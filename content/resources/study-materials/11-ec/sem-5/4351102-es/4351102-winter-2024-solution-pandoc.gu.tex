\documentclass[10pt,a4paper]{article}

% content/resources/templates/preamble.tex
\usepackage[margin=0.6in]{geometry}
\author{Milav Dabgar}
\usepackage{amsmath,amssymb,amsthm}
\usepackage{booktabs}
\usepackage{multirow}
\usepackage{xcolor}
\usepackage{tcolorbox}
\tcbuselibrary{breakable,skins}
\usepackage[colorlinks=true,linkcolor=blue]{hyperref}
\usepackage{titlesec}
\usepackage{enumitem}
\usepackage{tikz}
\usepackage{pgfplots}
\usepackage{circuitikz}
\usepackage[version=4]{mhchem}
\usepackage{longtable}
\usepackage{array}
\usepackage{float}
\usepackage{caption}
\usepackage{listings}

\lstset{
  basicstyle=\small\ttfamily,
  breaklines=true,
  breakatwhitespace=false,
  postbreak=\mbox{\textcolor{red}{$\hookrightarrow$}\space},
  float=false,
  numbers=left,
  numberstyle=\tiny\color{gray},
  numbersep=10pt,
  xleftmargin=2em,
  keywordstyle=\color{blue},
  commentstyle=\color{green!60!black},
  stringstyle=\color{purple},
  backgroundcolor=\color{gray!5},
  showstringspaces=false,
  tabsize=2,
  captionpos=b,
  keepspaces=true,
  columns=flexible
}

\pgfplotsset{compat=1.18}
\usetikzlibrary{shapes,arrows,positioning,calc,patterns,decorations.pathmorphing,decorations.markings,arrows.meta}

% Color scheme
\definecolor{headcolor}{RGB}{0,102,204}
\definecolor{keycolor}{RGB}{220,20,60}
\definecolor{solutioncolor}{RGB}{34,139,34}
\definecolor{mnemoniccolor}{RGB}{148,0,211}
\definecolor{codecolor}{RGB}{0,0,100}

% Spacing
\setlength{\parskip}{3pt}
\setlist[itemize]{nosep}
\setlist[enumerate]{nosep}

% Title formatting
\titleformat{\section}{\Large\bfseries\color{headcolor}}{\thesection}{1em}{}
\titleformat{\subsection}{\large\bfseries\color{headcolor}}{\thesubsection}{1em}{}

% Pandoc tightlist compatibility
\providecommand{\tightlist}{%
  \setlength{\itemsep}{0pt}\setlength{\parskip}{0pt}}

% Pandoc longtable compatibility
\newcounter{none}
\def\thenone{}


% content/resources/templates/gujarati-boxes.tex
\usepackage{fontspec}
\usepackage{polyglossia}

% Set Gujarati as main language (document is primarily in Gujarati)
% Note: gloss-gujarati.ldf doesn't exist in polyglossia, but it will use hyphenation patterns
\setdefaultlanguage{gujarati}
\setotherlanguage{english}

% Configure Gujarati font properly
% Use Language=Default to prevent polyglossia from trying to add language-specific features
% that don't exist for Gujarati, which causes "empty feature" warnings
\newfontfamily\gujaratifont[Script=Gujarati,AutoFakeBold=2.5,AutoFakeSlant=0.3]{Noto Sans Gujarati}
\setmainfont[Script=Gujarati,AutoFakeBold=2.5,AutoFakeSlant=0.3]{Noto Sans Gujarati}
% Use Noto Sans Gujarati for monospace to support Gujarati in text
\setmonofont[Scale=0.9]{Noto Sans Gujarati}

% Configure English to use the same font
\newfontfamily\englishfont[Script=Gujarati,AutoFakeBold=2.5,AutoFakeSlant=0.3]{Noto Sans Gujarati}

% Translations for polyglossia
\gappto\captionsgujarati{
  \renewcommand{\tablename}{કોષ્ટક}
  \renewcommand{\figurename}{આકૃતિ}
}

% Helper for TikZ nodes to ensure Gujarati font
\newcommand{\gu}[1]{{\gujaratifont #1}}

% Custom environments
\newtcolorbox{solutionbox}{
    breakable,
    enhanced,
    colback=solutioncolor!5!white,
    colframe=solutioncolor!75!black,
    fonttitle=\bfseries,
    title=જવાબ
}

\newtcolorbox{solutionboxnobreak}{
 colback=solutioncolor!5!white,
 colframe=solutioncolor!75!black,
 fonttitle=\bfseries,
 title=જવાબ
}

\newtcolorbox{keyformula}{
 breakable,
 enhanced,
 colback=keycolor!5!white,
 colframe=keycolor!75!black,
 fonttitle=\bfseries,
 title=રાસાયણિક સમીકરણ/સૂત્ર
}

\newtcolorbox{mnemonicbox}{
 breakable,
 enhanced,
 colback=mnemoniccolor!5!white,
 colframe=mnemoniccolor!75!black,
 fonttitle=\bfseries,
 title=મેમરી ટ્રીક
}


\begin{document}

\begin{center}
{\Huge\bfseries\color{headcolor} Subject Name (Gujarati)}\\[5pt]
{\LARGE 4351102 -- Winter 2024}\\[3pt]
{\large Semester 1 Study Material}\\[3pt]
{\normalsize\textit{Detailed Solutions and Explanations}}
\end{center}

\vspace{10pt}

\subsection*{પ્રશ્ન 1(અ) [3
ગુણ]}\label{uxaaauxab0uxab6uxaa8-1uxa85-3-uxa97uxaa3}

\textbf{ATmega32 ની વિશેષતાઓ લખો.}

\begin{solutionbox}

{\def\LTcaptype{none} % do not increment counter
\begin{longtable}[]{@{}ll@{}}
\toprule\noalign{}
વિશેષતા & વર્ણન \\
\midrule\noalign{}
\endhead
\bottomrule\noalign{}
\endlastfoot
\textbf{આર્કિટેક્ચર} & 8-bit RISC પ્રોસેસર \\
\textbf{મેમરી} & 32KB ફ્લેશ, 2KB SRAM, 1KB EEPROM \\
\textbf{I/O પોર્ટ્સ} & 32 પ્રોગ્રામેબલ I/O પિન્સ \\
\textbf{ટાઇમર્સ} & 3 ટાઇમર્સ (Timer0, Timer1, Timer2) \\
\textbf{ADC} & 10-bit, 8-channel ADC \\
\textbf{કમ્યુનિકેશન} & USART, SPI, I2C (TWI) \\
\end{longtable}
}

\begin{itemize}
\tightlist
\item
  \textbf{હાઇ પર્ફોર્મન્સ}: 16MHz પર 16 MIPS
\item
  \textbf{લો પાવર}: બહુવિધ સ્લીપ મોડ્સ
\item
  \textbf{ઓપરેટિંગ વોલ્ટેજ}: 2.7V થી 5.5V
\end{itemize}

\end{solutionbox}
\begin{mnemonicbox}
``ARM-TIC'' (Architecture-RISC, Memory-32KB,
Timers-3, I/O-32pins, Communication-3types)

\end{mnemonicbox}
\begin{center}\rule{0.5\linewidth}{0.5pt}\end{center}

\subsection*{પ્રશ્ન 1(બ) [4
ગુણ]}\label{uxaaauxab0uxab6uxaa8-1uxaac-4-uxa97uxaa3}

\textbf{માઇક્રોકંટ્રોલર પસંદ કરવા માટેના માપદંડો લખી સમજાવો.}

\begin{solutionbox}

{\def\LTcaptype{none} % do not increment counter
\begin{longtable}[]{@{}ll@{}}
\toprule\noalign{}
માપદંડ & વિચારણા \\
\midrule\noalign{}
\endhead
\bottomrule\noalign{}
\endlastfoot
\textbf{પર્ફોર્મન્સ} & સ્પીડ, ઇન્સ્ટ્રક્શન સેટ, આર્કિટેક્ચર \\
\textbf{મેમરી} & RAM, ROM, EEPROM આવશ્યકતાઓ \\
\textbf{I/O જરૂરિયાતો} & પિન્સની સંખ્યા, સ્પેશિયલ ફંક્શન્સ \\
\textbf{પાવર કન્ઝમ્પશન} & બેટરી લાઇફ, સ્લીપ મોડ્સ \\
\textbf{કિંમત} & યુનિટ પ્રાઇસ, ડેવલપમેન્ટ કોસ્ટ \\
\textbf{ડેવલપમેન્ટ ટૂલ્સ} & કમ્પાઇલર, ડીબગર ઉપલબ્ધતા \\
\end{longtable}
}

\begin{itemize}
\tightlist
\item
  \textbf{એપ્લિકેશન જરૂરિયાતો}: રિયલ-ટાઇમ કન્સ્ટ્રેઇન્ટ્સ, પ્રોસેસિંગ નીડ્સ
\item
  \textbf{પેકેજ સાઇઝ}: ફાઇનલ પ્રોડક્ટમાં સ્પેસ લિમિટેશન્સ
\item
  \textbf{પેરિફેરલ સપોર્ટ}: ADC, ટાઇમર્સ, કમ્યુનિકેશન ઇન્ટરફેસ
\end{itemize}

\end{solutionbox}
\begin{mnemonicbox}
``PM-IPCD'' (Performance, Memory, I/O, Power, Cost,
Development)

\end{mnemonicbox}
\begin{center}\rule{0.5\linewidth}{0.5pt}\end{center}

\subsection*{પ્રશ્ન 1(ક) [7
ગુણ]}\label{uxaaauxab0uxab6uxaa8-1uxa95-7-uxa97uxaa3}

\textbf{Embedded System ને વ્યાખ્યાયિત કરો. નાના, મધ્યમ અને વિશાળ Embedded
System ની ઉપયોગિતાની યાદી બનાવો.}

\begin{solutionbox}

\textbf{વ્યાખ્યા}: Embedded System એ મોટા યાંત્રિક અથવા ઇલેક્ટ્રિકલ સિસ્ટમમાં
ચોક્કસ કામ કરતું કમ્પ્યુટર સિસ્ટમ છે, જે વિશિષ્ટ કામો રિયલ-ટાઇમ મર્યાદા સાથે કરવા
માટે ડિઝાઇન કરવામાં આવે છે.

\textbf{એપ્લિકેશન ટેબલ}:

{\def\LTcaptype{none} % do not increment counter
\begin{longtable}[]{@{}lll@{}}
\toprule\noalign{}
સિસ્ટમ પ્રકાર & મેમરી સાઇઝ & એપ્લિકેશન્સ \\
\midrule\noalign{}
\endhead
\bottomrule\noalign{}
\endlastfoot
\textbf{નાના સ્કેલ} & \textless64KB & કેલ્ક્યુલેટર, ડિજિટલ વોચ, રમકડાં \\
\textbf{મધ્યમ સ્કેલ} & 64KB-1MB & મોબાઇલ ફોન, રાઉટર, પ્રિન્ટર \\
\textbf{વિશાળ સ્કેલ} & \textgreater1MB & ઓટોમોબાઇલ, એરક્રાફ્ટ સિસ્ટમ,
સેટેલાઇટ \\
\end{longtable}
}

\begin{center}
\textbf{Mermaid Diagram (Code)}
\begin{verbatim}
{Shaded}
{Highlighting}[]
graph TD
    A[Embedded System] {-{-}{} B[Small Scale]}
    A {-{-}{} C[Medium Scale]  }
    A {-{-}{} D[Large Scale]}
    B {-{-}{} E[Calculator{}br/{}Digital Watch{}br/{}Remote Control]}
    C {-{-}{} F[Mobile Phone{}br/{}Router{}br/{}Printer]}
    D {-{-}{} G[Car ECU{}br/{}Aircraft Control{}br/{}Medical Equipment]}
{Highlighting}
{Shaded}
\end{verbatim}
\end{center}

\textbf{લાક્ષણિકતાઓ}:

\begin{itemize}
\tightlist
\item
  \textbf{રિયલ-ટાઇમ ઓપરેશન}: પ્રિડિક્ટેબલ રિસ્પોન્સ ટાઇમ
\item
  \textbf{રિસોર્સ કન્સ્ટ્રેઇન્ટ્સ}: મર્યાદિત મેમરી અને પ્રોસેસિંગ પાવર
\item
  \textbf{ડેડિકેટેડ ફંક્શનાલિટી}: સિંગલ-પર્પઝ ડિઝાઇન
\end{itemize}

\end{solutionbox}
\begin{mnemonicbox}
``SML-CMP''
(Small-Calculator/Medium-Mobile/Large-Lifesupport)

\end{mnemonicbox}
\begin{center}\rule{0.5\linewidth}{0.5pt}\end{center}

\subsection*{પ્રશ્ન 1(ક) OR [7
ગુણ]}\label{uxaaauxab0uxab6uxaa8-1uxa95-or-7-uxa97uxaa3}

\textbf{Embedded system નો સામાન્ય બ્લોક ડાયાગ્રામ દોરી સમજાવો.}

\begin{solutionbox}

\begin{center}
\textbf{Mermaid Diagram (Code)}
\begin{verbatim}
{Shaded}
{Highlighting}[]
graph LR
    A[Input Interface] {-{-}{} B[Processor/Controller]}
    B {-{-}{} C[Output Interface]}
    B {-{-}{} D[Memory{}br/{}RAM/ROM/EEPROM]}
    B {-{-}{} E[Communication{}br/{}Interface]}
    F[Sensors] {-{-}{} A}
    C {-{-}{} G[Actuators/Display]}
    E {-{-}{} H[External Systems]}
    I[Power Supply] {-{-}{} B}
{Highlighting}
{Shaded}
\end{verbatim}
\end{center}

\textbf{બ્લોક ફંક્શન્સ}:

{\def\LTcaptype{none} % do not increment counter
\begin{longtable}[]{@{}ll@{}}
\toprule\noalign{}
બ્લોક & કાર્ય \\
\midrule\noalign{}
\endhead
\bottomrule\noalign{}
\endlastfoot
\textbf{પ્રોસેસર} & સેન્ટ્રલ પ્રોસેસિંગ યુનિટ (CPU/MCU) \\
\textbf{ઇનપુટ ઇન્ટરફેસ} & સેન્સર ડેટા એક્વિઝિશન, યુઝર ઇનપુટ \\
\textbf{આઉટપુટ ઇન્ટરફેસ} & એક્ચ્યુએટર કંટ્રોલ, ડિસ્પ્લે આઉટપુટ \\
\textbf{મેમરી} & પ્રોગ્રામ સ્ટોરેજ, ડેટા સ્ટોરેજ \\
\textbf{કમ્યુનિકેશન} & બાહ્ય સિસ્ટમ કનેક્ટિવિટી \\
\end{longtable}
}

\begin{itemize}
\tightlist
\item
  \textbf{ઇનપુટ પ્રોસેસિંગ}: ADC, ડિજિટલ ઇનપુટ કન્ડિશનિંગ
\item
  \textbf{આઉટપુટ કંટ્રોલ}: PWM, રિલે ડ્રાઇવર્સ, LED ડિસ્પ્લે
\item
  \textbf{પાવર મેનેજમેન્ટ}: વોલ્ટેજ રેગ્યુલેશન, પાવર ઓપ્ટિમાઇઝેશન
\end{itemize}

\end{solutionbox}
\begin{mnemonicbox}
``PIOMCP'' (Processor, Input, Output, Memory,
Communication, Power)

\end{mnemonicbox}
\begin{center}\rule{0.5\linewidth}{0.5pt}\end{center}

\subsection*{પ્રશ્ન 2(અ) [3
ગુણ]}\label{uxaaauxab0uxab6uxaa8-2uxa85-3-uxa97uxaa3}

\textbf{EEPROM નું પૂરું નામ લખો અને તેના વિશે સમજાવો.}

\begin{solutionbox}

\textbf{પૂરું નામ}: Electrically Erasable Programmable Read-Only Memory

\textbf{EEPROM રજિસ્ટર્સ}:

{\def\LTcaptype{none} % do not increment counter
\begin{longtable}[]{@{}ll@{}}
\toprule\noalign{}
રજિસ્ટર & કાર્ય \\
\midrule\noalign{}
\endhead
\bottomrule\noalign{}
\endlastfoot
\textbf{EEAR} & EEPROM Address Register \\
\textbf{EEDR} & EEPROM Data Register \\
\textbf{EECR} & EEPROM Control Register \\
\end{longtable}
}

\begin{itemize}
\tightlist
\item
  \textbf{EEAR}: EEPROM એક્સેસ માટે 10-bit એડ્રેસ (0-1023) હોલ્ડ કરે છે
\item
  \textbf{EEDR}: રીડ/રાઇટ ઓપરેશન માટે ડેટા રજિસ્ટર
\item
  \textbf{EECR}: કંટ્રોલ બિટ્સ - EERE (Read Enable), EEWE (Write Enable)
\end{itemize}

\end{solutionbox}
\begin{mnemonicbox}
``AAD-CRE'' (Address-EEAR, Data-EEDR, Control-EECR)

\end{mnemonicbox}
\begin{center}\rule{0.5\linewidth}{0.5pt}\end{center}

\subsection*{પ્રશ્ન 2(બ) [4
ગુણ]}\label{uxaaauxab0uxab6uxaa8-2uxaac-4-uxa97uxaa3}

\textbf{ATmega32માં રીસેટ સર્કિટ વિશે સમજાવો.}

\begin{solutionbox}

\textbf{રીસેટ સોર્સ ટેબલ}:

{\def\LTcaptype{none} % do not increment counter
\begin{longtable}[]{@{}ll@{}}
\toprule\noalign{}
રીસેટ પ્રકાર & ટ્રિગર કન્ડિશન \\
\midrule\noalign{}
\endhead
\bottomrule\noalign{}
\endlastfoot
\textbf{પાવર-ઓન રીસેટ} & VCC થ્રેશહોલ્ડ ઉપર વધે છે \\
\textbf{એક્સટર્નલ રીસેટ} & RESET પિન લો પુલ કરવામાં આવે છે \\
\textbf{બ્રાઉન-આઉટ રીસેટ} & VCC થ્રેશહોલ્ડ નીચે પડે છે \\
\textbf{વોચડોગ રીસેટ} & વોચડોગ ટાઇમર ઓવરફ્લો \\
\end{longtable}
}

\begin{center}
\textbf{Mermaid Diagram (Code)}
\begin{verbatim}
{Shaded}
{Highlighting}[]
graph TD
    A[Power{-on] {-}{-}{} E[Reset Vector]}
    B[External Pin] {-{-}{} E}
    C[Brown{-out] {-}{-}{} E}
    D[Watchdog] {-{-}{} E}
    E {-{-}{} F[Program Counter = 0x0000]}
{Highlighting}
{Shaded}
\end{verbatim}
\end{center}

\begin{itemize}
\tightlist
\item
  \textbf{રીસેટ ડ્યુરેશન}: મિનિમમ 2 ક્લોક સાઇકલ્સ
\item
  \textbf{રીસેટ વેક્ટર}: પ્રોગ્રામ એક્ઝિક્યુશન એડ્રેસ 0x0000 થી શરૂ થાય છે
\item
  \textbf{હાર્ડવેર કનેક્શન}: એક્સટર્નલ રીસેટ માટે પુલ-અપ રેઝિસ્ટર જરૂરી
\end{itemize}

\end{solutionbox}
\begin{mnemonicbox}
``PEBW'' (Power-on, External, Brown-out, Watchdog)

\end{mnemonicbox}
\begin{center}\rule{0.5\linewidth}{0.5pt}\end{center}

\subsection*{પ્રશ્ન 2(ક) [7
ગુણ]}\label{uxaaauxab0uxab6uxaa8-2uxa95-7-uxa97uxaa3}

\textbf{રિયલ ટાઇમ ઓપરેટિંગ સિસ્ટમની વ્યાખ્યા આપો અને તેની લાક્ષણિકતાઓ સમજાવો.}

\begin{solutionbox}

\textbf{વ્યાખ્યા}: રિયલ ટાઇમ ઓપરેટિંગ સિસ્ટમ (RTOS) એ એવું ઓપરેટિંગ સિસ્ટમ છે જે કડક
ટાઇમિંગ કન્સ્ટ્રેઇન્ટ્સ અને પ્રિડિક્ટેબલ રિસ્પોન્સ ટાઇમ સાથે રિયલ-ટાઇમ એપ્લિકેશન્સ હેન્ડલ
કરવા માટે ડિઝાઇન કરવામાં આવે છે.

\textbf{લાક્ષણિકતાઓ ટેબલ}:

{\def\LTcaptype{none} % do not increment counter
\begin{longtable}[]{@{}ll@{}}
\toprule\noalign{}
લાક્ષણિકતા & વર્ણન \\
\midrule\noalign{}
\endhead
\bottomrule\noalign{}
\endlastfoot
\textbf{ડિટર્મિનિસ્ટિક} & પ્રિડિક્ટેબલ એક્ઝિક્યુશન ટાઇમ \\
\textbf{પ્રીએમ્પ્ટિવ} & હાઇ પ્રાયોરિટી ટાસ્ક લો પ્રાયોરિટીને ઇન્ટરપ્ટ કરે છે \\
\textbf{મલ્ટિટાસ્કિંગ} & મલ્ટિપલ ટાસ્ક એક્ઝિક્યુશન \\
\textbf{ફાસ્ટ રિસ્પોન્સ} & મિનિમલ ઇન્ટરપ્ટ લેટન્સી \\
\textbf{પ્રાયોરિટી-બેસ્ડ} & પ્રાયોરિટી આધારિત ટાસ્ક શિડ્યુલિંગ \\
\textbf{રિસોર્સ મેનેજમેન્ટ} & એફિશિયન્ટ મેમરી અને CPU ઉપયોગ \\
\end{longtable}
}

\begin{center}
\textbf{Mermaid Diagram (Code)}
\begin{verbatim}
{Shaded}
{Highlighting}[]
graph TD
    A[RTOS] {-{-}{} B[Hard Real{-}time]}
    A {-{-}{} C[Soft Real{-}time]}
    B {-{-}{} D[Strict Deadlines{}br/{}Safety Critical]}
    C {-{-}{} E[Flexible Deadlines{}br/{}Performance Critical]}
{Highlighting}
{Shaded}
\end{verbatim}
\end{center}

\begin{itemize}
\tightlist
\item
  \textbf{ટાસ્ક શિડ્યુલિંગ}: રાઉન્ડ-રોબિન, પ્રાયોરિટી-બેસ્ડ અલ્ગોરિધમ્સ
\item
  \textbf{ઇન્ટર-ટાસ્ક કમ્યુનિકેશન}: સેમાફોર્સ, મેસેજ ક્યુ
\item
  \textbf{મેમરી મેનેજમેન્ટ}: પ્રિડિક્ટેબિલિટી માટે સ્ટેટિક એલોકેશન
\end{itemize}

\end{solutionbox}
\begin{mnemonicbox}
``DPM-FPR'' (Deterministic, Preemptive,
Multitasking, Fast, Priority, Resource)

\end{mnemonicbox}
\begin{center}\rule{0.5\linewidth}{0.5pt}\end{center}

\subsection*{પ્રશ્ન 2(અ) OR [3
ગુણ]}\label{uxaaauxab0uxab6uxaa8-2uxa85-or-3-uxa97uxaa3}

\textbf{AVR ફેમિલી વિશે સમજાવો.}

\begin{solutionbox}

\textbf{AVR ફેમિલી વર્ગીકરણ}:

{\def\LTcaptype{none} % do not increment counter
\begin{longtable}[]{@{}ll@{}}
\toprule\noalign{}
AVR પ્રકાર & વિશેષતાઓ \\
\midrule\noalign{}
\endhead
\bottomrule\noalign{}
\endlastfoot
\textbf{ATtiny} & 8-32 પિન્સ, બેસિક ફીચર્સ \\
\textbf{ATmega} & 28-100 પિન્સ, ફુલ ફીચર્સ \\
\textbf{ATxmega} & એડવાન્સ ફીચર્સ, DMA \\
\end{longtable}
}

\begin{itemize}
\tightlist
\item
  \textbf{આર્કિટેક્ચર}: 8-bit RISC, હાર્વર્ડ આર્કિટેક્ચર
\item
  \textbf{ઇન્સ્ટ્રક્શન સેટ}: 130+ ઇન્સ્ટ્રક્શન્સ, સિંગલ સાઇકલ એક્ઝિક્યુશન
\item
  \textbf{મેમરી}: ફ્લેશ પ્રોગ્રામ મેમરી, SRAM, EEPROM
\end{itemize}

\end{solutionbox}
\begin{mnemonicbox}
``TAX'' (Tiny-basic, mega-full, Xmega-advanced)

\end{mnemonicbox}
\begin{center}\rule{0.5\linewidth}{0.5pt}\end{center}

\subsection*{પ્રશ્ન 2(બ) OR [4
ગુણ]}\label{uxaaauxab0uxab6uxaa8-2uxaac-or-4-uxa97uxaa3}

\textbf{ATmega32માં ક્લોક સોર્સની પસંદગી માટે ફ્યૂઝ બિટ્સનું મહત્વ સમજાવો.}

\begin{solutionbox}

\textbf{ક્લોક સોર્સ સિલેક્શન}:

{\def\LTcaptype{none} % do not increment counter
\begin{longtable}[]{@{}ll@{}}
\toprule\noalign{}
ફ્યૂઝ બિટ્સ & ક્લોક સોર્સ \\
\midrule\noalign{}
\endhead
\bottomrule\noalign{}
\endlastfoot
\textbf{CKSEL3:0} & ક્લોક સોર્સ સિલેક્શન \\
\textbf{SUT1:0} & સ્ટાર્ટ-અપ ટાઇમ સિલેક્શન \\
\end{longtable}
}

\textbf{ક્લોક ઓપ્શન્સ ટેબલ}:

{\def\LTcaptype{none} % do not increment counter
\begin{longtable}[]{@{}lll@{}}
\toprule\noalign{}
CKSEL મૂલ્ય & ક્લોક સોર્સ & ફ્રીક્વન્સી \\
\midrule\noalign{}
\endhead
\bottomrule\noalign{}
\endlastfoot
0001 & એક્સટર્નલ ક્રિસ્ટલ & 1-8 MHz \\
0010 & એક્સટર્નલ ક્રિસ્ટલ & 8+ MHz \\
0100 & ઇન્ટર્નલ RC & 8 MHz \\
0000 & એક્સટર્નલ ક્લોક & યુઝર ડિફાઇન્ડ \\
\end{longtable}
}

\begin{itemize}
\tightlist
\item
  \textbf{ક્રિસ્ટલ સિલેક્શન}: એક્સટર્નલ ક્રિસ્ટલ અને કૅપેસિટર જરૂરી
\item
  \textbf{RC ઓસિલેટર}: બિલ્ટ-ઇન, ઓછું એક્યુરેટ પણ સુવિધાજનક
\item
  \textbf{સ્ટાર્ટ-અપ ટાઇમ}: ક્રિસ્ટલ સ્ટેબિલાઇઝેશનની મંજૂરી આપે છે
\end{itemize}

\end{solutionbox}
\begin{mnemonicbox}
``CRIS'' (Crystal, RC, Internal, Start-up)

\end{mnemonicbox}
\begin{center}\rule{0.5\linewidth}{0.5pt}\end{center}

\subsection*{પ્રશ્ન 2(ક) OR [7
ગુણ]}\label{uxaaauxab0uxab6uxaa8-2uxa95-or-7-uxa97uxaa3}

\textbf{ATmega32નો પિન ડાયાગ્રામ દોરી MISO, MOSI, SCK \&AREF Pin નું કાર્ય
સમજાવો.}

\begin{solutionbox}

\begin{verbatim}
        +{-{-}{-}{-}{-}{-}{-}{-}{-}{-}+}
    PB0 |1      40| PA0
    PB1 |2      39| PA1  
    PB2 |3      38| PA2
    PB3 |4      37| PA3
    PB4 |5      36| PA4
MOSI PB5|6      35| PA5
MISO PB6|7      34| PA6
 SCK PB7|8      33| PA7
   RESET|9      32| AREF
    VCC |10     31| GND
    GND |11     30| AVCC
   XTAL2|12     29| PC7
   XTAL1|13     28| PC6
        +{-{-}{-}{-}{-}{-}{-}{-}{-}{-}+}
\end{verbatim}

\textbf{પિન ફંક્શન્સ ટેબલ}:

{\def\LTcaptype{none} % do not increment counter
\begin{longtable}[]{@{}lll@{}}
\toprule\noalign{}
પિન & કાર્ય & વર્ણન \\
\midrule\noalign{}
\endhead
\bottomrule\noalign{}
\endlastfoot
\textbf{MOSI} & Master Out Slave In & માસ્ટરથી સ્લેવમાં SPI ડેટા આઉટપુટ \\
\textbf{MISO} & Master In Slave Out & સ્લેવથી માસ્ટરમાં SPI ડેટા ઇનપુટ \\
\textbf{SCK} & Serial Clock & SPI ક્લોક સિગ્નલ \\
\textbf{AREF} & Analog Reference & ADC રેફરન્સ વોલ્ટેજ \\
\end{longtable}
}

\begin{itemize}
\tightlist
\item
  \textbf{SPI કમ્યુનિકેશન}: MOSI, MISO, SCK મળીને સીરિયલ ડેટા ટ્રાન્સફર માટે કામ
  કરે છે
\item
  \textbf{ADC રેફરન્સ}: AREF, ADC કન્વર્ઝન માટે સ્થિર વોલ્ટેજ રેફરન્સ પ્રદાન કરે છે
\item
  \textbf{પિન મલ્ટિપ્લેક્સિંગ}: આ પિન્સ GPIO તરીકે વૈકલ્પિક કાર્યો ધરાવે છે
\end{itemize}

\end{solutionbox}
\begin{mnemonicbox}
``MMS-A'' (MOSI-out, MISO-in, SCK-clock,
AREF-reference)

\end{mnemonicbox}
\begin{center}\rule{0.5\linewidth}{0.5pt}\end{center}

\subsection*{પ્રશ્ન 3(અ) [3
ગુણ]}\label{uxaaauxab0uxab6uxaa8-3uxa85-3-uxa97uxaa3}

\textbf{ATmega32 માં DDR I/O રજિસ્ટરની ભૂમિકા સમજાવો.}

\begin{solutionbox}

\textbf{DDR (Data Direction Register) કાર્યો}:

{\def\LTcaptype{none} % do not increment counter
\begin{longtable}[]{@{}ll@{}}
\toprule\noalign{}
બિટ મૂલ્ય & પિન કન્ફિગરેશન \\
\midrule\noalign{}
\endhead
\bottomrule\noalign{}
\endlastfoot
\textbf{0} & ઇનપુટ પિન \\
\textbf{1} & આઉટપુટ પિન \\
\end{longtable}
}

\begin{itemize}
\tightlist
\item
  \textbf{પોર્ટ કંટ્રોલ}: દરેક પોર્ટનું અનુરૂપ DDR (DDRA, DDRB, DDRC, DDRD) છે
\item
  \textbf{બિટ-વાઇઝ કંટ્રોલ}: વ્યક્તિગત પિન દિશા કંટ્રોલ
\item
  \textbf{ડિફોલ્ટ સ્થિતિ}: રીસેટ પછી બધા પિન્સ ઇનપુટ (DDR = 0x00)
\end{itemize}

\textbf{કોડ ઉદાહરણ}:

\begin{verbatim}
DDRA = 0xFF;  // બધા Port A પિન્સ આઉટપુટ તરીકે
DDRB = 0x0F;  // PB0{-PB3 આઉટપુટ, PB4{-}PB7 ઇનપુટ}
\end{verbatim}

\end{solutionbox}
\begin{mnemonicbox}
``DDR-IO'' (Data Direction Register controls
Input/Output)

\end{mnemonicbox}
\begin{center}\rule{0.5\linewidth}{0.5pt}\end{center}

\subsection*{પ્રશ્ન 3(બ) [4
ગુણ]}\label{uxaaauxab0uxab6uxaa8-3uxaac-4-uxa97uxaa3}

\textbf{Port B પરથી ડેટાને રીડ કરાવી Port C પર મોકલવા માટેનો AVR C પ્રોગ્રામ
લખો.}

\begin{solutionbox}

\begin{verbatim}
\#include {avr/io.h}

int main(void)
\{
    unsigned char data;
    
    // Port B ને ઇનપુટ તરીકે કન્ફિગર કરો
    DDRB = 0x00;
    
    // Port C ને આઉટપુટ તરીકે કન્ફિગર કરો 
    DDRC = 0xFF;
    
    while(1)
    \{
        // Port B થી ડેટા રીડ કરો
        data = PINB;
        
        // Port C પર ડેટા મોકલો
        PORTC = data;
    \}
    
    return 0;
\}
\end{verbatim}

\textbf{પ્રોગ્રામ સમજૂતી}:

\begin{itemize}
\tightlist
\item
  \textbf{DDRB = 0x00}: બધા Port B પિન્સને ઇનપુટ તરીકે સેટ કરે છે
\item
  \textbf{DDRC = 0xFF}: બધા Port C પિન્સને આઉટપુટ તરીકે સેટ કરે છે
\item
  \textbf{PINB}: Port B પિન્સની વર્તમાન સ્થિતિ રીડ કરે છે
\item
  \textbf{PORTC}: Port C આઉટપુટ પિન્સ પર ડેટા લખે છે
\end{itemize}

\end{solutionbox}
\begin{mnemonicbox}
``RSTO'' (Read-PINB, Set-DDR, Transfer-data,
Output-PORTC)

\end{mnemonicbox}
\begin{center}\rule{0.5\linewidth}{0.5pt}\end{center}

\subsection*{પ્રશ્ન 3(ક) [7
ગુણ]}\label{uxaaauxab0uxab6uxaa8-3uxa95-7-uxa97uxaa3}

\textbf{PORT B ના પિન નં 1 પર ડોર સેન્સર જોડાયેલ છે અને PORT C ના પિન નં 7 પર
LED જોડાયેલ છે. દરવાજા ઉપર લાગેલા સેન્સરને મોનિટર કરતાં રહો અને જ્યારે દરવાજો ખુલે
ત્યારે LED ચાલુ થાય તે માટેનો AVR C પ્રોગ્રામ લખો.}

\begin{solutionbox}

\begin{verbatim}
\#include {avr/io.h}

int main(void)
\{
    // PB1 ને ઇનપુટ તરીકે કન્ફિગર કરો (ડોર સેન્સર)
    DDRB \&= {(}1{}1);  // બિટ 1 ક્લિયર કરો
    
    // PC7 ને આઉટપુટ તરીકે કન્ફિગર કરો (LED)
    DDRC |= (1{}7);   // બિટ 7 સેટ કરો
    
    // PB1 માટે પુલ{-અપ એનેબલ કરો}
    PORTB |= (1{}1);
    
    while(1)
    \{
        // ડોર સેન્સરની સ્થિતિ ચેક કરો
        if(PINB \& (1{}1))
        \{
            // દરવાજો બંધ {- LED બંધ કરો}
            PORTC \&= {(}1{}7);
        \}
        else
        \{
            // દરવાજો ખુલ્લો {- LED ચાલુ કરો}
            PORTC |= (1{}7);
        \}
    \}
    
    return 0;
\}
\end{verbatim}

\textbf{હાર્ડવેર કનેક્શન}:

\begin{itemize}
\tightlist
\item
  \textbf{ડોર સેન્સર}: PB1 અને GND વચ્ચે જોડાયેલ
\item
  \textbf{LED}: કરન્ટ લિમિટિંગ રેઝિસ્ટર દ્વારા PC7 સાથે જોડાયેલ
\item
  \textbf{પુલ-અપ}: PB1 માટે ઇન્ટર્નલ પુલ-અપ એનેબલ
\end{itemize}

\textbf{પ્રોગ્રામ લોજિક}:

\begin{itemize}
\tightlist
\item
  \textbf{સેન્સર બંધ}: PB1 = HIGH, LED OFF
\item
  \textbf{સેન્સર ખુલ્લું}: PB1 = LOW, LED ON
\end{itemize}

\end{solutionbox}
\begin{mnemonicbox}
``DCOL'' (Door-sensor, Configure-pins, Open-check,
LED-control)

\end{mnemonicbox}
\begin{center}\rule{0.5\linewidth}{0.5pt}\end{center}

\subsection*{પ્રશ્ન 3(અ) OR [3
ગુણ]}\label{uxaaauxab0uxab6uxaa8-3uxa85-or-3-uxa97uxaa3}

\textbf{AVR C પ્રોગ્રામ ના ડેટા ટાઇપની ચર્ચા કરો.}

\begin{solutionbox}

\textbf{AVR C ડેટા ટાઇપ્સ ટેબલ}:

{\def\LTcaptype{none} % do not increment counter
\begin{longtable}[]{@{}lll@{}}
\toprule\noalign{}
ડેટા ટાઇપ & સાઇઝ & રેન્જ \\
\midrule\noalign{}
\endhead
\bottomrule\noalign{}
\endlastfoot
\textbf{char} & 8-bit & -128 થી 127 \\
\textbf{unsigned char} & 8-bit & 0 થી 255 \\
\textbf{int} & 16-bit & -32768 થી 32767 \\
\textbf{unsigned int} & 16-bit & 0 થી 65535 \\
\textbf{long} & 32-bit & -2^{3}^{1} થી 2^{3}^{1}-1 \\
\textbf{float} & 32-bit & IEEE 754 ફોર્મેટ \\
\end{longtable}
}

\begin{itemize}
\tightlist
\item
  \textbf{મેમરી એફિશિયન્સી}: સૌથી નાનો યોગ્ય ડેટા ટાઇપ વાપરો
\item
  \textbf{અનસાઇન્ડ ટાઇપ્સ}: ફક્ત પોઝિટિવ વેલ્યુ માટે, રેન્જ બમાવે છે
\item
  \textbf{બિટ ફિલ્ડ્સ}: સ્પેસિફિક બિટ-વિડ્થ વેરિએબલ્સ ડિફાઇન કરી શકાય છે
\end{itemize}

\end{solutionbox}
\begin{mnemonicbox}
``CIL-FUB'' (Char-8bit, Int-16bit, Long-32bit,
Float-32bit, Unsigned-positive, Bit-specific)

\end{mnemonicbox}
\begin{center}\rule{0.5\linewidth}{0.5pt}\end{center}

\subsection*{પ્રશ્ન 3(બ) OR [4
ગુણ]}\label{uxaaauxab0uxab6uxaa8-3uxaac-or-4-uxa97uxaa3}

\textbf{સિરિયલ કોમ્યુનિકેશન પ્રોટોકોલ સમજાવો.}

\begin{solutionbox}

\textbf{સિરિયલ કોમ્યુનિકેશન પેરામીટર્સ}:

{\def\LTcaptype{none} % do not increment counter
\begin{longtable}[]{@{}ll@{}}
\toprule\noalign{}
પેરામીટર & વર્ણન \\
\midrule\noalign{}
\endhead
\bottomrule\noalign{}
\endlastfoot
\textbf{બોડ રેટ} & ડેટા ટ્રાન્સમિશન સ્પીડ (બિટ્સ/સેકન્ડ) \\
\textbf{ડેટા બિટ્સ} & ડેટા બિટ્સની સંખ્યા (5-9) \\
\textbf{પેરિટી} & એરર ચેકિંગ (None, Even, Odd) \\
\textbf{સ્ટોપ બિટ્સ} & ફ્રેમના અંતનું માર્કર (1 અથવા 2) \\
\end{longtable}
}

\begin{verbatim}
sequenceDiagram
    participant TX as Transmitter
    participant RX as Receiver
    TX{-RX: Start Bit (0)}
    TX{-RX: Data Bits (8)}
    TX{-RX: Parity Bit (Optional)}
    TX{-RX: Stop Bit(s) (1)}
\end{verbatim}

\begin{itemize}
\tightlist
\item
  \textbf{એસિંક્રોનસ}: કોઈ ક્લોક સિગ્નલ નથી, સ્ટાર્ટ/સ્ટોપ બિટ્સ વાપરે છે
\item
  \textbf{RS232 સ્ટાન્ડર્ડ}: \pm12V લેવલ્સ, TTL લેવલ્સમાં કન્વર્ટ થાય છે
\item
  \textbf{સામાન્ય બોડ રેટ્સ}: 9600, 19200, 38400, 115200
\end{itemize}

\end{solutionbox}
\begin{mnemonicbox}
``BDPS'' (Baud-rate, Data-bits, Parity-check,
Stop-bits)

\end{mnemonicbox}
\begin{center}\rule{0.5\linewidth}{0.5pt}\end{center}

\subsection*{પ્રશ્ન 3(ક) OR [7
ગુણ]}\label{uxaaauxab0uxab6uxaa8-3uxa95-or-7-uxa97uxaa3}

\textbf{Port B ના પિન નં. 0 અને પિન નં. 1 ને રીડ કરી નીચે આપેલા ટેબલ પ્રમાણે
ASCII કેરેક્ટર Port D પર મોકલાવા માટેનો AVR C પ્રોગ્રામ લખો}

\begin{solutionbox}

\begin{verbatim}
\#include {avr/io.h}

int main(void)
\{
    unsigned char input;
    
    // PB1 અને PB0 ને ઇનપુટ તરીકે કન્ફિગર કરો
    DDRB \&= {((}1{}1)|(1{}0));
    
    // Port D ને આઉટપુટ તરીકે કન્ફિગર કરો
    DDRD = 0xFF;
    
    // PB1 અને PB0 માટે પુલ{-અપ એનેબલ કરો}
    PORTB |= (1{}1)|(1{}0);
    
    while(1)
    \{
        // PB1 અને PB0 રીડ કરો
        input = PINB \& 0x03;  // અન્ય બિટ્સ માસ્ક કરો
        
        switch(input)
        \{
            case 0x00:  // Pin1=0, Pin0=0
                PORTD = {0};  // ASCII {0 = 0x30}
                break;
                
            case 0x01:  // Pin1=0, Pin0=1
                PORTD = {1};  // ASCII {1 = 0x31}
                break;
                
            case 0x02:  // Pin1=1, Pin0=0
                PORTD = {2};  // ASCII {2 = 0x32}
                break;
                
            case 0x03:  // Pin1=1, Pin0=1
                PORTD = {3};  // ASCII {3 = 0x33}
                break;
        \}
    \}
    
    return 0;
\}
\end{verbatim}

\textbf{ટ્રુથ ટેબલ અમલીકરણ}:

{\def\LTcaptype{none} % do not increment counter
\begin{longtable}[]{@{}llll@{}}
\toprule\noalign{}
Pin1 & Pin0 & ઇનપુટ મૂલ્ય & ASCII આઉટપુટ \\
\midrule\noalign{}
\endhead
\bottomrule\noalign{}
\endlastfoot
0 & 0 & 0x00 & `0' (0x30) \\
0 & 1 & 0x01 & `1' (0x31) \\
1 & 0 & 0x02 & `2' (0x32) \\
1 & 1 & 0x03 & `3' (0x33) \\
\end{longtable}
}

\end{solutionbox}
\begin{mnemonicbox}
``MATS'' (Mask-inputs, ASCII-conversion,
Truth-table, Switch-case)

\end{mnemonicbox}
\begin{center}\rule{0.5\linewidth}{0.5pt}\end{center}

\subsection*{પ્રશ્ન 4(અ) [3
ગુણ]}\label{uxaaauxab0uxab6uxaa8-4uxa85-3-uxa97uxaa3}

\textbf{ATmega32 સાથે રિલે ડ્રાઇવર ULN2803નું ઇન્ટરફેસિંગ ડાયાગ્રામ દોરો.}

\begin{solutionbox}

\begin{verbatim}
ATmega32          ULN2803         Relay
                                 
PC0 {-{-}{-}{-}{-}{-}|1    18|{-}{-}{-}{-}{-}{-}{-}{-}{-}{-}{-} +12V}
PC1 {-{-}{-}{-}{-}{-}|2    17|    }
PC2 {-{-}{-}{-}{-}{-}|3    16|    }
PC3 {-{-}{-}{-}{-}{-}|4    15|    }
PC4 {-{-}{-}{-}{-}{-}|5    14|    }
PC5 {-{-}{-}{-}{-}{-}|6    13|    }
PC6 {-{-}{-}{-}{-}{-}|7    12|    }
PC7 {-{-}{-}{-}{-}{-}|8    11|    }
           |9    10|{-{-}{-}{-}{-}{-}{-}{-}{-}{-}{-} GND}
           ULN2803    
                       
    COM1 of Relay connected to +12V
    NO1 of Relay connected to Load
    GND common for all
\end{verbatim}

\textbf{કોમ્પોનન્ટ ફંક્શન્સ}:

\begin{itemize}
\tightlist
\item
  \textbf{ULN2803}: ડાર્લિંગ્ટન ટ્રાન્ઝિસ્ટર એરે, કરન્ટ એમ્પ્લિફિકેશન
\item
  \textbf{પ્રોટેક્શન ડાયોડ્સ}: ઇન્ડક્ટિવ લોડ્સ માટે બિલ્ટ-ઇન ફ્લાયબેક ડાયોડ્સ
\item
  \textbf{રિલે કોઇલ}: 12V જરૂરી, ULN2803 આઉટપુટ દ્વારા કંટ્રોલ
\end{itemize}

\end{solutionbox}
\begin{mnemonicbox}
``UPC'' (ULN-driver, Port-control, Current-amplify)

\end{mnemonicbox}
\begin{center}\rule{0.5\linewidth}{0.5pt}\end{center}

\subsection*{પ્રશ્ન 4(બ) [4
ગુણ]}\label{uxaaauxab0uxab6uxaa8-4uxaac-4-uxa97uxaa3}

\textbf{પોલિંગ મેથડથી A/D કન્વર્ટરને પ્રોગ્રામ કરવા માટેના સ્ટેપ્સ લખો.}

\begin{solutionbox}

\textbf{ADC પ્રોગ્રામિંગ સ્ટેપ્સ}:

{\def\LTcaptype{none} % do not increment counter
\begin{longtable}[]{@{}ll@{}}
\toprule\noalign{}
સ્ટેપ & ક્રિયા \\
\midrule\noalign{}
\endhead
\bottomrule\noalign{}
\endlastfoot
\textbf{1} & ADMUX રજિસ્ટર કન્ફિગર કરો (રેફરન્સ, ચેનલ) \\
\textbf{2} & ADCSRA રજિસ્ટર કન્ફિગર કરો (એનેબલ, પ્રીસ્કેલર) \\
\textbf{3} & કન્વર્ઝન સ્ટાર્ટ કરો (ADSC બિટ સેટ કરો) \\
\textbf{4} & કન્વર્ઝન પૂર્ણ થવાની રાહ જુઓ (ADIF ફ્લેગ પોલ કરો) \\
\textbf{5} & ADCL અને ADCH થી પરિણામ રીડ કરો \\
\end{longtable}
}

\textbf{કોડ અમલીકરણ}:

\begin{verbatim}
// સ્ટેપ 1: ADMUX કન્ફિગર કરો
ADMUX = (1{}REFS0);  // AVCC રેફરન્સ, ચેનલ 0

// સ્ટેપ 2: પ્રીસ્કેલર સાથે ADC એનેબલ કરો
ADCSRA = (1{}ADEN)|(1{}ADPS2)|(1{}ADPS1)|(1{}ADPS0);

// સ્ટેપ 3: કન્વર્ઝન સ્ટાર્ટ કરો
ADCSRA |= (1{}ADSC);

// સ્ટેપ 4: પૂર્ણતાની રાહ જુઓ
while(!(ADCSRA \& (1{}ADIF)));

// સ્ટેપ 5: પરિણામ રીડ કરો
result = ADC;  // ADCL અને ADCH નું સંયોજન
\end{verbatim}

\end{solutionbox}
\begin{mnemonicbox}
``CCSWR'' (Configure-ADMUX, Configure-ADCSRA,
Start-conversion, Wait-complete, Read-result)

\end{mnemonicbox}
\begin{center}\rule{0.5\linewidth}{0.5pt}\end{center}

\subsection*{પ્રશ્ન 4(ક) [7
ગુણ]}\label{uxaaauxab0uxab6uxaa8-4uxa95-7-uxa97uxaa3}

\textbf{I2C 2 વાયર સિરિયલ ઇન્ટરફેસ પ્રોટોકોલ વિસ્તારવાર સમજાવો}

\begin{solutionbox}

\textbf{I2C પ્રોટોકોલ ફીચર્સ}:

{\def\LTcaptype{none} % do not increment counter
\begin{longtable}[]{@{}ll@{}}
\toprule\noalign{}
ફીચર & વર્ણન \\
\midrule\noalign{}
\endhead
\bottomrule\noalign{}
\endlastfoot
\textbf{બે વાયર} & SDA (ડેટા) અને SCL (ક્લોક) \\
\textbf{મલ્ટિ-માસ્ટર} & બહુવિધ માસ્ટર બસ કંટ્રોલ કરી શકે છે \\
\textbf{એડ્રેસિંગ} & 7-bit અથવા 10-bit ડિવાઇસ એડ્રેસ \\
\textbf{બાઇડાયરેક્શનલ} & બંને દિશામાં ડેટા ફ્લો \\
\end{longtable}
}

\begin{verbatim}
sequenceDiagram
    participant M as Master
    participant S as Slave
    M{-S: Start Condition}
    M{-S: Slave Address + R/W}
    S{-M: ACK}
    M{-S: Data Byte}
    S{-M: ACK}
    M{-S: Stop Condition}
\end{verbatim}

\textbf{I2C ફ્રેમ સ્ટ્રક્ચર}:

\begin{itemize}
\tightlist
\item
  \textbf{સ્ટાર્ટ કન્ડિશન}: SCL હાઇ હોય ત્યારે SDA લો જાય છે
\item
  \textbf{એડ્રેસ ફ્રેમ}: 7-bit એડ્રેસ + R/W બિટ
\item
  \textbf{ડેટા ફ્રેમ}: 8-bit ડેટા + ACK/NACK
\item
  \textbf{સ્ટોપ કન્ડિશન}: SCL હાઇ હોય ત્યારે SDA હાઇ જાય છે
\end{itemize}

\textbf{ATmega32 માં TWI રજિસ્ટર્સ}:

{\def\LTcaptype{none} % do not increment counter
\begin{longtable}[]{@{}ll@{}}
\toprule\noalign{}
રજિસ્ટર & કાર્ય \\
\midrule\noalign{}
\endhead
\bottomrule\noalign{}
\endlastfoot
\textbf{TWCR} & કંટ્રોલ અને સ્ટેટસ \\
\textbf{TWDR} & ડેટા રજિસ્ટર \\
\textbf{TWAR} & એડ્રેસ રજિસ્ટર \\
\textbf{TWSR} & સ્ટેટસ રજિસ્ટર \\
\end{longtable}
}

\begin{itemize}
\tightlist
\item
  \textbf{ક્લોક સ્ટ્રેચિંગ}: સ્લેવ માસ્ટરને ધીરે કરવા માટે SCL લો હોલ્ડ કરી શકે છે
\item
  \textbf{આર્બિટ્રેશન}: મલ્ટિ-માસ્ટર સિસ્ટમ્સમાં કોલિઝન અટકાવે છે
\item
  \textbf{પુલ-અપ રેઝિસ્ટર્સ}: SDA અને SCL બંને લાઇન્સ પર જરૂરી (સામાન્ય રીતે 4.7kΩ)
\end{itemize}

\end{solutionbox}
\begin{mnemonicbox}
``SAD-CSA'' (Start-Address-Data,
Control-Status-Address)

\end{mnemonicbox}
\begin{center}\rule{0.5\linewidth}{0.5pt}\end{center}

\subsection*{પ્રશ્ન 4(અ) OR [3
ગુણ]}\label{uxaaauxab0uxab6uxaa8-4uxa85-or-3-uxa97uxaa3}

\textbf{8-બિટ ટાઇમરનો ઉપયોગ કરી DC મોટરની સ્પીડ કંટ્રોલ કરવા માટે કોઈ પણ એક
PWM મોડ સમજાવો.}

\begin{solutionbox}

\textbf{ફાસ્ટ PWM મોડ (મોડ 3)}:

{\def\LTcaptype{none} % do not increment counter
\begin{longtable}[]{@{}ll@{}}
\toprule\noalign{}
પેરામીટર & મૂલ્ય \\
\midrule\noalign{}
\endhead
\bottomrule\noalign{}
\endlastfoot
\textbf{WGM બિટ્સ} & WGM01=1, WGM00=1 \\
\textbf{TOP મૂલ્ય} & 0xFF (255) \\
\textbf{રેઝોલ્યુશન} & 8-bit \\
\textbf{ફ્રીક્વન્સી} & fclk/(256\timesprescaler) \\
\end{longtable}
}

\textbf{PWM કન્ફિગરેશન}:

\begin{verbatim}
// ફાસ્ટ PWM માટે Timer0 કન્ફિગર કરો
TCCR0 = (1{}WGM01)|(1{}WGM00)|(1{}COM01)|(1{}CS01);

// ડ્યુટી સાઇકલ સેટ કરો (0{-255)}
OCR0 = 128;  // 50\% ડ્યુટી સાઇકલ
\end{verbatim}

\begin{center}
\textbf{Mermaid Diagram (Code)}
\begin{verbatim}
{Shaded}
{Highlighting}[]
graph LR
    A[Timer0] {-{-}{} B[PWM Signal]}
    B {-{-}{} C[Motor Driver]}
    C {-{-}{} D[DC Motor]}
    E[OCR0 Value] {-{-}{} A}
{Highlighting}
{Shaded}
\end{verbatim}
\end{center}

\begin{itemize}
\tightlist
\item
  \textbf{ડ્યુટી સાઇકલ કંટ્રોલ}: OCR0 મૂલ્ય મોટરની સ્પીડ નક્કી કરે છે
\item
  \textbf{નોન-ઇન્વર્ટિંગ મોડ}: હાઇ પલ્સ વિડ્થ = OCR0/255
\item
  \textbf{મોટર કંટ્રોલ}: વધારે ડ્યુટી સાઇકલ = વધારે સ્પીડ
\end{itemize}

\end{solutionbox}
\begin{mnemonicbox}
``FTO'' (Fast-PWM, Timer0, OCR0-control)

\end{mnemonicbox}
\begin{center}\rule{0.5\linewidth}{0.5pt}\end{center}

\subsection*{પ્રશ્ન 4(બ) OR [4
ગુણ]}\label{uxaaauxab0uxab6uxaa8-4uxaac-or-4-uxa97uxaa3}

\textbf{SPI ડિવાઇસમાંથી ડેટા રીડ કરવા માટેના સ્ટેપ્સ લખો.}

\begin{solutionbox}

\textbf{SPI રીડ સ્ટેપ્સ}:

{\def\LTcaptype{none} % do not increment counter
\begin{longtable}[]{@{}ll@{}}
\toprule\noalign{}
સ્ટેપ & ક્રિયા \\
\midrule\noalign{}
\endhead
\bottomrule\noalign{}
\endlastfoot
\textbf{1} & SPI કંટ્રોલ રજિસ્ટર (SPCR) કન્ફિગર કરો \\
\textbf{2} & સ્લેવ સિલેક્ટ કરવા માટે SS પિન લો કરો \\
\textbf{3} & SPDR માં ડમી ડેટા લખો \\
\textbf{4} & ટ્રાન્સમિશન પૂર્ણ થવાની રાહ જુઓ (SPIF ફ્લેગ) \\
\textbf{5} & SPDR થી રિસીવ કરેલો ડેટા રીડ કરો \\
\textbf{6} & સ્લેવ ડિસિલેક્ટ કરવા માટે SS પિન હાઇ કરો \\
\end{longtable}
}

\textbf{કોડ અમલીકરણ}:

\begin{verbatim}
// સ્ટેપ 1: SPI ને માસ્ટર તરીકે કન્ફિગર કરો
SPCR = (1{}SPE)|(1{}MSTR)|(1{}SPR0);

// સ્ટેપ 2: સ્લેવ સિલેક્ટ કરો
PORTB \&= {(}1{}SS);

// સ્ટેપ 3: ડમી બાઇટ મોકલો
SPDR = 0xFF;

// સ્ટેપ 4: પૂર્ણતાની રાહ જુઓ
while(!(SPSR \& (1{}SPIF)));

// સ્ટેપ 5: ડેટા રીડ કરો
data = SPDR;

// સ્ટેપ 6: સ્લેવ ડિસિલેક્ટ કરો
PORTB |= (1{}SS);
\end{verbatim}

\textbf{SPI ટાઇમિંગ}:

\begin{itemize}
\tightlist
\item
  \textbf{ક્લોક પોલેરિટી}: CPOL બિટ આઇડલ સ્ટેટ નક્કી કરે છે
\item
  \textbf{ક્લોક ફેઝ}: CPHA બિટ સેમ્પલિંગ એજ નક્કી કરે છે
\item
  \textbf{ડેટા ઓર્ડર}: MSB ફર્સ્ટ (ડિફોલ્ટ) અથવા LSB ફર્સ્ટ
\end{itemize}

\end{solutionbox}
\begin{mnemonicbox}
``CSWWRD'' (Configure, Select, Write-dummy, Wait,
Read-data, Deselect)

\end{mnemonicbox}
\begin{center}\rule{0.5\linewidth}{0.5pt}\end{center}

\subsection*{પ્રશ્ન 4(ક) OR [7
ગુણ]}\label{uxaaauxab0uxab6uxaa8-4uxa95-or-7-uxa97uxaa3}

\textbf{ATmega32 સાથે LM35 ઇન્ટરફેસિંગ ડાયાગ્રામ દોરી સમજાવો.}

\begin{solutionbox}

\begin{verbatim}
    LM35 Temperature Sensor
    
    +5V {-{-}{-}{-}{-} VCC (Pin 1)}
               |
    ATmega32   |    LM35
    PA0 {{-}{-}{-}{-}{-} OUTPUT (Pin 2)}
               |
    GND {-{-}{-}{-}{-} GND (Pin 3)}
    
    Optional: 0.1µF capacitor between 
    VCC and GND for noise filtering
\end{verbatim}

\textbf{LM35 સ્પેસિફિકેશન્સ}:

{\def\LTcaptype{none} % do not increment counter
\begin{longtable}[]{@{}ll@{}}
\toprule\noalign{}
પેરામીટર & મૂલ્ય \\
\midrule\noalign{}
\endhead
\bottomrule\noalign{}
\endlastfoot
\textbf{આઉટપુટ} & 10mV/^\circC \\
\textbf{રેન્જ} & 0^\circC થી 100^\circC \\
\textbf{સપ્લાય} & 4V થી 30V \\
\textbf{એક્યુરસી} & \pm0.5^\circC \\
\end{longtable}
}

\textbf{ટેમ્પરેચર રીડિંગ માટે ADC કોડ}:

\begin{verbatim}
\#include {avr/io.h}

unsigned int readTemperature(void)
\{
    unsigned int adcValue, temperature;
    
    // ADC કન્ફિગર કરો
    ADMUX = (1{}REFS0);  // AVCC રેફરન્સ, PA0
    ADCSRA = (1{}ADEN)|(1{}ADPS2)|(1{}ADPS1)|(1{}ADPS0);
    
    // કન્વર્ઝન સ્ટાર્ટ કરો
    ADCSRA |= (1{}ADSC);
    
    // પૂર્ણતાની રાહ જુઓ
    while(!(ADCSRA \& (1{}ADIF)));
    
    // ADC મૂલ્ય રીડ કરો
    adcValue = ADC;
    
    // ટેમ્પરેચરમાં કન્વર્ટ કરો
    // ADC = (Vin  1024) / Vref
    // Vin = (10mV/^)  Temp
    temperature = (adcValue * 500) / 1024;
    
    return temperature;
\}
\end{verbatim}

\textbf{ટેમ્પરેચર કેલ્ક્યુલેશન}:

\begin{itemize}
\tightlist
\item
  \textbf{ADC રેઝોલ્યુશન}: 10-bit (0-1023)
\item
  \textbf{રેફરન્સ વોલ્ટેજ}: 5V
\item
  \textbf{LM35 આઉટપુટ}: 10mV/^\circC
\item
  \textbf{ફોર્મ્યુલા}: Temp = (ADC \times 5000mV) / (1024 \times 10mV/^\circC)
\end{itemize}

\end{solutionbox}
\begin{mnemonicbox}
``VARC'' (Voltage-output, ADC-conversion,
Reference-5V, Calculation-formula)

\end{mnemonicbox}
\begin{center}\rule{0.5\linewidth}{0.5pt}\end{center}

\subsection*{પ્રશ્ન 5(અ) [3
ગુણ]}\label{uxaaauxab0uxab6uxaa8-5uxa85-3-uxa97uxaa3}

\textbf{Timer 0 માટે વર્કિંગ બ્લોક ડાયાગ્રામ દોરો.}

\begin{solutionbox}

\begin{center}
\textbf{Mermaid Diagram (Code)}
\begin{verbatim}
{Shaded}
{Highlighting}[]
graph LR
    A[System Clock] {-{-}{} B[Prescaler]}
    B {-{-}{} C[Timer/Counter 0]}
    C {-{-}{} D[Compare Unit]}
    C {-{-}{} E[Overflow Flag]}
    D {-{-}{} F[OCR0]}
    D {-{-}{} G[PWM Output]}
    H[External Clock] {-{-}{} B}
    
    style C fill:\#f9f,stroke:\#333,stroke{-width:4px}
{Highlighting}
{Shaded}
\end{verbatim}
\end{center}

\textbf{Timer0 કોમ્પોનન્ટ્સ}:

{\def\LTcaptype{none} % do not increment counter
\begin{longtable}[]{@{}ll@{}}
\toprule\noalign{}
કોમ્પોનન્ટ & કાર્ય \\
\midrule\noalign{}
\endhead
\bottomrule\noalign{}
\endlastfoot
\textbf{પ્રીસ્કેલર} & ક્લોક ડિવિઝન (1,8,64,256,1024) \\
\textbf{કાઉન્ટર} & 8-bit અપ કાઉન્ટર (0-255) \\
\textbf{કોમ્પેર યુનિટ} & કાઉન્ટરને OCR0 સાથે કોમ્પેર કરે છે \\
\textbf{ઓવરફ્લો} & કાઉન્ટર ઓવરફ્લો થાય ત્યારે ફ્લેગ સેટ કરે છે \\
\end{longtable}
}

\begin{itemize}
\tightlist
\item
  \textbf{ક્લોક સોર્સ}: ઇન્ટર્નલ ક્લોક અથવા એક્સટર્નલ પિન
\item
  \textbf{મોડ્સ}: નોર્મલ, CTC, ફાસ્ટ PWM, ફેઝ કરેક્ટ PWM
\item
  \textbf{ઇન્ટરપ્ટ}: ટાઇમર ઓવરફ્લો અને કોમ્પેર મેચ
\end{itemize}

\end{solutionbox}
\begin{mnemonicbox}
``PCCO'' (Prescaler, Counter, Compare, Overflow)

\end{mnemonicbox}
\begin{center}\rule{0.5\linewidth}{0.5pt}\end{center}

\subsection*{પ્રશ્ન 5(બ) [4
ગુણ]}\label{uxaaauxab0uxab6uxaa8-5uxaac-4-uxa97uxaa3}

\textbf{ATmega32 સાથે MAX7221 ઇન્ટરફેસિંગ ડાયાગ્રામ દોરો}

\begin{solutionbox}

\begin{verbatim}
ATmega32                    MAX7221
                           
PB5(MOSI) {-{-}{-}{-}{-}{-}{-}{-}{-}{-}{-} DIN (Pin 1)}
PB7(SCK)  {-{-}{-}{-}{-}{-}{-}{-}{-}{-}{-} CLK (Pin 13)}
PB4(SS)   {-{-}{-}{-}{-}{-}{-}{-}{-}{-}{-} CS  (Pin 12)}
                       
                       V+ (Pin 19) {{-}{-}{-} +5V}
                       GND(Pin 4,9) {{-}{-}{-} GND}
                       
         7{-Segment Display Connections:}
         SEG A{-G, DP connected to Pins 14{-}17, 20{-}23}
         DIG 0{-7 connected to Pins 2{-}3, 5{-}8, 10{-}11}
\end{verbatim}

\textbf{MAX7221 ફીચર્સ}:

{\def\LTcaptype{none} % do not increment counter
\begin{longtable}[]{@{}ll@{}}
\toprule\noalign{}
ફીચર & વર્ણન \\
\midrule\noalign{}
\endhead
\bottomrule\noalign{}
\endlastfoot
\textbf{ડિસ્પ્લે ડ્રાઇવર} & 8-digit 7-segment LED ડ્રાઇવર \\
\textbf{SPI ઇન્ટરફેસ} & સીરિયલ ડેટા ઇનપુટ \\
\textbf{કરન્ટ કંટ્રોલ} & એડજસ્ટેબલ સેગમેન્ટ કરન્ટ \\
\textbf{શટડાઉન મોડ} & પાવર સેવિંગ ફીચર \\
\end{longtable}
}

\textbf{ઇનિશિયલાઇઝેશન કોડ}:

\begin{verbatim}
void MAX7221\_init(void)
\{
    // SPI પિન્સ કન્ફિગર કરો
    DDRB |= (1{}PB5)|(1{}PB7)|(1{}PB4);  // MOSI, SCK, SS આઉટપુટ તરીકે
    
    // SPI ઇનિશિયલાઇઝ કરો
    SPCR = (1{}SPE)|(1{}MSTR)|(1{}SPR0);
    
    // MAX7221 વેક અપ કરો
    MAX7221\_write(0x0C, 0x01);  // શટડાઉન રજિસ્ટર
    
    // ડિકોડ મોડ સેટ કરો
    MAX7221\_write(0x09, 0xFF);  // બધા ડિજિટ્સ માટે BCD ડિકોડ
    
    // ઇન્ટેન્સિટી સેટ કરો
    MAX7221\_write(0x0A, 0x08);  // મધ્યમ બ્રાઇટનેસ
    
    // સ્કેન લિમિટ સેટ કરો
    MAX7221\_write(0x0B, 0x07);  // બધા 8 ડિજિટ્સ ડિસ્પ્લે કરો
\}
\end{verbatim}

\end{solutionbox}
\begin{mnemonicbox}
``SCD-ISS'' (SPI-interface, Current-control,
Decode-mode, Initialize-setup, Scan-limit)

\end{mnemonicbox}
\begin{center}\rule{0.5\linewidth}{0.5pt}\end{center}

\subsection*{પ્રશ્ન 5(ક) [7
ગુણ]}\label{uxaaauxab0uxab6uxaa8-5uxa95-7-uxa97uxaa3}

\textbf{વેધર મોનિટરિંગ સિસ્ટમ સમજાવો.}

\begin{solutionbox}

\textbf{સિસ્ટમ બ્લોક ડાયાગ્રામ}:

\begin{center}
\textbf{Mermaid Diagram (Code)}
\begin{verbatim}
{Shaded}
{Highlighting}[]
graph TD
    A[Temperature Sensor{br/{}LM35] {-}{-}{} E[ATmega32{}br/{}Microcontroller]}
    B[Humidity Sensor{br/{}DHT11] {-}{-}{} E}
    C[Pressure Sensor{br/{}BMP180] {-}{-}{} E}
    D[Light Sensor{br/{}LDR] {-}{-}{} E}
    E {-{-}{} F[LCD Display{}br/{}16x2]}
    E {-{-}{} G[Data Logger{}br/{}EEPROM]}
    E {-{-}{} H[Wireless Module{}br/{}ESP8266]}
    H {-{-}{} I[Cloud Server]}
    J[Power Supply{br/{}Battery/Solar] {-}{-}{} E}
{Highlighting}
{Shaded}
\end{verbatim}
\end{center}

\textbf{સિસ્ટમ કોમ્પોનન્ટ્સ}:

{\def\LTcaptype{none} % do not increment counter
\begin{longtable}[]{@{}lll@{}}
\toprule\noalign{}
કોમ્પોનન્ટ & કાર્ય & ઇન્ટરફેસ \\
\midrule\noalign{}
\endhead
\bottomrule\noalign{}
\endlastfoot
\textbf{LM35} & ટેમ્પરેચર માપન & ADC \\
\textbf{DHT11} & હ્યુમિડિટી અને ટેમ્પરેચર & ડિજિટલ I/O \\
\textbf{BMP180} & વાતાવરણીય દબાણ & I2C \\
\textbf{LCD} & લોકલ ડિસ્પ્લે & પેરેલલ \\
\textbf{ESP8266} & WiFi કનેક્ટિવિટી & UART \\
\textbf{EEPROM} & ડેટા સ્ટોરેજ & I2C \\
\end{longtable}
}

\textbf{ફીચર્સ અને એપ્લિકેશન્સ}:

\begin{itemize}
\tightlist
\item
  \textbf{રિયલ-ટાઇમ મોનિટરિંગ}: સતત સેન્સર ડેટા કલેક્શન
\item
  \textbf{ડેટા લોગિંગ}: EEPROM માં હિસ્ટોરિકલ ડેટા સ્ટોરેજ
\item
  \textbf{રિમોટ એક્સેસ}: ક્લાઉડ અપલોડ માટે WiFi કનેક્ટિવિટી
\item
  \textbf{પાવર મેનેજમેન્ટ}: સોલાર ચાર્જિંગ સાથે બેટરી બેકઅપ
\item
  \textbf{એલર્ટ સિસ્ટમ}: થ્રેશહોલ્ડ-બેસ્ડ વોર્નિંગ્સ
\item
  \textbf{એગ્રિકલ્ચરલ યુઝ}: ક્રોપ મોનિટરિંગ, ઇરિગેશન કંટ્રોલ
\item
  \textbf{હોમ ઓટોમેશન}: HVAC કંટ્રોલ, એનર્જી મેનેજમેન્ટ
\end{itemize}

\textbf{સોફ્ટવેર ફંક્શન્સ}:

\begin{itemize}
\tightlist
\item
  \textbf{સેન્સર રીડિંગ}: ADC કન્વર્ઝન, I2C કમ્યુનિકેશન
\item
  \textbf{ડેટા પ્રોસેસિંગ}: કેલિબ્રેશન, ફિલ્ટરિંગ, એવરેજિંગ
\item
  \textbf{ડિસ્પ્લે અપડેટ}: LCD ફોર્મેટિંગ, યુઝર ઇન્ટરફેસ
\item
  \textbf{કમ્યુનિકેશન}: WiFi ડેટા ટ્રાન્સમિશન, પ્રોટોકોલ હેન્ડલિંગ
\item
  \textbf{સ્ટોરેજ મેનેજમેન્ટ}: EEPROM રીડ/રાઇટ, ડેટા કમ્પ્રેશન
\end{itemize}

\end{solutionbox}
\begin{mnemonicbox}
``SMART-W'' (Sensors, Monitoring, Alert, Remote,
Temperature, Weather)

\end{mnemonicbox}
\begin{center}\rule{0.5\linewidth}{0.5pt}\end{center}

\subsection*{પ્રશ્ન 5(અ) OR [3
ગુણ]}\label{uxaaauxab0uxab6uxaa8-5uxa85-or-3-uxa97uxaa3}

\textbf{ટાઇમર/કાઉન્ટર કંટ્રોલ રજિસ્ટર TCCR0 દોરી સમજાવો.}

\begin{solutionbox}

\textbf{TCCR0 રજિસ્ટર બિટ સ્ટ્રક્ચર}:

\begin{verbatim}
Bit:    7     6     5     4     3     2     1    0
      +{-{-}{-}{-}+{-}{-}{-}{-}{-}+{-}{-}{-}{-}{-}+{-}{-}{-}{-}{-}+{-}{-}{-}{-}{-}+{-}{-}{-}{-}{-}+{-}{-}{-}{-}{-}+{-}{-}{-}{-}{-}+}
TCCR0 |FOC0|WGM00|COM01|COM00|WGM01| CS02| CS01| CS00|
      +{-{-}{-}{-}+{-}{-}{-}{-}{-}+{-}{-}{-}{-}{-}+{-}{-}{-}{-}{-}+{-}{-}{-}{-}{-}+{-}{-}{-}{-}{-}+{-}{-}{-}{-}{-}+{-}{-}{-}{-}{-}+}
\end{verbatim}

\textbf{બિટ ફંક્શન્સ ટેબલ}:

{\def\LTcaptype{none} % do not increment counter
\begin{longtable}[]{@{}lll@{}}
\toprule\noalign{}
બિટ & નામ & કાર્ય \\
\midrule\noalign{}
\endhead
\bottomrule\noalign{}
\endlastfoot
\textbf{FOC0} & Force Output Compare & ફોર્સ કોમ્પેર મેચ \\
\textbf{WGM01:00} & Waveform Generation & ટાઇમર મોડ સિલેક્શન \\
\textbf{COM01:00} & Compare Output Mode & આઉટપુટ પિન બિહેવિયર \\
\textbf{CS02:00} & Clock Select & પ્રીસ્કેલર સિલેક્શન \\
\end{longtable}
}

\textbf{ક્લોક સિલેક્ટ ઓપ્શન્સ}:

{\def\LTcaptype{none} % do not increment counter
\begin{longtable}[]{@{}ll@{}}
\toprule\noalign{}
CS02:00 & ક્લોક સોર્સ \\
\midrule\noalign{}
\endhead
\bottomrule\noalign{}
\endlastfoot
000 & કોઈ ક્લોક નહીં (બંધ) \\
001 & clk/1 (કોઈ પ્રીસ્કેલિંગ નહીં) \\
010 & clk/8 \\
011 & clk/64 \\
100 & clk/256 \\
101 & clk/1024 \\
110 & T0 પર એક્સટર્નલ ક્લોક (ફોલિંગ) \\
111 & T0 પર એક્સટર્નલ ક્લોક (રાઇઝિંગ) \\
\end{longtable}
}

\textbf{વેવફોર્મ જનરેશન મોડ્સ}:

{\def\LTcaptype{none} % do not increment counter
\begin{longtable}[]{@{}lll@{}}
\toprule\noalign{}
WGM01:00 & મોડ & વર્ણન \\
\midrule\noalign{}
\endhead
\bottomrule\noalign{}
\endlastfoot
00 & નોર્મલ & 0xFF સુધી કાઉન્ટ \\
01 & PWM, ફેઝ કરેક્ટ & અપ/ડાઉન કાઉન્ટ \\
10 & CTC & કોમ્પેર પર ટાઇમર ક્લિયર \\
11 & ફાસ્ટ PWM & 0xFF સુધી કાઉન્ટ \\
\end{longtable}
}

\end{solutionbox}
\begin{mnemonicbox}
``FWC-CS'' (Force, Waveform, Compare, Clock-Select)

\end{mnemonicbox}
\begin{center}\rule{0.5\linewidth}{0.5pt}\end{center}

\subsection*{પ્રશ્ન 5(બ) OR [4
ગુણ]}\label{uxaaauxab0uxab6uxaa8-5uxaac-or-4-uxa97uxaa3}

\textbf{મોટર ડ્રાઇવર L293D નું કાર્ય સમજાવો.}

\begin{solutionbox}

\textbf{L293D મોટર ડ્રાઇવર ફીચર્સ}:

{\def\LTcaptype{none} % do not increment counter
\begin{longtable}[]{@{}ll@{}}
\toprule\noalign{}
ફીચર & સ્પેસિફિકેશન \\
\midrule\noalign{}
\endhead
\bottomrule\noalign{}
\endlastfoot
\textbf{ચેનલ્સ} & ડ્યુઅલ H-બ્રિજ, 2 મોટર્સ \\
\textbf{સપ્લાય વોલ્ટેજ} & 4.5V થી 36V \\
\textbf{આઉટપુટ કરન્ટ} & ચેનલ દીઠ 600mA \\
\textbf{લોજિક વોલ્ટેજ} & 5V TTL કોમ્પેટિબલ \\
\textbf{પ્રોટેક્શન} & થર્મલ શટડાઉન \\
\end{longtable}
}

\textbf{પિન કન્ફિગરેશન}:

\begin{verbatim}
        L293D
    +{-{-}{-}{-}{-}{-}{-}{-}{-}+}
EN1 |1      16| VCC1 (+5V)
IN1 |2      15| IN4
OUT1|3      14| OUT4
GND |4      13| GND
GND |5      12| GND
OUT2|6      11| OUT3
IN2 |7      10| IN3
VCC2|8       9| EN2
    +{-{-}{-}{-}{-}{-}{-}{-}{-}+}
\end{verbatim}

\textbf{H-બ્રિજ ઓપરેશન}:

{\def\LTcaptype{none} % do not increment counter
\begin{longtable}[]{@{}lll@{}}
\toprule\noalign{}
IN1 & IN2 & મોટર એક્શન \\
\midrule\noalign{}
\endhead
\bottomrule\noalign{}
\endlastfoot
0 & 0 & સ્ટોપ (બ્રેક) \\
0 & 1 & CCW રોટેટ \\
1 & 0 & CW રોટેટ \\
1 & 1 & સ્ટોપ (બ્રેક) \\
\end{longtable}
}

\textbf{કંટ્રોલ ફંક્શન્સ}:

\begin{itemize}
\tightlist
\item
  \textbf{ડાયરેક્શન કંટ્રોલ}: IN1, IN2 રોટેશન ડાયરેક્શન નક્કી કરે છે
\item
  \textbf{સ્પીડ કંટ્રોલ}: એનેબલ પિન્સ (EN1, EN2) પર PWM
\item
  \textbf{ડ્યુઅલ સપ્લાય}: લોજિક માટે VCC1, મોટર પાવર માટે VCC2
\item
  \textbf{એનેબલ કંટ્રોલ}: EN પિન્સ મોટર ઓપરેશન એનેબલ/ડિસેબલ કરે છે
\end{itemize}

\textbf{એપ્લિકેશન્સ}:

\begin{itemize}
\tightlist
\item
  \textbf{રોબોટિક્સ}: ડિફરન્શિયલ ડ્રાઇવ રોબોટ્સ
\item
  \textbf{ઓટોમેશન}: કન્વેયર બેલ્ટ કંટ્રોલ
\item
  \textbf{RC વેહિકલ્સ}: મોટર સ્પીડ અને ડાયરેક્શન કંટ્રોલ
\end{itemize}

\end{solutionbox}
\begin{mnemonicbox}
``DHIE'' (Dual-channel, H-bridge, Input-control,
Enable-PWM)

\end{mnemonicbox}
\begin{center}\rule{0.5\linewidth}{0.5pt}\end{center}

\subsection*{પ્રશ્ન 5(ક) OR [7
ગુણ]}\label{uxaaauxab0uxab6uxaa8-5uxa95-or-7-uxa97uxaa3}

\textbf{ઓટોમેટિક જૂસ વેન્ડિંગ મશીન સમજાવો.}

\begin{solutionbox}

\textbf{સિસ્ટમ બ્લોક ડાયાગ્રામ}:

\begin{center}
\textbf{Mermaid Diagram (Code)}
\begin{verbatim}
{Shaded}
{Highlighting}[]
graph TD
    A[Keypad Input] {-{-}{} H[ATmega32{}br/{}Controller]}
    B[Coin Sensor] {-{-}{} H}
    C[LCD Display] {-{-}{} H}
    H {-{-}{} D[Pump Motors]}
    H {-{-}{} E[Solenoid Valves]}
    H {-{-}{} F[Coin Return{}br/{}Mechanism]}
    H {-{-}{} G[Level Sensors]}
    I[Power Supply] {-{-}{} H}
    J[Juice Containers] {-{-}{} D}
    D {-{-}{} K[Mixing Chamber]}
    E {-{-}{} K}
    K {-{-}{} L[Dispensing Unit]}
{Highlighting}
{Shaded}
\end{verbatim}
\end{center}

\textbf{સિસ્ટમ કોમ્પોનન્ટ્સ}:

{\def\LTcaptype{none} % do not increment counter
\begin{longtable}[]{@{}lll@{}}
\toprule\noalign{}
કોમ્પોનન્ટ & કાર્ય & ઇન્ટરફેસ \\
\midrule\noalign{}
\endhead
\bottomrule\noalign{}
\endlastfoot
\textbf{કીપેડ} & જૂસ સિલેક્શન & ડિજિટલ I/O \\
\textbf{કોઇન સેન્સર} & પેમેન્ટ ડિટેક્શન & ઇન્ટરપ્ટ \\
\textbf{LCD ડિસ્પ્લે} & યુઝર ઇન્ટરફેસ & પેરેલલ \\
\textbf{પંપ મોટર્સ} & જૂસ પંપિંગ & PWM કંટ્રોલ \\
\textbf{સોલેનોઇડ વાલ્વ} & ફ્લો કંટ્રોલ & ડિજિટલ આઉટપુટ \\
\textbf{લેવલ સેન્સર્સ} & કન્ટેનર મોનિટરિંગ & ADC/ડિજિટલ \\
\end{longtable}
}

\textbf{ઓપરેશન સિક્વન્સ}:

\begin{enumerate}
\tightlist
\item
  \textbf{મેન્યુ ડિસ્પ્લે}: ઉપલબ્ધ જૂસ અને કિંમતો બતાવો
\item
  \textbf{યુઝર સિલેક્શન}: કસ્ટમર કીપેડ વાયા જૂસ ટાઇપ સિલેક્ટ કરે છે
\item
  \textbf{પેમેન્ટ પ્રોસેસ}: કોઇન ઇન્સર્શન અને વેલિડેશન
\item
  \textbf{લેવલ ચેક}: ઇંગ્રીડિયન્ટ ઉપલબ્ધતા વેરિફાઇ કરો
\item
  \textbf{ડિસ્પેન્સિંગ}: સિક્વન્સમાં પંપ્સ અને વાલ્વ એક્ટિવેટ કરો
\item
  \textbf{મિક્સિંગ}: મિક્સિંગ રેશિયો અને ટાઇમ કંટ્રોલ કરો
\item
  \textbf{કમ્પ્લિશન}: કમ્પ્લિશન મેસેજ ડિસ્પ્લે કરો અને ચેન્જ રિટર્ન કરો
\end{enumerate}

\textbf{કંટ્રોલ અલ્ગોરિધમ}:

\begin{verbatim}
void dispensJuice(uint8\_t selection, uint16\_t amount)
\{
    // ઇંગ્રીડિયન્ટ લેવલ્સ ચેક કરો
    if(checkLevels(selection))
    \{
        // મિક્સિંગ રેશિયો કેલ્ક્યુલેટ કરો
        calculateRatio(selection);
        
        // ડિસ્પેન્સિંગ સિક્વન્સ સ્ટાર્ટ કરો
        activatePump(selection, amount);
        
        // મિક્સિંગ ટાઇમ કંટ્રોલ કરો
        startTimer(MIXING\_TIME);
        
        // ટ્રાન્ઝેક્શન પૂર્ણ કરો
        displayMessage("તમારા જૂસનો આનંદ માણો!");
    \}
    else
    \{
        displayMessage("ઇંગ્રીડિયન્ટ ઉપલબ્ધ નથી");
        returnCoins();
    \}
\}
\end{verbatim}

\textbf{ફીચર્સ}:

\begin{itemize}
\tightlist
\item
  \textbf{મલ્ટિપલ ફ્લેવર્સ}: વિવિધ જૂસ કોમ્બિનેશન્સ
\item
  \textbf{પેમેન્ટ સિસ્ટમ}: કોઇન એક્સેપ્ટન્સ અને ચેન્જ રિટર્ન
\item
  \textbf{ઇન્વેન્ટરી મેનેજમેન્ટ}: લેવલ મોનિટરિંગ અને એલર્ટ્સ
\item
  \textbf{યુઝર ઇન્ટરફેસ}: મેન્યુ ડિસ્પ્લે અને સિલેક્શન
\item
  \textbf{સેફ્ટી ફીચર્સ}: ઓવરફ્લો પ્રોટેક્શન, ઇમર્જન્સી સ્ટોપ
\item
  \textbf{મેઇન્ટેનન્સ મોડ}: સર્વિસ અને ક્લીનિંગ સાઇકલ્સ
\end{itemize}

\textbf{એપ્લિકેશન્સ}:

\begin{itemize}
\tightlist
\item
  \textbf{કમર્શિયલ}: શોપિંગ મોલ્સ, ઓફિસો, સ્કૂલો
\item
  \textbf{ઇન્ડસ્ટ્રિયલ}: ફેક્ટરી કેફેટેરિયા, હોસ્પિટલો
\item
  \textbf{પબ્લિક પ્લેસીસ}: એરપોર્ટ્સ, ટ્રેન સ્ટેશન્સ
\end{itemize}

\end{solutionbox}
\begin{mnemonicbox}
``JUMPS'' (Juice-selection, User-interface,
Mixing-control, Payment-system, Sensors-monitoring)

\end{mnemonicbox}

\end{document}
