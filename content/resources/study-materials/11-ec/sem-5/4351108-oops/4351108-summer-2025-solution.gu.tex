\documentclass{article}
% Adjust the relative path to point to the latex-templates directory

% content/resources/templates/preamble.tex
\usepackage[margin=0.6in]{geometry}
\author{Milav Dabgar}
\usepackage{amsmath,amssymb,amsthm}
\usepackage{booktabs}
\usepackage{multirow}
\usepackage{xcolor}
\usepackage{tcolorbox}
\tcbuselibrary{breakable,skins}
\usepackage[colorlinks=true,linkcolor=blue]{hyperref}
\usepackage{titlesec}
\usepackage{enumitem}
\usepackage{tikz}
\usepackage{pgfplots}
\usepackage{circuitikz}
\usepackage[version=4]{mhchem}
\usepackage{longtable}
\usepackage{array}
\usepackage{float}
\usepackage{caption}
\usepackage{listings}

\lstset{
  basicstyle=\small\ttfamily,
  breaklines=true,
  breakatwhitespace=false,
  postbreak=\mbox{\textcolor{red}{$\hookrightarrow$}\space},
  float=false,
  numbers=left,
  numberstyle=\tiny\color{gray},
  numbersep=10pt,
  xleftmargin=2em,
  keywordstyle=\color{blue},
  commentstyle=\color{green!60!black},
  stringstyle=\color{purple},
  backgroundcolor=\color{gray!5},
  showstringspaces=false,
  tabsize=2,
  captionpos=b,
  keepspaces=true,
  columns=flexible
}

\pgfplotsset{compat=1.18}
\usetikzlibrary{shapes,arrows,positioning,calc,patterns,decorations.pathmorphing,decorations.markings,arrows.meta}

% Color scheme
\definecolor{headcolor}{RGB}{0,102,204}
\definecolor{keycolor}{RGB}{220,20,60}
\definecolor{solutioncolor}{RGB}{34,139,34}
\definecolor{mnemoniccolor}{RGB}{148,0,211}
\definecolor{codecolor}{RGB}{0,0,100}

% Spacing
\setlength{\parskip}{3pt}
\setlist[itemize]{nosep}
\setlist[enumerate]{nosep}

% Title formatting
\titleformat{\section}{\Large\bfseries\color{headcolor}}{\thesection}{1em}{}
\titleformat{\subsection}{\large\bfseries\color{headcolor}}{\thesubsection}{1em}{}

% Pandoc tightlist compatibility
\providecommand{\tightlist}{%
  \setlength{\itemsep}{0pt}\setlength{\parskip}{0pt}}

% Pandoc longtable compatibility
\newcounter{none}
\def\thenone{}


% content/resources/templates/gujarati-boxes.tex
\usepackage{fontspec}
\usepackage{polyglossia}

% Set Gujarati as main language (document is primarily in Gujarati)
% Note: gloss-gujarati.ldf doesn't exist in polyglossia, but it will use hyphenation patterns
\setdefaultlanguage{gujarati}
\setotherlanguage{english}

% Configure Gujarati font properly
% Use Language=Default to prevent polyglossia from trying to add language-specific features
% that don't exist for Gujarati, which causes "empty feature" warnings
\newfontfamily\gujaratifont[Script=Gujarati,AutoFakeBold=2.5,AutoFakeSlant=0.3]{Noto Sans Gujarati}
\setmainfont[Script=Gujarati,AutoFakeBold=2.5,AutoFakeSlant=0.3]{Noto Sans Gujarati}
% Use Noto Sans Gujarati for monospace to support Gujarati in text
\setmonofont[Scale=0.9]{Noto Sans Gujarati}

% Configure English to use the same font
\newfontfamily\englishfont[Script=Gujarati,AutoFakeBold=2.5,AutoFakeSlant=0.3]{Noto Sans Gujarati}

% Translations for polyglossia
\gappto\captionsgujarati{
  \renewcommand{\tablename}{કોષ્ટક}
  \renewcommand{\figurename}{આકૃતિ}
}

% Helper for TikZ nodes to ensure Gujarati font
\newcommand{\gu}[1]{{\gujaratifont #1}}

% Custom environments
\newtcolorbox{solutionbox}{
    breakable,
    enhanced,
    colback=solutioncolor!5!white,
    colframe=solutioncolor!75!black,
    fonttitle=\bfseries,
    title=જવાબ
}

\newtcolorbox{solutionboxnobreak}{
 colback=solutioncolor!5!white,
 colframe=solutioncolor!75!black,
 fonttitle=\bfseries,
 title=જવાબ
}

\newtcolorbox{keyformula}{
 breakable,
 enhanced,
 colback=keycolor!5!white,
 colframe=keycolor!75!black,
 fonttitle=\bfseries,
 title=રાસાયણિક સમીકરણ/સૂત્ર
}

\newtcolorbox{mnemonicbox}{
 breakable,
 enhanced,
 colback=mnemoniccolor!5!white,
 colframe=mnemoniccolor!75!black,
 fonttitle=\bfseries,
 title=મેમરી ટ્રીક
}


% Custom commands for GTU solutions
% This file defines semantic commands for consistent formatting

% Question command with automatic formatting
\newcommand{\question}[2]{%
  \section*{Question #1}%
  \textbf{#2}%
}

% OR question variant
\newcommand{\questionor}[2]{%
  \section*{Question #1 OR}%
  \textbf{#2}%
}

% Proper table environment with caption
\newenvironment{answertable}[1]{%
  \begin{table}[htbp]
  \centering
  \caption{#1}
}{%
  \end{table}
}

% Proper figure environment for diagrams
\newenvironment{answerdiagram}[1]{%
  \begin{figure}[htbp]
  \centering
  \caption{#1}
}{%
  \end{figure}
}

% Semantic markup for key terms
\newcommand{\keyword}[1]{\textbf{#1}}
\newcommand{\code}[1]{\texttt{#1}}
\newcommand{\classname}[1]{\texttt{#1}}
\newcommand{\methodname}[1]{\texttt{#1}}

% Proper quotation marks
\newcommand{\mnemonic}[1]{``#1''}


\title{OOPS અને પાયથોન પ્રોગ્રામિંગ (4351108) - સમર 2025 સોલ્યુશન}
\date{May 14, 2025}

\begin{document}
\maketitle

\questionmarks{1(a)}{3}{Python માં for લૂપનો ઉદ્દેશ્ય શું છે? ઉદાહરણ સાથે સમજાવો.}

\begin{solutionbox}
for લૂપનો ઉપયોગ કોઈ sequence (જેમ કે લિસ્ટ, ટપલ, સ્ટ્રિંગ) અથવા અન્ય iterable ઓબ્જેક્ટ પર પુનરાવર્તન કરવા માટે અને sequence ના દરેક આઇટમ માટે કોડનો બ્લોક ચલાવવા માટે થાય છે.

\textbf{કોડ ઉદાહરણ:}
\begin{lstlisting}[language=Python]
# દરેક ફળને લિસ્ટમાંથી પ્રિન્ટ કરો
fruits = ["apple", "banana", "cherry"]
for fruit in fruits:
    print(fruit)
\end{lstlisting}

\begin{itemize}
    \item \keyword{પુનરાવર્તન}: સ્વયંસંચાલિત રીતે દરેક આઇટમ માટે કોડ પુનરાવર્તિત કરે છે
    \item \keyword{સરળતા}: કાઉન્ટર્સ સાથે while લૂપ્સ કરતાં સ્વચ્છ
\end{itemize}
\end{solutionbox}

\begin{mnemonicbox}
\mnemonic{દરેક આઇટમ માટે કરો}
\end{mnemonicbox}

\questionmarks{1(b)}{4}{Python માં variable ડિફાઇન કરવાના નિયમો જણાવો અને Python માં ડેટાપ્રકારો (data types) ની યાદી આપો.}

\begin{solutionbox}
\textbf{વેરિએબલ ડિફાઈન કરવાના નિયમો:}
\begin{center}
\captionof{table}{વેરિએબલ નિયમો}
\begin{tabulary}{\linewidth}{|L|L|L|}
\hline
\textbf{નિયમ} & \textbf{ઉદાહરણ} & \textbf{અમાન્ય ઉદાહરણ} \\ \hline
અક્ષર અથવા અંડરસ્કોરથી શરૂ થવું જોઈએ & \code{name = "John"} & \code{1name = "John"} \\ \hline
અક્ષરો, નંબરો, અંડરસ્કોર સમાવિષ્ટ કરી શકે & \code{user\_1 = "Alice"} & \code{user-1 = "Alice"} \\ \hline
કેસ-સેન્સિટિવ & \code{age} $\neq$ \code{Age} & - \\ \hline
રિઝર્વ્ડ કીવર્ડનો ઉપયોગ ન કરી શકાય & \code{count = 5} & \code{if = 5} \\ \hline
\end{tabulary}
\end{center}

\textbf{પાયથોન ડેટા ટાઈપ્સ:}
\begin{center}
\captionof{table}{ડેટા ટાઈપ્સ}
\begin{tabulary}{\linewidth}{|L|L|L|}
\hline
\textbf{ડેટા ટાઈપ} & \textbf{વિવરણ} & \textbf{ઉદાહરણ} \\ \hline
int & પૂર્ણાંક સંખ્યાઓ & \code{x = 10} \\ \hline
float & દશાંશ સંખ્યાઓ & \code{y = 10.5} \\ \hline
str & ટેક્સ્ટ સ્ટ્રિંગ્સ & \code{name = "John"} \\ \hline
bool & બૂલિયન મૂલ્યો & \code{is\_active = True} \\ \hline
list & ક્રમબદ્ધ, બદલી શકાય તેવા & \code{["apple", "banana"]} \\ \hline
tuple & ક્રમબદ્ધ, બદલી ન શકાય તેવા & \code{(10, 20)} \\ \hline
dict & કી-વેલ્યુ જોડી & \code{\{"name": "John"\}} \\ \hline
set & અનોર્ડર્ડ અનન્ય & \code{\{1, 2, 3\}} \\ \hline
\end{tabulary}
\end{center}
\end{solutionbox}

\begin{mnemonicbox}
\mnemonic{SILB-DTS: String, Integer, List, Boolean, Dictionary, Tuple, Set}
\end{mnemonicbox}

\questionmarks{1(c)}{7}{1 થી N સુધીના પ્રાઇમ નંબર પ્રિન્ટ કરવા પ્રોગ્રામ બનાવો.}

\begin{solutionbox}
\begin{lstlisting}[language=Python]
def print_primes(n):
    print("1 અને", n, "વચ્ચેના પ્રાઇમ નંબરો:")
    
    for num in range(2, n + 1):
        is_prime = True
        
        # Check if num is divisible by any number from 2 to sqrt(num)
        for i in range(2, int(num**0.5) + 1):
            if num % i == 0:
                is_prime = False
                break
                
        if is_prime:
            print(num, end=" ")

# Get input from user
N = int(input("N નંબર દાખલ કરો: "))
print_primes(N)
\end{lstlisting}

\textbf{એલ્ગોરિધમ ડાયાગ્રામ:}
\begin{center}
\begin{tikzpicture}[gtu flow]
    \node [gtu start] (start) {શરુઆત};
    \node [gtu input, below=0.5cm of start] (input) {N દાખલ કરો};
    \node [gtu process, below=0.5cm of input] (init) {num = 2 સેટ કરો};
    \node [gtu decision, below=0.5cm of init] (checkN) {num $\le$ N?};
    \node [gtu process, below left=0.5cm and 0cm of checkN, xshift=-2cm] (assume) {ધારો કે Prime};
    \node [gtu stop, right=2cm of checkN] (end) {સમાપ્ત};
    
    \node [gtu decision, below=0.5cm of assume] (checkI) {i $\le \sqrt{num}$?};
    \node [gtu decision, below=0.5cm of checkI] (div) {num \% i == 0?};
    \node [gtu output, right=2cm of checkI] (print) {num પ્રિન્ટ કરો};
    
    \path [gtu arrow] (start) -- (input);
    \path [gtu arrow] (input) -- (init);
    \path [gtu arrow] (init) -- (checkN);
    \path [gtu arrow] (checkN) -- node[left] {હા} (assume);
    \path [gtu arrow] (checkN) -- node[above] {ના} (end);
    \path [gtu arrow] (assume) -- (checkI);
    \path [gtu arrow] (checkI) -- node[above] {ના} (print);
    \path [gtu arrow] (checkI) -- node[right] {હા} (div);
    \path [gtu arrow] (div) -| node[right] {ના} ([xshift=0.5cm]checkI.east) -- (checkI);
    \path [gtu arrow] (div.west) -- node[above] {હા} ++(-0.5,0) |- (checkN);
    \path [gtu arrow] (print) |- (checkN);
\end{tikzpicture}
\captionof{figure}{Prime Number Algorithm}
\end{center}

\begin{itemize}
    \item \keyword{ટાઇમ કોમ્પ્લેક્સિટી}: $O(N\sqrt{N})$
    \item \keyword{સ્પેસ કોમ્પ્લેક્સિટી}: $O(1)$
\end{itemize}
\end{solutionbox}

\begin{mnemonicbox}
\mnemonic{ભાગ કરીને પ્રાઇમ નક્કી કરો}
\end{mnemonicbox}

\questionmarks{1(c OR)}{7}{Python માં break, continue, અને pass સ્ટેટમેન્ટનું કાર્ય અને ઉદાહરણ સાથે સમજાવો.}

\begin{solutionbox}
\begin{center}
\captionof{table}{કંટ્રોલ સ્ટેટમેન્ટ્સ}
\begin{tabulary}{\linewidth}{|L|L|L|}
\hline
\textbf{સ્ટેટમેન્ટ} & \textbf{ઉદ્દેશ} & \textbf{ઉદાહરણ} \\ \hline
break & લૂપને સંપૂર્ણપણે સમાપ્ત કરે છે & લૂપ બંધ કરો \\ \hline
continue & આગલા પુનરાવર્તન પર જાય છે & આઇટમ્સ છોડો \\ \hline
pass & કંઈ કરતું નથી (પ્લેસહોલ્ડર) & ભવિષ્યનો કોડ \\ \hline
\end{tabulary}
\end{center}

\textbf{ફ્લો કંટ્રોલ ડાયાગ્રામ:}
\begin{center}
\begin{tikzpicture}[gtu flow]
    \node [gtu start] (start) {લૂપ શરૂઆત};
    \node [gtu decision, below=0.5cm of start] (cond) {શરત};
    \node [gtu decision, below=0.5cm of cond] (break) {break?};
    \node [gtu decision, below=0.5cm of break] (cont) {continue?};
    \node [gtu decision, below=0.5cm of cont] (pass) {pass?};
    \node [gtu process, below=0.5cm of pass] (exec) {કોડ એક્ઝિક્યુટ કરો};
    \node [gtu stop, right=2cm of break] (exit) {લૂપ બહાર};
    
    \path [gtu arrow] (start) -- (cond);
    \path [gtu arrow] (cond) -- node[right] {સાચી} (break);
    \path [gtu arrow] (break) -- node[above] {હા} (exit);
    \path [gtu arrow] (break) -- node[right] {ના} (cont);
    \path [gtu arrow] (cont) -- node[right] {ના} (pass);
    \path [gtu arrow] (cont.west) -- node[above] {હા} ++(-0.5,0) |- (cond);
    \path [gtu arrow] (pass) -- (exec);
    \path [gtu arrow] (exec.west) -- ++(-0.5,0) |- (cond);
\end{tikzpicture}
\captionof{figure}{Loop Control Logic}
\end{center}

\textbf{ઉદાહરણ કોડ:}
\begin{lstlisting}[language=Python]
for i in range(5):
    if i == 2: continue  # Skip 2
    if i == 4: break     # Stop at 4
    if i == 0: pass      # Do nothing
    print(i)
# Output: 0, 1, 3
\end{lstlisting}
\end{solutionbox}

\begin{mnemonicbox}
\mnemonic{BCP - સંપૂર્ણપણે બંધ કરો, આંશિક રીતે ચાલુ રાખો, શાંતિથી પસાર થાઓ}
\end{mnemonicbox}

\questionmarks{2(a)}{3}{યુઝરે આપેલ વર્ષ લીપ વર્ષ છે કે નહીં તે માટે પ્રોગ્રામ બનાવો.}

\begin{solutionbox}
\begin{lstlisting}[language=Python]
year = int(input("વર્ષ દાખલ કરો: "))

if (year % 4 == 0 and year % 100 != 0) or (year % 400 == 0):
    print(f"{year} લીપ વર્ષ છે")
else:
    print(f"{year} લીપ વર્ષ નથી")
\end{lstlisting}

\textbf{નિર્ણય વૃક્ષ:}
\begin{center}
\begin{tikzpicture}[gtu flow]
    \node [gtu start] (start) {શરુઆત};
    \node [gtu input, below=0.5cm of start] (input) {વર્ષ દાખલ કરો};
    \node [gtu decision, below=0.5cm of input] (div4) {વર્ષ \% 4 == 0?};
    \node [gtu decision, below left=1cm and -1cm of div4] (div100) {\% 100 == 0?};
    \node [gtu decision, below=1cm of div100] (div400) {\% 400 == 0?};
    \node [gtu block, below=1cm of div400] (leap) {લીપ વર્ષ છે};
    \node [gtu block, right=2cm of div100] (notleap) {લીપ વર્ષ નથી};

    \path [gtu arrow] (start) -- (input);
    \path [gtu arrow] (input) -- (div4);
    \path [gtu arrow] (div4) -- node[left] {હા} (div100);
    \path [gtu arrow] (div4) -| node[above] {ના} (notleap);
    \path [gtu arrow] (div100) -- node[left] {હા} (div400);
    \path [gtu arrow] (div100) -- node[right] {ના} (leap); 
    \path [gtu arrow] (div400) -- node[left] {હા} (leap);
    \path [gtu arrow] (div400) -| node[above] {ના} (notleap);
\end{tikzpicture}
\captionof{figure}{Leap Year Logic}
\end{center}
\end{solutionbox}

\begin{mnemonicbox}
\mnemonic{4 હા, 100 ના, 400 હા}
\end{mnemonicbox}

\questionmarks{2(b)}{4}{Python માં લિસ્ટ અને ટ્યુપલ વચ્ચેના મુખ્ય તફાવત શું છે?}

\begin{solutionbox}
\begin{center}
\captionof{table}{લિસ્ટ અને ટ્યુપલ}
\begin{tabulary}{\linewidth}{|L|L|L|}
\hline
\textbf{વિશેષતા} & \textbf{લિસ્ટ} & \textbf{ટ્યુપલ} \\ \hline
સિન્ટેક્સ & \code{[]} & \code{()} \\ \hline
પરિવર્તનશીલતા & મ્યુટેબલ (બદલી શકાય) & ઇમ્યુટેબલ (બદલી ન શકાય) \\ \hline
પર્ફોર્મન્સ & ધીમું & ઝડપી \\ \hline
ઉપયોગ & જ્યારે ફેરફાર જરૂરી હોય & જ્યારે ડેટા ફિક્સ હોય \\ \hline
મેમરી & વધુ મેમરી & ઓછી મેમરી \\ \hline
\end{tabulary}
\end{center}

\textbf{તુલના ડાયાગ્રામ:}
\begin{center}
\begin{tikzpicture}[gtu flow]
    \node [gtu block] (list) {\textbf{List}\\ \code{["a", "b"]}\\ \code{.append("c")}};
    \node [gtu block, right=2cm of list] (tuple) {\textbf{Tuple}\\ \code{("a", "b")}\\ Read Only};
    
    \draw [gtu arrow] (list) -- node[above] {Mutable} (tuple); 
    \node [below=0.2cm of list] {બદલી શકાય};
    \node [below=0.2cm of tuple] {બદલી ન શકાય};
\end{tikzpicture}
\captionof{figure}{List vs Tuple}
\end{center}
\end{solutionbox}

\begin{mnemonicbox}
\mnemonic{LIST - બદલી શકાય, TUPLE - બદલી ન શકાય}
\end{mnemonicbox}

\questionmarks{2(c)}{7}{યુઝરે દાખલ કરેલ તમામ positive number છે કે નહી તે શોધવાનો પ્રોગ્રામ બનાવો. જ્યારે યુઝર negative number દાખલ કરે, ત્યારે ઇનપુટ લેવાનું બંધ કરો અને તમામ positive number નો સરવાળો કરો.}

\begin{solutionbox}
\begin{lstlisting}[language=Python]
def sum_positives():
    total_sum = 0
    while True:
        num = float(input("નંબર દાખલ કરો (negative બંધ કરવા): "))
        if num < 0:
            break
        total_sum += num
    print(f"બધા પોઝિટિવ નંબરનો સરવાળો: {total_sum}")

sum_positives()
\end{lstlisting}

\textbf{પ્રક્રિયા ફ્લો:}
\begin{center}
\begin{tikzpicture}[gtu flow]
    \node [gtu start] (start) {શરુઆત};
    \node [gtu process, below=0.5cm of start] (init) {total = 0};
    \node [gtu input, below=0.5cm of init] (input) {નંબર દાખલ કરો};
    \node [gtu decision, below=0.5cm of input] (check) {નંબર < 0?};
    \node [gtu process, right=1.5cm of check] (add) {total += નંબર};
    \node [gtu output, below=0.5cm of check] (print) {સરવાળો દર્શાવો};
    \node [gtu stop, below=0.5cm of print] (end) {અંત};

    \path [gtu arrow] (start) -- (init);
    \path [gtu arrow] (init) -- (input);
    \path [gtu arrow] (input) -- (check);
    \path [gtu arrow] (check) -- node[right] {ના} (add);
    \path [gtu arrow] (add) |- (input);
    \path [gtu arrow] (check) -- node[right] {હા} (print);
    \path [gtu arrow] (print) -- (end);
\end{tikzpicture}
\captionof{figure}{Summation Logic}
\end{center}
\end{solutionbox}

\begin{mnemonicbox}
\mnemonic{નેગેટિવ આવે ત્યાં સુધી સરવાળો કરો}
\end{mnemonicbox}

\questionmarks{2(a OR)}{3}{તમે આપેલ ત્રણ number માંથી મોટો number શોધવાનું પ્રોગ્રામ બનાવો.}

\begin{solutionbox}
\begin{lstlisting}[language=Python]
n1 = float(input("પહેલી સંખ્યા: "))
n2 = float(input("બીજી સંખ્યા: "))
n3 = float(input("ત્રીજી સંખ્યા: "))

if n1 >= n2 and n1 >= n3:
    mx = n1
elif n2 >= n1 and n2 >= n3:
    mx = n2
else:
    mx = n3

print(f"મહત્તમ સંખ્યા: {mx}")
\end{lstlisting}

\textbf{તુલના લોજિક:}
\begin{center}
\begin{tikzpicture}[gtu flow]
    \node [gtu input] (in) {ત્રણ સંખ્યા દાખલ કરો};
    \node [gtu decision, below=0.5cm of in] (c1) {n1 મહત્તમ?};
    \node [gtu decision, right=1cm of c1] (c2) {n2 મહત્તમ?};
    \node [gtu process, below=0.5cm of c1] (r1) {max=n1};
    \node [gtu process, below=0.5cm of c2] (r2) {max=n2};
    \node [gtu process, right=1cm of r2] (r3) {max=n3};
    
    \path [gtu arrow] (in) -- (c1);
    \path [gtu arrow] (c1) -- node[right] {હા} (r1);
    \path [gtu arrow] (c1) -- node[above] {ના} (c2);
    \path [gtu arrow] (c2) -- node[right] {હા} (r2);
    \path [gtu arrow] (c2) -- node[above] {ના} (r3);
\end{tikzpicture}
\captionof{figure}{Maximum Finder Logic}
\end{center}
\end{solutionbox}

\begin{mnemonicbox}
\mnemonic{દરેકની તુલના કરો, મોટામાં મોટો લો}
\end{mnemonicbox}

\questionmarks{2(b OR)}{4}{str = "abcdefghijklmnopqrstuvwxyz" આપેલ છે. ઉપરોક્ત સ્ટ્રિંગમાંથી દરેક બીજાં અક્ષર જુદો કાઢવા માટે Python પ્રોગ્રામ લખો.}

\begin{solutionbox}
\begin{lstlisting}[language=Python]
s = "abcdefghijklmnopqrstuvwxyz"
# સિન્ટેક્સ: [start:end:step]
result = s[0::2]
print("દરેક બીજો અક્ષર:", result)
# આઉટપુટ: acegikmoqsuwy
\end{lstlisting}

\textbf{સ્ટ્રિંગ સ્લાઇસિંગ ડાયાગ્રામ:}
\begin{center}
\begin{tikzpicture}[gtu flow]
    \foreach \x/\l in {0/a, 1/b, 2/c, 3/d, 4/e, 5/f} {
        \node[draw, minimum size=0.6cm] (n\x) at (\x*0.8, 0) {\l};
        \node[font=\footnotesize, below=0.1cm of n\x] {\x};
    }
    \node[right=0.2cm of n5] {...};
    
    \draw[->, thick, red] (n0.north) -- ++(0, 0.3) node[above] {Start};
    \draw[->, thick, blue] (n0) to[bend left] (n2);
    \draw[->, thick, blue] (n2) to[bend left] (n4);
    
    \node[above=0.8cm of n2] {Step=2};
\end{tikzpicture}
\captionof{figure}{String Slicing Step 2}
\end{center}
\end{solutionbox}

\begin{mnemonicbox}
\mnemonic{સ્લાઇસ સ્ટેપ સિલેક્ટર}
\end{mnemonicbox}

\questionmarks{2(c OR)}{7}{વિદ્યાર્થીઓના નામ અને તેમના માર્ક્સ સંગ્રહિત કરવા માટે ડિક્શનરી બનાવવાનું Python પ્રોગ્રામ લખો. 75 થી વધુ માર્ક્સ મેળવનાર વિદ્યાર્થીઓના નામ ડિસ્પ્લે કરવો.}

\begin{solutionbox}
\begin{lstlisting}[language=Python]
students = {}
n = int(input("વિદ્યાર્થીઓની સંખ્યા: "))

# ઇનપુટ લૂપ
for i in range(n):
    name = input("નામ: ")
    marks = float(input("માર્ક્સ: "))
    students[name] = marks

print("\n75 થી વધુ માર્ક્સ મેળવનાર:")
for name, marks in students.items():
    if marks > 75:
        print(f"{name}: {marks}")
\end{lstlisting}

\textbf{પ્રક્રિયા ડાયાગ્રામ:}
\begin{center}
\begin{tikzpicture}[gtu flow]
    \node [gtu start] (start) {શરુઆત};
    \node [gtu block, below=0.5cm of start] (dict) {ખાલી ડિક્શનરી};
    \node [gtu process, below=0.5cm of dict] (loop) {ઇનપુટ લૂપ};
    \node [gtu block, below=0.5cm of loop] (data) {ડેટા સંગ્રહિત};
    \node [gtu decision, below=0.5cm of data] (check) {માર્ક્સ > 75?};
    \node [gtu output, right=1cm of check] (disp) {નામ દર્શાવો};
    
    \path [gtu arrow] (start) -- (dict);
    \path [gtu arrow] (dict) -- (loop);
    \path [gtu arrow] (loop) -- (data);
    \path [gtu arrow] (data) -- (check);
    \path [gtu arrow] (check) -- node[above] {હા} (disp);
    \path [gtu arrow] (check) -- node[left] {ના} ++(0,-1);
    \path [gtu arrow] (disp) |- ++(0,-1);
\end{tikzpicture}
\captionof{figure}{Dictionary Filtering}
\end{center}
\end{solutionbox}

\begin{mnemonicbox}
\mnemonic{બધું સંગ્રહો, કેટલાક ફિલ્ટર કરો}
\end{mnemonicbox}

\questionmarks{3(a)}{3}{સ્પેસને બહાર રાખીને સ્ટ્રિંગની લંબાઈ શોધવાનો પ્રોગ્રામ લખો.}

\begin{solutionbox}
\begin{lstlisting}[language=Python]
s = input("સ્ટ્રિંગ દાખલ કરો: ")
no_spaces = s.replace(" ", "")
length = len(no_spaces)
print(f"સ્પેસ વિના લંબાઈ: {length}")
\end{lstlisting}

\textbf{વિઝ્યુલાઇઝેશન:}
\begin{center}
\begin{tikzpicture}[gtu flow]
    \node [gtu block] (orig) {"Hello World"};
    \node [gtu process, right=1cm of orig] (rep) {replace(" ","")};
    \node [gtu block, right=1cm of rep] (new) {"HelloWorld"};
    \node [gtu output, below=0.5cm of new] (len) {લંબાઈ: 10};
    
    \path [gtu arrow] (orig) -- (rep);
    \path [gtu arrow] (rep) -- (new);
    \path [gtu arrow] (new) -- (len);
\end{tikzpicture}
\captionof{figure}{Space Removal}
\end{center}
\end{solutionbox}

\begin{mnemonicbox}
\mnemonic{અક્ષરો ગણો, સ્પેસ છોડો}
\end{mnemonicbox}

\questionmarks{3(b)}{4}{Python માં ડિક્શનરી methods યાદી આપો અને દરેકને યોગ્ય ઉદાહરણ સાથે સમજાવો.}

\begin{solutionbox}
\begin{center}
\captionof{table}{ડિક્શનરી મેથડ્સ}
\begin{tabulary}{\linewidth}{|L|L|L|}
\hline
\textbf{મેથડ} & \textbf{વિવરણ} & \textbf{ઉદાહરણ} \\ \hline
\code{get(k)} & કી માટે મૂલ્ય મેળવો & \code{d.get('a')} \\ \hline
\code{keys()} & બધી કી મેળવો & \code{list(d.keys())} \\ \hline
\code{values()} & બધા મૂલ્યો મેળવો & \code{list(d.values())} \\ \hline
\code{items()} & (key, value) જોડી & \code{d.items()} \\ \hline
\code{pop(k)} & આઇટમ દૂર કરો & \code{d.pop('a')} \\ \hline
\code{update()} & ડિક્શનરી અપડેટ કરો & \code{d.update(d2)} \\ \hline
\code{clear()} & ડિક્શનરી ખાલી કરો & \code{d.clear()} \\ \hline
\end{tabulary}
\end{center}
\end{solutionbox}

\begin{mnemonicbox}
\mnemonic{GCUP-KPIV}
\end{mnemonicbox}

\questionmarks{3(c)}{7}{Python ના લિસ્ટ ડેટા ટાઇપને સમજાવો.}

\begin{solutionbox}
લિસ્ટ ક્રમબદ્ધ, પરિવર્તનશીલ સંગ્રહ છે જે ડુપ્લિકેટ ઘટકો અને મિશ્ર પ્રકારોને મંજૂરી આપે છે.

\textbf{મુખ્ય વિશેષતાઓ:}
\begin{itemize}
    \item \textbf{ક્રમબદ્ધ}: આઇટમ્સ ક્રમ જાળવી રાખે છે.
    \item \textbf{પરિવર્તનશીલ}: આઇટમ્સ ઉમેરી, દૂર કરી, બદલી શકાય છે.
    \item \textbf{વિવિધ}: int, str, float એકસાથે સ્ટોર કરી શકાય છે.
\end{itemize}

\textbf{લિસ્ટ ઓપરેશન્સ ડાયાગ્રામ:}
\begin{center}
\begin{tikzpicture}[gtu flow]
    \node [gtu block] (init) {\code{[1, 2]}};
    \node [gtu process, right=1cm of init] (append) {\code{.append(3)}};
    \node [gtu block, right=1cm of append] (res1) {\code{[1, 2, 3]}};
    \node [gtu process, below=1cm of init] (pop) {\code{.pop(0)}};
    \node [gtu block, below=1cm of res1] (res2) {\code{[2, 3]}};
    
    \path [gtu arrow] (init) -- (append);
    \path [gtu arrow] (append) -- (res1);
    \path [gtu arrow] (res1) -- (pop);
    \path [gtu arrow] (pop) -- (res2);
\end{tikzpicture}
\captionof{figure}{List Operations}
\end{center}
\end{solutionbox}

\begin{mnemonicbox}
\mnemonic{CAMP-IS: Create, Access, Modify, Process}
\end{mnemonicbox}

\questionmarks{3(a OR)}{3}{યુઝર પાસેથી સ્ટ્રિંગ ઇનપુટ લેવા માટેનું પ્રોગ્રામ લખો અને નવી સ્ટ્રિંગ બનાવ્યા વિના તેને reverse order માં છાપો.}

\begin{solutionbox}
\begin{lstlisting}[language=Python]
s = input("સ્ટ્રિંગ દાખલ કરો: ")
print(f"ઉલટી: {s[::-1]}")
\end{lstlisting}

\textbf{રિવર્સિંગ લોજિક:}
\begin{center}
\begin{tikzpicture}[gtu flow]
    \foreach \x/\l in {0/H, 1/e, 2/l, 3/l, 4/o} {
        \node[draw, minimum size=0.6cm] (n\x) at (\x*0.8, 0) {\l};
        \node[font=\footnotesize, below=0.1cm of n\x] {\x};
    }
    
    \draw[<-, thick, blue] (0,-0.6) -- (3.2,-0.6);
    \node[below=0.7cm of n2] {Step = -1 (પાછળની તરફ)};
\end{tikzpicture}
\captionof{figure}{String Reversal}
\end{center}
\end{solutionbox}

\begin{mnemonicbox}
\mnemonic{પાછળની તરફ સ્લાઇસ કરો}
\end{mnemonicbox}

\questionmarks{3(b OR)}{4}{Python માં ડિક્શનરી ઓપરેશન્સની યાદી આપો અને દરેકને યોગ્ય ઉદાહરણ સાથે સમજાવો.}

\begin{solutionbox}
\begin{center}
\captionof{table}{ડિક્શનરી ઓપરેશન્સ}
\begin{tabulary}{\linewidth}{|L|L|L|}
\hline
\textbf{ઓપરેશન} & \textbf{કોડ} & \textbf{વિવરણ} \\ \hline
એક્સેસ & \code{d['key']} & મૂલ્ય મેળવવા \\ \hline
ઉમેરો/બદલો & \code{d['k'] = v} & દાખલ/અપડેટ કરો \\ \hline
ડિલીટ & \code{del d['k']} & જોડી દૂર કરો \\ \hline
ચકાસણી & \code{'k' in d} & મેમ્બરશિપ \\ \hline
લંબાઈ & \code{len(d)} & આઇટમ્સ ગણવા \\ \hline
\end{tabulary}
\end{center}
\end{solutionbox}

\begin{mnemonicbox}
\mnemonic{CADMIL: Create Access Delete Modify Iterate Length}
\end{mnemonicbox}

\questionmarks{3(c OR)}{7}{Python ના સેટ ડેટા ટાઇપને વિગતે સમજાવો.}

\begin{solutionbox}
સેટ અનન્ય ઘટકોનો અનોર્ડર્ડ સંગ્રહ છે.

\textbf{સેટ લાક્ષણિકતાઓ:}
\begin{itemize}
    \item \textbf{અનન્ય}: કોઈ ડુપ્લિકેટ્સ નહીં.
    \item \textbf{અનોર્ડર્ડ}: કોઈ ઇન્ડેક્સ એક્સેસ નહીં.
    \item \textbf{ગાણિતિક}: યુનિયન, ઇન્ટરસેક્શન સપોર્ટ કરે છે.
\end{itemize}

\textbf{સેટ ઓપરેશન્સ ડાયાગ્રામ:}
\begin{center}
\begin{tikzpicture}[gtu flow]
    \node [gtu start] (A) {A: \{1,2\}};
    \node [gtu start, right=1cm of A] (B) {B: \{2,3\}};
    
    \node [gtu block, below=1cm of A, xshift=1cm] (union) {A | B\\\{1,2,3\}};
    \node [gtu block, below=0.5cm of union] (inter) {A \& B\\\{2\}};
    
    \path [gtu arrow] (A) -- (union);
    \path [gtu arrow] (B) -- (union);
\end{tikzpicture}
\captionof{figure}{Set Union \& Intersection}
\end{center}
\end{solutionbox}

\begin{mnemonicbox}
\mnemonic{SUMO: Set Unique Mutable Ordered-less}
\end{mnemonicbox}

\questionmarks{4(a)}{3}{statistics મોડ્યુલને સમજાવો અને તેમાંની ત્રણ પદ્ધતિઓ સાથે ઉદાહરણ આપો.}

\begin{solutionbox}
\begin{center}
\captionof{table}{Statistics Methods}
\begin{tabulary}{\linewidth}{|L|L|L|}
\hline
\textbf{મેથડ} & \textbf{વિવરણ} & \textbf{ઉદાહરણ} \\ \hline
\code{mean()} & સરેરાશ & \code{mean([1,2,3])} $\to$ 2 \\ \hline
\code{median()} & મધ્ય મૂલ્ય & \code{median([1,5,9])} $\to$ 5 \\ \hline
\code{mode()} & સૌથી સામાન્ય & \code{mode([1,1,2])} $\to$ 1 \\ \hline
\end{tabulary}
\end{center}
\end{solutionbox}

\begin{mnemonicbox}
\mnemonic{MMM Stats}
\end{mnemonicbox}

\questionmarks{4(b)}{4}{Python માં યુઝર ડિફાઇન્ડ ફંક્શન અને યુઝર ડિફાઇન્ડ મોડ્યુલને સમજાવો.}

\begin{solutionbox}
\begin{center}
\captionof{table}{ફંક્શન અને મોડ્યુલ}
\begin{tabulary}{\linewidth}{|L|L|L|}
\hline
\textbf{વિશેષતા} & \textbf{ફંક્શન} & \textbf{મોડ્યુલ} \\ \hline
એકમ & કોડ બ્લોક & ફાઇલ (.py) \\ \hline
નિર્માણ & \code{def name():} & .py તરીકે સાચવો \\ \hline
ઉપયોગ & \code{name()} કોલ & \code{import name} \\ \hline
સ્કોપ & લોકલ & ગ્લોબલ/ઇમ્પોર્ટેડ \\ \hline
\end{tabulary}
\end{center}

\textbf{મોડ્યુલ માળખું:}
\begin{center}
\begin{tikzpicture}[gtu flow]
    \node [gtu block] (main) {Main Program};
    \node [gtu block, right=2cm of main] (mod) {Module.py};
    \draw [gtu arrow] (main) -- node[above] {import} (mod);
    \draw [gtu arrow] (mod) -- node[below] {functions} (main);
\end{tikzpicture}
\captionof{figure}{Import Relationship}
\end{center}
\end{solutionbox}

\begin{mnemonicbox}
\mnemonic{FIR-MID}
\end{mnemonicbox}

\questionmarks{4(c)}{7}{Using recursion આપેલ આંકડાના ફેક્ટોરિયલને શોધવા માટે યુઝર ડિફાઇન્ડ ફંક્શનનો ઉપયોગ કરીને Python કોડ લખો.}

\begin{solutionbox}
\begin{lstlisting}[language=Python]
def factorial(n):
    # બેઝ કેસ
    if n == 0 or n == 1:
        return 1
    # રિકર્સિવ કેસ
    else:
        return n * factorial(n-1)

num = int(input("નંબર દાખલ કરો: "))
print(f"ફેક્ટોરિયલ: {factorial(num)}")
\end{lstlisting}

\textbf{રિકર્ઝન વિઝ્યુલાઇઝેશન:}
\begin{center}
\begin{tikzpicture}[gtu flow]
    \node [gtu process] (call4) {fact(4)};
    \node [gtu process, right=0.5cm of call4] (call3) {fact(3)};
    \node [gtu process, right=0.5cm of call3] (call2) {fact(2)};
    \node [gtu process, right=0.5cm of call2] (call1) {fact(1)};
    \node [gtu output, below=0.5cm of call1] (ret1) {return 1};
    
    \path [gtu arrow] (call4) -- (call3);
    \path [gtu arrow] (call3) -- (call2);
    \path [gtu arrow] (call2) -- (call1);
    \path [gtu arrow] (call1) -- (ret1);
    \path [gtu arrow, dashed] (ret1) -- node[below] {પરિણામ} (call4.south);
\end{tikzpicture}
\captionof{figure}{Recursion Chain}
\end{center}
\end{solutionbox}

\begin{mnemonicbox}
\mnemonic{સંખ્યા ગુણ્યા (સંખ્યા માઇનસ વન)!}
\end{mnemonicbox}

\questionmarks{4(a OR)}{3}{મેથ મોડ્યુલને સમજાવો અને તેમાંની ત્રણ methods ઉદાહરણ સાથે સમજાવો.}

\begin{solutionbox}
\begin{center}
\captionof{table}{Math Methods}
\begin{tabulary}{\linewidth}{|L|L|L|}
\hline
\textbf{મેથડ} & \textbf{વિવરણ} & \textbf{ઉદાહરણ} \\ \hline
\code{sqrt()} & વર્ગમૂળ & \code{sqrt(16)} $\to$ 4.0 \\ \hline
\code{pow()} & પાવર & \code{pow(2,3)} $\to$ 8.0 \\ \hline
\code{ceil()} & ઉપર રાઉન્ડ & \code{ceil(4.1)} $\to$ 5 \\ \hline
\end{tabulary}
\end{center}
\end{solutionbox}

\begin{mnemonicbox}
\mnemonic{SPT Math}
\end{mnemonicbox}

\questionmarks{4(b OR)}{4}{Python માં global અને local variables સમજાવો.}

\begin{solutionbox}
\begin{center}
\captionof{table}{વેરિએબલ સ્કોપ}
\begin{tabulary}{\linewidth}{|L|L|L|}
\hline
\textbf{પ્રકાર} & \textbf{સ્કોપ} & \textbf{એક્સેસ} \\ \hline
ગ્લોબલ & આખો પ્રોગ્રામ & ગમે ત્યાં \\ \hline
લોકલ & ફંક્શનની અંદર & માત્ર ફંક્શનમાં \\ \hline
\end{tabulary}
\end{center}

\textbf{સ્કોપ ડાયાગ્રામ:}
\begin{center}
\begin{tikzpicture}[gtu flow]
    \node [gtu block, minimum width=5cm, minimum height=3cm, label=above:Global Scope] (global) {};
    \node [gtu block, minimum width=3cm, minimum height=1.5cm, fill=white, label=above:Function Scope] (local) at (global.center) {};
    
    \node at ([yshift=1cm]global.center) {global\_var};
    \node at (local.center) {local\_var};
\end{tikzpicture}
\captionof{figure}{Scope Hierarchy}
\end{center}
\end{solutionbox}

\begin{mnemonicbox}
\mnemonic{GLOBAL બધે જાય, LOCAL ફક્ત ફંક્શનમાં રહે}
\end{mnemonicbox}

\questionmarks{4(c OR)}{7}{આપેલ સ્ટ્રિંગ પેલિન્ડ્રોમ છે કે નહીં તે તપાસવા માટે યુઝર ડિફાઇન્ડ ફંક્શન બનાવો.}

\begin{solutionbox}
\begin{lstlisting}[language=Python]
def is_palindrome(text):
    raw = text.replace(" ", "").lower()
    return raw == raw[::-1]

s = input("સ્ટ્રિંગ દાખલ કરો: ")
if is_palindrome(s):
    print("પેલિન્ડ્રોમ છે")
else:
    print("પેલિન્ડ્રોમ નથી")
\end{lstlisting}

\textbf{લોજિક ફ્લો:}
\begin{center}
\begin{tikzpicture}[gtu flow]
    \node [gtu start] (in) {ઇનપુટ};
    \node [gtu process, right=1cm of in] (clean) {સાફ કરો (space/case)};
    \node [gtu decision, right=1cm of clean] (check) {રિવર્સ સમાન છે?};
    \node [gtu output, right=1cm of check] (res) {પરિણામ};
    
    \path [gtu arrow] (in) -- (clean);
    \path [gtu arrow] (clean) -- (check);
    \path [gtu arrow] (check) -- (res);
\end{tikzpicture}
\captionof{figure}{Palindrome Check}
\end{center}
\end{solutionbox}

\begin{mnemonicbox}
\mnemonic{સાફ કરો, ઉલટાવો, સરખાવો}
\end{mnemonicbox}

\questionmarks{5(a)}{3}{ક્લાસ અને ઑબ્જેક્ટને વ્યાખ્યાયિત કરો અને ઉદાહરણ સાથે સમજાવો.}

\begin{solutionbox}
\begin{itemize}
    \item \textbf{ક્લાસ}: બ્લુપ્રિન્ટ/ટેમ્પલેટ.
    \item \textbf{ઓબ્જેક્ટ}: ક્લાસનો ઇન્સ્ટન્સ.
\end{itemize}

\textbf{સંબંધ ડાયાગ્રામ:}
\begin{center}
\begin{tikzpicture}[gtu flow]
    \node [gtu class] (cls) {Class: Dog};
    \node [gtu block, below left=1cm of cls] (o1) {Dog("Rex")};
    \node [gtu block, below right=1cm of cls] (o2) {Dog("Buddy")};
    
    \draw [gtu arrow] (cls) -- (o1);
    \draw [gtu arrow] (cls) -- (o2);
\end{tikzpicture}
\captionof{figure}{Instantiation}
\end{center}
\end{solutionbox}

\begin{mnemonicbox}
\mnemonic{CAMBO}
\end{mnemonicbox}

\questionmarks{5(b)}{4}{કન્સ્ટ્રક્ટરનું વર્ગીકરણ કરો. જેમાંથી એકને વિગતે સમજાવો.}

\begin{solutionbox}
\textbf{પ્રકારો}: Default, Parameterized, Non-parameterized, Copy.

\textbf{Parameterized Constructor:}
\begin{lstlisting}[language=Python]
class Student:
    def __init__(self, name):
        self.name = name
s = Student("Alice")
\end{lstlisting}

\textbf{એક્ઝિક્યુશન ફ્લો:}
\begin{center}
\begin{tikzpicture}[gtu flow]
    \node [gtu start] (new) {Object Created};
    \node [gtu process, right=1cm of new] (init) {\_\_init\_\_ કોલ};
    \node [gtu process, right=1cm of init] (set) {Attrs Set};
    \node [gtu stop, right=1cm of set] (rdy) {તૈયાર};
    
    \path [gtu arrow] (new) -- (init);
    \path [gtu arrow] (init) -- (set);
    \path [gtu arrow] (set) -- (rdy);
\end{tikzpicture}
\captionof{figure}{Constructor Lifecycle}
\end{center}
\end{solutionbox}

\begin{mnemonicbox}
\mnemonic{PICAN}
\end{mnemonicbox}

\questionmarks{5(c)}{7}{hierarchical inheritance માટે Python કોડ વિકસાવો અને સમજાવો.}

\begin{solutionbox}
\begin{lstlisting}[language=Python]
class Vehicle:
    def start(self): print("Engine On")

class Car(Vehicle):
    def drive(self): print("Driving")

class Bike(Vehicle):
    def ride(self): print("Riding")

c = Car(); c.start()
b = Bike(); b.start()
\end{lstlisting}

\textbf{ઇન્હેરિટન્સ ટ્રી:}
\begin{center}
\begin{tikzpicture}[gtu flow]
    \node [gtu class] (parent) {Vehicle};
    \node [gtu class, below left=1cm and 0.5cm of parent] (c1) {Car};
    \node [gtu class, below right=1cm and 0.5cm of parent] (c2) {Bike};
    
    \draw [gtu arrow] (parent) -- (c1);
    \draw [gtu arrow] (parent) -- (c2);
\end{tikzpicture}
\captionof{figure}{Hierarchical Inheritance}
\end{center}
\end{solutionbox}

\begin{mnemonicbox}
\mnemonic{પેરેન્ટ્સ શેર કરે, ચિલ્ડ્રન સ્પેશિયલાઇઝ કરે}
\end{mnemonicbox}

\questionmarks{5(a OR)}{3}{Python માં \_\_init\_\_ method શું છે? તેના હેતુને યોગ્ય ઉદાહરણ સાથે સમજાવો.}

\begin{solutionbox}
ખાસ મેથડ જે ઓબ્જેક્ટ બનાવતી વખતે એટ્રિબ્યુટ્સ શરૂ કરવા માટે આપોઆપ કોલ થાય છે.

\textbf{ઉદાહરણ:}
\begin{lstlisting}[language=Python]
class Rect:
    def __init__(self, w, h):
        self.w = w
        self.h = h
\end{lstlisting}
\end{solutionbox}

\begin{mnemonicbox}
\mnemonic{ASAP: Attributes Set At Production}
\end{mnemonicbox}

\questionmarks{5(b OR)}{4}{Python class માટે methods નું વર્ગીકરણ કરો. તે માંથી એકને વિગતવાર સમજાવો.}

\begin{solutionbox}
\begin{center}
\captionof{table}{મેથડ પ્રકારો}
\begin{tabulary}{\linewidth}{|L|L|L|}
\hline
\textbf{પ્રકાર} & \textbf{એક્સેસ} & \textbf{ડેકોરેટર} \\ \hline
Instance & \code{self} & None \\ \hline
Class & \code{cls} & \code{@classmethod} \\ \hline
Static & None & \code{@staticmethod} \\ \hline
\end{tabulary}
\end{center}

\textbf{Instance Method}:
\begin{lstlisting}[language=Python]
def display(self):
    print(self.name)
\end{lstlisting}
\end{solutionbox}

\begin{mnemonicbox}
\mnemonic{SIAM: Self Is Always Mentioned}
\end{mnemonicbox}

\questionmarks{5(c OR)}{7}{પોલીમોર્ફિઝમ માટે Python કોડ વિકસાવો અને સમજાવો.}

\begin{solutionbox}
\begin{lstlisting}[language=Python]
class Dog:
    def sound(self): return "Woof"

class Cat:
    def sound(self): return "Meow"

animals = [Dog(), Cat()]
for a in animals:
    print(a.sound())
\end{lstlisting}

\textbf{પોલીમોર્ફિઝમ ડાયાગ્રામ:}
\begin{center}
\begin{tikzpicture}[gtu flow]
    \node [gtu class] (call) {Call .sound()};
    \node [gtu class, below left=1cm of call] (dog) {Dog: Woof};
    \node [gtu class, below right=1cm of call] (cat) {Cat: Meow};
    
    \draw [gtu arrow] (call) -- node[left] {If Dog} (dog);
    \draw [gtu arrow] (call) -- node[right] {If Cat} (cat);
\end{tikzpicture}
\captionof{figure}{Dynamic Binding}
\end{center}
\end{solutionbox}

\begin{mnemonicbox}
\mnemonic{એક મેથડ, વિવિધ વર્તન}
\end{mnemonicbox}

\end{document}
