\documentclass[10pt,a4paper]{article}

% content/resources/templates/preamble.tex
\usepackage[margin=0.6in]{geometry}
\author{Milav Dabgar}
\usepackage{amsmath,amssymb,amsthm}
\usepackage{booktabs}
\usepackage{multirow}
\usepackage{xcolor}
\usepackage{tcolorbox}
\tcbuselibrary{breakable,skins}
\usepackage[colorlinks=true,linkcolor=blue]{hyperref}
\usepackage{titlesec}
\usepackage{enumitem}
\usepackage{tikz}
\usepackage{pgfplots}
\usepackage{circuitikz}
\usepackage[version=4]{mhchem}
\usepackage{longtable}
\usepackage{array}
\usepackage{float}
\usepackage{caption}
\usepackage{listings}

\lstset{
  basicstyle=\small\ttfamily,
  breaklines=true,
  breakatwhitespace=false,
  postbreak=\mbox{\textcolor{red}{$\hookrightarrow$}\space},
  float=false,
  numbers=left,
  numberstyle=\tiny\color{gray},
  numbersep=10pt,
  xleftmargin=2em,
  keywordstyle=\color{blue},
  commentstyle=\color{green!60!black},
  stringstyle=\color{purple},
  backgroundcolor=\color{gray!5},
  showstringspaces=false,
  tabsize=2,
  captionpos=b,
  keepspaces=true,
  columns=flexible
}

\pgfplotsset{compat=1.18}
\usetikzlibrary{shapes,arrows,positioning,calc,patterns,decorations.pathmorphing,decorations.markings,arrows.meta}

% Color scheme
\definecolor{headcolor}{RGB}{0,102,204}
\definecolor{keycolor}{RGB}{220,20,60}
\definecolor{solutioncolor}{RGB}{34,139,34}
\definecolor{mnemoniccolor}{RGB}{148,0,211}
\definecolor{codecolor}{RGB}{0,0,100}

% Spacing
\setlength{\parskip}{3pt}
\setlist[itemize]{nosep}
\setlist[enumerate]{nosep}

% Title formatting
\titleformat{\section}{\Large\bfseries\color{headcolor}}{\thesection}{1em}{}
\titleformat{\subsection}{\large\bfseries\color{headcolor}}{\thesubsection}{1em}{}

% Pandoc tightlist compatibility
\providecommand{\tightlist}{%
  \setlength{\itemsep}{0pt}\setlength{\parskip}{0pt}}

% Pandoc longtable compatibility
\newcounter{none}
\def\thenone{}


% content/resources/templates/gujarati-boxes.tex
\usepackage{fontspec}
\usepackage{polyglossia}

% Set Gujarati as main language (document is primarily in Gujarati)
% Note: gloss-gujarati.ldf doesn't exist in polyglossia, but it will use hyphenation patterns
\setdefaultlanguage{gujarati}
\setotherlanguage{english}

% Configure Gujarati font properly
% Use Language=Default to prevent polyglossia from trying to add language-specific features
% that don't exist for Gujarati, which causes "empty feature" warnings
\newfontfamily\gujaratifont[Script=Gujarati,AutoFakeBold=2.5,AutoFakeSlant=0.3]{Noto Sans Gujarati}
\setmainfont[Script=Gujarati,AutoFakeBold=2.5,AutoFakeSlant=0.3]{Noto Sans Gujarati}
% Use Noto Sans Gujarati for monospace to support Gujarati in text
\setmonofont[Scale=0.9]{Noto Sans Gujarati}

% Configure English to use the same font
\newfontfamily\englishfont[Script=Gujarati,AutoFakeBold=2.5,AutoFakeSlant=0.3]{Noto Sans Gujarati}

% Translations for polyglossia
\gappto\captionsgujarati{
  \renewcommand{\tablename}{કોષ્ટક}
  \renewcommand{\figurename}{આકૃતિ}
}

% Helper for TikZ nodes to ensure Gujarati font
\newcommand{\gu}[1]{{\gujaratifont #1}}

% Custom environments
\newtcolorbox{solutionbox}{
    breakable,
    enhanced,
    colback=solutioncolor!5!white,
    colframe=solutioncolor!75!black,
    fonttitle=\bfseries,
    title=જવાબ
}

\newtcolorbox{solutionboxnobreak}{
 colback=solutioncolor!5!white,
 colframe=solutioncolor!75!black,
 fonttitle=\bfseries,
 title=જવાબ
}

\newtcolorbox{keyformula}{
 breakable,
 enhanced,
 colback=keycolor!5!white,
 colframe=keycolor!75!black,
 fonttitle=\bfseries,
 title=રાસાયણિક સમીકરણ/સૂત્ર
}

\newtcolorbox{mnemonicbox}{
 breakable,
 enhanced,
 colback=mnemoniccolor!5!white,
 colframe=mnemoniccolor!75!black,
 fonttitle=\bfseries,
 title=મેમરી ટ્રીક
}


\begin{document}

\begin{center}
{\Huge\bfseries\color{headcolor} Subject Name (Gujarati)}\\[5pt]
{\LARGE 4351108 -- Winter 2024}\\[3pt]
{\large Semester 1 Study Material}\\[3pt]
{\normalsize\textit{Detailed Solutions and Explanations}}
\end{center}

\vspace{10pt}

\subsection*{પ્રશ્ન 1(અ) [3
ગુણ]}\label{uxaaauxab0uxab6uxaa8-1uxa85-3-uxa97uxaa3}

\textbf{પાયથોન પ્રોગ્રામિંગ ભાષાના લક્ષણોની યાદી બનાવો.}

\begin{solutionbox}

{\def\LTcaptype{none} % do not increment counter
\begin{longtable}[]{@{}ll@{}}
\toprule\noalign{}
લક્ષણ & વર્ણન \\
\midrule\noalign{}
\endhead
\bottomrule\noalign{}
\endlastfoot
\textbf{સરળ અને સહેલું} & સ્વચ્છ, વાંચી શકાય તેવું syntax \\
\textbf{મફત અને ઓપન સોર્સ} & કોઈ કિંમત નહીં, community driven \\
\textbf{ક્રોસ-પ્લેટફોર્મ} & Windows, Linux, Mac પર ચાલે છે \\
\textbf{Interpreted} & compilation ની જરૂર નથી \\
\textbf{Object-Oriented} & classes અને objects ને support કરે છે \\
\textbf{મોટી લાઇબ્રેરીઓ} & સમૃદ્ધ standard library \\
\end{longtable}
}

\textbf{યાદી રાખવાની ટ્રિક:} ``સરળ મફત ક્રોસ Interpreted ઓબ્જેક્ટ મોટી''

\end{solutionbox}
\begin{center}\rule{0.5\linewidth}{0.5pt}\end{center}

\subsection*{પ્રશ્ન 1(બ) [4
ગુણ]}\label{uxaaauxab0uxab6uxaa8-1uxaac-4-uxa97uxaa3}

\textbf{પાયથોન પ્રોગ્રામિંગ ભાષાની એપ્લિકેશનો લખો.}

\begin{solutionbox}

{\def\LTcaptype{none} % do not increment counter
\begin{longtable}[]{@{}ll@{}}
\toprule\noalign{}
એપ્લિકેશન ક્ષેત્ર & ઉદાહરણો \\
\midrule\noalign{}
\endhead
\bottomrule\noalign{}
\endlastfoot
\textbf{વેબ ડેવલપમેન્ટ} & Django, Flask frameworks \\
\textbf{ડેટા સાયન્સ} & NumPy, Pandas, Matplotlib \\
\textbf{મશીન લર્નિંગ} & TensorFlow, Scikit-learn \\
\textbf{ડેસ્કટોપ GUI} & Tkinter, PyQt applications \\
\textbf{ગેમ ડેવલપમેન્ટ} & Pygame library \\
\textbf{ઓટોમેશન} & Scripting અને testing \\
\end{longtable}
}

\textbf{યાદી રાખવાની ટ્રિક:} ``વેબ ડેટા મશીન ડેસ્કટોપ ગેમ ઓટો''

\end{solutionbox}
\begin{center}\rule{0.5\linewidth}{0.5pt}\end{center}

\subsection*{પ્રશ્ન 1(ક) [7
ગુણ]}\label{uxaaauxab0uxab6uxaa8-1uxa95-7-uxa97uxaa3}

\textbf{પાયથોનમાં વિવિધ ડેટાટાઇપ્સ સમજાવો.}

\begin{solutionbox}

{\def\LTcaptype{none} % do not increment counter
\begin{longtable}[]{@{}lll@{}}
\toprule\noalign{}
ડેટા ટાઇપ & ઉદાહરણ & વર્ણન \\
\midrule\noalign{}
\endhead
\bottomrule\noalign{}
\endlastfoot
\textbf{int} & \texttt{x\ =\ 5} & પૂર્ણાંક સંખ્યાઓ \\
\textbf{float} & \texttt{y\ =\ 3.14} & દશાંશ સંખ્યાઓ \\
\textbf{str} & \texttt{name\ =\ "John"} & ટેક્સ્ટ ડેટા \\
\textbf{bool} & \texttt{flag\ =\ True} & True/False મૂલ્યો \\
\textbf{list} & \texttt{[1,\ 2,\ 3]} & ક્રમબદ્ધ, બદલી શકાય તેવું \\
\textbf{tuple} & \texttt{(1,\ 2,\ 3)} & ક્રમબદ્ધ, બદલી ન શકાય તેવું \\
\textbf{dict} & \texttt{\{"a":\ 1\}} & Key-value જોડી \\
\textbf{set} & \texttt{\{1,\ 2,\ 3\}} & અનન્ય ઘટકો \\
\end{longtable}
}

\textbf{કોડ ઉદાહરણ:}

\begin{verbatim}
\# Numeric types
age = 25          \# int
price = 99.99     \# float

\# Text type
name = "Python"   \# str

\# Boolean type
is\_valid = True   \# bool

\# Collection types
numbers = [1, 2, 3]        \# list
coordinates = (10, 20)     \# tuple
student = \{"name": "John"\ }\# dict
unique\_ids = \{1, 2, 3\     }\# set
\end{verbatim}

\textbf{યાદી રાખવાની ટ્રિક:} ``Integer Float String Boolean List Tuple
Dict Set''

\end{solutionbox}
\begin{center}\rule{0.5\linewidth}{0.5pt}\end{center}

\subsection*{પ્રશ્ન 1(ક OR) [7
ગુણ]}\label{uxaaauxab0uxab6uxaa8-1uxa95-or-7-uxa97uxaa3}

\textbf{એરિથમેટિક, એસાઇનમેન્ટ અને આઇડેન્ટિટી ઓપરેટરો ઉદાહરણ સાથે સમજાવો.}

\begin{solutionbox}

\textbf{એરિથમેટિક ઓપરેટરો:}

{\def\LTcaptype{none} % do not increment counter
\begin{longtable}[]{@{}lll@{}}
\toprule\noalign{}
ઓપરેટર & ઓપરેશન & ઉદાહરણ \\
\midrule\noalign{}
\endhead
\bottomrule\noalign{}
\endlastfoot
\texttt{+} & બાકીદારી & \texttt{5\ +\ 3\ =\ 8} \\
\texttt{-} & બાદબાકી & \texttt{5\ -\ 3\ =\ 2} \\
\texttt{*} & ગુણાકાર & \texttt{5\ *\ 3\ =\ 15} \\
\texttt{/} & ભાગાકાર & \texttt{10\ /\ 3\ =\ 3.33} \\
\texttt{//} & Floor Division & \texttt{10\ //\ 3\ =\ 3} \\
\texttt{\%} & બાકી & \texttt{10\ \%\ 3\ =\ 1} \\
\texttt{**} & ઘાત & \texttt{2\ **\ 3\ =\ 8} \\
\end{longtable}
}

\textbf{એસાઇનમેન્ટ ઓપરેટરો:}

{\def\LTcaptype{none} % do not increment counter
\begin{longtable}[]{@{}lll@{}}
\toprule\noalign{}
ઓપરેટર & ઉદાહરણ & સમકક્ષ \\
\midrule\noalign{}
\endhead
\bottomrule\noalign{}
\endlastfoot
\texttt{=} & \texttt{x\ =\ 5} & મૂલ્ય આપો \\
\texttt{+=} & \texttt{x\ +=\ 3} & \texttt{x\ =\ x\ +\ 3} \\
\texttt{-=} & \texttt{x\ -=\ 2} & \texttt{x\ =\ x\ -\ 2} \\
\texttt{*=} & \texttt{x\ *=\ 4} & \texttt{x\ =\ x\ *\ 4} \\
\end{longtable}
}

\textbf{આઇડેન્ટિટી ઓપરેટરો:}

{\def\LTcaptype{none} % do not increment counter
\begin{longtable}[]{@{}lll@{}}
\toprule\noalign{}
ઓપરેટર & હેતુ & ઉદાહરણ \\
\midrule\noalign{}
\endhead
\bottomrule\noalign{}
\endlastfoot
\texttt{is} & સમાન ઓબ્જેક્ટ & \texttt{x\ is\ y} \\
\texttt{is\ not} & વિવિધ ઓબ્જેક્ટ & \texttt{x\ is\ not\ y} \\
\end{longtable}
}

\textbf{કોડ ઉદાહરણ:}

\begin{verbatim}
\# Arithmetic
a = 10 + 5    \# 15
b = 10 // 3   \# 3

\# Assignment
x = 5
x += 3        \# x બને છે 8

\# Identity
list1 = [1, 2, 3]
list2 = [1, 2, 3]
print(list1 is list2)      \# False
print(list1 is not list2)  \# True
\end{verbatim}

\textbf{યાદી રાખવાની ટ્રિક:} ``Add Assign Identity''

\end{solutionbox}
\begin{center}\rule{0.5\linewidth}{0.5pt}\end{center}

\subsection*{પ્રશ્ન 2(અ) [3
ગુણ]}\label{uxaaauxab0uxab6uxaa8-2uxa85-3-uxa97uxaa3}

\textbf{નીચેનામાંથી કયા આઇડેન્ટિફાયર્સ ના નામો અમાન્ય છે?} **(i) Total Marks
(ii)Total\_Marks (iii)total-Marks (iv) Hundred\$ (v) \_Percentage (vi)
True**

\begin{solutionbox}

{\def\LTcaptype{none} % do not increment counter
\begin{longtable}[]{@{}lll@{}}
\toprule\noalign{}
આઇડેન્ટિફાયર & માન્ય/અમાન્ય & કારણ \\
\midrule\noalign{}
\endhead
\bottomrule\noalign{}
\endlastfoot
Total Marks & \textbf{અમાન્ય} & સ્પેસ છે \\
Total\_Marks & માન્ય & અન્ડરસ્કોર મંજૂર છે \\
total-Marks & \textbf{અમાન્ય} & હાઇફન મંજૂર નથી \\
Hundred\$ & \textbf{અમાન્ય} & \$ સિમ્બોલ મંજૂર નથી \\
\_Percentage & માન્ય & અન્ડરસ્કોરથી શરૂ થઈ શકે છે \\
True & \textbf{અમાન્ય} & આરક્ષિત કીવર્ડ છે \\
\end{longtable}
}

\textbf{અમાન્ય આઇડેન્ટિફાયર્સ:} Total Marks, total-Marks, Hundred\$, True

\textbf{યાદી રાખવાની ટ્રિક:} ``સ્પેસ હાઇફન ડોલર કીવર્ડ = અમાન્ય''

\end{solutionbox}
\begin{center}\rule{0.5\linewidth}{0.5pt}\end{center}

\subsection*{પ્રશ્ન 2(બ) [4
ગુણ]}\label{uxaaauxab0uxab6uxaa8-2uxaac-4-uxa97uxaa3}

\textbf{આપેલ ત્રણ સંખ્યાઓમાંથી મહત્તમ સંખ્યા શોધવા માટે પ્રોગ્રામ લખો.}

\begin{solutionbox}

\begin{verbatim}
\# ત્રણ સંખ્યાઓ input લો
num1 = float(input("પ્રથમ સંખ્યા દાખલ કરો: "))
num2 = float(input("બીજી સંખ્યા દાખલ કરો: "))
num3 = float(input("ત્રીજી સંખ્યા દાખલ કરો: "))

\# if{-elif{-}else વાપરીને મહત્તમ શોધો}
if num1 {=} num2 and num1 {=} num3:
    maximum = num1
elif num2 {=} num1 and num2 {=} num3:
    maximum = num2
else:
    maximum = num3

\# પરિણામ દર્શાવો
print(f"મહત્તમ સંખ્યા છે: \{maximum\}")
\end{verbatim}

\textbf{max() ફંક્શન વાપરીને વૈકલ્પિક રીત:}

\begin{verbatim}
num1, num2, num3 = map(float, input("3 સંખ્યાઓ દાખલ કરો: ").split())
maximum = max(num1, num2, num3)
print(f"મહત્તમ: \{maximum\}")
\end{verbatim}

\textbf{યાદી રાખવાની ટ્રિક:} ``Input Compare Display''

\end{solutionbox}
\begin{center}\rule{0.5\linewidth}{0.5pt}\end{center}

\subsection*{પ્રશ્ન 2(ક) [7
ગુણ]}\label{uxaaauxab0uxab6uxaa8-2uxa95-7-uxa97uxaa3}

\textbf{પાયથોનમાં ડિક્શનરી સમજાવો. ડિક્શનરીમાં ઘટકો ઉમેરવા, બદલવા અને કાઢી
નાખવા માટેના સ્ટેટમેન્ટ લખો.}

\begin{solutionbox}

\textbf{ડિક્શનરી} એ key-value જોડીઓનો સંગ્રહ છે જે ક્રમબદ્ધ, બદલાય તેવો અને
ડુપ્લિકેટ keys નથી મંજૂર કરે છે.

\textbf{ઓપરેશન્સ ટેબલ:}

{\def\LTcaptype{none} % do not increment counter
\begin{longtable}[]{@{}lll@{}}
\toprule\noalign{}
ઓપરેશન & સિન્ટેક્સ & ઉદાહરણ \\
\midrule\noalign{}
\endhead
\bottomrule\noalign{}
\endlastfoot
\textbf{બનાવો} & \texttt{dict\_name\ =\ \{\}} &
\texttt{student\ =\ \{\}} \\
\textbf{ઉમેરો} & \texttt{dict[key]\ =\ value} &
\texttt{student[\textquotesingle{}name\textquotesingle{}]\ =\ \textquotesingle{}John\textquotesingle{}} \\
\textbf{બદલો} & \texttt{dict[key]\ =\ new\_value} &
\texttt{student[\textquotesingle{}name\textquotesingle{}]\ =\ \textquotesingle{}Jane\textquotesingle{}} \\
\textbf{ડિલીટ કરો} & \texttt{del\ dict[key]} &
\texttt{del\ student[\textquotesingle{}name\textquotesingle{}]} \\
\textbf{એક્સેસ કરો} & \texttt{dict[key]} &
\texttt{print(student[\textquotesingle{}name\textquotesingle{}])} \\
\end{longtable}
}

\textbf{કોડ ઉદાહરણ:}

\begin{verbatim}
\# ખાલી ડિક્શનરી બનાવો
student = \{\}

\# ઘટકો ઉમેરો
student[{name}] = {John}
student[{age}] = 20
student[{grade}] = {A}

\# ઘટક બદલો
student[{age}] = 21

\# ઘટક ડિલીટ કરો
del student[{grade}]

\# ડિક્શનરી દર્શાવો
print(student)  \# આઉટપુટ: \{{name: John, age: 21\}}

\# અન્ય methods
student.pop({age})           \# Remove અને મૂલ્ય return કરે
student.update(\{{city}: {Mumbai}\)  }\# અનેક items ઉમેરો
\end{verbatim}

\textbf{ડિક્શનરીના ગુણધર્મો:}

\begin{itemize}
\tightlist
\item
  \textbf{ક્રમબદ્ધ}: insertion order જાળવે છે (Python 3.7+)
\item
  \textbf{બદલાય તેવું}: બનાવ્યા પછી બદલી શકાય છે
\item
  \textbf{ડુપ્લિકેટ્સ નહીં}: Keys અનન્ય હોવા જરૂરી છે
\end{itemize}

\textbf{યાદી રાખવાની ટ્રિક:} ``Key-Value ક્રમબદ્ધ બદલાય અનન્ય''

\end{solutionbox}
\begin{center}\rule{0.5\linewidth}{0.5pt}\end{center}

\subsection*{પ્રશ્ન 2(અ OR) [3
ગુણ]}\label{uxaaauxab0uxab6uxaa8-2uxa85-or-3-uxa97uxaa3}

\textbf{નીચેની પેટર્ન દર્શાવવા માટેનો પ્રોગ્રામ લખો.}

\begin{solutionbox}

\begin{verbatim}
\# પેટર્ન પ્રોગ્રામ
for i in range(1, 6):
    for j in range(1, i + 1):
        print(j, end=" ")
    print()  \# દરેક રો પછી નવી લાઇન
\end{verbatim}

\textbf{આઉટપુટ:}

\begin{verbatim}
1
1 2
1 2 3
1 2 3 4
1 2 3 4 5
\end{verbatim}

\textbf{યાદી રાખવાની ટ્રિક:} ``બાહ્ય રો આંતરિક કોલમ પ્રિન્ટ''

\end{solutionbox}
\begin{center}\rule{0.5\linewidth}{0.5pt}\end{center}

\subsection*{પ્રશ્ન 2(બ OR) [4
ગુણ]}\label{uxaaauxab0uxab6uxaa8-2uxaac-or-4-uxa97uxaa3}

\textbf{વપરાશકર્તા દ્વારા દાખલ કરેલ પૂર્ણાંક સંખ્યાના અંકોનો સરવાળો શોધવા માટે
પ્રોગ્રામ લખો.}

\begin{solutionbox}

\begin{verbatim}
\# વપરાશકર્તા પાસેથી સંખ્યા input લો
number = int(input("સંખ્યા દાખલ કરો: "))
original\_number = number
sum\_digits = 0

\# અંકો કાઢો અને સરવાળો કરો
while number {} 0:
    digit = number \% 10    \# છેલ્લો અંક મેળવો
    sum\_digits += digit    \# સરવાળામાં ઉમેરો
    number = number // 10  \# છેલ્લો અંક દૂર કરો

\# પરિણામ દર્શાવો
print(f"\{original\_number\} ના અંકોનો સરવાળો છે: \{sum\_digits\}")
\end{verbatim}

\textbf{વૈકલ્પિક રીત:}

\begin{verbatim}
number = input("સંખ્યા દાખલ કરો: ")
sum\_digits = sum(int(digit) for digit in number)
print(f"અંકોનો સરવાળો: \{sum\_digits\}")
\end{verbatim}

\textbf{યાદી રાખવાની ટ્રિક:} ``Input કાઢો સરવાળો દર્શાવો''

\end{solutionbox}
\begin{center}\rule{0.5\linewidth}{0.5pt}\end{center}

\subsection*{પ્રશ્ન 2(ક OR) [7
ગુણ]}\label{uxaaauxab0uxab6uxaa8-2uxa95-or-7-uxa97uxaa3}

\textbf{લિસ્ટમાં સ્લાઇસિંગ અને કન્કેટનેશન ઓપરેશન સમજાવો.}

\begin{solutionbox}

\textbf{લિસ્ટ સ્લાઇસિંગ:} લિસ્ટનો ભાગ કાઢવા માટે
\texttt{[start:stop:step]} સિન્ટેક્સ વાપરવું.

\textbf{સ્લાઇસિંગ સિન્ટેક્સ ટેબલ:}

{\def\LTcaptype{none} % do not increment counter
\begin{longtable}[]{@{}lll@{}}
\toprule\noalign{}
સિન્ટેક્સ & વર્ણન & ઉદાહરણ \\
\midrule\noalign{}
\endhead
\bottomrule\noalign{}
\endlastfoot
\texttt{list[start:stop]} & start થી stop-1 સુધીના ઘટકો &
\texttt{nums[1:4]} \\
\texttt{list[:stop]} & શરૂઆતથી stop-1 સુધી & \texttt{nums[:3]} \\
\texttt{list[start:]} & start થી અંત સુધી & \texttt{nums[2:]} \\
\texttt{list[::step]} & step સાથે બધા ઘટકો &
\texttt{nums[::2]} \\
\texttt{list[::-1]} & રિવર્સ લિસ્ટ & \texttt{nums[::-1]} \\
\end{longtable}
}

\textbf{કન્કેટનેશન:} બે અથવા વધુ લિસ્ટને \texttt{+} ઓપરેટર અથવા
\texttt{extend()} મેથડ વાપરીને જોડવું.

\textbf{કોડ ઉદાહરણ:}

\begin{verbatim}
\# લિસ્ટ બનાવો
list1 = [1, 2, 3, 4, 5]
list2 = [6, 7, 8]

\# સ્લાઇસિંગ ઓપરેશન્સ
print(list1[1:4])    \# [2, 3, 4]
print(list1[:3])     \# [1, 2, 3]
print(list1[2:])     \# [3, 4, 5]
print(list1[::2])    \# [1, 3, 5]
print(list1[::{-}1])   \# [5, 4, 3, 2, 1]

\# કન્કેટનેશન ઓપરેશન્સ
result1 = list1 + list2           \# [1, 2, 3, 4, 5, 6, 7, 8]
list1.extend(list2)               \# list2 ને list1 માં ઉમેરે છે
combined = [*list1, *list2]       \# Unpacking operator વાપરીને
\end{verbatim}

\textbf{મુખ્ય મુદ્દા:}

\begin{itemize}
\tightlist
\item
  \textbf{સ્લાઇસિંગ}: મૂળ લિસ્ટ બદલ્યા વગર નવી લિસ્ટ બનાવે છે
\item
  \textbf{કન્કેટનેશન}: લિસ્ટને એક લિસ્ટમાં જોડે છે
\item
  \textbf{નેગેટિવ ઇન્ડેક્સિંગ}: \texttt{list[-1]} છેલ્લો ઘટક આપે છે
\end{itemize}

\textbf{યાદી રાખવાની ટ્રિક:} ``સ્લાઇસ કાઢો કન્કેટ જોડો''

\end{solutionbox}
\begin{center}\rule{0.5\linewidth}{0.5pt}\end{center}

\subsection*{પ્રશ્ન 3(અ) [3
ગુણ]}\label{uxaaauxab0uxab6uxaa8-3uxa85-3-uxa97uxaa3}

\textbf{પાયથોનમાં લિસ્ટ વ્યાખ્યાયિત કરો. લિસ્ટના અંતમાં એલિમેન્ટ ઉમેરવા માટે વપરાતા
ફંક્શનનું નામ લખો.}

\begin{solutionbox}

\textbf{લિસ્ટ વ્યાખ્યા:} એક \textbf{લિસ્ટ} એ આઇટમ્સનો ક્રમબદ્ધ સંગ્રહ છે જે બદલાય
તેવો અને ડુપ્લિકેટ મૂલ્યો મંજૂર કરે છે.

\textbf{ગુણધર્મો ટેબલ:}

{\def\LTcaptype{none} % do not increment counter
\begin{longtable}[]{@{}ll@{}}
\toprule\noalign{}
ગુણધર્મ & વર્ણન \\
\midrule\noalign{}
\endhead
\bottomrule\noalign{}
\endlastfoot
\textbf{ક્રમબદ્ધ} & આઇટમ્સનો નિશ્ચિત ક્રમ છે \\
\textbf{બદલાય તેવું} & બનાવ્યા પછી બદલી શકાય છે \\
\textbf{ડુપ્લિકેટ્સ} & ડુપ્લિકેટ મૂલ્યો મંજૂર કરે છે \\
\textbf{ઇન્ડેક્સ્ડ} & ઇન્ડેક્સ દ્વારા આઇટમ્સ access કરવાય છે \\
\end{longtable}
}

\textbf{ઘટક ઉમેરવા માટેનું ફંક્શન:} \texttt{append()}

\textbf{ઉદાહરણ:}

\begin{verbatim}
\# લિસ્ટ બનાવો
fruits = [{apple}, {banana}]

\# અંતમાં ઘટક ઉમેરો
fruits.append({orange})
print(fruits)  \# [{apple, banana, orange]}
\end{verbatim}

\textbf{યાદી રાખવાની ટ્રિક:} ``લિસ્ટ append અંત''

\end{solutionbox}
\begin{center}\rule{0.5\linewidth}{0.5pt}\end{center}

\subsection*{પ્રશ્ન 3(બ) [4
ગુણ]}\label{uxaaauxab0uxab6uxaa8-3uxaac-4-uxa97uxaa3}

\textbf{પાયથોનમાં ટ્યુપલ વ્યાખ્યાયિત કરો. ટ્યુપલના છેલ્લા એલિમેન્ટને એક્સેસ કરવા માટેનું
સ્ટેટમેન્ટ લખો.}

\begin{solutionbox}

\textbf{ટ્યુપલ વ્યાખ્યા:} એક \textbf{ટ્યુપલ} એ આઇટમ્સનો ક્રમબદ્ધ સંગ્રહ છે જે બદલાય
તેવો નથી અને ડુપ્લિકેટ મૂલ્યો મંજૂર કરે છે.

\textbf{ગુણધર્મો ટેબલ:}

{\def\LTcaptype{none} % do not increment counter
\begin{longtable}[]{@{}ll@{}}
\toprule\noalign{}
ગુણધર્મ & વર્ણન \\
\midrule\noalign{}
\endhead
\bottomrule\noalign{}
\endlastfoot
\textbf{ક્રમબદ્ધ} & આઇટમ્સનો નિશ્ચિત ક્રમ છે \\
\textbf{બદલાય તેવું નથી} & બનાવ્યા પછી બદલી શકાય નહીં \\
\textbf{ડુપ્લિકેટ્સ} & ડુપ્લિકેટ મૂલ્યો મંજૂર કરે છે \\
\textbf{ઇન્ડેક્સ્ડ} & ઇન્ડેક્સ દ્વારા આઇટમ્સ access કરવાય છે \\
\end{longtable}
}

\textbf{છેલ્લા એલિમેન્ટને એક્સેસ કરવું:}

\begin{verbatim}
\# મેથડ 1: નેગેટિવ ઇન્ડેક્સ વાપરીને
my\_tuple = (10, 20, 30, 40, 50)
last\_element = my\_tuple[{-}1]
print(last\_element)  \# આઉટપુટ: 50

\# મેથડ 2: length વાપરીને
last\_element = my\_tuple[len(my\_tuple) {-} 1]
print(last\_element)  \# આઉટપુટ: 50
\end{verbatim}

\textbf{યાદી રાખવાની ટ્રિક:} ``ટ્યુપલ બદલાય નહિ નેગેટિવ ઇન્ડેક્સ''

\end{solutionbox}
\begin{center}\rule{0.5\linewidth}{0.5pt}\end{center}

\subsection*{પ્રશ્ન 3(ક) [7
ગુણ]}\label{uxaaauxab0uxab6uxaa8-3uxa95-7-uxa97uxaa3}

\textbf{નીચેના સેટ ઓપરેશન્સ માટે સ્ટેટમેન્ટ લખો: ખાલી સેટ બનાવો, સેટમાં એક ઘટક ઉમેરો,
સેટમાંથી એક ઘટક દૂર કરો, બે સેટનું યુનિયન, બે સેટનું છેદ, બે સેટ વચ્ચેનો તફાવત અને બે સેટ
વચ્ચે સિમેટ્રિક તફાવત.}

\begin{solutionbox}

\textbf{સેટ ઓપરેશન્સ ટેબલ:}

{\def\LTcaptype{none} % do not increment counter
\begin{longtable}[]{@{}
  >{\raggedright\arraybackslash}p{(\linewidth - 6\tabcolsep) * \real{0.2895}}
  >{\raggedright\arraybackslash}p{(\linewidth - 6\tabcolsep) * \real{0.2105}}
  >{\raggedright\arraybackslash}p{(\linewidth - 6\tabcolsep) * \real{0.2632}}
  >{\raggedright\arraybackslash}p{(\linewidth - 6\tabcolsep) * \real{0.2368}}@{}}
\toprule\noalign{}
\begin{minipage}[b]{\linewidth}\raggedright
ઓપરેશન
\end{minipage} & \begin{minipage}[b]{\linewidth}\raggedright
મેથડ
\end{minipage} & \begin{minipage}[b]{\linewidth}\raggedright
ઓપરેટર
\end{minipage} & \begin{minipage}[b]{\linewidth}\raggedright
ઉદાહરણ
\end{minipage} \\
\midrule\noalign{}
\endhead
\bottomrule\noalign{}
\endlastfoot
\textbf{ખાલી બનાવો} & \texttt{set()} & - & \texttt{s\ =\ set()} \\
\textbf{ઘટક ઉમેરો} & \texttt{add()} & - & \texttt{s.add(5)} \\
\textbf{ઘટક દૂર કરો} & \texttt{remove()} & - & \texttt{s.remove(5)} \\
\textbf{યુનિયન} & \texttt{union()} & \texttt{\textbar{}} &
\texttt{A.union(B)} અથવા \texttt{A\ \textbar{}\ B} \\
\textbf{છેદ} & \texttt{intersection()} & \texttt{\&} &
\texttt{A.intersection(B)} અથવા \texttt{A\ \&\ B} \\
\textbf{તફાવત} & \texttt{difference()} & \texttt{-} &
\texttt{A.difference(B)} અથવા \texttt{A\ -\ B} \\
\textbf{સિમેટ્રિક તફાવત} & \texttt{symmetric\_difference()} &
\texttt{\^{}} & \texttt{A.symmetric\_difference(B)} અથવા
\texttt{A\ \^{}\ B} \\
\end{longtable}
}

\textbf{કોડ ઉદાહરણ:}

\begin{verbatim}
\# ખાલી સેટ બનાવો
my\_set = set()

\# ઘટકો ઉમેરો
my\_set.add(10)
my\_set.add(20)

\# ઘટક દૂર કરો
my\_set.remove(10)

\# ઓપરેશન્સ માટે બે સેટ બનાવો
A = \{1, 2, 3, 4\}
B = \{3, 4, 5, 6\}

\# યુનિયન (બધા અનન્ય ઘટકો)
union\_result = A.union(B)        \# \{1, 2, 3, 4, 5, 6\}

\# છેદ (સામાન્ય ઘટકો)
intersection\_result = A.intersection(B)  \# \{3, 4\}

\# તફાવત (A {- B)}
difference\_result = A.difference(B)      \# \{1, 2\}

\# સિમેટ્રિક તફાવત (A અથવા B માં છે, પરંતુ બંનેમાં નહીં)
sym\_diff\_result = A.symmetric\_difference(B)  \# \{1, 2, 5, 6\}
\end{verbatim}

\textbf{યાદી રાખવાની ટ્રિક:} ``બનાવો ઉમેરો દૂર કરો યુનિયન છેદ તફાવત સિમેટ્રિક''

\end{solutionbox}
\begin{center}\rule{0.5\linewidth}{0.5pt}\end{center}

\subsection*{પ્રશ્ન 3(અ OR) [3
ગુણ]}\label{uxaaauxab0uxab6uxaa8-3uxa85-or-3-uxa97uxaa3}

\textbf{પાયથોનમાં સ્ટ્રિંગ વ્યાખ્યાયિત કરો. ઉદાહરણ વાપરીને સમજાવો (i) સ્ટ્રિંગ કેવી
રીતે બનાવવી. (ii) ઇન્ડેક્સિંગનો ઉપયોગ કરીને વ્યક્તિગત અક્ષરોને એક્સેસ કરવું.}

\begin{solutionbox}

\textbf{સ્ટ્રિંગ વ્યાખ્યા:} એક \textbf{સ્ટ્રિંગ} એ અવતરણચિહ્ન (સિંગલ અથવા ડબલ) માં
બંધ કરેલા અક્ષરોનો ક્રમ છે.

\textbf{(i) સ્ટ્રિંગ બનાવવું:}

\begin{verbatim}
\# સિંગલ અવતરણ
name = {Python}

\# ડબલ અવતરણ
message = "Hello World"

\# ટ્રિપલ અવતરણ (મલ્ટિલાઇન)
text = """આ એક
મલ્ટિલાઇન સ્ટ્રિંગ છે"""
\end{verbatim}

\textbf{(ii) અક્ષરોને એક્સેસ કરવું:}

\begin{verbatim}
word = "PYTHON"
print(word[0])    \# P (પ્રથમ અક્ષર)
print(word[2])    \# T (ત્રીજો અક્ષર)
print(word[{-}1])   \# N (છેલ્લો અક્ષર)
print(word[{-}2])   \# O (છેલ્લાથી બીજો)
\end{verbatim}

\textbf{યાદી રાખવાની ટ્રિક:} ``સ્ટ્રિંગ અવતરણ ઇન્ડેક્સ એક્સેસ''

\end{solutionbox}
\begin{center}\rule{0.5\linewidth}{0.5pt}\end{center}

\subsection*{પ્રશ્ન 3(બ OR) [4
ગુણ]}\label{uxaaauxab0uxab6uxaa8-3uxaac-or-4-uxa97uxaa3}

\textbf{ફોર લૂપ અને વ્હાઇલ લૂપનો ઉપયોગ કરીને લિસ્ટ ટ્રાવર્સિંગ સમજાવો.}

\begin{solutionbox}

\textbf{લિસ્ટ ટ્રાવર્સિંગ} મતલબ લિસ્ટના દરેક ઘટકને એક પછી એક મુલાકાત લેવી.

\textbf{ફોર લૂપ ટ્રાવર્સિંગ:}

\begin{verbatim}
numbers = [10, 20, 30, 40, 50]

\# મેથડ 1: સીધો iteration
for num in numbers:
    print(num)

\# મેથડ 2: ઇન્ડેક્સ વાપરીને
for i in range(len(numbers)):
    print(f"ઇન્ડેક્સ \{i\}: \{numbers[i]\}")
\end{verbatim}

\textbf{વ્હાઇલ લૂપ ટ્રાવર્સિંગ:}

\begin{verbatim}
numbers = [10, 20, 30, 40, 50]
i = 0

while i {} len(numbers):
    print(f"ઇન્ડેક્સ \{i\} પર ઘટક: \{numbers[i]\}")
    i += 1
\end{verbatim}

\textbf{તુલના ટેબલ:}

{\def\LTcaptype{none} % do not increment counter
\begin{longtable}[]{@{}lll@{}}
\toprule\noalign{}
લૂપ પ્રકાર & ફાયદો & ઉપયોગ \\
\midrule\noalign{}
\endhead
\bottomrule\noalign{}
\endlastfoot
\textbf{ફોર લૂપ} & સરળ સિન્ટેક્સ & જ્યારે iteration ની સંખ્યા ખબર હોય \\
\textbf{વ્હાઇલ લૂપ} & વધુ નિયંત્રણ & જ્યારે શરત આધારિત iteration જોઈએ \\
\end{longtable}
}

\textbf{યાદી રાખવાની ટ્રિક:} ``ફોર સરળ વ્હાઇલ નિયંત્રણ''

\end{solutionbox}
\begin{center}\rule{0.5\linewidth}{0.5pt}\end{center}

\subsection*{પ્રશ્ન 3(ક OR) [7
ગુણ]}\label{uxaaauxab0uxab6uxaa8-3uxa95-or-7-uxa97uxaa3}

\textbf{એક એવો પ્રોગ્રામ લખો કે જેનાથી વર્ગમાં n વિદ્યાર્થીઓના રોલ નંબર, નામ અને
માર્ક્સ સાથેની ડિક્શનરી બનાવી શકાય અને 75 થી વધુ ગુણ મેળવનારા વિદ્યાર્થીઓના નામ
ડિસ્પ્લે કરી શકાય.}

\begin{solutionbox}

\begin{verbatim}
\# વિદ્યાર્થીઓની સંખ્યા input કરો
n = int(input("વિદ્યાર્થીઓની સંખ્યા દાખલ કરો: "))

\# ખાલી ડિક્શનરી બનાવો
students = \{\}

\# વિદ્યાર્થીઓનો ડેટા input કરો
for i in range(n):
    print(f"{n}વિદ્યાર્થી \{i + 1\} ની વિગતો દાખલ કરો:")
    roll\_no = int(input("રોલ નંબર: "))
    name = input("નામ: ")
    marks = float(input("માર્ક્સ: "))
    
    \# ડિક્શનરીમાં સ્ટોર કરો
    students[roll\_no] = \{
        {name}: name,
        {marks}: marks
    \}

\# 75 થી વધુ માર્ક્સ વાળા વિદ્યાર્થીઓ દર્શાવો
print("{n}75 થી વધુ માર્ક્સ વાળા વિદ્યાર્થીઓ:")
print("{-"} * 30)

high\_performers = []
for roll\_no, data in students.items():
    if data[{marks}] {} 75:
        high\_performers.append(data[{name}])
        print(f"નામ: \{data[{name}]\}, માર્ક્સ: \{data[{marks}]\}")

if not high\_performers:
    print("કોઈ વિદ્યાર્થીએ 75 થી વધુ માર્ક્સ મેળવ્યા નથી")
else:
    print(f"{n}કુલ હાઇ પર્ફોર્મર્સ: \{len(high\_performers)\}")
\end{verbatim}

\textbf{સેમ્પલ આઉટપુટ:}

\begin{verbatim}
વિદ્યાર્થીઓની સંખ્યા દાખલ કરો: 2

વિદ્યાર્થી 1 ની વિગતો દાખલ કરો:
રોલ નંબર: 101
નામ: John
માર્ક્સ: 80

વિદ્યાર્થી 2 ની વિગતો દાખલ કરો:
રોલ નંબર: 102
નામ: Alice
માર્ક્સ: 70

75 થી વધુ માર્ક્સ વાળા વિદ્યાર્થીઓ:
------------------------------
નામ: John, માર્ક્સ: 80.0

કુલ હાઇ પર્ફોર્મર્સ: 1
\end{verbatim}

\textbf{યાદી રાખવાની ટ્રિક:} ``Input સ્ટોર ફિલ્ટર ડિસ્પ્લે''

\end{solutionbox}
\begin{center}\rule{0.5\linewidth}{0.5pt}\end{center}

\subsection*{પ્રશ્ન 4(અ) [3
ગુણ]}\label{uxaaauxab0uxab6uxaa8-4uxa85-3-uxa97uxaa3}

\textbf{રેન્ડમ મોડ્યુલમાં ઉપલબ્ધ કોઈપણ ત્રણ ફંક્શન લખો. દરેક ફંક્શનનું સિન્ટેક્સ અને
ઉદાહરણ લખો.}

\begin{solutionbox}

\textbf{રેન્ડમ મોડ્યુલ ફંક્શન્સ:}

{\def\LTcaptype{none} % do not increment counter
\begin{longtable}[]{@{}
  >{\raggedright\arraybackslash}p{(\linewidth - 6\tabcolsep) * \real{0.2778}}
  >{\raggedright\arraybackslash}p{(\linewidth - 6\tabcolsep) * \real{0.2222}}
  >{\raggedright\arraybackslash}p{(\linewidth - 6\tabcolsep) * \real{0.2500}}
  >{\raggedright\arraybackslash}p{(\linewidth - 6\tabcolsep) * \real{0.2500}}@{}}
\toprule\noalign{}
\begin{minipage}[b]{\linewidth}\raggedright
ફંક્શન
\end{minipage} & \begin{minipage}[b]{\linewidth}\raggedright
સિન્ટેક્સ
\end{minipage} & \begin{minipage}[b]{\linewidth}\raggedright
હેતુ
\end{minipage} & \begin{minipage}[b]{\linewidth}\raggedright
ઉદાહરણ
\end{minipage} \\
\midrule\noalign{}
\endhead
\bottomrule\noalign{}
\endlastfoot
\textbf{random()} & \texttt{random.random()} & 0.0 થી 1.0 સુધી રેન્ડમ ફ્લોટ
& \texttt{0.7534} \\
\textbf{randint()} & \texttt{random.randint(a,\ b)} & a થી b સુધી રેન્ડમ
ઇન્ટીજર & \texttt{randint(1,\ 10)} \\
\textbf{choice()} & \texttt{random.choice(seq)} & સિક્વન્સમાંથી રેન્ડમ ઘટક &
\texttt{choice([\textquotesingle{}a\textquotesingle{},\ \textquotesingle{}b\textquotesingle{},\ \textquotesingle{}c\textquotesingle{}])} \\
\end{longtable}
}

\textbf{કોડ ઉદાહરણ:}

\begin{verbatim}
import random

\# random() {- 0.0 અને 1.0 વચ્ચે ફ્લોટ બનાવે છે}
num = random.random()
print(num)  \# ઉદાહરણ: 0.7234567

\# randint() {- આપેલ રેન્જ વચ્ચે ઇન્ટીજર બનાવે છે}
dice = random.randint(1, 6)
print(dice)  \# ઉદાહરણ: 4

\# choice() {- સિક્વન્સમાંથી રેન્ડમ ઘટક પસંદ કરે છે}
colors = [{red}, {blue}, {green}]
selected = random.choice(colors)
print(selected)  \# ઉદાહરણ: {blue}
\end{verbatim}

\textbf{યાદી રાખવાની ટ્રિક:} ``Random Randint Choice''

\end{solutionbox}
\begin{center}\rule{0.5\linewidth}{0.5pt}\end{center}

\subsection*{પ્રશ્ન 4(બ) [4
ગુણ]}\label{uxaaauxab0uxab6uxaa8-4uxaac-4-uxa97uxaa3}

\textbf{ફંક્શનના ફાયદા લખો.}

\begin{solutionbox}

\textbf{ફંક્શનના ફાયદા:}

{\def\LTcaptype{none} % do not increment counter
\begin{longtable}[]{@{}ll@{}}
\toprule\noalign{}
ફાયદો & વર્ણન \\
\midrule\noalign{}
\endhead
\bottomrule\noalign{}
\endlastfoot
\textbf{કોડ પુનઃઉપયોગ} & એકવાર લખો, અનેકવાર વાપરો \\
\textbf{મોડ્યુલરિટી} & મોટા પ્રોગ્રામને નાના ભાગોમાં વહેંચો \\
\textbf{સરળ ડીબગિંગ} & એરર્સ અલગ પાડીને સહેલાઈથી ઠીક કરો \\
\textbf{વાંચવાની સુવિધા} & કોડ વધુ સંગઠિત અને સ્પષ્ટ બનાવે છે \\
\textbf{જાળવણી} & સહેલાઈથી અપડેટ અને બદલાવ કરી શકાય છે \\
\textbf{પુનરાવર્તન ટાળો} & ડુપ્લિકેટ કોડ ઓછો કરે છે \\
\end{longtable}
}

\textbf{ઉદાહરણ:}

\begin{verbatim}
\# ફંક્શન વગર (પુનરાવર્તન)
num1 = 5
square1 = num1 * num1
print(square1)

num2 = 8
square2 = num2 * num2
print(square2)

\# ફંક્શન સાથે (પુનઃઉપયોગ)
def calculate\_square(num):
    return num * num

print(calculate\_square(5))  \# 25
print(calculate\_square(8))  \# 64
\end{verbatim}

\textbf{યાદી રાખવાની ટ્રિક:} ``પુનઃઉપયોગ મોડ્યુલર ડીબગ વાંચો જાળવો ટાળો''

\end{solutionbox}
\begin{center}\rule{0.5\linewidth}{0.5pt}\end{center}

\subsection*{પ્રશ્ન 4(ક) [7
ગુણ]}\label{uxaaauxab0uxab6uxaa8-4uxa95-7-uxa97uxaa3}

\textbf{એક પ્રોગ્રામ લખો જે વપરાશકર્તાને સ્ટ્રિંગ માટે પૂછે અને સ્ટ્રિંગમાં દરેક `a' નું
સ્થાન પ્રિન્ટ કરે.}

\begin{solutionbox}

\begin{verbatim}
\# વપરાશકર્તા પાસેથી સ્ટ્રિંગ input લો
text = input("સ્ટ્રિંગ દાખલ કરો: ")

\# {a ની બધી positions શોધો}
positions = []
for i in range(len(text)):
    if text[i].lower() == {a}:  \# {a અને A બંને માટે ચેક કરો}
        positions.append(i)

\# પરિણામો દર્શાવો
if positions:
    print(f"અક્ષર {a આ positions પર મળ્યું: }\{positions\}")
    print("વિગતવાર સ્થાનો:")
    for pos in positions:
        print(f"Position \{pos\}: {}\{text[pos]\}{"})
else:
    print("અક્ષર {a સ્ટ્રિંગમાં મળ્યું નથી"})

\# enumerate વાપરીને વૈકલ્પિક રીત
print("{n}વૈકલ્પિક રીત:")
for index, char in enumerate(text):
    if char.lower() == {a}:
        print(f"{a position }\{index\} પર મળ્યું")
\end{verbatim}

\textbf{સેમ્પલ આઉટપુટ:}

\begin{verbatim}
સ્ટ્રિંગ દાખલ કરો: Python Programming

અક્ષર 'a' આ positions પર મળ્યું: [12]
વિગતવાર સ્થાનો:
Position 12: 'a'

વૈકલ્પિક રીત:
'a' position 12 પર મળ્યું
\end{verbatim}

\textbf{સુધારેલું વર્ઝન:}

\begin{verbatim}
text = input("સ્ટ્રિંગ દાખલ કરો: ")
count = 0

print(f"{}\{text\}{ માં a શોધી રહ્યા છીએ"})
print("{-"} * 30)

for i, char in enumerate(text):
    if char.lower() == {a}:
        count += 1
        print(f"{a ઇન્ડેક્સ }\{i\} પર મળ્યું (અક્ષર: {}\{char\}{)"})

print(f"{n}{a ની કુલ આવૃત્તિઓ: }\{count\}")
\end{verbatim}

\textbf{યાદી રાખવાની ટ્રિક:} ``Input લૂપ ચેક સ્ટોર ડિસ્પ્લે''

\end{solutionbox}
\begin{center}\rule{0.5\linewidth}{0.5pt}\end{center}

\subsection*{પ્રશ્ન 4(અ OR) [3
ગુણ]}\label{uxaaauxab0uxab6uxaa8-4uxa85-or-3-uxa97uxaa3}

\textbf{લોકલ અને ગ્લોબલ વેરિયેબલ સમજાવો.}

\begin{solutionbox}

\textbf{વેરિયેબલ સ્કોપ પ્રકારો:}

{\def\LTcaptype{none} % do not increment counter
\begin{longtable}[]{@{}llll@{}}
\toprule\noalign{}
વેરિયેબલ પ્રકાર & સ્કોપ & એક્સેસ & ઉદાહરણ \\
\midrule\noalign{}
\endhead
\bottomrule\noalign{}
\endlastfoot
\textbf{લોકલ} & ફંક્શનની અંદર જ & ફંક્શનની અંદર &
\texttt{def\ func():\ x\ =\ 5} \\
\textbf{ગ્લોબલ} & આખા પ્રોગ્રામમાં & પ્રોગ્રામમાં ગમે ત્યાં & \texttt{x\ =\ 5}
(ફંક્શનની બહાર) \\
\end{longtable}
}

\textbf{કોડ ઉદાહરણ:}

\begin{verbatim}
\# ગ્લોબલ વેરિયેબલ
global\_var = "હું ગ્લોબલ છું"

def my\_function():
    \# લોકલ વેરિયેબલ
    local\_var = "હું લોકલ છું"
    print(global\_var)  \# ગ્લોબલ એક્સેસ કરી શકાય છે
    print(local\_var)   \# લોકલ એક્સેસ કરી શકાય છે

my\_function()
print(global\_var)      \# ગ્લોબલ એક્સેસ કરી શકાય છે
\# print(local\_var)     \# એરર {- લોકલ એક્સેસ કરી શકાતું નથી}
\end{verbatim}

\textbf{ગ્લોબલ કીવર્ડ:}

\begin{verbatim}
counter = 0  \# ગ્લોબલ વેરિયેબલ

def increment():
    global counter  \# બદલવા માટે ગ્લોબલ તરીકે declare કરો
    counter += 1

increment()
print(counter)  \# આઉટપુટ: 1
\end{verbatim}

\textbf{યાદી રાખવાની ટ્રિક:} ``લોકલ અંદર ગ્લોબલ બધે''

\end{solutionbox}
\begin{center}\rule{0.5\linewidth}{0.5pt}\end{center}

\subsection*{પ્રશ્ન 4(બ OR) [4
ગુણ]}\label{uxaaauxab0uxab6uxaa8-4uxaac-or-4-uxa97uxaa3}

\textbf{યુઝર ડિફાઇન્ડ ફંક્શનનું બનાવટ અને ઉપયોગ ઉદાહરણ સાથે સમજાવો.}

\begin{solutionbox}

\textbf{ફંક્શન બનાવટ સિન્ટેક્સ:}

\begin{verbatim}
def function\_name(parameters):
    """વૈકલ્પિક docstring"""
    \# ફંક્શન બોડી
    return value  \# વૈકલ્પિક
\end{verbatim}

\textbf{ફંક્શનના ઘટકો:}

{\def\LTcaptype{none} % do not increment counter
\begin{longtable}[]{@{}lll@{}}
\toprule\noalign{}
ઘટક & હેતુ & ઉદાહરણ \\
\midrule\noalign{}
\endhead
\bottomrule\noalign{}
\endlastfoot
\textbf{def} & ફંક્શન વ્યાખ્યાયિત કરવાનું કીવર્ડ & \texttt{def} \\
\textbf{function\_name} & ફંક્શનનું નામ & \texttt{calculate\_area} \\
\textbf{parameters} & ઇનપુટ મૂલ્યો & \texttt{(length,\ width)} \\
\textbf{return} & આઉટપુટ મૂલ્ય & \texttt{return\ result} \\
\end{longtable}
}

\textbf{ઉદાહરણ:}

\begin{verbatim}
\# ફંક્શન વ્યાખ્યા
def greet\_user(name, age):
    """નામ અને ઉંમર સાથે વપરાશકર્તાને શુભેચ્છા આપવાનું ફંક્શન"""
    message = f"હેલો \{name\}! તમારી ઉંમર \{age\} વર્ષ છે."
    return message

\# ફંક્શન કૉલ
user\_name = "જોહન"
user\_age = 25
greeting = greet\_user(user\_name, user\_age)
print(greeting)  \# આઉટપુટ: હેલો જોહન! તમારી ઉંમર 25 વર્ષ છે.

\# ડિફોલ્ટ પેરામીટર સાથે ફંક્શન
def calculate\_power(base, exponent=2):
    return base ** exponent

print(calculate\_power(5))     \# 25 (ડિફોલ્ટ exponent=2 વાપરીને)
print(calculate\_power(5, 3))  \# 125 (exponent=3 વાપરીને)
\end{verbatim}

\textbf{યાદી રાખવાની ટ્રિક:} ``વ્યાખ્યાયિત કૉલ રિટર્ન પેરામીટર''

\end{solutionbox}
\begin{center}\rule{0.5\linewidth}{0.5pt}\end{center}

\subsection*{પ્રશ્ન 4(ક OR) [7
ગુણ]}\label{uxaaauxab0uxab6uxaa8-4uxa95-or-7-uxa97uxaa3}

\textbf{calcFact() નામનું યુઝર ડિફાઇન્ડ ફંક્શન બનાવવા માટેનો પ્રોગ્રામ લખો કે જે
આર્ગ્યુમેન્ટ તરીકે આપેલ સંખ્યાના ફેક્ટોરિયલની ગણતરી કરી તેને ડિસ્પ્લે કરે.}

\begin{solutionbox}

\begin{verbatim}
def calcFact(number):
    """
    સંખ્યાનું ફેક્ટોરિયલ કેલ્ક્યુલેટ કરવાનું ફંક્શન
    ઇનપુટ: number (integer)
    આઉટપુટ: factorial (integer)
    """
    if number {} 0:
        return "નેગેટિવ સંખ્યા માટે ફેક્ટોરિયલ વ્યાખ્યાયિત નથી"
    elif number == 0 or number == 1:
        return 1
    else:
        factorial = 1
        for i in range(2, number + 1):
            factorial *= i
        return factorial

\# મુખ્ય પ્રોગ્રામ
try:
    \# વપરાશકર્તા પાસેથી ઇનપુટ
    num = int(input("સંખ્યા દાખલ કરો: "))
    
    \# ફંક્શન કૉલ
    result = calcFact(num)
    
    \# પરિણામ દર્શાવો
    if isinstance(result, str):
        print(result)
    else:
        print(f"\{num\} નું ફેક્ટોરિયલ છે: \{result\}")
        
except ValueError:
    print("કૃપા કરીને યોગ્ય ઇન્ટીજર દાખલ કરો")

\# વિવિધ મૂલ્યો સાથે ટેસ્ટ કરો
print("{n}વિવિધ મૂલ્યો સાથે ટેસ્ટિંગ:")
test\_values = [0, 1, 5, 10, {-}3]
for val in test\_values:
    result = calcFact(val)
    print(f"calcFact(\{val\}) = \{result\}")
\end{verbatim}

\textbf{રીકર્સિવ વર્ઝન:}

\begin{verbatim}
def calcFactRecursive(n):
    """ફેક્ટોરિયલ કેલ્ક્યુલેટ કરવાનું રીકર્સિવ ફંક્શન"""
    if n {} 0:
        return "નેગેટિવ સંખ્યા માટે અવ્યાખ્યાયિત"
elif

n == 0 or

n == 1:

        return 1
    else:
        return n * calcFactRecursive(n {-} 1)

\# ઉદાહરણ ઉપયોગ
number = int(input("સંખ્યા દાખલ કરો: "))
result = calcFactRecursive(number)
print(f"ફેક્ટોરિયલ: \{result\}")
\end{verbatim}

\textbf{સેમ્પલ આઉટપુટ:}

\begin{verbatim}
સંખ્યા દાખલ કરો: 5
5 નું ફેક્ટોરિયલ છે: 120

વિવિધ મૂલ્યો સાથે ટેસ્ટિંગ:
calcFact(0) = 1
calcFact(1) = 1
calcFact(5) = 120
calcFact(10) = 3628800
calcFact(-3) = નેગેટિવ સંખ્યા માટે ફેક્ટોરિયલ વ્યાખ્યાયિત નથી
\end{verbatim}

\textbf{યાદી રાખવાની ટ્રિક:} ``વ્યાખ્યાયિત ચેક લૂપ ગુણાકાર રિટર્ન''

\end{solutionbox}
\begin{center}\rule{0.5\linewidth}{0.5pt}\end{center}

\subsection*{પ્રશ્ન 5(અ) [3
ગુણ]}\label{uxaaauxab0uxab6uxaa8-5uxa85-3-uxa97uxaa3}

\textbf{ક્લાસ અને ઓબ્જેક્ટ વચ્ચે તફાવત આપો.}

\begin{solutionbox}

\textbf{ક્લાસ વર્સિસ ઓબ્જેક્ટ તુલના:}

{\def\LTcaptype{none} % do not increment counter
\begin{longtable}[]{@{}
  >{\raggedright\arraybackslash}p{(\linewidth - 4\tabcolsep) * \real{0.3478}}
  >{\raggedright\arraybackslash}p{(\linewidth - 4\tabcolsep) * \real{0.3043}}
  >{\raggedright\arraybackslash}p{(\linewidth - 4\tabcolsep) * \real{0.3478}}@{}}
\toprule\noalign{}
\begin{minipage}[b]{\linewidth}\raggedright
બાબત
\end{minipage} & \begin{minipage}[b]{\linewidth}\raggedright
ક્લાસ
\end{minipage} & \begin{minipage}[b]{\linewidth}\raggedright
ઓબ્જેક્ટ
\end{minipage} \\
\midrule\noalign{}
\endhead
\bottomrule\noalign{}
\endlastfoot
\textbf{વ્યાખ્યા} & બ્લૂપ્રિન્ટ/ટેમ્પ્લેટ & ક્લાસનું ઇન્સ્ટન્સ \\
\textbf{મેમરી} & મેમરી એલોકેટ નથી & મેમરી એલોકેટ છે \\
\textbf{બનાવટ} & \texttt{class} કીવર્ડ વાપરીને વ્યાખ્યાયિત & ક્લાસના નામ
વાપરીને બનાવાય છે \\
\textbf{એટ્રિબ્યુટ્સ} & વ્યાખ્યાયિત પરંતુ ઇનિશિયલાઇઝ નથી & વાસ્તવિક મૂલ્યો છે \\
\textbf{ઉદાહરણ} & \texttt{class\ Car:} & \texttt{my\_car\ =\ Car()} \\
\end{longtable}
}

\textbf{કોડ ઉદાહરણ:}

\begin{verbatim}
\# ક્લાસ વ્યાખ્યા (બ્લૂપ્રિન્ટ)
class Student:
    def \_\_init\_\_(self, name, age):
        self.name = name
        self.age = age

\# ઓબ્જેક્ટ બનાવટ (ઇન્સ્ટન્સ)
student1 = Student("જોહન", 20)  \# ઓબ્જેક્ટ 1
student2 = Student("આલિસ", 19) \# ઓબ્જેક્ટ 2

print(student1.name)  \# જોહન
print(student2.name)  \# આલિસ
\end{verbatim}

\textbf{યાદી રાખવાની ટ્રિક:} ``ક્લાસ બ્લૂપ્રિન્ટ ઓબ્જેક્ટ ઇન્સ્ટન્સ''

\end{solutionbox}
\begin{center}\rule{0.5\linewidth}{0.5pt}\end{center}

\subsection*{પ્રશ્ન 5(બ) [4
ગુણ]}\label{uxaaauxab0uxab6uxaa8-5uxaac-4-uxa97uxaa3}

\textbf{ક્લાસમાં કન્સ્ટ્રક્ટરનો હેતુ જણાવો.}

\begin{solutionbox}

\textbf{કન્સ્ટ્રક્ટરનો હેતુ:}

{\def\LTcaptype{none} % do not increment counter
\begin{longtable}[]{@{}ll@{}}
\toprule\noalign{}
હેતુ & વર્ણન \\
\midrule\noalign{}
\endhead
\bottomrule\noalign{}
\endlastfoot
\textbf{ઓબ્જેક્ટ ઇનિશિયલાઇઝ કરો} & એટ્રિબ્યુટ્સને પ્રારંભિક મૂલ્યો આપો \\
\textbf{ઓટોમેટિક એક્ઝીક્યુશન} & ઓબ્જેક્ટ બનાવતી વખતે આપોઆપ કૉલ થાય છે \\
\textbf{મેમરી સેટઅપ} & ઓબ્જેક્ટ એટ્રિબ્યુટ્સ માટે મેમરી એલોકેટ કરે છે \\
\textbf{ડિફોલ્ટ મૂલ્યો} & એટ્રિબ્યુટ્સને ડિફોલ્ટ મૂલ્યો આપે છે \\
\end{longtable}
}

\textbf{કન્સ્ટ્રક્ટરના પ્રકારો:}

{\def\LTcaptype{none} % do not increment counter
\begin{longtable}[]{@{}lll@{}}
\toprule\noalign{}
પ્રકાર & વર્ણન & ઉદાહરણ \\
\midrule\noalign{}
\endhead
\bottomrule\noalign{}
\endlastfoot
\textbf{ડિફોલ્ટ} & કોઈ પેરામીટર નથી & \texttt{def\ \_\_init\_\_(self):} \\
\textbf{પેરામીટરાઇઝ્ડ} & પેરામીટર લે છે &
\texttt{def\ \_\_init\_\_(self,\ name):} \\
\end{longtable}
}

\textbf{ઉદાહરણ:}

\begin{verbatim}
class Rectangle:
    def \_\_init\_\_(self, length=0, width=0):  \# કન્સ્ટ્રક્ટર
        self.length = length  \# એટ્રિબ્યુટ ઇનિશિયલાઇઝ કરો
        self.width = width    \# એટ્રિબ્યુટ ઇનિશિયલાઇઝ કરો
        print("રેક્ટેંગલ ઓબ્જેક્ટ બન્યું!")
    
    def area(self):
        return self.length * self.width

\# ઓબ્જેક્ટ બનાવટ {- કન્સ્ટ્રક્ટર આપોઆપ કૉલ થાય છે}
rect1 = Rectangle(10, 5)  \# આઉટપુટ: રેક્ટેંગલ ઓબ્જેક્ટ બન્યું!
rect2 = Rectangle()       \# ડિફોલ્ટ મૂલ્યો વાપરે છે

print(rect1.area())       \# 50
print(rect2.area())       \# 0
\end{verbatim}

\textbf{યાદી રાખવાની ટ્રિક:} ``ઇનિશિયલાઇઝ ઓટોમેટિક મેમરી ડિફોલ્ટ''

\end{solutionbox}
\begin{center}\rule{0.5\linewidth}{0.5pt}\end{center}

\subsection*{પ્રશ્ન 5(ક) [7
ગુણ]}\label{uxaaauxab0uxab6uxaa8-5uxa95-7-uxa97uxaa3}

\textbf{``Student'' નામનો ક્લાસ બનાવવા માટે પ્રોગ્રામ લખો જેમાં નામ, રોલ નંબર
અને માર્ક્સ જેવા એટ્રિબ્યુટ્સ હોય. વિદ્યાર્થીની માહિતી પ્રદર્શિત કરવાની મેથડ બનાવો.
``Student'' ક્લાસનો ઓબ્જેક્ટ બનાવો અને મેથડનો ઉપયોગ કેવી રીતે કરવો તે બતાવો.}

\begin{solutionbox}

\begin{verbatim}
class Student:
    def \_\_init\_\_(self, name, roll\_number, marks):
        """વિદ્યાર્થી એટ્રિબ્યુટ્સ ઇનિશિયલાઇઝ કરવા માટે કન્સ્ટ્રક્ટર"""
        self.name = name
        self.roll\_number = roll\_number
        self.marks = marks
    
    def display\_info(self):
        """વિદ્યાર્થીની માહિતી દર્શાવવાની મેથડ"""
        print("{-"} * 30)
        print("વિદ્યાર્થીની માહિતી")
        print("{-"} * 30)
        print(f"નામ: \{self.name\}")
        print(f"રોલ નંબર: \{self.roll\_number\}")
        print(f"માર્ક્સ: \{self.marks\}")
        print("{-"} * 30)
    
    def calculate\_grade(self):
        """માર્ક્સ આધારે ગ્રેડ કેલ્ક્યુલેટ કરવાની મેથડ"""
        if self.marks {=} 90:
            return {A+}
        elif self.marks {=} 80:
            return {A}
        elif self.marks {=} 70:
            return {B}
        elif self.marks {=} 60:
            return {C}
        else:
            return {F}
    
    def display\_grade(self):
        """ગ્રેડ દર્શાવવાની મેથડ"""
        grade = self.calculate\_grade()
        print(f"ગ્રેડ: \{grade\}")

\# Student ક્લાસના ઓબ્જેક્ટ્સ બનાવવું
print("Student ઓબ્જેક્ટ્સ બનાવી રહ્યા છીએ:")
student1 = Student("જોહન દોય", 101, 85)
student2 = Student("આલિસ સ્મિથ", 102, 92)
student3 = Student("બોબ જોહન્સન", 103, 78)

\# માહિતી દર્શાવવા માટે મેથડ્સનો ઉપયોગ
print("{n}=== વિદ્યાર્થી 1 ની વિગતો ===")
student1.display\_info()
student1.display\_grade()

print("{n}=== વિદ્યાર્થી 2 ની વિગતો ===")
student2.display\_info()
student2.display\_grade()

print("{n}=== વિદ્યાર્થી 3 ની વિગતો ===")
student3.display\_info()
student3.display\_grade()

\# એટ્રિબ્યુટ્સને સીધી એક્સેસ કરવું
print(f"{n}સીધી એક્સેસ {- વિદ્યાર્થી 1 નું નામ: }\{student1.name\}")
print(f"સીધી એક્સેસ {- વિદ્યાર્થી 2 ના માર્ક્સ: }\{student2.marks\}")
\end{verbatim}

\textbf{સેમ્પલ આઉટપુટ:}

\begin{verbatim}
Student ઓબ્જેક્ટ્સ બનાવી રહ્યા છીએ:

=== વિદ્યાર્થી 1 ની વિગતો ===
------------------------------
વિદ્યાર્થીની માહિતી
------------------------------
નામ: જોહન દોય
રોલ નંબર: 101
માર્ક્સ: 85
------------------------------
ગ્રેડ: A

=== વિદ્યાર્થી 2 ની વિગતો ===
------------------------------
વિદ્યાર્થીની માહિતી
------------------------------
નામ: આલિસ સ્મિથ
રોલ નંબર: 102
માર્ક્સ: 92
------------------------------
ગ્રેડ: A+
\end{verbatim}

\textbf{ક્લાસના ઘટકો:}

\begin{itemize}
\tightlist
\item
  \textbf{એટ્રિબ્યુટ્સ}: name, roll\_number, marks
\item
  \textbf{કન્સ્ટ્રક્ટર}: \texttt{\_\_init\_\_()} મેથડ
\item
  \textbf{મેથડ્સ}: display\_info(), calculate\_grade(), display\_grade()
\item
  \textbf{ઓબ્જેક્ટ્સ}: student1, student2, student3
\end{itemize}

\textbf{યાદી રાખવાની ટ્રિક:} ``ક્લાસ એટ્રિબ્યુટ્સ કન્સ્ટ્રક્ટર મેથડ્સ ઓબ્જેક્ટ્સ''

\end{solutionbox}
\begin{center}\rule{0.5\linewidth}{0.5pt}\end{center}

\subsection*{પ્રશ્ન 5(અ OR) [3
ગુણ]}\label{uxaaauxab0uxab6uxaa8-5uxa85-or-3-uxa97uxaa3}

\textbf{એન્કેપ્સ્યુલેશનનો હેતુ જણાવો.}

\begin{solutionbox}

\textbf{એન્કેપ્સ્યુલેશનનો હેતુ:}

{\def\LTcaptype{none} % do not increment counter
\begin{longtable}[]{@{}ll@{}}
\toprule\noalign{}
હેતુ & વર્ણન \\
\midrule\noalign{}
\endhead
\bottomrule\noalign{}
\endlastfoot
\textbf{ડેટા છુપાવવું} & આંતરિક implementation વિગતો છુપાવે છે \\
\textbf{ડેટા સુરક્ષા} & અનધિકૃત એક્સેસથી ડેટાને બચાવે છે \\
\textbf{નિયંત્રિત એક્સેસ} & મેથડ્સ દ્વારા નિયંત્રિત એક્સેસ આપે છે \\
\textbf{કોડ સિક્યુરિટી} & ડેટાના આકસ્મિક ફેરફારને અટકાવે છે \\
\textbf{મોડ્યુલરિટી} & સંબંધિત ડેટા અને મેથડ્સ એકસાથે રાખે છે \\
\end{longtable}
}

\textbf{અમલીકરણ ઉદાહરણ:}

\begin{verbatim}
class BankAccount:
    def \_\_init\_\_(self, balance):
        self.\_\_balance = balance  \# પ્રાઇવેટ એટ્રિબ્યુટ
    
    def get\_balance(self):       \# Getter મેથડ
        return self.\_\_balance
    
    def deposit(self, amount):   \# નિયંત્રિત એક્સેસ
        if amount {} 0:
            self.\_\_balance += amount

account = BankAccount(1000)
print(account.get\_balance())     \# 1000
\# print(account.\_\_balance)       \# એરર {- સીધી એક્સેસ કરી શકાતી નથી}
\end{verbatim}

\textbf{ફાયદા:}

\begin{itemize}
\tightlist
\item
  \textbf{સિક્યુરિટી}: ડેટાને સીધી એક્સેસ કરી શકાતી નથી
\item
  \textbf{જાળવણી}: આંતરિક implementation સહેલાઈથી બદલી શકાય છે
\item
  \textbf{વેલિડેશન}: getter/setter મેથડ્સમાં વેલિડેશન ઉમેરી શકાય છે
\end{itemize}

\textbf{યાદી રાખવાની ટ્રિક:} ``છુપાવો સુરક્ષા નિયંત્રણ સિક્યુર મોડ્યુલર''

\end{solutionbox}
\begin{center}\rule{0.5\linewidth}{0.5pt}\end{center}

\subsection*{પ્રશ્ન 5(બ OR) [4
ગુણ]}\label{uxaaauxab0uxab6uxaa8-5uxaac-or-4-uxa97uxaa3}

\textbf{મલ્ટિલેવલ ઇન્હેરિટન્સ સમજાવો.}

\begin{solutionbox}

\textbf{મલ્ટિલેવલ ઇન્હેરિટન્સ} એ જ્યારે એક ક્લાસ બીજી ક્લાસમાંથી ઇન્હેરિટ કરે છે, જે
બદલામાં બીજી ક્લાસમાંથી ઇન્હેરિટ કરે છે, આમ એક શૃંખલા બને છે.

\textbf{સ્ટ્રક્ચર ડાયાગ્રામ:}

\begin{verbatim}
    +{-{-}{-}{-}{-}{-}{-}{-}{-}{-}+}
    | GrandPa  |  (Base Class)
    +{-{-}{-}{-}{-}{-}{-}{-}{-}{-}+}
         \^{}
         |
    +{-{-}{-}{-}{-}{-}{-}{-}{-}{-}+}
    |  Parent  |  (Derived from GrandPa)
    +{-{-}{-}{-}{-}{-}{-}{-}{-}{-}+}
         \^{}
         |
    +{-{-}{-}{-}{-}{-}{-}{-}{-}{-}+}
    |  Child   |  (Derived from Parent)
    +{-{-}{-}{-}{-}{-}{-}{-}{-}{-}+}
\end{verbatim}

\textbf{લક્ષણો ટેબલ:}

{\def\LTcaptype{none} % do not increment counter
\begin{longtable}[]{@{}llll@{}}
\toprule\noalign{}
લેવલ & ક્લાસ & ઇન્હેરિટ કરે છે & એક્સેસ કરે છે \\
\midrule\noalign{}
\endhead
\bottomrule\noalign{}
\endlastfoot
\textbf{લેવલ 1} & GrandPa & કોઈથી નહીં & પોતાની મેથડ્સ \\
\textbf{લેવલ 2} & Parent & GrandPa & GrandPa + પોતાની મેથડ્સ \\
\textbf{લેવલ 3} & Child & Parent & GrandPa + Parent + પોતાની \\
\end{longtable}
}

\textbf{કોડ ઉદાહરણ:}

\begin{verbatim}
\# લેવલ 1 {- મૂળ ક્લાસ}
class Vehicle:
    def \_\_init\_\_(self, brand):
        self.brand = brand
    
    def start(self):
        print(f"\{self.brand\} વાહન શરૂ થયું")

\# લેવલ 2 {- Vehicle માંથી ઇન્હેરિટ}
class Car(Vehicle):
    def \_\_init\_\_(self, brand, model):
        super().\_\_init\_\_(brand)
        self.model = model
    
    def drive(self):
        print(f"\{self.brand\} \{self.model\} ચાલી રહી છે")

\# લેવલ 3 {- Car માંથી ઇન્હેરિટ}
class SportsCar(Car):
    def \_\_init\_\_(self, brand, model, top\_speed):
        super().\_\_init\_\_(brand, model)
        self.top\_speed = top\_speed
    
    def race(self):
        print(f"\{self.brand\} \{self.model\} \{self.top\_speed\} km/h ઝડપે રેસ કરી રહી છે")

\# ઓબ્જેક્ટ બનાવીને મેથડ્સ વાપરવા
ferrari = SportsCar("Ferrari", "F8", 340)
ferrari.start()    \# Vehicle ક્લાસમાંથી
ferrari.drive()    \# Car ક્લાસમાંથી
ferrari.race()     \# SportsCar ક્લાસમાંથી
\end{verbatim}

\textbf{યાદી રાખવાની ટ્રિક:} ``શૃંખલા ઇન્હેરિટ લેવલ એક્સેસ''

\end{solutionbox}
\begin{center}\rule{0.5\linewidth}{0.5pt}\end{center}

\subsection*{પ્રશ્ન 5(ક OR) [7
ગુણ]}\label{uxaaauxab0uxab6uxaa8-5uxa95-or-7-uxa97uxaa3}

\textbf{હાઇબ્રિડ ઇન્હેરિટન્સનું કાર્ય દર્શાવતો પાયથોન પ્રોગ્રામ લખો.}

\begin{solutionbox}

\textbf{હાઇબ્રિડ ઇન્હેરિટન્સ} એક પ્રોગ્રામમાં બહુવિધ પ્રકારની ઇન્હેરિટન્સ (સિંગલ,
મલ્ટિપલ, મલ્ટિલેવલ) ને જોડે છે.

\textbf{સ્ટ્રક્ચર ડાયાગ્રામ:}

\begin{verbatim}
    +{-{-}{-}{-}{-}{-}{-}{-}{-}{-}+}
    |  Animal  |  (Base Class)
    +{-{-}{-}{-}{-}{-}{-}{-}{-}{-}+}
         \^{}
         |
    +{-{-}{-}{-}{-}{-}{-}{-}{-}{-}+}
    | Mammal   |  (Single Inheritance)
    +{-{-}{-}{-}{-}{-}{-}{-}{-}{-}+}
         \^{}
         |
    +{-{-}{-}{-}{-}{-}{-}{-}{-}{-}+     +{-}{-}{-}{-}{-}{-}{-}{-}{-}{-}+}
    |   Dog    |     |   Bird   |  (Single Inheritance)
    +{-{-}{-}{-}{-}{-}{-}{-}{-}{-}+     +{-}{-}{-}{-}{-}{-}{-}{-}{-}{-}+}
         \^{                 \^{}}
         |                 |
         +{-{-}{-}{-}{-}{-}{-}+{-}{-}{-}{-}{-}{-}{-}{-}{-}+}
                 |
         +{-{-}{-}{-}{-}{-}{-}{-}{-}{-}{-}{-}{-}{-}{-}+}
         |  FlyingDog    |  (Multiple Inheritance)
         +{-{-}{-}{-}{-}{-}{-}{-}{-}{-}{-}{-}{-}{-}{-}+}
\end{verbatim}

\textbf{કોડ ઉદાહરણ:}

\begin{verbatim}
\# મૂળ ક્લાસ
class Animal:
    def \_\_init\_\_(self, name):
        self.name = name
        print(f"પ્રાણી \{self.name\} બન્યું")
    
    def eat(self):
        print(f"\{self.name\} ખાય છે")
    
    def sleep(self):
        print(f"\{self.name\} સૂએ છે")

\# Animal માંથી સિંગલ ઇન્હેરિટન્સ
class Mammal(Animal):
    def \_\_init\_\_(self, name, fur\_color):
        super().\_\_init\_\_(name)
        self.fur\_color = fur\_color
    
    def give\_birth(self):
        print(f"\{self.name\} જીવતા બાળકોને જન્મ આપે છે")

\# Animal માંથી સિંગલ ઇન્હેરિટન્સ
class Bird(Animal):
    def \_\_init\_\_(self, name, wing\_span):
        super().\_\_init\_\_(name)
        self.wing\_span = wing\_span
    
    def fly(self):
        print(f"\{self.name\} \{self.wing\_span\}cm પાંખો સાથે ઉડે છે")
    
    def lay\_eggs(self):
        print(f"\{self.name\} ઇંડા આપે છે")

\# Mammal માંથી સિંગલ ઇન્હેરિટન્સ
class Dog(Mammal):
    def \_\_init\_\_(self, name, fur\_color, breed):
        super().\_\_init\_\_(name, fur\_color)
        self.breed = breed
    
    def bark(self):
        print(f"\{self.name\} \{self.breed\} કૂકે છે")
    
    def guard(self):
        print(f"\{self.name\} ઘરની રક્ષા કરે છે")

\# Dog અને Bird માંથી મલ્ટિપલ ઇન્હેરિટન્સ (હાઇબ્રિડ)
class FlyingDog(Dog, Bird):
    def \_\_init\_\_(self, name, fur\_color, breed, wing\_span):
        \# બંને પેરેન્ટ ક્લાસને ઇનિશિયલાઇઝ કરો
        Dog.\_\_init\_\_(self, name, fur\_color, breed)
        Bird.\_\_init\_\_(self, name, wing\_span)
        print(f"જાદુઈ \{self.name\} બન્યું જેમાં mammal અને bird બંનેના લક્ષણ છે!")
    
    def fly\_and\_bark(self):
        print(f"\{self.name\} એક સાથે ઉડે છે અને કૂકે છે!")
    
    def show\_abilities(self):
        print(f"{n}\{self.name\} ની ક્ષમતાઓ:")
        print("{-"} * 25)
        self.eat()          \# Animal માંથી
        self.sleep()        \# Animal માંથી
        self.give\_birth()   \# Mammal માંથી
        self.bark()         \# Dog માંથી
        self.guard()        \# Dog માંથી
        self.fly()          \# Bird માંથી
        self.lay\_eggs()     \# Bird માંથી
        self.fly\_and\_bark() \# પોતાની મેથડ

\# પ્રદર્શન
print("=== હાઇબ્રિડ ઇન્હેરિટન્સ ડેમો ==={n}")

\# ઓબ્જેક્ટ્સ બનાવો
print("1. સામાન્ય કૂતરો બનાવી રહ્યા છીએ:")
dog1 = Dog("બડી", "સુવર્ણ", "રિટ્રીવર")
dog1.bark()
dog1.guard()

print("{n}2. સામાન્ય પક્ષી બનાવી રહ્યા છીએ:")
bird1 = Bird("ગરુડ", 200)
bird1.fly()
bird1.lay\_eggs()

print("{n}3. જાદુઈ ઉડતો કૂતરો બનાવી રહ્યા છીએ:")
flying\_dog = FlyingDog("સુપરડોગ", "રજતી", "હસ્કી", 150)
flying\_dog.show\_abilities()

\# મેથડ રિઝોલ્યુશન ઓર્ડર
print(f"{n}FlyingDog માટે મેથડ રિઝોલ્યુશન ઓર્ડર:")
for i, cls in enumerate(FlyingDog.\_\_mro\_\_):
    print(f"\{i+1\}. \{cls.\_\_name\_\_\}")
\end{verbatim}

\textbf{સેમ્પલ આઉટપુટ:}

\begin{verbatim}
=== હાઇબ્રિડ ઇન્હેરિટન્સ ડેમો ===

1. સામાન્ય કૂતરો બનાવી રહ્યા છીએ:
પ્રાણી બડી બન્યું
બડી રિટ્રીવર કૂકે છે
બડી ઘરની રક્ષા કરે છે

2. સામાન્ય પક્ષી બનાવી રહ્યા છીએ:
પ્રાણી ગરુડ બન્યું
ગરુડ 200cm પાંખો સાથે ઉડે છે
ગરુડ ઇંડા આપે છે

3. જાદુઈ ઉડતો કૂતરો બનાવી રહ્યા છીએ:
પ્રાણી સુપરડોગ બન્યું
પ્રાણી સુપરડોગ બન્યું
જાદુઈ સુપરડોગ બન્યું જેમાં mammal અને bird બંનેના લક્ષણ છે!

સુપરડોગ ની ક્ષમતાઓ:
-------------------------
સુપરડોગ ખાય છે
સુપરડોગ સૂએ છે
સુપરડોગ જીવતા બાળકોને જન્મ આપે છે
સુપરડોગ હસ્કી કૂકે છે
સુપરડોગ ઘરની રક્ષા કરે છે
સુપરડોગ 150cm પાંખો સાથે ઉડે છે
સુપરડોગ ઇંડા આપે છે
સુપરડોગ એક સાથે ઉડે છે અને કૂકે છે!
\end{verbatim}

\textbf{આ ઉદાહરણમાં ઇન્હેરિટન્સના પ્રકારો:}

\begin{enumerate}
\tightlist
\item
  \textbf{સિંગલ}: Mammal \leftarrow Animal, Bird \leftarrow Animal, Dog \leftarrow Mammal
\item
  \textbf{મલ્ટિપલ}: FlyingDog \leftarrow Dog + Bird
\item
  \textbf{મલ્ટિલેવલ}: FlyingDog \leftarrow Dog \leftarrow Mammal \leftarrow Animal
\item
  \textbf{હાઇબ્રિડ}: ઉપરોક્ત બધાનું સંયોજન
\end{enumerate}

\textbf{મુખ્ય લક્ષણો:}

\begin{itemize}
\tightlist
\item
  \textbf{મલ્ટિપલ પેરેન્ટ ક્લાસ}: FlyingDog Dog અને Bird બંનેમાંથી ઇન્હેરિટ કરે છે
\item
  \textbf{મેથડ રિઝોલ્યુશન ઓર્ડર}: Python MRO ને અનુસરીને મેથડ conflicts ને હલ કરે
  છે
\item
  \textbf{super() નો ઉપયોગ}: પેરેન્ટ ક્લાસના યોગ્ય ઇનિશિયલાઇઝેશન માટે
\item
  \textbf{સંયુક્ત કાર્યક્ષમતા}: બધી પેરેન્ટ ક્લાસની મેથડ્સ તકે પહોંચ
\end{itemize}

\textbf{યાદી રાખવાની ટ્રિક:} ``હાઇબ્રિડ મલ્ટિપલ સિંગલ મલ્ટિલેવલ સંયુક્ત''

\end{solutionbox}

\end{document}
