\documentclass[10pt,a4paper]{article}

% content/resources/templates/preamble.tex
\usepackage[margin=0.6in]{geometry}
\author{Milav Dabgar}
\usepackage{amsmath,amssymb,amsthm}
\usepackage{booktabs}
\usepackage{multirow}
\usepackage{xcolor}
\usepackage{tcolorbox}
\tcbuselibrary{breakable,skins}
\usepackage[colorlinks=true,linkcolor=blue]{hyperref}
\usepackage{titlesec}
\usepackage{enumitem}
\usepackage{tikz}
\usepackage{pgfplots}
\usepackage{circuitikz}
\usepackage[version=4]{mhchem}
\usepackage{longtable}
\usepackage{array}
\usepackage{float}
\usepackage{caption}
\usepackage{listings}

\lstset{
  basicstyle=\small\ttfamily,
  breaklines=true,
  breakatwhitespace=false,
  postbreak=\mbox{\textcolor{red}{$\hookrightarrow$}\space},
  float=false,
  numbers=left,
  numberstyle=\tiny\color{gray},
  numbersep=10pt,
  xleftmargin=2em,
  keywordstyle=\color{blue},
  commentstyle=\color{green!60!black},
  stringstyle=\color{purple},
  backgroundcolor=\color{gray!5},
  showstringspaces=false,
  tabsize=2,
  captionpos=b,
  keepspaces=true,
  columns=flexible
}

\pgfplotsset{compat=1.18}
\usetikzlibrary{shapes,arrows,positioning,calc,patterns,decorations.pathmorphing,decorations.markings,arrows.meta}

% Color scheme
\definecolor{headcolor}{RGB}{0,102,204}
\definecolor{keycolor}{RGB}{220,20,60}
\definecolor{solutioncolor}{RGB}{34,139,34}
\definecolor{mnemoniccolor}{RGB}{148,0,211}
\definecolor{codecolor}{RGB}{0,0,100}

% Spacing
\setlength{\parskip}{3pt}
\setlist[itemize]{nosep}
\setlist[enumerate]{nosep}

% Title formatting
\titleformat{\section}{\Large\bfseries\color{headcolor}}{\thesection}{1em}{}
\titleformat{\subsection}{\large\bfseries\color{headcolor}}{\thesubsection}{1em}{}

% Pandoc tightlist compatibility
\providecommand{\tightlist}{%
  \setlength{\itemsep}{0pt}\setlength{\parskip}{0pt}}

% Pandoc longtable compatibility
\newcounter{none}
\def\thenone{}


% content/resources/templates/english-boxes.tex
% This file is currently empty - it exists to maintain consistency with the import structure.
% Add custom environments here if needed in the future.


\begin{document}

\begin{center}
{\Huge\bfseries\color{headcolor} Subject Name Solutions}\\[5pt]
{\LARGE 4351108 -- Summer 2024}\\[3pt]
{\large Semester 1 Study Material}\\[3pt]
{\normalsize\textit{Detailed Solutions and Explanations}}
\end{center}

\vspace{10pt}

\subsection*{Question 1(a) [3 marks]}\label{q1a}

\textbf{Explain for loop working in Python.}

\begin{solutionbox}

For loop repeats code block for each item in sequence like list, tuple,
or string.

\textbf{Syntax Table:}

{\def\LTcaptype{none} % do not increment counter
\begin{longtable}[]{@{}lll@{}}
\toprule\noalign{}
Component & Syntax & Example \\
\midrule\noalign{}
\endhead
\bottomrule\noalign{}
\endlastfoot
Basic & \texttt{for\ variable\ in\ sequence:} &
\texttt{for\ i\ in\ [1,2,3]:} \\
Range & \texttt{for\ i\ in\ range(n):} &
\texttt{for\ i\ in\ range(5):} \\
String & \texttt{for\ char\ in\ string:} &
\texttt{for\ c\ in\ "hello":} \\
\end{longtable}
}

\textbf{Diagram:}

\begin{verbatim}
Start {-{-} Check if items left in sequence}
         |
         v
    Execute loop body
         |
         v
    Move to next item {-{-} Check if items left}
         |                      |
         v                      v
    Items left? {-{-}{-}{-}No{-}{-}{-}{-} End}
         |
        Yes
         |
         v
    Back to Execute loop body
\end{verbatim}

\begin{itemize}
\tightlist
\item
  \textbf{Iteration}: Loop variable gets each value from sequence one by
  one
\item
  \textbf{Automatic}: Python handles moving to next item automatically
\item
  \textbf{Flexible}: Works with lists, strings, tuples, ranges
\end{itemize}

\end{solutionbox}
\begin{mnemonicbox}
``For Each Item, Execute Block''

\end{mnemonicbox}
\begin{center}\rule{0.5\linewidth}{0.5pt}\end{center}

\subsection*{Question 1(b) [4 marks]}\label{q1b}

\textbf{Explain working of if-elif-else in Python.}

\begin{solutionbox}

Multi-way decision structure that checks multiple conditions in
sequence.

\textbf{Structure Table:}

{\def\LTcaptype{none} % do not increment counter
\begin{longtable}[]{@{}lll@{}}
\toprule\noalign{}
Statement & Purpose & Syntax \\
\midrule\noalign{}
\endhead
\bottomrule\noalign{}
\endlastfoot
if & First condition & \texttt{if\ condition1:} \\
elif & Alternative conditions & \texttt{elif\ condition2:} \\
else & Default case & \texttt{else:} \\
\end{longtable}
}

\textbf{Flow Diagram:}

\begin{verbatim}
    Start
      |
      v
 Check if condition
      |
   True/ {False}
     /    {}
    v      v
Execute   Check elif
if block   condition
    |         |
    v      True/ {False}
   End       /    {}
            v      v
       Execute   Check next elif
       elif      or else
       block        |
          |         v
          v      Execute
         End     else block
                    |
                    v
                   End
\end{verbatim}

\begin{itemize}
\tightlist
\item
  \textbf{Sequential}: Checks conditions top to bottom
\item
  \textbf{Exclusive}: Only one block executes
\item
  \textbf{Optional}: elif and else are optional
\end{itemize}

\end{solutionbox}
\begin{mnemonicbox}
``If This, Else If That, Else Default''

\end{mnemonicbox}
\begin{center}\rule{0.5\linewidth}{0.5pt}\end{center}

\subsection*{Question 1(c) [7 marks]}\label{q1c}

\textbf{Explain structure of a Python Program.}

\begin{solutionbox}

Python program has organized structure with specific components in
logical order.

\textbf{Program Structure Table:}

{\def\LTcaptype{none} % do not increment counter
\begin{longtable}[]{@{}lll@{}}
\toprule\noalign{}
Component & Purpose & Example \\
\midrule\noalign{}
\endhead
\bottomrule\noalign{}
\endlastfoot
Comments & Documentation & \texttt{\#\ This\ is\ comment} \\
Import & External modules & \texttt{import\ math} \\
Constants & Fixed values & \texttt{PI\ =\ 3.14} \\
Functions & Reusable code & \texttt{def\ function\_name():} \\
Classes & Objects blueprint & \texttt{class\ ClassName:} \\
Main code & Program execution &
\texttt{if\ \_\_name\_\_\ ==\ "\_\_main\_\_":} \\
\end{longtable}
}

\textbf{Program Architecture:}

\begin{verbatim}
    ┌─────────────────────┐
    │     Comments        │
    │   \# Documentation   │
    └─────────────────────┘
              │
              v
    ┌─────────────────────┐
    │    Import Section   │
    │   import modules    │
    └─────────────────────┘
              │
              v
    ┌─────────────────────┐
    │    Constants \&      │
    │     Variables       │
    └─────────────────────┘
              │
              v
    ┌─────────────────────┐
    │   Function          │
    │   Definitions       │
    └─────────────────────┘
              │
              v
    ┌─────────────────────┐
    │   Class             │
    │   Definitions       │
    └─────────────────────┘
              │
              v
    ┌─────────────────────┐
    │   Main Program      │
    │     Execution       │
    └─────────────────────┘
\end{verbatim}

\begin{itemize}
\tightlist
\item
  \textbf{Modular}: Each section has specific purpose
\item
  \textbf{Readable}: Clear organization helps understanding
\item
  \textbf{Maintainable}: Easy to modify and debug
\item
  \textbf{Standard}: Follows Python conventions
\end{itemize}

\textbf{Simple Example:}

\begin{verbatim}
\# Program to calculate area
import math

PI = 3.14159

def calculate\_area(radius):
    return PI * radius * radius

\# Main execution
radius = float(input("Enter radius: "))
area = calculate\_area(radius)
print(f"Area = \{area\}")
\end{verbatim}

\end{solutionbox}
\begin{mnemonicbox}
``Comment, Import, Constant, Function, Class, Main''

\end{mnemonicbox}
\begin{center}\rule{0.5\linewidth}{0.5pt}\end{center}

\subsection*{Question 1(c OR) [7
marks]}\label{question-1c-or-7-marks}

\textbf{Explain features of Python Programming Language.}

\begin{solutionbox}

Python has unique characteristics that make it popular for beginners and
professionals.

\textbf{Python Features Table:}

{\def\LTcaptype{none} % do not increment counter
\begin{longtable}[]{@{}lll@{}}
\toprule\noalign{}
Feature & Description & Benefit \\
\midrule\noalign{}
\endhead
\bottomrule\noalign{}
\endlastfoot
Simple & Easy syntax & Quick learning \\
Interpreted & No compilation & Fast development \\
Object-Oriented & Classes and objects & Code reusability \\
Open Source & Free to use & No licensing cost \\
Cross-Platform & Runs everywhere & High portability \\
\end{longtable}
}

\textbf{Feature Categories:}

\begin{verbatim}
         Python Features
              │
    ┌─────────┼─────────┐
    │         │         │
    v         v         v
Language   Technical  Community
Features   Features   Features
    │         │         │
    v         v         v
{- Simple   {-} Interpreted {-} Open Source}
{- Readable {-} Portable    {-} Large Library}
{- Dynamic  {-} Extensible  {-} Active Support}
\end{verbatim}

\begin{itemize}
\tightlist
\item
  \textbf{Beginner-Friendly}: Simple syntax like English language
\item
  \textbf{Versatile}: Used for web, AI, data science, automation
\item
  \textbf{Rich Libraries}: Huge collection of pre-built modules
\item
  \textbf{Dynamic Typing}: No need to declare variable types
\item
  \textbf{Interactive}: Can test code line by line in interpreter
\item
  \textbf{High-Level}: Handles memory management automatically
\end{itemize}

\textbf{Code Example:}

\begin{verbatim}
\# Simple Python syntax
name = "Python"
print(f"Hello, \{name\}!")
\end{verbatim}

\end{solutionbox}
\begin{mnemonicbox}
``Simple, Interpreted, Object-Oriented, Open,
Cross-platform''

\end{mnemonicbox}
\begin{center}\rule{0.5\linewidth}{0.5pt}\end{center}

\subsection*{Question 2(a) [3 marks]}\label{q2a}

\textbf{Explain any 3 operations done on Strings.}

\begin{solutionbox}

String operations manipulate and process text data in various ways.

\textbf{String Operations Table:}

{\def\LTcaptype{none} % do not increment counter
\begin{longtable}[]{@{}llll@{}}
\toprule\noalign{}
Operation & Method & Example & Result \\
\midrule\noalign{}
\endhead
\bottomrule\noalign{}
\endlastfoot
Concatenation & \texttt{+} & \texttt{"Hello"\ +\ "World"} &
\texttt{"HelloWorld"} \\
Length & \texttt{len()} & \texttt{len("Python")} & \texttt{6} \\
Uppercase & \texttt{.upper()} & \texttt{"hello".upper()} &
\texttt{"HELLO"} \\
\end{longtable}
}

\textbf{Operation Examples:}

\begin{verbatim}
text = "Python"
\# 1. Concatenation
result1 = text + " Programming"
\# 2. Find length
result2 = len(text)
\# 3. Convert to uppercase
result3 = text.upper()
\end{verbatim}

\begin{itemize}
\tightlist
\item
  \textbf{Concatenation}: Joins two or more strings together
\item
  \textbf{Length}: Counts total characters in string
\item
  \textbf{Case Conversion}: Changes letter cases (upper/lower)
\end{itemize}

\end{solutionbox}
\begin{mnemonicbox}
``Combine, Count, Convert''

\end{mnemonicbox}
\begin{center}\rule{0.5\linewidth}{0.5pt}\end{center}

\subsection*{Question 2(b) [4 marks]}\label{q2b}

\textbf{Develop a Python program to convert temperature from Fahrenheit
to Celsius unit using eq: C=(F-32)/1.8}

\begin{solutionbox}

Program converts temperature using mathematical formula with user input.

\textbf{Algorithm Table:}

{\def\LTcaptype{none} % do not increment counter
\begin{longtable}[]{@{}lll@{}}
\toprule\noalign{}
Step & Action & Code \\
\midrule\noalign{}
\endhead
\bottomrule\noalign{}
\endlastfoot
1 & Get input & \texttt{fahrenheit\ =\ float(input())} \\
2 & Apply formula & \texttt{celsius\ =\ (fahrenheit\ -\ 32)\ /\ 1.8} \\
3 & Display result & \texttt{print(f"Celsius:\ \{celsius\}")} \\
\end{longtable}
}

\textbf{Complete Program:}

\begin{verbatim}
\# Temperature conversion program
fahrenheit = float(input("Enter temperature in Fahrenheit: "))
celsius = (fahrenheit {-} 32) / 1.8
print(f"Temperature in Celsius: \{celsius:.2f\}")
\end{verbatim}

\textbf{Test Cases:}

\begin{itemize}
\item
  Input: 32^\circF \rightarrow Output: 0.00^\circC
\item
  Input: 100^\circF \rightarrow Output: 37.78^\circC
\item
  \textbf{User Input}: Gets Fahrenheit temperature from user
\item
  \textbf{Formula Application}: Uses given conversion equation
\item
  \textbf{Formatted Output}: Shows result with decimal places
\end{itemize}

\end{solutionbox}
\begin{mnemonicbox}
``Input, Calculate, Output''

\end{mnemonicbox}
\begin{center}\rule{0.5\linewidth}{0.5pt}\end{center}

\subsection*{Question 2(c) [7 marks]}\label{q2c}

\textbf{Explain in detail working of list data types in Python.}

\begin{solutionbox}

List is ordered, mutable collection that stores multiple items in single
variable.

\textbf{List Characteristics Table:}

{\def\LTcaptype{none} % do not increment counter
\begin{longtable}[]{@{}lll@{}}
\toprule\noalign{}
Property & Description & Example \\
\midrule\noalign{}
\endhead
\bottomrule\noalign{}
\endlastfoot
Ordered & Items have position & \texttt{[1,\ 2,\ 3]} \\
Mutable & Can be changed & \texttt{list[0]\ =\ 10} \\
Indexed & Access by position & \texttt{list[0]} \\
Mixed Types & Different data types &
\texttt{[1,\ "hello",\ 3.14]} \\
\end{longtable}
}

\textbf{List Operations Diagram:}

\begin{verbatim}
    List: [10, 20, 30, 40]
           |   |   |   |
    Index: 0   1   2   3
        
    Operations:
    ┌─────────────┐  ┌─────────────┐
    │   Access    │  │   Modify    │
    │  list[0]    │  │ list[0]=50  │
    └─────────────┘  └─────────────┘
           │                │
           v                v
         "10"          [50, 20, 30, 40]
\end{verbatim}

\textbf{Common List Methods:}

{\def\LTcaptype{none} % do not increment counter
\begin{longtable}[]{@{}lll@{}}
\toprule\noalign{}
Method & Purpose & Example \\
\midrule\noalign{}
\endhead
\bottomrule\noalign{}
\endlastfoot
append() & Add item at end & \texttt{list.append(5)} \\
insert() & Add at position & \texttt{list.insert(1,\ 15)} \\
remove() & Delete item & \texttt{list.remove(20)} \\
pop() & Remove last item & \texttt{list.pop()} \\
len() & Get length & \texttt{len(list)} \\
\end{longtable}
}

\textbf{Example Code:}

\begin{verbatim}
\# Creating and using lists
numbers = [1, 2, 3, 4, 5]
numbers.append(6)        \# Add 6 at end
numbers.insert(0, 0)     \# Add 0 at beginning
print(numbers[2])        \# Access 3rd element
numbers.remove(3)        \# Remove value 3
\end{verbatim}

\begin{itemize}
\tightlist
\item
  \textbf{Dynamic Size}: Can grow or shrink during execution
\item
  \textbf{Zero Indexing}: First element at index 0
\item
  \textbf{Slicing}: Can extract portions using [start:end]
\item
  \textbf{Nested Lists}: Can contain other lists
\end{itemize}

\end{solutionbox}
\begin{mnemonicbox}
``Ordered, Mutable, Indexed, Mixed''

\end{mnemonicbox}
\begin{center}\rule{0.5\linewidth}{0.5pt}\end{center}

\subsection*{Question 2(a OR) [3
marks]}\label{question-2a-or-3-marks}

\textbf{Explain String formatting in Python.}

\begin{solutionbox}

String formatting creates formatted strings by inserting values into
templates.

\textbf{Formatting Methods Table:}

{\def\LTcaptype{none} % do not increment counter
\begin{longtable}[]{@{}lll@{}}
\toprule\noalign{}
Method & Syntax & Example \\
\midrule\noalign{}
\endhead
\bottomrule\noalign{}
\endlastfoot
f-strings & \texttt{f"text\ \{variable\}"} &
\texttt{f"Hello\ \{name\}"} \\
format() & \texttt{"text\ \{\}".format(value)} &
\texttt{"Age:\ \{\}".format(25)} \\
\% operator & \texttt{"text\ \%s"\ \%\ value} &
\texttt{"Name:\ \%s"\ \%\ "John"} \\
\end{longtable}
}

\textbf{Example Usage:}

\begin{verbatim}
name = "Alice"
age = 25
\# f{-string formatting}
message = f"Hello \{name\}, you are \{age\} years old"
\end{verbatim}

\begin{itemize}
\tightlist
\item
  \textbf{Placeholder}: \{\} marks where values go
\item
  \textbf{Dynamic}: Values inserted at runtime
\item
  \textbf{Readable}: Makes code cleaner than concatenation
\end{itemize}

\end{solutionbox}
\begin{mnemonicbox}
``Format, Insert, Display''

\end{mnemonicbox}
\begin{center}\rule{0.5\linewidth}{0.5pt}\end{center}

\subsection*{Question 2(b OR) [4
marks]}\label{question-2b-or-4-marks}

\textbf{Develop a Python program to identify whether the scanned number
is even or odd and print an appropriate message.}

\begin{solutionbox}

Program checks if number is divisible by 2 to determine even or odd.

\textbf{Logic Table:}

{\def\LTcaptype{none} % do not increment counter
\begin{longtable}[]{@{}lll@{}}
\toprule\noalign{}
Condition & Result & Message \\
\midrule\noalign{}
\endhead
\bottomrule\noalign{}
\endlastfoot
number \% 2 == 0 & Even & ``Number is even'' \\
number \% 2 != 0 & Odd & ``Number is odd'' \\
\end{longtable}
}

\textbf{Complete Program:}

\begin{verbatim}
\# Even/Odd checker program
number = int(input("Enter a number: "))
if number \% 2 == 0:
    print(f"\{number\} is even")
else:
    print(f"\{number\} is odd")
\end{verbatim}

\textbf{Test Cases:}

\begin{itemize}
\item
  Input: 4 \rightarrow Output: ``4 is even''
\item
  Input: 7 \rightarrow Output: ``7 is odd''
\item
  \textbf{Modulo Operator}: \% gives remainder after division
\item
  \textbf{Conditional Logic}: if-else determines result
\item
  \textbf{User Feedback}: Clear message about result
\end{itemize}

\end{solutionbox}
\begin{mnemonicbox}
``Input, Check Remainder, Display Result''

\end{mnemonicbox}
\begin{center}\rule{0.5\linewidth}{0.5pt}\end{center}

\subsection*{Question 2(c OR) [7
marks]}\label{question-2c-or-7-marks}

\textbf{Explain in detail working of Set data types in Python.}

\begin{solutionbox}

Set is unordered collection of unique items with no duplicate values
allowed.

\textbf{Set Characteristics Table:}

{\def\LTcaptype{none} % do not increment counter
\begin{longtable}[]{@{}lll@{}}
\toprule\noalign{}
Property & Description & Example \\
\midrule\noalign{}
\endhead
\bottomrule\noalign{}
\endlastfoot
Unordered & No fixed position & \texttt{\{1,\ 3,\ 2\}} \\
Unique & No duplicates & \texttt{\{1,\ 2,\ 3\}} \\
Mutable & Can be modified & \texttt{set.add(4)} \\
Iterable & Can loop through & \texttt{for\ item\ in\ set:} \\
\end{longtable}
}

\textbf{Set Operations Diagram:}

\begin{verbatim}
    Set A: \{1, 2, 3\    Set B: \{3, 4, 5\}}
         {                    /}
          {                  /}
           v                v
    ┌─────────────────────────────┐
    │     Set Operations          │
    ├─────────────────────────────┤
    │ Union: \{1, 2, 3, 4, 5\      │}
    │ Intersection: \{3\           │}
    │ Difference: \{1, 2\          │}
    │ Symmetric Diff: \{1,2,4,5\   │}
    └─────────────────────────────┘
\end{verbatim}

\textbf{Set Methods Table:}

{\def\LTcaptype{none} % do not increment counter
\begin{longtable}[]{@{}lll@{}}
\toprule\noalign{}
Method & Purpose & Example \\
\midrule\noalign{}
\endhead
\bottomrule\noalign{}
\endlastfoot
add() & Add single item & \texttt{set.add(6)} \\
update() & Add multiple items & \texttt{set.update([7,\ 8])} \\
remove() & Delete item & \texttt{set.remove(3)} \\
union() & Combine sets & \texttt{set1.union(set2)} \\
intersection() & Common items & \texttt{set1.intersection(set2)} \\
\end{longtable}
}

\textbf{Example Code:}

\begin{verbatim}
\# Creating and using sets
fruits = \{"apple", "banana", "orange"\}
fruits.add("mango")              \# Add single item
fruits.update(["grape", "kiwi"]) \# Add multiple
fruits.remove("banana")          \# Remove item
print(len(fruits))               \# Count items
\end{verbatim}

\begin{itemize}
\tightlist
\item
  \textbf{Automatic Deduplication}: Removes duplicate values
  automatically
\item
  \textbf{Fast Membership}: Quick checking if item exists
\item
  \textbf{Mathematical Operations}: Union, intersection, difference
\item
  \textbf{No Indexing}: Cannot access items by position
\end{itemize}

\end{solutionbox}
\begin{mnemonicbox}
``Unique, Unordered, Mutable, Mathematical''

\end{mnemonicbox}
\begin{center}\rule{0.5\linewidth}{0.5pt}\end{center}

\subsection*{Question 3(a) [3 marks]}\label{q3a}

\textbf{Explain working of any 3 methods of math module.}

\begin{solutionbox}

Math module provides mathematical functions for complex calculations.

\textbf{Math Methods Table:}

{\def\LTcaptype{none} % do not increment counter
\begin{longtable}[]{@{}llll@{}}
\toprule\noalign{}
Method & Purpose & Example & Result \\
\midrule\noalign{}
\endhead
\bottomrule\noalign{}
\endlastfoot
math.sqrt() & Square root & \texttt{math.sqrt(16)} & \texttt{4.0} \\
math.pow() & Power calculation & \texttt{math.pow(2,\ 3)} &
\texttt{8.0} \\
math.ceil() & Round up & \texttt{math.ceil(4.3)} & \texttt{5} \\
\end{longtable}
}

\textbf{Usage Example:}

\begin{verbatim}
import math
number = 16
result1 = math.sqrt(number)  \# Square root
result2 = math.pow(2, 4)     \# 2 to power 4
result3 = math.ceil(7.2)     \# Round up to 8
\end{verbatim}

\begin{itemize}
\tightlist
\item
  \textbf{Precision}: More accurate than basic operators
\item
  \textbf{Import Required}: Must import math module first
\item
  \textbf{Return Values}: Usually return float numbers
\end{itemize}

\end{solutionbox}
\begin{mnemonicbox}
``Square root, Power, Ceiling''

\end{mnemonicbox}
\begin{center}\rule{0.5\linewidth}{0.5pt}\end{center}

\subsection*{Question 3(b) [4 marks]}\label{q3b}

\textbf{Develop a Python program to find sum of all elements in a list
using for loop.}

\begin{solutionbox}

Program iterates through list and accumulates sum of all elements.

\textbf{Algorithm Table:}

{\def\LTcaptype{none} % do not increment counter
\begin{longtable}[]{@{}lll@{}}
\toprule\noalign{}
Step & Action & Code \\
\midrule\noalign{}
\endhead
\bottomrule\noalign{}
\endlastfoot
1 & Initialize sum & \texttt{total\ =\ 0} \\
2 & Loop through list & \texttt{for\ element\ in\ list:} \\
3 & Add to sum & \texttt{total\ +=\ element} \\
4 & Display result & \texttt{print(total)} \\
\end{longtable}
}

\textbf{Complete Program:}

\begin{verbatim}
\# Sum of list elements
numbers = [10, 20, 30, 40, 50]
total = 0
for element in numbers:
    total += element
print(f"Sum of all elements: \{total\}")
\end{verbatim}

\textbf{Test Case:}

\begin{itemize}
\item
  Input: [1, 2, 3, 4, 5] \rightarrow Output: 15
\item
  \textbf{Accumulator}: Variable stores running total
\item
  \textbf{Iteration}: Loop visits each element once
\item
  \textbf{Addition}: Adds each element to running sum
\end{itemize}

\end{solutionbox}
\begin{mnemonicbox}
``Initialize, Loop, Add, Display''

\end{mnemonicbox}
\begin{center}\rule{0.5\linewidth}{0.5pt}\end{center}

\subsection*{Question 3(c) [7 marks]}\label{q3c}

\textbf{Develop a Python program to check if two lists are having
similar length. If yes then merge them and create a dictionary from
them.}

\begin{solutionbox}

Program compares list lengths and creates dictionary if they match.

\textbf{Logic Flow Table:}

{\def\LTcaptype{none} % do not increment counter
\begin{longtable}[]{@{}lll@{}}
\toprule\noalign{}
Step & Condition & Action \\
\midrule\noalign{}
\endhead
\bottomrule\noalign{}
\endlastfoot
1 & Check lengths & \texttt{len(list1)\ ==\ len(list2)} \\
2 & If equal & Merge and create dictionary \\
3 & If not equal & Display error message \\
\end{longtable}
}

\textbf{Process Diagram:}

\begin{verbatim}
    List1: [a, b, c]     List2: [1, 2, 3]
       |                     |
       v                     v
     len(List1) == len(List2) ?
             |
        Yes / { No}
           /   {}
          v     v
    Create Dict  Error
    \{a:1, b:2,   Message
     c:3\}
\end{verbatim}

\textbf{Complete Program:}

\begin{verbatim}
\# Merge lists into dictionary
list1 = [{name}, {age}, {city}]
list2 = [{John}, 25, {Mumbai}]

if len(list1) == len(list2):
    \# Create dictionary using zip
    result\_dict = dict(zip(list1, list2))
    print("Dictionary created:", result\_dict)
else:
    print("Lists have different lengths, cannot merge")
\end{verbatim}

\textbf{Expected Output:}

\begin{verbatim}
Dictionary created: {'name': 'John', 'age': 25, 'city': 'Mumbai'}
\end{verbatim}

\begin{itemize}
\tightlist
\item
  \textbf{Length Comparison}: Ensures lists can be paired properly
\item
  \textbf{zip() Function}: Pairs elements from both lists
\item
  \textbf{dict() Constructor}: Creates dictionary from paired elements
\item
  \textbf{Error Handling}: Prevents incorrect pairing
\end{itemize}

\textbf{Alternative Method:}

\begin{verbatim}
\# Manual dictionary creation
result\_dict = \{\}
for i in range(len(list1)):
    result\_dict[list1[i]] = list2[i]
\end{verbatim}

\end{solutionbox}
\begin{mnemonicbox}
``Check Length, Zip, Create Dictionary''

\end{mnemonicbox}
\begin{center}\rule{0.5\linewidth}{0.5pt}\end{center}

\subsection*{Question 3(a OR) [3
marks]}\label{question-3a-or-3-marks}

\textbf{Explain working of any 3 methods of statistics module.}

\begin{solutionbox}

Statistics module provides functions for statistical calculations on
numeric data.

\textbf{Statistics Methods Table:}

{\def\LTcaptype{none} % do not increment counter
\begin{longtable}[]{@{}llll@{}}
\toprule\noalign{}
Method & Purpose & Example & Result \\
\midrule\noalign{}
\endhead
\bottomrule\noalign{}
\endlastfoot
statistics.mean() & Average value & \texttt{mean([1,2,3,4,5])} &
\texttt{3.0} \\
statistics.median() & Middle value & \texttt{median([1,2,3,4,5])} &
\texttt{3} \\
statistics.mode() & Most frequent & \texttt{mode([1,1,2,3])} &
\texttt{1} \\
\end{longtable}
}

\textbf{Usage Example:}

\begin{verbatim}
import statistics
data = [10, 20, 30, 40, 50]
avg = statistics.mean(data)      \# Calculate average
mid = statistics.median(data)    \# Find middle value
\end{verbatim}

\begin{itemize}
\tightlist
\item
  \textbf{Data Analysis}: Helps understand data patterns
\item
  \textbf{Built-in Functions}: No need to write complex formulas
\item
  \textbf{Accurate Results}: Handles edge cases properly
\end{itemize}

\end{solutionbox}
\begin{mnemonicbox}
``Mean, Median, Mode''

\end{mnemonicbox}
\begin{center}\rule{0.5\linewidth}{0.5pt}\end{center}

\subsection*{Question 3(c OR) [7
marks]}\label{question-3c-or-7-marks}

\textbf{Develop a Python program to count the number of times a
character appears in a given string using a dictionary.}

\begin{solutionbox}

Program creates dictionary where keys are characters and values are
their counts.

\textbf{Character Counting Algorithm:}

{\def\LTcaptype{none} % do not increment counter
\begin{longtable}[]{@{}
  >{\raggedright\arraybackslash}p{(\linewidth - 4\tabcolsep) * \real{0.3000}}
  >{\raggedright\arraybackslash}p{(\linewidth - 4\tabcolsep) * \real{0.4000}}
  >{\raggedright\arraybackslash}p{(\linewidth - 4\tabcolsep) * \real{0.3000}}@{}}
\toprule\noalign{}
\begin{minipage}[b]{\linewidth}\raggedright
Step
\end{minipage} & \begin{minipage}[b]{\linewidth}\raggedright
Action
\end{minipage} & \begin{minipage}[b]{\linewidth}\raggedright
Code
\end{minipage} \\
\midrule\noalign{}
\endhead
\bottomrule\noalign{}
\endlastfoot
1 & Initialize dictionary & \texttt{char\_count\ =\ \{\}} \\
2 & Loop through string & \texttt{for\ char\ in\ string:} \\
3 & Count occurrences &
\texttt{char\_count[char]\ =\ char\_count.get(char,\ 0)\ +\ 1} \\
4 & Display results & \texttt{print(char\_count)} \\
\end{longtable}
}

\textbf{Counting Process:}

\begin{verbatim}
    String: "hello"
         |
         v
    Loop through each character
         |
    ┌────┴────┬────┬────┬────┬────┐
    │    h    │ e  │ l  │ l  │ o  │
    └────┬────┴────┴────┴────┴────┘
         v
    Dictionary: \{{h:1, e:1, l:2, o:1\}}
\end{verbatim}

\textbf{Complete Program:}

\begin{verbatim}
\# Character frequency counter
text = input("Enter a string: ")
char\_count = \{\}

for char in text:
    if char in char\_count:
        char\_count[char] += 1
    else:
        char\_count[char] = 1

print("Character frequencies:")
for char, count in char\_count.items():
    print(f"{}\{char\}{: }\{count\}")
\end{verbatim}

\textbf{Alternative Method (More Pythonic):}

\begin{verbatim}
\# Using get() method
text = "programming"
char\_count = \{\}

for char in text:
    char\_count[char] = char\_count.get(char, 0) + 1

print(char\_count)
\end{verbatim}

\textbf{Example Output:}

\begin{verbatim}
Input: "hello"
Output: {'h': 1, 'e': 1, 'l': 2, 'o': 1}
\end{verbatim}

\begin{itemize}
\tightlist
\item
  \textbf{Dictionary Keys}: Each unique character becomes a key
\item
  \textbf{Dictionary Values}: Count of character occurrences
\item
  \textbf{get() Method}: Returns 0 if key doesn't exist, avoiding errors
\item
  \textbf{Iteration}: Processes each character in string once
\end{itemize}

\end{solutionbox}
\begin{mnemonicbox}
``Loop, Check, Count, Store''

\end{mnemonicbox}
\begin{center}\rule{0.5\linewidth}{0.5pt}\end{center}

\subsection*{Question 4(a) [3 marks]}\label{q4a}

\textbf{Explain working of Python class and objects with example.}

\begin{solutionbox}

Class is blueprint for creating objects. Objects are instances of
classes.

\textbf{Class-Object Relationship:}

{\def\LTcaptype{none} % do not increment counter
\begin{longtable}[]{@{}lll@{}}
\toprule\noalign{}
Concept & Purpose & Example \\
\midrule\noalign{}
\endhead
\bottomrule\noalign{}
\endlastfoot
Class & Template/Blueprint & \texttt{class\ Car:} \\
Object & Instance of class & \texttt{my\_car\ =\ Car()} \\
Attributes & Data in class & \texttt{self.color\ =\ "red"} \\
Methods & Functions in class & \texttt{def\ start(self):} \\
\end{longtable}
}

\textbf{Class Structure:}

\begin{verbatim}
         Class: Car
    ┌─────────────────────┐
    │   Attributes:       │
    │   {- color           │}
    │   {- model           │}
    │                     │
    │   Methods:          │
    │   {- start()         │}
    │   {- stop()          │}
    └─────────────────────┘
              │
              v
    Object: my\_car = Car()
\end{verbatim}

\textbf{Example Code:}

\begin{verbatim}
class Student:
    def \_\_init\_\_(self, name, age):
        self.name = name  \# Attribute
        self.age = age    \# Attribute
    
    def display(self):    \# Method
        print(f"Name: \{self.name\}, Age: \{self.age\}")

\# Creating objects
student1 = Student("Alice", 20)
student1.display()
\end{verbatim}

\begin{itemize}
\tightlist
\item
  \textbf{Encapsulation}: Groups related data and functions together
\item
  \textbf{Reusability}: One class can create multiple objects
\item
  \textbf{Organization}: Better code structure and maintenance
\end{itemize}

\end{solutionbox}
\begin{mnemonicbox}
``Class Blueprint, Object Instance''

\end{mnemonicbox}
\begin{center}\rule{0.5\linewidth}{0.5pt}\end{center}

\subsection*{Question 4(b) [4 marks]}\label{q4b}

\textbf{Develop a Python program to print all odd numbers in a list.}

\begin{solutionbox}

Program filters list elements and displays only odd numbers.

\textbf{Odd Number Check Table:}

{\def\LTcaptype{none} % do not increment counter
\begin{longtable}[]{@{}lll@{}}
\toprule\noalign{}
Number & number \% 2 & Result \\
\midrule\noalign{}
\endhead
\bottomrule\noalign{}
\endlastfoot
1 & 1 & Odd \\
2 & 0 & Even \\
3 & 1 & Odd \\
4 & 0 & Even \\
\end{longtable}
}

\textbf{Complete Program:}

\begin{verbatim}
\# Print odd numbers from list
numbers = [1, 2, 3, 4, 5, 6, 7, 8, 9, 10]

print("Odd numbers in the list:")
for number in numbers:
    if number \% 2 != 0:
        print(number, end=" ")
\end{verbatim}

\textbf{Alternative Methods:}

\begin{verbatim}
\# Method 2: List comprehension
odd\_numbers = [num for num in numbers if num \% 2 != 0]
print(odd\_numbers)

\# Method 3: Using filter
odd\_numbers = list(filter(lambda x: x \% 2 != 0, numbers))
print(odd\_numbers)
\end{verbatim}

\textbf{Expected Output:}

\begin{verbatim}
Odd numbers in the list:
1 3 5 7 9
\end{verbatim}

\begin{itemize}
\tightlist
\item
  \textbf{Modulo Operation}: \% operator finds remainder
\item
  \textbf{Condition Check}: If remainder is not 0, number is odd
\item
  \textbf{Loop Iteration}: Checks each number in list
\end{itemize}

\end{solutionbox}
\begin{mnemonicbox}
``Loop, Check Remainder, Print Odd''

\end{mnemonicbox}
\begin{center}\rule{0.5\linewidth}{0.5pt}\end{center}

\subsection*{Question 4(c) [7 marks]}\label{q4c}

\textbf{Explain working of user defined functions in Python.}

\begin{solutionbox}

User-defined functions are custom functions created by programmers to
perform specific tasks.

\textbf{Function Components Table:}

{\def\LTcaptype{none} % do not increment counter
\begin{longtable}[]{@{}lll@{}}
\toprule\noalign{}
Component & Purpose & Syntax \\
\midrule\noalign{}
\endhead
\bottomrule\noalign{}
\endlastfoot
def keyword & Function declaration & \texttt{def\ function\_name():} \\
Parameters & Input values & \texttt{def\ func(param1,\ param2):} \\
Body & Function code & Indented statements \\
return & Output value & \texttt{return\ value} \\
\end{longtable}
}

\textbf{Function Structure:}

\begin{verbatim}
    def function\_name(parameters):
         │        │        │
         │        │        └─ Input values
         │        └─ Function identifier
         └─ Keyword to define function
         
    Function Body (indented)
         │
         v
    ┌─────────────────────┐
    │  Local variables    │
    │  Processing logic   │
    │  Calculations       │
    └─────────────────────┘
         │
         v
    return result (optional)
\end{verbatim}

\textbf{Types of Functions:}

{\def\LTcaptype{none} % do not increment counter
\begin{longtable}[]{@{}lll@{}}
\toprule\noalign{}
Type & Description & Example \\
\midrule\noalign{}
\endhead
\bottomrule\noalign{}
\endlastfoot
No parameters & Takes no input & \texttt{def\ greet():} \\
With parameters & Takes input & \texttt{def\ add(a,\ b):} \\
Return value & Gives output & \texttt{return\ a\ +\ b} \\
No return & Performs action & \texttt{print("Hello")} \\
\end{longtable}
}

\textbf{Example Functions:}

\begin{verbatim}
\# Function with no parameters
def greet():
    print("Hello, World!")

\# Function with parameters and return value
def calculate\_area(length, width):
    area = length * width
    return area

\# Function with default parameters
def introduce(name, age=18):
    print(f"My name is \{name\} and I am \{age\} years old")

\# Using functions
greet()
result = calculate\_area(5, 3)
print(f"Area: \{result\}")
introduce("Alice", 25)
introduce("Bob")  \# Uses default age
\end{verbatim}

\textbf{Function Benefits:}

\begin{itemize}
\tightlist
\item
  \textbf{Reusability}: Write once, use multiple times
\item
  \textbf{Modularity}: Break complex problems into smaller parts
\item
  \textbf{Maintainability}: Easy to update and debug
\item
  \textbf{Readability}: Makes code more organized and understandable
\item
  \textbf{Testing}: Can test individual functions separately
\end{itemize}

\textbf{Variable Scope:}

\begin{itemize}
\tightlist
\item
  \textbf{Local Variables}: Exist only inside function
\item
  \textbf{Global Variables}: Accessible throughout program
\item
  \textbf{Parameters}: Act as local variables
\end{itemize}

\end{solutionbox}
\begin{mnemonicbox}
``Define, Parameters, Body, Return''

\end{mnemonicbox}
\begin{center}\rule{0.5\linewidth}{0.5pt}\end{center}

\subsection*{Question 4(a OR) [3
marks]}\label{question-4a-or-3-marks}

\textbf{Explain working constructors in Python.}

\begin{solutionbox}

Constructor is special method that initializes objects when they are
created.

\textbf{Constructor Details Table:}

{\def\LTcaptype{none} % do not increment counter
\begin{longtable}[]{@{}lll@{}}
\toprule\noalign{}
Aspect & Description & Syntax \\
\midrule\noalign{}
\endhead
\bottomrule\noalign{}
\endlastfoot
Method name & Always \texttt{\_\_init\_\_} &
\texttt{def\ \_\_init\_\_(self):} \\
Purpose & Initialize object & Set initial values \\
Automatic call & Called during object creation &
\texttt{obj\ =\ Class()} \\
Parameters & Can accept arguments &
\texttt{def\ \_\_init\_\_(self,\ param):} \\
\end{longtable}
}

\textbf{Constructor Example:}

\begin{verbatim}
class Student:
    def \_\_init\_\_(self, name, age):
        self.name = name
        self.age = age
        print("Student object created")

\# Object creation automatically calls constructor
student1 = Student("Alice", 20)
\end{verbatim}

\begin{itemize}
\tightlist
\item
  \textbf{Automatic Execution}: Runs immediately when object is created
\item
  \textbf{Initialization}: Sets up object's initial state
\item
  \textbf{self Parameter}: Refers to current object being created
\end{itemize}

\end{solutionbox}
\begin{mnemonicbox}
``Initialize, Automatic, Self''

\end{mnemonicbox}
\begin{center}\rule{0.5\linewidth}{0.5pt}\end{center}

\subsection*{Question 4(b OR) [4
marks]}\label{question-4b-or-4-marks}

\textbf{Develop a Python program to find smallest number in a list
without using min function.}

\begin{solutionbox}

Program manually compares all elements to find the smallest value.

\textbf{Finding Minimum Algorithm:}

{\def\LTcaptype{none} % do not increment counter
\begin{longtable}[]{@{}lll@{}}
\toprule\noalign{}
Step & Action & Code \\
\midrule\noalign{}
\endhead
\bottomrule\noalign{}
\endlastfoot
1 & Assume first is smallest & \texttt{smallest\ =\ list[0]} \\
2 & Compare with others & \texttt{for\ num\ in\ list[1:]:} \\
3 & Update if smaller found &
\texttt{if\ num\ \textless{}\ smallest:} \\
4 & Display result & \texttt{print(smallest)} \\
\end{longtable}
}

\textbf{Complete Program:}

\begin{verbatim}
\# Find smallest number without min()
numbers = [45, 23, 67, 12, 89, 5, 34]

smallest = numbers[0]  \# Assume first is smallest

for i in range(1, len(numbers)):
    if numbers[i] {} smallest:
        smallest = numbers[i]

print(f"Smallest number: \{smallest\}")
\end{verbatim}

\textbf{Alternative Method:}

\begin{verbatim}
\# Using for loop with list elements
numbers = [45, 23, 67, 12, 89, 5, 34]
smallest = numbers[0]

for num in numbers[1:]:
    if num {} smallest:
        smallest = num

print(f"Smallest number: \{smallest\}")
\end{verbatim}

\textbf{Expected Output:}

\begin{verbatim}
Smallest number: 5
\end{verbatim}

\begin{itemize}
\tightlist
\item
  \textbf{Comparison Logic}: Compare each element with current smallest
\item
  \textbf{Update Strategy}: Replace smallest when smaller number found
\item
  \textbf{Linear Search}: Check all elements once
\end{itemize}

\end{solutionbox}
\begin{mnemonicbox}
``Assume, Compare, Update, Display''

\end{mnemonicbox}
\begin{center}\rule{0.5\linewidth}{0.5pt}\end{center}

\subsection*{Question 4(c OR) [7
marks]}\label{question-4c-or-7-marks}

\textbf{Explain working of user defined Modules in Python.}

\begin{solutionbox}

User-defined modules are custom Python files containing functions,
classes, and variables that can be imported and used in other programs.

\textbf{Module Components Table:}

{\def\LTcaptype{none} % do not increment counter
\begin{longtable}[]{@{}lll@{}}
\toprule\noalign{}
Component & Purpose & Example \\
\midrule\noalign{}
\endhead
\bottomrule\noalign{}
\endlastfoot
Functions & Reusable code blocks & \texttt{def\ calculate\_area():} \\
Classes & Object blueprints & \texttt{class\ Shape:} \\
Variables & Shared data & \texttt{PI\ =\ 3.14159} \\
Constants & Fixed values & \texttt{MAX\_SIZE\ =\ 100} \\
\end{longtable}
}

\textbf{Module Creation Process:}

\begin{verbatim}
    Step 1: Create .py file
         |
         v
    Step 2: Write functions/classes
         |
         v
    Step 3: Save file
         |
         v
    Step 4: Import in other programs
         |
         v
    Step 5: Use module functions
\end{verbatim}

\textbf{Example Module Creation:}

\textbf{File: math\_operations.py}

\begin{verbatim}
\# User{-defined module}
PI = 3.14159

def calculate\_circle\_area(radius):
    return PI * radius * radius

def calculate\_rectangle\_area(length, width):
    return length * width

class Calculator:
    def add(self, a, b):
        return a + b
    
    def subtract(self, a, b):
        return a {-} b
\end{verbatim}

\textbf{Using the Module:}

\textbf{Import Methods Table:}

{\def\LTcaptype{none} % do not increment counter
\begin{longtable}[]{@{}
  >{\raggedright\arraybackslash}p{(\linewidth - 4\tabcolsep) * \real{0.3478}}
  >{\raggedright\arraybackslash}p{(\linewidth - 4\tabcolsep) * \real{0.3478}}
  >{\raggedright\arraybackslash}p{(\linewidth - 4\tabcolsep) * \real{0.3043}}@{}}
\toprule\noalign{}
\begin{minipage}[b]{\linewidth}\raggedright
Method
\end{minipage} & \begin{minipage}[b]{\linewidth}\raggedright
Syntax
\end{minipage} & \begin{minipage}[b]{\linewidth}\raggedright
Usage
\end{minipage} \\
\midrule\noalign{}
\endhead
\bottomrule\noalign{}
\endlastfoot
Import entire module & \texttt{import\ math\_operations} &
\texttt{math\_operations.calculate\_circle\_area(5)} \\
Import specific function &
\texttt{from\ math\_operations\ import\ calculate\_circle\_area} &
\texttt{calculate\_circle\_area(5)} \\
Import with alias & \texttt{import\ math\_operations\ as\ math\_ops} &
\texttt{math\_ops.PI} \\
Import all & \texttt{from\ math\_operations\ import\ *} &
\texttt{calculate\_circle\_area(5)} \\
\end{longtable}
}

\textbf{Main Program:}

\begin{verbatim}
\# main.py {- Using the module}
import math\_operations

\# Using module functions
radius = 5
area = math\_operations.calculate\_circle\_area(radius)
print(f"Circle area: \{area\}")

\# Using module variables
print(f"PI value: \{math\_operations.PI\}")

\# Using module classes
calc = math\_operations.Calculator()
result = calc.add(10, 20)
print(f"Addition result: \{result\}")
\end{verbatim}

\textbf{Module Benefits:}

\begin{itemize}
\tightlist
\item
  \textbf{Code Reusability}: Write once, use in multiple programs
\item
  \textbf{Organization}: Keep related functions together
\item
  \textbf{Namespace}: Avoid naming conflicts
\item
  \textbf{Maintainability}: Easy to update and debug
\item
  \textbf{Collaboration}: Share modules with other developers
\end{itemize}

\textbf{Module Search Path:}

\begin{enumerate}
\tightlist
\item
  Current directory
\item
  PYTHONPATH environment variable
\item
  Standard library directories
\item
  Site-packages directory
\end{enumerate}

\textbf{Best Practices:}

\begin{itemize}
\tightlist
\item
  Use descriptive module names
\item
  Include docstrings for documentation
\item
  Keep related functionality together
\item
  Avoid circular imports
\end{itemize}

\end{solutionbox}
\begin{mnemonicbox}
``Create File, Define Functions, Import, Use''

\end{mnemonicbox}
\begin{center}\rule{0.5\linewidth}{0.5pt}\end{center}

\subsection*{Question 5(a) [3 marks]}\label{q5a}

\textbf{Explain single inheritance in Python with example.}

\begin{solutionbox}

Single inheritance is when one class inherits properties and methods
from exactly one parent class.

\textbf{Inheritance Structure Table:}

{\def\LTcaptype{none} % do not increment counter
\begin{longtable}[]{@{}lll@{}}
\toprule\noalign{}
Component & Role & Example \\
\midrule\noalign{}
\endhead
\bottomrule\noalign{}
\endlastfoot
Parent Class & Base/Super class & \texttt{class\ Animal:} \\
Child Class & Derived/Sub class & \texttt{class\ Dog(Animal):} \\
Inheritance & \texttt{class\ Child(Parent):} &
\texttt{class\ Dog(Animal):} \\
\end{longtable}
}

\textbf{Inheritance Diagram:}

\begin{verbatim}
    Parent Class: Animal
    ┌─────────────────────┐
    │   Attributes:       │
    │   {- name            │}
    │   {- age             │}
    │                     │
    │   Methods:          │
    │   {- eat()           │}
    │   {- sleep()         │}
    └─────────────────────┘
              │
              │ inherits
              v
    Child Class: Dog
    ┌─────────────────────┐
    │ Inherited:          │
    │   {- name, age       │}
    │   {- eat(), sleep()  │}
    │                     │
    │ Own Methods:        │
    │   {- bark()          │}
    └─────────────────────┘
\end{verbatim}

\textbf{Example Code:}

\begin{verbatim}
\# Parent class
class Animal:
    def \_\_init\_\_(self, name):
        self.name = name
    
    def eat(self):
        print(f"\{self.name\} is eating")

\# Child class inheriting from Animal
class Dog(Animal):
    def bark(self):
        print(f"\{self.name\} is barking")

\# Using inheritance
my\_dog = Dog("Buddy")
my\_dog.eat()    \# Inherited method
my\_dog.bark()   \# Own method
\end{verbatim}

\begin{itemize}
\tightlist
\item
  \textbf{Code Reuse}: Child class gets parent's functionality
  automatically
\item
  \textbf{Extension}: Child can add new methods and attributes
\item
  \textbf{Is-a Relationship}: Dog is-a Animal
\end{itemize}

\end{solutionbox}
\begin{mnemonicbox}
``One Parent, One Child''

\end{mnemonicbox}
\begin{center}\rule{0.5\linewidth}{0.5pt}\end{center}

\subsection*{Question 5(b) [4 marks]}\label{q5b}

\textbf{Explain concept of abstraction in Python with its advantages.}

\begin{solutionbox}

Abstraction hides complex implementation details and shows only
essential features to the user.

\textbf{Abstraction Concepts Table:}

{\def\LTcaptype{none} % do not increment counter
\begin{longtable}[]{@{}lll@{}}
\toprule\noalign{}
Concept & Description & Example \\
\midrule\noalign{}
\endhead
\bottomrule\noalign{}
\endlastfoot
Abstract Class & Cannot be instantiated & \texttt{class\ Shape(ABC):} \\
Abstract Method & Must be implemented & \texttt{@abstractmethod} \\
Interface & Defines method structure & \texttt{def\ area(self):} \\
\end{longtable}
}

\textbf{Abstraction Implementation:}

\begin{verbatim}
from abc import ABC, abstractmethod

\# Abstract class
class Shape(ABC):
    @abstractmethod
    def area(self):
        pass
    
    @abstractmethod
    def perimeter(self):
        pass

\# Concrete class
class Rectangle(Shape):
    def \_\_init\_\_(self, length, width):
        self.length = length
        self.width = width
    
    def area(self):
        return self.length * self.width
    
    def perimeter(self):
        return 2 * (self.length + self.width)
\end{verbatim}

\textbf{Advantages Table:}

{\def\LTcaptype{none} % do not increment counter
\begin{longtable}[]{@{}
  >{\raggedright\arraybackslash}p{(\linewidth - 4\tabcolsep) * \real{0.3333}}
  >{\raggedright\arraybackslash}p{(\linewidth - 4\tabcolsep) * \real{0.3939}}
  >{\raggedright\arraybackslash}p{(\linewidth - 4\tabcolsep) * \real{0.2727}}@{}}
\toprule\noalign{}
\begin{minipage}[b]{\linewidth}\raggedright
Advantage
\end{minipage} & \begin{minipage}[b]{\linewidth}\raggedright
Description
\end{minipage} & \begin{minipage}[b]{\linewidth}\raggedright
Benefit
\end{minipage} \\
\midrule\noalign{}
\endhead
\bottomrule\noalign{}
\endlastfoot
Simplicity & Hide complex details & Easier to use \\
Security & Hide internal implementation & Data protection \\
Maintainability & Change implementation without affecting users &
Flexible updates \\
Code Organization & Clear structure & Better design \\
\end{longtable}
}

\begin{itemize}
\tightlist
\item
  \textbf{Hide Complexity}: Users don't need to know internal workings
\item
  \textbf{Consistent Interface}: All child classes follow same structure
\item
  \textbf{Force Implementation}: Abstract methods must be defined in
  child classes
\end{itemize}

\end{solutionbox}
\begin{mnemonicbox}
``Hide Details, Show Interface''

\end{mnemonicbox}
\begin{center}\rule{0.5\linewidth}{0.5pt}\end{center}

\subsection*{Question 5(c) [7 marks]}\label{q5c}

\textbf{Develop a Python program to demonstrate working of multiple and
multi-level inheritances.}

\begin{solutionbox}

Program shows both inheritance types: multiple (multiple parents) and
multi-level (chain of inheritance).

\textbf{Inheritance Types Comparison:}

{\def\LTcaptype{none} % do not increment counter
\begin{longtable}[]{@{}
  >{\raggedright\arraybackslash}p{(\linewidth - 4\tabcolsep) * \real{0.2308}}
  >{\raggedright\arraybackslash}p{(\linewidth - 4\tabcolsep) * \real{0.4231}}
  >{\raggedright\arraybackslash}p{(\linewidth - 4\tabcolsep) * \real{0.3462}}@{}}
\toprule\noalign{}
\begin{minipage}[b]{\linewidth}\raggedright
Type
\end{minipage} & \begin{minipage}[b]{\linewidth}\raggedright
Structure
\end{minipage} & \begin{minipage}[b]{\linewidth}\raggedright
Example
\end{minipage} \\
\midrule\noalign{}
\endhead
\bottomrule\noalign{}
\endlastfoot
Multiple & Child inherits from 2+ parents & \texttt{class\ C(A,\ B):} \\
Multi-level & Grandparent \rightarrow Parent \rightarrow Child & \texttt{class\ C(B):} where
\texttt{class\ B(A):} \\
\end{longtable}
}

\textbf{Inheritance Hierarchy:}

\begin{verbatim}
Multiple Inheritance:
    Father    Mother
      {       /}
       {     /}
        Child

Multi{-level Inheritance:}
    Animal
      |
      v
    Mammal
      |
      v
     Dog
\end{verbatim}

\textbf{Complete Program:}

\begin{verbatim}
\# Multi{-level Inheritance Demo}
print("=== Multi{-level Inheritance ==="})

class Animal:
    def \_\_init\_\_(self, name):
        self.name = name
    
    def eat(self):
        print(f"\{self.name\} can eat")

class Mammal(Animal):  \# Inherits from Animal
    def breathe(self):
        print(f"\{self.name\} breathes air")

class Dog(Mammal):     \# Inherits from Mammal (which inherits from Animal)
    def bark(self):
        print(f"\{self.name\} can bark")

\# Using multi{-level inheritance}
my\_dog = Dog("Buddy")
my\_dog.eat()     \# From Animal (grandparent)
my\_dog.breathe() \# From Mammal (parent)
my\_dog.bark()    \# Own method

print("{n}=== Multiple Inheritance ===")

class Father:
    def father\_method(self):
        print("Method from Father class")

class Mother:
    def mother\_method(self):
        print("Method from Mother class")

class Child(Father, Mother):  \# Inherits from both Father and Mother
    def child\_method(self):
        print("Method from Child class")

\# Using multiple inheritance
child = Child()
child.father\_method()  \# From Father
child.mother\_method()  \# From Mother
child.child\_method()   \# Own method

\# Checking inheritance
print(f"{n}Child inherits from Father: \{issubclass(Child, Father)\}")
print(f"Child inherits from Mother: \{issubclass(Child, Mother)\}")
\end{verbatim}

\textbf{Expected Output:}

\begin{verbatim}
=== Multi-level Inheritance ===
Buddy can eat
Buddy breathes air
Buddy can bark

=== Multiple Inheritance ===
Method from Father class
Method from Mother class
Method from Child class

Child inherits from Father: True
Child inherits from Mother: True
\end{verbatim}

\textbf{Key Differences:}

{\def\LTcaptype{none} % do not increment counter
\begin{longtable}[]{@{}lll@{}}
\toprule\noalign{}
Aspect & Multiple & Multi-level \\
\midrule\noalign{}
\endhead
\bottomrule\noalign{}
\endlastfoot
Parents & 2 or more direct parents & Single parent chain \\
Syntax & \texttt{class\ C(A,\ B):} & \texttt{class\ C(B):} where
\texttt{B(A):} \\
Inheritance & Horizontal & Vertical \\
Complexity & Higher (diamond problem) & Lower \\
\end{longtable}
}

\textbf{Method Resolution Order (MRO):}

\begin{itemize}
\tightlist
\item
  \textbf{Multiple}: Python follows left-to-right order
\item
  \textbf{Multi-level}: Goes up the inheritance chain
\end{itemize}

\end{solutionbox}
\begin{mnemonicbox}
``Multiple Parents, Multi-level Chain''

\end{mnemonicbox}
\begin{center}\rule{0.5\linewidth}{0.5pt}\end{center}

\subsection*{Question 5(a OR) [3
marks]}\label{question-5a-or-3-marks}

\textbf{Explain working of 3 types of methods in Python.}

\begin{solutionbox}

Python classes have three types of methods based on how they access
class data.

\textbf{Method Types Table:}

{\def\LTcaptype{none} % do not increment counter
\begin{longtable}[]{@{}llll@{}}
\toprule\noalign{}
Method Type & Decorator & First Parameter & Purpose \\
\midrule\noalign{}
\endhead
\bottomrule\noalign{}
\endlastfoot
Instance Method & None & \texttt{self} & Access instance data \\
Class Method & \texttt{@classmethod} & \texttt{cls} & Access class
data \\
Static Method & \texttt{@staticmethod} & None & Utility functions \\
\end{longtable}
}

\textbf{Example Code:}

\begin{verbatim}
class Student:
    school\_name = "ABC School"  \# Class variable
    
    def \_\_init\_\_(self, name):
        self.name = name        \# Instance variable
    
    \# Instance method
    def display\_info(self):
        print(f"Student: \{self.name\}")
    
    \# Class method
    @classmethod
    def get\_school(cls):
        return cls.school\_name
    
    \# Static method
    @staticmethod
    def is\_adult(age):
        return age {=} 18

\# Usage
student = Student("Alice")
student.display\_info()           \# Instance method
print(Student.get\_school())      \# Class method
print(Student.is\_adult(20))      \# Static method
\end{verbatim}

\begin{itemize}
\tightlist
\item
  \textbf{Instance Methods}: Work with object-specific data using
  \texttt{self}
\item
  \textbf{Class Methods}: Work with class-wide data using \texttt{cls}
\item
  \textbf{Static Methods}: Independent utility functions
\end{itemize}

\end{solutionbox}
\begin{mnemonicbox}
``Instance Self, Class Cls, Static None''

\end{mnemonicbox}
\begin{center}\rule{0.5\linewidth}{0.5pt}\end{center}

\subsection*{Question 5(b OR) [4
marks]}\label{question-5b-or-4-marks}

\textbf{Explain polymorphism through inheritance in Python.}

\begin{solutionbox}

Polymorphism allows objects of different classes to be treated as
objects of common base class, with each implementing methods
differently.

\textbf{Polymorphism Concept Table:}

{\def\LTcaptype{none} % do not increment counter
\begin{longtable}[]{@{}
  >{\raggedright\arraybackslash}p{(\linewidth - 4\tabcolsep) * \real{0.2667}}
  >{\raggedright\arraybackslash}p{(\linewidth - 4\tabcolsep) * \real{0.4333}}
  >{\raggedright\arraybackslash}p{(\linewidth - 4\tabcolsep) * \real{0.3000}}@{}}
\toprule\noalign{}
\begin{minipage}[b]{\linewidth}\raggedright
Aspect
\end{minipage} & \begin{minipage}[b]{\linewidth}\raggedright
Description
\end{minipage} & \begin{minipage}[b]{\linewidth}\raggedright
Example
\end{minipage} \\
\midrule\noalign{}
\endhead
\bottomrule\noalign{}
\endlastfoot
Same Interface & Common method names & \texttt{area()} method \\
Different Implementation & Each class has own version & Rectangle vs
Circle area \\
Runtime Decision & Method chosen during execution & Dynamic binding \\
\end{longtable}
}

\textbf{Polymorphism Example:}

\begin{verbatim}
\# Base class
class Shape:
    def area(self):
        pass

\# Different implementations
class Rectangle(Shape):
    def \_\_init\_\_(self, length, width):
        self.length = length
        self.width = width
    
    def area(self):
        return self.length * self.width

class Circle(Shape):
    def \_\_init\_\_(self, radius):
        self.radius = radius
    
    def area(self):
        return 3.14 * self.radius * self.radius

\# Polymorphic behavior
shapes = [Rectangle(5, 3), Circle(4)]

for shape in shapes:
    print(f"Area: \{shape.area()\}")  \# Same method, different results
\end{verbatim}

\textbf{Benefits:}

\begin{itemize}
\tightlist
\item
  \textbf{Flexibility}: Same code works with different object types
\item
  \textbf{Extensibility}: Easy to add new classes without changing
  existing code
\item
  \textbf{Maintainability}: Changes in one class don't affect others
\end{itemize}

\end{solutionbox}
\begin{mnemonicbox}
``Same Name, Different Behavior''

\end{mnemonicbox}
\begin{center}\rule{0.5\linewidth}{0.5pt}\end{center}

\subsection*{Question 5(c OR) [7
marks]}\label{question-5c-or-7-marks}

\textbf{Develop a Python program to demonstrate working of hybrid
inheritance.}

\begin{solutionbox}

Hybrid inheritance combines multiple and multi-level inheritance in
single program structure.

\textbf{Hybrid Inheritance Structure:}

\begin{verbatim}
        Animal (Base)
          |
          v
        Mammal
       /      {}
      v        v
     Dog      Cat
       {      /}
        {    /}
         v  v
      Hybrid Pet
\end{verbatim}

\textbf{Inheritance Types in Hybrid:}

{\def\LTcaptype{none} % do not increment counter
\begin{longtable}[]{@{}lll@{}}
\toprule\noalign{}
Level & Type & Classes \\
\midrule\noalign{}
\endhead
\bottomrule\noalign{}
\endlastfoot
1 & Single & Animal \rightarrow Mammal \\
2 & Multiple & Mammal \rightarrow Dog, Cat \\
3 & Multiple & Dog, Cat \rightarrow Pet \\
\end{longtable}
}

\textbf{Complete Program:}

\begin{verbatim}
\# Hybrid Inheritance Demonstration

print("=== Hybrid Inheritance Demo ===")

\# Base class (Level 1)
class Animal:
    def \_\_init\_\_(self, name):
        self.name = name
    
    def eat(self):
        print(f"\{self.name\} can eat")
    
    def sleep(self):
        print(f"\{self.name\} can sleep")

\# Single inheritance (Level 2)
class Mammal(Animal):
    def breathe(self):
        print(f"\{self.name\} breathes air")
    
    def give\_birth(self):
        print(f"\{self.name\} gives birth to babies")

\# Multiple inheritance branches (Level 3)
class Dog(Mammal):
    def bark(self):
        print(f"\{self.name\} barks: Woof!")
    
    def loyalty(self):
        print(f"\{self.name\} is loyal to owner")

class Cat(Mammal):
    def meow(self):
        print(f"\{self.name\} meows: Meow!")
    
    def independence(self):
        print(f"\{self.name\} is independent")

\# Hybrid class {- Multiple inheritance (Level 4)}
class HybridPet(Dog, Cat):
    def \_\_init\_\_(self, name, breed):
        super().\_\_init\_\_(name)
        self.breed = breed
    
    def play(self):
        print(f"\{self.name\} loves to play")
    
    def show\_info(self):
        print(f"Name: \{self.name\}, Breed: \{self.breed\}")

\# Creating and using hybrid inheritance
print("{n}{-{-}{-} Creating Hybrid Pet {-}{-}{-}"})
pet = HybridPet("Buddy", "Labrador{-Persian Mix"})

print("{n}{-{-}{-} Methods from Animal (Great{-}grandparent) {-}{-}{-}"})
pet.eat()
pet.sleep()

print("{n}{-{-}{-} Methods from Mammal (Grandparent) {-}{-}{-}"})
pet.breathe()
pet.give\_birth()

print("{n}{-{-}{-} Methods from Dog (Parent 1) {-}{-}{-}"})
pet.bark()
pet.loyalty()

print("{n}{-{-}{-} Methods from Cat (Parent 2) {-}{-}{-}"})
pet.meow()
pet.independence()

print("{n}{-{-}{-} Own Methods {-}{-}{-}"})
pet.play()
pet.show\_info()

print("{n}{-{-}{-} Inheritance Chain {-}{-}{-}"})
print(f"MRO (Method Resolution Order): \{HybridPet.\_\_mro\_\_\}")

\# Checking inheritance relationships
print(f"{n}Is HybridPet subclass of Animal? \{issubclass(HybridPet, Animal)\}")
print(f"Is HybridPet subclass of Dog? \{issubclass(HybridPet, Dog)\}")
print(f"Is HybridPet subclass of Cat? \{issubclass(HybridPet, Cat)\}")
\end{verbatim}

\textbf{Expected Output:}

\begin{verbatim}
=== Hybrid Inheritance Demo ===

--- Creating Hybrid Pet ---

--- Methods from Animal (Great-grandparent) ---
Buddy can eat
Buddy can sleep

--- Methods from Mammal (Grandparent) ---
Buddy breathes air
Buddy gives birth to babies

--- Methods from Dog (Parent 1) ---
Buddy barks: Woof!
Buddy is loyal to owner

--- Methods from Cat (Parent 2) ---
Buddy meows: Meow!
Buddy is independent

--- Own Methods ---
Buddy loves to play
Name: Buddy, Breed: Labrador-Persian Mix

--- Inheritance Chain ---
MRO (Method Resolution Order): (<class '__main__.HybridPet'>, <class '__main__.Dog'>, <class '__main__.Cat'>, <class '__main__.Mammal'>, <class '__main__.Animal'>, <class 'object'>)

Is HybridPet subclass of Animal? True
Is HybridPet subclass of Dog? True
Is HybridPet subclass of Cat? True
\end{verbatim}

\textbf{Key Features of Hybrid Inheritance:}

\begin{itemize}
\tightlist
\item
  \textbf{Complex Structure}: Combines different inheritance types
\item
  \textbf{Method Resolution Order}: Python follows specific order for
  method lookup
\item
  \textbf{Diamond Problem}: Handled automatically by Python's MRO
\item
  \textbf{Flexibility}: Access to methods from multiple parent classes
\end{itemize}

\textbf{Advantages:}

\begin{itemize}
\tightlist
\item
  \textbf{Rich Functionality}: Inherits from multiple sources
\item
  \textbf{Code Reuse}: Maximum utilization of existing code
\item
  \textbf{Relationship Modeling}: Represents complex real-world
  relationships
\end{itemize}

\textbf{Challenges:}

\begin{itemize}
\tightlist
\item
  \textbf{Complexity}: Harder to understand and maintain
\item
  \textbf{Name Conflicts}: Multiple parents may have same method names
\item
  \textbf{Memory Usage}: Objects carry more overhead
\end{itemize}

\end{solutionbox}
\begin{mnemonicbox}
``Hybrid Combines All Types''

\end{mnemonicbox}

\end{document}
