\documentclass{article}
% Adjust the relative path to point to the latex-templates directory

% content/resources/templates/preamble.tex
\usepackage[margin=0.6in]{geometry}
\author{Milav Dabgar}
\usepackage{amsmath,amssymb,amsthm}
\usepackage{booktabs}
\usepackage{multirow}
\usepackage{xcolor}
\usepackage{tcolorbox}
\tcbuselibrary{breakable,skins}
\usepackage[colorlinks=true,linkcolor=blue]{hyperref}
\usepackage{titlesec}
\usepackage{enumitem}
\usepackage{tikz}
\usepackage{pgfplots}
\usepackage{circuitikz}
\usepackage[version=4]{mhchem}
\usepackage{longtable}
\usepackage{array}
\usepackage{float}
\usepackage{caption}
\usepackage{listings}

\lstset{
  basicstyle=\small\ttfamily,
  breaklines=true,
  breakatwhitespace=false,
  postbreak=\mbox{\textcolor{red}{$\hookrightarrow$}\space},
  float=false,
  numbers=left,
  numberstyle=\tiny\color{gray},
  numbersep=10pt,
  xleftmargin=2em,
  keywordstyle=\color{blue},
  commentstyle=\color{green!60!black},
  stringstyle=\color{purple},
  backgroundcolor=\color{gray!5},
  showstringspaces=false,
  tabsize=2,
  captionpos=b,
  keepspaces=true,
  columns=flexible
}

\pgfplotsset{compat=1.18}
\usetikzlibrary{shapes,arrows,positioning,calc,patterns,decorations.pathmorphing,decorations.markings,arrows.meta}

% Color scheme
\definecolor{headcolor}{RGB}{0,102,204}
\definecolor{keycolor}{RGB}{220,20,60}
\definecolor{solutioncolor}{RGB}{34,139,34}
\definecolor{mnemoniccolor}{RGB}{148,0,211}
\definecolor{codecolor}{RGB}{0,0,100}

% Spacing
\setlength{\parskip}{3pt}
\setlist[itemize]{nosep}
\setlist[enumerate]{nosep}

% Title formatting
\titleformat{\section}{\Large\bfseries\color{headcolor}}{\thesection}{1em}{}
\titleformat{\subsection}{\large\bfseries\color{headcolor}}{\thesubsection}{1em}{}

% Pandoc tightlist compatibility
\providecommand{\tightlist}{%
  \setlength{\itemsep}{0pt}\setlength{\parskip}{0pt}}

% Pandoc longtable compatibility
\newcounter{none}
\def\thenone{}


% content/resources/templates/english-boxes.tex
% This file is currently empty - it exists to maintain consistency with the import structure.
% Add custom environments here if needed in the future.


% Custom commands for GTU solutions
% This file defines semantic commands for consistent formatting

% Question command with automatic formatting
\newcommand{\question}[2]{%
  \section*{Question #1}%
  \textbf{#2}%
}

% OR question variant
\newcommand{\questionor}[2]{%
  \section*{Question #1 OR}%
  \textbf{#2}%
}

% Proper table environment with caption
\newenvironment{answertable}[1]{%
  \begin{table}[htbp]
  \centering
  \caption{#1}
}{%
  \end{table}
}

% Proper figure environment for diagrams
\newenvironment{answerdiagram}[1]{%
  \begin{figure}[htbp]
  \centering
  \caption{#1}
}{%
  \end{figure}
}

% Semantic markup for key terms
\newcommand{\keyword}[1]{\textbf{#1}}
\newcommand{\code}[1]{\texttt{#1}}
\newcommand{\classname}[1]{\texttt{#1}}
\newcommand{\methodname}[1]{\texttt{#1}}

% Proper quotation marks
\newcommand{\mnemonic}[1]{``#1''}


\title{OOPS \& Python Programming (4351108) - Summer 2024 Solution}
\date{May 18, 2024}

\begin{document}
\maketitle

\questionmarks{1(a)}{3}{Explain for loop working in Python.}

\begin{solutionbox}
For loop repeats code block for each item in sequence like list, tuple, or string.

\textbf{Syntax Table}:
\begin{center}
\captionof{table}{For Loop Syntax}
\begin{tabulary}{\linewidth}{|L|L|L|}
\hline
\textbf{Component} & \textbf{Syntax} & \textbf{Example} \\ \hline
Basic & \code{for variable in sequence:} & \code{for i in [1,2,3]:} \\ \hline
Range & \code{for i in range(n):} & \code{for i in range(5):} \\ \hline
String & \code{for char in string:} & \code{for c in "hello":} \\ \hline
\end{tabulary}
\end{center}

\textbf{Diagram}:
\begin{center}
\begin{tikzpicture}[gtu flow]
    \node [gtu start] (start) {Start};
    \node [gtu decision, below=1cm of start] (check) {Items left?};
    \node [gtu process, below=1cm of check] (execute) {Execute loop body};
    \node [gtu process, right=1cm of execute, text width=2cm] (next) {Move to\\next item};
    \node [gtu stop, right=3cm of check] (end) {End};

    \path [gtu arrow] (start) -- (check);
    \path [gtu arrow] (check) -- node[right] {Yes} (execute);
    \path [gtu arrow] (execute) -- (next);
    \path [gtu arrow] (next) |- (check);
    \path [gtu arrow] (check) -- node[above] {No} (end);
\end{tikzpicture}
\captionof{figure}{For Loop Execution Flow}
\end{center}

\begin{itemize}
    \item \keyword{Iteration}: Loop variable gets each value from sequence one by one
    \item \keyword{Automatic}: Python handles moving to next item automatically
    \item \keyword{Flexible}: Works with lists, strings, tuples, ranges
\end{itemize}
\end{solutionbox}

\begin{mnemonicbox}
\mnemonic{For Each Item, Execute Block}
\end{mnemonicbox}

\questionmarks{1(b)}{4}{Explain working of if-elif-else in Python.}

\begin{solutionbox}
Multi-way decision structure that checks multiple conditions in sequence.

\textbf{Structure Table}:
\begin{center}
\captionof{table}{If-Elif-Else Structure}
\begin{tabulary}{\linewidth}{|L|L|L|}
\hline
\textbf{Statement} & \textbf{Purpose} & \textbf{Syntax} \\ \hline
if & First condition & \code{if condition1:} \\ \hline
elif & Alternative conditions & \code{elif condition2:} \\ \hline
else & Default case & \code{else:} \\ \hline
\end{tabulary}
\end{center}

\textbf{Flow Diagram}:
\begin{center}
\begin{tikzpicture}[gtu flow]
    \node [gtu start] (start) {Start};
    \node [gtu decision, below=0.5cm of start] (cond1) {if condition?};
    \node [gtu process, left=0.5cm of cond1] (block1) {Execute if block};
    
    \node [gtu decision, below=1.5cm of cond1] (cond2) {elif condition?};
    \node [gtu process, left=0.5cm of cond2] (block2) {Execute elif block};
    
    \node [gtu process, right=0.5cm of cond2] (block3) {Execute else block};
    \node [gtu stop, below=1.5cm of cond2] (end) {End};

    \path [gtu arrow] (start) -- (cond1);
    \path [gtu arrow] (cond1) -- node[above] {True} (block1);
    \path [gtu arrow] (cond1) -- node[right] {False} (cond2);
    \path [gtu arrow] (cond2) -- node[above] {True} (block2);
    \path [gtu arrow] (cond2) -- node[above] {False} (block3);
    
    \path [gtu arrow] (block1) |- (end);
    \path [gtu arrow] (block2) |- (end);
    \path [gtu arrow] (block3) |- (end);
\end{tikzpicture}
\captionof{figure}{If-Elif-Else Logic Flow}
\end{center}

\begin{itemize}
    \item \keyword{Sequential}: Checks conditions top to bottom
    \item \keyword{Exclusive}: Only one block executes
    \item \keyword{Optional}: elif and else are optional
\end{itemize}
\end{solutionbox}

\begin{mnemonicbox}
\mnemonic{If This, Else If That, Else Default}
\end{mnemonicbox}

\questionmarks{1(c)}{7}{Explain structure of a Python Program.}

\begin{solutionbox}
Python program has organized structure with specific components in logical order.

\textbf{Program Structure Table}:
\begin{center}
\captionof{table}{Python Program Structure}
\begin{tabulary}{\linewidth}{|L|L|L|}
\hline
\textbf{Component} & \textbf{Purpose} & \textbf{Example} \\ \hline
Comments & Documentation & \code{\# This is comment} \\ \hline
Import & External modules & \code{import math} \\ \hline
Constants & Fixed values & \code{PI = 3.14} \\ \hline
Functions & Reusable code & \code{def function\_name():} \\ \hline
Classes & Objects blueprint & \code{class ClassName:} \\ \hline
Main code & Program execution & \code{if \_\_name\_\_ == "\_\_main\_\_":} \\ \hline
\end{tabulary}
\end{center}

\textbf{Program Architecture}:
\begin{center}
\begin{tikzpicture}[gtu flow]
    \node [gtu block] (comments) {Comments\\(\# Documentation)};
    \node [gtu block, below=0.5cm of comments] (imports) {Import Section\\(import modules)};
    \node [gtu block, below=0.5cm of imports] (const) {Constants \&\\Variables};
    \node [gtu block, below=0.5cm of const] (funcs) {Function\\Definitions};
    \node [gtu block, below=0.5cm of funcs] (classes) {Class\\Definitions};
    \node [gtu block, below=0.5cm of classes] (main) {Main Program\\Execution};

    \path [gtu arrow] (comments) -- (imports);
    \path [gtu arrow] (imports) -- (const);
    \path [gtu arrow] (const) -- (funcs);
    \path [gtu arrow] (funcs) -- (classes);
    \path [gtu arrow] (classes) -- (main);
\end{tikzpicture}
\captionof{figure}{Python Program Structure}
\end{center}

\begin{itemize}
    \item \keyword{Modular}: Each section has specific purpose
    \item \keyword{Readable}: Clear organization helps understanding
    \item \keyword{Maintainable}: Easy to modify and debug
    \item \keyword{Standard}: Follows Python conventions
\end{itemize}

\textbf{Simple Example}:
\begin{lstlisting}[language=Python]
# Program to calculate area
import math

PI = 3.14159

def calculate_area(radius):
    return PI * radius * radius

# Main execution
radius = float(input("Enter radius: "))
area = calculate_area(radius)
print(f"Area = {area}")
\end{lstlisting}
\end{solutionbox}

\begin{mnemonicbox}
\mnemonic{Comment, Import, Constant, Function, Class, Main}
\end{mnemonicbox}

\questionmarks{1(c OR)}{7}{Explain features of Python Programming Language.}

\begin{solutionbox}
Python has unique characteristics that make it popular for beginners and professionals.

\textbf{Python Features Table}:
\begin{center}
\captionof{table}{Python Features}
\begin{tabulary}{\linewidth}{|L|L|L|}
\hline
\textbf{Feature} & \textbf{Description} & \textbf{Benefit} \\ \hline
Simple & Easy syntax & Quick learning \\ \hline
Interpreted & No compilation & Fast development \\ \hline
Object-Oriented & Classes and objects & Code reusability \\ \hline
Open Source & Free to use & No licensing cost \\ \hline
Cross-Platform & Runs everywhere & High portability \\ \hline
\end{tabulary}
\end{center}

\textbf{Feature Categories}:
\begin{center}
\begin{tikzpicture}[gtu flow, level 1/.style={sibling distance=4cm}, level 2/.style={sibling distance=1.5cm}]
    \node [gtu root] {Python Features}
        child {node [gtu block] {Language}
            child {node [gtu child] {Simple}}
            child {node [gtu child] {Readable}}
            child {node [gtu child] {Dynamic}}
        }
        child {node [gtu block] {Technical}
            child {node [gtu child] {Interpreted}}
            child {node [gtu child] {Portable}}
            child {node [gtu child] {Extensible}}
        }
        child {node [gtu block] {Community}
            child {node [gtu child] {Open Source}}
            child {node [gtu child] {Libraries}}
            child {node [gtu child] {Support}}
        };
\end{tikzpicture}
\captionof{figure}{Python Features Hierarchy}
\end{center}

\begin{itemize}
    \item \keyword{Beginner-Friendly}: Simple syntax like English language
    \item \keyword{Versatile}: Used for web, AI, data science, automation
    \item \keyword{Rich Libraries}: Huge collection of pre-built modules
    \item \keyword{Dynamic Typing}: No need to declare variable types
\end{itemize}

\textbf{Code Example}:
\begin{lstlisting}[language=Python]
# Simple Python syntax
name = "Python"
print(f"Hello, {name}!")
\end{lstlisting}
\end{solutionbox}

\begin{mnemonicbox}
\mnemonic{Simple, Interpreted, Object-Oriented, Open, Cross-platform}
\end{mnemonicbox}

\questionmarks{2(a)}{3}{Explain any 3 operations done on Strings.}

\begin{solutionbox}
String operations manipulate and process text data in various ways.

\textbf{String Operations Table}:
\begin{center}
\captionof{table}{String Operations}
\begin{tabulary}{\linewidth}{|L|L|L|L|}
\hline
\textbf{Operation} & \textbf{Method} & \textbf{Example} & \textbf{Result} \\ \hline
Concatenation & \code{+} & \code{"Hello" + "World"} & \code{"HelloWorld"} \\ \hline
Length & \code{len()} & \code{len("Python")} & \code{6} \\ \hline
Uppercase & \code{.upper()} & \code{"hello".upper()} & \code{"HELLO"} \\ \hline
\end{tabulary}
\end{center}

\textbf{Operation Examples}:
\begin{lstlisting}[language=Python]
text = "Python"
# 1. Concatenation
result1 = text + " Programming"
# 2. Find length
result2 = len(text)
# 3. Convert to uppercase
result3 = text.upper()
\end{lstlisting}
\end{solutionbox}

\begin{mnemonicbox}
\mnemonic{Combine, Count, Convert}
\end{mnemonicbox}

\questionmarks{2(b)}{4}{Develop a Python program to convert temperature from Fahrenheit to Celsius unit using eq: C=(F-32)/1.8}

\begin{solutionbox}
Program converts temperature using mathematical formula with user input.

\textbf{Algorithm Table}:
\begin{center}
\captionof{table}{Conversion Algorithm}
\begin{tabulary}{\linewidth}{|L|L|L|}
\hline
\textbf{Step} & \textbf{Action} & \textbf{Code} \\ \hline
1 & Get input & \code{fahrenheit = float(input())} \\ \hline
2 & Apply formula & \code{celsius = (fahrenheit - 32) / 1.8} \\ \hline
3 & Display result & \code{print(f"Celsius: \{celsius\}")} \\ \hline
\end{tabulary}
\end{center}

\textbf{Complete Program}:
\begin{lstlisting}[language=Python]
# Temperature conversion program
fahrenheit = float(input("Enter temperature in Fahrenheit: "))
celsius = (fahrenheit - 32) / 1.8
print(f"Temperature in Celsius: {celsius:.2f}")
\end{lstlisting}

\textbf{Test Cases}:
\begin{itemize}
    \item Input: 32$^\circ$F $\rightarrow$ Output: 0.00$^\circ$C
    \item Input: 100$^\circ$F $\rightarrow$ Output: 37.78$^\circ$C
\end{itemize}
\end{solutionbox}

\begin{mnemonicbox}
\mnemonic{Input, Calculate, Output}
\end{mnemonicbox}

\questionmarks{2(c)}{7}{Explain in detail working of list data types in Python.}

\begin{solutionbox}
List is ordered, mutable collection that stores multiple items in single variable.

\textbf{List Characteristics Table}:
\begin{center}
\captionof{table}{List Characteristics}
\begin{tabulary}{\linewidth}{|L|L|L|}
\hline
\textbf{Property} & \textbf{Description} & \textbf{Example} \\ \hline
Ordered & Items have position & \code{[1, 2, 3]} \\ \hline
Mutable & Can be changed & \code{list[0] = 10} \\ \hline
Indexed & Access by position & \code{list[0]} \\ \hline
Mixed Types & Different data types & \code{[1, "hello", 3.14]} \\ \hline
\end{tabulary}
\end{center}

\textbf{List Operations Diagram}:
\begin{center}
\begin{tikzpicture}[gtu flow]
    \node [gtu block, minimum width=4cm] (list) {List: [10, 20, 30, 40]\\Index: 0~~ 1~~ 2~~ 3};
    
    \node [gtu input, below left=1cm and 0.5cm of list] (access) {Access\\list[0]};
    \node [gtu process, below right=1cm and 0.5cm of list] (modify) {Modify\\list[0]=50};
    
    \node [gtu output, below=1cm of access] (res1) {"10"};
    \node [gtu output, below=1cm of modify] (res2) {[50, 20, 30, 40]};
    
    \path [gtu arrow] (list) -- (access);
    \path [gtu arrow] (list) -- (modify);
    \path [gtu arrow] (access) -- (res1);
    \path [gtu arrow] (modify) -- (res2);
\end{tikzpicture}
\captionof{figure}{List Operations}
\end{center}

\textbf{Common List Methods}:
\begin{itemize}
    \item \code{append()}: Add item at end
    \item \code{insert()}: Add at position
    \item \code{remove()}: Delete item
    \item \code{pop()}: Remove last item
\end{itemize}

\textbf{Example Code}:
\begin{lstlisting}[language=Python]
# Creating and using lists
numbers = [1, 2, 3, 4, 5]
numbers.append(6)        # Add 6 at end
numbers.insert(0, 0)     # Add 0 at beginning
print(numbers[2])        # Access 3rd element
numbers.remove(3)        # Remove value 3
\end{lstlisting}
\end{solutionbox}

\begin{mnemonicbox}
\mnemonic{Ordered, Mutable, Indexed, Mixed}
\end{mnemonicbox}

\questionmarks{2(a OR)}{3}{Explain String formatting in Python.}

\begin{solutionbox}
String formatting creates formatted strings by inserting values into templates.

\textbf{Formatting Methods Table}:
\begin{center}
\captionof{table}{Formatting Methods}
\begin{tabulary}{\linewidth}{|L|L|L|}
\hline
\textbf{Method} & \textbf{Syntax} & \textbf{Example} \\ \hline
f-strings & \code{f"text \{variable\}"} & \code{f"Hello \{name\}"} \\ \hline
format() & \code{"text \{\}".format(value)} & \code{"Age: \{\}".format(25)} \\ \hline
\% operator & \code{"text \%s" \% value} & \code{"Name: \%s" \% "John"} \\ \hline
\end{tabulary}
\end{center}

\textbf{Example Usage}:
\begin{lstlisting}[language=Python]
name = "Alice"
age = 25
# f-string formatting
message = f"Hello {name}, you are {age} years old"
\end{lstlisting}
\end{solutionbox}

\begin{mnemonicbox}
\mnemonic{Format, Insert, Display}
\end{mnemonicbox}

\questionmarks{2(b OR)}{4}{Develop a Python program to identify whether the scanned number is even or odd and print an appropriate message.}

\begin{solutionbox}
Program checks if number is divisible by 2 to determine even or odd.

\textbf{Logic Table}:
\begin{center}
\captionof{table}{Even/Odd Logic}
\begin{tabulary}{\linewidth}{|L|L|L|}
\hline
\textbf{Condition} & \textbf{Result} & \textbf{Message} \\ \hline
number \% 2 == 0 & Even & "Number is even" \\ \hline
number \% 2 != 0 & Odd & "Number is odd" \\ \hline
\end{tabulary}
\end{center}

\textbf{Complete Program}:
\begin{lstlisting}[language=Python]
# Even/Odd checker program
number = int(input("Enter a number: "))
if number % 2 == 0:
    print(f"{number} is even")
else:
    print(f"{number} is odd")
\end{lstlisting}

\textbf{Test Cases}:
\begin{itemize}
    \item Input: 4 $\rightarrow$ Output: "4 is even"
    \item Input: 7 $\rightarrow$ Output: "7 is odd"
\end{itemize}
\end{solutionbox}

\begin{mnemonicbox}
\mnemonic{Input, Check Remainder, Display Result}
\end{mnemonicbox}

\questionmarks{2(c OR)}{7}{Explain in detail working of Set data types in Python.}

\begin{solutionbox}
Set is unordered collection of unique items with no duplicate values allowed.

\textbf{Set Characteristics Table}:
\begin{center}
\captionof{table}{Set Characteristics}
\begin{tabulary}{\linewidth}{|L|L|L|}
\hline
\textbf{Property} & \textbf{Description} & \textbf{Example} \\ \hline
Unordered & No fixed position & \code{\{1, 3, 2\}} \\ \hline
Unique & No duplicates & \code{\{1, 2, 3\}} \\ \hline
Mutable & Can be modified & \code{set.add(4)} \\ \hline
Iterable & Can loop through & \code{for item in set:} \\ \hline
\end{tabulary}
\end{center}

\textbf{Set Operations Diagram}:
\begin{center}
\begin{tikzpicture}[gtu flow]
    \node [gtu start] (setA) {Set A: \{1, 2, 3\}};
    \node [gtu start, right=2cm of setA] (setB) {Set B: \{3, 4, 5\}};
    
    \node [gtu block, below=1cm of setA, xshift=2cm] (union) {Union\\\{1, 2, 3, 4, 5\}};
    \node [gtu block, below=0.5cm of union] (ix) {Intersection\\\{3\}};
    \node [gtu block, below=0.5cm of ix] (diff) {Difference\\\{1, 2\}};
    
    \path [gtu arrow] (setA) -- (union);
    \path [gtu arrow] (setB) -- (union);
\end{tikzpicture}
\captionof{figure}{Set Operations}
\end{center}

\textbf{Set Methods Table}:
\begin{center}
\captionof{table}{Set Methods}
\begin{tabulary}{\linewidth}{|L|L|L|}
\hline
\textbf{Method} & \textbf{Purpose} & \textbf{Example} \\ \hline
add() & Add single item & \code{set.add(6)} \\ \hline
update() & Add multiple items & \code{set.update([7, 8])} \\ \hline
remove() & Delete item & \code{set.remove(3)} \\ \hline
union() & Combine sets & \code{set1.union(set2)} \\ \hline
intersection() & Common items & \code{set1.intersection(set2)} \\ \hline
\end{tabulary}
\end{center}

\textbf{Example Code}:
\begin{lstlisting}[language=Python]
# Creating and using sets
fruits = {"apple", "banana", "orange"}
fruits.add("mango")              # Add single item
fruits.update(["grape", "kiwi"]) # Add multiple
fruits.remove("banana")          # Remove item
print(len(fruits))               # Count items
\end{lstlisting}
\end{solutionbox}

\begin{mnemonicbox}
\mnemonic{Unique, Unordered, Mutable, Mathematical}
\end{mnemonicbox}

\questionmarks{3(a)}{3}{Explain working of any 3 methods of math module.}

\begin{solutionbox}
Math module provides mathematical functions for complex calculations.

\textbf{Math Methods Table}:
\begin{center}
\captionof{table}{Math Methods}
\begin{tabulary}{\linewidth}{|L|L|L|L|}
\hline
\textbf{Method} & \textbf{Purpose} & \textbf{Example} & \textbf{Result} \\ \hline
math.sqrt() & Square root & \code{math.sqrt(16)} & \code{4.0} \\ \hline
math.pow() & Power calculation & \code{math.pow(2, 3)} & \code{8.0} \\ \hline
math.ceil() & Round up & \code{math.ceil(4.3)} & \code{5} \\ \hline
\end{tabulary}
\end{center}

\textbf{Usage Example}:
\begin{lstlisting}[language=Python]
import math
number = 16
result1 = math.sqrt(number)  # Square root
result2 = math.pow(2, 4)     # 2 to power 4
result3 = math.ceil(7.2)     # Round up to 8
\end{lstlisting}
\end{solutionbox}

\begin{mnemonicbox}
\mnemonic{Square root, Power, Ceiling}
\end{mnemonicbox}

\questionmarks{3(b)}{4}{Develop a Python program to find sum of all elements in a list using for loop.}

\begin{solutionbox}
Program iterates through list and accumulates sum of all elements.

\textbf{Algorithm Table}:
\begin{center}
\captionof{table}{Summation Algorithm}
\begin{tabulary}{\linewidth}{|L|L|L|}
\hline
\textbf{Step} & \textbf{Action} & \textbf{Code} \\ \hline
1 & Initialize sum & \code{total = 0} \\ \hline
2 & Loop through list & \code{for element in list:} \\ \hline
3 & Add to sum & \code{total += element} \\ \hline
4 & Display result & \code{print(total)} \\ \hline
\end{tabulary}
\end{center}

\textbf{Complete Program}:
\begin{lstlisting}[language=Python]
# Sum of list elements
numbers = [10, 20, 30, 40, 50]
total = 0
for element in numbers:
    total += element
print(f"Sum of all elements: {total}")
\end{lstlisting}

\textbf{Test Case}:
\begin{itemize}
    \item Input: [1, 2, 3, 4, 5] $\rightarrow$ Output: 15
\end{itemize}
\end{solutionbox}

\begin{mnemonicbox}
\mnemonic{Initialize, Loop, Add, Display}
\end{mnemonicbox}

\questionmarks{3(c)}{7}{Develop a Python program to check if two lists are having similar length. If yes then merge them and create a dictionary from them.}

\begin{solutionbox}
Program compares list lengths and creates dictionary if they match.

\textbf{Logic Flow Table}:
\begin{center}
\captionof{table}{Merge Logic}
\begin{tabulary}{\linewidth}{|L|L|L|}
\hline
\textbf{Step} & \textbf{Condition} & \textbf{Action} \\ \hline
1 & Check lengths & \code{len(list1) == len(list2)} \\ \hline
2 & If equal & Merge and create dictionary \\ \hline
3 & If not equal & Display error message \\ \hline
\end{tabulary}
\end{center}

\textbf{Process Diagram}:
\begin{center}
\begin{tikzpicture}[gtu flow]
    \node [gtu block] (l1) {List1};
    \node [gtu block, right=2cm of l1] (l2) {List2};
    \node [gtu decision, below=1cm of l1, xshift=2cm] (check) {Length Equal?};
    \node [gtu process, below left=1cm and 0.5cm of check] (create) {Create Dict\\dict(zip(l1,l2))};
    \node [gtu process, below right=1cm and 0.5cm of check] (err) {Error Message};
    
    \path [gtu arrow] (l1) -- (check);
    \path [gtu arrow] (l2) -- (check);
    \path [gtu arrow] (check) -- node[left] {Yes} (create);
    \path [gtu arrow] (check) -- node[right] {No} (err);
\end{tikzpicture}
\captionof{figure}{List Merge Logic}
\end{center}

\textbf{Complete Program}:
\begin{lstlisting}[language=Python]
# Merge lists into dictionary
list1 = ['name', 'age', 'city']
list2 = ['John', 25, 'Mumbai']

if len(list1) == len(list2):
    # Create dictionary using zip
    result_dict = dict(zip(list1, list2))
    print("Dictionary created:", result_dict)
else:
    print("Lists have different lengths, cannot merge")
\end{lstlisting}

\textbf{Expected Output}:
\begin{lstlisting}
Dictionary created: {'name': 'John', 'age': 25, 'city': 'Mumbai'}
\end{lstlisting}
\end{solutionbox}

\begin{mnemonicbox}
\mnemonic{Check Length, Zip, Create Dictionary}
\end{mnemonicbox}

\questionmarks{3(a OR)}{3}{Explain working of any 3 methods of statistics module.}

\begin{solutionbox}
Statistics module provides functions for statistical calculations on numeric data.

\textbf{Statistics Methods Table}:
\begin{center}
\captionof{table}{Statistics Methods}
\begin{tabulary}{\linewidth}{|L|L|L|L|}
\hline
\textbf{Method} & \textbf{Purpose} & \textbf{Example} & \textbf{Result} \\ \hline
statistics.mean() & Average value & \code{mean([1,2,3,4,5])} & \code{3.0} \\ \hline
statistics.median() & Middle value & \code{median([1,2,3,4,5])} & \code{3} \\ \hline
statistics.mode() & Most frequent & \code{mode([1,1,2,3])} & \code{1} \\ \hline
\end{tabulary}
\end{center}

\textbf{Usage Example}:
\begin{lstlisting}[language=Python]
import statistics
data = [10, 20, 30, 40, 50]
avg = statistics.mean(data)      # Calculate average
mid = statistics.median(data)    # Find middle value
\end{lstlisting}
\end{solutionbox}

\begin{mnemonicbox}
\mnemonic{Mean, Median, Mode}
\end{mnemonicbox}

\questionmarks{3(c OR)}{7}{Develop a Python program to count the number of times a character appears in a given string using a dictionary.}

\begin{solutionbox}
Program creates dictionary where keys are characters and values are their counts.

\textbf{Character Counting Algorithm}:
\begin{center}
\captionof{table}{Counting Algorithm}
\begin{tabulary}{\linewidth}{|L|L|L|}
\hline
\textbf{Step} & \textbf{Action} & \textbf{Code} \\ \hline
1 & Initialize dictionary & \code{char\_count = \{\}} \\ \hline
2 & Loop through string & \code{for char in string:} \\ \hline
3 & Count occurrences & \code{char\_count[char] = get() + 1} \\ \hline
4 & Display results & \code{print(char\_count)} \\ \hline
\end{tabulary}
\end{center}

\textbf{Counting Process}:
\begin{center}
\begin{tikzpicture}[gtu flow]
    \node [gtu start] (string) {String: "hello"};
    \node [gtu process, below=1cm of string] (loop) {Loop each char};
    \node [gtu database, below=1cm of loop] (dict) {Dictionary\\\{'h':1, 'e':1, 'l':2, 'o':1\}};
    
    \path [gtu arrow] (string) -- (loop);
    \path [gtu arrow] (loop) -- (dict);
\end{tikzpicture}
\captionof{figure}{Character Counting Logic}
\end{center}

\textbf{Complete Program}:
\begin{lstlisting}[language=Python]
# Character frequency counter
text = input("Enter a string: ")
char_count = {}

for char in text:
    if char in char_count:
        char_count[char] += 1
    else:
        char_count[char] = 1

print("Character frequencies:")
for char, count in char_count.items():
    print(f"'{char}': {count}")
\end{lstlisting}
\end{solutionbox}

\begin{mnemonicbox}
\mnemonic{Loop, Check, Count, Store}
\end{mnemonicbox}

\questionmarks{4(a)}{3}{Explain working of Python class and objects with example.}

\begin{solutionbox}
Class is blueprint for creating objects. Objects are instances of classes.

\textbf{Class-Object Relationship}:
\begin{center}
\captionof{table}{Class vs Object}
\begin{tabulary}{\linewidth}{|L|L|L|}
\hline
\textbf{Concept} & \textbf{Purpose} & \textbf{Example} \\ \hline
Class & Template/Blueprint & \code{class Car:} \\ \hline
Object & Instance of class & \code{my\_car = Car()} \\ \hline
Attributes & Data in class & \code{self.color = "red"} \\ \hline
Methods & Functions in class & \code{def start(self):} \\ \hline
\end{tabulary}
\end{center}

\textbf{Class Structure Diagram}:
\begin{center}
\begin{tikzpicture}[gtu flow]
    \node [gtu class] (class) {\textbf{Class: Car}\\Attributes: color, model\\Methods: start(), stop()};
    \node [gtu object, below=1.5cm of class] (obj) {\textbf{Object}\\my\_car = Car()};
    
    \draw [gtu arrow] (class) -- node[right] {Instantiate} (obj);
\end{tikzpicture}
\captionof{figure}{Class and Object}
\end{center}

\textbf{Example Code}:
\begin{lstlisting}[language=Python]
class Student:
    def __init__(self, name, age):
        self.name = name  # Attribute
        self.age = age    # Attribute
    
    def display(self):    # Method
        print(f"Name: {self.name}, Age: {self.age}")

# Creating objects
student1 = Student("Alice", 20)
student1.display()
\end{lstlisting}
\end{solutionbox}

\begin{mnemonicbox}
\mnemonic{Class Blueprint, Object Instance}
\end{mnemonicbox}

\questionmarks{4(b)}{4}{Develop a Python program to print all odd numbers in a list.}

\begin{solutionbox}
Program filters list elements and displays only odd numbers.

\textbf{Odd Number Check Table}:
\begin{center}
\captionof{table}{Odd Number Logic}
\begin{tabulary}{\linewidth}{|L|L|L|}
\hline
\textbf{Number} & \textbf{Mod 2 (mod)} & \textbf{Result} \\ \hline
1 & 1 & Odd \\ \hline
2 & 0 & Even \\ \hline
\end{tabulary}
\end{center}

\textbf{Complete Program}:
\begin{lstlisting}[language=Python]
# Print odd numbers from list
numbers = [1, 2, 3, 4, 5, 6, 7, 8, 9, 10]

print("Odd numbers in the list:")
for number in numbers:
    if number % 2 != 0:
        print(number, end=" ")
\end{lstlisting}

\textbf{Expected Output}:
\begin{lstlisting}
Odd numbers in the list:
1 3 5 7 9
\end{lstlisting}
\end{solutionbox}

\begin{mnemonicbox}
\mnemonic{Loop, Check Remainder, Print Odd}
\end{mnemonicbox}

\questionmarks{4(c)}{7}{Explain working of user defined functions in Python.}

\begin{solutionbox}
User-defined functions are custom functions created by programmers to perform specific tasks.

\textbf{Function Components Table}:
\begin{center}
\captionof{table}{Function Components}
\begin{tabulary}{\linewidth}{|L|L|L|}
\hline
\textbf{Component} & \textbf{Purpose} & \textbf{Syntax} \\ \hline
def keyword & Function declaration & \code{def function\_name():} \\ \hline
Parameters & Input values & \code{def func(param1, param2):} \\ \hline
Body & Function code & Indented statements \\ \hline
return & Output value & \code{return value} \\ \hline
\end{tabulary}
\end{center}

\textbf{Function Structure}:
\begin{center}
\begin{tikzpicture}[gtu flow]
    \node [gtu process] (def) {def function\_name(parameters):};
    \node [gtu block, below=0.5cm of def] (body) {Function Body\\- Local variables\\- Processing logic};
    \node [gtu output, below=0.5cm of body] (ret) {return result};
    
    \path [gtu arrow] (def) -- (body);
    \path [gtu arrow] (body) -- (ret);
\end{tikzpicture}
\captionof{figure}{Function Anatomy}
\end{center}

\textbf{Types of Functions}:
\begin{itemize}
    \item No parameters: \code{def greet():}
    \item With parameters: \code{def add(a, b):}
    \item Return value: \code{return a + b}
    \item No return: \code{print("Hello")}
\end{itemize}

\textbf{Example Functions}:
\begin{lstlisting}[language=Python]
# Function with parameters and return value
def calculate_area(length, width):
    area = length * width
    return area

# Using functions
result = calculate_area(5, 3)
print(f"Area: {result}")
\end{lstlisting}
\end{solutionbox}

\begin{mnemonicbox}
\mnemonic{Define, Parameters, Body, Return}
\end{mnemonicbox}

\questionmarks{4(a OR)}{3}{Explain working constructors in Python.}

\begin{solutionbox}
Constructor is special method that initializes objects when they are created.

\textbf{Constructor Details Table}:
\begin{center}
\captionof{table}{Constructor Details}
\begin{tabulary}{\linewidth}{|L|L|L|}
\hline
\textbf{Aspect} & \textbf{Description} & \textbf{Syntax} \\ \hline
Method name & Always \code{\_\_init\_\_} & \code{def \_\_init\_\_(self):} \\ \hline
Purpose & Initialize object & Set initial values \\ \hline
Automatic call & Called during object creation & \code{obj = Class()} \\ \hline
\end{tabulary}
\end{center}

\textbf{Constructor Example}:
\begin{lstlisting}[language=Python]
class Student:
    def __init__(self, name, age):
        self.name = name
        self.age = age
        print("Student object created")

# Object creation automatically calls constructor
student1 = Student("Alice", 20)
\end{lstlisting}

\begin{itemize}
    \item \keyword{Automatic Execution}: Runs immediately when object is created
    \item \keyword{Initialization}: Sets up object's initial state
    \item \keyword{self Parameter}: Refers to current object being created
\end{itemize}
\end{solutionbox}

\begin{mnemonicbox}
\mnemonic{Initialize, Automatic, Self}
\end{mnemonicbox}

\questionmarks{4(b OR)}{4}{Develop a Python program to find smallest number in a list without using min function.}

\begin{solutionbox}
Program manually compares all elements to find the smallest value.

\textbf{Finding Minimum Algorithm}:
\begin{center}
\captionof{table}{Min Finding Algorithm}
\begin{tabulary}{\linewidth}{|L|L|L|}
\hline
\textbf{Step} & \textbf{Action} & \textbf{Code} \\ \hline
1 & Assume first is smallest & \code{smallest = list[0]} \\ \hline
2 & Compare with others & \code{for num in list[1:]:} \\ \hline
3 & Update if smaller found & \code{if num < smallest:} \\ \hline
4 & Display result & \code{print(smallest)} \\ \hline
\end{tabulary}
\end{center}

\textbf{Complete Program}:
\begin{lstlisting}[language=Python]
# Find smallest number without min()
numbers = [45, 23, 67, 12, 89, 5, 34]

smallest = numbers[0]  # Assume first is smallest

for i in range(1, len(numbers)):
    if numbers[i] < smallest:
        smallest = numbers[i]

print(f"Smallest number: {smallest}")
\end{lstlisting}

\textbf{Expected Output}:
\begin{lstlisting}
Smallest number: 5
\end{lstlisting}
\end{solutionbox}

\begin{mnemonicbox}
\mnemonic{Assume, Compare, Update, Display}
\end{mnemonicbox}

\questionmarks{4(c OR)}{7}{Explain working of user defined Modules in Python.}

\begin{solutionbox}
User-defined modules are custom Python files containing functions, classes, and variables that can be imported and used in other programs.

\textbf{Module Components}: Functions, Classes, Variables, Constants.

\textbf{Module Creation Process}:
\begin{center}
\begin{tikzpicture}[gtu flow]
    \node [gtu start] (step1) {Step 1: Create .py file};
    \node [gtu process, below=0.5cm of step1] (step2) {Step 2: Write functions/classes};
    \node [gtu process, below=0.5cm of step2] (step3) {Step 3: Save file};
    \node [gtu process, below=0.5cm of step3] (step4) {Step 4: Import in other programs};
    \node [gtu stop, below=0.5cm of step4] (step5) {Step 5: Use module};
    
    \path [gtu arrow] (step1) -- (step2);
    \path [gtu arrow] (step2) -- (step3);
    \path [gtu arrow] (step3) -- (step4);
    \path [gtu arrow] (step4) -- (step5);
\end{tikzpicture}
\captionof{figure}{Module Lifecycle}
\end{center}

\textbf{Example Module (math\_operations.py)}:
\begin{lstlisting}[language=Python]
PI = 3.14159

def calculate_circle_area(radius):
    return PI * radius * radius
\end{lstlisting}

\textbf{Main Program}:
\begin{lstlisting}[language=Python]
import math_operations

# Using module functions
radius = 5
area = math_operations.calculate_circle_area(radius)
print(f"Circle area: {area}")
\end{lstlisting}

\textbf{Module Benefits}:
\begin{itemize}
    \item \keyword{Code Reusability}: Write once, use in multiple programs
    \item \keyword{Organization}: Keep related functions together
    \item \keyword{Namespace}: Avoid naming conflicts
\end{itemize}
\end{solutionbox}

\begin{mnemonicbox}
\mnemonic{Create File, Define Functions, Import, Use}
\end{mnemonicbox}

\questionmarks{5(a)}{3}{Explain single inheritance in Python with example.}

\begin{solutionbox}
Single inheritance is when one class inherits properties and methods from exactly one parent class.

\textbf{Inheritance Structure}: Parent Class (Base) $\rightarrow$ Child Class (Derived).

\textbf{Inheritance Diagram}:
\begin{center}
\begin{tikzpicture}[gtu flow]
    \node [gtu class] (parent) {\textbf{Parent: Animal}\\Attributes: name, age\\Methods: eat(), sleep()};
    \node [gtu class, below=1.5cm of parent] (child) {\textbf{Child: Dog}\\Inherited: name, eat()\\Own: bark()};
    
    \draw [gtu arrow, dashed] (parent) -- node[right] {inherits} (child);
\end{tikzpicture}
\captionof{figure}{Single Inheritance}
\end{center}

\textbf{Example Code}:
\begin{lstlisting}[language=Python]
class Animal:
    def eat(self):
        print("Eating")

class Dog(Animal):
    def bark(self):
        print("Barking")

my_dog = Dog()
my_dog.eat()    # Inherited
my_dog.bark()   # Own
\end{lstlisting}
\end{solutionbox}

\begin{mnemonicbox}
\mnemonic{One Parent, One Child}
\end{mnemonicbox}

\questionmarks{5(b)}{4}{Explain concept of abstraction in Python with its advantages.}

\begin{solutionbox}
Abstraction hides complex implementation details and shows only essential features to the user.

\textbf{Abstraction Concepts}:
\begin{itemize}
    \item \textbf{Abstract Class}: Cannot be instantiated (\code{class Shape(ABC):})
    \item \textbf{Abstract Method}: Must be implemented by child (\code{@abstractmethod})
\end{itemize}

\textbf{Implementation}:
\begin{lstlisting}[language=Python]
from abc import ABC, abstractmethod

class Shape(ABC):
    @abstractmethod
    def area(self):
        pass

class Rectangle(Shape):
    def area(self):
        return self.length * self.width
\end{lstlisting}

\textbf{Advantages}:
\begin{itemize}
    \item \keyword{Simplicity}: Hides complex details
    \item \keyword{Security}: Hides internal implementation
    \item \keyword{Maintainability}: Change implementation safely
\end{itemize}
\end{solutionbox}

\begin{mnemonicbox}
\mnemonic{Hide Details, Show Interface}
\end{mnemonicbox}

\questionmarks{5(c)}{7}{Develop a Python program to demonstrate working of multiple and multi-level inheritances.}

\begin{solutionbox}
Program shows both inheritance types: multiple (multiple parents) and multi-level (chain of inheritance).

\textbf{Inheritance Hierarchy}:
\begin{center}
\begin{tikzpicture}[gtu flow]
    % Multiple
    \node [gtu class] (father) {Father};
    \node [gtu class, right=1cm of father] (mother) {Mother};
    \node [gtu class, below right=1cm and -0.5cm of father] (child) {Child};
    \draw [gtu arrow] (father) -- (child);
    \draw [gtu arrow] (mother) -- (child);
    \node [above=0.2cm of father] {Multiple};
    
    % Multi-level
    \node [gtu class, right=3cm of mother] (gpar) {Animal};
    \node [gtu class, below=0.5cm of gpar] (par) {Mammal};
    \node [gtu class, below=0.5cm of par] (kid) {Dog};
    \draw [gtu arrow] (gpar) -- (par);
    \draw [gtu arrow] (par) -- (kid);
    \node [above=0.2cm of gpar] {Multi-level};
\end{tikzpicture}
\captionof{figure}{Inheritance Types}
\end{center}

\textbf{Complete Program}:
\begin{lstlisting}[language=Python]
print("=== Multi-level Inheritance ===")
class Animal:
    def eat(self): print("Eating")
class Mammal(Animal):
    def breathe(self): print("Breathing")
class Dog(Mammal):
    def bark(self): print("Barking")

d = Dog()
d.eat(); d.breathe(); d.bark()

print("\n=== Multiple Inheritance ===")
class Father:
    def f_method(self): print("Father")
class Mother:
    def m_method(self): print("Mother")
class Child(Father, Mother):
    pass

c = Child()
c.f_method(); c.m_method()
\end{lstlisting}
\end{solutionbox}

\begin{mnemonicbox}
\mnemonic{Multiple Parents, Multi-level Chain}
\end{mnemonicbox}

\questionmarks{5(a OR)}{3}{Explain working of 3 types of methods in Python.}

\begin{solutionbox}
Python classes have three types of methods based on how they access class data.

\textbf{Method Types Table}:
\begin{center}
\captionof{table}{Method Types}
\begin{tabulary}{\linewidth}{|L|L|L|}
\hline
\textbf{Method Type} & \textbf{First Parameter} & \textbf{Purpose} \\ \hline
Instance Method & \code{self} & Access instance data \\ \hline
Class Method & \code{cls} & Access class data \\ \hline
Static Method & None & Utility functions \\ \hline
\end{tabulary}
\end{center}

\textbf{Example Code}:
\begin{lstlisting}[language=Python]
class Student:
    school = "ABC"
    def display(self): pass           # Instance
    @classmethod
    def get_school(cls): pass         # Class
    @staticmethod
    def is_adult(age): pass           # Static
\end{lstlisting}
\end{solutionbox}

\begin{mnemonicbox}
\mnemonic{Instance Self, Class Cls, Static None}
\end{mnemonicbox}

\questionmarks{5(b OR)}{4}{Explain polymorphism through inheritance in Python.}

\begin{solutionbox}
Polymorphism allows objects of different classes to be treated as objects of common base class.

\textbf{Key Concept}: Same method name, different implementation.

\textbf{Example}:
\begin{lstlisting}[language=Python]
class Shape:
    def area(self): pass

class Rectangle(Shape):
    def area(self): return self.l * self.w

class Circle(Shape):
    def area(self): return 3.14 * self.r * self.r

shapes = [Rectangle(5,3), Circle(4)]
for s in shapes:
    print(s.area())  # Polymorphic call
\end{lstlisting}

\begin{itemize}
    \item \keyword{Flexibility}: Same code works with different object types
    \item \keyword{Extensibility}: Easy to add new classes
\end{itemize}
\end{solutionbox}

\begin{mnemonicbox}
\mnemonic{Same Name, Different Behavior}
\end{mnemonicbox}

\questionmarks{5(c OR)}{7}{Develop a Python program to demonstrate working of hybrid inheritance.}

\begin{solutionbox}
Hybrid inheritance combines multiple and multi-level inheritance.

\textbf{Structure}: 
\begin{itemize}
    \item Animal $\rightarrow$ Mammal (Single)
    \item Mammal $\rightarrow$ Dog, Cat (Hierarchical)
    \item Dog, Cat $\rightarrow$ Pet (Multiple)
\end{itemize}

\textbf{Diagram}:
\begin{center}
\begin{tikzpicture}[gtu flow]
    \node [gtu class] (animal) {Animal};
    \node [gtu class, below=0.5cm of animal] (mammal) {Mammal};
    \node [gtu class, below left=1cm and 0.5cm of mammal] (dog) {Dog};
    \node [gtu class, below right=1cm and 0.5cm of mammal] (cat) {Cat};
    \node [gtu class, below=2.5cm of mammal] (pet) {Pet};
    
    \draw [gtu arrow] (animal) -- (mammal);
    \draw [gtu arrow] (mammal) -- (dog);
    \draw [gtu arrow] (mammal) -- (cat);
    \draw [gtu arrow] (dog) -- (pet);
    \draw [gtu arrow] (cat) -- (pet);
\end{tikzpicture}
\captionof{figure}{Hybrid Inheritance}
\end{center}

\textbf{Complete Program}:
\begin{lstlisting}[language=Python]
# Hybrid Inheritance Demo
class Animal:
    def __init__(self, name): self.name = name

class Mammal(Animal):
    def breathe(self): print("Breathing")

class Dog(Mammal):
    def bark(self): print("Barking")

class Cat(Mammal):
    def meow(self): print("Meowing")

class Pet(Dog, Cat):
    def play(self): print("Playing")

# Usage
p = Pet("Buddy")
p.breathe()  # From Mammal
p.bark()     # From Dog
p.meow()     # From Cat
p.play()     # Own
\end{lstlisting}
\end{solutionbox}

\end{document}
