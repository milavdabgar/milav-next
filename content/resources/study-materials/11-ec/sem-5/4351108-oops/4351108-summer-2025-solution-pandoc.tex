\documentclass[10pt,a4paper]{article}

% content/resources/templates/preamble.tex
\usepackage[margin=0.6in]{geometry}
\author{Milav Dabgar}
\usepackage{amsmath,amssymb,amsthm}
\usepackage{booktabs}
\usepackage{multirow}
\usepackage{xcolor}
\usepackage{tcolorbox}
\tcbuselibrary{breakable,skins}
\usepackage[colorlinks=true,linkcolor=blue]{hyperref}
\usepackage{titlesec}
\usepackage{enumitem}
\usepackage{tikz}
\usepackage{pgfplots}
\usepackage{circuitikz}
\usepackage[version=4]{mhchem}
\usepackage{longtable}
\usepackage{array}
\usepackage{float}
\usepackage{caption}
\usepackage{listings}

\lstset{
  basicstyle=\small\ttfamily,
  breaklines=true,
  breakatwhitespace=false,
  postbreak=\mbox{\textcolor{red}{$\hookrightarrow$}\space},
  float=false,
  numbers=left,
  numberstyle=\tiny\color{gray},
  numbersep=10pt,
  xleftmargin=2em,
  keywordstyle=\color{blue},
  commentstyle=\color{green!60!black},
  stringstyle=\color{purple},
  backgroundcolor=\color{gray!5},
  showstringspaces=false,
  tabsize=2,
  captionpos=b,
  keepspaces=true,
  columns=flexible
}

\pgfplotsset{compat=1.18}
\usetikzlibrary{shapes,arrows,positioning,calc,patterns,decorations.pathmorphing,decorations.markings,arrows.meta}

% Color scheme
\definecolor{headcolor}{RGB}{0,102,204}
\definecolor{keycolor}{RGB}{220,20,60}
\definecolor{solutioncolor}{RGB}{34,139,34}
\definecolor{mnemoniccolor}{RGB}{148,0,211}
\definecolor{codecolor}{RGB}{0,0,100}

% Spacing
\setlength{\parskip}{3pt}
\setlist[itemize]{nosep}
\setlist[enumerate]{nosep}

% Title formatting
\titleformat{\section}{\Large\bfseries\color{headcolor}}{\thesection}{1em}{}
\titleformat{\subsection}{\large\bfseries\color{headcolor}}{\thesubsection}{1em}{}

% Pandoc tightlist compatibility
\providecommand{\tightlist}{%
  \setlength{\itemsep}{0pt}\setlength{\parskip}{0pt}}

% Pandoc longtable compatibility
\newcounter{none}
\def\thenone{}


% content/resources/templates/english-boxes.tex
% This file is currently empty - it exists to maintain consistency with the import structure.
% Add custom environments here if needed in the future.


\begin{document}

\begin{center}
{\Huge\bfseries\color{headcolor} Subject Name Solutions}\\[5pt]
{\LARGE 4351108 -- Summer 2025}\\[3pt]
{\large Semester 1 Study Material}\\[3pt]
{\normalsize\textit{Detailed Solutions and Explanations}}
\end{center}

\vspace{10pt}

\subsection*{Question 1(a) [3 marks]}\label{q1a}

\textbf{What is the purpose of a for loop in Python? Write an example.}

\begin{solutionbox}
A for loop is used to iterate over a sequence (like
list, tuple, string) or other iterable objects and execute a block of
code for each item in the sequence.

\textbf{Code Example:}

\begin{verbatim}
\# Print each fruit in a list
fruits = ["apple", "banana", "cherry"]
for fruit in fruits:
    print(fruit)
\end{verbatim}

\begin{itemize}
\tightlist
\item
  \textbf{Iteration}: Automatically repeats code for each item
\item
  \textbf{Simplicity}: Cleaner than using while loops with counters
\end{itemize}

\end{solutionbox}
\begin{mnemonicbox}
``For Each Item Do''

\end{mnemonicbox}
\subsection*{Question 1(b) [4 marks]}\label{q1b}

\textbf{List out rules for defining variables in python and list out
data types in python.}

\begin{solutionbox}

\textbf{Rules for defining variables:}

{\def\LTcaptype{none} % do not increment counter
\begin{longtable}[]{@{}
  >{\raggedright\arraybackslash}p{(\linewidth - 4\tabcolsep) * \real{0.1935}}
  >{\raggedright\arraybackslash}p{(\linewidth - 4\tabcolsep) * \real{0.2903}}
  >{\raggedright\arraybackslash}p{(\linewidth - 4\tabcolsep) * \real{0.5161}}@{}}
\toprule\noalign{}
\begin{minipage}[b]{\linewidth}\raggedright
Rule
\end{minipage} & \begin{minipage}[b]{\linewidth}\raggedright
Example
\end{minipage} & \begin{minipage}[b]{\linewidth}\raggedright
Invalid Example
\end{minipage} \\
\midrule\noalign{}
\endhead
\bottomrule\noalign{}
\endlastfoot
Must start with letter or underscore & \texttt{name\ =\ "John"} &
\texttt{1name\ =\ "John"} \\
Can contain letters, numbers, underscores & \texttt{user\_1\ =\ "Alice"}
& \texttt{user-1\ =\ "Alice"} \\
Case-sensitive & \texttt{age} and \texttt{Age} are different & \\
Cannot use reserved keywords & \texttt{count\ =\ 5} &
\texttt{if\ =\ 5} \\
\end{longtable}
}

\textbf{Python Data Types:}

{\def\LTcaptype{none} % do not increment counter
\begin{longtable}[]{@{}
  >{\raggedright\arraybackslash}p{(\linewidth - 4\tabcolsep) * \real{0.3333}}
  >{\raggedright\arraybackslash}p{(\linewidth - 4\tabcolsep) * \real{0.3939}}
  >{\raggedright\arraybackslash}p{(\linewidth - 4\tabcolsep) * \real{0.2727}}@{}}
\toprule\noalign{}
\begin{minipage}[b]{\linewidth}\raggedright
Data Type
\end{minipage} & \begin{minipage}[b]{\linewidth}\raggedright
Description
\end{minipage} & \begin{minipage}[b]{\linewidth}\raggedright
Example
\end{minipage} \\
\midrule\noalign{}
\endhead
\bottomrule\noalign{}
\endlastfoot
int & Integer numbers & \texttt{x\ =\ 10} \\
float & Decimal numbers & \texttt{y\ =\ 10.5} \\
str & Text strings & \texttt{name\ =\ "John"} \\
bool & Boolean values & \texttt{is\_active\ =\ True} \\
list & Ordered, changeable collection &
\texttt{fruits\ =\ ["apple",\ "banana"]} \\
tuple & Ordered, unchangeable collection &
\texttt{coordinates\ =\ (10,\ 20)} \\
dict & Key-value pairs &
\texttt{person\ =\ \{"name":\ "John",\ "age":\ 30\}} \\
set & Unordered collection of unique items &
\texttt{numbers\ =\ \{1,\ 2,\ 3\}} \\
\end{longtable}
}

\begin{itemize}
\tightlist
\item
  \textbf{Variable rules}: Make them descriptive and meaningful
\item
  \textbf{Data types}: Python automatically determines the type
\end{itemize}

\end{solutionbox}
\begin{mnemonicbox}
``SILB-DTS'' (String, Integer, List, Boolean,
Dictionary, Tuple, Set)

\end{mnemonicbox}
\subsection*{Question 1(c) [7 marks]}\label{q1c}

\textbf{Create a program to print prime numbers between 1 to N.}

\begin{solutionbox}

\begin{verbatim}
def print\_primes(n):
    print("Prime numbers between 1 and", n, "are:")
    
    for num in range(2, n + 1):
        is\_prime = True
        
        \# Check if num is divisible by any number from 2 to sqrt(num)
        for i in range(2, int(num**0.5) + 1):
if num \%

i == 0:

                is\_prime = False
                break
                
        if is\_prime:
            print(num, end=" ")

\# Get input from user
N = int(input("Enter a number N: "))
print\_primes(N)
\end{verbatim}

\textbf{Algorithm Diagram:}

\begin{verbatim}
flowchart LR
    A[Start] {-{-} B[Input N]}
    B {-{-} C[Initialize num = 2]}
    C {-{-} D\{Is num = N?\}}
    D {-{-}|Yes| E[Assume num is prime]}
    D {-{-}|No| L[End]}
    E {-{-} F[Set i = 2]}
    F {-{-} G\{"Is i = sqrt(num)?"\}}
    G {-{-}|Yes| H\{Is num divisible by i?\}}
    G {-{-}|No| J[Print num]}
    H {-{-}|Yes| I[num is not prime]}
    H {-{-}|No| K[Increment i]}
    K {-{-} G}
    I {-{-} M[Increment num]}
    J {-{-} M}
    M {-{-} D}
\end{verbatim}

\begin{itemize}
\tightlist
\item
  \textbf{Time complexity}: O(N\sqrtN) - Optimized with square root approach
\item
  \textbf{Space complexity}: O(1) - Only uses constant space
\end{itemize}

\end{solutionbox}
\begin{mnemonicbox}
``Divide To Decide Prime''

\end{mnemonicbox}
\subsection*{Question 1(c) OR [7
marks]}\label{q1c}

\textbf{Explain working of break, continue and pass statement in Python
with examples.}

\begin{solutionbox}

{\def\LTcaptype{none} % do not increment counter
\begin{longtable}[]{@{}
  >{\raggedright\arraybackslash}p{(\linewidth - 4\tabcolsep) * \real{0.3793}}
  >{\raggedright\arraybackslash}p{(\linewidth - 4\tabcolsep) * \real{0.3103}}
  >{\raggedright\arraybackslash}p{(\linewidth - 4\tabcolsep) * \real{0.3103}}@{}}
\toprule\noalign{}
\begin{minipage}[b]{\linewidth}\raggedright
Statement
\end{minipage} & \begin{minipage}[b]{\linewidth}\raggedright
Purpose
\end{minipage} & \begin{minipage}[b]{\linewidth}\raggedright
Example
\end{minipage} \\
\midrule\noalign{}
\endhead
\bottomrule\noalign{}
\endlastfoot
break & Terminates the loop completely & Stop loop when condition met \\
continue & Skips current iteration, continues with next & Skip specific
items \\
pass & Null operation, does nothing & Placeholder for future code \\
\end{longtable}
}

\textbf{1. break statement:}

\begin{verbatim}
\# Exit loop when finding number 5
for num in range(1, 10):
    if num == 5:
        print("Found 5, breaking loop")
        break
    print(num)
\# Output: 1 2 3 4 Found 5, breaking loop
\end{verbatim}

\textbf{2. continue statement:}

\begin{verbatim}
\# Skip even numbers
for num in range(1, 6):
    if num \% 2 == 0:
        continue
    print(num)
\# Output: 1 3 5
\end{verbatim}

\textbf{3. pass statement:}

\begin{verbatim}
\# Empty function implementation
def my\_function():
    pass

\# Empty conditional block
x = 10
if x {} 5:
    pass  \# will implement later
\end{verbatim}

\textbf{Flow Control Diagram:}

\begin{verbatim}
flowchart LR
    A[Loop Start] {-{-} B\{Condition\}}
    B {-{-}|True| C[Process]}
    C {-{-} D\{break?\}}
    D {-{-}|Yes| E[Exit Loop]}
    D {-{-}|No| F\{continue?\}}
    F {-{-}|Yes| G[Skip to Next Iteration]}
    F {-{-}|No| H\{pass?\}}
    H {-{-}|Yes| I[Do Nothing]}
    H {-{-}|No| J[Execute Code]}
    I {-{-} B}
    J {-{-} B}
    G {-{-} B}
\end{verbatim}

\begin{itemize}
\tightlist
\item
  \textbf{break}: Exits completely from the loop
\item
  \textbf{continue}: Jumps to the next iteration
\item
  \textbf{pass}: Does nothing, placeholder for future code
\end{itemize}

\end{solutionbox}
\begin{mnemonicbox}
``BCP - Break Completely, Continue Partially, Pass
silently''

\end{mnemonicbox}
\subsection*{Question 2(a) [3 marks]}\label{q2a}

\textbf{Create a program that asks the user for a year and prints out
whether it is a leap year or not.}

\begin{solutionbox}

\begin{verbatim}
def is\_leap\_year(year):
    \# A leap year is divisible by 4
    \# But if it{s divisible by 100, it must also be divisible by 400}
    if (year \% 4 == 0 and year \% 100 != 0) or (year \% 400 == 0):
        return True
    else:
        return False

\# Get input from user
year = int(input("Enter a year: "))

\# Check if it{s a leap year}
if is\_leap\_year(year):
    print(f"\{year\} is a leap year")
else:
    print(f"\{year\} is not a leap year")
\end{verbatim}

\textbf{Decision Tree:}

\begin{verbatim}
flowchart LR
    A[Start] {-{-} B[Input year]}
    B {-{-} C\{year \% 4 == 0?\}}
    C {-{-}|Yes| D\{year \% 100 == 0?\}}
    C {-{-}|No| E[Not a Leap Year]}
    D {-{-}|Yes| F\{year \% 400 == 0?\}}
    D {-{-}|No| G[Leap Year]}
    F {-{-}|Yes| G}
    F {-{-}|No| E}
\end{verbatim}

\begin{itemize}
\tightlist
\item
  \textbf{Rule 1}: Divisible by 4, not by 100
\item
  \textbf{Rule 2}: Or divisible by 400
\end{itemize}

\end{solutionbox}
\begin{mnemonicbox}
``4 Yes, 100 No, 400 Yes''

\end{mnemonicbox}
\subsection*{Question 2(b) [4 marks]}\label{q2b}

\textbf{What are the key differences between a list and a tuple in
Python?}

\begin{solutionbox}

{\def\LTcaptype{none} % do not increment counter
\begin{longtable}[]{@{}
  >{\raggedright\arraybackslash}p{(\linewidth - 4\tabcolsep) * \real{0.4091}}
  >{\raggedright\arraybackslash}p{(\linewidth - 4\tabcolsep) * \real{0.2727}}
  >{\raggedright\arraybackslash}p{(\linewidth - 4\tabcolsep) * \real{0.3182}}@{}}
\toprule\noalign{}
\begin{minipage}[b]{\linewidth}\raggedright
Feature
\end{minipage} & \begin{minipage}[b]{\linewidth}\raggedright
List
\end{minipage} & \begin{minipage}[b]{\linewidth}\raggedright
Tuple
\end{minipage} \\
\midrule\noalign{}
\endhead
\bottomrule\noalign{}
\endlastfoot
Syntax & Created using \texttt{[]} & Created using \texttt{()} \\
Mutability & Mutable (can be changed) & Immutable (cannot be changed) \\
Methods & Many methods (append, remove, etc.) & Limited methods (count,
index) \\
Performance & Slower & Faster \\
Use Case & When modification needed & When data shouldn't change \\
Memory & Uses more memory & Uses less memory \\
\end{longtable}
}

\textbf{Comparison Diagram:}

\begin{center}
\textbf{Mermaid Diagram (Code)}
\begin{verbatim}
{Shaded}
{Highlighting}[]
graph TD
    subgraph List
    A["fruits = [{apple{}, {}banana{}]"] {-}{-}{} B["fruits.append({}orange{})"]}
    end
    subgraph Tuple
    C["coordinates = (10, 20)"] {-{-}{} D[Cannot modify elements]}
    end
{Highlighting}
{Shaded}
\end{verbatim}
\end{center}

\begin{itemize}
\tightlist
\item
  \textbf{Lists}: When you need to modify the collection
\item
  \textbf{Tuples}: When you need immutable data (faster, safer)
\end{itemize}

\end{solutionbox}
\begin{mnemonicbox}
``LIST - Lets Items Stay Transformable, TUPLE -
Totally Unchangeable Permanent List Elements''

\end{mnemonicbox}
\subsection*{Question 2(c) [7 marks]}\label{q2c}

\textbf{Create a program to find the sum of all the positive numbers
entered by the user. As soon as the user enters a negative number, stop
taking in any further input from the user and display the sum.}

\begin{solutionbox}

\begin{verbatim}
def sum\_positives():
    total\_sum = 0
    
    while True:
        num = float(input("Enter a number (negative to stop): "))
        
        \# Check if number is negative
        if num {} 0:
            break
            
        \# Add positive number to total
        total\_sum += num
    
    print(f"Sum of all positive numbers: \{total\_sum\}")

\# Run the function
sum\_positives()
\end{verbatim}

\textbf{Process Flow:}

\begin{verbatim}
flowchart LR
    A[Start] {-{-} B[Initialize total\_sum = 0]}
    B {-{-} C[Input number]}
    C {-{-} D\{Is number  0?\}}
    D {-{-}|Yes| E[Display sum]}
    D {-{-}|No| F[Add to total\_sum]}
    F {-{-} C}
    E {-{-} G[End]}
\end{verbatim}

\begin{itemize}
\tightlist
\item
  \textbf{Loop control}: Terminates on negative input
\item
  \textbf{Accumulator}: Adds each positive number to running total
\end{itemize}

\end{solutionbox}
\begin{mnemonicbox}
``Sum Till Negative''

\end{mnemonicbox}
\subsection*{Question 2(a) OR [3
marks]}\label{q2a}

\textbf{Create a program to find a maximum number among the given three
numbers.}

\begin{solutionbox}

\begin{verbatim}
\# Get three numbers from user
num1 = float(input("Enter first number: "))
num2 = float(input("Enter second number: "))
num3 = float(input("Enter third number: "))

\# Find maximum using if{-else}
if num1 {=} num2 and num1 {=} num3:
    maximum = num1
elif num2 {=} num1 and num2 {=} num3:
    maximum = num2
else:
    maximum = num3

print(f"Maximum number is: \{maximum\}")

\# Alternative using built{-in max() function}
\# maximum = max(num1, num2, num3)
\# print(f"Maximum number is: \{maximum\")}
\end{verbatim}

\textbf{Comparison Logic:}

\begin{verbatim}
flowchart LR
    A[Start] {-{-} B[Input num1, num2, num3]}
    B {-{-} C\{num1 = num2 AND num1 = num3?\}}
    C {-{-}|Yes| D[maximum = num1]}
    C {-{-}|No| E\{num2 = num1 AND num2 = num3?\}}
    E {-{-}|Yes| F[maximum = num2]}
    E {-{-}|No| G[maximum = num3]}
    D {-{-} H[Display maximum]}
    F {-{-} H}
    G {-{-} H}
    H {-{-} I[End]}
\end{verbatim}

\begin{itemize}
\tightlist
\item
  \textbf{Comparison}: Uses logical operators to find maximum
\item
  \textbf{Alternative}: Built-in max() function for simplicity
\end{itemize}

\end{solutionbox}
\begin{mnemonicbox}
``Compare Each, Take Largest''

\end{mnemonicbox}
\subsection*{Question 2(b) OR [4
marks]}\label{q2b}

\textbf{Given the str=``abcdefghijklmnopqrstuvwxyz''. Write a python
program to extract every second character from above string.}

\begin{solutionbox}

\begin{verbatim}
\# Given string
str = "abcdefghijklmnopqrstuvwxyz"

\# Extract every second character using slicing
\# The syntax is [start:end:step]
\# start=0 (beginning), end=len(str) (end of string), step=2 (every second character)
result = str[0::2]

print("Original string:", str)
print("Every second character:", result)
\# Output: Every second character: acegikmoqsuwy
\end{verbatim}

\textbf{String Slicing Diagram:}

\begin{verbatim}
+---+---+---+---+---+---+---+---+---+---+---+
| a | b | c | d | e | f | g | h | i | j | k |...
+---+---+---+---+---+---+---+---+---+---+---+
  ^       ^       ^       ^       ^
  |       |       |       |       |
  0       2       4       6       8   (indices)
\end{verbatim}

\begin{itemize}
\tightlist
\item
  \textbf{String slicing}: [start:end:step] syntax
\item
  \textbf{Step value}: 2 selects every second character
\end{itemize}

\end{solutionbox}
\begin{mnemonicbox}
``Slice Step Selector''

\end{mnemonicbox}
\subsection*{Question 2(c) OR [7
marks]}\label{q2c}

\textbf{Write a Python program to create a dictionary that stores
student names and their marks. Display the names of students who have
scored more than 75 marks.}

\begin{solutionbox}

\begin{verbatim}
def high\_scorers():
    \# Create empty dictionary
    students = \{\}
    
    \# Get number of students
    n = int(input("Enter number of students: "))
    
    \# Input student data
    for i in range(n):
        name = input(f"Enter name of student \{i+1\}: ")
        marks = float(input(f"Enter marks of student \{i+1\}: "))
        students[name] = marks
    
    \# Display dictionary
    print("{n}Student Records:", students)
    
    \# Display high scorers
    print("{n}Students who scored more than 75 marks:")
    for name, marks in students.items():
        if marks {} 75:
            print(f"\{name\}: \{marks\}")

\# Run the function
high\_scorers()
\end{verbatim}

\textbf{Process Diagram:}

\begin{verbatim}
flowchart TD
    A[Start] {-{-} B[Create empty dictionary]}
    B {-{-} C[Input n students]}
    C {-{-} D[Loop through n times]}
    D {-{-} E[Input name and marks]}
    E {-{-} F[Add to dictionary]}
    F {-{-} D}
    D {-{-} G[Display all records]}
    G {-{-} H[Loop through dictionary]}
    H {-{-} I\{marks  75?\}}
    I {-{-}|Yes| J[Display name]}
    I {-{-}|No| K[Skip]}
    J {-{-} H}
    K {-{-} H}
    H {-{-} L[End]}
\end{verbatim}

\begin{itemize}
\tightlist
\item
  \textbf{Dictionary}: Key-value pairs of student names and marks
\item
  \textbf{Conditional filtering}: Selects high scorers (\textgreater75)
\end{itemize}

\end{solutionbox}
\begin{mnemonicbox}
``Store All, Filter Some''

\end{mnemonicbox}
\subsection*{Question 3(a) [3 marks]}\label{q3a}

\textbf{Write a program to find the length of a string excluding
spaces.}

\begin{solutionbox}

\begin{verbatim}
def length\_without\_spaces():
    \# Get input string
    input\_string = input("Enter a string: ")
    
    \# Remove spaces and calculate length
    \# Method 1: Using replace
    no\_spaces = input\_string.replace(" ", "")
    length = len(no\_spaces)
    
    \# Method 2: Using a counter
    \# count = 0
    \# for char in input\_string:
    \#     if char != " ":
    \#         count += 1
    
    print(f"Original string: {}\{input\_string\}{"})
    print(f"Length excluding spaces: \{length\}")

\# Run the function
length\_without\_spaces()
\end{verbatim}

\textbf{String Processing:}

\begin{verbatim}
"Hello World" \rightarrow "HelloWorld" \rightarrow Length: 10
\end{verbatim}

\begin{itemize}
\tightlist
\item
  \textbf{Space removal}: Using replace() or filtering
\item
  \textbf{String length}: Calculated after space removal
\end{itemize}

\end{solutionbox}
\begin{mnemonicbox}
``Count Characters, Skip Spaces''

\end{mnemonicbox}
\subsection*{Question 3(b) [4 marks]}\label{q3b}

\textbf{List the dictionary methods in python and explain each with
suitable examples.}

\begin{solutionbox}

{\def\LTcaptype{none} % do not increment counter
\begin{longtable}[]{@{}lll@{}}
\toprule\noalign{}
Method & Description & Example \\
\midrule\noalign{}
\endhead
\bottomrule\noalign{}
\endlastfoot
\texttt{clear()} & Removes all items & \texttt{dict.clear()} \\
\texttt{copy()} & Returns a shallow copy &
\texttt{new\_dict\ =\ dict.copy()} \\
\texttt{get()} & Returns value for key &
\texttt{value\ =\ dict.get(\textquotesingle{}key\textquotesingle{},\ default)} \\
\texttt{items()} & Returns key-value pairs &
\texttt{for\ k,\ v\ in\ dict.items():} \\
\texttt{keys()} & Returns all keys &
\texttt{for\ k\ in\ dict.keys():} \\
\texttt{values()} & Returns all values &
\texttt{for\ v\ in\ dict.values():} \\
\texttt{pop()} & Removes item with key &
\texttt{value\ =\ dict.pop(\textquotesingle{}key\textquotesingle{})} \\
\texttt{update()} & Updates dictionary &
\texttt{dict.update(\{\textquotesingle{}key\textquotesingle{}:\ value\})} \\
\end{longtable}
}

\textbf{Code Example:}

\begin{verbatim}
student = \{{name}: {John}, {age}: 20, {grade}: {A}\}

\# get method
print(student.get({name}))  \# Output: John
print(student.get({city}, {Not found}))  \# Output: Not found

\# update method
student.update(\{{city}: {New York}, {grade}: {A+}\)}
print(student)  \# \{{name: John, age: 20, grade: A+, city: New York\}}

\# pop method
removed = student.pop({age})
print(removed)  \# 20
print(student)  \# \{{name: John, grade: A+, city: New York\}}
\end{verbatim}

\begin{itemize}
\tightlist
\item
  \textbf{Access methods}: get(), keys(), values(), items()
\item
  \textbf{Modification methods}: update(), pop(), clear()
\end{itemize}

\end{solutionbox}
\begin{mnemonicbox}
``GCUP-KPIV'' (Get-Copy-Update-Pop,
Keys-Pop-Items-Values)

\end{mnemonicbox}
\subsection*{Question 3(c) [7 marks]}\label{q3c}

\textbf{Explain Python's List data type in detail.}

\begin{solutionbox}

\textbf{Python List}: An ordered, mutable collection that can store
items of different data types.

{\def\LTcaptype{none} % do not increment counter
\begin{longtable}[]{@{}
  >{\raggedright\arraybackslash}p{(\linewidth - 4\tabcolsep) * \real{0.2903}}
  >{\raggedright\arraybackslash}p{(\linewidth - 4\tabcolsep) * \real{0.4194}}
  >{\raggedright\arraybackslash}p{(\linewidth - 4\tabcolsep) * \real{0.2903}}@{}}
\toprule\noalign{}
\begin{minipage}[b]{\linewidth}\raggedright
Feature
\end{minipage} & \begin{minipage}[b]{\linewidth}\raggedright
Description
\end{minipage} & \begin{minipage}[b]{\linewidth}\raggedright
Example
\end{minipage} \\
\midrule\noalign{}
\endhead
\bottomrule\noalign{}
\endlastfoot
Creation & Using square brackets &
\texttt{my\_list\ =\ [1,\ \textquotesingle{}hello\textquotesingle{},\ True]} \\
Indexing & Zero-based, negative indices & \texttt{my\_list[0]},
\texttt{my\_list[-1]} \\
Slicing & Extract parts & \texttt{my\_list[1:3]} \\
Mutability & Can be modified & \texttt{my\_list[0]\ =\ 10} \\
Methods & Many built-in methods & \texttt{append()}, \texttt{insert()},
\texttt{remove()} \\
Nesting & Lists within lists &
\texttt{nested\ =\ [[1,\ 2],\ [3,\ 4]]} \\
\end{longtable}
}

\textbf{Common List Methods:}

{\def\LTcaptype{none} % do not increment counter
\begin{longtable}[]{@{}lll@{}}
\toprule\noalign{}
Method & Purpose & Example \\
\midrule\noalign{}
\endhead
\bottomrule\noalign{}
\endlastfoot
\texttt{append()} & Add item to end & \texttt{my\_list.append(5)} \\
\texttt{insert()} & Add at position &
\texttt{my\_list.insert(1,\ \textquotesingle{}new\textquotesingle{})} \\
\texttt{remove()} & Remove by value &
\texttt{my\_list.remove(\textquotesingle{}hello\textquotesingle{})} \\
\texttt{pop()} & Remove by index & \texttt{my\_list.pop(2)} \\
\texttt{sort()} & Sort list & \texttt{my\_list.sort()} \\
\texttt{reverse()} & Reverse order & \texttt{my\_list.reverse()} \\
\end{longtable}
}

\textbf{List Operations Diagram:}

\begin{center}
\textbf{Mermaid Diagram (Code)}
\begin{verbatim}
{Shaded}
{Highlighting}[]
graph LR
    A["fruits = [{apple{}, {}banana{}]"] {-}{-}{} B["fruits.append({}orange{})"]}
    B {-{-}{} C["fruits.insert(1, {}mango{})"]}
    C {-{-}{} D["fruits.pop(0)"]}
    D {-{-}{} E["fruits.sort()"]}
    E {-{-}{} F["[{}mango{}, {}orange{}]"]}
{Highlighting}
{Shaded}
\end{verbatim}
\end{center}

\begin{itemize}
\tightlist
\item
  \textbf{Versatility}: Stores different data types in one collection
\item
  \textbf{Dynamic sizing}: Grows or shrinks as needed
\end{itemize}

\end{solutionbox}
\begin{mnemonicbox}
``CAMP-IS'' (Create, Access, Modify, Process, Index,
Slice)

\end{mnemonicbox}
\subsection*{Question 3(a) OR [3
marks]}\label{q3a}

\textbf{Write a program to input a string from the user and print it in
the reverse order without creating a new string.}

\begin{solutionbox}

\begin{verbatim}
def reverse\_string():
    \# Get input string
    input\_string = input("Enter a string: ")
    
    \# Print original string
    print(f"Original string: \{input\_string\}")
    
    \# Print reversed string using slice notation
    \# The syntax is [start:end:step]
    \# start=None (default), end=None (default), step={-1 (reverse)}
    print(f"Reversed string: \{input\_string[::{-}1]\}")

\# Run the function
reverse\_string()
\end{verbatim}

\textbf{String Reversing Visualization:}

\begin{verbatim}
"Hello" \rightarrow "olleH"

Indices:  0   1   2   3   4
String:   H   e   l   l   o
Reversed: o   l   l   e   H
Indices: -1  -2  -3  -4  -5
\end{verbatim}

\begin{itemize}
\tightlist
\item
  \textbf{Slicing with negative step}: Reverses without new string
\item
  \textbf{Efficient}: No extra memory used for new string
\end{itemize}

\end{solutionbox}
\begin{mnemonicbox}
``Slice Backwards''

\end{mnemonicbox}
\subsection*{Question 3(b) OR [4
marks]}\label{q3b}

\textbf{List the dictionary operations in python and explain each with
suitable examples.}

\begin{solutionbox}

{\def\LTcaptype{none} % do not increment counter
\begin{longtable}[]{@{}lll@{}}
\toprule\noalign{}
Operation & Description & Example \\
\midrule\noalign{}
\endhead
\bottomrule\noalign{}
\endlastfoot
Creation & Create a new dictionary &
\texttt{d\ =\ \{\textquotesingle{}key\textquotesingle{}:\ \textquotesingle{}value\textquotesingle{}\}} \\
Access & Access by key &
\texttt{value\ =\ d[\textquotesingle{}key\textquotesingle{}]} \\
Assignment & Add or update items &
\texttt{d[\textquotesingle{}new\_key\textquotesingle{}]\ =\ \textquotesingle{}new\_value\textquotesingle{}} \\
Deletion & Remove items &
\texttt{del\ d[\textquotesingle{}key\textquotesingle{}]} \\
Membership & Check if key exists &
\texttt{if\ \textquotesingle{}key\textquotesingle{}\ in\ d:} \\
Length & Count items & \texttt{len(d)} \\
Iteration & Loop through items & \texttt{for\ key\ in\ d:} \\
Comprehension & Create new dict &
\texttt{\{x:\ x**2\ for\ x\ in\ range(5)\}} \\
\end{longtable}
}

\textbf{Code Example:}

\begin{verbatim}
\# Creation
student = \{{name}: {John}, {age}: 20\}

\# Access
print(student[{name}])  \# Output: John

\# Assignment
student[{grade}] = {A}  \# Add new key{-value pair}
student[{age}] = 21     \# Update existing value

\# Membership test
if {grade} in student:
    print("Grade exists")  \# Will be printed

\# Deletion
del student[{age}]
print(student)  \# \{{name: John, grade: A\}}

\# Dictionary comprehension
squares = \{x: x**2 for x in range(1, 5)\}
print(squares)  \# \{1: 1, 2: 4, 3: 9, 4: 16\}
\end{verbatim}

\begin{itemize}
\tightlist
\item
  \textbf{Key-based access}: Fast lookup by keys
\item
  \textbf{Dynamic structure}: Add/remove items as needed
\end{itemize}

\end{solutionbox}
\begin{mnemonicbox}
``CADMIL'' (Create, Access, Delete, Modify, Iterate,
Length)

\end{mnemonicbox}
\subsection*{Question 3(c) OR [7
marks]}\label{q3c}

\textbf{Explain Python's set data type in detail.}

\begin{solutionbox}

\textbf{Python Set}: An unordered collection of unique, immutable items.

{\def\LTcaptype{none} % do not increment counter
\begin{longtable}[]{@{}
  >{\raggedright\arraybackslash}p{(\linewidth - 4\tabcolsep) * \real{0.2903}}
  >{\raggedright\arraybackslash}p{(\linewidth - 4\tabcolsep) * \real{0.4194}}
  >{\raggedright\arraybackslash}p{(\linewidth - 4\tabcolsep) * \real{0.2903}}@{}}
\toprule\noalign{}
\begin{minipage}[b]{\linewidth}\raggedright
Feature
\end{minipage} & \begin{minipage}[b]{\linewidth}\raggedright
Description
\end{minipage} & \begin{minipage}[b]{\linewidth}\raggedright
Example
\end{minipage} \\
\midrule\noalign{}
\endhead
\bottomrule\noalign{}
\endlastfoot
Creation & Using curly braces or set() &
\texttt{my\_set\ =\ \{1,\ 2,\ 3\}} or \texttt{set([1,\ 2,\ 3])} \\
Uniqueness & No duplicates allowed & \texttt{\{1,\ 2,\ 2,\ 3\}} becomes
\texttt{\{1,\ 2,\ 3\}} \\
Unordered & No indexing & Cannot use \texttt{my\_set[0]} \\
Mutability & Set itself is mutable, but elements must be immutable & Can
add/remove items \\
Math Operations & Set theory operations & union, intersection,
difference \\
Use Cases & Remove duplicates, membership testing & Fast lookups \\
\end{longtable}
}

\textbf{Common Set Operations:}

{\def\LTcaptype{none} % do not increment counter
\begin{longtable}[]{@{}
  >{\raggedright\arraybackslash}p{(\linewidth - 6\tabcolsep) * \real{0.2619}}
  >{\raggedright\arraybackslash}p{(\linewidth - 6\tabcolsep) * \real{0.2381}}
  >{\raggedright\arraybackslash}p{(\linewidth - 6\tabcolsep) * \real{0.1905}}
  >{\raggedright\arraybackslash}p{(\linewidth - 6\tabcolsep) * \real{0.3095}}@{}}
\toprule\noalign{}
\begin{minipage}[b]{\linewidth}\raggedright
Operation
\end{minipage} & \begin{minipage}[b]{\linewidth}\raggedright
Operator
\end{minipage} & \begin{minipage}[b]{\linewidth}\raggedright
Method
\end{minipage} & \begin{minipage}[b]{\linewidth}\raggedright
Description
\end{minipage} \\
\midrule\noalign{}
\endhead
\bottomrule\noalign{}
\endlastfoot
Union & \texttt{\textbackslash{}\textbar{}} & \texttt{union()} & All
elements from both sets \\
Intersection & \texttt{\&} & \texttt{intersection()} & Common
elements \\
Difference & \texttt{-} & \texttt{difference()} & Elements in first but
not second \\
Symmetric Difference & \texttt{\^{}} & \texttt{symmetric\_difference()}
& Elements in either but not both \\
\end{longtable}
}

\textbf{Set Operations Diagram:}

\begin{center}
\textbf{Mermaid Diagram (Code)}
\begin{verbatim}
{Shaded}
{Highlighting}[]
graph TD
    A["A = \{1, 2, 3\"] {-}{-}{} B["B = \{3, 4, 5\}"]}
    A {-{-}{} C["A | B = \{1, 2, 3, 4, 5\}"]}
    A {-{-}{} D["A \& B = \{3\}"]}
    A {-{-}{} E["A {-} B = \{1, 2\}"]}
    A {-{-}{} F["A \^{} B = \{1, 2, 4, 5\}"]}
{Highlighting}
{Shaded}
\end{verbatim}
\end{center}

\begin{itemize}
\tightlist
\item
  \textbf{Fast membership}: O(1) average time complexity
\item
  \textbf{Mathematical operations}: Set theory operations built-in
\end{itemize}

\end{solutionbox}
\begin{mnemonicbox}
``SUMO'' (Sets are Unique, Mutable, and Ordered-less)

\end{mnemonicbox}
\subsection*{Question 4(a) [3 marks]}\label{q4a}

\textbf{Explain statistics module with any three methods.}

\begin{solutionbox}

The statistics module provides functions for calculating mathematical
statistics of numeric data.

{\def\LTcaptype{none} % do not increment counter
\begin{longtable}[]{@{}
  >{\raggedright\arraybackslash}p{(\linewidth - 4\tabcolsep) * \real{0.2667}}
  >{\raggedright\arraybackslash}p{(\linewidth - 4\tabcolsep) * \real{0.4333}}
  >{\raggedright\arraybackslash}p{(\linewidth - 4\tabcolsep) * \real{0.3000}}@{}}
\toprule\noalign{}
\begin{minipage}[b]{\linewidth}\raggedright
Method
\end{minipage} & \begin{minipage}[b]{\linewidth}\raggedright
Description
\end{minipage} & \begin{minipage}[b]{\linewidth}\raggedright
Example
\end{minipage} \\
\midrule\noalign{}
\endhead
\bottomrule\noalign{}
\endlastfoot
\texttt{mean()} & Arithmetic average &
\texttt{statistics.mean([1,\ 2,\ 3,\ 4,\ 5])} returns 3.0 \\
\texttt{median()} & Middle value &
\texttt{statistics.median([1,\ 3,\ 5,\ 7,\ 9])} returns 5 \\
\texttt{mode()} & Most common value &
\texttt{statistics.mode([1,\ 2,\ 2,\ 3,\ 4])} returns 2 \\
\texttt{stdev()} & Standard deviation &
\texttt{statistics.stdev([1,\ 2,\ 3,\ 4,\ 5])} returns
1.58\ldots{} \\
\end{longtable}
}

\textbf{Code Example:}

\begin{verbatim}
import statistics

data = [2, 5, 7, 9, 12, 13, 14, 5]

\# Mean (average)
print("Mean:", statistics.mean(data))  \# Output: 8.375

\# Median (middle value)
print("Median:", statistics.median(data))  \# Output: 8.0

\# Mode (most frequent)
print("Mode:", statistics.mode(data))  \# Output: 5
\end{verbatim}

\begin{itemize}
\tightlist
\item
  \textbf{Data analysis}: Functions for statistical calculations
\item
  \textbf{Built-in module}: No external installation needed
\end{itemize}

\end{solutionbox}
\begin{mnemonicbox}
``MMM Stats'' (Mean, Median, Mode Statistics)

\end{mnemonicbox}
\subsection*{Question 4(b) [4 marks]}\label{q4b}

\textbf{Explain function of user define function and user defined module
in Python.}

\begin{solutionbox}

{\def\LTcaptype{none} % do not increment counter
\begin{longtable}[]{@{}
  >{\raggedright\arraybackslash}p{(\linewidth - 4\tabcolsep) * \real{0.1731}}
  >{\raggedright\arraybackslash}p{(\linewidth - 4\tabcolsep) * \real{0.4231}}
  >{\raggedright\arraybackslash}p{(\linewidth - 4\tabcolsep) * \real{0.4038}}@{}}
\toprule\noalign{}
\begin{minipage}[b]{\linewidth}\raggedright
Feature
\end{minipage} & \begin{minipage}[b]{\linewidth}\raggedright
User-defined Function
\end{minipage} & \begin{minipage}[b]{\linewidth}\raggedright
User-defined Module
\end{minipage} \\
\midrule\noalign{}
\endhead
\bottomrule\noalign{}
\endlastfoot
Definition & Block of reusable code & Python file with
functions/classes \\
Purpose & Code organization and reuse & Organizing related code \\
Creation & Using \texttt{def} keyword & Creating .py file \\
Usage & Call by function name & Import using \texttt{import}
statement \\
Scope & Local to function & Accessible after import \\
Benefits & Reduces redundancy & Promotes code organization \\
\end{longtable}
}

\textbf{User-defined Function Example:}

\begin{verbatim}
\# Function definition
def calculate\_area(length, width):
    """Calculate area of rectangle"""
    area = length * width
    return area

\# Function call
result = calculate\_area(5, 3)
print("Area:", result)  \# Output: 15
\end{verbatim}

\textbf{User-defined Module Example:}

\begin{verbatim}
\# File: geometry.py
def calculate\_area(length, width):
    return length * width

def calculate\_perimeter(length, width):
    return 2 * (length + width)

\# In another file
import geometry

area = geometry.calculate\_area(5, 3)
print("Area:", area)  \# Output: 15
\end{verbatim}

\textbf{Module Organization:}

\begin{center}
\textbf{Mermaid Diagram (Code)}
\begin{verbatim}
{Shaded}
{Highlighting}[]
graph LR
    A[Main Program] {-{-}{} B[import geometry]}
    B {-{-}{} C[geometry.py]}
    C {-{-}{} D[calculate\_area]}
    C {-{-}{} E[calculate\_perimeter]}
{Highlighting}
{Shaded}
\end{verbatim}
\end{center}

\begin{itemize}
\tightlist
\item
  \textbf{Function benefits}: Code reuse, modular design
\item
  \textbf{Module benefits}: Organized code, namespace separation
\end{itemize}

\end{solutionbox}
\begin{mnemonicbox}
``FIR-MID'' (Functions for Internal Reuse, Modules
for Inter-file Distribution)

\end{mnemonicbox}
\subsection*{Question 4(c) [7 marks]}\label{q4c}

\textbf{Write a Python code using user defined function to find the
factorial of a given number using recursion.}

\begin{solutionbox}

\begin{verbatim}
def factorial(n):
    """
    Calculate factorial of n using recursion
    n! = n * (n{-1)!}
    """
    \# Base case: factorial of 0 or 1 is 1
if

n == 0 or

n == 1:

        return 1
    
    \# Recursive case: n! = n * (n{-1)!}
    else:
        return n * factorial(n{-}1)

\# Get input from user
number = int(input("Enter a positive integer: "))

\# Check if input is valid
if number {} 0:
    print("Factorial is not defined for negative numbers.")
else:
    \# Calculate and display result
    result = factorial(number)
    print(f"Factorial of \{number\} is \{result\}")
\end{verbatim}

\textbf{Recursive Function Visualization:}

\begin{center}
\textbf{Mermaid Diagram (Code)}
\begin{verbatim}
{Shaded}
{Highlighting}[]
graph LR
    A["factorial(4)"] {-{-}{} B["4 * factorial(3)"]}
    B {-{-}{} C["4 * (3 * factorial(2))"]}
    C {-{-}{} D["4 * (3 * (2 * factorial(1)))"]}
    D {-{-}{} E["4 * (3 * (2 * 1))"]}
    E {-{-}{} F["4 * (3 * 2)"]}
    F {-{-}{} G["4 * 6"]}
    G {-{-}{} H["24"]}
{Highlighting}
{Shaded}
\end{verbatim}
\end{center}

\begin{itemize}
\tightlist
\item
  \textbf{Base case}: Stops recursion when n=0 or n=1
\item
  \textbf{Recursive case}: Breaks problem into smaller subproblems
\end{itemize}

\end{solutionbox}
\begin{mnemonicbox}
``Factorial = Number times (Number minus one)!''

\end{mnemonicbox}
\subsection*{Question 4(a) OR [3
marks]}\label{q4a}

\textbf{Explain math module with any three methods.}

\begin{solutionbox}

The math module provides access to mathematical functions defined by the
C standard.

{\def\LTcaptype{none} % do not increment counter
\begin{longtable}[]{@{}lll@{}}
\toprule\noalign{}
Method & Description & Example \\
\midrule\noalign{}
\endhead
\bottomrule\noalign{}
\endlastfoot
\texttt{math.sqrt()} & Square root & \texttt{math.sqrt(16)} returns
4.0 \\
\texttt{math.pow()} & Power function & \texttt{math.pow(2,\ 3)} returns
8.0 \\
\texttt{math.floor()} & Round down & \texttt{math.floor(4.7)} returns
4 \\
\texttt{math.ceil()} & Round up & \texttt{math.ceil(4.2)} returns 5 \\
\texttt{math.sin()} & Sine function & \texttt{math.sin(math.pi/2)}
returns 1.0 \\
\end{longtable}
}

\textbf{Code Example:}

\begin{verbatim}
import math

\# Square root
print("Square root of 25:", math.sqrt(25))  \# Output: 5.0

\# Power
print("2 raised to power 3:", math.pow(2, 3))  \# Output: 8.0

\# Constants
print("Value of pi:", math.pi)  \# Output: 3.141592653589793
\end{verbatim}

\begin{itemize}
\tightlist
\item
  \textbf{Mathematical operations}: Advanced math functions
\item
  \textbf{Constants}: Mathematical constants like pi and e
\end{itemize}

\end{solutionbox}
\begin{mnemonicbox}
``SPT Math'' (Square root, Power, Trigonometry in
Math module)

\end{mnemonicbox}
\subsection*{Question 4(b) OR [4
marks]}\label{q4b}

\textbf{Explain the concepts of global and local variables in Python.}

\begin{solutionbox}

{\def\LTcaptype{none} % do not increment counter
\begin{longtable}[]{@{}
  >{\raggedright\arraybackslash}p{(\linewidth - 6\tabcolsep) * \real{0.3415}}
  >{\raggedright\arraybackslash}p{(\linewidth - 6\tabcolsep) * \real{0.1707}}
  >{\raggedright\arraybackslash}p{(\linewidth - 6\tabcolsep) * \real{0.2927}}
  >{\raggedright\arraybackslash}p{(\linewidth - 6\tabcolsep) * \real{0.1951}}@{}}
\toprule\noalign{}
\begin{minipage}[b]{\linewidth}\raggedright
Variable Type
\end{minipage} & \begin{minipage}[b]{\linewidth}\raggedright
Scope
\end{minipage} & \begin{minipage}[b]{\linewidth}\raggedright
Definition
\end{minipage} & \begin{minipage}[b]{\linewidth}\raggedright
Access
\end{minipage} \\
\midrule\noalign{}
\endhead
\bottomrule\noalign{}
\endlastfoot
Local & Inside function & Defined within function & Only within the
function \\
Global & Entire program & Defined outside functions & Anywhere in the
program \\
\end{longtable}
}

\textbf{Example:}

\begin{verbatim}
\# Global variable
total = 0

def add\_numbers(a, b):
    \# Local variables
    result = a + b
    
    \# Accessing global variable
    global total
    total += result
    
    return result

\# Function call
sum\_result = add\_numbers(5, 3)
print("Sum:", sum\_result)  \# Output: 8
print("Total:", total)  \# Output: 8
\end{verbatim}

\textbf{Variable Scope Diagram:}

\begin{center}
\textbf{Mermaid Diagram (Code)}
\begin{verbatim}
{Shaded}
{Highlighting}[]
graph LR
    A[Global Scope] {-{-}{} B[total]}
    A {-{-}{} C[add\_numbers function]}
    C {-{-}{} D[Local Scope]}
    D {-{-}{} E[a, b, result]}
    D {-{-}{} F[global total]}
    F {-{-}{} B}
{Highlighting}
{Shaded}
\end{verbatim}
\end{center}

\begin{itemize}
\tightlist
\item
  \textbf{Global}: Accessible everywhere but needs \texttt{global}
  keyword to modify
\item
  \textbf{Local}: Limited to function scope, freed after function
  execution
\end{itemize}

\end{solutionbox}
\begin{mnemonicbox}
``GLOBAL Goes Everywhere, LOCAL Lives in Functions''

\end{mnemonicbox}
\subsection*{Question 4(c) OR [7
marks]}\label{q4c}

\textbf{Create code with user defined function to check if given string
is palindrome or not.}

\begin{solutionbox}

\begin{verbatim}
def is\_palindrome(text):
    """
    Check if a string is a palindrome.
    A palindrome reads the same forwards and backwards.
    """
    \# Remove spaces and convert to lowercase
    cleaned\_text = text.replace(" ", "").lower()
    
    \# Check if the string equals its reverse
    return cleaned\_text == cleaned\_text[::{-}1]

def check\_palindrome():
    \# Get input from user
    input\_string = input("Enter a string: ")
    
    \# Check if it{s a palindrome}
    if is\_palindrome(input\_string):
        print(f"{}\{input\_string\}{ is a palindrome!"})
    else:
        print(f"{}\{input\_string\}{ is not a palindrome."})
    
    \# Examples for reference
    print("{n}Examples of palindromes:")
    print("{radar "}, is\_palindrome("radar"))
    print("{level "}, is\_palindrome("level"))
    print("{A man a plan a canal Panama "}, is\_palindrome("A man a plan a canal Panama"))

\# Run the function
check\_palindrome()
\end{verbatim}

\textbf{Palindrome Testing Process:}

\begin{verbatim}
flowchart LR
    A[Start] {-{-} B[Input string]}
    B {-{-} C[Clean string: remove spaces, convert to lowercase]}
    C {-{-} D[Check if string equals its reverse]}
    D {-{-}|Yes| E[Return True]}
    D {-{-}|No| F[Return False]}
    E {-{-} G[Display result]}
    F {-{-} G}
    G {-{-} H[End]}
\end{verbatim}

\begin{itemize}
\tightlist
\item
  \textbf{String cleaning}: Removes spaces, converts to lowercase
\item
  \textbf{Comparison}: Checks against reversed string
\item
  \textbf{Example palindromes}: ``radar'', ``madam'', ``A man a plan a
  canal Panama''
\end{itemize}

\end{solutionbox}
\begin{mnemonicbox}
``Clean, Reverse, Compare''

\end{mnemonicbox}
\subsection*{Question 5(a) [3 marks]}\label{q5a}

\textbf{Define class and object with example.}

\begin{solutionbox}

\textbf{Class}: A blueprint for creating objects that defines attributes
and methods.

\textbf{Object}: An instance of a class with specific attribute values.

\textbf{Code Example:}

\begin{verbatim}
\# Class definition
class Dog:
    \# Class attribute
    species = "Canis familiaris"
    
    \# Constructor (initializes instance attributes)
    def \_\_init\_\_(self, name, age):
        self.name = name
        self.age = age
    
    \# Instance method
    def bark(self):
        return f"\{self.name\} says Woof!"

\# Creating objects (instances)
dog1 = Dog("Rex", 3)
dog2 = Dog("Buddy", 5)

\# Accessing attributes and methods
print(dog1.name)  \# Output: Rex
print(dog2.species)  \# Output: Canis familiaris
print(dog1.bark())  \# Output: Rex says Woof!
\end{verbatim}

\textbf{Class-Object Relationship:}

\begin{verbatim}
classDiagram
    class Dog \{
        +species: string
        +name: string
        +age: int
        +\_\_init\_\_(name, age)
        +bark()
    \}
    Dog {|{-}{-} dog1}
    Dog {|{-}{-} dog2}
    class dog1 \{
        name = "Rex"
        age = 3
    \}
    class dog2 \{
        name = "Buddy"
        age = 5
    \}
\end{verbatim}

\begin{itemize}
\tightlist
\item
  \textbf{Class}: Template with attributes and methods
\item
  \textbf{Object}: Concrete instance with specific values
\end{itemize}

\end{solutionbox}
\begin{mnemonicbox}
``CAMBO'' (Classes Are Molds, Build Objects)

\end{mnemonicbox}
\subsection*{Question 5(b) [4 marks]}\label{q5b}

\textbf{Classify constructor. Explain any one in detail.}

\begin{solutionbox}

{\def\LTcaptype{none} % do not increment counter
\begin{longtable}[]{@{}
  >{\raggedright\arraybackslash}p{(\linewidth - 4\tabcolsep) * \real{0.4146}}
  >{\raggedright\arraybackslash}p{(\linewidth - 4\tabcolsep) * \real{0.3171}}
  >{\raggedright\arraybackslash}p{(\linewidth - 4\tabcolsep) * \real{0.2683}}@{}}
\toprule\noalign{}
\begin{minipage}[b]{\linewidth}\raggedright
Constructor Type
\end{minipage} & \begin{minipage}[b]{\linewidth}\raggedright
Description
\end{minipage} & \begin{minipage}[b]{\linewidth}\raggedright
When Used
\end{minipage} \\
\midrule\noalign{}
\endhead
\bottomrule\noalign{}
\endlastfoot
Default constructor & Created by Python if none defined & Simple class
creation \\
Parameterized constructor & Takes parameters to initialize & Customized
object creation \\
Non-parameterized constructor & Takes no parameters & Basic
initialization \\
Copy constructor & Creates object from existing object & Object
duplication \\
\end{longtable}
}

\textbf{Parameterized Constructor Example:}

\begin{verbatim}
class Student:
    \# Parameterized constructor
    def \_\_init\_\_(self, name, roll\_no, marks):
        self.name = name
        self.roll\_no = roll\_no
        self.marks = marks
        
    def display(self):
        print(f"Name: \{self.name\}, Roll No: \{self.roll\_no\}, Marks: \{self.marks\}")

\# Creating objects with parameters
student1 = Student("Alice", 101, 85)
student2 = Student("Bob", 102, 78)

\# Displaying student information
student1.display()  \# Output: Name: Alice, Roll No: 101, Marks: 85
student2.display()  \# Output: Name: Bob, Roll No: 102, Marks: 78
\end{verbatim}

\textbf{Constructor Flow:}

\begin{verbatim}
flowchart LR
    A[Create Student object] {-{-} B[\_\_init\_\_ called]}
    B {-{-} C[Initialize name attribute]}
    C {-{-} D[Initialize roll\_no attribute]}
    D {-{-} E[Initialize marks attribute]}
    E {-{-} F[Object ready to use]}
\end{verbatim}

\begin{itemize}
\tightlist
\item
  \textbf{Purpose}: Initialize object attributes
\item
  \textbf{Self parameter}: Reference to the instance being created
\item
  \textbf{Automatic call}: Called when object is created
\end{itemize}

\end{solutionbox}
\begin{mnemonicbox}
``PICAN'' (Parameters Initialize Constructor And
Name)

\end{mnemonicbox}
\subsection*{Question 5(c) [7 marks]}\label{q5c}

\textbf{Develop and explain a python code to implement hierarchical
inheritance.}

\begin{solutionbox}

\begin{verbatim}
\# Base class
class Vehicle:
    def \_\_init\_\_(self, make, model, year):
        self.make = make
        self.model = model
        self.year = year
    
    def display\_info(self):
        return f"\{self.year\} \{self.make\} \{self.model\}"
    
    def start\_engine(self):
        return "Engine started!"

\# Derived class 1
class Car(Vehicle):
    def \_\_init\_\_(self, make, model, year, doors):
        \# Call parent class constructor
        super().\_\_init\_\_(make, model, year)
        self.doors = doors
    
    def drive(self):
        return "Car is being driven!"

\# Derived class 2
class Motorcycle(Vehicle):
    def \_\_init\_\_(self, make, model, year, has\_sidecar):
        \# Call parent class constructor
        super().\_\_init\_\_(make, model, year)
        self.has\_sidecar = has\_sidecar
    
    def wheelie(self):
        if not self.has\_sidecar:
            return "Performing wheelie!"
        else:
            return "Cannot perform wheelie with sidecar!"

\# Create objects
car = Car("Toyota", "Corolla", 2023, 4)
motorcycle = Motorcycle("Honda", "CBR", 2024, False)

\# Use methods from parent class
print(car.display\_info())  \# Output: 2023 Toyota Corolla
print(motorcycle.start\_engine())  \# Output: Engine started!

\# Use methods from specific classes
print(car.drive())  \# Output: Car is being driven!
print(motorcycle.wheelie())  \# Output: Performing wheelie!
\end{verbatim}

\textbf{Hierarchical Inheritance Diagram:}

\begin{verbatim}
classDiagram
    Vehicle {|{-}{-} Car}
    Vehicle {|{-}{-} Motorcycle}

    class Vehicle \{
        +make
        +model
        +year
        +\_\_init\_\_(make, model, year)
        +display\_info()
        +start\_engine()
    \}
    
    class Car \{
        +doors
        +\_\_init\_\_(make, model, year, doors)
        +drive()
    \}
    
    class Motorcycle \{
        +has\_sidecar
        +\_\_init\_\_(make, model, year, has\_sidecar)
        +wheelie()
    \}
\end{verbatim}

\begin{itemize}
\tightlist
\item
  \textbf{Base class}: Common attributes/methods for all vehicles
\item
  \textbf{Derived classes}: Specialized behaviors for specific vehicle
  types
\item
  \textbf{Method inheritance}: Child classes inherit parent class
  methods
\end{itemize}

\end{solutionbox}
\begin{mnemonicbox}
``Parents Share, Children Specialize''

\end{mnemonicbox}
\subsection*{Question 5(a) OR [3
marks]}\label{q5a}

\textbf{What is the \textbf{init} method in Python? Explain its purpose
with a suitable example.}

\begin{solutionbox}

The \texttt{\_\_init\_\_} method is a special method (constructor) in
Python classes that is automatically called when an object is created.

\textbf{Purpose:}

\begin{enumerate}
\tightlist
\item
  Initialize object attributes
\item
  Set up the initial state of the object
\item
  Execute code that must run when object is created
\end{enumerate}

\textbf{Example:}

\begin{verbatim}
class Rectangle:
    def \_\_init\_\_(self, length, width):
        \# Initialize attributes
        self.length = length
        self.width = width
        self.area = length * width  \# Calculated attribute
        
        \# Print confirmation message
        print(f"Rectangle created with dimensions \{length\}x\{width\}")
    
    def display(self):
        return f"Rectangle: \{self.length\}x\{self.width\}, Area: \{self.area\}"

\# Create rectangle objects
rect1 = Rectangle(5, 3)  \# \_\_init\_\_ called automatically
rect2 = Rectangle(10, 2)  \# \_\_init\_\_ called automatically

\# Display information
print(rect1.display())
print(rect2.display())
\end{verbatim}

\begin{itemize}
\tightlist
\item
  \textbf{Automatic execution}: Called when object is created
\item
  \textbf{Self parameter}: References the current instance
\item
  \textbf{Multiple parameters}: Can accept any number of arguments
\end{itemize}

\end{solutionbox}
\begin{mnemonicbox}
``ASAP'' (Attributes Set At Production)

\end{mnemonicbox}
\subsection*{Question 5(b) OR [4
marks]}\label{q5b}

\textbf{Classify methods in Python class. Explain any one in detail.}

\begin{solutionbox}

{\def\LTcaptype{none} % do not increment counter
\begin{longtable}[]{@{}
  >{\raggedright\arraybackslash}p{(\linewidth - 4\tabcolsep) * \real{0.3243}}
  >{\raggedright\arraybackslash}p{(\linewidth - 4\tabcolsep) * \real{0.3514}}
  >{\raggedright\arraybackslash}p{(\linewidth - 4\tabcolsep) * \real{0.3243}}@{}}
\toprule\noalign{}
\begin{minipage}[b]{\linewidth}\raggedright
Method Type
\end{minipage} & \begin{minipage}[b]{\linewidth}\raggedright
Description
\end{minipage} & \begin{minipage}[b]{\linewidth}\raggedright
Definition
\end{minipage} \\
\midrule\noalign{}
\endhead
\bottomrule\noalign{}
\endlastfoot
Instance Method & Operates on object instance & Regular method with
\texttt{self} \\
Class Method & Operates on class itself & Decorated with
\texttt{@classmethod} \\
Static Method & Doesn't need class or instance & Decorated with
\texttt{@staticmethod} \\
Magic/Dunder Method & Special built-in methods & Surrounded by double
underscores \\
\end{longtable}
}

\textbf{Instance Method Example:}

\begin{verbatim}
class Student:
    \# Class variable
    school = "ABC School"
    
    def \_\_init\_\_(self, name, age):
        \# Instance variables
        self.name = name
        self.age = age
    
    \# Instance method {- operates on instance}
    def display\_info(self):
        return f"Name: \{self.name\}, Age: \{self.age\}, School: \{self.school\}"
    
    \# Instance method with parameter
    def is\_eligible(self, min\_age):
        return self.age {=} min\_age

\# Create object
student = Student("John", 15)

\# Call instance methods
print(student.display\_info())  \# Output: Name: John, Age: 15, School: ABC School
print(student.is\_eligible(16))  \# Output: False
\end{verbatim}

\textbf{Method Classification:}

\begin{verbatim}
classDiagram
    class Student \{
        +name: string
        +age: int
        +school: string
        +\_\_init\_\_(name, age)
        +display\_info()
        +is\_eligible(min\_age)
        +@classmethod create\_from\_birth\_year(cls, name, birth\_year)
        +@staticmethod validate\_name(name)
    \}
\end{verbatim}

\begin{itemize}
\tightlist
\item
  \textbf{Instance methods}: Access and modify object state
\item
  \textbf{Self parameter}: Reference to the instance
\item
  \textbf{Object-specific}: Results depend on the instance state
\end{itemize}

\end{solutionbox}
\begin{mnemonicbox}
``SIAM'' (Self Is Always Mentioned in instance
methods)

\end{mnemonicbox}
\subsection*{Question 5(c) OR [7
marks]}\label{q5c}

\textbf{Develop a Python code for Polymorphism and explain it.}

\begin{solutionbox}

\begin{verbatim}
\# Base class
class Animal:
    def \_\_init\_\_(self, name):
        self.name = name
    
    def make\_sound(self):
        \# Generic sound {- will be overridden by subclasses}
        return "Some generic sound"

\# Derived class 1
class Dog(Animal):
    def make\_sound(self):
        \# Override base class method
        return "Woof!"

\# Derived class 2
class Cat(Animal):
    def make\_sound(self):
        \# Override base class method
        return "Meow!"

\# Derived class 3
class Cow(Animal):
    def make\_sound(self):
        \# Override base class method
        return "Moo!"

\# Function using polymorphism
def animal\_sound(animal):
    \# Same function works for any Animal subclass
    return animal.make\_sound()

\# Create objects of different classes
dog = Dog("Rex")
cat = Cat("Whiskers")
cow = Cow("Daisy")

\# Demonstrate polymorphism
animals = [dog, cat, cow]
for animal in animals:
    print(f"\{animal.name\} says: \{animal\_sound(animal)\}")

\# Output:
\# Rex says: Woof!
\# Whiskers says: Meow!
\# Daisy says: Moo!
\end{verbatim}

\textbf{Polymorphism Diagram:}

\begin{verbatim}
classDiagram
    Animal {|{-}{-} Dog}
    Animal {|{-}{-} Cat}
    Animal {|{-}{-} Cow}

    class Animal \{
        +name: string
        +\_\_init\_\_(name)
        +make\_sound()
    \}
    
    class Dog \{
        +make\_sound()
    \}
    
    class Cat \{
        +make\_sound()
    \}
    
    class Cow \{
        +make\_sound()
    \}
\end{verbatim}

\begin{itemize}
\tightlist
\item
  \textbf{Method overriding}: Subclasses implement their own versions
\item
  \textbf{Single interface}: Same method name for different behavior
\item
  \textbf{Flexibility}: Code works with any class in the hierarchy
\item
  \textbf{Dynamic binding}: Correct method called based on object type
\end{itemize}

\end{solutionbox}
\begin{mnemonicbox}
``Same Method, Different Behavior''

\end{mnemonicbox}

\end{document}
