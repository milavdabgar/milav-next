\documentclass[10pt,a4paper]{article}

% content/resources/templates/preamble.tex
\usepackage[margin=0.6in]{geometry}
\author{Milav Dabgar}
\usepackage{amsmath,amssymb,amsthm}
\usepackage{booktabs}
\usepackage{multirow}
\usepackage{xcolor}
\usepackage{tcolorbox}
\tcbuselibrary{breakable,skins}
\usepackage[colorlinks=true,linkcolor=blue]{hyperref}
\usepackage{titlesec}
\usepackage{enumitem}
\usepackage{tikz}
\usepackage{pgfplots}
\usepackage{circuitikz}
\usepackage[version=4]{mhchem}
\usepackage{longtable}
\usepackage{array}
\usepackage{float}
\usepackage{caption}
\usepackage{listings}

\lstset{
  basicstyle=\small\ttfamily,
  breaklines=true,
  breakatwhitespace=false,
  postbreak=\mbox{\textcolor{red}{$\hookrightarrow$}\space},
  float=false,
  numbers=left,
  numberstyle=\tiny\color{gray},
  numbersep=10pt,
  xleftmargin=2em,
  keywordstyle=\color{blue},
  commentstyle=\color{green!60!black},
  stringstyle=\color{purple},
  backgroundcolor=\color{gray!5},
  showstringspaces=false,
  tabsize=2,
  captionpos=b,
  keepspaces=true,
  columns=flexible
}

\pgfplotsset{compat=1.18}
\usetikzlibrary{shapes,arrows,positioning,calc,patterns,decorations.pathmorphing,decorations.markings,arrows.meta}

% Color scheme
\definecolor{headcolor}{RGB}{0,102,204}
\definecolor{keycolor}{RGB}{220,20,60}
\definecolor{solutioncolor}{RGB}{34,139,34}
\definecolor{mnemoniccolor}{RGB}{148,0,211}
\definecolor{codecolor}{RGB}{0,0,100}

% Spacing
\setlength{\parskip}{3pt}
\setlist[itemize]{nosep}
\setlist[enumerate]{nosep}

% Title formatting
\titleformat{\section}{\Large\bfseries\color{headcolor}}{\thesection}{1em}{}
\titleformat{\subsection}{\large\bfseries\color{headcolor}}{\thesubsection}{1em}{}

% Pandoc tightlist compatibility
\providecommand{\tightlist}{%
  \setlength{\itemsep}{0pt}\setlength{\parskip}{0pt}}

% Pandoc longtable compatibility
\newcounter{none}
\def\thenone{}


% content/resources/templates/english-boxes.tex
% This file is currently empty - it exists to maintain consistency with the import structure.
% Add custom environments here if needed in the future.


\begin{document}

\begin{center}
{\Huge\bfseries\color{headcolor} Subject Name Solutions}\\[5pt]
{\LARGE 4351108 -- Winter 2023}\\[3pt]
{\large Semester 1 Study Material}\\[3pt]
{\normalsize\textit{Detailed Solutions and Explanations}}
\end{center}

\vspace{10pt}

\subsection*{Question 1(a) [3 marks]}\label{q1a}

\textbf{List any 6 applications of Python programming language.}

\begin{solutionbox}

\textbf{Table of Python Applications:}

{\def\LTcaptype{none} % do not increment counter
\begin{longtable}[]{@{}ll@{}}
\toprule\noalign{}
Application Area & Description \\
\midrule\noalign{}
\endhead
\bottomrule\noalign{}
\endlastfoot
\textbf{Web Development} & Django, Flask frameworks \\
\textbf{Data Science} & Analysis and visualization \\
\textbf{Machine Learning} & AI model development \\
\textbf{Desktop Applications} & GUI using Tkinter, PyQt \\
\textbf{Game Development} & Pygame library \\
\textbf{Automation} & Scripting and testing \\
\end{longtable}
}

\end{solutionbox}
\begin{mnemonicbox}
``Web Data Machine Desktop Game Auto''

\end{mnemonicbox}
\subsection*{Question 1(b) [4 marks]}\label{q1b}

\textbf{List any 8 features of Python programming language.}

\begin{solutionbox}

\textbf{Table of Python Features:}

{\def\LTcaptype{none} % do not increment counter
\begin{longtable}[]{@{}ll@{}}
\toprule\noalign{}
Feature & Description \\
\midrule\noalign{}
\endhead
\bottomrule\noalign{}
\endlastfoot
\textbf{Simple Syntax} & Easy to read and write \\
\textbf{Interpreted} & No compilation needed \\
\textbf{Object-Oriented} & Supports OOP concepts \\
\textbf{Dynamic Typing} & Variables don't need type declaration \\
\textbf{Cross-Platform} & Runs on multiple OS \\
\textbf{Large Libraries} & Rich standard library \\
\textbf{Open Source} & Free to use and modify \\
\textbf{Interactive} & REPL environment \\
\end{longtable}
}

\end{solutionbox}
\begin{mnemonicbox}
``Simple Interpreted Object Dynamic Cross Large Open
Interactive''

\end{mnemonicbox}
\subsection*{Question 1(c) [7 marks]}\label{q1c}

\textbf{Explain working of for and while loops in Python.}

\begin{solutionbox}

\textbf{For Loop:}

\begin{itemize}
\tightlist
\item
  \textbf{Iteration}: Repeats over sequences (lists, strings, ranges)
\item
  \textbf{Syntax}: \texttt{for\ variable\ in\ sequence:}
\item
  \textbf{Automatic}: Handles iteration automatically
\end{itemize}

\textbf{While Loop:}

\begin{itemize}
\tightlist
\item
  \textbf{Condition-based}: Continues while condition is true
\item
  \textbf{Manual control}: Programmer controls iteration
\item
  \textbf{Risk}: Can create infinite loops if condition never becomes
  false
\end{itemize}

\textbf{Diagram:}

\begin{verbatim}
    Start
      |
   Initialize
      |
    Condition? {-{-}{-}{-}No{-}{-}{-}{-} End}
      |Yes
    Execute
      |
   Update
      |
    (loop back)
\end{verbatim}

\textbf{Code Example:}

\begin{verbatim}
\# For loop
for i in range(5):
    print(i)

\# While loop
i = 0
while i {} 5:
    print(i)
    i += 1
\end{verbatim}

\end{solutionbox}
\begin{mnemonicbox}
``For Automatic, While Manual''

\end{mnemonicbox}
\subsection*{Question 1(c OR) [7
marks]}\label{question-1c-or-7-marks}

\textbf{Explain working of break continue and pass statements in
Python.}

\begin{solutionbox}

\textbf{Break Statement:}

\begin{itemize}
\tightlist
\item
  \textbf{Exit}: Terminates the entire loop
\item
  \textbf{Usage}: When specific condition is met
\item
  \textbf{Effect}: Control moves to next statement after loop
\end{itemize}

\textbf{Continue Statement:}

\begin{itemize}
\tightlist
\item
  \textbf{Skip}: Skips current iteration only
\item
  \textbf{Usage}: Skip specific values in iteration
\item
  \textbf{Effect}: Moves to next iteration
\end{itemize}

\textbf{Pass Statement:}

\begin{itemize}
\tightlist
\item
  \textbf{Placeholder}: Does nothing, syntactic placeholder
\item
  \textbf{Usage}: When syntax requires statement but no action needed
\item
  \textbf{Effect}: No operation performed
\end{itemize}

\textbf{Code Examples:}

\begin{verbatim}
\# Break
for i in range(10):
if

i == 5:

        break
    print(i)  \# prints 0,1,2,3,4

\# Continue
for i in range(5):
if

i == 2:

        continue
    print(i)  \# prints 0,1,3,4

\# Pass
if True:
    pass  \# placeholder
\end{verbatim}

\end{solutionbox}
\begin{mnemonicbox}
``Break Exits, Continue Skips, Pass Waits''

\end{mnemonicbox}
\subsection*{Question 2(a) [3 marks]}\label{q2a}

\textbf{Develop a Python program to increment each element of list by
one.}

\begin{solutionbox}

\textbf{Code:}

\begin{verbatim}
\# Method 1 {- Using for loop}
numbers = [1, 2, 3, 4, 5]
for i in range(len(numbers)):
    numbers[i] += 1
print(numbers)

\# Method 2 {- List comprehension}
numbers = [1, 2, 3, 4, 5]
result = [x + 1 for x in numbers]
print(result)
\end{verbatim}

\end{solutionbox}
\begin{mnemonicbox}
``Loop Index or Comprehension''

\end{mnemonicbox}
\subsection*{Question 2(b) [4 marks]}\label{q2b}

\textbf{Develop a Python program to read three numbers from the user and
find the average of the numbers.}

\begin{solutionbox}

\textbf{Code:}

\begin{verbatim}
\# Input three numbers
num1 = float(input("Enter first number: "))
num2 = float(input("Enter second number: "))
num3 = float(input("Enter third number: "))

\# Calculate average
average = (num1 + num2 + num3) / 3

\# Display result
print(f"Average is: \{average\}")
\end{verbatim}

\textbf{Key Points:}

\begin{itemize}
\tightlist
\item
  \textbf{Input}: Use \texttt{float()} for decimal numbers
\item
  \textbf{Formula}: Sum divided by count
\item
  \textbf{Output}: Use f-string for formatting
\end{itemize}

\end{solutionbox}
\begin{mnemonicbox}
``Input Float, Sum Divide, Format Output''

\end{mnemonicbox}
\subsection*{Question 2(c) [7 marks]}\label{q2c}

\textbf{Explain Python's list data type in detail.}

\begin{solutionbox}

\textbf{List Characteristics:}

\begin{itemize}
\tightlist
\item
  \textbf{Ordered}: Elements maintain sequence
\item
  \textbf{Mutable}: Can be modified after creation
\item
  \textbf{Heterogeneous}: Can store different data types
\item
  \textbf{Indexed}: Access elements using index (0-based)
\end{itemize}

\textbf{List Operations Table:}

{\def\LTcaptype{none} % do not increment counter
\begin{longtable}[]{@{}lll@{}}
\toprule\noalign{}
Operation & Syntax & Description \\
\midrule\noalign{}
\endhead
\bottomrule\noalign{}
\endlastfoot
\textbf{Creation} & \texttt{list\ =\ [1,2,3]} & Create new list \\
\textbf{Access} & \texttt{list[0]} & Get element by index \\
\textbf{Append} & \texttt{list.append(4)} & Add element at end \\
\textbf{Insert} & \texttt{list.insert(1,5)} & Add at specific
position \\
\textbf{Remove} & \texttt{list.remove(2)} & Remove first occurrence \\
\textbf{Pop} & \texttt{list.pop()} & Remove and return last \\
\textbf{Slice} & \texttt{list[1:3]} & Get sublist \\
\end{longtable}
}

\textbf{Code Example:}

\begin{verbatim}
\# List creation and operations
fruits = [{apple}, {banana}, {orange}]
fruits.append({mango})
fruits.insert(1, {grape})
print(fruits[0])  \# apple
print(len(fruits))  \# 5
\end{verbatim}

\end{solutionbox}
\begin{mnemonicbox}
``Ordered Mutable Heterogeneous Indexed''

\end{mnemonicbox}
\subsection*{Question 2(a OR) [3
marks]}\label{question-2a-or-3-marks}

\textbf{Develop a Python program to find sum of all elements in a list
using for loop.}

\begin{solutionbox}

\textbf{Code:}

\begin{verbatim}
\# Method 1 {- Traditional for loop}
numbers = [10, 20, 30, 40, 50]
total = 0
for num in numbers:
    total += num
print(f"Sum is: \{total\}")

\# Method 2 {- Using range and index}
numbers = [10, 20, 30, 40, 50]
total = 0
for i in range(len(numbers)):
    total += numbers[i]
print(f"Sum is: \{total\}")
\end{verbatim}

\end{solutionbox}
\begin{mnemonicbox}
``Initialize Zero, Loop Add, Print Total''

\end{mnemonicbox}
\subsection*{Question 2(b OR) [4
marks]}\label{question-2b-or-4-marks}

\textbf{Develop a Python program to get input from user for principal,
rate and no of years then calculate and display simple interest from
that.}

\begin{solutionbox}

\textbf{Code:}

\begin{verbatim}
\# Get input from user
principal = float(input("Enter principal amount: "))
rate = float(input("Enter rate of interest: "))
time = float(input("Enter time in years: "))

\# Calculate simple interest
simple\_interest = (principal * rate * time) / 100

\# Display results
print(f"Principal: \{principal\}")
print(f"Rate: \{rate\}\%")
print(f"Time: \{time\} years")
print(f"Simple Interest: \{simple\_interest\}")
print(f"Total Amount: \{principal + simple\_interest\}")
\end{verbatim}

\textbf{Formula:}

\begin{itemize}
\tightlist
\item
  \textbf{Simple Interest} = (P \times R \times T) / 100
\item
  \textbf{Total Amount} = Principal + Simple Interest
\end{itemize}

\end{solutionbox}
\begin{mnemonicbox}
``Principal Rate Time, Multiply Divide Hundred''

\end{mnemonicbox}
\subsection*{Question 2(c OR) [7
marks]}\label{question-2c-or-7-marks}

\textbf{Explain Python's tuple data type in detail.}

\begin{solutionbox}

\textbf{Tuple Characteristics:}

\begin{itemize}
\tightlist
\item
  \textbf{Ordered}: Elements maintain sequence
\item
  \textbf{Immutable}: Cannot be modified after creation
\item
  \textbf{Heterogeneous}: Can store different data types
\item
  \textbf{Indexed}: Access using index (0-based)
\end{itemize}

\textbf{Tuple Operations Table:}

{\def\LTcaptype{none} % do not increment counter
\begin{longtable}[]{@{}lll@{}}
\toprule\noalign{}
Operation & Syntax & Description \\
\midrule\noalign{}
\endhead
\bottomrule\noalign{}
\endlastfoot
\textbf{Creation} & \texttt{tuple\ =\ (1,2,3)} & Create new tuple \\
\textbf{Access} & \texttt{tuple[0]} & Get element by index \\
\textbf{Count} & \texttt{tuple.count(2)} & Count occurrences \\
\textbf{Index} & \texttt{tuple.index(3)} & Find first index \\
\textbf{Slice} & \texttt{tuple[1:3]} & Get sub-tuple \\
\textbf{Length} & \texttt{len(tuple)} & Get tuple size \\
\textbf{Concatenate} & \texttt{tuple1\ +\ tuple2} & Join tuples \\
\end{longtable}
}

\textbf{Code Example:}

\begin{verbatim}
\# Tuple creation and operations
coordinates = (10, 20, 30)
print(coordinates[0])  \# 10
print(len(coordinates))  \# 3
x, y, z = coordinates  \# tuple unpacking
new\_tuple = coordinates + (40, 50)
\end{verbatim}

\textbf{Key Differences from List:}

\begin{itemize}
\tightlist
\item
  \textbf{Immutable}: Cannot change elements
\item
  \textbf{Performance}: Faster than lists
\item
  \textbf{Usage}: For fixed data collections
\end{itemize}

\end{solutionbox}
\begin{mnemonicbox}
``Ordered Immutable Heterogeneous Indexed''

\end{mnemonicbox}
\subsection*{Question 3(a) [3 marks]}\label{q3a}

\textbf{Explain any 3 random module methods.}

\begin{solutionbox}

\textbf{Random Module Methods Table:}

{\def\LTcaptype{none} % do not increment counter
\begin{longtable}[]{@{}lll@{}}
\toprule\noalign{}
Method & Syntax & Description \\
\midrule\noalign{}
\endhead
\bottomrule\noalign{}
\endlastfoot
\textbf{random()} & \texttt{random.random()} & Float between 0.0 to
1.0 \\
\textbf{randint()} & \texttt{random.randint(1,10)} & Integer between
given range \\
\textbf{choice()} & \texttt{random.choice(list)} & Random element from
sequence \\
\end{longtable}
}

\textbf{Code Example:}

\begin{verbatim}
import random

\# Generate random float
print(random.random())  \# 0.7234567

\# Generate random integer
print(random.randint(1, 10))  \# 7

\# Choose random element
colors = [{red}, {blue}, {green}]
print(random.choice(colors))  \# blue
\end{verbatim}

\end{solutionbox}
\begin{mnemonicbox}
``Random Float, Randint Integer, Choice Select''

\end{mnemonicbox}
\subsection*{Question 3(b) [4 marks]}\label{q3b}

\textbf{Develop a Python program that asks the user for a string and
prints out the location of each `a' in the string.}

\begin{solutionbox}

\textbf{Code:}

\begin{verbatim}
\# Get string from user
text = input("Enter a string: ")

\# Find all positions of {a}
positions = []
for i in range(len(text)):
    if text[i].lower() == {a}:
        positions.append(i)

\# Display results
if positions:
    print(f"Letter {a found at positions: }\{positions\}")
else:
    print("Letter {a not found in the string"})

\# Alternative method using enumerate
text = input("Enter a string: ")
for index, char in enumerate(text):
    if char.lower() == {a}:
        print(f"{a found at position }\{index\}")
\end{verbatim}

\textbf{Key Points:}

\begin{itemize}
\tightlist
\item
  \textbf{Case-insensitive}: Use \texttt{.lower()} to find both `a' and
  `A'
\item
  \textbf{Index tracking}: Use range or enumerate
\item
  \textbf{Output format}: Clear position indication
\end{itemize}

\end{solutionbox}
\begin{mnemonicbox}
``Loop Index Check Append Print''

\end{mnemonicbox}
\subsection*{Question 3(c) [7 marks]}\label{q3c}

\textbf{Explain Python's string data type in detail.}

\begin{solutionbox}

\textbf{String Characteristics:}

\begin{itemize}
\tightlist
\item
  \textbf{Immutable}: Cannot be changed after creation
\item
  \textbf{Sequence}: Ordered collection of characters
\item
  \textbf{Indexed}: Access characters using index
\item
  \textbf{Unicode}: Supports all languages and symbols
\end{itemize}

\textbf{String Methods Table:}

{\def\LTcaptype{none} % do not increment counter
\begin{longtable}[]{@{}lll@{}}
\toprule\noalign{}
Method & Example & Description \\
\midrule\noalign{}
\endhead
\bottomrule\noalign{}
\endlastfoot
\textbf{upper()} & \texttt{"hello".upper()} & Convert to uppercase \\
\textbf{lower()} & \texttt{"HELLO".lower()} & Convert to lowercase \\
\textbf{strip()} & \texttt{"\ hello\ ".strip()} & Remove whitespace \\
\textbf{split()} & \texttt{"a,b,c".split(",")} & Split into list \\
\textbf{replace()} & \texttt{"hello".replace("l","x")} & Replace
substring \\
\textbf{find()} & \texttt{"hello".find("e")} & Find substring index \\
\textbf{join()} & \texttt{",".join(["a","b"])} & Join list
elements \\
\end{longtable}
}

\textbf{String Operations:}

\begin{verbatim}
\# String creation
name = "Python Programming"

\# String indexing and slicing
print(name[0])      \# P
print(name[0:6])    \# Python
print(name[{-}1])     \# g

\# String formatting
age = 25
message = f"I am \{age\} years old"
\end{verbatim}

\textbf{Key Features:}

\begin{itemize}
\tightlist
\item
  \textbf{Concatenation}: Using + operator
\item
  \textbf{Repetition}: Using * operator
\item
  \textbf{Membership}: Using `in' operator
\item
  \textbf{Formatting}: f-strings, .format(), \% formatting
\end{itemize}

\end{solutionbox}
\begin{mnemonicbox}
``Immutable Sequence Indexed Unicode''

\end{mnemonicbox}
\subsection*{Question 3(a OR) [3
marks]}\label{question-3a-or-3-marks}

\textbf{Explain any 3 math module methods.}

\begin{solutionbox}

\textbf{Math Module Methods Table:}

{\def\LTcaptype{none} % do not increment counter
\begin{longtable}[]{@{}lll@{}}
\toprule\noalign{}
Method & Syntax & Description \\
\midrule\noalign{}
\endhead
\bottomrule\noalign{}
\endlastfoot
\textbf{sqrt()} & \texttt{math.sqrt(16)} & Square root calculation \\
\textbf{pow()} & \texttt{math.pow(2,3)} & Power calculation \\
\textbf{ceil()} & \texttt{math.ceil(4.3)} & Round up to integer \\
\end{longtable}
}

\textbf{Code Example:}

\begin{verbatim}
import math

\# Square root
print(math.sqrt(25))    \# 5.0

\# Power
print(math.pow(2, 3))   \# 8.0

\# Ceiling
print(math.ceil(4.2))   \# 5
\end{verbatim}

\end{solutionbox}
\begin{mnemonicbox}
``Square Root, Power Up, Ceiling Round''

\end{mnemonicbox}
\subsection*{Question 3(b OR) [4
marks]}\label{question-3b-or-4-marks}

\textbf{Develop a Python program to get a string from the user and count
total no. of Vowels present in that string.}

\begin{solutionbox}

\textbf{Code:}

\begin{verbatim}
\# Get string from user
text = input("Enter a string: ")

\# Define vowels
vowels = "aeiouAEIOU"

\# Count vowels
vowel\_count = 0
for char in text:
    if char in vowels:
        vowel\_count += 1

\# Display result
print(f"Total vowels in {}\{text\}{: }\{vowel\_count\}")

\# Alternative method using list comprehension
text = input("Enter a string: ")
vowels = "aeiouAEIOU"
count = sum(1 for char in text if char in vowels)
print(f"Total vowels: \{count\}")
\end{verbatim}

\textbf{Key Points:}

\begin{itemize}
\tightlist
\item
  \textbf{Vowel definition}: Include both cases
\item
  \textbf{Loop through}: Each character in string
\item
  \textbf{Count logic}: Check membership and increment
\end{itemize}

\end{solutionbox}
\begin{mnemonicbox}
``Define Vowels, Loop Check, Count Increment''

\end{mnemonicbox}
\subsection*{Question 3(c OR) [7
marks]}\label{question-3c-or-7-marks}

\textbf{Explain Python's set data type in detail.}

\begin{solutionbox}

\textbf{Set Characteristics:}

\begin{itemize}
\tightlist
\item
  \textbf{Unordered}: No fixed sequence of elements
\item
  \textbf{Mutable}: Can add/remove elements
\item
  \textbf{Unique}: No duplicate elements allowed
\item
  \textbf{Iterable}: Can loop through elements
\end{itemize}

\textbf{Set Operations Table:}

{\def\LTcaptype{none} % do not increment counter
\begin{longtable}[]{@{}lll@{}}
\toprule\noalign{}
Operation & Syntax & Description \\
\midrule\noalign{}
\endhead
\bottomrule\noalign{}
\endlastfoot
\textbf{Creation} & \texttt{set\ =\ \{1,2,3\}} & Create new set \\
\textbf{Add} & \texttt{set.add(4)} & Add single element \\
\textbf{Remove} & \texttt{set.remove(2)} & Remove element (error if not
found) \\
\textbf{Discard} & \texttt{set.discard(2)} & Remove element (no
error) \\
\textbf{Union} & \texttt{set1\ \textbar{}\ set2} & Combine sets \\
\textbf{Intersection} & \texttt{set1\ \&\ set2} & Common elements \\
\textbf{Difference} & \texttt{set1\ -\ set2} & Elements in set1 only \\
\end{longtable}
}

\textbf{Set Mathematical Operations:}

\begin{verbatim}
\# Set creation
A = \{1, 2, 3, 4\}
B = \{3, 4, 5, 6\}

\# Set operations
print(A | B)    \# Union: \{1,2,3,4,5,6\}
print(A \& B)    \# Intersection: \{3,4\}
print(A {-} B)    \# Difference: \{1,2\}
print(A \^{} B)    \# Symmetric difference: \{1,2,5,6\}
\end{verbatim}

\textbf{Key Uses:}

\begin{itemize}
\tightlist
\item
  \textbf{Remove duplicates}: From lists
\item
  \textbf{Mathematical operations}: Union, intersection
\item
  \textbf{Membership testing}: Fast lookup
\end{itemize}

\end{solutionbox}
\begin{mnemonicbox}
``Unordered Mutable Unique Iterable''

\end{mnemonicbox}
\subsection*{Question 4(a) [3 marks]}\label{q4a}

\textbf{What is the class in Python. How is it different from an
object?}

\begin{solutionbox}

\textbf{Class vs Object Comparison:}

{\def\LTcaptype{none} % do not increment counter
\begin{longtable}[]{@{}lll@{}}
\toprule\noalign{}
Aspect & Class & Object \\
\midrule\noalign{}
\endhead
\bottomrule\noalign{}
\endlastfoot
\textbf{Definition} & Blueprint or template & Instance of class \\
\textbf{Memory} & No memory allocated & Memory allocated \\
\textbf{Existence} & Logical entity & Physical entity \\
\textbf{Creation} & Using class keyword & Using class constructor \\
\end{longtable}
}

\textbf{Example:}

\begin{verbatim}
\# Class definition (blueprint)
class Car:
    def \_\_init\_\_(self, brand):
        self.brand = brand

\# Object creation (instances)
car1 = Car("Toyota")  \# Object 1
car2 = Car("Honda")   \# Object 2
\end{verbatim}

\textbf{Key Points:}

\begin{itemize}
\tightlist
\item
  \textbf{Class}: Template defining properties and methods
\item
  \textbf{Object}: Actual instance with specific values
\item
  \textbf{Relationship}: One class, multiple objects
\end{itemize}

\end{solutionbox}
\begin{mnemonicbox}
``Class Blueprint, Object Instance''

\end{mnemonicbox}
\subsection*{Question 4(b) [4 marks]}\label{q4b}

\textbf{Explain any four methods of dictionary data type of Python.}

\begin{solutionbox}

\textbf{Dictionary Methods Table:}

{\def\LTcaptype{none} % do not increment counter
\begin{longtable}[]{@{}lll@{}}
\toprule\noalign{}
Method & Syntax & Description \\
\midrule\noalign{}
\endhead
\bottomrule\noalign{}
\endlastfoot
\textbf{keys()} & \texttt{dict.keys()} & Get all keys \\
\textbf{values()} & \texttt{dict.values()} & Get all values \\
\textbf{items()} & \texttt{dict.items()} & Get key-value pairs \\
\textbf{get()} &
\texttt{dict.get(\textquotesingle{}key\textquotesingle{})} & Get value
safely \\
\end{longtable}
}

\textbf{Code Example:}

\begin{verbatim}
student = \{{name}: {John}, {age}: 20, {grade}: {A}\}

\# Dictionary methods
print(student.keys())    \# dict\_keys([{name, age, grade])}
print(student.values())  \# dict\_values([{John, 20, A])}
print(student.items())   \# dict\_items([({name, John), ...])}
print(student.get({name}))  \# John
\end{verbatim}

\end{solutionbox}
\begin{mnemonicbox}
``Keys Values Items Get''

\end{mnemonicbox}
\subsection*{Question 4(c) [7 marks]}\label{q4c}

\textbf{Develop a Python program that defines a user-defined module for
performing some tasks. Import this module and use its functions.}

\begin{solutionbox}

\textbf{Module Creation (math\_operations.py):}

\begin{verbatim}
\# math\_operations.py
def add(a, b):
    """Add two numbers"""
    return a + b

def multiply(a, b):
    """Multiply two numbers"""
    return a * b

def factorial(n):
    """Calculate factorial"""
    if n {=} 1:
        return 1
    return n * factorial(n {-} 1)

PI = 3.14159

def circle\_area(radius):
    """Calculate circle area"""
    return PI * radius * radius
\end{verbatim}

\textbf{Main Program (main.py):}

\begin{verbatim}
\# Import entire module
import math\_operations

\# Use module functions
result1 = math\_operations.add(5, 3)
result2 = math\_operations.multiply(4, 6)
result3 = math\_operations.factorial(5)
area = math\_operations.circle\_area(5)

print(f"Addition: \{result1\}")
print(f"Multiplication: \{result2\}")
print(f"Factorial: \{result3\}")
print(f"Circle Area: \{area\}")

\# Import specific functions
from math\_operations import add, multiply
print(f"Direct call: \{add(10, 20)\}")
\end{verbatim}

\textbf{Key Points:}

\begin{itemize}
\tightlist
\item
  \textbf{Module creation}: Separate .py file with functions
\item
  \textbf{Import methods}: import module or from module import function
\item
  \textbf{Usage}: Access using module.function() or direct function()
\end{itemize}

\end{solutionbox}
\begin{mnemonicbox}
``Create Import Use''

\end{mnemonicbox}
\subsection*{Question 4(a OR) [3
marks]}\label{question-4a-or-3-marks}

\textbf{Define types of methods available in Python classes.}

\begin{solutionbox}

\textbf{Types of Methods Table:}

{\def\LTcaptype{none} % do not increment counter
\begin{longtable}[]{@{}
  >{\raggedright\arraybackslash}p{(\linewidth - 4\tabcolsep) * \real{0.3824}}
  >{\raggedright\arraybackslash}p{(\linewidth - 4\tabcolsep) * \real{0.2353}}
  >{\raggedright\arraybackslash}p{(\linewidth - 4\tabcolsep) * \real{0.3824}}@{}}
\toprule\noalign{}
\begin{minipage}[b]{\linewidth}\raggedright
Method Type
\end{minipage} & \begin{minipage}[b]{\linewidth}\raggedright
Syntax
\end{minipage} & \begin{minipage}[b]{\linewidth}\raggedright
Description
\end{minipage} \\
\midrule\noalign{}
\endhead
\bottomrule\noalign{}
\endlastfoot
\textbf{Instance Method} & \texttt{def\ method(self):} & Access instance
variables \\
\textbf{Class Method} & \texttt{@classmethod\ def\ method(cls):} &
Access class variables \\
\textbf{Static Method} & \texttt{@staticmethod\ def\ method():} &
Independent of class/instance \\
\end{longtable}
}

\textbf{Example:}

\begin{verbatim}
class MyClass:
    class\_var = "Class Variable"
    
    def instance\_method(self):  \# Instance method
        return "Instance method"
    
    @classmethod
    def class\_method(cls):      \# Class method
        return cls.class\_var
    
    @staticmethod
    def static\_method():        \# Static method
        return "Static method"
\end{verbatim}

\end{solutionbox}
\begin{mnemonicbox}
``Instance Self, Class Cls, Static None''

\end{mnemonicbox}
\subsection*{Question 4(b OR) [4
marks]}\label{question-4b-or-4-marks}

\textbf{Explain any four methods of string data type of Python.}

\begin{solutionbox}

\textbf{String Methods Table:}

{\def\LTcaptype{none} % do not increment counter
\begin{longtable}[]{@{}
  >{\raggedright\arraybackslash}p{(\linewidth - 4\tabcolsep) * \real{0.2759}}
  >{\raggedright\arraybackslash}p{(\linewidth - 4\tabcolsep) * \real{0.2759}}
  >{\raggedright\arraybackslash}p{(\linewidth - 4\tabcolsep) * \real{0.4483}}@{}}
\toprule\noalign{}
\begin{minipage}[b]{\linewidth}\raggedright
Method
\end{minipage} & \begin{minipage}[b]{\linewidth}\raggedright
Syntax
\end{minipage} & \begin{minipage}[b]{\linewidth}\raggedright
Description
\end{minipage} \\
\midrule\noalign{}
\endhead
\bottomrule\noalign{}
\endlastfoot
\textbf{startswith()} &
\texttt{str.startswith(\textquotesingle{}pre\textquotesingle{})} & Check
if starts with substring \\
\textbf{endswith()} &
\texttt{str.endswith(\textquotesingle{}suf\textquotesingle{})} & Check
if ends with substring \\
\textbf{isdigit()} & \texttt{str.isdigit()} & Check if all digits \\
\textbf{count()} &
\texttt{str.count(\textquotesingle{}sub\textquotesingle{})} & Count
substring occurrences \\
\end{longtable}
}

\textbf{Code Example:}

\begin{verbatim}
text = "Hello World 123"

\# String methods
print(text.startswith({Hello}))  \# True
print(text.endswith({123}))      \# True
print({123}.isdigit())           \# True
print(text.count({l}))           \# 3
\end{verbatim}

\end{solutionbox}
\begin{mnemonicbox}
``Start End Digit Count''

\end{mnemonicbox}
\subsection*{Question 4(c OR) [7
marks]}\label{question-4c-or-7-marks}

\textbf{Develop a Python program to find factorial of a number using
recursive user defined function.}

\begin{solutionbox}

\textbf{Code:}

\begin{verbatim}
def factorial(n):
    """
    Calculate factorial using recursion
    Base case: factorial(0) = 1, factorial(1) = 1
    Recursive case: factorial(n) = n * factorial(n{-1)}
    """
    \# Base case
if

n == 0 or

n == 1:

        return 1
    
    \# Recursive case
    else:
        return n * factorial(n {-} 1)

\# Main program
try:
    num = int(input("Enter a number: "))
    
    if num {} 0:
        print("Factorial not defined for negative numbers")
    else:
        result = factorial(num)
        print(f"Factorial of \{num\} is \{result\}")
        
except ValueError:
    print("Please enter a valid integer")

\# Test cases
print(f"Factorial of 5: \{factorial(5)\}")  \# 120
print(f"Factorial of 0: \{factorial(0)\}")  \# 1
\end{verbatim}

\textbf{Recursion Flow:}

\begin{verbatim}
factorial(5)
    |
5 * factorial(4)
        |
    4 * factorial(3)
            |
        3 * factorial(2)
                |
            2 * factorial(1)
                    |
                return 1

Result: 5 * 4 * 3 * 2 * 1 = 120
\end{verbatim}

\textbf{Key Points:}

\begin{itemize}
\tightlist
\item
  \textbf{Base case}: Stops recursion (n=0 or n=1)
\item
  \textbf{Recursive case}: Function calls itself
\item
  \textbf{Error handling}: Check for negative input
\end{itemize}

\end{solutionbox}
\begin{mnemonicbox}
``Base Stop, Recursive Call, Error Check''

\end{mnemonicbox}
\subsection*{Question 5(a) [3 marks]}\label{q5a}

\textbf{Develop a python program to Implement single inheritance.}

\begin{solutionbox}

\textbf{Code:}

\begin{verbatim}
\# Parent class
class Animal:
    def \_\_init\_\_(self, name):
        self.name = name
    
    def speak(self):
        print(f"\{self.name\} makes a sound")
    
    def eat(self):
        print(f"\{self.name\} is eating")

\# Child class inheriting from Animal
class Dog(Animal):
    def \_\_init\_\_(self, name, breed):
        super().\_\_init\_\_(name)  \# Call parent constructor
        self.breed = breed
    
    def bark(self):
        print(f"\{self.name\} is barking")
    
    def speak(self):  \# Override parent method
        print(f"\{self.name\} says Woof!")

\# Create objects and test
dog = Dog("Buddy", "Golden Retriever")
dog.speak()  \# Buddy says Woof!
dog.eat()    \# Buddy is eating (inherited)
dog.bark()   \# Buddy is barking (own method)
\end{verbatim}

\end{solutionbox}
\begin{mnemonicbox}
``Parent Child Inherit Override''

\end{mnemonicbox}
\subsection*{Question 5(b) [4 marks]}\label{q5b}

\textbf{Explain the significance of constructors in Python classes.}

\begin{solutionbox}

\textbf{Constructor Significance:}

{\def\LTcaptype{none} % do not increment counter
\begin{longtable}[]{@{}ll@{}}
\toprule\noalign{}
Aspect & Description \\
\midrule\noalign{}
\endhead
\bottomrule\noalign{}
\endlastfoot
\textbf{Initialization} & Automatically called when object is created \\
\textbf{Setup} & Initialize instance variables with values \\
\textbf{Memory} & Allocate memory for object attributes \\
\textbf{Validation} & Validate input parameters during creation \\
\end{longtable}
}

\textbf{Constructor Types:}

\begin{verbatim}
class Student:
    \# Default constructor
    def \_\_init\_\_(self):
        self.name = "Unknown"
        self.age = 0
    
    \# Parameterized constructor
    def \_\_init\_\_(self, name, age):
        self.name = name
        self.age = age
        print(f"Student \{name\} created")
    
    \# Constructor with default parameters
    def \_\_init\_\_(self, name="Unknown", age=0):
        self.name = name
        self.age = age
\end{verbatim}

\textbf{Key Benefits:}

\begin{itemize}
\tightlist
\item
  \textbf{Automatic execution}: No need to call manually
\item
  \textbf{Object state}: Ensures proper initialization
\item
  \textbf{Code reusability}: Common setup code in one place
\end{itemize}

\end{solutionbox}
\begin{mnemonicbox}
``Initialize Setup Memory Validate''

\end{mnemonicbox}
\subsection*{Question 5(c) [7 marks]}\label{q5c}

\textbf{Develop a Python program to demonstrate method overriding using
inheritance.}

\begin{solutionbox}

\textbf{Code:}

\begin{verbatim}
\# Base class
class Shape:
    def \_\_init\_\_(self, name):
        self.name = name
    
    def area(self):
        print(f"Area calculation for \{self.name\}")
        return 0
    
    def display(self):
        print(f"This is a \{self.name\}")

\# Derived class 1
class Rectangle(Shape):
    def \_\_init\_\_(self, length, width):
        super().\_\_init\_\_("Rectangle")
        self.length = length
        self.width = width
    
    \# Override area method
    def area(self):
        area\_value = self.length * self.width
        print(f"Rectangle area: \{area\_value\}")
        return area\_value

\# Derived class 2
class Circle(Shape):
    def \_\_init\_\_(self, radius):
        super().\_\_init\_\_("Circle")
        self.radius = radius
    
    \# Override area method
    def area(self):
        area\_value = 3.14 * self.radius * self.radius
        print(f"Circle area: \{area\_value\}")
        return area\_value
    
    \# Override display method
    def display(self):
        super().display()  \# Call parent method
        print(f"Radius: \{self.radius\}")

\# Test method overriding
shapes = [
    Rectangle(5, 4),
    Circle(3),
    Shape("Generic Shape")
]

for shape in shapes:
    shape.display()
    shape.area()
    print("{-"} * 20)
\end{verbatim}

\textbf{Method Overriding Diagram:}

\begin{verbatim}
    Shape (Base)
    |{-{-} area()}
    |{-{-} display()}
         |
    Rectangle    Circle
    |{-{-} area()   |{-}{-} area()}
                 |{-{-} display()}
\end{verbatim}

\textbf{Key Points:}

\begin{itemize}
\tightlist
\item
  \textbf{Same method name}: In parent and child classes
\item
  \textbf{Different implementation}: Child class provides specific logic
\item
  \textbf{Runtime decision}: Correct method called based on object type
\item
  \textbf{Super() usage}: Access parent class method
\end{itemize}

\end{solutionbox}
\begin{mnemonicbox}
``Same Name Different Logic Runtime Decision''

\end{mnemonicbox}
\subsection*{Question 5(a OR) [3
marks]}\label{question-5a-or-3-marks}

\textbf{Explain concept of data encapsulation in Python.}

\begin{solutionbox}

\textbf{Data Encapsulation:}

{\def\LTcaptype{none} % do not increment counter
\begin{longtable}[]{@{}ll@{}}
\toprule\noalign{}
Aspect & Description \\
\midrule\noalign{}
\endhead
\bottomrule\noalign{}
\endlastfoot
\textbf{Definition} & Bundling data and methods together \\
\textbf{Access Control} & Restrict direct access to internal data \\
\textbf{Data Hiding} & Internal implementation hidden from outside \\
\textbf{Interface} & Provide controlled access through methods \\
\end{longtable}
}

\textbf{Implementation:}

\begin{verbatim}
class BankAccount:
    def \_\_init\_\_(self, balance):
        self.\_\_balance = balance  \# Private attribute
    
    def deposit(self, amount):    \# Public method
        if amount {} 0:
            self.\_\_balance += amount
    
    def get\_balance(self):        \# Public method
        return self.\_\_balance
    
    def \_\_validate(self):         \# Private method
        return self.\_\_balance {=} 0

\# Usage
account = BankAccount(1000)
account.deposit(500)
print(account.get\_balance())  \# 1500
\# print(account.\_\_balance)    \# Error {- cannot access private}
\end{verbatim}

\end{solutionbox}
\begin{mnemonicbox}
``Bundle Data Hide Interface''

\end{mnemonicbox}
\subsection*{Question 5(b OR) [4
marks]}\label{question-5b-or-4-marks}

\textbf{Explain concept of abstract classes in Python.}

\begin{solutionbox}

\textbf{Abstract Classes:}

{\def\LTcaptype{none} % do not increment counter
\begin{longtable}[]{@{}ll@{}}
\toprule\noalign{}
Concept & Description \\
\midrule\noalign{}
\endhead
\bottomrule\noalign{}
\endlastfoot
\textbf{Definition} & Class that cannot be instantiated directly \\
\textbf{Abstract Methods} & Methods declared but not implemented \\
\textbf{Implementation} & Subclasses must implement abstract methods \\
\textbf{Purpose} & Define common interface for related classes \\
\end{longtable}
}

\textbf{Implementation using ABC:}

\begin{verbatim}
from abc import ABC, abstractmethod

class Animal(ABC):  \# Abstract class
    @abstractmethod
    def make\_sound(self):  \# Abstract method
        pass
    
    def sleep(self):       \# Concrete method
        print("Animal is sleeping")

class Dog(Animal):
    def make\_sound(self):  \# Must implement
        print("Woof!")

class Cat(Animal):
    def make\_sound(self):  \# Must implement
        print("Meow!")

\# Usage
dog = Dog()
dog.make\_sound()  \# Woof!
\# animal = Animal()  \# Error {- cannot instantiate}
\end{verbatim}

\textbf{Key Features:}

\begin{itemize}
\tightlist
\item
  \textbf{Cannot instantiate}: Abstract class cannot create objects
\item
  \textbf{Force implementation}: Subclasses must implement abstract
  methods
\item
  \textbf{Common interface}: Ensures consistent method signatures
\end{itemize}

\end{solutionbox}
\begin{mnemonicbox}
``Cannot Instantiate Force Implementation Common
Interface''

\end{mnemonicbox}
\subsection*{Question 5(c OR) [7
marks]}\label{question-5c-or-7-marks}

\textbf{Develop a python program to Implement multiple inheritance.}

\begin{solutionbox}

\textbf{Code:}

\begin{verbatim}
\# First parent class
class Father:
    def \_\_init\_\_(self):
        self.father\_name = "John"
        print("Father constructor called")
    
    def show\_father(self):
        print(f"Father: \{self.father\_name\}")
    
    def work(self):
        print("Father works as Engineer")

\# Second parent class
class Mother:
    def \_\_init\_\_(self):
        self.mother\_name = "Mary"
        print("Mother constructor called")
    
    def show\_mother(self):
        print(f"Mother: \{self.mother\_name\}")
    
    def work(self):
        print("Mother works as Doctor")

\# Child class inheriting from both parents
class Child(Father, Mother):
    def \_\_init\_\_(self):
        Father.\_\_init\_\_(self)  \# Call father{s constructor}
        Mother.\_\_init\_\_(self)  \# Call mother{s constructor}
        self.child\_name = "Alice"
        print("Child constructor called")
    
    def show\_child(self):
        print(f"Child: \{self.child\_name\}")
    
    def show\_family(self):
        self.show\_father()
        self.show\_mother()
        self.show\_child()

\# Create child object and test
child = Child()
print("{n}Family Details:")
child.show\_family()
print("{n}Method Resolution:")
child.work()  \# Calls Father{s work method (MRO)}

\# Check Method Resolution Order
print(f"{n}MRO: \{Child.\_\_mro\_\_\}")
\end{verbatim}

\textbf{Multiple Inheritance Diagram:}

\begin{verbatim}
    Father        Mother
    |              |
    |              |
    +{-{-}{-}{-}{-}{-}+{-}{-}{-}{-}{-}{-}{-}+}
           |
         Child
\end{verbatim}

\textbf{Key Points:}

\begin{itemize}
\tightlist
\item
  \textbf{Multiple parents}: Child inherits from both Father and Mother
\item
  \textbf{Method Resolution Order (MRO)}: Determines which method is
  called
\item
  \textbf{Constructor calls}: Explicitly call parent constructors
\item
  \textbf{Diamond problem}: Python handles with MRO
\end{itemize}

\textbf{Output:}

\begin{verbatim}
Father constructor called
Mother constructor called  
Child constructor called

Family Details:
Father: John
Mother: Mary
Child: Alice

Method Resolution:
Father works as Engineer
\end{verbatim}

\end{solutionbox}
\begin{mnemonicbox}
``Multiple Parents MRO Constructor Diamond''

\end{mnemonicbox}

\end{document}
