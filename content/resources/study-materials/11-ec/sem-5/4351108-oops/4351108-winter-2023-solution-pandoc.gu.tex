\documentclass[10pt,a4paper]{article}

% content/resources/templates/preamble.tex
\usepackage[margin=0.6in]{geometry}
\author{Milav Dabgar}
\usepackage{amsmath,amssymb,amsthm}
\usepackage{booktabs}
\usepackage{multirow}
\usepackage{xcolor}
\usepackage{tcolorbox}
\tcbuselibrary{breakable,skins}
\usepackage[colorlinks=true,linkcolor=blue]{hyperref}
\usepackage{titlesec}
\usepackage{enumitem}
\usepackage{tikz}
\usepackage{pgfplots}
\usepackage{circuitikz}
\usepackage[version=4]{mhchem}
\usepackage{longtable}
\usepackage{array}
\usepackage{float}
\usepackage{caption}
\usepackage{listings}

\lstset{
  basicstyle=\small\ttfamily,
  breaklines=true,
  breakatwhitespace=false,
  postbreak=\mbox{\textcolor{red}{$\hookrightarrow$}\space},
  float=false,
  numbers=left,
  numberstyle=\tiny\color{gray},
  numbersep=10pt,
  xleftmargin=2em,
  keywordstyle=\color{blue},
  commentstyle=\color{green!60!black},
  stringstyle=\color{purple},
  backgroundcolor=\color{gray!5},
  showstringspaces=false,
  tabsize=2,
  captionpos=b,
  keepspaces=true,
  columns=flexible
}

\pgfplotsset{compat=1.18}
\usetikzlibrary{shapes,arrows,positioning,calc,patterns,decorations.pathmorphing,decorations.markings,arrows.meta}

% Color scheme
\definecolor{headcolor}{RGB}{0,102,204}
\definecolor{keycolor}{RGB}{220,20,60}
\definecolor{solutioncolor}{RGB}{34,139,34}
\definecolor{mnemoniccolor}{RGB}{148,0,211}
\definecolor{codecolor}{RGB}{0,0,100}

% Spacing
\setlength{\parskip}{3pt}
\setlist[itemize]{nosep}
\setlist[enumerate]{nosep}

% Title formatting
\titleformat{\section}{\Large\bfseries\color{headcolor}}{\thesection}{1em}{}
\titleformat{\subsection}{\large\bfseries\color{headcolor}}{\thesubsection}{1em}{}

% Pandoc tightlist compatibility
\providecommand{\tightlist}{%
  \setlength{\itemsep}{0pt}\setlength{\parskip}{0pt}}

% Pandoc longtable compatibility
\newcounter{none}
\def\thenone{}


% content/resources/templates/gujarati-boxes.tex
\usepackage{fontspec}
\usepackage{polyglossia}

% Set Gujarati as main language (document is primarily in Gujarati)
% Note: gloss-gujarati.ldf doesn't exist in polyglossia, but it will use hyphenation patterns
\setdefaultlanguage{gujarati}
\setotherlanguage{english}

% Configure Gujarati font properly
% Use Language=Default to prevent polyglossia from trying to add language-specific features
% that don't exist for Gujarati, which causes "empty feature" warnings
\newfontfamily\gujaratifont[Script=Gujarati,AutoFakeBold=2.5,AutoFakeSlant=0.3]{Noto Sans Gujarati}
\setmainfont[Script=Gujarati,AutoFakeBold=2.5,AutoFakeSlant=0.3]{Noto Sans Gujarati}
% Use Noto Sans Gujarati for monospace to support Gujarati in text
\setmonofont[Scale=0.9]{Noto Sans Gujarati}

% Configure English to use the same font
\newfontfamily\englishfont[Script=Gujarati,AutoFakeBold=2.5,AutoFakeSlant=0.3]{Noto Sans Gujarati}

% Translations for polyglossia
\gappto\captionsgujarati{
  \renewcommand{\tablename}{કોષ્ટક}
  \renewcommand{\figurename}{આકૃતિ}
}

% Helper for TikZ nodes to ensure Gujarati font
\newcommand{\gu}[1]{{\gujaratifont #1}}

% Custom environments
\newtcolorbox{solutionbox}{
    breakable,
    enhanced,
    colback=solutioncolor!5!white,
    colframe=solutioncolor!75!black,
    fonttitle=\bfseries,
    title=જવાબ
}

\newtcolorbox{solutionboxnobreak}{
 colback=solutioncolor!5!white,
 colframe=solutioncolor!75!black,
 fonttitle=\bfseries,
 title=જવાબ
}

\newtcolorbox{keyformula}{
 breakable,
 enhanced,
 colback=keycolor!5!white,
 colframe=keycolor!75!black,
 fonttitle=\bfseries,
 title=રાસાયણિક સમીકરણ/સૂત્ર
}

\newtcolorbox{mnemonicbox}{
 breakable,
 enhanced,
 colback=mnemoniccolor!5!white,
 colframe=mnemoniccolor!75!black,
 fonttitle=\bfseries,
 title=મેમરી ટ્રીક
}


\begin{document}

\begin{center}
{\Huge\bfseries\color{headcolor} Subject Name (Gujarati)}\\[5pt]
{\LARGE 4351108 -- Winter 2023}\\[3pt]
{\large Semester 1 Study Material}\\[3pt]
{\normalsize\textit{Detailed Solutions and Explanations}}
\end{center}

\vspace{10pt}

\subsection*{પ્રશ્ન 1(અ) [3
ગુણ]}\label{uxaaauxab0uxab6uxaa8-1uxa85-3-uxa97uxaa3}

\textbf{પાયથન પ્રોગ્રામિંગ લેન્ગવેજના કોઈ પણ 6 ઉપયોગો લખો.}

\begin{solutionbox}

\textbf{પાયથનના ઉપયોગોનું ટેબલ:}

{\def\LTcaptype{none} % do not increment counter
\begin{longtable}[]{@{}ll@{}}
\toprule\noalign{}
ઉપયોગ ક્ષેત્ર & વર્ણન \\
\midrule\noalign{}
\endhead
\bottomrule\noalign{}
\endlastfoot
\textbf{વેબ ડેવેલપમેન્ટ} & Django, Flask frameworks \\
\textbf{ડેટા સાયન્સ} & Analysis અને visualization \\
\textbf{મશીન લર્નિંગ} & AI model development \\
\textbf{ડેસ્કટોપ એપ્લિકેશન} & GUI using Tkinter, PyQt \\
\textbf{ગેમ ડેવેલપમેન્ટ} & Pygame library \\
\textbf{ઑટોમેશન} & Scripting અને testing \\
\end{longtable}
}

\textbf{સ્મરણ સૂત્ર:} ``Web Data Machine Desktop Game Auto''

\end{solutionbox}
\subsection*{પ્રશ્ન 1(બ) [4
ગુણ]}\label{uxaaauxab0uxab6uxaa8-1uxaac-4-uxa97uxaa3}

\textbf{પાયથન પ્રોગ્રામિંગ લેન્ગવેજની કોઈ પણ 8 વિશેષતાઓ લખો.}

\begin{solutionbox}

\textbf{પાયથનની વિશેષતાઓનું ટેબલ:}

{\def\LTcaptype{none} % do not increment counter
\begin{longtable}[]{@{}ll@{}}
\toprule\noalign{}
વિશેષતા & વર્ણન \\
\midrule\noalign{}
\endhead
\bottomrule\noalign{}
\endlastfoot
\textbf{સરળ સિન્ટેક્સ} & વાંચવા અને લખવામાં સરળ \\
\textbf{ઇન્ટરપ્રિટેડ} & Compilation ની જરૂર નથી \\
\textbf{ઑબ્જેક્ટ-ઓરિએન્ટેડ} & OOP concepts સપોર્ટ કરે છે \\
\textbf{ડાયનેમિક ટાઇપિંગ} & Variables ને type declaration જરૂરી નથી \\
\textbf{ક્રોસ-પ્લેટફોર્મ} & Multiple OS પર ચાલે છે \\
\textbf{મોટી લાઇબ્રેરીઓ} & Rich standard library \\
\textbf{ઓપન સોર્સ} & ઉપયોગ અને modify કરવા માટે મફત \\
\textbf{ઇન્ટરેક્ટિવ} & REPL environment \\
\end{longtable}
}

\textbf{સ્મરણ સૂત્ર:} ``Simple Interpreted Object Dynamic Cross Large Open
Interactive''

\end{solutionbox}
\subsection*{પ્રશ્ન 1(ક) [7
ગુણ]}\label{uxaaauxab0uxab6uxaa8-1uxa95-7-uxa97uxaa3}

\textbf{પાયથનની for અને while લૂપનું કાર્ય સમજાવો.}

\begin{solutionbox}

\textbf{For Loop:}

\begin{itemize}
\tightlist
\item
  \textbf{પુનરાવર્તન}: Sequences પર પુનરાવર્તન કરે છે (lists, strings, ranges)
\item
  \textbf{સિન્ટેક્સ}: \texttt{for\ variable\ in\ sequence:}
\item
  \textbf{આપોઆપ}: Iteration આપોઆપ handle કરે છે
\end{itemize}

\textbf{While Loop:}

\begin{itemize}
\tightlist
\item
  \textbf{શરત આધારિત}: જ્યાં સુધી શરત સાચી રહે છે
\item
  \textbf{મેન્યુઅલ નિયંત્રણ}: Programmer iteration નો નિયંત્રણ કરે છે
\item
  \textbf{જોખમ}: શરત કદી false ન બને તો infinite loop બની શકે છે
\end{itemize}

\textbf{ડાયાગ્રામ:}

\begin{verbatim}
    Start
      |
   Initialize
      |
    Condition? {-{-}{-}{-}No{-}{-}{-}{-} End}
      |Yes
    Execute
      |
   Update
      |
    (loop back)
\end{verbatim}

\textbf{કોડ ઉદાહરણ:}

\begin{verbatim}
\# For loop
for i in range(5):
    print(i)

\# While loop
i = 0
while i {} 5:
    print(i)
    i += 1
\end{verbatim}

\textbf{સ્મરણ સૂત્ર:} ``For આપોઆપ, While મેન્યુઅલ''

\end{solutionbox}
\subsection*{પ્રશ્ન 1(ક OR) [7
ગુણ]}\label{uxaaauxab0uxab6uxaa8-1uxa95-or-7-uxa97uxaa3}

\textbf{પાયથનના break, continue અને pass સ્ટેટમેન્ટના કાર્ય સમજાવો.}

\begin{solutionbox}

\textbf{Break Statement:}

\begin{itemize}
\tightlist
\item
  \textbf{બહાર નીકળવું}: સંપૂર્ણ loop ને terminate કરે છે
\item
  \textbf{ઉપયોગ}: જ્યારે કોઈ specific condition મળે છે
\item
  \textbf{અસર}: Control loop પછીના statement પર જાય છે
\end{itemize}

\textbf{Continue Statement:}

\begin{itemize}
\tightlist
\item
  \textbf{છોડીને આગળ}: ફક્ત current iteration skip કરે છે
\item
  \textbf{ઉપયોગ}: Iteration માં specific values skip કરવા માટે
\item
  \textbf{અસર}: Next iteration પર જાય છે
\end{itemize}

\textbf{Pass Statement:}

\begin{itemize}
\tightlist
\item
  \textbf{Placeholder}: કંઈ કરતું નથી, syntactic placeholder
\item
  \textbf{ઉપયોગ}: જ્યારે syntax statement જોઈએ પણ કોઈ action નહીં
\item
  \textbf{અસર}: કોઈ operation perform કરતું નથી
\end{itemize}

\textbf{કોડ ઉદાહરણો:}

\begin{verbatim}
\# Break
for i in range(10):
if

i == 5:

        break
    print(i)  \# prints 0,1,2,3,4

\# Continue
for i in range(5):
if

i == 2:

        continue
    print(i)  \# prints 0,1,3,4

\# Pass
if True:
    pass  \# placeholder
\end{verbatim}

\textbf{સ્મરણ સૂત્ર:} ``Break બહાર, Continue છોડીને, Pass રાહ''

\end{solutionbox}
\subsection*{પ્રશ્ન 2(અ) [3
ગુણ]}\label{uxaaauxab0uxab6uxaa8-2uxa85-3-uxa97uxaa3}

\textbf{લિસ્ટના દરેક ઘટકનું મૂલ્ય 1 થી વધારવા માટેનો પાયથન પ્રોગ્રામ લખો.}

\begin{solutionbox}

\textbf{કોડ:}

\begin{verbatim}
\# Method 1 {- Using for loop}
numbers = [1, 2, 3, 4, 5]
for i in range(len(numbers)):
    numbers[i] += 1
print(numbers)

\# Method 2 {- List comprehension}
numbers = [1, 2, 3, 4, 5]
result = [x + 1 for x in numbers]
print(result)
\end{verbatim}

\textbf{સ્મરણ સૂત્ર:} ``Loop Index અથવા Comprehension''

\end{solutionbox}
\subsection*{પ્રશ્ન 2(બ) [4
ગુણ]}\label{uxaaauxab0uxab6uxaa8-2uxaac-4-uxa97uxaa3}

\textbf{વપરાશકર્તા પાસેથી 3 સંખ્યા લઈ તેની સરેરાશ શોધવા માટેનો પાયથન પ્રોગ્રામ
લખો.}

\begin{solutionbox}

\textbf{કોડ:}

\begin{verbatim}
\# Input three numbers
num1 = float(input("Enter first number: "))
num2 = float(input("Enter second number: "))
num3 = float(input("Enter third number: "))

\# Calculate average
average = (num1 + num2 + num3) / 3

\# Display result
print(f"Average is: \{average\}")
\end{verbatim}

\textbf{મુખ્ય મુદ્દાઓ:}

\begin{itemize}
\tightlist
\item
  \textbf{ઇનપુટ}: દશાંશ સંખ્યાઓ માટે \texttt{float()} ઉપયોગ કરો
\item
  \textbf{સૂત્ર}: બધાનો સરવાળો કરીને સંખ્યા વડે ભાગો
\item
  \textbf{આઉટપુટ}: Formatting માટે f-string ઉપયોગ કરો
\end{itemize}

\textbf{સ્મરણ સૂત્ર:} ``Input Float, Sum Divide, Format Output''

\end{solutionbox}
\subsection*{પ્રશ્ન 2(ક) [7
ગુણ]}\label{uxaaauxab0uxab6uxaa8-2uxa95-7-uxa97uxaa3}

\textbf{પાયથનનો list ડેટા ટાઈપ વિસ્તારથી સમજાવો.}

\begin{solutionbox}

\textbf{લિસ્ટની લાક્ષણિકતાઓ:}

\begin{itemize}
\tightlist
\item
  \textbf{ક્રમબદ્ધ}: Elements sequence જાળવે છે
\item
  \textbf{બદલાવ પાત્ર}: બનાવ્યા પછી modify કરી શકાય છે
\item
  \textbf{વિવિધ પ્રકારની}: વિવિધ data types store કરી શકે છે
\item
  \textbf{ઇન્ડેક્સવાળી}: Index વડે elements ને access કરી શકાય છે (0-based)
\end{itemize}

\textbf{લિસ્ટ ઑપરેશન્સ ટેબલ:}

{\def\LTcaptype{none} % do not increment counter
\begin{longtable}[]{@{}lll@{}}
\toprule\noalign{}
ઑપરેશન & Syntax & વર્ણન \\
\midrule\noalign{}
\endhead
\bottomrule\noalign{}
\endlastfoot
\textbf{બનાવવું} & \texttt{list\ =\ [1,2,3]} & નવી list બનાવો \\
\textbf{એક્સેસ} & \texttt{list[0]} & Index વડે element મેળવો \\
\textbf{Append} & \texttt{list.append(4)} & અંતે element ઉમેરો \\
\textbf{Insert} & \texttt{list.insert(1,5)} & Specific position પર
ઉમેરો \\
\textbf{Remove} & \texttt{list.remove(2)} & પહેલું occurrence દૂર કરો \\
\textbf{Pop} & \texttt{list.pop()} & છેલ્લું element દૂર કરીને return કરો \\
\textbf{Slice} & \texttt{list[1:3]} & Sublist મેળવો \\
\end{longtable}
}

\textbf{કોડ ઉદાહરણ:}

\begin{verbatim}
\# List creation and operations
fruits = [{apple}, {banana}, {orange}]
fruits.append({mango})
fruits.insert(1, {grape})
print(fruits[0])  \# apple
print(len(fruits))  \# 5
\end{verbatim}

\textbf{સ્મરણ સૂત્ર:} ``ક્રમબદ્ધ બદલાવપાત્ર વિવિધપ્રકારની ઇન્ડેક્સવાળી''

\end{solutionbox}
\subsection*{પ્રશ્ન 2(અ OR) [3
ગુણ]}\label{uxaaauxab0uxab6uxaa8-2uxa85-or-3-uxa97uxaa3}

\textbf{for લૂપની મદદથી લિસ્ટના દરેક ઘટકનો સરવાળો શોધવા માટેનો પાયથન પ્રોગ્રામ
લખો.}

\begin{solutionbox}

\textbf{કોડ:}

\begin{verbatim}
\# Method 1 {- Traditional for loop}
numbers = [10, 20, 30, 40, 50]
total = 0
for num in numbers:
    total += num
print(f"Sum is: \{total\}")

\# Method 2 {- Using range and index}
numbers = [10, 20, 30, 40, 50]
total = 0
for i in range(len(numbers)):
    total += numbers[i]
print(f"Sum is: \{total\}")
\end{verbatim}

\textbf{સ્મરણ સૂત્ર:} ``શૂન્ય શરૂઆત, લૂપ ઉમેરો, કુલ પ્રિન્ટ''

\end{solutionbox}
\subsection*{પ્રશ્ન 2(બ OR) [4
ગુણ]}\label{uxaaauxab0uxab6uxaa8-2uxaac-or-4-uxa97uxaa3}

\textbf{વપરાશકર્તા પાસેથી લીધેલા મૂડલ, વ્યાજદર અને વર્ષ પરથી સાદું વ્યાજ શોધવા
માટેનો પાયથન પ્રોગ્રામ લખો.}

\begin{solutionbox}

\textbf{કોડ:}

\begin{verbatim}
\# Get input from user
principal = float(input("Enter principal amount: "))
rate = float(input("Enter rate of interest: "))
time = float(input("Enter time in years: "))

\# Calculate simple interest
simple\_interest = (principal * rate * time) / 100

\# Display results
print(f"Principal: \{principal\}")
print(f"Rate: \{rate\}\%")
print(f"Time: \{time\} years")
print(f"Simple Interest: \{simple\_interest\}")
print(f"Total Amount: \{principal + simple\_interest\}")
\end{verbatim}

\textbf{સૂત્ર:}

\begin{itemize}
\tightlist
\item
  \textbf{સાદું વ્યાજ} = (P \times R \times T) / 100
\item
  \textbf{કુલ રકમ} = મૂડલ + સાદું વ્યાજ
\end{itemize}

\textbf{સ્મરણ સૂત્ર:} ``મૂડલ દર સમય, ગુણો ભાગો સો''

\end{solutionbox}
\subsection*{પ્રશ્ન 2(ક OR) [7
ગુણ]}\label{uxaaauxab0uxab6uxaa8-2uxa95-or-7-uxa97uxaa3}

\textbf{પાયથનનો tuple ડેટા ટાઈપ વિસ્તારથી સમજાવો.}

\begin{solutionbox}

\textbf{ટ્યૂપલની લાક્ષણિકતાઓ:}

\begin{itemize}
\tightlist
\item
  \textbf{ક્રમબદ્ધ}: Elements sequence જાળવે છે
\item
  \textbf{અપરિવર્તનીય}: બનાવ્યા પછી modify કરી શકાતું નથી
\item
  \textbf{વિવિધ પ્રકારની}: વિવિધ data types store કરી શકે છે
\item
  \textbf{ઇન્ડેક્સવાળી}: Index વડે access કરી શકાય છે (0-based)
\end{itemize}

\textbf{ટ્યૂપલ ઑપરેશન્સ ટેબલ:}

{\def\LTcaptype{none} % do not increment counter
\begin{longtable}[]{@{}lll@{}}
\toprule\noalign{}
ઑપરેશન & Syntax & વર્ણન \\
\midrule\noalign{}
\endhead
\bottomrule\noalign{}
\endlastfoot
\textbf{બનાવવું} & \texttt{tuple\ =\ (1,2,3)} & નવું tuple બનાવો \\
\textbf{એક્સેસ} & \texttt{tuple[0]} & Index વડે element મેળવો \\
\textbf{Count} & \texttt{tuple.count(2)} & Occurrences ગિનો \\
\textbf{Index} & \texttt{tuple.index(3)} & પહેલો index શોધો \\
\textbf{Slice} & \texttt{tuple[1:3]} & Sub-tuple મેળવો \\
\textbf{Length} & \texttt{len(tuple)} & Tuple નું size મેળવો \\
\textbf{Concatenate} & \texttt{tuple1\ +\ tuple2} & Tuples જોડો \\
\end{longtable}
}

\textbf{કોડ ઉદાહરણ:}

\begin{verbatim}
\# Tuple creation and operations
coordinates = (10, 20, 30)
print(coordinates[0])  \# 10
print(len(coordinates))  \# 3
x, y, z = coordinates  \# tuple unpacking
new\_tuple = coordinates + (40, 50)
\end{verbatim}

\textbf{લિસ્ટ સાથે મુખ્ય તફાવત:}

\begin{itemize}
\tightlist
\item
  \textbf{અપરિવર્તનીય}: Elements બદલી શકાતા નથી
\item
  \textbf{પરફોર્મન્સ}: Lists કરતા વધુ ઝડપી
\item
  \textbf{ઉપયોગ}: Fixed data collections માટે
\end{itemize}

\textbf{સ્મરણ સૂત્ર:} ``ક્રમબદ્ધ અપરિવર્તનીય વિવિધપ્રકારની ઇન્ડેક્સવાળી''

\end{solutionbox}
\subsection*{પ્રશ્ન 3(અ) [3
ગુણ]}\label{uxaaauxab0uxab6uxaa8-3uxa85-3-uxa97uxaa3}

\textbf{random મોડ્યૂલની કોઈ પણ 3 મેથડ સમજાવો.}

\begin{solutionbox}

\textbf{Random મોડ્યૂલ મેથડ્સ ટેબલ:}

{\def\LTcaptype{none} % do not increment counter
\begin{longtable}[]{@{}lll@{}}
\toprule\noalign{}
મેથડ & Syntax & વર્ણન \\
\midrule\noalign{}
\endhead
\bottomrule\noalign{}
\endlastfoot
\textbf{random()} & \texttt{random.random()} & 0.0 થી 1.0 વચ્ચે float \\
\textbf{randint()} & \texttt{random.randint(1,10)} & આપેલી range વચ્ચે
integer \\
\textbf{choice()} & \texttt{random.choice(list)} & Sequence માંથી random
element \\
\end{longtable}
}

\textbf{કોડ ઉદાહરણ:}

\begin{verbatim}
import random

\# Generate random float
print(random.random())  \# 0.7234567

\# Generate random integer
print(random.randint(1, 10))  \# 7

\# Choose random element
colors = [{red}, {blue}, {green}]
print(random.choice(colors))  \# blue
\end{verbatim}

\textbf{સ્મરણ સૂત્ર:} ``Random Float, Randint Integer, Choice Select''

\end{solutionbox}
\subsection*{પ્રશ્ન 3(બ) [4
ગુણ]}\label{uxaaauxab0uxab6uxaa8-3uxaac-4-uxa97uxaa3}

\textbf{વપરાશકર્તા પાસેથી એક સ્ટ્રિંગ લઈને એમાંના દરેક `a' નું સ્થાન પ્રિન્ટ કરવાનો
પાયથન પ્રોગ્રામ લખો.}

\begin{solutionbox}

\textbf{કોડ:}

\begin{verbatim}
\# Get string from user
text = input("Enter a string: ")

\# Find all positions of {a}
positions = []
for i in range(len(text)):
    if text[i].lower() == {a}:
        positions.append(i)

\# Display results
if positions:
    print(f"Letter {a found at positions: }\{positions\}")
else:
    print("Letter {a not found in the string"})

\# Alternative method using enumerate
text = input("Enter a string: ")
for index, char in enumerate(text):
    if char.lower() == {a}:
        print(f"{a found at position }\{index\}")
\end{verbatim}

\textbf{મુખ્ય મુદ્દાઓ:}

\begin{itemize}
\tightlist
\item
  \textbf{Case-insensitive}: `a' અને `A' બંને શોધવા માટે \texttt{.lower()}
  ઉપયોગ કરો
\item
  \textbf{Index tracking}: range અથવા enumerate ઉપયોગ કરો
\item
  \textbf{આઉટપુટ ફોર્મેટ}: સ્પષ્ટ position indication
\end{itemize}

\textbf{સ્મરણ સૂત્ર:} ``લૂપ ઇન્ડેક્સ ચેક ઉમેરો પ્રિન્ટ''

\end{solutionbox}
\subsection*{પ્રશ્ન 3(ક) [7
ગુણ]}\label{uxaaauxab0uxab6uxaa8-3uxa95-7-uxa97uxaa3}

\textbf{પાયથનનો string ડેટા ટાઈપ વિસ્તારથી સમજાવો.}

\begin{solutionbox}

\textbf{સ્ટ્રિંગની લાક્ષણિકતાઓ:}

\begin{itemize}
\tightlist
\item
  \textbf{અપરિવર્તનીય}: બનાવ્યા પછી બદલી શકાતું નથી
\item
  \textbf{અનુક્રમ}: Characters નો ordered collection
\item
  \textbf{ઇન્ડેક્સવાળી}: Index વડે characters ને access કરી શકાય છે
\item
  \textbf{યુનિકોડ}: બધી ભાષાઓ અને symbols સપોર્ટ કરે છે
\end{itemize}

\textbf{સ્ટ્રિંગ મેથડ્સ ટેબલ:}

{\def\LTcaptype{none} % do not increment counter
\begin{longtable}[]{@{}lll@{}}
\toprule\noalign{}
મેથડ & ઉદાહરણ & વર્ણન \\
\midrule\noalign{}
\endhead
\bottomrule\noalign{}
\endlastfoot
\textbf{upper()} & \texttt{"hello".upper()} & Uppercase માં convert
કરો \\
\textbf{lower()} & \texttt{"HELLO".lower()} & Lowercase માં convert
કરો \\
\textbf{strip()} & \texttt{"\ hello\ ".strip()} & Whitespace દૂર કરો \\
\textbf{split()} & \texttt{"a,b,c".split(",")} & List માં split કરો \\
\textbf{replace()} & \texttt{"hello".replace("l","x")} & Substring
replace કરો \\
\textbf{find()} & \texttt{"hello".find("e")} & Substring index શોધો \\
\textbf{join()} & \texttt{",".join(["a","b"])} & List elements join
કરો \\
\end{longtable}
}

\textbf{સ્ટ્રિંગ ઑપરેશન્સ:}

\begin{verbatim}
\# String creation
name = "Python Programming"

\# String indexing and slicing
print(name[0])      \# P
print(name[0:6])    \# Python
print(name[{-}1])     \# g

\# String formatting
age = 25
message = f"I am \{age\} years old"
\end{verbatim}

\textbf{મુખ્ય વિશેષતાઓ:}

\begin{itemize}
\tightlist
\item
  \textbf{જોડાણ}: + operator વડે
\item
  \textbf{પુનરાવર્તન}: * operator વડે
\item
  \textbf{સભ્યપદ}: `in' operator વડે
\item
  \textbf{ફોર્મેટિંગ}: f-strings, .format(), \% formatting
\end{itemize}

\textbf{સ્મરણ સૂત્ર:} ``અપરિવર્તનીય અનુક્રમ ઇન્ડેક્સવાળી યુનિકોડ''

\end{solutionbox}
\subsection*{પ્રશ્ન 3(અ OR) [3
ગુણ]}\label{uxaaauxab0uxab6uxaa8-3uxa85-or-3-uxa97uxaa3}

\textbf{math મોડ્યૂલની કોઈ પણ 3 મેથડ સમજાવો.}

\begin{solutionbox}

\textbf{Math મોડ્યૂલ મેથડ્સ ટેબલ:}

{\def\LTcaptype{none} % do not increment counter
\begin{longtable}[]{@{}lll@{}}
\toprule\noalign{}
મેથડ & Syntax & વર્ણન \\
\midrule\noalign{}
\endhead
\bottomrule\noalign{}
\endlastfoot
\textbf{sqrt()} & \texttt{math.sqrt(16)} & Square root ગણતરી \\
\textbf{pow()} & \texttt{math.pow(2,3)} & Power ગણતરી \\
\textbf{ceil()} & \texttt{math.ceil(4.3)} & Integer માં round up \\
\end{longtable}
}

\textbf{કોડ ઉદાહરણ:}

\begin{verbatim}
import math

\# Square root
print(math.sqrt(25))    \# 5.0

\# Power
print(math.pow(2, 3))   \# 8.0

\# Ceiling
print(math.ceil(4.2))   \# 5
\end{verbatim}

\textbf{સ્મરણ સૂત્ર:} ``Square Root, Power Up, Ceiling Round''

\end{solutionbox}
\subsection*{પ્રશ્ન 3(બ OR) [4
ગુણ]}\label{uxaaauxab0uxab6uxaa8-3uxaac-or-4-uxa97uxaa3}

\textbf{વપરાશકર્તા પાસેથી એક સ્ટ્રિંગ લઈને એમાં રહેલા અંગ્રેજી સ્વરોની સંખ્યા શોધવાનો
પાયથન પ્રોગ્રામ લખો.}

\begin{solutionbox}

\textbf{કોડ:}

\begin{verbatim}
\# Get string from user
text = input("Enter a string: ")

\# Define vowels
vowels = "aeiouAEIOU"

\# Count vowels
vowel\_count = 0
for char in text:
    if char in vowels:
        vowel\_count += 1

\# Display result
print(f"Total vowels in {}\{text\}{: }\{vowel\_count\}")

\# Alternative method using list comprehension
text = input("Enter a string: ")
vowels = "aeiouAEIOU"
count = sum(1 for char in text if char in vowels)
print(f"Total vowels: \{count\}")
\end{verbatim}

\textbf{મુખ્ય મુદ્દાઓ:}

\begin{itemize}
\tightlist
\item
  \textbf{સ્વર વ્યાખ્યા}: બંને cases શામેલ કરો
\item
  \textbf{લૂપ કરો}: String ના દરેક character માં
\item
  \textbf{ગિનતી તર્ક}: Membership check કરો અને increment કરો
\end{itemize}

\textbf{સ્મરણ સૂત્ર:} ``સ્વર વ્યાખ્યા, લૂપ ચેક, ગિનતી વધારો''

\end{solutionbox}
\subsection*{પ્રશ્ન 3(ક OR) [7
ગુણ]}\label{uxaaauxab0uxab6uxaa8-3uxa95-or-7-uxa97uxaa3}

\textbf{પાયથનનો set ડેટા ટાઈપ વિસ્તારથી સમજાવો.}

\begin{solutionbox}

\textbf{સેટની લાક્ષણિકતાઓ:}

\begin{itemize}
\tightlist
\item
  \textbf{અક્રમ}: Elements નો કોઈ નિશ્ચિત sequence નથી
\item
  \textbf{બદલાવ પાત્ર}: Elements ઉમેરી/દૂર કરી શકાય છે
\item
  \textbf{અનન્ય}: Duplicate elements allowed નથી
\item
  \textbf{પુનરાવર્તનીય}: Elements માં loop કરી શકાય છે
\end{itemize}

\textbf{સેટ ઑપરેશન્સ ટેબલ:}

{\def\LTcaptype{none} % do not increment counter
\begin{longtable}[]{@{}lll@{}}
\toprule\noalign{}
ઑપરેશન & Syntax & વર્ણન \\
\midrule\noalign{}
\endhead
\bottomrule\noalign{}
\endlastfoot
\textbf{બનાવવું} & \texttt{set\ =\ \{1,2,3\}} & નવો set બનાવો \\
\textbf{Add} & \texttt{set.add(4)} & Single element ઉમેરો \\
\textbf{Remove} & \texttt{set.remove(2)} & Element દૂર કરો (error if not
found) \\
\textbf{Discard} & \texttt{set.discard(2)} & Element દૂર કરો (no
error) \\
\textbf{સંયોજન} & \texttt{set1\ \textbar{}\ set2} & Sets જોડો \\
\textbf{છેદ} & \texttt{set1\ \&\ set2} & સામાન્ય elements \\
\textbf{તફાવત} & \texttt{set1\ -\ set2} & ફક્ત set1 માંના elements \\
\end{longtable}
}

\textbf{સેટ ગણિતીય ઑપરેશન્સ:}

\begin{verbatim}
\# Set creation
A = \{1, 2, 3, 4\}
B = \{3, 4, 5, 6\}

\# Set operations
print(A | B)    \# Union: \{1,2,3,4,5,6\}
print(A \& B)    \# Intersection: \{3,4\}
print(A {-} B)    \# Difference: \{1,2\}
print(A \^{} B)    \# Symmetric difference: \{1,2,5,6\}
\end{verbatim}

\textbf{મુખ્ય ઉપયોગો:}

\begin{itemize}
\tightlist
\item
  \textbf{Duplicates દૂર કરવા}: Lists માંથી
\item
  \textbf{ગણિતીય ઑપરેશન્સ}: સંયોજન, છેદ
\item
  \textbf{સભ્યપદ પરીક્ષા}: ઝડપી lookup
\end{itemize}

\textbf{સ્મરણ સૂત્ર:} ``અક્રમ બદલાવપાત્ર અનન્ય પુનરાવર્તનીય''

\end{solutionbox}
\subsection*{પ્રશ્ન 4(અ) [3
ગુણ]}\label{uxaaauxab0uxab6uxaa8-4uxa85-3-uxa97uxaa3}

\textbf{પાયથનમાં ક્લાસ શું છે? તે ઓબ્જેક્ટથી કઈ રીતે અલગ છે?}

\begin{solutionbox}

\textbf{ક્લાસ વિ. ઓબ્જેક્ટ સરખામણી:}

{\def\LTcaptype{none} % do not increment counter
\begin{longtable}[]{@{}lll@{}}
\toprule\noalign{}
પાસું & ક્લાસ & ઓબ્જેક્ટ \\
\midrule\noalign{}
\endhead
\bottomrule\noalign{}
\endlastfoot
\textbf{વ્યાખ્યા} & Blueprint અથવા template & ક્લાસનું instance \\
\textbf{મેમરી} & કોઈ memory allocate થતી નથી & Memory allocate થાય છે \\
\textbf{અસ્તિત્વ} & Logical entity & Physical entity \\
\textbf{બનાવટ} & class keyword ઉપયોગ કરીને & Class constructor ઉપયોગ
કરીને \\
\end{longtable}
}

\textbf{ઉદાહરણ:}

\begin{verbatim}
\# Class definition (blueprint)
class Car:
    def \_\_init\_\_(self, brand):
        self.brand = brand

\# Object creation (instances)
car1 = Car("Toyota")  \# Object 1
car2 = Car("Honda")   \# Object 2
\end{verbatim}

\textbf{મુખ્ય મુદ્દાઓ:}

\begin{itemize}
\tightlist
\item
  \textbf{ક્લાસ}: Properties અને methods વ્યાખ્યાયિત કરતું template
\item
  \textbf{ઓબ્જેક્ટ}: Specific values સાથેનું actual instance
\item
  \textbf{સંબંધ}: એક ક્લાસ, અનેક ઓબ્જેક્ટ્સ
\end{itemize}

\textbf{સ્મરણ સૂત્ર:} ``ક્લાસ Blueprint, ઓબ્જેક્ટ Instance''

\end{solutionbox}
\subsection*{પ્રશ્ન 4(બ) [4
ગુણ]}\label{uxaaauxab0uxab6uxaa8-4uxaac-4-uxa97uxaa3}

\textbf{dictionary ડેટા ટાઈપની કોઈ પણ 4 મેથડ સમજાવો.}

\begin{solutionbox}

\textbf{Dictionary મેથડ્સ ટેબલ:}

{\def\LTcaptype{none} % do not increment counter
\begin{longtable}[]{@{}lll@{}}
\toprule\noalign{}
મેથડ & Syntax & વર્ણન \\
\midrule\noalign{}
\endhead
\bottomrule\noalign{}
\endlastfoot
\textbf{keys()} & \texttt{dict.keys()} & બધી keys મેળવો \\
\textbf{values()} & \texttt{dict.values()} & બધી values મેળવો \\
\textbf{items()} & \texttt{dict.items()} & Key-value pairs મેળવો \\
\textbf{get()} &
\texttt{dict.get(\textquotesingle{}key\textquotesingle{})} & Value
સુરક્ષિત રીતે મેળવો \\
\end{longtable}
}

\textbf{કોડ ઉદાહરણ:}

\begin{verbatim}
student = \{{name}: {John}, {age}: 20, {grade}: {A}\}

\# Dictionary methods
print(student.keys())    \# dict\_keys([{name, age, grade])}
print(student.values())  \# dict\_values([{John, 20, A])}
print(student.items())   \# dict\_items([({name, John), ...])}
print(student.get({name}))  \# John
\end{verbatim}

\textbf{સ્મરણ સૂત્ર:} ``Keys Values Items Get''

\end{solutionbox}
\subsection*{પ્રશ્ન 4(ક) [7
ગુણ]}\label{uxaaauxab0uxab6uxaa8-4uxa95-7-uxa97uxaa3}

\textbf{કોઈ કાર્યો કરવા માટે યુઝર ડિફાઈન્ડ મોડ્યૂલ બનાવી તેને ઈમ્પોર્ટ કરી તેના
ફંક્શનનો ઉપયોગ કરવાનો પાયથન પ્રોગ્રામ લખો.}

\begin{solutionbox}

\textbf{મોડ્યૂલ બનાવવું (math\_operations.py):}

\begin{verbatim}
\# math\_operations.py
def add(a, b):
    """Add two numbers"""
    return a + b

def multiply(a, b):
    """Multiply two numbers"""
    return a * b

def factorial(n):
    """Calculate factorial"""
    if n {=} 1:
        return 1
    return n * factorial(n {-} 1)

PI = 3.14159

def circle\_area(radius):
    """Calculate circle area"""
    return PI * radius * radius
\end{verbatim}

\textbf{મુખ્ય પ્રોગ્રામ (main.py):}

\begin{verbatim}
\# Import entire module
import math\_operations

\# Use module functions
result1 = math\_operations.add(5, 3)
result2 = math\_operations.multiply(4, 6)
result3 = math\_operations.factorial(5)
area = math\_operations.circle\_area(5)

print(f"Addition: \{result1\}")
print(f"Multiplication: \{result2\}")
print(f"Factorial: \{result3\}")
print(f"Circle Area: \{area\}")

\# Import specific functions
from math\_operations import add, multiply
print(f"Direct call: \{add(10, 20)\}")
\end{verbatim}

\textbf{મુખ્ય મુદ્દાઓ:}

\begin{itemize}
\tightlist
\item
  \textbf{મોડ્યૂલ બનાવવું}: Functions સાથે અલગ .py ફાઈલ
\item
  \textbf{Import પદ્ધતિઓ}: import module અથવા from module import function
\item
  \textbf{ઉપયોગ}: module.function() અથવા direct function() વડે access
\end{itemize}

\textbf{સ્મરણ સૂત્ર:} ``બનાવો Import ઉપયોગ''

\end{solutionbox}
\subsection*{પ્રશ્ન 4(અ OR) [3
ગુણ]}\label{uxaaauxab0uxab6uxaa8-4uxa85-or-3-uxa97uxaa3}

\textbf{પાયથન ક્લાસની મેથડ્સના પ્રકારો ટૂંકમાં સમજાવો.}

\begin{solutionbox}

\textbf{મેથડ્સના પ્રકારોનું ટેબલ:}

{\def\LTcaptype{none} % do not increment counter
\begin{longtable}[]{@{}
  >{\raggedright\arraybackslash}p{(\linewidth - 4\tabcolsep) * \real{0.3824}}
  >{\raggedright\arraybackslash}p{(\linewidth - 4\tabcolsep) * \real{0.2353}}
  >{\raggedright\arraybackslash}p{(\linewidth - 4\tabcolsep) * \real{0.3824}}@{}}
\toprule\noalign{}
\begin{minipage}[b]{\linewidth}\raggedright
મેથડ પ્રકાર
\end{minipage} & \begin{minipage}[b]{\linewidth}\raggedright
Syntax
\end{minipage} & \begin{minipage}[b]{\linewidth}\raggedright
વર્ણન
\end{minipage} \\
\midrule\noalign{}
\endhead
\bottomrule\noalign{}
\endlastfoot
\textbf{Instance Method} & \texttt{def\ method(self):} & Instance
variables ને access કરે છે \\
\textbf{Class Method} & \texttt{@classmethod\ def\ method(cls):} & Class
variables ને access કરે છે \\
\textbf{Static Method} & \texttt{@staticmethod\ def\ method():} &
Class/instance થી સ્વતંત્ર \\
\end{longtable}
}

\textbf{ઉદાહરણ:}

\begin{verbatim}
class MyClass:
    class\_var = "Class Variable"
    
    def instance\_method(self):  \# Instance method
        return "Instance method"
    
    @classmethod
    def class\_method(cls):      \# Class method
        return cls.class\_var
    
    @staticmethod
    def static\_method():        \# Static method
        return "Static method"
\end{verbatim}

\textbf{સ્મરણ સૂત્ર:} ``Instance Self, Class Cls, Static કંઈ નહીં''

\end{solutionbox}
\subsection*{પ્રશ્ન 4(બ OR) [4
ગુણ]}\label{uxaaauxab0uxab6uxaa8-4uxaac-or-4-uxa97uxaa3}

\textbf{string ડેટા ટાઈપની કોઈ પણ 4 મેથડ સમજાવો.}

\begin{solutionbox}

\textbf{String મેથડ્સ ટેબલ:}

{\def\LTcaptype{none} % do not increment counter
\begin{longtable}[]{@{}
  >{\raggedright\arraybackslash}p{(\linewidth - 4\tabcolsep) * \real{0.2759}}
  >{\raggedright\arraybackslash}p{(\linewidth - 4\tabcolsep) * \real{0.2759}}
  >{\raggedright\arraybackslash}p{(\linewidth - 4\tabcolsep) * \real{0.4483}}@{}}
\toprule\noalign{}
\begin{minipage}[b]{\linewidth}\raggedright
મેથડ
\end{minipage} & \begin{minipage}[b]{\linewidth}\raggedright
Syntax
\end{minipage} & \begin{minipage}[b]{\linewidth}\raggedright
વર્ણન
\end{minipage} \\
\midrule\noalign{}
\endhead
\bottomrule\noalign{}
\endlastfoot
\textbf{startswith()} &
\texttt{str.startswith(\textquotesingle{}pre\textquotesingle{})} &
Substring થી શરૂ થાય છે કે ચેક કરો \\
\textbf{endswith()} &
\texttt{str.endswith(\textquotesingle{}suf\textquotesingle{})} &
Substring થી અંત થાય છે કે ચેક કરો \\
\textbf{isdigit()} & \texttt{str.isdigit()} & બધા digits છે કે ચેક કરો \\
\textbf{count()} &
\texttt{str.count(\textquotesingle{}sub\textquotesingle{})} & Substring
ની occurrences ગિનો \\
\end{longtable}
}

\textbf{કોડ ઉદાહરણ:}

\begin{verbatim}
text = "Hello World 123"

\# String methods
print(text.startswith({Hello}))  \# True
print(text.endswith({123}))      \# True
print({123}.isdigit())           \# True
print(text.count({l}))           \# 3
\end{verbatim}

\textbf{સ્મરણ સૂત્ર:} ``Start End Digit Count''

\end{solutionbox}
\subsection*{પ્રશ્ન 4(ક OR) [7
ગુણ]}\label{uxaaauxab0uxab6uxaa8-4uxa95-or-7-uxa97uxaa3}

\textbf{રિકર્સીવ ફંક્શનની મદદથી આપેલ નંબરનો ફેક્ટોરીયલ શોધવા માટેનો પાયથન
પ્રોગ્રામ લખો.}

\begin{solutionbox}

\textbf{કોડ:}

\begin{verbatim}
def factorial(n):
    """
    Recursion વડે factorial ગણો
    Base case: factorial(0) = 1, factorial(1) = 1
    Recursive case: factorial(n) = n * factorial(n{-1)}
    """
    \# Base case
if

n == 0 or

n == 1:

        return 1
    
    \# Recursive case
    else:
        return n * factorial(n {-} 1)

\# Main program
try:
    num = int(input("Enter a number: "))
    
    if num {} 0:
        print("Factorial not defined for negative numbers")
    else:
        result = factorial(num)
        print(f"Factorial of \{num\} is \{result\}")
        
except ValueError:
    print("Please enter a valid integer")

\# Test cases
print(f"Factorial of 5: \{factorial(5)\}")  \# 120
print(f"Factorial of 0: \{factorial(0)\}")  \# 1
\end{verbatim}

\textbf{Recursion Flow:}

\begin{verbatim}
factorial(5)
    |
5 * factorial(4)
        |
    4 * factorial(3)
            |
        3 * factorial(2)
                |
            2 * factorial(1)
                    |
                return 1

Result: 5 * 4 * 3 * 2 * 1 = 120
\end{verbatim}

\textbf{મુખ્ય મુદ્દાઓ:}

\begin{itemize}
\tightlist
\item
  \textbf{Base case}: Recursion બંધ કરે છે (n=0 અથવા n=1)
\item
  \textbf{Recursive case}: Function પોતાને call કરે છે
\item
  \textbf{Error handling}: Negative input માટે ચેક કરો
\end{itemize}

\textbf{સ્મરણ સૂત્ર:} ``Base બંધ, Recursive Call, Error ચેક''

\end{solutionbox}
\subsection*{પ્રશ્ન 5(અ) [3
ગુણ]}\label{uxaaauxab0uxab6uxaa8-5uxa85-3-uxa97uxaa3}

\textbf{સિંગલ ઇન્હેરિટન્સ બતાવવા માટેનો પાયથન પ્રોગ્રામ લખો.}

\begin{solutionbox}

\textbf{કોડ:}

\begin{verbatim}
\# Parent class
class Animal:
    def \_\_init\_\_(self, name):
        self.name = name
    
    def speak(self):
        print(f"\{self.name\} makes a sound")
    
    def eat(self):
        print(f"\{self.name\} is eating")

\# Child class inheriting from Animal
class Dog(Animal):
    def \_\_init\_\_(self, name, breed):
        super().\_\_init\_\_(name)  \# Call parent constructor
        self.breed = breed
    
    def bark(self):
        print(f"\{self.name\} is barking")
    
    def speak(self):  \# Override parent method
        print(f"\{self.name\} says Woof!")

\# Create objects and test
dog = Dog("Buddy", "Golden Retriever")
dog.speak()  \# Buddy says Woof!
dog.eat()    \# Buddy is eating (inherited)
dog.bark()   \# Buddy is barking (own method)
\end{verbatim}

\textbf{સ્મરણ સૂત્ર:} ``Parent Child Inherit Override''

\end{solutionbox}
\subsection*{પ્રશ્ન 5(બ) [4
ગુણ]}\label{uxaaauxab0uxab6uxaa8-5uxaac-4-uxa97uxaa3}

\textbf{પાયથન ક્લાસમાં કન્સ્ટ્રક્ટરનું મહત્વ સમજાવો.}

\begin{solutionbox}

\textbf{કન્સ્ટ્રક્ટરનું મહત્વ:}

{\def\LTcaptype{none} % do not increment counter
\begin{longtable}[]{@{}ll@{}}
\toprule\noalign{}
પાસું & વર્ણન \\
\midrule\noalign{}
\endhead
\bottomrule\noalign{}
\endlastfoot
\textbf{ઇનિશિયલાઇઝેશન} & ઓબ્જેક્ટ બનાવવામાં આવે ત્યારે આપોઆપ call થાય છે \\
\textbf{સેટઅપ} & Instance variables ને values સાથે initialize કરે છે \\
\textbf{મેમરી} & Object attributes માટે memory allocate કરે છે \\
\textbf{વેલિડેશન} & Creation દરમિયાન input parameters validate કરે છે \\
\end{longtable}
}

\textbf{કન્સ્ટ્રક્ટરના પ્રકારો:}

\begin{verbatim}
class Student:
    \# Default constructor
    def \_\_init\_\_(self):
        self.name = "Unknown"
        self.age = 0
    
    \# Parameterized constructor
    def \_\_init\_\_(self, name, age):
        self.name = name
        self.age = age
        print(f"Student \{name\} created")
    
    \# Constructor with default parameters
    def \_\_init\_\_(self, name="Unknown", age=0):
        self.name = name
        self.age = age
\end{verbatim}

\textbf{મુખ્ય ફાયદાઓ:}

\begin{itemize}
\tightlist
\item
  \textbf{આપોઆપ execution}: Manual call કરવાની જરૂર નથી
\item
  \textbf{ઓબ્જેક્ટ state}: યોગ્ય initialization ensure કરે છે
\item
  \textbf{કોડ પુનઃઉપયોગ}: એક જગ્યાએ common setup code
\end{itemize}

\textbf{સ્મરણ સૂત્ર:} ``Initialize Setup Memory Validate''

\end{solutionbox}
\subsection*{પ્રશ્ન 5(ક) [7
ગુણ]}\label{uxaaauxab0uxab6uxaa8-5uxa95-7-uxa97uxaa3}

\textbf{ઇન્હેરિટન્સ દ્વારા થતું મેથડ ઓવરરાઇડિંગ બતાવવા માટેનો પાયથન પ્રોગ્રામ લખો.}

\begin{solutionbox}

\textbf{કોડ:}

\begin{verbatim}
\# Base class
class Shape:
    def \_\_init\_\_(self, name):
        self.name = name
    
    def area(self):
        print(f"Area calculation for \{self.name\}")
        return 0
    
    def display(self):
        print(f"This is a \{self.name\}")

\# Derived class 1
class Rectangle(Shape):
    def \_\_init\_\_(self, length, width):
        super().\_\_init\_\_("Rectangle")
        self.length = length
        self.width = width
    
    \# Override area method
    def area(self):
        area\_value = self.length * self.width
        print(f"Rectangle area: \{area\_value\}")
        return area\_value

\# Derived class 2
class Circle(Shape):
    def \_\_init\_\_(self, radius):
        super().\_\_init\_\_("Circle")
        self.radius = radius
    
    \# Override area method
    def area(self):
        area\_value = 3.14 * self.radius * self.radius
        print(f"Circle area: \{area\_value\}")
        return area\_value
    
    \# Override display method
    def display(self):
        super().display()  \# Call parent method
        print(f"Radius: \{self.radius\}")

\# Test method overriding
shapes = [
    Rectangle(5, 4),
    Circle(3),
    Shape("Generic Shape")
]

for shape in shapes:
    shape.display()
    shape.area()
    print("{-"} * 20)
\end{verbatim}

\textbf{મેથડ ઓવરરાઇડિંગ ડાયાગ્રામ:}

\begin{verbatim}
    Shape (Base)
    |{-{-} area()}
    |{-{-} display()}
         |
    Rectangle    Circle
    |{-{-} area()   |{-}{-} area()}
                 |{-{-} display()}
\end{verbatim}

\textbf{મુખ્ય મુદ્દાઓ:}

\begin{itemize}
\tightlist
\item
  \textbf{સમાન મેથડ નામ}: Parent અને child classes માં
\item
  \textbf{અલગ implementation}: Child class specific logic આપે છે
\item
  \textbf{Runtime નિર્ણય}: Object type આધારે યોગ્ય method call થાય છે
\item
  \textbf{Super() ઉપયોગ}: Parent class method ને access કરવા માટે
\end{itemize}

\textbf{સ્મરણ સૂત્ર:} ``સમાન નામ અલગ તર્ક Runtime નિર્ણય''

\end{solutionbox}
\subsection*{પ્રશ્ન 5(અ OR) [3
ગુણ]}\label{uxaaauxab0uxab6uxaa8-5uxa85-or-3-uxa97uxaa3}

\textbf{પાયથનમાં ડેટા એન્કેપ્સ્યુલેશનનો ખ્યાલ સમજાવો.}

\begin{solutionbox}

\textbf{ડેટા એન્કેપ્સ્યુલેશન:}

{\def\LTcaptype{none} % do not increment counter
\begin{longtable}[]{@{}ll@{}}
\toprule\noalign{}
પાસું & વર્ણન \\
\midrule\noalign{}
\endhead
\bottomrule\noalign{}
\endlastfoot
\textbf{વ્યાખ્યા} & Data અને methods ને એકસાથે બાંધવું \\
\textbf{એક્સેસ કન્ટ્રોલ} & Internal data ને direct access પર પ્રતિબંધ \\
\textbf{ડેટા છુપાવવું} & Internal implementation બહારથી છુપાવવું \\
\textbf{ઇન્ટરફેસ} & Methods દ્વારા controlled access પ્રદાન કરવું \\
\end{longtable}
}

\textbf{અમલીકરણ:}

\begin{verbatim}
class BankAccount:
    def \_\_init\_\_(self, balance):
        self.\_\_balance = balance  \# Private attribute
    
    def deposit(self, amount):    \# Public method
        if amount {} 0:
            self.\_\_balance += amount
    
    def get\_balance(self):        \# Public method
        return self.\_\_balance
    
    def \_\_validate(self):         \# Private method
        return self.\_\_balance {=} 0

\# Usage
account = BankAccount(1000)
account.deposit(500)
print(account.get\_balance())  \# 1500
\# print(account.\_\_balance)    \# Error {- cannot access private}
\end{verbatim}

\textbf{સ્મરણ સૂત્ર:} ``બાંધો ડેટા છુપાવો ઇન્ટરફેસ''

\end{solutionbox}
\subsection*{પ્રશ્ન 5(બ OR) [4
ગુણ]}\label{uxaaauxab0uxab6uxaa8-5uxaac-or-4-uxa97uxaa3}

\textbf{પાયથનમાં એબ્સ્ટ્રેક્ટ ક્લાસનો ખ્યાલ સમજાવો.}

\begin{solutionbox}

\textbf{એબ્સ્ટ્રેક્ટ ક્લાસ:}

{\def\LTcaptype{none} % do not increment counter
\begin{longtable}[]{@{}ll@{}}
\toprule\noalign{}
કન્સેપ્ટ & વર્ણન \\
\midrule\noalign{}
\endhead
\bottomrule\noalign{}
\endlastfoot
\textbf{વ્યાખ્યા} & સીધા instantiate ન થઈ શકતો ક્લાસ \\
\textbf{એબ્સ્ટ્રેક્ટ મેથડ્સ} & Declared પણ implemented નથી \\
\textbf{અમલીકરણ} & Subclasses એ abstract methods implement કરવા જોઈએ \\
\textbf{હેતુ} & Related classes માટે common interface વ્યાખ્યાયિત કરવો \\
\end{longtable}
}

\textbf{ABC વડે અમલીકરણ:}

\begin{verbatim}
from abc import ABC, abstractmethod

class Animal(ABC):  \# Abstract class
    @abstractmethod
    def make\_sound(self):  \# Abstract method
        pass
    
    def sleep(self):       \# Concrete method
        print("Animal is sleeping")

class Dog(Animal):
    def make\_sound(self):  \# Must implement
        print("Woof!")

class Cat(Animal):
    def make\_sound(self):  \# Must implement
        print("Meow!")

\# Usage
dog = Dog()
dog.make\_sound()  \# Woof!
\# animal = Animal()  \# Error {- cannot instantiate}
\end{verbatim}

\textbf{મુખ્ય વિશેષતાઓ:}

\begin{itemize}
\tightlist
\item
  \textbf{Instantiate ન થઈ શકે}: Abstract class objects બનાવી શકાતા નથી
\item
  \textbf{અમલીકરણ દબાણ}: Subclasses એ abstract methods implement કરવા
  જોઈએ
\item
  \textbf{કોમન ઇન્ટરફેસ}: સુસંગત method signatures ensure કરે છે
\end{itemize}

\textbf{સ્મરણ સૂત્ર:} ``Instantiate ન થાય અમલીકરણ દબાણ કોમન ઇન્ટરફેસ''

\end{solutionbox}
\subsection*{પ્રશ્ન 5(ક OR) [7
ગુણ]}\label{uxaaauxab0uxab6uxaa8-5uxa95-or-7-uxa97uxaa3}

\textbf{મલ્ટિપલ ઇન્હેરિટન્સ બતાવવા માટેનો પાયથન પ્રોગ્રામ લખો.}

\begin{solutionbox}

\textbf{કોડ:}

\begin{verbatim}
\# First parent class
class Father:
    def \_\_init\_\_(self):
        self.father\_name = "John"
        print("Father constructor called")
    
    def show\_father(self):
        print(f"Father: \{self.father\_name\}")
    
    def work(self):
        print("Father works as Engineer")

\# Second parent class
class Mother:
    def \_\_init\_\_(self):
        self.mother\_name = "Mary"
        print("Mother constructor called")
    
    def show\_mother(self):
        print(f"Mother: \{self.mother\_name\}")
    
    def work(self):
        print("Mother works as Doctor")

\# Child class inheriting from both parents
class Child(Father, Mother):
    def \_\_init\_\_(self):
        Father.\_\_init\_\_(self)  \# Call father{s constructor}
        Mother.\_\_init\_\_(self)  \# Call mother{s constructor}
        self.child\_name = "Alice"
        print("Child constructor called")
    
    def show\_child(self):
        print(f"Child: \{self.child\_name\}")
    
    def show\_family(self):
        self.show\_father()
        self.show\_mother()
        self.show\_child()

\# Create child object and test
child = Child()
print("{n}Family Details:")
child.show\_family()
print("{n}Method Resolution:")
child.work()  \# Calls Father{s work method (MRO)}

\# Check Method Resolution Order
print(f"{n}MRO: \{Child.\_\_mro\_\_\}")
\end{verbatim}

\textbf{મલ્ટિપલ ઇન્હેરિટન્સ ડાયાગ્રામ:}

\begin{verbatim}
    Father        Mother
    |              |
    |              |
    +{-{-}{-}{-}{-}{-}+{-}{-}{-}{-}{-}{-}{-}+}
           |
         Child
\end{verbatim}

\textbf{મુખ્ય મુદ્દાઓ:}

\begin{itemize}
\tightlist
\item
  \textbf{અનેક પેરેન્ટ્સ}: Child બંને Father અને Mother થી inherit કરે છે
\item
  \textbf{મેથડ રિઝોલ્યુશન ઓર્ડર (MRO)}: કયો method call થશે તે નક્કી કરે છે
\item
  \textbf{કન્સ્ટ્રક્ટર કોલ્સ}: Parent constructors ને સ્પષ્ટપણે call કરવા
\item
  \textbf{ડાયમંડ પ્રોબ્લેમ}: Python MRO વડે handle કરે છે
\end{itemize}

\textbf{આઉટપુટ:}

\begin{verbatim}
Father constructor called
Mother constructor called  
Child constructor called

Family Details:
Father: John
Mother: Mary
Child: Alice

Method Resolution:
Father works as Engineer
\end{verbatim}

\textbf{સ્મરણ સૂત્ર:} ``અનેક પેરેન્ટ્સ MRO કન્સ્ટ્રક્ટર ડાયમંડ''

\end{solutionbox}

\end{document}
