\documentclass[10pt,a4paper]{article}

% content/resources/templates/preamble.tex
\usepackage[margin=0.6in]{geometry}
\author{Milav Dabgar}
\usepackage{amsmath,amssymb,amsthm}
\usepackage{booktabs}
\usepackage{multirow}
\usepackage{xcolor}
\usepackage{tcolorbox}
\tcbuselibrary{breakable,skins}
\usepackage[colorlinks=true,linkcolor=blue]{hyperref}
\usepackage{titlesec}
\usepackage{enumitem}
\usepackage{tikz}
\usepackage{pgfplots}
\usepackage{circuitikz}
\usepackage[version=4]{mhchem}
\usepackage{longtable}
\usepackage{array}
\usepackage{float}
\usepackage{caption}
\usepackage{listings}

\lstset{
  basicstyle=\small\ttfamily,
  breaklines=true,
  breakatwhitespace=false,
  postbreak=\mbox{\textcolor{red}{$\hookrightarrow$}\space},
  float=false,
  numbers=left,
  numberstyle=\tiny\color{gray},
  numbersep=10pt,
  xleftmargin=2em,
  keywordstyle=\color{blue},
  commentstyle=\color{green!60!black},
  stringstyle=\color{purple},
  backgroundcolor=\color{gray!5},
  showstringspaces=false,
  tabsize=2,
  captionpos=b,
  keepspaces=true,
  columns=flexible
}

\pgfplotsset{compat=1.18}
\usetikzlibrary{shapes,arrows,positioning,calc,patterns,decorations.pathmorphing,decorations.markings,arrows.meta}

% Color scheme
\definecolor{headcolor}{RGB}{0,102,204}
\definecolor{keycolor}{RGB}{220,20,60}
\definecolor{solutioncolor}{RGB}{34,139,34}
\definecolor{mnemoniccolor}{RGB}{148,0,211}
\definecolor{codecolor}{RGB}{0,0,100}

% Spacing
\setlength{\parskip}{3pt}
\setlist[itemize]{nosep}
\setlist[enumerate]{nosep}

% Title formatting
\titleformat{\section}{\Large\bfseries\color{headcolor}}{\thesection}{1em}{}
\titleformat{\subsection}{\large\bfseries\color{headcolor}}{\thesubsection}{1em}{}

% Pandoc tightlist compatibility
\providecommand{\tightlist}{%
  \setlength{\itemsep}{0pt}\setlength{\parskip}{0pt}}

% Pandoc longtable compatibility
\newcounter{none}
\def\thenone{}


% content/resources/templates/english-boxes.tex
% This file is currently empty - it exists to maintain consistency with the import structure.
% Add custom environments here if needed in the future.


\begin{document}

\begin{center}
{\Huge\bfseries\color{headcolor} Subject Name Solutions}\\[5pt]
{\LARGE 4351108 -- Winter 2024}\\[3pt]
{\large Semester 1 Study Material}\\[3pt]
{\normalsize\textit{Detailed Solutions and Explanations}}
\end{center}

\vspace{10pt}

\subsection*{Question 1(a) [3 marks]}\label{q1a}

\textbf{List out features of python programming language.}

\begin{solutionbox}

{\def\LTcaptype{none} % do not increment counter
\begin{longtable}[]{@{}ll@{}}
\toprule\noalign{}
Feature & Description \\
\midrule\noalign{}
\endhead
\bottomrule\noalign{}
\endlastfoot
\textbf{Simple \& Easy} & Clean, readable syntax \\
\textbf{Free \& Open Source} & No cost, community driven \\
\textbf{Cross-platform} & Runs on Windows, Linux, Mac \\
\textbf{Interpreted} & No compilation needed \\
\textbf{Object-Oriented} & Supports classes and objects \\
\textbf{Large Libraries} & Rich standard library \\
\end{longtable}
}

\end{solutionbox}
\begin{mnemonicbox}
``Simple Free Cross Interpreted Object Large''

\end{mnemonicbox}
\begin{center}\rule{0.5\linewidth}{0.5pt}\end{center}

\subsection*{Question 1(b) [4 marks]}\label{q1b}

\textbf{Write applications of python programming language.}

\begin{solutionbox}

{\def\LTcaptype{none} % do not increment counter
\begin{longtable}[]{@{}ll@{}}
\toprule\noalign{}
Application Area & Examples \\
\midrule\noalign{}
\endhead
\bottomrule\noalign{}
\endlastfoot
\textbf{Web Development} & Django, Flask frameworks \\
\textbf{Data Science} & NumPy, Pandas, Matplotlib \\
\textbf{Machine Learning} & TensorFlow, Scikit-learn \\
\textbf{Desktop GUI} & Tkinter, PyQt applications \\
\textbf{Game Development} & Pygame library \\
\textbf{Automation} & Scripting and testing \\
\end{longtable}
}

\end{solutionbox}
\begin{mnemonicbox}
``Web Data Machine Desktop Game Auto''

\end{mnemonicbox}
\begin{center}\rule{0.5\linewidth}{0.5pt}\end{center}

\subsection*{Question 1(c) [7 marks]}\label{q1c}

\textbf{Explain various datatypes in python.}

\begin{solutionbox}

{\def\LTcaptype{none} % do not increment counter
\begin{longtable}[]{@{}lll@{}}
\toprule\noalign{}
Data Type & Example & Description \\
\midrule\noalign{}
\endhead
\bottomrule\noalign{}
\endlastfoot
\textbf{int} & \texttt{x\ =\ 5} & Whole numbers \\
\textbf{float} & \texttt{y\ =\ 3.14} & Decimal numbers \\
\textbf{str} & \texttt{name\ =\ "John"} & Text data \\
\textbf{bool} & \texttt{flag\ =\ True} & True/False values \\
\textbf{list} & \texttt{[1,\ 2,\ 3]} & Ordered, mutable \\
\textbf{tuple} & \texttt{(1,\ 2,\ 3)} & Ordered, immutable \\
\textbf{dict} & \texttt{\{"a":\ 1\}} & Key-value pairs \\
\textbf{set} & \texttt{\{1,\ 2,\ 3\}} & Unique elements \\
\end{longtable}
}

\textbf{Code Example:}

\begin{verbatim}
\# Numeric types
age = 25          \# int
price = 99.99     \# float

\# Text type
name = "Python"   \# str

\# Boolean type
is\_valid = True   \# bool

\# Collection types
numbers = [1, 2, 3]        \# list
coordinates = (10, 20)     \# tuple
student = \{"name": "John"\ }\# dict
unique\_ids = \{1, 2, 3\     }\# set
\end{verbatim}

\end{solutionbox}
\begin{mnemonicbox}
``Integer Float String Boolean List Tuple Dict Set''

\end{mnemonicbox}
\begin{center}\rule{0.5\linewidth}{0.5pt}\end{center}

\subsection*{Question 1(c OR) [7
marks]}\label{question-1c-or-7-marks}

\textbf{Explain arithmetic, assignment, and identity operators with
example.}

\begin{solutionbox}

\textbf{Arithmetic Operators:}

{\def\LTcaptype{none} % do not increment counter
\begin{longtable}[]{@{}lll@{}}
\toprule\noalign{}
Operator & Operation & Example \\
\midrule\noalign{}
\endhead
\bottomrule\noalign{}
\endlastfoot
\texttt{+} & Addition & \texttt{5\ +\ 3\ =\ 8} \\
\texttt{-} & Subtraction & \texttt{5\ -\ 3\ =\ 2} \\
\texttt{*} & Multiplication & \texttt{5\ *\ 3\ =\ 15} \\
\texttt{/} & Division & \texttt{10\ /\ 3\ =\ 3.33} \\
\texttt{//} & Floor Division & \texttt{10\ //\ 3\ =\ 3} \\
\texttt{\%} & Modulus & \texttt{10\ \%\ 3\ =\ 1} \\
\texttt{**} & Exponent & \texttt{2\ **\ 3\ =\ 8} \\
\end{longtable}
}

\textbf{Assignment Operators:}

{\def\LTcaptype{none} % do not increment counter
\begin{longtable}[]{@{}lll@{}}
\toprule\noalign{}
Operator & Example & Equivalent \\
\midrule\noalign{}
\endhead
\bottomrule\noalign{}
\endlastfoot
\texttt{=} & \texttt{x\ =\ 5} & Assign value \\
\texttt{+=} & \texttt{x\ +=\ 3} & \texttt{x\ =\ x\ +\ 3} \\
\texttt{-=} & \texttt{x\ -=\ 2} & \texttt{x\ =\ x\ -\ 2} \\
\texttt{*=} & \texttt{x\ *=\ 4} & \texttt{x\ =\ x\ *\ 4} \\
\end{longtable}
}

\textbf{Identity Operators:}

{\def\LTcaptype{none} % do not increment counter
\begin{longtable}[]{@{}lll@{}}
\toprule\noalign{}
Operator & Purpose & Example \\
\midrule\noalign{}
\endhead
\bottomrule\noalign{}
\endlastfoot
\texttt{is} & Same object & \texttt{x\ is\ y} \\
\texttt{is\ not} & Different object & \texttt{x\ is\ not\ y} \\
\end{longtable}
}

\textbf{Code Example:}

\begin{verbatim}
\# Arithmetic
a = 10 + 5    \# 15
b = 10 // 3   \# 3

\# Assignment
x = 5
x += 3        \# x becomes 8

\# Identity
list1 = [1, 2, 3]
list2 = [1, 2, 3]
print(list1 is list2)      \# False
print(list1 is not list2)  \# True
\end{verbatim}

\end{solutionbox}
\begin{mnemonicbox}
``Add Assign Identity''

\end{mnemonicbox}
\begin{center}\rule{0.5\linewidth}{0.5pt}\end{center}

\subsection*{Question 2(a) [3 marks]}\label{q2a}

\textbf{Which of the following identifier names are invalid?} **(i)
Total Marks (ii)Total\_Marks (iii)total-Marks (iv) Hundred\$ (v)
\_Percentage (vi) True**

\begin{solutionbox}

{\def\LTcaptype{none} % do not increment counter
\begin{longtable}[]{@{}lll@{}}
\toprule\noalign{}
Identifier & Valid/Invalid & Reason \\
\midrule\noalign{}
\endhead
\bottomrule\noalign{}
\endlastfoot
Total Marks & \textbf{Invalid} & Contains space \\
Total\_Marks & Valid & Underscore allowed \\
total-Marks & \textbf{Invalid} & Hyphen not allowed \\
Hundred\$ & \textbf{Invalid} & \$ symbol not allowed \\
\_Percentage & Valid & Can start with underscore \\
True & \textbf{Invalid} & Reserved keyword \\
\end{longtable}
}

\textbf{Invalid identifiers:} Total Marks, total-Marks, Hundred\$, True

\end{solutionbox}
\begin{mnemonicbox}
``Space Hyphen Dollar Keyword = Invalid''

\end{mnemonicbox}
\begin{center}\rule{0.5\linewidth}{0.5pt}\end{center}

\subsection*{Question 2(b) [4 marks]}\label{q2b}

\textbf{Write a program to find a maximum number among the given three
numbers.}

\begin{solutionbox}

\begin{verbatim}
\# Input three numbers
num1 = float(input("Enter first number: "))
num2 = float(input("Enter second number: "))
num3 = float(input("Enter third number: "))

\# Find maximum using if{-elif{-}else}
if num1 {=} num2 and num1 {=} num3:
    maximum = num1
elif num2 {=} num1 and num2 {=} num3:
    maximum = num2
else:
    maximum = num3

\# Display result
print(f"Maximum number is: \{maximum\}")
\end{verbatim}

\textbf{Alternative using max() function:}

\begin{verbatim}
num1, num2, num3 = map(float, input("Enter 3 numbers: ").split())
maximum = max(num1, num2, num3)
print(f"Maximum: \{maximum\}")
\end{verbatim}

\end{solutionbox}
\begin{mnemonicbox}
``Input Compare Display''

\end{mnemonicbox}
\begin{center}\rule{0.5\linewidth}{0.5pt}\end{center}

\subsection*{Question 2(c) [7 marks]}\label{q2c}

\textbf{Explain dictionaries in Python. Write statements to add, modify,
and delete elements in a dictionary.}

\begin{solutionbox}

\textbf{Dictionary} is a collection of key-value pairs that is ordered,
changeable, and does not allow duplicate keys.

\textbf{Operations Table:}

{\def\LTcaptype{none} % do not increment counter
\begin{longtable}[]{@{}lll@{}}
\toprule\noalign{}
Operation & Syntax & Example \\
\midrule\noalign{}
\endhead
\bottomrule\noalign{}
\endlastfoot
\textbf{Create} & \texttt{dict\_name\ =\ \{\}} &
\texttt{student\ =\ \{\}} \\
\textbf{Add} & \texttt{dict[key]\ =\ value} &
\texttt{student[\textquotesingle{}name\textquotesingle{}]\ =\ \textquotesingle{}John\textquotesingle{}} \\
\textbf{Modify} & \texttt{dict[key]\ =\ new\_value} &
\texttt{student[\textquotesingle{}name\textquotesingle{}]\ =\ \textquotesingle{}Jane\textquotesingle{}} \\
\textbf{Delete} & \texttt{del\ dict[key]} &
\texttt{del\ student[\textquotesingle{}name\textquotesingle{}]} \\
\textbf{Access} & \texttt{dict[key]} &
\texttt{print(student[\textquotesingle{}name\textquotesingle{}])} \\
\end{longtable}
}

\textbf{Code Example:}

\begin{verbatim}
\# Create empty dictionary
student = \{\}

\# Add elements
student[{name}] = {John}
student[{age}] = 20
student[{grade}] = {A}

\# Modify element
student[{age}] = 21

\# Delete element
del student[{grade}]

\# Display dictionary
print(student)  \# Output: \{{name: John, age: 21\}}

\# Other methods
student.pop({age})           \# Remove and return value
student.update(\{{city}: {Mumbai}\)  }\# Add multiple items
\end{verbatim}

\textbf{Dictionary Properties:}

\begin{itemize}
\tightlist
\item
  \textbf{Ordered}: Maintains insertion order (Python 3.7+)
\item
  \textbf{Changeable}: Can modify after creation
\item
  \textbf{No Duplicates}: Keys must be unique
\end{itemize}

\end{solutionbox}
\begin{mnemonicbox}
``Key-Value Ordered Changeable Unique''

\end{mnemonicbox}
\begin{center}\rule{0.5\linewidth}{0.5pt}\end{center}

\subsection*{Question 2(a OR) [3
marks]}\label{question-2a-or-3-marks}

\textbf{Write a program to display the following pattern.}

\begin{solutionbox}

\begin{verbatim}
\# Pattern program
for i in range(1, 6):
    for j in range(1, i + 1):
        print(j, end=" ")
    print()  \# New line after each row
\end{verbatim}

\textbf{Output:}

\begin{verbatim}
1
1 2
1 2 3
1 2 3 4
1 2 3 4 5
\end{verbatim}

\end{solutionbox}
\begin{mnemonicbox}
``Outer Row Inner Column Print''

\end{mnemonicbox}
\begin{center}\rule{0.5\linewidth}{0.5pt}\end{center}

\subsection*{Question 2(b OR) [4
marks]}\label{question-2b-or-4-marks}

\textbf{Write a program to find the sum of digits of an integer number,
input by the user.}

\begin{solutionbox}

\begin{verbatim}
\# Input number from user
number = int(input("Enter a number: "))
original\_number = number
sum\_digits = 0

\# Extract and sum digits
while number {} 0:
    digit = number \% 10    \# Get last digit
    sum\_digits += digit    \# Add to sum
    number = number // 10  \# Remove last digit

\# Display result
print(f"Sum of digits of \{original\_number\} is: \{sum\_digits\}")
\end{verbatim}

\textbf{Alternative Method:}

\begin{verbatim}
number = input("Enter number: ")
sum\_digits = sum(int(digit) for digit in number)
print(f"Sum of digits: \{sum\_digits\}")
\end{verbatim}

\end{solutionbox}
\begin{mnemonicbox}
``Input Extract Sum Display''

\end{mnemonicbox}
\begin{center}\rule{0.5\linewidth}{0.5pt}\end{center}

\subsection*{Question 2(c OR) [7
marks]}\label{question-2c-or-7-marks}

\textbf{Explain slicing and concatenation operation on list.}

\begin{solutionbox}

\textbf{List Slicing:} Extracting portion of list using
\texttt{[start:stop:step]} syntax.

\textbf{Slicing Syntax Table:}

{\def\LTcaptype{none} % do not increment counter
\begin{longtable}[]{@{}lll@{}}
\toprule\noalign{}
Syntax & Description & Example \\
\midrule\noalign{}
\endhead
\bottomrule\noalign{}
\endlastfoot
\texttt{list[start:stop]} & Elements from start to stop-1 &
\texttt{nums[1:4]} \\
\texttt{list[:stop]} & From beginning to stop-1 &
\texttt{nums[:3]} \\
\texttt{list[start:]} & From start to end & \texttt{nums[2:]} \\
\texttt{list[::step]} & All elements with step &
\texttt{nums[::2]} \\
\texttt{list[::-1]} & Reverse list & \texttt{nums[::-1]} \\
\end{longtable}
}

\textbf{Concatenation:} Joining two or more lists using \texttt{+}
operator or \texttt{extend()} method.

\textbf{Code Example:}

\begin{verbatim}
\# Create lists
list1 = [1, 2, 3, 4, 5]
list2 = [6, 7, 8]

\# Slicing operations
print(list1[1:4])    \# [2, 3, 4]
print(list1[:3])     \# [1, 2, 3]
print(list1[2:])     \# [3, 4, 5]
print(list1[::2])    \# [1, 3, 5]
print(list1[::{-}1])   \# [5, 4, 3, 2, 1]

\# Concatenation operations
result1 = list1 + list2           \# [1, 2, 3, 4, 5, 6, 7, 8]
list1.extend(list2)               \# Adds list2 to list1
combined = [*list1, *list2]       \# Using unpacking operator
\end{verbatim}

\textbf{Key Points:}

\begin{itemize}
\tightlist
\item
  \textbf{Slicing}: Creates new list without modifying original
\item
  \textbf{Concatenation}: Combines lists into single list
\item
  \textbf{Negative indexing}: \texttt{list[-1]} gives last element
\end{itemize}

\end{solutionbox}
\begin{mnemonicbox}
``Slice Extract Concat Join''

\end{mnemonicbox}
\begin{center}\rule{0.5\linewidth}{0.5pt}\end{center}

\subsection*{Question 3(a) [3 marks]}\label{q3a}

\textbf{Define a list in Python. Write name of the function used to add
an element to the end of a list.}

\begin{solutionbox}

\textbf{List Definition:} A \textbf{list} is an ordered collection of
items that is changeable and allows duplicate values.

\textbf{Properties Table:}

{\def\LTcaptype{none} % do not increment counter
\begin{longtable}[]{@{}ll@{}}
\toprule\noalign{}
Property & Description \\
\midrule\noalign{}
\endhead
\bottomrule\noalign{}
\endlastfoot
\textbf{Ordered} & Items have defined order \\
\textbf{Changeable} & Can modify after creation \\
\textbf{Duplicates} & Allows duplicate values \\
\textbf{Indexed} & Items accessed by index \\
\end{longtable}
}

\textbf{Function to add element:} \texttt{append()}

\textbf{Example:}

\begin{verbatim}
\# Create list
fruits = [{apple}, {banana}]

\# Add element to end
fruits.append({orange})
print(fruits)  \# [{apple, banana, orange]}
\end{verbatim}

\end{solutionbox}
\begin{mnemonicbox}
``List Append End''

\end{mnemonicbox}
\begin{center}\rule{0.5\linewidth}{0.5pt}\end{center}

\subsection*{Question 3(b) [4 marks]}\label{q3b}

\textbf{Define a tuple in Python. Write statement to access last element
of a tuple.}

\begin{solutionbox}

\textbf{Tuple Definition:} A \textbf{tuple} is an ordered collection of
items that is unchangeable and allows duplicate values.

\textbf{Properties Table:}

{\def\LTcaptype{none} % do not increment counter
\begin{longtable}[]{@{}ll@{}}
\toprule\noalign{}
Property & Description \\
\midrule\noalign{}
\endhead
\bottomrule\noalign{}
\endlastfoot
\textbf{Ordered} & Items have defined order \\
\textbf{Unchangeable} & Cannot modify after creation \\
\textbf{Duplicates} & Allows duplicate values \\
\textbf{Indexed} & Items accessed by index \\
\end{longtable}
}

\textbf{Accessing Last Element:}

\begin{verbatim}
\# Method 1: Using negative index
my\_tuple = (10, 20, 30, 40, 50)
last\_element = my\_tuple[{-}1]
print(last\_element)  \# Output: 50

\# Method 2: Using length
last\_element = my\_tuple[len(my\_tuple) {-} 1]
print(last\_element)  \# Output: 50
\end{verbatim}

\end{solutionbox}
\begin{mnemonicbox}
``Tuple Unchangeable Negative Index''

\end{mnemonicbox}
\begin{center}\rule{0.5\linewidth}{0.5pt}\end{center}

\subsection*{Question 3(c) [7 marks]}\label{q3c}

\textbf{Write statements for following set operations: create empty set,
add an element to a set, remove an element from set, Union of two sets,
Intersection of two sets, Difference between two sets and symmetric
difference between two sets.}

\begin{solutionbox}

\textbf{Set Operations Table:}

{\def\LTcaptype{none} % do not increment counter
\begin{longtable}[]{@{}
  >{\raggedright\arraybackslash}p{(\linewidth - 6\tabcolsep) * \real{0.2895}}
  >{\raggedright\arraybackslash}p{(\linewidth - 6\tabcolsep) * \real{0.2105}}
  >{\raggedright\arraybackslash}p{(\linewidth - 6\tabcolsep) * \real{0.2632}}
  >{\raggedright\arraybackslash}p{(\linewidth - 6\tabcolsep) * \real{0.2368}}@{}}
\toprule\noalign{}
\begin{minipage}[b]{\linewidth}\raggedright
Operation
\end{minipage} & \begin{minipage}[b]{\linewidth}\raggedright
Method
\end{minipage} & \begin{minipage}[b]{\linewidth}\raggedright
Operator
\end{minipage} & \begin{minipage}[b]{\linewidth}\raggedright
Example
\end{minipage} \\
\midrule\noalign{}
\endhead
\bottomrule\noalign{}
\endlastfoot
\textbf{Create Empty} & \texttt{set()} & - & \texttt{s\ =\ set()} \\
\textbf{Add Element} & \texttt{add()} & - & \texttt{s.add(5)} \\
\textbf{Remove Element} & \texttt{remove()} & - &
\texttt{s.remove(5)} \\
\textbf{Union} & \texttt{union()} & \texttt{\textbar{}} &
\texttt{A.union(B)} or \texttt{A\ \textbar{}\ B} \\
\textbf{Intersection} & \texttt{intersection()} & \texttt{\&} &
\texttt{A.intersection(B)} or \texttt{A\ \&\ B} \\
\textbf{Difference} & \texttt{difference()} & \texttt{-} &
\texttt{A.difference(B)} or \texttt{A\ -\ B} \\
\textbf{Symmetric Diff} & \texttt{symmetric\_difference()} &
\texttt{\^{}} & \texttt{A.symmetric\_difference(B)} or
\texttt{A\ \^{}\ B} \\
\end{longtable}
}

\textbf{Code Example:}

\begin{verbatim}
\# Create empty set
my\_set = set()

\# Add elements
my\_set.add(10)
my\_set.add(20)

\# Remove element
my\_set.remove(10)

\# Create two sets for operations
A = \{1, 2, 3, 4\}
B = \{3, 4, 5, 6\}

\# Union (all unique elements)
union\_result = A.union(B)        \# \{1, 2, 3, 4, 5, 6\}

\# Intersection (common elements)
intersection\_result = A.intersection(B)  \# \{3, 4\}

\# Difference (A {- B)}
difference\_result = A.difference(B)      \# \{1, 2\}

\# Symmetric difference (elements in A or B, but not both)
sym\_diff\_result = A.symmetric\_difference(B)  \# \{1, 2, 5, 6\}
\end{verbatim}

\end{solutionbox}
\begin{mnemonicbox}
``Create Add Remove Union Intersect Differ
Symmetric''

\end{mnemonicbox}
\begin{center}\rule{0.5\linewidth}{0.5pt}\end{center}

\subsection*{Question 3(a OR) [3
marks]}\label{question-3a-or-3-marks}

\textbf{Define a string in Python. Using example illustrate (i) How to
create a string. (ii) Accessing individual characters using indexing.}

\begin{solutionbox}

\textbf{String Definition:} A \textbf{string} is a sequence of
characters enclosed in quotes (single or double).

\textbf{(i) Creating String:}

\begin{verbatim}
\# Single quotes
name = {Python}

\# Double quotes
message = "Hello World"

\# Triple quotes (multiline)
text = """This is a
multiline string"""
\end{verbatim}

\textbf{(ii) Accessing Characters:}

\begin{verbatim}
word = "PYTHON"
print(word[0])    \# P (first character)
print(word[2])    \# T (third character)
print(word[{-}1])   \# N (last character)
print(word[{-}2])   \# O (second last)
\end{verbatim}

\end{solutionbox}
\begin{mnemonicbox}
``String Quotes Index Access''

\end{mnemonicbox}
\begin{center}\rule{0.5\linewidth}{0.5pt}\end{center}

\subsection*{Question 3(b OR) [4
marks]}\label{question-3b-or-4-marks}

\textbf{Explain list traversing using for loop and while loop.}

\begin{solutionbox}

\textbf{List Traversing} means visiting each element of list one by one.

\textbf{For Loop Traversing:}

\begin{verbatim}
numbers = [10, 20, 30, 40, 50]

\# Method 1: Direct iteration
for num in numbers:
    print(num)

\# Method 2: Using index
for i in range(len(numbers)):
    print(f"Index \{i\}: \{numbers[i]\}")
\end{verbatim}

\textbf{While Loop Traversing:}

\begin{verbatim}
numbers = [10, 20, 30, 40, 50]
i = 0

while i {} len(numbers):
    print(f"Element at index \{i\}: \{numbers[i]\}")
    i += 1
\end{verbatim}

\textbf{Comparison Table:}

{\def\LTcaptype{none} % do not increment counter
\begin{longtable}[]{@{}
  >{\raggedright\arraybackslash}p{(\linewidth - 4\tabcolsep) * \real{0.3438}}
  >{\raggedright\arraybackslash}p{(\linewidth - 4\tabcolsep) * \real{0.3438}}
  >{\raggedright\arraybackslash}p{(\linewidth - 4\tabcolsep) * \real{0.3125}}@{}}
\toprule\noalign{}
\begin{minipage}[b]{\linewidth}\raggedright
Loop Type
\end{minipage} & \begin{minipage}[b]{\linewidth}\raggedright
Advantage
\end{minipage} & \begin{minipage}[b]{\linewidth}\raggedright
Use Case
\end{minipage} \\
\midrule\noalign{}
\endhead
\bottomrule\noalign{}
\endlastfoot
\textbf{For Loop} & Simpler syntax & When number of iterations known \\
\textbf{While Loop} & More control & When condition-based iteration
needed \\
\end{longtable}
}

\end{solutionbox}
\begin{mnemonicbox}
``For Simple While Control''

\end{mnemonicbox}
\begin{center}\rule{0.5\linewidth}{0.5pt}\end{center}

\subsection*{Question 3(c OR) [7
marks]}\label{question-3c-or-7-marks}

\textbf{Write a program to create a dictionary with the roll number,
name, and marks of n students and display the names of students who have
scored marks above 75.}

\begin{solutionbox}

\begin{verbatim}
\# Input number of students
n = int(input("Enter number of students: "))

\# Create empty dictionary
students = \{\}

\# Input student data
for i in range(n):
    print(f"{n}Enter details for student \{i + 1\}:")
    roll\_no = int(input("Roll number: "))
    name = input("Name: ")
    marks = float(input("Marks: "))
    
    \# Store in dictionary
    students[roll\_no] = \{
        {name}: name,
        {marks}: marks
    \}

\# Display students with marks above 75
print("{n}Students with marks above 75:")
print("{-"} * 30)

high\_performers = []
for roll\_no, data in students.items():
    if data[{marks}] {} 75:
        high\_performers.append(data[{name}])
        print(f"Name: \{data[{name}]\}, Marks: \{data[{marks}]\}")

if not high\_performers:
    print("No student scored above 75 marks")
else:
    print(f"{n}Total high performers: \{len(high\_performers)\}")
\end{verbatim}

\textbf{Sample Output:}

\begin{verbatim}
Enter number of students: 2

Enter details for student 1:
Roll number: 101
Name: John
Marks: 80

Enter details for student 2:
Roll number: 102
Name: Alice
Marks: 70

Students with marks above 75:
------------------------------
Name: John, Marks: 80.0

Total high performers: 1
\end{verbatim}

\end{solutionbox}
\begin{mnemonicbox}
``Input Store Filter Display''

\end{mnemonicbox}
\begin{center}\rule{0.5\linewidth}{0.5pt}\end{center}

\subsection*{Question 4(a) [3 marks]}\label{q4a}

\textbf{Write any three functions available in random module. Write
syntax and example of each function.}

\begin{solutionbox}

\textbf{Random Module Functions:}

{\def\LTcaptype{none} % do not increment counter
\begin{longtable}[]{@{}
  >{\raggedright\arraybackslash}p{(\linewidth - 6\tabcolsep) * \real{0.2778}}
  >{\raggedright\arraybackslash}p{(\linewidth - 6\tabcolsep) * \real{0.2222}}
  >{\raggedright\arraybackslash}p{(\linewidth - 6\tabcolsep) * \real{0.2500}}
  >{\raggedright\arraybackslash}p{(\linewidth - 6\tabcolsep) * \real{0.2500}}@{}}
\toprule\noalign{}
\begin{minipage}[b]{\linewidth}\raggedright
Function
\end{minipage} & \begin{minipage}[b]{\linewidth}\raggedright
Syntax
\end{minipage} & \begin{minipage}[b]{\linewidth}\raggedright
Purpose
\end{minipage} & \begin{minipage}[b]{\linewidth}\raggedright
Example
\end{minipage} \\
\midrule\noalign{}
\endhead
\bottomrule\noalign{}
\endlastfoot
\textbf{random()} & \texttt{random.random()} & Random float 0.0 to 1.0 &
\texttt{0.7534} \\
\textbf{randint()} & \texttt{random.randint(a,\ b)} & Random integer a
to b & \texttt{randint(1,\ 10)} \\
\textbf{choice()} & \texttt{random.choice(seq)} & Random element from
sequence &
\texttt{choice([\textquotesingle{}a\textquotesingle{},\ \textquotesingle{}b\textquotesingle{},\ \textquotesingle{}c\textquotesingle{}])} \\
\end{longtable}
}

\textbf{Code Example:}

\begin{verbatim}
import random

\# random() {- generates float between 0.0 and 1.0}
num = random.random()
print(num)  \# Example: 0.7234567

\# randint() {- generates integer between given range}
dice = random.randint(1, 6)
print(dice)  \# Example: 4

\# choice() {- selects random element from sequence}
colors = [{red}, {blue}, {green}]
selected = random.choice(colors)
print(selected)  \# Example: {blue}
\end{verbatim}

\end{solutionbox}
\begin{mnemonicbox}
``Random Randint Choice''

\end{mnemonicbox}
\begin{center}\rule{0.5\linewidth}{0.5pt}\end{center}

\subsection*{Question 4(b) [4 marks]}\label{q4b}

\textbf{Write the advantages of functions.}

\begin{solutionbox}

\textbf{Function Advantages:}

{\def\LTcaptype{none} % do not increment counter
\begin{longtable}[]{@{}ll@{}}
\toprule\noalign{}
Advantage & Description \\
\midrule\noalign{}
\endhead
\bottomrule\noalign{}
\endlastfoot
\textbf{Code Reusability} & Write once, use multiple times \\
\textbf{Modularity} & Break large program into smaller parts \\
\textbf{Easy Debugging} & Isolate and fix errors easily \\
\textbf{Readability} & Makes code more organized and clear \\
\textbf{Maintainability} & Easy to update and modify \\
\textbf{Avoid Repetition} & Reduces duplicate code \\
\end{longtable}
}

\textbf{Example:}

\begin{verbatim}
\# Without function (repetitive)
num1 = 5
square1 = num1 * num1
print(square1)

num2 = 8
square2 = num2 * num2
print(square2)

\# With function (reusable)
def calculate\_square(num):
    return num * num

print(calculate\_square(5))  \# 25
print(calculate\_square(8))  \# 64
\end{verbatim}

\end{solutionbox}
\begin{mnemonicbox}
``Reuse Modular Debug Read Maintain Avoid''

\end{mnemonicbox}
\begin{center}\rule{0.5\linewidth}{0.5pt}\end{center}

\subsection*{Question 4(c) [7 marks]}\label{q4c}

\textbf{Write a program that asks the user for a string and prints out
the location of each `a' in the string.}

\begin{solutionbox}

\begin{verbatim}
\# Input string from user
text = input("Enter a string: ")

\# Find all positions of {a}
positions = []
for i in range(len(text)):
    if text[i].lower() == {a}:  \# Check for both {a and A}
        positions.append(i)

\# Display results
if positions:
    print(f"Character {a found at positions: }\{positions\}")
    print("Detailed locations:")
    for pos in positions:
        print(f"Position \{pos\}: {}\{text[pos]\}{"})
else:
    print("Character {a not found in the string"})

\# Alternative method using enumerate
print("{n}Alternative approach:")
for index, char in enumerate(text):
    if char.lower() == {a}:
        print(f"{a found at position }\{index\}")
\end{verbatim}

\textbf{Sample Output:}

\begin{verbatim}
Enter a string: Python Programming

Character 'a' found at positions: [12]
Detailed locations:
Position 12: 'a'

Alternative approach:
'a' found at position 12
\end{verbatim}

\textbf{Enhanced Version:}

\begin{verbatim}
text = input("Enter a string: ")
count = 0

print(f"Searching for {a in: }\{text\}{"})
print("{-"} * 30)

for i, char in enumerate(text):
    if char.lower() == {a}:
        count += 1
        print(f"Found {a at index }\{i\} (character: {}\{char\}{)"})

print(f"{n}Total occurrences of {a: }\{count\}")
\end{verbatim}

\end{solutionbox}
\begin{mnemonicbox}
``Input Loop Check Store Display''

\end{mnemonicbox}
\begin{center}\rule{0.5\linewidth}{0.5pt}\end{center}

\subsection*{Question 4(a OR) [3
marks]}\label{question-4a-or-3-marks}

\textbf{Explain local and global variables.}

\begin{solutionbox}

\textbf{Variable Scope Types:}

{\def\LTcaptype{none} % do not increment counter
\begin{longtable}[]{@{}
  >{\raggedright\arraybackslash}p{(\linewidth - 6\tabcolsep) * \real{0.3846}}
  >{\raggedright\arraybackslash}p{(\linewidth - 6\tabcolsep) * \real{0.1795}}
  >{\raggedright\arraybackslash}p{(\linewidth - 6\tabcolsep) * \real{0.2051}}
  >{\raggedright\arraybackslash}p{(\linewidth - 6\tabcolsep) * \real{0.2308}}@{}}
\toprule\noalign{}
\begin{minipage}[b]{\linewidth}\raggedright
Variable Type
\end{minipage} & \begin{minipage}[b]{\linewidth}\raggedright
Scope
\end{minipage} & \begin{minipage}[b]{\linewidth}\raggedright
Access
\end{minipage} & \begin{minipage}[b]{\linewidth}\raggedright
Example
\end{minipage} \\
\midrule\noalign{}
\endhead
\bottomrule\noalign{}
\endlastfoot
\textbf{Local} & Inside function only & Within function &
\texttt{def\ func():\ x\ =\ 5} \\
\textbf{Global} & Entire program & Anywhere in program &
\texttt{x\ =\ 5} (outside function) \\
\end{longtable}
}

\textbf{Code Example:}

\begin{verbatim}
\# Global variable
global\_var = "I am global"

def my\_function():
    \# Local variable
    local\_var = "I am local"
    print(global\_var)  \# Can access global
    print(local\_var)   \# Can access local

my\_function()
print(global\_var)      \# Can access global
\# print(local\_var)     \# Error {- cannot access local}
\end{verbatim}

\textbf{Global Keyword:}

\begin{verbatim}
counter = 0  \# Global variable

def increment():
    global counter  \# Declare as global to modify
    counter += 1

increment()
print(counter)  \# Output: 1
\end{verbatim}

\end{solutionbox}
\begin{mnemonicbox}
``Local Inside Global Everywhere''

\end{mnemonicbox}
\begin{center}\rule{0.5\linewidth}{0.5pt}\end{center}

\subsection*{Question 4(b OR) [4
marks]}\label{question-4b-or-4-marks}

\textbf{Explain creation and use of user defined function with example.}

\begin{solutionbox}

\textbf{Function Creation Syntax:}

\begin{verbatim}
def function\_name(parameters):
    """Optional docstring"""
    \# Function body
    return value  \# Optional
\end{verbatim}

\textbf{Function Components:}

{\def\LTcaptype{none} % do not increment counter
\begin{longtable}[]{@{}lll@{}}
\toprule\noalign{}
Component & Purpose & Example \\
\midrule\noalign{}
\endhead
\bottomrule\noalign{}
\endlastfoot
\textbf{def} & Keyword to define function & \texttt{def} \\
\textbf{function\_name} & Name of function & \texttt{calculate\_area} \\
\textbf{parameters} & Input values & \texttt{(length,\ width)} \\
\textbf{return} & Output value & \texttt{return\ result} \\
\end{longtable}
}

\textbf{Example:}

\begin{verbatim}
\# Function definition
def greet\_user(name, age):
    """Function to greet user with name and age"""
    message = f"Hello \{name\}! You are \{age\} years old."
    return message

\# Function call
user\_name = "John"
user\_age = 25
greeting = greet\_user(user\_name, user\_age)
print(greeting)  \# Output: Hello John! You are 25 years old.

\# Function with default parameter
def calculate\_power(base, exponent=2):
    return base ** exponent

print(calculate\_power(5))     \# 25 (using default exponent=2)
print(calculate\_power(5, 3))  \# 125 (using exponent=3)
\end{verbatim}

\end{solutionbox}
\begin{mnemonicbox}
``Define Call Return Parameter''

\end{mnemonicbox}
\begin{center}\rule{0.5\linewidth}{0.5pt}\end{center}

\subsection*{Question 4(c OR) [7
marks]}\label{question-4c-or-7-marks}

\textbf{Write a program to create a user defined function calcFact() to
calculate and display the factorial of a number passed as an argument.}

\begin{solutionbox}

\begin{verbatim}
def calcFact(number):
    """
    Function to calculate factorial of a number
    Input: number (integer)
    Output: factorial (integer)
    """
    if number {} 0:
        return "Factorial is not defined for negative numbers"
    elif number == 0 or number == 1:
        return 1
    else:
        factorial = 1
        for i in range(2, number + 1):
            factorial *= i
        return factorial

\# Main program
try:
    \# Input from user
    num = int(input("Enter a number: "))
    
    \# Call function
    result = calcFact(num)
    
    \# Display result
    if isinstance(result, str):
        print(result)
    else:
        print(f"Factorial of \{num\} is: \{result\}")
        
except ValueError:
    print("Please enter a valid integer")

\# Test with multiple values
print("{n}Testing with different values:")
test\_values = [0, 1, 5, 10, {-}3]
for val in test\_values:
    result = calcFact(val)
    print(f"calcFact(\{val\}) = \{result\}")
\end{verbatim}

\textbf{Recursive Version:}

\begin{verbatim}
def calcFactRecursive(n):
    """Recursive function to calculate factorial"""
    if n {} 0:
        return "Undefined for negative numbers"
elif

n == 0 or

n == 1:

        return 1
    else:
        return n * calcFactRecursive(n {-} 1)

\# Example usage
number = int(input("Enter number: "))
result = calcFactRecursive(number)
print(f"Factorial: \{result\}")
\end{verbatim}

\textbf{Sample Output:}

\begin{verbatim}
Enter a number: 5
Factorial of 5 is: 120

Testing with different values:
calcFact(0) = 1
calcFact(1) = 1
calcFact(5) = 120
calcFact(10) = 3628800
calcFact(-3) = Factorial is not defined for negative numbers
\end{verbatim}

\end{solutionbox}
\begin{mnemonicbox}
``Define Check Loop Multiply Return''

\end{mnemonicbox}
\begin{center}\rule{0.5\linewidth}{0.5pt}\end{center}

\subsection*{Question 5(a) [3 marks]}\label{q5a}

\textbf{Give difference between class and object.}

\begin{solutionbox}

\textbf{Class vs Object Comparison:}

{\def\LTcaptype{none} % do not increment counter
\begin{longtable}[]{@{}
  >{\raggedright\arraybackslash}p{(\linewidth - 4\tabcolsep) * \real{0.3478}}
  >{\raggedright\arraybackslash}p{(\linewidth - 4\tabcolsep) * \real{0.3043}}
  >{\raggedright\arraybackslash}p{(\linewidth - 4\tabcolsep) * \real{0.3478}}@{}}
\toprule\noalign{}
\begin{minipage}[b]{\linewidth}\raggedright
Aspect
\end{minipage} & \begin{minipage}[b]{\linewidth}\raggedright
Class
\end{minipage} & \begin{minipage}[b]{\linewidth}\raggedright
Object
\end{minipage} \\
\midrule\noalign{}
\endhead
\bottomrule\noalign{}
\endlastfoot
\textbf{Definition} & Blueprint/template & Instance of class \\
\textbf{Memory} & No memory allocated & Memory allocated \\
\textbf{Creation} & Defined using \texttt{class} keyword & Created using
class name \\
\textbf{Attributes} & Defined but not initialized & Have actual
values \\
\textbf{Example} & \texttt{class\ Car:} & \texttt{my\_car\ =\ Car()} \\
\end{longtable}
}

\textbf{Code Example:}

\begin{verbatim}
\# Class definition (blueprint)
class Student:
    def \_\_init\_\_(self, name, age):
        self.name = name
        self.age = age

\# Object creation (instances)
student1 = Student("John", 20)  \# Object 1
student2 = Student("Alice", 19) \# Object 2

print(student1.name)  \# John
print(student2.name)  \# Alice
\end{verbatim}

\end{solutionbox}
\begin{mnemonicbox}
``Class Blueprint Object Instance''

\end{mnemonicbox}
\begin{center}\rule{0.5\linewidth}{0.5pt}\end{center}

\subsection*{Question 5(b) [4 marks]}\label{q5b}

\textbf{State the purpose of a constructor in a class.}

\begin{solutionbox}

\textbf{Constructor Purpose:}

{\def\LTcaptype{none} % do not increment counter
\begin{longtable}[]{@{}ll@{}}
\toprule\noalign{}
Purpose & Description \\
\midrule\noalign{}
\endhead
\bottomrule\noalign{}
\endlastfoot
\textbf{Initialize Objects} & Set initial values to attributes \\
\textbf{Automatic Execution} & Called automatically when object
created \\
\textbf{Memory Setup} & Allocate memory for object attributes \\
\textbf{Default Values} & Provide default values to attributes \\
\end{longtable}
}

\textbf{Types of Constructors:}

{\def\LTcaptype{none} % do not increment counter
\begin{longtable}[]{@{}lll@{}}
\toprule\noalign{}
Type & Description & Example \\
\midrule\noalign{}
\endhead
\bottomrule\noalign{}
\endlastfoot
\textbf{Default} & No parameters & \texttt{def\ \_\_init\_\_(self):} \\
\textbf{Parameterized} & Takes parameters &
\texttt{def\ \_\_init\_\_(self,\ name):} \\
\end{longtable}
}

\textbf{Example:}

\begin{verbatim}
class Rectangle:
    def \_\_init\_\_(self, length=0, width=0):  \# Constructor
        self.length = length  \# Initialize attribute
        self.width = width    \# Initialize attribute
        print("Rectangle object created!")
    
    def area(self):
        return self.length * self.width

\# Object creation {- constructor called automatically}
rect1 = Rectangle(10, 5)  \# Output: Rectangle object created!
rect2 = Rectangle()       \# Uses default values

print(rect1.area())       \# 50
print(rect2.area())       \# 0
\end{verbatim}

\end{solutionbox}
\begin{mnemonicbox}
``Initialize Automatic Memory Default''

\end{mnemonicbox}
\begin{center}\rule{0.5\linewidth}{0.5pt}\end{center}

\subsection*{Question 5(c) [7 marks]}\label{q5c}

\textbf{Write a program to create a class ``Student'' with attributes
such as name, roll number, and marks. Implement method to display
student information. Create object of the student class and show how to
use method.}

\begin{solutionbox}

\begin{verbatim}
class Student:
    def \_\_init\_\_(self, name, roll\_number, marks):
        """Constructor to initialize student attributes"""
        self.name = name
        self.roll\_number = roll\_number
        self.marks = marks
    
    def display\_info(self):
        """Method to display student information"""
        print("{-"} * 30)
        print("STUDENT INFORMATION")
        print("{-"} * 30)
        print(f"Name: \{self.name\}")
        print(f"Roll Number: \{self.roll\_number\}")
        print(f"Marks: \{self.marks\}")
        print("{-"} * 30)
    
    def calculate\_grade(self):
        """Method to calculate grade based on marks"""
        if self.marks {=} 90:
            return {A+}
        elif self.marks {=} 80:
            return {A}
        elif self.marks {=} 70:
            return {B}
        elif self.marks {=} 60:
            return {C}
        else:
            return {F}
    
    def display\_grade(self):
        """Method to display grade"""
        grade = self.calculate\_grade()
        print(f"Grade: \{grade\}")

\# Creating objects of Student class
print("Creating Student Objects:")
student1 = Student("John Doe", 101, 85)
student2 = Student("Alice Smith", 102, 92)
student3 = Student("Bob Johnson", 103, 78)

\# Using methods to display information
print("{n}=== Student 1 Details ===")
student1.display\_info()
student1.display\_grade()

print("{n}=== Student 2 Details ===")
student2.display\_info()
student2.display\_grade()

print("{n}=== Student 3 Details ===")
student3.display\_info()
student3.display\_grade()

\# Accessing attributes directly
print(f"{n}Direct access {- Student 1 name: }\{student1.name\}")
print(f"Direct access {- Student 2 marks: }\{student2.marks\}")
\end{verbatim}

\textbf{Sample Output:}

\begin{verbatim}
Creating Student Objects:

=== Student 1 Details ===
------------------------------
STUDENT INFORMATION
------------------------------
Name: John Doe
Roll Number: 101
Marks: 85
------------------------------
Grade: A

=== Student 2 Details ===
------------------------------
STUDENT INFORMATION
------------------------------
Name: Alice Smith
Roll Number: 102
Marks: 92
------------------------------
Grade: A+
\end{verbatim}

\textbf{Class Components:}

\begin{itemize}
\tightlist
\item
  \textbf{Attributes}: name, roll\_number, marks
\item
  \textbf{Constructor}: \texttt{\_\_init\_\_()} method
\item
  \textbf{Methods}: display\_info(), calculate\_grade(),
  display\_grade()
\item
  \textbf{Objects}: student1, student2, student3
\end{itemize}

\end{solutionbox}
\begin{mnemonicbox}
``Class Attributes Constructor Methods Objects''

\end{mnemonicbox}
\begin{center}\rule{0.5\linewidth}{0.5pt}\end{center}

\subsection*{Question 5(a OR) [3
marks]}\label{question-5a-or-3-marks}

\textbf{State the purpose of encapsulation.}

\begin{solutionbox}

\textbf{Encapsulation Purpose:}

{\def\LTcaptype{none} % do not increment counter
\begin{longtable}[]{@{}ll@{}}
\toprule\noalign{}
Purpose & Description \\
\midrule\noalign{}
\endhead
\bottomrule\noalign{}
\endlastfoot
\textbf{Data Hiding} & Hide internal implementation details \\
\textbf{Data Protection} & Protect data from unauthorized access \\
\textbf{Controlled Access} & Provide controlled access through
methods \\
\textbf{Code Security} & Prevent accidental modification of data \\
\textbf{Modularity} & Keep related data and methods together \\
\end{longtable}
}

\textbf{Implementation Example:}

\begin{verbatim}
class BankAccount:
    def \_\_init\_\_(self, balance):
        self.\_\_balance = balance  \# Private attribute
    
    def get\_balance(self):       \# Getter method
        return self.\_\_balance
    
    def deposit(self, amount):   \# Controlled access
        if amount {} 0:
            self.\_\_balance += amount

account = BankAccount(1000)
print(account.get\_balance())     \# 1000
\# print(account.\_\_balance)       \# Error {- cannot access directly}
\end{verbatim}

\textbf{Benefits:}

\begin{itemize}
\tightlist
\item
  \textbf{Security}: Data cannot be accessed directly
\item
  \textbf{Maintenance}: Easy to modify internal implementation
\item
  \textbf{Validation}: Can add validation in getter/setter methods
\end{itemize}

\end{solutionbox}
\begin{mnemonicbox}
``Hide Protect Control Secure Modular''

\end{mnemonicbox}
\begin{center}\rule{0.5\linewidth}{0.5pt}\end{center}

\subsection*{Question 5(b OR) [4
marks]}\label{question-5b-or-4-marks}

\textbf{Explain multilevel inheritance.}

\begin{solutionbox}

\textbf{Multilevel Inheritance} is when a class inherits from another
class, which in turn inherits from another class, forming a chain.

\textbf{Structure Diagram:}

\begin{verbatim}
    +{-{-}{-}{-}{-}{-}{-}{-}{-}{-}+}
    | GrandPa  |  (Base Class)
    +{-{-}{-}{-}{-}{-}{-}{-}{-}{-}+}
         \^{}
         |
    +{-{-}{-}{-}{-}{-}{-}{-}{-}{-}+}
    |  Parent  |  (Derived from GrandPa)
    +{-{-}{-}{-}{-}{-}{-}{-}{-}{-}+}
         \^{}
         |
    +{-{-}{-}{-}{-}{-}{-}{-}{-}{-}+}
    |  Child   |  (Derived from Parent)
    +{-{-}{-}{-}{-}{-}{-}{-}{-}{-}+}
\end{verbatim}

\textbf{Characteristics Table:}

{\def\LTcaptype{none} % do not increment counter
\begin{longtable}[]{@{}llll@{}}
\toprule\noalign{}
Level & Class & Inherits From & Access To \\
\midrule\noalign{}
\endhead
\bottomrule\noalign{}
\endlastfoot
\textbf{Level 1} & GrandPa & None & Own methods \\
\textbf{Level 2} & Parent & GrandPa & GrandPa + Own methods \\
\textbf{Level 3} & Child & Parent & GrandPa + Parent + Own \\
\end{longtable}
}

\textbf{Code Example:}

\begin{verbatim}
\# Level 1 {- Base class}
class Vehicle:
    def \_\_init\_\_(self, brand):
        self.brand = brand
    
    def start(self):
        print(f"\{self.brand\} vehicle started")

\# Level 2 {- Inherits from Vehicle}
class Car(Vehicle):
    def \_\_init\_\_(self, brand, model):
        super().\_\_init\_\_(brand)
        self.model = model
    
    def drive(self):
        print(f"\{self.brand\} \{self.model\} is driving")

\# Level 3 {- Inherits from Car}
class SportsCar(Car):
    def \_\_init\_\_(self, brand, model, top\_speed):
        super().\_\_init\_\_(brand, model)
        self.top\_speed = top\_speed
    
    def race(self):
        print(f"\{self.brand\} \{self.model\} racing at \{self.top\_speed\} km/h")

\# Creating object and using methods
ferrari = SportsCar("Ferrari", "F8", 340)
ferrari.start()    \# From Vehicle class
ferrari.drive()    \# From Car class
ferrari.race()     \# From SportsCar class
\end{verbatim}

\end{solutionbox}
\begin{mnemonicbox}
``Chain Inherit Level Access''

\end{mnemonicbox}
\begin{center}\rule{0.5\linewidth}{0.5pt}\end{center}

\subsection*{Question 5(c OR) [7
marks]}\label{question-5c-or-7-marks}

\textbf{Write a Python program to demonstrate working of hybrid
inheritance.}

\begin{solutionbox}

\textbf{Hybrid Inheritance} combines multiple types of inheritance
(single, multiple, multilevel) in one program.

\textbf{Structure Diagram:}

\begin{verbatim}
    +{-{-}{-}{-}{-}{-}{-}{-}{-}{-}+}
    |  Animal  |  (Base Class)
    +{-{-}{-}{-}{-}{-}{-}{-}{-}{-}+}
         \^{}
         |
    +{-{-}{-}{-}{-}{-}{-}{-}{-}{-}+}
    | Mammal   |  (Single Inheritance)
    +{-{-}{-}{-}{-}{-}{-}{-}{-}{-}+}
         \^{}
         |
    +{-{-}{-}{-}{-}{-}{-}{-}{-}{-}+     +{-}{-}{-}{-}{-}{-}{-}{-}{-}{-}+}
    |   Dog    |     |   Bird   |  (Single Inheritance)
    +{-{-}{-}{-}{-}{-}{-}{-}{-}{-}+     +{-}{-}{-}{-}{-}{-}{-}{-}{-}{-}+}
         \^{                 \^{}}
         |                 |
         +{-{-}{-}{-}{-}{-}{-}+{-}{-}{-}{-}{-}{-}{-}{-}{-}+}
                 |
         +{-{-}{-}{-}{-}{-}{-}{-}{-}{-}{-}{-}{-}{-}{-}+}
         |  FlyingDog    |  (Multiple Inheritance)
         +{-{-}{-}{-}{-}{-}{-}{-}{-}{-}{-}{-}{-}{-}{-}+}
\end{verbatim}

\textbf{Code Example:}

\begin{verbatim}
\# Base class
class Animal:
    def \_\_init\_\_(self, name):
        self.name = name
        print(f"Animal \{self.name\} created")
    
    def eat(self):
        print(f"\{self.name\} is eating")
    
    def sleep(self):
        print(f"\{self.name\} is sleeping")

\# Single inheritance from Animal
class Mammal(Animal):
    def \_\_init\_\_(self, name, fur\_color):
        super().\_\_init\_\_(name)
        self.fur\_color = fur\_color
    
    def give\_birth(self):
        print(f"\{self.name\} gives birth to live babies")

\# Single inheritance from Animal
class Bird(Animal):
    def \_\_init\_\_(self, name, wing\_span):
        super().\_\_init\_\_(name)
        self.wing\_span = wing\_span
    
    def fly(self):
        print(f"\{self.name\} is flying with \{self.wing\_span\}cm wings")
    
    def lay\_eggs(self):
        print(f"\{self.name\} lays eggs")

\# Single inheritance from Mammal
class Dog(Mammal):
    def \_\_init\_\_(self, name, fur\_color, breed):
        super().\_\_init\_\_(name, fur\_color)
        self.breed = breed
    
    def bark(self):
        print(f"\{self.name\} the \{self.breed\} is barking")
    
    def guard(self):
        print(f"\{self.name\} is guarding the house")

\# Multiple inheritance from Dog and Bird (Hybrid)
class FlyingDog(Dog, Bird):
    def \_\_init\_\_(self, name, fur\_color, breed, wing\_span):
        \# Initialize both parent classes
        Dog.\_\_init\_\_(self, name, fur\_color, breed)
        Bird.\_\_init\_\_(self, name, wing\_span)
        print(f"Magical \{self.name\} created with both mammal and bird features!")
    
    def fly\_and\_bark(self):
        print(f"\{self.name\} is flying and barking at the same time!")
    
    def show\_abilities(self):
        print(f"{n}\{self.name\}{s Abilities:"})
        print("{-"} * 25)
        self.eat()          \# From Animal
        self.sleep()        \# From Animal
        self.give\_birth()   \# From Mammal
        self.bark()         \# From Dog
        self.guard()        \# From Dog
        self.fly()          \# From Bird
        self.lay\_eggs()     \# From Bird
        self.fly\_and\_bark() \# Own method

\# Demonstration
print("=== Hybrid Inheritance Demo ==={n}")

\# Create objects
print("1. Creating regular dog:")
dog1 = Dog("Buddy", "Golden", "Retriever")
dog1.bark()
dog1.guard()

print("{n}2. Creating regular bird:")
bird1 = Bird("Eagle", 200)
bird1.fly()
bird1.lay\_eggs()

print("{n}3. Creating magical flying dog:")
flying\_dog = FlyingDog("Superdog", "Silver", "Husky", 150)
flying\_dog.show\_abilities()

\# Method Resolution Order
print(f"{n}Method Resolution Order for FlyingDog:")
for i, cls in enumerate(FlyingDog.\_\_mro\_\_):
    print(f"\{i+1\}. \{cls.\_\_name\_\_\}")
\end{verbatim}

\textbf{Sample Output:}

\begin{verbatim}
=== Hybrid Inheritance Demo ===

1. Creating regular dog:
Animal Buddy created
Buddy the Retriever is barking
Buddy is guarding the house

2. Creating regular bird:
Animal Eagle created
Eagle is flying with 200cm wings
Eagle lays eggs

3. Creating magical flying dog:
Animal Superdog created
Animal Superdog created
Magical Superdog created with both mammal and bird features!

Superdog's Abilities:
-------------------------
Superdog is eating
Superdog is sleeping
Superdog gives birth to live babies
Superdog the Husky is barking
Superdog is guarding the house
Superdog is flying with 150cm wings
Superdog lays eggs
Superdog is flying and barking at the same time!
\end{verbatim}

\textbf{Inheritance Types in This Example:}

\begin{enumerate}
\tightlist
\item
  \textbf{Single}: Mammal \leftarrow Animal, Bird \leftarrow Animal, Dog \leftarrow Mammal
\item
  \textbf{Multiple}: FlyingDog \leftarrow Dog + Bird
\item
  \textbf{Multilevel}: FlyingDog \leftarrow Dog \leftarrow Mammal \leftarrow Animal
\item
  \textbf{Hybrid}: Combination of all above
\end{enumerate}

\textbf{Key Features:}

\begin{itemize}
\tightlist
\item
  \textbf{Multiple Parent Classes}: FlyingDog inherits from both Dog and
  Bird
\item
  \textbf{Method Resolution Order}: Python follows MRO to resolve method
  conflicts
\item
  \textbf{Super() Usage}: Proper initialization of parent classes
\item
  \textbf{Combined Functionality}: Access to methods from all parent
  classes
\end{itemize}

\end{solutionbox}
\begin{mnemonicbox}
``Hybrid Multiple Single Multilevel Combined''

\end{mnemonicbox}

\end{document}
