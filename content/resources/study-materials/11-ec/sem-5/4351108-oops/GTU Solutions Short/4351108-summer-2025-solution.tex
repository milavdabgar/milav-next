\documentclass{article}
% Adjust the relative path to point to the latex-templates directory

% content/resources/templates/preamble.tex
\usepackage[margin=0.6in]{geometry}
\author{Milav Dabgar}
\usepackage{amsmath,amssymb,amsthm}
\usepackage{booktabs}
\usepackage{multirow}
\usepackage{xcolor}
\usepackage{tcolorbox}
\tcbuselibrary{breakable,skins}
\usepackage[colorlinks=true,linkcolor=blue]{hyperref}
\usepackage{titlesec}
\usepackage{enumitem}
\usepackage{tikz}
\usepackage{pgfplots}
\usepackage{circuitikz}
\usepackage[version=4]{mhchem}
\usepackage{longtable}
\usepackage{array}
\usepackage{float}
\usepackage{caption}
\usepackage{listings}

\lstset{
  basicstyle=\small\ttfamily,
  breaklines=true,
  breakatwhitespace=false,
  postbreak=\mbox{\textcolor{red}{$\hookrightarrow$}\space},
  float=false,
  numbers=left,
  numberstyle=\tiny\color{gray},
  numbersep=10pt,
  xleftmargin=2em,
  keywordstyle=\color{blue},
  commentstyle=\color{green!60!black},
  stringstyle=\color{purple},
  backgroundcolor=\color{gray!5},
  showstringspaces=false,
  tabsize=2,
  captionpos=b,
  keepspaces=true,
  columns=flexible
}

\pgfplotsset{compat=1.18}
\usetikzlibrary{shapes,arrows,positioning,calc,patterns,decorations.pathmorphing,decorations.markings,arrows.meta}

% Color scheme
\definecolor{headcolor}{RGB}{0,102,204}
\definecolor{keycolor}{RGB}{220,20,60}
\definecolor{solutioncolor}{RGB}{34,139,34}
\definecolor{mnemoniccolor}{RGB}{148,0,211}
\definecolor{codecolor}{RGB}{0,0,100}

% Spacing
\setlength{\parskip}{3pt}
\setlist[itemize]{nosep}
\setlist[enumerate]{nosep}

% Title formatting
\titleformat{\section}{\Large\bfseries\color{headcolor}}{\thesection}{1em}{}
\titleformat{\subsection}{\large\bfseries\color{headcolor}}{\thesubsection}{1em}{}

% Pandoc tightlist compatibility
\providecommand{\tightlist}{%
  \setlength{\itemsep}{0pt}\setlength{\parskip}{0pt}}

% Pandoc longtable compatibility
\newcounter{none}
\def\thenone{}


% content/resources/templates/english-boxes.tex

% Custom environments
\newtcolorbox{solutionbox}{
 breakable,
 enhanced,
 colback=solutioncolor!5!white,
 colframe=solutioncolor!75!black,
 fonttitle=\bfseries,
 title=Solution
}

\newtcolorbox{solutionboxnobreak}{
 colback=solutioncolor!5!white,
 colframe=solutioncolor!75!black,
 fonttitle=\bfseries,
 title=Solution
}

\newtcolorbox{keyformula}{
 breakable,
 enhanced,
 colback=keycolor!5!white,
 colframe=keycolor!75!black,
 fonttitle=\bfseries,
 title=Key Formula
}

\newtcolorbox{mnemonicboxenv}{
 breakable,
 enhanced,
 colback=mnemoniccolor!5!white,
 colframe=mnemoniccolor!75!black,
 fonttitle=\bfseries,
 title=Mnemonic
}

\newcommand{\mnemonicbox}[1]{%
  \begin{mnemonicboxenv}
    #1
  \end{mnemonicboxenv}
}


% Custom commands for GTU solutions
% This file defines semantic commands for consistent formatting

% Question command with automatic formatting
\newcommand{\question}[2]{%
  \section*{Question #1}%
  \textbf{#2}%
}

% OR question variant
\newcommand{\questionor}[2]{%
  \section*{Question #1 OR}%
  \textbf{#2}%
}

% Proper table environment with caption
\newenvironment{answertable}[1]{%
  \begin{table}[htbp]
  \centering
  \caption{#1}
}{%
  \end{table}
}

% Proper figure environment for diagrams
\newenvironment{answerdiagram}[1]{%
  \begin{figure}[htbp]
  \centering
  \caption{#1}
}{%
  \end{figure}
}

% Semantic markup for key terms
\newcommand{\keyword}[1]{\textbf{#1}}
\newcommand{\code}[1]{\texttt{#1}}
\newcommand{\classname}[1]{\texttt{#1}}
\newcommand{\methodname}[1]{\texttt{#1}}

% Proper quotation marks
\newcommand{\mnemonic}[1]{``#1''}


\title{OOPS \& Python Programming (4351108) - Summer 2025 Solution}
\date{May 14, 2025}

\begin{document}
\maketitle

\questionmarks{1(a)}{3}{What is the purpose of a for loop in Python? Write an example.}

\begin{solutionbox}
A for loop is used to iterate over a sequence (like list, tuple, string) or other iterable objects and execute a block of code for each item in the sequence.

\textbf{Code Example:}
\begin{lstlisting}[language=Python]
# Print each fruit in a list
fruits = ["apple", "banana", "cherry"]
for fruit in fruits:
    print(fruit)
\end{lstlisting}

\begin{itemize}
    \item \keyword{Iteration}: Automatically repeats code for each item
    \item \keyword{Simplicity}: Cleaner than using while loops with counters
\end{itemize}
\end{solutionbox}

\begin{mnemonicbox}
\mnemonic{For Each Item Do}
\end{mnemonicbox}

\questionmarks{1(b)}{4}{List out rules for defining variables in python and list out data types in python.}

\begin{solutionbox}
\textbf{Rules for defining variables:}
\begin{center}
\captionof{table}{Variable Rules}
\begin{tabulary}{\linewidth}{|L|L|L|}
\hline
\textbf{Rule} & \textbf{Example} & \textbf{Invalid Example} \\ \hline
Must start with letter or underscore & \code{name = "John"} & \code{1name = "John"} \\ \hline
Can contain letters, numbers, underscores & \code{user\_1 = "Alice"} & \code{user-1 = "Alice"} \\ \hline
Case-sensitive & \code{age} $\neq$ \code{Age} & - \\ \hline
Cannot use reserved keywords & \code{count = 5} & \code{if = 5} \\ \hline
\end{tabulary}
\end{center}

\textbf{Python Data Types:}
\begin{center}
\captionof{table}{Data Types}
\begin{tabulary}{\linewidth}{|L|L|L|}
\hline
\textbf{Data Type} & \textbf{Description} & \textbf{Example} \\ \hline
int & Integer numbers & \code{x = 10} \\ \hline
float & Decimal numbers & \code{y = 10.5} \\ \hline
str & Text strings & \code{name = "John"} \\ \hline
bool & Boolean values & \code{is\_active = True} \\ \hline
list & Ordered, changeable & \code{["apple", "banana"]} \\ \hline
tuple & Ordered, unchangeable & \code{(10, 20)} \\ \hline
dict & Key-value pairs & \code{\{"name": "John"\}} \\ \hline
set & Unordered, unique & \code{\{1, 2, 3\}} \\ \hline
\end{tabulary}
\end{center}
\end{solutionbox}

\begin{mnemonicbox}
\mnemonic{SILB-DTS: String, Integer, List, Boolean, Dictionary, Tuple, Set}
\end{mnemonicbox}

\questionmarks{1(c)}{7}{Create a program to print prime numbers between 1 to N.}

\begin{solutionbox}
\begin{lstlisting}[language=Python]
def print_primes(n):
    print("Prime numbers between 1 and", n, "are:")
    
    for num in range(2, n + 1):
        is_prime = True
        
        # Check if num is divisible by any number from 2 to sqrt(num)
        for i in range(2, int(num**0.5) + 1):
            if num % i == 0:
                is_prime = False
                break
                
        if is_prime:
            print(num, end=" ")

# Get input from user
N = int(input("Enter a number N: "))
print_primes(N)
\end{lstlisting}

\textbf{Algorithm Diagram:}
\begin{center}
\begin{tikzpicture}[gtu flow]
    \node [gtu start] (start) {Start};
    \node [gtu input, below=0.5cm of start] (input) {Input N};
    \node [gtu process, below=0.5cm of input] (init) {num = 2};
    \node [gtu decision, below=0.5cm of init] (checkN) {num $\le$ N?};
    \node [gtu process, below left=0.5cm and 0cm of checkN, xshift=-2cm] (assume) {Assume Prime};
    \node [gtu stop, right=2cm of checkN] (end) {End};
    
    \node [gtu decision, below=0.5cm of assume] (checkI) {i $\le \sqrt{num}$?};
    \node [gtu decision, below=0.5cm of checkI] (div) {num \% i == 0?};
    \node [gtu output, right=2cm of checkI] (print) {Print num};
    
    \path [gtu arrow] (start) -- (input);
    \path [gtu arrow] (input) -- (init);
    \path [gtu arrow] (init) -- (checkN);
    \path [gtu arrow] (checkN) -- node[left] {Yes} (assume);
    \path [gtu arrow] (checkN) -- node[above] {No} (end);
    \path [gtu arrow] (assume) -- (checkI);
    \path [gtu arrow] (checkI) -- node[above] {No} (print);
    \path [gtu arrow] (checkI) -- node[right] {Yes} (div);
    \path [gtu arrow] (div) -| node[right] {No} ([xshift=0.5cm]checkI.east) -- (checkI);
    \path [gtu arrow] (div.west) -- node[above] {Yes} ++(-0.5,0) |- (checkN);
    \path [gtu arrow] (print) |- (checkN);
\end{tikzpicture}
\captionof{figure}{Prime Number Algorithm}
\end{center}

\begin{itemize}
    \item \keyword{Time complexity}: $O(N\sqrt{N})$
    \item \keyword{Space complexity}: $O(1)$
\end{itemize}
\end{solutionbox}

\begin{mnemonicbox}
\mnemonic{Divide To Decide Prime}
\end{mnemonicbox}

\questionmarks{1(c OR)}{7}{Explain working of break, continue and pass statement in Python with examples.}

\begin{solutionbox}
\begin{center}
\captionof{table}{Control Statements}
\begin{tabulary}{\linewidth}{|L|L|L|}
\hline
\textbf{Statement} & \textbf{Purpose} & \textbf{Example} \\ \hline
break & Terminates loop completely & Stop search \\ \hline
continue & Skips to next iteration & Skip even nums \\ \hline
pass & Does nothing (placeholder) & Empty function \\ \hline
\end{tabulary}
\end{center}

\textbf{Flow Control Diagram:}
\begin{center}
\begin{tikzpicture}[gtu flow]
    \node [gtu start] (start) {Loop Start};
    \node [gtu decision, below=0.5cm of start] (cond) {Condition};
    \node [gtu decision, below=0.5cm of cond] (break) {break?};
    \node [gtu decision, below=0.5cm of break] (cont) {continue?};
    \node [gtu decision, below=0.5cm of cont] (pass) {pass?};
    \node [gtu process, below=0.5cm of pass] (exec) {Execute Code};
    \node [gtu stop, right=2cm of break] (exit) {Exit Loop};
    
    \path [gtu arrow] (start) -- (cond);
    \path [gtu arrow] (cond) -- node[right] {True} (break);
    \path [gtu arrow] (break) -- node[above] {Yes} (exit);
    \path [gtu arrow] (break) -- node[right] {No} (cont);
    \path [gtu arrow] (cont) -- node[right] {No} (pass);
    \path [gtu arrow] (cont.west) -- node[above] {Yes} ++(-0.5,0) |- (cond);
    \path [gtu arrow] (pass) -- (exec);
    \path [gtu arrow] (exec.west) -- ++(-0.5,0) |- (cond);
\end{tikzpicture}
\captionof{figure}{Loop Control Logic}
\end{center}

\textbf{Example Code:}
\begin{lstlisting}[language=Python]
for i in range(5):
    if i == 2: continue  # Skip 2
    if i == 4: break     # Stop at 4
    if i == 0: pass      # Do nothing
    print(i)
# Output: 0, 1, 3
\end{lstlisting}
\end{solutionbox}

\begin{mnemonicbox}
\mnemonic{BCP: Break Completely, Continue Partially, Pass silently}
\end{mnemonicbox}

\questionmarks{2(a)}{3}{Create a program that asks the user for a year and prints out whether it is a leap year or not.}

\begin{solutionbox}
\begin{lstlisting}[language=Python]
year = int(input("Enter a year: "))

if (year % 4 == 0 and year % 100 != 0) or (year % 400 == 0):
    print(f"{year} is a leap year")
else:
    print(f"{year} is not a leap year")
\end{lstlisting}

\textbf{Decision Tree:}
\begin{center}
\begin{tikzpicture}[gtu flow]
    \node [gtu start] (start) {Start};
    \node [gtu input, below=0.5cm of start] (input) {Input Year};
    \node [gtu decision, below=0.5cm of input] (div4) {Target \% 4 == 0?};
    \node [gtu decision, below left=1cm and -1cm of div4] (div100) {\% 100 == 0?};
    \node [gtu decision, below=1cm of div100] (div400) {\% 400 == 0?};
    \node [gtu block, below=1cm of div400] (leap) {Leap Year};
    \node [gtu block, right=2cm of div100] (notleap) {Not Leap};

    \path [gtu arrow] (start) -- (input);
    \path [gtu arrow] (input) -- (div4);
    \path [gtu arrow] (div4) -- node[left] {Yes} (div100);
    \path [gtu arrow] (div4) -| node[above] {No} (notleap);
    \path [gtu arrow] (div100) -- node[left] {Yes} (div400);
    \path [gtu arrow] (div100) -- node[right] {No} (leap); 
    \path [gtu arrow] (div400) -- node[left] {Yes} (leap);
    \path [gtu arrow] (div400) -| node[above] {No} (notleap);
\end{tikzpicture}
\captionof{figure}{Leap Year Logic}
\end{center}
\end{solutionbox}

\begin{mnemonicbox}
\mnemonic{4 Yes, 100 No, 400 Yes}
\end{mnemonicbox}

\questionmarks{2(b)}{4}{What are the key differences between a list and a tuple in Python?}

\begin{solutionbox}
\begin{center}
\captionof{table}{List vs Tuple}
\begin{tabulary}{\linewidth}{|L|L|L|}
\hline
\textbf{Feature} & \textbf{List} & \textbf{Tuple} \\ \hline
Syntax & \code{[]} & \code{()} \\ \hline
Mutability & Mutable (Changeable) & Immutable (Fixed) \\ \hline
Performance & Slower & Faster \\ \hline
Use Case & Dynamic collections & Fixed data \\ \hline
Memory & More memory & Less memory \\ \hline
\end{tabulary}
\end{center}

\textbf{Comparison Diagram:}
\begin{center}
\begin{tikzpicture}[gtu flow]
    \node [gtu block] (list) {\textbf{List}\\ \code{["a", "b"]}\\ \code{.append("c")}};
    \node [gtu block, right=2cm of list] (tuple) {\textbf{Tuple}\\ \code{("a", "b")}\\ Read Only};
    
    \draw [gtu arrow] (list) -- node[above] {Mutable} (tuple); 
    \node [below=0.2cm of list] {Can Change};
    \node [below=0.2cm of tuple] {Cannot Change};
\end{tikzpicture}
\captionof{figure}{List vs Tuple}
\end{center}
\end{solutionbox}

\begin{mnemonicbox}
\mnemonic{LIST: Transformable, TUPLE: Unchangeable}
\end{mnemonicbox}

\questionmarks{2(c)}{7}{Create a program to find the sum of all the positive numbers entered by the user. As soon as the user enters a negative number, stop taking in any further input from the user and display the sum.}

\begin{solutionbox}
\begin{lstlisting}[language=Python]
def sum_positives():
    total_sum = 0
    while True:
        num = float(input("Enter number (negative to stop): "))
        if num < 0:
            break
        total_sum += num
    print(f"Sum of positive numbers: {total_sum}")

sum_positives()
\end{lstlisting}

\textbf{Process Flow:}
\begin{center}
\begin{tikzpicture}[gtu flow]
    \node [gtu start] (start) {Start};
    \node [gtu process, below=0.5cm of start] (init) {total = 0};
    \node [gtu input, below=0.5cm of init] (input) {Input Num};
    \node [gtu decision, below=0.5cm of input] (check) {Num < 0?};
    \node [gtu process, right=1.5cm of check] (add) {total += Num};
    \node [gtu output, below=0.5cm of check] (print) {Print total};
    \node [gtu stop, below=0.5cm of print] (end) {End};

    \path [gtu arrow] (start) -- (init);
    \path [gtu arrow] (init) -- (input);
    \path [gtu arrow] (input) -- (check);
    \path [gtu arrow] (check) -- node[right] {No} (add);
    \path [gtu arrow] (add) |- (input);
    \path [gtu arrow] (check) -- node[right] {Yes} (print);
    \path [gtu arrow] (print) -- (end);
\end{tikzpicture}
\captionof{figure}{Summation Logic}
\end{center}
\end{solutionbox}

\begin{mnemonicbox}
\mnemonic{Sum Till Negative}
\end{mnemonicbox}

\questionmarks{2(a OR)}{3}{Create a program to find a maximum number among the given three numbers.}

\begin{solutionbox}
\begin{lstlisting}[language=Python]
n1 = float(input("Num 1: "))
n2 = float(input("Num 2: "))
n3 = float(input("Num 3: "))

if n1 >= n2 and n1 >= n3:
    mx = n1
elif n2 >= n1 and n2 >= n3:
    mx = n2
else:
    mx = n3

print(f"Max: {mx}")
\end{lstlisting}

\textbf{Logic Flow:}
\begin{center}
\begin{tikzpicture}[gtu flow]
    \node [gtu input] (in) {Input 3 nums};
    \node [gtu decision, below=0.5cm of in] (c1) {n1 max?};
    \node [gtu decision, right=1cm of c1] (c2) {n2 max?};
    \node [gtu process, below=0.5cm of c1] (r1) {max=n1};
    \node [gtu process, below=0.5cm of c2] (r2) {max=n2};
    \node [gtu process, right=1cm of r2] (r3) {max=n3};
    
    \path [gtu arrow] (in) -- (c1);
    \path [gtu arrow] (c1) -- node[right] {Yes} (r1);
    \path [gtu arrow] (c1) -- node[above] {No} (c2);
    \path [gtu arrow] (c2) -- node[right] {Yes} (r2);
    \path [gtu arrow] (c2) -- node[above] {No} (r3);
\end{tikzpicture}
\captionof{figure}{Maximum Finder Logic}
\end{center}
\end{solutionbox}

\begin{mnemonicbox}
\mnemonic{Compare Each, Take Largest}
\end{mnemonicbox}

\questionmarks{2(b OR)}{4}{Given the str="abcdefghijklmnopqrstuvwxyz". Write a python program to extract every second character from above string.}

\begin{solutionbox}
\begin{lstlisting}[language=Python]
s = "abcdefghijklmnopqrstuvwxyz"
# Slice syntax: [start:end:step]
result = s[0::2]
print("Result:", result)
# Output: acegikmoqsuwy
\end{lstlisting}

\textbf{Slicing Visualization:}
\begin{center}
\begin{tikzpicture}[gtu flow]
    \foreach \x/\l in {0/a, 1/b, 2/c, 3/d, 4/e, 5/f} {
        \node[draw, minimum size=0.6cm] (n\x) at (\x*0.8, 0) {\l};
        \node[font=\footnotesize, below=0.1cm of n\x] {\x};
    }
    \node[right=0.2cm of n5] {...};
    
    \draw[->, thick, red] (n0.north) -- ++(0, 0.3) node[above] {Start};
    \draw[->, thick, blue] (n0) to[bend left] (n2);
    \draw[->, thick, blue] (n2) to[bend left] (n4);
    
    \node[above=0.8cm of n2] {Step=2};
\end{tikzpicture}
\captionof{figure}{String Slicing Step 2}
\end{center}
\end{solutionbox}

\begin{mnemonicbox}
\mnemonic{Slice Step Selector}
\end{mnemonicbox}

\questionmarks{2(c OR)}{7}{Write a Python program to create a dictionary that stores student names and their marks. Display the names of students who have scored more than 75 marks.}

\begin{solutionbox}
\begin{lstlisting}[language=Python]
students = {}
n = int(input("Enter count: "))

# Input Loop
for i in range(n):
    name = input("Name: ")
    marks = float(input("Marks: "))
    students[name] = marks

print("\nHigh Scorers (>75):")
for name, marks in students.items():
    if marks > 75:
        print(f"{name}: {marks}")
\end{lstlisting}

\textbf{Process Diagram:}
\begin{center}
\begin{tikzpicture}[gtu flow]
    \node [gtu start] (start) {Start};
    \node [gtu block, below=0.5cm of start] (dict) {Empty Dict};
    \node [gtu process, below=0.5cm of dict] (loop) {Input Loop};
    \node [gtu block, below=0.5cm of loop] (data) {Data Stored};
    \node [gtu decision, below=0.5cm of data] (check) {Marks > 75?};
    \node [gtu output, right=1cm of check] (disp) {Display Name};
    
    \path [gtu arrow] (start) -- (dict);
    \path [gtu arrow] (dict) -- (loop);
    \path [gtu arrow] (loop) -- (data);
    \path [gtu arrow] (data) -- (check);
    \path [gtu arrow] (check) -- node[above] {Yes} (disp);
    \path [gtu arrow] (check) -- node[left] {No} ++(0,-1);
    \path [gtu arrow] (disp) |- ++(0,-1);
\end{tikzpicture}
\captionof{figure}{Dictionary Filtering}
\end{center}
\end{solutionbox}

\begin{mnemonicbox}
\mnemonic{Store All, Filter Some}
\end{mnemonicbox}

\questionmarks{3(a)}{3}{Write a program to find the length of a string excluding spaces.}

\begin{solutionbox}
\begin{lstlisting}[language=Python]
s = input("Enter string: ")
no_spaces = s.replace(" ", "")
length = len(no_spaces)
print(f"Length excluding spaces: {length}")
\end{lstlisting}

\textbf{Visualization:}
\begin{center}
\begin{tikzpicture}[gtu flow]
    \node [gtu block] (orig) {"Hello World"};
    \node [gtu process, right=1cm of orig] (rep) {replace(" ","")};
    \node [gtu block, right=1cm of rep] (new) {"HelloWorld"};
    \node [gtu output, below=0.5cm of new] (len) {Length: 10};
    
    \path [gtu arrow] (orig) -- (rep);
    \path [gtu arrow] (rep) -- (new);
    \path [gtu arrow] (new) -- (len);
\end{tikzpicture}
\captionof{figure}{Space Removal}
\end{center}
\end{solutionbox}

\begin{mnemonicbox}
\mnemonic{Count Characters, Skip Spaces}
\end{mnemonicbox}

\questionmarks{3(b)}{4}{List the dictionary methods in python and explain each with suitable examples.}

\begin{solutionbox}
\begin{center}
\captionof{table}{Dictionary Methods}
\begin{tabulary}{\linewidth}{|L|L|L|}
\hline
\textbf{Method} & \textbf{Description} & \textbf{Example} \\ \hline
\code{get(k)} & Returns value for key & \code{d.get('a')} \\ \hline
\code{keys()} & Returns all keys & \code{list(d.keys())} \\ \hline
\code{values()} & Returns all values & \code{list(d.values())} \\ \hline
\code{items()} & Returns (key, value) pairs & \code{d.items()} \\ \hline
\code{pop(k)} & Removes item & \code{d.pop('a')} \\ \hline
\code{update()} & Merges dicts & \code{d.update(d2)} \\ \hline
\code{clear()} & Empties dict & \code{d.clear()} \\ \hline
\end{tabulary}
\end{center}
\end{solutionbox}

\begin{mnemonicbox}
\mnemonic{GCUP-KPIV}
\end{mnemonicbox}

\questionmarks{3(c)}{7}{Explain Python's List data type in detail.}

\begin{solutionbox}
List is an ordered, mutable collection that allows duplicate elements and mixed types.

\textbf{Key Features:}
\begin{itemize}
    \item \textbf{Ordered}: Items maintain order.
    \item \textbf{Mutable}: Can add, remove, change items.
    \item \textbf{Heterogeneous}: Can store int, str, float together.
\end{itemize}

\textbf{List Operations Diagram:}
\begin{center}
\begin{tikzpicture}[gtu flow]
    \node [gtu block] (init) {\code{[1, 2]}};
    \node [gtu process, right=1cm of init] (append) {\code{.append(3)}};
    \node [gtu block, right=1cm of append] (res1) {\code{[1, 2, 3]}};
    \node [gtu process, below=1cm of init] (pop) {\code{.pop(0)}};
    \node [gtu block, below=1cm of res1] (res2) {\code{[2, 3]}};
    
    \path [gtu arrow] (init) -- (append);
    \path [gtu arrow] (append) -- (res1);
    \path [gtu arrow] (res1) -- (pop);
    \path [gtu arrow] (pop) -- (res2);
\end{tikzpicture}
\captionof{figure}{List Operations}
\end{center}
\end{solutionbox}

\begin{mnemonicbox}
\mnemonic{CAMP-IS: Create, Access, Modify, Process}
\end{mnemonicbox}

\questionmarks{3(a OR)}{3}{Write a program to input a string from the user and print it in the reverse order without creating a new string.}

\begin{solutionbox}
\begin{lstlisting}[language=Python]
s = input("Enter string: ")
print(f"Reversed: {s[::-1]}")
\end{lstlisting}

\textbf{Reversing Logic:}
\begin{center}
\begin{tikzpicture}[gtu flow]
    \foreach \x/\l in {0/H, 1/e, 2/l, 3/l, 4/o} {
        \node[draw, minimum size=0.6cm] (n\x) at (\x*0.8, 0) {\l};
        \node[font=\footnotesize, below=0.1cm of n\x] {\x};
    }
    
    \draw[<-, thick, blue] (0,-0.6) -- (3.2,-0.6);
    \node[below=0.7cm of n2] {Step = -1 (Backwards)};
\end{tikzpicture}
\captionof{figure}{String Reversal}
\end{center}
\end{solutionbox}

\begin{mnemonicbox}
\mnemonic{Slice Backwards}
\end{mnemonicbox}

\questionmarks{3(b OR)}{4}{List the dictionary operations in python and explain each with suitable examples.}

\begin{solutionbox}
\begin{center}
\captionof{table}{Dictionary Operations}
\begin{tabulary}{\linewidth}{|L|L|L|}
\hline
\textbf{Operation} & \textbf{Syntax} & \textbf{Description} \\ \hline
Access & \code{d['key']} & Get value \\ \hline
Add/Mod & \code{d['k'] = v} & Insert/Update \\ \hline
Delete & \code{del d['k']} & Remove pair \\ \hline
Check & \code{'k' in d} & Membership \\ \hline
Length & \code{len(d)} & Count items \\ \hline
\end{tabulary}
\end{center}
\end{solutionbox}

\begin{mnemonicbox}
\mnemonic{CADMIL: Create Access Delete Modify Iterate Length}
\end{mnemonicbox}

\questionmarks{3(c OR)}{7}{Explain Python's set data type in detail.}

\begin{solutionbox}
Set is an unordered collection of unique elements.

\textbf{Set Characteristics:}
\begin{itemize}
    \item \textbf{Unique}: No duplicates.
    \item \textbf{Unordered}: No index access.
    \item \textbf{Math Ops}: Supports union, intersection.
\end{itemize}

\textbf{Set Operations Diagram:}
\begin{center}
\begin{tikzpicture}[gtu flow]
    \node [gtu start] (A) {A: \{1,2\}};
    \node [gtu start, right=1cm of A] (B) {B: \{2,3\}};
    
    \node [gtu block, below=1cm of A, xshift=1cm] (union) {A | B\\\{1,2,3\}};
    \node [gtu block, below=0.5cm of union] (inter) {A \& B\\\{2\}};
    
    \path [gtu arrow] (A) -- (union);
    \path [gtu arrow] (B) -- (union);
\end{tikzpicture}
\captionof{figure}{Set Union \& Intersection}
\end{center}
\end{solutionbox}

\begin{mnemonicbox}
\mnemonic{SUMO: Set Unique Mutable Ordered-less}
\end{mnemonicbox}

\questionmarks{4(a)}{3}{Explain statistics module with any three methods.}

\begin{solutionbox}
\begin{center}
\captionof{table}{Statistics Methods}
\begin{tabulary}{\linewidth}{|L|L|L|}
\hline
\textbf{Method} & \textbf{Description} & \textbf{Example} \\ \hline
\code{mean()} & Average & \code{mean([1,2,3])} $\to$ 2 \\ \hline
\code{median()} & Middle value & \code{median([1,5,9])} $\to$ 5 \\ \hline
\code{mode()} & Most Frequent & \code{mode([1,1,2])} $\to$ 1 \\ \hline
\end{tabulary}
\end{center}
\end{solutionbox}

\begin{mnemonicbox}
\mnemonic{MMM Stats}
\end{mnemonicbox}

\questionmarks{4(b)}{4}{Explain function of user define function and user defined module in Python.}

\begin{solutionbox}
\begin{center}
\captionof{table}{Function vs Module}
\begin{tabulary}{\linewidth}{|L|L|L|}
\hline
\textbf{Feature} & \textbf{Function} & \textbf{Module} \\ \hline
Unit & Code Block & File (.py) \\ \hline
Creation & \code{def name():} & Save as .py \\ \hline
Usage & Call \code{name()} & \code{import name} \\ \hline
Scope & Local & Global/Imported \\ \hline
\end{tabulary}
\end{center}

\textbf{Module Structure:}
\begin{center}
\begin{tikzpicture}[gtu flow]
    \node [gtu block] (main) {Main Program};
    \node [gtu block, right=2cm of main] (mod) {Module.py};
    \draw [gtu arrow] (main) -- node[above] {import} (mod);
    \draw [gtu arrow] (mod) -- node[below] {functions} (main);
\end{tikzpicture}
\captionof{figure}{Import Relationship}
\end{center}
\end{solutionbox}

\begin{mnemonicbox}
\mnemonic{FIR-MID}
\end{mnemonicbox}

\questionmarks{4(c)}{7}{Write a Python code using user defined function to find the factorial of a given number using recursion.}

\begin{solutionbox}
\begin{lstlisting}[language=Python]
def factorial(n):
    # Base case
    if n == 0 or n == 1:
        return 1
    # Recursive case
    else:
        return n * factorial(n-1)

num = int(input("Enter num: "))
print(f"Factorial: {factorial(num)}")
\end{lstlisting}

\textbf{Recursion Visualization:}
\begin{center}
\begin{tikzpicture}[gtu flow]
    \node [gtu process] (call4) {fact(4)};
    \node [gtu process, right=0.5cm of call4] (call3) {fact(3)};
    \node [gtu process, right=0.5cm of call3] (call2) {fact(2)};
    \node [gtu process, right=0.5cm of call2] (call1) {fact(1)};
    \node [gtu output, below=0.5cm of call1] (ret1) {return 1};
    
    \path [gtu arrow] (call4) -- (call3);
    \path [gtu arrow] (call3) -- (call2);
    \path [gtu arrow] (call2) -- (call1);
    \path [gtu arrow] (call1) -- (ret1);
    \path [gtu arrow, dashed] (ret1) -- node[below] {result} (call4.south);
\end{tikzpicture}
\captionof{figure}{Recursion Chain}
\end{center}
\end{solutionbox}

\begin{mnemonicbox}
\mnemonic{Number times (Number minus one)!}
\end{mnemonicbox}

\questionmarks{4(a OR)}{3}{Explain math module with any three methods.}

\begin{solutionbox}
\begin{center}
\captionof{table}{Math Methods}
\begin{tabulary}{\linewidth}{|L|L|L|}
\hline
\textbf{Method} & \textbf{Description} & \textbf{Example} \\ \hline
\code{sqrt()} & Square Root & \code{sqrt(16)} $\to$ 4.0 \\ \hline
\code{pow()} & Power & \code{pow(2,3)} $\to$ 8.0 \\ \hline
\code{ceil()} & Round Up & \code{ceil(4.1)} $\to$ 5 \\ \hline
\end{tabulary}
\end{center}
\end{solutionbox}

\begin{mnemonicbox}
\mnemonic{SPT Math}
\end{mnemonicbox}

\questionmarks{4(b OR)}{4}{Explain the concepts of global and local variables in Python.}

\begin{solutionbox}
\begin{center}
\captionof{table}{Variable Scope}
\begin{tabulary}{\linewidth}{|L|L|L|}
\hline
\textbf{Type} & \textbf{Scope} & \textbf{Access} \\ \hline
Global & Entire Program & Anywhere \\ \hline
Local & Inside Function & Within Function Only \\ \hline
\end{tabulary}
\end{center}

\textbf{Scope Diagram:}
\begin{center}
\begin{tikzpicture}[gtu flow]
    \node [gtu block, minimum width=5cm, minimum height=3cm, label=above:Global Scope] (global) {};
    \node [gtu block, minimum width=3cm, minimum height=1.5cm, fill=white, label=above:Function Scope] (local) at (global.center) {};
    
    \node at ([yshift=1cm]global.center) {global\_var};
    \node at (local.center) {local\_var};
\end{tikzpicture}
\captionof{figure}{Scope Hierarchy}
\end{center}
\end{solutionbox}

\begin{mnemonicbox}
\mnemonic{GLOBAL Goes Everywhere, LOCAL Lives in Functions}
\end{mnemonicbox}

\questionmarks{4(c OR)}{7}{Create code with user defined function to check if given string is palindrome or not.}

\begin{solutionbox}
\begin{lstlisting}[language=Python]
def is_palindrome(text):
    raw = text.replace(" ", "").lower()
    return raw == raw[::-1]

s = input("Enter string: ")
if is_palindrome(s):
    print("Palindrome")
else:
    print("Not Palindrome")
\end{lstlisting}

\textbf{Logic Flow:}
\begin{center}
\begin{tikzpicture}[gtu flow]
    \node [gtu start] (in) {Input};
    \node [gtu process, right=1cm of in] (clean) {Clean (space/case)};
    \node [gtu decision, right=1cm of clean] (check) {Equals Reverse?};
    \node [gtu output, right=1cm of check] (res) {Result};
    
    \path [gtu arrow] (in) -- (clean);
    \path [gtu arrow] (clean) -- (check);
    \path [gtu arrow] (check) -- (res);
\end{tikzpicture}
\captionof{figure}{Palindrome Check}
\end{center}
\end{solutionbox}

\begin{mnemonicbox}
\mnemonic{Clean, Reverse, Compare}
\end{mnemonicbox}

\questionmarks{5(a)}{3}{Define class and object with example.}

\begin{solutionbox}
\begin{itemize}
    \item \textbf{Class}: Blueprint/Template.
    \item \textbf{Object}: Instance of class.
\end{itemize}

\textbf{Relationship Diagram:}
\begin{center}
\begin{tikzpicture}[gtu flow]
    \node [gtu class] (cls) {Class: Dog};
    \node [gtu block, below left=1cm of cls] (o1) {Dog("Rex")};
    \node [gtu block, below right=1cm of cls] (o2) {Dog("Buddy")};
    
    \draw [gtu arrow] (cls) -- (o1);
    \draw [gtu arrow] (cls) -- (o2);
\end{tikzpicture}
\captionof{figure}{Instantiation}
\end{center}
\end{solutionbox}

\begin{mnemonicbox}
\mnemonic{CAMBO: Classes Are Molds, Build Objects}
\end{mnemonicbox}

\questionmarks{5(b)}{4}{Classify constructor. Explain any one in detail.}

\begin{solutionbox}
\textbf{Types}: Default, Parameterized, Non-parameterized, Copy.

\textbf{Parameterized Constructor:}
\begin{lstlisting}[language=Python]
class Student:
    def __init__(self, name):
        self.name = name
s = Student("Alice")
\end{lstlisting}

\textbf{Execution Flow:}
\begin{center}
\begin{tikzpicture}[gtu flow]
    \node [gtu start] (new) {Object Created};
    \node [gtu process, right=1cm of new] (init) {\_\_init\_\_ called};
    \node [gtu process, right=1cm of init] (set) {Attrs Set};
    \node [gtu stop, right=1cm of set] (rdy) {Ready};
    
    \path [gtu arrow] (new) -- (init);
    \path [gtu arrow] (init) -- (set);
    \path [gtu arrow] (set) -- (rdy);
\end{tikzpicture}
\captionof{figure}{Constructor Lifecycle}
\end{center}
\end{solutionbox}

\begin{mnemonicbox}
\mnemonic{PICAN}
\end{mnemonicbox}

\questionmarks{5(c)}{7}{Develop and explain a python code to implement hierarchical inheritance.}

\begin{solutionbox}
\begin{lstlisting}[language=Python]
class Vehicle:
    def start(self): print("Engine On")

class Car(Vehicle):
    def drive(self): print("Driving")

class Bike(Vehicle):
    def ride(self): print("Riding")

c = Car(); c.start()
b = Bike(); b.start()
\end{lstlisting}

\textbf{Inheritance Tree:}
\begin{center}
\begin{tikzpicture}[gtu flow]
    \node [gtu class] (parent) {Vehicle};
    \node [gtu class, below left=1cm and 0.5cm of parent] (c1) {Car};
    \node [gtu class, below right=1cm and 0.5cm of parent] (c2) {Bike};
    
    \draw [gtu arrow] (parent) -- (c1);
    \draw [gtu arrow] (parent) -- (c2);
\end{tikzpicture}
\captionof{figure}{Hierarchical Inheritance}
\end{center}
\end{solutionbox}

\begin{mnemonicbox}
\mnemonic{Parents Share, Children Specialize}
\end{mnemonicbox}

\questionmarks{5(a OR)}{3}{What is the \_\_init\_\_ method in Python? Explain its purpose with a suitable example.}

\begin{solutionbox}
Special method automatically called during object creation to initialize attributes.

\textbf{Example:}
\begin{lstlisting}[language=Python]
class Rect:
    def __init__(self, w, h):
        self.w = w
        self.h = h
\end{lstlisting}
\end{solutionbox}

\begin{mnemonicbox}
\mnemonic{ASAP: Attributes Set At Production}
\end{mnemonicbox}

\questionmarks{5(b OR)}{4}{Classify methods in Python class. Explain any one in detail.}

\begin{solutionbox}
\begin{center}
\captionof{table}{Method Types}
\begin{tabulary}{\linewidth}{|L|L|L|}
\hline
\textbf{Type} & \textbf{Access} & \textbf{Decorator} \\ \hline
Instance & \code{self} & None \\ \hline
Class & \code{cls} & \code{@classmethod} \\ \hline
Static & None & \code{@staticmethod} \\ \hline
\end{tabulary}
\end{center}

\textbf{Instance Method}:
\begin{lstlisting}[language=Python]
def display(self):
    print(self.name)
\end{lstlisting}
\end{solutionbox}

\begin{mnemonicbox}
\mnemonic{SIAM: Self Is Always Mentioned}
\end{mnemonicbox}

\questionmarks{5(c OR)}{7}{Develop a Python code for Polymorphism and explain it.}

\begin{solutionbox}
\begin{lstlisting}[language=Python]
class Dog:
    def sound(self): return "Woof"

class Cat:
    def sound(self): return "Meow"

animals = [Dog(), Cat()]
for a in animals:
    print(a.sound())
\end{lstlisting}

\textbf{Polymorphism Diagram:}
\begin{center}
\begin{tikzpicture}[gtu flow]
    \node [gtu class] (call) {Call .sound()};
    \node [gtu class, below left=1cm of call] (dog) {Dog: Woof};
    \node [gtu class, below right=1cm of call] (cat) {Cat: Meow};
    
    \draw [gtu arrow] (call) -- node[left] {If Dog} (dog);
    \draw [gtu arrow] (call) -- node[right] {If Cat} (cat);
\end{tikzpicture}
\captionof{figure}{Dynamic Binding}
\end{center}
\end{solutionbox}

\begin{mnemonicbox}
\mnemonic{Same Method, Different Behavior}
\end{mnemonicbox}

\end{document}
