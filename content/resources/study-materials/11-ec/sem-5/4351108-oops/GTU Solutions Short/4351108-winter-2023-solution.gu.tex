\documentclass{article}
% Adjust the relative path to point to the latex-templates directory

% content/resources/templates/preamble.tex
\usepackage[margin=0.6in]{geometry}
\author{Milav Dabgar}
\usepackage{amsmath,amssymb,amsthm}
\usepackage{booktabs}
\usepackage{multirow}
\usepackage{xcolor}
\usepackage{tcolorbox}
\tcbuselibrary{breakable,skins}
\usepackage[colorlinks=true,linkcolor=blue]{hyperref}
\usepackage{titlesec}
\usepackage{enumitem}
\usepackage{tikz}
\usepackage{pgfplots}
\usepackage{circuitikz}
\usepackage[version=4]{mhchem}
\usepackage{longtable}
\usepackage{array}
\usepackage{float}
\usepackage{caption}
\usepackage{listings}

\lstset{
  basicstyle=\small\ttfamily,
  breaklines=true,
  breakatwhitespace=false,
  postbreak=\mbox{\textcolor{red}{$\hookrightarrow$}\space},
  float=false,
  numbers=left,
  numberstyle=\tiny\color{gray},
  numbersep=10pt,
  xleftmargin=2em,
  keywordstyle=\color{blue},
  commentstyle=\color{green!60!black},
  stringstyle=\color{purple},
  backgroundcolor=\color{gray!5},
  showstringspaces=false,
  tabsize=2,
  captionpos=b,
  keepspaces=true,
  columns=flexible
}

\pgfplotsset{compat=1.18}
\usetikzlibrary{shapes,arrows,positioning,calc,patterns,decorations.pathmorphing,decorations.markings,arrows.meta}

% Color scheme
\definecolor{headcolor}{RGB}{0,102,204}
\definecolor{keycolor}{RGB}{220,20,60}
\definecolor{solutioncolor}{RGB}{34,139,34}
\definecolor{mnemoniccolor}{RGB}{148,0,211}
\definecolor{codecolor}{RGB}{0,0,100}

% Spacing
\setlength{\parskip}{3pt}
\setlist[itemize]{nosep}
\setlist[enumerate]{nosep}

% Title formatting
\titleformat{\section}{\Large\bfseries\color{headcolor}}{\thesection}{1em}{}
\titleformat{\subsection}{\large\bfseries\color{headcolor}}{\thesubsection}{1em}{}

% Pandoc tightlist compatibility
\providecommand{\tightlist}{%
  \setlength{\itemsep}{0pt}\setlength{\parskip}{0pt}}

% Pandoc longtable compatibility
\newcounter{none}
\def\thenone{}


% content/resources/templates/gujarati-boxes.tex
\usepackage{fontspec}
\usepackage{polyglossia}

% Set Gujarati as main language (document is primarily in Gujarati)
% Note: gloss-gujarati.ldf doesn't exist in polyglossia, but it will use hyphenation patterns
\setdefaultlanguage{gujarati}
\setotherlanguage{english}

% Configure Gujarati font properly
% Use Language=Default to prevent polyglossia from trying to add language-specific features
% that don't exist for Gujarati, which causes "empty feature" warnings
\newfontfamily\gujaratifont[Script=Gujarati,AutoFakeBold=2.5,AutoFakeSlant=0.3]{Noto Sans Gujarati}
\setmainfont[Script=Gujarati,AutoFakeBold=2.5,AutoFakeSlant=0.3]{Noto Sans Gujarati}
% Use Noto Sans Gujarati for monospace to support Gujarati in text
\setmonofont[Scale=0.9]{Noto Sans Gujarati}

% Configure English to use the same font
\newfontfamily\englishfont[Script=Gujarati,AutoFakeBold=2.5,AutoFakeSlant=0.3]{Noto Sans Gujarati}

% Translations for polyglossia
\gappto\captionsgujarati{
  \renewcommand{\tablename}{કોષ્ટક}
  \renewcommand{\figurename}{આકૃતિ}
}

% Helper for TikZ nodes to ensure Gujarati font
\newcommand{\gu}[1]{{\gujaratifont #1}}

% Custom environments
\newtcolorbox{solutionbox}{
    breakable,
    enhanced,
    colback=solutioncolor!5!white,
    colframe=solutioncolor!75!black,
    fonttitle=\bfseries,
    title=જવાબ
}

\newtcolorbox{solutionboxnobreak}{
 colback=solutioncolor!5!white,
 colframe=solutioncolor!75!black,
 fonttitle=\bfseries,
 title=જવાબ
}

\newtcolorbox{keyformula}{
 breakable,
 enhanced,
 colback=keycolor!5!white,
 colframe=keycolor!75!black,
 fonttitle=\bfseries,
 title=રાસાયણિક સમીકરણ/સૂત્ર
}

\newtcolorbox{mnemonicbox}{
 breakable,
 enhanced,
 colback=mnemoniccolor!5!white,
 colframe=mnemoniccolor!75!black,
 fonttitle=\bfseries,
 title=મેમરી ટ્રીક
}


% Custom commands for GTU solutions
% This file defines semantic commands for consistent formatting

% Question command with automatic formatting
\newcommand{\question}[2]{%
  \section*{Question #1}%
  \textbf{#2}%
}

% OR question variant
\newcommand{\questionor}[2]{%
  \section*{Question #1 OR}%
  \textbf{#2}%
}

% Proper table environment with caption
\newenvironment{answertable}[1]{%
  \begin{table}[htbp]
  \centering
  \caption{#1}
}{%
  \end{table}
}

% Proper figure environment for diagrams
\newenvironment{answerdiagram}[1]{%
  \begin{figure}[htbp]
  \centering
  \caption{#1}
}{%
  \end{figure}
}

% Semantic markup for key terms
\newcommand{\keyword}[1]{\textbf{#1}}
\newcommand{\code}[1]{\texttt{#1}}
\newcommand{\classname}[1]{\texttt{#1}}
\newcommand{\methodname}[1]{\texttt{#1}}

% Proper quotation marks
\newcommand{\mnemonic}[1]{``#1''}


\title{OOPS \& Python Programming (4351108) - Winter 2023 Solution}
\date{December 06, 2023}

\begin{document}
\maketitle

\questionmarks{1(અ)}{3}{પાયથન પ્રોગ્રામિંગ લેન્ગવેજના કોઈ પણ 6 ઉપયોગો લખો.}

\begin{solutionbox}
\begin{center}
\captionof{table}{પાયથનના ઉપયોગો}
\begin{tabulary}{\linewidth}{|L|L|}
\hline
\textbf{ઉપયોગ ક્ષેત્ર} & \textbf{વર્ણન} \\ \hline
\keyword{વેબ ડેવેલપમેન્ટ} & Django, Flask frameworks \\ \hline
\keyword{ડેટા સાયન્સ} & Analysis અને visualization \\ \hline
\keyword{મશીન લર્નિંગ} & AI model development \\ \hline
\keyword{ડેસ્કટોપ એપ્લિકેશન} & GUI using Tkinter, PyQt \\ \hline
\keyword{ગેમ ડેવેલપમેન્ટ} & Pygame library \\ \hline
\keyword{ઑટોમેશન} & Scripting અને testing \\ \hline
\end{tabulary}
\end{center}
\end{solutionbox}

\begin{mnemonicbox}
\mnemonic{Web Data Machine Desktop Game Auto}
\end{mnemonicbox}

\questionmarks{1(બ)}{4}{પાયથન પ્રોગ્રામિંગ લેન્ગવેજની કોઈ પણ 8 વિશેષતાઓ લખો.}

\begin{solutionbox}
\begin{center}
\captionof{table}{પાયથનની વિશેષતાઓ}
\begin{tabulary}{\linewidth}{|L|L|}
\hline
\textbf{વિશેષતા} & \textbf{વર્ણન} \\ \hline
\keyword{સરળ સિન્ટેક્સ} & વાંચવા અને લખવામાં સરળ \\ \hline
\keyword{ઇન્ટરપ્રિટેડ} & Compilation ની જરૂર નથી \\ \hline
\keyword{ઑબ્જેક્ટ-ઓરિએન્ટેડ} & OOP concepts સપોર્ટ કરે છે \\ \hline
\keyword{ડાયનેમિક ટાઇપિંગ} & Variables ને type declaration જરૂરી નથી \\ \hline
\keyword{ક્રોસ-પ્લેટફોર્મ} & Multiple OS પર ચાલે છે \\ \hline
\keyword{મોટી લાઇબ્રેરીઓ} & Rich standard library \\ \hline
\keyword{ઓપન સોર્સ} & ઉપયોગ અને modify કરવા માટે મફત \\ \hline
\keyword{ઇન્ટરેક્ટિવ} & REPL environment \\ \hline
\end{tabulary}
\end{center}
\end{solutionbox}

\begin{mnemonicbox}
\mnemonic{Simple Interpreted Object Dynamic Cross Large Open Interactive}
\end{mnemonicbox}

\questionmarks{1(ક)}{7}{પાયથનની for અને while લૂપનું કાર્ય સમજાવો.}

\begin{solutionbox}
\textbf{For Loop:}
\begin{itemize}
    \item \keyword{પુનરાવર્તન}: Sequences પર પુનરાવર્તન કરે છે (lists, strings, ranges)
    \item \keyword{સિન્ટેક્સ}: \code{for variable in sequence:}
    \item \keyword{આપોઆપ}: Iteration આપોઆપ handle કરે છે
\end{itemize}

\textbf{While Loop:}
\begin{itemize}
    \item \keyword{શરત આધારિત}: જ્યાં સુધી શરત સાચી રહે છે
    \item \keyword{મેન્યુઅલ નિયંત્રણ}: Programmer iteration નો નિયંત્રણ કરે છે
    \item \keyword{જોખમ}: શરત કદી false ન બને તો infinite loop બની શકે છે
\end{itemize}

\textbf{લૂપ ડાયાગ્રામ:}
\begin{center}
\begin{tikzpicture}[gtu flow]
    \node [gtu start] (start) {Start};
    \node [gtu decision, below=0.5cm of start] (cond) {Condition?};
    \node [gtu process, below=0.5cm of cond] (exec) {Execute Body};
    \node [gtu process, below=0.5cm of exec] (update) {Update};
    \node [gtu stop, right=2cm of cond] (end) {End};
    
    \path [gtu arrow] (start) -- (cond);
    \path [gtu arrow] (cond) -- node[right] {True} (exec);
    \path [gtu arrow] (exec) -- (update);
    \path [gtu arrow] (update.west) -- ++(-0.5,0) |- (cond);
    \path [gtu arrow] (cond) -- node[above] {False} (end);
\end{tikzpicture}
\captionof{figure}{General Loop Flow}
\end{center}

\textbf{કોડ ઉદાહરણ:}
\begin{lstlisting}[language=Python]
# For loop
for i in range(5):
    print(i)

# While loop
i = 0
while i < 5:
    print(i)
    i += 1
\end{lstlisting}
\end{solutionbox}

\begin{mnemonicbox}
\mnemonic{For Automatic, While Manual}
\end{mnemonicbox}

\questionmarks{1(ક OR)}{7}{પાયથનના break, continue અને pass સ્ટેટમેન્ટના કાર્ય સમજાવો.}

\begin{solutionbox}
\textbf{Break Statement:}
\begin{itemize}
    \item \keyword{બહાર નીકળવું}: સંપૂર્ણ loop ને terminate કરે છે
    \item \keyword{ઉપયોગ}: જ્યારે કોઈ specific condition મળે છે
    \item \keyword{અસર}: Control loop પછીના statement પર જાય છે
\end{itemize}

\textbf{Continue Statement:}
\begin{itemize}
    \item \keyword{છોડીને આગળ}: ફક્ત current iteration skip કરે છે
    \item \keyword{ઉપયોગ}: Iteration માં specific values skip કરવા માટે
    \item \keyword{અસર}: Next iteration પર જાય છે
\end{itemize}

\textbf{Pass Statement:}
\begin{itemize}
    \item \keyword{Placeholder}: કંઈ કરતું નથી, syntactic placeholder
    \item \keyword{ઉપયોગ}: જ્યારે syntax statement જોઈએ પણ કોઈ action નહીં
    \item \keyword{અસર}: કોઈ operation perform કરતું નથી
\end{itemize}

\textbf{Control Flow Visualization:}
\begin{center}
\begin{tikzpicture}[gtu flow]
    \node [gtu start] (start) {Loop Start};
    \node [gtu decision, below=0.5cm of start] (cond) {Condition};
    \node [gtu decision, below=0.5cm of cond] (break) {break?};
    \node [gtu decision, below=0.5cm of break] (cont) {continue?};
    \node [gtu decision, below=0.5cm of cont] (pass) {pass?};
    \node [gtu process, below=0.5cm of pass] (exec) {Execute Code};
    \node [gtu stop, right=2cm of break] (exit) {Exit Loop};
    
    \path [gtu arrow] (start) -- (cond);
    \path [gtu arrow] (cond) -- node[right] {True} (break);
    \path [gtu arrow] (break) -- node[above] {Yes} (exit);
    \path [gtu arrow] (break) -- node[right] {No} (cont);
    \path [gtu arrow] (cont) -- node[right] {No} (pass);
    \path [gtu arrow] (cont.west) -- node[above] {Yes} ++(-0.5,0) |- (cond);
    \path [gtu arrow] (pass) -- (exec);
    \path [gtu arrow] (exec.west) -- ++(-0.5,0) |- (cond);
\end{tikzpicture}
\captionof{figure}{Loop Control Statements}
\end{center}

\textbf{કોડ ઉદાહરણો:}
\begin{lstlisting}[language=Python]
# Break
for i in range(10):
    if i == 5: break
    print(i)  # 0,1,2,3,4

# Continue
for i in range(5):
    if i == 2: continue
    print(i)  # 0,1,3,4

# Pass
if True: pass  # placeholder
\end{lstlisting}
\end{solutionbox}

\begin{mnemonicbox}
\mnemonic{Break Exits, Continue Skips, Pass Waits}
\end{mnemonicbox}

\questionmarks{2(અ)}{3}{લિસ્ટના દરેક ઘટકનું મૂલ્ય 1 થી વધારવા માટેનો પાયથન પ્રોગ્રામ લખો.}

\begin{solutionbox}
\textbf{કોડ:}
\begin{lstlisting}[language=Python]
# Method 1 - Using for loop
numbers = [1, 2, 3, 4, 5]
for i in range(len(numbers)):
    numbers[i] += 1
print(numbers)

# Method 2 - List comprehension
numbers = [1, 2, 3, 4, 5]
result = [x + 1 for x in numbers]
print(result)
\end{lstlisting}
\end{solutionbox}

\begin{mnemonicbox}
\mnemonic{Loop Index or Comprehension}
\end{mnemonicbox}

\questionmarks{2(બ)}{4}{વપરાશકર્તા પાસેથી 3 સંખ્યા લઈ તેની સરેરાશ શોધવા માટેનો પાયથન પ્રોગ્રામ લખો.}

\begin{solutionbox}
\textbf{કોડ:}
\begin{lstlisting}[language=Python]
# Input three numbers
num1 = float(input("Enter first number: "))
num2 = float(input("Enter second number: "))
num3 = float(input("Enter third number: "))

# Calculate average
average = (num1 + num2 + num3) / 3

# Display result
print(f"Average is: {average}")
\end{lstlisting}

\textbf{મુખ્ય મુદ્દાઓ:}
\begin{itemize}
    \item \keyword{ઇનપુટ}: દશાંશ સંખ્યાઓ માટે \code{float()} ઉપયોગ કરો
    \item \keyword{સૂત્ર}: બધાનો સરવાળો કરીને સંખ્યા વડે ભાગો
    \item \keyword{આઉટપુટ}: Formatting માટે f-string ઉપયોગ કરો
\end{itemize}
\end{solutionbox}

\begin{mnemonicbox}
\mnemonic{Input Float, Sum Divide, Format Output}
\end{mnemonicbox}

\questionmarks{2(ક)}{7}{પાયથનનો list ડેટા ટાઈપ વિસ્તારથી સમજાવો.}

\begin{solutionbox}
\textbf{લિસ્ટની લાક્ષણિકતાઓ:}
\begin{itemize}
    \item \keyword{ક્રમબદ્ધ}: Elements sequence જાળવે છે
    \item \keyword{બદલાવ પાત્ર}: બનાવ્યા પછી modify કરી શકાય છે
    \item \keyword{વિવિધ પ્રકારની}: વિવિધ data types store કરી શકે છે
    \item \keyword{ઇન્ડેક્સવાળી}: Index વડે elements ને access કરી શકાય છે (0-based)
\end{itemize}

\textbf{લિસ્ટ ઑપરેશન્સ ટેબલ:}
\begin{center}
\captionof{table}{List Operations}
\begin{tabulary}{\linewidth}{|L|L|L|}
\hline
\textbf{ઑપરેશન} & \textbf{Syntax} & \textbf{વર્ણન} \\ \hline
\textbf{બનાવવું} & \code{list = [1,2,3]} & નવી list બનાવો \\ \hline
\textbf{એક્સેસ} & \code{list[0]} & Index વડે element મેળવો \\ \hline
\textbf{Append} & \code{list.append(4)} & અંતે element ઉમેરો \\ \hline
\textbf{Insert} & \code{list.insert(1,5)} & Specific position પર ઉમેરો \\ \hline
\textbf{Remove} & \code{list.remove(2)} & પહેલું occurrence દૂર કરો \\ \hline
\textbf{Pop} & \code{list.pop()} & છેલ્લું element દૂર કરીને return કરો \\ \hline
\textbf{Slice} & \code{list[1:3]} & Sublist મેળવો \\ \hline
\end{tabulary}
\end{center}

\textbf{List Structure Diagram:}
\begin{center}
\begin{tikzpicture}[gtu flow]
    \node [gtu block] (list) {\code{[10, 20, 30]}};
    \node [gtu process, right=1.5cm of list] (op1) {append(40)};
    \node [gtu block, right=1.5cm of op1] (res1) {\code{[10, 20, 30, 40]}};
    \node [gtu process, below=1cm of list] (op2) {pop(0)};
    \node [gtu block, below=1cm of res1] (res2) {\code{[20, 30, 40]}};
    
    \draw [gtu arrow] (list) -- (op1);
    \draw [gtu arrow] (op1) -- (res1);
    \draw [gtu arrow] (res1) -- (op2);
    \draw [gtu arrow] (op2) -- (res2);
\end{tikzpicture}
\captionof{figure}{List Mutation}
\end{center}

\textbf{કોડ ઉદાહરણ:}
\begin{lstlisting}[language=Python]
# List creation and operations
fruits = ['apple', 'banana', 'orange']
fruits.append('mango')
fruits.insert(1, 'grape')
print(fruits[0])  # apple
print(len(fruits))  # 5
\end{lstlisting}
\end{solutionbox}

\begin{mnemonicbox}
\mnemonic{Ordered Mutable Heterogeneous Indexed}
\end{mnemonicbox}

\questionmarks{2(અ OR)}{3}{for લૂપની મદદથી લિસ્ટના દરેક ઘટકનો સરવાળો શોધવા માટેનો પાયથન પ્રોગ્રામ લખો.}

\begin{solutionbox}
\textbf{કોડ:}
\begin{lstlisting}[language=Python]
# Method 1 - Traditional for loop
numbers = [10, 20, 30, 40, 50]
total = 0
for num in numbers:
    total += num
print(f"Sum is: {total}")

# Method 2 - Using range and index
numbers = [10, 20, 30, 40, 50]
total = 0
for i in range(len(numbers)):
    total += numbers[i]
print(f"Sum is: {total}")
\end{lstlisting}
\end{solutionbox}

\begin{mnemonicbox}
\mnemonic{Initialize Zero, Loop Add, Print Total}
\end{mnemonicbox}

\questionmarks{2(બ OR)}{4}{વપરાશકર્તા પાસેથી લીધેલા મૂડલ, વ્યાજદર અને વર્ષ પરથી સાદું વ્યાજ શોધવા માટેનો પાયથન પ્રોગ્રામ લખો.}

\begin{solutionbox}
\textbf{કોડ:}
\begin{lstlisting}[language=Python]
# Get input from user
principal = float(input("Enter principal amount: "))
rate = float(input("Enter rate of interest: "))
time = float(input("Enter time in years: "))

# Calculate simple interest
simple_interest = (principal * rate * time) / 100

# Display results
print(f"Principal: {principal}")
print(f"Rate: {rate}%")
print(f"Time: {time} years")
print(f"Simple Interest: {simple_interest}")
print(f"Total Amount: {principal + simple_interest}")
\end{lstlisting}

\textbf{સૂત્ર:}
\begin{itemize}
    \item \textbf{સાદું વ્યાજ} = (P $\times$ R $\times$ T) / 100
    \item \textbf{કુલ રકમ} = મૂડલ + સાદું વ્યાજ
\end{itemize}
\end{solutionbox}

\begin{mnemonicbox}
\mnemonic{Principal Rate Time, Multiply Divide Hundred}
\end{mnemonicbox}

\questionmarks{2(ક OR)}{7}{પાયથનનો tuple ડેટા ટાઈપ વિસ્તારથી સમજાવો.}

\begin{solutionbox}
\textbf{ટ્યૂપલની લાક્ષણિકતાઓ:}
\begin{itemize}
    \item \keyword{ક્રમબદ્ધ}: Elements sequence જાળવે છે
    \item \keyword{અપરિવર્તનીય}: બનાવ્યા પછી modify કરી શકાતું નથી
    \item \keyword{વિવિધ પ્રકારની}: વિવિધ data types store કરી શકે છે
    \item \keyword{ઇન્ડેક્સવાળી}: Index વડે access કરી શકાય છે (0-based)
\end{itemize}

\textbf{ટ્યૂપલ ઑપરેશન્સ ટેબલ:}
\begin{center}
\captionof{table}{Tuple Operations}
\begin{tabulary}{\linewidth}{|L|L|L|}
\hline
\textbf{ઑપરેશન} & \textbf{Syntax} & \textbf{વર્ણન} \\ \hline
\textbf{બનાવવું} & \code{tuple = (1,2,3)} & નવું tuple બનાવો \\ \hline
\textbf{એક્સેસ} & \code{tuple[0]} & Index વડે element મેળવો \\ \hline
\textbf{Count} & \code{tuple.count(2)} & Occurrences ગિનો \\ \hline
\textbf{Index} & \code{tuple.index(3)} & પહેલો index શોધો \\ \hline
\textbf{Slice} & \code{tuple[1:3]} & Sub-tuple મેળવો \\ \hline
\textbf{Length} & \code{len(tuple)} & Tuple નું size મેળવો \\ \hline
\textbf{Concatenate} & \code{tuple1 + tuple2} & Tuples જોડો \\ \hline
\end{tabulary}
\end{center}

\textbf{Tuple Comparison Diagram:}
\begin{center}
\begin{tikzpicture}[gtu flow]
    \node [gtu block] (tuple) {\textbf{Tuple}\\ \code{(1, 2, 3)}\\ Immutable};
    \node [gtu block, right=2cm of tuple] (list) {\textbf{List}\\ \code{[1, 2, 3]}\\ Mutable};
    
    \draw [gtu arrow] (tuple) -- node[above] {Fixed} (list);
    \node [below=0.2cm of tuple] {Read-Only};
    \node [below=0.2cm of list] {Read-Write};
\end{tikzpicture}
\captionof{figure}{Tuple vs List}
\end{center}

\textbf{કોડ ઉદાહરણ:}
\begin{lstlisting}[language=Python]
# Tuple creation and operations
coordinates = (10, 20, 30)
print(coordinates[0])  # 10
print(len(coordinates))  # 3
x, y, z = coordinates  # tuple unpacking
new_tuple = coordinates + (40, 50)
\end{lstlisting}
\end{solutionbox}

\begin{mnemonicbox}
\mnemonic{Ordered Immutable Heterogeneous Indexed}
\end{mnemonicbox}

\questionmarks{3(અ)}{3}{random મોડ્યૂલની કોઈ પણ 3 મેથડ સમજાવો.}

\begin{solutionbox}
\begin{center}
\captionof{table}{Random Module Methods}
\begin{tabulary}{\linewidth}{|L|L|L|}
\hline
\textbf{મેથડ} & \textbf{Syntax} & \textbf{વર્ણન} \\ \hline
\textbf{random()} & \code{random.random()} & 0.0 થી 1.0 વચ્ચે float \\ \hline
\textbf{randint()} & \code{random.randint(a,b)} & આપેલી range વચ્ચે integer \\ \hline
\textbf{choice()} & \code{random.choice(list)} & Sequence માંથી random element \\ \hline
\end{tabulary}
\end{center}

\textbf{કોડ ઉદાહરણ:}
\begin{lstlisting}[language=Python]
import random
print(random.random())        # 0.7234567
print(random.randint(1, 10))  # 7
print(random.choice(['r','g','b'])) # g
\end{lstlisting}
\end{solutionbox}

\begin{mnemonicbox}
\mnemonic{Random Float, Randint Integer, Choice Select}
\end{mnemonicbox}

\questionmarks{3(બ)}{4}{વપરાશકર્તા પાસેથી એક સ્ટ્રિંગ લઈને એમાંના દરેક 'a' નું સ્થાન પ્રિન્ટ કરવાનો પાયથન પ્રોગ્રામ લખો.}

\begin{solutionbox}
\textbf{કોડ:}
\begin{lstlisting}[language=Python]
# Get string from user
text = input("Enter a string: ")

# Find all positions of 'a'
positions = []
for i in range(len(text)):
    if text[i].lower() == 'a':
        positions.append(i)

# Display results
if positions:
    print(f"Letter 'a' found at positions: {positions}")
else:
    print("Letter 'a' not found in the string")
\end{lstlisting}

\textbf{મુખ્ય મુદ્દાઓ:}
\begin{itemize}
    \item \keyword{Case-insensitive}: 'a' અને 'A' બંને શોધવા માટે \code{.lower()} ઉપયોગ કરો
    \item \keyword{Index tracking}: range અથવા enumerate ઉપયોગ કરો
    \item \keyword{આઉટપુટ ફોર્મેટ}: સ્પષ્ટ position indication
\end{itemize}
\end{solutionbox}

\begin{mnemonicbox}
\mnemonic{Loop Index Check Append Print}
\end{mnemonicbox}

\questionmarks{3(ક)}{7}{પાયથનનો string ડેટા ટાઈપ વિસ્તારથી સમજાવો.}

\begin{solutionbox}
\textbf{સ્ટ્રિંગની લાક્ષણિકતાઓ:}
\begin{itemize}
    \item \keyword{અપરિવર્તનીય}: બનાવ્યા પછી બદલી શકાતું નથી
    \item \keyword{અનુક્રમ}: Characters નો ordered collection
    \item \keyword{ઇન્ડેક્સવાળી}: Index વડે characters ને access કરી શકાય છે
    \item \keyword{યુનિકોડ}: બધી ભાષાઓ અને symbols સપોર્ટ કરે છે
\end{itemize}

\textbf{સ્ટ્રિંગ મેથડ્સ ટેબલ:}
\begin{center}
\captionof{table}{String Methods}
\begin{tabulary}{\linewidth}{|L|L|L|}
\hline
\textbf{મેથડ} & \textbf{ઉદાહરણ} & \textbf{વર્ણન} \\ \hline
\textbf{upper()} & \code{"s".upper()} & Uppercase માં convert કરો \\ \hline
\textbf{lower()} & \code{"S".lower()} & Lowercase માં convert કરો \\ \hline
\textbf{strip()} & \code{" s ".strip()} & Whitespace દૂર કરો \\ \hline
\textbf{split()} & \code{"a,b".split(",")} & List માં split કરો \\ \hline
\textbf{replace()} & \code{"s".replace("s","x")} & Substring replace કરો \\ \hline
\textbf{find()} & \code{"s".find("e")} & Substring index શોધો \\ \hline
\end{tabulary}
\end{center}

\textbf{String Structure Diagram:}
\begin{center}
\begin{tikzpicture}[gtu flow]
    \node [gtu block] (str) {String: "Hello"};
    \node [gtu block, right=1cm of str] (mem) {Address: 0x100};
    \node [gtu process, below=1cm of str] (op) {s[0] = 'h'};
    \node [gtu block, below=1cm of mem] (err) {Error: Immutable};
    
    \draw [gtu arrow] (str) -- (mem);
    \draw [gtu arrow] (str) -- (op);
    \draw [gtu arrow] (op) -- (err);
\end{tikzpicture}
\captionof{figure}{String Immutability}
\end{center}

\textbf{કોડ ઉદાહરણ:}
\begin{lstlisting}[language=Python]
name = "Python Programming"
print(name[0])      # P
print(name[0:6])    # Python
message = f"I love {name}"
\end{lstlisting}
\end{solutionbox}

\begin{mnemonicbox}
\mnemonic{Immutable Sequence Indexed Unicode}
\end{mnemonicbox}

\questionmarks{3(અ OR)}{3}{math મોડ્યૂલની કોઈ પણ 3 મેથડ સમજાવો.}

\begin{solutionbox}
\begin{center}
\captionof{table}{Math Module Methods}
\begin{tabulary}{\linewidth}{|L|L|L|}
\hline
\textbf{મેથડ} & \textbf{Syntax} & \textbf{વર્ણન} \\ \hline
\textbf{sqrt()} & \code{math.sqrt(16)} & Square root ગણતરી \\ \hline
\textbf{pow()} & \code{math.pow(2,3)} & Power ગણતરી \\ \hline
\textbf{ceil()} & \code{math.ceil(4.3)} & Integer માં round up \\ \hline
\end{tabulary}
\end{center}

\textbf{કોડ ઉદાહરણ:}
\begin{lstlisting}[language=Python]
import math
print(math.sqrt(25))    # 5.0
print(math.pow(2, 3))   # 8.0
print(math.ceil(4.2))   # 5
\end{lstlisting}
\end{solutionbox}

\begin{mnemonicbox}
\mnemonic{Square Root, Power Up, Ceiling Round}
\end{mnemonicbox}

\questionmarks{3(બ OR)}{4}{વપરાશકર્તા પાસેથી એક સ્ટ્રિંગ લઈને એમાં રહેલા અંગ્રેજી સ્વરોની સંખ્યા શોધવાનો પાયથન પ્રોગ્રામ લખો.}

\begin{solutionbox}
\textbf{કોડ:}
\begin{lstlisting}[language=Python]
# Get string from user
text = input("Enter a string: ")

# Define vowels
vowels = "aeiouAEIOU"

# Count vowels
vowel_count = 0
for char in text:
    if char in vowels:
        vowel_count += 1

# Display result
print(f"Total vowels in '{text}': {vowel_count}")
\end{lstlisting}
\end{solutionbox}

\begin{mnemonicbox}
\mnemonic{Define Vowels, Loop Check, Count Increment}
\end{mnemonicbox}

\questionmarks{3(ક OR)}{7}{પાયથનનો set ડેટા ટાઈપ વિસ્તારથી સમજાવો.}

\begin{solutionbox}
\textbf{સેટની લાક્ષણિકતાઓ:}
\begin{itemize}
    \item \keyword{અક્રમ}: Elements નો કોઈ નિશ્ચિત sequence નથી
    \item \keyword{બદલાવ પાત્ર}: Elements ઉમેરી/દૂર કરી શકાય છે
    \item \keyword{અનન્ય}: Duplicate elements allowed નથી
    \item \keyword{પુનરાવર્તનીય}: Elements માં loop કરી શકાય છે
\end{itemize}

\textbf{સેટ ઑપરેશન્સ ટેબલ:}
\begin{center}
\captionof{table}{Set Operations}
\begin{tabulary}{\linewidth}{|L|L|L|}
\hline
\textbf{ઑપરેશન} & \textbf{Syntax} & \textbf{વર્ણન} \\ \hline
\textbf{બનાવવું} & \code{set = \{1,2,3\}} & નવો set બનાવો \\ \hline
\textbf{Add} & \code{set.add(4)} & Single element ઉમેરો \\ \hline
\textbf{Remove} & \code{set.remove(2)} & Element દૂર કરો \\ \hline
\textbf{સંયોજન} & \code{s1 | s2} & Sets જોડો \\ \hline
\textbf{છેદ} & \code{s1 \& s2} & સામાન્ય elements \\ \hline
\textbf{તફાવત} & \code{s1 - s2} & ફક્ત set1 માંના elements \\ \hline
\end{tabulary}
\end{center}

\textbf{Set Venn Diagram:}
\begin{center}
\begin{tikzpicture}[gtu flow]
    \node [gtu state, minimum size=1.5cm] (A) {A};
    \node [gtu state, minimum size=1.5cm, right=1cm of A] (B) {B};
    
    \node [gtu block, below=1cm of A, xshift=1cm] (ops) {Operators};
    \draw [gtu arrow] (A) -- node[below] {\&} (B);
    \draw [gtu arrow] (A) |- node[left] {|} (ops);
    \draw [gtu arrow] (B) |- node[right] {-} (ops);
\end{tikzpicture}
\captionof{figure}{Set Logic}
\end{center}

\textbf{કોડ ઉદાહરણ:}
\begin{lstlisting}[language=Python]
A = {1, 2, 3, 4}
B = {3, 4, 5, 6}
print(A | B)    # Union: {1,2,3,4,5,6}
print(A & B)    # Intersection: {3,4}
\end{lstlisting}
\end{solutionbox}

\begin{mnemonicbox}
\mnemonic{Unordered Mutable Unique Iterable}
\end{mnemonicbox}

\questionmarks{4(અ)}{3}{પાયથનમાં ક્લાસ શું છે? તે ઓબ્જેક્ટથી કઈ રીતે અલગ છે?}

\begin{solutionbox}
\textbf{ક્લાસ વિ. ઓબ્જેક્ટ સરખામણી:}
\begin{center}
\captionof{table}{Class vs Object}
\begin{tabulary}{\linewidth}{|L|L|L|}
\hline
\textbf{પાસું} & \textbf{ક્લાસ} & \textbf{ઓબ્જેક્ટ} \\ \hline
\textbf{વ્યાખ્યા} & Blueprint અથવા template & ક્લાસનું instance \\ \hline
\textbf{મેમરી} & કોઈ memory allocate થતી નથી & Memory allocate થાય છે \\ \hline
\textbf{અસ્તિત્વ} & Logical entity & Physical entity \\ \hline
\textbf{બનાવટ} & class keyword ઉપયોગ કરીને & Class constructor ઉપયોગ કરીને \\ \hline
\end{tabulary}
\end{center}

\textbf{Relationship Diagram:}
\begin{center}
\begin{tikzpicture}[gtu flow]
    \node [gtu class] (cls) {Class Car\\Blueprint};
    \node [gtu block, below left=1cm of cls] (o1) {Object 1\\Toyota};
    \node [gtu block, below right=1cm of cls] (o2) {Object 2\\Honda};
    
    \draw [gtu arrow] (cls) -- (o1);
    \draw [gtu arrow] (cls) -- (o2);
\end{tikzpicture}
\captionof{figure}{Class to Objects}
\end{center}
\end{solutionbox}

\begin{mnemonicbox}
\mnemonic{Class Blueprint, Object Instance}
\end{mnemonicbox}

\questionmarks{4(બ)}{4}{dictionary ડેટા ટાઈપની કોઈ પણ 4 મેથડ સમજાવો.}

\begin{solutionbox}
\textbf{Dictionary મેથડ્સ ટેબલ:}
\begin{center}
\captionof{table}{Dictionary Methods}
\begin{tabulary}{\linewidth}{|L|L|L|}
\hline
\textbf{મેથડ} & \textbf{Syntax} & \textbf{વર્ણન} \\ \hline
\textbf{keys()} & \code{d.keys()} & બધી keys મેળવો \\ \hline
\textbf{values()} & \code{d.values()} & બધી values મેળવો \\ \hline
\textbf{items()} & \code{d.items()} & Key-value pairs મેળવો \\ \hline
\textbf{get()} & \code{d.get('k')} & Value સુરક્ષિત રીતે મેળવો \\ \hline
\end{tabulary}
\end{center}

\textbf{કોડ ઉદાહરણ:}
\begin{lstlisting}[language=Python]
student = {'name': 'John', 'grade': 'A'}
print(student.keys())    # ['name', 'grade']
print(student.values())  # ['John', 'A']
print(student.get('age')) # None (no error)
\end{lstlisting}
\end{solutionbox}

\begin{mnemonicbox}
\mnemonic{Keys Values Items Get}
\end{mnemonicbox}

\questionmarks{4(ક)}{7}{કોઈ કાર્યો કરવા માટે યુઝર ડિફાઈન્ડ મોડ્યૂલ બનાવી તેને ઈમ્પોર્ટ કરી તેના ફંક્શનનો ઉપયોગ કરવાનો પાયથન પ્રોગ્રામ લખો.}

\begin{solutionbox}
\textbf{Module Structure:}
\begin{center}
\begin{tikzpicture}[gtu flow]
    \node [gtu block] (mod) {math\_operations.py\\ \code{add(), multiply()}};
    \node [gtu block, right=2cm of mod] (main) {main.py\\ \code{import math\_operations}};
    
    \draw [gtu arrow] (main) -- node[above] {Uses} (mod);
\end{tikzpicture}
\captionof{figure}{Module Import}
\end{center}

\textbf{મોડ્યૂલ બનાવવું (math\_operations.py):}
\begin{lstlisting}[language=Python]
def add(a, b):
    return a + b

def multiply(a, b):
    return a * b

PI = 3.14159
\end{lstlisting}

\textbf{મુખ્ય પ્રોગ્રામ (main.py):}
\begin{lstlisting}[language=Python]
import math_operations as mo

res1 = mo.add(5, 3)
res2 = mo.multiply(4, 6)

print(f"Addition: {res1}")
print(f"Multiplication: {res2}")
print(f"PI: {mo.PI}")
\end{lstlisting}

\textbf{મુખ્ય મુદ્દાઓ:}
\begin{itemize}
    \item \keyword{મોડ્યૂલ બનાવવું}: Functions સાથે અલગ .py ફાઈલ
    \item \keyword{Import પદ્ધતિઓ}: \code{import module} અથવા \code{from module import func}
    \item \keyword{ઉપયોગ}: \code{module.function()}
\end{itemize}
\end{solutionbox}

\begin{mnemonicbox}
\mnemonic{Create Import Use}
\end{mnemonicbox}

\questionmarks{4(અ OR)}{3}{પાયથન ક્લાસની મેથડ્સના પ્રકારો ટૂંકમાં સમજાવો.}

\begin{solutionbox}
\begin{center}
\captionof{table}{Method Types}
\begin{tabulary}{\linewidth}{|L|L|L|}
\hline
\textbf{પ્રકાર} & \textbf{Decorator} & \textbf{First Argument} \\ \hline
\textbf{Instance} & None & \code{self} \\ \hline
\textbf{Class} & \code{@classmethod} & \code{cls} \\ \hline
\textbf{Static} & \code{@staticmethod} & None \\ \hline
\end{tabulary}
\end{center}

\textbf{ઉદાહરણ:}
\begin{lstlisting}[language=Python]
class Demo:
    def inst(self): pass
    @classmethod
    def cls_m(cls): pass
    @staticmethod
    def stat(): pass
\end{lstlisting}
\end{solutionbox}

\begin{mnemonicbox}
\mnemonic{Instance Self, Class Cls, Static None}
\end{mnemonicbox}

\questionmarks{4(બ OR)}{4}{string ડેટા ટાઈપની કોઈ પણ 4 મેથડ સમજાવો.}

\begin{solutionbox}
\textbf{String Methods: Checks & Counts}
\begin{center}
\captionof{table}{String Check Methods}
\begin{tabulary}{\linewidth}{|L|L|L|}
\hline
\textbf{મેથડ} & \textbf{Syntax} & \textbf{વર્ણન} \\ \hline
\textbf{startswith} & \code{s.startswith('a')} & Substring થી શરૂ થાય છે કે ચેક કરો \\ \hline
\textbf{endswith} & \code{s.endswith('z')} & Substring થી અંત થાય છે કે ચેક કરો \\ \hline
\textbf{isdigit} & \code{s.isdigit()} & બધા digits છે કે ચેક કરો \\ \hline
\textbf{count} & \code{s.count('a')} & Substring ની occurrences ગિનો \\ \hline
\end{tabulary}
\end{center}

\textbf{કોડ ઉદાહરણ:}
\begin{lstlisting}[language=Python]
s = "Hello World 123"
print(s.startswith("He")) # True
print(s.endswith("23"))   # True
print("123".isdigit())    # True
print(s.count("l"))       # 3
\end{lstlisting}
\end{solutionbox}

\begin{mnemonicbox}
\mnemonic{Start End Digit Count}
\end{mnemonicbox}

\questionmarks{4(ક OR)}{7}{રિકર્સીવ ફંક્શનની મદદથી આપેલ નંબરનો ફેક્ટોરીયલ શોધવા માટેનો પાયથન પ્રોગ્રામ લખો.}

\begin{solutionbox}
\textbf{કોડ:}
\begin{lstlisting}[language=Python]
def factorial(n):
    # Base case
    if n == 0 or n == 1:
        return 1
    # Recursive case
    else:
        return n * factorial(n - 1)

try:
    num = int(input("Enter a number: "))
    if num < 0:
        print("Negative number not allowed")
    else:
        print(f"Factorial of {num} is {factorial(num)}")
except ValueError:
    print("Invalid input")
\end{lstlisting}

\textbf{Recursion Stack Visualization:}
\begin{center}
\begin{tikzpicture}[gtu flow]
    \node [gtu process] (call5) {fact(5)};
    \node [gtu process, below=0.5cm of call5] (call4) {5 * fact(4)};
    \node [gtu process, below=0.5cm of call4] (call3) {4 * fact(3)};
    \node [gtu process, below=0.5cm of call3] (call2) {3 * fact(2)};
    \node [gtu process, below=0.5cm of call2] (call1) {2 * fact(1)};
    \node [gtu output, right=1cm of call1] (ret1) {return 1};
    
    \draw [gtu arrow] (call5) -- (call4);
    \draw [gtu arrow] (call4) -- (call3);
    \draw [gtu arrow] (call3) -- (call2);
    \draw [gtu arrow] (call2) -- (call1);
    \draw [gtu dashed arrow] (call1) -- (ret1);
    \draw [gtu dashed arrow] (ret1) |- (call5); 
\end{tikzpicture}
\captionof{figure}{Recursive Calls}
\end{center}
\end{solutionbox}

\begin{mnemonicbox}
\mnemonic{Base Stop, Recursive Call, Error Check}
\end{mnemonicbox}

\questionmarks{5(અ)}{3}{સિંગલ ઇન્હેરિટન્સ બતાવવા માટેનો પાયથન પ્રોગ્રામ લખો.}

\begin{solutionbox}
\textbf{કોડ:}
\begin{lstlisting}[language=Python]
class Animal:
    def speak(self): print("Animal Speak")

class Dog(Animal):
    def bark(self): print("Dog Bark")

d = Dog()
d.speak()  # Inherited
d.bark()   # Own
\end{lstlisting}

\textbf{Inheritance Diagram:}
\begin{center}
\begin{tikzpicture}[gtu flow]
    \node [gtu class] (parent) {Animal\\speak()};
    \node [gtu class, below=1cm of parent] (child) {Dog\\bark()};
    \draw [gtu arrow] (parent) -- (child);
\end{tikzpicture}
\captionof{figure}{Single Inheritance}
\end{center}
\end{solutionbox}

\begin{mnemonicbox}
\mnemonic{Parent Child Inherit Override}
\end{mnemonicbox}

\questionmarks{5(બ)}{4}{પાયથન ક્લાસમાં કન્સ્ટ્રક્ટરનું મહત્વ સમજાવો.}

\begin{solutionbox}
\textbf{કન્સ્ટ્રક્ટરનું મહત્વ:}
\begin{center}
\captionof{table}{Constructor Features}
\begin{tabulary}{\linewidth}{|L|L|}
\hline
\textbf{પાસું} & \textbf{વર્ણન} \\ \hline
\textbf{ઇનિશિયલાઇઝેશન} & ઓબ્જેક્ટ બનાવવામાં આવે ત્યારે આપોઆપ call થાય છે \\ \hline
\textbf{સેટઅપ} & Instance variables ને values સાથે initialize કરે છે \\ \hline
\textbf{મેમરી} & Object attributes માટે memory allocate કરે છે \\ \hline
\textbf{વેલિડેશન} & Creation દરમિયાન input parameters validate કરે છે \\ \hline
\end{tabulary}
\end{center}

\textbf{Lifecycle Flow:}
\begin{center}
\begin{tikzpicture}[gtu flow]
    \node [gtu start] (create) {New Object};
    \node [gtu process, right=1cm of create] (init) {\_\_init\_\_()};
    \node [gtu stop, right=1cm of init] (ready) {Ready};
    \draw [gtu arrow] (create) -- (init);
    \draw [gtu arrow] (init) -- (ready);
\end{tikzpicture}
\captionof{figure}{Constructor Flow}
\end{center}
\end{solutionbox}

\begin{mnemonicbox}
\mnemonic{Initialize Setup Memory Validate}
\end{mnemonicbox}

\questionmarks{5(ક)}{7}{ઇન્હેરિટન્સ દ્વારા થતું મેથડ ઓવરરાઇડિંગ બતાવવા માટેનો પાયથન પ્રોગ્રામ લખો.}

\begin{solutionbox}
\textbf{કોડ:}
\begin{lstlisting}[language=Python]
class Shape:
    def area(self): print("Shape Area")

class Circle(Shape):
    def area(self): print("Circle Area")

class Rect(Shape):
    def area(self): print("Rect Area")

shapes = [Circle(), Rect()]
for s in shapes:
    s.area()
\end{lstlisting}

\textbf{Method Overriding Hierarchy:}
\begin{center}
\begin{tikzpicture}[gtu flow]
    \node [gtu class] (base) {Shape\\area()};
    \node [gtu class, below left=1cm of base] (c1) {Circle\\Override area()};
    \node [gtu class, below right=1cm of base] (c2) {Rect\\Override area()};
    
    \draw [gtu arrow] (base) -- (c1);
    \draw [gtu arrow] (base) -- (c2);
\end{tikzpicture}
\captionof{figure}{Polymorphism}
\end{center}

\textbf{મુખ્ય મુદ્દાઓ:}
\begin{itemize}
    \item \keyword{સમાન મેથડ નામ}: Parent અને child classes માં
    \item \keyword{અલગ implementation}: Child class specific logic આપે છે
    \item \keyword{Runtime નિર્ણય}: Object type આધારે યોગ્ય method call થાય છે
\end{itemize}
\end{solutionbox}

\begin{mnemonicbox}
\mnemonic{Same Name Different Logic Runtime Decision}
\end{mnemonicbox}

\questionmarks{5(અ OR)}{3}{પાયથનમાં ડેટા એન્કેપ્સ્યુલેશનનો ખ્યાલ સમજાવો.}

\begin{solutionbox}
\textbf{ડેટા એન્કેપ્સ્યુલેશન:}
\begin{center}
\captionof{table}{Encapsulation}
\begin{tabulary}{\linewidth}{|L|L|}
\hline
\textbf{પાસું} & \textbf{વર્ણન} \\ \hline
\textbf{વ્યાખ્યા} & Data અને methods ને એકસાથે બાંધવું \\ \hline
\textbf{ડેટા છુપાવવું} & Private attributes (\_\_var) \\ \hline
\textbf{એક્સેસ} & Public methods (getters/setters) \\ \hline
\end{tabulary}
\end{center}

\textbf{Encapsulation Diagram:}
\begin{center}
\begin{tikzpicture}[gtu flow]
    \node [gtu block] (data) {Private Data\\ \_\_balance};
    \node [gtu class, fit=(data), label=above:Class Bank] (cls) {};
    \node [gtu process, right=2cm of cls] (user) {User};
    
    \draw [gtu arrow] (user) -- node[above] {deposit()} (cls);
    \draw [gtu arrow] (cls) -- node[below] {get\_balance()} (user);
    \draw [gtu arrow, dashed] (user) -- node[below, red] {Direct X} (data);
\end{tikzpicture}
\captionof{figure}{Secure Access}
\end{center}
\end{solutionbox}

\begin{mnemonicbox}
\mnemonic{Bundle Data Hide Interface}
\end{mnemonicbox}

\questionmarks{5(બ OR)}{4}{પાયથનમાં એબ્સ્ટ્રેક્ટ ક્લાસનો ખ્યાલ સમજાવો.}

\begin{solutionbox}
\textbf{એબ્સ્ટ્રેક્ટ ક્લાસ:}
\begin{itemize}
    \item \keyword{વ્યાખ્યા}: સીધા instantiate ન થઈ શકતો ક્લાસ
    \item \keyword{એબ્સ્ટ્રેક્ટ મેથડ્સ}: Declared પણ implemented નથી
    \item \keyword{અમલીકરણ}: Subclasses એ abstract methods implement કરવા જોઈએ
    \item \keyword{મોડ્યૂલ}: \code{abc} મોડ્યૂલ ઉપયોગ કરે છે
\end{itemize}

\textbf{Abstraction Logic:}
\begin{center}
\begin{tikzpicture}[gtu flow]
    \node [gtu class, dashed] (abs) {Abstract Class\\ \textit{sound()}};
    \node [gtu class, below=1cm of abs] (impl) {Dog\\sound(): "Woof"};
    \draw [gtu arrow] (abs) -- node[right] {Implements} (impl);
\end{tikzpicture}
\captionof{figure}{Abstract Base Class}
\end{center}
\end{solutionbox}

\begin{mnemonicbox}
\mnemonic{Cannot Instantiate Force Implementation Common Interface}
\end{mnemonicbox}

\questionmarks{5(ક OR)}{7}{મલ્ટિપલ ઇન્હેરિટન્સ બતાવવા માટેનો પાયથન પ્રોગ્રામ લખો.}

\begin{solutionbox}
\textbf{કોડ:}
\begin{lstlisting}[language=Python]
class Father:
    def work(self): print("Father Engineer")

class Mother:
    def work(self): print("Mother Doctor")

class Child(Father, Mother):
    pass

c = Child()
c.work() # Father (MRO)
print(Child.__mro__)
\end{lstlisting}

\textbf{Multiple Inheritance Structure:}
\begin{center}
\begin{tikzpicture}[gtu flow]
    \node [gtu class] (f) {Father};
    \node [gtu class, right=1.5cm of f] (m) {Mother};
    \node [gtu class, below=1cm of f, xshift=1cm] (c) {Child};
    
    \draw [gtu arrow] (f) -- (c);
    \draw [gtu arrow] (m) -- (c);
\end{tikzpicture}
\captionof{figure}{Multiple Parents}
\end{center}

\textbf{મુખ્ય મુદ્દાઓ:}
\begin{itemize}
    \item \keyword{Structure}: Child બંને Father અને Mother થી inherit કરે છે
    \item \keyword{MRO}: મેથડ રિઝોલ્યુશન ઓર્ડર નક્કી કરે છે
    \item \keyword{Diamond Problem}: C3 linearization વડે solve થાય છે
\end{itemize}
\end{solutionbox}

\begin{mnemonicbox}
\mnemonic{Multiple Parents MRO Constructor Diamond}
\end{mnemonicbox}

\end{document}
