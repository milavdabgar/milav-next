\documentclass{article}
% Adjust the relative path to point to the latex-templates directory

% content/resources/templates/preamble.tex
\usepackage[margin=0.6in]{geometry}
\author{Milav Dabgar}
\usepackage{amsmath,amssymb,amsthm}
\usepackage{booktabs}
\usepackage{multirow}
\usepackage{xcolor}
\usepackage{tcolorbox}
\tcbuselibrary{breakable,skins}
\usepackage[colorlinks=true,linkcolor=blue]{hyperref}
\usepackage{titlesec}
\usepackage{enumitem}
\usepackage{tikz}
\usepackage{pgfplots}
\usepackage{circuitikz}
\usepackage[version=4]{mhchem}
\usepackage{longtable}
\usepackage{array}
\usepackage{float}
\usepackage{caption}
\usepackage{listings}

\lstset{
  basicstyle=\small\ttfamily,
  breaklines=true,
  breakatwhitespace=false,
  postbreak=\mbox{\textcolor{red}{$\hookrightarrow$}\space},
  float=false,
  numbers=left,
  numberstyle=\tiny\color{gray},
  numbersep=10pt,
  xleftmargin=2em,
  keywordstyle=\color{blue},
  commentstyle=\color{green!60!black},
  stringstyle=\color{purple},
  backgroundcolor=\color{gray!5},
  showstringspaces=false,
  tabsize=2,
  captionpos=b,
  keepspaces=true,
  columns=flexible
}

\pgfplotsset{compat=1.18}
\usetikzlibrary{shapes,arrows,positioning,calc,patterns,decorations.pathmorphing,decorations.markings,arrows.meta}

% Color scheme
\definecolor{headcolor}{RGB}{0,102,204}
\definecolor{keycolor}{RGB}{220,20,60}
\definecolor{solutioncolor}{RGB}{34,139,34}
\definecolor{mnemoniccolor}{RGB}{148,0,211}
\definecolor{codecolor}{RGB}{0,0,100}

% Spacing
\setlength{\parskip}{3pt}
\setlist[itemize]{nosep}
\setlist[enumerate]{nosep}

% Title formatting
\titleformat{\section}{\Large\bfseries\color{headcolor}}{\thesection}{1em}{}
\titleformat{\subsection}{\large\bfseries\color{headcolor}}{\thesubsection}{1em}{}

% Pandoc tightlist compatibility
\providecommand{\tightlist}{%
  \setlength{\itemsep}{0pt}\setlength{\parskip}{0pt}}

% Pandoc longtable compatibility
\newcounter{none}
\def\thenone{}


% content/resources/templates/gujarati-boxes.tex
\usepackage{fontspec}
\usepackage{polyglossia}

% Set Gujarati as main language (document is primarily in Gujarati)
% Note: gloss-gujarati.ldf doesn't exist in polyglossia, but it will use hyphenation patterns
\setdefaultlanguage{gujarati}
\setotherlanguage{english}

% Configure Gujarati font properly
% Use Language=Default to prevent polyglossia from trying to add language-specific features
% that don't exist for Gujarati, which causes "empty feature" warnings
\newfontfamily\gujaratifont[Script=Gujarati,AutoFakeBold=2.5,AutoFakeSlant=0.3]{Noto Sans Gujarati}
\setmainfont[Script=Gujarati,AutoFakeBold=2.5,AutoFakeSlant=0.3]{Noto Sans Gujarati}
% Use Noto Sans Gujarati for monospace to support Gujarati in text
\setmonofont[Scale=0.9]{Noto Sans Gujarati}

% Configure English to use the same font
\newfontfamily\englishfont[Script=Gujarati,AutoFakeBold=2.5,AutoFakeSlant=0.3]{Noto Sans Gujarati}

% Translations for polyglossia
\gappto\captionsgujarati{
  \renewcommand{\tablename}{કોષ્ટક}
  \renewcommand{\figurename}{આકૃતિ}
}

% Helper for TikZ nodes to ensure Gujarati font
\newcommand{\gu}[1]{{\gujaratifont #1}}

% Custom environments
\newtcolorbox{solutionbox}{
    breakable,
    enhanced,
    colback=solutioncolor!5!white,
    colframe=solutioncolor!75!black,
    fonttitle=\bfseries,
    title=જવાબ
}

\newtcolorbox{solutionboxnobreak}{
 colback=solutioncolor!5!white,
 colframe=solutioncolor!75!black,
 fonttitle=\bfseries,
 title=જવાબ
}

\newtcolorbox{keyformula}{
 breakable,
 enhanced,
 colback=keycolor!5!white,
 colframe=keycolor!75!black,
 fonttitle=\bfseries,
 title=રાસાયણિક સમીકરણ/સૂત્ર
}

\newtcolorbox{mnemonicbox}{
 breakable,
 enhanced,
 colback=mnemoniccolor!5!white,
 colframe=mnemoniccolor!75!black,
 fonttitle=\bfseries,
 title=મેમરી ટ્રીક
}


% Custom commands for GTU solutions
% This file defines semantic commands for consistent formatting

% Question command with automatic formatting
\newcommand{\question}[2]{%
  \section*{Question #1}%
  \textbf{#2}%
}

% OR question variant
\newcommand{\questionor}[2]{%
  \section*{Question #1 OR}%
  \textbf{#2}%
}

% Proper table environment with caption
\newenvironment{answertable}[1]{%
  \begin{table}[htbp]
  \centering
  \caption{#1}
}{%
  \end{table}
}

% Proper figure environment for diagrams
\newenvironment{answerdiagram}[1]{%
  \begin{figure}[htbp]
  \centering
  \caption{#1}
}{%
  \end{figure}
}

% Semantic markup for key terms
\newcommand{\keyword}[1]{\textbf{#1}}
\newcommand{\code}[1]{\texttt{#1}}
\newcommand{\classname}[1]{\texttt{#1}}
\newcommand{\methodname}[1]{\texttt{#1}}

% Proper quotation marks
\newcommand{\mnemonic}[1]{``#1''}


\title{OOPS \& Python Programming (4351108) - Winter 2024 Solution}
\date{November 25, 2024}

\begin{document}
\maketitle

\questionmarks{1(અ)}{3}{પાયથોન પ્રોગ્રામિંગ ભાષાના લક્ષણોની યાદી બનાવો.}

\begin{solutionbox}
\begin{center}
\captionof{table}{પાયથોનના લક્ષણો}
\begin{tabulary}{\linewidth}{|L|L|}
\hline
\textbf{લક્ષણ} & \textbf{વર્ણન} \\ \hline
\keyword{સરળ અને સહેલું} & સ્વચ્છ, વાંચી શકાય તેવું syntax \\ \hline
\keyword{મફત અને ઓપન સોર્સ} & કોઈ કિંમત નહીં, community driven \\ \hline
\keyword{ક્રોસ-પ્લેટફોર્મ} & Windows, Linux, Mac પર ચાલે છે \\ \hline
\keyword{Interpreted} & compilation ની જરૂર નથી \\ \hline
\keyword{Object-Oriented} & classes અને objects ને support કરે છે \\ \hline
\keyword{મોટી લાઇબ્રેરીઓ} & સમૃદ્ધ standard library \\ \hline
\end{tabulary}
\end{center}
\end{solutionbox}

\begin{mnemonicbox}
\mnemonic{Simple Free Cross Interpreted Object Large}
\end{mnemonicbox}

\questionmarks{1(બ)}{4}{પાયથોન પ્રોગ્રામિંગ ભાષાની એપ્લિકેશનો લખો.}

\begin{solutionbox}
\begin{center}
\captionof{table}{પાયથોન એપ્લિકેશનો}
\begin{tabulary}{\linewidth}{|L|L|}
\hline
\textbf{એપ્લિકેશન ક્ષેત્ર} & \textbf{ઉદાહરણો} \\ \hline
\keyword{વેબ ડેવલપમેન્ટ} & Django, Flask frameworks \\ \hline
\keyword{ડેટા સાયન્સ} & NumPy, Pandas, Matplotlib \\ \hline
\keyword{મશીન લર્નિંગ} & TensorFlow, Scikit-learn \\ \hline
\keyword{ડેસ્કટોપ GUI} & Tkinter, PyQt applications \\ \hline
\keyword{ગેમ ડેવલપમેન્ટ} & Pygame library \\ \hline
\keyword{ઓટોમેશન} & Scripting અને testing \\ \hline
\end{tabulary}
\end{center}
\end{solutionbox}

\begin{mnemonicbox}
\mnemonic{Web Data Machine Desktop Game Auto}
\end{mnemonicbox}

\questionmarks{1(ક)}{7}{પાયથોનમાં વિવિધ ડેટાટાઇપ્સ સમજાવો.}

\begin{solutionbox}
\textbf{Data Type Hierarchy:}
\begin{center}
\begin{tikzpicture}[gtu flow]
    \node [gtu block] (root) {Data Types};
    \node [gtu block, below left=1cm of root, xshift=-2cm] (num) {Numeric\\ \code{int, float, complex}};
    \node [gtu block, below=1cm of root] (seq) {Sequence\\ \code{str, list, tuple}};
    \node [gtu block, below right=1cm of root, xshift=2cm] (map) {Mapping\\ \code{dict}};
    \node [gtu block, right=0.5cm of map] (bool) {Boolean\\ \code{bool}};
    
    \draw [gtu arrow] (root) -- (num);
    \draw [gtu arrow] (root) -- (seq);
    \draw [gtu arrow] (root) -- (map);
    \draw [gtu arrow] (root) -- (bool);
\end{tikzpicture}
\captionof{figure}{Python Data Types}
\end{center}

\begin{center}
\captionof{table}{Python Data Types}
\begin{tabulary}{\linewidth}{|L|L|L|}
\hline
\textbf{ડેટા ટાઇપ} & \textbf{ઉદાહરણ} & \textbf{વર્ણન} \\ \hline
\textbf{int} & \code{x = 5} & પૂર્ણાંક સંખ્યાઓ \\ \hline
\textbf{float} & \code{y = 3.14} & દશાંશ સંખ્યાઓ \\ \hline
\textbf{str} & \code{name = "John"} & ટેક્સ્ટ ડેટા \\ \hline
\textbf{bool} & \code{flag = True} & True/False મૂલ્યો \\ \hline
\textbf{list} & \code{[1, 2, 3]} & ક્રમબદ્ધ, બદલી શકાય તેવું \\ \hline
\textbf{tuple} & \code{(1, 2, 3)} & ક્રમબદ્ધ, બદલી ન શકાય તેવું \\ \hline
\textbf{dict} & \code{\{"a": 1\}} & Key-value જોડી \\ \hline
\textbf{set} & \code{\{1, 2, 3\}} & અનન્ય ઘટકો \\ \hline
\end{tabulary}
\end{center}

\textbf{કોડ ઉદાહરણ:}
\begin{lstlisting}[language=Python]
# Numeric types
age = 25          # int
price = 99.99     # float

# Text type
name = "Python"   # str

# Boolean type
is_valid = True   # bool

# Collection types
numbers = [1, 2, 3]        # list
coordinates = (10, 20)     # tuple
student = {"name": "John"} # dict
unique_ids = {1, 2, 3}     # set
\end{lstlisting}
\end{solutionbox}

\begin{mnemonicbox}
\mnemonic{Integer Float String Boolean List Tuple Dict Set}
\end{mnemonicbox}

\questionmarks{1(ક OR)}{7}{એરિથમેટિક, એસાઇનમેન્ટ અને આઇડેન્ટિટી ઓપરેટરો ઉદાહરણ સાથે સમજાવો.}

\begin{solutionbox}
\textbf{એરિથમેટિક ઓપરેટરો:}
\begin{center}
\captionof{table}{એરિથમેટિક ઓપરેટરો}
\begin{tabulary}{\linewidth}{|C|C|L|}
\hline
\textbf{Op} & \textbf{Name} & \textbf{ઉદાહરણ} \\ \hline
\code{+} & બાકીદારી & \code{5 + 3 = 8} \\ \hline
\code{-} & બાદબાકી & \code{5 - 3 = 2} \\ \hline
\code{*} & ગુણાકાર & \code{5 * 3 = 15} \\ \hline
\code{/} & ભાગાકાર & \code{10 / 3 = 3.33} \\ \hline
\code{//} & Floor Div & \code{10 // 3 = 3} \\ \hline
\code{\%} & Modulus & \code{10 \% 3 = 1} \\ \hline
\code{**} & ઘાત & \code{2 ** 3 = 8} \\ \hline
\end{tabulary}
\end{center}

\textbf{એસાઇનમેન્ટ ઓપરેટરો:}
\begin{center}
\captionof{table}{એસાઇનમેન્ટ ઓપરેટરો}
\begin{tabulary}{\linewidth}{|C|C|L|}
\hline
\textbf{Op} & \textbf{ઉદાહરણ} & \textbf{સમકક્ષ} \\ \hline
\code{=} & \code{x = 5} & મૂલ્ય આપો \\ \hline
\code{+=} & \code{x += 3} & \code{x = x + 3} \\ \hline
\code{-=} & \code{x -= 2} & \code{x = x - 2} \\ \hline
\code{*=} & \code{x *= 4} & \code{x = x * 4} \\ \hline
\end{tabulary}
\end{center}

\textbf{આઇડેન્ટિટી ઓપરેટરો:}
\begin{center}
\captionof{table}{આઇડેન્ટિટી ઓપરેટરો}
\begin{tabulary}{\linewidth}{|L|L|L|}
\hline
\textbf{Op} & \textbf{હેતુ} & \textbf{ઉદાહરણ} \\ \hline
\code{is} & સમાન ઓબ્જેક્ટ & \code{x is y} \\ \hline
\code{is not} & વિવિધ ઓબ્જેક્ટ & \code{x is not y} \\ \hline
\end{tabulary}
\end{center}

\textbf{કોડ ઉદાહરણ:}
\begin{lstlisting}[language=Python]
# Arithmetic
a = 10 + 5    # 15
b = 10 // 3   # 3

# Assignment
x = 5
x += 3        # x બને છે 8

# Identity
list1 = [1, 2, 3]
list2 = [1, 2, 3]
print(list1 is list2)      # False
print(list1 is not list2)  # True
\end{lstlisting}
\end{solutionbox}

\begin{mnemonicbox}
\mnemonic{Add Assign Identity}
\end{mnemonicbox}

\questionmarks{2(અ)}{3}{નીચેનામાંથી કયા આઇડેન્ટિફાયર્સ ના નામો અમાન્ય છે? (i) Total Marks (ii) Total\_Marks (iii) total-Marks (iv) Hundred\$ (v) \_Percentage (vi) True}

\begin{solutionbox}
\begin{center}
\captionof{table}{Identifier Validity}
\begin{tabulary}{\linewidth}{|L|L|L|}
\hline
\textbf{આઇડેન્ટિફાયર} & \textbf{માન્ય/અમાન્ય} & \textbf{કારણ} \\ \hline
Total Marks & \textbf{અમાન્ય} & સ્પેસ છે \\ \hline
Total\_Marks & માન્ય & અન્ડરસ્કોર મંજૂર છે \\ \hline
total-Marks & \textbf{અમાન્ય} & હાઇફન મંજૂર નથી \\ \hline
Hundred\$ & \textbf{અમાન્ય} & \$ સિમ્બોલ મંજૂર નથી \\ \hline
\_Percentage & માન્ય & અન્ડરસ્કોરથી શરૂ થઈ શકે છે \\ \hline
True & \textbf{અમાન્ય} & આરક્ષિત કીવર્ડ છે \\ \hline
\end{tabulary}
\end{center}

\textbf{અમાન્ય આઇડેન્ટિફાયર્સ:} Total Marks, total-Marks, Hundred\$, True
\end{solutionbox}

\begin{mnemonicbox}
\mnemonic{Space Hyphen Dollar Keyword = Invalid}
\end{mnemonicbox}

\questionmarks{2(બ)}{4}{આપેલ ત્રણ સંખ્યાઓમાંથી મહત્તમ સંખ્યા શોધવા માટે પ્રોગ્રામ લખો.}

\begin{solutionbox}
\textbf{કોડ:}
\begin{lstlisting}[language=Python]
# ત્રણ સંખ્યાઓ input લો
num1 = float(input("પ્રથમ સંખ્યા દાખલ કરો: "))
num2 = float(input("બીજી સંખ્યા દાખલ કરો: "))
num3 = float(input("ત્રીજી સંખ્યા દાખલ કરો: "))

# if-elif-else વાપરીને મહત્તમ શોધો
if num1 >= num2 and num1 >= num3:
    maximum = num1
elif num2 >= num1 and num2 >= num3:
    maximum = num2
else:
    maximum = num3

# પરિણામ દર્શાવો
print(f"મહત્તમ સંખ્યા છે: {maximum}")
\end{lstlisting}

\textbf{Alternative using max() function:}
\begin{lstlisting}[language=Python]
num1, num2, num3 = map(float, input("3 સંખ્યાઓ દાખલ કરો: ").split())
maximum = max(num1, num2, num3)
print(f"મહત્તમ: {maximum}")
\end{lstlisting}
\end{solutionbox}

\begin{mnemonicbox}
\mnemonic{Input Compare Display}
\end{mnemonicbox}

\questionmarks{2(ક)}{7}{પાયથોનમાં ડિક્શનરી સમજાવો. ડિક્શનરીમાં ઘટકો ઉમેરવા, બદલવા અને કાઢી નાખવા માટેના સ્ટેટમેન્ટ લખો.}

\begin{solutionbox}
\textbf{ડિક્શનરી:} એક collection જે ક્રમબદ્ધ, બદલાય તેવો અને ડુપ્લિકેટ keys નથી મંજૂર કરે છે.

\textbf{Dictionary Operations:}
\begin{center}
\begin{tikzpicture}[gtu flow]
    \node [gtu block] (dict) {Dict: \code{\{'a': 1\}}};
    \node [gtu process, right=1.5cm of dict] (add) {Add: \code{d['b']=2}};
    \node [gtu block, right=1.5cm of add] (res1) {\code{\{'a':1, 'b':2\}}};
    \node [gtu process, below=1cm of dict] (mod) {Modify: \code{d['a']=10}};
    \node [gtu block, below=1cm of res1] (res2) {\code{\{'a':10, 'b':2\}}};
    
    \draw [gtu arrow] (dict) -- (add);
    \draw [gtu arrow] (add) -- (res1);
    \draw [gtu arrow] (res1) |- (mod);
    \draw [gtu arrow] (mod) -- (res2);
\end{tikzpicture}
\captionof{figure}{Dictionary Operations}
\end{center}

\begin{center}
\captionof{table}{Dictionary Methods}
\begin{tabulary}{\linewidth}{|L|L|L|}
\hline
\textbf{ઓપરેશન} & \textbf{સિન્ટેક્સ} & \textbf{ઉદાહરણ} \\ \hline
\textbf{બનાવો} & \code{dict = \{\}} & \code{student = \{\}} \\ \hline
\textbf{ઉમેરો} & \code{d[k] = v} & \code{student['name'] = 'John'} \\ \hline
\textbf{બદલો} & \code{d[k] = new} & \code{student['name'] = 'Jane'} \\ \hline
\textbf{ડિલીટ} & \code{del d[k]} & \code{del student['name']} \\ \hline
\textbf{એક્સેસ} & \code{d[k]} & \code{print(student['name'])} \\ \hline
\end{tabulary}
\end{center}

\textbf{કોડ ઉદાહરણ:}
\begin{lstlisting}[language=Python]
# ખાલી ડિક્શનરી બનાવો
student = {}

# ઘટકો ઉમેરો
student['name'] = 'John'
student['age'] = 20

# ઘટક બદલો
student['age'] = 21

# ઘટક ડિલીટ કરો
del student['name']

# ડિક્શનરી દર્શાવો
print(student)  # આઉટપુટ: {'age': 21}
\end{lstlisting}
\end{solutionbox}

\begin{mnemonicbox}
\mnemonic{Key-Value Ordered Changeable Unique}
\end{mnemonicbox}

\questionmarks{2(અ OR)}{3}{નીચેની પેટર્ન દર્શાવવા માટેનો પ્રોગ્રામ લખો.}

\begin{solutionbox}
\textbf{પેટર્ન:}
\begin{verbatim}
1
1 2
1 2 3
1 2 3 4
1 2 3 4 5
\end{verbatim}

\textbf{કોડ:}
\begin{lstlisting}[language=Python]
# પેટર્ન પ્રોગ્રામ
for i in range(1, 6):
    for j in range(1, i + 1):
        print(j, end=" ")
    print()  # દરેક રો પછી નવી લાઇન
\end{lstlisting}
\end{solutionbox}

\begin{mnemonicbox}
\mnemonic{Outer Row Inner Column Print}
\end{mnemonicbox}

\questionmarks{2(બ OR)}{4}{વપરાશકર્તા દ્વારા દાખલ કરેલ પૂર્ણાંક સંખ્યાના અંકોનો સરવાળો શોધવા માટે પ્રોગ્રામ લખો.}

\begin{solutionbox}
\textbf{કોડ:}
\begin{lstlisting}[language=Python]
# વપરાશકર્તા પાસેથી સંખ્યા input લો
number = int(input("સંખ્યા દાખલ કરો: "))
original_number = number
sum_digits = 0

# અંકો કાઢો અને સરવાળો કરો
while number > 0:
    digit = number % 10    # છેલ્લો અંક મેળવો
    sum_digits += digit    # સરવાળામાં ઉમેરો
    number = number // 10  # છેલ્લો અંક દૂર કરો

# પરિણામ દર્શાવો
print(f"{original_number} ના અંકોનો સરવાળો છે: {sum_digits}")
\end{lstlisting}

\textbf{વૈકલ્પિક રીત:}
\begin{lstlisting}[language=Python]
number = input("સંખ્યા દાખલ કરો: ")
sum_digits = sum(int(digit) for digit in number)
print(f"અંકોનો સરવાળો: {sum_digits}")
\end{lstlisting}
\end{solutionbox}

\begin{mnemonicbox}
\mnemonic{Input Extract Sum Display}
\end{mnemonicbox}

\questionmarks{2(ક OR)}{7}{લિસ્ટમાં સ્લાઇસિંગ અને કન્કેટનેશન ઓપરેશન સમજાવો.}

\begin{solutionbox}
\textbf{લિસ્ટ સ્લાઇસિંગ:} \code{[start:stop:step]} સિન્ટેક્સ વાપરીને લિસ્ટનો ભાગ કાઢવો.

\textbf{Slicing Visualization:}
\begin{center}
\begin{tikzpicture}[gtu flow]
    \node [gtu block] (list) {\code{['A', 'B', 'C', 'D', 'E']}};
    \node [above=0.2cm of list] {0 \quad 1 \quad 2 \quad 3 \quad 4};
    \node [below=0.2cm of list] {-5 \quad -4 \quad -3 \quad -2 \quad -1};
    \node [below=1cm of list] (slice) {\code{list[1:4]} $\to$ \code{['B', 'C', 'D']}};
    \node [below=0.5cm of slice] (rev) {\code{list[::-1]} $\to$ \code{['E', 'D', 'C', 'B', 'A']}};
    \draw [gtu arrow] (list) -- (slice);
\end{tikzpicture}
\captionof{figure}{List Indexing \& Slicing}
\end{center}

\textbf{ઓપરેશન્સ ટેબલ:}
\begin{center}
\captionof{table}{લિસ્ટ ઓપરેશન્સ}
\begin{tabulary}{\linewidth}{|L|L|L|}
\hline
\textbf{સિન્ટેક્સ} & \textbf{વર્ણન} & \textbf{ઉદાહરણ} \\ \hline
\code{l[start:stop]} & start થી stop-1 સુધી & \code{nums[1:4]} \\ \hline
\code{l[:stop]} & શરૂઆતથી & \code{nums[:3]} \\ \hline
\code{l[::step]} & step સાથે & \code{nums[::2]} \\ \hline
\code{l1 + l2} & કન્કેટનેશન & \code{[1]+[2]} \\ \hline
\end{tabulary}
\end{center}

\textbf{કોડ ઉદાહરણ:}
\begin{lstlisting}[language=Python]
list1 = [1, 2, 3, 4, 5]
list2 = [6, 7, 8]

# સ્લાઇસિંગ
print(list1[1:4])    # [2, 3, 4]
print(list1[::-1])   # [5, 4, 3, 2, 1]

# કન્કેટનેશન
result = list1 + list2  # [1, 2, 3... 8]
list1.extend(list2)     # list1 ને modify કરે છે
\end{lstlisting}
\end{solutionbox}

\begin{mnemonicbox}
\mnemonic{Slice Extract Concat Join}
\end{mnemonicbox}

\questionmarks{3(અ)}{3}{પાયથોનમાં લિસ્ટ વ્યાખ્યાયિત કરો. લિસ્ટના અંતમાં એલિમેન્ટ ઉમેરવા માટે વપરાતા ફંક્શનનું નામ લખો.}

\begin{solutionbox}
\textbf{લિસ્ટ:} એક ક્રમબદ્ધ, બદલાય તેવો સંગ્રહ.

\textbf{ગુણધર્મો:}
\begin{itemize}
    \item \keyword{ક્રમબદ્ધ}: આઇટમ્સનો નિશ્ચિત ક્રમ છે
    \item \keyword{બદલાય તેવું}: બનાવ્યા પછી બદલી શકાય છે
    \item \keyword{ડુપ્લિકેટ્સ}: ડુપ્લિકેટ મૂલ્યો મંજૂર કરે છે
\end{itemize}

\textbf{ઘટક ઉમેરવા માટેનું ફંક્શન:} \code{append()}

\textbf{ઉદાહરણ:}
\begin{lstlisting}[language=Python]
fruits = ['apple', 'banana']
fruits.append('orange')
print(fruits)  # ['apple', 'banana', 'orange']
\end{lstlisting}
\end{solutionbox}

\begin{mnemonicbox}
\mnemonic{List Append End}
\end{mnemonicbox}

\questionmarks{3(બ)}{4}{પાયથોનમાં ટ્યુપલ વ્યાખ્યાયિત કરો. ટ્યુપલના છેલ્લા એલિમેન્ટને એક્સેસ કરવા માટેનું સ્ટેટમેન્ટ લખો.}

\begin{solutionbox}
\textbf{ટ્યુપલ:} એક ક્રમબદ્ધ, બદલાય તેવો ન હોય તેવો સંગ્રહ.

\textbf{છેલ્લા એલિમેન્ટને એક્સેસ કરવું:}
\begin{lstlisting}[language=Python]
my_tuple = (10, 20, 30, 40, 50)

# મેથડ 1: નેગેટિવ ઇન્ડેક્સ
last = my_tuple[-1]  # 50

# મેથડ 2: Length
last = my_tuple[len(my_tuple) - 1] # 50
\end{lstlisting}
\end{solutionbox}

\begin{mnemonicbox}
\mnemonic{Tuple Unchangeable Negative Index}
\end{mnemonicbox}

\questionmarks{3(ક)}{7}{નીચેના સેટ ઓપરેશન્સ માટે સ્ટેટમેન્ટ લખો: ખાલી સેટ બનાવો, સેટમાં એક ઘટક ઉમેરો, સેટમાંથી એક ઘટક દૂર કરો, બે સેટનું યુનિયન, બે સેટનું છેદ, બે સેટ વચ્ચેનો તફાવત અને બે સેટ વચ્ચે સિમેટ્રિક તફાવત.}

\begin{solutionbox}
\textbf{સેટ ઓપરેશન્સ ટેબલ:}
\begin{center}
\captionof{table}{સેટ ઓપરેશન્સ}
\begin{tabulary}{\linewidth}{|L|L|L|}
\hline
\textbf{ઓપરેશન} & \textbf{મેથડ/Op} & \textbf{ઉદાહરણ} \\ \hline
ખાલી બનાવો & \code{set()} & \code{s = set()} \\ \hline
ઉમેરો & \code{add()} & \code{s.add(5)} \\ \hline
દૂર કરો & \code{remove()} & \code{s.remove(5)} \\ \hline
યુનિયન & \code{union() |} & \code{A | B} \\ \hline
છેદ (Intersect) & \code{intersection() \&} & \code{A \& B} \\ \hline
તફાવત (Diff) & \code{difference() -} & \code{A - B} \\ \hline
સિમેટ્રિક તફાવત & \code{sym\_diff() \^{}} & \code{A \^{} B} \\ \hline
\end{tabulary}
\end{center}

\textbf{Set Relations Diagram:}
\begin{center}
\begin{tikzpicture}[gtu flow]
    \node [gtu state, minimum size=1.5cm, fill=blue!10] (A) {A};
    \node [gtu state, minimum size=1.5cm, fill=red!10, right=1cm of A] (B) {B};
    \node [below=0.5cm of A] (U) {Union: All};
    \node [below=0.5cm of B] (I) {Intersect: Common};
    \draw [gtu arrow] (A) -- node[above] {\&} (B);
\end{tikzpicture}
\captionof{figure}{Set Relations}
\end{center}

\textbf{કોડ ઉદાહરણ:}
\begin{lstlisting}[language=Python]
s = set()
s.add(10)
s.remove(10)

A = {1, 2, 3}
B = {3, 4, 5}
print(A | B)  # {1, 2, 3, 4, 5}
print(A & B)  # {3}
print(A - B)  # {1, 2}
print(A ^ B)  # {1, 2, 4, 5}
\end{lstlisting}
\end{solutionbox}

\begin{mnemonicbox}
\mnemonic{Create Add Remove Union Intersect Differ Symmetric}
\end{mnemonicbox}

\questionmarks{3(અ OR)}{3}{પાયથોનમાં સ્ટ્રિંગ વ્યાખ્યાયિત કરો. ઉદાહરણ વાપરીને સમજાવો (i) સ્ટ્રિંગ કેવી રીતે બનાવવી. (ii) ઇન્ડેક્સિંગનો ઉપયોગ કરીને વ્યક્તિગત અક્ષરોને એક્સેસ કરવું.}

\begin{solutionbox}
\textbf{સ્ટ્રિંગ:} અવતરણચિહ્ન (સિંગલ, ડબલ) માં બંધ કરેલા અક્ષરોનો ક્રમ.

\textbf{String Structure:}
\begin{center}
\begin{tikzpicture}[gtu flow]
    \node [gtu block] (str) {P \quad Y \quad T \quad H \quad O \quad N};
    \node [above=0.2cm of str] {\small 0 \quad 1 \quad 2 \quad 3 \quad 4 \quad 5};
    \node [below=0.2cm of str] {\small -6 \quad -5 \quad -4 \quad -3 \quad -2 \quad -1};
\end{tikzpicture}
\captionof{figure}{String Indexing}
\end{center}

\textbf{કોડ:}
\begin{lstlisting}[language=Python]
# બનાવટ
s1 = 'Hello'
s2 = "World"

# એક્સેસિંગ
word = "PYTHON"
print(word[0])   # P
print(word[-1])  # N
\end{lstlisting}
\end{solutionbox}

\begin{mnemonicbox}
\mnemonic{String Quotes Index Access}
\end{mnemonicbox}

\questionmarks{3(બ OR)}{4}{ફોર લૂપ અને વ્હાઇલ લૂપનો ઉપયોગ કરીને લિસ્ટ ટ્રાવર્સિંગ સમજાવો.}

\begin{solutionbox}
\textbf{List Traversing:} લિસ્ટના દરેક ઘટકને મુલાકાત લેવી.

\textbf{Comparison:}
\begin{center}
\captionof{table}{Loop Comparison}
\begin{tabulary}{\linewidth}{|L|L|}
\hline
\textbf{ફોર લૂપ} & \textbf{વ્હાઇલ લૂપ} \\ \hline
સરળ સિન્ટેક્સ & વધુ નિયંત્રણ \\ \hline
Iterations ખબર હોય ત્યારે & શરત આધારિત માટે \\ \hline
\end{tabulary}
\end{center}

\textbf{કોડ ઉદાહરણ:}
\begin{lstlisting}[language=Python]
nums = [10, 20, 30]

# For Loop
for x in nums:
    print(x)

# While Loop
i = 0
while i < len(nums):
    print(nums[i])
    i += 1
\end{lstlisting}
\end{solutionbox}

\begin{mnemonicbox}
\mnemonic{For Simple While Control}
\end{mnemonicbox}

\questionmarks{3(ક OR)}{7}{એક એવો પ્રોગ્રામ લખો કે જેનાથી વર્ગમાં n વિદ્યાર્થીઓના રોલ નંબર, નામ અને માર્ક્સ સાથેની ડિક્શનરી બનાવી શકાય અને 75 થી વધુ ગુણ મેળવનારા વિદ્યાર્થીઓના નામ ડિસ્પ્લે કરી શકાય.}

\begin{solutionbox}
\textbf{કોડ:}
\begin{lstlisting}[language=Python]
# વિદ્યાર્થીઓની સંખ્યા input કરો
n = int(input("વિદ્યાર્થીઓની સંખ્યા દાખલ કરો: "))
students = {}

# ડેટા input કરો
for i in range(n):
    print(f"\nStudent {i + 1}:")
    roll = int(input("Roll: "))
    name = input("Name: ")
    marks = float(input("Marks: "))
    
    students[roll] = {'name': name, 'marks': marks}

# 75 થી વધુ માર્ક્સ વાળા
print("\nStudents with marks > 75:")
found = False
for roll, data in students.items():
    if data['marks'] > 75:
        print(f"Name: {data['name']}, Marks: {data['marks']}")
        found = True

if not found:
    print("None found")
\end{lstlisting}
\end{solutionbox}

\begin{mnemonicbox}
\mnemonic{Input Store Filter Display}
\end{mnemonicbox}

\questionmarks{4(અ)}{3}{રેન્ડમ મોડ્યુલમાં ઉપલબ્ધ કોઈપણ ત્રણ ફંક્શન લખો. દરેક ફંક્શનનું સિન્ટેક્સ અને ઉદાહરણ લખો.}

\begin{solutionbox}
\begin{center}
\captionof{table}{Random Functions}
\begin{tabulary}{\linewidth}{|L|L|L|}
\hline
\textbf{ફંક્શન} & \textbf{વર્ણન} & \textbf{ઉદાહરણ} \\ \hline
\code{random()} & 0.0 થી 1.0 Float & \code{0.75} \\ \hline
\code{randint(a,b)} & a થી b Integer & \code{5} \\ \hline
\code{choice(seq)} & Random element & \code{'red'} \\ \hline
\end{tabulary}
\end{center}

\textbf{કોડ:}
\begin{lstlisting}[language=Python]
import random
print(random.random())
print(random.randint(1, 10))
print(random.choice(['a', 'b', 'c']))
\end{lstlisting}
\end{solutionbox}

\begin{mnemonicbox}
\mnemonic{Random Randint Choice}
\end{mnemonicbox}

\questionmarks{4(બ)}{4}{ફંક્શનના ફાયદા લખો.}

\begin{solutionbox}
\textbf{ફાયદા:}
\begin{itemize}
    \item \keyword{Code Reusability}: કોડ પુનઃઉપયોગ.
    \item \keyword{Modularity}: પ્રોગ્રામને નાના ભાગોમાં વહેંચવો.
    \item \keyword{Debugging}: એરર્સ શોધવી સરળ છે.
    \item \keyword{Readability}: સંગઠિત કોડ.
\end{itemize}

\textbf{Concept Map:}
\begin{center}
\begin{tikzpicture}[gtu flow]
    \node [gtu start] (func) {Function};
    \node [gtu block, above right=1cm of func] (reuse) {Reuse};
    \node [gtu block, below right=1cm of func] (modular) {Modular};
    \node [gtu block, below left=1cm of func] (debug) {Debug};
    \node [gtu block, above left=1cm of func] (read) {Readable};
    \draw [gtu arrow] (func) -- (reuse);
    \draw [gtu arrow] (func) -- (modular);
    \draw [gtu arrow] (func) -- (debug);
    \draw [gtu arrow] (func) -- (read);
\end{tikzpicture}
\captionof{figure}{Function Advantages}
\end{center}
\end{solutionbox}

\begin{mnemonicbox}
\mnemonic{Reuse Modular Debug Read Maintain Avoid}
\end{mnemonicbox}

\questionmarks{4(ક)}{7}{એક પ્રોગ્રામ લખો જે વપરાશકર્તાને સ્ટ્રિંગ માટે પૂછે અને સ્ટ્રિંગમાં દરેક 'a' નું સ્થાન પ્રિન્ટ કરે.}

\begin{solutionbox}
\textbf{કોડ:}
\begin{lstlisting}[language=Python]
text = input("સ્ટ્રિંગ દાખલ કરો: ")
positions = []

# positions શોધો
for i in range(len(text)):
    if text[i].lower() == 'a':
        positions.append(i)

# Display
if positions:
    print(f"'a' found at indices: {positions}")
    for pos in positions:
        print(f"Index {pos}: '{text[pos]}'")
else:
    print("'a' not found")
\end{lstlisting}
\end{solutionbox}

\begin{mnemonicbox}
\mnemonic{Input Loop Check Store Display}
\end{mnemonicbox}

\questionmarks{4(અ OR)}{3}{લોકલ અને ગ્લોબલ વેરિયેબલ સમજાવો.}

\begin{solutionbox}
\begin{center}
\captionof{table}{Variable Scopes}
\begin{tabulary}{\linewidth}{|L|L|L|}
\hline
\textbf{પ્રકાર} & \textbf{સ્કોપ} & \textbf{એક્સેસ} \\ \hline
\textbf{Local} & ફંક્શનની અંદર & ફંક્શનમાં જ \\ \hline
\textbf{Global} & આખા પ્રોગ્રામમાં & બધે જ \\ \hline
\end{tabulary}
\end{center}

\textbf{Scope Visualization:}
\begin{center}
\begin{tikzpicture}[gtu flow]
    \node [gtu block, minimum width=4cm, minimum height=3cm] (global) {};
    \node [above] at (global.north) {Global Scope (All Access)};
    \node [gtu block, fill=white] at (global.center) (local) {Local Scope\\(Inside Function)};
    \draw [gtu arrow, <->] (global.west) -- node[above, rotate=90] {Access} (local.west);
\end{tikzpicture}
\captionof{figure}{Variable Scope}
\end{center}

\textbf{કોડ:}
\begin{lstlisting}[language=Python]
g = 10  # Global

def func():
    l = 5    # Local
    print(g) # Access Global
\end{lstlisting}
\end{solutionbox}

\begin{mnemonicbox}
\mnemonic{Local Inside Global Everywhere}
\end{mnemonicbox}

\questionmarks{4(બ OR)}{4}{યુઝર ડિફાઇન્ડ ફંક્શનનું બનાવટ અને ઉપયોગ ઉદાહરણ સાથે સમજાવો.}

\begin{solutionbox}
\textbf{સિન્ટેક્સ:}
\begin{lstlisting}[language=Python]
def function_name(params):
    """Docstring"""
    # Body
    return value
\end{lstlisting}

\textbf{ઘટકો:}
1. \textbf{def}: કીવર્ડ
2. \textbf{Name}: નામ
3. \textbf{Parameters}: ઇનપુટ
4. \textbf{Return}: આઉટપુટ

\textbf{ઉદાહરણ:}
\begin{lstlisting}[language=Python]
def greet(name):
    return f"Hello {name}"

msg = greet("John")
print(msg)
\end{lstlisting}
\end{solutionbox}

\begin{mnemonicbox}
\mnemonic{Define Call Return Parameter}
\end{mnemonicbox}

\questionmarks{4(ક OR)}{7}{calcFact() નામનું યુઝર ડિફાઇન્ડ ફંક્શન બનાવવા માટેનો પ્રોગ્રામ લખો કે જે આર્ગ્યુમેન્ટ તરીકે આપેલ સંખ્યાના ફેક્ટોરિયલની ગણતરી કરી તેને ડિસ્પ્લે કરે.}

\begin{solutionbox}
\textbf{કોડ:}
\begin{lstlisting}[language=Python]
def calcFact(n):
    if n < 0:
        return "Undefined"
    elif n == 0 or n == 1:
        return 1
    else:
        fact = 1
        for i in range(2, n + 1):
            fact *= i
        return fact

# Logic
num = int(input("સંખ્યા દાખલ કરો: "))
print(f"Factorial of {num} is {calcFact(num)}")
\end{lstlisting}

\textbf{Recursive Visual:}
\begin{center}
\begin{tikzpicture}[gtu flow]
    \node [gtu process] (f3) {fact(3)};
    \node [gtu process, below=0.5cm of f3] (f2) {3 * fact(2)};
    \node [gtu process, below=0.5cm of f2] (f1) {2 * fact(1)};
    \node [gtu output, right=1cm of f1] (res) {return 1};
    \draw [gtu arrow] (f3) -- (f2);
    \draw [gtu arrow] (f2) -- (f1);
    \draw [gtu arrow] (f1) -- (res);
    \draw [gtu arrow, dashed] (res) |- (f3);
\end{tikzpicture}
\captionof{figure}{Recursion Stack}
\end{center}
\end{solutionbox}

\begin{mnemonicbox}
\mnemonic{Define Check Loop Multiply Return}
\end{mnemonicbox}

\questionmarks{5(અ)}{3}{ક્લાસ અને ઓબ્જેક્ટ વચ્ચે તફાવત આપો.}

\begin{solutionbox}
\textbf{તફાવત:}
\begin{center}
\captionof{table}{Class vs Object}
\begin{tabulary}{\linewidth}{|L|L|L|}
\hline
\textbf{બાબત} & \textbf{Class} & \textbf{Object} \\ \hline
વ્યાખ્યા & બ્લૂપ્રિન્ટ & ઇન્સ્ટન્સ \\ \hline
મેમરી & એલોકેટ નથી & એલોકેટ છે \\ \hline
બનાવટ & \code{class} કીવર્ડ & કન્સ્ટ્રક્ટર કૉલ \\ \hline
\end{tabulary}
\end{center}

\textbf{Diagram:}
\begin{center}
\begin{tikzpicture}[gtu flow]
    \node [gtu class] (cls) {Class: Car Plan};
    \node [gtu block, below left=1cm of cls] (o1) {Object: Audi};
    \node [gtu block, below right=1cm of cls] (o2) {Object: BMW};
    \draw [gtu arrow] (cls) -- (o1);
    \draw [gtu arrow] (cls) -- (o2);
\end{tikzpicture}
\captionof{figure}{Blueprint vs Instances}
\end{center}
\end{solutionbox}

\begin{mnemonicbox}
\mnemonic{Class Blueprint Object Instance}
\end{mnemonicbox}

\questionmarks{5(બ)}{4}{ક્લાસમાં કન્સ્ટ્રક્ટરનો હેતુ જણાવો.}

\begin{solutionbox}
\textbf{હેતુ:}
\begin{itemize}
    \item \keyword{Initialize}: ઓબ્જેક્ટ ઇનિશિયલાઇઝ કરો.
    \item \keyword{Automatic}: આપોઆપ કૉલ થાય છે.
    \item \keyword{Memory}: મેમરી એલોકેટ કરે છે.
\end{itemize}

\textbf{Lifecycle:}
\begin{center}
\begin{tikzpicture}[gtu flow]
    \node [gtu start] (new) {New Object()};
    \node [gtu process, right=1cm of new] (init) {\_\_init\_\_()};
    \node [gtu stop, right=1cm of init] (ready) {Object Ready};
    \draw [gtu arrow] (new) -- (init);
    \draw [gtu arrow] (init) -- (ready);
\end{tikzpicture}
\captionof{figure}{Constructor Flow}
\end{center}
\end{solutionbox}

\begin{mnemonicbox}
\mnemonic{Initialize Automatic Memory Default}
\end{mnemonicbox}

\questionmarks{5(ક)}{7}{"Student" નામનો ક્લાસ બનાવવા માટે પ્રોગ્રામ લખો જેમાં નામ, રોલ નંબર અને માર્ક્સ જેવા એટ્રિબ્યુટ્સ હોય. વિદ્યાર્થીની માહિતી પ્રદર્શિત કરવાની મેથડ બનાવો. "Student" ક્લાસનો ઓબ્જેક્ટ બનાવો અને મેથડનો ઉપયોગ કેવી રીતે કરવો તે બતાવો.}

\begin{solutionbox}
\textbf{કોડ:}
\begin{lstlisting}[language=Python]
class Student:
    def __init__(self, name, roll, marks):
        self.name = name
        self.roll = roll
        self.marks = marks
    
    def display_info(self):
        print("-" * 20)
        print(f"Name: {self.name}")
        print(f"Roll: {self.roll}")
        print(f"Marks: {self.marks}")
        print("-" * 20)

# Objects બનાવો
s1 = Student("John", 101, 85)
s2 = Student("Alice", 102, 90)

# મેથડ વાપરો
s1.display_info()
s2.display_info()
\end{lstlisting}
\end{solutionbox}

\begin{mnemonicbox}
\mnemonic{Class Attributes Constructor Methods Objects}
\end{mnemonicbox}

\questionmarks{5(અ OR)}{3}{એન્કેપ્સ્યુલેશનનો હેતુ જણાવો.}

\begin{solutionbox}
\textbf{Encapsulation:} ડેટા અને મેથડ્સને સાથે રાખવું અને ડેટાને સુરક્ષિત કરવું.

\textbf{હેતુ:}
\begin{itemize}
    \item \keyword{Data Hiding}: ડેટા છુપાવવો.
    \item \keyword{Security}: સુરક્ષા.
    \item \keyword{Controlled Access}: નિયંત્રિત એક્સેસ.
\end{itemize}

\textbf{કોડ:}
\begin{lstlisting}[language=Python]
class Bank:
    def __init__(self):
        self.__bal = 0  # Private
    
    def deposit(self, amt):
        self.__bal += amt
\end{lstlisting}
\end{solutionbox}

\begin{mnemonicbox}
\mnemonic{Hide Protect Control Secure Modular}
\end{mnemonicbox}

\questionmarks{5(બ OR)}{4}{મલ્ટિલેવલ ઇન્હેરિટન્સ સમજાવો.}

\begin{solutionbox}
\textbf{વ્યાખ્યા:} ઇન્હેરિટન્સની શૃંખલા (A $\leftarrow$ B $\leftarrow$ C).

\textbf{Diagram:}
\begin{center}
\begin{tikzpicture}[gtu flow]
    \node [gtu class] (gp) {GrandPa};
    \node [gtu class, below=0.8cm of gp] (p) {Parent};
    \node [gtu class, below=0.8cm of p] (c) {Child};
    \draw [gtu arrow] (gp) -- (p);
    \draw [gtu arrow] (p) -- (c);
\end{tikzpicture}
\captionof{figure}{Multilevel Inheritance}
\end{center}

\textbf{કોડ:}
\begin{lstlisting}[language=Python]
class A: pass
class B(A): pass
class C(B): pass
obj = C() # A, B, C ના features છે
\end{lstlisting}
\end{solutionbox}

\begin{mnemonicbox}
\mnemonic{Chain Inherit Level Access}
\end{mnemonicbox}

\questionmarks{5(ક OR)}{7}{હાઇબ્રિડ ઇન્હેરિટન્સનું કાર્ય દર્શાવતો પાયથોન પ્રોગ્રામ લખો.}

\begin{solutionbox}
\textbf{Hybrid Inheritance:} વિવિધ ઇન્હેરિટન્સનું સંયોજન.

\textbf{Diagram:}
\begin{center}
\begin{tikzpicture}[gtu flow]
    \node [gtu class] (animal) {Animal};
    \node [gtu class, below left=1cm of animal] (mammal) {Mammal};
    \node [gtu class, below right=1cm of animal] (bird) {Bird};
    \node [gtu class, below=1cm of animal, yshift=-2cm] (fd) {FlyingDog};
    \draw [gtu arrow] (animal) -- (mammal);
    \draw [gtu arrow] (animal) -- (bird);
    \draw [gtu arrow] (mammal) -- (fd);
    \draw [gtu arrow] (bird) -- (fd);
\end{tikzpicture}
\captionof{figure}{Hybrid Inheritance}
\end{center}

\textbf{કોડ:}
\begin{lstlisting}[language=Python]
class Animal:
    def __init__(self): print("Animal")

class Mammal(Animal):
    def feed(self): print("Milk")

class Bird(Animal):
    def fly(self): print("Flying")

class FlyingDog(Mammal, Bird):
    def bark(self): print("Bark")

# Object
fd = FlyingDog()
fd.feed()  # Mammal
fd.fly()   # Bird
fd.bark()  # Own
\end{lstlisting}
\end{solutionbox}

\begin{mnemonicbox}
\mnemonic{Hybrid Multiple Single Multilevel Combined}
\end{mnemonicbox}

\end{document}
