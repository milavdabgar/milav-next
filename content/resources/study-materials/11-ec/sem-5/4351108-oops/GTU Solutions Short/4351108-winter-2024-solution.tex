\documentclass{article}
% Adjust the relative path to point to the latex-templates directory

% content/resources/templates/preamble.tex
\usepackage[margin=0.6in]{geometry}
\author{Milav Dabgar}
\usepackage{amsmath,amssymb,amsthm}
\usepackage{booktabs}
\usepackage{multirow}
\usepackage{xcolor}
\usepackage{tcolorbox}
\tcbuselibrary{breakable,skins}
\usepackage[colorlinks=true,linkcolor=blue]{hyperref}
\usepackage{titlesec}
\usepackage{enumitem}
\usepackage{tikz}
\usepackage{pgfplots}
\usepackage{circuitikz}
\usepackage[version=4]{mhchem}
\usepackage{longtable}
\usepackage{array}
\usepackage{float}
\usepackage{caption}
\usepackage{listings}

\lstset{
  basicstyle=\small\ttfamily,
  breaklines=true,
  breakatwhitespace=false,
  postbreak=\mbox{\textcolor{red}{$\hookrightarrow$}\space},
  float=false,
  numbers=left,
  numberstyle=\tiny\color{gray},
  numbersep=10pt,
  xleftmargin=2em,
  keywordstyle=\color{blue},
  commentstyle=\color{green!60!black},
  stringstyle=\color{purple},
  backgroundcolor=\color{gray!5},
  showstringspaces=false,
  tabsize=2,
  captionpos=b,
  keepspaces=true,
  columns=flexible
}

\pgfplotsset{compat=1.18}
\usetikzlibrary{shapes,arrows,positioning,calc,patterns,decorations.pathmorphing,decorations.markings,arrows.meta}

% Color scheme
\definecolor{headcolor}{RGB}{0,102,204}
\definecolor{keycolor}{RGB}{220,20,60}
\definecolor{solutioncolor}{RGB}{34,139,34}
\definecolor{mnemoniccolor}{RGB}{148,0,211}
\definecolor{codecolor}{RGB}{0,0,100}

% Spacing
\setlength{\parskip}{3pt}
\setlist[itemize]{nosep}
\setlist[enumerate]{nosep}

% Title formatting
\titleformat{\section}{\Large\bfseries\color{headcolor}}{\thesection}{1em}{}
\titleformat{\subsection}{\large\bfseries\color{headcolor}}{\thesubsection}{1em}{}

% Pandoc tightlist compatibility
\providecommand{\tightlist}{%
  \setlength{\itemsep}{0pt}\setlength{\parskip}{0pt}}

% Pandoc longtable compatibility
\newcounter{none}
\def\thenone{}


% content/resources/templates/english-boxes.tex
% This file is currently empty - it exists to maintain consistency with the import structure.
% Add custom environments here if needed in the future.


% Custom commands for GTU solutions
% This file defines semantic commands for consistent formatting

% Question command with automatic formatting
\newcommand{\question}[2]{%
  \section*{Question #1}%
  \textbf{#2}%
}

% OR question variant
\newcommand{\questionor}[2]{%
  \section*{Question #1 OR}%
  \textbf{#2}%
}

% Proper table environment with caption
\newenvironment{answertable}[1]{%
  \begin{table}[htbp]
  \centering
  \caption{#1}
}{%
  \end{table}
}

% Proper figure environment for diagrams
\newenvironment{answerdiagram}[1]{%
  \begin{figure}[htbp]
  \centering
  \caption{#1}
}{%
  \end{figure}
}

% Semantic markup for key terms
\newcommand{\keyword}[1]{\textbf{#1}}
\newcommand{\code}[1]{\texttt{#1}}
\newcommand{\classname}[1]{\texttt{#1}}
\newcommand{\methodname}[1]{\texttt{#1}}

% Proper quotation marks
\newcommand{\mnemonic}[1]{``#1''}


\title{OOPS \& Python Programming (4351108) - Winter 2024 Solution}
\date{November 25, 2024}

\begin{document}
\maketitle

\questionmarks{1(a)}{3}{List out features of python programming language.}

\begin{solutionbox}
\begin{center}
\captionof{table}{Features of Python}
\begin{tabulary}{\linewidth}{|L|L|}
\hline
\textbf{Feature} & \textbf{Description} \\ \hline
\keyword{Simple \& Easy} & Clean, readable syntax \\ \hline
\keyword{Free \& Open Source} & No cost, community driven \\ \hline
\keyword{Cross-platform} & Runs on Windows, Linux, Mac \\ \hline
\keyword{Interpreted} & No compilation needed \\ \hline
\keyword{Object-Oriented} & Supports classes and objects \\ \hline
\keyword{Large Libraries} & Rich standard library \\ \hline
\end{tabulary}
\end{center}
\end{solutionbox}

\begin{mnemonicbox}
\mnemonic{Simple Free Cross Interpreted Object Large}
\end{mnemonicbox}

\questionmarks{1(b)}{4}{Write applications of python programming language.}

\begin{solutionbox}
\begin{center}
\captionof{table}{Python Applications}
\begin{tabulary}{\linewidth}{|L|L|}
\hline
\textbf{Application Area} & \textbf{Examples} \\ \hline
\keyword{Web Development} & Django, Flask frameworks \\ \hline
\keyword{Data Science} & NumPy, Pandas, Matplotlib \\ \hline
\keyword{Machine Learning} & TensorFlow, Scikit-learn \\ \hline
\keyword{Desktop GUI} & Tkinter, PyQt applications \\ \hline
\keyword{Game Development} & Pygame library \\ \hline
\keyword{Automation} & Scripting and testing \\ \hline
\end{tabulary}
\end{center}
\end{solutionbox}

\begin{mnemonicbox}
\mnemonic{Web Data Machine Desktop Game Auto}
\end{mnemonicbox}

\questionmarks{1(c)}{7}{Explain various datatypes in python.}

\begin{solutionbox}
\textbf{Data Type Hierarchy:}
\begin{center}
\begin{tikzpicture}[gtu flow]
    \node [gtu block] (root) {Data Types};
    \node [gtu block, below left=1cm of root, xshift=-2cm] (num) {Numeric\\ \code{int, float, complex}};
    \node [gtu block, below=1cm of root] (seq) {Sequence\\ \code{str, list, tuple}};
    \node [gtu block, below right=1cm of root, xshift=2cm] (map) {Mapping\\ \code{dict}};
    \node [gtu block, right=0.5cm of map] (bool) {Boolean\\ \code{bool}};
    
    \draw [gtu arrow] (root) -- (num);
    \draw [gtu arrow] (root) -- (seq);
    \draw [gtu arrow] (root) -- (map);
    \draw [gtu arrow] (root) -- (bool);
\end{tikzpicture}
\captionof{figure}{Python Data Types}
\end{center}

\begin{center}
\captionof{table}{Python Data Types}
\begin{tabulary}{\linewidth}{|L|L|L|}
\hline
\textbf{Data Type} & \textbf{Example} & \textbf{Description} \\ \hline
\textbf{int} & \code{x = 5} & Whole numbers \\ \hline
\textbf{float} & \code{y = 3.14} & Decimal numbers \\ \hline
\textbf{str} & \code{name = "John"} & Text data \\ \hline
\textbf{bool} & \code{flag = True} & True/False values \\ \hline
\textbf{list} & \code{[1, 2, 3]} & Ordered, mutable \\ \hline
\textbf{tuple} & \code{(1, 2, 3)} & Ordered, immutable \\ \hline
\textbf{dict} & \code{\{"a": 1\}} & Key-value pairs \\ \hline
\textbf{set} & \code{\{1, 2, 3\}} & Unique elements \\ \hline
\end{tabulary}
\end{center}

\textbf{Code Example:}
\begin{lstlisting}[language=Python]
# Numeric types
age = 25          # int
price = 99.99     # float

# Text type
name = "Python"   # str

# Boolean type
is_valid = True   # bool

# Collection types
numbers = [1, 2, 3]        # list
coordinates = (10, 20)     # tuple
student = {"name": "John"} # dict
unique_ids = {1, 2, 3}     # set
\end{lstlisting}
\end{solutionbox}

\begin{mnemonicbox}
\mnemonic{Integer Float String Boolean List Tuple Dict Set}
\end{mnemonicbox}

\questionmarks{1(c OR)}{7}{Explain arithmetic, assignment, and identity operators with example.}

\begin{solutionbox}
\textbf{Arithmetic Operators:}
\begin{center}
\captionof{table}{Arithmetic Operators}
\begin{tabulary}{\linewidth}{|C|C|L|}
\hline
\textbf{Op} & \textbf{Name} & \textbf{Example} \\ \hline
\code{+} & Addition & \code{5 + 3 = 8} \\ \hline
\code{-} & Subtraction & \code{5 - 3 = 2} \\ \hline
\code{*} & Multiplication & \code{5 * 3 = 15} \\ \hline
\code{/} & Division & \code{10 / 3 = 3.33} \\ \hline
\code{//} & Floor Div & \code{10 // 3 = 3} \\ \hline
\code{\%} & Modulus & \code{10 \% 3 = 1} \\ \hline
\code{**} & Exponent & \code{2 ** 3 = 8} \\ \hline
\end{tabulary}
\end{center}

\textbf{Assignment Operators:}
\begin{center}
\captionof{table}{Assignment Operators}
\begin{tabulary}{\linewidth}{|C|C|L|}
\hline
\textbf{Op} & \textbf{Example} & \textbf{Equivalent} \\ \hline
\code{=} & \code{x = 5} & Assign value \\ \hline
\code{+=} & \code{x += 3} & \code{x = x + 3} \\ \hline
\code{-=} & \code{x -= 2} & \code{x = x - 2} \\ \hline
\code{*=} & \code{x *= 4} & \code{x = x * 4} \\ \hline
\end{tabulary}
\end{center}

\textbf{Identity Operators:}
\begin{center}
\captionof{table}{Identity Operators}
\begin{tabulary}{\linewidth}{|L|L|L|}
\hline
\textbf{Op} & \textbf{Purpose} & \textbf{Example} \\ \hline
\code{is} & Same object & \code{x is y} \\ \hline
\code{is not} & Different object & \code{x is not y} \\ \hline
\end{tabulary}
\end{center}

\textbf{Code Example:}
\begin{lstlisting}[language=Python]
# Arithmetic
a = 10 + 5    # 15
b = 10 // 3   # 3

# Assignment
x = 5
x += 3        # x becomes 8

# Identity
list1 = [1, 2, 3]
list2 = [1, 2, 3]
print(list1 is list2)      # False
print(list1 is not list2)  # True
\end{lstlisting}
\end{solutionbox}

\begin{mnemonicbox}
\mnemonic{Add Assign Identity}
\end{mnemonicbox}

\questionmarks{2(a)}{3}{Which of the following identifier names are invalid? (i) Total Marks (ii) Total\_Marks (iii) total-Marks (iv) Hundred\$ (v) \_Percentage (vi) True}

\begin{solutionbox}
\begin{center}
\captionof{table}{Identifier Validity}
\begin{tabulary}{\linewidth}{|L|L|L|}
\hline
\textbf{Identifier} & \textbf{Status} & \textbf{Reason} \\ \hline
Total Marks & \textbf{Invalid} & Contains space \\ \hline
Total\_Marks & Valid & Underscore allowed \\ \hline
total-Marks & \textbf{Invalid} & Hyphen not allowed \\ \hline
Hundred\$ & \textbf{Invalid} & \$ symbol not allowed \\ \hline
\_Percentage & Valid & Can start with underscore \\ \hline
True & \textbf{Invalid} & Reserved keyword \\ \hline
\end{tabulary}
\end{center}

\textbf{Invalid identifiers:} Total Marks, total-Marks, Hundred\$, True
\end{solutionbox}

\begin{mnemonicbox}
\mnemonic{Space Hyphen Dollar Keyword = Invalid}
\end{mnemonicbox}

\questionmarks{2(b)}{4}{Write a program to find a maximum number among the given three numbers.}

\begin{solutionbox}
\textbf{Code:}
\begin{lstlisting}[language=Python]
# Input three numbers
num1 = float(input("Enter first number: "))
num2 = float(input("Enter second number: "))
num3 = float(input("Enter third number: "))

# Find maximum using if-elif-else
if num1 >= num2 and num1 >= num3:
    maximum = num1
elif num2 >= num1 and num2 >= num3:
    maximum = num2
else:
    maximum = num3

# Display result
print(f"Maximum number is: {maximum}")
\end{lstlisting}

\textbf{Alternative using max() function:}
\begin{lstlisting}[language=Python]
num1, num2, num3 = map(float, input("Enter 3 numbers: ").split())
maximum = max(num1, num2, num3)
print(f"Maximum: {maximum}")
\end{lstlisting}
\end{solutionbox}

\begin{mnemonicbox}
\mnemonic{Input Compare Display}
\end{mnemonicbox}

\questionmarks{2(c)}{7}{Explain dictionaries in Python. Write statements to add, modify, and delete elements in a dictionary.}

\begin{solutionbox}
\textbf{Dictionary Definition:}
A dictionary is a collection of key-value pairs that is ordered, changeable, and does not allow duplicate keys.

\textbf{Dictionary Operations:}
\begin{center}
\begin{tikzpicture}[gtu flow]
    \node [gtu block] (dict) {Dict: \code{\{'a': 1\}}};
    \node [gtu process, right=1.5cm of dict] (add) {Add: \code{d['b']=2}};
    \node [gtu block, right=1.5cm of add] (res1) {\code{\{'a':1, 'b':2\}}};
    \node [gtu process, below=1cm of dict] (mod) {Modify: \code{d['a']=10}};
    \node [gtu block, below=1cm of res1] (res2) {\code{\{'a':10, 'b':2\}}};
    
    \draw [gtu arrow] (dict) -- (add);
    \draw [gtu arrow] (add) -- (res1);
    \draw [gtu arrow] (res1) |- (mod);
    \draw [gtu arrow] (mod) -- (res2);
\end{tikzpicture}
\captionof{figure}{Dictionary Operations}
\end{center}

\begin{center}
\captionof{table}{Dictionary Methods}
\begin{tabulary}{\linewidth}{|L|L|L|}
\hline
\textbf{Operation} & \textbf{Syntax} & \textbf{Example} \\ \hline
\textbf{Create} & \code{dict = \{\}} & \code{student = \{\}} \\ \hline
\textbf{Add} & \code{d[k] = v} & \code{student['name'] = 'John'} \\ \hline
\textbf{Modify} & \code{d[k] = new} & \code{student['name'] = 'Jane'} \\ \hline
\textbf{Delete} & \code{del d[k]} & \code{del student['name']} \\ \hline
\textbf{Access} & \code{d[k]} & \code{print(student['name'])} \\ \hline
\end{tabulary}
\end{center}

\textbf{Code Example:}
\begin{lstlisting}[language=Python]
# Create empty dictionary
student = {}

# Add elements
student['name'] = 'John'
student['age'] = 20

# Modify element
student['age'] = 21

# Delete element
del student['name']

# Display dictionary
print(student)  # Output: {'age': 21}
\end{lstlisting}
\end{solutionbox}

\begin{mnemonicbox}
\mnemonic{Key-Value Ordered Changeable Unique}
\end{mnemonicbox}

\questionmarks{2(a OR)}{3}{Write a program to display the following pattern.}

\begin{solutionbox}
\textbf{Pattern:}
\begin{verbatim}
1
1 2
1 2 3
1 2 3 4
1 2 3 4 5
\end{verbatim}

\textbf{Code:}
\begin{lstlisting}[language=Python]
# Pattern program
for i in range(1, 6):
    for j in range(1, i + 1):
        print(j, end=" ")
    print()  # New line after each row
\end{lstlisting}
\end{solutionbox}

\begin{mnemonicbox}
\mnemonic{Outer Row Inner Column Print}
\end{mnemonicbox}

\questionmarks{2(b OR)}{4}{Write a program to find the sum of digits of an integer number, input by the user.}

\begin{solutionbox}
\textbf{Code:}
\begin{lstlisting}[language=Python]
# Input number from user
number = int(input("Enter a number: "))
original_number = number
sum_digits = 0

# Extract and sum digits
while number > 0:
    digit = number % 10    # Get last digit
    sum_digits += digit    # Add to sum
    number = number // 10  # Remove last digit

# Display result
print(f"Sum of digits of {original_number} is: {sum_digits}")
\end{lstlisting}

\textbf{Alternative Method:}
\begin{lstlisting}[language=Python]
number = input("Enter number: ")
sum_digits = sum(int(digit) for digit in number)
print(f"Sum of digits: {sum_digits}")
\end{lstlisting}
\end{solutionbox}

\begin{mnemonicbox}
\mnemonic{Input Extract Sum Display}
\end{mnemonicbox}

\questionmarks{2(c OR)}{7}{Explain slicing and concatenation operation on list.}

\begin{solutionbox}
\textbf{List Slicing:}
Extracting portion of list using \code{[start:stop:step]} syntax.

\textbf{Slicing Visualization:}
\begin{center}
\begin{tikzpicture}[gtu flow]
    \node [gtu block] (list) {\code{['A', 'B', 'C', 'D', 'E']}};
    
    % Indices Top
    \node [above=0.2cm of list] {0 \quad 1 \quad 2 \quad 3 \quad 4};
    
    % Indices Bottom
    \node [below=0.2cm of list] {-5 \quad -4 \quad -3 \quad -2 \quad -1};
    
    \node [below=1cm of list] (slice) {\code{list[1:4]} $\to$ \code{['B', 'C', 'D']}};
    \node [below=0.5cm of slice] (rev) {\code{list[::-1]} $\to$ \code{['E', 'D', 'C', 'B', 'A']}};
    
    \draw [gtu arrow] (list) -- (slice);
\end{tikzpicture}
\captionof{figure}{List Indexing \& Slicing}
\end{center}

\textbf{Operations Table:}
\begin{center}
\captionof{table}{List Operations}
\begin{tabulary}{\linewidth}{|L|L|L|}
\hline
\textbf{Syntax} & \textbf{Description} & \textbf{Example} \\ \hline
\code{l[start:stop]} & Elements from start to stop-1 & \code{nums[1:4]} \\ \hline
\code{l[:stop]} & From beginning & \code{nums[:3]} \\ \hline
\code{l[::step]} & With step & \code{nums[::2]} \\ \hline
\code{l1 + l2} & Concatenation & \code{[1]+[2]} \\ \hline
\end{tabulary}
\end{center}

\textbf{Code Example:}
\begin{lstlisting}[language=Python]
list1 = [1, 2, 3, 4, 5]
list2 = [6, 7, 8]

# Slicing
print(list1[1:4])    # [2, 3, 4]
print(list1[::-1])   # [5, 4, 3, 2, 1]

# Concatenation
result = list1 + list2  # [1, 2, 3... 8]
list1.extend(list2)     # Modifies list1
\end{lstlisting}
\end{solutionbox}

\begin{mnemonicbox}
\mnemonic{Slice Extract Concat Join}
\end{mnemonicbox}

\questionmarks{3(a)}{3}{Define a list in Python. Write name of the function used to add an element to the end of a list.}

\begin{solutionbox}
\textbf{List Definition:}
A \textbf{list} is an ordered collection of items that is changeable and allows duplicate values.

\textbf{Properties:}
\begin{itemize}
    \item \keyword{Ordered}: Items have defined order
    \item \keyword{Changeable}: Can modify after creation
    \item \keyword{Duplicates}: Allows duplicate values
\end{itemize}

\textbf{Function to add element:} \code{append()}

\textbf{Example:}
\begin{lstlisting}[language=Python]
fruits = ['apple', 'banana']
fruits.append('orange')
print(fruits)  # ['apple', 'banana', 'orange']
\end{lstlisting}
\end{solutionbox}

\begin{mnemonicbox}
\mnemonic{List Append End}
\end{mnemonicbox}

\questionmarks{3(b)}{4}{Define a tuple in Python. Write statement to access last element of a tuple.}

\begin{solutionbox}
\textbf{Tuple Definition:}
A \textbf{tuple} is an ordered collection of items that is unchangeable and allows duplicate values.

\textbf{Accessing Last Element:}
\begin{lstlisting}[language=Python]
my_tuple = (10, 20, 30, 40, 50)

# Method 1: Negative index
last = my_tuple[-1]  # 50

# Method 2: Lens
last = my_tuple[len(my_tuple) - 1] # 50
\end{lstlisting}
\end{solutionbox}

\begin{mnemonicbox}
\mnemonic{Tuple Unchangeable Negative Index}
\end{mnemonicbox}

\questionmarks{3(c)}{7}{Write statements for following set operations: create empty set, add an element to a set, remove an element from set, Union of two sets, Intersection of two sets, Difference between two sets and symmetric difference between two sets.}

\begin{solutionbox}
\textbf{Set Operations Table:}
\begin{center}
\captionof{table}{Set Operations}
\begin{tabulary}{\linewidth}{|L|L|L|}
\hline
\textbf{Operation} & \textbf{Method/Op} & \textbf{Example} \\ \hline
Create Empty & \code{set()} & \code{s = set()} \\ \hline
Add & \code{add()} & \code{s.add(5)} \\ \hline
Remove & \code{remove()} & \code{s.remove(5)} \\ \hline
Union & \code{union() |} & \code{A | B} \\ \hline
Intersection & \code{intersection() \&} & \code{A \& B} \\ \hline
Difference & \code{difference() -} & \code{A - B} \\ \hline
Symm. Diff & \code{sym\_diff() \^{}} & \code{A \^{} B} \\ \hline
\end{tabulary}
\end{center}

\textbf{Set Venn Diagram:}
\begin{center}
\begin{tikzpicture}[gtu flow]
    \node [gtu state, minimum size=1.5cm, fill=blue!10] (A) {A};
    \node [gtu state, minimum size=1.5cm, fill=red!10, right=1cm of A] (B) {B};
    
    \node [below=0.5cm of A] (U) {Union: All};
    \node [below=0.5cm of B] (I) {Intersect: Common};
    
    \draw [gtu arrow] (A) -- node[above] {\&} (B);
\end{tikzpicture}
\captionof{figure}{Set Relations}
\end{center}

\textbf{Code Example:}
\begin{lstlisting}[language=Python]
s = set()
s.add(10)
s.remove(10)

A = {1, 2, 3}
B = {3, 4, 5}
print(A | B)  # {1, 2, 3, 4, 5}
print(A & B)  # {3}
print(A - B)  # {1, 2}
print(A ^ B)  # {1, 2, 4, 5}
\end{lstlisting}
\end{solutionbox}

\begin{mnemonicbox}
\mnemonic{Create Add Remove Union Intersect Differ Symmetric}
\end{mnemonicbox}

\questionmarks{3(a OR)}{3}{Define a string in Python. Using example illustrate (i) How to create a string. (ii) Accessing individual characters using indexing.}

\begin{solutionbox}
\textbf{String Definition:}
A string is a sequence of characters enclosed in quotes (single, double, or triple).

\textbf{String Structure:}
\begin{center}
\begin{tikzpicture}[gtu flow]
    \node [gtu block] (str) {P \quad Y \quad T \quad H \quad O \quad N};
    \node [above=0.2cm of str] {\small 0 \quad 1 \quad 2 \quad 3 \quad 4 \quad 5};
    \node [below=0.2cm of str] {\small -6 \quad -5 \quad -4 \quad -3 \quad -2 \quad -1};
\end{tikzpicture}
\captionof{figure}{String Indexing}
\end{center}

\textbf{Code:}
\begin{lstlisting}[language=Python]
# Creation
s1 = 'Hello'
s2 = "World"

# Accessing
word = "PYTHON"
print(word[0])   # P
print(word[-1])  # N
\end{lstlisting}
\end{solutionbox}

\begin{mnemonicbox}
\mnemonic{String Quotes Index Access}
\end{mnemonicbox}

\questionmarks{3(b OR)}{4}{Explain list traversing using for loop and while loop.}

\begin{solutionbox}
\textbf{List Traversing} means visiting each element of list one by one.

\textbf{Comparison:}
\begin{center}
\captionof{table}{Loops Comparison}
\begin{tabulary}{\linewidth}{|L|L|}
\hline
\textbf{For Loop} & \textbf{While Loop} \\ \hline
Simpler syntax & More control \\ \hline
Best for fixed iterations & Best for condition-based \\ \hline
\end{tabulary}
\end{center}

\textbf{Code Example:}
\begin{lstlisting}[language=Python]
nums = [10, 20, 30]

# For Loop
for x in nums:
    print(x)

# While Loop
i = 0
while i < len(nums):
    print(nums[i])
    i += 1
\end{lstlisting}
\end{solutionbox}

\begin{mnemonicbox}
\mnemonic{For Simple While Control}
\end{mnemonicbox}

\questionmarks{3(c OR)}{7}{Write a program to create a dictionary with the roll number, name, and marks of n students and display the names of students who have scored marks above 75.}

\begin{solutionbox}
\textbf{Code:}
\begin{lstlisting}[language=Python]
# Input number of students
n = int(input("Enter number of students: "))
students = {}

# Input data
for i in range(n):
    print(f"\nStudent {i + 1}:")
    roll = int(input("Roll: "))
    name = input("Name: ")
    marks = float(input("Marks: "))
    
    students[roll] = {'name': name, 'marks': marks}

# Display high performers
print("\nStudents with marks > 75:")
found = False
for roll, data in students.items():
    if data['marks'] > 75:
        print(f"Name: {data['name']}, Marks: {data['marks']}")
        found = True

if not found:
    print("None found")
\end{lstlisting}
\end{solutionbox}

\begin{mnemonicbox}
\mnemonic{Input Store Filter Display}
\end{mnemonicbox}

\questionmarks{4(a)}{3}{Write any three functions available in random module. Write syntax and example of each function.}

\begin{solutionbox}
\begin{center}
\captionof{table}{Random Functions}
\begin{tabulary}{\linewidth}{|L|L|L|}
\hline
\textbf{Function} & \textbf{Description} & \textbf{Example} \\ \hline
\code{random()} & Float 0.0 to 1.0 & \code{0.75} \\ \hline
\code{randint(a,b)} & Integer a to b & \code{5} \\ \hline
\code{choice(seq)} & Random element & \code{'red'} \\ \hline
\end{tabulary}
\end{center}

\textbf{Code:}
\begin{lstlisting}[language=Python]
import random
print(random.random())
print(random.randint(1, 10))
print(random.choice(['a', 'b', 'c']))
\end{lstlisting}
\end{solutionbox}

\begin{mnemonicbox}
\mnemonic{Random Randint Choice}
\end{mnemonicbox}

\questionmarks{4(b)}{4}{Write the advantages of functions.}

\begin{solutionbox}
\textbf{Advantages:}
\begin{itemize}
    \item \keyword{Code Reusability}: Write once, use multiple times.
    \item \keyword{Modularity}: Break complex problems into smaller parts.
    \item \keyword{Debugging}: Easier to isolate and fix errors.
    \item \keyword{Readability}: Code is more organized.
\end{itemize}

\textbf{Concept Map:}
\begin{center}
\begin{tikzpicture}[gtu flow]
    \node [gtu start] (func) {Function};
    \node [gtu block, above right=1cm of func] (reuse) {Reuse};
    \node [gtu block, below right=1cm of func] (modular) {Modular};
    \node [gtu block, below left=1cm of func] (debug) {Debug};
    \node [gtu block, above left=1cm of func] (read) {Readable};
    
    \draw [gtu arrow] (func) -- (reuse);
    \draw [gtu arrow] (func) -- (modular);
    \draw [gtu arrow] (func) -- (debug);
    \draw [gtu arrow] (func) -- (read);
\end{tikzpicture}
\captionof{figure}{Function Advantages}
\end{center}
\end{solutionbox}

\begin{mnemonicbox}
\mnemonic{Reuse Modular Debug Read Maintain Avoid}
\end{mnemonicbox}

\questionmarks{4(c)}{7}{Write a program that asks the user for a string and prints out the location of each 'a' in the string.}

\begin{solutionbox}
\textbf{Code:}
\begin{lstlisting}[language=Python]
text = input("Enter a string: ")
positions = []

# Find positions
for i in range(len(text)):
    if text[i].lower() == 'a':
        positions.append(i)

# Display
if positions:
    print(f"'a' found at indices: {positions}")
    for pos in positions:
        print(f"Index {pos}: '{text[pos]}'")
else:
    print("'a' not found")
\end{lstlisting}
\end{solutionbox}

\begin{mnemonicbox}
\mnemonic{Input Loop Check Store Display}
\end{mnemonicbox}

\questionmarks{4(a OR)}{3}{Explain local and global variables.}

\begin{solutionbox}
\textbf{Scope Comparison:}
\begin{center}
\captionof{table}{Variable Scopes}
\begin{tabulary}{\linewidth}{|L|L|L|}
\hline
\textbf{Type} & \textbf{Scope} & \textbf{Access} \\ \hline
\textbf{Local} & Inside function & Function only \\ \hline
\textbf{Global} & Entire program & Everywhere \\ \hline
\end{tabulary}
\end{center}

\textbf{Scope Visualization:}
\begin{center}
\begin{tikzpicture}[gtu flow]
    \node [gtu block, minimum width=4cm, minimum height=3cm] (global) {};
    \node [above] at (global.north) {Global Scope (All Access)};
    
    \node [gtu block, fill=white] at (global.center) (local) {Local Scope\\(Inside Function)};
    
    \draw [gtu arrow, <->] (global.west) -- node[above, rotate=90] {Access} (local.west);
\end{tikzpicture}
\captionof{figure}{Variable Scope}
\end{center}

\textbf{Code:}
\begin{lstlisting}[language=Python]
g = 10  # Global

def func():
    l = 5    # Local
    print(g) # Access Global
    # global g; g = 20 # To modify
\end{lstlisting}
\end{solutionbox}

\begin{mnemonicbox}
\mnemonic{Local Inside Global Everywhere}
\end{mnemonicbox}

\questionmarks{4(b OR)}{4}{Explain creation and use of user defined function with example.}

\begin{solutionbox}
\textbf{Function Syntax:}
\begin{lstlisting}[language=Python]
def function_name(params):
    """Docstring"""
    # Body
    return value
\end{lstlisting}

\textbf{Components:}
1. \textbf{def}: Keyword
2. \textbf{Name}: Identifier
3. \textbf{Parameters}: Inputs
4. \textbf{Return}: Output

\textbf{Example:}
\begin{lstlisting}[language=Python]
def greet(name):
    return f"Hello {name}"

msg = greet("John")
print(msg)
\end{lstlisting}
\end{solutionbox}

\begin{mnemonicbox}
\mnemonic{Define Call Return Parameter}
\end{mnemonicbox}

\questionmarks{4(c OR)}{7}{Write a program to create a user defined function calcFact() to calculate and display the factorial of a number passed as an argument.}

\begin{solutionbox}
\textbf{Code:}
\begin{lstlisting}[language=Python]
def calcFact(n):
    if n < 0:
        return "Undefined"
    elif n == 0 or n == 1:
        return 1
    else:
        fact = 1
        for i in range(2, n + 1):
            fact *= i
        return fact

# Logic
num = int(input("Enter number: "))
print(f"Factorial of {num} is {calcFact(num)}")
\end{lstlisting}

\textbf{Recursive Visual:}
\begin{center}
\begin{tikzpicture}[gtu flow]
    \node [gtu process] (f3) {fact(3)};
    \node [gtu process, below=0.5cm of f3] (f2) {3 * fact(2)};
    \node [gtu process, below=0.5cm of f2] (f1) {2 * fact(1)};
    \node [gtu output, right=1cm of f1] (res) {return 1};
    
    \draw [gtu arrow] (f3) -- (f2);
    \draw [gtu arrow] (f2) -- (f1);
    \draw [gtu arrow] (f1) -- (res);
    \draw [gtu arrow, dashed] (res) |- (f3);
\end{tikzpicture}
\captionof{figure}{Recursion Stack}
\end{center}
\end{solutionbox}

\begin{mnemonicbox}
\mnemonic{Define Check Loop Multiply Return}
\end{mnemonicbox}

\questionmarks{5(a)}{3}{Give difference between class and object.}

\begin{solutionbox}
\textbf{Comparison:}
\begin{center}
\captionof{table}{Class vs Object}
\begin{tabulary}{\linewidth}{|L|L|L|}
\hline
\textbf{Feature} & \textbf{Class} & \textbf{Object} \\ \hline
Definition & Blueprint & Instance \\ \hline
Memory & Not allocated & Allocated \\ \hline
Keyword & \code{class} & Constructor call \\ \hline
\end{tabulary}
\end{center}

\textbf{Analogy:}
\begin{center}
\begin{tikzpicture}[gtu flow]
    \node [gtu class] (cls) {Class: Car Plan};
    \node [gtu block, below left=1cm of cls] (o1) {Object: Audi};
    \node [gtu block, below right=1cm of cls] (o2) {Object: BMW};
    
    \draw [gtu arrow] (cls) -- (o1);
    \draw [gtu arrow] (cls) -- (o2);
\end{tikzpicture}
\captionof{figure}{Blueprint vs Instances}
\end{center}
\end{solutionbox}

\begin{mnemonicbox}
\mnemonic{Class Blueprint Object Instance}
\end{mnemonicbox}

\questionmarks{5(b)}{4}{State the purpose of a constructor in a class.}

\begin{solutionbox}
\textbf{Purpose:}
\begin{itemize}
    \item \keyword{Initialize}: Set initial state of object.
    \item \keyword{Automatic}: Called when object is created.
    \item \keyword{Memory}: Allocates required memory.
\end{itemize}

\textbf{Lifecycle:}
\begin{center}
\begin{tikzpicture}[gtu flow]
    \node [gtu start] (new) {New Object()};
    \node [gtu process, right=1cm of new] (init) {\_\_init\_\_()};
    \node [gtu stop, right=1cm of init] (ready) {Object Ready};
    
    \draw [gtu arrow] (new) -- (init);
    \draw [gtu arrow] (init) -- (ready);
\end{tikzpicture}
\captionof{figure}{Constructor Flow}
\end{center}

\textbf{Code:}
\begin{lstlisting}[language=Python]
class Demo:
    def __init__(self, val):
        self.val = val

obj = Demo(10) # Calls __init__
\end{lstlisting}
\end{solutionbox}

\begin{mnemonicbox}
\mnemonic{Initialize Automatic Memory Default}
\end{mnemonicbox}

\questionmarks{5(c)}{7}{Write a program to create a class "Student" with attributes such as name, roll number, and marks. Implement method to display student information. Create object of the student class and show how to use method.}

\begin{solutionbox}
\textbf{Code:}
\begin{lstlisting}[language=Python]
class Student:
    def __init__(self, name, roll, marks):
        self.name = name
        self.roll = roll
        self.marks = marks
    
    def display_info(self):
        print("-" * 20)
        print(f"Name: {self.name}")
        print(f"Roll: {self.roll}")
        print(f"Marks: {self.marks}")
        print("-" * 20)

# Create objects
s1 = Student("John", 101, 85)
s2 = Student("Alice", 102, 90)

# Use method
s1.display_info()
s2.display_info()
\end{lstlisting}

\textbf{Output:}
\begin{verbatim}
--------------------
Name: John
Roll: 101
Marks: 85
--------------------
\end{verbatim}
\end{solutionbox}

\begin{mnemonicbox}
\mnemonic{Class Attributes Constructor Methods Objects}
\end{mnemonicbox}

\questionmarks{5(a OR)}{3}{State the purpose of encapsulation.}

\begin{solutionbox}
\textbf{Encapsulation} is bundling data and methods, and restricting direct access to data.

\textbf{Purpose:}
\begin{itemize}
    \item \keyword{Data Hiding}: Protects internal state.
    \item \keyword{Security}: Prevents accidental modification.
    \item \keyword{Controlled Access}: Use getters/setters.
\end{itemize}

\textbf{Code:}
\begin{lstlisting}[language=Python]
class Bank:
    def __init__(self):
        self.__bal = 0  # Private
    
    def deposit(self, amt):
        self.__bal += amt
\end{lstlisting}
\end{solutionbox}

\begin{mnemonicbox}
\mnemonic{Hide Protect Control Secure Modular}
\end{mnemonicbox}

\questionmarks{5(b OR)}{4}{Explain multilevel inheritance.}

\begin{solutionbox}
\textbf{Definition:} Chain of inheritance (A $\leftarrow$ B $\leftarrow$ C).

\textbf{Diagram:}
\begin{center}
\begin{tikzpicture}[gtu flow]
    \node [gtu class] (gp) {GrandPa};
    \node [gtu class, below=0.8cm of gp] (p) {Parent};
    \node [gtu class, below=0.8cm of p] (c) {Child};
    
    \draw [gtu arrow] (gp) -- (p);
    \draw [gtu arrow] (p) -- (c);
\end{tikzpicture}
\captionof{figure}{Multilevel Inheritance}
\end{center}

\textbf{Code:}
\begin{lstlisting}[language=Python]
class A: pass
class B(A): pass
class C(B): pass

obj = C() # Has features of A, B, C
\end{lstlisting}
\end{solutionbox}

\begin{mnemonicbox}
\mnemonic{Chain Inherit Level Access}
\end{mnemonicbox}

\questionmarks{5(c OR)}{7}{Write a Python program to demonstrate working of hybrid inheritance.}

\begin{solutionbox}
\textbf{Hybrid Inheritance:} Combination of multiple inheritance types (e.g., Diamond problem).

\textbf{Diagram:}
\begin{center}
\begin{tikzpicture}[gtu flow]
    \node [gtu class] (animal) {Animal};
    \node [gtu class, below left=1cm of animal] (mammal) {Mammal};
    \node [gtu class, below right=1cm of animal] (bird) {Bird};
    \node [gtu class, below=1cm of animal, yshift=-2cm] (fd) {FlyingDog};
    
    \draw [gtu arrow] (animal) -- (mammal);
    \draw [gtu arrow] (animal) -- (bird);
    \draw [gtu arrow] (mammal) -- (fd);
    \draw [gtu arrow] (bird) -- (fd);
\end{tikzpicture}
\captionof{figure}{Hybrid Inheritance}
\end{center}

\textbf{Code:}
\begin{lstlisting}[language=Python]
class Animal:
    def __init__(self): print("Animal")

class Mammal(Animal):
    def feed(self): print("Milk")

class Bird(Animal):
    def fly(self): print("Flying")

class FlyingDog(Mammal, Bird):
    def bark(self): print("Bark")

# Object
fd = FlyingDog()
fd.feed()  # Mammal
fd.fly()   # Bird
fd.bark()  # Own
\end{lstlisting}
\end{solutionbox}

\begin{mnemonicbox}
\mnemonic{Hybrid Multiple Single Multilevel Combined}
\end{mnemonicbox}

\end{document}
