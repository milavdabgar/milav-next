\documentclass{article}
% Adjust the relative path to point to the latex-templates directory

% content/resources/templates/preamble.tex
\usepackage[margin=0.6in]{geometry}
\author{Milav Dabgar}
\usepackage{amsmath,amssymb,amsthm}
\usepackage{booktabs}
\usepackage{multirow}
\usepackage{xcolor}
\usepackage{tcolorbox}
\tcbuselibrary{breakable,skins}
\usepackage[colorlinks=true,linkcolor=blue]{hyperref}
\usepackage{titlesec}
\usepackage{enumitem}
\usepackage{tikz}
\usepackage{pgfplots}
\usepackage{circuitikz}
\usepackage[version=4]{mhchem}
\usepackage{longtable}
\usepackage{array}
\usepackage{float}
\usepackage{caption}
\usepackage{listings}

\lstset{
  basicstyle=\small\ttfamily,
  breaklines=true,
  breakatwhitespace=false,
  postbreak=\mbox{\textcolor{red}{$\hookrightarrow$}\space},
  float=false,
  numbers=left,
  numberstyle=\tiny\color{gray},
  numbersep=10pt,
  xleftmargin=2em,
  keywordstyle=\color{blue},
  commentstyle=\color{green!60!black},
  stringstyle=\color{purple},
  backgroundcolor=\color{gray!5},
  showstringspaces=false,
  tabsize=2,
  captionpos=b,
  keepspaces=true,
  columns=flexible
}

\pgfplotsset{compat=1.18}
\usetikzlibrary{shapes,arrows,positioning,calc,patterns,decorations.pathmorphing,decorations.markings,arrows.meta}

% Color scheme
\definecolor{headcolor}{RGB}{0,102,204}
\definecolor{keycolor}{RGB}{220,20,60}
\definecolor{solutioncolor}{RGB}{34,139,34}
\definecolor{mnemoniccolor}{RGB}{148,0,211}
\definecolor{codecolor}{RGB}{0,0,100}

% Spacing
\setlength{\parskip}{3pt}
\setlist[itemize]{nosep}
\setlist[enumerate]{nosep}

% Title formatting
\titleformat{\section}{\Large\bfseries\color{headcolor}}{\thesection}{1em}{}
\titleformat{\subsection}{\large\bfseries\color{headcolor}}{\thesubsection}{1em}{}

% Pandoc tightlist compatibility
\providecommand{\tightlist}{%
  \setlength{\itemsep}{0pt}\setlength{\parskip}{0pt}}

% Pandoc longtable compatibility
\newcounter{none}
\def\thenone{}


% content/resources/templates/english-boxes.tex

% Custom environments
\newtcolorbox{solutionbox}{
 breakable,
 enhanced,
 colback=solutioncolor!5!white,
 colframe=solutioncolor!75!black,
 fonttitle=\bfseries,
 title=Solution
}

\newtcolorbox{solutionboxnobreak}{
 colback=solutioncolor!5!white,
 colframe=solutioncolor!75!black,
 fonttitle=\bfseries,
 title=Solution
}

\newtcolorbox{keyformula}{
 breakable,
 enhanced,
 colback=keycolor!5!white,
 colframe=keycolor!75!black,
 fonttitle=\bfseries,
 title=Key Formula
}

\newtcolorbox{mnemonicboxenv}{
 breakable,
 enhanced,
 colback=mnemoniccolor!5!white,
 colframe=mnemoniccolor!75!black,
 fonttitle=\bfseries,
 title=Mnemonic
}

\newcommand{\mnemonicbox}[1]{%
  \begin{mnemonicboxenv}
    #1
  \end{mnemonicboxenv}
}


% Custom commands for GTU solutions
% This file defines semantic commands for consistent formatting

% Question command with automatic formatting
\newcommand{\question}[2]{%
  \section*{Question #1}%
  \textbf{#2}%
}

% OR question variant
\newcommand{\questionor}[2]{%
  \section*{Question #1 OR}%
  \textbf{#2}%
}

% Proper table environment with caption
\newenvironment{answertable}[1]{%
  \begin{table}[htbp]
  \centering
  \caption{#1}
}{%
  \end{table}
}

% Proper figure environment for diagrams
\newenvironment{answerdiagram}[1]{%
  \begin{figure}[htbp]
  \centering
  \caption{#1}
}{%
  \end{figure}
}

% Semantic markup for key terms
\newcommand{\keyword}[1]{\textbf{#1}}
\newcommand{\code}[1]{\texttt{#1}}
\newcommand{\classname}[1]{\texttt{#1}}
\newcommand{\methodname}[1]{\texttt{#1}}

% Proper quotation marks
\newcommand{\mnemonic}[1]{``#1''}


\title{OOPS \& Python Programming (4351108) - Winter 2023 Solution}
\date{December 06, 2023}

\begin{document}
\maketitle

\questionmarks{1(a)}{3}{List any 6 applications of Python programming language.}

\begin{solutionbox}
\begin{center}
\captionof{table}{Python Applications}
\begin{tabulary}{\linewidth}{|L|L|}
\hline
\textbf{Application Area} & \textbf{Description} \\ \hline
\keyword{Web Development} & Django, Flask frameworks \\ \hline
\keyword{Data Science} & Analysis and visualization \\ \hline
\keyword{Machine Learning} & AI model development \\ \hline
\keyword{Desktop Applications} & GUI using Tkinter, PyQt \\ \hline
\keyword{Game Development} & Pygame library \\ \hline
\keyword{Automation} & Scripting and testing \\ \hline
\end{tabulary}
\end{center}
\end{solutionbox}

\begin{mnemonicbox}
\mnemonic{Web Data Machine Desktop Game Auto}
\end{mnemonicbox}

\questionmarks{1(b)}{4}{List any 8 features of Python programming language.}

\begin{solutionbox}
\begin{center}
\captionof{table}{Python Features}
\begin{tabulary}{\linewidth}{|L|L|}
\hline
\textbf{Feature} & \textbf{Description} \\ \hline
\keyword{Simple Syntax} & Easy to read and write \\ \hline
\keyword{Interpreted} & No compilation needed \\ \hline
\keyword{Object-Oriented} & Supports OOP concepts \\ \hline
\keyword{Dynamic Typing} & Variables don't need type declaration \\ \hline
\keyword{Cross-Platform} & Runs on multiple OS \\ \hline
\keyword{Large Libraries} & Rich standard library \\ \hline
\keyword{Open Source} & Free to use and modify \\ \hline
\keyword{Interactive} & REPL environment \\ \hline
\end{tabulary}
\end{center}
\end{solutionbox}

\begin{mnemonicbox}
\mnemonic{Simple Interpreted Object Dynamic Cross Large Open Interactive}
\end{mnemonicbox}

\questionmarks{1(c)}{7}{Explain working of for and while loops in Python.}

\begin{solutionbox}
\textbf{For Loop:}
\begin{itemize}
    \item \keyword{Iteration}: Repeats over sequences (lists, strings, ranges)
    \item \keyword{Syntax}: \code{for variable in sequence:}
    \item \keyword{Automatic}: Handles iteration automatically
\end{itemize}

\textbf{While Loop:}
\begin{itemize}
    \item \keyword{Condition-based}: Continues while condition is true
    \item \keyword{Manual control}: Programmer controls iteration
    \item \keyword{Risk}: Can create infinite loops if condition never becomes false
\end{itemize}

\textbf{Loop Logic Diagram:}
\begin{center}
\begin{tikzpicture}[gtu flow]
    \node [gtu start] (start) {Start};
    \node [gtu decision, below=0.5cm of start] (cond) {Condition?};
    \node [gtu process, below=0.5cm of cond] (exec) {Execute Body};
    \node [gtu process, below=0.5cm of exec] (update) {Update};
    \node [gtu stop, right=2cm of cond] (end) {End};
    
    \path [gtu arrow] (start) -- (cond);
    \path [gtu arrow] (cond) -- node[right] {True} (exec);
    \path [gtu arrow] (exec) -- (update);
    \path [gtu arrow] (update.west) -- ++(-0.5,0) |- (cond);
    \path [gtu arrow] (cond) -- node[above] {False} (end);
\end{tikzpicture}
\captionof{figure}{General Loop Flow}
\end{center}

\textbf{Code Example:}
\begin{lstlisting}[language=Python]
# For loop
for i in range(5):
    print(i)

# While loop
i = 0
while i < 5:
    print(i)
    i += 1
\end{lstlisting}
\end{solutionbox}

\begin{mnemonicbox}
\mnemonic{For Automatic, While Manual}
\end{mnemonicbox}

\questionmarks{1(c OR)}{7}{Explain working of break continue and pass statements in Python.}

\begin{solutionbox}
\textbf{Break Statement:}
\begin{itemize}
    \item \keyword{Exit}: Terminates the entire loop
    \item \keyword{Usage}: When specific condition is met
    \item \keyword{Effect}: Control moves to next statement after loop
\end{itemize}

\textbf{Continue Statement:}
\begin{itemize}
    \item \keyword{Skip}: Skips current iteration only
    \item \keyword{Usage}: Skip specific values in iteration
    \item \keyword{Effect}: Moves to next iteration
\end{itemize}

\textbf{Pass Statement:}
\begin{itemize}
    \item \keyword{Placeholder}: Does nothing, syntactic placeholder
    \item \keyword{Usage}: When syntax requires statement but no action needed
    \item \keyword{Effect}: No operation performed
\end{itemize}

\textbf{Control Flow Visualization:}
\begin{center}
\begin{tikzpicture}[gtu flow]
    \node [gtu start] (start) {Loop Start};
    \node [gtu decision, below=0.5cm of start] (cond) {Condition};
    \node [gtu decision, below=0.5cm of cond] (break) {break?};
    \node [gtu decision, below=0.5cm of break] (cont) {continue?};
    \node [gtu decision, below=0.5cm of cont] (pass) {pass?};
    \node [gtu process, below=0.5cm of pass] (exec) {Execute Code};
    \node [gtu stop, right=2cm of break] (exit) {Exit Loop};
    
    \path [gtu arrow] (start) -- (cond);
    \path [gtu arrow] (cond) -- node[right] {True} (break);
    \path [gtu arrow] (break) -- node[above] {Yes} (exit);
    \path [gtu arrow] (break) -- node[right] {No} (cont);
    \path [gtu arrow] (cont) -- node[right] {No} (pass);
    \path [gtu arrow] (cont.west) -- node[above] {Yes} ++(-0.5,0) |- (cond);
    \path [gtu arrow] (pass) -- (exec);
    \path [gtu arrow] (exec.west) -- ++(-0.5,0) |- (cond);
\end{tikzpicture}
\captionof{figure}{Loop Control Statements}
\end{center}

\textbf{Code Examples:}
\begin{lstlisting}[language=Python]
# Break
for i in range(10):
    if i == 5: break
    print(i)  # 0,1,2,3,4

# Continue
for i in range(5):
    if i == 2: continue
    print(i)  # 0,1,3,4

# Pass
if True: pass  # placeholder
\end{lstlisting}
\end{solutionbox}

\begin{mnemonicbox}
\mnemonic{Break Exits, Continue Skips, Pass Waits}
\end{mnemonicbox}

\questionmarks{2(a)}{3}{Develop a Python program to increment each element of list by one.}

\begin{solutionbox}
\textbf{Code:}
\begin{lstlisting}[language=Python]
# Method 1 - Using for loop
numbers = [1, 2, 3, 4, 5]
for i in range(len(numbers)):
    numbers[i] += 1
print(numbers)

# Method 2 - List comprehension
numbers = [1, 2, 3, 4, 5]
result = [x + 1 for x in numbers]
print(result)
\end{lstlisting}
\end{solutionbox}

\begin{mnemonicbox}
\mnemonic{Loop Index or Comprehension}
\end{mnemonicbox}

\questionmarks{2(b)}{4}{Develop a Python program to read three numbers from the user and find the average of the numbers.}

\begin{solutionbox}
\textbf{Code:}
\begin{lstlisting}[language=Python]
# Input three numbers
num1 = float(input("Enter first number: "))
num2 = float(input("Enter second number: "))
num3 = float(input("Enter third number: "))

# Calculate average
average = (num1 + num2 + num3) / 3

# Display result
print(f"Average is: {average}")
\end{lstlisting}

\textbf{Key Points:}
\begin{itemize}
    \item \keyword{Input}: Use \code{float()} for decimal numbers
    \item \keyword{Formula}: Sum divided by count
    \item \keyword{Output}: Use f-string for formatting
\end{itemize}
\end{solutionbox}

\begin{mnemonicbox}
\mnemonic{Input Float, Sum Divide, Format Output}
\end{mnemonicbox}

\questionmarks{2(c)}{7}{Explain Python's list data type in detail.}

\begin{solutionbox}
\textbf{List Characteristics:}
\begin{itemize}
    \item \keyword{Ordered}: Elements maintain sequence
    \item \keyword{Mutable}: Can be modified after creation
    \item \keyword{Heterogeneous}: Can store different data types
    \item \keyword{Indexed}: Access elements using index (0-based)
\end{itemize}

\textbf{List Operations Table:}
\begin{center}
\captionof{table}{List Operations}
\begin{tabulary}{\linewidth}{|L|L|L|}
\hline
\textbf{Operation} & \textbf{Syntax} & \textbf{Description} \\ \hline
\textbf{Creation} & \code{list = [1,2,3]} & Create new list \\ \hline
\textbf{Access} & \code{list[0]} & Get element by index \\ \hline
\textbf{Append} & \code{list.append(4)} & Add element at end \\ \hline
\textbf{Insert} & \code{list.insert(1,5)} & Add at specific position \\ \hline
\textbf{Remove} & \code{list.remove(2)} & Remove first occurrence \\ \hline
\textbf{Pop} & \code{list.pop()} & Remove and return last \\ \hline
\textbf{Slice} & \code{list[1:3]} & Get sublist \\ \hline
\end{tabulary}
\end{center}

\textbf{List Structure Diagram:}
\begin{center}
\begin{tikzpicture}[gtu flow]
    \node [gtu block] (list) {\code{[10, 20, 30]}};
    \node [gtu process, right=1.5cm of list] (op1) {append(40)};
    \node [gtu block, right=1.5cm of op1] (res1) {\code{[10, 20, 30, 40]}};
    \node [gtu process, below=1cm of list] (op2) {pop(0)};
    \node [gtu block, below=1cm of res1] (res2) {\code{[20, 30, 40]}};
    
    \draw [gtu arrow] (list) -- (op1);
    \draw [gtu arrow] (op1) -- (res1);
    \draw [gtu arrow] (res1) -- (op2);
    \draw [gtu arrow] (op2) -- (res2);
\end{tikzpicture}
\captionof{figure}{List Mutation}
\end{center}

\textbf{Code Example:}
\begin{lstlisting}[language=Python]
# List creation and operations
fruits = ['apple', 'banana', 'orange']
fruits.append('mango')
fruits.insert(1, 'grape')
print(fruits[0])  # apple
print(len(fruits))  # 5
\end{lstlisting}
\end{solutionbox}

\begin{mnemonicbox}
\mnemonic{Ordered Mutable Heterogeneous Indexed}
\end{mnemonicbox}

\questionmarks{2(a OR)}{3}{Develop a Python program to find sum of all elements in a list using for loop.}

\begin{solutionbox}
\textbf{Code:}
\begin{lstlisting}[language=Python]
# Method 1 - Traditional for loop
numbers = [10, 20, 30, 40, 50]
total = 0
for num in numbers:
    total += num
print(f"Sum is: {total}")

# Method 2 - Using range and index
numbers = [10, 20, 30, 40, 50]
total = 0
for i in range(len(numbers)):
    total += numbers[i]
print(f"Sum is: {total}")
\end{lstlisting}
\end{solutionbox}

\begin{mnemonicbox}
\mnemonic{Initialize Zero, Loop Add, Print Total}
\end{mnemonicbox}

\questionmarks{2(b OR)}{4}{Develop a Python program to get input from user for principal, rate and no of years then calculate and display simple interest from that.}

\begin{solutionbox}
\textbf{Code:}
\begin{lstlisting}[language=Python]
# Get input from user
principal = float(input("Enter principal amount: "))
rate = float(input("Enter rate of interest: "))
time = float(input("Enter time in years: "))

# Calculate simple interest
simple_interest = (principal * rate * time) / 100

# Display results
print(f"Principal: {principal}")
print(f"Rate: {rate}%")
print(f"Time: {time} years")
print(f"Simple Interest: {simple_interest}")
print(f"Total Amount: {principal + simple_interest}")
\end{lstlisting}

\textbf{Formula:}
\begin{itemize}
    \item \textbf{Simple Interest} = (P $\times$ R $\times$ T) / 100
    \item \textbf{Total Amount} = Principal + Simple Interest
\end{itemize}
\end{solutionbox}

\begin{mnemonicbox}
\mnemonic{Principal Rate Time, Multiply Divide Hundred}
\end{mnemonicbox}

\questionmarks{2(c OR)}{7}{Explain Python's tuple data type in detail.}

\begin{solutionbox}
\textbf{Tuple Characteristics:}
\begin{itemize}
    \item \keyword{Ordered}: Elements maintain sequence
    \item \keyword{Immutable}: Cannot be modified after creation
    \item \keyword{Heterogeneous}: Can store different data types
    \item \keyword{Indexed}: Access using index (0-based)
\end{itemize}

\textbf{Tuple Operations Table:}
\begin{center}
\captionof{table}{Tuple Operations}
\begin{tabulary}{\linewidth}{|L|L|L|}
\hline
\textbf{Operation} & \textbf{Syntax} & \textbf{Description} \\ \hline
\textbf{Creation} & \code{tuple = (1,2,3)} & Create new tuple \\ \hline
\textbf{Access} & \code{tuple[0]} & Get element by index \\ \hline
\textbf{Count} & \code{tuple.count(2)} & Count occurrences \\ \hline
\textbf{Index} & \code{tuple.index(3)} & Find first index \\ \hline
\textbf{Slice} & \code{tuple[1:3]} & Get sub-tuple \\ \hline
\textbf{Length} & \code{len(tuple)} & Get tuple size \\ \hline
\textbf{Concatenate} & \code{tuple1 + tuple2} & Join tuples \\ \hline
\end{tabulary}
\end{center}

\textbf{Tuple Comparison Diagram:}
\begin{center}
\begin{tikzpicture}[gtu flow]
    \node [gtu block] (tuple) {\textbf{Tuple}\\ \code{(1, 2, 3)}\\ Immutable};
    \node [gtu block, right=2cm of tuple] (list) {\textbf{List}\\ \code{[1, 2, 3]}\\ Mutable};
    
    \draw [gtu arrow] (tuple) -- node[above] {Fixed} (list);
    \node [below=0.2cm of tuple] {Read-Only};
    \node [below=0.2cm of list] {Read-Write};
\end{tikzpicture}
\captionof{figure}{Tuple vs List}
\end{center}

\textbf{Code Example:}
\begin{lstlisting}[language=Python]
# Tuple creation and operations
coordinates = (10, 20, 30)
print(coordinates[0])  # 10
print(len(coordinates))  # 3
x, y, z = coordinates  # tuple unpacking
new_tuple = coordinates + (40, 50)
\end{lstlisting}
\end{solutionbox}

\begin{mnemonicbox}
\mnemonic{Ordered Immutable Heterogeneous Indexed}
\end{mnemonicbox}

\questionmarks{3(a)}{3}{Explain any 3 random module methods.}

\begin{solutionbox}
\begin{center}
\captionof{table}{Random Module Methods}
\begin{tabulary}{\linewidth}{|L|L|L|}
\hline
\textbf{Method} & \textbf{Syntax} & \textbf{Description} \\ \hline
\textbf{random()} & \code{random.random()} & Float between 0.0 to 1.0 \\ \hline
\textbf{randint()} & \code{random.randint(a,b)} & Integer between a and b \\ \hline
\textbf{choice()} & \code{random.choice(list)} & Random element from sequence \\ \hline
\end{tabulary}
\end{center}

\textbf{Code Example:}
\begin{lstlisting}[language=Python]
import random
print(random.random())        # 0.7234567
print(random.randint(1, 10))  # 7
print(random.choice(['r','g','b'])) # g
\end{lstlisting}
\end{solutionbox}

\begin{mnemonicbox}
\mnemonic{Random Float, Randint Integer, Choice Select}
\end{mnemonicbox}

\questionmarks{3(b)}{4}{Develop a Python program that asks the user for a string and prints out the location of each 'a' in the string.}

\begin{solutionbox}
\textbf{Code:}
\begin{lstlisting}[language=Python]
# Get string from user
text = input("Enter a string: ")

# Find all positions of 'a'
positions = []
for i in range(len(text)):
    if text[i].lower() == 'a':
        positions.append(i)

# Display results
if positions:
    print(f"Letter 'a' found at positions: {positions}")
else:
    print("Letter 'a' not found in the string")
\end{lstlisting}

\textbf{Key Points:}
\begin{itemize}
    \item \keyword{Case-insensitive}: Use \code{.lower()} to find both 'a' and 'A'
    \item \keyword{Index tracking}: Use range or enumerate
    \item \keyword{Output format}: Clear position indication
\end{itemize}
\end{solutionbox}

\begin{mnemonicbox}
\mnemonic{Loop Index Check Append Print}
\end{mnemonicbox}

\questionmarks{3(c)}{7}{Explain Python's string data type in detail.}

\begin{solutionbox}
\textbf{String Characteristics:}
\begin{itemize}
    \item \keyword{Immutable}: Cannot be changed after creation
    \item \keyword{Sequence}: Ordered collection of characters
    \item \keyword{Indexed}: Access characters using index
    \item \keyword{Unicode}: Supports all languages and symbols
\end{itemize}

\textbf{String Methods Table:}
\begin{center}
\captionof{table}{String Methods}
\begin{tabulary}{\linewidth}{|L|L|L|}
\hline
\textbf{Method} & \textbf{Example} & \textbf{Description} \\ \hline
\textbf{upper()} & \code{"s".upper()} & Convert to uppercase \\ \hline
\textbf{lower()} & \code{"S".lower()} & Convert to lowercase \\ \hline
\textbf{strip()} & \code{" s ".strip()} & Remove whitespace \\ \hline
\textbf{split()} & \code{"a,b".split(",")} & Split into list \\ \hline
\textbf{replace()} & \code{"s".replace("s","x")} & Replace substring \\ \hline
\textbf{find()} & \code{"s".find("e")} & Find substring index \\ \hline
\end{tabulary}
\end{center}

\textbf{String Structure Diagram:}
\begin{center}
\begin{tikzpicture}[gtu flow]
    \node [gtu block] (str) {String: "Hello"};
    \node [gtu block, right=1cm of str] (mem) {Address: 0x100};
    \node [gtu process, below=1cm of str] (op) {s[0] = 'h'};
    \node [gtu block, below=1cm of mem] (err) {Error: Immutable};
    
    \draw [gtu arrow] (str) -- (mem);
    \draw [gtu arrow] (str) -- (op);
    \draw [gtu arrow] (op) -- (err);
\end{tikzpicture}
\captionof{figure}{String Immutability}
\end{center}

\textbf{Code Example:}
\begin{lstlisting}[language=Python]
name = "Python Programming"
print(name[0])      # P
print(name[0:6])    # Python
message = f"I love {name}"
\end{lstlisting}
\end{solutionbox}

\begin{mnemonicbox}
\mnemonic{Immutable Sequence Indexed Unicode}
\end{mnemonicbox}

\questionmarks{3(a OR)}{3}{Explain any 3 math module methods.}

\begin{solutionbox}
\begin{center}
\captionof{table}{Math Module Methods}
\begin{tabulary}{\linewidth}{|L|L|L|}
\hline
\textbf{Method} & \textbf{Syntax} & \textbf{Description} \\ \hline
\textbf{sqrt()} & \code{math.sqrt(16)} & Square root calculation \\ \hline
\textbf{pow()} & \code{math.pow(2,3)} & Power calculation \\ \hline
\textbf{ceil()} & \code{math.ceil(4.3)} & Round up to integer \\ \hline
\end{tabulary}
\end{center}

\textbf{Code Example:}
\begin{lstlisting}[language=Python]
import math
print(math.sqrt(25))    # 5.0
print(math.pow(2, 3))   # 8.0
print(math.ceil(4.2))   # 5
\end{lstlisting}
\end{solutionbox}

\begin{mnemonicbox}
\mnemonic{Square Root, Power Up, Ceiling Round}
\end{mnemonicbox}

\questionmarks{3(b OR)}{4}{Develop a Python program to get a string from the user and count total no. of Vowels present in that string.}

\begin{solutionbox}
\textbf{Code:}
\begin{lstlisting}[language=Python]
# Get string from user
text = input("Enter a string: ")

# Define vowels
vowels = "aeiouAEIOU"

# Count vowels
vowel_count = 0
for char in text:
    if char in vowels:
        vowel_count += 1

# Display result
print(f"Total vowels in '{text}': {vowel_count}")
\end{lstlisting}
\end{solutionbox}

\begin{mnemonicbox}
\mnemonic{Define Vowels, Loop Check, Count Increment}
\end{mnemonicbox}

\questionmarks{3(c OR)}{7}{Explain Python's set data type in detail.}

\begin{solutionbox}
\textbf{Set Characteristics:}
\begin{itemize}
    \item \keyword{Unordered}: No fixed sequence of elements
    \item \keyword{Mutable}: Can add/remove elements
    \item \keyword{Unique}: No duplicate elements allowed
    \item \keyword{Iterable}: Can loop through elements
\end{itemize}

\textbf{Set Operations Table:}
\begin{center}
\captionof{table}{Set Operations}
\begin{tabulary}{\linewidth}{|L|L|L|}
\hline
\textbf{Operation} & \textbf{Syntax} & \textbf{Description} \\ \hline
\textbf{Creation} & \code{set = \{1,2,3\}} & Create new set \\ \hline
\textbf{Add} & \code{set.add(4)} & Add single element \\ \hline
\textbf{Remove} & \code{set.remove(2)} & Remove element \\ \hline
\textbf{Union} & \code{s1 | s2} & Combine sets \\ \hline
\textbf{Intersection} & \code{s1 \& s2} & Common elements \\ \hline
\textbf{Difference} & \code{s1 - s2} & Elements in s1 only \\ \hline
\end{tabulary}
\end{center}

\textbf{Set Venn Diagram:}
\begin{center}
\begin{tikzpicture}[gtu flow]
    \node [gtu state, minimum size=1.5cm] (A) {A};
    \node [gtu state, minimum size=1.5cm, right=1cm of A] (B) {B};
    
    \node [gtu block, below=1cm of A, xshift=1cm] (ops) {Operators};
    \draw [gtu arrow] (A) -- node[below] {\&} (B);
    \draw [gtu arrow] (A) |- node[left] {|} (ops);
    \draw [gtu arrow] (B) |- node[right] {-} (ops);
\end{tikzpicture}
\captionof{figure}{Set Logic}
\end{center}

\textbf{Code Example:}
\begin{lstlisting}[language=Python]
A = {1, 2, 3, 4}
B = {3, 4, 5, 6}
print(A | B)    # Union: {1,2,3,4,5,6}
print(A & B)    # Intersection: {3,4}
\end{lstlisting}
\end{solutionbox}

\begin{mnemonicbox}
\mnemonic{Unordered Mutable Unique Iterable}
\end{mnemonicbox}

\questionmarks{4(a)}{3}{What is the class in Python. How is it different from an object?}

\begin{solutionbox}
\textbf{Class vs Object comparison:}
\begin{center}
\captionof{table}{Class vs Object}
\begin{tabulary}{\linewidth}{|L|L|L|}
\hline
\textbf{Aspect} & \textbf{Class} & \textbf{Object} \\ \hline
\textbf{Definition} & Blueprint/Template & Instance of Class \\ \hline
\textbf{Memory} & No memory allocated & Memory allocated \\ \hline
\textbf{Existence} & Logical entity & Physical entity \\ \hline
\textbf{Creation} & \code{class} keyword & Constructor \\ \hline
\end{tabulary}
\end{center}

\textbf{Relationship Diagram:}
\begin{center}
\begin{tikzpicture}[gtu flow]
    \node [gtu class] (cls) {Class Car\\Blueprint};
    \node [gtu block, below left=1cm of cls] (o1) {Object 1\\Toyota};
    \node [gtu block, below right=1cm of cls] (o2) {Object 2\\Honda};
    
    \draw [gtu arrow] (cls) -- (o1);
    \draw [gtu arrow] (cls) -- (o2);
\end{tikzpicture}
\captionof{figure}{Class to Objects}
\end{center}
\end{solutionbox}

\begin{mnemonicbox}
\mnemonic{Class Blueprint, Object Instance}
\end{mnemonicbox}

\questionmarks{4(b)}{4}{Explain any four methods of dictionary data type of Python.}

\begin{solutionbox}
\textbf{Dictionary Methods Table:}
\begin{center}
\captionof{table}{Dictionary Methods}
\begin{tabulary}{\linewidth}{|L|L|L|}
\hline
\textbf{Method} & \textbf{Syntax} & \textbf{Description} \\ \hline
\textbf{keys()} & \code{d.keys()} & Get all keys \\ \hline
\textbf{values()} & \code{d.values()} & Get all values \\ \hline
\textbf{items()} & \code{d.items()} & Get key-value pairs \\ \hline
\textbf{get()} & \code{d.get('k')} & Get value safely \\ \hline
\end{tabulary}
\end{center}

\textbf{Code Example:}
\begin{lstlisting}[language=Python]
student = {'name': 'John', 'grade': 'A'}
print(student.keys())    # ['name', 'grade']
print(student.values())  # ['John', 'A']
print(student.get('age')) # None (no error)
\end{lstlisting}
\end{solutionbox}

\begin{mnemonicbox}
\mnemonic{Keys Values Items Get}
\end{mnemonicbox}

\questionmarks{4(c)}{7}{Develop a Python program that defines a user-defined module for performing some tasks. Import this module and use its functions.}

\begin{solutionbox}
\textbf{Module Structure:}
\begin{center}
\begin{tikzpicture}[gtu flow]
    \node [gtu block] (mod) {math\_operations.py\\ \code{add(), multiply()}};
    \node [gtu block, right=2cm of mod] (main) {main.py\\ \code{import math\_operations}};
    
    \draw [gtu arrow] (main) -- node[above] {Uses} (mod);
\end{tikzpicture}
\captionof{figure}{Module Import}
\end{center}

\textbf{Module (math\_operations.py):}
\begin{lstlisting}[language=Python]
def add(a, b):
    return a + b

def multiply(a, b):
    return a * b

PI = 3.14159
\end{lstlisting}

\textbf{Main Program (main.py):}
\begin{lstlisting}[language=Python]
import math_operations as mo

res1 = mo.add(5, 3)
res2 = mo.multiply(4, 6)

print(f"Addition: {res1}")
print(f"Multiplication: {res2}")
print(f"PI: {mo.PI}")
\end{lstlisting}

\textbf{Key Points:}
\begin{itemize}
    \item \keyword{Module creation}: Separate .py file with functions
    \item \keyword{Import}: \code{import module} or \code{from module import func}
    \item \keyword{Usage}: \code{module.function()}
\end{itemize}
\end{solutionbox}

\begin{mnemonicbox}
\mnemonic{Create Import Use}
\end{mnemonicbox}

\questionmarks{4(a OR)}{3}{Define types of methods available in Python classes.}

\begin{solutionbox}
\begin{center}
\captionof{table}{Method Types}
\begin{tabulary}{\linewidth}{|L|L|L|}
\hline
\textbf{Type} & \textbf{Decorator} & \textbf{First Argument} \\ \hline
\textbf{Instance} & None & \code{self} \\ \hline
\textbf{Class} & \code{@classmethod} & \code{cls} \\ \hline
\textbf{Static} & \code{@staticmethod} & None \\ \hline
\end{tabulary}
\end{center}

\textbf{Example:}
\begin{lstlisting}[language=Python]
class Demo:
    def inst(self): pass
    @classmethod
    def cls_m(cls): pass
    @staticmethod
    def stat(): pass
\end{lstlisting}
\end{solutionbox}

\begin{mnemonicbox}
\mnemonic{Instance Self, Class Cls, Static None}
\end{mnemonicbox}

\questionmarks{4(b OR)}{4}{Explain any four methods of string data type of Python.}

\begin{solutionbox}
\textbf{String Methods: Checks & Counts}
\begin{center}
\captionof{table}{String Check Methods}
\begin{tabulary}{\linewidth}{|L|L|L|}
\hline
\textbf{Method} & \textbf{Syntax} & \textbf{Description} \\ \hline
\textbf{startswith} & \code{s.startswith('a')} & Starts with substring? \\ \hline
\textbf{endswith} & \code{s.endswith('z')} & Ends with substring? \\ \hline
\textbf{isdigit} & \code{s.isdigit()} & All digits? \\ \hline
\textbf{count} & \code{s.count('a')} & Count occurrences \\ \hline
\end{tabulary}
\end{center}

\textbf{Code Example:}
\begin{lstlisting}[language=Python]
s = "Hello World 123"
print(s.startswith("He")) # True
print(s.endswith("23"))   # True
print("123".isdigit())    # True
print(s.count("l"))       # 3
\end{lstlisting}
\end{solutionbox}

\begin{mnemonicbox}
\mnemonic{Start End Digit Count}
\end{mnemonicbox}

\questionmarks{4(c OR)}{7}{Develop a Python program to find factorial of a number using recursive user defined function.}

\begin{solutionbox}
\textbf{Code:}
\begin{lstlisting}[language=Python]
def factorial(n):
    # Base case
    if n == 0 or n == 1:
        return 1
    # Recursive case
    else:
        return n * factorial(n - 1)

try:
    num = int(input("Enter a number: "))
    if num < 0:
        print("Negative number not allowed")
    else:
        print(f"Factorial of {num} is {factorial(num)}")
except ValueError:
    print("Invalid input")
\end{lstlisting}

\textbf{Recursion Stack Visualization:}
\begin{center}
\begin{tikzpicture}[gtu flow]
    \node [gtu process] (call5) {fact(5)};
    \node [gtu process, below=0.5cm of call5] (call4) {5 * fact(4)};
    \node [gtu process, below=0.5cm of call4] (call3) {4 * fact(3)};
    \node [gtu process, below=0.5cm of call3] (call2) {3 * fact(2)};
    \node [gtu process, below=0.5cm of call2] (call1) {2 * fact(1)};
    \node [gtu output, right=1cm of call1] (ret1) {return 1};
    
    \draw [gtu arrow] (call5) -- (call4);
    \draw [gtu arrow] (call4) -- (call3);
    \draw [gtu arrow] (call3) -- (call2);
    \draw [gtu arrow] (call2) -- (call1);
    \draw [gtu dashed arrow] (call1) -- (ret1);
    \draw [gtu dashed arrow] (ret1) |- (call5); 
\end{tikzpicture}
\captionof{figure}{Recursive Calls}
\end{center}
\end{solutionbox}

\begin{mnemonicbox}
\mnemonic{Base Stop, Recursive Call, Error Check}
\end{mnemonicbox}

\questionmarks{5(a)}{3}{Develop a python program to Implement single inheritance.}

\begin{solutionbox}
\textbf{Code:}
\begin{lstlisting}[language=Python]
class Animal:
    def speak(self): print("Animal Speak")

class Dog(Animal):
    def bark(self): print("Dog Bark")

d = Dog()
d.speak()  # Inherited
d.bark()   # Own
\end{lstlisting}

\textbf{Inheritance Diagram:}
\begin{center}
\begin{tikzpicture}[gtu flow]
    \node [gtu class] (parent) {Animal\\speak()};
    \node [gtu class, below=1cm of parent] (child) {Dog\\bark()};
    \draw [gtu arrow] (parent) -- (child);
\end{tikzpicture}
\captionof{figure}{Single Inheritance}
\end{center}
\end{solutionbox}

\begin{mnemonicbox}
\mnemonic{Parent Child Inherit Override}
\end{mnemonicbox}

\questionmarks{5(b)}{4}{Explain the significance of constructors in Python classes.}

\begin{solutionbox}
\textbf{Constructor Significance Table:}
\begin{center}
\captionof{table}{Constructor Features}
\begin{tabulary}{\linewidth}{|L|L|}
\hline
\textbf{Aspect} & \textbf{Description} \\ \hline
\textbf{Initialization} & Called automatically on creation \\ \hline
\textbf{Setup} & Set initial attribute values \\ \hline
\textbf{Memory} & Allocate memory for object \\ \hline
\textbf{Validation} & Validate inputs \\ \hline
\end{tabulary}
\end{center}

\textbf{Lifecycle Flow:}
\begin{center}
\begin{tikzpicture}[gtu flow]
    \node [gtu start] (create) {New Object};
    \node [gtu process, right=1cm of create] (init) {\_\_init\_\_()};
    \node [gtu stop, right=1cm of init] (ready) {Ready};
    \draw [gtu arrow] (create) -- (init);
    \draw [gtu arrow] (init) -- (ready);
\end{tikzpicture}
\captionof{figure}{Constructor Flow}
\end{center}
\end{solutionbox}

\begin{mnemonicbox}
\mnemonic{Initialize Setup Memory Validate}
\end{mnemonicbox}

\questionmarks{5(c)}{7}{Develop a Python program to demonstrate method overriding using inheritance.}

\begin{solutionbox}
\textbf{Code:}
\begin{lstlisting}[language=Python]
class Shape:
    def area(self): print("Shape Area")

class Circle(Shape):
    def area(self): print("Circle Area")

class Rect(Shape):
    def area(self): print("Rect Area")

shapes = [Circle(), Rect()]
for s in shapes:
    s.area()
\end{lstlisting}

\textbf{Method Overriding Hierarchy:}
\begin{center}
\begin{tikzpicture}[gtu flow]
    \node [gtu class] (base) {Shape\\area()};
    \node [gtu class, below left=1cm of base] (c1) {Circle\\Override area()};
    \node [gtu class, below right=1cm of base] (c2) {Rect\\Override area()};
    
    \draw [gtu arrow] (base) -- (c1);
    \draw [gtu arrow] (base) -- (c2);
\end{tikzpicture}
\captionof{figure}{Polymorphism}
\end{center}

\textbf{Key Points:}
\begin{itemize}
    \item \keyword{Same Name}: Method name same in parent/child
    \item \keyword{Different Logic}: Child provides specific implementation
    \item \keyword{Runtime Decision}: Object type determines method
\end{itemize}
\end{solutionbox}

\begin{mnemonicbox}
\mnemonic{Same Name Different Logic Runtime Decision}
\end{mnemonicbox}

\questionmarks{5(a OR)}{3}{Explain concept of data encapsulation in Python.}

\begin{solutionbox}
\textbf{Comparison:}
\begin{center}
\captionof{table}{Encapsulation}
\begin{tabulary}{\linewidth}{|L|L|}
\hline
\textbf{Aspect} & \textbf{Description} \\ \hline
\textbf{Definition} & Bundling data \& methods \\ \hline
\textbf{Data Hiding} & Private attributes (\_\_var) \\ \hline
\textbf{Access} & Public methods (getters/setters) \\ \hline
\end{tabulary}
\end{center}

\textbf{Encapsulation Diagram:}
\begin{center}
\begin{tikzpicture}[gtu flow]
    \node [gtu block] (data) {Private Data\\ \_\_balance};
    \node [gtu class, fit=(data), label=above:Class Bank] (cls) {};
    \node [gtu process, right=2cm of cls] (user) {User};
    
    \draw [gtu arrow] (user) -- node[above] {deposit()} (cls);
    \draw [gtu arrow] (cls) -- node[below] {get\_balance()} (user);
    \draw [gtu arrow, dashed] (user) -- node[below, red] {Direct X} (data);
\end{tikzpicture}
\captionof{figure}{Secure Access}
\end{center}
\end{solutionbox}

\begin{mnemonicbox}
\mnemonic{Bundle Data Hide Interface}
\end{mnemonicbox}

\questionmarks{5(b OR)}{4}{Explain concept of abstract classes in Python.}

\begin{solutionbox}
\textbf{Abstract Class Properties:}
\begin{itemize}
    \item \keyword{Definition}: Class that cannot be instantiated
    \item \keyword{Abstract Methods}: Methods without implementation
    \item \keyword{Implementation}: Child MUST implement these methods
    \item \keyword{Module}: Uses \code{abc} module
\end{itemize}

\textbf{Abstraction Logic:}
\begin{center}
\begin{tikzpicture}[gtu flow]
    \node [gtu class, dashed] (abs) {Abstract Class\\ \textit{sound()}};
    \node [gtu class, below=1cm of abs] (impl) {Dog\\sound(): "Woof"};
    \draw [gtu arrow] (abs) -- node[right] {Implements} (impl);
\end{tikzpicture}
\captionof{figure}{Abstract Base Class}
\end{center}
\end{solutionbox}

\begin{mnemonicbox}
\mnemonic{Cannot Instantiate Force Implementation Common Interface}
\end{mnemonicbox}

\questionmarks{5(c OR)}{7}{Develop a python program to Implement multiple inheritance.}

\begin{solutionbox}
\textbf{Code:}
\begin{lstlisting}[language=Python]
class Father:
    def work(self): print("Father Engineer")

class Mother:
    def work(self): print("Mother Doctor")

class Child(Father, Mother):
    pass

c = Child()
c.work() # Father (MRO)
print(Child.__mro__)
\end{lstlisting}

\textbf{Multiple Inheritance Structure:}
\begin{center}
\begin{tikzpicture}[gtu flow]
    \node [gtu class] (f) {Father};
    \node [gtu class, right=1.5cm of f] (m) {Mother};
    \node [gtu class, below=1cm of f, xshift=1cm] (c) {Child};
    
    \draw [gtu arrow] (f) -- (c);
    \draw [gtu arrow] (m) -- (c);
\end{tikzpicture}
\captionof{figure}{Multiple Parents}
\end{center}

\textbf{Key Points:}
\begin{itemize}
    \item \keyword{Structure}: Child inherits from >1 parent
    \item \keyword{MRO}: Method Resolution Order determines priority
    \item \keyword{Diamond Problem}: Solved by C3 linearization
\end{itemize}
\end{solutionbox}

\begin{mnemonicbox}
\mnemonic{Multiple Parents MRO Constructor Diamond}
\end{mnemonicbox}

\end{document}
