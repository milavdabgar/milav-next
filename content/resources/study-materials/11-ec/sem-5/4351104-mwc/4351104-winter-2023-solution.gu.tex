\documentclass{article}

% content/resources/templates/preamble.tex
\usepackage[margin=0.6in]{geometry}
\author{Milav Dabgar}
\usepackage{amsmath,amssymb,amsthm}
\usepackage{booktabs}
\usepackage{multirow}
\usepackage{xcolor}
\usepackage{tcolorbox}
\tcbuselibrary{breakable,skins}
\usepackage[colorlinks=true,linkcolor=blue]{hyperref}
\usepackage{titlesec}
\usepackage{enumitem}
\usepackage{tikz}
\usepackage{pgfplots}
\usepackage{circuitikz}
\usepackage[version=4]{mhchem}
\usepackage{longtable}
\usepackage{array}
\usepackage{float}
\usepackage{caption}
\usepackage{listings}

\lstset{
  basicstyle=\small\ttfamily,
  breaklines=true,
  breakatwhitespace=false,
  postbreak=\mbox{\textcolor{red}{$\hookrightarrow$}\space},
  float=false,
  numbers=left,
  numberstyle=\tiny\color{gray},
  numbersep=10pt,
  xleftmargin=2em,
  keywordstyle=\color{blue},
  commentstyle=\color{green!60!black},
  stringstyle=\color{purple},
  backgroundcolor=\color{gray!5},
  showstringspaces=false,
  tabsize=2,
  captionpos=b,
  keepspaces=true,
  columns=flexible
}

\pgfplotsset{compat=1.18}
\usetikzlibrary{shapes,arrows,positioning,calc,patterns,decorations.pathmorphing,decorations.markings,arrows.meta}

% Color scheme
\definecolor{headcolor}{RGB}{0,102,204}
\definecolor{keycolor}{RGB}{220,20,60}
\definecolor{solutioncolor}{RGB}{34,139,34}
\definecolor{mnemoniccolor}{RGB}{148,0,211}
\definecolor{codecolor}{RGB}{0,0,100}

% Spacing
\setlength{\parskip}{3pt}
\setlist[itemize]{nosep}
\setlist[enumerate]{nosep}

% Title formatting
\titleformat{\section}{\Large\bfseries\color{headcolor}}{\thesection}{1em}{}
\titleformat{\subsection}{\large\bfseries\color{headcolor}}{\thesubsection}{1em}{}

% Pandoc tightlist compatibility
\providecommand{\tightlist}{%
  \setlength{\itemsep}{0pt}\setlength{\parskip}{0pt}}

% Pandoc longtable compatibility
\newcounter{none}
\def\thenone{}


% content/resources/templates/gujarati-boxes.tex
\usepackage{fontspec}
\usepackage{polyglossia}

% Set Gujarati as main language (document is primarily in Gujarati)
% Note: gloss-gujarati.ldf doesn't exist in polyglossia, but it will use hyphenation patterns
\setdefaultlanguage{gujarati}
\setotherlanguage{english}

% Configure Gujarati font properly
% Use Language=Default to prevent polyglossia from trying to add language-specific features
% that don't exist for Gujarati, which causes "empty feature" warnings
\newfontfamily\gujaratifont[Script=Gujarati,AutoFakeBold=2.5,AutoFakeSlant=0.3]{Noto Sans Gujarati}
\setmainfont[Script=Gujarati,AutoFakeBold=2.5,AutoFakeSlant=0.3]{Noto Sans Gujarati}
% Use Noto Sans Gujarati for monospace to support Gujarati in text
\setmonofont[Scale=0.9]{Noto Sans Gujarati}

% Configure English to use the same font
\newfontfamily\englishfont[Script=Gujarati,AutoFakeBold=2.5,AutoFakeSlant=0.3]{Noto Sans Gujarati}

% Translations for polyglossia
\gappto\captionsgujarati{
  \renewcommand{\tablename}{કોષ્ટક}
  \renewcommand{\figurename}{આકૃતિ}
}

% Helper for TikZ nodes to ensure Gujarati font
\newcommand{\gu}[1]{{\gujaratifont #1}}

% Custom environments
\newtcolorbox{solutionbox}{
    breakable,
    enhanced,
    colback=solutioncolor!5!white,
    colframe=solutioncolor!75!black,
    fonttitle=\bfseries,
    title=જવાબ
}

\newtcolorbox{solutionboxnobreak}{
 colback=solutioncolor!5!white,
 colframe=solutioncolor!75!black,
 fonttitle=\bfseries,
 title=જવાબ
}

\newtcolorbox{keyformula}{
 breakable,
 enhanced,
 colback=keycolor!5!white,
 colframe=keycolor!75!black,
 fonttitle=\bfseries,
 title=રાસાયણિક સમીકરણ/સૂત્ર
}

\newtcolorbox{mnemonicbox}{
 breakable,
 enhanced,
 colback=mnemoniccolor!5!white,
 colframe=mnemoniccolor!75!black,
 fonttitle=\bfseries,
 title=મેમરી ટ્રીક
}


% Custom commands for GTU solutions
% This file defines semantic commands for consistent formatting

% Question command with automatic formatting
\newcommand{\question}[2]{%
  \section*{Question #1}%
  \textbf{#2}%
}

% OR question variant
\newcommand{\questionor}[2]{%
  \section*{Question #1 OR}%
  \textbf{#2}%
}

% Proper table environment with caption
\newenvironment{answertable}[1]{%
  \begin{table}[htbp]
  \centering
  \caption{#1}
}{%
  \end{table}
}

% Proper figure environment for diagrams
\newenvironment{answerdiagram}[1]{%
  \begin{figure}[htbp]
  \centering
  \caption{#1}
}{%
  \end{figure}
}

% Semantic markup for key terms
\newcommand{\keyword}[1]{\textbf{#1}}
\newcommand{\code}[1]{\texttt{#1}}
\newcommand{\classname}[1]{\texttt{#1}}
\newcommand{\methodname}[1]{\texttt{#1}}

% Proper quotation marks
\newcommand{\mnemonic}[1]{``#1''}


\title{મોબાઈલ અને વાયરલેસ કમ્યુનિકેશન (4351104) - શિયાળા 2023 ઉકેલ}
\date{December 12, 2023}

\begin{document}
\maketitle

\questionmarks{1(અ)}{3}{અમ્બ્રેલા સેલ આકૃતિ દોરી સમજાવો.}
\begin{solutionbox}
    \begin{center}
        \begin{tikzpicture}[gtu flow]
            \node[gtu block] (umbrella) {અમ્બ્રેલા સેલ ટાવર \\ (વિશાળ કવરેજ વિસ્તાર)};
            \node[gtu block, below left=1.5cm and 1cm of umbrella] (micro1) {માઈક્રો સેલ 1};
            \node[gtu block, below=1.5cm of umbrella] (micro2) {માઈક્રો સેલ 2};
            \node[gtu block, below right=1.5cm and 1cm of umbrella] (micro3) {માઈક્રો સેલ 3};
            
            \node[gtu block, below=1cm of micro1] (users1) {ગીચ વિસ્તારના \\ વપરાશકર્તાઓ};
            \node[gtu block, below=1cm of micro2] (users2) {ગીચ વિસ્તારના \\ વપરાશકર્તાઓ};
            \node[gtu block, below=1cm of micro3] (users3) {ગીચ વિસ્તારના \\ વપરાશકર્તાઓ};
            
            \draw[gtu arrow] (umbrella) -- (micro1);
            \draw[gtu arrow] (umbrella) -- (micro2);
            \draw[gtu arrow] (umbrella) -- (micro3);
            \draw[gtu arrow] (micro1) -- (users1);
            \draw[gtu arrow] (micro2) -- (users2);
            \draw[gtu arrow] (micro3) -- (users3);
        \end{tikzpicture}
    \end{center}

    \begin{itemize}
        \item \textbf{અમ્બ્રેલા સેલ}: નાના સેલોને આવરી લેતા વિશાળ કવરેજ વાળા સેલ.
        \item \textbf{હેતુ}: માઈક્રો/પિકો સેલોમાંથી વધારે ટ્રાફિક સંભાળે છે.
        \item \textbf{કવરેજ}: ઉચ્ચ-ટ્રાફિક વિસ્તારો માટે બેકઅપ કવરેજ પૂરું પાડે છે.
    \end{itemize}
\end{solutionbox}
\begin{mnemonicbox}
    \mnemonic{મારા મોટા છત્ર નીચે}
\end{mnemonicbox}

\questionmarks{1(બ)}{4}{ફુલ ફોર્મ લખો : (i) CCH (ii) TCH (iii) SCH (iv) BCCH}
\begin{solutionbox}
    \begin{tabulary}{\linewidth}{|L|L|L|}
        \hline
        \textbf{સંક્ષેપ} & \textbf{પૂરું નામ} & \textbf{કાર્ય} \\
        \hline
        CCH & Control Channel & નિયંત્રણ માહિતી વહન કરે છે \\
        \hline
        TCH & Traffic Channel & અવાજ/ડેટા ટ્રાફિક વહન કરે છે \\
        \hline
        SCH & Synchronization Channel & સમય સિંક્રોનાઈઝેશન પૂરું પાડે છે \\
        \hline
        BCCH & Broadcast Control Channel & સિસ્ટમ માહિતી પ્રસારિત કરે છે \\
        \hline
    \end{tabulary}
\end{solutionbox}
\begin{mnemonicbox}
    \mnemonic{કંટ્રોલ ટ્રાફિક સિંક બ્રોડકાસ્ટ}
\end{mnemonicbox}

\questionmarks{1(ક)}{7}{સેલ શું છે? અલગ અલગ પ્રકારના સેલ સમજાવો.}
\begin{solutionbox}
    \textbf{સેલ} એ સેલ્યુલર કમ્યુનિકેશનમાં એક બેઝ સ્ટેશન દ્વારા આવરી લેવાતો મૂળભૂત કવરેજ વિસ્તાર છે.

    \begin{tabulary}{\linewidth}{|L|L|L|L|}
        \hline
        \textbf{સેલનો પ્રકાર} & \textbf{કવરેજ} & \textbf{પાવર} & \textbf{ઉપયોગ} \\
        \hline
        \textbf{માક્રો સેલ} & 1-30 km & ઉચ્ચ & ગ્રામ્ય વિસ્તારો \\
        \hline
        \textbf{માઈક્રો સેલ} & 100m-2km & મધ્યમ & શહેરી વિસ્તારો \\
        \hline
        \textbf{પિકો સેલ} & 10-100m & નીચું & ઇન્ડોર કવરેજ \\
        \hline
        \textbf{ફેમ્ટો સેલ} & 10-30m & ખૂબ નીચું & ઘર/ઓફિસ \\
        \hline
    \end{tabulary}

    \begin{center}
        \begin{tikzpicture}[gtu flow]
            \node[gtu block] (macro) {માક્રો સેલ \\ (વિશાળ વિસ્તાર)};
            \node[gtu block, right=1cm of macro] (micro) {માઈક્રો સેલ \\ (શહેર કવરેજ)};
            \node[gtu block, right=1cm of micro] (pico) {પિકો સેલ \\ (બિલ્ડિંગ)};
            \node[gtu block, right=1cm of pico] (femto) {ફેમ્ટો સેલ \\ (રૂમ)};
            
            \draw[gtu arrow] (macro) -- (micro);
            \draw[gtu arrow] (micro) -- (pico);
            \draw[gtu arrow] (pico) -- (femto);
        \end{tikzpicture}
    \end{center}

    \begin{itemize}
        \item \textbf{કાર્ય}: દરેક સેલ મોબાઈલ વપરાશકર્તાઓને વાયરલેસ સેવા પૂરી પાડે છે.
        \item \textbf{આવૃત્તિ પુનઃઉપયોગ}: બિન-સંલગ્ન સેલોમાં સમાન આવૃત્તિઓનો ઉપયોગ.
        \item \textbf{હેન્ડઓફ}: વપરાશકર્તાઓ સેલો વચ્ચે નિરંતર ખસી શકે છે.
    \end{itemize}
\end{solutionbox}
\begin{mnemonicbox}
    \mnemonic{ઘણા મોબાઈલ લોકો કવરેજ શોધે છે}
\end{mnemonicbox}

\questionmarks{1(ક અથવા)}{7}{હેન્ડઓફ શું છે? સોફ્ટ અને હાર્ડ હેન્ડઓફ સમજાવો.}
\begin{solutionbox}
    \textbf{હેન્ડઓફ} એ મોબાઈલ ખસતા સમયે ચાલુ કોલને એક સેલમાંથી બીજા સેલમાં સ્થાનાંતરિત કરવાની પ્રક્રિયા છે.

    \begin{tabulary}{\linewidth}{|L|L|L|}
        \hline
        \textbf{લક્ષણ} & \textbf{હાર્ડ હેન્ડઓફ} & \textbf{સોફ્ટ હેન્ડઓફ} \\
        \hline
        \textbf{કનેક્શન} & તોડ્યા પછી જોડાણ & જોડાણ પછી તોડવું \\
        \hline
        \textbf{ચેનલો} & એક સમયે એક & એકસાથે ઘણા \\
        \hline
        \textbf{ટેક્નોલોજી} & GSM, TDMA & CDMA \\
        \hline
        \textbf{ગુણવત્તા} & થોડી વિક્ષેપ & સરળ સંક્રમણ \\
        \hline
    \end{tabulary}

    \begin{center}
        \begin{tikzpicture}[node distance=1.5cm, auto]
            % Actors
            \node[draw, rectangle] (M) {મોબાઈલ};
            \node[draw, rectangle, right=3cm of M] (BS1) {બેઝ સ્ટેશન 1};
            \node[draw, rectangle, right=3cm of BS1] (BS2) {બેઝ સ્ટેશન 2};

            % Timelines
            \draw[thick] (M) -- ++(0,-6);
            \draw[thick] (BS1) -- ++(0,-6);
            \draw[thick] (BS2) -- ++(0,-6);

            % Hard Handoff
            \node[anchor=west] at (0.2,-0.5) {\textbf{હાર્ડ હેન્ડઓફ}};
            \draw[->] (0,-1) -- (3,-1) node[midway, above] {જોડાયેલ};
            \draw[<-] (0,-1.5) -- (3,-1.5) node[midway, above] {સિગ્નલ નબળું પડે છે};
            \draw[->] (3,-2) -- (6,-2) node[midway, above] {હેન્ડઓફ વિનંતી};
            \draw[->] (0,-2.5) -- (6,-2.5) node[midway, above] {નવું કનેક્શન};

            % Soft Handoff
            \node[anchor=west] at (0.2,-3.5) {\textbf{સોફ્ટ હેન્ડઓફ}};
            \draw[->] (0,-4) -- (3,-4) node[midway, above] {જોડાયેલ};
            \draw[->] (0,-4.5) -- (6,-4.5) node[midway, above] {બેવડું કનેક્શન};
            \draw[->] (0,-5) -- (3,-5) node[midway, above] {નબળા સિગ્નલને છોડો};
        \end{tikzpicture}
    \end{center}

    \begin{itemize}
        \item \textbf{પ્રારંભ}: સિગ્નલ મજબૂતાઈના માપ પર આધારિત.
        \item \textbf{MAHO}: Mobile Assisted Handoff નિર્ણયની ચોકસાઈ સુધારે છે.
    \end{itemize}
\end{solutionbox}
\begin{mnemonicbox}
    \mnemonic{હાર્ડ દુખાવે, સોફ્ટ સરળ}
\end{mnemonicbox}

\questionmarks{2(અ)}{3}{ફુલ ફોર્મ લખો : (i) SIM (ii) LTE (iii) WCDMA}
\begin{solutionbox}
    \begin{tabulary}{\linewidth}{|L|L|L|}
        \hline
        \textbf{સંક્ષેપ} & \textbf{પૂરું નામ} & \textbf{હેતુ} \\
        \hline
        SIM & Subscriber Identity Module & વપરાશકર્તા પ્રમાણીકરણ \\
        \hline
        LTE & Long Term Evolution & 4G ટેક્નોલોજી \\
        \hline
        WCDMA & Wideband Code Division Multiple Access & 3G માનક \\
        \hline
    \end{tabulary}
\end{solutionbox}
\begin{mnemonicbox}
    \mnemonic{સબ્સ્ક્રાઈબરનું લાંબા વાઈડબેન્ડ કનેક્શન}
\end{mnemonicbox}

\questionmarks{2(બ)}{4}{મોબાઈલ હેન્ડસેટની બ્લોક આકૃતિ દોરો.}
\begin{solutionbox}
    \begin{center}
        \begin{tikzpicture}[gtu flow]
            \node[gtu block] (antenna) {એન્ટેના};
            \node[gtu block, below=1cm of antenna] (rf) {RF સેક્શન};
            \node[gtu block, below=1cm of rf] (baseband) {બેઝબેન્ડ પ્રોસેસર};
            
            \node[gtu block, left=1cm of baseband] (display) {ડિસ્પ્લે/કીપેડ};
            \node[gtu block, right=1cm of baseband] (audio) {ઓડિયો સેક્શન};
            \node[gtu block, below right=1cm and 0.1cm of baseband] (memory) {મેમરી};
            
            \node[gtu block, below left=1.5cm and 0.5cm of baseband] (power) {પાવર મેનેજમેન્ટ};
            \node[gtu block, left=1cm of power] (battery) {બૅટરી};

            \draw[gtu arrow] (antenna) -- (rf);
            \draw[gtu arrow] (rf) -- (baseband);
            \draw[gtu arrow, <->] (baseband) -- (display);
            \draw[gtu arrow, <->] (baseband) -- (audio);
            \draw[gtu arrow, <->] (baseband) -- (memory);
            
            \draw[gtu arrow] (battery) -- (power);
            \draw[gtu arrow] (power) -| (rf);
            \draw[gtu arrow] (power) -| (baseband);
            \draw[gtu arrow] (power) -| (display);
        \end{tikzpicture}
    \end{center}

    \begin{itemize}
        \item \textbf{RF સેક્શન}: રેડિયો સિગ્નલ મોકલે/મેળવે છે.
        \item \textbf{બેઝબેન્ડ}: ડિજિટલ સિગ્નલ અને પ્રોટોકોલ પ્રોસેસ કરે છે.
        \item \textbf{ઓડિયો}: અવાજનું ઇનપુટ/આઉટપુટ સંભાળે છે.
        \item \textbf{પાવર મેનેજમેન્ટ}: બૅટરીનો ઉપયોગ કાર્યક્ષમતાથી નિયંત્રિત કરે છે.
    \end{itemize}
\end{solutionbox}
\begin{mnemonicbox}
    \mnemonic{રેડિયો બેઝબેન્ડ ઓડિયો પાવર}
\end{mnemonicbox}

\questionmarks{2(ક)}{7}{GSM આર્કિટેક્ચર આકૃતિ સાથે સમજાવો.}
\begin{solutionbox}
    \begin{center}
        \begin{tikzpicture}[gtu flow]
            \node[gtu block] (ms) {MS};
            
            % BSS
            \node[gtu block, right=1.5cm of ms] (bts) {BTS};
            \node[gtu block, right=1cm of bts] (bsc) {BSC};
            \node[draw, dashed, fit=(bts)(bsc), label={[anchor=south]north:BSS}] (bss) {};
            
            % NSS
            \node[gtu block, right=1cm of bsc] (msc) {MSC};
            \node[gtu block, above=1cm of msc] (hlr) {HLR};
            \node[gtu block, below=1cm of msc] (vlr) {VLR};
            \node[gtu block, right=1cm of hlr] (auc) {AuC};
            \node[draw, dashed, fit=(msc)(hlr)(vlr)(auc), label={[anchor=south]north:NSS}] (nss) {};
            
            \node[gtu block, right=1.5cm of msc] (pstn) {PSTN};

            \draw[gtu arrow] (ms) -- (bts);
            \draw[gtu arrow] (bts) -- (bsc);
            \draw[gtu arrow] (bsc) -- (msc);
            \draw[gtu arrow] (msc) -- (pstn);
            \draw[gtu arrow] (msc) -- (hlr);
            \draw[gtu arrow] (msc) -- (vlr);
            \draw[gtu arrow] (hlr) -- (auc);
        \end{tikzpicture}
    \end{center}

    \begin{tabulary}{\linewidth}{|L|L|}
        \hline
        \textbf{ઘટક} & \textbf{કાર્ય} \\
        \hline
        \textbf{MS} & Mobile Station (હેન્ડસેટ) \\
        \hline
        \textbf{BTS} & Base Transceiver Station \\
        \hline
        \textbf{BSC} & Base Station Controller \\
        \hline
        \textbf{MSC} & Mobile Switching Center \\
        \hline
        \textbf{HLR} & Home Location Register \\
        \hline
        \textbf{VLR} & Visitor Location Register \\
        \hline
    \end{tabulary}

    \begin{itemize}
        \item \textbf{BSS}: Base Station Subsystem રેડિયો ઇન્ટરફેસ સંભાળે છે.
        \item \textbf{NSS}: Network Switching Subsystem કોલો મેનેજ કરે છે.
        \item \textbf{પ્રમાણીકરણ}: AuC સબ્સ્ક્રાઈબરની ઓળખ ચકાસે છે.
    \end{itemize}
\end{solutionbox}
\begin{mnemonicbox}
    \mnemonic{મોબાઈલ બેઝ નેટવર્ક ઘર કોલ કરે છે}
\end{mnemonicbox}

\questionmarks{2(અ અથવા)}{3}{ફુલ ફોર્મ લખો : (i) RSSI (ii) MAHO (iii) NCHO}
\begin{solutionbox}
    \begin{tabulary}{\linewidth}{|L|L|L|}
        \hline
        \textbf{સંક્ષેપ} & \textbf{પૂરું નામ} & \textbf{કાર્ય} \\
        \hline
        RSSI & Received Signal Strength Indicator & સિગ્નલ ગુણવત્તા માપ \\
        \hline
        MAHO & Mobile Assisted Handoff & મોબાઈલ હેન્ડઓફ નિર્ણયમાં મદદ કરે છે \\
        \hline
        NCHO & Network Controlled Handoff & નેટવર્ક હેન્ડઓફ નક્કી કરે છે \\
        \hline
    \end{tabulary}
\end{solutionbox}
\begin{mnemonicbox}
    \mnemonic{પ્રાપ્ત મોબાઈલ નેટવર્ક સિગ્નલો}
\end{mnemonicbox}

\questionmarks{2(બ અથવા)}{4}{બેઝબેન્ડ સેક્શનની બ્લોક આકૃતિ દોરો.}
\begin{solutionbox}
    \begin{center}
        \begin{tikzpicture}[gtu flow]
            \node[gtu block] (adc) {ADC/DAC};
            \node[gtu block, right=1cm of adc] (dsp) {DSP};
            \node[gtu block, right=1cm of dsp] (chan) {ચેનલ \\ કોડેક};
            \node[gtu block, right=1cm of chan] (speech) {સ્પીચ \\ કોડેક};
            \node[gtu block, below=1cm of speech] (audio) {ઓડિયો ઇન્ટરફેસ};
            
            \node[gtu block, below=1cm of dsp] (proto) {પ્રોટોકોલ સ્ટેક};
            \node[gtu block, right=1cm of proto] (ctrl) {કંટ્રોલ ઇન્ટરફેસ};

            \draw[gtu arrow] (adc) -- (dsp);
            \draw[gtu arrow] (dsp) -- (chan);
            \draw[gtu arrow] (chan) -- (speech);
            \draw[gtu arrow] (speech) -- (audio);
            \draw[gtu arrow] (dsp) -- (proto);
            \draw[gtu arrow] (proto) -- (ctrl);
        \end{tikzpicture}
    \end{center}

    \begin{itemize}
        \item \textbf{ADC/DAC}: Analog to Digital કન્વર્ઝન.
        \item \textbf{DSP}: Digital Signal Processor.
        \item \textbf{ચેનલ કોડેક}: ભૂલ સુધારણા કોડિંગ.
        \item \textbf{સ્પીચ કોડેક}: અવાજ સંકોચન/વિસ્તારણ.
    \end{itemize}
\end{solutionbox}
\begin{mnemonicbox}
    \mnemonic{એનાલોગ ડિજિટલ સ્પીચ પ્રોટોકોલ}
\end{mnemonicbox}

\questionmarks{2(ક અથવા)}{7}{GSM સિગ્નલ પ્રોસેસિંગ આકૃતિ સાથે સમજાવો.}
\begin{solutionbox}
    \begin{center}
        \begin{tikzpicture}[gtu flow]
            \node[gtu block] (speech) {અવાજ};
            \node[gtu block, right=0.8cm of speech] (scodec) {સ્પીચ \\ કોડેક};
            \node[gtu block, right=0.8cm of scodec] (ccodec) {ચેનલ \\ કોડેક};
            \node[gtu block, below=1cm of ccodec] (inter) {ઇન્ટરલીવિંગ};
            \node[gtu block, left=0.8cm of inter] (burst) {બર્સ્ટ \\ ફોર્મેટર};
            \node[gtu block, left=0.8cm of burst] (gmsk) {GMSK \\ મોડ્યુલેટર};
            \node[gtu block, left=0.8cm of gmsk] (rf) {RF ટ્રાન્સમિટર};

            \draw[gtu arrow] (speech) -- (scodec);
            \draw[gtu arrow] (scodec) -- (ccodec);
            \draw[gtu arrow] (ccodec) -- (inter);
            \draw[gtu arrow] (inter) -- (burst);
            \draw[gtu arrow] (burst) -- (gmsk);
            \draw[gtu arrow] (gmsk) -- (rf);
        \end{tikzpicture}
    \end{center}

    \begin{tabulary}{\linewidth}{|L|L|L|}
        \hline
        \textbf{તબક્કો} & \textbf{કાર્ય} & \textbf{હેતુ} \\
        \hline
        \textbf{સ્પીચ કોડેક} & અવાજને 13 kbps માં સંકોચે છે & બેન્ડવિડ્થ કાર્યક્ષમતા \\
        \hline
        \textbf{ચેનલ કોડેક} & ભૂલ સુધારણા ઉમેરે છે & સિગ્નલ વિશ્વસનીયતા \\
        \hline
        \textbf{ઇન્ટરલીવિંગ} & બર્સ્ટ ભૂલો વિતરિત કરે છે & ભૂલ સુરક્ષા \\
        \hline
        \textbf{GMSK} & Gaussian MSK મોડ્યુલેશન & સ્પેક્ટ્રલ કાર્યક્ષમતા \\
        \hline
    \end{tabulary}

    \begin{itemize}
        \item \textbf{પ્રોસેસિંગ રેટ}: 270.833 kbps કુલ બિટ રેટ.
        \item \textbf{ફ્રેમ સ્ટ્રક્ચર}: TDMA ફ્રેમ દીઠ 8 ટાઈમ સ્લોટ.
        \item \textbf{ફ્રીક્વન્સી હોપિંગ}: પ્રતિ સેકન્ડ 217 હોપ્સ.
    \end{itemize}
\end{solutionbox}
\begin{mnemonicbox}
    \mnemonic{સ્પીચ ચેનલ ઇન્ટરલીવ મોડ્યુલેટેડ રેડિયો}
\end{mnemonicbox}

\questionmarks{3(અ)}{3}{સેલ સ્પ્લિટિંગ સમજાવો.}
\begin{solutionbox}
    સેલ સ્પ્લિટિંગ ગીચતાવાળા સેલોને નાના સેલોમાં વિભાજિત કરીને ક્ષમતા વધારે છે.
    \begin{itemize}
        \item \textbf{પ્રક્રિયા}: ઉચ્ચ-પાવર સેલને ઘણા નીચા-પાવર સેલો સાથે બદલવું.
        \item \textbf{ફાયદો}: આવૃત્તિ પુનઃઉપયોગ દ્વારા સિસ્ટમ ક્ષમતા વધારે છે.
        \item \textbf{અમલીકરણ}: એન્ટેનાની ઊંચાઈ અને ટ્રાન્સમિટ પાવર ઘટાડવું.
    \end{itemize}
\end{solutionbox}
\begin{mnemonicbox}
    \mnemonic{સ્પ્લિટ નાના સેલો}
\end{mnemonicbox}

\questionmarks{3(બ)}{4}{મોબાઈલ હેન્ડસેટમાં વપરાતી Li-Ion બૅટરી વિશે તેના ફાયદા અને નુકસાનો સાથે સમજાવો.}
\begin{solutionbox}
    \begin{tabulary}{\linewidth}{|L|L|}
        \hline
        \textbf{ફાયદા} & \textbf{નુકસાનો} \\
        \hline
        \textbf{ઉચ્ચ એનર્જી ડેન્સિટી} & \textbf{સુરક્ષાની ચિંતાઓ} \\
        \hline
        \textbf{મેમરી ઇફેક્ટ નથી} & \textbf{સમય સાથે બગાડ} \\
        \hline
        \textbf{નીચું સેલ્ફ-ડિસ્ચાર્જ} & \textbf{તાપમાન સંવેદનશીલ} \\
        \hline
        \textbf{હળવું વજન} & \textbf{મોંઘું} \\
        \hline
    \end{tabulary}

    \begin{itemize}
        \item \textbf{કેમિસ્ટ્રી}: લિથિયમ આયન ઇલેક્ટ્રોડ વચ્ચે ફરે છે.
        \item \textbf{વોલ્ટેજ}: પ્રતિ સેલ 3.7V નોમિનલ.
        \item \textbf{ક્ષમતા}: mAh (મિલિએમ્પિયર-કલાક) માં માપવામાં આવે છે.
    \end{itemize}
\end{solutionbox}
\begin{mnemonicbox}
    \mnemonic{લાઇટ આયન એનર્જી સેફ્ટી}
\end{mnemonicbox}

\questionmarks{3(ક)}{7}{GPRS સમજાવો.}
\begin{solutionbox}
    \textbf{GPRS} (General Packet Radio Service) GSM પર પેકેટ-સ્વિચ્ડ ડેટા સેવા પૂરી પાડે છે.

    \begin{tabulary}{\linewidth}{|L|L|}
        \hline
        \textbf{લક્ષણ} & \textbf{સ્પેસિફિકેશન} \\
        \hline
        \textbf{ડેટા રેટ} & 171.2 kbps સુધી \\
        \hline
        \textbf{ટેક્નોલોજી} & પેકેટ સ્વિચિંગ \\
        \hline
        \textbf{ચેનલો} & બહુવિધ ટાઈમ સ્લોટનો ઉપયોગ \\
        \hline
        \textbf{બિલિંગ} & ડેટા વોલ્યુમ પર આધારિત \\
        \hline
    \end{tabulary}

    \begin{center}
        \begin{tikzpicture}[gtu flow]
            \node[gtu block] (mobile) {મોબાઈલ};
            \node[gtu block, right=1cm of mobile] (bss) {BSS};
            \node[gtu block, right=1cm of bss] (pcu) {PCU};
            \node[gtu block, below=1cm of pcu] (sgsn) {SGSN};
            \node[gtu block, left=1cm of sgsn] (ggsn) {GGSN};
            \node[gtu block, left=1cm of ggsn] (net) {ઇન્ટરનેટ};

            \draw[gtu arrow] (mobile) -- (bss);
            \draw[gtu arrow] (bss) -- (pcu);
            \draw[gtu arrow] (pcu) -- (sgsn);
            \draw[gtu arrow] (sgsn) -- (ggsn);
            \draw[gtu arrow] (ggsn) -- (net);
        \end{tikzpicture}
    \end{center}

    \begin{itemize}
        \item \textbf{PCU}: Packet Control Unit પેકેટ ડેટા મેનેજ કરે છે.
        \item \textbf{SGSN}: Serving GPRS Support Node.
        \item \textbf{GGSN}: Gateway GPRS Support Node.
        \item \textbf{ક્લાસ}: વિવિધ સ્પીડ/સ્લોટ કોમ્બિનેશન સાથે ક્લાસ 1-12.
    \end{itemize}
\end{solutionbox}
\begin{mnemonicbox}
    \mnemonic{જનરલ પેકેટ રેડિયો સર્વિસ}
\end{mnemonicbox}

\questionmarks{3(અ અથવા)}{3}{સેલ સેક્ટરિંગ સમજાવો.}
\begin{solutionbox}
    સેલ સેક્ટરિંગ ડાયરેક્શનલ એન્ટેના વાપરીને ઓમ્નિડાયરેક્શનલ સેલને સેક્ટરોમાં વિભાજિત કરે છે.
    \begin{itemize}
        \item \textbf{સામાન્ય}: 3-સેક્ટર (120$^{\circ}$) અથવા 6-સેક્ટર (60$^{\circ}$) કોન્ફિગરેશન.
        \item \textbf{ફાયદો}: કો-ચેનલ ઇન્ટરફેરન્સ ઘટાડે છે.
        \item \textbf{અમલીકરણ}: સમાન સાઇટ પર ડાયરેક્શનલ એન્ટેના.
    \end{itemize}
\end{solutionbox}
\begin{mnemonicbox}
    \mnemonic{સેક્ટર ઇન્ટરફેરન્સ ઘટાડે છે}
\end{mnemonicbox}

\questionmarks{3(બ અથવા)}{4}{મોબાઈલ હેન્ડસેટમાં વપરાતી Li-Po બૅટરી વિશે તેના ફાયદા અને નુકસાનો સાથે સમજાવો.}
\begin{solutionbox}
    \begin{tabulary}{\linewidth}{|L|L|}
        \hline
        \textbf{ફાયદા} & \textbf{નુકસાનો} \\
        \hline
        \textbf{લવચીક આકાર} & \textbf{નીચી એનર્જી ડેન્સિટી} \\
        \hline
        \textbf{અતિ-પાતળી ડિઝાઇન} & \textbf{ઓછું જીવનકાળ} \\
        \hline
        \textbf{હળવું વજન} & \textbf{સુરક્ષા જોખમો} \\
        \hline
        \textbf{મેમરી ઇફેક્ટ નથી} & \textbf{વધુ કિંમત} \\
        \hline
    \end{tabulary}

    \begin{itemize}
        \item \textbf{ટેક્નોલોજી}: લિથિયમ પોલિમર ઇલેક્ટ્રોલાઇટ.
        \item \textbf{ફોર્મ ફેક્ટર}: વિવિધ આકારોમાં મોલ્ડ કરી શકાય છે.
        \item \textbf{વોલ્ટેજ}: પ્રતિ સેલ 3.7V નોમિનલ.
    \end{itemize}
\end{solutionbox}
\begin{mnemonicbox}
    \mnemonic{પોલિમર લવચીક પાતળું હળવું}
\end{mnemonicbox}

\questionmarks{3(ક અથવા)}{7}{EDGE સમજાવો.}
\begin{solutionbox}
    \textbf{EDGE} (Enhanced Data rates for GSM Evolution) GSM ડેટા રેટ સુધારે છે.

    \begin{tabulary}{\linewidth}{|L|L|L|}
        \hline
        \textbf{પેરામીટર} & \textbf{GSM} & \textbf{EDGE} \\
        \hline
        \textbf{મોડ્યુલેશન} & GMSK & 8-PSK \\
        \hline
        \textbf{ડેટા રેટ} & 9.6 kbps & 384 kbps સુધી \\
        \hline
        \textbf{ભૂલ સુધારણા} & મૂળભૂત & અદ્યતન \\
        \hline
        \textbf{સ્પેક્ટ્રમ} & GSM જેવું જ & GSM જેવું જ \\
        \hline
    \end{tabulary}

    \begin{center}
        \begin{tikzpicture}[gtu flow]
            \node[gtu block] (data) {ડેટા};
            \node[gtu block, right=0.8cm of data] (adapt) {એડાપ્ટીવ \\ કોડિંગ};
            \node[gtu block, right=0.8cm of adapt] (psk) {8-PSK \\ મોડ્યુલેશન};
            \node[gtu block, below=1cm of psk] (link) {લિંક \\ એડાપ્ટેશન};
            \node[gtu block, left=0.8cm of link] (rx) {વધારેલ \\ રિસેપ્શન};

            \draw[gtu arrow] (data) -- (adapt);
            \draw[gtu arrow] (adapt) -- (psk);
            \draw[gtu arrow] (psk) -- (link);
            \draw[gtu arrow] (link) -- (rx);
        \end{tikzpicture}
    \end{center}

    \begin{itemize}
        \item \textbf{8-PSK}: 8-Phase Shift Keying પ્રતિ સિમ્બોલ 3 બિટ્સ પૂરી પાડે છે.
        \item \textbf{લિંક એડાપ્ટેશન}: ચેનલ ગુણવત્તા આધારે કોડિંગ સ્કીમ એડજસ્ટ કરે છે.
        \item \textbf{ઇન્ક્રિમેન્ટલ રિડન્ડન્સી}: ભૂલ સુધારણા કાર્યક્ષમતા સુધારે છે.
    \end{itemize}
\end{solutionbox}
\begin{mnemonicbox}
    \mnemonic{એન્હાન્સ્ડ ડેટા GSM ઇવોલ્યુશન}
\end{mnemonicbox}

\questionmarks{4(અ)}{3}{DSSS ટ્રાન્સમિટર અને રિસીવરની બ્લોક આકૃતિ દોરો.}
\begin{solutionbox}
    \begin{center}
        \begin{tikzpicture}[gtu flow]
            % Transmitter
            \node[gtu block] (data) {ડેટા};
            \node[gtu block, right=1cm of data] (spread) {સ્પ્રેડર};
            \node[gtu block, right=1cm of spread] (mod) {મોડ્યુલેટર};
            \node[gtu block, right=1cm of mod] (rf) {RF આઉટપુટ};
            \node[gtu block, below=0.8cm of spread] (pn) {PN કોડ};

            \draw[gtu arrow] (data) -- (spread);
            \draw[gtu arrow] (spread) -- (mod);
            \draw[gtu arrow] (mod) -- (rf);
            \draw[gtu arrow] (pn) -- (spread);
            
            \node[above=0.2cm of spread] {Tx};
        \end{tikzpicture}
        
        \vspace{0.5cm}

        \begin{tikzpicture}[gtu flow]
            % Receiver
            \node[gtu block] (rfin) {RF ઇનપુટ};
            \node[gtu block, right=1cm of rfin] (demod) {ડિમોડ્યુલેટર};
            \node[gtu block, right=1cm of demod] (despread) {ડિસ્પ્રેડર};
            \node[gtu block, right=1cm of despread] (dout) {ડેટા આઉટપુટ};
            \node[gtu block, below=0.8cm of despread] (pnrx) {PN કોડ};

            \draw[gtu arrow] (rfin) -- (demod);
            \draw[gtu arrow] (demod) -- (despread);
            \draw[gtu arrow] (despread) -- (dout);
            \draw[gtu arrow] (pnrx) -- (despread);

            \node[above=0.2cm of despread] {Rx};
        \end{tikzpicture}
    \end{center}

    \begin{itemize}
        \item \textbf{સ્પ્રેડર}: ડેટાને PN સિક્વન્સ સાથે ગુણાકાર કરે છે.
        \item \textbf{ડિસ્પ્રેડર}: પ્રાપ્ત સિગ્નલને સમાન PN કોડ સાથે કોરિલેટ કરે છે.
        \item \textbf{પ્રોસેસિંગ ગેઇન}: સ્પ્રેડ અને મૂળ બેન્ડવિડ્થનો ગુણોત્તર.
    \end{itemize}
\end{solutionbox}
\begin{mnemonicbox}
    \mnemonic{ડાયરેક્ટ સિક્વન્સ સ્પ્રેડ સ્પેક્ટ્રમ}
\end{mnemonicbox}

\questionmarks{4(બ)}{4}{CDMA અને GSM વચ્ચે તફાવત આપો.}
\begin{solutionbox}
    \begin{tabulary}{\linewidth}{|L|L|L|}
        \hline
        \textbf{પેરામીટર} & \textbf{CDMA} & \textbf{GSM} \\
        \hline
        \textbf{મલ્ટિપલ એક્સેસ} & કોડ ડિવિઝન & ટાઈમ ડિવિઝન \\
        \hline
        \textbf{ક્ષમતા} & વધુ (સોફ્ટ ક્ષમતા) & નિયત ક્ષમતા \\
        \hline
        \textbf{હેન્ડઓફ} & સોફ્ટ હેન્ડઓફ & હાર્ડ હેન્ડઓફ \\
        \hline
        \textbf{પાવર કંટ્રોલ} & મહત્વપૂર્ણ & ઓછું મહત્વપૂર્ણ \\
        \hline
        \textbf{ફ્રીક્વન્સી પ્લાનિંગ} & જરૂરી નથી & જરૂરી \\
        \hline
        \textbf{અવાજની ગુણવત્તા} & વધુ સારી & સારી \\
        \hline
    \end{tabulary}
\end{solutionbox}
\begin{mnemonicbox}
    \mnemonic{કોડ ડિવિઝન વિ ટાઈમ ડિવિઝન}
\end{mnemonicbox}

\questionmarks{4(ક)}{7}{સ્પ્રેડ સ્પેક્ટ્રમનો ખ્યાલ તેના ઉપયોગો સાથે સમજાવો.}
\begin{solutionbox}
    \textbf{સ્પ્રેડ સ્પેક્ટ્રમ} સિગ્નલની બેન્ડવિડ્થને ડેટા ટ્રાન્સમિશન માટે જરૂરી કરતાં ઘણી વિશાળ ફેલાવે છે.

    \begin{center}
        \begin{tikzpicture}[gtu flow]
            \node[gtu block] (narrow) {નેરોબેન્ડ \\ સિગ્નલ};
            \node[gtu block, right=1cm of narrow] (code) {સ્પ્રેડિંગ \\ કોડ};
            \node[gtu block, right=1cm of code] (wide) {વાઈડબેન્ડ \\ સિગ્નલ};
            \node[gtu block, below=1cm of wide] (tx) {ટ્રાન્સમિશન};
            \node[gtu block, left=1cm of tx] (despread) {ડિસ્પ્રેડિંગ};
            \node[gtu block, left=1cm of despread] (orig) {મૂળ \\ સિગ્નલ};

            \draw[gtu arrow] (narrow) -- (code);
            \draw[gtu arrow] (code) -- (wide);
            \draw[gtu arrow] (wide) -- (tx);
            \draw[gtu arrow] (tx) -- (despread);
            \draw[gtu arrow] (despread) -- (orig);
        \end{tikzpicture}
    \end{center}

    \begin{tabulary}{\linewidth}{|L|L|L|}
        \hline
        \textbf{પ્રકાર} & \textbf{પદ્ધતિ} & \textbf{એપ્લિકેશન} \\
        \hline
        \textbf{DSSS} & PN સિક્વન્સ ગુણાકાર & CDMA, WiFi \\
        \hline
        \textbf{FHSS} & ફ્રીક્વન્સી હોપિંગ & Bluetooth \\
        \hline
        \textbf{THSS} & ટાઈમ હોપિંગ & UWB સિસ્ટમો \\
        \hline
    \end{tabulary}

    \textbf{ફાયદા}:
    \begin{itemize}
        \item \textbf{એન્ટી-જેમિંગ}: ઇન્ટરફેરન્સ સામે પ્રતિકાર.
        \item \textbf{લો પાવર ડેન્સિટી}: શોધવામાં મુશ્કેલ.
        \item \textbf{મલ્ટિપલ એક્સેસ}: ઘણા વપરાશકર્તાઓ સ્પેક્ટ્રમ શેર કરે છે.
        \item \textbf{મલ્ટિપાથ રેઝિસ્ટન્સ}: વિલંબિત સિગ્નલો રિઝોલ્વ કરે છે.
    \end{itemize}

    \textbf{એપ્લિકેશનો}: GPS, WiFi, Bluetooth, લશ્કરી કમ્યુનિકેશન.
\end{solutionbox}
\begin{mnemonicbox}
    \mnemonic{સ્પ્રેડ સિગ્નલ સ્પેક્ટ્રમ સિક્યુરિટી}
\end{mnemonicbox}

\questionmarks{4(અ અથવા)}{3}{FHSS ટ્રાન્સમિટરની બ્લોક આકૃતિ દોરો.}
\begin{solutionbox}
    \begin{center}
        \begin{tikzpicture}[gtu flow]
            \node[gtu block] (data) {ડેટા};
            \node[gtu block, right=1cm of data] (mod) {મોડ્યુલેટર};
            \node[gtu block, right=1cm of mod] (synth) {ફ્રીક્વન્સી \\ સિન્થેસાઇઝર};
            \node[gtu block, right=1cm of synth] (rf) {RF આઉટપુટ};
            \node[gtu block, below=1cm of synth] (gen) {હોપિંગ સિક્વન્સ \\ જનરેટર};

            \draw[gtu arrow] (data) -- (mod);
            \draw[gtu arrow] (mod) -- (synth);
            \draw[gtu arrow] (synth) -- (rf);
            \draw[gtu arrow] (gen) -- (synth);
        \end{tikzpicture}
    \end{center}

    \begin{itemize}
        \item \textbf{ફ્રીક્વન્સી સિન્થેસાઇઝર}: કેરિયર ફ્રીક્વન્સી ઝડપથી બદલે છે.
        \item \textbf{હોપિંગ સિક્વન્સ}: સ્યુડો-રેન્ડમ ફ્રીક્વન્સી પેટર્ન.
        \item \textbf{ડ્વેલ ટાઈમ}: દરેક ફ્રીક્વન્સી પર વિતાવેલો સમય.
    \end{itemize}
\end{solutionbox}
\begin{mnemonicbox}
    \mnemonic{ફ્રીક્વન્સી હોપિંગ સ્પ્રેડ સ્પેક્ટ્રમ}
\end{mnemonicbox}

\questionmarks{4(બ અથવા)}{4}{CDMA માં કોલ પ્રોસેસિંગ સમજાવો.}
\begin{solutionbox}
    \begin{tabulary}{\linewidth}{|L|L|L|}
        \hline
        \textbf{તબક્કો} & \textbf{પ્રક્રિયા} & \textbf{વર્ણન} \\
        \hline
        \textbf{સિસ્ટમ એક્સેસ} & પાવર કંટ્રોલ & મોબાઈલ પાવર એડજસ્ટ કરે છે \\
        \hline
        \textbf{કોલ સેટઅપ} & ચેનલ અસાઈનમેન્ટ & વોલ્શ કોડ અસાઈન કરો \\
        \hline
        \textbf{ટ્રાફિક} & સોફ્ટ હેન્ડઓફ & બહુવિધ બેઝ સ્ટેશનો \\
        \hline
        \textbf{કોલ રિલીઝ} & પાવર ડાઉન & ક્રમશઃ પાવર ઘટાડો \\
        \hline
    \end{tabulary}

    \begin{itemize}
        \item \textbf{રેક રિસીવર}: મલ્ટિપાથ સિગ્નલો કમ્બાઇન કરે છે.
        \item \textbf{પાવર કંટ્રોલ}: પ્રતિ સેકન્ડ 800 વખત.
        \item \textbf{સોફ્ટ કેપેસિટી}: લોડ સાથે ક્રમશઃ બગડે છે.
    \end{itemize}
\end{solutionbox}
\begin{mnemonicbox}
    \mnemonic{કોડ ડિવિઝન મલ્ટિપલ એક્સેસ}
\end{mnemonicbox}

\questionmarks{4(ક અથવા)}{7}{HSDPA સમજાવો.}
\begin{solutionbox}
    \textbf{HSDPA} (High Speed Downlink Packet Access) WCDMA ડાઉનલિંક ડેટા રેટ વધારે છે.

    \begin{tabulary}{\linewidth}{|L|L|}
        \hline
        \textbf{લક્ષણ} & \textbf{સુધારો} \\
        \hline
        \textbf{ડેટા રેટ} & 14.4 Mbps સુધી \\
        \hline
        \textbf{મોડ્યુલેશન} & 16-QAM \\
        \hline
        \textbf{HARQ} & હાઇબ્રિડ ARQ \\
        \hline
        \textbf{ફાસ્ટ શેડ્યુલિંગ} & 2ms TTI \\
        \hline
    \end{tabulary}

    \begin{center}
        \begin{tikzpicture}[gtu flow]
            \node[gtu block] (nodeb) {NodeB};
            \node[gtu block, right=1cm of nodeb] (chan) {HS-DSCH};
            \node[gtu block, right=1cm of chan] (qam) {16-QAM};
            \node[gtu block, below=1cm of qam] (harq) {HARQ};
            \node[gtu block, left=1cm of harq] (mobile) {મોબાઈલ};

            \draw[gtu arrow] (nodeb) -- (chan);
            \draw[gtu arrow] (chan) -- (qam);
            \draw[gtu arrow] (qam) -- (harq);
            \draw[gtu arrow] (harq) -- (mobile);
        \end{tikzpicture}
    \end{center}

    \begin{itemize}
        \item \textbf{HS-DSCH}: High Speed Downlink Shared Channel.
        \item \textbf{AMC}: Adaptive Modulation and Coding.
        \item \textbf{ફાસ્ટ સેલ સિલેક્શન}: સેલ એજ પર્ફોર્મન્સ સુધારે છે.
        \item \textbf{MIMO}: બહુવિધ એન્ટેના કોન્ફિગરેશન શક્ય.
    \end{itemize}
\end{solutionbox}
\begin{mnemonicbox}
    \mnemonic{હાઇ સ્પીડ ડાઉનલિંક પેકેટ એક્સેસ}
\end{mnemonicbox}

\questionmarks{5(અ)}{3}{LTE ના સ્પેસિફિકેશન જણાવો.}
\begin{solutionbox}
    \begin{tabulary}{\linewidth}{|L|L|}
        \hline
        \textbf{પેરામીટર} & \textbf{સ્પેસિફિકેશન} \\
        \hline
        \textbf{પીક ડેટા રેટ} & 300 Mbps DL, 75 Mbps UL \\
        \hline
        \textbf{બેન્ડવિડ્થ} & 1.4 થી 20 MHz \\
        \hline
        \textbf{લેટન્સી} & <10ms યુઝર પ્લેન \\
        \hline
        \textbf{મોબિલિટી} & 350 km/h સુધી \\
        \hline
        \textbf{સ્પેક્ટ્રમ કાર્યક્ષમતા} & 3G કરતાં 3-4x વધારે સારી \\
        \hline
    \end{tabulary}
\end{solutionbox}
\begin{mnemonicbox}
    \mnemonic{લોંગ ટર્મ ઇવોલ્યુશન સ્પેસિફિકેશનો}
\end{mnemonicbox}

\questionmarks{5(બ)}{4}{OFDM રિસીવર બ્લોક આકૃતિ દોરી સમજાવો.}
\begin{solutionbox}
    \begin{center}
        \begin{tikzpicture}[gtu flow]
            \node[gtu block] (rf) {RF ઇનપુટ};
            \node[gtu block, right=0.8cm of rf] (adc) {ADC};
            \node[gtu block, right=0.8cm of adc] (cp) {CP દૂર કરો};
            \node[gtu block, below=1cm of cp] (fft) {FFT};
            \node[gtu block, left=0.8cm of fft] (demod) {ડિમોડ્યુલેટર};
            \node[gtu block, left=0.8cm of demod] (data) {ડેટા આઉટપુટ};

            \draw[gtu arrow] (rf) -- (adc);
            \draw[gtu arrow] (adc) -- (cp);
            \draw[gtu arrow] (cp) -- (fft);
            \draw[gtu arrow] (fft) -- (demod);
            \draw[gtu arrow] (demod) -- (data);
        \end{tikzpicture}
    \end{center}

    \begin{itemize}
        \item \textbf{FFT}: Fast Fourier Transform સમય ડોમેઇનને ફ્રીક્વન્સી ડોમેઇનમાં કન્વર્ટ કરે છે.
        \item \textbf{સાયક્લિક પ્રીફિક્સ}: ઇન્ટર-સિમ્બોલ ઇન્ટરફેરન્સ સામે રક્ષણ કરે છે.
        \item \textbf{સબકેરિયર્સ}: બહુવિધ ફ્રીક્વન્સીઓ પર સમાંતર ટ્રાન્સમિશન.
        \item \textbf{ડિમોડ્યુલેશન}: સબકેરિયર દીઠ QPSK/16QAM/64QAM.
    \end{itemize}
\end{solutionbox}
\begin{mnemonicbox}
    \mnemonic{ઓર્થોગોનલ ફ્રીક્વન્સી ડિવિઝન મલ્ટિપ્લેક્સિંગ}
\end{mnemonicbox}

\questionmarks{5(ક)}{7}{બ્લુટૂથ ટેક્નોલોજી તેના ઉપયોગો સાથે સમજાવો.}
\begin{solutionbox}
    \textbf{બ્લુટૂથ} પર્સનલ એરિયા નેટવર્ક માટે ટૂંકી રેન્જની વાયરલેસ કમ્યુનિકેશન ટેક્નોલોજી છે.

    \begin{tabulary}{\linewidth}{|L|L|}
        \hline
        \textbf{પેરામીટર} & \textbf{સ્પેસિફિકેશન} \\
        \hline
        \textbf{રેન્જ} & 10m (ક્લાસ 2) \\
        \hline
        \textbf{ફ્રીક્વન્સી} & 2.4 GHz ISM બેન્ડ \\
        \hline
        \textbf{ડેટા રેટ} & 3 Mbps સુધી \\
        \hline
        \textbf{ટોપોલોજી} & પિકોનેટ (8 ડિવાઇસો) \\
        \hline
    \end{tabulary}

    \begin{center}
        \begin{tikzpicture}[gtu flow]
            \node[gtu block] (master) {માસ્ટર ડિવાઇસ};
            \node[gtu block, below left=1.2cm and 1cm of master] (s1) {સ્લેવ 1};
            \node[gtu block, below=1.2cm of master] (s2) {સ્લેવ 2};
            \node[gtu block, below right=1.2cm and 1cm of master] (s3) {સ્લેવ 3};
            
            \node[gtu block, above right=1cm and 0.5cm of master] (scatter) {સ્કેટરનેટ};
            \node[gtu block, right=1cm of scatter] (pico2) {બીજું \\ પિકોનેટ};

            \draw[gtu arrow] (master) -- (s1);
            \draw[gtu arrow] (master) -- (s2);
            \draw[gtu arrow] (master) -- (s3);
            \draw[gtu arrow, dashed] (master) -- (scatter);
            \draw[gtu arrow, dashed] (scatter) -- (pico2);
        \end{tikzpicture}
    \end{center}

    \textbf{પ્રોટોકોલ સ્ટેક}:
    \begin{itemize}
        \item \textbf{RF લેયર}: ફિઝિકલ રેડિયો ઇન્ટરફેસ.
        \item \textbf{બેઝબેન્ડ}: મીડિયમ એક્સેસ કંટ્રોલ.
        \item \textbf{L2CAP}: લોજિકલ લિંક કંટ્રોલ.
        \item \textbf{એપ્લિકેશનો}: વિવિધ પ્રોફાઇલ્સ (A2DP, HID, વગેરે).
    \end{itemize}

    \textbf{ઉપયોગો}:
    \begin{itemize}
        \item ઓડિયો ડિવાઇસો (હેડફોન્સ, સ્પીકર્સ)
        \item ડિવાઇસો વચ્ચે ફાઇલ ટ્રાન્સફર
        \item ઇનપુટ ડિવાઇસો (કીબોર્ડ, માઉસ)
        \item હેલ્થ મોનિટરિંગ ડિવાઇસો
        \item સ્માર્ટ હોમ ઓટોમેશન
    \end{itemize}
\end{solutionbox}
\begin{mnemonicbox}
    \mnemonic{બ્લુ ટૂથ પર્સનલ એરિયા નેટવર્ક}
\end{mnemonicbox}

\questionmarks{5(અ અથવા)}{3}{5G ટેક્નોલોજીના ફાયદા જણાવો.}
\begin{solutionbox}
    \begin{tabulary}{\linewidth}{|L|L|}
        \hline
        \textbf{ફાયદો} & \textbf{લાભ} \\
        \hline
        \textbf{અલ્ટ્રા-લો લેટન્સી} & <1ms પ્રતિક્રિયા સમય \\
        \hline
        \textbf{ઉચ્ચ ડેટા રેટ} & 10 Gbps સુધી \\
        \hline
        \textbf{મેસિવ કનેક્ટિવિટી} & 1M ડિવાઇસો/km\textsuperscript{2} \\
        \hline
        \textbf{નેટવર્ક સ્લાઇસિંગ} & કસ્ટમાઇઝ્ડ સેવાઓ \\
        \hline
        \textbf{એનર્જી કાર્યક્ષમતા} & 90\% વધુ કાર્યક્ષમ \\
        \hline
    \end{tabulary}
\end{solutionbox}
\begin{mnemonicbox}
    \mnemonic{પાંચમી જનરેશનના ફાયદા}
\end{mnemonicbox}

\questionmarks{5(બ અથવા)}{4}{OFDM ટ્રાન્સમિટર બ્લોક આકૃતિ દોરી સમજાવો.}
\begin{solutionbox}
    \begin{center}
        \begin{tikzpicture}[gtu flow]
            \node[gtu block] (data) {ડેટા ઇનપુટ};
            \node[gtu block, right=0.8cm of data] (mod) {મોડ્યુલેટર};
            \node[gtu block, right=0.8cm of mod] (ifft) {IFFT};
            \node[gtu block, below=1cm of ifft] (cp) {CP ઉમેરો};
            \node[gtu block, left=0.8cm of cp] (dac) {DAC};
            \node[gtu block, left=0.8cm of dac] (rf) {RF આઉટપુટ};

            \draw[gtu arrow] (data) -- (mod);
            \draw[gtu arrow] (mod) -- (ifft);
            \draw[gtu arrow] (ifft) -- (cp);
            \draw[gtu arrow] (cp) -- (dac);
            \draw[gtu arrow] (dac) -- (rf);
        \end{tikzpicture}
    \end{center}

    \begin{itemize}
        \item \textbf{મોડ્યુલેશન}: બિટ્સને સિમ્બોલ્સમાં મેપ કરે છે (QPSK/QAM).
        \item \textbf{IFFT}: ઇન્વર્સ FFT ફ્રીક્વન્સીને ટાઈમ ડોમેઇનમાં કન્વર્ટ કરે છે.
        \item \textbf{સાયક્લિક પ્રીફિક્સ}: છેવટના સેમ્પલ્સને શરૂઆતમાં કૉપિ કરે છે.
        \item \textbf{DAC}: ટ્રાન્સમિશન માટે ડિજિટલ ટુ એનાલોગ કન્વર્ટર.
    \end{itemize}
\end{solutionbox}
\begin{mnemonicbox}
    \mnemonic{ઓર્થોગોનલ ફ્રીક્વન્સી ડિવિઝન મલ્ટિપ્લેક્સિંગ ટ્રાન્સમિટર}
\end{mnemonicbox}

\questionmarks{5(ક અથવા)}{7}{Zigbee ટેક્નોલોજી તેના ઉપયોગો સાથે સમજાવો.}
\begin{solutionbox}
    \textbf{Zigbee} IEEE 802.15.4 પર આધારિત લો-પાવર વાયરલેસ મેશ નેટવર્કિંગ સ્ટાન્ડર્ડ છે.

    \begin{tabulary}{\linewidth}{|L|L|}
        \hline
        \textbf{પેરામીટર} & \textbf{સ્પેસિફિકેશન} \\
        \hline
        \textbf{રેન્જ} & 10-100m \\
        \hline
        \textbf{ડેટા રેટ} & 250 kbps \\
        \hline
        \textbf{પાવર} & ખૂબ નીચું (બૅટરી વર્ષો) \\
        \hline
        \textbf{ટોપોલોજી} & મેશ નેટવર્ક \\
        \hline
        \textbf{ફ્રીક્વન્સી} & વૈશ્વિક રીતે 2.4 GHz \\
        \hline
    \end{tabulary}

    \begin{center}
        \begin{tikzpicture}[gtu flow]
            \node[gtu block] (coord) {કોઓર્ડિનેટર};
            \node[gtu block, below left=1.2cm and 1cm of coord] (r1) {રાઉટર 1};
            \node[gtu block, below right=1.2cm and 1cm of coord] (r2) {રાઉટર 2};
            
            \node[gtu block, below=1cm of r1] (ed1) {એન્ડ ડિવાઇસ 1};
            \node[gtu block, below left=1cm and 0.5cm of r1] (ed2) {એન્ડ ડિવાઇસ 2};
            \node[gtu block, below=1cm of r2] (ed3) {એન્ડ ડિવાઇસ 3};
            
            \node[gtu block, right=1cm of r2] (r3) {રાઉટર 3};
            \node[gtu block, below=1cm of r3] (ed4) {એન્ડ ડિવાઇસ 4};

            \draw[gtu arrow] (coord) -- (r1);
            \draw[gtu arrow] (coord) -- (r2);
            \draw[gtu arrow] (r1) -- (ed1);
            \draw[gtu arrow] (r1) -- (ed2);
            \draw[gtu arrow] (r2) -- (ed3);
            \draw[gtu arrow] (r2) -- (r3);
            \draw[gtu arrow] (r3) -- (ed4);
        \end{tikzpicture}
    \end{center}

    \textbf{નેટવર્ક રોલ્સ}:
    \begin{itemize}
        \item \textbf{કોઓર્ડિનેટર}: નેટવર્ક મેનેજર.
        \item \textbf{રાઉટર}: મેસેજ ફોરવર્ડ કરે છે.
        \item \textbf{એન્ડ ડિવાઇસ}: સાદા સેન્સર્સ/એક્ચ્યુએટર્સ.
    \end{itemize}

    \textbf{ઉપયોગો}:
    \begin{itemize}
        \item હોમ ઓટોમેશન (લાઇટ્સ, થર્મોસ્ટેટ્સ)
        \item ઇન્ડસ્ટ્રિયલ મોનિટરિંગ
        \item સ્માર્ટ ગ્રિડ સિસ્ટમો
        \item હેલ્થકેર મોનિટરિંગ
        \item કૃષિ સેન્સર્સ
        \item બિલ્ડિંગ મેનેજમેન્ટ સિસ્ટમો
    \end{itemize}

    \textbf{લક્ષણો}:
    \begin{itemize}
        \item \textbf{સેલ્ફ-હીલિંગ}: ઓટોમેટિક રૂટ ડિસ્કવરી.
        \item \textbf{ઓછી કિંમત}: સાદો અમલીકરણ.
        \item \textbf{સુરક્ષિત}: AES એન્ક્રિપ્શન.
        \item \textbf{વિશ્વસનીય}: મેશ રિડન્ડન્સી.
    \end{itemize}
\end{solutionbox}
\begin{mnemonicbox}
    \mnemonic{Zigbee મેશ નેટવર્ક એપ્લિકેશનો}
\end{mnemonicbox}

\end{document}
