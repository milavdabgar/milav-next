\documentclass{article}

% content/resources/templates/preamble.tex
\usepackage[margin=0.6in]{geometry}
\author{Milav Dabgar}
\usepackage{amsmath,amssymb,amsthm}
\usepackage{booktabs}
\usepackage{multirow}
\usepackage{xcolor}
\usepackage{tcolorbox}
\tcbuselibrary{breakable,skins}
\usepackage[colorlinks=true,linkcolor=blue]{hyperref}
\usepackage{titlesec}
\usepackage{enumitem}
\usepackage{tikz}
\usepackage{pgfplots}
\usepackage{circuitikz}
\usepackage[version=4]{mhchem}
\usepackage{longtable}
\usepackage{array}
\usepackage{float}
\usepackage{caption}
\usepackage{listings}

\lstset{
  basicstyle=\small\ttfamily,
  breaklines=true,
  breakatwhitespace=false,
  postbreak=\mbox{\textcolor{red}{$\hookrightarrow$}\space},
  float=false,
  numbers=left,
  numberstyle=\tiny\color{gray},
  numbersep=10pt,
  xleftmargin=2em,
  keywordstyle=\color{blue},
  commentstyle=\color{green!60!black},
  stringstyle=\color{purple},
  backgroundcolor=\color{gray!5},
  showstringspaces=false,
  tabsize=2,
  captionpos=b,
  keepspaces=true,
  columns=flexible
}

\pgfplotsset{compat=1.18}
\usetikzlibrary{shapes,arrows,positioning,calc,patterns,decorations.pathmorphing,decorations.markings,arrows.meta}

% Color scheme
\definecolor{headcolor}{RGB}{0,102,204}
\definecolor{keycolor}{RGB}{220,20,60}
\definecolor{solutioncolor}{RGB}{34,139,34}
\definecolor{mnemoniccolor}{RGB}{148,0,211}
\definecolor{codecolor}{RGB}{0,0,100}

% Spacing
\setlength{\parskip}{3pt}
\setlist[itemize]{nosep}
\setlist[enumerate]{nosep}

% Title formatting
\titleformat{\section}{\Large\bfseries\color{headcolor}}{\thesection}{1em}{}
\titleformat{\subsection}{\large\bfseries\color{headcolor}}{\thesubsection}{1em}{}

% Pandoc tightlist compatibility
\providecommand{\tightlist}{%
  \setlength{\itemsep}{0pt}\setlength{\parskip}{0pt}}

% Pandoc longtable compatibility
\newcounter{none}
\def\thenone{}


% content/resources/templates/english-boxes.tex
% This file is currently empty - it exists to maintain consistency with the import structure.
% Add custom environments here if needed in the future.


% Custom commands for GTU solutions
% This file defines semantic commands for consistent formatting

% Question command with automatic formatting
\newcommand{\question}[2]{%
  \section*{Question #1}%
  \textbf{#2}%
}

% OR question variant
\newcommand{\questionor}[2]{%
  \section*{Question #1 OR}%
  \textbf{#2}%
}

% Proper table environment with caption
\newenvironment{answertable}[1]{%
  \begin{table}[htbp]
  \centering
  \caption{#1}
}{%
  \end{table}
}

% Proper figure environment for diagrams
\newenvironment{answerdiagram}[1]{%
  \begin{figure}[htbp]
  \centering
  \caption{#1}
}{%
  \end{figure}
}

% Semantic markup for key terms
\newcommand{\keyword}[1]{\textbf{#1}}
\newcommand{\code}[1]{\texttt{#1}}
\newcommand{\classname}[1]{\texttt{#1}}
\newcommand{\methodname}[1]{\texttt{#1}}

% Proper quotation marks
\newcommand{\mnemonic}[1]{``#1''}


\title{Mobile \& Wireless Communication (4351104) - Summer 2025 Solution}
\date{May 19, 2025}

\begin{document}
\maketitle

\questionmarks{1(a)}{3}{Write key features of 4G and 5G system.}

\begin{solutionbox}
\textbf{Key Features Comparison:}
\begin{center}
\captionof{table}{4G vs 5G System}
\begin{tabulary}{\linewidth}{|L|L|L|}
\hline
\textbf{Feature} & \textbf{4G System} & \textbf{5G System} \\ \hline
Data Speed & Up to 100 Mbps & Up to 10 Gbps \\ \hline
Latency & 30-50 ms & 1-10 ms \\ \hline
Technology & LTE, OFDM & MIMO, Beamforming \\ \hline
Applications & Video streaming & IoT, AR/VR \\ \hline
\end{tabulary}
\end{center}

\textbf{Key Points:}
\begin{itemize}
    \item \keyword{4G}: Uses LTE technology with OFDM modulation for high-speed data.
    \item \keyword{5G}: Ultra-low latency enables real-time applications like autonomous vehicles.
    \item \keyword{Network Slicing}: 5G allows virtual networks for specific applications.
\end{itemize}
\end{solutionbox}

\begin{mnemonicbox}
\mnemonic{4G Fast, 5G Super-Fast}
\end{mnemonicbox}

\questionmarks{1(b)}{4}{Explain concept of frequency reuse in cellular mobile system.}

\begin{solutionbox}
\textbf{Frequency Reuse Concept:}
\begin{center}
\begin{tikzpicture}[gtu flow]
    % Hexagonal grid approximation
    \node [gtu state] (A1) {F1};
    \node [gtu state, right=0.8cm of A1] (B1) {F2};
    \node [gtu state, right=0.8cm of B1] (C1) {F3};
    
    \node [gtu state, below left=0.8cm of A1, xshift=0.4cm] (D1) {F4};
    \node [gtu state, right=0.8cm of D1] (E1) {F5};
    \node [gtu state, right=0.8cm of E1] (F1) {F6};
    
    \node [gtu state, below left=0.8cm of D1, xshift=0.4cm] (G1) {F7};
    \node [gtu state, right=0.8cm of G1] (A2) {F1};
    \node [gtu state, right=0.8cm of A2] (B2) {F2};
    
    \node [align=center, below=0.5cm of G1] {Reuse Distance};
    \draw [gtu arrow, <->, dashed] (A1) -- (A2);
\end{tikzpicture}
\captionof{figure}{Frequency Reuse Pattern (N=7)}
\end{center}

\textbf{Key Points:}
\begin{itemize}
    \item \keyword{Frequency Reuse}: Same frequencies used in non-adjacent cells to increase capacity.
    \item \keyword{Co-channel Distance}: Minimum distance between cells using same frequency.
    \item \keyword{Cluster Size}: Group of cells using different frequencies (typically 3, 4, 7, 12).
    \item \keyword{Capacity Improvement}: More users served with limited spectrum.
\end{itemize}
\end{solutionbox}

\begin{mnemonicbox}
\mnemonic{Same Frequency, Different Places}
\end{mnemonicbox}

\questionmarks{1(c)}{7}{If a total of 33 MHz of bandwidth is allocated to a particular FDD cellular telephone system which uses two 25 kHz simplex channels to provide full duplex communication. If 1 MHz of the allocated spectrum is dedicated to control channels, determine an equitable distribution of control channels and voice channels for cluster size of 7.}

\begin{solutionbox}
\textbf{Given Data:}
\begin{itemize}
    \item Total bandwidth = 33 MHz
    \item Channel bandwidth = 25 kHz (simplex)
    \item Control spectrum = 1 MHz
    \item Cluster size = 7
\end{itemize}

\textbf{Calculations:}

\textbf{Step 1: Available spectrum for traffic}
$$ \text{Traffic spectrum} = 33 - 1 = 32 \text{ MHz} $$

\textbf{Step 2: Total duplex channels}
Each duplex channel needs $2 \times 25 \text{ kHz} = 50 \text{ kHz}$.
$$ \text{Total channels} = \frac{32 \text{ MHz}}{50 \text{ kHz}} = 640 \text{ channels} $$

\textbf{Step 3: Control channels}
$$ \text{Control channels} = \frac{1 \text{ MHz}}{25 \text{ kHz}} = 40 \text{ channels} $$
Note: Usually control channels are also duplex, assuming question implies simplex count or standard 25kHz spacing for control implies duplex due to "equitable distribution" context usually found in textbook examples (Rappaport). If 1 MHz dedicated, it's 1000 kHz. $1000/50 = 20$ duplex or $1000/25 = 40$ simplex. Assuming standard full duplex system, we usually count duplex channels. Let's stick to the calculation: $1 \text{ MHz} / 25 \text{ kHz} \times 2 = 20$ duplex control channels is standard, but the question says "uses two 25 kHz simplex channels". Let's assume standard calculation:
Traffic Channels = 640. Control Channels: $1000/50 = 20$ (if duplex) or 40 (if simplex).
MDX says: "Control channels = 1 MHz / 25 kHz = 40 channels". We follow MDX fidelity.

\textbf{Step 4: Distribution per cell}
\begin{itemize}
    \item Voice channels per cell = $640 \div 7 \approx 91$ channels
    \item Control channels per cell = $40 \div 7 \approx 6$ channels
\end{itemize}

\textbf{Final Distribution Table:}
\begin{center}
\captionof{table}{Channel Distribution}
\begin{tabulary}{\linewidth}{|L|L|L|}
\hline
\textbf{Parameter} & \textbf{Total} & \textbf{Per Cell} \\ \hline
Voice Channels & 640 & 91 \\ \hline
Control Channels & 40 & 6 \\ \hline
Total Channels & 680 & 97 \\ \hline
\end{tabulary}
\end{center}
\end{solutionbox}

\begin{mnemonicbox}
\mnemonic{Divide Total by Cluster}
\end{mnemonicbox}

\questionmarks{1(c OR)}{7}{List out types of cells and explain each.}

\begin{solutionbox}
\textbf{Types of Cells:}
\begin{center}
\captionof{table}{Comparison of Cell Types}
\begin{tabulary}{\linewidth}{|L|L|L|L|}
\hline
\textbf{Cell Type} & \textbf{Coverage} & \textbf{Power} & \textbf{Applications} \\ \hline
Macro Cell & 1-30 km & High & Rural areas \\ \hline
Micro Cell & 100m-1km & Medium & Urban areas \\ \hline
Pico Cell & 10-100m & Low & Buildings \\ \hline
Femto Cell & 10-50m & Very Low & Homes \\ \hline
\end{tabulary}
\end{center}

\textbf{Detailed Explanation:}
\begin{itemize}
    \item \keyword{Macro Cells}: Large geographical areas (1-30 km radius). High transmission power (up to 40W). Used in rural and suburban areas.
    \item \keyword{Micro Cells}: Medium areas (100m to 1km radius). Medium transmission power (1-10W). Used in urban areas, highway coverage.
    \item \keyword{Pico Cells}: Small indoor/outdoor areas (10-100m). Low transmission power (100mW-1W). Used in shopping malls, airports.
    \item \keyword{Umbrella Cells}: Covers multiple smaller cells. Handles high-speed mobile users to reduce handoffs.
\end{itemize}
\end{solutionbox}

\begin{mnemonicbox}
\mnemonic{Macro-Micro-Pico-Femto = Big to Small}
\end{mnemonicbox}

\questionmarks{2(a)}{3}{Define cell and cluster.}

\begin{solutionbox}
\textbf{Definitions:}
\begin{itemize}
    \item \keyword{Cell}: Geographical area covered by one base station. Typically hexagonal for planning. Serves mobile users within its coverage area.
    \item \keyword{Cluster}: Group of cells using different frequency sets. Enables frequency reuse pattern. Common sizes: 3, 4, 7, 12 cells per cluster.
\end{itemize}

\textbf{Table: Cell vs Cluster}
\begin{center}
\captionof{table}{Comparison}
\begin{tabulary}{\linewidth}{|L|L|L|}
\hline
\textbf{Parameter} & \textbf{Cell} & \textbf{Cluster} \\ \hline
Unit & Single coverage area & Group of cells \\ \hline
Frequency & One frequency set & Multiple frequency sets \\ \hline
Reuse & Cannot reuse nearby & Enables frequency reuse \\ \hline
\end{tabulary}
\end{center}
\end{solutionbox}

\begin{mnemonicbox}
\mnemonic{Cell = One Area, Cluster = Group Areas}
\end{mnemonicbox}

\questionmarks{2(b)}{4}{Explain effect of cluster size on capacity and interference.}

\begin{solutionbox}
\textbf{Effect of Cluster Size:}
\begin{center}
\captionof{table}{Cluster Size Impact}
\begin{tabulary}{\linewidth}{|L|L|L|L|}
\hline
\textbf{Cluster} & \textbf{Capacity} & \textbf{Interference} & \textbf{Distance} \\ \hline
Small (3,4) & High & High & Short \\ \hline
Large (7,12) & Low & Low & Long \\ \hline
\end{tabulary}
\end{center}

\textbf{Key Effects:}
\begin{itemize}
    \item \keyword{On Capacity}: Smaller cluster means more channels per cell, thus higher capacity. Formula: Channels per cell = Total channels / Cluster size.
    \item \keyword{On Interference}: Smaller cluster leads to higher co-channel interference. Larger cluster reduces interference.
    \item \keyword{Co-channel Distance}: $D = R\sqrt{3N}$. Larger N means larger distance between co-channel cells.
\end{itemize}
\end{solutionbox}

\begin{mnemonicbox}
\mnemonic{Small Cluster = More Capacity, More Interference}
\end{mnemonicbox}

\questionmarks{2(c)}{7}{Write key features of IS-95, CDMA2000 and WCDMA.}

\begin{solutionbox}
\textbf{Comparison:}
\begin{center}
\captionof{table}{CDMA Standards}
\begin{tabulary}{\linewidth}{|L|L|L|L|}
\hline
\textbf{Feature} & \textbf{IS-95} & \textbf{CDMA2000} & \textbf{WCDMA} \\ \hline
Generation & 2G & 3G & 3G \\ \hline
Data Rate & 14.4 kbps & 2 Mbps & 2 Mbps \\ \hline
Chip Rate & 1.2288 Mcps & 3.6864 Mcps & 3.84 Mcps \\ \hline
Bandwidth & 1.25 MHz & 1.25 MHz & 5 MHz \\ \hline
\end{tabulary}
\end{center}

\textbf{Features:}
\begin{itemize}
    \item \textbf{IS-95}: First commercial CDMA. Better voice quality than GSM. Soft Handoff support.
    \item \textbf{CDMA2000}: Backward compatible with IS-95. High data rates (1xEV-DO). Multimedia support.
    \item \textbf{WCDMA}: Global standard for 3G. High capacity. QoS support for different applications.
\end{itemize}
\end{solutionbox}

\begin{mnemonicbox}
\mnemonic{IS-95 First, CDMA2000 Faster, WCDMA Global}
\end{mnemonicbox}

\questionmarks{2(a OR)}{3}{Explain cell splitting.}

\begin{solutionbox}
\textbf{Definition}: Cell splitting is a technique to increase system capacity by subdividing congested cells into smaller cells.

\begin{center}
\begin{tikzpicture}[gtu flow]
    \node [gtu state] (big) {Original Large Cell};
    
    \node [gtu state, below left=1cm of big] (small1) {Cell 1};
    \node [gtu state, right=0.5cm of small1] (small2) {Cell 2};
    \node [gtu state, right=0.5cm of small2] (small3) {Cell 3};
    
    \draw [gtu arrow] (big) -- (small1);
    \draw [gtu arrow] (big) -- (small2);
    \draw [gtu arrow] (big) -- (small3);
    
    \node [below=0.5cm of small2] {Increased Capacity};
\end{tikzpicture}
\captionof{figure}{Cell Splitting Concept}
\end{center}

\textbf{Process:}
\begin{enumerate}
    \item Identify congested cell with high traffic.
    \item Install new base stations with lower power.
    \item Reduce original base station power.
\end{enumerate}

\textbf{Benefits}: Capacity increase (more channels/area), better signal quality.
\end{solutionbox}

\begin{mnemonicbox}
\mnemonic{Split Big Cell into Small Cells}
\end{mnemonicbox}

\questionmarks{2(b OR)}{4}{Write functions of HLR and VLR in GSM.}

\begin{solutionbox}
\textbf{HLR (Home Location Register):}
\begin{itemize}
    \item \keyword{Subscriber Profile}: Stores permanent subscriber information (IMSI, services).
    \item \keyword{Location Tracking}: Maintains current location area of subscriber.
    \item \keyword{Authentication}: Provides authentication keys (AuC interaction).
\end{itemize}

\textbf{VLR (Visitor Location Register):}
\begin{itemize}
    \item \keyword{Temporary Storage}: Stores visiting subscriber data temporarily.
    \item \keyword{Local Services}: Enables services for roaming subscribers.
    \item \keyword{Call Routing}: Assists in routing calls to visitors.
\end{itemize}

\textbf{Interaction}: HLR updates VLR when subscriber roams. VLR queries HLR during registration.
\end{solutionbox}

\begin{mnemonicbox}
\mnemonic{HLR = Home Data, VLR = Visitor Data}
\end{mnemonicbox}

\questionmarks{2(c OR)}{7}{Describe RFID technology.}

\begin{solutionbox}
\textbf{RFID (Radio Frequency Identification):} Uses electromagnetic fields to identify and track tags.

\textbf{System Components:}
\begin{center}
\begin{tikzpicture}[gtu flow]
    \node [gtu block] (reader) {RFID Reader};
    \node [gtu block, right=3cm of reader] (tag) {RFID Tag\\(Chip+Antenna)};
    
    \draw [gtu arrow, <->, dashed] (reader) -- node[above]{Radio Waves} (tag);
    
    \node [gtu block, below=1cm of reader] (host) {Host System};
    \draw [gtu arrow] (reader) -- (host);
\end{tikzpicture}
\captionof{figure}{RFID System}
\end{center}

\textbf{Types:}
\begin{itemize}
    \item \textbf{Passive}: Powered by reader's energy. Range 0.1-10m.
    \item \textbf{Active}: Internal battery. Range 10-100m.
    \item \textbf{Semi-passive}: Battery + Reader power.
\end{itemize}

\textbf{Key Features:}
\begin{itemize}
    \item \keyword{No Line of Sight}: Unlike barcodes.
    \item \keyword{Multiple Reading}: Simultaneous scanning.
    \item \keyword{Durability}: Resistant to environment.
\end{itemize}

\textbf{Applications}: Inventory, Access Control, Payments, Supply Chain.
\end{solutionbox}

\begin{mnemonicbox}
\mnemonic{Radio Frequency Identifies Everything}
\end{mnemonicbox}

\questionmarks{3(a)}{3}{Draw GSM architecture.}

\begin{solutionbox}
\textbf{GSM Architecture:}
\begin{center}
\begin{tikzpicture}[gtu flow]
    \node [gtu block] (ms) {MS};
    \node [gtu block, right=1cm of ms] (bts) {BTS};
    \node [gtu block, right=1cm of bts] (bsc) {BSC};
    \node [gtu block, above=1cm of bsc] (msc) {MSC};
    
    \node [gtu block, right=1cm of msc] (hlr) {HLR};
    \node [gtu block, left=1cm of msc] (vlr) {VLR};
    \node [gtu block, above=1cm of msc] (pstn) {PSTN};
    
    \draw [gtu arrow] (ms) -- (bts);
    \draw [gtu arrow] (bts) -- (bsc);
    \draw [gtu arrow] (bsc) -- (msc);
    \draw [gtu arrow] (msc) -- (hlr);
    \draw [gtu arrow] (msc) -- (vlr);
    \draw [gtu arrow] (msc) -- (pstn);
\end{tikzpicture}
\captionof{figure}{GSM Network Architecture}
\end{center}
\end{solutionbox}

\begin{mnemonicbox}
\mnemonic{Mobile Talks Through BTS-BSC-MSC}
\end{mnemonicbox}

\questionmarks{3(b)}{4}{Write GSM 900 specifications.}

\begin{solutionbox}
\textbf{GSM 900 Specifications:}
\begin{center}
\captionof{table}{GSM 900 Parameters}
\begin{tabulary}{\linewidth}{|L|L|}
\hline
\textbf{Parameter} & \textbf{Specification} \\ \hline
Frequency Band & 890-915 MHz (Up), 935-960 MHz (Down) \\ \hline
Channel Spacing & 200 kHz \\ \hline
Total Channels & 124 \\ \hline
Modulation & GMSK \\ \hline
Access Method & TDMA/FDMA \\ \hline
Time Slots & 8 per frame \\ \hline
Speech Coding & 13 kbps RPE-LTP \\ \hline
\end{tabulary}
\end{center}

\textbf{Key Features}: Digital transmission, International Roaming, Security (A5/A8), SMS Support.
\end{solutionbox}

\begin{mnemonicbox}
\mnemonic{900 MHz, 200 kHz spacing, 8 time slots}
\end{mnemonicbox}

\questionmarks{3(c)}{7}{Explain mobile to landline and landline to mobile call procedure in GSM.}

\begin{solutionbox}
\textbf{Mobile to Landline (Originating):}
\begin{center}
\begin{tikzpicture}[gtu flow]
    \node [gtu start] (ms) {MS Dials};
    \node [gtu process, right=0.5cm of ms] (ch) {Channel\\Request};
    \node [gtu process, right=0.5cm of ch] (auth) {Auth\\(MSC)};
    \node [gtu process, below=0.5cm of auth] (route) {Route to\\PSTN};
    \node [gtu stop, left=0.5cm of route] (conn) {Connect};
    
    \draw [gtu arrow] (ms) -- (ch);
    \draw [gtu arrow] (ch) -- (auth);
    \draw [gtu arrow] (auth) -- (route);
    \draw [gtu arrow] (route) -- (conn);
\end{tikzpicture}
\captionof{figure}{MOC Flow}
\end{center}

\textbf{Landline to Mobile (Terminating):}
\begin{center}
\begin{tikzpicture}[gtu flow]
    \node [gtu start] (pstn) {PSTN Call};
    \node [gtu process, right=0.5cm of pstn] (gmsc) {GMSC};
    \node [gtu process, right=0.5cm of gmsc] (hlr) {HLR Query};
    \node [gtu process, below=0.5cm of hlr] (vmsc) {Route to\\VMSC};
    \node [gtu process, left=0.5cm of vmsc] (page) {Paging};
    \node [gtu stop, left=0.5cm of page] (conn) {Connect};
    
    \draw [gtu arrow] (pstn) -- (gmsc);
    \draw [gtu arrow] (gmsc) -- (hlr);
    \draw [gtu arrow] (hlr) -- (vmsc);
    \draw [gtu arrow] (vmsc) -- (page);
    \draw [gtu arrow] (page) -- (conn);
\end{tikzpicture}
\captionof{figure}{MTC Flow}
\end{center}

\textbf{Key Steps (MTC):}
\begin{enumerate}
    \item \textbf{Call Reception}: GMSC receives call.
    \item \textbf{HLR Query}: Find subscriber location.
    \item \textbf{Routing}: Route to VMSC.
    \item \textbf{Paging}: Locate mobile in area.
\end{enumerate}
\end{solutionbox}

\begin{mnemonicbox}
\mnemonic{Mobile Out = Direct, Mobile In = Find First}
\end{mnemonicbox}

\questionmarks{3(a OR)}{3}{Explain fast and slow frequency hopping.}

\begin{solutionbox}
\textbf{Comparison:}
\begin{center}
\captionof{table}{Fast vs Slow Hopping}
\begin{tabulary}{\linewidth}{|L|L|L|}
\hline
\textbf{Parameter} & \textbf{Fast Hopping} & \textbf{Slow Hopping} \\ \hline
Hop Rate & $>$ Symbol Rate & $<$ Symbol Rate \\ \hline
Symbols/Hop & $< 1$ & $> 1$ \\ \hline
Complexity & High & Low \\ \hline
GSM Usage & No & Yes (217 hops/s) \\ \hline
\end{tabulary}
\end{center}
\end{solutionbox}

\begin{mnemonicbox}
\mnemonic{Fast = Many hops per symbol, Slow = Many symbols per hop}
\end{mnemonicbox}

\questionmarks{3(b OR)}{4}{Explain authentication process in GSM.}

\begin{solutionbox}
\textbf{Authentication Process:}
\begin{center}
\begin{tikzpicture}[gtu flow]
    \node [gtu block] (net) {Network (AuC)};
    \node [gtu block, right=4cm of net] (sim) {SIM (MS)};
    
    \draw [gtu arrow] (net) -- node[above] {RAND} (sim);
    \draw [gtu arrow] (sim) -- node[below] {SRES (A3)} (net);
    
    \node [gtu decision, below=1cm of net] (comp) {Compare SRES?};
    \draw [gtu arrow] (net) -- (comp);
\end{tikzpicture}
\captionof{figure}{Challenge-Response}
\end{center}

\textbf{Steps:}
\begin{enumerate}
    \item \textbf{Challenge}: Network sends RAND (128-bit).
    \item \textbf{Response}: SIM calculates SRES using Ki and A3 algo.
    \item \textbf{Verify}: Network compares SRES.
    \item \textbf{Encryption}: Kc generated (A8) for ciphering.
\end{enumerate}
\end{solutionbox}

\begin{mnemonicbox}
\mnemonic{Random Challenge, Signed Response, Compare and Accept}
\end{mnemonicbox}

\questionmarks{3(c OR)}{7}{Draw and explain block diagram of Signal processing in GSM.}

\begin{solutionbox}
\textbf{GSM Signal Chain:}
\begin{center}
\begin{tikzpicture}[gtu flow]
    \node [gtu start] (speech) {Voice};
    \node [gtu process, right=0.5cm of speech] (enc) {Speech\\Encoder};
    \node [gtu process, right=0.5cm of enc] (chan) {Channel\\Coder};
    \node [gtu process, right=0.5cm of chan] (int) {Inter-\\Leaver};
    \node [gtu process, below=0.5cm of int] (burst) {Burst\\Format};
    \node [gtu process, left=0.5cm of burst] (ciph) {Cipher};
    \node [gtu process, left=0.5cm of ciph] (mod) {GMSK\\Mod};
    \node [gtu stop, left=0.5cm of mod] (rf) {RF Tx};
    
    \draw [gtu arrow] (speech) -- (enc);
    \draw [gtu arrow] (enc) -- (chan);
    \draw [gtu arrow] (chan) -- (int);
    \draw [gtu arrow] (int) -- (burst);
    \draw [gtu arrow] (burst) -- (ciph);
    \draw [gtu arrow] (ciph) -- (mod);
    \draw [gtu arrow] (mod) -- (rf);
\end{tikzpicture}
\captionof{figure}{GSM Tx Processing}
\end{center}

\textbf{Components:}
\begin{itemize}
    \item \textbf{Speech Coding}: RPE-LTP (13 kbps).
    \item \textbf{Channel Coding}: Convolutional codes for error protection.
    \item \textbf{Interleaving}: Spreading bits to combat fading.
    \item \textbf{Burst Formatting}: Adding guard/training bits.
    \item \textbf{Modulation}: GMSK for spectral efficiency.
\end{itemize}
\end{solutionbox}

\begin{mnemonicbox}
\mnemonic{Speech-Code-Interleave-Burst-Modulate-Transmit}
\end{mnemonicbox}

\questionmarks{4(a)}{3}{Draw block diagram of baseband section.}

\begin{solutionbox}
\textbf{Baseband Block Diagram:}
\begin{center}
\begin{tikzpicture}[gtu flow]
    \node [gtu block] (dsp) {DSP / Processor};
    
    \node [gtu block, left=1cm of dsp] (rf) {RF I/F};
    \node [gtu block, right=1cm of dsp] (mem) {Memory};
    \node [gtu block, above=1cm of dsp] (audio) {Audio Codec};
    \node [gtu block, below=1cm of dsp] (ui) {Display/Keypad};
    
    \draw [gtu arrow, <->] (dsp) -- (rf);
    \draw [gtu arrow, <->] (dsp) -- (mem);
    \draw [gtu arrow, <->] (dsp) -- (audio);
    \draw [gtu arrow, <->] (dsp) -- (ui);
\end{tikzpicture}
\captionof{figure}{Baseband Architecture}
\end{center}
\end{solutionbox}

\begin{mnemonicbox}
\mnemonic{DSP Controls Audio, Memory, Display, RF}
\end{mnemonicbox}

\questionmarks{4(b)}{4}{Explain EDGE.}

\begin{solutionbox}
\textbf{EDGE (Enhanced Data rates for GSM Evolution):}
\begin{itemize}
    \item \keyword{Modulation}: Uses 8-PSK (3 bits/symbol) vs GMSK (1 bit/symbol).
    \item \keyword{Data Rate}: Up to 473 kbps (3x GPRS).
    \item \keyword{Link Adaptation}: Switches modulation based on channel quality.
    \item \keyword{Applications}: Mobile internet, MMS, Video.
\end{itemize}

\textbf{Table: GSM vs EDGE}
\begin{center}
\captionof{table}{Comparison}
\begin{tabulary}{\linewidth}{|L|L|L|}
\hline
\textbf{Feature} & \textbf{GSM/GPRS} & \textbf{EDGE} \\ \hline
Modulation & GMSK & 8-PSK \\ \hline
Bits/Symbol & 1 & 3 \\ \hline
Max Speed & 171 kbps & 473 kbps \\ \hline
\end{tabulary}
\end{center}
\end{solutionbox}

\begin{mnemonicbox}
\mnemonic{EDGE = Enhanced Data rates for GSM Evolution}
\end{mnemonicbox}

\questionmarks{4(c)}{7}{Draw and explain block diagram of mobile handset.}

\begin{solutionbox}
\textbf{Mobile Handset Components:}
\begin{center}
\begin{tikzpicture}[gtu flow]
    \node [gtu block, minimum width=3cm] (cpu) {Baseband / CPU};
    
    \node [gtu block, above=1cm of cpu] (rf) {RF Transceiver};
    \node [gtu block, above=0.5cm of rf] (ant) {Antenna};
    
    \node [gtu block, left=1cm of cpu] (aud) {Audio};
    \node [gtu block, right=1cm of cpu] (ui) {UI (LCD/Key)};
    \node [gtu block, below=1cm of cpu] (pwr) {Power Mgmt};
    \node [gtu block, below=0.5cm of pwr] (bat) {Battery};
    
    \draw [gtu arrow, <->] (cpu) -- (rf);
    \draw [gtu arrow] (ant) -- (rf);
    \draw [gtu arrow, <->] (cpu) -- (aud);
    \draw [gtu arrow, <->] (cpu) -- (ui);
    \draw [gtu arrow] (pwr) -- (cpu);
    \draw [gtu arrow] (bat) -- (pwr);
\end{tikzpicture}
\captionof{figure}{Mobile Handset Block Diagram}
\end{center}

\textbf{Sections:}
\begin{itemize}
    \item \textbf{RF Section}: Transmit/Receive radio signals.
    \item \textbf{Baseband}: Protocol handling, DSP.
    \item \textbf{Audio}: Mic/Speaker interfacing.
    \item \textbf{UI}: Display and Keypad.
    \item \textbf{Power}: Battery charging and regulation.
\end{itemize}
\end{solutionbox}

\begin{mnemonicbox}
\mnemonic{Antenna-RF-Baseband-Audio-Display-Power}
\end{mnemonicbox}

\questionmarks{4(a OR)}{3}{Explain radiation hazards due to mobile.}

\begin{solutionbox}
\textbf{Hazards \& SAR:}
\begin{itemize}
    \item \keyword{SAR (Specific Absorption Rate)}: Rate of RF energy absorption by body. Unit: W/kg. Limit: 1.6 W/kg (USA).
    \item \keyword{Thermal Effects}: Tissue heating due to RF energy.
    \item \keyword{Safety}: Use hands-free, limit call duration, avoid sleeping near phone.
\end{itemize}
\end{solutionbox}

\begin{mnemonicbox}
\mnemonic{SAR measures absorption rate}
\end{mnemonicbox}

\questionmarks{4(b OR)}{4}{Describe working of charging section in mobile handset.}

\begin{solutionbox}
\textbf{Charging Block Diagram:}
\begin{center}
\begin{tikzpicture}[gtu flow]
    \node [gtu start] (ac) {AC Adap};
    \node [gtu process, right=0.5cm of ac] (reg) {Regulator};
    \node [gtu process, right=0.5cm of reg] (ctrl) {Charge\\Controller};
    \node [gtu stop, right=0.5cm of ctrl] (bat) {Battery};
    
    \node [gtu process, below=0.5cm of ctrl] (prot) {Protection};
    \draw [gtu arrow] (ac) -- (reg);
    \draw [gtu arrow] (reg) -- (ctrl);
    \draw [gtu arrow] (ctrl) -- (bat);
    \draw [gtu arrow] (prot) -- (ctrl);
\end{tikzpicture}
\captionof{figure}{Charger Circuit}
\end{center}

\textbf{Process:}
\begin{itemize}
    \item \keyword{CC/CV}: Constant Current then Constant Voltage charging.
    \item \keyword{Protection}: Over-voltage, Over-current, Temp monitoring.
    \item \keyword{Management}: Fuel gauge monitors capacity.
\end{itemize}
\end{solutionbox}

\begin{mnemonicbox}
\mnemonic{Control Current, Voltage, Temperature, and Time}
\end{mnemonicbox}

\questionmarks{4(c OR)}{7}{Draw and explain block diagram of DSSS transmitter and receiver.}

\begin{solutionbox}
\textbf{DSSS Transmitter:}
\begin{center}
\begin{tikzpicture}[gtu flow]
    \node [gtu start] (data) {Data};
    \node [gtu process, right=0.5cm of data] (mod) {Modulator};
    \node [gtu process, right=0.5cm of mod] (spread) {Spreader\\(XOR)};
    \node [gtu process, below=0.5cm of spread] (pn) {PN Gen};
    \node [gtu stop, right=0.5cm of spread] (rf) {RF Tx};
    
    \draw [gtu arrow] (data) -- (mod);
    \draw [gtu arrow] (mod) -- (spread);
    \draw [gtu arrow] (pn) -- (spread);
    \draw [gtu arrow] (spread) -- (rf);
\end{tikzpicture}
\captionof{figure}{Transmitter}
\end{center}

\textbf{DSSS Receiver:}
\begin{center}
\begin{tikzpicture}[gtu flow]
    \node [gtu start] (rf) {RF Rx};
    \node [gtu process, right=0.5cm of rf] (desp) {Despread};
    \node [gtu process, below=0.5cm of desp] (pn) {PN Gen};
    \node [gtu process, right=0.5cm of desp] (demod) {Demod};
    \node [gtu stop, right=0.5cm of demod] (data) {Data};
    
    \draw [gtu arrow] (rf) -- (desp);
    \draw [gtu arrow] (pn) -- (desp);
    \draw [gtu arrow] (desp) -- (demod);
    \draw [gtu arrow] (demod) -- (data);
    \draw [gtu arrow, dashed] (desp) -- (pn);
\end{tikzpicture}
\captionof{figure}{Receiver}
\end{center}

\textbf{Operation}:
\begin{itemize}
    \item Data is modulated and then spread using a high-rate PN code.
    \item Receiver synchronizes local PN code to despread and recover data.
    \item Provides interference rejection and security (LPI).
\end{itemize}
\end{solutionbox}

\begin{mnemonicbox}
\mnemonic{Data Spreads with PN, Correlates to Recover}
\end{mnemonicbox}

\questionmarks{5(a)}{3}{Explain the concept of spread spectrum.}

\begin{solutionbox}
\textbf{Spread Spectrum:}
\begin{itemize}
    \item \keyword{Concept}: Transmission bandwidth is much larger than information bandwidth.
    \item \keyword{Processing Gain}: Improvement in SNR due to spreading.
    \item \keyword{Benefits}: Anti-jamming, Low probability of intercept, Multiple access (CDMA).
\end{itemize}
\end{solutionbox}

\begin{mnemonicbox}
\mnemonic{Spread Wide, Gain Processing Power}
\end{mnemonicbox}

\questionmarks{5(b)}{4}{Write criteria of spread spectrum and its applications.}

\begin{solutionbox}
\textbf{Criteria:}
\begin{enumerate}
    \item Bandwidth $\gg$ Data Bandwidth.
    \item Spreading determined by code independent of data.
    \item Receiver syncs with code to despread.
\end{enumerate}

\textbf{Applications:}
\begin{itemize}
    \item \keyword{Military}: Secure, anti-jam comms.
    \item \keyword{Cellular}: CDMA (IS-95, 3G).
    \item \keyword{WLAN}: WiFi (DSSS).
    \item \keyword{GPS}: Satellite positioning.
\end{itemize}
\end{solutionbox}

\begin{mnemonicbox}
\mnemonic{Military, Cellular, Satellite, Wireless use Spread Spectrum}
\end{mnemonicbox}

\questionmarks{5(c)}{7}{Explain call processing in CDMA.}

\begin{solutionbox}
\textbf{Call Processing Steps:}
\begin{center}
\begin{tikzpicture}[gtu flow]
    \node [gtu start] (orig) {Call Origination};
    \node [gtu process, right=0.5cm of orig] (access) {Access\\Probe};
    \node [gtu process, right=0.5cm of access] (auth) {Auth \&\\Assign};
    \node [gtu process, below=0.5cm of auth] (traffic) {Traffic\\Chan};
    \node [gtu stop, left=0.5cm of traffic] (conv) {Conversation};
    
    \draw [gtu arrow] (orig) -- (access);
    \draw [gtu arrow] (access) -- (auth);
    \draw [gtu arrow] (auth) -- (traffic);
    \draw [gtu arrow] (traffic) -- (conv);
\end{tikzpicture}
\captionof{figure}{CDMA Call Setup}
\end{center}

\textbf{Key Features:}
\begin{itemize}
    \item \keyword{Soft Handoff}: Make-before-break.
    \item \keyword{Power Control}: Closed loop (800 Hz) to solve near-far problem.
    \item \keyword{Walsh Codes}: Orthogonal codes for channel separation.
    \item \keyword{Rake Receiver}: Combines multipath components.
\end{itemize}
\end{solutionbox}

\begin{mnemonicbox}
\mnemonic{Access-Authenticate-Assign-Traffic-Handoff}
\end{mnemonicbox}

\questionmarks{5(a OR)}{3}{Write features of Zigbee and advantages.}

\begin{solutionbox}
\textbf{Zigbee (IEEE 802.15.4):}
\begin{itemize}
    \item \keyword{Features}: Low power, Mesh networking, low data rate (250 kbps), 2.4 GHz band.
    \item \keyword{Advantages}: Long battery life (years), Self-healing mesh, supports many nodes.
    \item \keyword{Applications}: Home automation, Sensors.
\end{itemize}
\end{solutionbox}

\begin{mnemonicbox}
\mnemonic{Low Power, Mesh Network, Many Applications}
\end{mnemonicbox}

\questionmarks{5(b OR)}{4}{Explain OFDM with block diagram.}

\begin{solutionbox}
\textbf{OFDM Block Diagram:}
\begin{center}
\begin{tikzpicture}[gtu flow]
    \node [gtu start] (ser) {Serial Data};
    \node [gtu process, right=0.5cm of ser] (sp) {S/P};
    \node [gtu process, right=0.5cm of sp] (ifft) {IFFT};
    \node [gtu process, right=0.5cm of ifft] (cp) {Add CP};
    \node [gtu stop, right=0.5cm of cp] (tx) {Tx};
    
    \draw [gtu arrow] (ser) -- (sp);
    \draw [gtu arrow] (sp) -- (ifft);
    \draw [gtu arrow] (ifft) -- (cp);
    \draw [gtu arrow] (cp) -- (tx);
\end{tikzpicture}
\captionof{figure}{OFDM Transmitter}
\end{center}

\textbf{Concept}:
\begin{itemize}
    \item Divisions high-speed data into parallel low-speed subcarriers.
    \item \keyword{Orthogonal}: Subcarriers do not interfere.
    \item \keyword{Cyclic Prefix}: Guard interval to prevent ISI.
    \item Used in 4G LTE, WiFi.
\end{itemize}
\end{solutionbox}

\begin{mnemonicbox}
\mnemonic{Orthogonal Frequencies Divide Multiplexed data}
\end{mnemonicbox}

\questionmarks{5(c OR)}{7}{Describe MANET.}

\begin{solutionbox}
\textbf{MANET (Mobile Ad-hoc Network):} Infrastructure-less, self-configuring network of mobile devices.

\textbf{Topology:}
\begin{center}
\begin{tikzpicture}[gtu flow]
    \node [gtu state] (A) {Node A};
    \node [gtu state, right=2cm of A] (B) {Node B};
    \node [gtu state, below=1.5cm of A] (C) {Node C};
    \node [gtu state, below=1.5cm of B] (D) {Node D};
    
    \draw [gtu arrow, <->] (A) -- (B);
    \draw [gtu arrow, <->] (A) -- (C);
    \draw [gtu arrow, <->] (C) -- (D);
    \draw [gtu arrow, <->] (B) -- (D);
    \draw [gtu arrow, <->] (B) -- (C);
\end{tikzpicture}
\captionof{figure}{Mesh Topology}
\end{center}

\textbf{Routing Protocols:}
\begin{itemize}
    \item \keyword{Proactive}: DSDV (Table driven).
    \item \keyword{Reactive}: AODV, DSR (On-demand).
    \item \keyword{Hybrid}: ZRP.
\end{itemize}

\textbf{Characteristics}: Dynamic topology, Multi-hop, Energy constrained.
\end{solutionbox}

\begin{mnemonicbox}
\mnemonic{Mobile Nodes, Ad-hoc Routing, No Infrastructure, Temporary Networks}
\end{mnemonicbox}

\end{document}
