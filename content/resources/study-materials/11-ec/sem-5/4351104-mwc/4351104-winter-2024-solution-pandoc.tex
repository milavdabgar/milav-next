\documentclass[10pt,a4paper]{article}

% content/resources/templates/preamble.tex
\usepackage[margin=0.6in]{geometry}
\author{Milav Dabgar}
\usepackage{amsmath,amssymb,amsthm}
\usepackage{booktabs}
\usepackage{multirow}
\usepackage{xcolor}
\usepackage{tcolorbox}
\tcbuselibrary{breakable,skins}
\usepackage[colorlinks=true,linkcolor=blue]{hyperref}
\usepackage{titlesec}
\usepackage{enumitem}
\usepackage{tikz}
\usepackage{pgfplots}
\usepackage{circuitikz}
\usepackage[version=4]{mhchem}
\usepackage{longtable}
\usepackage{array}
\usepackage{float}
\usepackage{caption}
\usepackage{listings}

\lstset{
  basicstyle=\small\ttfamily,
  breaklines=true,
  breakatwhitespace=false,
  postbreak=\mbox{\textcolor{red}{$\hookrightarrow$}\space},
  float=false,
  numbers=left,
  numberstyle=\tiny\color{gray},
  numbersep=10pt,
  xleftmargin=2em,
  keywordstyle=\color{blue},
  commentstyle=\color{green!60!black},
  stringstyle=\color{purple},
  backgroundcolor=\color{gray!5},
  showstringspaces=false,
  tabsize=2,
  captionpos=b,
  keepspaces=true,
  columns=flexible
}

\pgfplotsset{compat=1.18}
\usetikzlibrary{shapes,arrows,positioning,calc,patterns,decorations.pathmorphing,decorations.markings,arrows.meta}

% Color scheme
\definecolor{headcolor}{RGB}{0,102,204}
\definecolor{keycolor}{RGB}{220,20,60}
\definecolor{solutioncolor}{RGB}{34,139,34}
\definecolor{mnemoniccolor}{RGB}{148,0,211}
\definecolor{codecolor}{RGB}{0,0,100}

% Spacing
\setlength{\parskip}{3pt}
\setlist[itemize]{nosep}
\setlist[enumerate]{nosep}

% Title formatting
\titleformat{\section}{\Large\bfseries\color{headcolor}}{\thesection}{1em}{}
\titleformat{\subsection}{\large\bfseries\color{headcolor}}{\thesubsection}{1em}{}

% Pandoc tightlist compatibility
\providecommand{\tightlist}{%
  \setlength{\itemsep}{0pt}\setlength{\parskip}{0pt}}

% Pandoc longtable compatibility
\newcounter{none}
\def\thenone{}


% content/resources/templates/english-boxes.tex
% This file is currently empty - it exists to maintain consistency with the import structure.
% Add custom environments here if needed in the future.


\begin{document}

\begin{center}
{\Huge\bfseries\color{headcolor} Subject Name Solutions}\\[5pt]
{\LARGE 4351104 -- Winter 2024}\\[3pt]
{\large Semester 1 Study Material}\\[3pt]
{\normalsize\textit{Detailed Solutions and Explanations}}
\end{center}

\vspace{10pt}

\subsection*{Question 1(a) [3 marks]}\label{q1a}

\textbf{Explain umbrella cell.}

\begin{solutionbox}
\textbf{Umbrella cell} is a large coverage area cell
that overlays smaller cells to provide continuous coverage and handle
overflow traffic.


{\def\LTcaptype{none} % do not increment counter
\vspace{-5pt}
\captionof{table}{Umbrella Cell Characteristics}
\vspace{-10pt}
\begin{longtable}[]{@{}ll@{}}
\toprule\noalign{}
Feature & Description \\
\midrule\noalign{}
\endhead
\bottomrule\noalign{}
\endlastfoot
\textbf{Coverage} & Large geographic area \\
\textbf{Purpose} & Handle overflow traffic from microcells \\
\textbf{Antenna} & High-power, elevated position \\
\textbf{Users} & Fast-moving vehicles, emergency calls \\
\end{longtable}
}

\begin{itemize}
\tightlist
\item
  \textbf{Large coverage}: Covers wide geographical area with high-power
  base station
\item
  \textbf{Traffic management}: Handles calls when smaller cells are
  congested
\item
  \textbf{Mobility support}: Serves fast-moving users crossing multiple
  cell boundaries
\end{itemize}

\end{solutionbox}
\begin{mnemonicbox}
``Umbrella Covers Large Areas''

\end{mnemonicbox}
\begin{center}\rule{0.5\linewidth}{0.5pt}\end{center}

\subsection*{Question 1(b) [4 marks]}\label{q1b}

\textbf{Define cell and cluster.}

\begin{solutionbox}
\textbf{Cell} and \textbf{cluster} are fundamental
concepts in cellular communication systems.


{\def\LTcaptype{none} % do not increment counter
\vspace{-5pt}
\captionof{table}{Cell vs Cluster Comparison}
\vspace{-10pt}
\begin{longtable}[]{@{}
  >{\raggedright\arraybackslash}p{(\linewidth - 4\tabcolsep) * \real{0.4231}}
  >{\raggedright\arraybackslash}p{(\linewidth - 4\tabcolsep) * \real{0.2308}}
  >{\raggedright\arraybackslash}p{(\linewidth - 4\tabcolsep) * \real{0.3462}}@{}}
\toprule\noalign{}
\begin{minipage}[b]{\linewidth}\raggedright
Parameter
\end{minipage} & \begin{minipage}[b]{\linewidth}\raggedright
Cell
\end{minipage} & \begin{minipage}[b]{\linewidth}\raggedright
Cluster
\end{minipage} \\
\midrule\noalign{}
\endhead
\bottomrule\noalign{}
\endlastfoot
\textbf{Definition} & Single coverage area served by one base station &
Group of cells using different frequencies \\
\textbf{Size} & Limited by antenna power and interference & Contains N
cells (typically 3, 4, 7, 12) \\
\textbf{Frequency} & Uses specific frequency set & Uses all available
frequencies once \\
\textbf{Purpose} & Provide coverage to specific area & Enable frequency
reuse pattern \\
\end{longtable}
}

\begin{itemize}
\tightlist
\item
  \textbf{Cell}: Geographic area served by single base station with
  specific frequency allocation
\item
  \textbf{Cluster}: Group of adjacent cells that collectively use entire
  frequency spectrum
\item
  \textbf{Frequency reuse}: Same frequencies can be reused in different
  clusters
\item
  \textbf{Pattern repetition}: Cluster pattern repeats throughout
  coverage area
\end{itemize}

\end{solutionbox}
\begin{mnemonicbox}
``Cells Cluster for Complete Coverage''

\end{mnemonicbox}
\begin{center}\rule{0.5\linewidth}{0.5pt}\end{center}

\subsection*{Question 1(c) [7 marks]}\label{q1c}

\textbf{Describe fundamental concept behind cellular communication
systems.}

\begin{solutionbox}
\textbf{Cellular communication} divides service area
into small cells to maximize spectrum efficiency and capacity.

\textbf{Diagram:}

\begin{verbatim}
    +{-{-}{-}{-}{-}{-}{-}+{-}{-}{-}{-}{-}{-}{-}+{-}{-}{-}{-}{-}{-}{-}+}
    |   A   |   B   |   C   |
    |  f1   |  f2   |  f3   |
    +{-{-}{-}{-}{-}{-}{-}+{-}{-}{-}{-}{-}{-}{-}+{-}{-}{-}{-}{-}{-}{-}+}
    |   D   |   E   |   F   |
    |  f4   |  f5   |  f6   |
    +{-{-}{-}{-}{-}{-}{-}+{-}{-}{-}{-}{-}{-}{-}+{-}{-}{-}{-}{-}{-}{-}+}
    |   G   |   H   |   I   |
    |  f7   |  f1   |  f2   |
    +{-{-}{-}{-}{-}{-}{-}+{-}{-}{-}{-}{-}{-}{-}+{-}{-}{-}{-}{-}{-}{-}+}
\end{verbatim}


{\def\LTcaptype{none} % do not increment counter
\vspace{-5pt}
\captionof{table}{Cellular System Benefits}
\vspace{-10pt}
\begin{longtable}[]{@{}ll@{}}
\toprule\noalign{}
Concept & Advantage \\
\midrule\noalign{}
\endhead
\bottomrule\noalign{}
\endlastfoot
\textbf{Frequency Reuse} & Same frequencies used multiple times \\
\textbf{Cell Division} & Smaller coverage areas, more capacity \\
\textbf{Handoff} & Seamless call transfer between cells \\
\textbf{Power Control} & Reduced interference, longer battery life \\
\end{longtable}
}

\begin{itemize}
\tightlist
\item
  \textbf{Small cell concept}: Service area divided into hexagonal cells
  for efficient coverage
\item
  \textbf{Frequency reuse}: Limited spectrum used multiple times with
  adequate separation
\item
  \textbf{Base station control}: Each cell served by low-power base
  station
\item
  \textbf{Capacity improvement}: More users supported compared to single
  large coverage area
\item
  \textbf{Interference management}: Co-channel interference controlled
  through proper cell planning
\end{itemize}

\end{solutionbox}
\begin{mnemonicbox}
``Small Cells Support Spectrum Sharing Successfully''

\end{mnemonicbox}
\begin{center}\rule{0.5\linewidth}{0.5pt}\end{center}

\subsection*{Question 1(c OR) [7
marks]}\label{question-1c-or-7-marks}

\textbf{Explain co-channel interference in cellular communication.}

\begin{solutionbox}
\textbf{Co-channel interference} occurs when cells
using same frequencies are too close, causing signal degradation.

\begin{center}
\textbf{Mermaid Diagram (Code)}
\begin{verbatim}
{Shaded}
{Highlighting}[]
graph TD
    A[Cell A {- f1] {-}{-}{} B[Interference Zone]}
    C[Cell C {- f1] {-}{-}{} B}
    B {-{-}{} D[Degraded Signal Quality]}
    E[Distance D] {-{-}{} F[Reduced Interference]}
{Highlighting}
{Shaded}
\end{verbatim}
\end{center}


{\def\LTcaptype{none} % do not increment counter
\vspace{-5pt}
\captionof{table}{Co-channel Interference Parameters}
\vspace{-10pt}
\begin{longtable}[]{@{}
  >{\raggedright\arraybackslash}p{(\linewidth - 4\tabcolsep) * \real{0.3333}}
  >{\raggedright\arraybackslash}p{(\linewidth - 4\tabcolsep) * \real{0.3939}}
  >{\raggedright\arraybackslash}p{(\linewidth - 4\tabcolsep) * \real{0.2727}}@{}}
\toprule\noalign{}
\begin{minipage}[b]{\linewidth}\raggedright
Parameter
\end{minipage} & \begin{minipage}[b]{\linewidth}\raggedright
Description
\end{minipage} & \begin{minipage}[b]{\linewidth}\raggedright
Impact
\end{minipage} \\
\midrule\noalign{}
\endhead
\bottomrule\noalign{}
\endlastfoot
\textbf{Reuse Distance} & Distance between co-channel cells & Higher
distance = Less interference \\
\textbf{C/I Ratio} & Carrier to Interference ratio & Must be \geq 18 dB for
good quality \\
\textbf{Cluster Size} & Number of cells in cluster & Larger cluster =
More separation \\
\end{longtable}
}

\begin{itemize}
\tightlist
\item
  \textbf{Signal overlap}: Same frequency signals from different cells
  interfere
\item
  \textbf{Quality degradation}: Causes call drops and poor voice quality
\item
  \textbf{Distance factor}: Interference reduces with square of distance
\item
  \textbf{Mitigation methods}: Proper cell planning, power control,
  antenna design
\end{itemize}

\end{solutionbox}
\begin{mnemonicbox}
``Co-channel Causes Call Quality Concerns''

\end{mnemonicbox}
\begin{center}\rule{0.5\linewidth}{0.5pt}\end{center}

\subsection*{Question 2(a) [3 marks]}\label{q2a}

\textbf{Explain cell splitting.}

\begin{solutionbox}
\textbf{Cell splitting} divides congested cells into
smaller cells to increase system capacity.

\textbf{Diagram:}

\begin{verbatim}
Original Large Cell          After Cell Splitting
    +{-{-}{-}{-}{-}{-}{-}+                   +{-}{-}{-}+{-}{-}{-}+}
    |       |                   | A | B |
    |   X   |                  +{-{-}{-}+{-}{-}{-}+}
    |       |                   | C | D |
    +{-{-}{-}{-}{-}{-}{-}+                   +{-}{-}{-}+{-}{-}{-}+}
\end{verbatim}

\begin{itemize}
\tightlist
\item
  \textbf{Capacity increase}: Each new cell handles fewer users with
  better service quality
\item
  \textbf{Power reduction}: New base stations use lower power to cover
  smaller areas
\item
  \textbf{Frequency management}: Original frequencies distributed among
  new smaller cells
\end{itemize}

\end{solutionbox}
\begin{mnemonicbox}
``Split Cells Serve Subscribers Successfully''

\end{mnemonicbox}
\begin{center}\rule{0.5\linewidth}{0.5pt}\end{center}

\subsection*{Question 2(b) [4 marks]}\label{q2b}

\textbf{Explain channel assignment strategies.}

\begin{solutionbox}
\textbf{Channel assignment} strategies determine how
frequencies are allocated to cells for optimal performance.


{\def\LTcaptype{none} % do not increment counter
\vspace{-5pt}
\captionof{table}{Channel Assignment Strategies}
\vspace{-10pt}
\begin{longtable}[]{@{}
  >{\raggedright\arraybackslash}p{(\linewidth - 6\tabcolsep) * \real{0.2000}}
  >{\raggedright\arraybackslash}p{(\linewidth - 6\tabcolsep) * \real{0.2600}}
  >{\raggedright\arraybackslash}p{(\linewidth - 6\tabcolsep) * \real{0.2400}}
  >{\raggedright\arraybackslash}p{(\linewidth - 6\tabcolsep) * \real{0.3000}}@{}}
\toprule\noalign{}
\begin{minipage}[b]{\linewidth}\raggedright
Strategy
\end{minipage} & \begin{minipage}[b]{\linewidth}\raggedright
Description
\end{minipage} & \begin{minipage}[b]{\linewidth}\raggedright
Advantages
\end{minipage} & \begin{minipage}[b]{\linewidth}\raggedright
Disadvantages
\end{minipage} \\
\midrule\noalign{}
\endhead
\bottomrule\noalign{}
\endlastfoot
\textbf{Fixed} & Channels permanently assigned to cells & Simple,
predictable & Inefficient during low traffic \\
\textbf{Dynamic} & Channels assigned based on demand & Efficient
spectrum use & Complex implementation \\
\textbf{Hybrid} & Combination of fixed and dynamic & Balanced approach &
Moderate complexity \\
\end{longtable}
}

\begin{itemize}
\tightlist
\item
  \textbf{Fixed assignment}: Each cell has predetermined set of channels
\item
  \textbf{Dynamic assignment}: Channels allocated in real-time based on
  traffic demand
\item
  \textbf{Load balancing}: Distributes traffic evenly across available
  channels
\item
  \textbf{Interference avoidance}: Considers co-channel interference in
  assignment decisions
\end{itemize}

\end{solutionbox}
\begin{mnemonicbox}
``Dynamic Distribution Delivers Optimal Performance''

\end{mnemonicbox}
\begin{center}\rule{0.5\linewidth}{0.5pt}\end{center}

\subsection*{Question 2(c) [7 marks]}\label{q2c}

\textbf{Calculate voice and control channels per cell for 33MHz
bandwidth, 25KHz simplex channels, 7-cell reuse, 1MHz for control.}

\begin{solutionbox}
\textbf{Calculation} for channel allocation in cellular
system.

\textbf{Given Data:}

\begin{itemize}
\tightlist
\item
  Total bandwidth = 33 MHz
\item
  Channel bandwidth = 25 KHz (simplex)
\item
  Full duplex requires = 2 \times 25 KHz = 50 KHz
\item
  Control spectrum = 1 MHz
\item
  Cluster size = 7 cells
\end{itemize}

\textbf{Calculations:}

\textbf{Step 1: Total available channels} Total channels = 33 MHz \div 25
KHz = 1320 channels

\textbf{Step 2: Control channels} Control channels = 1 MHz \div 25 KHz = 40
channels

\textbf{Step 3: Voice channels} Voice channels = 1320 - 40 = 1280
channels

\textbf{Step 4: Duplex voice channels} Duplex voice channels = 1280 \div 2
= 640 channels

\textbf{Step 5: Channels per cell} Voice channels per cell = 640 \div 7 \approx
91 channels Control channels per cell = 40 \div 7 \approx 6 channels

\textbf{Final Answer:}

\begin{itemize}
\tightlist
\item
  \textbf{Voice channels per cell: 91}
\item
  \textbf{Control channels per cell: 6}
\end{itemize}

\end{solutionbox}
\begin{mnemonicbox}
``Calculate Carefully for Channel Count''

\end{mnemonicbox}
\begin{center}\rule{0.5\linewidth}{0.5pt}\end{center}

\subsection*{Question 2(a OR) [3
marks]}\label{question-2a-or-3-marks}

\textbf{Write functions of FCCH and SCH in GSM.}

\begin{solutionbox}
\textbf{FCCH} and \textbf{SCH} are essential control
channels in GSM system for synchronization.


{\def\LTcaptype{none} % do not increment counter
\vspace{-5pt}
\captionof{table}{FCCH and SCH Functions}
\vspace{-10pt}
\begin{longtable}[]{@{}
  >{\raggedright\arraybackslash}p{(\linewidth - 4\tabcolsep) * \real{0.3000}}
  >{\raggedright\arraybackslash}p{(\linewidth - 4\tabcolsep) * \real{0.3667}}
  >{\raggedright\arraybackslash}p{(\linewidth - 4\tabcolsep) * \real{0.3333}}@{}}
\toprule\noalign{}
\begin{minipage}[b]{\linewidth}\raggedright
Channel
\end{minipage} & \begin{minipage}[b]{\linewidth}\raggedright
Full Form
\end{minipage} & \begin{minipage}[b]{\linewidth}\raggedright
Function
\end{minipage} \\
\midrule\noalign{}
\endhead
\bottomrule\noalign{}
\endlastfoot
\textbf{FCCH} & Frequency Correction Channel & Provides frequency
reference to mobile \\
\textbf{SCH} & Synchronization Channel & Provides timing and cell
identity \\
\end{longtable}
}

\begin{itemize}
\tightlist
\item
  \textbf{FCCH function}: Enables mobile to synchronize with base
  station frequency
\item
  \textbf{SCH function}: Carries BSIC (Base Station Identity Code) and
  frame number
\item
  \textbf{Timing correction}: Both channels help mobile achieve proper
  timing synchronization
\end{itemize}

\end{solutionbox}
\begin{mnemonicbox}
``FCCH Fixes Frequency, SCH Synchronizes System''

\end{mnemonicbox}
\begin{center}\rule{0.5\linewidth}{0.5pt}\end{center}

\subsection*{Question 2(b OR) [4
marks]}\label{question-2b-or-4-marks}

\textbf{Write GSM 900 specifications.}

\begin{solutionbox}
\textbf{GSM 900} operates in 900 MHz frequency band
with specific technical parameters.


{\def\LTcaptype{none} % do not increment counter
\vspace{-5pt}
\captionof{table}{GSM 900 Specifications}
\vspace{-10pt}
\begin{longtable}[]{@{}ll@{}}
\toprule\noalign{}
Parameter & Specification \\
\midrule\noalign{}
\endhead
\bottomrule\noalign{}
\endlastfoot
\textbf{Uplink Frequency} & 890-915 MHz \\
\textbf{Downlink Frequency} & 935-960 MHz \\
\textbf{Duplex Separation} & 45 MHz \\
\textbf{Channel Spacing} & 200 KHz \\
\textbf{Total Channels} & 124 channels \\
\textbf{Access Method} & TDMA/FDMA \\
\textbf{Modulation} & GMSK \\
\textbf{Power Classes} & 2W, 8W, 20W \\
\end{longtable}
}

\begin{itemize}
\tightlist
\item
  \textbf{Frequency bands}: Separate uplink and downlink frequencies for
  full duplex operation
\item
  \textbf{TDMA structure}: 8 time slots per carrier frequency
\end{itemize}

\end{solutionbox}
\begin{mnemonicbox}
``GSM 900 Gives Great Global Coverage''

\end{mnemonicbox}
\begin{center}\rule{0.5\linewidth}{0.5pt}\end{center}

\subsection*{Question 2(c OR) [7
marks]}\label{question-2c-or-7-marks}

\textbf{Draw and explain GSM architecture.}

\begin{solutionbox}
\textbf{GSM architecture} consists of three main
subsystems working together for mobile communication.

\begin{verbatim}
graph TB
    MS[Mobile Station] {-{-} BSS[Base Station Subsystem]}
    BSS {-{-} NSS[Network Switching Subsystem]}
    BSS {-{-} BTS[Base Transceiver Station]}
    BSS {-{-} BSC[Base Station Controller]}
    NSS {-{-} MSC[Mobile Switching Center]}
    NSS {-{-} HLR[Home Location Register]}
    NSS {-{-} VLR[Visitor Location Register]}
    NSS {-{-} AuC[Authentication Center]}
    MSC {-{-} PSTN[Public Switched Telephone Network]}
\end{verbatim}


{\def\LTcaptype{none} % do not increment counter
\vspace{-5pt}
\captionof{table}{GSM Architecture Components}
\vspace{-10pt}
\begin{longtable}[]{@{}
  >{\raggedright\arraybackslash}p{(\linewidth - 4\tabcolsep) * \real{0.3333}}
  >{\raggedright\arraybackslash}p{(\linewidth - 4\tabcolsep) * \real{0.3636}}
  >{\raggedright\arraybackslash}p{(\linewidth - 4\tabcolsep) * \real{0.3030}}@{}}
\toprule\noalign{}
\begin{minipage}[b]{\linewidth}\raggedright
Subsystem
\end{minipage} & \begin{minipage}[b]{\linewidth}\raggedright
Components
\end{minipage} & \begin{minipage}[b]{\linewidth}\raggedright
Function
\end{minipage} \\
\midrule\noalign{}
\endhead
\bottomrule\noalign{}
\endlastfoot
\textbf{Mobile Station} & Mobile Equipment + SIM & User interface and
identity \\
\textbf{BSS} & BTS + BSC & Radio interface and control \\
\textbf{NSS} & MSC, HLR, VLR, AuC & Switching and database management \\
\end{longtable}
}

\begin{itemize}
\tightlist
\item
  \textbf{Mobile Station}: Consists of mobile equipment and SIM card for
  user identification
\item
  \textbf{Base Station Subsystem}: Handles radio communication and
  resource management
\item
  \textbf{Network Switching Subsystem}: Manages call switching, routing,
  and subscriber databases
\item
  \textbf{Interfaces}: A-bis (BTS-BSC), A (BSC-MSC) interfaces connect
  subsystems
\end{itemize}

\end{solutionbox}
\begin{mnemonicbox}
``Mobile Base Network - Complete Communication
Chain''

\end{mnemonicbox}
\begin{center}\rule{0.5\linewidth}{0.5pt}\end{center}

\subsection*{Question 3(a) [3 marks]}\label{q3a}

\textbf{Draw block diagram of signal processing in GSM.}

\begin{solutionbox}
\textbf{Signal processing} in GSM involves multiple
stages for voice and data transmission.

\textbf{Diagram:}

\begin{verbatim}
Speech  Speech  Channel  Interleaving  Burst  RF
Input    Coding    Coding              Formatting  Processing
  ↓        ↓         ↓         ↓           ↓         ↓
13kbps  22.8kbps  Error  Reordering  Time  Modulation
                  Protection              Slot    \& Transmission
\end{verbatim}

\begin{itemize}
\tightlist
\item
  \textbf{Speech coding}: Converts analog speech to 13 kbps digital data
  using RPE-LTP
\item
  \textbf{Channel coding}: Adds error correction bits increasing rate to
  22.8 kbps
\item
  \textbf{Interleaving}: Reorders data to combat burst errors from
  fading
\end{itemize}

\end{solutionbox}
\begin{mnemonicbox}
``Speech Signals Systematically Processed
Successfully''

\end{mnemonicbox}
\begin{center}\rule{0.5\linewidth}{0.5pt}\end{center}

\subsection*{Question 3(b) [4 marks]}\label{q3b}

\textbf{Write functions of Common Control Channels in GSM.}

\begin{solutionbox}
\textbf{Common Control Channels} manage system
information and access procedures in GSM.


{\def\LTcaptype{none} % do not increment counter
\vspace{-5pt}
\captionof{table}{Common Control Channels Functions}
\vspace{-10pt}
\begin{longtable}[]{@{}ll@{}}
\toprule\noalign{}
Channel & Function \\
\midrule\noalign{}
\endhead
\bottomrule\noalign{}
\endlastfoot
\textbf{FCCH} & Frequency correction and synchronization \\
\textbf{SCH} & Frame synchronization and cell identification \\
\textbf{BCCH} & Broadcasts system information and cell parameters \\
\textbf{RACH} & Random access for call initiation by mobile \\
\textbf{AGCH} & Assigns dedicated channels to mobiles \\
\textbf{PCH} & Pages mobiles for incoming calls \\
\end{longtable}
}

\begin{itemize}
\tightlist
\item
  \textbf{Broadcast function}: BCCH continuously transmits system
  information
\item
  \textbf{Access management}: RACH allows mobiles to request service
\item
  \textbf{Channel assignment}: AGCH allocates resources for active calls
\item
  \textbf{Paging service}: PCH notifies mobiles of incoming calls
\end{itemize}

\end{solutionbox}
\begin{mnemonicbox}
``Common Channels Control Communication Completely''

\end{mnemonicbox}
\begin{center}\rule{0.5\linewidth}{0.5pt}\end{center}

\subsection*{Question 3(c) [7 marks]}\label{q3c}

\textbf{Explain GSM identifiers.}

\begin{solutionbox}
\textbf{GSM identifiers} uniquely identify subscribers,
equipment, and network elements.


{\def\LTcaptype{none} % do not increment counter
\vspace{-5pt}
\captionof{table}{GSM Identifiers}
\vspace{-10pt}
\begin{longtable}[]{@{}
  >{\raggedright\arraybackslash}p{(\linewidth - 6\tabcolsep) * \real{0.2927}}
  >{\raggedright\arraybackslash}p{(\linewidth - 6\tabcolsep) * \real{0.2683}}
  >{\raggedright\arraybackslash}p{(\linewidth - 6\tabcolsep) * \real{0.2195}}
  >{\raggedright\arraybackslash}p{(\linewidth - 6\tabcolsep) * \real{0.2195}}@{}}
\toprule\noalign{}
\begin{minipage}[b]{\linewidth}\raggedright
Identifier
\end{minipage} & \begin{minipage}[b]{\linewidth}\raggedright
Full Form
\end{minipage} & \begin{minipage}[b]{\linewidth}\raggedright
Purpose
\end{minipage} & \begin{minipage}[b]{\linewidth}\raggedright
Format
\end{minipage} \\
\midrule\noalign{}
\endhead
\bottomrule\noalign{}
\endlastfoot
\textbf{IMSI} & International Mobile Subscriber Identity & Unique
subscriber ID & 15 digits \\
\textbf{IMEI} & International Mobile Equipment Identity & Unique
equipment ID & 15 digits \\
\textbf{MSISDN} & Mobile Station ISDN Number & Phone number & Variable
length \\
\textbf{TMSI} & Temporary Mobile Subscriber Identity & Temporary ID for
security & 32 bits \\
\textbf{LAI} & Location Area Identity & Geographic area identification &
MCC+MNC+LAC \\
\textbf{BSIC} & Base Station Identity Code & Cell identification & 6
bits \\
\end{longtable}
}

\begin{itemize}
\tightlist
\item
  \textbf{IMSI structure}: MCC (3) + MNC (2-3) + MSIN (9-10 digits)
\item
  \textbf{Security purpose}: TMSI protects subscriber identity over
  radio interface
\item
  \textbf{Location management}: LAI helps in efficient paging and
  location updates
\item
  \textbf{Network planning}: BSIC prevents confusion between adjacent
  cells
\end{itemize}

\end{solutionbox}
\begin{mnemonicbox}
``Important Mobile System Identifiers Ensure
Security''

\end{mnemonicbox}
\begin{center}\rule{0.5\linewidth}{0.5pt}\end{center}

\subsection*{Question 3(a OR) [3
marks]}\label{question-3a-or-3-marks}

\textbf{Compare Fast and Slow frequency hopping.}

\begin{solutionbox}
\textbf{Frequency hopping} techniques differ in hopping
rate relative to symbol rate.


{\def\LTcaptype{none} % do not increment counter
\vspace{-5pt}
\captionof{table}{Fast vs Slow Frequency Hopping}
\vspace{-10pt}
\begin{longtable}[]{@{}lll@{}}
\toprule\noalign{}
Parameter & Fast Hopping & Slow Hopping \\
\midrule\noalign{}
\endhead
\bottomrule\noalign{}
\endlastfoot
\textbf{Hopping Rate} & \textgreater{} Symbol rate & \textless{} Symbol
rate \\
\textbf{Symbols per Hop} & \textless{} 1 & \textgreater{} 1 \\
\textbf{Complexity} & High & Low \\
\textbf{Applications} & Military, Bluetooth & GSM, CDMA \\
\end{longtable}
}

\begin{itemize}
\tightlist
\item
  \textbf{Fast hopping}: Multiple hops per symbol, better security but
  more complex
\item
  \textbf{Slow hopping}: Multiple symbols per hop, simpler
  implementation
\end{itemize}

\end{solutionbox}
\begin{mnemonicbox}
``Fast Frequently Flips, Slow Stays Stable''

\end{mnemonicbox}
\begin{center}\rule{0.5\linewidth}{0.5pt}\end{center}

\subsection*{Question 3(b OR) [4
marks]}\label{question-3b-or-4-marks}

\textbf{Calculate number of users in GSM 900 band without frequency
reuse.}

\begin{solutionbox}
\textbf{Calculation} for maximum users in GSM 900
without frequency reuse.

\textbf{Given GSM 900 Parameters:}

\begin{itemize}
\tightlist
\item
  Uplink: 890-915 MHz (25 MHz)
\item
  Downlink: 935-960 MHz (25 MHz)
\item
  Channel spacing: 200 KHz
\item
  Time slots per channel: 8
\end{itemize}

\textbf{Calculations:}

\textbf{Step 1: Available channels} Total channels = 25 MHz \div 200 KHz =
125 channels

\textbf{Step 2: Usable channels} Guard channels removed \approx 124 channels

\textbf{Step 3: Simultaneous users} Users per channel = 8 time slots
Total users = 124 \times 8 = 992 users

\end{solutionbox}
\begin{solutionbox}

\end{solutionbox}
\begin{mnemonicbox}
``Calculate Channels Times Time-slots''

\end{mnemonicbox}
\begin{center}\rule{0.5\linewidth}{0.5pt}\end{center}

\subsection*{Question 3(c OR) [7
marks]}\label{question-3c-or-7-marks}

\textbf{Draw and explain general block diagram of mobile handset.}

\begin{solutionbox}
\textbf{Mobile handset} consists of several functional
blocks working together.

\begin{verbatim}
graph TB
    A[Antenna] {-{-} B[RF Section]}
    B {-{-} C[IF Section]}
    C {-{-} D[Baseband Processor]}
    D {-{-} E[Audio Section]}
    D {-{-} F[Display Unit]}
    D {-{-} G[Keypad]}
    H[Power Management] {-{-} D}
    I[Battery] {-{-} H}
    J[SIM Interface] {-{-} D}
\end{verbatim}


{\def\LTcaptype{none} % do not increment counter
\vspace{-5pt}
\captionof{table}{Mobile Handset Blocks}
\vspace{-10pt}
\begin{longtable}[]{@{}ll@{}}
\toprule\noalign{}
Block & Function \\
\midrule\noalign{}
\endhead
\bottomrule\noalign{}
\endlastfoot
\textbf{RF Section} & Signal transmission and reception \\
\textbf{Baseband} & Digital signal processing \\
\textbf{Audio} & Voice input/output processing \\
\textbf{Power Management} & Battery and power control \\
\textbf{User Interface} & Display, keypad, speaker, microphone \\
\end{longtable}
}

\begin{itemize}
\tightlist
\item
  \textbf{RF processing}: Handles radio frequency transmission and
  reception
\item
  \textbf{Digital processing}: Baseband performs channel coding, speech
  processing
\item
  \textbf{User interface}: Provides interaction through display, keypad,
  audio
\item
  \textbf{Power control}: Manages battery usage and charging functions
\end{itemize}

\end{solutionbox}
\begin{mnemonicbox}
``Mobile Manages Multiple Modules Simultaneously''

\end{mnemonicbox}
\begin{center}\rule{0.5\linewidth}{0.5pt}\end{center}

\subsection*{Question 4(a) [3 marks]}\label{q4a}

\textbf{Write radiation hazards due to mobile.}

\begin{solutionbox}
\textbf{Radiation hazards} from mobile phones are a
health concern due to RF energy exposure.


{\def\LTcaptype{none} % do not increment counter
\vspace{-5pt}
\captionof{table}{Mobile Radiation Hazards}
\vspace{-10pt}
\begin{longtable}[]{@{}lll@{}}
\toprule\noalign{}
Hazard & Effect & Prevention \\
\midrule\noalign{}
\endhead
\bottomrule\noalign{}
\endlastfoot
\textbf{SAR Exposure} & Tissue heating & Use hands-free devices \\
\textbf{Brain Effects} & Memory, sleep issues & Limit call duration \\
\textbf{Cancer Risk} & Potential tumor risk & Keep phone away from
body \\
\end{longtable}
}

\begin{itemize}
\tightlist
\item
  \textbf{SAR (Specific Absorption Rate)}: Measures RF energy absorbed
  by body tissue
\item
  \textbf{Thermal effects}: RF energy can cause localized heating of
  tissues
\item
  \textbf{Non-thermal effects}: Possible impacts on cellular functions
  and DNA
\end{itemize}

\end{solutionbox}
\begin{mnemonicbox}
``Safety Awareness Reduces Radiation Risk''

\end{mnemonicbox}
\begin{center}\rule{0.5\linewidth}{0.5pt}\end{center}

\subsection*{Question 4(b) [4 marks]}\label{q4b}

\textbf{Explain working of baseband section in mobile handset.}

\begin{solutionbox}
\textbf{Baseband section} performs digital signal
processing functions in mobile handset.


{\def\LTcaptype{none} % do not increment counter
\vspace{-5pt}
\captionof{table}{Baseband Section Functions}
\vspace{-10pt}
\begin{longtable}[]{@{}ll@{}}
\toprule\noalign{}
Function & Description \\
\midrule\noalign{}
\endhead
\bottomrule\noalign{}
\endlastfoot
\textbf{Speech Processing} & Encode/decode voice using vocoder \\
\textbf{Channel Coding} & Add error correction and detection \\
\textbf{Modulation} & Convert digital data to analog signals \\
\textbf{Protocol Processing} & Handle signaling and call control \\
\end{longtable}
}

\begin{itemize}
\tightlist
\item
  \textbf{Digital signal processor}: Executes speech coding algorithms
  (GSM: RPE-LTP)
\item
  \textbf{Error correction}: Implements convolutional coding for
  reliable transmission
\item
  \textbf{Control functions}: Manages call setup, handoff, and power
  control
\item
  \textbf{Interface}: Connects RF section with user interface components
\end{itemize}

\end{solutionbox}
\begin{mnemonicbox}
``Baseband Brings Better Communication Control''

\end{mnemonicbox}
\begin{center}\rule{0.5\linewidth}{0.5pt}\end{center}

\subsection*{Question 4(c) [7 marks]}\label{q4c}

\textbf{Explain working of DSSS transmitter and receiver.}

\begin{solutionbox}
\textbf{DSSS (Direct Sequence Spread Spectrum)} spreads
signal bandwidth using pseudorandom codes.

\textbf{Transmitter Diagram:}

\begin{center}
\textbf{Mermaid Diagram (Code)}
\begin{verbatim}
{Shaded}
{Highlighting}[]
graph LR
    A[Data Input] {-{-}{} B[PN Code Generator]}
    A {-{-}{} C[XOR Gate]}
    B {-{-}{} C}
    C {-{-}{} D[Modulator]}
    D {-{-}{} E[RF Output]}
{Highlighting}
{Shaded}
\end{verbatim}
\end{center}

\textbf{Receiver Diagram:}

\begin{center}
\textbf{Mermaid Diagram (Code)}
\begin{verbatim}
{Shaded}
{Highlighting}[]
graph LR
    F[RF Input] {-{-}{} G[Demodulator]}
    G {-{-}{} H[XOR Gate]}
    I[PN Code Generator] {-{-}{} H}
    H {-{-}{} J[Data Output]}
{Highlighting}
{Shaded}
\end{verbatim}
\end{center}


{\def\LTcaptype{none} % do not increment counter
\vspace{-5pt}
\captionof{table}{DSSS Process}
\vspace{-10pt}
\begin{longtable}[]{@{}
  >{\raggedright\arraybackslash}p{(\linewidth - 4\tabcolsep) * \real{0.2333}}
  >{\raggedright\arraybackslash}p{(\linewidth - 4\tabcolsep) * \real{0.4333}}
  >{\raggedright\arraybackslash}p{(\linewidth - 4\tabcolsep) * \real{0.3333}}@{}}
\toprule\noalign{}
\begin{minipage}[b]{\linewidth}\raggedright
Stage
\end{minipage} & \begin{minipage}[b]{\linewidth}\raggedright
Transmitter
\end{minipage} & \begin{minipage}[b]{\linewidth}\raggedright
Receiver
\end{minipage} \\
\midrule\noalign{}
\endhead
\bottomrule\noalign{}
\endlastfoot
\textbf{Spreading} & Data XOR with PN code & Received signal XOR with
PN \\
\textbf{Modulation} & Spread signal modulated & Demodulate received
signal \\
\textbf{Processing} & Bandwidth increased & Original data recovered \\
\end{longtable}
}

\begin{itemize}
\tightlist
\item
  \textbf{Spreading process}: Original data XORed with high-rate
  pseudorandom sequence
\item
  \textbf{Bandwidth expansion}: Signal bandwidth increased by processing
  gain factor
\item
  \textbf{Despreading}: Receiver uses same PN code to recover original
  data
\item
  \textbf{Interference rejection}: Spread spectrum provides resistance
  to jamming
\end{itemize}

\end{solutionbox}
\begin{mnemonicbox}
``Direct Sequence Spreads Signals Successfully''

\end{mnemonicbox}
\begin{center}\rule{0.5\linewidth}{0.5pt}\end{center}

\subsection*{Question 4(a OR) [3
marks]}\label{question-4a-or-3-marks}

\textbf{Calculate processing gain for DSSS system with 10 Mcps chip rate
and 1 Mbps data rate.}

\begin{solutionbox}
\textbf{Processing gain} determines spread spectrum
system's performance improvement.

\textbf{Given:}

\begin{itemize}
\tightlist
\item
  Chip rate (Rc) = 10 million chips per second = 10 \times 10^{6} cps
\item
  Data rate (Rd) = 1 Mbps = 1 \times 10^{6} bps
\end{itemize}

\textbf{Calculation:} Processing Gain (Gp) = Chip rate \div Data rate Gp =
Rc \div Rd = (10 \times 10^{6}) \div (1 \times 10^{6}) = 10

\textbf{In dB:} Gp (dB) = 10 log_{1}_{0}(10) = 10 \times 1 = 10 dB

\end{solutionbox}
\begin{solutionbox}

\end{solutionbox}
\begin{mnemonicbox}
``Processing Power Provides Protection''

\end{mnemonicbox}
\begin{center}\rule{0.5\linewidth}{0.5pt}\end{center}

\subsection*{Question 4(b OR) [4
marks]}\label{question-4b-or-4-marks}

\textbf{Explain how data rate is improved in EDGE.}

\begin{solutionbox}
\textbf{EDGE (Enhanced Data rates for GSM Evolution)}
improves data rates through advanced modulation.


{\def\LTcaptype{none} % do not increment counter
\vspace{-5pt}
\captionof{table}{EDGE Improvements}
\vspace{-10pt}
\begin{longtable}[]{@{}
  >{\raggedright\arraybackslash}p{(\linewidth - 6\tabcolsep) * \real{0.3235}}
  >{\raggedright\arraybackslash}p{(\linewidth - 6\tabcolsep) * \real{0.1471}}
  >{\raggedright\arraybackslash}p{(\linewidth - 6\tabcolsep) * \real{0.1765}}
  >{\raggedright\arraybackslash}p{(\linewidth - 6\tabcolsep) * \real{0.3529}}@{}}
\toprule\noalign{}
\begin{minipage}[b]{\linewidth}\raggedright
Parameter
\end{minipage} & \begin{minipage}[b]{\linewidth}\raggedright
GSM
\end{minipage} & \begin{minipage}[b]{\linewidth}\raggedright
EDGE
\end{minipage} & \begin{minipage}[b]{\linewidth}\raggedright
Improvement
\end{minipage} \\
\midrule\noalign{}
\endhead
\bottomrule\noalign{}
\endlastfoot
\textbf{Modulation} & GMSK & 8-PSK & 3 bits per symbol vs 1 bit \\
\textbf{Data Rate} & 9.6 kbps & 43.2 kbps per slot & \textasciitilde4.5x
increase \\
\textbf{Coding} & Fixed & Adaptive & Link adaptation \\
\textbf{Applications} & Voice, SMS & Multimedia, Internet & Enhanced
services \\
\end{longtable}
}

\begin{itemize}
\tightlist
\item
  \textbf{8-PSK modulation}: Transmits 3 bits per symbol instead of 1
  bit in GMSK
\item
  \textbf{Link adaptation}: Dynamically selects coding scheme based on
  channel quality
\item
  \textbf{Backward compatibility}: Works with existing GSM
  infrastructure
\item
  \textbf{Enhanced applications}: Supports multimedia and higher data
  rate services
\end{itemize}

\end{solutionbox}
\begin{mnemonicbox}
``EDGE Enhances Exchange Efficiently''

\end{mnemonicbox}
\begin{center}\rule{0.5\linewidth}{0.5pt}\end{center}

\subsection*{Question 4(c OR) [7
marks]}\label{question-4c-or-7-marks}

\textbf{Explain call processing in CDMA.}

\begin{solutionbox}
\textbf{CDMA call processing} involves unique
procedures for code-based multiple access.

\begin{center}
\textbf{Mermaid Diagram (Code)}
\begin{verbatim}
{Shaded}
{Highlighting}[]
graph LR
    A[Mobile Power On] {-{-}{} B[Pilot Channel Search]}
    B {-{-}{} C[Sync Channel Read]}
    C {-{-}{} D[Paging Channel Monitor]}
    D {-{-}{} E[Access Channel Request]}
    E {-{-}{} F[Traffic Channel Assignment]}
    F {-{-}{} G[Active Call State]}
    G {-{-}{} H[Soft Handoff]}
{Highlighting}
{Shaded}
\end{verbatim}
\end{center}


{\def\LTcaptype{none} % do not increment counter
\vspace{-5pt}
\captionof{table}{CDMA Call Processing Stages}
\vspace{-10pt}
\begin{longtable}[]{@{}lll@{}}
\toprule\noalign{}
Stage & Process & Function \\
\midrule\noalign{}
\endhead
\bottomrule\noalign{}
\endlastfoot
\textbf{Initialization} & Pilot acquisition & Find strongest base
station \\
\textbf{Idle State} & Monitor paging & Listen for incoming calls \\
\textbf{Access} & Random access & Request service from network \\
\textbf{Traffic} & Dedicated channel & Active communication \\
\textbf{Handoff} & Soft handoff & Seamless cell transition \\
\end{longtable}
}

\begin{itemize}
\tightlist
\item
  \textbf{Pilot channel}: Provides timing reference and system
  identification
\item
  \textbf{Rake receiver}: Combines multipath signals for improved
  performance
\item
  \textbf{Power control}: Maintains optimal signal levels for all users
\item
  \textbf{Soft handoff}: Mobile communicates with multiple base stations
  simultaneously
\item
  \textbf{Code assignment}: Each user assigned unique spreading code
\end{itemize}

\end{solutionbox}
\begin{mnemonicbox}
``CDMA Calls Connect Carefully and Clearly''

\end{mnemonicbox}
\begin{center}\rule{0.5\linewidth}{0.5pt}\end{center}

\subsection*{Question 5(a) [3 marks]}\label{q5a}

\textbf{Compare CDMA and GSM.}

\begin{solutionbox}
\textbf{CDMA} and \textbf{GSM} represent different
approaches to cellular communication.


{\def\LTcaptype{none} % do not increment counter
\vspace{-5pt}
\captionof{table}{CDMA vs GSM Comparison}
\vspace{-10pt}
\begin{longtable}[]{@{}lll@{}}
\toprule\noalign{}
Parameter & CDMA & GSM \\
\midrule\noalign{}
\endhead
\bottomrule\noalign{}
\endlastfoot
\textbf{Access Method} & Code Division & Time/Frequency Division \\
\textbf{Capacity} & Higher & Lower \\
\textbf{Handoff} & Soft handoff & Hard handoff \\
\textbf{Security} & Better (spreading codes) & Good (encryption) \\
\textbf{Global Usage} & Limited & Widespread \\
\textbf{Power Control} & Continuous & Periodic \\
\end{longtable}
}

\begin{itemize}
\tightlist
\item
  \textbf{Multiple access}: CDMA uses unique codes, GSM uses time slots
\item
  \textbf{Call quality}: CDMA provides soft handoff, GSM has hard
  handoff
\end{itemize}

\end{solutionbox}
\begin{mnemonicbox}
``Choose CDMA or GSM Carefully''

\end{mnemonicbox}
\begin{center}\rule{0.5\linewidth}{0.5pt}\end{center}

\subsection*{Question 5(b) [4 marks]}\label{q5b}

\textbf{Write advantages of CDMA.}

\begin{solutionbox}
\textbf{CDMA advantages} make it suitable for
high-capacity cellular systems.


{\def\LTcaptype{none} % do not increment counter
\vspace{-5pt}
\captionof{table}{CDMA Advantages}
\vspace{-10pt}
\begin{longtable}[]{@{}ll@{}}
\toprule\noalign{}
Advantage & Benefit \\
\midrule\noalign{}
\endhead
\bottomrule\noalign{}
\endlastfoot
\textbf{High Capacity} & More users per spectrum \\
\textbf{Soft Handoff} & Seamless call transfer \\
\textbf{Variable Rate} & Adapts to speech patterns \\
\textbf{Privacy} & Inherent security through spreading \\
\textbf{Multipath Resistance} & Uses rake receiver \\
\textbf{Power Control} & Optimizes battery life \\
\textbf{Frequency Planning} & Same frequency in all cells \\
\end{longtable}
}

\begin{itemize}
\tightlist
\item
  \textbf{Spectrum efficiency}: Higher capacity compared to FDMA/TDMA
  systems
\item
  \textbf{Quality advantage}: Soft handoff eliminates call drops during
  cell transitions
\item
  \textbf{Security benefit}: Spread spectrum provides inherent privacy
  protection
\item
  \textbf{Simplified planning}: No frequency reuse planning required
\end{itemize}

\end{solutionbox}
\begin{mnemonicbox}
``CDMA Creates Considerable Communication Capacity''

\end{mnemonicbox}
\begin{center}\rule{0.5\linewidth}{0.5pt}\end{center}

\subsection*{Question 5(c) [7 marks]}\label{q5c}

\textbf{Explain MANET in brief and write its applications.}

\begin{solutionbox}
\textbf{MANET (Mobile Ad Hoc Network)} is
infrastructure-less network of mobile devices.

\begin{center}
\textbf{Mermaid Diagram (Code)}
\begin{verbatim}
{Shaded}
{Highlighting}[]
graph LR
    A[Mobile Node A] {-.{-}{} B[Mobile Node B]}
    B {-.{-}{} C[Mobile Node C]}
    C {-.{-}{} D[Mobile Node D]}
    A {-.{-}{} C}
    B {-.{-}{} D}
    A {-.{-}{} D}

    style A fill:\#f9f
    style B fill:\#9ff
    style C fill:\#ff9
    style D fill:\#9f9
{Highlighting}
{Shaded}
\end{verbatim}
\end{center}


{\def\LTcaptype{none} % do not increment counter
\vspace{-5pt}
\captionof{table}{MANET Characteristics vs Applications}
\vspace{-10pt}
\begin{longtable}[]{@{}
  >{\raggedright\arraybackslash}p{(\linewidth - 4\tabcolsep) * \real{0.4103}}
  >{\raggedright\arraybackslash}p{(\linewidth - 4\tabcolsep) * \real{0.2308}}
  >{\raggedright\arraybackslash}p{(\linewidth - 4\tabcolsep) * \real{0.3590}}@{}}
\toprule\noalign{}
\begin{minipage}[b]{\linewidth}\raggedright
Characteristic
\end{minipage} & \begin{minipage}[b]{\linewidth}\raggedright
Feature
\end{minipage} & \begin{minipage}[b]{\linewidth}\raggedright
Applications
\end{minipage} \\
\midrule\noalign{}
\endhead
\bottomrule\noalign{}
\endlastfoot
\textbf{Self-organizing} & No fixed infrastructure & Military
communications \\
\textbf{Dynamic topology} & Nodes move freely & Emergency response \\
\textbf{Multi-hop routing} & Intermediate node relay & Disaster
recovery \\
\textbf{Distributed control} & No central authority & Sensor networks \\
\textbf{Resource constraints} & Limited battery, bandwidth & Vehicular
networks \\
\end{longtable}
}

\textbf{Applications:}

\begin{itemize}
\tightlist
\item
  \textbf{Military operations}: Battlefield communications without
  infrastructure
\item
  \textbf{Emergency services}: Disaster response and rescue operations
\item
  \textbf{Sensor networks}: Environmental monitoring and data collection
\item
  \textbf{Vehicular networks}: Car-to-car communication for traffic
  management
\item
  \textbf{Personal area networks}: Device-to-device communication
\item
  \textbf{Academic research}: Collaborative computing environments
\end{itemize}

\textbf{Advantages:}

\begin{itemize}
\tightlist
\item
  \textbf{Rapid deployment}: No infrastructure setup required
\item
  \textbf{Self-healing}: Automatic route reconfiguration when nodes fail
\item
  \textbf{Cost effective}: No base station installation costs
\end{itemize}

\textbf{Disadvantages:}

\begin{itemize}
\tightlist
\item
  \textbf{Limited bandwidth}: Shared wireless medium
\item
  \textbf{Security challenges}: Vulnerable to attacks
\item
  \textbf{Power constraints}: Battery-dependent operation
\end{itemize}

\end{solutionbox}
\begin{mnemonicbox}
``Mobile Ad Hoc Networks Enable Everywhere''

\end{mnemonicbox}
\begin{center}\rule{0.5\linewidth}{0.5pt}\end{center}

\subsection*{Question 5(a OR) [3
marks]}\label{question-5a-or-3-marks}

\textbf{Write key features of WCDMA.}

\begin{solutionbox}
\textbf{WCDMA (Wideband CDMA)} is the 3G standard
offering enhanced capabilities.


{\def\LTcaptype{none} % do not increment counter
\vspace{-5pt}
\captionof{table}{WCDMA Key Features}
\vspace{-10pt}
\begin{longtable}[]{@{}ll@{}}
\toprule\noalign{}
Feature & Specification \\
\midrule\noalign{}
\endhead
\bottomrule\noalign{}
\endlastfoot
\textbf{Chip Rate} & 3.84 Mcps \\
\textbf{Bandwidth} & 5 MHz \\
\textbf{Data Rates} & Up to 2 Mbps \\
\textbf{Spreading} & Variable spreading factor \\
\textbf{Power Control} & Fast closed-loop \\
\textbf{Handoff} & Soft and softer handoff \\
\end{longtable}
}

\begin{itemize}
\tightlist
\item
  \textbf{Wideband operation}: 5 MHz bandwidth provides high data rates
\item
  \textbf{Variable spreading}: Adapts to different service requirements
\end{itemize}

\end{solutionbox}
\begin{mnemonicbox}
``WCDMA Widens Communication Data Magnificently''

\end{mnemonicbox}
\begin{center}\rule{0.5\linewidth}{0.5pt}\end{center}

\subsection*{Question 5(b OR) [4
marks]}\label{question-5b-or-4-marks}

\textbf{Enlist advantages of 5G.}

\begin{solutionbox}
\textbf{5G advantages} represent significant
improvements over previous generations.


{\def\LTcaptype{none} % do not increment counter
\vspace{-5pt}
\captionof{table}{5G Advantages}
\vspace{-10pt}
\begin{longtable}[]{@{}ll@{}}
\toprule\noalign{}
Advantage & Benefit \\
\midrule\noalign{}
\endhead
\bottomrule\noalign{}
\endlastfoot
\textbf{Ultra-high Speed} & Up to 20 Gbps peak data rate \\
\textbf{Low Latency} & \textless1ms for critical applications \\
\textbf{Massive IoT} & 1 million devices per km^{2} \\
\textbf{Network Slicing} & Customized virtual networks \\
\textbf{Enhanced Coverage} & Better indoor and edge coverage \\
\textbf{Energy Efficiency} & 100x more efficient than 4G \\
\textbf{High Reliability} & 99.999\% availability \\
\end{longtable}
}

\begin{itemize}
\tightlist
\item
  \textbf{Enhanced mobile broadband}: Supports AR/VR and 4K/8K video
  streaming
\item
  \textbf{Ultra-reliable communications}: Enables autonomous vehicles
  and remote surgery
\item
  \textbf{Massive machine communications}: Supports smart cities and
  Industry 4.0
\item
  \textbf{Flexible network architecture}: Software-defined networking
  capabilities
\end{itemize}

\end{solutionbox}
\begin{mnemonicbox}
``5G Generates Great Gigabit Growth''

\end{mnemonicbox}
\begin{center}\rule{0.5\linewidth}{0.5pt}\end{center}

\subsection*{Question 5(c OR) [7
marks]}\label{question-5c-or-7-marks}

\textbf{Explain working of OFDM with block diagram.}

\begin{solutionbox}
\textbf{OFDM (Orthogonal Frequency Division
Multiplexing)} uses multiple subcarriers for high-speed data
transmission.

\textbf{OFDM Transmitter:}

\begin{center}
\textbf{Mermaid Diagram (Code)}
\begin{verbatim}
{Shaded}
{Highlighting}[]
graph LR
    A[Serial Data] {-{-}{} B[Serial to Parallel]}
    B {-{-}{} C[QAM Mapping]}
    C {-{-}{} D[IFFT]}
    D {-{-}{} E[Add Cyclic Prefix]}
    E {-{-}{} F[Parallel to Serial]}
    F {-{-}{} G[RF Transmission]}
{Highlighting}
{Shaded}
\end{verbatim}
\end{center}

\textbf{OFDM Receiver:}

\begin{center}
\textbf{Mermaid Diagram (Code)}
\begin{verbatim}
{Shaded}
{Highlighting}[]
graph LR
    H[RF Reception] {-{-}{} I[Serial to Parallel]}
    I {-{-}{} J[Remove Cyclic Prefix]}
    J {-{-}{} K[FFT]}
    K {-{-}{} L[QAM Demapping]}
    L {-{-}{} M[Parallel to Serial]}
    M {-{-}{} N[Serial Data]}
{Highlighting}
{Shaded}
\end{verbatim}
\end{center}


{\def\LTcaptype{none} % do not increment counter
\vspace{-5pt}
\captionof{table}{OFDM Process Steps}
\vspace{-10pt}
\begin{longtable}[]{@{}
  >{\raggedright\arraybackslash}p{(\linewidth - 4\tabcolsep) * \real{0.1489}}
  >{\raggedright\arraybackslash}p{(\linewidth - 4\tabcolsep) * \real{0.4468}}
  >{\raggedright\arraybackslash}p{(\linewidth - 4\tabcolsep) * \real{0.4043}}@{}}
\toprule\noalign{}
\begin{minipage}[b]{\linewidth}\raggedright
Stage
\end{minipage} & \begin{minipage}[b]{\linewidth}\raggedright
Transmitter Function
\end{minipage} & \begin{minipage}[b]{\linewidth}\raggedright
Receiver Function
\end{minipage} \\
\midrule\noalign{}
\endhead
\bottomrule\noalign{}
\endlastfoot
\textbf{Data Conversion} & Serial to parallel conversion & Parallel to
serial reconstruction \\
\textbf{Modulation} & QAM mapping on subcarriers & QAM demapping \\
\textbf{Transform} & IFFT creates time domain signal & FFT recovers
frequency domain \\
\textbf{Guard Period} & Cyclic prefix prevents ISI & Cyclic prefix
removal \\
\end{longtable}
}

\textbf{Key Features:}

\begin{itemize}
\tightlist
\item
  \textbf{Orthogonal subcarriers}: Multiple parallel low-rate data
  streams prevent interference
\item
  \textbf{FFT/IFFT processing}: Efficient digital implementation using
  fast transforms
\item
  \textbf{Cyclic prefix}: Guard interval prevents inter-symbol
  interference from multipath
\item
  \textbf{Spectral efficiency}: High data rates achieved in limited
  bandwidth
\item
  \textbf{Multipath resistance}: Individual subcarriers experience flat
  fading
\end{itemize}

\textbf{Applications:}

\begin{itemize}
\tightlist
\item
  \textbf{WiFi (802.11)}: Wireless LAN communications
\item
  \textbf{LTE/4G}: Mobile broadband networks
\item
  \textbf{Digital TV}: DVB-T terrestrial broadcasting
\item
  \textbf{WiMAX}: Broadband wireless access
\end{itemize}

\textbf{Advantages:}

\begin{itemize}
\tightlist
\item
  \textbf{High spectral efficiency}: Optimal bandwidth utilization
\item
  \textbf{Robustness}: Resistant to frequency selective fading
\item
  \textbf{Flexibility}: Adaptive modulation per subcarrier
\item
  \textbf{Implementation}: Digital signal processing simplifies hardware
\end{itemize}


{\def\LTcaptype{none} % do not increment counter
\vspace{-5pt}
\captionof{table}{OFDM Parameters}
\vspace{-10pt}
\begin{longtable}[]{@{}ll@{}}
\toprule\noalign{}
Parameter & Typical Values \\
\midrule\noalign{}
\endhead
\bottomrule\noalign{}
\endlastfoot
\textbf{Subcarriers} & 64, 128, 256, 512, 1024 \\
\textbf{Modulation} & BPSK, QPSK, 16-QAM, 64-QAM \\
\textbf{Cyclic Prefix} & 1/4, 1/8, 1/16 of symbol duration \\
\textbf{Applications} & WiFi, LTE, DVB, WiMAX \\
\end{longtable}
}

\end{solutionbox}
\begin{mnemonicbox}
``OFDM Offers Outstanding Data Multiplexing''

\end{mnemonicbox}
\begin{center}\rule{0.5\linewidth}{0.5pt}\end{center}


\end{document}
