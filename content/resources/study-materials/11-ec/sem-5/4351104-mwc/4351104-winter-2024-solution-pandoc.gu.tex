\documentclass[10pt,a4paper]{article}

% content/resources/templates/preamble.tex
\usepackage[margin=0.6in]{geometry}
\author{Milav Dabgar}
\usepackage{amsmath,amssymb,amsthm}
\usepackage{booktabs}
\usepackage{multirow}
\usepackage{xcolor}
\usepackage{tcolorbox}
\tcbuselibrary{breakable,skins}
\usepackage[colorlinks=true,linkcolor=blue]{hyperref}
\usepackage{titlesec}
\usepackage{enumitem}
\usepackage{tikz}
\usepackage{pgfplots}
\usepackage{circuitikz}
\usepackage[version=4]{mhchem}
\usepackage{longtable}
\usepackage{array}
\usepackage{float}
\usepackage{caption}
\usepackage{listings}

\lstset{
  basicstyle=\small\ttfamily,
  breaklines=true,
  breakatwhitespace=false,
  postbreak=\mbox{\textcolor{red}{$\hookrightarrow$}\space},
  float=false,
  numbers=left,
  numberstyle=\tiny\color{gray},
  numbersep=10pt,
  xleftmargin=2em,
  keywordstyle=\color{blue},
  commentstyle=\color{green!60!black},
  stringstyle=\color{purple},
  backgroundcolor=\color{gray!5},
  showstringspaces=false,
  tabsize=2,
  captionpos=b,
  keepspaces=true,
  columns=flexible
}

\pgfplotsset{compat=1.18}
\usetikzlibrary{shapes,arrows,positioning,calc,patterns,decorations.pathmorphing,decorations.markings,arrows.meta}

% Color scheme
\definecolor{headcolor}{RGB}{0,102,204}
\definecolor{keycolor}{RGB}{220,20,60}
\definecolor{solutioncolor}{RGB}{34,139,34}
\definecolor{mnemoniccolor}{RGB}{148,0,211}
\definecolor{codecolor}{RGB}{0,0,100}

% Spacing
\setlength{\parskip}{3pt}
\setlist[itemize]{nosep}
\setlist[enumerate]{nosep}

% Title formatting
\titleformat{\section}{\Large\bfseries\color{headcolor}}{\thesection}{1em}{}
\titleformat{\subsection}{\large\bfseries\color{headcolor}}{\thesubsection}{1em}{}

% Pandoc tightlist compatibility
\providecommand{\tightlist}{%
  \setlength{\itemsep}{0pt}\setlength{\parskip}{0pt}}

% Pandoc longtable compatibility
\newcounter{none}
\def\thenone{}


% content/resources/templates/gujarati-boxes.tex
\usepackage{fontspec}
\usepackage{polyglossia}

% Set Gujarati as main language (document is primarily in Gujarati)
% Note: gloss-gujarati.ldf doesn't exist in polyglossia, but it will use hyphenation patterns
\setdefaultlanguage{gujarati}
\setotherlanguage{english}

% Configure Gujarati font properly
% Use Language=Default to prevent polyglossia from trying to add language-specific features
% that don't exist for Gujarati, which causes "empty feature" warnings
\newfontfamily\gujaratifont[Script=Gujarati,AutoFakeBold=2.5,AutoFakeSlant=0.3]{Noto Sans Gujarati}
\setmainfont[Script=Gujarati,AutoFakeBold=2.5,AutoFakeSlant=0.3]{Noto Sans Gujarati}
% Use Noto Sans Gujarati for monospace to support Gujarati in text
\setmonofont[Scale=0.9]{Noto Sans Gujarati}

% Configure English to use the same font
\newfontfamily\englishfont[Script=Gujarati,AutoFakeBold=2.5,AutoFakeSlant=0.3]{Noto Sans Gujarati}

% Translations for polyglossia
\gappto\captionsgujarati{
  \renewcommand{\tablename}{કોષ્ટક}
  \renewcommand{\figurename}{આકૃતિ}
}

% Helper for TikZ nodes to ensure Gujarati font
\newcommand{\gu}[1]{{\gujaratifont #1}}

% Custom environments
\newtcolorbox{solutionbox}{
    breakable,
    enhanced,
    colback=solutioncolor!5!white,
    colframe=solutioncolor!75!black,
    fonttitle=\bfseries,
    title=જવાબ
}

\newtcolorbox{solutionboxnobreak}{
 colback=solutioncolor!5!white,
 colframe=solutioncolor!75!black,
 fonttitle=\bfseries,
 title=જવાબ
}

\newtcolorbox{keyformula}{
 breakable,
 enhanced,
 colback=keycolor!5!white,
 colframe=keycolor!75!black,
 fonttitle=\bfseries,
 title=રાસાયણિક સમીકરણ/સૂત્ર
}

\newtcolorbox{mnemonicbox}{
 breakable,
 enhanced,
 colback=mnemoniccolor!5!white,
 colframe=mnemoniccolor!75!black,
 fonttitle=\bfseries,
 title=મેમરી ટ્રીક
}


\begin{document}

\begin{center}
{\Huge\bfseries\color{headcolor} Subject Name (Gujarati)}\\[5pt]
{\LARGE 4351104 -- Winter 2024}\\[3pt]
{\large Semester 1 Study Material}\\[3pt]
{\normalsize\textit{Detailed Solutions and Explanations}}
\end{center}

\vspace{10pt}

\subsection*{પ્રશ્ન 1(અ) [3
ગુણ]}\label{uxaaauxab0uxab6uxaa8-1uxa85-3-uxa97uxaa3}

\textbf{અમ્બ્રેલા સેલ સમજાવો.}

\begin{solutionbox}
\textbf{અમ્બ્રેલા સેલ} મોટા કવરેજ એરિયાનો સેલ છે જે નાના સેલ્સને
ઢાંકીને સતત કવરેજ પૂરું પાડે છે.

\textbf{ટેબલ: અમ્બ્રેલા સેલની લાક્ષણિકતાઓ}

{\def\LTcaptype{none} % do not increment counter
\begin{longtable}[]{@{}ll@{}}
\toprule\noalign{}
લક્ષણ & વર્ણન \\
\midrule\noalign{}
\endhead
\bottomrule\noalign{}
\endlastfoot
\textbf{કવરેજ} & મોટો ભૌગોલિક વિસ્તાર \\
\textbf{હેતુ} & માઇક્રોસેલ્સમાંથી overflow traffic સંભાળવો \\
\textbf{એન્ટેના} & હાઇ-પાવર, ઊંચી જગ્યાએ મૂકેલ \\
\textbf{યુઝર્સ} & ઝડપથી ફરતા વાહનો, emergency calls \\
\end{longtable}
}

\begin{itemize}
\tightlist
\item
  \textbf{મોટું કવરેજ}: હાઇ-પાવર બેઝ સ્ટેશન સાથે વિશાળ ભૌગોલિક વિસ્તાર ઢાંકે છે
\item
  \textbf{Traffic management}: નાના સેલ્સ ભરપૂર હોય ત્યારે calls સંભાળે છે
\item
  \textbf{ગતિશીલતા સપોર્ટ}: બહુવિધ સેલ બાઉન્ડરી પાર કરતા ઝડપી યુઝર્સને સેવા આપે છે
\end{itemize}

\end{solutionbox}
\begin{mnemonicbox}
``Umbrella Covers Large Areas''

\end{mnemonicbox}
\begin{center}\rule{0.5\linewidth}{0.5pt}\end{center}

\subsection*{પ્રશ્ન 1(બ) [4
ગુણ]}\label{uxaaauxab0uxab6uxaa8-1uxaac-4-uxa97uxaa3}

\textbf{સેલ અને ક્લસ્ટર વ્યાખ્યાયિત કરો.}

\begin{solutionbox}
\textbf{સેલ} અને \textbf{ક્લસ્ટર} સેલ્યુલર કોમ્યુનિકેશન સિસ્ટમના
મૂળભૂત ખ્યાલો છે.

\textbf{ટેબલ: સેલ vs ક્લસ્ટર સરખામણી}

{\def\LTcaptype{none} % do not increment counter
\begin{longtable}[]{@{}
  >{\raggedright\arraybackslash}p{(\linewidth - 4\tabcolsep) * \real{0.4348}}
  >{\raggedright\arraybackslash}p{(\linewidth - 4\tabcolsep) * \real{0.2174}}
  >{\raggedright\arraybackslash}p{(\linewidth - 4\tabcolsep) * \real{0.3478}}@{}}
\toprule\noalign{}
\begin{minipage}[b]{\linewidth}\raggedright
પેરામીટર
\end{minipage} & \begin{minipage}[b]{\linewidth}\raggedright
સેલ
\end{minipage} & \begin{minipage}[b]{\linewidth}\raggedright
ક્લસ્ટર
\end{minipage} \\
\midrule\noalign{}
\endhead
\bottomrule\noalign{}
\endlastfoot
\textbf{વ્યાખ્યા} & એક બેઝ સ્ટેશન દ્વારા સેવા આપવામાં આવતો એક કવરેજ વિસ્તાર &
અલગ-અલગ frequencies વાપરતા સેલ્સનું જૂથ \\
\textbf{સાઇઝ} & એન્ટેના પાવર અને interference દ્વારા મર્યાદિત & N સેલ્સ ધરાવે છે
(સામાન્ય રીતે 3, 4, 7, 12) \\
\textbf{Frequency} & ચોક્કસ frequency set વાપરે છે & બધી ઉપલબ્ધ frequencies
એકવાર વાપરે છે \\
\textbf{હેતુ} & ચોક્કસ વિસ્તારને કવરેજ આપવું & Frequency reuse pattern શક્ય
બનાવવું \\
\end{longtable}
}

\begin{itemize}
\tightlist
\item
  \textbf{સેલ}: એક બેઝ સ્ટેશન દ્વારા સેવા આપવામાં આવતો ભૌગોલિક વિસ્તાર
\item
  \textbf{ક્લસ્ટર}: સંપૂર્ણ frequency spectrum વાપરતા પડોશી સેલ્સનું જૂથ
\item
  \textbf{Frequency reuse}: અલગ-અલગ ક્લસ્ટર્સમાં સમાન frequencies ફરીથી
  વાપરી શકાય
\item
  \textbf{Pattern repetition}: ક્લસ્ટર pattern સમગ્ર કવરેજમાં પુનરાવર્તિત થાય છે
\end{itemize}

\end{solutionbox}
\begin{mnemonicbox}
``Cells Cluster for Complete Coverage''

\end{mnemonicbox}
\begin{center}\rule{0.5\linewidth}{0.5pt}\end{center}

\subsection*{પ્રશ્ન 1(ક) [7
ગુણ]}\label{uxaaauxab0uxab6uxaa8-1uxa95-7-uxa97uxaa3}

\textbf{સેલ્યુલર કોમ્યુનિકેશન સિસ્ટમ પાછળના મૂળભૂત ખ્યાલનું વર્ણન કરો.}

\begin{solutionbox}
\textbf{સેલ્યુલર કોમ્યુનિકેશન} સર્વિસ એરિયાને નાના સેલ્સમાં વહેંચીને
spectrum efficiency અને capacity વધારે છે.

\textbf{આકૃતિ:}

\begin{verbatim}
    +{-{-}{-}{-}{-}{-}{-}+{-}{-}{-}{-}{-}{-}{-}+{-}{-}{-}{-}{-}{-}{-}+}
    |   A   |   B   |   C   |
    |  f1   |  f2   |  f3   |
    +{-{-}{-}{-}{-}{-}{-}+{-}{-}{-}{-}{-}{-}{-}+{-}{-}{-}{-}{-}{-}{-}+}
    |   D   |   E   |   F   |
    |  f4   |  f5   |  f6   |
    +{-{-}{-}{-}{-}{-}{-}+{-}{-}{-}{-}{-}{-}{-}+{-}{-}{-}{-}{-}{-}{-}+}
    |   G   |   H   |   I   |
    |  f7   |  f1   |  f2   |
    +{-{-}{-}{-}{-}{-}{-}+{-}{-}{-}{-}{-}{-}{-}+{-}{-}{-}{-}{-}{-}{-}+}
\end{verbatim}

\textbf{ટેબલ: સેલ્યુલર સિસ્ટમના ફાયદા}

{\def\LTcaptype{none} % do not increment counter
\begin{longtable}[]{@{}ll@{}}
\toprule\noalign{}
ખ્યાલ & ફાયદો \\
\midrule\noalign{}
\endhead
\bottomrule\noalign{}
\endlastfoot
\textbf{Frequency Reuse} & સમાન frequencies બહુવાર વાપરી શકાય \\
\textbf{Cell Division} & નાના કવરેજ વિસ્તારો, વધુ capacity \\
\textbf{Handoff} & સેલ્સ વચ્ચે seamless call transfer \\
\textbf{Power Control} & ઓછી interference, લાંબુ battery life \\
\end{longtable}
}

\begin{itemize}
\tightlist
\item
  \textbf{નાના સેલનો ખ્યાલ}: કાર્યક્ષમ કવરેજ માટે સર્વિસ એરિયાને hexagonal સેલ્સમાં
  વહેંચાય છે
\item
  \textbf{Frequency reuse}: મર્યાદિત spectrum યોગ્ય separation સાથે બહુવાર
  વાપરાય છે
\item
  \textbf{બેઝ સ્ટેશન કંટ્રોલ}: દરેક સેલને low-power બેઝ સ્ટેશન દ્વારા સેવા આપવામાં આવે
  છે
\item
  \textbf{Capacity improvement}: એક મોટા કવરેજ વિસ્તાર કરતાં વધુ યુઝર્સને સપોર્ટ
  મળે છે
\item
  \textbf{Interference management}: યોગ્ય સેલ પ્લાનિંગ દ્વારા co-channel
  interference નિયંત્રિત કરાય છે
\end{itemize}

\end{solutionbox}
\begin{mnemonicbox}
``Small Cells Support Spectrum Sharing
Successfully''

\end{mnemonicbox}
\begin{center}\rule{0.5\linewidth}{0.5pt}\end{center}

\subsection*{પ્રશ્ન 1(ક OR) [7
ગુણ]}\label{uxaaauxab0uxab6uxaa8-1uxa95-or-7-uxa97uxaa3}

\textbf{સેલ્યુલર કોમ્યુનિકેશનમાં કો-ચેનલ ઇન્ટર્ફીરન્સ સમજાવો.}

\begin{solutionbox}
\textbf{કો-ચેનલ ઇન્ટર્ફીરન્સ} જ્યારે સમાન frequencies વાપરતા
સેલ્સ ખૂબ નજીક હોય ત્યારે થાય છે.

\begin{center}
\textbf{Mermaid Diagram (Code)}
\begin{verbatim}
{Shaded}
{Highlighting}[]
graph TD
    A[Cell A {- f1] {-}{-}{} B[Interference Zone]}
    C[Cell C {- f1] {-}{-}{} B}
    B {-{-}{} D[Degraded Signal Quality]}
    E[Distance D] {-{-}{} F[Reduced Interference]}
{Highlighting}
{Shaded}
\end{verbatim}
\end{center}

\textbf{ટેબલ: કો-ચેનલ ઇન્ટર્ફીરન્સ પેરામીટર્સ}

{\def\LTcaptype{none} % do not increment counter
\begin{longtable}[]{@{}
  >{\raggedright\arraybackslash}p{(\linewidth - 4\tabcolsep) * \real{0.4762}}
  >{\raggedright\arraybackslash}p{(\linewidth - 4\tabcolsep) * \real{0.2857}}
  >{\raggedright\arraybackslash}p{(\linewidth - 4\tabcolsep) * \real{0.2381}}@{}}
\toprule\noalign{}
\begin{minipage}[b]{\linewidth}\raggedright
પેરામીટર
\end{minipage} & \begin{minipage}[b]{\linewidth}\raggedright
વર્ણન
\end{minipage} & \begin{minipage}[b]{\linewidth}\raggedright
અસર
\end{minipage} \\
\midrule\noalign{}
\endhead
\bottomrule\noalign{}
\endlastfoot
\textbf{Reuse Distance} & કો-ચેનલ સેલ્સ વચ્ચેનું અંતર & વધુ અંતર = ઓછી
interference \\
\textbf{C/I Ratio} & Carrier to Interference ratio & સારી quality માટે \geq
18 dB હોવું જોઈએ \\
\textbf{Cluster Size} & ક્લસ્ટરમાં સેલ્સની સંખ્યા & મોટું ક્લસ્ટર = વધુ
separation \\
\end{longtable}
}

\begin{itemize}
\tightlist
\item
  \textbf{Signal overlap}: અલગ સેલ્સના સમાન frequency signals interfere કરે
  છે
\item
  \textbf{Quality degradation}: call drops અને ખરાબ voice quality નું કારણ
  બને છે
\item
  \textbf{Distance factor}: અંતરના વર્ગના પ્રમાણમાં interference ઘટે છે
\item
  \textbf{ઘટાડવાની પદ્ધતિઓ}: યોગ્ય સેલ પ્લાનિંગ, power control, antenna
  design
\end{itemize}

\end{solutionbox}
\begin{mnemonicbox}
``Co-channel Causes Call Quality Concerns''

\end{mnemonicbox}
\begin{center}\rule{0.5\linewidth}{0.5pt}\end{center}

\subsection*{પ્રશ્ન 2(અ) [3
ગુણ]}\label{uxaaauxab0uxab6uxaa8-2uxa85-3-uxa97uxaa3}

\textbf{સેલ સ્પ્લિટિંગ સમજાવો.}

\begin{solutionbox}
\textbf{સેલ સ્પ્લિટિંગ} ભીડવાળા સેલ્સને નાના સેલ્સમાં વહેંચીને સિસ્ટમ
capacity વધારે છે.

\textbf{આકૃતિ:}

\begin{verbatim}
Original Large Cell          After Cell Splitting
    +{-{-}{-}{-}{-}{-}{-}+                   +{-}{-}{-}+{-}{-}{-}+}
    |       |                   | A | B |
    |   X   |                  +{-{-}{-}+{-}{-}{-}+}
    |       |                   | C | D |
    +{-{-}{-}{-}{-}{-}{-}+                   +{-}{-}{-}+{-}{-}{-}+}
\end{verbatim}

\begin{itemize}
\tightlist
\item
  \textbf{Capacity વધારો}: દરેક નવો સેલ ઓછા યુઝર્સને બેહતર સર્વિસ quality સાથે
  handle કરે છે
\item
  \textbf{Power ઘટાડો}: નવા બેઝ સ્ટેશન્સ નાના વિસ્તારોને ઢાંકવા માટે ઓછી power
  વાપરે છે
\item
  \textbf{Frequency management}: મૂળ frequencies નવા નાના સેલ્સમાં વહેંચાય છે
\end{itemize}

\end{solutionbox}
\begin{mnemonicbox}
``Split Cells Serve Subscribers Successfully''

\end{mnemonicbox}
\begin{center}\rule{0.5\linewidth}{0.5pt}\end{center}

\subsection*{પ્રશ્ન 2(બ) [4
ગુણ]}\label{uxaaauxab0uxab6uxaa8-2uxaac-4-uxa97uxaa3}

\textbf{ચેનલ વહેંચણીની વ્યૂહરચના સમજાવો.}

\begin{solutionbox}
\textbf{ચેનલ assignment} વ્યૂહરચનાઓ નક્કી કરે છે કે optimal
performance માટે સેલ્સને frequencies કેવી રીતે ફાળવવી.

\textbf{ટેબલ: ચેનલ Assignment વ્યૂહરચનાઓ}

{\def\LTcaptype{none} % do not increment counter
\begin{longtable}[]{@{}
  >{\raggedright\arraybackslash}p{(\linewidth - 6\tabcolsep) * \real{0.3214}}
  >{\raggedright\arraybackslash}p{(\linewidth - 6\tabcolsep) * \real{0.2143}}
  >{\raggedright\arraybackslash}p{(\linewidth - 6\tabcolsep) * \real{0.2143}}
  >{\raggedright\arraybackslash}p{(\linewidth - 6\tabcolsep) * \real{0.2500}}@{}}
\toprule\noalign{}
\begin{minipage}[b]{\linewidth}\raggedright
વ્યૂહરચના
\end{minipage} & \begin{minipage}[b]{\linewidth}\raggedright
વર્ણન
\end{minipage} & \begin{minipage}[b]{\linewidth}\raggedright
ફાયદા
\end{minipage} & \begin{minipage}[b]{\linewidth}\raggedright
નુકસાન
\end{minipage} \\
\midrule\noalign{}
\endhead
\bottomrule\noalign{}
\endlastfoot
\textbf{Fixed} & સેલ્સને કાયમી ચેનલ્સ ફાળવવા & સરળ, અનુમાનિત & ઓછા traffic
દરમિયાન બિનકાર્યક્ષમ \\
\textbf{Dynamic} & demand પર આધારિત ચેનલ assignment & કાર્યક્ષમ spectrum
વપરાશ & જટિલ implementation \\
\textbf{Hybrid} & Fixed અને dynamic નું મિશ્રણ & સંતુલિત approach & મધ્યમ
જટિલતા \\
\end{longtable}
}

\begin{itemize}
\tightlist
\item
  \textbf{Fixed assignment}: દરેક સેલને પૂર્વનિર્ધારિત ચેનલ્સનો સેટ હોય છે
\item
  \textbf{Dynamic assignment}: traffic demand પર આધારિત real-time માં
  ચેનલ્સ ફાળવાય છે
\item
  \textbf{Load balancing}: ઉપલબ્ધ ચેનલ્સમાં traffic સમાનરૂપે વહેંચાય છે
\item
  \textbf{Interference avoidance}: assignment માં co-channel interference
  ધ્યાનમાં લેવાય છે
\end{itemize}

\end{solutionbox}
\begin{mnemonicbox}
``Dynamic Distribution Delivers Optimal
Performance''

\end{mnemonicbox}
\begin{center}\rule{0.5\linewidth}{0.5pt}\end{center}

\subsection*{પ્રશ્ન 2(ક) [7
ગુણ]}\label{uxaaauxab0uxab6uxaa8-2uxa95-7-uxa97uxaa3}

\textbf{33MHz bandwidth, 25KHz simplex channels, 7-cell reuse, 1MHz
control માટે સેલ દીઠ voice અને control channels ની ગણતરી કરો.}

\begin{solutionbox}
સેલ્યુલર સિસ્ટમમાં \textbf{ચેનલ allocation} માટે ગણતરી.

\textbf{આપેલ ડેટા:}

\begin{itemize}
\tightlist
\item
  Total bandwidth = 33 MHz
\item
  Channel bandwidth = 25 KHz (simplex)
\item
  Full duplex માટે જરૂરી = 2 \times 25 KHz = 50 KHz
\item
  Control spectrum = 1 MHz
\item
  Cluster size = 7 cells
\end{itemize}

\textbf{ગણતરીઓ:}

\textbf{પગલું 1: કુલ ઉપલબ્ધ ચેનલ્સ} Total channels = 33 MHz \div 25 KHz = 1320
channels

\textbf{પગલું 2: Control channels} Control channels = 1 MHz \div 25 KHz = 40
channels

\textbf{પગલું 3: Voice channels} Voice channels = 1320 - 40 = 1280
channels

\textbf{પગલું 4: Duplex voice channels} Duplex voice channels = 1280 \div 2 =
640 channels

\textbf{પગલું 5: સેલ દીઠ ચેનલ્સ} Voice channels per cell = 640 \div 7 \approx 91
channels Control channels per cell = 40 \div 7 \approx 6 channels

\textbf{અંતિમ જવાબ:}

\begin{itemize}
\tightlist
\item
  \textbf{સેલ દીઠ Voice channels: 91}
\item
  \textbf{સેલ દીઠ Control channels: 6}
\end{itemize}

\end{solutionbox}
\begin{mnemonicbox}
``Calculate Carefully for Channel Count''

\end{mnemonicbox}
\begin{center}\rule{0.5\linewidth}{0.5pt}\end{center}

\subsection*{પ્રશ્ન 2(અ OR) [3
ગુણ]}\label{uxaaauxab0uxab6uxaa8-2uxa85-or-3-uxa97uxaa3}

\textbf{GSM માં FCCH અને SCH ના કાર્યો લખો.}

\begin{solutionbox}
\textbf{FCCH} અને \textbf{SCH} synchronization માટે GSM
સિસ્ટમમાં જરૂરી control channels છે.

\textbf{ટેબલ: FCCH અને SCH કાર્યો}

{\def\LTcaptype{none} % do not increment counter
\begin{longtable}[]{@{}
  >{\raggedright\arraybackslash}p{(\linewidth - 4\tabcolsep) * \real{0.2609}}
  >{\raggedright\arraybackslash}p{(\linewidth - 4\tabcolsep) * \real{0.4783}}
  >{\raggedright\arraybackslash}p{(\linewidth - 4\tabcolsep) * \real{0.2609}}@{}}
\toprule\noalign{}
\begin{minipage}[b]{\linewidth}\raggedright
ચેનલ
\end{minipage} & \begin{minipage}[b]{\linewidth}\raggedright
Full Form
\end{minipage} & \begin{minipage}[b]{\linewidth}\raggedright
કાર્ય
\end{minipage} \\
\midrule\noalign{}
\endhead
\bottomrule\noalign{}
\endlastfoot
\textbf{FCCH} & Frequency Correction Channel & Mobile ને frequency
reference પૂરું પાડે છે \\
\textbf{SCH} & Synchronization Channel & Timing અને cell identity પૂરું પાડે
છે \\
\end{longtable}
}

\begin{itemize}
\tightlist
\item
  \textbf{FCCH કાર્ય}: Mobile ને બેઝ સ્ટેશન frequency સાથે synchronize કરવામાં
  મદદ કરે છે
\item
  \textbf{SCH કાર્ય}: BSIC (Base Station Identity Code) અને frame number
  વહન કરે છે
\item
  \textbf{Timing correction}: બંને ચેનલ્સ mobile ને યોગ્ય timing
  synchronization મેળવવામાં મદદ કરે છે
\end{itemize}

\end{solutionbox}
\begin{mnemonicbox}
``FCCH Fixes Frequency, SCH Synchronizes System''

\end{mnemonicbox}
\begin{center}\rule{0.5\linewidth}{0.5pt}\end{center}

\subsection*{પ્રશ્ન 2(બ OR) [4
ગુણ]}\label{uxaaauxab0uxab6uxaa8-2uxaac-or-4-uxa97uxaa3}

\textbf{GSM 900 specifications લખો.}

\begin{solutionbox}
\textbf{GSM 900} 900 MHz frequency band માં ચોક્કસ તકનીકી
પેરામીટર્સ સાથે કાર્ય કરે છે.

\textbf{ટેબલ: GSM 900 Specifications}

{\def\LTcaptype{none} % do not increment counter
\begin{longtable}[]{@{}ll@{}}
\toprule\noalign{}
પેરામીટર & Specification \\
\midrule\noalign{}
\endhead
\bottomrule\noalign{}
\endlastfoot
\textbf{Uplink Frequency} & 890-915 MHz \\
\textbf{Downlink Frequency} & 935-960 MHz \\
\textbf{Duplex Separation} & 45 MHz \\
\textbf{Channel Spacing} & 200 KHz \\
\textbf{Total Channels} & 124 channels \\
\textbf{Access Method} & TDMA/FDMA \\
\textbf{Modulation} & GMSK \\
\textbf{Power Classes} & 2W, 8W, 20W \\
\end{longtable}
}

\begin{itemize}
\tightlist
\item
  \textbf{Frequency bands}: Full duplex operation માટે અલગ uplink અને
  downlink frequencies
\item
  \textbf{TDMA structure}: દરેક carrier frequency પર 8 time slots
\end{itemize}

\end{solutionbox}
\begin{mnemonicbox}
``GSM 900 Gives Great Global Coverage''

\end{mnemonicbox}
\begin{center}\rule{0.5\linewidth}{0.5pt}\end{center}

\subsection*{પ્રશ્ન 2(ક OR) [7
ગુણ]}\label{uxaaauxab0uxab6uxaa8-2uxa95-or-7-uxa97uxaa3}

\textbf{GSM આર્કિટેક્ચર દોરો અને સમજાવો.}

\begin{solutionbox}
\textbf{GSM આર્કિટેક્ચર} mobile communication માટે સાથે કાર્ય
કરતા ત્રણ મુખ્ય subsystems ધરાવે છે.

\begin{verbatim}
graph TB
    MS[Mobile Station] {-{-} BSS[Base Station Subsystem]}
    BSS {-{-} NSS[Network Switching Subsystem]}
    BSS {-{-} BTS[Base Transceiver Station]}
    BSS {-{-} BSC[Base Station Controller]}
    NSS {-{-} MSC[Mobile Switching Center]}
    NSS {-{-} HLR[Home Location Register]}
    NSS {-{-} VLR[Visitor Location Register]}
    NSS {-{-} AuC[Authentication Center]}
    MSC {-{-} PSTN[Public Switched Telephone Network]}
\end{verbatim}

\textbf{ટેબલ: GSM આર્કિટેક્ચર Components}

{\def\LTcaptype{none} % do not increment counter
\begin{longtable}[]{@{}
  >{\raggedright\arraybackslash}p{(\linewidth - 4\tabcolsep) * \real{0.3793}}
  >{\raggedright\arraybackslash}p{(\linewidth - 4\tabcolsep) * \real{0.4138}}
  >{\raggedright\arraybackslash}p{(\linewidth - 4\tabcolsep) * \real{0.2069}}@{}}
\toprule\noalign{}
\begin{minipage}[b]{\linewidth}\raggedright
Subsystem
\end{minipage} & \begin{minipage}[b]{\linewidth}\raggedright
Components
\end{minipage} & \begin{minipage}[b]{\linewidth}\raggedright
કાર્ય
\end{minipage} \\
\midrule\noalign{}
\endhead
\bottomrule\noalign{}
\endlastfoot
\textbf{Mobile Station} & Mobile Equipment + SIM & User interface અને
identity \\
\textbf{BSS} & BTS + BSC & Radio interface અને control \\
\textbf{NSS} & MSC, HLR, VLR, AuC & Switching અને database management \\
\end{longtable}
}

\begin{itemize}
\tightlist
\item
  \textbf{Mobile Station}: યુઝર identification માટે mobile equipment અને
  SIM card ધરાવે છે
\item
  \textbf{Base Station Subsystem}: Radio communication અને resource
  management handle કરે છે
\item
  \textbf{Network Switching Subsystem}: Call switching, routing, અને
  subscriber databases manage કરે છે
\item
  \textbf{Interfaces}: A-bis (BTS-BSC), A (BSC-MSC) interfaces
  subsystems ને connect કરે છે
\end{itemize}

\end{solutionbox}
\begin{mnemonicbox}
``Mobile Base Network - Complete Communication
Chain''

\end{mnemonicbox}
\begin{center}\rule{0.5\linewidth}{0.5pt}\end{center}

\subsection*{પ્રશ્ન 3(અ) [3
ગુણ]}\label{uxaaauxab0uxab6uxaa8-3uxa85-3-uxa97uxaa3}

\textbf{GSM માં signal processing નો block diagram દોરો.}

\begin{solutionbox}
GSM માં \textbf{signal processing} voice અને data
transmission માટે અનેક stages ધરાવે છે.

\textbf{આકૃતિ:}

\begin{verbatim}
Speech  Speech  Channel  Interleaving  Burst  RF
Input    Coding    Coding              Formatting  Processing
  ↓        ↓         ↓         ↓           ↓         ↓
13kbps  22.8kbps  Error  Reordering  Time  Modulation
                  Protection              Slot    \& Transmission
\end{verbatim}

\begin{itemize}
\tightlist
\item
  \textbf{Speech coding}: RPE-LTP વાપરીને analog speech ને 13 kbps digital
  data માં convert કરે છે
\item
  \textbf{Channel coding}: Error correction bits ઉમેરીને rate 22.8 kbps
  સુધી વધારે છે
\item
  \textbf{Interleaving}: Fading થી burst errors સામે લડવા માટે data ફરીથી
  order કરે છે
\end{itemize}

\end{solutionbox}
\begin{mnemonicbox}
``Speech Signals Systematically Processed
Successfully''

\end{mnemonicbox}
\begin{center}\rule{0.5\linewidth}{0.5pt}\end{center}

\subsection*{પ્રશ્ન 3(બ) [4
ગુણ]}\label{uxaaauxab0uxab6uxaa8-3uxaac-4-uxa97uxaa3}

\textbf{GSM માં Common Control Channels ના કાર્યો લખો.}

\begin{solutionbox}
\textbf{Common Control Channels} GSM માં system
information અને access procedures manage કરે છે.

\textbf{ટેબલ: Common Control Channels કાર્યો}

{\def\LTcaptype{none} % do not increment counter
\begin{longtable}[]{@{}ll@{}}
\toprule\noalign{}
ચેનલ & કાર્ય \\
\midrule\noalign{}
\endhead
\bottomrule\noalign{}
\endlastfoot
\textbf{FCCH} & Frequency correction અને synchronization \\
\textbf{SCH} & Frame synchronization અને cell identification \\
\textbf{BCCH} & System information અને cell parameters broadcast કરે છે \\
\textbf{RACH} & Mobile દ્વારા call initiation માટે random access \\
\textbf{AGCH} & Mobiles ને dedicated channels assign કરે છે \\
\textbf{PCH} & Incoming calls માટે mobiles ને page કરે છે \\
\end{longtable}
}

\begin{itemize}
\tightlist
\item
  \textbf{Broadcast કાર્ય}: BCCH સતત system information transmit કરે છે
\item
  \textbf{Access management}: RACH mobiles ને service request કરવાની મંજૂરી
  આપે છે
\item
  \textbf{Channel assignment}: AGCH active calls માટે resources allocate
  કરે છે
\item
  \textbf{Paging service}: PCH mobiles ને incoming calls ની જાણ કરે છે
\end{itemize}

\end{solutionbox}
\begin{mnemonicbox}
``Common Channels Control Communication Completely''

\end{mnemonicbox}
\begin{center}\rule{0.5\linewidth}{0.5pt}\end{center}

\subsection*{પ્રશ્ન 3(ક) [7
ગુણ]}\label{uxaaauxab0uxab6uxaa8-3uxa95-7-uxa97uxaa3}

\textbf{GSM આઇડેન્ટિફાયર્સ સમજાવો.}

\begin{solutionbox}
\textbf{GSM identifiers} subscribers, equipment, અને
network elements ને uniquely identify કરે છે.

\textbf{ટેબલ: GSM Identifiers}

{\def\LTcaptype{none} % do not increment counter
\begin{longtable}[]{@{}
  >{\raggedright\arraybackslash}p{(\linewidth - 6\tabcolsep) * \real{0.3158}}
  >{\raggedright\arraybackslash}p{(\linewidth - 6\tabcolsep) * \real{0.2895}}
  >{\raggedright\arraybackslash}p{(\linewidth - 6\tabcolsep) * \real{0.1579}}
  >{\raggedright\arraybackslash}p{(\linewidth - 6\tabcolsep) * \real{0.2368}}@{}}
\toprule\noalign{}
\begin{minipage}[b]{\linewidth}\raggedright
Identifier
\end{minipage} & \begin{minipage}[b]{\linewidth}\raggedright
Full Form
\end{minipage} & \begin{minipage}[b]{\linewidth}\raggedright
હેતુ
\end{minipage} & \begin{minipage}[b]{\linewidth}\raggedright
Format
\end{minipage} \\
\midrule\noalign{}
\endhead
\bottomrule\noalign{}
\endlastfoot
\textbf{IMSI} & International Mobile Subscriber Identity & Unique
subscriber ID & 15 digits \\
\textbf{IMEI} & International Mobile Equipment Identity & Unique
equipment ID & 15 digits \\
\textbf{MSISDN} & Mobile Station ISDN Number & Phone number & Variable
length \\
\textbf{TMSI} & Temporary Mobile Subscriber Identity & Security માટે
temporary ID & 32 bits \\
\textbf{LAI} & Location Area Identity & Geographic area identification &
MCC+MNC+LAC \\
\textbf{BSIC} & Base Station Identity Code & Cell identification & 6
bits \\
\end{longtable}
}

\begin{itemize}
\tightlist
\item
  \textbf{IMSI structure}: MCC (3) + MNC (2-3) + MSIN (9-10 digits)
\item
  \textbf{Security હેતુ}: TMSI radio interface પર subscriber identity ની
  સુરક્ષા કરે છે
\item
  \textbf{Location management}: LAI કાર્યક્ષમ paging અને location updates
  માં મદદ કરે છે
\item
  \textbf{Network planning}: BSIC પડોશી સેલ્સ વચ્ચે confusion અટકાવે છે
\end{itemize}

\end{solutionbox}
\begin{mnemonicbox}
``Important Mobile System Identifiers Ensure
Security''

\end{mnemonicbox}
\begin{center}\rule{0.5\linewidth}{0.5pt}\end{center}

\subsection*{પ્રશ્ન 3(અ OR) [3
ગુણ]}\label{uxaaauxab0uxab6uxaa8-3uxa85-or-3-uxa97uxaa3}

\textbf{ઝડપી અને ધીમી frequency hopping ની તુલના કરો.}

\begin{solutionbox}
\textbf{Frequency hopping} techniques symbol rate ના
સંબંધમાં hopping rate માં અલગ પડે છે.

\textbf{ટેબલ: Fast vs Slow Frequency Hopping}

{\def\LTcaptype{none} % do not increment counter
\begin{longtable}[]{@{}lll@{}}
\toprule\noalign{}
પેરામીટર & Fast Hopping & Slow Hopping \\
\midrule\noalign{}
\endhead
\bottomrule\noalign{}
\endlastfoot
\textbf{Hopping Rate} & \textgreater{} Symbol rate & \textless{} Symbol
rate \\
\textbf{Symbols per Hop} & \textless{} 1 & \textgreater{} 1 \\
\textbf{જટિલતા} & ઊંચી & નીચી \\
\textbf{Applications} & Military, Bluetooth & GSM, CDMA \\
\end{longtable}
}

\begin{itemize}
\tightlist
\item
  \textbf{Fast hopping}: પ્રતિ symbol બહુવિધ hops, બેહતર security પણ વધુ
  જટિલ
\item
  \textbf{Slow hopping}: પ્રતિ hop બહુવિધ symbols, સરળ implementation
\end{itemize}

\end{solutionbox}
\begin{mnemonicbox}
``Fast Frequently Flips, Slow Stays Stable''

\end{mnemonicbox}
\begin{center}\rule{0.5\linewidth}{0.5pt}\end{center}

\subsection*{પ્રશ્ન 3(બ OR) [4
ગુણ]}\label{uxaaauxab0uxab6uxaa8-3uxaac-or-4-uxa97uxaa3}

\textbf{Frequency reuse નો ઉપયોગ કર્યા વિના GSM 900 band માં એકસાથે વાત કરી
શકે તેવા વપરાશકર્તાઓની સંખ્યાની ગણતરી કરો.}

\begin{solutionbox}
Frequency reuse વિના GSM 900 માં મહત્તમ યુઝર્સ માટે
\textbf{ગણતરી}.

\textbf{આપેલ GSM 900 પેરામીટર્સ:}

\begin{itemize}
\tightlist
\item
  Uplink: 890-915 MHz (25 MHz)
\item
  Downlink: 935-960 MHz (25 MHz)
\item
  Channel spacing: 200 KHz
\item
  પ્રતિ ચેનલ time slots: 8
\end{itemize}

\textbf{ગણતરીઓ:}

\textbf{પગલું 1: ઉપલબ્ધ ચેનલ્સ} Total channels = 25 MHz \div 200 KHz = 125
channels

\textbf{પગલું 2: વાપરી શકાય તેવા ચેનલ્સ} Guard channels કાઢ્યા પછી \approx 124
channels

\textbf{પગલું 3: એકસાથે યુઝર્સ} પ્રતિ ચેનલ યુઝર્સ = 8 time slots કુલ યુઝર્સ = 124 \times
8 = 992 યુઝર્સ

\end{solutionbox}
\begin{solutionbox}

\end{solutionbox}
\begin{mnemonicbox}
``Calculate Channels Times Time-slots''

\end{mnemonicbox}
\begin{center}\rule{0.5\linewidth}{0.5pt}\end{center}

\subsection*{પ્રશ્ન 3(ક OR) [7
ગુણ]}\label{uxaaauxab0uxab6uxaa8-3uxa95-or-7-uxa97uxaa3}

\textbf{મોબાઇલ હેન્ડસેટનો સામાન્ય block diagram દોરો અને સમજાવો.}

\begin{solutionbox}
\textbf{મોબાઇલ હેન્ડસેટ} સાથે કાર્ય કરતા અનેક functional blocks
ધરાવે છે.

\begin{verbatim}
graph TB
    A[Antenna] {-{-} B[RF Section]}
    B {-{-} C[IF Section]}
    C {-{-} D[Baseband Processor]}
    D {-{-} E[Audio Section]}
    D {-{-} F[Display Unit]}
    D {-{-} G[Keypad]}
    H[Power Management] {-{-} D}
    I[Battery] {-{-} H}
    J[SIM Interface] {-{-} D}
\end{verbatim}

\textbf{ટેબલ: મોબાઇલ હેન્ડસેટ બ્લોક્સ}

{\def\LTcaptype{none} % do not increment counter
\begin{longtable}[]{@{}ll@{}}
\toprule\noalign{}
બ્લોક & કાર્ય \\
\midrule\noalign{}
\endhead
\bottomrule\noalign{}
\endlastfoot
\textbf{RF Section} & Signal transmission અને reception \\
\textbf{Baseband} & Digital signal processing \\
\textbf{Audio} & Voice input/output processing \\
\textbf{Power Management} & Battery અને power control \\
\textbf{User Interface} & Display, keypad, speaker, microphone \\
\end{longtable}
}

\begin{itemize}
\tightlist
\item
  \textbf{RF processing}: Radio frequency transmission અને reception
  handle કરે છે
\item
  \textbf{Digital processing}: Baseband channel coding, speech
  processing કરે છે
\item
  \textbf{User interface}: Display, keypad, audio દ્વારા interaction પૂરું
  પાડે છે
\item
  \textbf{Power control}: Battery usage અને charging functions manage કરે
  છે
\end{itemize}

\end{solutionbox}
\begin{mnemonicbox}
``Mobile Manages Multiple Modules Simultaneously''

\end{mnemonicbox}
\begin{center}\rule{0.5\linewidth}{0.5pt}\end{center}

\subsection*{પ્રશ્ન 4(અ) [3
ગુણ]}\label{uxaaauxab0uxab6uxaa8-4uxa85-3-uxa97uxaa3}

\textbf{મોબાઈલના કારણે રેડિયેશનના જોખમો લખો.}

\begin{solutionbox}
મોબાઇલ ફોનમાંથી \textbf{રેડિયેશન જોખમો} RF energy exposure ને
કારણે આરોગ્યની ચિંતા છે.

\textbf{ટેબલ: મોબાઇલ રેડિયેશન જોખમો}

{\def\LTcaptype{none} % do not increment counter
\begin{longtable}[]{@{}
  >{\raggedright\arraybackslash}p{(\linewidth - 4\tabcolsep) * \real{0.3158}}
  >{\raggedright\arraybackslash}p{(\linewidth - 4\tabcolsep) * \real{0.2632}}
  >{\raggedright\arraybackslash}p{(\linewidth - 4\tabcolsep) * \real{0.4211}}@{}}
\toprule\noalign{}
\begin{minipage}[b]{\linewidth}\raggedright
જોખમ
\end{minipage} & \begin{minipage}[b]{\linewidth}\raggedright
અસર
\end{minipage} & \begin{minipage}[b]{\linewidth}\raggedright
રોકથામ
\end{minipage} \\
\midrule\noalign{}
\endhead
\bottomrule\noalign{}
\endlastfoot
\textbf{SAR Exposure} & Tissue heating & Hands-free devices વાપરો \\
\textbf{મગજ પર અસર} & Memory, sleep ની સમસ્યાઓ & Call duration મર્યાદિત
રાખો \\
\textbf{કેન્સરનું જોખમ} & સંભવિત tumor નું જોખમ & ફોન શરીરથી દૂર રાખો \\
\end{longtable}
}

\begin{itemize}
\tightlist
\item
  \textbf{SAR (Specific Absorption Rate)}: શરીરના tissue દ્વારા absorbed
  RF energy માપે છે
\item
  \textbf{Thermal effects}: RF energy tissue ના localized heating નું કારણ
  બની શકે છે
\item
  \textbf{Non-thermal effects}: Cellular functions અને DNA પર સંભવિત અસરો
\end{itemize}

\end{solutionbox}
\begin{mnemonicbox}
``Safety Awareness Reduces Radiation Risk''

\end{mnemonicbox}
\begin{center}\rule{0.5\linewidth}{0.5pt}\end{center}

\subsection*{પ્રશ્ન 4(બ) [4
ગુણ]}\label{uxaaauxab0uxab6uxaa8-4uxaac-4-uxa97uxaa3}

\textbf{મોબાઈલ હેન્ડસેટમાં બેઝબેન્ડ વિભાગની કામગીરી સમજાવો.}

\begin{solutionbox}
\textbf{બેઝબેન્ડ વિભાગ} મોબાઇલ હેન્ડસેટમાં digital signal
processing કાર્યો કરે છે.

\textbf{ટેબલ: બેઝબેન્ડ વિભાગના કાર્યો}

{\def\LTcaptype{none} % do not increment counter
\begin{longtable}[]{@{}ll@{}}
\toprule\noalign{}
કાર્ય & વર્ણન \\
\midrule\noalign{}
\endhead
\bottomrule\noalign{}
\endlastfoot
\textbf{Speech Processing} & Vocoder વાપરીને voice encode/decode કરે છે \\
\textbf{Channel Coding} & Error correction અને detection ઉમેરે છે \\
\textbf{Modulation} & Digital data ને analog signals માં convert કરે છે \\
\textbf{Protocol Processing} & Signaling અને call control handle કરે છે \\
\end{longtable}
}

\begin{itemize}
\tightlist
\item
  \textbf{Digital signal processor}: Speech coding algorithms execute કરે
  છે (GSM: RPE-LTP)
\item
  \textbf{Error correction}: વિશ્વસનીય transmission માટે convolutional
  coding implement કરે છે
\item
  \textbf{Control functions}: Call setup, handoff, અને power control
  manage કરે છે
\item
  \textbf{Interface}: RF section ને user interface components સાથે connect
  કરે છે
\end{itemize}

\end{solutionbox}
\begin{mnemonicbox}
``Baseband Brings Better Communication Control''

\end{mnemonicbox}
\begin{center}\rule{0.5\linewidth}{0.5pt}\end{center}

\subsection*{પ્રશ્ન 4(ક) [7
ગુણ]}\label{uxaaauxab0uxab6uxaa8-4uxa95-7-uxa97uxaa3}

\textbf{DSSS ટ્રાન્સમીટર અને રીસીવરની કામગીરી સમજાવો.}

\begin{solutionbox}
\textbf{DSSS (Direct Sequence Spread Spectrum)}
pseudorandom codes વાપરીને signal bandwidth spread કરે છે.

\textbf{ટ્રાન્સમીટર આકૃતિ:}

\begin{center}
\textbf{Mermaid Diagram (Code)}
\begin{verbatim}
{Shaded}
{Highlighting}[]
graph LR
    A[Data Input] {-{-}{} B[PN Code Generator]}
    A {-{-}{} C[XOR Gate]}
    B {-{-}{} C}
    C {-{-}{} D[Modulator]}
    D {-{-}{} E[RF Output]}
{Highlighting}
{Shaded}
\end{verbatim}
\end{center}

\textbf{રીસીવર આકૃતિ:}

\begin{center}
\textbf{Mermaid Diagram (Code)}
\begin{verbatim}
{Shaded}
{Highlighting}[]
graph LR
    F[RF Input] {-{-}{} G[Demodulator]}
    G {-{-}{} H[XOR Gate]}
    I[PN Code Generator] {-{-}{} H}
    H {-{-}{} J[Data Output]}
{Highlighting}
{Shaded}
\end{verbatim}
\end{center}

\textbf{ટેબલ: DSSS પ્રક્રિયા}

{\def\LTcaptype{none} % do not increment counter
\begin{longtable}[]{@{}
  >{\raggedright\arraybackslash}p{(\linewidth - 4\tabcolsep) * \real{0.2400}}
  >{\raggedright\arraybackslash}p{(\linewidth - 4\tabcolsep) * \real{0.4400}}
  >{\raggedright\arraybackslash}p{(\linewidth - 4\tabcolsep) * \real{0.3200}}@{}}
\toprule\noalign{}
\begin{minipage}[b]{\linewidth}\raggedright
સ્ટેજ
\end{minipage} & \begin{minipage}[b]{\linewidth}\raggedright
ટ્રાન્સમીટર
\end{minipage} & \begin{minipage}[b]{\linewidth}\raggedright
રીસીવર
\end{minipage} \\
\midrule\noalign{}
\endhead
\bottomrule\noalign{}
\endlastfoot
\textbf{Spreading} & Data XOR with PN code & Received signal XOR with
PN \\
\textbf{Modulation} & Spread signal modulated & Demodulate received
signal \\
\textbf{Processing} & Bandwidth વધારાય છે & Original data recover થાય
છે \\
\end{longtable}
}

\begin{itemize}
\tightlist
\item
  \textbf{Spreading પ્રક્રિયા}: Original data ને high-rate pseudorandom
  sequence સાથે XOR કરવામાં આવે છે
\item
  \textbf{Bandwidth expansion}: Processing gain factor દ્વારા signal
  bandwidth વધે છે
\item
  \textbf{Despreading}: Receiver સમાન PN code વાપરીને original data
  recover કરે છે
\item
  \textbf{Interference rejection}: Spread spectrum jamming સામે પ્રતિકાર
  પૂરો પાડે છે
\end{itemize}

\end{solutionbox}
\begin{mnemonicbox}
``Direct Sequence Spreads Signals Successfully''

\end{mnemonicbox}
\begin{center}\rule{0.5\linewidth}{0.5pt}\end{center}

\subsection*{પ્રશ્ન 4(અ OR) [3
ગુણ]}\label{uxaaauxab0uxab6uxaa8-4uxa85-or-3-uxa97uxaa3}

\textbf{10 Mcps chip rate અને 1 Mbps data rate સાથે DSSS સિસ્ટમ માટે
processing gain ની ગણતરી કરો.}

\begin{solutionbox}
\textbf{Processing gain} spread spectrum સિસ્ટમના
performance improvement નક્કી કરે છે.

\textbf{આપેલ:}

\begin{itemize}
\tightlist
\item
  Chip rate (Rc) = 10 million chips per second = 10 \times 10^{6} cps
\item
  Data rate (Rd) = 1 Mbps = 1 \times 10^{6} bps
\end{itemize}

\textbf{ગણતરી:} Processing Gain (Gp) = Chip rate \div Data rate Gp = Rc \div
Rd = (10 \times 10^{6}) \div (1 \times 10^{6}) = 10

\textbf{dB માં:} Gp (dB) = 10 log_{1}_{0}(10) = 10 \times 1 = 10 dB

\end{solutionbox}
\begin{solutionbox}

\end{solutionbox}
\begin{mnemonicbox}
``Processing Power Provides Protection''

\end{mnemonicbox}
\begin{center}\rule{0.5\linewidth}{0.5pt}\end{center}

\subsection*{પ્રશ્ન 4(બ OR) [4
ગુણ]}\label{uxaaauxab0uxab6uxaa8-4uxaac-or-4-uxa97uxaa3}

\textbf{EDGE માં data rate કેવી રીતે વધારાયેલ છે તે સમજાવો.}

\begin{solutionbox}
\textbf{EDGE (Enhanced Data rates for GSM Evolution)}
advanced modulation દ્વારા data rates સુધારે છે.

\textbf{ટેબલ: EDGE સુધારાઓ}

{\def\LTcaptype{none} % do not increment counter
\begin{longtable}[]{@{}
  >{\raggedright\arraybackslash}p{(\linewidth - 6\tabcolsep) * \real{0.3448}}
  >{\raggedright\arraybackslash}p{(\linewidth - 6\tabcolsep) * \real{0.1724}}
  >{\raggedright\arraybackslash}p{(\linewidth - 6\tabcolsep) * \real{0.2069}}
  >{\raggedright\arraybackslash}p{(\linewidth - 6\tabcolsep) * \real{0.2759}}@{}}
\toprule\noalign{}
\begin{minipage}[b]{\linewidth}\raggedright
પેરામીટર
\end{minipage} & \begin{minipage}[b]{\linewidth}\raggedright
GSM
\end{minipage} & \begin{minipage}[b]{\linewidth}\raggedright
EDGE
\end{minipage} & \begin{minipage}[b]{\linewidth}\raggedright
સુધારો
\end{minipage} \\
\midrule\noalign{}
\endhead
\bottomrule\noalign{}
\endlastfoot
\textbf{Modulation} & GMSK & 8-PSK & 3 bits per symbol vs 1 bit \\
\textbf{Data Rate} & 9.6 kbps & 43.2 kbps per slot & \textasciitilde4.5x
વધારો \\
\textbf{Coding} & Fixed & Adaptive & Link adaptation \\
\textbf{Applications} & Voice, SMS & Multimedia, Internet & Enhanced
services \\
\end{longtable}
}

\begin{itemize}
\tightlist
\item
  \textbf{8-PSK modulation}: GMSK ના 1 bit નાં બદલે પ્રતિ symbol 3 bits
  transmit કરે છે
\item
  \textbf{Link adaptation}: Channel quality પર આધારિત coding scheme
  dynamically select કરે છે
\item
  \textbf{Backward compatibility}: હાલની GSM infrastructure સાથે કાર્ય કરે
  છે
\item
  \textbf{Enhanced applications}: Multimedia અને higher data rate
  services support કરે છે
\end{itemize}

\end{solutionbox}
\begin{mnemonicbox}
``EDGE Enhances Exchange Efficiently''

\end{mnemonicbox}
\begin{center}\rule{0.5\linewidth}{0.5pt}\end{center}

\subsection*{પ્રશ્ન 4(ક OR) [7
ગુણ]}\label{uxaaauxab0uxab6uxaa8-4uxa95-or-7-uxa97uxaa3}

\textbf{CDMA માં કોલ પ્રોસેસિંગ સમજાવો.}

\begin{solutionbox}
\textbf{CDMA call processing} code-based multiple access
માટે unique procedures ધરાવે છે.

\begin{center}
\textbf{Mermaid Diagram (Code)}
\begin{verbatim}
{Shaded}
{Highlighting}[]
graph LR
    A[Mobile Power On] {-{-}{} B[Pilot Channel Search]}
    B {-{-}{} C[Sync Channel Read]}
    C {-{-}{} D[Paging Channel Monitor]}
    D {-{-}{} E[Access Channel Request]}
    E {-{-}{} F[Traffic Channel Assignment]}
    F {-{-}{} G[Active Call State]}
    G {-{-}{} H[Soft Handoff]}
{Highlighting}
{Shaded}
\end{verbatim}
\end{center}

\textbf{ટેબલ: CDMA કોલ પ્રોસેસિંગ સ્ટેજો}

{\def\LTcaptype{none} % do not increment counter
\begin{longtable}[]{@{}lll@{}}
\toprule\noalign{}
સ્ટેજ & પ્રક્રિયા & કાર્ય \\
\midrule\noalign{}
\endhead
\bottomrule\noalign{}
\endlastfoot
\textbf{Initialization} & Pilot acquisition & સૌથી મજબૂત બેઝ સ્ટેશન શોધવું \\
\textbf{Idle State} & Monitor paging & Incoming calls માટે સાંભળવું \\
\textbf{Access} & Random access & Network પાસેથી service request કરવી \\
\textbf{Traffic} & Dedicated channel & Active communication \\
\textbf{Handoff} & Soft handoff & Seamless cell transition \\
\end{longtable}
}

\begin{itemize}
\tightlist
\item
  \textbf{Pilot channel}: Timing reference અને system identification પૂરું
  પાડે છે
\item
  \textbf{Rake receiver}: Improved performance માટે multipath signals
  combine કરે છે
\item
  \textbf{Power control}: બધા યુઝર્સ માટે optimal signal levels maintain કરે
  છે
\item
  \textbf{Soft handoff}: Mobile બહુવિધ બેઝ સ્ટેશન્સ સાથે એકસાથે communicate કરે
  છે
\item
  \textbf{Code assignment}: દરેક યુઝરને unique spreading code assign કરવામાં
  આવે છે
\end{itemize}

\end{solutionbox}
\begin{mnemonicbox}
``CDMA Calls Connect Carefully and Clearly''

\end{mnemonicbox}
\begin{center}\rule{0.5\linewidth}{0.5pt}\end{center}

\subsection*{પ્રશ્ન 5(અ) [3
ગુણ]}\label{uxaaauxab0uxab6uxaa8-5uxa85-3-uxa97uxaa3}

\textbf{CDMA અને GSM ની સરખામણી કરો.}

\begin{solutionbox}
\textbf{CDMA} અને \textbf{GSM} cellular communication માટે
અલગ અલગ approaches રજૂ કરે છે.

\textbf{ટેબલ: CDMA vs GSM સરખામણી}

{\def\LTcaptype{none} % do not increment counter
\begin{longtable}[]{@{}lll@{}}
\toprule\noalign{}
પેરામીટર & CDMA & GSM \\
\midrule\noalign{}
\endhead
\bottomrule\noalign{}
\endlastfoot
\textbf{Access Method} & Code Division & Time/Frequency Division \\
\textbf{Capacity} & વધુ & ઓછી \\
\textbf{Handoff} & Soft handoff & Hard handoff \\
\textbf{Security} & બેહતર (spreading codes) & સારી (encryption) \\
\textbf{Global Usage} & મર્યાદિત & વ્યાપક \\
\textbf{Power Control} & Continuous & Periodic \\
\end{longtable}
}

\begin{itemize}
\tightlist
\item
  \textbf{Multiple access}: CDMA unique codes વાપરે છે, GSM time slots
  વાપરે છે
\item
  \textbf{Call quality}: CDMA soft handoff પૂરું પાડે છે, GSM hard handoff કરે
  છે
\end{itemize}

\end{solutionbox}
\begin{mnemonicbox}
``Choose CDMA or GSM Carefully''

\end{mnemonicbox}
\begin{center}\rule{0.5\linewidth}{0.5pt}\end{center}

\subsection*{પ્રશ્ન 5(બ) [4
ગુણ]}\label{uxaaauxab0uxab6uxaa8-5uxaac-4-uxa97uxaa3}

\textbf{CDMA ના લાભો લખો.}

\begin{solutionbox}
\textbf{CDMA લાભો} તેને high-capacity cellular systems માટે
યોગ્ય બનાવે છે.

\textbf{ટેબલ: CDMA લાભો}

{\def\LTcaptype{none} % do not increment counter
\begin{longtable}[]{@{}ll@{}}
\toprule\noalign{}
લાભ & ફાયદો \\
\midrule\noalign{}
\endhead
\bottomrule\noalign{}
\endlastfoot
\textbf{High Capacity} & પ્રતિ spectrum વધુ યુઝર્સ \\
\textbf{Soft Handoff} & Seamless call transfer \\
\textbf{Variable Rate} & Speech patterns ને અનુકૂળ \\
\textbf{Privacy} & Spreading દ્વારા inherent security \\
\textbf{Multipath Resistance} & Rake receiver વાપરે છે \\
\textbf{Power Control} & Battery life optimize કરે છે \\
\textbf{Frequency Planning} & બધા સેલ્સમાં સમાન frequency \\
\end{longtable}
}

\begin{itemize}
\tightlist
\item
  \textbf{Spectrum efficiency}: FDMA/TDMA systems કરતાં વધુ capacity
\item
  \textbf{Quality લાભ}: Soft handoff cell transitions દરમિયાન call drops
  દૂર કરે છે
\item
  \textbf{Security ફાયદો}: Spread spectrum inherent privacy protection
  પૂરું પાડે છે
\item
  \textbf{Simplified planning}: Frequency reuse planning ની જરૂર નથી
\end{itemize}

\end{solutionbox}
\begin{mnemonicbox}
``CDMA Creates Considerable Communication Capacity''

\end{mnemonicbox}
\begin{center}\rule{0.5\linewidth}{0.5pt}\end{center}

\subsection*{પ્રશ્ન 5(ક) [7
ગુણ]}\label{uxaaauxab0uxab6uxaa8-5uxa95-7-uxa97uxaa3}

\textbf{MANET ને સંક્ષિપ્તમાં સમજાવો અને તેની ઉપયોગો લખો.}

\begin{solutionbox}
\textbf{MANET (Mobile Ad Hoc Network)} મોબાઇલ ડિવાઇસેસનું
infrastructure-less network છે.

\begin{center}
\textbf{Mermaid Diagram (Code)}
\begin{verbatim}
{Shaded}
{Highlighting}[]
graph LR
    A[Mobile Node A] {-.{-}{} B[Mobile Node B]}
    B {-.{-}{} C[Mobile Node C]}
    C {-.{-}{} D[Mobile Node D]}
    A {-.{-}{} C}
    B {-.{-}{} D}
    A {-.{-}{} D}
    
    style A fill:\#f9f
    style B fill:\#9ff
    style C fill:\#ff9
    style D fill:\#9f9
{Highlighting}
{Shaded}
\end{verbatim}
\end{center}

\textbf{ટેબલ: MANET લાક્ષણિકતાઓ vs ઉપયોગો}

{\def\LTcaptype{none} % do not increment counter
\begin{longtable}[]{@{}
  >{\raggedright\arraybackslash}p{(\linewidth - 4\tabcolsep) * \real{0.4074}}
  >{\raggedright\arraybackslash}p{(\linewidth - 4\tabcolsep) * \real{0.2593}}
  >{\raggedright\arraybackslash}p{(\linewidth - 4\tabcolsep) * \real{0.3333}}@{}}
\toprule\noalign{}
\begin{minipage}[b]{\linewidth}\raggedright
લાક્ષણિકતા
\end{minipage} & \begin{minipage}[b]{\linewidth}\raggedright
વિશેષતા
\end{minipage} & \begin{minipage}[b]{\linewidth}\raggedright
ઉપયોગો
\end{minipage} \\
\midrule\noalign{}
\endhead
\bottomrule\noalign{}
\endlastfoot
\textbf{Self-organizing} & કોઈ fixed infrastructure નથી & લશ્કરી
સંદેશાવ્યવહાર \\
\textbf{Dynamic topology} & Nodes મુક્તપણે ફરે છે & Emergency response \\
\textbf{Multi-hop routing} & Intermediate node relay & Disaster
recovery \\
\textbf{Distributed control} & કોઈ central authority નથી & Sensor
networks \\
\textbf{Resource constraints} & મર્યાદિત battery, bandwidth & Vehicular
networks \\
\end{longtable}
}

\textbf{ઉપયોગો:}

\begin{itemize}
\tightlist
\item
  \textbf{લશ્કરી ઓપરેશન્સ}: Infrastructure વિના battlefield communications
\item
  \textbf{Emergency services}: Disaster response અને rescue operations
\item
  \textbf{Sensor networks}: Environmental monitoring અને data collection
\item
  \textbf{Vehicular networks}: Traffic management માટે car-to-car
  communication
\item
  \textbf{Personal area networks}: Device-to-device communication
\item
  \textbf{Academic research}: Collaborative computing environments
\end{itemize}

\textbf{ફાયદા:}

\begin{itemize}
\tightlist
\item
  \textbf{Rapid deployment}: Infrastructure setup ની જરૂર નથી
\item
  \textbf{Self-healing}: Nodes fail થાય ત્યારે automatic route
  reconfiguration
\item
  \textbf{Cost effective}: Base station installation costs નથી
\end{itemize}

\textbf{નુકસાન:}

\begin{itemize}
\tightlist
\item
  \textbf{Limited bandwidth}: Shared wireless medium
\item
  \textbf{Security challenges}: Attacks માટે vulnerable
\item
  \textbf{Power constraints}: Battery-dependent operation
\end{itemize}

\end{solutionbox}
\begin{mnemonicbox}
``Mobile Ad Hoc Networks Enable Everywhere''

\end{mnemonicbox}
\begin{center}\rule{0.5\linewidth}{0.5pt}\end{center}

\subsection*{પ્રશ્ન 5(અ OR) [3
ગુણ]}\label{uxaaauxab0uxab6uxaa8-5uxa85-or-3-uxa97uxaa3}

\textbf{WCDMA ના મુખ્ય લક્ષણો લખો.}

\begin{solutionbox}
\textbf{WCDMA (Wideband CDMA)} enhanced capabilities પૂરી
પાડતો 3G standard છે.

\textbf{ટેબલ: WCDMA મુખ્ય લક્ષણો}

{\def\LTcaptype{none} % do not increment counter
\begin{longtable}[]{@{}ll@{}}
\toprule\noalign{}
લક્ષણ & Specification \\
\midrule\noalign{}
\endhead
\bottomrule\noalign{}
\endlastfoot
\textbf{Chip Rate} & 3.84 Mcps \\
\textbf{Bandwidth} & 5 MHz \\
\textbf{Data Rates} & 2 Mbps સુધી \\
\textbf{Spreading} & Variable spreading factor \\
\textbf{Power Control} & Fast closed-loop \\
\textbf{Handoff} & Soft અને softer handoff \\
\end{longtable}
}

\begin{itemize}
\tightlist
\item
  \textbf{Wideband operation}: 5 MHz bandwidth high data rates પૂરી પાડે છે
\item
  \textbf{Variable spreading}: અલગ-અલગ service requirements ને અનુકૂળ થાય છે
\end{itemize}

\end{solutionbox}
\begin{mnemonicbox}
``WCDMA Widens Communication Data Magnificently''

\end{mnemonicbox}
\begin{center}\rule{0.5\linewidth}{0.5pt}\end{center}

\subsection*{પ્રશ્ન 5(બ OR) [4
ગુણ]}\label{uxaaauxab0uxab6uxaa8-5uxaac-or-4-uxa97uxaa3}

\textbf{5G ના લાભો લખો.}

\begin{solutionbox}
\textbf{5G લાભો} અગાઉની generations કરતાં નોંધપાત્ર સુધારાઓ
રજૂ કરે છે.

\textbf{ટેબલ: 5G લાભો}

{\def\LTcaptype{none} % do not increment counter
\begin{longtable}[]{@{}ll@{}}
\toprule\noalign{}
લાભ & ફાયદો \\
\midrule\noalign{}
\endhead
\bottomrule\noalign{}
\endlastfoot
\textbf{Ultra-high Speed} & 20 Gbps સુધી peak data rate \\
\textbf{Low Latency} & Critical applications માટે \textless1ms \\
\textbf{Massive IoT} & પ્રતિ km^{2} 1 million devices \\
\textbf{Network Slicing} & Customized virtual networks \\
\textbf{Enhanced Coverage} & બેહતર indoor અને edge coverage \\
\textbf{Energy Efficiency} & 4G કરતાં 100x વધુ કાર્યક્ષમ \\
\textbf{High Reliability} & 99.999\% availability \\
\end{longtable}
}

\begin{itemize}
\tightlist
\item
  \textbf{Enhanced mobile broadband}: AR/VR અને 4K/8K video streaming
  support કરે છે
\item
  \textbf{Ultra-reliable communications}: Autonomous vehicles અને remote
  surgery શક્ય બનાવે છે
\item
  \textbf{Massive machine communications}: Smart cities અને Industry 4.0
  support કરે છે
\item
  \textbf{Flexible network architecture}: Software-defined networking
  capabilities
\end{itemize}

\end{solutionbox}
\begin{mnemonicbox}
``5G Generates Great Gigabit Growth''

\end{mnemonicbox}
\begin{center}\rule{0.5\linewidth}{0.5pt}\end{center}

\subsection*{પ્રશ્ન 5(ક OR) [7
ગુણ]}\label{uxaaauxab0uxab6uxaa8-5uxa95-or-7-uxa97uxaa3}

\textbf{બ્લોક ડાયાગ્રામ સાથે OFDM ની કામગીરી સમજાવો.}

\begin{solutionbox}
\textbf{OFDM (Orthogonal Frequency Division
Multiplexing)} high-speed data transmission માટે બહુવિધ subcarriers વાપરે
છે.

\textbf{OFDM ટ્રાન્સમીટર:}

\begin{center}
\textbf{Mermaid Diagram (Code)}
\begin{verbatim}
{Shaded}
{Highlighting}[]
graph LR
    A[Serial Data] {-{-}{} B[Serial to Parallel]}
    B {-{-}{} C[QAM Mapping]}
    C {-{-}{} D[IFFT]}
    D {-{-}{} E[Add Cyclic Prefix]}
    E {-{-}{} F[Parallel to Serial]}
    F {-{-}{} G[RF Transmission]}
{Highlighting}
{Shaded}
\end{verbatim}
\end{center}

\textbf{OFDM રીસીવર:}

\begin{center}
\textbf{Mermaid Diagram (Code)}
\begin{verbatim}
{Shaded}
{Highlighting}[]
graph LR
    H[RF Reception] {-{-}{} I[Serial to Parallel]}
    I {-{-}{} J[Remove Cyclic Prefix]}
    J {-{-}{} K[FFT]}
    K {-{-}{} L[QAM Demapping]}
    L {-{-}{} M[Parallel to Serial]}
    M {-{-}{} N[Serial Data]}
{Highlighting}
{Shaded}
\end{verbatim}
\end{center}

\textbf{ટેબલ: OFDM પ્રક્રિયાના પગલાં}

{\def\LTcaptype{none} % do not increment counter
\begin{longtable}[]{@{}
  >{\raggedright\arraybackslash}p{(\linewidth - 4\tabcolsep) * \real{0.1714}}
  >{\raggedright\arraybackslash}p{(\linewidth - 4\tabcolsep) * \real{0.4571}}
  >{\raggedright\arraybackslash}p{(\linewidth - 4\tabcolsep) * \real{0.3714}}@{}}
\toprule\noalign{}
\begin{minipage}[b]{\linewidth}\raggedright
સ્ટેજ
\end{minipage} & \begin{minipage}[b]{\linewidth}\raggedright
ટ્રાન્સમીટર કાર્ય
\end{minipage} & \begin{minipage}[b]{\linewidth}\raggedright
રીસીવર કાર્ય
\end{minipage} \\
\midrule\noalign{}
\endhead
\bottomrule\noalign{}
\endlastfoot
\textbf{Data Conversion} & Serial to parallel conversion & Parallel to
serial reconstruction \\
\textbf{Modulation} & Subcarriers પર QAM mapping & QAM demapping \\
\textbf{Transform} & IFFT time domain signal બનાવે છે & FFT frequency
domain recover કરે છે \\
\textbf{Guard Period} & Cyclic prefix ISI અટકાવે છે & Cyclic prefix
removal \\
\end{longtable}
}

\textbf{મુખ્ય લક્ષણો:}

\begin{itemize}
\tightlist
\item
  \textbf{Orthogonal subcarriers}: બહુવિધ parallel low-rate data streams
  interference અટકાવે છે
\item
  \textbf{FFT/IFFT processing}: Fast transforms વાપરીને કાર્યક્ષમ digital
  implementation
\item
  \textbf{Cyclic prefix}: Multipath થી inter-symbol interference અટકાવતો
  guard interval
\item
  \textbf{Spectral efficiency}: મર્યાદિત bandwidth માં high data rates
  હાંસલ કરાય છે
\item
  \textbf{Multipath resistance}: વ્યક્તિગત subcarriers flat fading અનુભવે છે
\end{itemize}

\textbf{ઉપયોગો:}

\begin{itemize}
\tightlist
\item
  \textbf{WiFi (802.11)}: Wireless LAN communications
\item
  \textbf{LTE/4G}: Mobile broadband networks
\item
  \textbf{Digital TV}: DVB-T terrestrial broadcasting
\item
  \textbf{WiMAX}: Broadband wireless access
\end{itemize}

\textbf{ફાયદા:}

\begin{itemize}
\tightlist
\item
  \textbf{High spectral efficiency}: Optimal bandwidth utilization
\item
  \textbf{મજબૂતાઈ}: Frequency selective fading સામે પ્રતિકારક
\item
  \textbf{લવચીકતા}: પ્રતિ subcarrier adaptive modulation
\item
  \textbf{Implementation}: Digital signal processing hardware સરળ બનાવે છે
\end{itemize}

\textbf{ટેબલ: OFDM પેરામીટર્સ}

{\def\LTcaptype{none} % do not increment counter
\begin{longtable}[]{@{}ll@{}}
\toprule\noalign{}
પેરામીટર & સામાન્ય મૂલ્યો \\
\midrule\noalign{}
\endhead
\bottomrule\noalign{}
\endlastfoot
\textbf{Subcarriers} & 64, 128, 256, 512, 1024 \\
\textbf{Modulation} & BPSK, QPSK, 16-QAM, 64-QAM \\
\textbf{Cyclic Prefix} & Symbol duration નો 1/4, 1/8, 1/16 \\
\textbf{Applications} & WiFi, LTE, DVB, WiMAX \\
\end{longtable}
}

\end{solutionbox}
\begin{mnemonicbox}
``OFDM Offers Outstanding Data Multiplexing''

\end{mnemonicbox}
\begin{center}\rule{0.5\linewidth}{0.5pt}\end{center}


\end{document}
