\documentclass{article}

% content/resources/templates/preamble.tex
\usepackage[margin=0.6in]{geometry}
\author{Milav Dabgar}
\usepackage{amsmath,amssymb,amsthm}
\usepackage{booktabs}
\usepackage{multirow}
\usepackage{xcolor}
\usepackage{tcolorbox}
\tcbuselibrary{breakable,skins}
\usepackage[colorlinks=true,linkcolor=blue]{hyperref}
\usepackage{titlesec}
\usepackage{enumitem}
\usepackage{tikz}
\usepackage{pgfplots}
\usepackage{circuitikz}
\usepackage[version=4]{mhchem}
\usepackage{longtable}
\usepackage{array}
\usepackage{float}
\usepackage{caption}
\usepackage{listings}

\lstset{
  basicstyle=\small\ttfamily,
  breaklines=true,
  breakatwhitespace=false,
  postbreak=\mbox{\textcolor{red}{$\hookrightarrow$}\space},
  float=false,
  numbers=left,
  numberstyle=\tiny\color{gray},
  numbersep=10pt,
  xleftmargin=2em,
  keywordstyle=\color{blue},
  commentstyle=\color{green!60!black},
  stringstyle=\color{purple},
  backgroundcolor=\color{gray!5},
  showstringspaces=false,
  tabsize=2,
  captionpos=b,
  keepspaces=true,
  columns=flexible
}

\pgfplotsset{compat=1.18}
\usetikzlibrary{shapes,arrows,positioning,calc,patterns,decorations.pathmorphing,decorations.markings,arrows.meta}

% Color scheme
\definecolor{headcolor}{RGB}{0,102,204}
\definecolor{keycolor}{RGB}{220,20,60}
\definecolor{solutioncolor}{RGB}{34,139,34}
\definecolor{mnemoniccolor}{RGB}{148,0,211}
\definecolor{codecolor}{RGB}{0,0,100}

% Spacing
\setlength{\parskip}{3pt}
\setlist[itemize]{nosep}
\setlist[enumerate]{nosep}

% Title formatting
\titleformat{\section}{\Large\bfseries\color{headcolor}}{\thesection}{1em}{}
\titleformat{\subsection}{\large\bfseries\color{headcolor}}{\thesubsection}{1em}{}

% Pandoc tightlist compatibility
\providecommand{\tightlist}{%
  \setlength{\itemsep}{0pt}\setlength{\parskip}{0pt}}

% Pandoc longtable compatibility
\newcounter{none}
\def\thenone{}


% content/resources/templates/gujarati-boxes.tex
\usepackage{fontspec}
\usepackage{polyglossia}

% Set Gujarati as main language (document is primarily in Gujarati)
% Note: gloss-gujarati.ldf doesn't exist in polyglossia, but it will use hyphenation patterns
\setdefaultlanguage{gujarati}
\setotherlanguage{english}

% Configure Gujarati font properly
% Use Language=Default to prevent polyglossia from trying to add language-specific features
% that don't exist for Gujarati, which causes "empty feature" warnings
\newfontfamily\gujaratifont[Script=Gujarati,AutoFakeBold=2.5,AutoFakeSlant=0.3]{Noto Sans Gujarati}
\setmainfont[Script=Gujarati,AutoFakeBold=2.5,AutoFakeSlant=0.3]{Noto Sans Gujarati}
% Use Noto Sans Gujarati for monospace to support Gujarati in text
\setmonofont[Scale=0.9]{Noto Sans Gujarati}

% Configure English to use the same font
\newfontfamily\englishfont[Script=Gujarati,AutoFakeBold=2.5,AutoFakeSlant=0.3]{Noto Sans Gujarati}

% Translations for polyglossia
\gappto\captionsgujarati{
  \renewcommand{\tablename}{કોષ્ટક}
  \renewcommand{\figurename}{આકૃતિ}
}

% Helper for TikZ nodes to ensure Gujarati font
\newcommand{\gu}[1]{{\gujaratifont #1}}

% Custom environments
\newtcolorbox{solutionbox}{
    breakable,
    enhanced,
    colback=solutioncolor!5!white,
    colframe=solutioncolor!75!black,
    fonttitle=\bfseries,
    title=જવાબ
}

\newtcolorbox{solutionboxnobreak}{
 colback=solutioncolor!5!white,
 colframe=solutioncolor!75!black,
 fonttitle=\bfseries,
 title=જવાબ
}

\newtcolorbox{keyformula}{
 breakable,
 enhanced,
 colback=keycolor!5!white,
 colframe=keycolor!75!black,
 fonttitle=\bfseries,
 title=રાસાયણિક સમીકરણ/સૂત્ર
}

\newtcolorbox{mnemonicbox}{
 breakable,
 enhanced,
 colback=mnemoniccolor!5!white,
 colframe=mnemoniccolor!75!black,
 fonttitle=\bfseries,
 title=મેમરી ટ્રીક
}


% Custom commands for GTU solutions
% This file defines semantic commands for consistent formatting

% Question command with automatic formatting
\newcommand{\question}[2]{%
  \section*{Question #1}%
  \textbf{#2}%
}

% OR question variant
\newcommand{\questionor}[2]{%
  \section*{Question #1 OR}%
  \textbf{#2}%
}

% Proper table environment with caption
\newenvironment{answertable}[1]{%
  \begin{table}[htbp]
  \centering
  \caption{#1}
}{%
  \end{table}
}

% Proper figure environment for diagrams
\newenvironment{answerdiagram}[1]{%
  \begin{figure}[htbp]
  \centering
  \caption{#1}
}{%
  \end{figure}
}

% Semantic markup for key terms
\newcommand{\keyword}[1]{\textbf{#1}}
\newcommand{\code}[1]{\texttt{#1}}
\newcommand{\classname}[1]{\texttt{#1}}
\newcommand{\methodname}[1]{\texttt{#1}}

% Proper quotation marks
\newcommand{\mnemonic}[1]{``#1''}


\title{Mobile \& Wireless Communication (4351104) - Winter 2024 Solution (Gujarati)}
\date{November 29, 2024}

\begin{document}
\maketitle

\questionmarks{1(a)}{3}{અમ્બ્રેલા સેલ સમજાવો.}
\begin{solutionbox}
\textbf{અમ્બ્રેલા સેલ} મોટા કવરેજ એરિયાનો સેલ છે જે નાના સેલ્સને ઢાંકીને સતત કવરેજ પૂરું પાડે છે.

\begin{table}[H]
\centering
\caption{અમ્બ્રેલા સેલની લાક્ષણિકતાઓ}
\begin{tabulary}{\textwidth}{L L}
\hline
\textbf{લક્ષણ} & \textbf{વર્ણન} \\
\hline
\textbf{કવરેજ} & મોટો ભૌગોલિક વિસ્તાર \\
\textbf{હેતુ} & માઇક્રોસેલ્સમાંથી overflow traffic સંભાળવો \\
\textbf{એન્ટેના} & હાઇ-પાવર, ઊંચી જગ્યાએ મૂકેલ \\
\textbf{યુઝર્સ} & ઝડપથી ફરતા વાહનો, emergency calls \\
\hline
\end{tabulary}
\end{table}

\begin{itemize}
    \item \textbf{મોટું કવરેજ}: હાઇ-પાવર બેઝ સ્ટેશન સાથે વિશાળ ભૌગોલિક વિસ્તાર ઢાંકે છે
    \item \textbf{Traffic management}: નાના સેલ્સ ભરપૂર હોય ત્યારે calls સંભાળે છે
    \item \textbf{ગતિશીલતા સપોર્ટ}: બહુવિધ સેલ બાઉન્ડરી પાર કરતા ઝડપી યુઝર્સને સેવા આપે છે
\end{itemize}

\begin{mnemonicbox}
\mnemonic{Umbrella Covers Large Areas}
\end{mnemonicbox}
\end{solutionbox}

\questionmarks{1(b)}{4}{સેલ અને ક્લસ્ટર વ્યાખ્યાયિત કરો.}
\begin{solutionbox}
\textbf{સેલ} અને \textbf{ક્લસ્ટર} સેલ્યુલર કોમ્યુનિકેશન સિસ્ટમના મૂળભૂત ખ્યાલો છે.

\begin{table}[H]
\centering
\caption{સેલ vs ક્લસ્ટર સરખામણી}
\begin{tabulary}{\textwidth}{L L L}
\hline
\textbf{પેરામીટર} & \textbf{સેલ} & \textbf{ક્લસ્ટર} \\
\hline
\textbf{વ્યાખ્યા} & એક બેઝ સ્ટેશન દ્વારા સેવા આપવામાં આવતો એક કવરેજ વિસ્તાર & અલગ-અલગ frequencies વાપરતા સેલ્સનું જૂથ \\
\textbf{સાઇઝ} & એન્ટેના પાવર અને interference દ્વારા મર્યાદિત & N સેલ્સ ધરાવે છે (સામાન્ય રીતે 3, 4, 7, 12) \\
\textbf{Frequency} & ચોક્કસ frequency set વાપરે છે & બધી ઉપલબ્ધ frequencies એકવાર વાપરે છે \\
\textbf{હેતુ} & ચોક્કસ વિસ્તારને કવરેજ આપવું & Frequency reuse pattern શક્ય બનાવવું \\
\hline
\end{tabulary}
\end{table}

\begin{itemize}
    \item \textbf{સેલ}: એક બેઝ સ્ટેશન દ્વારા સેવા આપવામાં આવતો ભૌગોલિક વિસ્તાર
    \item \textbf{ક્લસ્ટર}: સંપૂર્ણ frequency spectrum વાપરતા પડોશી સેલ્સનું જૂથ
    \item \textbf{Frequency reuse}: અલગ-અલગ ક્લસ્ટર્સમાં સમાન frequencies ફરીથી વાપરી શકાય
    \item \textbf{Pattern repetition}: ક્લસ્ટર pattern સમગ્ર કવરેજમાં પુનરાવર્તિત થાય છે
\end{itemize}

\begin{mnemonicbox}
\mnemonic{Cells Cluster for Complete Coverage}
\end{mnemonicbox}
\end{solutionbox}

\questionmarks{1(c)}{7}{સેલ્યુલર કોમ્યુનિકેશન સિસ્ટમ પાછળના મૂળભૂત ખ્યાલનું વર્ણન કરો.}
\begin{solutionbox}
\textbf{સેલ્યુલર કોમ્યુનિકેશન} સર્વિસ એરિયાને નાના સેલ્સમાં વહેંચીને spectrum efficiency અને capacity વધારે છે.

\begin{figure}[H]
    \centering
    \begin{tikzpicture}[gtu flow]
        % Hexagonal grid
        \foreach \x/\y/\t/\c in {0/0/A/f1, 1.5/0.866/B/f2, 3/0/C/f3, 0/1.732/D/f4, 1.5/2.598/E/f5, 3/1.732/F/f6, 1.5/-0.866/G/f7, 4.5/0.866/A'/f1, 4.5/2.598/B'/f2} {
            \node[regular polygon, regular polygon sides=6, minimum size=2cm, draw, fill=blue!5] at (\x,\y) {\begin{tabular}{c}\textbf{\t} \\ \footnotesize \c\end{tabular}};
        }
        % Labels
        \node[above] at (1.5, 3.2) {\footnotesize Cluster 1 (7 Cells)};
        \node[above] at (4.5, 3.2) {\footnotesize Frequency Reuse};
    \end{tikzpicture}
    \caption{Cellular System Concept with Frequency Reuse}
\end{figure}

\begin{table}[H]
\centering
\caption{સેલ્યુલર સિસ્ટમના ફાયદા}
\begin{tabulary}{\textwidth}{L L}
\hline
\textbf{ખ્યાલ} & \textbf{ફાયદો} \\
\hline
\textbf{Frequency Reuse} & સમાન frequencies બહુવાર વાપરી શકાય \\
\textbf{Cell Division} & નાના કવરેજ વિસ્તારો, વધુ capacity \\
\textbf{Handoff} & સેલ્સ વચ્ચે seamless call transfer \\
\textbf{Power Control} & ઓછી interference, લાંબુ battery life \\
\hline
\end{tabulary}
\end{table}

\begin{itemize}
    \item \textbf{નાના સેલનો ખ્યાલ}: કાર્યક્ષમ કવરેજ માટે સર્વિસ એરિયાને hexagonal સેલ્સમાં વહેંચાય છે
    \item \textbf{Frequency reuse}: મર્યાદિત spectrum યોગ્ય separation સાથે બહુવાર વાપરાય છે
    \item \textbf{બેઝ સ્ટેશન કંટ્રોલ}: દરેક સેલને low-power બેઝ સ્ટેશન દ્વારા સેવા આપવામાં આવે છે
    \item \textbf{Capacity improvement}: એક મોટા કવરેજ વિસ્તાર કરતાં વધુ યુઝર્સને સપોર્ટ મળે છે
    \item \textbf{Interference management}: યોગ્ય સેલ પ્લાનિંગ દ્વારા co-channel interference નિયંત્રિત કરાય છે
\end{itemize}

\begin{mnemonicbox}
\mnemonic{Small Cells Support Spectrum Sharing Successfully}
\end{mnemonicbox}
\end{solutionbox}

\questionmarks{1(c OR)}{7}{સેલ્યુલર કોમ્યુનિકેશનમાં કો-ચેનલ ઇન્ટર્ફીરન્સ સમજાવો.}
\begin{solutionbox}
\textbf{કો-ચેનલ ઇન્ટર્ફીરન્સ} જ્યારે સમાન frequencies વાપરતા સેલ્સ ખૂબ નજીક હોય ત્યારે થાય છે.

\begin{figure}[H]
    \centering
    \begin{tikzpicture}[gtu flow]
        % Cell A
        \node[regular polygon, regular polygon sides=6, minimum size=2.5cm, draw, fill=red!10] (A) at (0,0) {\textbf{Cell A} (f1)};
        \node[above] at (A.north) {Base Station A};
        
        % Cell B
        \node[regular polygon, regular polygon sides=6, minimum size=2.5cm, draw, fill=red!10] (C) at (6,0) {\textbf{Cell C} (f1)};
        \node[above] at (C.north) {Base Station C};
        
        % Mobile in Interference Zone
        \node[draw, circle, fill=yellow!20, minimum size=1cm] (M) at (3, -1.5) {Mobile};
        \node[below] at (M.south) {\footnotesize Interference Zone};
        
        % Definition of D
        \draw[<->, dashed] (A.center) -- (C.center) node[midway, above] {Reuse Distance (D)};
        
        % Interference Signals
        \draw[->, gtu arrow, red, dashed] (A) -- (M) node[midway, left] {\footnotesize Signal};
        \draw[->, gtu arrow, red, dashed] (C) -- (M) node[midway, right] {\footnotesize Interference};
    \end{tikzpicture}
    \caption{Co-channel Interference Mechanism}
\end{figure}

\begin{table}[H]
\centering
\caption{કો-ચેનલ ઇન્ટર્ફીરન્સ પેરામીટર્સ}
\begin{tabulary}{\textwidth}{L L L}
\hline
\textbf{પેરામીટર} & \textbf{વર્ણન} & \textbf{અસર} \\
\hline
\textbf{Reuse Distance} & કો-ચેનલ સેલ્સ વચ્ચેનું અંતર & વધુ અંતર = ઓછી interference \\
\textbf{C/I Ratio} & Carrier to Interference ratio & સારી quality માટે $\ge$ 18 dB હોવું જોઈએ \\
\textbf{Cluster Size} & ક્લસ્ટરમાં સેલ્સની સંખ્યા & મોટું ક્લસ્ટર = વધુ separation \\
\hline
\end{tabulary}
\end{table}

\begin{itemize}
    \item \textbf{Signal overlap}: અલગ સેલ્સના સમાન frequency signals interfere કરે છે
    \item \textbf{Quality degradation}: call drops અને ખરાબ voice quality નું કારણ બને છે
    \item \textbf{Distance factor}: અંતરના વર્ગના પ્રમાણમાં interference ઘટે છે
    \item \textbf{ઘટાડવાની પદ્ધતિઓ}: યોગ્ય સેલ પ્લાનિંગ, power control, antenna design
\end{itemize}

\begin{mnemonicbox}
\mnemonic{Co-channel Causes Call Quality Concerns}
\end{mnemonicbox}
\end{solutionbox}

\questionmarks{2(a)}{3}{સેલ સ્પ્લિટિંગ સમજાવો.}
\begin{solutionbox}
\textbf{સેલ સ્પ્લિટિંગ} ભીડવાળા સેલ્સને નાના સેલ્સમાં વહેંચીને સિસ્ટમ capacity વધારે છે.

\begin{figure}[H]
    \centering
    \begin{tikzpicture}[gtu flow]
        % Original Large Cell
        \node[regular polygon, regular polygon sides=6, minimum size=3cm, draw, dashed, fill=gray!5] at (0,0) {};
        \node at (0,0.5) {\textbf{Original}};
        
        % Split Cells inside
        \foreach \x/\y in {0/0, 0.75/0.433, -0.75/0.433, 0.75/-0.433, -0.75/-0.433, 0/0.866, 0/-0.866} {
             \node[regular polygon, regular polygon sides=6, minimum size=1cm, draw, fill=green!10] at (\x,\y) {};
        }
        
        \draw[->, gtu arrow] (2,0) -- (4,0) node[midway, above] {Splitting};
        
        % Result description
        \node[gtu block, minimum width=3cm, align=left] at (6,0) {
            \textbf{New Microcells}\\
            - Radius: $R/2$\\
            - Capacity: $\approx 4\times$
        };
    \end{tikzpicture}
    \caption{Cell Splitting Concept}
\end{figure}

\begin{itemize}
    \item \textbf{Capacity વધારો}: દરેક નવો સેલ ઓછા યુઝર્સને બેહતર સર્વિસ quality સાથે handle કરે છે
    \item \textbf{Power ઘટાડો}: નવા બેઝ સ્ટેશન્સ નાના વિસ્તારોને ઢાંકવા માટે ઓછી power વાપરે છે
    \item \textbf{Frequency management}: મૂળ frequencies નવા નાના સેલ્સમાં વહેંચાય છે
\end{itemize}

\begin{mnemonicbox}
\mnemonic{Split Cells Serve Subscribers Successfully}
\end{mnemonicbox}
\end{solutionbox}

\questionmarks{2(b)}{4}{ચેનલ વહેંચણીની વ્યૂહરચના સમજાવો.}
\begin{solutionbox}
\textbf{ચેનલ assignment} વ્યૂહરચનાઓ નક્કી કરે છે કે optimal performance માટે સેલ્સને frequencies કેવી રીતે ફાળવવી.

\begin{table}[H]
\centering
\caption{ચેનલ Assignment વ્યૂહરચનાઓ}
\begin{tabulary}{\textwidth}{L L L L}
\hline
\textbf{વ્યૂહરચના} & \textbf{વર્ણન} & \textbf{ફાયદા} & \textbf{નુકસાન} \\
\hline
\textbf{Fixed} & સેલ્સને કાયમી ચેનલ્સ ફાળવવા & સરળ, અનુમાનિત & ઓછા traffic દરમિયાન બિનકાર્યક્ષમ \\
\textbf{Dynamic} & demand પર આધારિત ચેનલ assignment & કાર્યક્ષમ spectrum વપરાશ & જટિલ implementation \\
\textbf{Hybrid} & Fixed અને dynamic નું મિશ્રણ & સંતુલિત approach & મધ્યમ જટિલતા \\
\hline
\end{tabulary}
\end{table}

\begin{itemize}
    \item \textbf{Fixed assignment}: દરેક સેલને પૂર્વનિર્ધારિત ચેનલ્સનો સેટ હોય છે
    \item \textbf{Dynamic assignment}: traffic demand પર આધારિત real-time માં ચેનલ્સ ફાળવાય છે
    \item \textbf{Load balancing}: ઉપલબ્ધ ચેનલ્સમાં traffic સમાનરૂપે વહેંચાય છે
    \item \textbf{Interference avoidance}: assignment માં co-channel interference ધ્યાનમાં લેવાય છે
\end{itemize}

\begin{mnemonicbox}
\mnemonic{Dynamic Distribution Delivers Optimal Performance}
\end{mnemonicbox}
\end{solutionbox}

\questionmarks{2(c)}{7}{33MHz bandwidth, 25KHz simplex channels, 7-cell reuse, 1MHz control માટે સેલ દીઠ voice અને control channels ની ગણતરી કરો.}
\begin{solutionbox}
સેલ્યુલર સિસ્ટમમાં \textbf{ચેનલ allocation} માટે ગણતરી.

\textbf{આપેલ ડેટા:}
\begin{itemize}
    \item Total bandwidth = 33 MHz
    \item Channel bandwidth = 25 KHz (simplex)
    \item Full duplex માટે જરૂરી = $2 \times 25$ KHz = 50 KHz
    \item Control spectrum = 1 MHz
    \item Cluster size = 7 cells
\end{itemize}

\textbf{ગણતરીઓ:}

\textbf{પગલું 1: કુલ ઉપલબ્ધ ચેનલ્સ}
\[ \text{Total channels} = \frac{33 \text{ MHz}}{25 \text{ KHz}} = \frac{33000}{25} = 1320 \text{ channels} \]

\textbf{પગલું 2: Control channels}
\[ \text{Control channels} = \frac{1 \text{ MHz}}{25 \text{ KHz}} = \frac{1000}{25} = 40 \text{ channels} \]

\textbf{પગલું 3: Voice channels}
\[ \text{Voice channels} = 1320 - 40 = 1280 \text{ channels} \]

\textbf{પગલું 4: Duplex voice channels}
\[ \text{Duplex voice channels} = \frac{1280}{2} = 640 \text{ channels} \]

\textbf{પગલું 5: સેલ દીઠ ચેનલ્સ}
\[ \text{Voice channels per cell} = \frac{640}{7} \approx 91 \text{ channels} \]
\[ \text{Control channels per cell} = \frac{40}{7} \approx 6 \text{ channels} \]

\textbf{અંતિમ જવાબ:}
\begin{itemize}
    \item \textbf{સેલ દીઠ Voice channels: 91}
    \item \textbf{સેલ દીઠ Control channels: 6}
\end{itemize}

\begin{mnemonicbox}
\mnemonic{Calculate Carefully for Channel Count}
\end{mnemonicbox}
\end{solutionbox}

\questionmarks{2(a OR)}{3}{GSM માં FCCH અને SCH ના કાર્યો લખો.}
\begin{solutionbox}
\textbf{FCCH} અને \textbf{SCH} synchronization માટે GSM સિસ્ટમમાં જરૂરી control channels છે.

\begin{table}[H]
\centering
\caption{FCCH અને SCH કાર્યો}
\begin{tabulary}{\textwidth}{L L L}
\hline
\textbf{ચેનલ} & \textbf{Full Form} & \textbf{કાર્ય} \\
\hline
\textbf{FCCH} & Frequency Correction Channel & Mobile ને frequency reference પૂરું પાડે છે \\
\textbf{SCH} & Synchronization Channel & Timing અને cell identity પૂરું પાડે છે \\
\hline
\end{tabulary}
\end{table}

\begin{itemize}
    \item \textbf{FCCH કાર્ય}: Mobile ને બેઝ સ્ટેશન frequency સાથે synchronize કરવામાં મદદ કરે છે
    \item \textbf{SCH કાર્ય}: BSIC (Base Station Identity Code) અને frame number વહન કરે છે
    \item \textbf{Timing correction}: બંને ચેનલ્સ mobile ને યોગ્ય timing synchronization મેળવવામાં મદદ કરે છે
\end{itemize}

\begin{mnemonicbox}
\mnemonic{FCCH Fixes Frequency, SCH Synchronizes System}
\end{mnemonicbox}
\end{solutionbox}

\questionmarks{2(b OR)}{4}{GSM 900 specifications લખો.}
\begin{solutionbox}
\textbf{GSM 900} 900 MHz frequency band માં ચોક્કસ તકનીકી પેરામીટર્સ સાથે કાર્ય કરે છે.

\begin{table}[H]
\centering
\caption{GSM 900 Specifications}
\begin{tabulary}{\textwidth}{L L}
\hline
\textbf{પેરામીટર} & \textbf{Specification} \\
\hline
\textbf{Uplink Frequency} & 890-915 MHz \\
\textbf{Downlink Frequency} & 935-960 MHz \\
\textbf{Duplex Separation} & 45 MHz \\
\textbf{Channel Spacing} & 200 KHz \\
\textbf{Total Channels} & 124 channels \\
\textbf{Access Method} & TDMA/FDMA \\
\textbf{Modulation} & GMSK \\
\textbf{Power Classes} & 2W, 8W, 20W \\
\hline
\end{tabulary}
\end{table}

\begin{itemize}
    \item \textbf{Frequency bands}: Full duplex operation માટે અલગ uplink અને downlink frequencies
    \item \textbf{TDMA structure}: દરેક carrier frequency પર 8 time slots
\end{itemize}

\begin{mnemonicbox}
\mnemonic{GSM 900 Gives Great Global Coverage}
\end{mnemonicbox}
\end{solutionbox}

\questionmarks{2(c OR)}{7}{GSM આર્કિટેક્ચર દોરો અને સમજાવો.}
\begin{solutionbox}
\textbf{GSM આર્કિટેક્ચર} mobile communication માટે સાથે કાર્ય કરતા ત્રણ મુખ્ય subsystems ધરાવે છે.

\begin{figure}[H]
    \centering
    \begin{tikzpicture}[gtu flow]
        % Nodes
        \node[gtu block] (MS) {Mobile Station\\(MS)};
        \node[gtu block, right=of MS, xshift=1cm] (BSS) {Base Station Subsystem\\(BSS)};
        \node[gtu block, right=of BSS, xshift=1cm] (NSS) {Network Switching\\Subsystem (NSS)};
        \node[gtu block, right=of NSS, xshift=1cm] (PSTN) {PSTN/ISDN};
        
        % Internal Components (Simplified representation)
        \node[below=0.5cm of BSS, font=\footnotesize, align=center] {BTS + BSC};
        \node[below=0.5cm of NSS, font=\footnotesize, align=center] {MSC, HLR,\\VLR, AuC, EIR};
        
        % Connections
        \draw[gtu arrow] (MS) -- (BSS) node[midway, above] {\footnotesize Um};
        \draw[gtu arrow] (BSS) -- (NSS) node[midway, above] {\footnotesize A};
        \draw[gtu arrow] (NSS) -- (PSTN);
        
        % Grouping
        \draw[dashed] ($(BSS.north west)+(-0.3,0.3)$) rectangle ($(BSS.south east)+(0.3,-0.8)$);
        \draw[dashed] ($(NSS.north west)+(-0.3,0.3)$) rectangle ($(NSS.south east)+(0.3,-0.8)$);
    \end{tikzpicture}
    \caption{GSM System Architecture}
\end{figure}

\begin{table}[H]
\centering
\caption{GSM આર્કિટેક્ચર Components}
\begin{tabulary}{\textwidth}{L L L}
\hline
\textbf{Subsystem} & \textbf{Components} & \textbf{કાર્ય} \\
\hline
\textbf{Mobile Station} & Mobile Equipment + SIM & User interface અને identity \\
\textbf{BSS} & BTS + BSC & Radio interface અને control \\
\textbf{NSS} & MSC, HLR, VLR, AuC & Switching અને database management \\
\hline
\end{tabulary}
\end{table}

\begin{itemize}
    \item \textbf{Mobile Station}: યુઝર identification માટે mobile equipment અને SIM card ધરાવે છે
    \item \textbf{Base Station Subsystem}: Radio communication અને resource management handle કરે છે
    \item \textbf{Network Switching Subsystem}: Call switching, routing, અને subscriber databases manage કરે છે
    \item \textbf{Interfaces}: A-bis (BTS-BSC), A (BSC-MSC) interfaces subsystems ને connect કરે છે
\end{itemize}

\begin{mnemonicbox}
\mnemonic{Mobile Base Network - Complete Communication Chain}
\end{mnemonicbox}
\end{solutionbox}

\questionmarks{3(a)}{3}{GSM માં signal processing નો block diagram દોરો.}
\begin{solutionbox}
GSM માં \textbf{signal processing} voice અને data transmission માટે અનેક stages ધરાવે છે.

\begin{figure}[H]
    \centering
    \begin{tikzpicture}[gtu flow, node distance=1.5cm]
        \node[gtu block] (Speech) {Speech Input};
        \node[gtu block, right=of Speech, text width=2cm] (Code) {Speech\\Coding\\(13 kbps)};
        \node[gtu block, right=of Code, text width=2cm] (Chan) {Channel\\Coding\\(22.8 kbps)};
        \node[gtu block, below=of Chan, text width=2cm] (Int) {Interleaving};
        \node[gtu block, left=of Int, text width=2cm] (Burst) {Burst\\Formatting};
        \node[gtu block, left=of Burst, text width=2cm] (RF) {Modulation\\\& RF};

        \draw[gtu arrow] (Speech) -- (Code);
        \draw[gtu arrow] (Code) -- (Chan);
        \draw[gtu arrow] (Chan) -- (Int);
        \draw[gtu arrow] (Int) -- (Burst);
        \draw[gtu arrow] (Burst) -- (RF);
    \end{tikzpicture}
    \caption{GSM Signal Processing Steps}
\end{figure}

\begin{itemize}
    \item \textbf{Speech coding}: RPE-LTP વાપરીને analog speech ને 13 kbps digital data માં convert કરે છે
    \item \textbf{Channel coding}: Error correction bits ઉમેરીને rate 22.8 kbps સુધી વધારે છે
    \item \textbf{Interleaving}: Fading થી burst errors સામે લડવા માટે data ફરીથી order કરે છે
\end{itemize}

\begin{mnemonicbox}
\mnemonic{Speech Signals Systematically Processed Successfully}
\end{mnemonicbox}
\end{solutionbox}

\questionmarks{3(b)}{4}{GSM માં Common Control Channels ના કાર્યો લખો.}
\begin{solutionbox}
\textbf{Common Control Channels} GSM માં system information અને access procedures manage કરે છે.

\begin{table}[H]
\centering
\caption{Common Control Channels કાર્યો}
\begin{tabulary}{\textwidth}{L L}
\hline
\textbf{ચેનલ} & \textbf{કાર્ય} \\
\hline
\textbf{FCCH} & Frequency correction અને synchronization \\
\textbf{SCH} & Frame synchronization અને cell identification \\
\textbf{BCCH} & System information અને cell parameters broadcast કરે છે \\
\textbf{RACH} & Mobile દ્વારા call initiation માટે random access \\
\textbf{AGCH} & Mobiles ને dedicated channels assign કરે છે \\
\textbf{PCH} & Incoming calls માટે mobiles ને page કરે છે \\
\hline
\end{tabulary}
\end{table}

\begin{itemize}
    \item \textbf{Broadcast કાર્ય}: BCCH સતત system information transmit કરે છે
    \item \textbf{Access management}: RACH mobiles ને service request કરવાની મંજૂરી આપે છે
    \item \textbf{Channel assignment}: AGCH active calls માટે resources allocate કરે છે
    \item \textbf{Paging service}: PCH mobiles ને incoming calls ની જાણ કરે છે
\end{itemize}

\begin{mnemonicbox}
\mnemonic{Common Channels Control Communication Completely}
\end{mnemonicbox}
\end{solutionbox}

\questionmarks{3(c)}{7}{GSM આઇડેન્ટિફાયર્સ સમજાવો.}
\begin{solutionbox}
\textbf{GSM identifiers} subscribers, equipment, અને network elements ને uniquely identify કરે છે.

\begin{table}[H]
\centering
\caption{GSM Identifiers}
\begin{tabulary}{\textwidth}{L L L L}
\hline
\textbf{Identifier} & \textbf{Full Form} & \textbf{હેતુ} & \textbf{Format} \\
\hline
\textbf{IMSI} & International Mobile Subscriber Identity & Unique subscriber ID & 15 digits \\
\textbf{IMEI} & International Mobile Equipment Identity & Unique equipment ID & 15 digits \\
\textbf{MSISDN} & Mobile Station ISDN Number & Phone number & Variable length \\
\textbf{TMSI} & Temporary Mobile Subscriber Identity & Security માટે temporary ID & 32 bits \\
\textbf{LAI} & Location Area Identity & Geographic area identification & MCC+MNC+LAC \\
\textbf{BSIC} & Base Station Identity Code & Cell identification & 6 bits \\
\hline
\end{tabulary}
\end{table}

\begin{itemize}
    \item \textbf{IMSI structure}: MCC (3) + MNC (2-3) + MSIN (9-10 digits)
    \item \textbf{Security હેતુ}: TMSI radio interface પર subscriber identity ની સુરક્ષા કરે છે
    \item \textbf{Location management}: LAI કાર્યક્ષમ paging અને location updates માં મદદ કરે છે
    \item \textbf{Network planning}: BSIC પડોશી સેલ્સ વચ્ચે confusion અટકાવે છે
\end{itemize}

\begin{mnemonicbox}
\mnemonic{Important Mobile System Identifiers Ensure Security}
\end{mnemonicbox}
\end{solutionbox}

\questionmarks{3(a OR)}{3}{ઝડપી અને ધીમી frequency hopping ની તુલના કરો.}
\begin{solutionbox}
\textbf{Frequency hopping} techniques symbol rate ના સંબંધમાં hopping rate માં અલગ પડે છે.

\begin{table}[H]
\centering
\caption{Fast vs Slow Frequency Hopping}
\begin{tabulary}{\textwidth}{L L L}
\hline
\textbf{પેરામીટર} & \textbf{Fast Hopping} & \textbf{Slow Hopping} \\
\hline
\textbf{Hopping Rate} & $>$ Symbol rate & $<$ Symbol rate \\
\textbf{Symbols per Hop} & $<$ 1 & $>$ 1 \\
\textbf{જટિલતા} & ઊંચી & નીચી \\
\textbf{Applications} & Military, Bluetooth & GSM, CDMA \\
\hline
\end{tabulary}
\end{table}

\begin{itemize}
    \item \textbf{Fast hopping}: પ્રતિ symbol બહુવિધ hops, બેહતર security પણ વધુ જટિલ
    \item \textbf{Slow hopping}: પ્રતિ hop બહુવિધ symbols, સરળ implementation
\end{itemize}

\begin{mnemonicbox}
\mnemonic{Fast Frequently Flips, Slow Stays Stable}
\end{mnemonicbox}
\end{solutionbox}

\questionmarks{3(b OR)}{4}{Frequency reuse નો ઉપયોગ કર્યા વિના GSM 900 band માં એકસાથે વાત કરી શકે તેવા વપરાશકર્તાઓની સંખ્યાની ગણતરી કરો.}
\begin{solutionbox}
Frequency reuse વિના GSM 900 માં મહત્તમ યુઝર્સ માટે \textbf{ગણતરી}.

\textbf{આપેલ GSM 900 પેરામીટર્સ:}
\begin{itemize}
    \item Uplink: 890-915 MHz (25 MHz)
    \item Downlink: 935-960 MHz (25 MHz)
    \item Channel spacing: 200 KHz
    \item પ્રતિ ચેનલ time slots: 8
\end{itemize}

\textbf{ગણતરીઓ:}

\textbf{પગલું 1: ઉપલબ્ધ ચેનલ્સ}
\[ \text{Total channels} = \frac{25 \text{ MHz}}{200 \text{ KHz}} = \frac{25000}{200} = 125 \text{ channels} \]

\textbf{પગલું 2: વાપરી શકાય તેવા ચેનલ્સ}
\[ \text{Guard channels કાઢ્યા પછી} \approx 124 \text{ channels} \]

\textbf{પગલું 3: એકસાથે યુઝર્સ}
\[ \text{પ્રતિ ચેનલ યુઝર્સ} = 8 \text{ time slots} \]
\[ \text{કુલ યુઝર્સ} = 124 \times 8 = 992 \text{ યુઝર્સ} \]

\textbf{જવાબ: 992 યુઝર્સ એકસાથે વાત કરી શકે છે}

\begin{mnemonicbox}
\mnemonic{Calculate Channels Times Time-slots}
\end{mnemonicbox}
\end{solutionbox}

\questionmarks{3(c OR)}{7}{મોબાઇલ હેન્ડસેટનો સામાન્ય block diagram દોરો અને સમજાવો.}
\begin{solutionbox}
\textbf{મોબાઇલ હેન્ડસેટ} સાથે કાર્ય કરતા અનેક functional blocks ધરાવે છે.

\begin{figure}[H]
    \centering
    \begin{tikzpicture}[gtu flow]
        % Main Blocks
        \node[gtu block] (Base) {Baseband\\Processor};
        \node[gtu block, above=of Base] (IF) {IF Section};
        \node[gtu block, above=of IF] (RF) {RF Section};
        \node[gtu block, above=of RF] (Ant) {Antenna};
        
        % Peripherals
        \node[gtu block, left=of Base] (Audio) {Audio\\Section};
        \node[gtu block, right=of Base] (UI) {Display/\\Keypad};
        \node[gtu block, below=of Base] (Pwr) {Power\\Management};
        \node[gtu block, right=of Pwr] (Batt) {Battery};
        \node[gtu block, left=of Pwr] (SIM) {SIM\\Interface};
        
        % Connections
        \draw[<->, gtu arrow] (Ant) -- (RF);
        \draw[<->, gtu arrow] (RF) -- (IF);
        \draw[<->, gtu arrow] (IF) -- (Base);
        \draw[<->, gtu arrow] (Base) -- (Audio);
        \draw[<->, gtu arrow] (Base) -- (UI);
        \draw[<->, gtu arrow] (Base) -- (Pwr);
        \draw[<->, gtu arrow] (Base) -- (SIM);
        \draw[->, gtu arrow] (Batt) -- (Pwr);
    \end{tikzpicture}
    \caption{Mobile Handset Block Diagram}
\end{figure}

\begin{table}[H]
\centering
\caption{મોબાઇલ હેન્ડસેટ બ્લોક્સ}
\begin{tabulary}{\textwidth}{L L}
\hline
\textbf{બ્લોક} & \textbf{કાર્ય} \\
\hline
\textbf{RF Section} & Signal transmission અને reception \\
\textbf{Baseband} & Digital signal processing \\
\textbf{Audio} & Voice input/output processing \\
\textbf{Power Management} & Battery અને power control \\
\textbf{User Interface} & Display, keypad, speaker, microphone \\
\hline
\end{tabulary}
\end{table}

\begin{itemize}
    \item \textbf{RF processing}: Radio frequency transmission અને reception handle કરે છે
    \item \textbf{Digital processing}: Baseband channel coding, speech processing કરે છે
    \item \textbf{User interface}: Display, keypad, audio દ્વારા interaction પૂરું પાડે છે
    \item \textbf{Power control}: Battery usage અને charging functions manage કરે છે
\end{itemize}

\begin{mnemonicbox}
\mnemonic{Mobile Manages Multiple Modules Simultaneously}
\end{mnemonicbox}
\end{solutionbox}

\questionmarks{4(a)}{3}{મોબાઈલના કારણે રેડિયેશનના જોખમો લખો.}
\begin{solutionbox}
મોબાઇલ ફોનમાંથી \textbf{રેડિયેશન જોખમો} RF energy exposure ને કારણે આરોગ્યની ચિંતા છે.

\begin{table}[H]
\centering
\caption{મોબાઇલ રેડિયેશન જોખમો}
\begin{tabulary}{\textwidth}{L L L}
\hline
\textbf{જોખમ} & \textbf{અસર} & \textbf{રોકથામ} \\
\hline
\textbf{SAR Exposure} & Tissue heating & Hands-free devices વાપરો \\
\textbf{મગજ પર અસર} & Memory, sleep ની સમસ્યાઓ & Call duration મર્યાદિત રાખો \\
\textbf{કેન્સરનું જોખમ} & સંભવિત tumor નું જોખમ & ફોન શરીરથી દૂર રાખો \\
\hline
\end{tabulary}
\end{table}

\begin{itemize}
    \item \textbf{SAR (Specific Absorption Rate)}: શરીરના tissue દ્વારા absorbed RF energy માપે છે
    \item \textbf{Thermal effects}: RF energy tissue ના localized heating નું કારણ બની શકે છે
    \item \textbf{Non-thermal effects}: Cellular functions અને DNA પર સંભવિત અસરો
\end{itemize}

\begin{mnemonicbox}
\mnemonic{Safety Awareness Reduces Radiation Risk}
\end{mnemonicbox}
\end{solutionbox}

\questionmarks{4(b)}{4}{મોબાઈલ હેન્ડસેટમાં બેઝબેન્ડ વિભાગની કામગીરી સમજાવો.}
\begin{solutionbox}
\textbf{બેઝબેન્ડ વિભાગ} મોબાઇલ હેન્ડસેટમાં digital signal processing કાર્યો કરે છે.

\begin{table}[H]
\centering
\caption{બેઝબેન્ડ વિભાગના કાર્યો}
\begin{tabulary}{\textwidth}{L L}
\hline
\textbf{કાર્ય} & \textbf{વર્ણન} \\
\hline
\textbf{Speech Processing} & Vocoder વાપરીને voice encode/decode કરે છે \\
\textbf{Channel Coding} & Error correction અને detection ઉમેરે છે \\
\textbf{Modulation} & Digital data ને analog signals માં convert કરે છે \\
\textbf{Protocol Processing} & Signaling અને call control handle કરે છે \\
\hline
\end{tabulary}
\end{table}

\begin{itemize}
    \item \textbf{Digital signal processor}: Speech coding algorithms execute કરે છે (GSM: RPE-LTP)
    \item \textbf{Error correction}: વિશ્વસનીય transmission માટે convolutional coding implement કરે છે
    \item \textbf{Control functions}: Call setup, handoff, અને power control manage કરે છે
    \item \textbf{Interface}: RF section ને user interface components સાથે connect કરે છે
\end{itemize}

\begin{mnemonicbox}
\mnemonic{Baseband Brings Better Communication Control}
\end{mnemonicbox}
\end{solutionbox}

\questionmarks{4(c)}{7}{DSSS ટ્રાન્સમીટર અને રીસીવરની કામગીરી સમજાવો.}
\begin{solutionbox}
\textbf{DSSS (Direct Sequence Spread Spectrum)} pseudorandom codes વાપરીને signal bandwidth spread કરે છે.

\begin{figure}[H]
    \centering
    \begin{tikzpicture}[gtu flow]
        % Transmitter
        \node[gtu block] (Data) {Data\\Input};
        \node[gtu block, regular polygon, regular polygon sides=4, shape border rotate=45, right=of Data] (XOR) {XOR};
        \node[gtu block, below=of XOR] (PN) {PN Code\\Gen};
        \node[gtu block, right=of XOR] (Mod) {Modulator};
        \node[right=of Mod] (Out) {RF Output};
        
        \draw[gtu arrow] (Data) -- (XOR);
        \draw[gtu arrow] (PN) -- (XOR);
        \draw[gtu arrow] (XOR) -- (Mod);
        \draw[gtu arrow] (Mod) -- (Out);
        
        \node[above=0.2cm of XOR] {Transmitter};
    \end{tikzpicture}
    
    \vspace{0.5cm}
    
    \begin{tikzpicture}[gtu flow]
        % Receiver
        \node (In) {RF Input};
        \node[gtu block, right=of In] (Demod) {Demodulator};
        \node[gtu block, regular polygon, regular polygon sides=4, shape border rotate=45, right=of Demod] (XOR) {XOR};
        \node[gtu block, below=of XOR] (PN) {PN Code\\Gen};
        \node[gtu block, right=of XOR] (Data) {Data\\Output};
        
        \draw[gtu arrow] (In) -- (Demod);
        \draw[gtu arrow] (Demod) -- (XOR);
        \draw[gtu arrow] (PN) -- (XOR);
        \draw[gtu arrow] (XOR) -- (Data);
        
        \node[above=0.2cm of XOR] {Receiver};
    \end{tikzpicture}
    \caption{DSSS Transmitter and Receiver}
\end{figure}

\begin{table}[H]
\centering
\caption{DSSS પ્રક્રિયા}
\begin{tabulary}{\textwidth}{L L L}
\hline
\textbf{સ્ટેજ} & \textbf{ટ્રાન્સમીટર} & \textbf{રીસીવર} \\
\hline
\textbf{Spreading} & Data XOR with PN code & Received signal XOR with PN \\
\textbf{Modulation} & Spread signal modulated & Demodulate received signal \\
\textbf{Processing} & Bandwidth વધારાય છે & Original data recover થાય છે \\
\hline
\end{tabulary}
\end{table}

\begin{itemize}
    \item \textbf{Spreading પ્રક્રિયા}: Original data ને high-rate pseudorandom sequence સાથે XOR કરવામાં આવે છે
    \item \textbf{Bandwidth expansion}: Processing gain factor દ્વારા signal bandwidth વધે છે
    \item \textbf{Despreading}: Receiver સમાન PN code વાપરીને original data recover કરે છે
    \item \textbf{Interference rejection}: Spread spectrum jamming સામે પ્રતિકાર પૂરો પાડે છે
\end{itemize}

\begin{mnemonicbox}
\mnemonic{Direct Sequence Spreads Signals Successfully}
\end{mnemonicbox}
\end{solutionbox}

\questionmarks{4(a OR)}{3}{10 Mcps chip rate અને 1 Mbps data rate સાથે DSSS સિસ્ટમ માટે processing gain ની ગણતરી કરો.}
\begin{solutionbox}
\textbf{Processing gain} spread spectrum સિસ્ટમના performance improvement નક્કી કરે છે.

\textbf{આપેલ:}
\begin{itemize}
    \item Chip rate (Rc) = 10 million chips per second = $10 \times 10^6$ cps
    \item Data rate (Rd) = 1 Mbps = $1 \times 10^6$ bps
\end{itemize}

\textbf{ગણતરી:}
\[ \text{Processing Gain (Gp)} = \frac{\text{Chip rate}}{\text{Data rate}} \]
\[ Gp = \frac{Rc}{Rd} = \frac{10 \times 10^6}{1 \times 10^6} = 10 \]

\textbf{dB માં:}
\[ Gp \text{ (dB)} = 10 \log_{10}(10) = 10 \times 1 = 10 \text{ dB} \]

\textbf{જવાબ: Processing Gain = 10 અથવા 10 dB}

\begin{mnemonicbox}
\mnemonic{Processing Power Provides Protection}
\end{mnemonicbox}
\end{solutionbox}

\questionmarks{4(b OR)}{4}{EDGE માં data rate કેવી રીતે વધારાયેલ છે તે સમજાવો.}
\begin{solutionbox}
\textbf{EDGE (Enhanced Data rates for GSM Evolution)} advanced modulation દ્વારા data rates સુધારે છે.

\begin{table}[H]
\centering
\caption{EDGE સુધારાઓ}
\begin{tabulary}{\textwidth}{L L L L}
\hline
\textbf{પેરામીટર} & \textbf{GSM} & \textbf{EDGE} & \textbf{સુધારો} \\
\hline
\textbf{Modulation} & GMSK & 8-PSK & 3 bits per symbol vs 1 bit \\
\textbf{Data Rate} & 9.6 kbps & 43.2 kbps per slot & $\approx$ 4.5x વધારો \\
\textbf{Coding} & Fixed & Adaptive & Link adaptation \\
\textbf{Applications} & Voice, SMS & Multimedia, Internet & Enhanced services \\
\hline
\end{tabulary}
\end{table}

\begin{itemize}
    \item \textbf{8-PSK modulation}: GMSK ના 1 bit નાં બદલે પ્રતિ symbol 3 bits transmit કરે છે
    \item \textbf{Link adaptation}: Channel quality પર આધારિત coding scheme dynamically select કરે છે
    \item \textbf{Backward compatibility}: હાલની GSM infrastructure સાથે કાર્ય કરે છે
    \item \textbf{Enhanced applications}: Multimedia અને higher data rate services support કરે છે
\end{itemize}

\begin{mnemonicbox}
\mnemonic{EDGE Enhances Exchange Efficiently}
\end{mnemonicbox}
\end{solutionbox}

\questionmarks{4(c OR)}{7}{CDMA માં કોલ પ્રોસેસિંગ સમજાવો.}
\begin{solutionbox}
\textbf{CDMA call processing} code-based multiple access માટે unique procedures ધરાવે છે.

\begin{figure}[H]
    \centering
    \begin{tikzpicture}[gtu flow]
        \node[gtu block] (Power) {Mobile\\Power On};
        \node[gtu block, right=of Power] (Pilot) {Pilot Channel\\Acquisition};
        \node[gtu block, right=of Pilot] (Sync) {Sync Channel\\Processing};
        \node[gtu block, below=of Sync] (Page) {Paging Channel\\Monitor};
        \node[gtu block, left=of Page] (Access) {Access Channel\\Request};
        \node[gtu block, left=of Access] (Traffic) {Traffic Channel\\Assignment};
        \node[gtu block, below=of Traffic] (Active) {Active Call\\State};
        
        \draw[gtu arrow] (Power) -- (Pilot);
        \draw[gtu arrow] (Pilot) -- (Sync);
        \draw[gtu arrow] (Sync) -- (Page);
        \draw[gtu arrow] (Page) -- (Access);
        \draw[gtu arrow] (Access) -- (Traffic);
        \draw[gtu arrow] (Traffic) -- (Active);
    \end{tikzpicture}
    \caption{CDMA Call Initialization Process}
\end{figure}

\begin{table}[H]
\centering
\caption{CDMA કોલ પ્રોસેસિંગ સ્ટેજો}
\begin{tabulary}{\textwidth}{L L L}
\hline
\textbf{સ્ટેજ} & \textbf{પ્રક્રિયા} & \textbf{કાર્ય} \\
\hline
\textbf{Initialization} & Pilot acquisition & સૌથી મજબૂત બેઝ સ્ટેશન શોધવું \\
\textbf{Idle State} & Monitor paging & Incoming calls માટે સાંભળવું \\
\textbf{Access} & Random access & Network પાસેથી service request કરવી \\
\textbf{Traffic} & Dedicated channel & Active communication \\
\textbf{Handoff} & Soft handoff & Seamless cell transition \\
\hline
\end{tabulary}
\end{table}

\begin{itemize}
    \item \textbf{Pilot channel}: Timing reference અને system identification પૂરું પાડે છે
    \item \textbf{Rake receiver}: Improved performance માટે multipath signals combine કરે છે
    \item \textbf{Power control}: બધા યુઝર્સ માટે optimal signal levels maintain કરે છે
    \item \textbf{Soft handoff}: Mobile બહુવિધ બેઝ સ્ટેશન્સ સાથે એકસાથે communicate કરે છે
    \item \textbf{Code assignment}: દરેક યુઝરને unique spreading code assign કરવામાં આવે છે
\end{itemize}

\begin{mnemonicbox}
\mnemonic{CDMA Calls Connect Carefully and Clearly}
\end{mnemonicbox}
\end{solutionbox}

\questionmarks{5(a)}{3}{CDMA અને GSM ની સરખામણી કરો.}
\begin{solutionbox}
\textbf{CDMA} અને \textbf{GSM} cellular communication માટે અલગ અલગ approaches રજૂ કરે છે.

\begin{table}[H]
\centering
\caption{CDMA vs GSM સરખામણી}
\begin{tabulary}{\textwidth}{L L L}
\hline
\textbf{પેરામીટર} & \textbf{CDMA} & \textbf{GSM} \\
\hline
\textbf{Access Method} & Code Division & Time/Frequency Division \\
\textbf{Capacity} & વધુ & ઓછી \\
\textbf{Handoff} & Soft handoff & Hard handoff \\
\textbf{Security} & બેહતર (spreading codes) & સારી (encryption) \\
\textbf{Global Usage} & મર્યાદિત & વ્યાપક \\
\textbf{Power Control} & Continuous & Periodic \\
\hline
\end{tabulary}
\end{table}

\begin{itemize}
    \item \textbf{Multiple access}: CDMA unique codes વાપરે છે, GSM time slots વાપરે છે
    \item \textbf{Call quality}: CDMA soft handoff પૂરું પાડે છે, GSM hard handoff કરે છે
\end{itemize}

\begin{mnemonicbox}
\mnemonic{Choose CDMA or GSM Carefully}
\end{mnemonicbox}
\end{solutionbox}

\questionmarks{5(b)}{4}{CDMA ના લાભો લખો.}
\begin{solutionbox}
\textbf{CDMA લાભો} તેને high-capacity cellular systems માટે યોગ્ય બનાવે છે.

\begin{table}[H]
\centering
\caption{CDMA લાભો}
\begin{tabulary}{\textwidth}{L L}
\hline
\textbf{લાભ} & \textbf{ફાયદો} \\
\hline
\textbf{High Capacity} & પ્રતિ spectrum વધુ યુઝર્સ \\
\textbf{Soft Handoff} & Seamless call transfer \\
\textbf{Variable Rate} & Speech patterns ને અનુકૂળ \\
\textbf{Privacy} & Spreading દ્વારા inherent security \\
\textbf{Multipath Resistance} & Rake receiver વાપરે છે \\
\textbf{Power Control} & Battery life optimize કરે છે \\
\textbf{Frequency Planning} & બધા સેલ્સમાં સમાન frequency \\
\hline
\end{tabulary}
\end{table}

\begin{itemize}
    \item \textbf{Spectrum efficiency}: FDMA/TDMA systems કરતાં વધુ capacity
    \item \textbf{Quality લાભ}: Soft handoff cell transitions દરમિયાન call drops દૂર કરે છે
    \item \textbf{Security ફાયદો}: Spread spectrum inherent privacy protection પૂરું પાડે છે
    \item \textbf{Simplified planning}: Frequency reuse planning ની જરૂર નથી
\end{itemize}

\begin{mnemonicbox}
\mnemonic{CDMA Creates Considerable Communication Capacity}
\end{mnemonicbox}
\end{solutionbox}

\questionmarks{5(c)}{7}{MANET ને સંક્ષિપ્તમાં સમજાવો અને તેની ઉપયોગો લખો.}
\begin{solutionbox}
\textbf{MANET (Mobile Ad Hoc Network)} મોબાઇલ ડિવાઇસેસનું infrastructure-less network છે.

\begin{figure}[H]
    \centering
    \begin{tikzpicture}[gtu flow]
        % Nodes
        \node[draw, circle, fill=red!10] (A) at (0,0) {A};
        \node[draw, circle, fill=blue!10] (B) at (2,1) {B};
        \node[draw, circle, fill=green!10] (C) at (2,-1) {C};
        \node[draw, circle, fill=yellow!10] (D) at (4,0) {D};
        
        % Connections
        \draw[dashed, gtu arrow] (A) -- (B);
        \draw[dashed, gtu arrow] (A) -- (C);
        \draw[dashed, gtu arrow] (B) -- (D);
        \draw[dashed, gtu arrow] (C) -- (D);
        \draw[dashed, gtu arrow] (B) -- (C);
        
        \node[below=2cm of C] {No Central Base Station};
    \end{tikzpicture}
    \caption{Structure of Mobile Ad Hoc Network}
\end{figure}

\begin{table}[H]
\centering
\caption{MANET લાક્ષણિકતાઓ vs ઉપયોગો}
\begin{tabulary}{\textwidth}{L L L}
\hline
\textbf{લાક્ષણિકતા} & \textbf{વિશેષતા} & \textbf{ઉપયોગો} \\
\hline
\textbf{Self-organizing} & કોઈ fixed infrastructure નથી & લશ્કરી સંદેશાવ્યવહાર \\
\textbf{Dynamic topology} & Nodes મુક્તપણે ફરે છે & Emergency response \\
\textbf{Multi-hop routing} & Intermediate node relay & Disaster recovery \\
\textbf{Distributed control} & કોઈ central authority નથી & Sensor networks \\
\textbf{Resource constraints} & મર્યાદિત battery, bandwidth & Vehicular networks \\
\hline
\end{tabulary}
\end{table}

\textbf{ઉપયોગો:}
\begin{itemize}
    \item \textbf{લશ્કરી ઓપરેશન્સ}: Infrastructure વિના battlefield communications
    \item \textbf{Emergency services}: Disaster response અને rescue operations
    \item \textbf{Sensor networks}: Environmental monitoring અને data collection
    \item \textbf{Vehicular networks}: Traffic management માટે car-to-car communication
    \item \textbf{Personal area networks}: Device-to-device communication
    \item \textbf{Academic research}: Collaborative computing environments
\end{itemize}

\textbf{ફાયદા:}
\begin{itemize}
    \item \textbf{Rapid deployment}: Infrastructure setup ની જરૂર નથી
    \item \textbf{Self-healing}: Nodes fail થાય ત્યારે automatic route reconfiguration
    \item \textbf{Cost effective}: Base station installation costs નથી
\end{itemize}

\textbf{નુકસાન:}
\begin{itemize}
    \item \textbf{Limited bandwidth}: Shared wireless medium
    \item \textbf{Security challenges}: Attacks માટે vulnerable
    \item \textbf{Power constraints}: Battery-dependent operation
\end{itemize}

\begin{mnemonicbox}
\mnemonic{Mobile Ad Hoc Networks Enable Everywhere}
\end{mnemonicbox}
\end{solutionbox}

\questionmarks{5(a OR)}{3}{WCDMA ના મુખ્ય લક્ષણો લખો.}
\begin{solutionbox}
\textbf{WCDMA (Wideband CDMA)} enhanced capabilities પૂરી પાડતો 3G standard છે.

\begin{table}[H]
\centering
\caption{WCDMA મુખ્ય લક્ષણો}
\begin{tabulary}{\textwidth}{L L}
\hline
\textbf{લક્ષણ} & \textbf{Specification} \\
\hline
\textbf{Chip Rate} & 3.84 Mcps \\
\textbf{Bandwidth} & 5 MHz \\
\textbf{Data Rates} & 2 Mbps સુધી \\
\textbf{Spreading} & Variable spreading factor \\
\textbf{Power Control} & Fast closed-loop \\
\textbf{Handoff} & Soft અને softer handoff \\
\hline
\end{tabulary}
\end{table}

\begin{itemize}
    \item \textbf{Wideband operation}: 5 MHz bandwidth high data rates પૂરી પાડે છે
    \item \textbf{Variable spreading}: અલગ-અલગ service requirements ને અનુકૂળ થાય છે
\end{itemize}

\begin{mnemonicbox}
\mnemonic{WCDMA Widens Communication Data Magnificently}
\end{mnemonicbox}
\end{solutionbox}

\questionmarks{5(b OR)}{4}{5G ના લાભો લખો.}
\begin{solutionbox}
\textbf{5G લાભો} અગાઉની generations કરતાં નોંધપાત્ર સુધારાઓ રજૂ કરે છે.

\begin{table}[H]
\centering
\caption{5G લાભો}
\begin{tabulary}{\textwidth}{L L}
\hline
\textbf{લાભ} & \textbf{ફાયદો} \\
\hline
\textbf{Ultra-high Speed} & 20 Gbps સુધી peak data rate \\
\textbf{Low Latency} & Critical applications માટે $<$1ms \\
\textbf{Massive IoT} & પ્રતિ km$^2$ 1 million devices \\
\textbf{Network Slicing} & Customized virtual networks \\
\textbf{Enhanced Coverage} & બેહતર indoor અને edge coverage \\
\textbf{Energy Efficiency} & 4G કરતાં 100x વધુ કાર્યક્ષમ \\
\textbf{High Reliability} & 99.999\% availability \\
\hline
\end{tabulary}
\end{table}

\begin{itemize}
    \item \textbf{Enhanced mobile broadband}: AR/VR અને 4K/8K video streaming support કરે છે
    \item \textbf{Ultra-reliable communications}: Autonomous vehicles અને remote surgery શક્ય બનાવે છે
    \item \textbf{Massive machine communications}: Smart cities અને Industry 4.0 support કરે છે
    \item \textbf{Flexible network architecture}: Software-defined networking capabilities
\end{itemize}

\begin{mnemonicbox}
\mnemonic{5G Generates Great Gigabit Growth}
\end{mnemonicbox}
\end{solutionbox}

\questionmarks{5(c OR)}{7}{બ્લોક ડાયાગ્રામ સાથે OFDM ની કામગીરી સમજાવો.}
\begin{solutionbox}
\textbf{OFDM (Orthogonal Frequency Division Multiplexing)} high-speed data transmission માટે બહુવિધ subcarriers વાપરે છે.

\begin{figure}[H]
    \centering
    \begin{tikzpicture}[gtu flow]
        % Transmitter
        \node (In) {Serial Data};
        \node[gtu block, right=of In] (SP) {S/P};
        \node[gtu block, right=of SP] (Map) {QAM\\Map};
        \node[gtu block, right=of Map] (IFFT) {IFFT};
        \node[gtu block, below=of IFFT] (CP) {Add\\CP};
        \node[gtu block, left=of CP] (PS) {P/S};
        \node[left=of PS] (Out) {Tx};
        
        \draw[gtu arrow] (In) -- (SP);
        \draw[gtu arrow] (SP) -- (Map);
        \draw[gtu arrow] (Map) -- (IFFT);
        \draw[gtu arrow] (IFFT) -- (CP);
        \draw[gtu arrow] (CP) -- (PS);
        \draw[gtu arrow] (PS) -- (Out);
        
        \node[above=0.2cm of Map] {OFDM Transmitter};

        % Receiver flow below...
        \node[below=2cm of In] (RxIn) {Rx};
        \node[gtu block, right=of RxIn] (RxSP) {S/P};
        \node[gtu block, right=of RxSP] (RmCP) {Rem\\CP};
        \node[gtu block, right=of RmCP] (FFT) {FFT};
        \node[gtu block, below=of FFT] (DeMap) {QAM\\DeMap};
        \node[gtu block, left=of DeMap] (RxPS) {P/S};
        \node[left=of RxPS] (RxOut) {Data};
        
        \draw[gtu arrow] (RxIn) -- (RxSP);
        \draw[gtu arrow] (RxSP) -- (RmCP);
        \draw[gtu arrow] (RmCP) -- (FFT);
        \draw[gtu arrow] (FFT) -- (DeMap);
        \draw[gtu arrow] (DeMap) -- (RxPS);
        \draw[gtu arrow] (RxPS) -- (RxOut);
        
        \node[above=0.2cm of RmCP] {OFDM Receiver};
    \end{tikzpicture}
    \caption{OFDM Transmitter and Receiver Block Diagram}
\end{figure}

\begin{table}[H]
\centering
\caption{OFDM પ્રક્રિયાના પગલાં}
\begin{tabulary}{\textwidth}{L L L}
\hline
\textbf{સ્ટેજ} & \textbf{ટ્રાન્સમીટર કાર્ય} & \textbf{રીસીવર કાર્ય} \\
\hline
\textbf{Data Conversion} & Serial to parallel conversion & Parallel to serial reconstruction \\
\textbf{Modulation} & Subcarriers પર QAM mapping & QAM demapping \\
\textbf{Transform} & IFFT time domain signal બનાવે છે & FFT frequency domain recover કરે છે \\
\textbf{Guard Period} & Cyclic prefix ISI અટકાવે છે & Cyclic prefix removal \\
\hline
\end{tabulary}
\end{table}

\textbf{મુખ્ય લક્ષણો:}
\begin{itemize}
    \item \textbf{Orthogonal subcarriers}: બહુવિધ parallel low-rate data streams interference અટકાવે છે
    \item \textbf{FFT/IFFT processing}: Fast transforms વાપરીને કાર્યક્ષમ digital implementation
    \item \textbf{Cyclic prefix}: Multipath થી inter-symbol interference અટકાવતો guard interval
    \item \textbf{Spectral efficiency}: મર્યાદિત bandwidth માં high data rates હાંસલ કરાય છે
    \item \textbf{Multipath resistance}: વ્યક્તિગત subcarriers flat fading અનુભવે છે
\end{itemize}

\textbf{ઉપયોગો:}
\begin{itemize}
    \item \textbf{WiFi (802.11)}: Wireless LAN communications
    \item \textbf{LTE/4G}: Mobile broadband networks
    \item \textbf{Digital TV}: DVB-T terrestrial broadcasting
    \item \textbf{WiMAX}: Broadband wireless access
\end{itemize}

\textbf{ફાયદા:}
\begin{itemize}
    \item \textbf{High spectral efficiency}: Optimal bandwidth utilization
    \item \textbf{મજબૂતાઈ}: Frequency selective fading સામે પ્રતિકારક
    \item \textbf{લવચીકતા}: પ્રતિ subcarrier adaptive modulation
    \item \textbf{Implementation}: Digital signal processing hardware સરળ બનાવે છે
\end{itemize}

\begin{table}[H]
\centering
\caption{OFDM પેરામીટર્સ}
\begin{tabulary}{\textwidth}{L L}
\hline
\textbf{પેરામીટર} & \textbf{સામાન્ય મૂલ્યો} \\
\hline
\textbf{Subcarriers} & 64, 128, 256, 512, 1024 \\
\textbf{Modulation} & BPSK, QPSK, 16-QAM, 64-QAM \\
\textbf{Cyclic Prefix} & Symbol duration નો 1/4, 1/8, 1/16 \\
\textbf{Applications} & WiFi, LTE, DVB, WiMAX \\
\hline
\end{tabulary}
\end{table}

\begin{mnemonicbox}
\mnemonic{OFDM Offers Outstanding Data Multiplexing}
\end{mnemonicbox}
\end{solutionbox}

\end{document}
