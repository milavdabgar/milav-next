\documentclass{article}

% content/resources/templates/preamble.tex
\usepackage[margin=0.6in]{geometry}
\author{Milav Dabgar}
\usepackage{amsmath,amssymb,amsthm}
\usepackage{booktabs}
\usepackage{multirow}
\usepackage{xcolor}
\usepackage{tcolorbox}
\tcbuselibrary{breakable,skins}
\usepackage[colorlinks=true,linkcolor=blue]{hyperref}
\usepackage{titlesec}
\usepackage{enumitem}
\usepackage{tikz}
\usepackage{pgfplots}
\usepackage{circuitikz}
\usepackage[version=4]{mhchem}
\usepackage{longtable}
\usepackage{array}
\usepackage{float}
\usepackage{caption}
\usepackage{listings}

\lstset{
  basicstyle=\small\ttfamily,
  breaklines=true,
  breakatwhitespace=false,
  postbreak=\mbox{\textcolor{red}{$\hookrightarrow$}\space},
  float=false,
  numbers=left,
  numberstyle=\tiny\color{gray},
  numbersep=10pt,
  xleftmargin=2em,
  keywordstyle=\color{blue},
  commentstyle=\color{green!60!black},
  stringstyle=\color{purple},
  backgroundcolor=\color{gray!5},
  showstringspaces=false,
  tabsize=2,
  captionpos=b,
  keepspaces=true,
  columns=flexible
}

\pgfplotsset{compat=1.18}
\usetikzlibrary{shapes,arrows,positioning,calc,patterns,decorations.pathmorphing,decorations.markings,arrows.meta}

% Color scheme
\definecolor{headcolor}{RGB}{0,102,204}
\definecolor{keycolor}{RGB}{220,20,60}
\definecolor{solutioncolor}{RGB}{34,139,34}
\definecolor{mnemoniccolor}{RGB}{148,0,211}
\definecolor{codecolor}{RGB}{0,0,100}

% Spacing
\setlength{\parskip}{3pt}
\setlist[itemize]{nosep}
\setlist[enumerate]{nosep}

% Title formatting
\titleformat{\section}{\Large\bfseries\color{headcolor}}{\thesection}{1em}{}
\titleformat{\subsection}{\large\bfseries\color{headcolor}}{\thesubsection}{1em}{}

% Pandoc tightlist compatibility
\providecommand{\tightlist}{%
  \setlength{\itemsep}{0pt}\setlength{\parskip}{0pt}}

% Pandoc longtable compatibility
\newcounter{none}
\def\thenone{}


% content/resources/templates/english-boxes.tex
% This file is currently empty - it exists to maintain consistency with the import structure.
% Add custom environments here if needed in the future.


% Custom commands for GTU solutions
% This file defines semantic commands for consistent formatting

% Question command with automatic formatting
\newcommand{\question}[2]{%
  \section*{Question #1}%
  \textbf{#2}%
}

% OR question variant
\newcommand{\questionor}[2]{%
  \section*{Question #1 OR}%
  \textbf{#2}%
}

% Proper table environment with caption
\newenvironment{answertable}[1]{%
  \begin{table}[htbp]
  \centering
  \caption{#1}
}{%
  \end{table}
}

% Proper figure environment for diagrams
\newenvironment{answerdiagram}[1]{%
  \begin{figure}[htbp]
  \centering
  \caption{#1}
}{%
  \end{figure}
}

% Semantic markup for key terms
\newcommand{\keyword}[1]{\textbf{#1}}
\newcommand{\code}[1]{\texttt{#1}}
\newcommand{\classname}[1]{\texttt{#1}}
\newcommand{\methodname}[1]{\texttt{#1}}

% Proper quotation marks
\newcommand{\mnemonic}[1]{``#1''}


\title{Mobile \& Wireless Communication (4351104) - Winter 2024 Solution}
\date{November 29, 2024}

\begin{document}
\maketitle

\questionmarks{1(a)}{3}{Explain umbrella cell.}
\begin{solutionbox}
\textbf{Umbrella cell} is a large coverage area cell that overlays smaller cells to provide continuous coverage and handle overflow traffic.

\begin{table}[H]
\centering
\caption{Umbrella Cell Characteristics}
\begin{tabulary}{\textwidth}{L L}
\hline
\textbf{Feature} & \textbf{Description} \\
\hline
\textbf{Coverage} & Large geographic area \\
\textbf{Purpose} & Handle overflow traffic from microcells \\
\textbf{Antenna} & High-power, elevated position \\
\textbf{Users} & Fast-moving vehicles, emergency calls \\
\hline
\end{tabulary}
\end{table}

\begin{itemize}
    \item \textbf{Large coverage}: Covers wide geographical area with high-power base station
    \item \textbf{Traffic management}: Handles calls when smaller cells are congested
    \item \textbf{Mobility support}: Serves fast-moving users crossing multiple cell boundaries
\end{itemize}

\begin{mnemonicbox}
\mnemonic{Umbrella Covers Large Areas}
\end{mnemonicbox}
\end{solutionbox}

\questionmarks{1(b)}{4}{Define cell and cluster.}
\begin{solutionbox}
\textbf{Cell} and \textbf{cluster} are fundamental concepts in cellular communication systems.

\begin{table}[H]
\centering
\caption{Cell vs Cluster Comparison}
\begin{tabulary}{\textwidth}{L L L}
\hline
\textbf{Parameter} & \textbf{Cell} & \textbf{Cluster} \\
\hline
\textbf{Definition} & Single coverage area served by one base station & Group of cells using different frequencies \\
\textbf{Size} & Limited by antenna power and interference & Contains N cells (typically 3, 4, 7, 12) \\
\textbf{Frequency} & Uses specific frequency set & Uses all available frequencies once \\
\textbf{Purpose} & Provide coverage to specific area & Enable frequency reuse pattern \\
\hline
\end{tabulary}
\end{table}

\begin{itemize}
    \item \textbf{Cell}: Geographic area served by single base station with specific frequency allocation
    \item \textbf{Cluster}: Group of adjacent cells that collectively use entire frequency spectrum
    \item \textbf{Frequency reuse}: Same frequencies can be reused in different clusters
    \item \textbf{Pattern repetition}: Cluster pattern repeats throughout coverage area
\end{itemize}

\begin{mnemonicbox}
\mnemonic{Cells Cluster for Complete Coverage}
\end{mnemonicbox}
\end{solutionbox}

\questionmarks{1(c)}{7}{Describe fundamental concept behind cellular communication systems.}
\begin{solutionbox}
\textbf{Cellular communication} divides service area into small cells to maximize spectrum efficiency and capacity.

\begin{figure}[H]
    \centering
    \begin{tikzpicture}[gtu flow]
        % Hexagonal grid
        \foreach \x/\y/\t/\c in {0/0/A/f1, 1.5/0.866/B/f2, 3/0/C/f3, 0/1.732/D/f4, 1.5/2.598/E/f5, 3/1.732/F/f6, 1.5/-0.866/G/f7, 4.5/0.866/A'/f1, 4.5/2.598/B'/f2} {
            \node[regular polygon, regular polygon sides=6, minimum size=2cm, draw, fill=blue!5] at (\x,\y) {\begin{tabular}{c}\textbf{\t} \\ \footnotesize \c\end{tabular}};
        }
        % Labels
        \node[above] at (1.5, 3.2) {\footnotesize Cluster 1 (7 Cells)};
        \node[above] at (4.5, 3.2) {\footnotesize Frequency Reuse};
    \end{tikzpicture}
    \caption{Cellular System Concept with Frequency Reuse}
\end{figure}

\begin{table}[H]
\centering
\caption{Cellular System Benefits}
\begin{tabulary}{\textwidth}{L L}
\hline
\textbf{Concept} & \textbf{Advantage} \\
\hline
\textbf{Frequency Reuse} & Same frequencies used multiple times \\
\textbf{Cell Division} & Smaller coverage areas, more capacity \\
\textbf{Handoff} & Seamless call transfer between cells \\
\textbf{Power Control} & Reduced interference, longer battery life \\
\hline
\end{tabulary}
\end{table}

\begin{itemize}
    \item \textbf{Small cell concept}: Service area divided into hexagonal cells for efficient coverage
    \item \textbf{Frequency reuse}: Limited spectrum used multiple times with adequate separation
    \item \textbf{Base station control}: Each cell served by low-power base station
    \item \textbf{Capacity improvement}: More users supported compared to single large coverage area
    \item \textbf{Interference management}: Co-channel interference controlled through proper cell planning
\end{itemize}

\begin{mnemonicbox}
\mnemonic{Small Cells Support Spectrum Sharing Successfully}
\end{mnemonicbox}
\end{solutionbox}

\questionmarks{1(c OR)}{7}{Explain co-channel interference in cellular communication.}
\begin{solutionbox}
\textbf{Co-channel interference} occurs when cells using same frequencies are too close, causing signal degradation.

\begin{figure}[H]
    \centering
    \begin{tikzpicture}[gtu flow]
        % Cell A
        \node[regular polygon, regular polygon sides=6, minimum size=2.5cm, draw, fill=red!10] (A) at (0,0) {\textbf{Cell A} (f1)};
        \node[above] at (A.north) {Base Station A};
        
        % Cell B
        \node[regular polygon, regular polygon sides=6, minimum size=2.5cm, draw, fill=red!10] (C) at (6,0) {\textbf{Cell C} (f1)};
        \node[above] at (C.north) {Base Station C};
        
        % Mobile in Interference Zone
        \node[draw, circle, fill=yellow!20, minimum size=1cm] (M) at (3, -1.5) {Mobile};
        \node[below] at (M.south) {\footnotesize Interference Zone};
        
        % Definition of D
        \draw[<->, dashed] (A.center) -- (C.center) node[midway, above] {Reuse Distance (D)};
        
        % Interference Signals
        \draw[->, gtu arrow, red, dashed] (A) -- (M) node[midway, left] {\footnotesize Signal};
        \draw[->, gtu arrow, red, dashed] (C) -- (M) node[midway, right] {\footnotesize Interference};
    \end{tikzpicture}
    \caption{Co-channel Interference Mechanism}
\end{figure}

\begin{table}[H]
\centering
\caption{Co-channel Interference Parameters}
\begin{tabulary}{\textwidth}{L L L}
\hline
\textbf{Parameter} & \textbf{Description} & \textbf{Impact} \\
\hline
\textbf{Reuse Distance} & Distance between co-channel cells & Higher distance = Less interference \\
\textbf{C/I Ratio} & Carrier to Interference ratio & Must be $\ge$ 18 dB for good quality \\
\textbf{Cluster Size} & Number of cells in cluster & Larger cluster = More separation \\
\hline
\end{tabulary}
\end{table}

\begin{itemize}
    \item \textbf{Signal overlap}: Same frequency signals from different cells interfere
    \item \textbf{Quality degradation}: Causes call drops and poor voice quality
    \item \textbf{Distance factor}: Interference reduces with square of distance
    \item \textbf{Mitigation methods}: Proper cell planning, power control, antenna design
\end{itemize}

\begin{mnemonicbox}
\mnemonic{Co-channel Causes Call Quality Concerns}
\end{mnemonicbox}
\end{solutionbox}

\questionmarks{2(a)}{3}{Explain cell splitting.}
\begin{solutionbox}
\textbf{Cell splitting} divides congested cells into smaller cells to increase system capacity.

\begin{figure}[H]
    \centering
    \begin{tikzpicture}[gtu flow]
        % Original Large Cell
        \node[regular polygon, regular polygon sides=6, minimum size=3cm, draw, dashed, fill=gray!5] at (0,0) {};
        \node at (0,0.5) {\textbf{Original}};
        
        % Split Cells inside
        \foreach \x/\y in {0/0, 0.75/0.433, -0.75/0.433, 0.75/-0.433, -0.75/-0.433, 0/0.866, 0/-0.866} {
             \node[regular polygon, regular polygon sides=6, minimum size=1cm, draw, fill=green!10] at (\x,\y) {};
        }
        
        \draw[->, gtu arrow] (2,0) -- (4,0) node[midway, above] {Splitting};
        
        % Result description
        \node[gtu block, minimum width=3cm, align=left] at (6,0) {
            \textbf{New Microcells}\\
            - Radius: $R/2$\\
            - Capacity: $\approx 4\times$
        };
    \end{tikzpicture}
    \caption{Cell Splitting Concept}
\end{figure}

\begin{itemize}
    \item \textbf{Capacity increase}: Each new cell handles fewer users with better service quality
    \item \textbf{Power reduction}: New base stations use lower power to cover smaller areas
    \item \textbf{Frequency management}: Original frequencies distributed among new smaller cells
\end{itemize}

\begin{mnemonicbox}
\mnemonic{Split Cells Serve Subscribers Successfully}
\end{mnemonicbox}
\end{solutionbox}

\questionmarks{2(b)}{4}{Explain channel assignment strategies.}
\begin{solutionbox}
\textbf{Channel assignment} strategies determine how frequencies are allocated to cells for optimal performance.

\begin{table}[H]
\centering
\caption{Channel Assignment Strategies}
\begin{tabulary}{\textwidth}{L L L L}
\hline
\textbf{Strategy} & \textbf{Description} & \textbf{Advantages} & \textbf{Disadvantages} \\
\hline
\textbf{Fixed} & Channels permanently assigned to cells & Simple, predictable & Inefficient during low traffic \\
\textbf{Dynamic} & Channels assigned based on demand & Efficient spectrum use & Complex implementation \\
\textbf{Hybrid} & Combination of fixed and dynamic & Balanced approach & Moderate complexity \\
\hline
\end{tabulary}
\end{table}

\begin{itemize}
    \item \textbf{Fixed assignment}: Each cell has predetermined set of channels
    \item \textbf{Dynamic assignment}: Channels allocated in real-time based on traffic demand
    \item \textbf{Load balancing}: Distributes traffic evenly across available channels
    \item \textbf{Interference avoidance}: Considers co-channel interference in assignment decisions
\end{itemize}

\begin{mnemonicbox}
\mnemonic{Dynamic Distribution Delivers Optimal Performance}
\end{mnemonicbox}
\end{solutionbox}

\questionmarks{2(c)}{7}{Calculate voice and control channels per cell for 33MHz bandwidth, 25KHz simplex channels, 7-cell reuse, 1MHz for control.}
\begin{solutionbox}
\textbf{Calculation} for channel allocation in cellular system.

\textbf{Given Data:}
\begin{itemize}
    \item Total bandwidth = 33 MHz
    \item Channel bandwidth = 25 KHz (simplex)
    \item Full duplex requires = $2 \times 25$ KHz = 50 KHz
    \item Control spectrum = 1 MHz
    \item Cluster size = 7 cells
\end{itemize}

\textbf{Calculations:}

\textbf{Step 1: Total available channels}
\[ \text{Total channels} = \frac{33 \text{ MHz}}{25 \text{ KHz}} = \frac{33000}{25} = 1320 \text{ channels} \]

\textbf{Step 2: Control channels}
\[ \text{Control channels} = \frac{1 \text{ MHz}}{25 \text{ KHz}} = \frac{1000}{25} = 40 \text{ channels} \]

\textbf{Step 3: Voice channels}
\[ \text{Voice channels} = 1320 - 40 = 1280 \text{ channels} \]

\textbf{Step 4: Duplex voice channels}
\[ \text{Duplex voice channels} = \frac{1280}{2} = 640 \text{ channels} \]

\textbf{Step 5: Channels per cell}
\[ \text{Voice channels per cell} = \frac{640}{7} \approx 91 \text{ channels} \]
\[ \text{Control channels per cell} = \frac{40}{7} \approx 6 \text{ channels} \]

\textbf{Final Answer:}
\begin{itemize}
    \item \textbf{Voice channels per cell: 91}
    \item \textbf{Control channels per cell: 6}
\end{itemize}

\begin{mnemonicbox}
\mnemonic{Calculate Carefully for Channel Count}
\end{mnemonicbox}
\end{solutionbox}

\questionmarks{2(a OR)}{3}{Write functions of FCCH and SCH in GSM.}
\begin{solutionbox}
\textbf{FCCH} and \textbf{SCH} are essential control channels in GSM system for synchronization.

\begin{table}[H]
\centering
\caption{FCCH and SCH Functions}
\begin{tabulary}{\textwidth}{L L L}
\hline
\textbf{Channel} & \textbf{Full Form} & \textbf{Function} \\
\hline
\textbf{FCCH} & Frequency Correction Channel & Provides frequency reference to mobile \\
\textbf{SCH} & Synchronization Channel & Provides timing and cell identity \\
\hline
\end{tabulary}
\end{table}

\begin{itemize}
    \item \textbf{FCCH function}: Enables mobile to synchronize with base station frequency
    \item \textbf{SCH function}: Carries BSIC (Base Station Identity Code) and frame number
    \item \textbf{Timing correction}: Both channels help mobile achieve proper timing synchronization
\end{itemize}

\begin{mnemonicbox}
\mnemonic{FCCH Fixes Frequency, SCH Synchronizes System}
\end{mnemonicbox}
\end{solutionbox}

\questionmarks{2(b OR)}{4}{Write GSM 900 specifications.}
\begin{solutionbox}
\textbf{GSM 900} operates in 900 MHz frequency band with specific technical parameters.

\begin{table}[H]
\centering
\caption{GSM 900 Specifications}
\begin{tabulary}{\textwidth}{L L}
\hline
\textbf{Parameter} & \textbf{Specification} \\
\hline
\textbf{Uplink Frequency} & 890-915 MHz \\
\textbf{Downlink Frequency} & 935-960 MHz \\
\textbf{Duplex Separation} & 45 MHz \\
\textbf{Channel Spacing} & 200 KHz \\
\textbf{Total Channels} & 124 channels \\
\textbf{Access Method} & TDMA/FDMA \\
\textbf{Modulation} & GMSK \\
\textbf{Power Classes} & 2W, 8W, 20W \\
\hline
\end{tabulary}
\end{table}

\begin{itemize}
    \item \textbf{Frequency bands}: Separate uplink and downlink frequencies for full duplex operation
    \item \textbf{TDMA structure}: 8 time slots per carrier frequency
\end{itemize}

\begin{mnemonicbox}
\mnemonic{GSM 900 Gives Great Global Coverage}
\end{mnemonicbox}
\end{solutionbox}

\questionmarks{2(c OR)}{7}{Draw and explain GSM architecture.}
\begin{solutionbox}
\textbf{GSM architecture} consists of three main subsystems working together for mobile communication.

\begin{figure}[H]
    \centering
    \begin{tikzpicture}[gtu flow]
        % Nodes
        \node[gtu block] (MS) {Mobile Station\\(MS)};
        \node[gtu block, right=of MS, xshift=1cm] (BSS) {Base Station Subsystem\\(BSS)};
        \node[gtu block, right=of BSS, xshift=1cm] (NSS) {Network Switching\\Subsystem (NSS)};
        \node[gtu block, right=of NSS, xshift=1cm] (PSTN) {PSTN/ISDN};
        
        % Internal Components (Simplified representation)
        \node[below=0.5cm of BSS, font=\footnotesize, align=center] {BTS + BSC};
        \node[below=0.5cm of NSS, font=\footnotesize, align=center] {MSC, HLR,\\VLR, AuC, EIR};
        
        % Connections
        \draw[gtu arrow] (MS) -- (BSS) node[midway, above] {\footnotesize Um};
        \draw[gtu arrow] (BSS) -- (NSS) node[midway, above] {\footnotesize A};
        \draw[gtu arrow] (NSS) -- (PSTN);
        
        % Grouping
        \draw[dashed] ($(BSS.north west)+(-0.3,0.3)$) rectangle ($(BSS.south east)+(0.3,-0.8)$);
        \draw[dashed] ($(NSS.north west)+(-0.3,0.3)$) rectangle ($(NSS.south east)+(0.3,-0.8)$);
    \end{tikzpicture}
    \caption{GSM System Architecture}
\end{figure}

\begin{table}[H]
\centering
\caption{GSM Architecture Components}
\begin{tabulary}{\textwidth}{L L L}
\hline
\textbf{Subsystem} & \textbf{Components} & \textbf{Function} \\
\hline
\textbf{Mobile Station} & Mobile Equipment + SIM & User interface and identity \\
\textbf{BSS} & BTS + BSC & Radio interface and control \\
\textbf{NSS} & MSC, HLR, VLR, AuC & Switching and database management \\
\hline
\end{tabulary}
\end{table}

\begin{itemize}
    \item \textbf{Mobile Station}: Consists of mobile equipment and SIM card for user identification
    \item \textbf{Base Station Subsystem}: Handles radio communication and resource management
    \item \textbf{Network Switching Subsystem}: Manages call switching, routing, and subscriber databases
    \item \textbf{Interfaces}: A-bis (BTS-BSC), A (BSC-MSC) interfaces connect subsystems
\end{itemize}

\begin{mnemonicbox}
\mnemonic{Mobile Base Network - Complete Communication Chain}
\end{mnemonicbox}
\end{solutionbox}

\questionmarks{3(a)}{3}{Draw block diagram of signal processing in GSM.}
\begin{solutionbox}
\textbf{Signal processing} in GSM involves multiple stages for voice and data transmission.

\begin{figure}[H]
    \centering
    \begin{tikzpicture}[gtu flow, node distance=1.5cm]
        \node[gtu block] (Speech) {Speech Input};
        \node[gtu block, right=of Speech, text width=2cm] (Code) {Speech\\Coding\\(13 kbps)};
        \node[gtu block, right=of Code, text width=2cm] (Chan) {Channel\\Coding\\(22.8 kbps)};
        \node[gtu block, below=of Chan, text width=2cm] (Int) {Interleaving};
        \node[gtu block, left=of Int, text width=2cm] (Burst) {Burst\\Formatting};
        \node[gtu block, left=of Burst, text width=2cm] (RF) {Modulation\\\& RF};

        \draw[gtu arrow] (Speech) -- (Code);
        \draw[gtu arrow] (Code) -- (Chan);
        \draw[gtu arrow] (Chan) -- (Int);
        \draw[gtu arrow] (Int) -- (Burst);
        \draw[gtu arrow] (Burst) -- (RF);
    \end{tikzpicture}
    \caption{GSM Signal Processing Steps}
\end{figure}

\begin{itemize}
    \item \textbf{Speech coding}: Converts analog speech to 13 kbps digital data using RPE-LTP
    \item \textbf{Channel coding}: Adds error correction bits increasing rate to 22.8 kbps
    \item \textbf{Interleaving}: Reorders data to combat burst errors from fading
\end{itemize}

\begin{mnemonicbox}
\mnemonic{Speech Signals Systematically Processed Successfully}
\end{mnemonicbox}
\end{solutionbox}

\questionmarks{3(b)}{4}{Write functions of Common Control Channels in GSM.}
\begin{solutionbox}
\textbf{Common Control Channels} manage system information and access procedures in GSM.

\begin{table}[H]
\centering
\caption{Common Control Channels Functions}
\begin{tabulary}{\textwidth}{L L}
\hline
\textbf{Channel} & \textbf{Function} \\
\hline
\textbf{FCCH} & Frequency correction and synchronization \\
\textbf{SCH} & Frame synchronization and cell identification \\
\textbf{BCCH} & Broadcasts system information and cell parameters \\
\textbf{RACH} & Random access for call initiation by mobile \\
\textbf{AGCH} & Assigns dedicated channels to mobiles \\
\textbf{PCH} & Pages mobiles for incoming calls \\
\hline
\end{tabulary}
\end{table}

\begin{itemize}
    \item \textbf{Broadcast function}: BCCH continuously transmits system information
    \item \textbf{Access management}: RACH allows mobiles to request service
    \item \textbf{Channel assignment}: AGCH allocates resources for active calls
    \item \textbf{Paging service}: PCH notifies mobiles of incoming calls
\end{itemize}

\begin{mnemonicbox}
\mnemonic{Common Channels Control Communication Completely}
\end{mnemonicbox}
\end{solutionbox}

\questionmarks{3(c)}{7}{Explain GSM identifiers.}
\begin{solutionbox}
\textbf{GSM identifiers} uniquely identify subscribers, equipment, and network elements.

\begin{table}[H]
\centering
\caption{GSM Identifiers}
\begin{tabulary}{\textwidth}{L L L L}
\hline
\textbf{Identifier} & \textbf{Full Form} & \textbf{Purpose} & \textbf{Format} \\
\hline
\textbf{IMSI} & International Mobile Subscriber Identity & Unique subscriber ID & 15 digits \\
\textbf{IMEI} & International Mobile Equipment Identity & Unique equipment ID & 15 digits \\
\textbf{MSISDN} & Mobile Station ISDN Number & Phone number & Variable length \\
\textbf{TMSI} & Temporary Mobile Subscriber Identity & Temporary ID for security & 32 bits \\
\textbf{LAI} & Location Area Identity & Geographic area identification & MCC+MNC+LAC \\
\textbf{BSIC} & Base Station Identity Code & Cell identification & 6 bits \\
\hline
\end{tabulary}
\end{table}

\begin{itemize}
    \item \textbf{IMSI structure}: MCC (3) + MNC (2-3) + MSIN (9-10 digits)
    \item \textbf{Security purpose}: TMSI protects subscriber identity over radio interface
    \item \textbf{Location management}: LAI helps in efficient paging and location updates
    \item \textbf{Network planning}: BSIC prevents confusion between adjacent cells
\end{itemize}

\begin{mnemonicbox}
\mnemonic{Important Mobile System Identifiers Ensure Security}
\end{mnemonicbox}
\end{solutionbox}

\questionmarks{3(a OR)}{3}{Compare Fast and Slow frequency hopping.}
\begin{solutionbox}
\textbf{Frequency hopping} techniques differ in hopping rate relative to symbol rate.

\begin{table}[H]
\centering
\caption{Fast vs Slow Frequency Hopping}
\begin{tabulary}{\textwidth}{L L L}
\hline
\textbf{Parameter} & \textbf{Fast Hopping} & \textbf{Slow Hopping} \\
\hline
\textbf{Hopping Rate} & $>$ Symbol rate & $<$ Symbol rate \\
\textbf{Symbols per Hop} & $<$ 1 & $>$ 1 \\
\textbf{Complexity} & High & Low \\
\textbf{Applications} & Military, Bluetooth & GSM, CDMA \\
\hline
\end{tabulary}
\end{table}

\begin{itemize}
    \item \textbf{Fast hopping}: Multiple hops per symbol, better security but more complex
    \item \textbf{Slow hopping}: Multiple symbols per hop, simpler implementation
\end{itemize}

\begin{mnemonicbox}
\mnemonic{Fast Frequently Flips, Slow Stays Stable}
\end{mnemonicbox}
\end{solutionbox}

\questionmarks{3(b OR)}{4}{Calculate number of users in GSM 900 band without frequency reuse.}
\begin{solutionbox}
\textbf{Calculation} for maximum users in GSM 900 without frequency reuse.

\textbf{Given GSM 900 Parameters:}
\begin{itemize}
    \item Uplink: 890-915 MHz (25 MHz)
    \item Downlink: 935-960 MHz (25 MHz)
    \item Channel spacing: 200 KHz
    \item Time slots per channel: 8
\end{itemize}

\textbf{Calculations:}

\textbf{Step 1: Available channels}
\[ \text{Total channels} = \frac{25 \text{ MHz}}{200 \text{ KHz}} = \frac{25000}{200} = 125 \text{ channels} \]

\textbf{Step 2: Usable channels}
\[ \text{Guard channels removed} \approx 124 \text{ channels} \]

\textbf{Step 3: Simultaneous users}
\[ \text{Users per channel} = 8 \text{ time slots} \]
\[ \text{Total users} = 124 \times 8 = 992 \text{ users} \]

\textbf{Answer: 992 users can talk simultaneously}

\begin{mnemonicbox}
\mnemonic{Calculate Channels Times Time-slots}
\end{mnemonicbox}
\end{solutionbox}

\questionmarks{3(c OR)}{7}{Draw and explain general block diagram of mobile handset.}
\begin{solutionbox}
\textbf{Mobile handset} consists of several functional blocks working together.

\begin{figure}[H]
    \centering
    \begin{tikzpicture}[gtu flow]
        % Main Blocks
        \node[gtu block] (Base) {Baseband\\Processor};
        \node[gtu block, above=of Base] (IF) {IF Section};
        \node[gtu block, above=of IF] (RF) {RF Section};
        \node[gtu block, above=of RF] (Ant) {Antenna};
        
        % Peripherals
        \node[gtu block, left=of Base] (Audio) {Audio\\Section};
        \node[gtu block, right=of Base] (UI) {Display/\\Keypad};
        \node[gtu block, below=of Base] (Pwr) {Power\\Management};
        \node[gtu block, right=of Pwr] (Batt) {Battery};
        \node[gtu block, left=of Pwr] (SIM) {SIM\\Interface};
        
        % Connections
        \draw[<->, gtu arrow] (Ant) -- (RF);
        \draw[<->, gtu arrow] (RF) -- (IF);
        \draw[<->, gtu arrow] (IF) -- (Base);
        \draw[<->, gtu arrow] (Base) -- (Audio);
        \draw[<->, gtu arrow] (Base) -- (UI);
        \draw[<->, gtu arrow] (Base) -- (Pwr);
        \draw[<->, gtu arrow] (Base) -- (SIM);
        \draw[->, gtu arrow] (Batt) -- (Pwr);
    \end{tikzpicture}
    \caption{Mobile Handset Block Diagram}
\end{figure}

\begin{table}[H]
\centering
\caption{Mobile Handset Blocks}
\begin{tabulary}{\textwidth}{L L}
\hline
\textbf{Block} & \textbf{Function} \\
\hline
\textbf{RF Section} & Signal transmission and reception \\
\textbf{Baseband} & Digital signal processing \\
\textbf{Audio} & Voice input/output processing \\
\textbf{Power Management} & Battery and power control \\
\textbf{User Interface} & Display, keypad, speaker, microphone \\
\hline
\end{tabulary}
\end{table}

\begin{itemize}
    \item \textbf{RF processing}: Handles radio frequency transmission and reception
    \item \textbf{Digital processing}: Baseband performs channel coding, speech processing
    \item \textbf{User interface}: Provides interaction through display, keypad, audio
    \item \textbf{Power control}: Manages battery usage and charging functions
\end{itemize}

\begin{mnemonicbox}
\mnemonic{Mobile Manages Multiple Modules Simultaneously}
\end{mnemonicbox}
\end{solutionbox}

\questionmarks{4(a)}{3}{Write radiation hazards due to mobile.}
\begin{solutionbox}
\textbf{Radiation hazards} from mobile phones are a health concern due to RF energy exposure.

\begin{table}[H]
\centering
\caption{Mobile Radiation Hazards}
\begin{tabulary}{\textwidth}{L L L}
\hline
\textbf{Hazard} & \textbf{Effect} & \textbf{Prevention} \\
\hline
\textbf{SAR Exposure} & Tissue heating & Use hands-free devices \\
\textbf{Brain Effects} & Memory, sleep issues & Limit call duration \\
\textbf{Cancer Risk} & Potential tumor risk & Keep phone away from body \\
\hline
\end{tabulary}
\end{table}

\begin{itemize}
    \item \textbf{SAR (Specific Absorption Rate)}: Measures RF energy absorbed by body tissue
    \item \textbf{Thermal effects}: RF energy can cause localized heating of tissues
    \item \textbf{Non-thermal effects}: Possible impacts on cellular functions and DNA
\end{itemize}

\begin{mnemonicbox}
\mnemonic{Safety Awareness Reduces Radiation Risk}
\end{mnemonicbox}
\end{solutionbox}

\questionmarks{4(b)}{4}{Explain working of baseband section in mobile handset.}
\begin{solutionbox}
\textbf{Baseband section} performs digital signal processing functions in mobile handset.

\begin{table}[H]
\centering
\caption{Baseband Section Functions}
\begin{tabulary}{\textwidth}{L L}
\hline
\textbf{Function} & \textbf{Description} \\
\hline
\textbf{Speech Processing} & Encode/decode voice using vocoder \\
\textbf{Channel Coding} & Add error correction and detection \\
\textbf{Modulation} & Convert digital data to analog signals \\
\textbf{Protocol Processing} & Handle signaling and call control \\
\hline
\end{tabulary}
\end{table}

\begin{itemize}
    \item \textbf{Digital signal processor}: Executes speech coding algorithms (GSM: RPE-LTP)
    \item \textbf{Error correction}: Implements convolutional coding for reliable transmission
    \item \textbf{Control functions}: Manages call setup, handoff, and power control
    \item \textbf{Interface}: Connects RF section with user interface components
\end{itemize}

\begin{mnemonicbox}
\mnemonic{Baseband Brings Better Communication Control}
\end{mnemonicbox}
\end{solutionbox}

\questionmarks{4(c)}{7}{Explain working of DSSS transmitter and receiver.}
\begin{solutionbox}
\textbf{DSSS (Direct Sequence Spread Spectrum)} spreads signal bandwidth using pseudorandom codes.

\begin{figure}[H]
    \centering
    \begin{tikzpicture}[gtu flow]
        % Transmitter
        \node[gtu block] (Data) {Data\\Input};
        \node[gtu block, regular polygon, regular polygon sides=4, shape border rotate=45, right=of Data] (XOR) {XOR};
        \node[gtu block, below=of XOR] (PN) {PN Code\\Gen};
        \node[gtu block, right=of XOR] (Mod) {Modulator};
        \node[right=of Mod] (Out) {RF Output};
        
        \draw[gtu arrow] (Data) -- (XOR);
        \draw[gtu arrow] (PN) -- (XOR);
        \draw[gtu arrow] (XOR) -- (Mod);
        \draw[gtu arrow] (Mod) -- (Out);
        
        \node[above=0.2cm of XOR] {Transmitter};
    \end{tikzpicture}
    
    \vspace{0.5cm}
    
    \begin{tikzpicture}[gtu flow]
        % Receiver
        \node (In) {RF Input};
        \node[gtu block, right=of In] (Demod) {Demodulator};
        \node[gtu block, regular polygon, regular polygon sides=4, shape border rotate=45, right=of Demod] (XOR) {XOR};
        \node[gtu block, below=of XOR] (PN) {PN Code\\Gen};
        \node[gtu block, right=of XOR] (Data) {Data\\Output};
        
        \draw[gtu arrow] (In) -- (Demod);
        \draw[gtu arrow] (Demod) -- (XOR);
        \draw[gtu arrow] (PN) -- (XOR);
        \draw[gtu arrow] (XOR) -- (Data);
        
        \node[above=0.2cm of XOR] {Receiver};
    \end{tikzpicture}
    \caption{DSSS Transmitter and Receiver}
\end{figure}

\begin{table}[H]
\centering
\caption{DSSS Process}
\begin{tabulary}{\textwidth}{L L L}
\hline
\textbf{Stage} & \textbf{Transmitter} & \textbf{Receiver} \\
\hline
\textbf{Spreading} & Data XOR with PN code & Received signal XOR with PN \\
\textbf{Modulation} & Spread signal modulated & Demodulate received signal \\
\textbf{Processing} & Bandwidth increased & Original data recovered \\
\hline
\end{tabulary}
\end{table}

\begin{itemize}
    \item \textbf{Spreading process}: Original data XORed with high-rate pseudorandom sequence
    \item \textbf{Bandwidth expansion}: Signal bandwidth increased by processing gain factor
    \item \textbf{Despreading}: Receiver uses same PN code to recover original data
    \item \textbf{Interference rejection}: Spread spectrum provides resistance to jamming
\end{itemize}

\begin{mnemonicbox}
\mnemonic{Direct Sequence Spreads Signals Successfully}
\end{mnemonicbox}
\end{solutionbox}

\questionmarks{4(a OR)}{3}{Calculate processing gain for DSSS system with 10 Mcps chip rate and 1 Mbps data rate.}
\begin{solutionbox}
\textbf{Processing gain} determines spread spectrum system's performance improvement.

\textbf{Given:}
\begin{itemize}
    \item Chip rate (Rc) = 10 million chips per second = $10 \times 10^6$ cps
    \item Data rate (Rd) = 1 Mbps = $1 \times 10^6$ bps
\end{itemize}

\textbf{Calculation:}
\[ \text{Processing Gain (Gp)} = \frac{\text{Chip rate}}{\text{Data rate}} \]
\[ Gp = \frac{Rc}{Rd} = \frac{10 \times 10^6}{1 \times 10^6} = 10 \]

\textbf{In dB:}
\[ Gp \text{ (dB)} = 10 \log_{10}(10) = 10 \times 1 = 10 \text{ dB} \]

\textbf{Answer: Processing Gain = 10 or 10 dB}

\begin{mnemonicbox}
\mnemonic{Processing Power Provides Protection}
\end{mnemonicbox}
\end{solutionbox}

\questionmarks{4(b OR)}{4}{Explain how data rate is improved in EDGE.}
\begin{solutionbox}
\textbf{EDGE (Enhanced Data rates for GSM Evolution)} improves data rates through advanced modulation.

\begin{table}[H]
\centering
\caption{EDGE Improvements}
\begin{tabulary}{\textwidth}{L L L L}
\hline
\textbf{Parameter} & \textbf{GSM} & \textbf{EDGE} & \textbf{Improvement} \\
\hline
\textbf{Modulation} & GMSK & 8-PSK & 3 bits per symbol vs 1 bit \\
\textbf{Data Rate} & 9.6 kbps & 43.2 kbps per slot & $\approx$ 4.5x increase \\
\textbf{Coding} & Fixed & Adaptive & Link adaptation \\
\textbf{Applications} & Voice, SMS & Multimedia, Internet & Enhanced services \\
\hline
\end{tabulary}
\end{table}

\begin{itemize}
    \item \textbf{8-PSK modulation}: Transmits 3 bits per symbol instead of 1 bit in GMSK
    \item \textbf{Link adaptation}: Dynamically selects coding scheme based on channel quality
    \item \textbf{Backward compatibility}: Works with existing GSM infrastructure
    \item \textbf{Enhanced applications}: Supports multimedia and higher data rate services
\end{itemize}

\begin{mnemonicbox}
\mnemonic{EDGE Enhances Exchange Efficiently}
\end{mnemonicbox}
\end{solutionbox}

\questionmarks{4(c OR)}{7}{Explain call processing in CDMA.}
\begin{solutionbox}
\textbf{CDMA call processing} involves unique procedures for code-based multiple access.

\begin{figure}[H]
    \centering
    \begin{tikzpicture}[gtu flow]
        \node[gtu block] (Power) {Mobile\\Power On};
        \node[gtu block, right=of Power] (Pilot) {Pilot Channel\\Acquisition};
        \node[gtu block, right=of Pilot] (Sync) {Sync Channel\\Processing};
        \node[gtu block, below=of Sync] (Page) {Paging Channel\\Monitor};
        \node[gtu block, left=of Page] (Access) {Access Channel\\Request};
        \node[gtu block, left=of Access] (Traffic) {Traffic Channel\\Assignment};
        \node[gtu block, below=of Traffic] (Active) {Active Call\\State};
        
        \draw[gtu arrow] (Power) -- (Pilot);
        \draw[gtu arrow] (Pilot) -- (Sync);
        \draw[gtu arrow] (Sync) -- (Page);
        \draw[gtu arrow] (Page) -- (Access);
        \draw[gtu arrow] (Access) -- (Traffic);
        \draw[gtu arrow] (Traffic) -- (Active);
    \end{tikzpicture}
    \caption{CDMA Call Initialization Process}
\end{figure}

\begin{table}[H]
\centering
\caption{CDMA Call Processing Stages}
\begin{tabulary}{\textwidth}{L L L}
\hline
\textbf{Stage} & \textbf{Process} & \textbf{Function} \\
\hline
\textbf{Initialization} & Pilot acquisition & Find strongest base station \\
\textbf{Idle State} & Monitor paging & Listen for incoming calls \\
\textbf{Access} & Random access & Request service from network \\
\textbf{Traffic} & Dedicated channel & Active communication \\
\textbf{Handoff} & Soft handoff & Seamless cell transition \\
\hline
\end{tabulary}
\end{table}

\begin{itemize}
    \item \textbf{Pilot channel}: Provides timing reference and system identification
    \item \textbf{Rake receiver}: Combines multipath signals for improved performance
    \item \textbf{Power control}: Maintains optimal signal levels for all users
    \item \textbf{Soft handoff}: Mobile communicates with multiple base stations simultaneously
    \item \textbf{Code assignment}: Each user assigned unique spreading code
\end{itemize}

\begin{mnemonicbox}
\mnemonic{CDMA Calls Connect Carefully and Clearly}
\end{mnemonicbox}
\end{solutionbox}

\questionmarks{5(a)}{3}{Compare CDMA and GSM.}
\begin{solutionbox}
\textbf{CDMA} and \textbf{GSM} represent different approaches to cellular communication.

\begin{table}[H]
\centering
\caption{CDMA vs GSM Comparison}
\begin{tabulary}{\textwidth}{L L L}
\hline
\textbf{Parameter} & \textbf{CDMA} & \textbf{GSM} \\
\hline
\textbf{Access Method} & Code Division & Time/Frequency Division \\
\textbf{Capacity} & Higher & Lower \\
\textbf{Handoff} & Soft handoff & Hard handoff \\
\textbf{Security} & Better (spreading codes) & Good (encryption) \\
\textbf{Global Usage} & Limited & Widespread \\
\textbf{Power Control} & Continuous & Periodic \\
\hline
\end{tabulary}
\end{table}

\begin{itemize}
    \item \textbf{Multiple access}: CDMA uses unique codes, GSM uses time slots
    \item \textbf{Call quality}: CDMA provides soft handoff, GSM has hard handoff
\end{itemize}

\begin{mnemonicbox}
\mnemonic{Choose CDMA or GSM Carefully}
\end{mnemonicbox}
\end{solutionbox}

\questionmarks{5(b)}{4}{Write advantages of CDMA.}
\begin{solutionbox}
\textbf{CDMA advantages} make it suitable for high-capacity cellular systems.

\begin{table}[H]
\centering
\caption{CDMA Advantages}
\begin{tabulary}{\textwidth}{L L}
\hline
\textbf{Advantage} & \textbf{Benefit} \\
\hline
\textbf{High Capacity} & More users per spectrum \\
\textbf{Soft Handoff} & Seamless call transfer \\
\textbf{Variable Rate} & Adapts to speech patterns \\
\textbf{Privacy} & Inherent security through spreading \\
\textbf{Multipath Resistance} & Uses rake receiver \\
\textbf{Power Control} & Optimizes battery life \\
\textbf{Frequency Planning} & Same frequency in all cells \\
\hline
\end{tabulary}
\end{table}

\begin{itemize}
    \item \textbf{Spectrum efficiency}: Higher capacity compared to FDMA/TDMA systems
    \item \textbf{Quality advantage}: Soft handoff eliminates call drops during cell transitions
    \item \textbf{Security benefit}: Spread spectrum provides inherent privacy protection
    \item \textbf{Simplified planning}: No frequency reuse planning required
\end{itemize}

\begin{mnemonicbox}
\mnemonic{CDMA Creates Considerable Communication Capacity}
\end{mnemonicbox}
\end{solutionbox}

\questionmarks{5(c)}{7}{Explain MANET in brief and write its applications.}
\begin{solutionbox}
\textbf{MANET (Mobile Ad Hoc Network)} is infrastructure-less network of mobile devices.

\begin{figure}[H]
    \centering
    \begin{tikzpicture}[gtu flow]
        % Nodes
        \node[draw, circle, fill=red!10] (A) at (0,0) {A};
        \node[draw, circle, fill=blue!10] (B) at (2,1) {B};
        \node[draw, circle, fill=green!10] (C) at (2,-1) {C};
        \node[draw, circle, fill=yellow!10] (D) at (4,0) {D};
        
        % Connections
        \draw[dashed, gtu arrow] (A) -- (B);
        \draw[dashed, gtu arrow] (A) -- (C);
        \draw[dashed, gtu arrow] (B) -- (D);
        \draw[dashed, gtu arrow] (C) -- (D);
        \draw[dashed, gtu arrow] (B) -- (C);
        
        \node[below=2cm of C] {No Central Base Station};
    \end{tikzpicture}
    \caption{Structure of Mobile Ad Hoc Network}
\end{figure}

\begin{table}[H]
\centering
\caption{MANET Characteristics vs Applications}
\begin{tabulary}{\textwidth}{L L L}
\hline
\textbf{Characteristic} & \textbf{Feature} & \textbf{Applications} \\
\hline
\textbf{Self-organizing} & No fixed infrastructure & Military communications \\
\textbf{Dynamic topology} & Nodes move freely & Emergency response \\
\textbf{Multi-hop routing} & Intermediate node relay & Disaster recovery \\
\textbf{Distributed control} & No central authority & Sensor networks \\
\textbf{Resource constraints} & Limited battery, bandwidth & Vehicular networks \\
\hline
\end{tabulary}
\end{table}

\textbf{Applications:}
\begin{itemize}
    \item \textbf{Military operations}: Battlefield communications without infrastructure
    \item \textbf{Emergency services}: Disaster response and rescue operations
    \item \textbf{Sensor networks}: Environmental monitoring and data collection
    \item \textbf{Vehicular networks}: Car-to-car communication for traffic management
    \item \textbf{Personal area networks}: Device-to-device communication
    \item \textbf{Academic research}: Collaborative computing environments
\end{itemize}

\textbf{Advantages:}
\begin{itemize}
    \item \textbf{Rapid deployment}: No infrastructure setup required
    \item \textbf{Self-healing}: Automatic route reconfiguration when nodes fail
    \item \textbf{Cost effective}: No base station installation costs
\end{itemize}

\textbf{Disadvantages:}
\begin{itemize}
    \item \textbf{Limited bandwidth}: Shared wireless medium
    \item \textbf{Security challenges}: Vulnerable to attacks
    \item \textbf{Power constraints}: Battery-dependent operation
\end{itemize}

\begin{mnemonicbox}
\mnemonic{Mobile Ad Hoc Networks Enable Everywhere}
\end{mnemonicbox}
\end{solutionbox}

\questionmarks{5(a OR)}{3}{Write key features of WCDMA.}
\begin{solutionbox}
\textbf{WCDMA (Wideband CDMA)} is the 3G standard offering enhanced capabilities.

\begin{table}[H]
\centering
\caption{WCDMA Key Features}
\begin{tabulary}{\textwidth}{L L}
\hline
\textbf{Feature} & \textbf{Specification} \\
\hline
\textbf{Chip Rate} & 3.84 Mcps \\
\textbf{Bandwidth} & 5 MHz \\
\textbf{Data Rates} & Up to 2 Mbps \\
\textbf{Spreading} & Variable spreading factor \\
\textbf{Power Control} & Fast closed-loop \\
\textbf{Handoff} & Soft and softer handoff \\
\hline
\end{tabulary}
\end{table}

\begin{itemize}
    \item \textbf{Wideband operation}: 5 MHz bandwidth provides high data rates
    \item \textbf{Variable spreading}: Adapts to different service requirements
\end{itemize}

\begin{mnemonicbox}
\mnemonic{WCDMA Widens Communication Data Magnificently}
\end{mnemonicbox}
\end{solutionbox}

\questionmarks{5(b OR)}{4}{Enlist advantages of 5G.}
\begin{solutionbox}
\textbf{5G advantages} represent significant improvements over previous generations.

\begin{table}[H]
\centering
\caption{5G Advantages}
\begin{tabulary}{\textwidth}{L L}
\hline
\textbf{Advantage} & \textbf{Benefit} \\
\hline
\textbf{Ultra-high Speed} & Up to 20 Gbps peak data rate \\
\textbf{Low Latency} & $<$1ms for critical applications \\
\textbf{Massive IoT} & 1 million devices per km$^2$ \\
\textbf{Network Slicing} & Customized virtual networks \\
\textbf{Enhanced Coverage} & Better indoor and edge coverage \\
\textbf{Energy Efficiency} & 100x more efficient than 4G \\
\textbf{High Reliability} & 99.999\% availability \\
\hline
\end{tabulary}
\end{table}

\begin{itemize}
    \item \textbf{Enhanced mobile broadband}: Supports AR/VR and 4K/8K video streaming
    \item \textbf{Ultra-reliable communications}: Enables autonomous vehicles and remote surgery
    \item \textbf{Massive machine communications}: Supports smart cities and Industry 4.0
    \item \textbf{Flexible network architecture}: Software-defined networking capabilities
\end{itemize}

\begin{mnemonicbox}
\mnemonic{5G Generates Great Gigabit Growth}
\end{mnemonicbox}
\end{solutionbox}

\questionmarks{5(c OR)}{7}{Explain working of OFDM with block diagram.}
\begin{solutionbox}
\textbf{OFDM (Orthogonal Frequency Division Multiplexing)} uses multiple subcarriers for high-speed data transmission.

\begin{figure}[H]
    \centering
    \begin{tikzpicture}[gtu flow]
        % Transmitter
        \node (In) {Serial Data};
        \node[gtu block, right=of In] (SP) {S/P};
        \node[gtu block, right=of SP] (Map) {QAM\\Map};
        \node[gtu block, right=of Map] (IFFT) {IFFT};
        \node[gtu block, below=of IFFT] (CP) {Add\\CP};
        \node[gtu block, left=of CP] (PS) {P/S};
        \node[left=of PS] (Out) {Tx};
        
        \draw[gtu arrow] (In) -- (SP);
        \draw[gtu arrow] (SP) -- (Map);
        \draw[gtu arrow] (Map) -- (IFFT);
        \draw[gtu arrow] (IFFT) -- (CP);
        \draw[gtu arrow] (CP) -- (PS);
        \draw[gtu arrow] (PS) -- (Out);
        
        \node[above=0.2cm of Map] {OFDM Transmitter};

        % Receiver flow below...
        \node[below=2cm of In] (RxIn) {Rx};
        \node[gtu block, right=of RxIn] (RxSP) {S/P};
        \node[gtu block, right=of RxSP] (RmCP) {Rem\\CP};
        \node[gtu block, right=of RmCP] (FFT) {FFT};
        \node[gtu block, below=of FFT] (DeMap) {QAM\\DeMap};
        \node[gtu block, left=of DeMap] (RxPS) {P/S};
        \node[left=of RxPS] (RxOut) {Data};
        
        \draw[gtu arrow] (RxIn) -- (RxSP);
        \draw[gtu arrow] (RxSP) -- (RmCP);
        \draw[gtu arrow] (RmCP) -- (FFT);
        \draw[gtu arrow] (FFT) -- (DeMap);
        \draw[gtu arrow] (DeMap) -- (RxPS);
        \draw[gtu arrow] (RxPS) -- (RxOut);
        
        \node[above=0.2cm of RmCP] {OFDM Receiver};
    \end{tikzpicture}
    \caption{OFDM Transmitter and Receiver Block Diagram}
\end{figure}

\begin{table}[H]
\centering
\caption{OFDM Process Steps}
\begin{tabulary}{\textwidth}{L L L}
\hline
\textbf{Stage} & \textbf{Transmitter Function} & \textbf{Receiver Function} \\
\hline
\textbf{Data Conversion} & Serial to parallel conversion & Parallel to serial reconstruction \\
\textbf{Modulation} & QAM mapping on subcarriers & QAM demapping \\
\textbf{Transform} & IFFT creates time domain signal & FFT recovers frequency domain \\
\textbf{Guard Period} & Cyclic prefix prevents ISI & Cyclic prefix removal \\
\hline
\end{tabulary}
\end{table}

\textbf{Key Features:}
\begin{itemize}
    \item \textbf{Orthogonal subcarriers}: Multiple parallel low-rate data streams prevent interference
    \item \textbf{FFT/IFFT processing}: Efficient digital implementation using fast transforms
    \item \textbf{Cyclic prefix}: Guard interval prevents inter-symbol interference from multipath
    \item \textbf{Spectral efficiency}: High data rates achieved in limited bandwidth
    \item \textbf{Multipath resistance}: Individual subcarriers experience flat fading
\end{itemize}

\textbf{Applications:}
\begin{itemize}
    \item \textbf{WiFi (802.11)}: Wireless LAN communications
    \item \textbf{LTE/4G}: Mobile broadband networks
    \item \textbf{Digital TV}: DVB-T terrestrial broadcasting
    \item \textbf{WiMAX}: Broadband wireless access
\end{itemize}

\textbf{Advantages:}
\begin{itemize}
    \item \textbf{High spectral efficiency}: Optimal bandwidth utilization
    \item \textbf{Robustness}: Resistant to frequency selective fading
    \item \textbf{Flexibility}: Adaptive modulation per subcarrier
    \item \textbf{Implementation}: Digital signal processing simplifies hardware
\end{itemize}

\begin{table}[H]
\centering
\caption{OFDM Parameters}
\begin{tabulary}{\textwidth}{L L}
\hline
\textbf{Parameter} & \textbf{Typical Values} \\
\hline
\textbf{Subcarriers} & 64, 128, 256, 512, 1024 \\
\textbf{Modulation} & BPSK, QPSK, 16-QAM, 64-QAM \\
\textbf{Cyclic Prefix} & 1/4, 1/8, 1/16 of symbol duration \\
\textbf{Applications} & WiFi, LTE, DVB, WiMAX \\
\hline
\end{tabulary}
\end{table}

\begin{mnemonicbox}
\mnemonic{OFDM Offers Outstanding Data Multiplexing}
\end{mnemonicbox}
\end{solutionbox}

\end{document}
