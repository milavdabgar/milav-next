\documentclass{article}

% content/resources/templates/preamble.tex
\usepackage[margin=0.6in]{geometry}
\author{Milav Dabgar}
\usepackage{amsmath,amssymb,amsthm}
\usepackage{booktabs}
\usepackage{multirow}
\usepackage{xcolor}
\usepackage{tcolorbox}
\tcbuselibrary{breakable,skins}
\usepackage[colorlinks=true,linkcolor=blue]{hyperref}
\usepackage{titlesec}
\usepackage{enumitem}
\usepackage{tikz}
\usepackage{pgfplots}
\usepackage{circuitikz}
\usepackage[version=4]{mhchem}
\usepackage{longtable}
\usepackage{array}
\usepackage{float}
\usepackage{caption}
\usepackage{listings}

\lstset{
  basicstyle=\small\ttfamily,
  breaklines=true,
  breakatwhitespace=false,
  postbreak=\mbox{\textcolor{red}{$\hookrightarrow$}\space},
  float=false,
  numbers=left,
  numberstyle=\tiny\color{gray},
  numbersep=10pt,
  xleftmargin=2em,
  keywordstyle=\color{blue},
  commentstyle=\color{green!60!black},
  stringstyle=\color{purple},
  backgroundcolor=\color{gray!5},
  showstringspaces=false,
  tabsize=2,
  captionpos=b,
  keepspaces=true,
  columns=flexible
}

\pgfplotsset{compat=1.18}
\usetikzlibrary{shapes,arrows,positioning,calc,patterns,decorations.pathmorphing,decorations.markings,arrows.meta}

% Color scheme
\definecolor{headcolor}{RGB}{0,102,204}
\definecolor{keycolor}{RGB}{220,20,60}
\definecolor{solutioncolor}{RGB}{34,139,34}
\definecolor{mnemoniccolor}{RGB}{148,0,211}
\definecolor{codecolor}{RGB}{0,0,100}

% Spacing
\setlength{\parskip}{3pt}
\setlist[itemize]{nosep}
\setlist[enumerate]{nosep}

% Title formatting
\titleformat{\section}{\Large\bfseries\color{headcolor}}{\thesection}{1em}{}
\titleformat{\subsection}{\large\bfseries\color{headcolor}}{\thesubsection}{1em}{}

% Pandoc tightlist compatibility
\providecommand{\tightlist}{%
  \setlength{\itemsep}{0pt}\setlength{\parskip}{0pt}}

% Pandoc longtable compatibility
\newcounter{none}
\def\thenone{}


% content/resources/templates/gujarati-boxes.tex
\usepackage{fontspec}
\usepackage{polyglossia}

% Set Gujarati as main language (document is primarily in Gujarati)
% Note: gloss-gujarati.ldf doesn't exist in polyglossia, but it will use hyphenation patterns
\setdefaultlanguage{gujarati}
\setotherlanguage{english}

% Configure Gujarati font properly
% Use Language=Default to prevent polyglossia from trying to add language-specific features
% that don't exist for Gujarati, which causes "empty feature" warnings
\newfontfamily\gujaratifont[Script=Gujarati,AutoFakeBold=2.5,AutoFakeSlant=0.3]{Noto Sans Gujarati}
\setmainfont[Script=Gujarati,AutoFakeBold=2.5,AutoFakeSlant=0.3]{Noto Sans Gujarati}
% Use Noto Sans Gujarati for monospace to support Gujarati in text
\setmonofont[Scale=0.9]{Noto Sans Gujarati}

% Configure English to use the same font
\newfontfamily\englishfont[Script=Gujarati,AutoFakeBold=2.5,AutoFakeSlant=0.3]{Noto Sans Gujarati}

% Translations for polyglossia
\gappto\captionsgujarati{
  \renewcommand{\tablename}{કોષ્ટક}
  \renewcommand{\figurename}{આકૃતિ}
}

% Helper for TikZ nodes to ensure Gujarati font
\newcommand{\gu}[1]{{\gujaratifont #1}}

% Custom environments
\newtcolorbox{solutionbox}{
    breakable,
    enhanced,
    colback=solutioncolor!5!white,
    colframe=solutioncolor!75!black,
    fonttitle=\bfseries,
    title=જવાબ
}

\newtcolorbox{solutionboxnobreak}{
 colback=solutioncolor!5!white,
 colframe=solutioncolor!75!black,
 fonttitle=\bfseries,
 title=જવાબ
}

\newtcolorbox{keyformula}{
 breakable,
 enhanced,
 colback=keycolor!5!white,
 colframe=keycolor!75!black,
 fonttitle=\bfseries,
 title=રાસાયણિક સમીકરણ/સૂત્ર
}

\newtcolorbox{mnemonicbox}{
 breakable,
 enhanced,
 colback=mnemoniccolor!5!white,
 colframe=mnemoniccolor!75!black,
 fonttitle=\bfseries,
 title=મેમરી ટ્રીક
}


% Custom commands for GTU solutions
% This file defines semantic commands for consistent formatting

% Question command with automatic formatting
\newcommand{\question}[2]{%
  \section*{Question #1}%
  \textbf{#2}%
}

% OR question variant
\newcommand{\questionor}[2]{%
  \section*{Question #1 OR}%
  \textbf{#2}%
}

% Proper table environment with caption
\newenvironment{answertable}[1]{%
  \begin{table}[htbp]
  \centering
  \caption{#1}
}{%
  \end{table}
}

% Proper figure environment for diagrams
\newenvironment{answerdiagram}[1]{%
  \begin{figure}[htbp]
  \centering
  \caption{#1}
}{%
  \end{figure}
}

% Semantic markup for key terms
\newcommand{\keyword}[1]{\textbf{#1}}
\newcommand{\code}[1]{\texttt{#1}}
\newcommand{\classname}[1]{\texttt{#1}}
\newcommand{\methodname}[1]{\texttt{#1}}

% Proper quotation marks
\newcommand{\mnemonic}[1]{``#1''}


\title{Mobile \& Wireless Communication (4351104) - Summer 2024 Solution}
\date{May 23, 2024}

\begin{document}
\maketitle

\questionmarks{1(અ)}{3}{સિલેક્ટિવ સેલ સમજાવો.}

\begin{solutionbox}
\textbf{સિલેક્ટિવ સેલની લાક્ષણિકતાઓ:}
\begin{center}
\captionof{table}{સિલેક્ટિવ સેલના લક્ષણો}
\begin{tabulary}{\linewidth}{|L|L|}
\hline
\textbf{લક્ષણ} & \textbf{વર્ણન} \\ \hline
હેતુ & ચોક્કસ વિસ્તારો માટે કવરેજ આપે છે \\ \hline
કદ & નાનો કવરેજ વિસ્તાર \\ \hline
ઉપયોગ & ઇન્ડોર લોકેશન, ટનલ, બિલ્ડિંગ \\ \hline
એન્ટેના & ડાયરેક્શનલ એન્ટેના સિસ્ટમ \\ \hline
\end{tabulary}
\end{center}

\begin{itemize}
    \item \keyword{સિલેક્ટિવ કવરેજ}: સિગ્નલની જરૂર હોય તેવા ચોક્કસ ભૌગોલિક વિસ્તારોને લક્ષ્ય બનાવે છે.
    \item \keyword{ઇન્ડોર સોલ્યૂશન}: મુખ્યત્વે બિલ્ડિંગ કવરેજ વધારવા માટે વપરાય છે.
    \item \keyword{ડાયરેક્શનલ ટ્રાન્સમિશન}: કાર્યક્ષમતા માટે ફોકસ્ડ બીમ પેટર્ન વાપરે છે.
\end{itemize}
\end{solutionbox}

\begin{mnemonicbox}
\mnemonic{Select Special Spots}
\end{mnemonicbox}

\questionmarks{1(બ)}{4}{અમ્બ્રેલા સેલ દોરો અને સમજાવો.}

\begin{solutionbox}
\textbf{અમ્બ્રેલા સેલ માળખું:}
\begin{center}
\begin{tikzpicture}[gtu flow]
    \node [gtu block, minimum width=6cm] (umbrella) {Umbrella Cell (Macro)};
    
    \node [gtu block, below left=1cm of umbrella, xshift=1cm] (micro) {Micro Cell};
    \node [gtu block, below right=1cm of umbrella, xshift=-1cm] (pico) {Pico Cell};
    
    \draw [gtu arrow] (umbrella) -- (micro);
    \draw [gtu arrow] (umbrella) -- (pico);
\end{tikzpicture}
\captionof{figure}{અમ્બ્રેલા સેલ કન્સેપ્ટ}
\end{center}

\textbf{લક્ષણો:}
\begin{center}
\captionof{table}{અમ્બ્રેલા સેલના લક્ષણો}
\begin{tabulary}{\linewidth}{|L|L|}
\hline
\textbf{પેરામીટર} & \textbf{વર્ણન} \\ \hline
કવરેજ & મોટા વિસ્તારનું કવરેજ \\ \hline
હેતુ & નાના સેલ્સને ઓવરલે કરે છે \\ \hline
હેન્ડઓફ & ઇન્ટર-સેલ ટ્રાન્ઝિશન સંચાલિત કરે છે \\ \hline
ક્ષમતા & ઓવરફ્લો ટ્રાફિક હેન્ડલ કરે છે \\ \hline
\end{tabulary}
\end{center}

\begin{itemize}
    \item \keyword{મોટું કવરેજ}: નાના સેલ્સ ઉપર વિશાળ વિસ્તારનું સિગ્નલ કવરેજ પૂરું પાડે છે.
    \item \keyword{ટ્રાફિક મેનેજમેન્ટ}: માઇક્રો અને પિકો સેલ્સમાંથી ઓવરફ્લો હેન્ડલ કરે છે.
    \item \keyword{સીમલેસ હેન્ડઓફ}: હલનચલન દરમિયાન સતત કમ્યુનિકેશન સુનિશ્ચિત કરે છે.
\end{itemize}
\end{solutionbox}

\begin{mnemonicbox}
\mnemonic{Umbrella Covers All}
\end{mnemonicbox}

\questionmarks{1(ક)}{7}{સેલ શું છે? ફ્રીક્વન્સી રીયૂઝ વિગતવાર સમજાવો.}

\begin{solutionbox}
\textbf{સેલ અને ફ્રીક્વન્સી રીયૂઝ:}
\begin{center}
\captionof{table}{કન્સેપ્ટ}
\begin{tabulary}{\linewidth}{|L|L|L|}
\hline
\textbf{કન્સેપ્ટ} & \textbf{વ્યાખ્યા} & \textbf{હેતુ} \\ \hline
સેલ & ભૌગોલિક કવરેજ વિસ્તાર & સેવા પ્રદાન \\ \hline
ફ્રીક્વન્સી રીયૂઝ & અલગ સેલ્સમાં સમાન ફ્રીક્વન્સી & સ્પેક્ટ્રમ કાર્યક્ષમતા \\ \hline
ક્લસ્ટર & અનોખી ફ્રીક્વન્સીઓનું જૂથ & ઇન્ટરફેરન્સ કંટ્રોલ \\ \hline
રીયૂઝ ડિસ્ટન્સ & સમાન ફ્રીક્વન્સીઓ વચ્ચે લઘુત્તમ અંતર & સિગ્નલ ગુણવત્તા \\ \hline
\end{tabulary}
\end{center}

\textbf{કન્સેપ્ટ મેપ:}
\begin{center}
\begin{tikzpicture}[gtu flow]
    \node [gtu block] (cell) {સેલ કન્સેપ્ટ\\(હેક્સાગોનલ)};
    \node [gtu block, right=1cm of cell] (bs) {બેઝ સ્ટેશન\\કવરેજ};
    
    \node [gtu block, below=1cm of cell] (reuse) {ફ્રીક્વન્સી રીયૂઝ};
    \node [gtu block, below left=1cm of reuse] (cluster) {ક્લસ્ટર પેટર્ન\\(N=4,7,12)};
    \node [gtu block, below right=1cm of reuse] (cochannel) {કો-ચેનલ\\રીયૂઝ};
    
    \draw [gtu arrow] (cell) -- (bs);
    \draw [gtu arrow] (cell) -- (reuse);
    \draw [gtu arrow] (reuse) -- (cluster);
    \draw [gtu arrow] (reuse) -- (cochannel);
\end{tikzpicture}
\captionof{figure}{ફ્રીક્વન્સી રીયૂઝ માળખું}
\end{center}

\begin{itemize}
    \item \keyword{સેલની વ્યાખ્યા}: એક બેઝ સ્ટેશન એન્ટેના દ્વારા કવર થતો ભૌગોલિક વિસ્તાર.
    \item \keyword{હેક્સાગોનલ પેટર્ન}: ગેપ વિના કવરેજ માટે સૌથી કાર્યક્ષમ આકાર.
    \item \keyword{ફ્રીક્વન્સી રીયૂઝ}: ક્ષમતા માટે બિન-નજીકના સેલ્સમાં સમાન ફ્રીક્વન્સી વપરાય છે.
    \item \keyword{ક્લસ્ટર સાઇઝ}: ફ્રીક્વન્સી રીયૂઝ પેટર્ન નક્કી કરે છે (N=4,7,12).
\end{itemize}
\end{solutionbox}

\begin{mnemonicbox}
\mnemonic{Cells Reuse Frequencies Efficiently}
\end{mnemonicbox}

\questionmarks{1(ક OR)}{7}{સેલ્યુલર કન્સેપ્ટને વિગતવાર સમજાવો.}

\begin{solutionbox}
\textbf{સેલ્યુલર સિસ્ટમના ઘટકો:}
\begin{center}
\captionof{table}{ઘટકો}
\begin{tabulary}{\linewidth}{|L|L|L|}
\hline
\textbf{ઘટક} & \textbf{કાર્ય} & \textbf{ફાયદો} \\ \hline
સેલ ડિવિઝન & વિસ્તારને વહેંચવું & કવરેજ ઓપ્ટિમાઇઝેશન \\ \hline
બેઝ સ્ટેશનો & સેલ્સની સેવા & સિગ્નલ ટ્રાન્સમિશન \\ \hline
MSC & કૉલ રૂટિંગ & નેટવર્ક કનેક્ટિવિટી \\ \hline
ફ્રીક્વન્સી પ્લાનિંગ & સ્પેક્ટ્રમ એલોકેશન & ઇન્ટરફેરન્સ કંટ્રોલ \\ \hline
\end{tabulary}
\end{center}

\textbf{સેલ્યુલર કન્સેપ્ટ ફ્લો:}
\begin{center}
\begin{tikzpicture}[gtu flow]
    \node [gtu start] (large) {મોટો કવરેજ વિસ્તાર};
    \node [gtu process, below=0.5cm of large] (div) {સેલ ડિવિઝન};
    \node [gtu block, below=0.5cm of div] (bs) {બહુવિધ બેઝ સ્ટેશનો};
    \node [gtu decision, below=0.5cm of bs] (reuse) {ફ્રીક્વન્સી રીયૂઝ};
    \node [gtu stop, below=0.5cm of reuse] (cap) {હાઇ કેપેસિટી સિસ્ટમ};
    
    \draw [gtu arrow] (large) -- (div);
    \draw [gtu arrow] (div) -- (bs);
    \draw [gtu arrow] (bs) -- (reuse);
    \draw [gtu arrow] (reuse) -- (cap);
\end{tikzpicture}
\captionof{figure}{સેલ્યુલર કન્સેપ્ટ}
\end{center}

\begin{itemize}
    \item \keyword{વિસ્તાર વિભાજન}: મોટા સર્વિસ વિસ્તારને નાના હેક્સાગોનલ સેલ્સમાં વહેંચવામાં આવે છે.
    \item \keyword{પાવર કંટ્રોલ}: લો પાવર ટ્રાન્સમિટર ઇન્ટરફેરન્સ ઘટાડે છે.
    \item \keyword{ફ્રીક્વન્સી કાર્યક્ષમતા}: દૂરના સેલ્સમાં સમાન ફ્રીક્વન્સી ફરીથી વાપરવામાં આવે છે.
    \item \keyword{ક્ષમતા વૃદ્ધિ}: વધુ સાથે સાથે વપરાશકર્તાઓની સેવા કરવામાં આવે છે.
\end{itemize}
\end{solutionbox}

\begin{mnemonicbox}
\mnemonic{Divide Area For Better Service}
\end{mnemonicbox}

\questionmarks{2(અ)}{3}{પૂર્ણ સ્વરૂપ લખો: (i) IMEI (ii) LTE (iii) GSM}

\begin{solutionbox}
\begin{center}
\captionof{table}{પૂર્ણ સ્વરૂપો}
\begin{tabulary}{\linewidth}{|L|L|L|}
\hline
\textbf{સંક્ષેપ} & \textbf{પૂર્ણ સ્વરૂપ} & \textbf{હેતુ} \\ \hline
\textbf{IMEI} & International Mobile Equipment Identity & ડિવાઇસ ઓળખ \\ \hline
\textbf{LTE} & Long Term Evolution & 4G સ્ટાન્ડર્ડ \\ \hline
\textbf{GSM} & Global System for Mobile Communication & 2G સ્ટાન્ડર્ડ \\ \hline
\end{tabulary}
\end{center}
\end{solutionbox}

\begin{mnemonicbox}
\mnemonic{Identity Long-term Global}
\end{mnemonicbox}

\questionmarks{2(બ)}{4}{MAHO ને વિગતવાર સમજાવો.}

\begin{solutionbox}
\textbf{MAHO (Mobile Assisted Handoff):}
\begin{center}
\captionof{table}{લાક્ષણિકતાઓ}
\begin{tabulary}{\linewidth}{|L|L|}
\hline
\textbf{લક્ષણ} & \textbf{વર્ણન} \\ \hline
કાર્ય & હેન્ડઓફ નિર્ણયમાં મોબાઇલ મદદ કરે છે \\ \hline
માપ & સિગ્નલ સ્ટ્રેંથ મોનિટરિંગ \\ \hline
રિપોર્ટિંગ & મોબાઇલ નેટવર્કને રિપોર્ટ કરે છે \\ \hline
\end{tabulary}
\end{center}

\textbf{પ્રક્રિયા ફ્લો:}
\begin{center}
\begin{tikzpicture}[gtu flow]
    \node [gtu start] (mob) {મોબાઇલ યુનિટ};
    \node [gtu process, right=1cm of mob, align=center] (meas) {સિગ્નલ સ્ટ્રેંથ\\માપો};
    \node [gtu process, right=1cm of meas, align=center] (rep) {બેઝ સ્ટેશનને\\રિપોર્ટ};
    \node [gtu decision, below=1cm of rep] (net) {નેટવર્ક નિર્ણય};
    \node [gtu stop, left=1cm of net] (ho) {હેન્ડઓફ ચલાવો};
    
    \draw [gtu arrow] (mob) -- (meas);
    \draw [gtu arrow] (meas) -- (rep);
    \draw [gtu arrow] (rep) -- (net);
    \draw [gtu arrow] (net) -- (ho);
\end{tikzpicture}
\captionof{figure}{MAHO પ્રક્રિયા}
\end{center}

\begin{itemize}
    \item \keyword{મોબાઇલ સહાયતા}: મોબાઇલ યુનિટ પડોશી સેલ સિગ્નલ્સ માપે છે.
    \item \keyword{રિપોર્ટિંગ}: સતત માપ રિપોર્ટ્સ નેટવર્કને મોકલવામાં આવે છે.
    \item \keyword{ગુણવત્તા સુધારણા}: મોબાઇલ ઇનપુટ સાથે બેહતર હેન્ડઓફ નિર્ણયો.
\end{itemize}
\end{solutionbox}

\begin{mnemonicbox}
\mnemonic{Mobile Assists Network Decisions}
\end{mnemonicbox}

\questionmarks{2(ક)}{7}{GSM આર્કિટેક્ચર આકૃતિ સાથે સમજાવો}

\begin{solutionbox}
\textbf{GSM આર્કિટેક્ચર:}
\begin{center}
\begin{tikzpicture}[gtu flow]
    % Nodes
    \node [gtu block] (ms) {મોબાઇલ સ્ટેશન\\(MS)};
    \node [gtu block, right=1cm of ms] (bts) {બેઝ ટ્રાન્સીવર\\સ્ટેશન (BTS)};
    \node [gtu block, right=1cm of bts] (bsc) {બેઝ સ્ટેશન\\કંટ્રોલર (BSC)};
    \node [gtu block, above=1cm of bsc] (msc) {મોબાઇલ સ્વિચિંગ\\સેન્ટર (MSC)};
    
    \node [gtu block, right=1cm of msc] (hlr) {HLR / VLR};
    \node [gtu block, left=1cm of msc] (auc) {AuC / EIR};
    \node [gtu block, above=1cm of msc] (pstn) {PSTN / ISDN};
    
    % Connections
    \draw [gtu arrow] (ms) -- (bts);
    \draw [gtu arrow] (bts) -- (bsc);
    \draw [gtu arrow] (bsc) -- (msc);
    \draw [gtu arrow] (msc) -- (hlr);
    \draw [gtu arrow] (msc) -- (auc);
    \draw [gtu arrow] (msc) -- (pstn);
\end{tikzpicture}
\captionof{figure}{GSM નેટવર્ક આર્કિટેક્ચર}
\end{center}

\begin{center}
\captionof{table}{ઘટકો}
\begin{tabulary}{\linewidth}{|L|L|L|}
\hline
\textbf{ઘટક} & \textbf{પૂર્ણ સ્વરૂપ} & \textbf{કાર્ય} \\ \hline
MS & Mobile Station & વપરાશકર્તા ઉપકરણ \\ \hline
BTS & Base Transceiver & રેડિયો ઇન્ટરફેસ \\ \hline
BSC & Base Controller & રિસોર્સ મેનેજમેન્ટ \\ \hline
MSC & Mobile Switching & કૉલ સ્વિચિંગ \\ \hline
HLR & Home Location & કાયમી ડેટાબેઝ \\ \hline
VLR & Visitor Location & અસ્થાયી ડેટાબેઝ \\ \hline
\end{tabulary}
\end{center}

\begin{itemize}
    \item \keyword{રેડિયો સબસિસ્ટમ}: BTS અને BSC રેડિયો કમ્યુનિકેશન હેન્ડલ કરે છે.
    \item \keyword{નેટવર્ક સબસિસ્ટમ}: MSC, HLR, VLR કૉલ્સ અને મોબિલિટી મેનેજ કરે છે.
    \item \keyword{ઓથેન્ટિકેશન}: AuC સિક્યુરિટી ફંક્શન્સ પૂરા પાડે છે.
\end{itemize}
\end{solutionbox}

\begin{mnemonicbox}
\mnemonic{Mobile Base Network Database}
\end{mnemonicbox}

\questionmarks{2(અ OR)}{3}{સેલ સ્પ્લિટિંગ સમજાવો.}

\begin{solutionbox}
\textbf{સેલ સ્પ્લિટિંગ પ્રક્રિયા:}
\begin{center}
\captionof{table}{પ્રક્રિયાના પગલાં}
\begin{tabulary}{\linewidth}{|L|L|L|}
\hline
\textbf{પગલું} & \textbf{ક્રિયા} & \textbf{પરિણામ} \\ \hline
1 & પાવર ઘટાડો & નાનું કવરેજ \\ \hline
2 & બેઝ સ્ટેશનો ઉમેરો & કવરેજ ગેપ્સ ભરો \\ \hline
3 & ફ્રીક્વન્સી પ્લાનિંગ & ઇન્ટરફેરન્સ કંટ્રોલ જાળવો \\ \hline
4 & ક્ષમતા વૃદ્ધિ & વધુ વપરાશકર્તાઓની સેવા \\ \hline
\end{tabulary}
\end{center}

\begin{itemize}
    \item \keyword{પાવર રિડક્શન}: કવરેજ ઘટાડવા માટે ઓરિજિનલ સેલ પાવર ઘટાડવામાં આવે છે.
    \item \keyword{નવા સેલ્સ}: કવરેજ ગેપ્સમાં વધારાના બેઝ સ્ટેશનો ઇન્સ્ટોલ કરવામાં આવે છે.
    \item \keyword{ક્ષમતા લાભ}: વધુ સેલ્સ એટલે સમાન વિસ્તારમાં વધુ વપરાશકર્તા ક્ષમતા.
\end{itemize}
\end{solutionbox}

\begin{mnemonicbox}
\mnemonic{Split Cells Double Capacity}
\end{mnemonicbox}

\questionmarks{2(બ OR)}{4}{હેન્ડઓફ શું છે? સોફ્ટ અને હાર્ડ હેન્ડઓફ સમજાવો.}

\begin{solutionbox}
\textbf{હેન્ડઓફ પ્રકારો:}
\begin{center}
\captionof{table}{હાર્ડ vs સોફ્ટ હેન્ડઓફ}
\begin{tabulary}{\linewidth}{|L|L|L|}
\hline
\textbf{પ્રકાર} & \textbf{પ્રક્રિયા} & \textbf{ટેકનોલોજી} \\ \hline
હાર્ડ & \keyword{બ્રેક-ધેન-મેક} & GSM, TDMA \\ \hline
સોફ્ટ & \keyword{મેક-ધેન-બ્રેક} & CDMA \\ \hline
\end{tabulary}
\end{center}

\textbf{નિર્ણય ફ્લો:}
\begin{center}
\begin{tikzpicture}[gtu flow]
    \node [gtu start] (move) {મોબાઇલ મૂવિંગ};
    \node [gtu decision, right=1cm of move] (type) {પ્રકાર?};
    
    \node [gtu process, above right=1cm of type] (hard) {હાર્ડ: ડિસ્કનેક્ટ $\to$ કનેક્ટ};
    \node [gtu process, below right=1cm of type] (soft) {સોફ્ટ: કનેક્ટ $\to$ ડિસ્કનેક્ટ};
    
    \draw [gtu arrow] (move) -- (type);
    \draw [gtu arrow] (type) |- (hard);
    \draw [gtu arrow] (type) |- (soft);
\end{tikzpicture}
\captionof{figure}{હેન્ડઓફ મિકેનિઝમ}
\end{center}
\end{solutionbox}

\begin{mnemonicbox}
\mnemonic{Hard Breaks Soft Connects}
\end{mnemonicbox}

\questionmarks{2(ક OR)}{7}{GSM સિગ્નલ પ્રોસેસિંગ આકૃતિ સાથે સમજાવો}

\begin{solutionbox}
\textbf{GSM સિગ્નલ ચેન:}
\begin{center}
\begin{tikzpicture}[gtu flow]
    \node [gtu start] (voice) {વૉઇસ ઇનપુટ};
    \node [gtu process, below=0.5cm of voice] (codec) {સ્પીચ કોડેક};
    \node [gtu process, below=0.5cm of codec] (channel) {ચેનલ કોડિંગ};
    \node [gtu process, below=0.5cm of channel] (inter) {ઇન્ટરલીવિંગ};
    \node [gtu process, right=1cm of inter] (encrypt) {એન્ક્રિપ્શન};
    \node [gtu process, above=0.5cm of encrypt] (burst) {બર્સ્ટ ફોર્મેટિંગ};
    \node [gtu process, above=0.5cm of burst] (mod) {મોડ્યુલેશન};
    \node [gtu stop, above=0.5cm of mod] (rf) {RF ટ્રાન્સમિશન};
    
    \draw [gtu arrow] (voice) -- (codec);
    \draw [gtu arrow] (codec) -- (channel);
    \draw [gtu arrow] (channel) -- (inter);
    \draw [gtu arrow] (inter) -- (encrypt);
    \draw [gtu arrow] (encrypt) -- (burst);
    \draw [gtu arrow] (burst) -- (mod);
    \draw [gtu arrow] (mod) -- (rf);
\end{tikzpicture}
\captionof{figure}{સિગ્નલ પ્રોસેસિંગ સ્ટેજ}
\end{center}

\textbf{પ્રોસેસિંગ સ્ટેજ:}
\begin{itemize}
    \item \keyword{સ્પીચ પ્રોસેસિંગ}: RPE-LTP કોડેક વાપરીને વૉઇસ કમ્પ્રેસ કરવામાં આવે છે.
    \item \keyword{એરર પ્રોટેક્શન}: કન્વોલ્યુશનલ કોડિંગ રિડન્ડન્સી ઉમેરે છે.
    \item \keyword{બર્સ્ટ સ્ટ્રક્ચર}: ડેટાને ટાઇમ સ્લોટ્સમાં ગોઠવવામાં આવે છે.
    \item \keyword{સિક્યુરિટી લેયર}: A5 અલ્ગોરિધમ ડેટાને એન્ક્રિપ્ટ કરે છે.
    \item \keyword{મોડ્યુલેશન}: RF ટ્રાન્સમિશન માટે GMSK મોડ્યુલેશન.
\end{itemize}
\end{solutionbox}

\begin{mnemonicbox}
\mnemonic{Voice Coded Interleaved Encrypted Modulated}
\end{mnemonicbox}

\questionmarks{3(અ)}{3}{સેલ સેક્ટરિંગ સમજાવો.}

\begin{solutionbox}
\textbf{સેલ સેક્ટરિંગના ફાયદા:}
\begin{center}
\captionof{table}{સેક્ટરિંગ}
\begin{tabulary}{\linewidth}{|L|L|}
\hline
\textbf{લક્ષણ} & \textbf{વર્ણન} \\ \hline
એન્ટેના પેટર્ન & ઓમ્નિડાયરેક્શનલને બદલે ડાયરેક્શનલ \\ \hline
સેક્ટર્સ & સેલ દીઠ 3 અથવા 6 સેક્ટર્સ \\ \hline
ક્ષમતા & 3x અથવા 6x ક્ષમતા વૃદ્ધિ \\ \hline
ઇન્ટરફેરન્સ & કો-ચેનલ ઇન્ટરફેરન્સ ઘટાડે છે \\ \hline
\end{tabulary}
\end{center}

\begin{itemize}
    \item \keyword{ડાયરેક્શનલ એન્ટેના}: ઓમ્નિડાયરેક્શનલને સેક્ટર એન્ટેના સાથે બદલો.
    \item \keyword{ક્ષમતા ગુણાકાર}: દરેક સેક્ટરને અલગ સેલ તરીકે ગણવામાં આવે છે.
    \item \keyword{ઇન્ટરફેરન્સ ઘટાડો}: ડાયરેક્શનલ પેટર્ન ઇન્ટરફેરન્સ ઘટાડે છે.
\end{itemize}
\end{solutionbox}

\begin{mnemonicbox}
\mnemonic{Sector Antennas Triple Capacity}
\end{mnemonicbox}

\questionmarks{3(બ)}{4}{GSM કૉલ પ્રક્રિયા સમજાવો.}

\begin{solutionbox}
\textbf{કૉલ સિક્વન્સ:}
\begin{center}
\begin{tikzpicture}[gtu flow]
    \node [gtu start] (req) {કૉલ રિક્વેસ્ટ (મોબાઇલ)};
    \node [gtu process, below=0.5cm of req] (bts) {BTS $\to$ BSC};
    \node [gtu process, below=0.5cm of bts] (msc) {BSC $\to$ MSC};
    \node [gtu decision, below=0.5cm of msc] (auth) {ઓથેન્ટિકેશન (HLR)};
    \node [gtu process, below=0.5cm of auth] (conn) {કનેક્શન (PSTN)};
    
    \draw [gtu arrow] (req) -- (bts);
    \draw [gtu arrow] (bts) -- (msc);
    \draw [gtu arrow] (msc) -- (auth);
    \draw [gtu arrow] (auth) -- (conn);
\end{tikzpicture}
\captionof{figure}{કૉલ સેટઅપ ફ્લો}
\end{center}

\textbf{પગલાં:}
1. \keyword{ઓથેન્ટિકેશન}: વપરાશકર્તા ચકાસણી.
2. \keyword{ચેનલ એલોકેશન}: રિસોર્સ એસાઇનમેન્ટ.
3. \keyword{રૂટિંગ}: પાથ સ્થાપના.
4. \keyword{કનેક્શન}: કમ્યુનિકેશન લિંક.
\end{solutionbox}

\begin{mnemonicbox}
\mnemonic{Authenticate Allocate Route Connect}
\end{mnemonicbox}

\questionmarks{3(ક)}{7}{GPRS સમજાવો.}

\begin{solutionbox}
\textbf{GPRS લક્ષણો:}
\begin{itemize}
    \item \keyword{પેકેટ સ્વિચિંગ}: ડેટા પેકેટ્સમાં ટ્રાન્સમિટ કરવામાં આવે છે.
    \item \keyword{હંમેશા-ઓન}: ડેટા એક્સેસ માટે ડાયલ-અપની જરૂર નથી.
    \item \keyword{વધુ સ્પીડ}: સર્કિટ-સ્વિચ્ડ ડેટા કરતાં નોંધપાત્ર સુધારો (114 kbps).
\end{itemize}

\textbf{GPRS આર્કિટેક્ચર:}
\begin{center}
\begin{tikzpicture}[gtu flow]
    \node [gtu block] (net) {GPRS નેટવર્ક};
    \node [gtu block, below left=1cm of net] (sgsn) {SGSN};
    \node [gtu block, below right=1cm of net] (ggsn) {GGSN};
    
    \node [gtu block, below=1cm of sgsn] (pkt) {પેકેટ ડેટા};
    \node [gtu block, below=1cm of ggsn] (net2) {ઇન્ટરનેટ/બાહ્ય};
    
    \draw [gtu arrow] (net) -- (sgsn);
    \draw [gtu arrow] (net) -- (ggsn);
    \draw [gtu arrow] (sgsn) -- (pkt);
    \draw [gtu arrow] (ggsn) -- (net2);
\end{tikzpicture}
\captionof{figure}{GPRS નોડ્સ}
\end{center}

\begin{itemize}
    \item \textbf{SGSN}: Service GPRS Support Node (મોબિલિટી).
    \item \textbf{GGSN}: Gateway GPRS Support Node (બાહ્ય).
    \item ઇન્ટરનેટ અને ઇમેઇલ સેવાઓ સક્ષમ કરે છે.
\end{itemize}
\end{solutionbox}

\begin{mnemonicbox}
\mnemonic{General Packet Radio Service}
\end{mnemonicbox}

\questionmarks{3(અ OR)}{3}{CDMA ના ફાયદા સમજાવો}

\begin{solutionbox}
\textbf{CDMA ફાયદા:}
\begin{center}
\captionof{table}{CDMA લાભો}
\begin{tabulary}{\linewidth}{|L|L|}
\hline
\textbf{ફાયદો} & \textbf{વર્ણન} \\ \hline
ક્ષમતા & વધુ વપરાશકર્તા ક્ષમતા \\ \hline
સિક્યુરિટી & બિલ્ટ-ઇન એન્ક્રિપ્શન (સ્પ્રેડ સ્પેક્ટ્રમ) \\ \hline
ગુણવત્તા & બેહતર વૉઇસ ગુણવત્તા (સોફ્ટ હેન્ડઓફ) \\ \hline
પાવર & કાર્યક્ષમ પાવર કંટ્રોલ \\ \hline
\end{tabulary}
\end{center}
\end{solutionbox}

\begin{mnemonicbox}
\mnemonic{Capacity Security Quality}
\end{mnemonicbox}

\questionmarks{3(બ OR)}{4}{ફ્રીક્વન્સી હોપિંગ તકનીકો સમજાવો.}

\begin{solutionbox}
\textbf{ફ્રીક્વન્સી હોપિંગ (FH):}
\begin{center}
\captionof{table}{FH પ્રકારો}
\begin{tabulary}{\linewidth}{|L|L|}
\hline
\textbf{પ્રકાર} & \textbf{હેતુ} \\ \hline
સ્લો FH & સિમ્બોલ રેટ કરતાં ઓછું (GSM) \\ \hline
ફાસ્ટ FH & સિમ્બોલ રેટ કરતાં વધારે (મિલિટરી) \\ \hline
\end{tabulary}
\end{center}

\textbf{મિકેનિઝમ:}
\begin{center}
\begin{tikzpicture}[gtu flow]
    \node [gtu start] (data) {ડેટા};
    \node [gtu process, right=0.5cm of data] (spread) {સ્પ્રેડ સ્પેક્ટ્રમ};
    \node [gtu process, right=0.5cm of spread] (synth) {Freq Synth};
    \node [gtu process, below=0.5cm of synth] (patt) {Hop Pattern};
    \node [gtu stop, right=0.5cm of synth] (rf) {RF Tx};
    
    \draw [gtu arrow] (data) -- (spread);
    \draw [gtu arrow] (spread) -- (synth);
    \draw [gtu arrow] (patt) -- (synth);
    \draw [gtu arrow] (synth) -- (rf);
\end{tikzpicture}
\captionof{figure}{ફ્રીક્વન્સી હોપિંગ}
\end{center}

\begin{itemize}
    \item ઇન્ટરફેરન્સ અને જામિંગ ઘટાડે છે.
    \item સિક્યુરિટી વધારે છે.
\end{itemize}
\end{solutionbox}

\begin{mnemonicbox}
\mnemonic{Frequency Hops For Security}
\end{mnemonicbox}

\questionmarks{3(ક OR)}{7}{EDGE સમજાવો.}

\begin{solutionbox}
\textbf{EDGE (Enhanced Data rates for GSM Evolution):}
\begin{center}
\captionof{table}{EDGE સ્પેસિફિકેશન્સ}
\begin{tabulary}{\linewidth}{|L|L|}
\hline
\textbf{પેરામીટર} & \textbf{મૂલ્ય/ફાયદો} \\ \hline
ડેટા રેટ & 384 kbps સુધી (3x GPRS) \\ \hline
મોડ્યુલેશન & 8-PSK (હાઇયર ઓર્ડર) \\ \hline
સુસંગતતા & GSM સાથે બેકવર્ડ કમ્પેટિબલ \\ \hline
\end{tabulary}
\end{center}

\textbf{એન્હાન્સમેન્ટ ફ્લો:}
\begin{center}
\begin{tikzpicture}[gtu flow]
    \node [gtu block] (edge) {EDGE};
    \node [gtu block, below left=1cm of edge] (mod) {8-PSK};
    \node [gtu block, below=1cm of edge] (link) {લિંક એડેપ્ટેશન};
    \node [gtu block, below right=1cm of edge] (red) {Inc. Redundancy};
    
    \draw [gtu arrow] (edge) -- (mod);
    \draw [gtu arrow] (edge) -- (link);
    \draw [gtu arrow] (edge) -- (red);
\end{tikzpicture}
\captionof{figure}{EDGE સુધારાઓ}
\end{center}

\begin{itemize}
    \item \keyword{8-PSK}: GMSK ને બદલે 8-PSK ડેટા રેટ વધારે છે.
    \item \keyword{લિંક એડેપ્ટેશન}: મોડ્યુલેશન સ્કીમ ચેનલ કંડિશન્સ મુજબ એડજસ્ટ થાય છે.
    \item \keyword{3G સ્ટેપિંગ સ્ટોન}: 2G અને 3G ટેકનોલોજીઓ વચ્ચે પુલ.
\end{itemize}
\end{solutionbox}

\begin{mnemonicbox}
\mnemonic{Enhanced Data Gets Excellence}
\end{mnemonicbox}

\questionmarks{4(અ)}{3}{FHSS ટ્રાન્સમિટર બ્લોક આકૃતિ દોરો}

\begin{solutionbox}
\textbf{FHSS ટ્રાન્સમિટર:}
\begin{center}
\begin{tikzpicture}[gtu flow]
    \node [gtu start] (input) {ડેટા ઇનપુટ};
    \node [gtu process, right=1cm of input] (mod) {મોડ્યુલેટર};
    \node [gtu process, right=1cm of mod] (synth) {Freq\\Synthesizer};
    \node [gtu process, below=1cm of synth] (pn) {PN Seq\\Generator};
    \node [gtu process, right=1cm of synth] (amp) {RF Amp};
    \node [gtu stop, right=1cm of amp] (ant) {Antenna};
    
    \draw [gtu arrow] (input) -- (mod);
    \draw [gtu arrow] (mod) -- (synth);
    \draw [gtu arrow] (pn) -- (synth);
    \draw [gtu arrow] (synth) -- (amp);
    \draw [gtu arrow] (amp) -- (ant);
\end{tikzpicture}
\captionof{figure}{FHSS ટ્રાન્સમિટર બ્લોક આકૃતિ}
\end{center}
\end{solutionbox}

\begin{mnemonicbox}
\mnemonic{Data Modulated Frequency Hops}
\end{mnemonicbox}

\questionmarks{4(બ)}{4}{CDMA માં કૉલ પ્રોસેસિંગ સમજાવો}

\begin{solutionbox}
\textbf{પ્રોસેસિંગ ફેઝ:}
\begin{center}
\captionof{table}{CDMA કૉલ સ્ટેજ}
\begin{tabulary}{\linewidth}{|L|L|L|}
\hline
\textbf{ફેઝ} & \textbf{પ્રક્રિયા} & \textbf{હેતુ} \\ \hline
એક્સેસ & સિસ્ટમ એક્સેસ & પ્રારંભિક કનેક્શન \\ \hline
ઓથેન્ટિકેશન & આઇડેન્ટિટી વેરિફિકેશન & સિક્યુરિટી \\ \hline
ટ્રાફિક & કમ્યુનિકેશન & ડેટા ટ્રાન્સફર \\ \hline
રિલીઝ & કૉલ ટર્મિનેશન & રિસોર્સ ક્લિનઅપ \\ \hline
\end{tabulary}
\end{center}

\begin{itemize}
    \item \keyword{સિસ્ટમ એક્સેસ}: મોબાઇલ પાઇલટ ચેનલ એક્વાયર કરે છે.
    \item \keyword{ઓથેન્ટિકેશન}: નેટવર્ક સબ્સ્ક્રાઇબર ક્રેડેન્શિયલ્સ ચકાસે છે.
    \item \keyword{ટ્રાફિક સ્ટેટ}: પાવર કંટ્રોલ સાથે સક્રિય કમ્યુનિકેશન.
    \item \keyword{કૉલ રિલીઝ}: કૉલ સમાપ્ત થાય ત્યારે રિસોર્સ મુક્ત કરવામાં આવે છે.
\end{itemize}
\end{solutionbox}

\begin{mnemonicbox}
\mnemonic{Access Authenticate Transfer Release}
\end{mnemonicbox}

\questionmarks{4(ક)}{7}{OFDM રિસીવર બ્લોક આકૃતિ દોરી સમજાવો}

\begin{solutionbox}
\textbf{OFDM રિસીવર આકૃતિ:}
\begin{center}
\begin{tikzpicture}[gtu flow]
    \node [gtu start] (rf) {RF In};
    \node [gtu process, right=0.5cm of rf] (down) {Down\\Convert};
    \node [gtu process, right=0.5cm of down] (adc) {ADC};
    \node [gtu process, right=0.5cm of adc] (cp) {Remove\\CP};
    \node [gtu process, below=1cm of cp] (fft) {FFT};
    \node [gtu process, left=0.5cm of fft] (ps) {Par $\to$ Ser};
    \node [gtu process, left=0.5cm of ps] (dec) {Decode};
    \node [gtu stop, left=0.5cm of dec] (out) {Data Out};
    
    \draw [gtu arrow] (rf) -- (down);
    \draw [gtu arrow] (down) -- (adc);
    \draw [gtu arrow] (adc) -- (cp);
    \draw [gtu arrow] (cp) -- (fft);
    \draw [gtu arrow] (fft) -- (ps);
    \draw [gtu arrow] (ps) -- (dec);
    \draw [gtu arrow] (dec) -- (out);
\end{tikzpicture}
\captionof{figure}{OFDM રિસીવર}
\end{center}

\textbf{કાર્યો:}
\begin{itemize}
    \item \keyword{ડાઉન કન્વર્ટર}: RF સિગ્નલને બેસબેન્ડમાં કન્વર્ટ કરે છે.
    \item \keyword{રિમૂવ CP}: ISI દૂર કરવા માટે સાયક્લિક પ્રીફિક્સ રિમૂવ કરવામાં આવે છે.
    \item \keyword{FFT}: ઓર્થોગોનલ સબકેરિયર્સને અલગ કરે છે.
    \item \keyword{ચેનલ ડિકોડર}: એરર કરેક્શન અને ડેટા રિકવરી.
\end{itemize}
\end{solutionbox}

\begin{mnemonicbox}
\mnemonic{Receive Convert Remove Transform Decode}
\end{mnemonicbox}

\questionmarks{4(અ OR)}{3}{મોબાઇલને કારણે રેડિયેશનનું જોખમ સમજાવો.}

\begin{solutionbox}
\textbf{રેડિયેશન અસરો:}
\begin{center}
\captionof{table}{જોખમો}
\begin{tabulary}{\linewidth}{|L|L|}
\hline
\textbf{પેરામીટર} & \textbf{અસર/વિગત} \\ \hline
SAR & Specific Absorption Rate (હીટિંગ) \\ \hline
ફ્રીક્વન્સી & 900/1800 MHz (પેનિટ્રેશન) \\ \hline
સેફ્ટી & આંતરરાષ્ટ્રીય સ્ટાન્ડર્ડ લિમિટ્સ \\ \hline
\end{tabulary}
\end{center}

\begin{itemize}
    \item \keyword{SAR}: એનર્જી એબસોર્પ્શન રેટ માપે છે.
    \item \keyword{થર્મલ અસરો}: વધુ SAR ટિશ્યુ હીટિંગનું કારણ બની શકે છે.
\end{itemize}
\end{solutionbox}

\begin{mnemonicbox}
\mnemonic{SAR Safety Absorption Rate}
\end{mnemonicbox}

\questionmarks{4(બ OR)}{4}{મોબાઇલ હેન્ડસેટમાં વપરાતી લિ-પો પ્રકારની બેટરીઓ સમજાવો.}

\begin{solutionbox}
\textbf{લિ-પો લાક્ષણિકતાઓ:}
\begin{center}
\captionof{table}{લિ-પો લક્ષણો}
\begin{tabulary}{\linewidth}{|L|L|}
\hline
\textbf{લક્ષણ} & \textbf{ફાયદો} \\ \hline
કેમિસ્ટ્રી & લિથિયમ પોલિમર (સોલિડ/જેલ) \\ \hline
આકાર & ફ્લેક્સિબલ ફોર્મ ફેક્ટર \\ \hline
ડેન્સિટી & હાઇ એનર્જી ડેન્સિટી \\ \hline
વજન & હલકું \\ \hline
\end{tabulary}
\end{center}

\begin{itemize}
    \item લિક્વિડ ઇલેક્ટ્રોલાઇટને બદલે પોલિમર વાપરે છે.
    \item ડિવાઇસ ડિઝાઇન મુજબ આકાર આપી શકાય છે.
    \item રેપિડ ચાર્જિંગ પ્રોટોકોલ્સને સપોર્ટ કરે છે.
\end{itemize}
\end{solutionbox}

\begin{mnemonicbox}
\mnemonic{Lithium Polymer Power}
\end{mnemonicbox}

\questionmarks{4(ક OR)}{7}{મોબાઇલ હેન્ડસેટ બ્લોક ડાયાગ્રામ સમજાવો.}

\begin{solutionbox}
\textbf{હેન્ડસેટ બ્લોક ડાયાગ્રામ:}
\begin{center}
\begin{tikzpicture}[gtu flow]
    % Central Unit
    \node [gtu block, minimum width=3cm] (base) {બેસબેન્ડ\\પ્રોસેસર};
    
    % Peripherals
    \node [gtu block, above=1cm of base] (rf) {RF સેક્શન};
    \node [gtu block, above=0.5cm of rf] (ant) {એન્ટેના};
    
    \node [gtu block, left=1cm of base] (aud) {ઓડિયો\\કોડેક};
    \node [gtu block, right=1cm of base] (ui) {ડિસ્પ્લે/\\કીપેડ};
    
    \node [gtu block, below=1cm of base] (pwr) {પાવર મેનેજમેન્ટ};
    \node [gtu block, below=0.5cm of pwr] (bat) {બેટરી};
    
    \node [gtu block, below left=1cm of base] (sim) {SIM};
    
    % Connections
    \draw [gtu arrow] (ant) -- (rf);
    \draw [gtu arrow] (rf) -- (base);
    \draw [gtu arrow] (base) -- (rf);
    
    \draw [gtu arrow] (base) -- (aud);
    \draw [gtu arrow] (base) -- (ui);
    \draw [gtu arrow] (sim) -| (base);
    \draw [gtu arrow] (pwr) -- (base);
    \draw [gtu arrow] (pwr) -- (rf);
    \draw [gtu arrow] (bat) -- (pwr);
\end{tikzpicture}
\captionof{figure}{મોબાઇલ હેન્ડસેટ ઘટકો}
\end{center}

\textbf{ઘટકો:}
\begin{itemize}
    \item \keyword{RF સેક્શન}: રેડિયો ટ્રાન્સમિશન/રિસેપ્શન.
    \item \keyword{બેસબેન્ડ}: પ્રોટોકોલ હેન્ડલિંગ.
    \item \keyword{ઓડિયો}: વૉઇસ અને સાઉન્ડ પ્રોસેસિંગ.
    \item \keyword{પાવર}: બેટરી અને ચાર્જિંગ કંટ્રોલ.
    \item \keyword{SIM}: આઇડેન્ટિટી અને ઓથેન્ટિકેશન.
\end{itemize}
\end{solutionbox}

\begin{mnemonicbox}
\mnemonic{Radio Baseband Audio Power Interface}
\end{mnemonicbox}

\questionmarks{5(અ)}{3}{CDMA અને GSM ની સરખામણી કરો}

\begin{solutionbox}
\begin{center}
\captionof{table}{CDMA vs GSM}
\begin{tabulary}{\linewidth}{|L|L|L|}
\hline
\textbf{લક્ષણ} & \textbf{CDMA} & \textbf{GSM} \\ \hline
એક્સેસ & Code Division & Time Division \\ \hline
ક્ષમતા & વધુ & ઓછી \\ \hline
હેન્ડઓફ & Soft & Hard \\ \hline
SIM & જરૂરી નથી & જરૂરી \\ \hline
\end{tabulary}
\end{center}
\end{solutionbox}

\begin{mnemonicbox}
\mnemonic{Code vs Time Division}
\end{mnemonicbox}

\questionmarks{5(બ)}{4}{HSDPA સમજાવો.}

\begin{solutionbox}
\textbf{HSDPA (High Speed Downlink Packet Access):}
\begin{center}
\captionof{table}{લક્ષણો}
\begin{tabulary}{\linewidth}{|L|L|}
\hline
\textbf{લક્ષણ} & \textbf{વર્ણન} \\ \hline
ડેટા રેટ & 14.4 Mbps સુધી \\ \hline
ટેકનોલોજી & 3.5G (UMTS એન્હાન્સમેન્ટ) \\ \hline
દિશા & ડાઉનલિંક ઓપ્ટિમાઇઝેશન \\ \hline
મોડ્યુલેશન & એડેપ્ટિવ (QPSK $\to$ 16-QAM) \\ \hline
\end{tabulary}
\end{center}

\begin{itemize}
    \item 3G ડાઉનલિંક ઝડપ નોંધપાત્ર રીતે વધારે છે.
    \item ફાસ્ટ શેડ્યુલિંગ (2ms ઈન્ટરવલ).
\end{itemize}
\end{solutionbox}

\begin{mnemonicbox}
\mnemonic{High Speed Download Access}
\end{mnemonicbox}

\questionmarks{5(ક)}{7}{બ્લૂટૂથના આર્કિટેક્ચર, સુવિધાઓ અને ફાયદા સમજાવો.}

\begin{solutionbox}
\textbf{બ્લૂટૂથ સ્ટેક:}
\begin{center}
\begin{tikzpicture}[gtu flow]
    \node [gtu block] (app) {એપ્લિકેશન લેયર};
    \node [gtu block, below=0.5cm of app] (l2cap) {L2CAP};
    \node [gtu block, below=0.5cm of l2cap] (hci) {HCI};
    \node [gtu block, below=0.5cm of hci] (lm) {લિંક મેનેજર};
    \node [gtu block, below=0.5cm of lm] (bb) {બેસબેન્ડ};
    \node [gtu block, below=0.5cm of bb] (radio) {રેડિયો લેયર};
    
    \draw [gtu arrow] (app) -- (l2cap);
    \draw [gtu arrow] (l2cap) -- (hci);
    \draw [gtu arrow] (hci) -- (lm);
    \draw [gtu arrow] (lm) -- (bb);
    \draw [gtu arrow] (bb) -- (radio);
\end{tikzpicture}
\captionof{figure}{બ્લૂટૂથ પ્રોટોકોલ સ્ટેક}
\end{center}

\textbf{ફાયદા અને લક્ષણો:}
\begin{itemize}
    \item \keyword{રેન્જ}: 10 મીટર (PAN).
    \item \keyword{ફ્રીક્વન્સી}: 2.4 GHz ISM.
    \item \keyword{નેટવર્ક}: પીકોનેટ (1 માસ્ટર, 7 સ્લેવ).
    \item \keyword{લો પાવર}: બેટરી કાર્યક્ષમતા.
\end{itemize}

\textbf{એપ્લિકેશન્સ}: ઓડિયો, ડેટા ટ્રાન્સફર, પેરિફેરલ્સ.
\end{solutionbox}

\begin{mnemonicbox}
\mnemonic{Blue Personal Area Network}
\end{mnemonicbox}

\questionmarks{5(અ OR)}{3}{RFID ની મૂળભૂત વિભાવના સમજાવો.}

\begin{solutionbox}
\textbf{RFID (Radio Frequency Identification):}
\begin{center}
\captionof{table}{ઘટકો}
\begin{tabulary}{\linewidth}{|L|L|}
\hline
\textbf{ઘટક} & \textbf{કાર્ય} \\ \hline
ટેગ & ઓળખ ડેટા સ્ટોર કરે છે \\ \hline
રીડર & RF દ્વારા ટેગ વાંચે છે \\ \hline
એન્ટેના & કમ્યુનિકેશન \\ \hline
બેકએન્ડ & ડેટા પ્રોસેસિંગ \\ \hline
\end{tabulary}
\end{center}

\begin{itemize}
    \item ઓળખ માટે RF તરંગોનો ઉપયોગ કરે છે.
    \item કોન્ટેક્ટલેસ ઓપરેશન.
\end{itemize}
\end{solutionbox}

\begin{mnemonicbox}
\mnemonic{Radio Frequency Identifies}
\end{mnemonicbox}

\questionmarks{5(બ OR)}{4}{5G સિસ્ટમનું આર્કિટેક્ચર સમજાવો.}

\begin{solutionbox}
\textbf{5G આર્કિટેક્ચર ઘટકો:}
\begin{center}
\captionof{table}{મુખ્ય કાર્યો}
\begin{tabulary}{\linewidth}{|L|L|}
\hline
\textbf{નોડ} & \textbf{કાર્ય} \\ \hline
gNodeB & 5G બેઝ સ્ટેશન \\ \hline
AMF & એક્સેસ અને મોબિલિટી મેનેજમેન્ટ \\ \hline
SMF & સેશન મેનેજમેન્ટ \\ \hline
UPF & યુઝર પ્લેન ફંક્શન \\ \hline
\end{tabulary}
\end{center}

\begin{itemize}
    \item \keyword{સર્વિસ બેઝ્ડ}: મોડ્યુલર નેટવર્ક ફંક્શન્સ.
    \item \keyword{નેટવર્ક સ્લાઇસિંગ}: વર્ચ્યુઅલ નેટવર્ક્સ.
    \item \keyword{એજ કમ્પ્યુટિંગ}: વપરાશકર્તાઓની નજીક પ્રોસેસિંગ.
\end{itemize}
\end{solutionbox}

\begin{mnemonicbox}
\mnemonic{Service Based Network Slicing}
\end{mnemonicbox}

\questionmarks{5(ક OR)}{7}{MANET ને વિગતવાર સમજાવો.}

\begin{solutionbox}
\textbf{MANET (Mobile Ad-hoc Network):}
\begin{itemize}
    \item \keyword{ઇન્ફ્રાસ્ટ્રક્ચર-લેસ}: બેઝ સ્ટેશનોની જરૂર નથી.
    \item \keyword{ડાયનેમિક ટોપોલોજી}: નોડ્સ મુક્તપણે ફરે છે.
    \item \keyword{મલ્ટી-હોપ}: ઇન્ટરમીડિયેટ નોડ્સ દ્વારા મેસેજ રિલે થાય છે.
\end{itemize}

\textbf{ટોપોલોજી આકૃતિ:}
\begin{center}
\begin{tikzpicture}[gtu flow]
    \node [gtu state] (a) {નોડ A};
    \node [gtu state, right=2cm of a] (b) {નોડ B};
    \node [gtu state, below=2cm of a] (c) {નોડ C};
    \node [gtu state, below=2cm of b] (d) {નોડ D};
    
    \draw [gtu arrow, <->] (a) -- (b);
    \draw [gtu arrow, <->] (a) -- (c);
    \draw [gtu arrow, <->] (c) -- (d);
    \draw [gtu arrow, <->] (b) -- (d);
    \draw [gtu arrow, <->] (b) -- (c);
\end{tikzpicture}
\captionof{figure}{મેશ ટોપોલોજી}
\end{center}

\textbf{સરખામણી:}
\begin{center}
\captionof{table}{MANET vs સેલ્યુલર}
\begin{tabulary}{\linewidth}{|L|L|L|}
\hline
\textbf{લક્ષણ} & \textbf{MANET} & \textbf{સેલ્યુલર} \\ \hline
ઇન્ફ્રાસ્ટ્રક્ચર & કોઈ નથી & બેઝ સ્ટેશનો જરૂરી \\ \hline
કિંમત & ઓછી & વધુ \\ \hline
રેન્જ & મલ્ટી-હોપ & સિંગલ હોપ \\ \hline
\end{tabulary}
\end{center}
\end{solutionbox}

\begin{mnemonicbox}
\mnemonic{Mobile Adhoc Network}
\end{mnemonicbox}

\end{document}
