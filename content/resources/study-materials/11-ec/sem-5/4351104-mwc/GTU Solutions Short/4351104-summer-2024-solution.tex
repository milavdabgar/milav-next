\documentclass{article}
% Adjust the relative path to point to the latex-templates directory

% content/resources/templates/preamble.tex
\usepackage[margin=0.6in]{geometry}
\author{Milav Dabgar}
\usepackage{amsmath,amssymb,amsthm}
\usepackage{booktabs}
\usepackage{multirow}
\usepackage{xcolor}
\usepackage{tcolorbox}
\tcbuselibrary{breakable,skins}
\usepackage[colorlinks=true,linkcolor=blue]{hyperref}
\usepackage{titlesec}
\usepackage{enumitem}
\usepackage{tikz}
\usepackage{pgfplots}
\usepackage{circuitikz}
\usepackage[version=4]{mhchem}
\usepackage{longtable}
\usepackage{array}
\usepackage{float}
\usepackage{caption}
\usepackage{listings}

\lstset{
  basicstyle=\small\ttfamily,
  breaklines=true,
  breakatwhitespace=false,
  postbreak=\mbox{\textcolor{red}{$\hookrightarrow$}\space},
  float=false,
  numbers=left,
  numberstyle=\tiny\color{gray},
  numbersep=10pt,
  xleftmargin=2em,
  keywordstyle=\color{blue},
  commentstyle=\color{green!60!black},
  stringstyle=\color{purple},
  backgroundcolor=\color{gray!5},
  showstringspaces=false,
  tabsize=2,
  captionpos=b,
  keepspaces=true,
  columns=flexible
}

\pgfplotsset{compat=1.18}
\usetikzlibrary{shapes,arrows,positioning,calc,patterns,decorations.pathmorphing,decorations.markings,arrows.meta}

% Color scheme
\definecolor{headcolor}{RGB}{0,102,204}
\definecolor{keycolor}{RGB}{220,20,60}
\definecolor{solutioncolor}{RGB}{34,139,34}
\definecolor{mnemoniccolor}{RGB}{148,0,211}
\definecolor{codecolor}{RGB}{0,0,100}

% Spacing
\setlength{\parskip}{3pt}
\setlist[itemize]{nosep}
\setlist[enumerate]{nosep}

% Title formatting
\titleformat{\section}{\Large\bfseries\color{headcolor}}{\thesection}{1em}{}
\titleformat{\subsection}{\large\bfseries\color{headcolor}}{\thesubsection}{1em}{}

% Pandoc tightlist compatibility
\providecommand{\tightlist}{%
  \setlength{\itemsep}{0pt}\setlength{\parskip}{0pt}}

% Pandoc longtable compatibility
\newcounter{none}
\def\thenone{}


% content/resources/templates/english-boxes.tex
% This file is currently empty - it exists to maintain consistency with the import structure.
% Add custom environments here if needed in the future.


% Custom commands for GTU solutions
% This file defines semantic commands for consistent formatting

% Question command with automatic formatting
\newcommand{\question}[2]{%
  \section*{Question #1}%
  \textbf{#2}%
}

% OR question variant
\newcommand{\questionor}[2]{%
  \section*{Question #1 OR}%
  \textbf{#2}%
}

% Proper table environment with caption
\newenvironment{answertable}[1]{%
  \begin{table}[htbp]
  \centering
  \caption{#1}
}{%
  \end{table}
}

% Proper figure environment for diagrams
\newenvironment{answerdiagram}[1]{%
  \begin{figure}[htbp]
  \centering
  \caption{#1}
}{%
  \end{figure}
}

% Semantic markup for key terms
\newcommand{\keyword}[1]{\textbf{#1}}
\newcommand{\code}[1]{\texttt{#1}}
\newcommand{\classname}[1]{\texttt{#1}}
\newcommand{\methodname}[1]{\texttt{#1}}

% Proper quotation marks
\newcommand{\mnemonic}[1]{``#1''}


\title{Mobile \& Wireless Communication (4351104) - Summer 2024 Solution}
\date{May 23, 2024}

\begin{document}
\maketitle

\questionmarks{1(a)}{3}{Explain selective cell.}

\begin{solutionbox}
\textbf{Selective Cell Characteristics:}
\begin{center}
\captionof{table}{Selective Cell Features}
\begin{tabulary}{\linewidth}{|L|L|}
\hline
\textbf{Feature} & \textbf{Description} \\ \hline
Purpose & Provides coverage for specific areas \\ \hline
Size & Small coverage area \\ \hline
Application & Indoor locations, tunnels, buildings \\ \hline
Antenna & Directional antenna system \\ \hline
\end{tabulary}
\end{center}

\begin{itemize}
    \item \keyword{Selective coverage}: Targets specific geographical areas needing signal.
    \item \keyword{Indoor solution}: Primarily used for building coverage enhancement.
    \item \keyword{Directional transmission}: Uses focused beam patterns for efficiency.
\end{itemize}
\end{solutionbox}

\begin{mnemonicbox}
\mnemonic{Select Special Spots}
\end{mnemonicbox}

\questionmarks{1(b)}{4}{Draw and explain umbrella cell.}

\begin{solutionbox}
\textbf{Umbrella Cell Structure:}
\begin{center}
\begin{tikzpicture}[gtu flow]
    \node [gtu block, minimum width=6cm] (umbrella) {Umbrella Cell (Macro)};
    
    \node [gtu block, below left=1cm of umbrella, xshift=1cm] (micro) {Micro Cell};
    \node [gtu block, below right=1cm of umbrella, xshift=-1cm] (pico) {Pico Cell};
    
    \draw [gtu arrow] (umbrella) -- (micro);
    \draw [gtu arrow] (umbrella) -- (pico);
\end{tikzpicture}
\captionof{figure}{Umbrella Cell Concept}
\end{center}

\textbf{Features:}
\begin{center}
\captionof{table}{Umbrella Cell Features}
\begin{tabulary}{\linewidth}{|L|L|}
\hline
\textbf{Parameter} & \textbf{Description} \\ \hline
Coverage & Large area coverage \\ \hline
Purpose & Overlays smaller cells \\ \hline
Handoff & Manages inter-cell transitions \\ \hline
Capacity & Handles overflow traffic \\ \hline
\end{tabulary}
\end{center}

\begin{itemize}
    \item \keyword{Large coverage}: Provides wide area signal coverage over smaller cells.
    \item \keyword{Traffic management}: Handles overflow from micro and pico cells.
    \item \keyword{Seamless handoff}: Ensures continuous communication during movement.
\end{itemize}
\end{solutionbox}

\begin{mnemonicbox}
\mnemonic{Umbrella Covers All}
\end{mnemonicbox}

\questionmarks{1(c)}{7}{What is the cell? Explain frequency reuse.}

\begin{solutionbox}
\textbf{Cell \& Frequency Reuse:}
\begin{center}
\captionof{table}{Concepts}
\begin{tabulary}{\linewidth}{|L|L|L|}
\hline
\textbf{Concept} & \textbf{Definition} & \textbf{Purpose} \\ \hline
Cell & Geographic coverage area & Service provision \\ \hline
Frequency Reuse & Same frequency in diff cells & Spectrum efficiency \\ \hline
Cluster & Group with unique freqs & Interference control \\ \hline
Reuse Distance & Min distance same freq & Signal quality \\ \hline
\end{tabulary}
\end{center}

\textbf{Concept Map:}
\begin{center}
\begin{tikzpicture}[gtu flow]
    \node [gtu block] (cell) {Cell Concept\\(Hexagonal)};
    \node [gtu block, right=1cm of cell] (bs) {Base Station\\Coverage};
    
    \node [gtu block, below=1cm of cell] (reuse) {Frequency Reuse};
    \node [gtu block, below left=1cm of reuse] (cluster) {Cluster Pattern\\(N=4,7,12)};
    \node [gtu block, below right=1cm of reuse] (cochannel) {Co-channel\\Reuse};
    
    \draw [gtu arrow] (cell) -- (bs);
    \draw [gtu arrow] (cell) -- (reuse);
    \draw [gtu arrow] (reuse) -- (cluster);
    \draw [gtu arrow] (reuse) -- (cochannel);
\end{tikzpicture}
\captionof{figure}{Frequency Reuse Structure}
\end{center}

\begin{itemize}
    \item \keyword{Cell definition}: Geographical area covered by one base station.
    \item \keyword{Hexagonal pattern}: Most efficient shape for coverage.
    \item \keyword{Frequency reuse}: Reusing same frequencies in non-adjacent cells.
    \item \keyword{Cluster size}: Determines reuse pattern (N).
\end{itemize}
\end{solutionbox}

\begin{mnemonicbox}
\mnemonic{Cells Reuse Frequencies Efficiently}
\end{mnemonicbox}

\questionmarks{1(c OR)}{7}{Explain cellular concept in detail.}

\begin{solutionbox}
\textbf{Cellular System Components:}
\begin{center}
\captionof{table}{Components}
\begin{tabulary}{\linewidth}{|L|L|L|}
\hline
\textbf{Component} & \textbf{Function} & \textbf{Benefit} \\ \hline
Cell Division & Split area & Coverage optimization \\ \hline
Base Stations & Serve cells & Signal transmission \\ \hline
MSC & Routing & Network connectivity \\ \hline
Freq Planning & Allocation & Interference control \\ \hline
\end{tabulary}
\end{center}

\textbf{Cellular Concept Flow:}
\begin{center}
\begin{tikzpicture}[gtu flow]
    \node [gtu start] (large) {Large Area};
    \node [gtu process, below=0.5cm of large] (div) {Cell Division};
    \node [gtu block, below=0.5cm of div] (bs) {Multiple Base Stations};
    \node [gtu decision, below=0.5cm of bs] (reuse) {Frequency Reuse};
    \node [gtu stop, below=0.5cm of reuse] (cap) {High Capacity};
    
    \draw [gtu arrow] (large) -- (div);
    \draw [gtu arrow] (div) -- (bs);
    \draw [gtu arrow] (bs) -- (reuse);
    \draw [gtu arrow] (reuse) -- (cap);
\end{tikzpicture}
\captionof{figure}{Cellular Concept}
\end{center}

\begin{itemize}
    \item \keyword{Area division}: Large area divided into small hexagonal cells.
    \item \keyword{Power control}: Low power transmitters reduce interference.
    \item \keyword{Frequency efficiency}: Frequencies reused in distant cells.
    \item \keyword{Capacity}: Supports more simultaneous users.
\end{itemize}
\end{solutionbox}

\begin{mnemonicbox}
\mnemonic{Divide Area For Better Service}
\end{mnemonicbox}

\questionmarks{2(a)}{3}{Define full forms: (i) IMEI (ii) LTE (iii) GSM}

\begin{solutionbox}
\begin{center}
\captionof{table}{Full Forms}
\begin{tabulary}{\linewidth}{|L|L|L|}
\hline
\textbf{Abbr} & \textbf{Full Form} & \textbf{Purpose} \\ \hline
\textbf{IMEI} & International Mobile Equipment Identity & Device ID \\ \hline
\textbf{LTE} & Long Term Evolution & 4G Standard \\ \hline
\textbf{GSM} & Global System for Mobile Communication & 2G Standard \\ \hline
\end{tabulary}
\end{center}
\end{solutionbox}

\begin{mnemonicbox}
\mnemonic{Identity Long-term Global}
\end{mnemonicbox}

\questionmarks{2(b)}{4}{Explain MAHO in detail.}

\begin{solutionbox}
\textbf{MAHO (Mobile Assisted Handoff):}
\begin{center}
\captionof{table}{Characteristics}
\begin{tabulary}{\linewidth}{|L|L|}
\hline
\textbf{Feature} & \textbf{Description} \\ \hline
Function & Mobile helps in handoff decision \\ \hline
Measurement & Signal strength monitoring \\ \hline
Reporting & Mobile reports to network \\ \hline
\end{tabulary}
\end{center}

\textbf{Process Flow:}
\begin{center}
\begin{tikzpicture}[gtu flow]
    \node [gtu start] (mob) {Mobile Unit};
    \node [gtu process, right=1cm of mob, align=center] (meas) {Measure Signal\\Strength};
    \node [gtu process, right=1cm of meas, align=center] (rep) {Report to\\Base Station};
    \node [gtu decision, below=1cm of rep] (net) {Network Decision};
    \node [gtu stop, left=1cm of net] (ho) {Execute Handoff};
    
    \draw [gtu arrow] (mob) -- (meas);
    \draw [gtu arrow] (meas) -- (rep);
    \draw [gtu arrow] (rep) -- (net);
    \draw [gtu arrow] (net) -- (ho);
\end{tikzpicture}
\captionof{figure}{MAHO Process}
\end{center}

\begin{itemize}
    \item Mobile measures neighboring cell signals.
    \item Sends reports to the network.
    \item Network uses this data for better handoff decisions.
\end{itemize}
\end{solutionbox}

\begin{mnemonicbox}
\mnemonic{Mobile Assists Network Decisions}
\end{mnemonicbox}

\questionmarks{2(c)}{7}{Explain GSM architecture with diagram}

\begin{solutionbox}
\textbf{GSM Architecture:}
\begin{center}
\begin{tikzpicture}[gtu flow]
    % Nodes
    \node [gtu block] (ms) {Mobile Station\\(MS)};
    \node [gtu block, right=1cm of ms] (bts) {Base Transceiver\\Station (BTS)};
    \node [gtu block, right=1cm of bts] (bsc) {Base Station\\Controller (BSC)};
    \node [gtu block, above=1cm of bsc] (msc) {Mobile Switching\\Center (MSC)};
    
    \node [gtu block, right=1cm of msc] (hlr) {HLR / VLR};
    \node [gtu block, left=1cm of msc] (auc) {AuC / EIR};
    \node [gtu block, above=1cm of msc] (pstn) {PSTN / ISDN};
    
    % Connections
    \draw [gtu arrow] (ms) -- (bts);
    \draw [gtu arrow] (bts) -- (bsc);
    \draw [gtu arrow] (bsc) -- (msc);
    \draw [gtu arrow] (msc) -- (hlr);
    \draw [gtu arrow] (msc) -- (auc);
    \draw [gtu arrow] (msc) -- (pstn);
\end{tikzpicture}
\captionof{figure}{GSM Network Architecture}
\end{center}

\begin{center}
\captionof{table}{Components}
\begin{tabulary}{\linewidth}{|L|L|L|}
\hline
\textbf{Component} & \textbf{Full Form} & \textbf{Function} \\ \hline
MS & Mobile Station & User equipment \\ \hline
BTS & Base Transceiver & Radio interface \\ \hline
BSC & Base Controller & Resource mgmt \\ \hline
MSC & Mobile Switching & Call switching \\ \hline
HLR & Home Location & Perm. DB \\ \hline
VLR & Visitor Location & Temp. DB \\ \hline
\end{tabulary}
\end{center}

\begin{itemize}
    \item \keyword{Radio Subsystem}: BTS handles radio, BSC manages resources.
    \item \keyword{Network Subsystem}: MSC switches calls, HLR/VLR manage data.
    \item \keyword{Authentication}: AuC handles security.
\end{itemize}
\end{solutionbox}

\begin{mnemonicbox}
\mnemonic{Mobile Base Network Database}
\end{mnemonicbox}

\questionmarks{2(a OR)}{3}{Explain cell splitting.}

\begin{solutionbox}
\textbf{Cell Splitting Process:}
\begin{center}
\captionof{table}{Process Steps}
\begin{tabulary}{\linewidth}{|L|L|L|}
\hline
\textbf{Step} & \textbf{Action} & \textbf{Result} \\ \hline
1 & Reduce Power & Smaller coverage \\ \hline
2 & Add Base Stations & Fill gaps \\ \hline
3 & Freq Planning & Control interference \\ \hline
4 & Incr Capacity & More users \\ \hline
\end{tabulary}
\end{center}

\begin{itemize}
    \item \keyword{Power reduction}: Shrinks original cell coverage.
    \item \keyword{New cells}: Added in gaps to maintain coverage.
    \item \keyword{Capacity gain}: Higher user density handling.
\end{itemize}
\end{solutionbox}

\begin{mnemonicbox}
\mnemonic{Split Cells Double Capacity}
\end{mnemonicbox}

\questionmarks{2(b OR)}{4}{What is handoff? Explain soft and hard handoffs.}

\begin{solutionbox}
\textbf{Handoff Types:}
\begin{center}
\captionof{table}{Hard vs Soft Handoff}
\begin{tabulary}{\linewidth}{|L|L|L|}
\hline
\textbf{Type} & \textbf{Process} & \textbf{Tech} \\ \hline
Hard & \keyword{Break-then-make} & GSM, TDMA \\ \hline
Soft & \keyword{Make-then-break} & CDMA \\ \hline
\end{tabulary}
\end{center}

\textbf{Decision Flow:}
\begin{center}
\begin{tikzpicture}[gtu flow]
    \node [gtu start] (move) {Mobile Moves};
    \node [gtu decision, right=1cm of move] (type) {Type?};
    
    \node [gtu process, above right=1cm of type] (hard) {Hard: Disconnect $\to$ Connect};
    \node [gtu process, below right=1cm of type] (soft) {Soft: Connect $\to$ Disconnect};
    
    \draw [gtu arrow] (move) -- (type);
    \draw [gtu arrow] (type) |- (hard);
    \draw [gtu arrow] (type) |- (soft);
\end{tikzpicture}
\captionof{figure}{Handoff Mechanism}
\end{center}
\end{solutionbox}

\begin{mnemonicbox}
\mnemonic{Hard Breaks Soft Connects}
\end{mnemonicbox}

\questionmarks{2(c OR)}{7}{Explain GSM signal processing with diagram}

\begin{solutionbox}
\textbf{GSM Signal Chain:}
\begin{center}
\begin{tikzpicture}[gtu flow]
    \node [gtu start] (voice) {Voice Input};
    \node [gtu process, below=0.5cm of voice] (codec) {Speech Codec};
    \node [gtu process, below=0.5cm of codec] (channel) {Channel Coding};
    \node [gtu process, below=0.5cm of channel] (inter) {Interleaving};
    \node [gtu process, right=1cm of inter] (encrypt) {Encryption};
    \node [gtu process, above=0.5cm of encrypt] (burst) {Burst Format};
    \node [gtu process, above=0.5cm of burst] (mod) {Modulation};
    \node [gtu stop, above=0.5cm of mod] (rf) {RF Tx};
    
    \draw [gtu arrow] (voice) -- (codec);
    \draw [gtu arrow] (codec) -- (channel);
    \draw [gtu arrow] (channel) -- (inter);
    \draw [gtu arrow] (inter) -- (encrypt);
    \draw [gtu arrow] (encrypt) -- (burst);
    \draw [gtu arrow] (burst) -- (mod);
    \draw [gtu arrow] (mod) -- (rf);
\end{tikzpicture}
\captionof{figure}{Signal Processing Steps}
\end{center}

\textbf{Processing Stages:}
\begin{itemize}
    \item \keyword{Speech Codec}: Compresses voice (RPE-LTP).
    \item \keyword{Channel Coding}: Adds error protection.
    \item \keyword{Interleaving}: Protects against burst errors.
    \item \keyword{Encryption}: Secures data (A5 algorithm).
    \item \keyword{Modulation}: GMSK for transmission.
\end{itemize}
\end{solutionbox}

\begin{mnemonicbox}
\mnemonic{Voice Coded Interleaved Encrypted Modulated}
\end{mnemonicbox}

\questionmarks{3(a)}{3}{Explain cell sectoring.}

\begin{solutionbox}
\textbf{Cell Sectoring Benefits:}
\begin{center}
\captionof{table}{Sectoring}
\begin{tabulary}{\linewidth}{|L|L|}
\hline
\textbf{Feature} & \textbf{Description} \\ \hline
Antenna & Directional (not omnidirectional) \\ \hline
Sectors & 3 or 6 per cell \\ \hline
Capacity & Increases (3x or 6x) \\ \hline
Interference & Reduced co-channel interference \\ \hline
\end{tabulary}
\end{center}

\begin{itemize}
    \item Uses directional antennas to divide cell.
    \item Each sector acts like a new cell, increasing capacity.
    \item Reduces interference by focusing energy.
\end{itemize}
\end{solutionbox}

\begin{mnemonicbox}
\mnemonic{Sector Antennas Triple Capacity}
\end{mnemonicbox}

\questionmarks{3(b)}{4}{Explain GSM call procedure.}

\begin{solutionbox}
\textbf{Call Sequence:}
\begin{center}
\begin{tikzpicture}[gtu flow]
    \node [gtu start] (req) {Call Request (Mobile)};
    \node [gtu process, below=0.5cm of req] (bts) {BTS $\to$ BSC};
    \node [gtu process, below=0.5cm of bts] (msc) {BSC $\to$ MSC};
    \node [gtu decision, below=0.5cm of msc] (auth) {Authenticate (HLR)};
    \node [gtu process, below=0.5cm of auth] (conn) {Connect (PSTN)};
    
    \draw [gtu arrow] (req) -- (bts);
    \draw [gtu arrow] (bts) -- (msc);
    \draw [gtu arrow] (msc) -- (auth);
    \draw [gtu arrow] (auth) -- (conn);
\end{tikzpicture}
\captionof{figure}{Call Setup Flow}
\end{center}

\textbf{Steps:}
1. \keyword{Authentication}: Verify user.
2. \keyword{Allocation}: Assign channel.
3. \keyword{Routing}: Determine path.
4. \keyword{Connection}: Establish link.
\end{solutionbox}

\begin{mnemonicbox}
\mnemonic{Authenticate Allocate Route Connect}
\end{mnemonicbox}

\questionmarks{3(c)}{7}{Explain GPRS.}

\begin{solutionbox}
\textbf{GPRS Features:}
\begin{itemize}
    \item \keyword{Packet Switched}: Info sent in packets.
    \item \keyword{Always On}: Constant connection.
    \item \keyword{Speed}: Up to 114 kbps.
\end{itemize}

\textbf{GPRS Architecture:}
\begin{center}
\begin{tikzpicture}[gtu flow]
    \node [gtu block] (net) {GPRS Network};
    \node [gtu block, below left=1cm of net] (sgsn) {SGSN};
    \node [gtu block, below right=1cm of net] (ggsn) {GGSN};
    
    \node [gtu block, below=1cm of sgsn] (pkt) {Packet Data};
    \node [gtu block, below=1cm of ggsn] (net2) {Internet/Ext};
    
    \draw [gtu arrow] (net) -- (sgsn);
    \draw [gtu arrow] (net) -- (ggsn);
    \draw [gtu arrow] (sgsn) -- (pkt);
    \draw [gtu arrow] (ggsn) -- (net2);
\end{tikzpicture}
\captionof{figure}{GPRS Nodes}
\end{center}

\begin{itemize}
    \item \textbf{SGSN}: Service GPRS Support Node (Mobility).
    \item \textbf{GGSN}: Gateway GPRS Support Node (External).
    \item Enables internet and email services.
\end{itemize}
\end{solutionbox}

\begin{mnemonicbox}
\mnemonic{General Packet Radio Service}
\end{mnemonicbox}

\questionmarks{3(a OR)}{3}{Explain advantage of CDMA}

\begin{solutionbox}
\textbf{Advantages:}
\begin{center}
\captionof{table}{CDMA Benefits}
\begin{tabulary}{\linewidth}{|L|L|}
\hline
\textbf{Advantage} & \textbf{Description} \\ \hline
Capacity & Higher user capacity \\ \hline
Security & Built-in encryption (Spread Spectrum) \\ \hline
Quality & Better voice quality (Soft Handoff) \\ \hline
Power & Efficient power control \\ \hline
\end{tabulary}
\end{center}
\end{solutionbox}

\begin{mnemonicbox}
\mnemonic{Capacity Security Quality}
\end{mnemonicbox}

\questionmarks{3(b OR)}{4}{Explain frequency hopping techniques.}

\begin{solutionbox}
\textbf{Frequency Hopping (FH):}
\begin{center}
\captionof{table}{FH Types}
\begin{tabulary}{\linewidth}{|L|L|}
\hline
\textbf{Type} & \textbf{Rate vs Symbol Rate} \\ \hline
Slow FH & Rate $<$ Symbol Rate (GSM) \\ \hline
Fast FH & Rate $>$ Symbol Rate (Military) \\ \hline
\end{tabulary}
\end{center}

\textbf{Mechanism:}
\begin{center}
\begin{tikzpicture}[gtu flow]
    \node [gtu start] (data) {Data};
    \node [gtu process, right=0.5cm of data] (spread) {Spread Spectrum};
    \node [gtu process, right=0.5cm of spread] (synth) {Freq Synth};
    \node [gtu process, below=0.5cm of synth] (patt) {Hop Pattern};
    \node [gtu stop, right=0.5cm of synth] (rf) {RF Tx};
    
    \draw [gtu arrow] (data) -- (spread);
    \draw [gtu arrow] (spread) -- (synth);
    \draw [gtu arrow] (patt) -- (synth);
    \draw [gtu arrow] (synth) -- (rf);
\end{tikzpicture}
\captionof{figure}{Frequency Hopping}
\end{center}

\begin{itemize}
    \item Reduces interference and jamming.
    \item Increases security.
\end{itemize}
\end{solutionbox}

\begin{mnemonicbox}
\mnemonic{Frequency Hops For Security}
\end{mnemonicbox}

\questionmarks{3(c OR)}{7}{Explain EDGE.}

\begin{solutionbox}
\textbf{EDGE (Enhanced Data rates for GSM Evolution):}
\begin{center}
\captionof{table}{EDGE Specs}
\begin{tabulary}{\linewidth}{|L|L|}
\hline
\textbf{Parameter} & \textbf{Value/Benefit} \\ \hline
Data Rate & Up to 384 kbps (3x GPRS) \\ \hline
Modulation & 8-PSK (Higher order) \\ \hline
Compatibility & Backward compatible with GSM \\ \hline
\end{tabulary}
\end{center}

\textbf{Enhancement Flow:}
\begin{center}
\begin{tikzpicture}[gtu flow]
    \node [gtu block] (edge) {EDGE};
    \node [gtu block, below left=1cm of edge] (mod) {8-PSK};
    \node [gtu block, below=1cm of edge] (link) {Link Adapt};
    \node [gtu block, below right=1cm of edge] (red) {Inc. Redundancy};
    
    \draw [gtu arrow] (edge) -- (mod);
    \draw [gtu arrow] (edge) -- (link);
    \draw [gtu arrow] (edge) -- (red);
\end{tikzpicture}
\captionof{figure}{EDGE Improvements}
\end{center}

\begin{itemize}
    \item \keyword{8-PSK}: Increases throughput via higher order modulation.
    \item \keyword{Link Adaptation}: Adjusts to channel quality.
    \item \keyword{Bridge to 3G}: Step between 2G and 3G.
\end{itemize}
\end{solutionbox}

\begin{mnemonicbox}
\mnemonic{Enhanced Data Gets Excellence}
\end{mnemonicbox}

\questionmarks{4(a)}{3}{Draw FHSS transmitter block diagram}

\begin{solutionbox}
\textbf{FHSS Transmitter:}
\begin{center}
\begin{tikzpicture}[gtu flow]
    \node [gtu start] (input) {Data Input};
    \node [gtu process, right=1cm of input] (mod) {Modulator};
    \node [gtu process, right=1cm of mod] (synth) {Freq\\Synthesizer};
    \node [gtu process, below=1cm of synth] (pn) {PN Seq\\Generator};
    \node [gtu process, right=1cm of synth] (amp) {RF Amp};
    \node [gtu stop, right=1cm of amp] (ant) {Antenna};
    
    \draw [gtu arrow] (input) -- (mod);
    \draw [gtu arrow] (mod) -- (synth);
    \draw [gtu arrow] (pn) -- (synth);
    \draw [gtu arrow] (synth) -- (amp);
    \draw [gtu arrow] (amp) -- (ant);
\end{tikzpicture}
\captionof{figure}{FHSS Transmitter Block Diagram}
\end{center}
\end{solutionbox}

\begin{mnemonicbox}
\mnemonic{Data Modulated Frequency Hops}
\end{mnemonicbox}

\questionmarks{4(b)}{4}{Explain call processing in CDMA}

\begin{solutionbox}
\textbf{Processing Phases:}
\begin{center}
\captionof{table}{Call Stages}
\begin{tabulary}{\linewidth}{|L|L|L|}
\hline
\textbf{Phase} & \textbf{Process} & \textbf{Purpose} \\ \hline
Access & Sync & Initial connection \\ \hline
Authentication & Verify & Security \\ \hline
Traffic & Comm & Data transfer \\ \hline
Release & Terminate & Cleanup \\ \hline
\end{tabulary}
\end{center}

\begin{itemize}
    \item \keyword{Access}: Mobile acquires pilot, syncs.
    \item \keyword{Auth}: Credentials verified.
    \item \keyword{Traffic}: Active call with power control.
    \item \keyword{Release}: Free resources.
\end{itemize}
\end{solutionbox}

\begin{mnemonicbox}
\mnemonic{Access Authenticate Transfer Release}
\end{mnemonicbox}

\questionmarks{4(c)}{7}{Draw OFDM receiver and explain its working}

\begin{solutionbox}
\textbf{OFDM Receiver Diagram:}
\begin{center}
\begin{tikzpicture}[gtu flow]
    \node [gtu start] (rf) {RF In};
    \node [gtu process, right=0.5cm of rf] (down) {Down\\Convert};
    \node [gtu process, right=0.5cm of down] (adc) {ADC};
    \node [gtu process, right=0.5cm of adc] (cp) {Remove\\CP};
    \node [gtu process, below=1cm of cp] (fft) {FFT};
    \node [gtu process, left=0.5cm of fft] (ps) {Par $\to$ Ser};
    \node [gtu process, left=0.5cm of ps] (dec) {Decode};
    \node [gtu stop, left=0.5cm of dec] (out) {Data Out};
    
    \draw [gtu arrow] (rf) -- (down);
    \draw [gtu arrow] (down) -- (adc);
    \draw [gtu arrow] (adc) -- (cp);
    \draw [gtu arrow] (cp) -- (fft);
    \draw [gtu arrow] (fft) -- (ps);
    \draw [gtu arrow] (ps) -- (dec);
    \draw [gtu arrow] (dec) -- (out);
\end{tikzpicture}
\captionof{figure}{OFDM Receiver}
\end{center}

\textbf{Functions:}
\begin{itemize}
    \item \keyword{Down Converter}: RF to baseband.
    \item \keyword{Remove CP}: Eliminates Inter-Symbol Interference (ISI).
    \item \keyword{FFT}: Separates orthogonal subcarriers.
    \item \keyword{Decoder}: Corrects errors and recovers data.
\end{itemize}
\end{solutionbox}

\begin{mnemonicbox}
\mnemonic{Receive Convert Remove Transform Decode}
\end{mnemonicbox}

\questionmarks{4(a OR)}{3}{Explain radiation hazard due to mobile.}

\begin{solutionbox}
\textbf{Radiation Effects:}
\begin{center}
\captionof{table}{Hazards}
\begin{tabulary}{\linewidth}{|L|L|}
\hline
\textbf{Parameter} & \textbf{Effect/Details} \\ \hline
SAR & Specific Absorption Rate (Heating) \\ \hline
Frequency & 900/1800 MHz (Penetration) \\ \hline
Safety & Regulated limits for exposure \\ \hline
\end{tabulary}
\end{center}

\begin{itemize}
    \item \keyword{SAR}: Measures energy absorbed by body tissue.
    \item \keyword{Thermal}: High SAR causes tissue heating.
\end{itemize}
\end{solutionbox}

\begin{mnemonicbox}
\mnemonic{SAR Safety Absorption Rate}
\end{mnemonicbox}

\questionmarks{4(b OR)}{4}{Explain Li-Po type batteries used in mobile handset.}

\begin{solutionbox}
\textbf{Li-Po Characteristics:}
\begin{center}
\captionof{table}{Li-Po Features}
\begin{tabulary}{\linewidth}{|L|L|}
\hline
\textbf{Feature} & \textbf{Benefit} \\ \hline
Chemistry & Lithium Polymer (Solid/Gel) \\ \hline
Shape & Flexible, thin form factor \\ \hline
Density & High energy density \\ \hline
Weight & Lightweight \\ \hline
\end{tabulary}
\end{center}

\begin{itemize}
    \item Uses polymer electrolyte, allowing flexible shapes.
    \item Safe and lightweight compared to liquid electrolytes.
    \item Supports fast charging.
\end{itemize}
\end{solutionbox}

\begin{mnemonicbox}
\mnemonic{Lithium Polymer Power}
\end{mnemonicbox}

\questionmarks{4(c OR)}{7}{Explain mobile handset block diagram.}

\begin{solutionbox}
\textbf{Handset Diagram:}
\begin{center}
\begin{tikzpicture}[gtu flow]
    % Central Unit
    \node [gtu block, minimum width=3cm] (base) {Baseband\\Processor};
    
    % Peripherals
    \node [gtu block, above=1cm of base] (rf) {RF Section};
    \node [gtu block, above=0.5cm of rf] (ant) {Antenna};
    
    \node [gtu block, left=1cm of base] (aud) {Audio\\Codec};
    \node [gtu block, right=1cm of base] (ui) {Display/\\Keypad};
    
    \node [gtu block, below=1cm of base] (pwr) {Power Mgmt};
    \node [gtu block, below=0.5cm of pwr] (bat) {Battery};
    
    \node [gtu block, below left=1cm of base] (sim) {SIM};
    
    % Connections
    \draw [gtu arrow] (ant) -- (rf);
    \draw [gtu arrow] (rf) -- (base);
    \draw [gtu arrow] (base) -- (rf);
    
    \draw [gtu arrow] (base) -- (aud);
    \draw [gtu arrow] (base) -- (ui);
    \draw [gtu arrow] (sim) -| (base);
    \draw [gtu arrow] (pwr) -- (base);
    \draw [gtu arrow] (pwr) -- (rf);
    \draw [gtu arrow] (bat) -- (pwr);
\end{tikzpicture}
\captionof{figure}{Mobile Handset Blocks}
\end{center}

\textbf{Components:}
\begin{itemize}
    \item \keyword{RF Section}: Radio transmission/reception.
    \item \keyword{Baseband}: Protocols and processing.
    \item \keyword{Audio}: Voice processing.
    \item \keyword{Power}: Manages battery and charging.
    \item \keyword{SIM}: User identity.
\end{itemize}
\end{solutionbox}

\begin{mnemonicbox}
\mnemonic{Radio Baseband Audio Power Interface}
\end{mnemonicbox}

\questionmarks{5(a)}{3}{Compare CDMA and GSM}

\begin{solutionbox}
\begin{center}
\captionof{table}{CDMA vs GSM}
\begin{tabulary}{\linewidth}{|L|L|L|}
\hline
\textbf{Feature} & \textbf{CDMA} & \textbf{GSM} \\ \hline
Access & Code Division & Time Division \\ \hline
Capacity & Higher & Lower \\ \hline
Handoff & Soft & Hard \\ \hline
SIM & Not always req & Required \\ \hline
\end{tabulary}
\end{center}
\end{solutionbox}

\begin{mnemonicbox}
\mnemonic{Code vs Time Division}
\end{mnemonicbox}

\questionmarks{5(b)}{4}{Explain HSDPA.}

\begin{solutionbox}
\textbf{HSDPA (High Speed Downlink Packet Access):}
\begin{center}
\captionof{table}{Features}
\begin{tabulary}{\linewidth}{|L|L|}
\hline
\textbf{Feature} & \textbf{Description} \\ \hline
Data Rate & Up to 14.4 Mbps \\ \hline
Tech & 3.5G (UMTS Enhancement) \\ \hline
Direction & Optimized for Downlink \\ \hline
Modulation & Adaptive (QPSK $\to$ 16-QAM) \\ \hline
\end{tabulary}
\end{center}

\begin{itemize}
    \item Improves 3G downlink speeds significantly.
    \item Faster scheduling (2ms) for responsiveness.
\end{itemize}
\end{solutionbox}

\begin{mnemonicbox}
\mnemonic{High Speed Download Access}
\end{mnemonicbox}

\questionmarks{5(c)}{7}{Explain architecture, features and advantage of Bluetooth.}

\begin{solutionbox}
\textbf{Bluetooth Stack:}
\begin{center}
\begin{tikzpicture}[gtu flow]
    \node [gtu block] (app) {Application Layer};
    \node [gtu block, below=0.5cm of app] (l2cap) {L2CAP};
    \node [gtu block, below=0.5cm of l2cap] (hci) {HCI};
    \node [gtu block, below=0.5cm of hci] (lm) {Link Manager};
    \node [gtu block, below=0.5cm of lm] (bb) {Baseband};
    \node [gtu block, below=0.5cm of bb] (radio) {Radio Layer};
    
    \draw [gtu arrow] (app) -- (l2cap);
    \draw [gtu arrow] (l2cap) -- (hci);
    \draw [gtu arrow] (hci) -- (lm);
    \draw [gtu arrow] (lm) -- (bb);
    \draw [gtu arrow] (bb) -- (radio);
\end{tikzpicture}
\captionof{figure}{Bluetooth Protocol Stack}
\end{center}

\textbf{Features \& Advantages:}
\begin{itemize}
    \item \keyword{Range}: 10m (PAN).
    \item \keyword{Freq}: 2.4 GHz ISM (Unlicensed).
    \item \keyword{Net}: Piconet (1 Master, 7 Slaves).
    \item \keyword{Low Power}: Battery efficient.
\end{itemize}

\textbf{Applications}: Audio, Data transfer, Peripherals.
\end{solutionbox}

\begin{mnemonicbox}
\mnemonic{Blue Personal Area Network}
\end{mnemonicbox}

\questionmarks{5(a OR)}{3}{Explain basic concept of RFID.}

\begin{solutionbox}
\textbf{RFID (Radio Frequency Identification):}
\begin{center}
\captionof{table}{Components}
\begin{tabulary}{\linewidth}{|L|L|}
\hline
\textbf{Component} & \textbf{Function} \\ \hline
Tag & Stores ID data \\ \hline
Reader & Reads tag via RF \\ \hline
Antenna & Communication \\ \hline
Backend & Processing \\ \hline
\end{tabulary}
\end{center}

\begin{itemize}
    \item Contactless identification using RF waves.
    \item Automatic reading within range.
\end{itemize}
\end{solutionbox}

\begin{mnemonicbox}
\mnemonic{Radio Frequency Identifies}
\end{mnemonicbox}

\questionmarks{5(b OR)}{4}{Explain architecture of 5G system.}

\begin{solutionbox}
\textbf{5G Architecture Components:}
\begin{center}
\captionof{table}{Key Functions}
\begin{tabulary}{\linewidth}{|L|L|}
\hline
\textbf{Node} & \textbf{Function} \\ \hline
gNodeB & 5G Base Station \\ \hline
AMF & Access \& Mobility Mgmt \\ \hline
SMF & Session Mgmt \\ \hline
UPF & User Plane Function \\ \hline
\end{tabulary}
\end{center}

\begin{itemize}
    \item \keyword{Service Based}: Modular functions.
    \item \keyword{Network Slicing}: Virtual networks.
    \item \keyword{Edge Compute}: Low latency.
\end{itemize}
\end{solutionbox}

\begin{mnemonicbox}
\mnemonic{Service Based Network Slicing}
\end{mnemonicbox}

\questionmarks{5(c OR)}{7}{Explain MANET in detail.}

\begin{solutionbox}
\textbf{MANET (Mobile Ad-hoc Network):}
\begin{itemize}
    \item \keyword{Infrastructure-less}: No base stations.
    \item \keyword{Dynamic Topology}: Nodes move freely.
    \item \keyword{Multi-hop}: Relays messages via peers.
\end{itemize}

\textbf{Topology Diagram:}
\begin{center}
\begin{tikzpicture}[gtu flow]
    \node [gtu state] (a) {Node A};
    \node [gtu state, right=2cm of a] (b) {Node B};
    \node [gtu state, below=2cm of a] (c) {Node C};
    \node [gtu state, below=2cm of b] (d) {Node D};
    
    \draw [gtu arrow, <->] (a) -- (b);
    \draw [gtu arrow, <->] (a) -- (c);
    \draw [gtu arrow, <->] (c) -- (d);
    \draw [gtu arrow, <->] (b) -- (d);
    \draw [gtu arrow, <->] (b) -- (c);
\end{tikzpicture}
\captionof{figure}{Mesh Topology}
\end{center}

\textbf{Comparison:}
\begin{center}
\captionof{table}{MANET vs Cellular}
\begin{tabulary}{\linewidth}{|L|L|L|}
\hline
\textbf{Feature} & \textbf{MANET} & \textbf{Cellular} \\ \hline
Infra & None & Towers req \\ \hline
Cost & Low & High \\ \hline
Range & Multi-hop & Single hop \\ \hline
\end{tabulary}
\end{center}
\end{solutionbox}

\begin{mnemonicbox}
\mnemonic{Mobile Adhoc Network}
\end{mnemonicbox}

\end{document}
