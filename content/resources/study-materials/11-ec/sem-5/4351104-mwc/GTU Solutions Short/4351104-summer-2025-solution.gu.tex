\documentclass{article}

% content/resources/templates/preamble.tex
\usepackage[margin=0.6in]{geometry}
\author{Milav Dabgar}
\usepackage{amsmath,amssymb,amsthm}
\usepackage{booktabs}
\usepackage{multirow}
\usepackage{xcolor}
\usepackage{tcolorbox}
\tcbuselibrary{breakable,skins}
\usepackage[colorlinks=true,linkcolor=blue]{hyperref}
\usepackage{titlesec}
\usepackage{enumitem}
\usepackage{tikz}
\usepackage{pgfplots}
\usepackage{circuitikz}
\usepackage[version=4]{mhchem}
\usepackage{longtable}
\usepackage{array}
\usepackage{float}
\usepackage{caption}
\usepackage{listings}

\lstset{
  basicstyle=\small\ttfamily,
  breaklines=true,
  breakatwhitespace=false,
  postbreak=\mbox{\textcolor{red}{$\hookrightarrow$}\space},
  float=false,
  numbers=left,
  numberstyle=\tiny\color{gray},
  numbersep=10pt,
  xleftmargin=2em,
  keywordstyle=\color{blue},
  commentstyle=\color{green!60!black},
  stringstyle=\color{purple},
  backgroundcolor=\color{gray!5},
  showstringspaces=false,
  tabsize=2,
  captionpos=b,
  keepspaces=true,
  columns=flexible
}

\pgfplotsset{compat=1.18}
\usetikzlibrary{shapes,arrows,positioning,calc,patterns,decorations.pathmorphing,decorations.markings,arrows.meta}

% Color scheme
\definecolor{headcolor}{RGB}{0,102,204}
\definecolor{keycolor}{RGB}{220,20,60}
\definecolor{solutioncolor}{RGB}{34,139,34}
\definecolor{mnemoniccolor}{RGB}{148,0,211}
\definecolor{codecolor}{RGB}{0,0,100}

% Spacing
\setlength{\parskip}{3pt}
\setlist[itemize]{nosep}
\setlist[enumerate]{nosep}

% Title formatting
\titleformat{\section}{\Large\bfseries\color{headcolor}}{\thesection}{1em}{}
\titleformat{\subsection}{\large\bfseries\color{headcolor}}{\thesubsection}{1em}{}

% Pandoc tightlist compatibility
\providecommand{\tightlist}{%
  \setlength{\itemsep}{0pt}\setlength{\parskip}{0pt}}

% Pandoc longtable compatibility
\newcounter{none}
\def\thenone{}


% content/resources/templates/gujarati-boxes.tex
\usepackage{fontspec}
\usepackage{polyglossia}

% Set Gujarati as main language (document is primarily in Gujarati)
% Note: gloss-gujarati.ldf doesn't exist in polyglossia, but it will use hyphenation patterns
\setdefaultlanguage{gujarati}
\setotherlanguage{english}

% Configure Gujarati font properly
% Use Language=Default to prevent polyglossia from trying to add language-specific features
% that don't exist for Gujarati, which causes "empty feature" warnings
\newfontfamily\gujaratifont[Script=Gujarati,AutoFakeBold=2.5,AutoFakeSlant=0.3]{Noto Sans Gujarati}
\setmainfont[Script=Gujarati,AutoFakeBold=2.5,AutoFakeSlant=0.3]{Noto Sans Gujarati}
% Use Noto Sans Gujarati for monospace to support Gujarati in text
\setmonofont[Scale=0.9]{Noto Sans Gujarati}

% Configure English to use the same font
\newfontfamily\englishfont[Script=Gujarati,AutoFakeBold=2.5,AutoFakeSlant=0.3]{Noto Sans Gujarati}

% Translations for polyglossia
\gappto\captionsgujarati{
  \renewcommand{\tablename}{કોષ્ટક}
  \renewcommand{\figurename}{આકૃતિ}
}

% Helper for TikZ nodes to ensure Gujarati font
\newcommand{\gu}[1]{{\gujaratifont #1}}

% Custom environments
\newtcolorbox{solutionbox}{
    breakable,
    enhanced,
    colback=solutioncolor!5!white,
    colframe=solutioncolor!75!black,
    fonttitle=\bfseries,
    title=જવાબ
}

\newtcolorbox{solutionboxnobreak}{
 colback=solutioncolor!5!white,
 colframe=solutioncolor!75!black,
 fonttitle=\bfseries,
 title=જવાબ
}

\newtcolorbox{keyformula}{
 breakable,
 enhanced,
 colback=keycolor!5!white,
 colframe=keycolor!75!black,
 fonttitle=\bfseries,
 title=રાસાયણિક સમીકરણ/સૂત્ર
}

\newtcolorbox{mnemonicbox}{
 breakable,
 enhanced,
 colback=mnemoniccolor!5!white,
 colframe=mnemoniccolor!75!black,
 fonttitle=\bfseries,
 title=મેમરી ટ્રીક
}


% Custom commands for GTU solutions
% This file defines semantic commands for consistent formatting

% Question command with automatic formatting
\newcommand{\question}[2]{%
  \section*{Question #1}%
  \textbf{#2}%
}

% OR question variant
\newcommand{\questionor}[2]{%
  \section*{Question #1 OR}%
  \textbf{#2}%
}

% Proper table environment with caption
\newenvironment{answertable}[1]{%
  \begin{table}[htbp]
  \centering
  \caption{#1}
}{%
  \end{table}
}

% Proper figure environment for diagrams
\newenvironment{answerdiagram}[1]{%
  \begin{figure}[htbp]
  \centering
  \caption{#1}
}{%
  \end{figure}
}

% Semantic markup for key terms
\newcommand{\keyword}[1]{\textbf{#1}}
\newcommand{\code}[1]{\texttt{#1}}
\newcommand{\classname}[1]{\texttt{#1}}
\newcommand{\methodname}[1]{\texttt{#1}}

% Proper quotation marks
\newcommand{\mnemonic}[1]{``#1''}


\title{Mobile \& Wireless Communication (4351104) - Summer 2025 Solution}
\date{May 19, 2025}

\begin{document}
\maketitle

\questionmarks{1(અ)}{3}{4G અને 5G સિસ્ટમની મુખ્ય વિશેષતાઓ લખો.}

\begin{solutionbox}
\textbf{મુખ્ય વિશેષતાઓ તુલના:}
\begin{center}
\captionof{table}{4G વિ 5G સિસ્ટમ}
\begin{tabulary}{\linewidth}{|L|L|L|}
\hline
\textbf{લક્ષણ} & \textbf{4G સિસ્ટમ} & \textbf{5G સિસ્ટમ} \\ \hline
ડેટા સ્પીડ & 100 Mbps સુધી & 10 Gbps સુધી \\ \hline
લેટન્સી & 30-50 ms & 1-10 ms \\ \hline
ટેકનોલોજી & LTE, OFDM & MIMO, Beamforming \\ \hline
એપ્લિકેશન & વિડિયો સ્ટ્રીમિંગ & IoT, AR/VR \\ \hline
\end{tabulary}
\end{center}

\textbf{મુખ્ય મુદ્દાઓ:}
\begin{itemize}
    \item \keyword{4G}: હાઇ-સ્પીડ ડેટા માટે OFDM મોડ્યુલેશન સાથે LTE ટેકનોલોજી વાપરે છે.
    \item \keyword{5G}: અલ્ટ્રા-લો લેટન્સી સ્વાયત્ત વાહનો જેવી રીઅલ-ટાઇમ એપ્લિકેશન માટે સક્ષમ બનાવે છે.
    \item \keyword{નેટવર્ક સ્લાઇસિંગ}: 5G ચોક્કસ એપ્લિકેશન માટે વર્ચ્યુઅલ નેટવર્કની મંજૂરી આપે છે.
\end{itemize}
\end{solutionbox}

\begin{mnemonicbox}
\mnemonic{4G Fast, 5G Super-Fast}
\end{mnemonicbox}

\questionmarks{1(બ)}{4}{સેલ્યુલર મોબાઇલ સિસ્ટમમાં ફ્રીક્વન્સી રીયુઝનો કોન્સેપ્ટ સમજાવો.}

\begin{solutionbox}
\textbf{ફ્રીક્વન્સી રીયુઝ કોન્સેપ્ટ:}
\begin{center}
\begin{tikzpicture}[gtu flow]
    % Hexagonal grid approximation with Gujarati labels if necessary, here F1..F7 are universal
    \node [gtu state] (A1) {F1};
    \node [gtu state, right=0.8cm of A1] (B1) {F2};
    \node [gtu state, right=0.8cm of B1] (C1) {F3};
    
    \node [gtu state, below left=0.8cm of A1, xshift=0.4cm] (D1) {F4};
    \node [gtu state, right=0.8cm of D1] (E1) {F5};
    \node [gtu state, right=0.8cm of E1] (F1) {F6};
    
    \node [gtu state, below left=0.8cm of D1, xshift=0.4cm] (G1) {F7};
    \node [gtu state, right=0.8cm of G1] (A2) {F1};
    \node [gtu state, right=0.8cm of A2] (B2) {F2};
    
    \node [align=center, below=0.5cm of G1] {રીયુઝ ડિસ્ટન્સ};
    \draw [gtu arrow, <->, dashed] (A1) -- (A2);
\end{tikzpicture}
\captionof{figure}{ફ્રીક્વન્સી રીયુઝ પેટર્ન (N=7)}
\end{center}

\textbf{મુખ્ય મુદ્દાઓ:}
\begin{itemize}
    \item \keyword{ફ્રીક્વન્સી રીયુઝ}: કેપેસિટી વધારવા માટે બિન-સંલગ્ન સેલમાં સમાન ફ્રીક્વન્સીનો ઉપયોગ.
    \item \keyword{કો-ચેનલ અંતર}: સમાન ફ્રીક્વન્સીનો ઉપયોગ કરતા સેલ વચ્ચે ન્યૂનતમ અંતર.
    \item \keyword{ક્લસ્ટર સાઇઝ}: અલગ ફ્રીક્વન્સીનો ઉપયોગ કરતા સેલનું જૂથ (સામાન્ય રીતે 3, 4, 7, 12).
    \item \keyword{કેપેસિટી વૃદ્ધિ}: મર્યાદિત સ્પેક્ટ્રમ સાથે વધુ વપરાશકર્તાઓને સેવા.
\end{itemize}
\end{solutionbox}

\begin{mnemonicbox}
\mnemonic{Same Frequency, Different Places}
\end{mnemonicbox}

\questionmarks{1(ક)}{7}{જો કોઈ ચોક્કસ FDD સેલ્યુલર ટેલિફોન સિસ્ટમને કુલ 33 MHz બેન્ડવિડ્થ ફાળવવામાં આવે છે... (Refer to original for full text)}

\begin{solutionbox}
\textbf{આપેલ માહિતી:}
\begin{itemize}
    \item કુલ બેન્ડવિડ્થ = 33 MHz
    \item ચેનલ બેન્ડવિડ્થ = 25 kHz (સિમ્પ્લેક્સ)
    \item કંટ્રોલ સ્પેક્ટ્રમ = 1 MHz
    \item ક્લસ્ટર સાઇઝ = 7
\end{itemize}

\textbf{ગણતરીઓ:}

\textbf{પગલું 1: ટ્રાફિક માટે ઉપલબ્ધ સ્પેક્ટ્રમ}
$$ \text{ટ્રાફિક સ્પેક્ટ્રમ} = 33 - 1 = 32 \text{ MHz} $$

\textbf{પગલું 2: કુલ ડુપ્લેક્સ ચેનલો}
દરેક ડુપ્લેક્સ ચેનલને $2 \times 25 \text{ kHz} = 50 \text{ kHz}$ જોઈએ.
$$ \text{કુલ ચેનલો} = \frac{32 \text{ MHz}}{50 \text{ kHz}} = 640 \text{ ચેનલો} $$

\textbf{પગલું 3: કંટ્રોલ ચેનલો}
$$ \text{કંટ્રોલ ચેનલો} = \frac{1 \text{ MHz}}{25 \text{ kHz}} = 40 \text{ ચેનલો} $$

\textbf{પગલું 4: પ્રતિ સેલ વિતરણ}
\begin{itemize}
    \item પ્રતિ સેલ વોઇસ ચેનલો = $640 \div 7 \approx 91$ ચેનલો
    \item પ્રતિ સેલ કંટ્રોલ ચેનલો = $40 \div 7 \approx 6$ ચેનલો
\end{itemize}

\textbf{અંતિમ વિતરણ કોષ્ટક:}
\begin{center}
\captionof{table}{ચેનલ વિતરણ}
\begin{tabulary}{\linewidth}{|L|L|L|}
\hline
\textbf{પેરામીટર} & \textbf{કુલ} & \textbf{પ્રતિ સેલ} \\ \hline
વોઇસ ચેનલો & 640 & 91 \\ \hline
કંટ્રોલ ચેનલો & 40 & 6 \\ \hline
કુલ ચેનલો & 680 & 97 \\ \hline
\end{tabulary}
\end{center}
\end{solutionbox}

\begin{mnemonicbox}
\mnemonic{Divide Total by Cluster}
\end{mnemonicbox}

\questionmarks{1(ક OR)}{7}{સેલના પ્રકારોની યાદી બનાવો અને દરેકને સમજાવો.}

\begin{solutionbox}
\textbf{સેલના પ્રકારો:}
\begin{center}
\captionof{table}{સેલ પ્રકારોની સરખામણી}
\begin{tabulary}{\linewidth}{|L|L|L|L|}
\hline
\textbf{સેલ પ્રકાર} & \textbf{કવરેજ} & \textbf{પાવર} & \textbf{એપ્લિકેશન} \\ \hline
મેક્રો સેલ & 1-30 km & હાઇ & ગ્રામીણ \\ \hline
માઇક્રો સેલ & 100m-1km & મધ્યમ & શહેરી \\ \hline
પિકો સેલ & 10-100m & લો & બિલ્ડિંગ \\ \hline
ફેમ્ટો સેલ & 10-50m & ખૂબ લો & ઘર \\ \hline
\end{tabulary}
\end{center}

\textbf{વિગતવાર સમજૂતી:}
\begin{itemize}
    \item \keyword{મેક્રો સેલ}: મોટા ભૌગોલિક વિસ્તારો. હાઇ ટ્રાન્સમિશન પાવર.
    \item \keyword{માઇક્રો સેલ}: મધ્યમ વિસ્તારો. શહેરી વિસ્તારો અને હાઇવે માટે.
    \item \keyword{પિકો સેલ}: નાના ઇન્ડોર/આઉટડોર વિસ્તારો. શોપિંગ મોલ, એરપોર્ટ.
    \item \keyword{અમ્બ્રેલા સેલ}: અનેક નાના સેલને આવરી લે છે. હાઇ-સ્પીડ યુઝર્સ માટે હેન્ડઓફ ઘટાડે છે.
\end{itemize}
\end{solutionbox}

\begin{mnemonicbox}
\mnemonic{Macro-Micro-Pico-Femto = Big to Small}
\end{mnemonicbox}

\questionmarks{2(અ)}{3}{સેલ અને ક્લસ્ટર વ્યાખ્યાયિત કરો.}

\begin{solutionbox}
\textbf{વ્યાખ્યાઓ:}
\begin{itemize}
    \item \keyword{સેલ}: એક બેઝ સ્ટેશન દ્વારા આવરાયેલ ભૌગોલિક વિસ્તાર. સામાન્ય રીતે ષટ્કોણ આકાર.
    \item \keyword{ક્લસ્ટર}: અલગ ફ્રીક્વન્સી સેટનો ઉપયોગ કરતા સેલનું જૂથ. ફ્રીક્વન્સી રીયુઝ સક્ષમ બનાવે છે.
\end{itemize}

\textbf{તફાવત:}
\begin{center}
\captionof{table}{સેલ વિ ક્લસ્ટર}
\begin{tabulary}{\linewidth}{|L|L|L|}
\hline
\textbf{પેરામીટર} & \textbf{સેલ} & \textbf{ક્લસ્ટર} \\ \hline
એકમ & એકલ વિસ્તાર & સેલનું જૂથ \\ \hline
ફ્રીક્વન્સી & એક સેટ & અનેક સેટ \\ \hline
\end{tabulary}
\end{center}
\end{solutionbox}

\begin{mnemonicbox}
\mnemonic{Cell = One Area, Cluster = Group Areas}
\end{mnemonicbox}

\questionmarks{2(બ)}{4}{ક્ષમતા અને ઇન્ટર્ફેરન્સ પર ક્લસ્ટરના સાઇઝની અસર સમજાવો.}

\begin{solutionbox}
\textbf{ક્લસ્ટર સાઇઝની અસર:}
\begin{center}
\captionof{table}{ક્લસ્ટર સાઇઝ પ્રભાવ}
\begin{tabulary}{\linewidth}{|L|L|L|L|}
\hline
\textbf{ક્લસ્ટર} & \textbf{ક્ષમતા} & \textbf{ઇન્ટર્ફેરન્સ} & \textbf{અંતર} \\ \hline
નાનું (3,4) & હાઇ & હાઇ & ટૂંકું \\ \hline
મોટું (7,12) & લો & લો & લાંબું \\ \hline
\end{tabulary}
\end{center}

\textbf{મુખ્ય અસરો:}
\begin{itemize}
    \item \keyword{ક્ષમતા પર}: નાનું ક્લસ્ટર એટલે પ્રતિ સેલ વધુ ચેનલો, તેથી વધુ ક્ષમતા.
    \item \keyword{ઇન્ટર્ફેરન્સ પર}: નાનું ક્લસ્ટર વધુ કો-ચેનલ ઇન્ટર્ફેરન્સ લાવે છે.
    \item \keyword{કો-ચેનલ અંતર}: $D = R\sqrt{3N}$. મોટું N એટલે કો-ચેનલ સેલ વચ્ચે વધુ અંતર.
\end{itemize}
\end{solutionbox}

\begin{mnemonicbox}
\mnemonic{Small Cluster = More Capacity, More Interference}
\end{mnemonicbox}

\questionmarks{2(ક)}{7}{IS-95, CDMA2000 અને WCDMA ની મુખ્ય વિશેષતાઓ લખો.}

\begin{solutionbox}
\textbf{સરખામણી:}
\begin{center}
\captionof{table}{CDMA સ્ટાન્ડર્ડ્સ}
\begin{tabulary}{\linewidth}{|L|L|L|L|}
\hline
\textbf{લક્ષણ} & \textbf{IS-95} & \textbf{CDMA2000} & \textbf{WCDMA} \\ \hline
જનરેશન & 2G & 3G & 3G \\ \hline
ડેટા રેટ & 14.4 kbps & 2 Mbps & 2 Mbps \\ \hline
ચિપ રેટ & 1.2288 Mcps & 3.6864 Mcps & 3.84 Mcps \\ \hline
બેન્ડવિડ્થ & 1.25 MHz & 1.25 MHz & 5 MHz \\ \hline
\end{tabulary}
\end{center}

\textbf{વિશેષતાઓ:}
\begin{itemize}
    \item \textbf{IS-95}: પ્રથમ કોમર્શિયલ CDMA. GSM કરતાં સારી વોઇસ ક્વોલિટી. સોફ્ટ હેન્ડઓફ.
    \item \textbf{CDMA2000}: IS-95 સાથે બેકવર્ડ કમ્પેટિબલ. હાઇ ડેટા રેટ. મલ્ટિમીડિયા સપોર્ટ.
    \item \textbf{WCDMA}: 3G માટે ગ્લોબલ સ્ટાન્ડર્ડ. હાઇ કેપેસિટી. ઇન્ટરનેશનલ રોમિંગ.
\end{itemize}
\end{solutionbox}

\begin{mnemonicbox}
\mnemonic{IS-95 First, CDMA2000 Faster, WCDMA Global}
\end{mnemonicbox}

\questionmarks{2(અ OR)}{3}{સેલ સ્પ્લિટિંગ સમજાવો.}

\begin{solutionbox}
\textbf{વ્યાખ્યા}: સેલ સ્પ્લિટિંગ એ ભીડભાડવાળા સેલને નાના સેલમાં વિભાજિત કરીને સિસ્ટમ ક્ષમતા વધારવાની તકનીક છે.

\begin{center}
\begin{tikzpicture}[gtu flow]
    \node [gtu state] (big) {મૂળ મોટો સેલ};
    
    \node [gtu state, below left=1cm of big] (small1) {સેલ 1};
    \node [gtu state, right=0.5cm of small1] (small2) {સેલ 2};
    \node [gtu state, right=0.5cm of small2] (small3) {સેલ 3};
    
    \draw [gtu arrow] (big) -- (small1);
    \draw [gtu arrow] (big) -- (small2);
    \draw [gtu arrow] (big) -- (small3);
    
    \node [below=0.5cm of small2] {વધેલી ક્ષમતા};
\end{tikzpicture}
\captionof{figure}{સેલ સ્પ્લિટિંગ કન્સેપ્ટ}
\end{center}

\textbf{ફાયદા}: ક્ષમતા વૃદ્ધિ, વધુ સારી સિગ્નલ ક્વોલિટી.
\end{solutionbox}

\begin{mnemonicbox}
\mnemonic{Split Big Cell into Small Cells}
\end{mnemonicbox}

\questionmarks{2(બ OR)}{4}{GSM માં HLR અને VLR ના કાર્યો લખો.}

\begin{solutionbox}
\textbf{HLR (હોમ લોકેશન રજિસ્ટર):}
\begin{itemize}
    \item \keyword{સબ્સ્ક્રાઇબર પ્રોફાઇલ}: કાયમી ડેટા સંગ્રહિત કરે છે.
    \item \keyword{લોકેશન ટ્રેકિંગ}: સબ્સ્ક્રાઇબરનું વર્તમાન લોકેશન જાળવે છે.
    \item \keyword{ઓથેન્ટિકેશન}: ઓથેન્ટિકેશન કીઝ પ્રદાન કરે છે.
\end{itemize}

\textbf{VLR (વિઝિટર લોકેશન રજિસ્ટર):}
\begin{itemize}
    \item \keyword{અસ્થાયી સંગ્રહ}: વિઝિટિંગ સબ્સ્ક્રાઇબર ડેટા રાખે છે.
    \item \keyword{સ્થાનિક સેવાઓ}: રોમિંગ સક્ષમ બનાવે છે.
    \item \keyword{કોલ રાઉટિંગ}: વિઝિટર્સને કોલ રાઉટ કરવામાં મદદ કરે છે.
\end{itemize}
\end{solutionbox}

\begin{mnemonicbox}
\mnemonic{HLR = Home Data, VLR = Visitor Data}
\end{mnemonicbox}

\questionmarks{2(ક OR)}{7}{RFID ટેકનોલોજીનું વર્ણન કરો.}

\begin{solutionbox}
\textbf{RFID (Radio Frequency Identification):} ઓળખ અને ટ્રેકિંગ માટે રેડિયો તરંગોનો ઉપયોગ કરે છે.

\textbf{સિસ્ટમ ઘટકો:}
\begin{center}
\begin{tikzpicture}[gtu flow]
    \node [gtu block] (reader) {RFID રીડર};
    \node [gtu block, right=3cm of reader] (tag) {RFID ટેગ\\(ચિપ+એન્ટેના)};
    
    \draw [gtu arrow, <->, dashed] (reader) -- node[above]{રેડિયો તરંગો} (tag);
    
    \node [gtu block, below=1cm of reader] (host) {હોસ્ટ સિસ્ટમ};
    \draw [gtu arrow] (reader) -- (host);
\end{tikzpicture}
\captionof{figure}{RFID સિસ્ટમ}
\end{center}

\textbf{વિશેષતાઓ:}
\begin{itemize}
    \item \keyword{લાઇન ઓફ સાઇટ નહીં}: સીધા દૃશ્યની જરૂર નથી.
    \item \keyword{મલ્ટિપલ રીડિંગ}: એકસાથે અનેક ટેગ વાંચી શકાય છે.
    \item \keyword{ટકાઉપણું}: પર્યાવરણ સામે પ્રતિરોધક.
\end{itemize}
\end{solutionbox}

\begin{mnemonicbox}
\mnemonic{Radio Frequency Identifies Everything}
\end{mnemonicbox}

\questionmarks{3(અ)}{3}{GSM આર્કિટેક્ચર દોરો.}

\begin{solutionbox}
\textbf{GSM આર્કિટેક્ચર:}
\begin{center}
\begin{tikzpicture}[gtu flow]
    \node [gtu block] (ms) {MS};
    \node [gtu block, right=1cm of ms] (bts) {BTS};
    \node [gtu block, right=1cm of bts] (bsc) {BSC};
    \node [gtu block, above=1cm of bsc] (msc) {MSC};
    
    \node [gtu block, right=1cm of msc] (hlr) {HLR};
    \node [gtu block, left=1cm of msc] (vlr) {VLR};
    \node [gtu block, above=1cm of msc] (pstn) {PSTN};
    
    \draw [gtu arrow] (ms) -- (bts);
    \draw [gtu arrow] (bts) -- (bsc);
    \draw [gtu arrow] (bsc) -- (msc);
    \draw [gtu arrow] (msc) -- (hlr);
    \draw [gtu arrow] (msc) -- (vlr);
    \draw [gtu arrow] (msc) -- (pstn);
\end{tikzpicture}
\captionof{figure}{GSM નેટવર્ક આર્કિટેક્ચર}
\end{center}
\end{solutionbox}

\begin{mnemonicbox}
\mnemonic{Mobile Talks Through BTS-BSC-MSC}
\end{mnemonicbox}

\questionmarks{3(બ)}{4}{GSM 900 ના સ્પેશિફિકેશન લખો.}

\begin{solutionbox}
\textbf{GSM 900 સ્પેશિફિકેશન:}
\begin{center}
\captionof{table}{પેરામીટર્સ}
\begin{tabulary}{\linewidth}{|L|L|}
\hline
\textbf{પેરામીટર} & \textbf{સ્પેશિફિકેશન} \\ \hline
ફ્રીક્વન્સી બેન્ડ & 890-915 MHz (Up), 935-960 MHz (Down) \\ \hline
ચેનલ સ્પેસિંગ & 200 kHz \\ \hline
કુલ ચેનલો & 124 \\ \hline
મોડ્યુલેશન & GMSK \\ \hline
એક્સેસ મેથડ & TDMA/FDMA \\ \hline
ટાઇમ સ્લોટ & 8 પ્રતિ ફ્રેમ \\ \hline
સ્પીચ કોડિંગ & 13 kbps RPE-LTP \\ \hline
\end{tabulary}
\end{center}
\end{solutionbox}

\begin{mnemonicbox}
\mnemonic{900 MHz, 200 kHz spacing, 8 time slots}
\end{mnemonicbox}

\questionmarks{3(ક)}{7}{GSM માં મોબાઇલ થી લેન્ડલાઇન અને લેન્ડલાઇન થી મોબાઇલ કોલ પ્રક્રિયા સમજાવો.}

\begin{solutionbox}
\textbf{મોબાઇલ થી લેન્ડલાઇન (MOC):}
\begin{center}
\begin{tikzpicture}[gtu flow]
    \node [gtu start] (ms) {MS ડાયલ};
    \node [gtu process, right=0.5cm of ms] (ch) {ચેનલ\\રિક્વેસ્ટ};
    \node [gtu process, right=0.5cm of ch] (auth) {Auth\\(MSC)};
    \node [gtu process, below=0.5cm of auth] (route) {PSTN પર\\રાઉટ};
    \node [gtu stop, left=0.5cm of route] (conn) {કનેક્ટ};
    
    \draw [gtu arrow] (ms) -- (ch);
    \draw [gtu arrow] (ch) -- (auth);
    \draw [gtu arrow] (auth) -- (route);
    \draw [gtu arrow] (route) -- (conn);
\end{tikzpicture}
\captionof{figure}{MOC ફ્લો}
\end{center}

\textbf{લેન્ડલાઇન થી મોબાઇલ (MTC):}
\begin{center}
\begin{tikzpicture}[gtu flow]
    \node [gtu start] (pstn) {PSTN કોલ};
    \node [gtu process, right=0.5cm of pstn] (gmsc) {GMSC};
    \node [gtu process, right=0.5cm of gmsc] (hlr) {HLR ક્વેરી};
    \node [gtu process, below=0.5cm of hlr] (vmsc) {VMSC પર\\રાઉટ};
    \node [gtu process, left=0.5cm of vmsc] (page) {પેજિંગ};
    \node [gtu stop, left=0.5cm of page] (conn) {કનેક્ટ};
    
    \draw [gtu arrow] (pstn) -- (gmsc);
    \draw [gtu arrow] (gmsc) -- (hlr);
    \draw [gtu arrow] (hlr) -- (vmsc);
    \draw [gtu arrow] (vmsc) -- (page);
    \draw [gtu arrow] (page) -- (conn);
\end{tikzpicture}
\captionof{figure}{MTC ફ્લો}
\end{center}
\end{solutionbox}

\begin{mnemonicbox}
\mnemonic{Mobile Out = Direct, Mobile In = Find First}
\end{mnemonicbox}

\questionmarks{3(અ OR)}{3}{ફાસ્ટ અને સ્લો ફ્રીક્વન્સી હોપિંગ સમજાવો.}

\begin{solutionbox}
\textbf{તફાવત:}
\begin{center}
\captionof{table}{ફાસ્ટ વિ સ્લો હોપિંગ}
\begin{tabulary}{\linewidth}{|L|L|L|}
\hline
\textbf{પેરામીટર} & \textbf{ફાસ્ટ હોપિંગ} & \textbf{સ્લો હોપિંગ} \\ \hline
હોપ રેટ & $>$ સિમ્બોલ રેટ & $<$ સિમ્બોલ રેટ \\ \hline
સિમ્બોલ/હોપ & $< 1$ & $> 1$ \\ \hline
જટિલતા & હાઇ & લો \\ \hline
GSM ઉપયોગ & ના & હા (217 hops/s) \\ \hline
\end{tabulary}
\end{center}
\end{solutionbox}

\begin{mnemonicbox}
\mnemonic{Fast = Many hops per symbol, Slow = Many symbols per hop}
\end{mnemonicbox}

\questionmarks{3(બ OR)}{4}{GSM માં ઓથેન્ટિકેશન પ્રક્રિયા સમજાવો.}

\begin{solutionbox}
\textbf{ઓથેન્ટિકેશન પ્રક્રિયા:}
\begin{center}
\begin{tikzpicture}[gtu flow]
    \node [gtu block] (net) {નેટવર્ક (AuC)};
    \node [gtu block, right=4cm of net] (sim) {SIM (MS)};
    
    \draw [gtu arrow] (net) -- node[above] {RAND} (sim);
    \draw [gtu arrow] (sim) -- node[below] {SRES (A3)} (net);
    
    \node [gtu decision, below=1cm of net] (comp) { સરખામણી SRES?};
    \draw [gtu arrow] (net) -- (comp);
\end{tikzpicture}
\captionof{figure}{ચેલેન્જ-રિસ્પોન્સ}
\end{center}

\textbf{પગલાં:}
\begin{enumerate}
    \item \textbf{ચેલેન્જ}: નેટવર્ક RAND (128-bit) મોકલે છે.
    \item \textbf{રિસ્પોન્સ}: SIM Ki અને A3 વાપરીને SRES ગણે છે.
    \item \textbf{વેરિફિકેશન}: નેટવર્ક SRES સરખાવે છે.
\end{enumerate}
\end{solutionbox}

\begin{mnemonicbox}
\mnemonic{Random Challenge, Signed Response, Compare and Accept}
\end{mnemonicbox}

\questionmarks{3(ક OR)}{7}{GSM માં સિગ્નલ પ્રોસેસિંગનો બ્લોક ડાયાગ્રામ દોરો અને સમજાવો.}

\begin{solutionbox}
\textbf{GSM સિગ્નલ ચેન:}
\begin{center}
\begin{tikzpicture}[gtu flow]
    \node [gtu start] (speech) {વોઇસ};
    \node [gtu process, right=0.5cm of speech] (enc) {સ્પીચ\\એન્કોડર};
    \node [gtu process, right=0.5cm of enc] (chan) {ચેનલ\\કોડર};
    \node [gtu process, right=0.5cm of chan] (int) {ઇન્ટર-\\લીવર};
    \node [gtu process, below=0.5cm of int] (burst) {બર્સ્ટ\\ફોર્મેટ};
    \node [gtu process, left=0.5cm of burst] (ciph) {સાિફર};
    \node [gtu process, left=0.5cm of ciph] (mod) {GMSK\\Mod};
    \node [gtu stop, left=0.5cm of mod] (rf) {RF Tx};
    
    \draw [gtu arrow] (speech) -- (enc);
    \draw [gtu arrow] (enc) -- (chan);
    \draw [gtu arrow] (chan) -- (int);
    \draw [gtu arrow] (int) -- (burst);
    \draw [gtu arrow] (burst) -- (ciph);
    \draw [gtu arrow] (ciph) -- (mod);
    \draw [gtu arrow] (mod) -- (rf);
\end{tikzpicture}
\captionof{figure}{GSM Tx પ્રોસેસિંગ}
\end{center}

\textbf{ઘટકો:}
\begin{itemize}
    \item \textbf{સ્પીચ કોડિંગ}: RPE-LTP (13 kbps).
    \item \textbf{ચેનલ કોડિંગ}: એરર પ્રોટેક્શન માટે કન્વોલ્યુશનલ કોડ્સ.
    \item \textbf{ઇન્ટરલીવિંગ}: ફેડિંગ સામે લડવા માટે બિટ્સ ફેલાવે છે.
    \item \textbf{બર્સ્ટ ફોર્મેટિંગ}: ગાર્ડ/ટ્રેનિંગ બિટ્સ ઉમેરે છે.
    \item \textbf{મોડ્યુલેશન}: સ્પેક્ટ્રલ એફિશિયન્સી માટે GMSK.
\end{itemize}
\end{solutionbox}

\begin{mnemonicbox}
\mnemonic{Speech-Code-Interleave-Burst-Modulate-Transmit}
\end{mnemonicbox}

\questionmarks{4(અ)}{3}{બેઝબેન્ડ સેક્શનનો બ્લોક ડાયાગ્રામ દોરો.}

\begin{solutionbox}
\textbf{બેઝબેન્ડ બ્લોક ડાયાગ્રામ:}
\begin{center}
\begin{tikzpicture}[gtu flow]
    \node [gtu block] (dsp) {DSP / પ્રોસેસર};
    
    \node [gtu block, left=1cm of dsp] (rf) {RF I/F};
    \node [gtu block, right=1cm of dsp] (mem) {મેમરી};
    \node [gtu block, above=1cm of dsp] (audio) {ઓડિયો કોડેક};
    \node [gtu block, below=1cm of dsp] (ui) {ડિસ્પ્લે/કીપેડ};
    
    \draw [gtu arrow, <->] (dsp) -- (rf);
    \draw [gtu arrow, <->] (dsp) -- (mem);
    \draw [gtu arrow, <->] (dsp) -- (audio);
    \draw [gtu arrow, <->] (dsp) -- (ui);
\end{tikzpicture}
\captionof{figure}{બેઝબેન્ડ આર્કિટેક્ચર}
\end{center}
\end{solutionbox}

\begin{mnemonicbox}
\mnemonic{DSP Controls Audio, Memory, Display, RF}
\end{mnemonicbox}

\questionmarks{4(બ)}{4}{EDGE સમજાવો.}

\begin{solutionbox}
\textbf{EDGE (Enhanced Data rates for GSM Evolution):}
\begin{itemize}
    \item \keyword{મોડ્યુલેશન}: 8-PSK (3 bits/symbol) વાપરે છે vs GMSK (1 bit/symbol).
    \item \keyword{ડેટા રેટ}: 473 kbps સુધી (GPRS કરતા 3x).
    \item \keyword{લિંક અડેપ્ટેશન}: ચેનલ ક્વોલિટીના આધારે મોડ્યુલેશન સ્વિચ કરે છે.
\end{itemize}
\end{solutionbox}

\begin{mnemonicbox}
\mnemonic{EDGE = Enhanced Data rates for GSM Evolution}
\end{mnemonicbox}

\questionmarks{4(ક)}{7}{મોબાઇલ હેન્ડસેટનો બ્લોક ડાયાગ્રામ દોરો અને સમજાવો.}

\begin{solutionbox}
\textbf{મોબાઇલ હેન્ડસેટ ઘટકો:}
\begin{center}
\begin{tikzpicture}[gtu flow]
    \node [gtu block, minimum width=3cm] (cpu) {બેઝબેન્ડ / CPU};
    
    \node [gtu block, above=1cm of cpu] (rf) {RF ટ્રાન્સીવર};
    \node [gtu block, above=0.5cm of rf] (ant) {એન્ટેના};
    
    \node [gtu block, left=1cm of cpu] (aud) {ઓડિયો};
    \node [gtu block, right=1cm of cpu] (ui) {UI (LCD/Key)};
    \node [gtu block, below=1cm of cpu] (pwr) {પાવર Mgmt};
    \node [gtu block, below=0.5cm of pwr] (bat) {બેટરી};
    
    \draw [gtu arrow, <->] (cpu) -- (rf);
    \draw [gtu arrow] (ant) -- (rf);
    \draw [gtu arrow, <->] (cpu) -- (aud);
    \draw [gtu arrow, <->] (cpu) -- (ui);
    \draw [gtu arrow] (pwr) -- (cpu);
    \draw [gtu arrow] (bat) -- (pwr);
\end{tikzpicture}
\captionof{figure}{મોબાઇલ હેન્ડસેટ ડાયાગ્રામ}
\end{center}

\textbf{વિભાગો:}
\begin{itemize}
    \item \textbf{RF સેક્શન}: રેડિયો સિગ્નલ ટ્રાન્સમિટ/રિસીવ કરે છે.
    \item \textbf{બેઝબેન્ડ}: પ્રોટોકોલ હેન્ડલિંગ, DSP.
    \item \textbf{ઓડિયો}: માઇક/સ્પીકર ઇન્ટરફેસિંગ.
    \item \textbf{UI}: ડિસ્પ્લે અને કીપેડ.
    \item \textbf{પાવર}: બેટરી ચાર્જિંગ અને રેગ્યુલેશન.
\end{itemize}
\end{solutionbox}

\begin{mnemonicbox}
\mnemonic{Antenna-RF-Baseband-Audio-Display-Power}
\end{mnemonicbox}

\questionmarks{4(અ OR)}{3}{મોબાઇલના કારણે રેડિયેશનના જોખમો સમજાવો.}

\begin{solutionbox}
\textbf{જોખમો અને SAR:}
\begin{itemize}
    \item \keyword{SAR (Specific Absorption Rate)}: શરીર દ્વારા RF એનર્જી શોષણનો દર. એકમ: W/kg.
    \item \keyword{થર્મલ અસરો}: RF એનર્જીને કારણે ટિશ્યુ હીટિંગ.
    \item \keyword{સુરક્ષા}: હેન્ડ્સ-ફ્રી વાપરો, કોલ અવધિ મર્યાદિત કરો.
\end{itemize}
\end{solutionbox}

\begin{mnemonicbox}
\mnemonic{SAR measures absorption rate}
\end{mnemonicbox}

\questionmarks{4(બ OR)}{4}{મોબાઇલ હેન્ડસેટમાં ચાર્જિંગ સેક્શનનું કાર્ય વર્ણન કરો.}

\begin{solutionbox}
\textbf{ચાર્જિંગ બ્લોક ડાયાગ્રામ:}
\begin{center}
\begin{tikzpicture}[gtu flow]
    \node [gtu start] (ac) {AC Adap};
    \node [gtu process, right=0.5cm of ac] (reg) {રેગ્યુલેટર};
    \node [gtu process, right=0.5cm of reg] (ctrl) {ચાર્જ\\કંટ્રોલર};
    \node [gtu stop, right=0.5cm of ctrl] (bat) {બેટરી};
    
    \node [gtu process, below=0.5cm of ctrl] (prot) {પ્રોટેક્શન};
    \draw [gtu arrow] (ac) -- (reg);
    \draw [gtu arrow] (reg) -- (ctrl);
    \draw [gtu arrow] (ctrl) -- (bat);
    \draw [gtu arrow] (prot) -- (ctrl);
\end{tikzpicture}
\captionof{figure}{ચાર્જર સર્કિટ}
\end{center}

\textbf{પ્રક્રિયા:}
\begin{itemize}
    \item \keyword{CC/CV}: કોન્સ્ટન્ટ કરન્ટ પછી કોન્સ્ટન્ટ વોલ્ટેજ ચાર્જિંગ.
    \item \keyword{પ્રોટેક્શન}: ઓવર-વોલ્ટેજ, ઓવર-કરન્ટ, ટેમ્પરેચર મોનિટરિંગ.
\end{itemize}
\end{solutionbox}

\begin{mnemonicbox}
\mnemonic{Control Current, Voltage, Temperature, and Time}
\end{mnemonicbox}

\questionmarks{4(ક OR)}{7}{DSSS ટ્રાન્સમિટર અને રિસીવરનો બ્લોક ડાયાગ્રામ દોરો અને સમજાવો.}

\begin{solutionbox}
\textbf{DSSS ટ્રાન્સમિટર:}
\begin{center}
\begin{tikzpicture}[gtu flow]
    \node [gtu start] (data) {ડેટા};
    \node [gtu process, right=0.5cm of data] (mod) {મોડ્યુલેટર};
    \node [gtu process, right=0.5cm of mod] (spread) {સ્પ્રેડર\\(XOR)};
    \node [gtu process, below=0.5cm of spread] (pn) {PN Gen};
    \node [gtu stop, right=0.5cm of spread] (rf) {RF Tx};
    
    \draw [gtu arrow] (data) -- (mod);
    \draw [gtu arrow] (mod) -- (spread);
    \draw [gtu arrow] (pn) -- (spread);
    \draw [gtu arrow] (spread) -- (rf);
\end{tikzpicture}
\captionof{figure}{ટ્રાન્સમિટર}
\end{center}

\textbf{DSSS રિસીવર:}
\begin{center}
\begin{tikzpicture}[gtu flow]
    \node [gtu start] (rf) {RF Rx};
    \node [gtu process, right=0.5cm of rf] (desp) {ડિસ્પ્રેડ};
    \node [gtu process, below=0.5cm of desp] (pn) {PN Gen};
    \node [gtu process, right=0.5cm of desp] (demod) {ડિમોડ};
    \node [gtu stop, right=0.5cm of demod] (data) {ડેટા};
    
    \draw [gtu arrow] (rf) -- (desp);
    \draw [gtu arrow] (pn) -- (desp);
    \draw [gtu arrow] (desp) -- (demod);
    \draw [gtu arrow] (demod) -- (data);
    \draw [gtu arrow, dashed] (desp) -- (pn);
\end{tikzpicture}
\captionof{figure}{રિસીવર}
\end{center}

\textbf{ઓપરેશન}:
\begin{itemize}
    \item ડેટા મોડ્યુલેટ થાય છે અને હાઇ-રેટ PN કોડથી સ્પ્રેડ થાય છે.
    \item રિસીવર ડેટા રિકવર કરવા માટે લોકલ PN કોડ સિંક્રોનાઇઝ કરે છે.
\end{itemize}
\end{solutionbox}

\begin{mnemonicbox}
\mnemonic{Data Spreads with PN, Correlates to Recover}
\end{mnemonicbox}

\questionmarks{5(અ)}{3}{સ્પ્રેડ સ્પેક્ટ્રમની કોન્સેપ્ટ સમજાવો.}

\begin{solutionbox}
\textbf{સ્પ્રેડ સ્પેક્ટ્રમ:}
\begin{itemize}
    \item \keyword{કોન્સેપ્ટ}: ટ્રાન્સમિશન બેન્ડવિડ્થ માહિતી બેન્ડવિડ્થ કરતાં ઘણી મોટી હોય છે.
    \item \keyword{પ્રોસેસિંગ ગેઇન}: સ્પ્રેડિંગને કારણે SNR માં સુધારો.
    \item \keyword{ફાયદા}: એન્ટી-જામિંગ, ઇન્ટરસેપ્ટની ઓછી શક્યતા, મલ્ટિપલ એક્સેસ (CDMA).
\end{itemize}
\end{solutionbox}

\begin{mnemonicbox}
\mnemonic{Spread Wide, Gain Processing Power}
\end{mnemonicbox}

\questionmarks{5(બ)}{4}{સ્પ્રેડ સ્પેક્ટ્રમ ક્રાઇટેરિયા અને તેની એપ્લિકેશન લખો.}

\begin{solutionbox}
\textbf{ક્રાઇટેરિયા:}
\begin{enumerate}
    \item બેન્ડવિડ્થ $\gg$ ડેટા બેન્ડવિડ્થ.
    \item સ્પ્રેડિંગ કોડ ડેટાથી સ્વતંત્ર હોય છે.
    \item રિસીવર કોડ સાથે સિંક થાય છે.
\end{enumerate}

\textbf{એપ્લિકેશન:}
\begin{itemize}
    \item \keyword{મિલિટરી}: સુરક્ષિત, એન્ટી-જામ કોમ્યુનિકેશન.
    \item \keyword{સેલ્યુલર}: CDMA (IS-95, 3G).
    \item \keyword{WLAN}: WiFi (DSSS).
    \item \keyword{GPS}: સેટેલાઇટ પોઝિશનિંગ.
\end{itemize}
\end{solutionbox}

\begin{mnemonicbox}
\mnemonic{Military, Cellular, Satellite, Wireless use Spread Spectrum}
\end{mnemonicbox}

\questionmarks{5(ક)}{7}{CDMA માં કોલ પ્રોસેસિંગ સમજાવો.}

\begin{solutionbox}
\textbf{કોલ પ્રોસેસિંગ:}
\begin{center}
\begin{tikzpicture}[gtu flow]
    \node [gtu start] (orig) {કોલ ઓરિજિનેશન};
    \node [gtu process, right=0.5cm of orig] (access) {એક્સેસ\\પ્રોબ};
    \node [gtu process, right=0.5cm of access] (auth) {Auth \&\\Assign};
    \node [gtu process, below=0.5cm of auth] (traffic) {ટ્રાફિક\\ચેનલ};
    \node [gtu stop, left=0.5cm of traffic] (conv) {વાતચીત};
    
    \draw [gtu arrow] (orig) -- (access);
    \draw [gtu arrow] (access) -- (auth);
    \draw [gtu arrow] (auth) -- (traffic);
    \draw [gtu arrow] (traffic) -- (conv);
\end{tikzpicture}
\captionof{figure}{CDMA કોલ સેટઅપ}
\end{center}

\textbf{મુખ્ય લાક્ષણિકતાઓ:}
\begin{itemize}
    \item \keyword{સોફ્ટ હેન્ડઓફ}: મેક-બિફોર-બ્રેક.
    \item \keyword{પાવર કંટ્રોલ}: ક્લોઝ્ડ લૂપ (800 Hz) નીયર-ફાર સમસ્યા ઉકેલવા.
    \item \keyword{વોલ્શ કોડ્સ}: ચેનલ સેપરેશન માટે ઓર્થોગોનલ કોડ્સ.
\end{itemize}
\end{solutionbox}

\begin{mnemonicbox}
\mnemonic{Access-Authenticate-Assign-Traffic-Handoff}
\end{mnemonicbox}

\questionmarks{5(અ OR)}{3}{ઝિગબીની વિશેષતાઓ અને ફાયદાઓ લખો.}

\begin{solutionbox}
\textbf{Zigbee (IEEE 802.15.4):}
\begin{itemize}
    \item \keyword{વિશેષતાઓ}: લો પાવર, મેશ નેટવર્કિંગ, લો ડેટા રેટ (250 kbps).
    \item \keyword{ફાયદા}: લાંબી બેટરી લાઇફ (વર્ષો), સેલ્ફ-હીલિંગ મેશ.
    \item \keyword{એપ્લિકેશન}: હોમ ઓટોમેશન, સેન્સર.
\end{itemize}
\end{solutionbox}

\begin{mnemonicbox}
\mnemonic{Low Power, Mesh Network, Many Applications}
\end{mnemonicbox}

\questionmarks{5(બ OR)}{4}{બ્લોક ડાયાગ્રામ સાથે OFDM સમજાવો.}

\begin{solutionbox}
\textbf{OFDM બ્લોક ડાયાગ્રામ:}
\begin{center}
\begin{tikzpicture}[gtu flow]
    \node [gtu start] (ser) {સીરિયલ ડેટા};
    \node [gtu process, right=0.5cm of ser] (sp) {S/P};
    \node [gtu process, right=0.5cm of sp] (ifft) {IFFT};
    \node [gtu process, right=0.5cm of ifft] (cp) {Add CP};
    \node [gtu stop, right=0.5cm of cp] (tx) {Tx};
    
    \draw [gtu arrow] (ser) -- (sp);
    \draw [gtu arrow] (sp) -- (ifft);
    \draw [gtu arrow] (ifft) -- (cp);
    \draw [gtu arrow] (cp) -- (tx);
\end{tikzpicture}
\captionof{figure}{OFDM ટ્રાન્સમિટર}
\end{center}

\textbf{કોન્સેપ્ટ}:
\begin{itemize}
    \item હાઇ-સ્પીડ ડેટાને પેરેલલ લો-સ્પીડ સબકેરિયરમાં વિભાજિત કરે છે.
    \item \keyword{ઓર્થોગોનલ}: સબકેરિયર ઇન્ટર્ફેર કરતા નથી.
    \item \keyword{સાઇક્લિક પ્રીફિક્સ}: ISI અટકાવવા ગાર્ડ ઇન્ટરવલ.
\end{itemize}
\end{solutionbox}

\begin{mnemonicbox}
\mnemonic{Orthogonal Frequencies Divide Multiplexed data}
\end{mnemonicbox}

\questionmarks{5(ક OR)}{7}{MANET નું વર્ણન કરો.}

\begin{solutionbox}
\textbf{MANET (Mobile Ad-hoc Network):} ઇન્ફ્રાસ્ટ્રક્ચર વિનાનું સ્વ-કોન્ફિગરિંગ મોબાઇલ નેટવર્ક.

\textbf{ટોપોલોજી:}
\begin{center}
\begin{tikzpicture}[gtu flow]
    \node [gtu state] (A) {નોડ A};
    \node [gtu state, right=2cm of A] (B) {નોડ B};
    \node [gtu state, below=1.5cm of A] (C) {નોડ C};
    \node [gtu state, below=1.5cm of B] (D) {નોડ D};
    
    \draw [gtu arrow, <->] (A) -- (B);
    \draw [gtu arrow, <->] (A) -- (C);
    \draw [gtu arrow, <->] (C) -- (D);
    \draw [gtu arrow, <->] (B) -- (D);
    \draw [gtu arrow, <->] (B) -- (C);
\end{tikzpicture}
\captionof{figure}{મેશ ટોપોલોજી}
\end{center}

\textbf{રાઉટિંગ પ્રોટોકોલ:}
\begin{itemize}
    \item \keyword{પ્રોએક્ટિવ}: DSDV (ટેબલ ડ્રિવન).
    \item \keyword{રિએક્ટિવ}: AODV, DSR (ઓન-ડિમાન્ડ).
    \item \keyword{હાઇબ્રિડ}: ZRP.
\end{itemize}
\end{solutionbox}

\begin{mnemonicbox}
\mnemonic{Mobile Nodes, Ad-hoc Routing, No Infrastructure, Temporary Networks}
\end{mnemonicbox}

\end{document}
