\documentclass[10pt,a4paper]{article}

% content/resources/templates/preamble.tex
\usepackage[margin=0.6in]{geometry}
\author{Milav Dabgar}
\usepackage{amsmath,amssymb,amsthm}
\usepackage{booktabs}
\usepackage{multirow}
\usepackage{xcolor}
\usepackage{tcolorbox}
\tcbuselibrary{breakable,skins}
\usepackage[colorlinks=true,linkcolor=blue]{hyperref}
\usepackage{titlesec}
\usepackage{enumitem}
\usepackage{tikz}
\usepackage{pgfplots}
\usepackage{circuitikz}
\usepackage[version=4]{mhchem}
\usepackage{longtable}
\usepackage{array}
\usepackage{float}
\usepackage{caption}
\usepackage{listings}

\lstset{
  basicstyle=\small\ttfamily,
  breaklines=true,
  breakatwhitespace=false,
  postbreak=\mbox{\textcolor{red}{$\hookrightarrow$}\space},
  float=false,
  numbers=left,
  numberstyle=\tiny\color{gray},
  numbersep=10pt,
  xleftmargin=2em,
  keywordstyle=\color{blue},
  commentstyle=\color{green!60!black},
  stringstyle=\color{purple},
  backgroundcolor=\color{gray!5},
  showstringspaces=false,
  tabsize=2,
  captionpos=b,
  keepspaces=true,
  columns=flexible
}

\pgfplotsset{compat=1.18}
\usetikzlibrary{shapes,arrows,positioning,calc,patterns,decorations.pathmorphing,decorations.markings,arrows.meta}

% Color scheme
\definecolor{headcolor}{RGB}{0,102,204}
\definecolor{keycolor}{RGB}{220,20,60}
\definecolor{solutioncolor}{RGB}{34,139,34}
\definecolor{mnemoniccolor}{RGB}{148,0,211}
\definecolor{codecolor}{RGB}{0,0,100}

% Spacing
\setlength{\parskip}{3pt}
\setlist[itemize]{nosep}
\setlist[enumerate]{nosep}

% Title formatting
\titleformat{\section}{\Large\bfseries\color{headcolor}}{\thesection}{1em}{}
\titleformat{\subsection}{\large\bfseries\color{headcolor}}{\thesubsection}{1em}{}

% Pandoc tightlist compatibility
\providecommand{\tightlist}{%
  \setlength{\itemsep}{0pt}\setlength{\parskip}{0pt}}

% Pandoc longtable compatibility
\newcounter{none}
\def\thenone{}


% content/resources/templates/gujarati-boxes.tex
\usepackage{fontspec}
\usepackage{polyglossia}

% Set Gujarati as main language (document is primarily in Gujarati)
% Note: gloss-gujarati.ldf doesn't exist in polyglossia, but it will use hyphenation patterns
\setdefaultlanguage{gujarati}
\setotherlanguage{english}

% Configure Gujarati font properly
% Use Language=Default to prevent polyglossia from trying to add language-specific features
% that don't exist for Gujarati, which causes "empty feature" warnings
\newfontfamily\gujaratifont[Script=Gujarati,AutoFakeBold=2.5,AutoFakeSlant=0.3]{Noto Sans Gujarati}
\setmainfont[Script=Gujarati,AutoFakeBold=2.5,AutoFakeSlant=0.3]{Noto Sans Gujarati}
% Use Noto Sans Gujarati for monospace to support Gujarati in text
\setmonofont[Scale=0.9]{Noto Sans Gujarati}

% Configure English to use the same font
\newfontfamily\englishfont[Script=Gujarati,AutoFakeBold=2.5,AutoFakeSlant=0.3]{Noto Sans Gujarati}

% Translations for polyglossia
\gappto\captionsgujarati{
  \renewcommand{\tablename}{કોષ્ટક}
  \renewcommand{\figurename}{આકૃતિ}
}

% Helper for TikZ nodes to ensure Gujarati font
\newcommand{\gu}[1]{{\gujaratifont #1}}

% Custom environments
\newtcolorbox{solutionbox}{
    breakable,
    enhanced,
    colback=solutioncolor!5!white,
    colframe=solutioncolor!75!black,
    fonttitle=\bfseries,
    title=જવાબ
}

\newtcolorbox{solutionboxnobreak}{
 colback=solutioncolor!5!white,
 colframe=solutioncolor!75!black,
 fonttitle=\bfseries,
 title=જવાબ
}

\newtcolorbox{keyformula}{
 breakable,
 enhanced,
 colback=keycolor!5!white,
 colframe=keycolor!75!black,
 fonttitle=\bfseries,
 title=રાસાયણિક સમીકરણ/સૂત્ર
}

\newtcolorbox{mnemonicbox}{
 breakable,
 enhanced,
 colback=mnemoniccolor!5!white,
 colframe=mnemoniccolor!75!black,
 fonttitle=\bfseries,
 title=મેમરી ટ્રીક
}


\begin{document}

\begin{center}
{\Huge\bfseries\color{headcolor} Subject Name (Gujarati)}\\[5pt]
{\LARGE 4351104 -- Winter 2023}\\[3pt]
{\large Semester 1 Study Material}\\[3pt]
{\normalsize\textit{Detailed Solutions and Explanations}}
\end{center}

\vspace{10pt}

\subsection*{પ્રશ્ન 1(અ) [3
ગુણ]}\label{uxaaauxab0uxab6uxaa8-1uxa85-3-uxa97uxaa3}

\textbf{અમ્બ્રેલા સેલ આકૃતિ દોરી સમજાવો.}

\begin{solutionbox}

\begin{center}
\textbf{Mermaid Diagram (Code)}
\begin{verbatim}
{Shaded}
{Highlighting}[]
graph TD
    A[વિશાળ કવરેજ વિસ્તાર] {-{-}{} B[અમ્બ્રેલા સેલ ટાવર]}
    B {-{-}{} C[માઈક્રો સેલ 1]}
    B {-{-}{} D[માઈક્રો સેલ 2]}
    B {-{-}{} E[માઈક્રો સેલ 3]}
    C {-{-}{} F[ગીચ વિસ્તારના વપરાશકર્તાઓ]}
    D {-{-}{} G[ગીચ વિસ્તારના વપરાશકર્તાઓ]}
    E {-{-}{} H[ગીચ વિસ્તારના વપરાશકર્તાઓ]}
{Highlighting}
{Shaded}
\end{verbatim}
\end{center}

\begin{itemize}
\tightlist
\item
  \textbf{અમ્બ્રેલા સેલ}: નાના સેલોને આવરી લેતા વિશાળ કવરેજ વાળા સેલ
\item
  \textbf{હેતુ}: માઈક્રો/પિકો સેલોમાંથી વધારે ટ્રાફિક સંભાળે છે
\item
  \textbf{કવરેજ}: ઉચ્ચ-ટ્રાફિક વિસ્તારો માટે બેકઅપ કવરેજ પૂરું પાડે છે
\end{itemize}

\end{solutionbox}
\begin{mnemonicbox}
``મારા મોટા છત્ર નીચે''

\end{mnemonicbox}
\subsection*{પ્રશ્ન 1(બ) [4
ગુણ]}\label{uxaaauxab0uxab6uxaa8-1uxaac-4-uxa97uxaa3}

\textbf{ફુલ ફોર્મ લખો : (i) CCH (ii) TCH (iii) SCH (iv) BCCH}

\begin{solutionbox}

{\def\LTcaptype{none} % do not increment counter
\begin{longtable}[]{@{}lll@{}}
\toprule\noalign{}
સંક્ષેપ & પૂરું નામ & કાર્ય \\
\midrule\noalign{}
\endhead
\bottomrule\noalign{}
\endlastfoot
CCH & Control Channel & નિયંત્રણ માહિતી વહન કરે છે \\
TCH & Traffic Channel & અવાજ/ડેટા ટ્રાફિક વહન કરે છે \\
SCH & Synchronization Channel & સમય સિંક્રોનાઈઝેશન પૂરું પાડે છે \\
BCCH & Broadcast Control Channel & સિસ્ટમ માહિતી પ્રસારિત કરે છે \\
\end{longtable}
}

\end{solutionbox}
\begin{mnemonicbox}
``કંટ્રોલ ટ્રાફિક સિંક બ્રોડકાસ્ટ''

\end{mnemonicbox}
\subsection*{પ્રશ્ન 1(ક) [7
ગુણ]}\label{uxaaauxab0uxab6uxaa8-1uxa95-7-uxa97uxaa3}

\textbf{સેલ શું છે? અલગ અલગ પ્રકારના સેલ સમજાવો.}

\begin{solutionbox}
\textbf{સેલ} એ સેલ્યુલર કમ્યુનિકેશનમાં એક બેઝ સ્ટેશન દ્વારા આવરી
લેવાતો મૂળભૂત કવરેજ વિસ્તાર છે.

{\def\LTcaptype{none} % do not increment counter
\begin{longtable}[]{@{}llll@{}}
\toprule\noalign{}
સેલનો પ્રકાર & કવરેજ & પાવર & ઉપયોગ \\
\midrule\noalign{}
\endhead
\bottomrule\noalign{}
\endlastfoot
\textbf{માક્રો સેલ} & 1-30 km & ઉચ્ચ & ગ્રામ્ય વિસ્તારો \\
\textbf{માઈક્રો સેલ} & 100m-2km & મધ્યમ & શહેરી વિસ્તારો \\
\textbf{પિકો સેલ} & 10-100m & નીચું & ઇન્ડોર કવરેજ \\
\textbf{ફેમ્ટો સેલ} & 10-30m & ખૂબ નીચું & ઘર/ઓફિસ \\
\end{longtable}
}

\begin{center}
\textbf{Mermaid Diagram (Code)}
\begin{verbatim}
{Shaded}
{Highlighting}[]
graph TD
    A[માક્રો સેલ] {-{-}{} B[વિશાળ વિસ્તાર કવરેજ]}
    C[માઈક્રો સેલ] {-{-}{} D[શહેર કવરેજ]}
    E[પિકો સેલ] {-{-}{} F[બિલ્ડિંગ કવરેજ]}
    G[ફેમ્ટો સેલ] {-{-}{} H[રૂમ કવરેજ]}
{Highlighting}
{Shaded}
\end{verbatim}
\end{center}

\begin{itemize}
\tightlist
\item
  \textbf{કાર્ય}: દરેક સેલ મોબાઈલ વપરાશકર્તાઓને વાયરલેસ સેવા પૂરી પાડે છે
\item
  \textbf{આવૃત્તિ પુનઃઉપયોગ}: બિન-સંલગ્ન સેલોમાં સમાન આવૃત્તિઓનો ઉપયોગ
\item
  \textbf{હેન્ડઓફ}: વપરાશકર્તાઓ સેલો વચ્ચે નિરંતર ખસી શકે છે
\end{itemize}

\end{solutionbox}
\begin{mnemonicbox}
``ઘણા મોબાઈલ લોકો કવરેજ શોધે છે''

\end{mnemonicbox}
\subsection*{પ્રશ્ન 1(ક અથવા) [7
ગુણ]}\label{uxaaauxab0uxab6uxaa8-1uxa95-uxa85uxaa5uxab5-7-uxa97uxaa3}

\textbf{હેન્ડઓફ શું છે? સોફ્ટ અને હાર્ડ હેન્ડઓફ સમજાવો.}

\begin{solutionbox}
\textbf{હેન્ડઓફ} એ મોબાઈલ ખસતા સમયે ચાલુ કોલને એક સેલમાંથી બીજા
સેલમાં સ્થાનાંતરિત કરવાની પ્રક્રિયા છે.

{\def\LTcaptype{none} % do not increment counter
\begin{longtable}[]{@{}lll@{}}
\toprule\noalign{}
લક્ષણ & હાર્ડ હેન્ડઓફ & સોફ્ટ હેન્ડઓફ \\
\midrule\noalign{}
\endhead
\bottomrule\noalign{}
\endlastfoot
\textbf{કનેક્શન} & તોડ્યા પછી જોડાણ & જોડાણ પછી તોડવું \\
\textbf{ચેનલો} & એક સમયે એક & એકસાથે ઘણા \\
\textbf{ટેક્નોલોજી} & GSM, TDMA & CDMA \\
\textbf{ગુણવત્તા} & થોડી વિક્ષેપ & સરળ સંક્રમણ \\
\end{longtable}
}

\begin{verbatim}
sequenceDiagram
    participant M as મોબાઈલ
    participant BS1 as બેઝ સ્ટેશન 1
    participant BS2 as બેઝ સ્ટેશન 2
    
    Note over M,BS2: હાર્ડ હેન્ડઓફ
    M{-BS1: જોડાયેલ}
    BS1{-{-}M: સિગ્નલ નબળું પડે છે}
    BS1{-BS2: હેન્ડઓફ વિનંતી}
    M{-BS2: નવું કનેક્શન}
    
    Note over M,BS2: સોફ્ટ હેન્ડઓફ
    M{-BS1: જોડાયેલ}
    M{-BS2: બેવડું કનેક્શન}
    M{-{-}BS1: નબળા સિગ્નલને છોડો}
\end{verbatim}

\begin{itemize}
\tightlist
\item
  \textbf{પ્રારંભ}: સિગ્નલ મજબૂતાઈના માપ પર આધારિત
\item
  \textbf{MAHO}: Mobile Assisted Handoff નિર્ણયની ચોકસાઈ સુધારે છે
\end{itemize}

\end{solutionbox}
\begin{mnemonicbox}
``હાર્ડ દુખાવે, સોફ્ટ સરળ''

\end{mnemonicbox}
\subsection*{પ્રશ્ન 2(અ) [3
ગુણ]}\label{uxaaauxab0uxab6uxaa8-2uxa85-3-uxa97uxaa3}

\textbf{ફુલ ફોર્મ લખો : (i) SIM (ii) LTE (iii) WCDMA}

\begin{solutionbox}

{\def\LTcaptype{none} % do not increment counter
\begin{longtable}[]{@{}lll@{}}
\toprule\noalign{}
સંક્ષેપ & પૂરું નામ & હેતુ \\
\midrule\noalign{}
\endhead
\bottomrule\noalign{}
\endlastfoot
SIM & Subscriber Identity Module & વપરાશકર્તા પ્રમાણીકરણ \\
LTE & Long Term Evolution & 4G ટેક્નોલોજી \\
WCDMA & Wideband Code Division Multiple Access & 3G માનક \\
\end{longtable}
}

\end{solutionbox}
\begin{mnemonicbox}
``સબ્સ્ક્રાઈબરનું લાંબા વાઈડબેન્ડ કનેક્શન''

\end{mnemonicbox}
\subsection*{પ્રશ્ન 2(બ) [4
ગુણ]}\label{uxaaauxab0uxab6uxaa8-2uxaac-4-uxa97uxaa3}

\textbf{મોબાઈલ હેન્ડસેટની બ્લોક આકૃતિ દોરો.}

\begin{solutionbox}

\begin{center}
\textbf{Mermaid Diagram (Code)}
\begin{verbatim}
{Shaded}
{Highlighting}[]
graph TD
    A[એન્ટેના] {-{-}{} B[RF સેક્શન]}
    B {-{-}{} C[બેઝબેન્ડ પ્રોસેસર]}
    C {-{-}{} D[ઓડિયો સેક્શન]}
    C {-{-}{} E[ડિસ્પ્લે/કીપેડ]}
    C {-{-}{} F[મેમરી]}
    G[બૅટરી] {-{-}{} H[પાવર મેનેજમેન્ટ]}
    H {-{-}{} B}
    H {-{-}{} C}
    H {-{-}{} D}
{Highlighting}
{Shaded}
\end{verbatim}
\end{center}

\begin{itemize}
\tightlist
\item
  \textbf{RF સેક્શન}: રેડિયો સિગ્નલ મોકલે/મેળવે છે
\item
  \textbf{બેઝબેન્ડ}: ડિજિટલ સિગ્નલ અને પ્રોટોકોલ પ્રોસેસ કરે છે
\item
  \textbf{ઓડિયો}: અવાજનું ઇનપુટ/આઉટપુટ સંભાળે છે
\item
  \textbf{પાવર મેનેજમેન્ટ}: બૅટરીનો ઉપયોગ કાર્યક્ષમતાથી નિયંત્રિત કરે છે
\end{itemize}

\end{solutionbox}
\begin{mnemonicbox}
``રેડિયો બેઝબેન્ડ ઓડિયો પાવર''

\end{mnemonicbox}
\subsection*{પ્રશ્ન 2(ક) [7
ગુણ]}\label{uxaaauxab0uxab6uxaa8-2uxa95-7-uxa97uxaa3}

\textbf{GSM આર્કિટેક્ચર આકૃતિ સાથે સમજાવો.}

\begin{solutionbox}

\begin{center}
\textbf{Mermaid Diagram (Code)}
\begin{verbatim}
{Shaded}
{Highlighting}[]
graph LR
    A[MS] {-{-}{} B[BTS]}
    B {-{-}{} C[BSC]}
    C {-{-}{} D[MSC]}
    D {-{-}{} E[HLR]}
    D {-{-}{} F[VLR]}
    D {-{-}{} G[AuC]}
    D {-{-}{} H[PSTN]}
    
    subgraph BSS
    B
    C
    end
    
    subgraph NSS
    D
    E
    F
    G
    end
{Highlighting}
{Shaded}
\end{verbatim}
\end{center}

{\def\LTcaptype{none} % do not increment counter
\begin{longtable}[]{@{}ll@{}}
\toprule\noalign{}
ઘટક & કાર્ય \\
\midrule\noalign{}
\endhead
\bottomrule\noalign{}
\endlastfoot
\textbf{MS} & Mobile Station (હેન્ડસેટ) \\
\textbf{BTS} & Base Transceiver Station \\
\textbf{BSC} & Base Station Controller \\
\textbf{MSC} & Mobile Switching Center \\
\textbf{HLR} & Home Location Register \\
\textbf{VLR} & Visitor Location Register \\
\end{longtable}
}

\begin{itemize}
\tightlist
\item
  \textbf{BSS}: Base Station Subsystem રેડિયો ઇન્ટરફેસ સંભાળે છે
\item
  \textbf{NSS}: Network Switching Subsystem કોલો મેનેજ કરે છે
\item
  \textbf{પ્રમાણીકરણ}: AuC સબ્સ્ક્રાઈબરની ઓળખ ચકાસે છે
\end{itemize}

\end{solutionbox}
\begin{mnemonicbox}
``મોબાઈલ બેઝ નેટવર્ક ઘર કોલ કરે છે''

\end{mnemonicbox}
\subsection*{પ્રશ્ન 2(અ અથવા) [3
ગુણ]}\label{uxaaauxab0uxab6uxaa8-2uxa85-uxa85uxaa5uxab5-3-uxa97uxaa3}

\textbf{ફુલ ફોર્મ લખો : (i) RSSI (ii) MAHO (iii) NCHO}

\begin{solutionbox}

{\def\LTcaptype{none} % do not increment counter
\begin{longtable}[]{@{}lll@{}}
\toprule\noalign{}
સંક્ષેપ & પૂરું નામ & કાર્ય \\
\midrule\noalign{}
\endhead
\bottomrule\noalign{}
\endlastfoot
RSSI & Received Signal Strength Indicator & સિગ્નલ ગુણવત્તા માપ \\
MAHO & Mobile Assisted Handoff & મોબાઈલ હેન્ડઓફ નિર્ણયમાં મદદ કરે છે \\
NCHO & Network Controlled Handoff & નેટવર્ક હેન્ડઓફ નક્કી કરે છે \\
\end{longtable}
}

\end{solutionbox}
\begin{mnemonicbox}
``પ્રાપ્ત મોબાઈલ નેટવર્ક સિગ્નલો''

\end{mnemonicbox}
\subsection*{પ્રશ્ન 2(બ અથવા) [4
ગુણ]}\label{uxaaauxab0uxab6uxaa8-2uxaac-uxa85uxaa5uxab5-4-uxa97uxaa3}

\textbf{બેઝબેન્ડ સેક્શનની બ્લોક આકૃતિ દોરો.}

\begin{solutionbox}

\begin{center}
\textbf{Mermaid Diagram (Code)}
\begin{verbatim}
{Shaded}
{Highlighting}[]
graph LR
    A[ADC/DAC] {-{-}{} B[DSP]}
    B {-{-}{} C[ચેનલ કોડેક]}
    C {-{-}{} D[સ્પીચ કોડેક]}
    D {-{-}{} E[ઓડિયો ઇન્ટરફેસ]}
    B {-{-}{} F[પ્રોટોકોલ સ્ટેક]}
    F {-{-}{} G[કંટ્રોલ ઇન્ટરફેસ]}
{Highlighting}
{Shaded}
\end{verbatim}
\end{center}

\begin{itemize}
\tightlist
\item
  \textbf{ADC/DAC}: Analog to Digital કન્વર્ઝન
\item
  \textbf{DSP}: Digital Signal Processor
\item
  \textbf{ચેનલ કોડેક}: ભૂલ સુધારણા કોડિંગ
\item
  \textbf{સ્પીચ કોડેક}: અવાજ સંકોચન/વિસ્તારણ
\end{itemize}

\end{solutionbox}
\begin{mnemonicbox}
``એનાલોગ ડિજિટલ સ્પીચ પ્રોટોકોલ''

\end{mnemonicbox}
\subsection*{પ્રશ્ન 2(ક અથવા) [7
ગુણ]}\label{uxaaauxab0uxab6uxaa8-2uxa95-uxa85uxaa5uxab5-7-uxa97uxaa3}

\textbf{GSM સિગ્નલ પ્રોસેસિંગ આકૃતિ સાથે સમજાવો.}

\begin{solutionbox}

\begin{center}
\textbf{Mermaid Diagram (Code)}
\begin{verbatim}
{Shaded}
{Highlighting}[]
graph LR
    A[અવાજ] {-{-}{} B[સ્પીચ કોડેક]}
    B {-{-}{} C[ચેનલ કોડેક]}
    C {-{-}{} D[ઇન્ટરલીવિંગ]}
    D {-{-}{} E[બર્સ્ટ ફોર્મેટર]}
    E {-{-}{} F[GMSK મોડ્યુલેટર]}
    F {-{-}{} G[RF ટ્રાન્સમિટર]}
{Highlighting}
{Shaded}
\end{verbatim}
\end{center}

{\def\LTcaptype{none} % do not increment counter
\begin{longtable}[]{@{}lll@{}}
\toprule\noalign{}
તબક્કો & કાર્ય & હેતુ \\
\midrule\noalign{}
\endhead
\bottomrule\noalign{}
\endlastfoot
\textbf{સ્પીચ કોડેક} & અવાજને 13 kbps માં સંકોચે છે & બેન્ડવિડ્થ કાર્યક્ષમતા \\
\textbf{ચેનલ કોડેક} & ભૂલ સુધારણા ઉમેરે છે & સિગ્નલ વિશ્વસનીયતા \\
\textbf{ઇન્ટરલીવિંગ} & બર્સ્ટ ભૂલો વિતરિત કરે છે & ભૂલ સુરક્ષા \\
\textbf{GMSK} & Gaussian MSK મોડ્યુલેશન & સ્પેક્ટ્રલ કાર્યક્ષમતા \\
\end{longtable}
}

\begin{itemize}
\tightlist
\item
  \textbf{પ્રોસેસિંગ રેટ}: 270.833 kbps કુલ બિટ રેટ
\item
  \textbf{ફ્રેમ સ્ટ્રક્ચર}: TDMA ફ્રેમ દીઠ 8 ટાઈમ સ્લોટ
\item
  \textbf{ફ્રીક્વન્સી હોપિંગ}: પ્રતિ સેકન્ડ 217 હોપ્સ
\end{itemize}

\end{solutionbox}
\begin{mnemonicbox}
``સ્પીચ ચેનલ ઇન્ટરલીવ મોડ્યુલેટેડ રેડિયો''

\end{mnemonicbox}
\subsection*{પ્રશ્ન 3(અ) [3
ગુણ]}\label{uxaaauxab0uxab6uxaa8-3uxa85-3-uxa97uxaa3}

\textbf{સેલ સ્પ્લિટિંગ સમજાવો.}

\begin{solutionbox}
સેલ સ્પ્લિટિંગ ગીચતાવાળા સેલોને નાના સેલોમાં વિભાજિત કરીને ક્ષમતા
વધારે છે.

\begin{itemize}
\tightlist
\item
  \textbf{પ્રક્રિયા}: ઉચ્ચ-પાવર સેલને ઘણા નીચા-પાવર સેલો સાથે બદલવું
\item
  \textbf{ફાયદો}: આવૃત્તિ પુનઃઉપયોગ દ્વારા સિસ્ટમ ક્ષમતા વધારે છે
\item
  \textbf{અમલીકરણ}: એન્ટેનાની ઊંચાઈ અને ટ્રાન્સમિટ પાવર ઘટાડવું
\end{itemize}

\end{solutionbox}
\begin{mnemonicbox}
``સ્પ્લિટ નાના સેલો''

\end{mnemonicbox}
\subsection*{પ્રશ્ન 3(બ) [4
ગુણ]}\label{uxaaauxab0uxab6uxaa8-3uxaac-4-uxa97uxaa3}

\textbf{મોબાઈલ હેન્ડસેટમાં વપરાતી Li-Ion બૅટરી વિશે તેના ફાયદા અને નુકસાનો સાથે
સમજાવો.}

\begin{solutionbox}

{\def\LTcaptype{none} % do not increment counter
\begin{longtable}[]{@{}ll@{}}
\toprule\noalign{}
ફાયદા & નુકસાનો \\
\midrule\noalign{}
\endhead
\bottomrule\noalign{}
\endlastfoot
\textbf{ઉચ્ચ એનર્જી ડેન્સિટી} & \textbf{સુરક્ષાની ચિંતાઓ} \\
\textbf{મેમરી ઇફેક્ટ નથી} & \textbf{સમય સાથે બગાડ} \\
\textbf{નીચું સેલ્ફ-ડિસ્ચાર્જ} & \textbf{તાપમાન સંવેદનશીલ} \\
\textbf{હળવું વજન} & \textbf{મોંઘું} \\
\end{longtable}
}

\begin{itemize}
\tightlist
\item
  \textbf{કેમિસ્ટ્રી}: લિથિયમ આયન ઇલેક્ટ્રોડ વચ્ચે ફરે છે
\item
  \textbf{વોલ્ટેજ}: પ્રતિ સેલ 3.7V નોમિનલ
\item
  \textbf{ક્ષમતા}: mAh (મિલિએમ્પિયર-કલાક) માં માપવામાં આવે છે
\end{itemize}

\end{solutionbox}
\begin{mnemonicbox}
``લાઇટ આયન એનર્જી સેફ્ટી''

\end{mnemonicbox}
\subsection*{પ્રશ્ન 3(ક) [7
ગુણ]}\label{uxaaauxab0uxab6uxaa8-3uxa95-7-uxa97uxaa3}

\textbf{GPRS સમજાવો.}

\begin{solutionbox}
\textbf{GPRS} (General Packet Radio Service) GSM પર
પેકેટ-સ્વિચ્ડ ડેટા સેવા પૂરી પાડે છે.

{\def\LTcaptype{none} % do not increment counter
\begin{longtable}[]{@{}ll@{}}
\toprule\noalign{}
લક્ષણ & સ્પેસિફિકેશન \\
\midrule\noalign{}
\endhead
\bottomrule\noalign{}
\endlastfoot
\textbf{ડેટા રેટ} & 171.2 kbps સુધી \\
\textbf{ટેક્નોલોજી} & પેકેટ સ્વિચિંગ \\
\textbf{ચેનલો} & બહુવિધ ટાઈમ સ્લોટનો ઉપયોગ \\
\textbf{બિલિંગ} & ડેટા વોલ્યુમ પર આધારિત \\
\end{longtable}
}

\begin{center}
\textbf{Mermaid Diagram (Code)}
\begin{verbatim}
{Shaded}
{Highlighting}[]
graph LR
    A[મોબાઈલ] {-{-}{} B[BSS]}
    B {-{-}{} C[PCU]}
    C {-{-}{} D[SGSN]}
    D {-{-}{} E[GGSN]}
    E {-{-}{} F[ઇન્ટરનેટ]}
{Highlighting}
{Shaded}
\end{verbatim}
\end{center}

\begin{itemize}
\tightlist
\item
  \textbf{PCU}: Packet Control Unit પેકેટ ડેટા મેનેજ કરે છે
\item
  \textbf{SGSN}: Serving GPRS Support Node
\item
  \textbf{GGSN}: Gateway GPRS Support Node
\item
  \textbf{ક્લાસ}: વિવિધ સ્પીડ/સ્લોટ કોમ્બિનેશન સાથે ક્લાસ 1-12
\end{itemize}

\end{solutionbox}
\begin{mnemonicbox}
``જનરલ પેકેટ રેડિયો સર્વિસ''

\end{mnemonicbox}
\subsection*{પ્રશ્ન 3(અ અથવા) [3
ગુણ]}\label{uxaaauxab0uxab6uxaa8-3uxa85-uxa85uxaa5uxab5-3-uxa97uxaa3}

\textbf{સેલ સેક્ટરિંગ સમજાવો.}

\begin{solutionbox}
સેલ સેક્ટરિંગ ડાયરેક્શનલ એન્ટેના વાપરીને ઓમ્નિડાયરેક્શનલ સેલને સેક્ટરોમાં
વિભાજિત કરે છે.

\begin{itemize}
\tightlist
\item
  \textbf{સામાન્ય}: 3-સેક્ટર (120^\circ) અથવા 6-સેક્ટર (60^\circ) કોન્ફિગરેશન
\item
  \textbf{ફાયદો}: કો-ચેનલ ઇન્ટરફેરન્સ ઘટાડે છે
\item
  \textbf{અમલીકરણ}: સમાન સાઇટ પર ડાયરેક્શનલ એન્ટેના
\end{itemize}

\end{solutionbox}
\begin{mnemonicbox}
``સેક્ટર ઇન્ટરફેરન્સ ઘટાડે છે''

\end{mnemonicbox}
\subsection*{પ્રશ્ન 3(બ અથવા) [4
ગુણ]}\label{uxaaauxab0uxab6uxaa8-3uxaac-uxa85uxaa5uxab5-4-uxa97uxaa3}

\textbf{મોબાઈલ હેન્ડસેટમાં વપરાતી Li-Po બૅટરી વિશે તેના ફાયદા અને નુકસાનો સાથે
સमજાવો.}

\begin{solutionbox}

{\def\LTcaptype{none} % do not increment counter
\begin{longtable}[]{@{}ll@{}}
\toprule\noalign{}
ફાયદા & નુકસાનો \\
\midrule\noalign{}
\endhead
\bottomrule\noalign{}
\endlastfoot
\textbf{લવચીક આકાર} & \textbf{નીચી એનર્જી ડેન્સિટી} \\
\textbf{અતિ-પાતળી ડિઝાઇન} & \textbf{ઓછું જીવનકાળ} \\
\textbf{હળવું વજન} & \textbf{સુરક્ષા જોખમો} \\
\textbf{મેમરી ઇફેક્ટ નથી} & \textbf{વધુ કિંમત} \\
\end{longtable}
}

\begin{itemize}
\tightlist
\item
  \textbf{ટેક્નોલોજી}: લિથિયમ પોલિમર ઇલેક્ટ્રોલાઇટ
\item
  \textbf{ફોર્મ ફેક્ટર}: વિવિધ આકારોમાં મોલ્ડ કરી શકાય છે
\item
  \textbf{વોલ્ટેજ}: પ્રતિ સેલ 3.7V નોમિનલ
\end{itemize}

\end{solutionbox}
\begin{mnemonicbox}
``પોલિમર લવચીક પાતળું હળવું''

\end{mnemonicbox}
\subsection*{પ્રશ્ન 3(ક અથવા) [7
ગુણ]}\label{uxaaauxab0uxab6uxaa8-3uxa95-uxa85uxaa5uxab5-7-uxa97uxaa3}

\textbf{EDGE સમજાવો.}

\begin{solutionbox}
\textbf{EDGE} (Enhanced Data rates for GSM Evolution) GSM
ડેટા રેટ સુધારે છે.

{\def\LTcaptype{none} % do not increment counter
\begin{longtable}[]{@{}lll@{}}
\toprule\noalign{}
પેરામીટર & GSM & EDGE \\
\midrule\noalign{}
\endhead
\bottomrule\noalign{}
\endlastfoot
\textbf{મોડ્યુલેશન} & GMSK & 8-PSK \\
\textbf{ડેટા રેટ} & 9.6 kbps & 384 kbps સુધી \\
\textbf{ભૂલ સુધારણા} & મૂળભૂત & અદ્યતન \\
\textbf{સ્પેક્ટ્રમ} & GSM જેવું જ & GSM જેવું જ \\
\end{longtable}
}

\begin{center}
\textbf{Mermaid Diagram (Code)}
\begin{verbatim}
{Shaded}
{Highlighting}[]
graph LR
    A[ડેટા] {-{-}{} B[એડાપ્ટીવ કોડિંગ]}
    B {-{-}{} C[8{-}PSK મોડ્યુલેશન]}
    C {-{-}{} D[લિંક એડાપ્ટેશન]}
    D {-{-}{} E[વધારેલ રિસેપ્શન]}
{Highlighting}
{Shaded}
\end{verbatim}
\end{center}

\begin{itemize}
\tightlist
\item
  \textbf{8-PSK}: 8-Phase Shift Keying પ્રતિ સિમ્બોલ 3 બિટ્સ પૂરી પાડે છે
\item
  \textbf{લિંક એડાપ્ટેશન}: ચેનલ ગુણવત્તા આધારે કોડિંગ સ્કીમ એડજસ્ટ કરે છે
\item
  \textbf{ઇન્ક્રિમેન્ટલ રિડન્ડન્સી}: ભૂલ સુધારણા કાર્યક્ષમતા સુધારે છે
\end{itemize}

\end{solutionbox}
\begin{mnemonicbox}
``એન્હાન્સ્ડ ડેટા GSM ઇવોલ્યુશન''

\end{mnemonicbox}
\subsection*{પ્રશ્ન 4(અ) [3
ગુણ]}\label{uxaaauxab0uxab6uxaa8-4uxa85-3-uxa97uxaa3}

\textbf{DSSS ટ્રાન્સમિટર અને રિસીવરની બ્લોક આકૃતિ દોરો.}

\begin{solutionbox}

\begin{verbatim}
Transmitter:
Data {-{-} Spreader {-}{-} Modulator {-}{-} RF Out}
         \^{}
         |
      PN Code

Receiver:
RF In {-{-} Demodulator {-}{-} Despreader {-}{-} Data Out}
                           \^{}
                           |
                        PN Code
\end{verbatim}

\begin{itemize}
\tightlist
\item
  \textbf{સ્પ્રેડર}: ડેટાને PN સિક્વન્સ સાથે ગુણાકાર કરે છે
\item
  \textbf{ડિસ્પ્રેડર}: પ્રાપ્ત સિગ્નલને સમાન PN કોડ સાથે કોરિલેટ કરે છે
\item
  \textbf{પ્રોસેસિંગ ગેઇન}: સ્પ્રેડ અને મૂળ બેન્ડવિડ્થનો ગુણોત્તર
\end{itemize}

\end{solutionbox}
\begin{mnemonicbox}
``ડાયરેક્ટ સિક્વન્સ સ્પ્રેડ સ્પેક્ટ્રમ''

\end{mnemonicbox}
\subsection*{પ્રશ્ન 4(બ) [4
ગુણ]}\label{uxaaauxab0uxab6uxaa8-4uxaac-4-uxa97uxaa3}

\textbf{CDMA અને GSM વચ્ચે તફાવત આપો.}

\begin{solutionbox}

{\def\LTcaptype{none} % do not increment counter
\begin{longtable}[]{@{}lll@{}}
\toprule\noalign{}
પેરામીટર & CDMA & GSM \\
\midrule\noalign{}
\endhead
\bottomrule\noalign{}
\endlastfoot
\textbf{મલ્ટિપલ એક્સેસ} & કોડ ડિવિઝન & ટાઈમ ડિવિઝન \\
\textbf{ક્ષમતા} & વધુ (સોફ્ટ ક્ષમતા) & નિયત ક્ષમતા \\
\textbf{હેન્ડઓફ} & સોફ્ટ હેન્ડઓફ & હાર્ડ હેન્ડઓફ \\
\textbf{પાવર કંટ્રોલ} & મહત્વપૂર્ણ & ઓછું મહત્વપૂર્ણ \\
\textbf{ફ્રીક્વન્સી પ્લાનિંગ} & જરૂરી નથી & જરૂરી \\
\textbf{અવાજની ગુણવત્તા} & વધુ સારી & સારી \\
\end{longtable}
}

\end{solutionbox}
\begin{mnemonicbox}
``કોડ ડિવિઝન વિ ટાઈમ ડિવિઝન''

\end{mnemonicbox}
\subsection*{પ્રશ્ન 4(ક) [7
ગુણ]}\label{uxaaauxab0uxab6uxaa8-4uxa95-7-uxa97uxaa3}

\textbf{સ્પ્રેડ સ્પેક્ટ્રમનો ખ્યાલ તેના ઉપયોગો સાથે સમજાવો.}

\begin{solutionbox}
\textbf{સ્પ્રેડ સ્પેક્ટ્રમ} સિગ્નલની બેન્ડવિડ્થને ડેટા ટ્રાન્સમિશન માટે
જરૂરી કરતાં ઘણી વિશાળ ફેલાવે છે.

\begin{center}
\textbf{Mermaid Diagram (Code)}
\begin{verbatim}
{Shaded}
{Highlighting}[]
graph LR
    A[નેરોબેન્ડ સિગ્નલ] {-{-}{} B[સ્પ્રેડિંગ કોડ]}
    B {-{-}{} C[વાઈડબેન્ડ સિગ્નલ]}
    C {-{-}{} D[ટ્રાન્સમિશન]}
    D {-{-}{} E[ડિસ્પ્રેડિંગ]}
    E {-{-}{} F[મૂળ સિગ્નલ]}
{Highlighting}
{Shaded}
\end{verbatim}
\end{center}

{\def\LTcaptype{none} % do not increment counter
\begin{longtable}[]{@{}lll@{}}
\toprule\noalign{}
પ્રકાર & પદ્ધતિ & એપ્લિકેશન \\
\midrule\noalign{}
\endhead
\bottomrule\noalign{}
\endlastfoot
\textbf{DSSS} & PN સિક્વન્સ ગુણાકાર & CDMA, WiFi \\
\textbf{FHSS} & ફ્રીક્વન્સી હોપિંગ & Bluetooth \\
\textbf{THSS} & ટાઈમ હોપિંગ & UWB સિસ્ટમો \\
\end{longtable}
}

\textbf{ફાયદા}:

\begin{itemize}
\tightlist
\item
  \textbf{એન્ટી-જેમિંગ}: ઇન્ટરફેરન્સ સામે પ્રતિકાર
\item
  \textbf{લો પાવર ડેન્સિટી}: શોધવામાં મુશ્કેલ
\item
  \textbf{મલ્ટિપલ એક્સેસ}: ઘણા વપરાશકર્તાઓ સ્પેક્ટ્રમ શેર કરે છે
\item
  \textbf{મલ્ટિપાથ રેઝિસ્ટન્સ}: વિલંબિત સિગ્નલો રિઝોલ્વ કરે છે
\end{itemize}

\textbf{એપ્લિકેશનો}: GPS, WiFi, Bluetooth, લશ્કરી કમ્યુનિકેશન

\end{solutionbox}
\begin{mnemonicbox}
``સ્પ્રેડ સિગ્નલ સ્પેક્ટ્રમ સિક્યુરિટી''

\end{mnemonicbox}
\subsection*{પ્રશ્ન 4(અ અથવા) [3
ગુણ]}\label{uxaaauxab0uxab6uxaa8-4uxa85-uxa85uxaa5uxab5-3-uxa97uxaa3}

\textbf{FHSS ટ્રાન્સમિટરની બ્લોક આકૃતિ દોરો.}

\begin{solutionbox}

\begin{verbatim}
Data {-{-} Modulator {-}{-} Frequency {-}{-} RF Out}
                       Synthesizer
                           \^{}
                           |
                    Hopping Sequence
                       Generator
\end{verbatim}

\begin{itemize}
\tightlist
\item
  \textbf{ફ્રીક્વન્સી સિન્થેસાઇઝર}: કેરિયર ફ્રીક્વન્સી ઝડપથી બદલે છે
\item
  \textbf{હોપિંગ સિક્વન્સ}: સ્યુડો-રેન્ડમ ફ્રીક્વન્સી પેટર્ન
\item
  \textbf{ડ્વેલ ટાઈમ}: દરેક ફ્રીક્વન્સી પર વિત
\end{itemize}

\end{solutionbox}
\begin{mnemonicbox}
``ફ્રીક્વન્સી હોપિંગ સ્પ્રેડ સ્પેક્ટ્રમ''

\end{mnemonicbox}
\subsection*{પ્રશ્ન 4(બ અથવા) [4
ગુણ]}\label{uxaaauxab0uxab6uxaa8-4uxaac-uxa85uxaa5uxab5-4-uxa97uxaa3}

\textbf{CDMA માં કોલ પ્રોસેસિંગ સમજાવો.}

\begin{solutionbox}

{\def\LTcaptype{none} % do not increment counter
\begin{longtable}[]{@{}lll@{}}
\toprule\noalign{}
તબક્કો & પ્રક્રિયા & વર્ણન \\
\midrule\noalign{}
\endhead
\bottomrule\noalign{}
\endlastfoot
\textbf{સિસ્ટમ એક્સેસ} & પાવર કંટ્રોલ & મોબાઈલ પાવર એડજસ્ટ કરે છે \\
\textbf{કોલ સેટઅપ} & ચેનલ અસાઈનમેન્ટ & વોલ્શ કોડ અસાઈન કરો \\
\textbf{ટ્રાફિક} & સોફ્ટ હેન્ડઓફ & બહુવિધ બેઝ સ્ટેશનો \\
\textbf{કોલ રિલીઝ} & પાવર ડાઉન & ક્રમશઃ પાવર ઘટાડો \\
\end{longtable}
}

\begin{itemize}
\tightlist
\item
  \textbf{રેક રિસીવર}: મલ્ટિપાથ સિગ્નલો કમ્બાઇન કરે છે
\item
  \textbf{પાવર કંટ્રોલ}: પ્રતિ સેકન્ડ 800 વખત
\item
  \textbf{સોફ્ટ કેપેસિટી}: લોડ સાથે ક્રમશઃ બગડે છે
\end{itemize}

\end{solutionbox}
\begin{mnemonicbox}
``કોડ ડિવિઝન મલ્ટિપલ એક્સેસ''

\end{mnemonicbox}
\subsection*{પ્રશ્ન 4(ક અથવા) [7
ગુણ]}\label{uxaaauxab0uxab6uxaa8-4uxa95-uxa85uxaa5uxab5-7-uxa97uxaa3}

\textbf{HSDPA સમજાવો.}

\begin{solutionbox}
\textbf{HSDPA} (High Speed Downlink Packet Access) WCDMA
ડાઉનલિંક ડેટા રેટ વધારે છે.

{\def\LTcaptype{none} % do not increment counter
\begin{longtable}[]{@{}ll@{}}
\toprule\noalign{}
લક્ષણ & સુધારો \\
\midrule\noalign{}
\endhead
\bottomrule\noalign{}
\endlastfoot
\textbf{ડેટા રેટ} & 14.4 Mbps સુધી \\
\textbf{મોડ્યુલેશન} & 16-QAM \\
\textbf{HARQ} & હાઇબ્રિડ ARQ \\
\textbf{ફાસ્ટ શેડ્યુલિંગ} & 2ms TTI \\
\end{longtable}
}

\begin{center}
\textbf{Mermaid Diagram (Code)}
\begin{verbatim}
{Shaded}
{Highlighting}[]
graph LR
    A[NodeB] {-{-}{} B[HS{-}DSCH]}
    B {-{-}{} C[16{-}QAM]}
    C {-{-}{} D[HARQ]}
    D {-{-}{} E[મોબાઈલ]}
{Highlighting}
{Shaded}
\end{verbatim}
\end{center}

\begin{itemize}
\tightlist
\item
  \textbf{HS-DSCH}: High Speed Downlink Shared Channel
\item
  \textbf{AMC}: Adaptive Modulation and Coding
\item
  \textbf{ફાસ્ટ સેલ સિલેક્શન}: સેલ એજ પર્ફોર્મન્સ સુધારે છે
\item
  \textbf{MIMO}: બહુવિધ એન્ટેના કોન્ફિગરેશન શક્ય
\end{itemize}

\end{solutionbox}
\begin{mnemonicbox}
``હાઇ સ્પીડ ડાઉનલિંક પેકેટ એક્સેસ''

\end{mnemonicbox}
\subsection*{પ્રશ્ન 5(અ) [3
ગુણ]}\label{uxaaauxab0uxab6uxaa8-5uxa85-3-uxa97uxaa3}

\textbf{LTE ના સ્પેસિફિકેશન જણાવો.}

\begin{solutionbox}

{\def\LTcaptype{none} % do not increment counter
\begin{longtable}[]{@{}ll@{}}
\toprule\noalign{}
પેરામીટર & સ્પેસિફિકેશન \\
\midrule\noalign{}
\endhead
\bottomrule\noalign{}
\endlastfoot
\textbf{પીક ડેટા રેટ} & 300 Mbps DL, 75 Mbps UL \\
\textbf{બેન્ડવિડ્થ} & 1.4 થી 20 MHz \\
\textbf{લેટન્સી} & \textless10ms યુઝર પ્લેન \\
\textbf{મોબિલિટી} & 350 km/h સુધી \\
\textbf{સ્પેક્ટ્રમ કાર્યક્ષમતા} & 3G કરતાં 3-4x વધારે સારી \\
\end{longtable}
}

\end{solutionbox}
\begin{mnemonicbox}
``લોંગ ટર્મ ઇવોલ્યુશન સ્પેસિફિકેશનો''

\end{mnemonicbox}
\subsection*{પ્રશ્ન 5(બ) [4
ગુણ]}\label{uxaaauxab0uxab6uxaa8-5uxaac-4-uxa97uxaa3}

\textbf{OFDM રિસીવર બ્લોક આકૃતિ દોરી સમજાવો.}

\begin{solutionbox}

\begin{center}
\textbf{Mermaid Diagram (Code)}
\begin{verbatim}
{Shaded}
{Highlighting}[]
graph LR
    A[RF ઇનપુટ] {-{-}{} B[ADC]}
    B {-{-}{} C[CP દૂર કરો]}
    C {-{-}{} D[FFT]}
    D {-{-}{} E[ડિમોડ્યુલેટર]}
    E {-{-}{} F[ડેટા આઉટપુટ]}
{Highlighting}
{Shaded}
\end{verbatim}
\end{center}

\begin{itemize}
\tightlist
\item
  \textbf{FFT}: Fast Fourier Transform સમય ડોમેઇનને ફ્રીક્વન્સી ડોમેઇનમાં કન્વર્ટ
  કરે છે
\item
  \textbf{સાયક્લિક પ્રીફિક્સ}: ઇન્ટર-સિમ્બોલ ઇન્ટરફેરન્સ સામે રક્ષણ કરે છે
\item
  \textbf{સબકેરિયર્સ}: બહુવિધ ફ્રીક્વન્સીઓ પર સમાંતર ટ્રાન્સમિશન
\item
  \textbf{ડિમોડ્યુલેશન}: સબકેરિયર દીઠ QPSK/16QAM/64QAM
\end{itemize}

\end{solutionbox}
\begin{mnemonicbox}
``ઓર્થોગોનલ ફ્રીક્વન્સી ડિવિઝન મલ્ટિપ્લેક્સિંગ''

\end{mnemonicbox}
\subsection*{પ્રશ્ન 5(ક) [7
ગુણ]}\label{uxaaauxab0uxab6uxaa8-5uxa95-7-uxa97uxaa3}

\textbf{બ્લુટૂથ ટેક્નોલોજી તેના ઉપયોગો સાથે સમજાવો.}

\begin{solutionbox}
\textbf{બ્લુટૂથ} પર્સનલ એરિયા નેટવર્ક માટે ટૂંકી રેન્જની વાયરલેસ
કમ્યુનિકેશન ટેક્નોલોજી છે.

{\def\LTcaptype{none} % do not increment counter
\begin{longtable}[]{@{}ll@{}}
\toprule\noalign{}
પેરામીટર & સ્પેસિફિકેશન \\
\midrule\noalign{}
\endhead
\bottomrule\noalign{}
\endlastfoot
\textbf{રેન્જ} & 10m (ક્લાસ 2) \\
\textbf{ફ્રીક્વન્સી} & 2.4 GHz ISM બેન્ડ \\
\textbf{ડેટા રેટ} & 3 Mbps સુધી \\
\textbf{ટોપોલોજી} & પિકોનેટ (8 ડિવાઇસો) \\
\end{longtable}
}

\begin{center}
\textbf{Mermaid Diagram (Code)}
\begin{verbatim}
{Shaded}
{Highlighting}[]
graph TD
    A[માસ્ટર ડિવાઇસ] {-{-}{} B[સ્લેવ 1]}
    A {-{-}{} C[સ્લેવ 2]}
    A {-{-}{} D[સ્લેવ 3]}
    E[સ્કેટરનેટ] {-{-}{} A}
    E {-{-}{} F[બીજું પિકોનેટ]}
{Highlighting}
{Shaded}
\end{verbatim}
\end{center}

\textbf{પ્રોટોકોલ સ્ટેક}:

\begin{itemize}
\tightlist
\item
  \textbf{RF લેયર}: ફિઝિકલ રેડિયો ઇન્ટરફેસ
\item
  \textbf{બેઝબેન્ડ}: મીડિયમ એક્સેસ કંટ્રોલ
\item
  \textbf{L2CAP}: લોજિકલ લિંક કંટ્રોલ
\item
  \textbf{એપ્લિકેશનો}: વિવિધ પ્રોફાઇલ્સ (A2DP, HID, વગેરે)
\end{itemize}

\textbf{ઉપયોગો}:

\begin{itemize}
\tightlist
\item
  ઓડિયો ડિવાઇસો (હેડફોન્સ, સ્પીકર્સ)
\item
  ડિવાઇસો વચ્ચે ફાઇલ ટ્રાન્સફર
\item
  ઇનપુટ ડિવાઇસો (કીબોર્ડ, માઉસ)
\item
  હેલ્થ મોનિટરિંગ ડિવાઇસો
\item
  સ્માર્ટ હોમ ઓટોમેશન
\end{itemize}

\end{solutionbox}
\begin{mnemonicbox}
``બ્લુ ટૂથ પર્સનલ એરિયા નેટવર્ક''

\end{mnemonicbox}
\subsection*{પ્રશ્ન 5(અ અથવા) [3
ગુણ]}\label{uxaaauxab0uxab6uxaa8-5uxa85-uxa85uxaa5uxab5-3-uxa97uxaa3}

\textbf{5G ટેક્નોલોજીના ફાયદા જણાવો.}

\begin{solutionbox}

{\def\LTcaptype{none} % do not increment counter
\begin{longtable}[]{@{}ll@{}}
\toprule\noalign{}
ફાયદો & લાભ \\
\midrule\noalign{}
\endhead
\bottomrule\noalign{}
\endlastfoot
\textbf{અલ્ટ્રા-લો લેટન્સી} & \textless1ms પ્રતિક્રિયા સમય \\
\textbf{ઉચ્ચ ડેટા રેટ} & 10 Gbps સુધી \\
\textbf{મેસિવ કનેક્ટિવિટી} & 1M ડિવાઇસો/km^{2} \\
\textbf{નેટવર્ક સ્લાઇસિંગ} & કસ્ટમાઇઝ્ડ સેવાઓ \\
\textbf{એનર્જી કાર્યક્ષમતા} & 90\% વધુ કાર્યક્ષમ \\
\end{longtable}
}

\end{solutionbox}
\begin{mnemonicbox}
``પાંચમી જનરેશનના ફાયદા''

\end{mnemonicbox}
\subsection*{પ્રશ્ન 5(બ અથવા) [4
ગુણ]}\label{uxaaauxab0uxab6uxaa8-5uxaac-uxa85uxaa5uxab5-4-uxa97uxaa3}

\textbf{OFDM ટ્રાન્સમિટર બ્લોક આકૃતિ દોરી સમજાવો.}

\begin{solutionbox}

\begin{center}
\textbf{Mermaid Diagram (Code)}
\begin{verbatim}
{Shaded}
{Highlighting}[]
graph LR
    A[ડેટા ઇનપુટ] {-{-}{} B[મોડ્યુલેટર]}
    B {-{-}{} C[IFFT]}
    C {-{-}{} D[CP ઉમેરો]}
    D {-{-}{} E[DAC]}
    E {-{-}{} F[RF આઉટપુટ]}
{Highlighting}
{Shaded}
\end{verbatim}
\end{center}

\begin{itemize}
\tightlist
\item
  \textbf{મોડ્યુલેશન}: બિટ્સને સિમ્બોલ્સમાં મેપ કરે છે (QPSK/QAM)
\item
  \textbf{IFFT}: ઇન્વર્સ FFT ફ્રીક્વન્સીને ટાઈમ ડોમેઇનમાં કન્વર્ટ કરે છે
\item
  \textbf{સાયક્લિક પ્રીફિક્સ}: છેવટના સેમ્પલ્સને શરૂઆતમાં કૉપિ કરે છે
\item
  \textbf{DAC}: ટ્રાન્સમિશન માટે ડિજિટલ ટુ એનાલોગ કન્વર્ટર
\end{itemize}

\end{solutionbox}
\begin{mnemonicbox}
``ઓર્થોગોનલ ફ્રીક્વન્સી ડિવિઝન મલ્ટિપ્લેક્સિંગ ટ્રાન્સમિટર''

\end{mnemonicbox}
\subsection*{પ્રશ્ન 5(ક અથવા) [7
ગુણ]}\label{uxaaauxab0uxab6uxaa8-5uxa95-uxa85uxaa5uxab5-7-uxa97uxaa3}

\textbf{Zigbee ટેક્નોલોજી તેના ઉપયોગો સાથે સમજાવો.}

\begin{solutionbox}
\textbf{Zigbee} IEEE 802.15.4 પર આધારિત લો-પાવર વાયરલેસ મેશ
નેટવર્કિંગ સ્ટાન્ડર્ડ છે.

{\def\LTcaptype{none} % do not increment counter
\begin{longtable}[]{@{}ll@{}}
\toprule\noalign{}
પેરામીટર & સ્પેસિફિકેશન \\
\midrule\noalign{}
\endhead
\bottomrule\noalign{}
\endlastfoot
\textbf{રેન્જ} & 10-100m \\
\textbf{ડેટા રેટ} & 250 kbps \\
\textbf{પાવર} & ખૂબ નીચું (બૅટરી વર્ષો) \\
\textbf{ટોપોલોજી} & મેશ નેટવર્ક \\
\textbf{ફ્રીક્વન્સી} & વૈશ્વિક રીતે 2.4 GHz \\
\end{longtable}
}

\begin{center}
\textbf{Mermaid Diagram (Code)}
\begin{verbatim}
{Shaded}
{Highlighting}[]
graph TD
    A[કોઓર્ડિનેટર] {-{-}{} B[રાઉટર 1]}
    A {-{-}{} C[રાઉટર 2]}
    B {-{-}{} D[એન્ડ ડિવાઇસ 1]}
    B {-{-}{} E[એન્ડ ડિવાઇસ 2]}
    C {-{-}{} F[એન્ડ ડિવાઇસ 3]}
    C {-{-}{} G[રાઉટર 3]}
    G {-{-}{} H[એન્ડ ડિવાઇસ 4]}
{Highlighting}
{Shaded}
\end{verbatim}
\end{center}

\textbf{નેટવર્ક રોલ્સ}:

\begin{itemize}
\tightlist
\item
  \textbf{કોઓર્ડિનેટર}: નેટવર્ક મેનેજર
\item
  \textbf{રાઉટર}: મેસેજ ફોરવર્ડ કરે છે
\item
  \textbf{એન્ડ ડિવાઇસ}: સાદા સેન્સર્સ/એક્ચ્યુએટર્સ
\end{itemize}

\textbf{ઉપયોગો}:

\begin{itemize}
\tightlist
\item
  હોમ ઓટોમેશન (લાઇટ્સ, થર્મોસ્ટેટ્સ)
\item
  ઇન્ડસ્ટ્રિયલ મોનિટરિંગ
\item
  સ્માર્ટ ગ્રિડ સિસ્ટમો
\item
  હેલ્થકેર મોનિટરિંગ
\item
  કૃષિ સેન્સર્સ
\item
  બિલ્ડિંગ મેનેજમેન્ટ સિસ્ટમો
\end{itemize}

\textbf{લક્ષણો}:

\begin{itemize}
\tightlist
\item
  \textbf{સેલ્ફ-હીલિંગ}: ઓટોમેટિક રૂટ ડિસ્કવરી
\item
  \textbf{ઓછી કિંમત}: સાદો અમલીકરણ
\item
  \textbf{સુરક્ષિત}: AES એન્ક્રિપ્શન
\item
  \textbf{વિશ્વસનીય}: મેશ રિડન્ડન્સી
\end{itemize}

\end{solutionbox}
\begin{mnemonicbox}
``Zigbee મેશ નેટવર્ક એપ્લિકેશનો''

\end{mnemonicbox}

\end{document}
