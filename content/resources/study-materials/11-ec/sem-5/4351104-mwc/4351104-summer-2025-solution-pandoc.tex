\documentclass[10pt,a4paper]{article}

% content/resources/templates/preamble.tex
\usepackage[margin=0.6in]{geometry}
\author{Milav Dabgar}
\usepackage{amsmath,amssymb,amsthm}
\usepackage{booktabs}
\usepackage{multirow}
\usepackage{xcolor}
\usepackage{tcolorbox}
\tcbuselibrary{breakable,skins}
\usepackage[colorlinks=true,linkcolor=blue]{hyperref}
\usepackage{titlesec}
\usepackage{enumitem}
\usepackage{tikz}
\usepackage{pgfplots}
\usepackage{circuitikz}
\usepackage[version=4]{mhchem}
\usepackage{longtable}
\usepackage{array}
\usepackage{float}
\usepackage{caption}
\usepackage{listings}

\lstset{
  basicstyle=\small\ttfamily,
  breaklines=true,
  breakatwhitespace=false,
  postbreak=\mbox{\textcolor{red}{$\hookrightarrow$}\space},
  float=false,
  numbers=left,
  numberstyle=\tiny\color{gray},
  numbersep=10pt,
  xleftmargin=2em,
  keywordstyle=\color{blue},
  commentstyle=\color{green!60!black},
  stringstyle=\color{purple},
  backgroundcolor=\color{gray!5},
  showstringspaces=false,
  tabsize=2,
  captionpos=b,
  keepspaces=true,
  columns=flexible
}

\pgfplotsset{compat=1.18}
\usetikzlibrary{shapes,arrows,positioning,calc,patterns,decorations.pathmorphing,decorations.markings,arrows.meta}

% Color scheme
\definecolor{headcolor}{RGB}{0,102,204}
\definecolor{keycolor}{RGB}{220,20,60}
\definecolor{solutioncolor}{RGB}{34,139,34}
\definecolor{mnemoniccolor}{RGB}{148,0,211}
\definecolor{codecolor}{RGB}{0,0,100}

% Spacing
\setlength{\parskip}{3pt}
\setlist[itemize]{nosep}
\setlist[enumerate]{nosep}

% Title formatting
\titleformat{\section}{\Large\bfseries\color{headcolor}}{\thesection}{1em}{}
\titleformat{\subsection}{\large\bfseries\color{headcolor}}{\thesubsection}{1em}{}

% Pandoc tightlist compatibility
\providecommand{\tightlist}{%
  \setlength{\itemsep}{0pt}\setlength{\parskip}{0pt}}

% Pandoc longtable compatibility
\newcounter{none}
\def\thenone{}


% content/resources/templates/english-boxes.tex
% This file is currently empty - it exists to maintain consistency with the import structure.
% Add custom environments here if needed in the future.


\begin{document}

\begin{center}
{\Huge\bfseries\color{headcolor} Subject Name Solutions}\\[5pt]
{\LARGE 4351104 -- Summer 2025}\\[3pt]
{\large Semester 1 Study Material}\\[3pt]
{\normalsize\textit{Detailed Solutions and Explanations}}
\end{center}

\vspace{10pt}

\subsection*{Question 1(a) [3 marks]}\label{q1a}

\textbf{Write key features of 4G and 5G system.}

\begin{solutionbox}


{\def\LTcaptype{none} % do not increment counter
\vspace{-5pt}
\captionof{table}{Key Features Comparison}
\vspace{-10pt}
\begin{longtable}[]{@{}lll@{}}
\toprule\noalign{}
Feature & 4G System & 5G System \\
\midrule\noalign{}
\endhead
\bottomrule\noalign{}
\endlastfoot
\textbf{Data Speed} & Up to 100 Mbps & Up to 10 Gbps \\
\textbf{Latency} & 30-50 ms & 1-10 ms \\
\textbf{Technology} & LTE, OFDM & MIMO, Beamforming \\
\textbf{Applications} & Video streaming & IoT, AR/VR \\
\end{longtable}
}

\textbf{Key Points:}

\begin{itemize}
\tightlist
\item
  \textbf{4G}: Uses LTE technology with OFDM modulation for high-speed
  data
\item
  \textbf{5G}: Ultra-low latency enables real-time applications like
  autonomous vehicles
\item
  \textbf{Network Slicing}: 5G allows virtual networks for specific
  applications
\end{itemize}

\end{solutionbox}
\begin{mnemonicbox}
``4G Fast, 5G Super-Fast''

\end{mnemonicbox}
\begin{center}\rule{0.5\linewidth}{0.5pt}\end{center}

\subsection*{Question 1(b) [4 marks]}\label{q1b}

\textbf{Explain concept of frequency reuse in cellular mobile system.}

\begin{solutionbox}

\textbf{Diagram:}

\begin{verbatim}
    F1      F2      F3
   +{-{-}{-}+   +{-}{-}{-}+   +{-}{-}{-}+}
   | A |   | B |   | C |
   +{-{-}{-}+   +{-}{-}{-}+   +{-}{-}{-}+}
    F4      F5      F6
   +{-{-}{-}+   +{-}{-}{-}+   +{-}{-}{-}+}
   | D |   | E |   | F |
   +{-{-}{-}+   +{-}{-}{-}+   +{-}{-}{-}+}
    F7      F1      F2
   +{-{-}{-}+   +{-}{-}{-}+   +{-}{-}{-}+}
   | G |   | A |   | B |
   +{-{-}{-}+   +{-}{-}{-}+   +{-}{-}{-}+}
\end{verbatim}

\textbf{Key Points:}

\begin{itemize}
\tightlist
\item
  \textbf{Frequency Reuse}: Same frequencies used in non-adjacent cells
  to increase capacity
\item
  \textbf{Co-channel Distance}: Minimum distance between cells using
  same frequency
\item
  \textbf{Cluster Size}: Group of cells using different frequencies
  (typically 3, 4, 7, 12)
\item
  \textbf{Capacity Improvement}: More users served with limited spectrum
\end{itemize}

\end{solutionbox}
\begin{mnemonicbox}
``Same Frequency, Different Places''

\end{mnemonicbox}
\begin{center}\rule{0.5\linewidth}{0.5pt}\end{center}

\subsection*{Question 1(c) [7 marks]}\label{q1c}

\textbf{If a total of 33 MHz of bandwidth is allocated to a particular
FDD cellular telephone system which uses two 25 kHz simplex channels to
provide full duplex communication. If 1 MHz of the allocated spectrum is
dedicated to control channels, determine an equitable distribution of
control channels and voice channels for cluster size of 7.}

\begin{solutionbox}

\textbf{Given Data:}

\begin{itemize}
\tightlist
\item
  Total bandwidth = 33 MHz
\item
  Channel bandwidth = 25 kHz (simplex)
\item
  Control spectrum = 1 MHz
\item
  Cluster size = 7
\end{itemize}

\textbf{Calculations:}

\textbf{Step 1: Available spectrum for traffic} Traffic spectrum = 33 -
1 = 32 MHz

\textbf{Step 2: Total duplex channels} Each duplex channel needs 2 \times 25
kHz = 50 kHz Total channels = 32 MHz \div 50 kHz = 640 channels

\textbf{Step 3: Control channels} Control channels = 1 MHz \div 25 kHz = 40
channels

\textbf{Step 4: Distribution per cell}

\begin{itemize}
\tightlist
\item
  Voice channels per cell = 640 \div 7 \approx 91 channels
\item
  Control channels per cell = 40 \div 7 \approx 6 channels
\end{itemize}

\textbf{Final Distribution Table:}

{\def\LTcaptype{none} % do not increment counter
\begin{longtable}[]{@{}lll@{}}
\toprule\noalign{}
Parameter & Total & Per Cell \\
\midrule\noalign{}
\endhead
\bottomrule\noalign{}
\endlastfoot
\textbf{Voice Channels} & 640 & 91 \\
\textbf{Control Channels} & 40 & 6 \\
\textbf{Total Channels} & 680 & 97 \\
\end{longtable}
}

\end{solutionbox}
\begin{mnemonicbox}
``Divide Total by Cluster''

\end{mnemonicbox}
\begin{center}\rule{0.5\linewidth}{0.5pt}\end{center}

\subsection*{Question 1(c OR) [7
marks]}\label{question-1c-or-7-marks}

\textbf{List out types of cells and explain each.}

\begin{solutionbox}


{\def\LTcaptype{none} % do not increment counter
\vspace{-5pt}
\captionof{table}{Types of Cells}
\vspace{-10pt}
\begin{longtable}[]{@{}llll@{}}
\toprule\noalign{}
Cell Type & Coverage & Power & Applications \\
\midrule\noalign{}
\endhead
\bottomrule\noalign{}
\endlastfoot
\textbf{Macro Cell} & 1-30 km & High & Rural areas \\
\textbf{Micro Cell} & 100m-1km & Medium & Urban areas \\
\textbf{Pico Cell} & 10-100m & Low & Buildings \\
\textbf{Femto Cell} & 10-50m & Very Low & Homes \\
\end{longtable}
}

\textbf{Detailed Explanation:}

\textbf{Macro Cells:}

\begin{itemize}
\tightlist
\item
  \textbf{Coverage}: Large geographical areas (1-30 km radius)
\item
  \textbf{Power}: High transmission power (up to 40W)
\item
  \textbf{Usage}: Rural and suburban areas with low user density
\end{itemize}

\textbf{Micro Cells:}

\begin{itemize}
\tightlist
\item
  \textbf{Coverage}: Medium areas (100m to 1km radius)\\
\item
  \textbf{Power}: Medium transmission power (1-10W)
\item
  \textbf{Usage}: Urban areas, highway coverage
\end{itemize}

\textbf{Pico Cells:}

\begin{itemize}
\tightlist
\item
  \textbf{Coverage}: Small indoor/outdoor areas (10-100m)
\item
  \textbf{Power}: Low transmission power (100mW-1W)
\item
  \textbf{Usage}: Shopping malls, airports, offices
\end{itemize}

\textbf{Umbrella Cells:}

\begin{itemize}
\tightlist
\item
  \textbf{Special Type}: Covers multiple smaller cells
\item
  \textbf{Purpose}: Handles high-speed mobile users
\item
  \textbf{Advantage}: Reduces handoffs for fast-moving users
\end{itemize}

\end{solutionbox}
\begin{mnemonicbox}
``Macro-Micro-Pico-Femto = Big to Small''

\end{mnemonicbox}
\begin{center}\rule{0.5\linewidth}{0.5pt}\end{center}

\subsection*{Question 2(a) [3 marks]}\label{q2a}

\textbf{Define cell and cluster.}

\begin{solutionbox}

\textbf{Definitions:}

\textbf{Cell:}

\begin{itemize}
\tightlist
\item
  \textbf{Definition}: Geographical area covered by one base station
\item
  \textbf{Shape}: Typically hexagonal for planning purposes
\item
  \textbf{Function}: Serves mobile users within its coverage area
\end{itemize}

\textbf{Cluster:}

\begin{itemize}
\tightlist
\item
  \textbf{Definition}: Group of cells using different frequency sets
\item
  \textbf{Purpose}: Enables frequency reuse pattern
\item
  \textbf{Common Sizes}: 3, 4, 7, 12 cells per cluster
\end{itemize}


{\def\LTcaptype{none} % do not increment counter
\vspace{-5pt}
\captionof{table}{Cell vs Cluster}
\vspace{-10pt}
\begin{longtable}[]{@{}lll@{}}
\toprule\noalign{}
Parameter & Cell & Cluster \\
\midrule\noalign{}
\endhead
\bottomrule\noalign{}
\endlastfoot
\textbf{Unit} & Single coverage area & Group of cells \\
\textbf{Frequency} & One frequency set & Multiple frequency sets \\
\textbf{Reuse} & Cannot reuse nearby & Enables frequency reuse \\
\end{longtable}
}

\end{solutionbox}
\begin{mnemonicbox}
``Cell = One Area, Cluster = Group Areas''

\end{mnemonicbox}
\begin{center}\rule{0.5\linewidth}{0.5pt}\end{center}

\subsection*{Question 2(b) [4 marks]}\label{q2b}

\textbf{Explain effect of cluster size on capacity and interference.}

\begin{solutionbox}

\textbf{Effects Table:}

{\def\LTcaptype{none} % do not increment counter
\begin{longtable}[]{@{}llll@{}}
\toprule\noalign{}
Cluster Size & Capacity & Interference & Co-channel Distance \\
\midrule\noalign{}
\endhead
\bottomrule\noalign{}
\endlastfoot
\textbf{Small (3,4)} & High & High & Short \\
\textbf{Large (7,12)} & Low & Low & Long \\
\end{longtable}
}

\textbf{Key Effects:}

\textbf{On Capacity:}

\begin{itemize}
\tightlist
\item
  \textbf{Smaller Cluster}: More channels per cell, higher capacity
\item
  \textbf{Larger Cluster}: Fewer channels per cell, lower capacity
\item
  \textbf{Formula}: Channels per cell = Total channels \div Cluster size
\end{itemize}

\textbf{On Interference:}

\begin{itemize}
\tightlist
\item
  \textbf{Smaller Cluster}: Higher co-channel interference
\item
  \textbf{Larger Cluster}: Lower co-channel interference
\item
  \textbf{Trade-off}: Capacity vs.~Quality
\end{itemize}

\textbf{Co-channel Distance:}

\begin{itemize}
\tightlist
\item
  \textbf{Relationship}: D = R\sqrt(3N) where N = cluster size
\item
  \textbf{Effect}: Larger N means larger distance between co-channel
  cells
\end{itemize}

\end{solutionbox}
\begin{mnemonicbox}
``Small Cluster = More Capacity, More Interference''

\end{mnemonicbox}
\begin{center}\rule{0.5\linewidth}{0.5pt}\end{center}

\subsection*{Question 2(c) [7 marks]}\label{q2c}

\textbf{Write key features of IS-95, CDMA2000 and WCDMA.}

\begin{solutionbox}

\textbf{Comparison Table:}

{\def\LTcaptype{none} % do not increment counter
\begin{longtable}[]{@{}llll@{}}
\toprule\noalign{}
Feature & IS-95 & CDMA2000 & WCDMA \\
\midrule\noalign{}
\endhead
\bottomrule\noalign{}
\endlastfoot
\textbf{Generation} & 2G & 3G & 3G \\
\textbf{Data Rate} & 14.4 kbps & 2 Mbps & 2 Mbps \\
\textbf{Chip Rate} & 1.2288 Mcps & 3.6864 Mcps & 3.84 Mcps \\
\textbf{Bandwidth} & 1.25 MHz & 1.25 MHz & 5 MHz \\
\end{longtable}
}

\textbf{IS-95 Features:}

\begin{itemize}
\tightlist
\item
  \textbf{Technology}: First commercial CDMA system
\item
  \textbf{Voice Quality}: Better than GSM in some conditions
\item
  \textbf{Soft Handoff}: Maintains multiple connections during handoff
\item
  \textbf{Power Control}: Precise power control reduces interference
\end{itemize}

\textbf{CDMA2000 Features:}

\begin{itemize}
\tightlist
\item
  \textbf{Backward Compatibility}: Works with IS-95 networks
\item
  \textbf{High Data Rate}: Up to 2 Mbps for 1xEV-DO
\item
  \textbf{Multimedia}: Supports voice, data, and video
\item
  \textbf{Efficiency}: Better spectrum efficiency than IS-95
\end{itemize}

\textbf{WCDMA Features:}

\begin{itemize}
\tightlist
\item
  \textbf{Global Standard}: Used worldwide for 3G
\item
  \textbf{High Capacity}: Supports more simultaneous users
\item
  \textbf{QoS Support}: Different service classes for applications
\item
  \textbf{International Roaming}: Global compatibility
\end{itemize}

\end{solutionbox}
\begin{mnemonicbox}
``IS-95 First, CDMA2000 Faster, WCDMA Global''

\end{mnemonicbox}
\begin{center}\rule{0.5\linewidth}{0.5pt}\end{center}

\subsection*{Question 2(a OR) [3
marks]}\label{question-2a-or-3-marks}

\textbf{Explain cell splitting.}

\begin{solutionbox}

\textbf{Definition:} Cell splitting is a technique to increase system
capacity by subdividing congested cells into smaller cells.

\begin{center}
\textbf{Mermaid Diagram (Code)}
\begin{verbatim}
{Shaded}
{Highlighting}[]
graph TD
    A[Original Large Cell] {-{-}{} B[Split into 4 Smaller Cells]}
    B {-{-}{} C[Cell 1]}
    B {-{-}{} D[Cell 2]}
    B {-{-}{} E[Cell 3]}
    B {-{-}{} F[Cell 4]}
{Highlighting}
{Shaded}
\end{verbatim}
\end{center}

\textbf{Process:}

\begin{itemize}
\tightlist
\item
  \textbf{Step 1}: Identify congested cell with high traffic
\item
  \textbf{Step 2}: Install new base stations with lower power
\item
  \textbf{Step 3}: Reduce original base station power
\item
  \textbf{Step 4}: Create multiple smaller coverage areas
\end{itemize}

\textbf{Benefits:}

\begin{itemize}
\tightlist
\item
  \textbf{Capacity Increase}: More channels available in same area
\item
  \textbf{Better Signal Quality}: Shorter distances improve signal
  strength
\end{itemize}

\end{solutionbox}
\begin{mnemonicbox}
``Split Big Cell into Small Cells''

\end{mnemonicbox}
\begin{center}\rule{0.5\linewidth}{0.5pt}\end{center}

\subsection*{Question 2(b OR) [4
marks]}\label{question-2b-or-4-marks}

\textbf{Write functions of HLR and VLR in GSM.}

\begin{solutionbox}

\textbf{Functions Table:}

{\def\LTcaptype{none} % do not increment counter
\begin{longtable}[]{@{}lll@{}}
\toprule\noalign{}
Database & Full Form & Primary Functions \\
\midrule\noalign{}
\endhead
\bottomrule\noalign{}
\endlastfoot
\textbf{HLR} & Home Location Register & Permanent subscriber data \\
\textbf{VLR} & Visitor Location Register & Temporary visitor data \\
\end{longtable}
}

\textbf{HLR Functions:}

\begin{itemize}
\tightlist
\item
  \textbf{Subscriber Profile}: Stores permanent subscriber information
  (IMSI, services)
\item
  \textbf{Location Tracking}: Maintains current location area of
  subscriber
\item
  \textbf{Authentication}: Provides authentication keys for security
\item
  \textbf{Service Management}: Controls subscribed services and
  restrictions
\end{itemize}

\textbf{VLR Functions:}

\begin{itemize}
\tightlist
\item
  \textbf{Temporary Storage}: Stores visiting subscriber data
  temporarily
\item
  \textbf{Local Services}: Enables services for roaming subscribers\\
\item
  \textbf{Call Routing}: Assists in routing calls to visiting
  subscribers
\item
  \textbf{Authentication Copy}: Maintains copy of authentication data
  from HLR
\end{itemize}

\textbf{Interaction:}

\begin{itemize}
\tightlist
\item
  HLR updates VLR when subscriber roams to new area
\item
  VLR requests subscriber data from HLR during registration
\end{itemize}

\end{solutionbox}
\begin{mnemonicbox}
``HLR = Home Data, VLR = Visitor Data''

\end{mnemonicbox}
\begin{center}\rule{0.5\linewidth}{0.5pt}\end{center}

\subsection*{Question 2(c OR) [7
marks]}\label{question-2c-or-7-marks}

\textbf{Describe RFID technology.}

\begin{solutionbox}

\textbf{RFID Overview:} Radio Frequency Identification uses
electromagnetic fields to identify and track tags attached to objects.

\textbf{System Components:}

\begin{center}
\textbf{Mermaid Diagram (Code)}
\begin{verbatim}
{Shaded}
{Highlighting}[]
graph LR
    A[RFID Reader] {-{-}{} B[Radio Waves]}
    B {-{-}{} C[RFID Tag]}
    C {-{-}{} D[Stored Data]}
    C {-{-}{} B}
    B {-{-}{} A}
{Highlighting}
{Shaded}
\end{verbatim}
\end{center}

\textbf{Types Table:}

{\def\LTcaptype{none} % do not increment counter
\begin{longtable}[]{@{}llll@{}}
\toprule\noalign{}
Type & Power Source & Range & Applications \\
\midrule\noalign{}
\endhead
\bottomrule\noalign{}
\endlastfoot
\textbf{Passive} & Reader's energy & 0.1-10m & Access cards \\
\textbf{Active} & Internal battery & 10-100m & Vehicle tracking \\
\textbf{Semi-passive} & Battery + Reader & 1-30m & Temperature
sensors \\
\end{longtable}
}

\textbf{Key Features:}

\begin{itemize}
\tightlist
\item
  \textbf{No Line of Sight}: Works without direct visual contact
\item
  \textbf{Multiple Reading}: Can read multiple tags simultaneously
\item
  \textbf{Data Storage}: Can store and update information
\item
  \textbf{Durability}: Resistant to environmental conditions
\end{itemize}

\textbf{Applications:}

\begin{itemize}
\tightlist
\item
  \textbf{Inventory Management}: Warehouse and retail tracking
\item
  \textbf{Access Control}: Building and vehicle access
\item
  \textbf{Payment Systems}: Contactless payment cards
\item
  \textbf{Supply Chain}: Product tracking from manufacturing to sale
\end{itemize}

\textbf{Advantages:}

\begin{itemize}
\tightlist
\item
  \textbf{Fast Reading}: Instant identification without scanning
\item
  \textbf{Automation}: Reduces manual data entry errors
\item
  \textbf{Real-time Tracking}: Continuous monitoring of assets
\end{itemize}

\end{solutionbox}
\begin{mnemonicbox}
``Radio Frequency Identifies Everything''

\end{mnemonicbox}
\begin{center}\rule{0.5\linewidth}{0.5pt}\end{center}

\subsection*{Question 3(a) [3 marks]}\label{q3a}

\textbf{Draw GSM architecture.}

\begin{solutionbox}

\begin{center}
\textbf{Mermaid Diagram (Code)}
\begin{verbatim}
{Shaded}
{Highlighting}[]
graph TD
    A[Mobile Station] {-{-}{} B[BTS {-} Base Transceiver Station]}
    B {-{-}{} C[BSC {-} Base Station Controller]}
    C {-{-}{} D[MSC {-} Mobile Switching Center]}
    D {-{-}{} E[HLR {-} Home Location Register]}
    D {-{-}{} F[VLR {-} Visitor Location Register]}
    D {-{-}{} G[PSTN/ISDN]}

    H[Authentication Center] {-{-}{} D}
    I[Equipment Identity Register] {-{-}{} D}
{Highlighting}
{Shaded}
\end{verbatim}
\end{center}

\textbf{Components:}

\begin{itemize}
\tightlist
\item
  \textbf{MS}: Mobile Station (handset + SIM)
\item
  \textbf{BTS}: Radio interface with mobile
\item
  \textbf{BSC}: Controls multiple BTS, handles handoffs
\item
  \textbf{MSC}: Switching and call control
\item
  \textbf{HLR/VLR}: Database for subscriber information
\end{itemize}

\end{solutionbox}
\begin{mnemonicbox}
``Mobile Talks Through BTS-BSC-MSC''

\end{mnemonicbox}
\begin{center}\rule{0.5\linewidth}{0.5pt}\end{center}

\subsection*{Question 3(b) [4 marks]}\label{q3b}

\textbf{Write GSM 900 specifications.}

\begin{solutionbox}

\textbf{GSM 900 Specifications Table:}

{\def\LTcaptype{none} % do not increment counter
\begin{longtable}[]{@{}ll@{}}
\toprule\noalign{}
Parameter & Specification \\
\midrule\noalign{}
\endhead
\bottomrule\noalign{}
\endlastfoot
\textbf{Frequency Band} & 890-915 MHz (Uplink), 935-960 MHz
(Downlink) \\
\textbf{Channel Spacing} & 200 kHz \\
\textbf{Total Channels} & 124 channels \\
\textbf{Modulation} & GMSK (Gaussian MSK) \\
\textbf{Access Method} & TDMA/FDMA \\
\textbf{Frame Duration} & 4.615 ms \\
\textbf{Time Slots} & 8 per frame \\
\textbf{Speech Coding} & 13 kbps RPE-LTP \\
\end{longtable}
}

\textbf{Key Features:}

\begin{itemize}
\tightlist
\item
  \textbf{Digital Transmission}: Superior voice quality compared to
  analog
\item
  \textbf{International Roaming}: Global compatibility standard
\item
  \textbf{Security}: Encryption and authentication built-in
\item
  \textbf{SMS Support}: Short message service capability
\end{itemize}

\textbf{Coverage:}

\begin{itemize}
\tightlist
\item
  \textbf{Cell Radius}: Up to 35 km (rural areas)
\item
  \textbf{Power Classes}: 5 classes from 0.8W to 20W
\end{itemize}

\end{solutionbox}
\begin{mnemonicbox}
``900 MHz, 200 kHz spacing, 8 time slots''

\end{mnemonicbox}
\begin{center}\rule{0.5\linewidth}{0.5pt}\end{center}

\subsection*{Question 3(c) [7 marks]}\label{q3c}

\textbf{Explain mobile to landline and landline to mobile call procedure
in GSM.}

\begin{solutionbox}

\textbf{Mobile to Landline Call Procedure:}

\begin{verbatim}
sequenceDiagram
    participant MS as Mobile Station
    participant BTS as BTS/BSC
    participant MSC as MSC
    participant PSTN as PSTN/Landline

    MS{-BTS: Call Request}
    BTS{-MSC: Forward Request}
    MSC{-MSC: Authenticate User}
    MSC{-PSTN: Route Call}
    PSTN{-MSC: Ring Response}
    MSC{-BTS: Ring Indication}
    BTS{-MS: Ring Back Tone}
    PSTN{-MSC: Call Answered}
    MSC{-MS: Connect Call}
\end{verbatim}

\textbf{Steps:}

\begin{enumerate}
\tightlist
\item
  \textbf{Call Initiation}: Mobile dials landline number
\item
  \textbf{Channel Assignment}: BSC assigns traffic channel
\item
  \textbf{Authentication}: MSC verifies subscriber
\item
  \textbf{Routing}: MSC routes call to PSTN gateway
\item
  \textbf{Connection}: End-to-end connection established
\end{enumerate}

\textbf{Landline to Mobile Call Procedure:}

\begin{verbatim}
sequenceDiagram
    participant PSTN as PSTN/Landline
    participant MSC as Gateway MSC
    participant HLR as HLR
    participant VMSC as Visited MSC
    participant MS as Mobile Station

    PSTN{-MSC: Call to Mobile}
    MSC{-HLR: Location Request}
    HLR{-VMSC: Get Routing Number}
    VMSC{-MSC: Return Routing Number}
    MSC{-VMSC: Route Call}
    VMSC{-MS: Page Mobile}
    MS{-VMSC: Page Response}
    VMSC{-MS: Ring Mobile}
\end{verbatim}

\textbf{Steps:}

\begin{enumerate}
\tightlist
\item
  \textbf{Call Reception}: PSTN receives call to mobile number
\item
  \textbf{HLR Query}: Gateway MSC queries HLR for location
\item
  \textbf{Location Update}: HLR provides current MSC information
\item
  \textbf{Paging}: Visited MSC pages mobile in location area
\item
  \textbf{Connection}: Mobile responds and call is connected
\end{enumerate}

\textbf{Key Differences:}

\begin{itemize}
\tightlist
\item
  \textbf{Mobile Originating}: Direct routing through serving MSC
\item
  \textbf{Mobile Terminating}: Requires location lookup through HLR
\end{itemize}

\end{solutionbox}
\begin{mnemonicbox}
``Mobile Out = Direct, Mobile In = Find First''

\end{mnemonicbox}
\begin{center}\rule{0.5\linewidth}{0.5pt}\end{center}

\subsection*{Question 3(a OR) [3
marks]}\label{question-3a-or-3-marks}

\textbf{Explain fast and slow frequency hopping.}

\begin{solutionbox}

\textbf{Frequency Hopping Types:}


{\def\LTcaptype{none} % do not increment counter
\vspace{-5pt}
\captionof{table}{Fast vs Slow Hopping}
\vspace{-10pt}
\begin{longtable}[]{@{}lll@{}}
\toprule\noalign{}
Parameter & Fast Hopping & Slow Hopping \\
\midrule\noalign{}
\endhead
\bottomrule\noalign{}
\endlastfoot
\textbf{Hop Rate} & \textgreater{} Symbol Rate & \textless{} Symbol
Rate \\
\textbf{Symbols per Hop} & \textless{} 1 & \textgreater{} 1 \\
\textbf{Complexity} & High & Low \\
\textbf{GSM Usage} & Not used & Used (217 hops/sec) \\
\end{longtable}
}

\textbf{Fast Frequency Hopping:}

\begin{itemize}
\tightlist
\item
  \textbf{Definition}: Frequency changes multiple times per symbol
\item
  \textbf{Characteristics}: Very high hop rate, complex implementation
\item
  \textbf{Advantage}: Excellent interference resistance
\end{itemize}

\textbf{Slow Frequency Hopping:}

\begin{itemize}
\tightlist
\item
  \textbf{Definition}: Multiple symbols transmitted per frequency
\item
  \textbf{GSM Implementation}: 217 hops per second
\item
  \textbf{Advantage}: Simple to implement, effective interference
  averaging
\end{itemize}

\end{solutionbox}
\begin{mnemonicbox}
``Fast = Many hops per symbol, Slow = Many symbols
per hop''

\end{mnemonicbox}
\begin{center}\rule{0.5\linewidth}{0.5pt}\end{center}

\subsection*{Question 3(b OR) [4
marks]}\label{question-3b-or-4-marks}

\textbf{Explain authentication process in GSM.}

\begin{solutionbox}

\textbf{Authentication Process:}

\begin{verbatim}
sequenceDiagram
    participant MS as Mobile Station
    participant MSC as MSC/VLR
    participant HLR as HLR/AuC

    MS{-MSC: Location Update Request}
    MSC{-HLR: Send IMSI}
    HLR{-HLR: Generate RAND, SRES, Kc}
    HLR{-MSC: Return Triplet (RAND, SRES, Kc)}
    MSC{-MS: Authentication Request (RAND)}
    MS{-MS: Calculate SRES using A3 algorithm}
    MS{-MSC: Authentication Response (SRES)}
    MSC{-MSC: Compare SRES values}
    MSC{-MS: Accept/Reject}
\end{verbatim}

\textbf{Key Components:}

\begin{itemize}
\tightlist
\item
  \textbf{RAND}: Random number (128 bits)
\item
  \textbf{SRES}: Signed response (32 bits)\\
\item
  \textbf{Kc}: Cipher key (64 bits)
\item
  \textbf{Ki}: Individual subscriber authentication key
\end{itemize}

\textbf{Process Steps:}

\begin{enumerate}
\tightlist
\item
  \textbf{Challenge}: Network sends random number (RAND)
\item
  \textbf{Response}: Mobile calculates SRES using Ki and RAND
\item
  \textbf{Verification}: Network compares received and expected SRES
\item
  \textbf{Result}: Authentication success or failure
\end{enumerate}

\textbf{Security Features:}

\begin{itemize}
\tightlist
\item
  \textbf{Mutual Authentication}: Prevents fake base stations
\item
  \textbf{Unique Keys}: Each subscriber has individual Ki
\item
  \textbf{Challenge-Response}: Prevents replay attacks
\end{itemize}

\end{solutionbox}
\begin{mnemonicbox}
``Random Challenge, Signed Response, Compare and
Accept''

\end{mnemonicbox}
\begin{center}\rule{0.5\linewidth}{0.5pt}\end{center}

\subsection*{Question 3(c OR) [7
marks]}\label{question-3c-or-7-marks}

\textbf{Draw and explain block diagram of Signal processing in GSM.}

\begin{solutionbox}

\textbf{GSM Signal Processing Block Diagram:}

\begin{center}
\textbf{Mermaid Diagram (Code)}
\begin{verbatim}
{Shaded}
{Highlighting}[]
graph LR
    A[Speech Input] {-{-}{} B[Speech Coder]}
    B {-{-}{} C[Channel Coder]}
    C {-{-}{} D[Interleaver]}
    D {-{-}{} E[Burst Formatter]}
    E {-{-}{} F[Modulator]}
    F {-{-}{} G[RF Section]}
    G {-{-}{} H[Antenna]}

    I[Antenna] {-{-}{} J[RF Section]}
    J {-{-}{} K[Demodulator]}
    K {-{-}{} L[Burst Detector]}
    L {-{-}{} M[De{-}interleaver]}
    M {-{-}{} N[Channel Decoder]}
    N {-{-}{} O[Speech Decoder]}
    O {-{-}{} P[Speech Output]}
{Highlighting}
{Shaded}
\end{verbatim}
\end{center}

\textbf{Transmitter Processing:}

\textbf{Speech Coding:}

\begin{itemize}
\tightlist
\item
  \textbf{Function}: Converts analog speech to 13 kbps digital
\item
  \textbf{Algorithm}: RPE-LTP (Regular Pulse Excitation - Long Term
  Prediction)
\item
  \textbf{Frame Size}: 20 ms speech frames
\end{itemize}

\textbf{Channel Coding:}

\begin{itemize}
\tightlist
\item
  \textbf{Purpose}: Adds redundancy for error correction
\item
  \textbf{Types}: Convolutional coding, block coding
\item
  \textbf{Output}: Protected 22.8 kbps data stream
\end{itemize}

\textbf{Interleaving:}

\begin{itemize}
\tightlist
\item
  \textbf{Function}: Spreads coded bits across multiple time slots
\item
  \textbf{Benefit}: Combats burst errors from fading
\item
  \textbf{Types}: Block interleaving over 8 time slots
\end{itemize}

\textbf{Burst Formatting:}

\begin{itemize}
\tightlist
\item
  \textbf{Process}: Organizes data into GSM burst structure
\item
  \textbf{Components}: Training sequence, guard bits, data bits
\item
  \textbf{Types}: Normal burst, access burst, sync burst
\end{itemize}

\textbf{Modulation:}

\begin{itemize}
\tightlist
\item
  \textbf{Technique}: GMSK (Gaussian Minimum Shift Keying)
\item
  \textbf{Bandwidth}: 200 kHz channel spacing
\item
  \textbf{Symbol Rate}: 270.833 kbps
\end{itemize}

\textbf{Receiver Processing:}

\begin{itemize}
\tightlist
\item
  \textbf{Demodulation}: Recovers digital bits from RF signal
\item
  \textbf{Equalization}: Compensates for multipath distortion
\item
  \textbf{Error Correction}: Uses channel coding redundancy
\item
  \textbf{Speech Decoding}: Reconstructs original speech
\end{itemize}

\textbf{Key Features:}

\begin{itemize}
\tightlist
\item
  \textbf{Digital Processing}: All operations in digital domain
\item
  \textbf{Error Protection}: Multiple levels of error correction
\item
  \textbf{Adaptive}: Parameters adjust to channel conditions
\end{itemize}

\end{solutionbox}
\begin{mnemonicbox}
``Speech-Code-Interleave-Burst-Modulate-Transmit''

\end{mnemonicbox}
\begin{center}\rule{0.5\linewidth}{0.5pt}\end{center}

\subsection*{Question 4(a) [3 marks]}\label{q4a}

\textbf{Draw block diagram of baseband section.}

\begin{solutionbox}

\textbf{Baseband Section Block Diagram:}

\begin{center}
\textbf{Mermaid Diagram (Code)}
\begin{verbatim}
{Shaded}
{Highlighting}[]
graph TD
    A[DSP Processor] {-{-}{} B[Audio Codec]}
    B {-{-}{} C[Speaker]}
    D[Microphone] {-{-}{} B}
    A {-{-}{} E[Memory Interface]}
    E {-{-}{} F[Flash Memory]}
    E {-{-}{} G[RAM]}
    A {-{-}{} H[Control Interface]}
    H {-{-}{} I[Keypad]}
    H {-{-}{} J[Display]}
    A {-{-}{} K[RF Interface]}
    A {-{-}{} L[SIM Interface]}
{Highlighting}
{Shaded}
\end{verbatim}
\end{center}

\textbf{Components:}

\begin{itemize}
\tightlist
\item
  \textbf{DSP}: Digital signal processing for speech and data
\item
  \textbf{Audio Codec}: Analog-to-digital conversion
\item
  \textbf{Memory}: Program storage (Flash) and working memory (RAM)
\item
  \textbf{Control}: User interface management
\item
  \textbf{Interfaces}: RF section, SIM card connections
\end{itemize}

\textbf{Functions:}

\begin{itemize}
\tightlist
\item
  \textbf{Signal Processing}: Speech coding, echo cancellation
\item
  \textbf{Protocol Stack}: GSM layer 1, 2, 3 protocols
\item
  \textbf{User Interface}: Display, keypad, audio management
\end{itemize}

\end{solutionbox}
\begin{mnemonicbox}
``DSP Controls Audio, Memory, Display, RF''

\end{mnemonicbox}
\begin{center}\rule{0.5\linewidth}{0.5pt}\end{center}

\subsection*{Question 4(b) [4 marks]}\label{q4b}

\textbf{Explain EDGE.}

\begin{solutionbox}

\textbf{EDGE Overview:} Enhanced Data rates for GSM Evolution - improves
data transmission in GSM networks.

\textbf{Key Features Table:}

{\def\LTcaptype{none} % do not increment counter
\begin{longtable}[]{@{}lll@{}}
\toprule\noalign{}
Parameter & GSM/GPRS & EDGE \\
\midrule\noalign{}
\endhead
\bottomrule\noalign{}
\endlastfoot
\textbf{Modulation} & GMSK & 8-PSK \\
\textbf{Data Rate} & 9.6-171 kbps & Up to 473 kbps \\
\textbf{Generation} & 2.5G & 2.75G \\
\textbf{Symbol Rate} & 270.833 ksps & 270.833 ksps \\
\end{longtable}
}

\textbf{Technical Improvements:}

\begin{itemize}
\tightlist
\item
  \textbf{Advanced Modulation}: 8-PSK carries 3 bits per symbol vs 1 bit
  in GMSK
\item
  \textbf{Link Adaptation}: Automatically switches between GMSK and
  8-PSK
\item
  \textbf{Enhanced Coding}: Better error correction schemes
\item
  \textbf{Incremental Redundancy}: Improved retransmission strategy
\end{itemize}

\textbf{Benefits:}

\begin{itemize}
\tightlist
\item
  \textbf{Higher Data Rates}: 3x faster than GPRS
\item
  \textbf{Backward Compatibility}: Works with existing GSM
  infrastructure
\item
  \textbf{Cost Effective}: Software upgrade to existing networks
\item
  \textbf{Multimedia Support}: Enables better mobile internet experience
\end{itemize}

\textbf{Applications:}

\begin{itemize}
\tightlist
\item
  \textbf{Mobile Internet}: Faster web browsing
\item
  \textbf{Email}: Quick email with attachments
\item
  \textbf{Multimedia Messaging}: MMS support
\item
  \textbf{Video Calls}: Basic video communication
\end{itemize}

\end{solutionbox}
\begin{mnemonicbox}
``EDGE = Enhanced Data rates for GSM Evolution''

\end{mnemonicbox}
\begin{center}\rule{0.5\linewidth}{0.5pt}\end{center}

\subsection*{Question 4(c) [7 marks]}\label{q4c}

\textbf{Draw and explain block diagram of mobile handset.}

\begin{solutionbox}

\textbf{Mobile Handset Block Diagram:}

\begin{center}
\textbf{Mermaid Diagram (Code)}
\begin{verbatim}
{Shaded}
{Highlighting}[]
graph TD
    A[Antenna] {-{-}{} B[Antenna Switch]}
    B {-{-}{} C[RF Transceiver]}
    C {-{-}{} D[Baseband Processor]}
    D {-{-}{} E[Audio Section]}
    E {-{-}{} F[Speaker/Microphone]}

    D {-{-}{} G[Display Controller]}
    G {-{-}{} H[LCD Display]}
    
    D {-{-}{} I[Keypad Interface]}
    I {-{-}{} J[Keypad]}
    
    D {-{-}{} K[Memory Controller]}
    K {-{-}{} L[Flash Memory]}
    K {-{-}{} M[RAM]}
    
    D {-{-}{} N[SIM Interface]}
    N {-{-}{} O[SIM Card]}
    
    P[Battery] {-{-}{} Q[Power Management]}
    Q {-{-}{} C}
    Q {-{-}{} D}
    Q {-{-}{} R[Charging Circuit]}
{Highlighting}
{Shaded}
\end{verbatim}
\end{center}

\textbf{Major Sections:}

\textbf{RF Section:}

\begin{itemize}
\tightlist
\item
  \textbf{Antenna}: Transmits and receives radio signals
\item
  \textbf{Duplexer}: Separates TX and RX signals
\item
  \textbf{RF Transceiver}: Up/down conversion, amplification
\item
  \textbf{Frequency Synthesizer}: Generates carrier frequencies
\end{itemize}

\textbf{Baseband Section:}

\begin{itemize}
\tightlist
\item
  \textbf{DSP}: Digital signal processing for speech and data
\item
  \textbf{Protocol Stack}: Implements GSM protocols
\item
  \textbf{Control Unit}: Manages all mobile functions
\item
  \textbf{Memory Interface}: Controls program and data storage
\end{itemize}

\textbf{Audio Section:}

\begin{itemize}
\tightlist
\item
  \textbf{Audio Codec}: A/D and D/A conversion
\item
  \textbf{Audio Amplifier}: Drives speaker
\item
  \textbf{Microphone Amplifier}: Amplifies voice input
\item
  \textbf{Hands-free Support}: External audio accessories
\end{itemize}

\textbf{User Interface:}

\begin{itemize}
\tightlist
\item
  \textbf{Display}: Shows information to user (LCD/OLED)
\item
  \textbf{Keypad}: User input interface
\item
  \textbf{LED Indicators}: Status indication
\item
  \textbf{Vibrator}: Alert mechanism
\end{itemize}

\textbf{Power Management:}

\begin{itemize}
\tightlist
\item
  \textbf{Battery}: Energy storage (Li-ion typically)
\item
  \textbf{Charging Circuit}: Battery charging control
\item
  \textbf{Power Regulation}: Voltage regulation for all sections
\item
  \textbf{Power Saving}: Sleep modes and power optimization
\end{itemize}

\textbf{Memory System:}

\begin{itemize}
\tightlist
\item
  \textbf{Flash Memory}: Program storage and user data
\item
  \textbf{RAM}: Working memory for program execution
\item
  \textbf{SIM Interface}: Secure element for subscriber identity
\end{itemize}

\textbf{Interconnections:}

\begin{itemize}
\tightlist
\item
  \textbf{Control Bus}: Command and control signals
\item
  \textbf{Data Bus}: Information transfer
\item
  \textbf{Power Bus}: Power distribution
\item
  \textbf{Audio Bus}: Voice and audio signals
\end{itemize}

\textbf{Operation:}

\begin{enumerate}
\tightlist
\item
  \textbf{Receive}: Antenna \rightarrow RF \rightarrow Baseband \rightarrow Audio \rightarrow Speaker
\item
  \textbf{Transmit}: Microphone \rightarrow Audio \rightarrow Baseband \rightarrow RF \rightarrow Antenna
\item
  \textbf{Control}: User input \rightarrow Baseband \rightarrow Display output
\item
  \textbf{Processing}: All operations controlled by baseband processor
\end{enumerate}

\end{solutionbox}
\begin{mnemonicbox}
``Antenna-RF-Baseband-Audio-Display-Power''

\end{mnemonicbox}
\begin{center}\rule{0.5\linewidth}{0.5pt}\end{center}

\subsection*{Question 4(a OR) [3
marks]}\label{question-4a-or-3-marks}

\textbf{Explain radiation hazards due to mobile.}

\begin{solutionbox}

\textbf{Radiation Hazards:}

\textbf{SAR (Specific Absorption Rate):}

\begin{itemize}
\tightlist
\item
  \textbf{Definition}: Rate of energy absorption by human body
\item
  \textbf{Unit}: Watts per kilogram (W/kg)
\item
  \textbf{Limit}: 2.0 W/kg (Europe), 1.6 W/kg (USA)
\end{itemize}

\textbf{Health Concerns Table:}

{\def\LTcaptype{none} % do not increment counter
\begin{longtable}[]{@{}lll@{}}
\toprule\noalign{}
Effect & Risk Level & Symptoms \\
\midrule\noalign{}
\endhead
\bottomrule\noalign{}
\endlastfoot
\textbf{Thermal} & Confirmed & Tissue heating \\
\textbf{Non-thermal} & Under study & Headaches, fatigue \\
\textbf{Long-term} & Uncertain & Cancer concerns \\
\end{longtable}
}

\textbf{Prevention Measures:}

\begin{itemize}
\tightlist
\item
  \textbf{Distance}: Keep phone away from body during calls
\item
  \textbf{Duration}: Limit call duration
\item
  \textbf{Hands-free}: Use headsets or speakerphone
\item
  \textbf{Low SAR}: Choose phones with lower SAR values
\end{itemize}

\textbf{Safety Guidelines:}

\begin{itemize}
\tightlist
\item
  Avoid sleeping with phone near head
\item
  Use airplane mode when not needed
\item
  Keep calls short and use text when possible
\end{itemize}

\end{solutionbox}
\begin{mnemonicbox}
``SAR measures absorption rate''

\end{mnemonicbox}
\begin{center}\rule{0.5\linewidth}{0.5pt}\end{center}

\subsection*{Question 4(b OR) [4
marks]}\label{question-4b-or-4-marks}

\textbf{Describe working of charging section in mobile handset.}

\begin{solutionbox}

\textbf{Charging Section Block Diagram:}

\begin{center}
\textbf{Mermaid Diagram (Code)}
\begin{verbatim}
{Shaded}
{Highlighting}[]
graph LR
    A[AC Adapter] {-{-}{} B[Rectifier]}
    B {-{-}{} C[Voltage Regulator]}
    C {-{-}{} D[Charging Controller]}
    D {-{-}{} E[Battery]}
    D {-{-}{} F[Current Monitor]}
    F {-{-}{} G[Protection Circuit]}
    G {-{-}{} H[Temperature Sensor]}
{Highlighting}
{Shaded}
\end{verbatim}
\end{center}

\textbf{Components \& Functions:}

\textbf{Charging Controller:}

\begin{itemize}
\tightlist
\item
  \textbf{Function}: Controls charging current and voltage
\item
  \textbf{Types}: Linear and switching mode controllers
\item
  \textbf{Protection}: Prevents overcharging and overheating
\end{itemize}

\textbf{Charging Process:}

\begin{enumerate}
\tightlist
\item
  \textbf{Constant Current}: Initial high current charging (fast charge)
\item
  \textbf{Constant Voltage}: Voltage maintained, current decreases
\item
  \textbf{Trickle Charge}: Low current maintenance charging
\item
  \textbf{Cut-off}: Charging stops when battery full
\end{enumerate}

\textbf{Protection Features:}

\begin{itemize}
\tightlist
\item
  \textbf{Over-voltage Protection}: Prevents damage from high voltage
\item
  \textbf{Over-current Protection}: Limits maximum charging current
\item
  \textbf{Temperature Monitoring}: Stops charging if battery gets too
  hot
\item
  \textbf{Reverse Polarity}: Prevents damage from wrong connection
\end{itemize}

\textbf{Battery Management:}

\begin{itemize}
\tightlist
\item
  \textbf{Fuel Gauge}: Monitors battery capacity
\item
  \textbf{Cell Balancing}: Ensures equal charging of battery cells
\item
  \textbf{Health Monitoring}: Tracks battery condition over time
\end{itemize}

\end{solutionbox}
\begin{mnemonicbox}
``Control Current, Voltage, Temperature, and Time''

\end{mnemonicbox}
\begin{center}\rule{0.5\linewidth}{0.5pt}\end{center}

\subsection*{Question 4(c OR) [7
marks]}\label{question-4c-or-7-marks}

\textbf{Draw and explain block diagram of DSSS transmitter and
receiver.}

\begin{solutionbox}

\textbf{DSSS Transmitter Block Diagram:}

\begin{center}
\textbf{Mermaid Diagram (Code)}
\begin{verbatim}
{Shaded}
{Highlighting}[]
graph LR
    A[Data Input] {-{-}{} B[Data Modulator]}
    B {-{-}{} C[Spreader/Mixer]}
    D[PN Code Generator] {-{-}{} C}
    C {-{-}{} E[RF Modulator]}
    E {-{-}{} F[Power Amplifier]}
    F {-{-}{} G[Antenna]}
{Highlighting}
{Shaded}
\end{verbatim}
\end{center}

\textbf{DSSS Receiver Block Diagram:}

\begin{center}
\textbf{Mermaid Diagram (Code)}
\begin{verbatim}
{Shaded}
{Highlighting}[]
graph LR
    H[Antenna] {-{-}{} I[RF Amplifier]}
    I {-{-}{} J[RF Demodulator]}
    J {-{-}{} K[Correlator/Despreader]}
    L[PN Code Generator] {-{-}{} K}
    K {-{-}{} M[Data Demodulator]}
    M {-{-}{} N[Data Output]}
    K {-{-}{} O[Synchronization]}
    O {-{-}{} L}
{Highlighting}
{Shaded}
\end{verbatim}
\end{center}

\textbf{Transmitter Operation:}

\textbf{Data Modulation:}

\begin{itemize}
\tightlist
\item
  \textbf{Input}: Original data stream (low rate)
\item
  \textbf{Modulation}: BPSK or QPSK modulation
\item
  \textbf{Output}: Modulated narrowband signal
\end{itemize}

\textbf{Spreading Process:}

\begin{itemize}
\tightlist
\item
  \textbf{PN Code}: Pseudo-random binary sequence (high rate)
\item
  \textbf{Spreading}: XOR operation between data and PN code
\item
  \textbf{Result}: Wideband spread spectrum signal
\end{itemize}

\textbf{RF Modulation:}

\begin{itemize}
\tightlist
\item
  \textbf{Carrier}: High frequency carrier signal
\item
  \textbf{Modulation}: Spread signal modulates RF carrier
\item
  \textbf{Transmission}: Signal transmitted through antenna
\end{itemize}

\textbf{Receiver Operation:}

\textbf{RF Processing:}

\begin{itemize}
\tightlist
\item
  \textbf{Reception}: Antenna receives spread spectrum signal
\item
  \textbf{Amplification}: Low noise amplifier boosts weak signal
\item
  \textbf{Demodulation}: Recovers baseband spread signal
\end{itemize}

\textbf{Despreading Process:}

\begin{itemize}
\tightlist
\item
  \textbf{Correlation}: Received signal correlated with same PN code
\item
  \textbf{Synchronization}: PN code timing synchronized with received
  signal
\item
  \textbf{Output}: Original narrowband data signal recovered
\end{itemize}

\textbf{Key Parameters:}

\begin{itemize}
\tightlist
\item
  \textbf{Processing Gain}: Ratio of spread bandwidth to data bandwidth
\item
  \textbf{Chip Rate}: Rate of PN code (higher than data rate)
\item
  \textbf{Spreading Factor}: Processing gain value
\end{itemize}

\textbf{Advantages:}

\begin{itemize}
\tightlist
\item
  \textbf{Interference Rejection}: Resistant to narrowband interference
\item
  \textbf{Low Probability of Intercept}: Difficult to detect and jam
\item
  \textbf{Multiple Access}: Many users can share same frequency
\item
  \textbf{Multipath Resistance}: Reduces fading effects
\end{itemize}

\textbf{Applications:}

\begin{itemize}
\tightlist
\item
  \textbf{CDMA Cellular}: IS-95, CDMA2000, WCDMA
\item
  \textbf{GPS}: Global positioning system
\item
  \textbf{WiFi}: 802.11b spread spectrum mode
\item
  \textbf{Military}: Secure communications
\end{itemize}

\end{solutionbox}
\begin{mnemonicbox}
``Data Spreads with PN, Correlates to Recover''

\end{mnemonicbox}
\begin{center}\rule{0.5\linewidth}{0.5pt}\end{center}

\subsection*{Question 5(a) [3 marks]}\label{q5a}

\textbf{Explain the concept of spread spectrum.}

\begin{solutionbox}

\textbf{Spread Spectrum Concept:} A communication technique where the
transmitted signal bandwidth is much wider than the minimum required
bandwidth.

\textbf{Basic Principle:}

{\def\LTcaptype{none} % do not increment counter
\begin{longtable}[]{@{}lll@{}}
\toprule\noalign{}
Parameter & Before Spreading & After Spreading \\
\midrule\noalign{}
\endhead
\bottomrule\noalign{}
\endlastfoot
\textbf{Bandwidth} & Narrow (data rate) & Wide (chip rate) \\
\textbf{Power Density} & High & Low \\
\textbf{Interference} & Vulnerable & Resistant \\
\end{longtable}
}

\textbf{Key Characteristics:}

\begin{itemize}
\tightlist
\item
  \textbf{Bandwidth Expansion}: Signal spread over wide frequency range
\item
  \textbf{Processing Gain}: Improvement in signal-to-noise ratio
\item
  \textbf{Pseudo-random Sequence}: Spreading code known only to intended
  receiver
\item
  \textbf{Security}: Difficult for unauthorized users to intercept
\end{itemize}

\textbf{Benefits:}

\begin{itemize}
\tightlist
\item
  \textbf{Jam Resistance}: Immune to intentional interference
\item
  \textbf{Low Power Density}: Coexists with narrowband systems
\item
  \textbf{Multiple Access}: Many users share same spectrum
\item
  \textbf{Privacy}: Encrypted-like transmission
\end{itemize}

\end{solutionbox}
\begin{mnemonicbox}
``Spread Wide, Gain Processing Power''

\end{mnemonicbox}
\begin{center}\rule{0.5\linewidth}{0.5pt}\end{center}

\subsection*{Question 5(b) [4 marks]}\label{q5b}

\textbf{Write criteria of spread spectrum and its applications.}

\begin{solutionbox}

\textbf{Spread Spectrum Criteria:}

\textbf{Technical Criteria:}

\begin{enumerate}
\tightlist
\item
  \textbf{Bandwidth}: Transmitted bandwidth \textgreater\textgreater{}
  Information bandwidth
\item
  \textbf{Processing Gain}: Gp = Spread BW / Data BW \geq 10 dB
\item
  \textbf{Pseudo-random}: Spreading sequence appears random
\item
  \textbf{Synchronization}: Receiver must sync with transmitter code
\end{enumerate}

\textbf{Performance Criteria Table:}

{\def\LTcaptype{none} % do not increment counter
\begin{longtable}[]{@{}lll@{}}
\toprule\noalign{}
Criteria & Requirement & Benefit \\
\midrule\noalign{}
\endhead
\bottomrule\noalign{}
\endlastfoot
\textbf{Processing Gain} & \textgreater{} 10 dB & Interference
rejection \\
\textbf{Code Length} & Long period & Security and randomness \\
\textbf{Cross-correlation} & Low & Multiple user separation \\
\textbf{Auto-correlation} & Sharp peak & Synchronization \\
\end{longtable}
}

\textbf{Applications:}

\textbf{Military Communications:}

\begin{itemize}
\tightlist
\item
  \textbf{Anti-jam}: Resistant to enemy jamming
\item
  \textbf{LPI/LPD}: Low probability of intercept/detection
\item
  \textbf{Secure}: Encrypted transmission
\end{itemize}

\textbf{Cellular Systems:}

\begin{itemize}
\tightlist
\item
  \textbf{CDMA}: IS-95, CDMA2000, WCDMA
\item
  \textbf{Capacity}: Multiple users per frequency
\item
  \textbf{Quality}: Reduced interference
\end{itemize}

\textbf{Satellite Communications:}

\begin{itemize}
\tightlist
\item
  \textbf{GPS}: Global positioning system
\item
  \textbf{Weather}: Satellite data transmission
\item
  \textbf{Broadcasting}: Satellite radio/TV
\end{itemize}

\textbf{Wireless Networks:}

\begin{itemize}
\tightlist
\item
  \textbf{WiFi}: 802.11b DSSS mode
\item
  \textbf{Bluetooth}: Frequency hopping
\item
  \textbf{Cordless Phones}: 2.4 GHz band
\end{itemize}

\end{solutionbox}
\begin{mnemonicbox}
``Military, Cellular, Satellite, Wireless use Spread
Spectrum''

\end{mnemonicbox}
\begin{center}\rule{0.5\linewidth}{0.5pt}\end{center}

\subsection*{Question 5(c) [7 marks]}\label{q5c}

\textbf{Explain call processing in CDMA.}

\begin{solutionbox}

\textbf{CDMA Call Processing Sequence:}

\begin{verbatim}
sequenceDiagram
    participant MS as Mobile Station
    participant BTS as Base Station
    participant BSC as Base Station Controller
    participant MSC as Mobile Switching Center

    Note over MS,MSC: Call Origination
    MS{-BTS: Access Request (Random Access)}
    BTS{-MS: Access Grant (Assign Code)}
    MS{-BTS: Call Setup Request}
    BTS{-BSC: Forward Call Request}
    BSC{-MSC: Route Call Setup}
    MSC{-BSC: Assign Traffic Channel}
    BSC{-BTS: Allocate Walsh Code}
    BTS{-MS: Traffic Channel Assignment}
    MS{-BTS: Confirm Assignment}
    Note over MS,MSC: Call in Progress
\end{verbatim}

\textbf{Call Origination Process:}

\textbf{Step 1: System Access}

\begin{itemize}
\tightlist
\item
  \textbf{Random Access}: Mobile sends access probe on access channel
\item
  \textbf{Power Control}: Gradually increases power until acknowledged
\item
  \textbf{Code Assignment}: Base station assigns unique spreading code
\end{itemize}

\textbf{Step 2: Authentication}

\begin{itemize}
\tightlist
\item
  \textbf{Challenge}: Network sends authentication challenge
\item
  \textbf{Response}: Mobile responds with calculated authentication
\item
  \textbf{Validation}: Network validates mobile identity
\end{itemize}

\textbf{Step 3: Channel Assignment}

\begin{itemize}
\tightlist
\item
  \textbf{Walsh Code}: Unique orthogonal code assigned for forward link
\item
  \textbf{PN Offset}: Base station identified by PN sequence offset
\item
  \textbf{Power Level}: Initial transmission power set
\end{itemize}

\textbf{Step 4: Traffic Channel Setup}

\begin{itemize}
\tightlist
\item
  \textbf{Service Options}: Voice, data, or multimedia service
  negotiated
\item
  \textbf{Rate Set}: Transmission rate configured (Rate Set 1 or 2)
\item
  \textbf{Handoff Parameters}: Neighboring cell information provided
\end{itemize}

\textbf{Call Processing Features:}

\textbf{Soft Handoff:}

\begin{itemize}
\tightlist
\item
  \textbf{Multiple Connections}: Mobile maintains links to multiple base
  stations
\item
  \textbf{Diversity}: Improves call quality and reliability
\item
  \textbf{Make-before-Break}: New connection established before old one
  dropped
\end{itemize}

\textbf{Power Control:}

\begin{itemize}
\tightlist
\item
  \textbf{Closed Loop}: Rapid power adjustments (800 Hz rate)
\item
  \textbf{Open Loop}: Initial power estimation
\item
  \textbf{Purpose}: Minimize interference, maximize capacity
\end{itemize}

\textbf{Variable Rate Vocoder:}

\begin{itemize}
\tightlist
\item
  \textbf{Rate Adaptation}: Transmission rate varies with speech
  activity
\item
  \textbf{Silence Detection}: Lower rates during speech pauses
\item
  \textbf{Capacity}: Increases system capacity
\end{itemize}

\textbf{Call Termination Process:}

\begin{verbatim}
sequenceDiagram
    participant PSTN as PSTN
    participant MSC as MSC
    participant HLR as HLR
    participant BSC as BSC/BTS
    participant MS as Mobile Station

    PSTN{-MSC: Incoming Call}
    MSC{-HLR: Location Request}
    HLR{-MSC: Routing Information}
    MSC{-BSC: Page Mobile}
    BSC{-MS: Paging Message}
    MS{-BSC: Page Response}
    BSC{-MSC: Page Response}
    MSC{-BSC: Setup Traffic Channel}
    BSC{-MS: Channel Assignment}
    MS{-BSC: Assignment Complete}
    Note over PSTN,MS: Call Connected
\end{verbatim}

\textbf{Key CDMA Features:}

\textbf{Rake Receiver:}

\begin{itemize}
\tightlist
\item
  \textbf{Multipath Combining}: Combines multiple signal paths
\item
  \textbf{Diversity Gain}: Improves signal quality
\item
  \textbf{Finger Assignment}: Each finger tracks different path
\end{itemize}

\textbf{Capacity Advantages:}

\begin{itemize}
\tightlist
\item
  \textbf{Frequency Reuse}: Same frequency used in all cells
\item
  \textbf{Interference Limited}: Capacity limited by interference, not
  frequency
\item
  \textbf{Voice Activity}: Statistical multiplexing increases capacity
\end{itemize}

\textbf{Quality Features:}

\begin{itemize}
\tightlist
\item
  \textbf{Error Correction}: Forward error correction coding
\item
  \textbf{Interleaving}: Protects against burst errors
\item
  \textbf{Adaptive Rates}: Data rate adapts to channel conditions
\end{itemize}

\textbf{Call States:}

\begin{enumerate}
\tightlist
\item
  \textbf{Idle}: Mobile monitoring paging channel
\item
  \textbf{Access}: Attempting to access system
\item
  \textbf{Traffic}: Active call in progress
\item
  \textbf{Handoff}: Transitioning between base stations
\end{enumerate}

\end{solutionbox}
\begin{mnemonicbox}
``Access-Authenticate-Assign-Traffic-Handoff''

\end{mnemonicbox}
\begin{center}\rule{0.5\linewidth}{0.5pt}\end{center}

\subsection*{Question 5(a OR) [3
marks]}\label{question-5a-or-3-marks}

\textbf{Write features of Zigbee and advantages.}

\begin{solutionbox}

\textbf{Zigbee Features:}

\textbf{Technical Specifications Table:}

{\def\LTcaptype{none} % do not increment counter
\begin{longtable}[]{@{}ll@{}}
\toprule\noalign{}
Parameter & Specification \\
\midrule\noalign{}
\endhead
\bottomrule\noalign{}
\endlastfoot
\textbf{Standard} & IEEE 802.15.4 \\
\textbf{Frequency} & 2.4 GHz, 915 MHz, 868 MHz \\
\textbf{Data Rate} & 250 kbps (2.4 GHz) \\
\textbf{Range} & 10-100 meters \\
\textbf{Power} & Ultra-low power \\
\end{longtable}
}

\textbf{Key Features:}

\begin{itemize}
\tightlist
\item
  \textbf{Mesh Network}: Self-organizing and self-healing network
\item
  \textbf{Low Power}: Battery life up to years
\item
  \textbf{Low Cost}: Inexpensive hardware implementation
\item
  \textbf{Simple Protocol}: Easy to implement and deploy
\end{itemize}

\textbf{Advantages:}

\begin{itemize}
\tightlist
\item
  \textbf{Long Battery Life}: Optimized for battery-powered devices
\item
  \textbf{Network Reliability}: Multiple routing paths available
\item
  \textbf{Scalability}: Supports thousands of nodes
\item
  \textbf{Interoperability}: Standard ensures device compatibility
\end{itemize}

\textbf{Applications:}

\begin{itemize}
\tightlist
\item
  \textbf{Home automation, Industrial monitoring, Smart lighting}
\end{itemize}

\end{solutionbox}
\begin{mnemonicbox}
``Low Power, Mesh Network, Many Applications''

\end{mnemonicbox}
\begin{center}\rule{0.5\linewidth}{0.5pt}\end{center}

\subsection*{Question 5(b OR) [4
marks]}\label{question-5b-or-4-marks}

\textbf{Explain OFDM with block diagram.}

\begin{solutionbox}

\textbf{OFDM Block Diagram:}

\begin{center}
\textbf{Mermaid Diagram (Code)}
\begin{verbatim}
{Shaded}
{Highlighting}[]
graph LR
    A[Serial Data] {-{-}{} B[Serial to Parallel]}
    B {-{-}{} C[QAM Modulator]}
    C {-{-}{} D[IFFT]}
    D {-{-}{} E[Add Cyclic Prefix]}
    E {-{-}{} F[Parallel to Serial]}
    F {-{-}{} G[RF Transmission]}

    H[RF Reception] {-{-}{} I[Serial to Parallel]}
    I {-{-}{} J[Remove Cyclic Prefix]}
    J {-{-}{} K[FFT]}
    K {-{-}{} L[QAM Demodulator]}
    L {-{-}{} M[Parallel to Serial]}
    M {-{-}{} N[Serial Data]}
{Highlighting}
{Shaded}
\end{verbatim}
\end{center}

\textbf{OFDM Principle:} Orthogonal Frequency Division Multiplexing
divides high-speed data into multiple parallel low-speed streams
transmitted simultaneously on different frequencies.

\textbf{Key Components:}

\textbf{IFFT/FFT:}

\begin{itemize}
\tightlist
\item
  \textbf{IFFT}: Inverse Fast Fourier Transform creates orthogonal
  subcarriers
\item
  \textbf{FFT}: Fast Fourier Transform recovers data at receiver
\item
  \textbf{Orthogonality}: Subcarriers don't interfere with each other
\end{itemize}

\textbf{Cyclic Prefix:}

\begin{itemize}
\tightlist
\item
  \textbf{Function}: Prevents inter-symbol interference
\item
  \textbf{Implementation}: Copy of signal end added to beginning
\item
  \textbf{Length}: Longer than channel delay spread
\end{itemize}

\textbf{Advantages:}

\begin{itemize}
\tightlist
\item
  \textbf{Spectral Efficiency}: High data rate in limited bandwidth
\item
  \textbf{Multipath Immunity}: Resistant to fading channels
\item
  \textbf{Flexible}: Easy to implement with DSP
\end{itemize}

\textbf{Applications:}

\begin{itemize}
\tightlist
\item
  \textbf{4G LTE}: Mobile communication standard
\item
  \textbf{WiFi}: 802.11a/g/n/ac standards
\item
  \textbf{Digital TV}: DVB-T, ISDB-T standards
\end{itemize}

\end{solutionbox}
\begin{mnemonicbox}
``Orthogonal Frequencies Divide Multiplexed data''

\end{mnemonicbox}
\begin{center}\rule{0.5\linewidth}{0.5pt}\end{center}

\subsection*{Question 5(c OR) [7
marks]}\label{question-5c-or-7-marks}

\textbf{Describe MANET.}

\begin{solutionbox}

\textbf{MANET Overview:} Mobile Ad-hoc Network is a self-configuring
network of mobile devices connected wirelessly without fixed
infrastructure.

\textbf{Network Topology:}

\begin{verbatim}
     A {-{-}{-}{-} B}
     |    / |
     |   /  |
     C {-{-}{-}{-} D {-}{-}{-}{-} E}
       {   /}
        { /}
         F
\end{verbatim}

\textbf{Key Characteristics:}

\textbf{Architecture Table:}

{\def\LTcaptype{none} % do not increment counter
\begin{longtable}[]{@{}lll@{}}
\toprule\noalign{}
Parameter & MANET & Cellular Network \\
\midrule\noalign{}
\endhead
\bottomrule\noalign{}
\endlastfoot
\textbf{Infrastructure} & No fixed base stations & Base stations
required \\
\textbf{Topology} & Dynamic, changes frequently & Fixed cell
structure \\
\textbf{Routing} & Multi-hop peer-to-peer & Single hop to base
station \\
\textbf{Cost} & Low deployment cost & High infrastructure cost \\
\end{longtable}
}

\textbf{MANET Features:}

\textbf{Dynamic Topology:}

\begin{itemize}
\tightlist
\item
  \textbf{Mobile Nodes}: All nodes can move freely
\item
  \textbf{Changing Links}: Network connections change as nodes move
\item
  \textbf{Self-Organization}: Network automatically reconfigures
\end{itemize}

\textbf{Multi-hop Communication:}

\begin{itemize}
\tightlist
\item
  \textbf{Relay Function}: Nodes act as routers for other nodes
\item
  \textbf{Path Discovery}: Dynamic route finding to destination
\item
  \textbf{Distributed Control}: No central coordination needed
\end{itemize}

\textbf{Routing Protocols:}

\textbf{Proactive Protocols:}

\begin{itemize}
\tightlist
\item
  \textbf{DSDV}: Destination Sequenced Distance Vector
\item
  \textbf{Characteristic}: Maintain routing tables continuously
\item
  \textbf{Advantage}: Routes available immediately
\item
  \textbf{Disadvantage}: High overhead in mobile environment
\end{itemize}

\textbf{Reactive Protocols:}

\begin{itemize}
\tightlist
\item
  \textbf{AODV}: Ad-hoc On-demand Distance Vector
\item
  \textbf{DSR}: Dynamic Source Routing
\item
  \textbf{Characteristic}: Find routes only when needed
\item
  \textbf{Advantage}: Lower overhead
\item
  \textbf{Disadvantage}: Route discovery delay
\end{itemize}

\textbf{Hybrid Protocols:}

\begin{itemize}
\tightlist
\item
  \textbf{ZRP}: Zone Routing Protocol
\item
  \textbf{Combination}: Proactive within zone, reactive between zones
\item
  \textbf{Balance}: Overhead vs.~delay optimization
\end{itemize}

\textbf{Advantages:}

\begin{itemize}
\tightlist
\item
  \textbf{No Infrastructure}: Quick deployment without base stations
\item
  \textbf{Flexibility}: Network adapts to changing topology
\item
  \textbf{Cost Effective}: Lower setup and maintenance costs
\item
  \textbf{Robustness}: No single point of failure
\end{itemize}

\textbf{Disadvantages:}

\begin{itemize}
\tightlist
\item
  \textbf{Limited Bandwidth}: Shared wireless medium
\item
  \textbf{Power Consumption}: Routing functions drain battery
\item
  \textbf{Security Issues}: Vulnerable to attacks
\item
  \textbf{Scalability}: Performance degrades with network size
\end{itemize}

\textbf{Applications:}

\textbf{Military Operations:}

\begin{itemize}
\tightlist
\item
  \textbf{Battlefield Communications}: Soldier-to-soldier communication
\item
  \textbf{Emergency Response}: Disaster relief coordination
\item
  \textbf{Surveillance}: Sensor network deployment
\end{itemize}

\textbf{Commercial Applications:}

\begin{itemize}
\tightlist
\item
  \textbf{Vehicular Networks}: Car-to-car communication
\item
  \textbf{Sensor Networks}: Environmental monitoring
\item
  \textbf{Conference Networks}: Temporary meeting networks
\item
  \textbf{Personal Area Networks}: Device interconnection
\end{itemize}

\textbf{Challenges:}

\textbf{Technical Challenges:}

\begin{itemize}
\tightlist
\item
  \textbf{Routing Overhead}: Control message bandwidth consumption
\item
  \textbf{Quality of Service}: Difficulty in guaranteeing service levels
\item
  \textbf{Power Management}: Energy-efficient operation
\item
  \textbf{Interference}: Co-channel interference from multiple hops
\end{itemize}

\textbf{Security Challenges:}

\begin{itemize}
\tightlist
\item
  \textbf{Authentication}: Verifying node identity
\item
  \textbf{Data Integrity}: Ensuring message authenticity
\item
  \textbf{Privacy}: Protecting user information
\item
  \textbf{Denial of Service}: Preventing network attacks
\end{itemize}

\textbf{Performance Metrics:}

\begin{itemize}
\tightlist
\item
  \textbf{Throughput}: Data delivery rate
\item
  \textbf{Delay}: End-to-end packet delivery time
\item
  \textbf{Packet Loss}: Percentage of lost packets
\item
  \textbf{Energy Consumption}: Battery life optimization
\end{itemize}

\textbf{Future Trends:}

\begin{itemize}
\tightlist
\item
  \textbf{Integration}: Combination with cellular and WiFi networks
\item
  \textbf{IoT Applications}: Internet of Things device networks
\item
  \textbf{5G Integration}: Part of 5G network architecture
\item
  \textbf{AI-based Routing}: Machine learning for optimal routing
\end{itemize}

\end{solutionbox}
\begin{mnemonicbox}
``Mobile Nodes, Ad-hoc Routing, No Infrastructure,
Temporary Networks''

\end{mnemonicbox}

\end{document}
