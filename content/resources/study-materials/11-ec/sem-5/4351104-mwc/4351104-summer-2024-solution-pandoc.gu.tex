\documentclass[10pt,a4paper]{article}

% content/resources/templates/preamble.tex
\usepackage[margin=0.6in]{geometry}
\author{Milav Dabgar}
\usepackage{amsmath,amssymb,amsthm}
\usepackage{booktabs}
\usepackage{multirow}
\usepackage{xcolor}
\usepackage{tcolorbox}
\tcbuselibrary{breakable,skins}
\usepackage[colorlinks=true,linkcolor=blue]{hyperref}
\usepackage{titlesec}
\usepackage{enumitem}
\usepackage{tikz}
\usepackage{pgfplots}
\usepackage{circuitikz}
\usepackage[version=4]{mhchem}
\usepackage{longtable}
\usepackage{array}
\usepackage{float}
\usepackage{caption}
\usepackage{listings}

\lstset{
  basicstyle=\small\ttfamily,
  breaklines=true,
  breakatwhitespace=false,
  postbreak=\mbox{\textcolor{red}{$\hookrightarrow$}\space},
  float=false,
  numbers=left,
  numberstyle=\tiny\color{gray},
  numbersep=10pt,
  xleftmargin=2em,
  keywordstyle=\color{blue},
  commentstyle=\color{green!60!black},
  stringstyle=\color{purple},
  backgroundcolor=\color{gray!5},
  showstringspaces=false,
  tabsize=2,
  captionpos=b,
  keepspaces=true,
  columns=flexible
}

\pgfplotsset{compat=1.18}
\usetikzlibrary{shapes,arrows,positioning,calc,patterns,decorations.pathmorphing,decorations.markings,arrows.meta}

% Color scheme
\definecolor{headcolor}{RGB}{0,102,204}
\definecolor{keycolor}{RGB}{220,20,60}
\definecolor{solutioncolor}{RGB}{34,139,34}
\definecolor{mnemoniccolor}{RGB}{148,0,211}
\definecolor{codecolor}{RGB}{0,0,100}

% Spacing
\setlength{\parskip}{3pt}
\setlist[itemize]{nosep}
\setlist[enumerate]{nosep}

% Title formatting
\titleformat{\section}{\Large\bfseries\color{headcolor}}{\thesection}{1em}{}
\titleformat{\subsection}{\large\bfseries\color{headcolor}}{\thesubsection}{1em}{}

% Pandoc tightlist compatibility
\providecommand{\tightlist}{%
  \setlength{\itemsep}{0pt}\setlength{\parskip}{0pt}}

% Pandoc longtable compatibility
\newcounter{none}
\def\thenone{}


% content/resources/templates/gujarati-boxes.tex
\usepackage{fontspec}
\usepackage{polyglossia}

% Set Gujarati as main language (document is primarily in Gujarati)
% Note: gloss-gujarati.ldf doesn't exist in polyglossia, but it will use hyphenation patterns
\setdefaultlanguage{gujarati}
\setotherlanguage{english}

% Configure Gujarati font properly
% Use Language=Default to prevent polyglossia from trying to add language-specific features
% that don't exist for Gujarati, which causes "empty feature" warnings
\newfontfamily\gujaratifont[Script=Gujarati,AutoFakeBold=2.5,AutoFakeSlant=0.3]{Noto Sans Gujarati}
\setmainfont[Script=Gujarati,AutoFakeBold=2.5,AutoFakeSlant=0.3]{Noto Sans Gujarati}
% Use Noto Sans Gujarati for monospace to support Gujarati in text
\setmonofont[Scale=0.9]{Noto Sans Gujarati}

% Configure English to use the same font
\newfontfamily\englishfont[Script=Gujarati,AutoFakeBold=2.5,AutoFakeSlant=0.3]{Noto Sans Gujarati}

% Translations for polyglossia
\gappto\captionsgujarati{
  \renewcommand{\tablename}{કોષ્ટક}
  \renewcommand{\figurename}{આકૃતિ}
}

% Helper for TikZ nodes to ensure Gujarati font
\newcommand{\gu}[1]{{\gujaratifont #1}}

% Custom environments
\newtcolorbox{solutionbox}{
    breakable,
    enhanced,
    colback=solutioncolor!5!white,
    colframe=solutioncolor!75!black,
    fonttitle=\bfseries,
    title=જવાબ
}

\newtcolorbox{solutionboxnobreak}{
 colback=solutioncolor!5!white,
 colframe=solutioncolor!75!black,
 fonttitle=\bfseries,
 title=જવાબ
}

\newtcolorbox{keyformula}{
 breakable,
 enhanced,
 colback=keycolor!5!white,
 colframe=keycolor!75!black,
 fonttitle=\bfseries,
 title=રાસાયણિક સમીકરણ/સૂત્ર
}

\newtcolorbox{mnemonicbox}{
 breakable,
 enhanced,
 colback=mnemoniccolor!5!white,
 colframe=mnemoniccolor!75!black,
 fonttitle=\bfseries,
 title=મેમરી ટ્રીક
}


\begin{document}

\begin{center}
{\Huge\bfseries\color{headcolor} Subject Name (Gujarati)}\\[5pt]
{\LARGE 4351104 -- Summer 2024}\\[3pt]
{\large Semester 1 Study Material}\\[3pt]
{\normalsize\textit{Detailed Solutions and Explanations}}
\end{center}

\vspace{10pt}

\subsection*{પ્રશ્ન 1(અ) [3
ગુણ]}\label{uxaaauxab0uxab6uxaa8-1uxa85-3-uxa97uxaa3}

\textbf{સિલેક્ટિવ સેલ સમજાવો.}

\begin{solutionbox}


{\def\LTcaptype{none} % do not increment counter
\vspace{-5pt}
\captionof{table}{સિલેક્ટિવ સેલની લાક્ષણિકતાઓ}
\vspace{-10pt}
\begin{longtable}[]{@{}ll@{}}
\toprule\noalign{}
લક્ષણ & વર્ણન \\
\midrule\noalign{}
\endhead
\bottomrule\noalign{}
\endlastfoot
હેતુ & ચોક્કસ વિસ્તારો માટે કવરેજ આપે છે \\
કદ & નાનો કવરેજ વિસ્તાર \\
ઉપયોગ & ઇન્ડોર લોકેશન, ટનલ, બિલ્ડિંગ \\
એન્ટેના & \textbf{ડાયરેક્શનલ એન્ટેના સિસ્ટમ} \\
\end{longtable}
}

\begin{itemize}
\tightlist
\item
  \textbf{સિલેક્ટિવ કવરેજ}: સિગ્નલની જરૂર હોય તેવા ચોક્કસ ભૌગોલિક વિસ્તારોને લક્ષ્ય
  બનાવે છે
\item
  \textbf{ઇન્ડોર સોલ્યૂશન}: મુખ્યત્વે બિલ્ડિંગ કવરેજ વધારવા માટે વપરાય છે
\item
  \textbf{ડાયરેક્શનલ ટ્રાન્સમિશન}: કાર્યક્ષમતા માટે ફોકસ્ડ બીમ પેટર્ન વાપરે છે
\end{itemize}

\end{solutionbox}
\begin{mnemonicbox}
``સિલેક્ટ સ્પેશિયલ સ્પોટ્સ''

\end{mnemonicbox}
\subsection*{પ્રશ્ન 1(બ) [4
ગુણ]}\label{uxaaauxab0uxab6uxaa8-1uxaac-4-uxa97uxaa3}

\textbf{અમ્બ્રેલા સેલ દોરો અને સમજાવો.}

\begin{solutionbox}

\begin{verbatim}
        Umbrella Cell
           +{-{-}{-}+}
          /     {}
         /       {}
        +         +
       / {       / }
      +   +     +   +
     Micro    Pico
     Cells    Cells
\end{verbatim}


{\def\LTcaptype{none} % do not increment counter
\vspace{-5pt}
\captionof{table}{અમ્બ્રેલા સેલના લક્ષણો}
\vspace{-10pt}
\begin{longtable}[]{@{}ll@{}}
\toprule\noalign{}
પેરામીટર & વર્ણન \\
\midrule\noalign{}
\endhead
\bottomrule\noalign{}
\endlastfoot
કવરેજ & મોટા વિસ્તારનું કવરેજ \\
હેતુ & નાના સેલ્સને ઓવરલે કરે છે \\
હેન્ડઓફ & ઇન્ટર-સેલ ટ્રાન્ઝિશન સંચાલિત કરે છે \\
ક્ષમતા & \textbf{ઓવરફ્લો ટ્રાફિક હેન્ડલ કરે છે} \\
\end{longtable}
}

\begin{itemize}
\tightlist
\item
  \textbf{મોટું કવરેજ}: નાના સેલ્સ ઉપર વિશાળ વિસ્તારનું સિગ્નલ કવરેજ પૂરું પાડે છે
\item
  \textbf{ટ્રાફિક મેનેજમેન્ટ}: માઇક્રો અને પિકો સેલ્સમાંથી ઓવરફ્લો હેન્ડલ કરે છે
\item
  \textbf{સીમલેસ હેન્ડઓફ}: હલનચલન દરમિયાન સતત કમ્યુનિકેશન સુનિશ્ચિત કરે છે
\end{itemize}

\end{solutionbox}
\begin{mnemonicbox}
``અમ્બ્રેલા બધાને કવર કરે છે''

\end{mnemonicbox}
\subsection*{પ્રશ્ન 1(ક) [7
ગુણ]}\label{uxaaauxab0uxab6uxaa8-1uxa95-7-uxa97uxaa3}

\textbf{સેલ શું છે? ફ્રીક્વન્સી રીયૂઝ વિગતવાર સમજાવો.}

\begin{solutionbox}


{\def\LTcaptype{none} % do not increment counter
\vspace{-5pt}
\captionof{table}{સેલ અને ફ્રીક્વન્સી રીયૂઝ કન્સેપ્ટ}
\vspace{-10pt}
\begin{longtable}[]{@{}lll@{}}
\toprule\noalign{}
કન્સેપ્ટ & વ્યાખ્યા & હેતુ \\
\midrule\noalign{}
\endhead
\bottomrule\noalign{}
\endlastfoot
સેલ & ભૌગોલિક કવરેજ વિસ્તાર & સેવા પ્રદાન \\
ફ્રીક્વન્સી રીયૂઝ & અલગ સેલ્સમાં સમાન ફ્રીક્વન્સી & સ્પેક્ટ્રમ કાર્યક્ષમતા \\
ક્લસ્ટર & અનોખી ફ્રીક્વન્સીઓ ધરાવતા સેલ્સનું જૂથ & ઇન્ટરફેરન્સ કંટ્રોલ \\
રીયૂઝ ડિસ્ટન્સ & સમાન ફ્રીક્વન્સીઓ વચ્ચેનું લઘુત્તમ અંતર & સિગ્નલ ગુણવત્તા \\
\end{longtable}
}

\begin{center}
\textbf{Mermaid Diagram (Code)}
\begin{verbatim}
{Shaded}
{Highlighting}[]
graph TD
    A[સેલ કન્સેપ્ટ] {-{-}{} B[હેક્સાગોનલ આકાર]}
    A {-{-}{} C[બેઝ સ્ટેશન કવરેજ]}
    D[ફ્રીક્વન્સી રીયૂઝ] {-{-}{} E[ક્લસ્ટર પેટર્ન]}
    D {-{-}{} F[કો{-}ચેનલ રીયૂઝ]}
    E {-{-}{} G[N=4,7,12 પેટર્ન]}
{Highlighting}
{Shaded}
\end{verbatim}
\end{center}

\begin{itemize}
\tightlist
\item
  \textbf{સેલની વ્યાખ્યા}: એક બેઝ સ્ટેશન એન્ટેના દ્વારા કવર થતો ભૌગોલિક વિસ્તાર
\item
  \textbf{હેક્સાગોનલ પેટર્ન}: ગેપ વિના કવરેજ માટે સૌથી કાર્યક્ષમ આકાર
\item
  \textbf{ફ્રીક્વન્સી રીયૂઝ}: ક્ષમતા માટે બિન-નજીકના સેલ્સમાં સમાન ફ્રીક્વન્સી વપરાય
  છે
\item
  \textbf{ક્લસ્ટર સાઇઝ}: ફ્રીક્વન્સી રીયૂઝ પેટર્ન નક્કી કરે છે (N=4,7,12)
\item
  \textbf{કો-ચેનલ ઇન્ટરફેરન્સ}: લઘુત્તમ રીયૂઝ અંતર દ્વારા નિયંત્રિત
\end{itemize}

\end{solutionbox}
\begin{mnemonicbox}
``સેલ્સ રીયૂઝ ફ્રીક્વન્સીઝ એફિશિયન્ટલી''

\end{mnemonicbox}
\subsection*{પ્રશ્ન 1(ક) OR [7
ગુણ]}\label{uxaaauxab0uxab6uxaa8-1uxa95-or-7-uxa97uxaa3}

\textbf{સેલ્યુલર કન્સેપ્ટને વિગતવાર સમજાવો.}

\begin{solutionbox}


{\def\LTcaptype{none} % do not increment counter
\vspace{-5pt}
\captionof{table}{સેલ્યુલર સિસ્ટમના ઘટકો}
\vspace{-10pt}
\begin{longtable}[]{@{}lll@{}}
\toprule\noalign{}
ઘટક & કાર્ય & ફાયદો \\
\midrule\noalign{}
\endhead
\bottomrule\noalign{}
\endlastfoot
સેલ ડિવિઝન & વિસ્તારને સેલ્સમાં વહેંચવું & કવરેજ ઓપ્ટિમાઇઝેશન \\
બેઝ સ્ટેશનો & વ્યક્તિગત સેલ્સની સેવા & સિગ્નલ ટ્રાન્સમિશન \\
મોબાઇલ સ્વિચિંગ & કૉલ રૂટિંગ & નેટવર્ક કનેક્ટિવિટી \\
ફ્રીક્વન્સી પ્લાનિંગ & સ્પેક્ટ્રમ એલોકેશન & ઇન્ટરફેરન્સ કંટ્રોલ \\
\end{longtable}
}

\begin{center}
\textbf{Mermaid Diagram (Code)}
\begin{verbatim}
{Shaded}
{Highlighting}[]
graph LR
    A[મોટો કવરેજ વિસ્તાર] {-{-}{} B[સેલ ડિવિઝન]}
    B {-{-}{} C[બહુવિધ બેઝ સ્ટેશનો]}
    C {-{-}{} D[ફ્રીક્વન્સી રીયૂઝ]}
    D {-{-}{} E[હાઇ કેપેસિટી સિસ્ટમ]}
{Highlighting}
{Shaded}
\end{verbatim}
\end{center}

\begin{itemize}
\tightlist
\item
  \textbf{વિસ્તાર વિભાજન}: મોટા સર્વિસ વિસ્તારને નાના હેક્સાગોનલ સેલ્સમાં વહેંચવામાં
  આવે છે
\item
  \textbf{પાવર કંટ્રોલ}: લો પાવર ટ્રાન્સમિટર ઇન્ટરફેરન્સ ઘટાડે છે
\item
  \textbf{ફ્રીક્વન્સી કાર્યક્ષમતા}: દૂરના સેલ્સમાં સમાન ફ્રીક્વન્સી ફરીથી વાપરવામાં
  આવે છે
\item
  \textbf{ક્ષમતા વૃદ્ધિ}: વધુ સાથે સાથે વપરાશકર્તાઓની સેવા કરવામાં આવે છે
\item
  \textbf{સીમલેસ કવરેજ}: બધા સેલ્સમાં સતત સેવા
\end{itemize}

\end{solutionbox}
\begin{mnemonicbox}
``ડિવાઇડ એરિયા ફોર બેટર સર્વિસ''

\end{mnemonicbox}
\subsection*{પ્રશ્ન 2(અ) [3
ગુણ]}\label{uxaaauxab0uxab6uxaa8-2uxa85-3-uxa97uxaa3}

\textbf{પૂર્ણ સ્વરૂપ લખો: (i) IMEI (ii) LTE (iii) GSM}

\begin{solutionbox}


{\def\LTcaptype{none} % do not increment counter
\vspace{-5pt}
\captionof{table}{પૂર્ણ સ્વરૂપો}
\vspace{-10pt}
\begin{longtable}[]{@{}lll@{}}
\toprule\noalign{}
સંક્ષેપ & પૂર્ણ સ્વરૂપ & હેતુ \\
\midrule\noalign{}
\endhead
\bottomrule\noalign{}
\endlastfoot
IMEI & International Mobile Equipment Identity & ડિવાઇસ ઓળખ \\
LTE & Long Term Evolution & 4G ટેકનોલોજી સ્ટાન્ડર્ડ \\
GSM & Global System for Mobile Communication & 2G સેલ્યુલર સ્ટાન્ડર્ડ \\
\end{longtable}
}

\end{solutionbox}
\begin{mnemonicbox}
``આઇડેન્ટિટી, લોંગ-ટર્મ, ગ્લોબલ''

\end{mnemonicbox}
\subsection*{પ્રશ્ન 2(બ) [4
ગુણ]}\label{uxaaauxab0uxab6uxaa8-2uxaac-4-uxa97uxaa3}

\textbf{MAHO ને વિગતવાર સમજાવો.}

\begin{solutionbox}


{\def\LTcaptype{none} % do not increment counter
\vspace{-5pt}
\captionof{table}{MAHO લાક્ષણિકતાઓ}
\vspace{-10pt}
\begin{longtable}[]{@{}ll@{}}
\toprule\noalign{}
લક્ષણ & વર્ણન \\
\midrule\noalign{}
\endhead
\bottomrule\noalign{}
\endlastfoot
પૂર્ણ સ્વરૂપ & Mobile Assisted Handoff \\
કાર્ય & હેન્ડઓફ નિર્ણયમાં મોબાઇલ મદદ કરે છે \\
માપ & સિગ્નલ સ્ટ્રેંથ મોનિટરિંગ \\
રિપોર્ટિંગ & મોબાઇલ નેટવર્કને રિપોર્ટ કરે છે \\
\end{longtable}
}

\begin{verbatim}
sequenceDiagram
    Mobile{-Base Station: સિગ્નલ સ્ટ્રેંથ રિપોર્ટ}
    Base Station{-MSC: હેન્ડઓફ રિક્વેસ્ટ}
    MSC{-Target BS: હેન્ડઓફ તૈયાર કરો}
    Target BS{-MSC: તૈયાર પુષ્ટિ}
    MSC{-Mobile: હેન્ડઓફ કમાન્ડ}
\end{verbatim}

\begin{itemize}
\tightlist
\item
  \textbf{મોબાઇલ સહાયતા}: મોબાઇલ યુનિટ પડોશી સેલ સિગ્નલ્સ માપે છે
\item
  \textbf{સિગ્નલ રિપોર્ટિંગ}: સતત માપ રિપોર્ટ્સ નેટવર્કને મોકલવામાં આવે છે
\item
  \textbf{નિર્ણય સહાયતા}: નેટવર્ક હેન્ડઓફ નિર્ણયો માટે મોબાઇલ ડેટા વાપરે છે
\item
  \textbf{ગુણવત્તા સુધારણા}: મોબાઇલ ઇનપુટ સાથે બેહતર હેન્ડઓફ નિર્ણયો
\end{itemize}

\end{solutionbox}
\begin{mnemonicbox}
``મોબાઇલ એસિસ્ટ્સ નેટવર્ક ડિસિઝન્સ''

\end{mnemonicbox}
\subsection*{પ્રશ્ન 2(ક) [7
ગુણ]}\label{uxaaauxab0uxab6uxaa8-2uxa95-7-uxa97uxaa3}

\textbf{GSM આર્કિટેક્ચર આકૃતિ સાથે સમજાવો}

\begin{solutionbox}

\begin{center}
\textbf{Mermaid Diagram (Code)}
\begin{verbatim}
{Shaded}
{Highlighting}[]
graph LR
    A[મોબાઇલ સ્ટેશન] {-{-}{} B[બેઝ ટ્રાન્સીવર સ્ટેશન]}
    B {-{-}{} C[બેઝ સ્ટેશન કંટ્રોલર]}
    C {-{-}{} D[મોબાઇલ સ્વિચિંગ સેન્ટર]}
    D {-{-}{} E[હોમ લોકેશન રજિસ્ટર]}
    D {-{-}{} F[વિઝિટર લોકેશન રજિસ્ટર]}
    D {-{-}{} G[ઓથેન્ટિકેશન સેન્ટર]}
    D {-{-}{} H[PSTN/ISDN]}
{Highlighting}
{Shaded}
\end{verbatim}
\end{center}


{\def\LTcaptype{none} % do not increment counter
\vspace{-5pt}
\captionof{table}{GSM આર્કિટેક્ચર ઘટકો}
\vspace{-10pt}
\begin{longtable}[]{@{}lll@{}}
\toprule\noalign{}
ઘટક & કાર્ય & હેતુ \\
\midrule\noalign{}
\endhead
\bottomrule\noalign{}
\endlastfoot
MS & મોબાઇલ સ્ટેશન & વપરાશકર્તા ઉપકરણ \\
BTS & બેઝ ટ્રાન્સીવર & રેડિયો ઇન્ટરફેસ \\
BSC & બેઝ સ્ટેશન કંટ્રોલર & રેડિયો રિસોર્સ મેનેજમેન્ટ \\
MSC & મોબાઇલ સ્વિચિંગ સેન્ટર & કૉલ સ્વિચિંગ \\
HLR & હોમ લોકેશન રજિસ્ટર & સબ્સ્ક્રાઇબર ડેટાબેઝ \\
VLR & વિઝિટર લોકેશન રજિસ્ટર & અસ્થાયી સબ્સ્ક્રાઇબર ડેટા \\
\end{longtable}
}

\begin{itemize}
\tightlist
\item
  \textbf{રેડિયો સબસિસ્ટમ}: BTS અને BSC રેડિયો કમ્યુનિકેશન હેન્ડલ કરે છે
\item
  \textbf{નેટવર્ક સબસિસ્ટમ}: MSC, HLR, VLR કૉલ્સ અને મોબિલિટી મેનેજ કરે છે
\item
  \textbf{ડેટાબેઝ મેનેજમેન્ટ}: HLR પર્મેનન્ટ, VLR ટેમ્પરરી ડેટા સ્ટોર કરે છે
\item
  \textbf{ઓથેન્ટિકેશન}: AuC સિક્યુરિટી ફંક્શન્સ પૂરા પાડે છે
\end{itemize}

\end{solutionbox}
\begin{mnemonicbox}
``મોબાઇલ બેઝ નેટવર્ક ડેટાબેઝ''

\end{mnemonicbox}
\subsection*{પ્રશ્ન 2(અ) OR [3
ગુણ]}\label{uxaaauxab0uxab6uxaa8-2uxa85-or-3-uxa97uxaa3}

\textbf{સેલ સ્પ્લિટિંગ સમજાવો.}

\begin{solutionbox}


{\def\LTcaptype{none} % do not increment counter
\vspace{-5pt}
\captionof{table}{સેલ સ્પ્લિટિંગ પ્રક્રિયા}
\vspace{-10pt}
\begin{longtable}[]{@{}lll@{}}
\toprule\noalign{}
પગલું & ક્રિયા & પરિણામ \\
\midrule\noalign{}
\endhead
\bottomrule\noalign{}
\endlastfoot
1 & ટ્રાન્સમિટ પાવર ઘટાડો & નાનું કવરેજ \\
2 & નવા બેઝ સ્ટેશનો ઉમેરો & કવરેજ ગેપ્સ ભરો \\
3 & ફ્રીક્વન્સી પ્લાનિંગ & ઇન્ટરફેરન્સ કંટ્રોલ જાળવો \\
4 & ક્ષમતા વૃદ્ધિ & વધુ વપરાશકર્તાઓની સેવા \\
\end{longtable}
}

\begin{itemize}
\tightlist
\item
  \textbf{પાવર રિડક્શન}: કવરેજ ઘટાડવા માટે ઓરિજિનલ સેલ પાવર ઘટાડવામાં આવે છે
\item
  \textbf{નવા સેલ્સ}: કવરેજ ગેપ્સમાં વધારાના બેઝ સ્ટેશનો ઇન્સ્ટોલ કરવામાં આવે છે
\item
  \textbf{ક્ષમતા લાભ}: વધુ સેલ્સ એટલે સમાન વિસ્તારમાં વધુ વપરાશકર્તા ક્ષમતા
\end{itemize}

\end{solutionbox}
\begin{mnemonicbox}
``સ્પ્લિટ સેલ્સ ડબલ કેપેસિટી''

\end{mnemonicbox}
\subsection*{પ્રશ્ન 2(બ) OR [4
ગુણ]}\label{uxaaauxab0uxab6uxaa8-2uxaac-or-4-uxa97uxaa3}

\textbf{હેન્ડઓફ શું છે? સોફ્ટ અને હાર્ડ હેન્ડઓફ સમજાવો.}

\begin{solutionbox}


{\def\LTcaptype{none} % do not increment counter
\vspace{-5pt}
\captionof{table}{હેન્ડઓફ પ્રકારોની સરખામણી}
\vspace{-10pt}
\begin{longtable}[]{@{}llll@{}}
\toprule\noalign{}
પ્રકાર & પ્રક્રિયા & ટેકનોલોજી & ગુણવત્તા \\
\midrule\noalign{}
\endhead
\bottomrule\noalign{}
\endlastfoot
હાર્ડ હેન્ડઓફ & બ્રેક-ધેન-મેક & GSM, TDMA & ટૂંકો વિક્ષેપ \\
સોફ્ટ હેન્ડઓફ & મેક-ધેન-બ્રેક & CDMA & સીમલેસ ટ્રાન્ઝિશન \\
\end{longtable}
}

\begin{center}
\textbf{Mermaid Diagram (Code)}
\begin{verbatim}
{Shaded}
{Highlighting}[]
graph LR
    A[મોબાઇલ મૂવિંગ] {-{-}{} B\{હેન્ડઓફ પ્રકાર\}}
    B {-{-}{}|હાર્ડ| C[જૂનું ડિસ્કનેક્ટ, નવું કનેક્ટ]}
    B {-{-}{}|સોફ્ટ| D[નવું કનેક્ટ, પછી જૂનું ડિસ્કનેક્ટ]}
{Highlighting}
{Shaded}
\end{verbatim}
\end{center}

\begin{itemize}
\tightlist
\item
  \textbf{હેન્ડઓફ વ્યાખ્યા}: એક સેલમાંથી બીજા સેલમાં કૉલ ટ્રાન્સફર કરવાની પ્રક્રિયા
\item
  \textbf{હાર્ડ હેન્ડઓફ}: નવું કનેક્શન સ્થાપિત કરતા પહેલા કનેક્શન તૂટી જાય છે
\item
  \textbf{સોફ્ટ હેન્ડઓફ}: જૂનું તોડતા પહેલા નવું કનેક્શન સ્થાપિત કરવામાં આવે છે
\item
  \textbf{ગુણવત્તા તફાવત}: સોફ્ટ હેન્ડઓફ બેહતર કૉલ ગુણવત્તા પૂરી પાડે છે
\end{itemize}

\end{solutionbox}
\begin{mnemonicbox}
``હાર્ડ બ્રેક્સ, સોફ્ટ કનેક્ટ્સ''

\end{mnemonicbox}
\subsection*{પ્રશ્ન 2(ક) OR [7
ગુણ]}\label{uxaaauxab0uxab6uxaa8-2uxa95-or-7-uxa97uxaa3}

\textbf{GSM સિગ્નલ પ્રોસેસિંગ આકૃતિ સાથે સમજાવો}

\begin{solutionbox}

\begin{center}
\textbf{Mermaid Diagram (Code)}
\begin{verbatim}
{Shaded}
{Highlighting}[]
graph LR
    A[વૉઇસ ઇનપુટ] {-{-}{} B[સ્પીચ કોડેક]}
    B {-{-}{} C[ચેનલ કોડિંગ]}
    C {-{-}{} D[ઇન્ટરલીવિંગ]}
    D {-{-}{} E[એન્ક્રિપ્શન]}
    E {-{-}{} F[બર્સ્ટ ફોર્મેટિંગ]}
    F {-{-}{} G[મોડ્યુલેશન]}
    G {-{-}{} H[RF ટ્રાન્સમિશન]}
{Highlighting}
{Shaded}
\end{verbatim}
\end{center}


{\def\LTcaptype{none} % do not increment counter
\vspace{-5pt}
\captionof{table}{GSM સિગ્નલ પ્રોસેસિંગ સ્ટેજ}
\vspace{-10pt}
\begin{longtable}[]{@{}lll@{}}
\toprule\noalign{}
સ્ટેજ & કાર્ય & હેતુ \\
\midrule\noalign{}
\endhead
\bottomrule\noalign{}
\endlastfoot
સ્પીચ કોડેક & વૉઇસ કમ્પ્રેશન & બેન્ડવિડ્થ કાર્યક્ષમતા \\
ચેનલ કોડિંગ & એરર કરેક્શન & ટ્રાન્સમિશન વિશ્વસનીયતા \\
ઇન્ટરલીવિંગ & બર્સ્ટ એરર પ્રોટેક્શન & ડેટા અખંડિતતા \\
એન્ક્રિપ્શન & સિક્યુરિટી & પ્રાઇવેસી પ્રોટેક્શન \\
મોડ્યુલેશન & RF કન્વર્ઝન & એર ઇન્ટરફેસ \\
\end{longtable}
}

\begin{itemize}
\tightlist
\item
  \textbf{સ્પીચ પ્રોસેસિંગ}: RPE-LTP કોડેક વાપરીને વૉઇસ કમ્પ્રેસ કરવામાં આવે છે
\item
  \textbf{એરર પ્રોટેક્શન}: કન્વોલ્યુશનલ કોડિંગ રિડન્ડન્સી ઉમેરે છે
\item
  \textbf{સિક્યુરિટી લેયર}: A5 અલ્ગોરિધમ ડેટાને એન્ક્રિપ્ટ કરે છે
\item
  \textbf{બર્સ્ટ સ્ટ્રક્ચર}: ડેટાને ટાઇમ સ્લોટ્સમાં ગોઠવવામાં આવે છે
\item
  \textbf{મોડ્યુલેશન}: RF ટ્રાન્સમિશન માટે GMSK મોડ્યુલેશન
\end{itemize}

\end{solutionbox}
\begin{mnemonicbox}
``વૉઇસ કોડેડ ઇન્ટરલીવ્ડ એન્ક્રિપ્ટેડ મોડ્યુલેટેડ''

\end{mnemonicbox}
\subsection*{પ્રશ્ન 3(અ) [3
ગુણ]}\label{uxaaauxab0uxab6uxaa8-3uxa85-3-uxa97uxaa3}

\textbf{સેલ સેક્ટરિંગ સમજાવો.}

\begin{solutionbox}


{\def\LTcaptype{none} % do not increment counter
\vspace{-5pt}
\captionof{table}{સેલ સેક્ટરિંગના ફાયદા}
\vspace{-10pt}
\begin{longtable}[]{@{}ll@{}}
\toprule\noalign{}
લક્ષણ & વર્ણન \\
\midrule\noalign{}
\endhead
\bottomrule\noalign{}
\endlastfoot
એન્ટેના પેટર્ન & ઓમ્નિડાયરેક્શનલને બદલે ડાયરેક્શનલ \\
સેક્ટર્સ & સેલ દીઠ 3 અથવા 6 સેક્ટર્સ \\
ક્ષમતા & 3x અથવા 6x ક્ષમતા વૃદ્ધિ \\
ઇન્ટરફેરન્સ & કો-ચેનલ ઇન્ટરફેરન્સ ઘટાડે છે \\
\end{longtable}
}

\begin{itemize}
\tightlist
\item
  \textbf{ડાયરેક્શનલ એન્ટેના}: ઓમ્નિડાયરેક્શનલને સેક્ટર એન્ટેના સાથે બદલો
\item
  \textbf{ક્ષમતા ગુણાકાર}: દરેક સેક્ટરને અલગ સેલ તરીકે ગણવામાં આવે છે
\item
  \textbf{ઇન્ટરફેરન્સ ઘટાડો}: ડાયરેક્શનલ પેટર્ન ઇન્ટરફેરન્સ ઘટાડે છે
\end{itemize}

\end{solutionbox}
\begin{mnemonicbox}
``સેક્ટર એન્ટેના ટ્રિપલ કેપેસિટી''

\end{mnemonicbox}
\subsection*{પ્રશ્ન 3(બ) [4
ગુણ]}\label{uxaaauxab0uxab6uxaa8-3uxaac-4-uxa97uxaa3}

\textbf{GSM કૉલ પ્રક્રિયા સમજાવો.}

\begin{solutionbox}

\begin{verbatim}
sequenceDiagram
    Mobile{-BTS: કૉલ રિક્વેસ્ટ}
    BTS{-BSC: ફોરવર્ડ રિક્વેસ્ટ}
    BSC{-MSC: રૂટ કૉલ}
    MSC{-HLR: વપરાશકર્તાને ઓથેન્ટિકેટ કરો}
    HLR{-MSC: ઓથેન્ટિકેશન OK}
    MSC{-PSTN: કનેક્શન સ્થાપિત કરો}
\end{verbatim}


{\def\LTcaptype{none} % do not increment counter
\vspace{-5pt}
\captionof{table}{કૉલ સેટઅપ પગલાં}
\vspace{-10pt}
\begin{longtable}[]{@{}lll@{}}
\toprule\noalign{}
પગલું & પ્રક્રિયા & હેતુ \\
\midrule\noalign{}
\endhead
\bottomrule\noalign{}
\endlastfoot
1 & ઓથેન્ટિકેશન & વપરાશકર્તા ચકાસણી \\
2 & ચેનલ એલોકેશન & રિસોર્સ એસાઇનમેન્ટ \\
3 & કૉલ રૂટિંગ & પાથ સ્થાપના \\
4 & કનેક્શન સેટઅપ & કમ્યુનિકેશન લિંક \\
\end{longtable}
}

\begin{itemize}
\tightlist
\item
  \textbf{ઓથેન્ટિકેશન}: નેટવર્ક સબ્સ્ક્રાઇબર આઇડેન્ટિટી ચકાસે છે
\item
  \textbf{રિસોર્સ એલોકેશન}: કૉલ માટે ટ્રાફિક ચેનલ અસાઇન કરવામાં આવે છે
\item
  \textbf{રૂટિંગ}: નેટવર્ક દ્વારા કૉલ પાથ નક્કી કરવામાં આવે છે
\item
  \textbf{કનેક્શન}: એન્ડ-ટુ-એન્ડ કમ્યુનિકેશન સ્થાપિત કરવામાં આવે છે
\end{itemize}

\end{solutionbox}
\begin{mnemonicbox}
``ઓથેન્ટિકેટ એલોકેટ રૂટ કનેક્ટ''

\end{mnemonicbox}
\subsection*{પ્રશ્ન 3(ક) [7
ગુણ]}\label{uxaaauxab0uxab6uxaa8-3uxa95-7-uxa97uxaa3}

\textbf{GPRS સમજાવો.}

\begin{solutionbox}


{\def\LTcaptype{none} % do not increment counter
\vspace{-5pt}
\captionof{table}{GPRS લક્ષણો}
\vspace{-10pt}
\begin{longtable}[]{@{}lll@{}}
\toprule\noalign{}
લક્ષણ & વર્ણન & ફાયદો \\
\midrule\noalign{}
\endhead
\bottomrule\noalign{}
\endlastfoot
ટેકનોલોજી & General Packet Radio Service & ડેટા સેવા \\
ડેટા રેટ & 114 kbps સુધી & હાઇ સ્પીડ \\
કનેક્શન & પેકેટ સ્વિચ્ડ & હંમેશા ઓન \\
એપ્લિકેશન્સ & ઇન્ટરનેટ, ઇમેઇલ & ડેટા સેવાઓ \\
\end{longtable}
}

\begin{verbatim}
graph TB
    A[GPRS નેટવર્ક] {-{-} B[SGSN]}
    A {-{-} C[GGSN]}
    B {-{-} D[પેકેટ ડેટા]}
    C {-{-} E[ઇન્ટરનેટ ગેટવે]}
    F[મોબાઇલ] {-{-} B}
    C {-{-} G[બાહ્ય નેટવર્ક્સ]}
\end{verbatim}

\begin{itemize}
\tightlist
\item
  \textbf{પેકેટ સ્વિચિંગ}: ડેટા સર્કિટ્સમાં નહીં પણ પેકેટ્સમાં ટ્રાન્સમિટ કરવામાં આવે છે
\item
  \textbf{હંમેશા-ઓન કનેક્શન}: ડેટા એક્સેસ માટે ડાયલ-અપની જરૂર નથી
\item
  \textbf{વધુ સ્પીડ}: સર્કિટ-સ્વિચ્ડ ડેટા કરતાં નોંધપાત્ર સુધારો
\item
  \textbf{નવા નોડ્સ}: GSM આર્કિટેક્ચરમાં SGSN અને GGSN ઉમેરવામાં આવ્યા
\item
  \textbf{ઇન્ટરનેટ એક્સેસ}: IP નેટવર્ક્સ સાથે સીધું કનેક્શન
\end{itemize}

\end{solutionbox}
\begin{mnemonicbox}
``જનરલ પેકેટ રેડિયો સર્વિસ''

\end{mnemonicbox}
\subsection*{પ્રશ્ન 3(અ) OR [3
ગુણ]}\label{uxaaauxab0uxab6uxaa8-3uxa85-or-3-uxa97uxaa3}

\textbf{CDMA ના ફાયદા સમજાવો}

\begin{solutionbox}


{\def\LTcaptype{none} % do not increment counter
\vspace{-5pt}
\captionof{table}{CDMA ફાયદા}
\vspace{-10pt}
\begin{longtable}[]{@{}ll@{}}
\toprule\noalign{}
ફાયદો & વર્ણન \\
\midrule\noalign{}
\endhead
\bottomrule\noalign{}
\endlastfoot
ક્ષમતા & વધુ વપરાશકર્તા ક્ષમતા \\
સિક્યુરિટી & બિલ્ટ-ઇન એન્ક્રિપ્શન \\
ગુણવત્તા & બેહતર વૉઇસ ગુણવત્તા \\
પાવર & કાર્યક્ષમ પાવર કંટ્રોલ \\
\end{longtable}
}

\begin{itemize}
\tightlist
\item
  \textbf{વધેલી ક્ષમતા}: ફ્રીક્વન્સી બેન્ડ દીઠ વધુ વપરાશકર્તાઓ
\item
  \textbf{વિકસિત સિક્યુરિટી}: સ્પ્રેડ સ્પેક્ટ્રમ કુદરતી એન્ક્રિપ્શન પૂરું પાડે છે
\item
  \textbf{સોફ્ટ હેન્ડઓફ}: હેન્ડઓફ દરમિયાન બેહતર કૉલ ગુણવત્તા
\end{itemize}

\end{solutionbox}
\begin{mnemonicbox}
``કેપેસિટી સિક્યુરિટી ક્વોલિટી''

\end{mnemonicbox}
\subsection*{પ્રશ્ન 3(બ) OR [4
ગુણ]}\label{uxaaauxab0uxab6uxaa8-3uxaac-or-4-uxa97uxaa3}

\textbf{ફ્રીક્વન્સી હોપિંગ તકનીકો સમજાવો.}

\begin{solutionbox}


{\def\LTcaptype{none} % do not increment counter
\vspace{-5pt}
\captionof{table}{ફ્રીક્વન્સી હોપિંગ પ્રકારો}
\vspace{-10pt}
\begin{longtable}[]{@{}lll@{}}
\toprule\noalign{}
પ્રકાર & હોપિંગ રેટ & એપ્લિકેશન \\
\midrule\noalign{}
\endhead
\bottomrule\noalign{}
\endlastfoot
સ્લો FH & સિમ્બોલ રેટ કરતાં ઓછું & GSM \\
ફાસ્ટ FH & સિમ્બોલ રેટ કરતાં વધારે & મિલિટરી \\
\end{longtable}
}

\begin{center}
\textbf{Mermaid Diagram (Code)}
\begin{verbatim}
{Shaded}
{Highlighting}[]
graph LR
    A[ડેટા] {-{-}{} B[સ્પ્રેડ સ્પેક્ટ્રમ]}
    B {-{-}{} C[ફ્રીક્વન્સી સિન્થેસાઇઝર]}
    C {-{-}{} D[હોપ સીક્વન્સ]}
    D {-{-}{} E[RF ટ્રાન્સમિશન]}
{Highlighting}
{Shaded}
\end{verbatim}
\end{center}

\begin{itemize}
\tightlist
\item
  \textbf{ફ્રીક્વન્સી હોપિંગ}: કેરિયર ફ્રીક્વન્સી પેટર્ન મુજબ બદલાય છે
\item
  \textbf{ઇન્ટરફેરન્સ રેઝિસ્ટન્સ}: નેરોબેન્ડ ઇન્ટરફેરન્સની અસર ઘટાડે છે
\item
  \textbf{સિક્યુરિટી એન્હાન્સમેન્ટ}: હોપિંગ સિગ્નલ્સને ઇન્ટરસેપ્ટ કરવું મુશ્કેલ
\item
  \textbf{GSM ઇમ્પ્લિમેન્ટેશન}: ગુણવત્તા માટે સ્લો ફ્રીક્વન્સી હોપિંગ વપરાય છે
\end{itemize}

\end{solutionbox}
\begin{mnemonicbox}
``ફ્રીક્વન્સી હોપ્સ ફોર સિક્યુરિટી''

\end{mnemonicbox}
\subsection*{પ્રશ્ન 3(ક) OR [7
ગુણ]}\label{uxaaauxab0uxab6uxaa8-3uxa95-or-7-uxa97uxaa3}

\textbf{EDGE સમજાવો.}

\begin{solutionbox}


{\def\LTcaptype{none} % do not increment counter
\vspace{-5pt}
\captionof{table}{EDGE સ્પેસિફિકેશન્સ}
\vspace{-10pt}
\begin{longtable}[]{@{}lll@{}}
\toprule\noalign{}
પેરામીટર & મૂલ્ય & સુધારો \\
\midrule\noalign{}
\endhead
\bottomrule\noalign{}
\endlastfoot
પૂર્ણ સ્વરૂપ & Enhanced Data rate for GSM Evolution & - \\
ડેટા રેટ & 384 kbps સુધી & 3x GPRS \\
મોડ્યુલેશન & 8-PSK & હાઇયર ઓર્ડર \\
સુસંગતતા & GSM/GPRS & બેકવર્ડ કમ્પેટિબલ \\
\end{longtable}
}

\begin{verbatim}
graph TB
    A[EDGE એન્હાન્સમેન્ટ] {-{-} B[8{-}PSK મોડ્યુલેશન]}
    A {-{-} C[લિંક એડેપ્ટેશન]}
    A {-{-} D[ઇન્ક્રિમેન્ટલ રિડન્ડન્સી]}
    B {-{-} E[વધુ ડેટા રેટ]}
    C {-{-} F[બેહતર ગુણવત્તા]}
    D {-{-} G[એરર કરેક્શન]}
\end{verbatim}

\begin{itemize}
\tightlist
\item
  \textbf{એન્હાન્સ્ડ મોડ્યુલેશન}: GMSK ને બદલે 8-PSK ડેટા રેટ વધારે છે
\item
  \textbf{લિંક એડેપ્ટેશન}: મોડ્યુલેશન સ્કીમ ચેનલ કંડિશન્સ મુજબ એડજસ્ટ થાય છે
\item
  \textbf{ઇન્ક્રિમેન્ટલ રિડન્ડન્સી}: સુધારેલી એરર કરેક્શન મિકેનિઝમ
\item
  \textbf{બેકવર્ડ કમ્પેટિબિલિટી}: હાલના GSM/GPRS ઇન્ફ્રાસ્ટ્રક્ચર સાથે કામ કરે છે
\item
  \textbf{3G સ્ટેપિંગ સ્ટોન}: 2G અને 3G ટેકનોલોજીઓ વચ્ચે પુલ
\end{itemize}

\end{solutionbox}
\begin{mnemonicbox}
``એન્હાન્સ્ડ ડેટા ગેટ્સ એક્સેલન્સ''

\end{mnemonicbox}
\subsection*{પ્રશ્ન 4(અ) [3
ગુણ]}\label{uxaaauxab0uxab6uxaa8-4uxa85-3-uxa97uxaa3}

\textbf{FHSS ટ્રાન્સમિટર બ્લોક આકૃતિ દોરો}

\begin{solutionbox}

\begin{verbatim}
Data {-{-} Modulator {-}{-} Frequency {-}{-} RF Amp {-}{-} Antenna}
Input               Synthesizer                   
                         \^{}
                    PN Sequence
                    Generator
\end{verbatim}


{\def\LTcaptype{none} % do not increment counter
\vspace{-5pt}
\captionof{table}{FHSS ઘટકો}
\vspace{-10pt}
\begin{longtable}[]{@{}ll@{}}
\toprule\noalign{}
ઘટક & કાર્ય \\
\midrule\noalign{}
\endhead
\bottomrule\noalign{}
\endlastfoot
PN Generator & હોપિંગ સીક્વન્સ બનાવે છે \\
ફ્રીક્વન્સી સિન્થેસાઇઝર & કેરિયર ફ્રીક્વન્સી બદલે છે \\
મોડ્યુલેટર & ડેટાને મોડ્યુલેટ કરે છે \\
\end{longtable}
}

\end{solutionbox}
\begin{mnemonicbox}
``ડેટા મોડ્યુલેટેડ ફ્રીક્વન્સી હોપ્સ''

\end{mnemonicbox}
\subsection*{પ્રશ્ન 4(બ) [4
ગુણ]}\label{uxaaauxab0uxab6uxaa8-4uxaac-4-uxa97uxaa3}

\textbf{CDMA માં કૉલ પ્રોસેસિંગ સમજાવો}

\begin{solutionbox}


{\def\LTcaptype{none} % do not increment counter
\vspace{-5pt}
\captionof{table}{CDMA કૉલ પ્રોસેસિંગ}
\vspace{-10pt}
\begin{longtable}[]{@{}lll@{}}
\toprule\noalign{}
ફેઝ & પ્રક્રિયા & હેતુ \\
\midrule\noalign{}
\endhead
\bottomrule\noalign{}
\endlastfoot
એક્સેસ & સિસ્ટમ એક્સેસ & પ્રારંભિક કનેક્શન \\
ઓથેન્ટિકેશન & આઇડેન્ટિટી વેરિફિકેશન & સિક્યુરિટી \\
ટ્રાફિક & કમ્યુનિકેશન & ડેટા ટ્રાન્સફર \\
રિલીઝ & કૉલ ટર્મિનેશન & રિસોર્સ ક્લિનઅપ \\
\end{longtable}
}

\begin{itemize}
\tightlist
\item
  \textbf{સિસ્ટમ એક્સેસ}: મોબાઇલ પાઇલટ ચેનલ એક્વાયર કરે છે અને સિંક્રોનાઇઝ થાય છે
\item
  \textbf{ઓથેન્ટિકેશન}: નેટવર્ક સબ્સ્ક્રાઇબર ક્રેડેન્શિયલ્સ ચકાસે છે
\item
  \textbf{ટ્રાફિક સ્ટેટ}: પાવર કંટ્રોલ સાથે સક્રિય કમ્યુનિકેશન
\item
  \textbf{કૉલ રિલીઝ}: કૉલ સમાપ્ત થાય ત્યારે રિસોર્સ મુક્ત કરવામાં આવે છે
\end{itemize}

\end{solutionbox}
\begin{mnemonicbox}
``એક્સેસ ઓથેન્ટિકેટ ટ્રાન્સફર રિલીઝ''

\end{mnemonicbox}
\subsection*{પ્રશ્ન 4(ક) [7
ગુણ]}\label{uxaaauxab0uxab6uxaa8-4uxa95-7-uxa97uxaa3}

\textbf{OFDM રિસીવર બ્લોક આકૃતિ દોરી સમજાવો}

\begin{solutionbox}

\begin{verbatim}
RF      {-{-} Down    {-}{-} ADC {-}{-} Remove  {-}{-} FFT {-}{-} Parallel {-}{-} Channel {-}{-} Data}
Input       Converter           Cyclic            to Serial    Decoder     Output
                                Prefix            Converter
\end{verbatim}


{\def\LTcaptype{none} % do not increment counter
\vspace{-5pt}
\captionof{table}{OFDM રિસીવર ફંક્શન્સ}
\vspace{-10pt}
\begin{longtable}[]{@{}lll@{}}
\toprule\noalign{}
ઘટક & કાર્ય & હેતુ \\
\midrule\noalign{}
\endhead
\bottomrule\noalign{}
\endlastfoot
ડાઉન કન્વર્ટર & RF to baseband & ફ્રીક્વન્સી કન્વર્ઝન \\
ADC & એનાલોગ ટુ ડિજિટલ & સિગ્નલ ડિજિટાઇઝેશન \\
રિમૂવ CP & સાયક્લિક પ્રીફિક્સ રિમૂવલ & ISI એલિમિનેશન \\
FFT & ફાસ્ટ ફૂરિયર ટ્રાન્સફોર્મ & સબકેરિયર સેપરેશન \\
ચેનલ ડિકોડર & એરર કરેક્શન & ડેટા રિકવરી \\
\end{longtable}
}

\begin{itemize}
\tightlist
\item
  \textbf{RF પ્રોસેસિંગ}: પ્રાપ્ત RF સિગ્નલને બેસબેન્ડમાં કન્વર્ટ કરે છે
\item
  \textbf{ડિજિટલ કન્વર્ઝન}: ADC એનાલોગ સિગ્નલને સેમ્પલ કરે છે
\item
  \textbf{પ્રીફિક્સ રિમૂવલ}: ISI દૂર કરવા માટે સાયક્લિક પ્રીફિક્સ રિમૂવ કરવામાં આવે
  છે
\item
  \textbf{FFT પ્રોસેસિંગ}: ઓર્થોગોનલ સબકેરિયર્સને અલગ કરે છે
\item
  \textbf{ડેટા રિકવરી}: ચેનલ ડિકોડિંગ મૂળ ડેટા પુનઃપ્રાપ્ત કરે છે
\end{itemize}

\end{solutionbox}
\begin{mnemonicbox}
``રિસીવ કન્વર્ટ રિમૂવ ટ્રાન્સફોર્મ ડિકોડ''

\end{mnemonicbox}
\subsection*{પ્રશ્ન 4(અ) OR [3
ગુણ]}\label{uxaaauxab0uxab6uxaa8-4uxa85-or-3-uxa97uxaa3}

\textbf{મોબાઇલને કારણે રેડિયેશનનું જોખમ સમજાવો.}

\begin{solutionbox}


{\def\LTcaptype{none} % do not increment counter
\vspace{-5pt}
\captionof{table}{મોબાઇલ રેડિયેશન અસરો}
\vspace{-10pt}
\begin{longtable}[]{@{}lll@{}}
\toprule\noalign{}
પેરામીટર & મૂલ્ય & અસર \\
\midrule\noalign{}
\endhead
\bottomrule\noalign{}
\endlastfoot
SAR & સ્પેસિફિક એબસોર્પ્શન રેટ & ટિશ્યુ હીટિંગ \\
ફ્રીક્વન્સી & 900/1800 MHz & પેનિટ્રેશન ડેપ્થ \\
પાવર & ટ્રાન્સમિટ પાવર & એક્સપોઝર લેવલ \\
\end{longtable}
}

\begin{itemize}
\tightlist
\item
  \textbf{SAR માપ}: સ્પેસિફિક એબસોર્પ્શન રેટ એનર્જી એબસોર્પ્શન માપે છે
\item
  \textbf{થર્મલ અસરો}: વધુ SAR ટિશ્યુ હીટિંગનું કારણ બની શકે છે
\item
  \textbf{સેફ્ટી લિમિટ્સ}: આંતરરાષ્ટ્રીય સ્ટાન્ડર્ડ SAR વેલ્યુઝને મર્યાદિત કરે છે
\end{itemize}

\end{solutionbox}
\begin{mnemonicbox}
``SAR સેફ્ટી એબસોર્પ્શન રેટ''

\end{mnemonicbox}
\subsection*{પ્રશ્ન 4(બ) OR [4
ગુણ]}\label{uxaaauxab0uxab6uxaa8-4uxaac-or-4-uxa97uxaa3}

\textbf{મોબાઇલ હેન્ડસેટમાં વપરાતી લિ-પો પ્રકારની બેટરીઓ સમજાવો.}

\begin{solutionbox}


{\def\LTcaptype{none} % do not increment counter
\vspace{-5pt}
\captionof{table}{લિ-પો બેટરી લાક્ષણિકતાઓ}
\vspace{-10pt}
\begin{longtable}[]{@{}lll@{}}
\toprule\noalign{}
લક્ષણ & વર્ણન & ફાયદો \\
\midrule\noalign{}
\endhead
\bottomrule\noalign{}
\endlastfoot
કેમિસ્ટ્રી & લિથિયમ પોલિમર & હાઇ એનર્જી ડેન્સિટી \\
આકાર & ફ્લેક્સિબલ ફોર્મ ફેક્ટર & ડિઝાઇન ફ્રીડમ \\
વજન & હલકું & પોર્ટેબિલિટી \\
ચાર્જિંગ & ફાસ્ટ ચાર્જિંગ & વપરાશકર્તા સુવિધા \\
\end{longtable}
}

\begin{itemize}
\tightlist
\item
  \textbf{પોલિમર ઇલેક્ટ્રોલાઇટ}: લિક્વિડ ઇલેક્ટ્રોલાઇટને બદલે પોલિમર વાપરે છે
\item
  \textbf{ફ્લેક્સિબલ પેકેજિંગ}: ડિવાઇસ ડિઝાઇન મુજબ આકાર આપી શકાય છે
\item
  \textbf{હાઇ એનર્જી ડેન્સિટી}: નાના કદમાં વધુ ક્ષમતા
\item
  \textbf{ફાસ્ટ ચાર્જિંગ}: રેપિડ ચાર્જિંગ પ્રોટોકોલ્સને સપોર્ટ કરે છે
\end{itemize}

\end{solutionbox}
\begin{mnemonicbox}
``લિથિયમ પોલિમર પાવર''

\end{mnemonicbox}
\subsection*{પ્રશ્ન 4(ક) OR [7
ગુણ]}\label{uxaaauxab0uxab6uxaa8-4uxa95-or-7-uxa97uxaa3}

\textbf{મોબાઇલ હેન્ડસેટ બ્લોક ડાયાગ્રામ સમજાવો.}

\begin{solutionbox}

\begin{verbatim}
graph TB
    A[એન્ટેના] {-{-} B[RF સેક્શન]}
    B {-{-} C[બેસબેન્ડ પ્રોસેસર]}
    C {-{-} D[ઓડિયો કોડેક]}
    C {-{-} E[ડિસ્પ્લે કંટ્રોલર]}
    C {-{-} F[કીપેડ ઇન્ટરફેસ]}
    G[બેટરી] {-{-} H[પાવર મેનેજમેન્ટ]}
    H {-{-} B}
    H {-{-} C}
    I[SIM ઇન્ટરફેસ] {-{-} C}
\end{verbatim}


{\def\LTcaptype{none} % do not increment counter
\vspace{-5pt}
\captionof{table}{મોબાઇલ હેન્ડસેટ ઘટકો}
\vspace{-10pt}
\begin{longtable}[]{@{}lll@{}}
\toprule\noalign{}
સેક્શન & કાર્ય & હેતુ \\
\midrule\noalign{}
\endhead
\bottomrule\noalign{}
\endlastfoot
RF સેક્શન & રેડિયો ફ્રીક્વન્સી પ્રોસેસિંગ & એર ઇન્ટરફેસ \\
બેસબેન્ડ & ડિજિટલ સિગ્નલ પ્રોસેસિંગ & પ્રોટોકોલ હેન્ડલિંગ \\
ઓડિયો કોડેક & વૉઇસ પ્રોસેસિંગ & સાઉન્ડ કન્વર્ઝન \\
પાવર મેનેજમેન્ટ & બેટરી કંટ્રોલ & પાવર એફિશિયન્સી \\
SIM ઇન્ટરફેસ & સબ્સ્ક્રાઇબર આઇડેન્ટિટી & વપરાશકર્તા ઓથેન્ટિકેશન \\
\end{longtable}
}

\begin{itemize}
\tightlist
\item
  \textbf{RF સેક્શન}: રેડિયો સિગ્નલ્સનું ટ્રાન્સમિશન અને રિસેપ્શન હેન્ડલ કરે છે
\item
  \textbf{બેસબેન્ડ પ્રોસેસર}: કમ્યુનિકેશન પ્રોટોકોલ્સ ઇમ્પ્લિમેન્ટ કરે છે
\item
  \textbf{ઓડિયો સબસિસ્ટમ}: વૉઇસ અને ઓડિયો સિગ્નલ્સ પ્રોસેસ કરે છે
\item
  \textbf{પાવર મેનેજમેન્ટ}: બેટરી ઉપયોગ અને ચાર્જિંગ કંટ્રોલ કરે છે
\item
  \textbf{યુઝર ઇન્ટરફેસ}: ડિસ્પ્લે, કીપેડ અને યુઝર ઇન્ટરેક્શન
\end{itemize}

\end{solutionbox}
\begin{mnemonicbox}
``રેડિયો બેસબેન્ડ ઓડિયો પાવર ઇન્ટરફેસ''

\end{mnemonicbox}
\subsection*{પ્રશ્ન 5(અ) [3
ગુણ]}\label{uxaaauxab0uxab6uxaa8-5uxa85-3-uxa97uxaa3}

\textbf{CDMA અને GSM ની સરખામણી કરો}

\begin{solutionbox}


{\def\LTcaptype{none} % do not increment counter
\vspace{-5pt}
\captionof{table}{CDMA vs GSM સરખામણી}
\vspace{-10pt}
\begin{longtable}[]{@{}lll@{}}
\toprule\noalign{}
લક્ષણ & CDMA & GSM \\
\midrule\noalign{}
\endhead
\bottomrule\noalign{}
\endlastfoot
એક્સેસ મેથડ & કોડ ડિવિઝન & ટાઇમ ડિવિઝન \\
ક્ષમતા & વધુ & ઓછી \\
હેન્ડઓફ & સોફ્ટ & હાર્ડ \\
SIM કાર્ડ & જરૂરી નથી & જરૂરી \\
\end{longtable}
}

\end{solutionbox}
\begin{mnemonicbox}
``કોડ વર્સ ટાઇમ ડિવિઝન''

\end{mnemonicbox}
\subsection*{પ્રશ્ન 5(બ) [4
ગુણ]}\label{uxaaauxab0uxab6uxaa8-5uxaac-4-uxa97uxaa3}

\textbf{HSDPA સમજાવો.}

\begin{solutionbox}


{\def\LTcaptype{none} % do not increment counter
\vspace{-5pt}
\captionof{table}{HSDPA લક્ષણો}
\vspace{-10pt}
\begin{longtable}[]{@{}ll@{}}
\toprule\noalign{}
લક્ષણ & વર્ણન \\
\midrule\noalign{}
\endhead
\bottomrule\noalign{}
\endlastfoot
પૂર્ણ સ્વરૂપ & High Speed Downlink Packet Access \\
ડેટા રેટ & 14.4 Mbps સુધી \\
ટેકનોલોજી & 3.5G એન્હાન્સમેન્ટ \\
દિશા & ડાઉનલિંક ઓપ્ટિમાઇઝેશન \\
\end{longtable}
}

\begin{itemize}
\tightlist
\item
  \textbf{3.5G ટેકનોલોજી}: 3G UMTS સિસ્ટમનું એન્હાન્સમેન્ટ
\item
  \textbf{હાઇ સ્પીડ ડાઉનલિંક}: ડાઉનલોડ એપ્લિકેશન્સ માટે ઓપ્ટિમાઇઝ્ડ
\item
  \textbf{એડેપ્ટિવ મોડ્યુલેશન}: ચેનલ આધારિત QPSK થી 16-QAM
\item
  \textbf{ફાસ્ટ શેડ્યુલિંગ}: 2ms શેડ્યુલિંગ ઇન્ટરવલ્સ
\end{itemize}

\end{solutionbox}
\begin{mnemonicbox}
``હાઇ સ્પીડ ડાઉનલોડ એક્સેસ''

\end{mnemonicbox}
\subsection*{પ્રશ્ન 5(ક) [7
ગુણ]}\label{uxaaauxab0uxab6uxaa8-5uxa95-7-uxa97uxaa3}

\textbf{બ્લૂટૂથના આર્કિટેક્ચર, સુવિધાઓ અને ફાયદા સમજાવો.}

\begin{solutionbox}

\begin{center}
\textbf{Mermaid Diagram (Code)}
\begin{verbatim}
{Shaded}
{Highlighting}[]
graph LR
    A[એપ્લિકેશન લેયર] {-{-}{} B[L2CAP]}
    B {-{-}{} C[HCI]}
    C {-{-}{} D[લિંક મેનેજર]}
    D {-{-}{} E[બેસબેન્ડ]}
    E {-{-}{} F[રેડિયો લેયર]}
{Highlighting}
{Shaded}
\end{verbatim}
\end{center}


{\def\LTcaptype{none} % do not increment counter
\vspace{-5pt}
\captionof{table}{બ્લૂટૂથ લક્ષણો}
\vspace{-10pt}
\begin{longtable}[]{@{}lll@{}}
\toprule\noalign{}
લક્ષણ & વર્ણન & ફાયદો \\
\midrule\noalign{}
\endhead
\bottomrule\noalign{}
\endlastfoot
રેન્જ & 10 મીટર & પર્સનલ એરિયા નેટવર્ક \\
ફ્રીક્વન્સી & 2.4 GHz ISM & અનલાઇસન્સ્ડ બેન્ડ \\
ટોપોલોજી & સ્ટાર/સ્કેટરનેટ & ફ્લેક્સિબલ કનેક્શન્સ \\
પાવર & લો પાવર & બેટરી એફિશિયન્સી \\
\end{longtable}
}


{\def\LTcaptype{none} % do not increment counter
\vspace{-5pt}
\captionof{table}{બ્લૂટૂથ એપ્લિકેશન્સ}
\vspace{-10pt}
\begin{longtable}[]{@{}ll@{}}
\toprule\noalign{}
એપ્લિકેશન & ઉપયોગ કેસ \\
\midrule\noalign{}
\endhead
\bottomrule\noalign{}
\endlastfoot
ઓડિયો & વાયરલેસ હેડફોન્સ \\
ડેટા & ફાઇલ ટ્રાન્સફર \\
ઇનપુટ & વાયરલેસ કીબોર્ડ/માઉસ \\
નેટવર્કિંગ & ઇન્ટરનેટ શેરિંગ \\
\end{longtable}
}

\begin{itemize}
\tightlist
\item
  \textbf{શોર્ટ રેન્જ}: પર્સનલ એરિયા નેટવર્ક્સ માટે ડિઝાઇન કરવામાં આવ્યું
\item
  \textbf{લો પાવર}: બેટરી-પાવર્ડ ડિવાઇસ માટે ઓપ્ટિમાઇઝ કરવામાં આવ્યું
\item
  \textbf{ફ્રીક્વન્સી હોપિંગ}: ઇન્ટરફેરન્સ રેઝિસ્ટન્સ માટે 79 ચેનલ્સ
\item
  \textbf{માસ્ટર-સ્લેવ}: એક માસ્ટર 7 સ્લેવ્સ સાથે કનેક્ટ થઈ શકે છે
\item
  \textbf{એપ્લિકેશન્સ}: ઓડિયો, ડેટા ટ્રાન્સફર, ઇનપુટ ડિવાઇસ
\end{itemize}

\end{solutionbox}
\begin{mnemonicbox}
``બ્લૂ પર્સનલ એરિયા નેટવર્ક''

\end{mnemonicbox}
\subsection*{પ્રશ્ન 5(અ) OR [3
ગુણ]}\label{uxaaauxab0uxab6uxaa8-5uxa85-or-3-uxa97uxaa3}

\textbf{RFID ની મૂળભૂત વિભાવના સમજાવો.}

\begin{solutionbox}


{\def\LTcaptype{none} % do not increment counter
\vspace{-5pt}
\captionof{table}{RFID ઘટકો}
\vspace{-10pt}
\begin{longtable}[]{@{}ll@{}}
\toprule\noalign{}
ઘટક & કાર્ય \\
\midrule\noalign{}
\endhead
\bottomrule\noalign{}
\endlastfoot
RFID ટેગ & ઓળખ ડેટા સ્ટોર કરે છે \\
RFID રીડર & ટેગ માહિતી વાંચે છે \\
એન્ટેના & RF કમ્યુનિકેશન \\
બેકએન્ડ સિસ્ટમ & ડેટા પ્રોસેસિંગ \\
\end{longtable}
}

\begin{itemize}
\tightlist
\item
  \textbf{રેડિયો ફ્રીક્વન્સી આઇડેન્ટિફિકેશન}: ઓળખ માટે RF તરંગોનો ઉપયોગ કરે છે
\item
  \textbf{કોન્ટેક્ટલેસ ઓપરેશન}: ભૌતિક સંપર્કની જરૂર નથી
\item
  \textbf{ઓટોમેટિક આઇડેન્ટિફિકેશન}: રેન્જમાં હોય તેવા ટેગ્સ આપોઆપ વાંચે છે
\end{itemize}

\end{solutionbox}
\begin{mnemonicbox}
``રેડિયો ફ્રીક્વન્સી આઇડેન્ટિફાઇઝ''

\end{mnemonicbox}
\subsection*{પ્રશ્ન 5(બ) OR [4
ગુણ]}\label{uxaaauxab0uxab6uxaa8-5uxaac-or-4-uxa97uxaa3}

\textbf{5G સિસ્ટમનું આર્કિટેક્ચર સમજાવો.}

\begin{solutionbox}


{\def\LTcaptype{none} % do not increment counter
\vspace{-5pt}
\captionof{table}{5G આર્કિટેક્ચર ઘટકો}
\vspace{-10pt}
\begin{longtable}[]{@{}ll@{}}
\toprule\noalign{}
ઘટક & કાર્ય \\
\midrule\noalign{}
\endhead
\bottomrule\noalign{}
\endlastfoot
gNodeB & 5G બેઝ સ્ટેશન \\
AMF & Access and Mobility Function \\
SMF & Session Management Function \\
UPF & User Plane Function \\
\end{longtable}
}

\begin{itemize}
\tightlist
\item
  \textbf{સર્વિસ-બેઝ્ડ આર્કિટેક્ચર}: મોડ્યુલર નેટવર્ક ફંક્શન્સ
\item
  \textbf{નેટવર્ક સ્લાઇસિંગ}: વિવિધ સેવાઓ માટે વર્ચ્યુઅલ નેટવર્ક્સ
\item
  \textbf{એજ કમ્પ્યુટિંગ}: વપરાશકર્તાઓની નજીક પ્રોસેસિંગ
\item
  \textbf{મેસિવ MIMO}: બહુવિધ એન્ટેના ટેકનોલોજી
\end{itemize}

\end{solutionbox}
\begin{mnemonicbox}
``સર્વિસ બેઝ્ડ નેટવર્ક સ્લાઇસિંગ''

\end{mnemonicbox}
\subsection*{પ્રશ્ન 5(ક) OR [7
ગુણ]}\label{uxaaauxab0uxab6uxaa8-5uxa95-or-7-uxa97uxaa3}

\textbf{MANET ને વિગતવાર સમજાવો.}

\begin{solutionbox}


{\def\LTcaptype{none} % do not increment counter
\vspace{-5pt}
\captionof{table}{MANET લાક્ષણિકતાઓ}
\vspace{-10pt}
\begin{longtable}[]{@{}lll@{}}
\toprule\noalign{}
લક્ષણ & વર્ણન & ફાયદો \\
\midrule\noalign{}
\endhead
\bottomrule\noalign{}
\endlastfoot
ઇન્ફ્રાસ્ટ્રક્ચર & ઇન્ફ્રાસ્ટ્રક્ચર-લેસ & બેઝ સ્ટેશનોની જરૂર નથી \\
મોબિલિટી & મોબાઇલ નોડ્સ & ડાયનેમિક ટોપોલોજી \\
રૂટિંગ & મલ્ટી-હોપ રૂટિંગ & વિસ્તૃત કવરેજ \\
સેલ્ફ-ઓર્ગેનાઇઝિંગ & ઓટોમેટિક કન્ફિગરેશન & સરળ ડિપ્લોયમેન્ટ \\
\end{longtable}
}

\begin{center}
\textbf{Mermaid Diagram (Code)}
\begin{verbatim}
{Shaded}
{Highlighting}[]
graph LR
    A[નોડ A] {-{-}{} B[નોડ B]}
    B {-{-}{} C[નોડ C]}
    A {-{-}{} D[નોડ D]}
    C {-{-}{} E[નોડ E]}
    D {-{-}{} E}
    B {-{-}{} E}
{Highlighting}
{Shaded}
\end{verbatim}
\end{center}


{\def\LTcaptype{none} % do not increment counter
\vspace{-5pt}
\captionof{table}{MANET vs સેલ્યુલર નેટવર્ક}
\vspace{-10pt}
\begin{longtable}[]{@{}lll@{}}
\toprule\noalign{}
પેરામીટર & MANET & સેલ્યુલર \\
\midrule\noalign{}
\endhead
\bottomrule\noalign{}
\endlastfoot
ઇન્ફ્રાસ્ટ્રક્ચર & કોઈ નથી & બેઝ સ્ટેશનો જરૂરી \\
ટોપોલોજી & ડાયનેમિક & ફિક્સ્ડ \\
રેન્જ & મલ્ટી-હોપ & સિંગલ હોપ \\
કિંમત & ઓછી & વધુ ઇન્ફ્રાસ્ટ્રક્ચર કોસ્ટ \\
\end{longtable}
}

\begin{itemize}
\tightlist
\item
  \textbf{મોબાઇલ એડ-હોક નેટવર્ક}: મોબાઇલ ડિવાઇસનું સેલ્ફ-કન્ફિગરિંગ નેટવર્ક
\item
  \textbf{કોઈ ઇન્ફ્રાસ્ટ્રક્ચર નથી}: નોડ્સ બેઝ સ્ટેશનો વિના સીધું કમ્યુનિકેટ કરે છે
\item
  \textbf{ડાયનેમિક રૂટિંગ}: નોડ્સ હલે તેમ રૂટ્સ બદલાય છે
\item
  \textbf{મલ્ટી-હોપ કમ્યુનિકેશન}: મેસેજ ઇન્ટરમીડિયેટ નોડ્સ દ્વારા રિલે થાય છે
\item
  \textbf{એપ્લિકેશન્સ}: મિલિટરી, ડિઝાસ્ટર રિકવરી, સેન્સર નેટવર્ક્સ
\end{itemize}

\end{solutionbox}
\begin{mnemonicbox}
``મોબાઇલ એડહોક નેટવર્ક''

\end{mnemonicbox}

\end{document}
