\documentclass[10pt,a4paper]{article}

% content/resources/templates/preamble.tex
\usepackage[margin=0.6in]{geometry}
\author{Milav Dabgar}
\usepackage{amsmath,amssymb,amsthm}
\usepackage{booktabs}
\usepackage{multirow}
\usepackage{xcolor}
\usepackage{tcolorbox}
\tcbuselibrary{breakable,skins}
\usepackage[colorlinks=true,linkcolor=blue]{hyperref}
\usepackage{titlesec}
\usepackage{enumitem}
\usepackage{tikz}
\usepackage{pgfplots}
\usepackage{circuitikz}
\usepackage[version=4]{mhchem}
\usepackage{longtable}
\usepackage{array}
\usepackage{float}
\usepackage{caption}
\usepackage{listings}

\lstset{
  basicstyle=\small\ttfamily,
  breaklines=true,
  breakatwhitespace=false,
  postbreak=\mbox{\textcolor{red}{$\hookrightarrow$}\space},
  float=false,
  numbers=left,
  numberstyle=\tiny\color{gray},
  numbersep=10pt,
  xleftmargin=2em,
  keywordstyle=\color{blue},
  commentstyle=\color{green!60!black},
  stringstyle=\color{purple},
  backgroundcolor=\color{gray!5},
  showstringspaces=false,
  tabsize=2,
  captionpos=b,
  keepspaces=true,
  columns=flexible
}

\pgfplotsset{compat=1.18}
\usetikzlibrary{shapes,arrows,positioning,calc,patterns,decorations.pathmorphing,decorations.markings,arrows.meta}

% Color scheme
\definecolor{headcolor}{RGB}{0,102,204}
\definecolor{keycolor}{RGB}{220,20,60}
\definecolor{solutioncolor}{RGB}{34,139,34}
\definecolor{mnemoniccolor}{RGB}{148,0,211}
\definecolor{codecolor}{RGB}{0,0,100}

% Spacing
\setlength{\parskip}{3pt}
\setlist[itemize]{nosep}
\setlist[enumerate]{nosep}

% Title formatting
\titleformat{\section}{\Large\bfseries\color{headcolor}}{\thesection}{1em}{}
\titleformat{\subsection}{\large\bfseries\color{headcolor}}{\thesubsection}{1em}{}

% Pandoc tightlist compatibility
\providecommand{\tightlist}{%
  \setlength{\itemsep}{0pt}\setlength{\parskip}{0pt}}

% Pandoc longtable compatibility
\newcounter{none}
\def\thenone{}


% content/resources/templates/english-boxes.tex
% This file is currently empty - it exists to maintain consistency with the import structure.
% Add custom environments here if needed in the future.


\begin{document}

\begin{center}
{\Huge\bfseries\color{headcolor} Subject Name Solutions}\\[5pt]
{\LARGE 4351104 -- Summer 2024}\\[3pt]
{\large Semester 1 Study Material}\\[3pt]
{\normalsize\textit{Detailed Solutions and Explanations}}
\end{center}

\vspace{10pt}

\subsection*{Question 1(a) [3 marks]}\label{q1a}

\textbf{Explain selective cell.}

\begin{solutionbox}


{\def\LTcaptype{none} % do not increment counter
\vspace{-5pt}
\captionof{table}{Selective Cell Characteristics}
\vspace{-10pt}
\begin{longtable}[]{@{}ll@{}}
\toprule\noalign{}
Feature & Description \\
\midrule\noalign{}
\endhead
\bottomrule\noalign{}
\endlastfoot
Purpose & Provides coverage for specific areas \\
Size & Small coverage area \\
Application & Indoor locations, tunnels, buildings \\
Antenna & Directional antenna system \\
\end{longtable}
}

\begin{itemize}
\tightlist
\item
  \textbf{Selective coverage}: Targets specific geographical areas
  needing signal
\item
  \textbf{Indoor solution}: Primarily used for building coverage
  enhancement
\item
  \textbf{Directional transmission}: Uses focused beam patterns for
  efficiency
\end{itemize}

\end{solutionbox}
\begin{mnemonicbox}
``Select Special Spots''

\end{mnemonicbox}
\subsection*{Question 1(b) [4 marks]}\label{q1b}

\textbf{Draw and explain umbrella cell.}

\begin{solutionbox}

\begin{verbatim}
        Umbrella Cell
           +{-{-}{-}+}
          /     {}
         /       {}
        +         +
       / {       / }
      +   +     +   +
     Micro    Pico
     Cells    Cells
\end{verbatim}


{\def\LTcaptype{none} % do not increment counter
\vspace{-5pt}
\captionof{table}{Umbrella Cell Features}
\vspace{-10pt}
\begin{longtable}[]{@{}ll@{}}
\toprule\noalign{}
Parameter & Description \\
\midrule\noalign{}
\endhead
\bottomrule\noalign{}
\endlastfoot
Coverage & Large area coverage \\
Purpose & Overlays smaller cells \\
Handoff & Manages inter-cell transitions \\
Capacity & Handles overflow traffic \\
\end{longtable}
}

\begin{itemize}
\tightlist
\item
  \textbf{Large coverage}: Provides wide area signal coverage over
  smaller cells
\item
  \textbf{Traffic management}: Handles overflow from micro and pico
  cells
\item
  \textbf{Seamless handoff}: Ensures continuous communication during
  movement
\end{itemize}

\end{solutionbox}
\begin{mnemonicbox}
``Umbrella Covers All''

\end{mnemonicbox}
\subsection*{Question 1(c) [7 marks]}\label{q1c}

\textbf{What is the cell? Explain frequency reuse.}

\begin{solutionbox}


{\def\LTcaptype{none} % do not increment counter
\vspace{-5pt}
\captionof{table}{Cell and Frequency Reuse Concepts}
\vspace{-10pt}
\begin{longtable}[]{@{}
  >{\raggedright\arraybackslash}p{(\linewidth - 4\tabcolsep) * \real{0.3000}}
  >{\raggedright\arraybackslash}p{(\linewidth - 4\tabcolsep) * \real{0.4000}}
  >{\raggedright\arraybackslash}p{(\linewidth - 4\tabcolsep) * \real{0.3000}}@{}}
\toprule\noalign{}
\begin{minipage}[b]{\linewidth}\raggedright
Concept
\end{minipage} & \begin{minipage}[b]{\linewidth}\raggedright
Definition
\end{minipage} & \begin{minipage}[b]{\linewidth}\raggedright
Purpose
\end{minipage} \\
\midrule\noalign{}
\endhead
\bottomrule\noalign{}
\endlastfoot
Cell & Geographic coverage area & Service provision \\
Frequency Reuse & Same frequency in different cells & Spectrum
efficiency \\
Cluster & Group of cells with unique frequencies & Interference
control \\
Reuse Distance & Minimum distance between same frequencies & Signal
quality \\
\end{longtable}
}

\begin{center}
\textbf{Mermaid Diagram (Code)}
\begin{verbatim}
{Shaded}
{Highlighting}[]
graph TD
    A[Cell Concept] {-{-}{} B[Hexagonal Shape]}
    A {-{-}{} C[Base Station Coverage]}
    D[Frequency Reuse] {-{-}{} E[Cluster Pattern]}
    D {-{-}{} F[Co{-}channel Reuse]}
    E {-{-}{} G[N=4,7,12 patterns]}
{Highlighting}
{Shaded}
\end{verbatim}
\end{center}

\begin{itemize}
\tightlist
\item
  \textbf{Cell definition}: Geographical area covered by one base
  station antenna
\item
  \textbf{Hexagonal pattern}: Most efficient shape for coverage without
  gaps
\item
  \textbf{Frequency reuse}: Same frequencies used in non-adjacent cells
  for capacity
\item
  \textbf{Cluster size}: Determines frequency reuse pattern (N=4,7,12)
\item
  \textbf{Co-channel interference}: Controlled by minimum reuse distance
\end{itemize}

\end{solutionbox}
\begin{mnemonicbox}
``Cells Reuse Frequencies Efficiently''

\end{mnemonicbox}
\subsection*{Question 1(c) OR [7
marks]}\label{q1c}

\textbf{Explain cellular concept in detail.}

\begin{solutionbox}


{\def\LTcaptype{none} % do not increment counter
\vspace{-5pt}
\captionof{table}{Cellular System Components}
\vspace{-10pt}
\begin{longtable}[]{@{}lll@{}}
\toprule\noalign{}
Component & Function & Benefit \\
\midrule\noalign{}
\endhead
\bottomrule\noalign{}
\endlastfoot
Cell Division & Area split into cells & Coverage optimization \\
Base Stations & Serve individual cells & Signal transmission \\
Mobile Switching & Call routing & Network connectivity \\
Frequency Planning & Spectrum allocation & Interference control \\
\end{longtable}
}

\begin{center}
\textbf{Mermaid Diagram (Code)}
\begin{verbatim}
{Shaded}
{Highlighting}[]
graph LR
    A[Large Coverage Area] {-{-}{} B[Cell Division]}
    B {-{-}{} C[Multiple Base Stations]}
    C {-{-}{} D[Frequency Reuse]}
    D {-{-}{} E[High Capacity System]}
{Highlighting}
{Shaded}
\end{verbatim}
\end{center}

\begin{itemize}
\tightlist
\item
  \textbf{Area division}: Large service area divided into smaller
  hexagonal cells
\item
  \textbf{Power control}: Low power transmitters reduce interference
\item
  \textbf{Frequency efficiency}: Same frequencies reused in distant
  cells
\item
  \textbf{Capacity increase}: More simultaneous users served
\item
  \textbf{Seamless coverage}: Continuous service across all cells
\end{itemize}

\end{solutionbox}
\begin{mnemonicbox}
``Divide Area For Better Service''

\end{mnemonicbox}
\subsection*{Question 2(a) [3 marks]}\label{q2a}

\textbf{Define full forms: (i) IMEI (ii) LTE (iii) GSM}

\begin{solutionbox}


{\def\LTcaptype{none} % do not increment counter
\vspace{-5pt}
\captionof{table}{Full Forms}
\vspace{-10pt}
\begin{longtable}[]{@{}
  >{\raggedright\arraybackslash}p{(\linewidth - 4\tabcolsep) * \real{0.4118}}
  >{\raggedright\arraybackslash}p{(\linewidth - 4\tabcolsep) * \real{0.3235}}
  >{\raggedright\arraybackslash}p{(\linewidth - 4\tabcolsep) * \real{0.2647}}@{}}
\toprule\noalign{}
\begin{minipage}[b]{\linewidth}\raggedright
Abbreviation
\end{minipage} & \begin{minipage}[b]{\linewidth}\raggedright
Full Form
\end{minipage} & \begin{minipage}[b]{\linewidth}\raggedright
Purpose
\end{minipage} \\
\midrule\noalign{}
\endhead
\bottomrule\noalign{}
\endlastfoot
IMEI & International Mobile Equipment Identity & Device
identification \\
LTE & Long Term Evolution & 4G technology standard \\
GSM & Global System for Mobile Communication & 2G cellular standard \\
\end{longtable}
}

\end{solutionbox}
\begin{mnemonicbox}
``Identity, Long-term, Global''

\end{mnemonicbox}
\subsection*{Question 2(b) [4 marks]}\label{q2b}

\textbf{Explain MAHO in detail.}

\begin{solutionbox}


{\def\LTcaptype{none} % do not increment counter
\vspace{-5pt}
\captionof{table}{MAHO Characteristics}
\vspace{-10pt}
\begin{longtable}[]{@{}ll@{}}
\toprule\noalign{}
Feature & Description \\
\midrule\noalign{}
\endhead
\bottomrule\noalign{}
\endlastfoot
Full Form & Mobile Assisted Handoff \\
Function & Mobile helps in handoff decision \\
Measurement & Signal strength monitoring \\
Reporting & Mobile reports to network \\
\end{longtable}
}

\begin{verbatim}
sequenceDiagram
    Mobile{-Base Station: Signal strength report}
    Base Station{-MSC: Handoff request}
    MSC{-Target BS: Prepare handoff}
    Target BS{-MSC: Ready confirmation}
    MSC{-Mobile: Handoff command}
\end{verbatim}

\begin{itemize}
\tightlist
\item
  \textbf{Mobile assistance}: Mobile unit measures neighboring cell
  signals
\item
  \textbf{Signal reporting}: Continuous measurement reports sent to
  network
\item
  \textbf{Decision support}: Network uses mobile data for handoff
  decisions
\item
  \textbf{Quality improvement}: Better handoff decisions with mobile
  input
\end{itemize}

\end{solutionbox}
\begin{mnemonicbox}
``Mobile Assists Network Decisions''

\end{mnemonicbox}
\subsection*{Question 2(c) [7 marks]}\label{q2c}

\textbf{Explain GSM architecture with diagram}

\begin{solutionbox}

\begin{center}
\textbf{Mermaid Diagram (Code)}
\begin{verbatim}
{Shaded}
{Highlighting}[]
graph LR
    A[Mobile Station] {-{-}{} B[Base Transceiver Station]}
    B {-{-}{} C[Base Station Controller]}
    C {-{-}{} D[Mobile Switching Center]}
    D {-{-}{} E[Home Location Register]}
    D {-{-}{} F[Visitor Location Register]}
    D {-{-}{} G[Authentication Center]}
    D {-{-}{} H[PSTN/ISDN]}
{Highlighting}
{Shaded}
\end{verbatim}
\end{center}


{\def\LTcaptype{none} % do not increment counter
\vspace{-5pt}
\captionof{table}{GSM Architecture Components}
\vspace{-10pt}
\begin{longtable}[]{@{}lll@{}}
\toprule\noalign{}
Component & Function & Purpose \\
\midrule\noalign{}
\endhead
\bottomrule\noalign{}
\endlastfoot
MS & Mobile Station & User equipment \\
BTS & Base Transceiver & Radio interface \\
BSC & Base Station Controller & Radio resource management \\
MSC & Mobile Switching Center & Call switching \\
HLR & Home Location Register & Subscriber database \\
VLR & Visitor Location Register & Temporary subscriber data \\
\end{longtable}
}

\begin{itemize}
\tightlist
\item
  \textbf{Radio subsystem}: BTS and BSC handle radio communications
\item
  \textbf{Network subsystem}: MSC, HLR, VLR manage calls and mobility
\item
  \textbf{Database management}: HLR stores permanent, VLR stores
  temporary data
\item
  \textbf{Authentication}: AuC provides security functions
\end{itemize}

\end{solutionbox}
\begin{mnemonicbox}
``Mobile Base Network Database''

\end{mnemonicbox}
\subsection*{Question 2(a) OR [3
marks]}\label{q2a}

\textbf{Explain cell splitting.}

\begin{solutionbox}


{\def\LTcaptype{none} % do not increment counter
\vspace{-5pt}
\captionof{table}{Cell Splitting Process}
\vspace{-10pt}
\begin{longtable}[]{@{}lll@{}}
\toprule\noalign{}
Step & Action & Result \\
\midrule\noalign{}
\endhead
\bottomrule\noalign{}
\endlastfoot
1 & Reduce transmit power & Smaller coverage \\
2 & Add new base stations & Fill coverage gaps \\
3 & Frequency planning & Maintain interference control \\
4 & Capacity increase & More users served \\
\end{longtable}
}

\begin{itemize}
\tightlist
\item
  \textbf{Power reduction}: Original cell power decreased to shrink
  coverage
\item
  \textbf{New cells}: Additional base stations installed in coverage
  gaps
\item
  \textbf{Capacity gain}: More cells mean higher user capacity in same
  area
\end{itemize}

\end{solutionbox}
\begin{mnemonicbox}
``Split Cells Double Capacity''

\end{mnemonicbox}
\subsection*{Question 2(b) OR [4
marks]}\label{q2b}

\textbf{What is handoff? Explain soft and hard handoffs.}

\begin{solutionbox}


{\def\LTcaptype{none} % do not increment counter
\vspace{-5pt}
\captionof{table}{Handoff Types Comparison}
\vspace{-10pt}
\begin{longtable}[]{@{}llll@{}}
\toprule\noalign{}
Type & Process & Technology & Quality \\
\midrule\noalign{}
\endhead
\bottomrule\noalign{}
\endlastfoot
Hard Handoff & Break-then-make & GSM, TDMA & Brief interruption \\
Soft Handoff & Make-then-break & CDMA & Seamless transition \\
\end{longtable}
}

\begin{center}
\textbf{Mermaid Diagram (Code)}
\begin{verbatim}
{Shaded}
{Highlighting}[]
graph LR
    A[Mobile Moving] {-{-}{} B\{Handoff Type\}}
    B {-{-}{}|Hard| C[Disconnect old, Connect new]}
    B {-{-}{}|Soft| D[Connect new, then disconnect old]}
{Highlighting}
{Shaded}
\end{verbatim}
\end{center}

\begin{itemize}
\tightlist
\item
  \textbf{Handoff definition}: Process of transferring call from one
  cell to another
\item
  \textbf{Hard handoff}: Connection broken before establishing new
  connection
\item
  \textbf{Soft handoff}: New connection established before breaking old
  one
\item
  \textbf{Quality difference}: Soft handoff provides better call quality
\end{itemize}

\end{solutionbox}
\begin{mnemonicbox}
``Hard Breaks, Soft Connects''

\end{mnemonicbox}
\subsection*{Question 2(c) OR [7
marks]}\label{q2c}

\textbf{Explain GSM signal processing with diagram}

\begin{solutionbox}

\begin{center}
\textbf{Mermaid Diagram (Code)}
\begin{verbatim}
{Shaded}
{Highlighting}[]
graph LR
    A[Voice Input] {-{-}{} B[Speech Codec]}
    B {-{-}{} C[Channel Coding]}
    C {-{-}{} D[Interleaving]}
    D {-{-}{} E[Encryption]}
    E {-{-}{} F[Burst Formatting]}
    F {-{-}{} G[Modulation]}
    G {-{-}{} H[RF Transmission]}
{Highlighting}
{Shaded}
\end{verbatim}
\end{center}


{\def\LTcaptype{none} % do not increment counter
\vspace{-5pt}
\captionof{table}{GSM Signal Processing Stages}
\vspace{-10pt}
\begin{longtable}[]{@{}lll@{}}
\toprule\noalign{}
Stage & Function & Purpose \\
\midrule\noalign{}
\endhead
\bottomrule\noalign{}
\endlastfoot
Speech Codec & Voice compression & Bandwidth efficiency \\
Channel Coding & Error correction & Transmission reliability \\
Interleaving & Burst error protection & Data integrity \\
Encryption & Security & Privacy protection \\
Modulation & RF conversion & Air interface \\
\end{longtable}
}

\begin{itemize}
\tightlist
\item
  \textbf{Speech processing}: Voice compressed using RPE-LTP codec
\item
  \textbf{Error protection}: Convolutional coding adds redundancy
\item
  \textbf{Security layer}: A5 algorithm encrypts data
\item
  \textbf{Burst structure}: Data organized in time slots
\item
  \textbf{Modulation}: GMSK modulation for RF transmission
\end{itemize}

\end{solutionbox}
\begin{mnemonicbox}
``Voice Coded Interleaved Encrypted Modulated''

\end{mnemonicbox}
\subsection*{Question 3(a) [3 marks]}\label{q3a}

\textbf{Explain cell sectoring.}

\begin{solutionbox}


{\def\LTcaptype{none} % do not increment counter
\vspace{-5pt}
\captionof{table}{Cell Sectoring Benefits}
\vspace{-10pt}
\begin{longtable}[]{@{}ll@{}}
\toprule\noalign{}
Feature & Description \\
\midrule\noalign{}
\endhead
\bottomrule\noalign{}
\endlastfoot
Antenna Pattern & Directional instead of omnidirectional \\
Sectors & 3 or 6 sectors per cell \\
Capacity & 3x or 6x capacity increase \\
Interference & Reduced co-channel interference \\
\end{longtable}
}

\begin{itemize}
\tightlist
\item
  \textbf{Directional antennas}: Replace omnidirectional with sector
  antennas
\item
  \textbf{Capacity multiplication}: Each sector treated as separate cell
\item
  \textbf{Interference reduction}: Directional pattern reduces
  interference
\end{itemize}

\end{solutionbox}
\begin{mnemonicbox}
``Sector Antennas Triple Capacity''

\end{mnemonicbox}
\subsection*{Question 3(b) [4 marks]}\label{q3b}

\textbf{Explain GSM call procedure.}

\begin{solutionbox}

\begin{verbatim}
sequenceDiagram
    Mobile{-BTS: Call request}
    BTS{-BSC: Forward request}
    BSC{-MSC: Route call}
    MSC{-HLR: Authenticate user}
    HLR{-MSC: Authentication OK}
    MSC{-PSTN: Establish connection}
\end{verbatim}


{\def\LTcaptype{none} % do not increment counter
\vspace{-5pt}
\captionof{table}{Call Setup Steps}
\vspace{-10pt}
\begin{longtable}[]{@{}lll@{}}
\toprule\noalign{}
Step & Process & Purpose \\
\midrule\noalign{}
\endhead
\bottomrule\noalign{}
\endlastfoot
1 & Authentication & User verification \\
2 & Channel allocation & Resource assignment \\
3 & Call routing & Path establishment \\
4 & Connection setup & Communication link \\
\end{longtable}
}

\begin{itemize}
\tightlist
\item
  \textbf{Authentication}: Network verifies subscriber identity
\item
  \textbf{Resource allocation}: Traffic channel assigned to call
\item
  \textbf{Routing}: Call path determined through network
\item
  \textbf{Connection}: End-to-end communication established
\end{itemize}

\end{solutionbox}
\begin{mnemonicbox}
``Authenticate Allocate Route Connect''

\end{mnemonicbox}
\subsection*{Question 3(c) [7 marks]}\label{q3c}

\textbf{Explain GPRS.}

\begin{solutionbox}


{\def\LTcaptype{none} % do not increment counter
\vspace{-5pt}
\captionof{table}{GPRS Features}
\vspace{-10pt}
\begin{longtable}[]{@{}lll@{}}
\toprule\noalign{}
Feature & Description & Benefit \\
\midrule\noalign{}
\endhead
\bottomrule\noalign{}
\endlastfoot
Technology & General Packet Radio Service & Data service \\
Data Rate & Up to 114 kbps & High speed \\
Connection & Packet switched & Always on \\
Applications & Internet, email & Data services \\
\end{longtable}
}

\begin{verbatim}
graph TB
    A[GPRS Network] {-{-} B[SGSN]}
    A {-{-} C[GGSN]}
    B {-{-} D[Packet Data]}
    C {-{-} E[Internet Gateway]}
    F[Mobile] {-{-} B}
    C {-{-} G[External Networks]}
\end{verbatim}

\begin{itemize}
\tightlist
\item
  \textbf{Packet switching}: Data transmitted in packets, not circuits
\item
  \textbf{Always-on connection}: No dial-up required for data access
\item
  \textbf{Higher speeds}: Significant improvement over circuit-switched
  data
\item
  \textbf{New nodes}: SGSN and GGSN added to GSM architecture
\item
  \textbf{Internet access}: Direct connection to IP networks
\end{itemize}

\end{solutionbox}
\begin{mnemonicbox}
``General Packet Radio Service''

\end{mnemonicbox}
\subsection*{Question 3(a) OR [3
marks]}\label{q3a}

\textbf{Explain advantage of CDMA}

\begin{solutionbox}


{\def\LTcaptype{none} % do not increment counter
\vspace{-5pt}
\captionof{table}{CDMA Advantages}
\vspace{-10pt}
\begin{longtable}[]{@{}ll@{}}
\toprule\noalign{}
Advantage & Description \\
\midrule\noalign{}
\endhead
\bottomrule\noalign{}
\endlastfoot
Capacity & Higher user capacity \\
Security & Built-in encryption \\
Quality & Better voice quality \\
Power & Efficient power control \\
\end{longtable}
}

\begin{itemize}
\tightlist
\item
  \textbf{Increased capacity}: More users per frequency band
\item
  \textbf{Enhanced security}: Spread spectrum provides natural
  encryption
\item
  \textbf{Soft handoff}: Better call quality during handoffs
\end{itemize}

\end{solutionbox}
\begin{mnemonicbox}
``Capacity Security Quality''

\end{mnemonicbox}
\subsection*{Question 3(b) OR [4
marks]}\label{q3b}

\textbf{Explain frequency hopping techniques.}

\begin{solutionbox}


{\def\LTcaptype{none} % do not increment counter
\vspace{-5pt}
\captionof{table}{Frequency Hopping Types}
\vspace{-10pt}
\begin{longtable}[]{@{}lll@{}}
\toprule\noalign{}
Type & Hopping Rate & Application \\
\midrule\noalign{}
\endhead
\bottomrule\noalign{}
\endlastfoot
Slow FH & Less than symbol rate & GSM \\
Fast FH & Greater than symbol rate & Military \\
\end{longtable}
}

\begin{center}
\textbf{Mermaid Diagram (Code)}
\begin{verbatim}
{Shaded}
{Highlighting}[]
graph LR
    A[Data] {-{-}{} B[Spread Spectrum]}
    B {-{-}{} C[Frequency Synthesizer]}
    C {-{-}{} D[Hop Sequence]}
    D {-{-}{} E[RF Transmission]}
{Highlighting}
{Shaded}
\end{verbatim}
\end{center}

\begin{itemize}
\tightlist
\item
  \textbf{Frequency hopping}: Carrier frequency changes according to
  pattern
\item
  \textbf{Interference resistance}: Reduces effect of narrowband
  interference
\item
  \textbf{Security enhancement}: Difficult to intercept hopping signals
\item
  \textbf{GSM implementation}: Slow frequency hopping used for quality
\end{itemize}

\end{solutionbox}
\begin{mnemonicbox}
``Frequency Hops For Security''

\end{mnemonicbox}
\subsection*{Question 3(c) OR [7
marks]}\label{q3c}

\textbf{Explain EDGE.}

\begin{solutionbox}


{\def\LTcaptype{none} % do not increment counter
\vspace{-5pt}
\captionof{table}{EDGE Specifications}
\vspace{-10pt}
\begin{longtable}[]{@{}lll@{}}
\toprule\noalign{}
Parameter & Value & Improvement \\
\midrule\noalign{}
\endhead
\bottomrule\noalign{}
\endlastfoot
Full Form & Enhanced Data rate for GSM Evolution & - \\
Data Rate & Up to 384 kbps & 3x GPRS \\
Modulation & 8-PSK & Higher order \\
Compatibility & GSM/GPRS & Backward compatible \\
\end{longtable}
}

\begin{verbatim}
graph TB
    A[EDGE Enhancement] {-{-} B[8{-}PSK Modulation]}
    A {-{-} C[Link Adaptation]}
    A {-{-} D[Incremental Redundancy]}
    B {-{-} E[Higher Data Rate]}
    C {-{-} F[Better Quality]}
    D {-{-} G[Error Correction]}
\end{verbatim}

\begin{itemize}
\tightlist
\item
  \textbf{Enhanced modulation}: 8-PSK instead of GMSK increases data
  rate
\item
  \textbf{Link adaptation}: Modulation scheme adapts to channel
  conditions
\item
  \textbf{Incremental redundancy}: Improved error correction mechanism
\item
  \textbf{Backward compatibility}: Works with existing GSM/GPRS
  infrastructure
\item
  \textbf{3G stepping stone}: Bridge between 2G and 3G technologies
\end{itemize}

\end{solutionbox}
\begin{mnemonicbox}
``Enhanced Data Gets Excellence''

\end{mnemonicbox}
\subsection*{Question 4(a) [3 marks]}\label{q4a}

\textbf{Draw FHSS transmitter block diagram}

\begin{solutionbox}

\begin{verbatim}
Data {-{-} Modulator {-}{-} Frequency {-}{-} RF Amp {-}{-} Antenna}
Input               Synthesizer                   
                         \^{}
                    PN Sequence
                    Generator
\end{verbatim}


{\def\LTcaptype{none} % do not increment counter
\vspace{-5pt}
\captionof{table}{FHSS Components}
\vspace{-10pt}
\begin{longtable}[]{@{}ll@{}}
\toprule\noalign{}
Component & Function \\
\midrule\noalign{}
\endhead
\bottomrule\noalign{}
\endlastfoot
PN Generator & Produces hopping sequence \\
Frequency Synthesizer & Changes carrier frequency \\
Modulator & Modulates data \\
\end{longtable}
}

\end{solutionbox}
\begin{mnemonicbox}
``Data Modulated Frequency Hops''

\end{mnemonicbox}
\subsection*{Question 4(b) [4 marks]}\label{q4b}

\textbf{Explain call processing in CDMA}

\begin{solutionbox}


{\def\LTcaptype{none} % do not increment counter
\vspace{-5pt}
\captionof{table}{CDMA Call Processing}
\vspace{-10pt}
\begin{longtable}[]{@{}lll@{}}
\toprule\noalign{}
Phase & Process & Purpose \\
\midrule\noalign{}
\endhead
\bottomrule\noalign{}
\endlastfoot
Access & System access & Initial connection \\
Authentication & Identity verification & Security \\
Traffic & Communication & Data transfer \\
Release & Call termination & Resource cleanup \\
\end{longtable}
}

\begin{itemize}
\tightlist
\item
  \textbf{System access}: Mobile acquires pilot channel and synchronizes
\item
  \textbf{Authentication}: Network verifies subscriber credentials
\item
  \textbf{Traffic state}: Active communication with power control
\item
  \textbf{Call release}: Resources freed when call ends
\end{itemize}

\end{solutionbox}
\begin{mnemonicbox}
``Access Authenticate Transfer Release''

\end{mnemonicbox}
\subsection*{Question 4(c) [7 marks]}\label{q4c}

\textbf{Draw OFDM receiver and explain its working}

\begin{solutionbox}

\begin{verbatim}
RF      {-{-} Down    {-}{-} ADC {-}{-} Remove  {-}{-} FFT {-}{-} Parallel {-}{-} Channel {-}{-} Data}
Input       Converter           Cyclic            to Serial    Decoder     Output
                                Prefix            Converter
\end{verbatim}


{\def\LTcaptype{none} % do not increment counter
\vspace{-5pt}
\captionof{table}{OFDM Receiver Functions}
\vspace{-10pt}
\begin{longtable}[]{@{}lll@{}}
\toprule\noalign{}
Component & Function & Purpose \\
\midrule\noalign{}
\endhead
\bottomrule\noalign{}
\endlastfoot
Down Converter & RF to baseband & Frequency conversion \\
ADC & Analog to digital & Signal digitization \\
Remove CP & Cyclic prefix removal & ISI elimination \\
FFT & Fast Fourier Transform & Subcarrier separation \\
Channel Decoder & Error correction & Data recovery \\
\end{longtable}
}

\begin{itemize}
\tightlist
\item
  \textbf{RF processing}: Converts received RF signal to baseband
\item
  \textbf{Digital conversion}: ADC samples the analog signal
\item
  \textbf{Prefix removal}: Cyclic prefix removed to eliminate ISI
\item
  \textbf{FFT processing}: Separates orthogonal subcarriers
\item
  \textbf{Data recovery}: Channel decoding recovers original data
\end{itemize}

\end{solutionbox}
\begin{mnemonicbox}
``Receive Convert Remove Transform Decode''

\end{mnemonicbox}
\subsection*{Question 4(a) OR [3
marks]}\label{q4a}

\textbf{Explain radiation hazard due to mobile.}

\begin{solutionbox}


{\def\LTcaptype{none} % do not increment counter
\vspace{-5pt}
\captionof{table}{Mobile Radiation Effects}
\vspace{-10pt}
\begin{longtable}[]{@{}lll@{}}
\toprule\noalign{}
Parameter & Value & Effect \\
\midrule\noalign{}
\endhead
\bottomrule\noalign{}
\endlastfoot
SAR & Specific Absorption Rate & Tissue heating \\
Frequency & 900/1800 MHz & Penetration depth \\
Power & Transmit power & Exposure level \\
\end{longtable}
}

\begin{itemize}
\tightlist
\item
  \textbf{SAR measurement}: Specific Absorption Rate measures energy
  absorption
\item
  \textbf{Thermal effects}: High SAR can cause tissue heating
\item
  \textbf{Safety limits}: International standards limit SAR values
\end{itemize}

\end{solutionbox}
\begin{mnemonicbox}
``SAR Safety Absorption Rate''

\end{mnemonicbox}
\subsection*{Question 4(b) OR [4
marks]}\label{q4b}

\textbf{Explain Li-Po type batteries used in mobile handset.}

\begin{solutionbox}


{\def\LTcaptype{none} % do not increment counter
\vspace{-5pt}
\captionof{table}{Li-Po Battery Characteristics}
\vspace{-10pt}
\begin{longtable}[]{@{}lll@{}}
\toprule\noalign{}
Feature & Description & Advantage \\
\midrule\noalign{}
\endhead
\bottomrule\noalign{}
\endlastfoot
Chemistry & Lithium Polymer & High energy density \\
Shape & Flexible form factor & Design freedom \\
Weight & Lightweight & Portability \\
Charging & Fast charging & User convenience \\
\end{longtable}
}

\begin{itemize}
\tightlist
\item
  \textbf{Polymer electrolyte}: Uses polymer instead of liquid
  electrolyte
\item
  \textbf{Flexible packaging}: Can be shaped to fit device design
\item
  \textbf{High energy density}: More capacity in smaller size
\item
  \textbf{Fast charging}: Supports rapid charging protocols
\end{itemize}

\end{solutionbox}
\begin{mnemonicbox}
``Lithium Polymer Power''

\end{mnemonicbox}
\subsection*{Question 4(c) OR [7
marks]}\label{q4c}

\textbf{Explain mobile handset block diagram.}

\begin{solutionbox}

\begin{verbatim}
graph TB
    A[Antenna] {-{-} B[RF Section]}
    B {-{-} C[Baseband Processor]}
    C {-{-} D[Audio Codec]}
    C {-{-} E[Display Controller]}
    C {-{-} F[Keypad Interface]}
    G[Battery] {-{-} H[Power Management]}
    H {-{-} B}
    H {-{-} C}
    I[SIM Interface] {-{-} C}
\end{verbatim}


{\def\LTcaptype{none} % do not increment counter
\vspace{-5pt}
\captionof{table}{Mobile Handset Components}
\vspace{-10pt}
\begin{longtable}[]{@{}lll@{}}
\toprule\noalign{}
Section & Function & Purpose \\
\midrule\noalign{}
\endhead
\bottomrule\noalign{}
\endlastfoot
RF Section & Radio frequency processing & Air interface \\
Baseband & Digital signal processing & Protocol handling \\
Audio Codec & Voice processing & Sound conversion \\
Power Management & Battery control & Power efficiency \\
SIM Interface & Subscriber identity & User authentication \\
\end{longtable}
}

\begin{itemize}
\tightlist
\item
  \textbf{RF section}: Handles transmission and reception of radio
  signals
\item
  \textbf{Baseband processor}: Implements communication protocols
\item
  \textbf{Audio subsystem}: Processes voice and audio signals
\item
  \textbf{Power management}: Controls battery usage and charging
\item
  \textbf{User interface}: Display, keypad, and user interaction
\end{itemize}

\end{solutionbox}
\begin{mnemonicbox}
``Radio Baseband Audio Power Interface''

\end{mnemonicbox}
\subsection*{Question 5(a) [3 marks]}\label{q5a}

\textbf{Compare CDMA and GSM}

\begin{solutionbox}


{\def\LTcaptype{none} % do not increment counter
\vspace{-5pt}
\captionof{table}{CDMA vs GSM Comparison}
\vspace{-10pt}
\begin{longtable}[]{@{}lll@{}}
\toprule\noalign{}
Feature & CDMA & GSM \\
\midrule\noalign{}
\endhead
\bottomrule\noalign{}
\endlastfoot
Access Method & Code Division & Time Division \\
Capacity & Higher & Lower \\
Handoff & Soft & Hard \\
SIM Card & Not required & Required \\
\end{longtable}
}

\end{solutionbox}
\begin{mnemonicbox}
``Code vs Time Division''

\end{mnemonicbox}
\subsection*{Question 5(b) [4 marks]}\label{q5b}

\textbf{Explain HSDPA.}

\begin{solutionbox}


{\def\LTcaptype{none} % do not increment counter
\vspace{-5pt}
\captionof{table}{HSDPA Features}
\vspace{-10pt}
\begin{longtable}[]{@{}ll@{}}
\toprule\noalign{}
Feature & Description \\
\midrule\noalign{}
\endhead
\bottomrule\noalign{}
\endlastfoot
Full Form & High Speed Downlink Packet Access \\
Data Rate & Up to 14.4 Mbps \\
Technology & 3.5G enhancement \\
Direction & Downlink optimization \\
\end{longtable}
}

\begin{itemize}
\tightlist
\item
  \textbf{3.5G technology}: Enhancement to 3G UMTS system
\item
  \textbf{High speed downlink}: Optimized for download applications
\item
  \textbf{Adaptive modulation}: QPSK to 16-QAM based on channel
\item
  \textbf{Fast scheduling}: 2ms scheduling intervals
\end{itemize}

\end{solutionbox}
\begin{mnemonicbox}
``High Speed Download Access''

\end{mnemonicbox}
\subsection*{Question 5(c) [7 marks]}\label{q5c}

\textbf{Explain architecture, features and advantage of Bluetooth.}

\begin{solutionbox}

\begin{center}
\textbf{Mermaid Diagram (Code)}
\begin{verbatim}
{Shaded}
{Highlighting}[]
graph LR
    A[Application Layer] {-{-}{} B[L2CAP]}
    B {-{-}{} C[HCI]}
    C {-{-}{} D[Link Manager]}
    D {-{-}{} E[Baseband]}
    E {-{-}{} F[Radio Layer]}
{Highlighting}
{Shaded}
\end{verbatim}
\end{center}


{\def\LTcaptype{none} % do not increment counter
\vspace{-5pt}
\captionof{table}{Bluetooth Features}
\vspace{-10pt}
\begin{longtable}[]{@{}lll@{}}
\toprule\noalign{}
Feature & Description & Advantage \\
\midrule\noalign{}
\endhead
\bottomrule\noalign{}
\endlastfoot
Range & 10 meters & Personal area network \\
Frequency & 2.4 GHz ISM & Unlicensed band \\
Topology & Star/Scatternet & Flexible connections \\
Power & Low power & Battery efficiency \\
\end{longtable}
}


{\def\LTcaptype{none} % do not increment counter
\vspace{-5pt}
\captionof{table}{Bluetooth Applications}
\vspace{-10pt}
\begin{longtable}[]{@{}ll@{}}
\toprule\noalign{}
Application & Use Case \\
\midrule\noalign{}
\endhead
\bottomrule\noalign{}
\endlastfoot
Audio & Wireless headphones \\
Data & File transfer \\
Input & Wireless keyboard/mouse \\
Networking & Internet sharing \\
\end{longtable}
}

\begin{itemize}
\tightlist
\item
  \textbf{Short range}: Designed for personal area networks
\item
  \textbf{Low power}: Optimized for battery-powered devices
\item
  \textbf{Frequency hopping}: 79 channels for interference resistance
\item
  \textbf{Master-slave}: One master can connect to 7 slaves
\item
  \textbf{Applications}: Audio, data transfer, input devices
\end{itemize}

\end{solutionbox}
\begin{mnemonicbox}
``Blue Personal Area Network''

\end{mnemonicbox}
\subsection*{Question 5(a) OR [3
marks]}\label{q5a}

\textbf{Explain basic concept of RFID.}

\begin{solutionbox}


{\def\LTcaptype{none} % do not increment counter
\vspace{-5pt}
\captionof{table}{RFID Components}
\vspace{-10pt}
\begin{longtable}[]{@{}ll@{}}
\toprule\noalign{}
Component & Function \\
\midrule\noalign{}
\endhead
\bottomrule\noalign{}
\endlastfoot
RFID Tag & Stores identification data \\
RFID Reader & Reads tag information \\
Antenna & RF communication \\
Backend System & Data processing \\
\end{longtable}
}

\begin{itemize}
\tightlist
\item
  \textbf{Radio frequency identification}: Uses RF waves for
  identification
\item
  \textbf{Contactless operation}: No physical contact required
\item
  \textbf{Automatic identification}: Reads tags automatically in range
\end{itemize}

\end{solutionbox}
\begin{mnemonicbox}
``Radio Frequency Identifies''

\end{mnemonicbox}
\subsection*{Question 5(b) OR [4
marks]}\label{q5b}

\textbf{Explain architecture of 5G system.}

\begin{solutionbox}


{\def\LTcaptype{none} % do not increment counter
\vspace{-5pt}
\captionof{table}{5G Architecture Components}
\vspace{-10pt}
\begin{longtable}[]{@{}ll@{}}
\toprule\noalign{}
Component & Function \\
\midrule\noalign{}
\endhead
\bottomrule\noalign{}
\endlastfoot
gNodeB & 5G base station \\
AMF & Access and Mobility Function \\
SMF & Session Management Function \\
UPF & User Plane Function \\
\end{longtable}
}

\begin{itemize}
\tightlist
\item
  \textbf{Service-based architecture}: Modular network functions
\item
  \textbf{Network slicing}: Virtual networks for different services
\item
  \textbf{Edge computing}: Processing closer to users
\item
  \textbf{Massive MIMO}: Multiple antenna technology
\end{itemize}

\end{solutionbox}
\begin{mnemonicbox}
``Service Based Network Slicing''

\end{mnemonicbox}
\subsection*{Question 5(c) OR [7
marks]}\label{q5c}

\textbf{Explain MANET in detail.}

\begin{solutionbox}


{\def\LTcaptype{none} % do not increment counter
\vspace{-5pt}
\captionof{table}{MANET Characteristics}
\vspace{-10pt}
\begin{longtable}[]{@{}lll@{}}
\toprule\noalign{}
Feature & Description & Benefit \\
\midrule\noalign{}
\endhead
\bottomrule\noalign{}
\endlastfoot
Infrastructure & Infrastructure-less & No base stations needed \\
Mobility & Mobile nodes & Dynamic topology \\
Routing & Multi-hop routing & Extended coverage \\
Self-organizing & Automatic configuration & Easy deployment \\
\end{longtable}
}

\begin{center}
\textbf{Mermaid Diagram (Code)}
\begin{verbatim}
{Shaded}
{Highlighting}[]
graph LR
    A[Node A] {-{-}{} B[Node B]}
    B {-{-}{} C[Node C]}
    A {-{-}{} D[Node D]}
    C {-{-}{} E[Node E]}
    D {-{-}{} E}
    B {-{-}{} E}
{Highlighting}
{Shaded}
\end{verbatim}
\end{center}


{\def\LTcaptype{none} % do not increment counter
\vspace{-5pt}
\captionof{table}{MANET vs Cellular Network}
\vspace{-10pt}
\begin{longtable}[]{@{}lll@{}}
\toprule\noalign{}
Parameter & MANET & Cellular \\
\midrule\noalign{}
\endhead
\bottomrule\noalign{}
\endlastfoot
Infrastructure & None & Base stations required \\
Topology & Dynamic & Fixed \\
Range & Multi-hop & Single hop \\
Cost & Low & High infrastructure cost \\
\end{longtable}
}

\begin{itemize}
\tightlist
\item
  \textbf{Mobile Ad-hoc Network}: Self-configuring network of mobile
  devices
\item
  \textbf{No infrastructure}: Nodes communicate directly without base
  stations
\item
  \textbf{Dynamic routing}: Routes change as nodes move
\item
  \textbf{Multi-hop communication}: Messages relay through intermediate
  nodes
\item
  \textbf{Applications}: Military, disaster recovery, sensor networks
\end{itemize}

\end{solutionbox}
\begin{mnemonicbox}
``Mobile Adhoc Network''

\end{mnemonicbox}

\end{document}
