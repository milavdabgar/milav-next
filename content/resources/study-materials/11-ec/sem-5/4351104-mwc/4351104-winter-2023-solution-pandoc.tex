\documentclass[10pt,a4paper]{article}

% content/resources/templates/preamble.tex
\usepackage[margin=0.6in]{geometry}
\author{Milav Dabgar}
\usepackage{amsmath,amssymb,amsthm}
\usepackage{booktabs}
\usepackage{multirow}
\usepackage{xcolor}
\usepackage{tcolorbox}
\tcbuselibrary{breakable,skins}
\usepackage[colorlinks=true,linkcolor=blue]{hyperref}
\usepackage{titlesec}
\usepackage{enumitem}
\usepackage{tikz}
\usepackage{pgfplots}
\usepackage{circuitikz}
\usepackage[version=4]{mhchem}
\usepackage{longtable}
\usepackage{array}
\usepackage{float}
\usepackage{caption}
\usepackage{listings}

\lstset{
  basicstyle=\small\ttfamily,
  breaklines=true,
  breakatwhitespace=false,
  postbreak=\mbox{\textcolor{red}{$\hookrightarrow$}\space},
  float=false,
  numbers=left,
  numberstyle=\tiny\color{gray},
  numbersep=10pt,
  xleftmargin=2em,
  keywordstyle=\color{blue},
  commentstyle=\color{green!60!black},
  stringstyle=\color{purple},
  backgroundcolor=\color{gray!5},
  showstringspaces=false,
  tabsize=2,
  captionpos=b,
  keepspaces=true,
  columns=flexible
}

\pgfplotsset{compat=1.18}
\usetikzlibrary{shapes,arrows,positioning,calc,patterns,decorations.pathmorphing,decorations.markings,arrows.meta}

% Color scheme
\definecolor{headcolor}{RGB}{0,102,204}
\definecolor{keycolor}{RGB}{220,20,60}
\definecolor{solutioncolor}{RGB}{34,139,34}
\definecolor{mnemoniccolor}{RGB}{148,0,211}
\definecolor{codecolor}{RGB}{0,0,100}

% Spacing
\setlength{\parskip}{3pt}
\setlist[itemize]{nosep}
\setlist[enumerate]{nosep}

% Title formatting
\titleformat{\section}{\Large\bfseries\color{headcolor}}{\thesection}{1em}{}
\titleformat{\subsection}{\large\bfseries\color{headcolor}}{\thesubsection}{1em}{}

% Pandoc tightlist compatibility
\providecommand{\tightlist}{%
  \setlength{\itemsep}{0pt}\setlength{\parskip}{0pt}}

% Pandoc longtable compatibility
\newcounter{none}
\def\thenone{}


% content/resources/templates/english-boxes.tex
% This file is currently empty - it exists to maintain consistency with the import structure.
% Add custom environments here if needed in the future.


\begin{document}

\begin{center}
{\Huge\bfseries\color{headcolor} Subject Name Solutions}\\[5pt]
{\LARGE 4351104 -- Winter 2023}\\[3pt]
{\large Semester 1 Study Material}\\[3pt]
{\normalsize\textit{Detailed Solutions and Explanations}}
\end{center}

\vspace{10pt}

\subsection*{Question 1(a) [3 marks]}\label{q1a}

\textbf{Draw \& Explain umbrella cell.}

\begin{solutionbox}

\begin{center}
\textbf{Mermaid Diagram (Code)}
\begin{verbatim}
{Shaded}
{Highlighting}[]
graph TD
    A[Large Coverage Area] {-{-}{} B[Umbrella Cell Tower]}
    B {-{-}{} C[Micro Cell 1]}
    B {-{-}{} D[Micro Cell 2]}
    B {-{-}{} E[Micro Cell 3]}
    C {-{-}{} F[Users in Dense Area]}
    D {-{-}{} G[Users in Dense Area]}
    E {-{-}{} H[Users in Dense Area]}
{Highlighting}
{Shaded}
\end{verbatim}
\end{center}

\begin{itemize}
\tightlist
\item
  \textbf{Umbrella Cell}: Large coverage cell overlaying smaller cells
\item
  \textbf{Purpose}: Handles overflow traffic from micro/pico cells
\item
  \textbf{Coverage}: Provides backup coverage for high-traffic areas
\end{itemize}

\end{solutionbox}
\begin{mnemonicbox}
``Under My Big Umbrella''

\end{mnemonicbox}
\subsection*{Question 1(b) [4 marks]}\label{q1b}

\textbf{Define full forms: (i) CCH (ii) TCH (iii) SCH (iv) BCCH}

\begin{solutionbox}

{\def\LTcaptype{none} % do not increment counter
\begin{longtable}[]{@{}lll@{}}
\toprule\noalign{}
Acronym & Full Form & Function \\
\midrule\noalign{}
\endhead
\bottomrule\noalign{}
\endlastfoot
CCH & Control Channel & Carries control information \\
TCH & Traffic Channel & Carries voice/data traffic \\
SCH & Synchronization Channel & Provides timing sync \\
BCCH & Broadcast Control Channel & Broadcasts system info \\
\end{longtable}
}

\end{solutionbox}
\begin{mnemonicbox}
``Control Traffic Sync Broadcast''

\end{mnemonicbox}
\subsection*{Question 1(c) [7 marks]}\label{q1c}

\textbf{What is cell? Explain different types of cells.}

\begin{solutionbox}
\textbf{Cell} is the basic coverage area served by one
base station in cellular communication.

{\def\LTcaptype{none} % do not increment counter
\begin{longtable}[]{@{}llll@{}}
\toprule\noalign{}
Cell Type & Coverage & Power & Usage \\
\midrule\noalign{}
\endhead
\bottomrule\noalign{}
\endlastfoot
\textbf{Macro Cell} & 1-30 km & High & Rural areas \\
\textbf{Micro Cell} & 100m-2km & Medium & Urban areas \\
\textbf{Pico Cell} & 10-100m & Low & Indoor coverage \\
\textbf{Femto Cell} & 10-30m & Very Low & Home/office \\
\end{longtable}
}

\begin{center}
\textbf{Mermaid Diagram (Code)}
\begin{verbatim}
{Shaded}
{Highlighting}[]
graph TD
    A[Macro Cell] {-{-}{} B[Large Area Coverage]}
    C[Micro Cell] {-{-}{} D[City Coverage]}
    E[Pico Cell] {-{-}{} F[Building Coverage]}
    G[Femto Cell] {-{-}{} H[Room Coverage]}
{Highlighting}
{Shaded}
\end{verbatim}
\end{center}

\begin{itemize}
\tightlist
\item
  \textbf{Function}: Each cell provides wireless service to mobile users
\item
  \textbf{Frequency Reuse}: Same frequencies used in non-adjacent cells
\item
  \textbf{Handoff}: Users move between cells seamlessly
\end{itemize}

\end{solutionbox}
\begin{mnemonicbox}
``Many Mobile People Find coverage''

\end{mnemonicbox}
\subsection*{Question 1(c OR) [7
marks]}\label{question-1c-or-7-marks}

\textbf{What is handoff? Explain soft and hard handoffs.}

\begin{solutionbox}
\textbf{Handoff} is the process of transferring an
ongoing call from one cell to another as mobile moves.

{\def\LTcaptype{none} % do not increment counter
\begin{longtable}[]{@{}lll@{}}
\toprule\noalign{}
Feature & Hard Handoff & Soft Handoff \\
\midrule\noalign{}
\endhead
\bottomrule\noalign{}
\endlastfoot
\textbf{Connection} & Break-before-make & Make-before-break \\
\textbf{Channels} & One at a time & Multiple simultaneously \\
\textbf{Technology} & GSM, TDMA & CDMA \\
\textbf{Quality} & Brief interruption & Seamless transition \\
\end{longtable}
}

\begin{verbatim}
sequenceDiagram
    participant M as Mobile
    participant BS1 as Base Station 1
    participant BS2 as Base Station 2

    Note over M,BS2: Hard Handoff
    M{-BS1: Connected}
    BS1{-{-}M: Signal weakens}
    BS1{-BS2: Handoff request}
    M{-BS2: New connection}
    
    Note over M,BS2: Soft Handoff
    M{-BS1: Connected}
    M{-BS2: Dual connection}
    M{-{-}BS1: Drop weak signal}
\end{verbatim}

\begin{itemize}
\tightlist
\item
  \textbf{Initiation}: Based on signal strength measurements
\item
  \textbf{MAHO}: Mobile Assisted Handoff improves decision accuracy
\end{itemize}

\end{solutionbox}
\begin{mnemonicbox}
``Hard Hurts, Soft Smooth''

\end{mnemonicbox}
\subsection*{Question 2(a) [3 marks]}\label{q2a}

\textbf{Define full forms: (i) SIM (ii) LTE (iii) WCDMA}

\begin{solutionbox}

{\def\LTcaptype{none} % do not increment counter
\begin{longtable}[]{@{}lll@{}}
\toprule\noalign{}
Acronym & Full Form & Purpose \\
\midrule\noalign{}
\endhead
\bottomrule\noalign{}
\endlastfoot
SIM & Subscriber Identity Module & User authentication \\
LTE & Long Term Evolution & 4G technology \\
WCDMA & Wideband Code Division Multiple Access & 3G standard \\
\end{longtable}
}

\end{solutionbox}
\begin{mnemonicbox}
``Subscriber's Long Wideband connection''

\end{mnemonicbox}
\subsection*{Question 2(b) [4 marks]}\label{q2b}

\textbf{Draw mobile handset block diagram.}

\begin{solutionbox}

\begin{center}
\textbf{Mermaid Diagram (Code)}
\begin{verbatim}
{Shaded}
{Highlighting}[]
graph TD
    A[Antenna] {-{-}{} B[RF Section]}
    B {-{-}{} C[Baseband Processor]}
    C {-{-}{} D[Audio Section]}
    C {-{-}{} E[Display/Keypad]}
    C {-{-}{} F[Memory]}
    G[Battery] {-{-}{} H[Power Management]}
    H {-{-}{} B}
    H {-{-}{} C}
    H {-{-}{} D}
{Highlighting}
{Shaded}
\end{verbatim}
\end{center}

\begin{itemize}
\tightlist
\item
  \textbf{RF Section}: Transmits/receives radio signals
\item
  \textbf{Baseband}: Processes digital signals and protocols
\item
  \textbf{Audio}: Handles voice input/output
\item
  \textbf{Power Management}: Controls battery usage efficiently
\end{itemize}

\end{solutionbox}
\begin{mnemonicbox}
``Radio Baseband Audio Power''

\end{mnemonicbox}
\subsection*{Question 2(c) [7 marks]}\label{q2c}

\textbf{Explain GSM architecture with diagram.}

\begin{solutionbox}

\begin{center}
\textbf{Mermaid Diagram (Code)}
\begin{verbatim}
{Shaded}
{Highlighting}[]
graph LR
    A[MS] {-{-}{} B[BTS]}
    B {-{-}{} C[BSC]}
    C {-{-}{} D[MSC]}
    D {-{-}{} E[HLR]}
    D {-{-}{} F[VLR]}
    D {-{-}{} G[AuC]}
    D {-{-}{} H[PSTN]}

    subgraph BSS
    B
    C
    end
    
    subgraph NSS
    D
    E
    F
    G
    end
{Highlighting}
{Shaded}
\end{verbatim}
\end{center}

{\def\LTcaptype{none} % do not increment counter
\begin{longtable}[]{@{}ll@{}}
\toprule\noalign{}
Component & Function \\
\midrule\noalign{}
\endhead
\bottomrule\noalign{}
\endlastfoot
\textbf{MS} & Mobile Station (handset) \\
\textbf{BTS} & Base Transceiver Station \\
\textbf{BSC} & Base Station Controller \\
\textbf{MSC} & Mobile Switching Center \\
\textbf{HLR} & Home Location Register \\
\textbf{VLR} & Visitor Location Register \\
\end{longtable}
}

\begin{itemize}
\tightlist
\item
  \textbf{BSS}: Base Station Subsystem handles radio interface
\item
  \textbf{NSS}: Network Switching Subsystem manages calls
\item
  \textbf{Authentication}: AuC verifies subscriber identity
\end{itemize}

\end{solutionbox}
\begin{mnemonicbox}
``Mobile Base Network calls Home''

\end{mnemonicbox}
\subsection*{Question 2(a OR) [3
marks]}\label{question-2a-or-3-marks}

\textbf{Define full forms: (i) RSSI (ii) MAHO (iii) NCHO}

\begin{solutionbox}

{\def\LTcaptype{none} % do not increment counter
\begin{longtable}[]{@{}
  >{\raggedright\arraybackslash}p{(\linewidth - 4\tabcolsep) * \real{0.3000}}
  >{\raggedright\arraybackslash}p{(\linewidth - 4\tabcolsep) * \real{0.3667}}
  >{\raggedright\arraybackslash}p{(\linewidth - 4\tabcolsep) * \real{0.3333}}@{}}
\toprule\noalign{}
\begin{minipage}[b]{\linewidth}\raggedright
Acronym
\end{minipage} & \begin{minipage}[b]{\linewidth}\raggedright
Full Form
\end{minipage} & \begin{minipage}[b]{\linewidth}\raggedright
Function
\end{minipage} \\
\midrule\noalign{}
\endhead
\bottomrule\noalign{}
\endlastfoot
RSSI & Received Signal Strength Indicator & Signal quality
measurement \\
MAHO & Mobile Assisted Handoff & Mobile helps handoff decision \\
NCHO & Network Controlled Handoff & Network decides handoff \\
\end{longtable}
}

\end{solutionbox}
\begin{mnemonicbox}
``Received Mobile Network signals''

\end{mnemonicbox}
\subsection*{Question 2(b OR) [4
marks]}\label{question-2b-or-4-marks}

\textbf{Draw baseband section block diagram.}

\begin{solutionbox}

\begin{center}
\textbf{Mermaid Diagram (Code)}
\begin{verbatim}
{Shaded}
{Highlighting}[]
graph LR
    A[ADC/DAC] {-{-}{} B[DSP]}
    B {-{-}{} C[Channel Codec]}
    C {-{-}{} D[Speech Codec]}
    D {-{-}{} E[Audio Interface]}
    B {-{-}{} F[Protocol Stack]}
    F {-{-}{} G[Control Interface]}
{Highlighting}
{Shaded}
\end{verbatim}
\end{center}

\begin{itemize}
\tightlist
\item
  \textbf{ADC/DAC}: Analog to Digital conversion
\item
  \textbf{DSP}: Digital Signal Processor
\item
  \textbf{Channel Codec}: Error correction coding
\item
  \textbf{Speech Codec}: Voice compression/decompression
\end{itemize}

\end{solutionbox}
\begin{mnemonicbox}
``Analog Digital Speech Protocol''

\end{mnemonicbox}
\subsection*{Question 2(c OR) [7
marks]}\label{question-2c-or-7-marks}

\textbf{Explain GSM signal processing with diagram.}

\begin{solutionbox}

\begin{center}
\textbf{Mermaid Diagram (Code)}
\begin{verbatim}
{Shaded}
{Highlighting}[]
graph LR
    A[Speech] {-{-}{} B[Speech Codec]}
    B {-{-}{} C[Channel Codec]}
    C {-{-}{} D[Interleaving]}
    D {-{-}{} E[Burst Formatter]}
    E {-{-}{} F[GMSK Modulator]}
    F {-{-}{} G[RF Transmitter]}
{Highlighting}
{Shaded}
\end{verbatim}
\end{center}

{\def\LTcaptype{none} % do not increment counter
\begin{longtable}[]{@{}lll@{}}
\toprule\noalign{}
Stage & Function & Purpose \\
\midrule\noalign{}
\endhead
\bottomrule\noalign{}
\endlastfoot
\textbf{Speech Codec} & Compress voice to 13 kbps & Bandwidth
efficiency \\
\textbf{Channel Codec} & Add error correction & Signal reliability \\
\textbf{Interleaving} & Distribute burst errors & Error protection \\
\textbf{GMSK} & Gaussian MSK modulation & Spectral efficiency \\
\end{longtable}
}

\begin{itemize}
\tightlist
\item
  \textbf{Processing Rate}: 270.833 kbps gross bit rate
\item
  \textbf{Frame Structure}: 8 time slots per TDMA frame
\item
  \textbf{Frequency Hopping}: 217 hops per second
\end{itemize}

\end{solutionbox}
\begin{mnemonicbox}
``Speech Channel Interleaves Modulated Radio''

\end{mnemonicbox}
\subsection*{Question 3(a) [3 marks]}\label{q3a}

\textbf{Explain cell splitting.}

\begin{solutionbox}
Cell splitting divides congested cells into smaller
cells to increase capacity.

\begin{itemize}
\tightlist
\item
  \textbf{Process}: Replace high-power cell with multiple low-power
  cells
\item
  \textbf{Benefit}: Increases system capacity by frequency reuse
\item
  \textbf{Implementation}: Reduce antenna height and transmit power
\end{itemize}

\end{solutionbox}
\begin{mnemonicbox}
``Split Small Cells''

\end{mnemonicbox}
\subsection*{Question 3(b) [4 marks]}\label{q3b}

\textbf{Explain Li-Ion type batteries used in mobile handset with its
advantages and disadvantages.}

\begin{solutionbox}

{\def\LTcaptype{none} % do not increment counter
\begin{longtable}[]{@{}ll@{}}
\toprule\noalign{}
Advantages & Disadvantages \\
\midrule\noalign{}
\endhead
\bottomrule\noalign{}
\endlastfoot
\textbf{High energy density} & \textbf{Safety concerns} \\
\textbf{No memory effect} & \textbf{Degradation over time} \\
\textbf{Low self-discharge} & \textbf{Temperature sensitive} \\
\textbf{Lightweight} & \textbf{Expensive} \\
\end{longtable}
}

\begin{itemize}
\tightlist
\item
  \textbf{Chemistry}: Lithium ions move between electrodes
\item
  \textbf{Voltage}: 3.7V nominal per cell
\item
  \textbf{Capacity}: Measured in mAh (milliampere-hours)
\end{itemize}

\end{solutionbox}
\begin{mnemonicbox}
``Light Ion Energy Safety''

\end{mnemonicbox}
\subsection*{Question 3(c) [7 marks]}\label{q3c}

\textbf{Explain GPRS.}

\begin{solutionbox}
\textbf{GPRS} (General Packet Radio Service) provides
packet-switched data service over GSM.

{\def\LTcaptype{none} % do not increment counter
\begin{longtable}[]{@{}ll@{}}
\toprule\noalign{}
Feature & Specification \\
\midrule\noalign{}
\endhead
\bottomrule\noalign{}
\endlastfoot
\textbf{Data Rate} & Up to 171.2 kbps \\
\textbf{Technology} & Packet switching \\
\textbf{Channels} & Uses multiple time slots \\
\textbf{Billing} & Based on data volume \\
\end{longtable}
}

\begin{center}
\textbf{Mermaid Diagram (Code)}
\begin{verbatim}
{Shaded}
{Highlighting}[]
graph LR
    A[Mobile] {-{-}{} B[BSS]}
    B {-{-}{} C[PCU]}
    C {-{-}{} D[SGSN]}
    D {-{-}{} E[GGSN]}
    E {-{-}{} F[Internet]}
{Highlighting}
{Shaded}
\end{verbatim}
\end{center}

\begin{itemize}
\tightlist
\item
  \textbf{PCU}: Packet Control Unit manages packet data
\item
  \textbf{SGSN}: Serving GPRS Support Node
\item
  \textbf{GGSN}: Gateway GPRS Support Node
\item
  \textbf{Classes}: Class 1-12 with different speed/slot combinations
\end{itemize}

\end{solutionbox}
\begin{mnemonicbox}
``General Packet Radio Service''

\end{mnemonicbox}
\subsection*{Question 3(a OR) [3
marks]}\label{question-3a-or-3-marks}

\textbf{Explain cell sectoring.}

\begin{solutionbox}
Cell sectoring divides omnidirectional cell into
sectors using directional antennas.

\begin{itemize}
\tightlist
\item
  \textbf{Common}: 3-sector (120^\circ) or 6-sector (60^\circ) configurations
\item
  \textbf{Benefit}: Reduces co-channel interference
\item
  \textbf{Implementation}: Directional antennas at same site
\end{itemize}

\end{solutionbox}
\begin{mnemonicbox}
``Sector Reduces Interference''

\end{mnemonicbox}
\subsection*{Question 3(b OR) [4
marks]}\label{question-3b-or-4-marks}

\textbf{Explain Li-Po type batteries used in mobile handset with its
advantages and disadvantages.}

\begin{solutionbox}

{\def\LTcaptype{none} % do not increment counter
\begin{longtable}[]{@{}ll@{}}
\toprule\noalign{}
Advantages & Disadvantages \\
\midrule\noalign{}
\endhead
\bottomrule\noalign{}
\endlastfoot
\textbf{Flexible shape} & \textbf{Lower energy density} \\
\textbf{Ultra-thin design} & \textbf{Shorter lifespan} \\
\textbf{Lightweight} & \textbf{Safety risks} \\
\textbf{No memory effect} & \textbf{Higher cost} \\
\end{longtable}
}

\begin{itemize}
\tightlist
\item
  \textbf{Technology}: Lithium Polymer electrolyte
\item
  \textbf{Form Factor}: Can be molded into various shapes
\item
  \textbf{Voltage}: 3.7V nominal per cell
\end{itemize}

\end{solutionbox}
\begin{mnemonicbox}
``Polymer Flexible Thin Light''

\end{mnemonicbox}
\subsection*{Question 3(c OR) [7
marks]}\label{question-3c-or-7-marks}

\textbf{Explain EDGE.}

\begin{solutionbox}
\textbf{EDGE} (Enhanced Data rates for GSM Evolution)
improves GSM data rates.

{\def\LTcaptype{none} % do not increment counter
\begin{longtable}[]{@{}lll@{}}
\toprule\noalign{}
Parameter & GSM & EDGE \\
\midrule\noalign{}
\endhead
\bottomrule\noalign{}
\endlastfoot
\textbf{Modulation} & GMSK & 8-PSK \\
\textbf{Data Rate} & 9.6 kbps & Up to 384 kbps \\
\textbf{Error Correction} & Basic & Advanced \\
\textbf{Spectrum} & Same as GSM & Same as GSM \\
\end{longtable}
}

\begin{center}
\textbf{Mermaid Diagram (Code)}
\begin{verbatim}
{Shaded}
{Highlighting}[]
graph LR
    A[Data] {-{-}{} B[Adaptive Coding]}
    B {-{-}{} C[8{-}PSK Modulation]}
    C {-{-}{} D[Link Adaptation]}
    D {-{-}{} E[Enhanced Reception]}
{Highlighting}
{Shaded}
\end{verbatim}
\end{center}

\begin{itemize}
\tightlist
\item
  \textbf{8-PSK}: 8-Phase Shift Keying provides 3 bits per symbol
\item
  \textbf{Link Adaptation}: Adjusts coding scheme based on channel
  quality
\item
  \textbf{Incremental Redundancy}: Improves error correction efficiency
\end{itemize}

\end{solutionbox}
\begin{mnemonicbox}
``Enhanced Data GSM Evolution''

\end{mnemonicbox}
\subsection*{Question 4(a) [3 marks]}\label{q4a}

\textbf{Draw DSSS transmitter and receiver block diagram.}

\begin{solutionbox}

\begin{verbatim}
Transmitter:
Data {-{-} Spreader {-}{-} Modulator {-}{-} RF Out}
         \^{}
         |
      PN Code

Receiver:
RF In {-{-} Demodulator {-}{-} Despreader {-}{-} Data Out}
                           \^{}
                           |
                        PN Code
\end{verbatim}

\begin{itemize}
\tightlist
\item
  \textbf{Spreader}: Multiplies data with PN sequence
\item
  \textbf{Despreader}: Correlates received signal with same PN code
\item
  \textbf{Processing Gain}: Ratio of spread to original bandwidth
\end{itemize}

\end{solutionbox}
\begin{mnemonicbox}
``Direct Sequence Spread Spectrum''

\end{mnemonicbox}
\subsection*{Question 4(b) [4 marks]}\label{q4b}

\textbf{Compare CDMA and GSM.}

\begin{solutionbox}

{\def\LTcaptype{none} % do not increment counter
\begin{longtable}[]{@{}lll@{}}
\toprule\noalign{}
Parameter & CDMA & GSM \\
\midrule\noalign{}
\endhead
\bottomrule\noalign{}
\endlastfoot
\textbf{Multiple Access} & Code Division & Time Division \\
\textbf{Capacity} & Higher (soft capacity) & Fixed capacity \\
\textbf{Handoff} & Soft handoff & Hard handoff \\
\textbf{Power Control} & Critical & Less critical \\
\textbf{Frequency Planning} & Not required & Required \\
\textbf{Voice Quality} & Better & Good \\
\end{longtable}
}

\end{solutionbox}
\begin{mnemonicbox}
``Code Division vs Time Division''

\end{mnemonicbox}
\subsection*{Question 4(c) [7 marks]}\label{q4c}

\textbf{Explain concept of spread spectrum with applications.}

\begin{solutionbox}
\textbf{Spread Spectrum} spreads signal bandwidth much
wider than required for data transmission.

\begin{center}
\textbf{Mermaid Diagram (Code)}
\begin{verbatim}
{Shaded}
{Highlighting}[]
graph LR
    A[Narrowband Signal] {-{-}{} B[Spreading Code]}
    B {-{-}{} C[Wideband Signal]}
    C {-{-}{} D[Transmission]}
    D {-{-}{} E[Despreading]}
    E {-{-}{} F[Original Signal]}
{Highlighting}
{Shaded}
\end{verbatim}
\end{center}

{\def\LTcaptype{none} % do not increment counter
\begin{longtable}[]{@{}lll@{}}
\toprule\noalign{}
Type & Method & Application \\
\midrule\noalign{}
\endhead
\bottomrule\noalign{}
\endlastfoot
\textbf{DSSS} & PN sequence multiplication & CDMA, WiFi \\
\textbf{FHSS} & Frequency hopping & Bluetooth \\
\textbf{THSS} & Time hopping & UWB systems \\
\end{longtable}
}

\textbf{Benefits}:

\begin{itemize}
\tightlist
\item
  \textbf{Anti-jamming}: Resistant to interference
\item
  \textbf{Low Power Density}: Difficult to detect
\item
  \textbf{Multiple Access}: Many users share spectrum
\item
  \textbf{Multipath Resistance}: Resolves delayed signals
\end{itemize}

\textbf{Applications}: GPS, WiFi, Bluetooth, Military communications

\end{solutionbox}
\begin{mnemonicbox}
``Spread Signal Spectrum Security''

\end{mnemonicbox}
\subsection*{Question 4(a OR) [3
marks]}\label{question-4a-or-3-marks}

\textbf{Draw FHSS transmitter block diagram.}

\begin{solutionbox}

\begin{verbatim}
Data {-{-} Modulator {-}{-} Frequency {-}{-} RF Out}
                       Synthesizer
                           \^{}
                           |
                    Hopping Sequence
                       Generator
\end{verbatim}

\begin{itemize}
\tightlist
\item
  \textbf{Frequency Synthesizer}: Changes carrier frequency rapidly
\item
  \textbf{Hopping Sequence}: Pseudo-random frequency pattern
\item
  \textbf{Dwell Time}: Time spent on each frequency
\end{itemize}

\end{solutionbox}
\begin{mnemonicbox}
``Frequency Hopping Spread Spectrum''

\end{mnemonicbox}
\subsection*{Question 4(b OR) [4
marks]}\label{question-4b-or-4-marks}

\textbf{Explain call processing in CDMA.}

\begin{solutionbox}

{\def\LTcaptype{none} % do not increment counter
\begin{longtable}[]{@{}lll@{}}
\toprule\noalign{}
Phase & Process & Description \\
\midrule\noalign{}
\endhead
\bottomrule\noalign{}
\endlastfoot
\textbf{System Access} & Power control & Mobile adjusts power \\
\textbf{Call Setup} & Channel assignment & Assign Walsh code \\
\textbf{Traffic} & Soft handoff & Multiple base stations \\
\textbf{Call Release} & Power down & Gradual power reduction \\
\end{longtable}
}

\begin{itemize}
\tightlist
\item
  \textbf{Rake Receiver}: Combines multipath signals
\item
  \textbf{Power Control}: 800 times per second
\item
  \textbf{Soft Capacity}: Degrades gracefully with load
\end{itemize}

\end{solutionbox}
\begin{mnemonicbox}
``Code Division Multiple Access''

\end{mnemonicbox}
\subsection*{Question 4(c OR) [7
marks]}\label{question-4c-or-7-marks}

\textbf{Explain HSDPA.}

\begin{solutionbox}
\textbf{HSDPA} (High Speed Downlink Packet Access)
enhances WCDMA downlink data rates.

{\def\LTcaptype{none} % do not increment counter
\begin{longtable}[]{@{}ll@{}}
\toprule\noalign{}
Feature & Enhancement \\
\midrule\noalign{}
\endhead
\bottomrule\noalign{}
\endlastfoot
\textbf{Data Rate} & Up to 14.4 Mbps \\
\textbf{Modulation} & 16-QAM \\
\textbf{HARQ} & Hybrid ARQ \\
\textbf{Fast Scheduling} & 2ms TTI \\
\end{longtable}
}

\begin{center}
\textbf{Mermaid Diagram (Code)}
\begin{verbatim}
{Shaded}
{Highlighting}[]
graph LR
    A[NodeB] {-{-}{} B[HS{-}DSCH]}
    B {-{-}{} C[16{-}QAM]}
    C {-{-}{} D[HARQ]}
    D {-{-}{} E[Mobile]}
{Highlighting}
{Shaded}
\end{verbatim}
\end{center}

\begin{itemize}
\tightlist
\item
  \textbf{HS-DSCH}: High Speed Downlink Shared Channel
\item
  \textbf{AMC}: Adaptive Modulation and Coding
\item
  \textbf{Fast Cell Selection}: Improves cell edge performance
\item
  \textbf{MIMO}: Multiple antenna configurations possible
\end{itemize}

\end{solutionbox}
\begin{mnemonicbox}
``High Speed Downlink Packet Access''

\end{mnemonicbox}
\subsection*{Question 5(a) [3 marks]}\label{q5a}

\textbf{List LTE specifications.}

\begin{solutionbox}

{\def\LTcaptype{none} % do not increment counter
\begin{longtable}[]{@{}ll@{}}
\toprule\noalign{}
Parameter & Specification \\
\midrule\noalign{}
\endhead
\bottomrule\noalign{}
\endlastfoot
\textbf{Peak Data Rate} & 300 Mbps DL, 75 Mbps UL \\
\textbf{Bandwidth} & 1.4 to 20 MHz \\
\textbf{Latency} & \textless10ms user plane \\
\textbf{Mobility} & Up to 350 km/h \\
\textbf{Spectrum Efficiency} & 3-4x better than 3G \\
\end{longtable}
}

\end{solutionbox}
\begin{mnemonicbox}
``Long Term Evolution specifications''

\end{mnemonicbox}
\subsection*{Question 5(b) [4 marks]}\label{q5b}

\textbf{Draw OFDM receiver and explain its working.}

\begin{solutionbox}

\begin{center}
\textbf{Mermaid Diagram (Code)}
\begin{verbatim}
{Shaded}
{Highlighting}[]
graph LR
    A[RF Input] {-{-}{} B[ADC]}
    B {-{-}{} C[Remove CP]}
    C {-{-}{} D[FFT]}
    D {-{-}{} E[Demodulator]}
    E {-{-}{} F[Data Output]}
{Highlighting}
{Shaded}
\end{verbatim}
\end{center}

\begin{itemize}
\tightlist
\item
  \textbf{FFT}: Fast Fourier Transform converts time to frequency domain
\item
  \textbf{Cyclic Prefix}: Guards against inter-symbol interference
\item
  \textbf{Subcarriers}: Parallel transmission on multiple frequencies
\item
  \textbf{Demodulation}: QPSK/16QAM/64QAM per subcarrier
\end{itemize}

\end{solutionbox}
\begin{mnemonicbox}
``Orthogonal Frequency Division Multiplexing''

\end{mnemonicbox}
\subsection*{Question 5(c) [7 marks]}\label{q5c}

\textbf{Explain Bluetooth Technology \& list its applications.}

\begin{solutionbox}
\textbf{Bluetooth} is short-range wireless
communication technology for personal area networks.

{\def\LTcaptype{none} % do not increment counter
\begin{longtable}[]{@{}ll@{}}
\toprule\noalign{}
Parameter & Specification \\
\midrule\noalign{}
\endhead
\bottomrule\noalign{}
\endlastfoot
\textbf{Range} & 10m (Class 2) \\
\textbf{Frequency} & 2.4 GHz ISM band \\
\textbf{Data Rate} & Up to 3 Mbps \\
\textbf{Topology} & Piconet (8 devices) \\
\end{longtable}
}

\begin{center}
\textbf{Mermaid Diagram (Code)}
\begin{verbatim}
{Shaded}
{Highlighting}[]
graph TD
    A[Master Device] {-{-}{} B[Slave 1]}
    A {-{-}{} C[Slave 2]}
    A {-{-}{} D[Slave 3]}
    E[Scatternet] {-{-}{} A}
    E {-{-}{} F[Another Piconet]}
{Highlighting}
{Shaded}
\end{verbatim}
\end{center}

\textbf{Protocol Stack}:

\begin{itemize}
\tightlist
\item
  \textbf{RF Layer}: Physical radio interface
\item
  \textbf{Baseband}: Medium access control
\item
  \textbf{L2CAP}: Logical Link Control
\item
  \textbf{Applications}: Various profiles (A2DP, HID, etc.)
\end{itemize}

\textbf{Applications}:

\begin{itemize}
\tightlist
\item
  Audio devices (headphones, speakers)
\item
  File transfer between devices
\item
  Input devices (keyboard, mouse)
\item
  Health monitoring devices
\item
  Smart home automation
\end{itemize}

\end{solutionbox}
\begin{mnemonicbox}
``Blue Tooth Personal Area Network''

\end{mnemonicbox}
\subsection*{Question 5(a OR) [3
marks]}\label{question-5a-or-3-marks}

\textbf{List advantages of 5G Technology.}

\begin{solutionbox}

{\def\LTcaptype{none} % do not increment counter
\begin{longtable}[]{@{}ll@{}}
\toprule\noalign{}
Advantage & Benefit \\
\midrule\noalign{}
\endhead
\bottomrule\noalign{}
\endlastfoot
\textbf{Ultra-low latency} & \textless1ms response time \\
\textbf{High data rates} & Up to 10 Gbps \\
\textbf{Massive connectivity} & 1M devices/km^{2} \\
\textbf{Network slicing} & Customized services \\
\textbf{Energy efficiency} & 90\% more efficient \\
\end{longtable}
}

\end{solutionbox}
\begin{mnemonicbox}
``Fifth Generation advantages''

\end{mnemonicbox}
\subsection*{Question 5(b OR) [4
marks]}\label{question-5b-or-4-marks}

\textbf{Draw OFDM transmitter and explain its working.}

\begin{solutionbox}

\begin{center}
\textbf{Mermaid Diagram (Code)}
\begin{verbatim}
{Shaded}
{Highlighting}[]
graph LR
    A[Data Input] {-{-}{} B[Modulator]}
    B {-{-}{} C[IFFT]}
    C {-{-}{} D[Add CP]}
    D {-{-}{} E[DAC]}
    E {-{-}{} F[RF Output]}
{Highlighting}
{Shaded}
\end{verbatim}
\end{center}

\begin{itemize}
\tightlist
\item
  \textbf{Modulation}: Maps bits to symbols (QPSK/QAM)
\item
  \textbf{IFFT}: Inverse FFT converts frequency to time domain
\item
  \textbf{Cyclic Prefix}: Copies end samples to beginning
\item
  \textbf{DAC}: Digital to Analog Converter for transmission
\end{itemize}

\end{solutionbox}
\begin{mnemonicbox}
``Orthogonal Frequency Division Multiplexing
Transmitter''

\end{mnemonicbox}
\subsection*{Question 5(c OR) [7
marks]}\label{question-5c-or-7-marks}

\textbf{Explain Zigbee Technology \& list its applications.}

\begin{solutionbox}
\textbf{Zigbee} is low-power wireless mesh networking
standard based on IEEE 802.15.4.

{\def\LTcaptype{none} % do not increment counter
\begin{longtable}[]{@{}ll@{}}
\toprule\noalign{}
Parameter & Specification \\
\midrule\noalign{}
\endhead
\bottomrule\noalign{}
\endlastfoot
\textbf{Range} & 10-100m \\
\textbf{Data Rate} & 250 kbps \\
\textbf{Power} & Very low (battery years) \\
\textbf{Topology} & Mesh network \\
\textbf{Frequency} & 2.4 GHz globally \\
\end{longtable}
}

\begin{center}
\textbf{Mermaid Diagram (Code)}
\begin{verbatim}
{Shaded}
{Highlighting}[]
graph TD
    A[Coordinator] {-{-}{} B[Router 1]}
    A {-{-}{} C[Router 2]}
    B {-{-}{} D[End Device 1]}
    B {-{-}{} E[End Device 2]}
    C {-{-}{} F[End Device 3]}
    C {-{-}{} G[Router 3]}
    G {-{-}{} H[End Device 4]}
{Highlighting}
{Shaded}
\end{verbatim}
\end{center}

\textbf{Network Roles}:

\begin{itemize}
\tightlist
\item
  \textbf{Coordinator}: Network manager
\item
  \textbf{Router}: Forwards messages
\item
  \textbf{End Device}: Simple sensors/actuators
\end{itemize}

\textbf{Applications}:

\begin{itemize}
\tightlist
\item
  Home automation (lights, thermostats)
\item
  Industrial monitoring
\item
  Smart grid systems
\item
  Healthcare monitoring
\item
  Agricultural sensors
\item
  Building management systems
\end{itemize}

\textbf{Features}:

\begin{itemize}
\tightlist
\item
  \textbf{Self-healing}: Automatic route discovery
\item
  \textbf{Low cost}: Simple implementation
\item
  \textbf{Secure}: AES encryption
\item
  \textbf{Reliable}: Mesh redundancy
\end{itemize}

\end{solutionbox}
\begin{mnemonicbox}
``Zigbee Mesh Network Applications''

\end{mnemonicbox}

\end{document}
