\documentclass[10pt,a4paper]{article}

% content/resources/templates/preamble.tex
\usepackage[margin=0.6in]{geometry}
\author{Milav Dabgar}
\usepackage{amsmath,amssymb,amsthm}
\usepackage{booktabs}
\usepackage{multirow}
\usepackage{xcolor}
\usepackage{tcolorbox}
\tcbuselibrary{breakable,skins}
\usepackage[colorlinks=true,linkcolor=blue]{hyperref}
\usepackage{titlesec}
\usepackage{enumitem}
\usepackage{tikz}
\usepackage{pgfplots}
\usepackage{circuitikz}
\usepackage[version=4]{mhchem}
\usepackage{longtable}
\usepackage{array}
\usepackage{float}
\usepackage{caption}
\usepackage{listings}

\lstset{
  basicstyle=\small\ttfamily,
  breaklines=true,
  breakatwhitespace=false,
  postbreak=\mbox{\textcolor{red}{$\hookrightarrow$}\space},
  float=false,
  numbers=left,
  numberstyle=\tiny\color{gray},
  numbersep=10pt,
  xleftmargin=2em,
  keywordstyle=\color{blue},
  commentstyle=\color{green!60!black},
  stringstyle=\color{purple},
  backgroundcolor=\color{gray!5},
  showstringspaces=false,
  tabsize=2,
  captionpos=b,
  keepspaces=true,
  columns=flexible
}

\pgfplotsset{compat=1.18}
\usetikzlibrary{shapes,arrows,positioning,calc,patterns,decorations.pathmorphing,decorations.markings,arrows.meta}

% Color scheme
\definecolor{headcolor}{RGB}{0,102,204}
\definecolor{keycolor}{RGB}{220,20,60}
\definecolor{solutioncolor}{RGB}{34,139,34}
\definecolor{mnemoniccolor}{RGB}{148,0,211}
\definecolor{codecolor}{RGB}{0,0,100}

% Spacing
\setlength{\parskip}{3pt}
\setlist[itemize]{nosep}
\setlist[enumerate]{nosep}

% Title formatting
\titleformat{\section}{\Large\bfseries\color{headcolor}}{\thesection}{1em}{}
\titleformat{\subsection}{\large\bfseries\color{headcolor}}{\thesubsection}{1em}{}

% Pandoc tightlist compatibility
\providecommand{\tightlist}{%
  \setlength{\itemsep}{0pt}\setlength{\parskip}{0pt}}

% Pandoc longtable compatibility
\newcounter{none}
\def\thenone{}


% content/resources/templates/english-boxes.tex
% This file is currently empty - it exists to maintain consistency with the import structure.
% Add custom environments here if needed in the future.


\begin{document}

\begin{center}
{\Huge\bfseries\color{headcolor} Subject Name Solutions}\\[5pt]
{\LARGE 4341603 -- Winter 2023}\\[3pt]
{\large Semester 1 Study Material}\\[3pt]
{\normalsize\textit{Detailed Solutions and Explanations}}
\end{center}

\vspace{10pt}

\subsection*{Question 1(a) [3 marks]}\label{q1a}

\textbf{Define human learning and explain how machine learning is
different from human learning?}

\begin{solutionbox}


{\def\LTcaptype{none} % do not increment counter
\vspace{-5pt}
\captionof{table}{Human Learning vs Machine Learning}
\vspace{-10pt}
\begin{longtable}[]{@{}
  >{\raggedright\arraybackslash}p{(\linewidth - 4\tabcolsep) * \real{0.1905}}
  >{\raggedright\arraybackslash}p{(\linewidth - 4\tabcolsep) * \real{0.3810}}
  >{\raggedright\arraybackslash}p{(\linewidth - 4\tabcolsep) * \real{0.4286}}@{}}
\toprule\noalign{}
\begin{minipage}[b]{\linewidth}\raggedright
Aspect
\end{minipage} & \begin{minipage}[b]{\linewidth}\raggedright
Human Learning
\end{minipage} & \begin{minipage}[b]{\linewidth}\raggedright
Machine Learning
\end{minipage} \\
\midrule\noalign{}
\endhead
\bottomrule\noalign{}
\endlastfoot
\textbf{Method} & Experience, trial and error & Data and algorithms \\
\textbf{Speed} & Slow, gradual & Fast processing \\
\textbf{Data Requirement} & Limited examples needed & Large datasets
required \\
\end{longtable}
}

\begin{itemize}
\tightlist
\item
  \textbf{Human Learning}: Process of acquiring knowledge through
  experience, observation, and reasoning
\item
  \textbf{Machine Learning}: Automated learning from data using
  algorithms to identify patterns
\end{itemize}

\end{solutionbox}
\begin{mnemonicbox}
``Humans Experience, Machines Analyze Data'' (HEMAD)

\end{mnemonicbox}
\begin{center}\rule{0.5\linewidth}{0.5pt}\end{center}

\subsection*{Question 1(b) [4 marks]}\label{q1b}

\textbf{Describe the use of machine learning in finance and banking.}

\begin{solutionbox}

\textbf{Applications in Finance and Banking:}

{\def\LTcaptype{none} % do not increment counter
\begin{longtable}[]{@{}
  >{\raggedright\arraybackslash}p{(\linewidth - 4\tabcolsep) * \real{0.4194}}
  >{\raggedright\arraybackslash}p{(\linewidth - 4\tabcolsep) * \real{0.2903}}
  >{\raggedright\arraybackslash}p{(\linewidth - 4\tabcolsep) * \real{0.2903}}@{}}
\toprule\noalign{}
\begin{minipage}[b]{\linewidth}\raggedright
Application
\end{minipage} & \begin{minipage}[b]{\linewidth}\raggedright
Purpose
\end{minipage} & \begin{minipage}[b]{\linewidth}\raggedright
Benefit
\end{minipage} \\
\midrule\noalign{}
\endhead
\bottomrule\noalign{}
\endlastfoot
\textbf{Fraud Detection} & Identify suspicious transactions & Reduce
financial losses \\
\textbf{Credit Scoring} & Assess loan default risk & Better lending
decisions \\
\textbf{Algorithmic Trading} & Automated trading decisions & Faster
market responses \\
\end{longtable}
}

\begin{itemize}
\tightlist
\item
  \textbf{Risk Assessment}: ML analyzes customer data to predict
  creditworthiness
\item
  \textbf{Customer Service}: Chatbots provide 24/7 support using NLP
\item
  \textbf{Regulatory Compliance}: Automated monitoring for suspicious
  activities
\end{itemize}

\end{solutionbox}
\begin{mnemonicbox}
``Finance Needs Smart Analysis'' (FNSA)

\end{mnemonicbox}
\begin{center}\rule{0.5\linewidth}{0.5pt}\end{center}

\subsection*{Question 1(c) [7 marks]}\label{q1c}

\textbf{Give difference between Supervised Learning, Unsupervised
Learning and Reinforcement Learning.}

\begin{solutionbox}

\textbf{Comparison Table:}

{\def\LTcaptype{none} % do not increment counter
\begin{longtable}[]{@{}
  >{\raggedright\arraybackslash}p{(\linewidth - 6\tabcolsep) * \real{0.1286}}
  >{\raggedright\arraybackslash}p{(\linewidth - 6\tabcolsep) * \real{0.2714}}
  >{\raggedright\arraybackslash}p{(\linewidth - 6\tabcolsep) * \real{0.3000}}
  >{\raggedright\arraybackslash}p{(\linewidth - 6\tabcolsep) * \real{0.3000}}@{}}
\toprule\noalign{}
\begin{minipage}[b]{\linewidth}\raggedright
Feature
\end{minipage} & \begin{minipage}[b]{\linewidth}\raggedright
Supervised Learning
\end{minipage} & \begin{minipage}[b]{\linewidth}\raggedright
Unsupervised Learning
\end{minipage} & \begin{minipage}[b]{\linewidth}\raggedright
Reinforcement Learning
\end{minipage} \\
\midrule\noalign{}
\endhead
\bottomrule\noalign{}
\endlastfoot
\textbf{Data Type} & Labeled data & Unlabeled data & Environment
interaction \\
\textbf{Goal} & Predict output & Find patterns & Maximize rewards \\
\textbf{Examples} & Classification, Regression & Clustering, Association
& Game playing, Robotics \\
\textbf{Feedback} & Immediate & None & Delayed rewards \\
\end{longtable}
}

\textbf{Key Characteristics:}

\begin{itemize}
\tightlist
\item
  \textbf{Supervised Learning}: Teacher-guided learning with correct
  answers provided
\item
  \textbf{Unsupervised Learning}: Self-discovery of hidden patterns in
  data
\item
  \textbf{Reinforcement Learning}: Learning through trial and error with
  rewards/penalties
\end{itemize}

\end{solutionbox}
\begin{mnemonicbox}
``Supervised Teachers, Unsupervised Explores,
Reinforcement Rewards'' (STUER)

\end{mnemonicbox}
\begin{center}\rule{0.5\linewidth}{0.5pt}\end{center}

\subsection*{Question 1(c OR) [7
marks]}\label{question-1c-or-7-marks}

\textbf{Explain different tools and technology used in machine
learning.}

\begin{solutionbox}

\textbf{ML Tools and Technologies:}

{\def\LTcaptype{none} % do not increment counter
\begin{longtable}[]{@{}lll@{}}
\toprule\noalign{}
Category & Tools & Purpose \\
\midrule\noalign{}
\endhead
\bottomrule\noalign{}
\endlastfoot
\textbf{Programming} & Python, R, Java & Algorithm implementation \\
\textbf{Libraries} & Scikit-learn, TensorFlow & Ready-made algorithms \\
\textbf{Visualization} & Matplotlib, Seaborn & Data visualization \\
\textbf{Data Processing} & Pandas, NumPy & Data manipulation \\
\end{longtable}
}

\textbf{Key Technologies:}

\begin{itemize}
\tightlist
\item
  \textbf{Cloud Platforms}: AWS, Google Cloud for scalable computing
\item
  \textbf{Development Environments}: Jupyter Notebook, Google Colab
\item
  \textbf{Big Data Tools}: Spark, Hadoop for large datasets
\end{itemize}

\end{solutionbox}
\begin{mnemonicbox}
``Python Libraries Visualize Data Effectively''
(PLVDE)

\end{mnemonicbox}
\begin{center}\rule{0.5\linewidth}{0.5pt}\end{center}

\subsection*{Question 2(a) [3 marks]}\label{q2a}

\textbf{Define outliers with one example.}

\begin{solutionbox}

\textbf{Definition}: Outliers are data points that significantly differ
from other observations in a dataset.

\textbf{Example Table:}

{\def\LTcaptype{none} % do not increment counter
\begin{longtable}[]{@{}ll@{}}
\toprule\noalign{}
Student Heights (cm) & Classification \\
\midrule\noalign{}
\endhead
\bottomrule\noalign{}
\endlastfoot
165, 170, 168, 172 & Normal values \\
195 & Outlier (too tall) \\
140 & Outlier (too short) \\
\end{longtable}
}

\begin{itemize}
\tightlist
\item
  \textbf{Detection}: Values beyond 1.5 \times IQR from quartiles
\item
  \textbf{Impact}: Can skew statistical analysis and model performance
\end{itemize}

\end{solutionbox}
\begin{mnemonicbox}
``Outliers Stand Apart'' (OSA)

\end{mnemonicbox}
\begin{center}\rule{0.5\linewidth}{0.5pt}\end{center}

\subsection*{Question 2(b) [4 marks]}\label{q2b}

\textbf{Explain regression steps in detail.}

\begin{solutionbox}

\textbf{Regression Process Steps:}

\begin{verbatim}
flowchart LR
    A[Data Collection] {-{-} B[Data Preprocessing]}
    B {-{-} C[Feature Selection]}
    C {-{-} D[Model Training]}
    D {-{-} E[Model Evaluation]}
    E {-{-} F[Prediction]}
\end{verbatim}

\textbf{Detailed Steps:}

\begin{itemize}
\tightlist
\item
  \textbf{Data Collection}: Gather relevant dataset with input-output
  pairs
\item
  \textbf{Preprocessing}: Clean data, handle missing values, normalize
  features
\item
  \textbf{Feature Selection}: Choose relevant variables that affect
  target
\item
  \textbf{Model Training}: Fit regression line to minimize prediction
  errors
\end{itemize}

\end{solutionbox}
\begin{mnemonicbox}
``Data Preprocessing Features Train Evaluation
Predicts'' (DPFTEP)

\end{mnemonicbox}
\begin{center}\rule{0.5\linewidth}{0.5pt}\end{center}

\subsection*{Question 2(c) [7 marks]}\label{q2c}

\textbf{Define Accuracy and for the following binary classifier's
confusion matrix, find the various measurement parameters like 1.
Accuracy 2. Precision.}

\begin{solutionbox}

\textbf{Confusion Matrix Analysis:}

{\def\LTcaptype{none} % do not increment counter
\begin{longtable}[]{@{}lll@{}}
\toprule\noalign{}
& Predicted No & Predicted Yes \\
\midrule\noalign{}
\endhead
\bottomrule\noalign{}
\endlastfoot
\textbf{Actual No} & 10 (TN) & 3 (FP) \\
\textbf{Actual Yes} & 2 (FN) & 15 (TP) \\
\end{longtable}
}

\textbf{Calculations:}

{\def\LTcaptype{none} % do not increment counter
\begin{longtable}[]{@{}llll@{}}
\toprule\noalign{}
Metric & Formula & Calculation & Result \\
\midrule\noalign{}
\endhead
\bottomrule\noalign{}
\endlastfoot
\textbf{Accuracy} & (TP+TN)/(TP+TN+FP+FN) & (15+10)/(15+10+3+2) &
83.33\% \\
\textbf{Precision} & TP/(TP+FP) & 15/(15+3) & 83.33\% \\
\end{longtable}
}

\textbf{Definitions:}

\begin{itemize}
\tightlist
\item
  \textbf{Accuracy}: Proportion of correct predictions out of total
  predictions
\item
  \textbf{Precision}: Proportion of true positive predictions out of all
  positive predictions
\end{itemize}

\end{solutionbox}
\begin{mnemonicbox}
``Accuracy Counts All, Precision Picks Positives''
(ACAPP)

\end{mnemonicbox}
\begin{center}\rule{0.5\linewidth}{0.5pt}\end{center}

\subsection*{Question 2(a OR) [3
marks]}\label{question-2a-or-3-marks}

\textbf{Identify basic steps of feature subset selection.}

\begin{solutionbox}

\textbf{Feature Subset Selection Steps:}

\begin{verbatim}
flowchart LR
    A[Original Features] {-{-} B[Generate Subsets]}
    B {-{-} C[Evaluate Subsets]}
    C {-{-} D[Select Best Subset]}
\end{verbatim}

\textbf{Basic Steps:}

\begin{itemize}
\tightlist
\item
  \textbf{Generation}: Create different combinations of features
\item
  \textbf{Evaluation}: Test each subset using performance metrics
\item
  \textbf{Selection}: Choose optimal subset based on criteria
\end{itemize}

\end{solutionbox}
\begin{mnemonicbox}
``Generate, Evaluate, Select'' (GES)

\end{mnemonicbox}
\begin{center}\rule{0.5\linewidth}{0.5pt}\end{center}

\subsection*{Question 2(b OR) [4
marks]}\label{question-2b-or-4-marks}

\textbf{Discuss the strength and weakness of the KNN algorithm.}

\begin{solutionbox}

\textbf{KNN Algorithm Analysis:}

{\def\LTcaptype{none} % do not increment counter
\begin{longtable}[]{@{}
  >{\raggedright\arraybackslash}p{(\linewidth - 2\tabcolsep) * \real{0.4783}}
  >{\raggedright\arraybackslash}p{(\linewidth - 2\tabcolsep) * \real{0.5217}}@{}}
\toprule\noalign{}
\begin{minipage}[b]{\linewidth}\raggedright
Strengths
\end{minipage} & \begin{minipage}[b]{\linewidth}\raggedright
Weaknesses
\end{minipage} \\
\midrule\noalign{}
\endhead
\bottomrule\noalign{}
\endlastfoot
Simple to understand & Computationally expensive \\
No training required & Sensitive to irrelevant features \\
Works with non-linear data & Performance degrades with high
dimensions \\
Effective for small datasets & Requires optimal K value selection \\
\end{longtable}
}

\textbf{Key Points:}

\begin{itemize}
\tightlist
\item
  \textbf{Lazy Learning}: No explicit training phase required
\item
  \textbf{Distance-Based}: Classification based on neighbor proximity
\item
  \textbf{Memory-Intensive}: Stores entire training dataset
\end{itemize}

\end{solutionbox}
\begin{mnemonicbox}
``Simple but Slow, Effective but Expensive'' (SBSEBE)

\end{mnemonicbox}
\begin{center}\rule{0.5\linewidth}{0.5pt}\end{center}

\subsection*{Question 2(c OR) [7
marks]}\label{question-2c-or-7-marks}

\textbf{Define Error-rate and for the following binary classifier's
confusion matrix, find the various measurement parameters like 1. Error
value 2. Recall.}

\begin{solutionbox}

\textbf{Confusion Matrix Analysis:}

{\def\LTcaptype{none} % do not increment counter
\begin{longtable}[]{@{}lll@{}}
\toprule\noalign{}
& Predicted No & Predicted Yes \\
\midrule\noalign{}
\endhead
\bottomrule\noalign{}
\endlastfoot
\textbf{Actual No} & 20 (TN) & 3 (FP) \\
\textbf{Actual Yes} & 2 (FN) & 15 (TP) \\
\end{longtable}
}

\textbf{Calculations:}

{\def\LTcaptype{none} % do not increment counter
\begin{longtable}[]{@{}llll@{}}
\toprule\noalign{}
Metric & Formula & Calculation & Result \\
\midrule\noalign{}
\endhead
\bottomrule\noalign{}
\endlastfoot
\textbf{Error Rate} & (FP+FN)/(TP+TN+FP+FN) & (3+2)/(15+20+3+2) &
12.5\% \\
\textbf{Recall} & TP/(TP+FN) & 15/(15+2) & 88.24\% \\
\end{longtable}
}

\textbf{Definitions:}

\begin{itemize}
\tightlist
\item
  \textbf{Error Rate}: Proportion of incorrect predictions out of total
  predictions
\item
  \textbf{Recall}: Proportion of actual positives correctly identified
\end{itemize}

\end{solutionbox}
\begin{mnemonicbox}
``Error Excludes, Recall Retrieves'' (EERR)

\end{mnemonicbox}
\begin{center}\rule{0.5\linewidth}{0.5pt}\end{center}

\subsection*{Question 3(a) [3 marks]}\label{q3a}

\textbf{Give any three examples of unsupervised learning.}

\begin{solutionbox}

\textbf{Unsupervised Learning Examples:}

{\def\LTcaptype{none} % do not increment counter
\begin{longtable}[]{@{}
  >{\raggedright\arraybackslash}p{(\linewidth - 4\tabcolsep) * \real{0.2571}}
  >{\raggedright\arraybackslash}p{(\linewidth - 4\tabcolsep) * \real{0.3714}}
  >{\raggedright\arraybackslash}p{(\linewidth - 4\tabcolsep) * \real{0.3714}}@{}}
\toprule\noalign{}
\begin{minipage}[b]{\linewidth}\raggedright
Example
\end{minipage} & \begin{minipage}[b]{\linewidth}\raggedright
Description
\end{minipage} & \begin{minipage}[b]{\linewidth}\raggedright
Application
\end{minipage} \\
\midrule\noalign{}
\endhead
\bottomrule\noalign{}
\endlastfoot
\textbf{Customer Segmentation} & Group customers by behavior & Marketing
strategies \\
\textbf{Document Classification} & Organize documents by topics &
Information retrieval \\
\textbf{Gene Sequencing} & Group similar DNA patterns & Medical
research \\
\end{longtable}
}

\begin{itemize}
\tightlist
\item
  \textbf{Market Basket Analysis}: Finding product purchase patterns
\item
  \textbf{Social Network Analysis}: Identifying community structures
\item
  \textbf{Anomaly Detection}: Detecting unusual patterns in data
\end{itemize}

\end{solutionbox}
\begin{mnemonicbox}
``Customers, Documents, Genes Group Automatically''
(CDGGA)

\end{mnemonicbox}
\begin{center}\rule{0.5\linewidth}{0.5pt}\end{center}

\subsection*{Question 3(b) [4 marks]}\label{q3b}

\textbf{Find Mean and Median for the following data:
4,6,7,8,9,12,14,15,20}

\begin{solutionbox}

\textbf{Statistical Calculations:}

{\def\LTcaptype{none} % do not increment counter
\begin{longtable}[]{@{}lll@{}}
\toprule\noalign{}
Statistic & Calculation & Result \\
\midrule\noalign{}
\endhead
\bottomrule\noalign{}
\endlastfoot
\textbf{Mean} & (4+6+7+8+9+12+14+15+20)/9 & 10.56 \\
\textbf{Median} & Middle value (5th position) & 9 \\
\end{longtable}
}

\textbf{Step-by-step:}

\begin{itemize}
\tightlist
\item
  \textbf{Data}: Already sorted: 4,6,7,8,9,12,14,15,20
\item
  \textbf{Mean}: Sum all values \div count = 95 \div 9 = 10.56
\item
  \textbf{Median}: Middle value in sorted list = 9 (5th position)
\end{itemize}

\end{solutionbox}
\begin{mnemonicbox}
``Mean Averages All, Median Middle Value'' (MAAMV)

\end{mnemonicbox}
\begin{center}\rule{0.5\linewidth}{0.5pt}\end{center}

\subsection*{Question 3(c) [7 marks]}\label{q3c}

\textbf{Describe k-fold cross validation method in detail.}

\begin{solutionbox}

\textbf{K-Fold Cross Validation Process:}

\begin{verbatim}
flowchart LR
    A[Original Dataset] {-{-} B[Split into K folds]}
    B {-{-} C[Train on K{-}1 folds]}
    C {-{-} D[Test on 1 fold]}
    D {-{-} E[Repeat K times]}
    E {-{-} F[Average Results]}
\end{verbatim}

\textbf{Process Steps:}

{\def\LTcaptype{none} % do not increment counter
\begin{longtable}[]{@{}
  >{\raggedright\arraybackslash}p{(\linewidth - 4\tabcolsep) * \real{0.2143}}
  >{\raggedright\arraybackslash}p{(\linewidth - 4\tabcolsep) * \real{0.4643}}
  >{\raggedright\arraybackslash}p{(\linewidth - 4\tabcolsep) * \real{0.3214}}@{}}
\toprule\noalign{}
\begin{minipage}[b]{\linewidth}\raggedright
Step
\end{minipage} & \begin{minipage}[b]{\linewidth}\raggedright
Description
\end{minipage} & \begin{minipage}[b]{\linewidth}\raggedright
Purpose
\end{minipage} \\
\midrule\noalign{}
\endhead
\bottomrule\noalign{}
\endlastfoot
\textbf{1. Data Division} & Split data into K equal parts & Ensure
balanced testing \\
\textbf{2. Iterative Training} & Use K-1 folds for training & Maximum
data utilization \\
\textbf{3. Validation} & Test on remaining fold & Unbiased evaluation \\
\textbf{4. Averaging} & Calculate mean performance & Robust performance
estimate \\
\end{longtable}
}

\textbf{Advantages:}

\begin{itemize}
\tightlist
\item
  \textbf{Unbiased Estimation}: Each data point used for both training
  and testing
\item
  \textbf{Reduced Overfitting}: Multiple validation rounds increase
  reliability
\item
  \textbf{Efficient Data Use}: All data utilized for both training and
  validation
\end{itemize}

\end{solutionbox}
\begin{mnemonicbox}
``K-fold Keeps Keen Knowledge'' (KKKK)

\end{mnemonicbox}
\begin{center}\rule{0.5\linewidth}{0.5pt}\end{center}

\subsection*{Question 3(a OR) [3
marks]}\label{question-3a-or-3-marks}

\textbf{Give any three applications of multiple linear regression.}

\begin{solutionbox}

\textbf{Multiple Linear Regression Applications:}

{\def\LTcaptype{none} % do not increment counter
\begin{longtable}[]{@{}
  >{\raggedright\arraybackslash}p{(\linewidth - 4\tabcolsep) * \real{0.3939}}
  >{\raggedright\arraybackslash}p{(\linewidth - 4\tabcolsep) * \real{0.3333}}
  >{\raggedright\arraybackslash}p{(\linewidth - 4\tabcolsep) * \real{0.2727}}@{}}
\toprule\noalign{}
\begin{minipage}[b]{\linewidth}\raggedright
Application
\end{minipage} & \begin{minipage}[b]{\linewidth}\raggedright
Variables
\end{minipage} & \begin{minipage}[b]{\linewidth}\raggedright
Purpose
\end{minipage} \\
\midrule\noalign{}
\endhead
\bottomrule\noalign{}
\endlastfoot
\textbf{House Price Prediction} & Size, location, age & Real estate
valuation \\
\textbf{Sales Forecasting} & Marketing spend, season, economy & Business
planning \\
\textbf{Medical Diagnosis} & Symptoms, age, history & Disease
prediction \\
\end{longtable}
}

\begin{itemize}
\tightlist
\item
  \textbf{Stock Market Analysis}: Multiple economic indicators predict
  stock prices
\item
  \textbf{Academic Performance}: Study hours, attendance, previous
  grades predict scores
\item
  \textbf{Marketing ROI}: Various marketing channels impact sales
  revenue
\end{itemize}

\end{solutionbox}
\begin{mnemonicbox}
``Houses, Sales, Medicine Predict Multiple
Variables'' (HSMPV)

\end{mnemonicbox}
\begin{center}\rule{0.5\linewidth}{0.5pt}\end{center}

\subsection*{Question 3(b OR) [4
marks]}\label{question-3b-or-4-marks}

\textbf{Find Standard Deviation for the following data:
4,15,20,28,35,45}

\begin{solutionbox}

\textbf{Standard Deviation Calculation:}

{\def\LTcaptype{none} % do not increment counter
\begin{longtable}[]{@{}lll@{}}
\toprule\noalign{}
Step & Calculation & Value \\
\midrule\noalign{}
\endhead
\bottomrule\noalign{}
\endlastfoot
\textbf{Mean} & (4+15+20+28+35+45)/6 & 24.5 \\
\textbf{Variance} & Σ(xi-mean)^{2}/n & 236.92 \\
\textbf{Std Dev} & \sqrtVariance & 15.39 \\
\end{longtable}
}

\textbf{Detailed Calculation:}

\begin{itemize}
\tightlist
\item
  \textbf{Deviations from mean}: (-20.5)^{2}, (-9.5)^{2}, (-4.5)^{2}, (3.5)^{2},
  (10.5)^{2}, (20.5)^{2}
\item
  \textbf{Squared deviations}: 420.25, 90.25, 20.25, 12.25, 110.25,
  420.25
\item
  \textbf{Sum}: 1073.5
\item
  \textbf{Variance}: 1073.5/6 = 178.92
\item
  \textbf{Standard Deviation}: \sqrt178.92 = 13.38
\end{itemize}

\end{solutionbox}
\begin{mnemonicbox}
``Deviation Measures Data Spread'' (DMDS)

\end{mnemonicbox}
\begin{center}\rule{0.5\linewidth}{0.5pt}\end{center}

\subsection*{Question 3(c OR) [7
marks]}\label{question-3c-or-7-marks}

\textbf{Explain Bagging, Boosting in detail.}

\begin{solutionbox}

\textbf{Ensemble Methods Comparison:}

{\def\LTcaptype{none} % do not increment counter
\begin{longtable}[]{@{}lll@{}}
\toprule\noalign{}
Aspect & Bagging & Boosting \\
\midrule\noalign{}
\endhead
\bottomrule\noalign{}
\endlastfoot
\textbf{Strategy} & Parallel training & Sequential training \\
\textbf{Data Sampling} & Random with replacement & Weighted sampling \\
\textbf{Combination} & Simple averaging/voting & Weighted combination \\
\textbf{Bias-Variance} & Reduces variance & Reduces bias \\
\end{longtable}
}

\textbf{Bagging (Bootstrap Aggregating):}

\begin{verbatim}
flowchart LR
    A[Original Data] {-{-} B[Bootstrap Sample 1]}
    A {-{-} C[Bootstrap Sample 2]}
    A {-{-} D[Bootstrap Sample n]}
    B {-{-} E[Model 1]}
    C {-{-} F[Model 2]}
    D {-{-} G[Model n]}
    E {-{-} H[Final Prediction]}
    F {-{-} H}
    G {-{-} H}
\end{verbatim}

\textbf{Boosting Process:}

\begin{itemize}
\tightlist
\item
  \textbf{Sequential Learning}: Each model learns from previous model's
  mistakes
\item
  \textbf{Weight Adjustment}: Increase weight of misclassified examples
\item
  \textbf{Final Prediction}: Weighted combination of all models
\end{itemize}

\textbf{Key Differences:}

\begin{itemize}
\tightlist
\item
  \textbf{Bagging}: Independent models trained in parallel, reduces
  overfitting
\item
  \textbf{Boosting}: Dependent models trained sequentially, improves
  accuracy
\end{itemize}

\end{solutionbox}
\begin{mnemonicbox}
``Bagging Builds Parallel, Boosting Builds
Sequential'' (BBPBS)

\end{mnemonicbox}
\begin{center}\rule{0.5\linewidth}{0.5pt}\end{center}

\subsection*{Question 4(a) [3 marks]}\label{q4a}

\textbf{Define: Support, Confidence.}

\begin{solutionbox}

\textbf{Association Rule Metrics:}

{\def\LTcaptype{none} % do not increment counter
\begin{longtable}[]{@{}
  >{\raggedright\arraybackslash}p{(\linewidth - 4\tabcolsep) * \real{0.2759}}
  >{\raggedright\arraybackslash}p{(\linewidth - 4\tabcolsep) * \real{0.4138}}
  >{\raggedright\arraybackslash}p{(\linewidth - 4\tabcolsep) * \real{0.3103}}@{}}
\toprule\noalign{}
\begin{minipage}[b]{\linewidth}\raggedright
Metric
\end{minipage} & \begin{minipage}[b]{\linewidth}\raggedright
Definition
\end{minipage} & \begin{minipage}[b]{\linewidth}\raggedright
Formula
\end{minipage} \\
\midrule\noalign{}
\endhead
\bottomrule\noalign{}
\endlastfoot
\textbf{Support} & Frequency of itemset in transactions & Support(A) =
Count(A)/Total transactions \\
\textbf{Confidence} & Conditional probability of rule & Confidence(A\rightarrowB)
= Support(A\cupB)/Support(A) \\
\end{longtable}
}

\textbf{Example:}

\begin{itemize}
\tightlist
\item
  \textbf{Support(Bread)} = 0.6 (60\% transactions contain bread)
\item
  \textbf{Confidence(Bread\rightarrowButter)} = 0.8 (80\% of bread buyers also buy
  butter)
\end{itemize}

\textbf{Applications:}

\begin{itemize}
\tightlist
\item
  \textbf{Market Basket Analysis}: Finding product associations
\item
  \textbf{Recommendation Systems}: Suggesting related items
\end{itemize}

\end{solutionbox}
\begin{mnemonicbox}
``Support Shows Frequency, Confidence Shows
Connection'' (SSFC)

\end{mnemonicbox}
\begin{center}\rule{0.5\linewidth}{0.5pt}\end{center}

\subsection*{Question 4(b) [4 marks]}\label{q4b}

\textbf{Illustrate any two applications of logistic regression.}

\begin{solutionbox}

\textbf{Logistic Regression Applications:}

{\def\LTcaptype{none} % do not increment counter
\begin{longtable}[]{@{}
  >{\raggedright\arraybackslash}p{(\linewidth - 6\tabcolsep) * \real{0.2766}}
  >{\raggedright\arraybackslash}p{(\linewidth - 6\tabcolsep) * \real{0.3404}}
  >{\raggedright\arraybackslash}p{(\linewidth - 6\tabcolsep) * \real{0.1702}}
  >{\raggedright\arraybackslash}p{(\linewidth - 6\tabcolsep) * \real{0.2128}}@{}}
\toprule\noalign{}
\begin{minipage}[b]{\linewidth}\raggedright
Application
\end{minipage} & \begin{minipage}[b]{\linewidth}\raggedright
Input Variables
\end{minipage} & \begin{minipage}[b]{\linewidth}\raggedright
Output
\end{minipage} & \begin{minipage}[b]{\linewidth}\raggedright
Use Case
\end{minipage} \\
\midrule\noalign{}
\endhead
\bottomrule\noalign{}
\endlastfoot
\textbf{Email Spam Detection} & Word frequency, sender, subject &
Spam/Not Spam & Email filtering \\
\textbf{Medical Diagnosis} & Symptoms, age, test results & Disease/No
Disease & Healthcare \\
\end{longtable}
}

\textbf{Key Features:}

\begin{itemize}
\tightlist
\item
  \textbf{Binary Classification}: Predicts probability between 0 and 1
\item
  \textbf{S-shaped Curve}: Uses sigmoid function for probability
  estimation
\item
  \textbf{Linear Decision Boundary}: Separates classes with linear
  boundary
\end{itemize}

\textbf{Real-world Examples:}

\begin{itemize}
\tightlist
\item
  \textbf{Marketing}: Customer purchase probability based on
  demographics
\item
  \textbf{Finance}: Credit approval based on credit history and income
\end{itemize}

\end{solutionbox}
\begin{mnemonicbox}
``Logistic Limits Linear Logic'' (LLLL)

\end{mnemonicbox}
\begin{center}\rule{0.5\linewidth}{0.5pt}\end{center}

\subsection*{Question 4(c) [7 marks]}\label{q4c}

\textbf{Discuss the main purpose of Numpy and Pandas in machine
learning.}

\begin{solutionbox}

\textbf{NumPy and Pandas in ML:}

{\def\LTcaptype{none} % do not increment counter
\begin{longtable}[]{@{}lll@{}}
\toprule\noalign{}
Library & Purpose & Key Features \\
\midrule\noalign{}
\endhead
\bottomrule\noalign{}
\endlastfoot
\textbf{NumPy} & Numerical computing & Arrays, mathematical functions \\
\textbf{Pandas} & Data manipulation & DataFrames, data cleaning \\
\end{longtable}
}

\textbf{NumPy Functions:}

\begin{center}
\textbf{Mermaid Diagram (Code)}
\begin{verbatim}
{Shaded}
{Highlighting}[]
graph TD
    A[NumPy] {-{-}{} B[Array Operations]}
    A {-{-}{} C[Mathematical Functions]}
    A {-{-}{} D[Linear Algebra]}
    A {-{-}{} E[Random Numbers]}
{Highlighting}
{Shaded}
\end{verbatim}
\end{center}

\textbf{Pandas Capabilities:}

\begin{itemize}
\tightlist
\item
  \textbf{Data Import/Export}: Read CSV, Excel, JSON files
\item
  \textbf{Data Cleaning}: Handle missing values, duplicates
\item
  \textbf{Data Transformation}: Group, merge, pivot operations
\item
  \textbf{Statistical Analysis}: Descriptive statistics, correlation
\end{itemize}

\textbf{Integration with ML:}

\begin{itemize}
\tightlist
\item
  \textbf{Data Preprocessing}: Clean and prepare data for algorithms
\item
  \textbf{Feature Engineering}: Create new features from existing data
\item
  \textbf{Model Input}: Convert data to format required by ML algorithms
\end{itemize}

\textbf{Key Benefits:}

\begin{itemize}
\tightlist
\item
  \textbf{Performance}: Optimized C/C++ backend for speed
\item
  \textbf{Memory Efficiency}: Efficient data storage and manipulation
\item
  \textbf{Ecosystem Integration}: Works seamlessly with scikit-learn,
  matplotlib
\end{itemize}

\end{solutionbox}
\begin{mnemonicbox}
``NumPy Numbers, Pandas Processes Data'' (NNPD)

\end{mnemonicbox}
\begin{center}\rule{0.5\linewidth}{0.5pt}\end{center}

\subsection*{Question 4(a OR) [3
marks]}\label{question-4a-or-3-marks}

\textbf{Give any three examples of Supervised Learning.}

\begin{solutionbox}

\textbf{Supervised Learning Examples:}

{\def\LTcaptype{none} % do not increment counter
\begin{longtable}[]{@{}
  >{\raggedright\arraybackslash}p{(\linewidth - 4\tabcolsep) * \real{0.2903}}
  >{\raggedright\arraybackslash}p{(\linewidth - 4\tabcolsep) * \real{0.1935}}
  >{\raggedright\arraybackslash}p{(\linewidth - 4\tabcolsep) * \real{0.5161}}@{}}
\toprule\noalign{}
\begin{minipage}[b]{\linewidth}\raggedright
Example
\end{minipage} & \begin{minipage}[b]{\linewidth}\raggedright
Type
\end{minipage} & \begin{minipage}[b]{\linewidth}\raggedright
Input \rightarrow Output
\end{minipage} \\
\midrule\noalign{}
\endhead
\bottomrule\noalign{}
\endlastfoot
\textbf{Email Classification} & Classification & Email features \rightarrow
Spam/Not Spam \\
\textbf{House Price Prediction} & Regression & House features \rightarrow Price \\
\textbf{Image Recognition} & Classification & Pixel values \rightarrow Object
class \\
\end{longtable}
}

\begin{itemize}
\tightlist
\item
  \textbf{Medical Diagnosis}: Patient symptoms \rightarrow Disease classification
\item
  \textbf{Stock Price Prediction}: Market indicators \rightarrow Future price
\item
  \textbf{Speech Recognition}: Audio signals \rightarrow Text transcription
\end{itemize}

\end{solutionbox}
\begin{mnemonicbox}
``Emails, Houses, Images Learn Supervised'' (EHILS)

\end{mnemonicbox}
\begin{center}\rule{0.5\linewidth}{0.5pt}\end{center}

\subsection*{Question 4(b OR) [4
marks]}\label{question-4b-or-4-marks}

\textbf{Explain any two applications of the apriori algorithm.}

\begin{solutionbox}

\textbf{Apriori Algorithm Applications:}

{\def\LTcaptype{none} % do not increment counter
\begin{longtable}[]{@{}
  >{\raggedright\arraybackslash}p{(\linewidth - 4\tabcolsep) * \real{0.3095}}
  >{\raggedright\arraybackslash}p{(\linewidth - 4\tabcolsep) * \real{0.3095}}
  >{\raggedright\arraybackslash}p{(\linewidth - 4\tabcolsep) * \real{0.3810}}@{}}
\toprule\noalign{}
\begin{minipage}[b]{\linewidth}\raggedright
Application
\end{minipage} & \begin{minipage}[b]{\linewidth}\raggedright
Description
\end{minipage} & \begin{minipage}[b]{\linewidth}\raggedright
Business Value
\end{minipage} \\
\midrule\noalign{}
\endhead
\bottomrule\noalign{}
\endlastfoot
\textbf{Market Basket Analysis} & Find products bought together &
Cross-selling strategies \\
\textbf{Web Usage Mining} & Discover website navigation patterns &
Improve user experience \\
\end{longtable}
}

\textbf{Market Basket Analysis:}

\begin{itemize}
\tightlist
\item
  \textbf{Example}: ``Customers who buy bread and milk also buy eggs''
\item
  \textbf{Business Impact}: Product placement, promotional offers
\item
  \textbf{Implementation}: Analyze transaction data to find frequent
  itemsets
\end{itemize}

\textbf{Web Usage Mining:}

\begin{itemize}
\tightlist
\item
  \textbf{Example}: ``Users visiting page A often visit page B next''
\item
  \textbf{Website Optimization}: Improve navigation, recommend content
\item
  \textbf{User Experience}: Personalized website layouts
\end{itemize}

\textbf{Algorithm Process:}

\begin{itemize}
\tightlist
\item
  \textbf{Generate Candidates}: Create frequent itemsets
\item
  \textbf{Prune}: Remove infrequent items
\item
  \textbf{Generate Rules}: Create association rules with confidence
\end{itemize}

\end{solutionbox}
\begin{mnemonicbox}
``Apriori Analyzes Associations Automatically''
(AAAA)

\end{mnemonicbox}
\begin{center}\rule{0.5\linewidth}{0.5pt}\end{center}

\subsection*{Question 4(c OR) [7
marks]}\label{question-4c-or-7-marks}

\textbf{Explain the features and applications of Matplotlib.}

\begin{solutionbox}

\textbf{Matplotlib Features and Applications:}

{\def\LTcaptype{none} % do not increment counter
\begin{longtable}[]{@{}
  >{\raggedright\arraybackslash}p{(\linewidth - 4\tabcolsep) * \real{0.4000}}
  >{\raggedright\arraybackslash}p{(\linewidth - 4\tabcolsep) * \real{0.2889}}
  >{\raggedright\arraybackslash}p{(\linewidth - 4\tabcolsep) * \real{0.3111}}@{}}
\toprule\noalign{}
\begin{minipage}[b]{\linewidth}\raggedright
Feature Category
\end{minipage} & \begin{minipage}[b]{\linewidth}\raggedright
Capabilities
\end{minipage} & \begin{minipage}[b]{\linewidth}\raggedright
Applications
\end{minipage} \\
\midrule\noalign{}
\endhead
\bottomrule\noalign{}
\endlastfoot
\textbf{Plot Types} & Line, bar, scatter, histogram & Data
exploration \\
\textbf{Customization} & Colors, labels, styles & Professional
presentations \\
\textbf{Subplots} & Multiple plots in one figure & Comparative
analysis \\
\textbf{3D Plotting} & Three-dimensional visualizations & Scientific
modeling \\
\end{longtable}
}

\textbf{Key Features:}

\begin{center}
\textbf{Mermaid Diagram (Code)}
\begin{verbatim}
{Shaded}
{Highlighting}[]
graph TD
    A[Matplotlib] {-{-}{} B[2D Plotting]}
    A {-{-}{} C[3D Plotting]}
    A {-{-}{} D[Interactive Plots]}
    A {-{-}{} E[Publication Quality]}
    B {-{-}{} F[Line Charts]}
    B {-{-}{} G[Bar Charts]}
    B {-{-}{} H[Scatter Plots]}
    C {-{-}{} I[Surface Plots]}
    C {-{-}{} J[3D Scatter]}
{Highlighting}
{Shaded}
\end{verbatim}
\end{center}

\textbf{Applications in Machine Learning:}

\begin{itemize}
\tightlist
\item
  \textbf{Data Exploration}: Visualize data distribution and patterns
\item
  \textbf{Model Performance}: Plot accuracy, loss curves during training
\item
  \textbf{Result Presentation}: Display predictions vs actual values
\item
  \textbf{Feature Analysis}: Correlation matrices, feature importance
  plots
\end{itemize}

\textbf{Advanced Capabilities:}

\begin{itemize}
\tightlist
\item
  \textbf{Animation}: Create animated plots for time-series data
\item
  \textbf{Interactive Widgets}: Add sliders, buttons for user
  interaction
\item
  \textbf{Integration}: Works with Jupyter notebooks, web applications
\end{itemize}

\textbf{Benefits:}

\begin{itemize}
\tightlist
\item
  \textbf{Flexibility}: Highly customizable plotting options
\item
  \textbf{Community}: Large user base with extensive documentation
\item
  \textbf{Compatibility}: Integrates with NumPy, Pandas seamlessly
\end{itemize}

\end{solutionbox}
\begin{mnemonicbox}
``Matplotlib Makes Meaningful Visual Displays''
(MMVD)

\end{mnemonicbox}
\begin{center}\rule{0.5\linewidth}{0.5pt}\end{center}

\subsection*{Question 5(a) [3 marks]}\label{q5a}

\textbf{List out the major features of Numpy.}

\begin{solutionbox}

\textbf{NumPy Major Features:}

{\def\LTcaptype{none} % do not increment counter
\begin{longtable}[]{@{}
  >{\raggedright\arraybackslash}p{(\linewidth - 4\tabcolsep) * \real{0.2903}}
  >{\raggedright\arraybackslash}p{(\linewidth - 4\tabcolsep) * \real{0.4194}}
  >{\raggedright\arraybackslash}p{(\linewidth - 4\tabcolsep) * \real{0.2903}}@{}}
\toprule\noalign{}
\begin{minipage}[b]{\linewidth}\raggedright
Feature
\end{minipage} & \begin{minipage}[b]{\linewidth}\raggedright
Description
\end{minipage} & \begin{minipage}[b]{\linewidth}\raggedright
Benefit
\end{minipage} \\
\midrule\noalign{}
\endhead
\bottomrule\noalign{}
\endlastfoot
\textbf{N-dimensional Arrays} & Efficient array operations & Fast
mathematical computations \\
\textbf{Broadcasting} & Operations on different sized arrays & Flexible
array manipulation \\
\textbf{Linear Algebra} & Matrix operations, decompositions & Scientific
computing support \\
\end{longtable}
}

\begin{itemize}
\tightlist
\item
  \textbf{Universal Functions}: Element-wise operations on arrays
\item
  \textbf{Memory Efficiency}: Contiguous memory layout for speed
\item
  \textbf{C/C++ Integration}: Interface with compiled languages
\end{itemize}

\end{solutionbox}
\begin{mnemonicbox}
``NumPy Numbers Need Neat Operations'' (NNNNO)

\end{mnemonicbox}
\begin{center}\rule{0.5\linewidth}{0.5pt}\end{center}

\subsection*{Question 5(b) [4 marks]}\label{q5b}

\textbf{How to load an iris dataset csv file in a Pandas Dataframe
program? Explain with example.}

\begin{solutionbox}

\textbf{Loading Iris Dataset:}

\begin{verbatim}
import pandas as pd

\# Method 1: Load from file
df = pd.read\_csv({iris.csv})

\# Method 2: Load from sklearn
from sklearn.datasets import load\_iris
iris = load\_iris()
df = pd.DataFrame(iris.data, columns=iris.feature\_names)
df[{target}] = iris.target

\# Display basic information
print(df.head())
print(df.info())
print(df.describe())
\end{verbatim}

\textbf{Code Explanation:}

\begin{itemize}
\tightlist
\item
  \textbf{pd.read\_csv()}: Reads CSV file into DataFrame
\item
  \textbf{columns parameter}: Assigns column names
\item
  \textbf{head()}: Shows first 5 rows
\item
  \textbf{info()}: Displays data types and memory usage
\end{itemize}

\end{solutionbox}
\begin{mnemonicbox}
``Pandas Reads CSV Files Easily'' (PRCFE)

\end{mnemonicbox}
\begin{center}\rule{0.5\linewidth}{0.5pt}\end{center}

\subsection*{Question 5(c) [7 marks]}\label{q5c}

\textbf{Compare and Contrast Supervised Learning and Unsupervised
Learning.}

\begin{solutionbox}

\textbf{Comprehensive Comparison:}

{\def\LTcaptype{none} % do not increment counter
\begin{longtable}[]{@{}
  >{\raggedright\arraybackslash}p{(\linewidth - 4\tabcolsep) * \real{0.1667}}
  >{\raggedright\arraybackslash}p{(\linewidth - 4\tabcolsep) * \real{0.3958}}
  >{\raggedright\arraybackslash}p{(\linewidth - 4\tabcolsep) * \real{0.4375}}@{}}
\toprule\noalign{}
\begin{minipage}[b]{\linewidth}\raggedright
Aspect
\end{minipage} & \begin{minipage}[b]{\linewidth}\raggedright
Supervised Learning
\end{minipage} & \begin{minipage}[b]{\linewidth}\raggedright
Unsupervised Learning
\end{minipage} \\
\midrule\noalign{}
\endhead
\bottomrule\noalign{}
\endlastfoot
\textbf{Data Type} & Labeled (input-output pairs) & Unlabeled (input
only) \\
\textbf{Learning Goal} & Predict target variable & Discover hidden
patterns \\
\textbf{Evaluation} & Accuracy, precision, recall & Silhouette score,
inertia \\
\textbf{Complexity} & Less complex to evaluate & More complex to
validate \\
\textbf{Applications} & Classification, regression & Clustering,
dimensionality reduction \\
\end{longtable}
}

\textbf{Detailed Comparison:}

\begin{center}
\textbf{Mermaid Diagram (Code)}
\begin{verbatim}
{Shaded}
{Highlighting}[]
graph TD
    A[Machine Learning] {-{-}{} B[Supervised]}
    A {-{-}{} C[Unsupervised]}
    B {-{-}{} D[Classification]}
    B {-{-}{} E[Regression]}
    C {-{-}{} F[Clustering]}
    C {-{-}{} G[Association Rules]}
{Highlighting}
{Shaded}
\end{verbatim}
\end{center}

\textbf{Supervised Learning Characteristics:}

\begin{itemize}
\tightlist
\item
  \textbf{Training Process}: Learn from examples with known correct
  answers
\item
  \textbf{Performance Measurement}: Direct comparison with actual
  outcomes
\item
  \textbf{Common Algorithms}: Decision trees, SVM, neural networks
\item
  \textbf{Business Applications}: Fraud detection, medical diagnosis,
  price prediction
\end{itemize}

\textbf{Unsupervised Learning Characteristics:}

\begin{itemize}
\tightlist
\item
  \textbf{Exploration}: Find unknown patterns without guidance
\item
  \textbf{Validation Challenges}: No ground truth for direct comparison
\item
  \textbf{Common Algorithms}: K-means, hierarchical clustering, PCA
\item
  \textbf{Business Applications}: Customer segmentation, market
  research, anomaly detection
\end{itemize}

\textbf{Key Contrasts:}

\begin{itemize}
\tightlist
\item
  \textbf{Feedback}: Supervised has immediate feedback, unsupervised
  relies on domain expertise
\item
  \textbf{Data Requirements}: Supervised needs expensive labeled data,
  unsupervised uses readily available unlabeled data
\item
  \textbf{Problem Types}: Supervised solves prediction problems,
  unsupervised solves discovery problems
\end{itemize}

\end{solutionbox}
\begin{mnemonicbox}
``Supervised Seeks Specific Solutions, Unsupervised
Uncovers Unknown'' (SSSUU)

\end{mnemonicbox}
\begin{center}\rule{0.5\linewidth}{0.5pt}\end{center}

\subsection*{Question 5(a OR) [3
marks]}\label{question-5a-or-3-marks}

\textbf{List out the applications of Pandas.}

\begin{solutionbox}

\textbf{Pandas Applications:}

{\def\LTcaptype{none} % do not increment counter
\begin{longtable}[]{@{}
  >{\raggedright\arraybackslash}p{(\linewidth - 4\tabcolsep) * \real{0.3611}}
  >{\raggedright\arraybackslash}p{(\linewidth - 4\tabcolsep) * \real{0.3611}}
  >{\raggedright\arraybackslash}p{(\linewidth - 4\tabcolsep) * \real{0.2778}}@{}}
\toprule\noalign{}
\begin{minipage}[b]{\linewidth}\raggedright
Application
\end{minipage} & \begin{minipage}[b]{\linewidth}\raggedright
Description
\end{minipage} & \begin{minipage}[b]{\linewidth}\raggedright
Industry
\end{minipage} \\
\midrule\noalign{}
\endhead
\bottomrule\noalign{}
\endlastfoot
\textbf{Data Cleaning} & Handle missing values, duplicates & All
industries \\
\textbf{Financial Analysis} & Stock market, trading data & Finance \\
\textbf{Business Intelligence} & Sales reports, KPI analysis &
Business \\
\end{longtable}
}

\begin{itemize}
\tightlist
\item
  \textbf{Scientific Research}: Experimental data analysis
\item
  \textbf{Web Analytics}: Website traffic, user behavior analysis
\item
  \textbf{Healthcare}: Patient records, clinical trial data
\end{itemize}

\end{solutionbox}
\begin{mnemonicbox}
``Pandas Processes Data Perfectly'' (PPDP)

\end{mnemonicbox}
\begin{center}\rule{0.5\linewidth}{0.5pt}\end{center}

\subsection*{Question 5(b OR) [4
marks]}\label{question-5b-or-4-marks}

\textbf{How to plot a vertical line and horizontal line in matplotlib?
Explain with examples.}

\begin{solutionbox}

\textbf{Matplotlib Line Plotting:}

\begin{verbatim}
import matplotlib.pyplot as plt
import numpy as np

\# Create sample data
x = np.linspace(0, 10, 100)
y = np.sin(x)

\# Plot the main curve
plt.plot(x, y, label={sin(x)})

\# Vertical line at x = 5
plt.axvline(x=5, color={red}, linestyle={{-}{-}}, label={Vertical Line})

\# Horizontal line at y = 0.5
plt.axhline(y=0.5, color={green}, linestyle={:}, label={Horizontal Line})

\# Formatting
plt.xlabel({X{-}axis})
plt.ylabel({Y{-}axis})
plt.legend()
plt.title({Vertical and Horizontal Lines})
plt.grid(True)
plt.show()
\end{verbatim}

\textbf{Key Functions:}

\begin{itemize}
\tightlist
\item
  \textbf{axvline()}: Creates vertical line at specified x-coordinate
\item
  \textbf{axhline()}: Creates horizontal line at specified y-coordinate
\item
  \textbf{Parameters}: color, linestyle, linewidth, alpha
\end{itemize}

\end{solutionbox}
\begin{mnemonicbox}
``Matplotlib Makes Lines Easily'' (MMLE)

\end{mnemonicbox}
\begin{center}\rule{0.5\linewidth}{0.5pt}\end{center}

\subsection*{Question 5(c OR) [7
marks]}\label{question-5c-or-7-marks}

\textbf{Describe the concept of clustering using appropriate real-world
examples.}

\begin{solutionbox}

\textbf{Clustering Concept and Applications:}

{\def\LTcaptype{none} % do not increment counter
\begin{longtable}[]{@{}
  >{\raggedright\arraybackslash}p{(\linewidth - 4\tabcolsep) * \real{0.3077}}
  >{\raggedright\arraybackslash}p{(\linewidth - 4\tabcolsep) * \real{0.3654}}
  >{\raggedright\arraybackslash}p{(\linewidth - 4\tabcolsep) * \real{0.3269}}@{}}
\toprule\noalign{}
\begin{minipage}[b]{\linewidth}\raggedright
Clustering Type
\end{minipage} & \begin{minipage}[b]{\linewidth}\raggedright
Real-World Example
\end{minipage} & \begin{minipage}[b]{\linewidth}\raggedright
Business Impact
\end{minipage} \\
\midrule\noalign{}
\endhead
\bottomrule\noalign{}
\endlastfoot
\textbf{Customer Segmentation} & Group customers by purchase behavior &
Targeted marketing campaigns \\
\textbf{Image Segmentation} & Medical imaging for tumor detection &
Improved diagnosis accuracy \\
\textbf{Gene Analysis} & Group genes with similar expression & Drug
discovery and treatment \\
\end{longtable}
}

\textbf{Clustering Process:}

\begin{verbatim}
flowchart LR
    A[Raw Data] {-{-} B[Feature Selection]}
    B {-{-} C[Distance Calculation]}
    C {-{-} D[Cluster Formation]}
    D {-{-} E[Cluster Validation]}
    E {-{-} F[Business Insights]}
\end{verbatim}

\textbf{Detailed Examples:}

\textbf{1. Customer Segmentation:}

\begin{itemize}
\tightlist
\item
  \textbf{Data}: Purchase history, demographics, website behavior
\item
  \textbf{Clusters}: High-value customers, price-sensitive buyers,
  occasional shoppers
\item
  \textbf{Business Value}: Customized marketing, product
  recommendations, retention strategies
\end{itemize}

\textbf{2. Social Media Analysis:}

\begin{itemize}
\tightlist
\item
  \textbf{Data}: User interactions, post topics, engagement patterns
\item
  \textbf{Clusters}: Influencers, casual users, brand advocates
\item
  \textbf{Applications}: Viral marketing, content strategy, community
  management
\end{itemize}

\textbf{3. Market Research:}

\begin{itemize}
\tightlist
\item
  \textbf{Data}: Survey responses, product preferences, demographics
\item
  \textbf{Clusters}: Market segments with similar needs
\item
  \textbf{Insights}: Product development, pricing strategy, market
  positioning
\end{itemize}

\textbf{Clustering Algorithms:}

\begin{itemize}
\tightlist
\item
  \textbf{K-Means}: Partitions data into k clusters
\item
  \textbf{Hierarchical}: Creates tree-like cluster structure
\item
  \textbf{DBSCAN}: Finds clusters of varying density
\end{itemize}

\textbf{Validation Methods:}

\begin{itemize}
\tightlist
\item
  \textbf{Silhouette Score}: Measures cluster quality
\item
  \textbf{Elbow Method}: Determines optimal number of clusters
\item
  \textbf{Domain Expertise}: Business knowledge validation
\end{itemize}

\textbf{Benefits:}

\begin{itemize}
\tightlist
\item
  \textbf{Pattern Discovery}: Reveals hidden data structures
\item
  \textbf{Decision Support}: Provides insights for business decisions
\item
  \textbf{Automation}: Reduces manual data analysis effort
\end{itemize}

\end{solutionbox}
\begin{mnemonicbox}
``Clustering Creates Clear Categories'' (CCCC)

\end{mnemonicbox}

\end{document}
