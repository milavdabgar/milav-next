\documentclass[10pt,a4paper]{article}

% content/resources/templates/preamble.tex
\usepackage[margin=0.6in]{geometry}
\author{Milav Dabgar}
\usepackage{amsmath,amssymb,amsthm}
\usepackage{booktabs}
\usepackage{multirow}
\usepackage{xcolor}
\usepackage{tcolorbox}
\tcbuselibrary{breakable,skins}
\usepackage[colorlinks=true,linkcolor=blue]{hyperref}
\usepackage{titlesec}
\usepackage{enumitem}
\usepackage{tikz}
\usepackage{pgfplots}
\usepackage{circuitikz}
\usepackage[version=4]{mhchem}
\usepackage{longtable}
\usepackage{array}
\usepackage{float}
\usepackage{caption}
\usepackage{listings}

\lstset{
  basicstyle=\small\ttfamily,
  breaklines=true,
  breakatwhitespace=false,
  postbreak=\mbox{\textcolor{red}{$\hookrightarrow$}\space},
  float=false,
  numbers=left,
  numberstyle=\tiny\color{gray},
  numbersep=10pt,
  xleftmargin=2em,
  keywordstyle=\color{blue},
  commentstyle=\color{green!60!black},
  stringstyle=\color{purple},
  backgroundcolor=\color{gray!5},
  showstringspaces=false,
  tabsize=2,
  captionpos=b,
  keepspaces=true,
  columns=flexible
}

\pgfplotsset{compat=1.18}
\usetikzlibrary{shapes,arrows,positioning,calc,patterns,decorations.pathmorphing,decorations.markings,arrows.meta}

% Color scheme
\definecolor{headcolor}{RGB}{0,102,204}
\definecolor{keycolor}{RGB}{220,20,60}
\definecolor{solutioncolor}{RGB}{34,139,34}
\definecolor{mnemoniccolor}{RGB}{148,0,211}
\definecolor{codecolor}{RGB}{0,0,100}

% Spacing
\setlength{\parskip}{3pt}
\setlist[itemize]{nosep}
\setlist[enumerate]{nosep}

% Title formatting
\titleformat{\section}{\Large\bfseries\color{headcolor}}{\thesection}{1em}{}
\titleformat{\subsection}{\large\bfseries\color{headcolor}}{\thesubsection}{1em}{}

% Pandoc tightlist compatibility
\providecommand{\tightlist}{%
  \setlength{\itemsep}{0pt}\setlength{\parskip}{0pt}}

% Pandoc longtable compatibility
\newcounter{none}
\def\thenone{}


% content/resources/templates/gujarati-boxes.tex
\usepackage{fontspec}
\usepackage{polyglossia}

% Set Gujarati as main language (document is primarily in Gujarati)
% Note: gloss-gujarati.ldf doesn't exist in polyglossia, but it will use hyphenation patterns
\setdefaultlanguage{gujarati}
\setotherlanguage{english}

% Configure Gujarati font properly
% Use Language=Default to prevent polyglossia from trying to add language-specific features
% that don't exist for Gujarati, which causes "empty feature" warnings
\newfontfamily\gujaratifont[Script=Gujarati,AutoFakeBold=2.5,AutoFakeSlant=0.3]{Noto Sans Gujarati}
\setmainfont[Script=Gujarati,AutoFakeBold=2.5,AutoFakeSlant=0.3]{Noto Sans Gujarati}
% Use Noto Sans Gujarati for monospace to support Gujarati in text
\setmonofont[Scale=0.9]{Noto Sans Gujarati}

% Configure English to use the same font
\newfontfamily\englishfont[Script=Gujarati,AutoFakeBold=2.5,AutoFakeSlant=0.3]{Noto Sans Gujarati}

% Translations for polyglossia
\gappto\captionsgujarati{
  \renewcommand{\tablename}{કોષ્ટક}
  \renewcommand{\figurename}{આકૃતિ}
}

% Helper for TikZ nodes to ensure Gujarati font
\newcommand{\gu}[1]{{\gujaratifont #1}}

% Custom environments
\newtcolorbox{solutionbox}{
    breakable,
    enhanced,
    colback=solutioncolor!5!white,
    colframe=solutioncolor!75!black,
    fonttitle=\bfseries,
    title=જવાબ
}

\newtcolorbox{solutionboxnobreak}{
 colback=solutioncolor!5!white,
 colframe=solutioncolor!75!black,
 fonttitle=\bfseries,
 title=જવાબ
}

\newtcolorbox{keyformula}{
 breakable,
 enhanced,
 colback=keycolor!5!white,
 colframe=keycolor!75!black,
 fonttitle=\bfseries,
 title=રાસાયણિક સમીકરણ/સૂત્ર
}

\newtcolorbox{mnemonicbox}{
 breakable,
 enhanced,
 colback=mnemoniccolor!5!white,
 colframe=mnemoniccolor!75!black,
 fonttitle=\bfseries,
 title=મેમરી ટ્રીક
}


\begin{document}

\begin{center}
{\Huge\bfseries\color{headcolor} Subject Name (Gujarati)}\\[5pt]
{\LARGE 4341603 -- Summer 2025}\\[3pt]
{\large Semester 1 Study Material}\\[3pt]
{\normalsize\textit{Detailed Solutions and Explanations}}
\end{center}

\vspace{10pt}

\subsection*{પ્રશ્ન 1(અ) [3
ગુણ]}\label{uxaaauxab0uxab6uxaa8-1uxa85-3-uxa97uxaa3}

\textbf{મશીન લર્નિંગની વ્યાખ્યા આપો. મશીન લર્નિંગની કોઈપણ બે ઉપયોગીતાઓ આપો.}

\begin{solutionbox}

મશીન લર્નિંગ એ આર્ટિફિશિયલ ઇન્ટેલિજન્સનો એક ભાગ છે જે કમ્પ્યુટરને ડેટામાંથી શીખવા અને
દરેક કાર્ય માટે સ્પષ્ટ પ્રોગ્રામિંગ વિના નિર્ણયો લેવાની ક્ષમતા આપે છે.

\textbf{ઉપયોગીતાઓ:}

\begin{itemize}
\tightlist
\item
  \textbf{ઈમેઇલ સ્પામ ડિટેક્શન}: આપોઆપ સ્પામ ઈમેઇલ ઓળખે અને ફિલ્ટર કરે છે
\item
  \textbf{સુઝાવ સિસ્ટમ}: Amazon જેવી ઈ-કોમર્સ સાઇટ્સ પર પ્રોડક્ટ સુઝાવે છે
\end{itemize}

\textbf{ટેબલ: ML વિ ટ્રેડિશનલ પ્રોગ્રામિંગ}

{\def\LTcaptype{none} % do not increment counter
\begin{longtable}[]{@{}ll@{}}
\toprule\noalign{}
પરંપરાગત પ્રોગ્રામિંગ & મશીન લર્નિંગ \\
\midrule\noalign{}
\endhead
\bottomrule\noalign{}
\endlastfoot
ઇનપુટ ડેટા + પ્રોગ્રામ \rightarrow આઉટપુટ & ઇનપુટ ડેટા + આઉટપુટ \rightarrow પ્રોગ્રામ \\
નિયમો સ્પષ્ટપણે કોડ કરવામાં આવે છે & નિયમો ડેટામાંથી શીખવામાં આવે છે \\
\end{longtable}
}

\end{solutionbox}
\begin{mnemonicbox}
``ML = ડેટામાંથી શીખવું બનાવો''

\end{mnemonicbox}
\subsection*{પ્રશ્ન 1(બ) [4
ગુણ]}\label{uxaaauxab0uxab6uxaa8-1uxaac-4-uxa97uxaa3}

\textbf{વ્યાખ્યા આપો: અંડર ફિટિંગ અને ઓવર ફિટિંગ.}

\begin{solutionbox}

\textbf{અંડરફિટિંગ} ત્યારે થાય છે જ્યારે મોડલ ડેટામાં છુપાયેલા પેટર્ન કેપ્ચર કરવા માટે
ખૂબ સાદું હોય છે, જેના પરિણામે ટ્રેનિંગ અને ટેસ્ટ બંને ડેટા પર નબળી કામગીરી થાય છે.

\textbf{ઓવરફિટિંગ} ત્યારે થાય છે જ્યારે મોડલ ટ્રેનિંગ ડેટાને અવાજ સહિત ખૂબ સારી રીતે
શીખે છે, જેના કારણે નવા અદ્રશ્ય ડેટા પર નબળી કામગીરી થાય છે.

\textbf{ટેબલ: સરખામણી}

{\def\LTcaptype{none} % do not increment counter
\begin{longtable}[]{@{}lll@{}}
\toprule\noalign{}
પાસું & અંડરફિટિંગ & ઓવરફિટિંગ \\
\midrule\noalign{}
\endhead
\bottomrule\noalign{}
\endlastfoot
\textbf{ટ્રેનિંગ એક્યુરેસી} & ઓછી & વધારે \\
\textbf{ટેસ્ટ એક્યુરેસી} & ઓછી & ઓછી \\
\textbf{મોડલ કોમ્પ્લેક્સિટી} & ખૂબ સાદું & ખૂબ જટિલ \\
\textbf{સોલ્યુશન} & કોમ્પ્લેક્સિટી વધારો & કોમ્પ્લેક્સિટી ઘટાડો \\
\end{longtable}
}

\end{solutionbox}
\begin{mnemonicbox}
``અંડર = ઓછું કામ, ઓવર = વધુ પડતું શીખવું''

\end{mnemonicbox}
\subsection*{પ્રશ્ન 1(ક) [7
ગુણ]}\label{uxaaauxab0uxab6uxaa8-1uxa95-7-uxa97uxaa3}

\textbf{મશીન લર્નિંગના વિવિધ પ્રકારો યોગ્ય ઉદાહરણની મદદથી વર્ણવો.}

\begin{solutionbox}

\textbf{ટેબલ: મશીન લર્નિંગના પ્રકારો}

{\def\LTcaptype{none} % do not increment counter
\begin{longtable}[]{@{}lll@{}}
\toprule\noalign{}
પ્રકાર & વર્ણન & ઉદાહરણ \\
\midrule\noalign{}
\endhead
\bottomrule\noalign{}
\endlastfoot
\textbf{સુપરવાઇઝ્ડ} & લેબલ કરેલ ટ્રેનિંગ ડેટા વાપરે છે & ઈમેઇલ વર્ગીકરણ \\
\textbf{અનસુપરવાઇઝ્ડ} & લેબલ કરેલ ડેટા નથી, પેટર્ન શોધે છે & કસ્ટમર સેગમેન્ટેશન \\
\textbf{રિઇન્ફોર્સમેન્ટ} & પુરસ્કાર/દંડ દ્વારા શીખે છે & ગેમ રમતું AI \\
\end{longtable}
}

\textbf{સુપરવાઇઝ્ડ લર્નિંગ} ઇનપુટ-આઉટપુટ જોડીઓ વાપરીને મોડલ ટ્રેન કરે છે. અલ્ગોરિધમ
ઉદાહરણોમાંથી શીખીને નવા ડેટા માટે પરિણામોની આગાહી કરે છે.

\textbf{અનસુપરવાઇઝ્ડ લર્નિંગ} ટાર્ગેટ લેબલ વિના ડેટામાં છુપાયેલા પેટર્ન શોધે છે. તે સમાન
ડેટા પોઇન્ટ્સને એકસાથે જૂથબદ્ધ કરે છે.

\textbf{રિઇન્ફોર્સમેન્ટ લર્નિંગ} સારા કાર્યો માટે પુરસ્કાર અને ખરાબ કાર્યો માટે દંડ
આપીને એજન્ટને નિર્ણય લેવાનું શીખવે છે.

\textbf{ડાયાગ્રામ:}

\begin{center}
\textbf{Mermaid Diagram (Code)}
\begin{verbatim}
{Shaded}
{Highlighting}[]
graph TD
    A[મશીન લર્નિંગ] {-{-}{} B[સુપરવાઇઝ્ડ લર્નિંગ]}
    A {-{-}{} C[અનસુપરવાઇઝ્ડ લર્નિંગ]}
    A {-{-}{} D[રિઇન્ફોર્સમેન્ટ લર્નિંગ]}
    B {-{-}{} E[ક્લાસિફિકેશન]}
    B {-{-}{} F[રિગ્રેશન]}
    C {-{-}{} G[ક્લસ્ટરિંગ]}
    C {-{-}{} H[એસોસિએશન રૂલ્સ]}
{Highlighting}
{Shaded}
\end{verbatim}
\end{center}

\end{solutionbox}
\begin{mnemonicbox}
``સુપર અન-સુપરવાઇઝ્ડ રિઇન્ફોર્સ શીખવું''

\end{mnemonicbox}
\subsection*{પ્રશ્ન 1(ક) અથવા [7
ગુણ]}\label{uxaaauxab0uxab6uxaa8-1uxa95-uxa85uxaa5uxab5-7-uxa97uxaa3}

\textbf{મશીન લર્નિંગમાં ઉપયોગ થતી વિવિધ ટૂલ્સ અને ટેકનોલોજી વર્ણવો.}

\begin{solutionbox}

\textbf{ટેબલ: ML ટૂલ્સ અને ટેકનોલોજીઓ}

{\def\LTcaptype{none} % do not increment counter
\begin{longtable}[]{@{}lll@{}}
\toprule\noalign{}
કેટેગરી & ટૂલ્સ & હેતુ \\
\midrule\noalign{}
\endhead
\bottomrule\noalign{}
\endlastfoot
\textbf{પ્રોગ્રામિંગ} & Python, R & મુખ્ય ડેવલપમેન્ટ \\
\textbf{લાઇબ્રેરીઓ} & Scikit-learn, TensorFlow & મોડલ બિલ્ડિંગ \\
\textbf{ડેટા પ્રોસેસિંગ} & Pandas, NumPy & ડેટા મેનિપ્યુલેશન \\
\textbf{વિઝ્યુલાઇઝેશન} & Matplotlib, Seaborn & ડેટા પ્લોટિંગ \\
\end{longtable}
}

\textbf{Python} તેની સરળતા અને વ્યાપક લાઇબ્રેરીઓને કારણે સૌથી લોકપ્રિય ભાષા છે.

\textbf{Scikit-learn} ડેટા માઇનિંગ અને વિશ્લેષણ માટે સરળ ટૂલ્સ પ્રદાન કરે છે, જે
શરૂઆતીઓ માટે પરફેક્ટ છે.

\textbf{TensorFlow} અને \textbf{PyTorch} ડીપ લર્નિંગ એપ્લિકેશન માટે એડવાન્સ
ફ્રેમવર્ક છે.

\textbf{Jupyter Notebook} પ્રયોગ માટે ઇન્ટરેક્ટિવ ડેવલપમેન્ટ એન્વાયર્નમેન્ટ ઓફર કરે
છે.

\textbf{ડાયાગ્રામ:}

\begin{center}
\textbf{Mermaid Diagram (Code)}
\begin{verbatim}
{Shaded}
{Highlighting}[]
graph LR
    A[ડેટા] {-{-}{} B[Pandas/NumPy]}
    B {-{-}{} C[Scikit{-}learn]}
    C {-{-}{} D[મોડલ]}
    D {-{-}{} E[Matplotlib]}
{Highlighting}
{Shaded}
\end{verbatim}
\end{center}

\end{solutionbox}
\begin{mnemonicbox}
``Python Pandas Scikit Tensor Jupyter''

\end{mnemonicbox}
\subsection*{પ્રશ્ન 2(અ) [3
ગુણ]}\label{uxaaauxab0uxab6uxaa8-2uxa85-3-uxa97uxaa3}

\textbf{Qualitative ડેટા અને Quantitative ડેટા વચ્ચેનો તફાવત આપો.}

\begin{solutionbox}

\textbf{ટેબલ: Qualitative વિ Quantitative ડેટા}

{\def\LTcaptype{none} % do not increment counter
\begin{longtable}[]{@{}ll@{}}
\toprule\noalign{}
Qualitative ડેટા & Quantitative ડેટા \\
\midrule\noalign{}
\endhead
\bottomrule\noalign{}
\endlastfoot
\textbf{બિન-સંખ્યાત્મક} કેટેગરીઓ & \textbf{સંખ્યાત્મક} મૂલ્યો \\
રંગો, નામો, ગ્રેડ્સ & ઊંચાઈ, વજન, કિંમત \\
માપી શકાતું નથી & માપી શકાય છે \\
\end{longtable}
}

\textbf{Qualitative ડેટા} એવા ગુણો અથવા લક્ષણોનું વર્ણન કરે છે જે સંખ્યાત્મક રીતે
માપી શકાતા નથી.

\textbf{Quantitative ડેટા} સંખ્યાઓ તરીકે વ્યક્ત કરેલા માપી શકાય તેવા જથ્થાઓનું
પ્રતિનિધિત્વ કરે છે.

\end{solutionbox}
\begin{mnemonicbox}
``Quality = કેટેગરીઓ, Quantity = સંખ્યાઓ''

\end{mnemonicbox}
\subsection*{પ્રશ્ન 2(બ) [4
ગુણ]}\label{uxaaauxab0uxab6uxaa8-2uxaac-4-uxa97uxaa3}

\textbf{નીચે આપેલા ડેટાનું mean અને median શોધો: 3,4,5,5,7,8,9,11,12,14.}

\begin{solutionbox}

\textbf{આપેલ ડેટા:} 3, 4, 5, 5, 7, 8, 9, 11, 12, 14

\textbf{Mean ગણતરી:}

\begin{itemize}
\tightlist
\item
  સરવાળો = 3+4+5+5+7+8+9+11+12+14 = 78
\item
  સંખ્યાઓની ગિનતી = 10
\item
  \textbf{Mean = 78/10 = 7.8}
\end{itemize}

\textbf{Median ગણતરી:}

\begin{itemize}
\tightlist
\item
  ડેટા પહેલેથી જ સોર્ટ થયેલ છે
\item
  10 સંખ્યાઓ માટે: Median = (5મી + 6ઠી મૂલ્ય)/2
\item
  \textbf{Median = (7+8)/2 = 7.5}
\end{itemize}

\textbf{ટેબલ: પરિણામો}

{\def\LTcaptype{none} % do not increment counter
\begin{longtable}[]{@{}ll@{}}
\toprule\noalign{}
માપદંડ & મૂલ્ય \\
\midrule\noalign{}
\endhead
\bottomrule\noalign{}
\endlastfoot
\textbf{Mean} & 7.8 \\
\textbf{Median} & 7.5 \\
\end{longtable}
}

\end{solutionbox}
\begin{mnemonicbox}
``Mean = સરેરાશ, Median = મધ્યક''

\end{mnemonicbox}
\subsection*{પ્રશ્ન 2(ક) [7
ગુણ]}\label{uxaaauxab0uxab6uxaa8-2uxa95-7-uxa97uxaa3}

\textbf{મશીન લર્નિંગની એક્ટિવિટી વિગતવાર વર્ણવો.}

\begin{solutionbox}

\textbf{ટેબલ: મશીન લર્નિંગ એક્ટિવિટીઓ}

{\def\LTcaptype{none} % do not increment counter
\begin{longtable}[]{@{}lll@{}}
\toprule\noalign{}
એક્ટિવિટી & વર્ણન & ઉદાહરણ \\
\midrule\noalign{}
\endhead
\bottomrule\noalign{}
\endlastfoot
\textbf{ડેટા કલેક્શન} & સંબંધિત ડેટા એકત્રિત કરવું & સર્વે પ્રતિભાવો \\
\textbf{ડેટા પ્રીપ્રોસેસિંગ} & ડેટા સાફ અને તૈયાર કરવું & ડુપ્લિકેટ્સ દૂર કરવા \\
\textbf{ફીચર સિલેક્શન} & મહત્વપૂર્ણ વેરિયેબલ્સ પસંદ કરવા & લોન માટે ઉંમર, આવક \\
\textbf{મોડલ ટ્રેનિંગ} & અલ્ગોરિધમને પેટર્ન શીખવવું & ટ્રેનિંગ ડેટા ખવડાવવો \\
\textbf{મોડલ ઇવેલ્યુએશન} & મોડલની કામગીરી પરીક્ષણ & એક્યુરેસી મેઝરમેન્ટ \\
\end{longtable}
}

\textbf{ડેટા કલેક્શન} ડેટાબેસ, સેન્સર્સ અથવા સર્વે જેવા વિવિધ સ્રોતોમાંથી માહિતી
એકત્રિત કરવાનો સમાવેશ કરે છે.

\textbf{ડેટા પ્રીપ્રોસેસિંગ} વિશ્લેષણ માટે કાચા ડેટાને સાફ, રૂપાંતર અને ગોઠવવાનો
સમાવેશ કરે છે.

\textbf{ફીચર સિલેક્શન} આગાહીઓમાં યોગદાન આપતા સૌથી સંબંધિત વેરિયેબલ્સ ઓળખે છે.

\textbf{મોડલ ટ્રેનિંગ} તૈયાર કરેલા ટ્રેનિંગ ડેટામાંથી પેટર્ન શીખવા માટે અલ્ગોરિધમ્સનો
ઉપયોગ કરે છે.

\textbf{મોડલ ઇવેલ્યુએશન} ટ્રેન કરેલ મોડલ નવા, અદ્રશ્ય ડેટા પર કેટલી સારી કામગીરી
કરે છે તેનું પરીક્ષણ કરે છે.

\textbf{ડાયાગ્રામ:}

\begin{verbatim}
flowchart LR
    A[ડેટા કલેક્શન] {-{-} B[ડેટા પ્રીપ્રોસેસિંગ]}
    B {-{-} C[ફીચર સિલેક્શન]}
    C {-{-} D[મોડલ ટ્રેનિંગ]}
    D {-{-} E[મોડલ ઇવેલ્યુએશન]}
    E {-{-} F[ડિપ્લોયમેન્ટ]}
\end{verbatim}

\end{solutionbox}
\begin{mnemonicbox}
``કલેક્ટ પ્રોસેસ ફીચર ટ્રેન ઇવેલ્યુએટ ડિપ્લોય''

\end{mnemonicbox}
\subsection*{પ્રશ્ન 2(અ) અથવા [3
ગુણ]}\label{uxaaauxab0uxab6uxaa8-2uxa85-uxa85uxaa5uxab5-3-uxa97uxaa3}

\textbf{Predictive મોડલ અને Descriptive મોડલ વચ્ચેનો તફાવત આપો.}

\begin{solutionbox}

\textbf{ટેબલ: Predictive વિ Descriptive મોડલ્સ}

{\def\LTcaptype{none} % do not increment counter
\begin{longtable}[]{@{}ll@{}}
\toprule\noalign{}
Predictive મોડલ & Descriptive મોડલ \\
\midrule\noalign{}
\endhead
\bottomrule\noalign{}
\endlastfoot
\textbf{ભવિષ્યના} પરિણામોની આગાહી કરે છે & \textbf{વર્તમાન} પેટર્નનું સમજૂતી આપે
છે \\
સુપરવાઇઝ્ડ લર્નિંગ વાપરે છે & અનસુપરવાઇઝ્ડ લર્નિંગ વાપરે છે \\
સ્ટોક પ્રાઇસ પ્રિડિક્શન & કસ્ટમર સેગમેન્ટેશન \\
\end{longtable}
}

\textbf{Predictive મોડલ્સ} ભવિષ્યની ઘટનાઓ અથવા અજાણ્યા પરિણામોની આગાહી કરવા
માટે ઐતિહાસિક ડેટાનો ઉપયોગ કરે છે.

\textbf{Descriptive મોડલ્સ} વર્તમાન પેટર્ન અને સંબંધોને સમજવા માટે હાલના ડેટાનું
વિશ્લેષણ કરે છે.

\end{solutionbox}
\begin{mnemonicbox}
``Predict = ભવિષ્ય, Describe = વર્તમાન''

\end{mnemonicbox}
\subsection*{પ્રશ્ન 2(બ) અથવા [4
ગુણ]}\label{uxaaauxab0uxab6uxaa8-2uxaac-uxa85uxaa5uxab5-4-uxa97uxaa3}

\textbf{નીચે આપેલા ડેટાને યોગ્ય ડેટા ટાઇપની મદદથી classify કરો: hair color,
gender, blood group type, time of day.}

\begin{solutionbox}

\textbf{ટેબલ: ડેટા ટાઇપ ક્લાસિફિકેશન}

{\def\LTcaptype{none} % do not increment counter
\begin{longtable}[]{@{}lll@{}}
\toprule\noalign{}
ડેટા & પ્રકાર & કારણ \\
\midrule\noalign{}
\endhead
\bottomrule\noalign{}
\endlastfoot
\textbf{Hair color} & Nominal & કોઈ ક્રમ વિના કેટેગરીઓ \\
\textbf{Gender} & Nominal & કોઈ ક્રમ વિના કેટેગરીઓ \\
\textbf{Blood group} & Nominal & કોઈ ક્રમ વિના કેટેગરીઓ \\
\textbf{Time of day} & Continuous & માપી શકાય તેવી માત્રા \\
\end{longtable}
}

\textbf{Nominal ડેટા} કોઈ કુદરતી ક્રમ વિના કેટેગરીઓનું પ્રતિનિધિત્વ કરે છે.

\textbf{Continuous ડેટા} શ્રેણીમાં કોઈપણ મૂલ્ય લઈ શકે છે અને માપી શકાય છે.

\end{solutionbox}
\begin{mnemonicbox}
``નામો = Nominal, સંખ્યાઓ = Numerical''

\end{mnemonicbox}
\subsection*{પ્રશ્ન 2(ક) અથવા [7
ગુણ]}\label{uxaaauxab0uxab6uxaa8-2uxa95-uxa85uxaa5uxab5-7-uxa97uxaa3}

\textbf{ડેટા પ્રી-પ્રોસેસિંગમાં ઉપયોગ થતી વિવિધ મેથડ્સ વર્ણવો.}

\begin{solutionbox}

\textbf{ટેબલ: ડેટા પ્રીપ્રોસેસિંગ મેથડ્સ}

{\def\LTcaptype{none} % do not increment counter
\begin{longtable}[]{@{}
  >{\raggedright\arraybackslash}p{(\linewidth - 4\tabcolsep) * \real{0.3077}}
  >{\raggedright\arraybackslash}p{(\linewidth - 4\tabcolsep) * \real{0.3462}}
  >{\raggedright\arraybackslash}p{(\linewidth - 4\tabcolsep) * \real{0.3462}}@{}}
\toprule\noalign{}
\begin{minipage}[b]{\linewidth}\raggedright
મેથડ
\end{minipage} & \begin{minipage}[b]{\linewidth}\raggedright
હેતુ
\end{minipage} & \begin{minipage}[b]{\linewidth}\raggedright
ઉદાહરણ
\end{minipage} \\
\midrule\noalign{}
\endhead
\bottomrule\noalign{}
\endlastfoot
\textbf{ડેટા ક્લીનિંગ} & ભૂલો અને અસંગતતાઓ દૂર કરવી & ટાઇપોઝ ઠીક કરવા, ડુપ્લિકેટ્સ
દૂર કરવા \\
\textbf{ડેટા ઇન્ટીગ્રેશન} & બહુવિધ સ્રોતો એકસાથે જોડવા & કસ્ટમર ડેટાબેસ મર્જ
કરવા \\
\textbf{ડેટા ટ્રાન્સફોર્મેશન} & યોગ્ય ફોર્મેટમાં કન્વર્ટ કરવું & 0-1 મૂલ્યો નોર્મલાઇઝ
કરવા \\
\textbf{ડેટા રિડક્શન} & ડેટાસેટનું કદ ઘટાડવું & મહત્વપૂર્ણ ફીચર્સ પસંદ કરવા \\
\end{longtable}
}

\textbf{ડેટા ક્લીનિંગ} ભૂલભરેલ, અધૂરા અથવા અપ્રસ્તુત ડેટાને દૂર કરે છે અથવા સુધારે છે.

\textbf{ડેટા ઇન્ટીગ્રેશન} બહુવિધ સ્રોતોમાંથી ડેટાને એકીકૃત ડેટાસેટમાં જોડે છે.

\textbf{ડેટા ટ્રાન્સફોર્મેશન} વિશ્લેષણ માટે ડેટાને યોગ્ય ફોર્મેટમાં કન્વર્ટ કરે છે.

\textbf{ડેટા રિડક્શન} માહિતીની ગુણવત્તા જાળવીને ડેટાસેટનું કદ ઘટાડે છે.

\textbf{ડાયાગ્રામ:}

\begin{center}
\textbf{Mermaid Diagram (Code)}
\begin{verbatim}
{Shaded}
{Highlighting}[]
graph LR
    A[કાચો ડેટા] {-{-}{} B[ડેટા ક્લીનિંગ]}
    B {-{-}{} C[ડેટા ઇન્ટીગ્રેશન]}
    C {-{-}{} D[ડેટા ટ્રાન્સફોર્મેશન]}
    D {-{-}{} E[ડેટા રિડક્શન]}
    E {-{-}{} F[પ્રોસેસ થયેલ ડેટા]}
{Highlighting}
{Shaded}
\end{verbatim}
\end{center}

\end{solutionbox}
\begin{mnemonicbox}
``ક્લીન ઇન્ટીગ્રેટ ટ્રાન્સફોર્મ રિડ્યુસ''

\end{mnemonicbox}
\subsection*{પ્રશ્ન 3(અ) [3
ગુણ]}\label{uxaaauxab0uxab6uxaa8-3uxa85-3-uxa97uxaa3}

\textbf{Classification અને Regression વચ્ચેનો તફાવત આપો.}

\begin{solutionbox}

\textbf{ટેબલ: Classification વિ Regression}

{\def\LTcaptype{none} % do not increment counter
\begin{longtable}[]{@{}ll@{}}
\toprule\noalign{}
Classification & Regression \\
\midrule\noalign{}
\endhead
\bottomrule\noalign{}
\endlastfoot
\textbf{ડિસ્ક્રીટ} આઉટપુટ & \textbf{કન્ટિન્યુઅસ} આઉટપુટ \\
કેટેગરીઓની આગાહી કરે છે & સંખ્યાત્મક મૂલ્યોની આગાહી કરે છે \\
ઈમેઇલ: સ્પામ/બિન-સ્પામ & ઘરની કિંમત આગાહી \\
\end{longtable}
}

\textbf{Classification} ઇનપુટ ડેટામાંથી ડિસ્ક્રીટ કેટેગરીઓ અથવા ક્લાસની આગાહી કરે
છે.

\textbf{Regression} ઇનપુટ ડેટામાંથી કન્ટિન્યુઅસ સંખ્યાત્મક મૂલ્યોની આગાહી કરે છે.

\end{solutionbox}
\begin{mnemonicbox}
``Class = કેટેગરીઓ, Regress = વાસ્તવિક સંખ્યાઓ''

\end{mnemonicbox}
\subsection*{પ્રશ્ન 3(બ) [4
ગુણ]}\label{uxaaauxab0uxab6uxaa8-3uxaac-4-uxa97uxaa3}

\textbf{યોગ્ય ઉદાહરણ લઈને confusion matrix લખો. તેના માટે accuracy અને error
rate ગણો.}

\begin{solutionbox}

\textbf{ઉદાહરણ: ઈમેઇલ ક્લાસિફિકેશન}

\textbf{ટેબલ: Confusion Matrix}

{\def\LTcaptype{none} % do not increment counter
\begin{longtable}[]{@{}lll@{}}
\toprule\noalign{}
& પ્રિડિક્ટેડ સ્પામ & પ્રિડિક્ટેડ નોટ સ્પામ \\
\midrule\noalign{}
\endhead
\bottomrule\noalign{}
\endlastfoot
\textbf{વાસ્તવિક સ્પામ} & 85 (TP) & 15 (FN) \\
\textbf{વાસ્તવિક નોટ સ્પામ} & 10 (FP) & 90 (TN) \\
\end{longtable}
}

\textbf{ગણતરીઓ:}

\begin{itemize}
\tightlist
\item
  \textbf{Accuracy = (TP+TN)/(TP+TN+FP+FN) = (85+90)/200 = 87.5\%}
\item
  \textbf{Error Rate = (FP+FN)/(TP+TN+FP+FN) = (10+15)/200 = 12.5\%}
\end{itemize}

\textbf{મુખ્ય શબ્દો:}

\begin{itemize}
\tightlist
\item
  \textbf{TP}: True Positive - યોગ્ય રીતે સ્પામ આગાહી
\item
  \textbf{TN}: True Negative - યોગ્ય રીતે નોટ સ્પામ આગાહી
\end{itemize}

\end{solutionbox}
\begin{mnemonicbox}
``True Positive True Negative = યોગ્ય આગાહીઓ''

\end{mnemonicbox}
\subsection*{પ્રશ્ન 3(ક) [7
ગુણ]}\label{uxaaauxab0uxab6uxaa8-3uxa95-7-uxa97uxaa3}

\textbf{KNN અલ્ગોરિધમ વિગતવાર વર્ણવો.}

\begin{solutionbox}

\textbf{K-Nearest Neighbors (KNN)} એક સરળ ક્લાસિફિકેશન અલ્ગોરિધમ છે જે તેમના K
નજીકના પડોશીઓના મેજોરિટી ક્લાસના આધારે ડેટા પોઇન્ટ્સને ક્લાસિફાઇ કરે છે.

\textbf{ટેબલ: KNN અલ્ગોરિધમ સ્ટેપ્સ}

{\def\LTcaptype{none} % do not increment counter
\begin{longtable}[]{@{}lll@{}}
\toprule\noalign{}
સ્ટેપ & વર્ણન & ઉદાહરણ \\
\midrule\noalign{}
\endhead
\bottomrule\noalign{}
\endlastfoot
\textbf{K પસંદ કરો} & પડોશીઓની સંખ્યા પસંદ કરો & K=3 \\
\textbf{અંતર ગણો} & બધા પોઇન્ટ્સનો અંતર શોધો & Euclidean અંતર \\
\textbf{પડોશીઓ શોધો} & K સૌથી નજીકના પોઇન્ટ્સ ઓળખો & 3 નજીકના પોઇન્ટ્સ \\
\textbf{વોટ કરો} & મેજોરિટી ક્લાસ જીતે છે & 2 બિલાડી, 1 કૂતરો \rightarrow બિલાડી \\
\end{longtable}
}

\textbf{કામગીરી પ્રક્રિયા:}

\begin{enumerate}
\tightlist
\item
  \textbf{અંતર ગણો} ટેસ્ટ પોઇન્ટ અને બધા ટ્રેનિંગ પોઇન્ટ્સ વચ્ચે
\item
  \textbf{અંતર સોર્ટ કરો} અને K નજીકના પડોશીઓ પસંદ કરો
\item
  \textbf{વોટ ગણો} પડોશીઓ વચ્ચે દરેક ક્લાસમાંથી
\item
  \textbf{ક્લાસ અસાઇન કરો} મેજોરિટી વોટ સાથે
\end{enumerate}

\textbf{ડાયાગ્રામ:}

\begin{center}
\textbf{Mermaid Diagram (Code)}
\begin{verbatim}
{Shaded}
{Highlighting}[]
graph LR
    A[નવો ડેટા પોઇન્ટ] {-{-}{} B[અંતર ગણો]}
    B {-{-}{} C[K નજીકના શોધો]}
    C {-{-}{} D[મેજોરિટી વોટ]}
    D {-{-}{} E[ક્લાસ આગાહી કરો]}
{Highlighting}
{Shaded}
\end{verbatim}
\end{center}

\textbf{ફાયદાઓ:}

\begin{itemize}
\tightlist
\item
  \textbf{લાગુ કરવામાં સરળ} અને સમજવામાં આસાન
\item
  \textbf{ટ્રેનિંગની જરૂર નથી} - આળસુ લર્નિંગ અલ્ગોરિધમ
\end{itemize}

\end{solutionbox}
\begin{mnemonicbox}
``K નજીકના પડોશીઓ ક્લાસિફિકેશન માટે વોટ કરે છે''

\end{mnemonicbox}
\subsection*{પ્રશ્ન 3(અ) અથવા [3
ગુણ]}\label{uxaaauxab0uxab6uxaa8-3uxa85-uxa85uxaa5uxab5-3-uxa97uxaa3}

\textbf{Multiple linear regression ની કોઈપણ ત્રણ ઉપયોગીતાઓ આપો.}

\begin{solutionbox}

\textbf{Multiple Linear Regression ની ઉપયોગીતાઓ:}

\textbf{ટેબલ: ઉપયોગીતાઓ}

{\def\LTcaptype{none} % do not increment counter
\begin{longtable}[]{@{}lll@{}}
\toprule\noalign{}
ઉપયોગીતા & વેરિયેબલ્સ & હેતુ \\
\midrule\noalign{}
\endhead
\bottomrule\noalign{}
\endlastfoot
\textbf{ઘરની કિંમત આગાહી} & કદ, સ્થાન, ઉંમર & પ્રોપર્ટીની કિંમત અંદાજ \\
\textbf{સેલ્સ ફોરકાસ્ટિંગ} & જાહેરાત, સીઝન, કિંમત & આવકની આગાહી કરવી \\
\textbf{મેડિકલ ડાયગ્નોસિસ} & લક્ષણો, ઉંમર, ઇતિહાસ & જોખમ આકારણી \\
\end{longtable}
}

\textbf{Multiple Linear Regression} એક કન્ટિન્યુઅસ આઉટપુટ વેરિયેબલની આગાહી
કરવા માટે બહુવિધ ઇનપુટ વેરિયેબલ્સનો ઉપયોગ કરે છે.

\end{solutionbox}
\begin{mnemonicbox}
``બહુવિધ ઇનપુટ્સ, એક આઉટપુટ''

\end{mnemonicbox}
\subsection*{પ્રશ્ન 3(બ) અથવા [4
ગુણ]}\label{uxaaauxab0uxab6uxaa8-3uxaac-uxa85uxaa5uxab5-4-uxa97uxaa3}

\textbf{Bagging, boosting અને stacking વિગતવાર વર્ણવો.}

\begin{solutionbox}

\textbf{ટેબલ: Ensemble મેથડ્સ}

{\def\LTcaptype{none} % do not increment counter
\begin{longtable}[]{@{}lll@{}}
\toprule\noalign{}
મેથડ & અભિગમ & ઉદાહરણ \\
\midrule\noalign{}
\endhead
\bottomrule\noalign{}
\endlastfoot
\textbf{Bagging} & પેરેલલ ટ્રેનિંગ, સરેરાશ પરિણામો & Random Forest \\
\textbf{Boosting} & સિક્વેન્શિયલ ટ્રેનિંગ, ભૂલોમાંથી શીખે & AdaBoost \\
\textbf{Stacking} & મેટા-લર્નર મોડલ્સ કન્બાઇન કરે & Neural network
combiner \\
\end{longtable}
}

\textbf{Bagging} વિવિધ ડેટા સબસેટ્સ પર બહુવિધ મોડલ્સને ટ્રેન કરે છે અને આગાહીઓની
સરેરાશ કાઢે છે.

\textbf{Boosting} મોડલ્સને ક્રમિક રીતે ટ્રેન કરે છે, દરેક અગાઉના મોડલની ભૂલોમાંથી
શીખે છે.

\textbf{Stacking} બેઝ મોડલ્સની આગાહીઓને કેવી રીતે કન્બાઇન કરવી તે શીખવા માટે
મેટા-મોડલનો ઉપયોગ કરે છે.

\end{solutionbox}
\begin{mnemonicbox}
``Bag પેરેલલ, Boost સિક્વેન્શિયલ, Stack મેટા''

\end{mnemonicbox}
\subsection*{પ્રશ્ન 3(ક) અથવા [7
ગુણ]}\label{uxaaauxab0uxab6uxaa8-3uxa95-uxa85uxaa5uxab5-7-uxa97uxaa3}

\textbf{Single linear regression તેની ઉપયોગીતાઓ સાથે વર્ણવો.}

\begin{solutionbox}

\textbf{Single Linear Regression} એક ઇનપુટ વેરિયેબલ (X) અને એક આઉટપુટ વેરિયેબલ
(Y) વચ્ચે શ્રેષ્ઠ સીધો રેખા સંબંધ શોધે છે.

\textbf{ફોર્મ્યુલા: Y = a + bX}

\begin{itemize}
\tightlist
\item
  \textbf{a}: Y-intercept
\item
  \textbf{b}: લાઇનનો Slope
\end{itemize}

\textbf{ટેબલ: ઉપયોગ ઉદાહરણ - ઘરની કિંમત વિ કદ}

{\def\LTcaptype{none} % do not increment counter
\begin{longtable}[]{@{}ll@{}}
\toprule\noalign{}
ઘરનું કદ (sq ft) & કિંમત (લાખ) \\
\midrule\noalign{}
\endhead
\bottomrule\noalign{}
\endlastfoot
1000 & 50 \\
1500 & 75 \\
2000 & 100 \\
\end{longtable}
}

\textbf{કામકાજની પ્રક્રિયા:}

\begin{enumerate}
\tightlist
\item
  \textbf{ડેટા એકત્રિત કરો} ઇનપુટ-આઉટપુટ જોડીઓ સાથે
\item
  \textbf{પોઇન્ટ્સ પ્લોટ કરો} સ્કેટર ગ્રાફ પર
\item
  \textbf{શ્રેષ્ઠ લાઇન શોધો} જે ભૂલ ન્યૂનતમ કરે
\item
  \textbf{આગાહીઓ કરો} લાઇન સમીકરણ વાપરીને
\end{enumerate}

\textbf{ડાયાગ્રામ:}

\begin{verbatim}
    કિંમત |
          |    *
       75 |      *
          |        *
       50 |  *
          |\_\_\_\_\_\_\_\_\_\_\_\_\_\_\_\_
             1000  1500  કદ
\end{verbatim}

\textbf{ઉપયોગીતાઓ:}

\begin{itemize}
\tightlist
\item
  \textbf{સેલ્સ વિ જાહેરાત}: વધુ જાહેરાત \rightarrow વધુ સેલ્સ
\item
  \textbf{તાપમાન વિ આઇસક્રીમ સેલ્સ}: ગરમ હવામાન \rightarrow વધુ સેલ્સ
\end{itemize}

\end{solutionbox}
\begin{mnemonicbox}
``એક X એક Y ની લાઇન સાથે આગાહી કરે છે''

\end{mnemonicbox}
\subsection*{પ્રશ્ન 4(અ) [3
ગુણ]}\label{uxaaauxab0uxab6uxaa8-4uxa85-3-uxa97uxaa3}

\textbf{વ્યાખ્યા આપો: (1)support (2)confidence.}

\begin{solutionbox}

\textbf{Support} માપે છે કે આઇટમસેટ ડેટાસેટમાં કેટલી વાર દેખાય છે.

\textbf{Confidence} માપે છે કે જ્યારે antecedent હાજર હોય ત્યારે consequent માં
આઇટમ્સ કેટલી વાર દેખાય છે.

\textbf{ટેબલ: વ્યાખ્યાઓ}

{\def\LTcaptype{none} % do not increment counter
\begin{longtable}[]{@{}
  >{\raggedright\arraybackslash}p{(\linewidth - 4\tabcolsep) * \real{0.3333}}
  >{\raggedright\arraybackslash}p{(\linewidth - 4\tabcolsep) * \real{0.3333}}
  >{\raggedright\arraybackslash}p{(\linewidth - 4\tabcolsep) * \real{0.3333}}@{}}
\toprule\noalign{}
\begin{minipage}[b]{\linewidth}\raggedright
માપદંડ
\end{minipage} & \begin{minipage}[b]{\linewidth}\raggedright
ફોર્મ્યુલા
\end{minipage} & \begin{minipage}[b]{\linewidth}\raggedright
ઉદાહરણ
\end{minipage} \\
\midrule\noalign{}
\endhead
\bottomrule\noalign{}
\endlastfoot
\textbf{Support} & Count(itemset)/કુલ transactions & બ્રેડ 60\%
transactions માં દેખાય છે \\
\textbf{Confidence} & Support(A\cupB)/Support(A) & બ્રેડ ખરીદનારા 80\% લોકો
બટર પણ ખરીદે છે \\
\end{longtable}
}

\textbf{Support = આવૃત્તિની આવર્તન} \textbf{Confidence = નિયમની વિશ્વસનીયતા}

\end{solutionbox}
\begin{mnemonicbox}
``Support = કેટલી વાર, Confidence = કેટલું વિશ્વસનીય''

\end{mnemonicbox}
\subsection*{પ્રશ્ન 4(બ) [4
ગુણ]}\label{uxaaauxab0uxab6uxaa8-4uxaac-4-uxa97uxaa3}

\textbf{Unsupervised learning ની ઉપયોગીતાઓ વર્ણવો.}

\begin{solutionbox}

\textbf{ટેબલ: Unsupervised Learning ઉપયોગીતાઓ}

{\def\LTcaptype{none} % do not increment counter
\begin{longtable}[]{@{}lll@{}}
\toprule\noalign{}
ઉપયોગીતા & હેતુ & ઉદાહરણ \\
\midrule\noalign{}
\endhead
\bottomrule\noalign{}
\endlastfoot
\textbf{કસ્ટમર સેગમેન્ટેશન} & સમાન કસ્ટમર્સને જૂથબદ્ધ કરવા & માર્કેટિંગ કેમ્પેઇન્સ \\
\textbf{ડેટા કમ્પ્રેશન} & ડેટાનું કદ ઘટાડવું & ઇમેજ કમ્પ્રેશન \\
\textbf{અનોમલી ડિટેક્શન} & અસામાન્ય પેટર્ન શોધવા & ફ્રોડ ડિટેક્શન \\
\textbf{રેકમેન્ડેશન સિસ્ટમ્સ} & સમાન આઇટમ્સ સુઝાવવા & મ્યુઝિક રેકમેન્ડેશન્સ \\
\end{longtable}
}

\textbf{કસ્ટમર સેગમેન્ટેશન} લક્ષિત માર્કેટિંગ માટે સમાન ખરીદી વર્તણૂક ધરાવતા કસ્ટમર્સને
જૂથબદ્ધ કરે છે.

\textbf{ડેટા કમ્પ્રેશન} પેટર્ન શોધીને અને રિડન્ડન્સી દૂર કરીને સ્ટોરેજ સ્પેસ ઘટાડે છે.

\textbf{અનોમલી ડિટેક્શન} અસામાન્ય પેટર્ન ઓળખે છે જે ફ્રોડ અથવા ભૂલો સૂચવી શકે છે.

\end{solutionbox}
\begin{mnemonicbox}
``સેગમેન્ટ કમ્પ્રેસ ડિટેક્ટ રેકમેન્ડ''

\end{mnemonicbox}
\subsection*{પ્રશ્ન 4(ક) [7
ગુણ]}\label{uxaaauxab0uxab6uxaa8-4uxa95-7-uxa97uxaa3}

\textbf{Apriori અલ્ગોરિધમ યોગ્ય ઉદાહરણ સાથે વર્ણવો.}

\begin{solutionbox}

\textbf{Apriori Algorithm} માર્કેટ બાસ્કેટ એનાલિસિસ માટે ફ્રીક્વન્ટ આઇટમસેટ્સ શોધે
છે અને એસોસિએશન રૂલ્સ જનરેટ કરે છે.

\textbf{ટેબલ: અલ્ગોરિધમ સ્ટેપ્સ}

{\def\LTcaptype{none} % do not increment counter
\begin{longtable}[]{@{}
  >{\raggedright\arraybackslash}p{(\linewidth - 4\tabcolsep) * \real{0.2143}}
  >{\raggedright\arraybackslash}p{(\linewidth - 4\tabcolsep) * \real{0.4643}}
  >{\raggedright\arraybackslash}p{(\linewidth - 4\tabcolsep) * \real{0.3214}}@{}}
\toprule\noalign{}
\begin{minipage}[b]{\linewidth}\raggedright
સ્ટેપ
\end{minipage} & \begin{minipage}[b]{\linewidth}\raggedright
વર્ણન
\end{minipage} & \begin{minipage}[b]{\linewidth}\raggedright
ઉદાહરણ
\end{minipage} \\
\midrule\noalign{}
\endhead
\bottomrule\noalign{}
\endlastfoot
\textbf{ફ્રીક્વન્ટ 1-itemsets શોધો} & વ્યક્તિગત આઇટમ્સ ગણો & \{બ્રેડ\}:4,
\{દૂધ\}:3 \\
\textbf{2-itemsets જનરેટ કરો} & ફ્રીક્વન્ટ આઇટમ્સ કન્બાઇન કરો &
\{બ્રેડ,દૂધ\}:2 \\
\textbf{મિનિમમ સપોર્ટ લાગુ કરો} & ઇન્ફ્રીક્વન્ટ સેટ્સ ફિલ્ટર કરો & support \geq 50\%
જો રાખો \\
\textbf{રૂલ્સ જનરેટ કરો} & if-then રૂલ્સ બનાવો & બ્રેડ \rightarrow દૂધ \\
\end{longtable}
}

\textbf{ઉદાહરણ ડેટાસેટ:}

\begin{itemize}
\tightlist
\item
  Transaction 1: \{બ્રેડ, દૂધ, ઈંડા\}
\item
  Transaction 2: \{બ્રેડ, દૂધ\}
\item
  Transaction 3: \{બ્રેડ, ઈંડા\}
\item
  Transaction 4: \{દૂધ, ઈંડા\}
\end{itemize}

\textbf{કામકાજની પ્રક્રિયા:}

\begin{enumerate}
\tightlist
\item
  \textbf{ડેટાબેઝ સ્કેન કરો} આઇટમ ફ્રીક્વન્સીઝ ગણવા માટે
\item
  \textbf{કેન્ડિડેટ આઇટમસેટ્સ જનરેટ કરો} વધતા કદની
\item
  \textbf{ઇન્ફ્રીક્વન્ટ આઇટમસેટ્સ પ્રૂન કરો} મિનિમમ સપોર્ટથી નીચે
\item
  \textbf{એસોસિએશન રૂલ્સ જનરેટ કરો} ફ્રીક્વન્ટ આઇટમસેટ્સમાંથી
\end{enumerate}

\textbf{ડાયાગ્રામ:}

\begin{verbatim}
flowchart LR
    A[Transaction Database] {-{-} B[ફ્રીક્વન્ટ 1{-}itemsets શોધો]}
    B {-{-} C[2{-}itemsets જનરેટ કરો]}
    C {-{-} D[મિન સપોર્ટ લાગુ કરો]}
    D {-{-} E[રૂલ્સ જનરેટ કરો]}
\end{verbatim}

\end{solutionbox}
\begin{mnemonicbox}
``A-priori જ્ઞાન ફ્રીક્વન્ટ પેટર્ન શોધવામાં મદદ કરે છે''

\end{mnemonicbox}
\subsection*{પ્રશ્ન 4(અ) અથવા [3
ગુણ]}\label{uxaaauxab0uxab6uxaa8-4uxa85-uxa85uxaa5uxab5-3-uxa97uxaa3}

\textbf{Clustering અને Classification ના તફાવતની યાદી આપો.}

\begin{solutionbox}

\textbf{ટેબલ: Clustering વિ Classification}

{\def\LTcaptype{none} % do not increment counter
\begin{longtable}[]{@{}ll@{}}
\toprule\noalign{}
Clustering & Classification \\
\midrule\noalign{}
\endhead
\bottomrule\noalign{}
\endlastfoot
\textbf{અનસુપરવાઇઝ્ડ} લર્નિંગ & \textbf{સુપરવાઇઝ્ડ} લર્નિંગ \\
લેબલ કરેલ ડેટા નથી & લેબલ કરેલ ટ્રેનિંગ ડેટા વાપરે છે \\
સમાન ડેટાને જૂથબદ્ધ કરે છે & પૂર્વનિર્ધારિત લેબલ્સ અસાઇન કરે છે \\
\end{longtable}
}

\textbf{Clustering} અનલેબલ ડેટામાં છુપાયેલા જૂથો શોધે છે.

\textbf{Classification} ટ્રેન કરેલા મોડલ્સ વાપરીને નવા ડેટાને જાણીતી કેટેગરીઓમાં
અસાઇન કરે છે.

\end{solutionbox}
\begin{mnemonicbox}
``Cluster = અજાણ્યા જૂથો, Classify = જાણીતા લેબલ્સ''

\end{mnemonicbox}
\subsection*{પ્રશ્ન 4(બ) અથવા [4
ગુણ]}\label{uxaaauxab0uxab6uxaa8-4uxaac-uxa85uxaa5uxab5-4-uxa97uxaa3}

\textbf{Clustering ની પ્રોસેસ વિગતવાર વર્ણવો.}

\begin{solutionbox}

\textbf{ટેબલ: Clustering પ્રોસેસ સ્ટેપ્સ}

{\def\LTcaptype{none} % do not increment counter
\begin{longtable}[]{@{}
  >{\raggedright\arraybackslash}p{(\linewidth - 4\tabcolsep) * \real{0.2143}}
  >{\raggedright\arraybackslash}p{(\linewidth - 4\tabcolsep) * \real{0.4643}}
  >{\raggedright\arraybackslash}p{(\linewidth - 4\tabcolsep) * \real{0.3214}}@{}}
\toprule\noalign{}
\begin{minipage}[b]{\linewidth}\raggedright
સ્ટેપ
\end{minipage} & \begin{minipage}[b]{\linewidth}\raggedright
વર્ણન
\end{minipage} & \begin{minipage}[b]{\linewidth}\raggedright
હેતુ
\end{minipage} \\
\midrule\noalign{}
\endhead
\bottomrule\noalign{}
\endlastfoot
\textbf{ડેટા પ્રિપેરેશન} & ડેટા સાફ અને નોર્મલાઇઝ કરો & ગુણવત્તાપૂર્ણ ઇનપુટ સુનિશ્ચિત
કરવું \\
\textbf{ડિસ્ટન્સ મેટ્રિક} & સમાનતાનું માપ પસંદ કરો & Euclidean, Manhattan \\
\textbf{અલ્ગોરિધમ સિલેક્શન} & ક્લસ્ટરિંગ મેથડ પસંદ કરો & K-means,
Hierarchical \\
\textbf{ક્લસ્ટર વેલિડેશન} & ક્લસ્ટર ગુણવત્તાનું મૂલ્યાંકન કરો & Silhouette score \\
\end{longtable}
}

\textbf{Clustering પ્રોસેસ} તેમની લાક્ષણિકતાઓના આધારે સમાન ડેટા પોઇન્ટ્સને એકસાથે
જૂથબદ્ધ કરે છે.

\textbf{મુખ્ય નિર્ણયોમાં ક્લસ્ટર્સની સંખ્યા અને યોગ્ય ડિસ્ટન્સ મેટ્રિક્સ પસંદ કરવાનો
સમાવેશ થાય છે.}

\textbf{વેલિડેશન સુનિશ્ચિત કરે છે કે ક્લસ્ટર્સ અર્થપૂર્ણ અને સારી રીતે અલગ છે.}

\end{solutionbox}
\begin{mnemonicbox}
``પ્રિપેર ડિસ્ટન્સ અલ્ગોરિધમ વેલિડેટ''

\end{mnemonicbox}
\subsection*{પ્રશ્ન 4(ક) અથવા [7
ગુણ]}\label{uxaaauxab0uxab6uxaa8-4uxa95-uxa85uxaa5uxab5-7-uxa97uxaa3}

\textbf{K-means clustering અલ્ગોરિધમ યોગ્ય ઉદાહરણ સાથે વર્ણવો.}

\begin{solutionbox}

\textbf{K-means} વિથિન-ક્લસ્ટર સમ ઓફ સ્ક્વેર્સ ન્યૂનતમ કરીને ડેટાને K ક્લસ્ટર્સમાં
વિભાજિત કરે છે.

\textbf{ટેબલ: અલ્ગોરિધમ સ્ટેપ્સ}

{\def\LTcaptype{none} % do not increment counter
\begin{longtable}[]{@{}lll@{}}
\toprule\noalign{}
સ્ટેપ & વર્ણન & ઉદાહરણ \\
\midrule\noalign{}
\endhead
\bottomrule\noalign{}
\endlastfoot
\textbf{સેન્ટ્રોઇડ્સ ઇનિશિયલાઇઝ કરો} & રેન્ડમ K સેન્ટર પોઇન્ટ્સ & C1(2,3),
C2(8,7) \\
\textbf{પોઇન્ટ્સ અસાઇન કરો} & દરેક પોઇન્ટ નજીકના સેન્ટ્રોઇડને & Point(1,2) \rightarrow
C1 \\
\textbf{સેન્ટ્રોઇડ્સ અપડેટ કરો} & અસાઇન થયેલા પોઇન્ટ્સનો મીન & નવું C1(1.5,
2.5) \\
\textbf{રિપીટ કરો} & સેન્ટ્રોઇડ્સ હલનચલન બંધ ન થાય ત્યાં સુધી & કન્વર્જન્સ \\
\end{longtable}
}

\textbf{ઉદાહરણ: કસ્ટમર આવક વિ ઉંમર}

\begin{itemize}
\tightlist
\item
  કસ્ટમર 1: (આવક=30k, ઉંમર=25)
\item
  કસ્ટમર 2: (આવક=35k, ઉંમર=30)
\item
  કસ્ટમર 3: (આવક=70k, ઉંમર=45)
\item
  કસ્ટમર 4: (આવક=75k, ઉંમર=50)
\end{itemize}

\textbf{કામકાજની પ્રક્રિયા:}

\begin{enumerate}
\tightlist
\item
  \textbf{K=2 પસંદ કરો} યુવા/વૃદ્ધ કસ્ટમર્સ માટે ક્લસ્ટર્સ
\item
  \textbf{સેન્ટ્રોઇડ્સ ઇનિશિયલાઇઝ કરો} રેન્ડમ રીતે
\item
  \textbf{અંતર ગણો} દરેક કસ્ટમરથી સેન્ટ્રોઇડ્સ સુધી
\item
  \textbf{કસ્ટમર્સ અસાઇન કરો} નજીકના સેન્ટ્રોઇડને
\item
  \textbf{સેન્ટ્રોઇડ પોઝિશન્સ અપડેટ કરો} અસાઇન થયેલા કસ્ટમર્સના કેન્દ્રમાં
\item
  \textbf{સ્થિર ન થાય ત્યાં સુધી રિપીટ કરો}
\end{enumerate}

\textbf{ડાયાગ્રામ:}

\begin{verbatim}
flowchart LR
    A[K પસંદ કરો] {-{-} B[સેન્ટ્રોઇડ્સ ઇનિશિયલાઇઝ કરો]}
    B {-{-} C[પોઇન્ટ્સને નજીકના સેન્ટ્રોઇડને અસાઇન કરો]}
    C {-{-} D[સેન્ટ્રોઇડ પોઝિશન્સ અપડેટ કરો]}
    D {-{-} E\{કન્વર્જ થયું?\}}
    E {-{-}|ના| C}
    E {-{-}|હા| F[અંતિમ ક્લસ્ટર્સ]}
\end{verbatim}

\textbf{ફાયદાઓ:}

\begin{itemize}
\tightlist
\item
  \textbf{સરળ અને ઝડપી} મોટા ડેટાસેટ્સ માટે
\item
  \textbf{ગોળાકાર ક્લસ્ટર્સ સાથે સારું કામ કરે છે}
\end{itemize}

\end{solutionbox}
\begin{mnemonicbox}
``K સેન્ટ્રોઇડ્સ તેમના અસાઇન થયેલા પોઇન્ટ્સનો મીન કરે છે''

\end{mnemonicbox}
\subsection*{પ્રશ્ન 5(અ) [3
ગુણ]}\label{uxaaauxab0uxab6uxaa8-5uxa85-3-uxa97uxaa3}

\textbf{Matplotlib ની ઉપયોગીતાઓની યાદી આપો.}

\begin{solutionbox}

\textbf{ટેબલ: Matplotlib ઉપયોગીતાઓ}

{\def\LTcaptype{none} % do not increment counter
\begin{longtable}[]{@{}lll@{}}
\toprule\noalign{}
ઉપયોગીતા & હેતુ & ઉદાહરણ \\
\midrule\noalign{}
\endhead
\bottomrule\noalign{}
\endlastfoot
\textbf{ડેટા વિઝ્યુલાઇઝેશન} & ચાર્ટ્સ અને ગ્રાફ્સ બનાવવા & બાર ચાર્ટ્સ,
હિસ્ટોગ્રામ્સ \\
\textbf{સાયન્ટિફિક પ્લોટિંગ} & સંશોધન પ્રેઝન્ટેશન્સ & ગાણિતિક ફંક્શન્સ \\
\textbf{ડેશબોર્ડ ક્રિએશન} & ઇન્ટરેક્ટિવ ડિસ્પ્લે & બિઝનેસ મેટ્રિક્સ \\
\end{longtable}
}

\textbf{Matplotlib} સ્ટેટિક, એનિમેટેડ અને ઇન્ટરેક્ટિવ વિઝ્યુલાઇઝેશન્સ બનાવવા માટે
Python ની પ્રાથમિક પ્લોટિંગ લાઇબ્રેરી છે.

\textbf{મુખ્ય ફીચર્સમાં બહુવિધ પ્લોટ ટાઇપ્સ માટેનું સપોર્ટ અને કસ્ટમાઇઝેબલ સ્ટાઇલિંગનો
સમાવેશ થાય છે.}

\end{solutionbox}
\begin{mnemonicbox}
``Mat-plot-lib = ગાણિત પ્લોટિંગ લાઇબ્રેરી''

\end{mnemonicbox}
\subsection*{પ્રશ્ન 5(બ) [4
ગુણ]}\label{uxaaauxab0uxab6uxaa8-5uxaac-4-uxa97uxaa3}

\textbf{હોરિઝોન્ટલ અને વર્ટિકલ લાઇન પ્લોટ કરવાનો કોડ matplotlib ની મદદથી
લખો.}

\begin{solutionbox}

\textbf{કોડ બ્લોક:}

\begin{verbatim}
import matplotlib.pyplot as plt

\# ફિગર બનાવો
plt.figure(figsize=(8, 6))

\# x=3 પર વર્ટિકલ લાઇન પ્લોટ કરો
plt.axvline(x=3, color={red}, linestyle={{-}{-}}, label={વર્ટિકલ લાઇન})

\# y=2 પર હોરિઝોન્ટલ લાઇન પ્લોટ કરો
plt.axhline(y=2, color={blue}, linestyle={{-}}, label={હોરિઝોન્ટલ લાઇન})

\# લેબલ્સ અને ટાઇટલ ઉમેરો
plt.xlabel({X{-}અક્ષ})
plt.ylabel({Y{-}અક્ષ})
plt.title({વર્ટિકલ અને હોરિઝોન્ટલ લાઇન્સ})
plt.legend()
plt.grid(True)
plt.show()
\end{verbatim}

\textbf{મુખ્ય ફંક્શન્સ:}

\begin{itemize}
\tightlist
\item
  \textbf{axvline()}: વર્ટિકલ લાઇન બનાવે છે
\item
  \textbf{axhline()}: હોરિઝોન્ટલ લાઇન બનાવે છે
\end{itemize}

\end{solutionbox}
\begin{mnemonicbox}
``axvline = વર્ટિકલ, axhline = હોરિઝોન્ટલ''

\end{mnemonicbox}
\subsection*{પ્રશ્ન 5(ક) [7
ગુણ]}\label{uxaaauxab0uxab6uxaa8-5uxa95-7-uxa97uxaa3}

\textbf{Scikit-Learn ની વિશેષતાઓ અને ઉપયોગીતાઓ સમજાવો.}

\begin{solutionbox}

\textbf{ટેબલ: Scikit-Learn વિશેષતાઓ}

{\def\LTcaptype{none} % do not increment counter
\begin{longtable}[]{@{}lll@{}}
\toprule\noalign{}
વિશેષતા & વર્ણન & ઉદાહરણ \\
\midrule\noalign{}
\endhead
\bottomrule\noalign{}
\endlastfoot
\textbf{સરળ API} & ઉપયોગમાં સરળ ઇન્ટરફેસ & fit(), predict() \\
\textbf{બહુવિધ અલ્ગોરિધમ્સ} & વિવિધ ML મેથડ્સ & SVM, Random Forest \\
\textbf{ડેટા પ્રીપ્રોસેસિંગ} & બિલ્ટ-ઇન ડેટા ટૂલ્સ & StandardScaler \\
\textbf{મોડલ ઇવેલ્યુએશન} & પરફોર્મન્સ મેટ્રિક્સ & accuracy\_score \\
\end{longtable}
}

\textbf{Scikit-Learn} ડેટા એનાલિસિસ માટે સરળ ટૂલ્સ પ્રદાન કરતી Python ની સૌથી
લોકપ્રિય મશીન લર્નિંગ લાઇબ્રેરી છે.

\textbf{મુખ્ય શક્તિઓ:}

\begin{itemize}
\tightlist
\item
  \textbf{સુસંગત ઇન્ટરફેસ} બધા અલ્ગોરિધમ્સમાં
\item
  \textbf{ઉત્કૃષ્ટ દસ્તાવેજીકરણ} ઉદાહરણો સાથે
\item
  \textbf{સક્રિય કમ્યુનિટી} સપોર્ટ અને ડેવલપમેન્ટ
\end{itemize}

\textbf{ઉપયોગીતાઓ:}

\begin{itemize}
\tightlist
\item
  \textbf{ક્લાસિફિકેશન}: ઈમેઇલ સ્પામ ડિટેક્શન
\item
  \textbf{રિગ્રેશન}: ઘરની કિંમત આગાહી
\item
  \textbf{ક્લસ્ટરિંગ}: કસ્ટમર સેગમેન્ટેશન
\item
  \textbf{ડાયમેન્શનાલિટી રિડક્શન}: ડેટા વિઝ્યુલાઇઝેશન
\end{itemize}

\textbf{ડાયાગ્રામ:}

\begin{center}
\textbf{Mermaid Diagram (Code)}
\begin{verbatim}
{Shaded}
{Highlighting}[]
graph TD
    A[Scikit{-Learn] {-}{-}{} B[ક્લાસિફિકેશન]}
    A {-{-}{} C[રિગ્રેશન]}
    A {-{-}{} D[ક્લસ્ટરિંગ]}
    A {-{-}{} E[પ્રીપ્રોસેસિંગ]}
    B {-{-}{} F[SVM, Decision Trees]}
    C {-{-}{} G[Linear, Polynomial]}
    D {-{-}{} H[K{-}means, DBSCAN]}
    E {-{-}{} I[Scaling, Encoding]}
{Highlighting}
{Shaded}
\end{verbatim}
\end{center}

\end{solutionbox}
\begin{mnemonicbox}
``Scikit = મશીન લર્નિંગ માટે સાયન્સ કિટ''

\end{mnemonicbox}
\subsection*{પ્રશ્ન 5(અ) અથવા [3
ગુણ]}\label{uxaaauxab0uxab6uxaa8-5uxa85-uxa85uxaa5uxab5-3-uxa97uxaa3}

\textbf{NumPy નો મશીન લર્નિંગના સંદર્ભમાં ઉપયોગ આપો.}

\begin{solutionbox}

\textbf{ટેબલ: ML માં NumPy નો હેતુ}

{\def\LTcaptype{none} % do not increment counter
\begin{longtable}[]{@{}
  >{\raggedright\arraybackslash}p{(\linewidth - 4\tabcolsep) * \real{0.2903}}
  >{\raggedright\arraybackslash}p{(\linewidth - 4\tabcolsep) * \real{0.4194}}
  >{\raggedright\arraybackslash}p{(\linewidth - 4\tabcolsep) * \real{0.2903}}@{}}
\toprule\noalign{}
\begin{minipage}[b]{\linewidth}\raggedright
હેતુ
\end{minipage} & \begin{minipage}[b]{\linewidth}\raggedright
વર્ણન
\end{minipage} & \begin{minipage}[b]{\linewidth}\raggedright
ફાયદો
\end{minipage} \\
\midrule\noalign{}
\endhead
\bottomrule\noalign{}
\endlastfoot
\textbf{ન્યુમેરિકલ કમ્પ્યુટિંગ} & ઝડપી array ઓપરેશન્સ & કાર્યક્ષમ ગણતરીઓ \\
\textbf{ફાઉન્ડેશન લાઇબ્રેરી} & અન્ય લાઇબ્રેરીઓ માટે આધાર & Pandas, Scikit-learn
તેનો ઉપયોગ કરે છે \\
\textbf{ગાણિતિક ફંક્શન્સ} & બિલ્ટ-ઇન મેથ ઓપરેશન્સ & સ્ટેટિસ્ટિક્સ, લિનિયર
આલ્જીબ્રા \\
\end{longtable}
}

\textbf{NumPy} Python મશીન લર્નિંગ એપ્લિકેશન્સમાં ન્યુમેરિકલ કમ્પ્યુટિંગ માટે પાયો
પ્રદાન કરે છે.

\textbf{મોટા ડેટાસેટ્સ હેન્ડલ કરવા અને ગાણિતિક ઓપરેશન્સ કાર્યક્ષમ રીતે કરવા માટે
જરૂરી છે.}

\end{solutionbox}
\begin{mnemonicbox}
``Num-Py = ન્યુમેરિકલ Python''

\end{mnemonicbox}
\subsection*{પ્રશ્ન 5(બ) અથવા [4
ગુણ]}\label{uxaaauxab0uxab6uxaa8-5uxaac-uxa85uxaa5uxab5-4-uxa97uxaa3}

\textbf{csv ફાઈલને pandas માં ઇમ્પોર્ટ કરવાના સ્ટેપ લખો.}

\begin{solutionbox}

\textbf{કોડ બ્લોક:}

\begin{verbatim}
import pandas as pd

\# સ્ટેપ 1: pandas લાઇબ્રેરી ઇમ્પોર્ટ કરો
\# સ્ટેપ 2: read\_csv() ફંક્શન વાપરો
data = pd.read\_csv({filename.csv})

\# સ્ટેપ 3: પ્રથમ કેટલીક પંક્તિઓ ડિસ્પ્લે કરો
print(data.head())

\# વૈકલ્પિક: પેરામીટર્સ સ્પેસિફાઇ કરો
data = pd.read\_csv({file.csv}, 
                   delimiter={,},
                   header=0,
                   index\_col=0)
\end{verbatim}

\textbf{સ્ટેપ્સ:}

\begin{enumerate}
\tightlist
\item
  \textbf{pandas ઇમ્પોર્ટ કરો} લાઇબ્રેરી
\item
  \textbf{read\_csv() વાપરો} ફાઇલનેમ સાથે
\item
  \textbf{ડેટા વેરિફાઇ કરો} head() મેથડ સાથે
\end{enumerate}

\end{solutionbox}
\begin{mnemonicbox}
``ઇમ્પોર્ટ રીડ વેરિફાઇ''

\end{mnemonicbox}
\subsection*{પ્રશ્ન 5(ક) અથવા [7
ગુણ]}\label{uxaaauxab0uxab6uxaa8-5uxa95-uxa85uxaa5uxab5-7-uxa97uxaa3}

\textbf{Pandas ની વિશેષતાઓ અને ઉપયોગીતાઓ સમજાવો.}

\begin{solutionbox}

\textbf{ટેબલ: Pandas વિશેષતાઓ}

{\def\LTcaptype{none} % do not increment counter
\begin{longtable}[]{@{}lll@{}}
\toprule\noalign{}
વિશેષતા & વર્ણન & ઉદાહરણ \\
\midrule\noalign{}
\endhead
\bottomrule\noalign{}
\endlastfoot
\textbf{ડેટા સ્ટ્રક્ચર્સ} & DataFrame અને Series & ટેબ્યુલર ડેટા હેન્ડલિંગ \\
\textbf{ડેટા I/O} & બહુવિધ ફોર્મેટ્સ રીડ/રાઇટ & CSV, Excel, JSON \\
\textbf{ડેટા ક્લીનિંગ} & મિસિંગ વેલ્યુઝ હેન્ડલ કરવા & dropna(), fillna() \\
\textbf{ડેટા એનાલિસિસ} & સ્ટેટિસ્ટિકલ ઓપરેશન્સ & groupby(), describe() \\
\end{longtable}
}

\textbf{Pandas} મશીન લર્નિંગ પ્રોજેક્ટ્સમાં Python માં પ્રાથમિક ડેટા મેનિપ્યુલેશન
લાઇબ્રેરી છે.

\textbf{મુખ્ય ક્ષમતાઓ:}

\begin{itemize}
\tightlist
\item
  \textbf{ડેટા લોડિંગ} વિવિધ ફાઇલ ફોર્મેટ્સમાંથી
\item
  \textbf{ડેટા ક્લીનિંગ} અને પ્રીપ્રોસેસિંગ ઓપરેશન્સ
\item
  \textbf{ડેટા ટ્રાન્સફોર્મેશન} અને રીશેપિંગ
\item
  \textbf{સ્ટેટિસ્ટિકલ એનાલિસિસ} અને એગ્રિગેશન
\end{itemize}

\textbf{ઉપયોગીતાઓ:}

\begin{itemize}
\tightlist
\item
  \textbf{ડેટા પ્રીપ્રોસેસિંગ}: ML પહેલાં ડેટાસેટ્સ સાફ કરવા
\item
  \textbf{એક્સ્પ્લોરેટરી એનાલિસિસ}: ડેટા પેટર્ન સમજવા
\item
  \textbf{ફીચર એન્જિનિયરિંગ}: નવા વેરિયેબલ્સ બનાવવા
\item
  \textbf{ડેટા ઇન્ટીગ્રેશન}: બહુવિધ ડેટા સ્રોતો મર્જ કરવા
\end{itemize}

\textbf{ડાયાગ્રામ:}

\begin{center}
\textbf{Mermaid Diagram (Code)}
\begin{verbatim}
{Shaded}
{Highlighting}[]
graph LR
    A[કાચો ડેટા] {-{-}{} B[Pandas DataFrame]}
    B {-{-}{} C[ડેટા ક્લીનિંગ]}
    C {-{-}{} D[ડેટા એનાલિસિસ]}
    D {-{-}{} E[ફીચર એન્જિનિયરિંગ]}
    E {-{-}{} F[ML તૈયાર ડેટા]}
{Highlighting}
{Shaded}
\end{verbatim}
\end{center}

\textbf{ફાયદાઓ:}

\begin{itemize}
\tightlist
\item
  \textbf{સાહજિક સિન્ટેક્સ} ડેટા ઓપરેશન્સ માટે
\item
  \textbf{હાઇ પરફોર્મન્સ} ઓપ્ટિમાઇઝ્ડ ઓપરેશન્સ સાથે
\item
  \textbf{ઇન્ટીગ્રેશન} અન્ય ML લાઇબ્રેરીઓ સાથે
\end{itemize}

\end{solutionbox}
\begin{mnemonicbox}
``Pandas = એનાલિસિસ માટે પેનલ ડેટા''

\end{mnemonicbox}

\end{document}
