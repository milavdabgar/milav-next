\documentclass[10pt,a4paper]{article}

% content/resources/templates/preamble.tex
\usepackage[margin=0.6in]{geometry}
\author{Milav Dabgar}
\usepackage{amsmath,amssymb,amsthm}
\usepackage{booktabs}
\usepackage{multirow}
\usepackage{xcolor}
\usepackage{tcolorbox}
\tcbuselibrary{breakable,skins}
\usepackage[colorlinks=true,linkcolor=blue]{hyperref}
\usepackage{titlesec}
\usepackage{enumitem}
\usepackage{tikz}
\usepackage{pgfplots}
\usepackage{circuitikz}
\usepackage[version=4]{mhchem}
\usepackage{longtable}
\usepackage{array}
\usepackage{float}
\usepackage{caption}
\usepackage{listings}

\lstset{
  basicstyle=\small\ttfamily,
  breaklines=true,
  breakatwhitespace=false,
  postbreak=\mbox{\textcolor{red}{$\hookrightarrow$}\space},
  float=false,
  numbers=left,
  numberstyle=\tiny\color{gray},
  numbersep=10pt,
  xleftmargin=2em,
  keywordstyle=\color{blue},
  commentstyle=\color{green!60!black},
  stringstyle=\color{purple},
  backgroundcolor=\color{gray!5},
  showstringspaces=false,
  tabsize=2,
  captionpos=b,
  keepspaces=true,
  columns=flexible
}

\pgfplotsset{compat=1.18}
\usetikzlibrary{shapes,arrows,positioning,calc,patterns,decorations.pathmorphing,decorations.markings,arrows.meta}

% Color scheme
\definecolor{headcolor}{RGB}{0,102,204}
\definecolor{keycolor}{RGB}{220,20,60}
\definecolor{solutioncolor}{RGB}{34,139,34}
\definecolor{mnemoniccolor}{RGB}{148,0,211}
\definecolor{codecolor}{RGB}{0,0,100}

% Spacing
\setlength{\parskip}{3pt}
\setlist[itemize]{nosep}
\setlist[enumerate]{nosep}

% Title formatting
\titleformat{\section}{\Large\bfseries\color{headcolor}}{\thesection}{1em}{}
\titleformat{\subsection}{\large\bfseries\color{headcolor}}{\thesubsection}{1em}{}

% Pandoc tightlist compatibility
\providecommand{\tightlist}{%
  \setlength{\itemsep}{0pt}\setlength{\parskip}{0pt}}

% Pandoc longtable compatibility
\newcounter{none}
\def\thenone{}


% content/resources/templates/gujarati-boxes.tex
\usepackage{fontspec}
\usepackage{polyglossia}

% Set Gujarati as main language (document is primarily in Gujarati)
% Note: gloss-gujarati.ldf doesn't exist in polyglossia, but it will use hyphenation patterns
\setdefaultlanguage{gujarati}
\setotherlanguage{english}

% Configure Gujarati font properly
% Use Language=Default to prevent polyglossia from trying to add language-specific features
% that don't exist for Gujarati, which causes "empty feature" warnings
\newfontfamily\gujaratifont[Script=Gujarati,AutoFakeBold=2.5,AutoFakeSlant=0.3]{Noto Sans Gujarati}
\setmainfont[Script=Gujarati,AutoFakeBold=2.5,AutoFakeSlant=0.3]{Noto Sans Gujarati}
% Use Noto Sans Gujarati for monospace to support Gujarati in text
\setmonofont[Scale=0.9]{Noto Sans Gujarati}

% Configure English to use the same font
\newfontfamily\englishfont[Script=Gujarati,AutoFakeBold=2.5,AutoFakeSlant=0.3]{Noto Sans Gujarati}

% Translations for polyglossia
\gappto\captionsgujarati{
  \renewcommand{\tablename}{કોષ્ટક}
  \renewcommand{\figurename}{આકૃતિ}
}

% Helper for TikZ nodes to ensure Gujarati font
\newcommand{\gu}[1]{{\gujaratifont #1}}

% Custom environments
\newtcolorbox{solutionbox}{
    breakable,
    enhanced,
    colback=solutioncolor!5!white,
    colframe=solutioncolor!75!black,
    fonttitle=\bfseries,
    title=જવાબ
}

\newtcolorbox{solutionboxnobreak}{
 colback=solutioncolor!5!white,
 colframe=solutioncolor!75!black,
 fonttitle=\bfseries,
 title=જવાબ
}

\newtcolorbox{keyformula}{
 breakable,
 enhanced,
 colback=keycolor!5!white,
 colframe=keycolor!75!black,
 fonttitle=\bfseries,
 title=રાસાયણિક સમીકરણ/સૂત્ર
}

\newtcolorbox{mnemonicbox}{
 breakable,
 enhanced,
 colback=mnemoniccolor!5!white,
 colframe=mnemoniccolor!75!black,
 fonttitle=\bfseries,
 title=મેમરી ટ્રીક
}


\begin{document}

\begin{center}
{\Huge\bfseries\color{headcolor} Subject Name (Gujarati)}\\[5pt]
{\LARGE 4341603 -- Summer 2023}\\[3pt]
{\large Semester 1 Study Material}\\[3pt]
{\normalsize\textit{Detailed Solutions and Explanations}}
\end{center}

\vspace{10pt}

\subsection*{પ્રશ્ન 1(અ) [3
માર્ક્સ]}\label{uxaaauxab0uxab6uxaa8-1uxa85-3-uxaaeuxab0uxa95uxab8}

\textbf{હ્યુમન લર્નિંગને વ્યાખ્યાયિત કરો. હ્યુમન લર્નિંગના પ્રકારોની યાદી બનાવો.}

\begin{solutionbox}

હ્યુમન લર્નિંગ એ પ્રક્રિયા છે જેના દ્વારા માનવીઓ અનુભવ, અભ્યાસ અથવા સૂચનાઓ દ્વારા નવા
જ્ઞાન, કૌશલ્યો, વર્તન મેળવે છે અથવા હાલનાઓમાં ફેરફાર કરે છે.

\textbf{હ્યુમન લર્નિંગના પ્રકારો:}

{\def\LTcaptype{none} % do not increment counter
\begin{longtable}[]{@{}ll@{}}
\toprule\noalign{}
પ્રકાર & વર્ણન \\
\midrule\noalign{}
\endhead
\bottomrule\noalign{}
\endlastfoot
\textbf{સુપરવાઇઝ્ડ લર્નિંગ} & શિક્ષક/માર્ગદર્શકની મદદથી શીખવું \\
\textbf{અનસુપરવાઇઝ્ડ લર્નિંગ} & બાહ્ય માર્ગદર્શન વિના સ્વ-નિર્દેશિત શીખવું \\
\textbf{રિઇનફોર્સમેન્ટ લર્નિંગ} & ફીડબેક સાથે પ્રયાસ અને ભૂલ દ્વારા શીખવું \\
\end{longtable}
}

\end{solutionbox}
\begin{mnemonicbox}
``SUR - Supervised, Unsupervised, Reinforcement''

\end{mnemonicbox}
\subsection*{પ્રશ્ન 1(બ) [4
માર્ક્સ]}\label{uxaaauxab0uxab6uxaa8-1uxaac-4-uxaaeuxab0uxa95uxab8}

\textbf{ક્વાલિટેટિવ ડેટા અને ક્વોન્ટિટેટિવ ડેટા વચ્ચે તફાવત આપો.}

\begin{solutionbox}

\textbf{ટેબલ: ક્વાલિટેટિવ vs ક્વોન્ટિટેટિવ ડેટા}

{\def\LTcaptype{none} % do not increment counter
\begin{longtable}[]{@{}lll@{}}
\toprule\noalign{}
લક્ષણ & ક્વાલિટેટિવ ડેટા & ક્વોન્ટિટેટિવ ડેટા \\
\midrule\noalign{}
\endhead
\bottomrule\noalign{}
\endlastfoot
\textbf{પ્રકૃતિ} & વર્ણનાત્મક, કેટેગોરિકલ & સંખ્યાત્મક, માપી શકાય તેવું \\
\textbf{વિશ્લેષણ} & વ્યક્તિગત અર્થઘટન & આંકડાકીય વિશ્લેષણ \\
\textbf{ઉદાહરણો} & રંગો, નામો, લિંગ & ઊંચાઈ, વજન, ઉંમર \\
\textbf{પ્રતિનિધિત્વ} & શબ્દો, કેટેગરીઓ & સંખ્યાઓ, ગ્રાફ્સ \\
\end{longtable}
}

\end{solutionbox}
\begin{mnemonicbox}
``QUAN-Numbers, QUAL-Words''

\end{mnemonicbox}
\subsection*{પ્રશ્ન 1(ક) [7
માર્ક્સ]}\label{uxaaauxab0uxab6uxaa8-1uxa95-7-uxaaeuxab0uxa95uxab8}

\textbf{મશીન લર્નિંગના વિવિધ પ્રકારોની સરખામણી કરો.}

\begin{solutionbox}

\textbf{ટેબલ: મશીન લર્નિંગના પ્રકારોની સરખામણી}

{\def\LTcaptype{none} % do not increment counter
\begin{longtable}[]{@{}
  >{\raggedright\arraybackslash}p{(\linewidth - 6\tabcolsep) * \real{0.1622}}
  >{\raggedright\arraybackslash}p{(\linewidth - 6\tabcolsep) * \real{0.4054}}
  >{\raggedright\arraybackslash}p{(\linewidth - 6\tabcolsep) * \real{0.1622}}
  >{\raggedright\arraybackslash}p{(\linewidth - 6\tabcolsep) * \real{0.2703}}@{}}
\toprule\noalign{}
\begin{minipage}[b]{\linewidth}\raggedright
પ્રકાર
\end{minipage} & \begin{minipage}[b]{\linewidth}\raggedright
ટ્રેનિંગ ડેટા
\end{minipage} & \begin{minipage}[b]{\linewidth}\raggedright
ધ્યેય
\end{minipage} & \begin{minipage}[b]{\linewidth}\raggedright
ઉદાહરણો
\end{minipage} \\
\midrule\noalign{}
\endhead
\bottomrule\noalign{}
\endlastfoot
\textbf{સુપરવાઇઝ્ડ} & લેબલ્ડ ડેટા & પરિણામોની આગાહી & ક્લાસિફિકેશન, રિગ્રેશન \\
\textbf{અનસુપરવાઇઝ્ડ} & અનલેબલ્ડ ડેટા & પેટર્ન શોધવા & ક્લસ્ટરિંગ, એસોસિએશન \\
\textbf{રિઇનફોર્સમેન્ટ} & રિવોર્ડ/પેનલ્ટી & રિવોર્ડ મેક્સિમાઇઝ કરવા & ગેમિંગ,
રોબોટિક્સ \\
\end{longtable}
}

\textbf{મુખ્ય તફાવતો:}

\begin{itemize}
\tightlist
\item
  \textbf{સુપરવાઇઝ્ડ}: ટ્રેનિંગ માટે ઇનપુટ-આઉટપુટ જોડીનો ઉપયોગ કરે છે
\item
  \textbf{અનસુપરવાઇઝ્ડ}: ડેટામાં છુપાયેલા પેટર્ન શોધે છે
\item
  \textbf{રિઇનફોર્સમેન્ટ}: પર્યાવરણ સાથે ક્રિયાપ્રતિક્રિયા દ્વારા શીખે છે
\end{itemize}

\end{solutionbox}
\begin{mnemonicbox}
``SUR-LAP: Supervised-Labeled, Unsupervised-Reveal,
Reinforcement-Action''

\end{mnemonicbox}
\subsection*{પ્રશ્ન 1(ક OR) [7
માર્ક્સ]}\label{uxaaauxab0uxab6uxaa8-1uxa95-or-7-uxaaeuxab0uxa95uxab8}

\textbf{મશીન લર્નિંગ વ્યાખ્યાયિત કરો. મશીન લર્નિંગની કોઈપણ ચાર એપ્લિકેશનને ટૂંકમાં
સમજાવો.}

\begin{solutionbox}

મશીન લર્નિંગ આર્ટિફિશિયલ ઇન્ટેલિજન્સનો ઉપવિભાગ છે જે કમ્પ્યુટરોને સ્પષ્ટ પ્રોગ્રામિંગ
વિના ડેટામાંથી શીખવા અને નિર્ણયો લેવા સક્ષમ બનાવે છે.

\textbf{ચાર એપ્લિકેશનો:}

{\def\LTcaptype{none} % do not increment counter
\begin{longtable}[]{@{}ll@{}}
\toprule\noalign{}
એપ્લિકેશન & વર્ણન \\
\midrule\noalign{}
\endhead
\bottomrule\noalign{}
\endlastfoot
\textbf{ઈમેઇલ સ્પામ ડિટેક્શન} & ઈમેઇલને સ્પામ અથવા વૈધ તરીકે વર્ગીકૃત કરે છે \\
\textbf{ઇમેજ રેકગ્નિશન} & ફોટોમાં ઓબ્જેક્ટ્સ ઓળખે છે \\
\textbf{રેકમેન્ડેશન સિસ્ટમ} & યુઝર્સને પ્રોડક્ટ્સ/કન્ટેન્ટ સૂચવે છે \\
\textbf{મેડિકલ ડાયગ્નોસિસ} & રોગોની શોધમાં ડૉક્ટરોની મદદ કરે છે \\
\end{longtable}
}

\end{solutionbox}
\begin{mnemonicbox}
``SIRM - Spam, Image, Recommendation, Medical''

\end{mnemonicbox}
\subsection*{પ્રશ્ન 2(અ) [3
માર્ક્સ]}\label{uxaaauxab0uxab6uxaa8-2uxa85-3-uxaaeuxab0uxa95uxab8}

\textbf{નીચેના ઉદાહરણોનો યોગ્ય ડેટા પ્રકાર જણાવો.}

\begin{solutionbox}

\textbf{ડેટા પ્રકાર વર્ગીકરણ:}

{\def\LTcaptype{none} % do not increment counter
\begin{longtable}[]{@{}ll@{}}
\toprule\noalign{}
ઉદાહરણ & ડેટા પ્રકાર \\
\midrule\noalign{}
\endhead
\bottomrule\noalign{}
\endlastfoot
\textbf{વિદ્યાર્થીઓની રાષ્ટ્રીયતા} & કેટેગોરિકલ (નોમિનલ) \\
\textbf{વિદ્યાર્થીઓની શિક્ષણ સ્થિતિ} & કેટેગોરિકલ (ઓર્ડિનલ) \\
\textbf{વિદ્યાર્થીઓની ઊંચાઈ} & ન્યુમેરિકલ (કન્ટિન્યુઅસ) \\
\end{longtable}
}

\end{solutionbox}
\begin{mnemonicbox}
``NCN - Nominal, Categorical, Numerical''

\end{mnemonicbox}
\subsection*{પ્રશ્ન 2(બ) [4
માર્ક્સ]}\label{uxaaauxab0uxab6uxaa8-2uxaac-4-uxaaeuxab0uxa95uxab8}

\textbf{ડેટા પ્રી-પ્રોસેસિંગ ટૂંકમાં સમજાવો.}

\begin{solutionbox}

ડેટા પ્રી-પ્રોસેસિંગ એ મશીન લર્નિંગ અલ્ગોરિધમ માટે કાચા ડેટાને તૈયાર કરવાની તકનીક છે.

\textbf{મુખ્ય સ્ટેપ્સ:}

{\def\LTcaptype{none} % do not increment counter
\begin{longtable}[]{@{}ll@{}}
\toprule\noalign{}
સ્ટેપ & હેતુ \\
\midrule\noalign{}
\endhead
\bottomrule\noalign{}
\endlastfoot
\textbf{ડેટા ક્લીનિંગ} & ભૂલો અને અસંગતતાઓ દૂર કરવી \\
\textbf{ડેટા ઇન્ટીગ્રેશન} & બહુવિધ સ્ત્રોતોમાંથી ડેટાને જોડવો \\
\textbf{ડેટા ટ્રાન્સફોર્મેશન} & ડેટાને યોગ્ય ફોર્મેટમાં બદલવો \\
\textbf{ડેટા રિડક્શન} & માહિતી જાળવીને ડેટાનું કદ ઘટાડવું \\
\end{longtable}
}

\end{solutionbox}
\begin{mnemonicbox}
``CITR - Clean, Integrate, Transform, Reduce''

\end{mnemonicbox}
\subsection*{પ્રશ્ન 2(ક) [7
માર્ક્સ]}\label{uxaaauxab0uxab6uxaa8-2uxa95-7-uxaaeuxab0uxa95uxab8}

\textbf{K-ફોલ્ડ ક્રોસ વેલિડેશન વિગતવાર સમજાવો.}

\begin{solutionbox}

K-ફોલ્ડ ક્રોસ વેલિડેશન એ ડેટાને K સમાન ભાગોમાં વિભાજિત કરીને મોડેલ પરફોર્મન્સ
મૂલ્યાંકનની તકનીક છે.

\textbf{પ્રક્રિયા:}

\begin{center}
\textbf{Mermaid Diagram (Code)}
\begin{verbatim}
{Shaded}
{Highlighting}[]
graph LR
    A[મૂળ ડેટાસેટ] {-{-}{} B[K ફોલ્ડમાં વિભાજિત કરો]}
    B {-{-}{} C[K{-}1 ફોલ્ડ ટ્રેનિંગ માટે વાપરો]}
    C {-{-}{} D[1 ફોલ્ડ ટેસ્ટિંગ માટે વાપરો]}
    D {-{-}{} E[K વખત પુનરાવર્તન કરો]}
    E {-{-}{} F[પરિણામોની સરેરાશ કાઢો]}
{Highlighting}
{Shaded}
\end{verbatim}
\end{center}

\textbf{સ્ટેપ્સ:}

\begin{itemize}
\tightlist
\item
  \textbf{વિભાજન}: ડેટાસેટને K સમાન ભાગોમાં વહેંચો
\item
  \textbf{ટ્રેનિંગ}: K-1 ફોલ્ડનો ઉપયોગ ટ્રેનિંગ માટે કરો
\item
  \textbf{ટેસ્ટ}: બાકીના ફોલ્ડનો ઉપયોગ વેલિડેશન માટે કરો
\item
  \textbf{પુનરાવર્તન}: K વખત પ્રક્રિયા કરો
\item
  \textbf{સરેરાશ}: સરેરાશ પરફોર્મન્સ કાઢો
\end{itemize}

\textbf{ફાયદા:}

\begin{itemize}
\tightlist
\item
  ઓવરફિટિંગ ઘટાડે છે
\item
  મર્યાદિત ડેટાનો બહેતર ઉપયોગ
\item
  વધુ વિશ્વસનીય પરફોર્મન્સ અંદાજ
\end{itemize}

\end{solutionbox}
\begin{mnemonicbox}
``DTRA - Divide, Train, Repeat, Average''

\end{mnemonicbox}
\subsection*{પ્રશ્ન 2(અ OR) [3
માર્ક્સ]}\label{uxaaauxab0uxab6uxaa8-2uxa85-or-3-uxaaeuxab0uxa95uxab8}

\textbf{નીચેના શબ્દો વ્યાખ્યાયિત કરો: i) Mean, ii) Outliers, iii)
Interquartile range}

\begin{solutionbox}

\textbf{આંકડાકીય શબ્દો:}

{\def\LTcaptype{none} % do not increment counter
\begin{longtable}[]{@{}ll@{}}
\toprule\noalign{}
શબ્દ & વ્યાખ્યા \\
\midrule\noalign{}
\endhead
\bottomrule\noalign{}
\endlastfoot
\textbf{Mean} & ડેટાસેટમાં બધી વેલ્યુઝની સરેરાશ \\
\textbf{Outliers} & અન્ય ડેટા પોઇન્ટ્સથી નોંધપાત્ર રીતે અલગ ડેટા પોઇન્ટ્સ \\
\textbf{Interquartile Range} & 75મા અને 25મા પર્સેન્ટાઇલ વચ્ચેનો તફાવત \\
\end{longtable}
}

\end{solutionbox}
\begin{mnemonicbox}
``MOI - Mean, Outliers, Interquartile''

\end{mnemonicbox}
\subsection*{પ્રશ્ન 2(બ OR) [4
માર્ક્સ]}\label{uxaaauxab0uxab6uxaa8-2uxaac-or-4-uxaaeuxab0uxa95uxab8}

\textbf{કન્ફ્યુશન મેટ્રિક્સની રચના સમજાવો.}

\begin{solutionbox}

\textbf{કન્ફ્યુશન મેટ્રિક્સ સ્ટ્રક્ચર:}

{\def\LTcaptype{none} % do not increment counter
\begin{longtable}[]{@{}lll@{}}
\toprule\noalign{}
& આગાહી & \\
\midrule\noalign{}
\endhead
\bottomrule\noalign{}
\endlastfoot
\textbf{વાસ્તવિક} & \textbf{પોઝિટિવ} & \textbf{નેગેટિવ} \\
\textbf{પોઝિટિવ} & True Positive (TP) & False Negative (FN) \\
\textbf{નેગેટિવ} & False Positive (FP) & True Negative (TN) \\
\end{longtable}
}

\textbf{ઘટકો:}

\begin{itemize}
\tightlist
\item
  \textbf{TP}: સાચી રીતે આગાહી કરેલા પોઝિટિવ કેસો
\item
  \textbf{TN}: સાચી રીતે આગાહી કરેલા નેગેટિવ કેસો
\item
  \textbf{FP}: ખોટી રીતે પોઝિટિવ તરીકે આગાહી કરેલા
\item
  \textbf{FN}: ખોટી રીતે નેગેટિવ તરીકે આગાહી કરેલા
\end{itemize}

\end{solutionbox}
\begin{mnemonicbox}
``TTFF - True True, False False''

\end{mnemonicbox}
\subsection*{પ્રશ્ન 2(ક OR) [7
માર્ક્સ]}\label{uxaaauxab0uxab6uxaa8-2uxa95-or-7-uxaaeuxab0uxa95uxab8}

\textbf{ફીચર સબસેટની પસંદગી પર ટૂંકી નોંધ લખો.}

\begin{solutionbox}

ફીચર સબસેટ સિલેક્શન એ મૂળ ફીચર સેટમાંથી સંબંધિત ફીચર્સ પસંદ કરવાની પ્રક્રિયા છે.

\textbf{મેથડ્સ:}

{\def\LTcaptype{none} % do not increment counter
\begin{longtable}[]{@{}ll@{}}
\toprule\noalign{}
મેથડ & વર્ણન \\
\midrule\noalign{}
\endhead
\bottomrule\noalign{}
\endlastfoot
\textbf{ફિલ્ટર મેથડ્સ} & ફીચર્સ રેન્ક કરવા આંકડાકીય માપદંડોનો ઉપયોગ \\
\textbf{રેપર મેથડ્સ} & ફીચર સબસેટ્સ મૂલ્યાંકન માટે ML અલ્ગોરિધમનો ઉપયોગ \\
\textbf{એમ્બેડેડ મેથડ્સ} & મોડેલ ટ્રેનિંગ દરમિયાન ફીચર સિલેક્શન \\
\end{longtable}
}

\textbf{ફાયદા:}

\begin{itemize}
\tightlist
\item
  \textbf{ઘટાડેલી જટિલતા}: ઓછા ફીચર્સ, સરળ મોડેલ્સ
\item
  \textbf{સુધારેલ પરફોર્મન્સ}: નોઇઝ અને અપ્રસ્તુત ફીચર્સ દૂર કરે છે
\item
  \textbf{ઝડપી ટ્રેનિંગ}: ઓછો કમ્પ્યુટેશનલ ઓવરહેડ
\end{itemize}

\textbf{લોકપ્રિય તકનીકો:}

\begin{itemize}
\tightlist
\item
  Chi-square ટેસ્ટ
\item
  Recursive Feature Elimination
\item
  LASSO રેગ્યુલરાઇઝેશન
\end{itemize}

\end{solutionbox}
\begin{mnemonicbox}
``FWE - Filter, Wrapper, Embedded''

\end{mnemonicbox}
\subsection*{પ્રશ્ન 3(અ) [3
માર્ક્સ]}\label{uxaaauxab0uxab6uxaa8-3uxa85-3-uxaaeuxab0uxa95uxab8}

\textbf{પ્રેડિક્ટિવ મોડેલ અને ડીસ્ક્રિપ્ટિવ મોડેલ વચ્ચેનો તફાવત આપો.}

\begin{solutionbox}

\textbf{મોડેલ પ્રકાર સરખામણી:}

{\def\LTcaptype{none} % do not increment counter
\begin{longtable}[]{@{}lll@{}}
\toprule\noalign{}
લક્ષણ & પ્રેડિક્ટિવ મોડેલ & ડીસ્ક્રિપ્ટિવ મોડેલ \\
\midrule\noalign{}
\endhead
\bottomrule\noalign{}
\endlastfoot
\textbf{હેતુ} & ભાવિ પરિણામોની આગાહી & વર્તમાન પેટર્ન સમજવા \\
\textbf{આઉટપુટ} & આગાહીઓ/વર્ગીકરણ & અંતર્દૃષ્ટિ/સારાંશ \\
\textbf{ઉદાહરણો} & રિગ્રેશન, ક્લાસિફિકેશન & ક્લસ્ટરિંગ, એસોસિએશન રૂલ્સ \\
\end{longtable}
}

\end{solutionbox}
\begin{mnemonicbox}
``PF-DC: Predictive-Future, Descriptive-Current''

\end{mnemonicbox}
\subsection*{પ્રશ્ન 3(બ) [4
માર્ક્સ]}\label{uxaaauxab0uxab6uxaa8-3uxaac-4-uxaaeuxab0uxa95uxab8}

\textbf{ક્લાસિફિકેશન અને રિગ્રેશન વચ્ચેના તફાવતની ચર્ચા કરો.}

\begin{solutionbox}

\textbf{ક્લાસિફિકેશન vs રિગ્રેશન:}

{\def\LTcaptype{none} % do not increment counter
\begin{longtable}[]{@{}lll@{}}
\toprule\noalign{}
પાસું & ક્લાસિફિકેશન & રિગ્રેશન \\
\midrule\noalign{}
\endhead
\bottomrule\noalign{}
\endlastfoot
\textbf{આઉટપુટ} & ડિસ્ક્રીટ કેટેગરીઓ & કન્ટિન્યુઅસ વેલ્યુઝ \\
\textbf{ધ્યેય} & ક્લાસ લેબલ્સની આગાહી & ન્યુમેરિકલ વેલ્યુઝની આગાહી \\
\textbf{ઉદાહરણો} & સ્પામ ડિટેક્શન, ઇમેજ રેકગ્નિશન & કિંમત આગાહી, તાપમાન \\
\textbf{મૂલ્યાંકન} & Accuracy, precision, recall & MSE, RMSE, R-squared \\
\end{longtable}
}

\end{solutionbox}
\begin{mnemonicbox}
``CCNM - Classification-Categories,
Regression-Numbers''

\end{mnemonicbox}
\subsection*{પ્રશ્ન 3(ક) [7
માર્ક્સ]}\label{uxaaauxab0uxab6uxaa8-3uxa95-7-uxaaeuxab0uxa95uxab8}

\textbf{ક્લાસિફિકેશનને વ્યાખ્યાયિત કરો. ક્લાસિફિકેશન લર્નિંગના સ્ટેપને વિગતોમાં
સમજાવો.}

\begin{solutionbox}

ક્લાસિફિકેશન એ સુપરવાઇઝ્ડ લર્નિંગ તકનીક છે જે ઇનપુટ ડેટા માટે ડિસ્ક્રીટ ક્લાસ લેબલ્સની
આગાહી કરે છે.

\textbf{ક્લાસિફિકેશન લર્નિંગ સ્ટેપ્સ:}

\begin{center}
\textbf{Mermaid Diagram (Code)}
\begin{verbatim}
{Shaded}
{Highlighting}[]
graph LR
    A[ડેટા કલેક્શન] {-{-}{} B[ડેટા પ્રીપ્રોસેસિંગ]}
    B {-{-}{} C[ફીચર સિલેક્શન]}
    C {-{-}{} D[ટ્રેન{-}ટેસ્ટ સ્પ્લિટ]}
    D {-{-}{} E[મોડેલ ટ્રેનિંગ]}
    E {-{-}{} F[મોડેલ મૂલ્યાંકન]}
    F {-{-}{} G[મોડેલ ડિપ્લોયમેન્ટ]}
{Highlighting}
{Shaded}
\end{verbatim}
\end{center}

\textbf{વિગતવાર સ્ટેપ્સ:}

\begin{itemize}
\tightlist
\item
  \textbf{ડેટા કલેક્શન}: લેબલ્ડ ટ્રેનિંગ ડેટા એકત્ર કરવો
\item
  \textbf{પ્રીપ્રોસેસિંગ}: ડેટાને સાફ કરવો અને તૈયાર કરવો
\item
  \textbf{ફીચર સિલેક્શન}: સંબંધિત લક્ષણો પસંદ કરવા
\item
  \textbf{ડેટા સ્પ્લિટ}: ટ્રેનિંગ અને ટેસ્ટિંગ સેટમાં વિભાજન
\item
  \textbf{ટ્રેનિંગ}: ટ્રેનિંગ ડેટાનો ઉપયોગ કરીને મોડેલ બનાવવું
\item
  \textbf{મૂલ્યાંકન}: મોડેલ પરફોર્મન્સ ચકાસવી
\item
  \textbf{ડિપ્લોયમેન્ટ}: આગાહીઓ માટે મોડેલનો ઉપયોગ
\end{itemize}

\end{solutionbox}
\begin{mnemonicbox}
``DCFSTED - Data, Clean, Features, Split, Train,
Evaluate, Deploy''

\end{mnemonicbox}
\subsection*{પ્રશ્ન 3(અ OR) [3
માર્ક્સ]}\label{uxaaauxab0uxab6uxaa8-3uxa85-or-3-uxaaeuxab0uxa95uxab8}

\textbf{બેગિંગ અને બૂસ્ટિંગ વચ્ચેનો તફાવત આપો.}

\begin{solutionbox}

\textbf{બેગિંગ vs બૂસ્ટિંગ:}

{\def\LTcaptype{none} % do not increment counter
\begin{longtable}[]{@{}lll@{}}
\toprule\noalign{}
લક્ષણ & બેગિંગ & બૂસ્ટિંગ \\
\midrule\noalign{}
\endhead
\bottomrule\noalign{}
\endlastfoot
\textbf{સેમ્પલિંગ} & બૂટસ્ટ્રેપ સેમ્પલિંગ & ક્રમાનુગત વેઇટેડ સેમ્પલિંગ \\
\textbf{ટ્રેનિંગ} & પેરેલલ ટ્રેનિંગ & ક્રમાનુગત ટ્રેનિંગ \\
\textbf{ફોકસ} & વેરિયન્સ ઘટાડવું & બાયસ ઘટાડવું \\
\end{longtable}
}

\end{solutionbox}
\begin{mnemonicbox}
``BPV-BSB: Bagging-Parallel-Variance,
Boosting-Sequential-Bias''

\end{mnemonicbox}
\subsection*{પ્રશ્ન 3(બ OR) [4
માર્ક્સ]}\label{uxaaauxab0uxab6uxaa8-3uxaac-or-4-uxaaeuxab0uxa95uxab8}

\textbf{લોજિસ્ટિક રિગ્રેશનના વિવિધ પ્રકારો સંક્ષિપ્તમાં સમજાવો.}

\begin{solutionbox}

\textbf{લોજિસ્ટિક રિગ્રેશનના પ્રકારો:}

{\def\LTcaptype{none} % do not increment counter
\begin{longtable}[]{@{}lll@{}}
\toprule\noalign{}
પ્રકાર & ક્લાસો & ઉપયોગ \\
\midrule\noalign{}
\endhead
\bottomrule\noalign{}
\endlastfoot
\textbf{બાઇનરી} & 2 ક્લાસો & હા/ના, પાસ/ફેઇલ \\
\textbf{મલ્ટિનોમિયલ} & 3+ ક્લાસો (અવ્યવસ્થિત) & રંગ વર્ગીકરણ \\
\textbf{ઓર્ડિનલ} & 3+ ક્લાસો (ક્રમાંકિત) & રેટિંગ સ્કેલ \\
\end{longtable}
}

\end{solutionbox}
\begin{mnemonicbox}
``BMO - Binary, Multinomial, Ordinal''

\end{mnemonicbox}
\subsection*{પ્રશ્ન 3(ક OR) [7
માર્ક્સ]}\label{uxaaauxab0uxab6uxaa8-3uxa95-or-7-uxaaeuxab0uxa95uxab8}

\textbf{k-NN અલ્ગોરિધમ લખો અને તેના ઉપયોગ બતાવો.}

\begin{solutionbox}

K-નિયરેસ્ટ નેઇબર્સ (k-NN) એ લેઝી લર્નિંગ અલ્ગોરિધમ છે જે k નજીકના પડોશીઓના બહુમતી
ક્લાસના આધારે ડેટા પોઇન્ટ્સને વર્ગીકૃત કરે છે.

\textbf{અલ્ગોરિધમ:}

\begin{verbatim}
1. k ની વેલ્યુ પસંદ કરો
2. બધા ટ્રેનિંગ પોઇન્ટ્સ સાથે અંતર કાઢો
3. k નજીકના પડોશીઓ પસંદ કરો
4. ક્લાસિફિકેશન માટે: બહુમતી મત
   રિગ્રેશન માટે: k પડોશીઓની સરેરાશ
5. ટેસ્ટ પોઇન્ટને ક્લાસ/વેલ્યુ અસાઇન કરો
\end{verbatim}

\textbf{અંતર ગણતરી:}

\begin{itemize}
\tightlist
\item
  \textbf{યુક્લિડિયન ડિસ્ટન્સ}: \sqrt[(x_{1}-x_{2})^{2} + (y_{1}-y_{2})^{2}]
\end{itemize}

\textbf{એપ્લિકેશનો:}

\begin{itemize}
\tightlist
\item
  \textbf{રેકમેન્ડેશન સિસ્ટમ્સ}: સમાન યુઝર પ્રાધાન્યો
\item
  \textbf{ઇમેજ રેકગ્નિશન}: પેટર્ન મેચિંગ
\item
  \textbf{મેડિકલ ડાયગ્નોસિસ}: લક્ષણોની સમાનતા
\end{itemize}

\textbf{ફાયદા:}

\begin{itemize}
\tightlist
\item
  અમલમાં મૂકવામાં સરળ
\item
  ટ્રેનિંગની જરૂર નથી
\item
  નાના ડેટાસેટ સાથે સારું કામ કરે છે
\end{itemize}

\end{solutionbox}
\begin{mnemonicbox}
``CDSA - Choose, Distance, Select, Assign''

\end{mnemonicbox}
\subsection*{પ્રશ્ન 4(અ) [3
માર્ક્સ]}\label{uxaaauxab0uxab6uxaa8-4uxa85-3-uxaaeuxab0uxa95uxab8}

\textbf{સપોર્ટ વેક્ટર મશીનની એપ્લિકેશનોની યાદી બનાવો.}

\begin{solutionbox}

\textbf{SVM એપ્લિકેશનો:}

{\def\LTcaptype{none} % do not increment counter
\begin{longtable}[]{@{}ll@{}}
\toprule\noalign{}
એપ્લિકેશન & ડોમેન \\
\midrule\noalign{}
\endhead
\bottomrule\noalign{}
\endlastfoot
\textbf{ટેક્સ્ટ ક્લાસિફિકેશન} & ડોક્યુમેન્ટ કેટેગોરાઇઝેશન \\
\textbf{ઇમેજ રેકગ્નિશન} & ફેસ ડિટેક્શન \\
\textbf{બાયોઇન્ફોર્મેટિક્સ} & જીન ક્લાસિફિકેશન \\
\end{longtable}
}

\end{solutionbox}
\begin{mnemonicbox}
``TIB - Text, Image, Bio''

\end{mnemonicbox}
\subsection*{પ્રશ્ન 4(બ) [4
માર્ક્સ]}\label{uxaaauxab0uxab6uxaa8-4uxaac-4-uxaaeuxab0uxa95uxab8}

\textbf{k-means અલ્ગોરિધમ માટે સ્યુડો કોડ બનાવો.}

\begin{solutionbox}

\textbf{K-means સ્યુડો કોડ:}

\begin{verbatim}
BEGIN K-means
1. k ક્લસ્ટર સેન્ટ્રોઇડ્સને રેન્ડમલી ઇનિશિયલાઇઝ કરો
2. REPEAT
   a. દરેક પોઇન્ટને નજીકના સેન્ટ્રોઇડને અસાઇન કરો
   b. અસાઇન કરેલા પોઇન્ટ્સના મીન તરીકે સેન્ટ્રોઇડ્સ અપડેટ કરો
   c. ટોટલ વિથિન-ક્લસ્ટર સમ ઓફ સ્ક્વેર્સ કાઢો
3. UNTIL કન્વર્જન્સ અથવા મેક્સ આવર્તન
4. RETURN ફાઇનલ ક્લસ્ટર્સ અને સેન્ટ્રોઇડ્સ
END
\end{verbatim}

\end{solutionbox}
\begin{mnemonicbox}
``IAUC - Initialize, Assign, Update, Check''

\end{mnemonicbox}
\subsection*{પ્રશ્ન 4(ક) [7
માર્ક્સ]}\label{uxaaauxab0uxab6uxaa8-4uxa95-7-uxaaeuxab0uxa95uxab8}

\textbf{અનસુપરવાઇઝ્ડ લર્નિંગની એપ્લિકેશનો લખો અને સમજાવો.}

\begin{solutionbox}

અનસુપરવાઇઝ્ડ લર્નિંગ લેબલ્ડ ઉદાહરણો વિના ડેટામાં છુપાયેલા પેટર્ન શોધે છે.

\textbf{મુખ્ય એપ્લિકેશનો:}

{\def\LTcaptype{none} % do not increment counter
\begin{longtable}[]{@{}lll@{}}
\toprule\noalign{}
એપ્લિકેશન & વર્ણન & ઉદાહરણ \\
\midrule\noalign{}
\endhead
\bottomrule\noalign{}
\endlastfoot
\textbf{કસ્ટમર સેગ્મેન્ટેશન} & વર્તન પ્રમાણે ગ્રાહકોનું ગ્રુપિંગ & માર્કેટ રિસર્ચ \\
\textbf{એનોમેલી ડિટેક્શન} & અસામાન્ય પેટર્ન ઓળખવા & ફ્રોડ ડિટેક્શન \\
\textbf{ડેટા કમ્પ્રેશન} & ડાયમેન્શનાલિટી ઘટાડવી & ઇમેજ કમ્પ્રેશન \\
\textbf{એસોસિએશન રૂલ્સ} & આઇટમ સંબંધો શોધવા & માર્કેટ બાસ્કેટ વિશ્લેષણ \\
\end{longtable}
}

\textbf{ક્લસ્ટરિંગ એપ્લિકેશનો:}

\begin{itemize}
\tightlist
\item
  \textbf{માર્કેટ રિસર્ચ}: કસ્ટમર ગ્રુપિંગ
\item
  \textbf{સોશિયલ નેટવર્ક વિશ્લેષણ}: કમ્યુનિટી ડિટેક્શન
\item
  \textbf{જીન સીક્વેન્સિંગ}: બાયોલોજિકલ ક્લાસિફિકેશન
\end{itemize}

\textbf{ડાયમેન્શનાલિટી રિડક્શન:}

\begin{itemize}
\tightlist
\item
  \textbf{વિઝ્યુઅલાઇઝેશન}: હાઇ-ડાયમેન્શનલ ડેટા પ્લોટિંગ
\item
  \textbf{ફીચર એક્સ્ટ્રેક્શન}: નોઇઝ રિડક્શન
\end{itemize}

\end{solutionbox}
\begin{mnemonicbox}
``CADA - Customer, Anomaly, Data, Association''

\end{mnemonicbox}
\subsection*{પ્રશ્ન 4(અ OR) [3
માર્ક્સ]}\label{uxaaauxab0uxab6uxaa8-4uxa85-or-3-uxaaeuxab0uxa95uxab8}

\textbf{રિગ્રેશનની એપ્લિકેશનોની યાદી બનાવો.}

\begin{solutionbox}

\textbf{રિગ્રેશન એપ્લિકેશનો:}

{\def\LTcaptype{none} % do not increment counter
\begin{longtable}[]{@{}ll@{}}
\toprule\noalign{}
એપ્લિકેશન & હેતુ \\
\midrule\noalign{}
\endhead
\bottomrule\noalign{}
\endlastfoot
\textbf{સ્ટોક પ્રાઇસ પ્રેડિક્શન} & ફાઇનાન્શિયલ ફોરકાસ્ટિંગ \\
\textbf{સેલ્સ ફોરકાસ્ટિંગ} & બિઝનેસ પ્લાનિંગ \\
\textbf{મેડિકલ ડાયગ્નોસિસ} & રિસ્ક એસેસમેન્ટ \\
\end{longtable}
}

\end{solutionbox}
\begin{mnemonicbox}
``SSM - Stock, Sales, Medical''

\end{mnemonicbox}
\subsection*{પ્રશ્ન 4(બ OR) [4
માર્ક્સ]}\label{uxaaauxab0uxab6uxaa8-4uxaac-or-4-uxaaeuxab0uxa95uxab8}

\textbf{નીચેના શબ્દો વ્યાખ્યાયિત કરો: i) Support ii) Confidence}

\begin{solutionbox}

\textbf{એસોસિએશન રૂલ શબ્દો:}

{\def\LTcaptype{none} % do not increment counter
\begin{longtable}[]{@{}
  >{\raggedright\arraybackslash}p{(\linewidth - 4\tabcolsep) * \real{0.2222}}
  >{\raggedright\arraybackslash}p{(\linewidth - 4\tabcolsep) * \real{0.4444}}
  >{\raggedright\arraybackslash}p{(\linewidth - 4\tabcolsep) * \real{0.3333}}@{}}
\toprule\noalign{}
\begin{minipage}[b]{\linewidth}\raggedright
શબ્દ
\end{minipage} & \begin{minipage}[b]{\linewidth}\raggedright
વ્યાખ્યા
\end{minipage} & \begin{minipage}[b]{\linewidth}\raggedright
ફોર્મ્યુલા
\end{minipage} \\
\midrule\noalign{}
\endhead
\bottomrule\noalign{}
\endlastfoot
\textbf{Support} & ડેટાબેઝમાં આઇટમસેટની આવર્તન & Support(A) =
\textbar A\textbar{} / \textbar D\textbar{} \\
\textbf{Confidence} & રૂલની શરતી સંભાવના & Confidence(A\rightarrowB) = Support(A\cupB)
/ Support(A) \\
\end{longtable}
}

\textbf{ઉદાહરણ:}

\begin{itemize}
\tightlist
\item
  જો 30\% ટ્રાન્ઝેક્શનમાં બ્રેડ અને દૂધ હોય: Support = 0.3
\item
  જો 80\% બ્રેડ ખરીદનારાઓ દૂધ પણ ખરીદે: Confidence = 0.8
\end{itemize}

\end{solutionbox}
\begin{mnemonicbox}
``SF-CP: Support-Frequency, Confidence-Probability''

\end{mnemonicbox}
\subsection*{પ્રશ્ન 4(ક OR) [7
માર્ક્સ]}\label{uxaaauxab0uxab6uxaa8-4uxa95-or-7-uxaaeuxab0uxa95uxab8}

\textbf{apriori algorithm ને વિગતવાર સમજાવો.}

\begin{solutionbox}

Apriori અલ્ગોરિધમ એપ્રિઓરી પ્રોપર્ટીનો ઉપયોગ કરીને ટ્રાન્ઝેક્શનલ ડેટામાં ફ્રીક્વન્ટ
આઇટમસેટ્સ શોધે છે.

\textbf{અલ્ગોરિધમ સ્ટેપ્સ:}

\begin{center}
\textbf{Mermaid Diagram (Code)}
\begin{verbatim}
{Shaded}
{Highlighting}[]
graph LR
    A[ફ્રીક્વન્ટ 1{-આઇટમસેટ્સ શોધો] {-}{-}{} B[કેન્ડિડેટ 2{-}આઇટમસેટ્સ જનરેટ કરો]}
    B {-{-}{} C[એપ્રિઓરી પ્રોપર્ટી વાપરીને પ્રૂન કરો]}
    C {-{-}{} D[ડેટાબેઝમાં સપોર્ટ કાઉન્ટ કરો]}
    D {-{-}{} E[ફ્રીક્વન્ટ k{-}આઇટમસેટ્સ શોધો]}
    E {-{-}{} F\{વધુ કેન્ડિડેટ્સ?\}}
    F {-{-}{}|હા| B}
    F {-{-}{}|ના| G[રૂલ્સ જનરેટ કરો]}
{Highlighting}
{Shaded}
\end{verbatim}
\end{center}

\textbf{એપ્રિઓરી પ્રોપર્ટી:}

\begin{itemize}
\tightlist
\item
  જો આઇટમસેટ ફ્રીક્વન્ટ છે, તો તેના બધા સબસેટ્સ ફ્રીક્વન્ટ છે
\item
  જો આઇટમસેટ ઇનફ્રીક્વન્ટ છે, તો તેના બધા સુપરસેટ્સ ઇનફ્રીક્વન્ટ છે
\end{itemize}

\textbf{સ્ટેપ્સ:}

\begin{enumerate}
\tightlist
\item
  \textbf{ડેટાબેઝ સ્કેન}: 1-આઇટમ સપોર્ટ કાઉન્ટ કરો
\item
  \textbf{કેન્ડિડેટ્સ જનરેટ}: ફ્રીક્વન્ટ k-આઇટમસેટ્સમાંથી k+1 આઇટમસેટ્સ બનાવો
\item
  \textbf{પ્રૂન}: ઇનફ્રીક્વન્ટ સબસેટ્સ સાથેના કેન્ડિડેટ્સ દૂર કરો
\item
  \textbf{સપોર્ટ કાઉન્ટ}: કેન્ડિડેટ ફ્રીક્વન્સી માટે ડેટાબેઝ સ્કેન કરો
\item
  \textbf{પુનરાવર્તન}: નવા ફ્રીક્વન્ટ આઇટમસેટ્સ ન મળે ત્યાં સુધી
\end{enumerate}

\textbf{એપ્લિકેશનો:}

\begin{itemize}
\tightlist
\item
  માર્કેટ બાસ્કેટ વિશ્લેષણ
\item
  વેબ યુઝેજ પેટર્ન
\item
  પ્રોટીન સીક્વન્સ
\end{itemize}

\end{solutionbox}
\begin{mnemonicbox}
``SGPCR - Scan, Generate, Prune, Count, Repeat''

\end{mnemonicbox}
\subsection*{પ્રશ્ન 5(અ) [3
માર્ક્સ]}\label{uxaaauxab0uxab6uxaa8-5uxa85-3-uxaaeuxab0uxa95uxab8}

\textbf{matplotlib ના મુખ્ય ફીચર્સની યાદી બનાવો.}

\begin{solutionbox}

\textbf{Matplotlib ફીચર્સ:}

{\def\LTcaptype{none} % do not increment counter
\begin{longtable}[]{@{}ll@{}}
\toprule\noalign{}
ફીચર & વર્ણન \\
\midrule\noalign{}
\endhead
\bottomrule\noalign{}
\endlastfoot
\textbf{મલ્ટિપલ પ્લોટ ટાઇપ્સ} & લાઇન, બાર, સ્કેટર, હિસ્ટોગ્રામ \\
\textbf{કસ્ટમાઇઝેશન} & કલર્સ, સ્ટાઇલ્સ, લેબલ્સ \\
\textbf{એક્સપોર્ટ ઓપ્શન્સ} & PNG, PDF, SVG ફોર્મેટ્સ \\
\end{longtable}
}

\end{solutionbox}
\begin{mnemonicbox}
``MCE - Multiple, Customization, Export''

\end{mnemonicbox}
\subsection*{પ્રશ્ન 5(બ) [4
માર્ક્સ]}\label{uxaaauxab0uxab6uxaa8-5uxaac-4-uxaaeuxab0uxa95uxab8}

\textbf{Numpy ના પ્રોગ્રામમાં iris ડેટાસેટ કેવી રીતે લોડ કરવો? સમજાવો.}

\begin{solutionbox}

\textbf{NumPy માં Iris ડેટાસેટ લોડ કરવું:}

\begin{verbatim}
import numpy as np
from sklearn.datasets import load\_iris

\# iris ડેટાસેટ લોડ કરો
iris = load\_iris()
data = iris.data    \# ફીચર્સ
target = iris.target \# લેબલ્સ
\end{verbatim}

\textbf{સ્ટેપ્સ:}

\begin{itemize}
\tightlist
\item
  \textbf{Import}: જરૂરી લાઇબ્રેરીઓ import કરો
\item
  \textbf{Load}: sklearn ના load\_iris() ફંક્શનનો ઉપયોગ કરો
\item
  \textbf{Extract}: ફીચર્સ અને ટાર્ગેટ એરે મેળવો
\item
  \textbf{Access}: .data અને .target એટ્રિબ્યુટ્સનો ઉપયોગ કરો
\end{itemize}

\end{solutionbox}
\begin{mnemonicbox}
``ILEA - Import, Load, Extract, Access''

\end{mnemonicbox}
\subsection*{પ્રશ્ન 5(ક) [7
માર્ક્સ]}\label{uxaaauxab0uxab6uxaa8-5uxa95-7-uxaaeuxab0uxa95uxab8}

\textbf{Pandas ની વિશેષતાઓ અને એપ્લિકેશનો સમજાવો.}

\begin{solutionbox}

Pandas એ Python માટે શક્તિશાળી ડેટા મેનિપ્યુલેશન અને વિશ્લેષણ લાઇબ્રેરી છે.

\textbf{મુખ્ય ફીચર્સ:}

{\def\LTcaptype{none} % do not increment counter
\begin{longtable}[]{@{}ll@{}}
\toprule\noalign{}
ફીચર & વર્ણન \\
\midrule\noalign{}
\endhead
\bottomrule\noalign{}
\endlastfoot
\textbf{DataFrame} & 2D લેબલ્ડ ડેટા સ્ટ્રક્ચર \\
\textbf{Series} & 1D લેબલ્ડ એરે \\
\textbf{Data I/O} & વિવિધ ફાઇલ ફોર્મેટ્સ વાંચવા/લખવા \\
\textbf{Data Cleaning} & મિસિંગ વેલ્યુઝ હેન્ડલ કરવા \\
\textbf{Grouping} & ગ્રુપ અને એગ્રીગેટ ઓપરેશન્સ \\
\end{longtable}
}

\textbf{એપ્લિકેશનો:}

{\def\LTcaptype{none} % do not increment counter
\begin{longtable}[]{@{}ll@{}}
\toprule\noalign{}
એપ્લિકેશન & ઉપયોગ \\
\midrule\noalign{}
\endhead
\bottomrule\noalign{}
\endlastfoot
\textbf{ડેટા એનાલિસિસ} & આંકડાકીય વિશ્લેષણ \\
\textbf{ડેટા ક્લીનિંગ} & ML માટે પ્રીપ્રોસેસિંગ \\
\textbf{ફાઇનાન્શિયલ એનાલિસિસ} & સ્ટોક માર્કેટ ડેટા \\
\textbf{વેબ સ્ક્રેપિંગ} & HTML ટેબલ્સ પાર્સ કરવા \\
\end{longtable}
}

\textbf{સામાન્ય ઓપરેશન્સ:}

\begin{itemize}
\tightlist
\item
  \textbf{ડેટા વાંચવો}: pd.read\_csv(), pd.read\_excel()
\item
  \textbf{ફિલ્ટરિંગ}: df[df[`column'] \textgreater{} value]
\item
  \textbf{ગ્રુપિંગ}: df.groupby(`column').mean()
\end{itemize}

\end{solutionbox}
\begin{mnemonicbox}
``DSDCG - DataFrame, Series, Data I/O, Cleaning,
Grouping''

\end{mnemonicbox}
\subsection*{પ્રશ્ન 5(અ OR) [3
માર્ક્સ]}\label{uxaaauxab0uxab6uxaa8-5uxa85-or-3-uxaaeuxab0uxa95uxab8}

\textbf{matplotlib ની એપ્લિકેશનોની યાદી બનાવો.}

\begin{solutionbox}

\textbf{Matplotlib એપ્લિકેશનો:}

{\def\LTcaptype{none} % do not increment counter
\begin{longtable}[]{@{}ll@{}}
\toprule\noalign{}
એપ્લિકેશન & હેતુ \\
\midrule\noalign{}
\endhead
\bottomrule\noalign{}
\endlastfoot
\textbf{સાયન્ટિફિક વિઝ્યુઅલાઇઝેશન} & રિસર્ચ ડેટા પ્લોટિંગ \\
\textbf{બિઝનેસ એનાલિટિક્સ} & ડેશબોર્ડ બનાવવું \\
\textbf{એજ્યુકેશનલ કન્ટેન્ટ} & શિક્ષણ સામગ્રી \\
\end{longtable}
}

\end{solutionbox}
\begin{mnemonicbox}
``SBE - Scientific, Business, Educational''

\end{mnemonicbox}
\subsection*{પ્રશ્ન 5(બ OR) [4
માર્ક્સ]}\label{uxaaauxab0uxab6uxaa8-5uxaac-or-4-uxaaeuxab0uxa95uxab8}

\textbf{Pandas માં csv ફાઇલ ઇમ્પોર્ટ કરવાના સ્ટેપ્સ લખો અને સમજાવો.}

\begin{solutionbox}

\textbf{Pandas માં CSV ઇમ્પોર્ટ કરવાના સ્ટેપ્સ:}

\begin{verbatim}
import pandas as pd

\# સ્ટેપ 1: pandas લાઇબ્રેરી import કરો
\# સ્ટેપ 2: read\_csv() ફંક્શનનો ઉપયોગ કરો
df = pd.read\_csv({filename.csv})

\# વૈકલ્પિક પેરામીટર્સ
df = pd.read\_csv({file.csv}, 
                 header=0,     \# પ્રથમ પંક્તિ હેડર તરીકે
                 sep={,},      \# કોમા સેપરેટર
                 index\_col=0)  \# પ્રથમ કૉલમ ઇન્ડેક્સ તરીકે
\end{verbatim}

\textbf{પ્રક્રિયા:}

\begin{itemize}
\tightlist
\item
  \textbf{Import}: pandas લાઇબ્રેરી import કરો
\item
  \textbf{Read}: pd.read\_csv() ફંક્શનનો ઉપયોગ કરો
\item
  \textbf{Specify}: ફાઇલ પાથ અને પેરામીટર્સ ઉમેરો
\item
  \textbf{Store}: DataFrame વેરિએબલમાં અસાઇન કરો
\end{itemize}

\end{solutionbox}
\begin{mnemonicbox}
``IRSS - Import, Read, Specify, Store''

\end{mnemonicbox}
\subsection*{પ્રશ્ન 5(ક OR) [7
માર્ક્સ]}\label{uxaaauxab0uxab6uxaa8-5uxa95-or-7-uxaaeuxab0uxa95uxab8}

\textbf{Scikit-Learn ની વિશેષતાઓ અને એપ્લિકેશનો સમજાવો.}

\begin{solutionbox}

Scikit-Learn એ Python માટે વ્યાપક મશીન લર્નિંગ લાઇબ્રેરી છે.

\textbf{મુખ્ય ફીચર્સ:}

{\def\LTcaptype{none} % do not increment counter
\begin{longtable}[]{@{}ll@{}}
\toprule\noalign{}
ફીચર & વર્ણન \\
\midrule\noalign{}
\endhead
\bottomrule\noalign{}
\endlastfoot
\textbf{અલ્ગોરિધમ્સ} & ક્લાસિફિકેશન, રિગ્રેશન, ક્લસ્ટરિંગ \\
\textbf{પ્રીપ્રોસેસિંગ} & ડેટા સ્કેલિંગ અને ટ્રાન્સફોર્મેશન \\
\textbf{મોડેલ સિલેક્શન} & ક્રોસ-વેલિડેશન અને ગ્રિડ સર્ચ \\
\textbf{મેટ્રિક્સ} & પરફોર્મન્સ મૂલ્યાંકન ટૂલ્સ \\
\end{longtable}
}

\textbf{એપ્લિકેશનો:}

{\def\LTcaptype{none} % do not increment counter
\begin{longtable}[]{@{}ll@{}}
\toprule\noalign{}
ડોમેન & ઉપયોગ \\
\midrule\noalign{}
\endhead
\bottomrule\noalign{}
\endlastfoot
\textbf{હેલ્થકેર} & રોગ આગાહી \\
\textbf{ફાઇનાન્સ} & ક્રેડિટ સ્કોરિંગ \\
\textbf{માર્કેટિંગ} & કસ્ટમર સેગ્મેન્ટેશન \\
\textbf{ટેકનોલોજી} & રેકમેન્ડેશન સિસ્ટમ્સ \\
\end{longtable}
}

\textbf{અલ્ગોરિધમ કેટેગરીઓ:}

\begin{itemize}
\tightlist
\item
  \textbf{સુપરવાઇઝ્ડ}: SVM, Random Forest, Linear Regression
\item
  \textbf{અનસુપરવાઇઝ્ડ}: K-means, DBSCAN, PCA
\item
  \textbf{એન્સેમ્બલ}: Bagging, Boosting
\end{itemize}

\textbf{વર્કફ્લો:}

\begin{enumerate}
\tightlist
\item
  \textbf{ડેટા તૈયારી}: પ્રીપ્રોસેસિંગ
\item
  \textbf{મોડેલ સિલેક્શન}: અલ્ગોરિધમ પસંદ કરો
\item
  \textbf{ટ્રેનિંગ}: ડેટા પર મોડેલ ફિટ કરો
\item
  \textbf{મૂલ્યાંકન}: પરફોર્મન્સ આકારો
\item
  \textbf{આગાહી}: ફોરકાસ્ટ બનાવો
\end{enumerate}

\end{solutionbox}
\begin{mnemonicbox}
``APME - Algorithms, Preprocessing, Metrics,
Evaluation''

\end{mnemonicbox}

\end{document}
