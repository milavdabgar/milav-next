\documentclass[10pt,a4paper]{article}

% content/resources/templates/preamble.tex
\usepackage[margin=0.6in]{geometry}
\author{Milav Dabgar}
\usepackage{amsmath,amssymb,amsthm}
\usepackage{booktabs}
\usepackage{multirow}
\usepackage{xcolor}
\usepackage{tcolorbox}
\tcbuselibrary{breakable,skins}
\usepackage[colorlinks=true,linkcolor=blue]{hyperref}
\usepackage{titlesec}
\usepackage{enumitem}
\usepackage{tikz}
\usepackage{pgfplots}
\usepackage{circuitikz}
\usepackage[version=4]{mhchem}
\usepackage{longtable}
\usepackage{array}
\usepackage{float}
\usepackage{caption}
\usepackage{listings}

\lstset{
  basicstyle=\small\ttfamily,
  breaklines=true,
  breakatwhitespace=false,
  postbreak=\mbox{\textcolor{red}{$\hookrightarrow$}\space},
  float=false,
  numbers=left,
  numberstyle=\tiny\color{gray},
  numbersep=10pt,
  xleftmargin=2em,
  keywordstyle=\color{blue},
  commentstyle=\color{green!60!black},
  stringstyle=\color{purple},
  backgroundcolor=\color{gray!5},
  showstringspaces=false,
  tabsize=2,
  captionpos=b,
  keepspaces=true,
  columns=flexible
}

\pgfplotsset{compat=1.18}
\usetikzlibrary{shapes,arrows,positioning,calc,patterns,decorations.pathmorphing,decorations.markings,arrows.meta}

% Color scheme
\definecolor{headcolor}{RGB}{0,102,204}
\definecolor{keycolor}{RGB}{220,20,60}
\definecolor{solutioncolor}{RGB}{34,139,34}
\definecolor{mnemoniccolor}{RGB}{148,0,211}
\definecolor{codecolor}{RGB}{0,0,100}

% Spacing
\setlength{\parskip}{3pt}
\setlist[itemize]{nosep}
\setlist[enumerate]{nosep}

% Title formatting
\titleformat{\section}{\Large\bfseries\color{headcolor}}{\thesection}{1em}{}
\titleformat{\subsection}{\large\bfseries\color{headcolor}}{\thesubsection}{1em}{}

% Pandoc tightlist compatibility
\providecommand{\tightlist}{%
  \setlength{\itemsep}{0pt}\setlength{\parskip}{0pt}}

% Pandoc longtable compatibility
\newcounter{none}
\def\thenone{}


% content/resources/templates/gujarati-boxes.tex
\usepackage{fontspec}
\usepackage{polyglossia}

% Set Gujarati as main language (document is primarily in Gujarati)
% Note: gloss-gujarati.ldf doesn't exist in polyglossia, but it will use hyphenation patterns
\setdefaultlanguage{gujarati}
\setotherlanguage{english}

% Configure Gujarati font properly
% Use Language=Default to prevent polyglossia from trying to add language-specific features
% that don't exist for Gujarati, which causes "empty feature" warnings
\newfontfamily\gujaratifont[Script=Gujarati,AutoFakeBold=2.5,AutoFakeSlant=0.3]{Noto Sans Gujarati}
\setmainfont[Script=Gujarati,AutoFakeBold=2.5,AutoFakeSlant=0.3]{Noto Sans Gujarati}
% Use Noto Sans Gujarati for monospace to support Gujarati in text
\setmonofont[Scale=0.9]{Noto Sans Gujarati}

% Configure English to use the same font
\newfontfamily\englishfont[Script=Gujarati,AutoFakeBold=2.5,AutoFakeSlant=0.3]{Noto Sans Gujarati}

% Translations for polyglossia
\gappto\captionsgujarati{
  \renewcommand{\tablename}{કોષ્ટક}
  \renewcommand{\figurename}{આકૃતિ}
}

% Helper for TikZ nodes to ensure Gujarati font
\newcommand{\gu}[1]{{\gujaratifont #1}}

% Custom environments
\newtcolorbox{solutionbox}{
    breakable,
    enhanced,
    colback=solutioncolor!5!white,
    colframe=solutioncolor!75!black,
    fonttitle=\bfseries,
    title=જવાબ
}

\newtcolorbox{solutionboxnobreak}{
 colback=solutioncolor!5!white,
 colframe=solutioncolor!75!black,
 fonttitle=\bfseries,
 title=જવાબ
}

\newtcolorbox{keyformula}{
 breakable,
 enhanced,
 colback=keycolor!5!white,
 colframe=keycolor!75!black,
 fonttitle=\bfseries,
 title=રાસાયણિક સમીકરણ/સૂત્ર
}

\newtcolorbox{mnemonicbox}{
 breakable,
 enhanced,
 colback=mnemoniccolor!5!white,
 colframe=mnemoniccolor!75!black,
 fonttitle=\bfseries,
 title=મેમરી ટ્રીક
}


\begin{document}

\begin{center}
{\Huge\bfseries\color{headcolor} Subject Name (Gujarati)}\\[5pt]
{\LARGE 4341603 -- Summer 2024}\\[3pt]
{\large Semester 1 Study Material}\\[3pt]
{\normalsize\textit{Detailed Solutions and Explanations}}
\end{center}

\vspace{10pt}

\subsection*{પ્રશ્ન 1(અ) [3
ગુણ]}\label{uxaaauxab0uxab6uxaa8-1uxa85-3-uxa97uxaa3}

\textbf{યોગ્ય ઉદાહરણનો ઉપયોગ કરીને મશીન લર્નિંગને વ્યાખ્યાયિત કરો}

\begin{solutionbox}

મશીન લર્નિંગ આર્ટિફિશિયલ ઇન્ટેલિજન્સનો એક ભાગ છે જે કમ્પ્યુટર્સને ડેટામાંથી શીખવા અને
દરેક કાર્ય માટે સ્પષ્ટ રીતે પ્રોગ્રામ કર્યા વિના નિર્ણયો લેવા માટે સક્ષમ બનાવે છે.

\textbf{ટેબલ: મશીન લર્નિંગના મુખ્ય ઘટકો}

{\def\LTcaptype{none} % do not increment counter
\begin{longtable}[]{@{}ll@{}}
\toprule\noalign{}
ઘટક & વર્ણન \\
\midrule\noalign{}
\endhead
\bottomrule\noalign{}
\endlastfoot
\textbf{ડેટા} & ટ્રેનિંગ માટે ઉપયોગમાં લેવાતી ઇનપુટ માહિતી \\
\textbf{અલ્ગોરિધમ} & પેટર્ન શીખતા ગાણિતિક મોડેલ \\
\textbf{ટ્રેનિંગ} & અલ્ગોરિધમને શીખવવાની પ્રક્રિયા \\
\textbf{પ્રિડિક્શન} & શીખેલા પેટર્ન આધારિત આઉટપુટ \\
\end{longtable}
}

\textbf{ઉદાહરણ}: ઇમેઇલ સ્પામ ડિટેક્શન સિસ્ટમ હજારો ઇમેઇલોમાંથી ``સ્પામ'' અથવા
``નોટ સ્પામ'' તરીકે લેબલ કરેલા ઇમેઇલોમાંથી શીખે છે અને નવા ઇમેઇલોને આપોઆપ વર્ગીકૃત કરે
છે.

\end{solutionbox}
\begin{mnemonicbox}
``ડેટા ડ્રાઇવ્સ ડિસિઝન્સ'' - ડેટા અલ્ગોરિધમને બુદ્ધિશાળી
નિર્ણયો લેવા માટે પ્રશિક્ષિત કરે છે

\end{mnemonicbox}
\subsection*{પ્રશ્ન 1(બ) [4
ગુણ]}\label{uxaaauxab0uxab6uxaa8-1uxaac-4-uxa97uxaa3}

\textbf{સ્કેમેટિક રેખાકૃતિના ઉપયોગ કરીને મશીન લર્નિંગની પ્રક્રિયા સમજાવો}

\begin{solutionbox}

મશીન લર્નિંગ પ્રક્રિયામાં ડેટા સંગ્રહથી લઈને મોડેલ ડિપ્લોયમેન્ટ સુધીના વ્યવસ્થિત પગલાંઓનો
સમાવેશ થાય છે.

\begin{verbatim}
flowchart LR
    A[ડેટા સંગ્રહ] {-{-} B[ડેટા પ્રીપ્રોસેસિંગ]}
    B {-{-} C[ફીચર સિલેક્શન]}
    C {-{-} D[મોડેલ સિલેક્શન]}
    D {-{-} E[ટ્રેનિંગ]}
    E {-{-} F[વેલિડેશન]}
    F {-{-} G\{પરફોર્મન્સ સારું?\}}
    G {-{-}|ના| D}
    G {-{-}|હા| H[ટેસ્ટિંગ]}
    H {-{-} I[ડિપ્લોયમેન્ટ]}
\end{verbatim}

\textbf{પ્રક્રિયાના પગલાં:}

\begin{itemize}
\tightlist
\item
  \textbf{ડેટા સંગ્રહ}: સંબંધિત ડેટાસેટ એકત્રિત કરવું
\item
  \textbf{પ્રીપ્રોસેસિંગ}: ડેટાને સાફ અને તૈયાર કરવું
\item
  \textbf{ટ્રેનિંગ}: ટ્રેનિંગ ડેટાનો ઉપયોગ કરીને અલ્ગોરિધમને શીખવવું
\item
  \textbf{વેલિડેશન}: મોડેલની કામગીરીને ચકાસવી
\item
  \textbf{ડિપ્લોયમેન્ટ}: વાસ્તવિક પ્રિડિક્શન માટે મોડેલનો ઉપયોગ
\end{itemize}

\end{solutionbox}
\begin{mnemonicbox}
``કમ્પ્યુટર્સ કેન ટ્રુલી થિંક'' - કલેક્ટ, ક્લીન, ટ્રેન, ટેસ્ટ

\end{mnemonicbox}
\subsection*{પ્રશ્ન 1(ક) [7
ગુણ]}\label{uxaaauxab0uxab6uxaa8-1uxa95-7-uxa97uxaa3}

\textbf{યોગ્ય એપ્લિકેશન સાથે વિવિધ પ્રકારના મશીન લર્નિંગ સમજાવો}

\begin{solutionbox}

મશીન લર્નિંગ અલ્ગોરિધમ્સને લર્નિંગ એપ્રોચ અને ઉપલબ્ધ ડેટાના આધારે વર્ગીકૃત કરવામાં આવે
છે.

\textbf{ટેબલ: મશીન લર્નિંગના પ્રકારો}

{\def\LTcaptype{none} % do not increment counter
\begin{longtable}[]{@{}llll@{}}
\toprule\noalign{}
પ્રકાર & લર્નિંગ મેથડ & ડેટા આવશ્યકતા & ઉદાહરણ એપ્લિકેશન \\
\midrule\noalign{}
\endhead
\bottomrule\noalign{}
\endlastfoot
\textbf{સુપરવાઇઝ્ડ} & લેબલ્ડ ડેટાનો ઉપયોગ & ઇનપુટ-આઉટપુટ જોડીઓ & ઇમેઇલ
ક્લાસિફિકેશન \\
\textbf{અનસુપરવાઇઝ્ડ} & છુપાયેલા પેટર્ન શોધે & માત્ર ઇનપુટ ડેટા & કસ્ટમર સેગમેન્ટેશન \\
\textbf{રિઇનફોર્સમેન્ટ} & રિવોર્ડ્સ દ્વારા શીખે & એન્વાયર્નમેન્ટ ફીડબેક & ગેમ પ્લેઇંગ
AI \\
\end{longtable}
}

\textbf{એપ્લિકેશન્સ:}

\begin{itemize}
\tightlist
\item
  \textbf{સુપરવાઇઝ્ડ લર્નિંગ}: મેડિકલ ડાયગ્નોસિસ, ઇમેજ રેકોગ્નિશન, ફ્રોડ ડિટેક્શન
\item
  \textbf{અનસુપરવાઇઝ્ડ લર્નિંગ}: માર્કેટ રિસર્ચ, એનોમેલી ડિટેક્શન, રેકમેન્ડેશન સિસ્ટમ્સ
\item
  \textbf{રિઇનફોર્સમેન્ટ લર્નિંગ}: ઓટોનોમસ વેહિકલ્સ, રોબોટિક્સ, સ્ટ્રેટેજિક ગેમ્સ
\end{itemize}

\textbf{ડાયાગ્રામ: લર્નિંગ ટાઇપ્સ}

\begin{verbatim}
mindmap
  root((મશીન લર્નિંગ))
    સુપરવાઇઝ્ડ
      ક્લાસિફિકેશન
      રિગ્રેશન
    અનસુપરવાઇઝ્ડ
      ક્લસ્ટરિંગ
      એસોસિએશન
    રિઇનફોર્સમેન્ટ
      પોલિસી લર્નિંગ
      વેલ્યુ ફંક્શન
\end{verbatim}

\end{solutionbox}
\begin{mnemonicbox}
``સ્ટુડન્ટ્સ યુઝ્યુઅલી રીમેમ્બર'' - સુપરવાઇઝ્ડ, અનસુપરવાઇઝ્ડ,
રિઇનફોર્સમેન્ટ

\end{mnemonicbox}
\subsection*{પ્રશ્ન 1(ક) OR [7
ગુણ]}\label{uxaaauxab0uxab6uxaa8-1uxa95-or-7-uxa97uxaa3}

\textbf{મશીન લર્નિંગમાં વિવિધ સમસ્યાઓ શું છે? ત્રણ સમસ્યાઓ કે જે મશીન લર્નિંગનો ઉપયોગ
કરીને ઉકેલી શકાતી નથી.}

\begin{solutionbox}

\textbf{ટેબલ: મશીન લર્નિંગની સમસ્યાઓ}

{\def\LTcaptype{none} % do not increment counter
\begin{longtable}[]{@{}lll@{}}
\toprule\noalign{}
સમસ્યા કેટેગરી & વર્ણન & અસર \\
\midrule\noalign{}
\endhead
\bottomrule\noalign{}
\endlastfoot
\textbf{ડેટા ક્વોલિટી} & અધૂરો, નોઇઝી, પક્ષપાતી ડેટા & નબળું મોડેલ પરફોર્મન્સ \\
\textbf{ઓવરફિટિંગ} & મોડેલ ટ્રેનિંગ ડેટાને યાદ રાખે છે & નબળું જનરલાઇઝેશન \\
\textbf{કમ્પ્યુટેશનલ} & ઉચ્ચ પ્રોસેસિંગ આવશ્યકતાઓ & રિસોર્સ મર્યાદાઓ \\
\textbf{ઇન્ટરપ્રિટેબિલિટી} & બ્લેક બોક્સ મોડેલ્સ & પારદર્શિતાનો અભાવ \\
\end{longtable}
}

\textbf{ML માટે અનુપયુક્ત સમસ્યાઓ:}

\begin{enumerate}
\tightlist
\item
  \textbf{સિમ્પલ રૂલ-બેસ્ડ ટાસ્ક} - મૂળભૂત ગણતરીઓ, સિમ્પલ if-then લોજિક
\item
  \textbf{નૈતિક નિર્ણયો} - માનવીય મૂલ્યોની આવશ્યકતા ધરાવતા નૈતિક जजમેન્ટ્સ
\item
  \textbf{ક્રિએટિવ એક્સપ્રેશન} - માનવીય લાગણીની આવશ્યકતા ધરાવતી મૂળ કલાત્મક
  સર્જના
\end{enumerate}

\textbf{અન્ય સમસ્યાઓ:}

\begin{itemize}
\tightlist
\item
  \textbf{પ્રાઇવસી ચિંતાઓ}: સંવેદનશીલ ડેટા હેન્ડલિંગ
\item
  \textbf{બાયસ પ્રોપેગેશન}: અન્યાયકારક અલ્ગોરિધમિક નિર્ણયો
\item
  \textbf{ફીચર સિલેક્શન}: સંબંધિત ઇનપુટ વેરિએબલ્સ પસંદ કરવા
\end{itemize}

\end{solutionbox}
\begin{mnemonicbox}
``ડેટા ડ્રાઇવ્સ ક્વોલિટી'' - ડેટા ક્વોલિટી સીધી રીતે મોડેલ
ક્વોલિટીને અસર કરે છે

\end{mnemonicbox}
\subsection*{પ્રશ્ન 2(અ) [3
ગુણ]}\label{uxaaauxab0uxab6uxaa8-2uxa85-3-uxa97uxaa3}

\textbf{સામાન્ય મશીન લર્નિંગ સમસ્યામાં વિવિધ પ્રકારના ડેટાનો સારાંશ આપો}

\begin{solutionbox}

\textbf{ટેબલ: મશીન લર્નિંગમાં ડેટા પ્રકારો}

{\def\LTcaptype{none} % do not increment counter
\begin{longtable}[]{@{}lll@{}}
\toprule\noalign{}
ડેટા પ્રકાર & વર્ણન & ઉદાહરણ \\
\midrule\noalign{}
\endhead
\bottomrule\noalign{}
\endlastfoot
\textbf{ન્યુમેરિકલ} & માત્રાત્મક મૂલ્યો & ઉંમર: 25, ઊંચાઈ: 170cm \\
\textbf{કેટેગોરિકલ} & અસ્પષ્ટ કેટેગરીઓ & રંગ: લાલ, વાદળી, લીલો \\
\textbf{ઓર્ડિનલ} & ક્રમબદ્ધ કેટેગરીઓ & રેટિંગ: નબળું, સારું, ઉત્તમ \\
\textbf{બાઇનરી} & બે શક્ય મૂલ્યો & લિંગ: પુરુષ/સ્ત્રી \\
\end{longtable}
}

\textbf{લક્ષણો:}

\begin{itemize}
\tightlist
\item
  \textbf{સ્ટ્રક્ચર્ડ}: ટેબલોમાં વ્યવસ્થિત (ડેટાબેસેસ, સ્પ્રેડશીટ્સ)
\item
  \textbf{અનસ્ટ્રક્ચર્ડ}: ઇમેજ, ટેક્સ્ટ, ઓડિયો ફાઇલો
\item
  \textbf{ટાઇમ-સીરીઝ}: સમય પર ડેટા પોઇન્ટ્સ
\end{itemize}

\end{solutionbox}
\begin{mnemonicbox}
``નંબર્સ કાઉન્ટ બેટર દેન વર્ડ્સ'' - ન્યુમેરિકલ, કેટેગોરિકલ,
બાઇનરી, ટેક્સ્ટ

\end{mnemonicbox}
\subsection*{પ્રશ્ન 2(બ) [4
ગુણ]}\label{uxaaauxab0uxab6uxaa8-2uxaac-4-uxa97uxaa3}

\textbf{બંને એટ્રિબ્યુટ માટે વેરિયન્સ ગણતરી કરો. નક્કી કરો કે કઈ એટ્રિબ્યુટ મીનની
આસપાસ સ્પ્રેડ આઉટ છે}

\begin{solutionbox}

\textbf{આપેલ ડેટા:}

\begin{itemize}
\tightlist
\item
  એટ્રિબ્યુટ 1: 32, 37, 47, 50, 59
\item
  એટ્રિબ્યુટ 2: 48, 40, 41, 47, 49
\end{itemize}

\textbf{ગણતરીઓ:}

\textbf{એટ્રિબ્યુટ 1:}

\begin{itemize}
\tightlist
\item
  મીન = (32+37+47+50+59)/5 = 225/5 = 45
\item
  વેરિયન્સ = [(32-45)^{2} + (37-45)^{2} + (47-45)^{2} + (50-45)^{2} + (59-45)^{2}]/5
\item
  વેરિયન્સ = [169 + 64 + 4 + 25 + 196]/5 = 458/5 = 91.6
\end{itemize}

\textbf{એટ્રિબ્યુટ 2:}

\begin{itemize}
\tightlist
\item
  મીન = (48+40+41+47+49)/5 = 225/5 = 45
\item
  વેરિયન્સ = [(48-45)^{2} + (40-45)^{2} + (41-45)^{2} + (47-45)^{2} + (49-45)^{2}]/5
\item
  વેરિયન્સ = [9 + 25 + 16 + 4 + 16]/5 = 70/5 = 14
\end{itemize}

\textbf{પરિણામ}: એટ્રિબ્યુટ 1 (વેરિયન્સ = 91.6) એટ્રિબ્યુટ 2 (વેરિયન્સ = 14) કરતાં
વધુ સ્પ્રેડ આઉટ છે.

\end{solutionbox}
\begin{mnemonicbox}
``હાયર વેરિયન્સ શોઝ સ્પ્રેડ'' - વધુ વેરિયન્સ વધુ વિખેરાઈને
દર્શાવે છે

\end{mnemonicbox}
\subsection*{પ્રશ્ન 2(ક) [7
ગુણ]}\label{uxaaauxab0uxab6uxaa8-2uxa95-7-uxa97uxaa3}

\textbf{ડેટા ગુણવત્તા સમસ્યા તરફ દોરી જતા ફેક્ટર્સની યાદી બનાવો. આઉટલાયર્સ અને
મિસિંગ વેલ્યુ કેવી રીતે હેન્ડલ કરવું}

\begin{solutionbox}

\textbf{ટેબલ: ડેટા ગુણવત્તા સમસ્યાઓ}

{\def\LTcaptype{none} % do not increment counter
\begin{longtable}[]{@{}lll@{}}
\toprule\noalign{}
ફેક્ટર & કારણ & સોલ્યુશન \\
\midrule\noalign{}
\endhead
\bottomrule\noalign{}
\endlastfoot
\textbf{અપૂર્ણતા} & મિસિંગ ડેટા કલેક્શન & ઇમ્પ્યુટેશન ટેકનિક્સ \\
\textbf{અસંગતતા} & વિવિધ ડેટા ફોર્મેટ્સ & સ્ટેન્ડર્ડાઇઝેશન \\
\textbf{અચોક્કસતા} & હ્યુમન/સેન્સર એરર્સ & વેલિડેશન રૂલ્સ \\
\textbf{નોઇઝ} & રેન્ડમ વેરિએશન્સ & ફિલ્ટરિંગ મેથડ્સ \\
\end{longtable}
}

\textbf{આઉટલાયર્સ હેન્ડલ કરવું:}

\begin{itemize}
\tightlist
\item
  \textbf{ડિટેક્શન}: સ્ટેટિસ્ટિકલ મેથડ્સ (Z-score, IQR)
\item
  \textbf{ટ્રીટમેન્ટ}: એક્સ્ટ્રીમ વેલ્યુઝને રીમૂવ, ટ્રાન્સફોર્મ, અથવા કેપ કરવી
\item
  \textbf{વિઝ્યુઅલાઇઝેશન}: બોક્સ પ્લોટ્સ, સ્કેટર પ્લોટ્સ
\end{itemize}

\textbf{મિસિંગ વેલ્યુઝ હેન્ડલ કરવું:}

\begin{itemize}
\tightlist
\item
  \textbf{ડિલીશન}: અપૂર્ણ રેકોર્ડ્સ રીમૂવ કરવા
\item
  \textbf{ઇમ્પ્યુટેશન}: મીન, મીડિયન, અથવા મોડ સાથે ભરવું
\item
  \textbf{પ્રિડિક્શન}: મિસિંગ વેલ્યુઝની આગાહી કરવા માટે ML નો ઉપયોગ
\end{itemize}

\textbf{કોડ ઉદાહરણ:}

\begin{verbatim}
\# મિસિંગ વેલ્યુઝ હેન્ડલ કરવું
df.fillna(df.mean())  \# મીન ઇમ્પ્યુટેશન
df.dropna()          \# મિસિંગ રોઝ રીમૂવ કરવા
\end{verbatim}

\end{solutionbox}
\begin{mnemonicbox}
``ક્લીન ડેટા મેક્સ મોડેલ્સ'' - સાફ ડેટા બેહતર મોડેલ્સ બનાવે છે

\end{mnemonicbox}
\subsection*{પ્રશ્ન 2(અ) OR [3
ગુણ]}\label{uxaaauxab0uxab6uxaa8-2uxa85-or-3-uxa97uxaa3}

\textbf{વિવિધ મશીન લર્નિંગ પ્રવૃત્તિઓ આપો}

\begin{solutionbox}

\textbf{ટેબલ: મશીન લર્નિંગ પ્રવૃત્તિઓ}

{\def\LTcaptype{none} % do not increment counter
\begin{longtable}[]{@{}
  >{\raggedright\arraybackslash}p{(\linewidth - 4\tabcolsep) * \real{0.3636}}
  >{\raggedright\arraybackslash}p{(\linewidth - 4\tabcolsep) * \real{0.2727}}
  >{\raggedright\arraybackslash}p{(\linewidth - 4\tabcolsep) * \real{0.3636}}@{}}
\toprule\noalign{}
\begin{minipage}[b]{\linewidth}\raggedright
પ્રવૃત્તિ
\end{minipage} & \begin{minipage}[b]{\linewidth}\raggedright
હેતુ
\end{minipage} & \begin{minipage}[b]{\linewidth}\raggedright
ઉદાહરણ
\end{minipage} \\
\midrule\noalign{}
\endhead
\bottomrule\noalign{}
\endlastfoot
\textbf{ડેટા કલેક્શન} & સંબંધિત માહિતી એકત્રિત કરવી & સર્વે, સેન્સર્સ, ડેટાબેસેસ \\
\textbf{ડેટા પ્રીપ્રોસેસિંગ} & ડેટાને સાફ અને તૈયાર કરવું & નોઇઝ રીમૂવ કરવું, મિસિંગ
વેલ્યુઝ હેન્ડલ કરવું \\
\textbf{ફીચર એન્જિનિયરિંગ} & અર્થપૂર્ણ વેરિએબલ્સ બનાવવા & રો ડેટામાંથી ફીચર્સ
એક્સ્ટ્રેક્ટ કરવા \\
\textbf{મોડેલ ટ્રેનિંગ} & અલ્ગોરિધમને પેટર્ન શીખવવા & ટ્રેનિંગ ડેટાસેટનો ઉપયોગ \\
\textbf{મોડેલ ઇવેલ્યુએશન} & પરફોર્મન્સ આકારણી & ટેસ્ટ એક્યુરસી, પ્રિસિઝન, રિકોલ \\
\textbf{મોડેલ ડિપ્લોયમેન્ટ} & મોડેલને પ્રોડક્શનમાં મૂકવું & વેબ સર્વિસેસ, મોબાઇલ એપ્સ \\
\end{longtable}
}

\textbf{મુખ્ય પ્રવૃત્તિઓ:}

\begin{itemize}
\tightlist
\item
  \textbf{એક્સ્પ્લોરેટરી ડેટા એનાલિસિસ}: ડેટા પેટર્ન સમજવા
\item
  \textbf{હાયપરપેરામીટર ટ્યુનિંગ}: મોડેલ સેટિંગ્સ ઓપ્ટિમાઇઝ કરવા
\item
  \textbf{ક્રોસ-વેલિડેશન}: મજબૂત પરફોર્મન્સ આકારણી
\end{itemize}

\end{solutionbox}
\begin{mnemonicbox}
``ડેટા મોડેલ્સ પર્ફોર્મ એક્સેલન્ટલી'' - ડેટા તૈયારી, મોડેલ
બિલ્ડિંગ, પરફોર્મન્સ ઇવેલ્યુએશન, એક્ઝિક્યુશન

\end{mnemonicbox}
\subsection*{પ્રશ્ન 2(બ) OR [4
ગુણ]}\label{uxaaauxab0uxab6uxaa8-2uxaac-or-4-uxa97uxaa3}

\textbf{નીચેની સંખ્યાઓના મીન અને મીડિયન ની ગણતરી કરો:
12,15,18,20,22,24,28,30}

\begin{solutionbox}

\textbf{આપેલ સંખ્યાઓ:} 12, 15, 18, 20, 22, 24, 28, 30

\textbf{મીન ગણતરી:} મીન = (12+15+18+20+22+24+28+30)/8 = 169/8 = 21.125

\textbf{મીડિયન ગણતરી:}

\begin{itemize}
\tightlist
\item
  સંખ્યાઓ પહેલેથી સૉર્ટ કરેલી છે: 12, 15, 18, 20, 22, 24, 28, 30
\item
  સમ કાઉન્ટ (8 સંખ્યાઓ)
\item
  મીડિયન = (4મી સંખ્યા + 5મી સંખ્યા)/2 = (20 + 22)/2 = 21
\end{itemize}

\textbf{ટેબલ: સ્ટેટિસ્ટિકલ સમરી}

{\def\LTcaptype{none} % do not increment counter
\begin{longtable}[]{@{}lll@{}}
\toprule\noalign{}
માપદંડ & મૂલ્ય & વર્ણન \\
\midrule\noalign{}
\endhead
\bottomrule\noalign{}
\endlastfoot
\textbf{મીન} & 21.125 & સરેરાશ મૂલ્ય \\
\textbf{મીડિયન} & 21 & મધ્યમ મૂલ્ય \\
\textbf{કાઉન્ટ} & 8 & કુલ સંખ્યાઓ \\
\end{longtable}
}

\end{solutionbox}
\begin{mnemonicbox}
``મિડલ મેક્સ મીડિયન'' - મધ્યમ મૂલ્ય મીડિયન આપે છે

\end{mnemonicbox}
\subsection*{પ્રશ્ન 2(ક) OR [7
ગુણ]}\label{uxaaauxab0uxab6uxaa8-2uxa95-or-7-uxa97uxaa3}

\textbf{ડેટા પ્રીપ્રોસેસિંગના સંદર્ભમાં ડાયમેન્શનાલિટી રિડક્શન અને ફીચર સબસેટ સિલેક્શન
પર ટૂંકી નોંધ લખો}

\begin{solutionbox}

\textbf{ડાયમેન્શનાલિટી રિડક્શન} અપ્રસ્તુત ફીચર્સને દૂર કરે છે અને કોમ્પ્યુટેશનલ જટિલતા
ઘટાડે છે જ્યારે મહત્વપૂર્ણ માહિતી જાળવી રાખે છે.

\textbf{ટેબલ: ડાયમેન્શનાલિટી રિડક્શન ટેકનિક્સ}

{\def\LTcaptype{none} % do not increment counter
\begin{longtable}[]{@{}lll@{}}
\toprule\noalign{}
ટેકનિક & મેથડ & વપરાશ \\
\midrule\noalign{}
\endhead
\bottomrule\noalign{}
\endlastfoot
\textbf{PCA} & પ્રિન્સિપલ કમ્પોનન્ટ એનાલિસિસ & લીનિયર રિડક્શન \\
\textbf{LDA} & લીનિયર ડિસ્ક્રિમિનન્ટ એનાલિસિસ & ક્લાસિફિકેશન ટાસ્ક્સ \\
\textbf{t-SNE} & નોન-લીનિયર એમ્બેડિંગ & વિઝ્યુઅલાઇઝેશન \\
\textbf{ફીચર સિલેક્શન} & મહત્વપૂર્ણ ફીચર્સ પસંદ કરવા & ઓવરફિટિંગ ઘટાડવું \\
\end{longtable}
}

\textbf{ફીચર સબસેટ સિલેક્શન મેથડ્સ:}

\begin{itemize}
\tightlist
\item
  \textbf{ફિલ્ટર મેથડ્સ}: સ્ટેટિસ્ટિકલ ટેસ્ટ્સ, કોરિલેશન એનાલિસિસ
\item
  \textbf{રેપર મેથડ્સ}: ફોરવર્ડ/બેકવર્ડ સિલેક્શન
\item
  \textbf{એમ્બેડેડ મેથડ્સ}: LASSO, રિજ રિગ્રેશન
\end{itemize}

\textbf{ફાયદાઓ:}

\begin{itemize}
\tightlist
\item
  \textbf{કોમ્પ્યુટેશનલ કાર્યક્ષમતા}: ઝડપી ટ્રેનિંગ અને પ્રિડિક્શન
\item
  \textbf{સ્ટોરેજ રિડક્શન}: ઓછી મેમરી આવશ્યકતાઓ
\item
  \textbf{નોઇઝ રિડક્શન}: અપ્રસ્તુત ફીચર્સ દૂર કરવા
\item
  \textbf{વિઝ્યુઅલાઇઝેશન}: 2D/3D પ્લોટિંગ સક્ષમ કરવું
\end{itemize}

\textbf{કોડ ઉદાહરણ:}

\begin{verbatim}
from sklearn.decomposition import PCA
pca = PCA(n\_components=2)
reduced\_data = pca.fit\_transform(data)
\end{verbatim}

\end{solutionbox}
\begin{mnemonicbox}
``રિડ્યુસ ફીચર્સ, ઇમ્પ્રૂવ પર્ફોર્મન્સ'' - ઓછા ફીચર્સ ઘણીવાર
બેહતર મોડેલ્સ તરફ દોરી જાય છે

\end{mnemonicbox}
\subsection*{પ્રશ્ન 3(અ) [3
ગુણ]}\label{uxaaauxab0uxab6uxaa8-3uxa85-3-uxa97uxaa3}

\textbf{શું બાયસ ML મોડેલના પરફોર્મન્સને અસર કરે છે? ટૂંકમાં સમજાવો}

\begin{solutionbox}

હા, બાયસ પ્રિડિક્શન્સમાં સિસ્ટેમેટિક એરર્સ બનાવીને ML મોડેલના પરફોર્મન્સને નોંધપાત્ર
રીતે અસર કરે છે.

\textbf{ટેબલ: બાયસના પ્રકારો}

{\def\LTcaptype{none} % do not increment counter
\begin{longtable}[]{@{}lll@{}}
\toprule\noalign{}
બાયસ પ્રકાર & વર્ણન & અસર \\
\midrule\noalign{}
\endhead
\bottomrule\noalign{}
\endlastfoot
\textbf{સિલેક્શન બાયસ} & બિન-પ્રતિનિધિત્વકારી ડેટા & નબળું જનરલાઇઝેશન \\
\textbf{કન્ફર્મેશન બાયસ} & અપેક્ષિત પરિણામોની તરફેણ & ત્રાંસા નિષ્કર્ષો \\
\textbf{અલ્ગોરિધમિક બાયસ} & મોડેલ ધારણાઓ & અન્યાયકારક પ્રિડિક્શન્સ \\
\end{longtable}
}

\textbf{પરફોર્મન્સ પર અસરો:}

\begin{itemize}
\tightlist
\item
  \textbf{અંડરફિટિંગ}: ઉચ્ચ બાયસ અતિ સરળ મોડેલ્સ તરફ દોરી જાય છે
\item
  \textbf{નબળી ચોકસાઈ}: સિસ્ટેમેટિક એરર્સ એકંદર પરફોર્મન્સ ઘટાડે છે
\item
  \textbf{અન્યાયકારક નિર્ણયો}: પક્ષપાતી મોડેલ્સ જૂથો સામે ભેદભાવ કરે છે
\end{itemize}

\textbf{ઘટાડવાની વ્યૂહરચનાઓ:}

\begin{itemize}
\tightlist
\item
  વિવિધ ટ્રેનિંગ ડેટા
\item
  ક્રોસ-વેલિડેશન ટેકનિક્સ
\item
  બાયસ ડિટેક્શન અલ્ગોરિધમ્સ
\end{itemize}

\end{solutionbox}
\begin{mnemonicbox}
``બાયસ બ્રેક્સ બેટર પર્ફોર્મન્સ'' - બાયસ મોડેલની અસરકારકતા
ઘટાડે છે

\end{mnemonicbox}
\subsection*{પ્રશ્ન 3(બ) [4
ગુણ]}\label{uxaaauxab0uxab6uxaa8-3uxaac-4-uxa97uxaa3}

\textbf{ક્રોસ-વેલિડેશન અને બૂટસ્ટ્રેપ સેમ્પલિંગની સરખામણી કરો}

\begin{solutionbox}

\textbf{ટેબલ: ક્રોસ-વેલિડેશન vs બૂટસ્ટ્રેપ સેમ્પલિંગ}

{\def\LTcaptype{none} % do not increment counter
\begin{longtable}[]{@{}lll@{}}
\toprule\noalign{}
પાસું & ક્રોસ-વેલિડેશન & બૂટસ્ટ્રેપ સેમ્પલિંગ \\
\midrule\noalign{}
\endhead
\bottomrule\noalign{}
\endlastfoot
\textbf{મેથડ} & ડેટાને ફોલ્ડ્સમાં વિભાજિત કરવું & રિપ્લેસમેન્ટ સાથે સેમ્પલ કરવું \\
\textbf{ડેટા ઉપયોગ} & બધો ડેટા વાપરે છે & મલ્ટિપલ સેમ્પલ્સ બનાવે છે \\
\textbf{હેતુ} & મોડેલ ઇવેલ્યુએશન & અનિશ્ચિતતાનો અંદાજ \\
\textbf{ઓવરલેપ} & સેટ્સ વચ્ચે કોઈ ઓવરલેપ નથી & ડુપ્લિકેટ સેમ્પલ્સની મંજૂરી \\
\end{longtable}
}

\textbf{ક્રોસ-વેલિડેશન:}

\begin{itemize}
\tightlist
\item
  ડેટાને k સમાન ભાગોમાં વહેંચે છે
\item
  k-1 ભાગોમાં ટ્રેન કરે છે, 1 ભાગમાં ટેસ્ટ કરે છે
\item
  મજબૂત ઇવેલ્યુએશન માટે k વખત પુનરાવર્તન કરે છે
\end{itemize}

\textbf{બૂટસ્ટ્રેપ સેમ્પલિંગ:}

\begin{itemize}
\tightlist
\item
  રિપ્લેસમેન્ટ સાથે રેન્ડમ સેમ્પલ્સ બનાવે છે
\item
  સમાન સાઇઝના મલ્ટિપલ ડેટાસેટ્સ જનરેટ કરે છે
\item
  કોન્ફિડન્સ ઇન્ટરવલ્સનો અંદાજ કાઢે છે
\end{itemize}

\textbf{એપ્લિકેશન્સ:}

\begin{itemize}
\tightlist
\item
  \textbf{ક્રોસ-વેલિડેશન}: મોડેલ સિલેક્શન, હાયપરપેરામીટર ટ્યુનિંગ
\item
  \textbf{બૂટસ્ટ્રેપ}: સ્ટેટિસ્ટિકલ ઇન્ફરન્સ, કોન્ફિડન્સ એસ્ટિમેશન
\end{itemize}

\end{solutionbox}
\begin{mnemonicbox}
``ક્રોસ ચેક્સ, બૂટસ્ટ્રેપ બિલ્ડ્સ'' - ક્રોસ-વેલિડેશન પરફોર્મન્સ ચેક
કરે છે, બૂટસ્ટ્રેપ કોન્ફિડન્સ બિલ્ડ કરે છે

\end{mnemonicbox}
\subsection*{પ્રશ્ન 3(ક) [7
ગુણ]}\label{uxaaauxab0uxab6uxaa8-3uxa95-7-uxa97uxaa3}

\textbf{કન્ફ્યુઝન મેટ્રિક્સ ગણતરી અને મેટ્રિક્સ}

\begin{solutionbox}

\textbf{આપેલ માહિતી:}

\begin{itemize}
\tightlist
\item
  True Positive (TP): 83 (પ્રિડિક્ટેડ ખરીદશે, વાસ્તવમાં ખરીદ્યું)
\item
  False Positive (FP): 7 (પ્રિડિક્ટેડ ખરીદશે, નથી ખરીદ્યું)
\item
  False Negative (FN): 5 (પ્રિડિક્ટેડ નહીં ખરીદે, વાસ્તવમાં ખરીદ્યું)
\item
  True Negative (TN): 5 (પ્રિડિક્ટેડ નહીં ખરીદે, નથી ખરીદ્યું)
\end{itemize}

\textbf{કન્ફ્યુઝન મેટ્રિક્સ:}

{\def\LTcaptype{none} % do not increment counter
\begin{longtable}[]{@{}lll@{}}
\toprule\noalign{}
& પ્રિડિક્ટેડ ખરીદશે & પ્રિડિક્ટેડ નહીં ખરીદે \\
\midrule\noalign{}
\endhead
\bottomrule\noalign{}
\endlastfoot
\textbf{વાસ્તવમાં ખરીદે} & 83 (TP) & 5 (FN) \\
\textbf{વાસ્તવમાં નહીં ખરીદે} & 7 (FP) & 5 (TN) \\
\end{longtable}
}

\textbf{ગણતરીઓ:}

\textbf{અ) એરર રેટ:} એરર રેટ = (FP + FN) / કુલ = (7 + 5) / 100 = 0.12 =
12\%

\textbf{બ) પ્રિસિઝન:} પ્રિસિઝન = TP / (TP + FP) = 83 / (83 + 7) = 83/90 =
0.922 = 92.2\%

\textbf{ક) રિકોલ:} રિકોલ = TP / (TP + FN) = 83 / (83 + 5) = 83/88 =
0.943 = 94.3\%

\textbf{ડ) F-મેઝર:} F-મેઝર = 2 \times (પ્રિસિઝન \times રિકોલ) / (પ્રિસિઝન + રિકોલ)
F-મેઝર = 2 \times (0.922 \times 0.943) / (0.922 + 0.943) = 0.932 = 93.2\%

\textbf{ટેબલ: પરફોર્મન્સ મેટ્રિક્સ}

{\def\LTcaptype{none} % do not increment counter
\begin{longtable}[]{@{}lll@{}}
\toprule\noalign{}
મેટ્રિક & મૂલ્ય & અર્થઘટન \\
\midrule\noalign{}
\endhead
\bottomrule\noalign{}
\endlastfoot
\textbf{એરર રેટ} & 12\% & મોડેલ 12\% ખોટી આગાહીઓ કરે છે \\
\textbf{પ્રિસિઝન} & 92.2\% & પ્રિડિક્ટેડ ખરીદદારોમાંથી 92.2\% ખરેખર ખરીદે છે \\
\textbf{રિકોલ} & 94.3\% & મોડેલ 94.3\% વાસ્તવિક ખરીદદારોને ઓળખે છે \\
\textbf{F-મેઝર} & 93.2\% & સંતુલિત પરફોર્મન્સ માપદંડ \\
\end{longtable}
}

\end{solutionbox}
\begin{mnemonicbox}
``પર્ફેક્ટ રિકોલ ફાઇન્ડ્સ એવરીવન'' - પ્રિસિઝન ચોકસાઈ માપે
છે, રિકોલ બધા પોઝિટિવ શોધે છે

\end{mnemonicbox}
\subsection*{પ્રશ્ન 3(અ) OR [3
ગુણ]}\label{uxaaauxab0uxab6uxaa8-3uxa85-or-3-uxa97uxaa3}

\textbf{સંક્ષિપ્તમાં વ્યાખ્યાયિત કરો: અ) ટાર્ગેટ ફંક્શન બ) કોસ્ટ ફંક્શન ક) લોસ ફંક્શન}

\begin{solutionbox}

\textbf{ટેબલ: ફંક્શન વ્યાખ્યાઓ}

{\def\LTcaptype{none} % do not increment counter
\begin{longtable}[]{@{}
  >{\raggedright\arraybackslash}p{(\linewidth - 4\tabcolsep) * \real{0.3636}}
  >{\raggedright\arraybackslash}p{(\linewidth - 4\tabcolsep) * \real{0.4091}}
  >{\raggedright\arraybackslash}p{(\linewidth - 4\tabcolsep) * \real{0.2273}}@{}}
\toprule\noalign{}
\begin{minipage}[b]{\linewidth}\raggedright
ફંક્શન
\end{minipage} & \begin{minipage}[b]{\linewidth}\raggedright
વ્યાખ્યા
\end{minipage} & \begin{minipage}[b]{\linewidth}\raggedright
હેતુ
\end{minipage} \\
\midrule\noalign{}
\endhead
\bottomrule\noalign{}
\endlastfoot
\textbf{ટાર્ગેટ ફંક્શન} & ઇનપુટથી આઉટપુટ સુધીની આદર્શ મેપિંગ & આપણે શું શીખવા માગીએ
છીએ \\
\textbf{કોસ્ટ ફંક્શન} & એકંદર મોડેલ એરરને માપે છે & કુલ પરફોર્મન્સનું મૂલ્યાંકન \\
\textbf{લોસ ફંક્શન} & એક પ્રિડિક્શન માટે એરર માપે છે & વ્યક્તિગત પ્રિડિક્શન એરર \\
\end{longtable}
}

\textbf{વિગતવાર સમજૂતી:}

\begin{itemize}
\tightlist
\item
  \textbf{ટાર્ગેટ ફંક્શન}: f(x) = y, સાચો સંબંધ જેનો આપણે અંદાજ કાઢવા માગીએ છીએ
\item
  \textbf{કોસ્ટ ફંક્શન}: તમામ લોસ ફંક્શન્સની સરેરાશ, J = (1/n)Σloss(yi, ŷi)
\item
  \textbf{લોસ ફંક્શન}: એક સેમ્પલ માટે એરર, દા.ત., (yi - ŷi)^{2}
\end{itemize}

\textbf{સંબંધ}: કોસ્ટ ફંક્શન સામાન્ય રીતે તમામ ટ્રેનિંગ ઉદાહરણોમાં લોસ ફંક્શન્સની
સરેરાશ હોય છે.

\end{solutionbox}
\begin{mnemonicbox}
``ટાર્ગેટ કોસ્ટ્સ લેસ'' - ટાર્ગેટ ફંક્શન આદર્શ છે, કોસ્ટ ફંક્શન
એકંદર એરર માપે છે, લોસ ફંક્શન વ્યક્તિગત એરર માપે છે

\end{mnemonicbox}
\subsection*{પ્રશ્ન 3(બ) OR [4
ગુણ]}\label{uxaaauxab0uxab6uxaa8-3uxaac-or-4-uxa97uxaa3}

\textbf{બેલેન્સ્ડ ફિટ, અંડરફિટ અને ઓવરફિટ સમજાવો}

\begin{solutionbox}

\textbf{ટેબલ: મોડેલ ફિટિંગ પ્રકારો}

{\def\LTcaptype{none} % do not increment counter
\begin{longtable}[]{@{}llll@{}}
\toprule\noalign{}
ફિટ પ્રકાર & ટ્રેનિંગ એરર & વેલિડેશન એરર & લક્ષણો \\
\midrule\noalign{}
\endhead
\bottomrule\noalign{}
\endlastfoot
\textbf{અંડરફિટ} & ઊંચો & ઊંચો & ખૂબ સાદું મોડેલ \\
\textbf{બેલેન્સ્ડ ફિટ} & નીચો & નીચો & આદર્શ જટિલતા \\
\textbf{ઓવરફિટ} & ખૂબ નીચો & ઊંચો & ખૂબ જટિલ મોડેલ \\
\end{longtable}
}

\textbf{વિઝ્યુઅલાઇઝેશન:}

\begin{center}
\textbf{Mermaid Diagram (Code)}
\begin{verbatim}
{Shaded}
{Highlighting}[]
graph LR
    A[અંડરફિટ] {-{-}{} B[બેલેન્સ્ડ ફિટ]}
    B {-{-}{} C[ઓવરફિટ]}
    A {-{-}{} D[હાઇ બાયસ]}
    C {-{-}{} E[હાઇ વેરિયન્સ]}
    B {-{-}{} F[ઓપ્ટિમલ પરફોર્મન્સ]}
{Highlighting}
{Shaded}
\end{verbatim}
\end{center}

\textbf{લક્ષણો:}

\begin{itemize}
\tightlist
\item
  \textbf{અંડરફિટ}: મોડેલ ખૂબ સાદું, પેટર્ન કેપ્ચર કરી શકતું નથી
\item
  \textbf{બેલેન્સ્ડ ફિટ}: યોગ્ય જટિલતા, સારું જનરલાઇઝેશન
\item
  \textbf{ઓવરફિટ}: મોડેલ ખૂબ જટિલ, ટ્રેનિંગ ડેટાને યાદ રાખે છે
\end{itemize}

\textbf{સોલ્યુશન્સ:}

\begin{itemize}
\tightlist
\item
  \textbf{અંડરફિટ}: મોડેલ જટિલતા વધારવી, ફીચર્સ ઉમેરવા
\item
  \textbf{ઓવરફિટ}: રેગ્યુલરાઇઝેશન, ક્રોસ-વેલિડેશન, વધુ ડેટા
\end{itemize}

\end{solutionbox}
\begin{mnemonicbox}
``બેલેન્સ બ્રિંગ્સ બેસ્ટ રિઝલ્ટ્સ'' - સંતુલિત મોડેલ્સ નવા ડેટા પર
શ્રેષ્ઠ પરફોર્મ કરે છે

\end{mnemonicbox}
\subsection*{પ્રશ્ન 4(અ) [3
ગુણ]}\label{uxaaauxab0uxab6uxaa8-4uxa85-3-uxa97uxaa3}

\textbf{ક્લાસિફિકેશન લર્નિંગ સ્ટેપ્સ આપો}

\begin{solutionbox}

\textbf{ટેબલ: ક્લાસિફિકેશન લર્નિંગ સ્ટેપ્સ}

{\def\LTcaptype{none} % do not increment counter
\begin{longtable}[]{@{}lll@{}}
\toprule\noalign{}
સ્ટેપ & વર્ણન & હેતુ \\
\midrule\noalign{}
\endhead
\bottomrule\noalign{}
\endlastfoot
\textbf{ડેટા કલેક્શન} & લેબલ્ડ ઉદાહરણો એકત્રિત કરવા & ટ્રેનિંગ મટેરિયલ પ્રદાન
કરવું \\
\textbf{પ્રીપ્રોસેસિંગ} & ડેટાને સાફ અને તૈયાર કરવું & ડેટા ગુણવત્તા સુધારવી \\
\textbf{ફીચર સિલેક્શન} & સંબંધિત એટ્રિબ્યુટ્સ પસંદ કરવા & જટિલતા ઘટાડવી \\
\textbf{મોડેલ ટ્રેનિંગ} & ટ્રેનિંગ ડેટામાંથી શીખવું & ક્લાસિફાયર બનાવવું \\
\textbf{ઇવેલ્યુએશન} & મોડેલ પરફોર્મન્સ ટેસ્ટ કરવું & ચોકસાઈ આકારવી \\
\textbf{ડિપ્લોયમેન્ટ} & નવી આગાહીઓ માટે ઉપયોગ & પ્રેક્ટિકલ એપ્લિકેશન \\
\end{longtable}
}

\textbf{વિગતવાર પ્રક્રિયા:}

\begin{enumerate}
\tightlist
\item
  \textbf{ડેટાસેટ તૈયાર કરવું} ઇનપુટ ફીચર્સ અને ક્લાસ લેબલ્સ સાથે
\item
  \textbf{ડેટા સ્પ્લિટ કરવું} ટ્રેનિંગ અને ટેસ્ટિંગ સેટ્સમાં
\item
  \textbf{ક્લાસિફાયર ટ્રેન કરવું} ટ્રેનિંગ ડેટાનો ઉપયોગ કરીને
\item
  \textbf{મોડેલ વેલિડેટ કરવું} ટેસ્ટ ડેટાનો ઉપયોગ કરીને
\item
  \textbf{પેરામીટર્સ ફાઇન-ટ્યુન કરવા} આદર્શ પરફોર્મન્સ માટે
\end{enumerate}

\end{solutionbox}
\begin{mnemonicbox}
``ડેટા પ્રેપેરેશન ફેસિલિટેટ્સ મોડેલ એક્સેલન્સ'' - ડેટા પ્રેપ, ફીચર
સિલેક્શન, મોડેલ ટ્રેનિંગ, ઇવેલ્યુએશન

\end{mnemonicbox}
\subsection*{પ્રશ્ન 4(બ) [4
ગુણ]}\label{uxaaauxab0uxab6uxaa8-4uxaac-4-uxa97uxaa3}

\textbf{લીનિયર રિલેશનશિપ ગણતરી}

\begin{solutionbox}

\textbf{આપેલ ડેટા:}

{\def\LTcaptype{none} % do not increment counter
\begin{longtable}[]{@{}ll@{}}
\toprule\noalign{}
કલાકો (X) & પરીક્ષા સ્કોર (Y) \\
\midrule\noalign{}
\endhead
\bottomrule\noalign{}
\endlastfoot
2 & 85 \\
3 & 80 \\
4 & 75 \\
5 & 70 \\
6 & 60 \\
\end{longtable}
}

\textbf{લીનિયર રિગ્રેશન ગણતરી:}

\textbf{સ્ટેપ 1: મીન્સ કેલ્ક્યુલેટ કરવા}

\begin{itemize}
\tightlist
\item
  X̄ = (2+3+4+5+6)/5 = 4
\item
  Ȳ = (85+80+75+70+60)/5 = 74
\end{itemize}

\textbf{સ્ટેપ 2: સ્લોપ (b) કેલ્ક્યુલેટ કરવું}

\begin{itemize}
\tightlist
\item
  ન્યુમેરેટર = Σ(X-X̄)(Y-Ȳ) = (2-4)(85-74) + (3-4)(80-74) + (4-4)(75-74) +
  (5-4)(70-74) + (6-4)(60-74)
\item
  = (-2)(11) + (-1)(6) + (0)(1) + (1)(-4) + (2)(-14) = -22 - 6 + 0 - 4 -
  28 = -60
\item
  ડિનોમિનેટર = Σ(X-X̄)^{2} = (-2)^{2} + (-1)^{2} + (0)^{2} + (1)^{2} + (2)^{2} = 4 + 1 + 0 +
  1 + 4 = 10
\item
  b = -60/10 = -6
\end{itemize}

\textbf{સ્ટેપ 3: ઇન્ટરસેપ્ટ (a) કેલ્ક્યુલેટ કરવું}

\begin{itemize}
\tightlist
\item
  a = Ȳ - b\timesX̄ = 74 - (-6)\times4 = 74 + 24 = 98
\end{itemize}

\textbf{લીનિયર ઇક્વેશન: Y = 98 - 6X}

\textbf{અર્થઘટન}: સ્માર્ટફોન ઉપયોગના દરેક વધારાના કલાક માટે, પરીક્ષા સ્કોર 6
પોઇન્ટ ઘટે છે.

\end{solutionbox}
\begin{mnemonicbox}
``મોર ફોન, લેસ સ્કોર'' - ફોનના ઉપયોગ અને ગ્રેડ્સ વચ્ચે નેગેટિવ
કોરિલેશન

\end{mnemonicbox}
\subsection*{પ્રશ્ન 4(ક) [7
ગુણ]}\label{uxaaauxab0uxab6uxaa8-4uxa95-7-uxa97uxaa3}

\textbf{વર્ગીકરણના પગલાંને વિગતવાર સમજાવો}

\begin{solutionbox}

ક્લાસિફિકેશન એ સુપરવાઇઝ્ડ લર્નિંગ પ્રક્રિયા છે જે ઇનપુટ ડેટાને પૂર્વનિર્ધારિત કેટેગરીઓ
અથવા ક્લાસોમાં સોંપે છે.

\textbf{વિગતવાર ક્લાસિફિકેશન સ્ટેપ્સ:}

\textbf{1. સમસ્યા વ્યાખ્યા}

\begin{itemize}
\tightlist
\item
  ક્લાસો અને ઉદ્દેશ્યો વ્યાખ્યાયિત કરવા
\item
  ઇનપુટ ફીચર્સ અને ટાર્ગેટ વેરિએબલ ઓળખવા
\item
  સફળતાના માપદંડો નક્કી કરવા
\end{itemize}

\textbf{2. ડેટા કલેક્શન અને તૈયારી}

\begin{verbatim}
flowchart LR
    A[રો ડેટા] {-{-} B[ડેટા ક્લીનિંગ]}
    B {-{-} C[મિસિંગ વેલ્યુઝ હેન્ડલ કરવું]}
    C {-{-} D[આઉટલાયર્સ રીમૂવ કરવા]}
    D {-{-} E[ફીચર એન્જિનિયરિંગ]}
    E {-{-} F[ડેટા સ્પ્લિટિંગ]}
\end{verbatim}

\textbf{3. ફીચર એન્જિનિયરિંગ}

\begin{itemize}
\tightlist
\item
  \textbf{ફીચર સિલેક્શન}: સંબંધિત એટ્રિબ્યુટ્સ પસંદ કરવા
\item
  \textbf{ફીચર એક્સ્ટ્રેક્શન}: નવા અર્થપૂર્ણ ફીચર્સ બનાવવા
\item
  \textbf{નોર્મલાઇઝેશન}: ફીચર્સને સમાન રેન્જમાં સ્કેલ કરવા
\end{itemize}

\textbf{4. મોડેલ સિલેક્શન અને ટ્રેનિંગ}

\textbf{ટેબલ: સામાન્ય ક્લાસિફિકેશન અલ્ગોરિધમ્સ}

{\def\LTcaptype{none} % do not increment counter
\begin{longtable}[]{@{}lll@{}}
\toprule\noalign{}
અલ્ગોરિધમ & શ્રેષ્ઠ માટે & ફાયદાઓ \\
\midrule\noalign{}
\endhead
\bottomrule\noalign{}
\endlastfoot
\textbf{ડિસિઝન ટ્રી} & ઇન્ટરપ્રિટેબલ રૂલ્સ & સમજવામાં સરળ \\
\textbf{SVM} & હાઇ-ડાયમેન્શનલ ડેટા & સારું જનરલાઇઝેશન \\
\textbf{ન્યુરલ નેટવર્ક્સ} & જટિલ પેટર્ન્સ & ઉચ્ચ ચોકસાઈ \\
\textbf{નાઇવ બેઝ} & ટેક્સ્ટ ક્લાસિફિકેશન & ઝડપી ટ્રેનિંગ \\
\end{longtable}
}

\textbf{5. મોડેલ ઇવેલ્યુએશન}

\begin{itemize}
\tightlist
\item
  \textbf{કન્ફ્યુઝન મેટ્રિક્સ}: વિગતવાર પરફોર્મન્સ એનાલિસિસ
\item
  \textbf{ક્રોસ-વેલિડેશન}: મજબૂત પરફોર્મન્સ અંદાજ
\item
  \textbf{મેટ્રિક્સ}: એક્યુરસી, પ્રિસિઝન, રિકોલ, F1-સ્કોર
\end{itemize}

\textbf{6. હાયપરપેરામીટર ટ્યુનિંગ}

\begin{itemize}
\tightlist
\item
  આદર્શ પેરામીટર્સ માટે ગ્રિડ સર્ચ
\item
  પેરામીટર સિલેક્શન માટે વેલિડેશન સેટ
\end{itemize}

\textbf{7. અંતિમ ઇવેલ્યુએશન અને ડિપ્લોયમેન્ટ}

\begin{itemize}
\tightlist
\item
  અદ્રશ્ય ડેટા પર ટેસ્ટ કરવું
\item
  પ્રોડક્શન ઉપયોગ માટે મોડેલ ડિપ્લોય કરવું
\item
  સમય જતાં પરફોર્મન્સ મોનિટર કરવું
\end{itemize}

\end{solutionbox}
\begin{mnemonicbox}
``પ્રોપર ડેટા મોડેલિંગ ઇવેલ્યુએટ્સ પર્ફોર્મન્સ થોરોલી'' -
પ્રોબ્લેમ ડેફિનિશન, ડેટા પ્રેપ, મોડેલિંગ, ઇવેલ્યુએશન, પર્ફોર્મન્સ ટેસ્ટિંગ, ટ્યુનિંગ

\end{mnemonicbox}
\subsection*{પ્રશ્ન 4(અ) OR [3
ગુણ]}\label{uxaaauxab0uxab6uxaa8-4uxa85-or-3-uxa97uxaa3}

\textbf{શું k વેલ્યુની પસંદગી KNN અલ્ગોરિધમના પરફોર્મન્સને પ્રભાવિત કરે છે? ટૂંકમાં
સમજાવો}

\begin{solutionbox}

હા, k વેલ્યુ ડિસિઝન બાઉન્ડરી અને મોડેલ જટિલતાને અસર કરીને KNN અલ્ગોરિધમના
પરફોર્મન્સને નોંધપાત્ર રીતે પ્રભાવિત કરે છે.

\textbf{ટેબલ: K વેલ્યુની અસર}

{\def\LTcaptype{none} % do not increment counter
\begin{longtable}[]{@{}lll@{}}
\toprule\noalign{}
K વેલ્યુ & અસર & પરફોર્મન્સ \\
\midrule\noalign{}
\endhead
\bottomrule\noalign{}
\endlastfoot
\textbf{નાનું K (k=1)} & નોઇઝ પ્રત્યે સંવેદનશીલ & હાઇ વેરિયન્સ, લો બાયસ \\
\textbf{મધ્યમ K} & સંતુલિત નિર્ણયો & આદર્શ પરફોર્મન્સ \\
\textbf{મોટું K} & સ્મૂથ બાઉન્ડરીઝ & લો વેરિયન્સ, હાઇ બાયસ \\
\end{longtable}
}

\textbf{અસર એનાલિસિસ:}

\begin{itemize}
\tightlist
\item
  \textbf{k=1}: ટ્રેનિંગ ડેટા પર ઓવરફિટ થઈ શકે, આઉટલાયર્સ પ્રત્યે સંવેદનશીલ
\item
  \textbf{આદર્શ k}: સામાન્ય રીતે વિષમ સંખ્યા, બાયસ-વેરિયન્સ ટ્રેડઓફને સંતુલિત કરે
\item
  \textbf{મોટું k}: અંડરફિટ થઈ શકે, સ્થાનિક પેટર્ન્સ ગુમાવે
\end{itemize}

\textbf{સિલેક્શન વ્યૂહરચના:}

\begin{itemize}
\tightlist
\item
  આદર્શ k શોધવા માટે ક્રોસ-વેલિડેશનનો ઉપયોગ
\item
  શરૂઆતના બિંદુ તરીકે k = \sqrtn ટ્રાય કરો
\item
  કોમ્પ્યુટેશનલ કોસ્ટ vs ચોકસાઈનો વિચાર કરો
\end{itemize}

\end{solutionbox}
\begin{mnemonicbox}
``સ્મોલ K વેરીઝ, લાર્જ K સ્મૂથ્સ'' - નાનું k વેરિયન્સ બનાવે,
મોટું k સ્મૂથ બાઉન્ડરીઝ બનાવે

\end{mnemonicbox}
\subsection*{પ્રશ્ન 4(બ) OR [4
ગુણ]}\label{uxaaauxab0uxab6uxaa8-4uxaac-or-4-uxa97uxaa3}

\textbf{SVM મોડેલમાં સપોર્ટ વેક્ટર્સને વ્યાખ્યાયિત કરો}

\begin{solutionbox}

સપોર્ટ વેક્ટર્સ એ મહત્વપૂર્ણ ડેટા પોઇન્ટ્સ છે જે સપોર્ટ વેક્ટર મશીન અલ્ગોરિધમમાં ડિસિઝન
બાઉન્ડરી (હાયપરપ્લેન)ની સૌથી નજીક આવેલા હોય છે.

\textbf{ટેબલ: સપોર્ટ વેક્ટર લક્ષણો}

{\def\LTcaptype{none} % do not increment counter
\begin{longtable}[]{@{}
  >{\raggedright\arraybackslash}p{(\linewidth - 4\tabcolsep) * \real{0.3158}}
  >{\raggedright\arraybackslash}p{(\linewidth - 4\tabcolsep) * \real{0.3158}}
  >{\raggedright\arraybackslash}p{(\linewidth - 4\tabcolsep) * \real{0.3684}}@{}}
\toprule\noalign{}
\begin{minipage}[b]{\linewidth}\raggedright
પાસું
\end{minipage} & \begin{minipage}[b]{\linewidth}\raggedright
વર્ણન
\end{minipage} & \begin{minipage}[b]{\linewidth}\raggedright
મહત્વ
\end{minipage} \\
\midrule\noalign{}
\endhead
\bottomrule\noalign{}
\endlastfoot
\textbf{સ્થાન} & હાયપરપ્લેનની સૌથી નજીકના પોઇન્ટ્સ & ડિસિઝન બાઉન્ડરી વ્યાખ્યાયિત
કરે \\
\textbf{અંતર} & બાઉન્ડરીથી સમાન અંતર & મેક્સિમમ માર્જિન \\
\textbf{ભૂમિકા} & હાયપરપ્લેનને સપોર્ટ કરે & આદર્શ વિભાજન નક્કી કરે \\
\textbf{સંવેદનશીલતા} & તેમને રીમૂવ કરવાથી મોડેલ બદલાય & મોડેલ સ્ટ્રક્ચર માટે
મહત્વપૂર્ણ \\
\end{longtable}
}

\textbf{મુખ્ય ગુણધર્મો:}

\begin{itemize}
\tightlist
\item
  \textbf{માર્જિન ડેફિનિશન}: સપોર્ટ વેક્ટર્સ ક્લાસો વચ્ચે મેક્સિમમ માર્જિન નક્કી કરે છે
\item
  \textbf{મોડેલ ડિપેન્ડન્સી}: માત્ર સપોર્ટ વેક્ટર્સ જ અંતિમ મોડેલને અસર કરે છે
\item
  \textbf{બાઉન્ડરી ફોર્મેશન}: આદર્શ વિભાજક હાયપરપ્લેન બનાવે છે
\end{itemize}

\textbf{ડાયાગ્રામ:}

\begin{verbatim}
      ક્લાસ A  |     |  ક્લાસ B
         o     |     |     x
           o   |     |   x
         o   O |     | X   x
           o   |     |   x  
         o     |     |     x
              
         સપોર્ટ વેક્ટર્સ: O અને X
         હાયપરપ્લેન: {-{-}{-}|{-}{-}{-}}
\end{verbatim}

\textbf{ગાણિતિક મહત્વ}: સપોર્ટ વેક્ટર્સ yi(w·xi + b) = 1 કન્સ્ટ્રેઇન્ટને સંતુષ્ટ કરે છે,
જ્યાં તેઓ માર્જિન બાઉન્ડરી પર બરાબર સ્થિત હોય છે.

\end{solutionbox}
\begin{mnemonicbox}
``સપોર્ટ વેક્ટર્સ સપોર્ટ ડિસિઝન્સ'' - આ વેક્ટર્સ ડિસિઝન
બાઉન્ડરીને સપોર્ટ કરે છે

\end{mnemonicbox}
\subsection*{પ્રશ્ન 4(ક) OR [7
ગુણ]}\label{uxaaauxab0uxab6uxaa8-4uxa95-or-7-uxa97uxaa3}

\textbf{લોજિસ્ટિક રિગ્રેશનને વિગતવાર સમજાવો}

\begin{solutionbox}

લોજિસ્ટિક રિગ્રેશન એ બાઇનરી ક્લાસિફિકેશન માટે વપરાતી સ્ટેટિસ્ટિકલ મેથડ છે જે
લોજિસ્ટિક ફંક્શનનો ઉપયોગ કરીને ક્લાસ મેમ્બરશિપની સંભાવનાને મોડેલ કરે છે.

\textbf{ગાણિતિક આધાર:}

\textbf{સિગ્મોઇડ ફંક્શન:} σ(z) = 1 / (1 + e\^{}(-z)) જ્યાં z = β_{0} + β_{1}x_{1} +
β_{2}x_{2} + \ldots{} + β_{n}x_{n}

\textbf{ટેબલ: લીનિયર vs લોજિસ્ટિક રિગ્રેશન}

{\def\LTcaptype{none} % do not increment counter
\begin{longtable}[]{@{}lll@{}}
\toprule\noalign{}
પાસું & લીનિયર રિગ્રેશન & લોજિસ્ટિક રિગ્રેશન \\
\midrule\noalign{}
\endhead
\bottomrule\noalign{}
\endlastfoot
\textbf{આઉટપુટ} & સતત મૂલ્યો & સંભાવનાઓ (0-1) \\
\textbf{ફંક્શન} & લીનિયર & સિગ્મોઇડ (S-કર્વ) \\
\textbf{હેતુ} & આગાહી & ક્લાસિફિકેશન \\
\textbf{એરર ફંક્શન} & મીન સ્ક્વેર્ડ એરર & લોગ-લાઇકલીહુડ \\
\end{longtable}
}

\textbf{મુખ્ય ઘટકો:}

\textbf{1. લોજિસ્ટિક ફંક્શન ગુણધર્મો:}

\begin{itemize}
\tightlist
\item
  \textbf{S-આકારનો કર્વ}: 0 અને 1 વચ્ચે સ્મૂથ ટ્રાન્ઝિશન
\item
  \textbf{એસિમ્પ્ટોટ્સ}: 0 અને 1ની નજીક પહોંચે પણ ક્યારેય પહોંચતું નથી
\item
  \textbf{મોનોટોનિક}: હંમેશા વધતું ફંક્શન
\end{itemize}

\textbf{2. મોડેલ ટ્રેનિંગ:}

\begin{itemize}
\tightlist
\item
  \textbf{મેક્સિમમ લાઇકલીહુડ એસ્ટિમેશન}: જોયેલા ડેટાની સંભાવના વધારતા પેરામીટર્સ
  શોધવા
\item
  \textbf{ગ્રેડિયન્ટ ડિસેન્ટ}: પુનરાવર્તક ઓપ્ટિમાઇઝેશન અલ્ગોરિધમ
\item
  \textbf{કોસ્ટ ફંક્શન}: લોગ-લોસ અથવા ક્રોસ-એન્ટ્રોપી
\end{itemize}

\textbf{3. નિર્ણય લેવું:}

\begin{itemize}
\tightlist
\item
  \textbf{થ્રેશોલ્ડ}: બાઇનરી ક્લાસિફિકેશન માટે સામાન્ય રીતે 0.5
\item
  \textbf{પ્રોબેબિલિટી આઉટપુટ}: P(y=1\textbar x) ક્લાસ સંભાવના આપે છે
\item
  \textbf{ડિસિઝન રૂલ}: P(y=1\textbar x) \textgreater{} 0.5 હોય તો પોઝિટિવ
  તરીકે ક્લાસિફાય કરવું
\end{itemize}

\textbf{ફાયદાઓ:}

\begin{itemize}
\tightlist
\item
  \textbf{પ્રોબેબિલિસ્ટિક આઉટપુટ}: આગાહીઓમાં વિશ્વાસ પ્રદાન કરે છે
\item
  \textbf{કોઈ ધારણાઓ નથી}: ઇન્ડિપેન્ડન્ટ વેરિએબલ્સના વિતરણ વિશે
\item
  \textbf{ઓછું ઓવરફિટિંગ}: જટિલ મોડેલ્સની તુલનામાં
\item
  \textbf{ઝડપી ટ્રેનિંગ}: કાર્યક્ષમ કોમ્પ્યુટેશન
\end{itemize}

\textbf{એપ્લિકેશન્સ:}

\begin{itemize}
\tightlist
\item
  મેડિકલ ડાયગ્નોસિસ
\item
  માર્કેટિંગ રિસ્પોન્સ આગાહી
\item
  ક્રેડિટ એપ્રૂવલ નિર્ણયો
\item
  ઇમેઇલ સ્પામ ડિટેક્શન
\end{itemize}

\textbf{કોડ ઉદાહરણ:}

\begin{verbatim}
from sklearn.linear\_model import LogisticRegression
model = LogisticRegression()
model.fit(X\_train, y\_train)
predictions = model.predict(X\_test)
probabilities = model.predict\_proba(X\_test)
\end{verbatim}

\end{solutionbox}
\begin{mnemonicbox}
``સિગ્મોઇડ સ્ક્વેશેસ ઇન્ફિનિટ ઇનપુટ'' - સિગ્મોઇડ ફંક્શન કોઈપણ
વાસ્તવિક સંખ્યાને સંભાવનામાં કન્વર્ટ કરે છે

\end{mnemonicbox}
\subsection*{પ્રશ્ન 5(અ) [3
ગુણ]}\label{uxaaauxab0uxab6uxaa8-5uxa85-3-uxa97uxaa3}

\textbf{Matplotlib python library પર ટૂંકી નોંધ લખો}

\begin{solutionbox}

Matplotlib એ ડેટા સાયન્સ અને મશીન લર્નિંગમાં સ્ટેટિક, એનિમેટેડ અને ઇન્ટરેક્ટિવ
વિઝ્યુઅલાઇઝેશન બનાવવા માટેની વ્યાપક Python લાઇબ્રેરી છે.

\textbf{ટેબલ: Matplotlib મુખ્ય ફીચર્સ}

{\def\LTcaptype{none} % do not increment counter
\begin{longtable}[]{@{}lll@{}}
\toprule\noalign{}
ફીચર & હેતુ & ઉદાહરણ \\
\midrule\noalign{}
\endhead
\bottomrule\noalign{}
\endlastfoot
\textbf{Pyplot} & MATLAB-જેવું પ્લોટિંગ ઇન્ટરફેસ & લાઇન પ્લોટ્સ, સ્કેટર પ્લોટ્સ \\
\textbf{Object-oriented} & એડવાન્સ્ડ કસ્ટમાઇઝેશન & ફિગર અને એક્સેસ ઓબ્જેક્ટ્સ \\
\textbf{મલ્ટિપલ ફોર્મેટ્સ} & વિવિધ ફોર્મેટમાં સેવ કરવું & PNG, PDF, SVG, EPS \\
\textbf{સબપ્લોટ્સ} & એક ફિગરમાં મલ્ટિપલ પ્લોટ્સ & ગ્રિડ એરેન્જમેન્ટ્સ \\
\end{longtable}
}

\textbf{સામાન્ય પ્લોટ પ્રકારો:}

\begin{itemize}
\tightlist
\item
  \textbf{લાઇન પ્લોટ}: સમય પર વલણો
\item
  \textbf{સ્કેટર પ્લોટ}: વેરિએબલ્સ વચ્ચે સંબંધ
\item
  \textbf{હિસ્ટોગ્રામ}: ડેટા વિતરણ
\item
  \textbf{બાર ચાર્ટ}: કેટેગોરિકલ કમ્પેરિઝન્સ
\item
  \textbf{બોક્સ પ્લોટ}: સ્ટેટિસ્ટિકલ સમરીઝ
\end{itemize}

\textbf{મૂળભૂત ઉપયોગ:}

\begin{verbatim}
import matplotlib.pyplot as plt
plt.plot(x, y)
plt.xlabel({X લેબલ})
plt.ylabel({Y લેબલ})
plt.title({પ્લોટ ટાઇટલ})
plt.show()
\end{verbatim}

\textbf{એપ્લિકેશન્સ}: ડેટા એક્સ્પ્લોરેશન, મોડેલ પરફોર્મન્સ વિઝ્યુઅલાઇઝેશન, પ્રેઝન્ટેશન
ગ્રાફિક્સ

\end{solutionbox}
\begin{mnemonicbox}
``Matplotlib મેક્સ પ્રિટી પ્લોટ્સ'' - ડેટા વિઝ્યુઅલાઇઝેશન માટે
આવશ્યક ટૂલ

\end{mnemonicbox}
\subsection*{પ્રશ્ન 5(બ) [4
ગુણ]}\label{uxaaauxab0uxab6uxaa8-5uxaac-4-uxa97uxaa3}

\textbf{દ્વિ-પરિમાણીય ડેટા માટે K-means ક્લસ્ટરિંગ}

\begin{solutionbox}

\textbf{આપેલ પોઇન્ટ્સ:}
\{(2,3),(3,3),(4,3),(5,3),(6,3),(7,3),(8,3),(25,20),(26,20),(27,20),(28,20),(29,20),(30,20)\}

\textbf{K-means અલ્ગોરિધમ સ્ટેપ્સ:}

\textbf{સ્ટેપ 1: સેન્ટ્રોઇડ્સ ઇનિશિયલાઇઝ કરવા}

\begin{itemize}
\tightlist
\item
  ક્લસ્ટર 1: (4, 3) - ડાબા જૂથમાંથી પસંદ કરેલું
\item
  ક્લસ્ટર 2: (27, 20) - જમણા જૂથમાંથી પસંદ કરેલું
\end{itemize}

\textbf{સ્ટેપ 2: નજીકના સેન્ટ્રોઇડને પોઇન્ટ્સ સોંપવા}

\textbf{ટેબલ: પોઇન્ટ એસાઇનમેન્ટ્સ}

{\def\LTcaptype{none} % do not increment counter
\begin{longtable}[]{@{}llll@{}}
\toprule\noalign{}
પોઇન્ટ & C1નું અંતર & C2નું અંતર & સોંપેલ ક્લસ્ટર \\
\midrule\noalign{}
\endhead
\bottomrule\noalign{}
\endlastfoot
(2,3) & 2.0 & 25.8 & ક્લસ્ટર 1 \\
(3,3) & 1.0 & 24.8 & ક્લસ્ટર 1 \\
(4,3) & 0.0 & 23.8 & ક્લસ્ટર 1 \\
(5,3) & 1.0 & 22.8 & ક્લસ્ટર 1 \\
(6,3) & 2.0 & 21.8 & ક્લસ્ટર 1 \\
(7,3) & 3.0 & 20.8 & ક્લસ્ટર 1 \\
(8,3) & 4.0 & 19.8 & ક્લસ્ટર 1 \\
(25,20) & 23.8 & 2.0 & ક્લસ્ટર 2 \\
(26,20) & 24.8 & 1.0 & ક્લસ્ટર 2 \\
(27,20) & 25.8 & 0.0 & ક્લસ્ટર 2 \\
(28,20) & 26.8 & 1.0 & ક્લસ્ટર 2 \\
(29,20) & 27.8 & 2.0 & ક્લસ્ટર 2 \\
(30,20) & 28.8 & 3.0 & ક્લસ્ટર 2 \\
\end{longtable}
}

\textbf{સ્ટેપ 3: સેન્ટ્રોઇડ્સ અપડેટ કરવા}

\begin{itemize}
\tightlist
\item
  નવું C1 = ((2+3+4+5+6+7+8)/7, (3+3+3+3+3+3+3)/7) = (5, 3)
\item
  નવું C2 = ((25+26+27+28+29+30)/6, (20+20+20+20+20+20)/6) = (27.5, 20)
\end{itemize}

\textbf{અંતિમ ક્લસ્ટર્સ:}

\begin{itemize}
\tightlist
\item
  \textbf{ક્લસ્ટર 1}: \{(2,3),(3,3),(4,3),(5,3),(6,3),(7,3),(8,3)\}
\item
  \textbf{ક્લસ્ટર 2}: \{(25,20),(26,20),(27,20),(28,20),(29,20),(30,20)\}
\end{itemize}

\end{solutionbox}
\begin{mnemonicbox}
``સેન્ટ્રોઇડ્સ એટ્રેક્ટ નિયરેસ્ટ નેબર્સ'' - પોઇન્ટ્સ નજીકના
સેન્ટ્રોઇડમાં જોડાય છે

\end{mnemonicbox}
\subsection*{પ્રશ્ન 5(ક) [7
ગુણ]}\label{uxaaauxab0uxab6uxaa8-5uxa95-7-uxa97uxaa3}

\textbf{Scikit-learn ના ફંક્શન્સ અને તેનો ઉપયોગ આપો: a. ડેટા પ્રીપ્રોસેસિંગ b.
મોડેલ સિલેક્શન c.~મોડેલ ઇવેલ્યુએશન અને મેટ્રિક્સ}

\begin{solutionbox}

Scikit-learn ડેટા પ્રીપ્રોસેસિંગથી લઈને મોડેલ ઇવેલ્યુએશન સુધીના મશીન લર્નિંગ વર્કફ્લો
માટે વ્યાપક સાધનો પ્રદાન કરે છે.

\textbf{a) ડેટા પ્રીપ્રોસેસિંગ ફંક્શન્સ:}

\textbf{ટેબલ: પ્રીપ્રોસેસિંગ ફંક્શન્સ}

{\def\LTcaptype{none} % do not increment counter
\begin{longtable}[]{@{}lll@{}}
\toprule\noalign{}
ફંક્શન & હેતુ & ઉદાહરણ ઉપયોગ \\
\midrule\noalign{}
\endhead
\bottomrule\noalign{}
\endlastfoot
\texttt{StandardScaler()} & ફીચર્સને નોર્મલાઇઝ કરવા & મીન દૂર કરવું, યુનિટ
વેરિયન્સ \\
\texttt{MinMaxScaler()} & [0,1] રેન્જમાં સ્કેલ કરવું & ફીચર સ્કેલિંગ \\
\texttt{LabelEncoder()} & કેટેગોરિકલ લેબલ્સ એન્કોડ કરવા & ટેક્સ્ટને નંબરમાં કન્વર્ટ
કરવું \\
\texttt{OneHotEncoder()} & ડમી વેરિએબલ્સ બનાવવા & કેટેગોરિકલ ફીચર્સ હેન્ડલ
કરવા \\
\texttt{train\_test\_split()} & ડેટાસેટ સ્પ્લિટ કરવું & ટ્રેનિંગ/ટેસ્ટિંગ વિભાજન \\
\end{longtable}
}

\textbf{કોડ ઉદાહરણ:}

\begin{verbatim}
from sklearn.preprocessing import StandardScaler
scaler = StandardScaler()
X\_scaled = scaler.fit\_transform(X)
\end{verbatim}

\textbf{b) મોડેલ સિલેક્શન ફંક્શન્સ:}

\textbf{ટેબલ: મોડેલ સિલેક્શન ટૂલ્સ}

{\def\LTcaptype{none} % do not increment counter
\begin{longtable}[]{@{}lll@{}}
\toprule\noalign{}
ફંક્શન & હેતુ & એપ્લિકેશન \\
\midrule\noalign{}
\endhead
\bottomrule\noalign{}
\endlastfoot
\texttt{GridSearchCV()} & હાયપરપેરામીટર ટ્યુનિંગ & આદર્શ પેરામીટર્સ શોધવા \\
\texttt{RandomizedSearchCV()} & રેન્ડમ પેરામીટર સર્ચ & ઝડપી પેરામીટર
ઓપ્ટિમાઇઝેશન \\
\texttt{cross\_val\_score()} & ક્રોસ-વેલિડેશન & મોડેલ પરફોર્મન્સ ઇવેલ્યુએશન \\
\texttt{StratifiedKFold()} & સ્ટ્રેટિફાઇડ સેમ્પલિંગ & સંતુલિત ક્રોસ-વેલિડેશન \\
\texttt{Pipeline()} & પ્રીપ્રોસેસિંગ અને મોડેલિંગ ભેગું કરવું & સ્ટ્રીમલાઇન્ડ વર્કફ્લો \\
\end{longtable}
}

\textbf{કોડ ઉદાહરણ:}

\begin{verbatim}
from sklearn.model\_selection import GridSearchCV
param\_grid = \{{C}: [0.1, 1, 10]\}
grid\_search = GridSearchCV(SVM(), param\_grid, cv=5)
grid\_search.fit(X\_train, y\_train)
\end{verbatim}

\textbf{c) મોડેલ ઇવેલ્યુએશન અને મેટ્રિક્સ ફંક્શન્સ:}

\textbf{ટેબલ: ઇવેલ્યુએશન મેટ્રિક્સ}

{\def\LTcaptype{none} % do not increment counter
\begin{longtable}[]{@{}lll@{}}
\toprule\noalign{}
ફંક્શન & હેતુ & વપરાશ કેસ \\
\midrule\noalign{}
\endhead
\bottomrule\noalign{}
\endlastfoot
\texttt{accuracy\_score()} & એકંદર ચોકસાઈ & સામાન્ય ક્લાસિફિકેશન \\
\texttt{precision\_score()} & પોઝિટિવ પ્રિડિક્શન ચોકસાઈ & ફોલ્સ પોઝિટિવ્સ
ઘટાડવા \\
\texttt{recall\_score()} & ટ્રુ પોઝિટિવ રેટ & ફોલ્સ નેગેટિવ્સ ઘટાડવા \\
\texttt{f1\_score()} & પ્રિસિઝન/રિકોલનું હાર્મોનિક મીન & સંતુલિત મેટ્રિક \\
\texttt{confusion\_matrix()} & વિગતવાર એરર એનાલિસિસ & ભૂલો સમજવી \\
\texttt{classification\_report()} & વ્યાપક મેટ્રિક્સ & સંપૂર્ણ મૂલ્યાંકન \\
\texttt{roc\_auc\_score()} & ROC કર્વ હેઠળનો વિસ્તાર & બાઇનરી ક્લાસિફિકેશન \\
\end{longtable}
}

\textbf{કોડ ઉદાહરણ:}

\begin{verbatim}
from sklearn.metrics import classification\_report
print(classification\_report(y\_true, y\_pred))
\end{verbatim}

\textbf{વર્કફ્લો ઇન્ટિગ્રેશન:}

\begin{itemize}
\tightlist
\item
  \textbf{પ્રીપ્રોસેસિંગ}: ડેટાને સાફ અને તૈયાર કરવું
\item
  \textbf{મોડેલ સિલેક્શન}: અલ્ગોરિધમ્સ પસંદ કરવા અને ટ્યુન કરવા
\item
  \textbf{ઇવેલ્યુએશન}: પરફોર્મન્સનું વ્યાપક આકારણી
\end{itemize}

\end{solutionbox}
\begin{mnemonicbox}
``પ્રીપ્રોસેસ, સિલેક્ટ, ઇવેલ્યુએટ'' - Scikit-learn માં સંપૂર્ણ
ML વર્કફ્લો

\end{mnemonicbox}
\subsection*{પ્રશ્ન 5(અ) OR [3
ગુણ]}\label{uxaaauxab0uxab6uxaa8-5uxa85-or-3-uxa97uxaa3}

\textbf{NumPy ના મુખ્ય ફીચર્સની યાદી બનાવો}

\begin{solutionbox}

NumPy (Numerical Python) Python માં વૈજ્ઞાનિક કોમ્પ્યુટિંગ માટેનું મૂળભૂત પેકેજ છે, જે
શક્તિશાળી એરે ઓપરેશન્સ અને ગાણિતિક ફંક્શન્સ પ્રદાન કરે છે.

\textbf{ટેબલ: NumPy ના મુખ્ય ફીચર્સ}

{\def\LTcaptype{none} % do not increment counter
\begin{longtable}[]{@{}lll@{}}
\toprule\noalign{}
ફીચર & વર્ણન & ફાયદો \\
\midrule\noalign{}
\endhead
\bottomrule\noalign{}
\endlastfoot
\textbf{N-dimensional Arrays} & કાર્યક્ષમ એરે ઓબ્જેક્ટ્સ & ઝડપી ગાણિતિક
ઓપરેશન્સ \\
\textbf{Broadcasting} & વિવિધ સાઇઝના એરે પર ઓપરેશન્સ & લવચીક કોમ્પ્યુટેશન્સ \\
\textbf{Linear Algebra} & મેટ્રિક્સ ઓપરેશન્સ, ડીકમ્પોઝિશન્સ & વૈજ્ઞાનિક
કોમ્પ્યુટિંગ \\
\textbf{Random Numbers} & રેન્ડમ સેમ્પલિંગ અને ડિસ્ટ્રિબ્યુશન્સ & સ્ટેટિસ્ટિકલ
સિમ્યુલેશન્સ \\
\textbf{Integration} & C/C++/Fortran સાથે કામ કરે છે & ઉચ્ચ પરફોર્મન્સ \\
\end{longtable}
}

\textbf{મુખ્ય ક્ષમતાઓ:}

\begin{itemize}
\tightlist
\item
  \textbf{ગાણિતિક ફંક્શન્સ}: ત્રિકોણમિતિ, લોગેરિધમિક, એક્સપોનેન્શિયલ
\item
  \textbf{એરે મેનિપ્યુલેશન}: રિશેપિંગ, સ્પ્લિટિંગ, જોઇનિંગ એરેઝ
\item
  \textbf{ઇન્ડેક્સિંગ}: એડવાન્સ્ડ સ્લાઇસિંગ અને બૂલિયન ઇન્ડેક્સિંગ
\item
  \textbf{મેમરી કાર્યક્ષમતા}: ઓપ્ટિમાઇઝ્ડ ડેટા સ્ટોરેજ
\end{itemize}

\textbf{એપ્લિકેશન્સ}: ડેટા એનાલિસિસ, મશીન લર્નિંગ, ઇમેજ પ્રોસેસિંગ, વૈજ્ઞાનિક સંશોધન

\end{solutionbox}
\begin{mnemonicbox}
``નંબર્સ નીડ NumPy's પાવર'' - ન્યુમેરિકલ કોમ્પ્યુટેશન્સ માટે
આવશ્યક

\end{mnemonicbox}
\subsection*{પ્રશ્ન 5(બ) OR [4
ગુણ]}\label{uxaaauxab0uxab6uxaa8-5uxaac-or-4-uxa97uxaa3}

\textbf{એક-પરિમાણીય ડેટા માટે K-means ક્લસ્ટરિંગ}

\begin{solutionbox}

\textbf{આપેલ ડેટાસેટ:} \{1,2,4,5,7,8,10,11,12,14,15,17\}

\textbf{3 ક્લસ્ટર્સ માટે K-means અલ્ગોરિધમ:}

\textbf{સ્ટેપ 1: સેન્ટ્રોઇડ્સ ઇનિશિયલાઇઝ કરવા}

\begin{itemize}
\tightlist
\item
  C1 = 3 (પ્રારંભિક મૂલ્યોની આસપાસ)
\item
  C2 = 9 (મધ્યમ મૂલ્યોની આસપાસ)
\item
  C3 = 15 (પછીના મૂલ્યોની આસપાસ)
\end{itemize}

\textbf{સ્ટેપ 2: નજીકના સેન્ટ્રોઇડને પોઇન્ટ્સ સોંપવા}

\textbf{ટેબલ: પોઇન્ટ એસાઇનમેન્ટ્સ (ઇટરેશન 1)}

{\def\LTcaptype{none} % do not increment counter
\begin{longtable}[]{@{}lllll@{}}
\toprule\noalign{}
પોઇન્ટ & C1નું અંતર & C2નું અંતર & C3નું અંતર & સોંપેલ ક્લસ્ટર \\
\midrule\noalign{}
\endhead
\bottomrule\noalign{}
\endlastfoot
1 & 2 & 8 & 14 & ક્લસ્ટર 1 \\
2 & 1 & 7 & 13 & ક્લસ્ટર 1 \\
4 & 1 & 5 & 11 & ક્લસ્ટર 1 \\
5 & 2 & 4 & 10 & ક્લસ્ટર 1 \\
7 & 4 & 2 & 8 & ક્લસ્ટર 2 \\
8 & 5 & 1 & 7 & ક્લસ્ટર 2 \\
10 & 7 & 1 & 5 & ક્લસ્ટર 2 \\
11 & 8 & 2 & 4 & ક્લસ્ટર 2 \\
12 & 9 & 3 & 3 & ક્લસ્ટર 2 \\
14 & 11 & 5 & 1 & ક્લસ્ટર 3 \\
15 & 12 & 6 & 0 & ક્લસ્ટર 3 \\
17 & 14 & 8 & 2 & ક્લસ્ટર 3 \\
\end{longtable}
}

\textbf{સ્ટેપ 3: સેન્ટ્રોઇડ્સ અપડેટ કરવા}

\begin{itemize}
\tightlist
\item
  નવું C1 = (1+2+4+5)/4 = 3
\item
  નવું C2 = (7+8+10+11+12)/5 = 9.6
\item
  નવું C3 = (14+15+17)/3 = 15.33
\end{itemize}

\textbf{અંતિમ ક્લસ્ટર્સ:}

\begin{itemize}
\tightlist
\item
  \textbf{ક્લસ્ટર 1}: \{1, 2, 4, 5\}
\item
  \textbf{ક્લસ્ટર 2}: \{7, 8, 10, 11, 12\}
\item
  \textbf{ક્લસ્ટર 3}: \{14, 15, 17\}
\end{itemize}

\end{solutionbox}
\begin{mnemonicbox}
``ગ્રુપ્સ ગેદર બાય ડિસ્ટન્સ'' - સમાન પોઇન્ટ્સ પ્રાકૃતિક ક્લસ્ટર્સ
બનાવે છે

\end{mnemonicbox}
\subsection*{પ્રશ્ન 5(ક) OR [7
ગુણ]}\label{uxaaauxab0uxab6uxaa8-5uxa95-or-7-uxa97uxaa3}

\textbf{Pandas library ના ફંક્શન્સ અને તેનો ઉપયોગ આપો: a. ડેટા પ્રીપ્રોસેસિંગ b.
ડેટા ઇન્સ્પેક્શન c.~ડેટા ક્લીનિંગ અને ટ્રાન્સફોર્મેશન}

\begin{solutionbox}

Pandas ડેટા મેનિપ્યુલેશન અને એનાલિસિસ માટેની શક્તિશાળી Python લાઇબ્રેરી છે, જે
ઉચ્ચ-સ્તરના ડેટા સ્ટ્રક્ચર્સ અને ઓપરેશન્સ પ્રદાન કરે છે.

\textbf{a) ડેટા પ્રીપ્રોસેસિંગ ફંક્શન્સ:}

\textbf{ટેબલ: પ્રીપ્રોસેસિંગ ફંક્શન્સ}

{\def\LTcaptype{none} % do not increment counter
\begin{longtable}[]{@{}lll@{}}
\toprule\noalign{}
ફંક્શન & હેતુ & ઉદાહરણ \\
\midrule\noalign{}
\endhead
\bottomrule\noalign{}
\endlastfoot
\texttt{read\_csv()} & CSV ફાઇલો લોડ કરવા &
\texttt{pd.read\_csv(\textquotesingle{}data.csv\textquotesingle{})} \\
\texttt{head()} & પ્રથમ n રોઝ જોવા & \texttt{df.head(10)} \\
\texttt{tail()} & છેલ્લા n રોઝ જોવા & \texttt{df.tail(5)} \\
\texttt{sample()} & રેન્ડમ સેમ્પલિંગ & \texttt{df.sample(100)} \\
\texttt{set\_index()} & કોલમને ઇન્ડેક્સ તરીકે સેટ કરવું &
\texttt{df.set\_index(\textquotesingle{}id\textquotesingle{})} \\
\end{longtable}
}

\textbf{b) ડેટા ઇન્સ્પેક્શન ફંક્શન્સ:}

\textbf{ટેબલ: ઇન્સ્પેક્શન ફંક્શન્સ}

{\def\LTcaptype{none} % do not increment counter
\begin{longtable}[]{@{}lll@{}}
\toprule\noalign{}
ફંક્શન & હેતુ & પ્રદાન કરેલી માહિતી \\
\midrule\noalign{}
\endhead
\bottomrule\noalign{}
\endlastfoot
\texttt{info()} & ડેટાસેટ ઓવરવ્યુ & ડેટા ટાઇપ્સ, મેમરી વપરાશ \\
\texttt{describe()} & સ્ટેટિસ્ટિકલ સમરી & મીન, std, min, max \\
\texttt{shape} & ડેટાસેટ ડાયમેન્શન્સ & (રોઝ, કોલમ્સ) \\
\texttt{dtypes} & ડેટા ટાઇપ્સ & કોલમ ડેટા ટાઇપ્સ \\
\texttt{isnull()} & મિસિંગ વેલ્યુઝ & નલ્સ માટે બૂલિયન માસ્ક \\
\texttt{value\_counts()} & યુનિક વેલ્યુઝ કાઉન્ટ કરવા & ફ્રીક્વન્સી ડિસ્ટ્રિબ્યુશન \\
\texttt{corr()} & કોરિલેશન મેટ્રિક્સ & ફીચર રિલેશનશિપ્સ \\
\end{longtable}
}

\textbf{કોડ ઉદાહરણ:}

\begin{verbatim}
\# ડેટા ઇન્સ્પેક્શન
print(df.info())
print(df.describe())
print(df.isnull().sum())
\end{verbatim}

\textbf{c) ડેટા ક્લીનિંગ અને ટ્રાન્સફોર્મેશન ફંક્શન્સ:}

\textbf{ટેબલ: ક્લીનિંગ ફંક્શન્સ}

{\def\LTcaptype{none} % do not increment counter
\begin{longtable}[]{@{}lll@{}}
\toprule\noalign{}
ફંક્શન & હેતુ & વપરાશ \\
\midrule\noalign{}
\endhead
\bottomrule\noalign{}
\endlastfoot
\texttt{dropna()} & મિસિંગ વેલ્યુઝ રીમૂવ કરવા & \texttt{df.dropna()} \\
\texttt{fillna()} & મિસિંગ વેલ્યુઝ ભરવા & \texttt{df.fillna(0)} \\
\texttt{drop\_duplicates()} & ડુપ્લિકેટ રોઝ રીમૂવ કરવા &
\texttt{df.drop\_duplicates()} \\
\texttt{replace()} & વેલ્યુઝ રિપ્લેસ કરવા &
\texttt{df.replace(\textquotesingle{}old\textquotesingle{},\ \textquotesingle{}new\textquotesingle{})} \\
\texttt{astype()} & ડેટા ટાઇપ્સ બદલવા &
\texttt{df[\textquotesingle{}col\textquotesingle{}].astype(\textquotesingle{}int\textquotesingle{})} \\
\texttt{apply()} & ડેટા પર ફંક્શન એપ્લાય કરવું &
\texttt{df.apply(lambda\ x:\ x*2)} \\
\texttt{groupby()} & ડેટા ગ્રુપ કરવું &
\texttt{df.groupby(\textquotesingle{}category\textquotesingle{})} \\
\texttt{merge()} & ડેટાસેટ્સ જોઇન કરવા & \texttt{pd.merge(df1,\ df2)} \\
\texttt{pivot()} & ડેટા રિશેપ કરવું &
\texttt{df.pivot(columns=\textquotesingle{}col\textquotesingle{})} \\
\end{longtable}
}

\textbf{એડવાન્સ્ડ ઓપરેશન્સ:}

\begin{itemize}
\tightlist
\item
  \textbf{સ્ટ્રિંગ ઓપરેશન્સ}: \texttt{str.contains()}, \texttt{str.replace()}
\item
  \textbf{તારીખ ઓપરેશન્સ}: \texttt{to\_datetime()}, \texttt{dt.year}
\item
  \textbf{કેટેગોરિકલ ડેટા}: \texttt{pd.Categorical()}
\end{itemize}

\textbf{વર્કફ્લો ઉદાહરણ:}

\begin{verbatim}
\# સંપૂર્ણ પ્રીપ્રોસેસિંગ પાઇપલાઇન
df = pd.read\_csv({data.csv})
df = df.dropna()
df[{category}] = df[{category}].astype({category})
df\_grouped = df.groupby({type}).mean()
\end{verbatim}

\textbf{ફાયદાઓ:}

\begin{itemize}
\tightlist
\item
  \textbf{સહજ સિન્ટેક્સ}: શીખવા અને વાપરવામાં સરળ
\item
  \textbf{પરફોર્મન્સ}: મોટા ડેટાસેટ્સ માટે ઓપ્ટિમાઇઝ્ડ
\item
  \textbf{ઇન્ટિગ્રેશન}: NumPy, Matplotlib સાથે સારી રીતે કામ કરે છે
\item
  \textbf{લવચીકતા}: વિવિધ ડેટા ફોર્મેટ્સ હેન્ડલ કરે છે
\end{itemize}

\end{solutionbox}
\begin{mnemonicbox}
``Pandas પ્રોસેસેસ ડેટા પર્ફેક્ટલી'' - વ્યાપક ડેટા મેનિપ્યુલેશન
ટૂલ

\end{mnemonicbox}

\end{document}
