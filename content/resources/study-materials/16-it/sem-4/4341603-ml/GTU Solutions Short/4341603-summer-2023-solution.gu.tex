\documentclass{article}

% content/resources/templates/preamble.tex
\usepackage[margin=0.6in]{geometry}
\author{Milav Dabgar}
\usepackage{amsmath,amssymb,amsthm}
\usepackage{booktabs}
\usepackage{multirow}
\usepackage{xcolor}
\usepackage{tcolorbox}
\tcbuselibrary{breakable,skins}
\usepackage[colorlinks=true,linkcolor=blue]{hyperref}
\usepackage{titlesec}
\usepackage{enumitem}
\usepackage{tikz}
\usepackage{pgfplots}
\usepackage{circuitikz}
\usepackage[version=4]{mhchem}
\usepackage{longtable}
\usepackage{array}
\usepackage{float}
\usepackage{caption}
\usepackage{listings}

\lstset{
  basicstyle=\small\ttfamily,
  breaklines=true,
  breakatwhitespace=false,
  postbreak=\mbox{\textcolor{red}{$\hookrightarrow$}\space},
  float=false,
  numbers=left,
  numberstyle=\tiny\color{gray},
  numbersep=10pt,
  xleftmargin=2em,
  keywordstyle=\color{blue},
  commentstyle=\color{green!60!black},
  stringstyle=\color{purple},
  backgroundcolor=\color{gray!5},
  showstringspaces=false,
  tabsize=2,
  captionpos=b,
  keepspaces=true,
  columns=flexible
}

\pgfplotsset{compat=1.18}
\usetikzlibrary{shapes,arrows,positioning,calc,patterns,decorations.pathmorphing,decorations.markings,arrows.meta}

% Color scheme
\definecolor{headcolor}{RGB}{0,102,204}
\definecolor{keycolor}{RGB}{220,20,60}
\definecolor{solutioncolor}{RGB}{34,139,34}
\definecolor{mnemoniccolor}{RGB}{148,0,211}
\definecolor{codecolor}{RGB}{0,0,100}

% Spacing
\setlength{\parskip}{3pt}
\setlist[itemize]{nosep}
\setlist[enumerate]{nosep}

% Title formatting
\titleformat{\section}{\Large\bfseries\color{headcolor}}{\thesection}{1em}{}
\titleformat{\subsection}{\large\bfseries\color{headcolor}}{\thesubsection}{1em}{}

% Pandoc tightlist compatibility
\providecommand{\tightlist}{%
  \setlength{\itemsep}{0pt}\setlength{\parskip}{0pt}}

% Pandoc longtable compatibility
\newcounter{none}
\def\thenone{}


% content/resources/templates/gujarati-boxes.tex
\usepackage{fontspec}
\usepackage{polyglossia}

% Set Gujarati as main language (document is primarily in Gujarati)
% Note: gloss-gujarati.ldf doesn't exist in polyglossia, but it will use hyphenation patterns
\setdefaultlanguage{gujarati}
\setotherlanguage{english}

% Configure Gujarati font properly
% Use Language=Default to prevent polyglossia from trying to add language-specific features
% that don't exist for Gujarati, which causes "empty feature" warnings
\newfontfamily\gujaratifont[Script=Gujarati,AutoFakeBold=2.5,AutoFakeSlant=0.3]{Noto Sans Gujarati}
\setmainfont[Script=Gujarati,AutoFakeBold=2.5,AutoFakeSlant=0.3]{Noto Sans Gujarati}
% Use Noto Sans Gujarati for monospace to support Gujarati in text
\setmonofont[Scale=0.9]{Noto Sans Gujarati}

% Configure English to use the same font
\newfontfamily\englishfont[Script=Gujarati,AutoFakeBold=2.5,AutoFakeSlant=0.3]{Noto Sans Gujarati}

% Translations for polyglossia
\gappto\captionsgujarati{
  \renewcommand{\tablename}{કોષ્ટક}
  \renewcommand{\figurename}{આકૃતિ}
}

% Helper for TikZ nodes to ensure Gujarati font
\newcommand{\gu}[1]{{\gujaratifont #1}}

% Custom environments
\newtcolorbox{solutionbox}{
    breakable,
    enhanced,
    colback=solutioncolor!5!white,
    colframe=solutioncolor!75!black,
    fonttitle=\bfseries,
    title=જવાબ
}

\newtcolorbox{solutionboxnobreak}{
 colback=solutioncolor!5!white,
 colframe=solutioncolor!75!black,
 fonttitle=\bfseries,
 title=જવાબ
}

\newtcolorbox{keyformula}{
 breakable,
 enhanced,
 colback=keycolor!5!white,
 colframe=keycolor!75!black,
 fonttitle=\bfseries,
 title=રાસાયણિક સમીકરણ/સૂત્ર
}

\newtcolorbox{mnemonicbox}{
 breakable,
 enhanced,
 colback=mnemoniccolor!5!white,
 colframe=mnemoniccolor!75!black,
 fonttitle=\bfseries,
 title=મેમરી ટ્રીક
}


% Custom commands for GTU solutions
% This file defines semantic commands for consistent formatting

% Question command with automatic formatting
\newcommand{\question}[2]{%
  \section*{Question #1}%
  \textbf{#2}%
}

% OR question variant
\newcommand{\questionor}[2]{%
  \section*{Question #1 OR}%
  \textbf{#2}%
}

% Proper table environment with caption
\newenvironment{answertable}[1]{%
  \begin{table}[htbp]
  \centering
  \caption{#1}
}{%
  \end{table}
}

% Proper figure environment for diagrams
\newenvironment{answerdiagram}[1]{%
  \begin{figure}[htbp]
  \centering
  \caption{#1}
}{%
  \end{figure}
}

% Semantic markup for key terms
\newcommand{\keyword}[1]{\textbf{#1}}
\newcommand{\code}[1]{\texttt{#1}}
\newcommand{\classname}[1]{\texttt{#1}}
\newcommand{\methodname}[1]{\texttt{#1}}

% Proper quotation marks
\newcommand{\mnemonic}[1]{``#1''}


\title{Fundamentals of Machine Learning (4341603) - Summer 2023 Solution}
\date{July 18, 2023}

\begin{document}
\maketitle

\questionmarks{1(a)}{3}{હ્યુમન લર્નિંગને વ્યાખ્યાયિત કરો. હ્યુમન લર્નિંગના પ્રકારોની યાદી બનાવો.}

\begin{solutionbox}
હ્યુમન લર્નિંગ એ પ્રક્રિયા છે જેના દ્વારા માનવીઓ અનુભવ, અભ્યાસ અથવા સૂચનાઓ દ્વારા નવા જ્ઞાન, કૌશલ્યો, વર્તન મેળવે છે અથવા હાલનાઓમાં ફેરફાર કરે છે.

\begin{center}
\captionof{table}{હ્યુમન લર્નિંગના પ્રકારો}
\begin{tabulary}{\linewidth}{|L|L|}
\hline
\textbf{પ્રકાર} & \textbf{વર્ણન} \\ \hline
\textbf{સુપરવાઇઝ્ડ લર્નિંગ} & શિક્ષક/માર્ગદર્શકની મદદથી શીખવું \\ \hline
\textbf{અનસુપરવાઇઝ્ડ લર્નિંગ} & બાહ્ય માર્ગદર્શન વિના સ્વ-નિર્દેશિત શીખવું \\ \hline
\textbf{રિઇનફોર્સમેન્ટ લર્નિંગ} & ફીડબેક સાથે પ્રયાસ અને ભૂલ દ્વારા શીખવું \\ \hline
\end{tabulary}
\end{center}

\begin{mnemonicbox}SUR - Supervised, Unsupervised, Reinforcement\end{mnemonicbox}
\end{solutionbox}

\questionmarks{1(b)}{4}{ક્વાલિટેટિવ ડેટા અને ક્વોન્ટિટેટિવ ડેટા વચ્ચે તફાવત આપો.}

\begin{solutionbox}
\begin{center}
\captionof{table}{ક્વાલિટેટિવ vs ક્વોન્ટિટેટિવ ડેટા}
\begin{tabulary}{\linewidth}{|L|L|L|}
\hline
\textbf{લક્ષણ} & \textbf{ક્વાલિટેટિવ ડેટા} & \textbf{ક્વોન્ટિટેટિવ ડેટા} \\ \hline
\textbf{પ્રકૃતિ} & વર્ણનાત્મક, કેટેગોરિકલ & સંખ્યાત્મક, માપી શકાય તેવું \\ \hline
\textbf{વિશ્લેષણ} & વ્યક્તિગત અર્થઘટન & આંકડાકીય વિશ્લેષણ \\ \hline
\textbf{ઉદાહરણો} & રંગો, નામો, લિંગ & ઊંચાઈ, વજન, ઉંમર \\ \hline
\textbf{પ્રતિનિધિત્વ} & શબ્દો, કેટેગરીઓ & સંખ્યાઓ, ગ્રાફ્સ \\ \hline
\end{tabulary}
\end{center}

\begin{mnemonicbox}QUAN-Numbers, QUAL-Words\end{mnemonicbox}
\end{solutionbox}

\questionmarks{1(c)}{7}{મશીન લર્નિંગના વિવિધ પ્રકારોની સરખામણી કરો.}

\begin{solutionbox}
\begin{center}
\captionof{table}{મશીન લર્નિંગના પ્રકારોની સરખામણી}
\begin{tabulary}{\linewidth}{|L|L|L|L|}
\hline
\textbf{પ્રકાર} & \textbf{ટ્રેનિંગ ડેટા} & \textbf{ધ્યેય} & \textbf{ઉદાહરણો} \\ \hline
\textbf{સુપરવાઇઝ્ડ} & લેબલ્ડ ડેટા & પરિણામોની આગાહી & ક્લાસિફિકેશન, રિગ્રેશન \\ \hline
\textbf{અનસુપરવાઇઝ્ડ} & અનલેબલ્ડ ડેટા & પેટર્ન શોધવા & ક્લસ્ટરિંગ, એસોસિએશન \\ \hline
\textbf{રિઇનફોર્સમેન્ટ} & રિવોર્ડ/પેનલ્ટી & રિવોર્ડ મેક્સિમાઇઝ કરવા & ગેમિંગ, રોબોટિક્સ \\ \hline
\end{tabulary}
\end{center}

\textbf{મુખ્ય તફાવતો:}
\begin{itemize}
    \item \textbf{સુપરવાઇઝ્ડ}: ટ્રેનિંગ માટે ઇનપુટ-આઉટપુટ જોડીનો ઉપયોગ કરે છે
    \item \textbf{અનસુપરવાઇઝ્ડ}: ડેટામાં છુપાયેલા પેટર્ન શોધે છે
    \item \textbf{રિઇનફોર્સમેન્ટ}: પર્યાવરણ સાથે ક્રિયાપ્રતિક્રિયા દ્વારા શીખે છે
\end{itemize}

\begin{mnemonicbox}SUR-LAP: Supervised-Labeled, Unsupervised-Reveal, Reinforcement-Action\end{mnemonicbox}
\end{solutionbox}

\questionmarks{1(c OR)}{7}{મશીન લર્નિંગ વ્યાખ્યાયિત કરો. મશીન લર્નિંગની કોઈપણ ચાર એપ્લિકેશનને ટૂંકમાં સમજાવો.}

\begin{solutionbox}
મશીન લર્નિંગ આર્ટિફિશિયલ ઇન્ટેલિજન્સનો ઉપવિભાગ છે જે કમ્પ્યુટરોને સ્પષ્ટ પ્રોગ્રામિંગ વિના ડેટામાંથી શીખવા અને નિર્ણયો લેવા સક્ષમ બનાવે છે.

\begin{center}
\captionof{table}{ચાર એપ્લિકેશનો}
\begin{tabulary}{\linewidth}{|L|L|}
\hline
\textbf{એપ્લિકેશન} & \textbf{વર્ણન} \\ \hline
\textbf{ઈમેઇલ સ્પામ ડિટેક્શન} & ઈમેઇલને સ્પામ અથવા વૈધ તરીકે વર્ગીકૃત કરે છે \\ \hline
\textbf{ઇમેજ રેકગ્નિશન} & ફોટોમાં ઓબ્જેક્ટ્સ ઓળખે છે \\ \hline
\textbf{રેકમેન્ડેશન સિસ્ટમ} & યુઝર્સને પ્રોડક્ટ્સ/કન્ટેન્ટ સૂચવે છે \\ \hline
\textbf{મેડિકલ ડાયગ્નોસિસ} & રોગોની શોધમાં ડૉક્ટરોની મદદ કરે છે \\ \hline
\end{tabulary}
\end{center}

\begin{mnemonicbox}SIRM - Spam, Image, Recommendation, Medical\end{mnemonicbox}
\end{solutionbox}

\questionmarks{2(a)}{3}{નીચેના ઉદાહરણોનો યોગ્ય ડેટા પ્રકાર જણાવો.}

\begin{solutionbox}
\begin{center}
\captionof{table}{ડેટા પ્રકાર વર્ગીકરણ}
\begin{tabulary}{\linewidth}{|L|L|}
\hline
\textbf{ઉદાહરણ} & \textbf{ડેટા પ્રકાર} \\ \hline
\textbf{વિદ્યાર્થીઓની રાષ્ટ્રીયતા} & કેટેગોરિકલ (નોમિનલ) \\ \hline
\textbf{વિદ્યાર્થીઓની શિક્ષણ સ્થિતિ} & કેટેગોરિકલ (ઓર્ડિનલ) \\ \hline
\textbf{વિદ્યાર્થીઓની ઊંચાઈ} & ન્યુમેરિકલ (કન્ટિન્યુઅસ) \\ \hline
\end{tabulary}
\end{center}

\begin{mnemonicbox}NCN - Nominal, Categorical, Numerical\end{mnemonicbox}
\end{solutionbox}

\questionmarks{2(b)}{4}{ડેટા પ્રી-પ્રોસેસિંગ ટૂંકમાં સમજાવો.}

\begin{solutionbox}
ડેટા પ્રી-પ્રોસેસિંગ એ મશીન લર્નિંગ અલ્ગોરિધમ માટે કાચા ડેટાને તૈયાર કરવાની તકનીક છે.

\begin{center}
\captionof{table}{મુખ્ય સ્ટેપ્સ}
\begin{tabulary}{\linewidth}{|L|L|}
\hline
\textbf{સ્ટેપ} & \textbf{હેતુ} \\ \hline
\textbf{ડેટા ક્લીનિંગ} & ભૂલો અને અસંગતતાઓ દૂર કરવી \\ \hline
\textbf{ડેટા ઇન્ટીગ્રેશન} & બહુવિધ સ્ત્રોતોમાંથી ડેટાને જોડવો \\ \hline
\textbf{ડેટા ટ્રાન્સફોર્મેશન} & ડેટાને યોગ્ય ફોર્મેટમાં બદલવો \\ \hline
\textbf{ડેટા રિડક્શન} & માહિતી જાળવીને ડેટાનું કદ ઘટાડવું \\ \hline
\end{tabulary}
\end{center}

\begin{mnemonicbox}CITR - Clean, Integrate, Transform, Reduce\end{mnemonicbox}
\end{solutionbox}

\questionmarks{2(c)}{7}{K-ફોલ્ડ ક્રોસ વેલિડેશન વિગતવાર સમજાવો.}

\begin{solutionbox}
K-ફોલ્ડ ક્રોસ વેલિડેશન એ ડેટાને K સમાન ભાગોમાં વિભાજિત કરીને મોડેલ પરફોર્મન્સ મૂલ્યાંકનની તકનીક છે.

\begin{center}
\begin{tikzpicture}[node distance=1.5cm, auto]
    \node [gtu block] (Data) {મૂળ ડેટાસેટ};
    \node [gtu block, below=1.2cm of Data] (Split) {K ફોલ્ડમાં વિભાજિત કરો};
    \node [gtu block, below left=1.8cm and -0.8cm of Split] (Train) {K-1 ફોલ્ડ ટ્રેનિંગ\\માટે વાપરો};
    \node [gtu block, below right=1.8cm and -0.8cm of Split] (Test) {1 ફોલ્ડ ટેસ્ટિંગ\\માટે વાપરો};
    \node [gtu state, below=2.5cm of Split] (Loop) {K વખત પુનરાવર્તન કરો};
    \node [gtu block, below=1.5cm of Loop] (Average) {પરિણામોની\\સરેરાશ કાઢો};

    \path [gtu arrow] (Data) -- (Split);
    \path [gtu arrow] (Split) -- (Train);
    \path [gtu arrow] (Split) -- (Test);
    \path [gtu arrow] (Train) -- (Loop);
    \path [gtu arrow] (Test) -- (Loop);
    \path [gtu arrow] (Loop) -- (Average);
\end{tikzpicture}
\captionof{figure}{K-Fold ક્રોસ વેલિડેશન પ્રક્રિયા}
\end{center}

\textbf{સ્ટેપ્સ:}
\begin{itemize}
    \item \textbf{વિભાજન}: ડેટાસેટને K સમાન ભાગોમાં વહેંચો
    \item \textbf{ટ્રેનિંગ}: K-1 ફોલ્ડનો ઉપયોગ ટ્રેનિંગ માટે કરો
    \item \textbf{ટેસ્ટ}: બાકીના ફોલ્ડનો ઉપયોગ વેલિડેશન માટે કરો
    \item \textbf{પુનરાવર્તન}: K વખત પ્રક્રિયા કરો
    \item \textbf{સરેરાશ}: સરેરાશ પરફોર્મન્સ કાઢો
\end{itemize}

\textbf{ફાયદા:}
\begin{itemize}
    \item ઓવરફિટિંગ ઘટાડે છે
    \item મર્યાદિત ડેટાનો બહેતર ઉપયોગ
    \item વધુ વિશ્વસનીય પરફોર્મન્સ અંદાજ
\end{itemize}

\begin{mnemonicbox}DTRA - Divide, Train, Repeat, Average\end{mnemonicbox}
\end{solutionbox}

\questionmarks{2(a OR)}{3}{નીચેના શબ્દો વ્યાખ્યાયિત કરો: i) Mean, ii) Outliers, iii) Interquartile range}

\begin{solutionbox}
\begin{center}
\captionof{table}{આંકડાકીય શબ્દો}
\begin{tabulary}{\linewidth}{|L|L|}
\hline
\textbf{શબ્દ} & \textbf{વ્યાખ્યા} \\ \hline
\textbf{Mean} & ડેટાસેટમાં બધી વેલ્યુઝની સરેરાશ \\ \hline
\textbf{Outliers} & અન્ય ડેટા પોઇન્ટ્સથી નોંધપાત્ર રીતે અલગ ડેટા પોઇન્ટ્સ \\ \hline
\textbf{Interquartile Range} & 75મા અને 25મા પર્સેન્ટાઇલ વચ્ચેનો તફાવત \\ \hline
\end{tabulary}
\end{center}

\begin{mnemonicbox}MOI - Mean, Outliers, Interquartile\end{mnemonicbox}
\end{solutionbox}

\questionmarks{2(b OR)}{4}{કન્ફ્યુશન મેટ્રિક્સની રચના સમજાવો.}

\begin{solutionbox}
\textbf{કન્ફ્યુશન મેટ્રિક્સ સ્ટ્રક્ચર:}

\begin{center}
\captionof{table}{કન્ફ્યુશન મેટ્રિક્સ}
\begin{tabulary}{\linewidth}{|L|L|L|}
\hline
 & \textbf{આગાહી પોઝિટિવ} & \textbf{આગાહી નેગેટિવ} \\ \hline
\textbf{વાસ્તવિક પોઝિટિવ} & True Positive (TP) & False Negative (FN) \\ \hline
\textbf{વાસ્તવિક નેગેટિવ} & False Positive (FP) & True Negative (TN) \\ \hline
\end{tabulary}
\end{center}

\textbf{ઘટકો:}
\begin{itemize}
    \item \textbf{TP}: સાચી રીતે આગાહી કરેલા પોઝિટિવ કેસો
    \item \textbf{TN}: સાચી રીતે આગાહી કરેલા નેગેટિવ કેસો
    \item \textbf{FP}: ખોટી રીતે પોઝિટિવ તરીકે આગાહી કરેલા
    \item \textbf{FN}: ખોટી રીતે નેગેટિવ તરીકે આગાહી કરેલા
\end{itemize}

\begin{mnemonicbox}TTFF - True True, False False\end{mnemonicbox}
\end{solutionbox}

\questionmarks{2(c OR)}{7}{ફીચર સબસેટની પસંદગી પર ટૂંકી નોંધ લખો.}

\begin{solutionbox}
ફીચર સબસેટ સિલેક્શન એ મૂળ ફીચર સેટમાંથી સંબંધિત ફીચર્સ પસંદ કરવાની પ્રક્રિયા છે.

\begin{center}
\captionof{table}{મેથડ્સ}
\begin{tabulary}{\linewidth}{|L|L|}
\hline
\textbf{મેથડ} & \textbf{વર્ણન} \\ \hline
\textbf{ફિલ્ટર મેથડ્સ} & ફીચર્સ રેન્ક કરવા આંકડાકીય માપદંડોનો ઉપયોગ \\ \hline
\textbf{રેપર મેથડ્સ} & ફીચર સબસેટ્સ મૂલ્યાંકન માટે ML અલ્ગોરિધમનો ઉપયોગ \\ \hline
\textbf{એમ્બેડેડ મેથડ્સ} & મોડેલ ટ્રેનિંગ દરમિયાન ફીચર સિલેક્શન \\ \hline
\end{tabulary}
\end{center}

\textbf{ફાયદા:}
\begin{itemize}
    \item **ઘટાડેલી જટિલતા**: ઓછા ફીચર્સ, સરળ મોડેલ્સ
    \item **સુધારેલ પરફોર્મન્સ**: નોઇઝ અને અપ્રસ્તુત ફીચર્સ દૂર કરે છે
    \item **ઝડપી ટ્રેનિંગ**: ઓછો કમ્પ્યુટેશનલ ઓવરહેડ
\end{itemize}

\textbf{લોકપ્રિય તકનીકો:}
\begin{itemize}
    \item Chi-square ટેસ્ટ
    \item Recursive Feature Elimination
    \item LASSO રેગ્યુલરાઇઝેશન
\end{itemize}

\begin{mnemonicbox}FWE - Filter, Wrapper, Embedded\end{mnemonicbox}
\end{solutionbox}

\questionmarks{3(a)}{3}{પ્રેડિક્ટિવ મોડેલ અને ડીસ્ક્રિપ્ટિવ મોડેલ વચ્ચેનો તફાવત આપો.}

\begin{solutionbox}
\begin{center}
\captionof{table}{પ્રેડિક્ટિવ vs ડીસ્ક્રિપ્ટિવ મોડેલ}
\begin{tabulary}{\linewidth}{|L|L|L|}
\hline
\textbf{લક્ષણ} & \textbf{પ્રેડિક્ટિવ મોડેલ} & \textbf{ડીસ્ક્રિપ્ટિવ મોડેલ} \\ \hline
\textbf{હેતુ} & ભાવિ પરિણામોની આગાહી & વર્તમાન પેટર્ન સમજવા \\ \hline
\textbf{આઉટપુટ} & આગાહીઓ/વર્ગીકરણ & અંતર્દૃષ્ટિ/સારાંશ \\ \hline
\textbf{ઉદાહરણો} & રિગ્રેશન, ક્લાસિફિકેશન & ક્લસ્ટરિંગ, એસોસિએશન રૂલ્સ \\ \hline
\end{tabulary}
\end{center}

\begin{mnemonicbox}PF-DC: Predictive-Future, Descriptive-Current\end{mnemonicbox}
\end{solutionbox}

\questionmarks{3(b)}{4}{ક્લાસિફિકેશન અને રિગ્રેશન વચ્ચેના તફાવતની ચર્ચા કરો.}

\begin{solutionbox}
\begin{center}
\captionof{table}{ક્લાસિફિકેશન vs રિગ્રેશન}
\begin{tabulary}{\linewidth}{|L|L|L|}
\hline
\textbf{પાસું} & \textbf{ક્લાસિફિકેશન} & \textbf{રિગ્રેશન} \\ \hline
\textbf{આઉટપુટ} & ડિસ્ક્રીટ કેટેગરીઓ & કન્ટિન્યુઅસ વેલ્યુઝ \\ \hline
\textbf{ધ્યેય} & ક્લાસ લેબલ્સની આગાહી & ન્યુમેરિકલ વેલ્યુઝની આગાહી \\ \hline
\textbf{ઉદાહરણો} & સ્પામ ડિટેક્શન, ઇમેજ રેકગ્નિશન & કિંમત આગાહી, તાપમાન \\ \hline
\textbf{મૂલ્યાંકન} & Accuracy, precision, recall & MSE, RMSE, R-squared \\ \hline
\end{tabulary}
\end{center}

\begin{mnemonicbox}CCNM - Classification-Categories, Regression-Numbers\end{mnemonicbox}
\end{solutionbox}

\questionmarks{3(c)}{7}{ક્લાસિફિકેશનને વ્યાખ્યાયિત કરો. ક્લાસિફિકેશન લર્નિંગના સ્ટેપને વિગતોમાં સમજાવો.}

\begin{solutionbox}
ક્લાસિફિકેશન એ સુપરવાઇઝ્ડ લર્નિંગ તકનીક છે જે ઇનપુટ ડેટા માટે ડિસ્ક્રીટ ક્લાસ લેબલ્સની આગાહી કરે છે.

\begin{center}
\begin{tikzpicture}[node distance=1.2cm, auto]
    \node [gtu block] (Data) {ડેટા કલેક્શન};
    \node [gtu block, below=0.8cm of Data] (Pre) {ડેટા પ્રીપ્રોસેસિંગ};
    \node [gtu block, below=0.8cm of Pre] (Feature) {ફીચર સિલેક્શન};
    \node [gtu block, below=0.8cm of Feature] (Split) {ટ્રેન-ટેસ્ટ સ્પ્લિટ};
    
    \node [gtu block, below left=1.2cm and -1.5cm of Split] (Train) {મોડેલ ટ્રેનિંગ};
    \node [gtu block, below right=1.2cm and -1.5cm of Split] (Eval) {મોડેલ મૂલ્યાંકન};
    
    \node [gtu block, below=2cm of Split] (Deploy) {મોડેલ ડિપ્લોયમેન્ટ};

    \path [gtu arrow] (Data) -- (Pre);
    \path [gtu arrow] (Pre) -- (Feature);
    \path [gtu arrow] (Feature) -- (Split);
    \path [gtu arrow] (Split) -- (Train);
    \path [gtu arrow] (Split) -- (Eval);
    \path [gtu arrow] (Train) -- (Deploy);
    \path [gtu arrow] (Eval) -- (Deploy);
\end{tikzpicture}
\captionof{figure}{ક્લાસિફિકેશન લર્નિંગ સ્ટેપ્સ}
\end{center}

\textbf{વિગતવાર સ્ટેપ્સ:}
\begin{itemize}
    \item \textbf{ડેટા કલેક્શન}: લેબલ્ડ ટ્રેનિંગ ડેટા એકત્ર કરવો
    \item \textbf{પ્રીપ્રોસેસિંગ}: ડેટાને સાફ કરવો અને તૈયાર કરવો
    \item \textbf{ફીચર સિલેક્શન}: સંબંધિત લક્ષણો પસંદ કરવા
    \item \textbf{ડેટા સ્પ્લિટ}: ટ્રેનિંગ અને ટેસ્ટિંગ સેટમાં વિભાજન
    \item \textbf{ટ્રેનિંગ}: ટ્રેનિંગ ડેટાનો ઉપયોગ કરીને મોડેલ બનાવવું
    \item \textbf{મૂલ્યાંકન}: મોડેલ પરફોર્મન્સ ચકાસવી
    \item \textbf{ડિપ્લોયમેન્ટ}: આગાહીઓ માટે મોડેલનો ઉપયોગ
\end{itemize}

\begin{mnemonicbox}DCFSTED - Data, Clean, Features, Split, Train, Evaluate, Deploy\end{mnemonicbox}
\end{solutionbox}

\questionmarks{3(a OR)}{3}{બેગિંગ અને બૂસ્ટિંગ વચ્ચેનો તફાવત આપો.}

\begin{solutionbox}
\begin{center}
\captionof{table}{બેગિંગ vs બૂસ્ટિંગ}
\begin{tabulary}{\linewidth}{|L|L|L|}
\hline
\textbf{લક્ષણ} & \textbf{બેગિંગ} & \textbf{બૂસ્ટિંગ} \\ \hline
\textbf{સેમ્પલિંગ} & બૂટસ્ટ્રેપ સેમ્પલિંગ & ક્રમાનુગત વેઇટેડ સેમ્પલિંગ \\ \hline
\textbf{ટ્રેનિંગ} & પેરેલલ ટ્રેનિંગ & ક્રમાનુગત ટ્રેનિંગ \\ \hline
\textbf{ફોકસ} & વેરિયન્સ ઘટાડવું & બાયસ ઘટાડવું \\ \hline
\end{tabulary}
\end{center}

\begin{mnemonicbox}BPV-BSB: Bagging-Parallel-Variance, Boosting-Sequential-Bias\end{mnemonicbox}
\end{solutionbox}

\questionmarks{3(b OR)}{4}{લોજિસ્ટિક રિગ્રેશનના વિવિધ પ્રકારો સંક્ષિપ્તમાં સમજાવો.}

\begin{solutionbox}
\begin{center}
\captionof{table}{લોજિસ્ટિક રિગ્રેશનના પ્રકારો}
\begin{tabulary}{\linewidth}{|L|L|L|}
\hline
\textbf{પ્રકાર} & \textbf{ક્લાસો} & \textbf{ઉપયોગ} \\ \hline
\textbf{બાઇનરી} & 2 ક્લાસો & હા/ના, પાસ/ફેઇલ \\ \hline
\textbf{મલ્ટિનોમિયલ} & 3+ ક્લાસો (અવ્યવસ્થિત) & રંગ જાતિ \\ \hline
\textbf{ઓર્ડિનલ} & 3+ ક્લાસો (ક્રમાંકિત) & રેટિંગ સ્કેલ \\ \hline
\end{tabulary}
\end{center}

\begin{mnemonicbox}BMO - Binary, Multinomial, Ordinal\end{mnemonicbox}
\end{solutionbox}

\questionmarks{3(c OR)}{7}{k-NN અલ્ગોરિધમ લખો અને તેના ઉપયોગ બતાવો.}

\begin{solutionbox}
K-નિયરેસ્ટ નેઇબર્સ (k-NN) એ લેઝી લર્નિંગ અલ્ગોરિધમ છે જે k નજીકના પડોશીઓના બહુમતી ક્લાસના આધારે ડેટા પોઇન્ટ્સને વર્ગીકૃત કરે છે.

\textbf{અલ્ગોરિધમ સ્ટેપ્સ:}
\begin{enumerate}
    \item k ની વેલ્યુ પસંદ કરો
    \item બધા ટ્રેનિંગ પોઇન્ટ્સ સાથે અંતર કાઢો
    \item k નજીકના પડોશીઓ પસંદ કરો
    \item ક્લાસિફિકેશન માટે: બહુમતી મત; રિગ્રેશન માટે: k પડોશીઓની સરેરાશ
    \item ટેસ્ટ પોઇન્ટને ક્લાસ/વેલ્યુ અસાઇન કરો
\end{enumerate}

\textbf{અંતર ગણતરી:}
\begin{itemize}
    \item \textbf{યુક્લિડિયન ડિસ્ટન્સ}: $\sqrt{(x_1-x_2)^2 + (y_1-y_2)^2}$
\end{itemize}

\textbf{એપ્લિકેશનો:}
\begin{itemize}
    \item \textbf{રેકમેન્ડેશન સિસ્ટમ્સ}: સમાન યુઝર પ્રાધાન્યો
    \item \textbf{ઇમેજ રેકગ્નિશન}: પેટર્ન મેચિંગ
    \item \textbf{મેડિકલ ડાયગ્નોસિસ}: લક્ષણોની સમાનતા
\end{itemize}

\textbf{ફાયદા:}
\begin{itemize}
    \item અમલમાં મૂકવામાં સરળ
    \item ટ્રેનિંગની જરૂર નથી
    \item નાના ડેટાસેટ સાથે સારું કામ કરે છે
\end{itemize}

\begin{mnemonicbox}CDSA - Choose, Distance, Select, Assign\end{mnemonicbox}
\end{solutionbox}

\questionmarks{4(a)}{3}{સપોર્ટ વેક્ટર મશીનની એપ્લિકેશનોની યાદી બનાવો.}

\begin{solutionbox}
\begin{center}
\captionof{table}{SVM એપ્લિકેશનો}
\begin{tabulary}{\linewidth}{|L|L|}
\hline
\textbf{એપ્લિકેશન} & \textbf{ડોમેન} \\ \hline
\textbf{ટેક્સ્ટ ક્લાસિફિકેશન} & ડોક્યુમેન્ટ કેટેગોરાઇઝેશન \\ \hline
\textbf{ઇમેજ રેકગ્નિશન} & ફેસ ડિટેક્શન \\ \hline
\textbf{બાયોઇન્ફોર્મેટિક્સ} & જીન ક્લાસિફિકેશન \\ \hline
\end{tabulary}
\end{center}

\begin{mnemonicbox}TIB - Text, Image, Bio\end{mnemonicbox}
\end{solutionbox}

\questionmarks{4(b)}{4}{k-means અલ્ગોરિધમ માટે સ્યુડો કોડ બનાવો.}

\begin{solutionbox}
\textbf{K-means સ્યુડો કોડ:}
\begin{lstlisting}[language=python, frame=single]
BEGIN K-means
1. Initialize k cluster centroids randomly
2. REPEAT
   a. Assign each point to nearest centroid
   b. Update centroids to mean of assigned points
   c. Calculate total within-cluster sum of squares
3. UNTIL convergence or max iterations
4. RETURN final clusters and centroids
END
\end{lstlisting}

\begin{mnemonicbox}IAUC - Initialize, Assign, Update, Check\end{mnemonicbox}
\end{solutionbox}

\questionmarks{4(c)}{7}{અનસુપરવાઇઝ્ડ લર્નિંગની એપ્લિકેશનો લખો અને સમજાવો.}

\begin{solutionbox}
અનસુપરવાઇઝ્ડ લર્નિંગ લેબલ્ડ ઉદાહરણો વિના ડેટામાં છુપાયેલા પેટર્ન શોધે છે.

\begin{center}
\captionof{table}{મુખ્ય એપ્લિકેશનો}
\begin{tabulary}{\linewidth}{|L|L|L|}
\hline
\textbf{એપ્લિકેશન} & \textbf{વર્ણન} & \textbf{ઉદાહરણ} \\ \hline
\textbf{કસ્ટમર સેગ્મેન્ટેશન} & વર્તન પ્રમાણે ગ્રાહકોનું ગ્રુપિંગ & માર્કેટ રિસર્ચ \\ \hline
\textbf{એનોમેલી ડિટેક્શન} & અસામાન્ય પેટર્ન ઓળખવા & ફ્રોડ ડિટેક્શન \\ \hline
\textbf{ડેટા કમ્પ્રેશન} & ડાયમેન્શનાલિટી ઘટાડવી & ઇમેજ કમ્પ્રેશન \\ \hline
\textbf{એસોસિએશન રૂલ્સ} & આઇટમ સંબંધો શોધવા & માર્કેટ બાસ્કેટ વિશ્લેષણ \\ \hline
\end{tabulary}
\end{center}

\textbf{ક્લસ્ટરિંગ એપ્લિકેશનો:}
\begin{itemize}
    \item \textbf{માર્કેટ રિસર્ચ}: કસ્ટમર ગ્રુપિંગ
    \item \textbf{સોશિયલ નેટવર્ક વિશ્લેષણ}: કમ્યુનિટી ડિટેક્શન
    \item \textbf{જીન સીક્વેન્સિંગ}: બાયોલોજિકલ ક્લાસિફિકેશન
\end{itemize}

\textbf{ડાયમેન્શનાલિટી રિડક્શન:}
\begin{itemize}
    \item \textbf{વિઝ્યુઅલાઇઝેશન}: હાઇ-ડાયમેન્શનલ ડેટા પ્લોટિંગ
    \item \textbf{ફીચર એક્સ્ટ્રેક્શન}: નોઇઝ રિડક્શન
\end{itemize}

\begin{mnemonicbox}CADA - Customer, Anomaly, Data, Association\end{mnemonicbox}
\end{solutionbox}

\questionmarks{4(a OR)}{3}{રિગ્રેશનની એપ્લિકેશનોની યાદી બનાવો.}

\begin{solutionbox}
\begin{center}
\captionof{table}{રિગ્રેશન એપ્લિકેશનો}
\begin{tabulary}{\linewidth}{|L|L|}
\hline
\textbf{એપ્લિકેશન} & \textbf{હેતુ} \\ \hline
\textbf{સ્ટોક પ્રાઇસ પ્રેડિક્શન} & ફાઇનાન્શિયલ ફોરકાસ્ટિંગ \\ \hline
\textbf{સેલ્સ ફોરકાસ્ટિંગ} & બિઝનેસ પ્લાનિંગ \\ \hline
\textbf{મેડિકલ ડાયગ્નોસિસ} & રિસ્ક એસેસમેન્ટ \\ \hline
\end{tabulary}
\end{center}

\begin{mnemonicbox}SSM - Stock, Sales, Medical\end{mnemonicbox}
\end{solutionbox}

\questionmarks{4(b OR)}{4}{નીચેના શબ્દો વ્યાખ્યાયિત કરો: i) Support ii) Confidence}

\begin{solutionbox}
\begin{center}
\captionof{table}{એસોસિએશન રૂલ શબ્દો}
\begin{tabulary}{\linewidth}{|L|L|L|}
\hline
\textbf{શબ્દ} & \textbf{વ્યાખ્યા} & \textbf{ફોર્મ્યુલા} \\ \hline
\textbf{Support} & ડેટાબેઝમાં આઇટમસેટની આવર્તન & $Support(A) = \frac{|A|}{|D|}$ \\ \hline
\textbf{Confidence} & રૂલની શરતી સંભાવના & $Confidence(A \to B) = \frac{Support(A \cup B)}{Support(A)}$ \\ \hline
\end{tabulary}
\end{center}

\textbf{ઉદાહરણ:}
\begin{itemize}
    \item જો 30\% ટ્રાન્ઝેક્શનમાં બ્રેડ અને દૂધ હોય: Support = 0.3
    \item જો 80\% બ્રેડ ખરીદનારાઓ દૂધ પણ ખરીદે: Confidence = 0.8
\end{itemize}

\begin{mnemonicbox}SF-CP: Support-Frequency, Confidence-Probability\end{mnemonicbox}
\end{solutionbox}

\questionmarks{4(c OR)}{7}{apriori algorithm ને વિગતવાર સમજાવો.}

\begin{solutionbox}
Apriori અલ્ગોરિધમ એપ્રિઓરી પ્રોપર્ટીનો ઉપયોગ કરીને ટ્રાન્ઝેક્શનલ ડેટામાં ફ્રીક્વન્ટ આઇટમસેટ્સ શોધે છે.

\begin{center}
\begin{tikzpicture}[node distance=1.5cm, auto]
    \node [gtu block] (Find) {ફ્રીક્વન્ટ 1-આઇટમસેટ્સ\\શોધો};
    \node [gtu block, below=1.2cm of Find] (Gen) {કેન્ડિડેટ 2-આઇટમસેટ્સ\\જનરેટ કરો};
    \node [gtu block, below=1.2cm of Gen] (Prune) {એપ્રિઓરી પ્રોપર્ટી\\વાપરીને પ્રૂન કરો};
    \node [gtu block, below=1.2cm of Prune] (Count) {ડેટાબેઝમાં સપોર્ટ\\કાઉન્ટ કરો};
    \node [gtu block, below=1.2cm of Count] (FindK) {ફ્રીક્વન્ટ k-આઇટમસેટ્સ\\શોધો};
    \node [gtu state, right=2cm of Prune] (Check) {વધુ\\કેન્ડિડેટ્સ?};

    \path [gtu arrow] (Find) -- (Gen);
    \path [gtu arrow] (Gen) -- (Prune);
    \path [gtu arrow] (Prune) -- (Count);
    \path [gtu arrow] (Count) -- (FindK);
    \path [gtu arrow] (FindK) -| (Check);
    \path [gtu arrow] (Check) |- node[above, near end] {હા} (Gen);
    \path [gtu arrow] (Check) -- node[right] {ના} ++(0,-2) node[gtu block, anchor=north] {રૂલ્સ જનરેટ કરો};

\end{tikzpicture}
\captionof{figure}{Apriori અલ્ગોરિધમ પ્રક્રિયા}
\end{center}

\textbf{એપ્રિઓરી પ્રોપર્ટી:}
\begin{itemize}
    \item જો આઇટમસેટ ફ્રીક્વન્ટ છે, તો તેના બધા સબસેટ્સ ફ્રીક્વન્ટ છે
    \item જો આઇટમસેટ ઇનફ્રીક્વન્ટ છે, તો તેના બધા સુપરસેટ્સ ઇનફ્રીક્વન્ટ છે
\end{itemize}

\textbf{સ્ટેપ્સ:}
\begin{enumerate}
    \item **ડેટાબેઝ સ્કેન**: 1-આઇટમ સપોર્ટ કાઉન્ટ કરો
    \item **કેન્ડિડેટ્સ જનરેટ**: ફ્રીક્વન્ટ k-આઇટમસેટ્સમાંથી k+1 આઇટમસેટ્સ બનાવો
    \item **પ્રૂન**: ઇનફ્રીક્વન્ટ સબસેટ્સ સાથેના કેન્ડિડેટ્સ દૂર કરો
    \item **સપોર્ટ કાઉન્ટ**: કેન્ડિડેટ ફ્રીક્વન્સી માટે ડેટાબેઝ સ્કેન કરો
    \item **પુનરાવર્તન**: નવા ફ્રીક્વન્ટ આઇટમસેટ્સ ન મળે ત્યાં સુધી
\end{enumerate}

\textbf{એપ્લિકેશનો:}
\begin{itemize}
    \item માર્કેટ બાસ્કેટ વિશ્લેષણ
    \item વેબ યુઝેજ પેટર્ન
    \item પ્રોટીન સીક્વન્સ
\end{itemize}

\begin{mnemonicbox}SGPCR - Scan, Generate, Prune, Count, Repeat\end{mnemonicbox}
\end{solutionbox}

\questionmarks{5(a)}{3}{matplotlib ના મુખ્ય ફીચર્સની યાદી બનાવો.}

\begin{solutionbox}
\begin{center}
\captionof{table}{Matplotlib ફીચર્સ}
\begin{tabulary}{\linewidth}{|L|L|}
\hline
\textbf{ફીચર} & \textbf{વર્ણન} \\ \hline
\textbf{મલ્ટિપલ પ્લોટ ટાઇપ્સ} & લાઇન, બાર, સ્કેટર, હિસ્ટોગ્રામ \\ \hline
\textbf{કસ્ટમાઇઝેશન} & કલર્સ, સ્ટાઇલ્સ, લેબલ્સ \\ \hline
\textbf{એક્સપોર્ટ ઓપ્શન્સ} & PNG, PDF, SVG ફોર્મેટ્સ \\ \hline
\end{tabulary}
\end{center}

\begin{mnemonicbox}MCE - Multiple, Customization, Export\end{mnemonicbox}
\end{solutionbox}

\questionmarks{5(b)}{4}{Numpy ના પ્રોગ્રામમાં iris ડેટાસેટ કેવી રીતે લોડ કરવો? સમજાવો.}

\begin{solutionbox}
\textbf{NumPy માં Iris ડેટાસેટ લોડ કરવું:}
\begin{lstlisting}[language=python]
import numpy as np
from sklearn.datasets import load_iris

# iris ડેટાસેટ લોડ કરો
iris = load_iris()
data = iris.data    # ફીચર્સ
target = iris.target # લેબલ્સ
\end{lstlisting}

\textbf{સ્ટેપ્સ:}
\begin{itemize}
    \item \textbf{Import}: જરૂરી લાઇબ્રેરીઓ import કરો
    \item \textbf{Load}: sklearn ના \code{load\_iris()} ફંક્શનનો ઉપયોગ કરો
    \item \textbf{Extract}: ફીચર્સ અને ટાર્ગેટ એરે મેળવો
    \item \textbf{Access}: \code{.data} અને \code{.target} એટ્રિબ્યુટ્સનો ઉપયોગ કરો
\end{itemize}

\begin{mnemonicbox}ILEA - Import, Load, Extract, Access\end{mnemonicbox}
\end{solutionbox}

\questionmarks{5(c)}{7}{Pandas ની વિશેષતાઓ અને એપ્લિકેશનો સમજાવો.}

\begin{solutionbox}
Pandas એ Python માટે શક્તિશાળી ડેટા મેનિપ્યુલેશન અને વિશ્લેષણ લાઇબ્રેરી છે.

\begin{center}
\captionof{table}{મુખ્ય ફીચર્સ}
\begin{tabulary}{\linewidth}{|L|L|}
\hline
\textbf{ફીચર} & \textbf{વર્ણન} \\ \hline
\textbf{DataFrame} & 2D લેબલ્ડ ડેટા સ્ટ્રક્ચર \\ \hline
\textbf{Series} & 1D લેબલ્ડ એરે \\ \hline
\textbf{Data I/O} & વિવિધ ફાઇલ ફોર્મેટ્સ વાંચવા/લખવા \\ \hline
\textbf{Data Cleaning} & મિસિંગ વેલ્યુઝ હેન્ડલ કરવા \\ \hline
\textbf{Grouping} & ગ્રુપ અને એગ્રીગેટ ઓપરેશન્સ \\ \hline
\end{tabulary}
\end{center}

\textbf{એપ્લિકેશનો:}
\begin{itemize}
    \item \textbf{ડેટા એનાલિસિસ}: આંકડાકીય વિશ્લેષણ
    \item \textbf{ડેટા ક્લીનિંગ}: ML માટે પ્રીપ્રોસેસિંગ
    \item \textbf{ફાઇનાન્શિયલ એનાલિસિસ}: સ્ટોક માર્કેટ ડેટા
    \item \textbf{વેબ સ્ક્રેપિંગ}: HTML ટેબલ્સ પાર્સ કરવા
\end{itemize}

\textbf{સામાન્ય ઓપરેશન્સ:}
\begin{itemize}
    \item \textbf{ડેટા વાંચવો}: \code{pd.read\_csv()}, \code{pd.read\_excel()}
    \item \textbf{ફિલ્ટરિંગ}: \code{df[df['column'] > value]}
    \item \textbf{ગ્રુપિંગ}: \code{df.groupby('column').mean()}
\end{itemize}

\begin{mnemonicbox}DSDCG - DataFrame, Series, Data I/O, Cleaning, Grouping\end{mnemonicbox}
\end{solutionbox}

\questionmarks{5(a OR)}{3}{matplotlib ની એપ્લિકેશનોની યાદી બનાવો.}

\begin{solutionbox}
\begin{center}
\captionof{table}{Matplotlib એપ્લિકેશનો}
\begin{tabulary}{\linewidth}{|L|L|}
\hline
\textbf{એપ્લિકેશન} & \textbf{હેતુ} \\ \hline
\textbf{સાયન્ટિફિક વિઝ્યુઅલાઇઝેશન} & રિસર્ચ ડેટા પ્લોટિંગ \\ \hline
\textbf{બિઝનેસ એનાલિટિક્સ} & ડેશબોર્ડ બનાવવું \\ \hline
\textbf{એજ્યુકેશનલ કન્ટેન્ટ} & શિક્ષણ સામગ્રી \\ \hline
\end{tabulary}
\end{center}

\begin{mnemonicbox}SBE - Scientific, Business, Educational\end{mnemonicbox}
\end{solutionbox}

\questionmarks{5(b OR)}{4}{Pandas માં csv ફાઇલ ઇમ્પોર્ટ કરવાના સ્ટેપ્સ લખો અને સમજાવો.}

\begin{solutionbox}
\textbf{Pandas માં CSV ઇમ્પોર્ટ કરવાના સ્ટેપ્સ:}
\begin{lstlisting}[language=python]
import pandas as pd

# સ્ટેપ 1: pandas લાઇબ્રેરી import કરો
# સ્ટેપ 2: read_csv() ફંક્શનનો ઉપયોગ કરો
df = pd.read_csv('filename.csv')

# વૈકલ્પિક પેરામીટર્સ
df = pd.read_csv('file.csv', 
                 header=0,     # પ્રથમ પંક્તિ હેડર તરીકે
                 sep=',',      # કોમા સેપરેટર
                 index_col=0)  # પ્રથમ કૉલમ ઇન્ડેક્સ તરીકે
\end{lstlisting}

\textbf{પ્રક્રિયા:}
\begin{itemize}
    \item \textbf{Import}: pandas લાઇબ્રેરી import કરો
    \item \textbf{Read}: \code{pd.read\_csv()} ફંક્શનનો ઉપયોગ કરો
    \item \textbf{Specify}: ફાઇલ પાથ અને પેરામીટર્સ ઉમેરો
    \item \textbf{Store}: DataFrame વેરિએબલમાં અસાઇન કરો
\end{itemize}

\begin{mnemonicbox}IRSS - Import, Read, Specify, Store\end{mnemonicbox}
\end{solutionbox}

\questionmarks{5(c OR)}{7}{Scikit-Learn ની વિશેષતાઓ અને એપ્લિકેશનો સમજાવો.}

\begin{solutionbox}
Scikit-Learn એ Python માટે વ્યાપક મશીન લર્નિંગ લાઇબ્રેરી છે.

\begin{center}
\captionof{table}{મુખ્ય ફીચર્સ}
\begin{tabulary}{\linewidth}{|L|L|}
\hline
\textbf{ફીચર} & \textbf{વર્ણન} \\ \hline
\textbf{અલ્ગોરિધમ્સ} & ક્લાસિફિકેશન, રિગ્રેશન, ક્લસ્ટરિંગ \\ \hline
\textbf{પ્રીપ્રોસેસિંગ} & ડેટા સ્કેલિંગ અને ટ્રાન્સફોર્મેશન \\ \hline
\textbf{મોડેલ સિલેક્શન} & ક્રોસ-વેલિડેશન અને ગ્રિડ સર્ચ \\ \hline
\textbf{મેટ્રિક્સ} & પરફોર્મન્સ મૂલ્યાંકન ટૂલ્સ \\ \hline
\end{tabulary}
\end{center}

\textbf{એપ્લિકેશનો:}
\begin{itemize}
    \item \textbf{હેલ્થકેર}: રોગ આગાહી
    \item \textbf{ફાઇનાન્સ}: ક્રેડિટ સ્કોરિંગ
    \item \textbf{માર્કેટિંગ}: કસ્ટમર સેગ્મેન્ટેશન
    \item \textbf{ટેકનોલોજી}: રેકમેન્ડેશન સિસ્ટમ્સ
\end{itemize}

\textbf{અલ્ગોરિધમ કેટેગરીઓ:}
\begin{itemize}
    \item \textbf{સુપરવાઇઝ્ડ}: SVM, Random Forest, Linear Regression
    \item \textbf{અનસુપરવાઇઝ્ડ}: K-means, DBSCAN, PCA
    \item \textbf{એન્સેમ્બલ}: Bagging, Boosting
\end{itemize}

\textbf{વર્કફ્લો:}
\begin{enumerate}
    \item \textbf{ડેટા તૈયારી}: પ્રીપ્રોસેસિંગ
    \item \textbf{મોડેલ સિલેક્શન}: અલ્ગોરિધમ પસંદ કરો
    \item \textbf{ટ્રેનિંગ}: ડેટા પર મોડેલ ફિટ કરો
    \item \textbf{મૂલ્યાંકન}: પરફોર્મન્સ આકારો
    \item \textbf{આગાહી}: ફોરકાસ્ટ બનાવો
\end{enumerate}

\begin{mnemonicbox}APME - Algorithms, Preprocessing, Metrics, Evaluation\end{mnemonicbox}
\end{solutionbox}

\end{document}
