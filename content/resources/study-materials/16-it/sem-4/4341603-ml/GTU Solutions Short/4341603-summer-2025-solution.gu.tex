\documentclass{article}

% content/resources/templates/preamble.tex
\usepackage[margin=0.6in]{geometry}
\author{Milav Dabgar}
\usepackage{amsmath,amssymb,amsthm}
\usepackage{booktabs}
\usepackage{multirow}
\usepackage{xcolor}
\usepackage{tcolorbox}
\tcbuselibrary{breakable,skins}
\usepackage[colorlinks=true,linkcolor=blue]{hyperref}
\usepackage{titlesec}
\usepackage{enumitem}
\usepackage{tikz}
\usepackage{pgfplots}
\usepackage{circuitikz}
\usepackage[version=4]{mhchem}
\usepackage{longtable}
\usepackage{array}
\usepackage{float}
\usepackage{caption}
\usepackage{listings}

\lstset{
  basicstyle=\small\ttfamily,
  breaklines=true,
  breakatwhitespace=false,
  postbreak=\mbox{\textcolor{red}{$\hookrightarrow$}\space},
  float=false,
  numbers=left,
  numberstyle=\tiny\color{gray},
  numbersep=10pt,
  xleftmargin=2em,
  keywordstyle=\color{blue},
  commentstyle=\color{green!60!black},
  stringstyle=\color{purple},
  backgroundcolor=\color{gray!5},
  showstringspaces=false,
  tabsize=2,
  captionpos=b,
  keepspaces=true,
  columns=flexible
}

\pgfplotsset{compat=1.18}
\usetikzlibrary{shapes,arrows,positioning,calc,patterns,decorations.pathmorphing,decorations.markings,arrows.meta}

% Color scheme
\definecolor{headcolor}{RGB}{0,102,204}
\definecolor{keycolor}{RGB}{220,20,60}
\definecolor{solutioncolor}{RGB}{34,139,34}
\definecolor{mnemoniccolor}{RGB}{148,0,211}
\definecolor{codecolor}{RGB}{0,0,100}

% Spacing
\setlength{\parskip}{3pt}
\setlist[itemize]{nosep}
\setlist[enumerate]{nosep}

% Title formatting
\titleformat{\section}{\Large\bfseries\color{headcolor}}{\thesection}{1em}{}
\titleformat{\subsection}{\large\bfseries\color{headcolor}}{\thesubsection}{1em}{}

% Pandoc tightlist compatibility
\providecommand{\tightlist}{%
  \setlength{\itemsep}{0pt}\setlength{\parskip}{0pt}}

% Pandoc longtable compatibility
\newcounter{none}
\def\thenone{}


% content/resources/templates/gujarati-boxes.tex
\usepackage{fontspec}
\usepackage{polyglossia}

% Set Gujarati as main language (document is primarily in Gujarati)
% Note: gloss-gujarati.ldf doesn't exist in polyglossia, but it will use hyphenation patterns
\setdefaultlanguage{gujarati}
\setotherlanguage{english}

% Configure Gujarati font properly
% Use Language=Default to prevent polyglossia from trying to add language-specific features
% that don't exist for Gujarati, which causes "empty feature" warnings
\newfontfamily\gujaratifont[Script=Gujarati,AutoFakeBold=2.5,AutoFakeSlant=0.3]{Noto Sans Gujarati}
\setmainfont[Script=Gujarati,AutoFakeBold=2.5,AutoFakeSlant=0.3]{Noto Sans Gujarati}
% Use Noto Sans Gujarati for monospace to support Gujarati in text
\setmonofont[Scale=0.9]{Noto Sans Gujarati}

% Configure English to use the same font
\newfontfamily\englishfont[Script=Gujarati,AutoFakeBold=2.5,AutoFakeSlant=0.3]{Noto Sans Gujarati}

% Translations for polyglossia
\gappto\captionsgujarati{
  \renewcommand{\tablename}{કોષ્ટક}
  \renewcommand{\figurename}{આકૃતિ}
}

% Helper for TikZ nodes to ensure Gujarati font
\newcommand{\gu}[1]{{\gujaratifont #1}}

% Custom environments
\newtcolorbox{solutionbox}{
    breakable,
    enhanced,
    colback=solutioncolor!5!white,
    colframe=solutioncolor!75!black,
    fonttitle=\bfseries,
    title=જવાબ
}

\newtcolorbox{solutionboxnobreak}{
 colback=solutioncolor!5!white,
 colframe=solutioncolor!75!black,
 fonttitle=\bfseries,
 title=જવાબ
}

\newtcolorbox{keyformula}{
 breakable,
 enhanced,
 colback=keycolor!5!white,
 colframe=keycolor!75!black,
 fonttitle=\bfseries,
 title=રાસાયણિક સમીકરણ/સૂત્ર
}

\newtcolorbox{mnemonicbox}{
 breakable,
 enhanced,
 colback=mnemoniccolor!5!white,
 colframe=mnemoniccolor!75!black,
 fonttitle=\bfseries,
 title=મેમરી ટ્રીક
}


% Custom commands for GTU solutions
% This file defines semantic commands for consistent formatting

% Question command with automatic formatting
\newcommand{\question}[2]{%
  \section*{Question #1}%
  \textbf{#2}%
}

% OR question variant
\newcommand{\questionor}[2]{%
  \section*{Question #1 OR}%
  \textbf{#2}%
}

% Proper table environment with caption
\newenvironment{answertable}[1]{%
  \begin{table}[htbp]
  \centering
  \caption{#1}
}{%
  \end{table}
}

% Proper figure environment for diagrams
\newenvironment{answerdiagram}[1]{%
  \begin{figure}[htbp]
  \centering
  \caption{#1}
}{%
  \end{figure}
}

% Semantic markup for key terms
\newcommand{\keyword}[1]{\textbf{#1}}
\newcommand{\code}[1]{\texttt{#1}}
\newcommand{\classname}[1]{\texttt{#1}}
\newcommand{\methodname}[1]{\texttt{#1}}

% Proper quotation marks
\newcommand{\mnemonic}[1]{``#1''}


\title{Fundamentals of Machine Learning (4341603) - Summer 2025 Solution}
\date{May 17, 2025}

\begin{document}
\maketitle

\questionmarks{1(a)}{3}{મશીન લર્નિંગની વ્યાખ્યા આપો. મશીન લર્નિંગની કોઈપણ બે ઉપયોગીતાઓ આપો.}
\begin{solutionbox}
મશીન લર્નિંગ એ આર્ટિફિશિયલ ઇન્ટેલિજન્સનો એક ભાગ છે જે કમ્પ્યુટરને ડેટામાંથી શીખવા અને દરેક કાર્ય માટે સ્પષ્ટ પ્રોગ્રામિંગ વિના નિર્ણયો લેવાની ક્ષમતા આપે છે.

\textbf{ઉપયોગીતાઓ:}
\begin{itemize}
    \item \textbf{ઈમેઇલ સ્પામ ડિટેક્શન}: આપોઆપ સ્પામ ઈમેઇલ ઓળખે અને ફિલ્ટર કરે છે
    \item \textbf{સુઝાવ સિસ્ટમ}: Amazon જેવી ઈ-કોમર્સ સાઇટ્સ પર પ્રોડક્ટ સુઝાવે છે
\end{itemize}

\begin{center}
\captionof{table}{ML વિ ટ્રેડિશનલ પ્રોગ્રામિંગ}
\begin{tabulary}{\linewidth}{L L}
\hline
\textbf{પરંપરાગત પ્રોગ્રામિંગ} & \textbf{મશીન લર્નિંગ} \\
\hline
ઇનપુટ ડેટા + પ્રોગ્રામ \(\to\) આઉટપુટ & ઇનપુટ ડેટા + આઉટપુટ \(\to\) પ્રોગ્રામ \\
નિયમો સ્પષ્ટપણે કોડ કરવામાં આવે છે & નિયમો ડેટામાંથી શીખવામાં આવે છે \\
\hline
\end{tabulary}
\end{center}
\end{solutionbox}

\begin{mnemonicbox}
\mnemonic{ML = ડેટામાંથી શીખવું બનાવો}
\end{mnemonicbox}

\questionmarks{1(b)}{4}{વ્યાખ્યા આપો: અંડર ફિટિંગ અને ઓવર ફિટિંગ.}
\begin{solutionbox}
\textbf{અંડરફિટિંગ} ત્યારે થાય છે જ્યારે મોડલ ડેટામાં છુપાયેલા પેટર્ન કેપ્ચર કરવા માટે ખૂબ સાદું હોય છે, જેના પરિણામે ટ્રેનિંગ અને ટેસ્ટ બંને ડેટા પર નબળી કામગીરી થાય છે.

\textbf{ઓવરફિટિંગ} ત્યારે થાય છે જ્યારે મોડલ ટ્રેનિંગ ડેટાને અવાજ સહિત ખૂબ સારી રીતે શીખે છે, જેના કારણે નવા અદ્રશ્ય ડેટા પર નબળી કામગીરી થાય છે.

\begin{center}
\captionof{table}{સરખામણી}
\begin{tabulary}{\linewidth}{L L L}
\hline
\textbf{પાસું} & \textbf{અંડરફિટિંગ} & \textbf{ઓવરફિટિંગ} \\
\hline
\textbf{ટ્રેનિંગ એક્યુરેસી} & ઓછી & વધારે \\
\textbf{ટેસ્ટ એક્યુરેસી} & ઓછી & ઓછી \\
\textbf{મોડલ કોમ્પ્લેક્સિટી} & ખૂબ સાદું & ખૂબ જટિલ \\
\textbf{સોલ્યુશન} & કોમ્પ્લેક્સિટી વધારો & કોમ્પ્લેક્સિટી ઘટાડો \\
\hline
\end{tabulary}
\end{center}
\end{solutionbox}

\begin{mnemonicbox}
\mnemonic{અંડર = ઓછું કામ, ઓવર = વધુ પડતું શીખવું}
\end{mnemonicbox}

\questionmarks{1(c)}{7}{મશીન લર્નિંગના વિવિધ પ્રકારો યોગ્ય ઉદાહરણની મદદથી વર્ણવો.}
\begin{solutionbox}
\begin{center}
\captionof{table}{મશીન લર્નિંગના પ્રકારો}
\begin{tabulary}{\linewidth}{L L L}
\hline
\textbf{પ્રકાર} & \textbf{વર્ણન} & \textbf{ઉદાહરણ} \\
\hline
\textbf{સુપરવાઇઝ્ડ} & લેબલ કરેલ ટ્રેનિંગ ડેટા વાપરે છે & ઈમેઇલ વર્ગીકરણ \\
\textbf{અનસુપરવાઇઝ્ડ} & લેબલ કરેલ ડેટા નથી, પેટર્ન શોધે છે & કસ્ટમર સેગમેન્ટેશન \\
\textbf{રિઇન્ફોર્સમેન્ટ} & પુરસ્કાર/દંડ દ્વારા શીખે છે & ગેમ રમતું AI \\
\hline
\end{tabulary}
\end{center}

\textbf{સુપરવાઇઝ્ડ લર્નિંગ} ઇનપુટ-આઉટપુટ જોડીઓ વાપરીને મોડલ ટ્રેન કરે છે. અલ્ગોરિધમ ઉદાહરણોમાંથી શીખીને નવા ડેટા માટે પરિણામોની આગાહી કરે છે.

\textbf{અનસુપરવાઇઝ્ડ લર્નિંગ} ટાર્ગેટ લેબલ વિના ડેટામાં છુપાયેલા પેટર્ન શોધે છે. તે સમાન ડેટા પોઇન્ટ્સને એકસાથે જૂથબદ્ધ કરે છે.

\textbf{રિઇન્ફોર્સમેન્ટ લર્નિંગ} સારા કાર્યો માટે પુરસ્કાર અને ખરાબ કાર્યો માટે દંડ આપીને એજન્ટને નિર્ણય લેવાનું શીખવે છે.

\begin{center}
\begin{tikzpicture}[node distance=1.5cm]
    \node [gtu block] (ML) {મશીન લર્નિંગ};
    \node [gtu block, below left=1.5cm and 1cm of ML] (Super) {સુપરવાઇઝ્ડ લર્નિંગ};
    \node [gtu block, below=1.5cm of ML] (Unsuper) {અનસુપરવાઇઝ્ડ લર્નિંગ};
    \node [gtu block, below right=1.5cm and 1cm of ML] (Reinf) {રિઇન્ફોર્સમેન્ટ લર્નિંગ};
    
    \node [gtu state, below=0.8cm of Super, text width=2.5cm] (S_Apps) {ક્લાસિફિકેશન\\રિગ્રેશન};
    \node [gtu state, below=0.8cm of Unsuper, text width=2.5cm] (U_Apps) {ક્લસ્ટરિંગ\\એસોસિએશન રૂલ્સ};
    
    \path [gtu arrow] (ML) -- (Super);
    \path [gtu arrow] (ML) -- (Unsuper);
    \path [gtu arrow] (ML) -- (Reinf);
    \path [gtu arrow] (Super) -- (S_Apps);
    \path [gtu arrow] (Unsuper) -- (U_Apps);
\end{tikzpicture}
\captionof{figure}{મશીન લર્નિંગના પ્રકારો}
\end{center}
\end{solutionbox}

\begin{mnemonicbox}
\mnemonic{સુપર અન-સુપરવાઇઝ્ડ રિઇન્ફોર્સ શીખવું}
\end{mnemonicbox}

\questionmarks{1(c) અથવા}{7}{મશીન લર્નિંગમાં ઉપયોગ થતી વિવિધ ટૂલ્સ અને ટેકનોલોજી વર્ણવો.}
\begin{solutionbox}
\begin{center}
\captionof{table}{ML ટૂલ્સ અને ટેકનોલોજીઓ}
\begin{tabulary}{\linewidth}{L L L}
\hline
\textbf{કેટેગરી} & \textbf{ટૂલ્સ} & \textbf{હેતુ} \\
\hline
\textbf{પ્રોગ્રામિંગ} & Python, R & મુખ્ય ડેવલપમેન્ટ \\
\textbf{લાઇબ્રેરીઓ} & Scikit-learn, TensorFlow & મોડલ બિલ્ડિંગ \\
\textbf{ડેટા પ્રોસેસિંગ} & Pandas, NumPy & ડેટા મેનિપ્યુલેશન \\
\textbf{વિઝ્યુલાઇઝેશન} & Matplotlib, Seaborn & ડેટા પ્લોટિંગ \\
\hline
\end{tabulary}
\end{center}

\textbf{Python} તેની સરળતા અને વ્યાપક લાઇબ્રેરીઓને કારણે સૌથી લોકપ્રિય ભાષા છે.

\textbf{Scikit-learn} ડેટા માઇનિંગ અને વિશ્લેષણ માટે સરળ ટૂલ્સ પ્રદાન કરે છે, જે શરૂઆતીઓ માટે પરફેક્ટ છે.

\textbf{TensorFlow} અને \textbf{PyTorch} ડીપ લર્નિંગ એપ્લિકેશન માટે એડવાન્સ ફ્રેમવર્ક છે.

\textbf{Jupyter Notebook} પ્રયોગ માટે ઇન્ટરેક્ટિવ ડેવલપમેન્ટ એન્વાયર્નમેન્ટ ઓફર કરે છે.

\begin{center}
\begin{tikzpicture}[node distance=1.5cm, auto]
    \node [gtu block] (Data) {ડેટા};
    \node [gtu block, right=of Data] (Libs) {Pandas/NumPy};
    \node [gtu block, right=of Libs] (Learn) {Scikit-learn};
    \node [gtu block, right=of Learn] (Model) {મોડલ};
    \node [gtu block, right=of Model] (Viz) {Matplotlib};
    
    \path [gtu arrow] (Data) -- (Libs);
    \path [gtu arrow] (Libs) -- (Learn);
    \path [gtu arrow] (Learn) -- (Model);
    \path [gtu arrow] (Model) -- (Viz);
\end{tikzpicture}
\captionof{figure}{ML ટૂલ્સ વર્કફ્લો}
\end{center}
\end{solutionbox}

\begin{mnemonicbox}
\mnemonic{Python Pandas Scikit Tensor Jupyter}
\end{mnemonicbox}

\questionmarks{2(a)}{3}{Qualitative ડેટા અને Quantitative ડેટા વચ્ચેનો તફાવત આપો.}
\begin{solutionbox}
\begin{center}
\captionof{table}{Qualitative વિ Quantitative ડેટા}
\begin{tabulary}{\linewidth}{L L}
\hline
\textbf{Qualitative ડેટા} & \textbf{Quantitative ડેટા} \\
\hline
\textbf{બિન-સંખ્યાત્મક} કેટેગરીઓ & \textbf{સંખ્યાત્મક} મૂલ્યો \\
રંગો, નામો, ગ્રેડ્સ & ઊંચાઈ, વજન, કિંમત \\
માપી શકાતું નથી & માપી શકાય છે \\
\hline
\end{tabulary}
\end{center}

\textbf{Qualitative ડેટા} એવા ગુણો અથવા લક્ષણોનું વર્ણન કરે છે જે સંખ્યાત્મક રીતે માપી શકાતા નથી.

\textbf{Quantitative ડેટા} સંખ્યાઓ તરીકે વ્યક્ત કરેલા માપી શકાય તેવા જથ્થાઓનું પ્રતિનિધિત્વ કરે છે.
\end{solutionbox}

\begin{mnemonicbox}
\mnemonic{Quality = કેટેગરીઓ, Quantity = સંખ્યાઓ}
\end{mnemonicbox}

\questionmarks{2(b)}{4}{નીચે આપેલા ડેટાનું mean અને median શોધો: 3,4,5,5,7,8,9,11,12,14.}
\begin{solutionbox}
\textbf{આપેલ ડેટા:} 3, 4, 5, 5, 7, 8, 9, 11, 12, 14

\textbf{Mean ગણતરી:}
\begin{itemize}
    \item સરવાળો = \(3+4+5+5+7+8+9+11+12+14 = 78\)
    \item સંખ્યાઓની ગિનતી = 10
    \item \textbf{Mean = \(78/10 = 7.8\)}
\end{itemize}

\textbf{Median ગણતરી:}
\begin{itemize}
    \item ડેટા પહેલેથી જ સોર્ટ થયેલ છે
    \item 10 સંખ્યાઓ માટે: Median = (5મી + 6ઠી મૂલ્ય)/2
    \item \textbf{Median = \((7+8)/2 = 7.5\)}
\end{itemize}

\begin{center}
\captionof{table}{પરિણામો}
\begin{tabulary}{\linewidth}{L L}
\hline
\textbf{માપદંડ} & \textbf{મૂલ્ય} \\
\hline
\textbf{Mean} & 7.8 \\
\textbf{Median} & 7.5 \\
\hline
\end{tabulary}
\end{center}
\end{solutionbox}

\begin{mnemonicbox}
\mnemonic{Mean = સરેરાશ, Median = મધ્યક}
\end{mnemonicbox}

\questionmarks{2(c)}{7}{મશીન લર્નિંગની એક્ટિવિટી વિગતવાર વર્ણવો.}
\begin{solutionbox}
\begin{center}
\captionof{table}{મશીન લર્નિંગ એક્ટિવિટીઓ}
\begin{tabulary}{\linewidth}{L L L}
\hline
\textbf{એક્ટિવિટી} & \textbf{વર્ણન} & \textbf{ઉદાહરણ} \\
\hline
\textbf{ડેટા કલેક્શન} & સંબંધિત ડેટા એકત્રિત કરવું & સર્વે પ્રતિભાવો \\
\textbf{ડેટા પ્રીપ્રોસેસિંગ} & ડેટા સાફ અને તૈયાર કરવું & ડુપ્લિકેટ્સ દૂર કરવા \\
\textbf{ફીચર સિલેક્શન} & મહત્વપૂર્ણ વેરિયેબલ્સ પસંદ કરવા & લોન માટે ઉંમર, આવક \\
\textbf{મોડલ ટ્રેનિંગ} & અલ્ગોરિધમને પેટર્ન શીખવવું & ટ્રેનિંગ ડેટા ખવડાવવો \\
\textbf{મોડલ ઇવેલ્યુએશન} & મોડલની કામગીરી પરીક્ષણ & એક્યુરેસી મેઝરમેન્ટ \\
\hline
\end{tabulary}
\end{center}

\textbf{ડેટા કલેક્શન} ડેટાબેસ, સેન્સર્સ અથવા સર્વે જેવા વિવિધ સ્રોતોમાંથી માહિતી એકત્રિત કરવાનો સમાવેશ કરે છે.

\textbf{ડેટા પ્રીપ્રોસેસિંગ} વિશ્લેષણ માટે કાચા ડેટાને સાફ, રૂપાંતર અને ગોઠવવાનો સમાવેશ કરે છે.

\textbf{ફીચર સિલેક્શન} આગાહીઓમાં યોગદાન આપતા સૌથી સંબંધિત વેરિયેબલ્સ ઓળખે છે.

\textbf{મોડલ ટ્રેનિંગ} તૈયાર કરેલા ટ્રેનિંગ ડેટામાંથી પેટર્ન શીખવા માટે અલ્ગોરિધમ્સનો ઉપયોગ કરે છે.

\textbf{મોડલ ઇવેલ્યુએશન} ટ્રેન કરેલ મોડલ નવા, અદ્રશ્ય ડેટા પર કેટલી સારી કામગીરી કરે છે તેનું પરીક્ષણ કરે છે.

\begin{center}
\begin{tikzpicture}[node distance=1.5cm, auto]
    \node [gtu block] (Coll) {ડેટા કલેક્શન};
    \node [gtu block, below=0.8cm of Coll] (Prep) {ડેટા પ્રીપ્રોસેસિંગ};
    \node [gtu block, below=0.8cm of Prep] (Feat) {ફીચર સિલેક્શન};
    \node [gtu block, below=0.8cm of Feat] (Train) {મોડલ ટ્રેનિંગ};
    \node [gtu block, below=0.8cm of Train] (Eval) {મોડલ ઇવેલ્યુએશન};
    \node [gtu block, below=0.8cm of Eval] (Deploy) {ડિપ્લોયમેન્ટ};
    
    \path [gtu arrow] (Coll) -- (Prep);
    \path [gtu arrow] (Prep) -- (Feat);
    \path [gtu arrow] (Feat) -- (Train);
    \path [gtu arrow] (Train) -- (Eval);
    \path [gtu arrow] (Eval) -- (Deploy);
\end{tikzpicture}
\captionof{figure}{મશીન લર્નિંગ એક્ટિવિટીઝ}
\end{center}
\end{solutionbox}

\begin{mnemonicbox}
\mnemonic{કલેક્ટ પ્રોસેસ ફીચર ટ્રેન ઇવેલ્યુએટ ડિપ્લોય}
\end{mnemonicbox}

\questionmarks{2(a) અથવા}{3}{Predictive મોડલ અને Descriptive મોડલ વચ્ચેનો તફાવત આપો.}
\begin{solutionbox}
\begin{center}
\captionof{table}{Predictive વિ Descriptive મોડલ્સ}
\begin{tabulary}{\linewidth}{L L}
\hline
\textbf{Predictive મોડલ} & \textbf{Descriptive મોડલ} \\
\hline
\textbf{ભવિષ્યના} પરિણામોની આગાહી કરે છે & \textbf{વર્તમાન} પેટર્નનું સમજૂતી આપે છે \\
સુપરવાઇઝ્ડ લર્નિંગ વાપરે છે & અનસુપરવાઇઝ્ડ લર્નિંગ વાપરે છે \\
સ્ટોક પ્રાઇસ પ્રિડિક્શન & કસ્ટમર સેગમેન્ટેશન \\
\hline
\end{tabulary}
\end{center}

\textbf{Predictive મોડલ્સ} ભવિષ્યની ઘટનાઓ અથવા અજાણ્યા પરિણામોની આગાહી કરવા માટે ઐતિહાસિક ડેટાનો ઉપયોગ કરે છે.

\textbf{Descriptive મોડલ્સ} વર્તમાન પેટર્ન અને સંબંધોને સમજવા માટે હાલના ડેટાનું વિશ્લેષણ કરે છે.
\end{solutionbox}

\begin{mnemonicbox}
\mnemonic{Predict = ભવિષ્ય, Describe = વર્તમાન}
\end{mnemonicbox}

\questionmarks{2(b) અથવા}{4}{નીચે આપેલા ડેટાને યોગ્ય ડેટા ટાઇપની મદદથી classify કરો: hair color, gender, blood group type, time of day.}
\begin{solutionbox}
\begin{center}
\captionof{table}{ડેટા ટાઇપ ક્લાસિફિકેશન}
\begin{tabulary}{\linewidth}{L L L}
\hline
\textbf{ડેટા} & \textbf{પ્રકાર} & \textbf{કારણ} \\
\hline
\textbf{Hair color} & Nominal & કોઈ ક્રમ વિના કેટેગરીઓ \\
\textbf{Gender} & Nominal & કોઈ ક્રમ વિના કેટેગરીઓ \\
\textbf{Blood group} & Nominal & કોઈ ક્રમ વિના કેટેગરીઓ \\
\textbf{Time of day} & Continuous & માપી શકાય તેવી માત્રા \\
\hline
\end{tabulary}
\end{center}

\textbf{Nominal ડેટા} કોઈ કુદરતી ક્રમ વિના કેટેગરીઓનું પ્રતિનિધિત્વ કરે છે.

\textbf{Continuous ડેટા} શ્રેણીમાં કોઈપણ મૂલ્ય લઈ શકે છે અને માપી શકાય છે.
\end{solutionbox}

\begin{mnemonicbox}
\mnemonic{નામો = Nominal, સંખ્યાઓ = Numerical}
\end{mnemonicbox}

\questionmarks{2(c) અથવા}{7}{ડેટા પ્રી-પ્રોસેસિંગમાં ઉપયોગ થતી વિવિધ મેથડ્સ વર્ણવો.}
\begin{solutionbox}
\begin{center}
\captionof{table}{ડેટા પ્રીપ્રોસેસિંગ મેથડ્સ}
\begin{tabulary}{\linewidth}{L L L}
\hline
\textbf{મેથડ} & \textbf{હેતુ} & \textbf{ઉદાહરણ} \\
\hline
\textbf{ડેટા ક્લીનિંગ} & ભૂલો અને અસંગતતાઓ દૂર કરવી & ટાઇપોઝ ઠીક કરવા, ડુપ્લિકેટ્સ દૂર કરવા \\
\textbf{ડેટા ઇન્ટીગ્રેશન} & બહુવિધ સ્રોતો એકસાથે જોડવા & કસ્ટમર ડેટાબેસ મર્જ કરવા \\
\textbf{ડેટા ટ્રાન્સફોર્મેશન} & યોગ્ય ફોર્મેટમાં કન્વર્ટ કરવું & 0-1 મૂલ્યો નોર્મલાઇઝ કરવા \\
\textbf{ડેટા રિડક્શન} & ડેટાસેટનું કદ ઘટાડવું & મહત્વપૂર્ણ ફીચર્સ પસંદ કરવા \\
\hline
\end{tabulary}
\end{center}

\textbf{ડેટા ક્લીનિંગ} ભૂલભરેલ, અધૂરા અથવા અપ્રસ્તુત ડેટાને દૂર કરે છે અથવા સુધારે છે.

\textbf{ડેટા ઇન્ટીગ્રેશન} બહુવિધ સ્રોતોમાંથી ડેટાને એકીકૃત ડેટાસેટમાં જોડે છે.

\textbf{ડેટા ટ્રાન્સફોર્મેશન} વિશ્લેષણ માટે ડેટાને યોગ્ય ફોર્મેટમાં કન્વર્ટ કરે છે.

\textbf{ડેટા રિડક્શન} માહિતીની ગુણવત્તા જાળવીને ડેટાસેટનું કદ ઘટાડે છે.

\begin{center}
\begin{tikzpicture}[node distance=1.5cm, auto]
    \node [gtu block] (Raw) {કાચો ડેટા};
    \node [gtu block, right=of Raw] (Clean) {ડેટા ક્લીનિંગ};
    \node [gtu block, right=of Clean] (Integ) {ડેટા ઇન્ટીગ્રેશન};
    \node [gtu block, below=1cm of Integ] (Trans) {ડેટા ટ્રાન્સફોર્મેશન};
    \node [gtu block, left=of Trans] (Reduc) {ડેટા રિડક્શન};
    \node [gtu block, left=of Reduc] (Proc) {પ્રોસેસ થયેલ ડેટા};
    
    \path [gtu arrow] (Raw) -- (Clean);
    \path [gtu arrow] (Clean) -- (Integ);
    \path [gtu arrow] (Integ) -- (Trans);
    \path [gtu arrow] (Trans) -- (Reduc);
    \path [gtu arrow] (Reduc) -- (Proc);
\end{tikzpicture}
\captionof{figure}{ડેટા પ્રીપ્રોસેસિંગ પાઇપલાઇન}
\end{center}
\end{solutionbox}

\begin{mnemonicbox}
\mnemonic{ક્લીન ઇન્ટીગ્રેટ ટ્રાન્સફોર્મ રિડ્યુસ}
\end{mnemonicbox}

\questionmarks{3(a)}{3}{Classification અને Regression વચ્ચેનો તફાવત આપો.}
\begin{solutionbox}
\begin{center}
\captionof{table}{Classification વિ Regression}
\begin{tabulary}{\linewidth}{L L}
\hline
\textbf{Classification} & \textbf{Regression} \\
\hline
\textbf{ડિસ્ક્રીટ} આઉટપુટ & \textbf{કન્ટિન્યુઅસ} આઉટપુટ \\
કેટેગરીઓની આગાહી કરે છે & સંખ્યાત્મક મૂલ્યોની આગાહી કરે છે \\
ઈમેઇલ: સ્પામ/બિન-સ્પામ & ઘરની કિંમત આગાહી \\
\hline
\end{tabulary}
\end{center}

\textbf{Classification} ઇનપુટ ડેટામાંથી ડિસ્ક્રીટ કેટેગરીઓ અથવા ક્લાસની આગાહી કરે છે.

\textbf{Regression} ઇનપુટ ડેટામાંથી કન્ટિન્યુઅસ સંખ્યાત્મક મૂલ્યોની આગાહી કરે છે.
\end{solutionbox}

\begin{mnemonicbox}
\mnemonic{Class = કેટેગરીઓ, Regress = વાસ્તવિક સંખ્યાઓ}
\end{mnemonicbox}

\questionmarks{3(b)}{4}{યોગ્ય ઉદાહરણ લઈને confusion matrix લખો. તેના માટે accuracy અને error rate ગણો.}
\begin{solutionbox}
\textbf{ઉદાહરણ: ઈમેઇલ ક્લાસિફિકેશન}

\begin{center}
\captionof{table}{Confusion Matrix}
\begin{tabulary}{\linewidth}{L L L}
\hline
 & \textbf{પ્રિડિક્ટેડ સ્પામ} & \textbf{પ્રિડિક્ટેડ નોટ સ્પામ} \\
\hline
\textbf{વાસ્તવિક સ્પામ} & 85 (TP) & 15 (FN) \\
\textbf{વાસ્તવિક નોટ સ્પામ} & 10 (FP) & 90 (TN) \\
\hline
\end{tabulary}
\end{center}

\textbf{ગણતરીઓ:}
\begin{itemize}
    \item \textbf{Accuracy} = \((TP+TN)/(TP+TN+FP+FN) = (85+90)/200 = 87.5\%\)
    \item \textbf{Error Rate} = \((FP+FN)/(TP+TN+FP+FN) = (10+15)/200 = 12.5\%\)
\end{itemize}

\textbf{મુખ્ય શબ્દો:}
\begin{itemize}
    \item \textbf{TP}: True Positive - યોગ્ય રીતે સ્પામ આગાહી
    \item \textbf{TN}: True Negative - યોગ્ય રીતે નોટ સ્પામ આગાહી
\end{itemize}
\end{solutionbox}

\begin{mnemonicbox}
\mnemonic{True Positive True Negative = યોગ્ય આગાહીઓ}
\end{mnemonicbox}

\questionmarks{3(c)}{7}{KNN અલ્ગોરિધમ વિગતવાર વર્ણવો.}
\begin{solutionbox}
\textbf{K-Nearest Neighbors (KNN)} એક સરળ ક્લાસિફિકેશન અલ્ગોરિધમ છે જે તેમના K નજીકના પડોશીઓના મેજોરિટી ક્લાસના આધારે ડેટા પોઇન્ટ્સને ક્લાસિફાઇ કરે છે.

\begin{center}
\captionof{table}{KNN અલ્ગોરિધમ સ્ટેપ્સ}
\begin{tabulary}{\linewidth}{L L L}
\hline
\textbf{સ્ટેપ} & \textbf{વર્ણન} & \textbf{ઉદાહરણ} \\
\hline
\textbf{K પસંદ કરો} & પડોશીઓની સંખ્યા પસંદ કરો & K=3 \\
\textbf{અંતર ગણો} & બધા પોઇન્ટ્સનો અંતર શોધો & Euclidean અંતર \\
\textbf{પડોશીઓ શોધો} & K સૌથી નજીકના પોઇન્ટ્સ ઓળખો & 3 નજીકના પોઇન્ટ્સ \\
\textbf{વોટ કરો} & મેજોરિટી ક્લાસ જીતે છે & 2 બિલાડી, 1 કૂતરો \(\to\) બિલાડી \\
\hline
\end{tabulary}
\end{center}

\textbf{કામગીરી પ્રક્રિયા:}
\begin{enumerate}
    \item \textbf{અંતર ગણો} ટેસ્ટ પોઇન્ટ અને બધા ટ્રેનિંગ પોઇન્ટ્સ વચ્ચે
    \item \textbf{અંતર સોર્ટ કરો} અને K નજીકના પડોશીઓ પસંદ કરો
    \item \textbf{વોટ ગણો} પડોશીઓ વચ્ચે દરેક ક્લાસમાંથી
    \item \textbf{ક્લાસ અસાઇન કરો} મેજોરિટી વોટ સાથે
\end{enumerate}

\begin{center}
\begin{tikzpicture}[node distance=1.5cm, auto]
    \node [gtu block] (New) {નવો ડેટા પોઇન્ટ};
    \node [gtu block, right=of New] (Calc) {અંતર ગણો};
    \node [gtu block, right=of Calc] (Find) {K નજીકના શોધો};
    \node [gtu block, below=1cm of Find] (Vote) {મેજોરિટી વોટ};
    \node [gtu block, left=of Vote] (Pred) {ક્લાસ આગાહી કરો};
    
    \path [gtu arrow] (New) -- (Calc);
    \path [gtu arrow] (Calc) -- (Find);
    \path [gtu arrow] (Find) -- (Vote);
    \path [gtu arrow] (Vote) -- (Pred);
\end{tikzpicture}
\captionof{figure}{KNN પ્રોસેસ ફ્લો}
\end{center}

\textbf{ફાયદાઓ:}
\begin{itemize}
    \item \textbf{લાગુ કરવામાં સરળ} અને સમજવામાં આસાન
    \item \textbf{ટ્રેનિંગની જરૂર નથી} - આળસુ લર્નિંગ અલ્ગોરિધમ
\end{itemize}
\end{solutionbox}

\begin{mnemonicbox}
\mnemonic{K નજીકના પડોશીઓ ક્લાસિફિકેશન માટે વોટ કરે છે}
\end{mnemonicbox}

\questionmarks{3(a) અથવા}{3}{Multiple linear regression ની કોઈપણ ત્રણ ઉપયોગીતાઓ આપો.}
\begin{solutionbox}
\textbf{Multiple Linear Regression ની ઉપયોગીતાઓ:}

\begin{center}
\captionof{table}{ઉપયોગીતાઓ}
\begin{tabulary}{\linewidth}{L L L}
\hline
\textbf{ઉપયોગીતા} & \textbf{વેરિયેબલ્સ} & \textbf{હેતુ} \\
\hline
\textbf{ઘરની કિંમત આગાહી} & કદ, સ્થાન, ઉંમર & પ્રોપર્ટીની કિંમત અંદાજ \\
\textbf{સેલ્સ ફોરકાસ્ટિંગ} & જાહેરાત, સીઝન, કિંમત & આવકની આગાહી કરવી \\
\textbf{મેડિકલ ડાયગ્નોસિસ} & લક્ષણો, ઉંમર, ઇતિહાસ & જોખમ આકારણી \\
\hline
\end{tabulary}
\end{center}

\textbf{Multiple Linear Regression} એક કન્ટિન્યુઅસ આઉટપુટ વેરિયેબલની આગાહી કરવા માટે બહુવિધ ઇનપુટ વેરિયેબલ્સનો ઉપયોગ કરે છે.
\end{solutionbox}

\begin{mnemonicbox}
\mnemonic{બહુવિધ ઇનપુટ્સ, એક આઉટપુટ}
\end{mnemonicbox}

\questionmarks{3(b) અથવા}{4}{Bagging, boosting અને stacking વિગતવાર વર્ણવો.}
\begin{solutionbox}
\begin{center}
\captionof{table}{Ensemble મેથડ્સ}
\begin{tabulary}{\linewidth}{L L L}
\hline
\textbf{મેથડ} & \textbf{અભિગમ} & \textbf{ઉદાહરણ} \\
\hline
\textbf{Bagging} & પેરેલલ ટ્રેનિંગ, સરેરાશ પરિણામો & Random Forest \\
\textbf{Boosting} & સિક્વેન્શિયલ ટ્રેનિંગ, ભૂલોમાંથી શીખે & AdaBoost \\
\textbf{Stacking} & મેટા-લર્નર મોડલ્સ કન્બાઇન કરે & Neural network combiner \\
\hline
\end{tabulary}
\end{center}

\textbf{Bagging} વિવિધ ડેટા સબસેટ્સ પર બહુવિધ મોડલ્સને ટ્રેન કરે છે અને આગાહીઓની સરેરાશ કાઢે છે.

\textbf{Boosting} મોડલ્સને ક્રમિક રીતે ટ્રેન કરે છે, દરેક અગાઉના મોડલની ભૂલોમાંથી શીખે છે.

\textbf{Stacking} બેઝ મોડલ્સની આગાહીઓને કેવી રીતે કન્બાઇન કરવી તે શીખવા માટે મેટા-મોડલનો ઉપયોગ કરે છે.
\end{solutionbox}

\begin{mnemonicbox}
\mnemonic{Bag પેરેલલ, Boost સિક્વેન્શિયલ, Stack મેટા}
\end{mnemonicbox}

\questionmarks{3(c) અથવા}{7}{Single linear regression તેની ઉપયોગીતાઓ સાથે વર્ણવો.}
\begin{solutionbox}
\textbf{Single Linear Regression} એક ઇનપુટ વેરિયેબલ (X) અને એક આઉટપુટ વેરિયેબલ (Y) વચ્ચે શ્રેષ્ઠ સીધો રેખા સંબંધ શોધે છે.

\textbf{ફોર્મ્યુલા:} \(Y = a + bX\)
\begin{itemize}
    \item \textbf{a}: Y-intercept
    \item \textbf{b}: લાઇનનો Slope
\end{itemize}

\begin{center}
\captionof{table}{ઉપયોગ ઉદાહરણ - ઘરની કિંમત વિ કદ}
\begin{tabulary}{\linewidth}{L L}
\hline
\textbf{ઘરનું કદ (sq ft)} & \textbf{કિંમત (લાખ)} \\
\hline
1000 & 50 \\
1500 & 75 \\
2000 & 100 \\
\hline
\end{tabulary}
\end{center}

\textbf{કામકાજની પ્રક્રિયા:}
\begin{enumerate}
    \item \textbf{ડેટા એકત્રિત કરો} ઇનપુટ-આઉટપુટ જોડીઓ સાથે
    \item \textbf{પોઇન્ટ્સ પ્લોટ કરો} સ્કેટર ગ્રાફ પર
    \item \textbf{શ્રેષ્ઠ લાઇન શોધો} જે ભૂલ ન્યૂનતમ કરે
    \item \textbf{આગાહીઓ કરો} લાઇન સમીકરણ વાપરીને
\end{enumerate}

\begin{center}
\begin{tikzpicture}
    \begin{axis}[
        xlabel={કદ (sq ft)},
        ylabel={કિંમત (લાખ)},
        xmin=0, xmax=2500,
        ymin=0, ymax=120,
        axis lines=middle,
        grid=major,
        width=8cm,
        height=6cm,
        scatter/classes={a={mark=*,draw=black,fill=blue}}
    ]
    \addplot[scatter,only marks, scatter src=explicit symbolic]
    coordinates {
        (1000, 50) [a]
        (1500, 75) [a]
        (2000, 100) [a]
    };
    \addplot[domain=0:2500, color=red, thick] {0.05*x};
    \end{axis}
    \node [below] at (current axis.south) {રિગ્રેશન લાઇન સાથે સ્કેટર પ્લોટ};
\end{tikzpicture}
\captionof{figure}{લિનિયર રિગ્રેશન વિઝ્યુલાઇઝેશન}
\end{center}

\textbf{ઉપયોગીતાઓ:}
\begin{itemize}
    \item \textbf{સેલ્સ વિ જાહેરાત}: વધુ જાહેરાત \(\to\) વધુ સેલ્સ
    \item \textbf{તાપમાન વિ આઇસક્રીમ સેલ્સ}: ગરમ હવામાન \(\to\) વધુ સેલ્સ
\end{itemize}
\end{solutionbox}

\begin{mnemonicbox}
\mnemonic{એક X એક Y ની લાઇન સાથે આગાહી કરે છે}
\end{mnemonicbox}

\questionmarks{4(a)}{3}{વ્યાખ્યા આપો: (1)support (2)confidence.}
\begin{solutionbox}
\textbf{Support} માપે છે કે આઇટમસેટ ડેટાસેટમાં કેટલી વાર દેખાય છે.

\textbf{Confidence} માપે છે કે જ્યારે antecedent હાજર હોય ત્યારે consequent માં આઇટમ્સ કેટલી વાર દેખાય છે.

\begin{center}
\captionof{table}{વ્યાખ્યાઓ}
\begin{tabulary}{\linewidth}{L L L}
\hline
\textbf{માપદંડ} & \textbf{ફોર્મ્યુલા} & \textbf{ઉદાહરણ} \\
\hline
\textbf{Support} & Count(itemset)/કુલ transactions & બ્રેડ 60\% transactions માં દેખાય છે \\
\textbf{Confidence} & Support(A\(\cup\)B)/Support(A) & બ્રેડ ખરીદનારા 80\% લોકો બટર પણ ખરીદે છે \\
\hline
\end{tabulary}
\end{center}

\textbf{Support = આવૃત્તિની આવર્તન} \\
\textbf{Confidence = નિયમની વિશ્વસનીયતા}
\end{solutionbox}

\begin{mnemonicbox}
\mnemonic{Support = કેટલી વાર, Confidence = કેટલું વિશ્વસનીય}
\end{mnemonicbox}

\questionmarks{4(b)}{4}{Unsupervised learning ની ઉપયોગીતાઓ વર્ણવો.}
\begin{solutionbox}
\begin{center}
\captionof{table}{Unsupervised Learning ઉપયોગીતાઓ}
\begin{tabulary}{\linewidth}{L L L}
\hline
\textbf{ઉપયોગીતા} & \textbf{હેતુ} & \textbf{ઉદાહરણ} \\
\hline
\textbf{કસ્ટમર સેગમેન્ટેશન} & સમાન કસ્ટમર્સને જૂથબદ્ધ કરવા & માર્કેટિંગ કેમ્પેઇન્સ \\
\textbf{ડેટા કમ્પ્રેશન} & ડેટાનું કદ ઘટાડવું & ઇમેજ કમ્પ્રેશન \\
\textbf{અનોમલી ડિટેક્શન} & અસામાન્ય પેટર્ન શોધવા & ફ્રોડ ડિટેક્શન \\
\textbf{રેકમેન્ડેશન સિસ્ટમ્સ} & સમાન આઇટમ્સ સુઝાવવા & મ્યુઝિક રેકમેન્ડેશન્સ \\
\hline
\end{tabulary}
\end{center}

\textbf{કસ્ટમર સેગમેન્ટેશન} લક્ષિત માર્કેટિંગ માટે સમાન ખરીદી વર્તણૂક ધરાવતા કસ્ટમર્સને જૂથબદ્ધ કરે છે.

\textbf{ડેટા કમ્પ્રેશન} પેટર્ન શોધીને અને રિડન્ડન્સી દૂર કરીને સ્ટોરેજ સ્પેસ ઘટાડે છે.

\textbf{અનોમલી ડિટેક્શન} અસામાન્ય પેટર્ન ઓળખે છે જે ફ્રોડ અથવા ભૂલો સૂચવી શકે છે.
\end{solutionbox}

\begin{mnemonicbox}
\mnemonic{સેગમેન્ટ કમ્પ્રેસ ડિટેક્ટ રેકમેન્ડ}
\end{mnemonicbox}

\questionmarks{4(c)}{7}{Apriori અલ્ગોરિધમ યોગ્ય ઉદાહરણ સાથે વર્ણવો.}
\begin{solutionbox}
\textbf{Apriori Algorithm} માર્કેટ બાસ્કેટ એનાલિસિસ માટે ફ્રીક્વન્ટ આઇટમસેટ્સ શોધે છે અને એસોસિએશન રૂલ્સ જનરેટ કરે છે.

\begin{center}
\captionof{table}{અલ્ગોરિધમ સ્ટેપ્સ}
\begin{tabulary}{\linewidth}{L L L}
\hline
\textbf{સ્ટેપ} & \textbf{વર્ણન} & \textbf{ઉદાહરણ} \\
\hline
\textbf{ફ્રીક્વન્ટ 1-itemsets શોધો} & વ્યક્તિગત આઇટમ્સ ગણો & \{બ્રેડ\}:4, \{દૂધ\}:3 \\
\textbf{2-itemsets જનરેટ કરો} & ફ્રીક્વન્ટ આઇટમ્સ કન્બાઇન કરો & \{બ્રેડ,દૂધ\}:2 \\
\textbf{મિનિમમ સપોર્ટ લાગુ કરો} & ઇન્ફ્રીક્વન્ટ સેટ્સ ફિલ્ટર કરો & support \(\ge\) 50\% જો રાખો \\
\textbf{રૂલ્સ જનરેટ કરો} & if-then રૂલ્સ બનાવો & બ્રેડ \(\to\) દૂધ \\
\hline
\end{tabulary}
\end{center}

\textbf{ઉદાહરણ ડેટાસેટ:}
\begin{itemize}
    \item Transaction 1: \{બ્રેડ, દૂધ, ઈંડા\}
    \item Transaction 2: \{બ્રેડ, દૂધ\}
    \item Transaction 3: \{બ્રેડ, ઈંડા\}
    \item Transaction 4: \{દૂધ, ઈંડા\}
\end{itemize}

\textbf{કામકાજની પ્રક્રિયા:}
\begin{enumerate}
    \item \textbf{ડેટાબેઝ સ્કેન કરો} આઇટમ ફ્રીક્વન્સીઝ ગણવા માટે
    \item \textbf{કેન્ડિડેટ આઇટમસેટ્સ જનરેટ કરો} વધતા કદની
    \item \textbf{ઇન્ફ્રીક્વન્ટ આઇટમસેટ્સ પ્રૂન કરો} મિનિમમ સપોર્ટથી નીચે
    \item \textbf{એસોસિએશન રૂલ્સ જનરેટ કરો} ફ્રીક્વન્ટ આઇટમસેટ્સમાંથી
\end{enumerate}

\begin{center}
\begin{tikzpicture}[node distance=1.5cm, auto]
    \node [gtu block] (DB) {Transaction Database};
    \node [gtu block, right=of DB] (Freq1) {ફ્રીક્વન્ટ 1-set};
    \node [gtu block, right=of Freq1] (Gen2) {2-itemsets જનરેટ};
    \node [gtu block, below=1cm of Gen2] (MinSup) {મિન સપોર્ટ લાગુ};
    \node [gtu block, left=of MinSup] (Rules) {રૂલ્સ જનરેટ};
    
    \path [gtu arrow] (DB) -- (Freq1);
    \path [gtu arrow] (Freq1) -- (Gen2);
    \path [gtu arrow] (Gen2) -- (MinSup);
    \path [gtu arrow] (MinSup) -- (Rules);
\end{tikzpicture}
\captionof{figure}{Apriori અલ્ગોરિધમ સ્ટેપ્સ}
\end{center}
\end{solutionbox}

\begin{mnemonicbox}
\mnemonic{A-priori જ્ઞાન ફ્રીક્વન્ટ પેટર્ન શોધવામાં મદદ કરે છે}
\end{mnemonicbox}

\questionmarks{4(a) અથવા}{3}{Clustering અને Classification ના તફાવતની યાદી આપો.}
\begin{solutionbox}
\begin{center}
\captionof{table}{Clustering વિ Classification}
\begin{tabulary}{\linewidth}{L L}
\hline
\textbf{Clustering} & \textbf{Classification} \\
\hline
\textbf{અનસુપરવાઇઝ્ડ} લર્નિંગ & \textbf{સુપરવાઇઝ્ડ} લર્નિંગ \\
લેબલ કરેલ ડેટા નથી & લેબલ કરેલ ટ્રેનિંગ ડેટા વાપરે છે \\
સમાન ડેટાને જૂથબદ્ધ કરે છે & પૂર્વનિર્ધારિત લેબલ્સ અસાઇન કરે છે \\
\hline
\end{tabulary}
\end{center}

\textbf{Clustering} અનલેબલ ડેટામાં છુપાયેલા જૂથો શોધે છે.

\textbf{Classification} ટ્રેન કરેલા મોડલ્સ વાપરીને નવા ડેટાને જાણીતી કેટેગરીઓમાં અસાઇન કરે છે.
\end{solutionbox}

\begin{mnemonicbox}
\mnemonic{Cluster = અજાણ્યા જૂથો, Classify = જાણીતા લેબલ્સ}
\end{mnemonicbox}

\questionmarks{4(b) અથવા}{4}{Clustering ની પ્રોસેસ વિગતવાર વર્ણવો.}
\begin{solutionbox}
\begin{center}
\captionof{table}{Clustering પ્રોસેસ સ્ટેપ્સ}
\begin{tabulary}{\linewidth}{L L L}
\hline
\textbf{સ્ટેપ} & \textbf{વર્ણન} & \textbf{હેતુ} \\
\hline
\textbf{ડેટા પ્રિપેરેશન} & ડેટા સાફ અને નોર્મલાઇઝ કરો & ગુણવત્તાપૂર્ણ ઇનપુટ સુનિશ્ચિત કરવું \\
\textbf{ડિસ્ટન્સ મેટ્રિક} & સમાનતાનું માપ પસંદ કરો & Euclidean, Manhattan \\
\textbf{અલ્ગોરિધમ સિલેક્શન} & ક્લસ્ટરિંગ મેથડ પસંદ કરો & K-means, Hierarchical \\
\textbf{ક્લસ્ટર વેલિડેશન} & ક્લસ્ટર ગુણવત્તાનું મૂલ્યાંકન કરો & Silhouette score \\
\hline
\end{tabulary}
\end{center}

\textbf{Clustering પ્રોસેસ} તેમની લાક્ષણિકતાઓના આધારે સમાન ડેટા પોઇન્ટ્સને એકસાથે જૂથબદ્ધ કરે છે.

\textbf{મુખ્ય નિર્ણયોમાં ક્લસ્ટર્સની સંખ્યા અને યોગ્ય ડિસ્ટન્સ મેટ્રિક્સ પસંદ કરવાનો સમાવેશ થાય છે.}

\textbf{વેલિડેશન સુનિશ્ચિત કરે છે કે ક્લસ્ટર્સ અર્થપૂર્ણ અને સારી રીતે અલગ છે.}
\end{solutionbox}

\begin{mnemonicbox}
\mnemonic{પ્રિપેર ડિસ્ટન્સ અલ્ગોરિધમ વેલિડેટ}
\end{mnemonicbox}

\questionmarks{4(c) અથવા}{7}{K-means clustering અલ્ગોરિધમ યોગ્ય ઉદાહરણ સાથે વર્ણવો.}
\begin{solutionbox}
\textbf{K-means} વિથિન-ક્લસ્ટર સમ ઓફ સ્ક્વેર્સ ન્યૂનતમ કરીને ડેટાને K ક્લસ્ટર્સમાં વિભાજિત કરે છે.

\begin{center}
\captionof{table}{અલ્ગોરિધમ સ્ટેપ્સ}
\begin{tabulary}{\linewidth}{L L L}
\hline
\textbf{સ્ટેપ} & \textbf{વર્ણન} & \textbf{ઉદાહરણ} \\
\hline
\textbf{સેન્ટ્રોઇડ્સ ઇનિશિયલાઇઝ કરો} & રેન્ડમ K સેન્ટર પોઇન્ટ્સ & C1(2,3), C2(8,7) \\
\textbf{પોઇન્ટ્સ અસાઇન કરો} & દરેક પોઇન્ટ નજીકના સેન્ટ્રોઇડને & Point(1,2) \(\to\) C1 \\
\textbf{સેન્ટ્રોઇડ્સ અપડેટ કરો} & અસાઇન થયેલા પોઇન્ટ્સનો મીન & નવું C1(1.5, 2.5) \\
\textbf{રિપીટ કરો} & સેન્ટ્રોઇડ્સ હલનચલન બંધ ન થાય ત્યાં સુધી & કન્વર્જન્સ \\
\hline
\end{tabulary}
\end{center}

\textbf{ઉદાહરણ: કસ્ટમર આવક વિ ઉંમર}
\begin{itemize}
    \item કસ્ટમર 1: (આવક=30k, ઉંમર=25)
    \item કસ્ટમર 2: (આવક=35k, ઉંમર=30)
    \item કસ્ટમર 3: (આવક=70k, ઉંમર=45)
    \item કસ્ટમર 4: (આવક=75k, ઉંમર=50)
\end{itemize}

\textbf{કામકાજની પ્રક્રિયા:}
\begin{enumerate}
    \item \textbf{K=2 પસંદ કરો} યુવા/વૃદ્ધ કસ્ટમર્સ માટે ક્લસ્ટર્સ
    \item \textbf{સેન્ટ્રોઇડ્સ ઇનિશિયલાઇઝ કરો} રેન્ડમ રીતે
    \item \textbf{અંતર ગણો} દરેક કસ્ટમરથી સેન્ટ્રોઇડ્સ સુધી
    \item \textbf{કસ્ટમર્સ અસાઇન કરો} નજીકના સેન્ટ્રોઇડને
    \item \textbf{સેન્ટ્રોઇડ પોઝિશન્સ અપડેટ કરો} અસાઇન થયેલા કસ્ટમર્સના કેન્દ્રમાં
    \item \textbf{સ્થિર ન થાય ત્યાં સુધી રિપીટ કરો}
\end{enumerate}

\begin{center}
\begin{tikzpicture}[node distance=1.5cm, auto]
    \node [gtu block] (K) {K પસંદ કરો};
    \node [gtu block, right=of K] (Init) {સેન્ટ્રોઇડ્સ ઇનિશિયલાઇઝ};
    \node [gtu block, right=of Init] (Assign) {પોઇન્ટ્સ અસાઇન};
    \node [gtu block, below=1cm of Assign] (Update) {સેન્ટ્રોઇડ્સ અપડેટ};
    \node [diamond, draw, aspect=2, left=of Update] (Check) {કન્વર્જ?};
    \node [gtu block, left=of Check] (Final) {અંતિમ ક્લસ્ટર્સ};
    
    \path [gtu arrow] (K) -- (Init);
    \path [gtu arrow] (Init) -- (Assign);
    \path [gtu arrow] (Assign) -- (Update);
    \path [gtu arrow] (Update) -- (Check);
    \path [gtu arrow] (Check) -- node[above] {હા} (Final);
    \path [gtu arrow] (Check.north) |- node[near start, right] {ના} (Assign.south);
\end{tikzpicture}
\captionof{figure}{K-Means લોજિક}
\end{center}
\end{solutionbox}

\begin{mnemonicbox}
\mnemonic{K સેન્ટ્રોઇડ્સ તેમના અસાઇન થયેલા પોઇન્ટ્સનો મીન કરે છે}
\end{mnemonicbox}

\questionmarks{5(a)}{3}{Matplotlib ની ઉપયોગીતાઓની યાદી આપો.}
\begin{solutionbox}
\begin{center}
\captionof{table}{Matplotlib ઉપયોગીતાઓ}
\begin{tabulary}{\linewidth}{L L L}
\hline
\textbf{ઉપયોગીતા} & \textbf{હેતુ} & \textbf{ઉદાહરણ} \\
\hline
\textbf{ડેટા વિઝ્યુલાઇઝેશન} & ચાર્ટ્સ અને ગ્રાફ્સ બનાવવા & બાર ચાર્ટ્સ, હિસ્ટોગ્રામ્સ \\
\textbf{સાયન્ટિફિક પ્લોટિંગ} & સંશોધન પ્રેઝન્ટેશન્સ & ગાણિતિક ફંક્શન્સ \\
\textbf{ડેશબોર્ડ ક્રિએશન} & ઇન્ટરેક્ટિવ ડિસ્પ્લે & બિઝનેસ મેટ્રિક્સ \\
\hline
\end{tabulary}
\end{center}

\textbf{Matplotlib} સ્ટેટિક, એનિમેટેડ અને ઇન્ટરેક્ટિવ વિઝ્યુલાઇઝેશન્સ બનાવવા માટે Python ની પ્રાથમિક પ્લોટિંગ લાઇબ્રેરી છે.
\end{solutionbox}

\begin{mnemonicbox}
\mnemonic{Mat-plot-lib = ગાણિત પ્લોટિંગ લાઇબ્રેરી}
\end{mnemonicbox}

\questionmarks{5(b)}{4}{હોરિઝોન્ટલ અને વર્ટિકલ લાઇન પ્લોટ કરવાનો કોડ matplotlib ની મદદથી લખો.}
\begin{solutionbox}
\begin{lstlisting}[language=Python]
import matplotlib.pyplot as plt

# ફિગર બનાવો
plt.figure(figsize=(8, 6))

# x=3 પર વર્ટિકલ લાઇન પ્લોટ કરો
plt.axvline(x=3, color='red', linestyle='--', label='વર્ટિકલ લાઇન')

# y=2 પર હોરિઝોન્ટલ લાઇન પ્લોટ કરો
plt.axhline(y=2, color='blue', linestyle='-', label='હોરિઝોન્ટલ લાઇન')

# લેબલ્સ અને ટાઇટલ ઉમેરો
plt.xlabel('X-અક્ષ')
plt.ylabel('Y-અક્ષ')
plt.title('વર્ટિકલ અને હોરિઝોન્ટલ લાઇન્સ')
plt.legend()
plt.grid(True)
plt.show()
\end{lstlisting}

\textbf{મુખ્ય ફંક્શન્સ:}
\begin{itemize}
    \item \code{axvline()}: વર્ટિકલ લાઇન બનાવે છે
    \item \code{axhline()}: હોરિઝોન્ટલ લાઇન બનાવે છે
\end{itemize}
\end{solutionbox}

\begin{mnemonicbox}
\mnemonic{axvline = વર્ટિકલ, axhline = હોરિઝોન્ટલ}
\end{mnemonicbox}

\questionmarks{5(c)}{7}{Scikit-Learn ની વિશેષતાઓ અને ઉપયોગીતાઓ સમજાવો.}
\begin{solutionbox}
\begin{center}
\captionof{table}{Scikit-Learn વિશેષતાઓ}
\begin{tabulary}{\linewidth}{L L L}
\hline
\textbf{વિશેષતા} & \textbf{વર્ણન} & \textbf{ઉદાહરણ} \\
\hline
\textbf{સરળ API} & ઉપયોગમાં સરળ ઇન્ટરફેસ & fit(), predict() \\
\textbf{બહુવિધ અલ્ગોરિધમ્સ} & વિવિધ ML મેથડ્સ & SVM, Random Forest \\
\textbf{ડેટા પ્રીપ્રોસેસિંગ} & બિલ્ટ-ઇન ડેટા ટૂલ્સ & StandardScaler \\
\textbf{મોડલ ઇવેલ્યુએશન} & પરફોર્મન્સ મેટ્રિક્સ & accuracy\_score \\
\hline
\end{tabulary}
\end{center}

\textbf{Scikit-Learn} ડેટા એનાલિસિસ માટે સરળ ટૂલ્સ પ્રદાન કરતી Python ની સૌથી લોકપ્રિય મશીન લર્નિંગ લાઇબ્રેરી છે.

\textbf{ઉપયોગીતાઓ:}
\begin{itemize}
    \item \textbf{ક્લાસિફિકેશન}: ઈમેઇલ સ્પામ ડિટેક્શન
    \item \textbf{રિગ્રેશન}: ઘરની કિંમત આગાહી
    \item \textbf{ક્લસ્ટરિંગ}: કસ્ટમર સેગમેન્ટેશન
    \item \textbf{ડાયમેન્શનાલિટી રિડક્શન}: ડેટા વિઝ્યુલાઇઝેશન
\end{itemize}

\begin{center}
\begin{tikzpicture}[node distance=1.5cm]
    \node [gtu block] (SK) {Scikit-Learn};
    
    \node [gtu block, below left=1.5cm and 0.5cm of SK] (Class) {ક્લાસિફિકેશન};
    \node [gtu block, below left=1.5cm and -1.5cm of SK] (Reg) {રિગ્રેશન};
    \node [gtu block, below right=1.5cm and -1.5cm of SK] (Clust) {ક્લસ્ટરિંગ};
    \node [gtu block, below right=1.5cm and 0.5cm of SK] (Prep) {પ્રીપ્રોસેસિંગ};
    
    \node [gtu state, below=0.8cm of Class, text width=2cm] (C_Ex) {SVM\\Trees};
    \node [gtu state, below=0.8cm of Reg, text width=2cm] (R_Ex) {Linear\\Poly};
    \node [gtu state, below=0.8cm of Clust, text width=2cm] (Cl_Ex) {K-means\\DBSCAN};
    \node [gtu state, below=0.8cm of Prep, text width=2cm] (P_Ex) {Scale\\Encode};
    
    \path [gtu arrow] (SK) -- (Class);
    \path [gtu arrow] (SK) -- (Reg);
    \path [gtu arrow] (SK) -- (Clust);
    \path [gtu arrow] (SK) -- (Prep);
    
    \path [gtu arrow] (Class) -- (C_Ex);
    \path [gtu arrow] (Reg) -- (R_Ex);
    \path [gtu arrow] (Clust) -- (Cl_Ex);
    \path [gtu arrow] (Prep) -- (P_Ex);
\end{tikzpicture}
\captionof{figure}{Scikit-Learn કમ્પોનન્ટ્સ}
\end{center}
\end{solutionbox}

\begin{mnemonicbox}
\mnemonic{Scikit = મશીન લર્નિંગ માટે સાયન્સ કિટ}
\end{mnemonicbox}

\questionmarks{5(a) અથવા}{3}{NumPy નો મશીન લર્નિંગના સંદર્ભમાં ઉપયોગ આપો.}
\begin{solutionbox}
\begin{center}
\captionof{table}{ML માં NumPy નો હેતુ}
\begin{tabulary}{\linewidth}{L L L}
\hline
\textbf{હેતુ} & \textbf{વર્ણન} & \textbf{ફાયદો} \\
\hline
\textbf{ન્યુમેરિકલ કમ્પ્યુટિંગ} & ઝડપી array ઓપરેશન્સ & કાર્યક્ષમ ગણતરીઓ \\
\textbf{ફાઉન્ડેશન લાઇબ્રેરી} & અન્ય લાઇબ્રેરીઓ માટે આધાર & Pandas, Scikit-learn તેનો ઉપયોગ કરે છે \\
\textbf{ગાણિતિક ફંક્શન્સ} & બિલ્ટ-ઇન મેથ ઓપરેશન્સ & સ્ટેટિસ્ટિક્સ, લિનિયર આલ્જીબ્રા \\
\hline
\end{tabulary}
\end{center}

\textbf{NumPy} Python મશીન લર્નિંગ એપ્લિકેશન્સમાં ન્યુમેરિકલ કમ્પ્યુટિંગ માટે પાયો પ્રદાન કરે છે.
\end{solutionbox}

\begin{mnemonicbox}
\mnemonic{Num-Py = ન્યુમેરિકલ Python}
\end{mnemonicbox}

\questionmarks{5(b) અથવા}{4}{csv ફાઈલને pandas માં ઇમ્પોર્ટ કરવાના સ્ટેપ લખો.}
\begin{solutionbox}
\begin{lstlisting}[language=Python]
import pandas as pd

# સ્ટેપ 1: pandas લાઇબ્રેરી ઇમ્પોર્ટ કરો
# સ્ટેપ 2: read_csv() ફંક્શન વાપરો
data = pd.read_csv('filename.csv')

# સ્ટેપ 3: પ્રથમ કેટલીક પંક્તિઓ ડિસ્પ્લે કરો
print(data.head())

# વૈકલ્પિક: પેરામીટર્સ સ્પેસિફાઇ કરો
data = pd.read_csv('file.csv', 
                   delimiter=',',
                   header=0,
                   index_col=0)
\end{lstlisting}

\textbf{સ્ટેપ્સ:}
\begin{enumerate}
    \item \textbf{pandas ઇમ્પોર્ટ કરો} લાઇબ્રેરી
    \item \textbf{read\_csv() વાપરો} ફાઇલનેમ સાથે
    \item \textbf{ડેટા વેરિફાઇ કરો} head() મેથડ સાથે
\end{enumerate}
\end{solutionbox}

\begin{mnemonicbox}
\mnemonic{ઇમ્પોર્ટ રીડ વેરિફાઇ}
\end{mnemonicbox}

\questionmarks{5(c) અથવા}{7}{Pandas ની વિશેષતાઓ અને ઉપયોગીતાઓ સમજાવો.}
\begin{solutionbox}
\begin{center}
\captionof{table}{Pandas વિશેષતાઓ}
\begin{tabulary}{\linewidth}{L L L}
\hline
\textbf{વિશેષતા} & \textbf{વર્ણન} & \textbf{ઉદાહરણ} \\
\hline
\textbf{ડેટા સ્ટ્રક્ચર્સ} & DataFrame અને Series & ટેબ્યુલર ડેટા હેન્ડલિંગ \\
\textbf{ડેટા I/O} & બહુવિધ ફોર્મેટ્સ રીડ/રાઇટ & CSV, Excel, JSON \\
\textbf{ડેટા ક્લીનિંગ} & મિસિંગ વેલ્યુઝ હેન્ડલ કરવા & dropna(), fillna() \\
\textbf{ડેટા એનાલિસિસ} & સ્ટેટિસ્ટિકલ ઓપરેશન્સ & groupby(), describe() \\
\hline
\end{tabulary}
\end{center}

\textbf{Pandas} મશીન લર્નિંગ પ્રોજેક્ટ્સમાં Python માં પ્રાથમિક ડેટા મેનિપ્યુલેશન લાઇબ્રેરી છે.

\textbf{મુખ્ય ક્ષમતાઓ:}
\begin{itemize}
    \item \textbf{ડેટા લોડિંગ} વિવિધ ફાઇલ ફોર્મેટ્સમાંથી
    \item \textbf{ડેટા ક્લીનિંગ} અને પ્રીપ્રોસેસિંગ ઓપરેશન્સ
    \item \textbf{ડેટા ટ્રાન્સફોર્મેશન} અને રીશેપિંગ
    \item \textbf{સ્ટેટિસ્ટિકલ એનાલિસિસ} અને એગ્રિગેશન
\end{itemize}

\begin{center}
\begin{tikzpicture}[node distance=1.5cm, auto]
    \node [gtu block] (Raw) {કાચો ડેટા};
    \node [gtu block, right=of Raw] (DF) {Pandas DataFrame};
    \node [gtu block, right=of DF] (Clean) {ક્લીનિંગ};
    \node [gtu block, below=1cm of Clean] (Anal) {ડેટા એનાલિસિસ};
    \node [gtu block, left=of Anal] (Feat) {ફીચર એન્જિન.};
    \node [gtu block, left=of Feat] (Ready) {ML તૈયાર ડેટા};
    
    \path [gtu arrow] (Raw) -- (DF);
    \path [gtu arrow] (DF) -- (Clean);
    \path [gtu arrow] (Clean) -- (Anal);
    \path [gtu arrow] (Anal) -- (Feat);
    \path [gtu arrow] (Feat) -- (Ready);
\end{tikzpicture}
\captionof{figure}{Pandas વર્કફ્લો}
\end{center}
\end{solutionbox}

\begin{mnemonicbox}
\mnemonic{Pandas = એનાલિસિસ માટે પેનલ ડેટા}
\end{mnemonicbox}

\end{document}
