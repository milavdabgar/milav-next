\documentclass{article}

% content/resources/templates/preamble.tex
\usepackage[margin=0.6in]{geometry}
\author{Milav Dabgar}
\usepackage{amsmath,amssymb,amsthm}
\usepackage{booktabs}
\usepackage{multirow}
\usepackage{xcolor}
\usepackage{tcolorbox}
\tcbuselibrary{breakable,skins}
\usepackage[colorlinks=true,linkcolor=blue]{hyperref}
\usepackage{titlesec}
\usepackage{enumitem}
\usepackage{tikz}
\usepackage{pgfplots}
\usepackage{circuitikz}
\usepackage[version=4]{mhchem}
\usepackage{longtable}
\usepackage{array}
\usepackage{float}
\usepackage{caption}
\usepackage{listings}

\lstset{
  basicstyle=\small\ttfamily,
  breaklines=true,
  breakatwhitespace=false,
  postbreak=\mbox{\textcolor{red}{$\hookrightarrow$}\space},
  float=false,
  numbers=left,
  numberstyle=\tiny\color{gray},
  numbersep=10pt,
  xleftmargin=2em,
  keywordstyle=\color{blue},
  commentstyle=\color{green!60!black},
  stringstyle=\color{purple},
  backgroundcolor=\color{gray!5},
  showstringspaces=false,
  tabsize=2,
  captionpos=b,
  keepspaces=true,
  columns=flexible
}

\pgfplotsset{compat=1.18}
\usetikzlibrary{shapes,arrows,positioning,calc,patterns,decorations.pathmorphing,decorations.markings,arrows.meta}

% Color scheme
\definecolor{headcolor}{RGB}{0,102,204}
\definecolor{keycolor}{RGB}{220,20,60}
\definecolor{solutioncolor}{RGB}{34,139,34}
\definecolor{mnemoniccolor}{RGB}{148,0,211}
\definecolor{codecolor}{RGB}{0,0,100}

% Spacing
\setlength{\parskip}{3pt}
\setlist[itemize]{nosep}
\setlist[enumerate]{nosep}

% Title formatting
\titleformat{\section}{\Large\bfseries\color{headcolor}}{\thesection}{1em}{}
\titleformat{\subsection}{\large\bfseries\color{headcolor}}{\thesubsection}{1em}{}

% Pandoc tightlist compatibility
\providecommand{\tightlist}{%
  \setlength{\itemsep}{0pt}\setlength{\parskip}{0pt}}

% Pandoc longtable compatibility
\newcounter{none}
\def\thenone{}


% content/resources/templates/gujarati-boxes.tex
\usepackage{fontspec}
\usepackage{polyglossia}

% Set Gujarati as main language (document is primarily in Gujarati)
% Note: gloss-gujarati.ldf doesn't exist in polyglossia, but it will use hyphenation patterns
\setdefaultlanguage{gujarati}
\setotherlanguage{english}

% Configure Gujarati font properly
% Use Language=Default to prevent polyglossia from trying to add language-specific features
% that don't exist for Gujarati, which causes "empty feature" warnings
\newfontfamily\gujaratifont[Script=Gujarati,AutoFakeBold=2.5,AutoFakeSlant=0.3]{Noto Sans Gujarati}
\setmainfont[Script=Gujarati,AutoFakeBold=2.5,AutoFakeSlant=0.3]{Noto Sans Gujarati}
% Use Noto Sans Gujarati for monospace to support Gujarati in text
\setmonofont[Scale=0.9]{Noto Sans Gujarati}

% Configure English to use the same font
\newfontfamily\englishfont[Script=Gujarati,AutoFakeBold=2.5,AutoFakeSlant=0.3]{Noto Sans Gujarati}

% Translations for polyglossia
\gappto\captionsgujarati{
  \renewcommand{\tablename}{કોષ્ટક}
  \renewcommand{\figurename}{આકૃતિ}
}

% Helper for TikZ nodes to ensure Gujarati font
\newcommand{\gu}[1]{{\gujaratifont #1}}

% Custom environments
\newtcolorbox{solutionbox}{
    breakable,
    enhanced,
    colback=solutioncolor!5!white,
    colframe=solutioncolor!75!black,
    fonttitle=\bfseries,
    title=જવાબ
}

\newtcolorbox{solutionboxnobreak}{
 colback=solutioncolor!5!white,
 colframe=solutioncolor!75!black,
 fonttitle=\bfseries,
 title=જવાબ
}

\newtcolorbox{keyformula}{
 breakable,
 enhanced,
 colback=keycolor!5!white,
 colframe=keycolor!75!black,
 fonttitle=\bfseries,
 title=રાસાયણિક સમીકરણ/સૂત્ર
}

\newtcolorbox{mnemonicbox}{
 breakable,
 enhanced,
 colback=mnemoniccolor!5!white,
 colframe=mnemoniccolor!75!black,
 fonttitle=\bfseries,
 title=મેમરી ટ્રીક
}


% Custom commands for GTU solutions
% This file defines semantic commands for consistent formatting

% Question command with automatic formatting
\newcommand{\question}[2]{%
  \section*{Question #1}%
  \textbf{#2}%
}

% OR question variant
\newcommand{\questionor}[2]{%
  \section*{Question #1 OR}%
  \textbf{#2}%
}

% Proper table environment with caption
\newenvironment{answertable}[1]{%
  \begin{table}[htbp]
  \centering
  \caption{#1}
}{%
  \end{table}
}

% Proper figure environment for diagrams
\newenvironment{answerdiagram}[1]{%
  \begin{figure}[htbp]
  \centering
  \caption{#1}
}{%
  \end{figure}
}

% Semantic markup for key terms
\newcommand{\keyword}[1]{\textbf{#1}}
\newcommand{\code}[1]{\texttt{#1}}
\newcommand{\classname}[1]{\texttt{#1}}
\newcommand{\methodname}[1]{\texttt{#1}}

% Proper quotation marks
\newcommand{\mnemonic}[1]{``#1''}


\title{Fundamentals of Machine Learning (4341603) - Summer 2024 Solution}
\date{June 15, 2024}

\begin{document}
\maketitle

\questionmarks{1(a)}{3}{Define Machine Learning using suitable example?}
\begin{solutionbox}

મશીન લર્નિંગ આર્ટિફિશિયલ ઇન્ટેલિજન્સનો એક ભાગ છે જે કમ્પ્યુટર્સને ડેટામાંથી શીખવા અને દરેક કાર્ય માટે સ્પષ્ટ રીતે પ્રોગ્રામ કર્યા વિના નિર્ણયો લેવા માટે સક્ષમ બનાવે છે.

\begin{center}
\captionof{table}{મશીન લર્નિંગના મુખ્ય ઘટકો}
\begin{tabulary}{\linewidth}{L L}
\hline
\textbf{ઘટક} & \textbf{વર્ણન} \\
\hline
\textbf{Data} & ટ્રેનિંગ માટે ઉપયોગમાં લેવાતી ઇનપુટ માહિતી \\
\textbf{Algorithm} & પેટર્ન શીખતા ગાણિતિક મોડેલ \\
\textbf{Training} & અલ્ગોરિધમને શીખવવાની પ્રક્રિયા \\
\textbf{Prediction} & શીખેલા પેટર્ન આધારિત આઉટપુટ \\
\hline
\end{tabulary}
\end{center}

\textbf{ઉદાહરણ}: ઇમેઇલ સ્પામ ડિટેક્શન સિસ્ટમ હજારો ઇમેઇલોમાંથી "Spam" અથવા "Not Spam" તરીકે લેબલ કરેલા ઇમેઇલોમાંથી શીખે છે અને નવા ઇમેઇલોને આપોઆપ વર્ગીકૃત કરે છે.

\end{solutionbox}

\begin{mnemonicbox}
\mnemonic{Data Drives Decisions - ડેટા અલ્ગોરિધમને બુદ્ધિશાળી નિર્ણયો લેવા માટે પ્રશિક્ષિત કરે છે}
\end{mnemonicbox}

\questionmarks{1(b)}{4}{Explain the process of machine learning with the help of schematic representation}
\begin{solutionbox}

મશીન લર્નિંગ પ્રક્રિયામાં ડેટા સંગ્રહથી લઈને મોડેલ ડિપ્લોયમેન્ટ સુધીના વ્યવસ્થિત પગલાંઓનો સમાવેશ થાય છે.

\begin{center}
\begin{tikzpicture}[node distance=1.5cm, auto]
    \node [gtu block] (Data) {ડેટા સંગ્રહ};
    \node [gtu block, below=0.8cm of Data] (Pre) {ડેટા પ્રીપ્રોસેસિંગ};
    \node [gtu block, below=0.8cm of Pre] (Feature) {ફીચર સિલેક્શન};
    \node [gtu block, below=0.8cm of Feature] (Model) {મોડેલ સિલેક્શન};
    \node [gtu block, below=0.8cm of Model] (Train) {ટ્રેનિંગ};
    \node [gtu block, below=0.8cm of Train] (Valid) {વેલિડેશન};
    \node [diamond, draw, aspect=2, below=1cm of Valid, align=center] (Check) {પરફોર્મન્સ\\સારું?};
    \node [gtu block, below=1.5cm of Check] (Test) {ટેસ્ટિંગ};
    \node [gtu block, below=0.8cm of Test] (Deploy) {ડિપ્લોયમેન્ટ};

    \path [gtu arrow] (Data) -- (Pre);
    \path [gtu arrow] (Pre) -- (Feature);
    \path [gtu arrow] (Feature) -- (Model);
    \path [gtu arrow] (Model) -- (Train);
    \path [gtu arrow] (Train) -- (Valid);
    \path [gtu arrow] (Valid) -- (Check);
    \path [gtu arrow] (Check) -- node[right] {હા} (Test);
    \path [gtu arrow] (Check.east) -- ++(1,0) |- node[near start, right] {ના} (Model.east);
    \path [gtu arrow] (Test) -- (Deploy);
\end{tikzpicture}
\captionof{figure}{મશીન લર્નિંગ પ્રક્રિયા}
\end{center}

\textbf{પ્રક્રિયાના પગલાં:}
\begin{itemize}
    \item \keyword{Data Collection}: સંબંધિત ડેટાસેટ એકત્રિત કરવું
    \item \keyword{Preprocessing}: ડેટાને સાફ અને તૈયાર કરવું
    \item \keyword{Training}: ટ્રેનિંગ ડેટાનો ઉપયોગ કરીને અલ્ગોરિધમને શીખવવું
    \item \keyword{Validation}: મોડેલની કામગીરીને ચકાસવી
    \item \keyword{Deployment}: વાસ્તવિક પ્રિડિક્શન માટે મોડેલનો ઉપયોગ
\end{itemize}

\end{solutionbox}

\begin{mnemonicbox}
\mnemonic{Computers Can Truly Think - Collect, Clean, Train, Test}
\end{mnemonicbox}

\questionmarks{1(c)}{7}{Explain different types of machine learning with suitable application.}
\begin{solutionbox}

મશીન લર્નિંગ અલ્ગોરિધમ્સને લર્નિંગ એપ્રોચ અને ઉપલબ્ધ ડેટાના આધારે વર્ગીકૃત કરવામાં આવે છે.

\begin{center}
\captionof{table}{મશીન લર્નિંગના પ્રકારો}
\begin{tabulary}{\linewidth}{L L L L}
\hline
\textbf{પ્રકાર} & \textbf{લર્નિંગ મેથડ} & \textbf{ડેટા આવશ્યકતા} & \textbf{ઉદાહરણ એપ્લિકેશન} \\
\hline
\textbf{Supervised} & લેબલ્ડ ડેટાનો ઉપયોગ & ઇનપુટ-આઉટપુટ જોડીઓ & ઇમેઇલ ક્લાસિફિકેશન \\
\textbf{Unsupervised} & છુપાયેલા પેટર્ન શોધે & માત્ર ઇનપુટ ડેટા & કસ્ટમર સેગમેન્ટેશન \\
\textbf{Reinforcement} & રિવોર્ડ્સ દ્વારા શીખે & એન્વાયર્નમેન્ટ ફીડબેક & ગેમ પ્લેઇંગ AI \\
\hline
\end{tabulary}
\end{center}

\textbf{એપ્લિકેશન્સ:}
\begin{itemize}
    \item \textbf{Supervised Learning}: મેડિકલ ડાયગ્નોસિસ, ઇમેજ રેકોગ્નિશન, ફ્રોડ ડિટેક્શન
    \item \textbf{Unsupervised Learning}: માર્કેટ રિસર્ચ, એનોમેલી ડિટેક્શન, રેકમેન્ડેશન સિસ્ટમ્સ
    \item \textbf{Reinforcement Learning}: ઓટોનોમસ વેહિકલ્સ, રોબોટિક્સ, સ્ટ્રેટેજિક ગેમ્સ
\end{itemize}

\begin{center}
\begin{tikzpicture}[node distance=1.5cm]
    \node [gtu block] (Root) {મશીન લર્નિંગ};
    
    \node [gtu block, below left=1.5cm and 1cm of Root] (Super) {Supervised};
    \node [gtu block, below=1.5cm of Root] (Unsuper) {Unsupervised};
    \node [gtu block, below right=1.5cm and 1cm of Root] (Reinf) {Reinforcement};
    
    \node [gtu state, below=0.8cm of Super, text width=2.5cm] (S1) {Classification\\Regression};
    \node [gtu state, below=0.8cm of Unsuper, text width=2.5cm] (U1) {Clustering\\Association};
    \node [gtu state, below=0.8cm of Reinf, text width=2.5cm] (R1) {Policy Learning\\Value Function};

    \path [gtu arrow] (Root) -- (Super);
    \path [gtu arrow] (Root) -- (Unsuper);
    \path [gtu arrow] (Root) -- (Reinf);
    \path [gtu arrow] (Super) -- (S1);
    \path [gtu arrow] (Unsuper) -- (U1);
    \path [gtu arrow] (Reinf) -- (R1);
\end{tikzpicture}
\captionof{figure}{મશીન લર્નિંગના પ્રકારો}
\end{center}

\end{solutionbox}

\begin{mnemonicbox}
\mnemonic{Students Usually Remember - Supervised, Unsupervised, Reinforcement}
\end{mnemonicbox}

\questionmarks{1(c) OR}{7}{What are various issues with machine learning? List three problems that are not to be solved using machine learning.}
\begin{solutionbox}

\begin{center}
\captionof{table}{મશીન લર્નિંગની સમસ્યાઓ}
\begin{tabulary}{\linewidth}{L L L}
\hline
\textbf{સમસ્યા કેટેગરી} & \textbf{વર્ણન} & \textbf{અસર} \\
\hline
\textbf{Data Quality} & અધૂરો, નોઇઝી, પક્ષપાતી ડેટા & નબળું મોડેલ પરફોર્મન્સ \\
\textbf{Overfitting} & મોડેલ ટ્રેનિંગ ડેટાને યાદ રાખે છે & નબળું જનરલાઇઝેશન \\
\textbf{Computational} & ઉચ્ચ પ્રોસેસિંગ આવશ્યકતાઓ & રિસોર્સ મર્યાદાઓ \\
\textbf{Interpretability} & બ્લેક બોક્સ મોડેલ્સ & પારદર્શિતાનો અભાવ \\
\hline
\end{tabulary}
\end{center}

\textbf{ML માટે અનુપયુક્ત સમસ્યાઓ:}
\begin{enumerate}
    \item \textbf{Simple rule-based tasks} - મૂળભૂત ગણતરીઓ, સિમ્પલ if-then લોજિક
    \item \textbf{Ethical decisions} - માનવીય મૂલ્યોની આવશ્યકતા ધરાવતા નૈતિક નિર્ણયો
    \item \textbf{Creative expression} - માનવીય લાગણીની આવશ્યકતા ધરાવતી મૂળ કલાત્મક સર્જના
\end{enumerate}

\textbf{અન્ય સમસ્યાઓ:}
\begin{itemize}
    \item \keyword{Privacy concerns}: સંવેદનશીલ ડેટા હેન્ડલિંગ
    \item \keyword{Bias propagation}: અન્યાયકારક અલ્ગોરિધમિક નિર્ણયો
    \item \keyword{Feature selection}: સંબંધિત ઇનપુટ વેરિએબલ્સ પસંદ કરવા
\end{itemize}

\end{solutionbox}

\begin{mnemonicbox}
\mnemonic{Data Drives Quality - ડેટા ક્વોલિટી સીધી રીતે મોડેલ ક્વોલિટીને અસર કરે છે}
\end{mnemonicbox}

\questionmarks{2(a)}{3}{Give a summarized view of different types of data in a typical machine learning problem.}
\begin{solutionbox}

\begin{center}
\captionof{table}{મશીન લર્નિંગમાં ડેટા પ્રકારો}
\begin{tabulary}{\linewidth}{L L L}
\hline
\textbf{ડેટા પ્રકાર} & \textbf{વર્ણન} & \textbf{ઉદાહરણ} \\
\hline
\textbf{Numerical} & માત્રાત્મક મૂલ્યો & ઉંમર: 25, ઊંચાઈ: 170cm \\
\textbf{Categorical} & અસ્પષ્ટ કેટેગરીઓ & રંગ: લાલ, વાદળી, લીલો \\
\textbf{Ordinal} & ક્રમબદ્ધ કેટેગરીઓ & રેટિંગ: નબળું, સારું, ઉત્તમ \\
\textbf{Binary} & બે શક્ય મૂલ્યો & લિંગ: પુરુષ/સ્ત્રી \\
\hline
\end{tabulary}
\end{center}

\textbf{લક્ષણો:}
\begin{itemize}
    \item \textbf{Structured}: ટેબલોમાં વ્યવસ્થિત (ડેટાબેસેસ, સ્પ્રેડશીટ્સ)
    \item \textbf{Unstructured}: ઇમેજ, ટેક્સ્ટ, ઓડિયો ફાઇલો
    \item \textbf{Time-series}: સમય પર ડેટા પોઇન્ટ્સ
\end{itemize}

\end{solutionbox}

\begin{mnemonicbox}
\mnemonic{Numbers Count Better Than Words - Numerical, Categorical, Binary, Text}
\end{mnemonicbox}

\questionmarks{2(b)}{4}{Calculate variance for both attributes. Determine which attribute is spread out around mean.}
\begin{solutionbox}

\textbf{આપેલ ડેટા:}
\begin{itemize}
    \item Attribute 1: 32, 37, 47, 50, 59
    \item Attribute 2: 48, 40, 41, 47, 49
\end{itemize}

\textbf{ગણતરીઓ:}

\textbf{Attribute 1:}
\begin{itemize}
    \item Mean = $(32+37+47+50+59)/5 = 225/5 = 45$
    \item Variance = $[(32-45)^2 + (37-45)^2 + (47-45)^2 + (50-45)^2 + (59-45)^2]/5$
    \item Variance = $[169 + 64 + 4 + 25 + 196]/5 = 458/5 = 91.6$
\end{itemize}

\textbf{Attribute 2:}
\begin{itemize}
    \item Mean = $(48+40+41+47+49)/5 = 225/5 = 45$
    \item Variance = $[(48-45)^2 + (40-45)^2 + (41-45)^2 + (47-45)^2 + (49-45)^2]/5$
    \item Variance = $[9 + 25 + 16 + 4 + 16]/5 = 70/5 = 14$
\end{itemize}

\textbf{પરિણામ}: Attribute 1 (variance = 91.6) એ Attribute 2 (variance = 14) કરતાં વધુ સ્પ્રેડ આઉટ છે.

\end{solutionbox}

\begin{mnemonicbox}
\mnemonic{Higher Variance Shows Spread - વધુ વેરિયન્સ વધુ વિખેરાઈને દર્શાવે છે}
\end{mnemonicbox}

\questionmarks{2(c)}{7}{List Factors that lead to data quality issue. How to handle outliers and missing values.}
\begin{solutionbox}

\begin{center}
\captionof{table}{ડેટા ગુણવત્તા સમસ્યાઓ}
\begin{tabulary}{\linewidth}{L L L}
\hline
\textbf{ફેક્ટર} & \textbf{કારણ} & \textbf{સોલ્યુશન} \\
\hline
\textbf{Incompleteness} & મિસિંગ ડેટા કલેક્શન & ઇમ્પ્યુટેશન ટેકનિક્સ \\
\textbf{Inconsistency} & વિવિધ ડેટા ફોર્મેટ્સ & સ્ટેન્ડર્ડાઇઝેશન \\
\textbf{Inaccuracy} & હ્યુમન/સેન્સર એરર્સ & વેલિડેશન રૂલ્સ \\
\textbf{Noise} & રેન્ડમ વેરિએશન્સ & ફિલ્ટરિંગ મેથડ્સ \\
\hline
\end{tabulary}
\end{center}

\textbf{આઉટલાયર્સ હેન્ડલ કરવું:}
\begin{itemize}
    \item \textbf{Detection}: સ્ટેટિસ્ટિકલ મેથડ્સ (Z-score, IQR)
    \item \textbf{Treatment}: એક્સ્ટ્રીમ વેલ્યુઝને દૂર, ટ્રાન્સફોર્મ, અથવા કેપ કરવી
    \item \textbf{Visualization}: બોક્સ પ્લોટ્સ, સ્કેટર પ્લોટ્સ
\end{itemize}

\textbf{મિસિંગ વેલ્યુઝ હેન્ડલ કરવું:}
\begin{itemize}
    \item \textbf{Deletion}: અપૂર્ણ રેકોર્ડ્સ રીમૂવ કરવા
    \item \textbf{Imputation}: મીન, મીડિયન, અથવા મોડ સાથે ભરવું
    \item \textbf{Prediction}: મિસિંગ વેલ્યુઝની આગાહી કરવા માટે ML નો ઉપયોગ
\end{itemize}

\textbf{કોડ ઉદાહરણ:}
\begin{lstlisting}[language=Python]
# Handle missing values
df.fillna(df.mean())  # Mean imputation
df.dropna()          # Remove missing rows
\end{lstlisting}

\end{solutionbox}

\begin{mnemonicbox}
\mnemonic{Clean Data Makes Models - સાફ ડેટા બેહતર મોડેલ્સ બનાવે છે}
\end{mnemonicbox}

\questionmarks{2(a) OR}{3}{Give different machine learning activities.}
\begin{solutionbox}

\begin{center}
\captionof{table}{મશીન લર્નિંગ પ્રવૃત્તિઓ}
\begin{tabulary}{\linewidth}{L L L}
\hline
\textbf{પ્રવૃત્તિ} & \textbf{હેતુ} & \textbf{ઉદાહરણ} \\
\hline
\textbf{Data Collection} & સંબંધિત માહિતી એકત્રિત કરવી & સર્વે, સેન્સર્સ, ડેટાબેસેસ \\
\textbf{Data Preprocessing} & ડેટાને સાફ અને તૈયાર કરવું & નોઇઝ રીમૂવ કરવું, મિસિંગ વેલ્યુઝ \\
\textbf{Feature Engineering} & અર્થપૂર્ણ વેરિએબલ્સ બનાવવા & રો ડેટામાંથી ફીચર્સ એક્સ્ટ્રેક્ટ કરવા \\
\textbf{Model Training} & અલ્ગોરિધમને પેટર્ન શીખવવા & ટ્રેનિંગ ડેટાસેટનો ઉપયોગ \\
\textbf{Model Evaluation} & પરફોર્મન્સ આકારણી & ટેસ્ટ એક્યુરસી, પ્રિસિઝન, રિકોલ \\
\textbf{Model Deployment} & મોડેલને પ્રોડક્શનમાં મૂકવું & વેબ સર્વિસેસ, મોબાઇલ એપ્સ \\
\hline
\end{tabulary}
\end{center}

\textbf{મુખ્ય પ્રવૃત્તિઓ:}
\begin{itemize}
    \item \keyword{Exploratory Data Analysis}: ડેટા પેટર્ન સમજવા
    \item \keyword{Hyperparameter Tuning}: મોડેલ સેટિંગ્સ ઓપ્ટિમાઇઝ કરવા
    \item \keyword{Cross-validation}: મજબૂત પરફોર્મન્સ આકારણી
\end{itemize}

\end{solutionbox}

\begin{mnemonicbox}
\mnemonic{Data Models Perform Excellently - Data preparation, Model building, Performance evaluation, Execution}
\end{mnemonicbox}

\questionmarks{2(b) OR}{4}{Calculate mean and median of the following numbers: 12,15,18,20,22,24,28,30}
\begin{solutionbox}

\textbf{આપેલ સંખ્યાઓ:} 12, 15, 18, 20, 22, 24, 28, 30

\textbf{Mean ગણતરી:}
Mean = $(12+15+18+20+22+24+28+30)/8 = 169/8 = 21.125$

\textbf{Median ગણતરી:}
\begin{itemize}
    \item સંખ્યાઓ પહેલેથી સૉર્ટ કરેલી છે: 12, 15, 18, 20, 22, 24, 28, 30
    \item સમ સંખ્યા (8 સંખ્યાઓ)
    \item Median = (4મી સંખ્યા + 5મી સંખ્યા)/2 = $(20 + 22)/2 = 21$
\end{itemize}

\begin{center}
\captionof{table}{સ્ટેટિસ્ટિકલ સમરી}
\begin{tabulary}{\linewidth}{L L L}
\hline
\textbf{માપદંડ} & \textbf{મૂલ્ય} & \textbf{વર્ણન} \\
\hline
\textbf{Mean} & 21.125 & સરેરાશ મૂલ્ય \\
\textbf{Median} & 21 & મધ્યમ મૂલ્ય \\
\textbf{Count} & 8 & કુલ સંખ્યાઓ \\
\hline
\end{tabulary}
\end{center}

\end{solutionbox}

\begin{mnemonicbox}
\mnemonic{Middle Makes Median - Middle value gives median}
\end{mnemonicbox}

\questionmarks{2(c) OR}{7}{Write a short note on dimensionality reduction and feature subset selection in context with data preprocessing.}
\begin{solutionbox}

\keyword{Dimensionality Reduction} અપ્રસ્તુત ફીચર્સને દૂર કરે છે અને કોમ્પ્યુટેશનલ જટિલતા ઘટાડે છે જ્યારે મહત્વપૂર્ણ માહિતી જાળવી રાખે છે.

\begin{center}
\captionof{table}{ડાયમેન્શનાલિટી રિડક્શન ટેકનિક્સ}
\begin{tabulary}{\linewidth}{L L L}
\hline
\textbf{ટેકનિક} & \textbf{મેથડ} & \textbf{વપરાશ} \\
\hline
\textbf{PCA} & Principal Component Analysis & લીનિયર રિડક્શન \\
\textbf{LDA} & Linear Discriminant Analysis & ક્લાસિફિકેશન ટાસ્ક્સ \\
\textbf{t-SNE} & Non-linear embedding & વિઝ્યુઅલાઇઝેશન \\
\textbf{Feature Selection} & મહત્વપૂર્ણ ફીચર્સ પસંદ કરવા & ઓવરફિટિંગ ઘટાડવું \\
\hline
\end{tabulary}
\end{center}

\textbf{Feature Subset Selection Methods:}
\begin{itemize}
    \item \keyword{Filter Methods}: Statistical tests, correlation analysis
    \item \keyword{Wrapper Methods}: Forward/backward selection
    \item \keyword{Embedded Methods}: LASSO, Ridge regression
\end{itemize}

\textbf{ફાયદાઓ:}
\begin{itemize}
    \item \textbf{Computational Efficiency}: ઝડપી ટ્રેનિંગ અને પ્રિડિક્શન
    \item \textbf{Storage Reduction}: ઓછી મેમરી આવશ્યકતાઓ
    \item \textbf{Noise Reduction}: અપ્રસ્તુત ફીચર્સ દૂર કરવા
    \item \textbf{Visualization}: 2D/3D પ્લોટિંગ સક્ષમ કરવું
\end{itemize}

\begin{lstlisting}[language=Python]
from sklearn.decomposition import PCA
pca = PCA(n_components=2)
reduced_data = pca.fit_transform(data)
\end{lstlisting}

\end{solutionbox}

\begin{mnemonicbox}
\mnemonic{Reduce Features, Improve Performance - ઓછા ફીચર્સ ઘણીવાર બેહતર મોડેલ્સ તરફ દોરી જાય છે}
\end{mnemonicbox}

\questionmarks{3(a)}{3}{Does bias affect the performance of the ML model? Explain briefly.}
\begin{solutionbox}

હા, બાયસ પ્રિડિક્શન્સમાં સિસ્ટેમેટિક એરર્સ બનાવીને ML મોડેલના પરફોર્મન્સને નોંધપાત્ર રીતે અસર કરે છે.

\begin{center}
\captionof{table}{બાયસના પ્રકારો}
\begin{tabulary}{\linewidth}{L L L}
\hline
\textbf{બાયસ પ્રકાર} & \textbf{વર્ણન} & \textbf{અસર} \\
\hline
\textbf{Selection Bias} & બિન-પ્રતિનિધિત્વકારી ડેટા & નબળું જનરલાઇઝેશન \\
\textbf{Confirmation Bias} & અપેક્ષિત પરિણામોની તરફેણ & ત્રાંસા નિષ્કર્ષો \\
\textbf{Algorithmic Bias} & મોડેલ ધારણાઓ & અન્યાયકારક પ્રિડિક્શન્સ \\
\hline
\end{tabulary}
\end{center}

\textbf{પરફોર્મન્સ પર અસરો:}
\begin{itemize}
    \item \keyword{Underfitting}: ઉચ્ચ બાયસ અતિ સરળ મોડેલ્સ તરફ દોરી જાય છે
    \item \keyword{Poor Accuracy}: સિસ્ટેમેટિક એરર્સ એકંદર પરફોર્મન્સ ઘટાડે છે
    \item \keyword{Unfair Decisions}: પક્ષપાતી મોડેલ્સ જૂથો સામે ભેદભાવ કરે છે
\end{itemize}

\end{solutionbox}

\begin{mnemonicbox}
\mnemonic{Bias Breaks Better Performance - બાયસ મોડેલની અસરકારકતા ઘટાડે છે}
\end{mnemonicbox}

\questionmarks{3(b)}{4}{Compare cross-validation and bootstrap sampling}
\begin{solutionbox}

\begin{center}
\captionof{table}{Cross-validation vs Bootstrap Sampling}
\begin{tabulary}{\linewidth}{L L L}
\hline
\textbf{પાસું} & \textbf{Cross-validation} & \textbf{Bootstrap Sampling} \\
\hline
\textbf{મેથડ} & ડેટાને ફોલ્ડ્સમાં વિભાજિત કરવું & રિપ્લેસમેન્ટ સાથે સેમ્પલ કરવું \\
\textbf{ડેટા ઉપયોગ} & બધો ડેટા વાપરે છે & મલ્ટિપલ સેમ્પલ્સ બનાવે છે \\
\textbf{હેતુ} & મોડેલ ઇવેલ્યુએશન & અનિશ્ચિતતાનો અંદાજ \\
\textbf{ઓવરલેપ} & સેટ્સ વચ્ચે કોઈ ઓવરલેપ નથી & ડુપ્લિકેટ સેમ્પલ્સની મંજૂરી \\
\hline
\end{tabulary}
\end{center}

\textbf{મુખ્ય તફાવત:}
\begin{itemize}
    \item \textbf{Cross-validation}: ડેટાને k સમાન ભાગોમાં વહેંચે છે. k-1 ભાગોમાં ટ્રેન કરે છે, 1 ભાગમાં ટેસ્ટ કરે છે.
    \item \textbf{Bootstrap Sampling}: રિપ્લેસમેન્ટ સાથે રેન્ડમ સેમ્પલ્સ બનાવે છે. સમાન સાઇઝના મલ્ટિપલ ડેટાસેટ્સ જનરેટ કરે છે.
\end{itemize}

\end{solutionbox}

\begin{mnemonicbox}
\mnemonic{Cross Checks, Bootstrap Builds - Cross-validation checks performance, Bootstrap builds confidence}
\end{mnemonicbox}

\questionmarks{3(c)}{7}{Confusion Matrix Calculation and Metrics}
\begin{solutionbox}

\textbf{આપેલ માહિતી:}
\begin{itemize}
    \item True Positive (TP): 83
    \item False Positive (FP): 7
    \item False Negative (FN): 5
    \item True Negative (TN): 5
\end{itemize}

\begin{center}
\begin{tabulary}{\linewidth}{L L L}
\hline
 & \textbf{Predicted Buy} & \textbf{Predicted No Buy} \\
\hline
\textbf{Actually Buy} & 83 (TP) & 5 (FN) \\
\textbf{Actually No Buy} & 7 (FP) & 5 (TN) \\
\hline
\end{tabulary}
\end{center}

\textbf{ગણતરીઓ:}

\textbf{a) Error Rate:}
Error Rate = $(FP + FN) / Total = (7 + 5) / 100 = 0.12 = 12\%$

\textbf{b) Precision:}
Precision = $TP / (TP + FP) = 83 / (83 + 7) = 83/90 = 0.922 = 92.2\%$

\textbf{c) Recall:}
Recall = $TP / (TP + FN) = 83 / (83 + 5) = 83/88 = 0.943 = 94.3\%$

\textbf{d) F-measure:}
F-measure = $2 \times (Precision \times Recall) / (Precision + Recall)$
F-measure = $2 \times (0.922 \times 0.943) / (0.922 + 0.943) = 0.932 = 93.2\%$

\end{solutionbox}

\begin{mnemonicbox}
\mnemonic{Perfect Recall Finds Everyone - Precision measures accuracy, Recall finds all positives}
\end{mnemonicbox}

\questionmarks{3(a) OR}{3}{Define in brief: a) Target function b) Cost function c) Loss Function}
\begin{solutionbox}

\begin{center}
\captionof{table}{ફંક્શન વ્યાખ્યાઓ}
\begin{tabulary}{\linewidth}{L L L}
\hline
\textbf{ફંક્શન} & \textbf{વ્યાખ્યા} & \textbf{હેતુ} \\
\hline
\textbf{Target Function} & ઇનપુટથી આઉટપુટ સુધીની આદર્શ મેપિંગ & આપણે શું શીખવા માગીએ છીએ \\
\textbf{Cost Function} & એકંદર મોડેલ એરરને માપે છે & કુલ પરફોર્મન્સનું મૂલ્યાંકન \\
\textbf{Loss Function} & એક પ્રિડિક્શન માટે એરર માપે છે & વ્યક્તિગત પ્રિડિક્શન એરર \\
\hline
\end{tabulary}
\end{center}

\textbf{સંબંધ}: Cost function સામાન્ય રીતે તમામ ટ્રેનિંગ ઉદાહરણોમાં લોસ ફંક્શન્સની સરેરાશ હોય છે.

\end{solutionbox}

\begin{mnemonicbox}
\mnemonic{Target Costs Less - Target function is ideal, Cost function measures overall error, Loss function measures individual error}
\end{mnemonicbox}

\questionmarks{3(b) OR}{4}{Explain balanced fit, underfit and overfit}
\begin{solutionbox}

\begin{center}
\captionof{table}{મોડેલ ફિટિંગ પ્રકારો}
\begin{tabulary}{\linewidth}{L L L L}
\hline
\textbf{ફિટ પ્રકાર} & \textbf{Training Error} & \textbf{Validation Error} & \textbf{લક્ષણો} \\
\hline
\textbf{Underfit} & ઊંચો & ઊંચો & ખૂબ સાદું મોડેલ \\
\textbf{Balanced Fit} & નીચો & નીચો & આદર્શ જટિલતા \\
\textbf{Overfit} & ખૂબ નીચો & ઊંચો & ખૂબ જટિલ મોડેલ \\
\hline
\end{tabulary}
\end{center}

\begin{center}
\begin{tikzpicture}[node distance=2cm, auto]
    \node [gtu block] (Under) {Underfit\\(High Bias)};
    \node [gtu block, right=of Under] (Bal) {Balanced Fit\\(Optimal)};
    \node [gtu block, right=of Bal] (Over) {Overfit\\(High Variance)};
    
    \path [gtu arrow] (Under) -- (Bal);
    \path [gtu arrow] (Bal) -- (Over);
\end{tikzpicture}
\captionof{figure}{મોડેલ જટિલતા સ્પેક્ટ્રમ}
\end{center}

\textbf{સોલ્યુશન્સ:}
\begin{itemize}
    \item \textbf{Underfit}: મોડેલ જટિલતા વધારવી, ફીચર્સ ઉમેરવા
    \item \textbf{Overfit}: રેગ્યુલરાઇઝેશન, ક્રોસ-વેલિડેશન, વધુ ડેટા
\end{itemize}

\end{solutionbox}

\begin{mnemonicbox}
\mnemonic{Balance Brings Best Results - સંતુલિત મોડેલ્સ નવા ડેટા પર શ્રેષ્ઠ પરફોર્મ કરે છે}
\end{mnemonicbox}

\questionmarks{4(a)}{3}{Give classification learning steps.}
\begin{solutionbox}

\begin{center}
\captionof{table}{ક્લાસિફિકેશન લર્નિંગ સ્ટેપ્સ}
\begin{tabulary}{\linewidth}{L L L}
\hline
\textbf{સ્ટેપ} & \textbf{વર્ણન} & \textbf{હેતુ} \\
\hline
\textbf{Data Collection} & લેબલ્ડ ઉદાહરણો એકત્રિત કરવા & ટ્રેનિંગ મટેરિયલ પ્રદાન કરવું \\
\textbf{Preprocessing} & ડેટાને સાફ અને તૈયાર કરવું & ડેટા ગુણવત્તા સુધારવી \\
\textbf{Feature Selection} & સંબંધિત એટ્રિબ્યુટ્સ પસંદ કરવા & જટિલતા ઘટાડવી \\
\textbf{Model Training} & ટ્રેનિંગ ડેટામાંથી શીખવું & ક્લાસિફાયર બનાવવું \\
\textbf{Evaluation} & મોડેલ પરફોર્મન્સ ટેસ્ટ કરવું & ચોકસાઈ આકારવી \\
\textbf{Deployment} & નવી આગાહીઓ માટે ઉપયોગ & પ્રેક્ટિકલ એપ્લિકેશન \\
\hline
\end{tabulary}
\end{center}

\end{solutionbox}

\begin{mnemonicbox}
\mnemonic{Data Preparation Facilitates Model Excellence - Data prep, Feature selection, Model training, Evaluation}
\end{mnemonicbox}

\questionmarks{4(b)}{4}{Linear Relationship Calculation}
\begin{solutionbox}

\textbf{આપેલ ડેટા:} Hours (X) vs Exam Score (Y)

\textbf{Linear Regression ગણતરી:}

\textbf{સ્ટેપ 1: Mean કેલ્ક્યુલેટ કરવા}
\begin{itemize}
    \item $\bar{X} = (2+3+4+5+6)/5 = 4$
    \item $\bar{Y} = (85+80+75+70+60)/5 = 74$
\end{itemize}

\textbf{સ્ટેપ 2: Slope (b) કેલ્ક્યુલેટ કરવું}
\begin{itemize}
    \item ન્યુમેરેટર = $\sum(X-\bar{X})(Y-\bar{Y}) = -60$
    \item ડિનોમિનેટર = $\sum(X-\bar{X})^2 = 10$
    \item $b = -60/10 = -6$
\end{itemize}

\textbf{સ્ટેપ 3: Intercept (a) કેલ્ક્યુલેટ કરવું}
\begin{itemize}
    \item $a = \bar{Y} - b\times\bar{X} = 74 - (-6)\times4 = 74 + 24 = 98$
\end{itemize}

\textbf{Linear Equation}: $Y = 98 - 6X$

\textbf{અર્થઘટન}: સ્માર્ટફોન ઉપયોગના દરેક વધારાના કલાક માટે, પરીક્ષા સ્કોર 6 પોઇન્ટ ઘટે છે.

\end{solutionbox}

\begin{mnemonicbox}
\mnemonic{More Phone, Less Score - Negative correlation between phone use and grades}
\end{mnemonicbox}

\questionmarks{4(c)}{7}{Explain classification steps in detail}
\begin{solutionbox}

Classification એ સુપરવાઇઝ્ડ લર્નિંગ પ્રક્રિયા છે જે ઇનપુટ ડેટાને પૂર્વનિર્ધારિત કેટેગરીઓમાં સોંપે છે.

\textbf{વિગતવાર ક્લાસિફિકેશન સ્ટેપ્સ:}

\textbf{1. Problem Definition}
\begin{itemize}
    \item ક્લાસો અને ઉદ્દેશ્યો વ્યાખ્યાયિત કરવા
    \item ઇનપુટ ફીચર્સ અને ટાર્ગેટ વેરિએબલ ઓળખવા
\end{itemize}

\textbf{2. Data Collection and Preparation}
\begin{center}
\begin{tikzpicture}[node distance=1.2cm, auto]
    \node [gtu block] (Raw) {રો ડેટા};
    \node [gtu block, right=0.5cm of Raw] (Clean) {ડેટા ક્લીન};
    \node [gtu block, right=0.5cm of Clean] (Feat) {ફીચર્સ};
    \node [gtu block, right=0.5cm of Feat] (Split) {સ્પ્લિટ};
    
    \path [gtu arrow] (Raw) -- (Clean);
    \path [gtu arrow] (Clean) -- (Feat);
    \path [gtu arrow] (Feat) -- (Split);
\end{tikzpicture}
\end{center}

\textbf{3. Feature Engineering}
\begin{itemize}
    \item \textbf{Feature Selection}: સંબંધિત એટ્રિબ્યુટ્સ પસંદ કરવા
    \item \textbf{Normalization}: ફીચર્સને સમાન રેન્જમાં સ્કેલ કરવા
\end{itemize}

\textbf{4. Model Selection and Training}
\begin{center}
\captionof{table}{સામાન્ય ક્લાસિફિકેશન અલ્ગોરિધમ્સ}
\begin{tabulary}{\linewidth}{L L L}
\hline
\textbf{અલ્ગોરિધમ} & \textbf{શ્રેષ્ઠ માટે} & \textbf{ફાયદાઓ} \\
\hline
\textbf{Decision Tree} & ઇન્ટરપ્રિટેબલ રૂલ્સ & સમજવામાં સરળ \\
\textbf{SVM} & હાઇ-ડાયમેન્શનલ ડેટા & સારું જનરલાઇઝેશન \\
\textbf{Neural Networks} & જટિલ પેટર્ન્સ & ઉચ્ચ ચોકસાઈ \\
\hline
\end{tabulary}
\end{center}

\textbf{5. Model Evaluation}
\begin{itemize}
    \item \textbf{Confusion Matrix}: વિગતવાર પરફોર્મન્સ એનાલિસિસ
    \item \textbf{Metrics}: Accuracy, Precision, Recall, F1-score
\end{itemize}

\textbf{6. Final Evaluation and Deployment}
\begin{itemize}
    \item અદ્રશ્ય ડેટા પર ટેસ્ટ કરવું
    \item પ્રોડક્શન ઉપયોગ માટે મોડેલ ડિપ્લોય કરવું
\end{itemize}

\end{solutionbox}

\begin{mnemonicbox}
\mnemonic{Proper Data Modeling Evaluates Performance Thoroughly - Problem definition, Data prep, Modeling, Evaluation, Performance testing, Tuning}
\end{mnemonicbox}

\questionmarks{4(a) OR}{3}{Does the choice of the k value influence the performance of the KNN algorithm? Explain briefly}
\begin{solutionbox}

હા, k વેલ્યુ KNN અલ્ગોરિધમના પરફોર્મન્સને નોંધપાત્ર રીતે પ્રભાવિત કરે છે.

\begin{center}
\captionof{table}{K વેલ્યુની અસર}
\begin{tabulary}{\linewidth}{L L L}
\hline
\textbf{K વેલ્યુ} & \textbf{અસર} & \textbf{પરફોર્મન્સ} \\
\hline
\textbf{Small K (k=1)} & નોઇઝ પ્રત્યે સંવેદનશીલ & High variance, low bias \\
\textbf{Medium K} & સંતુલિત નિર્ણયો & આદર્શ પરફોર્મન્સ \\
\textbf{Large K} & સ્મૂથ બાઉન્ડરીઝ & Low variance, high bias \\
\hline
\end{tabulary}
\end{center}

\textbf{સિલેક્શન વ્યૂહરચના:} આદર્શ k શોધવા માટે ક્રોસ-વેલિડેશનનો ઉપયોગ કરો, ઘણીવાર $k = \sqrt{n}$ થી શરૂઆત.

\end{solutionbox}

\begin{mnemonicbox}
\mnemonic{Small K Varies, Large K Smooths - Small k creates variance, large k creates smooth boundaries}
\end{mnemonicbox}

\questionmarks{4(b) OR}{4}{Define Support Vectors in the SVM model.}
\begin{solutionbox}

Support Vectors એ મહત્વપૂર્ણ ડેટા પોઇન્ટ્સ છે જે ડિસિઝન બાઉન્ડરી (hyperplane)ની સૌથી નજીક આવેલા હોય છે.

\begin{center}
\captionof{table}{Support Vector લક્ષણો}
\begin{tabulary}{\linewidth}{L L L}
\hline
\textbf{પાસું} & \textbf{વર્ણન} & \textbf{મહત્વ} \\
\hline
\textbf{Link} & હાયપરપ્લેનની સૌથી નજીકના પોઇન્ટ્સ & ડિસિઝન બાઉન્ડરી વ્યાખ્યાયિત કરે \\
\textbf{Distance} & બાઉન્ડરીથી સમાન અંતર & Maximize margin \\
\textbf{Role} & હાયપરપ્લેનને સપોર્ટ કરે & આદર્શ વિભાજન નક્કી કરે \\
\hline
\end{tabulary}
\end{center}

\begin{center}
\begin{tikzpicture}
    % Axes
    \draw[->] (0,0) -- (4,0) node[right] {$x_1$};
    \draw[->] (0,0) -- (0,4) node[above] {$x_2$};
    
    % Points Class A (Circles)
    \foreach \p in {(1.5, 3), (2, 3.5), (1, 2.5)}
        \node[circle, draw, fill=white, inner sep=1.5pt] at \p {};
        
    % Points Class B (Crosses/Filled) - Using filled squares for B
    \foreach \p in {(3.5, 1), (3, 0.5), (4, 1.5)}
        \node[rectangle, draw, fill=black, inner sep=1.5pt] at \p {};
        
    % Support Vectors (closest ones)
    % A: (2.5, 2.5) approx
    % B: (3.5, 1.5) approx
    
    % Hyperplane y = -x + 4.5  (passing through 2.25, 2.25)
    \draw[thick] (1, 3.5) -- (3.5, 1);
    
    % Margins
    \draw[dashed] (0.5, 3) -- (3, 0.5);
    \draw[dashed] (1.5, 4) -- (4, 1.5);
    
    % Labels
    \node[right] at (3.5, 3.5) {Hyperplane};
    \draw[->] (3.5, 3.5) -- (2.25, 2.25);
    \node[below] at (2,0.5) {Support Vectors};
\end{tikzpicture}
\captionof{figure}{SVM હાયપરપ્લેન અને સપોર્ટ વેક્ટર્સ}
\end{center}

\end{solutionbox}

\begin{mnemonicbox}
\mnemonic{Support Vectors Support Decisions - These vectors support the decision boundary}
\end{mnemonicbox}

\questionmarks{4(c) OR}{7}{Explain logistic regression in detail.}
\begin{solutionbox}

Logistic Regression એ બાઇનરી ક્લાસિફિકેશન માટે વપરાતી સ્ટેટિસ્ટિકલ મેથડ છે.

\textbf{ગાણિતિક આધાર:}
\textbf{Sigmoid Function:} $\sigma(z) = 1 / (1 + e^{-z})$ જયાં $z = \beta_0 + \beta_1x_1 + ...$

\begin{center}
\captionof{table}{Linear vs Logistic Regression}
\begin{tabulary}{\linewidth}{L L L}
\hline
\textbf{પાસું} & \textbf{Linear Regression} & \textbf{Logistic Regression} \\
\hline
\textbf{Output} & સતત મૂલ્યો & સંભાવનાઓ (0-1) \\
\textbf{Function} & Linear & Sigmoid (S-curve) \\
\textbf{Purpose} & આગાહી & ક્લાસિફિકેશન \\
\textbf{Error} & Mean Squared Error & Log-likelihood \\
\hline
\end{tabulary}
\end{center}

\textbf{મુખ્ય ઘટકો:}
\begin{itemize}
    \item \textbf{Logistic Function}: S-આકારનો કર્વ જે મૂલ્યોને [0,1] માં મેપ કરે છે.
    \item \textbf{Decision Rule}: જો $P(y=1|x) > 0.5$, તો પોઝિટિવ તરીકે ક્લાસિફાય કરવું.
    \item \textbf{Training}: Maximum Likelihood Estimation નો ઉપયોગ કરે છે.
\end{itemize}

\textbf{એપ્લિકેશન્સ:} મેડિકલ ડાયગ્નોસિસ, ઇમેઇલ સ્પામ ડિટેક્શન, ક્રેડિટ એપ્રૂવલ.

\begin{lstlisting}[language=Python]
from sklearn.linear_model import LogisticRegression
model = LogisticRegression()
model.fit(X_train, y_train)
probabilities = model.predict_proba(X_test)
\end{lstlisting}

\end{solutionbox}

\begin{mnemonicbox}
\mnemonic{Sigmoid Squashes Infinite Input - Sigmoid function converts any real number to probability}
\end{mnemonicbox}

\questionmarks{5(a)}{3}{Write a short note on Matplotlib python library.}
\begin{solutionbox}

Matplotlib એ વિઝ્યુઅલાઇઝેશન બનાવવા માટેની વ્યાપક Python લાઇબ્રેરી છે.

\begin{center}
\captionof{table}{Matplotlib મુખ્ય ફીચર્સ}
\begin{tabulary}{\linewidth}{L L L}
\hline
\textbf{ફીચર} & \textbf{હેતુ} & \textbf{ઉદાહરણ} \\
\hline
\textbf{Pyplot} & MATLAB-જેવું ઇન્ટરફેસ & લાઇન પ્લોટ્સ \\
\textbf{Formats} & વિવિધ ફોર્મેટમાં સેવ કરવું & PNG, PDF, SVG \\
\textbf{Subplots} & એક ફિગરમાં મલ્ટિપલ પ્લોટ્સ & ગ્રિડ એરેન્જમેન્ટ્સ \\
\hline
\end{tabulary}
\end{center}

\textbf{મૂળભૂત ઉપયોગ:}
\begin{lstlisting}[language=Python]
import matplotlib.pyplot as plt
plt.plot(x, y)
plt.show()
\end{lstlisting}

\end{solutionbox}

\begin{mnemonicbox}
\mnemonic{Matplotlib Makes Pretty Plots - Essential tool for data visualization}
\end{mnemonicbox}

\questionmarks{5(b)}{4}{K-means clustering for two-dimensional data}
\begin{solutionbox}

\textbf{આપેલ પોઇન્ટ્સ:} બે જૂથો સ્પષ્ટ રીતે અલગ પડે છે.
Group 1: (2,3) થી (8,3). Group 2: (25,20) થી (30,20).

\textbf{અલ્ગોરિધમ સ્ટેપ્સ:}
\textbf{Step 1: સેન્ટ્રોઇડ્સ ઇનિશિયલાઇઝ કરવા}
C1 = (4, 3), C2 = (27, 20)

\textbf{Step 2: પોઇન્ટ્સ સોંપવા}
પોઇન્ટ્સ (2,3)...(8,3) C1 ની નજીક છે.
પોઇન્ટ્સ (25,20)...(30,20) C2 ની નજીક છે.

\textbf{Step 3: સેન્ટ્રોઇડ્સ અપડેટ કરવા}
નવું C1 = Group 1 નું Average = (5, 3)
નવું C2 = Group 2 નું Average = (27.5, 20)

\textbf{અંતિમ ક્લસ્ટર્સ:}
Cluster 1: ડાબા જૂથના પોઇન્ટ્સ. Cluster 2: જમણા જૂથના પોઇન્ટ્સ.

\end{solutionbox}

\begin{mnemonicbox}
\mnemonic{Centroids Attract Nearest Neighbors - Points join closest centroid}
\end{mnemonicbox}

\questionmarks{5(c)}{7}{Give functions and its use of Scikit-learn}
\begin{solutionbox}

\textbf{a) Data Preprocessing:}
\begin{itemize}
    \item \code{StandardScaler()}: ફીચર્સને નોર્મલાઇઝ કરવા
    \item \code{train\_test\_split()}: ડેટાસેટ સ્પ્લિટ કરવું
\end{itemize}

\textbf{b) Model Selection:}
\begin{itemize}
    \item \code{GridSearchCV()}: હાયપરપેરામીટર ટ્યુનિંગ
    \item \code{cross\_val\_score()}: ક્રોસ-વેલિડેશન
\end{itemize}

\textbf{c) Model Evaluation:}
\begin{itemize}
    \item \code{accuracy\_score()}: એકંદર ચોકસાઈ
    \item \code{confusion\_matrix()}: એરર એનાલિસિસ
    \item \code{classification\_report()}: વ્યાપક મેટ્રિક્સ
\end{itemize}

\begin{lstlisting}[language=Python]
from sklearn.metrics import classification_report
print(classification_report(y_true, y_pred))
\end{lstlisting}

\end{solutionbox}

\begin{mnemonicbox}
\mnemonic{Preprocess, Select, Evaluate - Complete ML workflow in Scikit-learn}
\end{mnemonicbox}

\questionmarks{5(a) OR}{3}{List out the major features of Numpy.}
\begin{solutionbox}

NumPy વૈજ્ઞાનિક કોમ્પ્યુટિંગ માટેનું મૂળભૂત પેકેજ છે.

\begin{itemize}
    \item \textbf{N-dimensional Arrays}: કાર્યક્ષમ એરે ઓબ્જેક્ટ્સ
    \item \textbf{Broadcasting}: વિવિધ સાઇઝના એરે પર ઓપરેશન્સ
    \item \textbf{Linear Algebra}: મેટ્રિક્સ ઓપરેશન્સ
    \item \textbf{Random Numbers}: સ્ટેટિસ્ટિકલ સિમ્યુલેશન્સ
\end{itemize}

\end{solutionbox}

\begin{mnemonicbox}
\mnemonic{Numbers Need Numpy's Power - Essential for numerical computations}
\end{mnemonicbox}

\questionmarks{5(b) OR}{4}{K-means clustering for one-dimensional data}
\begin{solutionbox}

\textbf{Dataset:} {1,2,4,5,7,8,10,11,12,14,15,17}

\textbf{K-means (k=3):}
\begin{itemize}
    \item \textbf{Cluster 1}: {1, 2, 4, 5} (Centroid $\approx$ 3)
    \item \textbf{Cluster 2}: {7, 8, 10, 11, 12} (Centroid $\approx$ 9.6)
    \item \textbf{Cluster 3}: {14, 15, 17} (Centroid $\approx$ 15.3)
\end{itemize}

\end{solutionbox}

\begin{mnemonicbox}
\mnemonic{Groups Gather by Distance - Similar points form natural clusters}
\end{mnemonicbox}

\questionmarks{5(c) OR}{7}{Give function and its use of Pandas library}
\begin{solutionbox}

\textbf{a) Preprocessing:}
\begin{itemize}
    \item \code{read\_csv()}: ડેટા લોડ કરવા
    \item \code{head()}, \code{tail()}: ડેટા જોવા
\end{itemize}

\textbf{b) Inspection:}
\begin{itemize}
    \item \code{info()}: ડેટા ટાઇપ્સ, મેમરી
    \item \code{describe()}: સ્ટેટિસ્ટિકલ સમરી
    \item \code{isnull()}: મિસિંગ વેલ્યુઝ ચેક કરવા
\end{itemize}

\textbf{c) Cleaning:}
\begin{itemize}
    \item \code{dropna()}, \code{fillna()}: મિસિંગ ડેટા હેન્ડલ કરવા
    \item \code{groupby()}: ડેટા એગ્રીગેટ કરવા
\end{itemize}

\begin{lstlisting}[language=Python]
df = pd.read_csv('data.csv')
print(df.describe())
\end{lstlisting}

\end{solutionbox}

\begin{mnemonicbox}
\mnemonic{Pandas Processes Data Perfectly - Comprehensive data manipulation tool}
\end{mnemonicbox}

\end{document}
