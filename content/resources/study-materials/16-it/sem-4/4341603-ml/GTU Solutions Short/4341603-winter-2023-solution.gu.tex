\documentclass{article}

% content/resources/templates/preamble.tex
\usepackage[margin=0.6in]{geometry}
\author{Milav Dabgar}
\usepackage{amsmath,amssymb,amsthm}
\usepackage{booktabs}
\usepackage{multirow}
\usepackage{xcolor}
\usepackage{tcolorbox}
\tcbuselibrary{breakable,skins}
\usepackage[colorlinks=true,linkcolor=blue]{hyperref}
\usepackage{titlesec}
\usepackage{enumitem}
\usepackage{tikz}
\usepackage{pgfplots}
\usepackage{circuitikz}
\usepackage[version=4]{mhchem}
\usepackage{longtable}
\usepackage{array}
\usepackage{float}
\usepackage{caption}
\usepackage{listings}

\lstset{
  basicstyle=\small\ttfamily,
  breaklines=true,
  breakatwhitespace=false,
  postbreak=\mbox{\textcolor{red}{$\hookrightarrow$}\space},
  float=false,
  numbers=left,
  numberstyle=\tiny\color{gray},
  numbersep=10pt,
  xleftmargin=2em,
  keywordstyle=\color{blue},
  commentstyle=\color{green!60!black},
  stringstyle=\color{purple},
  backgroundcolor=\color{gray!5},
  showstringspaces=false,
  tabsize=2,
  captionpos=b,
  keepspaces=true,
  columns=flexible
}

\pgfplotsset{compat=1.18}
\usetikzlibrary{shapes,arrows,positioning,calc,patterns,decorations.pathmorphing,decorations.markings,arrows.meta}

% Color scheme
\definecolor{headcolor}{RGB}{0,102,204}
\definecolor{keycolor}{RGB}{220,20,60}
\definecolor{solutioncolor}{RGB}{34,139,34}
\definecolor{mnemoniccolor}{RGB}{148,0,211}
\definecolor{codecolor}{RGB}{0,0,100}

% Spacing
\setlength{\parskip}{3pt}
\setlist[itemize]{nosep}
\setlist[enumerate]{nosep}

% Title formatting
\titleformat{\section}{\Large\bfseries\color{headcolor}}{\thesection}{1em}{}
\titleformat{\subsection}{\large\bfseries\color{headcolor}}{\thesubsection}{1em}{}

% Pandoc tightlist compatibility
\providecommand{\tightlist}{%
  \setlength{\itemsep}{0pt}\setlength{\parskip}{0pt}}

% Pandoc longtable compatibility
\newcounter{none}
\def\thenone{}


% content/resources/templates/gujarati-boxes.tex
\usepackage{fontspec}
\usepackage{polyglossia}

% Set Gujarati as main language (document is primarily in Gujarati)
% Note: gloss-gujarati.ldf doesn't exist in polyglossia, but it will use hyphenation patterns
\setdefaultlanguage{gujarati}
\setotherlanguage{english}

% Configure Gujarati font properly
% Use Language=Default to prevent polyglossia from trying to add language-specific features
% that don't exist for Gujarati, which causes "empty feature" warnings
\newfontfamily\gujaratifont[Script=Gujarati,AutoFakeBold=2.5,AutoFakeSlant=0.3]{Noto Sans Gujarati}
\setmainfont[Script=Gujarati,AutoFakeBold=2.5,AutoFakeSlant=0.3]{Noto Sans Gujarati}
% Use Noto Sans Gujarati for monospace to support Gujarati in text
\setmonofont[Scale=0.9]{Noto Sans Gujarati}

% Configure English to use the same font
\newfontfamily\englishfont[Script=Gujarati,AutoFakeBold=2.5,AutoFakeSlant=0.3]{Noto Sans Gujarati}

% Translations for polyglossia
\gappto\captionsgujarati{
  \renewcommand{\tablename}{કોષ્ટક}
  \renewcommand{\figurename}{આકૃતિ}
}

% Helper for TikZ nodes to ensure Gujarati font
\newcommand{\gu}[1]{{\gujaratifont #1}}

% Custom environments
\newtcolorbox{solutionbox}{
    breakable,
    enhanced,
    colback=solutioncolor!5!white,
    colframe=solutioncolor!75!black,
    fonttitle=\bfseries,
    title=જવાબ
}

\newtcolorbox{solutionboxnobreak}{
 colback=solutioncolor!5!white,
 colframe=solutioncolor!75!black,
 fonttitle=\bfseries,
 title=જવાબ
}

\newtcolorbox{keyformula}{
 breakable,
 enhanced,
 colback=keycolor!5!white,
 colframe=keycolor!75!black,
 fonttitle=\bfseries,
 title=રાસાયણિક સમીકરણ/સૂત્ર
}

\newtcolorbox{mnemonicbox}{
 breakable,
 enhanced,
 colback=mnemoniccolor!5!white,
 colframe=mnemoniccolor!75!black,
 fonttitle=\bfseries,
 title=મેમરી ટ્રીક
}


% Custom commands for GTU solutions
% This file defines semantic commands for consistent formatting

% Question command with automatic formatting
\newcommand{\question}[2]{%
  \section*{Question #1}%
  \textbf{#2}%
}

% OR question variant
\newcommand{\questionor}[2]{%
  \section*{Question #1 OR}%
  \textbf{#2}%
}

% Proper table environment with caption
\newenvironment{answertable}[1]{%
  \begin{table}[htbp]
  \centering
  \caption{#1}
}{%
  \end{table}
}

% Proper figure environment for diagrams
\newenvironment{answerdiagram}[1]{%
  \begin{figure}[htbp]
  \centering
  \caption{#1}
}{%
  \end{figure}
}

% Semantic markup for key terms
\newcommand{\keyword}[1]{\textbf{#1}}
\newcommand{\code}[1]{\texttt{#1}}
\newcommand{\classname}[1]{\texttt{#1}}
\newcommand{\methodname}[1]{\texttt{#1}}

% Proper quotation marks
\newcommand{\mnemonic}[1]{``#1''}


\title{Fundamentals of Machine Learning (4341603) - Winter 2023 Solution}
\date{February 2, 2024}

\begin{document}
\maketitle

\questionmarks{1(a)}{3}{Human learning વ્યાખ્યાયિત કરો અને સમજાવો કે machine learning human learning થી કેવી રીતે અલગ છે?}
\begin{solutionbox}
\textbf{Human Learning} એ અનુભવ, અવલોકન અને તર્ક દ્વારા જ્ઞાન મેળવવાની પ્રક્રિયા છે.

\begin{center}
\captionof{table}{Human Learning વિ Machine Learning}
\begin{tabulary}{\linewidth}{L L L}
\hline
\textbf{પાસાં} & \textbf{Human Learning} & \textbf{Machine Learning} \\
\hline
\textbf{પદ્ધતિ} & અનુભવ, પ્રયાસ અને ભૂલ & ડેટા અને અલ્ગોરિધમ \\
\textbf{ઝડપ} & ધીમી, ક્રમશઃ & ઝડપી પ્રોસેસિંગ \\
\textbf{ડેટા જરૂરિયાત} & મર્યાદિત ઉદાહરણો જોઈએ & મોટા ડેટાસેટ જરૂરી \\
\hline
\end{tabulary}
\end{center}

\textbf{Machine Learning}: ડેટામાં પેટર્ન ઓળખવા માટે અલ્ગોરિધમનો ઉપયોગ કરીને આપોઆપ શીખવાની પ્રક્રિયા.
\end{solutionbox}

\begin{mnemonicbox}
\mnemonic{Humans Experience, Machines Analyze Data (HEMAD)}
\end{mnemonicbox}

\questionmarks{1(b)}{4}{ફાઇનાન્સ અને બેંકિંગમાં મશીન લર્નિંગના ઉપયોગનું વર્ણન કરો.}
\begin{solutionbox}
\begin{center}
\captionof{table}{ફાઇનાન્સ અને બેંકિંગમાં ઉપયોગો}
\begin{tabulary}{\linewidth}{L L L}
\hline
\textbf{ઉપયોગ} & \textbf{હેતુ} & \textbf{ફાયદો} \\
\hline
\textbf{Fraud Detection} & શંકાસ્પદ ટ્રાન્ઝેક્શન ઓળખવા & નાણાકીય નુકસાન ઘટાડવું \\
\textbf{Credit Scoring} & લોન ડિફોલ્ટ રિસ્ક આંકવી & વધુ સારા લેન્ડિંગ નિર્ણયો \\
\textbf{Algorithmic Trading} & આપોઆપ ટ્રેડિંગ નિર્ણયો & ઝડપી માર્કેટ રિસ્પોન્સ \\
\hline
\end{tabulary}
\end{center}

\textbf{Risk Assessment}: ગ્રાહકની ક્રેડિટવર્થીનેસની આગાહી માટે ML ડેટાનું વિશ્લેષણ કરે છે.
\textbf{Customer Service}: NLP નો ઉપયોગ કરીને ચેટબોટ્સ 24/7 સપોર્ટ આપે છે.
\textbf{Regulatory Compliance}: શંકાસ્પદ પ્રવૃત્તિઓ માટે આપોઆપ મોનિટરિંગ.
\end{solutionbox}

\begin{mnemonicbox}
\mnemonic{Finance Needs Smart Analysis (FNSA)}
\end{mnemonicbox}

\questionmarks{1(c)}{7}{સુપરવાઇઝ્ડ લર્નિંગ, અનસુપરવાઇઝ્ડ લર્નિંગ અને રિઇન્ફોર્સમેન્ટ લર્નિંગ વચ્ચે તફાવત આપો.}
\begin{solutionbox}
\begin{center}
\captionof{table}{તુલનાત્મક કોષ્ટક}
\begin{tabulary}{\linewidth}{L L L L}
\hline
\textbf{લક્ષણ} & \textbf{Supervised} & \textbf{Unsupervised} & \textbf{Reinforcement} \\
\hline
\textbf{ડેટા પ્રકાર} & લેબલ્ડ ડેટા & અનલેબલ્ડ ડેટા & પર્યાવરણ ઇન્ટરેક્શન \\
\textbf{લક્ષ્ય} & આઉટપુટની આગાહી & પેટર્નો શોધવા & રિવોર્ડ વધારવા \\
\textbf{ઉદાહરણો} & Classification & Clustering & Game playing \\
\textbf{Feedback} & તાત્કાલિક & કંઈ નહીં & વિલંબિત પુરસ્કારો \\
\hline
\end{tabulary}
\end{center}

\textbf{Supervised Learning}: સાચા જવાબો સાથે શિક્ષક દ્વારા માર્ગદર્શિત શીખવું.

\textbf{Unsupervised Learning}: ડેટામાં છુપાયેલા પેટર્નોની સ્વ-શોધ.

\textbf{Reinforcement Learning}: પુરસ્કાર/દંડ સાથે ટ્રાયલ અને એરર દ્વારા શીખવું.
\end{solutionbox}

\begin{mnemonicbox}
\mnemonic{Supervised Teachers, Unsupervised Explores, Reinforcement Rewards (STUER)}
\end{mnemonicbox}

\questionmarks{1(c) OR}{7}{મશીન લર્નિંગમાં વપરાતા વિવિધ TOOLS અને ટેકનોલોજી સમજાવો.}
\begin{solutionbox}
\begin{center}
\captionof{table}{ML Tools અને Technologies}
\begin{tabulary}{\linewidth}{L L L}
\hline
\textbf{કેટેગરી} & \textbf{Tools} & \textbf{હેતુ} \\
\hline
\textbf{Programming} & Python, R, Java & અલ્ગોરિધમ ઇમ્પ્લિમેન્ટેશન \\
\textbf{Libraries} & Scikit-learn, TensorFlow & તૈયાર અલ્ગોરિધમ \\
\textbf{Visualization} & Matplotlib, Seaborn & ડેટા વિઝ્યુઅલાઇઝેશન \\
\textbf{Data Processing} & Pandas, NumPy & ડેટા મેનિપ્યુલેશન \\
\hline
\end{tabulary}
\end{center}

\textbf{મુખ્ય ટેકનોલોજીઓ:}
\begin{itemize}
    \item \textbf{Cloud Platforms}: AWS, Google Cloud સ્કેલેબલ કમ્પ્યુટિંગ માટે
    \item \textbf{Development Environments}: Jupyter Notebook, Google Colab
    \item \textbf{Big Data Tools}: મોટા ડેટાસેટ માટે Spark, Hadoop
\end{itemize}
\end{solutionbox}

\begin{mnemonicbox}
\mnemonic{Python Libraries Visualize Data Effectively (PLVDE)}
\end{mnemonicbox}

\questionmarks{2(a)}{3}{એક ઉદાહરણ સાથે outliers ને વ્યાખ્યાયિત કરો.}
\begin{solutionbox}
\textbf{વ્યાખ્યા}: Outliers એવા ડેટા પોઇન્ટ્સ છે જે ડેટાસેટમાં અન્ય અવલોકનોથી નોંધપાત્ર રીતે અલગ હોય છે.

\begin{center}
\captionof{table}{ઉદાહરણ: વિદ્યાર્થીઓની ઊંચાઈ}
\begin{tabulary}{\linewidth}{L L}
\hline
\textbf{ઊંચાઈ (cm)} & \textbf{વર્ગીકરણ} \\
\hline
165, 170, 168, 172 & સામાન્ય મૂલ્યો \\
195 & Outlier (ખૂબ ઊંચું) \\
140 & Outlier (ખૂબ નીચું) \\
\hline
\end{tabulary}
\end{center}

\textbf{શોધ}: Quartiles થી 1.5 $\times$ IQR થી વધુ મૂલ્યો.
\textbf{અસર}: આંકડાકીય વિશ્લેષણ અને મોડલ પર્ફોર્મન્સને અસર કરી શકે.
\end{solutionbox}

\begin{mnemonicbox}
\mnemonic{Outliers Stand Apart (OSA)}
\end{mnemonicbox}

\questionmarks{2(b)}{4}{રીગ્રેશન સ્ટેપ્સ વિગતવાર સમજાવો.}
\begin{solutionbox}
\begin{center}
\begin{tikzpicture}[node distance=1.5cm, auto]
    \node [gtu block] (Coll) {ડાટા કલેક્શન};
    \node [gtu block, below=0.8cm of Coll] (Prep) {ડાટા પ્રીપ્રોસેસિંગ};
    \node [gtu block, below=0.8cm of Prep] (Feat) {ફીચર સિલેક્શન};
    \node [gtu block, below=0.8cm of Feat] (Train) {મોડલ ટ્રેનિંગ};
    \node [gtu block, below=0.8cm of Train] (Eval) {મોડલ ઇવેલ્યુએશન};
    \node [gtu block, below=0.8cm of Eval] (Pred) {આગાહી (Prediction)};
    
    \path [gtu arrow] (Coll) -- (Prep);
    \path [gtu arrow] (Prep) -- (Feat);
    \path [gtu arrow] (Feat) -- (Train);
    \path [gtu arrow] (Train) -- (Eval);
    \path [gtu arrow] (Eval) -- (Pred);
\end{tikzpicture}
\captionof{figure}{રીગ્રેશન પ્રોસેસ સ્ટેપ્સ}
\end{center}

\textbf{વિગતવાર સ્ટેપ્સ:}
\begin{itemize}
    \item \textbf{Data Collection}: ઇનપુટ-આઉટપુટ જોડી સાથે સંબંધિત ડેટાસેટ એકત્રિત કરવું.
    \item \textbf{Preprocessing}: ડેટા સાફ કરવું, ખોવાયેલા મૂલ્યો સંભાળવા, features ને normalize કરવા.
    \item \textbf{Feature Selection}: લક્ષ્યને અસર કરતા સંબંધિત variables પસંદ કરવા.
    \item \textbf{Model Training}: આગાહીની ભૂલો ન્યૂનતમ કરવા માટે રીગ્રેશન લાઇન ફિટ કરવી.
\end{itemize}
\end{solutionbox}

\begin{mnemonicbox}
\mnemonic{Data Preprocessing Features Train Evaluation Predicts (DPFTEP)}
\end{mnemonicbox}

\questionmarks{2(c)}{7}{ચોકસાઈ વ્યાખ્યાયિત કરો અને નીચેના binary classifier ની confusion matrix માટે વિવિધ માપન પરિમાણો શોધો જેમ કે 1. Accuracy 2. Precision.}
\begin{solutionbox}
\textbf{Confusion Matrix વિશ્લેષણ:}

\begin{center}
\captionof{table}{આપેલ Confusion Matrix}
\begin{tabulary}{\linewidth}{L L L}
\hline
 & \textbf{અનુમાનિત ના} & \textbf{અનુમાનિત હા} \\
\hline
\textbf{વાસ્તવિક ના} & 10 (TN) & 3 (FP) \\
\textbf{વાસ્તવિક હા} & 2 (FN) & 15 (TP) \\
\hline
\end{tabulary}
\end{center}

\textbf{ગણતરીઓ:}
\begin{itemize}
    \item \textbf{Accuracy} = \((TP+TN)/(TP+TN+FP+FN) = (15+10)/(15+10+3+2) = 25/30 = 0.8333\)
    \item \textbf{પરિણામ: 83.33\%}
    
    \item \textbf{Precision} = \(TP/(TP+FP) = 15/(15+3) = 15/18 = 0.8333\)
    \item \textbf{પરિણામ: 83.33\%}
\end{itemize}

\textbf{વ્યાખ્યાઓ:}
\begin{itemize}
    \item \textbf{Accuracy}: કુલ આગાહીઓમાંથી સાચી આગાહીઓનું પ્રમાણ.
    \item \textbf{Precision}: બધી positive આગાહીઓમાંથી true positive આગાહીઓનું પ્રમાણ.
\end{itemize}
\end{solutionbox}

\begin{mnemonicbox}
\mnemonic{Accuracy Counts All, Precision Picks Positives (ACAPP)}
\end{mnemonicbox}

\questionmarks{2(a) OR}{3}{Feature સબસેટ પસંદગીના મૂળભૂત પગલાઓને ઓળખો.}
\begin{solutionbox}
\begin{center}
\begin{tikzpicture}[node distance=1.5cm, auto]
    \node [gtu block] (Orig) {મૂળ ફીચર્સ};
    \node [gtu block, right=of Orig] (Gen) {સબસેટ્સ જનરેટ};
    \node [gtu block, right=of Gen] (Eval) {સબસેટ્સ મૂલ્યાંકન};
    \node [gtu block, right=of Eval] (Sel) {શ્રેષ્ઠ પસંદગી};
    
    \path [gtu arrow] (Orig) -- (Gen);
    \path [gtu arrow] (Gen) -- (Eval);
    \path [gtu arrow] (Eval) -- (Sel);
\end{tikzpicture}
\captionof{figure}{Feature Subset Selection Process}
\end{center}

\textbf{મૂળભૂત પગલાઓ:}
\begin{enumerate}
    \item \textbf{Generation}: Features ના વિવિધ સંયોજનો બનાવવા.
    \item \textbf{Evaluation}: પ્રત્યેક સબસેટને પર્ફોર્મન્સ મેટ્રિક્સ વાપરીને ટેસ્ટ કરવા.
    \item \textbf{Selection}: માપદંડોના આધારે શ્રેષ્ઠ સબસેટ પસંદ કરવા.
\end{enumerate}
\end{solutionbox}

\begin{mnemonicbox}
\mnemonic{Generate, Evaluate, Select (GES)}
\end{mnemonicbox}

\questionmarks{2(b) OR}{4}{KNN અલ્ગોરિધમની તાકાત અને નબળાઈની ચર્ચા કરો.}
\begin{solutionbox}
\begin{center}
\captionof{table}{KNN અલ્ગોરિધમ વિશ્લેષણ}
\begin{tabulary}{\linewidth}{L L}
\hline
\textbf{તાકાતો} & \textbf{નબળાઈઓ} \\
\hline
સમજવામાં સરળ & કમ્પ્યુટેશનલી મોંઘું \\
Training ની જરૂર નથી (Lazy) & અપ્રસ્તુત features ને સંવેદનશીલ \\
Non-linear ડેટા સાથે કામ કરે & High dimensions સાથે performance ઘટે \\
નાના ડેટાસેટ માટે અસરકારક & શ્રેષ્ઠ K value પસંદગી જરૂરી \\
\hline
\end{tabulary}
\end{center}

\textbf{મુખ્ય મુદ્દાઓ:}
\begin{itemize}
    \item \textbf{Lazy Learning}: સ્પષ્ટ training phase ની જરૂર નથી.
    \item \textbf{Distance-Based}: પડોશીની નજીકતા આધારિત વર્ગીકરણ.
\end{itemize}
\end{solutionbox}

\begin{mnemonicbox}
\mnemonic{Simple but Slow, Effective but Expensive (SBSEBE)}
\end{mnemonicbox}

\questionmarks{2(c) OR}{7}{ભૂલ-દર વ્યાખ્યાયિત કરો અને નીચેના binary classifier ની confusion matrix માટે વિવિધ માપન પરિમાણો શોધો જેમ કે 1. Error value 2. Recall.}
\begin{solutionbox}
\textbf{Confusion Matrix વિશ્લેષણ:}

\begin{center}
\captionof{table}{આપેલ Confusion Matrix}
\begin{tabulary}{\linewidth}{L L L}
\hline
 & \textbf{અનુમાનિત ના} & \textbf{અનુમાનિત હા} \\
\hline
\textbf{વાસ્તવિક ના} & 20 (TN) & 3 (FP) \\
\textbf{વાસ્તવિક હા} & 2 (FN) & 15 (TP) \\
\hline
\end{tabulary}
\end{center}

\textbf{ગણતરીઓ:}
\begin{itemize}
    \item \textbf{Error Rate} = \((FP+FN)/(Total) = (3+2)/(15+20+3+2) = 5/40 = 0.125\)
    \item \textbf{પરિણામ: 12.5\%}
    
    \item \textbf{Recall} = \(TP/(TP+FN) = 15/(15+2) = 15/17 = 0.8824\)
    \item \textbf{પરિણામ: 88.24\%}
\end{itemize}

\textbf{વ્યાખ્યાઓ:}
\begin{itemize}
    \item \textbf{Error Rate}: કુલ આગાહીઓમાંથી ખોટી આગાહીઓનું પ્રમાણ.
    \item \textbf{Recall}: વાસ્તવિક positives માંથી સાચી રીતે ઓળખાયેલાનું પ્રમાણ.
\end{itemize}
\end{solutionbox}

\begin{mnemonicbox}
\mnemonic{Error Excludes, Recall Retrieves (EERR)}
\end{mnemonicbox}

\questionmarks{3(a)}{3}{Unsupervised learning ના કોઈ પણ ત્રણ ઉદાહરણો આપો.}
\begin{solutionbox}
\begin{center}
\captionof{table}{Unsupervised Learning ઉદાહરણો}
\begin{tabulary}{\linewidth}{L L L}
\hline
\textbf{ઉદાહરણ} & \textbf{વર્ણન} & \textbf{ઉપયોગ} \\
\hline
\textbf{Customer Segmentation} & વર્તન દ્વારા ગ્રાહકોને જૂથબદ્ધ કરવા & માર્કેટિંગ વ્યૂહરચના \\
\textbf{Document Classification} & વિષયો દ્વારા દસ્તાવેજો ગોઠવવા & માહિતી પુનઃપ્રાપ્તિ \\
\textbf{Gene Sequencing} & સમાન DNA પેટર્ન જૂથબદ્ધ કરવા & તબીબી સંશોધન \\
\hline
\end{tabulary}
\end{center}

\textbf{અન્ય ઉદાહરણો:}
\begin{itemize}
    \item \textbf{Market Basket Analysis}: ઉત્પાદન ખરીદીના પેટર્ન શોધવા.
    \item \textbf{Anomaly Detection}: ડેટામાં અસામાન્ય પેટર્ન શોધવા.
\end{itemize}
\end{solutionbox}

\begin{mnemonicbox}
\mnemonic{Customers, Documents, Genes Group Automatically (CDGGA)}
\end{mnemonicbox}

\questionmarks{3(b)}{4}{નીચેના ડેટા માટે સરેરાશ અને મધ્યક શોધો: 4,6,7,8,9,12,14,15,20}
\begin{solutionbox}
\textbf{ડેટા}: 4, 6, 7, 8, 9, 12, 14, 15, 20 (પહેલેથી જ સૉર્ટ થયેલ)

\textbf{સરેરાશ (Mean) ગણતરી:}
\begin{itemize}
    \item સરવાળો = \(4+6+7+8+9+12+14+15+20 = 95\)
    \item ગણતરી = 9
    \item \textbf{Mean} = \(95/9 = 10.56\)
\end{itemize}

\textbf{મધ્યક (Median) ગણતરી:}
\begin{itemize}
    \item N = 9 (એકી સંખ્યા)
    \item મધ્ય મૂલ્ય = \((N+1)/2 = 5\)મી સ્થિતિ
    \item 5મી સ્થિતિએ મૂલ્ય 9 છે
    \item \textbf{Median} = 9
\end{itemize}
\end{solutionbox}

\begin{mnemonicbox}
\mnemonic{Mean Averages All, Median Middle Value (MAAMV)}
\end{mnemonicbox}

\questionmarks{3(c)}{7}{k-ફોલ્ડ ક્રોસ વેલિડેશન પદ્ધતિનું વિગતવાર વર્ણન કરો.}
\begin{solutionbox}
\begin{center}
\begin{tikzpicture}[node distance=1.5cm, auto]
    \node [gtu block] (Data) {ડેટાસેટ};
    \node [gtu block, right=of Data] (Split) {K ભાગોમાં વિભાજન};
    \node [gtu block, below=1cm of Split] (Train) {K-1 ભાગો પર ટ્રેનિંગ};
    \node [gtu block, right=of Train] (Test) {1 ભાગ પર ટેસ્ટિંગ};
    \node [gtu block, right=of Test] (Avg) {સરેરાશ};
    
    \path [gtu arrow] (Data) -- (Split);
    \path [gtu arrow] (Split) -- (Train);
    \path [gtu arrow] (Train) -- (Test);
    \path [gtu arrow] (Test) -- node[above] {K વાર} (Avg);
    \path [gtu arrow] (Test.south) -- ++(0,-0.5) -| (Train.south);
\end{tikzpicture}
\captionof{figure}{K-Fold Cross Validation}
\end{center}

\textbf{પ્રોસેસ સ્ટેપ્સ:}
\begin{enumerate}
    \item \textbf{ડેટા વિભાજન}: ડેટાને K સમાન ભાગોમાં વહેંચવું.
    \item \textbf{પુનરાવર્તિત Training}: મોડલ ટ્રેનિંગ માટે K-1 folds નો ઉપયોગ કરવો.
    \item \textbf{Validation}: મોડલને બાકીના fold પર ટેસ્ટ કરવું.
    \item \textbf{સરેરાશ}: K વાર પુનરાવર્તન કરો અને performance મેટ્રિક્સની સરેરાશ કાઢો.
\end{enumerate}

\textbf{ફાયદાઓ:}
\begin{itemize}
    \item \textbf{નિષ્પક્ષ અંદાજ}: દરેક ડેટા પોઇન્ટ training અને testing બંને માટે વપરાય છે.
    \item \textbf{Overfitting ઘટાડવું}: અનેક validation રાઉન્ડ વિશ્વસનીયતા વધારે છે.
\end{itemize}
\end{solutionbox}

\begin{mnemonicbox}
\mnemonic{K-fold Keeps Keen Knowledge (KKKK)}
\end{mnemonicbox}

\questionmarks{3(a) OR}{3}{Multiple linear રીગ્રેશનની કોઈ પણ ત્રણ એપ્લિકેશન આપો.}
\begin{solutionbox}
\begin{center}
\captionof{table}{Multiple Linear Regression એપ્લિકેશન}
\begin{tabulary}{\linewidth}{L L L}
\hline
\textbf{એપ્લિકેશન} & \textbf{Variables} & \textbf{હેતુ} \\
\hline
\textbf{House Price Prediction} & Size, location, age & રિયલ એસ્ટેટ વેલ્યુએશન \\
\textbf{Sales Forecasting} & Marketing spend, season & બિઝનેસ પ્લાનિંગ \\
\textbf{Medical Diagnosis} & Symptoms, age, history & રોગની આગાહી \\
\hline
\end{tabulary}
\end{center}
\end{solutionbox}

\begin{mnemonicbox}
\mnemonic{Houses, Sales, Medicine Predict Multiple Variables (HSMPV)}
\end{mnemonicbox}

\questionmarks{3(b) OR}{4}{નીચેના ડેટા માટે માનક વિચલન શોધો: 4,15,20,28,35,45}
\begin{solutionbox}
\textbf{ડેટા}: 4, 15, 20, 28, 35, 45 (N=6)

\textbf{Step 1: સરેરાશ ગણતરી (Mean)}
\begin{itemize}
    \item સરવાળો = 147
    \item Mean (\(\bar{x}\)) = \(147/6 = 24.5\)
\end{itemize}

\textbf{Step 2: વર્ગ વિચલન (Squared Deviations)}
\begin{itemize}
    \item \((4-24.5)^2 = 420.25\), \((15-24.5)^2 = 90.25\), ...
\end{itemize}

\textbf{Step 3: Variance અને Std Dev}
\begin{itemize}
    \item વર્ગ વિચલનોનો સરવાળો = 1073.5
    \item Variance (\(\sigma^2\)) = \(1073.5/6 = 178.92\)
    \item \textbf{માનક વિચલન (\(\sigma\))} = \(\sqrt{178.92} = 13.376\)
\end{itemize}
\end{solutionbox}

\begin{mnemonicbox}
\mnemonic{Deviation Measures Data Spread (DMDS)}
\end{mnemonicbox}

\questionmarks{3(c) OR}{7}{બેગિંગ અને બૂસ્ટિંગને વિગતવાર સમજાવો.}
\begin{solutionbox}
\begin{center}
\captionof{table}{Bagging વિ Boosting}
\begin{tabulary}{\linewidth}{L L L}
\hline
\textbf{પાસું} & \textbf{Bagging} & \textbf{Boosting} \\
\hline
\textbf{વ્યૂહરચના} & સમાંતર training & ક્રમિક training \\
\textbf{ડેટા સેમ્પલિંગ} & રેન્ડમ (replacement સાથે) & વેઇટેડ સેમ્પલિંગ \\
\textbf{લક્ષ્ય} & Variance ઘટાડે & Bias ઘટાડે \\
\hline
\end{tabulary}
\end{center}

\textbf{Bagging (Bootstrap Aggregating):}
ડેટાના રેન્ડમ સબસેટ્સનો ઉપયોગ કરીને સમાંતરમાં બહુવિધ સ્વતંત્ર મોડલ્સને ટ્રેન કરે છે અને તેમની આગાહીઓની સરેરાશ કાઢે છે.

\begin{center}
\begin{tikzpicture}[node distance=1.5cm]
    \node [gtu block] (Data) {મૂળ ડેટા};
    \node [gtu block, below left=1.5cm and 1cm of Data] (S1) {સેમ્પલ 1};
    \node [gtu block, below=1.5cm of Data] (S2) {સેમ્પલ 2};
    \node [gtu block, below right=1.5cm and 1cm of Data] (Sn) {સેમ્પલ n};
    
    \node [gtu state, below=0.8cm of S1] (M1) {મોડલ 1};
    \node [gtu state, below=0.8cm of S2] (M2) {મોડલ 2};
    \node [gtu state, below=0.8cm of Sn] (Mn) {મોડલ n};
    
    \node [gtu block, below=2cm of M2] (Final) {અંતિમ આગાહી};
    
    \path [gtu arrow] (Data) -- (S1);
    \path [gtu arrow] (Data) -- (S2);
    \path [gtu arrow] (Data) -- (Sn);
    \path [gtu arrow] (S1) -- (M1);
    \path [gtu arrow] (S2) -- (M2);
    \path [gtu arrow] (Sn) -- (Mn);
    \path [gtu arrow] (M1) -- (Final);
    \path [gtu arrow] (M2) -- (Final);
    \path [gtu arrow] (Mn) -- (Final);
\end{tikzpicture}
\captionof{figure}{Bagging Process}
\end{center}

\textbf{Boosting:}
મોડલ્સને ક્રમિક રીતે ટ્રેન કરે છે, જ્યાં દરેક નવું મોડલ અગાઉના મોડલ્સ દ્વારા કરવામાં આવેલી ભૂલો પર ધ્યાન કેન્દ્રિત કરે છે.
\end{solutionbox}

\begin{mnemonicbox}
\mnemonic{Bagging Builds Parallel, Boosting Builds Sequential (BBPBS)}
\end{mnemonicbox}

\questionmarks{4(a)}{3}{વ્યાખ્યાયિત કરો: Support, Confidence.}
\begin{solutionbox}
\begin{center}
\captionof{table}{Association Rule મેટ્રિક્સ}
\begin{tabulary}{\linewidth}{L L}
\hline
\textbf{મેટ્રિક} & \textbf{વ્યાખ્યા અને ફોર્મ્યુલા} \\
\hline
\textbf{Support} & ટ્રાન્ઝેક્શનમાં itemset ની આવર્તન. \\
 & \(Support(A) = Count(A)/Total\) \\
\hline
\textbf{Confidence} & નિયમની શરતી સંભાવના. \\
 & \(Confidence(A \to B) = Support(A \cup B)/Support(A)\) \\
\hline
\end{tabulary}
\end{center}

\textbf{ઉદાહરણ:}
\begin{itemize}
    \item \textbf{Support}: 60\% ટ્રાન્ઝેક્શનમાં બ્રેડ છે.
    \item \textbf{Confidence}: 80\% બ્રેડ ખરીદનારા લોકો બટર પણ ખરીદે છે.
\end{itemize}
\end{solutionbox}

\begin{mnemonicbox}
\mnemonic{Support Shows Frequency, Confidence Shows Connection (SSFC)}
\end{mnemonicbox}

\questionmarks{4(b)}{4}{લોજિસ્ટિક રીગ્રેશનની કોઈ પણ બે એપ્લિકેશનને સમજાવો.}
\begin{solutionbox}
\begin{center}
\captionof{table}{Logistic Regression એપ્લિકેશન}
\begin{tabulary}{\linewidth}{L L L}
\hline
\textbf{એપ્લિકેશન} & \textbf{વર્ણન} & \textbf{પરિણામ} \\
\hline
\textbf{Email Spam} & સ્પામ ડિટેક્શન & Spam/Not Spam \\
\textbf{Medical Diagnosis} & લક્ષણો પરથી રોગની આગાહી & રોગ/સ્વસ્થ \\
\textbf{Credit Approval} & લોન રિસ્ક આકારણી & મંજૂર/નામંજૂર \\
\hline
\end{tabulary}
\end{center}

\textbf{મુખ્ય લાક્ષણિકતાઓ:}
\begin{itemize}
    \item \textbf{Binary Classification}: સંભાવના (0 થી 1) આગાહી કરે છે.
    \item \textbf{Sigmoid Function}: આઉટપુટને S-shaped વળાંકમાં મેપ કરે છે.
\end{itemize}
\end{solutionbox}

\begin{mnemonicbox}
\mnemonic{Logistic Limits Linear Logic (LLLL)}
\end{mnemonicbox}

\questionmarks{4(c)}{7}{Machine learning માં Numpy અને Pandas ના મુખ્ય હેતુની ચર્ચા કરો.}
\begin{solutionbox}
\textbf{NumPy} મોટા, મલ્ટી-ડાયમેન્શનલ એરે અને મેટ્રિસિસ માટે સપોર્ટ પૂરો પાડે છે.
\textbf{Pandas} ન્યૂમેરિકલ ટેબલ્સ અને ટાઇમ સીરિઝના મેનિપ્યુલેશન માટે ડેટા સ્ટ્રક્ચર્સ પૂરા પાડે છે.

\begin{center}
\begin{tikzpicture}[node distance=1.5cm]
    \node [gtu block] (NumPy) {NumPy};
    \node [gtu state, below left=1.5cm and 0.5cm of NumPy] (Array) {Array Ops};
    \node [gtu state, below right=1.5cm and 0.5cm of NumPy] (Math) {Math Funcs};
    \node [gtu state, below=1.5cm of NumPy] (LinAlg) {Linear Algebra};
    
    \path [gtu arrow] (NumPy) -- (Array);
    \path [gtu arrow] (NumPy) -- (Math);
    \path [gtu arrow] (NumPy) -- (LinAlg);
\end{tikzpicture}
\captionof{figure}{NumPy Features}
\end{center}

\begin{center}
\captionof{table}{તુલના}
\begin{tabulary}{\linewidth}{L L L}
\hline
\textbf{Library} & \textbf{પ્રાથમિક હેતુ} & \textbf{મુખ્ય Features} \\
\hline
\textbf{NumPy} & Numerical Computing & N-dim arrays, broadcasting \\
\textbf{Pandas} & Data Manipulation & DataFrames, cleaning \\
\hline
\end{tabulary}
\end{center}
\end{solutionbox}

\begin{mnemonicbox}
\mnemonic{NumPy Numbers, Pandas Processes Data (NNPD)}
\end{mnemonicbox}

\questionmarks{4(a) OR}{3}{સુપરવાઇઝ્ડ લર્નિંગના કોઈ પણ ત્રણ ઉદાહરણો આપો.}
\begin{solutionbox}
\begin{center}
\captionof{table}{Supervised Learning ઉદાહરણો}
\begin{tabulary}{\linewidth}{L L L}
\hline
\textbf{ઉદાહરણ} & \textbf{પ્રકાર} & \textbf{Input \(\to\) Output} \\
\hline
\textbf{Email Classification} & Classification & Email features \(\to\) Spam/Not \\
\textbf{House Price Prediction} & Regression & House features \(\to\) કિંમત \\
\textbf{Image Recognition} & Classification & Pixels \(\to\) Object Class \\
\hline
\end{tabulary}
\end{center}
\end{solutionbox}

\begin{mnemonicbox}
\mnemonic{Emails, Houses, Images Learn Supervised (EHILS)}
\end{mnemonicbox}

\questionmarks{4(b) OR}{4}{એપ્રિઓરી અલ્ગોરિધમના કોઈ પણ બે એપ્લિકેશનો સમજાવો.}
\begin{solutionbox}
\begin{center}
\captionof{table}{Apriori એપ્લિકેશન}
\begin{tabulary}{\linewidth}{L L}
\hline
\textbf{એપ્લિકેશન} & \textbf{વર્ણન} \\
\hline
\textbf{Market Basket Analysis} & એકસાથે ખરીદાતી વસ્તુઓ શોધવી (દા.ત., બ્રેડ અને બટર). \\
\hline
\textbf{Web Usage Mining} & વેબસાઇટ UX સુધારવા માટે વિશ્વસનીય નેવિગેશન પેટર્ન શોધવી. \\
\hline
\end{tabulary}
\end{center}
\end{solutionbox}

\begin{mnemonicbox}
\mnemonic{Apriori Analyzes Associations Automatically (AAAA)}
\end{mnemonicbox}

\questionmarks{4(c) OR}{7}{Matplotlib ની વિશેષતાઓ અને એપ્લિકેશનો સમજાવો.}
\begin{solutionbox}
\textbf{Matplotlib} Python માં સ્ટેટિક, એનિમેટેડ અને ઇન્ટરેક્ટિવ વિઝ્યુલાઇઝેશન બનાવવા માટેની વ્યાપક લાઇબ્રેરી છે.

\begin{center}
\begin{tikzpicture}[node distance=1.5cm]
    \node [gtu block] (Mat) {Matplotlib};
    \node [gtu block, below left=1.5cm and 1cm of Mat] (2D) {2D Plotting};
    \node [gtu block, below right=1.5cm and 1cm of Mat] (3D) {3D Plotting};
    
    \node [gtu state, below=0.8cm of 2D] (2Dtype) {Line, Bar, Scatter};
    \node [gtu state, below=0.8cm of 3D] (3Dtype) {Surface, Mesh};
    
    \path [gtu arrow] (Mat) -- (2D);
    \path [gtu arrow] (Mat) -- (3D);
    \path [gtu arrow] (2D) -- (2Dtype);
    \path [gtu arrow] (3D) -- (3Dtype);
\end{tikzpicture}
\captionof{figure}{Matplotlib ક્ષમતાઓ}
\end{center}

\textbf{એપ્લિકેશન્સ:}
\begin{itemize}
    \item \textbf{Data Exploration}: ડેટા સમજવા માટે હિસ્ટોગ્રામ, સ્કેટર પ્લોટ.
    \item \textbf{Model Performance}: લોસ કર્વ્સ અને એક્યુરસી પ્લોટ કરવા.
    \item \textbf{Result Presentation}: પ્રકાશન-ગુણવત્તાવાળા ગ્રાફ્સ.
\end{itemize}
\end{solutionbox}

\begin{mnemonicbox}
\mnemonic{Matplotlib Makes Meaningful Visual Displays (MMVD)}
\end{mnemonicbox}

\questionmarks{5(a)}{3}{Numpy ના મુખ્ય features ની યાદી બનાવો.}
\begin{solutionbox}
\textbf{NumPy ના Features:}
\begin{itemize}
    \item \textbf{N-dimensional Arrays}: ઝડપી અને કાર્યક્ષમ મલ્ટી-ડાયમેન્શનલ એરે ઓબ્જેક્ટ (ndarray).
    \item \textbf{Broadcasting}: વિવિધ આકારોના એરે પર ઓપરેશન્સ કરવા માટેના ફંક્શન્સ.
    \item \textbf{Linear Algebra}: મેટ્રિક્સ ઓપરેશન્સ અને Fourier ટ્રાન્સફોર્મ્સ માટે બિલ્ટ-ઇન સપોર્ટ.
    \item \textbf{C/C++ Integration}: C/C++ અને Fortran કોડને ઇન્ટિગ્રેટ કરવા માટેના સાધનો.
\end{itemize}
\end{solutionbox}

\begin{mnemonicbox}
\mnemonic{NumPy Numbers Need Neat Operations (NNNNO)}
\end{mnemonicbox}

\questionmarks{5(b)}{4}{પ્રોગ્રામમાં iris ડેટાસેટ Pandas Dataframe કેવી રીતે લોડ કરવો? ઉદાહરણ સાથે સમજાવો.}
\begin{solutionbox}
\begin{lstlisting}[language=Python]
import pandas as pd

# પદ્ધતિ 1: CSV ફાઇલમાંથી લોડ કરવું
df = pd.read_csv('iris.csv')

# પદ્ધતિ 2: sklearn માંથી (ML માં સામાન્ય)
from sklearn.datasets import load_iris
iris = load_iris()
df_iris = pd.DataFrame(iris.data, columns=iris.feature_names)

# પ્રથમ 5 પંક્તિઓ બતાવો
print(df.head())
\end{lstlisting}

\textbf{સમજૂતી:}
\begin{itemize}
    \item \code{pd.read\_csv()}: CSV ફાઇલો વાંચવા માટેનું ફંક્શન.
    \item \code{df.head()}: પ્રથમ n પંક્તિઓ પાછી આપે છે (ડિફોલ્ટ 5).
\end{itemize}
\end{solutionbox}

\begin{mnemonicbox}
\mnemonic{Pandas Reads CSV Files Easily (PRCFE)}
\end{mnemonicbox}

\questionmarks{5(c)}{7}{સુપરવાઇઝ્ડ લર્નિંગ અને અનસુપરવાઇઝ્ડ લર્નિંગની સરખામણી કરો અને કોન્ટ્રાસ્ટ કરો.}
\begin{solutionbox}
\begin{center}
\begin{tikzpicture}[node distance=1.5cm]
    \node [gtu block] (ML) {Machine Learning};
    \node [gtu block, below left=1.5cm and 1cm of ML] (Sup) {Supervised};
    \node [gtu block, below right=1.5cm and 1cm of ML] (Unsup) {Unsupervised};
    
    \node [gtu state, below=0.8cm of Sup] (SupEx) {Classification\\Regression};
    \node [gtu state, below=0.8cm of Unsup] (UnsupEx) {Clustering\\Association};
    
    \path [gtu arrow] (ML) -- (Sup);
    \path [gtu arrow] (ML) -- (Unsup);
    \path [gtu arrow] (Sup) -- (SupEx);
    \path [gtu arrow] (Unsup) -- (UnsupEx);
\end{tikzpicture}
\captionof{figure}{ML Learning Types}
\end{center}

\begin{center}
\captionof{table}{તુલના}
\begin{tabulary}{\linewidth}{L L L}
\hline
\textbf{પાસું} & \textbf{Supervised} & \textbf{Unsupervised} \\
\hline
\textbf{ડેટા} & લેબલ્ડ & અનલેબલ્ડ \\
\textbf{લક્ષ્ય} & આઉટપુટની આગાહી & પેટર્ન શોધવી \\
\textbf{Feedback} & સીધો પ્રતિસાદ & કોઈ પ્રતિસાદ નહીં \\
\textbf{જટિલતા} & વેલિડેશન સરળ છે & વેલિડેશન અઘરું છે \\
\hline
\end{tabulary}
\end{center}
\end{solutionbox}

\begin{mnemonicbox}
\mnemonic{Supervised Seeks Specific Solutions, Unsupervised Uncovers Unknown (SSSUU)}
\end{mnemonicbox}

\questionmarks{5(a) OR}{3}{Pandas ની એપ્લિકેશન્સની યાદી બનાવો.}
\begin{solutionbox}
\begin{center}
\captionof{table}{Pandas એપ્લિકેશન}
\begin{tabulary}{\linewidth}{L L L}
\hline
\textbf{એપ્લિકેશન} & \textbf{વર્ણન} & \textbf{ક્ષેત્ર} \\
\hline
\textbf{Data Cleaning} & ખોવાયેલા ડેટાને સંભાળવું & General ML \\
\textbf{Financial Analysis} & શેરબજારના વલણો & Finance \\
\textbf{Recommendation} & વપરાશકર્તા વર્તન વિશ્લેષણ & E-commerce \\
\hline
\end{tabulary}
\end{center}
\end{solutionbox}

\begin{mnemonicbox}
\mnemonic{Pandas Processes Data Perfectly (PPDP)}
\end{mnemonicbox}

\questionmarks{5(b) OR}{4}{Matplotlib લાઇબ્રેરીનો ઉપયોગ કરીને આકૃતિ કેવી રીતે બનાવવી? ઉદાહરણ સાથે સમજાવો.}
\begin{solutionbox}
\begin{lstlisting}[language=Python]
import matplotlib.pyplot as plt

# સરળ લાઇન પ્લોટ
plt.plot([1, 2, 3], [1, 4, 9])

# x = 2 પર વર્ટિકલ લાઇન (લાલ ડેશ)
plt.axvline(x=2, color='red', linestyle='--')

# y = 4 પર હોરિઝોન્ટલ લાઇન (લીલી સીધી)
plt.axhline(y=4, color='green', linestyle='-')

plt.show()
\end{lstlisting}

\textbf{ફંક્શન્સ:}
\begin{itemize}
    \item \code{axvline(x)}: અક્ષો પર ઊભી રેખા ઉમેરે છે.
    \item \code{axhline(y)}: અક્ષો પર આડી રેખા ઉમેરે છે.
\end{itemize}
\end{solutionbox}

\questionmarks{5(c) OR}{7}{યોગ્ય વાસ્તવિક વિશ્વ ઉદાહરણોનો ઉપયોગ કરીને clustering ના concept નું વર્ણન કરો.}
\begin{solutionbox}
\textbf{Clustering} એ એક unsupervised learning તકનીક છે જે સમાન ડેટા પોઇન્ટ્સને જૂથબદ્ધ કરે છે જેથી એક જ જૂથમાંના બિંદુઓ અન્ય જૂથોના મુકાબલે એકબીજા સાથે વધુ સમાન હોય.

\begin{center}
\captionof{table}{Clustering એપ્લિકેશન}
\begin{tabulary}{\linewidth}{L L L}
\hline
\textbf{પ્રકાર} & \textbf{ઉદાહરણ} & \textbf{અસર} \\
\hline
\textbf{Customer Seg.} & ખરીદી વર્તન દ્વારા જૂથ & Targeted માર્કેટિંગ \\
\textbf{Image Seg.} & MRI માં ગાંઠ શોધવી & સુધારેલ નિદાન \\
\textbf{Gene Analysis} & expression દ્વારા genes જૂથ & દવા શોધ \\
\hline
\end{tabulary}
\end{center}

\begin{center}
\begin{tikzpicture}[node distance=1.5cm, auto]
    \node [gtu block] (Raw) {કાચો ડેટા};
    \node [gtu block, right=of Raw] (Feat) {ફીચર સિલેક્શન};
    \node [gtu block, right=of Feat] (Dist) {અંતર ગણતરી};
    \node [gtu block, below=1cm of Feat] (Form) {ક્લસ્ટર રચના};
    \node [gtu block, right=of Form] (Val) {વેલિડેશન};
    \node [gtu block, right=of Val] (Ins) {બિઝનેસ ઇનસાઇટ્સ};
    
    \path [gtu arrow] (Raw) -- (Feat);
    \path [gtu arrow] (Feat) -- (Dist);
    \path [gtu arrow] (Dist) |- (Form);
    \path [gtu arrow] (Form) -- (Val);
    \path [gtu arrow] (Val) -- (Ins);
\end{tikzpicture}
\captionof{figure}{Clustering પ્રોસેસ}
\end{center}

\textbf{વાસ્તવિક ઉદાહરણો:}
\begin{enumerate}
    \item \textbf{Customer Segmentation}: ઉચ્ચ-મૂલ્યના ગ્રાહકો વિ. મોસમી ખરીદદારોને ઓળખવા.
    \item \textbf{Social Media Analysis}: વપરાશકર્તાઓને રુચિઓ દ્વારા જૂથબદ્ધ કરવા (દા.ત. સ્પોર્ટ્સ, ટેક).
    \item \textbf{Market Research}: સમાન ઉત્પાદન જરૂરિયાતો ધરાવતા સેગમેન્ટ્સ શોધવા.
\end{enumerate}
\end{solutionbox}

\begin{mnemonicbox}
\mnemonic{Clustering Creates Clear Categories (CCCC)}
\end{mnemonicbox}

\end{document}
