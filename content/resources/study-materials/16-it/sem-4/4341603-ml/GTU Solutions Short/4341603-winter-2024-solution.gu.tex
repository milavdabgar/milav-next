\documentclass{article}

% content/resources/templates/preamble.tex
\usepackage[margin=0.6in]{geometry}
\author{Milav Dabgar}
\usepackage{amsmath,amssymb,amsthm}
\usepackage{booktabs}
\usepackage{multirow}
\usepackage{xcolor}
\usepackage{tcolorbox}
\tcbuselibrary{breakable,skins}
\usepackage[colorlinks=true,linkcolor=blue]{hyperref}
\usepackage{titlesec}
\usepackage{enumitem}
\usepackage{tikz}
\usepackage{pgfplots}
\usepackage{circuitikz}
\usepackage[version=4]{mhchem}
\usepackage{longtable}
\usepackage{array}
\usepackage{float}
\usepackage{caption}
\usepackage{listings}

\lstset{
  basicstyle=\small\ttfamily,
  breaklines=true,
  breakatwhitespace=false,
  postbreak=\mbox{\textcolor{red}{$\hookrightarrow$}\space},
  float=false,
  numbers=left,
  numberstyle=\tiny\color{gray},
  numbersep=10pt,
  xleftmargin=2em,
  keywordstyle=\color{blue},
  commentstyle=\color{green!60!black},
  stringstyle=\color{purple},
  backgroundcolor=\color{gray!5},
  showstringspaces=false,
  tabsize=2,
  captionpos=b,
  keepspaces=true,
  columns=flexible
}

\pgfplotsset{compat=1.18}
\usetikzlibrary{shapes,arrows,positioning,calc,patterns,decorations.pathmorphing,decorations.markings,arrows.meta}

% Color scheme
\definecolor{headcolor}{RGB}{0,102,204}
\definecolor{keycolor}{RGB}{220,20,60}
\definecolor{solutioncolor}{RGB}{34,139,34}
\definecolor{mnemoniccolor}{RGB}{148,0,211}
\definecolor{codecolor}{RGB}{0,0,100}

% Spacing
\setlength{\parskip}{3pt}
\setlist[itemize]{nosep}
\setlist[enumerate]{nosep}

% Title formatting
\titleformat{\section}{\Large\bfseries\color{headcolor}}{\thesection}{1em}{}
\titleformat{\subsection}{\large\bfseries\color{headcolor}}{\thesubsection}{1em}{}

% Pandoc tightlist compatibility
\providecommand{\tightlist}{%
  \setlength{\itemsep}{0pt}\setlength{\parskip}{0pt}}

% Pandoc longtable compatibility
\newcounter{none}
\def\thenone{}


% content/resources/templates/gujarati-boxes.tex
\usepackage{fontspec}
\usepackage{polyglossia}

% Set Gujarati as main language (document is primarily in Gujarati)
% Note: gloss-gujarati.ldf doesn't exist in polyglossia, but it will use hyphenation patterns
\setdefaultlanguage{gujarati}
\setotherlanguage{english}

% Configure Gujarati font properly
% Use Language=Default to prevent polyglossia from trying to add language-specific features
% that don't exist for Gujarati, which causes "empty feature" warnings
\newfontfamily\gujaratifont[Script=Gujarati,AutoFakeBold=2.5,AutoFakeSlant=0.3]{Noto Sans Gujarati}
\setmainfont[Script=Gujarati,AutoFakeBold=2.5,AutoFakeSlant=0.3]{Noto Sans Gujarati}
% Use Noto Sans Gujarati for monospace to support Gujarati in text
\setmonofont[Scale=0.9]{Noto Sans Gujarati}

% Configure English to use the same font
\newfontfamily\englishfont[Script=Gujarati,AutoFakeBold=2.5,AutoFakeSlant=0.3]{Noto Sans Gujarati}

% Translations for polyglossia
\gappto\captionsgujarati{
  \renewcommand{\tablename}{કોષ્ટક}
  \renewcommand{\figurename}{આકૃતિ}
}

% Helper for TikZ nodes to ensure Gujarati font
\newcommand{\gu}[1]{{\gujaratifont #1}}

% Custom environments
\newtcolorbox{solutionbox}{
    breakable,
    enhanced,
    colback=solutioncolor!5!white,
    colframe=solutioncolor!75!black,
    fonttitle=\bfseries,
    title=જવાબ
}

\newtcolorbox{solutionboxnobreak}{
 colback=solutioncolor!5!white,
 colframe=solutioncolor!75!black,
 fonttitle=\bfseries,
 title=જવાબ
}

\newtcolorbox{keyformula}{
 breakable,
 enhanced,
 colback=keycolor!5!white,
 colframe=keycolor!75!black,
 fonttitle=\bfseries,
 title=રાસાયણિક સમીકરણ/સૂત્ર
}

\newtcolorbox{mnemonicbox}{
 breakable,
 enhanced,
 colback=mnemoniccolor!5!white,
 colframe=mnemoniccolor!75!black,
 fonttitle=\bfseries,
 title=મેમરી ટ્રીક
}


% Custom commands for GTU solutions
% This file defines semantic commands for consistent formatting

% Question command with automatic formatting
\newcommand{\question}[2]{%
  \section*{Question #1}%
  \textbf{#2}%
}

% OR question variant
\newcommand{\questionor}[2]{%
  \section*{Question #1 OR}%
  \textbf{#2}%
}

% Proper table environment with caption
\newenvironment{answertable}[1]{%
  \begin{table}[htbp]
  \centering
  \caption{#1}
}{%
  \end{table}
}

% Proper figure environment for diagrams
\newenvironment{answerdiagram}[1]{%
  \begin{figure}[htbp]
  \centering
  \caption{#1}
}{%
  \end{figure}
}

% Semantic markup for key terms
\newcommand{\keyword}[1]{\textbf{#1}}
\newcommand{\code}[1]{\texttt{#1}}
\newcommand{\classname}[1]{\texttt{#1}}
\newcommand{\methodname}[1]{\texttt{#1}}

% Proper quotation marks
\newcommand{\mnemonic}[1]{``#1''}


\title{Fundamentals of Machine Learning (4341603) - Winter 2024 Solution}
\date{November 28, 2024}

\begin{document}
\maketitle

\questionmarks{1(a)}{3}{હ્યુમન લર્નિંગનું સંક્ષિપ્ત વર્ણન કરો.}
\begin{solutionbox}
\textbf{હ્યુમન લર્નિંગ} એ પ્રક્રિયા છે જેના દ્વારા માનવ અનુભવ, પ્રેક્ટિસ અને શિક્ષણ દ્વારા જ્ઞાન, કૌશલ્ય અને વર્તણૂક પ્રાપ્ત કરે છે.

\begin{center}
\captionof{table}{હ્યુમન લર્નિંગ પ્રક્રિયા}
\begin{tabulary}{\linewidth}{L L}
\hline
\textbf{પાસું} & \textbf{વર્ણન} \\
\hline
\textbf{અવલોકન} & પર્યાવરણમાંથી માહિતી એકત્રિત કરવી \\
\textbf{અનુભવ} & ટ્રાયલ અને એરર દ્વારા શીખવું \\
\textbf{અભ્યાસ} & કૌશલ્ય સુધારવા માટે પુનરાવર્તન \\
\textbf{સ્મૃતિ} & માહિતી સંગ્રહ અને પુનઃપ્રાપ્તિ \\
\hline
\end{tabulary}
\end{center}

\begin{itemize}
    \item \textbf{લર્નિંગ પ્રકારો}: દ્રશ્ય, શ્રાવ્ય, ગતિશીલ લર્નિંગ શૈલીઓ.
    \item \textbf{ફીડબેક લૂપ}: ભૂલો અને સફળતાઓમાંથી શીખવું.
    \item \textbf{અનુકૂલન}: નવી પરિસ્થિતિઓમાં જ્ઞાન લાગુ કરવાની ક્ષમતા.
\end{itemize}
\end{solutionbox}

\begin{mnemonicbox}
\mnemonic{અવલોકન, અનુભવ, અભ્યાસ, સ્મૃતિ (AAPS)}
\end{mnemonicbox}

\questionmarks{1(b)}{4}{તફાવત કરો: Supervised લર્નિંગ v/s Unsupervised લર્નિંગ}
\begin{solutionbox}
\begin{center}
\captionof{table}{Supervised vs Unsupervised લર્નિંગ}
\begin{tabulary}{\linewidth}{L L L}
\hline
\textbf{પેરામીટર} & \textbf{Supervised લર્નિંગ} & \textbf{Unsupervised લર્નિંગ} \\
\hline
\textbf{ટ્રેનિંગ ડેટા} & લેબલ થયેલ ડેટા & લેબલ વિનાનો ડેટા \\
\textbf{ધ્યેય} & આઉટપુટ આગાહી કરવી & પેટર્ન શોધવું \\
\textbf{ઉદાહરણ} & Classification, Regression & Clustering, Association \\
\textbf{ફીડબેક} & સીધો ફીડબેક & કોઈ ફીડબેક નથી \\
\hline
\end{tabulary}
\end{center}

\begin{itemize}
    \item \textbf{Supervised}: શિક્ષક સાચા જવાબો સાથે માર્ગદર્શન આપે છે.
    \item \textbf{Unsupervised}: માર્ગદર્શન વિના પેટર્નની સ્વ-શોધ.
\end{itemize}
\end{solutionbox}

\begin{mnemonicbox}
\mnemonic{SL-લેબલ્સ, UL-અજાણ્યા}
\end{mnemonicbox}

\questionmarks{1(c)}{7}{મશીન લર્નિંગ એક્ટિવિટીની સૂચિ બનાવો. દરેકને વિગતવાર સમજાવો.}
\begin{solutionbox}
\begin{center}
\captionof{table}{મશીન લર્નિંગ એક્ટિવિટીઓ}
\begin{tabulary}{\linewidth}{L L L}
\hline
\textbf{એક્ટિવિટી} & \textbf{હેતુ} & \textbf{વર્ણન} \\
\hline
\textbf{ડેટા કલેક્શન} & કાચો ડેટા એકત્રિત કરવો & વિવિધ સ્રોતોમાંથી ડેટા લાવવો \\
\textbf{ડેટા પ્રીપ્રોસેસિંગ} & ડેટા સાફ કરવો & નોઈઝ અને મિસિંગ વેલ્યૂઝ દૂર કરવી \\
\textbf{ફીચર સિલેક્શન} & લક્ષણો પસંદ કરવા & મહત્વપૂર્ણ એટ્રિબ્યુટ્સ પસંદ કરવા \\
\textbf{મોડેલ ટ્રેનિંગ} & મોડેલ બનાવવું & અલગોરિધમ ટ્રેનિંગ \\
\textbf{મોડેલ ઇવેલ્યુએશન} & પરફોર્મન્સ માપવું & ચોકસાઈ ચકાસવી \\
\textbf{મોડેલ ડિપ્લોયમેન્ટ} & ઉપયોગમાં લેવું & એપ્લિકેશનમાં મૂકવું \\
\hline
\end{tabulary}
\end{center}

\begin{center}
\begin{tikzpicture}[node distance=1.5cm, auto]
    \node [gtu block] (Data) {ડેટા કલેક્શન};
    \node [gtu block, right=of Data] (Prep) {પ્રીપ્રોસેસિંગ};
    \node [gtu block, right=of Prep] (Feat) {ફીચર સિલેક્શન};
    \node [gtu block, below=1cm of Data] (Train) {મોડેલ ટ્રેનિંગ};
    \node [gtu block, right=of Train] (Eval) {ઇવેલ્યુએશન};
    \node [gtu block, right=of Eval] (Dep) {ડિપ્લોયમેન્ટ};
    \node [gtu block, below=1cm of Train] (Mon) {મોનિટરિંગ};
    
    \path [gtu arrow] (Data) -- (Prep);
    \path [gtu arrow] (Prep) -- (Feat);
    \path [gtu arrow] (Feat) -- (Train);
    \path [gtu arrow] (Train) -- (Eval);
    \path [gtu arrow] (Eval) -- (Dep);
    \path [gtu arrow] (Dep) -- (Mon);
\end{tikzpicture}
\captionof{figure}{મશીન લર્નિંગ ફ્લો}
\end{center}

\begin{itemize}
    \item \textbf{પુનરાવર્તિત પ્રક્રિયા}: મોડેલ સુધારણા માટે સાયકલ ચાલે છે.
    \item \textbf{ગુણવત્તા નિયંત્રણ}: સારા પરિણામો માટે દરેક સ્ટેપ મહત્વનું છે.
\end{itemize}
\end{solutionbox}

\begin{mnemonicbox}
\mnemonic{કપફટઇડમ (CPFTEDM)}
\end{mnemonicbox}

\questionmarks{1(c) OR}{7}{નીચેના ડેટા માટે મીન, મીડિયન અને મોડ શોધો: 1, 1, 1, 2, 4, 5, 5, 6, 6, 7, 7, 7, 7, 8, 9, 10, 11}
\begin{solutionbox}
\textbf{ડેટા}: 1, 1, 1, 2, 4, 5, 5, 6, 6, 7, 7, 7, 7, 8, 9, 10, 11 (કુલ 17)

\begin{center}
\captionof{table}{ડેટા વિશ્લેષણ}
\begin{tabulary}{\linewidth}{L L L L}
\hline
\textbf{માપ} & \textbf{સૂત્ર} & \textbf{ગણતરી} & \textbf{પરિણામ} \\
\hline
\textbf{મીન} & સરવાળો/ગણતરી & \(100/17\) & 5.88 \\
\textbf{મીડિયન} & મધ્ય વેલ્યુ & 9મી વેલ્યુ & 6 \\
\textbf{મોડ} & સૌથી વધુ & 7 (4 વાર) & 7 \\
\hline
\end{tabulary}
\end{center}

\textbf{ગણતરી:}
\begin{itemize}
    \item \textbf{સરવાળો}: 100
    \item \textbf{મીન}: \(100/17 = 5.88\)
    \item \textbf{મીડિયન}: 17 (એકી) સંખ્યા છે, તેથી \((17+1)/2 = 9\)મું પદ. 9મું પદ 6 છે.
    \item \textbf{મોડ}: 7 સૌથી વધુ વખત (4 વખત) આવે છે.
\end{itemize}
\end{solutionbox}

\begin{mnemonicbox}
\mnemonic{મમમ (MMM)}
\end{mnemonicbox}

\questionmarks{2(a)}{3}{મોડેલ ટ્રેનિંગ માટે હોલ્ડ આઉટ પદ્ધતિનો ઉપયોગ કરવાના પગલાં લખો.}
\begin{solutionbox}
\begin{center}
\captionof{table}{હોલ્ડ આઉટ મેથડ}
\begin{tabulary}{\linewidth}{L L L}
\hline
\textbf{પગલું} & \textbf{ક્રિયા} & \textbf{હેતુ} \\
\hline
\textbf{1} & ડેટાસેટ વિભાજન & 70-80\% ટ્રેનિંગ, 20-30\% ટેસ્ટિંગ \\
\textbf{2} & ટ્રેનિંગ & ટ્રેનિંગ સેટ પર મોડેલ શીખવવું \\
\textbf{3} & ટેસ્ટિંગ & ટેસ્ટ સેટ પર પરફોર્મન્સ માપવું \\
\hline
\end{tabulary}
\end{center}

\begin{itemize}
    \item \textbf{રેન્ડમ સ્પ્લિટ}: ડેટાનું પ્રતિનિધિત્વ જળવાય રહે.
    \item \textbf{નો ઓવરલેપ}: ટેસ્ટ ડેટા ટ્રેનિંગમાં વપરાતો નથી.
\end{itemize}
\end{solutionbox}

\begin{mnemonicbox}
\mnemonic{વિભાજન, ટ્રેન, ટેસ્ટ (VTT)}
\end{mnemonicbox}

\questionmarks{2(b)}{4}{કન્ફ્યુઝન મેટ્રિક્સની રચના સમજાવો.}
\begin{solutionbox}
\textbf{કન્ફ્યુઝન મેટ્રિક્સ} ક્લાસિફિકેશન મોડેલના પરફોર્મન્સનું કોષ્ટક છે.

\begin{center}
\captionof{table}{કન્ફ્યુઝન મેટ્રિક્સ}
\begin{tabulary}{\linewidth}{L L L}
\hline
 & \textbf{આગાહી: પોઝિટિવ} & \textbf{આગાહી: નેગેટિવ} \\
\hline
\textbf{વાસ્તવિક: પોઝિટિવ} & True Positive (TP) & False Negative (FN) \\
\textbf{વાસ્તવિક: નેગેટિવ} & False Positive (FP) & True Negative (TN) \\
\hline
\end{tabulary}
\end{center}

\textbf{મેટ્રિક્સ:}
\begin{itemize}
    \item \textbf{Accuracy}: \((TP+TN)/Total\)
    \item \textbf{Precision}: \(TP/(TP+FP)\)
\end{itemize}
\end{solutionbox}

\questionmarks{2(c)}{7}{ડેટા પ્રી-પ્રોસેસિંગ વ્યાખ્યાયિત કરો. ડેટા પ્રી-પ્રોસેસિંગમાં વપરાતી વિવિધ પદ્ધતિઓ સમજાવો.}
\begin{solutionbox}
\textbf{ડેટા પ્રી-પ્રોસેસિંગ} કાચા ડેટાને ઉપયોગી ફોર્મેટમાં ફેરવવાની પ્રક્રિયા છે.

\begin{center}
\captionof{table}{પદ્ધતિઓ}
\begin{tabulary}{\linewidth}{L L L}
\hline
\textbf{પદ્ધતિ} & \textbf{હેતુ} & \textbf{તકનીક} \\
\hline
\textbf{ક્લીનિંગ} & નોઈઝ દૂર કરવી & મિસિંગ વેલ્યૂઝ ભરવી \\
\textbf{ટ્રાન્સફોર્મેશન} & ફોર્મેટ બદલવું & નોર્મલાઈઝેશન \\
\textbf{રિડક્શન} & કદ ઘટાડવું & ફીચર સિલેક્શન \\
\textbf{ઈન્ટીગ્રેશન} & ડેટા જોડવો & મર્જિંગ \\
\hline
\end{tabulary}
\end{center}

\begin{center}
\begin{tikzpicture}[node distance=1.5cm, auto]
    \node [gtu block] (Raw) {કાચો ડેટા};
    \node [gtu block, right=of Raw] (Clean) {ક્લીનિંગ};
    \node [gtu block, right=of Clean] (Trans) {ટ્રાન્સફોર્મેશન};
    \node [gtu block, right=of Trans] (Red) {રિડક્શન};
    \node [gtu block, right=of Red] (Cleaned) {સાફ ડેટા};
    
    \path [gtu arrow] (Raw) -- (Clean);
    \path [gtu arrow] (Clean) -- (Trans);
    \path [gtu arrow] (Trans) -- (Red);
    \path [gtu arrow] (Red) -- (Cleaned);
\end{tikzpicture}
\captionof{figure}{પ્રી-પ્રોસેસિંગ સ્ટેપ્સ}
\end{center}
\end{solutionbox}

\begin{mnemonicbox}
\mnemonic{કતરઈ (CTRI)}
\end{mnemonicbox}

\questionmarks{2(a) OR}{3}{યોગ્ય ઉદાહરણ સાથે હિસ્ટોગ્રામ સમજાવો.}
\begin{solutionbox}
\textbf{હિસ્ટોગ્રામ} ડેટાના વિતરણ (distribution) નો ગ્રાફ છે.

\begin{center}
\captionof{table}{હિસ્ટોગ્રામ ઘટકો}
\begin{tabulary}{\linewidth}{L L}
\hline
\textbf{ઘટક} & \textbf{વર્ણન} \\
\hline
\textbf{X-axis} & Bins (રેન્જ) \\
\textbf{Y-axis} & આવર્તન (Frequency) \\
\textbf{Bars} & ઊંચાઈ આવર્તન દર્શાવે છે \\
\hline
\end{tabulary}
\end{center}

\textbf{ઉદાહરણ}: વિદ્યાર્થીઓના માર્ક્સ. 0-10, 10-20, વગેરે bins માં કેટલા વિદ્યાર્થીઓ છે તે બતાવે છે.
\end{solutionbox}

\begin{mnemonicbox}
\mnemonic{બએર (BAR)}
\end{mnemonicbox}

\questionmarks{2(b) OR}{4}{નીચેના ઉદાહરણોનો યોગ્ય ડેટા પ્રકાર જણાવો: i) વ્યક્તિનું લિંગ ii) વિદ્યાર્થીઓનો ક્રમ iii) ઘરની કિંમત iv) ફૂલનો રંગ}
\begin{solutionbox}
\begin{center}
\captionof{table}{ડેટા પ્રકારો}
\begin{tabulary}{\linewidth}{L L}
\hline
\textbf{ઉદાહરણ} & \textbf{ડેટા પ્રકાર} \\
\hline
\textbf{વ્યક્તિનું લિંગ} & Nominal Categorical \\
\textbf{વિદ્યાર્થીઓનો ક્રમ} & Ordinal Categorical \\
\textbf{ઘરની કિંમત} & Continuous Numerical \\
\textbf{ફૂલનો રંગ} & Nominal Categorical \\
\hline
\end{tabulary}
\end{center}
\end{solutionbox}

\begin{mnemonicbox}
\mnemonic{નોકો (NOCO)}
\end{mnemonicbox}

\questionmarks{2(c) OR}{7}{K-fold ક્રોસ વેલિડેશનનું વિગતવાર વર્ણન કરો.}
\begin{solutionbox}
\textbf{K-fold ક્રોસ વેલિડેશન} ડેટાને K ભાગોમાં વહેંચીને મોડેલનું મૂલ્યાંકન કરે છે.

\begin{center}
\captionof{table}{પ્રક્રિયા}
\begin{tabulary}{\linewidth}{L L}
\hline
\textbf{પગલું} & \textbf{વર્ણન} \\
\hline
\textbf{1} & ડેટાને K સમાન ભાગો (folds) માં વહેંચો. \\
\textbf{2} & K-1 ભાગો ટ્રેનિંગ માટે અને 1 ભાગ ટેસ્ટિંગ માટે વાપરો. \\
\textbf{3} & આ પ્રક્રિયા K વાર પુનરાવર્તિત કરો. \\
\textbf{4} & બધા પરિણામોની સરેરાશ લો. \\
\hline
\end{tabulary}
\end{center}

\begin{center}
\begin{tikzpicture}[node distance=1.5cm, auto]
    \node [gtu block] (Data) {ડેટાસેટ};
    \node [gtu block, right=of Data] (Folds) {K Folds};
    \node [gtu block, below=1cm of Data] (Train) {ટ્રેન K-1};
    \node [gtu block, right=of Train] (Test) {ટેસ્ટ 1};
    \node [gtu block, right=of Test] (Avg) {સરેરાશ};
    
    \path [gtu arrow] (Data) -- (Folds);
    \path [gtu arrow] (Folds) -- (Train);
    \path [gtu arrow] (Train) -- (Test);
    \path [gtu arrow] (Test) -- node[above]{K વાર} (Avg);
\end{tikzpicture}
\captionof{figure}{K-Fold Cross Validation}
\end{center}

\textbf{ફાયદા}: ઓવરફિટિંગ ઘટાડે છે અને દરેક ડેટા પોઇન્ટનો ઉપયોગ કરે છે.
\end{solutionbox}

\begin{mnemonicbox}
\mnemonic{વઉપસટ (DURAT)}
\end{mnemonicbox}

\questionmarks{3(a)}{3}{રીગ્રેશનની એપ્લિકેશનની યાદી બનાવો.}
\begin{solutionbox}
\begin{center}
\captionof{table}{રીગ્રેશન એપ્લિકેશન}
\begin{tabulary}{\linewidth}{L L L}
\hline
\textbf{ક્ષેત્ર} & \textbf{એપ્લિકેશન} & \textbf{હેતુ} \\
\hline
\textbf{ફાઇનાન્સ} & શેર કિંમત & વલણો જાણવા \\
\textbf{હેલ્થકેર} & દવાની માત્રા & સારવાર નક્કી કરવા \\
\textbf{રિયલ એસ્ટેટ} & ઘરની કિંમત & વેલ્યુએશન \\
\hline
\end{tabulary}
\end{center}
\end{solutionbox}

\begin{mnemonicbox}
\mnemonic{નહમર (FHMR)}
\end{mnemonicbox}

\questionmarks{3(b)}{4}{સિંગલ લિનિયર રીગ્રેશન પર ટૂંકી નોંધ લખો.}
\begin{solutionbox}
\textbf{સિંગલ લિનિયર રીગ્રેશન} એક input variable (X) અને output variable (Y) વચ્ચે રેખીય સંબંધ શોધે છે.

\begin{itemize}
    \item \textbf{સમીકરણ}: \(Y = a + bX\)
    \item \textbf{ધ્યેય}: બેસ્ટ ફિટ લાઇન બનાવવી જે એરર ઘટાડે.
    \item \textbf{સ્લોપ (b)}: X ના ફેરફાર સાથે Y નો ફેરફાર.
\end{itemize}
\end{solutionbox}

\begin{mnemonicbox}
\mnemonic{YABX}
\end{mnemonicbox}

\questionmarks{3(c)}{7}{K-NN અલગોરિધમ લખો અને ચર્ચા કરો.}
\begin{solutionbox}
\textbf{K-Nearest Neighbors (K-NN)} નવા ડેટા પોઇન્ટને તેના K નજીકના પડોશીઓના વર્ગ પ્રમાણે વર્ગીકૃત કરે છે.

\begin{center}
\captionof{table}{K-NN પગલાં}
\begin{tabulary}{\linewidth}{L L}
\hline
\textbf{પગલું} & \textbf{ક્રિયા} \\
\hline
\textbf{1} & K ની કિંમત નક્કી કરો. \\
\textbf{2} & બધા પોઇન્ટ્સથી અંતર ગણો (Euclidean Distance). \\
\textbf{3} & અંતરને ચડતા ક્રમમાં ગોઠવો. \\
\textbf{4} & નજીકના K પોઇન્ટ્સ પસંદ કરો. \\
\textbf{5} & બહુમતી (Majority) ક્લાસ અસાઇન કરો. \\
\hline
\end{tabulary}
\end{center}

\begin{center}
\begin{tikzpicture}[node distance=1.5cm, auto]
    \node [gtu block] (New) {નવો ડેટા};
    \node [gtu block, right=of New] (Calc) {અંતર ગણતરી};
    \node [gtu block, right=of Calc] (Sort) {સોર્ટિંગ};
    \node [gtu block, below=1cm of Calc] (Sel) {K પસંદગી};
    \node [gtu block, right=of Sel] (Vote) {વોટિંગ};
    \node [gtu block, right=of Vote] (Class) {વર્ગીકરણ};
    
    \path [gtu arrow] (New) -- (Calc);
    \path [gtu arrow] (Calc) -- (Sort);
    \path [gtu arrow] (Sort) -- (Sel);
    \path [gtu arrow] (Sel) -- (Vote);
    \path [gtu arrow] (Vote) -- (Class);
\end{tikzpicture}
\captionof{figure}{K-NN પ્રક્રિયા}
\end{center}
\end{solutionbox}

\begin{mnemonicbox}
\mnemonic{પગકમ (CCSM)}
\end{mnemonicbox}

\questionmarks{3(a) OR}{3}{હેલ્થકેર ક્ષેત્રમાં supervised learning ના કોઈપણ ત્રણ ઉદાહરણો લખો}
\begin{solutionbox}
\begin{itemize}
    \item \textbf{રોગ નિદાન}: લક્ષણો પરથી રોગ ઓળખવો.
    \item \textbf{દવાની અસર}: દર્દી પર દવાની અસરકારકતા મોડેલ કરવી.
    \item \textbf{મેડિકલ ઇમેજિંગ}: X-ray માંથી ટ્યુમર શોધવું.
\end{itemize}
\end{solutionbox}

\begin{mnemonicbox}
\mnemonic{રદમ (DDM)}
\end{mnemonicbox}

\questionmarks{3(b) OR}{4}{તફાવત કરો: Classification v/s Regression.}
\begin{solutionbox}
\begin{center}
\captionof{table}{Classification vs Regression}
\begin{tabulary}{\linewidth}{L L L}
\hline
\textbf{પાસું} & \textbf{Classification} & \textbf{Regression} \\
\hline
\textbf{આઉટપુટ} & શ્રેણીઓ (હા/ના) & સંખ્યાત્મક (કિંમત, તાપમાન) \\
\textbf{ઉદાહરણ} & સ્પામ ડિટેક્શન & ભાવ આગાહી \\
\textbf{મેટ્રિક્સ} & Accuracy & MSE, R2 \\
\hline
\end{tabulary}
\end{center}
\end{solutionbox}

\begin{mnemonicbox}
\mnemonic{CLASS-શ્રેણી, REG-સંખ્યા}
\end{mnemonicbox}

\questionmarks{3(c) OR}{7}{ક્લાસિફિકેશન લર્નિંગના સ્ટેપ્સને વિગતમાં સમજાવો.}
\begin{solutionbox}
\textbf{ક્લાસિફિકેશન} ડેટાને વર્ગોમાં વહેંચવાની પ્રક્રિયા છે.

\begin{center}
\captionof{table}{સ્ટેપ્સ}
\begin{tabulary}{\linewidth}{L L}
\hline
\textbf{પગલું} & \textbf{વર્ણન} \\
\hline
\textbf{1. કલેક્શન} & લેબલ ડેટા ભેગો કરવો. \\
\textbf{2. પ્રીપ્રોસેસિંગ} & ડેટા સાફ અને તૈયાર કરવો. \\
\textbf{3. ફીચર્સ} & મહત્વના ફીચર્સ પસંદ કરવા. \\
\textbf{4. ટ્રેનિંગ} & અલગોરિધમ મોડેલ બનાવવું. \\
\textbf{5. મૂલ્યાંકન} & ટેસ્ટિંગ કરવું. \\
\hline
\end{tabulary}
\end{center}

\begin{center}
\begin{tikzpicture}[node distance=1.5cm, auto]
    \node [gtu block] (Data) {ડેટા};
    \node [gtu block, right=of Data] (Prep) {પ્રીપ્રોસેસ};
    \node [gtu block, right=of Prep] (Mod) {મોડેલ};
    \node [gtu block, below=1cm of Data] (Train) {ટ્રેનિંગ};
    \node [gtu block, right=of Train] (Eval) {મૂલ્યાંકન};
    \node [gtu block, right=of Eval] (Dep) {ડિપ્લોય};
    
    \path [gtu arrow] (Data) -- (Prep);
    \path [gtu arrow] (Prep) -- (Mod);
    \path [gtu arrow] (Mod) -- (Train);
    \path [gtu arrow] (Train) -- (Eval);
    \path [gtu arrow] (Eval) -- (Dep);
\end{tikzpicture}
\captionof{figure}{ક્લાસિફિકેશન ફ્લો}
\end{center}
\end{solutionbox}

\begin{mnemonicbox}
\mnemonic{ડપફમટમડ (DCFMTED)}
\end{mnemonicbox}

\questionmarks{4(a)}{3}{તફાવત કરો: Clustering v/s Classification.}
\begin{solutionbox}
\begin{center}
\captionof{table}{Clustering vs Classification}
\begin{tabulary}{\linewidth}{L L L}
\hline
\textbf{પાસું} & \textbf{Clustering} & \textbf{Classification} \\
\hline
\textbf{પ્રકાર} & Unsupervised & Supervised \\
\textbf{ડેટા} & લેબલ વગરનો & લેબલવાળો \\
\textbf{હેતુ} & જૂથો શોધવા & વર્ગ આગાહી કરવી \\
\hline
\end{tabulary}
\end{center}
\end{solutionbox}

\begin{mnemonicbox}
\mnemonic{CL-અજાણ્યા, CLASS-જાણીતા}
\end{mnemonicbox}

\questionmarks{4(b)}{4}{Apriori અલગોરિધમના ફાયદા અને ગેરફાયદાની યાદી બનાવો.}
\begin{solutionbox}
\begin{center}
\captionof{table}{Apriori ફાયદા/ગેરફાયદા}
\begin{tabulary}{\linewidth}{L L}
\hline
\textbf{ફાયદા} & \textbf{ગેરફાયદા} \\
\hline
સરળ છે & ધીમું છે (Slow) \\
બધા પેટર્ન શોધે છે & મેમરી વધારે વાપરે છે \\
રૂલ્સ બનાવે છે & વારંવાર ડેટાબેઝ સ્કેન કરે છે \\
\hline
\end{tabulary}
\end{center}
\end{solutionbox}

\begin{mnemonicbox}
\mnemonic{સરળ-ધીમું}
\end{mnemonicbox}

\questionmarks{4(c)}{7}{unsupervised લર્નિંગની એપ્લિકેશનો લખો અને સમજાવો}
\begin{solutionbox}
\textbf{Unsupervised Learning} ડેટામાંથી પેટર્ન શોધે છે.

\begin{center}
\captionof{table}{એપ્લિકેશન્સ}
\begin{tabulary}{\linewidth}{L L L}
\hline
\textbf{ક્ષેત્ર} & \textbf{એપ્લિકેશન} & \textbf{તકનીક} \\
\hline
\textbf{માર્કેટિંગ} & ગ્રાહક સેગમેન્ટેશન & Clustering \\
\textbf{રિટેઇલ} & બાસ્કેટ એનાલિસિસ & Association Rules \\
\textbf{સુરક્ષા} & ફ્રોડ ડિટેક્શન & Anomaly Detection \\
\hline
\end{tabulary}
\end{center}

\begin{center}
\begin{tikzpicture}[node distance=1.5cm, auto]
    \node [gtu block] (UL) {Unsupervised};
    \node [gtu state, below left=1.5cm and 0.5cm of UL] (Clust) {Clustering\\(સેગમેન્ટેશન)};
    \node [gtu state, below right=1.5cm and 0.5cm of UL] (Assoc) {Association\\(રીટેઇલ)};
    \node [gtu state, below=1.5cm of UL] (Anom) {Anomaly\\(ફ્રોડ)};
    
    \path [gtu arrow] (UL) -- (Clust);
    \path [gtu arrow] (UL) -- (Assoc);
    \path [gtu arrow] (UL) -- (Anom);
\end{tikzpicture}
\captionof{figure}{એપ્લિકેશન્સ}
\end{center}
\end{solutionbox}

\begin{mnemonicbox}
\mnemonic{મરએડ (MRAD)}
\end{mnemonicbox}

\questionmarks{4(a) OR}{3}{Apriori અલગોરિધમની એપ્લિકેશનની યાદી બનાવો.}
\begin{solutionbox}
\begin{itemize}
    \item \textbf{માર્કેટ બાસ્કેટ એનાલિસિસ}: ગ્રાહક ખરીદી પેટર્ન.
    \item \textbf{વેબ માઇનિંગ}: પેજ વિઝિટ સિક્વન્સ.
    \item \textbf{બાયોઇન્ફોર્મેટિક્સ}: DNA સિક્વન્સ એનાલિસિસ.
\end{itemize}
\end{solutionbox}

\begin{mnemonicbox}
\mnemonic{રવબ (RWB)}
\end{mnemonicbox}

\questionmarks{4(b) OR}{4}{વ્યાખ્યાયિત કરો: Support and Confidence.}
\begin{solutionbox}
\begin{center}
\captionof{table}{મેટ્રિક્સ}
\begin{tabulary}{\linewidth}{L L}
\hline
\textbf{મેટ્રિક} & \textbf{વ્યાખ્યા} \\
\hline
\textbf{Support} & આઇટમસેટ કેટલી વાર દેખાય છે? (\(Count/Total\)) \\
\textbf{Confidence} & નિયમ (Rule) કેટલો સાચો છે? (\(A \to B\)) \\
\hline
\end{tabulary}
\end{center}
\end{solutionbox}

\questionmarks{4(c) OR}{7}{K-means ક્લસ્ટરિંગ અપ્રોચ વિગતવાર લખો અને સમજાવો.}
\begin{solutionbox}
\textbf{K-means} ડેટાને K ક્લસ્ટરમાં વહેંચે છે.

\begin{center}
\captionof{table}{સ્ટેપ્સ}
\begin{tabulary}{\linewidth}{L L}
\hline
\textbf{પગલું} & \textbf{ક્રિયા} \\
\hline
\textbf{1} & K પસંદ કરો. \\
\textbf{2} & K સેન્ટ્રોઇડ્સ (કેન્દ્રો) રેન્ડમલી મૂકો. \\
\textbf{3} & દરેક ડેટા પોઇન્ટને નજીકના સેન્ટ્રોઇડ સાથે જોડો. \\
\textbf{4} & નવા સેન્ટ્રોઇડ્સ ગણો (સરેરાશ). \\
\textbf{5} & ફેરફાર બંધ થાય ત્યાં સુધી 3-4 કરો. \\
\hline
\end{tabulary}
\end{center}

\begin{center}
\begin{tikzpicture}[node distance=1.5cm, auto]
    \node [gtu block] (K) {K પસંદ કરો};
    \node [gtu block, right=of K] (Init) {ઇનિશિયલાઇઝ};
    \node [gtu block, right=of Init] (Assign) {અસાઇન};
    \node [gtu block, below=1cm of Assign] (Update) {અપડેટ};
    
    \path [gtu arrow] (K) -- (Init);
    \path [gtu arrow] (Init) -- (Assign);
    \path [gtu arrow] (Assign) -- (Update);
    \path [gtu arrow] (Update.west) -- ++(-1,0) |- (Assign.west);
\end{tikzpicture}
\captionof{figure}{K-means પ્રક્રિયા}
\end{center}
\end{solutionbox}

\begin{mnemonicbox}
\mnemonic{પસઅપ (CIAUR)}
\end{mnemonicbox}

\questionmarks{5(a)}{3}{પ્રિડિક્ટિવ મોડેલ અને ડિસ્ક્રિપ્ટિવ મોડેલ વચ્ચેનો તફાવત આપો.}
\begin{solutionbox}
\begin{center}
\captionof{table}{Predictive vs Descriptive}
\begin{tabulary}{\linewidth}{L L L}
\hline
\textbf{પાસું} & \textbf{Predictive} & \textbf{Descriptive} \\
\hline
\textbf{હેતુ} & ભવિષ્યની આગાહી & વર્તમાન સમજવું \\
\textbf{ઉદાહરણ} & વેચાણ આગાહી & સેગમેન્ટેશન \\
\hline
\end{tabulary}
\end{center}
\end{solutionbox}

\questionmarks{5(b)}{4}{scikit-learn ની એપ્લિકેશનની સૂચિ બનાવો.}
\begin{solutionbox}
\begin{center}
\captionof{table}{Scikit-learn}
\begin{tabulary}{\linewidth}{L L}
\hline
\textbf{પ્રકાર} & \textbf{એપ્લિકેશન} \\
\hline
\textbf{Classification} & સ્પામ ફિલ્ટર \\
\textbf{Regression} & કિંમત આગાહી \\
\textbf{Clustering} & ગ્રાહક જૂથ \\
\textbf{Preprocessing} & ડેટા ક્લીનિંગ \\
\hline
\end{tabulary}
\end{center}
\end{solutionbox}

\questionmarks{5(c)}{7}{Numpy ના લક્ષણો અને એપ્લિકેશનો સમજાવો.}
\begin{solutionbox}
\textbf{NumPy} ગણતરી માટેની Python લાઈબ્રેરી છે.

\begin{center}
\captionof{table}{લક્ષણો}
\begin{tabulary}{\linewidth}{L L}
\hline
\textbf{લક્ષણ} & \textbf{ફાયદો} \\
\hline
\textbf{N-dim Arrays} & ડેટા સ્ટોરેજ \\
\textbf{Math Functions} & ગણતરી \\
\textbf{Performance} & ઝડપી (Fast) \\
\hline
\end{tabulary}
\end{center}

\begin{center}
\begin{tikzpicture}[node distance=1.5cm, auto]
    \node [gtu block] (Num) {NumPy};
    \node [gtu block, below left=1.5cm and 1cm of Num] (Feat) {લક્ષણો};
    \node [gtu block, below right=1.5cm and 1cm of Num] (App) {એપ્લિકેશન};
    \node [gtu state, below=0.5cm of Feat] {Arrays\\Maths};
    \node [gtu state, below=0.5cm of App] {ML, Science};
    
    \path [gtu arrow] (Num) -- (Feat);
    \path [gtu arrow] (Num) -- (App);
\end{tikzpicture}
\captionof{figure}{NumPy}
\end{center}
\end{solutionbox}

\begin{mnemonicbox}
\mnemonic{NઝEગવ (NFAMS)}
\end{mnemonicbox}

\questionmarks{5(a) OR}{3}{બેગિંગ પર ટૂંકી નોંધ લખો}
\begin{solutionbox}
\textbf{Bagging} એક સાથે અનેક મોડેલ ટ્રેન કરે છે.
\begin{itemize}
    \item \textbf{Bootstrap}: ડેટાના અનેક સેમ્પલ બનાવે છે.
    \item \textbf{Aggregation}: બધા મોડેલના પરિણામની સરેરાશ લે છે.
    \item \textbf{ઉદાહરણ}: Random Forest.
\end{itemize}
\end{solutionbox}

\begin{mnemonicbox}
\mnemonic{બટએ (BTA)}
\end{mnemonicbox}

\questionmarks{5(b) OR}{4}{Pandas લક્ષણોની યાદી આપો.}
\begin{solutionbox}
\begin{center}
\captionof{table}{Pandas લક્ષણો}
\begin{tabulary}{\linewidth}{L L}
\hline
\textbf{લક્ષણ} & \textbf{ઉપયોગ} \\
\hline
\textbf{DataFrame} & ડેટા ટેબલ \\
\textbf{File I/O} & Excel/CSV વાંચવું \\
\textbf{Cleaning} & ડેટા સાફ કરવો \\
\textbf{Grouping} & એનાલિસિસ \\
\hline
\end{tabulary}
\end{center}
\end{solutionbox}

\begin{mnemonicbox}
\mnemonic{ડફઇગ (DFIG)}
\end{mnemonicbox}

\questionmarks{5(c) OR}{7}{Matplotlib ની વિશેષતાઓ અને એપ્લિકેશનો સમજાવો.}
\begin{solutionbox}
\textbf{Matplotlib} ગ્રાફ અને ચાર્ટ બનાવતી લાઈબ્રેરી છે.

\begin{center}
\captionof{table}{વિશેષતાઓ}
\begin{tabulary}{\linewidth}{L L}
\hline
\textbf{વિશેષતા} & \textbf{વર્ણન} \\
\hline
\textbf{Plot Types} & Line, Bar, Scatter \\
\textbf{Customization} & કલર, સ્ટાઇલ \\
\textbf{Output} & PNG, PDF \\
\hline
\end{tabulary}
\end{center}

\begin{center}
\begin{tikzpicture}[node distance=1.5cm, auto]
    \node [gtu block] (Data) {ડેટા};
    \node [gtu block, right=of Data] (Code) {Matplotlib};
    \node [gtu block, right=of Code] (Plot) {પ્લોટ/ગ્રાફ};
    \node [gtu block, below=1cm of Code] (Out) {ફાઇલ/વેબ};
    
    \path [gtu arrow] (Data) -- (Code);
    \path [gtu arrow] (Code) -- (Plot);
    \path [gtu arrow] (Code) -- (Out);
\end{tikzpicture}
\captionof{figure}{વિઝ્યુલાઇઝેશન}
\end{center}
\end{solutionbox}

\end{document}
