\documentclass[10pt,a4paper]{article}

% content/resources/templates/preamble.tex
\usepackage[margin=0.6in]{geometry}
\author{Milav Dabgar}
\usepackage{amsmath,amssymb,amsthm}
\usepackage{booktabs}
\usepackage{multirow}
\usepackage{xcolor}
\usepackage{tcolorbox}
\tcbuselibrary{breakable,skins}
\usepackage[colorlinks=true,linkcolor=blue]{hyperref}
\usepackage{titlesec}
\usepackage{enumitem}
\usepackage{tikz}
\usepackage{pgfplots}
\usepackage{circuitikz}
\usepackage[version=4]{mhchem}
\usepackage{longtable}
\usepackage{array}
\usepackage{float}
\usepackage{caption}
\usepackage{listings}

\lstset{
  basicstyle=\small\ttfamily,
  breaklines=true,
  breakatwhitespace=false,
  postbreak=\mbox{\textcolor{red}{$\hookrightarrow$}\space},
  float=false,
  numbers=left,
  numberstyle=\tiny\color{gray},
  numbersep=10pt,
  xleftmargin=2em,
  keywordstyle=\color{blue},
  commentstyle=\color{green!60!black},
  stringstyle=\color{purple},
  backgroundcolor=\color{gray!5},
  showstringspaces=false,
  tabsize=2,
  captionpos=b,
  keepspaces=true,
  columns=flexible
}

\pgfplotsset{compat=1.18}
\usetikzlibrary{shapes,arrows,positioning,calc,patterns,decorations.pathmorphing,decorations.markings,arrows.meta}

% Color scheme
\definecolor{headcolor}{RGB}{0,102,204}
\definecolor{keycolor}{RGB}{220,20,60}
\definecolor{solutioncolor}{RGB}{34,139,34}
\definecolor{mnemoniccolor}{RGB}{148,0,211}
\definecolor{codecolor}{RGB}{0,0,100}

% Spacing
\setlength{\parskip}{3pt}
\setlist[itemize]{nosep}
\setlist[enumerate]{nosep}

% Title formatting
\titleformat{\section}{\Large\bfseries\color{headcolor}}{\thesection}{1em}{}
\titleformat{\subsection}{\large\bfseries\color{headcolor}}{\thesubsection}{1em}{}

% Pandoc tightlist compatibility
\providecommand{\tightlist}{%
  \setlength{\itemsep}{0pt}\setlength{\parskip}{0pt}}

% Pandoc longtable compatibility
\newcounter{none}
\def\thenone{}


% content/resources/templates/gujarati-boxes.tex
\usepackage{fontspec}
\usepackage{polyglossia}

% Set Gujarati as main language (document is primarily in Gujarati)
% Note: gloss-gujarati.ldf doesn't exist in polyglossia, but it will use hyphenation patterns
\setdefaultlanguage{gujarati}
\setotherlanguage{english}

% Configure Gujarati font properly
% Use Language=Default to prevent polyglossia from trying to add language-specific features
% that don't exist for Gujarati, which causes "empty feature" warnings
\newfontfamily\gujaratifont[Script=Gujarati,AutoFakeBold=2.5,AutoFakeSlant=0.3]{Noto Sans Gujarati}
\setmainfont[Script=Gujarati,AutoFakeBold=2.5,AutoFakeSlant=0.3]{Noto Sans Gujarati}
% Use Noto Sans Gujarati for monospace to support Gujarati in text
\setmonofont[Scale=0.9]{Noto Sans Gujarati}

% Configure English to use the same font
\newfontfamily\englishfont[Script=Gujarati,AutoFakeBold=2.5,AutoFakeSlant=0.3]{Noto Sans Gujarati}

% Translations for polyglossia
\gappto\captionsgujarati{
  \renewcommand{\tablename}{કોષ્ટક}
  \renewcommand{\figurename}{આકૃતિ}
}

% Helper for TikZ nodes to ensure Gujarati font
\newcommand{\gu}[1]{{\gujaratifont #1}}

% Custom environments
\newtcolorbox{solutionbox}{
    breakable,
    enhanced,
    colback=solutioncolor!5!white,
    colframe=solutioncolor!75!black,
    fonttitle=\bfseries,
    title=જવાબ
}

\newtcolorbox{solutionboxnobreak}{
 colback=solutioncolor!5!white,
 colframe=solutioncolor!75!black,
 fonttitle=\bfseries,
 title=જવાબ
}

\newtcolorbox{keyformula}{
 breakable,
 enhanced,
 colback=keycolor!5!white,
 colframe=keycolor!75!black,
 fonttitle=\bfseries,
 title=રાસાયણિક સમીકરણ/સૂત્ર
}

\newtcolorbox{mnemonicbox}{
 breakable,
 enhanced,
 colback=mnemoniccolor!5!white,
 colframe=mnemoniccolor!75!black,
 fonttitle=\bfseries,
 title=મેમરી ટ્રીક
}


\begin{document}

\begin{center}
{\Huge\bfseries\color{headcolor} Subject Name (Gujarati)}\\[5pt]
{\LARGE 4341603 -- Winter 2023}\\[3pt]
{\large Semester 1 Study Material}\\[3pt]
{\normalsize\textit{Detailed Solutions and Explanations}}
\end{center}

\vspace{10pt}

\subsection*{પ્રશ્ન 1(અ) [3
ગુણ]}\label{uxaaauxab0uxab6uxaa8-1uxa85-3-uxa97uxaa3}

\textbf{Human learning વ્યાખ્યાયિત કરો અને સમજાવો કે machine learning human
learning થી કેવી રીતે અલગ છે?}

\begin{solutionbox}

\textbf{Human Learning વિ Machine Learning કોષ્ટક:}

{\def\LTcaptype{none} % do not increment counter
\begin{longtable}[]{@{}lll@{}}
\toprule\noalign{}
પાસાં & Human Learning & Machine Learning \\
\midrule\noalign{}
\endhead
\bottomrule\noalign{}
\endlastfoot
\textbf{પદ્ધતિ} & અનુભવ, પ્રયાસ અને ભૂલ & ડેટા અને અલ્ગોરિધમ \\
\textbf{ઝડપ} & ધીમી, ક્રમશઃ & ઝડપી પ્રોસેસિંગ \\
\textbf{ડેટા જરૂરિયાત} & મર્યાદિત ઉદાહરણો જોઈએ & મોટા ડેટાસેટ જરૂરી \\
\end{longtable}
}

\begin{itemize}
\tightlist
\item
  \textbf{Human Learning}: અનુભવ, અવલોકન અને તર્ક દ્વારા જ્ઞાન મેળવવાની
  પ્રક્રિયા
\item
  \textbf{Machine Learning}: ડેટામાં પેટર્ન ઓળખવા માટે અલ્ગોરિધમનો ઉપયોગ કરીને
  આપોઆપ શીખવાની પ્રક્રિયા
\end{itemize}

\end{solutionbox}
\begin{mnemonicbox}
``Humans Experience, Machines Analyze Data'' (HEMAD)

\end{mnemonicbox}
\begin{center}\rule{0.5\linewidth}{0.5pt}\end{center}

\subsection*{પ્રશ્ન 1(બ) [4
ગુણ]}\label{uxaaauxab0uxab6uxaa8-1uxaac-4-uxa97uxaa3}

\textbf{ફાઇનાન્સ અને બેંકિંગમાં મશીન લર્નિંગના ઉપયોગનું વર્ણન કરો.}

\begin{solutionbox}

\textbf{ફાઇનાન્સ અને બેંકિંગમાં ઉપયોગો:}

{\def\LTcaptype{none} % do not increment counter
\begin{longtable}[]{@{}lll@{}}
\toprule\noalign{}
ઉપયોગ & હેતુ & ફાયદો \\
\midrule\noalign{}
\endhead
\bottomrule\noalign{}
\endlastfoot
\textbf{Fraud Detection} & શંકાસ્પદ ટ્રાન્ઝેક્શન ઓળખવા & નાણાકીય નુકસાન
ઘટાડવું \\
\textbf{Credit Scoring} & લોન ડિફોલ્ટ રિસ્ક આંકવી & વધુ સારા લેન્ડિંગ નિર્ણયો \\
\textbf{Algorithmic Trading} & આપોઆપ ટ્રેડિંગ નિર્ણયો & ઝડપી માર્કેટ
રિસ્પોન્સ \\
\end{longtable}
}

\begin{itemize}
\tightlist
\item
  \textbf{Risk Assessment}: ગ્રાહકની ક્રેડિટવર્થીનેસની આગાહી માટે ML ડેટાનું
  વિશ્લેષણ કરે છે
\item
  \textbf{Customer Service}: NLP નો ઉપયોગ કરીને ચેટબોટ્સ 24/7 સપોર્ટ આપે છે
\item
  \textbf{Regulatory Compliance}: શંકાસ્પદ પ્રવૃત્તિઓ માટે આપોઆપ મોનિટરિંગ
\end{itemize}

\end{solutionbox}
\begin{mnemonicbox}
``Finance Needs Smart Analysis'' (FNSA)

\end{mnemonicbox}
\begin{center}\rule{0.5\linewidth}{0.5pt}\end{center}

\subsection*{પ્રશ્ન 1(ક) [7
ગુણ]}\label{uxaaauxab0uxab6uxaa8-1uxa95-7-uxa97uxaa3}

\textbf{સુપરવાઇઝ્ડ લર્નિંગ, અનસુપરવાઇઝ્ડ લર્નિંગ અને રિઇન્ફોર્સમેન્ટ લર્નિંગ વચ્ચે તફાવત
આપો.}

\begin{solutionbox}

\textbf{તુલનાત્મક કોષ્ટક:}

{\def\LTcaptype{none} % do not increment counter
\begin{longtable}[]{@{}
  >{\raggedright\arraybackslash}p{(\linewidth - 6\tabcolsep) * \real{0.1286}}
  >{\raggedright\arraybackslash}p{(\linewidth - 6\tabcolsep) * \real{0.2714}}
  >{\raggedright\arraybackslash}p{(\linewidth - 6\tabcolsep) * \real{0.3000}}
  >{\raggedright\arraybackslash}p{(\linewidth - 6\tabcolsep) * \real{0.3000}}@{}}
\toprule\noalign{}
\begin{minipage}[b]{\linewidth}\raggedright
લક્ષણ
\end{minipage} & \begin{minipage}[b]{\linewidth}\raggedright
Supervised Learning
\end{minipage} & \begin{minipage}[b]{\linewidth}\raggedright
Unsupervised Learning
\end{minipage} & \begin{minipage}[b]{\linewidth}\raggedright
Reinforcement Learning
\end{minipage} \\
\midrule\noalign{}
\endhead
\bottomrule\noalign{}
\endlastfoot
\textbf{ડેટા પ્રકાર} & લેબલ્ડ ડેટા & અનલેબલ્ડ ડેટા & પર્યાવરણ ઇન્ટરેક્શન \\
\textbf{લક્ષ્ય} & આઉટપુટની આગાહી & પેટર્નો શોધવા & રિવોર્ડ વધારવા \\
\textbf{ઉદાહરણો} & Classification, Regression & Clustering, Association
& Game playing, Robotics \\
\textbf{Feedback} & તાત્કાલિક & કંઈ નહીં & વિલંબિત પુરસ્કારો \\
\end{longtable}
}

\textbf{મુખ્ય લાક્ષણિકતાઓ:}

\begin{itemize}
\tightlist
\item
  \textbf{Supervised Learning}: સાચા જવાબો સાથે શિક્ષક દ્વારા માર્ગદર્શિત
  શીખવું
\item
  \textbf{Unsupervised Learning}: ડેટામાં છુપાયેલા પેટર્નોની સ્વ-શોધ
\item
  \textbf{Reinforcement Learning}: પુરસ્કાર/દંડ સાથે ટ્રાયલ અને એરર દ્વારા શીખવું
\end{itemize}

\end{solutionbox}
\begin{mnemonicbox}
``Supervised Teachers, Unsupervised Explores,
Reinforcement Rewards'' (STUER)

\end{mnemonicbox}
\begin{center}\rule{0.5\linewidth}{0.5pt}\end{center}

\subsection*{પ્રશ્ન 1(ક OR) [7
ગુણ]}\label{uxaaauxab0uxab6uxaa8-1uxa95-or-7-uxa97uxaa3}

\textbf{મશીન લર્નિંગમાં વપરાતા વિવિધ TOOLS અને ટેકનોલોજી સમજાવો.}

\begin{solutionbox}

\textbf{ML Tools અને Technologies:}

{\def\LTcaptype{none} % do not increment counter
\begin{longtable}[]{@{}lll@{}}
\toprule\noalign{}
કેટેગરી & Tools & હેતુ \\
\midrule\noalign{}
\endhead
\bottomrule\noalign{}
\endlastfoot
\textbf{Programming} & Python, R, Java & અલ્ગોરિધમ ઇમ્પ્લિમેન્ટેશન \\
\textbf{Libraries} & Scikit-learn, TensorFlow & તૈયાર અલ્ગોરિધમ \\
\textbf{Visualization} & Matplotlib, Seaborn & ડેટા વિઝ્યુઅલાઇઝેશન \\
\textbf{Data Processing} & Pandas, NumPy & ડેટા મેનિપ્યુલેશન \\
\end{longtable}
}

\textbf{મુખ્ય ટેકનોલોજીઓ:}

\begin{itemize}
\tightlist
\item
  \textbf{Cloud Platforms}: AWS, Google Cloud સ્કેલેબલ કમ્પ્યુટિંગ માટે
\item
  \textbf{Development Environments}: Jupyter Notebook, Google Colab
\item
  \textbf{Big Data Tools}: મોટા ડેટાસેટ માટે Spark, Hadoop
\end{itemize}

\end{solutionbox}
\begin{mnemonicbox}
``Python Libraries Visualize Data Effectively''
(PLVDE)

\end{mnemonicbox}
\begin{center}\rule{0.5\linewidth}{0.5pt}\end{center}

\subsection*{પ્રશ્ન 2(અ) [3
ગુણ]}\label{uxaaauxab0uxab6uxaa8-2uxa85-3-uxa97uxaa3}

\textbf{એક ઉદાહરણ સાથે outliers ને વ્યાખ્યાયિત કરો.}

\begin{solutionbox}

\textbf{વ્યાખ્યા}: Outliers એવા ડેટા પોઇન્ટ્સ છે જે ડેટાસેટમાં અન્ય અવલોકનોથી
નોંધપાત્ર રીતે અલગ હોય છે.

\textbf{ઉદાહરણ કોષ્ટક:}

{\def\LTcaptype{none} % do not increment counter
\begin{longtable}[]{@{}ll@{}}
\toprule\noalign{}
વિદ્યાર્થીઓની ઊંચાઈ (cm) & વર્ગીકરણ \\
\midrule\noalign{}
\endhead
\bottomrule\noalign{}
\endlastfoot
165, 170, 168, 172 & સામાન્ય મૂલ્યો \\
195 & Outlier (ખૂબ ઊંચું) \\
140 & Outlier (ખૂબ નીચું) \\
\end{longtable}
}

\begin{itemize}
\tightlist
\item
  \textbf{શોધ}: Quartiles થી 1.5 \times IQR થી વધુ મૂલ્યો
\item
  \textbf{અસર}: આંકડાકીય વિશ્લેષણ અને મોડલ પર્ફોર્મન્સને અસર કરી શકે
\end{itemize}

\end{solutionbox}
\begin{mnemonicbox}
``Outliers Stand Apart'' (OSA)

\end{mnemonicbox}
\begin{center}\rule{0.5\linewidth}{0.5pt}\end{center}

\subsection*{પ્રશ્ન 2(બ) [4
ગુણ]}\label{uxaaauxab0uxab6uxaa8-2uxaac-4-uxa97uxaa3}

\textbf{રીગ્રેશન સ્ટેપ્સ વિગતવાર સમજાવો.}

\begin{solutionbox}

\textbf{રીગ્રેશન પ્રોસેસ સ્ટેપ્સ:}

\begin{verbatim}
flowchart LR
    A[Data Collection] {-{-} B[Data Preprocessing]}
    B {-{-} C[Feature Selection]}
    C {-{-} D[Model Training]}
    D {-{-} E[Model Evaluation]}
    E {-{-} F[Prediction]}
\end{verbatim}

\textbf{વિગતવાર સ્ટેપ્સ:}

\begin{itemize}
\tightlist
\item
  \textbf{Data Collection}: ઇનપુટ-આઉટપુટ જોડી સાથે સંબંધિત ડેટાસેટ એકત્રિત કરવું
\item
  \textbf{Preprocessing}: ડેટા સાફ કરવું, ખોવાયેલા મૂલ્યો સંભાળવા, features ને
  normalize કરવા
\item
  \textbf{Feature Selection}: લક્ષ્યને અસર કરતા સંબંધિત variables પસંદ કરવા
\item
  \textbf{Model Training}: આગાહીની ભૂલો ન્યૂનતમ કરવા માટે રીગ્રેશન લાઇન ફિટ
  કરવી
\end{itemize}

\end{solutionbox}
\begin{mnemonicbox}
``Data Preprocessing Features Train Evaluation
Predicts'' (DPFTEP)

\end{mnemonicbox}
\begin{center}\rule{0.5\linewidth}{0.5pt}\end{center}

\subsection*{પ્રશ્ન 2(ક) [7
ગુણ]}\label{uxaaauxab0uxab6uxaa8-2uxa95-7-uxa97uxaa3}

\textbf{ચોકસાઈ વ્યાખ્યાયિત કરો અને નીચેના binary classifier ની confusion
matrix માટે વિવિધ માપન પરિમાણો શોધો જેમ કે 1. Accuracy 2. Precision.}

\begin{solutionbox}

\textbf{Confusion Matrix વિશ્લેષણ:}

{\def\LTcaptype{none} % do not increment counter
\begin{longtable}[]{@{}lll@{}}
\toprule\noalign{}
& અનુમાનિત ના & અનુમાનિત હા \\
\midrule\noalign{}
\endhead
\bottomrule\noalign{}
\endlastfoot
\textbf{વાસ્તવિક ના} & 10 (TN) & 3 (FP) \\
\textbf{વાસ્તવિક હા} & 2 (FN) & 15 (TP) \\
\end{longtable}
}

\textbf{ગણતરીઓ:}

{\def\LTcaptype{none} % do not increment counter
\begin{longtable}[]{@{}llll@{}}
\toprule\noalign{}
મેટ્રિક & ફોર્મ્યુલા & ગણતરી & પરિણામ \\
\midrule\noalign{}
\endhead
\bottomrule\noalign{}
\endlastfoot
\textbf{Accuracy} & (TP+TN)/(TP+TN+FP+FN) & (15+10)/(15+10+3+2) &
83.33\% \\
\textbf{Precision} & TP/(TP+FP) & 15/(15+3) & 83.33\% \\
\end{longtable}
}

\textbf{વ્યાખ્યાઓ:}

\begin{itemize}
\tightlist
\item
  \textbf{Accuracy}: કુલ આગાહીઓમાંથી સાચી આગાહીઓનું પ્રમાણ
\item
  \textbf{Precision}: બધી positive આગાહીઓમાંથી true positive આગાહીઓનું
  પ્રમાણ
\end{itemize}

\end{solutionbox}
\begin{mnemonicbox}
``Accuracy Counts All, Precision Picks Positives''
(ACAPP)

\end{mnemonicbox}
\begin{center}\rule{0.5\linewidth}{0.5pt}\end{center}

\subsection*{પ્રશ્ન 2(અ OR) [3
ગુણ]}\label{uxaaauxab0uxab6uxaa8-2uxa85-or-3-uxa97uxaa3}

\textbf{Feature સબસેટ પસંદગીના મૂળભૂત પગલાઓને ઓળખો.}

\begin{solutionbox}

\textbf{Feature Subset Selection સ્ટેપ્સ:}

\begin{verbatim}
flowchart LR
    A[Original Features] {-{-} B[Generate Subsets]}
    B {-{-} C[Evaluate Subsets]}
    C {-{-} D[Select Best Subset]}
\end{verbatim}

\textbf{મૂળભૂત પગલાઓ:}

\begin{itemize}
\tightlist
\item
  \textbf{Generation}: Features ના વિવિધ સંયોજનો બનાવવા
\item
  \textbf{Evaluation}: પ્રત્યેક સબસેટને પર્ફોર્મન્સ મેટ્રિક્સ વાપરીને ટેસ્ટ કરવા
\item
  \textbf{Selection}: માપદંડોના આધારે શ્રેષ્ઠ સબસેટ પસંદ કરવા
\end{itemize}

\end{solutionbox}
\begin{mnemonicbox}
``Generate, Evaluate, Select'' (GES)

\end{mnemonicbox}
\begin{center}\rule{0.5\linewidth}{0.5pt}\end{center}

\subsection*{પ્રશ્ન 2(બ OR) [4
ગુણ]}\label{uxaaauxab0uxab6uxaa8-2uxaac-or-4-uxa97uxaa3}

\textbf{KNN અલ્ગોરિધમની તાકાત અને નબળાઈની ચર્ચા કરો.}

\begin{solutionbox}

\textbf{KNN અલ્ગોરિધમ વિશ્લેષણ:}

{\def\LTcaptype{none} % do not increment counter
\begin{longtable}[]{@{}ll@{}}
\toprule\noalign{}
તાકાતો & નબળાઈઓ \\
\midrule\noalign{}
\endhead
\bottomrule\noalign{}
\endlastfoot
સમજવામાં સરળ & કમ્પ્યુટેશનલી મોંઘું \\
Training ની જરૂર નથી & અપ્રસ્તુત features ને સંવેદનશીલ \\
Non-linear ડેટા સાથે કામ કરે & High dimensions સાથે performance ઘટે \\
નાના ડેટાસેટ માટે અસરકારક & શ્રેષ્ઠ K value પસંદગી જરૂરી \\
\end{longtable}
}

\textbf{મુખ્ય મુદ્દાઓ:}

\begin{itemize}
\tightlist
\item
  \textbf{Lazy Learning}: સ્પષ્ટ training phase ની જરૂર નથી
\item
  \textbf{Distance-Based}: પડોશીની નજીકતા આધારિત વર્ગીકરણ
\item
  \textbf{Memory-Intensive}: સંપૂર્ણ training ડેટાસેટ સ્ટોર કરે છે
\end{itemize}

\end{solutionbox}
\begin{mnemonicbox}
``Simple but Slow, Effective but Expensive''
(SBSEBE)

\end{mnemonicbox}
\begin{center}\rule{0.5\linewidth}{0.5pt}\end{center}

\subsection*{પ્રશ્ન 2(ક OR) [7
ગુણ]}\label{uxaaauxab0uxab6uxaa8-2uxa95-or-7-uxa97uxaa3}

\textbf{ભૂલ-દર વ્યાખ્યાયિત કરો અને નીચેના binary classifier ની confusion
matrix માટે વિવિધ માપન પરિમાણો શોધો જેમ કે 1. Error value 2. Recall.}

\begin{solutionbox}

\textbf{Confusion Matrix વિશ્લેષણ:}

{\def\LTcaptype{none} % do not increment counter
\begin{longtable}[]{@{}lll@{}}
\toprule\noalign{}
& અનુમાનિત ના & અનુમાનિત હા \\
\midrule\noalign{}
\endhead
\bottomrule\noalign{}
\endlastfoot
\textbf{વાસ્તવિક ના} & 20 (TN) & 3 (FP) \\
\textbf{વાસ્તવિક હા} & 2 (FN) & 15 (TP) \\
\end{longtable}
}

\textbf{ગણતરીઓ:}

{\def\LTcaptype{none} % do not increment counter
\begin{longtable}[]{@{}llll@{}}
\toprule\noalign{}
મેટ્રિક & ફોર્મ્યુલા & ગણતરી & પરિણામ \\
\midrule\noalign{}
\endhead
\bottomrule\noalign{}
\endlastfoot
\textbf{Error Rate} & (FP+FN)/(TP+TN+FP+FN) & (3+2)/(15+20+3+2) &
12.5\% \\
\textbf{Recall} & TP/(TP+FN) & 15/(15+2) & 88.24\% \\
\end{longtable}
}

\textbf{વ્યાખ્યાઓ:}

\begin{itemize}
\tightlist
\item
  \textbf{Error Rate}: કુલ આગાહીઓમાંથી ખોટી આગાહીઓનું પ્રમાણ
\item
  \textbf{Recall}: વાસ્તવિક positives માંથી સાચી રીતે ઓળખાયેલાનું પ્રમાણ
\end{itemize}

\end{solutionbox}
\begin{mnemonicbox}
``Error Excludes, Recall Retrieves'' (EERR)

\end{mnemonicbox}
\begin{center}\rule{0.5\linewidth}{0.5pt}\end{center}

\subsection*{પ્રશ્ન 3(અ) [3
ગુણ]}\label{uxaaauxab0uxab6uxaa8-3uxa85-3-uxa97uxaa3}

\textbf{Unsupervised learning ના કોઈ પણ ત્રણ ઉદાહરણો આપો.}

\begin{solutionbox}

\textbf{Unsupervised Learning ઉદાહરણો:}

{\def\LTcaptype{none} % do not increment counter
\begin{longtable}[]{@{}
  >{\raggedright\arraybackslash}p{(\linewidth - 4\tabcolsep) * \real{0.2571}}
  >{\raggedright\arraybackslash}p{(\linewidth - 4\tabcolsep) * \real{0.3714}}
  >{\raggedright\arraybackslash}p{(\linewidth - 4\tabcolsep) * \real{0.3714}}@{}}
\toprule\noalign{}
\begin{minipage}[b]{\linewidth}\raggedright
ઉદાહરણ
\end{minipage} & \begin{minipage}[b]{\linewidth}\raggedright
વર્ણન
\end{minipage} & \begin{minipage}[b]{\linewidth}\raggedright
ઉપયોગ
\end{minipage} \\
\midrule\noalign{}
\endhead
\bottomrule\noalign{}
\endlastfoot
\textbf{Customer Segmentation} & વર્તન દ્વારા ગ્રાહકોને જૂથબદ્ધ કરવા & માર્કેટિંગ
વ્યૂહરચના \\
\textbf{Document Classification} & વિષયો દ્વારા દસ્તાવેજો ગોઠવવા & માહિતી
પુનઃપ્રાપ્તિ \\
\textbf{Gene Sequencing} & સમાન DNA પેટર્ન જૂથબદ્ધ કરવા & તબીબી સંશોધન \\
\end{longtable}
}

\begin{itemize}
\tightlist
\item
  \textbf{Market Basket Analysis}: ઉત્પાદન ખરીદીના પેટર્ન શોધવા
\item
  \textbf{Social Network Analysis}: સમુદાયની રચનાઓ ઓળખવી
\item
  \textbf{Anomaly Detection}: ડેટામાં અસામાન્ય પેટર્ન શોધવા
\end{itemize}

\end{solutionbox}
\begin{mnemonicbox}
``Customers, Documents, Genes Group Automatically''
(CDGGA)

\end{mnemonicbox}
\begin{center}\rule{0.5\linewidth}{0.5pt}\end{center}

\subsection*{પ્રશ્ન 3(બ) [4
ગુણ]}\label{uxaaauxab0uxab6uxaa8-3uxaac-4-uxa97uxaa3}

\textbf{નીચેના ડેટા માટે સરેરાશ અને મધ્યક શોધો: 4,6,7,8,9,12,14,15,20}

\begin{solutionbox}

\textbf{આંકડાકીય ગણતરીઓ:}

{\def\LTcaptype{none} % do not increment counter
\begin{longtable}[]{@{}lll@{}}
\toprule\noalign{}
આંકડા & ગણતરી & પરિણામ \\
\midrule\noalign{}
\endhead
\bottomrule\noalign{}
\endlastfoot
\textbf{સરેરાશ (Mean)} & (4+6+7+8+9+12+14+15+20)/9 & 10.56 \\
\textbf{મધ્યક (Median)} & મધ્ય મૂલ્ય (5મી સ્થિતિ) & 9 \\
\end{longtable}
}

\textbf{પગલું-દર-પગલું:}

\begin{itemize}
\tightlist
\item
  \textbf{ડેટા}: પહેલેથી જ સૉર્ટ થયેલ: 4,6,7,8,9,12,14,15,20
\item
  \textbf{સરેરાશ}: બધા મૂલ્યોનો સરવાળો \div ગણતરી = 95 \div 9 = 10.56
\item
  \textbf{મધ્યક}: સૉર્ટ કરેલ યાદીમાં મધ્ય મૂલ્ય = 9 (5મી સ્થિતિ)
\end{itemize}

\end{solutionbox}
\begin{mnemonicbox}
``Mean Averages All, Median Middle Value'' (MAAMV)

\end{mnemonicbox}
\begin{center}\rule{0.5\linewidth}{0.5pt}\end{center}

\subsection*{પ્રશ્ન 3(ક) [7
ગુણ]}\label{uxaaauxab0uxab6uxaa8-3uxa95-7-uxa97uxaa3}

\textbf{k-ફોલ્ડ ક્રોસ વેલિડેશન પદ્ધતિનું વિગતવાર વર્ણન કરો.}

\begin{solutionbox}

\textbf{K-Fold Cross Validation પ્રોસેસ:}

\begin{verbatim}
flowchart LR
    A[Original Dataset] {-{-} B[Split into K folds]}
    B {-{-} C[Train on K{-}1 folds]}
    C {-{-} D[Test on 1 fold]}
    D {-{-} E[Repeat K times]}
    E {-{-} F[Average Results]}
\end{verbatim}

\textbf{પ્રોસેસ સ્ટેપ્સ:}

{\def\LTcaptype{none} % do not increment counter
\begin{longtable}[]{@{}
  >{\raggedright\arraybackslash}p{(\linewidth - 4\tabcolsep) * \real{0.2143}}
  >{\raggedright\arraybackslash}p{(\linewidth - 4\tabcolsep) * \real{0.4643}}
  >{\raggedright\arraybackslash}p{(\linewidth - 4\tabcolsep) * \real{0.3214}}@{}}
\toprule\noalign{}
\begin{minipage}[b]{\linewidth}\raggedright
પગલું
\end{minipage} & \begin{minipage}[b]{\linewidth}\raggedright
વર્ણન
\end{minipage} & \begin{minipage}[b]{\linewidth}\raggedright
હેતુ
\end{minipage} \\
\midrule\noalign{}
\endhead
\bottomrule\noalign{}
\endlastfoot
\textbf{1. ડેટા વિભાજન} & ડેટાને K સમાન ભાગોમાં વહેંચવું & સંતુલિત પરીક્ષણ સુનિશ્ચિત
કરવું \\
\textbf{2. પુનરાવર્તિત Training} & Training માટે K-1 folds નો ઉપયોગ & મહત્તમ
ડેટા ઉપયોગ \\
\textbf{3. Validation} & બાકીના fold પર ટેસ્ટ કરવું & નિષ્પક્ષ મૂલ્યાંકન \\
\textbf{4. સરેરાશ} & સરેરાશ performance ગણવું & મજબૂત performance અંદાજ \\
\end{longtable}
}

\textbf{ફાયદાઓ:}

\begin{itemize}
\tightlist
\item
  \textbf{નિષ્પક્ષ અંદાજ}: દરેક ડેટા પોઇન્ટ training અને testing બંને માટે વાપરાય
\item
  \textbf{Overfitting ઘટાડવું}: અનેક validation રાઉન્ડ વિશ્વસનીયતા વધારે
\item
  \textbf{કાર્યક્ષમ ડેટા ઉપયોગ}: બધો ડેટા training અને validation બંને માટે
  ઉપયોગ
\end{itemize}

\end{solutionbox}
\begin{mnemonicbox}
``K-fold Keeps Keen Knowledge'' (KKKK)

\end{mnemonicbox}
\begin{center}\rule{0.5\linewidth}{0.5pt}\end{center}

\subsection*{પ્રશ્ન 3(અ OR) [3
ગુણ]}\label{uxaaauxab0uxab6uxaa8-3uxa85-or-3-uxa97uxaa3}

\textbf{Multiple linear રીગ્રેશનની કોઈ પણ ત્રણ એપ્લિકેશન આપો.}

\begin{solutionbox}

\textbf{Multiple Linear Regression એપ્લિકેશન:}

{\def\LTcaptype{none} % do not increment counter
\begin{longtable}[]{@{}
  >{\raggedright\arraybackslash}p{(\linewidth - 4\tabcolsep) * \real{0.3939}}
  >{\raggedright\arraybackslash}p{(\linewidth - 4\tabcolsep) * \real{0.3333}}
  >{\raggedright\arraybackslash}p{(\linewidth - 4\tabcolsep) * \real{0.2727}}@{}}
\toprule\noalign{}
\begin{minipage}[b]{\linewidth}\raggedright
એપ્લિકેશન
\end{minipage} & \begin{minipage}[b]{\linewidth}\raggedright
Variables
\end{minipage} & \begin{minipage}[b]{\linewidth}\raggedright
હેતુ
\end{minipage} \\
\midrule\noalign{}
\endhead
\bottomrule\noalign{}
\endlastfoot
\textbf{House Price Prediction} & Size, location, age & રિયલ એસ્ટેટ
વેલ્યુએશન \\
\textbf{Sales Forecasting} & Marketing spend, season, economy & બિઝનેસ
પ્લાનિંગ \\
\textbf{Medical Diagnosis} & Symptoms, age, history & રોગની આગાહી \\
\end{longtable}
}

\begin{itemize}
\tightlist
\item
  \textbf{Stock Market Analysis}: અનેક આર્થિક સૂચકાંકો શેર કિંમતોની આગાહી કરે
\item
  \textbf{Academic Performance}: અભ્યાસના કલાકો, હાજરી, અગાઉના ગ્રેડ સ્કોરની
  આગાહી
\item
  \textbf{Marketing ROI}: વિવિધ માર્કેટિંગ ચેનલો વેચાણ આવક પર અસર કરે
\end{itemize}

\end{solutionbox}
\begin{mnemonicbox}
``Houses, Sales, Medicine Predict Multiple
Variables'' (HSMPV)

\end{mnemonicbox}
\begin{center}\rule{0.5\linewidth}{0.5pt}\end{center}

\subsection*{પ્રશ્ન 3(બ OR) [4
ગુણ]}\label{uxaaauxab0uxab6uxaa8-3uxaac-or-4-uxa97uxaa3}

\textbf{નીચેના ડેટા માટે માનક વિચલન શોધો: 4,15,20,28,35,45}

\begin{solutionbox}

\textbf{માનક વિચલન ગણતરી:}

{\def\LTcaptype{none} % do not increment counter
\begin{longtable}[]{@{}lll@{}}
\toprule\noalign{}
પગલું & ગણતરી & મૂલ્ય \\
\midrule\noalign{}
\endhead
\bottomrule\noalign{}
\endlastfoot
\textbf{સરેરાશ} & (4+15+20+28+35+45)/6 & 24.5 \\
\textbf{Variance} & Σ(xi-mean)^{2}/n & 178.92 \\
\textbf{Std Dev} & \sqrtVariance & 13.38 \\
\end{longtable}
}

\textbf{વિગતવાર ગણતરી:}

\begin{itemize}
\tightlist
\item
  \textbf{સરેરાશથી વિચલન}: (-20.5)^{2}, (-9.5)^{2}, (-4.5)^{2}, (3.5)^{2}, (10.5)^{2},
  (20.5)^{2}
\item
  \textbf{વર્ગ વિચલન}: 420.25, 90.25, 20.25, 12.25, 110.25, 420.25
\item
  \textbf{સરવાળો}: 1073.5
\item
  \textbf{Variance}: 1073.5/6 = 178.92
\item
  \textbf{માનક વિચલન}: \sqrt178.92 = 13.38
\end{itemize}

\end{solutionbox}
\begin{mnemonicbox}
``Deviation Measures Data Spread'' (DMDS)

\end{mnemonicbox}
\begin{center}\rule{0.5\linewidth}{0.5pt}\end{center}

\subsection*{પ્રશ્ન 3(ક OR) [7
ગુણ]}\label{uxaaauxab0uxab6uxaa8-3uxa95-or-7-uxa97uxaa3}

\textbf{બેગિંગ અને બૂસ્ટિંગને વિગતવાર સમજાવો.}

\begin{solutionbox}

\textbf{Ensemble Methods તુલના:}

{\def\LTcaptype{none} % do not increment counter
\begin{longtable}[]{@{}lll@{}}
\toprule\noalign{}
પાસું & Bagging & Boosting \\
\midrule\noalign{}
\endhead
\bottomrule\noalign{}
\endlastfoot
\textbf{વ્યૂહરચના} & સમાંતર training & ક્રમિક training \\
\textbf{ડેટા સેમ્પલિંગ} & રેન્ડમ with replacement & વેઇટેડ સેમ્પલિંગ \\
\textbf{સંયોજન} & સરળ સરેરાશ/voting & વેઇટેડ સંયોજન \\
\textbf{Bias-Variance} & Variance ઘટાડે & Bias ઘટાડે \\
\end{longtable}
}

\textbf{Bagging (Bootstrap Aggregating):}

\begin{verbatim}
flowchart LR
    A[Original Data] {-{-} B[Bootstrap Sample 1]}
    A {-{-} C[Bootstrap Sample 2]}
    A {-{-} D[Bootstrap Sample n]}
    B {-{-} E[Model 1]}
    C {-{-} F[Model 2]}
    D {-{-} G[Model n]}
    E {-{-} H[Final Prediction]}
    F {-{-} H}
    G {-{-} H}
\end{verbatim}

\textbf{Boosting પ્રોસેસ:}

\begin{itemize}
\tightlist
\item
  \textbf{ક્રમિક શીખવું}: દરેક મોડલ અગાઉના મોડલની ભૂલોમાંથી શીખે છે
\item
  \textbf{વેઇટ એડજસ્ટમેન્ટ}: ખોટા વર્ગીકૃત ઉદાહરણોનું વેઇટ વધારવું
\item
  \textbf{અંતિમ આગાહી}: બધા મોડલ્સનું વેઇટેડ સંયોજન
\end{itemize}

\textbf{મુખ્ય તફાવતો:}

\begin{itemize}
\tightlist
\item
  \textbf{Bagging}: સ્વતંત્ર મોડલ્સ સમાંતરમાં trained, overfitting ઘટાડે
\item
  \textbf{Boosting}: આશ્રિત મોડલ્સ ક્રમિક trained, accuracy સુધારે
\end{itemize}

\end{solutionbox}
\begin{mnemonicbox}
``Bagging Builds Parallel, Boosting Builds
Sequential'' (BBPBS)

\end{mnemonicbox}
\begin{center}\rule{0.5\linewidth}{0.5pt}\end{center}

\subsection*{પ્રશ્ન 4(અ) [3
ગુણ]}\label{uxaaauxab0uxab6uxaa8-4uxa85-3-uxa97uxaa3}

\textbf{વ્યાખ્યાયિત કરો: Support, Confidence.}

\begin{solutionbox}

\textbf{Association Rule મેટ્રિક્સ:}

{\def\LTcaptype{none} % do not increment counter
\begin{longtable}[]{@{}
  >{\raggedright\arraybackslash}p{(\linewidth - 4\tabcolsep) * \real{0.2759}}
  >{\raggedright\arraybackslash}p{(\linewidth - 4\tabcolsep) * \real{0.4138}}
  >{\raggedright\arraybackslash}p{(\linewidth - 4\tabcolsep) * \real{0.3103}}@{}}
\toprule\noalign{}
\begin{minipage}[b]{\linewidth}\raggedright
મેટ્રિક
\end{minipage} & \begin{minipage}[b]{\linewidth}\raggedright
વ્યાખ્યા
\end{minipage} & \begin{minipage}[b]{\linewidth}\raggedright
ફોર્મ્યુલા
\end{minipage} \\
\midrule\noalign{}
\endhead
\bottomrule\noalign{}
\endlastfoot
\textbf{Support} & ટ્રાન્ઝેક્શનમાં itemset ની આવર્તન & Support(A) =
Count(A)/કુલ ટ્રાન્ઝેક્શન \\
\textbf{Confidence} & નિયમની શરતી સંભાવના & Confidence(A\rightarrowB) =
Support(A\cupB)/Support(A) \\
\end{longtable}
}

\textbf{ઉદાહરણ:}

\begin{itemize}
\tightlist
\item
  \textbf{Support(Bread)} = 0.6 (60\% ટ્રાન્ઝેક્શનમાં બ્રેડ છે)
\item
  \textbf{Confidence(Bread\rightarrowButter)} = 0.8 (80\% બ્રેડ ખરીદનારા બટર પણ ખરીદે
  છે)
\end{itemize}

\textbf{ઉપયોગો:}

\begin{itemize}
\tightlist
\item
  \textbf{Market Basket Analysis}: ઉત્પાદન સંબંધો શોધવા
\item
  \textbf{Recommendation Systems}: સંબંધિત વસ્તુઓ સૂચવવી
\end{itemize}

\end{solutionbox}
\begin{mnemonicbox}
``Support Shows Frequency, Confidence Shows
Connection'' (SSFC)

\end{mnemonicbox}
\begin{center}\rule{0.5\linewidth}{0.5pt}\end{center}

\subsection*{પ્રશ્ન 4(બ) [4
ગુણ]}\label{uxaaauxab0uxab6uxaa8-4uxaac-4-uxa97uxaa3}

\textbf{લોજિસ્ટિક રીગ્રેશનની કોઈ પણ બે એપ્લિકેશનને સમજાવો.}

\begin{solutionbox}

\textbf{Logistic Regression એપ્લિકેશન:}

{\def\LTcaptype{none} % do not increment counter
\begin{longtable}[]{@{}
  >{\raggedright\arraybackslash}p{(\linewidth - 6\tabcolsep) * \real{0.2766}}
  >{\raggedright\arraybackslash}p{(\linewidth - 6\tabcolsep) * \real{0.3404}}
  >{\raggedright\arraybackslash}p{(\linewidth - 6\tabcolsep) * \real{0.1702}}
  >{\raggedright\arraybackslash}p{(\linewidth - 6\tabcolsep) * \real{0.2128}}@{}}
\toprule\noalign{}
\begin{minipage}[b]{\linewidth}\raggedright
એપ્લિકેશન
\end{minipage} & \begin{minipage}[b]{\linewidth}\raggedright
Input Variables
\end{minipage} & \begin{minipage}[b]{\linewidth}\raggedright
Output
\end{minipage} & \begin{minipage}[b]{\linewidth}\raggedright
ઉપયોગનો કેસ
\end{minipage} \\
\midrule\noalign{}
\endhead
\bottomrule\noalign{}
\endlastfoot
\textbf{Email Spam Detection} & શબ્દ આવર્તન, sender, subject & Spam/Not
Spam & Email filtering \\
\textbf{Medical Diagnosis} & લક્ષણો, ઉંમર, ટેસ્ટ પરિણામો & રોગ/કોઈ રોગ નથી &
આરોગ્યસેવા \\
\end{longtable}
}

\textbf{મુખ્ય લાક્ષણિકતાઓ:}

\begin{itemize}
\tightlist
\item
  \textbf{Binary Classification}: 0 અને 1 વચ્ચે સંભાવના આગાહી કરે છે
\item
  \textbf{S-shaped Curve}: સંભાવના અંદાજ માટે sigmoid function વાપરે છે
\item
  \textbf{Linear Decision Boundary}: linear boundary સાથે વર્ગો અલગ કરે છે
\end{itemize}

\textbf{વાસ્તવિક જીવનના ઉદાહરણો:}

\begin{itemize}
\tightlist
\item
  \textbf{Marketing}: demographics આધારે ગ્રાહક ખરીદીની સંભાવના
\item
  \textbf{Finance}: ક્રેડિટ હિસ્ટ્રી અને આવક આધારે ક્રેડિટ મંજૂરી
\end{itemize}

\end{solutionbox}
\begin{mnemonicbox}
``Logistic Limits Linear Logic'' (LLLL)

\end{mnemonicbox}
\begin{center}\rule{0.5\linewidth}{0.5pt}\end{center}

\subsection*{પ્રશ્ન 4(ક) [7
ગુણ]}\label{uxaaauxab0uxab6uxaa8-4uxa95-7-uxa97uxaa3}

\textbf{Machine learning માં Numpy અને Pandas ના મુખ્ય હેતુની ચર્ચા કરો.}

\begin{solutionbox}

\textbf{ML માં NumPy અને Pandas:}

{\def\LTcaptype{none} % do not increment counter
\begin{longtable}[]{@{}lll@{}}
\toprule\noalign{}
Library & હેતુ & મુખ્ય લાક્ષણિકતાઓ \\
\midrule\noalign{}
\endhead
\bottomrule\noalign{}
\endlastfoot
\textbf{NumPy} & Numerical computing & Arrays, mathematical functions \\
\textbf{Pandas} & Data manipulation & DataFrames, data cleaning \\
\end{longtable}
}

\textbf{NumPy Functions:}

\begin{center}
\textbf{Mermaid Diagram (Code)}
\begin{verbatim}
{Shaded}
{Highlighting}[]
graph TD
    A[NumPy] {-{-}{} B[Array Operations]}
    A {-{-}{} C[Mathematical Functions]}
    A {-{-}{} D[Linear Algebra]}
    A {-{-}{} E[Random Numbers]}
{Highlighting}
{Shaded}
\end{verbatim}
\end{center}

\textbf{Pandas ક્ષમતાઓ:}

\begin{itemize}
\tightlist
\item
  \textbf{Data Import/Export}: CSV, Excel, JSON ફાઇલો વાંચવી
\item
  \textbf{Data Cleaning}: ખોવાયેલા મૂલ્યો, duplicates સંભાળવા
\item
  \textbf{Data Transformation}: Group, merge, pivot operations
\item
  \textbf{Statistical Analysis}: વર્ણનાત્મક આંકડા, correlation
\end{itemize}

\textbf{ML સાથે Integration:}

\begin{itemize}
\tightlist
\item
  \textbf{Data Preprocessing}: અલ્ગોરિધમ માટે ડેટા સાફ અને તૈયાર કરવો
\item
  \textbf{Feature Engineering}: હાલના ડેટામાંથી નવા features બનાવવા
\item
  \textbf{Model Input}: ML અલ્ગોરિધમ દ્વારા જરૂરી ફોર્મેટમાં ડેટા કન્વર્ટ કરવો
\end{itemize}

\textbf{મુખ્ય ફાયદાઓ:}

\begin{itemize}
\tightlist
\item
  \textbf{Performance}: ઝડપ માટે C/C++ backend optimized
\item
  \textbf{Memory Efficiency}: કાર્યક્ષમ ડેટા સ્ટોરેજ અને manipulation
\item
  \textbf{Ecosystem Integration}: scikit-learn, matplotlib સાથે
  seamlessly કામ કરે
\end{itemize}

\end{solutionbox}
\begin{mnemonicbox}
``NumPy Numbers, Pandas Processes Data'' (NNPD)

\end{mnemonicbox}
\begin{center}\rule{0.5\linewidth}{0.5pt}\end{center}

\subsection*{પ્રશ્ન 4(અ OR) [3
ગુણ]}\label{uxaaauxab0uxab6uxaa8-4uxa85-or-3-uxa97uxaa3}

\textbf{સુપરવાઇઝ્ડ લર્નિંગના કોઈ પણ ત્રણ ઉદાહરણો આપો.}

\begin{solutionbox}

\textbf{Supervised Learning ઉદાહરણો:}

{\def\LTcaptype{none} % do not increment counter
\begin{longtable}[]{@{}
  >{\raggedright\arraybackslash}p{(\linewidth - 4\tabcolsep) * \real{0.2903}}
  >{\raggedright\arraybackslash}p{(\linewidth - 4\tabcolsep) * \real{0.1935}}
  >{\raggedright\arraybackslash}p{(\linewidth - 4\tabcolsep) * \real{0.5161}}@{}}
\toprule\noalign{}
\begin{minipage}[b]{\linewidth}\raggedright
ઉદાહરણ
\end{minipage} & \begin{minipage}[b]{\linewidth}\raggedright
પ્રકાર
\end{minipage} & \begin{minipage}[b]{\linewidth}\raggedright
Input \rightarrow Output
\end{minipage} \\
\midrule\noalign{}
\endhead
\bottomrule\noalign{}
\endlastfoot
\textbf{Email Classification} & Classification & Email features \rightarrow
Spam/Not Spam \\
\textbf{House Price Prediction} & Regression & House features \rightarrow કિંમત \\
\textbf{Image Recognition} & Classification & Pixel values \rightarrow Object
class \\
\end{longtable}
}

\begin{itemize}
\tightlist
\item
  \textbf{Medical Diagnosis}: દર્દીના લક્ષણો \rightarrow રોગ વર્ગીકરણ
\item
  \textbf{Stock Price Prediction}: માર્કેટ સૂચકાંકો \rightarrow ભાવિ કિંમત
\item
  \textbf{Speech Recognition}: Audio signals \rightarrow Text transcription
\end{itemize}

\end{solutionbox}
\begin{mnemonicbox}
``Emails, Houses, Images Learn Supervised'' (EHILS)

\end{mnemonicbox}
\begin{center}\rule{0.5\linewidth}{0.5pt}\end{center}

\subsection*{પ્રશ્ન 4(બ OR) [4
ગુણ]}\label{uxaaauxab0uxab6uxaa8-4uxaac-or-4-uxa97uxaa3}

\textbf{એપ્રિઓરી અલ્ગોરિધમના કોઈ પણ બે એપ્લિકેશનો સમજાવો.}

\begin{solutionbox}

\textbf{Apriori Algorithm એપ્લિકેશન:}

{\def\LTcaptype{none} % do not increment counter
\begin{longtable}[]{@{}
  >{\raggedright\arraybackslash}p{(\linewidth - 4\tabcolsep) * \real{0.3095}}
  >{\raggedright\arraybackslash}p{(\linewidth - 4\tabcolsep) * \real{0.3095}}
  >{\raggedright\arraybackslash}p{(\linewidth - 4\tabcolsep) * \real{0.3810}}@{}}
\toprule\noalign{}
\begin{minipage}[b]{\linewidth}\raggedright
એપ્લિકેશન
\end{minipage} & \begin{minipage}[b]{\linewidth}\raggedright
વર્ણન
\end{minipage} & \begin{minipage}[b]{\linewidth}\raggedright
બિઝનેસ વેલ્યુ
\end{minipage} \\
\midrule\noalign{}
\endhead
\bottomrule\noalign{}
\endlastfoot
\textbf{Market Basket Analysis} & એકસાથે ખરીદાતા ઉત્પાદનો શોધવા &
Cross-selling વ્યૂહરચના \\
\textbf{Web Usage Mining} & વેબસાઇટ navigation પેટર્ન શોધવા & વપરાશકર્તા
અનુભવ સુધારવો \\
\end{longtable}
}

\textbf{Market Basket Analysis:}

\begin{itemize}
\tightlist
\item
  \textbf{ઉદાહરણ}: ``બ્રેડ અને મિલ્ક ખરીદનારા ગ્રાહકો ઈંડા પણ ખરીદે છે''
\item
  \textbf{બિઝનેસ અસર}: ઉત્પાદન પ્લેસમેન્ટ, પ્રમોશનલ ઓફર
\item
  \textbf{Implementation}: frequent itemsets શોધવા માટે transaction ડેટાનું
  વિશ્લેષણ
\end{itemize}

\textbf{Web Usage Mining:}

\begin{itemize}
\tightlist
\item
  \textbf{ઉદાહરણ}: ``પેજ A visit કરનારા users ઘણીવાર આગળ પેજ B visit કરે
  છે''
\item
  \textbf{વેબસાઇટ Optimization}: navigation સુધારવી, content recommend કરવું
\item
  \textbf{User Experience}: વ્યક્તિગત વેબસાઇટ layouts
\end{itemize}

\textbf{Algorithm પ્રોસેસ:}

\begin{itemize}
\tightlist
\item
  \textbf{Generate Candidates}: frequent itemsets બનાવવા
\item
  \textbf{Prune}: infrequent items દૂર કરવા
\item
  \textbf{Generate Rules}: confidence સાથે association rules બનાવવા
\end{itemize}

\end{solutionbox}
\begin{mnemonicbox}
``Apriori Analyzes Associations Automatically''
(AAAA)

\end{mnemonicbox}
\begin{center}\rule{0.5\linewidth}{0.5pt}\end{center}

\subsection*{પ્રશ્ન 4(ક OR) [7
ગુણ]}\label{uxaaauxab0uxab6uxaa8-4uxa95-or-7-uxa97uxaa3}

\textbf{Matplotlib ની વિશેષતાઓ અને એપ્લિકેશનો સમજાવો.}

\begin{solutionbox}

\textbf{Matplotlib Features અને Applications:}

{\def\LTcaptype{none} % do not increment counter
\begin{longtable}[]{@{}lll@{}}
\toprule\noalign{}
Feature કેટેગરી & ક્ષમતાઓ & એપ્લિકેશન \\
\midrule\noalign{}
\endhead
\bottomrule\noalign{}
\endlastfoot
\textbf{Plot Types} & Line, bar, scatter, histogram & ડેટા exploration \\
\textbf{Customization} & રંગો, labels, styles & વ્યવસાયિક presentations \\
\textbf{Subplots} & એક figure માં અનેક plots & તુલનાત્મક વિશ્લેષણ \\
\textbf{3D Plotting} & ત્રિ-પરિમાણીય visualizations & વૈજ્ઞાનિક modeling \\
\end{longtable}
}

\textbf{મુખ્ય Features:}

\begin{center}
\textbf{Mermaid Diagram (Code)}
\begin{verbatim}
{Shaded}
{Highlighting}[]
graph TD
    A[Matplotlib] {-{-}{} B[2D Plotting]}
    A {-{-}{} C[3D Plotting]}
    A {-{-}{} D[Interactive Plots]}
    A {-{-}{} E[Publication Quality]}
    B {-{-}{} F[Line Charts]}
    B {-{-}{} G[Bar Charts]}
    B {-{-}{} H[Scatter Plots]}
    C {-{-}{} I[Surface Plots]}
    C {-{-}{} J[3D Scatter]}
{Highlighting}
{Shaded}
\end{verbatim}
\end{center}

\textbf{Machine Learning માં Applications:}

\begin{itemize}
\tightlist
\item
  \textbf{Data Exploration}: ડેટા વિતરણ અને પેટર્ન visualize કરવા
\item
  \textbf{Model Performance}: training દરમિયાન accuracy, loss curves
  plot કરવા
\item
  \textbf{Result Presentation}: predictions vs actual values દેખાડવા
\item
  \textbf{Feature Analysis}: Correlation matrices, feature importance
  plots
\end{itemize}

\textbf{અદ્યતન ક્ષમતાઓ:}

\begin{itemize}
\tightlist
\item
  \textbf{Animation}: time-series ડેટા માટે animated plots બનાવવા
\item
  \textbf{Interactive Widgets}: વપરાશકર્તા interaction માટે sliders,
  buttons ઉમેરવા
\item
  \textbf{Integration}: Jupyter notebooks, web applications સાથે કામ કરે છે
\end{itemize}

\textbf{ફાયદાઓ:}

\begin{itemize}
\tightlist
\item
  \textbf{Flexibility}: અત્યંત customizable plotting options
\item
  \textbf{Community}: વ્યાપક documentation સાથે મોટો વપરાશકર્તા આધાર
\item
  \textbf{Compatibility}: NumPy, Pandas સાથે seamlessly integrate થાય છે
\end{itemize}

\end{solutionbox}
\begin{mnemonicbox}
``Matplotlib Makes Meaningful Visual Displays''
(MMVD)

\end{mnemonicbox}
\begin{center}\rule{0.5\linewidth}{0.5pt}\end{center}

\subsection*{પ્રશ્ન 5(અ) [3
ગુણ]}\label{uxaaauxab0uxab6uxaa8-5uxa85-3-uxa97uxaa3}

\textbf{Numpy ના મુખ્ય features ની યાદી બનાવો.}

\begin{solutionbox}

\textbf{NumPy મુખ્ય Features:}

{\def\LTcaptype{none} % do not increment counter
\begin{longtable}[]{@{}
  >{\raggedright\arraybackslash}p{(\linewidth - 4\tabcolsep) * \real{0.2903}}
  >{\raggedright\arraybackslash}p{(\linewidth - 4\tabcolsep) * \real{0.4194}}
  >{\raggedright\arraybackslash}p{(\linewidth - 4\tabcolsep) * \real{0.2903}}@{}}
\toprule\noalign{}
\begin{minipage}[b]{\linewidth}\raggedright
Feature
\end{minipage} & \begin{minipage}[b]{\linewidth}\raggedright
વર્ણન
\end{minipage} & \begin{minipage}[b]{\linewidth}\raggedright
ફાયદો
\end{minipage} \\
\midrule\noalign{}
\endhead
\bottomrule\noalign{}
\endlastfoot
\textbf{N-dimensional Arrays} & કાર્યક્ષમ array operations & ઝડપી
mathematical computations \\
\textbf{Broadcasting} & વિવિધ size ના arrays પર operations & લવચીક array
manipulation \\
\textbf{Linear Algebra} & Matrix operations, decompositions & વૈજ્ઞાનિક
computing support \\
\end{longtable}
}

\begin{itemize}
\tightlist
\item
  \textbf{Universal Functions}: arrays પર element-wise operations
\item
  \textbf{Memory Efficiency}: ઝડપ માટે contiguous memory layout
\item
  \textbf{C/C++ Integration}: compiled languages સાથે interface
\end{itemize}

\end{solutionbox}
\begin{mnemonicbox}
``NumPy Numbers Need Neat Operations'' (NNNNO)

\end{mnemonicbox}
\begin{center}\rule{0.5\linewidth}{0.5pt}\end{center}

\subsection*{પ્રશ્ન 5(બ) [4
ગુણ]}\label{uxaaauxab0uxab6uxaa8-5uxaac-4-uxa97uxaa3}

\textbf{પ્રોગ્રામમાં iris ડેટાસેટ Pandas Dataframe કેવી રીતે લોડ કરવો? ઉદાહરણ
સાથે સમજાવો.}

\begin{solutionbox}

\textbf{Iris ડેટાસેટ લોડ કરવું:}

\begin{verbatim}
import pandas as pd

\# પદ્ધતિ 1: ફાઇલમાંથી લોડ કરવું
df = pd.read\_csv({iris.csv})

\# પદ્ધતિ 2: sklearn માંથી લોડ કરવું
from sklearn.datasets import load\_iris
iris = load\_iris()
df = pd.DataFrame(iris.data, columns=iris.feature\_names)
df[{target}] = iris.target

\# મૂળભૂત માહિતી દેખાડવી
print(df.head())
print(df.info())
print(df.describe())
\end{verbatim}

\textbf{કોડ સમજાવટ:}

\begin{itemize}
\tightlist
\item
  \textbf{pd.read\_csv()}: CSV ફાઇલને DataFrame માં વાંચે છે
\item
  \textbf{columns parameter}: column નામો assign કરે છે
\item
  \textbf{head()}: પ્રથમ 5 rows બતાવે છે
\item
  \textbf{info()}: data types અને memory usage બતાવે છે
\end{itemize}

\end{solutionbox}
\begin{mnemonicbox}
``Pandas Reads CSV Files Easily'' (PRCFE)

\end{mnemonicbox}
\begin{center}\rule{0.5\linewidth}{0.5pt}\end{center}

\subsection*{પ્રશ્ન 5(ક) [7
ગુણ]}\label{uxaaauxab0uxab6uxaa8-5uxa95-7-uxa97uxaa3}

\textbf{સુપરવાઇઝ્ડ લર્નિંગ અને અનસુપરવાઇઝ્ડ લર્નિંગની સરખામણી કરો અને કોન્ટ્રાસ્ટ
કરો.}

\begin{solutionbox}

\textbf{વ્યાપક તુલના:}

{\def\LTcaptype{none} % do not increment counter
\begin{longtable}[]{@{}
  >{\raggedright\arraybackslash}p{(\linewidth - 4\tabcolsep) * \real{0.1667}}
  >{\raggedright\arraybackslash}p{(\linewidth - 4\tabcolsep) * \real{0.3958}}
  >{\raggedright\arraybackslash}p{(\linewidth - 4\tabcolsep) * \real{0.4375}}@{}}
\toprule\noalign{}
\begin{minipage}[b]{\linewidth}\raggedright
પાસું
\end{minipage} & \begin{minipage}[b]{\linewidth}\raggedright
Supervised Learning
\end{minipage} & \begin{minipage}[b]{\linewidth}\raggedright
Unsupervised Learning
\end{minipage} \\
\midrule\noalign{}
\endhead
\bottomrule\noalign{}
\endlastfoot
\textbf{ડેટા પ્રકાર} & Labeled (input-output જોડી) & Unlabeled (માત્ર
input) \\
\textbf{શીખવાનું લક્ષ્ય} & Target variable ની આગાહી કરવી & છુપાયેલા પેટર્ન
શોધવા \\
\textbf{મૂલ્યાંકન} & Accuracy, precision, recall & Silhouette score,
inertia \\
\textbf{જટિલતા} & મૂલ્યાંકન માટે ઓછું જટિલ & validate કરવું વધુ જટિલ \\
\textbf{એપ્લિકેશન} & Classification, regression & Clustering,
dimensionality reduction \\
\end{longtable}
}

\textbf{વિગતવાર તુલના:}

\begin{center}
\textbf{Mermaid Diagram (Code)}
\begin{verbatim}
{Shaded}
{Highlighting}[]
graph TD
    A[Machine Learning] {-{-}{} B[Supervised]}
    A {-{-}{} C[Unsupervised]}
    B {-{-}{} D[Classification]}
    B {-{-}{} E[Regression]}
    C {-{-}{} F[Clustering]}
    C {-{-}{} G[Association Rules]}
{Highlighting}
{Shaded}
\end{verbatim}
\end{center}

\textbf{Supervised Learning લાક્ષણિકતાઓ:}

\begin{itemize}
\tightlist
\item
  \textbf{Training પ્રોસેસ}: જાણીતા સાચા જવાબો સાથેના ઉદાહરણોમાંથી શીખવું
\item
  \textbf{Performance Measurement}: વાસ્તવિક પરિણામો સાથે સીધી તુલના
\item
  \textbf{સામાન્ય Algorithms}: Decision trees, SVM, neural networks
\item
  \textbf{બિઝનેસ એપ્લિકેશન}: Fraud detection, medical diagnosis, price
  prediction
\end{itemize}

\textbf{Unsupervised Learning લાક્ષણિકતાઓ:}

\begin{itemize}
\tightlist
\item
  \textbf{Exploration}: માર્ગદર્શન વિના અજાણ્યા પેટર્ન શોધવા
\item
  \textbf{Validation Challenges}: સીધી તુલના માટે ground truth નથી
\item
  \textbf{સામાન્ય Algorithms}: K-means, hierarchical clustering, PCA
\item
  \textbf{બિઝનેસ એપ્લિકેશન}: Customer segmentation, market research,
  anomaly detection
\end{itemize}

\textbf{મુખ્ય કોન્ટ્રાસ્ટ:}

\begin{itemize}
\tightlist
\item
  \textbf{Feedback}: Supervised માં તાત્કાલિક feedback, unsupervised
  domain expertise પર આધાર રાખે
\item
  \textbf{ડેટા જરૂરિયાતો}: Supervised ને મોંઘા labeled ડેટાની જરૂર,
  unsupervised સહેલાઈથી ઉપલબ્ધ unlabeled ડેટા વાપરે
\item
  \textbf{સમસ્યાના પ્રકારો}: Supervised prediction સમસ્યાઓ હલ કરે,
  unsupervised discovery સમસ્યાઓ હલ કરે
\end{itemize}

\end{solutionbox}
\begin{mnemonicbox}
``Supervised Seeks Specific Solutions, Unsupervised
Uncovers Unknown'' (SSSUU)

\end{mnemonicbox}
\begin{center}\rule{0.5\linewidth}{0.5pt}\end{center}

\subsection*{પ્રશ્ન 5(અ OR) [3
ગુણ]}\label{uxaaauxab0uxab6uxaa8-5uxa85-or-3-uxa97uxaa3}

\textbf{Pandas ની એપ્લિકેશન્સની યાદી બનાવો.}

\begin{solutionbox}

\textbf{Pandas એપ્લિકેશન:}

{\def\LTcaptype{none} % do not increment counter
\begin{longtable}[]{@{}lll@{}}
\toprule\noalign{}
એપ્લિકેશન & વર્ણન & ઇન્ડસ્ટ્રી \\
\midrule\noalign{}
\endhead
\bottomrule\noalign{}
\endlastfoot
\textbf{Data Cleaning} & ખોવાયેલા મૂલ્યો, duplicates સંભાળવા & બધા
industries \\
\textbf{Financial Analysis} & Stock market, trading ડેટા & ફાઇનાન્સ \\
\textbf{Business Intelligence} & Sales reports, KPI analysis & બિઝનેસ \\
\end{longtable}
}

\begin{itemize}
\tightlist
\item
  \textbf{Scientific Research}: પ્રાયોગિક ડેટા વિશ્લેષણ
\item
  \textbf{Web Analytics}: વેબસાઇટ ટ્રાફિક, વપરાશકર્તા વર્તન વિશ્લેષણ
\item
  \textbf{Healthcare}: દર્દીના રેકોર્ડ, clinical trial ડેટા
\end{itemize}

\end{solutionbox}
\begin{mnemonicbox}
``Pandas Processes Data Perfectly'' (PPDP)

\end{mnemonicbox}
\begin{center}\rule{0.5\linewidth}{0.5pt}\end{center}

\subsection*{પ્રશ્ન 5(બ OR) [4
ગુણ]}\label{uxaaauxab0uxab6uxaa8-5uxaac-or-4-uxa97uxaa3}

\textbf{Matplotlib લાઇબ્રેરીનો ઉપયોગ કરીને આકૃતિ કેવી રીતે બનાવવી? ઉદાહરણ સાથે
સમજાવો.}

\begin{solutionbox}

\textbf{Matplotlib Line Plotting:}

\begin{verbatim}
import matplotlib.pyplot as plt
import numpy as np

\# સેમ્પલ ડેટા બનાવવું
x = np.linspace(0, 10, 100)
y = np.sin(x)

\# મુખ્ય curve plot કરવું
plt.plot(x, y, label={sin(x)})

\# x = 5 પર વર્ટિકલ લાઇન
plt.axvline(x=5, color={red}, linestyle={{-}{-}}, label={Vertical Line})

\# y = 0.5 પર હોરિઝોન્ટલ લાઇન
plt.axhline(y=0.5, color={green}, linestyle={:}, label={Horizontal Line})

\# ફોર્મેટિંગ
plt.xlabel({X{-}axis})
plt.ylabel({Y{-}axis})
plt.legend()
plt.title({Vertical અને Horizontal Lines})
plt.grid(True)
plt.show()
\end{verbatim}

\textbf{મુખ્ય Functions:}

\begin{itemize}
\tightlist
\item
  \textbf{axvline()}: નિર્દિષ્ટ x-coordinate પર vertical line બનાવે
\item
  \textbf{axhline()}: નિર્દિષ્ટ y-coordinate પર horizontal line બનાવે
\item
  \textbf{Parameters}: color, linestyle, linewidth, alpha
\end{itemize}

\end{solutionbox}
\begin{mnemonicbox}
``Matplotlib Makes Lines Easily'' (MMLE)

\end{mnemonicbox}
\begin{center}\rule{0.5\linewidth}{0.5pt}\end{center}

\subsection*{પ્રશ્ન 5(ક OR) [7
ગુણ]}\label{uxaaauxab0uxab6uxaa8-5uxa95-or-7-uxa97uxaa3}

\textbf{યોગ્ય વાસ્તવિક વિશ્વ ઉદાહરણોનો ઉપયોગ કરીને clustering ના concept નું
વર્ણન કરો.}

\begin{solutionbox}

\textbf{Clustering Concept અને Applications:}

{\def\LTcaptype{none} % do not increment counter
\begin{longtable}[]{@{}
  >{\raggedright\arraybackslash}p{(\linewidth - 4\tabcolsep) * \real{0.3077}}
  >{\raggedright\arraybackslash}p{(\linewidth - 4\tabcolsep) * \real{0.3654}}
  >{\raggedright\arraybackslash}p{(\linewidth - 4\tabcolsep) * \real{0.3269}}@{}}
\toprule\noalign{}
\begin{minipage}[b]{\linewidth}\raggedright
Clustering પ્રકાર
\end{minipage} & \begin{minipage}[b]{\linewidth}\raggedright
વાસ્તવિક જીવનનું ઉદાહરણ
\end{minipage} & \begin{minipage}[b]{\linewidth}\raggedright
બિઝનેસ અસર
\end{minipage} \\
\midrule\noalign{}
\endhead
\bottomrule\noalign{}
\endlastfoot
\textbf{Customer Segmentation} & ખરીદી વર્તન દ્વારા ગ્રાહકોને જૂથબદ્ધ કરવા &
Targeted marketing campaigns \\
\textbf{Image Segmentation} & ગાંઠ શોધવા માટે medical imaging & સુધારેલ
નિદાન accuracy \\
\textbf{Gene Analysis} & સમાન expression સાથે genes ને જૂથબદ્ધ કરવા & દવા
શોધ અને સારવાર \\
\end{longtable}
}

\textbf{Clustering પ્રોસેસ:}

\begin{verbatim}
flowchart LR
    A[Raw Data] {-{-} B[Feature Selection]}
    B {-{-} C[Distance Calculation]}
    C {-{-} D[Cluster Formation]}
    D {-{-} E[Cluster Validation]}
    E {-{-} F[Business Insights]}
\end{verbatim}

\textbf{વિગતવાર ઉદાહરણો:}

\textbf{1. Customer Segmentation:}

\begin{itemize}
\tightlist
\item
  \textbf{ડેટા}: ખરીદીનો ઇતિહાસ, demographics, વેબસાઇટ વર્તન
\item
  \textbf{Clusters}: ઉચ્ચ-મૂલ્યના ગ્રાહકો, કિંમત-સંવેદનશીલ ખરીદદારો, પ્રસંગોપાત
  દુકાનદારો
\item
  \textbf{બિઝનેસ વેલ્યુ}: કસ્ટમાઇઝ્ડ માર્કેટિંગ, ઉત્પાદન સિફારિશો, retention
  વ્યૂહરચના
\end{itemize}

\textbf{2. Social Media Analysis:}

\begin{itemize}
\tightlist
\item
  \textbf{ડેટા}: વપરાશકર્તા interactions, post topics, engagement પેટર્ન
\item
  \textbf{Clusters}: Influencers, casual users, brand advocates
\item
  \textbf{એપ્લિકેશન}: Viral marketing, content વ્યૂહરચના, community
  management
\end{itemize}

\textbf{3. Market Research:}

\begin{itemize}
\tightlist
\item
  \textbf{ડેટા}: Survey responses, ઉત્પાદન પસંદગીઓ, demographics
\item
  \textbf{Clusters}: સમાન જરૂરિયાતો સાથેના માર્કેટ segments
\item
  \textbf{Insights}: ઉત્પાદન વિકાસ, કિંમત વ્યૂહરચના, માર્કેટ positioning
\end{itemize}

\textbf{Clustering Algorithms:}

\begin{itemize}
\tightlist
\item
  \textbf{K-Means}: ડેટાને k clusters માં વિભાજિત કરે છે
\item
  \textbf{Hierarchical}: વૃક્ષ-જેવું cluster structure બનાવે છે
\item
  \textbf{DBSCAN}: વિવિધ ઘનતાના clusters શોધે છે
\end{itemize}

\textbf{Validation પદ્ધતિઓ:}

\begin{itemize}
\tightlist
\item
  \textbf{Silhouette Score}: cluster ગુણવત્તા માપે છે
\item
  \textbf{Elbow Method}: optimal clusters ની સંખ્યા નક્કી કરે છે
\item
  \textbf{Domain Expertise}: બિઝનેસ જ્ઞાન validation
\end{itemize}

\textbf{ફાયદાઓ:}

\begin{itemize}
\tightlist
\item
  \textbf{Pattern Discovery}: છુપાયેલ ડેટા structures જાહેર કરે છે
\item
  \textbf{Decision Support}: બિઝનેસ નિર્ણયો માટે insights પ્રદાન કરે છે
\item
  \textbf{Automation}: manual ડેટા વિશ્લેષણનો પ્રયાસ ઘટાડે છે
\end{itemize}

\end{solutionbox}
\begin{mnemonicbox}
``Clustering Creates Clear Categories'' (CCCC)

\end{mnemonicbox}

\end{document}
