\documentclass[10pt,a4paper]{article}

% content/resources/templates/preamble.tex
\usepackage[margin=0.6in]{geometry}
\author{Milav Dabgar}
\usepackage{amsmath,amssymb,amsthm}
\usepackage{booktabs}
\usepackage{multirow}
\usepackage{xcolor}
\usepackage{tcolorbox}
\tcbuselibrary{breakable,skins}
\usepackage[colorlinks=true,linkcolor=blue]{hyperref}
\usepackage{titlesec}
\usepackage{enumitem}
\usepackage{tikz}
\usepackage{pgfplots}
\usepackage{circuitikz}
\usepackage[version=4]{mhchem}
\usepackage{longtable}
\usepackage{array}
\usepackage{float}
\usepackage{caption}
\usepackage{listings}

\lstset{
  basicstyle=\small\ttfamily,
  breaklines=true,
  breakatwhitespace=false,
  postbreak=\mbox{\textcolor{red}{$\hookrightarrow$}\space},
  float=false,
  numbers=left,
  numberstyle=\tiny\color{gray},
  numbersep=10pt,
  xleftmargin=2em,
  keywordstyle=\color{blue},
  commentstyle=\color{green!60!black},
  stringstyle=\color{purple},
  backgroundcolor=\color{gray!5},
  showstringspaces=false,
  tabsize=2,
  captionpos=b,
  keepspaces=true,
  columns=flexible
}

\pgfplotsset{compat=1.18}
\usetikzlibrary{shapes,arrows,positioning,calc,patterns,decorations.pathmorphing,decorations.markings,arrows.meta}

% Color scheme
\definecolor{headcolor}{RGB}{0,102,204}
\definecolor{keycolor}{RGB}{220,20,60}
\definecolor{solutioncolor}{RGB}{34,139,34}
\definecolor{mnemoniccolor}{RGB}{148,0,211}
\definecolor{codecolor}{RGB}{0,0,100}

% Spacing
\setlength{\parskip}{3pt}
\setlist[itemize]{nosep}
\setlist[enumerate]{nosep}

% Title formatting
\titleformat{\section}{\Large\bfseries\color{headcolor}}{\thesection}{1em}{}
\titleformat{\subsection}{\large\bfseries\color{headcolor}}{\thesubsection}{1em}{}

% Pandoc tightlist compatibility
\providecommand{\tightlist}{%
  \setlength{\itemsep}{0pt}\setlength{\parskip}{0pt}}

% Pandoc longtable compatibility
\newcounter{none}
\def\thenone{}


% content/resources/templates/gujarati-boxes.tex
\usepackage{fontspec}
\usepackage{polyglossia}

% Set Gujarati as main language (document is primarily in Gujarati)
% Note: gloss-gujarati.ldf doesn't exist in polyglossia, but it will use hyphenation patterns
\setdefaultlanguage{gujarati}
\setotherlanguage{english}

% Configure Gujarati font properly
% Use Language=Default to prevent polyglossia from trying to add language-specific features
% that don't exist for Gujarati, which causes "empty feature" warnings
\newfontfamily\gujaratifont[Script=Gujarati,AutoFakeBold=2.5,AutoFakeSlant=0.3]{Noto Sans Gujarati}
\setmainfont[Script=Gujarati,AutoFakeBold=2.5,AutoFakeSlant=0.3]{Noto Sans Gujarati}
% Use Noto Sans Gujarati for monospace to support Gujarati in text
\setmonofont[Scale=0.9]{Noto Sans Gujarati}

% Configure English to use the same font
\newfontfamily\englishfont[Script=Gujarati,AutoFakeBold=2.5,AutoFakeSlant=0.3]{Noto Sans Gujarati}

% Translations for polyglossia
\gappto\captionsgujarati{
  \renewcommand{\tablename}{કોષ્ટક}
  \renewcommand{\figurename}{આકૃતિ}
}

% Helper for TikZ nodes to ensure Gujarati font
\newcommand{\gu}[1]{{\gujaratifont #1}}

% Custom environments
\newtcolorbox{solutionbox}{
    breakable,
    enhanced,
    colback=solutioncolor!5!white,
    colframe=solutioncolor!75!black,
    fonttitle=\bfseries,
    title=જવાબ
}

\newtcolorbox{solutionboxnobreak}{
 colback=solutioncolor!5!white,
 colframe=solutioncolor!75!black,
 fonttitle=\bfseries,
 title=જવાબ
}

\newtcolorbox{keyformula}{
 breakable,
 enhanced,
 colback=keycolor!5!white,
 colframe=keycolor!75!black,
 fonttitle=\bfseries,
 title=રાસાયણિક સમીકરણ/સૂત્ર
}

\newtcolorbox{mnemonicbox}{
 breakable,
 enhanced,
 colback=mnemoniccolor!5!white,
 colframe=mnemoniccolor!75!black,
 fonttitle=\bfseries,
 title=મેમરી ટ્રીક
}


\begin{document}

\begin{center}
{\Huge\bfseries\color{headcolor} Subject Name (Gujarati)}\\[5pt]
{\LARGE 4341603 -- Winter 2024}\\[3pt]
{\large Semester 1 Study Material}\\[3pt]
{\normalsize\textit{Detailed Solutions and Explanations}}
\end{center}

\vspace{10pt}

\subsection*{પ્રશ્ન 1(અ) [3
ગુણ]}\label{uxaaauxab0uxab6uxaa8-1uxa85-3-uxa97uxaa3}

\textbf{હ્યુમન લર્નિંગનું સંક્ષિપ્ત વર્ણન કરો.}

\begin{solutionbox}

હ્યુમન લર્નિંગ એ પ્રક્રિયા છે જેના દ્વારા માનવ અનુભવ, પ્રેક્ટિસ અને શિક્ષણ દ્વારા જ્ઞાન,
કૌશલ્ય અને વર્તણૂક પ્રાપ્ત કરે છે.


{\def\LTcaptype{none} % do not increment counter
\vspace{-5pt}
\captionof{table}{હ્યુમન લર્નિંગ પ્રક્રિયા}
\vspace{-10pt}
\begin{longtable}[]{@{}ll@{}}
\toprule\noalign{}
પાસું & વર્ણન \\
\midrule\noalign{}
\endhead
\bottomrule\noalign{}
\endlastfoot
\textbf{અવલોકન} & પર્યાવરણમાંથી માહિતી એકત્રિત કરવી \\
\textbf{અનુભવ} & ટ્રાયલ અને એરર દ્વારા શીખવું \\
\textbf{અભ્યાસ} & કૌશલ્ય સુધારવા માટે પુનરાવર્તન \\
\textbf{સ્મૃતિ} & માહિતી સંગ્રહ અને પુનઃપ્રાપ્તિ \\
\end{longtable}
}

\begin{itemize}
\tightlist
\item
  \textbf{લર્નિંગ પ્રકારો}: દ્રશ્ય, શ્રાવ્ય, ગતિશીલ લર્નિંગ શૈલીઓ
\item
  \textbf{ફીડબેક લૂપ}: ભૂલો અને સફળતાઓમાંથી શીખવું
\item
  \textbf{અનુકૂલન}: નવી પરિસ્થિતિઓમાં જ્ઞાન લાગુ કરવાની ક્ષમતા
\end{itemize}

\end{solutionbox}
\begin{mnemonicbox}
``AAPS'' - અવલોકન, અનુભવ, અભ્યાસ, સ્મૃતિ

\end{mnemonicbox}
\subsection*{પ્રશ્ન 1(બ) [4
ગુણ]}\label{uxaaauxab0uxab6uxaa8-1uxaac-4-uxa97uxaa3}

\textbf{તફાવત કરો: Supervised લર્નિંગ v/s Unsupervised લર્નિંગ}

\begin{solutionbox}

\textbf{તુલનાત્મક કોષ્ટક: Supervised vs Unsupervised લર્નિંગ}

{\def\LTcaptype{none} % do not increment counter
\begin{longtable}[]{@{}
  >{\raggedright\arraybackslash}p{(\linewidth - 4\tabcolsep) * \real{0.2037}}
  >{\raggedright\arraybackslash}p{(\linewidth - 4\tabcolsep) * \real{0.3889}}
  >{\raggedright\arraybackslash}p{(\linewidth - 4\tabcolsep) * \real{0.4074}}@{}}
\toprule\noalign{}
\begin{minipage}[b]{\linewidth}\raggedright
પેરામીટર
\end{minipage} & \begin{minipage}[b]{\linewidth}\raggedright
Supervised લર્નિંગ
\end{minipage} & \begin{minipage}[b]{\linewidth}\raggedright
Unsupervised લર્નિંગ
\end{minipage} \\
\midrule\noalign{}
\endhead
\bottomrule\noalign{}
\endlastfoot
\textbf{ટ્રેનિંગ ડેટા} & લેબલ થયેલ ડેટા (ઇનપુટ-આઉટપુટ જોડી) & લેબલ વિનાનો ડેટા
(માત્ર ઇનપુટ) \\
\textbf{ધ્યેય} & નવા ઇનપુટ માટે આઉટપુટ આગાહી કરવી & છુપાયેલ પેટર્ન શોધવું \\
\textbf{ઉદાહરણ} & Classification, Regression & Clustering,
Association \\
\textbf{ફીડબેક} & સીધો ફીડબેક ઉપલબ્ધ & કોઈ સીધો ફીડબેક નથી \\
\end{longtable}
}

\begin{itemize}
\tightlist
\item
  \textbf{Supervised}: શિક્ષક સાચા જવાબો સાથે લર્નિંગ માર્ગદર્શન કરે છે
\item
  \textbf{Unsupervised}: માર્ગદર્શન વિના પેટર્નની સ્વ-શોધ
\end{itemize}

\end{solutionbox}
\begin{mnemonicbox}
``SL-લેબલ્સ, UL-અજાણ્યા'' પેટર્ન

\end{mnemonicbox}
\subsection*{પ્રશ્ન 1(ક) [7
ગુણ]}\label{uxaaauxab0uxab6uxaa8-1uxa95-7-uxa97uxaa3}

\textbf{મશીન લર્નિંગ એક્ટિવિટીની સૂચિ બનાવો. દરેકને વિગતવાર સમજાવો.}

\begin{solutionbox}


{\def\LTcaptype{none} % do not increment counter
\vspace{-5pt}
\captionof{table}{મશીન લર્નિંગ એક્ટિવિટીઓ}
\vspace{-10pt}
\begin{longtable}[]{@{}
  >{\raggedright\arraybackslash}p{(\linewidth - 4\tabcolsep) * \real{0.4583}}
  >{\raggedright\arraybackslash}p{(\linewidth - 4\tabcolsep) * \real{0.2500}}
  >{\raggedright\arraybackslash}p{(\linewidth - 4\tabcolsep) * \real{0.2917}}@{}}
\toprule\noalign{}
\begin{minipage}[b]{\linewidth}\raggedright
એક્ટિવિટી
\end{minipage} & \begin{minipage}[b]{\linewidth}\raggedright
હેતુ
\end{minipage} & \begin{minipage}[b]{\linewidth}\raggedright
વર્ણન
\end{minipage} \\
\midrule\noalign{}
\endhead
\bottomrule\noalign{}
\endlastfoot
\textbf{ડેટા કલેક્શન} & કાચો ડેટા એકત્રિત કરવો & વિવિધ સ્રોતોમાંથી સંબંધિત ડેટા
એકત્રિત કરવો \\
\textbf{ડેટા પ્રીપ્રોસેસિંગ} & ડેટા સાફ અને તૈયાર કરવો & ખોવાયેલી વેલ્યૂઝ સંભાળવી,
સામાન્યીકરણ \\
\textbf{ફીચર સિલેક્શન} & મહત્વપૂર્ણ લક્ષણો પસંદ કરવા & લર્નિંગ માટે સંબંધિત એટ્રિબ્યુટ્સ
પસંદ કરવા \\
\textbf{મોડેલ ટ્રેનિંગ} & લર્નિંગ મોડેલ બનાવવું & તૈયાર ડેટાસેટ પર અલગોરિધમ ટ્રેનિંગ \\
\textbf{મોડેલ ઇવેલ્યુએશન} & પરફોર્મન્સ મૂલ્યાંકન & મોડેલની ચોકસાઈ અને અસરકારકતા
ચકાસવી \\
\textbf{મોડેલ ડિપ્લોયમેન્ટ} & મોડેલને ઉપયોગમાં લેવું & વાસ્તવિક દુનિયાની એપ્લિકેશનમાં
મોડેલ અમલીકરણ \\
\end{longtable}
}

\begin{verbatim}
flowchart LR
    A[ડેટા કલેક્શન] {-{-} B[ડેટા પ્રીપ્રોસેસિંગ]}
    B {-{-} C[ફીચર સિલેક્શન]}
    C {-{-} D[મોડેલ ટ્રેનિંગ]}
    D {-{-} E[મોડેલ ઇવેલ્યુએશન]}
    E {-{-} F[મોડેલ ડિપ્લોયમેન્ટ]}
    F {-{-} G[મોડેલ મોનિટરિંગ]}
\end{verbatim}

\begin{itemize}
\tightlist
\item
  \textbf{પુનરાવર્તિત પ્રક્રિયા}: મોડેલ સુધારણા માટે એક્ટિવિટીઓ પુનરાવર્તિત થાય છે
\item
  \textbf{ગુણવત્તા નિયંત્રણ}: દરેક પગલું બહેતર મોડેલ પરફોર્મન્સ સુનિશ્ચિત કરે છે
\end{itemize}

\end{solutionbox}
\begin{mnemonicbox}
``કપફટઇડમ'' - કલેક્શન, પ્રીપ્રોસેસિંગ, ફીચર, ટ્રેનિંગ,
ઇવેલ્યુએશન, ડિપ્લોયમેન્ટ, મોનિટરિંગ

\end{mnemonicbox}
\subsection*{પ્રશ્ન 1(ક OR) [7
ગુણ]}\label{uxaaauxab0uxab6uxaa8-1uxa95-or-7-uxa97uxaa3}

\textbf{નીચેના ડેટા માટે મીન, મીડિયન અને મોડ શોધો: 1, 1, 1, 2, 4, 5, 5, 6,
6, 7, 7, 7, 7, 8, 9, 10, 11}

\begin{solutionbox}

\textbf{ડેટા વિશ્લેષણ કોષ્ટક}

{\def\LTcaptype{none} % do not increment counter
\begin{longtable}[]{@{}
  >{\raggedright\arraybackslash}p{(\linewidth - 6\tabcolsep) * \real{0.3684}}
  >{\raggedright\arraybackslash}p{(\linewidth - 6\tabcolsep) * \real{0.1842}}
  >{\raggedright\arraybackslash}p{(\linewidth - 6\tabcolsep) * \real{0.2368}}
  >{\raggedright\arraybackslash}p{(\linewidth - 6\tabcolsep) * \real{0.2105}}@{}}
\toprule\noalign{}
\begin{minipage}[b]{\linewidth}\raggedright
આંકડાકીય માપ
\end{minipage} & \begin{minipage}[b]{\linewidth}\raggedright
સૂત્ર
\end{minipage} & \begin{minipage}[b]{\linewidth}\raggedright
ગણતરી
\end{minipage} & \begin{minipage}[b]{\linewidth}\raggedright
પરિણામ
\end{minipage} \\
\midrule\noalign{}
\endhead
\bottomrule\noalign{}
\endlastfoot
\textbf{મીન} & સરવાળો/ગણતરી & (1+1+1+2+4+5+5+6+6+7+7+7+7+8+9+10+11)/17 &
5.88 \\
\textbf{મીડિયન} & મધ્ય વેલ્યુ & ક્રમબદ્ધ ડેટામાં 7મી પોઝિશન & 6 \\
\textbf{મોડ} & સૌથી વધુ આવર્તન & 4 વખત દેખાતી વેલ્યુ & 7 \\
\end{longtable}
}

\textbf{પગલું-દર-પગલું ગણતરી:}

\begin{itemize}
\tightlist
\item
  \textbf{કુલ ગણતરી}: 17 વેલ્યુઝ
\item
  \textbf{સરવાળો}: 100
\item
  \textbf{મીન}: 100/17 = 5.88
\item
  \textbf{મીડિયન}: મધ્ય પોઝિશન (9મી) = 6
\item
  \textbf{મોડ}: 7 સૌથી વધુ 4 વખત દેખાય છે
\end{itemize}

\end{solutionbox}
\begin{mnemonicbox}
``મમમ'' - મીન=સરેરાશ, મીડિયન=મધ્ય, મોડ=સૌથી વધુ આવર્તન

\end{mnemonicbox}
\subsection*{પ્રશ્ન 2(અ) [3
ગુણ]}\label{uxaaauxab0uxab6uxaa8-2uxa85-3-uxa97uxaa3}

\textbf{મોડેલ ટ્રેનિંગ માટે હોલ્ડ આઉટ પદ્ધતિનો ઉપયોગ કરવાના પગલાં લખો.}

\begin{solutionbox}

\textbf{હોલ્ડ આઉટ મેથડ પગલાં}

{\def\LTcaptype{none} % do not increment counter
\begin{longtable}[]{@{}
  >{\raggedright\arraybackslash}p{(\linewidth - 4\tabcolsep) * \real{0.3500}}
  >{\raggedright\arraybackslash}p{(\linewidth - 4\tabcolsep) * \real{0.3500}}
  >{\raggedright\arraybackslash}p{(\linewidth - 4\tabcolsep) * \real{0.3000}}@{}}
\toprule\noalign{}
\begin{minipage}[b]{\linewidth}\raggedright
પગલું
\end{minipage} & \begin{minipage}[b]{\linewidth}\raggedright
ક્રિયા
\end{minipage} & \begin{minipage}[b]{\linewidth}\raggedright
હેતુ
\end{minipage} \\
\midrule\noalign{}
\endhead
\bottomrule\noalign{}
\endlastfoot
\textbf{1} & ડેટાસેટ વિભાજન (70-80\% ટ્રેનિંગ, 20-30\% ટેસ્ટિંગ) & ટ્રેનિંગ અને
મૂલ્યાંકન માટે ડેટા અલગ કરવો \\
\textbf{2} & ટ્રેનિંગ સેટ પર મોડેલ ટ્રેન કરવું & લર્નિંગ અલગોરિધમ બનાવવું \\
\textbf{3} & ટેસ્ટિંગ સેટ પર મોડેલ ટેસ્ટ કરવું & મોડેલ પરફોર્મન્સ મૂલ્યાંકન કરવું \\
\end{longtable}
}

\begin{itemize}
\tightlist
\item
  \textbf{રેન્ડમ સ્પ્લિટ}: બંને સેટમાં પ્રતિનિધિ વિતરણ સુનિશ્ચિત કરવું
\item
  \textbf{કોઈ ઓવરલેપ નહીં}: ટેસ્ટિંગ ડેટા ક્યારેય ટ્રેનિંગમાં ઉપયોગ થતો નથી
\item
  \textbf{સિંગલ સ્પ્લિટ}: ડેટાનું એક વખતનું વિભાજન
\end{itemize}

\end{solutionbox}
\begin{mnemonicbox}
``વટટ'' - વિભાજન, ટ્રેન, ટેસ્ટ

\end{mnemonicbox}
\subsection*{પ્રશ્ન 2(બ) [4
ગુણ]}\label{uxaaauxab0uxab6uxaa8-2uxaac-4-uxa97uxaa3}

\textbf{કન્ફ્યુઝન મેટ્રિક્સની રચના સમજાવો.}

\begin{solutionbox}

\textbf{કન્ફ્યુઝન મેટ્રિક્સ રચના}

{\def\LTcaptype{none} % do not increment counter
\begin{longtable}[]{@{}lll@{}}
\toprule\noalign{}
& આગાહી પોઝિટિવ & આગાહી નેગેટિવ \\
\midrule\noalign{}
\endhead
\bottomrule\noalign{}
\endlastfoot
\textbf{વાસ્તવિક પોઝિટિવ} & ટ્રુ પોઝિટિવ (TP) & ફોલ્સ નેગેટિવ (FN) \\
\textbf{વાસ્તવિક નેગેટિવ} & ફોલ્સ પોઝિટિવ (FP) & ટ્રુ નેગેટિવ (TN) \\
\end{longtable}
}

\textbf{ઘટકોની સમજૂતી:}

\begin{itemize}
\tightlist
\item
  \textbf{TP}: સાચી રીતે આગાહી કરેલા પોઝિટિવ કેસ
\item
  \textbf{TN}: સાચી રીતે આગાહી કરેલા નેગેટિવ કેસ
\item
  \textbf{FP}: ખોટી રીતે પોઝિટિવ તરીકે આગાહી (ટાઈપ I એરર)
\item
  \textbf{FN}: ખોટી રીતે નેગેટિવ તરીકે આગાહી (ટાઈપ II એરર)
\end{itemize}

\textbf{પરફોર્મન્સ મેટ્રિક્સ:}

\begin{itemize}
\tightlist
\item
  \textbf{એક્યુરેસી} = (TP+TN)/(TP+TN+FP+FN)
\item
  \textbf{પ્રિસિઝન} = TP/(TP+FP)
\end{itemize}

\end{solutionbox}
\begin{mnemonicbox}
``TPFN-FPTN'' મેટ્રિક્સ પોઝિશન માટે

\end{mnemonicbox}
\subsection*{પ્રશ્ન 2(ક) [7
ગુણ]}\label{uxaaauxab0uxab6uxaa8-2uxa95-7-uxa97uxaa3}

\textbf{ડેટા પ્રી-પ્રોસેસિંગ વ્યાખ્યાયિત કરો. ડેટા પ્રી-પ્રોસેસિંગમાં વપરાતી વિવિધ
પદ્ધતિઓ સમજાવો.}

\begin{solutionbox}

ડેટા પ્રી-પ્રોસેસિંગ એ કાચા ડેટાને સાફ, રૂપાંતરિત અને મશીન લર્નિંગ અલગોરિધમ માટે તૈયાર
કરવાની તકનીક છે.

\textbf{ડેટા પ્રી-પ્રોસેસિંગ પદ્ધતિઓ કોષ્ટક}

{\def\LTcaptype{none} % do not increment counter
\begin{longtable}[]{@{}
  >{\raggedright\arraybackslash}p{(\linewidth - 4\tabcolsep) * \real{0.3478}}
  >{\raggedright\arraybackslash}p{(\linewidth - 4\tabcolsep) * \real{0.2609}}
  >{\raggedright\arraybackslash}p{(\linewidth - 4\tabcolsep) * \real{0.3913}}@{}}
\toprule\noalign{}
\begin{minipage}[b]{\linewidth}\raggedright
પદ્ધતિ
\end{minipage} & \begin{minipage}[b]{\linewidth}\raggedright
હેતુ
\end{minipage} & \begin{minipage}[b]{\linewidth}\raggedright
તકનીકો
\end{minipage} \\
\midrule\noalign{}
\endhead
\bottomrule\noalign{}
\endlastfoot
\textbf{ડેટા ક્લીનિંગ} & નોઈઝ અને અસંગતતા દૂર કરવી & ખોવાયેલી વેલ્યૂઝ સંભાળવી,
ડુપ્લિકેટ દૂર કરવા \\
\textbf{ડેટા ટ્રાન્સફોર્મેશન} & ડેટા ફોર્મેટ કન્વર્ટ કરવું & નોર્મલાઈઝેશન,
સ્ટાન્ડર્ડાઈઝેશન \\
\textbf{ડેટા રિડક્શન} & ડેટાસેટ સાઈઝ ઘટાડવું & ફીચર સિલેક્શન, ડાઈમેન્શનાલિટી
રિડક્શન \\
\textbf{ડેટા ઈન્ટીગ્રેશન} & અનેક સ્રોતો જોડવા & ડેટાસેટ મર્જ કરવા, કોન્ફ્લિક્ટ હલ
કરવા \\
\end{longtable}
}

\begin{verbatim}
flowchart LR
    A[કાચો ડેટા] {-{-} B[ડેટા ક્લીનિંગ]}
    B {-{-} C[ડેટા ટ્રાન્સફોર્મેશન]}
    C {-{-} D[ડેટા રિડક્શન]}
    D {-{-} E[સાફ ડેટા]}
\end{verbatim}

\begin{itemize}
\tightlist
\item
  \textbf{ખોવાયેલી વેલ્યૂઝ}: ઈમ્પ્યુટેશન માટે મીન, મીડિયન, અથવા મોડનો ઉપયોગ
\item
  \textbf{આઉટલાયર્સ}: અત્યંત વેલ્યૂઝ શોધવી અને સંભાળવી
\item
  \textbf{ફીચર સ્કેલિંગ}: ડેટાને સમાન સ્કેલ પર નોર્મલાઈઝ કરવું
\end{itemize}

\end{solutionbox}
\begin{mnemonicbox}
``કતરઈ'' - ક્લીન, ટ્રાન્સફોર્મ, રિડ્યુસ, ઈન્ટીગ્રેટ

\end{mnemonicbox}
\subsection*{પ્રશ્ન 2(અ OR) [3
ગુણ]}\label{uxaaauxab0uxab6uxaa8-2uxa85-or-3-uxa97uxaa3}

\textbf{યોગ્ય ઉદાહરણ સાથે હિસ્ટોગ્રામ સમજાવો.}

\begin{solutionbox}

હિસ્ટોગ્રામ એ અંકશાસ્ત્રીય ડેટાના ફ્રીક્વન્સી ડિસ્ટ્રિબ્યુશનનું ગ્રાફિકલ પ્રતિનિધિત્વ છે જે
ડેટાને bins માં વિભાજિત કરે છે.

\textbf{હિસ્ટોગ્રામ ઘટકો કોષ્ટક}

{\def\LTcaptype{none} % do not increment counter
\begin{longtable}[]{@{}ll@{}}
\toprule\noalign{}
ઘટક & વર્ણન \\
\midrule\noalign{}
\endhead
\bottomrule\noalign{}
\endlastfoot
\textbf{X-axis} & ડેટા રેન્જ (bins) \\
\textbf{Y-axis} & આવર્તન \\
\textbf{બાર્સ} & ઊંચાઈ આવર્તન દર્શાવે છે \\
\end{longtable}
}

\textbf{ઉદાહરણ}: વિદ્યાર્થીઓના ગુણ વિતરણ:

\begin{itemize}
\tightlist
\item
  Bins: 0-20, 21-40, 41-60, 61-80, 81-100
\item
  ઊંચાઈ દરેક રેન્જમાં વિદ્યાર્થીઓની સંખ્યા દર્શાવે છે
\end{itemize}

\end{solutionbox}
\begin{mnemonicbox}
``બએર'' - Bins, Axes, રેન્જ

\end{mnemonicbox}
\subsection*{પ્રશ્ન 2(બ OR) [4
ગુણ]}\label{uxaaauxab0uxab6uxaa8-2uxaac-or-4-uxa97uxaa3}

\textbf{નીચેના ઉદાહરણોનો યોગ્ય ડેટા પ્રકાર જણાવો:} \textbf{i) વ્યક્તિનું લિંગ ii)
વિદ્યાર્થીઓનો ક્રમ iii) ઘરની કિંમત iv) ફૂલનો રંગ}

\begin{solutionbox}

\textbf{ડેટા પ્રકાર વર્ગીકરણ કોષ્ટક}

{\def\LTcaptype{none} % do not increment counter
\begin{longtable}[]{@{}
  >{\raggedright\arraybackslash}p{(\linewidth - 4\tabcolsep) * \real{0.2903}}
  >{\raggedright\arraybackslash}p{(\linewidth - 4\tabcolsep) * \real{0.4194}}
  >{\raggedright\arraybackslash}p{(\linewidth - 4\tabcolsep) * \real{0.2903}}@{}}
\toprule\noalign{}
\begin{minipage}[b]{\linewidth}\raggedright
ઉદાહરણ
\end{minipage} & \begin{minipage}[b]{\linewidth}\raggedright
ડેટા પ્રકાર
\end{minipage} & \begin{minipage}[b]{\linewidth}\raggedright
લક્ષણો
\end{minipage} \\
\midrule\noalign{}
\endhead
\bottomrule\noalign{}
\endlastfoot
\textbf{વ્યક્તિનું લિંગ} & Nominal Categorical & કોઈ પ્રાકૃતિક ક્રમ નથી
(પુરુષ/સ્ત્રી) \\
\textbf{વિદ્યાર્થીઓનો ક્રમ} & Ordinal Categorical & અર્થપૂર્ણ ક્રમ છે (1લો, 2જો,
3જો) \\
\textbf{ઘરની કિંમત} & Continuous Numerical & રેન્જમાં કોઈપણ વેલ્યુ લઈ શકે છે \\
\textbf{ફૂલનો રંગ} & Nominal Categorical & કોઈ પ્રાકૃતિક ક્રમ નથી (લાલ,
વાદળી, પીળો) \\
\end{longtable}
}

\begin{itemize}
\tightlist
\item
  \textbf{કેટેગોરિકલ ડેટા}: વિશિષ્ટ શ્રેણીઓનો મર્યાદિત સેટ
\item
  \textbf{ન્યુમેરિકલ ડેટા}: ગાણિતિક ઓપરેશન શક્ય છે
\item
  \textbf{ઓર્ડિનલ}: અર્થપૂર્ણ અનુક્રમ સાથેની શ્રેણીઓ
\end{itemize}

\end{solutionbox}
\begin{mnemonicbox}
``નોકો'' - Nominal, Ordinal, કન્ટિન્યુઅસ

\end{mnemonicbox}
\subsection*{પ્રશ્ન 2(ક OR) [7
ગુણ]}\label{uxaaauxab0uxab6uxaa8-2uxa95-or-7-uxa97uxaa3}

\textbf{K-fold ક્રોસ વેલિડેશનનું વિગતવાર વર્ણન કરો.}

\begin{solutionbox}

K-fold ક્રોસ વેલિડેશન એ મોડેલ મૂલ્યાંકન તકનીક છે જે મજબૂત પરફોર્મન્સ આકલન માટે ડેટાસેટને
K સમાન ભાગોમાં વિભાજિત કરે છે.

\textbf{K-fold પ્રક્રિયા કોષ્ટક}

{\def\LTcaptype{none} % do not increment counter
\begin{longtable}[]{@{}lll@{}}
\toprule\noalign{}
પગલું & ક્રિયા & હેતુ \\
\midrule\noalign{}
\endhead
\bottomrule\noalign{}
\endlastfoot
\textbf{1} & ડેટાને K સમાન folds માં વિભાજિત કરવો & K સબસેટ્સ બનાવવા \\
\textbf{2} & K-1 folds નો ટ્રેનિંગ માટે ઉપયોગ & મોડેલ ટ્રેન કરવું \\
\textbf{3} & 1 fold નો ટેસ્ટિંગ માટે ઉપયોગ & પરફોર્મન્સ મૂલ્યાંકન \\
\textbf{4} & K વખત પુનરાવર્તન & દરેક fold એક વખત ટેસ્ટ સેટ તરીકે સેવા આપે \\
\textbf{5} & બધા પરિણામોની સરેરાશ & અંતિમ પરફોર્મન્સ મેટ્રિક મેળવવું \\
\end{longtable}
}

\begin{verbatim}
flowchart LR
    A[મૂળ ડેટાસેટ] {-{-} B[K folds માં વિભાજન]}
    B {-{-} C[પુનરાવર્તન 1: folds 2{-}K પર ટ્રેન, fold 1 પર ટેસ્ટ]}
    C {-{-} D[પુનરાવર્તન 2: folds 1,3{-}K પર ટ્રેન, fold 2 પર ટેસ્ટ]}
    D {-{-} E[... K પુનરાવર્તનો માટે ચાલુ રાખો]}
    E {-{-} F[બધા K પરિણામોની સરેરાશ]}
\end{verbatim}

\textbf{ફાયદા:}

\begin{itemize}
\tightlist
\item
  \textbf{મજબૂત મૂલ્યાંકન}: દરેક ડેટા પોઇન્ટ ટ્રેનિંગ અને ટેસ્ટિંગ બંને માટે ઉપયોગ થાય છે
\item
  \textbf{ઓવરફિટિંગ ઘટાડવું}: બહુવિધ વેલિડેશન રાઉન્ડ
\item
  \textbf{બહેતર જનરલાઈઝેશન}: વધુ વિશ્વસનીય પરફોર્મન્સ અંદાજ
\end{itemize}

\textbf{સામાન્ય વેલ્યૂઝ}: સામાન્ય રીતે K=5 અથવા K=10 વપરાય છે

\end{solutionbox}
\begin{mnemonicbox}
``વઉપસટ'' - વિભાજન, ઉપયોગ, પુનરાવર્તન, સરેરાશ, ટેસ્ટ

\end{mnemonicbox}
\subsection*{પ્રશ્ન 3(અ) [3
ગુણ]}\label{uxaaauxab0uxab6uxaa8-3uxa85-3-uxa97uxaa3}

\textbf{રીગ્રેશનની એપ્લિકેશનની યાદી બનાવો.}

\begin{solutionbox}

\textbf{રીગ્રેશન એપ્લિકેશન કોષ્ટક}

{\def\LTcaptype{none} % do not increment counter
\begin{longtable}[]{@{}lll@{}}
\toprule\noalign{}
ડોમેન & એપ્લિકેશન & હેતુ \\
\midrule\noalign{}
\endhead
\bottomrule\noalign{}
\endlastfoot
\textbf{નાણાં} & શેર કિંમત આગાહી & બજાર ટ્રેન્ડ્સ આગાહી કરવી \\
\textbf{હેલ્થકેર} & દવાની માત્રા ગણતરી & શ્રેષ્ઠ સારવાર નક્કી કરવી \\
\textbf{માર્કેટિંગ} & વેચાણ આગાહી & આવક આગાહી કરવી \\
\textbf{રિયલ એસ્ટેટ} & પ્રોપર્ટી વેલ્યુએશન & ઘરની કિંમત અંદાજ કરવો \\
\end{longtable}
}

\begin{itemize}
\tightlist
\item
  \textbf{પ્રિડિક્ટિવ મોડલિંગ}: કન્ટિન્યુઅસ વેલ્યૂઝ આગાહી કરવી
\item
  \textbf{ટ્રેન્ડ એનાલિસિસ}: વેરિએબલ્સ વચ્ચેના સંબંધોને સમજવા
\item
  \textbf{રિસ્ક એસેસમેન્ટ}: ભાવિ પરિણામોનું મૂલ્યાંકન
\end{itemize}

\end{solutionbox}
\begin{mnemonicbox}
``નહમર'' - નાણાં, હેલ્થકેર, માર્કેટિંગ, રિયલ એસ્ટેટ

\end{mnemonicbox}
\subsection*{પ્રશ્ન 3(બ) [4
ગુણ]}\label{uxaaauxab0uxab6uxaa8-3uxaac-4-uxa97uxaa3}

\textbf{સિંગલ લિનિયર રીગ્રેશન પર ટૂંકી નોંધ લખો.}

\begin{solutionbox}

સિંગલ લિનિયર રીગ્રેશન એક સ્વતંત્ર વેરિએબલ (X) અને એક આશ્રિત વેરિએબલ (Y) વચ્ચેના સંબંધને
સીધી રેખાનો ઉપયોગ કરીને મોડેલ કરે છે.

\textbf{લિનિયર રીગ્રેશન ઘટકો}

{\def\LTcaptype{none} % do not increment counter
\begin{longtable}[]{@{}lll@{}}
\toprule\noalign{}
ઘટક & સૂત્ર & વર્ણન \\
\midrule\noalign{}
\endhead
\bottomrule\noalign{}
\endlastfoot
\textbf{સમીકરણ} & Y = a + bX & રેખીય સંબંધ \\
\textbf{સ્લોપ (b)} & Y માં ફેરફાર / X માં ફેરફાર & ફેરફારની દર \\
\textbf{ઇન્ટરસેપ્ટ (a)} & X=0 વખતે Y-વેલ્યુ & શરુઆતી બિંદુ \\
\textbf{એરર} & વાસ્તવિક - આગાહી & રેખામાંથી તફાવત \\
\end{longtable}
}

\begin{itemize}
\tightlist
\item
  \textbf{ધ્યેય}: એરર ઘટાડતી બેસ્ટ-ફિટ લાઇન શોધવી
\item
  \textbf{પદ્ધતિ}: લીસ્ટ સ્ક્વેર ઓપ્ટિમાઇઝેશન
\item
  \textbf{ધારણા}: વેરિએબલ્સ વચ્ચે રેખીય સંબંધ અસ્તિત્વમાં છે
\end{itemize}

\end{solutionbox}
\begin{mnemonicbox}
``YABX'' - Y બરાબર a પ્લસ b ગુણા X

\end{mnemonicbox}
\subsection*{પ્રશ્ન 3(ક) [7
ગુણ]}\label{uxaaauxab0uxab6uxaa8-3uxa95-7-uxa97uxaa3}

\textbf{K-NN અલગોરિધમ લખો અને ચર્ચા કરો.}

\begin{solutionbox}

K-નીયરેસ્ટ નેઇબર્સ (K-NN) એ લેઝી લર્નિંગ અલગોરિધમ છે જે ડેટા પોઇન્ટ્સને તેમના K નજીકના
પડોશીઓના મેજોરિટી ક્લાસના આધારે વર્ગીકૃત કરે છે.

\textbf{K-NN અલગોરિધમ પગલાં}

{\def\LTcaptype{none} % do not increment counter
\begin{longtable}[]{@{}lll@{}}
\toprule\noalign{}
પગલું & ક્રિયા & વર્ણન \\
\midrule\noalign{}
\endhead
\bottomrule\noalign{}
\endlastfoot
\textbf{1} & K વેલ્યુ પસંદ કરવી & પડોશીઓની સંખ્યા પસંદ કરવી \\
\textbf{2} & અંતર ગણતરી કરવી & બધા ટ્રેનિંગ પોઇન્ટ્સનું અંતર શોધવું \\
\textbf{3} & અંતર ક્રમાંકિત કરવા & ચડતા ક્રમમાં ગોઠવવા \\
\textbf{4} & K નજીકના પસંદ કરવા & K સૌથી નજીકના પોઇન્ટ્સ પસંદ કરવા \\
\textbf{5} & મેજોરિટી વોટિંગ & સૌથી સામાન્ય ક્લાસ અસાઇન કરવો \\
\end{longtable}
}

\begin{verbatim}
flowchart LR
    A[નવો ડેટા પોઇન્ટ] {-{-} B[બધા ટ્રેનિંગ પોઇન્ટ્સનું અંતર ગણતરી]}
    B {-{-} C[અંતર ક્રમાંકિત કરવા]}
    C {-{-} D[K નજીકના પડોશીઓ પસંદ કરવા]}
    D {-{-} E[મેજોરિટી વોટ]}
    E {-{-} F[ક્લાસ લેબલ અસાઇન કરવો]}
\end{verbatim}

\textbf{અંતર મેટ્રિક્સ:}

\begin{itemize}
\tightlist
\item
  \textbf{યુક્લિડિયન}: સૌથી સામાન્ય અંતર માપ
\item
  \textbf{મેનહેટન}: નિરપેક્ષ તફાવતોનો સરવાળો
\item
  \textbf{મિન્કોવસ્કી}: સામાન્યીકૃત અંતર મેટ્રિક
\end{itemize}

\textbf{ફાયદા:}

\begin{itemize}
\tightlist
\item
  \textbf{સરળ}: સમજવા અને અમલીકરણ માટે સરળ
\item
  \textbf{કોઈ ટ્રેનિંગ નહીં}: બધો ડેટા સ્ટોર કરે છે, કોઈ મોડેલ બિલ્ડિંગ નથી
\end{itemize}

\textbf{ગેરફાયદા:}

\begin{itemize}
\tightlist
\item
  \textbf{કોમ્પ્યુટેશનલી મહેંગું}: બધા પોઇન્ટ્સ ચેક કરવા પડે છે
\item
  \textbf{K પ્રત્યે સંવેદનશીલ}: પરફોર્મન્સ K વેલ્યુ પર આધાર રાખે છે
\end{itemize}

\end{solutionbox}
\begin{mnemonicbox}
``પગકમ'' - પસંદ, ગણતરી, ક્રમાંકન, મેજોરિટી વોટ

\end{mnemonicbox}
\subsection*{પ્રશ્ન 3(અ OR) [3
ગુણ]}\label{uxaaauxab0uxab6uxaa8-3uxa85-or-3-uxa97uxaa3}

\textbf{હેલ્થકેર ક્ષેત્રમાં supervised learning ના કોઈપણ ત્રણ ઉદાહરણો લખો}

\begin{solutionbox}

\textbf{હેલ્થકેર Supervised લર્નિંગ ઉદાહરણો}

{\def\LTcaptype{none} % do not increment counter
\begin{longtable}[]{@{}
  >{\raggedright\arraybackslash}p{(\linewidth - 6\tabcolsep) * \real{0.3824}}
  >{\raggedright\arraybackslash}p{(\linewidth - 6\tabcolsep) * \real{0.2059}}
  >{\raggedright\arraybackslash}p{(\linewidth - 6\tabcolsep) * \real{0.2353}}
  >{\raggedright\arraybackslash}p{(\linewidth - 6\tabcolsep) * \real{0.1765}}@{}}
\toprule\noalign{}
\begin{minipage}[b]{\linewidth}\raggedright
એપ્લિકેશન
\end{minipage} & \begin{minipage}[b]{\linewidth}\raggedright
ઇનપુટ
\end{minipage} & \begin{minipage}[b]{\linewidth}\raggedright
આઉટપુટ
\end{minipage} & \begin{minipage}[b]{\linewidth}\raggedright
હેતુ
\end{minipage} \\
\midrule\noalign{}
\endhead
\bottomrule\noalign{}
\endlastfoot
\textbf{રોગ નિદાન} & લક્ષણો, ટેસ્ટ પરિણામો & રોગનો પ્રકાર & તબીબી સ્થિતિઓ
ઓળખવી \\
\textbf{દવાની રિસ્પોન્સ આગાહી} & દર્દીનો ડેટા, આનુવંશિકતા & દવાની અસરકારકતા &
વ્યક્તિગત દવા \\
\textbf{મેડિકલ ઇમેજ એનાલિસિસ} & X-rays, MRI સ્કેન & ટ્યુમર શોધ & પ્રારંભિક રોગ
શોધ \\
\end{longtable}
}

\begin{itemize}
\tightlist
\item
  \textbf{પેટર્ન રેકગ્નિશન}: લેબલ કરેલા તબીબી ડેટામાંથી શીખવું
\item
  \textbf{ક્લિનિકલ ડિસિઝન સપોર્ટ}: ડોકટરોને નિદાનમાં મદદ કરવી
\item
  \textbf{પ્રિડિક્ટિવ મેડિસિન}: આરોગ્ય પરિણામો આગાહી કરવા
\end{itemize}

\end{solutionbox}
\begin{mnemonicbox}
``રદમ'' - રોગ નિદાન, દવાની રિસ્પોન્સ, મેડિકલ ઇમેજિંગ

\end{mnemonicbox}
\subsection*{પ્રશ્ન 3(બ OR) [4
ગુણ]}\label{uxaaauxab0uxab6uxaa8-3uxaac-or-4-uxa97uxaa3}

\textbf{તફાવત કરો: Classification v/s Regression.}

\begin{solutionbox}

\textbf{Classification vs Regression તુલના}

{\def\LTcaptype{none} % do not increment counter
\begin{longtable}[]{@{}lll@{}}
\toprule\noalign{}
પાસું & Classification & Regression \\
\midrule\noalign{}
\endhead
\bottomrule\noalign{}
\endlastfoot
\textbf{આઉટપુટ પ્રકાર} & વિશિષ્ટ શ્રેણીઓ/ક્લાસ & કન્ટિન્યુઅસ ન્યુમેરિકલ વેલ્યૂઝ \\
\textbf{ધ્યેય} & ક્લાસ લેબલ આગાહી કરવા & ન્યુમેરિકલ વેલ્યૂઝ આગાહી કરવી \\
\textbf{ઉદાહરણ} & ઇમેઇલ સ્પામ/ન સ્પામ & ઘરની કિંમત આગાહી \\
\textbf{મૂલ્યાંકન} & એક્યુરેસી, પ્રિસિઝન, રિકોલ & MAE, MSE, R-squared \\
\end{longtable}
}

\begin{itemize}
\tightlist
\item
  \textbf{Classification}: શ્રેણીઓ આગાહી કરે છે (હા/ના, લાલ/વાદળી/લીલો)
\item
  \textbf{Regression}: માત્રાઓ આગાહી કરે છે (કિંમત, તાપમાન, વજન)
\item
  \textbf{અલગોરિધમ}: કેટલાક બંને માટે કામ કરે છે, અન્ય વિશેષીકૃત છે
\end{itemize}

\end{solutionbox}
\begin{mnemonicbox}
``CLASS-શ્રેણીઓ, REG-વાસ્તવિક સંખ્યાઓ''

\end{mnemonicbox}
\subsection*{પ્રશ્ન 3(ક OR) [7
ગુણ]}\label{uxaaauxab0uxab6uxaa8-3uxa95-or-7-uxa97uxaa3}

\textbf{ક્લાસિફિકેશન લર્નિંગના સ્ટેપ્સને વિગતમાં સમજાવો.}

\begin{solutionbox}

ક્લાસિફિકેશન લર્નિંગમાં ઇનપુટ ડેટાને પૂર્વનિર્ધારિત શ્રેણીઓ અથવા ક્લાસમાં અસાઇન કરવા
માટે મોડેલ ટ્રેનિંગ શામેલ છે.

\textbf{ક્લાસિફિકેશન લર્નિંગ પગલાં}

{\def\LTcaptype{none} % do not increment counter
\begin{longtable}[]{@{}lll@{}}
\toprule\noalign{}
પગલું & પ્રક્રિયા & વર્ણન \\
\midrule\noalign{}
\endhead
\bottomrule\noalign{}
\endlastfoot
\textbf{1} & ડેટા કલેક્શન & લેબલ કરેલા ટ્રેનિંગ ઉદાહરણો એકત્રિત કરવા \\
\textbf{2} & ડેટા પ્રીપ્રોસેસિંગ & ડેટા સાફ અને તૈયાર કરવો \\
\textbf{3} & ફીચર સિલેક્શન & સંબંધિત એટ્રિબ્યુટ્સ પસંદ કરવા \\
\textbf{4} & મોડેલ સિલેક્શન & ક્લાસિફિકેશન અલગોરિધમ પસંદ કરવું \\
\textbf{5} & ટ્રેનિંગ & લેબલ કરેલા ડેટામાંથી શીખવું \\
\textbf{6} & મૂલ્યાંકન & મોડેલ પરફોર્મન્સ ટેસ્ટ કરવું \\
\textbf{7} & ડિપ્લોયમેન્ટ & આગાહી માટે મોડેલનો ઉપયોગ કરવો \\
\end{longtable}
}

\begin{verbatim}
flowchart LR
    A[લેબલ કરેલો ટ્રેનિંગ ડેટા] {-{-} B[પ્રીપ્રોસેસિંગ]}
    B {-{-} C[ફીચર સિલેક્શન]}
    C {-{-} D[અલગોરિધમ પસંદ કરવું]}
    D {-{-} E[મોડેલ ટ્રેન કરવું]}
    E {-{-} F[પરફોર્મન્સ મૂલ્યાંકન]}
    F {-{-} G\{સારી પરફોર્મન્સ?\}}
    G {-{-}|ના| D}
    G {-{-}|હા| H[મોડેલ ડિપ્લોય કરવું]}
\end{verbatim}

\textbf{મુખ્ય કન્સેપ્ટ્સ:}

\begin{itemize}
\tightlist
\item
  \textbf{Supervised લર્નિંગ}: લેબલ કરેલા ટ્રેનિંગ ડેટાની જરૂર છે
\item
  \textbf{ફીચર એન્જિનિયરિંગ}: કાચા ડેટાને ઉપયોગી ફીચર્સમાં રૂપાંતરિત કરવું
\item
  \textbf{ક્રોસ-વેલિડેશન}: મોડેલ સારી રીતે જનરલાઇઝ કરે છે તે સુનિશ્ચિત કરવું
\item
  \textbf{પરફોર્મન્સ મેટ્રિક્સ}: એક્યુરેસી, પ્રિસિઝન, રિકોલ, F1-સ્કોર
\end{itemize}

\textbf{સામાન્ય અલગોરિધમ:}

\begin{itemize}
\tightlist
\item
  \textbf{ડિસિઝન ટ્રી}: વ્યાખ્યા કરવા સરળ નિયમો
\item
  \textbf{SVM}: હાઇ-ડાઇમેન્શનલ ડેટા માટે અસરકારક
\item
  \textbf{ન્યુરલ નેટવર્ક}: જટિલ પેટર્ન સંભાળે છે
\end{itemize}

\end{solutionbox}
\begin{mnemonicbox}
``ડપફમટમડ'' - ડેટા, પ્રીપ્રોસેસ, ફીચર, મોડેલ, ટ્રેન,
મૂલ્યાંકન, ડિપ્લોય

\end{mnemonicbox}
\subsection*{પ્રશ્ન 4(અ) [3
ગુણ]}\label{uxaaauxab0uxab6uxaa8-4uxa85-3-uxa97uxaa3}

\textbf{તફાવત કરો: Clustering v/s Classification.}

\begin{solutionbox}

\textbf{Clustering vs Classification તુલના}

{\def\LTcaptype{none} % do not increment counter
\begin{longtable}[]{@{}lll@{}}
\toprule\noalign{}
પાસું & Clustering & Classification \\
\midrule\noalign{}
\endhead
\bottomrule\noalign{}
\endlastfoot
\textbf{લર્નિંગ પ્રકાર} & Unsupervised & Supervised \\
\textbf{ટ્રેનિંગ ડેટા} & લેબલ વિનાનો ડેટા & લેબલ કરેલો ડેટા \\
\textbf{ધ્યેય} & છુપાયેલા જૂથો શોધવા & જાણીતા ક્લાસ આગાહી કરવા \\
\textbf{આઉટપુટ} & જૂથ અસાઇનમેન્ટ & ક્લાસ આગાહીઓ \\
\end{longtable}
}

\begin{itemize}
\tightlist
\item
  \textbf{Clustering}: ડેટામાં અજાણ્યા પેટર્ન શોધે છે
\item
  \textbf{Classification}: નવા ઉદાહરણો આગાહી કરવા માટે જાણીતા ઉદાહરણોમાંથી
  શીખે છે
\item
  \textbf{મૂલ્યાંકન}: Clustering નું મૂલ્યાંકન classification કરતાં મુશ્કેલ છે
\end{itemize}

\end{solutionbox}
\begin{mnemonicbox}
``CL-અજાણ્યા જૂથો, CLASS-જાણીતી શ્રેણીઓ''

\end{mnemonicbox}
\subsection*{પ્રશ્ન 4(બ) [4
ગુણ]}\label{uxaaauxab0uxab6uxaa8-4uxaac-4-uxa97uxaa3}

\textbf{Apriori અલગોરિધમના ફાયદા અને ગેરફાયદાની યાદી બનાવો.}

\begin{solutionbox}

\textbf{Apriori અલગોરિધમના ફાયદા અને ગેરફાયદા}

{\def\LTcaptype{none} % do not increment counter
\begin{longtable}[]{@{}ll@{}}
\toprule\noalign{}
ફાયદા & ગેરફાયદા \\
\midrule\noalign{}
\endhead
\bottomrule\noalign{}
\endlastfoot
\textbf{સમજવામાં સરળ} & \textbf{કોમ્પ્યુટેશનલી મહેંગું} \\
\textbf{બધા ફ્રીક્વન્ટ આઇટમસેટ્સ શોધે છે} & \textbf{બહુવિધ ડેટાબેસ સ્કેન} \\
\textbf{સ્થાપિત અલગોરિધમ} & \textbf{મોટી મેમરી જરૂરિયાતો} \\
\textbf{એસોસિએશન રૂલ્સ જનરેટ કરે છે} & \textbf{નબળી સ્કેલેબિલિટી} \\
\end{longtable}
}

\textbf{ફાયદાની વિગતો:}

\begin{itemize}
\tightlist
\item
  \textbf{સરળતા}: સીધું તર્ક અને અમલીકરણ
\item
  \textbf{સંપૂર્ણતા}: બધા ફ્રીક્વન્ટ પેટર્ન શોધે છે
\item
  \textbf{રૂલ જનરેશન}: અર્થપૂર્ણ એસોસિએશન રૂલ્સ બનાવે છે
\end{itemize}

\textbf{ગેરફાયદાની વિગતો:}

\begin{itemize}
\tightlist
\item
  \textbf{પરફોર્મન્સ}: મોટા ડેટાસેટ પર ધીમું
\item
  \textbf{મેમરી}: ઘણા કેન્ડિડેટ આઇટમસેટ્સ સ્ટોર કરે છે
\item
  \textbf{સ્કેલેબિલિટી}: ડેટાના કદ સાથે પરફોર્મન્સ ઘટે છે
\end{itemize}

\end{solutionbox}
\begin{mnemonicbox}
``સરળ-ધીમું'' - ઉપયોગમાં સરળ પણ ધીમી પરફોર્મન્સ

\end{mnemonicbox}
\subsection*{પ્રશ્ન 4(ક) [7
ગુણ]}\label{uxaaauxab0uxab6uxaa8-4uxa95-7-uxa97uxaa3}

\textbf{unsupervised લર્નિંગની એપ્લિકેશનો લખો અને સમજાવો}

\begin{solutionbox}

Unsupervised લર્નિંગ લેબલ કરેલા ઉદાહરણો વિના ડેટામાં છુપાયેલા પેટર્ન શોધે છે.

\textbf{Unsupervised લર્નિંગ એપ્લિકેશન}

{\def\LTcaptype{none} % do not increment counter
\begin{longtable}[]{@{}llll@{}}
\toprule\noalign{}
ડોમેન & એપ્લિકેશન & તકનીક & હેતુ \\
\midrule\noalign{}
\endhead
\bottomrule\noalign{}
\endlastfoot
\textbf{માર્કેટિંગ} & કસ્ટમર સેગમેન્ટેશન & Clustering & સમાન કસ્ટમર્સને જૂથ
બનાવવા \\
\textbf{રિટેઇલ} & માર્કેટ બાસ્કેટ એનાલિસિસ & એસોસિએશન રૂલ્સ & ખરીદીના પેટર્ન
શોધવા \\
\textbf{એનોમેલી ડિટેક્શન} & ફ્રોડ ડિટેક્શન & આઉટલાયર ડિટેક્શન & અસામાન્ય વર્તન
ઓળખવું \\
\textbf{ડેટા કોમ્પ્રેશન} & ડાઇમેન્શનેલિટી રિડક્શન & PCA & ડેટાનું કદ ઘટાડવું \\
\textbf{રેકમેન્ડેશન} & કન્ટેન્ટ ફિલ્ટરિંગ & Clustering & સમાન આઇટમ્સ સૂચવવા \\
\end{longtable}
}

\begin{verbatim}
mindmap
  root((Unsupervised લર્નિંગ))
    Clustering
      કસ્ટમર સેગમેન્ટેશન
      ઇમેજ સેગમેન્ટેશન
      જીન સિક્વન્સિંગ
    એસોસિએશન રૂલ્સ
      માર્કેટ બાસ્કેટ એનાલિસિસ
      વેબ યુઝેજ માઇનિંગ
      પ્રોટીન સિક્વન્સ
    એનોમેલી ડિટેક્શન
      ફ્રોડ ડિટેક્શન
      નેટવર્ક સિક્યોરિટી
      ક્વોલિટી કન્ટ્રોલ
    ડાઇમેન્શનેલિટી રિડક્શન
      ડેટા વિઝ્યુઅલાઇઝેશન
      ફીચર એક્સટ્રેક્શન
      ડેટા કોમ્પ્રેશન
\end{verbatim}

\textbf{મુખ્ય ફાયદા:}

\begin{itemize}
\tightlist
\item
  \textbf{પેટર્ન ડિસ્કવરી}: છુપાયેલી સ્ટ્રક્ચર્સ બહાર કાઢે છે
\item
  \textbf{લેબલ્સની જરૂર નથી}: કાચા ડેટા સાથે કામ કરે છે
\item
  \textbf{એક્સપ્લોરેટરી એનાલિસિસ}: ડેટાની લાક્ષણિકતાઓ સમજવી
\end{itemize}

\textbf{સામાન્ય તકનીકો:}

\begin{itemize}
\tightlist
\item
  \textbf{K-means}: ડેટાને ક્લસ્ટરમાં વિભાજિત કરે છે
\item
  \textbf{હાયરાર્કિકલ ક્લસ્ટરિંગ}: ક્લસ્ટર હાયરાર્કી બનાવે છે
\item
  \textbf{Apriori}: એસોસિએશન રૂલ્સ શોધે છે
\end{itemize}

\end{solutionbox}
\begin{mnemonicbox}
``મરએડ'' - માર્કેટિંગ, રિટેઇલ, એનોમેલી, ડાઇમેન્શનેલિટી

\end{mnemonicbox}
\subsection*{પ્રશ્ન 4(અ OR) [3
ગુણ]}\label{uxaaauxab0uxab6uxaa8-4uxa85-or-3-uxa97uxaa3}

\textbf{Apriori અલગોરિધમની એપ્લિકેશનની યાદી બનાવો.}

\begin{solutionbox}

\textbf{Apriori અલગોરિધમ એપ્લિકેશન}

{\def\LTcaptype{none} % do not increment counter
\begin{longtable}[]{@{}lll@{}}
\toprule\noalign{}
ડોમેન & એપ્લિકેશન & હેતુ \\
\midrule\noalign{}
\endhead
\bottomrule\noalign{}
\endlastfoot
\textbf{રિટેઇલ} & માર્કેટ બાસ્કેટ એનાલિસિસ & એકસાથે ખરીદાતા આઇટમ્સ શોધવા \\
\textbf{વેબ માઇનિંગ} & વેબસાઇટ ઉપયોગ પેટર્ન & પેજ વિઝિટ સિક્વન્સ શોધવા \\
\textbf{બાયોઇન્ફોર્મેટિક્સ} & જીન પેટર્ન એનાલિસિસ & જીન એસોસિએશન ઓળખવા \\
\end{longtable}
}

\begin{itemize}
\tightlist
\item
  \textbf{એસોસિએશન રૂલ્સ}: ``જો A તો B'' સંબંધો
\item
  \textbf{ફ્રીક્વન્ટ પેટર્ન}: વારંવાર એકસાથે દેખાતા આઇટમ્સ
\item
  \textbf{ક્રોસ-સેલિંગ}: સંબંધિત પ્રોડક્ટ્સ રેકમેન્ડ કરવા
\end{itemize}

\end{solutionbox}
\begin{mnemonicbox}
``રવબ'' - રિટેઇલ, વેબ, બાયોઇન્ફોર્મેટિક્સ

\end{mnemonicbox}
\subsection*{પ્રશ્ન 4(બ OR) [4
ગુણ]}\label{uxaaauxab0uxab6uxaa8-4uxaac-or-4-uxa97uxaa3}

\textbf{વ્યાખ્યાયિત કરો: Support and Confidence.}

\begin{solutionbox}

\textbf{એસોસિએશન રૂલ મેટ્રિક્સ}

{\def\LTcaptype{none} % do not increment counter
\begin{longtable}[]{@{}
  >{\raggedright\arraybackslash}p{(\linewidth - 6\tabcolsep) * \real{0.2903}}
  >{\raggedright\arraybackslash}p{(\linewidth - 6\tabcolsep) * \real{0.2258}}
  >{\raggedright\arraybackslash}p{(\linewidth - 6\tabcolsep) * \real{0.2581}}
  >{\raggedright\arraybackslash}p{(\linewidth - 6\tabcolsep) * \real{0.2258}}@{}}
\toprule\noalign{}
\begin{minipage}[b]{\linewidth}\raggedright
મેટ્રિક
\end{minipage} & \begin{minipage}[b]{\linewidth}\raggedright
સૂત્ર
\end{minipage} & \begin{minipage}[b]{\linewidth}\raggedright
વર્ણન
\end{minipage} & \begin{minipage}[b]{\linewidth}\raggedright
રેન્જ
\end{minipage} \\
\midrule\noalign{}
\endhead
\bottomrule\noalign{}
\endlastfoot
\textbf{Support} & Support(A) = Count(A) / કુલ ટ્રાન્ઝેક્શન & આઇટમસેટ કેટલી વાર
દેખાય છે & 0 થી 1 \\
\textbf{Confidence} & Confidence(A\rightarrowB) = Support(A\cupB) / Support(A) & રૂલ
કેટલી વાર સાચું છે & 0 થી 1 \\
\end{longtable}
}

\textbf{Support ઉદાહરણ:}

\begin{itemize}
\tightlist
\item
  જો આઇટમસેટ \{બ્રેડ, દૂધ\} 10 માંથી 3 ટ્રાન્ઝેક્શનમાં દેખાય છે
\item
  Support = 3/10 = 0.3 (30\%)
\end{itemize}

\textbf{Confidence ઉદાહરણ:}

\begin{itemize}
\tightlist
\item
  રૂલ: ``બ્રેડ \rightarrow દૂધ''
\item
  જો \{બ્રેડ, દૂધ\} 3 વખત દેખાય છે, બ્રેડ એકલું 5 વખત દેખાય છે
\item
  Confidence = 3/5 = 0.6 (60\%)
\end{itemize}

\end{solutionbox}
\begin{mnemonicbox}
``SUP-કેટલી વાર, CONF-કેટલું વિશ્વસનીય

\end{mnemonicbox}
\subsection*{પ્રશ્ન 4(ક OR) [7
ગુણ]}\label{uxaaauxab0uxab6uxaa8-4uxa95-or-7-uxa97uxaa3}

\textbf{K-means ક્લસ્ટરિંગ અપ્રોચ વિગતવાર લખો અને સમજાવો.}

\begin{solutionbox}

K-means ક્લસ્ટરિંગ વિધિન-ક્લસ્ટર સમ ઓફ સ્ક્વેર્સને ન્યૂનતમ કરીને ડેટાને K ક્લસ્ટરમાં
વિભાજિત કરે છે.

\textbf{K-means અલગોરિધમ પગલાં}

{\def\LTcaptype{none} % do not increment counter
\begin{longtable}[]{@{}lll@{}}
\toprule\noalign{}
પગલું & ક્રિયા & વર્ણન \\
\midrule\noalign{}
\endhead
\bottomrule\noalign{}
\endlastfoot
\textbf{1} & K પસંદ કરવું & ક્લસ્ટરની સંખ્યા પસંદ કરવી \\
\textbf{2} & સેન્ટ્રોઇડ્સ ઇનિશિયલાઇઝ કરવા & K પોઇન્ટ્સ રેન્ડમલી મૂકવા \\
\textbf{3} & પોઇન્ટ્સ અસાઇન કરવા & દરેક પોઇન્ટ નજીકના સેન્ટ્રોઇડમાં \\
\textbf{4} & સેન્ટ્રોઇડ્સ અપડેટ કરવા & અસાઇન કરેલા પોઇન્ટ્સનો મીન ગણતરી કરવો \\
\textbf{5} & 3-4 પુનરાવર્તન & કન્વર્જન્સ સુધી \\
\end{longtable}
}

\begin{verbatim}
flowchart LR
    A[K વેલ્યુ પસંદ કરવી] {-{-} B[K સેન્ટ્રોઇડ્સ રેન્ડમલી ઇનિશિયલાઇઝ કરવા]}
    B {-{-} C[દરેક પોઇન્ટને નજીકના સેન્ટ્રોઇડમાં અસાઇન કરવો]}
    C {-{-} D[સેન્ટ્રોઇડ્સને ક્લસ્ટર મીન્સમાં અપડેટ કરવા]}
    D {-{-} E\{સેન્ટ્રોઇડ્સ બદલાયા?\}}
    E {-{-}|હા| C}
    E {-{-}|ના| F[અંતિમ ક્લસ્ટર્સ]}
\end{verbatim}

\textbf{અલગોરિધમ વિગતો:}

\begin{itemize}
\tightlist
\item
  \textbf{ડિસ્ટન્સ મેટ્રિક}: સામાન્ય રીતે યુક્લિડિયન ડિસ્ટન્સ
\item
  \textbf{કન્વર્જન્સ}: જ્યારે સેન્ટ્રોઇડ્સ નોંધપાત્ર રીતે હલવા બંધ કરે છે
\item
  \textbf{ઉદ્દેશ્ય}: વિધિન-ક્લસ્ટર સમ ઓફ સ્ક્વેર્સ (WCSS) ન્યૂનતમ કરવું
\end{itemize}

\textbf{ફાયદા:}

\begin{itemize}
\tightlist
\item
  \textbf{સરળ}: સમજવા અને અમલીકરણ માટે સરળ
\item
  \textbf{કાર્યક્ષમ}: લિનિયર ટાઈમ કોમ્પ્લેક્સિટી
\item
  \textbf{સ્કેલેબલ}: મોટા ડેટાસેટ સાથે સારી રીતે કામ કરે છે
\end{itemize}

\textbf{ગેરફાયદા:}

\begin{itemize}
\tightlist
\item
  \textbf{K સિલેક્શન}: પહેલેથી K પસંદ કરવું પડે છે
\item
  \textbf{ઇનિશિયલાઇઝેશન પ્રત્યે સંવેદનશીલ}: વિવિધ શરૂઆતી પોઇન્ટ્સ વિવિધ પરિણામો
  આપે છે
\item
  \textbf{ગોળાકાર ક્લસ્ટર્સ ધારે છે}: અનિયમિત આકાર સાથે કામ ન કરી શકે
\end{itemize}

\textbf{K પસંદ કરવું:}

\begin{itemize}
\tightlist
\item
  \textbf{એલ્બો મેથડ}: WCSS vs K પ્લોટ કરવું, ``એલ્બો'' શોધવું
\item
  \textbf{સિલ્હુએટ એનાલિસિસ}: ક્લસ્ટર ગુણવત્તા માપવી
\end{itemize}

\end{solutionbox}
\begin{mnemonicbox}
``પસઅપ'' - પસંદ K, સેન્ટ્રોઇડ ઇનિશિયલાઇઝ, અસાઇન, અપડેટ,
પુનરાવર્તન

\end{mnemonicbox}
\subsection*{પ્રશ્ન 5(અ) [3
ગુણ]}\label{uxaaauxab0uxab6uxaa8-5uxa85-3-uxa97uxaa3}

\textbf{પ્રિડિક્ટિવ મોડેલ અને ડિસ્ક્રિપ્ટિવ મોડેલ વચ્ચેનો તફાવત આપો.}

\begin{solutionbox}

\textbf{પ્રિડિક્ટિવ vs ડિસ્ક્રિપ્ટિવ મોડેલ્સ}

{\def\LTcaptype{none} % do not increment counter
\begin{longtable}[]{@{}lll@{}}
\toprule\noalign{}
પાસું & પ્રિડિક્ટિવ મોડેલ & ડિસ્ક્રિપ્ટિવ મોડેલ \\
\midrule\noalign{}
\endhead
\bottomrule\noalign{}
\endlastfoot
\textbf{હેતુ} & ભવિષ્યના પરિણામો આગાહી કરવા & વર્તમાન પેટર્ન સમજાવવા \\
\textbf{આઉટપુટ} & આગાહીઓ/વર્ગીકરણ & આંતરદૃષ્ટિ/સારાંશ \\
\textbf{ઉદાહરણ} & વેચાણ આગાહી, સ્પામ ડિટેક્શન & કસ્ટમર સેગમેન્ટેશન, ટ્રેન્ડ
એનાલિસિસ \\
\end{longtable}
}

\begin{itemize}
\tightlist
\item
  \textbf{પ્રિડિક્ટિવ}: ભવિષ્યની આગાહી કરવા માટે ઐતિહાસિક ડેટાનો ઉપયોગ કરે છે
\item
  \textbf{ડિસ્ક્રિપ્ટિવ}: પેટર્ન સમજવા માટે વર્તમાન ડેટાનું વિશ્લેષણ કરે છે
\item
  \textbf{ધ્યેય}: આગાહી vs સમજ
\end{itemize}

\end{solutionbox}
\begin{mnemonicbox}
``PRED-ભવિષ્ય, DESC-વર્તમાન''

\end{mnemonicbox}
\subsection*{પ્રશ્ન 5(બ) [4
ગુણ]}\label{uxaaauxab0uxab6uxaa8-5uxaac-4-uxa97uxaa3}

\textbf{scikit-learn ની એપ્લિકેશનની સૂચિ બનાવો.}

\begin{solutionbox}

\textbf{Scikit-learn એપ્લિકેશન}

{\def\LTcaptype{none} % do not increment counter
\begin{longtable}[]{@{}
  >{\raggedright\arraybackslash}p{(\linewidth - 4\tabcolsep) * \real{0.2424}}
  >{\raggedright\arraybackslash}p{(\linewidth - 4\tabcolsep) * \real{0.3939}}
  >{\raggedright\arraybackslash}p{(\linewidth - 4\tabcolsep) * \real{0.3636}}@{}}
\toprule\noalign{}
\begin{minipage}[b]{\linewidth}\raggedright
શ્રેણી
\end{minipage} & \begin{minipage}[b]{\linewidth}\raggedright
એપ્લિકેશન
\end{minipage} & \begin{minipage}[b]{\linewidth}\raggedright
અલગોરિધમ
\end{minipage} \\
\midrule\noalign{}
\endhead
\bottomrule\noalign{}
\endlastfoot
\textbf{Classification} & ઇમેઇલ ફિલ્ટરિંગ, ઇમેજ રેકગ્નિશન & SVM, Random
Forest, Naive Bayes \\
\textbf{Regression} & કિંમત આગાહી, રિસ્ક એસેસમેન્ટ & Linear Regression,
Decision Trees \\
\textbf{Clustering} & કસ્ટમર સેગમેન્ટેશન, ડેટા એક્સપ્લોરેશન & K-means, DBSCAN \\
\textbf{Preprocessing} & ડેટા ક્લીનિંગ, ફીચર સ્કેલિંગ & StandardScaler,
LabelEncoder \\
\end{longtable}
}

\begin{itemize}
\tightlist
\item
  \textbf{મશીન લર્નિંગ લાઇબ્રેરી}: વ્યાપક Python ટૂલકિટ
\item
  \textbf{સરળ ઇન્ટીગ્રેશન}: NumPy, Pandas સાથે કામ કરે છે
\item
  \textbf{સારી ડોક્યુમેન્ટેશન}: વ્યાપક ઉદાહરણો અને ટ્યુટોરિયલ
\end{itemize}

\end{solutionbox}
\begin{mnemonicbox}
``કરકપ'' - Classification, Regression, Clustering,
Preprocessing

\end{mnemonicbox}
\subsection*{પ્રશ્ન 5(ક) [7
ગુણ]}\label{uxaaauxab0uxab6uxaa8-5uxa95-7-uxa97uxaa3}

\textbf{Numpy ના લક્ષણો અને એપ્લિકેશનો સમજાવો.}

\begin{solutionbox}

NumPy (Numerical Python) એ Python માં વૈજ્ઞાનિક કોમ્પ્યુટિંગ માટેની મૂળભૂત લાઇબ્રેરી
છે, જે મોટા બહુ-પરિમાણીય એરે અને ગાણિતિક ફંક્શન્સ માટે સપોર્ટ પ્રદાન કરે છે.

\textbf{NumPy લક્ષણો કોષ્ટક}

{\def\LTcaptype{none} % do not increment counter
\begin{longtable}[]{@{}
  >{\raggedright\arraybackslash}p{(\linewidth - 4\tabcolsep) * \real{0.3182}}
  >{\raggedright\arraybackslash}p{(\linewidth - 4\tabcolsep) * \real{0.3636}}
  >{\raggedright\arraybackslash}p{(\linewidth - 4\tabcolsep) * \real{0.3182}}@{}}
\toprule\noalign{}
\begin{minipage}[b]{\linewidth}\raggedright
લક્ષણ
\end{minipage} & \begin{minipage}[b]{\linewidth}\raggedright
વર્ણન
\end{minipage} & \begin{minipage}[b]{\linewidth}\raggedright
ફાયદો
\end{minipage} \\
\midrule\noalign{}
\endhead
\bottomrule\noalign{}
\endlastfoot
\textbf{N-dimensional Arrays} & શક્તિશાળી એરે ઓબ્જેક્ટ્સ & કાર્યક્ષમ ડેટા
સ્ટોરેજ \\
\textbf{Broadcasting} & વિવિધ આકારના એરે પર ઓપરેશન & લવચીક ગણતરી \\
\textbf{Mathematical Functions} & ત્રિકોણમિતિ, લઘુગણક, આંકડાકીય & સંપૂર્ણ ગણિત
ટૂલકિટ \\
\textbf{Performance} & C/Fortran માં અમલીકરણ & ઝડપી એક્ઝીક્યુશન \\
\textbf{Memory Efficiency} & સતત મેમરી લેઆઉટ & મેમરી વપરાશ ઘટાડવું \\
\end{longtable}
}

\textbf{NumPy એપ્લિકેશન}

{\def\LTcaptype{none} % do not increment counter
\begin{longtable}[]{@{}
  >{\raggedright\arraybackslash}p{(\linewidth - 4\tabcolsep) * \real{0.2692}}
  >{\raggedright\arraybackslash}p{(\linewidth - 4\tabcolsep) * \real{0.5000}}
  >{\raggedright\arraybackslash}p{(\linewidth - 4\tabcolsep) * \real{0.2308}}@{}}
\toprule\noalign{}
\begin{minipage}[b]{\linewidth}\raggedright
ડોમેન
\end{minipage} & \begin{minipage}[b]{\linewidth}\raggedright
એપ્લિકેશન
\end{minipage} & \begin{minipage}[b]{\linewidth}\raggedright
હેતુ
\end{minipage} \\
\midrule\noalign{}
\endhead
\bottomrule\noalign{}
\endlastfoot
\textbf{મશીન લર્નિંગ} & ડેટા પ્રીપ્રોસેસિંગ, ફીચર એન્જિનિયરિંગ & ન્યુમેરિકલ ડેટા
સંભાળવો \\
\textbf{ઇમેજ પ્રોસેસિંગ} & ઇમેજ મેનિપ્યુલેશન, ફિલ્ટરિંગ & પિક્સેલ એરે પ્રોસેસ કરવા \\
\textbf{વૈજ્ઞાનિક કોમ્પ્યુટિંગ} & ન્યુમેરિકલ સિમ્યુલેશન, મોડેલિંગ & ગાણિતિક ગણતરીઓ \\
\textbf{ફાઇનાન્શિયલ એનાલિસિસ} & પોર્ટફોલિયો ઓપ્ટિમાઇઝેશન, રિસ્ક મોડેલિંગ &
માત્રાત્મક વિશ્લેષણ \\
\end{longtable}
}

\begin{verbatim}
mindmap
  root((NumPy))
    મુખ્ય લક્ષણો
      N{-dimensional Arrays}
      Broadcasting
      Mathematical Functions
      ઝડપી પરફોર્મન્સ
    એપ્લિકેશન
      મશીન લર્નિંગ
      ઇમેજ પ્રોસેસિંગ
      વૈજ્ઞાનિક કોમ્પ્યુટિંગ
      ફાઇનાન્શિયલ એનાલિસિસ
    ફાયદા
      મેમરી કાર્યક્ષમ
      ઉપયોગમાં સરળ
      સારું ઇન્ટીગ્રેશન
      ઇન્ડસ્ટ્રી સ્ટાન્ડર્ડ
\end{verbatim}

\textbf{મુખ્ય ક્ષમતાઓ:}

\begin{itemize}
\tightlist
\item
  \textbf{એરે ઓપરેશન્સ}: એલિમેન્ટ-વાઇઝ ઓપરેશન્સ, સ્લાઇસિંગ, ઇન્ડેક્સિંગ
\item
  \textbf{લિનિયર અલજેબ્રા}: મેટ્રિક્સ ઓપરેશન્સ, eigenvalues, decompositions
\item
  \textbf{રેન્ડમ નંબર જનરેશન}: આંકડાકીય વિતરણ, સેમ્પલિંગ
\item
  \textbf{ફૂરિયર ટ્રાન્સફોર્મ}: સિગ્નલ પ્રોસેસિંગ, ફ્રીક્વન્સી એનાલિસિસ
\end{itemize}

\textbf{ઇન્ટીગ્રેશન:}

\begin{itemize}
\tightlist
\item
  \textbf{Pandas}: DataFrames NumPy એરે પર બનેલા છે
\item
  \textbf{Matplotlib}: NumPy એરે પ્લોટ કરવા
\item
  \textbf{Scikit-learn}: ML અલગોરિધમ NumPy એરે વાપરે છે
\end{itemize}

\end{solutionbox}
\begin{mnemonicbox}
``NઝEગવ'' - N-dimensional, ઝડપી, એરે, ગણિત, વૈજ્ઞાનિક

\end{mnemonicbox}
\subsection*{પ્રશ્ન 5(અ OR) [3
ગુણ]}\label{uxaaauxab0uxab6uxaa8-5uxa85-or-3-uxa97uxaa3}

\textbf{બેગિંગ પર ટૂંકી નોંધ લખો}

\begin{solutionbox}

બેગિંગ (Bootstrap Aggregating) એ ensemble પદ્ધતિ છે જે ડેટાના વિવિધ સબસેટ પર
બહુવિધ મોડેલ ટ્રેનિંગ કરીને મોડેલ પરફોર્મન્સ સુધારે છે.

\textbf{બેગિંગ પ્રક્રિયા કોષ્ટક}

{\def\LTcaptype{none} % do not increment counter
\begin{longtable}[]{@{}
  >{\raggedright\arraybackslash}p{(\linewidth - 4\tabcolsep) * \real{0.3182}}
  >{\raggedright\arraybackslash}p{(\linewidth - 4\tabcolsep) * \real{0.4091}}
  >{\raggedright\arraybackslash}p{(\linewidth - 4\tabcolsep) * \real{0.2727}}@{}}
\toprule\noalign{}
\begin{minipage}[b]{\linewidth}\raggedright
પગલું
\end{minipage} & \begin{minipage}[b]{\linewidth}\raggedright
પ્રક્રિયા
\end{minipage} & \begin{minipage}[b]{\linewidth}\raggedright
હેતુ
\end{minipage} \\
\midrule\noalign{}
\endhead
\bottomrule\noalign{}
\endlastfoot
\textbf{Bootstrap Sampling} & બહુવિધ ટ્રેનિંગ સેટ બનાવવા & વિવિધ ડેટાસેટ જનરેટ
કરવા \\
\textbf{Train Models} & દરેક સબસેટ પર મોડેલ બનાવવું & બહુવિધ આગાહીકર્તા
બનાવવા \\
\textbf{Aggregate Results} & આગાહીઓ જોડવી (વોટિંગ/એવરેજિંગ) & ઓવરફિટિંગ
ઘટાડવું \\
\end{longtable}
}

\begin{itemize}
\tightlist
\item
  \textbf{વેરિયન્સ રિડક્શન}: એવરેજિંગ દ્વારા મોડેલ વેરિયન્સ ઘટાડે છે
\item
  \textbf{પેરેલલ ટ્રેનિંગ}: મોડેલ્સ સ્વતંત્ર રીતે ટ્રેન થાય છે
\item
  \textbf{ઉદાહરણ}: Random Forest ડિસિઝન ટ્રી સાથે બેગિંગ વાપરે છે
\end{itemize}

\end{solutionbox}
\begin{mnemonicbox}
``બટએ'' - Bootstrap, Train, Aggregate

\end{mnemonicbox}
\subsection*{પ્રશ્ન 5(બ OR) [4
ગુણ]}\label{uxaaauxab0uxab6uxaa8-5uxaac-or-4-uxa97uxaa3}

\textbf{Pandas લક્ષણોની યાદી આપો.}

\begin{solutionbox}

\textbf{Pandas લક્ષણો}

{\def\LTcaptype{none} % do not increment counter
\begin{longtable}[]{@{}
  >{\raggedright\arraybackslash}p{(\linewidth - 4\tabcolsep) * \real{0.3182}}
  >{\raggedright\arraybackslash}p{(\linewidth - 4\tabcolsep) * \real{0.3636}}
  >{\raggedright\arraybackslash}p{(\linewidth - 4\tabcolsep) * \real{0.3182}}@{}}
\toprule\noalign{}
\begin{minipage}[b]{\linewidth}\raggedright
લક્ષણ
\end{minipage} & \begin{minipage}[b]{\linewidth}\raggedright
વર્ણન
\end{minipage} & \begin{minipage}[b]{\linewidth}\raggedright
ફાયદો
\end{minipage} \\
\midrule\noalign{}
\endhead
\bottomrule\noalign{}
\endlastfoot
\textbf{DataFrame/Series} & સ્ટ્રક્ચર્ડ ડેટા કન્ટેનર & સરળ ડેટા મેનિપ્યુલેશન \\
\textbf{File I/O} & CSV, Excel, JSON રીડ/રાઇટ & વિવિધ ફોર્મેટ સંભાળવા \\
\textbf{Data Cleaning} & ખોવાયેલી વેલ્યૂઝ, ડુપ્લિકેટ સંભાળવા & સાફ ડેટા તૈયાર
કરવો \\
\textbf{Grouping/Aggregation} & ગ્રુપ બાય ઓપરેશન્સ, આંકડાકીય & ડેટા પેટર્ન
એનાલિઝ કરવા \\
\end{longtable}
}

\textbf{ડેટા ઓપરેશન્સ:}

\begin{itemize}
\tightlist
\item
  \textbf{ઇન્ડેક્સિંગ}: લવચીક ડેટા સિલેક્શન અને ફિલ્ટરિંગ
\item
  \textbf{મર્જિંગ}: joins સાથે ડેટાસેટ જોડવા
\item
  \textbf{રીશેપિંગ}: પિવોટ ટેબલ અને ડેટા ટ્રાન્સફોર્મેશન
\end{itemize}

\end{solutionbox}
\begin{mnemonicbox}
``ડફઇગ'' - DataFrame, ફાઇલ I/O, ઇન્ડેક્સિંગ, ગ્રુપિંગ

\end{mnemonicbox}
\subsection*{પ્રશ્ન 5(ક OR) [7
ગુણ]}\label{uxaaauxab0uxab6uxaa8-5uxa95-or-7-uxa97uxaa3}

\textbf{Matplotlib ની વિશેષતાઓ અને એપ્લિકેશનો સમજાવો.}

\begin{solutionbox}

Matplotlib એ Python માટેની એક વ્યાપક 2D પ્લોટિંગ લાઇબ્રેરી છે જે વિવિધ ફોર્મેટ અને
ઇન્ટરેક્ટિવ વાતાવરણમાં પ્રકાશન-ગુણવત્તાવાળા આકૃતિઓ બનાવે છે.

\textbf{Matplotlib લક્ષણો}

{\def\LTcaptype{none} % do not increment counter
\begin{longtable}[]{@{}
  >{\raggedright\arraybackslash}p{(\linewidth - 4\tabcolsep) * \real{0.3043}}
  >{\raggedright\arraybackslash}p{(\linewidth - 4\tabcolsep) * \real{0.3478}}
  >{\raggedright\arraybackslash}p{(\linewidth - 4\tabcolsep) * \real{0.3478}}@{}}
\toprule\noalign{}
\begin{minipage}[b]{\linewidth}\raggedright
લક્ષણ
\end{minipage} & \begin{minipage}[b]{\linewidth}\raggedright
વર્ણન
\end{minipage} & \begin{minipage}[b]{\linewidth}\raggedright
ક્ષમતા
\end{minipage} \\
\midrule\noalign{}
\endhead
\bottomrule\noalign{}
\endlastfoot
\textbf{Plot Types} & Line, bar, scatter, histogram, pie & વિવિધ
વિઝ્યુઅલાઇઝેશન વિકલ્પો \\
\textbf{Customization} & રંગો, ફોન્ટ્સ, સ્ટાઇલ, લેઆઉટ & વ્યાવસાયિક દેખાવ \\
\textbf{Interactive Features} & Zoom, pan, widgets & ગતિશીલ એક્સપ્લોરેશન \\
\textbf{Multiple Backends} & GUI, વેબ, ફાઇલ આઉટપુટ & લવચીક ડિપ્લોયમેન્ટ \\
\textbf{3D Plotting} & Surface, wireframe, scatter plots & ત્રિ-પરિમાણીય
વિઝ્યુઅલાઇઝેશન \\
\end{longtable}
}

\textbf{Matplotlib એપ્લિકેશન}

{\def\LTcaptype{none} % do not increment counter
\begin{longtable}[]{@{}lll@{}}
\toprule\noalign{}
ડોમેન & એપ્લિકેશન & વિઝ્યુઅલાઇઝેશન પ્રકાર \\
\midrule\noalign{}
\endhead
\bottomrule\noalign{}
\endlastfoot
\textbf{ડેટા સાયન્સ} & એક્સપ્લોરેટરી ડેટા એનાલિસિસ & હિસ્ટોગ્રામ, સ્કેટર પ્લોટ \\
\textbf{વૈજ્ઞાનિક સંશોધન} & પ્રકાશન આકૃતિઓ & લાઇન પ્લોટ, એરર બાર \\
\textbf{બિઝનેસ ઇન્ટેલિજન્સ} & ડેશબોર્ડ બનાવવું & બાર ચાર્ટ, ટ્રેન્ડ લાઇન \\
\textbf{મશીન લર્નિંગ} & મોડેલ પરફોર્મન્સ વિઝ્યુઅલાઇઝેશન & કન્ફ્યુઝન મેટ્રિક્સ, ROC
કર્વ \\
\textbf{એન્જિનિયરિંગ} & સિગ્નલ એનાલિસિસ & ટાઇમ સિરિઝ, ફ્રીક્વન્સી પ્લોટ \\
\end{longtable}
}

\begin{verbatim}
flowchart LR
    A[કાચો ડેટા] {-{-} B[Matplotlib પ્રોસેસિંગ]}
    B {-{-} C[સ્ટેટિક પ્લોટ]}
    B {-{-} D[ઇન્ટરેક્ટિવ પ્લોટ]}
    B {-{-} E[પ્રકાશન આકૃતિઓ]}
    C {-{-} F[PNG/PDF આઉટપુટ]}
    D {-{-} G[વેબ એપ્લિકેશન]}
    E {-{-} H[સંશોધન પેપર]}
\end{verbatim}

\textbf{મુખ્ય ઘટકો:}

\begin{itemize}
\tightlist
\item
  \textbf{Figure}: બધા પ્લોટ એલિમેન્ટ્સ માટે ટોપ-લેવલ કન્ટેનર
\item
  \textbf{Axes}: આકૃતિની અંદર વ્યક્તિગત પ્લોટ
\item
  \textbf{Artist}: આકૃતિ પર દોરવામાં આવતું બધું (રેખાઓ, ટેક્સ્ટ, વગેરે)
\item
  \textbf{Backend}: વિવિધ આઉટપુટ માટે રેન્ડરિંગ સંભાળે છે
\end{itemize}

\textbf{પ્લોટ કસ્ટમાઇઝેશન:}

\begin{itemize}
\tightlist
\item
  \textbf{રંગો/સ્ટાઇલ}: વિઝ્યુઅલ વિકલ્પોની વિશાળ શ્રેણી
\item
  \textbf{એનોટેશન}: ટેક્સ્ટ લેબલ, એરો, લેજેન્ડ
\item
  \textbf{સબપ્લોટ}: સિંગલ આકૃતિમાં બહુવિધ પ્લોટ
\item
  \textbf{લેઆઉટ}: ગ્રિડ ગોઠવણી, સ્પેસિંગ કન્ટ્રોલ
\end{itemize}

\textbf{ઇન્ટીગ્રેશન ફાયદા:}

\begin{itemize}
\tightlist
\item
  \textbf{NumPy એરે}: ન્યુમેરિકલ ડેટાનું સીધું પ્લોટિંગ
\item
  \textbf{Pandas}: બિલ્ટ-ઇન પ્લોટિંગ મેથડ્સ
\item
  \textbf{Jupyter Notebooks}: ઇનલાઇન પ્લોટ ડિસ્પ્લે
\item
  \textbf{વેબ ફ્રેમવર્ક}: એપ્લિકેશનમાં પ્લોટ એમ્બેડ કરવા
\end{itemize}

\textbf{આઉટપુટ ફોર્મેટ:}

\begin{itemize}
\tightlist
\item
  \textbf{રેસ્ટર}: વેબ ઉપયોગ માટે PNG, JPEG
\item
  \textbf{વેક્ટર}: પ્રકાશન માટે PDF, SVG
\item
  \textbf{ઇન્ટરેક્ટિવ}: વેબ ડિપ્લોયમેન્ટ માટે HTML
\end{itemize}

\end{solutionbox}
\begin{mnemonicbox}
``બવઇકવ'' - બહુવિધ પ્લોટ, વિઝ્યુઅલાઇઝેશન, ઇન્ટરેક્ટિવ,
કસ્ટમાઇઝેબલ, વૈજ્ઞાનિક

\end{mnemonicbox}

\end{document}
