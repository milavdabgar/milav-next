\documentclass{article}

% content/resources/templates/preamble.tex
\usepackage[margin=0.6in]{geometry}
\author{Milav Dabgar}
\usepackage{amsmath,amssymb,amsthm}
\usepackage{booktabs}
\usepackage{multirow}
\usepackage{xcolor}
\usepackage{tcolorbox}
\tcbuselibrary{breakable,skins}
\usepackage[colorlinks=true,linkcolor=blue]{hyperref}
\usepackage{titlesec}
\usepackage{enumitem}
\usepackage{tikz}
\usepackage{pgfplots}
\usepackage{circuitikz}
\usepackage[version=4]{mhchem}
\usepackage{longtable}
\usepackage{array}
\usepackage{float}
\usepackage{caption}
\usepackage{listings}

\lstset{
  basicstyle=\small\ttfamily,
  breaklines=true,
  breakatwhitespace=false,
  postbreak=\mbox{\textcolor{red}{$\hookrightarrow$}\space},
  float=false,
  numbers=left,
  numberstyle=\tiny\color{gray},
  numbersep=10pt,
  xleftmargin=2em,
  keywordstyle=\color{blue},
  commentstyle=\color{green!60!black},
  stringstyle=\color{purple},
  backgroundcolor=\color{gray!5},
  showstringspaces=false,
  tabsize=2,
  captionpos=b,
  keepspaces=true,
  columns=flexible
}

\pgfplotsset{compat=1.18}
\usetikzlibrary{shapes,arrows,positioning,calc,patterns,decorations.pathmorphing,decorations.markings,arrows.meta}

% Color scheme
\definecolor{headcolor}{RGB}{0,102,204}
\definecolor{keycolor}{RGB}{220,20,60}
\definecolor{solutioncolor}{RGB}{34,139,34}
\definecolor{mnemoniccolor}{RGB}{148,0,211}
\definecolor{codecolor}{RGB}{0,0,100}

% Spacing
\setlength{\parskip}{3pt}
\setlist[itemize]{nosep}
\setlist[enumerate]{nosep}

% Title formatting
\titleformat{\section}{\Large\bfseries\color{headcolor}}{\thesection}{1em}{}
\titleformat{\subsection}{\large\bfseries\color{headcolor}}{\thesubsection}{1em}{}

% Pandoc tightlist compatibility
\providecommand{\tightlist}{%
  \setlength{\itemsep}{0pt}\setlength{\parskip}{0pt}}

% Pandoc longtable compatibility
\newcounter{none}
\def\thenone{}


% content/resources/templates/gujarati-boxes.tex
\usepackage{fontspec}
\usepackage{polyglossia}

% Set Gujarati as main language (document is primarily in Gujarati)
% Note: gloss-gujarati.ldf doesn't exist in polyglossia, but it will use hyphenation patterns
\setdefaultlanguage{gujarati}
\setotherlanguage{english}

% Configure Gujarati font properly
% Use Language=Default to prevent polyglossia from trying to add language-specific features
% that don't exist for Gujarati, which causes "empty feature" warnings
\newfontfamily\gujaratifont[Script=Gujarati,AutoFakeBold=2.5,AutoFakeSlant=0.3]{Noto Sans Gujarati}
\setmainfont[Script=Gujarati,AutoFakeBold=2.5,AutoFakeSlant=0.3]{Noto Sans Gujarati}
% Use Noto Sans Gujarati for monospace to support Gujarati in text
\setmonofont[Scale=0.9]{Noto Sans Gujarati}

% Configure English to use the same font
\newfontfamily\englishfont[Script=Gujarati,AutoFakeBold=2.5,AutoFakeSlant=0.3]{Noto Sans Gujarati}

% Translations for polyglossia
\gappto\captionsgujarati{
  \renewcommand{\tablename}{કોષ્ટક}
  \renewcommand{\figurename}{આકૃતિ}
}

% Helper for TikZ nodes to ensure Gujarati font
\newcommand{\gu}[1]{{\gujaratifont #1}}

% Custom environments
\newtcolorbox{solutionbox}{
    breakable,
    enhanced,
    colback=solutioncolor!5!white,
    colframe=solutioncolor!75!black,
    fonttitle=\bfseries,
    title=જવાબ
}

\newtcolorbox{solutionboxnobreak}{
 colback=solutioncolor!5!white,
 colframe=solutioncolor!75!black,
 fonttitle=\bfseries,
 title=જવાબ
}

\newtcolorbox{keyformula}{
 breakable,
 enhanced,
 colback=keycolor!5!white,
 colframe=keycolor!75!black,
 fonttitle=\bfseries,
 title=રાસાયણિક સમીકરણ/સૂત્ર
}

\newtcolorbox{mnemonicbox}{
 breakable,
 enhanced,
 colback=mnemoniccolor!5!white,
 colframe=mnemoniccolor!75!black,
 fonttitle=\bfseries,
 title=મેમરી ટ્રીક
}


% Custom commands for GTU solutions
% This file defines semantic commands for consistent formatting

% Question command with automatic formatting
\newcommand{\question}[2]{%
  \section*{Question #1}%
  \textbf{#2}%
}

% OR question variant
\newcommand{\questionor}[2]{%
  \section*{Question #1 OR}%
  \textbf{#2}%
}

% Proper table environment with caption
\newenvironment{answertable}[1]{%
  \begin{table}[htbp]
  \centering
  \caption{#1}
}{%
  \end{table}
}

% Proper figure environment for diagrams
\newenvironment{answerdiagram}[1]{%
  \begin{figure}[htbp]
  \centering
  \caption{#1}
}{%
  \end{figure}
}

% Semantic markup for key terms
\newcommand{\keyword}[1]{\textbf{#1}}
\newcommand{\code}[1]{\texttt{#1}}
\newcommand{\classname}[1]{\texttt{#1}}
\newcommand{\methodname}[1]{\texttt{#1}}

% Proper quotation marks
\newcommand{\mnemonic}[1]{``#1''}


\title{ડિજિટલ માર્કેટિંગના આવશ્યક તત્વો (4341601) - ઉનાળા 2024 સોલ્યુશન}
\date{June 11, 2024}

\begin{document}
\maketitle

\questionmarks{1(a)}{3}{તફાવત આપો: ટ્રેડિશનલ માર્કેટિંગ અને ડિજિટલ માર્કેટિંગ.}

\begin{solutionbox}
\begin{center}
\captionof{table}{ટ્રેડિશનલ માર્કેટિંગ vs ડિજિટલ માર્કેટિંગ}
\begin{tabulary}{\linewidth}{|L|L|}
\hline
\textbf{ટ્રેડિશનલ માર્કેટિંગ} & \textbf{ડિજિટલ માર્કેટિંગ} \\ \hline
\textbf{ભૌતિક હાજરી} મુશ્કેલ & \textbf{ઓનલાઇન હાજરી} સરળ \\ \hline
\textbf{મર્યાદિત પહોંચ} (સ્થાનિક) & \textbf{વૈશ્વિક પહોંચ} (વિશ્વભરમાં) \\ \hline
\textbf{એક તરફી પ્રત્યાયન} & \textbf{દ્વિ-માર્ગી ઇન્ટરેક્ટિવ} પ્રત્યાયન \\ \hline
\textbf{જાહેરાત ખર્ચાળ} છે & \textbf{ખર્ચ-અસરકારક} ઝુંબેશ \\ \hline
\textbf{ROI માપવું મુશ્કેલ} & \textbf{સરળ ટ્રેકિંગ} અને એનાલિટિક્સ \\ \hline
\textbf{ધીમો પ્રતિસાદ} & \textbf{ત્વરિત પ્રતિસાદ} \\ \hline
\end{tabulary}
\end{center}
\end{solutionbox}

\begin{mnemonicbox}
\mnemonic{PITCH vs CLICK: Physical vs Interactive, Traditional vs Trackable, High-cost vs Cost-effective}
\end{mnemonicbox}

\questionmarks{1(b)}{4}{સર્ચ એન્જિન અલ્ગોરિધમનું કાર્ય સમજાવો.}

\begin{solutionbox}
સર્ચ એન્જિન અલ્ગોરિધમ્સ સંબંધિત પરિણામો પ્રદાન કરવા માટે વ્યવસ્થિત પ્રક્રિયાઓ દ્વારા કાર્ય કરે છે:

\begin{center}
\begin{tikzpicture}[node distance=1.5cm, auto]
    \node [gtu block] (Crawl) {વેબ ક્રોલિંગ};
    \node [gtu block, right=1cm of Crawl] (Index) {ઇન્ડેક્સિંગ};
    \node [gtu block, right=1cm of Index] (Query) {ક્વેરી પ્રોસેસિંગ};
    \node [gtu block, below=1cm of Crawl] (Rank) {રેન્કિંગ અલ્ગોરિધમ};
    \node [gtu state, right=1cm of Rank] (SERP) {SERP ડિસ્પ્લે};

    \path [gtu arrow] (Crawl) -- (Index);
    \path [gtu arrow] (Index) -- (Query);
    \path [gtu arrow] (Query) -- (Rank);
    \path [gtu arrow] (Rank) -- (SERP);
\end{tikzpicture}
\captionof{figure}{Search Engine Process Flow}
\end{center}

\begin{itemize}
    \item \keyword{Crawling}: સર્ચ બોટ્સ નવી સામગ્રી શોધવા માટે સતત વેબસાઇટ્સ સ્કેન કરે છે
    \item \keyword{Indexing}: વિશ્લેષિત સામગ્રી કીવર્ડ્સ સાથે વિશાળ ડેટાબેઝમાં સંગ્રહિત થાય છે
    \item \keyword{Query matching}: વપરાશકર્તાની સર્ચ ટર્મ્સ ઇન્ડેક્સ કરેલી સામગ્રી સાથે મેચ કરવામાં આવે છે
    \item \keyword{Ranking factors}: સામગ્રીની સુસંગતતા, ઓથોરિટી અને વપરાશકર્તા અનુભવ સ્થાન નક્કી કરે છે
\end{itemize}
\end{solutionbox}

\begin{mnemonicbox}
\mnemonic{CIRR: Crawl, Index, Rank, Results}
\end{mnemonicbox}

\questionmarks{1(c)}{7}{ડિજિટલ માર્કેટિંગ યોજનાના મુખ્ય ઘટકો સમજાવો.}

\begin{solutionbox}
સફળતા માટે ડિજિટલ માર્કેટિંગ યોજનામાં નીચેના મુખ્ય ઘટકોનો સમાવેશ થાય છે:

\begin{center}
\captionof{table}{ડિજિટલ માર્કેટિંગ યોજનાના ઘટકો}
\begin{tabulary}{\linewidth}{|L|L|L|}
\hline
\textbf{ઘટક} & \textbf{વર્ણન} & \textbf{હેતુ} \\ \hline
\textbf{પરિસ્થિતિ વિશ્લેષણ} & વર્તમાન બજાર સ્થિતિ અને SWOT & પ્રારંભિક બિંદુ સમજવું \\ \hline
\textbf{લક્ષ્ય પ્રેક્ષકો} & ડેમોગ્રાફિક્સ અને બાયર પર્સોના & કેન્દ્રિત માર્કેટિંગ પ્રયાસો \\ \hline
\textbf{લક્ષ્યો અને ઉદ્દેશ્યો} & SMART ગોલ્સ અને KPIs & માપી શકાય તેવા પરિણામો \\ \hline
\textbf{વ્યૂહરચના પસંદગી} & SEO, SEM, સોશિયલ મીડિયા, ઇમેઇલ & ચેનલ ઓપ્ટિમાઇઝેશન \\ \hline
\textbf{બજેટ ફાળવણી} & ચેનલ્સમાં સંસાધન વિતરણ & ખર્ચ વ્યવસ્થાપન \\ \hline
\textbf{કન્ટેન્ટ કેલેન્ડર} & શેડ્યૂલ કરેલ કન્ટેન્ટ પ્રકાશન & સતત જોડાણ \\ \hline
\textbf{Analytics સેટઅપ} & ટ્રેકિંગ ટૂલ્સ અને મેટ્રિક્સ & પ્રદર્શન મોનિટરિંગ \\ \hline
\end{tabulary}
\end{center}

\textbf{સફળતાના મુખ્ય પરિબળો:}

\begin{itemize}
    \item \keyword{Research-driven}: બજારની આંતરદૃષ્ટિ સાથેનો અભિગમ
    \item \keyword{Integration}: બહુવિધ ડિજિટલ ચેનલ્સમાં એકીકરણ
    \item \keyword{Flexibility}: પ્રદર્શન ડેટાના આધારે અનુકૂલન સાધવાની ક્ષમતા
\end{itemize}
\end{solutionbox}

\begin{mnemonicbox}
\mnemonic{STGSBC Analytics: Situation, Target, Goals, Strategy, Budget, Content, Analytics}
\end{mnemonicbox}

\questionmarks{1(c) OR}{7}{P.O.E.M. ફ્રેમવર્કના ઘટકો સમજાવો અને ડિજિટલ માર્કેટિંગમાં તેમની સુસંગતતા સમજાવો.}

\begin{solutionbox}
P.O.E.M. ફ્રેમવર્ક વ્યૂહાત્મક આયોજન માટે ડિજિટલ માર્કેટિંગ ચેનલોનું વર્ગીકરણ કરે છે:

\begin{center}
\begin{tikzpicture}[node distance=2cm, auto]
    \node [gtu state] (POEM) {P.O.E.M.};
    \node [gtu block, above left=1cm and 0.5cm of POEM] (Paid) {Paid Media};
    \node [gtu block, above right=1cm and 0.5cm of POEM] (Owned) {Owned Media};
    \node [gtu block, below left=1cm and 0.5cm of POEM] (Earned) {Earned Media};
    \node [gtu block, below right=1cm and 0.5cm of POEM] (Managed) {Managed Media};

    \path [gtu arrow] (POEM) -- (Paid);
    \path [gtu arrow] (POEM) -- (Owned);
    \path [gtu arrow] (POEM) -- (Earned);
    \path [gtu arrow] (POEM) -- (Managed);
\end{tikzpicture}
\captionof{figure}{P.O.E.M. Framework}
\end{center}

\begin{center}
\captionof{table}{P.O.E.M. ઘટકો}
\begin{tabulary}{\linewidth}{|L|L|L|L|}
\hline
\textbf{ઘટક} & \textbf{વ્યાખ્યા} & \textbf{ઉદાહરણો} & \textbf{સુસંગતતા} \\ \hline
\textbf{Paid Media} & ખરીદેલ જાહેરાત સ્થાન & Google Ads, Facebook Ads & \textbf{તાત્કાલિક પહોંચ} અને નિયંત્રણ \\ \hline
\textbf{Owned Media} & બ્રાન્ડ-નિયંત્રિત ચેનલ્સ & વેબસાઇટ, બ્લોગ્સ & \textbf{લાંબા ગાળાની સંપત્તિ} \\ \hline
\textbf{Earned Media} & ત્રીજા પક્ષની ભલામણો & રિવ્યૂઝ, શેર્સ & \textbf{વિશ્વસનીયતા} અને વિશ્વાસ \\ \hline
\textbf{Managed Media} & પ્રભાવિત પરંતુ માલિકીના નહીં & ઇન્ફ્લુએન્સર ભાગીદારી & \textbf{વિસ્તૃત પહોંચ} \\ \hline
\end{tabulary}
\end{center}

\textbf{વ્યૂહાત્મક લાભો:}

\begin{itemize}
    \item \keyword{Balanced approach}: તમામ મીડિયા પ્રકારોમાં સંતુલિત અભિગમ
    \item \keyword{Cost optimization}: ચેનલ મિશ્રણ દ્વારા ખર્ચ ઓપ્ટિમાઇઝેશન
    \item \keyword{Amplified impact}: જ્યારે ચેનલો એકસાથે કામ કરે છે ત્યારે પ્રભાવ વધે છે
\end{itemize}
\end{solutionbox}

\begin{mnemonicbox}
\mnemonic{POEM builds Digital SUCCESS: Paid, Owned, Earned, Managed}
\end{mnemonicbox}

\questionmarks{2(a)}{3}{SEO ની જરૂરિયાત વર્ણવો.}

\begin{solutionbox}
ઓનલાઇન દૃશ્યતા અને વ્યવસાય વૃદ્ધિ માટે SEO આવશ્યક છે:

\begin{itemize}
    \item \keyword{Organic traffic}: 68\% ઓનલાઇન અનુભવો સર્ચ એન્જિનથી શરૂ થાય છે
    \item \keyword{Cost-effective}: પેઇડ જાહેરાતોથી વિપરીત ઓર્ગેનિક રેન્કિંગ માટે સીધી ચુકવણી નથી
    \item \keyword{Trust building}: ઉચ્ચ રેન્કિંગ વપરાશકર્તાઓ સાથે વિશ્વસનીયતા બનાવે છે
    \item \keyword{Long-term results}: સમય જતાં ટકાઉ ટ્રાફિક વૃદ્ધિ
\end{itemize}
\end{solutionbox}

\begin{mnemonicbox}
\mnemonic{OCTL: Organic, Cost-effective, Trust, Long-term}
\end{mnemonicbox}

\questionmarks{2(b)}{4}{ઓન-પેજ અને ઓફ-પેજ ઓપ્ટિમાઇઝેશન વચ્ચે તફાવત આપો.}

\begin{solutionbox}
\begin{center}
\captionof{table}{On-Page vs Off-Page SEO}
\begin{tabulary}{\linewidth}{|L|L|}
\hline
\textbf{On-Page SEO} & \textbf{Off-Page SEO} \\ \hline
\textbf{વેબસાઇટ તત્વો}નું ઓપ્ટિમાઇઝેશન & \textbf{બાહ્ય પરિબળો}નું ઓપ્ટિમાઇઝેશન \\ \hline
\textbf{ટાઇટલ ટેગ્સ, મેટા વર્ણનો} & \textbf{અન્ય સાઇટ્સ પરથી બેકલિંક્સ} \\ \hline
\textbf{સામગ્રી ગુણવત્તા અને કીવર્ડ્સ} & \textbf{સોશિયલ મીડિયા સિગ્નલ્સ} \\ \hline
\textbf{આંતરિક લિંકિંગ માળખું} & \textbf{ડોમેન ઓથોરિટી બિલ્ડિંગ} \\ \hline
\textbf{સંપૂર્ણ નિયંત્રણ} (વેબસાઇટ માલિક) & \textbf{મર્યાદિત નિયંત્રણ} (અન્ય પર આધારિત) \\ \hline
\textbf{ટેકનિકલ ઓપ્ટિમાઇઝેશન} પર ફોકસ & \textbf{સત્તા અને લોકપ્રિયતા} પર ફોકસ \\ \hline
\end{tabulary}
\end{center}
\end{solutionbox}

\begin{mnemonicbox}
\mnemonic{IN vs OUT: Internal optimization vs Outbound authority}
\end{mnemonicbox}

\questionmarks{2(c)}{7}{SEO રેન્કિંગ સમજાવો અને SEO રેન્કિંગ સુધારવાની રીતો સમજાવો.}

\begin{solutionbox}
SEO રેન્કિંગ સર્ચ એન્જિન પરિણામ પૃષ્ઠો (SERPs) માં વેબસાઇટનું સ્થાન નક્કી કરે છે.

\begin{center}
\captionof{table}{Ranking Factors}
\begin{tabulary}{\linewidth}{|L|L|L|}
\hline
\textbf{પરિબળ શ્રેણી} & \textbf{તકનીકો} & \textbf{અસર સ્તર} \\ \hline
\textbf{સામગ્રી ગુણવત્તા} & મૌલિક, મૂલ્યવાન સામગ્રી & ઉચ્ચ \\ \hline
\textbf{કીવર્ડ્સ} & સંશોધન અને કુદરતી પ્લેસમેન્ટ & ઉચ્ચ \\ \hline
\textbf{ટેકનિકલ SEO} & સાઇટ સ્પીડ, મોબાઇલ-ફ્રેન્ડલી & મધ્યમ \\ \hline
\textbf{બેકલિંક્સ} & ગુણવત્તાયુક્ત લિંક બિલ્ડિંગ & ઉચ્ચ \\ \hline
\textbf{વપરાશકર્તા અનુભવ} & ઓછો બાઉન્સ રેટ, વધુ જોડાણ & મધ્યમ \\ \hline
\end{tabulary}
\end{center}

\textbf{સુધારણા વ્યૂહરચનાઓ:}

\begin{itemize}
    \item \keyword{Content optimization}: વ્યાપક, વપરાશકર્તા-કેન્દ્રિત સામગ્રી બનાવો
    \item \keyword{Keyword research}: સંબંધિત કીવર્ડ્સને લક્ષ્ય બનાવો
    \item \keyword{Technical fixes}: સાઇટ સ્પીડ અને મોબાઇલ રિસ્પોન્સિવનેસ સુધારો
    \item \keyword{Link building}: અધિકૃત સાઇટ્સ પરથી બેકલિંક્સ મેળવો
    \item \keyword{User signals}: જોડાણ મેટ્રિક્સમાં વધારો કરો
\end{itemize}

\textbf{સફળતા મેટ્રિક્સ:}

\begin{itemize}
    \item \keyword{SERP position} સુધારણા
    \item \keyword{Organic traffic} વૃદ્ધિ
    \item \keyword{Click-through rates} વધારો
\end{itemize}
\end{solutionbox}

\begin{mnemonicbox}
\mnemonic{CKTU for SEO SUCCESS: Content, Keywords, Technical, User-experience}
\end{mnemonicbox}

\questionmarks{2(a) OR}{3}{વ્યાખ્યાયિત કરો: 1. બેકલિંક્સ 2. વેબસાઇટ સ્પીડ 3. કીવર્ડ સ્ટફિંગ.}

\begin{solutionbox}
\begin{center}
\captionof{table}{SEO વ્યાખ્યાઓ}
\begin{tabulary}{\linewidth}{|L|L|}
\hline
\textbf{શબ્દ} & \textbf{વ્યાખ્યા} \\ \hline
\textbf{બેકલિંક્સ} & અન્ય વેબસાઇટ્સ પરથી તમારી સાઇટ પર આવતી હાઇપરલિંક્સ \\ \hline
\textbf{વેબસાઇટ સ્પીડ} & વેબ પેજીસને બ્રાઉઝરમાં સંપૂર્ણ લોડ થવામાં લાગતો સમય \\ \hline
\textbf{કીવર્ડ સ્ટફિંગ} & રેન્કિંગ સાથે છેડછાડ કરવા માટે કીવર્ડ્સનો અકુદરતી ઉપયોગ \\ \hline
\end{tabulary}
\end{center}
\end{solutionbox}

\begin{mnemonicbox}
\mnemonic{BWK: Backlinks, Website speed, Keyword stuffing}
\end{mnemonicbox}

\questionmarks{2(b) OR}{4}{બ્લેક હેટ અને વ્હાઇટ હેટ SEO તકનીકો વચ્ચે તફાવત આપો.}

\begin{solutionbox}
\begin{center}
\captionof{table}{White Hat vs Black Hat SEO}
\begin{tabulary}{\linewidth}{|L|L|}
\hline
\textbf{White Hat SEO} & \textbf{Black Hat SEO} \\ \hline
\textbf{નૈતિક પ્રથાઓ} (માર્ગદર્શિકા મુજબ) & \textbf{છેતરપિંડીયુક્ત યુક્તિઓ} (નિયમો વિરુદ્ધ) \\ \hline
\textbf{ગુણવત્તાયુક્ત સામગ્રી} સર્જન & \textbf{સામગ્રીની ચોરી} અને ડુપ્લિકેશન \\ \hline
\textbf{કુદરતી લિંક બિલ્ડિંગ} & \textbf{લિંક ફાર્મ્સ} અને પેઇડ લિંક્સ \\ \hline
\textbf{લાંબા ગાળાના પરિણામો} & \textbf{ઝડપી પરંતુ જોખમી} લાભ \\ \hline
\textbf{સર્ચ એન્જિન દ્વારા માન્ય} & \textbf{પેનલ્ટીનું જોખમ} \\ \hline
\end{tabulary}
\end{center}
\end{solutionbox}

\begin{mnemonicbox}
\mnemonic{GOOD vs BAD: Guidelines-following vs Penalty-risking}
\end{mnemonicbox}

\questionmarks{2(c) OR}{7}{કોઈપણ ત્રણ સામાન્ય SEO ટૂલ્સના નામ આપો અને તેમના કાર્યોનું વર્ણન કરો.}

\begin{solutionbox}
\begin{center}
\captionof{table}{સામાન્ય SEO ટૂલ્સ}
\begin{tabulary}{\linewidth}{|L|L|L|}
\hline
\textbf{SEO Tool} & \textbf{પ્રાથમિક કાર્યો} & \textbf{મુખ્ય વિશેષતાઓ} \\ \hline
\textbf{Google Analytics} & વેબસાઇટ ટ્રાફિક વિશ્લેષણ & મુલાકાતી વર્તન, કન્વર્ઝન ટ્રેકિંગ \\ \hline
\textbf{SEMrush} & કીવર્ડ અને હરીફ વિશ્લેષણ & કીવર્ડ મુશ્કેલી, બેકલિંક વિશ્લેષણ \\ \hline
\textbf{Yoast SEO} & ઓન-પેજ ઓપ્ટિમાઇઝેશન & કન્ટેન્ટ ઓપ્ટિમાઇઝેશન, ટેકનિકલ SEO \\ \hline
\end{tabulary}
\end{center}

\textbf{વિગતવાર કાર્યો:}

\begin{itemize}
    \item \keyword{Google Analytics}: વપરાશકર્તાની મુસાફરી, બાઉન્સ રેટ અને ગોલ કમ્પ્લીશન ટ્રેક કરે છે
    \item \keyword{SEMrush}: રેન્કિંગની તકો ઓળખે છે અને હરીફોની વ્યૂહરચના પર નજર રાખે છે
    \item \keyword{Yoast SEO}: વર્ડપ્રેસ માટે રીઅલ-ટાઇમ કન્ટેન્ટ અને મેટા ટેગ સૂચનો આપે છે
\end{itemize}

\textbf{લાભો:}

\begin{itemize}
    \item \keyword{Data-driven decisions}: વ્યાપક એનાલિટિક્સ દ્વારા નિર્ણયો
    \item \keyword{Competitive advantage}: બજારની આંતરદૃષ્ટિ સાથે
    \item \keyword{Efficiency}: ઓપ્ટિમાઇઝેશન કાર્યોમાં કાર્યક્ષમતા
\end{itemize}
\end{solutionbox}

\begin{mnemonicbox}
\mnemonic{GSY Tools: Google Analytics, SEMrush, Yoast}
\end{mnemonicbox}

\questionmarks{3(a)}{3}{કોઈપણ એક મલ્ટી-ટચ એટ્રિબ્યુશન મોડલ ઉદાહરણ સાથે સમજાવો.}

\begin{solutionbox}
\textbf{Linear Attribution Model} ગ્રાહક મુસાફરીના તમામ ટચપોઇન્ટ્સ પર સમાન રીતે ક્રેડિટ વિતરિત કરે છે.

\textbf{ઉદાહરણ:}
ગ્રાહક મુસાફરી: Social Media Ad $\rightarrow$ Email $\rightarrow$ Website Visit $\rightarrow$ Purchase

\textbf{ક્રેડિટ વિતરણ:}

\begin{itemize}
    \item Social Media Ad: 25\%
    \item Email: 25\%
    \item Website Visit: 25\%
    \item Purchase Page: 25\%
\end{itemize}
\end{solutionbox}

\begin{mnemonicbox}
\mnemonic{EQUAL Credit for ALL Touches: Linear = Equal distribution}
\end{mnemonicbox}

\questionmarks{3(b)}{4}{નીચેના મુખ્ય મેટ્રિક્સ સમજાવો: યુનિક વિઝિટર્સ, બાઉન્સ રેટ.}

\begin{solutionbox}
\begin{center}
\captionof{table}{મુખ્ય વેબ મેટ્રિક્સ}
\begin{tabulary}{\linewidth}{|L|L|L|}
\hline
\textbf{મેટ્રિક} & \textbf{વ્યાખ્યા} & \textbf{મહત્વ} \\ \hline
\textbf{Unique Visitors} & ચોક્કસ સમયગાળામાં વેબસાઇટની મુલાકાત લેનારા વ્યક્તિગત વપરાશકર્તાઓ & \textbf{પ્રેક્ષક પહોંચ} અને વૃદ્ધિ માપે છે \\ \hline
\textbf{Bounce Rate} & માત્ર એક પેજ જોઈને જતા રહેતા મુલાકાતીઓની ટકાવારી & \textbf{સામગ્રીની સુસંગતતા} દર્શાવે છે \\ \hline
\end{tabulary}
\end{center}

\textbf{ઓપ્ટિમાઇઝેશન ટિપ્સ:}

\begin{itemize}
    \item \keyword{Unique Visitors}: SEO અને સોશિયલ મીડિયા દ્વારા વધારો
    \item \keyword{Bounce Rate}: સારી સામગ્રી અને નેવિગેશનથી સુધારો
\end{itemize}
\end{solutionbox}

\begin{mnemonicbox}
\mnemonic{UV-BR: Unique Visitors measure reach, Bounce Rate measures engagement}
\end{mnemonicbox}

\questionmarks{3(c)}{7}{નીચેના ટ્રેકિંગ કોડ તેમના ફાયદા અને ગેરફાયદા સાથે સમજાવો: લોંગ ટ્રેકિંગ કોડ, UTM કોડ.}

\begin{solutionbox}
\begin{center}
\captionof{table}{ટ્રેકિંગ કોડ સરખામણી}
\begin{tabulary}{\linewidth}{|L|L|L|L|}
\hline
\textbf{ટ્રેકિંગ કોડ} & \textbf{વર્ણન} & \textbf{ફાયદા} & \textbf{ગેરફાયદા} \\ \hline
\textbf{Long Tracking Code} & વિસ્તૃત માહિતી સાથેના વિગતવાર પેરામીટર્સ & \textbf{વ્યાપક ડેટા}, \textbf{વિગતવાર આંતરદૃષ્ટિ} & \textbf{જટિલ URLs}, \textbf{વપરાશકર્તા માટે અગવડભર્યું} \\ \hline
\textbf{UTM Code} & ઝુંબેશ ટ્રેકિંગ માટે Urchin Tracking Module & \textbf{સરળ અમલીકરણ}, \textbf{ઝુંબેશ-વિશિષ્ટ} & \textbf{મર્યાદિત ડેટા}, \textbf{મેન્યુઅલ મેનેજમેન્ટ} \\ \hline
\end{tabulary}
\end{center}

\textbf{UTM Parameters:}

\begin{itemize}
    \item \code{utm\_source}: ટ્રાફિક સ્ત્રોત (google, facebook)
    \item \code{utm\_medium}: માર્કેટિંગ માધ્યમ (cpc, email)
    \item \code{utm\_campaign}: ઝુંબેશ નામ (summer\_sale)
\end{itemize}

\textbf{શ્રેષ્ઠ પ્રયાસો:}

\begin{itemize}
    \item \keyword{Consistent naming}: સુસંગત નામકરણ પદ્ધતિઓ
    \item \keyword{URL shortening}: લાંબા કોડ્સ માટે URL શોર્ટનર વાપરો
    \item \keyword{Regular monitoring}: ઝુંબેશ પ્રદર્શનનું નિયમિત નિરીક્ષણ
\end{itemize}
\end{solutionbox}

\begin{mnemonicbox}
\mnemonic{LONG vs SHORT: Comprehensive vs Simple tracking}
\end{mnemonicbox}

\questionmarks{3(a) OR}{3}{કોઈપણ એક સિંગલ-ટચ એટ્રિબ્યુશન મોડલ ઉદાહરણ સાથે સમજાવો.}

\begin{solutionbox}
\textbf{Last-Click Attribution Model} કન્વર્ઝન પહેલાના છેલ્લા ટચપોઇન્ટને 100\% ક્રેડિટ આપે છે.

\textbf{ઉદાહરણ:}
ગ્રાહક મુસાફરી: Social Media $\rightarrow$ Email $\rightarrow$ Google Search $\rightarrow$ Purchase

\textbf{ક્રેડિટ વિતરણ:}

\begin{itemize}
    \item Google Search: 100\%
    \item અન્ય ટચપોઇન્ટ્સ: 0\%
\end{itemize}

\textbf{ઉપયોગ}: તાત્કાલિક કન્વર્ઝન ડ્રાઇવરો પર ધ્યાન કેન્દ્રિત કરતું સરળ ઈ-કોમર્સ ટ્રેકિંગ.
\end{solutionbox}

\begin{mnemonicbox}
\mnemonic{LAST WINS ALL: Final touchpoint gets full credit}
\end{mnemonicbox}

\questionmarks{3(b) OR}{4}{નીચેના મુખ્ય મેટ્રિક્સ સમજાવો: પેજવ્યૂઝ, ન્યૂ વિઝિટ્સ.}

\begin{solutionbox}
\begin{center}
\captionof{table}{મેટ્રિક્સ વિશ્લેષણ}
\begin{tabulary}{\linewidth}{|L|L|L|}
\hline
\textbf{મેટ્રિક} & \textbf{વ્યાખ્યા} & \textbf{માપન મૂલ્ય} \\ \hline
\textbf{Pageviews} & પુનરાવર્તિત દૃશ્યો સહિત જોવાયેલા કુલ પૃષ્ઠો & \textbf{સામગ્રી લોકપ્રિયતા} અને સાઇટ વપરાશ \\ \hline
\textbf{New Visits} & વેબસાઇટ પર પ્રથમ વખતના મુલાકાતીઓની ટકાવારી & \textbf{પ્રેક્ષક વૃદ્ધિ} અને વિસ્તરણ \\ \hline
\end{tabulary}
\end{center}

\textbf{વિશ્લેષણ મહત્વ:}

\begin{itemize}
    \item \keyword{Pageviews}: ઉચ્ચ સંખ્યા આકર્ષક સામગ્રી સૂચવે છે
    \item \keyword{New Visits}: વૃદ્ધિ અસરકારક માર્કેટિંગ આઉટરીચ દર્શાવે છે
\end{itemize}
\end{solutionbox}

\begin{mnemonicbox}
\mnemonic{PN Metrics: Pageviews for engagement, New visits for growth}
\end{mnemonicbox}

\questionmarks{3(c) OR}{7}{વિવિધ પ્રકારના વેબ એનાલિટિક્સ ટૂલ્સનું વર્ણન કરો.}

\begin{solutionbox}
\begin{center}
\captionof{table}{Web Analytics Tools}
\begin{tabulary}{\linewidth}{|L|L|L|L|}
\hline
\textbf{ટૂલ કેટેગરી} & \textbf{હેતુ} & \textbf{ઉદાહરણો} & \textbf{મુખ્ય વિશેષતાઓ} \\ \hline
\textbf{Content Analytics} & કન્ટેન્ટ પ્રદર્શન ટ્રેકિંગ & Google Analytics & પેજ વ્યૂઝ, ટાઈમ ઓન પેજ \\ \hline
\textbf{Customer Analytics} & વપરાશકર્તા વર્તન વિશ્લેષણ & Hotjar, Crazy Egg & હીટમેપ્સ, રેકોર્ડિંગ્સ \\ \hline
\textbf{SEO Analytics} & સર્ચ ઓપ્ટિમાઇઝેશન & SEMrush, Ahrefs & કીવર્ડ્સ, બેકલિંક્સ \\ \hline
\textbf{Social Analytics} & સોશિયલ પ્રદર્શન & FB Insights & એન્ગેજમેન્ટ, રીચ \\ \hline
\textbf{A/B Testing} & કન્વર્ઝન ઓપ્ટિમાઇઝેશન & Optimizely & સ્પ્લિટ ટેસ્ટિંગ \\ \hline
\end{tabulary}
\end{center}

\textbf{પસંદગી માપદંડ:}

\begin{itemize}
    \item \keyword{Business objectives}: વ્યવસાયિક ઉદ્દેશ્યો સાથે સંરેખણ
    \item \keyword{Integration}: વર્તમાન ટૂલ્સ સાથે એકીકરણ ક્ષમતાઓ
    \item \keyword{Cost-effectiveness}: સંસ્થાના કદ માટે ખર્ચ-અસરકારકતા
\end{itemize}
\end{solutionbox}

\begin{mnemonicbox}
\mnemonic{CCSSA Analytics: Content, Customer, SEO, Social, A/B testing}
\end{mnemonicbox}

\questionmarks{4(a)}{3}{સોશિયલ મીડિયા માર્કેટિંગ સમજાવો.}

\begin{solutionbox}
સોશિયલ મીડિયા માર્કેટિંગ ઉત્પાદનોને પ્રોત્સાહન આપવા અને પ્રેક્ષકોને જોડવા માટે સોશિયલ પ્લેટફોર્મના ઉપયોગ કરે છે.

\textbf{મુખ્ય તત્વો:}

\begin{itemize}
    \item \keyword{Content creation}: લક્ષ્ય પ્રેક્ષકોના જોડાણ માટે
    \item \keyword{Community building}: સતત ક્રિયાપ્રતિક્રિયા દ્વારા
    \item \keyword{Brand awareness}: ઓર્ગેનિક અને પેઇડ વ્યૂહરચનાઓ દ્વારા
    \item \keyword{Customer support}: સોશિયલ ચેનલ્સ દ્વારા
\end{itemize}
\end{solutionbox}

\begin{mnemonicbox}
\mnemonic{CCBC: Content, Community, Brand awareness, Customer support}
\end{mnemonicbox}

\questionmarks{4(b)}{4}{ઇન્સ્ટાગ્રામ જાહેરાતોના પ્રકારો સમજાવો.}

\begin{solutionbox}
\begin{center}
\captionof{table}{Instagram Ad Types}
\begin{tabulary}{\linewidth}{|L|L|L|}
\hline
\textbf{Ad Type} & \textbf{ફોર્મેટ} & \textbf{શ્રેષ્ઠ ઉપયોગ} \\ \hline
\textbf{Photo Ads} & કેપ્શન સાથે સિંગલ ઇમેજ & \textbf{ઉત્પાદન પ્રદર્શન} અને જાગૃતિ \\ \hline
\textbf{Video Ads} & ટૂંકી વિડિઓ સામગ્રી & \textbf{સ્ટોરીટેલિંગ} અને જોડાણ \\ \hline
\textbf{Carousel Ads} & એકથી વધુ છબીઓ/વિડિઓઝ & \textbf{ઉત્પાદન કેટલોગ} \\ \hline
\textbf{Stories Ads} & પૂર્ણ-સ્ક્રીન વર્ટિકલ & \textbf{તાત્કાલિક પગલાં} \\ \hline
\end{tabulary}
\end{center}
\end{solutionbox}

\begin{mnemonicbox}
\mnemonic{PVCS Instagram: Photo, Video, Carousel, Stories}
\end{mnemonicbox}

\questionmarks{4(c)}{7}{યૂટ્યુબ માર્કેટિંગ સમજાવો અને ડિજિટલ માર્કેટિંગમાં તેનું મહત્વ સમજાવો.}

\begin{solutionbox}
યૂટ્યુબ માર્કેટિંગ બ્રાન્ડ પ્રમોશન અને પ્રેક્ષક જોડાણ માટે વિડિઓ સામગ્રીનો લાભ લે છે.

\begin{center}
\begin{tikzpicture}[node distance=1.5cm, auto]
    \node [gtu block] (YT) {યૂટ્યુબ માર્કેટિંગ};
    \node [gtu block, above right=1cm and 1cm of YT] (Content) {કન્ટેન્ટ વ્યૂહરચના};
    \node [gtu block, below right=1cm and 1cm of YT] (Channel) {ચેનલ ઓપ્ટિમાઇઝેશન};
    \node [gtu block, below left=1cm and 1cm of YT] (SEO) {વિડિઓ SEO};
    \node [gtu block, above left=1cm and 1cm of YT] (Analytics) {એનાલિટિક્સ અને વૃદ્ધિ};

    \path [gtu arrow] (YT) -- (Content);
    \path [gtu arrow] (YT) -- (Channel);
    \path [gtu arrow] (YT) -- (SEO);
    \path [gtu arrow] (YT) -- (Analytics);
\end{tikzpicture}
\captionof{figure}{YouTube Marketing Components}
\end{center}

\begin{center}
\captionof{table}{યૂટ્યુબ માર્કેટિંગ વ્યૂહરચનાઓ}
\begin{tabulary}{\linewidth}{|L|L|L|}
\hline
\textbf{ઘટક} & \textbf{વ્યૂહરચના} & \textbf{મહત્વ} \\ \hline
\textbf{Content Strategy} & શૈક્ષણિક, મનોરંજક વિડિઓઝ & \textbf{પ્રેક્ષક જોડાણ} અને મૂલ્ય \\ \hline
\textbf{Channel Optimization} & બ્રાન્ડિંગ, પ્લેલિસ્ટ્સ & \textbf{વ્યાવસાયિક હાજરી} \\ \hline
\textbf{Video SEO} & કીવર્ડ્સ, થમ્બનેલ્સ & \textbf{સર્ચ દૃશ્યતા} \\ \hline
\textbf{YouTube Ads} & TrueView, બમ્પર ads & \textbf{લક્ષિત પ્રમોશન} \\ \hline
\end{tabulary}
\end{center}

\textbf{ડિજિટલ માર્કેટિંગમાં મહત્વ:}

\begin{itemize}
    \item \keyword{Visual storytelling}: ભાવનાત્મક જોડાણો બનાવે છે
    \item \keyword{Search engine benefits}: યૂટ્યુબ બીજું સૌથી મોટું સર્ચ એન્જિન છે
    \item \keyword{Cross-platform integration}: અન્ય માર્કેટિંગ ચેનલો સાથે એકીકરણ
    \item \keyword{Cost-effective}: પરંપરાગત ટીવી જાહેરાતોની તુલનામાં
\end{itemize}
\end{solutionbox}

\begin{mnemonicbox}
\mnemonic{CCVA Success: Content, Channel, Video SEO, Ads for YouTube success}
\end{mnemonicbox}

\questionmarks{4(a) OR}{3}{માર્કેટિંગ વ્યૂહરચનાઓની સફળતાને ટ્રેક કરવા માટે ઇન્સ્ટાગ્રામ પર ઉપલબ્ધ મેટ્રિક્સની સૂચિ બનાવો.}

\begin{solutionbox}
\textbf{Instagram Analytics Metrics:}

\begin{itemize}
    \item \keyword{Reach}: કન્ટેન્ટ જોનારા અનન્ય એકાઉન્ટ્સની સંખ્યા
    \item \keyword{Impressions}: પુનરાવર્તનો સહિત કુલ કન્ટેન્ટ દૃશ્યો
    \item \keyword{Engagement Rate}: લાઈક્સ, કોમેન્ટ્સ, શેરની ટકાવારી
    \item \keyword{Profile Visits}: બિઝનેસ પ્રોફાઇલ પર આવતો ટ્રાફિક
    \item \keyword{Website Clicks}: બાહ્ય વેબસાઇટ પર જતો ટ્રાફિક
    \item \keyword{Story Completion Rate}: સંપૂર્ણ સ્ટોરીઝ જોનાર આની ટકાવારી
\end{itemize}
\end{solutionbox}

\begin{mnemonicbox}
\mnemonic{RIEPSW: Reach, Impressions, Engagement, Profile visits, Story completion, Website clicks}
\end{mnemonicbox}

\questionmarks{4(b) OR}{4}{યૂટ્યુબ જાહેરાતોના પ્રકારો સમજાવો.}

\begin{solutionbox}
\begin{center}
\captionof{table}{YouTube Ad Types}
\begin{tabulary}{\linewidth}{|L|L|L|L|}
\hline
\textbf{Ad Type} & \textbf{ફોર્મેટ} & \textbf{અવધિ} & \textbf{શ્રેષ્ઠ માટે} \\ \hline
\textbf{TrueView In-Stream} & સ્કિપેબલ વિડિઓ & 12s+ & \textbf{બ્રાન્ડ જાગૃતિ} \\ \hline
\textbf{TrueView Discovery} & થમ્બનેલ + ટેક્સ્ટ & ચલિત & \textbf{કન્ટેન્ટ પ્રમોશન} \\ \hline
\textbf{Bumper Ads} & નોન-સ્કિપેબલ & 6s & \textbf{ઝડપી મેસેજિંગ} \\ \hline
\textbf{Overlay Ads} & બેનર & સ્થિર & \textbf{વેબસાઇટ ટ્રાફિક} \\ \hline
\end{tabulary}
\end{center}
\end{solutionbox}

\begin{mnemonicbox}
\mnemonic{TTBO YouTube: TrueView In-stream, TrueView Discovery, Bumper, Overlay}
\end{mnemonicbox}

\questionmarks{4(c) OR}{7}{ફેસબુક જાહેરાતમાં ઉપલબ્ધ લક્ષ્યીકરણ વિકલ્પોનું વર્ણન કરો.}

\begin{solutionbox}
ફેસબુક ચોક્કસ પ્રેક્ષક પહોંચ માટે વ્યાપક લક્ષ્યીકરણ આપે છે:

\begin{center}
\captionof{table}{Facebook Targeting Options}
\begin{tabulary}{\linewidth}{|L|L|L|}
\hline
\textbf{લક્ષ્યીકરણ શ્રેણી} & \textbf{વિકલ્પો} & \textbf{હેતુ} \\ \hline
\textbf{Demographics} & ઉંમર, લિંગ, શિક્ષણ & \textbf{મૂળભૂત પ્રેક્ષક} વ્યાખ્યા \\ \hline
\textbf{Location} & દેશો, શહેરો & \textbf{ભૌગોલિક} લક્ષ્યીકરણ \\ \hline
\textbf{Interests} & શોખ, પ્રવૃત્તિઓ & \textbf{વર્તણૂકલક્ષી} લક્ષ્યીકરણ \\ \hline
\textbf{Behaviors} & ખરીદી ઇતિહાસ & \textbf{ક્રિયા-આધારિત} લક્ષ્યીકરણ \\ \hline
\textbf{Custom Audiences} & વેબસાઇટ મુલાકાતીઓ & \textbf{રિટાર્ગેટિંગ} \\ \hline
\textbf{Lookalike Audiences} & ગ્રાહકો જેવા & \textbf{પ્રેક્ષક વિસ્તરણ} \\ \hline
\end{tabulary}
\end{center}

\textbf{ઝુંબેશ ઓપ્ટિમાઇઝેશન:}

\begin{itemize}
    \item \keyword{Narrow targeting}: ચોક્કસ ઉત્પાદનો માટે
    \item \keyword{Broad targeting}: બ્રાન્ડ જાગૃતિ માટે
    \item \keyword{Dynamic audiences}: વપરાશકર્તા વર્તનના આધારે
\end{itemize}

\textbf{પ્રદર્શન લાભો:}

\begin{itemize}
    \item \keyword{Higher conversion}: ચોકસાઇ દ્વારા ઉચ્ચ રૂપાંતરણ
    \item \keyword{Cost efficiency}: સુસંગત પ્રેક્ષકો સાથે ખર્ચ કાર્યક્ષમતા
    \item \keyword{Scalable growth}: લુકઅલાઇક વિસ્તરણ દ્વારા વૃદ્ધિ
\end{itemize}
\end{solutionbox}

\begin{mnemonicbox}
\mnemonic{DLIBCL Targeting: Demographics, Location, Interests, Behaviors, Custom, Lookalike}
\end{mnemonicbox}

\questionmarks{5(a)}{3}{LinkedIn માર્કેટિંગનો ખ્યાલ સમજાવો.}

\begin{solutionbox}
LinkedIn માર્કેટિંગ વ્યવસાયિક નેટવર્કિંગ અને B2B સંબંધ નિર્માણ પર ધ્યાન કેન્દ્રિત કરે છે.

\textbf{મુખ્ય ખ્યાલો:}

\begin{itemize}
    \item \keyword{Professional audience}: B2B વેચાણ માટે લક્ષ્યીકરણ
    \item \keyword{Thought leadership}: ઇન્ડસ્ટ્રી કન્ટેન્ટ દ્વારા
    \item \keyword{Network expansion}: કનેક્શન્સ અને જૂથો દ્વારા
    \item \keyword{Lead generation}: લક્ષિત ઝુંબેશ દ્વારા
\end{itemize}
\end{solutionbox}

\begin{mnemonicbox}
\mnemonic{PTNL: Professional, Thought leadership, Network, Leads}
\end{mnemonicbox}

\questionmarks{5(b)}{4}{વિવિધ પ્રકારના ઇમેઇલ માર્કેટિંગ ઝુંબેશ સમજાવો.}

\begin{solutionbox}
\begin{center}
\captionof{table}{Email Marketing Campaigns}
\begin{tabulary}{\linewidth}{|L|L|L|}
\hline
\textbf{Campaign Type} & \textbf{હેતુ} & \textbf{સમય} \\ \hline
\textbf{Welcome Series} & નવા સબ્સ્ક્રાઇબર ઓનબોર્ડિંગ & સાઇન અપ પછી \textbf{તરત જ} \\ \hline
\textbf{Newsletter} & નિયમિત અપડેટ્સ & \textbf{સાપ્તાહિક/માસિક} \\ \hline
\textbf{Promotional} & વેચાણ અને ઑફર્સ & \textbf{ઇવેન્ટ-આધારિત} \\ \hline
\textbf{Abandoned Cart} & અધૂરી ખરીદી રિકવરી & \textbf{24-48 કલાક} પછી \\ \hline
\end{tabulary}
\end{center}
\end{solutionbox}

\begin{mnemonicbox}
\mnemonic{WNPA Emails: Welcome, Newsletter, Promotional, Abandoned cart}
\end{mnemonicbox}

\questionmarks{5(c)}{7}{Google Ads ઝુંબેશના વિવિધ પ્રકારો સમજાવો.}

\begin{solutionbox}
Google Ads વિવિધ માર્કેટિંગ ઉદ્દેશ્યો માટે બહુવિધ ઝુંબેશ પ્રકારો પ્રદાન કરે છે:

\begin{center}
\captionof{table}{Google Ads Campaign Types}
\begin{tabulary}{\linewidth}{|L|L|L|L|}
\hline
\textbf{ઝુંબેશ પ્રકાર} & \textbf{પ્લેટફોર્મ} & \textbf{Ad ફોર્મેટ} & \textbf{શ્રેષ્ઠ માટે} \\ \hline
\textbf{Search} & Google Search & Text ads & \textbf{ઉચ્ચ-ઇરાદા} કીવર્ડ્સ \\ \hline
\textbf{Display} & ભાગીદાર સાઇટ્સ & Banner ads & \textbf{બ્રાન્ડ જાગૃતિ} \\ \hline
\textbf{Video} & YouTube & Video ads & \textbf{જોડાણ} \\ \hline
\textbf{Shopping} & Google Shopping & Product listings & \textbf{ઇ-કોમર્સ} વેચાણ \\ \hline
\textbf{App} & બહુવિધ & Automated & \textbf{એપ્લિકેશન ડાઉનલોડ્સ} \\ \hline
\textbf{Smart} & ઓટોમેટેડ & મિશ્ર & \textbf{નાના વ્યવસાયો} \\ \hline
\end{tabulary}
\end{center}

\begin{center}
\begin{tikzpicture}[node distance=1.5cm, auto]
    \node [gtu state] (Obj) {માર્કેટિંગ ઉદ્દેશ્ય};
    \node [gtu decision, right=1cm of Obj] (Type) {ઝુંબેશ પ્રકાર};
    \node [gtu block, above right=0.5cm and 1cm of Type] (Search) {સર્ચ};
    \node [gtu block, right=1cm of Type] (Display) {ડિસ્પ્લે};
    \node [gtu block, below right=0.5cm and 1cm of Type] (Video) {વિડિઓ};
    \node [gtu block, above=0.5cm of Search] (Shopping) {શોપિંગ};
    
    \node [gtu block, right=0.5cm of Search] {Intent};
    \node [gtu block, right=0.5cm of Display] {Aware};
    \node [gtu block, right=0.5cm of Video] {Engage};
    \node [gtu block, right=0.5cm of Shopping] {Sales};

    \path [gtu arrow] (Obj) -- (Type);
    \path [gtu arrow] (Type) -- (Search);
    \path [gtu arrow] (Type) -- (Display);
    \path [gtu arrow] (Type) -- (Video);
    \path [gtu arrow] (Type) -- (Shopping);
\end{tikzpicture}
\captionof{figure}{Campaign Selection Flow}
\end{center}

\textbf{ઓપ્ટિમાઇઝેશન વ્યૂહરચનાઓ:}

\begin{itemize}
    \item \keyword{Keyword research}: સર્ચ ઝુંબેશ માટે
    \item \keyword{Audience targeting}: ડિસ્પ્લે ઝુંબેશ માટે
    \item \keyword{Conversion tracking}: ROI માપન માટે
\end{itemize}
\end{solutionbox}

\begin{mnemonicbox}
\mnemonic{SDVSAS Google: Search, Display, Video, Shopping, App, Smart campaigns}
\end{mnemonicbox}

\questionmarks{5(a) OR}{3}{Twitter માર્કેટિંગનો ખ્યાલ સમજાવો.}

\begin{solutionbox}
Twitter માર્કેટિંગ બ્રાન્ડ જોડાણ અને ગ્રાહક સેવા માટે રીઅલ-ટાઇમ સંચારનો ઉપયોગ કરે છે.

\textbf{મુખ્ય તત્વો:}

\begin{itemize}
    \item \keyword{Real-time emgagement}: ટ્રેન્ડિંગ વિષયો સાથે
    \item \keyword{Customer support}: સીધા પ્રતિસાદ દ્વારા
    \item \keyword{Content amplification}: રીટ્વીટ અને હેશટેગ્સ દ્વારા
    \item \keyword{Influencer partnerships}: વિસ્તૃત પહોંચ માટે
\end{itemize}
\end{solutionbox}

\begin{mnemonicbox}
\mnemonic{RCCI Twitter: Real-time, Customer support, Content amplification, Influencer partnerships}
\end{mnemonicbox}

\questionmarks{5(b) OR}{4}{SEO અને PPC વચ્ચે તફાવત આપો.}

\begin{solutionbox}
\begin{center}
\captionof{table}{SEO vs PPC}
\begin{tabulary}{\linewidth}{|L|L|}
\hline
\textbf{SEO (સર્ચ એન્જિન ઓપ્ટિમાઇઝેશન)} & \textbf{PPC (પે-પર-ક્લિક)} \\ \hline
\textbf{ઓર્ગેનિક પરિણામો} & \textbf{પેઇડ જાહેરાત} \\ \hline
\textbf{લાંબા ગાળાની વ્યૂહરચના} (3-6 મહિના) & \textbf{તાત્કાલિક પરિણામો} (કલાકોમાં) \\ \hline
\textbf{કોઈ સીધો ખર્ચ નહીં} & \textbf{ક્લિક દીઠ ખર્ચ} \\ \hline
\textbf{ટકાઉ ટ્રાફિક} વૃદ્ધિ & બજેટ પૂરું થતાં \textbf{ટ્રાફિક બંધ} \\ \hline
\textbf{વિશ્વાસ અને વિશ્વસનીયતા} & \textbf{ઓછો વિશ્વાસ} (જાહેરાતો) \\ \hline
\textbf{ચાલુ પ્રયત્નો} જરૂરી & \textbf{સતત બજેટ} જરૂરી \\ \hline
\end{tabulary}
\end{center}
\end{solutionbox}

\begin{mnemonicbox}
\mnemonic{OLNSTN vs PICRCR: Organic, Long-term, No cost vs Paid, Immediate, Cost-per-click}
\end{mnemonicbox}

\questionmarks{5(c) OR}{7}{Google Ads માં ઉપલબ્ધ વિવિધ બિડિંગ વ્યૂહરચનાઓ સમજાવો.}

\begin{solutionbox}
Google Ads વિવિધ ઝુંબેશ લક્ષ્યો માટે બહુવિધ બિડિંગ વ્યૂહરચનાઓ પ્રદાન કરે છે:

\begin{center}
\captionof{table}{Google Ads Bidding Strategies}
\begin{tabulary}{\linewidth}{|L|L|L|L|}
\hline
\textbf{બિડિંગ વ્યૂહરચના} & \textbf{પ્રકાર} & \textbf{લક્ષ્ય} & \textbf{શ્રેષ્ઠ માટે} \\ \hline
\textbf{Manual CPC} & મેન્યુઅલ & \textbf{ટ્રાફિક નિયંત્રણ} & \textbf{અનુભવી જાહેરાતકર્તાઓ} \\ \hline
\textbf{Enhanced CPC} & અર્ધ-સ્વચાલિત & \textbf{કન્વર્ઝન ઓપ્ટિમાઇઝેશન} & \textbf{સંતુલિત નિયંત્રણ} \\ \hline
\textbf{Target CPA} & સ્વચાલિત & \textbf{ખર્ચ પ્રતિ સંપાદન} & \textbf{લીડ જનરેશન} \\ \hline
\textbf{Target ROAS} & સ્વચાલિત & \textbf{જાહેરાત ખર્ચ પર વળતર} & \textbf{ઇ-કોમર્સ વેચાણ} \\ \hline
\textbf{Maximize Clicks} & સ્વચાલિત & \textbf{ટ્રાફિક વોલ્યુમ} & \textbf{બ્રાન્ડ જાગૃતિ} \\ \hline
\textbf{Maximize Conversions} & સ્વચાલિત & \textbf{કન્વર્ઝન વોલ્યુમ} & \textbf{ઝુંબેશ સ્કેલિંગ} \\ \hline
\end{tabulary}
\end{center}

\begin{center}
\begin{tikzpicture}[node distance=1.5cm, auto]
    \node [gtu state] (Goal) {લક્ષ્ય};
    
    \node [gtu block, below left=1cm and 2cm of Goal] (Traffic) {ટ્રાફિક};
    \node [gtu block, left=3cm of Goal] (Leads) {લીડ્સ};
    \node [gtu block, right=3cm of Goal] (Sales) {વેચાણ};
    \node [gtu block, below right=1cm and 2cm of Goal] (Control) {નિયંત્રણ};

    \node [gtu decision, below=0.5cm of Traffic] (MaxClick) {Max Clicks};
    \node [gtu decision, below=0.5cm of Leads] (CPA) {Target CPA};
    \node [gtu decision, below=0.5cm of Sales] (ROAS) {Target ROAS};
    \node [gtu decision, below=0.5cm of Control] (Manual) {Manual CPC};

    \path [gtu arrow] (Goal) -- (Traffic);
    \path [gtu arrow] (Goal) -- (Leads);
    \path [gtu arrow] (Goal) -- (Sales);
    \path [gtu arrow] (Goal) -- (Control);

    \path [gtu arrow] (Traffic) -- (MaxClick);
    \path [gtu arrow] (Leads) -- (CPA);
    \path [gtu arrow] (Sales) -- (ROAS);
    \path [gtu arrow] (Control) -- (Manual);
\end{tikzpicture}
\captionof{figure}{Bidding Strategy Selection}
\end{center}

\textbf{અમલીકરણ માર્ગદર્શિકા:}

\begin{itemize}
    \item \keyword{Manual CPC}: બિડ એડજસ્ટમેન્ટ અને કીવર્ડ-સ્તર નિયંત્રણ સાથે પ્રારંભ કરો
    \item \keyword{Target CPA}: ઐતિહાસિક કન્વર્ઝન ડેટાના આધારે સેટ કરો
    \item \keyword{Target ROAS}: પર્યાપ્ત કન્વર્ઝન ટ્રેકિંગ ડેટાની જરૂર છે
\end{itemize}
\end{solutionbox}

\begin{mnemonicbox}
\mnemonic{METMM Bidding: Manual, Enhanced, Target CPA, Target ROAS, Maximize clicks, Maximize conversions}
\end{mnemonicbox}

\end{document}
