\documentclass{article}

% content/resources/templates/preamble.tex
\usepackage[margin=0.6in]{geometry}
\author{Milav Dabgar}
\usepackage{amsmath,amssymb,amsthm}
\usepackage{booktabs}
\usepackage{multirow}
\usepackage{xcolor}
\usepackage{tcolorbox}
\tcbuselibrary{breakable,skins}
\usepackage[colorlinks=true,linkcolor=blue]{hyperref}
\usepackage{titlesec}
\usepackage{enumitem}
\usepackage{tikz}
\usepackage{pgfplots}
\usepackage{circuitikz}
\usepackage[version=4]{mhchem}
\usepackage{longtable}
\usepackage{array}
\usepackage{float}
\usepackage{caption}
\usepackage{listings}

\lstset{
  basicstyle=\small\ttfamily,
  breaklines=true,
  breakatwhitespace=false,
  postbreak=\mbox{\textcolor{red}{$\hookrightarrow$}\space},
  float=false,
  numbers=left,
  numberstyle=\tiny\color{gray},
  numbersep=10pt,
  xleftmargin=2em,
  keywordstyle=\color{blue},
  commentstyle=\color{green!60!black},
  stringstyle=\color{purple},
  backgroundcolor=\color{gray!5},
  showstringspaces=false,
  tabsize=2,
  captionpos=b,
  keepspaces=true,
  columns=flexible
}

\pgfplotsset{compat=1.18}
\usetikzlibrary{shapes,arrows,positioning,calc,patterns,decorations.pathmorphing,decorations.markings,arrows.meta}

% Color scheme
\definecolor{headcolor}{RGB}{0,102,204}
\definecolor{keycolor}{RGB}{220,20,60}
\definecolor{solutioncolor}{RGB}{34,139,34}
\definecolor{mnemoniccolor}{RGB}{148,0,211}
\definecolor{codecolor}{RGB}{0,0,100}

% Spacing
\setlength{\parskip}{3pt}
\setlist[itemize]{nosep}
\setlist[enumerate]{nosep}

% Title formatting
\titleformat{\section}{\Large\bfseries\color{headcolor}}{\thesection}{1em}{}
\titleformat{\subsection}{\large\bfseries\color{headcolor}}{\thesubsection}{1em}{}

% Pandoc tightlist compatibility
\providecommand{\tightlist}{%
  \setlength{\itemsep}{0pt}\setlength{\parskip}{0pt}}

% Pandoc longtable compatibility
\newcounter{none}
\def\thenone{}


% content/resources/templates/gujarati-boxes.tex
\usepackage{fontspec}
\usepackage{polyglossia}

% Set Gujarati as main language (document is primarily in Gujarati)
% Note: gloss-gujarati.ldf doesn't exist in polyglossia, but it will use hyphenation patterns
\setdefaultlanguage{gujarati}
\setotherlanguage{english}

% Configure Gujarati font properly
% Use Language=Default to prevent polyglossia from trying to add language-specific features
% that don't exist for Gujarati, which causes "empty feature" warnings
\newfontfamily\gujaratifont[Script=Gujarati,AutoFakeBold=2.5,AutoFakeSlant=0.3]{Noto Sans Gujarati}
\setmainfont[Script=Gujarati,AutoFakeBold=2.5,AutoFakeSlant=0.3]{Noto Sans Gujarati}
% Use Noto Sans Gujarati for monospace to support Gujarati in text
\setmonofont[Scale=0.9]{Noto Sans Gujarati}

% Configure English to use the same font
\newfontfamily\englishfont[Script=Gujarati,AutoFakeBold=2.5,AutoFakeSlant=0.3]{Noto Sans Gujarati}

% Translations for polyglossia
\gappto\captionsgujarati{
  \renewcommand{\tablename}{કોષ્ટક}
  \renewcommand{\figurename}{આકૃતિ}
}

% Helper for TikZ nodes to ensure Gujarati font
\newcommand{\gu}[1]{{\gujaratifont #1}}

% Custom environments
\newtcolorbox{solutionbox}{
    breakable,
    enhanced,
    colback=solutioncolor!5!white,
    colframe=solutioncolor!75!black,
    fonttitle=\bfseries,
    title=જવાબ
}

\newtcolorbox{solutionboxnobreak}{
 colback=solutioncolor!5!white,
 colframe=solutioncolor!75!black,
 fonttitle=\bfseries,
 title=જવાબ
}

\newtcolorbox{keyformula}{
 breakable,
 enhanced,
 colback=keycolor!5!white,
 colframe=keycolor!75!black,
 fonttitle=\bfseries,
 title=રાસાયણિક સમીકરણ/સૂત્ર
}

\newtcolorbox{mnemonicbox}{
 breakable,
 enhanced,
 colback=mnemoniccolor!5!white,
 colframe=mnemoniccolor!75!black,
 fonttitle=\bfseries,
 title=મેમરી ટ્રીક
}


% Custom commands for GTU solutions
% This file defines semantic commands for consistent formatting

% Question command with automatic formatting
\newcommand{\question}[2]{%
  \section*{Question #1}%
  \textbf{#2}%
}

% OR question variant
\newcommand{\questionor}[2]{%
  \section*{Question #1 OR}%
  \textbf{#2}%
}

% Proper table environment with caption
\newenvironment{answertable}[1]{%
  \begin{table}[htbp]
  \centering
  \caption{#1}
}{%
  \end{table}
}

% Proper figure environment for diagrams
\newenvironment{answerdiagram}[1]{%
  \begin{figure}[htbp]
  \centering
  \caption{#1}
}{%
  \end{figure}
}

% Semantic markup for key terms
\newcommand{\keyword}[1]{\textbf{#1}}
\newcommand{\code}[1]{\texttt{#1}}
\newcommand{\classname}[1]{\texttt{#1}}
\newcommand{\methodname}[1]{\texttt{#1}}

% Proper quotation marks
\newcommand{\mnemonic}[1]{``#1''}


\title{Essentials of Digital Marketing (4341601) - Winter 2024 Solution}
\date{November 22, 2024}

\begin{document}
\maketitle

\questionmarks{1(અ)}{3}{વેબસાઇટના SEO રેન્કિંગને પ્રભાવિત કરતા ત્રણ મહત્વપૂર્ણ પરિબળો સમજાવો.}

\begin{solutionbox}
\begin{center}
\captionof{table}{SEO રેન્કિંગ પરિબળો}
\begin{tabulary}{\linewidth}{|L|L|}
\hline
\textbf{પરિબળ} & \textbf{વર્ણન} \\ \hline
\textbf{કન્ટેન્ટ ક્વોલિટી} & તાજું, સંબંધિત, કીવર્ડ-ઓપ્ટિમાઇઝ્ડ કન્ટેન્ટ \\ \hline
\textbf{બેકલિંક્સ} & અન્ય ગુણવત્તાવાળી વેબસાઇટ્સમાંથી લિંક્સ \\ \hline
\textbf{ટેકનિકલ SEO} & સાઇટ સ્પીડ, મોબાઇલ-ફ્રેન્ડલી, SSL સર્ટિફિકેટ \\ \hline
\end{tabulary}
\end{center}

\begin{itemize}
    \item \textbf{કન્ટેન્ટ ક્વોલિટી}: સર્ચ એન્જિન મૂલ્યવાન કન્ટેન્ટને પ્રાથમિકતા આપે છે
    \item \textbf{બેકલિંક્સ}: અન્ય વેબસાઇટ્સ તરફથી વિશ્વાસની મતદાન તરીકે કામ કરે છે
    \item \textbf{ટેકનિકલ SEO}: સર્ચ એન્જિનને સાઇટને ક્રોલ અને ઇન્ડેક્સ કરવામાં મદદ કરે છે
\end{itemize}

\begin{mnemonicbox}CBT - Content, Backlinks, Technical\end{mnemonicbox}
\end{solutionbox}

\questionmarks{1(બ)}{4}{ડિજિટલ માર્કેટિંગમાં ડેટાની ગોપનીયતા અને તેનું મહત્વ વર્ણવો.}

\begin{solutionbox}
\textbf{ડેટા પ્રાઇવસી} એટલે ડિજિટલ માર્કેટિંગ પ્રવૃત્તિઓ દરમિયાન એકત્રિત કરાયેલી વ્યક્તિગત માહિતીનું સંરક્ષણ.

\begin{center}
\captionof{table}{ડેટા પ્રાઇવસીનું મહત્વ}
\begin{tabulary}{\linewidth}{|L|L|}
\hline
\textbf{પાસું} & \textbf{મહત્વ} \\ \hline
\textbf{યુઝર ટ્રસ્ટ} & ગ્રાહકોનો વિશ્વાસ અને વફાદારી બનાવે છે \\ \hline
\textbf{કાયદાકીય પાલન} & GDPR, CCPA નિયમોથી દંડ બચાવે છે \\ \hline
\textbf{બ્રાન્ડ પ્રતિષ્ઠા} & ડેટા બ્રીચથી નકારાત્મક પ્રચારને અટકાવે છે \\ \hline
\end{tabulary}
\end{center}

\begin{itemize}
    \item \textbf{યુઝર ટ્રસ્ટ}: જ્યારે ગ્રાહકો તમારી પ્રાઇવસી પ્રેક્ટિસ પર ભરોસો રાખે છે ત્યારે વધુ ડેટા શેર કરે છે
    \item \textbf{કાયદાકીય પાલન}: ડેટા પ્રોટેક્શન કાયદાઓનું ફરજિયાત પાલન
    \item \textbf{બ્રાન્ડ પ્રતિષ્ઠા}: ડેટા બ્રીચ બ્રાન્ડ ઇમેજને ગંભીર નુકસાન પહોંચાડી શકે છે
\end{itemize}

\begin{mnemonicbox}TLR - Trust, Legal, Reputation\end{mnemonicbox}
\end{solutionbox}

\questionmarks{1(ક)}{7}{ડિજિટલ માર્કેટિંગ યોજનાઓ માટેના મુખ્ય ઘટકો સમજાવો.}

\begin{solutionbox}
\begin{center}
\captionof{table}{ડિજિટલ માર્કેટિંગ પ્લાન ઘટકો}
\begin{tabulary}{\linewidth}{|L|L|}
\hline
\textbf{ઘટક} & \textbf{વર્ણન} \\ \hline
\textbf{લક્ષ્યો અને ઉદ્દેશ્યો} & વ્યવસાયિક ઉદ્દેશ્યો સાથે જોડાયેલા SMART લક્ષ્યો \\ \hline
\textbf{ટાર્ગેટ ઓડિયન્સ} & ડેમોગ્રાફિક્સ, સાયકોગ્રાફિક્સ અને વર્તન વિશ્લેષણ \\ \hline
\textbf{ચેનલ સ્ટ્રેટેજી} & યોગ્ય ડિજિટલ પ્લેટફોર્મની પસંદગી \\ \hline
\textbf{કન્ટેન્ટ સ્ટ્રેટેજી} & કન્ટેન્ટના પ્રકારો, થીમ્સ અને પબ્લિશિંગ શેડ્યૂલ \\ \hline
\textbf{બજેટ ફાળવણી} & ચેનલ્સમાં સંસાધનોની વિતરણ \\ \hline
\textbf{એનાલિટિક્સ અને KPIs} & માપદંડ ફ્રેમવર્ક અને સફળતાના મેટ્રિક્સ \\ \hline
\end{tabulary}
\end{center}

\begin{center}
\begin{tikzpicture}[node distance=1.5cm, auto]
    \node [gtu block, minimum width=3cm] (Plan) {ડિજિટલ માર્કેટિંગ\\પ્લાન};
    \node [gtu block, below left=1.5cm and 2cm of Plan] (Goals) {લક્ષ્યો અને\\ઉદ્દેશ્યો};
    \node [gtu block, below left=1.5cm and -0.5cm of Plan] (Audience) {ટાર્ગેટ\\ઓડિયન્સ};
    \node [gtu block, below=1.5cm of Plan] (Channels) {ચેનલ\\સ્ટ્રેટેજી};
    \node [gtu block, below right=1.5cm and -0.5cm of Plan] (Content) {કન્ટેન્ટ\\સ્ટ્રેટેજી};
    \node [gtu block, below right=1.5cm and 2cm of Plan] (Budget) {બજેટ\\ફાળવણી};
    \node [gtu block, below=1.5cm of Channels] (Analytics) {એનાલિટિક્સ અને\\KPIs};

    \path [gtu arrow] (Plan) -- (Goals);
    \path [gtu arrow] (Plan) -- (Audience);
    \path [gtu arrow] (Plan) -- (Channels);
    \path [gtu arrow] (Plan) -- (Content);
    \path [gtu arrow] (Plan) -- (Budget);
    \path [gtu arrow] (Plan) -- (Analytics);
\end{tikzpicture}
\captionof{figure}{ડિજિટલ માર્કેટિંગ પ્લાન માળખું}
\end{center}

\begin{itemize}
    \item \textbf{લક્ષ્યો અને ઉદ્દેશ્યો}: વિશિષ્ટ, માપવા યોગ્ય પરિણામો વ્યાખ્યાયિત કરો
    \item \textbf{ટાર્ગેટ ઓડિયન્સ}: વિગતવાર બાયર પર્સોના બનાવો
    \item \textbf{ચેનલ સ્ટ્રેટેજી}: સોશિયલ મીડિયા, ઇમેઇલ, SEO, PPC નું સર્વોત્તમ મિશ્રણ પસંદ કરો
    \item \textbf{કન્ટેન્ટ સ્ટ્રેટેજી}: આકર્ષક કન્ટેન્ટ કેલેન્ડર વિકસાવો
    \item \textbf{બજેટ ફાળવણી}: ROI ની સંભાવના આધારે સંસાધનોનું વિતરણ કરો
    \item \textbf{એનાલિટિક્સ અને KPIs}: પરફોર્મન્સ ટ્રેક કરો અને સતત ઑપ્ટિમાઇઝ કરો
\end{itemize}

\begin{mnemonicbox}GT-CCBA - Goals-Target, Channels-Content-Budget-Analytics\end{mnemonicbox}
\end{solutionbox}

\questionmarks{1(ક OR)}{7}{P.O.E.M ફ્રેમવર્કને વ્યાખ્યાયિત કરો અને ડિજિટલ માર્કેટિંગમાં તેનું મહત્વ સમજાવો.}

\begin{solutionbox}
\textbf{P.O.E.M.} એટલે \textbf{Paid, Owned, Earned, Media} ફ્રેમવર્ક ડિજિટલ માર્કેટિંગ સ્ટ્રેટેજી માટે.

\begin{center}
\captionof{table}{P.O.E.M ફ્રેમવર્ક}
\begin{tabulary}{\linewidth}{|L|L|L|}
\hline
\textbf{મીડિયા પ્રકાર} & \textbf{વર્ણન} & \textbf{ઉદાહરણો} \\ \hline
\textbf{Paid} & તમે પૈસા ચૂકવો છો તે મીડિયા & Google Ads, Facebook Ads, YouTube Ads \\ \hline
\textbf{Owned} & તમે નિયંત્રિત કરો છો તે મીડિયા & વેબસાઇટ, બ્લોગ, ઇમેઇલ લિસ્ટ, મોબાઇલ એપ \\ \hline
\textbf{Earned} & વિશ્વસનીયતા દ્વારા મેળવેલ મીડિયા & સોશિયલ શેર્સ, રિવ્યૂઝ, PR મેન્શન્સ \\ \hline
\end{tabulary}
\end{center}

\begin{center}
\begin{tikzpicture}[node distance=1.5cm, auto]
    \node [gtu block] (Main) {P.O.E.M ફ્રેમવર્ક};
    
    \node [gtu block, below left=1.5cm and 1cm of Main] (Paid) {Paid Media};
    \node [gtu block, below=1.5cm of Main] (Owned) {Owned Media};
    \node [gtu block, below right=1.5cm and 1cm of Main] (Earned) {Earned Media};
    
    \node [gtu state, below=0.8cm of Paid] (Reach) {તાત્કાલિક\\પહોંચ};
    \node [gtu state, below=0.8cm of Owned] (Asset) {લાંબા ગાળાની\\સંપત્તિ};
    \node [gtu state, below=0.8cm of Earned] (Trust) {વિશ્વાસ અને\\વિશ્વસનીયતા};
    
    \path [gtu arrow] (Main) -- (Paid);
    \path [gtu arrow] (Main) -- (Owned);
    \path [gtu arrow] (Main) -- (Earned);
    
    \path [gtu arrow] (Paid) -- (Reach);
    \path [gtu arrow] (Owned) -- (Asset);
    \path [gtu arrow] (Earned) -- (Trust);
\end{tikzpicture}
\captionof{figure}{P.O.E.M ફ્રેમવર્ક}
\end{center}

\begin{itemize}
    \item \textbf{Paid Media}: તાત્કાલિક દૃશ્યતા અને લક્ષિત પહોંચ પ્રદાન કરે છે
    \item \textbf{Owned Media}: લાંબા ગાળાની સંપત્તિ અને બ્રાન્ડ નિયંત્રણ બનાવે છે
    \item \textbf{Earned Media}: વિશ્વાસ અને અધિકૃત બ્રાન્ડ એડવોકસી બનાવે છે
\end{itemize}

\begin{mnemonicbox}POE - Pay, Own, Earn\end{mnemonicbox}
\end{solutionbox}

\questionmarks{2(અ)}{3}{Black hat અને White hat SEO ટેકનીક વચ્ચે તફાવત વર્ણવો.}

\begin{solutionbox}
\begin{center}
\captionof{table}{White Hat vs Black Hat SEO}
\begin{tabulary}{\linewidth}{|L|L|L|}
\hline
\textbf{પાસું} & \textbf{White Hat SEO} & \textbf{Black Hat SEO} \\ \hline
\textbf{પદ્ધતિઓ} & નૈતિક, માર્ગદર્શિકા-અનુપાલન & હેરાફેરીયુક્ત, નિયમ-ભંગ \\ \hline
\textbf{પરિણામો} & ટકાઉ લાંબા ગાળાની વૃદ્ધિ & ઝડપી પરંતુ અસ્થાયી લાભ \\ \hline
\textbf{જોખમ} & દંડથી સુરક્ષિત & દંડનું ઉચ્ચ જોખમ \\ \hline
\end{tabulary}
\end{center}

\begin{itemize}
    \item \textbf{White Hat SEO}: ટકાઉ પરિણામો માટે સર્ચ એન્જિન માર્ગદર્શિકાઓનું પાલન કરે છે
    \item \textbf{Black Hat SEO}: ઝડપી રેન્કિંગ લાભ માટે ભ્રામક પ્રથાઓનો ઉપયોગ કરે છે
    \item \textbf{જોખમ પરિબળ}: Black hat ટેકનીક્સના કારણે સંપૂર્ણ સાઇટ બેન થઈ શકે છે
\end{itemize}

\begin{mnemonicbox}WEB - White Ethical Benefits, Black Breaks-rules\end{mnemonicbox}
\end{solutionbox}

\questionmarks{2(બ)}{4}{સર્ચ એન્જિન અલ્ગોરિધમ કઈ રીતે કાર્ય કરે છે અને વેબસાઈટને કઈ રીતે રેંક આપે છે એ સમજાવો.}

\begin{solutionbox}
\begin{center}
\captionof{table}{સર્ચ એન્જિન પ્રક્રિયા}
\begin{tabulary}{\linewidth}{|L|L|}
\hline
\textbf{પ્રક્રિયા} & \textbf{કાર્ય} \\ \hline
\textbf{ક્રોલિંગ} & બોટ્સ વેબ પેજો શોધે અને સ્કેન કરે છે \\ \hline
\textbf{ઇન્ડેક્સિંગ} & પેજો સર્ચ એન્જિન ડેટાબેસમાં સંગ્રહિત થાય છે \\ \hline
\textbf{રેન્કિંગ} & અલ્ગોરિધમ પેજની સંબંધિતતા અને અધિકાર નક્કી કરે છે \\ \hline
\textbf{પરિણામો} & યુઝર ક્વેરીઝ માટે શ્રેષ્ઠ મેચ દર્શાવવામાં આવે છે \\ \hline
\end{tabulary}
\end{center}

\begin{itemize}
    \item \textbf{ક્રોલિંગ}: વેબ ક્રોલર્સ નવું કન્ટેન્ટ શોધવા માટે લિંક્સને ફોલો કરે છે
    \item \textbf{ઇન્ડેક્સિંગ}: કન્ટેન્ટનું વિશ્લેષણ કરીને મોટા ડેટાબેસમાં સંગ્રહિત કરવામાં આવે છે
    \item \textbf{રેન્કિંગ}: 200+ પરિબળો સર્ચ પરિણામ સ્થાનો નક્કી કરે છે
    \item \textbf{પરિણામો}: સૌથી સંબંધિત પેજો યુઝરોને પ્રથમ દર્શાવવામાં આવે છે
\end{itemize}

\begin{mnemonicbox}CIRR - Crawl, Index, Rank, Results\end{mnemonicbox}
\end{solutionbox}

\questionmarks{2(ક)}{7}{બેકલિંક્સ બનાવવા માટેની વ્યૂહરચનાઓનું વર્ણન કરો.}

\begin{solutionbox}
\begin{center}
\captionof{table}{બેકલિંક વ્યૂહરચનાઓ}
\begin{tabulary}{\linewidth}{|L|L|L|}
\hline
\textbf{વ્યૂહરચના} & \textbf{વર્ણન} & \textbf{અસરકારકતા} \\ \hline
\textbf{ગેસ્ટ પોસ્ટિંગ} & અન્ય વેબસાઇટ્સ માટે લેખો લખવા & ઉચ્ચ \\ \hline
\textbf{રિસોર્સ લિંક બિલ્ડિંગ} & ઉદ્યોગ ડાયરેક્ટરીમાં સૂચિબદ્ધ થવું & મધ્યમ \\ \hline
\textbf{બ્રોકન લિંક બિલ્ડિંગ} & તૂટેલી લિંક્સને તમારા કન્ટેન્ટ સાથે બદલવી & ઉચ્ચ \\ \hline
\textbf{કન્ટેન્ટ માર્કેટિંગ} & શેર કરવા યોગ્ય, મૂલ્યવાન કન્ટેન્ટ બનાવવું & ખૂબ ઉચ્ચ \\ \hline
\textbf{ઇન્ફ્લુએન્સર આઉટરીચ} & ઉદ્યોગ ઇન્ફ્લુએન્સર્સ સાથે ભાગીદારી & ઉચ્ચ \\ \hline
\end{tabulary}
\end{center}

\begin{center}
\begin{tikzpicture}[node distance=1.5cm, auto]
    \node [gtu block] (Backlinks) {બેકલિંક બિલ્ડિંગ};
    
    \node [gtu block, below left=1.5cm and 2cm of Backlinks] (Guest) {ગેસ્ટ\\પોસ્ટિંગ};
    \node [gtu block, below left=1.5cm and -0.5cm of Backlinks] (Resource) {રિસોર્સ\\લિંક્સ};
    \node [gtu block, below=1.5cm of Backlinks] (Broken) {બ્રોકન લિંક\\બિલ્ડિંગ};
    \node [gtu block, below right=1.5cm and -0.5cm of Backlinks] (Content) {કન્ટેન્ટ\\માર્કેટિંગ};
    \node [gtu block, below right=1.5cm and 2cm of Backlinks] (Influencer) {ઇન્ફ્લુએન્સર\\આઉટરીચ};
    
    \foreach \n in {Guest, Resource, Broken, Content, Influencer}
        \path [gtu arrow] (Backlinks) -- (\n);
\end{tikzpicture}
\captionof{figure}{બેકલિંક બિલ્ડિંગ વ્યૂહરચનાઓ}
\end{center}

\begin{itemize}
    \item \textbf{ગેસ્ટ પોસ્ટિંગ}: તમારા નિશમાં સંબંધો અને સત્તા બનાવે છે
    \item \textbf{રિસોર્સ લિંક બિલ્ડિંગ}: ડાયરેક્ટરીઓ દ્વારા વિશ્વસનીયતા સ્થાપિત કરે છે
    \item \textbf{બ્રોકન લિંક બિલ્ડિંગ}: તૂટેલા સંસાધનો ઠીક કરીને મૂલ્ય પ્રદાન કરે છે
    \item \textbf{કન્ટેન્ટ માર્કેટિંગ}: ગુણવત્તાયુક્ત કન્ટેન્ટ દ્વારા કુદરતી રીતે લિંક્સ આકર્ષે છે
    \item \textbf{ઇન્ફ્લુએન્સર આઉટરીચ}: લિંક તકો માટે સ્થાપિત પ્રેક્ષકોનો લાભ લે છે
\end{itemize}

\begin{mnemonicbox}GRBCI - Guest, Resource, Broken, Content, Influencer\end{mnemonicbox}
\end{solutionbox}

\questionmarks{2(અ OR)}{3}{સર્ચ એન્જિન રેન્કિંગ માટે backlinks, website speed અને performance ની અગત્યતા સમજાવો.}

\begin{solutionbox}
\begin{center}
\captionof{table}{SEO પરિબળ અસર}
\begin{tabulary}{\linewidth}{|L|L|}
\hline
\textbf{પરિબળ} & \textbf{SEO પર અસર} \\ \hline
\textbf{બેકલિંક્સ} & સત્તા અને વિશ્વાસના સંકેતો \\ \hline
\textbf{વેબસાઇટ સ્પીડ} & યુઝર એક્સપિરિયન્સ રેન્કિંગ પરિબળ \\ \hline
\textbf{પરફોર્મન્સ} & Core Web Vitals રેન્કિંગને અસર કરે છે \\ \hline
\end{tabulary}
\end{center}

\begin{itemize}
    \item \textbf{બેકલિંક્સ}: અન્ય વેબસાઇટ્સ તરફથી વિશ્વાસના વોટ તરીકે કામ કરે છે
    \item \textbf{વેબસાઇટ સ્પીડ}: ઝડપી સાઇટ્સ ઉચ્ચ રેન્ક કરે છે અને બાઉન્સ રેટ ઘટાડે છે
    \item \textbf{પરફોર્મન્સ}: Google સારા Core Web Vitals વાળી સાઇટ્સને પ્રાથમિકતા આપે છે
\end{itemize}

\begin{mnemonicbox}BSP - Backlinks, Speed, Performance\end{mnemonicbox}
\end{solutionbox}

\questionmarks{2(બ OR)}{4}{On-page અને Off-page SEO ટેકનીક વચ્ચે તફાવત વર્ણવો અને દરેકનું એક ઉદાહરણ આપો.}

\begin{solutionbox}
\begin{center}
\captionof{table}{On-Page vs Off-Page SEO}
\begin{tabulary}{\linewidth}{|L|L|L|}
\hline
\textbf{SEO પ્રકાર} & \textbf{ફોકસ} & \textbf{ઉદાહરણો} \\ \hline
\textbf{On-Page} & વેબસાઇટ ઑપ્ટિમાઇઝેશન & ટાઇટલ ટેગ્સ, મેટા વર્ણનો, કન્ટેન્ટ ઑપ્ટિમાઇઝેશન \\ \hline
\textbf{Off-Page} & બાહ્ય પરિબળો & બેકલિંક્સ, સોશિયલ સિગ્નલ્સ, બ્રાન્ડ મેન્શન્સ \\ \hline
\end{tabulary}
\end{center}

\begin{itemize}
    \item \textbf{On-Page SEO}: તમારી વેબસાઇટની અંદરના તત્વોને નિયંત્રિત કરે છે
    \item \textbf{Off-Page SEO}: બાહ્ય વેલિડેશન દ્વારા સત્તા બનાવે છે
    \item \textbf{ઉદાહરણો}: On-page માં કીવર્ડ ઑપ્ટિમાઇઝેશન; off-page માં લિંક બિલ્ડિંગ
\end{itemize}

\begin{mnemonicbox}IO - Internal Optimization, External Elevation\end{mnemonicbox}
\end{solutionbox}

\questionmarks{2(ક OR)}{7}{SEO રેન્કિંગમાં સુધારો કરવાની વિવિધ રીતો સમજાવો.}

\begin{solutionbox}
\begin{center}
\captionof{table}{SEO સુધારણા પદ્ધતિઓ}
\begin{tabulary}{\linewidth}{|L|L|L|}
\hline
\textbf{પદ્ધતિ} & \textbf{વર્ણન} & \textbf{અસર} \\ \hline
\textbf{કીવર્ડ રિસર્ચ} & સંબંધિત, ઓછી સ્પર્ધાવાળા કીવર્ડ્સ લક્ષિત કરવા & ઉચ્ચ \\ \hline
\textbf{કન્ટેન્ટ ઑપ્ટિમાઇઝેશન} & મૂલ્યવાન, કીવર્ડ-સમૃદ્ધ કન્ટેન્ટ બનાવવું & ખૂબ ઉચ્ચ \\ \hline
\textbf{ટેકનિકલ SEO} & સાઇટ સ્પીડ, મોબાઇલ-ફ્રેન્ડલીનેસ સુધારવી & ઉચ્ચ \\ \hline
\textbf{લિંક બિલ્ડિંગ} & ગુણવત્તાયુક્ત બેકલિંક્સ મેળવવી & ખૂબ ઉચ્ચ \\ \hline
\textbf{યુઝર એક્સપિરિયન્સ} & સાઇટ ઉપયોગીતા અને સંલગ્નતા વધારવી & મધ્યમ \\ \hline
\textbf{લોકલ SEO} & સ્થાનિક સર્ચ પરિણામો માટે ઑપ્ટિમાઇઝ કરવું & ઉચ્ચ (સ્થાનિક વ્યવસાય માટે) \\ \hline
\end{tabulary}
\end{center}

\begin{center}
\begin{tikzpicture}[node distance=1.5cm, auto]
    \node [gtu block] (SEO) {SEO સુધારણા};
    
    \node [gtu block, below left=1.5cm and 2.5cm of SEO] (Keys) {કીવર્ડ\\રિસર્ચ};
    \node [gtu block, below left=1.5cm and 0cm of SEO] (Content) {કન્ટેન્ટ\\ઑપ્ટિમાઇઝેશન};
    \node [gtu block, below=1.5cm of SEO] (Tech) {ટેકનિકલ\\SEO};
    \node [gtu block, below right=1.5cm and 0cm of SEO] (Links) {લિંક\\બિલ્ડિંગ};
    \node [gtu block, below right=1.5cm and 2.5cm of SEO] (UX) {યુઝર\\એક્સપિરિયન્સ};
    \node [gtu block, below=3.5cm of SEO] (Local) {લોકલ\\SEO};
    
    \foreach \n in {Keys, Content, Tech, Links, UX, Local}
        \path [gtu arrow] (SEO) -- (\n);
\end{tikzpicture}
\captionof{figure}{SEO સુધારવાની રીતો}
\end{center}

\begin{itemize}
    \item \textbf{કીવર્ડ રિસર્ચ}: બધા SEO પ્રયાસો માટે પાયો
    \item \textbf{કન્ટેન્ટ ઑપ્ટિમાઇઝેશન}: કીવર્ડ્સને લક્ષિત કરતી વખતે મૂલ્ય પ્રદાન કરે છે
    \item \textbf{ટેકનિકલ SEO}: સર્ચ એન્જિનો તમારી સાઇટને અસરકારક રીતે ક્રોલ કરી શકે તેની ખાતરી કરે છે
    \item \textbf{લિંક બિલ્ડિંગ}: ડોમેઇન ઑથોરિટી અને વિશ્વાસ બનાવે છે
    \item \textbf{યુઝર એક્સપિરિયન્સ}: બાઉન્સ રેટ ઘટાડે અને સંલગ્નતા વધારે છે
    \item \textbf{લોકલ SEO}: ભૌતિક સ્થાનો ધરાવતા વ્યવસાયો માટે મહત્વપૂર્ણ
\end{itemize}

\begin{mnemonicbox}KC-TLUL - Keywords, Content, Technical, Links, User-experience, Local\end{mnemonicbox}
\end{solutionbox}

\questionmarks{3(અ)}{3}{Single-touch અને multi-touch attribution મોડેલ વચ્ચેનો તફાવત વર્ણવો.}

\begin{solutionbox}
\begin{center}
\captionof{table}{એટ્રિબ્યુશન મોડલ્સ}
\begin{tabulary}{\linewidth}{|L|L|L|}
\hline
\textbf{મોડેલ પ્રકાર} & \textbf{ક્રેડિટ સોંપણી} & \textbf{ઉપયોગ કેસ} \\ \hline
\textbf{Single-Touch} & એક ટચપોઇન્ટને 100\% ક્રેડિટ & સરળ ગ્રાહક યાત્રાઓ \\ \hline
\textbf{Multi-Touch} & ટચપોઇન્ટ્સમાં ક્રેડિટ વિતરણ & જટિલ ગ્રાહક યાત્રાઓ \\ \hline
\end{tabulary}
\end{center}

\begin{itemize}
    \item \textbf{Single-Touch}: પ્રથમ-ક્લિક અથવા છેલ્લા-ક્લિકને સંપૂર્ણ ક્રેડિટ મળે છે
    \item \textbf{Multi-Touch}: લિનિયર, ટાઇમ-ડિકે, અથવા પોઝિશન-આધારિત એટ્રિબ્યુશન
    \item \textbf{ઉપયોગ}: Multi-touch વધુ સચોટ ગ્રાહક યાત્રા આંતરદૃષ્ટિ પ્રદાન કરે છે
\end{itemize}

\begin{mnemonicbox}SM - Single Simple, Multi Multiple\end{mnemonicbox}
\end{solutionbox}

\questionmarks{3(બ)}{4}{Google Analytics માં વ્યવસાયો કેવી રીતે લક્ષ્યો સેટ કરી શકે છે તે સમજાવો.}

\begin{solutionbox}
\begin{center}
\captionof{table}{ગોલ્સ સેટઅપ}
\begin{tabulary}{\linewidth}{|L|L|}
\hline
\textbf{પગલું} & \textbf{ક્રિયા} \\ \hline
\textbf{1. ગોલ્સ એક્સેસ} & Admin $\rightarrow$ View $\rightarrow$ Goals પર જાઓ \\ \hline
\textbf{2. ટેમ્પલેટ પસંદ કરો} & ટેમ્પલેટમાંથી પસંદ કરો અથવા કસ્ટમ બનાવો \\ \hline
\textbf{3. વિગતો કન્ફિગર કરો} & ગોલ નામ, પ્રકાર અને શરતો સેટ કરો \\ \hline
\textbf{4. સેટઅપ ચકાસો} & વેરિફિકેશન ફીચર વાપરીને ગોલ ટેસ્ટ કરો \\ \hline
\end{tabulary}
\end{center}

\begin{itemize}
    \item \textbf{ગોલ પ્રકારો}: ડેસ્ટિનેશન, અવધિ, પેજ/સેશન, ઇવેન્ટ ગોલ્સ
    \item \textbf{કન્ફિગરેશન}: ગોલ પૂર્ણતા માટે વિશિષ્ટ શરતો વ્યાખ્યાયિત કરો
    \item \textbf{વેરિફિકેશન}: અમલીકરણ પહેલાં ગોલ્સ યોગ્ય રીતે ટ્રેક કરે છે તેની ખાતરી કરો
    \item \textbf{મોનિટરિંગ}: ગોલ પરફોર્મન્સની નિયમિત સમીક્ષા અને ઑપ્ટિમાઇઝેશન
\end{itemize}

\begin{mnemonicbox}ACCV - Access, Choose, Configure, Verify\end{mnemonicbox}
\end{solutionbox}

\questionmarks{3(ક)}{7}{ડિજિટલ માર્કેટિંગની વ્યૂહરચના ઘડવામાં વેબ એનાલિટિક્સની શું ભૂમિકા છે? વિવિધ પ્રકારના વેબ એનાલિટિક્સ વિશે ચર્ચા કરો.}

\begin{solutionbox}
\textbf{વ્યૂહરચનામાં ભૂમિકા:}
વેબ એનાલિટિક્સ ડિજિટલ માર્કેટિંગમાં માહિતી-આધારિત નિર્ણય લેવા માટે ડેટા-આધારિત આંતરદૃષ્ટિ પ્રદાન કરે છે.

\begin{center}
\captionof{table}{વેબ એનાલિટિક્સ પ્રકારો}
\begin{tabulary}{\linewidth}{|L|L|L|}
\hline
\textbf{એનાલિટિક્સ પ્રકાર} & \textbf{હેતુ} & \textbf{મુખ્ય મેટ્રિક્સ} \\ \hline
\textbf{કન્ટેન્ટ એનાલિટિક્સ} & કન્ટેન્ટ પરફોર્મન્સ ટ્રેકિંગ & પેજ વ્યૂઝ, પેજ પર સમય, બાઉન્સ રેટ \\ \hline
\textbf{કસ્ટમર એનાલિટિક્સ} & યુઝર વર્તન વિશ્લેષણ & ડેમોગ્રાફિક્સ, રુચિઓ, કન્વર્શન પાથ \\ \hline
\textbf{સોશિયલ મીડિયા એનાલિટિક્સ} & સોશિયલ એન્ગેજમેન્ટ માપદંડ & શેર્સ, લાઇક્સ, કોમેન્ટ્સ, રીચ \\ \hline
\textbf{SEO એનાલિટિક્સ} & સર્ચ પરફોર્મન્સ ટ્રેકિંગ & કીવર્ડ્સ, રેન્કિંગ્સ, ઓર્ગેનિક ટ્રાફિક \\ \hline
\textbf{કન્વર્શન એનાલિટિક્સ} & ગોલ પૂર્ણતા ટ્રેકિંગ & કન્વર્શન રેટ, રેવન્યુ, ROI \\ \hline
\end{tabulary}
\end{center}

\begin{center}
\begin{tikzpicture}[node distance=1.5cm, auto]
    \node [gtu block, minimum width=3cm] (Strategy) {ડિજિટલ માર્કેટિંગ\\વ્યૂહરચના ઘડતર};
    \node [gtu block, below=1.5cm of Strategy] (Analytics) {વેબ એનાલિટિક્સ};
    
    \node [gtu block, below left=1.5cm and 3cm of Analytics] (Content) {કન્ટેન્ટ\\એનાલિટિક્સ};
    \node [gtu block, below left=1.5cm and 0.5cm of Analytics] (Customer) {કસ્ટમર\\એનાલિટિક્સ};
    \node [gtu block, below=1.5cm of Analytics] (Social) {સોશિયલ\\એનાલિટિક્સ};
    \node [gtu block, below right=1.5cm and 0.5cm of Analytics] (SEO) {SEO\\એનાલિટિક્સ};
    \node [gtu block, below right=1.5cm and 3cm of Analytics] (Conversion) {કન્વર્શન\\એનાલિટિક્સ};
    
    \path [gtu arrow] (Analytics) -- (Strategy);
    \foreach \n in {Content, Customer, Social, SEO, Conversion}
        \path [gtu arrow] (Analytics) -- (\n);
\end{tikzpicture}
\captionof{figure}{વેબ એનાલિટિક્સની ભૂમિકા}
\end{center}

\begin{itemize}
    \item \textbf{વ્યૂહરચનાત્મક ભૂમિકા}: તકો ઓળખે છે, પરફોર્મન્સ માપે છે, ઑપ્ટિમાઇઝેશન માર્ગદર્શન આપે છે
    \item \textbf{કન્ટેન્ટ એનાલિટિક્સ}: એન્ગેજમેન્ટ આધારે કન્ટેન્ટ વ્યૂહરચના ઑપ્ટિમાઇઝ કરવામાં મદદ કરે છે
    \item \textbf{કસ્ટમર એનાલિટિક્સ}: વધુ સારું ઓડિયન્સ ટાર્ગેટિંગ અને વ્યક્તિગતકરણ સક્ષમ કરે છે
    \item \textbf{સોશિયલ મીડિયા એનાલિટિક્સ}: સોશિયલ મીડિયા ROI અને એન્ગેજમેન્ટ માપે છે
    \item \textbf{SEO એનાલિટિક્સ}: ઓર્ગેનિક સર્ચ પરફોર્મન્સ અને તકો ટ્રેક કરે છે
    \item \textbf{કન્વર્શન એનાલિટિક્સ}: માર્કેટિંગ પ્રયાસોની બોટમ-લાઇન અસર માપે છે
\end{itemize}

\begin{mnemonicbox}CCSSC - Content, Customer, Social, SEO, Conversion\end{mnemonicbox}
\end{solutionbox}

\questionmarks{3(અ OR)}{3}{Unique visitors, Average Visit Duration, Bounce rate ની વ્યાખ્યા આપો.}

\begin{solutionbox}
\begin{center}
\captionof{table}{વેબ મેટ્રિક્સ વ્યાખ્યાઓ}
\begin{tabulary}{\linewidth}{|L|L|}
\hline
\textbf{મેટ્રિક} & \textbf{વ્યાખ્યા} \\ \hline
\textbf{યુનિક વિઝિટર્સ} & વિશિષ્ટ સમયગાળામાં સાઇટની મુલાકાત લેતા વ્યક્તિગત યુઝર્સ \\ \hline
\textbf{એવરેજ વિઝિટ ડ્યુરેશન} & પ્રતિ સેશન યુઝર્સ વેબસાઇટ પર વિતાવતો સરેરાશ સમય \\ \hline
\textbf{બાઉન્સ રેટ} & એક પેજ જોયા પછી છોડી જનારા વિઝિટર્સની ટકાવારી \\ \hline
\end{tabulary}
\end{center}

\begin{itemize}
    \item \textbf{યુનિક વિઝિટર્સ}: પુનઃ મુલાકાતોને ધ્યાનમાં લીધા વગર દરેક વ્યક્તિને એકવાર ગણે છે
    \item \textbf{એવરેજ વિઝિટ ડ્યુરેશન}: કન્ટેન્ટ એન્ગેજમેન્ટ અને સાઇટ સ્ટિકિનેસ દર્શાવે છે
    \item \textbf{બાઉન્સ રેટ}: ઉચ્ચ દર ખરાબ કન્ટેન્ટ મેચ અથવા સાઇટ સમસ્યાઓ સૂચવી શકે છે
\end{itemize}

\begin{mnemonicbox}UAB - Unique, Average, Bounce\end{mnemonicbox}
\end{solutionbox}

\questionmarks{3(બ OR)}{4}{વેબ એનાલિટિક્સમાં A/B testing વિશે સમજાવો.}

\begin{solutionbox}
\textbf{A/B Testing} એટલે કયું વધુ સારું પરફોર્મ કરે છે તે નક્કી કરવા માટે વેબપેજના બે વર્ઝનની તુલના કરવી.

\begin{center}
\captionof{table}{A/B ટેસ્ટિંગ ઘટકો}
\begin{tabulary}{\linewidth}{|L|L|}
\hline
\textbf{ઘટક} & \textbf{વર્ણન} \\ \hline
\textbf{વર્ઝન A} & મૂળ વેબપેજ (કંટ્રોલ) \\ \hline
\textbf{વર્ઝન B} & સુધારેલ વેબપેજ (વેરિઅન્ટ) \\ \hline
\textbf{ટ્રાફિક સ્પ્લિટ} & સામાન્ય રીતે 50/50 રેન્ડમ વિતરણ \\ \hline
\textbf{મેટ્રિક્સ} & કન્વર્શન રેટ, ક્લિક-થ્રુ રેટ, એન્ગેજમેન્ટ \\ \hline
\end{tabulary}
\end{center}

\begin{itemize}
    \item \textbf{પ્રક્રિયા}: બે વર્ઝન વચ્ચે ટ્રાફિક વિભાજિત કરીને પરફોર્મન્સ માપો
    \item \textbf{અવધિ}: આંકડાકીય મહત્વ માટે પૂરતા લાંબા સમય સુધી ટેસ્ટ ચલાવો
    \item \textbf{વુઝર પરફેક્ટ}: એક સમયે એક તત્વ ટેસ્ટ કરો (હેડલાઇન્સ, બટન્સ, ઇમેજો)
    \item \textbf{નિર્ણય}: ડેટા આધારે જીતનાર વર્ઝન અમલ કરો
\end{itemize}

\begin{mnemonicbox}ABCD - A-version, B-version, Compare, Decide\end{mnemonicbox}
\end{solutionbox}

\questionmarks{3(ક OR)}{7}{નીચેમુજબના ટ્રેકિંગ કોડના ફાયદા અને ગેરફાયદાઓ સમજાવો: Long tracking code, Obfuscated tracking code, UTM codes}

\begin{solutionbox}
\begin{center}
\captionof{table}{ટ્રેકિંગ કોડ પ્રકારો}
\begin{tabulary}{\linewidth}{|L|L|L|L|}
\hline
\textbf{ટ્રેકિંગ પ્રકાર} & \textbf{વર્ણન} & \textbf{ફાયદા} & \textbf{ગેરફાયદા} \\ \hline
\textbf{લોંગ ટ્રેકિંગ કોડ} & વ્યાપક ટ્રેકિંગ માટે વિગતવાર પેરામીટર્સ & સંપૂર્ણ ડેટા સંગ્રહ, વિગતવાર આંતરદૃષ્ટિ & ધીમી પેજ લોડ, જટિલ અમલીકરણ \\ \hline
\textbf{ઓબ્ફસ્કેટેડ ટ્રેકિંગ} & એન્ક્રિપ્ટેડ/છુપાયેલ ટ્રેકિંગ પેરામીટર્સ & ડેટા સુરક્ષા, હેરાફેરીથી અટકાવે છે & કઠિન ડિબગિંગ, જટિલ સેટઅપ \\ \hline
\textbf{UTM કોડ્સ} & કેમ્પેઇન ટ્રેકિંગ માટે URL પેરામીટર્સ & સરળ અમલીકરણ, કેમ્પેઇન એટ્રિબ્યુશન & મેન્યુઅલ ટેગિંગ જરૂરી, URL દેખાવ \\ \hline
\end{tabulary}
\end{center}

\begin{center}
\begin{tikzpicture}[node distance=1.5cm, auto]
    \node [gtu block] (Tracking) {ટ્રેકિંગ કોડ્સ};
    
    \node [gtu block, below left=1.5cm and 2cm of Tracking] (Long) {લોંગ ટ્રેકિંગ};
    \node [gtu block, below=1.5cm of Tracking] (Obfuscated) {ઓબ્ફસ્કેટેડ\\ટ્રેકિંગ};
    \node [gtu block, below right=1.5cm and 2cm of Tracking] (UTM) {UTM કોડ્સ};
    
    \node [gtu state, below=0.8cm of Long] (T1) {વ્યાપક\\ડેટા};
    \node [gtu state, below=0.8cm of Obfuscated] (T2) {સુરક્ષિત\\ટ્રેકિંગ};
    \node [gtu state, below=0.8cm of UTM] (T3) {કેમ્પેઇન\\એટ્રિબ્યુશન};
    
    \foreach \n in {Long, Obfuscated, UTM}
        \path [gtu arrow] (Tracking) -- (\n);
        
    \path [gtu arrow] (Long) -- (T1);
    \path [gtu arrow] (Obfuscated) -- (T2);
    \path [gtu arrow] (UTM) -- (T3);
\end{tikzpicture}
\captionof{figure}{ટ્રેકિંગ કોડ સરખામણી}
\end{center}

\begin{itemize}
    \item \textbf{લોંગ ટ્રેકિંગ કોડ}: એન્ટરપ્રાઇઝ-લેવલ વિગતવાર એનાલિટિક્સ માટે શ્રેષ્ઠ
    \item \textbf{ઓબ્ફસ્કેટેડ ટ્રેકિંગ}: સંવેદનશીલ ડેટા સુરક્ષા આવશ્યકતાઓ માટે આદર્શ
    \item \textbf{UTM કોડ્સ}: કેમ્પેઇન ટ્રેકિંગ અને ટ્રાફિક સોર્સ ઓળખ માટે સંપૂર્ણ
\end{itemize}

\begin{mnemonicbox}LOU - Long comprehensive, Obfuscated secure, UTM simple\end{mnemonicbox}
\end{solutionbox}


\questionmarks{4(a)}{3}{YouTube ads ના અલગ અલગ પ્રકાર સમજાવો.}

\begin{solutionbox}
\begin{center}
\captionof{table}{YouTube જાહેરાત પ્રકારો}
\begin{tabulary}{\linewidth}{|L|L|L|}
\hline
\textbf{જાહેરાત પ્રકાર} & \textbf{ફોર્મેટ} & \textbf{પ્લેસમેન્ટ} \\ \hline
\textbf{Skippable In-Stream} & 5-સેકન્ડ સ્કીપ વિકલ્પ & વિડિઓ પહેલાં/દરમિયાન \\ \hline
\textbf{Non-Skippable} & 15-20 સેકન્ડ, સ્કીપ નહીં & વિડિઓ પહેલાં/દરમિયાન \\ \hline
\textbf{Bumper Ads} & 6 સેકન્ડ, નોન-સ્કીપેબલ & વિડિઓ પહેલાં \\ \hline
\end{tabulary}
\end{center}

\begin{itemize}
    \item \textbf{Skippable In-Stream}: ખર્ચ-અસરકારક, માત્ર એન્ગેજ્ડ દર્શકો માટે ચૂકવણી
    \item \textbf{Non-Skippable}: ગેરંટી મેસેજ ડિલિવરી, ઉચ્ચ પૂર્ણતા દર
    \item \textbf{Bumper Ads}: બ્રાન્ડ જાગૃતિ, ઝડપી યાદગાર સંદેશાઓ
\end{itemize}

\begin{mnemonicbox}SNB - Skippable, Non-skippable, Bumper\end{mnemonicbox}
\end{solutionbox}

\questionmarks{4(b)}{4}{LinkedIn marketing નો ખ્યાલ સમજાવો અને ડિજિટલ માર્કેટિંગમાં તેનું મહત્વ ચર્ચો.}

\begin{solutionbox}
\textbf{LinkedIn Marketing} પ્રોફેશનલ નેટવર્કિંગ અને B2B સંબંધ નિર્માણ પર કેન્દ્રિત છે.

\begin{center}
\captionof{table}{LinkedIn માર્કેટિંગનું મહત્વ}
\begin{tabulary}{\linewidth}{|L|L|}
\hline
\textbf{પાસું} & \textbf{મહત્વ} \\ \hline
\textbf{પ્રોફેશનલ ઓડિયન્સ} & નિર્ણય લેનારાઓ અને ઉદ્યોગ વ્યાવસાયિકો \\ \hline
\textbf{B2B ફોકસ} & બિઝનેસ-ટુ-બિઝનેસ માર્કેટિંગ માટે આદર્શ \\ \hline
\textbf{કન્ટેન્ટ ઑથોરિટી} & થૉટ લીડરશિપ સ્થાપિત કરે છે \\ \hline
\textbf{નેટવર્કિંગ} & મુખ્ય વ્યવસાય સંપર્કો સીધી ઍક્સેસ \\ \hline
\end{tabulary}
\end{center}

\begin{itemize}
    \item \textbf{પ્રોફેશનલ ઓડિયન્સ}: ઉચ્ચ આવક, શિક્ષિત ડેમોગ્રાફિક્સ
    \item \textbf{B2B ફોકસ}: 80\% B2B લીડ્સ LinkedIn થી આવે છે
    \item \textbf{કન્ટેન્ટ ઑથોરિટી}: ઉદ્યોગ આંતરદૃષ્ટિ અને કુશળતા શેર કરો
    \item \textbf{નેટવર્કિંગ}: મૂલ્યવાન વ્યવસાય સંબંધો બનાવો
\end{itemize}

\begin{mnemonicbox}PBCN - Professional, B2B, Content, Networking\end{mnemonicbox}
\end{solutionbox}

\questionmarks{4(c)}{7}{Organic અને paid સોશિયલ મીડિયા માર્કેટિંગ વ્યૂહરચનાઓ વચ્ચેના મુખ્ય તફાવતોનું વર્ણન કરો. દરેક વ્યૂહરચના માટે બે ફાયદા અને ગેરફાયદા આપો.}

\begin{solutionbox}
\begin{center}
\captionof{table}{Organic vs Paid સોશિયલ મીડિયા}
\begin{tabulary}{\linewidth}{|L|L|L|L|}
\hline
\textbf{વ્યૂહરચના} & \textbf{વર્ણન} & \textbf{ફાયદા} & \textbf{ગેરફાયદા} \\ \hline
\textbf{Organic} & મફત કન્ટેન્ટ પોસ્ટિંગ અને એન્ગેજમેન્ટ & • ખર્ચ-અસરકારક \newline • અધિકૃત સંબંધો & • સીમિત પહોંચ \newline • સમય માંગી લેતું \\ \hline
\textbf{Paid} & સ્પોન્સર કન્ટેન્ટ અને જાહેરાતો & • તાત્કાલિક પહોંચ \newline • ચોક્કસ ટાર્ગેટિંગ & • બજેટ જરૂરી \newline • અસ્થાયી પરિણામો \\ \hline
\end{tabulary}
\end{center}

\begin{center}
\begin{tikzpicture}[node distance=1.5cm, auto]
    \node [gtu block] (SMM) {સોશિયલ મીડિયા માર્કેટિંગ};
    
    \node [gtu block, below left=1.2cm and 1cm of SMM] (Organic) {Organic સ્ટ્રેટેજી};
    \node [gtu block, below right=1.2cm and 1cm of SMM] (Paid) {Paid સ્ટ્રેટેજી};
    
    \node [gtu state, below=0.8cm of Organic] (OAdv) {ખર્ચ-અસરકારક\\અધિકૃતતા};
    \node [gtu state, below=0.8cm of Paid] (PAdv) {તાત્કાલિક પહોંચ\\ટાર્ગેટિંગ};
    
    \path [gtu arrow] (SMM) -- (Organic);
    \path [gtu arrow] (SMM) -- (Paid);
    \path [gtu arrow] (Organic) -- (OAdv);
    \path [gtu arrow] (Paid) -- (PAdv);
\end{tikzpicture}
\captionof{figure}{Organic vs Paid વ્યૂહરચનાઓ}
\end{center}

\textbf{Organic ફાયદા:}
\begin{itemize}
    \item \textbf{ખર્ચ-અસરકારક}: કોઈ જાહેરાત ખર્ચ જરૂરી નથી
    \item \textbf{અધિકૃત સંબંધો}: સાચી સમુદાય સંલગ્નતા બનાવે છે
\end{itemize}

\textbf{Organic ગેરફાયદા:}
\begin{itemize}
    \item \textbf{સીમિત પહોંચ}: અલ્ગોરિધમ પ્રતિબંધો દૃશ્યતા ઘટાડે છે
    \item \textbf{સમય માંગી લેતું}: સતત કન્ટેન્ટ સર્જન અને એન્ગેજમેન્ટ જરૂરી
\end{itemize}

\textbf{Paid ફાયદા:}
\begin{itemize}
    \item \textbf{તાત્કાલિક પહોંચ}: ટાર્ગેટ ઓડિયન્સ માટે ઇન્સ્ટન્ટ દૃશ્યતા
    \item \textbf{ચોક્કસ ટાર્ગેટિંગ}: અદ્યતન ડેમોગ્રાફિક અને રુચિ ટાર્ગેટિંગ
\end{itemize}

\textbf{Paid ગેરફાયદા:}
\begin{itemize}
    \item \textbf{બજેટ જરૂરી}: ચાલુ જાહેરાત ખર્ચ
    \item \textbf{અસ્થાયી પરિણામો}: જ્યારે જાહેરાત બંધ થાય ત્યારે પરિણામો અટકી જાય છે
\end{itemize}

\begin{mnemonicbox}OPAL - Organic Patient Authentic Low-cost, Paid Quick Targeted Expensive\end{mnemonicbox}
\end{solutionbox}

\questionmarks{4(a OR)}{3}{Twitter ads ના વિવિધ પ્રકાર કયા છે? કોઈપણ એક પ્રકાર ટૂંકમાં સમજાવો.}

\begin{solutionbox}
\begin{center}
\captionof{table}{Twitter જાહેરાત પ્રકારો}
\begin{tabulary}{\linewidth}{|L|L|}
\hline
\textbf{જાહેરાત પ્રકાર} & \textbf{હેતુ} \\ \hline
\textbf{Promoted Tweets} & ટ્વીટ દૃશ્યતા વધારવા \\ \hline
\textbf{Promoted Accounts} & વધુ ફોલોઅર્સ મેળવવા \\ \hline
\textbf{Promoted Trends} & ટ્રેન્ડિંગ વિષયો બૂસ્ટ કરવા \\ \hline
\end{tabulary}
\end{center}

\textbf{Promoted Tweets}: નિયમિત ટ્વીટ્સ જે વ્યવસાયો તેમના ફોલોઅર્સ સિવાયના વ્યાપક પ્રેક્ષકોને બતાવવા માટે ચૂકવણી કરે છે, જે યુઝર્સની ટાઈમલાઈન અને સર્ચ પરિણામોમાં "Promoted" લેબલ સાથે દેખાય છે.

\begin{mnemonicbox}PAT - Promoted tweets, Accounts, Trends\end{mnemonicbox}
\end{solutionbox}

\questionmarks{4(b OR)}{4}{Samsung માર્કેટમાં નવો સ્માર્ટ ફોન લોન્ચ કર્યો છે અને YouTube ads ચલાવવા માંગે છે. સોશિયલ મીડિયા માર્કેટિંગ એકસપર્ટ તરીકે તમે કયા પ્રકારનું YouTube ad format પસંદ કરશો અને શા માટે?}

\begin{solutionbox}
\textbf{ભલામણ કરેલ ફોર્મેટ: Skippable In-Stream Ads}

\begin{center}
\captionof{table}{જાહેરાત પસંદગી કારણ}
\begin{tabulary}{\linewidth}{|L|L|}
\hline
\textbf{કારણ} & \textbf{લાભ} \\ \hline
\textbf{ખર્ચ-અસરકારક} & જ્યારે યુઝર્સ 30+ સેકન્ડ જુએ ત્યારે જ ચૂકવો \\ \hline
\textbf{પ્રોડક્ટ ડેમોન્સ્ટ્રેશન} & લાંબુ ફોર્મેટ ફીચર શોકેસ કરવાની મંજૂરી આપે છે \\ \hline
\textbf{ઓડિયન્સ રુચિ} & સ્કીપ વિકલ્પ એન્ગેજ્ડ દર્શકો સુનિશ્ચિત કરે છે \\ \hline
\textbf{બ્રાન્ડ જાગૃતિ} & સ્માર્ટફોન રુચિ ધરાવતા વ્યાપક ઓડિયન્સ સુધી પહોંચે છે \\ \hline
\end{tabulary}
\end{center}

\begin{itemize}
    \item \textbf{પ્રોડક્ટ ડેમોન્સ્ટ્રેશન}: સ્માર્ટફોન્સને ફીચર્સના વિઝ્યુઅલ ડેમોન્સ્ટ્રેશનની જરૂર છે
    \item \textbf{ઓડિયન્સ રુચિ}: સ્કીપ વિકલ્પ ખરેખર રસ ધરાવતા દર્શકોને ફિલ્ટર કરે છે
    \item \textbf{ખર્ચ-અસરકારક}: માત્ર એન્ગેજ્ડ દર્શકો માટે ચૂકવણી કરો જે 30 સેકન્ડથી વધુ જુએ છે
    \item \textbf{બ્રાન્ડ જાગૃતિ}: નવા પ્રોડક્ટ લોન્ચ માટે વ્યાપક પહોંચ
\end{itemize}

\begin{mnemonicbox}PCAB - Product demo, Cost-effective, Audience interest, Brand awareness\end{mnemonicbox}
\end{solutionbox}

\questionmarks{4(c OR)}{7}{Facebook Page, Business Manager અને Facebook Ads ના મુખ્ય કાર્યોનું વર્ણન કરો. આ અસ્કયામતો વ્યવસાયોને તેમના માર્કેટિંગ પ્રયત્નોમાં કેવી રીતે મદદ કરી શકે?}

\begin{solutionbox}
\begin{center}
\captionof{table}{Facebook માર્કેટિંગ અસ્કયામતો}
\begin{tabulary}{\linewidth}{|L|L|L|}
\hline
\textbf{અસ્કયામત} & \textbf{મુખ્ય કાર્યો} & \textbf{માર્કેટિંગ લાભો} \\ \hline
\textbf{Facebook Page} & • બ્રાન્ડ હાજરી\newline • કન્ટેન્ટ શેરિંગ\newline • ગ્રાહક એન્ગેજમેન્ટ & • બ્રાન્ડ જાગૃતિ બનાવે છે\newline • સીધો ગ્રાહક સંચાર \\ \hline
\textbf{Business Manager} & • એકાઉન્ટ મેનેજમેન્ટ\newline • ટીમ એક્સેસ કંટ્રોલ\newline • અસ્કયામત સંસ્થા & • કેન્દ્રિય નિયંત્રણ\newline • સુરક્ષિત સહયોગ \\ \hline
\textbf{Facebook Ads} & • ટાર્ગેટ એડવર્ટાઇઝિંગ\newline • કેમ્પેઇન મેનેજમેન્ટ\newline • પરફોર્મન્સ ટ્રેકિંગ & • ચોક્કસ ઓડિયન્સ ટાર્ગેટિંગ\newline • માપી શકાય તેવું ROI \\ \hline
\end{tabulary}
\end{center}

\begin{center}
\begin{tikzpicture}[node distance=1.5cm, auto]
    \node [gtu block] (Main) {Facebook માર્કેટિંગ અસ્કયામતો};
    
    \node [gtu block, below left=1.5cm and 2cm of Main] (Page) {Facebook Page};
    \node [gtu block, below=1.5cm of Main] (Manager) {Business Manager};
    \node [gtu block, below right=1.5cm and 2cm of Main] (Ads) {Facebook Ads};
    
    \node [gtu state, below=0.8cm of Page] (PBen) {બ્રાન્ડ હાજરી};
    \node [gtu state, below=0.8cm of Manager] (MBen) {એકાઉન્ટ મેનેજમેન્ટ};
    \node [gtu state, below=0.8cm of Ads] (ABen) {ટાર્ગેટ એડવર્ટાઇઝિંગ};
    
    \foreach \n in {Page, Manager, Ads}
        \path [gtu arrow] (Main) -- (\n);
        
    \path [gtu arrow] (Page) -- (PBen);
    \path [gtu arrow] (Manager) -- (MBen);
    \path [gtu arrow] (Ads) -- (ABen);
\end{tikzpicture}
\captionof{figure}{Facebook માર્કેટિંગ ઇકોસિસ્ટમ}
\end{center}

\textbf{માર્કેટિંગ લાભો:}
\begin{itemize}
    \item \textbf{Facebook Page}: પ્રોફેશનલ બ્રાન્ડ હાજરી બનાવે છે અને ઓર્ગેનિક રીચ સક્ષમ કરે છે
    \item \textbf{Business Manager}: બહુવિધ એકાઉન્ટ્સ અને ટીમ મેમ્બર્સ માટે સુરક્ષા અને સંસ્થા પ્રદાન કરે છે
    \item \textbf{Facebook Ads}: વિગતવાર એનાલિટિક્સ અને ROI ટ્રેકિંગ સાથે લક્ષિત ઝુંબેશ પહોંચાડે છે
\end{itemize}

\textbf{એકીકરણ લાભો:}
\begin{itemize}
    \item \textbf{એકીકૃત વ્યૂહરચના}: વ્યાપક Facebook માર્કેટિંગ માટે ત્રણેય સાથે મળીને કામ કરે છે
    \item \textbf{ડેટા શેરિંગ}: પેજમાંથી પિક્સેલ ડેટા એડ ટાર્ગેટિંગ વધારે છે
    \item \textbf{બ્રાન્ડ સુસંગતતા}: ઓર્ગેનિક અને પેઇડ કન્ટેન્ટમાં સુસંગત મેસેજિંગ
\end{itemize}

\begin{mnemonicbox}PMA - Page presence, Manager control, Ads targeting\end{mnemonicbox}
\end{solutionbox}

\questionmarks{5(a)}{3}{Instagram Content અને Ads ના પ્રકારોની યાદી આપો.}

\begin{solutionbox}
\begin{center}
\captionof{table}{Instagram કન્ટેન્ટ અને જાહેરાતો}
\begin{tabulary}{\linewidth}{|L|L|}
\hline
\textbf{કન્ટેન્ટ પ્રકારો} & \textbf{જાહેરાત પ્રકારો} \\ \hline
\textbf{Posts} & Photo Ads \\ \hline
\textbf{Stories} & Video Ads \\ \hline
\textbf{Reels} & Carousel Ads \\ \hline
\textbf{IGTV} & Stories Ads \\ \hline
\textbf{Live} & Reels Ads \\ \hline
\end{tabulary}
\end{center}

\begin{itemize}
    \item \textbf{કન્ટેન્ટ પ્રકારો}: ઓર્ગેનિક એન્ગેજમેન્ટ માટે વિવિધ ફોર્મેટ્સ
    \item \textbf{જાહેરાત પ્રકારો}: ટાર્ગેટિંગ ક્ષમતાઓ સાથે પ્રાયોજિત વર્ઝન
    \item \textbf{એકીકરણ}: જાહેરાતો ઓર્ગેનિક કન્ટેન્ટ સાથે કુદરતી રીતે ભળી જાય છે
\end{itemize}

\begin{mnemonicbox}PSRIL - Posts, Stories, Reels, IGTV, Live\end{mnemonicbox}
\end{solutionbox}

\questionmarks{5(b)}{4}{Email marketing શું છે? Email marketing ના વિવિધ પ્રકારો કયા છે?}

\begin{solutionbox}
\textbf{Email Marketing} એ વ્યક્તિગત ઇમેઇલ સંદેશાઓ દ્વારા ગ્રાહકો સાથે સીધો ડિજિટલ સંચાર છે.

\begin{center}
\captionof{table}{ઇમેઇલ માર્કેટિંગ પ્રકારો}
\begin{tabulary}{\linewidth}{|L|L|L|}
\hline
\textbf{પ્રકાર} & \textbf{હેતુ} & \textbf{ઉદાહરણ} \\ \hline
\textbf{ન્યૂઝલેટર} & નિયમિત અપડેટ્સ અને માહિતી & માસિક કંપની સમાચાર \\ \hline
\textbf{પ્રમોશનલ} & વેચાણ અને ઓફર્સ & ડિસ્કાઉન્ટ કોડ્સ, નવા પ્રોડક્ટ્સ \\ \hline
\textbf{ટ્રાન્ઝેક્શનલ} & ખરીદી પુષ્ટિકરણ & ઓર્ડર રસીદો, શિપિંગ અપડેટ્સ \\ \hline
\textbf{વેલકમ સિરીઝ} & નવા સબસ્ક્રાઇબર ઓનબોર્ડિંગ & બ્રાન્ડ અને પ્રોડક્ટ્સ પરિચય \\ \hline
\end{tabulary}
\end{center}

\begin{itemize}
    \item \textbf{ન્યૂઝલેટર}: મૂલ્યવાન કન્ટેન્ટ દ્વારા સંબંધો બનાવે છે
    \item \textbf{પ્રમોશનલ}: વેચાણ અને રૂપાંતર વધારે છે
    \item \textbf{ટ્રાન્ઝેક્શનલ}: આવશ્યક ગ્રાહક સેવા માહિતી પ્રદાન કરે છે
    \item \textbf{વેલકમ સિરીઝ}: નવા સબસ્ક્રાઇબર્સને ગ્રાહકોમાં ફેરવે છે
\end{itemize}

\begin{mnemonicbox}NPTW - Newsletter, Promotional, Transactional, Welcome\end{mnemonicbox}
\end{solutionbox}

\questionmarks{5(c)}{7}{Google Ads માં ઉપલબ્ધ વિવિધ પ્રકારના ad extensions ને ઉદાહરણ સાથે સમજાવો.}

\begin{solutionbox}
\begin{center}
\captionof{table}{Google Ad Extensions}
\begin{tabulary}{\linewidth}{|L|L|L|}
\hline
\textbf{Extension પ્રકાર} & \textbf{કાર્ય} & \textbf{ઉદાહરણ} \\ \hline
\textbf{Sitelink Extensions} & વધારાની પેજ લિંક્સ & "About Us", "Contact", "Products" \\ \hline
\textbf{Call Extensions} & ફોન નંબર ડિસ્પ્લે & "+1-800-123-4567" \\ \hline
\textbf{Location Extensions} & વ્યવસાય સરનામું & "123 Main St, City, State" \\ \hline
\textbf{Callout Extensions} & હાઈલાઈટ ફીચર્સ & "Free Shipping", "24/7 Support" \\ \hline
\textbf{Price Extensions} & પ્રોડક્ટ/સર્વિસ કિંમત & "Basic Plan: \$19/month" \\ \hline
\textbf{App Extensions} & મોબાઇલ એપ ડાઉનલોડ & "Download our iOS/Android app" \\ \hline
\end{tabulary}
\end{center}

\begin{center}
\begin{tikzpicture}[node distance=1.5cm, auto]
    \node [gtu block] (Extensions) {Google Ad Extensions};
    
    \node [gtu block, below left=1.5cm and 3cm of Extensions] (Sitelink) {Sitelink};
    \node [gtu block, below left=1.5cm and 0.5cm of Extensions] (Call) {Call};
    \node [gtu block, below=1.5cm of Extensions] (Location) {Location};
    \node [gtu block, below right=1.5cm and 0.5cm of Extensions] (Callout) {Callout};
    \node [gtu block, below right=1.5cm and 3cm of Extensions] (Price) {Price};
    \node [gtu block, below=2.5cm of Extensions] (App) {App};
    
    \foreach \n in {Sitelink, Call, Location, Callout, Price, App}
        \path [gtu arrow] (Extensions) -- (\n);
\end{tikzpicture}
\captionof{figure}{Ad Extensions ના પ્રકારો}
\end{center}

\textbf{લાભો:}
\begin{itemize}
    \item \textbf{વધારેલ CTR}: એક્સટેન્શન જાહેરાતોને વધુ પ્રબળ અને માહિતીપ્રદ બનાવે છે
    \item \textbf{બેટર ક્વોલિટી સ્કોર}: સુધારેલ એડ પરફોર્મન્સ ઓછા ખર્ચ તરફ દોરી જાય છે
    \item \textbf{વધારેલ યુઝર એક્સપિરિયન્સ}: યુઝર્સને વધુ સંબંધિત માહિતી મળે છે
    \item \textbf{સ્પર્ધાત્મક લાભ}: સ્પર્ધકો કરતાં વધુ સ્ક્રીન જગ્યા
\end{itemize}

\textbf{અમલીકરણ:}
\begin{itemize}
    \item \textbf{ઓટોમેટિક}: Google આપમેળે સંબંધિત એક્સટેન્શન બતાવી શકે છે
    \item \textbf{મેન્યુઅલ}: જાહેરાતકર્તાઓ વિશિષ્ટ એક્સટેન્શન બનાવી અને કસ્ટમાઇઝ કરી શકે છે
    \item \textbf{પરફોર્મન્સ}: આગાહી કરેલ અસરના આધારે એક્સટેન્શન બતાવવામાં આવે છે
\end{itemize}

\begin{mnemonicbox}SCLCPA - Sitelink, Call, Location, Callout, Price, App\end{mnemonicbox}
\end{solutionbox}

\questionmarks{5(a OR)}{3}{Social media marketing નું મહત્વ અને ફાયદા સમજાવો.}

\begin{solutionbox}
\begin{center}
\captionof{table}{સોશિયલ મીડિયા લાભો}
\begin{tabulary}{\linewidth}{|L|L|}
\hline
\textbf{લાભ} & \textbf{અસર} \\ \hline
\textbf{બ્રાન્ડ જાગૃતિ} & દૃશ્યતા અને ઓળખ વધારે છે \\ \hline
\textbf{ગ્રાહક એન્ગેજમેન્ટ} & સીધી ક્રિયાપ્રતિક્રિયા અને સંબંધ નિર્માણ \\ \hline
\textbf{ખર્ચ-અસરકારક} & પરંપરાગત જાહેરાતની તુલનામાં ઓછો ખર્ચ \\ \hline
\end{tabulary}
\end{center}

\begin{itemize}
    \item \textbf{બ્રાન્ડ જાગૃતિ}: શેરિંગ અને વાયરલ કન્ટેન્ટ દ્વારા ઘાતાંકીય પહોંચ
    \item \textbf{ગ્રાહક એન્ગેજમેન્ટ}: રિયલ-ટાઇમ પ્રતિસાદ અને સમુદાય નિર્માણ
    \item \textbf{ખર્ચ-અસરકારક}: લક્ષિત જાહેરાત વિકલ્પો સાથે ઉચ્ચ ROI
\end{itemize}

\begin{mnemonicbox}BEC - Brand awareness, Engagement, Cost-effective\end{mnemonicbox}
\end{solutionbox}

\questionmarks{5(b OR)}{4}{PPC અને SEO વચ્ચેનો તફાવત આપો.}

\begin{solutionbox}
\begin{center}
\captionof{table}{PPC vs SEO}
\begin{tabulary}{\linewidth}{|L|L|L|}
\hline
\textbf{પાસું} & \textbf{PPC (Pay-Per-Click)} & \textbf{SEO (Search Engine Optimization)} \\ \hline
\textbf{ખર્ચ} & ચૂકવેલ જાહેરાત & ઓર્ગેનિક/મફત ટ્રાફિક \\ \hline
\textbf{પરિણામો} & તાત્કાલિક દૃશ્યતા & લાંબા ગાળાના ટકાઉ પરિણામો \\ \hline
\textbf{નિયંત્રણ} & જાહેરાતો પર સંપૂર્ણ નિયંત્રણ & રેન્કિંગ પર મર્યાદિત નિયંત્રણ \\ \hline
\textbf{અવધિ} & જ્યારે ચૂકવણી બંધ થાય ત્યારે પરિણામો અટકી જાય છે & લાંબા સમય સુધી ચાલતા પરિણામો \\ \hline
\end{tabulary}
\end{center}

\begin{itemize}
    \item \textbf{PPC}: તાત્કાલિક પરિણામો પરંતુ ચાલુ રોકાણ જરૂરી છે
    \item \textbf{SEO}: બનાવવા માટે સમય લે છે પરંતુ ટકાઉ લાંબા ગાળાનું મૂલ્ય પ્રદાન કરે છે
    \item \textbf{એકીકરણ}: શ્રેષ્ઠ પરિણામો બંને વ્યૂહરચનાઓને સંયોજિત કરવાથી મળે છે
    \item \textbf{બજેટ}: PPC ને જાહેરાત બજેટની જરૂર છે; SEO ને સમય રોકાણની જરૂર છે
\end{itemize}

\begin{mnemonicbox}ICRD - Immediate vs Continuous, Results vs Duration\end{mnemonicbox}
\end{solutionbox}

\questionmarks{5(c OR)}{7}{Google AdWords માં Quality Score નો ખ્યાલ અને જાહેરાત રેન્કિંગ પર તેની અસર સમજાવો.}

\begin{solutionbox}
\textbf{ક્વોલિટી સ્કોર} એ Google નું જાહેરાત ગુણવત્તા, કીવર્ડ્સ અને લેન્ડિંગ પેજોનું રેટિંગ (1-10) છે.

\begin{center}
\captionof{table}{ક્વોલિટી સ્કોર ઘટકો}
\begin{tabulary}{\linewidth}{|L|L|L|}
\hline
\textbf{ઘટક} & \textbf{વજન} & \textbf{અસર} \\ \hline
\textbf{અપેક્ષિત CTR} & ઉચ્ચ & જે સંભાવના છે કે યુઝર્સ ક્લિક કરશે \\ \hline
\textbf{જાહેરાત સુસંગતતા} & ઉચ્ચ & જાહેરાત સર્ચ ઈરાદા સાથે કેટલી નજીક છે \\ \hline
\textbf{લેન્ડિંગ પેજ અનુભવ} & મધ્યમ & પેજ ગુણવત્તા અને યુઝર એક્સપિરિયન્સ \\ \hline
\end{tabulary}
\end{center}

\begin{center}
\begin{tikzpicture}[node distance=1.5cm, auto]
    \node [gtu block] (Quality) {ક્વોલિટી સ્કોર};
    
    \node [gtu block, below left=1.2cm and 1cm of Quality] (CTR) {અપેક્ષિત CTR};
    \node [gtu block, below=1.2cm of Quality] (Relevance) {Ad સુસંગતતા};
    \node [gtu block, below right=1.2cm and 1cm of Quality] (LP) {લેન્ડિંગ પેજ\\અનુભવ};
    
    \node [gtu state, below=0.8cm of CTR] (Rank) {Ad રેન્કિંગ};
    \node [gtu state, below=0.8cm of Relevance] (CPC) {Cost Per Click};
    \node [gtu state, below=0.8cm of LP] (Pos) {Ad પોઝિશન};
    
    \foreach \n in {CTR, Relevance, LP}
        \path [gtu arrow] (Quality) -- (\n);
        
    \path [gtu arrow] (CTR) -- (Rank);
    \path [gtu arrow] (Relevance) -- (CPC);
    \path [gtu arrow] (LP) -- (Pos);
\end{tikzpicture}
\captionof{figure}{ક્વોલિટી સ્કોર અસર}
\end{center}

\textbf{જાહેરાત રેન્કિંગ પર અસર:}
\begin{center}
\captionof{table}{રેન્કિંગ અસર}
\begin{tabulary}{\linewidth}{|L|L|L|}
\hline
\textbf{ક્વોલિટી સ્કોર} & \textbf{Ad Rank અસર} & \textbf{ખર્ચ અસર} \\ \hline
\textbf{ઉચ્ચ (8-10)} & ઉચ્ચ સ્થાનો & ઓછો CPC \\ \hline
\textbf{મધ્યમ (5-7)} & સરેરાશ સ્થાનો & સરેરાશ CPC \\ \hline
\textbf{નીચું (1-4)} & નીચલા સ્થાનો & ઉચ્ચ CPC \\ \hline
\end{tabulary}
\end{center}

\textbf{ઉચ્ચ ક્વોલિટી સ્કોરના લાભો:}
\begin{itemize}
    \item \textbf{ઓછો ખર્ચ}: સ્પર્ધકો કરતા પ્રતિ ક્લિક ઓછું ચૂકવો
    \item \textbf{વધુ સારા સ્થાનો}: સર્ચ પરિણામોમાં ઉચ્ચ દેખાય છે
    \item \textbf{વધારેલ દૃશ્યતા}: વધુ જાહેરાત એક્સટેન્શન યોગ્યતા
    \item \textbf{સુધારેલ ROI}: ઓછા ખર્ચે વધુ સારું પ્રદર્શન
\end{itemize}

\textbf{ઑપ્ટિમાઇઝેશન વ્યૂહરચનાઓ:}
\begin{itemize}
    \item \textbf{કીવર્ડ સુસંગતતા}: કીવર્ડ્સને જાહેરાત કોપી સાથે નજીકથી મેચ કરો
    \item \textbf{Ad કોપી ગુણવત્તા}: આકર્ષક, સંબંધિત જાહેરાત ટેક્સ્ટ લખો
    \item \textbf{લેન્ડિંગ પેજ}: ઝડપી, સંબંધિત, યુઝર-ફ્રેન્ડલી પેજોની ખાતરી કરો
    \item \textbf{એકાઉન્ટ સ્ટ્રક્ચર}: કેમ્પેઇન અને એડ જૂથોને તાર્કિક રીતે ગોઠવો
\end{itemize}

\begin{mnemonicbox}EAL-RCP - Expected CTR, Ad relevance, Landing page affect Rank, Cost, Position\end{mnemonicbox}
\end{solutionbox}

\end{document}
