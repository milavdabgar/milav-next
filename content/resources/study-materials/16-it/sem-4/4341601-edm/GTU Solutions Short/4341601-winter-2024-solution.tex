\documentclass{article}

% content/resources/templates/preamble.tex
\usepackage[margin=0.6in]{geometry}
\author{Milav Dabgar}
\usepackage{amsmath,amssymb,amsthm}
\usepackage{booktabs}
\usepackage{multirow}
\usepackage{xcolor}
\usepackage{tcolorbox}
\tcbuselibrary{breakable,skins}
\usepackage[colorlinks=true,linkcolor=blue]{hyperref}
\usepackage{titlesec}
\usepackage{enumitem}
\usepackage{tikz}
\usepackage{pgfplots}
\usepackage{circuitikz}
\usepackage[version=4]{mhchem}
\usepackage{longtable}
\usepackage{array}
\usepackage{float}
\usepackage{caption}
\usepackage{listings}

\lstset{
  basicstyle=\small\ttfamily,
  breaklines=true,
  breakatwhitespace=false,
  postbreak=\mbox{\textcolor{red}{$\hookrightarrow$}\space},
  float=false,
  numbers=left,
  numberstyle=\tiny\color{gray},
  numbersep=10pt,
  xleftmargin=2em,
  keywordstyle=\color{blue},
  commentstyle=\color{green!60!black},
  stringstyle=\color{purple},
  backgroundcolor=\color{gray!5},
  showstringspaces=false,
  tabsize=2,
  captionpos=b,
  keepspaces=true,
  columns=flexible
}

\pgfplotsset{compat=1.18}
\usetikzlibrary{shapes,arrows,positioning,calc,patterns,decorations.pathmorphing,decorations.markings,arrows.meta}

% Color scheme
\definecolor{headcolor}{RGB}{0,102,204}
\definecolor{keycolor}{RGB}{220,20,60}
\definecolor{solutioncolor}{RGB}{34,139,34}
\definecolor{mnemoniccolor}{RGB}{148,0,211}
\definecolor{codecolor}{RGB}{0,0,100}

% Spacing
\setlength{\parskip}{3pt}
\setlist[itemize]{nosep}
\setlist[enumerate]{nosep}

% Title formatting
\titleformat{\section}{\Large\bfseries\color{headcolor}}{\thesection}{1em}{}
\titleformat{\subsection}{\large\bfseries\color{headcolor}}{\thesubsection}{1em}{}

% Pandoc tightlist compatibility
\providecommand{\tightlist}{%
  \setlength{\itemsep}{0pt}\setlength{\parskip}{0pt}}

% Pandoc longtable compatibility
\newcounter{none}
\def\thenone{}


% content/resources/templates/english-boxes.tex

% Custom environments
\newtcolorbox{solutionbox}{
 breakable,
 enhanced,
 colback=solutioncolor!5!white,
 colframe=solutioncolor!75!black,
 fonttitle=\bfseries,
 title=Solution
}

\newtcolorbox{solutionboxnobreak}{
 colback=solutioncolor!5!white,
 colframe=solutioncolor!75!black,
 fonttitle=\bfseries,
 title=Solution
}

\newtcolorbox{keyformula}{
 breakable,
 enhanced,
 colback=keycolor!5!white,
 colframe=keycolor!75!black,
 fonttitle=\bfseries,
 title=Key Formula
}

\newtcolorbox{mnemonicboxenv}{
 breakable,
 enhanced,
 colback=mnemoniccolor!5!white,
 colframe=mnemoniccolor!75!black,
 fonttitle=\bfseries,
 title=Mnemonic
}

\newcommand{\mnemonicbox}[1]{%
  \begin{mnemonicboxenv}
    #1
  \end{mnemonicboxenv}
}


% Custom commands for GTU solutions
% This file defines semantic commands for consistent formatting

% Question command with automatic formatting
\newcommand{\question}[2]{%
  \section*{Question #1}%
  \textbf{#2}%
}

% OR question variant
\newcommand{\questionor}[2]{%
  \section*{Question #1 OR}%
  \textbf{#2}%
}

% Proper table environment with caption
\newenvironment{answertable}[1]{%
  \begin{table}[htbp]
  \centering
  \caption{#1}
}{%
  \end{table}
}

% Proper figure environment for diagrams
\newenvironment{answerdiagram}[1]{%
  \begin{figure}[htbp]
  \centering
  \caption{#1}
}{%
  \end{figure}
}

% Semantic markup for key terms
\newcommand{\keyword}[1]{\textbf{#1}}
\newcommand{\code}[1]{\texttt{#1}}
\newcommand{\classname}[1]{\texttt{#1}}
\newcommand{\methodname}[1]{\texttt{#1}}

% Proper quotation marks
\newcommand{\mnemonic}[1]{``#1''}


\title{Essentials of Digital Marketing (4341601) - Winter 2024 Solution}
\date{November 22, 2024}

\begin{document}
\maketitle

\questionmarks{1(a)}{3}{Explain three important factors that influence a website's SEO ranking.}

\begin{solutionbox}
\begin{center}
\captionof{table}{SEO Ranking Factors}
\begin{tabulary}{\linewidth}{|L|L|}
\hline
\textbf{Factor} & \textbf{Description} \\ \hline
\textbf{Content Quality} & Fresh, relevant, keyword-optimized content that provides value to users \\ \hline
\textbf{Backlinks} & High-quality external websites linking to your site (domain authority) \\ \hline
\textbf{Technical SEO} & Site speed, mobile-friendliness, SSL certificate, and proper site structure \\ \hline
\end{tabulary}
\end{center}

\begin{itemize}
    \item \textbf{Content Quality}: Search engines prioritize websites with original, valuable content
    \item \textbf{Backlinks}: Act as votes of confidence from other websites
    \item \textbf{Technical SEO}: Ensures search engines can crawl and index your site efficiently
\end{itemize}

\begin{mnemonicbox}CBT - Content, Backlinks, Technical\end{mnemonicbox}
\end{solutionbox}

\questionmarks{1(b)}{4}{Define data privacy and its importance in digital marketing.}

\begin{solutionbox}
\textbf{Data Privacy} is the protection of personal information collected from users during digital marketing activities.

\begin{center}
\captionof{table}{Importance of Data Privacy}
\begin{tabulary}{\linewidth}{|L|L|}
\hline
\textbf{Aspect} & \textbf{Importance} \\ \hline
\textbf{User Trust} & Builds customer confidence and loyalty \\ \hline
\textbf{Legal Compliance} & Avoids penalties from GDPR, CCPA regulations \\ \hline
\textbf{Brand Reputation} & Prevents negative publicity from data breaches \\ \hline
\end{tabulary}
\end{center}

\begin{itemize}
    \item \textbf{User Trust}: Customers share more data when they trust your privacy practices
    \item \textbf{Legal Compliance}: Mandatory compliance with data protection laws
    \item \textbf{Brand Reputation}: Data breaches can severely damage brand image
\end{itemize}

\begin{mnemonicbox}TLR - Trust, Legal, Reputation\end{mnemonicbox}
\end{solutionbox}

\questionmarks{1(c)}{7}{Explain the key components of a digital marketing plan.}

\begin{solutionbox}
\begin{center}
\captionof{table}{Digital Marketing Plan Components}
\begin{tabulary}{\linewidth}{|L|L|}
\hline
\textbf{Component} & \textbf{Description} \\ \hline
\textbf{Goals \& Objectives} & SMART goals aligned with business objectives \\ \hline
\textbf{Target Audience} & Demographics, psychographics, and behavior analysis \\ \hline
\textbf{Channel Strategy} & Selection of appropriate digital platforms \\ \hline
\textbf{Content Strategy} & Content types, themes, and publishing schedule \\ \hline
\textbf{Budget Allocation} & Resource distribution across channels \\ \hline
\textbf{Analytics \& KPIs} & Measurement frameworks and success metrics \\ \hline
\end{tabulary}
\end{center}

\begin{center}
\begin{tikzpicture}[node distance=1.5cm, auto]
    \node [gtu block, minimum width=3cm] (Plan) {Digital Marketing\\Plan};
    \node [gtu block, below left=1.5cm and 2cm of Plan] (Goals) {Goals \&\\Objectives};
    \node [gtu block, below left=1.5cm and -0.5cm of Plan] (Audience) {Target\\Audience};
    \node [gtu block, below=1.5cm of Plan] (Channels) {Channel\\Strategy};
    \node [gtu block, below right=1.5cm and -0.5cm of Plan] (Content) {Content\\Strategy};
    \node [gtu block, below right=1.5cm and 2cm of Plan] (Budget) {Budget\\Allocation};
    \node [gtu block, below=1.5cm of Channels] (Analytics) {Analytics \&\\KPIs};

    \path [gtu arrow] (Plan) -- (Goals);
    \path [gtu arrow] (Plan) -- (Audience);
    \path [gtu arrow] (Plan) -- (Channels);
    \path [gtu arrow] (Plan) -- (Content);
    \path [gtu arrow] (Plan) -- (Budget);
    \path [gtu arrow] (Plan) -- (Analytics);
\end{tikzpicture}
\captionof{figure}{Digital Marketing Plan Structure}
\end{center}

\begin{itemize}
    \item \textbf{Goals \& Objectives}: Define specific, measurable outcomes
    \item \textbf{Target Audience}: Create detailed buyer personas
    \item \textbf{Channel Strategy}: Choose optimal mix of social media, email, SEO, PPC
    \item \textbf{Content Strategy}: Develop engaging content calendar
    \item \textbf{Budget Allocation}: Distribute resources based on ROI potential
    \item \textbf{Analytics \& KPIs}: Track performance and optimize continuously
\end{itemize}

\begin{mnemonicbox}GT-CCBA - Goals-Target, Channels-Content-Budget-Analytics\end{mnemonicbox}
\end{solutionbox}

\questionmarks{1(c OR)}{7}{Define the P.O.E.M. Framework and explain its importance in digital marketing.}

\begin{solutionbox}
\textbf{P.O.E.M.} stands for \textbf{Paid, Owned, Earned, Media} framework for digital marketing strategy.

\begin{center}
\captionof{table}{P.O.E.M Framework}
\begin{tabulary}{\linewidth}{|L|L|L|}
\hline
\textbf{Media Type} & \textbf{Description} & \textbf{Examples} \\ \hline
\textbf{Paid} & Media you pay for & Google Ads, Facebook Ads, YouTube Ads \\ \hline
\textbf{Owned} & Media you control & Website, Blog, Email list, Mobile app \\ \hline
\textbf{Earned} & Media gained through credibility & Social shares, Reviews, PR mentions \\ \hline
\end{tabulary}
\end{center}

\begin{center}
\begin{tikzpicture}[node distance=1.5cm, auto]
    \node [gtu block] (Main) {P.O.E.M Framework};
    
    \node [gtu block, below left=1.5cm and 1cm of Main] (Paid) {Paid Media};
    \node [gtu block, below=1.5cm of Main] (Owned) {Owned Media};
    \node [gtu block, below right=1.5cm and 1cm of Main] (Earned) {Earned Media};
    
    \node [gtu state, below=0.8cm of Paid] (Reach) {Immediate\\Reach};
    \node [gtu state, below=0.8cm of Owned] (Asset) {Long-term\\Asset};
    \node [gtu state, below=0.8cm of Earned] (Trust) {Trust \&\\Credibility};
    
    \path [gtu arrow] (Main) -- (Paid);
    \path [gtu arrow] (Main) -- (Owned);
    \path [gtu arrow] (Main) -- (Earned);
    
    \path [gtu arrow] (Paid) -- (Reach);
    \path [gtu arrow] (Owned) -- (Asset);
    \path [gtu arrow] (Earned) -- (Trust);
\end{tikzpicture}
\captionof{figure}{P.O.E.M Framework}
\end{center}

\begin{itemize}
    \item \textbf{Paid Media}: Provides immediate visibility and targeted reach
    \item \textbf{Owned Media}: Creates long-term assets and brand control
    \item \textbf{Earned Media}: Builds trust and authentic brand advocacy
\end{itemize}

\begin{mnemonicbox}POE - Pay, Own, Earn\end{mnemonicbox}
\end{solutionbox}

\questionmarks{2(a)}{3}{Differentiate between black hat and white hat SEO techniques.}

\begin{solutionbox}
\begin{center}
\captionof{table}{White Hat vs Black Hat SEO}
\begin{tabulary}{\linewidth}{|L|L|L|}
\hline
\textbf{Aspect} & \textbf{White Hat SEO} & \textbf{Black Hat SEO} \\ \hline
\textbf{Methods} & Ethical, guideline-compliant & Manipulative, rule-breaking \\ \hline
\textbf{Results} & Sustainable long-term growth & Quick but temporary gains \\ \hline
\textbf{Risk} & Safe from penalties & High risk of penalties \\ \hline
\end{tabulary}
\end{center}

\begin{itemize}
    \item \textbf{White Hat SEO}: Follows search engine guidelines for sustainable results
    \item \textbf{Black Hat SEO}: Uses deceptive practices for quick ranking gains
    \item \textbf{Risk Factor}: Black hat techniques can result in complete site bans
\end{itemize}

\begin{mnemonicbox}WEB - White Ethical Benefits, Black Breaks-rules\end{mnemonicbox}
\end{solutionbox}

\questionmarks{2(b)}{4}{Explain how search engine algorithms work and how they rank websites.}

\begin{solutionbox}
\begin{center}
\captionof{table}{Search Engine Process}
\begin{tabulary}{\linewidth}{|L|L|}
\hline
\textbf{Process} & \textbf{Function} \\ \hline
\textbf{Crawling} & Bots discover and scan web pages \\ \hline
\textbf{Indexing} & Pages stored in search engine database \\ \hline
\textbf{Ranking} & Algorithm determines page relevance and authority \\ \hline
\textbf{Results} & Best matches displayed for user queries \\ \hline
\end{tabulary}
\end{center}

\begin{itemize}
    \item \textbf{Crawling}: Web crawlers follow links to find new content
    \item \textbf{Indexing}: Content analyzed and stored in massive databases
    \item \textbf{Ranking}: 200+ factors determine search result positions
    \item \textbf{Results}: Most relevant pages shown first to users
\end{itemize}

\begin{mnemonicbox}CIRR - Crawl, Index, Rank, Results\end{mnemonicbox}
\end{solutionbox}

\questionmarks{2(c)}{7}{Describe the strategies for building backlinks.}

\begin{solutionbox}
\begin{center}
\captionof{table}{Backlink Strategies}
\begin{tabulary}{\linewidth}{|L|L|L|}
\hline
\textbf{Strategy} & \textbf{Description} & \textbf{Effectiveness} \\ \hline
\textbf{Guest Posting} & Write articles for other websites & High \\ \hline
\textbf{Resource Link Building} & Get listed in industry directories & Medium \\ \hline
\textbf{Broken Link Building} & Replace broken links with your content & High \\ \hline
\textbf{Content Marketing} & Create shareable, valuable content & Very High \\ \hline
\textbf{Influencer Outreach} & Partner with industry influencers & High \\ \hline
\end{tabulary}
\end{center}

\begin{center}
\begin{tikzpicture}[node distance=1.5cm, auto]
    \node [gtu block] (Backlinks) {Backlink Building};
    
    \node [gtu block, below left=1.5cm and 2cm of Backlinks] (Guest) {Guest\\Posting};
    \node [gtu block, below left=1.5cm and -0.5cm of Backlinks] (Resource) {Resource\\Links};
    \node [gtu block, below=1.5cm of Backlinks] (Broken) {Broken Link\\Building};
    \node [gtu block, below right=1.5cm and -0.5cm of Backlinks] (Content) {Content\\Marketing};
    \node [gtu block, below right=1.5cm and 2cm of Backlinks] (Influencer) {Influencer\\Outreach};
    
    \foreach \n in {Guest, Resource, Broken, Content, Influencer}
        \path [gtu arrow] (Backlinks) -- (\n);
\end{tikzpicture}
\captionof{figure}{Backlink Building Strategies}
\end{center}

\begin{itemize}
    \item \textbf{Guest Posting}: Builds relationships and authority in your niche
    \item \textbf{Resource Link Building}: Establishes credibility through directories
    \item \textbf{Broken Link Building}: Provides value by fixing broken resources
    \item \textbf{Content Marketing}: Naturally attracts links through quality content
    \item \textbf{Influencer Outreach}: Leverages established audiences for link opportunities
\end{itemize}

\begin{mnemonicbox}GRBCI - Guest, Resource, Broken, Content, Influencer\end{mnemonicbox}
\end{solutionbox}

\questionmarks{2(a OR)}{3}{Explain the importance of backlinks, website speed and performance in search engine ranking.}

\begin{solutionbox}
\begin{center}
\captionof{table}{SEO Factors Impact}
\begin{tabulary}{\linewidth}{|L|L|}
\hline
\textbf{Factor} & \textbf{Impact on SEO} \\ \hline
\textbf{Backlinks} & Authority and trust signals \\ \hline
\textbf{Website Speed} & User experience ranking factor \\ \hline
\textbf{Performance} & Core Web Vitals affect rankings \\ \hline
\end{tabulary}
\end{center}

\begin{itemize}
    \item \textbf{Backlinks}: Act as votes of confidence from other websites
    \item \textbf{Website Speed}: Faster sites rank higher and reduce bounce rates
    \item \textbf{Performance}: Google prioritizes sites with good Core Web Vitals
\end{itemize}

\begin{mnemonicbox}BSP - Backlinks, Speed, Performance\end{mnemonicbox}
\end{solutionbox}

\questionmarks{2(b OR)}{4}{Differentiate between on-page and off-page SEO, and provide examples of each.}

\begin{solutionbox}
\begin{center}
\captionof{table}{On-Page vs Off-Page SEO}
\begin{tabulary}{\linewidth}{|L|L|L|}
\hline
\textbf{SEO Type} & \textbf{Focus} & \textbf{Examples} \\ \hline
\textbf{On-Page} & Website optimization & Title tags, meta descriptions, content optimization \\ \hline
\textbf{Off-Page} & External factors & Backlinks, social signals, brand mentions \\ \hline
\end{tabulary}
\end{center}

\begin{itemize}
    \item \textbf{On-Page SEO}: Controls elements within your website
    \item \textbf{Off-Page SEO}: Builds authority through external validation
    \item \textbf{Examples}: On-page includes keyword optimization; off-page includes link building
\end{itemize}

\begin{mnemonicbox}IO - Internal Optimization, External Elevation\end{mnemonicbox}
\end{solutionbox}

\questionmarks{2(c OR)}{7}{Explain Different ways to improve SEO rankings.}

\begin{solutionbox}
\begin{center}
\captionof{table}{SEO Improvement Methods}
\begin{tabulary}{\linewidth}{|L|L|L|}
\hline
\textbf{Method} & \textbf{Description} & \textbf{Impact} \\ \hline
\textbf{Keyword Research} & Target relevant, low-competition keywords & High \\ \hline
\textbf{Content Optimization} & Create valuable, keyword-rich content & Very High \\ \hline
\textbf{Technical SEO} & Improve site speed, mobile-friendliness & High \\ \hline
\textbf{Link Building} & Acquire quality backlinks & Very High \\ \hline
\textbf{User Experience} & Enhance site usability and engagement & Medium \\ \hline
\textbf{Local SEO} & Optimize for local search results & High (for local business) \\ \hline
\end{tabulary}
\end{center}

\begin{center}
\begin{tikzpicture}[node distance=1.5cm, auto]
    \node [gtu block] (SEO) {SEO Improvement};
    
    \node [gtu block, below left=1.5cm and 2.5cm of SEO] (Keys) {Keyword\\Research};
    \node [gtu block, below left=1.5cm and 0cm of SEO] (Content) {Content\\Optimization};
    \node [gtu block, below=1.5cm of SEO] (Tech) {Technical\\SEO};
    \node [gtu block, below right=1.5cm and 0cm of SEO] (Links) {Link\\Building};
    \node [gtu block, below right=1.5cm and 2.5cm of SEO] (UX) {User\\Experience};
    \node [gtu block, below=3.5cm of SEO] (Local) {Local\\SEO};
    
    \foreach \n in {Keys, Content, Tech, Links, UX, Local}
        \path [gtu arrow] (SEO) -- (\n);
\end{tikzpicture}
\captionof{figure}{Ways to Improve SEO}
\end{center}

\begin{itemize}
    \item \textbf{Keyword Research}: Foundation for all SEO efforts
    \item \textbf{Content Optimization}: Provides value while targeting keywords
    \item \textbf{Technical SEO}: Ensures search engines can effectively crawl your site
    \item \textbf{Link Building}: Builds domain authority and trust
    \item \textbf{User Experience}: Reduces bounce rate and increases engagement
    \item \textbf{Local SEO}: Critical for businesses with physical locations
\end{itemize}

\begin{mnemonicbox}KC-TLUL - Keywords, Content, Technical, Links, User-experience, Local\end{mnemonicbox}
\end{solutionbox}

\questionmarks{3(a)}{3}{Differentiate between single-touch and multi-touch attribution models.}

\begin{solutionbox}
\begin{center}
\captionof{table}{Attribution Models}
\begin{tabulary}{\linewidth}{|L|L|L|}
\hline
\textbf{Model Type} & \textbf{Credit Assignment} & \textbf{Use Case} \\ \hline
\textbf{Single-Touch} & 100\% credit to one touchpoint & Simple customer journeys \\ \hline
\textbf{Multi-Touch} & Credit distributed across touchpoints & Complex customer journeys \\ \hline
\end{tabulary}
\end{center}

\begin{itemize}
    \item \textbf{Single-Touch}: First-click or last-click gets full credit
    \item \textbf{Multi-Touch}: Linear, time-decay, or position-based attribution
    \item \textbf{Usage}: Multi-touch provides more accurate customer journey insights
\end{itemize}

\begin{mnemonicbox}SM - Single Simple, Multi Multiple\end{mnemonicbox}
\end{solutionbox}

\questionmarks{3(b)}{4}{Explain how businesses can set up goals in Google Analytics.}

\begin{solutionbox}
\begin{center}
\captionof{table}{Setting up Goals}
\begin{tabulary}{\linewidth}{|L|L|}
\hline
\textbf{Step} & \textbf{Action} \\ \hline
\textbf{1. Access Goals} & Navigate to Admin $\rightarrow$ View $\rightarrow$ Goals \\ \hline
\textbf{2. Choose Template} & Select from template or create custom \\ \hline
\textbf{3. Configure Details} & Set goal name, type, and conditions \\ \hline
\textbf{4. Verify Setup} & Test goal using verification feature \\ \hline
\end{tabulary}
\end{center}

\begin{itemize}
    \item \textbf{Goal Types}: Destination, Duration, Pages/Session, Event goals
    \item \textbf{Configuration}: Define specific conditions for goal completion
    \item \textbf{Verification}: Ensure goals track correctly before implementation
    \item \textbf{Monitoring}: Regular review and optimization of goal performance
\end{itemize}

\begin{mnemonicbox}ACCV - Access, Choose, Configure, Verify\end{mnemonicbox}
\end{solutionbox}

\questionmarks{3(c)}{7}{What is the role of web analytics in formulation of digital marketing strategy? Discuss different types of web analytics.}

\begin{solutionbox}
\textbf{Role in Strategy:}
Web analytics provides data-driven insights for informed decision-making in digital marketing.

\begin{center}
\captionof{table}{Web Analytics Types}
\begin{tabulary}{\linewidth}{|L|L|L|}
\hline
\textbf{Analytics Type} & \textbf{Purpose} & \textbf{Key Metrics} \\ \hline
\textbf{Content Analytics} & Content performance tracking & Page views, time on page, bounce rate \\ \hline
\textbf{Customer Analytics} & User behavior analysis & Demographics, interests, conversion paths \\ \hline
\textbf{Social Media Analytics} & Social engagement measurement & Shares, likes, comments, reach \\ \hline
\textbf{SEO Analytics} & Search performance tracking & Keywords, rankings, organic traffic \\ \hline
\textbf{Conversion Analytics} & Goal completion tracking & Conversion rate, revenue, ROI \\ \hline
\end{tabulary}
\end{center}

\begin{center}
\begin{tikzpicture}[node distance=1.5cm, auto]
    \node [gtu block, minimum width=3cm] (Strategy) {Digital Marketing\\Strategy Formulation};
    \node [gtu block, below=1.5cm of Strategy] (Analytics) {Web Analytics};
    
    \node [gtu block, below left=1.5cm and 3cm of Analytics] (Content) {Content\\Analytics};
    \node [gtu block, below left=1.5cm and 0.5cm of Analytics] (Customer) {Customer\\Analytics};
    \node [gtu block, below=1.5cm of Analytics] (Social) {Social\\Analytics};
    \node [gtu block, below right=1.5cm and 0.5cm of Analytics] (SEO) {SEO\\Analytics};
    \node [gtu block, below right=1.5cm and 3cm of Analytics] (Conversion) {Conversion\\Analytics};
    
    \path [gtu arrow] (Analytics) -- (Strategy);
    \foreach \n in {Content, Customer, Social, SEO, Conversion}
        \path [gtu arrow] (Analytics) -- (\n);
\end{tikzpicture}
\captionof{figure}{Role of Web Analytics}
\end{center}

\begin{itemize}
    \item \textbf{Strategic Role}: Identifies opportunities, measures performance, guides optimization
    \item \textbf{Content Analytics}: Helps optimize content strategy based on engagement
    \item \textbf{Customer Analytics}: Enables better audience targeting and personalization
    \item \textbf{Social Media Analytics}: Measures social media ROI and engagement
    \item \textbf{SEO Analytics}: Tracks organic search performance and opportunities
    \item \textbf{Conversion Analytics}: Measures bottom-line impact of marketing efforts
\end{itemize}

\begin{mnemonicbox}CCSSC - Content, Customer, Social, SEO, Conversion\end{mnemonicbox}
\end{solutionbox}

\questionmarks{3(a OR)}{3}{Define the terms: Unique visitors, Average Visit Duration, Bounce rate.}

\begin{solutionbox}
\begin{center}
\captionof{table}{Web Metrics Definitions}
\begin{tabulary}{\linewidth}{|L|L|}
\hline
\textbf{Metric} & \textbf{Definition} \\ \hline
\textbf{Unique Visitors} & Individual users visiting site in specific time period \\ \hline
\textbf{Average Visit Duration} & Average time users spend on website per session \\ \hline
\textbf{Bounce Rate} & Percentage of visitors leaving after viewing one page \\ \hline
\end{tabulary}
\end{center}

\begin{itemize}
    \item \textbf{Unique Visitors}: Counts each person once, regardless of return visits
    \item \textbf{Average Visit Duration}: Indicates content engagement and site stickiness
    \item \textbf{Bounce Rate}: High rates may indicate poor content match or site issues
\end{itemize}

\begin{mnemonicbox}UAB - Unique, Average, Bounce\end{mnemonicbox}
\end{solutionbox}

\questionmarks{3(b OR)}{4}{Explain A/B testing in web analytics.}

\begin{solutionbox}
\textbf{A/B Testing} is comparing two versions of a webpage to determine which performs better.

\begin{center}
\captionof{table}{A/B Testing Components}
\begin{tabulary}{\linewidth}{|L|L|}
\hline
\textbf{Component} & \textbf{Description} \\ \hline
\textbf{Version A} & Original webpage (control) \\ \hline
\textbf{Version B} & Modified webpage (variant) \\ \hline
\textbf{Traffic Split} & Usually 50/50 random distribution \\ \hline
\textbf{Metrics} & Conversion rate, click-through rate, engagement \\ \hline
\end{tabulary}
\end{center}

\begin{itemize}
    \item \textbf{Process}: Split traffic between two versions and measure performance
    \item \textbf{Duration}: Run tests long enough for statistical significance
    \item \textbf{Variables}: Test one element at a time (headlines, buttons, images)
    \item \textbf{Decision}: Implement winning version based on data
\end{itemize}

\begin{mnemonicbox}ABCD - A-version, B-version, Compare, Decide\end{mnemonicbox}
\end{solutionbox}

\questionmarks{3(c OR)}{7}{Explain following tracking code with their pros and cons: Long tracking code, Obfuscated tracking code, UTM codes}

\begin{solutionbox}
\begin{center}
\captionof{table}{Tracking Code Types}
\begin{tabulary}{\linewidth}{|L|L|L|L|}
\hline
\textbf{Tracking Type} & \textbf{Description} & \textbf{Pros} & \textbf{Cons} \\ \hline
\textbf{Long Tracking Code} & Detailed parameters for comprehensive tracking & Complete data collection, detailed insights & Slow page load, complex implementation \\ \hline
\textbf{Obfuscated Tracking} & Encrypted/hidden tracking parameters & Data security, prevents tampering & Difficult debugging, complex setup \\ \hline
\textbf{UTM Codes} & URL parameters for campaign tracking & Easy implementation, campaign attribution & Manual tagging required, URL appearance \\ \hline
\end{tabulary}
\end{center}

\begin{center}
\begin{tikzpicture}[node distance=1.5cm, auto]
    \node [gtu block] (Tracking) {Tracking Codes};
    
    \node [gtu block, below left=1.5cm and 2cm of Tracking] (Long) {Long Tracking};
    \node [gtu block, below=1.5cm of Tracking] (Obfuscated) {Obfuscated\\Tracking};
    \node [gtu block, below right=1.5cm and 2cm of Tracking] (UTM) {UTM Codes};
    
    \node [gtu state, below=0.8cm of Long] (T1) {Comprehensive\\Data};
    \node [gtu state, below=0.8cm of Obfuscated] (T2) {Secure\\Tracking};
    \node [gtu state, below=0.8cm of UTM] (T3) {Campaign\\Attribution};
    
    \foreach \n in {Long, Obfuscated, UTM}
        \path [gtu arrow] (Tracking) -- (\n);
        
    \path [gtu arrow] (Long) -- (T1);
    \path [gtu arrow] (Obfuscated) -- (T2);
    \path [gtu arrow] (UTM) -- (T3);
\end{tikzpicture}
\captionof{figure}{Tracking Code Comparison}
\end{center}

\begin{itemize}
    \item \textbf{Long Tracking Code}: Best for enterprise-level detailed analytics
    \item \textbf{Obfuscated Tracking}: Ideal for sensitive data protection requirements
    \item \textbf{UTM Codes}: Perfect for campaign tracking and traffic source identification
\end{itemize}

\begin{mnemonicbox}LOU - Long comprehensive, Obfuscated secure, UTM simple\end{mnemonicbox}
\end{solutionbox}

\questionmarks{4(a)}{3}{Explain different types of YouTube ads.}

\begin{solutionbox}
\begin{center}
\captionof{table}{YouTube Ad Types}
\begin{tabulary}{\linewidth}{|L|L|L|}
\hline
\textbf{Ad Type} & \textbf{Format} & \textbf{Placement} \\ \hline
\textbf{Skippable In-Stream} & 5-second skip option & Before/during videos \\ \hline
\textbf{Non-Skippable} & 15-20 seconds, no skip & Before/during videos \\ \hline
\textbf{Bumper Ads} & 6 seconds, non-skippable & Before videos \\ \hline
\end{tabulary}
\end{center}

\begin{itemize}
    \item \textbf{Skippable In-Stream}: Cost-effective, pay only for engaged viewers
    \item \textbf{Non-Skippable}: Guaranteed message delivery, higher completion rates
    \item \textbf{Bumper Ads}: Brand awareness, quick memorable messages
\end{itemize}

\begin{mnemonicbox}SNB - Skippable, Non-skippable, Bumper\end{mnemonicbox}
\end{solutionbox}

\questionmarks{4(b)}{4}{Explain the concept of LinkedIn marketing and discuss its significance in the digital marketing landscape.}

\begin{solutionbox}
\textbf{LinkedIn Marketing} focuses on professional networking and B2B relationship building.

\begin{center}
\captionof{table}{Significance of LinkedIn Marketing}
\begin{tabulary}{\linewidth}{|L|L|}
\hline
\textbf{Aspect} & \textbf{Significance} \\ \hline
\textbf{Professional Audience} & Decision-makers and industry professionals \\ \hline
\textbf{B2B Focus} & Ideal for business-to-business marketing \\ \hline
\textbf{Content Authority} & Establishes thought leadership \\ \hline
\textbf{Networking} & Direct access to key business contacts \\ \hline
\end{tabulary}
\end{center}

\begin{itemize}
    \item \textbf{Professional Audience}: Higher income, educated demographics
    \item \textbf{B2B Focus}: 80\% of B2B leads come from LinkedIn
    \item \textbf{Content Authority}: Share industry insights and expertise
    \item \textbf{Networking}: Build valuable business relationships
\end{itemize}

\begin{mnemonicbox}PBCN - Professional, B2B, Content, Networking\end{mnemonicbox}
\end{solutionbox}

\questionmarks{4(c)}{7}{Describe the key differences between organic and paid social media marketing strategies. Provide two advantages and two disadvantages for each strategy.}

\begin{solutionbox}
\begin{center}
\captionof{table}{Organic vs Paid Social Media}
\begin{tabulary}{\linewidth}{|L|L|L|L|}
\hline
\textbf{Strategy} & \textbf{Description} & \textbf{Advantages} & \textbf{Disadvantages} \\ \hline
\textbf{Organic} & Free content posting and engagement & • Cost-effective \newline • Builds authentic relationships & • Limited reach \newline • Time-intensive \\ \hline
\textbf{Paid} & Sponsored content and advertisements & • Immediate reach \newline • Precise targeting & • Requires budget \newline • Temporary results \\ \hline
\end{tabulary}
\end{center}

\begin{center}
\begin{tikzpicture}[node distance=1.5cm, auto]
    \node [gtu block] (SMM) {Social Media Marketing};
    
    \node [gtu block, below left=1.2cm and 1cm of SMM] (Organic) {Organic Strategy};
    \node [gtu block, below right=1.2cm and 1cm of SMM] (Paid) {Paid Strategy};
    
    \node [gtu state, below=0.8cm of Organic] (OAdv) {Cost-effective\\Authentic};
    \node [gtu state, below=0.8cm of Paid] (PAdv) {Immediate reach\\Targeting};
    
    \path [gtu arrow] (SMM) -- (Organic);
    \path [gtu arrow] (SMM) -- (Paid);
    \path [gtu arrow] (Organic) -- (OAdv);
    \path [gtu arrow] (Paid) -- (PAdv);
\end{tikzpicture}
\captionof{figure}{Organic vs Paid Strategies}
\end{center}

\textbf{Organic Advantages:}
\begin{itemize}
    \item \textbf{Cost-effective}: No advertising spend required
    \item \textbf{Builds authentic relationships}: Genuine community engagement
\end{itemize}

\textbf{Organic Disadvantages:}
\begin{itemize}
    \item \textbf{Limited reach}: Algorithm restrictions reduce visibility
    \item \textbf{Time-intensive}: Requires consistent content creation and engagement
\end{itemize}

\textbf{Paid Advantages:}
\begin{itemize}
    \item \textbf{Immediate reach}: Instant visibility to target audience
    \item \textbf{Precise targeting}: Advanced demographic and interest targeting
\end{itemize}

\textbf{Paid Disadvantages:}
\begin{itemize}
    \item \textbf{Requires budget}: Ongoing advertising costs
    \item \textbf{Temporary results}: Results stop when advertising stops
\end{itemize}

\begin{mnemonicbox}OPAL - Organic Patient Authentic Low-cost, Paid Quick Targeted Expensive\end{mnemonicbox}
\end{solutionbox}

\questionmarks{4(a OR)}{3}{What are the different types of Twitter ads? Explain any one type briefly.}

\begin{solutionbox}
\begin{center}
\captionof{table}{Twitter Ad Types}
\begin{tabulary}{\linewidth}{|L|L|}
\hline
\textbf{Ad Type} & \textbf{Purpose} \\ \hline
\textbf{Promoted Tweets} & Increase tweet visibility \\ \hline
\textbf{Promoted Accounts} & Gain more followers \\ \hline
\textbf{Promoted Trends} & Boost trending topics \\ \hline
\end{tabulary}
\end{center}

\textbf{Promoted Tweets}: Regular tweets that businesses pay to show to wider audiences beyond their followers, appearing in users' timelines and search results with "Promoted" label.

\begin{mnemonicbox}PAT - Promoted tweets, Accounts, Trends\end{mnemonicbox}
\end{solutionbox}

\questionmarks{4(b OR)}{4}{Samsung launched a new smart phone in market and want to run YouTube ads. As social media marketing expert which type of YouTube ad format would you will choose and why?}

\begin{solutionbox}
\textbf{Recommended Format: Skippable In-Stream Ads}

\begin{center}
\captionof{table}{Ad Selection Reasoning}
\begin{tabulary}{\linewidth}{|L|L|}
\hline
\textbf{Reason} & \textbf{Benefit} \\ \hline
\textbf{Cost-Effective} & Pay only when users watch 30+ seconds \\ \hline
\textbf{Product Demonstration} & Longer format allows feature showcase \\ \hline
\textbf{Audience Interest} & Skip option ensures engaged viewers \\ \hline
\textbf{Brand Awareness} & Reaches broad audience with smartphone interest \\ \hline
\end{tabulary}
\end{center}

\begin{itemize}
    \item \textbf{Product Demonstration}: Smartphones need visual demonstration of features
    \item \textbf{Audience Interest}: Skip option filters for genuinely interested viewers
    \item \textbf{Cost-Effective}: Only pay for engaged viewers who watch beyond 30 seconds
    \item \textbf{Brand Awareness}: Broad reach for new product launch
\end{itemize}

\begin{mnemonicbox}PCAB - Product demo, Cost-effective, Audience interest, Brand awareness\end{mnemonicbox}
\end{solutionbox}

\questionmarks{4(c OR)}{7}{Describe the main functions of a Facebook Page, Business Manager, and Facebook Ads. How can these assets help businesses in their marketing efforts?}

\begin{solutionbox}
\begin{center}
\captionof{table}{Facebook Marketing Assets}
\begin{tabulary}{\linewidth}{|L|L|L|}
\hline
\textbf{Asset} & \textbf{Main Functions} & \textbf{Marketing Benefits} \\ \hline
\textbf{Facebook Page} & • Brand presence\newline • Content sharing\newline • Customer engagement & • Builds brand awareness\newline • Direct customer communication \\ \hline
\textbf{Business Manager} & • Account management\newline • Team access control\newline • Asset organization & • Centralized control\newline • Secure collaboration \\ \hline
\textbf{Facebook Ads} & • Targeted advertising\newline • Campaign management\newline • Performance tracking & • Precise audience targeting\newline • Measurable ROI \\ \hline
\end{tabulary}
\end{center}

\begin{center}
\begin{tikzpicture}[node distance=1.5cm, auto]
    \node [gtu block] (Main) {Facebook Marketing Assets};
    
    \node [gtu block, below left=1.5cm and 2cm of Main] (Page) {Facebook Page};
    \node [gtu block, below=1.5cm of Main] (Manager) {Business Manager};
    \node [gtu block, below right=1.5cm and 2cm of Main] (Ads) {Facebook Ads};
    
    \node [gtu state, below=0.8cm of Page] (PBen) {Brand Presence};
    \node [gtu state, below=0.8cm of Manager] (MBen) {Account Management};
    \node [gtu state, below=0.8cm of Ads] (ABen) {Targeted Advertising};
    
    \foreach \n in {Page, Manager, Ads}
        \path [gtu arrow] (Main) -- (\n);
        
    \path [gtu arrow] (Page) -- (PBen);
    \path [gtu arrow] (Manager) -- (MBen);
    \path [gtu arrow] (Ads) -- (ABen);
\end{tikzpicture}
\captionof{figure}{Facebook Marketing Ecosystem}
\end{center}

\textbf{Marketing Benefits:}
\begin{itemize}
    \item \textbf{Facebook Page}: Creates professional brand presence and enables organic reach
    \item \textbf{Business Manager}: Provides security and organization for multiple accounts and team members
    \item \textbf{Facebook Ads}: Delivers targeted campaigns with detailed analytics and ROI tracking
\end{itemize}

\textbf{Integration Benefits:}
\begin{itemize}
    \item \textbf{Unified Strategy}: All three work together for comprehensive Facebook marketing
    \item \textbf{Data Sharing}: Pixel data from page enhances ad targeting
    \item \textbf{Brand Consistency}: Consistent messaging across organic and paid content
\end{itemize}

\begin{mnemonicbox}PMA - Page presence, Manager control, Ads targeting\end{mnemonicbox}
\end{solutionbox}

\questionmarks{5(a)}{3}{List the Types of Instagram Content and Ads.}

\begin{solutionbox}
\begin{center}
\captionof{table}{Instagram Content and Ads}
\begin{tabulary}{\linewidth}{|L|L|}
\hline
\textbf{Content Types} & \textbf{Ad Types} \\ \hline
\textbf{Posts} & Photo Ads \\ \hline
\textbf{Stories} & Video Ads \\ \hline
\textbf{Reels} & Carousel Ads \\ \hline
\textbf{IGTV} & Stories Ads \\ \hline
\textbf{Live} & Reels Ads \\ \hline
\end{tabulary}
\end{center}

\begin{itemize}
    \item \textbf{Content Types}: Various formats for organic engagement
    \item \textbf{Ad Types}: Sponsored versions with targeting capabilities
    \item \textbf{Integration}: Ads blend naturally with organic content
\end{itemize}

\begin{mnemonicbox}PSRIL - Posts, Stories, Reels, IGTV, Live\end{mnemonicbox}
\end{solutionbox}

\questionmarks{5(b)}{4}{What is e-mail marketing? What are different types of e-mail marketing?}

\begin{solutionbox}
\textbf{Email Marketing} is direct digital communication with customers through personalized email messages.

\begin{center}
\captionof{table}{Email Marketing Types}
\begin{tabulary}{\linewidth}{|L|L|L|}
\hline
\textbf{Type} & \textbf{Purpose} & \textbf{Example} \\ \hline
\textbf{Newsletter} & Regular updates and information & Monthly company news \\ \hline
\textbf{Promotional} & Sales and offers & Discount codes, new products \\ \hline
\textbf{Transactional} & Purchase confirmations & Order receipts, shipping updates \\ \hline
\textbf{Welcome Series} & New subscriber onboarding & Introduction to brand and products \\ \hline
\end{tabulary}
\end{center}

\begin{itemize}
    \item \textbf{Newsletter}: Builds relationships through valuable content
    \item \textbf{Promotional}: Drives sales and conversions
    \item \textbf{Transactional}: Provides essential customer service information
    \item \textbf{Welcome Series}: Nurtures new subscribers into customers
\end{itemize}

\begin{mnemonicbox}NPTW - Newsletter, Promotional, Transactional, Welcome\end{mnemonicbox}
\end{solutionbox}

\questionmarks{5(c)}{7}{Explain different types of ad extensions available in Google Ads with an example of each.}

\begin{solutionbox}
\begin{center}
\captionof{table}{Google Ad Extensions}
\begin{tabulary}{\linewidth}{|L|L|L|}
\hline
\textbf{Extension Type} & \textbf{Function} & \textbf{Example} \\ \hline
\textbf{Sitelink Extensions} & Additional page links & "About Us", "Contact", "Products" \\ \hline
\textbf{Call Extensions} & Phone number display & "+1-800-123-4567" \\ \hline
\textbf{Location Extensions} & Business address & "123 Main St, City, State" \\ \hline
\textbf{Callout Extensions} & Highlight features & "Free Shipping", "24/7 Support" \\ \hline
\textbf{Price Extensions} & Product/service pricing & "Basic Plan: \$19/month" \\ \hline
\textbf{App Extensions} & Mobile app downloads & "Download our iOS/Android app" \\ \hline
\end{tabulary}
\end{center}

\begin{center}
\begin{tikzpicture}[node distance=1.5cm, auto]
    \node [gtu block] (Extensions) {Google Ad Extensions};
    
    \node [gtu block, below left=1.5cm and 3cm of Extensions] (Sitelink) {Sitelink};
    \node [gtu block, below left=1.5cm and 0.5cm of Extensions] (Call) {Call};
    \node [gtu block, below=1.5cm of Extensions] (Location) {Location};
    \node [gtu block, below right=1.5cm and 0.5cm of Extensions] (Callout) {Callout};
    \node [gtu block, below right=1.5cm and 3cm of Extensions] (Price) {Price};
    \node [gtu block, below=2.5cm of Extensions] (App) {App};
    
    \foreach \n in {Sitelink, Call, Location, Callout, Price, App}
        \path [gtu arrow] (Extensions) -- (\n);
\end{tikzpicture}
\captionof{figure}{Types of Ad Extensions}
\end{center}

\textbf{Benefits:}
\begin{itemize}
    \item \textbf{Increased CTR}: Extensions make ads more prominent and informative
    \item \textbf{Better Quality Score}: Improved ad performance leads to lower costs
    \item \textbf{Enhanced User Experience}: Users get more relevant information
    \item \textbf{Competitive Advantage}: More screen real estate than competitors
\end{itemize}

\textbf{Implementation:}
\begin{itemize}
    \item \textbf{Automatic}: Google may show relevant extensions automatically
    \item \textbf{Manual}: Advertisers can create and customize specific extensions
    \item \textbf{Performance}: Extensions shown based on predicted impact
\end{itemize}

\begin{mnemonicbox}SCLCPA - Sitelink, Call, Location, Callout, Price, App\end{mnemonicbox}
\end{solutionbox}

\questionmarks{5(a OR)}{3}{Explain importance and benefits of social media marketing.}

\begin{solutionbox}
\begin{center}
\captionof{table}{Social Media Benefits}
\begin{tabulary}{\linewidth}{|L|L|}
\hline
\textbf{Benefit} & \textbf{Impact} \\ \hline
\textbf{Brand Awareness} & Increases visibility and recognition \\ \hline
\textbf{Customer Engagement} & Direct interaction and relationship building \\ \hline
\textbf{Cost-Effective} & Lower costs compared to traditional advertising \\ \hline
\end{tabulary}
\end{center}

\begin{itemize}
    \item \textbf{Brand Awareness}: Exponential reach through sharing and viral content
    \item \textbf{Customer Engagement}: Real-time feedback and community building
    \item \textbf{Cost-Effective}: High ROI with targeted advertising options
\end{itemize}

\begin{mnemonicbox}BEC - Brand awareness, Engagement, Cost-effective\end{mnemonicbox}
\end{solutionbox}

\questionmarks{5(b OR)}{4}{Give the difference between PPC and SEO.}

\begin{solutionbox}
\begin{center}
\captionof{table}{PPC vs SEO}
\begin{tabulary}{\linewidth}{|L|L|L|}
\hline
\textbf{Aspect} & \textbf{PPC (Pay-Per-Click)} & \textbf{SEO (Search Engine Optimization)} \\ \hline
\textbf{Cost} & Paid advertising & Organic/Free traffic \\ \hline
\textbf{Results} & Immediate visibility & Long-term sustainable results \\ \hline
\textbf{Control} & Full control over ads & Limited control over rankings \\ \hline
\textbf{Duration} & Results stop when payments stop & Long-lasting results \\ \hline
\end{tabulary}
\end{center}

\begin{itemize}
    \item \textbf{PPC}: Immediate results but requires ongoing investment
    \item \textbf{SEO}: Takes time to build but provides sustainable long-term value
    \item \textbf{Integration}: Best results come from combining both strategies
    \item \textbf{Budget}: PPC needs advertising budget; SEO needs time investment
\end{itemize}

\begin{mnemonicbox}ICRD - Immediate vs Continuous, Results vs Duration\end{mnemonicbox}
\end{solutionbox}

\questionmarks{5(c OR)}{7}{Explain the concept of Quality Score in Google AdWords and its impact on ad rankings.}

\begin{solutionbox}
\textbf{Quality Score} is Google's rating (1-10) of ad quality, keywords, and landing pages.

\begin{center}
\captionof{table}{Quality Score Components}
\begin{tabulary}{\linewidth}{|L|L|L|}
\hline
\textbf{Component} & \textbf{Weight} & \textbf{Impact} \\ \hline
\textbf{Expected CTR} & High & Predicted likelihood users will click \\ \hline
\textbf{Ad Relevance} & High & How closely ad matches search intent \\ \hline
\textbf{Landing Page Experience} & Medium & Page quality and user experience \\ \hline
\end{tabulary}
\end{center}

\begin{center}
\begin{tikzpicture}[node distance=1.5cm, auto]
    \node [gtu block] (Quality) {Quality Score};
    
    \node [gtu block, below left=1.2cm and 1cm of Quality] (CTR) {Expected CTR};
    \node [gtu block, below=1.2cm of Quality] (Relevance) {Ad Relevance};
    \node [gtu block, below right=1.2cm and 1cm of Quality] (LP) {Landing Page\\Experience};
    
    \node [gtu state, below=0.8cm of CTR] (Rank) {Ad Ranking};
    \node [gtu state, below=0.8cm of Relevance] (CPC) {Cost Per Click};
    \node [gtu state, below=0.8cm of LP] (Pos) {Ad Position};
    
    \foreach \n in {CTR, Relevance, LP}
        \path [gtu arrow] (Quality) -- (\n);
        
    \path [gtu arrow] (CTR) -- (Rank);
    \path [gtu arrow] (Relevance) -- (CPC);
    \path [gtu arrow] (LP) -- (Pos);
\end{tikzpicture}
\captionof{figure}{Quality Score Impact}
\end{center}

\textbf{Impact on Ad Rankings:}
\begin{center}
\captionof{table}{Ranking Impact}
\begin{tabulary}{\linewidth}{|L|L|L|}
\hline
\textbf{Quality Score} & \textbf{Ad Rank Impact} & \textbf{Cost Impact} \\ \hline
\textbf{High (8-10)} & Higher positions & Lower CPC \\ \hline
\textbf{Medium (5-7)} & Average positions & Average CPC \\ \hline
\textbf{Low (1-4)} & Lower positions & Higher CPC \\ \hline
\end{tabulary}
\end{center}

\textbf{Benefits of High Quality Score:}
\begin{itemize}
    \item \textbf{Lower Costs}: Pay less per click than competitors
    \item \textbf{Better Positions}: Appear higher in search results
    \item \textbf{Increased Visibility}: More ad extension eligibility
    \item \textbf{Improved ROI}: Better performance at lower costs
\end{itemize}

\textbf{Optimization Strategies:}
\begin{itemize}
    \item \textbf{Keyword Relevance}: Match keywords closely to ad copy
    \item \textbf{Ad Copy Quality}: Write compelling, relevant ad text
    \item \textbf{Landing Page}: Ensure fast, relevant, user-friendly pages
    \item \textbf{Account Structure}: Organize campaigns and ad groups logically
\end{itemize}

\begin{mnemonicbox}EAL-RCP - Expected CTR, Ad relevance, Landing page affect Rank, Cost, Position\end{mnemonicbox}
\end{solutionbox}

\end{document}
