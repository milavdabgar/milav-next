\documentclass{article}

% content/resources/templates/preamble.tex
\usepackage[margin=0.6in]{geometry}
\author{Milav Dabgar}
\usepackage{amsmath,amssymb,amsthm}
\usepackage{booktabs}
\usepackage{multirow}
\usepackage{xcolor}
\usepackage{tcolorbox}
\tcbuselibrary{breakable,skins}
\usepackage[colorlinks=true,linkcolor=blue]{hyperref}
\usepackage{titlesec}
\usepackage{enumitem}
\usepackage{tikz}
\usepackage{pgfplots}
\usepackage{circuitikz}
\usepackage[version=4]{mhchem}
\usepackage{longtable}
\usepackage{array}
\usepackage{float}
\usepackage{caption}
\usepackage{listings}

\lstset{
  basicstyle=\small\ttfamily,
  breaklines=true,
  breakatwhitespace=false,
  postbreak=\mbox{\textcolor{red}{$\hookrightarrow$}\space},
  float=false,
  numbers=left,
  numberstyle=\tiny\color{gray},
  numbersep=10pt,
  xleftmargin=2em,
  keywordstyle=\color{blue},
  commentstyle=\color{green!60!black},
  stringstyle=\color{purple},
  backgroundcolor=\color{gray!5},
  showstringspaces=false,
  tabsize=2,
  captionpos=b,
  keepspaces=true,
  columns=flexible
}

\pgfplotsset{compat=1.18}
\usetikzlibrary{shapes,arrows,positioning,calc,patterns,decorations.pathmorphing,decorations.markings,arrows.meta}

% Color scheme
\definecolor{headcolor}{RGB}{0,102,204}
\definecolor{keycolor}{RGB}{220,20,60}
\definecolor{solutioncolor}{RGB}{34,139,34}
\definecolor{mnemoniccolor}{RGB}{148,0,211}
\definecolor{codecolor}{RGB}{0,0,100}

% Spacing
\setlength{\parskip}{3pt}
\setlist[itemize]{nosep}
\setlist[enumerate]{nosep}

% Title formatting
\titleformat{\section}{\Large\bfseries\color{headcolor}}{\thesection}{1em}{}
\titleformat{\subsection}{\large\bfseries\color{headcolor}}{\thesubsection}{1em}{}

% Pandoc tightlist compatibility
\providecommand{\tightlist}{%
  \setlength{\itemsep}{0pt}\setlength{\parskip}{0pt}}

% Pandoc longtable compatibility
\newcounter{none}
\def\thenone{}


% content/resources/templates/gujarati-boxes.tex
\usepackage{fontspec}
\usepackage{polyglossia}

% Set Gujarati as main language (document is primarily in Gujarati)
% Note: gloss-gujarati.ldf doesn't exist in polyglossia, but it will use hyphenation patterns
\setdefaultlanguage{gujarati}
\setotherlanguage{english}

% Configure Gujarati font properly
% Use Language=Default to prevent polyglossia from trying to add language-specific features
% that don't exist for Gujarati, which causes "empty feature" warnings
\newfontfamily\gujaratifont[Script=Gujarati,AutoFakeBold=2.5,AutoFakeSlant=0.3]{Noto Sans Gujarati}
\setmainfont[Script=Gujarati,AutoFakeBold=2.5,AutoFakeSlant=0.3]{Noto Sans Gujarati}
% Use Noto Sans Gujarati for monospace to support Gujarati in text
\setmonofont[Scale=0.9]{Noto Sans Gujarati}

% Configure English to use the same font
\newfontfamily\englishfont[Script=Gujarati,AutoFakeBold=2.5,AutoFakeSlant=0.3]{Noto Sans Gujarati}

% Translations for polyglossia
\gappto\captionsgujarati{
  \renewcommand{\tablename}{કોષ્ટક}
  \renewcommand{\figurename}{આકૃતિ}
}

% Helper for TikZ nodes to ensure Gujarati font
\newcommand{\gu}[1]{{\gujaratifont #1}}

% Custom environments
\newtcolorbox{solutionbox}{
    breakable,
    enhanced,
    colback=solutioncolor!5!white,
    colframe=solutioncolor!75!black,
    fonttitle=\bfseries,
    title=જવાબ
}

\newtcolorbox{solutionboxnobreak}{
 colback=solutioncolor!5!white,
 colframe=solutioncolor!75!black,
 fonttitle=\bfseries,
 title=જવાબ
}

\newtcolorbox{keyformula}{
 breakable,
 enhanced,
 colback=keycolor!5!white,
 colframe=keycolor!75!black,
 fonttitle=\bfseries,
 title=રાસાયણિક સમીકરણ/સૂત્ર
}

\newtcolorbox{mnemonicbox}{
 breakable,
 enhanced,
 colback=mnemoniccolor!5!white,
 colframe=mnemoniccolor!75!black,
 fonttitle=\bfseries,
 title=મેમરી ટ્રીક
}


% Custom commands for GTU solutions
% This file defines semantic commands for consistent formatting

% Question command with automatic formatting
\newcommand{\question}[2]{%
  \section*{Question #1}%
  \textbf{#2}%
}

% OR question variant
\newcommand{\questionor}[2]{%
  \section*{Question #1 OR}%
  \textbf{#2}%
}

% Proper table environment with caption
\newenvironment{answertable}[1]{%
  \begin{table}[htbp]
  \centering
  \caption{#1}
}{%
  \end{table}
}

% Proper figure environment for diagrams
\newenvironment{answerdiagram}[1]{%
  \begin{figure}[htbp]
  \centering
  \caption{#1}
}{%
  \end{figure}
}

% Semantic markup for key terms
\newcommand{\keyword}[1]{\textbf{#1}}
\newcommand{\code}[1]{\texttt{#1}}
\newcommand{\classname}[1]{\texttt{#1}}
\newcommand{\methodname}[1]{\texttt{#1}}

% Proper quotation marks
\newcommand{\mnemonic}[1]{``#1''}


\title{ડિજિટલ માર્કેટિંગની આવશ્યકતાઓ (4341601) - ઉનાળો 2025 ઉકેલ}
\date{May 13, 2025}

\begin{document}
\maketitle

\questionmarks{પ્રશ્ન 1(અ)}{3}{SEO ranking સમજાવો.}
\begin{solutionbox}
SEO ranking એ ચોક્કસ કીવર્ડ અથવા પ્રશ્નો માટે સર્ચ એન્જિન પરિણામ પૃષ્ઠો (SERPs) માં વેબસાઇટ અથવા વેબપેજની સ્થિતિને દર્શાવે છે.

\textbf{મુખ્ય ઘટકો:}

\begin{center}
\captionof{table}{SEO રેંકિંગ ઘટકો}
\begin{tabulary}{\linewidth}{L L}
\toprule
\textbf{પરિબળ} & \textbf{વર્ણન} \\
\midrule
\textbf{પેજ પોઝિશન} & પ્રથમ પેજ પર સંખ્યાત્મક સ્થિતિ (1-10) \\
\textbf{સર્ચ વિઝિબિલિટી} & સર્ચ પરિણામોમાં કેટલી વાર સાઇટ દેખાય છે \\
\textbf{કીવર્ડ રિલેવન્સ} & કન્ટેન્ટ અને સર્ચ શબ્દો વચ્ચેનો મેળ \\
\bottomrule
\end{tabulary}
\end{center}

\begin{itemize}
    \item \textbf{ઉંચી રેંકિંગ}: બહેતર દૃશ્યતા અને વધુ ઓર્ગેનિક ટ્રાફિક
    \item \textbf{એલ્ગોરિધમ-આધારિત}: Google 200+ રેંકિંગ પરિબળોનો ઉપયોગ કરે છે
    \item \textbf{ગતિશીલ પ્રકૃતિ}: એલ્ગોરિધમ અપડેટ્સના આધારે રેંકિંગ બદલાય છે
\end{itemize}

\begin{mnemonicbox}SERP સફળતા સ્માર્ટ SEO સાથે શરૂ થાય છે\end{mnemonicbox}
\end{solutionbox}

\questionmarks{પ્રશ્ન 1(બ)}{4}{ડિજિટલ માર્કેટિંગમાં P.O.E.M. ફ્રેમવર્કનું વર્ણન કરો}
\begin{solutionbox}
P.O.E.M. ફ્રેમવર્ક એ ડિજિટલ માર્કેટિંગ ચેનલો અને કન્ટેન્ટ ડિસ્ટ્રિબ્યુશનને વર્ગીકૃત કરવાનો વ્યૂહાત્મક અભિગમ છે.

\textbf{ફ્રેમવર્ક ઘટકો:}

\begin{center}
\captionof{table}{P.O.E.M. ફ્રેમવર્ક ઘટકો}
\begin{tabulary}{\linewidth}{L L L}
\toprule
\textbf{ચેનલ પ્રકાર} & \textbf{વ્યાખ્યા} & \textbf{ઉદાહરણો} \\
\midrule
\textbf{Paid} & ખરીદેલી જાહેરાત જગ્યા & Google Ads, Facebook Ads \\
\textbf{Owned} & બ્રાન્ડ-નિયંત્રિત પ્લેટફોર્મ & વેબસાઇટ, Email lists \\
\textbf{Earned} & ત્રીજા પક્ષની સમર્થન & રિવ્યુઝ, સોશિયલ શેર \\
\textbf{Managed} & નિયંત્રિત સોશિયલ હાજરી & Facebook Pages, Twitter \\
\bottomrule
\end{tabulary}
\end{center}

\begin{itemize}
    \item \textbf{સંકલિત અભિગમ}: મહત્તમ પહોંચ માટે બધી ચેનલોને સંયોજિત કરે છે
    \item \textbf{કોસ્ટ ઓપ્ટિમાઇઝેશન}: પેઇડ અને ઓર્ગેનિક પ્રયત્નોને સંતુલિત કરે છે
    \item \textbf{બ્રાન્ડ કંટ્રોલ}: બધી ચેનલોમાં સુસંગત મેસેજિંગ જાળવે છે
\end{itemize}

\begin{mnemonicbox}લોકો ઘણીવાર પૈસા કમાય છે\end{mnemonicbox}
\end{solutionbox}

\questionmarks{પ્રશ્ન 1(ક)}{7}{ડિજિટલ માર્કેટિંગમાં નૈતિકતા અને ડેટા ગોપનીયતાના મહત્વ પર ચર્ચા કરો. નૈતિક પ્રથાઓ અને ડેટા ગોપનીયતા પ્રત્યેની પ્રતિબદ્ધતા "ગૌરવપૂર્ણ ડિજિટલ માર્કેટિંગ" માં કેવી રીતે યોગદાન આપે છે}
\begin{solutionbox}
આજના ડેટા-સંચાલિત લેન્ડસ્કેપમાં નૈતિકતા અને ડેટા ગોપનીયતા જવાબદાર ડિજિટલ માર્કેટિંગ પ્રથાઓનો પાયો છે.

\textbf{નૈતિક મહત્વ:}

\begin{center}
\captionof{table}{ડિજિટલ માર્કેટિંગમાં નૈતિક મહત્વ}
\begin{tabulary}{\linewidth}{L L}
\toprule
\textbf{પાસું} & \textbf{મહત્વ} \\
\midrule
\textbf{ગ્રાહક વિશ્વાસ} & લાંબા ગાળાના સંબંધો બનાવે છે \\
\textbf{કાયદાકીય અનુપાલન} & GDPR/CCPA દંડથી બચાવે છે \\
\textbf{બ્રાન્ડ પ્રતિષ્ઠા} & સકારાત્મક છબી જાળવે છે \\
\textbf{બજાર ટકાઉપણું} & ઉદ્યોગની વિશ્વસનીયતા સુનિશ્ચિત કરે છે \\
\bottomrule
\end{tabulary}
\end{center}

\textbf{ડેટા ગોપનીયતા પ્રથાઓ:}
\begin{itemize}
    \item \textbf{પારદર્શક સંગ્રહ}: સ્પષ્ટ સંમતિ તંત્ર
    \item \textbf{ન્યૂનતમ ડેટા એકત્રીકરણ}: ફક્ત જરૂરી માહિતી
    \item \textbf{સુરક્ષિત સંગ્રહ}: એન્ક્રિપ્ટેડ ડેટાબેસ અને એક્સેસ કંટ્રોલ
    \item \textbf{વપરાશકર્તા અધિકારો}: સરળ opt-out અને ડેટા ડિલીશન વિકલ્પો
\end{itemize}

\textbf{ગૌરવપૂર્ણ ડિજિટલ માર્કેટિંગના ફાયદા:}
\begin{itemize}
    \item \textbf{વધેલી વિશ્વસનીયતા}: ગ્રાહકો નૈતિક બ્રાન્ડ્સ પર વિશ્વાસ કરે છે
    \item \textbf{સ્પર્ધાત્મક ફાયદો}: જવાબદાર પ્રથાઓ દ્વારા અલગતા
    \item \textbf{નિયમનકારી અનુપાલન}: ગોપનીયતા કાયદાઓ માટે સક્રિય અભિગમ
    \item \textbf{ટકાઉ વૃદ્ધિ}: લાંબા ગાળાના ગ્રાહક સંબંધો
\end{itemize}

\begin{mnemonicbox}પારદર્શિતા દ્વારા વિશ્વાસ જીતો\end{mnemonicbox}
\end{solutionbox}

\questionmarks{પ્રશ્ન 1(ક) વા}{7}{પરંપરાગત માર્કેટિંગ અને ડિજિટલ માર્કેટિંગ વચ્ચે તેમની પહોંચ, ટાર્ગેટિંગ, કોસ્ટ-અસરકારકતા અને સફળતા માપનના દ્રષ્ટિકોણથી તફાવત કરો.}
\begin{solutionbox}
\textbf{તુલનાત્મક વિશ્લેષણ:}

\begin{center}
\captionof{table}{પરંપરાગત વિ ડિજિટલ માર્કેટિંગ}
\begin{tabulary}{\linewidth}{L L L}
\toprule
\textbf{પરિબળ} & \textbf{પરંપરાગત માર્કેટિંગ} & \textbf{ડિજિટલ માર્કેટિંગ} \\
\midrule
\textbf{પહોંચ} & સ્થાનિક/પ્રાદેશિક મર્યાદાઓ & તરત જ વૈશ્વિક પ્રેક્ષકો \\
\textbf{ટાર્ગેટિંગ} & વ્યાપક ડેમોગ્રાફિક જૂથો & ચોક્કસ વર્તણૂકીય ટાર્ગેટિંગ \\
\textbf{ખર્ચ} & ઉંચા અગાઉથી રોકાણ & લવચીક બજેટ વિકલ્પો \\
\textbf{માપણ} & ROI ટ્રેક કરવું મુશ્કેલ & રીઅલ-ટાઇમ એનાલિટિક્સ ઉપલબ્ધ \\
\bottomrule
\end{tabulary}
\end{center}

\textbf{વિગતવાર તફાવતો:}

\textbf{પહોંચ ક્ષમતાઓ:}
\begin{itemize}
    \item \textbf{પરંપરાગત}: ભૌગોલિક અવરોધો, મર્યાદિત પ્રેક્ષકો
    \item \textbf{ડિજિટલ}: વિશ્વવ્યાપી પહોંચ, 24/7 ઉપલબ્ધતા
\end{itemize}

\textbf{ટાર્ગેટિંગ ચોકસાઈ:}
\begin{itemize}
    \item \textbf{પરંપરાગત}: મોટા બજાર અભિગમ, મર્યાદિત વિભાજન
    \item \textbf{ડિજિટલ}: વ્યક્તિગત-સ્તરે ટાર્ગેટિંગ, વર્તણૂકીય ડેટા વપરાશ
\end{itemize}

\textbf{કોસ્ટ સ્ટ્રક્ચર:}
\begin{itemize}
    \item \textbf{પરંપરાગત}: નિશ્ચિત ખર્ચ, ન્યૂનતમ ખર્ચ જરૂરિયાતો
    \item \textbf{ડિજિટલ}: પે-પર-ક્લિક, સ્કેલેબલ બજેટ, સૂક્ષ્મ રોકાણ
\end{itemize}

\textbf{સફળતા માપણ:}
\begin{itemize}
    \item \textbf{પરંપરાગત}: સર્વેક્ષણો, અનુમાનિત પહોંચ ગણતરીઓ
    \item \textbf{ડિજિટલ}: ક્લિક-થ્રુ રેટ્સ, કન્વર્ઝન ટ્રેકિંગ, એટ્રિબ્યુશન મોડેલ્સ
\end{itemize}

\begin{mnemonicbox}પહોંચ, ટાર્ગેટ, કોસ્ટ, માપો - ડિજિટલ બેહતર\end{mnemonicbox}
\end{solutionbox}

\questionmarks{પ્રશ્ન 2(અ)}{3}{White Hat SEO અને Black Hat SEO વચ્ચે સરખામણી કરો}
\begin{solutionbox}
\textbf{SEO પ્રથાઓની સરખામણી:}

\begin{center}
\captionof{table}{White Hat વિ Black Hat SEO}
\begin{tabulary}{\linewidth}{L L L}
\toprule
\textbf{પાસું} & \textbf{White Hat SEO} & \textbf{Black Hat SEO} \\
\midrule
\textbf{પદ્ધતિઓ} & નૈતિક, ગાઇડલાઇન-અનુકૂળ & ચાલાકીપૂર્ણ, નિયમ-ભંગ \\
\textbf{સમયમર્યાદા} & લાંબા ગાળાના ટકાઉ પરિણામો & ઝડપી પરંતુ અસ્થાયી લાભ \\
\textbf{જોખમ} & સર્ચ એન્જિન મંજૂર & દંડ અને પ્રતિબંધ જોખમો \\
\bottomrule
\end{tabulary}
\end{center}

\begin{itemize}
    \item \textbf{White Hat}: ગુણવત્તાયુક્ત કન્ટેન્ટ, કુદરતી લિંક બિલ્ડિંગ, વપરાશકર્તા-કેન્દ્રિત ઓપ્ટિમાઇઝેશન
    \item \textbf{Black Hat}: કીવર્ડ સ્ટફિંગ, છુપાયેલું ટેક્સ્ટ, લિંક ફાર્મિંગ
    \item \textbf{પરિણામો}: White Hat ઓથોરિટી બનાવે છે, Black Hat દંડનું જોખમ છે
\end{itemize}

\begin{mnemonicbox}વાઇટ છે રાઇટ, બ્લેક એટેક\end{mnemonicbox}
\end{solutionbox}

\questionmarks{પ્રશ્ન 2(બ)}{4}{Outdated content અને slow loading times ધરાવતી વેબસાઇટ assume કરો અને તેની search engine rankings સુધારવા માટેની SEO tactics apply કરો.}
\begin{solutionbox}
\textbf{SEO સુધારણા વ્યૂહરચના:}

\begin{center}
\begin{tikzpicture}[node distance=1.5cm]
    \node (issues) [gtu block, minimum width=3cm] {જૂની વેબસાઇટ સમસ્યાઓ};
    \node (content) [gtu block, below left=of issues, xshift=1cm] {કન્ટેન્ટ સમસ્યાઓ};
    \node (perf) [gtu block, below right=of issues, xshift=-1cm] {પરફોર્મન્સ સમસ્યાઓ};
    \node (fresh) [gtu block, below=of content] {તાજું કન્ટેન્ટ};
    \node (speed) [gtu block, below=of perf] {સ્પીડ ઓપ્ટિમાઇઝેશન};
    \node (rank) [gtu block, below right=of fresh, xshift=1cm] {સુધારેલી રેંકિંગ};

    \draw [gtu arrow] (issues) -| (content);
    \draw [gtu arrow] (issues) -| (perf);
    \draw [gtu arrow] (content) -- (fresh);
    \draw [gtu arrow] (perf) -- (speed);
    \draw [gtu arrow] (fresh) |- (rank);
    \draw [gtu arrow] (speed) |- (rank);
\end{tikzpicture}
\captionof{figure}{જૂની વેબસાઇટ માટે SEO યુક્તિઓ}
\end{center}

\textbf{વ્યૂહાત્મક ઉકેલો:}

\begin{center}
\captionof{table}{વેબસાઇટ સુધારણા માટે વ્યૂહાત્મક ઉકેલો}
\begin{tabulary}{\linewidth}{L L L}
\toprule
\textbf{સમસ્યા} & \textbf{SEO યુક્તિ} & \textbf{અમલીકરણ} \\
\midrule
\textbf{જૂનું કન્ટેન્ટ} & કન્ટેન્ટ રિફ્રેશ & વર્તમાન માહિતી સાથે અપડેટ \\
\textbf{ધીમું લોડિંગ} & પરફોર્મન્સ ઓપ્ટિમાઇઝેશન & ઇમેજ કોમ્પ્રેસ, કોડ મિનિમાઇઝ \\
\textbf{નબળું સ્ટ્રક્ચર} & ટેકનિકલ SEO & સાઇટ આર્કિટેક્ચર સુધારો \\
\bottomrule
\end{tabulary}
\end{center}

\begin{itemize}
    \item \textbf{કન્ટેન્ટ વ્યૂહરચના}: નિયમિત અપડેટ્સ, ટ્રેન્ડિંગ વિષયો, વપરાશકર્તા-સંબંધિત માહિતી
    \item \textbf{ટેકનિકલ ફિક્સ}: CDN અમલીકરણ, કેશિંગ, મોબાઇલ ઓપ્ટિમાઇઝેશન
    \item \textbf{મોનિટરિંગ}: પેજ સ્પીડ, વપરાશકર્તા એંગેજમેન્ટ મેટ્રિક્સ ટ્રેક કરો
\end{itemize}

\begin{mnemonicbox}કન્ટેન્ટ કરન્સી ક્લિક્સ બનાવે છે\end{mnemonicbox}
\end{solutionbox}

\questionmarks{પ્રશ્ન 2(ક)}{7}{on-page optimization, content quality, અને વેબસાઇટની ઝડપ કેવી રીતે સારી search engine rankings માટે યોગદાન આપે છે તે ચર્ચા કરો. આ વિસ્તારોમાં વેબસાઇટની visibility વધારવા માટેની કેટલીક વિશિષ્ટ તકનીકોના ઉદાહરણ આપો}
\begin{solutionbox}
\textbf{SEO રેંકિંગ પરિબળો:}

\begin{center}
\begin{tikzpicture}[node distance=1.5cm]
    \node (rank) [gtu block] {સર્ચ એન્જિન રેંકિંગ};
    
    \node (onpage) [gtu block, below left=of rank, xshift=-1cm] {On-Page ઓપ્ટિમાઇઝેશન};
    \node (content) [gtu block, below=of rank] {કન્ટેન્ટ ગુણવત્તા};
    \node (speed) [gtu block, below right=of rank, xshift=1cm] {વેબસાઇટ સ્પીડ};
    
    \draw [gtu arrow] (rank) -- (onpage);
    \draw [gtu arrow] (rank) -- (content);
    \draw [gtu arrow] (rank) -- (speed);
    
    \node (onpage_items) [gtu block, below=of onpage, align=left, font=\small] {ટાઇટલ ટેગ્સ\\મેટા ડિસ્ક્રિપ્શન\\હેડર સ્ટ્રક્ચર};
    \node (content_items) [gtu block, below=of content, align=left, font=\small] {મૂળ કન્ટેન્ટ\\કીવર્ડ રિલેવન્સ\\વપરાશકર્તા ઇન્ટેન્ટ મેચ};
    \node (speed_items) [gtu block, below=of speed, align=left, font=\small] {પેજ લોડ ટાઇમ\\મોબાઇલ પરફોર્મન્સ\\કોર વેબ વાઇટલ્સ};

    \draw [gtu arrow] (onpage) -- (onpage_items);
    \draw [gtu arrow] (content) -- (content_items);
    \draw [gtu arrow] (speed) -- (speed_items);
\end{tikzpicture}
\captionof{figure}{મુખ્ય SEO રેંકિંગ પરિબળો}
\end{center}

\textbf{On-Page ઓપ્ટિમાઇઝેશન ટેકનિક્સ:}

\begin{center}
\captionof{table}{On-Page ઓપ્ટિમાઇઝેશન બેસ્ટ પ્રેક્ટિસ}
\begin{tabulary}{\linewidth}{L L L}
\toprule
\textbf{એલિમેન્ટ} & \textbf{બેસ્ટ પ્રેક્ટિસ} & \textbf{ઉદાહરણ} \\
\midrule
\textbf{ટાઇટલ ટેગ્સ} & પ્રાથમિક કીવર્ડ શામેલ કરો & "શ્રેષ્ઠ ડિજિટલ માર્કેટિંગ ટૂલ્સ 2025" \\
\textbf{મેટા ડિસ્ક્રિપ્શન} & આકર્ષક 155-160 અક્ષરો & "ટોચના ડિજિટલ માર્કેટિંગ ટૂલ્સ શોધો..." \\
\textbf{હેડર ટેગ્સ} & વંશવેલો સ્ટ્રક્ચર & H1$\to$H2$\to$H3 તાર્કિક પ્રવાહ \\
\textbf{આંતરિક લિંકિંગ} & સંબંધિત પેજ કનેક્શન્સ & સંબંધિત બ્લોગ પોસ્ટ્સ લિંક કરો \\
\bottomrule
\end{tabulary}
\end{center}

\textbf{કન્ટેન્ટ ગુણવત્તા પરિબળો:}
\begin{itemize}
    \item \textbf{મૌલિકતા}: અનોખી, મૂલ્યવાન માહિતી
    \item \textbf{ઊંડાઈ}: વિષયનું વ્યાપક કવરેજ
    \item \textbf{તાજગી}: નિયમિત અપડેટ્સ અને વર્તમાન ડેટા
    \item \textbf{વપરાશકર્તા એંગેજમેન્ટ}: પેજ પર સમય, ઓછી બાઉન્સ રેટ
\end{itemize}

\textbf{વેબસાઇટ સ્પીડ ઓપ્ટિમાઇઝેશન:}
\begin{itemize}
    \item \textbf{ઇમેજ કોમ્પ્રેશન}: WebP ફોર્મેટ, લેઝી લોડિંગ
    \item \textbf{કોડ મિનિફિકેશન}: CSS, JavaScript ઓપ્ટિમાઇઝેશન
    \item \textbf{કેશિંગ વ્યૂહરચના}: બ્રાઉઝર અને સર્વર-સાઇડ કેશિંગ
    \item \textbf{CDN અમલીકરણ}: વૈશ્વિક કન્ટેન્ટ ડિલિવરી
\end{itemize}

\begin{mnemonicbox}ઓપ્ટિમાઇઝ, ગુણવત્તા, સ્પીડ = સફળતા\end{mnemonicbox}
\end{solutionbox}

\questionmarks{પ્રશ્ન 2(અ) વા}{3}{Search engine ની crawling થી ranking process માં આવેલા મુખ્ય steps ની ચર્ચા કરો.}
\begin{solutionbox}
\textbf{સર્ચ એન્જિન પ્રક્રિયા:}

\begin{center}
\captionof{table}{સર્ચ એન્જિન પ્રક્રિયા સ્ટેપ્સ}
\begin{tabulary}{\linewidth}{L L L}
\toprule
\textbf{સ્ટેપ} & \textbf{પ્રક્રિયા} & \textbf{વર્ણન} \\
\midrule
\textbf{1. ક્રોલિંગ} & શોધ & બોટ્સ નવા/અપડેટેડ પેજ શોધે છે \\
\textbf{2. ઇન્ડેક્સિંગ} & સંગ્રહ & કન્ટેન્ટનું વિશ્લેષણ અને સંગ્રહ \\
\textbf{3. રેંકિંગ} & મૂલ્યાંકન & એલ્ગોરિધમ રિલેવન્સ નક્કી કરે છે \\
\bottomrule
\end{tabulary}
\end{center}

\begin{itemize}
    \item \textbf{ક્રોલિંગ}: સ્પાઇડર બોટ્સ લિંક્સ અનુસરે છે, કન્ટેન્ટ શોધે છે
    \item \textbf{ઇન્ડેક્સિંગ}: કન્ટેન્ટ પાર્સ, કીવર્ડ્સ ઓળખાણ, ડેટાબેસ સંગ્રહ
    \item \textbf{રેંકિંગ}: એલ્ગોરિધમ મૂલ્યાંકન, SERP સ્થિતિ નિર્ધારણ
\end{itemize}

\begin{mnemonicbox}ક્રોલ, ઇન્ડેક્સ, રેંક - સર્ચ સફળતા\end{mnemonicbox}
\end{solutionbox}

\questionmarks{પ્રશ્ન 2(બ) વા}{4}{ઓછી search engine visibility ધરાવતી વેબસાઇટ પર on-page optimization ના concepts apply કરો. તેની rankings સુધારવા માટે ત્રણ વિશિષ્ટ on-page SEO tactics સૂચવો}
\begin{solutionbox}
\textbf{On-Page SEO સુધારણા યોજના:}

\begin{center}
\begin{tikzpicture}[node distance=1.5cm]
    \node (low) [gtu block] {નીચી વિઝિબિલિટી વેબસાઇટ};
    
    \node (title) [gtu block, below left=of low] {ટાઇટલ ઓપ્ટિમાઇઝેશન};
    \node (meta) [gtu block, below=of low] {મેટા ઓપ્ટિમાઇઝેશન};
    \node (content) [gtu block, below right=of low] {કન્ટેન્ટ ઓપ્ટિમાઇઝેશન};
    
    \node (better) [gtu block, below=of meta] {બેહતર રેંકિંગ};

    \draw [gtu arrow] (low) -| (title);
    \draw [gtu arrow] (low) -- (meta);
    \draw [gtu arrow] (low) -| (content);
    
    \draw [gtu arrow] (title) |- (better);
    \draw [gtu arrow] (meta) -- (better);
    \draw [gtu arrow] (content) |- (better);
\end{tikzpicture}
\captionof{figure}{On-Page ઓપ્ટિમાઇઝેશન યોજના}
\end{center}

\textbf{ત્રણ મુખ્ય યુક્તિઓ:}

\begin{center}
\captionof{table}{On-Page SEO યુક્તિઓ}
\begin{tabulary}{\linewidth}{L L L}
\toprule
\textbf{યુક્તિ} & \textbf{અમલીકરણ} & \textbf{અપેક્ષિત અસર} \\
\midrule
\textbf{ટાઇટલ ટેગ ઓપ્ટિમાઇઝેશન} & પ્રાથમિક કીવર્ડ્સ, બ્રાન્ડ નામ શામેલ કરો & બેહતર ક્લિક-થ્રુ રેટ્સ \\
\textbf{કન્ટેન્ટ સ્ટ્રક્ચર} & H1-H6 હેડર્સ, બુલેટ પોઇન્ટ્સ ઉમેરો & બેહતર વપરાશકર્તા અનુભવ \\
\textbf{આંતરિક લિંકિંગ} & સંબંધિત પેજો લિંક કરો, એન્કર ટેક્સ્ટ & વધેલી પેજ ઓથોરિટી \\
\bottomrule
\end{tabulary}
\end{center}

\begin{itemize}
    \item \textbf{કીવર્ડ પ્લેસમેન્ટ}: ટાઇટલ, હેડર્સ, પ્રથમ પેરાગ્રાફમાં વ્યૂહાત્મક સ્થાન
    \item \textbf{મેટા ડિસ્ક્રિપ્શન}: આકર્ષક 155-અક્ષર સારાંશ
    \item \textbf{ઇમેજ ઓપ્ટિમાઇઝેશન}: Alt ટેગ્સ, વર્ણનાત્મક ફાઇલનામ
\end{itemize}

\begin{mnemonicbox}ટાઇટલ, સ્ટ્રક્ચર, લિંક - સફળતાનું ચિંતન\end{mnemonicbox}
\end{solutionbox}

\questionmarks{પ્રશ્ન 2(ક) વા}{7}{વેબસાઇટની ઓનલાઇન હાજરી વધારવામાં SEO ની ભૂમિકા ચર્ચા કરો. ઉદાહરણ સાથે સમજાવો.}
\begin{solutionbox}
SEO વ્યવસાયો અને સંસ્થાઓ માટે મજબૂત ડિજિટલ ફૂટપ્રિન્ટ સ્થાપિત કરવા અને જાળવવામાં મહત્વપૂર્ણ ભૂમિકા ભજવે છે.

\textbf{ઓનલાઇન હાજરીમાં SEO ની ભૂમિકા:}

\begin{center}
\begin{tikzpicture}[node distance=1.2cm]
    \node (strat) [gtu block] {SEO વ્યૂહરચના};
    \node (vis) [gtu block, right=of strat] {વધેલી દૃશ્યતા};
    \node (traffic) [gtu block, right=of vis] {વધુ ઓર્ગેનિક ટ્રાફિક};
    \node (conv) [gtu block, below=of traffic] {ઊંચા કન્વર્ઝન};
    \node (auth) [gtu block, left=of conv] {વધેલી બ્રાન્ડ ઓથોરિટી};
    \node (pres) [gtu block, left=of auth] {ટકાઉ ઓનલાઇન હાજરી};

    \draw [gtu arrow] (strat) -- (vis);
    \draw [gtu arrow] (vis) -- (traffic);
    \draw [gtu arrow] (traffic) -- (conv);
    \draw [gtu arrow] (conv) -- (auth);
    \draw [gtu arrow] (auth) -- (pres);
\end{tikzpicture}
\captionof{figure}{SEO અસર ચક્ર}
\end{center}

\textbf{મુખ્ય યોગદાન:}

\begin{center}
\captionof{table}{ઓનલાઇન હાજરી પર SEO અસર}
\begin{tabulary}{\linewidth}{L L L}
\toprule
\textbf{પાસું} & \textbf{SEO અસર} & \textbf{વ્યાપારિક ફાયદો} \\
\midrule
\textbf{સર્ચ વિઝિબિલિટી} & ઊંચી SERP રેંકિંગ & વધુ સંભવિત ગ્રાહકો તમને શોધે છે \\
\textbf{વિશ્વસનીયતા} & આધિકારિક કન્ટેન્ટ & વપરાશકર્તાઓ ટોચની રેંકવાળા પરિણામો પર વિશ્વાસ કરે છે \\
\textbf{વપરાશકર્તા અનુભવ} & ઝડપી, મોબાઇલ-ફ્રેન્ડલી સાઇટ્સ & બેહતર એંગેજમેન્ટ મેટ્રિક્સ \\
\textbf{કોસ્ટ-અસરકારક} & ઓર્ગેનિક ટ્રાફિક જનરેશન & ઓછી ગ્રાહક અધિગ્રહણ કિંમતો \\
\bottomrule
\end{tabulary}
\end{center}

\textbf{ઉદાહરણ: ઇ-કોમર્સ સ્ટોર:}
એક સ્થાનિક ઇલેક્ટ્રોનિક્સ સ્ટોરે SEO વ્યૂહરચના અમલ કરી:
\begin{itemize}
    \item \textbf{પહેલાં}: "electronics store" માટે પેજ 3 પર રેંકિંગ
    \item \textbf{SEO ક્રિયાઓ}: પ્રોડક્ટ પેજ ઓપ્ટિમાઇઝ, લોકલ SEO, ગુણવત્તાયુક્ત કન્ટેન્ટ
    \item \textbf{પછી}: પેજ 1 રેંકિંગ, 300\% ટ્રાફિક વધારો, 150\% વેચાણ વૃદ્ધિ
\end{itemize}

\textbf{લાંબા ગાળાના ફાયદા:}
\begin{itemize}
    \item \textbf{ટકાઉ ટ્રાફિક}: પેઇડ જાહેરાતોથી વિપરીત, ઓર્ગેનિક પરિણામો સ્થિર રહે છે
    \item \textbf{બ્રાન્ડ બિલ્ડિંગ}: સતત દૃશ્યતા ઓળખ બનાવે છે
    \item \textbf{બજાર વિસ્તરણ}: ઉત્પાદનો માટે સક્રિયપણે શોધતા ગ્રાહકો સુધી પહોંચો
\end{itemize}

\begin{mnemonicbox}સર્ચ એન્જિન ઓપ્ટિમાઇઝેશન = ટકાઉ ઓનલાઇન સફળતા\end{mnemonicbox}
\end{solutionbox}


\questionmarks{પ્રશ્ન 3(અ)}{3}{Unique Visitors, Pageviews વ્યાખ્યાયિત કરો}
\begin{solutionbox}
\textbf{વેબ એનાલિટિક્સ વ્યાખ્યાઓ:}

\begin{center}
\captionof{table}{Unique Visitors અને Pageviews}
\begin{tabulary}{\linewidth}{L L L}
\toprule
\textbf{મેટ્રિક} & \textbf{વ્યાખ્યા} & \textbf{માપણ સમયગાળો} \\
\midrule
\textbf{Unique Visitors} & સાઇટની મુલાકાત લેતા અલગ વ્યક્તિઓ & ચોક્કસ સમયગાળો \\
\textbf{Pageviews} & જોવાયેલા કુલ પેજ & વ્યક્તિગત પેજ લોડ્સ \\
\bottomrule
\end{tabulary}
\end{center}

\begin{itemize}
    \item \textbf{Unique Visitors}: પેજ જોયા હોય તે ગમે તેટલા હોય, સત્ર દીઠ એકવાર ગણાય છે
    \item \textbf{Pageviews}: દરેક પેજ રિફ્રેશ અથવા નવા પેજ અલગ ગણાય છે
    \item \textbf{સંબંધ}: એક અનન્ય મુલાકાતી અનેક pageviews જનરેટ કરી શકે છે
\end{itemize}

\begin{mnemonicbox}અનોખા વપરાશકર્તાઓ, જોવાયેલા પેજો\end{mnemonicbox}
\end{solutionbox}

\questionmarks{પ્રશ્ન 3(બ)}{4}{વેબસાઇટની કાર્યક્ષમતાને સમજવા Content Analytics Tools કેવી રીતે યોગદાન આપે છે?}
\begin{solutionbox}
Content Analytics Tools વપરાશકર્તાઓ વેબસાઇટ કન્ટેન્ટ સાથે કેવી રીતે ક્રિયાપ્રતિક્રિયા કરે છે તેની અંતર્દૃષ્ટિ પ્રદાન કરે છે, જે ડેટા-સંચાલિત ઓપ્ટિમાઇઝેશન નિર્ણયો સક્ષમ કરે છે.

\textbf{યોગદાન ક્ષેત્રો:}

\begin{center}
\captionof{table}{Content Analytics અંતર્દૃષ્ટિ}
\begin{tabulary}{\linewidth}{L L L}
\toprule
\textbf{વિશ્લેષણ પ્રકાર} & \textbf{પ્રદાન કરેલી અંતર્દૃષ્ટિ} & \textbf{ઓપ્ટિમાઇઝેશન ક્રિયાઓ} \\
\midrule
\textbf{કન્ટેન્ટ પરફોર્મન્સ} & પેજ લોકપ્રિયતા, એંગેજમેન્ટ સમય & ઉચ્ચ પ્રદર્શન વિષયો પર ધ્યાન કેન્દ્રિત કરો \\
\textbf{વપરાશકર્તા વર્તન} & વાંચન પેટર્ન, સ્ક્રોલ ઊંડાઈ & કન્ટેન્ટ સ્ટ્રક્ચર સુધારો \\
\textbf{કન્વર્ઝન ટ્રેકિંગ} & કન્ટેન્ટ-ટુ-કન્વર્ઝન પાથ & કન્વર્ઝન ફનલ ઓપ્ટિમાઇઝ કરો \\
\bottomrule
\end{tabulary}
\end{center}

\begin{itemize}
    \item \textbf{પરફોર્મન્સ મેટ્રિક્સ}: બાઉન્સ રેટ, પેજ પર સમય, સોશિયલ શેર
    \item \textbf{કન્ટેન્ટ ગેપ}: ખૂટતા વિષયો, વપરાશકર્તા સર્ચ ક્વેરીઓ ઓળખો
    \item \textbf{A/B ટેસ્ટિંગ}: અસરકારકતા માટે કન્ટેન્ટ વેરિયેશન સરખાવો
    \item \textbf{ROI માપણ}: કન્ટેન્ટ પ્રયત્નોને વ્યાપારિક લક્ષ્યો સાથે જોડો
\end{itemize}

\begin{mnemonicbox}કન્ટેન્ટ એનાલિટિક્સ કાર્યક્ષમ અંતર્દૃષ્ટિ બનાવે છે\end{mnemonicbox}
\end{solutionbox}

\questionmarks{પ્રશ્ન 3(ક)}{7}{Web analytics માં વપરાતા જુદા જુદા attribution models ની ચર્ચા કરો. ઉદાહરણ સાથે.}
\begin{solutionbox}
Attribution models માર્કેટર્સને ગ્રાહક જર્નીમાં કયા ટચપોઇન્ટ્સ કન્વર્ઝનમાં યોગદાન આપે છે તે સમજવામાં મદદ કરે છે.

\textbf{Attribution Model પ્રકારો:}

\begin{center}
\begin{tikzpicture}[node distance=1.5cm]
    \node (attrib) [gtu block] {Attribution Models};
    \node (single) [gtu block, below left=of attrib, xshift=-1cm] {સિંગલ-ટચ મોડેલ્સ};
    \node (multi) [gtu block, below right=of attrib, xshift=1cm] {મલ્ટિ-ટચ મોડેલ્સ};

    \draw [gtu arrow] (attrib) -- (single);
    \draw [gtu arrow] (attrib) -- (multi);

    \node (single_types) [gtu block, below=of single, align=left, font=\small] {ફર્સ્ટ-ક્લિક\\લાસ્ટ-ક્લિક\\લાસ્ટ નોન-ડાયરેક્ટ};
    \node (multi_types) [gtu block, below=of multi, align=left, font=\small] {લિનિયર\\ટાઇમ-ડીકે\\પોઝિશન-બેસ્ડ\\ડેટા-ડ્રિવન};

    \draw [gtu arrow] (single) -- (single_types);
    \draw [gtu arrow] (multi) -- (multi_types);
\end{tikzpicture}
\captionof{figure}{Attribution Model વર્ગીકરણ}
\end{center}

\textbf{મોડેલ સરખામણી:}

\begin{center}
\captionof{table}{Attribution Model સરખામણી}
\begin{tabulary}{\linewidth}{L L L}
\toprule
\textbf{મોડેલ} & \textbf{ક્રેડિટ વિતરણ} & \textbf{શ્રેષ્ઠ ઉપયોગ કેસ} \\
\midrule
\textbf{ફર્સ્ટ-ક્લિક} & પ્રથમ ટચપોઇન્ટને 100\% & બ્રાન્ડ જાગૃતિ ઝુંબેશ \\
\textbf{લાસ્ટ-ક્લિક} & અંતિમ ટચપોઇન્ટને 100\% & ડાયરેક્ટ રિસ્પોન્સ માર્કેટિંગ \\
\textbf{લિનિયર} & બધા ટચપોઇન્ટ્સને સમાન ક્રેડિટ & લાંબા વેચાણ ચક્ર \\
\textbf{ટાઇમ-ડીકે} & તાજેતરની ક્રિયાપ્રતિક્રિયાઓને વધુ ક્રેડિટ & ટૂંકા વિચારણા સમયગાળો \\
\bottomrule
\end{tabulary}
\end{center}

\textbf{ઉદાહરણ ગ્રાહક જર્ની:}
1. \textbf{Facebook Ad} (જાગૃતિ) $\to$ 2. \textbf{Google Search} (સંશોધન) $\to$ 3. \textbf{Email} (કન્વર્ઝન)

\textbf{Attribution પરિણામો:}
\begin{itemize}
    \item \textbf{ફર્સ્ટ-ક્લિક}: Facebook Ad ને 100\% ક્રેડિટ
    \item \textbf{લાસ્ટ-ક્લિક}: Email ને 100\% ક્રેડિટ
    \item \textbf{લિનિયર}: દરેક ટચપોઇન્ટને 33.3\% ક્રેડિટ
    \item \textbf{ટાઇમ-ડીકે}: Email 50\%, Google 30\%, Facebook 20\%
\end{itemize}

\textbf{યોગ્ય મોડેલ પસંદ કરવું:}
\begin{itemize}
    \item \textbf{વ્યાપારિક લક્ષ્યો}: જાગૃતિ વિ. કન્વર્ઝન ફોકસ
    \item \textbf{વેચાણ ચક્ર લંબાઈ}: ટૂંકા વિ. લાંબા વિચારણા સમયગાળો
    \item \textbf{માર્કેટિંગ મિક્સ}: સિંગલ વિ. મલ્ટિ-ચેનલ વ્યૂહરચના
\end{itemize}

\begin{mnemonicbox}ફર્સ્ટ, લાસ્ટ, લિનિયર, ટાઇમ - Attribution ની ડિઝાઇન\end{mnemonicbox}
\end{solutionbox}

\questionmarks{પ્રશ્ન 3(અ) વા}{3}{Average Visit Duration, Bounce Rate, અને New Visits વ્યાખ્યાયિત કરો.}
\begin{solutionbox}
\textbf{વેબ એનાલિટિક્સ મેટ્રિક્સ:}

\begin{center}
\captionof{table}{વેબ એનાલિટિક્સ મેટ્રિક્સ}
\begin{tabulary}{\linewidth}{L L L}
\toprule
\textbf{મેટ્રિક} & \textbf{વ્યાખ્યા} & \textbf{ગણતરી} \\
\midrule
\textbf{Average Visit Duration} & સત્ર દીઠ વિતાવેલો સમય & કુલ સમય $\div$ સત્રો \\
\textbf{Bounce Rate} & સિંગલ-પેજ સત્રોની ટકાવારી & બાઉન્સ $\div$ કુલ સત્રો $\times$ 100 \\
\textbf{New Visits} & પ્રથમ વખતના મુલાકાતીઓની ટકાવારી & નવા વપરાશકર્તાઓ $\div$ કુલ વપરાશકર્તાઓ $\times$ 100 \\
\bottomrule
\end{tabulary}
\end{center}

\begin{itemize}
    \item \textbf{Visit Duration}: કન્ટેન્ટ એંગેજમેન્ટ અને વપરાશકર્તા રુચિ સૂચવે છે
    \item \textbf{Bounce Rate}: કન્ટેન્ટ સુસંગતતા અને સાઇટ ઉપયોગિતા દર્શાવે છે
    \item \textbf{New Visits}: પ્રેક્ષક વૃદ્ધિ અને અધિગ્રહણ અસરકારકતા માપે છે
\end{itemize}

\begin{mnemonicbox}અવધિ, બાઉન્સ, નવું - એનાલિટિક્સ સત્ય\end{mnemonicbox}
\end{solutionbox}

\questionmarks{પ્રશ્ન 3(બ) વા}{4}{વેબસાઇટની કાર્યક્ષમતાને સમજવા Customer Analytics Tools કેવી રીતે યોગદાન આપે છે?}
\begin{solutionbox}
Customer Analytics Tools વપરાશકર્તા વર્તન, પસંદગીઓ અને કન્વર્ઝન પેટર્નમાં ઊંડી અંતર્દૃષ્ટિ પ્રદાન કરે છે.

\textbf{મુખ્ય યોગદાન:}

\begin{center}
\captionof{table}{Customer Analytics યોગદાન}
\begin{tabulary}{\linewidth}{L L L}
\toprule
\textbf{એનાલિટિક્સ ક્ષેત્ર} & \textbf{અંતર્દૃષ્ટિ} & \textbf{પરફોર્મન્સ અસર} \\
\midrule
\textbf{વપરાશકર્તા વિભાજન} & ડેમોગ્રાફિક્સ, વર્તન પેટર્ન & લક્ષિત કન્ટેન્ટ બનાવટ \\
\textbf{જર્ની મેપિંગ} & કન્વર્ઝન પાથ, ડ્રોપ-ઓફ પોઇન્ટ્સ & ઓપ્ટિમાઇઝ્ડ વપરાશકર્તા અનુભવ \\
\textbf{લાઇફટાઇમ વેલ્યુ} & ગ્રાહક મૂલ્ય, રિટેન્શન રેટ્સ & ROI-કેન્દ્રિત વ્યૂહરચના \\
\bottomrule
\end{tabulary}
\end{center}

\begin{itemize}
    \item \textbf{વર્તણૂકીય વિશ્લેષણ}: ક્લિક પેટર્ન, નેવિગેશન પસંદગીઓ
    \item \textbf{કન્વર્ઝન ઓપ્ટિમાઇઝેશન}: વપરાશકર્તા જર્નીમાં ઘર્ષણ બિંદુઓ ઓળખો
    \item \textbf{વ્યક્તિગતકરણ}: વપરાશકર્તા પ્રોફાઇલ આધારિત કસ્ટમાઇઝ્ડ કન્ટેન્ટ
    \item \textbf{રિટેન્શન વ્યૂહરચના}: ગ્રાહકોને સંલગ્ન રાખતી બાબતો સમજવી
\end{itemize}

\begin{mnemonicbox}ગ્રાહક એનાલિટિક્સ સ્પર્ધાત્મક ફાયદા બનાવે છે\end{mnemonicbox}
\end{solutionbox}

\questionmarks{પ્રશ્ન 3(ક) વા}{7}{Google Analytics માં goals સેટ કરવા અને conversion rates ટ્રેક કરવાની પ્રક્રિયા પર ઉદાહરણ સાથે ચર્ચા કરો.}
\begin{solutionbox}
Google Analytics માં goals સેટ કરવું અને conversions ટ્રેક કરવું વેબસાઇટની સફળતા માપવા અને ROI ઓપ્ટિમાઇઝેશન સક્ષમ કરે છે.

\textbf{Goal Setup પ્રક્રિયા:}

\begin{center}
\begin{tikzpicture}[node distance=1.5cm]
    \node (goals) [gtu block] {Google Analytics Goals};
    \node (config) [gtu block, below=of goals] {Goal Configuration};
    \node (types) [gtu block, below=of config] {Goal Types};

    \node (dest) [gtu block, below left=of types, xshift=-2cm] {Destination};
    \node (dur) [gtu block, right=of dest] {Duration};
    \node (pgs) [gtu block, right=of dur] {Pages/Session};
    \node (evt) [gtu block, right=of pgs] {Event};

    \node (conv) [gtu block, below=of dur, xshift=1.5cm] {Conversion Tracking};

    \draw [gtu arrow] (goals) -- (config);
    \draw [gtu arrow] (config) -- (types);
    \draw [gtu arrow] (types) -- (dest);
    \draw [gtu arrow] (types) -- (dur);
    \draw [gtu arrow] (types) -- (pgs);
    \draw [gtu arrow] (types) -- (evt);
    
    \draw [gtu arrow] (dest) -- (conv);
    \draw [gtu arrow] (dur) -- (conv);
    \draw [gtu arrow] (pgs) -- (conv);
    \draw [gtu arrow] (evt) -- (conv);
\end{tikzpicture}
\captionof{figure}{Google Analytics Goal સેટઅપ}
\end{center}

\textbf{Goal પ્રકારો અને સેટઅપ:}

\begin{center}
\captionof{table}{Goal પ્રકારો}
\begin{tabulary}{\linewidth}{L L L}
\toprule
\textbf{Goal Type} & \textbf{વર્ણન} & \textbf{ઉદાહરણ સેટઅપ} \\
\midrule
\textbf{Destination} & ચોક્કસ પેજ મુલાકાતો & Thank you page URL \\
\textbf{Duration} & સત્રની લંબાઈ & સત્રો > 3 મિનિટ \\
\textbf{Pages/Session} & મુલાકાત દીઠ પેજ વ્યૂઝ & 5 થી વધુ પેજો \\
\textbf{Event} & ચોક્કસ ક્રિયાઓ & ડાઉનલોડ બટન ક્લિક \\
\bottomrule
\end{tabulary}
\end{center}

\textbf{ઉદાહરણ: ઇ-કોમર્સ કન્વર્ઝન સેટઅપ:}

\textbf{પગલું-દર-પગલું પ્રક્રિયા:}
\begin{enumerate}
    \item \textbf{Goals એક્સેસ કરો}: Admin $\to$ View $\to$ Goals $\to$ New Goal
    \item \textbf{Goal Type}: Destination (Thank you page)
    \item \textbf{Goal Details}: 
    \begin{itemize}
        \item નામ: "Purchase Completion"
        \item પ્રકાર: Destination
        \item Destination: "/thank-you"
    \end{itemize}
    \item \textbf{Funnel Setup}: ચેકઆઉટ સ્ટેપ્સ ઉમેરો
    \item \textbf{Value Assignment}: સરેરાશ ઓર્ડર વેલ્યુ
\end{enumerate}

\textbf{Conversion Rate ગણતરી:}
\begin{itemize}
    \item \textbf{ફોર્મ્યુલા}: (Conversions $\div$ Sessions) $\times$ 100
    \item \textbf{ઉદાહરણ}: 50 ખરીદીઓ $\div$ 2,000 સત્રો = 2.5\% conversion rate
\end{itemize}

\textbf{ટ્રેકિંગ ફાયદા:}
\begin{itemize}
    \item \textbf{પરફોર્મન્સ માપણ}: સ્પષ્ટ સફળતા મેટ્રિક્સ
    \item \textbf{ROI ગણતરી}: માર્કેટિંગ ચેનલોને આવક એટ્રિબ્યુશન
    \item \textbf{ઓપ્ટિમાઇઝેશન તકો}: સુધારણા ક્ષેત્રો ઓળખો
\end{itemize}

\begin{mnemonicbox}ગોલ્સ ગ્રેટ ગ્રોથ ગાઇડન્સ આપે છે\end{mnemonicbox}
\end{solutionbox}

\questionmarks{પ્રશ્ન 4(અ)}{3}{Marketers માટે ઉપલબ્ધ Twitter Ads ના પ્રકારો કયા છે?}
\begin{solutionbox}
\textbf{Twitter જાહેરાત વિકલ્પો:}

\begin{center}
\captionof{table}{Twitter Ad પ્રકારો}
\begin{tabulary}{\linewidth}{L L L}
\toprule
\textbf{Ad Type} & \textbf{હેતુ} & \textbf{ફોર્મેટ} \\
\midrule
\textbf{Promoted Tweets} & એંગેજમેન્ટ વધારો & નેટિવ ટ્વીટ દેખાવ \\
\textbf{Promoted Accounts} & ફોલોવર્સ વધારો & એકાઉન્ટ સૂચનો \\
\textbf{Promoted Trends} & વિષય દૃશ્યતા & ટ્રેન્ડિંગ સેક્શન પ્લેસમેન્ટ \\
\bottomrule
\end{tabulary}
\end{center}

\begin{itemize}
    \item \textbf{Promoted Tweets}: અસ્તિત્વમાં રહેલા ટ્વીટ્સની પહોંચ વધારો, ક્લિક્સ/કન્વર્ઝન ચલાવો
    \item \textbf{Promoted Accounts}: ફોલો કરવાની સંભાવના ધરાવતા વપરાશકર્તાઓને ટાર્ગેટ કરો, પ્રેક્ષક સાઇઝ વધારો
    \item \textbf{Promoted Trends}: ટ્રેન્ડિંગ વિષયોમાં પ્રીમિયમ પ્લેસમેન્ટ, ઉચ્ચ દૃશ્યતા
\end{itemize}

\begin{mnemonicbox}ટ્વીટ્સ, એકાઉન્ટ્સ, ટ્રેન્ડ્સ - Twitter જાહેરાતનો અંત\end{mnemonicbox}
\end{solutionbox}

\questionmarks{પ્રશ્ન 4(બ)}{4}{તમને કંપનીના આગામી વેબિનાર માટે LinkedIn જાહેરાત અભિયાન વિકસાવવા માટે નિમણૂક કરવામાં આવી છે. આ અભિયાન માટે LinkedIn Ads બનાવવા અને સુધારવા માટેની પ્રક્રિયા outline કરો. તેમાં તમે કયા પ્રકારની LinkedIn ads પસંદ કરશો, કયા content ઉપયોગમાં લેશો, અને LinkedIn Analytics નો ઉપયોગ કરીને અભિયાનની અસરકારકતા મૂલવવા અને સુધારવા માટે કેવી રીતે કાર્ય કરશો તે સમાવો.}
\begin{solutionbox}
\textbf{LinkedIn વેબિનાર કેમ્પેઇન વ્યૂહરચના:}

\begin{center}
\begin{tikzpicture}[node distance=1.5cm]
    \node (plan) [gtu block, minimum width=4cm] {વેબિનાર કેમ્પેઇન પ્લાનિંગ};
    \node (ads) [gtu block, below left=of plan] {Ad Types પસંદગી};
    \node (content) [gtu block, below right=of plan] {કન્ટેન્ટ વ્યૂહરચના};
    \node (spon) [gtu block, below=of ads] {Sponsored Content};
    \node (msg) [gtu block, below=of content] {પ્રોફેશનલ મેસેજિંગ};
    \node (opt) [gtu block, below=of plan, yshift=-3cm] {Analytics \& ઓપ્ટિમાઇઝેશન};

    \draw [gtu arrow] (plan) -| (ads);
    \draw [gtu arrow] (plan) -| (content);
    \draw [gtu arrow] (ads) -- (spon);
    \draw [gtu arrow] (content) -- (msg);
    \draw [gtu arrow] (spon) |- (opt);
    \draw [gtu arrow] (msg) |- (opt);
\end{tikzpicture}
\captionof{figure}{LinkedIn કેમ્પેઇન ફ્લો}
\end{center}

\textbf{કેમ્પેઇન ડેવલપમેન્ટ પ્રક્રિયા:}

\begin{center}
\captionof{table}{LinkedIn કેમ્પેઇન પ્રક્રિયા}
\begin{tabulary}{\linewidth}{L L L}
\toprule
\textbf{તબક્કો} & \textbf{કાર્ય આઇટમ્સ} & \textbf{અમલીકરણ} \\
\midrule
\textbf{Ad Selection} & Sponsored Content + Message Ads પસંદ કરો & એંગેજમેન્ટ માટે વિડિયો કન્ટેન્ટ \\
\textbf{Targeting} & પ્રોફેશનલ ડેમોગ્રાફિક્સ, જોબ ટાઇટલ & IT પ્રોફેશનલ્સ, નિર્ણય નિર્માતાઓ \\
\textbf{Content Creation} & વેલ્યુ પ્રોપોઝિશન, સ્પષ્ટ CTA & "નિષ્ણાત-સંચાલિત માર્કેટિંગ વેબિનારમાં જોડાઓ" \\
\textbf{Optimization} & A/B ટેસ્ટ હેડલાઇન્સ, CTR મોનિટર કરો & પરફોર્મન્સ ડેટાના આધારે એડજસ્ટ કરો \\
\bottomrule
\end{tabulary}
\end{center}

\textbf{ભલામણ કરેલા Ad Types:}
\begin{itemize}
    \item \textbf{Sponsored Content}: નેટિવ ફીડ પ્લેસમેન્ટ, પ્રોફેશનલ દેખાવ
    \item \textbf{Message Ads}: ડાયરેક્ટ ઇનબોક્સ ડિલિવરી, વ્યક્તિગત અભિગમ
    \item \textbf{Dynamic Ads}: પ્રોફાઇલ આધારિત વ્યક્તિગત ક્રિએટિવ
\end{itemize}

\textbf{કન્ટેન્ટ વ્યૂહરચના:}
\begin{itemize}
    \item \textbf{હેડલાઇન્સ}: "ડિજિટલ માર્કેટિંગમાં નિપુણતા: મફત નિષ્ણાત વેબિનાર"
    \item \textbf{વિઝ્યુઅલ્સ}: પ્રોફેશનલ સ્પીકર ફોટા, એજેન્ડા હાઇલાઇટ્સ
    \item \textbf{CTA}: "હમણાં રજિસ્ટર કરો - મર્યાદિત સીટો ઉપલબ્ધ"
\end{itemize}

\begin{mnemonicbox}પસંદ કરો, ટાર્ગેટ કરો, બનાવો, ઓપ્ટિમાઇઝ કરો - LinkedIn સફળતા\end{mnemonicbox}
\end{solutionbox}

\questionmarks{પ્રશ્ન 4(ક)}{7}{ડિજિટલ માર્કેટિંગ વ્યૂહરચનાઓમાં વિડિયો માર્કેટિંગની ભૂમિકા અને મહત્વ પર ચર્ચા કરો. YouTube Ads વ્યાપક વિડિયો માર્કેટિંગ વ્યૂહરચનામાં કેવી રીતે ફિટ થાય છે તે સમજાવો.}
\begin{solutionbox}
વિડિયો માર્કેટિંગ આધુનિક ડિજિટલ માર્કેટિંગ વ્યૂહરચનાઓનો પાયાનો પથ્થર બની ગયું છે, જે અપ્રતિમ એંગેજમેન્ટ અને કન્વર્ઝન સંભાવના પ્રદાન કરે છે.

\textbf{વિડિયો માર્કેટિંગ મહત્વ:}

\begin{center}
\begin{tikzpicture}[node distance=1.5cm]
    \node (video) [gtu block] {વિડિયો માર્કેટિંગ};
    
    \node (engage) [gtu block, below left=of video, xshift=-1cm] {ઉચ્ચ એંગેજમેન્ટ};
    \node (convert) [gtu block, below=of video] {બેહતર કન્વર્ઝન};
    \node (story) [gtu block, below right=of video, xshift=1cm] {બ્રાન્ડ સ્ટોરીટેલિંગ};
    
    \node (reach) [gtu block, below=of convert] {વધેલી પહોંચ};

    \draw [gtu arrow] (video) -- (engage);
    \draw [gtu arrow] (video) -- (convert);
    \draw [gtu arrow] (video) -- (story);
    
    \draw [gtu arrow] (engage) -- (reach);
    \draw [gtu arrow] (convert) -- (reach);
    \draw [gtu arrow] (story) -- (reach);
\end{tikzpicture}
\captionof{figure}{વિડિયો માર્કેટિંગ ફાયદા}
\end{center}

\textbf{વ્યૂહાત્મક મહત્વ:}

\begin{center}
\captionof{table}{વિડિયો માર્કેટિંગ અસર}
\begin{tabulary}{\linewidth}{L L L}
\toprule
\textbf{પાસું} & \textbf{અસર} & \textbf{વ્યાપારિક મૂલ્ય} \\
\midrule
\textbf{એંગેજમેન્ટ} & ટેક્સ્ટ કન્ટેન્ટ કરતાં 10x વધુ & વધેલું બ્રાન્ડ રિકૉલ \\
\textbf{કન્વર્ઝન} & ખરીદી માટે 80\% વધુ શક્યતા & ઉચ્ચ વેચાણ આવક \\
\textbf{SEO વેલ્યુ} & પ્રથમ રેંક કરવાની 53x વધુ શક્યતા & ઓર્ગેનિક ટ્રાફિક વૃદ્ધિ \\
\bottomrule
\end{tabulary}
\end{center}

\textbf{YouTube Ads ઇન્ટિગ્રેશન:}

\begin{itemize}
    \item \textbf{વ્યાપક વ્યૂહરચના કનેક્શન}: જાગૃતિ $\to$ વિચારણા $\to$ કન્વર્ઝન ફનલ ઇન્ટિગ્રેશન
    \item \textbf{ક્રોસ-પ્લેટફોર્મ ડિસ્ટ્રિબ્યુશન}: YouTube વિડિયો સોશિયલ મીડિયા/વેબસાઇટ માટે ફરીથી વાપરો
    \item \textbf{રિટાર્ગેટિંગ}: ફોલો-અપ ads માટે વિડિયો દર્શકોમાંથી કસ્ટમ ઓડિયન્સ બનાવો
\end{itemize}

\begin{mnemonicbox}વિડિયો એંગેજ કરે, કન્વર્ટ કરે અને માર્કેટિંગ એક્સેલન્સ સ્કેલ કરે\end{mnemonicbox}
\end{solutionbox}

\questionmarks{પ્રશ્ન 4(અ) વા}{3}{LinkedIn ના Campaign Manager ના બે મુખ્ય લક્ષણો નામ આપો.}
\begin{solutionbox}
\textbf{LinkedIn Campaign Manager લક્ષણો:}

\begin{center}
\captionof{table}{LinkedIn Campaign Manager લક્ષણો}
\begin{tabulary}{\linewidth}{L L L}
\toprule
\textbf{લક્ષણ} & \textbf{કાર્યક્ષમતા} & \textbf{ફાયદો} \\
\midrule
\textbf{ઓડિયન્સ ટાર્ગેટિંગ} & પ્રોફેશનલ ડેમોગ્રાફિક્સ, જોબ ફંક્શન્સ & ચોક્કસ B2B ટાર્ગેટિંગ \\
\textbf{પરફોર્મન્સ એનાલિટિક્સ} & રીઅલ-ટાઇમ મેટ્રિક્સ, કન્વર્ઝન ટ્રેકિંગ & ડેટા-સંચાલિત ઓપ્ટિમાઇઝેશન \\
\bottomrule
\end{tabulary}
\end{center}

\begin{itemize}
    \item \textbf{ઓડિયન્સ ટાર્ગેટિંગ}: ઉદ્યોગ, કંપની સાઇઝ, જોબ ટાઇટલ, કૌશલ્ય-આધારિત વિભાજન
    \item \textbf{પરફોર્મન્સ એનાલિટિક્સ}: CTR, CPC, કન્વર્ઝન ટ્રેકિંગ, A/B ટેસ્ટિંગ ક્ષમતાઓ
\end{itemize}

\begin{mnemonicbox}ચોક્કસ ટાર્ગેટ, પરફોર્મન્સ એનાલાઇઝ\end{mnemonicbox}
\end{solutionbox}

\questionmarks{પ્રશ્ન 4(બ) વા}{4}{તમને Instagram પર નવા Product launch માટે એક જાહેરાત અભિયાન બનાવવાનું કામ આપવામાં આવ્યું છે. Instagram Ads બનાવવા અને સુધારવા માટે તમે કયા પગલાં લેશો તે outline કરો, તેમાં ઉપયોગ કરવા માટેના content types પણ સમાવેશ કરો.}
\begin{solutionbox}
\textbf{Instagram પ્રોડક્ટ લૉન્ચ કેમ્પેઇન:}

\begin{center}
\begin{tikzpicture}[node distance=1.5cm]
    \node (strat) [gtu block] {પ્રોડક્ટ લૉન્ચ વ્યૂહરચના};
    
    \node (planning) [gtu block, below left=of strat, xshift=-1cm] {કન્ટેન્ટ પ્લાનિંગ};
    \node (format) [gtu block, below left=of strat, xshift=1cm] {Ad ફોર્મેટ};
    \node (targeting) [gtu block, below right=of strat, xshift=-1cm] {ટાર્ગેટિંગ};
    \node (opt) [gtu block, below right=of strat, xshift=1cm] {ઓપ્ટિમાઇઝેશન};

    \node (visuals) [gtu block, below=of planning] {વિઝ્યુઅલ કન્ટેન્ટ};
    \node (stories) [gtu block, below=of format] {Stories Ads};
    \node (interests) [gtu block, below=of targeting] {Interest Groups};
    \node (track) [gtu block, below=of opt] {પરફોર્મન્સ ટ્રેકિંગ};

    \draw [gtu arrow] (strat) -- (planning);
    \draw [gtu arrow] (strat) -- (format);
    \draw [gtu arrow] (strat) -- (targeting);
    \draw [gtu arrow] (strat) -- (opt);

    \draw [gtu arrow] (planning) -- (visuals);
    \draw [gtu arrow] (format) -- (stories);
    \draw [gtu arrow] (targeting) -- (interests);
    \draw [gtu arrow] (opt) -- (track);
\end{tikzpicture}
\captionof{figure}{Instagram લૉન્ચ વ્યૂહરચના}
\end{center}

\textbf{કેમ્પેઇન ડેવલપમેન્ટ સ્ટેપ્સ:}

\begin{center}
\captionof{table}{Instagram કેમ્પેઇન સ્ટેપ્સ}
\begin{tabulary}{\linewidth}{L L L}
\toprule
\textbf{સ્ટેપ} & \textbf{ક્રિયા} & \textbf{અમલીકરણ} \\
\midrule
\textbf{1. કન્ટેન્ટ ક્રિએશન} & વિઝ્યુઅલ સ્ટોરીટેલિંગ & પ્રોડક્ટ ફોટા, લાઇફસ્ટાઇલ ઇમેજ \\
\textbf{2. Ad Format} & Feed + Stories + Reels & મલ્ટિ-ફોર્મેટ અભિગમ \\
\textbf{3. Targeting Setup} & ડેમોગ્રાફિક્સ + રુચિઓ & Lookalike ઓડિયન્સ \\
\textbf{4. Budget Allocation} & દૈનિક ખર્ચ મર્યાદા & પરફોર્મન્સ-આધારિત ઓપ્ટિમાઇઝેશન \\
\bottomrule
\end{tabulary}
\end{center}

\textbf{કન્ટેન્ટ વ્યૂહરચના:}
\begin{itemize}
    \item \textbf{Feed Posts}: ઉચ્ચ-ગુણવત્તાની પ્રોડક્ટ ફોટોગ્રાફી, લાઇફસ્ટાઇલ કન્ટેક્સ્ટ
    \item \textbf{Stories Ads}: પડદા પાછળની કન્ટેન્ટ, યુઝર-જનરેટેડ કન્ટેન્ટ
    \item \textbf{Reels}: ટ્રેન્ડિંગ ઓડિયો, પ્રોડક્ટ ડેમોન્સ્ટ્રેશન, ટ્યુટોરિયલ્સ
    \item \textbf{Carousel Ads}: અનેક પ્રોડક્ટ એંગલ, ફીચર હાઇલાઇટ્સ
\end{itemize}

\begin{mnemonicbox}બનાવો, પસંદ કરો, ટાર્ગેટ કરો, ટ્રેક કરો - Instagram અસર\end{mnemonicbox}
\end{solutionbox}

\questionmarks{પ્રશ્ન 4(ક) વા}{7}{Facebook ના advertising એલ્ગોરિધમને સમજવાની મહત્વપૂર્ણતાનો સ્પષ્ટ કરો અને તે ad delivery ને કેવી રીતે અસર કરે છે તે સમજાવો.}
\begin{solutionbox}
Facebook ના advertising એલ્ગોરિધમને સમજવું ad પરફોર્મન્સને મહત્તમ બનાવવા અને રોકાણ પર શ્રેષ્ઠ પ્રતિસાદ મેળવવા માટે મહત્વપૂર્ણ છે.

\textbf{એલ્ગોરિધમ ઘટકો:}

\begin{center}
\begin{tikzpicture}[node distance=1.5cm]
    \node (algo) [gtu block] {Facebook એલ્ગોરિધમ};
    \node (adv) [gtu block, below left=of algo] {જાહેરાતકર્તા મૂલ્ય};
    \node (user) [gtu block, below right=of algo] {વપરાશકર્તા મૂલ્ય};
    \node (total) [gtu block, below=of algo, yshift=-2cm] {કુલ મૂલ્ય};
    \node (delivery) [gtu block, below=of total] {Ad ડિલિવરી નિર્ણય};

    \draw [gtu arrow] (algo) -- (adv);
    \draw [gtu arrow] (algo) -- (user);
    \draw [gtu arrow] (adv) -- (total);
    \draw [gtu arrow] (user) -- (total);
    \draw [gtu arrow] (total) -- (delivery);

    \node (adv_factors) [gtu block, below=of adv, align=center, font=\footnotesize] {બિડ રકમ\\Ad ગુણવત્તા};
    \node (user_factors) [gtu block, below=of user, align=center, font=\footnotesize] {વપરાશકર્તા રિલેવન્સ\\વપરાશકર્તા અનુભવ};

    \draw [gtu arrow] (adv) -- (adv_factors);
    \draw [gtu arrow] (user) -- (user_factors);
\end{tikzpicture}
\captionof{figure}{Facebook એલ્ગોરિધમ પરિબળો}
\end{center}

\textbf{એલ્ગોરિધમ પરિબળો:}

\begin{center}
\captionof{table}{Ad ડિલિવરી પરિબળો}
\begin{tabulary}{\linewidth}{L L L}
\toprule
\textbf{ઘટક} & \textbf{વેઇટ} & \textbf{ડિલિવરી પર અસર} \\
\midrule
\textbf{બિડ વ્યૂહરચના} & ઉચ્ચ & બજેટ એલોકેશન કાર્યક્ષમતા \\
\textbf{Ad રિલેવન્સ} & ઉચ્ચ & ગુણવત્તા સ્કોર નિર્ધારણ \\
\textbf{વપરાશકર્તા એંગેજમેન્ટ} & મધ્યમ & ઓડિયન્સ પ્રતિસાદ અનુમાન \\
\textbf{Landing Page} & મધ્યમ & એકંદર વપરાશકર્તા અનુભવ \\
\bottomrule
\end{tabulary}
\end{center}

\textbf{Ad ડિલિવરી પ્રક્રિયા:}
\begin{enumerate}
    \item \textbf{Auction Entry}: Ad રીઅલ-ટાઇમ બિડિંગમાં પ્રવેશ કરે છે
    \item \textbf{Value Calculation}: એલ્ગોરિધમ ad રિલેવન્સ અને ગુણવત્તા સ્કોર કરે છે
    \item \textbf{Winner Selection}: સૌથી ઊંચી કુલ વેલ્યુ પ્લેસમેન્ટ જીતે છે
    \item \textbf{Performance Feedback}: પરિણામો ભાવિ ડિલિવરીને પ્રભાવિત કરે છે
\end{enumerate}

\begin{mnemonicbox}એલ્ગોરિધમ જાગૃતિ જાહેરાત ફાયદો પ્રાપ્ત કરે છે\end{mnemonicbox}
\end{solutionbox}

\questionmarks{પ્રશ્ન 5(અ)}{3}{વિવિધ પ્રકારની Email Marketing ની સૂચિ બનાવો અને તેનું સંક્ષિપ્ત વર્ણન કરો.}
\begin{solutionbox}
\textbf{Email Marketing પ્રકારો:}

\begin{center}
\captionof{table}{Email Marketing પ્રકારો}
\begin{tabulary}{\linewidth}{L L L}
\toprule
\textbf{પ્રકાર} & \textbf{હેતુ} & \textbf{કન્ટેન્ટ ફોકસ} \\
\midrule
\textbf{Newsletter} & નિયમિત સંવાદ & કંપની અપડેટ્સ, ઉદ્યોગ સમાચાર \\
\textbf{Promotional} & વેચાણ અને ઓફર & ડિસ્કાઉન્ટ કોડ્સ, પ્રોડક્ટ લૉન્ચ \\
\textbf{Transactional} & ખરીદી પુષ્ટિકરણ & ઓર્ડર રિસીટ, શિપિંગ અપડેટ્સ \\
\bottomrule
\end{tabulary}
\end{center}

\begin{itemize}
    \item \textbf{Newsletter}: બ્રાન્ડ જાગૃતિ, ગ્રાહક જાળવણી, વિચાર નેતૃત્વ
    \item \textbf{Promotional}: વેચાણ વધારો, ઇવેન્ટ્સ પ્રમોટ કરો, મોસમી ઝુંબેશ
    \item \textbf{Transactional}: ઓર્ડર કન્ફર્મેશન, વેલકમ સિરીઝ, એકાઉન્ટ અપડેટ્સ
\end{itemize}

\begin{mnemonicbox}ન્યૂઝ, પ્રમોટ, ટ્રાન્ઝેક્ટ - Email નો અસર\end{mnemonicbox}
\end{solutionbox}

\questionmarks{પ્રશ્ન 5(બ)}{4}{તમે નવી પ્રોડક્ટ લૉન્ચ માટે એક Email Marketing અભિયાનની યોજના બનાવી રહ્યા છો. આ અભિયાન ડિઝાઇન અને અમલમાં લાવવાના પગલાંઓનું વિવરણ આપો, તેમાં તમે તેની સફળતા માપવા માટે Email Marketing Analytics નો કેવી રીતે ઉપયોગ કરશો તે પણ સમાવો.}
\begin{solutionbox}
\textbf{Email કેમ્પેઇન વ્યૂહરચના:}

\begin{center}
\begin{tikzpicture}[node distance=1.5cm]
    \node (launch) [gtu block] {પ્રોડક્ટ લૉન્ચ Email કેમ્પેઇન};
    
    \node (plan) [gtu block, below left=of launch, xshift=-1cm] {પ્લાનિંગ તબક્કો};
    \node (design) [gtu block, right=of plan] {ડિઝાઇન તબક્કો};
    \node (exec) [gtu block, right=of design] {એક્ઝિક્યુશન તબક્કો};
    \node (anal) [gtu block, right=of exec] {એનાલિટિક્સ તબક્કો};

    \node (target) [gtu block, below=of plan] {ટાર્ગેટ ઓડિયન્સ};
    \node (temp) [gtu block, below=of design] {Email ટેમ્પ્લેટ};
    \node (send) [gtu block, below=of exec] {Send Schedule};
    \node (meas) [gtu block, below=of anal] {પરિણામો માપો};

    \draw [gtu arrow] (launch) -- (plan);
    \draw [gtu arrow] (launch) -- (design);
    \draw [gtu arrow] (launch) -- (exec);
    \draw [gtu arrow] (launch) -- (anal);

    \draw [gtu arrow] (plan) -- (target);
    \draw [gtu arrow] (design) -- (temp);
    \draw [gtu arrow] (exec) -- (send);
    \draw [gtu arrow] (anal) -- (meas);
\end{tikzpicture}
\captionof{figure}{Email કેમ્પેઇન સ્ટેપ્સ}
\end{center}

\textbf{કેમ્પેઇન ડેવલપમેન્ટ પ્રક્રિયા:}

\begin{center}
\captionof{table}{Email કેમ્પેઇન તબક્કાઓ}
\begin{tabulary}{\linewidth}{L L L}
\toprule
\textbf{તબક્કો} & \textbf{પ્રવૃત્તિઓ} & \textbf{મુખ્ય ડિલિવરેબલ્સ} \\
\midrule
\textbf{પ્લાનિંગ} & ઓડિયન્સ સેગમેન્ટેશન, લક્ષ્ય સેટિંગ & ટાર્ગેટ લિસ્ટ્સ, KPIs \\
\textbf{ડિઝાઇન} & ટેમ્પ્લેટ ક્રિએશન, કન્ટેન્ટ લેખન & Email ટેમ્પ્લેટ્સ, કોપી \\
\textbf{એક્ઝિક્યુશન} & સેન્ડ શેડ્યુલિંગ, A/B ટેસ્ટિંગ & કેમ્પેઇન ડિપ્લોયમેન્ટ \\
\textbf{એનાલિટિક્સ} & પરફોર્મન્સ ટ્રેકિંગ, ઓપ્ટિમાઇઝેશન & રિપોર્ટ્સ, ઇનસાઇટ્સ \\
\bottomrule
\end{tabulary}
\end{center}

\textbf{એનાલિટિક્સ માપણ:}
\begin{itemize}
    \item \textbf{Open rates}: સબ્જેક્ટ લાઇન અસરકારકતા, સેન્ડર પ્રતિષ્ઠા
    \item \textbf{Click-through rates}: કન્ટેન્ટ રિલેવન્સ, call-to-action પરફોર્મન્સ
    \item \textbf{Conversion rates}: Landing page અસરકારકતા, ઓફર અપીલ
\end{itemize}

\begin{mnemonicbox}યોજના, ડિઝાઇન, અમલ, વિશ્લેષણ - Email સફળતા\end{mnemonicbox}
\end{solutionbox}

\questionmarks{પ્રશ્ન 5(ક)}{7}{આજના ડિજિટલ લેન્ડસ્કેપમાં સોશિયલ મીડિયા માર્કેટિંગનું મહત્વ ચર્ચાઓ.}
\begin{solutionbox}
સોશિયલ મીડિયા માર્કેટિંગ ડિજિટલ માર્કેટિંગ વ્યૂહરચનાઓનો અનિવાર્ય ઘટક બની ગયું છે, જે બ્રાન્ડ્સ ગ્રાહકો સાથે કેવી રીતે ક્રિયાપ્રતિક્રિયા કરે છે તેમાં મૂળભૂત રીતે પરિવર્તન લાવ્યું છે.

\textbf{વ્યૂહાત્મક મહત્વ:}

\begin{center}
\begin{tikzpicture}[node distance=1.5cm]
    \node (smm) [gtu block] {સોશિયલ મીડિયા માર્કેટિંગ};
    
    \node (brand) [gtu block, below left=of smm] {બ્રાન્ડ જાગૃતિ};
    \node (eng) [gtu block, below=of smm] {ગ્રાહક એંગેજમેન્ટ};
    \node (lead) [gtu block, below right=of smm] {લીડ જનરેશન};
    \node (serv) [gtu block, right=of lead] {ગ્રાહક સેવા};

    \node (growth) [gtu block, below=of eng] {બિઝનેસ વૃદ્ધિ};

    \draw [gtu arrow] (smm) -- (brand);
    \draw [gtu arrow] (smm) -- (eng);
    \draw [gtu arrow] (smm) -- (lead);
    \draw [gtu arrow] (smm) -- (serv);

    \draw [gtu arrow] (brand) -- (growth);
    \draw [gtu arrow] (eng) -- (growth);
    \draw [gtu arrow] (lead) -- (growth);
    \draw [gtu arrow] (serv) -- (growth);
\end{tikzpicture}
\captionof{figure}{સોશિયલ મીડિયા અસર}
\end{center}

\textbf{મુખ્ય મહત્વ ક્ષેત્રો:}

\begin{center}
\captionof{table}{સોશિયલ મીડિયા મૂલ્ય}
\begin{tabulary}{\linewidth}{L L L}
\toprule
\textbf{પાસું} & \textbf{અસર} & \textbf{વ્યાપારિક મૂલ્ય} \\
\midrule
\textbf{વૈશ્વિક પહોંચ} & વિશ્વભરમાં 4.8 બિલિયન વપરાશકર્તાઓ & વિશાળ ઓડિયન્સ સંભાવના \\
\textbf{કોસ્ટ અસરકારકતા} & પરંપરાગત મીડિયા કરતાં ઓછું & ઉચ્ચ ROI તકો \\
\textbf{રીઅલ-ટાઇમ એંગેજમેન્ટ} & તત્કાલ ગ્રાહક ક્રિયાપ્રતિક્રિયા & સુધારેલા સંબંધો \\
\bottomrule
\end{tabulary}
\end{center}

\textbf{પ્લેટફોર્મ-વિશિષ્ટ ફાયદા:}
\begin{itemize}
    \item \textbf{Facebook}: કમ્યુનિટી બિલ્ડિંગ, વિવિધ કન્ટેન્ટ, એડવાન્સ ટાર્ગેટિંગ
    \item \textbf{Instagram}: વિઝ્યુઅલ સ્ટોરીટેલિંગ, ઇન્ફ્લુએન્સર માર્કેટિંગ, શોપિંગ ફીચર્સ
    \item \textbf{LinkedIn}: B2B નેટવર્કિંગ, વિચાર નેતૃત્વ, લીડ જનરેશન
    \item \textbf{YouTube}: વિડિયો માર્કેટિંગ, SEO ફાયદા, શિક્ષણાત્મક કન્ટેન્ટ
\end{itemize}

\begin{mnemonicbox}સોશિયલ મીડિયા આધુનિક માર્કેટિંગને અર્થપૂર્ણ બનાવે છે\end{mnemonicbox}
\end{solutionbox}

\questionmarks{પ્રશ્ન 5(અ) વા}{3}{Google Ads Campaigns ના વિવિધ પ્રકારો કયા છે? દરેકનું સંક્ષિપ્ત વર્ણન આપો.}
\begin{solutionbox}
\textbf{Google Ads કેમ્પેઇન પ્રકારો:}

\begin{center}
\captionof{table}{Google Ads કેમ્પેઇન્સ}
\begin{tabulary}{\linewidth}{L L L}
\toprule
\textbf{કેમ્પેઇન પ્રકાર} & \textbf{હેતુ} & \textbf{પ્લેસમેન્ટ} \\
\midrule
\textbf{Search} & સર્ચ પરિણામોમાં ટેક્સ્ટ જાહેરાત & Google સર્ચ પેજ \\
\textbf{Display} & વેબસાઇટ્સમાં વિઝ્યુઅલ જાહેરાત & Google Display Network \\
\textbf{Video} & વિડિયો જાહેરાતો & YouTube પ્લેટફોર્મ \\
\textbf{Shopping} & પ્રોડક્ટ શોકેસ જાહેરાત & Google Shopping, Search \\
\textbf{App} & મોબાઇલ ઍપ પ્રમોશન & ક્રોસ-પ્લેટફોર્મ પ્લેસમેન્ટ \\
\bottomrule
\end{tabulary}
\end{center}

\begin{itemize}
    \item \textbf{Search}: કીવર્ડ-ટાર્ગેટેડ ટેક્સ્ટ જાહેરાત, ઉચ્ચ ઇન્ટેન્ટ ઓડિયન્સ
    \item \textbf{Display}: બેનર જાહેરાત, વ્યાપક પહોંચ, વિઝ્યુઅલ અપીલ
    \item \textbf{Video}: YouTube જાહેરાત, આકર્ષક કન્ટેન્ટ ફોર્મેટ
\end{itemize}

\begin{mnemonicbox}સર્ચ, ડિસ્પ્લે, વિડિયો, શોપિંગ, ઍપ - Google નો નકશો\end{mnemonicbox}
\end{solutionbox}

\questionmarks{પ્રશ્ન 5(બ) વા}{4}{ધારો કે તમે Google Ads નો ઉપયોગ કરીને Pay-Per-Click (PPC) કેમ્પેઇન સેટ કરી રહ્યા છો. કેમ્પેઇન બનાવવાની પ્રક્રિયા વર્ણવો, તેમાં Google Ads કેમ્પેઇનનો પ્રકાર પસંદ કરવો, ad extensions સેટ કરવું, અને ad performance ને optimize કરવા માટે bidding અને ranking વ્યૂહરચના પસંદ કરવી તે સમાવેશ કરો.}
\begin{solutionbox}
\textbf{PPC કેમ્પેઇન સેટઅપ પ્રક્રિયા:}

\begin{center}
\begin{tikzpicture}[node distance=1.5cm]
    \node (setup) [gtu block] {PPC કેમ્પેઇન સેટઅપ};
    \node (type) [gtu block, below left=of setup] {કેમ્પેઇન પ્રકાર};
    \node (ext) [gtu block, below=of setup] {Ad Extensions};
    \node (bid) [gtu block, below right=of setup] {Bidding વ્યૂહરચના};
    \node (opt) [gtu block, below=of layout] {પરફોર્મન્સ ઓપ્ટિમાઇઝેશન};

    \node (srch) [gtu block, below=of type] {સર્ચ કેમ્પેઇન};
    \node (sitelink) [gtu block, below=of ext] {Sitelink/Call Ext.};
    \node (cpc) [gtu block, below=of bid] {Manual CPC/Target CPA};

    \draw [gtu arrow] (setup) -- (type);
    \draw [gtu arrow] (setup) -- (ext);
    \draw [gtu arrow] (setup) -- (bid);
    
    \draw [gtu arrow] (type) -- (srch);
    \draw [gtu arrow] (ext) -- (sitelink);
    \draw [gtu arrow] (bid) -- (cpc);
\end{tikzpicture}
\captionof{figure}{PPC સેટઅપ પ્રક્રિયા}
\end{center}

\textbf{પગલું-દર-પગલું પ્રક્રિયા:}
\begin{enumerate}
    \item \textbf{કેમ્પેઇન પસંદગી}: સર્ચ કેમ્પેઇન પસંદ કરો
    \item \textbf{Ad Extensions}: Sitelinks, Callouts, Structured Snippets ઉમેરો
    \item \textbf{Bidding સેટઅપ}: Manual CPC અથવા Target CPA પસંદ કરો
    \item \textbf{ઓપ્ટિમાઇઝેશન}: પરફોર્મન્સ મોનિટર કરો
\end{enumerate}

\textbf{પરફોર્મન્સ ઓપ્ટિમાઇઝેશન:}
\begin{itemize}
    \item \textbf{કીવર્ડ સંશોધન}: નેગેટિવ કીવર્ડ્સ, લોંગ-ટેઇલ તકો
    \item \textbf{Ad કોપી ટેસ્ટિંગ}: અનેક વર્ઝન, પરફોર્મન્સ સરખામણી
    \item \textbf{ક્વોલિટી સ્કોર}: રિલેવન્સ, ક્લિક-થ્રુ રેટ, landing page અનુભવ
\end{itemize}

\begin{mnemonicbox}પસંદ કરો, વિસ્તૃત કરો, બિડ કરો, ઓપ્ટિમાઇઝ કરો - PPC સફળતાનો માર્ગ\end{mnemonicbox}
\end{solutionbox}

\questionmarks{પ્રશ્ન 5(ક) વા}{7}{સફળ Facebook Ads વ્યૂહરચનાના મુખ્ય ઘટકોનું વર્ણન કરો.}
\begin{solutionbox}
સફળ Facebook Ads વ્યૂહરચના માટે અનેક પરસ્પર જોડાયેલા ઘટકોમાં સાવચેતીપૂર્વક આયોજન, અમલીકરણ અને ઓપ્ટિમાઇઝેશનની જરૂર છે.

\textbf{વ્યૂહાત્મક ફ્રેમવર્ક:}

\begin{center}
\begin{tikzpicture}[node distance=1.5cm]
    \node (strat) [gtu block] {Facebook Ads વ્યૂહરચના};
    
    \node (target) [gtu block, below left=of strat, xshift=-1cm] {ઓડિયન્સ ટાર્ગેટિંગ};
    \node (create) [gtu block, below left=of strat, xshift=1cm] {ક્રિએટિવ ડેવલપમેન્ટ};
    \node (struct) [gtu block, below right=of strat, xshift=-1cm] {કેમ્પેઇન સ્ટ્રક્ચર};
    \node (opt) [gtu block, below right=of strat, xshift=1cm] {ઓપ્ટિમાઇઝેશન};

    \node (demos) [gtu block, below=of target, font=\footnotesize] {ડેમોગ્રાફિક્સ\\રુચિઓ\\વર્તણૂકો};
    \node (visuals) [gtu block, below=of create, font=\footnotesize] {વિઝ્યુઅલ ડિઝાઇન\\Ad કોપી\\વિડિયો કન્ટેન્ટ};
    \node (objs) [gtu block, below=of struct, font=\footnotesize] {ઉદ્દેશ્યો\\બજેટ\\શેડ્યુલિંગ};
    \node (tests) [gtu block, below=of opt, font=\footnotesize] {A/B ટેસ્ટિંગ\\મોનિટરિંગ};

    \draw [gtu arrow] (strat) -- (target);
    \draw [gtu arrow] (strat) -- (create);
    \draw [gtu arrow] (strat) -- (struct);
    \draw [gtu arrow] (strat) -- (opt);

    \draw [gtu arrow] (target) -- (demos);
    \draw [gtu arrow] (create) -- (visuals);
    \draw [gtu arrow] (struct) -- (objs);
    \draw [gtu arrow] (opt) -- (tests);
\end{tikzpicture}
\captionof{figure}{Facebook Ads વ્યૂહરચના}
\end{center}

\textbf{મુખ્ય વ્યૂહરચના ઘટકો:}

\begin{center}
\captionof{table}{વ્યૂહરચના ઘટકો}
\begin{tabulary}{\linewidth}{L L L}
\toprule
\textbf{ઘટક} & \textbf{એલિમેન્ટ્સ} & \textbf{સફળતા પરિબળો} \\
\midrule
\textbf{ઓડિયન્સ ટાર્ગેટિંગ} & ડેમોગ્રાફિક્સ, રુચિઓ, વર્તણૂકો & ચોક્કસ ટાર્ગેટિંગ, સંબંધિત પહોંચ \\
\textbf{ક્રિએટિવ એક્સેલન્સ} & વિઝ્યુઅલ્સ, કોપી, વિડિયો કન્ટેન્ટ & એંગેજમેન્ટ, બ્રાન્ડ સુસંગતતા \\
\textbf{કેમ્પેઇન સ્ટ્રક્ચર} & ઉદ્દેશ્યો, બજેટ, શેડ્યુલિંગ & સ્પષ્ટ લક્ષ્યો, કાર્યક્ષમ ખર્ચ \\
\textbf{ઓપ્ટિમાઇઝેશન} & ટેસ્ટિંગ, મોનિટરિંગ, એડજસ્ટમેન્ટ્સ & ડેટા-સંચાલિત નિર્ણયો \\
\bottomrule
\end{tabulary}
\end{center}

\begin{itemize}
    \item \textbf{ઓડિયન્સ ટાર્ગેટિંગ}: Core, Custom, અને Lookalike audiences
    \item \textbf{ક્રિએટિવ ડેવલપમેન્ટ}: ઉચ્ચ-ગુણવત્તાની ઇમેજ, વિડિયો, આકર્ષક ad કોપી
    \item \textbf{માપણ}: ROI, ROAS, અને Customer Lifetime Value
\end{itemize}

\begin{mnemonicbox}ચોક્કસ ટાર્ગેટ, આકર્ષક બનાવો, વ્યૂહાત્મક રચના, સતત ઓપ્ટિમાઇઝ કરો\end{mnemonicbox}
\end{solutionbox}

\end{document}
