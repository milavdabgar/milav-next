\documentclass{article}

% content/resources/templates/preamble.tex
\usepackage[margin=0.6in]{geometry}
\author{Milav Dabgar}
\usepackage{amsmath,amssymb,amsthm}
\usepackage{booktabs}
\usepackage{multirow}
\usepackage{xcolor}
\usepackage{tcolorbox}
\tcbuselibrary{breakable,skins}
\usepackage[colorlinks=true,linkcolor=blue]{hyperref}
\usepackage{titlesec}
\usepackage{enumitem}
\usepackage{tikz}
\usepackage{pgfplots}
\usepackage{circuitikz}
\usepackage[version=4]{mhchem}
\usepackage{longtable}
\usepackage{array}
\usepackage{float}
\usepackage{caption}
\usepackage{listings}

\lstset{
  basicstyle=\small\ttfamily,
  breaklines=true,
  breakatwhitespace=false,
  postbreak=\mbox{\textcolor{red}{$\hookrightarrow$}\space},
  float=false,
  numbers=left,
  numberstyle=\tiny\color{gray},
  numbersep=10pt,
  xleftmargin=2em,
  keywordstyle=\color{blue},
  commentstyle=\color{green!60!black},
  stringstyle=\color{purple},
  backgroundcolor=\color{gray!5},
  showstringspaces=false,
  tabsize=2,
  captionpos=b,
  keepspaces=true,
  columns=flexible
}

\pgfplotsset{compat=1.18}
\usetikzlibrary{shapes,arrows,positioning,calc,patterns,decorations.pathmorphing,decorations.markings,arrows.meta}

% Color scheme
\definecolor{headcolor}{RGB}{0,102,204}
\definecolor{keycolor}{RGB}{220,20,60}
\definecolor{solutioncolor}{RGB}{34,139,34}
\definecolor{mnemoniccolor}{RGB}{148,0,211}
\definecolor{codecolor}{RGB}{0,0,100}

% Spacing
\setlength{\parskip}{3pt}
\setlist[itemize]{nosep}
\setlist[enumerate]{nosep}

% Title formatting
\titleformat{\section}{\Large\bfseries\color{headcolor}}{\thesection}{1em}{}
\titleformat{\subsection}{\large\bfseries\color{headcolor}}{\thesubsection}{1em}{}

% Pandoc tightlist compatibility
\providecommand{\tightlist}{%
  \setlength{\itemsep}{0pt}\setlength{\parskip}{0pt}}

% Pandoc longtable compatibility
\newcounter{none}
\def\thenone{}


% content/resources/templates/english-boxes.tex

% Custom environments
\newtcolorbox{solutionbox}{
 breakable,
 enhanced,
 colback=solutioncolor!5!white,
 colframe=solutioncolor!75!black,
 fonttitle=\bfseries,
 title=Solution
}

\newtcolorbox{solutionboxnobreak}{
 colback=solutioncolor!5!white,
 colframe=solutioncolor!75!black,
 fonttitle=\bfseries,
 title=Solution
}

\newtcolorbox{keyformula}{
 breakable,
 enhanced,
 colback=keycolor!5!white,
 colframe=keycolor!75!black,
 fonttitle=\bfseries,
 title=Key Formula
}

\newtcolorbox{mnemonicboxenv}{
 breakable,
 enhanced,
 colback=mnemoniccolor!5!white,
 colframe=mnemoniccolor!75!black,
 fonttitle=\bfseries,
 title=Mnemonic
}

\newcommand{\mnemonicbox}[1]{%
  \begin{mnemonicboxenv}
    #1
  \end{mnemonicboxenv}
}


% Custom commands for GTU solutions
% This file defines semantic commands for consistent formatting

% Question command with automatic formatting
\newcommand{\question}[2]{%
  \section*{Question #1}%
  \textbf{#2}%
}

% OR question variant
\newcommand{\questionor}[2]{%
  \section*{Question #1 OR}%
  \textbf{#2}%
}

% Proper table environment with caption
\newenvironment{answertable}[1]{%
  \begin{table}[htbp]
  \centering
  \caption{#1}
}{%
  \end{table}
}

% Proper figure environment for diagrams
\newenvironment{answerdiagram}[1]{%
  \begin{figure}[htbp]
  \centering
  \caption{#1}
}{%
  \end{figure}
}

% Semantic markup for key terms
\newcommand{\keyword}[1]{\textbf{#1}}
\newcommand{\code}[1]{\texttt{#1}}
\newcommand{\classname}[1]{\texttt{#1}}
\newcommand{\methodname}[1]{\texttt{#1}}

% Proper quotation marks
\newcommand{\mnemonic}[1]{``#1''}


\title{Essentials of Digital Marketing (4341601) - Summer 2024 Solution}
\date{June 11, 2024}

\begin{document}
\maketitle

\questionmarks{1(a)}{3}{Differentiate: Traditional marketing and Digital marketing.}

\begin{solutionbox}
\begin{center}
\captionof{table}{Traditional vs Digital Marketing}
\begin{tabulary}{\linewidth}{|L|L|}
\hline
\textbf{Traditional Marketing} & \textbf{Digital Marketing} \\ \hline
\textbf{Physical presence} required & \textbf{Online presence} through internet \\ \hline
\textbf{Limited reach} to local audience & \textbf{Global reach} to worldwide audience \\ \hline
\textbf{One-way communication} & \textbf{Two-way interactive} communication \\ \hline
\textbf{High cost} for advertising & \textbf{Cost-effective} campaigns \\ \hline
\textbf{Difficult to measure} ROI & \textbf{Easy tracking} and analytics \\ \hline
\textbf{Slow feedback} from customers & \textbf{Instant feedback} and responses \\ \hline
\end{tabulary}
\end{center}
\end{solutionbox}

\begin{mnemonicbox}
\mnemonic{PITCH vs CLICK: Physical vs Interactive, Traditional vs Trackable, High-cost vs Cost-effective}
\end{mnemonicbox}

\questionmarks{1(b)}{4}{Explain working of search engine algorithm.}

\begin{solutionbox}
Search engine algorithms work through systematic processes to deliver relevant results:

\begin{center}
\begin{tikzpicture}[node distance=1.5cm, auto]
    \node [gtu block] (Crawl) {Web Crawling};
    \node [gtu block, right=1cm of Crawl] (Index) {Indexing};
    \node [gtu block, right=1cm of Index] (Query) {Query Processing};
    \node [gtu block, below=1cm of Crawl] (Rank) {Ranking Algorithm};
    \node [gtu state, right=1cm of Rank] (SERP) {SERP Display};

    \path [gtu arrow] (Crawl) -- (Index);
    \path [gtu arrow] (Index) -- (Query);
    \path [gtu arrow] (Query) -- (Rank);
    \path [gtu arrow] (Rank) -- (SERP);
\end{tikzpicture}
\captionof{figure}{Search Engine Process Flow}
\end{center}

\begin{itemize}
    \item \keyword{Crawling}: Search bots scan websites continuously to discover new content
    \item \keyword{Indexing}: Analyzed content is stored in massive databases with keywords
    \item \keyword{Query matching}: User search terms are matched with indexed content
    \item \keyword{Ranking factors}: Content relevance, authority, and user experience determine position
\end{itemize}
\end{solutionbox}

\begin{mnemonicbox}
\mnemonic{CIRR: Crawl, Index, Rank, Results}
\end{mnemonicbox}

\questionmarks{1(c)}{7}{Explain the key components of a digital marketing plan.}

\begin{solutionbox}
A comprehensive digital marketing plan includes essential components for success:

\begin{center}
\captionof{table}{Digital Marketing Plan Components}
\begin{tabulary}{\linewidth}{|L|L|L|}
\hline
\textbf{Component} & \textbf{Description} & \textbf{Purpose} \\ \hline
\textbf{Situation Analysis} & Current market position and SWOT & Understanding starting point \\ \hline
\textbf{Target Audience} & Demographics and buyer personas & Focused marketing efforts \\ \hline
\textbf{Goals \& Objectives} & SMART goals with KPIs & Measurable outcomes \\ \hline
\textbf{Strategy Selection} & SEO, SEM, Social Media, Email & Channel optimization \\ \hline
\textbf{Budget Allocation} & Resource distribution across channels & Cost management \\ \hline
\textbf{Content Calendar} & Scheduled content publication & Consistent engagement \\ \hline
\textbf{Analytics Setup} & Tracking tools and metrics & Performance monitoring \\ \hline
\end{tabulary}
\end{center}

\textbf{Key Success Factors:}

\begin{itemize}
    \item \keyword{Research-driven} approach with market insights
    \item \keyword{Integration} across multiple digital channels
    \item \keyword{Flexibility} to adapt based on performance data
\end{itemize}
\end{solutionbox}

\begin{mnemonicbox}
\mnemonic{STGSBC Analytics: Situation, Target, Goals, Strategy, Budget, Content, Analytics}
\end{mnemonicbox}

\questionmarks{1(c) OR}{7}{Explain the components of the P.O.E.M. framework and their relevance in digital marketing.}

\begin{solutionbox}
P.O.E.M. framework categorizes digital marketing channels for strategic planning:

\begin{center}
\begin{tikzpicture}[node distance=2cm, auto]
    \node [gtu state] (POEM) {P.O.E.M.};
    \node [gtu block, above left=1cm and 0.5cm of POEM] (Paid) {Paid Media};
    \node [gtu block, above right=1cm and 0.5cm of POEM] (Owned) {Owned Media};
    \node [gtu block, below left=1cm and 0.5cm of POEM] (Earned) {Earned Media};
    \node [gtu block, below right=1cm and 0.5cm of POEM] (Managed) {Managed Media};

    \path [gtu arrow] (POEM) -- (Paid);
    \path [gtu arrow] (POEM) -- (Owned);
    \path [gtu arrow] (POEM) -- (Earned);
    \path [gtu arrow] (POEM) -- (Managed);
\end{tikzpicture}
\captionof{figure}{P.O.E.M. Framework}
\end{center}

\begin{center}
\captionof{table}{P.O.E.M. Components}
\begin{tabulary}{\linewidth}{|L|L|L|L|}
\hline
\textbf{Component} & \textbf{Definition} & \textbf{Examples} & \textbf{Relevance} \\ \hline
\textbf{Paid Media} & Purchased advertising space & Google Ads, Facebook Ads & \textbf{Immediate reach} and control \\ \hline
\textbf{Owned Media} & Brand-controlled channels & Website, email lists, blogs & \textbf{Long-term asset} building \\ \hline
\textbf{Earned Media} & Third-party endorsements & Reviews, shares, mentions & \textbf{Credibility} and trust \\ \hline
\textbf{Managed Media} & Influenced but not owned & Influencer partnerships & \textbf{Extended reach} through others \\ \hline
\end{tabulary}
\end{center}

\textbf{Strategic Benefits:}

\begin{itemize}
    \item \keyword{Balanced approach} across all media types
    \item \keyword{Cost optimization} through channel mix
    \item \keyword{Amplified impact} when channels work together
\end{itemize}
\end{solutionbox}

\begin{mnemonicbox}
\mnemonic{POEM builds Digital SUCCESS: Paid, Owned, Earned, Managed}
\end{mnemonicbox}

\questionmarks{2(a)}{3}{Describe need of SEO.}

\begin{solutionbox}
SEO is essential for online visibility and business growth:

\begin{itemize}
    \item \keyword{Organic traffic}: 68\% of online experiences begin with search engines
    \item \keyword{Cost-effective}: No direct payment for organic rankings unlike paid ads
    \item \keyword{Trust building}: Higher rankings create credibility with users
    \item \keyword{Long-term results}: Sustainable traffic growth over time
\end{itemize}
\end{solutionbox}

\begin{mnemonicbox}
\mnemonic{OCTL: Organic, Cost-effective, Trust, Long-term}
\end{mnemonicbox}

\questionmarks{2(b)}{4}{Differentiate between on-page and off-page optimization.}

\begin{solutionbox}
\begin{center}
\captionof{table}{On-Page vs Off-Page SEO}
\begin{tabulary}{\linewidth}{|L|L|}
\hline
\textbf{On-Page SEO} & \textbf{Off-Page SEO} \\ \hline
\textbf{Website elements} optimization & \textbf{External factors} optimization \\ \hline
\textbf{Title tags, meta descriptions} & \textbf{Backlinks from other sites} \\ \hline
\textbf{Content quality and keywords} & \textbf{Social media signals} \\ \hline
\textbf{Internal linking structure} & \textbf{Domain authority building} \\ \hline
\textbf{Complete control} by website owner & \textbf{Limited control}, depends on others \\ \hline
\textbf{Technical optimization} focus & \textbf{Authority and popularity} focus \\ \hline
\end{tabulary}
\end{center}
\end{solutionbox}

\begin{mnemonicbox}
\mnemonic{IN vs OUT: Internal optimization vs Outbound authority}
\end{mnemonicbox}

\questionmarks{2(c)}{7}{Explain SEO ranking and ways to improve SEO ranking.}

\begin{solutionbox}
SEO ranking determines website position in search engine results pages (SERPs).

\begin{center}
\captionof{table}{Ranking Factors}
\begin{tabulary}{\linewidth}{|L|L|L|}
\hline
\textbf{Factor Category} & \textbf{Techniques} & \textbf{Impact Level} \\ \hline
\textbf{Content Quality} & Original, valuable content & High \\ \hline
\textbf{Keywords} & Research and natural placement & High \\ \hline
\textbf{Technical SEO} & Site speed, mobile-friendly & Medium \\ \hline
\textbf{Backlinks} & Quality link building & High \\ \hline
\textbf{User Experience} & Low bounce rate, high engagement & Medium \\ \hline
\end{tabulary}
\end{center}

\textbf{Improvement Strategies:}

\begin{itemize}
    \item \keyword{Content optimization}: Create comprehensive, user-focused content
    \item \keyword{Keyword research}: Target relevant, achievable keywords
    \item \keyword{Technical fixes}: Improve site speed and mobile responsiveness
    \item \keyword{Link building}: Earn quality backlinks from authoritative sites
    \item \keyword{User signals}: Enhance engagement metrics
\end{itemize}

\textbf{Success Metrics:}

\begin{itemize}
    \item \keyword{SERP position} improvements
    \item \keyword{Organic traffic} growth
    \item \keyword{Click-through rates} increase
\end{itemize}
\end{solutionbox}

\begin{mnemonicbox}
\mnemonic{CKTU for SEO SUCCESS: Content, Keywords, Technical, User-experience}
\end{mnemonicbox}

\questionmarks{2(a) OR}{3}{Define: 1. Backlinks 2. Website Speed 3. Keyword stuffing.}

\begin{solutionbox}
\begin{center}
\captionof{table}{SEO Definitions}
\begin{tabulary}{\linewidth}{|L|L|}
\hline
\textbf{Term} & \textbf{Definition} \\ \hline
\textbf{Backlinks} & Incoming hyperlinks from external websites pointing to your site \\ \hline
\textbf{Website Speed} & Time taken for web pages to load completely in browser \\ \hline
\textbf{Keyword Stuffing} & Overuse of keywords unnaturally in content to manipulate rankings \\ \hline
\end{tabulary}
\end{center}
\end{solutionbox}

\begin{mnemonicbox}
\mnemonic{BWK: Backlinks, Website speed, Keyword stuffing}
\end{mnemonicbox}

\questionmarks{2(b) OR}{4}{Differentiate between Black Hat and White Hat SEO Techniques.}

\begin{solutionbox}
\begin{center}
\captionof{table}{White Hat vs Black Hat SEO}
\begin{tabulary}{\linewidth}{|L|L|}
\hline
\textbf{White Hat SEO} & \textbf{Black Hat SEO} \\ \hline
\textbf{Ethical practices} following guidelines & \textbf{Manipulative tactics} violating rules \\ \hline
\textbf{Quality content} creation & \textbf{Content scraping} and duplication \\ \hline
\textbf{Natural link building} & \textbf{Link farms} and paid links \\ \hline
\textbf{Long-term results} & \textbf{Quick but risky} gains \\ \hline
\textbf{Search engine approved} & \textbf{Penalty risks} from search engines \\ \hline
\end{tabulary}
\end{center}
\end{solutionbox}

\begin{mnemonicbox}
\mnemonic{GOOD vs BAD: Guidelines-following vs Penalty-risking}
\end{mnemonicbox}

\questionmarks{2(c) OR}{7}{Give name of any three common SEO tools and describe their functions.}

\begin{solutionbox}
\begin{center}
\captionof{table}{Common SEO Tools}
\begin{tabulary}{\linewidth}{|L|L|L|}
\hline
\textbf{SEO Tool} & \textbf{Primary Functions} & \textbf{Key Features} \\ \hline
\textbf{Google Analytics} & Website traffic analysis & Visitor behavior, conversion tracking, audience insights \\ \hline
\textbf{SEMrush} & Keyword research and competitor analysis & Keyword difficulty, backlink analysis, site audit \\ \hline
\textbf{Yoast SEO} & On-page optimization (WordPress) & Content optimization, technical SEO, readability analysis \\ \hline
\end{tabulary}
\end{center}

\textbf{Detailed Functions:}

\begin{itemize}
    \item \keyword{Google Analytics}: Tracks user journey, bounce rates, and goal completions
    \item \keyword{SEMrush}: Identifies ranking opportunities and monitors competitor strategies
    \item \keyword{Yoast SEO}: Provides real-time optimization suggestions for content and meta tags
\end{itemize}

\textbf{Benefits:}

\begin{itemize}
    \item \keyword{Data-driven decisions} through comprehensive analytics
    \item \keyword{Competitive advantage} with market insights
    \item \keyword{Efficiency} in optimization tasks
\end{itemize}
\end{solutionbox}

\begin{mnemonicbox}
\mnemonic{GSY Tools: Google Analytics, SEMrush, Yoast}
\end{mnemonicbox}

\questionmarks{3(a)}{3}{Explain any one Multi-touch attribution model with example.}

\begin{solutionbox}
\textbf{Linear Attribution Model} distributes credit equally across all touchpoints in customer journey.

\textbf{Example Scenario:}
Customer journey: Social Media Ad $\rightarrow$ Email $\rightarrow$ Website Visit $\rightarrow$ Purchase

\textbf{Credit Distribution:}

\begin{itemize}
    \item Social Media Ad: 25\%
    \item Email: 25\%
    \item Website Visit: 25\%
    \item Purchase Page: 25\%
\end{itemize}
\end{solutionbox}

\begin{mnemonicbox}
\mnemonic{EQUAL Credit for ALL Touches: Linear = Equal distribution}
\end{mnemonicbox}

\questionmarks{3(b)}{4}{Explain following Key metrics: Unique Visitors, Bounce Rate.}

\begin{solutionbox}
\begin{center}
\captionof{table}{Key Web Metrics}
\begin{tabulary}{\linewidth}{|L|L|L|}
\hline
\textbf{Metric} & \textbf{Definition} & \textbf{Significance} \\ \hline
\textbf{Unique Visitors} & Count of individual users visiting website in specific period & Measures \textbf{audience reach} and growth \\ \hline
\textbf{Bounce Rate} & Percentage of visitors leaving after viewing only one page & Indicates \textbf{content relevance} and engagement \\ \hline
\end{tabulary}
\end{center}

\textbf{Optimization Tips:}

\begin{itemize}
    \item \keyword{Unique Visitors}: Increase through SEO and social media
    \item \keyword{Bounce Rate}: Improve with better content and site navigation
\end{itemize}
\end{solutionbox}

\begin{mnemonicbox}
\mnemonic{UV-BR: Unique Visitors measure reach, Bounce Rate measures engagement}
\end{mnemonicbox}

\questionmarks{3(c)}{7}{Explain following tracking code with their advantage and disadvantage: Long tracking code, UTM code.}

\begin{solutionbox}
\begin{center}
\captionof{table}{Tracking Code Comparison}
\begin{tabulary}{\linewidth}{|L|L|L|L|}
\hline
\textbf{Tracking Code Type} & \textbf{Description} & \textbf{Advantages} & \textbf{Disadvantages} \\ \hline
\textbf{Long Tracking Code} & Detailed parameters with extensive information & \textbf{Comprehensive data} collection, \textbf{Detailed insights} & \textbf{Complex URLs}, \textbf{User-unfriendly} appearance \\ \hline
\textbf{UTM Code} & Urchin Tracking Module parameters for campaign tracking & \textbf{Simple implementation}, \textbf{Campaign-specific} tracking & \textbf{Limited data}, \textbf{Manual management} required \\ \hline
\end{tabulary}
\end{center}

\textbf{UTM Parameters:}

\begin{itemize}
    \item \code{utm\_source}: Traffic source (google, facebook)
    \item \code{utm\_medium}: Marketing medium (cpc, email)
    \item \code{utm\_campaign}: Campaign name (summer\_sale)
\end{itemize}

\textbf{Best Practices:}

\begin{itemize}
    \item \keyword{Consistent naming} conventions
    \item \keyword{URL shortening} for long tracking codes
    \item \keyword{Regular monitoring} of campaign performance
\end{itemize}
\end{solutionbox}

\begin{mnemonicbox}
\mnemonic{LONG vs SHORT: Comprehensive vs Simple tracking}
\end{mnemonicbox}

\questionmarks{3(a) OR}{3}{Explain any one Single-touch attribution model with example.}

\begin{solutionbox}
\textbf{Last-Click Attribution Model} gives 100\% credit to the final touchpoint before conversion.

\textbf{Example Scenario:}
Customer journey: Social Media $\rightarrow$ Email $\rightarrow$ Google Search $\rightarrow$ Purchase

\textbf{Credit Distribution:}

\begin{itemize}
    \item Google Search: 100\%
    \item Other touchpoints: 0\%
\end{itemize}

\textbf{Use Case}: Simple e-commerce tracking focusing on immediate conversion drivers
\end{solutionbox}

\begin{mnemonicbox}
\mnemonic{LAST WINS ALL: Final touchpoint gets full credit}
\end{mnemonicbox}

\questionmarks{3(b) OR}{4}{Explain following Key metrics: Pageviews, New Visits.}

\begin{solutionbox}
\begin{center}
\captionof{table}{Metrics Analysis}
\begin{tabulary}{\linewidth}{|L|L|L|}
\hline
\textbf{Metric} & \textbf{Definition} & \textbf{Measurement Value} \\ \hline
\textbf{Pageviews} & Total number of pages viewed including repeat views & \textbf{Content popularity} and site usage \\ \hline
\textbf{New Visits} & Percentage of first-time visitors to website & \textbf{Audience growth} and reach expansion \\ \hline
\end{tabulary}
\end{center}

\textbf{Analysis Importance:}

\begin{itemize}
    \item \keyword{Pageviews}: Higher numbers indicate engaging content
    \item \keyword{New Visits}: Growth shows effective marketing outreach
\end{itemize}
\end{solutionbox}

\begin{mnemonicbox}
\mnemonic{PN Metrics: Pageviews for engagement, New visits for growth}
\end{mnemonicbox}

\questionmarks{3(c) OR}{7}{Describe various types of web analytics tool.}

\begin{solutionbox}
\begin{center}
\captionof{table}{Web Analytics Tools}
\begin{tabulary}{\linewidth}{|L|L|L|L|}
\hline
\textbf{Tool Category} & \textbf{Purpose} & \textbf{Examples} & \textbf{Key Features} \\ \hline
\textbf{Content Analytics} & Content performance tracking & Google Analytics, Adobe Analytics & Page views, time on page, bounce rate \\ \hline
\textbf{Customer Analytics} & User behavior analysis & Hotjar, Crazy Egg & Heatmaps, session recordings \\ \hline
\textbf{SEO Analytics} & Search optimization & SEMrush, Ahrefs & Keyword rankings, backlink analysis \\ \hline
\textbf{Social Media Analytics} & Social performance & Facebook Insights, Twitter Analytics & Engagement rates, reach metrics \\ \hline
\textbf{A/B Testing Tools} & Conversion optimization & Optimizely, VWO & Split testing, conversion tracking \\ \hline
\end{tabulary}
\end{center}

\textbf{Selection Criteria:}

\begin{itemize}
    \item \keyword{Business objectives} alignment
    \item \keyword{Integration capabilities} with existing tools
    \item \keyword{Cost-effectiveness} for organization size
\end{itemize}

\textbf{Implementation Benefits:}

\begin{itemize}
    \item \keyword{Data-driven decisions} for marketing strategy
    \item \keyword{ROI measurement} across channels
    \item \keyword{User experience} optimization
\end{itemize}
\end{solutionbox}

\begin{mnemonicbox}
\mnemonic{CCSSA Analytics: Content, Customer, SEO, Social, A/B testing}
\end{mnemonicbox}

\questionmarks{4(a)}{3}{Explain Social Media Marketing.}

\begin{solutionbox}
Social Media Marketing uses social platforms to promote products and engage audiences.

\textbf{Core Elements:}

\begin{itemize}
    \item \keyword{Content creation} for target audience engagement
    \item \keyword{Community building} through consistent interaction
    \item \keyword{Brand awareness} through organic and paid strategies
    \item \keyword{Customer support} via social channels
\end{itemize}
\end{solutionbox}

\begin{mnemonicbox}
\mnemonic{CCBC: Content, Community, Brand awareness, Customer support}
\end{mnemonicbox}

\questionmarks{4(b)}{4}{Explain types of Instagram Ads.}

\begin{solutionbox}
\begin{center}
\captionof{table}{Instagram Ad Types}
\begin{tabulary}{\linewidth}{|L|L|L|}
\hline
\textbf{Ad Type} & \textbf{Format} & \textbf{Best Use Case} \\ \hline
\textbf{Photo Ads} & Single image with caption & \textbf{Product showcasing} and brand awareness \\ \hline
\textbf{Video Ads} & Short video content & \textbf{Storytelling} and engagement \\ \hline
\textbf{Carousel Ads} & Multiple images/videos & \textbf{Product catalogs} and features \\ \hline
\textbf{Stories Ads} & Full-screen vertical format & \textbf{Immediate action} and urgency \\ \hline
\end{tabulary}
\end{center}
\end{solutionbox}

\begin{mnemonicbox}
\mnemonic{PVCS Instagram: Photo, Video, Carousel, Stories}
\end{mnemonicbox}

\questionmarks{4(c)}{7}{Explain YouTube Marketing and its importance in digital marketing.}

\begin{solutionbox}
YouTube Marketing leverages video content for brand promotion and audience engagement.

\begin{center}
\begin{tikzpicture}[node distance=1.5cm, auto]
    \node [gtu block] (YT) {YouTube Marketing};
    \node [gtu block, above right=1cm and 1cm of YT] (Content) {Content Strategy};
    \node [gtu block, below right=1cm and 1cm of YT] (Channel) {Channel Optimization};
    \node [gtu block, below left=1cm and 1cm of YT] (SEO) {Video SEO};
    \node [gtu block, above left=1cm and 1cm of YT] (Analytics) {Analytics \& Growth};

    \path [gtu arrow] (YT) -- (Content);
    \path [gtu arrow] (YT) -- (Channel);
    \path [gtu arrow] (YT) -- (SEO);
    \path [gtu arrow] (YT) -- (Analytics);
\end{tikzpicture}
\captionof{figure}{YouTube Marketing Components}
\end{center}

\begin{center}
\captionof{table}{YouTube Marketing Strategies}
\begin{tabulary}{\linewidth}{|L|L|L|}
\hline
\textbf{Component} & \textbf{Strategy} & \textbf{Importance} \\ \hline
\textbf{Content Strategy} & Educational, entertaining videos & \textbf{Audience engagement} and value delivery \\ \hline
\textbf{Channel Optimization} & Branding, playlists, descriptions & \textbf{Professional presence} and discoverability \\ \hline
\textbf{Video SEO} & Keywords, thumbnails, titles & \textbf{Search visibility} and organic reach \\ \hline
\textbf{YouTube Ads} & TrueView, bumper ads & \textbf{Targeted promotion} and quick results \\ \hline
\end{tabulary}
\end{center}

\textbf{Digital Marketing Importance:}

\begin{itemize}
    \item \keyword{Visual storytelling} builds emotional connections
    \item \keyword{Search engine benefits} (YouTube is 2nd largest search engine)
    \item \keyword{Cross-platform integration} with other marketing channels
    \item \keyword{Cost-effective} compared to traditional video advertising
\end{itemize}

\textbf{Success Metrics:}

\begin{itemize}
    \item \keyword{Watch time} and retention rates
    \item \keyword{Subscriber growth} and engagement
    \item \keyword{Conversion tracking} from video to website
\end{itemize}
\end{solutionbox}

\begin{mnemonicbox}
\mnemonic{CCVA Success: Content, Channel, Video SEO, Ads for YouTube success}
\end{mnemonicbox}

\questionmarks{4(a) OR}{3}{List the metrics available on Instagram for tracking the success of marketing strategies.}

\begin{solutionbox}
\textbf{Instagram Analytics Metrics:}

\begin{itemize}
    \item \keyword{Reach}: Number of unique accounts seeing content
    \item \keyword{Impressions}: Total content views including repeats
    \item \keyword{Engagement Rate}: Likes, comments, shares percentage
    \item \keyword{Profile Visits}: Traffic to Instagram business profile
    \item \keyword{Website Clicks}: Traffic driven to external website
    \item \keyword{Story Completion Rate}: Percentage viewing full stories
\end{itemize}
\end{solutionbox}

\begin{mnemonicbox}
\mnemonic{RIEPSW: Reach, Impressions, Engagement, Profile visits, Story completion, Website clicks}
\end{mnemonicbox}

\questionmarks{4(b) OR}{4}{Explain types of YouTube Ads.}

\begin{solutionbox}
\begin{center}
\captionof{table}{YouTube Ad Types}
\begin{tabulary}{\linewidth}{|L|L|L|L|}
\hline
\textbf{YouTube Ad Type} & \textbf{Format} & \textbf{Duration} & \textbf{Best For} \\ \hline
\textbf{TrueView In-Stream} & Skippable video ads & 12 seconds+ & \textbf{Brand awareness} campaigns \\ \hline
\textbf{TrueView Discovery} & Thumbnail + text & Variable & \textbf{Content promotion} \\ \hline
\textbf{Bumper Ads} & Non-skippable short ads & 6 seconds & \textbf{Quick messaging} \\ \hline
\textbf{Overlay Ads} & Banner on video & Static & \textbf{Website traffic} \\ \hline
\end{tabulary}
\end{center}
\end{solutionbox}

\begin{mnemonicbox}
\mnemonic{TTBO YouTube: TrueView In-stream, TrueView Discovery, Bumper, Overlay}
\end{mnemonicbox}

\questionmarks{4(c) OR}{7}{Describe the targeting options available in Facebook advertising.}

\begin{solutionbox}
Facebook offers comprehensive targeting for precise audience reach:

\begin{center}
\captionof{table}{Facebook Targeting Options}
\begin{tabulary}{\linewidth}{|L|L|L|}
\hline
\textbf{Targeting Category} & \textbf{Options} & \textbf{Purpose} \\ \hline
\textbf{Demographics} & Age, gender, education, income & \textbf{Basic audience} definition \\ \hline
\textbf{Location} & Countries, cities, radius & \textbf{Geographic} targeting \\ \hline
\textbf{Interests} & Hobbies, pages liked, activities & \textbf{Behavioral} targeting \\ \hline
\textbf{Behaviors} & Purchase history, device usage & \textbf{Action-based} targeting \\ \hline
\textbf{Custom Audiences} & Website visitors, email lists & \textbf{Retargeting} existing contacts \\ \hline
\textbf{Lookalike Audiences} & Similar to existing customers & \textbf{Audience expansion} \\ \hline
\end{tabulary}
\end{center}

\textbf{Advanced Features:}

\begin{itemize}
    \item \keyword{Detailed targeting} with inclusion/exclusion options
    \item \keyword{Audience insights} for optimization
    \item \keyword{A/B testing} different audience segments
\end{itemize}

\textbf{Campaign Optimization:}

\begin{itemize}
    \item \keyword{Narrow targeting} for specific products
    \item \keyword{Broad targeting} for brand awareness
    \item \keyword{Dynamic audiences} based on user behavior
\end{itemize}

\textbf{Performance Benefits:}

\begin{itemize}
    \item \keyword{Higher conversion} rates through precision
    \item \keyword{Cost efficiency} with relevant audiences
    \item \keyword{Scalable growth} through lookalike expansion
\end{itemize}
\end{solutionbox}

\begin{mnemonicbox}
\mnemonic{DLIBCL Targeting: Demographics, Location, Interests, Behaviors, Custom, Lookalike}
\end{mnemonicbox}

\questionmarks{5(a)}{3}{Explain the concept of LinkedIn marketing.}

\begin{solutionbox}
LinkedIn Marketing focuses on professional networking and B2B relationship building.

\textbf{Key Concepts:}

\begin{itemize}
    \item \keyword{Professional audience} targeting for B2B sales
    \item \keyword{Thought leadership} through industry content
    \item \keyword{Network expansion} via connections and groups
    \item \keyword{Lead generation} through targeted campaigns
\end{itemize}
\end{solutionbox}

\begin{mnemonicbox}
\mnemonic{PTNL: Professional, Thought leadership, Network, Leads}
\end{mnemonicbox}

\questionmarks{5(b)}{4}{Explain different types of email marketing campaigns.}

\begin{solutionbox}
\begin{center}
\captionof{table}{Email Marketing Campaigns}
\begin{tabulary}{\linewidth}{|L|L|L|}
\hline
\textbf{Campaign Type} & \textbf{Purpose} & \textbf{Timing} \\ \hline
\textbf{Welcome Series} & New subscriber onboarding & \textbf{Immediate} after signup \\ \hline
\textbf{Newsletter} & Regular content updates & \textbf{Weekly/Monthly} schedule \\ \hline
\textbf{Promotional} & Sales and special offers & \textbf{Event-based} campaigns \\ \hline
\textbf{Abandoned Cart} & Recovery incomplete purchases & \textbf{24-48 hours} after abandonment \\ \hline
\end{tabulary}
\end{center}
\end{solutionbox}

\begin{mnemonicbox}
\mnemonic{WNPA Emails: Welcome, Newsletter, Promotional, Abandoned cart}
\end{mnemonicbox}

\questionmarks{5(c)}{7}{Explain the different types of Google Ads Campaigns.}

\begin{solutionbox}
Google Ads offers multiple campaign types for different marketing objectives:

\begin{center}
\captionof{table}{Google Ads Campaign Types}
\begin{tabulary}{\linewidth}{|L|L|L|L|}
\hline
\textbf{Campaign Type} & \textbf{Platform} & \textbf{Ad Format} & \textbf{Best For} \\ \hline
\textbf{Search Campaigns} & Google Search & Text ads & \textbf{High-intent} keyword targeting \\ \hline
\textbf{Display Campaigns} & Partner websites & Banner/image ads & \textbf{Brand awareness} and retargeting \\ \hline
\textbf{Video Campaigns} & YouTube & Video ads & \textbf{Engagement} and storytelling \\ \hline
\textbf{Shopping Campaigns} & Google Shopping & Product listings & \textbf{E-commerce} sales \\ \hline
\textbf{App Campaigns} & Multiple platforms & Automated ads & \textbf{App downloads} and engagement \\ \hline
\textbf{Smart Campaigns} & Automated placement & Mixed formats & \textbf{Small business} automation \\ \hline
\end{tabulary}
\end{center}

\begin{center}
\begin{tikzpicture}[node distance=1.5cm, auto]
    \node [gtu state] (Obj) {Obj};
    \node [gtu decision, right=1cm of Obj] (Type) {Type};
    \node [gtu block, above right=0.5cm and 1cm of Type] (Search) {Search};
    \node [gtu block, right=1cm of Type] (Display) {Display};
    \node [gtu block, below right=0.5cm and 1cm of Type] (Video) {Video};
    \node [gtu block, above=0.5cm of Search] (Shopping) {Shopping};
    
    \node [gtu block, right=0.5cm of Search] {Intent};
    \node [gtu block, right=0.5cm of Display] {Aware};
    \node [gtu block, right=0.5cm of Video] {Engage};
    \node [gtu block, right=0.5cm of Shopping] {Sales};

    \path [gtu arrow] (Obj) -- (Type);
    \path [gtu arrow] (Type) -- (Search);
    \path [gtu arrow] (Type) -- (Display);
    \path [gtu arrow] (Type) -- (Video);
    \path [gtu arrow] (Type) -- (Shopping);
\end{tikzpicture}
\captionof{figure}{Campaign Selection Flow}
\end{center}

\textbf{Optimization Strategies:}

\begin{itemize}
    \item \keyword{Keyword research} for search campaigns
    \item \keyword{Audience targeting} for display campaigns
    \item \keyword{Creative testing} across all formats
    \item \keyword{Conversion tracking} for ROI measurement
\end{itemize}

\textbf{Budget Allocation:}

\begin{itemize}
    \item \keyword{Search}: 40-50\% for high-intent traffic
    \item \keyword{Display}: 20-30\% for brand building
    \item \keyword{Video}: 15-25\% for engagement
    \item \keyword{Shopping}: 10-20\% for e-commerce
\end{itemize}

\textbf{Performance Metrics:}

\begin{itemize}
    \item \keyword{Click-through rates} (CTR)
    \item \keyword{Cost per acquisition} (CPA)
    \item \keyword{Return on ad spend} (ROAS)
\end{itemize}
\end{solutionbox}

\begin{mnemonicbox}
\mnemonic{SDVSAS Google: Search, Display, Video, Shopping, App, Smart campaigns}
\end{mnemonicbox}

\questionmarks{5(a) OR}{3}{Explain the concept of Twitter Marketing.}

\begin{solutionbox}
Twitter Marketing utilizes real-time communication for brand engagement and customer service.

\textbf{Core Elements:}

\begin{itemize}
    \item \keyword{Real-time engagement} with trending topics
    \item \keyword{Customer support} through direct responses
    \item \keyword{Content amplification} via retweets and hashtags
    \item \keyword{Influencer partnerships} for extended reach
\end{itemize}
\end{solutionbox}

\begin{mnemonicbox}
\mnemonic{RCCI Twitter: Real-time, Customer support, Content amplification, Influencer partnerships}
\end{mnemonicbox}

\questionmarks{5(b) OR}{4}{Give the difference between SEO and PPC.}

\begin{solutionbox}
\begin{center}
\captionof{table}{SEO vs PPC}
\begin{tabulary}{\linewidth}{|L|L|}
\hline
\textbf{SEO (Search Engine Optimization)} & \textbf{PPC (Pay-Per-Click)} \\ \hline
\textbf{Organic results} placement & \textbf{Paid advertisement} placement \\ \hline
\textbf{Long-term strategy} (3-6 months) & \textbf{Immediate results} (within hours) \\ \hline
\textbf{No direct cost} per click & \textbf{Cost per click} payment \\ \hline
\textbf{Sustainable traffic} growth & \textbf{Traffic stops} when budget ends \\ \hline
\textbf{Trust and credibility} higher & \textbf{Lower trust} (marked as ads) \\ \hline
\textbf{Requires ongoing} SEO efforts & \textbf{Requires continuous} budget \\ \hline
\end{tabulary}
\end{center}
\end{solutionbox}

\begin{mnemonicbox}
\mnemonic{OLNSTN vs PICRCR: Organic, Long-term, No cost vs Paid, Immediate, Cost-per-click}
\end{mnemonicbox}

\questionmarks{5(c) OR}{7}{Explain various bidding strategies available in Google Ads.}

\begin{solutionbox}
Google Ads provides multiple bidding strategies for different campaign goals:

\begin{center}
\captionof{table}{Google Ads Bidding Strategies}
\begin{tabulary}{\linewidth}{|L|L|L|L|}
\hline
\textbf{Bidding Strategy} & \textbf{Type} & \textbf{Goal} & \textbf{Best For} \\ \hline
\textbf{Manual CPC} & Manual & \textbf{Traffic control} & \textbf{Experienced advertisers} \\ \hline
\textbf{Enhanced CPC} & Semi-automated & \textbf{Conversion optimization} & \textbf{Balanced control} \\ \hline
\textbf{Target CPA} & Automated & \textbf{Cost per acquisition} & \textbf{Lead generation} \\ \hline
\textbf{Target ROAS} & Automated & \textbf{Return on ad spend} & \textbf{E-commerce sales} \\ \hline
\textbf{Maximize Clicks} & Automated & \textbf{Traffic volume} & \textbf{Brand awareness} \\ \hline
\textbf{Maximize Conversions} & Automated & \textbf{Conversion volume} & \textbf{Campaign scaling} \\ \hline
\end{tabulary}
\end{center}

\begin{center}
\begin{tikzpicture}[node distance=1.5cm, auto]
    \node [gtu state] (Goal) {Goal};
    
    \node [gtu block, below left=1cm and 2cm of Goal] (Traffic) {Traffic};
    \node [gtu block, left=3cm of Goal] (Leads) {Leads};
    \node [gtu block, right=3cm of Goal] (Sales) {Sales};
    \node [gtu block, below right=1cm and 2cm of Goal] (Control) {Control};

    \node [gtu decision, below=0.5cm of Traffic] (MaxClick) {Max Clicks};
    \node [gtu decision, below=0.5cm of Leads] (CPA) {Target CPA};
    \node [gtu decision, below=0.5cm of Sales] (ROAS) {Target ROAS};
    \node [gtu decision, below=0.5cm of Control] (Manual) {Manual CPC};

    \path [gtu arrow] (Goal) -- (Traffic);
    \path [gtu arrow] (Goal) -- (Leads);
    \path [gtu arrow] (Goal) -- (Sales);
    \path [gtu arrow] (Goal) -- (Control);

    \path [gtu arrow] (Traffic) -- (MaxClick);
    \path [gtu arrow] (Leads) -- (CPA);
    \path [gtu arrow] (Sales) -- (ROAS);
    \path [gtu arrow] (Control) -- (Manual);
\end{tikzpicture}
\captionof{figure}{Bidding Strategy Selection}
\end{center}

\textbf{Implementation Guidelines:}

\begin{itemize}
    \item \keyword{Manual CPC}: Start with bid adjustments and keyword-level control
    \item \keyword{Enhanced CPC}: Allows Google to adjust bids up to 30\% for better conversions
    \item \keyword{Target CPA}: Set based on historical conversion data
    \item \keyword{Target ROAS}: Requires sufficient conversion tracking data
\end{itemize}

\textbf{Performance Monitoring:}

\begin{itemize}
    \item \keyword{Bid strategy reports} for effectiveness analysis
    \item \keyword{Search term reports} for keyword optimization
    \item \keyword{Auction insights} for competitor comparison
\end{itemize}
\end{solutionbox}

\begin{mnemonicbox}
\mnemonic{METMM Bidding: Manual, Enhanced, Target CPA, Target ROAS, Maximize clicks, Maximize conversions}
\end{mnemonicbox}

\end{document}
