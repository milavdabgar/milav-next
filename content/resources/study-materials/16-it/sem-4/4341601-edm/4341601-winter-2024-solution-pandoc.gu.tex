\documentclass[10pt,a4paper]{article}

% content/resources/templates/preamble.tex
\usepackage[margin=0.6in]{geometry}
\author{Milav Dabgar}
\usepackage{amsmath,amssymb,amsthm}
\usepackage{booktabs}
\usepackage{multirow}
\usepackage{xcolor}
\usepackage{tcolorbox}
\tcbuselibrary{breakable,skins}
\usepackage[colorlinks=true,linkcolor=blue]{hyperref}
\usepackage{titlesec}
\usepackage{enumitem}
\usepackage{tikz}
\usepackage{pgfplots}
\usepackage{circuitikz}
\usepackage[version=4]{mhchem}
\usepackage{longtable}
\usepackage{array}
\usepackage{float}
\usepackage{caption}
\usepackage{listings}

\lstset{
  basicstyle=\small\ttfamily,
  breaklines=true,
  breakatwhitespace=false,
  postbreak=\mbox{\textcolor{red}{$\hookrightarrow$}\space},
  float=false,
  numbers=left,
  numberstyle=\tiny\color{gray},
  numbersep=10pt,
  xleftmargin=2em,
  keywordstyle=\color{blue},
  commentstyle=\color{green!60!black},
  stringstyle=\color{purple},
  backgroundcolor=\color{gray!5},
  showstringspaces=false,
  tabsize=2,
  captionpos=b,
  keepspaces=true,
  columns=flexible
}

\pgfplotsset{compat=1.18}
\usetikzlibrary{shapes,arrows,positioning,calc,patterns,decorations.pathmorphing,decorations.markings,arrows.meta}

% Color scheme
\definecolor{headcolor}{RGB}{0,102,204}
\definecolor{keycolor}{RGB}{220,20,60}
\definecolor{solutioncolor}{RGB}{34,139,34}
\definecolor{mnemoniccolor}{RGB}{148,0,211}
\definecolor{codecolor}{RGB}{0,0,100}

% Spacing
\setlength{\parskip}{3pt}
\setlist[itemize]{nosep}
\setlist[enumerate]{nosep}

% Title formatting
\titleformat{\section}{\Large\bfseries\color{headcolor}}{\thesection}{1em}{}
\titleformat{\subsection}{\large\bfseries\color{headcolor}}{\thesubsection}{1em}{}

% Pandoc tightlist compatibility
\providecommand{\tightlist}{%
  \setlength{\itemsep}{0pt}\setlength{\parskip}{0pt}}

% Pandoc longtable compatibility
\newcounter{none}
\def\thenone{}


% content/resources/templates/gujarati-boxes.tex
\usepackage{fontspec}
\usepackage{polyglossia}

% Set Gujarati as main language (document is primarily in Gujarati)
% Note: gloss-gujarati.ldf doesn't exist in polyglossia, but it will use hyphenation patterns
\setdefaultlanguage{gujarati}
\setotherlanguage{english}

% Configure Gujarati font properly
% Use Language=Default to prevent polyglossia from trying to add language-specific features
% that don't exist for Gujarati, which causes "empty feature" warnings
\newfontfamily\gujaratifont[Script=Gujarati,AutoFakeBold=2.5,AutoFakeSlant=0.3]{Noto Sans Gujarati}
\setmainfont[Script=Gujarati,AutoFakeBold=2.5,AutoFakeSlant=0.3]{Noto Sans Gujarati}
% Use Noto Sans Gujarati for monospace to support Gujarati in text
\setmonofont[Scale=0.9]{Noto Sans Gujarati}

% Configure English to use the same font
\newfontfamily\englishfont[Script=Gujarati,AutoFakeBold=2.5,AutoFakeSlant=0.3]{Noto Sans Gujarati}

% Translations for polyglossia
\gappto\captionsgujarati{
  \renewcommand{\tablename}{કોષ્ટક}
  \renewcommand{\figurename}{આકૃતિ}
}

% Helper for TikZ nodes to ensure Gujarati font
\newcommand{\gu}[1]{{\gujaratifont #1}}

% Custom environments
\newtcolorbox{solutionbox}{
    breakable,
    enhanced,
    colback=solutioncolor!5!white,
    colframe=solutioncolor!75!black,
    fonttitle=\bfseries,
    title=જવાબ
}

\newtcolorbox{solutionboxnobreak}{
 colback=solutioncolor!5!white,
 colframe=solutioncolor!75!black,
 fonttitle=\bfseries,
 title=જવાબ
}

\newtcolorbox{keyformula}{
 breakable,
 enhanced,
 colback=keycolor!5!white,
 colframe=keycolor!75!black,
 fonttitle=\bfseries,
 title=રાસાયણિક સમીકરણ/સૂત્ર
}

\newtcolorbox{mnemonicbox}{
 breakable,
 enhanced,
 colback=mnemoniccolor!5!white,
 colframe=mnemoniccolor!75!black,
 fonttitle=\bfseries,
 title=મેમરી ટ્રીક
}


\begin{document}

\begin{center}
{\Huge\bfseries\color{headcolor} Subject Name (Gujarati)}\\[5pt]
{\LARGE 4341601 -- Winter 2024}\\[3pt]
{\large Semester 1 Study Material}\\[3pt]
{\normalsize\textit{Detailed Solutions and Explanations}}
\end{center}

\vspace{10pt}

\subsection*{પ્રશ્ન 1(અ) [3
ગુણ]}\label{uxaaauxab0uxab6uxaa8-1uxa85-3-uxa97uxaa3}

\textbf{વેબસાઇટના SEO રેન્કિંગને પ્રભાવિત કરતા ત્રણ મહત્વપૂર્ણ પરિબળો સમજાવો.}

\begin{solutionbox}

{\def\LTcaptype{none} % do not increment counter
\begin{longtable}[]{@{}ll@{}}
\toprule\noalign{}
પરિબળ & વર્ણન \\
\midrule\noalign{}
\endhead
\bottomrule\noalign{}
\endlastfoot
\textbf{કન્ટેન્ટ ક્વોલિટી} & તાજું, સંબંધિત, કીવર્ડ-ઓપ્ટિમાઇઝ્ડ કન્ટેન્ટ \\
\textbf{બેકલિંક્સ} & અન્ય ગુણવત્તાવાળી વેબસાઇટ્સમાંથી લિંક્સ \\
\textbf{ટેકનિકલ SEO} & સાઇટ સ્પીડ, મોબાઇલ-ફ્રેન્ડલી, SSL સર્ટિફિકેટ \\
\end{longtable}
}

\begin{itemize}
\tightlist
\item
  \textbf{કન્ટેન્ટ ક્વોલિટી}: સર્ચ એન્જિન મૂલ્યવાન કન્ટેન્ટને પ્રાથમિકતા આપે છે
\item
  \textbf{બેકલિંક્સ}: અન્ય વેબસાઇટ્સ તરફથી વિશ્વાસની મતદાન તરીકે કામ કરે છે
\item
  \textbf{ટેકનિકલ SEO}: સર્ચ એન્જિનને સાઇટને ક્રોલ અને ઇન્ડેક્સ કરવામાં મદદ કરે છે
\end{itemize}

\end{solutionbox}
\begin{mnemonicbox}
``CBT - Content, Backlinks, Technical''

\end{mnemonicbox}
\subsection*{પ્રશ્ન 1(બ) [4
ગુણ]}\label{uxaaauxab0uxab6uxaa8-1uxaac-4-uxa97uxaa3}

\textbf{ડિજિટલ માર્કેટિંગમાં ડેટાની ગોપનીયતા અને તેનું મહત્વ વર્ણવો.}

\begin{solutionbox}

\textbf{ડેટા પ્રાઇવસી} એટલે ડિજિટલ માર્કેટિંગ પ્રવૃત્તિઓ દરમિયાન એકત્રિત કરાયેલી
વ્યક્તિગત માહિતીનું સંરક્ષણ.

{\def\LTcaptype{none} % do not increment counter
\begin{longtable}[]{@{}ll@{}}
\toprule\noalign{}
પાસું & મહત્વ \\
\midrule\noalign{}
\endhead
\bottomrule\noalign{}
\endlastfoot
\textbf{યુઝર ટ્રસ્ટ} & ગ્રાહકોનો વિશ્વાસ અને વફાદારી બનાવે છે \\
\textbf{કાયદાકીય પાલન} & GDPR, CCPA નિયમોથી દંડ બચાવે છે \\
\textbf{બ્રાન્ડ પ્રતિષ્ઠા} & ડેટા બ્રીચથી નકારાત્મક પ્રચારને અટકાવે છે \\
\end{longtable}
}

\begin{itemize}
\tightlist
\item
  \textbf{યુઝર ટ્રસ્ટ}: જ્યારે ગ્રાહકો તમારી પ્રાઇવસી પ્રેક્ટિસ પર ભરોસો રાખે છે
  ત્યારે વધુ ડેટા શેર કરે છે
\item
  \textbf{કાયદાકીય પાલન}: ડેટા પ્રોટેક્શન કાયદાઓનું ફરજિયાત પાલન
\item
  \textbf{બ્રાન્ડ પ્રતિષ્ઠા}: ડેટા બ્રીચ બ્રાન્ડ ઇમેજને ગંભીર નુકસાન પહોંચાડી શકે છે
\end{itemize}

\end{solutionbox}
\begin{mnemonicbox}
``TLR - Trust, Legal, Reputation''

\end{mnemonicbox}
\subsection*{પ્રશ્ન 1(ક) [7
ગુણ]}\label{uxaaauxab0uxab6uxaa8-1uxa95-7-uxa97uxaa3}

\textbf{ડિજિટલ માર્કેટિંગ યોજનાઓ માટેના મુખ્ય ઘટકો સમજાવો.}

\begin{solutionbox}

{\def\LTcaptype{none} % do not increment counter
\begin{longtable}[]{@{}ll@{}}
\toprule\noalign{}
ઘટક & વર્ણન \\
\midrule\noalign{}
\endhead
\bottomrule\noalign{}
\endlastfoot
\textbf{લક્ષ્યો અને ઉદ્દેશ્યો} & વ્યવસાયિક ઉદ્દેશ્યો સાથે જોડાયેલા SMART લક્ષ્યો \\
\textbf{ટાર્ગેટ ઓડિયન્સ} & ડેમોગ્રાફિક્સ, સાયકોગ્રાફિક્સ અને વર્તન વિશ્લેષણ \\
\textbf{ચેનલ સ્ટ્રેટેજી} & યોગ્ય ડિજિટલ પ્લેટફોર્મની પસંદગી \\
\textbf{કન્ટેન્ટ સ્ટ્રેટેજી} & કન્ટેન્ટના પ્રકારો, થીમ્સ અને પબ્લિશિંગ શેડ્યૂલ \\
\textbf{બજેટ ફાળવણી} & ચેનલ્સમાં સંસાધનોની વિતરણ \\
\textbf{એનાલિટિક્સ અને KPIs} & માપદંડ ફ્રેમવર્ક અને સફળતાના મેટ્રિક્સ \\
\end{longtable}
}

\begin{center}
\textbf{Mermaid Diagram (Code)}
\begin{verbatim}
{Shaded}
{Highlighting}[]
graph TD
    A[ડિજિટલ માર્કેટિંગ પ્લાન] {-{-}{} B[લક્ષ્યો અને ઉદ્દેશ્યો]}
    A {-{-}{} C[ટાર્ગેટ ઓડિયન્સ]}
    A {-{-}{} D[ચેનલ સ્ટ્રેટેજી]}
    A {-{-}{} E[કન્ટેન્ટ સ્ટ્રેટેજી]}
    A {-{-}{} F[બજેટ ફાળવણી]}
    A {-{-}{} G[એનાલિટિક્સ અને KPIs]}
{Highlighting}
{Shaded}
\end{verbatim}
\end{center}

\begin{itemize}
\tightlist
\item
  \textbf{લક્ષ્યો અને ઉદ્દેશ્યો}: વિશિષ્ટ, માપવા યોગ્ય પરિણામો વ્યાખ્યાયિત કરો
\item
  \textbf{ટાર્ગેટ ઓડિયન્સ}: વિગતવાર બાયર પર્સોના બનાવો
\item
  \textbf{ચેનલ સ્ટ્રેટેજી}: સોશિયલ મીડિયા, ઇમેઇલ, SEO, PPC નું સર્વોત્તમ મિશ્રણ પસંદ
  કરો
\item
  \textbf{કન્ટેન્ટ સ્ટ્રેટેજી}: આકર્ષક કન્ટેન્ટ કેલેન્ડર વિકસાવો
\item
  \textbf{બજેટ ફાળવણી}: ROI ની સંભાવના આધારે સંસાધનોનું વિતરણ કરો
\item
  \textbf{એનાલિટિક્સ અને KPIs}: પરફોર્મન્સ ટ્રેક કરો અને સતત ઑપ્ટિમાઇઝ કરો
\end{itemize}

\end{solutionbox}
\begin{mnemonicbox}
``GT-CCBA - Goals-Target,
Channels-Content-Budget-Analytics''

\end{mnemonicbox}
\subsection*{પ્રશ્ન 1(ક OR) [7
ગુણ]}\label{uxaaauxab0uxab6uxaa8-1uxa95-or-7-uxa97uxaa3}

\textbf{P.O.E.M ફ્રેમવર્કને વ્યાખ્યાયિત કરો અને ડિજિટલ માર્કેટિંગમાં તેનું મહત્વ
સમજાવો.}

\begin{solutionbox}

\textbf{P.O.E.M.} એટલે \textbf{Paid, Owned, Earned, Media} ફ્રેમવર્ક ડિજિટલ
માર્કેટિંગ સ્ટ્રેટેજી માટે.

{\def\LTcaptype{none} % do not increment counter
\begin{longtable}[]{@{}
  >{\raggedright\arraybackslash}p{(\linewidth - 4\tabcolsep) * \real{0.4815}}
  >{\raggedright\arraybackslash}p{(\linewidth - 4\tabcolsep) * \real{0.2222}}
  >{\raggedright\arraybackslash}p{(\linewidth - 4\tabcolsep) * \real{0.2963}}@{}}
\toprule\noalign{}
\begin{minipage}[b]{\linewidth}\raggedright
મીડિયા પ્રકાર
\end{minipage} & \begin{minipage}[b]{\linewidth}\raggedright
વર્ણન
\end{minipage} & \begin{minipage}[b]{\linewidth}\raggedright
ઉદાહરણો
\end{minipage} \\
\midrule\noalign{}
\endhead
\bottomrule\noalign{}
\endlastfoot
\textbf{Paid} & તમે પૈસા ચૂકવો છો તે મીડિયા & Google Ads, Facebook Ads,
YouTube Ads \\
\textbf{Owned} & તમે નિયંત્રિત કરો છો તે મીડિયા & વેબસાઇટ, બ્લોગ, ઇમેઇલ લિસ્ટ,
મોબાઇલ એપ \\
\textbf{Earned} & વિશ્વસનીયતા દ્વારા મેળવેલ મીડિયા & સોશિયલ શેર્સ, રિવ્યૂઝ, PR
મેન્શન્સ \\
\end{longtable}
}

\begin{center}
\textbf{Mermaid Diagram (Code)}
\begin{verbatim}
{Shaded}
{Highlighting}[]
graph TD
    A[P.O.E.M Framework] {-{-}{} B[Paid Media]}
    A {-{-}{} C[Owned Media]}
    A {-{-}{} D[Earned Media]}
    B {-{-}{} E[તાત્કાલિક પહોંચ]}
    C {-{-}{} F[લાંબા ગાળાની સંપત્તિ]}
    D {-{-}{} G[વિશ્વાસ અને વિશ્વસનીયતા]}
{Highlighting}
{Shaded}
\end{verbatim}
\end{center}

\begin{itemize}
\tightlist
\item
  \textbf{Paid Media}: તાત્કાલિક દૃશ્યતા અને લક્ષિત પહોંચ પ્રદાન કરે છે
\item
  \textbf{Owned Media}: લાંબા ગાળાની સંપત્તિ અને બ્રાન્ડ નિયંત્રણ બનાવે છે
\item
  \textbf{Earned Media}: વિશ્વાસ અને અધિકૃત બ્રાન્ડ એડવોકસી બનાવે છે
\end{itemize}

\end{solutionbox}
\begin{mnemonicbox}
``POE - Pay, Own, Earn''

\end{mnemonicbox}
\subsection*{પ્રશ્ન 2(અ) [3
ગુણ]}\label{uxaaauxab0uxab6uxaa8-2uxa85-3-uxa97uxaa3}

\textbf{Black hat અને White hat SEO ટેકનીક વચ્ચે તફાવત વર્ણવો.}

\begin{solutionbox}

{\def\LTcaptype{none} % do not increment counter
\begin{longtable}[]{@{}lll@{}}
\toprule\noalign{}
પાસું & White Hat SEO & Black Hat SEO \\
\midrule\noalign{}
\endhead
\bottomrule\noalign{}
\endlastfoot
\textbf{પદ્ધતિઓ} & નૈતિક, માર્ગદર્શિકા-અનુપાલન & હેરાફેરીયુક્ત, નિયમ-ભંગ \\
\textbf{પરિણામો} & ટકાઉ લાંબા ગાળાની વૃદ્ધિ & ઝડપી પરંતુ અસ્થાયી લાભ \\
\textbf{જોખમ} & દંડથી સુરક્ષિત & દંડનું ઉચ્ચ જોખમ \\
\end{longtable}
}

\begin{itemize}
\tightlist
\item
  \textbf{White Hat SEO}: ટકાઉ પરિણામો માટે સર્ચ એન્જિન માર્ગદર્શિકાઓનું પાલન
  કરે છે
\item
  \textbf{Black Hat SEO}: ઝડપી રેન્કિંગ લાભ માટે ભ્રામક પ્રથાઓનો ઉપયોગ કરે છે
\item
  \textbf{જોખમ પરિબળ}: Black hat ટેકનીક્સના કારણે સંપૂર્ણ સાઇટ બેન થઈ શકે છે
\end{itemize}

\end{solutionbox}
\begin{mnemonicbox}
``WEB - White Ethical Benefits, Black Breaks-rules''

\end{mnemonicbox}
\subsection*{પ્રશ્ન 2(બ) [4
ગુણ]}\label{uxaaauxab0uxab6uxaa8-2uxaac-4-uxa97uxaa3}

\textbf{સર્ચ એન્જિન અલ્ગોરિધમ કઈ રીતે કાર્ય કરે છે અને વેબસાઈટને કઈ રીતે રેંક આપે છે એ
સમજાવો.}

\begin{solutionbox}

{\def\LTcaptype{none} % do not increment counter
\begin{longtable}[]{@{}ll@{}}
\toprule\noalign{}
પ્રક્રિયા & કાર્ય \\
\midrule\noalign{}
\endhead
\bottomrule\noalign{}
\endlastfoot
\textbf{ક્રોલિંગ} & બોટ્સ વેબ પેજો શોધે અને સ્કેન કરે છે \\
\textbf{ઇન્ડેક્સિંગ} & પેજો સર્ચ એન્જિન ડેટાબેસમાં સંગ્રહિત થાય છે \\
\textbf{રેન્કિંગ} & અલ્ગોરિધમ પેજની સંબંધિતતા અને અધિકાર નક્કી કરે છે \\
\textbf{પરિણામો} & યુઝર ક્વેરીઝ માટે શ્રેષ્ઠ મેચ દર્શાવવામાં આવે છે \\
\end{longtable}
}

\begin{itemize}
\tightlist
\item
  \textbf{ક્રોલિંગ}: વેબ ક્રોલર્સ નવું કન્ટેન્ટ શોધવા માટે લિંક્સને ફોલો કરે છે
\item
  \textbf{ઇન્ડેક્સિંગ}: કન્ટેન્ટનું વિશ્લેષણ કરીને મોટા ડેટાબેસમાં સંગ્રહિત કરવામાં આવે છે
\item
  \textbf{રેન્કિંગ}: 200+ પરિબળો સર્ચ પરિણામ સ્થાનો નક્કી કરે છે
\item
  \textbf{પરિણામો}: સૌથી સંબંધિત પેજો યુઝરોને પ્રથમ દર્શાવવામાં આવે છે
\end{itemize}

\end{solutionbox}
\begin{mnemonicbox}
``CIRR - Crawl, Index, Rank, Results''

\end{mnemonicbox}
\subsection*{પ્રશ્ન 2(ક) [7
ગુણ]}\label{uxaaauxab0uxab6uxaa8-2uxa95-7-uxa97uxaa3}

\textbf{બેકલિંક્સ બનાવવા માટેની વ્યૂહરચનાઓનું વર્ણન કરો.}

\begin{solutionbox}

{\def\LTcaptype{none} % do not increment counter
\begin{longtable}[]{@{}lll@{}}
\toprule\noalign{}
વ્યૂહરચના & વર્ણન & અસરકારકતા \\
\midrule\noalign{}
\endhead
\bottomrule\noalign{}
\endlastfoot
\textbf{ગેસ્ટ પોસ્ટિંગ} & અન્ય વેબસાઇટ્સ માટે લેખો લખવા & ઉચ્ચ \\
\textbf{રિસોર્સ લિંક બિલ્ડિંગ} & ઉદ્યોગ ડાયરેક્ટરીમાં સૂચિબદ્ધ થવું & મધ્યમ \\
\textbf{બ્રોકન લિંક બિલ્ડિંગ} & તૂટેલી લિંક્સને તમારા કન્ટેન્ટ સાથે બદલવી & ઉચ્ચ \\
\textbf{કન્ટેન્ટ માર્કેટિંગ} & શેર કરવા યોગ્ય, મૂલ્યવાન કન્ટેન્ટ બનાવવું & ખૂબ ઉચ્ચ \\
\textbf{ઇન્ફ્લુએન્સર આઉટરીચ} & ઉદ્યોગ ઇન્ફ્લુએન્સર્સ સાથે ભાગીદારી & ઉચ્ચ \\
\end{longtable}
}

\begin{center}
\textbf{Mermaid Diagram (Code)}
\begin{verbatim}
{Shaded}
{Highlighting}[]
graph TD
    A[બેકલિંક બિલ્ડિંગ] {-{-}{} B[ગેસ્ટ પોસ્ટિંગ]}
    A {-{-}{} C[રિસોર્સ લિંક્સ]}
    A {-{-}{} D[બ્રોકન લિંક બિલ્ડિંગ]}
    A {-{-}{} E[કન્ટેન્ટ માર્કેટિંગ]}
    A {-{-}{} F[ઇન્ફ્લુએન્સર આઉટરીચ]}
{Highlighting}
{Shaded}
\end{verbatim}
\end{center}

\begin{itemize}
\tightlist
\item
  \textbf{ગેસ્ટ પોસ્ટિંગ}: તમારા નિશમાં સંબંધો અને સત્તા બનાવે છે
\item
  \textbf{રિસોર્સ લિંક બિલ્ડિંગ}: ડાયરેક્ટરીઓ દ્વારા વિશ્વસનીયતા સ્થાપિત કરે છે
\item
  \textbf{બ્રોકન લિંક બિલ્ડિંગ}: તૂટેલા સંસાધનો ઠીક કરીને મૂલ્য પ્રદાન કરે છે
\item
  \textbf{કન્ટેન્ટ માર્કેટિંગ}: ગુણવત્તાયુક્ત કન્ટેન્ટ દ્વારા કુદરતી રીતે લિંક્સ આકર્ષે છે
\item
  \textbf{ઇન્ફ્લુએન્સર આઉટરીચ}: લિંક તકો માટે સ્થાપિત પ્રેક્ષકોનો લાભ લે છે
\end{itemize}

\end{solutionbox}
\begin{mnemonicbox}
``GRBCI - Guest, Resource, Broken, Content,
Influencer''

\end{mnemonicbox}
\subsection*{પ્રશ્ન 2(અ OR) [3
ગુણ]}\label{uxaaauxab0uxab6uxaa8-2uxa85-or-3-uxa97uxaa3}

\textbf{સર્ચ એન્જિન રેન્કિંગ માટે backlinks, website speed અને performance ની
અગત્યતા સમજાવો.}

\begin{solutionbox}

{\def\LTcaptype{none} % do not increment counter
\begin{longtable}[]{@{}ll@{}}
\toprule\noalign{}
પરિબળ & SEO પર અસર \\
\midrule\noalign{}
\endhead
\bottomrule\noalign{}
\endlastfoot
\textbf{બેકલિંક્સ} & સત્તા અને વિશ્વાસના સંકેતો \\
\textbf{વેબસાઇટ સ્પીડ} & યુઝર એક્સપિરિયન્સ રેન્કિંગ પરિબળ \\
\textbf{પરફોર્મન્સ} & Core Web Vitals રેન્કિંગને અસર કરે છે \\
\end{longtable}
}

\begin{itemize}
\tightlist
\item
  \textbf{બેકલિંક્સ}: અન્ય વેબસાઇટ્સ તરફથી વિશ્વાસના વોટ તરીકે કામ કરે છે
\item
  \textbf{વેબસાઇટ સ્પીડ}: ઝડપી સાઇટ્સ ઉચ્ચ રેન્ક કરે છે અને બાઉન્સ રેટ ઘટાડે છે
\item
  \textbf{પરફોર્મન્સ}: Google સારા Core Web Vitals વાળી સાઇટ્સને પ્રાથમિકતા
  આપે છે
\end{itemize}

\end{solutionbox}
\begin{mnemonicbox}
``BSP - Backlinks, Speed, Performance''

\end{mnemonicbox}
\subsection*{પ્રશ્ન 2(બ OR) [4
ગુણ]}\label{uxaaauxab0uxab6uxaa8-2uxaac-or-4-uxa97uxaa3}

\textbf{On-page અને Off-page SEO ટેકનીક વચ્ચે તફાવત વર્ણવો અને દરેકનું એક ઉદાહરણ
આપો.}

\begin{solutionbox}

{\def\LTcaptype{none} % do not increment counter
\begin{longtable}[]{@{}
  >{\raggedright\arraybackslash}p{(\linewidth - 4\tabcolsep) * \real{0.4400}}
  >{\raggedright\arraybackslash}p{(\linewidth - 4\tabcolsep) * \real{0.2400}}
  >{\raggedright\arraybackslash}p{(\linewidth - 4\tabcolsep) * \real{0.3200}}@{}}
\toprule\noalign{}
\begin{minipage}[b]{\linewidth}\raggedright
SEO પ્રકાર
\end{minipage} & \begin{minipage}[b]{\linewidth}\raggedright
ફોકસ
\end{minipage} & \begin{minipage}[b]{\linewidth}\raggedright
ઉદાહરણો
\end{minipage} \\
\midrule\noalign{}
\endhead
\bottomrule\noalign{}
\endlastfoot
\textbf{On-Page} & વેબસાઇટ ઑપ્ટિમાઇઝેશન & ટાઇટલ ટેગ્સ, મેટા વર્ણનો, કન્ટેન્ટ
ઑપ્ટિમાઇઝેશન \\
\textbf{Off-Page} & બાહ્ય પરિબળો & બેકલિંક્સ, સોશિયલ સિગ્નલ્સ, બ્રાન્ડ મેન્શન્સ \\
\end{longtable}
}

\begin{itemize}
\tightlist
\item
  \textbf{On-Page SEO}: તમારી વેબસાઇટની અંદરના તત્વોને નિયંત્રિત કરે છે
\item
  \textbf{Off-Page SEO}: બાહ્ય વેલિડેશન દ્વારા સત્તા બનાવે છે
\item
  \textbf{ઉદાહરણો}: On-page માં કીવર્ડ ઑપ્ટિમાઇઝેશન; off-page માં લિંક બિલ્ડિંગ
\end{itemize}

\end{solutionbox}
\begin{mnemonicbox}
``IO - Internal Optimization, External Elevation''

\end{mnemonicbox}
\subsection*{પ્રશ્ન 2(ક OR) [7
ગુણ]}\label{uxaaauxab0uxab6uxaa8-2uxa95-or-7-uxa97uxaa3}

\textbf{SEO રેન્કિંગમાં સુધારો કરવાની વિવિધ રીતો સમજાવો.}

\begin{solutionbox}

{\def\LTcaptype{none} % do not increment counter
\begin{longtable}[]{@{}
  >{\raggedright\arraybackslash}p{(\linewidth - 4\tabcolsep) * \real{0.3889}}
  >{\raggedright\arraybackslash}p{(\linewidth - 4\tabcolsep) * \real{0.3333}}
  >{\raggedright\arraybackslash}p{(\linewidth - 4\tabcolsep) * \real{0.2778}}@{}}
\toprule\noalign{}
\begin{minipage}[b]{\linewidth}\raggedright
પદ્ધતિ
\end{minipage} & \begin{minipage}[b]{\linewidth}\raggedright
વર્ણન
\end{minipage} & \begin{minipage}[b]{\linewidth}\raggedright
અસર
\end{minipage} \\
\midrule\noalign{}
\endhead
\bottomrule\noalign{}
\endlastfoot
\textbf{કીવર્ડ રિસર્ચ} & સંબંધિત, ઓછી સ્પર્ધાવાળા કીવર્ડ્સ લક્ષિત કરવા & ઉચ્ચ \\
\textbf{કન્ટેન્ટ ઑપ્ટિમાઇઝેશન} & મૂલ્યવાન, કીવર્ડ-સમૃદ્ધ કન્ટેન્ટ બનાવવું & ખૂબ ઉચ્ચ \\
\textbf{ટેકનિકલ SEO} & સાઇટ સ્પીડ, મોબાઇલ-ફ્રેન્ડલીનેસ સુધારવી & ઉચ્ચ \\
\textbf{લિંક બિલ્ડિંગ} & ગુણવત્તાયુક્ત બેકલિંક્સ મેળવવી & ખૂબ ઉચ્ચ \\
\textbf{યુઝર એક્સપિરિયન્સ} & સાઇટ ઉપયોગીતા અને સંલગ્નતા વધારવી & મધ્યમ \\
\textbf{લોકલ SEO} & સ્થાનિક સર્ચ પરિણામો માટે ઑપ્ટિમાઇઝ કરવું & ઉચ્ચ (સ્થાનિક
વ્યવસાય માટે) \\
\end{longtable}
}

\begin{center}
\textbf{Mermaid Diagram (Code)}
\begin{verbatim}
{Shaded}
{Highlighting}[]
graph TD
    A[SEO સુધારણા] {-{-}{} B[કીવર્ડ રિસર્ચ]}
    A {-{-}{} C[કન્ટેન્ટ ઑપ્ટિમાઇઝેશન]}
    A {-{-}{} D[ટેકનિકલ SEO]}
    A {-{-}{} E[લિંક બિલ્ડિંગ]}
    A {-{-}{} F[યુઝર એક્સપિરિયન્સ]}
    A {-{-}{} G[લોકલ SEO]}
{Highlighting}
{Shaded}
\end{verbatim}
\end{center}

\begin{itemize}
\tightlist
\item
  \textbf{કીવર્ડ રિસર્ચ}: બધા SEO પ્રયાસો માટે પાયો
\item
  \textbf{કન્ટેન્ટ ઑપ્ટિમાઇઝેશન}: કીવર્ડ્સને લક્ષિત કરતી વખતે મૂલ્ય પ્રદાન કરે છે
\item
  \textbf{ટેકનિકલ SEO}: સર્ચ એન્જિનો તમારી સાઇટને અસરકારક રીતે ક્રોલ કરી શકે
  તેની ખાતરી કરે છે
\item
  \textbf{લિંક બિલ્ડિંગ}: ડોમેઇન ઑથોરિટી અને વિશ્વાસ બનાવે છે
\item
  \textbf{યુઝર એક્સપિરિયન્સ}: બાઉન્સ રેટ ઘટાડે અને સંલગ્નતા વધારે છે
\item
  \textbf{લોકલ SEO}: ભૌતિક સ્થાનો ધરાવતા વ્યવસાયો માટે મહત્વપૂર્ણ
\end{itemize}

\end{solutionbox}
\begin{mnemonicbox}
``KC-TLUL - Keywords, Content, Technical, Links,
User-experience, Local''

\end{mnemonicbox}
\subsection*{પ્રશ્ન 3(અ) [3
ગુણ]}\label{uxaaauxab0uxab6uxaa8-3uxa85-3-uxa97uxaa3}

\textbf{Single-touch અને multi-touch attribution મોડેલ વચ્ચેનો તફાવત વર્ણવો.}

\begin{solutionbox}

{\def\LTcaptype{none} % do not increment counter
\begin{longtable}[]{@{}lll@{}}
\toprule\noalign{}
મોડેલ પ્રકાર & ક્રેડિટ સોંપણી & ઉપયોગ કેસ \\
\midrule\noalign{}
\endhead
\bottomrule\noalign{}
\endlastfoot
\textbf{Single-Touch} & એક ટચપોઇન્ટને 100\% ક્રેડિટ & સરળ ગ્રાહક યાત્રાઓ \\
\textbf{Multi-Touch} & ટચપોઇન્ટ્સમાં ક્રેડિટ વિતરણ & જટિલ ગ્રાહક યાત્રાઓ \\
\end{longtable}
}

\begin{itemize}
\tightlist
\item
  \textbf{Single-Touch}: પ્રથમ-ક્લિક અથવા છેલ્લા-ક્લિકને સંપૂર્ણ ક્રેડિટ મળે છે
\item
  \textbf{Multi-Touch}: લિનિયર, ટાઇમ-ડિકે, અથવા પોઝિશન-આધારિત એટ્રિબ્યુશન
\item
  \textbf{ઉપયોગ}: Multi-touch વધુ સચોટ ગ્રાહક યાત્રા આંતરદૃષ્ટિ પ્રદાન કરે છે
\end{itemize}

\end{solutionbox}
\begin{mnemonicbox}
``SM - Single Simple, Multi Multiple''

\end{mnemonicbox}
\subsection*{પ્રશ્ન 3(બ) [4
ગુણ]}\label{uxaaauxab0uxab6uxaa8-3uxaac-4-uxa97uxaa3}

\textbf{Google Analytics માં વ્યવસાયો કેવી રીતે લક્ષ્યો સેટ કરી શકે છે તે સમજાવો.}

\begin{solutionbox}

{\def\LTcaptype{none} % do not increment counter
\begin{longtable}[]{@{}ll@{}}
\toprule\noalign{}
પગલું & ક્રિયા \\
\midrule\noalign{}
\endhead
\bottomrule\noalign{}
\endlastfoot
\textbf{1. ગોલ્સ એક્સેસ} & Admin \rightarrow View \rightarrow Goals પર જાઓ \\
\textbf{2. ટેમ્પલેટ પસંદ કરો} & ટેમ્પલેટમાંથી પસંદ કરો અથવા કસ્ટમ બનાવો \\
\textbf{3. વિગતો કન્ફિગર કરો} & ગોલ નામ, પ્રકાર અને શરતો સેટ કરો \\
\textbf{4. સેટઅપ ચકાસો} & વેરિફિકેશન ફીચર વાપરીને ગોલ ટેસ્ટ કરો \\
\end{longtable}
}

\begin{itemize}
\tightlist
\item
  \textbf{ગોલ પ્રકારો}: ડેસ્ટિનેશન, અવધિ, પેજ/સેશન, ઇવેન્ટ ગોલ્સ
\item
  \textbf{કન્ફિગરેશન}: ગોલ પૂર્ણતા માટે વિશિષ્ટ શરતો વ્યાખ્યાયિત કરો
\item
  \textbf{વેરિફિકેશન}: અમલીકરણ પહેલાં ગોલ્સ યોગ્ય રીતે ટ્રેક કરે છે તેની ખાતરી કરો
\item
  \textbf{મોનિટરિંગ}: ગોલ પરફોર્મન્સની નિયમિત સમીક્ષા અને ઑપ્ટિમાઇઝેશન
\end{itemize}

\end{solutionbox}
\begin{mnemonicbox}
``ACCV - Access, Choose, Configure, Verify''

\end{mnemonicbox}
\subsection*{પ્રશ્ન 3(ક) [7
ગુણ]}\label{uxaaauxab0uxab6uxaa8-3uxa95-7-uxa97uxaa3}

\textbf{ડિજિટલ માર્કેટિંગની વ્યૂહરચના ઘડવામાં વેબ એનાલિટિક્સની શું ભૂમિકા છે? વિવિધ
પ્રકારના વેબ એનાલિટિક્સ વિશે ચર્ચા કરો.}

\begin{solutionbox}

\textbf{વ્યૂહરચનામાં ભૂમિકા:} વેબ એનાલિટિક્સ ડિજિટલ માર્કેટિંગમાં માહિતી-આધારિત
નિર્ણય લેવા માટે ડેટા-આધારિત આંતરદૃષ્ટિ પ્રદાન કરે છે.

{\def\LTcaptype{none} % do not increment counter
\begin{longtable}[]{@{}
  >{\raggedright\arraybackslash}p{(\linewidth - 4\tabcolsep) * \real{0.5000}}
  >{\raggedright\arraybackslash}p{(\linewidth - 4\tabcolsep) * \real{0.1389}}
  >{\raggedright\arraybackslash}p{(\linewidth - 4\tabcolsep) * \real{0.3611}}@{}}
\toprule\noalign{}
\begin{minipage}[b]{\linewidth}\raggedright
એનાલિટિક્સ પ્રકાર
\end{minipage} & \begin{minipage}[b]{\linewidth}\raggedright
હેતુ
\end{minipage} & \begin{minipage}[b]{\linewidth}\raggedright
મુખ્ય મેટ્રિક્સ
\end{minipage} \\
\midrule\noalign{}
\endhead
\bottomrule\noalign{}
\endlastfoot
\textbf{કન્ટેન્ટ એનાલિટિક્સ} & કન્ટેન્ટ પરફોર્મન્સ ટ્રેકિંગ & પેજ વ્યૂઝ, પેજ પર સમય,
બાઉન્સ રેટ \\
\textbf{કસ્ટમર એનાલિટિક્સ} & યુઝર વર્તન વિશ્લેષણ & ડેમોગ્રાફિક્સ, રુચિઓ, કન્વર્શન
પાથ \\
\textbf{સોશિયલ મીડિયા એનાલિટિક્સ} & સોશિયલ એન્ગેજમેન્ટ માપદંડ & શેર્સ, લાઇક્સ,
કોમેન્ટ્સ, રીચ \\
\textbf{SEO એનાલિટિક્સ} & સર્ચ પરફોર્મન્સ ટ્રેકિંગ & કીવર્ડ્સ, રેન્કિંગ્સ, ઓર્ગેનિક
ટ્રાફિક \\
\textbf{કન્વર્શન એનાલિટિક્સ} & ગોલ પૂર્ણતા ટ્રેકિંગ & કન્વર્શન રેટ, રેવન્યુ, ROI \\
\end{longtable}
}

\begin{center}
\textbf{Mermaid Diagram (Code)}
\begin{verbatim}
{Shaded}
{Highlighting}[]
graph TD
    A[વેબ એનાલિટિક્સ] {-{-}{} B[વ્યૂહરચના ઘડતર]}
    B {-{-}{} C[કન્ટેન્ટ એનાલિટિક્સ]}
    B {-{-}{} D[કસ્ટમર એનાલિટિક્સ]}
    B {-{-}{} E[સોશિયલ એનાલિટિક્સ]}
    B {-{-}{} F[SEO એનાલિટિક્સ]}
    B {-{-}{} G[કન્વર્શન એનાલિટિક્સ]}
{Highlighting}
{Shaded}
\end{verbatim}
\end{center}

\begin{itemize}
\tightlist
\item
  \textbf{વ્યૂહરચનાત્મક ભૂમિકા}: તકો ઓળખે છે, પરફોર્મન્સ માપે છે, ઑપ્ટિમાઇઝેશન
  માર્ગદર્શન આપે છે
\item
  \textbf{કન્ટેન્ટ એનાલિટિક્સ}: એન્ગેજમેન્ટ આધારે કન્ટેન્ટ વ્યૂહરચના ઑપ્ટિમાઇઝ કરવામાં
  મદદ કરે છે
\item
  \textbf{કસ્ટમર એનાલિટિક્સ}: વધુ સારું ઓડિયન્સ ટાર્ગેટિંગ અને વ્યક્તિગતકરણ સક્ષમ કરે
  છે
\item
  \textbf{સોશિયલ મીડિયા એનાલિટિક્સ}: સોશિયલ મીડિયા ROI અને એન્ગેજમેન્ટ માપે છે
\item
  \textbf{SEO એનાલિટિક્સ}: ઓર્ગેનિક સર્ચ પરફોર્મન્સ અને તકો ટ્રેક કરે છે
\item
  \textbf{કન્વર્શન એનાલિટિક્સ}: માર્કેટિંગ પ્રયાસોની બોટમ-લાઇન અસર માપે છે
\end{itemize}

\end{solutionbox}
\begin{mnemonicbox}
``CCSSC - Content, Customer, Social, SEO,
Conversion''

\end{mnemonicbox}
\subsection*{પ્રશ્ન 3(અ OR) [3
ગુણ]}\label{uxaaauxab0uxab6uxaa8-3uxa85-or-3-uxa97uxaa3}

\textbf{Unique visitors, Average Visit Duration, Bounce rate ની વ્યાખ્યા
આપો.}

\begin{solutionbox}

{\def\LTcaptype{none} % do not increment counter
\begin{longtable}[]{@{}ll@{}}
\toprule\noalign{}
મેટ્રિક & વ્યાખ્યા \\
\midrule\noalign{}
\endhead
\bottomrule\noalign{}
\endlastfoot
\textbf{યુનિક વિઝિટર્સ} & વિશિષ્ટ સમયગાળામાં સાઇટની મુલાકાત લેતા વ્યક્તિગત
યુઝર્સ \\
\textbf{એવરેજ વિઝિટ ડ્યુરેશન} & પ્રતિ સેશન યુઝર્સ વેબસાઇટ પર વિતાવતો સરેરાશ સમય \\
\textbf{બાઉન્સ રેટ} & એક પેજ જોયા પછી છોડી જનારા વિઝિટર્સની ટકાવારી \\
\end{longtable}
}

\begin{itemize}
\tightlist
\item
  \textbf{યુનિક વિઝિટર્સ}: પુનઃ મુલાકાતોને ધ્યાનમાં લીધા વગર દરેક વ્યક્તિને એકવાર
  ગણે છે
\item
  \textbf{એવરેજ વિઝિટ ડ્યુરેશન}: કન્ટેન્ટ એન્ગેજમેન્ટ અને સાઇટ સ્ટિકિનેસ દર્શાવે છે
\item
  \textbf{બાઉન્સ રેટ}: ઉચ્ચ દર ખરાબ કન્ટેન્ટ મેચ અથવા સાઇટ સમસ્યાઓ સૂચવી શકે છે
\end{itemize}

\end{solutionbox}
\begin{mnemonicbox}
``UAB - Unique, Average, Bounce''

\end{mnemonicbox}
\subsection*{પ્રશ્ન 3(બ OR) [4
ગુણ]}\label{uxaaauxab0uxab6uxaa8-3uxaac-or-4-uxa97uxaa3}

\textbf{વેબ એનાલિટિક્સમાં A/B testing વિશે સમજાવો.}

\begin{solutionbox}

\textbf{A/B Testing} એટલે કયું વધુ સારું પરફોર્મ કરે છે તે નક્કી કરવા માટે વેબપેજના બે
વર્ઝનની તુલના કરવી.

{\def\LTcaptype{none} % do not increment counter
\begin{longtable}[]{@{}ll@{}}
\toprule\noalign{}
ઘટક & વર્ણન \\
\midrule\noalign{}
\endhead
\bottomrule\noalign{}
\endlastfoot
\textbf{વર્ઝન A} & મૂળ વેબપેજ (કંટ્રોલ) \\
\textbf{વર્ઝન B} & સુધારેલ વેબપેજ (વેરિઅન્ટ) \\
\textbf{ટ્રાફિક સ્પ્લિટ} & સામાન્ય રીતે 50/50 રેન્ડમ વિતરણ \\
\textbf{મેટ્રિક્સ} & કન્વર્શન રેટ, ક્લિક-થ્રુ રેટ, એન્ગેજમેન્ટ \\
\end{longtable}
}

\begin{itemize}
\tightlist
\item
  \textbf{પ્રક્રિયા}: બે વર્ઝન વચ્ચે ટ્રાફિક વિભાજિત કરીને પરફોર્મન્સ માપો
\item
  \textbf{અવધિ}: આંકડાકીય મહત્વ માટે પૂરતા લાંબા સમય સુધી ટેસ્ટ ચલાવો
\item
  \textbf{વેરિએબલ્સ}: એક સમયે એક તત્વ ટેસ્ટ કરો (હેડલાઇન્સ, બટન્સ, ઇમેજો)
\item
  \textbf{નિર્ણય}: ડેટા આધારે જીતનાર વર્ઝન અમલ કરો
\end{itemize}

\end{solutionbox}
\begin{mnemonicbox}
``ABCD - A-version, B-version, Compare, Decide''

\end{mnemonicbox}
\subsection*{પ્રશ્ન 3(ક OR) [7
ગુણ]}\label{uxaaauxab0uxab6uxaa8-3uxa95-or-7-uxa97uxaa3}

\textbf{નીચેમુજબના ટ્રેકિંગ કોડના ફાયદા અને ગેરફાયદાઓ સમજાવો: Long tracking
code, Obfuscated tracking code, UTM codes}

\begin{solutionbox}

{\def\LTcaptype{none} % do not increment counter
\begin{longtable}[]{@{}
  >{\raggedright\arraybackslash}p{(\linewidth - 6\tabcolsep) * \real{0.3889}}
  >{\raggedright\arraybackslash}p{(\linewidth - 6\tabcolsep) * \real{0.1667}}
  >{\raggedright\arraybackslash}p{(\linewidth - 6\tabcolsep) * \real{0.1944}}
  >{\raggedright\arraybackslash}p{(\linewidth - 6\tabcolsep) * \real{0.2500}}@{}}
\toprule\noalign{}
\begin{minipage}[b]{\linewidth}\raggedright
ટ્રેકિંગ પ્રકાર
\end{minipage} & \begin{minipage}[b]{\linewidth}\raggedright
વર્ણન
\end{minipage} & \begin{minipage}[b]{\linewidth}\raggedright
ફાયદા
\end{minipage} & \begin{minipage}[b]{\linewidth}\raggedright
ગેરફાયદા
\end{minipage} \\
\midrule\noalign{}
\endhead
\bottomrule\noalign{}
\endlastfoot
\textbf{લોંગ ટ્રેકિંગ કોડ} & વ્યાપક ટ્રેકિંગ માટે વિગતવાર પેરામીટર્સ & સંપૂર્ણ ડેટા
સંગ્રહ, વિગતવાર આંતરદૃષ્ટિ & ધીમી પેજ લોડ, જટિલ અમલીકરણ \\
\textbf{ઓબ્ફસ્કેટેડ ટ્રેકિંગ} & એન્ક્રિપ્ટેડ/છુપાયેલ ટ્રેકિંગ પેરામીટર્સ & ડેટા સુરક્ષા,
હેરાફેરીથી અટકાવે છે & કઠિન ડિબગિંગ, જટિલ સેટઅપ \\
\textbf{UTM કોડ્સ} & કેમ્પેઇન ટ્રેકિંગ માટે URL પેરામીટર્સ & સરળ અમલીકરણ, કેમ્પેઇન
એટ્રિબ્યુશન & મેન્યુઅલ ટેગિંગ જરૂરી, URL દેખાવ \\
\end{longtable}
}

\begin{center}
\textbf{Mermaid Diagram (Code)}
\begin{verbatim}
{Shaded}
{Highlighting}[]
graph TD
    A[ટ્રેકિંગ કોડ્સ] {-{-}{} B[લોંગ ટ્રેકિંગ]}
    A {-{-}{} C[ઓબ્ફસ્કેટેડ ટ્રેકિંગ]}
    A {-{-}{} D[UTM કોડ્સ]}
    B {-{-}{} E[વ્યાપક ડેટા]}
    C {-{-}{} F[સુરક્ષિત ટ્રેકિંગ]}
    D {-{-}{} G[કેમ્પેઇન એટ્રિબ્યુશન]}
{Highlighting}
{Shaded}
\end{verbatim}
\end{center}

\begin{itemize}
\tightlist
\item
  \textbf{લોંગ ટ્રેકિંગ કોડ}: એન્ટરપ્રાઇઝ-લેવલ વિગતવાર એનાલિટિક્સ માટે શ્રેષ્ઠ
\item
  \textbf{ઓબ્ફસ્કેટેડ ટ્રેકિંગ}: સંવેદનશીલ ડેટા સુરક્ષા આવશ્યકતાઓ માટે આદર્શ
\item
  \textbf{UTM કોડ્સ}: કેમ્પેઇન ટ્રેકિંગ અને ટ્રાફિક સોર્સ ઓળખ માટે સંપૂર્ણ
\end{itemize}

\end{solutionbox}
\begin{mnemonicbox}
``LOU - Long comprehensive, Obfuscated secure, UTM
simple''

\end{mnemonicbox}
\subsection*{પ્રશ્ન 4(અ) [3
ગુણ]}\label{uxaaauxab0uxab6uxaa8-4uxa85-3-uxa97uxaa3}

\textbf{વિવિધ પ્રકારની YouTube ads સમજાવો.}

\begin{solutionbox}

{\def\LTcaptype{none} % do not increment counter
\begin{longtable}[]{@{}lll@{}}
\toprule\noalign{}
એડ પ્રકાર & ફોર્મેટ & પ્લેસમેન્ટ \\
\midrule\noalign{}
\endhead
\bottomrule\noalign{}
\endlastfoot
\textbf{સ્કિપેબલ ઇન-સ્ટ્રીમ} & 5-સેકન્ડ સ્કિપ વિકલ્પ & વિડિયો પહેલાં/દરમિયાન \\
\textbf{નોન-સ્કિપેબલ} & 15-20 સેકન્ડ, સ્કિપ નહીં & વિડિયો પહેલાં/દરમિયાન \\
\textbf{બમ્પર એડ્સ} & 6 સેકન્ડ, નોન-સ્કિપેબલ & વિડિયો પહેલાં \\
\end{longtable}
}

\begin{itemize}
\tightlist
\item
  \textbf{સ્કિપેબલ ઇન-સ્ટ્રીમ}: કિફાયતી, માત્ર એન્ગેજ્ડ વ્યૂઅર્સ માટે ચૂકવણી
\item
  \textbf{નોન-સ્કિપેબલ}: ગેરંટીડ મેસેજ ડિલિવરી, વધુ કમ્પ્લીશન રેટ
\item
  \textbf{બમ્પર એડ્સ}: બ્રાન્ડ અવેરનેસ, ઝડપી યાદગાર મેસેજ
\end{itemize}

\end{solutionbox}
\begin{mnemonicbox}
``SNB - Skippable, Non-skippable, Bumper''

\end{mnemonicbox}
\subsection*{પ્રશ્ન 4(બ) [4
ગુણ]}\label{uxaaauxab0uxab6uxaa8-4uxaac-4-uxa97uxaa3}

\textbf{LinkedIn marketing સમજાવો અને ડિજિટલ માર્કેટિંગ માં તેનું શું મહત્વ છે એના
વિશે ચર્ચા કરો.}

\begin{solutionbox}

\textbf{LinkedIn Marketing} વ્યાવસાયિક નેટવર્કિંગ અને B2B રિલેશનશિપ બિલ્ડિંગ પર
ધ્યાન કેન્દ્રિત કરે છે.

{\def\LTcaptype{none} % do not increment counter
\begin{longtable}[]{@{}ll@{}}
\toprule\noalign{}
પાસું & મહત્વ \\
\midrule\noalign{}
\endhead
\bottomrule\noalign{}
\endlastfoot
\textbf{વ્યાવસાયિક ઓડિયન્સ} & નિર્ણય લેનારા અને ઉદ્યોગ વ્યાવસાયિકો \\
\textbf{B2B ફોકસ} & બિઝનેસ-ટુ-બિઝનેસ માર્કેટિંગ માટે આદર્શ \\
\textbf{કન્ટેન્ટ ઓથોરિટી} & વિચારધારાનું નેતૃત્વ સ્થાપિત કરે છે \\
\textbf{નેટવર્કિંગ} & મુખ્ય વ્યાવસાયિક સંપર્કોની સીધી પહોંચ \\
\end{longtable}
}

\begin{itemize}
\tightlist
\item
  \textbf{વ્યાવસાયિક ઓડિયન્સ}: ઉચ્ચ આવક, શિક્ષિત ડેમોગ્રાફિક્સ
\item
  \textbf{B2B ફોકસ}: LinkedIn માંથી 80\% B2B લીડ્સ આવે છે
\item
  \textbf{કન્ટેન્ટ ઓથોરિટી}: ઉદ્યોગ આંતરદૃષ્ટિ અને નિપુણતા શેર કરો
\item
  \textbf{નેટવર્કિંગ}: મૂલ્યવાન વ્યાવસાયિક સંબંધો બનાવો
\end{itemize}

\end{solutionbox}
\begin{mnemonicbox}
``PBCN - Professional, B2B, Content, Networking''

\end{mnemonicbox}
\subsection*{પ્રશ્ન 4(ક) [7
ગુણ]}\label{uxaaauxab0uxab6uxaa8-4uxa95-7-uxa97uxaa3}

\textbf{Organic and Paid social media marketing strategies વચ્ચે મહત્વના
તફાવત વર્ણવી દરેક strategy ના બે ફાયદાઓ અને ગેરફાયદાઓ વર્ણવો.}

\begin{solutionbox}

{\def\LTcaptype{none} % do not increment counter
\begin{longtable}[]{@{}
  >{\raggedright\arraybackslash}p{(\linewidth - 6\tabcolsep) * \real{0.3030}}
  >{\raggedright\arraybackslash}p{(\linewidth - 6\tabcolsep) * \real{0.1818}}
  >{\raggedright\arraybackslash}p{(\linewidth - 6\tabcolsep) * \real{0.2121}}
  >{\raggedright\arraybackslash}p{(\linewidth - 6\tabcolsep) * \real{0.3030}}@{}}
\toprule\noalign{}
\begin{minipage}[b]{\linewidth}\raggedright
વ્યૂહરચના
\end{minipage} & \begin{minipage}[b]{\linewidth}\raggedright
વર્ણન
\end{minipage} & \begin{minipage}[b]{\linewidth}\raggedright
ફાયદાઓ
\end{minipage} & \begin{minipage}[b]{\linewidth}\raggedright
ગેરફાયદાઓ
\end{minipage} \\
\midrule\noalign{}
\endhead
\bottomrule\noalign{}
\endlastfoot
\textbf{ઓર્ગેનિક} & મફત કન્ટેન્ટ પોસ્ટિંગ અને એન્ગેજમેન્ટ & • કિફાયતી• અધિકૃત સંબંધો
બનાવે છે & • મર્યાદિત પહોંચ• સમય-સઘન \\
\textbf{પેઇડ} & સ્પોન્સર્ડ કન્ટેન્ટ અને જાહેરાતો & • તાત્કાલિક પહોંચ• ચોક્કસ ટાર્ગેટિંગ
& • બજેટ જરૂરી• અસ્થાયી પરિણામો \\
\end{longtable}
}

\begin{center}
\textbf{Mermaid Diagram (Code)}
\begin{verbatim}
{Shaded}
{Highlighting}[]
graph TD
    A[સોશિયલ મીડિયા માર્કેટિંગ] {-{-}{} B[ઓર્ગેનિક વ્યૂહરચના]}
    A {-{-}{} C[પેઇડ વ્યૂહરચના]}
    B {-{-}{} D[કિફાયતી]}
    B {-{-}{} E[અધિકૃત સંબંધો]}
    C {-{-}{} F[તાત્કાલિક પહોંચ]}
    C {-{-}{} G[ચોક્કસ ટાર્ગેટિંગ]}
{Highlighting}
{Shaded}
\end{verbatim}
\end{center}

\textbf{ઓર્ગેનિક ફાયદાઓ:}

\begin{itemize}
\tightlist
\item
  \textbf{કિફાયતી}: કોઈ જાહેરાત ખર્ચ જરૂરી નથી
\item
  \textbf{અધિકૃત સંબંધો બનાવે છે}: સાચું કોમ્યુનિટી એન્ગેજમેન્ટ
\end{itemize}

\textbf{ઓર્ગેનિક ગેરફાયદાઓ:}

\begin{itemize}
\tightlist
\item
  \textbf{મર્યાદિત પહોંચ}: અલ્ગોરિધમ પ્રતિબંધો દૃશ્યતા ઘટાડે છે
\item
  \textbf{સમય-સઘન}: સતત કન્ટેન્ટ નિર્માણ અને એન્ગેજમેન્ટ જરૂરી
\end{itemize}

\textbf{પેઇડ ફાયદાઓ:}

\begin{itemize}
\tightlist
\item
  \textbf{તાત્કાલિક પહોંચ}: લક્ષિત પ્રેક્ષકો માટે તાત્કાલિક દૃશ્યતા
\item
  \textbf{ચોક્કસ ટાર્ગેટિંગ}: એડવાન્સ ડેમોગ્રાફિક અને રુચિ ટાર્ગેટિંગ
\end{itemize}

\textbf{પેઇડ ગેરફાયદાઓ:}

\begin{itemize}
\tightlist
\item
  \textbf{બજેટ જરૂરી}: ચાલુ જાહેરાત ખર્ચ
\item
  \textbf{અસ્થાયી પરિણામો}: જાહેરાત બંધ થાય ત્યારે પરિણામો બંધ થાય છે
\end{itemize}

\end{solutionbox}
\begin{mnemonicbox}
``OPAL - Organic Patient Authentic Low-cost, Paid
Quick Targeted Expensive''

\end{mnemonicbox}
\subsection*{પ્રશ્ન 4(અ OR) [3
ગુણ]}\label{uxaaauxab0uxab6uxaa8-4uxa85-or-3-uxa97uxaa3}

\textbf{વિવિધ પ્રકારની Twitter ads કઈ કઈ છે અને કોઈપણ એક Ads નો પ્રકાર
વિસ્તારપૂર્વક સમજાવો.}

\begin{solutionbox}

{\def\LTcaptype{none} % do not increment counter
\begin{longtable}[]{@{}ll@{}}
\toprule\noalign{}
એડ પ્રકાર & હેતુ \\
\midrule\noalign{}
\endhead
\bottomrule\noalign{}
\endlastfoot
\textbf{પ્રોમોટેડ ટ્વીટ્સ} & ટ્વીટ દૃશ્યતા વધારવી \\
\textbf{પ્રોમોટેડ એકાઉન્ટ્સ} & વધુ ફોલોઅર્સ મેળવવા \\
\textbf{પ્રોમોટેડ ટ્રેન્ડ્સ} & ટ્રેન્ડિંગ ટોપિક્સને બૂસ્ટ કરવા \\
\end{longtable}
}

\textbf{પ્રોમોટેડ ટ્વીટ્સ}: નિયમિત ટ્વીટ્સ કે જેના માટે વ્યવસાયો તેમના ફોલોઅર્સથી
આગળ વ્યાપક પ્રેક્ષકોને બતાવવા પૈસા ચૂકવે છે, યુઝર્સના ટાઇમલાઇન અને સર્ચ પરિણામોમાં
``Promoted'' લેબલ સાથે દેખાય છે.

\end{solutionbox}
\begin{mnemonicbox}
``PAT - Promoted tweets, Accounts, Trends''

\end{mnemonicbox}
\subsection*{પ્રશ્ન 4(બ OR) [4
ગુણ]}\label{uxaaauxab0uxab6uxaa8-4uxaac-or-4-uxa97uxaa3}

\textbf{સેમસંગ કંપનીએ નવો સ્માર્ટફોન માર્કેટમાં વેચાણ માટે મૂક્યો છે અને YouTube ads
ચલાવવા માંગે છે. સોશિયલ મીડિયા માર્કેટિંગના નિષ્ણાત તરીકે તમે કઈ YouTube ads
ફોર્મેટ પસંદ કરશો અને શા માટે એ સમજાવો.}

\begin{solutionbox}

\textbf{ભલામણ કરેલ ફોર્મેટ: સ્કિપેબલ ઇન-સ્ટ્રીમ એડ્સ}

{\def\LTcaptype{none} % do not increment counter
\begin{longtable}[]{@{}ll@{}}
\toprule\noalign{}
કારણ & ફાયદો \\
\midrule\noalign{}
\endhead
\bottomrule\noalign{}
\endlastfoot
\textbf{કિફાયતી} & માત્ર જ્યારે યુઝર્સ 30+ સેકન્ડ જુએ ત્યારે જ ચૂકવણી \\
\textbf{પ્રોડક્ટ ડેમોન્સ્ટ્રેશન} & લાંબું ફોર્મેટ ફીચર શોકેસ કરવાની મંજૂરી આપે છે \\
\textbf{ઓડિયન્સ રુચિ} & સ્કિપ વિકલ્પ એન્ગેજ્ડ વ્યૂઅર્સની ખાતરી કરે છે \\
\textbf{બ્રાન્ડ અવેરનેસ} & સ્માર્ટફોન રુચિ સાથે વ્યાપક પ્રેક્ષકો સુધી પહોંચે છે \\
\end{longtable}
}

\begin{itemize}
\tightlist
\item
  \textbf{પ્રોડક્ટ ડેમોન્સ્ટ્રેશન}: સ્માર્ટફોન્સને ફીચર્સના વિઝ્યુઅલ ડેમોન્સ્ટ્રેશનની જરૂર છે
\item
  \textbf{ઓડિયન્સ રુચિ}: સ્કિપ વિકલ્પ સાચામાં રુચિ ધરાવતા વ્યૂઅર્સને ફિલ્ટર કરે છે
\item
  \textbf{કિફાયતી}: માત્ર એન્ગેજ્ડ વ્યૂઅર્સ માટે જ ચૂકવણી કરો જેઓ 30 સેકન્ડથી વધુ જુએ
  છે
\item
  \textbf{બ્રાન્ડ અવેરનેસ}: નવા પ્રોડક્ટ લોન્ચ માટે વ્યાપક પહોંચ
\end{itemize}

\end{solutionbox}
\begin{mnemonicbox}
``PCAB - Product demo, Cost-effective, Audience
interest, Brand awareness''

\end{mnemonicbox}
\subsection*{પ્રશ્ન 4(ક OR) [7
ગુણ]}\label{uxaaauxab0uxab6uxaa8-4uxa95-or-7-uxa97uxaa3}

\textbf{Facebook Page, Business Manager અને Facebook Ads નું મુખ્ય કાર્ય
વિસ્તારપૂર્વક વર્ણવો. આ ત્રણ assets તમારા વ્યવસાયના માર્કેટિંગમાં મદદરૂપ થઇ શકે છે?}

\begin{solutionbox}

{\def\LTcaptype{none} % do not increment counter
\begin{longtable}[]{@{}
  >{\raggedright\arraybackslash}p{(\linewidth - 4\tabcolsep) * \real{0.1714}}
  >{\raggedright\arraybackslash}p{(\linewidth - 4\tabcolsep) * \real{0.3429}}
  >{\raggedright\arraybackslash}p{(\linewidth - 4\tabcolsep) * \real{0.4857}}@{}}
\toprule\noalign{}
\begin{minipage}[b]{\linewidth}\raggedright
સંપત્તિ
\end{minipage} & \begin{minipage}[b]{\linewidth}\raggedright
મુખ્ય કાર્યો
\end{minipage} & \begin{minipage}[b]{\linewidth}\raggedright
માર્કેટિંગ ફાયદાઓ
\end{minipage} \\
\midrule\noalign{}
\endhead
\bottomrule\noalign{}
\endlastfoot
\textbf{Facebook Page} & • બ્રાન્ડ હાજરી• કન્ટેન્ટ શેરિંગ• કસ્ટમર એન્ગેજમેન્ટ & •
બ્રાન્ડ અવેરનેસ બનાવે છે• સીધો કસ્ટમર કોમ્યુનિકેશન \\
\textbf{Business Manager} & • એકાઉન્ટ મેનેજમેન્ટ• ટીમ એક્સેસ કંટ્રોલ• સંપત્તિ સંગઠન
& • કેન્દ્રીકૃત નિયંત્રણ• સુરક્ષિત સહયોગ \\
\textbf{Facebook Ads} & • લક્ષિત જાહેરાત• કેમ્પેઇન મેનેજમેન્ટ• પરફોર્મન્સ ટ્રેકિંગ & •
ચોક્કસ પ્રેક્ષક ટાર્ગેટિંગ• માપવા યોગ્ય ROI \\
\end{longtable}
}

\begin{center}
\textbf{Mermaid Diagram (Code)}
\begin{verbatim}
{Shaded}
{Highlighting}[]
graph TD
    A[Facebook માર્કેટિંગ સંપત્તિઓ] {-{-}{} B[Facebook Page]}
    A {-{-}{} C[Business Manager]}
    A {-{-}{} D[Facebook Ads]}
    B {-{-}{} E[બ્રાન્ડ હાજરી]}
    C {-{-}{} F[એકાઉન્ટ મેનેજમેન્ટ]}
    D {-{-}{} G[લક્ષિત જાહેરાત]}
{Highlighting}
{Shaded}
\end{verbatim}
\end{center}

\textbf{માર્કેટિંગ ફાયદાઓ:}

\begin{itemize}
\tightlist
\item
  \textbf{Facebook Page}: વ્યાવસાયિક બ્રાન્ડ હાજરી બનાવે છે અને ઓર્ગેનિક પહોંચ
  સક્ષમ કરે છે
\item
  \textbf{Business Manager}: બહુવિધ એકાઉન્ટ્સ અને ટીમ સભ્યો માટે સુરક્ષા અને સંગઠન
  પ્રદાન કરે છે
\item
  \textbf{Facebook Ads}: વિગતવાર એનાલિટિક્સ અને ROI ટ્રેકિંગ સાથે લક્ષિત કેમ્પેઇન્સ
  પહોંચાડે છે
\end{itemize}

\textbf{એકીકરણ ફાયદાઓ:}

\begin{itemize}
\tightlist
\item
  \textbf{યુનિફાઇડ વ્યૂહરચના}: ત્રણેય વ્યાપક Facebook માર્કેટિંગ માટે મળીને કામ કરે
  છે
\item
  \textbf{ડેટા શેરિંગ}: પેજના પિક્સેલ ડેટા એડ ટાર્ગેટિંગ વધારે છે
\item
  \textbf{બ્રાન્ડ સુસંગતતા}: ઓર્ગેનિક અને પેઇડ કન્ટેન્ટમાં સુસંગત મેસેજિંગ
\end{itemize}

\end{solutionbox}
\begin{mnemonicbox}
``PMA - Page presence, Manager control, Ads
targeting''

\end{mnemonicbox}
\subsection*{પ્રશ્ન 5(અ) [3
ગુણ]}\label{uxaaauxab0uxab6uxaa8-5uxa85-3-uxa97uxaa3}

\textbf{વિવિધ Instagram Content અને જાહેરાતોના પ્રકારોની યાદી બનાવો.}

\begin{solutionbox}

{\def\LTcaptype{none} % do not increment counter
\begin{longtable}[]{@{}ll@{}}
\toprule\noalign{}
કન્ટેન્ટ પ્રકારો & એડ પ્રકારો \\
\midrule\noalign{}
\endhead
\bottomrule\noalign{}
\endlastfoot
\textbf{પોસ્ટ્સ} & Photo Ads \\
\textbf{સ્ટોરીઝ} & Video Ads \\
\textbf{રીલ્સ} & Carousel Ads \\
\textbf{IGTV} & Stories Ads \\
\textbf{લાઇવ} & Reels Ads \\
\end{longtable}
}

\begin{itemize}
\tightlist
\item
  \textbf{કન્ટેન્ટ પ્રકારો}: ઓર્ગેનિક એન્ગેજમેન્ટ માટે વિવિધ ફોર્મેટ્સ
\item
  \textbf{એડ પ્રકારો}: ટાર્ગેટિંગ ક્ષમતાઓ સાથે સ્પોન્સર્ડ વર્ઝન્સ
\item
  \textbf{એકીકરણ}: એડ્સ ઓર્ગેનિક કન્ટેન્ટ સાથે કુદરતી રીતે ભળે છે
\end{itemize}

\end{solutionbox}
\begin{mnemonicbox}
``PSRIL - Posts, Stories, Reels, IGTV, Live''

\end{mnemonicbox}
\subsection*{પ્રશ્ન 5(બ) [4
ગુણ]}\label{uxaaauxab0uxab6uxaa8-5uxaac-4-uxa97uxaa3}

\textbf{ઈ-મેઈલ માર્કેટિંગ શું છે? વિવિભન્ન ઈ-મેઈલ માર્કેટિંગના પ્રકાર કયા છે?}

\begin{solutionbox}

\textbf{ઇમેઇલ માર્કેટિંગ} એટલે વ્યક્તિગત ઇમેઇલ સંદેશાઓ દ્વારા ગ્રાહકો સાથે સીધો
ડિજિટલ કોમ્યુનિકેશન.

{\def\LTcaptype{none} % do not increment counter
\begin{longtable}[]{@{}lll@{}}
\toprule\noalign{}
પ્રકાર & હેતુ & ઉદાહરણ \\
\midrule\noalign{}
\endhead
\bottomrule\noalign{}
\endlastfoot
\textbf{ન્યૂઝલેટર} & નિયમિત અપડેટ્સ અને માહિતી & માસિક કંપની સમાચાર \\
\textbf{પ્રમોશનલ} & વેચાણ અને ઓફર્સ & ડિસ્કાઉન્ટ કોડ્સ, નવા પ્રોડક્ટ્સ \\
\textbf{ટ્રાન્ઝેક્શનલ} & ખરીદી પુષ્ટિકરણ & ઓર્ડર રસીદો, શિપિંગ અપડેટ્સ \\
\textbf{વેલકમ સિરીઝ} & નવા સબ્સ્ક્રાઇબર ઓનબોર્ડિંગ & બ્રાન્ડ અને પ્રોડક્ટ્સનો
પરિચય \\
\end{longtable}
}

\begin{itemize}
\tightlist
\item
  \textbf{ન્યૂઝલેટર}: મૂલ્યવાન કન્ટેન્ટ દ્વારા સંબંધો બનાવે છે
\item
  \textbf{પ્રમોશનલ}: વેચાણ અને કન્વર્શન ચલાવે છે
\item
  \textbf{ટ્રાન્ઝેક્શનલ}: આવશ્યક કસ્ટમર સર્વિસ માહિતી પ્રદાન કરે છે
\item
  \textbf{વેલકમ સિરીઝ}: નવા સબ્સ્ક્રાઇબર્સને ગ્રાહકોમાં રૂપાંતરિત કરે છે
\end{itemize}

\end{solutionbox}
\begin{mnemonicbox}
``NPTW - Newsletter, Promotional, Transactional,
Welcome''

\end{mnemonicbox}
\subsection*{પ્રશ્ન 5(ક) [7
ગુણ]}\label{uxaaauxab0uxab6uxaa8-5uxa95-7-uxa97uxaa3}

\textbf{Google Ads માટે ઉપલબ્ધ વિવિધ પ્રકારના Ad extensions ઉદાહરણ સાથે
સમજાવો.}

\begin{solutionbox}

{\def\LTcaptype{none} % do not increment counter
\begin{longtable}[]{@{}
  >{\raggedright\arraybackslash}p{(\linewidth - 4\tabcolsep) * \real{0.5484}}
  >{\raggedright\arraybackslash}p{(\linewidth - 4\tabcolsep) * \real{0.1935}}
  >{\raggedright\arraybackslash}p{(\linewidth - 4\tabcolsep) * \real{0.2581}}@{}}
\toprule\noalign{}
\begin{minipage}[b]{\linewidth}\raggedright
એક્સટેન્શન પ્રકાર
\end{minipage} & \begin{minipage}[b]{\linewidth}\raggedright
કાર્ય
\end{minipage} & \begin{minipage}[b]{\linewidth}\raggedright
ઉદાહરણ
\end{minipage} \\
\midrule\noalign{}
\endhead
\bottomrule\noalign{}
\endlastfoot
\textbf{સાઇટલિંક એક્સટેન્શન્સ} & વધારાના પેજ લિંક્સ & ``અમારા વિશે'', ``સંપર્ક'',
``પ્રોડક્ટ્સ'' \\
\textbf{કોલ એક્સટેન્શન્સ} & ફોન નંબર ડિસ્પ્લે & ``+91-800-123-4567'' \\
\textbf{લોકેશન એક્સટેન્શન્સ} & વ્યવસાયિક સરનામું & ``123 મુખ્ય સ્ટ્રીટ, શહેર,
રાજ્ય'' \\
\textbf{કોલઆઉટ એક્સટેન્શન્સ} & ફીચર્સ હાઇલાઇટ & ``મફત શિપિંગ'', ``24/7
સહાય'' \\
\textbf{પ્રાઇસ એક્સટેન્શન્સ} & પ્રોડક્ટ/સર્વિસ કિંમત & ``બેસિક પ્લાન:
₹1900/મહિનો'' \\
\textbf{એપ એક્સટેન્શન્સ} & મોબાઇલ એપ ડાઉનલોડ્સ & ``અમારી iOS/Android એપ
ડાઉનલોડ કરો'' \\
\end{longtable}
}

\begin{center}
\textbf{Mermaid Diagram (Code)}
\begin{verbatim}
{Shaded}
{Highlighting}[]
graph TD
    A[Google Ad Extensions] {-{-}{} B[સાઇટલિંક એક્સટેન્શન્સ]}
    A {-{-}{} C[કોલ એક્સટેન્શન્સ]}
    A {-{-}{} D[લોકેશન એક્સટેન્શન્સ]}
    A {-{-}{} E[કોલઆઉટ એક્સટેન્શન્સ]}
    A {-{-}{} F[પ્રાઇસ એક્સટેન્શન્સ]}
    A {-{-}{} G[એપ એક્સટેન્શન્સ]}
{Highlighting}
{Shaded}
\end{verbatim}
\end{center}

\textbf{ફાયદાઓ:}

\begin{itemize}
\tightlist
\item
  \textbf{વધારેલ CTR}: એક્સટેન્શન્સ એડ્સને વધુ આકર્ષક અને માહિતીપ્રદ બનાવે છે
\item
  \textbf{બેહતર ક્વોલિટી સ્કોર}: સુધારેલ એડ પરફોર્મન્સ ઓછા ખર્ચ તરફ દોરી જાય છે
\item
  \textbf{વધારેલ યુઝર એક્સપિરિયન્સ}: યુઝર્સને વધુ સંબંધિત માહિતી મળે છે
\item
  \textbf{સ્પર્ધાત્મક લાભ}: સ્પર્ધકો કરતાં વધુ સ્ક્રીન રિયલ એસ્ટેટ
\end{itemize}

\textbf{અમલીકરણ:}

\begin{itemize}
\tightlist
\item
  \textbf{ઓટોમેટિક}: Google સંબંધિત એક્સટેન્શન્સ ઓટોમેટિક બતાવી શકે છે
\item
  \textbf{મેન્યુઅલ}: જાહેરાતકર્તાઓ વિશિષ્ટ એક્સટેન્શન્સ બનાવી અને કસ્ટમાઇઝ કરી શકે છે
\item
  \textbf{પરફોર્મન્સ}: અનુમાનિત અસર આધારે એક્સટેન્શન્સ બતાવવામાં આવે છે
\end{itemize}

\end{solutionbox}
\begin{mnemonicbox}
``SCLCPA - Sitelink, Call, Location, Callout, Price,
App''

\end{mnemonicbox}
\subsection*{પ્રશ્ન 5(અ OR) [3
ગુણ]}\label{uxaaauxab0uxab6uxaa8-5uxa85-or-3-uxa97uxaa3}

\textbf{Social media marketing નું મહત્વ અને ફાયદાઓ વર્ણવો.}

\begin{solutionbox}

{\def\LTcaptype{none} % do not increment counter
\begin{longtable}[]{@{}ll@{}}
\toprule\noalign{}
ફાયદો & અસર \\
\midrule\noalign{}
\endhead
\bottomrule\noalign{}
\endlastfoot
\textbf{બ્રાન્ડ અવેરનેસ} & દૃશ્યતા અને ઓળખ વધારે છે \\
\textbf{કસ્ટમર એન્ગેજમેન્ટ} & સીધો ઇન્ટરેક્શન અને રિલેશનશિપ બિલ્ડિંગ \\
\textbf{કિફાયતી} & પરંપરાગત જાહેરાતની તુલનામાં ઓછા ખર્ચે \\
\end{longtable}
}

\begin{itemize}
\tightlist
\item
  \textbf{બ્રાન્ડ અવેરનેસ}: શેરિંગ અને વાયરલ કન્ટેન્ટ દ્વારા ઘાતાંકીય પહોંચ
\item
  \textbf{કસ્ટમર એન્ગેજમેન્ટ}: રીઅલ-ટાઇમ ફીડબેક અને કોમ્યુનિટી બિલ્ડિંગ
\item
  \textbf{કિફાયતી}: લક્ષિત જાહેરાત વિકલ્પો સાથે ઉચ્ચ ROI
\end{itemize}

\end{solutionbox}
\begin{mnemonicbox}
``BEC - Brand awareness, Engagement,
Cost-effective''

\end{mnemonicbox}
\subsection*{પ્રશ્ન 5(બ OR) [4
ગુણ]}\label{uxaaauxab0uxab6uxaa8-5uxaac-or-4-uxa97uxaa3}

\textbf{PPC અને SEO વચ્ચેનો તફાવત આપો.}

\begin{solutionbox}

{\def\LTcaptype{none} % do not increment counter
\begin{longtable}[]{@{}lll@{}}
\toprule\noalign{}
પાસું & PPC (Pay-Per-Click) & SEO (Search Engine Optimization) \\
\midrule\noalign{}
\endhead
\bottomrule\noalign{}
\endlastfoot
\textbf{ખર્ચ} & પેઇડ જાહેરાત & ઓર્ગેનિક/મફત ટ્રાફિક \\
\textbf{પરિણામો} & તાત્કાલિક દૃશ્યતા & લાંબા ગાળાના ટકાઉ પરિણામો \\
\textbf{નિયંત્રણ} & એડ્સ પર સંપૂર્ણ નિયંત્રણ & રેન્કિંગ્સ પર મર્યાદિત નિયંત્રણ \\
\textbf{અવધિ} & ચૂકવણી બંધ થાય ત્યારે પરિણામો બંધ & લાંબા ગાળાના પરિણામો \\
\end{longtable}
}

\begin{itemize}
\tightlist
\item
  \textbf{PPC}: તાત્કાલિક પરિણામો પરંતુ ચાલુ રોકાણ જરૂરી
\item
  \textbf{SEO}: બનાવવામાં સમય લાગે છે પરંતુ ટકાઉ લાંબા ગાળાની વેલ્યુ પ્રદાન કરે છે
\item
  \textbf{એકીકરણ}: બંને વ્યૂહરચનાઓને જોડવાથી શ્રેષ્ઠ પરિણામો આવે છે
\item
  \textbf{બજેટ}: PPC ને જાહેરાત બજેટ; SEO ને સમય રોકાણ જરૂરી
\end{itemize}

\end{solutionbox}
\begin{mnemonicbox}
``ICRD - Immediate vs Continuous, Results vs
Duration''

\end{mnemonicbox}
\subsection*{પ્રશ્ન 5(ક OR) [7
ગુણ]}\label{uxaaauxab0uxab6uxaa8-5uxa95-or-7-uxa97uxaa3}

\textbf{ગૂગલ એડવર્ડ્સમાં ક્વોલિટી સ્કોર વિશે સમજાવો અને એડ રેન્કિંગ પર એની શું અસર થઇ
શકે?}

\begin{solutionbox}

\textbf{ક્વોલિટી સ્કોર} એટલે એડ ક્વોલિટી, કીવર્ડ્સ અને લેન્ડિંગ પેજનું Google નું રેટિંગ
(1-10).

{\def\LTcaptype{none} % do not increment counter
\begin{longtable}[]{@{}lll@{}}
\toprule\noalign{}
ઘટક & વેઇટ & અસર \\
\midrule\noalign{}
\endhead
\bottomrule\noalign{}
\endlastfoot
\textbf{એક્સપેક્ટેડ CTR} & ઉચ્ચ & યુઝર્સ ક્લિક કરશે તેની અનુમાનિત સંભાવના \\
\textbf{એડ રેલેવન્સ} & ઉચ્ચ & સર્ચ ઇન્ટેન્ટ સાથે એડ કેટલું નજીકથી મેચ કરે છે \\
\textbf{લેન્ડિંગ પેજ એક્સપિરિયન્સ} & મધ્યમ & પેજ ક્વોલિટી અને યુઝર એક્સપિરિયન્સ \\
\end{longtable}
}

\begin{center}
\textbf{Mermaid Diagram (Code)}
\begin{verbatim}
{Shaded}
{Highlighting}[]
graph TD
    A[ક્વોલિટી સ્કોર] {-{-}{} B[એક્સપેક્ટેડ CTR]}
    A {-{-}{} C[એડ રેલેવન્સ]}
    A {-{-}{} D[લેન્ડિંગ પેજ એક્સપિરિયન્સ]}
    B {-{-}{} E[એડ રેન્કિંગ]}
    C {-{-}{} F[કોસ્ટ પર ક્લિક]}
    D {-{-}{} G[એડ પોઝિશન]}
{Highlighting}
{Shaded}
\end{verbatim}
\end{center}

\textbf{એડ રેન્કિંગ્સ પર અસર:}

{\def\LTcaptype{none} % do not increment counter
\begin{longtable}[]{@{}lll@{}}
\toprule\noalign{}
ક્વોલિટી સ્કોર & એડ રેન્ક અસર & કોસ્ટ અસર \\
\midrule\noalign{}
\endhead
\bottomrule\noalign{}
\endlastfoot
\textbf{ઉચ્ચ (8-10)} & ઉચ્ચ પોઝિશન્સ & ઓછા CPC \\
\textbf{મધ્યમ (5-7)} & સરેરાશ પોઝિશન્સ & સરેરાશ CPC \\
\textbf{નીચા (1-4)} & ઓછા પોઝિશન્સ & વધારે CPC \\
\end{longtable}
}

\textbf{ઉચ્ચ ક્વોલિટી સ્કોરના ફાયદાઓ:}

\begin{itemize}
\tightlist
\item
  \textbf{ઓછા ખર્ચ}: સ્પર્ધકો કરતાં ક્લિક દીઠ ઓછું ચૂકવો
\item
  \textbf{બેહતર પોઝિશન્સ}: સર્ચ પરિણામોમાં ઉચ્ચ દેખાય છે
\item
  \textbf{વધારેલ દૃશ્યતા}: વધુ એડ એક્સટેન્શન પાત્રતા
\item
  \textbf{સુધારેલ ROI}: ઓછા ખર્ચે બેહતર પરફોર્મન્સ
\end{itemize}

\textbf{ઑપ્ટિમાઇઝેશન વ્યૂહરચનાઓ:}

\begin{itemize}
\tightlist
\item
  \textbf{કીવર્ડ રેલેવન્સ}: કીવર્ડ્સને એડ કોપી સાથે નજીકથી મેચ કરો
\item
  \textbf{એડ કોપી ક્વોલિટી}: આકર્ષક, સંબંધિત એડ ટેક્સ્ટ લખો
\item
  \textbf{લેન્ડિંગ પેજ}: ઝડપી, સંબંધિત, યુઝર-ફ્રેન્ડલી પેજની ખાતરી કરો
\item
  \textbf{એકાઉન્ટ સ્ટ્રક્ચર}: કેમ્પેઇન્સ અને એડ ગ્રુપ્સને તાર્કિક રીતે વ્યવસ્થિત કરો
\end{itemize}

\end{solutionbox}
\begin{mnemonicbox}
``EAL-RCP - Expected CTR, Ad relevance, Landing page
affect Rank, Cost, Position''

\end{mnemonicbox}

\end{document}
