\documentclass{article}
% Adjust the relative path to point to the latex-templates directory

% content/resources/templates/preamble.tex
\usepackage[margin=0.6in]{geometry}
\author{Milav Dabgar}
\usepackage{amsmath,amssymb,amsthm}
\usepackage{booktabs}
\usepackage{multirow}
\usepackage{xcolor}
\usepackage{tcolorbox}
\tcbuselibrary{breakable,skins}
\usepackage[colorlinks=true,linkcolor=blue]{hyperref}
\usepackage{titlesec}
\usepackage{enumitem}
\usepackage{tikz}
\usepackage{pgfplots}
\usepackage{circuitikz}
\usepackage[version=4]{mhchem}
\usepackage{longtable}
\usepackage{array}
\usepackage{float}
\usepackage{caption}
\usepackage{listings}

\lstset{
  basicstyle=\small\ttfamily,
  breaklines=true,
  breakatwhitespace=false,
  postbreak=\mbox{\textcolor{red}{$\hookrightarrow$}\space},
  float=false,
  numbers=left,
  numberstyle=\tiny\color{gray},
  numbersep=10pt,
  xleftmargin=2em,
  keywordstyle=\color{blue},
  commentstyle=\color{green!60!black},
  stringstyle=\color{purple},
  backgroundcolor=\color{gray!5},
  showstringspaces=false,
  tabsize=2,
  captionpos=b,
  keepspaces=true,
  columns=flexible
}

\pgfplotsset{compat=1.18}
\usetikzlibrary{shapes,arrows,positioning,calc,patterns,decorations.pathmorphing,decorations.markings,arrows.meta}

% Color scheme
\definecolor{headcolor}{RGB}{0,102,204}
\definecolor{keycolor}{RGB}{220,20,60}
\definecolor{solutioncolor}{RGB}{34,139,34}
\definecolor{mnemoniccolor}{RGB}{148,0,211}
\definecolor{codecolor}{RGB}{0,0,100}

% Spacing
\setlength{\parskip}{3pt}
\setlist[itemize]{nosep}
\setlist[enumerate]{nosep}

% Title formatting
\titleformat{\section}{\Large\bfseries\color{headcolor}}{\thesection}{1em}{}
\titleformat{\subsection}{\large\bfseries\color{headcolor}}{\thesubsection}{1em}{}

% Pandoc tightlist compatibility
\providecommand{\tightlist}{%
  \setlength{\itemsep}{0pt}\setlength{\parskip}{0pt}}

% Pandoc longtable compatibility
\newcounter{none}
\def\thenone{}


% content/resources/templates/english-boxes.tex
% This file is currently empty - it exists to maintain consistency with the import structure.
% Add custom environments here if needed in the future.


% Custom commands for GTU solutions
% This file defines semantic commands for consistent formatting

% Question command with automatic formatting
\newcommand{\question}[2]{%
  \section*{Question #1}%
  \textbf{#2}%
}

% OR question variant
\newcommand{\questionor}[2]{%
  \section*{Question #1 OR}%
  \textbf{#2}%
}

% Proper table environment with caption
\newenvironment{answertable}[1]{%
  \begin{table}[htbp]
  \centering
  \caption{#1}
}{%
  \end{table}
}

% Proper figure environment for diagrams
\newenvironment{answerdiagram}[1]{%
  \begin{figure}[htbp]
  \centering
  \caption{#1}
}{%
  \end{figure}
}

% Semantic markup for key terms
\newcommand{\keyword}[1]{\textbf{#1}}
\newcommand{\code}[1]{\texttt{#1}}
\newcommand{\classname}[1]{\texttt{#1}}
\newcommand{\methodname}[1]{\texttt{#1}}

% Proper quotation marks
\newcommand{\mnemonic}[1]{``#1''}


\title{Essentials of Digital Marketing (4341601) - Winter 2023 Solution}
\date{January 17, 2023}

\begin{document}
\maketitle

\questionmarks{1(a)}{3}{Describe the need of SEO in digital marketing.}

\begin{solutionbox}
SEO is essential in digital marketing for online visibility and business growth.

\begin{center}
\captionof{table}{Need for SEO}
\begin{tabulary}{\linewidth}{|L|L|}
\hline
\textbf{Need} & \textbf{Description} \\ \hline
\textbf{Visibility} & Helps websites appear in top search results \\ \hline
\textbf{Traffic} & Drives organic visitors without paid ads \\ \hline
\textbf{Credibility} & Higher rankings build trust with users \\ \hline
\textbf{Cost-effective} & Long-term results without continuous ad spending \\ \hline
\end{tabulary}
\end{center}

\begin{itemize}
    \item \textbf{Increased visibility}: SEO helps websites rank higher on search engines
    \item \textbf{Organic traffic}: Brings quality visitors without advertising costs
    \item \textbf{Brand credibility}: Top rankings establish business authority
\end{itemize}

\begin{mnemonicbox}VTC - Visibility, Traffic, Credibility\end{mnemonicbox}
\end{solutionbox}

\questionmarks{1(b)}{4}{Differentiate between traditional marketing and digital marketing.}

\begin{solutionbox}
Digital marketing offers targeted reach and measurable results compared to traditional methods.

\begin{center}
\captionof{table}{Traditional vs Digital Marketing}
\begin{tabulary}{\linewidth}{|L|L|L|}
\hline
\textbf{Aspect} & \textbf{Traditional Marketing} & \textbf{Digital Marketing} \\ \hline
\textbf{Reach} & Local/Regional & Global \\ \hline
\textbf{Cost} & High & Lower \\ \hline
\textbf{Targeting} & Broad audience & Specific demographics \\ \hline
\textbf{Measurement} & Difficult to track & Real-time analytics \\ \hline
\textbf{Interaction} & One-way communication & Two-way engagement \\ \hline
\end{tabulary}
\end{center}

\begin{itemize}
    \item \textbf{Cost efficiency}: Digital marketing requires lower investment
    \item \textbf{Real-time tracking}: Immediate performance measurement available
    \item \textbf{Global reach}: Access to worldwide audience instantly
\end{itemize}

\begin{mnemonicbox}GRIM - Global, Real-time, Interactive, Measurable\end{mnemonicbox}
\end{solutionbox}

\questionmarks{1(c)}{7}{Explain the components of the P.O.E.M. framework and their importance in digital marketing.}

\begin{solutionbox}
P.O.E.M. framework categorizes media types for comprehensive digital marketing strategy.

\begin{center}
\begin{tikzpicture}[node distance=1.5cm, auto]
    % Nodes
    \node [gtu block] (A) {P.O.E.M Framework};
    
    \node [gtu block, below left=1.2cm and 2cm of A] (B) {Paid Media};
    \node [gtu block, below=1.2cm of A] (C) {Owned Media};
    \node [gtu block, below right=1.2cm and 2cm of A] (D) {Earned Media};
    
    \node [gtu state, below=0.8cm of B] (B1) {Google Ads\\Facebook Ads};
    \node [gtu state, below=0.8cm of C] (C1) {Website\\Email Lists};
    \node [gtu state, below=0.8cm of D] (D1) {Social Shares\\Reviews};

    % Arrows
    \path [gtu arrow] (A) -- (B);
    \path [gtu arrow] (A) -- (C);
    \path [gtu arrow] (A) -- (D);
    
    \path [gtu arrow] (B) -- (B1);
    \path [gtu arrow] (C) -- (C1);
    \path [gtu arrow] (D) -- (D1);
\end{tikzpicture}
\captionof{figure}{P.O.E.M. Framework Components}
\end{center}

\begin{center}
\captionof{table}{P.O.E.M. Components}
\begin{tabulary}{\linewidth}{|L|L|L|L|}
\hline
\textbf{Component} & \textbf{Definition} & \textbf{Examples} & \textbf{Importance} \\ \hline
\textbf{Paid Media} & Promotional content through payment & Google Ads, Facebook Ads & Immediate visibility and traffic \\ \hline
\textbf{Owned Media} & Content controlled by brand & Website, Email lists & Build long-term relationships \\ \hline
\textbf{Earned Media} & Organic mentions by users & Reviews, Social shares & Authentic credibility \\ \hline
\end{tabulary}
\end{center}

\begin{itemize}
    \item \keyword{Paid Media}: Provides immediate reach and measurable ROI
    \item \keyword{Owned Media}: Creates direct customer relationships and brand control
    \item \keyword{Earned Media}: Builds authentic trust through user-generated content
\end{itemize}

\begin{mnemonicbox}POE - Pay for reach, Own relationships, Earn trust\end{mnemonicbox}
\end{solutionbox}

\questionmarks{1(c OR)}{7}{Explain the key components of digital marketing plan.}

\begin{solutionbox}
A digital marketing plan provides structured approach for online business success.

\begin{center}
\begin{tikzpicture}[node distance=1.5cm, auto]
    % Central Node
    \node [gtu block, minimum width=3cm] (Center) {Digital Marketing\\Plan};
    
    % Surrounding Nodes
    \node [gtu block, above=1.5cm of Center] (Goals) {Goals \& KPIs};
    \node [gtu block, above right=1cm and 1cm of Center] (Audience) {Target Audience};
    \node [gtu block, right=1.8cm of Center] (Strategy) {Strategy};
    \node [gtu block, below right=1cm and 1cm of Center] (Tactics) {Tactics};
    \node [gtu block, below=1.5cm of Center] (Budget) {Budget};
    \node [gtu block, below left=1cm and 1cm of Center] (Timeline) {Timeline};
    \node [gtu block, left=1.8cm of Center] (Research) {Research};
    \node [gtu block, above left=1cm and 1cm of Center] (Measurement) {Measurement};
    
    % Connections
    \foreach \n in {Goals, Audience, Strategy, Tactics, Budget, Timeline, Research, Measurement}
        \draw [gtu arrow, <->] (Center) -- (\n);
\end{tikzpicture}
\captionof{figure}{Key Components of Digital Marketing Plan}
\end{center}

\begin{center}
\captionof{table}{Digital Marketing Plan Components}
\begin{tabulary}{\linewidth}{|L|L|L|}
\hline
\textbf{Component} & \textbf{Description} & \textbf{Purpose} \\ \hline
\textbf{Market Research} & Industry and competitor analysis & Understanding market landscape \\ \hline
\textbf{Target Audience} & Demographics and psychographics & Focused messaging \\ \hline
\textbf{Goals \& KPIs} & Specific measurable objectives & Performance tracking \\ \hline
\textbf{Strategy \& Tactics} & Channels and content approach & Implementation roadmap \\ \hline
\textbf{Budget Allocation} & Resource distribution & Cost management \\ \hline
\textbf{Timeline} & Campaign scheduling & Project management \\ \hline
\textbf{Measurement} & Analytics and reporting & Continuous improvement \\ \hline
\end{tabulary}
\end{center}

\begin{itemize}
    \item \textbf{Clear objectives}: SMART goals ensure focused efforts
    \item \textbf{Audience targeting}: Precise demographics improve conversion rates
    \item \textbf{Performance tracking}: Regular measurement enables optimization
\end{itemize}

\begin{mnemonicbox}RATSBUM - Research, Audience, Tactics, Strategy, Budget, Measurement\end{mnemonicbox}
\end{solutionbox}

\questionmarks{2(a)}{3}{Differentiate between black hat and white hat SEO techniques.}

\begin{solutionbox}
White hat SEO follows guidelines while black hat uses prohibited methods for quick results.

\begin{center}
\captionof{table}{White Hat vs Black Hat SEO}
\begin{tabulary}{\linewidth}{|L|L|L|}
\hline
\textbf{Aspect} & \textbf{White Hat SEO} & \textbf{Black Hat SEO} \\ \hline
\textbf{Methods} & Ethical practices & Manipulative techniques \\ \hline
\textbf{Results} & Sustainable rankings & Temporary gains \\ \hline
\textbf{Risk} & Safe from penalties & High penalty risk \\ \hline
\textbf{Examples} & Quality content, natural links & Keyword stuffing, hidden text \\ \hline
\end{tabulary}
\end{center}

\begin{itemize}
    \item \keyword{White Hat}: Focuses on user experience and quality content
    \item \keyword{Black Hat}: Attempts to deceive search engine algorithms
    \item \keyword{Long-term impact}: White hat builds lasting success
\end{itemize}

\begin{mnemonicbox}WS-BT - White Sustainable, Black Temporary\end{mnemonicbox}
\end{solutionbox}

\questionmarks{2(b)}{4}{Discuss the factors that affect SEO rankings.}

\begin{solutionbox}
Multiple factors influence how search engines rank websites in results.

\begin{center}
\captionof{table}{SEO Ranking Factors}
\begin{tabulary}{\linewidth}{|L|L|}
\hline
\textbf{Factor Category} & \textbf{Specific Factors} \\ \hline
\textbf{Content Quality} & Relevance, originality, keyword optimization \\ \hline
\textbf{Technical SEO} & Page speed, mobile-friendliness, SSL \\ \hline
\textbf{User Experience} & Bounce rate, time on site, navigation \\ \hline
\textbf{Authority} & Backlinks, domain age, social signals \\ \hline
\end{tabulary}
\end{center}

\begin{itemize}
    \item \textbf{Content relevance}: High-quality, original content ranks better
    \item \textbf{Technical optimization}: Fast loading and mobile-friendly sites preferred
    \item \textbf{User engagement}: Low bounce rates indicate valuable content
    \item \textbf{External authority}: Quality backlinks boost credibility
\end{itemize}

\begin{mnemonicbox}CTUA - Content, Technical, User experience, Authority\end{mnemonicbox}
\end{solutionbox}

\questionmarks{2(c)}{7}{How social media can improve SEO rankings? Explain with a suitable example.}

\begin{solutionbox}
Social media indirectly boosts SEO through increased visibility and engagement signals.

\begin{center}
\begin{tikzpicture}[node distance=1.5cm, auto]
    \node [gtu block] (Activity) {Social Media Activity};
    
    \node [gtu block, above right=1cm and 2cm of Activity] (Visibility) {Increased Brand\\Visibility};
    \node [gtu block, below right=1cm and 2cm of Activity] (Traffic) {More Website\\Traffic};
    \node [gtu block, right=4cm of Activity] (Signals) {Social Signals};
    
    \node [gtu block, right=8cm of Activity] (Rankings) {Better SEO\\Rankings};
    
    \path [gtu arrow] (Activity) -- (Visibility);
    \path [gtu arrow] (Activity) -- (Traffic);
    \path [gtu arrow] (Activity) -- (Signals);
    
    \path [gtu arrow] (Visibility) -- (Rankings);
    \path [gtu arrow] (Traffic) -- (Rankings);
    \path [gtu arrow] (Signals) -- (Rankings);
\end{tikzpicture}
\captionof{figure}{Social Media Impact on SEO}
\end{center}

\begin{center}
\captionof{table}{Social Media SEO Impact}
\begin{tabulary}{\linewidth}{|L|L|L|}
\hline
\textbf{Social Media Impact} & \textbf{SEO Benefit} & \textbf{Example} \\ \hline
\textbf{Content Sharing} & Increased backlinks & Blog post shared on LinkedIn gets linked by industry sites \\ \hline
\textbf{Brand Mentions} & Authority building & Twitter mentions increase brand searches \\ \hline
\textbf{Traffic Generation} & User engagement signals & Facebook posts drive traffic, reducing bounce rate \\ \hline
\textbf{Local Presence} & Local SEO boost & Google My Business posts improve local rankings \\ \hline
\end{tabulary}
\end{center}

\textbf{Example}: A restaurant shares food photos on Instagram with location tags. This increases:
\begin{itemize}
    \item Local brand searches
    \item Website visits from social media
    \item User-generated content and reviews
    \item Overall online presence
\end{itemize}

\begin{itemize}
    \item \textbf{Social signals}: Search engines consider social engagement as quality indicator
    \item \textbf{Traffic boost}: Social media drives qualified visitors to website
    \item \textbf{Content amplification}: Social sharing increases content reach and potential backlinks
\end{itemize}

\begin{mnemonicbox}STAB - Signals, Traffic, Amplification, Branding\end{mnemonicbox}
\end{solutionbox}

\questionmarks{2(a OR)}{3}{Differentiate between on-page SEO and off-page SEO.}

\begin{solutionbox}
On-page SEO optimizes website elements while off-page builds external authority.

\begin{center}
\captionof{table}{On-Page vs Off-Page SEO}
\begin{tabulary}{\linewidth}{|L|L|L|}
\hline
\textbf{Aspect} & \textbf{On-Page SEO} & \textbf{Off-Page SEO} \\ \hline
\textbf{Location} & Within website & External websites \\ \hline
\textbf{Control} & Full control & Limited control \\ \hline
\textbf{Focus} & Content and technical & Authority and trust \\ \hline
\textbf{Examples} & Title tags, meta descriptions & Backlinks, social shares \\ \hline
\end{tabulary}
\end{center}

\begin{itemize}
    \item \keyword{On-page}: Optimizes content, HTML tags, and site structure
    \item \keyword{Off-page}: Builds authority through external signals and links
    \item \keyword{Combination}: Both needed for comprehensive SEO success
\end{itemize}

\begin{mnemonicbox}In-Out - Internal optimization, Outside authority\end{mnemonicbox}
\end{solutionbox}

\questionmarks{2(b OR)}{4}{Discuss different ways to improve SEO ranking.}

\begin{solutionbox}
Multiple strategies can enhance website visibility in search results.

\begin{center}
\captionof{table}{SEO Improvement Strategies}
\begin{tabulary}{\linewidth}{|L|L|}
\hline
\textbf{Strategy} & \textbf{Implementation} \\ \hline
\textbf{Content Optimization} & Keyword research, quality writing, regular updates \\ \hline
\textbf{Technical SEO} & Page speed, mobile optimization, SSL certificate \\ \hline
\textbf{Link Building} & Guest posting, directory submissions, partnerships \\ \hline
\textbf{User Experience} & Clear navigation, fast loading, engaging design \\ \hline
\end{tabulary}
\end{center}

\begin{itemize}
    \item \textbf{Quality content}: Create valuable, original content with target keywords
    \item \textbf{Technical excellence}: Ensure fast, mobile-friendly, secure website
    \item \textbf{Authority building}: Acquire high-quality backlinks from relevant sites
    \item \textbf{User satisfaction}: Focus on easy navigation and engaging experience
\end{itemize}

\begin{mnemonicbox}CTLU - Content, Technical, Links, User experience\end{mnemonicbox}
\end{solutionbox}

\questionmarks{2(c OR)}{7}{How will you do off page optimization for newly launched website?}

\begin{solutionbox}
Off-page optimization for new websites requires strategic approach to build authority.

\begin{center}
\begin{tikzpicture}[node distance=1.5cm, auto]
    \node [gtu block] (NewSite) {New Website\\Off-Page SEO};
    
    \node [gtu block, below left=1.5cm and 2cm of NewSite] (Directory) {Directory\\Submissions};
    \node [gtu block, below left=1.5cm and 0cm of NewSite] (Social) {Social Media\\Presence};
    \node [gtu block, below=1.5cm of NewSite] (Content) {Content\\Marketing};
    \node [gtu block, below right=1.5cm and 0cm of NewSite] (Local) {Local SEO};
    \node [gtu block, below right=1.5cm and 2cm of NewSite] (Relationship) {Relationship\\Building};
    
    \node [gtu block, below=4cm of NewSite] (Result) {Improved Rankings};
    
    \foreach \n in {Directory, Social, Content, Local, Relationship}
        \path [gtu arrow] (NewSite) -- (\n);
        
    \foreach \n in {Directory, Social, Content, Local, Relationship}
        \path [gtu arrow] (\n) -- (Result);
\end{tikzpicture}
\captionof{figure}{Off-Page Strategy for New Website}
\end{center}

\begin{center}
\captionof{table}{Off-Page Action Plan}
\begin{tabulary}{\linewidth}{|L|L|L|}
\hline
\textbf{Strategy} & \textbf{Action Steps} & \textbf{Timeline} \\ \hline
\textbf{Directory Submissions} & Submit to relevant business directories & Week 1-2 \\ \hline
\textbf{Social Media Setup} & Create profiles on major platforms & Week 1 \\ \hline
\textbf{Content Creation} & Develop shareable blog content & Ongoing \\ \hline
\textbf{Local SEO} & Google My Business, local citations & Week 2-3 \\ \hline
\textbf{Guest Posting} & Write for industry blogs with backlinks & Month 2-3 \\ \hline
\textbf{Influencer Outreach} & Connect with industry influencers & Month 2-4 \\ \hline
\end{tabulary}
\end{center}

\textbf{Implementation Steps:}
\begin{enumerate}
    \item \textbf{Research competitors}: Analyze their backlink profiles
    \item \textbf{Create valuable content}: Develop resources worth linking to
    \item \textbf{Build relationships}: Engage with industry professionals
    \item \textbf{Monitor progress}: Track backlinks and ranking improvements
\end{enumerate}

\begin{itemize}
    \item \textbf{Patience required}: Off-page SEO takes 3-6 months to show results
    \item \textbf{Quality focus}: Few high-quality links better than many low-quality ones
    \item \textbf{Consistency}: Regular outreach and content creation essential
\end{itemize}

\begin{mnemonicbox}DSCLIG - Directories, Social, Content, Local, Influencers, Guest posting\end{mnemonicbox}
\end{solutionbox}

\questionmarks{3(a)}{3}{Define the following key metrics: Unique visitors, Bounce rate, Pageviews.}

\begin{solutionbox}
These metrics measure website performance and user engagement effectively.

\begin{center}
\captionof{table}{Key Web Metrics}
\begin{tabulary}{\linewidth}{|L|L|L|}
\hline
\textbf{Metric} & \textbf{Definition} & \textbf{Importance} \\ \hline
\textbf{Unique Visitors} & Individual users visiting site in time period & Measures audience reach \\ \hline
\textbf{Bounce Rate} & Percentage leaving after viewing one page & Indicates content relevance \\ \hline
\textbf{Pageviews} & Total pages viewed during visits & Shows content consumption \\ \hline
\end{tabulary}
\end{center}

\begin{itemize}
    \item \textbf{Unique Visitors}: Counts each person once regardless of multiple visits
    \item \textbf{Bounce Rate}: High rates suggest poor content or user experience
    \item \textbf{Pageviews}: Higher numbers indicate engaging, discoverable content
\end{itemize}

\begin{mnemonicbox}UBP - Users, Bounces, Pages\end{mnemonicbox}
\end{solutionbox}

\questionmarks{3(b)}{4}{Explain A/B testing in web analytics.}

\begin{solutionbox}
A/B testing compares two versions to determine which performs better.

\begin{center}
\begin{tikzpicture}[node distance=1.5cm, auto]
    % Versions
    \node [gtu block, minimum width=2.5cm] (VerA) {Version A\\(Red Button)};
    \node [gtu block, minimum width=2.5cm, right=3cm of VerA] (VerB) {Version B\\(Blue Button)};
    
    % Traffic
    \node [gtu state, below=1cm of VerA] (TrafA) {50\% Traffic};
    \node [gtu state, below=1cm of VerB] (TrafB) {50\% Traffic};
    
    % Results
    \node [gtu block, below=1cm of TrafA] (ResA) {Result A\\5\% Click};
    \node [gtu block, below=1cm of TrafB] (ResB) {Result B\\8\% Click};
    
    % Connections
    \draw [gtu arrow] (VerA) -- (TrafA);
    \draw [gtu arrow] (VerB) -- (TrafB);
    \draw [gtu arrow] (TrafA) -- (ResA);
    \draw [gtu arrow] (TrafB) -- (ResB);
    
    % Winner
    \node [draw, star, star points=5, star point height=0.5cm, below=0.5cm of ResB, fill=yellow!20, align=center] (Win) {Winner!};
\end{tikzpicture}
\captionof{figure}{A/B Testing Process}
\end{center}

\begin{center}
\captionof{table}{A/B Testing Components}
\begin{tabulary}{\linewidth}{|L|L|}
\hline
\textbf{Component} & \textbf{Description} \\ \hline
\textbf{Hypothesis} & Prediction about what will improve performance \\ \hline
\textbf{Variables} & Elements being tested (headlines, buttons, colors) \\ \hline
\textbf{Traffic Split} & Random division of visitors between versions \\ \hline
\textbf{Measurement} & Comparing conversion rates or other metrics \\ \hline
\end{tabulary}
\end{center}

\begin{itemize}
    \item \textbf{Statistical significance}: Ensure enough data for reliable results
    \item \textbf{Single variable}: Test one element at a time for clear insights
    \item \textbf{Continuous improvement}: Regular testing optimizes performance
\end{itemize}

\begin{mnemonicbox}HTVM - Hypothesis, Test, Variables, Measure\end{mnemonicbox}
\end{solutionbox}

\questionmarks{3(c)}{7}{How businesses can set up goals in Google Analytics? Explain with a suitable example.}

\begin{solutionbox}
Google Analytics goals track important business actions and measure success.

\begin{center}
\begin{tikzpicture}[node distance=1.5cm, auto]
    \node [gtu block] (Goals) {Google Analytics Goals};
    
    \node [gtu block, below left=1.5cm and 3cm of Goals] (Dest) {Destination Goals};
    \node [gtu block, below left=1.5cm and 0.5cm of Goals] (Dur) {Duration Goals};
    \node [gtu block, below right=1.5cm and 0.5cm of Goals] (Pages) {Pages/Session Goals};
    \node [gtu block, below right=1.5cm and 3cm of Goals] (Event) {Event Goals};
    
    \node [gtu state, below=0.8cm of Dest] (Ex1) {Thank You Page};
    \node [gtu state, below=0.8cm of Dur] (Ex2) {Time on Site};
    \node [gtu state, below=0.8cm of Pages] (Ex3) {Page Views};
    \node [gtu state, below=0.8cm of Event] (Ex4) {Download PDF};
    
    \path [gtu arrow] (Goals) -- (Dest);
    \path [gtu arrow] (Goals) -- (Dur);
    \path [gtu arrow] (Goals) -- (Pages);
    \path [gtu arrow] (Goals) -- (Event);
    
    \path [gtu arrow] (Dest) -- (Ex1);
    \path [gtu arrow] (Dur) -- (Ex2);
    \path [gtu arrow] (Pages) -- (Ex3);
    \path [gtu arrow] (Event) -- (Ex4);
\end{tikzpicture}
\captionof{figure}{Goal Types in Google Analytics}
\end{center}

\begin{center}
\captionof{table}{Goal Types}
\begin{tabulary}{\linewidth}{|L|L|L|}
\hline
\textbf{Goal Type} & \textbf{Description} & \textbf{Business Example} \\ \hline
\textbf{Destination} & Reaching specific page & Contact form submission \\ \hline
\textbf{Duration} & Time spent on site & Engagement measurement \\ \hline
\textbf{Pages/Session} & Number of pages viewed & Content consumption \\ \hline
\textbf{Event} & Specific interactions & File downloads, video plays \\ \hline
\end{tabulary}
\end{center}

\textbf{Setup Process:}
\begin{enumerate}
    \item \textbf{Access Admin}: Go to Goals section in Admin panel
    \item \textbf{Choose Template}: Select relevant goal template or custom
    \item \textbf{Configure Details}: Set destination URL or event parameters
    \item \textbf{Verify Goal}: Test goal setup with Goal Flow reports
\end{enumerate}

\textbf{Example - E-commerce Store:}
\begin{itemize}
    \item \textbf{Goal}: Track purchase completions
    \item \textbf{Type}: Destination goal
    \item \textbf{Setup}: Track visits to "/order-confirmation" page
    \item \textbf{Value}: Assign monetary value to conversions
    \item \textbf{Funnel}: Set up checkout process steps
\end{itemize}

\begin{itemize}
    \item \textbf{Conversion tracking}: Measures business objective achievement
    \item \textbf{ROI calculation}: Assigns value to website interactions
    \item \textbf{Optimization insights}: Identifies improvement opportunities
\end{itemize}

\begin{mnemonicbox}DDPE - Destination, Duration, Pages, Events\end{mnemonicbox}
\end{solutionbox}

\questionmarks{3(a OR)}{3}{Define the following key metrics: New Visits, Pages/visit, Average Visit Duration.}

\begin{solutionbox}
These metrics analyze visitor behavior and website engagement patterns.

\begin{center}
\captionof{table}{Engagement Metrics}
\begin{tabulary}{\linewidth}{|L|L|L|}
\hline
\textbf{Metric} & \textbf{Definition} & \textbf{Significance} \\ \hline
\textbf{New Visits} & First-time visitors percentage & Measures audience growth \\ \hline
\textbf{Pages/Visit} & Average pages viewed per session & Content engagement level \\ \hline
\textbf{Average Visit Duration} & Time spent per visit & User interest indicator \\ \hline
\end{tabulary}
\end{center}

\begin{itemize}
    \item \textbf{New Visits}: High percentage shows effective marketing reach
    \item \textbf{Pages/Visit}: Higher numbers indicate compelling content
    \item \textbf{Visit Duration}: Longer time suggests valuable information
\end{itemize}

\begin{mnemonicbox}NPA - New visitors, Pages viewed, Average duration\end{mnemonicbox}
\end{solutionbox}

\questionmarks{3(b OR)}{4}{What are the different methods of data collection in website analytics?}

\begin{solutionbox}
Various methods capture user behavior data for analysis and optimization.

\begin{center}
\captionof{table}{Data Collection Methods}
\begin{tabulary}{\linewidth}{|L|L|L|}
\hline
\textbf{Method} & \textbf{Description} & \textbf{Data Collected} \\ \hline
\textbf{Page Tagging} & JavaScript code on pages & User interactions, page views \\ \hline
\textbf{Web Log Analysis} & Server log files examination & Technical data, errors \\ \hline
\textbf{Packet Sniffing} & Network traffic monitoring & Real-time user behavior \\ \hline
\textbf{Hybrid Approach} & Combination of methods & Comprehensive data set \\ \hline
\end{tabulary}
\end{center}

\begin{itemize}
    \item \textbf{Page Tagging}: Most common method using Google Analytics code
    \item \textbf{Server Logs}: Technical data about requests and responses
    \item \textbf{Real-time Tracking}: Immediate user behavior insights
    \item \textbf{Data Accuracy}: Multiple methods provide complete picture
\end{itemize}

\begin{mnemonicbox}PWPH - Page tagging, Web logs, Packet sniffing, Hybrid\end{mnemonicbox}
\end{solutionbox}

\questionmarks{3(c OR)}{7}{Explain different marketing attribution models with example.}

\begin{solutionbox}
Attribution models assign credit to marketing channels in customer journey.

\begin{center}
\begin{tikzpicture}[node distance=1.5cm, auto]
    % Journey
    \node [gtu state] (FB) {Facebook Ad};
    \node [gtu state, right=0.8cm of FB] (Google) {Google Search};
    \node [gtu state, right=0.8cm of Google] (Email) {Email};
    \node [gtu block, fill=green!10, right=0.8cm of Email] (Purchase) {Purchase (\$100)};
    
    \draw [gtu arrow] (FB) -- (Google);
    \draw [gtu arrow] (Google) -- (Email);
    \draw [gtu arrow] (Email) -- (Purchase);
    
    % Models
    \node [gtu block, below=1.5cm of FB, text width=2.5cm] (First) {\textbf{First-Click}\\100\% to Facebook};
    \node [gtu block, below=1.5cm of Email, text width=2.5cm] (Last) {\textbf{Last-Click}\\100\% to Email};
    \node [gtu block, below=3.5cm of Google, text width=3.5cm] (Linear) {\textbf{Linear}\\Equal Split (33\% each)};
\end{tikzpicture}
\captionof{figure}{Attribution Models Example}
\end{center}

\begin{center}
\captionof{table}{Attribution Models Comparison}
\begin{tabulary}{\linewidth}{|L|L|L|L|}
\hline
\textbf{Model} & \textbf{Credit Distribution} & \textbf{Best For} & \textbf{Example} \\ \hline
\textbf{First-Click} & 100\% to first interaction & Brand awareness campaigns & Social media ad gets full credit \\ \hline
\textbf{Last-Click} & 100\% to final interaction & Direct response campaigns & Email campaign gets full credit \\ \hline
\textbf{Linear} & Equal credit to all touchpoints & Multi-channel campaigns & Each channel gets 25\% credit \\ \hline
\textbf{Time-Decay} & More credit to recent interactions & Sales-focused campaigns & Recent touchpoints get higher credit \\ \hline
\textbf{Position-Based} & More credit to first and last & Awareness + conversion focus & 40\% first, 40\% last, 20\% middle \\ \hline
\end{tabulary}
\end{center}

\textbf{Example Scenario:} Customer journey: Facebook Ad $\to$ Google Search $\to$ Email $\to$ Purchase (\$100)
\begin{itemize}
    \item \textbf{First-Click}: Facebook Ad = \$100 credit
    \item \textbf{Last-Click}: Email = \$100 credit
    \item \textbf{Linear}: Facebook \$33, Google \$33, Email \$33 credit
    \item \textbf{Time-Decay}: Email \$60, Google \$30, Facebook \$10 credit
\end{itemize}

\begin{itemize}
    \item \textbf{Business alignment}: Choose model matching marketing objectives
    \item \textbf{Data insights}: Different models reveal various channel contributions
    \item \textbf{Optimization}: Helps allocate budget to effective channels
\end{itemize}

\begin{mnemonicbox}FLLTP - First, Last, Linear, Time-decay, Position-based\end{mnemonicbox}
\end{solutionbox}

\questionmarks{4(a)}{3}{Explain different types of YouTube ads.}

\begin{solutionbox}
YouTube offers various ad formats to reach audiences effectively.

\begin{center}
\captionof{table}{YouTube Ad Types}
\begin{tabulary}{\linewidth}{|L|L|L|L|}
\hline
\textbf{Ad Type} & \textbf{Format} & \textbf{Duration} & \textbf{Placement} \\ \hline
\textbf{Skippable} & Video ads with skip option & Any length & Before/during videos \\ \hline
\textbf{Non-Skippable} & Mandatory viewing & 15-20 seconds & Before/during videos \\ \hline
\textbf{Bumper Ads} & Short, non-skippable & 6 seconds & Before videos \\ \hline
\textbf{Discovery Ads} & Thumbnail with text & Variable & Search results, sidebar \\ \hline
\end{tabulary}
\end{center}

\begin{itemize}
    \item \textbf{Skippable ads}: Cost-effective for engagement-focused campaigns
    \item \textbf{Non-skippable}: Guaranteed exposure for brand awareness
    \item \textbf{Bumper ads}: Quick brand messages with high reach
\end{itemize}

\begin{mnemonicbox}SNBD - Skippable, Non-skippable, Bumper, Discovery\end{mnemonicbox}
\end{solutionbox}

\questionmarks{4(b)}{4}{How hashtags can be used in Twitter marketing?}

\begin{solutionbox}
Hashtags increase content discoverability and engagement on Twitter platform.

\begin{center}
\captionof{table}{Twitter Hashtag Strategy}
\begin{tabulary}{\linewidth}{|L|L|L|}
\hline
\textbf{Use Case} & \textbf{Strategy} & \textbf{Example} \\ \hline
\textbf{Trending Topics} & Join relevant conversations & \#BlackFriday for sales \\ \hline
\textbf{Brand Hashtags} & Create unique brand identifiers & \#JustDoIt for Nike \\ \hline
\textbf{Event Marketing} & Promote events and gatherings & \#TechConf2023 \\ \hline
\textbf{Categorization} & Organize content themes & \#MondayMotivation \\ \hline
\end{tabulary}
\end{center}

\begin{itemize}
    \item \textbf{Research trends}: Use trending hashtags for wider reach
    \item \textbf{Create branded}: Develop unique hashtags for campaigns
    \item \textbf{Monitor performance}: Track hashtag engagement and reach
    \item \textbf{Limit quantity}: Use 1-2 hashtags per tweet for best results
\end{itemize}

\begin{mnemonicbox}TBEC - Trending, Branded, Events, Categorization\end{mnemonicbox}
\end{solutionbox}

\questionmarks{4(c)}{7}{Explain social media marketing and its significance in the current digital landscape.}

\begin{solutionbox}
Social media marketing leverages platforms to build relationships and drive business results.

\begin{center}
\begin{tikzpicture}[node distance=1.5cm, auto]
    \node [gtu block, minimum width=3cm] (SMM) {Social Media\\Marketing};
    
    \node [gtu block, above left=1.5cm and 1cm of SMM] (Awareness) {Brand\\Awareness};
    \node [gtu block, above right=1.5cm and 1cm of SMM] (Engagement) {Customer\\Engagement};
    \node [gtu block, below left=1.5cm and 1cm of SMM] (Lead) {Lead\\Generation};
    \node [gtu block, below right=1.5cm and 1cm of SMM] (Service) {Customer\\Service};
    
    \node [gtu block, right=4cm of SMM] (Growth) {Business Growth};
    
    \path [gtu arrow] (SMM) -- (Awareness);
    \path [gtu arrow] (SMM) -- (Engagement);
    \path [gtu arrow] (SMM) -- (Lead);
    \path [gtu arrow] (SMM) -- (Service);
    
    \path [gtu arrow] (Awareness) -- (Growth);
    \path [gtu arrow] (Engagement) -- (Growth);
    \path [gtu arrow] (Lead) -- (Growth);
    \path [gtu arrow] (Service) -- (Growth);
\end{tikzpicture}
\captionof{figure}{Social Media Marketing Impact}
\end{center}

\begin{center}
\captionof{table}{Social Media Platforms}
\begin{tabulary}{\linewidth}{|L|L|L|L|}
\hline
\textbf{Platform} & \textbf{Primary Use} & \textbf{Audience} & \textbf{Content Type} \\ \hline
\textbf{Facebook} & Community building & Broad demographics & Posts, videos, events \\ \hline
\textbf{Instagram} & Visual storytelling & Younger audience & Photos, stories, reels \\ \hline
\textbf{LinkedIn} & Professional networking & Business professionals & Articles, company updates \\ \hline
\textbf{Twitter} & Real-time engagement & News, trends followers & Short messages, threads \\ \hline
\textbf{YouTube} & Video marketing & Video consumers & Educational, entertainment \\ \hline
\end{tabulary}
\end{center}

\textbf{Significance in Digital Landscape:}
\begin{itemize}
    \item \textbf{Direct communication}: Real-time interaction with customers
    \item \textbf{Cost-effective reach}: Lower costs compared to traditional advertising
    \item \textbf{Targeted advertising}: Precise demographic and interest targeting
    \item \textbf{Viral potential}: Content can reach massive audiences organically
    \item \textbf{Customer insights}: Valuable data about preferences and behavior
\end{itemize}

\textbf{Current Trends:}
\begin{itemize}
    \item \textbf{Video content dominance}: Short-form videos drive engagement
    \item \textbf{Social commerce}: Direct purchasing through platforms
    \item \textbf{Influencer partnerships}: Authentic endorsements from creators
\end{itemize}

\begin{mnemonicbox}CLEAR - Communication, Low-cost, Engagement, Analytics, Reach\end{mnemonicbox}
\end{solutionbox}

\questionmarks{4(a OR)}{3}{Explain different types of LinkedIn ads.}

\begin{solutionbox}
LinkedIn provides professional-focused advertising options for B2B marketing.

\begin{center}
\captionof{table}{LinkedIn Ad Types}
\begin{tabulary}{\linewidth}{|L|L|L|}
\hline
\textbf{Ad Type} & \textbf{Format} & \textbf{Best For} \\ \hline
\textbf{Sponsored Content} & Native posts in feed & Brand awareness, engagement \\ \hline
\textbf{Message Ads} & Direct messages to users & Lead generation, outreach \\ \hline
\textbf{Dynamic Ads} & Personalized banner ads & Website traffic, follower growth \\ \hline
\textbf{Text Ads} & Simple text with image & Cost-effective awareness \\ \hline
\end{tabulary}
\end{center}

\begin{itemize}
    \item \textbf{Professional targeting}: Reach users by job title, company, industry
    \item \textbf{B2B focus}: Ideal for business-to-business marketing campaigns
    \item \textbf{High-quality audience}: Professional mindset drives better engagement
\end{itemize}

\begin{mnemonicbox}SMDT - Sponsored, Message, Dynamic, Text\end{mnemonicbox}
\end{solutionbox}

\questionmarks{4(b OR)}{4}{Explain the concept of influencer marketing on Instagram.}

\begin{solutionbox}
Influencer marketing leverages popular users to promote products authentically.

\begin{center}
\captionof{table}{Influencer Tiers}
\begin{tabulary}{\linewidth}{|L|L|L|L|}
\hline
\textbf{Types} & \textbf{Followers} & \textbf{Best For} & \textbf{Cost} \\ \hline
\textbf{Nano} & 1K-10K & Local, niche products & Low \\ \hline
\textbf{Micro} & 10K-100K & Targeted, high engagement & Medium \\ \hline
\textbf{Macro} & 100K-1M & Brand awareness, reach & High \\ \hline
\textbf{Mega} & 1M+ & Mass market, celebrities & Very High \\ \hline
\end{tabulary}
\end{center}

\begin{itemize}
    \item \textbf{Authentic content}: Influencers create genuine product recommendations
    \item \textbf{High engagement}: Followers trust influencer opinions and suggestions
    \item \textbf{Targeted reach}: Choose influencers matching target audience demographics
    \item \textbf{Measurable results}: Track engagement, clicks, and conversions easily
\end{itemize}

\begin{mnemonicbox}NMAM - Nano, Micro, Macro, Mega influencers\end{mnemonicbox}
\end{solutionbox}

\questionmarks{4(c OR)}{7}{Describe the targeting options available in Facebook advertising.}

\begin{solutionbox}
Facebook provides comprehensive targeting capabilities for precise audience reach.

\begin{center}
\begin{tikzpicture}[node distance=1.5cm, auto]
    \node [gtu block] (Targeting) {Facebook Targeting};
    
    \node [gtu block, below left=1.5cm and 3cm of Targeting] (Demo) {Demographics};
    \node [gtu block, below left=1.5cm and 0.5cm of Targeting] (Interest) {Interests};
    \node [gtu block, below right=1.5cm and 0.5cm of Targeting] (Behavior) {Behaviors};
    \node [gtu block, below right=1.5cm and 3cm of Targeting] (Custom) {Custom\\Audiences};
    
    \node [gtu state, below=0.8cm of Demo] (D1) {Age, Gender,\\Location};
    \node [gtu state, below=0.8cm of Interest] (I1) {Hobbies, Activities};
    \node [gtu state, below=0.8cm of Behavior] (B1) {Purchase History,\\Devices};
    \node [gtu state, below=0.8cm of Custom] (C1) {Email Lists,\\Lookalikes};
    
    \path [gtu arrow] (Targeting) -- (Demo);
    \path [gtu arrow] (Targeting) -- (Interest);
    \path [gtu arrow] (Targeting) -- (Behavior);
    \path [gtu arrow] (Targeting) -- (Custom);
    
    \path [gtu arrow] (Demo) -- (D1);
    \path [gtu arrow] (Interest) -- (I1);
    \path [gtu arrow] (Behavior) -- (B1);
    \path [gtu arrow] (Custom) -- (C1);
\end{tikzpicture}
\captionof{figure}{Facebook Targeting Options}
\end{center}

\begin{center}
\captionof{table}{Targeting Categories}
\begin{tabulary}{\linewidth}{|L|L|L|}
\hline
\textbf{Category} & \textbf{Options} & \textbf{Use Case} \\ \hline
\textbf{Demographics} & Age, gender, location, education & Basic audience definition \\ \hline
\textbf{Interests} & Pages liked, activities, hobbies & Lifestyle-based targeting \\ \hline
\textbf{Behaviors} & Purchase history, device usage & Action-based targeting \\ \hline
\textbf{Custom} & Uploaded lists, website visitors & Retargeting campaigns \\ \hline
\textbf{Lookalike} & Similar to existing customers & Audience expansion \\ \hline
\end{tabulary}
\end{center}

\textbf{Campaign Strategy:}
\begin{enumerate}
    \item \textbf{Start Broad}: Begin with basic demographics and interests
    \item \textbf{Analyze Performance}: Use analytics to identify best-performing segments
    \item \textbf{Refine Targeting}: Narrow focus based on successful audiences
    \item \textbf{Create Lookalikes}: Expand reach with similar audience characteristics
    \item \textbf{Retarget Visitors}: Re-engage website visitors with custom audiences
\end{enumerate}

\begin{itemize}
    \item \textbf{Precision marketing}: Reach exactly the right people for products
    \item \textbf{Cost efficiency}: Targeted ads reduce wasted advertising spend
    \item \textbf{Performance optimization}: Continuous refinement improves results
\end{itemize}

\begin{mnemonicbox}DIBCCL - Demographics, Interests, Behaviors, Custom, Connections, Lookalike\end{mnemonicbox}
\end{solutionbox}

\questionmarks{5(a)}{3}{List the metrics used to measure the success of YouTube marketing campaigns.}

\begin{solutionbox}
YouTube provides comprehensive metrics to evaluate campaign performance effectively.

\begin{center}
\captionof{table}{YouTube Measurement Metrics}
\begin{tabulary}{\linewidth}{|L|L|}
\hline
\textbf{Metric Category} & \textbf{Specific Metrics} \\ \hline
\textbf{Reach Metrics} & Views, impressions, unique viewers \\ \hline
\textbf{Engagement Metrics} & Likes, comments, shares, subscribers \\ \hline
\textbf{Performance Metrics} & Click-through rate, conversion rate \\ \hline
\textbf{Retention Metrics} & Watch time, average view duration \\ \hline
\end{tabulary}
\end{center}

\begin{itemize}
    \item \textbf{Views and impressions}: Measure content reach and visibility
    \item \textbf{Engagement signals}: Indicate audience interest and content quality
    \item \textbf{Conversion tracking}: Links video performance to business goals
\end{itemize}

\begin{mnemonicbox}REPR - Reach, Engagement, Performance, Retention\end{mnemonicbox}
\end{solutionbox}

\questionmarks{5(b)}{4}{Differentiate between PPC and SEO.}

\begin{solutionbox}
PPC and SEO are complementary strategies with different approaches and timelines.

\begin{center}
\captionof{table}{PPC vs SEO}
\begin{tabulary}{\linewidth}{|L|L|L|}
\hline
\textbf{Aspect} & \textbf{PPC} & \textbf{SEO} \\ \hline
\textbf{Cost} & Immediate payment per click & Long-term content investment \\ \hline
\textbf{Results} & Instant visibility & Gradual ranking improvement \\ \hline
\textbf{Control} & Full positioning control & Limited ranking control \\ \hline
\textbf{Sustainability} & Stops when budget ends & Continues after work stops \\ \hline
\textbf{Targeting} & Precise audience targeting & Broad keyword targeting \\ \hline
\end{tabulary}
\end{center}

\begin{itemize}
    \item \textbf{PPC advantages}: Immediate results, precise targeting, measurable ROI
    \item \textbf{SEO advantages}: Long-term sustainability, credibility, cost-effectiveness
    \item \textbf{Combined approach}: Both strategies work better together
\end{itemize}

\begin{mnemonicbox}IRCST - Immediate vs Reactive, Control vs Sustainable, Targeted\end{mnemonicbox}
\end{solutionbox}

\questionmarks{5(c)}{7}{Explain the different types of Google Ads Campaigns.}

\begin{solutionbox}
Google Ads offers various campaign types for different marketing objectives.

\begin{center}
\begin{tikzpicture}[node distance=1.5cm, auto]
    \node [gtu block] (Ads) {Google Ads\\Campaigns};
    
    \node [gtu block, below left=1.5cm and 3cm of Ads] (Search) {Search};
    \node [gtu block, below left=1.5cm and 1cm of Ads] (Display) {Display};
    \node [gtu block, below=1.5cm of Ads] (Video) {Video};
    \node [gtu block, below right=1.5cm and 1cm of Ads] (Shopping) {Shopping};
    \node [gtu block, below right=1.5cm and 3cm of Ads] (App) {App};
    
    \node [gtu state, below=0.8cm of Search] (S1) {Text Ads};
    \node [gtu state, below=0.8cm of Display] (D1) {Banner Ads};
    \node [gtu state, below=0.8cm of Video] (V1) {YouTube Ads};
    \node [gtu state, below=0.8cm of Shopping] (Sh1) {Product Ads};
    \node [gtu state, below=0.8cm of App] (A1) {App Promo};
    
    \foreach \n in {Search, Display, Video, Shopping, App}
        \path [gtu arrow] (Ads) -- (\n);
        
    \path [gtu arrow] (Search) -- (S1);
    \path [gtu arrow] (Display) -- (D1);
    \path [gtu arrow] (Video) -- (V1);
    \path [gtu arrow] (Shopping) -- (Sh1);
    \path [gtu arrow] (App) -- (A1);
\end{tikzpicture}
\captionof{figure}{Google Ads Campaign Types}
\end{center}

\begin{center}
\captionof{table}{Google Ads Campaigns}
\begin{tabulary}{\linewidth}{|L|L|L|L|}
\hline
\textbf{Type} & \textbf{Format} & \textbf{Best For} & \textbf{Objective} \\ \hline
\textbf{Search} & Text ads on search results & High-intent keywords & Traffic, sales \\ \hline
\textbf{Display} & Visual ads on partner sites & Awareness, remarketing & Broad reach \\ \hline
\textbf{Video} & Video ads on YouTube & Engagement, branding & Promotion \\ \hline
\textbf{Shopping} & Product listings with images & E-commerce sales & Direct showcase \\ \hline
\textbf{App} & Automated app promotion & Mobile app downloads & App installs \\ \hline
\textbf{Perf Max} & Multi-channel automation & Maximum performance & AI optimization \\ \hline
\end{tabulary}
\end{center}

\textbf{Budget Allocation Recommendations:}
\begin{itemize}
    \item \textbf{Search}: 40-50\% of budget for high-converting keywords
    \item \textbf{Display}: 20-30\% for awareness and remarketing
    \item \textbf{Video}: 15-25\% for engagement and brand building
    \item \textbf{Shopping}: 30-40\% for e-commerce businesses
\end{itemize}

\begin{itemize}
    \item \textbf{Multi-campaign approach}: Combine types for comprehensive reach
    \item \textbf{Audience journey}: Different campaigns target various buying stages
    \item \textbf{Performance optimization}: Regular monitoring improves results
\end{itemize}

\begin{mnemonicbox}SDVSAP - Search, Display, Video, Shopping, App, Performance Max\end{mnemonicbox}
\end{solutionbox}

\questionmarks{5(a OR)}{3}{List the metrics available on Instagram for tracking the success of marketing strategies.}

\begin{solutionbox}
Instagram Insights provides comprehensive metrics for campaign performance analysis.

\begin{center}
\captionof{table}{Instagram Metrics}
\begin{tabulary}{\linewidth}{|L|L|}
\hline
\textbf{Metric Category} & \textbf{Specific Metrics} \\ \hline
\textbf{Reach Metrics} & Impressions, reach, profile visits \\ \hline
\textbf{Engagement Metrics} & Likes, comments, shares, saves \\ \hline
\textbf{Story Metrics} & Story views, taps forward/back, exits \\ \hline
\textbf{Audience Metrics} & Demographics, active times, locations \\ \hline
\end{tabulary}
\end{center}

\begin{itemize}
    \item \textbf{Content performance}: Track which posts drive highest engagement
    \item \textbf{Audience insights}: Understand follower demographics and behavior
    \item \textbf{Growth tracking}: Monitor follower count and engagement rate changes
\end{itemize}

\begin{mnemonicbox}RESA - Reach, Engagement, Stories, Audience\end{mnemonicbox}
\end{solutionbox}

\questionmarks{5(b OR)}{4}{Describe the benefits of email marketing in digital marketing.}

\begin{solutionbox}
Email marketing remains highly effective for customer communication and conversion.

\begin{center}
\captionof{table}{Email Marketing Benefits}
\begin{tabulary}{\linewidth}{|L|L|L|}
\hline
\textbf{Benefit} & \textbf{Description} & \textbf{Impact} \\ \hline
\textbf{High ROI} & \$42 return for every \$1 spent & Cost-effective \\ \hline
\textbf{Direct} & Personal inbox access & Intimate connection \\ \hline
\textbf{Segmentation} & Targeted messaging by groups & Improved relevance \\ \hline
\textbf{Automation} & Scheduled and triggered emails & Efficient nurturing \\ \hline
\textbf{Measurable} & Detailed analytics available & Data-driven optimization \\ \hline
\end{tabulary}
\end{center}

\begin{itemize}
    \item \textbf{Permission-based}: Subscribers actively chose to receive communications
    \item \textbf{Personalization}: Customized content based on user preferences and behavior
    \item \textbf{Scalability}: Reach thousands of customers with single campaign
\end{itemize}

\begin{mnemonicbox}HDSAM - High ROI, Direct, Segmented, Automated, Measurable\end{mnemonicbox}
\end{solutionbox}

\questionmarks{5(c OR)}{7}{Explain various bidding strategies available in Google Ads.}

\begin{solutionbox}
Google Ads offers multiple bidding strategies to optimize campaign performance based on objectives.

\begin{center}
\begin{tikzpicture}[node distance=1.5cm, auto]
    \node [gtu block] (Bidding) {Google Ads\\Bidding};
    
    \node [gtu block, below left=1.2cm and 2cm of Bidding] (Manual) {Manual Bidding};
    \node [gtu block, below right=1.2cm and 0.5cm of Bidding] (Automated) {Automated Bidding};
    
    \node [gtu state, below=0.8cm of Manual] (ManualEx) {Manual CPC\\Enhanced CPC};
    
    \node [gtu state, below=0.8cm of Automated, text width=6cm] (AutoEx) {Target CPA, Target ROAS\\Max Clicks, Max Conversions\\Target Impression Share};
    
    \path [gtu arrow] (Bidding) -- (Manual);
    \path [gtu arrow] (Bidding) -- (Automated);
    
    \path [gtu arrow] (Manual) -- (ManualEx);
    \path [gtu arrow] (Automated) -- (AutoEx);
\end{tikzpicture}
\captionof{figure}{Bidding Strategies}
\end{center}

\begin{center}
\captionof{table}{Bidding Strategy Guide}
\begin{tabulary}{\linewidth}{|L|L|L|}
\hline
\textbf{Strategy} & \textbf{Type} & \textbf{Goal} \\ \hline
\textbf{Manual CPC} & Manual & Full bid control \\ \hline
\textbf{Enhanced CPC} & Semi-auto & Optimize likely conversions \\ \hline
\textbf{Target CPA} & Automated & Specific cost per acquisition \\ \hline
\textbf{Target ROAS} & Automated & Return on ad spend \\ \hline
\textbf{Max Clicks} & Automated & Traffic generation \\ \hline
\textbf{Max Conversions} & Automated & Volume within budget \\ \hline
\end{tabulary}
\end{center}

\textbf{Strategy Selection Guide:}
\begin{itemize}
    \item \textbf{New Campaigns}: Maximize Clicks or Manual CPC to build data
    \item \textbf{Established}: Target CPA or ROAS for efficiency
    \item \textbf{Budget}: Ensure budget allows for automated strategy fluctuations
\end{itemize}

\begin{itemize}
    \item \textbf{Algorithm Learning}: Automated bidding improves with more data
    \item \textbf{Performance Goals}: Choose strategy matching business objectives
    \item \textbf{Budget Management}: Consider spending patterns with different strategies
\end{itemize}

\begin{mnemonicbox}METER-MT - Manual, Enhanced, Target CPA, Target ROAS, Maximize\end{mnemonicbox}
\end{solutionbox}

\end{document}
