\documentclass{article}

% content/resources/templates/preamble.tex
\usepackage[margin=0.6in]{geometry}
\author{Milav Dabgar}
\usepackage{amsmath,amssymb,amsthm}
\usepackage{booktabs}
\usepackage{multirow}
\usepackage{xcolor}
\usepackage{tcolorbox}
\tcbuselibrary{breakable,skins}
\usepackage[colorlinks=true,linkcolor=blue]{hyperref}
\usepackage{titlesec}
\usepackage{enumitem}
\usepackage{tikz}
\usepackage{pgfplots}
\usepackage{circuitikz}
\usepackage[version=4]{mhchem}
\usepackage{longtable}
\usepackage{array}
\usepackage{float}
\usepackage{caption}
\usepackage{listings}

\lstset{
  basicstyle=\small\ttfamily,
  breaklines=true,
  breakatwhitespace=false,
  postbreak=\mbox{\textcolor{red}{$\hookrightarrow$}\space},
  float=false,
  numbers=left,
  numberstyle=\tiny\color{gray},
  numbersep=10pt,
  xleftmargin=2em,
  keywordstyle=\color{blue},
  commentstyle=\color{green!60!black},
  stringstyle=\color{purple},
  backgroundcolor=\color{gray!5},
  showstringspaces=false,
  tabsize=2,
  captionpos=b,
  keepspaces=true,
  columns=flexible
}

\pgfplotsset{compat=1.18}
\usetikzlibrary{shapes,arrows,positioning,calc,patterns,decorations.pathmorphing,decorations.markings,arrows.meta}

% Color scheme
\definecolor{headcolor}{RGB}{0,102,204}
\definecolor{keycolor}{RGB}{220,20,60}
\definecolor{solutioncolor}{RGB}{34,139,34}
\definecolor{mnemoniccolor}{RGB}{148,0,211}
\definecolor{codecolor}{RGB}{0,0,100}

% Spacing
\setlength{\parskip}{3pt}
\setlist[itemize]{nosep}
\setlist[enumerate]{nosep}

% Title formatting
\titleformat{\section}{\Large\bfseries\color{headcolor}}{\thesection}{1em}{}
\titleformat{\subsection}{\large\bfseries\color{headcolor}}{\thesubsection}{1em}{}

% Pandoc tightlist compatibility
\providecommand{\tightlist}{%
  \setlength{\itemsep}{0pt}\setlength{\parskip}{0pt}}

% Pandoc longtable compatibility
\newcounter{none}
\def\thenone{}


% content/resources/templates/english-boxes.tex

% Custom environments
\newtcolorbox{solutionbox}{
 breakable,
 enhanced,
 colback=solutioncolor!5!white,
 colframe=solutioncolor!75!black,
 fonttitle=\bfseries,
 title=Solution
}

\newtcolorbox{solutionboxnobreak}{
 colback=solutioncolor!5!white,
 colframe=solutioncolor!75!black,
 fonttitle=\bfseries,
 title=Solution
}

\newtcolorbox{keyformula}{
 breakable,
 enhanced,
 colback=keycolor!5!white,
 colframe=keycolor!75!black,
 fonttitle=\bfseries,
 title=Key Formula
}

\newtcolorbox{mnemonicboxenv}{
 breakable,
 enhanced,
 colback=mnemoniccolor!5!white,
 colframe=mnemoniccolor!75!black,
 fonttitle=\bfseries,
 title=Mnemonic
}

\newcommand{\mnemonicbox}[1]{%
  \begin{mnemonicboxenv}
    #1
  \end{mnemonicboxenv}
}


% Custom commands for GTU solutions
% This file defines semantic commands for consistent formatting

% Question command with automatic formatting
\newcommand{\question}[2]{%
  \section*{Question #1}%
  \textbf{#2}%
}

% OR question variant
\newcommand{\questionor}[2]{%
  \section*{Question #1 OR}%
  \textbf{#2}%
}

% Proper table environment with caption
\newenvironment{answertable}[1]{%
  \begin{table}[htbp]
  \centering
  \caption{#1}
}{%
  \end{table}
}

% Proper figure environment for diagrams
\newenvironment{answerdiagram}[1]{%
  \begin{figure}[htbp]
  \centering
  \caption{#1}
}{%
  \end{figure}
}

% Semantic markup for key terms
\newcommand{\keyword}[1]{\textbf{#1}}
\newcommand{\code}[1]{\texttt{#1}}
\newcommand{\classname}[1]{\texttt{#1}}
\newcommand{\methodname}[1]{\texttt{#1}}

% Proper quotation marks
\newcommand{\mnemonic}[1]{``#1''}


\title{Essentials of Digital Marketing (4341601) - Summer 2025 Solution}
\date{May 13, 2025}

\begin{document}
\maketitle

\questionmarks{Question 1(a)}{3}{Explain SEO ranking?}
\begin{solutionbox}
SEO ranking refers to the position of a website or webpage in search engine results pages (SERPs) for specific keywords or queries.

\textbf{Key Components:}

\begin{center}
\captionof{table}{SEO Ranking Components}
\begin{tabulary}{\linewidth}{L L}
\toprule
\textbf{Factor} & \textbf{Description} \\
\midrule
\textbf{Page Position} & Numerical position (1-10) on first page \\
\textbf{Search Visibility} & How often site appears in search results \\
\textbf{Keyword Relevance} & Match between content and search terms \\
\bottomrule
\end{tabulary}
\end{center}

\begin{itemize}
    \item \textbf{Higher ranking}: Better visibility and more organic traffic
    \item \textbf{Algorithm-based}: Google uses 200+ ranking factors
    \item \textbf{Dynamic nature}: Rankings change based on algorithm updates
\end{itemize}

\begin{mnemonicbox}SERP Success Starts with Smart SEO\end{mnemonicbox}
\end{solutionbox}

\questionmarks{Question 1(b)}{4}{Describe the P.O.E.M. Framework in digital marketing}
\begin{solutionbox}
P.O.E.M. Framework is a strategic approach to categorize digital marketing channels and content distribution.

\textbf{Framework Components:}

\begin{center}
\captionof{table}{P.O.E.M. Framework Components}
\begin{tabulary}{\linewidth}{L L L}
\toprule
\textbf{Channel Type} & \textbf{Definition} & \textbf{Examples} \\
\midrule
\textbf{Paid} & Purchased advertising space & Google Ads, Facebook Ads \\
\textbf{Owned} & Brand-controlled platforms & Website, Email lists \\
\textbf{Earned} & Third-party endorsements & Reviews, Social shares \\
\textbf{Managed} & Controlled social presence & Facebook Pages, Twitter \\
\bottomrule
\end{tabulary}
\end{center}

\begin{itemize}
    \item \textbf{Integrated approach}: Combines all channels for maximum reach
    \item \textbf{Cost optimization}: Balances paid and organic efforts
    \item \textbf{Brand control}: Maintains consistent messaging across channels
\end{itemize}

\begin{mnemonicbox}People Often Earn Money\end{mnemonicbox}
\end{solutionbox}

\questionmarks{Question 1(c)}{7}{Discuss the importance of ethics and data privacy in digital marketing. How do ethical practices and a commitment to data privacy contribute to "Dignified Digital Marketing"}
\begin{solutionbox}
Ethics and data privacy form the foundation of responsible digital marketing practices in today's data-driven landscape.

\textbf{Ethical Importance:}

\begin{center}
\captionof{table}{Ethical Importance in Digital Marketing}
\begin{tabulary}{\linewidth}{L L}
\toprule
\textbf{Aspect} & \textbf{Significance} \\
\midrule
\textbf{Consumer Trust} & Builds long-term relationships \\
\textbf{Legal Compliance} & Avoids GDPR/CCPA penalties \\
\textbf{Brand Reputation} & Maintains positive image \\
\textbf{Market Sustainability} & Ensures industry credibility \\
\bottomrule
\end{tabulary}
\end{center}

\textbf{Data Privacy Practices:}

\begin{itemize}
    \item \textbf{Transparent collection}: Clear consent mechanisms
    \item \textbf{Minimal data gathering}: Only necessary information
    \item \textbf{Secure storage}: Encrypted databases and access controls
    \item \textbf{User rights}: Easy opt-out and data deletion options
\end{itemize}

\textbf{Dignified Digital Marketing Benefits:}

\begin{itemize}
    \item \textbf{Enhanced credibility}: Consumers trust ethical brands
    \item \textbf{Competitive advantage}: Differentiation through responsible practices
    \item \textbf{Regulatory compliance}: Proactive approach to privacy laws
    \item \textbf{Sustainable growth}: Long-term customer relationships
\end{itemize}

\begin{mnemonicbox}Trust Through Transparency Triumphs\end{mnemonicbox}
\end{solutionbox}

\questionmarks{Question 1(c) OR}{7}{Differentiate between traditional marketing and digital marketing in terms of their reach, targeting, cost-effectiveness, and measurement of success.}
\begin{solutionbox}
\textbf{Comparison Analysis:}

\begin{center}
\captionof{table}{Traditional vs Digital Marketing}
\begin{tabulary}{\linewidth}{L L L}
\toprule
\textbf{Factor} & \textbf{Traditional Marketing} & \textbf{Digital Marketing} \\
\midrule
\textbf{Reach} & Local/Regional limitations & Global audience instantly \\
\textbf{Targeting} & Broad demographic groups & Precise behavioral targeting \\
\textbf{Cost} & High upfront investments & Flexible budget options \\
\textbf{Measurement} & Difficult to track ROI & Real-time analytics available \\
\bottomrule
\end{tabulary}
\end{center}

\textbf{Detailed Differences:}

\textbf{Reach Capabilities:}
\begin{itemize}
    \item \textbf{Traditional}: Geographic constraints, limited audience
    \item \textbf{Digital}: Worldwide accessibility, 24/7 availability
\end{itemize}

\textbf{Targeting Precision:}
\begin{itemize}
    \item \textbf{Traditional}: Mass market approach, limited segmentation
    \item \textbf{Digital}: Individual-level targeting, behavioral data usage
\end{itemize}

\textbf{Cost Structure:}
\begin{itemize}
    \item \textbf{Traditional}: Fixed costs, minimum spend requirements
    \item \textbf{Digital}: Pay-per-click, scalable budgets, micro-investments
\end{itemize}

\textbf{Success Measurement:}
\begin{itemize}
    \item \textbf{Traditional}: Surveys, estimated reach calculations
    \item \textbf{Digital}: Click-through rates, conversion tracking, attribution models
\end{itemize}

\begin{mnemonicbox}Reach, Target, Cost, Measure - Digital's Better\end{mnemonicbox}
\end{solutionbox}

\questionmarks{Question 2(a)}{3}{Compare White Hat SEO and Black Hat SEO}
\begin{solutionbox}
\textbf{SEO Practices Comparison:}

\begin{center}
\captionof{table}{White Hat vs Black Hat SEO}
\begin{tabulary}{\linewidth}{L L L}
\toprule
\textbf{Aspect} & \textbf{White Hat SEO} & \textbf{Black Hat SEO} \\
\midrule
\textbf{Methods} & Ethical, guideline-compliant & Manipulative, rule-breaking \\
\textbf{Timeline} & Long-term sustainable results & Quick but temporary gains \\
\textbf{Risk} & Search engine approved & Penalty and ban risks \\
\bottomrule
\end{tabulary}
\end{center}

\begin{itemize}
    \item \textbf{White Hat}: Quality content, natural link building, user-focused optimization
    \item \textbf{Black Hat}: Keyword stuffing, hidden text, link farming
    \item \textbf{Consequences}: White Hat builds authority, Black Hat risks penalties
\end{itemize}

\begin{mnemonicbox}White is Right, Black Attacks\end{mnemonicbox}
\end{solutionbox}

\questionmarks{Question 2(b)}{4}{Assume a website with outdated content and slow loading times, apply SEO tactics to improve its search engine rankings.}
\begin{solutionbox}
\textbf{SEO Improvement Strategy:}

\begin{center}
\begin{tikzpicture}[node distance=1.5cm]
    \node (issues) [gtu block, minimum width=3cm] {Outdated Website Issues};
    \node (content) [gtu block, below left=of issues, xshift=1cm] {Content Issues};
    \node (perf) [gtu block, below right=of issues, xshift=-1cm] {Performance Problems};
    \node (fresh) [gtu block, below=of content] {Fresh Content};
    \node (speed) [gtu block, below=of perf] {Speed Optimization};
    \node (rank) [gtu block, below right=of fresh, xshift=1cm] {Improved Rankings};

    \draw [gtu arrow] (issues) -| (content);
    \draw [gtu arrow] (issues) -| (perf);
    \draw [gtu arrow] (content) -- (fresh);
    \draw [gtu arrow] (perf) -- (speed);
    \draw [gtu arrow] (fresh) |- (rank);
    \draw [gtu arrow] (speed) |- (rank);
\end{tikzpicture}
\captionof{figure}{SEO Tactics for Outdated Website}
\end{center}

\textbf{Tactical Solutions:}

\begin{center}
\captionof{table}{Tactical Solutions for Website Improvement}
\begin{tabulary}{\linewidth}{L L L}
\toprule
\textbf{Issue} & \textbf{SEO Tactic} & \textbf{Implementation} \\
\midrule
\textbf{Outdated Content} & Content refresh & Update with current information \\
\textbf{Slow Loading} & Performance optimization & Compress images, minimize code \\
\textbf{Poor Structure} & Technical SEO & Improve site architecture \\
\bottomrule
\end{tabulary}
\end{center}

\begin{itemize}
    \item \textbf{Content strategy}: Regular updates, trending topics, user-relevant information
    \item \textbf{Technical fixes}: CDN implementation, caching, mobile optimization
    \item \textbf{Monitoring}: Track page speed, user engagement metrics
\end{itemize}

\begin{mnemonicbox}Content Currency Creates Clicks\end{mnemonicbox}
\end{solutionbox}

\questionmarks{Question 2(c)}{7}{Discuss how on-page optimization, content quality, and website speed contribute to better search engine rankings. Provide examples of specific techniques within these areas that can enhance a website's visibility}
\begin{solutionbox}
\textbf{SEO Ranking Factors:}

\begin{center}
\begin{tikzpicture}[node distance=1.5cm]
    \node (rank) [gtu block] {Search Engine Rankings};
    
    \node (onpage) [gtu block, below left=of rank, xshift=-1cm] {On-Page Optimization};
    \node (content) [gtu block, below=of rank] {Content Quality};
    \node (speed) [gtu block, below right=of rank, xshift=1cm] {Website Speed};
    
    \draw [gtu arrow] (rank) -- (onpage);
    \draw [gtu arrow] (rank) -- (content);
    \draw [gtu arrow] (rank) -- (speed);
    
    \node (onpage_items) [gtu block, below=of onpage, align=left, font=\small] {Title Tags\\Meta Descriptions\\Header Structure};
    \node (content_items) [gtu block, below=of content, align=left, font=\small] {Original Content\\Keyword Relevance\\User Intent Match};
    \node (speed_items) [gtu block, below=of speed, align=left, font=\small] {Page Load Time\\Mobile Performance\\Core Web Vitals};

    \draw [gtu arrow] (onpage) -- (onpage_items);
    \draw [gtu arrow] (content) -- (content_items);
    \draw [gtu arrow] (speed) -- (speed_items);
\end{tikzpicture}
\captionof{figure}{Key SEO Ranking Factors}
\end{center}

\textbf{On-Page Optimization Techniques:}

\begin{center}
\captionof{table}{On-Page Optimization Best Practices}
\begin{tabulary}{\linewidth}{L L L}
\toprule
\textbf{Element} & \textbf{Best Practice} & \textbf{Example} \\
\midrule
\textbf{Title Tags} & Include primary keyword & "Best Digital Marketing Tools 2025" \\
\textbf{Meta Descriptions} & Compelling 155-160 characters & "Discover top digital marketing tools..." \\
\textbf{Header Tags} & Hierarchical structure & H1$\to$H2$\to$H3 logical flow \\
\textbf{Internal Linking} & Relevant page connections & Link related blog posts \\
\bottomrule
\end{tabulary}
\end{center}

\textbf{Content Quality Factors:}

\begin{itemize}
    \item \textbf{Originality}: Unique, valuable information
    \item \textbf{Depth}: Comprehensive topic coverage
    \item \textbf{Freshness}: Regular updates and current data
    \item \textbf{User engagement}: Time on page, low bounce rate
\end{itemize}

\textbf{Website Speed Optimization:}

\begin{itemize}
    \item \textbf{Image compression}: WebP format, lazy loading
    \item \textbf{Code minification}: CSS, JavaScript optimization
    \item \textbf{Caching strategies}: Browser and server-side caching
    \item \textbf{CDN implementation}: Global content delivery
\end{itemize}

\begin{mnemonicbox}Optimize, Quality, Speed = Success\end{mnemonicbox}
\end{solutionbox}

\questionmarks{Question 2(a) OR}{3}{Discuss the main steps involved in a search engine's process from crawling to ranking.}
\begin{solutionbox}
\textbf{Search Engine Process:}

\begin{center}
\captionof{table}{Search Engine Process Steps}
\begin{tabulary}{\linewidth}{L L L}
\toprule
\textbf{Step} & \textbf{Process} & \textbf{Description} \\
\midrule
\textbf{1. Crawling} & Discovery & Bots find new/updated pages \\
\textbf{2. Indexing} & Storage & Content analyzed and stored \\
\textbf{3. Ranking} & Evaluation & Algorithm determines relevance \\
\bottomrule
\end{tabulary}
\end{center}

\begin{itemize}
    \item \textbf{Crawling}: Spider bots follow links, discover content
    \item \textbf{Indexing}: Content parsed, keywords identified, database storage
    \item \textbf{Ranking}: Algorithm evaluation, SERP position determination
\end{itemize}

\begin{mnemonicbox}Crawl, Index, Rank - Search Success\end{mnemonicbox}
\end{solutionbox}

\questionmarks{Question 2(b) OR}{4}{Apply the concepts of on-page optimization to a website that has low search engine visibility. Suggest three specific on-page SEO tactics to improve its rankings}
\begin{solutionbox}
\textbf{On-Page SEO Improvement Plan:}

\begin{center}
\begin{tikzpicture}[node distance=1.5cm]
    \node (low) [gtu block] {Low Visibility Website};
    
    \node (title) [gtu block, below left=of low] {Title Optimization};
    \node (meta) [gtu block, below=of low] {Meta Optimization};
    \node (content) [gtu block, below right=of low] {Content Optimization};
    
    \node (better) [gtu block, below=of meta] {Better Rankings};

    \draw [gtu arrow] (low) -| (title);
    \draw [gtu arrow] (low) -- (meta);
    \draw [gtu arrow] (low) -| (content);
    
    \draw [gtu arrow] (title) |- (better);
    \draw [gtu arrow] (meta) -- (better);
    \draw [gtu arrow] (content) |- (better);
\end{tikzpicture}
\captionof{figure}{On-Page Optimization Plan}
\end{center}

\textbf{Three Key Tactics:}

\begin{center}
\captionof{table}{On-Page SEO Tactics}
\begin{tabulary}{\linewidth}{L L L}
\toprule
\textbf{Tactic} & \textbf{Implementation} & \textbf{Expected Impact} \\
\midrule
\textbf{Title Tag Optimization} & Include primary keywords, brand name & Improved click-through rates \\
\textbf{Content Structure} & Add H1-H6 headers, bullet points & Better user experience \\
\textbf{Internal Linking} & Link to related pages, anchor text & Enhanced page authority \\
\bottomrule
\end{tabulary}
\end{center}

\begin{itemize}
    \item \textbf{Keyword placement}: Strategic positioning in titles, headers, first paragraph
    \item \textbf{Meta descriptions}: Compelling 155-character summaries
    \item \textbf{Image optimization}: Alt tags, descriptive filenames
\end{itemize}

\begin{mnemonicbox}Title, Structure, Link - Think Success\end{mnemonicbox}
\end{solutionbox}

\questionmarks{Question 2(c) OR}{7}{Discuss the role of SEO in enhancing a website's online presence with example}
\begin{solutionbox}
SEO plays a crucial role in establishing and maintaining a strong digital footprint for businesses and organizations.

\textbf{SEO's Role in Online Presence:}

\begin{center}
\begin{tikzpicture}[node distance=1.2cm]
    \node (strat) [gtu block] {SEO Strategy};
    \node (vis) [gtu block, right=of strat] {Increased Visibility};
    \node (traffic) [gtu block, right=of vis] {More Organic Traffic};
    \node (conv) [gtu block, below=of traffic] {Higher Conversions};
    \node (auth) [gtu block, left=of conv] {Enhanced Brand Authority};
    \node (pres) [gtu block, left=of auth] {Sustained Online Presence};

    \draw [gtu arrow] (strat) -- (vis);
    \draw [gtu arrow] (vis) -- (traffic);
    \draw [gtu arrow] (traffic) -- (conv);
    \draw [gtu arrow] (conv) -- (auth);
    \draw [gtu arrow] (auth) -- (pres);
\end{tikzpicture}
\captionof{figure}{SEO Impact Cycle}
\end{center}

\textbf{Key Contributions:}

\begin{center}
\captionof{table}{SEO Impact on Online Presence}
\begin{tabulary}{\linewidth}{L L L}
\toprule
\textbf{Aspect} & \textbf{SEO Impact} & \textbf{Business Benefit} \\
\midrule
\textbf{Search Visibility} & Higher SERP rankings & More potential customers find you \\
\textbf{Credibility} & Authoritative content & Users trust top-ranked results \\
\textbf{User Experience} & Fast, mobile-friendly sites & Better engagement metrics \\
\textbf{Cost-Effective} & Organic traffic generation & Lower customer acquisition costs \\
\bottomrule
\end{tabulary}
\end{center}

\textbf{Example: E-commerce Store:}
A local electronics store implemented SEO strategy:
\begin{itemize}
    \item \textbf{Before}: Ranking on page 3 for "electronics store"
    \item \textbf{SEO actions}: Optimized product pages, local SEO, quality content
    \item \textbf{After}: Page 1 ranking, 300\% traffic increase, 150\% sales growth
\end{itemize}

\textbf{Long-term Benefits:}
\begin{itemize}
    \item \textbf{Sustainable traffic}: Unlike paid ads, organic results persist
    \item \textbf{Brand building}: Consistent visibility builds recognition
    \item \textbf{Market expansion}: Reach customers actively searching for products
\end{itemize}

\begin{mnemonicbox}Search Engine Optimization = Sustainable Online Success\end{mnemonicbox}
\end{solutionbox}

\questionmarks{Question 3(a)}{3}{Define Unique Visitors, Pageviews}
\begin{solutionbox}
\textbf{Web Analytics Definitions:}

\begin{center}
\captionof{table}{Unique Visitors vs Pageviews}
\begin{tabulary}{\linewidth}{L L L}
\toprule
\textbf{Metric} & \textbf{Definition} & \textbf{Measurement Period} \\
\midrule
\textbf{Unique Visitors} & Distinct individuals visiting site & Specific time period \\
\textbf{Pageviews} & Total pages viewed & Individual page loads \\
\bottomrule
\end{tabulary}
\end{center}

\begin{itemize}
    \item \textbf{Unique Visitors}: Counted once per session, regardless of pages viewed
    \item \textbf{Pageviews}: Each page refresh or new page counts separately
    \item \textbf{Relationship}: One unique visitor can generate multiple pageviews
\end{itemize}

\begin{mnemonicbox}Unique Users, Viewed Pages\end{mnemonicbox}
\end{solutionbox}

\questionmarks{Question 3(b)}{4}{How do Content Analytics Tools contribute to understanding a website's performance?}
\begin{solutionbox}
Content Analytics Tools provide insights into how users interact with website content, enabling data-driven optimization decisions.

\textbf{Contribution Areas:}

\begin{center}
\captionof{table}{Content Analytics Insights}
\begin{tabulary}{\linewidth}{L L L}
\toprule
\textbf{Analysis Type} & \textbf{Insights Provided} & \textbf{Optimization Actions} \\
\midrule
\textbf{Content Performance} & Page popularity, engagement time & Focus on high-performing topics \\
\textbf{User Behavior} & Reading patterns, scroll depth & Improve content structure \\
\textbf{Conversion Tracking} & Content-to-conversion paths & Optimize conversion funnels \\
\bottomrule
\end{tabulary}
\end{center}

\begin{itemize}
    \item \textbf{Performance metrics}: Bounce rate, time on page, social shares
    \item \textbf{Content gaps}: Identify missing topics, user search queries
    \item \textbf{A/B testing}: Compare content variations for effectiveness
    \item \textbf{ROI measurement}: Connect content efforts to business goals
\end{itemize}

\begin{mnemonicbox}Content Analytics Create Actionable Insights\end{mnemonicbox}
\end{solutionbox}

\questionmarks{Question 3(c)}{7}{Discuss the different attribution models used in web analytics with example.}
\begin{solutionbox}
Attribution models help marketers understand which touchpoints contribute to conversions in the customer journey.

\textbf{Attribution Model Types:}

\begin{center}
\begin{tikzpicture}[node distance=1.5cm]
    \node (attrib) [gtu block] {Attribution Models};
    \node (single) [gtu block, below left=of attrib, xshift=-1cm] {Single-Touch Models};
    \node (multi) [gtu block, below right=of attrib, xshift=1cm] {Multi-Touch Models};

    \draw [gtu arrow] (attrib) -- (single);
    \draw [gtu arrow] (attrib) -- (multi);

    \node (single_types) [gtu block, below=of single, align=left, font=\small] {First-Click\\Last-Click\\Last Non-Direct};
    \node (multi_types) [gtu block, below=of multi, align=left, font=\small] {Linear\\Time-Decay\\Position-Based\\Data-Driven};

    \draw [gtu arrow] (single) -- (single_types);
    \draw [gtu arrow] (multi) -- (multi_types);
\end{tikzpicture}
\captionof{figure}{Attribution Model Classification}
\end{center}

\textbf{Model Comparison:}

\begin{center}
\captionof{table}{Attribution Model Comparison}
\begin{tabulary}{\linewidth}{L L L}
\toprule
\textbf{Model} & \textbf{Credit Distribution} & \textbf{Best Use Case} \\
\midrule
\textbf{First-Click} & 100\% to first touchpoint & Brand awareness campaigns \\
\textbf{Last-Click} & 100\% to final touchpoint & Direct response marketing \\
\textbf{Linear} & Equal credit to all touchpoints & Long sales cycles \\
\textbf{Time-Decay} & More credit to recent interactions & Short consideration periods \\
\bottomrule
\end{tabulary}
\end{center}

\textbf{Example Customer Journey:}
1. \textbf{Facebook Ad} (Awareness) $\to$ 2. \textbf{Google Search} (Research) $\to$ 3. \textbf{Email} (Conversion)

\textbf{Attribution Results:}
\begin{itemize}
    \item \textbf{First-Click}: Facebook Ad gets 100\% credit
    \item \textbf{Last-Click}: Email gets 100\% credit
    \item \textbf{Linear}: Each touchpoint gets 33.3\% credit
    \item \textbf{Time-Decay}: Email 50\%, Google 30\%, Facebook 20\%
\end{itemize}

\textbf{Choosing the Right Model:}
\begin{itemize}
    \item \textbf{Business goals}: Awareness vs. conversion focus
    \item \textbf{Sales cycle length}: Short vs. long consideration periods
    \item \textbf{Marketing mix}: Single vs. multi-channel strategies
\end{itemize}

\begin{mnemonicbox}First, Last, Linear, Time - Attribution's Design\end{mnemonicbox}
\end{solutionbox}

\questionmarks{Question 3(a) OR}{3}{Define Average Visit Duration, Bounce Rate, and New Visits.}
\begin{solutionbox}
\textbf{Web Analytics Metrics:}

\begin{center}
\captionof{table}{Web Analytics Metrics Defined}
\begin{tabulary}{\linewidth}{L L L}
\toprule
\textbf{Metric} & \textbf{Definition} & \textbf{Calculation} \\
\midrule
\textbf{Average Visit Duration} & Time spent per session & Total time $\div$ Sessions \\
\textbf{Bounce Rate} & Single-page sessions percentage & Bounces $\div$ Total sessions $\times$ 100 \\
\textbf{New Visits} & First-time visitors percentage & New users $\div$ Total users $\times$ 100 \\
\bottomrule
\end{tabulary}
\end{center}

\begin{itemize}
    \item \textbf{Visit Duration}: Indicates content engagement and user interest
    \item \textbf{Bounce Rate}: Shows content relevance and site usability
    \item \textbf{New Visits}: Measures audience growth and acquisition effectiveness
\end{itemize}

\begin{mnemonicbox}Duration, Bounce, New - Analytics True\end{mnemonicbox}
\end{solutionbox}

\questionmarks{Question 3(b) OR}{4}{How do Customer Analytics Tools contribute to understanding a website's performance?}
\begin{solutionbox}
Customer Analytics Tools provide deep insights into user behavior, preferences, and conversion patterns.

\textbf{Key Contributions:}

\begin{center}
\captionof{table}{Customer Analytics Contributions}
\begin{tabulary}{\linewidth}{L L L}
\toprule
\textbf{Analytics Area} & \textbf{Insights} & \textbf{Performance Impact} \\
\midrule
\textbf{User Segmentation} & Demographics, behavior patterns & Targeted content creation \\
\textbf{Journey Mapping} & Conversion paths, drop-off points & Optimized user experience \\
\textbf{Lifetime Value} & Customer worth, retention rates & ROI-focused strategies \\
\bottomrule
\end{tabulary}
\end{center}

\begin{itemize}
    \item \textbf{Behavioral analysis}: Click patterns, navigation preferences
    \item \textbf{Conversion optimization}: Identify friction points in user journey
    \item \textbf{Personalization}: Customized content based on user profiles
    \item \textbf{Retention strategies}: Understanding what keeps customers engaged
\end{itemize}

\begin{mnemonicbox}Customer Analytics Create Competitive Advantages\end{mnemonicbox}
\end{solutionbox}

\questionmarks{Question 3(c) OR}{7}{Discuss the process of setting up goals and tracking conversion rates in Google Analytics with example.}
\begin{solutionbox}
Setting up goals and tracking conversions in Google Analytics enables measurement of website success and ROI optimization.

\textbf{Goal Setup Process:}

\begin{center}
\begin{tikzpicture}[node distance=1.5cm]
    \node (goals) [gtu block] {Google Analytics Goals};
    \node (config) [gtu block, below=of goals] {Goal Configuration};
    \node (types) [gtu block, below=of config] {Goal Types};

    \node (dest) [gtu block, below left=of types, xshift=-2cm] {Destination};
    \node (dur) [gtu block, right=of dest] {Duration};
    \node (pgs) [gtu block, right=of dur] {Pages/Session};
    \node (evt) [gtu block, right=of pgs] {Event};

    \node (conv) [gtu block, below=of dur, xshift=1.5cm] {Conversion Tracking};

    \draw [gtu arrow] (goals) -- (config);
    \draw [gtu arrow] (config) -- (types);
    \draw [gtu arrow] (types) -- (dest);
    \draw [gtu arrow] (types) -- (dur);
    \draw [gtu arrow] (types) -- (pgs);
    \draw [gtu arrow] (types) -- (evt);
    
    \draw [gtu arrow] (dest) -- (conv);
    \draw [gtu arrow] (dur) -- (conv);
    \draw [gtu arrow] (pgs) -- (conv);
    \draw [gtu arrow] (evt) -- (conv);
\end{tikzpicture}
\captionof{figure}{Google Analytics Goal Setup}
\end{center}

\textbf{Goal Types and Setup:}

\begin{center}
\captionof{table}{Goal Types}
\begin{tabulary}{\linewidth}{L L L}
\toprule
\textbf{Goal Type} & \textbf{Description} & \textbf{Example Setup} \\
\midrule
\textbf{Destination} & Specific page visits & Thank you page URL \\
\textbf{Duration} & Session length & Sessions > 3 minutes \\
\textbf{Pages/Session} & Page views per visit & More than 5 pages \\
\textbf{Event} & Specific actions & Download button click \\
\bottomrule
\end{tabulary}
\end{center}

\textbf{Example: E-commerce Conversion Setup:}

\textbf{Step-by-Step Process:}
\begin{enumerate}
    \item \textbf{Access Goals}: Admin $\to$ View $\to$ Goals $\to$ New Goal
    \item \textbf{Goal Type}: Destination (Thank you page)
    \item \textbf{Goal Details}: 
    \begin{itemize}
        \item Name: "Purchase Completion"
        \item Type: Destination
        \item Destination: "/thank-you"
    \end{itemize}
    \item \textbf{Funnel Setup}: Add checkout steps
    \item \textbf{Value Assignment}: Average order value
\end{enumerate}

\textbf{Conversion Rate Calculation:}
\begin{itemize}
    \item \textbf{Formula}: (Conversions $\div$ Sessions) $\times$ 100
    \item \textbf{Example}: 50 purchases $\div$ 2,000 sessions = 2.5\% conversion rate
\end{itemize}

\textbf{Tracking Benefits:}
\begin{itemize}
    \item \textbf{Performance measurement}: Clear success metrics
    \item \textbf{ROI calculation}: Revenue attribution to marketing channels
    \item \textbf{Optimization opportunities}: Identify improvement areas
\end{itemize}

\begin{mnemonicbox}Goals Give Great Growth Guidance\end{mnemonicbox}
\end{solutionbox}

\questionmarks{Question 4(a)}{3}{What are the types of Twitter Ads available for marketers?}
\begin{solutionbox}
\textbf{Twitter Advertising Options:}

\begin{center}
\captionof{table}{Twitter Ad Types}
\begin{tabulary}{\linewidth}{L L L}
\toprule
\textbf{Ad Type} & \textbf{Purpose} & \textbf{Format} \\
\midrule
\textbf{Promoted Tweets} & Increase engagement & Native tweet appearance \\
\textbf{Promoted Accounts} & Grow followers & Account suggestions \\
\textbf{Promoted Trends} & Topic visibility & Trending section placement \\
\bottomrule
\end{tabulary}
\end{center}

\begin{itemize}
    \item \textbf{Promoted Tweets}: Boost reach of existing tweets, drive clicks/conversions
    \item \textbf{Promoted Accounts}: Target users likely to follow, increase audience size
    \item \textbf{Promoted Trends}: Premium placement in trending topics, high visibility
\end{itemize}

\begin{mnemonicbox}Tweets, Accounts, Trends - Twitter Advertising Ends\end{mnemonicbox}
\end{solutionbox}

\questionmarks{Question 4(b)}{4}{You have been assigned to develop a LinkedIn advertising campaign for a company's upcoming webinar. Outline the process for creating and optimizing LinkedIn Ads for this campaign. Include the types of LinkedIn ads you would choose, the content you would use, and how you would leverage LinkedIn Analytics to assess and enhance the campaign's effectiveness.}
\begin{solutionbox}
\textbf{LinkedIn Webinar Campaign Strategy:}

\begin{center}
\begin{tikzpicture}[node distance=1.5cm]
    \node (plan) [gtu block, minimum width=4cm] {Webinar Campaign Planning};
    \node (ads) [gtu block, below left=of plan] {Ad Types Selection};
    \node (content) [gtu block, below right=of plan] {Content Strategy};
    \node (spon) [gtu block, below=of ads] {Sponsored Content};
    \node (msg) [gtu block, below=of content] {Professional Messaging};
    \node (opt) [gtu block, below=of plan, yshift=-3cm] {Analytics \& Optimization};

    \draw [gtu arrow] (plan) -| (ads);
    \draw [gtu arrow] (plan) -| (content);
    \draw [gtu arrow] (ads) -- (spon);
    \draw [gtu arrow] (content) -- (msg);
    \draw [gtu arrow] (spon) |- (opt);
    \draw [gtu arrow] (msg) |- (opt);
\end{tikzpicture}
\captionof{figure}{LinkedIn Campaign Flow}
\end{center}

\textbf{Campaign Development Process:}

\begin{center}
\captionof{table}{LinkedIn Campaign Process}
\begin{tabulary}{\linewidth}{L L L}
\toprule
\textbf{Phase} & \textbf{Action Items} & \textbf{Implementation} \\
\midrule
\textbf{Ad Selection} & Choose Sponsored Content + Message Ads & Video content for engagement \\
\textbf{Targeting} & Professional demographics, job titles & IT professionals, decision-makers \\
\textbf{Content Creation} & Value proposition, clear CTA & "Join Expert-Led Marketing Webinar" \\
\textbf{Optimization} & A/B test headlines, monitor CTR & Adjust based on performance data \\
\bottomrule
\end{tabulary}
\end{center}

\textbf{Recommended Ad Types:}
\begin{itemize}
    \item \textbf{Sponsored Content}: Native feed placement, professional appearance
    \item \textbf{Message Ads}: Direct inbox delivery, personalized approach
    \item \textbf{Dynamic Ads}: Personalized creative based on profile data
\end{itemize}

\textbf{Content Strategy:}
\begin{itemize}
    \item \textbf{Headlines}: "Master Digital Marketing: Free Expert Webinar"
    \item \textbf{Visuals}: Professional speaker photos, agenda highlights
    \item \textbf{CTA}: "Register Now - Limited Seats Available"
\end{itemize}

\begin{mnemonicbox}Select, Target, Create, Optimize - LinkedIn Success\end{mnemonicbox}
\end{solutionbox}

\questionmarks{Question 4(c)}{7}{Discuss the role and significance of video marketing in digital marketing strategies. How do YouTube Ads fit into a broader video marketing strategy?}
\begin{solutionbox}
Video marketing has become the cornerstone of modern digital marketing strategies, offering unparalleled engagement and conversion potential.

\textbf{Video Marketing Significance:}

\begin{center}
\begin{tikzpicture}[node distance=1.5cm]
    \node (video) [gtu block] {Video Marketing};
    
    \node (engage) [gtu block, below left=of video, xshift=-1cm] {High Engagement};
    \node (convert) [gtu block, below=of video] {Better Conversion};
    \node (story) [gtu block, below right=of video, xshift=1cm] {Brand Storytelling};
    
    \node (reach) [gtu block, below=of convert] {Increased Reach};

    \draw [gtu arrow] (video) -- (engage);
    \draw [gtu arrow] (video) -- (convert);
    \draw [gtu arrow] (video) -- (story);
    
    \draw [gtu arrow] (engage) -- (reach);
    \draw [gtu arrow] (convert) -- (reach);
    \draw [gtu arrow] (story) -- (reach);
\end{tikzpicture}
\captionof{figure}{Video Marketing Benefits}
\end{center}

\textbf{Strategic Importance:}

\begin{center}
\captionof{table}{Video Marketing Impact}
\begin{tabulary}{\linewidth}{L L L}
\toprule
\textbf{Aspect} & \textbf{Impact} & \textbf{Business Value} \\
\midrule
\textbf{Engagement} & 10x higher than text content & Increased brand recall \\
\textbf{Conversion} & 80\% more likely to purchase & Higher sales revenue \\
\textbf{SEO Value} & 53x more likely to rank first & Organic traffic growth \\
\bottomrule
\end{tabulary}
\end{center}

\textbf{YouTube Ads Integration:}

\begin{itemize}
    \item \textbf{Broader Strategy Connection}: Awareness $\to$ Consideration $\to$ Conversion funnel integration
    \item \textbf{Cross-platform distribution}: YouTube videos repurposed for social media/website
    \item \textbf{Retargeting}: Custom audiences created from video viewers for follow-up ads
\end{itemize}

\begin{mnemonicbox}Video Engages, Converts, and Scales Marketing Excellence\end{mnemonicbox}
\end{solutionbox}

\questionmarks{Question 4(a) OR}{3}{Name two key features of LinkedIn's Campaign Manager.}
\begin{solutionbox}
\textbf{LinkedIn Campaign Manager Features:}

\begin{center}
\captionof{table}{LinkedIn Campaign Manager Features}
\begin{tabulary}{\linewidth}{L L L}
\toprule
\textbf{Feature} & \textbf{Functionality} & \textbf{Benefit} \\
\midrule
\textbf{Audience Targeting} & Professional demographics, job functions & Precise B2B targeting \\
\textbf{Performance Analytics} & Real-time metrics, conversion tracking & Data-driven optimization \\
\bottomrule
\end{tabulary}
\end{center}

\begin{itemize}
    \item \textbf{Audience Targeting}: Industry, company size, job title, skills-based segmentation
    \item \textbf{Performance Analytics}: CTR, CPC, conversion tracking, A/B testing capabilities
\end{itemize}

\begin{mnemonicbox}Target Accurately, Analyze Performance\end{mnemonicbox}
\end{solutionbox}

\questionmarks{Question 4(b) OR}{4}{You are tasked with creating an advertising campaign on Instagram for a new product launch. Outline the steps you would take to create and optimize Instagram Ads, including the types of content you would use.}
\begin{solutionbox}
\textbf{Instagram Product Launch Campaign:}

\begin{center}
\begin{tikzpicture}[node distance=1.5cm]
    \node (strat) [gtu block] {Product Launch Strategy};
    
    \node (planning) [gtu block, below left=of strat, xshift=-1cm] {Content Planning};
    \node (format) [gtu block, below left=of strat, xshift=1cm] {Ad Format};
    \node (targeting) [gtu block, below right=of strat, xshift=-1cm] {Targeting};
    \node (opt) [gtu block, below right=of strat, xshift=1cm] {Optimization};

    \node (visuals) [gtu block, below=of planning] {Visual Content};
    \node (stories) [gtu block, below=of format] {Stories Ads};
    \node (interests) [gtu block, below=of targeting] {Interest Groups};
    \node (track) [gtu block, below=of opt] {Performance Tracking};

    \draw [gtu arrow] (strat) -- (planning);
    \draw [gtu arrow] (strat) -- (format);
    \draw [gtu arrow] (strat) -- (targeting);
    \draw [gtu arrow] (strat) -- (opt);

    \draw [gtu arrow] (planning) -- (visuals);
    \draw [gtu arrow] (format) -- (stories);
    \draw [gtu arrow] (targeting) -- (interests);
    \draw [gtu arrow] (opt) -- (track);
\end{tikzpicture}
\captionof{figure}{Instagram Launch Strategy}
\end{center}

\textbf{Campaign Development Steps:}

\begin{center}
\captionof{table}{Instagram Campaign Steps}
\begin{tabulary}{\linewidth}{L L L}
\toprule
\textbf{Step} & \textbf{Action} & \textbf{Implementation} \\
\midrule
\textbf{1. Content Creation} & Visual storytelling & Product photos, lifestyle images \\
\textbf{2. Ad Format} & Feed + Stories + Reels & Multi-format approach \\
\textbf{3. Targeting Setup} & Demographics + interests & Lookalike audiences \\
\textbf{4. Budget Allocation} & Daily spend limits & Performance-based optimization \\
\bottomrule
\end{tabulary}
\end{center}

\textbf{Content Strategy:}
\begin{itemize}
    \item \textbf{Feed Posts}: High-quality product photography, lifestyle contexts
    \item \textbf{Stories Ads}: Behind-the-scenes content, user-generated content
    \item \textbf{Reels}: Trending audio, product demonstrations, tutorials
    \item \textbf{Carousel Ads}: Multiple product angles, feature highlights
\end{itemize}

\begin{mnemonicbox}Create, Choose, Target, Track - Instagram Impact\end{mnemonicbox}
\end{solutionbox}

\questionmarks{Question 4(c) OR}{7}{Explain the importance of understanding Facebook's advertising algorithm and how it affects ad delivery.}
\begin{solutionbox}
Understanding Facebook's advertising algorithm is crucial for maximizing ad performance and achieving optimal return on investment.

\textbf{Algorithm Components:}

\begin{center}
\begin{tikzpicture}[node distance=1.5cm]
    \node (algo) [gtu block] {Facebook Algorithm};
    \node (adv) [gtu block, below left=of algo] {Advertiser Value};
    \node (user) [gtu block, below right=of algo] {User Value};
    \node (total) [gtu block, below=of algo, yshift=-2cm] {Total Value};
    \node (delivery) [gtu block, below=of total] {Ad Delivery Decision};

    \draw [gtu arrow] (algo) -- (adv);
    \draw [gtu arrow] (algo) -- (user);
    \draw [gtu arrow] (adv) -- (total);
    \draw [gtu arrow] (user) -- (total);
    \draw [gtu arrow] (total) -- (delivery);

    \node (adv_factors) [gtu block, below=of adv, align=center, font=\footnotesize] {Bid Amount\\Est. Action Rate\\Ad Quality};
    \node (user_factors) [gtu block, below=of user, align=center, font=\footnotesize] {User Relevance\\User Experience\\Feedback};

    \draw [gtu arrow] (adv) -- (adv_factors);
    \draw [gtu arrow] (user) -- (user_factors);
\end{tikzpicture}
\captionof{figure}{Facebook Algorithm Factors}
\end{center}

\textbf{Algorithm Factors:}

\begin{center}
\captionof{table}{Ad Delivery Factors}
\begin{tabulary}{\linewidth}{L L L}
\toprule
\textbf{Component} & \textbf{Weight} & \textbf{Impact on Delivery} \\
\midrule
\textbf{Bid Strategy} & High & Budget allocation efficiency \\
\textbf{Ad Relevance} & High & Quality score determination \\
\textbf{User Engagement} & Medium & Audience response prediction \\
\textbf{Landing Page} & Medium & Overall user experience \\
\bottomrule
\end{tabulary}
\end{center}

\textbf{Ad Delivery Process:}
\begin{enumerate}
    \item \textbf{Auction Entry}: Ad enters real-time bidding
    \item \textbf{Value Calculation}: Algorithm scores ad relevance and quality
    \item \textbf{Winner Selection}: Highest total value wins placement
    \item \textbf{Performance Feedback}: Results influence future delivery
\end{enumerate}

\begin{mnemonicbox}Algorithm Awareness Achieves Advertising Advantage\end{mnemonicbox}
\end{solutionbox}

\questionmarks{Question 5(a)}{3}{List and briefly describe the different types of Email Marketing.}
\begin{solutionbox}
\textbf{Email Marketing Types:}

\begin{center}
\captionof{table}{Email Marketing Types}
\begin{tabulary}{\linewidth}{L L L}
\toprule
\textbf{Type} & \textbf{Purpose} & \textbf{Content Focus} \\
\midrule
\textbf{Newsletter} & Regular communication & Company updates, industry news \\
\textbf{Promotional} & Sales and offers & Discount codes, product launches \\
\textbf{Transactional} & Purchase confirmation & Order receipts, shipping updates \\
\bottomrule
\end{tabulary}
\end{center}

\begin{itemize}
    \item \textbf{Newsletter}: Brand awareness, customer retention, thought leadership
    \item \textbf{Promotional}: Drive sales, promote events, seasonal campaigns
    \item \textbf{Transactional}: Order confirmations, welcome series, account updates
\end{itemize}

\begin{mnemonicbox}News, Promote, Transact - Email's Impact\end{mnemonicbox}
\end{solutionbox}

\questionmarks{Question 5(b)}{4}{You are planning an email marketing campaign for a new product launch. Outline the steps you would take to design and execute this campaign, including how you would use email marketing analytics to measure its success.}
\begin{solutionbox}
\textbf{Email Campaign Strategy:}

\begin{center}
\begin{tikzpicture}[node distance=1.5cm]
    \node (launch) [gtu block] {Product Launch Email Campaign};
    
    \node (plan) [gtu block, below left=of launch, xshift=-1cm] {Planning Phase};
    \node (design) [gtu block, right=of plan] {Design Phase};
    \node (exec) [gtu block, right=of design] {Execution Phase};
    \node (anal) [gtu block, right=of exec] {Analytics Phase};

    \node (target) [gtu block, below=of plan] {Target Audience};
    \node (temp) [gtu block, below=of design] {Email Template};
    \node (send) [gtu block, below=of exec] {Send Schedule};
    \node (meas) [gtu block, below=of anal] {Measure Results};

    \draw [gtu arrow] (launch) -- (plan);
    \draw [gtu arrow] (launch) -- (design);
    \draw [gtu arrow] (launch) -- (exec);
    \draw [gtu arrow] (launch) -- (anal);

    \draw [gtu arrow] (plan) -- (target);
    \draw [gtu arrow] (design) -- (temp);
    \draw [gtu arrow] (exec) -- (send);
    \draw [gtu arrow] (anal) -- (meas);
\end{tikzpicture}
\captionof{figure}{Email Campaign Steps}
\end{center}

\textbf{Campaign Development Process:}

\begin{center}
\captionof{table}{Email Campaign Phases}
\begin{tabulary}{\linewidth}{L L L}
\toprule
\textbf{Phase} & \textbf{Activities} & \textbf{Key Deliverables} \\
\midrule
\textbf{Planning} & Audience segmentation, goal setting & Target lists, KPIs \\
\textbf{Design} & Template creation, content writing & Email templates, copy \\
\textbf{Execution} & Send scheduling, A/B testing & Campaign deployment \\
\textbf{Analytics} & Performance tracking, optimization & Reports, insights \\
\bottomrule
\end{tabulary}
\end{center}

\textbf{Analytics Measurement:}
\begin{itemize}
    \item \textbf{Open rates}: Subject line effectiveness, sender reputation
    \item \textbf{Click-through rates}: Content relevance, call-to-action performance
    \item \textbf{Conversion rates}: Landing page effectiveness, offer appeal
\end{itemize}

\begin{mnemonicbox}Plan, Design, Execute, Analyze - Email Success\end{mnemonicbox}
\end{solutionbox}

\questionmarks{Question 5(c)}{7}{Discuss the importance of social media marketing in today's digital landscape.}
\begin{solutionbox}
Social media marketing has become an indispensable component of digital marketing strategies, fundamentally changing how brands interact with consumers.

\textbf{Strategic Importance:}

\begin{center}
\begin{tikzpicture}[node distance=1.5cm]
    \node (smm) [gtu block] {Social Media Marketing};
    
    \node (brand) [gtu block, below left=of smm] {Brand Awareness};
    \node (eng) [gtu block, below=of smm] {Customer Engagement};
    \node (lead) [gtu block, below right=of smm] {Lead Generation};
    \node (serv) [gtu block, right=of lead] {Customer Service};

    \node (growth) [gtu block, below=of eng] {Business Growth};

    \draw [gtu arrow] (smm) -- (brand);
    \draw [gtu arrow] (smm) -- (eng);
    \draw [gtu arrow] (smm) -- (lead);
    \draw [gtu arrow] (smm) -- (serv);

    \draw [gtu arrow] (brand) -- (growth);
    \draw [gtu arrow] (eng) -- (growth);
    \draw [gtu arrow] (lead) -- (growth);
    \draw [gtu arrow] (serv) -- (growth);
\end{tikzpicture}
\captionof{figure}{Social Media Impact}
\end{center}

\textbf{Key Significance Areas:}

\begin{center}
\captionof{table}{Social Media Value}
\begin{tabulary}{\linewidth}{L L L}
\toprule
\textbf{Aspect} & \textbf{Impact} & \textbf{Business Value} \\
\midrule
\textbf{Global Reach} & 4.8 billion users worldwide & Massive audience potential \\
\textbf{Cost Effectiveness} & Lower than traditional media & Higher ROI opportunities \\
\textbf{Real-time Engagement} & Instant customer interaction & Improved relationships \\
\bottomrule
\end{tabulary}
\end{center}

\textbf{Platform-Specific Benefits:}
\begin{itemize}
    \item \textbf{Facebook}: Community building, diverse content, advanced targeting
    \item \textbf{Instagram}: Visual storytelling, influencer marketing, shopping features
    \item \textbf{LinkedIn}: B2B networking, thought leadership, lead generation
    \item \textbf{YouTube}: Video marketing, SEO benefits, educational content
\end{itemize}

\begin{mnemonicbox}Social Media Makes Modern Marketing Meaningful\end{mnemonicbox}
\end{solutionbox}

\questionmarks{Question 5(a) OR}{3}{What are the different types of Google Ads Campaigns? Provide a brief description of each.}
\begin{solutionbox}
\textbf{Google Ads Campaign Types:}

\begin{center}
\captionof{table}{Google Ads Campaigns}
\begin{tabulary}{\linewidth}{L L L}
\toprule
\textbf{Campaign Type} & \textbf{Purpose} & \textbf{Placement} \\
\midrule
\textbf{Search} & Text ads in search results & Google Search pages \\
\textbf{Display} & Visual ads across websites & Google Display Network \\
\textbf{Video} & Video advertisements & YouTube platform \\
\textbf{Shopping} & Product showcase ads & Google Shopping, Search \\
\textbf{App} & Mobile app promotion & Cross-platform placement \\
\bottomrule
\end{tabulary}
\end{center}

\begin{itemize}
    \item \textbf{Search}: Keyword-targeted text ads, high intent audience
    \item \textbf{Display}: Banner ads, broad reach, visual appeal
    \item \textbf{Video}: YouTube ads, engaging content format
\end{itemize}

\begin{mnemonicbox}Search, Display, Video, Shopping, App - Google's Map\end{mnemonicbox}
\end{solutionbox}

\questionmarks{Question 5(b) OR}{4}{Imagine you are setting up a Pay-Per-Click (PPC) campaign using Google Ads. Describe the process of creating the campaign, including selecting the type of Google Ads campaign, setting up ad extensions, and choosing bidding and ranking strategies to optimize ad performance.}
\begin{solutionbox}
\textbf{PPC Campaign Setup Process:}

\begin{center}
\begin{tikzpicture}[node distance=1.5cm]
    \node (setup) [gtu block] {PPC Campaign Setup};
    \node (type) [gtu block, below left=of setup] {Campaign Type};
    \node (ext) [gtu block, below=of setup] {Ad Extensions};
    \node (bid) [gtu block, below right=of setup] {Bidding Strategy};
    \node (opt) [gtu block, below=of layout] {Performance Optimization};

    \node (srch) [gtu block, below=of type] {Search Campaign};
    \node (sitelink) [gtu block, below=of ext] {Sitelink/Call Ext.};
    \node (cpc) [gtu block, below=of bid] {Manual CPC/Target CPA};

    \draw [gtu arrow] (setup) -- (type);
    \draw [gtu arrow] (setup) -- (ext);
    \draw [gtu arrow] (setup) -- (bid);
    
    \draw [gtu arrow] (type) -- (srch);
    \draw [gtu arrow] (ext) -- (sitelink);
    \draw [gtu arrow] (bid) -- (cpc);
\end{tikzpicture}
\captionof{figure}{PPC Setup Process}
\end{center}

\textbf{Step-by-Step Process:}
\begin{enumerate}
    \item \textbf{Campaign Selection}: Choose Search Campaign for high-intent keyword targeting
    \item \textbf{Ad Extensions}: Add Sitelinks, Callouts, Structured Snippets
    \item \textbf{Bidding Setup}: Select Manual CPC or Target CPA/Maximize Conversions
    \item \textbf{Optimization}: Monitor performance keywords and ad testing
\end{enumerate}

\textbf{Performance Optimization:}
\begin{itemize}
    \item \textbf{Keyword research}: Negative keywords, long-tail opportunities
    \item \textbf{Ad copy testing}: Multiple versions, performance comparison
    \item \textbf{Quality Score}: Relevance, click-through rate, landing page experience
\end{itemize}

\begin{mnemonicbox}Select, Extend, Bid, Optimize - PPC Success Route\end{mnemonicbox}
\end{solutionbox}

\questionmarks{Question 5(c) OR}{7}{Describe the key components of a successful Facebook Ads strategy.}
\begin{solutionbox}
A successful Facebook Ads strategy requires careful planning, execution, and optimization across multiple interconnected components.

\textbf{Strategic Framework:}

\begin{center}
\begin{tikzpicture}[node distance=1.5cm]
    \node (strat) [gtu block] {Facebook Ads Strategy};
    
    \node (target) [gtu block, below left=of strat, xshift=-1cm] {Audience Targeting};
    \node (create) [gtu block, below left=of strat, xshift=1cm] {Creative Development};
    \node (struct) [gtu block, below right=of strat, xshift=-1cm] {Campaign Structure};
    \node (opt) [gtu block, below right=of strat, xshift=1cm] {Optimization};

    \node (demos) [gtu block, below=of target, font=\footnotesize] {Demographics\\Interests\\Behaviors};
    \node (visuals) [gtu block, below=of create, font=\footnotesize] {Visual Design\\Ad Copy\\Video Content};
    \node (objs) [gtu block, below=of struct, font=\footnotesize] {Objectives\\Budgets\\Scheduling};
    \node (tests) [gtu block, below=of opt, font=\footnotesize] {A/B Testing\\Monitoring\\Bid Opt};

    \draw [gtu arrow] (strat) -- (target);
    \draw [gtu arrow] (strat) -- (create);
    \draw [gtu arrow] (strat) -- (struct);
    \draw [gtu arrow] (strat) -- (opt);

    \draw [gtu arrow] (target) -- (demos);
    \draw [gtu arrow] (create) -- (visuals);
    \draw [gtu arrow] (struct) -- (objs);
    \draw [gtu arrow] (opt) -- (tests);
\end{tikzpicture}
\captionof{figure}{Facebook Ads Strategy}
\end{center}

\textbf{Key Strategy Components:}

\begin{center}
\captionof{table}{Strategy Components}
\begin{tabulary}{\linewidth}{L L L}
\toprule
\textbf{Component} & \textbf{Elements} & \textbf{Success Factors} \\
\midrule
\textbf{Audience Targeting} & Demographics, interests, behaviors & Precise targeting, relevant reach \\
\textbf{Creative Excellence} & Visuals, copy, video content & Engagement, brand consistency \\
\textbf{Campaign Structure} & Objectives, budgets, scheduling & Clear goals, efficient spending \\
\textbf{Optimization} & Testing, monitoring, adjustments & Data-driven decisions \\
\bottomrule
\end{tabulary}
\end{center}

\begin{itemize}
    \item \textbf{Audience Targeting}: Core, Custom, and Lookalike audiences
    \item \textbf{Creative Development}: High-quality images, video, compelling ad copy
    \item \textbf{Measurement}: ROI, ROAS, and Customer Lifetime Value
\end{itemize}

\begin{mnemonicbox}Target Accurately, Create Compellingly, Structure Strategically, Optimize Continuously\end{mnemonicbox}
\end{solutionbox}

\end{document}
