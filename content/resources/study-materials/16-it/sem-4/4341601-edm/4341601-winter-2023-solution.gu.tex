\documentclass{article}
% Adjust the relative path to point to the latex-templates directory

% content/resources/templates/preamble.tex
\usepackage[margin=0.6in]{geometry}
\author{Milav Dabgar}
\usepackage{amsmath,amssymb,amsthm}
\usepackage{booktabs}
\usepackage{multirow}
\usepackage{xcolor}
\usepackage{tcolorbox}
\tcbuselibrary{breakable,skins}
\usepackage[colorlinks=true,linkcolor=blue]{hyperref}
\usepackage{titlesec}
\usepackage{enumitem}
\usepackage{tikz}
\usepackage{pgfplots}
\usepackage{circuitikz}
\usepackage[version=4]{mhchem}
\usepackage{longtable}
\usepackage{array}
\usepackage{float}
\usepackage{caption}
\usepackage{listings}

\lstset{
  basicstyle=\small\ttfamily,
  breaklines=true,
  breakatwhitespace=false,
  postbreak=\mbox{\textcolor{red}{$\hookrightarrow$}\space},
  float=false,
  numbers=left,
  numberstyle=\tiny\color{gray},
  numbersep=10pt,
  xleftmargin=2em,
  keywordstyle=\color{blue},
  commentstyle=\color{green!60!black},
  stringstyle=\color{purple},
  backgroundcolor=\color{gray!5},
  showstringspaces=false,
  tabsize=2,
  captionpos=b,
  keepspaces=true,
  columns=flexible
}

\pgfplotsset{compat=1.18}
\usetikzlibrary{shapes,arrows,positioning,calc,patterns,decorations.pathmorphing,decorations.markings,arrows.meta}

% Color scheme
\definecolor{headcolor}{RGB}{0,102,204}
\definecolor{keycolor}{RGB}{220,20,60}
\definecolor{solutioncolor}{RGB}{34,139,34}
\definecolor{mnemoniccolor}{RGB}{148,0,211}
\definecolor{codecolor}{RGB}{0,0,100}

% Spacing
\setlength{\parskip}{3pt}
\setlist[itemize]{nosep}
\setlist[enumerate]{nosep}

% Title formatting
\titleformat{\section}{\Large\bfseries\color{headcolor}}{\thesection}{1em}{}
\titleformat{\subsection}{\large\bfseries\color{headcolor}}{\thesubsection}{1em}{}

% Pandoc tightlist compatibility
\providecommand{\tightlist}{%
  \setlength{\itemsep}{0pt}\setlength{\parskip}{0pt}}

% Pandoc longtable compatibility
\newcounter{none}
\def\thenone{}


% content/resources/templates/gujarati-boxes.tex
\usepackage{fontspec}
\usepackage{polyglossia}

% Set Gujarati as main language (document is primarily in Gujarati)
% Note: gloss-gujarati.ldf doesn't exist in polyglossia, but it will use hyphenation patterns
\setdefaultlanguage{gujarati}
\setotherlanguage{english}

% Configure Gujarati font properly
% Use Language=Default to prevent polyglossia from trying to add language-specific features
% that don't exist for Gujarati, which causes "empty feature" warnings
\newfontfamily\gujaratifont[Script=Gujarati,AutoFakeBold=2.5,AutoFakeSlant=0.3]{Noto Sans Gujarati}
\setmainfont[Script=Gujarati,AutoFakeBold=2.5,AutoFakeSlant=0.3]{Noto Sans Gujarati}
% Use Noto Sans Gujarati for monospace to support Gujarati in text
\setmonofont[Scale=0.9]{Noto Sans Gujarati}

% Configure English to use the same font
\newfontfamily\englishfont[Script=Gujarati,AutoFakeBold=2.5,AutoFakeSlant=0.3]{Noto Sans Gujarati}

% Translations for polyglossia
\gappto\captionsgujarati{
  \renewcommand{\tablename}{કોષ્ટક}
  \renewcommand{\figurename}{આકૃતિ}
}

% Helper for TikZ nodes to ensure Gujarati font
\newcommand{\gu}[1]{{\gujaratifont #1}}

% Custom environments
\newtcolorbox{solutionbox}{
    breakable,
    enhanced,
    colback=solutioncolor!5!white,
    colframe=solutioncolor!75!black,
    fonttitle=\bfseries,
    title=જવાબ
}

\newtcolorbox{solutionboxnobreak}{
 colback=solutioncolor!5!white,
 colframe=solutioncolor!75!black,
 fonttitle=\bfseries,
 title=જવાબ
}

\newtcolorbox{keyformula}{
 breakable,
 enhanced,
 colback=keycolor!5!white,
 colframe=keycolor!75!black,
 fonttitle=\bfseries,
 title=રાસાયણિક સમીકરણ/સૂત્ર
}

\newtcolorbox{mnemonicbox}{
 breakable,
 enhanced,
 colback=mnemoniccolor!5!white,
 colframe=mnemoniccolor!75!black,
 fonttitle=\bfseries,
 title=મેમરી ટ્રીક
}


% Custom commands for GTU solutions
% This file defines semantic commands for consistent formatting

% Question command with automatic formatting
\newcommand{\question}[2]{%
  \section*{Question #1}%
  \textbf{#2}%
}

% OR question variant
\newcommand{\questionor}[2]{%
  \section*{Question #1 OR}%
  \textbf{#2}%
}

% Proper table environment with caption
\newenvironment{answertable}[1]{%
  \begin{table}[htbp]
  \centering
  \caption{#1}
}{%
  \end{table}
}

% Proper figure environment for diagrams
\newenvironment{answerdiagram}[1]{%
  \begin{figure}[htbp]
  \centering
  \caption{#1}
}{%
  \end{figure}
}

% Semantic markup for key terms
\newcommand{\keyword}[1]{\textbf{#1}}
\newcommand{\code}[1]{\texttt{#1}}
\newcommand{\classname}[1]{\texttt{#1}}
\newcommand{\methodname}[1]{\texttt{#1}}

% Proper quotation marks
\newcommand{\mnemonic}[1]{``#1''}


\title{એસેન્શિયલ્સ ઓફ ડિજિટલ માર્કેટિંગ (4341601) - શિયાળો 2023 સોલ્યુશન}
\date{જાન્યુઆરી 17, 2023}

\begin{document}
\maketitle

\questionmarks{1(અ)}{3}{ડિજિટલ માર્કેટિંગમાં SEO ની જરૂરિયાતનું વર્ણન કરો.}

\begin{solutionbox}
SEO એ ઓનલાઈન દૃશ્યતા અને બિઝનેસ વૃદ્ધિ માટે આવશ્યક છે.

\begin{center}
\captionof{table}{SEO ની જરૂરિયાત}
\begin{tabulary}{\linewidth}{|L|L|}
\hline
\textbf{જરૂરિયાત} & \textbf{વર્ણન} \\ \hline
\textbf{દૃશ્યતા} & વેબસાઈટને ટોપ સર્ચ રિઝલ્ટમાં દેખાવવામાં મદદ કરે \\ \hline
\textbf{ટ્રાફિક} & પેઈડ એડ્સ વિના ઓર્ગેનિક વિઝિટર્સ લાવે \\ \hline
\textbf{વિશ્વસનીયતા} & ઉચ્ચ રેન્કિંગ યુઝર્સ સાથે વિશ્વાસ બનાવે \\ \hline
\textbf{ખર્ચ-અસરકારક} & સતત એડ ખર્ચ વિના લાંબા સમયના પરિણામો \\ \hline
\end{tabulary}
\end{center}

\begin{itemize}
    \item \textbf{વધારાની દૃશ્યતા}: SEO વેબસાઈટને સર્ચ એન્જિનમાં ઉચ્ચ રેન્ક કરવામાં મદદ કરે
    \item \textbf{ઓર્ગેનિક ટ્રાફિક}: જાહેરાતના ખર્ચ વિના ગુણવત્તાવાળા વિઝિટર્સ લાવે
    \item \textbf{બ્રાન્ડ વિશ્વસનીયતા}: ટોપ રેન્કિંગ બિઝનેસ ઓથોરિટી સ્થાપિત કરે
\end{itemize}

\begin{mnemonicbox}VTC - Visibility, Traffic, Credibility\end{mnemonicbox}
\end{solutionbox}

\questionmarks{1(બ)}{4}{ટ્રેડિશનલ માર્કેટિંગ અને ડિજિટલ માર્કેટિંગ વચ્ચે તફાવત આપો.}

\begin{solutionbox}
ડિજિટલ માર્કેટિંગ પરંપરાગત પદ્ધતિઓની સરખામણીમાં લક્ષિત પહોંચ અને માપી શકાય તેવા પરિણામો પ્રદાન કરે છે.

\begin{center}
\captionof{table}{ટ્રેડિશનલ vs ડિજિટલ માર્કેટિંગ}
\begin{tabulary}{\linewidth}{|L|L|L|}
\hline
\textbf{પાસું} & \textbf{ટ્રેડિશનલ માર્કેટિંગ} & \textbf{ડિજિટલ માર્કેટિંગ} \\ \hline
\textbf{પહોંચ} & સ્થાનિક/પ્રાદેશિક & વૈશ્વિક \\ \hline
\textbf{ખર્ચ} & ઊંચો & નીચો \\ \hline
\textbf{ટાર્ગેટિંગ} & વ્યાપક પ્રેક્ષકો & ચોક્કસ ડેમોગ્રાફિક્સ \\ \hline
\textbf{માપ} & ટ્રેક કરવું મુશ્કેલ & રીઅલ-ટાઈમ એનાલિટિક્સ \\ \hline
\textbf{ઇન્ટરેક્શન} & એકતરફી વાતચીત & બે-તરફી જોડાણ \\ \hline
\end{tabulary}
\end{center}

\begin{itemize}
    \item \textbf{ખર્ચ કાર્યક્ષમતા}: ડિજિટલ માર્કેટિંગ ઓછા રોકાણની જરૂર છે
    \item \textbf{રીઅલ-ટાઈમ ટ્રેકિંગ}: તાત્કાલિક પ્રદર્શન માપ ઉપલબ્ધ
    \item \textbf{વૈશ્વિક પહોંચ}: વિશ્વવ્યાપી દર્શકોને તાત્કાલિક પ્રવેશ
\end{itemize}

\begin{mnemonicbox}GRIM - Global, Real-time, Interactive, Measurable\end{mnemonicbox}
\end{solutionbox}

\questionmarks{1(ક)}{7}{P.O.E.M. ફ્રેમવર્કના કોમ્પોનેન્ટસ અને તેમનું ડિજિટલ માર્કેટિંગમાં મહત્વ સમજાવો.}

\begin{solutionbox}
P.O.E.M. ફ્રેમવર્ક વ્યાપક ડિજિટલ માર્કેટિંગ વ્યૂહરચના માટે મીડિયા પ્રકારોનું વર્ગીકરણ કરે છે.

\begin{center}
\begin{tikzpicture}[node distance=1.5cm, auto]
    % Nodes
    \node [gtu block] (A) {P.O.E.M ફ્રેમવર્ક};
    
    \node [gtu block, below left=1.2cm and 2cm of A] (B) {Paid Media};
    \node [gtu block, below=1.2cm of A] (C) {Owned Media};
    \node [gtu block, below right=1.2cm and 2cm of A] (D) {Earned Media};
    
    \node [gtu state, below=0.8cm of B] (B1) {Google Ads\\Facebook Ads};
    \node [gtu state, below=0.8cm of C] (C1) {વેબસાઈટ\\ઈમેલ લિસ્ટ};
    \node [gtu state, below=0.8cm of D] (D1) {સોશિયલ શેર\\રિવ્યુઝ};

    % Arrows
    \path [gtu arrow] (A) -- (B);
    \path [gtu arrow] (A) -- (C);
    \path [gtu arrow] (A) -- (D);
    
    \path [gtu arrow] (B) -- (B1);
    \path [gtu arrow] (C) -- (C1);
    \path [gtu arrow] (D) -- (D1);
\end{tikzpicture}
\captionof{figure}{P.O.E.M. ફ્રેમવર્ક કોમ્પોનેન્ટસ}
\end{center}

\begin{center}
\captionof{table}{P.O.E.M. કોમ્પોનેન્ટસ}
\begin{tabulary}{\linewidth}{|L|L|L|L|}
\hline
\textbf{કોમ્પોનેન્ટ} & \textbf{વ્યાખ્યા} & \textbf{ઉદાહરણો} & \textbf{મહત્વ} \\ \hline
\textbf{Paid Media} & પેમેન્ટ દ્વારા પ્રમોશનલ કન્ટેન્ટ & Google Ads, Facebook Ads & તાત્કાલિક દૃશ્યતા અને ટ્રાફિક \\ \hline
\textbf{Owned Media} & બ્રાન્ડ દ્વારા કંટ્રોલ કરાતું કન્ટેન્ટ & વેબસાઈટ, ઈમેલ લિસ્ટ & લાંબા ગાળાના સંબંધો બનાવે \\ \hline
\textbf{Earned Media} & યુઝર્સ દ્વારા ઓર્ગેનિક ઉલ્લેખ & રિવ્યુઝ, સોશિયલ શેર & પ્રામાણિક વિશ્વસનીયતા \\ \hline
\end{tabulary}
\end{center}

\begin{itemize}
    \item \keyword{Paid Media}: તાત્કાલિક પહોંચ અને માપી શકાય તેવા ROI પ્રદાન કરે
    \item \keyword{Owned Media}: સીધા ગ્રાહક સંબંધો અને બ્રાન્ડ કંટ્રોલ બનાવે
    \item \keyword{Earned Media}: યુઝર-જનરેટેડ કન્ટેન્ટ દ્વારા પ્રામાણિક વિશ્વાસ બનાવે
\end{itemize}

\begin{mnemonicbox}POE - Pay for reach, Own relationships, Earn trust\end{mnemonicbox}
\end{solutionbox}

\questionmarks{1(ક OR)}{7}{ડિજિટલ માર્કેટિંગ પ્લાનના મુખ્ય કોમ્પોનેન્ટસ સમજાવો.}

\begin{solutionbox}
ડિજિટલ માર્કેટિંગ પ્લાન ઓનલાઈન બિઝનેસ સફળતા માટે સંરચિત અભિગમ પ્રદાન કરે છે.

\begin{center}
\begin{tikzpicture}[node distance=1.5cm, auto]
    % Central Node
    \node [gtu block, minimum width=3cm] (Center) {ડિજિટલ માર્કેટિંગ\\પ્લાન};
    
    % Surrounding Nodes
    \node [gtu block, above=1.5cm of Center] (Goals) {ગોલ્સ \& KPIs};
    \node [gtu block, above right=1cm and 1cm of Center] (Audience) {ટાર્ગેટ ઑડિયન્સ};
    \node [gtu block, right=1.8cm of Center] (Strategy) {વ્યૂહરચના};
    \node [gtu block, below right=1cm and 1cm of Center] (Tactics) {ટેકટિક્સ};
    \node [gtu block, below=1.5cm of Center] (Budget) {બજેટ};
    \node [gtu block, below left=1cm and 1cm of Center] (Timeline) {ટાઈમલાઈન};
    \node [gtu block, left=1.8cm of Center] (Research) {માર્કેટ રિસર્ચ};
    \node [gtu block, above left=1cm and 1cm of Center] (Measurement) {માપ};
    
    % Connections
    \foreach \n in {Goals, Audience, Strategy, Tactics, Budget, Timeline, Research, Measurement}
        \draw [gtu arrow, <->] (Center) -- (\n);
\end{tikzpicture}
\captionof{figure}{ડિજિટલ માર્કેટિંગ પ્લાનના મુખ્ય કોમ્પોનેન્ટસ}
\end{center}

\begin{center}
\captionof{table}{ડિજિટલ માર્કેટિંગ પ્લાન કોમ્પોનેન્ટસ}
\begin{tabulary}{\linewidth}{|L|L|L|}
\hline
\textbf{કોમ્પોનેન્ટ} & \textbf{વર્ણન} & \textbf{હેતુ} \\ \hline
\textbf{માર્કેટ રિસર્ચ} & ઇન્ડસ્ટ્રી અને કોમ્પિટિટર એનાલિસિસ & માર્કેટ લેન્ડસ્કેપ સમજવું \\ \hline
\textbf{ટાર્ગેટ ઑડિયન્સ} & ડેમોગ્રાફિક્સ અને સાઈકોગ્રાફિક્સ & ફોકસ્ડ મેસેજિંગ \\ \hline
\textbf{ગોલ્સ \& KPIs} & ચોક્કસ માપી શકાય તેવા ઉદ્દેશ્યો & પ્રદર્શન ટ્રેકિંગ \\ \hline
\textbf{વ્યૂહરચના \& ટેકટિક્સ} & ચેનલ્સ અને કન્ટેન્ટ અભિગમ & અમલીકરણ રોડમેપ \\ \hline
\textbf{બજેટ એલોકેશન} & રિસોર્સ વિતરણ & ખર્ચ મેનેજમેન્ટ \\ \hline
\textbf{ટાઈમલાઈન} & કેમ્પેઈન શેડ્યુલિંગ & પ્રોજેક્ટ મેનેજમેન્ટ \\ \hline
\textbf{માપ} & એનાલિટિક્સ અને રિપોર્ટિંગ & સતત સુધારણા \\ \hline
\end{tabulary}
\end{center}

\begin{itemize}
    \item \textbf{સ્પષ્ટ ઉદ્દેશ્યો}: SMART ગોલ્સ ફોકસ્ડ પ્રયાસો સુનિશ્ચિત કરે
    \item \textbf{ઑડિયન્સ ટાર્ગેટિંગ}: ચોક્કસ ડેમોગ્રાફિક્સ કન્વર્ઝન રેટ સુધારે
    \item \textbf{પ્રદર્શન ટ્રેકિંગ}: નિયમિત માપ ઓપ્ટિમાઈઝેશન સક્ષમ કરે
\end{itemize}

\begin{mnemonicbox}RATSBUM - Research, Audience, Tactics, Strategy, Budget, Measurement\end{mnemonicbox}
\end{solutionbox}

\questionmarks{2(અ)}{3}{બ્લેક હેટ અને વ્હાઈટ હેટ SEO તકનીકો વચ્ચે તફાવત આપો.}

\begin{solutionbox}
વ્હાઈટ હેટ SEO ગાઈડલાઈન્સ અનુસરે છે જ્યારે બ્લેક હેટ ઝડપી પરિણામો માટે પ્રતિબંધિત પદ્ધતિઓનો ઉપયોગ કરે છે.

\begin{center}
\captionof{table}{વ્હાઈટ હેટ vs બ્લેક હેટ SEO}
\begin{tabulary}{\linewidth}{|L|L|L|}
\hline
\textbf{પાસું} & \textbf{વ્હાઈટ હેટ SEO} & \textbf{બ્લેક હેટ SEO} \\ \hline
\textbf{પદ્ધતિઓ} & નૈતિક પ્રેક્ટિસ & મેનિપ્યુલેટિવ ટેકનીક \\ \hline
\textbf{પરિણામો} & ટકાઉ રેન્કિંગ & અસ્થાયી લાભ \\ \hline
\textbf{જોખમ} & પેનાલ્ટીથી સુરક્ષિત & ઉચ્ચ પેનાલ્ટી જોખમ \\ \hline
\textbf{ઉદાહરણો} & ગુણવત્તાવાળું કન્ટેન્ટ, પ્રાકૃતિક લિંક & કીવર્ડ સ્ટફિંગ, છુપાયેલું ટેક્સ્ટ \\ \hline
\end{tabulary}
\end{center}

\begin{itemize}
    \item \keyword{વ્હાઈટ હેટ}: યુઝર એક્સપિરિયન્સ અને ગુણવત્તાવાળા કન્ટેન્ટ પર ફોકસ કરે
    \item \keyword{બ્લેક હેટ}: સર્ચ એન્જિન એલ્ગોરિધમને છેતરવાનો પ્રયાસ કરે
    \item \keyword{લાંબા ગાળાની અસર}: વ્હાઈટ હેટ ચાલુ સફળતા બનાવે
\end{itemize}

\begin{mnemonicbox}WS-BT - White Sustainable, Black Temporary\end{mnemonicbox}
\end{solutionbox}

\questionmarks{2(બ)}{4}{SEO રેન્કિંગને અસર કરતા પરિબળોની ચર્ચા કરો.}

\begin{solutionbox}
ઘણા પરિબળો સર્ચ એન્જિનો વેબસાઈટને પરિણામોમાં કેવી રીતે રેન્ક કરે છે તેને પ્રભાવિત કરે છે.

\begin{center}
\captionof{table}{SEO રેન્કિંગ પરિબળો}
\begin{tabulary}{\linewidth}{|L|L|}
\hline
\textbf{પરિબળ કેટેગરી} & \textbf{ચોક્કસ પરિબળો} \\ \hline
\textbf{કન્ટેન્ટ ગુણવત્તા} & સુસંગતતા, મૌલિકતા, કીવર્ડ ઓપ્ટિમાઈઝેશન \\ \hline
\textbf{ટેકનિકલ SEO} & પેજ સ્પીડ, મોબાઈલ-ફ્રેન્ડલીનેસ, SSL \\ \hline
\textbf{યુઝર એક્સપિરિયન્સ} & બાઉન્સ રેટ, સાઈટ પરનો સમય, નેવિગેશન \\ \hline
\textbf{ઓથોરિટી} & બેકલિંક્સ, ડોમેઈન એજ, સોશિયલ સિગ્નલ્સ \\ \hline
\end{tabulary}
\end{center}

\begin{itemize}
    \item \textbf{કન્ટેન્ટ સુસંગતતા}: ઉચ્ચ-ગુણવત્તાવાળું, મૌલિક કન્ટેન્ટ બેહતર રેન્ક કરે
    \item \textbf{ટેકનિકલ ઓપ્ટિમાઈઝેશન}: ઝડપી લોડિંગ અને મોબાઈલ-ફ્રેન્ડલી સાઈટ પસંદ
    \item \textbf{યુઝર એન્ગેજમેન્ટ}: ઓછા બાઉન્સ રેટ મૂલ્યવાન કન્ટેન્ટ દર્શાવે
    \item \textbf{બાહ્ય ઓથોરિટી}: ગુણવત્તાવાળા બેકલિંક્સ વિશ્વસનીયતા વધારે
\end{itemize}

\begin{mnemonicbox}CTUA - Content, Technical, User experience, Authority\end{mnemonicbox}
\end{solutionbox}

\questionmarks{2(ક)}{7}{કેવી રીતે સોશિયલ મીડિયા SEO રેન્કિંગને સુધારી શકે છે? યોગ્ય ઉદાહરણ સાથે સમજાવો.}

\begin{solutionbox}
સોશિયલ મીડિયા વધેલી દૃશ્યતા અને એન્ગેજમેન્ટ સિગ્નલ્સ દ્વારા પરોક્ષ રીતે SEO વધારે છે.

\begin{center}
\begin{tikzpicture}[node distance=1.5cm, auto]
    \node [gtu block] (Activity) {સોશિયલ મીડિયા એક્ટિવિટી};
    
    \node [gtu block, above right=1cm and 2cm of Activity] (Visibility) {વધેલી બ્રાન્ડ\\દૃશ્યતા};
    \node [gtu block, below right=1cm and 2cm of Activity] (Traffic) {વધુ વેબસાઈટ\\ટ્રાફિક};
    \node [gtu block, right=4cm of Activity] (Signals) {સોશિયલ સિગ્નલ્સ};
    
    \node [gtu block, right=8cm of Activity] (Rankings) {બેહતર SEO\\રેન્કિંગ};
    
    \path [gtu arrow] (Activity) -- (Visibility);
    \path [gtu arrow] (Activity) -- (Traffic);
    \path [gtu arrow] (Activity) -- (Signals);
    
    \path [gtu arrow] (Visibility) -- (Rankings);
    \path [gtu arrow] (Traffic) -- (Rankings);
    \path [gtu arrow] (Signals) -- (Rankings);
\end{tikzpicture}
\captionof{figure}{SEO પર સોશિયલ મીડિયા અસર}
\end{center}

\begin{center}
\captionof{table}{સોશિયલ મીડિયા SEO અસર}
\begin{tabulary}{\linewidth}{|L|L|L|}
\hline
\textbf{સોશિયલ મીડિયા અસર} & \textbf{SEO લાભ} & \textbf{ઉદાહરણ} \\ \hline
\textbf{કન્ટેન્ટ શેરિંગ} & વધેલા બેકલિંક્સ & LinkedIn પર શેર થયેલ બ્લોગ પોસ્ટ ઇન્ડસ્ટ્રી સાઈટ્સ દ્વારા લિંક થાય \\ \hline
\textbf{બ્રાન્ડ મેન્શન્સ} & ઓથોરિટી બિલ્ડિંગ & Twitter મેન્શન્સ બ્રાન્ડ સર્ચ વધારે \\ \hline
\textbf{ટ્રાફિક જનરેશન} & યુઝર એન્ગેજમેન્ટ સિગ્નલ્સ & Facebook પોસ્ટ ટ્રાફિક લાવે, બાઉન્સ રેટ ઘટાડે \\ \hline
\textbf{લોકલ પ્રેઝન્સ} & લોકલ SEO બૂસ્ટ & Google My Business પોસ્ટ લોકલ રેન્કિંગ સુધારે \\ \hline
\end{tabulary}
\end{center}

\textbf{ઉદાહરણ}: એક રેસ્ટોરન્ટ Instagram પર લોકેશન ટેગ્સ સાથે ફૂડ ફોટોઝ શેર કરે છે. આનાથી વધે છે:
\begin{itemize}
    \item લોકલ બ્રાન્ડ સર્ચ
    \item સોશિયલ મીડિયાથી વેબસાઈટ વિઝિટ
    \item યુઝર-જનરેટેડ કન્ટેન્ટ અને રિવ્યુઝ
    \item એકંદર ઓનલાઈન પ્રેઝન્સ
\end{itemize}

\begin{itemize}
    \item \textbf{સોશિયલ સિગ્નલ્સ}: સર્ચ એન્જિનો સોશિયલ એન્ગેજમેન્ટને ગુણવત્તા સૂચક માને
    \item \textbf{ટ્રાફિક બૂસ્ટ}: સોશિયલ મીડિયા યોગ્ય વિઝિટર્સ વેબસાઈટ પર લાવે
    \item \textbf{કન્ટેન્ટ એમ્પ્લિફિકેશન}: સોશિયલ શેરિંગ કન્ટેન્ટની પહોંચ અને સંભવિત બેકલિંક્સ વધારે
\end{itemize}

\begin{mnemonicbox}STAB - Signals, Traffic, Amplification, Branding\end{mnemonicbox}
\end{solutionbox}

\questionmarks{2(અ OR)}{3}{ઑન-પેજ SEO અને ઑફ-પેજ SEO વચ્ચે તફાવત આપો.}

\begin{solutionbox}
ઑન-પેજ SEO વેબસાઈટ એલિમેન્ટ્સ ઓપ્ટિમાઈઝ કરે છે જ્યારે ઑફ-પેજ બાહ્ય ઓથોરિટી બનાવે છે.

\begin{center}
\captionof{table}{ઑન-પેજ vs ઑફ-પેજ SEO}
\begin{tabulary}{\linewidth}{|L|L|L|}
\hline
\textbf{પાસું} & \textbf{ઑન-પેજ SEO} & \textbf{ઑફ-પેજ SEO} \\ \hline
\textbf{સ્થાન} & વેબસાઈટની અંદર & બાહ્ય વેબસાઈટ્સ \\ \hline
\textbf{કંટ્રોલ} & સંપૂર્ણ કંટ્રોલ & મર્યાદિત કંટ્રોલ \\ \hline
\textbf{ફોકસ} & કન્ટેન્ટ અને ટેકનિકલ & ઓથોરિટી અને ટ્રસ્ટ \\ \hline
\textbf{ઉદાહરણો} & ટાઈટલ ટેગ્સ, મેટા ડિસ્ક્રિપ્શન & બેકલિંક્સ, સોશિયલ શેર \\ \hline
\end{tabulary}
\end{center}

\begin{itemize}
    \item \keyword{ઑન-પેજ}: કન્ટેન્ટ, HTML ટેગ્સ અને સાઈટ સ્ટ્રક્ચર ઓપ્ટિમાઈઝ કરે
    \item \keyword{ઑફ-પેજ}: બાહ્ય સિગ્નલ્સ અને લિંક્સ દ્વારા ઓથોરિટી બનાવે
    \item \keyword{સંયોજન}: વ્યાપક SEO સફળતા માટે બંનેની જરૂર
\end{itemize}

\begin{mnemonicbox}In-Out - Internal optimization, Outside authority\end{mnemonicbox}
\end{solutionbox}

\questionmarks{2(બ OR)}{4}{SEO રેન્કિંગને સુધારવાની વિવિધ રીતોની ચર્ચા કરો.}

\begin{solutionbox}
ઘણી વ્યૂહરચનાઓ સર્ચ પરિણામોમાં વેબસાઈટની દૃશ્યતા વધારી શકે છે.

\begin{center}
\captionof{table}{SEO સુધારણા વ્યૂહરચનાઓ}
\begin{tabulary}{\linewidth}{|L|L|}
\hline
\textbf{વ્યૂહરચના} & \textbf{અમલીકરણ} \\ \hline
\textbf{કન્ટેન્ટ ઓપ્ટિમાઈઝેશન} & કીવર્ડ રિસર્ચ, ગુણવત્તાવાળું લેખન, નિયમિત અપડેટ્સ \\ \hline
\textbf{ટેકનિકલ SEO} & પેજ સ્પીડ, મોબાઈલ ઓપ્ટિમાઈઝેશન, SSL સર્ટિફિકેટ \\ \hline
\textbf{લિંક બિલ્ડિંગ} & ગેસ્ટ પોસ્ટિંગ, ડિરેક્ટરી સબમિશન, પાર્ટનરશિપ \\ \hline
\textbf{યુઝર એક્સપિરિયન્સ} & સ્પષ્ટ નેવિગેશન, ઝડપી લોડિંગ, આકર્ષક ડિઝાઈન \\ \hline
\end{tabulary}
\end{center}

\begin{itemize}
    \item \textbf{ગુણવત્તાવાળું કન્ટેન્ટ}: ટાર્ગેટ કીવર્ડ્સ સાથે મૂલ્યવાન, મૌલિક કન્ટેન્ટ બનાવો
    \item \textbf{ટેકનિકલ એક્સેલન્સ}: ઝડપી, મોબાઈલ-ફ્રેન્ડલી, સિક્યોર વેબસાઈટ સુનિશ્ચિત કરો
    \item \textbf{ઓથોરિટી બિલ્ડિંગ}: સંબંધિત સાઈટ્સથી ઉચ્ચ-ગુણવત્તાવાળા બેકલિંક્સ મેળવો
    \item \textbf{યુઝર સંતોષ}: સરળ નેવિગેશન અને આકર્ષક અનુભવ પર ધ્યાન આપો
\end{itemize}

\begin{mnemonicbox}CTLU - Content, Technical, Links, User experience\end{mnemonicbox}
\end{solutionbox}

\questionmarks{2(ક OR)}{7}{નવી લૉન્ચ થયેલી વેબસાઈટ માટે તમે ઑફ પેજ ઓપ્ટિમાઈઝેશન કેવી રીતે કરશો?}

\begin{solutionbox}
નવી વેબસાઈટ્સ માટે ઑફ-પેજ ઓપ્ટિમાઈઝેશન ઓથોરિટી બિલ્ડ કરવા માટે વ્યૂહરચનાત્મક અભિગમની જરૂર છે.

\begin{center}
\begin{tikzpicture}[node distance=1.5cm, auto]
    \node [gtu block] (NewSite) {નવી વેબસાઈટ\\ઑફ-પેજ SEO};
    
    \node [gtu block, below left=1.5cm and 2cm of NewSite] (Directory) {ડિરેક્ટરી\\સબમિશન};
    \node [gtu block, below left=1.5cm and 0cm of NewSite] (Social) {સોશિયલ મીડિયા\\પ્રેઝન્સ};
    \node [gtu block, below=1.5cm of NewSite] (Content) {કન્ટેન્ટ\\માર્કેટિંગ};
    \node [gtu block, below right=1.5cm and 0cm of NewSite] (Local) {લોકલ SEO};
    \node [gtu block, below right=1.5cm and 2cm of NewSite] (Relationship) {રિલેશનશિપ\\બિલ્ડિંગ};
    
    \node [gtu block, below=4cm of NewSite] (Result) {સુધારેલી રેન્કિંગ};
    
    \foreach \n in {Directory, Social, Content, Local, Relationship}
        \path [gtu arrow] (NewSite) -- (\n);
        
    \foreach \n in {Directory, Social, Content, Local, Relationship}
        \path [gtu arrow] (\n) -- (Result);
\end{tikzpicture}
\captionof{figure}{નવી વેબસાઈટ માટે ઑફ-પેજ વ્યૂહરચના}
\end{center}

\begin{center}
\captionof{table}{ઑફ-પેજ એક્શન પ્લાન}
\begin{tabulary}{\linewidth}{|L|L|L|}
\hline
\textbf{વ્યૂહરચના} & \textbf{એક્શન સ્ટેપ્સ} & \textbf{ટાઈમલાઈન} \\ \hline
\textbf{ડિરેક્ટરી સબમિશન} & સંબંધિત બિઝનેસ ડિરેક્ટરીમાં સબમિટ કરો & અઠવાડિયું 1-2 \\ \hline
\textbf{સોશિયલ મીડિયા સેટઅપ} & મુખ્ય પ્લેટફોર્મ પર પ્રોફાઈલ બનાવો & અઠવાડિયું 1 \\ \hline
\textbf{કન્ટેન્ટ ક્રિએશન} & શેર કરી શકાય તેવું બ્લોગ કન્ટેન્ટ વિકસાવો & ચાલુ \\ \hline
\textbf{લોકલ SEO} & Google My Business, લોકલ સાઈટેશન & અઠવાડિયું 2-3 \\ \hline
\textbf{ગેસ્ટ પોસ્ટિંગ} & બેકલિંક્સ સાથે ઇન્ડસ્ટ્રી બ્લોગ માટે લખો & મહિનો 2-3 \\ \hline
\textbf{ઇન્ફ્લુએન્સર આઉટરીચ} & ઇન્ડસ્ટ્રી ઇન્ફ્લુએન્સર્સ સાથે જોડાણ & મહિનો 2-4 \\ \hline
\end{tabulary}
\end{center}

\textbf{અમલીકરણ સ્ટેપ્સ:}
\begin{enumerate}
    \item \textbf{કોમ્પિટિટર્સ રિસર્ચ}: તેમના બેકલિંક પ્રોફાઈલનું વિશ્લેષણ કરો
    \item \textbf{મૂલ્યવાન કન્ટેન્ટ બનાવો}: લિંક કરવા યોગ્ય રિસોર્સ વિકસાવો
    \item \textbf{સંબંધો બનાવો}: ઇન્ડસ્ટ્રી પ્રોફેશનલ્સ સાથે જોડાણ કરો
    \item \textbf{પ્રગતિ મોનિટર કરો}: બેકલિંક્સ અને રેન્કિંગ સુધારણા ટ્રેક કરો
\end{enumerate}

\begin{itemize}
    \item \textbf{ધીરજની જરૂર}: ઑફ-પેજ SEO પરિણામો બતાવવા માટે 3-6 મહિના લે છે
    \item \textbf{ગુણવત્તા ફોકસ}: ઘણા નીચી-ગુણવત્તાવાળા કરતાં થોડા ઉચ્ચ-ગુણવત્તાવાળા લિંક્સ બેહતર
    \item \textbf{સુસંગતતા}: નિયમિત આઉટરીચ અને કન્ટેન્ટ ક્રિએશન આવશ્યક
\end{itemize}

\begin{mnemonicbox}DSCLIG - Directories, Social, Content, Local, Influencers, Guest posting\end{mnemonicbox}
\end{solutionbox}

\questionmarks{3(અ)}{3}{નીચેના કી મેટ્રિક્સને વ્યાખ્યાયિત કરો: યુનિક વિઝિટર્સ, બાઉન્સ દર, પેજવ્યૂસ.}

\begin{solutionbox}
આ મેટ્રિક્સ વેબસાઈટ પ્રદર્શન અને યુઝર એન્ગેજમેન્ટ અસરકારક રીતે માપે છે.

\begin{center}
\captionof{table}{કી વેબ મેટ્રિક્સ}
\begin{tabulary}{\linewidth}{|L|L|L|}
\hline
\textbf{મેટ્રિક} & \textbf{વ્યાખ્યા} & \textbf{મહત્વ} \\ \hline
\textbf{યુનિક વિઝિટર્સ} & સમયગાળામાં સાઈટ વિઝિટ કરતા વ્યક્તિગત યુઝર્સ & ઑડિયન્સ પહોંચ માપે \\ \hline
\textbf{બાઉન્સ દર} & એક પેજ જોયા પછી છોડી જતા લોકોની ટકાવારી & કન્ટેન્ટ સુસંગતતા દર્શાવે \\ \hline
\textbf{પેજવ્યૂસ} & વિઝિટ દરમિયાન જોવાયેલા કુલ પેજીસ & કન્ટેન્ટ વપરાશ બતાવે \\ \hline
\end{tabulary}
\end{center}

\begin{itemize}
    \item \textbf{યુનિક વિઝિટર્સ}: ઘણી વિઝિટ્સ છતાં દરેક વ્યક્તિને એકવાર ગણે
    \item \textbf{બાઉન્સ દર}: ઊંચા દર ખરાબ કન્ટેન્ટ અથવા યુઝર એક્સપિરિયન્સ સૂચવે
    \item \textbf{પેજવ્યૂસ}: વધુ નંબર આકર્ષક, શોધી શકાય તેવું કન્ટેન્ટ દર્શાવે
\end{itemize}

\begin{mnemonicbox}UBP - Users, Bounces, Pages\end{mnemonicbox}
\end{solutionbox}

\questionmarks{3(બ)}{4}{વેબ એનાલિટિક્સમાં A/B પરીક્ષણ સમજાવો.}

\begin{solutionbox}
A/B ટેસ્ટિંગ કયું વર્ઝન બેહતર પ્રદર્શન કરે છે તે નક્કી કરવા માટે બે વર્ઝનની સરખામણી કરે છે.

\begin{center}
\begin{tikzpicture}[node distance=1.5cm, auto]
    % Versions
    \node [gtu block, minimum width=2.5cm] (VerA) {વર્ઝન A\\(લાલ બટન)};
    \node [gtu block, minimum width=2.5cm, right=3cm of VerA] (VerB) {વર્ઝન B\\(વાદળી બટન)};
    
    % Traffic
    \node [gtu state, below=1cm of VerA] (TrafA) {50\% ટ્રાફિક};
    \node [gtu state, below=1cm of VerB] (TrafB) {50\% ટ્રાફિક};
    
    % Results
    \node [gtu block, below=1cm of TrafA] (ResA) {પરિણામ A\\5\% ક્લિક};
    \node [gtu block, below=1cm of TrafB] (ResB) {પરિણામ B\\8\% ક્લિક};
    
    % Connections
    \draw [gtu arrow] (VerA) -- (TrafA);
    \draw [gtu arrow] (VerB) -- (TrafB);
    \draw [gtu arrow] (TrafA) -- (ResA);
    \draw [gtu arrow] (TrafB) -- (ResB);
    
    % Winner
    \node [draw, star, star points=5, star point height=0.5cm, below=0.5cm of ResB, fill=yellow!20, align=center] (Win) {વિજેતા!};
\end{tikzpicture}
\captionof{figure}{A/B ટેસ્ટિંગ પ્રક્રિયા}
\end{center}

\begin{center}
\captionof{table}{A/B ટેસ્ટિંગ કોમ્પોનેન્ટસ}
\begin{tabulary}{\linewidth}{|L|L|}
\hline
\textbf{કોમ્પોનેન્ટ} & \textbf{વર્ણન} \\ \hline
\textbf{હાઈપોથિસિસ} & શું પ્રદર્શન સુધારશે તેની આગાહી \\ \hline
\textbf{વેરિએબલ્સ} & ટેસ્ટ કરવામાં આવતા એલિમેન્ટ્સ (હેડલાઈન, બટન, રંગો) \\ \hline
\textbf{ટ્રાફિક સ્પ્લિટ} & વર્ઝન વચ્ચે વિઝિટર્સનું રેન્ડમ વિભાજન \\ \hline
\textbf{માપ} & કન્વર્ઝન રેટ અથવા અન્ય મેટ્રિક્સની તુલના | \\ \hline
\end{tabulary}
\end{center}

\begin{itemize}
    \item \textbf{સ્ટેટિસ્ટિકલ સિગ્નિફિકન્સ}: વિશ્વસનીય પરિણામો માટે પૂરતા ડેટાની ખાતરી કરો
    \item \textbf{સિંગલ વેરિએબલ}: સ્પષ્ટ આંતરદૃષ્ટિ માટે એક સમયે એક એલિમેન્ટ ટેસ્ટ કરો
    \item \textbf{સતત સુધારણા}: નિયમિત ટેસ્ટિંગ પ્રદર્શન ઓપ્ટિમાઈઝ કરે
\end{itemize}

\begin{mnemonicbox}HTVM - Hypothesis, Test, Variables, Measure\end{mnemonicbox}
\end{solutionbox}

\questionmarks{3(ક)}{7}{કેવી રીતે વ્યવસાયો Google Analytics માં લક્ષ્યો સેટ કરી શકે છે? યોગ્ય ઉદાહરણ સાથે સમજાવો.}

\begin{solutionbox}
Google Analytics ગોલ્સ મહત્વપૂર્ણ બિઝનેસ એક્શન્સ ટ્રેક કરે અને સફળતા માપે છે.

\begin{center}
\begin{tikzpicture}[node distance=1.5cm, auto]
    \node [gtu block] (Goals) {Google Analytics\\ગોલ્સ};
    
    \node [gtu block, below left=1.5cm and 3cm of Goals] (Dest) {ડેસ્ટિનેશન ગોલ્સ};
    \node [gtu block, below left=1.5cm and 0.5cm of Goals] (Dur) {ડ્યૂરેશન ગોલ્સ};
    \node [gtu block, below right=1.5cm and 0.5cm of Goals] (Pages) {પેજીસ/સેશન ગોલ્સ};
    \node [gtu block, below right=1.5cm and 3cm of Goals] (Event) {ઈવેન્ટ ગોલ્સ};
    
    \node [gtu state, below=0.8cm of Dest] (Ex1) {થેંક યુ પેજ};
    \node [gtu state, below=0.8cm of Dur] (Ex2) {સાઈટ પરનો સમય};
    \node [gtu state, below=0.8cm of Pages] (Ex3) {પેજ વ્યૂઝ};
    \node [gtu state, below=0.8cm of Event] (Ex4) {PDF ડાઉનલોડ};
    
    \path [gtu arrow] (Goals) -- (Dest);
    \path [gtu arrow] (Goals) -- (Dur);
    \path [gtu arrow] (Goals) -- (Pages);
    \path [gtu arrow] (Goals) -- (Event);
    
    \path [gtu arrow] (Dest) -- (Ex1);
    \path [gtu arrow] (Dur) -- (Ex2);
    \path [gtu arrow] (Pages) -- (Ex3);
    \path [gtu arrow] (Event) -- (Ex4);
\end{tikzpicture}
\captionof{figure}{Google Analytics માં ગોલ ટાઈપ્સ}
\end{center}

\begin{center}
\captionof{table}{ગોલ ટાઈપ્સ}
\begin{tabulary}{\linewidth}{|L|L|L|}
\hline
\textbf{ગોલ ટાઈપ} & \textbf{વર્ણન} & \textbf{બિઝનેસ ઉદાહરણ} \\ \hline
\textbf{ડેસ્ટિનેશન} & ચોક્કસ પેજ પર પહોંચવું & કોન્ટેક્ટ ફોર્મ સબમિશન \\ \hline
\textbf{ડ્યૂરેશન} & સાઈટ પર વિતાવેલો સમય & એન્ગેજમેન્ટ માપ \\ \hline
\textbf{પેજીસ/સેશન} & જોવાયેલા પેજીસની સંખ્યા & કન્ટેન્ટ વપરાશ \\ \hline
\textbf{ઈવેન્ટ} & ચોક્કસ ઇન્ટરેક્શન & ફાઈલ ડાઉનલોડ, વિડિયો પ્લે \\ \hline
\end{tabulary}
\end{center}

\textbf{સેટઅપ પ્રક્રિયા:}
\begin{enumerate}
    \item \textbf{એડમિન એક્સેસ}: એડમિન પેનલમાં ગોલ્સ સેક્શનમાં જાઓ
    \item \textbf{ટેમ્પલેટ પસંદ કરો}: સંબંધિત ગોલ ટેમ્પલેટ અથવા કસ્ટમ પસંદ કરો
    \item \textbf{વિગતો કોન્ફિગર કરો}: ડેસ્ટિનેશન URL અથવા ઈવેન્ટ પેરામીટર્સ સેટ કરો
    \item \textbf{ગોલ વેરિફાઈ કરો}: ગોલ ફ્લો રિપોર્ટ્સ સાથે ગોલ સેટઅપ ટેસ્ટ કરો
\end{enumerate}

\textbf{ઉદાહરણ - ઈ-કોમર્સ સ્ટોર:}
\begin{itemize}
    \item \textbf{ગોલ}: પર્ચેઝ કમ્પ્લીશન ટ્રેક કરવું
    \item \textbf{ટાઈપ}: ડેસ્ટિનેશન ગોલ
    \item \textbf{સેટઅપ}: "/order-confirmation" પેજની વિઝિટ ટ્રેક કરો
    \item \textbf{વેલ્યુ}: કન્વર્ઝનને મનેટરી વેલ્યુ આપો
    \item \textbf{ફનલ}: ચેકઆઉટ પ્રક્રિયાના સ્ટેપ્સ સેટ કરો
\end{itemize}

\begin{itemize}
    \item \textbf{કન્વર્ઝન ટ્રેકિંગ}: બિઝનેસ ઓબ્જેક્ટિવ એચીવમેન્ટ માપે
    \item \textbf{ROI કેલ્ક્યુલેશન}: વેબસાઈટ ઇન્ટરેક્શનને વેલ્યુ આપે
    \item \textbf{ઓપ્ટિમાઈઝેશન ઇનસાઈટ્સ}: સુધારણાની તકો ઓળખે
\end{itemize}

\begin{mnemonicbox}DDPE - Destination, Duration, Pages, Events\end{mnemonicbox}
\end{solutionbox}

\questionmarks{3(અ OR)}{3}{નીચેના કી મેટ્રિક્સને વ્યાખ્યાયિત કરો: નવી મુલાકાતો, પેજીસ/વિઝિટ, સરેરાશ મુલાકાત ડ્યૂરેશન.}

\begin{solutionbox}
આ મેટ્રિક્સ વિઝિટર વર્તન અને વેબસાઈટ એન્ગેજમેન્ટ પેટર્નનું વિશ્લેષણ કરે છે.

\begin{center}
\captionof{table}{એન્ગેજમેન્ટ મેટ્રિક્સ}
\begin{tabulary}{\linewidth}{|L|L|L|}
\hline
\textbf{મેટ્રિક} & \textbf{વ્યાખ્યા} & \textbf{મહત્વ} \\ \hline
\textbf{નવી મુલાકાતો} & પ્રથમ વખતના વિઝિટર્સની ટકાવારી & ઑડિયન્સ વૃદ્ધિ માપે \\ \hline
\textbf{પેજીસ/વિઝિટ} & પ્રતિ સેશન જોવાયેલા સરેરાશ પેજીસ & કન્ટેન્ટ એન્ગેજમેન્ટ લેવલ \\ \hline
\textbf{સરેરાશ વિઝિટ ડ્યૂરેશન} & પ્રતિ વિઝિટ વિતાવેલો સમય & યુઝર ઇન્ટરેસ્ટ ઇન્ડિકેટર \\ \hline
\end{tabulary}
\end{center}

\begin{itemize}
    \item \textbf{નવી મુલાકાતો}: ઊંચી ટકાવારી અસરકારક માર્કેટિંગ પહોંચ બતાવે
    \item \textbf{પેજીસ/વિઝિટ}: વધુ નંબર આકર્ષક કન્ટેન્ટ દર્શાવે
    \item \textbf{વિઝિટ ડ્યૂરેશન}: લાંબો સમય મૂલ્યવાન માહિતી સૂચવે
\end{itemize}

\begin{mnemonicbox}NPA - New visitors, Pages viewed, Average duration\end{mnemonicbox}
\end{solutionbox}

\questionmarks{3(બ OR)}{4}{વેબસાઈટ એનાલિટિક્સમાં ડેટા એકત્ર કરવાની વિવિધ પદ્ધતિઓ શું છે?}

\begin{solutionbox}
વિવિધ પદ્ધતિઓ વિશ્લેષણ અને ઓપ્ટિમાઈઝેશન માટે યુઝર બિહેવિયર ડેટા કેપ્ચર કરે છે.

\begin{center}
\captionof{table}{ડેટા એકત્ર કરવાની પદ્ધતિઓ}
\begin{tabulary}{\linewidth}{|L|L|L|}
\hline
\textbf{પદ્ધતિ} & \textbf{વર્ણન} & \textbf{એકત્ર કરવામાં આવતો ડેટા} \\ \hline
\textbf{પેજ ટેગિંગ} & પેજીસ પર JavaScript કોડ & યુઝર ઇન્ટરેક્શન, પેજ વ્યૂઝ \\ \hline
\textbf{વેબ લોગ એનાલિસિસ} & સર્વર લોગ ફાઈલ્સની તપાસ & ટેકનિકલ ડેટા, એરર્સ \\ \hline
\textbf{પેકેટ સ્નિફિંગ} & નેટવર્ક ટ્રાફિક મોનિટરિંગ & રીઅલ-ટાઈમ યુઝર બિહેવિયર \\ \hline
\textbf{હાઈબ્રિડ અભિગમ} & પદ્ધતિઓનું સંયોજન & વ્યાપક ડેટા સેટ \\ \hline
\end{tabulary}
\end{center}

\begin{itemize}
    \item \textbf{પેજ ટેગિંગ}: Google Analytics કોડ વાપરતી સૌથી સામાન્ય પદ્ધતિ
    \item \textbf{સર્વર લોગ્સ}: રિક્વેસ્ટ અને રિસ્પોન્સ વિશે ટેકનિકલ ડેટા
    \item \textbf{રીઅલ-ટાઈમ ટ્રેકિંગ}: તાત્કાલિક યુઝર બિહેવિયર ઇનસાઈટ્સ
    \item \textbf{ડેટા એક્યુરસી}: ઘણી પદ્ધતિઓ સંપૂર્ણ ચિત્ર પ્રદાન કરે
\end{itemize}

\begin{mnemonicbox}PWPH - Page tagging, Web logs, Packet sniffing, Hybrid\end{mnemonicbox}
\end{solutionbox}

\questionmarks{3(ક OR)}{7}{વિવિધ માર્કેટિંગ એટ્રિબ્યુશન મોડલ્સને ઉદાહરણ સાથે સમજાવો.}

\begin{solutionbox}
એટ્રિબ્યુશન મોડલ્સ કસ્ટમર જર્નીમાં માર્કેટિંગ ચેનલ્સને ક્રેડિટ આપે છે.

\begin{center}
\begin{tikzpicture}[node distance=1.5cm, auto]
    % Journey
    \node [gtu state] (FB) {Facebook એડ};
    \node [gtu state, right=0.8cm of FB] (Google) {Google સર્ચ};
    \node [gtu state, right=0.8cm of Google] (Email) {ઈમેલ};
    \node [gtu block, fill=green!10, right=0.8cm of Email] (Purchase) {પર્ચેઝ (\$100)};
    
    \draw [gtu arrow] (FB) -- (Google);
    \draw [gtu arrow] (Google) -- (Email);
    \draw [gtu arrow] (Email) -- (Purchase);
    
    % Models
    \node [gtu block, below=1.5cm of FB, text width=2.5cm] (First) {\textbf{ફર્સ્ટ-ક્લિક}\\100\% Facebook ને};
    \node [gtu block, below=1.5cm of Email, text width=2.5cm] (Last) {\textbf{લાસ્ટ-ક્લિક}\\100\% ઈમેલને};
    \node [gtu block, below=3.5cm of Google, text width=3.5cm] (Linear) {\textbf{લિનિયર}\\સમાન (33\% દરેક)};
\end{tikzpicture}
\captionof{figure}{એટ્રિબ્યુશન મોડલ્સ ઉદાહરણ}
\end{center}

\begin{center}
\captionof{table}{એટ્રિબ્યુશન મોડલ્સ સરખામણી}
\begin{tabulary}{\linewidth}{|L|L|L|L|}
\hline
\textbf{એટ્રિબ્યુશન મોડલ} & \textbf{ક્રેડિટ વિતરણ} & \textbf{શ્રેષ્ઠ ઉપયોગ} & \textbf{ઉદાહરણ} \\ \hline
\textbf{ફર્સ્ટ-ક્લિક} & પ્રથમ ઇન્ટરેક્શનને 100\% & બ્રાન્ડ અવેરનેસ કેમ્પેઈન & સોશિયલ મીડિયા એડને સંપૂર્ણ ક્રેડિટ \\ \hline
\textbf{લાસ્ટ-ક્લિક} & અંતિમ ઇન્ટરેક્શનને 100\% & ડાયરેક્ટ રિસ્પોન્સ કેમ્પેઈન & ઈમેલ કેમ્પેઈનને સંપૂર્ણ ક્રેડિટ \\ \hline
\textbf{લિનિયર} & બધા ટચપોઈન્ટ્સને સમાન ક્રેડિટ & મલ્ટી-ચેનલ કેમ્પેઈન & દરેક ચેનલને 25\% ક્રેડિટ \\ \hline
\textbf{ટાઈમ-ડિકે} & તાજેતરના ઇન્ટરેક્શનને વધુ ક્રેડિટ & સેલ્સ-ફોકસ્ડ કેમ્પેઈન & તાજેતરના ટચપોઈન્ટ્સને વધુ ક્રેડિટ \\ \hline
\textbf{પોઝિશન-બેસ્ડ} & પ્રથમ અને અંતિમને વધુ ક્રેડિટ & અવેરનેસ + કન્વર્ઝન ફોકસ & 40\% પ્રથમ, 40\% અંતિમ, 20\% મધ્ય \\ \hline
\end{tabulary}
\end{center}

\textbf{ઉદાહરણ પરિસ્થિતિ:} કસ્ટમર જર્ની: Facebook એડ $\to$ Google સર્ચ $\to$ ઈમેલ $\to$ પર્ચેઝ (\$100)
\begin{itemize}
    \item \textbf{ફર્સ્ટ-ક્લિક}: Facebook એડ = \$100 ક્રેડિટ
    \item \textbf{લાસ્ટ-ક્લિક}: ઈમેલ = \$100 ક્રેડિટ
    \item \textbf{લિનિયર}: Facebook \$33, Google \$33, ઈમેલ \$33 ક્રેડિટ
    \item \textbf{ટાઈમ-ડિકે}: ઈમેલ \$60, Google \$30, Facebook \$10 ક્રેડિટ
\end{itemize}

\begin{itemize}
    \item \textbf{બિઝનેસ એલાઈનમેન્ટ}: માર્કેટિંગ ઉદ્દેશ્યો સાથે મેળ ખાતું મોડલ પસંદ કરો
    \item \textbf{ડેટા ઇનસાઈટ્સ}: વિવિધ મોડલ્સ વિવિધ ચેનલ યોગદાન દર્શાવે
    \item \textbf{ઓપ્ટિમાઈઝેશન}: અસરકારક ચેનલ્સમાં બજેટ એલોકેટ કરવામાં મદદ કરે
\end{itemize}

\begin{mnemonicbox}FLLTP - First, Last, Linear, Time-decay, Position-based\end{mnemonicbox}
\end{solutionbox}

\questionmarks{4(અ)}{3}{વિવિધ પ્રકારની YouTube જાહેરાતો સમજાવો.}

\begin{solutionbox}
YouTube ઑડિયન્સને અસરકારક રીતે પહોંચવા માટે વિવિધ એડ ફોર્મેટ્સ ઓફર કરે છે.

\begin{center}
\captionof{table}{YouTube એડ ટાઈપ્સ}
\begin{tabulary}{\linewidth}{|L|L|L|L|}
\hline
\textbf{એડ ટાઈપ} & \textbf{ફોર્મેટ} & \textbf{અવધિ} & \textbf{પ્લેસમેન્ટ} \\ \hline
\textbf{સ્કિપેબલ} & સ્કિપ ઓપ્શન સાથે વિડિયો એડ્સ & કોઈપણ લંબાઈ & વિડિયો પહેલાં/દરમિયાન \\ \hline
\textbf{નોન-સ્કિપેબલ} & ફરજિયાત જોવાશે & 15-20 સેકન્ડ & વિડિયો પહેલાં/દરમિયાન \\ \hline
\textbf{બમ્પર એડ્સ} & ટૂંકા, નોન-સ્કિપેબલ & 6 સેકન્ડ & વિડિયો પહેલાં \\ \hline
\textbf{ડિસ્કવરી એડ્સ} & ટેક્સ્ટ સાથે થમ્બનેઈલ & ચલ & સર્ચ પરિણામો, સાઈડબાર \\ \hline
\end{tabulary}
\end{center}

\begin{itemize}
    \item \textbf{સ્કિપેબલ એડ્સ}: એન્ગેજમેન્ટ-ફોકસ્ડ કેમ્પેઈન માટે કોસ્ટ-એફેક્ટિવ
    \item \textbf{નોન-સ્કિપેબલ}: બ્રાન્ડ અવેરનેસ માટે ગેરંટીડ એક્સપોઝર
    \item \textbf{બમ્પર એડ્સ}: ઉચ્ચ પહોંચ સાથે ઝડપી બ્રાન્ડ મેસેજ
\end{itemize}

\begin{mnemonicbox}SNBD - Skippable, Non-skippable, Bumper, Discovery\end{mnemonicbox}
\end{solutionbox}

\questionmarks{4(બ)}{4}{ટ્વિટર માર્કેટિંગમાં હેશટેગનો ઉપયોગ કેવી રીતે કરી શકાય?}

\begin{solutionbox}
હેશટેગ Twitter પ્લેટફોર્મ પર કન્ટેન્ટની શોધ અને એન્ગેજમેન્ટ વધારે છે.

\begin{center}
\captionof{table}{Twitter હેશટેગ વ્યૂહરચના}
\begin{tabulary}{\linewidth}{|L|L|L|}
\hline
\textbf{ઉપયોગ કેસ} & \textbf{વ્યૂહરચના} & \textbf{ઉદાહરણ} \\ \hline
\textbf{ટ્રેન્ડિંગ ટોપિક્સ} & સંબંધિત વાતચીતમાં જોડાવું & વેચાણ માટે \#BlackFriday \\ \hline
\textbf{બ્રાન્ડ હેશટેગ} & યુનિક બ્રાન્ડ આઈડેન્ટિફાયર બનાવવા & Nike માટે \#JustDoIt \\ \hline
\textbf{ઈવેન્ટ માર્કેટિંગ} & ઈવેન્ટ અને મેળાવડાને પ્રમોટ કરવા & \#TechConf2023 \\ \hline
\textbf{કેટેગરાઈઝેશન} & કન્ટેન્ટ થીમ્સનું આયોજન & \#MondayMotivation \\ \hline
\end{tabulary}
\end{center}

\begin{itemize}
    \item \textbf{ટ્રેન્ડ રિસર્ચ}: વ્યાપક પહોંચ માટે ટ્રેન્ડિંગ હેશટેગનો ઉપયોગ કરો
    \item \textbf{બ્રાન્ડેડ ક્રિએટ કરો}: કેમ્પેઈન માટે યુનિક હેશટેગ વિકસાવો
    \item \textbf{પ્રદર્શન મોનિટર કરો}: હેશટેગ એન્ગેજમેન્ટ અને પહોંચ ટ્રેક કરો
    \item \textbf{ક્વોન્ટિટી મર્યાદિત કરો}: શ્રેષ્ઠ પરિણામો માટે પ્રતિ ટ્વીટ 1-2 હેશટેગ વાપરો
\end{itemize}

\begin{mnemonicbox}TBEC - Trending, Branded, Events, Categorization\end{mnemonicbox}
\end{solutionbox}

\questionmarks{4(ક)}{7}{સોશિયલ મીડિયા માર્કેટિંગ અને તેનું વર્તમાન ડિજિટલ લેન્ડસ્કેપમાં મહત્વ સમજાવો.}

\begin{solutionbox}
સોશિયલ મીડિયા માર્કેટિંગ સંબંધો બનાવવા અને બિઝનેસ પરિણામો લાવવા માટે પ્લેટફોર્મ્સનો લાભ ઉઠાવે છે.

\begin{center}
\begin{tikzpicture}[node distance=1.5cm, auto]
    \node [gtu block, minimum width=3cm] (SMM) {સોશિયલ મીડિયા\\માર્કેટિંગ};
    
    \node [gtu block, above left=1.5cm and 1cm of SMM] (Awareness) {બ્રાન્ડ\\અવેરનેસ};
    \node [gtu block, above right=1.5cm and 1cm of SMM] (Engagement) {કસ્ટમર\\એન્ગેજમેન્ટ};
    \node [gtu block, below left=1.5cm and 1cm of SMM] (Lead) {લીડ\\જનરેશન};
    \node [gtu block, below right=1.5cm and 1cm of SMM] (Service) {કસ્ટમર\\સર્વિસ};
    
    \node [gtu block, right=4cm of SMM] (Growth) {બિઝનેસ ગ્રોથ};
    
    \path [gtu arrow] (SMM) -- (Awareness);
    \path [gtu arrow] (SMM) -- (Engagement);
    \path [gtu arrow] (SMM) -- (Lead);
    \path [gtu arrow] (SMM) -- (Service);
    
    \path [gtu arrow] (Awareness) -- (Growth);
    \path [gtu arrow] (Engagement) -- (Growth);
    \path [gtu arrow] (Lead) -- (Growth);
    \path [gtu arrow] (Service) -- (Growth);
\end{tikzpicture}
\captionof{figure}{સોશિયલ મીડિયા માર્કેટિંગ અસર}
\end{center}

\begin{center}
\captionof{table}{સોશિયલ મીડિયા પ્લેટફોર્મ્સ}
\begin{tabulary}{\linewidth}{|L|L|L|L|}
\hline
\textbf{પ્લેટફોર્મ} & \textbf{મુખ્ય ઉપયોગ} & \textbf{ઑડિયન્સ} & \textbf{કન્ટેન્ટ ટાઈપ} \\ \hline
\textbf{Facebook} & કમ્યુનિટી બિલ્ડિંગ & વ્યાપક ડેમોગ્રાફિક્સ & પોસ્ટ્સ, વિડિયો, ઈવેન્ટ્સ \\ \hline
\textbf{Instagram} & વિઝ્યુઅલ સ્ટોરીટેલિંગ & યુવા ઑડિયન્સ & ફોટોઝ, સ્ટોરીઝ, રીલ્સ \\ \hline
\textbf{LinkedIn} & પ્રોફેશનલ નેટવર્કિંગ & બિઝનેસ પ્રોફેશનલ્સ & આર્ટિકલ્સ, કંપની અપડેટ્સ \\ \hline
\textbf{Twitter} & રીઅલ-ટાઈમ એન્ગેજમેન્ટ & ન્યૂઝ, ટ્રેન્ડ્સ ફોલોવર્સ & ટૂંકા મેસેજ, થ્રેડ્સ \\ \hline
\textbf{YouTube} & વિડિયો માર્કેટિંગ & વિડિયો કન્ઝ્યુમર્સ & એજ્યુકેશનલ, એન્ટરટેઈનમેન્ટ \\ \hline
\end{tabulary}
\end{center}

\textbf{ડિજિટલ લેન્ડસ્કેપમાં મહત્વ:}
\begin{itemize}
    \item \textbf{ડાયરેક્ટ કમ્યુનિકેશન}: કસ્ટમર્સ સાથે રીઅલ-ટાઈમ ઇન્ટરેક્શન
    \item \textbf{કોસ્ટ-એફેક્ટિવ પહોંચ}: પરંપરાગત જાહેરાતની સરખામણીમાં ઓછા ખર્ચે
    \item \textbf{ટાર્ગેટેડ એડવર્ટાઈઝિંગ}: ચોક્કસ ડેમોગ્રાફિક અને રુચિ ટાર્ગેટિંગ
    \item \textbf{વાયરલ પોટેન્શિયલ}: કન્ટેન્ટ ઓર્ગેનિક રીતે મોટા ઑડિયન્સ સુધી પહોંચી શકે
    \item \textbf{કસ્ટમર ઇનસાઈટ્સ}: પસંદગીઓ અને વર્તન વિશે મૂલ્યવાન ડેટા
\end{itemize}

\textbf{વર્તમાન ટ્રેન્ડ્સ:}
\begin{itemize}
    \item \textbf{વિડિયો કન્ટેન્ટ ડોમિનન્સ}: ટૂંકા-ફોર્મ વિડિયો એન્ગેજમેન્ટ લાવે
    \item \textbf{સોશિયલ કોમર્સ}: પ્લેટફોર્મ દ્વારા ડાયરેક્ટ ખરીદી
    \item \textbf{ઇન્ફ્લુએન્સર પાર્ટનરશિપ}: ક્રિએટર્સ દ્વારા પ્રામાણિક એન્ડોર્સમેન્ટ
\end{itemize}

\begin{mnemonicbox}CLEAR - Communication, Low-cost, Engagement, Analytics, Reach\end{mnemonicbox}
\end{solutionbox}

\questionmarks{4(અ OR)}{3}{LinkedIn જાહેરાતોના વિવિધ પ્રકારો સમજાવો.}

\begin{solutionbox}
LinkedIn B2B માર્કેટિંગ માટે પ્રોફેશનલ-ફોકસ્ડ એડવર્ટાઈઝિંગ વિકલ્પો પ્રદાન કરે છે.

\begin{center}
\captionof{table}{LinkedIn એડ ટાઈપ્સ}
\begin{tabulary}{\linewidth}{|L|L|L|}
\hline
\textbf{એડ ટાઈપ} & \textbf{ફોર્મેટ} & \textbf{શ્રેષ્ઠ ઉપયોગ} \\ \hline
\textbf{સ્પોન્સર્ડ કન્ટેન્ટ} & ફીડમાં નેટિવ પોસ્ટ્સ & બ્રાન્ડ અવેરનેસ, એન્ગેજમેન્ટ \\ \hline
\textbf{મેસેજ એડ્સ} & યુઝર્સને ડાયરેક્ટ મેસેજ & લીડ જનરેશન, પર્સનલાઈઝ્ડ આઉટરીચ \\ \hline
\textbf{ડાયનેમિક એડ્સ} & પર્સનલાઈઝ્ડ બેનર એડ્સ & વેબસાઈટ ટ્રાફિક, ફોલોવર ગ્રોથ \\ \hline
\textbf{ટેક્સ્ટ એડ્સ} & ઈમેજ સાથે સાદું ટેક્સ્ટ & કોસ્ટ-એફેક્ટિવ અવેરનેસ કેમ્પેઈન \\ \hline
\end{tabulary}
\end{center}

\begin{itemize}
    \item \textbf{પ્રોફેશનલ ટાર્ગેટિંગ}: જોબ ટાઈટલ, કંપની, ઇન્ડસ્ટ્રી દ્વારા યુઝર્સ સુધી પહોંચો
    \item \textbf{B2B ફોકસ}: બિઝનેસ-ટુ-બિઝનેસ માર્કેટિંગ કેમ્પેઈન માટે આદર્શ
    \item \textbf{હાઈ-ક્વોલિટી ઑડિયન્સ}: પ્રોફેશનલ માનસિકતા બેહતર એન્ગેજમેન્ટ લાવે
\end{itemize}

\begin{mnemonicbox}SMDT - Sponsored, Message, Dynamic, Text\end{mnemonicbox}
\end{solutionbox}

\questionmarks{4(બ OR)}{4}{Instagram પર પ્રભાવક માર્કેટિંગનો કોન્સેપ્ટ સમજાવો.}

\begin{solutionbox}
ઇન્ફ્લુએન્સર માર્કેટિંગ પ્રોડક્ટ્સને પ્રામાણિક રીતે પ્રમોટ કરવા માટે લોકપ્રિય યુઝર્સનો લાભ ઉઠાવે છે.

\begin{center}
\captionof{table}{ઇન્ફ્લુએન્સર ટાઈપ્સ}
\begin{tabulary}{\linewidth}{|L|L|L|L|}
\hline
\textbf{પ્રકાર} & \textbf{ફોલોવર્સ} & \textbf{શ્રેષ્ઠ ઉપયોગ} & \textbf{ખર્ચ} \\ \hline
\textbf{નેનો} & 1K-10K & લોકલ, નિશ પ્રોડક્ટ્સ & ઓછો \\ \hline
\textbf{માઈક્રો} & 10K-100K & ટાર્ગેટેડ, ઉચ્ચ એન્ગેજમેન્ટ & મધ્યમ \\ \hline
\textbf{મેક્રો} & 100K-1M & બ્રાન્ડ અવેરનેસ, પહોંચ & ઊંચો \\ \hline
\textbf{મેગા} & 1M+ & માસ માર્કેટ, સેલિબ્રિટીઝ & ખૂબ ઊંચો \\ \hline
\end{tabulary}
\end{center}

\begin{itemize}
    \item \textbf{પ્રામાણિક કન્ટેન્ટ}: ઇન્ફ્લુએન્સર્સ અસલી પ્રોડક્ટ ભલામણો બનાવે
    \item \textbf{ઉચ્ચ એન્ગેજમેન્ટ}: ફોલોવર્સ ઇન્ફ્લુએન્સર મંતવ્યો અને સૂચનો પર વિશ્વાસ કરે
    \item \textbf{ટાર્ગેટેડ પહોંચ}: ટાર્ગેટ ઑડિયન્સ ડેમોગ્રાફિક્સ મેળ ખાતા ઇન્ફ્લુએન્સર્સ પસંદ કરો
    \item \textbf{માપી શકાય તેવા પરિણામો}: એન્ગેજમેન્ટ, ક્લિક્સ અને કન્વર્ઝન સરળતાથી ટ્રેક કરો
\end{itemize}

\begin{mnemonicbox}NMAM - Nano, Micro, Macro, Mega influencers\end{mnemonicbox}
\end{solutionbox}

\questionmarks{4(ક OR)}{7}{Facebook જાહેરાતમાં ઉપલબ્ધ લક્ષ્યીકરણ વિકલ્પોનું વર્ણન કરો.}

\begin{solutionbox}
Facebook ચોક્કસ ઑડિયન્સ પહોંચ માટે વ્યાપક ટાર્ગેટિંગ ક્ષમતાઓ પ્રદાન કરે છે.

\begin{center}
\begin{tikzpicture}[node distance=1.5cm, auto]
    \node [gtu block] (Targeting) {Facebook ટાર્ગેટિંગ};
    
    \node [gtu block, below left=1.5cm and 3cm of Targeting] (Demo) {ડેમોગ્રાફિક્સ};
    \node [gtu block, below left=1.5cm and 0.5cm of Targeting] (Interest) {રુચિઓ};
    \node [gtu block, below right=1.5cm and 0.5cm of Targeting] (Behavior) {વર્તન};
    \node [gtu block, below right=1.5cm and 3cm of Targeting] (Custom) {કસ્ટમ\\ઑડિયન્સ};
    
    \node [gtu state, below=0.8cm of Demo] (D1) {ઉંમર, લિંગ,\\સ્થાન};
    \node [gtu state, below=0.8cm of Interest] (I1) {શોખ, પ્રવૃત્તિઓ};
    \node [gtu state, below=0.8cm of Behavior] (B1) {ખરીદી હિસ્ટ્રી,\\ડિવાઈસ};
    \node [gtu state, below=0.8cm of Custom] (C1) {ઈમેલ લિસ્ટ,\\લુકલાઈક};
    
    \path [gtu arrow] (Targeting) -- (Demo);
    \path [gtu arrow] (Targeting) -- (Interest);
    \path [gtu arrow] (Targeting) -- (Behavior);
    \path [gtu arrow] (Targeting) -- (Custom);
    
    \path [gtu arrow] (Demo) -- (D1);
    \path [gtu arrow] (Interest) -- (I1);
    \path [gtu arrow] (Behavior) -- (B1);
    \path [gtu arrow] (Custom) -- (C1);
\end{tikzpicture}
\captionof{figure}{Facebook ટાર્ગેટિંગ વિકલ્પો}
\end{center}

\begin{center}
\captionof{table}{ટાર્ગેટિંગ કેટેગરી}
\begin{tabulary}{\linewidth}{|L|L|L|}
\hline
\textbf{કેટેગરી} & \textbf{વિકલ્પો} & \textbf{ઉપયોગ કેસ} \\ \hline
\textbf{ડેમોગ્રાફિક્સ} & ઉંમર, લિંગ, સ્થાન, શિક્ષણ & મૂળભૂત ઑડિયન્સ વ્યાખ્યા \\ \hline
\textbf{રુચિઓ} & લાઈક કરેલા પેજીસ, પ્રવૃત્તિઓ & લાઈફસ્ટાઈલ-બેસ્ડ ટાર્ગેટિંગ \\ \hline
\textbf{વર્તન} & ખરીદી હિસ્ટ્રી, ડિવાઈસ ઉપયોગ & એક્શન-બેસ્ડ ટાર્ગેટિંગ \\ \hline
\textbf{કસ્ટમ} & અપલોડ કરેલી લિસ્ટ, વિઝિટર્સ & રીટાર્ગેટિંગ કેમ્પેઈન \\ \hline
\textbf{લુકલાઈક} & હાલના કસ્ટમર્સ જેવા & ઑડિયન્સ એક્સપેન્શન \\ \hline
\end{tabulary}
\end{center}

\textbf{કેમ્પેઈન વ્યૂહરચના:}
\begin{enumerate}
    \item \textbf{બ્રોડ શરૂઆત}: મૂળભૂત ડેમોગ્રાફિક્સ અને રુચિઓથી શરૂ કરો
    \item \textbf{પ્રદર્શન વિશ્લેષણ}: શ્રેષ્ઠ પ્રદર્શન કરતા સેગમેન્ટ્સ ઓળખવા એનાલિટિક્સ વાપરો
    \item \textbf{ટાર્ગેટિંગ રિફાઈન કરો}: સફળ ઑડિયન્સ આધારે ફોકસ સાંકડો કરો
    \item \textbf{લુકલાઈક્સ બનાવો}: સમાન ઑડિયન્સ લાક્ષણિકતાઓ સાથે પહોંચ વિસ્તૃત કરો
    \item \textbf{વિઝિટર્સ રીટાર્ગેટ કરો}: કસ્ટમ ઑડિયન્સ સાથે વેબસાઈટ વિઝિટર્સ સાથે ફરીથી જોડાણ કરો
\end{enumerate}

\begin{itemize}
    \item \textbf{પ્રિસિઝન માર્કેટિંગ}: પ્રોડક્ટ્સ માટે બરાબર યોગ્ય લોકો સુધી પહોંચો
    \item \textbf{કોસ્ટ એફિશિયન્સી}: ટાર્ગેટેડ એડ્સ બગાડાયેલા એડવર્ટાઈઝિંગ ખર્ચ ઘટાડે
    \item \textbf{પ્રદર્શન ઓપ્ટિમાઈઝેશન}: સતત રિફાઈનમેન્ટ પરિણામો સુધારે
\end{itemize}

\begin{mnemonicbox}DIBCCL - Demographics, Interests, Behaviors, Custom, Connections, Lookalike\end{mnemonicbox}
\end{solutionbox}

\questionmarks{5(અ)}{3}{YouTube માર્કેટિંગ ઝુંબેશની સફળતાને માપવા માટે વપરાતા મેટ્રિક્સની સૂચિ બનાવો.}

\begin{solutionbox}
YouTube કેમ્પેઈન પ્રદર્શનનું અસરકારક મૂલ્યાંકન કરવા માટે વ્યાપક મેટ્રિક્સ પ્રદાન કરે છે.

\begin{center}
\captionof{table}{YouTube માપન મેટ્રિક્સ}
\begin{tabulary}{\linewidth}{|L|L|}
\hline
\textbf{મેટ્રિક કેટેગરી} & \textbf{ચોક્કસ મેટ્રિક્સ} \\ \hline
\textbf{પહોંચ મેટ્રિક્સ} & વ્યૂઝ, ઇમ્પ્રેશન્સ, યુનિક વ્યૂઅર્સ \\ \hline
\textbf{એન્ગેજમેન્ટ મેટ્રિક્સ} & લાઈક્સ, કોમેન્ટ્સ, શેર્સ, સબસ્ક્રાઈબર્સ \\ \hline
\textbf{પ્રદર્શન મેટ્રિક્સ} & ક્લિક-થ્રુ રેટ, કન્વર્ઝન રેટ \\ \hline
\textbf{રિટેન્શન મેટ્રિક્સ} & વોચ ટાઈમ, સરેરાશ વ્યૂ ડ્યૂરેશન \\ \hline
\end{tabulary}
\end{center}

\begin{itemize}
    \item \textbf{વ્યૂઝ અને ઇમ્પ્રેશન્સ}: કન્ટેન્ટ પહોંચ અને દૃશ્યતા માપે
    \item \textbf{એન્ગેજમેન્ટ સિગ્નલ્સ}: ઑડિયન્સ રુચિ અને કન્ટેન્ટ ગુણવત્તા દર્શાવે
    \item \textbf{કન્વર્ઝન ટ્રેકિંગ}: વિડિયો પ્રદર્શનને બિઝનેસ ગોલ્સ સાથે જોડે
\end{itemize}

\begin{mnemonicbox}REPR - Reach, Engagement, Performance, Retention\end{mnemonicbox}
\end{solutionbox}

\questionmarks{5(બ)}{4}{PPC અને SEO વચ્ચે તફાવત આપો.}

\begin{solutionbox}
PPC અને SEO વિવિધ અભિગમ અને ટાઈમલાઈન સાથે પૂરક વ્યૂહરચનાઓ છે.

\begin{center}
\captionof{table}{PPC vs SEO}
\begin{tabulary}{\linewidth}{|L|L|L|}
\hline
\textbf{પાસું} & \textbf{PPC} & \textbf{SEO} \\ \hline
\textbf{ખર્ચ} & પ્રતિ ક્લિક તાત્કાલિક પેમેન્ટ & કન્ટેન્ટમાં લાંબા ગાળાનું રોકાણ \\ \hline
\textbf{પરિણામો} & તાત્કાલિક દૃશ્યતા & ક્રમશ: રેન્કિંગ સુધારણા \\ \hline
\textbf{કંટ્રોલ} & પોઝિશનિંગ પર સંપૂર્ણ કંટ્રોલ & રેન્કિંગ પર મર્યાદિત કંટ્રોલ \\ \hline
\textbf{ટકાઉપણું} & બજેટ સમાપ્ત થવા પર બંધ & કામ બંધ કર્યા પછી પણ ચાલુ \\ \hline
\textbf{ટાર્ગેટિંગ} & ચોક્કસ ઑડિયન્સ ટાર્ગેટિંગ & બ્રોડ કીવર્ડ ટાર્ગેટિંગ \\ \hline
\end{tabulary}
\end{center}

\begin{itemize}
    \item \textbf{PPC ફાયદા}: તાત્કાલિક પરિણામો, ચોક્કસ ટાર્ગેટિંગ, માપી શકાય તેવા ROI
    \item \textbf{SEO ફાયદા}: લાંબા ગાળાની ટકાઉપણું, વિશ્વસનીયતા, કોસ્ટ-એફેક્ટિવનેસ
    \item \textbf{સંયુક્ત અભિગમ}: બંને વ્યૂહરચનાઓ એકસાથે વધુ સારી કામ કરે
\end{itemize}

\begin{mnemonicbox}IRCST - Immediate vs Reactive, Control vs Sustainable, Targeted\end{mnemonicbox}
\end{solutionbox}

\questionmarks{5(ક)}{7}{Google Ads ઝુંબેશના વિવિધ પ્રકારો સમજાવો.}

\begin{solutionbox}
Google Ads વિવિધ માર્કેટિંગ ઉદ્દેશ્યો માટે વિવિધ કેમ્પેઈન ટાઈપ્સ ઓફર કરે છે.

\begin{center}
\begin{tikzpicture}[node distance=1.5cm, auto]
    \node [gtu block] (Ads) {Google Ads\\કેમ્પેઈન્સ};
    
    \node [gtu block, below left=1.5cm and 3cm of Ads] (Search) {સર્ચ};
    \node [gtu block, below left=1.5cm and 1cm of Ads] (Display) {ડિસ્પ્લે};
    \node [gtu block, below=1.5cm of Ads] (Video) {વિડિયો};
    \node [gtu block, below right=1.5cm and 1cm of Ads] (Shopping) {શોપિંગ};
    \node [gtu block, below right=1.5cm and 3cm of Ads] (App) {એપ};
    
    \node [gtu state, below=0.8cm of Search] (S1) {ટેક્સ્ટ એડ્સ};
    \node [gtu state, below=0.8cm of Display] (D1) {બેનર એડ્સ};
    \node [gtu state, below=0.8cm of Video] (V1) {YouTube એડ્સ};
    \node [gtu state, below=0.8cm of Shopping] (Sh1) {પ્રોડક્ટ એડ્સ};
    \node [gtu state, below=0.8cm of App] (A1) {એપ પ્રમોશન};
    
    \foreach \n in {Search, Display, Video, Shopping, App}
        \path [gtu arrow] (Ads) -- (\n);
        
    \path [gtu arrow] (Search) -- (S1);
    \path [gtu arrow] (Display) -- (D1);
    \path [gtu arrow] (Video) -- (V1);
    \path [gtu arrow] (Shopping) -- (Sh1);
    \path [gtu arrow] (App) -- (A1);
\end{tikzpicture}
\captionof{figure}{Google Ads કેમ્પેઈન ટાઈપ્સ}
\end{center}

\begin{center}
\captionof{table}{Google Ads કેમ્પેઈન્સ}
\begin{tabulary}{\linewidth}{|L|L|L|L|}
\hline
\textbf{ટાઈપ} & \textbf{ફોર્મેટ} & \textbf{શ્રેષ્ઠ ઉપયોગ} & \textbf{ઉદ્દેશ્ય} \\ \hline
\textbf{સર્ચ} & સર્ચ પરિણામો પર ટેક્સ્ટ એડ્સ & હાઈ-ઇન્ટેન્ટ કીવર્ડ્સ & ટ્રાફિક, વેચાણ \\ \hline
\textbf{ડિસ્પ્લે} & પાર્ટનર સાઈટ્સ પર વિઝ્યુઅલ & બ્રાન્ડ અવેરનેસ & વ્યાપક પહોંચ \\ \hline
\textbf{વિડિયો} & YouTube પર વિડિયો એડ્સ & એન્ગેજમેન્ટ, બ્રાન્ડિંગ & પ્રમોશન \\ \hline
\textbf{શોપિંગ} & ઈમેજ સાથે પ્રોડક્ટ લિસ્ટિંગ્સ & ઈ-કોમર્સ વેચાણ & ડાયરેક્ટ શોકેસ \\ \hline
\textbf{એપ} & ઓટોમેટેડ એપ પ્રમોશન & મોબાઈલ એપ ડાઉનલોડ્સ & ઇન્સ્ટોલ વધારવા \\ \hline
\textbf{Perf Max} & મલ્ટી-ચેનલ ઓટોમેશન & મેક્સિમમ પ્રદર્શન & AI ઓપ્ટિમાઈઝેશન \\ \hline
\end{tabulary}
\end{center}

\textbf{બજેટ એલોકેશન ભલામણો:}
\begin{itemize}
    \item \textbf{સર્ચ}: ઉચ્ચ કન્વર્ટિંગ કીવર્ડ્સ માટે બજેટના 40-50\%
    \item \textbf{ડિસ્પ્લે}: અવેરનેસ અને રીમાર્કેટિંગ માટે 20-30\%
    \item \textbf{વિડિયો}: એન્ગેજમેન્ટ અને બ્રાન્ડ બિલ્ડિંગ માટે 15-25\%
    \item \textbf{શોપિંગ}: ઈ-કોમર્સ બિઝનેસ માટે 30-40\%
\end{itemize}

\begin{itemize}
    \item \textbf{મલ્ટી-કેમ્પેઈન અભિગમ}: વ્યાપક પહોંચ માટે ટાઈપ્સ સંયોજિત કરો
    \item \textbf{ઑડિયન્સ જર્ની}: વિવિધ કેમ્પેઈન્સ વિવિધ ખરીદી સ્ટેજ ટાર્ગેટ કરે
    \item \textbf{પ્રદર્શન ઓપ્ટિમાઈઝેશન}: નિયમિત મોનિટરિંગ પરિણામો સુધારે
\end{itemize}

\begin{mnemonicbox}SDVSAP - Search, Display, Video, Shopping, App, Performance Max\end{mnemonicbox}
\end{solutionbox}

\questionmarks{5(અ OR)}{3}{માર્કેટિંગ વ્યૂહરચનાઓની સફળતાને ટ્રેક કરવા માટે Instagram પર ઉપલબ્ધ મેટ્રિક્સની સૂચિ બનાવો.}

\begin{solutionbox}
Instagram Insights કેમ્પેઈન પ્રદર્શન વિશ્લેષણ માટે વ્યાપક મેટ્રિક્સ પ્રદાન કરે છે.

\begin{center}
\captionof{table}{Instagram મેટ્રિક્સ}
\begin{tabulary}{\linewidth}{|L|L|}
\hline
\textbf{મેટ્રિક કેટેગરી} & \textbf{ચોક્કસ મેટ્રિક્સ} \\ \hline
\textbf{પહોંચ મેટ્રિક્સ} & ઇમ્પ્રેશન્સ, પહોંચ, પ્રોફાઈલ વિઝિટ્સ \\ \hline
\textbf{એન્ગેજમેન્ટ મેટ્રિક્સ} & લાઈક્સ, કોમેન્ટ્સ, શેર્સ, સેવ્સ \\ \hline
\textbf{સ્ટોરી મેટ્રિક્સ} & સ્ટોરી વ્યૂઝ, ટેપ્સ ફોરવર્ડ/બેક, એક્ઝિટ્સ \\ \hline
\textbf{ઑડિયન્સ મેટ્રિક્સ} & ડેમોગ્રાફિક્સ, એક્ટિવ ટાઈમ્સ, લોકેશન્સ \\ \hline
\end{tabulary}
\end{center}

\begin{itemize}
    \item \textbf{કન્ટેન્ટ પ્રદર્શન}: કયા પોસ્ટ્સ સૌથી વધુ એન્ગેજમેન્ટ લાવે છે તે ટ્રેક કરો
    \item \textbf{ઑડિયન્સ ઇનસાઈટ્સ}: ફોલોવર ડેમોગ્રાફિક્સ અને વર્તન સમજો
    \item \textbf{ગ્રોથ ટ્રેકિંગ}: ફોલોવર કાઉન્ટ અને એન્ગેજમેન્ટ રેટ ફેરફારો મોનિટર કરો
\end{itemize}

\begin{mnemonicbox}RESA - Reach, Engagement, Stories, Audience\end{mnemonicbox}
\end{solutionbox}

\questionmarks{5(બ OR)}{4}{ડિજિટલ માર્કેટિંગમાં ઈમેલ માર્કેટિંગના ફાયદાઓનું વર્ણન કરો.}

\begin{solutionbox}
ઈમેલ માર્કેટિંગ કસ્ટમર કમ્યુનિકેશન અને કન્વર્ઝન માટે અત્યંત અસરકારક રહે છે.

\begin{center}
\captionof{table}{ઈમેલ માર્કેટિંગ ફાયદા}
\begin{tabulary}{\linewidth}{|L|L|L|}
\hline
\textbf{ફાયદો} & \textbf{વર્ણન} & \textbf{અસર} \\ \hline
\textbf{ઊંચો ROI} & દરેક \$1 ખર્ચ માટે \$42 રિટર્ન & કોસ્ટ-એફેક્ટિવ \\ \hline
\textbf{ડાયરેક્ટ} & પર્સનલ ઇનબોક્સ એક્સેસ & ઘનિષ્ઠ જોડાણ \\ \hline
\textbf{સેગમેન્ટેશન} & ગ્રુપ્સ દ્વારા ટાર્ગેટેડ મેસેજિંગ & સુધારેલી સુસંગતતા \\ \hline
\textbf{ઓટોમેશન} & શેડ્યુલ્ડ અને ટ્રિગર્ડ ઈમેલ્સ & કાર્યક્ષમ નર્ચરિંગ \\ \hline
\textbf{માપી શકાય તેવા} & વિગતવાર એનાલિટિક્સ ઉપલબ્ધ & ડેટા-ડ્રિવન ઓપ્ટિમાઈઝેશન \\ \hline
\end{tabulary}
\end{center}

\begin{itemize}
    \item \textbf{પરમિશન-બેસ્ડ}: સબસ્ક્રાઈબર્સે સક્રિય રીતે કમ્યુનિકેશન પ્રાપ્ત કરવાનું પસંદ કર્યું
    \item \textbf{પર્સનલાઈઝેશન}: યુઝર પસંદગીઓ અને વર્તન આધારે કસ્ટમાઈઝ્ડ કન્ટેન્ટ
    \item \textbf{સ્કેલેબિલિટી}: સિંગલ કેમ્પેઈન સાથે હજારો કસ્ટમર્સ સુધી પહોંચો
\end{itemize}

\begin{mnemonicbox}HDSAM - High ROI, Direct, Segmented, Automated, Measurable\end{mnemonicbox}
\end{solutionbox}

\questionmarks{5(ક OR)}{7}{Google Ads માં ઉપલબ્ધ વિવિધ બિડિંગ વ્યૂહરચનાઓ સમજાવો.}

\begin{solutionbox}
Google Ads ઉદ્દેશ્યો આધારે કેમ્પેઈન પ્રદર્શન ઓપ્ટિમાઈઝ કરવા માટે ઘણી બિડિંગ વ્યૂહરચનાઓ ઓફર કરે છે.

\begin{center}
\begin{tikzpicture}[node distance=1.5cm, auto]
    \node [gtu block] (Bidding) {Google Ads\\બિડિંગ};
    
    \node [gtu block, below left=1.2cm and 2cm of Bidding] (Manual) {મેન્યુઅલ બિડિંગ};
    \node [gtu block, below right=1.2cm and 0.5cm of Bidding] (Automated) {ઓટોમેટેડ બિડિંગ};
    
    \node [gtu state, below=0.8cm of Manual] (ManualEx) {મેન્યુઅલ CPC\\એન્હાન્સ્ડ CPC};
    
    \node [gtu state, below=0.8cm of Automated, text width=6cm] (AutoEx) {ટાર્ગેટ CPA, ટાર્ગેટ ROAS\\મેક્સિમાઈઝ ક્લિક્સ/કન્વર્ઝન્સ\\ટાર્ગેટ ઇમ્પ્રેશન શેર};
    
    \path [gtu arrow] (Bidding) -- (Manual);
    \path [gtu arrow] (Bidding) -- (Automated);
    
    \path [gtu arrow] (Manual) -- (ManualEx);
    \path [gtu arrow] (Automated) -- (AutoEx);
\end{tikzpicture}
\captionof{figure}{બિડિંગ વ્યૂહરચનાઓ}
\end{center}

\begin{center}
\captionof{table}{બિડિંગ વ્યૂહરચના માર્ગદર્શિકા}
\begin{tabulary}{\linewidth}{|L|L|L|}
\hline
\textbf{વ્યૂહરચના} & \textbf{ટાઈપ} & \textbf{ગોલ} \\ \hline
\textbf{મેન્યુઅલ CPC} & મેન્યુઅલ & સંપૂર્ણ કંટ્રોલ \\ \hline
\textbf{એન્હાન્સ્ડ CPC} & સેમી-ઓટો & કન્વર્ઝન ઓપ્ટિમાઈઝેશન \\ \hline
\textbf{ટાર્ગેટ CPA} & ઓટોમેટેડ & કોસ્ટ પર એક્વિઝિશન \\ \hline
\textbf{ટાર્ગેટ ROAS} & ઓટોમેટેડ & રિટર્ન ઓન એડ સ્પેન્ડ \\ \hline
\textbf{મેક્સિમાઈઝ ક્લિક્સ} & ઓટોમેટેડ & ટ્રાફિક જનરેશન \\ \hline
\textbf{મેક્સિમાઈઝ કન્વર્ઝન} & ઓટોમેટેડ & વોલ્યુમ ઓપ્ટિમાઈઝેશન \\ \hline
\end{tabulary}
\end{center}

\textbf{વ્યૂહરચના સિલેક્શન ગાઈડ:}
\begin{itemize}
    \item \textbf{નવા કેમ્પેઈન્સ}: ડેટા બિલ્ડ કરવા માટે મેક્સિમાઈઝ ક્લિક્સ અથવા મેન્યુઅલ CPC
    \item \textbf{સ્થાપિત}: કાર્યક્ષમતા માટે ટાર્ગેટ CPA અથવા ROAS
    \item \textbf{બજેટ}: ઓટોમેટેડ વ્યૂહરચના વધઘટ માટે બજેટ પરવાનગી આપે છે તેની ખાતરી કરો
\end{itemize}

\begin{itemize}
    \item \textbf{એલ્ગોરિધમ લર્નિંગ}: ઓટોમેટેડ બિડિંગ વધુ ડેટા સાથે સુધરે
    \item \textbf{પ્રદર્શન ગોલ્સ}: બિઝનેસ ઉદ્દેશ્યો મેળ ખાતી વ્યૂહરચના પસંદ કરો
    \item \textbf{બજેટ મેનેજમેન્ટ}: વિવિધ વ્યૂહરચનાઓ સાથે ખર્ચ પેટર્ન વિચારો
\end{itemize}

\begin{mnemonicbox}METER-MT - Manual, Enhanced, Target CPA, Target ROAS, Maximize\end{mnemonicbox}
\end{solutionbox}

\end{document}
