\documentclass[10pt,a4paper]{article}

% content/resources/templates/preamble.tex
\usepackage[margin=0.6in]{geometry}
\author{Milav Dabgar}
\usepackage{amsmath,amssymb,amsthm}
\usepackage{booktabs}
\usepackage{multirow}
\usepackage{xcolor}
\usepackage{tcolorbox}
\tcbuselibrary{breakable,skins}
\usepackage[colorlinks=true,linkcolor=blue]{hyperref}
\usepackage{titlesec}
\usepackage{enumitem}
\usepackage{tikz}
\usepackage{pgfplots}
\usepackage{circuitikz}
\usepackage[version=4]{mhchem}
\usepackage{longtable}
\usepackage{array}
\usepackage{float}
\usepackage{caption}
\usepackage{listings}

\lstset{
  basicstyle=\small\ttfamily,
  breaklines=true,
  breakatwhitespace=false,
  postbreak=\mbox{\textcolor{red}{$\hookrightarrow$}\space},
  float=false,
  numbers=left,
  numberstyle=\tiny\color{gray},
  numbersep=10pt,
  xleftmargin=2em,
  keywordstyle=\color{blue},
  commentstyle=\color{green!60!black},
  stringstyle=\color{purple},
  backgroundcolor=\color{gray!5},
  showstringspaces=false,
  tabsize=2,
  captionpos=b,
  keepspaces=true,
  columns=flexible
}

\pgfplotsset{compat=1.18}
\usetikzlibrary{shapes,arrows,positioning,calc,patterns,decorations.pathmorphing,decorations.markings,arrows.meta}

% Color scheme
\definecolor{headcolor}{RGB}{0,102,204}
\definecolor{keycolor}{RGB}{220,20,60}
\definecolor{solutioncolor}{RGB}{34,139,34}
\definecolor{mnemoniccolor}{RGB}{148,0,211}
\definecolor{codecolor}{RGB}{0,0,100}

% Spacing
\setlength{\parskip}{3pt}
\setlist[itemize]{nosep}
\setlist[enumerate]{nosep}

% Title formatting
\titleformat{\section}{\Large\bfseries\color{headcolor}}{\thesection}{1em}{}
\titleformat{\subsection}{\large\bfseries\color{headcolor}}{\thesubsection}{1em}{}

% Pandoc tightlist compatibility
\providecommand{\tightlist}{%
  \setlength{\itemsep}{0pt}\setlength{\parskip}{0pt}}

% Pandoc longtable compatibility
\newcounter{none}
\def\thenone{}


% content/resources/templates/gujarati-boxes.tex
\usepackage{fontspec}
\usepackage{polyglossia}

% Set Gujarati as main language (document is primarily in Gujarati)
% Note: gloss-gujarati.ldf doesn't exist in polyglossia, but it will use hyphenation patterns
\setdefaultlanguage{gujarati}
\setotherlanguage{english}

% Configure Gujarati font properly
% Use Language=Default to prevent polyglossia from trying to add language-specific features
% that don't exist for Gujarati, which causes "empty feature" warnings
\newfontfamily\gujaratifont[Script=Gujarati,AutoFakeBold=2.5,AutoFakeSlant=0.3]{Noto Sans Gujarati}
\setmainfont[Script=Gujarati,AutoFakeBold=2.5,AutoFakeSlant=0.3]{Noto Sans Gujarati}
% Use Noto Sans Gujarati for monospace to support Gujarati in text
\setmonofont[Scale=0.9]{Noto Sans Gujarati}

% Configure English to use the same font
\newfontfamily\englishfont[Script=Gujarati,AutoFakeBold=2.5,AutoFakeSlant=0.3]{Noto Sans Gujarati}

% Translations for polyglossia
\gappto\captionsgujarati{
  \renewcommand{\tablename}{કોષ્ટક}
  \renewcommand{\figurename}{આકૃતિ}
}

% Helper for TikZ nodes to ensure Gujarati font
\newcommand{\gu}[1]{{\gujaratifont #1}}

% Custom environments
\newtcolorbox{solutionbox}{
    breakable,
    enhanced,
    colback=solutioncolor!5!white,
    colframe=solutioncolor!75!black,
    fonttitle=\bfseries,
    title=જવાબ
}

\newtcolorbox{solutionboxnobreak}{
 colback=solutioncolor!5!white,
 colframe=solutioncolor!75!black,
 fonttitle=\bfseries,
 title=જવાબ
}

\newtcolorbox{keyformula}{
 breakable,
 enhanced,
 colback=keycolor!5!white,
 colframe=keycolor!75!black,
 fonttitle=\bfseries,
 title=રાસાયણિક સમીકરણ/સૂત્ર
}

\newtcolorbox{mnemonicbox}{
 breakable,
 enhanced,
 colback=mnemoniccolor!5!white,
 colframe=mnemoniccolor!75!black,
 fonttitle=\bfseries,
 title=મેમરી ટ્રીક
}


\begin{document}

\begin{center}
{\Huge\bfseries\color{headcolor} Subject Name (Gujarati)}\\[5pt]
{\LARGE 4341602 -- Winter 2024}\\[3pt]
{\large Semester 1 Study Material}\\[3pt]
{\normalsize\textit{Detailed Solutions and Explanations}}
\end{center}

\vspace{10pt}

\subsection*{પ્રશ્ન 1(અ) [3
ગુણ]}\label{uxaaauxab0uxab6uxaa8-1uxa85-3-uxa97uxaa3}

\textbf{OOP અને POP વચ્ચેનો તફાવત લખો.}

\begin{solutionbox}

{\def\LTcaptype{none} % do not increment counter
\begin{longtable}[]{@{}
  >{\raggedright\arraybackslash}p{(\linewidth - 4\tabcolsep) * \real{0.3793}}
  >{\raggedright\arraybackslash}p{(\linewidth - 4\tabcolsep) * \real{0.3103}}
  >{\raggedright\arraybackslash}p{(\linewidth - 4\tabcolsep) * \real{0.3103}}@{}}
\toprule\noalign{}
\begin{minipage}[b]{\linewidth}\raggedright
\textbf{પાસાં}
\end{minipage} & \begin{minipage}[b]{\linewidth}\raggedright
\textbf{OOP}
\end{minipage} & \begin{minipage}[b]{\linewidth}\raggedright
\textbf{POP}
\end{minipage} \\
\midrule\noalign{}
\endhead
\bottomrule\noalign{}
\endlastfoot
\textbf{અભિગમ} & બોટમ-અપ અભિગમ & ટોપ-ડાઉન અભિગમ \\
\textbf{ફોકસ} & ઓબ્જેક્ટ અને ક્લાસ & ફંક્શન અને પ્રોસીજર \\
\textbf{ડેટા સિક્યોરિટી} & એન્કેપ્સુલેશન દ્વારા ડેટા હાઇડિંગ & ડેટા હાઇડિંગ નથી \\
\textbf{પ્રોબ્લેમ સોલ્વિંગ} & સમસ્યાને ઓબ્જેક્ટમાં વિભાજિત કરો & સમસ્યાને ફંક્શનમાં
વિભાજિત કરો \\
\end{longtable}
}

\end{solutionbox}
\begin{mnemonicbox}
``ઓબ્જેક્ટ બોટમ, પ્રોસીજર ટોપ''

\end{mnemonicbox}
\subsection*{પ્રશ્ન 1(બ) [4
ગુણ]}\label{uxaaauxab0uxab6uxaa8-1uxaac-4-uxa97uxaa3}

\textbf{બાઇટ કોડ શું છે? JVM ને વિગતવાર સમજાવો.}

\begin{solutionbox}

\textbf{બાઇટ કોડ}: Java compiler દ્વારા સોર્સ કોડમાંથી જનરેટ થતો
પ્લેટફોર્મ-ઇન્ડિપેન્ડન્ટ ઇન્ટરમીડિયેટ કોડ.

\begin{center}
\textbf{Mermaid Diagram (Code)}
\begin{verbatim}
{Shaded}
{Highlighting}[]
graph LR
    A[Java Source Code] {-{-}{} B[Java Compiler javac]}
    B {-{-}{} C[Byte Code .class]}
    C {-{-}{} D[JVM]}
    D {-{-}{} E[Machine Code]}
{Highlighting}
{Shaded}
\end{verbatim}
\end{center}

\textbf{JVM કોમ્પોનન્ટ્સ}:

\begin{itemize}
\tightlist
\item
  \textbf{Class Loader}: .class ફાઇલોને મેમરીમાં લોડ કરે છે
\item
  \textbf{Memory Area}: Heap, stack, method area સ્ટોરેજ
\item
  \textbf{Execution Engine}: બાઇટકોડને ઇન્ટરપ્રેટ અને એક્ઝિક્યુટ કરે છે
\item
  \textbf{Garbage Collector}: ઓટોમેટિક મેમરી મેનેજમેન્ટ
\end{itemize}

\end{solutionbox}
\begin{mnemonicbox}
``બાઇટ કોડ દરેક જગ્યાએ ચાલે છે''

\end{mnemonicbox}
\subsection*{પ્રશ્ન 1(ક) [7
ગુણ]}\label{uxaaauxab0uxab6uxaa8-1uxa95-7-uxa97uxaa3}

\textbf{એરેના એલિમેન્ટ્સને ચડતા ક્રમમાં સૉર્ટ કરવા માટે જાવામાં પ્રોગ્રામ લખો}

\begin{solutionbox}

\begin{verbatim}
import java.util.Arrays;

public class ArraySort \{
    public static void main(String[] args) \{
        int[] arr = \{64, 34, 25, 12, 22, 11, 90\;}
        
        // Bubble Sort
        for(int i = 0; i {} arr.length{-}1; i++) \{
            for(int j = 0; j {} arr.length{-}i{-}1; j++) \{
                if(arr[j] {} arr[j+1]) \{
                    int temp = arr[j];
                    arr[j] = arr[j+1];
                    arr[j+1] = temp;
                \}
            \}
        \}
        
        System.out.println("Sorted array: " + Arrays.toString(arr));
    \}
\}
\end{verbatim}

\textbf{મુખ્ય મુદ્દાઓ}:

\begin{itemize}
\tightlist
\item
  \textbf{Bubble Sort}: બાજુના એલિમેન્ટ્સની તુલના કરે છે
\item
  \textbf{Time Complexity}: O(n^{2})
\item
  \textbf{Space Complexity}: O(1)
\end{itemize}

\end{solutionbox}
\begin{mnemonicbox}
``બબલ અપ ધ સ્મોલેસ્ટ''

\end{mnemonicbox}
\subsection*{પ્રશ્ન 1(ક OR) [7
ગુણ]}\label{uxaaauxab0uxab6uxaa8-1uxa95-or-7-uxa97uxaa3}

\textbf{કમાન્ડ લાઇન આર્ગ્યુમેન્ટ્સનો ઉપયોગ કરીને કોઈપણ દસ સંખ્યાઓમાંથી મહત્તમ શોધવા
માટે જાવામાં પ્રોગ્રામ લખો.}

\begin{solutionbox}

\begin{verbatim}
public class FindMaximum \{
    public static void main(String[] args) \{
        if(args.length != 10) \{
            System.out.println("કૃપા કરીને બરાબર 10 સંખ્યાઓ દાખલ કરો");
            return;
        \}
        
        int max = Integer.parseInt(args[0]);
        
        for(int i = 1; i {} args.length; i++) \{
            int num = Integer.parseInt(args[i]);
            if(num {} max) \{
                max = num;
            \}
        \}
        
        System.out.println("મહત્તમ સંખ્યા: " + max);
    \}
\}
\end{verbatim}

\textbf{મુખ્ય મુદ્દાઓ}:

\begin{itemize}
\tightlist
\item
  \textbf{Command Line}: args[] array આર્ગ્યુમેન્ટ્સ સ્ટોર કરે છે
\item
  \textbf{parseInt()}: સ્ટ્રિંગને ઇન્ટિજરમાં કન્વર્ટ કરે છે
\item
  \textbf{Validation}: Array length ચેક કરો
\end{itemize}

\end{solutionbox}
\begin{mnemonicbox}
``આર્ગ્યુમેન્ટ્સ મેક્સિમમ સર્ચ''

\end{mnemonicbox}
\subsection*{પ્રશ્ન 2(અ) [3
ગુણ]}\label{uxaaauxab0uxab6uxaa8-2uxa85-3-uxa97uxaa3}

\textbf{Wrapper ક્લાસ શું છે? ઉદાહરણ સાથે સમજાવો.}

\begin{solutionbox}

\textbf{Wrapper Class}: પ્રિમિટિવ ડેટા ટાઇપ્સને ઓબ્જેક્ટમાં કન્વર્ટ કરે છે.

{\def\LTcaptype{none} % do not increment counter
\begin{longtable}[]{@{}ll@{}}
\toprule\noalign{}
\textbf{Primitive} & \textbf{Wrapper Class} \\
\midrule\noalign{}
\endhead
\bottomrule\noalign{}
\endlastfoot
int & Integer \\
char & Character \\
boolean & Boolean \\
double & Double \\
\end{longtable}
}

\begin{verbatim}
// Boxing
Integer obj = Integer.valueOf(10);
// Unboxing  
int value = obj.intValue();
\end{verbatim}

\end{solutionbox}
\begin{mnemonicbox}
``પ્રિમિટિવ્સને ઓબ્જેક્ટમાં લપેટો''

\end{mnemonicbox}
\subsection*{પ્રશ્ન 2(બ) [4
ગુણ]}\label{uxaaauxab0uxab6uxaa8-2uxaac-4-uxa97uxaa3}

\textbf{જાવાના વિવિધ લક્ષણોની યાદી આપો. કોઈપણ બે સમજાવો.}

\begin{solutionbox}

\textbf{Java Features}:

\begin{itemize}
\tightlist
\item
  \textbf{Simple}: સરળ syntax, pointers નથી
\item
  \textbf{Platform Independent}: એકવાર લખો, દરેક જગ્યાએ ચલાવો\\
\item
  \textbf{Object Oriented}: ઓબ્જેક્ટ અને ક્લાસ પર આધારિત
\item
  \textbf{Secure}: explicit pointers નથી, bytecode verification
\end{itemize}

\textbf{વિગતવાર સમજૂતી}:

\begin{itemize}
\tightlist
\item
  \textbf{Platform Independence}: Java bytecode JVM વાળા કોઈપણ પ્લેટફોર્મ
  પર ચાલે છે
\item
  \textbf{Object Oriented}: inheritance, encapsulation, polymorphism,
  abstraction સપોર્ટ કરે છે
\end{itemize}

\end{solutionbox}
\begin{mnemonicbox}
``સિમ્પલ પ્લેટફોર્મ ઓબ્જેક્ટ સિક્યોરિટી''

\end{mnemonicbox}
\subsection*{પ્રશ્ન 2(ક) [7
ગુણ]}\label{uxaaauxab0uxab6uxaa8-2uxa95-7-uxa97uxaa3}

\textbf{ઓવરરાઇડિંગ પદ્ધતિ શું છે? ઉદાહરણ સાથે સમજાવો.}

\begin{solutionbox}

\textbf{Method Overriding}: ચાઇલ્ડ ક્લાસ પેરન્ટ ક્લાસની મેથડનું વિશિષ્ટ
implementation પ્રદાન કરે છે.

\begin{verbatim}
class Animal \{
    public void sound() \{
        System.out.println("પ્રાણી અવાજ કરે છે");
    \}
\}

class Dog extends Animal \{
    @Override
    public void sound() \{
        System.out.println("કૂતરો ભસે છે");
    \}
\}

public class Test \{
    public static void main(String[] args) \{
        Animal a = new Dog();
        a.sound(); // આઉટપુટ: કૂતરો ભસે છે
    \}
\}
\end{verbatim}

\textbf{મુખ્ય મુદ્દાઓ}:

\begin{itemize}
\tightlist
\item
  \textbf{Runtime Polymorphism}: ઓબ્જેક્ટ ટાઇપના આધારે મેથડ કોલ થાય છે
\item
  \textbf{@Override}: મેથડ ઓવરરાઇડિંગ માટે annotation
\item
  \textbf{Dynamic Binding}: રનટાઇમ પર મેથડ રિઝોલ્યુશન
\end{itemize}

\end{solutionbox}
\begin{mnemonicbox}
``ચાઇલ્ડ પેરન્ટ મેથડ બદલે છે''

\end{mnemonicbox}
\subsection*{પ્રશ્ન 2(અ OR) [3
ગુણ]}\label{uxaaauxab0uxab6uxaa8-2uxa85-or-3-uxa97uxaa3}

\textbf{જાવામાં Garbage collection સમજાવો.}

\begin{solutionbox}

\textbf{Garbage Collection}: ઓટોમેટિક મેમરી મેનેજમેન્ટ જે અનુપયોગી ઓબ્જેક્ટ્સને દૂર કરે
છે.

\begin{center}
\textbf{Mermaid Diagram (Code)}
\begin{verbatim}
{Shaded}
{Highlighting}[]
graph LR
    A[Object Created] {-{-}{} B[Object Used]}
    B {-{-}{} C[Object Unreferenced]}
    C {-{-}{} D[Garbage Collector]}
    D {-{-}{} E[Memory Freed]}
{Highlighting}
{Shaded}
\end{verbatim}
\end{center}

\textbf{મુખ્ય મુદ્દાઓ}:

\begin{itemize}
\tightlist
\item
  \textbf{Automatic}: મેન્યુઅલ મેમરી deallocation નથી
\item
  \textbf{Mark and Sweep}: અનુપયોગી ઓબ્જેક્ટ્સને ઓળખે અને દૂર કરે છે
\item
  \textbf{Heap Memory}: heap memory area પર કામ કરે છે
\end{itemize}

\end{solutionbox}
\begin{mnemonicbox}
``ઓટો ક્લીન અનયુઝ્ડ ઓબ્જેક્ટ્સ''

\end{mnemonicbox}
\subsection*{પ્રશ્ન 2(બ OR) [4
ગુણ]}\label{uxaaauxab0uxab6uxaa8-2uxaac-or-4-uxa97uxaa3}

\textbf{static કીવર્ડ ઉદાહરણ સાથે સમજાવો.}

\begin{solutionbox}

\textbf{Static Keyword}: ઇન્સ્ટન્સને બદલે ક્લાસનું છે.

\begin{verbatim}
class Student \{
    static String college = "GTU";  // Static variable
    String name;
    
    static void showCollege() \{     // Static method
        System.out.println("કૉલેજ: " + college);
    \}
\}
\end{verbatim}

\textbf{Static Features}:

\begin{itemize}
\tightlist
\item
  \textbf{Memory}: ક્લાસ લોડિંગ ટાઇમે લોડ થાય છે
\item
  \textbf{Access}: ઓબ્જેક્ટ વિના એક્સેસ કરી શકાય છે
\item
  \textbf{Sharing}: બધા instances વચ્ચે શેર થાય છે
\end{itemize}

\end{solutionbox}
\begin{mnemonicbox}
``ક્લાસ લેવલ મેમરી શેરિંગ''

\end{mnemonicbox}
\subsection*{પ્રશ્ન 2(ક OR) [7
ગુણ]}\label{uxaaauxab0uxab6uxaa8-2uxa95-or-7-uxa97uxaa3}

\textbf{કન્સ્ટ્રક્ટર શું છે? કોપી કન્સ્ટ્રક્ટરને ઉદાહરણ સાથે સમજાવો.}

\begin{solutionbox}

\textbf{Constructor}: ઓબ્જેક્ટ્સને initialize કરવા માટેની વિશેષ મેથડ.

\begin{verbatim}
class Person \{
    String name;
    int age;
    
    // Default constructor
    Person() \{
        name = "અજ્ઞાત";
        age = 0;
    \}
    
    // Parameterized constructor
    Person(String n, int a) \{
        name = n;
        age = a;
    \}
    
    // Copy constructor
    Person(Person p) \{
        name = p.name;
        age = p.age;
    \}
\}
\end{verbatim}

\textbf{Constructor Types}:

\begin{itemize}
\tightlist
\item
  \textbf{Default}: કોઈ પેરામીટર નથી
\item
  \textbf{Parameterized}: પેરામીટર લે છે
\item
  \textbf{Copy}: અસ્તિત્વમાં રહેલા ઓબ્જેક્ટમાંથી ઓબ્જેક્ટ બનાવે છે
\end{itemize}

\end{solutionbox}
\begin{mnemonicbox}
``ડિફોલ્ટ પેરામીટર કોપી''

\end{mnemonicbox}
\subsection*{પ્રશ્ન 3(અ) [3
ગુણ]}\label{uxaaauxab0uxab6uxaa8-3uxa85-3-uxa97uxaa3}

\textbf{super કીવર્ડ ઉદાહરણ સાથે સમજાવો.}

\begin{solutionbox}

\textbf{Super Keyword}: પેરન્ટ ક્લાસના સભ્યોનો સંદર્ભ આપે છે.

\begin{verbatim}
class Vehicle \{
    String brand = "જેનેરિક";
\}

class Car extends Vehicle \{
    String brand = "ટોયોટા";
    
    void display() \{
        System.out.println("ચાઇલ્ડ: " + brand);
        System.out.println("પેરન્ટ: " + super.brand);
    \}
\}
\end{verbatim}

\textbf{Super Uses}:

\begin{itemize}
\tightlist
\item
  \textbf{Variables}: પેરન્ટ ક્લાસના variables એક્સેસ કરો
\item
  \textbf{Methods}: પેરન્ટ ક્લાસની methods કૉલ કરો\\
\item
  \textbf{Constructor}: પેરન્ટ ક્લાસના constructor કૉલ કરો
\end{itemize}

\end{solutionbox}
\begin{mnemonicbox}
``સુપર પેરન્ટને કૉલ કરે છે''

\end{mnemonicbox}
\subsection*{પ્રશ્ન 3(બ) [4
ગુણ]}\label{uxaaauxab0uxab6uxaa8-3uxaac-4-uxa97uxaa3}

\textbf{inheritance ના વિવિધ પ્રકારોની યાદી આપો. multilevel inheritance
સમજાવો.}

\begin{solutionbox}

\textbf{Inheritance Types}:

{\def\LTcaptype{none} % do not increment counter
\begin{longtable}[]{@{}ll@{}}
\toprule\noalign{}
\textbf{Type} & \textbf{વર્ણન} \\
\midrule\noalign{}
\endhead
\bottomrule\noalign{}
\endlastfoot
Single & એક પેરન્ટ, એક ચાઇલ્ડ \\
Multilevel & inheritance ની ચેઇન \\
Hierarchical & એક પેરન્ટ, બહુવિધ ચિલ્ડ્રન \\
Multiple & બહુવિધ પેરન્ટ્સ (interfaces દ્વારા) \\
\end{longtable}
}

\textbf{Multilevel Inheritance}:

\begin{verbatim}
class Animal \{
    void eat() \{ System.out.println("ખાવું"); \}
\}

class Mammal extends Animal \{
    void breathe() \{ System.out.println("શ્વાસ લેવો"); \}
\}

class Dog extends Mammal \{
    void bark() \{ System.out.println("ભસવું"); \}
\}
\end{verbatim}

\end{solutionbox}
\begin{mnemonicbox}
``સિંગલ મલ્ટી હાયરાર્કિકલ મલ્ટિપલ''

\end{mnemonicbox}
\subsection*{પ્રશ્ન 3(ક) [7
ગુણ]}\label{uxaaauxab0uxab6uxaa8-3uxa95-7-uxa97uxaa3}

\textbf{ઇન્ટરફેસ શું છે? ઉદાહરણ સાથે multiple inheritance સમજાવો.}

\begin{solutionbox}

\textbf{Interface}: કોન્ટ્રાક્ટ જે ક્લાસે શું કરવું જોઈએ તે વ્યાખ્યાયિત કરે છે, કેવી રીતે
નહીં.

\begin{verbatim}
interface Flyable \{
    void fly();
\}

interface Swimmable \{
    void swim();
\}

class Duck implements Flyable, Swimmable \{
    public void fly() \{
        System.out.println("બતક ઉડી રહી છે");
    \}
    
    public void swim() \{
        System.out.println("બતક તરી રહી છે");
    \}
\}
\end{verbatim}

\textbf{Interface Features}:

\begin{itemize}
\tightlist
\item
  \textbf{Multiple Inheritance}: ક્લાસ બહુવિધ interfaces implement કરી શકે
  છે
\item
  \textbf{Abstract Methods}: બધી methods ડિફોલ્ટ રૂપે abstract છે
\item
  \textbf{Constants}: બધા variables public, static, final છે
\end{itemize}

\end{solutionbox}
\begin{mnemonicbox}
``મલ્ટિપલ એબ્સ્ટ્રાક્ટ કોન્સ્ટન્ટ્સ''

\end{mnemonicbox}
\subsection*{પ્રશ્ન 3(અ OR) [3
ગુણ]}\label{uxaaauxab0uxab6uxaa8-3uxa85-or-3-uxa97uxaa3}

\textbf{final કીવર્ડ ઉદાહરણ સાથે સમજાવો.}

\begin{solutionbox}

\textbf{Final Keyword}: modification, inheritance, અથવા overriding
પ્રતિબંધિત કરે છે.

\begin{verbatim}
final class Math \{           // inherit કરી શકાતું નથી
    final int PI = 3.14;     // modify કરી શકાતું નથી
    
    final void calculate() \{ // override કરી શકાતું નથી
        System.out.println("ગણતરી કરી રહ્યું છું");
    \}
\}
\end{verbatim}

\textbf{Final Uses}:

\begin{itemize}
\tightlist
\item
  \textbf{Class}: extend કરી શકાતું નથી
\item
  \textbf{Method}: override કરી શકાતું નથી
\item
  \textbf{Variable}: reassign કરી શકાતું નથી
\end{itemize}

\end{solutionbox}
\begin{mnemonicbox}
``ફાઇનલ ફેરફાર અટકાવે છે''

\end{mnemonicbox}
\subsection*{પ્રશ્ન 3(બ OR) [4
ગુણ]}\label{uxaaauxab0uxab6uxaa8-3uxaac-or-4-uxa97uxaa3}

\textbf{જાવામાં વિવિધ એક્સેસ કંટ્રોલ સમજાવો.}

\begin{solutionbox}

\textbf{Access Modifiers}:

{\def\LTcaptype{none} % do not increment counter
\begin{longtable}[]{@{}
  >{\raggedright\arraybackslash}p{(\linewidth - 8\tabcolsep) * \real{0.1842}}
  >{\raggedright\arraybackslash}p{(\linewidth - 8\tabcolsep) * \real{0.1974}}
  >{\raggedright\arraybackslash}p{(\linewidth - 8\tabcolsep) * \real{0.1974}}
  >{\raggedright\arraybackslash}p{(\linewidth - 8\tabcolsep) * \real{0.1711}}
  >{\raggedright\arraybackslash}p{(\linewidth - 8\tabcolsep) * \real{0.2500}}@{}}
\toprule\noalign{}
\begin{minipage}[b]{\linewidth}\raggedright
\textbf{Modifier}
\end{minipage} & \begin{minipage}[b]{\linewidth}\raggedright
\textbf{સેમ ક્લાસ}
\end{minipage} & \begin{minipage}[b]{\linewidth}\raggedright
\textbf{સેમ પેકેજ}
\end{minipage} & \begin{minipage}[b]{\linewidth}\raggedright
\textbf{સબક્લાસ}
\end{minipage} & \begin{minipage}[b]{\linewidth}\raggedright
\textbf{ડિફરન્ટ પેકેજ}
\end{minipage} \\
\midrule\noalign{}
\endhead
\bottomrule\noalign{}
\endlastfoot
public & ✓ & ✓ & ✓ & ✓ \\
protected & ✓ & ✓ & ✓ & ✗ \\
default & ✓ & ✓ & ✗ & ✗ \\
private & ✓ & ✗ & ✗ & ✗ \\
\end{longtable}
}

\end{solutionbox}
\begin{mnemonicbox}
``પબ્લિક પ્રોટેક્ટેડ ડિફોલ્ટ પ્રાઇવેટ''

\end{mnemonicbox}
\subsection*{પ્રશ્ન 3(ક OR) [7
ગુણ]}\label{uxaaauxab0uxab6uxaa8-3uxa95-or-7-uxa97uxaa3}

\textbf{પેકેજ શું છે? પેકેજ બનાવવાના પગલાં લખો અને તેનું ઉદાહરણ આપો.}

\begin{solutionbox}

\textbf{Package}: સંબંધિત ક્લાસ અને interfaces નું જૂથ.

\textbf{પેકેજ બનાવવાના પગલાં}:

\begin{enumerate}
\tightlist
\item
  \textbf{Declare}: ટોપ પર package statement વાપરો
\item
  \textbf{Compile}: javac -d . ClassName.java\\
\item
  \textbf{Run}: java packagename.ClassName
\end{enumerate}

\begin{verbatim}
// File: mypack/Calculator.java
package mypack;

public class Calculator \{
    public int add(int a, int b) \{
        return a + b;
    \}
\}

// File: Test.java
import mypack.Calculator;

public class Test \{
    public static void main(String[] args) \{
        Calculator calc = new Calculator();
        System.out.println(calc.add(5, 3));
    \}
\}
\end{verbatim}

\textbf{Package Benefits}:

\begin{itemize}
\tightlist
\item
  \textbf{Organization}: સંબંધિત ક્લાસોને જૂથ કરે છે
\item
  \textbf{Access Control}: પેકેજ-લેવલ પ્રોટેક્શન
\item
  \textbf{Namespace}: નામિંગ કન્ફ્લિક્ટ ટાળે છે
\end{itemize}

\end{solutionbox}
\begin{mnemonicbox}
``ડિક્લેર કમ્પાઇલ રન''

\end{mnemonicbox}
\subsection*{પ્રશ્ન 4(અ) [3
ગુણ]}\label{uxaaauxab0uxab6uxaa8-4uxa85-3-uxa97uxaa3}

\textbf{યોગ્ય ઉદાહરણ સાથે thread ની પ્રાથમિકતાઓ સમજાવો.}

\begin{solutionbox}

\textbf{Thread Priority}: thread execution order નક્કી કરે છે (1-10 સ્કેલ).

\begin{verbatim}
class MyThread extends Thread \{
    public void run() \{
        System.out.println(getName() + " પ્રાથમિકતા: " + getPriority());
    \}
\}

public class ThreadPriorityExample \{
    public static void main(String[] args) \{
        MyThread t1 = new MyThread();
        MyThread t2 = new MyThread();
        
        t1.setPriority(Thread.MIN\_PRIORITY);  // 1
        t2.setPriority(Thread.MAX\_PRIORITY);  // 10
        
        t1.start();
        t2.start();
    \}
\}
\end{verbatim}

\textbf{Priority Constants}:

\begin{itemize}
\tightlist
\item
  \textbf{MIN\_PRIORITY}: 1
\item
  \textbf{NORM\_PRIORITY}: 5\\
\item
  \textbf{MAX\_PRIORITY}: 10
\end{itemize}

\end{solutionbox}
\begin{mnemonicbox}
``મિન નોર્મલ મેક્સ''

\end{mnemonicbox}
\subsection*{પ્રશ્ન 4(બ) [4
ગુણ]}\label{uxaaauxab0uxab6uxaa8-4uxaac-4-uxa97uxaa3}

\textbf{થ્રેડ શું છે? થ્રેડ જીવન ચક્ર સમજાવો.}

\begin{solutionbox}

\textbf{Thread}: concurrent execution માટે lightweight process.

\begin{verbatim}
stateDiagram{-v2}
        direction LR
    [*] {-{-} New}
    New {-{-} Runnable: start()}
    Runnable {-{-} Running: CPU allocation}
    Running {-{-} Blocked: wait(), sleep()}
    Blocked {-{-} Runnable: notify(), timeout}
    Running {-{-} Dead: complete}
\end{verbatim}

\textbf{Thread States}:

\begin{itemize}
\tightlist
\item
  \textbf{New}: Thread બનાવ્યું પણ શરૂ થયું નથી
\item
  \textbf{Runnable}: ચાલવા માટે તૈયાર
\item
  \textbf{Running}: હાલમાં execute થઈ રહ્યું છે
\item
  \textbf{Blocked}: resource માટે રાહ જોઈ રહ્યું છે
\item
  \textbf{Dead}: execution પૂર્ણ થયું
\end{itemize}

\end{solutionbox}
\begin{mnemonicbox}
``ન્યૂ રનેબલ રનિંગ બ્લોક્ડ ડેડ''

\end{mnemonicbox}
\subsection*{પ્રશ્ન 4(ક) [7
ગુણ]}\label{uxaaauxab0uxab6uxaa8-4uxa95-7-uxa97uxaa3}

\textbf{જાવામાં એક પ્રોગ્રામ લખો જે રનેબલ ઇન્ટરફેસનો અમલ કરીને બહુવિધ થ્રેડો બનાવે
છે.}

\begin{solutionbox}

\begin{verbatim}
class MyRunnable implements Runnable \{
    private String threadName;
    
    MyRunnable(String name) \{
        threadName = name;
    \}
    
    public void run() \{
for(int

i = 1; i {=} 5; i++) \{

            System.out.println(threadName + " {- ગણતરી: "} + i);
            try \{
                Thread.sleep(1000);
            \} catch(InterruptedException e) \{
                e.printStackTrace();
            \}
        \}
    \}
\}

public class MultipleThreads \{
    public static void main(String[] args) \{
        Thread t1 = new Thread(new MyRunnable("થ્રેડ{-1"}));
        Thread t2 = new Thread(new MyRunnable("થ્રેડ{-2"}));
        Thread t3 = new Thread(new MyRunnable("થ્રેડ{-3"}));
        
        t1.start();
        t2.start(); 
        t3.start();
    \}
\}
\end{verbatim}

\textbf{મુખ્ય મુદ્દાઓ}:

\begin{itemize}
\tightlist
\item
  \textbf{Runnable Interface}: Thread ક્લાસ extend કરવા કરતાં સારું છે
\item
  \textbf{Thread.sleep()}: thread execution pause કરે છે
\item
  \textbf{Multiple Threads}: એકસાથે concurrent ચાલે છે
\end{itemize}

\end{solutionbox}
\begin{mnemonicbox}
``ઇમ્પ્લિમેન્ટ રનેબલ સ્ટાર્ટ મલ્ટિપલ''

\end{mnemonicbox}
\subsection*{પ્રશ્ન 4(અ OR) [3
ગુણ]}\label{uxaaauxab0uxab6uxaa8-4uxa85-or-3-uxa97uxaa3}

\textbf{ચાર અલગ-અલગ ઇનબિલ્ટ exception ની યાદી આપો. કોઈપણ એક ઇનબિલ્ટ
exception સમજાવો.}

\begin{solutionbox}

\textbf{Inbuilt Exceptions}:

\begin{itemize}
\tightlist
\item
  \textbf{NullPointerException}: null object એક્સેસ કરવું
\item
  \textbf{ArrayIndexOutOfBoundsException}: અમાન્ય array index
\item
  \textbf{ArithmeticException}: શૂન્યથી ભાગાકાર
\item
  \textbf{NumberFormatException}: અમાન્ય સંખ્યા ફોર્મેટ
\end{itemize}

\textbf{ArithmeticException}: arithmetic operation નિષ્ફળ થાય ત્યારે throw
થાય છે.

\begin{verbatim}
int result = 10 / 0; // ArithmeticException throw કરે છે
\end{verbatim}

\end{solutionbox}
\begin{mnemonicbox}
``નલ એરે એરિથમેટિક નંબર''

\end{mnemonicbox}
\subsection*{પ્રશ્ન 4(બ OR) [4
ગુણ]}\label{uxaaauxab0uxab6uxaa8-4uxaac-or-4-uxa97uxaa3}

\textbf{યોગ્ય ઉદાહરણ સાથે ટ્રાય અને કેચ સમજાવો.}

\begin{solutionbox}

\textbf{Try-Catch}: Exception handling મેકેનિઝમ.

\begin{verbatim}
public class TryCatchExample \{
    public static void main(String[] args) \{
        try \{
            int[] arr = \{1, 2, 3\;}
            System.out.println(arr[5]); // Index out of bounds
        \}
        catch(ArrayIndexOutOfBoundsException e) \{
            System.out.println("Array index એરર: " + e.getMessage());
        \}
        finally \{
            System.out.println("હંમેશા execute થાય છે");
        \}
    \}
\}
\end{verbatim}

\textbf{Exception Handling Flow}:

\begin{itemize}
\tightlist
\item
  \textbf{Try}: કોડ જે exception throw કરી શકે છે
\item
  \textbf{Catch}: વિશિષ્ટ exceptions handle કરે છે\\
\item
  \textbf{Finally}: હંમેશા execute થાય છે
\end{itemize}

\end{solutionbox}
\begin{mnemonicbox}
``ટ્રાય કેચ ફાઇનલી''

\end{mnemonicbox}
\subsection*{પ્રશ્ન 4(ક OR) [7
ગુણ]}\label{uxaaauxab0uxab6uxaa8-4uxa95-or-7-uxa97uxaa3}

\textbf{Exception શું છે? Arithmetic Exception નો ઉપયોગ દશાવતો પ્રોગ્રામ
લખો.}

\begin{solutionbox}

\textbf{Exception}: runtime error જે સામાન્ય પ્રોગ્રામ flow ને વિક્ષેપ કરે છે.

\begin{verbatim}
import java.util.Scanner;

public class ArithmeticExceptionExample \{
    public static void main(String[] args) \{
        Scanner sc = new Scanner(System.in);
        
        try \{
            System.out.print("પ્રથમ સંખ્યા દાખલ કરો: ");
            int num1 = sc.nextInt();
            
            System.out.print("બીજી સંખ્યા દાખલ કરો: ");
            int num2 = sc.nextInt();
            
            int result = num1 / num2;
            System.out.println("પરિણામ: " + result);
        \}
        catch(ArithmeticException e) \{
            System.out.println("એરર: શૂન્યથી ભાગાકાર કરી શકાતો નથી!");
        \}
        catch(Exception e) \{
            System.out.println("સામાન્ય એરર: " + e.getMessage());
        \}
        finally \{
            sc.close();
        \}
    \}
\}
\end{verbatim}

\textbf{Exception Types}:

\begin{itemize}
\tightlist
\item
  \textbf{Checked}: કમ્પાઇલ-ટાઇમ exceptions
\item
  \textbf{Unchecked}: રનટાઇમ exceptions
\item
  \textbf{Error}: સિસ્ટમ-લેવલ પ્રોબ્લેમ્સ
\end{itemize}

\end{solutionbox}
\begin{mnemonicbox}
``રનટાઇમ એરર ફ્લો ડિસરપ્ટ કરે છે''

\end{mnemonicbox}
\subsection*{પ્રશ્ન 5(અ) [3
ગુણ]}\label{uxaaauxab0uxab6uxaa8-5uxa85-3-uxa97uxaa3}

\textbf{JavaમાંArrayIndexOutOfBound અપવાદને ઉદાહરણ સાથે સમજાવો.}

\begin{solutionbox}

\textbf{ArrayIndexOutOfBoundsException}: અમાન્ય array index એક્સેસ કરતી વખતે
throw થાય છે.

\begin{verbatim}
public class ArrayIndexExample \{
    public static void main(String[] args) \{
        int[] numbers = \{10, 20, 30\;}
        
        try \{
            System.out.println(numbers[5]); // અમાન્ય index
        \}
        catch(ArrayIndexOutOfBoundsException e) \{
            System.out.println("અમાન્ય array index: " + e.getMessage());
        \}
    \}
\}
\end{verbatim}

\textbf{મુખ્ય મુદ્દાઓ}:

\begin{itemize}
\tightlist
\item
  \textbf{Valid Range}: 0 થી array.length-1
\item
  \textbf{Negative Index}: નકારાત્મક index પણ exception throw કરે છે
\item
  \textbf{Runtime Exception}: unchecked exception
\end{itemize}

\end{solutionbox}
\begin{mnemonicbox}
``એરે ઇન્ડેક્સ રેન્જ ચેક''

\end{mnemonicbox}
\subsection*{પ્રશ્ન 5(બ) [4
ગુણ]}\label{uxaaauxab0uxab6uxaa8-5uxaac-4-uxa97uxaa3}

\textbf{stream classes ની મૂળભૂત બાબતો સમજાવો.}

\begin{solutionbox}

\textbf{Stream Classes}: input/output operations handle કરે છે.

{\def\LTcaptype{none} % do not increment counter
\begin{longtable}[]{@{}ll@{}}
\toprule\noalign{}
\textbf{Stream Type} & \textbf{Classes} \\
\midrule\noalign{}
\endhead
\bottomrule\noalign{}
\endlastfoot
Byte Streams & InputStream, OutputStream \\
Character Streams & Reader, Writer \\
File Streams & FileInputStream, FileOutputStream \\
Buffered Streams & BufferedReader, BufferedWriter \\
\end{longtable}
}

\begin{center}
\textbf{Mermaid Diagram (Code)}
\begin{verbatim}
{Shaded}
{Highlighting}[]
graph TD
    A[Stream Classes] {-{-}{} B[Byte Streams]}
    A {-{-}{} C[Character Streams]}
    B {-{-}{} D[InputStream]}
    B {-{-}{} E[OutputStream]}
    C {-{-}{} F[Reader]}
    C {-{-}{} G[Writer]}
{Highlighting}
{Shaded}
\end{verbatim}
\end{center}

\textbf{Stream Features}:

\begin{itemize}
\tightlist
\item
  \textbf{Sequential}: ડેટા sequence માં flow કરે છે
\item
  \textbf{One Direction}: કાં તો input કાં output
\item
  \textbf{Automatic}: નીચલા સ્તરની વિગતો handle કરે છે
\end{itemize}

\end{solutionbox}
\begin{mnemonicbox}
``બાઇટ કેરેક્ટર ફાઇલ બફર્ડ''

\end{mnemonicbox}
\subsection*{પ્રશ્ન 5(ક) [7
ગુણ]}\label{uxaaauxab0uxab6uxaa8-5uxa95-7-uxa97uxaa3}

\textbf{ટેક્સ્ટ ફાઇલ બનાવવા માટે જાવા પ્રોગ્રામ લખો અને ટેક્સ્ટ ફાઇલ પર રીડ ઑપરેશન
કરો.}

\begin{solutionbox}

\begin{verbatim}
import java.io.*;

public class FileReadExample \{
    public static void main(String[] args) \{
        // ફાઇલ બનાવો અને લખો
        try \{
            FileWriter writer = new FileWriter("sample.txt");
            writer.write("હેલો વર્લ્ડ!{n}");
            writer.write("જાવા ફાઇલ હેન્ડલિંગ{n}");
            writer.write("GTU પરીક્ષા 2024");
            writer.close();
            System.out.println("ફાઇલ સફળતાપૂર્વક બનાવાઈ");
        \}
        catch(IOException e) \{
            System.out.println("ફાઇલ બનાવવામાં એરર: " + e.getMessage());
        \}
        
        // ફાઇલમાંથી વાંચો
        try \{
            BufferedReader reader = new BufferedReader(new FileReader("sample.txt"));
            String line;
            
            System.out.println("{n}ફાઇલની વિગતો:");
            while((line = reader.readLine()) != null) \{
                System.out.println(line);
            \}
            reader.close();
        \}
        catch(IOException e) \{
            System.out.println("ફાઇલ વાંચવામાં એરર: " + e.getMessage());
        \}
    \}
\}
\end{verbatim}

\textbf{મુખ્ય મુદ્દાઓ}:

\begin{itemize}
\tightlist
\item
  \textbf{FileWriter}: ફાઇલ બનાવે અને લખે છે
\item
  \textbf{BufferedReader}: કાર્યક્ષમ વાંચન
\item
  \textbf{Exception Handling}: IOException handle કરો
\end{itemize}

\end{solutionbox}
\begin{mnemonicbox}
``બનાવો લખો વાંચો બંધ કરો''

\end{mnemonicbox}
\subsection*{પ્રશ્ન 5(અ OR) [3
ગુણ]}\label{uxaaauxab0uxab6uxaa8-5uxa85-or-3-uxa97uxaa3}

\textbf{Java માં Divide by Zero Exception ને ઉદાહરણ સાથે સમજાવો.}

\begin{solutionbox}

\textbf{ArithmeticException}: શૂન્યથી ભાગાકાર ઑપરેશન દરમિયાન throw થાય છે.

\begin{verbatim}
public class DivideByZeroExample \{
    public static void main(String[] args) \{
        try \{
            int a = 10;
            int b = 0;
            int result = a / b;  // ArithmeticException throw કરે છે
            System.out.println("પરિણામ: " + result);
        \}
        catch(ArithmeticException e) \{
            System.out.println("શૂન્યથી ભાગાકાર કરી શકાતો નથી: " + e.getMessage());
        \}
    \}
\}
\end{verbatim}

\textbf{મુખ્ય મુદ્દાઓ}:

\begin{itemize}
\tightlist
\item
  \textbf{Integer Division}: માત્ર integer division by zero exception
  throw કરે છે
\item
  \textbf{Floating Point}: floating point division માટે Infinity return
  કરે છે
\item
  \textbf{Runtime Exception}: unchecked exception
\end{itemize}

\end{solutionbox}
\begin{mnemonicbox}
``શૂન્ય ભાગાકાર એરિથમેટિક એરર''

\end{mnemonicbox}
\subsection*{પ્રશ્ન 5(બ OR) [4
ગુણ]}\label{uxaaauxab0uxab6uxaa8-5uxaac-or-4-uxa97uxaa3}

\textbf{java I/O પ્રક્રિયા સમજાવો.}

\begin{solutionbox}

\textbf{Java I/O Process}: ડેટા વાંચવા અને લખવાની પદ્ધતિ.

\begin{center}
\textbf{Mermaid Diagram (Code)}
\begin{verbatim}
{Shaded}
{Highlighting}[]
graph LR
    A[Data Source] {-{-}{} B[Input Stream]}
    B {-{-}{} C[Java Program]}
    C {-{-}{} D[Output Stream]}
    D {-{-}{} E[Data Destination]}
{Highlighting}
{Shaded}
\end{verbatim}
\end{center}

\textbf{I/O Components}:

\begin{itemize}
\tightlist
\item
  \textbf{Stream}: ડેટાનો ક્રમ
\item
  \textbf{Buffer}: કાર્યક્ષમતા માટે અસ્થાયી સ્ટોરેજ
\item
  \textbf{File}: સ્થાયી સ્ટોરેજ
\item
  \textbf{Network}: દૂરસ્થ ડેટા ટ્રાન્સફર
\end{itemize}

\textbf{I/O Types}:

\begin{itemize}
\tightlist
\item
  \textbf{Byte-oriented}: કાચો ડેટા (images, videos)
\item
  \textbf{Character-oriented}: ટેક્સ્ટ ડેટા
\item
  \textbf{Synchronous}: blocking operations
\item
  \textbf{Asynchronous}: non-blocking operations
\end{itemize}

\end{solutionbox}
\begin{mnemonicbox}
``સ્ટ્રીમ બફર ફાઇલ નેટવર્ક''

\end{mnemonicbox}
\subsection*{પ્રશ્ન 5(ક OR) [7
ગુણ]}\label{uxaaauxab0uxab6uxaa8-5uxa95-or-7-uxa97uxaa3}

\textbf{ટેક્સ્ટ ફાઇલ બનાવવા માટે જાવા પ્રોગ્રામ લખો અને ટેક્સ્ટ ફાઇલ પર રાઇટ ઑપરેશન
કરો.}

\begin{solutionbox}

\begin{verbatim}
import java.io.*;
import java.util.Scanner;

public class FileWriteExample \{
    public static void main(String[] args) \{
        Scanner sc = new Scanner(System.in);
        
        try \{
            // FileWriter સાથે ફાઇલ બનાવો
            FileWriter writer = new FileWriter("student.txt");
            
            System.out.println("વિદ્યાર્થીની વિગતો દાખલ કરો:");
            System.out.print("નામ: ");
            String name = sc.nextLine();
            
            System.out.print("રોલ નંબર: ");
            String rollNo = sc.nextLine();
            
            System.out.print("શાખા: ");
            String branch = sc.nextLine();
            
            // ફાઇલમાં ડેટા લખો
            writer.write("વિદ્યાર્થીની માહિતી{n}");
            writer.write("=================={n}");
            writer.write("નામ: " + name + "{n}");
            writer.write("રોલ નંબર: " + rollNo + "{n}");
            writer.write("શાખા: " + branch + "{n}");
            writer.write("તારીખ: " + new java.util.Date() + "{n}");
            
            writer.close();
            System.out.println("{n}ડેટા સફળતાપૂર્વક ફાઇલમાં લખાયો!");
            
        \}
        catch(IOException e) \{
            System.out.println("ફાઇલમાં લખવામાં એરર: " + e.getMessage());
        \}
        finally \{
            sc.close();
        \}
    \}
\}
\end{verbatim}

\textbf{મુખ્ય મુદ્દાઓ}:

\begin{itemize}
\tightlist
\item
  \textbf{FileWriter}: ફાઇલમાં character data લખે છે
\item
  \textbf{BufferedWriter}: મોટા ડેટા માટે વધુ કાર્યક્ષમ
\item
  \textbf{Auto-close}: automatic closing માટે try-with-resources વાપરો
\end{itemize}

\end{solutionbox}
\begin{mnemonicbox}
``બનાવો લખો બંધ કરો હેન્ડલ કરો''

\end{mnemonicbox}

\end{document}
