\documentclass[10pt,a4paper]{article}

% content/resources/templates/preamble.tex
\usepackage[margin=0.6in]{geometry}
\author{Milav Dabgar}
\usepackage{amsmath,amssymb,amsthm}
\usepackage{booktabs}
\usepackage{multirow}
\usepackage{xcolor}
\usepackage{tcolorbox}
\tcbuselibrary{breakable,skins}
\usepackage[colorlinks=true,linkcolor=blue]{hyperref}
\usepackage{titlesec}
\usepackage{enumitem}
\usepackage{tikz}
\usepackage{pgfplots}
\usepackage{circuitikz}
\usepackage[version=4]{mhchem}
\usepackage{longtable}
\usepackage{array}
\usepackage{float}
\usepackage{caption}
\usepackage{listings}

\lstset{
  basicstyle=\small\ttfamily,
  breaklines=true,
  breakatwhitespace=false,
  postbreak=\mbox{\textcolor{red}{$\hookrightarrow$}\space},
  float=false,
  numbers=left,
  numberstyle=\tiny\color{gray},
  numbersep=10pt,
  xleftmargin=2em,
  keywordstyle=\color{blue},
  commentstyle=\color{green!60!black},
  stringstyle=\color{purple},
  backgroundcolor=\color{gray!5},
  showstringspaces=false,
  tabsize=2,
  captionpos=b,
  keepspaces=true,
  columns=flexible
}

\pgfplotsset{compat=1.18}
\usetikzlibrary{shapes,arrows,positioning,calc,patterns,decorations.pathmorphing,decorations.markings,arrows.meta}

% Color scheme
\definecolor{headcolor}{RGB}{0,102,204}
\definecolor{keycolor}{RGB}{220,20,60}
\definecolor{solutioncolor}{RGB}{34,139,34}
\definecolor{mnemoniccolor}{RGB}{148,0,211}
\definecolor{codecolor}{RGB}{0,0,100}

% Spacing
\setlength{\parskip}{3pt}
\setlist[itemize]{nosep}
\setlist[enumerate]{nosep}

% Title formatting
\titleformat{\section}{\Large\bfseries\color{headcolor}}{\thesection}{1em}{}
\titleformat{\subsection}{\large\bfseries\color{headcolor}}{\thesubsection}{1em}{}

% Pandoc tightlist compatibility
\providecommand{\tightlist}{%
  \setlength{\itemsep}{0pt}\setlength{\parskip}{0pt}}

% Pandoc longtable compatibility
\newcounter{none}
\def\thenone{}


% content/resources/templates/english-boxes.tex
% This file is currently empty - it exists to maintain consistency with the import structure.
% Add custom environments here if needed in the future.


\begin{document}

\begin{center}
{\Huge\bfseries\color{headcolor} Subject Name Solutions}\\[5pt]
{\LARGE 4341602 -- Summer 2024}\\[3pt]
{\large Semester 1 Study Material}\\[3pt]
{\normalsize\textit{Detailed Solutions and Explanations}}
\end{center}

\vspace{10pt}

\subsection*{Question 1(a) [3 marks]}\label{q1a}

\textbf{Explain the basic structure of Java program.}

\begin{solutionbox}

\textbf{Basic Structure Table:}

{\def\LTcaptype{none} % do not increment counter
\begin{longtable}[]{@{}
  >{\raggedright\arraybackslash}p{(\linewidth - 2\tabcolsep) * \real{0.4583}}
  >{\raggedright\arraybackslash}p{(\linewidth - 2\tabcolsep) * \real{0.5417}}@{}}
\toprule\noalign{}
\begin{minipage}[b]{\linewidth}\raggedright
Component
\end{minipage} & \begin{minipage}[b]{\linewidth}\raggedright
Description
\end{minipage} \\
\midrule\noalign{}
\endhead
\bottomrule\noalign{}
\endlastfoot
\textbf{Package declaration} & Optional, defines package membership \\
\textbf{Import statements} & Imports required classes/packages \\
\textbf{Class declaration} & Defines the main class \\
\textbf{Main method} & Entry point: public static void main(String[]
args) \\
\end{longtable}
}

\textbf{Diagram:}

\begin{verbatim}
+{-{-}{-}{-}{-}{-}{-}{-}{-}{-}{-}{-}{-}{-}{-}{-}{-}{-}{-}{-}{-}{-}{-}{-}{-}+}
|    Package Declaration  |
+{-{-}{-}{-}{-}{-}{-}{-}{-}{-}{-}{-}{-}{-}{-}{-}{-}{-}{-}{-}{-}{-}{-}{-}{-}+}
|    Import Statements    |
+{-{-}{-}{-}{-}{-}{-}{-}{-}{-}{-}{-}{-}{-}{-}{-}{-}{-}{-}{-}{-}{-}{-}{-}{-}+}
|    Class Declaration    |
|  +{-{-}{-}{-}{-}{-}{-}{-}{-}{-}{-}{-}{-}{-}{-}{-}{-}{-}{-}+  |}
|  |   Variables       |  |
|  +{-{-}{-}{-}{-}{-}{-}{-}{-}{-}{-}{-}{-}{-}{-}{-}{-}{-}{-}+  |}
|  |   Methods         |  |
|  |  +{-{-}{-}{-}{-}{-}{-}{-}{-}{-}{-}{-}{-}+  |  |}
|  |  | main method |  |  |
|  |  +{-{-}{-}{-}{-}{-}{-}{-}{-}{-}{-}{-}{-}+  |  |}
|  +{-{-}{-}{-}{-}{-}{-}{-}{-}{-}{-}{-}{-}{-}{-}{-}{-}{-}{-}+  |}
+{-{-}{-}{-}{-}{-}{-}{-}{-}{-}{-}{-}{-}{-}{-}{-}{-}{-}{-}{-}{-}{-}{-}{-}{-}+}
\end{verbatim}

\begin{itemize}
\tightlist
\item
  \textbf{Package}: Groups related classes
\item
  \textbf{Import}: Access external classes
\item
  \textbf{Class}: Blueprint for objects
\item
  \textbf{Main method}: Program execution starts here
\end{itemize}

\end{solutionbox}
\begin{mnemonicbox}
``PICM - Package, Import, Class, Main''

\end{mnemonicbox}
\subsection*{Question 1(b) [4 marks]}\label{q1b}

\textbf{List out different features of java. Explain any two.}

\begin{solutionbox}

\textbf{Java Features Table:}

{\def\LTcaptype{none} % do not increment counter
\begin{longtable}[]{@{}ll@{}}
\toprule\noalign{}
Feature & Description \\
\midrule\noalign{}
\endhead
\bottomrule\noalign{}
\endlastfoot
\textbf{Platform Independent} & Write once, run anywhere \\
\textbf{Object Oriented} & Everything is an object \\
\textbf{Simple} & Easy syntax, no pointers \\
\textbf{Secure} & Built-in security features \\
\textbf{Robust} & Strong memory management \\
\textbf{Multithreaded} & Concurrent execution support \\
\end{longtable}
}

\textbf{Detailed Explanation:}

\textbf{Platform Independence:}

\begin{itemize}
\tightlist
\item
  Java code compiles to bytecode
\item
  JVM interprets bytecode on any platform
\item
  Same program runs on Windows, Linux, Mac
\end{itemize}

\textbf{Object Oriented:}

\begin{itemize}
\tightlist
\item
  Encapsulation: Data hiding in classes
\item
  Inheritance: Code reuse through extends
\item
  Polymorphism: Same method, different behavior
\end{itemize}

\end{solutionbox}
\begin{mnemonicbox}
``POSRMM - Platform, Object, Simple, Robust,
Multithreaded, Memory''

\end{mnemonicbox}
\subsection*{Question 1(c) [7 marks]}\label{q1c}

\textbf{Write a program in java to find out sum of the digits of entered
number. (Ex. Number is 123 output is 6).}

\begin{solutionbox}

\begin{verbatim}
public class DigitSum \{
    public static void main(String[] args) \{
        int number = Integer.parseInt(args[0]);
        int sum = 0;
        int temp = Math.abs(number);
        
        while (temp {} 0) \{
            sum += temp \% 10;
            temp /= 10;
        \}
        
        System.out.println("Sum of digits: " + sum);
    \}
\}
\end{verbatim}

\textbf{Algorithm Table:}

{\def\LTcaptype{none} % do not increment counter
\begin{longtable}[]{@{}lll@{}}
\toprule\noalign{}
Step & Operation & Example (123) \\
\midrule\noalign{}
\endhead
\bottomrule\noalign{}
\endlastfoot
1 & Extract last digit (n\%10) & 123\%10 = 3 \\
2 & Add to sum & sum = 0+3 = 3 \\
3 & Remove last digit (n/10) & 123/10 = 12 \\
4 & Repeat until n=0 & Continue \\
\end{longtable}
}

\begin{itemize}
\tightlist
\item
  \textbf{Input}: Command line argument
\item
  \textbf{Process}: Extract digits using modulo
\item
  \textbf{Output}: Sum of all digits
\end{itemize}

\end{solutionbox}
\begin{mnemonicbox}
``EARD - Extract, Add, Remove, Done''

\end{mnemonicbox}
\subsection*{Question 1(c OR) [7
marks]}\label{question-1c-or-7-marks}

\textbf{Write a program in java to find out maximum from any ten numbers
using command line argument.}

\begin{solutionbox}

\begin{verbatim}
public class FindMaximum \{
    public static void main(String[] args) \{
        if (args.length {} 10) \{
            System.out.println("Please enter 10 numbers");
            return;
        \}
        
        int max = Integer.parseInt(args[0]);
        
        for (int i = 1; i {} 10; i++) \{
            int current = Integer.parseInt(args[i]);
            if (current {} max) \{
                max = current;
            \}
        \}
        
        System.out.println("Maximum number: " + max);
    \}
\}
\end{verbatim}

\textbf{Process Table:}

{\def\LTcaptype{none} % do not increment counter
\begin{longtable}[]{@{}lll@{}}
\toprule\noalign{}
Step & Action & Details \\
\midrule\noalign{}
\endhead
\bottomrule\noalign{}
\endlastfoot
1 & \textbf{Check args} & Ensure 10 numbers provided \\
2 & \textbf{Initialize max} & First number as initial max \\
3 & \textbf{Compare loop} & Check each remaining number \\
4 & \textbf{Update max} & If current \textgreater{} max, update \\
\end{longtable}
}

\begin{itemize}
\tightlist
\item
  \textbf{Validation}: Check argument count
\item
  \textbf{Comparison}: Standard maximum finding
\item
  \textbf{Output}: Display the largest number
\end{itemize}

\end{solutionbox}
\begin{mnemonicbox}
``VCIU - Validate, Compare, Initialize, Update''

\end{mnemonicbox}
\subsection*{Question 2(a) [3 marks]}\label{q2a}

\textbf{List out different concept of oop. Explain anyone in detail.}

\begin{solutionbox}

\textbf{OOP Concepts Table:}

{\def\LTcaptype{none} % do not increment counter
\begin{longtable}[]{@{}ll@{}}
\toprule\noalign{}
Concept & Description \\
\midrule\noalign{}
\endhead
\bottomrule\noalign{}
\endlastfoot
\textbf{Encapsulation} & Data hiding and bundling \\
\textbf{Inheritance} & Code reuse from parent class \\
\textbf{Polymorphism} & One interface, many forms \\
\textbf{Abstraction} & Hiding implementation details \\
\end{longtable}
}

\textbf{Encapsulation Details:}

\begin{itemize}
\tightlist
\item
  Combines data and methods in single unit
\item
  Uses private access modifiers for data
\item
  Provides public getter/setter methods
\item
  Protects data from unauthorized access
\end{itemize}

\textbf{Benefits:}

\begin{itemize}
\tightlist
\item
  \textbf{Security}: Data protection
\item
  \textbf{Maintenance}: Easy code updates
\item
  \textbf{Flexibility}: Change implementation easily
\end{itemize}

\end{solutionbox}
\begin{mnemonicbox}
``EIPA - Encapsulation, Inheritance, Polymorphism,
Abstraction''

\end{mnemonicbox}
\subsection*{Question 2(b) [4 marks]}\label{q2b}

\textbf{Explain JVM in detail.}

\begin{solutionbox}

\textbf{JVM Architecture Diagram:}

\begin{center}
\textbf{Mermaid Diagram (Code)}
\begin{verbatim}
{Shaded}
{Highlighting}[]
graph LR
    A[Java Source Code] {-{-}{} B[Java Compiler javac]}
    B {-{-}{} C[Bytecode .class]}
    C {-{-}{} D[JVM]}
    D {-{-}{} E[Class Loader]}
    D {-{-}{} F[Memory Areas]}
    D {-{-}{} G[Execution Engine]}
    G {-{-}{} H[Native OS]}
{Highlighting}
{Shaded}
\end{verbatim}
\end{center}

\textbf{JVM Components Table:}

{\def\LTcaptype{none} % do not increment counter
\begin{longtable}[]{@{}ll@{}}
\toprule\noalign{}
Component & Function \\
\midrule\noalign{}
\endhead
\bottomrule\noalign{}
\endlastfoot
\textbf{Class Loader} & Loads .class files into memory \\
\textbf{Memory Areas} & Heap, Stack, Method area \\
\textbf{Execution Engine} & Executes bytecode \\
\textbf{JIT Compiler} & Optimizes frequently used code \\
\end{longtable}
}

\begin{itemize}
\tightlist
\item
  \textbf{Platform Independence}: Same bytecode runs everywhere
\item
  \textbf{Memory Management}: Automatic garbage collection
\item
  \textbf{Security}: Bytecode verification before execution
\end{itemize}

\end{solutionbox}
\begin{mnemonicbox}
``CEMJ - Class loader, Execution, Memory, JIT''

\end{mnemonicbox}
\subsection*{Question 2(c) [7 marks]}\label{q2c}

\textbf{Explain constructor overloading with example.}

\begin{solutionbox}

\begin{verbatim}
public class Student \{
    private String name;
    private int age;
    private String course;
    
    // Default constructor
    public Student() \{
        this.name = "Unknown";
        this.age = 0;
        this.course = "Not Assigned";
    \}
    
    // Constructor with name
    public Student(String name) \{
        this.name = name;
        this.age = 0;
        this.course = "Not Assigned";
    \}
    
    // Constructor with name and age
    public Student(String name, int age) \{
        this.name = name;
        this.age = age;
        this.course = "Not Assigned";
    \}
    
    // Constructor with all parameters
    public Student(String name, int age, String course) \{
        this.name = name;
        this.age = age;
        this.course = course;
    \}
\}
\end{verbatim}

\textbf{Constructor Types Table:}

{\def\LTcaptype{none} % do not increment counter
\begin{longtable}[]{@{}lll@{}}
\toprule\noalign{}
Constructor & Parameters & Use Case \\
\midrule\noalign{}
\endhead
\bottomrule\noalign{}
\endlastfoot
\textbf{Default} & None & Basic object creation \\
\textbf{Single param} & Name only & Partial initialization \\
\textbf{Two param} & Name, Age & More specific data \\
\textbf{Full param} & All fields & Complete initialization \\
\end{longtable}
}

\begin{itemize}
\tightlist
\item
  \textbf{Same name}: All constructors have class name
\item
  \textbf{Different parameters}: Number or type varies
\item
  \textbf{Compile-time}: Decision made during compilation
\end{itemize}

\end{solutionbox}
\begin{mnemonicbox}
``SNDF - Same Name, Different Parameters, Flexible''

\end{mnemonicbox}
\subsection*{Question 2(a OR) [3
marks]}\label{question-2a-or-3-marks}

\textbf{What is wrapper class? Explain with example.}

\begin{solutionbox}

\textbf{Wrapper Classes Table:}

{\def\LTcaptype{none} % do not increment counter
\begin{longtable}[]{@{}ll@{}}
\toprule\noalign{}
Primitive & Wrapper Class \\
\midrule\noalign{}
\endhead
\bottomrule\noalign{}
\endlastfoot
\textbf{byte} & Byte \\
\textbf{int} & Integer \\
\textbf{float} & Float \\
\textbf{double} & Double \\
\textbf{char} & Character \\
\textbf{boolean} & Boolean \\
\end{longtable}
}

\textbf{Example:}

\begin{verbatim}
// Boxing {- primitive to object}
int num = 10;
Integer obj = Integer.valueOf(num);

// Unboxing {- object to primitive}
Integer wrapper = new Integer(20);
int value = wrapper.intValue();

// Auto{-boxing (Java 5+)}
Integer auto = 30;
int autoValue = auto;
\end{verbatim}

\begin{itemize}
\tightlist
\item
  \textbf{Boxing}: Convert primitive to wrapper object
\item
  \textbf{Unboxing}: Extract primitive from wrapper
\item
  \textbf{Collections}: Only objects allowed in collections
\end{itemize}

\end{solutionbox}
\begin{mnemonicbox}
``BUC - Boxing, Unboxing, Collections''

\end{mnemonicbox}
\subsection*{Question 2(b OR) [4
marks]}\label{question-2b-or-4-marks}

\textbf{Explain static keyword with example.}

\begin{solutionbox}

\begin{verbatim}
public class Counter \{
    private static int count = 0;  // Static variable
    private int id;                // Instance variable
    
    public Counter() \{
        count++;                   // Increment static count
        this.id = count;
    \}
    
    public static void showCount() \{  // Static method
        System.out.println("Total objects: " + count);
    \}
    
    public void showId() \{         // Instance method
        System.out.println("Object ID: " + id);
    \}
\}
\end{verbatim}

\textbf{Static Features Table:}

{\def\LTcaptype{none} % do not increment counter
\begin{longtable}[]{@{}ll@{}}
\toprule\noalign{}
Feature & Characteristics \\
\midrule\noalign{}
\endhead
\bottomrule\noalign{}
\endlastfoot
\textbf{Static Variable} & Shared among all instances \\
\textbf{Static Method} & Called without object creation \\
\textbf{Static Block} & Executed once when class loads \\
\textbf{Memory} & Stored in method area \\
\end{longtable}
}

\begin{itemize}
\tightlist
\item
  \textbf{Class level}: Belongs to class, not instance
\item
  \textbf{Memory efficiency}: Single copy for all objects
\item
  \textbf{Access}: Use class name to access
\end{itemize}

\end{solutionbox}
\begin{mnemonicbox}
``SCMA - Shared, Class-level, Memory, Access''

\end{mnemonicbox}
\subsection*{Question 2(c OR) [7
marks]}\label{question-2c-or-7-marks}

\textbf{What is constructor? Explain copy constructor with example.}

\begin{solutionbox}

\textbf{Constructor Definition:} Constructor is a special method that
initializes objects when they are created.

\begin{verbatim}
public class Book \{
    private String title;
    private String author;
    private int pages;
    
    // Default constructor
    public Book() \{
        this.title = "Unknown";
        this.author = "Unknown";
        this.pages = 0;
    \}
    
    // Parameterized constructor
    public Book(String title, String author, int pages) \{
        this.title = title;
        this.author = author;
        this.pages = pages;
    \}
    
    // Copy constructor
    public Book(Book other) \{
        this.title = other.title;
        this.author = other.author;
        this.pages = other.pages;
    \}
    
    public void display() \{
        System.out.println(title + " by " + author + 
                          " (" + pages + " pages)");
    \}
\}

// Usage
Book original = new Book("Java Guide", "James", 500);
Book copy = new Book(original);  // Copy constructor
\end{verbatim}

\textbf{Constructor Types Table:}

{\def\LTcaptype{none} % do not increment counter
\begin{longtable}[]{@{}lll@{}}
\toprule\noalign{}
Type & Purpose & Parameters \\
\midrule\noalign{}
\endhead
\bottomrule\noalign{}
\endlastfoot
\textbf{Default} & Basic initialization & None \\
\textbf{Parameterized} & Custom initialization & User-defined \\
\textbf{Copy} & Clone existing object & Same class object \\
\end{longtable}
}

\begin{itemize}
\tightlist
\item
  \textbf{Same name}: Constructor name = class name
\item
  \textbf{No return type}: Not even void
\item
  \textbf{Automatic call}: Called when object created
\end{itemize}

\end{solutionbox}
\begin{mnemonicbox}
``SNAC - Same Name, Automatic Call''

\end{mnemonicbox}
\subsection*{Question 3(a) [3 marks]}\label{q3a}

\textbf{Explain any four-string function in java with example.}

\begin{solutionbox}

\textbf{String Functions Table:}

{\def\LTcaptype{none} % do not increment counter
\begin{longtable}[]{@{}
  >{\raggedright\arraybackslash}p{(\linewidth - 4\tabcolsep) * \real{0.3571}}
  >{\raggedright\arraybackslash}p{(\linewidth - 4\tabcolsep) * \real{0.3214}}
  >{\raggedright\arraybackslash}p{(\linewidth - 4\tabcolsep) * \real{0.3214}}@{}}
\toprule\noalign{}
\begin{minipage}[b]{\linewidth}\raggedright
Function
\end{minipage} & \begin{minipage}[b]{\linewidth}\raggedright
Purpose
\end{minipage} & \begin{minipage}[b]{\linewidth}\raggedright
Example
\end{minipage} \\
\midrule\noalign{}
\endhead
\bottomrule\noalign{}
\endlastfoot
\textbf{length()} & Returns string length & ``Hello''.length() \rightarrow 5 \\
\textbf{charAt(index)} & Character at position & ``Java''.charAt(1) \rightarrow
`a' \\
\textbf{substring(start)} & Extract portion & ``Program''.substring(3) \rightarrow
``gram'' \\
\textbf{toUpperCase()} & Convert to uppercase & ``java''.toUpperCase() \rightarrow
``JAVA'' \\
\end{longtable}
}

\textbf{Code Example:}

\begin{verbatim}
String str = "Java Programming";

int len = str.length();           // 16
char ch = str.charAt(0);          // {J}
String sub = str.substring(5);    // "Programming"
String upper = str.toUpperCase(); // "JAVA PROGRAMMING"
\end{verbatim}

\begin{itemize}
\tightlist
\item
  \textbf{Immutable}: String objects cannot be changed
\item
  \textbf{Return new}: Methods return new string objects
\item
  \textbf{Zero-indexed}: Position counting starts from 0
\end{itemize}

\end{solutionbox}
\begin{mnemonicbox}
``LCST - Length, Character, Substring, Transform''

\end{mnemonicbox}
\subsection*{Question 3(b) [4 marks]}\label{q3b}

\textbf{List out different types of inheritance. Explain multilevel
inheritance.}

\begin{solutionbox}

\textbf{Inheritance Types Table:}

{\def\LTcaptype{none} % do not increment counter
\begin{longtable}[]{@{}ll@{}}
\toprule\noalign{}
Type & Description \\
\midrule\noalign{}
\endhead
\bottomrule\noalign{}
\endlastfoot
\textbf{Single} & One parent, one child \\
\textbf{Multilevel} & Chain of inheritance \\
\textbf{Hierarchical} & One parent, multiple children \\
\textbf{Multiple} & Multiple parents (via interfaces) \\
\end{longtable}
}

\textbf{Multilevel Inheritance Diagram:}

\begin{center}
\textbf{Mermaid Diagram (Code)}
\begin{verbatim}
{Shaded}
{Highlighting}[]
graph LR
    A[Vehicle] {-{-}{} B[Car]}
    B {-{-}{} C[SportsCar]}
{Highlighting}
{Shaded}
\end{verbatim}
\end{center}

\textbf{Example:}

\begin{verbatim}
class Vehicle \{
    protected String brand;
    public void start() \{
        System.out.println("Vehicle started");
    \}
\}

class Car extends Vehicle \{
    protected int doors;
    public void drive() \{
        System.out.println("Car is driving");
    \}
\}

class SportsCar extends Car \{
    private int maxSpeed;
    public void race() \{
        System.out.println("Sports car racing");
    \}
\}
\end{verbatim}

\begin{itemize}
\tightlist
\item
  \textbf{Chain inheritance}: Grandparent \rightarrow Parent \rightarrow Child
\item
  \textbf{Feature accumulation}: Child gets all ancestor features
\item
  \textbf{Method access}: Can call methods from all levels
\end{itemize}

\end{solutionbox}
\begin{mnemonicbox}
``SMHM - Single, Multilevel, Hierarchical, Multiple''

\end{mnemonicbox}
\subsection*{Question 3(c) [7 marks]}\label{q3c}

\textbf{What is interface? Explain multiple inheritance with example.}

\begin{solutionbox}

\textbf{Interface Definition:} Interface is a contract that defines what
methods a class must implement, without providing implementation.

\begin{verbatim}
interface Flyable \{
    void fly();
    void land();
\}

interface Swimmable \{
    void swim();
    void dive();
\}

// Multiple inheritance through interfaces
class Duck implements Flyable, Swimmable \{
    public void fly() \{
        System.out.println("Duck is flying");
    \}
    
    public void land() \{
        System.out.println("Duck landed on ground");
    \}
    
    public void swim() \{
        System.out.println("Duck is swimming");
    \}
    
    public void dive() \{
        System.out.println("Duck dived underwater");
    \}
\}
\end{verbatim}

\textbf{Interface vs Class Table:}

{\def\LTcaptype{none} % do not increment counter
\begin{longtable}[]{@{}lll@{}}
\toprule\noalign{}
Feature & Interface & Class \\
\midrule\noalign{}
\endhead
\bottomrule\noalign{}
\endlastfoot
\textbf{Methods} & Abstract (default/static allowed) & Concrete \\
\textbf{Variables} & public static final & Any type \\
\textbf{Inheritance} & Multiple allowed & Single only \\
\textbf{Instantiation} & Cannot create objects & Can create objects \\
\end{longtable}
}

\textbf{Multiple Inheritance Diagram:}

\begin{center}
\textbf{Mermaid Diagram (Code)}
\begin{verbatim}
{Shaded}
{Highlighting}[]
graph TD
    A[Flyable] {-{-}{} C[Duck]}
    B[Swimmable] {-{-}{} C[Duck]}
{Highlighting}
{Shaded}
\end{verbatim}
\end{center}

\begin{itemize}
\tightlist
\item
  \textbf{Contract}: Defines what, not how
\item
  \textbf{Multiple implementation}: One class, many interfaces
\item
  \textbf{Diamond problem solution}: Interfaces solve multiple
  inheritance issues
\end{itemize}

\end{solutionbox}
\begin{mnemonicbox}
``CMDS - Contract, Multiple, Diamond-solution''

\end{mnemonicbox}
\subsection*{Question 3(a OR) [3
marks]}\label{question-3a-or-3-marks}

\textbf{Explain this keyword with example.}

\begin{solutionbox}

\textbf{`this' Keyword Uses Table:}

{\def\LTcaptype{none} % do not increment counter
\begin{longtable}[]{@{}ll@{}}
\toprule\noalign{}
Use Case & Purpose \\
\midrule\noalign{}
\endhead
\bottomrule\noalign{}
\endlastfoot
\textbf{Instance variable} & Differentiate from parameter \\
\textbf{Method call} & Call another method of same class \\
\textbf{Constructor call} & Call another constructor \\
\textbf{Return object} & Return current object reference \\
\end{longtable}
}

\textbf{Example:}

\begin{verbatim}
public class Person \{
    private String name;
    private int age;
    
    public Person(String name, int age) \{
        this.name = name;  // Distinguish parameter from field
        this.age = age;
    \}
    
    public Person setName(String name) \{
        this.name = name;
        return this;       // Return current object
    \}
    
    public void display() \{
        this.printDetails(); // Call method of same class
    \}
    
    private void printDetails() \{
        System.out.println(this.name + " is " + this.age);
    \}
\}
\end{verbatim}

\begin{itemize}
\tightlist
\item
  \textbf{Current object}: Refers to current instance
\item
  \textbf{Parameter conflict}: Resolve naming conflicts
\item
  \textbf{Method chaining}: Enable fluent interface
\end{itemize}

\end{solutionbox}
\begin{mnemonicbox}
``CRPM - Current, Resolve, Parameter, Method''

\end{mnemonicbox}
\subsection*{Question 3(b OR) [4
marks]}\label{question-3b-or-4-marks}

\textbf{Explain method overriding with example.}

\begin{solutionbox}

\begin{verbatim}
class Animal \{
    public void makeSound() \{
        System.out.println("Animal makes a sound");
    \}
    
    public void sleep() \{
        System.out.println("Animal sleeps");
    \}
\}

class Dog extends Animal \{
    @Override
    public void makeSound() \{  // Method overriding
        System.out.println("Dog barks: Woof!");
    \}
    
    // sleep() method inherited as{-is}
\}

class Cat extends Animal \{
    @Override
    public void makeSound() \{  // Method overriding
        System.out.println("Cat meows: Meow!");
    \}
\}
\end{verbatim}

\textbf{Overriding Rules Table:}

{\def\LTcaptype{none} % do not increment counter
\begin{longtable}[]{@{}ll@{}}
\toprule\noalign{}
Rule & Description \\
\midrule\noalign{}
\endhead
\bottomrule\noalign{}
\endlastfoot
\textbf{Same signature} & Method name, parameters must match \\
\textbf{Inheritance} & Must be in parent-child relationship \\
\textbf{@Override} & Annotation for compiler checking \\
\textbf{Runtime decision} & Method called based on object type \\
\end{longtable}
}

\textbf{Usage:}

\begin{verbatim}
Animal animal1 = new Dog();
Animal animal2 = new Cat();

animal1.makeSound(); // Output: "Dog barks: Woof!"
animal2.makeSound(); // Output: "Cat meows: Meow!"
\end{verbatim}

\begin{itemize}
\tightlist
\item
  \textbf{Runtime polymorphism}: Decision made during execution
\item
  \textbf{Same interface}: Different behavior for different classes
\item
  \textbf{Dynamic binding}: Method resolution at runtime
\end{itemize}

\end{solutionbox}
\begin{mnemonicbox}
``SSRD - Same Signature, Runtime Decision''

\end{mnemonicbox}
\subsection*{Question 3(c OR) [7
marks]}\label{question-3c-or-7-marks}

\textbf{What is package? Write steps to create a package and give
example of it.}

\begin{solutionbox}

\textbf{Package Definition:} Package is a namespace that organizes
related classes and interfaces, providing access control and avoiding
naming conflicts.

\textbf{Steps to Create Package:}

{\def\LTcaptype{none} % do not increment counter
\begin{longtable}[]{@{}lll@{}}
\toprule\noalign{}
Step & Action & Command/Code \\
\midrule\noalign{}
\endhead
\bottomrule\noalign{}
\endlastfoot
1 & \textbf{Create directory} & mkdir com/company/utils \\
2 & \textbf{Add package declaration} & package com.company.utils; \\
3 & \textbf{Write class} & public class MathUtils \{ \} \\
4 & \textbf{Compile} & javac -d . MathUtils.java \\
5 & \textbf{Import and use} & import com.company.utils.*; \\
\end{longtable}
}

\textbf{Example Package Structure:}

\begin{verbatim}
src/
  com/
    company/
      utils/
        MathUtils.java
        StringUtils.java
      models/
        Student.java
\end{verbatim}

\textbf{MathUtils.java:}

\begin{verbatim}
package com.company.utils;

public class MathUtils \{
    public static int add(int a, int b) \{
        return a + b;
    \}
    
    public static int multiply(int a, int b) \{
        return a * b;
    \}
\}
\end{verbatim}

\textbf{Using Package:}

\begin{verbatim}
import com.company.utils.MathUtils;

public class Calculator \{
    public static void main(String[] args) \{
        int sum = MathUtils.add(5, 3);
        int product = MathUtils.multiply(4, 6);
        
        System.out.println("Sum: " + sum);
        System.out.println("Product: " + product);
    \}
\}
\end{verbatim}

\textbf{Package Benefits Table:}

{\def\LTcaptype{none} % do not increment counter
\begin{longtable}[]{@{}ll@{}}
\toprule\noalign{}
Benefit & Description \\
\midrule\noalign{}
\endhead
\bottomrule\noalign{}
\endlastfoot
\textbf{Organization} & Logical grouping of classes \\
\textbf{Namespace} & Avoid naming conflicts \\
\textbf{Access control} & Package-private access \\
\textbf{Maintenance} & Easier code management \\
\end{longtable}
}

\end{solutionbox}
\begin{mnemonicbox}
``ONAM - Organization, Namespace, Access,
Maintenance''

\end{mnemonicbox}
\subsection*{Question 4(a) [3 marks]}\label{q4a}

\textbf{Explain thread priorities with suitable example.}

\begin{solutionbox}

\textbf{Thread Priority Table:}

{\def\LTcaptype{none} % do not increment counter
\begin{longtable}[]{@{}lll@{}}
\toprule\noalign{}
Priority Level & Constant & Value \\
\midrule\noalign{}
\endhead
\bottomrule\noalign{}
\endlastfoot
\textbf{Minimum} & MIN\_PRIORITY & 1 \\
\textbf{Normal} & NORM\_PRIORITY & 5 \\
\textbf{Maximum} & MAX\_PRIORITY & 10 \\
\end{longtable}
}

\textbf{Example:}

\begin{verbatim}
class PriorityDemo extends Thread \{
    public PriorityDemo(String name) \{
        super(name);
    \}
    
    public void run() \{
for (int

i = 1; i {=} 5; i++) \{

            System.out.println(getName() + " {- Count: "} + i);
        \}
    \}
\}

public class ThreadPriorityExample \{
    public static void main(String[] args) \{
        PriorityDemo t1 = new PriorityDemo("High Priority");
        PriorityDemo t2 = new PriorityDemo("Low Priority");
        
        t1.setPriority(Thread.MAX\_PRIORITY);  // Priority 10
        t2.setPriority(Thread.MIN\_PRIORITY);  // Priority 1
        
        t1.start();
        t2.start();
    \}
\}
\end{verbatim}

\begin{itemize}
\tightlist
\item
  \textbf{Higher priority}: More likely to get CPU time
\item
  \textbf{Not guaranteed}: JVM decides actual scheduling
\item
  \textbf{Default priority}: Every thread starts with NORM\_PRIORITY
\end{itemize}

\end{solutionbox}
\begin{mnemonicbox}
``HNG - Higher priority, Not Guaranteed''

\end{mnemonicbox}
\subsection*{Question 4(b) [4 marks]}\label{q4b}

\textbf{What is Thread? Explain Thread life cycle.}

\begin{solutionbox}

\textbf{Thread Definition:} Thread is a lightweight sub-process that
allows concurrent execution of multiple tasks within a program.

\textbf{Thread Life Cycle Diagram:}

\begin{center}
\textbf{Mermaid Diagram (Code)}
\begin{verbatim}
{Shaded}
{Highlighting}[]
graph LR
    A[NEW] {-{-}{} B[RUNNABLE]}
    B {-{-}{} C[RUNNING]}
    C {-{-}{} D[BLOCKED/WAITING]}
    D {-{-}{} B}
    C {-{-}{} E[TERMINATED]}
{Highlighting}
{Shaded}
\end{verbatim}
\end{center}

\textbf{Thread States Table:}

{\def\LTcaptype{none} % do not increment counter
\begin{longtable}[]{@{}ll@{}}
\toprule\noalign{}
State & Description \\
\midrule\noalign{}
\endhead
\bottomrule\noalign{}
\endlastfoot
\textbf{NEW} & Thread created but not started \\
\textbf{RUNNABLE} & Ready to run, waiting for CPU \\
\textbf{RUNNING} & Currently executing \\
\textbf{BLOCKED/WAITING} & Waiting for resource/condition \\
\textbf{TERMINATED} & Execution completed \\
\end{longtable}
}

\textbf{State Transitions:}

\begin{itemize}
\item
  \textbf{NEW \rightarrow RUNNABLE}: start() method called
\item
  \textbf{RUNNABLE \rightarrow RUNNING}: Thread scheduler assigns CPU
\item
  \textbf{RUNNING \rightarrow BLOCKED}: Waiting for I/O or lock
\item
  \textbf{RUNNING \rightarrow TERMINATED}: run() method completes
\item
  \textbf{Concurrent execution}: Multiple threads run simultaneously
\item
  \textbf{JVM managed}: Thread scheduler controls execution
\item
  \textbf{Resource sharing}: Threads share memory space
\end{itemize}

\end{solutionbox}
\begin{mnemonicbox}
``NRBT - New, Runnable, Blocked, Terminated''

\end{mnemonicbox}
\subsection*{Question 4(c) [7 marks]}\label{q4c}

\textbf{Write a program in java that create the multiple threads by
implementing the Thread class.}

\begin{solutionbox}

\begin{verbatim}
class NumberPrinter extends Thread \{
    private String threadName;
    private int start;
    private int end;
    
    public NumberPrinter(String name, int start, int end) \{
        this.threadName = name;
        this.start = start;
        this.end = end;
    \}
    
    @Override
    public void run() \{
        System.out.println(threadName + " started");
        
for (int

i = start; i {=} end; i++) \{

            System.out.println(threadName + ": " + i);
            
            try \{
                Thread.sleep(500); // Pause for 500ms
            \} catch (InterruptedException e) \{
                System.out.println(threadName + " interrupted");
            \}
        \}
        
        System.out.println(threadName + " finished");
    \}
\}

public class MultipleThreadsExample \{
    public static void main(String[] args) \{
        // Create multiple threads
        NumberPrinter thread1 = new NumberPrinter("Thread{-1"}, 1, 5);
        NumberPrinter thread2 = new NumberPrinter("Thread{-2"}, 10, 15);
        NumberPrinter thread3 = new NumberPrinter("Thread{-3"}, 20, 25);
        
        // Start all threads
        thread1.start();
        thread2.start();
        thread3.start();
        
        System.out.println("All threads started from main");
    \}
\}
\end{verbatim}

\textbf{Implementation Steps Table:}

{\def\LTcaptype{none} % do not increment counter
\begin{longtable}[]{@{}ll@{}}
\toprule\noalign{}
Step & Action \\
\midrule\noalign{}
\endhead
\bottomrule\noalign{}
\endlastfoot
1 & \textbf{Extend Thread class} \\
2 & \textbf{Override run() method} \\
3 & \textbf{Create thread objects} \\
4 & \textbf{Call start() method} \\
\end{longtable}
}

\begin{itemize}
\tightlist
\item
  \textbf{Extends Thread}: Inherit threading capabilities
\item
  \textbf{Override run()}: Define thread's execution logic
\item
  \textbf{start() method}: Begin thread execution
\item
  \textbf{Concurrent execution}: All threads run simultaneously
\end{itemize}

\end{solutionbox}
\begin{mnemonicbox}
``EOCS - Extend, Override, Create, Start''

\end{mnemonicbox}
\subsection*{Question 4(a OR) [3
marks]}\label{question-4a-or-3-marks}

\textbf{Explain basic concept of Exception Handling.}

\begin{solutionbox}

\textbf{Exception Handling Concepts Table:}

{\def\LTcaptype{none} % do not increment counter
\begin{longtable}[]{@{}ll@{}}
\toprule\noalign{}
Concept & Description \\
\midrule\noalign{}
\endhead
\bottomrule\noalign{}
\endlastfoot
\textbf{Exception} & Runtime error that disrupts normal flow \\
\textbf{try block} & Code that might throw exception \\
\textbf{catch block} & Handles specific exception types \\
\textbf{finally block} & Always executes, cleanup code \\
\end{longtable}
}

\textbf{Exception Hierarchy:}

\begin{center}
\textbf{Mermaid Diagram (Code)}
\begin{verbatim}
{Shaded}
{Highlighting}[]
graph LR
    A[Throwable] {-{-}{} B[Exception]}
    A {-{-}{} C[Error]}
    B {-{-}{} D[RuntimeException]}
    B {-{-}{} E[Checked Exceptions]}
    D {-{-}{} F[NullPointerException]}
    D {-{-}{} G[ArrayIndexOutOfBoundsException]}
{Highlighting}
{Shaded}
\end{verbatim}
\end{center}

\textbf{Basic Syntax:}

\begin{verbatim}
try \{
    // Risky code
\} catch (ExceptionType e) \{
    // Handle exception
\} finally \{
    // Cleanup code
\}
\end{verbatim}

\begin{itemize}
\tightlist
\item
  \textbf{Graceful handling}: Program continues after exception
\item
  \textbf{Error prevention}: Avoid program crash
\item
  \textbf{Resource cleanup}: finally block ensures cleanup
\end{itemize}

\end{solutionbox}
\begin{mnemonicbox}
``TRCF - Try, Runtime error, Catch, Finally''

\end{mnemonicbox}
\subsection*{Question 4(b OR) [4
marks]}\label{question-4b-or-4-marks}

\textbf{Explain multiple catch with suitable example.}

\begin{solutionbox}

\begin{verbatim}
public class MultipleCatchExample \{
    public static void main(String[] args) \{
        try \{
            int[] numbers = \{10, 20, 30\;}
            int divisor = Integer.parseInt(args[0]);
            
            int result = numbers[5] / divisor;  // May cause multiple exceptions
            System.out.println("Result: " + result);
            
        \} catch (ArrayIndexOutOfBoundsException e) \{
            System.out.println("Array index error: " + e.getMessage());
            
        \} catch (ArithmeticException e) \{
            System.out.println("Math error: " + e.getMessage());
            
        \} catch (NumberFormatException e) \{
            System.out.println("Number format error: " + e.getMessage());
            
        \} catch (Exception e) \{  // Generic catch
            System.out.println("General error: " + e.getMessage());
            
        \} finally \{
            System.out.println("Cleanup completed");
        \}
    \}
\}
\end{verbatim}

\textbf{Multiple Catch Rules Table:}

{\def\LTcaptype{none} % do not increment counter
\begin{longtable}[]{@{}ll@{}}
\toprule\noalign{}
Rule & Description \\
\midrule\noalign{}
\endhead
\bottomrule\noalign{}
\endlastfoot
\textbf{Specific first} & Handle specific exceptions before general \\
\textbf{One catch executes} & Only first matching catch runs \\
\textbf{Order matters} & More specific to more general \\
\textbf{finally always} & finally block always executes \\
\end{longtable}
}

\textbf{Exception Flow:}

\begin{itemize}
\tightlist
\item
  \textbf{ArrayIndexOutOfBoundsException}: Invalid array access
\item
  \textbf{ArithmeticException}: Division by zero
\item
  \textbf{NumberFormatException}: Invalid number conversion
\item
  \textbf{Exception}: Catches any remaining exceptions
\end{itemize}

\end{solutionbox}
\begin{mnemonicbox}
``SOOF - Specific first, One executes, Order matters,
Finally''

\end{mnemonicbox}
\subsection*{Question 4(c OR) [7
marks]}\label{question-4c-or-7-marks}

\textbf{What is Exception? Write a program that show the use of
Arithmetic Exception.}

\begin{solutionbox}

\textbf{Exception Definition:} Exception is an event that occurs during
program execution and disrupts the normal flow of instructions.

\begin{verbatim}
public class ArithmeticExceptionDemo \{
    
    public static double divide(int numerator, int denominator) \{
        try \{
            if (denominator == 0) \{
                throw new ArithmeticException("Division by zero is not allowed");
            \}
            return (double) numerator / denominator;
            
        \} catch (ArithmeticException e) \{
            System.out.println("Arithmetic Exception caught: " + e.getMessage());
            return Double.NaN;  // Return Not{-a{-}Number}
        \}
    \}
    
    public static void calculatorDemo() \{
        int[] numbers = \{100, 50, 25, 0, {-}10\;}
        
        for (int i = 0; i {} numbers.length; i++) \{
            try \{
                int result = 100 / numbers[i];
                System.out.println("100 / " + numbers[i] + " = " + result);
                
            \} catch (ArithmeticException e) \{
                System.out.println("Cannot divide 100 by " + numbers[i] + 
                                 " {- "} + e.getMessage());
            \}
        \}
    \}
    
    public static void main(String[] args) \{
        System.out.println("=== Arithmetic Exception Demo ===");
        
        // Test custom divide method
        System.out.println("{n}1. Custom divide method:");
        System.out.println("10 / 2 = " + divide(10, 2));
        System.out.println("15 / 0 = " + divide(15, 0));
        
        // Test calculator demo
        System.out.println("{n}2. Calculator demo:");
        calculatorDemo();
        
        // Test with try{-catch{-}finally}
        System.out.println("{n}3. Try{-catch{-}finally demo:"});
        try \{
            int value = 50;
            int zero = 0;
            int result = value / zero;  // This will throw ArithmeticException
            
        \} catch (ArithmeticException e) \{
            System.out.println("Exception handled: " + e.toString());
            
        \} finally \{
            System.out.println("Finally block: Cleanup completed");
        \}
        
        System.out.println("Program continues normally after exception handling");
    \}
\}
\end{verbatim}

\textbf{Exception Types Table:}

{\def\LTcaptype{none} % do not increment counter
\begin{longtable}[]{@{}lll@{}}
\toprule\noalign{}
Type & Description & Example \\
\midrule\noalign{}
\endhead
\bottomrule\noalign{}
\endlastfoot
\textbf{Checked} & Must be handled at compile time & IOException \\
\textbf{Unchecked} & Runtime exceptions & ArithmeticException \\
\textbf{Error} & System-level problems & OutOfMemoryError \\
\end{longtable}
}

\textbf{ArithmeticException Causes:}

\begin{itemize}
\tightlist
\item
  \textbf{Division by zero}: Most common cause
\item
  \textbf{Modulo by zero}: Remainder operation with zero
\item
  \textbf{Invalid operations}: Mathematical impossibilities
\end{itemize}

\textbf{Program Flow:}

\begin{enumerate}
\tightlist
\item
  \textbf{Normal execution}: Try block runs
\item
  \textbf{Exception occurs}: ArithmeticException thrown
\item
  \textbf{Exception caught}: Catch block handles it
\item
  \textbf{Cleanup}: Finally block executes
\item
  \textbf{Continue}: Program continues after handling
\end{enumerate}

\end{solutionbox}
\begin{mnemonicbox}
``DZMI - Division by Zero, Mathematical Invalid''

\end{mnemonicbox}
\subsection*{Question 5(a) [3 marks]}\label{q5a}

\textbf{Explain ArrayIndexOutOfBound Exception in Java with example.}

\begin{solutionbox}

\textbf{ArrayIndexOutOfBound Exception Table:}

{\def\LTcaptype{none} % do not increment counter
\begin{longtable}[]{@{}lll@{}}
\toprule\noalign{}
Cause & Description & Example \\
\midrule\noalign{}
\endhead
\bottomrule\noalign{}
\endlastfoot
\textbf{Negative index} & Index less than 0 & arr[-1] \\
\textbf{Index \textgreater= length} & Index beyond array size &
arr[5] for size 3 \\
\textbf{Empty array} & Access on zero-length array & arr[0] for
length 0 \\
\end{longtable}
}

\textbf{Example:}

\begin{verbatim}
public class ArrayIndexDemo \{
    public static void main(String[] args) \{
        int[] numbers = \{10, 20, 30\;}
        
        try \{
            System.out.println(numbers[5]); // Index 5 { length 3}
        \} catch (ArrayIndexOutOfBoundsException e) \{
            System.out.println("Error: " + e.getMessage());
        \}
        
        try \{
            System.out.println(numbers[{-}1]); // Negative index
        \} catch (ArrayIndexOutOfBoundsException e) \{
            System.out.println("Error: Negative index");
        \}
    \}
\}
\end{verbatim}

\begin{itemize}
\tightlist
\item
  \textbf{Runtime exception}: Occurs during program execution
\item
  \textbf{Index validation}: Always check array bounds
\item
  \textbf{Prevention}: Use array.length for bounds checking
\end{itemize}

\end{solutionbox}
\begin{mnemonicbox}
``NIE - Negative, Index-exceed, Empty''

\end{mnemonicbox}
\subsection*{Question 5(b) [4 marks]}\label{q5b}

\textbf{Explain basics of stream classes.}

\begin{solutionbox}

\textbf{Stream Classes Hierarchy:}

\begin{center}
\textbf{Mermaid Diagram (Code)}
\begin{verbatim}
{Shaded}
{Highlighting}[]
graph TD
    A[InputStream] {-{-}{} B[FileInputStream]}
    A {-{-}{} C[BufferedInputStream]}
    D[OutputStream] {-{-}{} E[FileOutputStream]}
    D {-{-}{} F[BufferedOutputStream]}
    G[Reader] {-{-}{} H[FileReader]}
    G {-{-}{} I[BufferedReader]}
    J[Writer] {-{-}{} K[FileWriter]}
    J {-{-}{} L[BufferedWriter]}
{Highlighting}
{Shaded}
\end{verbatim}
\end{center}

\textbf{Stream Types Table:}

{\def\LTcaptype{none} % do not increment counter
\begin{longtable}[]{@{}
  >{\raggedright\arraybackslash}p{(\linewidth - 4\tabcolsep) * \real{0.4194}}
  >{\raggedright\arraybackslash}p{(\linewidth - 4\tabcolsep) * \real{0.2903}}
  >{\raggedright\arraybackslash}p{(\linewidth - 4\tabcolsep) * \real{0.2903}}@{}}
\toprule\noalign{}
\begin{minipage}[b]{\linewidth}\raggedright
Stream Type
\end{minipage} & \begin{minipage}[b]{\linewidth}\raggedright
Purpose
\end{minipage} & \begin{minipage}[b]{\linewidth}\raggedright
Classes
\end{minipage} \\
\midrule\noalign{}
\endhead
\bottomrule\noalign{}
\endlastfoot
\textbf{Byte Streams} & Handle binary data & InputStream,
OutputStream \\
\textbf{Character Streams} & Handle text data & Reader, Writer \\
\textbf{Buffered Streams} & Improve performance & BufferedReader,
BufferedWriter \\
\textbf{File Streams} & File operations & FileInputStream,
FileOutputStream \\
\end{longtable}
}

\textbf{Basic Operations:}

\begin{itemize}
\tightlist
\item
  \textbf{Input}: Read data from source
\item
  \textbf{Output}: Write data to destination
\item
  \textbf{Buffering}: Store data temporarily for efficiency
\item
  \textbf{Closing}: Release system resources
\end{itemize}

\textbf{Stream Benefits:}

\begin{itemize}
\tightlist
\item
  \textbf{Abstraction}: Uniform interface for I/O
\item
  \textbf{Efficiency}: Buffered operations
\item
  \textbf{Flexibility}: Various data sources/destinations
\end{itemize}

\end{solutionbox}
\begin{mnemonicbox}
``BCIF - Byte, Character, Input/Output, File''

\end{mnemonicbox}
\subsection*{Question 5(c) [7 marks]}\label{q5c}

\textbf{Write a java program to create a text file and perform write
operation on the text file.}

\begin{solutionbox}

\begin{verbatim}
import java.io.*;

public class FileWriteDemo \{
    
    public static void writeWithFileWriter() \{
        try \{
            FileWriter writer = new FileWriter("student\_data.txt");
            
            writer.write("Student Information System{n}");
            writer.write("=========================={n}");
            writer.write("ID: 101{n}");
            writer.write("Name: John Doe{n}");
            writer.write("Course: Java Programming{n}");
            writer.write("Grade: A+{n}");
            
            writer.close();
            System.out.println("File written successfully using FileWriter");
            
        \} catch (IOException e) \{
            System.out.println("Error writing file: " + e.getMessage());
        \}
    \}
    
    public static void writeWithBufferedWriter() \{
        try \{
            BufferedWriter buffWriter = new BufferedWriter(
                new FileWriter("course\_details.txt")
            );
            
            String[] courses = \{
                "Java Programming {- 4341602"},
                "Database Management {- 4341603"}, 
                "Web Development {- 4341604"},
                "Mobile App Development {- 4341605"}
            \;}
            
            buffWriter.write("Available Courses:{n}");
            buffWriter.write("=================={n}");
            
            for (String course : courses) \{
                buffWriter.write(course + "{n}");
            \}
            
            buffWriter.close();
            System.out.println("File written successfully using BufferedWriter");
            
        \} catch (IOException e) \{
            System.out.println("Error: " + e.getMessage());
        \}
    \}
    
    public static void writeWithTryWithResources() \{
        try (FileWriter writer = new FileWriter("marks\_record.txt")) \{
            
            writer.write("Semester 4 Marks Record{n}");
            writer.write("======================={n}");
            writer.write("Java Programming: 85{n}");
            writer.write("Database Management: 78{n}");
            writer.write("Web Development: 92{n}");
            writer.write("Total: 255/300{n}");
            writer.write("Percentage: 85\%{n}");
            
            System.out.println("File written with automatic resource management");
            
        \} catch (IOException e) \{
            System.out.println("File write error: " + e.getMessage());
        \}
    \}
    
    public static void main(String[] args) \{
        System.out.println("=== File Write Operations Demo ==={n}");
        
        // Method 1: Basic FileWriter
        writeWithFileWriter();
        
        // Method 2: BufferedWriter for better performance
        writeWithBufferedWriter();
        
        // Method 3: Try{-with{-}resources (recommended)}
        writeWithTryWithResources();
        
        System.out.println("{n}All file write operations completed!");
    \}
\}
\end{verbatim}

\textbf{File Write Methods Table:}

{\def\LTcaptype{none} % do not increment counter
\begin{longtable}[]{@{}llll@{}}
\toprule\noalign{}
Method & Performance & Resource Management & Use Case \\
\midrule\noalign{}
\endhead
\bottomrule\noalign{}
\endlastfoot
\textbf{FileWriter} & Basic & Manual close() & Simple writes \\
\textbf{BufferedWriter} & High & Manual close() & Large data \\
\textbf{Try-with-resources} & High & Automatic & Recommended \\
\end{longtable}
}

\textbf{Write Operation Steps:}

\begin{enumerate}
\tightlist
\item
  \textbf{Create writer object}: FileWriter or BufferedWriter
\item
  \textbf{Write data}: Use write() method
\item
  \textbf{Close stream}: Release resources
\item
  \textbf{Handle exceptions}: IOException management
\end{enumerate}

\textbf{File Operations:}

\begin{itemize}
\tightlist
\item
  \textbf{Create}: New file if doesn't exist
\item
  \textbf{Overwrite}: Replaces existing content
\item
  \textbf{Append}: Add to existing content (use append mode)
\end{itemize}

\end{solutionbox}
\begin{mnemonicbox}
``CWCH - Create, Write, Close, Handle''

\end{mnemonicbox}
\subsection*{Question 5(a OR) [3
marks]}\label{question-5a-or-3-marks}

\textbf{Explain Divide by Zero Exception in Java with example.}

\begin{solutionbox}

\textbf{Divide by Zero Exception Table:}

{\def\LTcaptype{none} % do not increment counter
\begin{longtable}[]{@{}lll@{}}
\toprule\noalign{}
Operation & Result & Exception \\
\midrule\noalign{}
\endhead
\bottomrule\noalign{}
\endlastfoot
\textbf{Integer division} & Undefined & ArithmeticException \\
\textbf{Float division} & Infinity & No exception \\
\textbf{Modulo by zero} & Undefined & ArithmeticException \\
\end{longtable}
}

\textbf{Example:}

\begin{verbatim}
public class DivideByZeroDemo \{
    public static void main(String[] args) \{
        // Integer division by zero
        try \{
            int result = 10 / 0;
        \} catch (ArithmeticException e) \{
            System.out.println("Integer division: " + e.getMessage());
        \}
        
        // Float division by zero (no exception)
        double floatResult = 10.0 / 0.0;
        System.out.println("Float division: " + floatResult); // Infinity
        
        // Modulo by zero
        try \{
            int remainder = 10 \% 0;
        \} catch (ArithmeticException e) \{
            System.out.println("Modulo error: " + e.getMessage());
        \}
    \}
\}
\end{verbatim}

\begin{itemize}
\tightlist
\item
  \textbf{Integer arithmetic}: Throws ArithmeticException
\item
  \textbf{Floating point}: Returns Infinity (IEEE 754 standard)
\item
  \textbf{Prevention}: Check denominator before division
\end{itemize}

\end{solutionbox}
\begin{mnemonicbox}
``IFM - Integer exception, Float infinity, Modulo
error''

\end{mnemonicbox}
\subsection*{Question 5(b OR) [4
marks]}\label{question-5b-or-4-marks}

\textbf{Explain try and catch block with example.}

\begin{solutionbox}

\textbf{Try-Catch Structure:}

\begin{verbatim}
try \{
    // Risky code that might throw exception
\} catch (SpecificException e) \{
    // Handle specific exception
\} catch (GeneralException e) \{
    // Handle general exception
\} finally \{
    // Always executes (optional)
\}
\end{verbatim}

\textbf{Example:}

\begin{verbatim}
public class TryCatchExample \{
    public static void validateAge(int age) \{
        try \{
            if (age {} 0) \{
                throw new IllegalArgumentException("Age cannot be negative");
            \}
            if (age {} 150) \{
                throw new IllegalArgumentException("Age seems unrealistic");
            \}
            System.out.println("Valid age: " + age);
            
        \} catch (IllegalArgumentException e) \{
            System.out.println("Validation error: " + e.getMessage());
        \}
    \}
    
    public static void main(String[] args) \{
        validateAge(25);    // Valid
        validateAge({-}5);    // Invalid
        validateAge(200);   // Invalid
    \}
\}
\end{verbatim}

\textbf{Try-Catch Flow Table:}

{\def\LTcaptype{none} % do not increment counter
\begin{longtable}[]{@{}lll@{}}
\toprule\noalign{}
Block & Purpose & Execution \\
\midrule\noalign{}
\endhead
\bottomrule\noalign{}
\endlastfoot
\textbf{try} & Contains risky code & Always executed first \\
\textbf{catch} & Handles exceptions & Only if exception occurs \\
\textbf{finally} & Cleanup code & Always executed \\
\end{longtable}
}

\begin{itemize}
\tightlist
\item
  \textbf{Exception matching}: First matching catch block executes
\item
  \textbf{Control flow}: Program continues after catch block
\item
  \textbf{Multiple catches}: Handle different exception types
\end{itemize}

\end{solutionbox}
\begin{mnemonicbox}
``TCF - Try risky, Catch exception, Finally cleanup''

\end{mnemonicbox}
\subsection*{Question 5(c OR) [7
marks]}\label{question-5c-or-7-marks}

\textbf{Write a java program to display the content of a text file and
perform append operation on the text file.}

\begin{solutionbox}

\begin{verbatim}
import java.io.*;

public class FileReadAppendDemo \{
    
    public static void createInitialFile() \{
        try (FileWriter writer = new FileWriter("student\_log.txt")) \{
            writer.write("Student Activity Log{n}");
            writer.write("==================={n}");
            writer.write("2024{-06{-}13: Course registration started}{n}");
            writer.write("2024{-06{-}14: Assignment 1 submitted}{n}");
            
            System.out.println("Initial file created successfully");
            
        \} catch (IOException e) \{
            System.out.println("Error creating file: " + e.getMessage());
        \}
    \}
    
    public static void displayFileContent(String fileName) \{
        System.out.println("{n}=== File Content ===");
        
        try (BufferedReader reader = new BufferedReader(new FileReader(fileName))) \{
            String line;
            int lineNumber = 1;
            
            while ((line = reader.readLine()) != null) \{
                System.out.println(lineNumber + ": " + line);
                lineNumber++;
            \}
            
        \} catch (FileNotFoundException e) \{
            System.out.println("File not found: " + fileName);
        \} catch (IOException e) \{
            System.out.println("Error reading file: " + e.getMessage());
        \}
    \}
    
    public static void appendToFile(String fileName, String content) \{
        try (FileWriter writer = new FileWriter(fileName, true)) \{ // true = append mode
            writer.write(content);
            System.out.println("Content appended successfully");
            
        \} catch (IOException e) \{
            System.out.println("Error appending to file: " + e.getMessage());
        \}
    \}
    
    public static void appendMultipleEntries(String fileName) \{
        String[] newEntries = \{
            "2024{-06{-}15: Quiz 1 completed}{n}",
            "2024{-06{-}16: Project proposal submitted}{n}",
            "2024{-06{-}17: Group study session}{n}",
            "2024{-06{-}18: Mid{-}term exam preparation}{n}"
        \;}
        
        try (BufferedWriter writer = new BufferedWriter(
                new FileWriter(fileName, true))) \{
            
            writer.write("{n}{-{-}{-} Recent Activities {-}{-}{-}}{n}");
            
            for (String entry : newEntries) \{
                writer.write(entry);
            \}
            
            writer.write("{-{-}{-} End of Log {-}{-}{-}}{n}");
            System.out.println("Multiple entries appended successfully");
            
        \} catch (IOException e) \{
            System.out.println("Error appending entries: " + e.getMessage());
        \}
    \}
    
    public static void main(String[] args) \{
        String fileName = "student\_log.txt";
        
        System.out.println("=== File Read and Append Operations ===");
        
        // Step 1: Create initial file
        createInitialFile();
        
        // Step 2: Display initial content
        displayFileContent(fileName);
        
        // Step 3: Append single entry
        appendToFile(fileName, "2024{-06{-}19: Lab session completed}{n}");
        
        // Step 4: Display content after first append
        System.out.println("{n}{-{-}{-} After first append {-}{-}{-}"});
        displayFileContent(fileName);
        
        // Step 5: Append multiple entries
        appendMultipleEntries(fileName);
        
        // Step 6: Display final content
        System.out.println("{n}{-{-}{-} Final file content {-}{-}{-}"});
        displayFileContent(fileName);
        
        // Step 7: File statistics
        showFileStatistics(fileName);
    \}
    
    public static void showFileStatistics(String fileName) \{
        try (BufferedReader reader = new BufferedReader(new FileReader(fileName))) \{
            int lineCount = 0;
            int charCount = 0;
            String line;
            
            while ((line = reader.readLine()) != null) \{
                lineCount++;
                charCount += line.length();
            \}
            
            System.out.println("{n}=== File Statistics ===");
            System.out.println("Total lines: " + lineCount);
            System.out.println("Total characters: " + charCount);
            
        \} catch (IOException e) \{
            System.out.println("Error reading file statistics: " + e.getMessage());
        \}
    \}
\}
\end{verbatim}

\textbf{File Operations Table:}

{\def\LTcaptype{none} % do not increment counter
\begin{longtable}[]{@{}lll@{}}
\toprule\noalign{}
Operation & Method & Purpose \\
\midrule\noalign{}
\endhead
\bottomrule\noalign{}
\endlastfoot
\textbf{Create} & FileWriter(filename) & Create new file \\
\textbf{Read} & BufferedReader.readLine() & Read file content \\
\textbf{Append} & FileWriter(filename, true) & Add to existing file \\
\textbf{Display} & System.out.println() & Show content \\
\end{longtable}
}

\textbf{File Operations Flow:}

\begin{enumerate}
\tightlist
\item
  \textbf{Create initial file}: Write initial content
\item
  \textbf{Display content}: Read and show current content
\item
  \textbf{Append data}: Add new information
\item
  \textbf{Display updated}: Show modified content
\item
  \textbf{Statistics}: Count lines and characters
\end{enumerate}

\textbf{Append vs Write:}

\begin{itemize}
\tightlist
\item
  \textbf{Write mode}: Overwrites existing content
\item
  \textbf{Append mode}: Adds to end of existing content
\item
  \textbf{Constructor parameter}: Second parameter true enables append
\end{itemize}

\textbf{Resource Management:}

\begin{itemize}
\tightlist
\item
  \textbf{Try-with-resources}: Automatic close()
\item
  \textbf{Exception handling}: FileNotFoundException, IOException
\item
  \textbf{Buffered operations}: Better performance for large files
\end{itemize}

\end{solutionbox}
\begin{mnemonicbox}
``CDADS - Create, Display, Append, Display,
Statistics''

\end{mnemonicbox}

\end{document}
