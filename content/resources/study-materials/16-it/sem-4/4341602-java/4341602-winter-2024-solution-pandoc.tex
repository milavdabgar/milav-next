\documentclass[10pt,a4paper]{article}

% content/resources/templates/preamble.tex
\usepackage[margin=0.6in]{geometry}
\author{Milav Dabgar}
\usepackage{amsmath,amssymb,amsthm}
\usepackage{booktabs}
\usepackage{multirow}
\usepackage{xcolor}
\usepackage{tcolorbox}
\tcbuselibrary{breakable,skins}
\usepackage[colorlinks=true,linkcolor=blue]{hyperref}
\usepackage{titlesec}
\usepackage{enumitem}
\usepackage{tikz}
\usepackage{pgfplots}
\usepackage{circuitikz}
\usepackage[version=4]{mhchem}
\usepackage{longtable}
\usepackage{array}
\usepackage{float}
\usepackage{caption}
\usepackage{listings}

\lstset{
  basicstyle=\small\ttfamily,
  breaklines=true,
  breakatwhitespace=false,
  postbreak=\mbox{\textcolor{red}{$\hookrightarrow$}\space},
  float=false,
  numbers=left,
  numberstyle=\tiny\color{gray},
  numbersep=10pt,
  xleftmargin=2em,
  keywordstyle=\color{blue},
  commentstyle=\color{green!60!black},
  stringstyle=\color{purple},
  backgroundcolor=\color{gray!5},
  showstringspaces=false,
  tabsize=2,
  captionpos=b,
  keepspaces=true,
  columns=flexible
}

\pgfplotsset{compat=1.18}
\usetikzlibrary{shapes,arrows,positioning,calc,patterns,decorations.pathmorphing,decorations.markings,arrows.meta}

% Color scheme
\definecolor{headcolor}{RGB}{0,102,204}
\definecolor{keycolor}{RGB}{220,20,60}
\definecolor{solutioncolor}{RGB}{34,139,34}
\definecolor{mnemoniccolor}{RGB}{148,0,211}
\definecolor{codecolor}{RGB}{0,0,100}

% Spacing
\setlength{\parskip}{3pt}
\setlist[itemize]{nosep}
\setlist[enumerate]{nosep}

% Title formatting
\titleformat{\section}{\Large\bfseries\color{headcolor}}{\thesection}{1em}{}
\titleformat{\subsection}{\large\bfseries\color{headcolor}}{\thesubsection}{1em}{}

% Pandoc tightlist compatibility
\providecommand{\tightlist}{%
  \setlength{\itemsep}{0pt}\setlength{\parskip}{0pt}}

% Pandoc longtable compatibility
\newcounter{none}
\def\thenone{}


% content/resources/templates/english-boxes.tex
% This file is currently empty - it exists to maintain consistency with the import structure.
% Add custom environments here if needed in the future.


\begin{document}

\begin{center}
{\Huge\bfseries\color{headcolor} Subject Name Solutions}\\[5pt]
{\LARGE 4341602 -- Winter 2024}\\[3pt]
{\large Semester 1 Study Material}\\[3pt]
{\normalsize\textit{Detailed Solutions and Explanations}}
\end{center}

\vspace{10pt}

\subsection*{Question 1(a) [3 marks]}\label{q1a}

\textbf{Write down the difference between oop and pop.}

\begin{solutionbox}

{\def\LTcaptype{none} % do not increment counter
\begin{longtable}[]{@{}
  >{\raggedright\arraybackslash}p{(\linewidth - 4\tabcolsep) * \real{0.4000}}
  >{\raggedright\arraybackslash}p{(\linewidth - 4\tabcolsep) * \real{0.3000}}
  >{\raggedright\arraybackslash}p{(\linewidth - 4\tabcolsep) * \real{0.3000}}@{}}
\toprule\noalign{}
\begin{minipage}[b]{\linewidth}\raggedright
\textbf{Aspect}
\end{minipage} & \begin{minipage}[b]{\linewidth}\raggedright
\textbf{OOP}
\end{minipage} & \begin{minipage}[b]{\linewidth}\raggedright
\textbf{POP}
\end{minipage} \\
\midrule\noalign{}
\endhead
\bottomrule\noalign{}
\endlastfoot
\textbf{Approach} & Bottom-up approach & Top-down approach \\
\textbf{Focus} & Objects and classes & Functions and procedures \\
\textbf{Data Security} & Data hiding through encapsulation & No data
hiding \\
\textbf{Problem Solving} & Divide problem into objects & Divide problem
into functions \\
\end{longtable}
}

\end{solutionbox}
\begin{mnemonicbox}
``Objects Bottom, Procedures Top''

\end{mnemonicbox}
\subsection*{Question 1(b) [4 marks]}\label{q1b}

\textbf{What is byte code? Explain JVM in detail.}

\begin{solutionbox}

\textbf{Byte Code}: Platform-independent intermediate code generated by
Java compiler from source code.

\begin{center}
\textbf{Mermaid Diagram (Code)}
\begin{verbatim}
{Shaded}
{Highlighting}[]
graph LR
    A[Java Source Code] {-{-}{} B[Java Compiler javac]}
    B {-{-}{} C[Byte Code .class]}
    C {-{-}{} D[JVM]}
    D {-{-}{} E[Machine Code]}
{Highlighting}
{Shaded}
\end{verbatim}
\end{center}

\textbf{JVM Components}:

\begin{itemize}
\tightlist
\item
  \textbf{Class Loader}: Loads .class files into memory
\item
  \textbf{Memory Area}: Heap, stack, method area storage
\item
  \textbf{Execution Engine}: Interprets and executes bytecode
\item
  \textbf{Garbage Collector}: Automatic memory management
\end{itemize}

\end{solutionbox}
\begin{mnemonicbox}
``Byte Code Runs Everywhere''

\end{mnemonicbox}
\subsection*{Question 1(c) [7 marks]}\label{q1c}

\textbf{Write a program in Java to sort the elements of an array in
ascending order}

\begin{solutionbox}

\begin{verbatim}
import java.util.Arrays;

public class ArraySort \{
    public static void main(String[] args) \{
        int[] arr = \{64, 34, 25, 12, 22, 11, 90\;}
        
        // Bubble Sort
        for(int i = 0; i {} arr.length{-}1; i++) \{
            for(int j = 0; j {} arr.length{-}i{-}1; j++) \{
                if(arr[j] {} arr[j+1]) \{
                    int temp = arr[j];
                    arr[j] = arr[j+1];
                    arr[j+1] = temp;
                \}
            \}
        \}
        
        System.out.println("Sorted array: " + Arrays.toString(arr));
    \}
\}
\end{verbatim}

\textbf{Key Points}:

\begin{itemize}
\tightlist
\item
  \textbf{Bubble Sort}: Compares adjacent elements
\item
  \textbf{Time Complexity}: O(n^{2})
\item
  \textbf{Space Complexity}: O(1)
\end{itemize}

\end{solutionbox}
\begin{mnemonicbox}
``Bubble Up The Smallest''

\end{mnemonicbox}
\subsection*{Question 1(c OR) [7
marks]}\label{question-1c-or-7-marks}

\textbf{Write a program in java to find out maximum from any ten numbers
using command line argument.}

\begin{solutionbox}

\begin{verbatim}
public class FindMaximum \{
    public static void main(String[] args) \{
        if(args.length != 10) \{
            System.out.println("Please enter exactly 10 numbers");
            return;
        \}
        
        int max = Integer.parseInt(args[0]);
        
        for(int i = 1; i {} args.length; i++) \{
            int num = Integer.parseInt(args[i]);
            if(num {} max) \{
                max = num;
            \}
        \}
        
        System.out.println("Maximum number: " + max);
    \}
\}
\end{verbatim}

\textbf{Key Points}:

\begin{itemize}
\tightlist
\item
  \textbf{Command Line}: args[] array stores arguments
\item
  \textbf{parseInt()}: Converts string to integer
\item
  \textbf{Validation}: Check array length
\end{itemize}

\end{solutionbox}
\begin{mnemonicbox}
``Arguments Maximum Search''

\end{mnemonicbox}
\subsection*{Question 2(a) [3 marks]}\label{q2a}

\textbf{What is wrapper class? Explain with example.}

\begin{solutionbox}

\textbf{Wrapper Class}: Converts primitive data types into objects.

{\def\LTcaptype{none} % do not increment counter
\begin{longtable}[]{@{}ll@{}}
\toprule\noalign{}
\textbf{Primitive} & \textbf{Wrapper Class} \\
\midrule\noalign{}
\endhead
\bottomrule\noalign{}
\endlastfoot
int & Integer \\
char & Character \\
boolean & Boolean \\
double & Double \\
\end{longtable}
}

\begin{verbatim}
// Boxing
Integer obj = Integer.valueOf(10);
// Unboxing  
int value = obj.intValue();
\end{verbatim}

\end{solutionbox}
\begin{mnemonicbox}
``Wrap Primitives Into Objects''

\end{mnemonicbox}
\subsection*{Question 2(b) [4 marks]}\label{q2b}

\textbf{List out different features of java. Explain any two.}

\begin{solutionbox}

\textbf{Java Features}:

\begin{itemize}
\tightlist
\item
  \textbf{Simple}: Easy syntax, no pointers
\item
  \textbf{Platform Independent}: Write once, run anywhere\\
\item
  \textbf{Object Oriented}: Based on objects and classes
\item
  \textbf{Secure}: No explicit pointers, bytecode verification
\end{itemize}

\textbf{Detailed Explanation}:

\begin{itemize}
\tightlist
\item
  \textbf{Platform Independence}: Java bytecode runs on any platform
  with JVM
\item
  \textbf{Object Oriented}: Supports inheritance, encapsulation,
  polymorphism, abstraction
\end{itemize}

\end{solutionbox}
\begin{mnemonicbox}
``Simple Platform Object Security''

\end{mnemonicbox}
\subsection*{Question 2(c) [7 marks]}\label{q2c}

\textbf{What is method overriding? Explain with example.}

\begin{solutionbox}

\textbf{Method Overriding}: Child class provides specific implementation
of parent class method.

\begin{verbatim}
class Animal \{
    public void sound() \{
        System.out.println("Animal makes sound");
    \}
\}

class Dog extends Animal \{
    @Override
    public void sound() \{
        System.out.println("Dog barks");
    \}
\}

public class Test \{
    public static void main(String[] args) \{
        Animal a = new Dog();
        a.sound(); // Output: Dog barks
    \}
\}
\end{verbatim}

\textbf{Key Points}:

\begin{itemize}
\tightlist
\item
  \textbf{Runtime Polymorphism}: Method called based on object type
\item
  \textbf{@Override}: Annotation for method overriding
\item
  \textbf{Dynamic Binding}: Method resolution at runtime
\end{itemize}

\end{solutionbox}
\begin{mnemonicbox}
``Child Changes Parent Method''

\end{mnemonicbox}
\subsection*{Question 2(a OR) [3
marks]}\label{question-2a-or-3-marks}

\textbf{Explain Garbage collection in java.}

\begin{solutionbox}

\textbf{Garbage Collection}: Automatic memory management that removes
unused objects.

\begin{center}
\textbf{Mermaid Diagram (Code)}
\begin{verbatim}
{Shaded}
{Highlighting}[]
graph LR
    A[Object Created] {-{-}{} B[Object Used]}
    B {-{-}{} C[Object Unreferenced]}
    C {-{-}{} D[Garbage Collector]}
    D {-{-}{} E[Memory Freed]}
{Highlighting}
{Shaded}
\end{verbatim}
\end{center}

\textbf{Key Points}:

\begin{itemize}
\tightlist
\item
  \textbf{Automatic}: No manual memory deallocation
\item
  \textbf{Mark and Sweep}: Identifies and removes unused objects
\item
  \textbf{Heap Memory}: Works on heap memory area
\end{itemize}

\end{solutionbox}
\begin{mnemonicbox}
``Auto Clean Unused Objects''

\end{mnemonicbox}
\subsection*{Question 2(b OR) [4
marks]}\label{question-2b-or-4-marks}

\textbf{Explain static keyword with example.}

\begin{solutionbox}

\textbf{Static Keyword}: Belongs to class rather than instance.

\begin{verbatim}
class Student \{
    static String college = "GTU";  // Static variable
    String name;
    
    static void showCollege() \{     // Static method
        System.out.println("College: " + college);
    \}
\}
\end{verbatim}

\textbf{Static Features}:

\begin{itemize}
\tightlist
\item
  \textbf{Memory}: Loaded at class loading time
\item
  \textbf{Access}: Can be accessed without object
\item
  \textbf{Sharing}: Shared among all instances
\end{itemize}

\end{solutionbox}
\begin{mnemonicbox}
``Class Level Memory Sharing''

\end{mnemonicbox}
\subsection*{Question 2(c OR) [7
marks]}\label{question-2c-or-7-marks}

\textbf{What is constructor? Explain copy constructor with example.}

\begin{solutionbox}

\textbf{Constructor}: Special method to initialize objects.

\begin{verbatim}
class Person \{
    String name;
    int age;
    
    // Default constructor
    Person() \{
        name = "Unknown";
        age = 0;
    \}
    
    // Parameterized constructor
    Person(String n, int a) \{
        name = n;
        age = a;
    \}
    
    // Copy constructor
    Person(Person p) \{
        name = p.name;
        age = p.age;
    \}
\}
\end{verbatim}

\textbf{Constructor Types}:

\begin{itemize}
\tightlist
\item
  \textbf{Default}: No parameters
\item
  \textbf{Parameterized}: Takes parameters
\item
  \textbf{Copy}: Creates object from existing object
\end{itemize}

\end{solutionbox}
\begin{mnemonicbox}
``Default Parameter Copy''

\end{mnemonicbox}
\subsection*{Question 3(a) [3 marks]}\label{q3a}

\textbf{Explain super keyword with example.}

\begin{solutionbox}

\textbf{Super Keyword}: References parent class members.

\begin{verbatim}
class Vehicle \{
    String brand = "Generic";
\}

class Car extends Vehicle \{
    String brand = "Toyota";
    
    void display() \{
        System.out.println("Child: " + brand);
        System.out.println("Parent: " + super.brand);
    \}
\}
\end{verbatim}

\textbf{Super Uses}:

\begin{itemize}
\tightlist
\item
  \textbf{Variables}: Access parent class variables
\item
  \textbf{Methods}: Call parent class methods\\
\item
  \textbf{Constructor}: Call parent class constructor
\end{itemize}

\end{solutionbox}
\begin{mnemonicbox}
``Super Calls Parent''

\end{mnemonicbox}
\subsection*{Question 3(b) [4 marks]}\label{q3b}

\textbf{List out different types of inheritance. Explain multilevel
inheritance.}

\begin{solutionbox}

\textbf{Inheritance Types}:

{\def\LTcaptype{none} % do not increment counter
\begin{longtable}[]{@{}ll@{}}
\toprule\noalign{}
\textbf{Type} & \textbf{Description} \\
\midrule\noalign{}
\endhead
\bottomrule\noalign{}
\endlastfoot
Single & One parent, one child \\
Multilevel & Chain of inheritance \\
Hierarchical & One parent, multiple children \\
Multiple & Multiple parents (via interfaces) \\
\end{longtable}
}

\textbf{Multilevel Inheritance}:

\begin{verbatim}
class Animal \{
    void eat() \{ System.out.println("Eating"); \}
\}

class Mammal extends Animal \{
    void breathe() \{ System.out.println("Breathing"); \}
\}

class Dog extends Mammal \{
    void bark() \{ System.out.println("Barking"); \}
\}
\end{verbatim}

\end{solutionbox}
\begin{mnemonicbox}
``Single Multi Hierarchical Multiple''

\end{mnemonicbox}
\subsection*{Question 3(c) [7 marks]}\label{q3c}

\textbf{What is interface? Explain multiple inheritance with example.}

\begin{solutionbox}

\textbf{Interface}: Contract that defines what class must do, not how.

\begin{verbatim}
interface Flyable \{
    void fly();
\}

interface Swimmable \{
    void swim();
\}

class Duck implements Flyable, Swimmable \{
    public void fly() \{
        System.out.println("Duck is flying");
    \}
    
    public void swim() \{
        System.out.println("Duck is swimming");
    \}
\}
\end{verbatim}

\textbf{Interface Features}:

\begin{itemize}
\tightlist
\item
  \textbf{Multiple Inheritance}: Class can implement multiple interfaces
\item
  \textbf{Abstract Methods}: All methods are abstract by default
\item
  \textbf{Constants}: All variables are public, static, final
\end{itemize}

\end{solutionbox}
\begin{mnemonicbox}
``Multiple Abstract Constants''

\end{mnemonicbox}
\subsection*{Question 3(a OR) [3
marks]}\label{question-3a-or-3-marks}

\textbf{Explain final keyword with example.}

\begin{solutionbox}

\textbf{Final Keyword}: Restricts modification, inheritance, or
overriding.

\begin{verbatim}
final class Math \{           // Cannot be inherited
    final int PI = 3.14;     // Cannot be modified
    
    final void calculate() \{ // Cannot be overridden
        System.out.println("Calculating");
    \}
\}
\end{verbatim}

\textbf{Final Uses}:

\begin{itemize}
\tightlist
\item
  \textbf{Class}: Cannot be extended
\item
  \textbf{Method}: Cannot be overridden
\item
  \textbf{Variable}: Cannot be reassigned
\end{itemize}

\end{solutionbox}
\begin{mnemonicbox}
``Final Stops Changes''

\end{mnemonicbox}
\subsection*{Question 3(b OR) [4
marks]}\label{question-3b-or-4-marks}

\textbf{Explain different access controls in Java.}

\begin{solutionbox}

\textbf{Access Modifiers}:

{\def\LTcaptype{none} % do not increment counter
\begin{longtable}[]{@{}
  >{\raggedright\arraybackslash}p{(\linewidth - 8\tabcolsep) * \real{0.1667}}
  >{\raggedright\arraybackslash}p{(\linewidth - 8\tabcolsep) * \real{0.1905}}
  >{\raggedright\arraybackslash}p{(\linewidth - 8\tabcolsep) * \real{0.2143}}
  >{\raggedright\arraybackslash}p{(\linewidth - 8\tabcolsep) * \real{0.1667}}
  >{\raggedright\arraybackslash}p{(\linewidth - 8\tabcolsep) * \real{0.2619}}@{}}
\toprule\noalign{}
\begin{minipage}[b]{\linewidth}\raggedright
\textbf{Modifier}
\end{minipage} & \begin{minipage}[b]{\linewidth}\raggedright
\textbf{Same Class}
\end{minipage} & \begin{minipage}[b]{\linewidth}\raggedright
\textbf{Same Package}
\end{minipage} & \begin{minipage}[b]{\linewidth}\raggedright
\textbf{Subclass}
\end{minipage} & \begin{minipage}[b]{\linewidth}\raggedright
\textbf{Different Package}
\end{minipage} \\
\midrule\noalign{}
\endhead
\bottomrule\noalign{}
\endlastfoot
public & ✓ & ✓ & ✓ & ✓ \\
protected & ✓ & ✓ & ✓ & ✗ \\
default & ✓ & ✓ & ✗ & ✗ \\
private & ✓ & ✗ & ✗ & ✗ \\
\end{longtable}
}

\end{solutionbox}
\begin{mnemonicbox}
``Public Protected Default Private''

\end{mnemonicbox}
\subsection*{Question 3(c OR) [7
marks]}\label{question-3c-or-7-marks}

\textbf{What is package? Write steps to create a package and give
example of it.}

\begin{solutionbox}

\textbf{Package}: Group of related classes and interfaces.

\textbf{Steps to Create Package}:

\begin{enumerate}
\tightlist
\item
  \textbf{Declare}: Use package statement at top
\item
  \textbf{Compile}: javac -d . ClassName.java\\
\item
  \textbf{Run}: java packagename.ClassName
\end{enumerate}

\begin{verbatim}
// File: mypack/Calculator.java
package mypack;

public class Calculator \{
    public int add(int a, int b) \{
        return a + b;
    \}
\}

// File: Test.java
import mypack.Calculator;

public class Test \{
    public static void main(String[] args) \{
        Calculator calc = new Calculator();
        System.out.println(calc.add(5, 3));
    \}
\}
\end{verbatim}

\textbf{Package Benefits}:

\begin{itemize}
\tightlist
\item
  \textbf{Organization}: Groups related classes
\item
  \textbf{Access Control}: Package-level protection
\item
  \textbf{Namespace}: Avoids naming conflicts
\end{itemize}

\end{solutionbox}
\begin{mnemonicbox}
``Declare Compile Run''

\end{mnemonicbox}
\subsection*{Question 4(a) [3 marks]}\label{q4a}

\textbf{Explain thread priorities with suitable example.}

\begin{solutionbox}

\textbf{Thread Priority}: Determines thread execution order (1-10
scale).

\begin{verbatim}
class MyThread extends Thread \{
    public void run() \{
        System.out.println(getName() + " Priority: " + getPriority());
    \}
\}

public class ThreadPriorityExample \{
    public static void main(String[] args) \{
        MyThread t1 = new MyThread();
        MyThread t2 = new MyThread();
        
        t1.setPriority(Thread.MIN\_PRIORITY);  // 1
        t2.setPriority(Thread.MAX\_PRIORITY);  // 10
        
        t1.start();
        t2.start();
    \}
\}
\end{verbatim}

\textbf{Priority Constants}:

\begin{itemize}
\tightlist
\item
  \textbf{MIN\_PRIORITY}: 1
\item
  \textbf{NORM\_PRIORITY}: 5\\
\item
  \textbf{MAX\_PRIORITY}: 10
\end{itemize}

\end{solutionbox}
\begin{mnemonicbox}
``Min Normal Max''

\end{mnemonicbox}
\subsection*{Question 4(b) [4 marks]}\label{q4b}

\textbf{What is Thread? Explain Thread life cycle.}

\begin{solutionbox}

\textbf{Thread}: Lightweight process for concurrent execution.

\begin{verbatim}
stateDiagram{-v2}
        direction LR
    [*] {-{-} New}
    New {-{-} Runnable: start()}
    Runnable {-{-} Running: CPU allocation}
    Running {-{-} Blocked: wait(), sleep()}
    Blocked {-{-} Runnable: notify(), timeout}
    Running {-{-} Dead: complete}
\end{verbatim}

\textbf{Thread States}:

\begin{itemize}
\tightlist
\item
  \textbf{New}: Thread created but not started
\item
  \textbf{Runnable}: Ready to run
\item
  \textbf{Running}: Currently executing
\item
  \textbf{Blocked}: Waiting for resource
\item
  \textbf{Dead}: Execution completed
\end{itemize}

\end{solutionbox}
\begin{mnemonicbox}
``New Runnable Running Blocked Dead''

\end{mnemonicbox}
\subsection*{Question 4(c) [7 marks]}\label{q4c}

\textbf{Write a program in java that create the multiple threads by
implementing the Runnable interface.}

\begin{solutionbox}

\begin{verbatim}
class MyRunnable implements Runnable \{
    private String threadName;
    
    MyRunnable(String name) \{
        threadName = name;
    \}
    
    public void run() \{
for(int

i = 1; i {=} 5; i++) \{

            System.out.println(threadName + " {- Count: "} + i);
            try \{
                Thread.sleep(1000);
            \} catch(InterruptedException e) \{
                e.printStackTrace();
            \}
        \}
    \}
\}

public class MultipleThreads \{
    public static void main(String[] args) \{
        Thread t1 = new Thread(new MyRunnable("Thread{-1"}));
        Thread t2 = new Thread(new MyRunnable("Thread{-2"}));
        Thread t3 = new Thread(new MyRunnable("Thread{-3"}));
        
        t1.start();
        t2.start(); 
        t3.start();
    \}
\}
\end{verbatim}

\textbf{Key Points}:

\begin{itemize}
\tightlist
\item
  \textbf{Runnable Interface}: Better than extending Thread class
\item
  \textbf{Thread.sleep()}: Pauses thread execution
\item
  \textbf{Multiple Threads}: Run concurrently
\end{itemize}

\end{solutionbox}
\begin{mnemonicbox}
``Implement Runnable Start Multiple''

\end{mnemonicbox}
\subsection*{Question 4(a OR) [3
marks]}\label{question-4a-or-3-marks}

\textbf{List four different inbuilt exceptions. Explain any one inbuilt
exception.}

\begin{solutionbox}

\textbf{Inbuilt Exceptions}:

\begin{itemize}
\tightlist
\item
  \textbf{NullPointerException}: Accessing null object
\item
  \textbf{ArrayIndexOutOfBoundsException}: Invalid array index
\item
  \textbf{ArithmeticException}: Division by zero
\item
  \textbf{NumberFormatException}: Invalid number format
\end{itemize}

\textbf{ArithmeticException}: Thrown when arithmetic operation fails.

\begin{verbatim}
int result = 10 / 0; // Throws ArithmeticException
\end{verbatim}

\end{solutionbox}
\begin{mnemonicbox}
``Null Array Arithmetic Number''

\end{mnemonicbox}
\subsection*{Question 4(b OR) [4
marks]}\label{question-4b-or-4-marks}

\textbf{Explain Try and Catch with suitable example.}

\begin{solutionbox}

\textbf{Try-Catch}: Exception handling mechanism.

\begin{verbatim}
public class TryCatchExample \{
    public static void main(String[] args) \{
        try \{
            int[] arr = \{1, 2, 3\;}
            System.out.println(arr[5]); // Index out of bounds
        \}
        catch(ArrayIndexOutOfBoundsException e) \{
            System.out.println("Array index error: " + e.getMessage());
        \}
        finally \{
            System.out.println("Always executed");
        \}
    \}
\}
\end{verbatim}

\textbf{Exception Handling Flow}:

\begin{itemize}
\tightlist
\item
  \textbf{Try}: Code that may throw exception
\item
  \textbf{Catch}: Handles specific exceptions\\
\item
  \textbf{Finally}: Always executes
\end{itemize}

\end{solutionbox}
\begin{mnemonicbox}
``Try Catch Finally''

\end{mnemonicbox}
\subsection*{Question 4(c OR) [7
marks]}\label{question-4c-or-7-marks}

\textbf{What is Exception? Write a program that show the use of
Arithmetic Exception.}

\begin{solutionbox}

\textbf{Exception}: Runtime error that disrupts normal program flow.

\begin{verbatim}
public class ArithmeticExceptionExample \{
    public static void main(String[] args) \{
        Scanner sc = new Scanner(System.in);
        
        try \{
            System.out.print("Enter first number: ");
            int num1 = sc.nextInt();
            
            System.out.print("Enter second number: ");
            int num2 = sc.nextInt();
            
            int result = num1 / num2;
            System.out.println("Result: " + result);
        \}
        catch(ArithmeticException e) \{
            System.out.println("Error: Cannot divide by zero!");
        \}
        catch(Exception e) \{
            System.out.println("General error: " + e.getMessage());
        \}
        finally \{
            sc.close();
        \}
    \}
\}
\end{verbatim}

\textbf{Exception Types}:

\begin{itemize}
\tightlist
\item
  \textbf{Checked}: Compile-time exceptions
\item
  \textbf{Unchecked}: Runtime exceptions
\item
  \textbf{Error}: System-level problems
\end{itemize}

\end{solutionbox}
\begin{mnemonicbox}
``Runtime Error Disrupts Flow''

\end{mnemonicbox}
\subsection*{Question 5(a) [3 marks]}\label{q5a}

\textbf{Explain ArrayIndexOutOfBound Exception in Java with example.}

\begin{solutionbox}

\textbf{ArrayIndexOutOfBoundsException}: Thrown when accessing invalid
array index.

\begin{verbatim}
public class ArrayIndexExample \{
    public static void main(String[] args) \{
        int[] numbers = \{10, 20, 30\;}
        
        try \{
            System.out.println(numbers[5]); // Invalid index
        \}
        catch(ArrayIndexOutOfBoundsException e) \{
            System.out.println("Invalid array index: " + e.getMessage());
        \}
    \}
\}
\end{verbatim}

\textbf{Key Points}:

\begin{itemize}
\tightlist
\item
  \textbf{Valid Range}: 0 to array.length-1
\item
  \textbf{Negative Index}: Also throws exception
\item
  \textbf{Runtime Exception}: Unchecked exception
\end{itemize}

\end{solutionbox}
\begin{mnemonicbox}
``Array Index Range Check''

\end{mnemonicbox}
\subsection*{Question 5(b) [4 marks]}\label{q5b}

\textbf{Explain basics of stream classes.}

\begin{solutionbox}

\textbf{Stream Classes}: Handle input/output operations.

{\def\LTcaptype{none} % do not increment counter
\begin{longtable}[]{@{}ll@{}}
\toprule\noalign{}
\textbf{Stream Type} & \textbf{Classes} \\
\midrule\noalign{}
\endhead
\bottomrule\noalign{}
\endlastfoot
Byte Streams & InputStream, OutputStream \\
Character Streams & Reader, Writer \\
File Streams & FileInputStream, FileOutputStream \\
Buffered Streams & BufferedReader, BufferedWriter \\
\end{longtable}
}

\begin{center}
\textbf{Mermaid Diagram (Code)}
\begin{verbatim}
{Shaded}
{Highlighting}[]
graph TD
    A[Stream Classes] {-{-}{} B[Byte Streams]}
    A {-{-}{} C[Character Streams]}
    B {-{-}{} D[InputStream]}
    B {-{-}{} E[OutputStream]}
    C {-{-}{} F[Reader]}
    C {-{-}{} G[Writer]}
{Highlighting}
{Shaded}
\end{verbatim}
\end{center}

\textbf{Stream Features}:

\begin{itemize}
\tightlist
\item
  \textbf{Sequential}: Data flows in sequence
\item
  \textbf{One Direction}: Either input or output
\item
  \textbf{Automatic}: Handles low-level details
\end{itemize}

\end{solutionbox}
\begin{mnemonicbox}
``Byte Character File Buffered''

\end{mnemonicbox}
\subsection*{Question 5(c) [7 marks]}\label{q5c}

\textbf{Write a java program to create a text file and perform read
operation on the text file.}

\begin{solutionbox}

\begin{verbatim}
import java.io.*;

public class FileReadExample \{
    public static void main(String[] args) \{
        // Create and write to file
        try \{
            FileWriter writer = new FileWriter("sample.txt");
            writer.write("Hello World!{n}");
            writer.write("Java File Handling{n}");
            writer.write("GTU Exam 2024");
            writer.close();
            System.out.println("File created successfully");
        \}
        catch(IOException e) \{
            System.out.println("Error creating file: " + e.getMessage());
        \}
        
        // Read from file
        try \{
            BufferedReader reader = new BufferedReader(new FileReader("sample.txt"));
            String line;
            
            System.out.println("{n}File contents:");
            while((line = reader.readLine()) != null) \{
                System.out.println(line);
            \}
            reader.close();
        \}
        catch(IOException e) \{
            System.out.println("Error reading file: " + e.getMessage());
        \}
    \}
\}
\end{verbatim}

\textbf{Key Points}:

\begin{itemize}
\tightlist
\item
  \textbf{FileWriter}: Creates and writes to file
\item
  \textbf{BufferedReader}: Efficient reading
\item
  \textbf{Exception Handling}: Handle IOException
\end{itemize}

\end{solutionbox}
\begin{mnemonicbox}
``Create Write Read Close''

\end{mnemonicbox}
\subsection*{Question 5(a OR) [3
marks]}\label{question-5a-or-3-marks}

\textbf{Explain Divide by Zero Exception in Java with example.}

\begin{solutionbox}

\textbf{ArithmeticException}: Thrown during divide by zero operation.

\begin{verbatim}
public class DivideByZeroExample \{
    public static void main(String[] args) \{
        try \{
            int a = 10;
            int b = 0;
            int result = a / b;  // Throws ArithmeticException
            System.out.println("Result: " + result);
        \}
        catch(ArithmeticException e) \{
            System.out.println("Cannot divide by zero: " + e.getMessage());
        \}
    \}
\}
\end{verbatim}

\textbf{Key Points}:

\begin{itemize}
\tightlist
\item
  \textbf{Integer Division}: Only integer division by zero throws
  exception
\item
  \textbf{Floating Point}: Returns Infinity for floating point division
\item
  \textbf{Runtime Exception}: Unchecked exception
\end{itemize}

\end{solutionbox}
\begin{mnemonicbox}
``Zero Division Arithmetic Error''

\end{mnemonicbox}
\subsection*{Question 5(b OR) [4
marks]}\label{question-5b-or-4-marks}

\textbf{Explain java I/O process.}

\begin{solutionbox}

\textbf{Java I/O Process}: Mechanism for reading and writing data.

\begin{center}
\textbf{Mermaid Diagram (Code)}
\begin{verbatim}
{Shaded}
{Highlighting}[]
graph LR
    A[Data Source] {-{-}{} B[Input Stream]}
    B {-{-}{} C[Java Program]}
    C {-{-}{} D[Output Stream]}
    D {-{-}{} E[Data Destination]}
{Highlighting}
{Shaded}
\end{verbatim}
\end{center}

\textbf{I/O Components}:

\begin{itemize}
\tightlist
\item
  \textbf{Stream}: Sequence of data
\item
  \textbf{Buffer}: Temporary storage for efficiency
\item
  \textbf{File}: Persistent storage
\item
  \textbf{Network}: Remote data transfer
\end{itemize}

\textbf{I/O Types}:

\begin{itemize}
\tightlist
\item
  \textbf{Byte-oriented}: Raw data (images, videos)
\item
  \textbf{Character-oriented}: Text data
\item
  \textbf{Synchronous}: Blocking operations
\item
  \textbf{Asynchronous}: Non-blocking operations
\end{itemize}

\end{solutionbox}
\begin{mnemonicbox}
``Stream Buffer File Network''

\end{mnemonicbox}
\subsection*{Question 5(c OR) [7
marks]}\label{question-5c-or-7-marks}

\textbf{Write a java program to create a text file and perform write
operation on the text file.}

\begin{solutionbox}

\begin{verbatim}
import java.io.*;
import java.util.Scanner;

public class FileWriteExample \{
    public static void main(String[] args) \{
        Scanner sc = new Scanner(System.in);
        
        try \{
            // Create file with FileWriter
            FileWriter writer = new FileWriter("student.txt");
            
            System.out.println("Enter student details:");
            System.out.print("Name: ");
            String name = sc.nextLine();
            
            System.out.print("Roll Number: ");
            String rollNo = sc.nextLine();
            
            System.out.print("Branch: ");
            String branch = sc.nextLine();
            
            // Write data to file
            writer.write("Student Information{n}");
            writer.write("=================={n}");
            writer.write("Name: " + name + "{n}");
            writer.write("Roll Number: " + rollNo + "{n}");
            writer.write("Branch: " + branch + "{n}");
            writer.write("Date: " + new java.util.Date() + "{n}");
            
            writer.close();
            System.out.println("{n}Data written to file successfully!");
            
        \}
        catch(IOException e) \{
            System.out.println("Error writing to file: " + e.getMessage());
        \}
        finally \{
            sc.close();
        \}
    \}
\}
\end{verbatim}

\textbf{Key Points}:

\begin{itemize}
\tightlist
\item
  \textbf{FileWriter}: Writes character data to file
\item
  \textbf{BufferedWriter}: More efficient for large data
\item
  \textbf{Auto-close}: Use try-with-resources for automatic closing
\end{itemize}

\end{solutionbox}
\begin{mnemonicbox}
``Create Write Close Handle''

\end{mnemonicbox}

\end{document}
