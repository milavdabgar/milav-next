\documentclass{article}

% content/resources/templates/preamble.tex
\usepackage[margin=0.6in]{geometry}
\author{Milav Dabgar}
\usepackage{amsmath,amssymb,amsthm}
\usepackage{booktabs}
\usepackage{multirow}
\usepackage{xcolor}
\usepackage{tcolorbox}
\tcbuselibrary{breakable,skins}
\usepackage[colorlinks=true,linkcolor=blue]{hyperref}
\usepackage{titlesec}
\usepackage{enumitem}
\usepackage{tikz}
\usepackage{pgfplots}
\usepackage{circuitikz}
\usepackage[version=4]{mhchem}
\usepackage{longtable}
\usepackage{array}
\usepackage{float}
\usepackage{caption}
\usepackage{listings}

\lstset{
  basicstyle=\small\ttfamily,
  breaklines=true,
  breakatwhitespace=false,
  postbreak=\mbox{\textcolor{red}{$\hookrightarrow$}\space},
  float=false,
  numbers=left,
  numberstyle=\tiny\color{gray},
  numbersep=10pt,
  xleftmargin=2em,
  keywordstyle=\color{blue},
  commentstyle=\color{green!60!black},
  stringstyle=\color{purple},
  backgroundcolor=\color{gray!5},
  showstringspaces=false,
  tabsize=2,
  captionpos=b,
  keepspaces=true,
  columns=flexible
}

\pgfplotsset{compat=1.18}
\usetikzlibrary{shapes,arrows,positioning,calc,patterns,decorations.pathmorphing,decorations.markings,arrows.meta}

% Color scheme
\definecolor{headcolor}{RGB}{0,102,204}
\definecolor{keycolor}{RGB}{220,20,60}
\definecolor{solutioncolor}{RGB}{34,139,34}
\definecolor{mnemoniccolor}{RGB}{148,0,211}
\definecolor{codecolor}{RGB}{0,0,100}

% Spacing
\setlength{\parskip}{3pt}
\setlist[itemize]{nosep}
\setlist[enumerate]{nosep}

% Title formatting
\titleformat{\section}{\Large\bfseries\color{headcolor}}{\thesection}{1em}{}
\titleformat{\subsection}{\large\bfseries\color{headcolor}}{\thesubsection}{1em}{}

% Pandoc tightlist compatibility
\providecommand{\tightlist}{%
  \setlength{\itemsep}{0pt}\setlength{\parskip}{0pt}}

% Pandoc longtable compatibility
\newcounter{none}
\def\thenone{}


% content/resources/templates/gujarati-boxes.tex
\usepackage{fontspec}
\usepackage{polyglossia}

% Set Gujarati as main language (document is primarily in Gujarati)
% Note: gloss-gujarati.ldf doesn't exist in polyglossia, but it will use hyphenation patterns
\setdefaultlanguage{gujarati}
\setotherlanguage{english}

% Configure Gujarati font properly
% Use Language=Default to prevent polyglossia from trying to add language-specific features
% that don't exist for Gujarati, which causes "empty feature" warnings
\newfontfamily\gujaratifont[Script=Gujarati,AutoFakeBold=2.5,AutoFakeSlant=0.3]{Noto Sans Gujarati}
\setmainfont[Script=Gujarati,AutoFakeBold=2.5,AutoFakeSlant=0.3]{Noto Sans Gujarati}
% Use Noto Sans Gujarati for monospace to support Gujarati in text
\setmonofont[Scale=0.9]{Noto Sans Gujarati}

% Configure English to use the same font
\newfontfamily\englishfont[Script=Gujarati,AutoFakeBold=2.5,AutoFakeSlant=0.3]{Noto Sans Gujarati}

% Translations for polyglossia
\gappto\captionsgujarati{
  \renewcommand{\tablename}{કોષ્ટક}
  \renewcommand{\figurename}{આકૃતિ}
}

% Helper for TikZ nodes to ensure Gujarati font
\newcommand{\gu}[1]{{\gujaratifont #1}}

% Custom environments
\newtcolorbox{solutionbox}{
    breakable,
    enhanced,
    colback=solutioncolor!5!white,
    colframe=solutioncolor!75!black,
    fonttitle=\bfseries,
    title=જવાબ
}

\newtcolorbox{solutionboxnobreak}{
 colback=solutioncolor!5!white,
 colframe=solutioncolor!75!black,
 fonttitle=\bfseries,
 title=જવાબ
}

\newtcolorbox{keyformula}{
 breakable,
 enhanced,
 colback=keycolor!5!white,
 colframe=keycolor!75!black,
 fonttitle=\bfseries,
 title=રાસાયણિક સમીકરણ/સૂત્ર
}

\newtcolorbox{mnemonicbox}{
 breakable,
 enhanced,
 colback=mnemoniccolor!5!white,
 colframe=mnemoniccolor!75!black,
 fonttitle=\bfseries,
 title=મેમરી ટ્રીક
}


% Custom commands for GTU solutions
% This file defines semantic commands for consistent formatting

% Question command with automatic formatting
\newcommand{\question}[2]{%
  \section*{Question #1}%
  \textbf{#2}%
}

% OR question variant
\newcommand{\questionor}[2]{%
  \section*{Question #1 OR}%
  \textbf{#2}%
}

% Proper table environment with caption
\newenvironment{answertable}[1]{%
  \begin{table}[htbp]
  \centering
  \caption{#1}
}{%
  \end{table}
}

% Proper figure environment for diagrams
\newenvironment{answerdiagram}[1]{%
  \begin{figure}[htbp]
  \centering
  \caption{#1}
}{%
  \end{figure}
}

% Semantic markup for key terms
\newcommand{\keyword}[1]{\textbf{#1}}
\newcommand{\code}[1]{\texttt{#1}}
\newcommand{\classname}[1]{\texttt{#1}}
\newcommand{\methodname}[1]{\texttt{#1}}

% Proper quotation marks
\newcommand{\mnemonic}[1]{``#1''}


\title{ઑબ્જેક્ટ ઓરિએન્ટેડ પ્રોગ્રામિંગ વિથ જાવા (4341602) - સમર 2023 સોલ્યુશન}
\date{July 15, 2023}

\begin{document}
\maketitle

\questionmarks{1(a)}{3}{પ્રોસિજર-ઓરિએન્ટેડ પ્રોગ્રામિંગ (POP) અને ઑબ્જેક્ટ-ઓરિએન્ટેડ પ્રોગ્રામિંગ (OOP) વચ્ચે તફાવત કરો.}

\begin{solutionbox}
\begin{center}
\captionof{table}{POP vs OOP}
\begin{tabulary}{\linewidth}{|L|L|L|}
\hline
\textbf{પાસું} & \textbf{POP} & \textbf{OOP} \\ \hline
ધ્યાન & ફંક્શન્સ/પ્રોસિજર્સ & ઑબ્જેક્ટ્સ અને ક્લાસ્સિઝ \\ \hline
ડેટા સિક્યોરિટી & ઓછી સુરક્ષા, ગ્લોબલ ડેટા & વધુ સુરક્ષા, ડેટા encapsulation \\ \hline
સમસ્યા ઉકેલ & ટોપ-ડાઉન એપ્રોચ & બોટમ-અપ એપ્રોચ \\ \hline
કોડ પુનઃઉપયોગ & મર્યાદિત & inheritance દ્વારા વધુ \\ \hline
ઉદાહરણો & C, Pascal & Java, C++, Python \\ \hline
\end{tabulary}
\end{center}

\begin{itemize}
    \item \keyword{POP}: પ્રોગ્રામ ફંક્શન્સમાં વિભાજિત, ડેટા ફંક્શન્સ વચ્ચે વહે છે
    \item \keyword{OOP}: પ્રોગ્રામ ઑબ્જેક્ટ્સની આસપાસ ગોઠવાયેલું જેમાં ડેટા અને મેથડ્સ બંને હોય છે
\end{itemize}
\end{solutionbox}

\begin{mnemonicbox}
\mnemonic{POP Functions, OOP Objects}
\end{mnemonicbox}

\questionmarks{1(b)}{4}{યોગ્ય ઉદાહરણ સાથે inheritance માં Super કીવર્ડ સમજાવો.}

\begin{solutionbox}
\textbf{Super કીવર્ડ} નો ઉપયોગ child class માંથી parent class ના members ને access કરવા માટે થાય છે.

\begin{center}
\captionof{table}{Super કીવર્ડના ઉપયોગો}
\begin{tabulary}{\linewidth}{|L|L|L|}
\hline
\textbf{ઉપયોગ} & \textbf{હેતુ} & \textbf{ઉદાહરણ} \\ \hline
\code{super()} & parent constructor ને કૉલ કરે & \code{super(name, age)} \\ \hline
\code{super.method()} & parent method ને કૉલ કરે & \code{super.display()} \\ \hline
\code{super.variable} & parent variable ને access કરે & \code{super.name} \\ \hline
\end{tabulary}
\end{center}

\begin{lstlisting}[language=Java,caption={Super કીવર્ડ ઉદાહરણ}]
class Animal {
    String name = "Animal";
    void eat() { System.out.println("Animal eats"); }
}

class Dog extends Animal {
    String name = "Dog";
    void eat() {
        super.eat(); // parent method ને કૉલ કરે છે
        System.out.println("Dog eats bones");
    }
    void display() {
        System.out.println(super.name); // "Animal" પ્રિન્ટ કરે છે
    }
}
\end{lstlisting}
\end{solutionbox}

\begin{mnemonicbox}
\mnemonic{Super Calls Parent}
\end{mnemonicbox}

\questionmarks{1(c)}{7}{વ્યાખ્યાયિત કરો: મેથડ ઓવરરાઇડિંગ. મેથડ ઓવરરાઇડિંગ માટેના નિયમોની યાદી બનાવો. એક જાવા પ્રોગ્રામ લખો જે મેથડ ઓવરરાઇડિંગને implement કરે છે.}

\begin{solutionbox}
\textbf{મેથડ ઓવરરાઇડિંગ}: Child class પોતાની parent class ના method નું specific implementation આપે છે સમાન signature સાથે.

\begin{center}
\captionof{table}{મેથડ ઓવરરાઇડિંગના નિયમો}
\begin{tabulary}{\linewidth}{|L|L|}
\hline
\textbf{નિયમ} & \textbf{વર્ણન} \\ \hline
સમાન નામ & મેથડનું નામ સમાન હોવું જોઈએ \\ \hline
સમાન parameters & Parameter list બરાબર મેચ થવી જોઈએ \\ \hline
IS-A સંબંધ & inheritance હોવું જરૂરી \\ \hline
Access modifier & visibility ઘટાડી શકાતી નથી \\ \hline
Return type & સમાન અથવા covariant હોવું જોઈએ \\ \hline
\end{tabulary}
\end{center}

\begin{lstlisting}[language=Java,caption={મેથડ ઓવરરાઇડિંગ ઉદાહરણ}]
class Shape {
    void draw() {
        System.out.println("Drawing a shape");
    }
}

class Circle extends Shape {
    @Override
    void draw() {
        System.out.println("Drawing a circle");
    }
}

class Main {
    public static void main(String[] args) {
        Shape s = new Circle();
        s.draw(); // આઉટપુટ: Drawing a circle
    }
}
\end{lstlisting}
\end{solutionbox}

\begin{mnemonicbox}
\mnemonic{Override Same Signature}
\end{mnemonicbox}

\questionmarks{1(c OR)}{7}{વર્ણવો: ઇન્ટરફેસ. મલ્ટિપલ inheritance ને દર્શાવવા માટે interface નો ઉપયોગ કરીને જાવા પ્રોગ્રામ લખો.}

\begin{solutionbox}
\textbf{Interface}: blueprint જેમાં abstract methods અને constants હોય છે. Classes interfaces ને implement કરીને multiple inheritance પ્રાપ્ત કરે છે.

\begin{center}
\captionof{table}{Interface ની વિશેષતાઓ}
\begin{tabulary}{\linewidth}{|L|L|}
\hline
\textbf{વિશેષતા} & \textbf{વર્ણન} \\ \hline
Abstract methods & કોઈ implementation નથી (Java 8 પહેલાં) \\ \hline
Constants & બધા variables public static final છે \\ \hline
Multiple inheritance & Class અનેક interfaces implement કરી શકે \\ \hline
Default methods & Concrete methods (Java 8+) \\ \hline
\end{tabulary}
\end{center}

\begin{lstlisting}[language=Java,caption={Interface ઉદાહરણ}]
interface Flyable {
    void fly();
}

interface Swimmable {
    void swim();
}

class Duck implements Flyable, Swimmable {
    public void fly() {
        System.out.println("Duck flies");
    }
    
    public void swim() {
        System.out.println("Duck swims");
    }
}

class Main {
    public static void main(String[] args) {
        Duck d = new Duck();
        d.fly();
        d.swim();
    }
}
\end{lstlisting}
\end{solutionbox}

\begin{mnemonicbox}
\mnemonic{Interface Multiple Implementation}
\end{mnemonicbox}

\questionmarks{2(a)}{3}{ઉદાહરણ સાથે Java પ્રોગ્રામ સ્ટ્રક્ચર સમજાવો.}

\begin{solutionbox}
\textbf{Java પ્રોગ્રામ સ્ટ્રક્ચર} માં package, imports, class declaration અને main method હોય છે.

\begin{center}
\begin{tikzpicture}[node distance=1.5cm, auto]
    \node [gtu block] (pkg) {Package statement};
    \node [gtu block, below=0.5cm of pkg] (imp) {Import statements};
    \node [gtu block, below=0.5cm of imp] (cls) {Class declaration};
    \node [gtu block, below=0.5cm of cls] (vars) {Variables};
    \node [gtu block, below=0.5cm of vars] (meth) {Methods};
    
    \path [gtu arrow] (pkg) -- (imp);
    \path [gtu arrow] (imp) -- (cls);
    \path [gtu arrow] (cls) -- (vars);
    \path [gtu arrow] (vars) -- (meth);
\end{tikzpicture}
\captionof{figure}{Java પ્રોગ્રામ સ્ટ્રક્ચર}
\end{center}

\begin{lstlisting}[language=Java,caption={પ્રોગ્રામ સ્ટ્રક્ચર ઉદાહરણ}]
package com.example;        // Package
import java.util.*;         // Import

public class HelloWorld {   // Class
    static int count = 0;   // Variable
    
    public static void main(String[] args) { // Main method
        System.out.println("Hello World");
    }
}
\end{lstlisting}
\end{solutionbox}

\begin{mnemonicbox}
\mnemonic{Package Import Class Main}
\end{mnemonicbox}

\questionmarks{2(b)}{4}{યોગ્ય ઉદાહરણ સાથે static કીવર્ડ સમજાવો.}

\begin{solutionbox}
\textbf{Static કીવર્ડ} class નું છે instance નું નહીં. memory એક વાર allocate થાય છે.

\begin{center}
\captionof{table}{Static ના ઉપયોગો}
\begin{tabulary}{\linewidth}{|L|L|L|}
\hline
\textbf{પ્રકાર} & \textbf{વર્ણન} & \textbf{ઉદાહરણ} \\ \hline
Static variable & બધા objects દ્વારા shared & \code{static int count} \\ \hline
Static method & object વિના કૉલ થાય & \code{static void display()} \\ \hline
Static block & main પહેલાં execute થાય & \code{static \{ \}} \\ \hline
\end{tabulary}
\end{center}

\begin{lstlisting}[language=Java,caption={Static કીવર્ડ ઉદાહરણ}]
class Student {
    static String college = "GTU";  // static variable
    String name;
    
    static void showCollege() {     // static method
        System.out.println(college);
    }
    
    static {                        // static block
        System.out.println("Static block executed");
    }
}

class Main {
    public static void main(String[] args) {
        Student.showCollege(); // કોઈ object ની જરૂર નથી
    }
}
\end{lstlisting}
\end{solutionbox}

\begin{mnemonicbox}
\mnemonic{Static Shared by Class}
\end{mnemonicbox}

\questionmarks{2(c)}{7}{વ્યાખ્યાયિત કરો: કન્સ્ટ્રક્ટર. તેના પ્રકારોની યાદી બનાવો. પેરામીટરાઇઝ્ડ અને કોપી કન્સ્ટ્રક્ટરને યોગ્ય ઉદાહરણ સાથે સમજાવો.}

\begin{solutionbox}
\textbf{કન્સ્ટ્રક્ટર}: objects ને initialize કરવા માટેની special method, class જેવું જ નામ, કોઈ return type નથી.

\begin{center}
\captionof{table}{કન્સ્ટ્રક્ટરના પ્રકારો}
\begin{tabulary}{\linewidth}{|L|L|L|}
\hline
\textbf{પ્રકાર} & \textbf{વર્ણન} & \textbf{ઉદાહરણ} \\ \hline
Default & કોઈ parameters નથી & \code{Student()} \\ \hline
Parameterized & parameters સાથે & \code{Student(String n)} \\ \hline
Copy & object ની copy બનાવે & \code{Student(Student s)} \\ \hline
\end{tabulary}
\end{center}

\begin{lstlisting}[language=Java,caption={કન્સ્ટ્રક્ટર ઉદાહરણ}]
class Student {
    String name;
    int age;
    
    // Parameterized constructor
    Student(String n, int a) {
        name = n;
        age = a;
    }
    
    // Copy constructor
    Student(Student s) {
        name = s.name;
        age = s.age;
    }
    
    void display() {
        System.out.println(name + " " + age);
    }
}

class Main {
    public static void main(String[] args) {
        Student s1 = new Student("John", 20);  // Parameterized
        Student s2 = new Student(s1);          // Copy
        s1.display();
        s2.display();
    }
}
\end{lstlisting}
\end{solutionbox}

\begin{mnemonicbox}
\mnemonic{Constructor Initializes Objects}
\end{mnemonicbox}

\questionmarks{2(a OR)}{3}{જાવામાં પ્રિમિટિવ ડેટા પ્રકારો અને યુઝર ડિફાઇન્ડ ડેટા પ્રકારો સમજાવો.}

\begin{solutionbox}
\textbf{પ્રિમિટિવ ડેટા ટાઇપ્સ}: Java language દ્વારા આપવામાં આવેલા built-in types.
\textbf{યુઝર ડિફાઇન્ડ ટાઇપ્સ}: programmer દ્વારા classes વાપરીને બનાવવામાં આવેલા custom types.

\begin{center}
\captionof{table}{ડેટા ટાઇપ્સ}
\begin{tabulary}{\linewidth}{|L|L|L|L|}
\hline
\textbf{કેટેગરી} & \textbf{પ્રકારો} & \textbf{સાઇઝ} & \textbf{ઉદાહરણ} \\ \hline
Primitive & byte, short, int, long & 1,2,4,8 bytes & \code{int x = 10;} \\ \hline
Primitive & float, double & 4,8 bytes & \code{double d = 3.14;} \\ \hline
Primitive & char, boolean & 2,1 bytes & \code{char c = 'A';} \\ \hline
User Defined & Class, Interface, Array & Variable & \code{Student s;} \\ \hline
\end{tabulary}
\end{center}

\begin{itemize}
    \item \keyword{Primitive}: stack માં store થાય, ઝડપી access
    \item \keyword{User Defined}: heap માં store થાય, જટિલ operations
\end{itemize}
\end{solutionbox}

\begin{mnemonicbox}
\mnemonic{Primitive Built-in, User Custom}
\end{mnemonicbox}

\questionmarks{2(b OR)}{4}{યોગ્ય ઉદાહરણ સાથે this કીવર્ડ સમજાવો.}

\begin{solutionbox}
\textbf{This કીવર્ડ} વર્તમાન object instance ને refer કરે છે, instance અને local variables વચ્ચે ભેદ પાડવા માટે વાપરાય છે.

\begin{center}
\captionof{table}{This કીવર્ડના ઉપયોગો}
\begin{tabulary}{\linewidth}{|L|L|L|}
\hline
\textbf{ઉપયોગ} & \textbf{હેતુ} & \textbf{ઉદાહરણ} \\ \hline
\code{this.variable} & instance variable access & \code{this.name = name;} \\ \hline
\code{this.method()} & instance method call & \code{this.display();} \\ \hline
\code{this()} & constructor call & \code{this(name, age);} \\ \hline
\end{tabulary}
\end{center}

\begin{lstlisting}[language=Java,caption={This કીવર્ડ ઉદાહરણ}]
class Student {
    String name;
    int age;
    
    Student(String name, int age) {
        this.name = name;    // this instance અને parameter વચ્ચે
        this.age = age;      // ભેદ પાડે છે
    }
    
    void setData(String name) {
        this.name = name;    // this વર્તમાન object ને refer કરે
    }
    
    void display() {
        System.out.println(this.name + " " + this.age);
    }
}
\end{lstlisting}
\end{solutionbox}

\begin{mnemonicbox}
\mnemonic{This Current Object}
\end{mnemonicbox}

\questionmarks{2(c OR)}{7}{ઇનહેરિટન્સ વ્યાખ્યાયિત કરો. તેના પ્રકારોની યાદી બનાવો. multilevel અને hierarchical ઇનહેરિટન્સને યોગ્ય ઉદાહરણ સાથે સમજાવો.}

\begin{solutionbox}
\textbf{Inheritance}: mechanism જેમાં child class parent class ના properties અને methods મેળવે છે.

\begin{center}
\captionof{table}{Inheritance ના પ્રકારો}
\begin{tabulary}{\linewidth}{|L|L|L|}
\hline
\textbf{પ્રકાર} & \textbf{વર્ણન} & \textbf{સ્ટ્રક્ચર} \\ \hline
Single & એક parent, એક child & A $\to$ B \\ \hline
Multilevel & inheritance ની chain & A $\to$ B $\to$ C \\ \hline
Hierarchical & એક parent, અનેક children & A $\to$ B, A $\to$ C \\ \hline
Multiple & અનેક parents (interfaces) & B,C $\to$ A \\ \hline
\end{tabulary}
\end{center}

\begin{center}
\begin{tikzpicture}[node distance=1.5cm]
    \node (A) {Animal};
    \node [right=1cm of A] (B) {Mammal};
    \node [right=1cm of B] (C) {Dog};
    \draw [->] (A) -- (B);
    \draw [->] (B) -- (C);
\end{tikzpicture}
\captionof{figure}{Multilevel Inheritance}
\end{center}

\begin{lstlisting}[language=Java,caption={Multilevel Inheritance ઉદાહરણ}]
class Animal {
    void eat() { System.out.println("Animal eats"); }
}

class Mammal extends Animal {
    void breathe() { System.out.println("Mammal breathes"); }
}

class Dog extends Mammal {
    void bark() { System.out.println("Dog barks"); }
}
\end{lstlisting}

\begin{center}
\begin{tikzpicture}[level distance=1.5cm, sibling distance=2.5cm]
  \node {Shape}
    child { node {Circle} }
    child { node {Rectangle} }
    child { node {Triangle} };
\end{tikzpicture}
\captionof{figure}{Hierarchical Inheritance}
\end{center}

\begin{lstlisting}[language=Java,caption={Hierarchical Inheritance ઉદાહરણ}]
class Shape {
    void draw() { System.out.println("Drawing shape"); }
}

class Circle extends Shape {
    void drawCircle() { System.out.println("Drawing circle"); }
}

class Rectangle extends Shape {
    void drawRectangle() { System.out.println("Drawing rectangle"); }
}
\end{lstlisting}
\end{solutionbox}

\begin{mnemonicbox}
\mnemonic{Inheritance Properties Shares}
\end{mnemonicbox}

\questionmarks{3(a)}{3}{જાવામાં ટાઇપ કન્વર્ઝન અને કાસ્ટિંગ સમજાવો.}

\begin{solutionbox}
\textbf{Type Conversion}: એક data type ને બીજામાં બદલવું.
\textbf{Casting}: programmer દ્વારા explicit type conversion.

\begin{center}
\captionof{table}{Type Conversion}
\begin{tabulary}{\linewidth}{|L|L|L|}
\hline
\textbf{Type} & \textbf{Description} & \textbf{Example} \\ \hline
Implicit (Widening) & Automatic, smaller to larger & int to double \\ \hline
Explicit (Narrowing) & Manual, larger to smaller & double to int \\ \hline
\end{tabulary}
\end{center}

\begin{lstlisting}[language=Java,caption={Conversion vs Casting}]
// Implicit conversion
int i = 10;
double d = i;        // int થી double (automatic)

// Explicit casting
double x = 10.5;
int y = (int) x;     // double થી int (manual)

// String conversion
String s = String.valueOf(i);    // int થી String
int z = Integer.parseInt("123"); // String થી int
\end{lstlisting}
\end{solutionbox}

\begin{mnemonicbox}
\mnemonic{Implicit Auto, Explicit Manual}
\end{mnemonicbox}

\questionmarks{3(b)}{4}{Java માં ઉપયોગમાં લેવાતા વિવિધ visibility controls સમજાવો.}

\begin{solutionbox}
\textbf{Visibility Controls (Access Modifiers)}: classes, methods અને variables ના access ને control કરે છે.

\begin{center}
\captionof{table}{Access Modifiers}
\begin{tabulary}{\linewidth}{|L|C|C|C|C|}
\hline
\textbf{Modifier} & \textbf{Class} & \textbf{Pkg} & \textbf{Sub} & \textbf{Diff Pkg} \\ \hline
private & \checkmark & \times & \times & \times \\ \hline
default & \checkmark & \checkmark & \times & \times \\ \hline
protected & \checkmark & \checkmark & \checkmark & \times \\ \hline
public & \checkmark & \checkmark & \checkmark & \checkmark \\ \hline
\end{tabulary}
\end{center}

\begin{lstlisting}[language=Java,caption={Access Modifiers ઉદાહરણ}]
class Example {
    private int x = 10;      // માત્ર class અંદર
    int y = 20;              // Package level
    protected int z = 30;    // Package + subclass
    public int w = 40;       // દરેક જગ્યાએ
    
    private void method1() { }    // Private method
    public void method2() { }     // Public method
}
\end{lstlisting}
\end{solutionbox}

\begin{mnemonicbox}
\mnemonic{Private Package Protected Public}
\end{mnemonicbox}

\questionmarks{3(c)}{7}{વ્યાખ્યાયિત કરો: થ્રેડ. થ્રેડ બનાવવા માટે ઉપયોગમાં લેવાતી વિવિધ પદ્ધતિઓની સૂચિ બનાવો. થ્રેડની લાઇફ સાઇકલ વિગતવાર સમજાવો.}

\begin{solutionbox}
\textbf{Thread}: lightweight subprocess જે program ના અનેક ભાગોને concurrent execution ની મંજૂરી આપે છે.

\begin{center}
\captionof{table}{Thread બનાવવા પદ્ધતિઓ}
\begin{tabulary}{\linewidth}{|L|L|L|}
\hline
\textbf{પદ્ધતિ} & \textbf{વર્ણન} & \textbf{ઉદાહરણ} \\ \hline
Extending & Thread class inherit & \code{class MyThread extends Thread} \\ \hline
Boolean & Runnable implement & \code{class MyTask implements Runnable} \\ \hline
\end{tabulary}
\end{center}

\begin{center}
\begin{tikzpicture}[node distance=1.8cm, auto]
    \node [gtu state] (new) {NEW};
    \node [gtu state, right=of new] (runnable) {RUNNABLE};
    \node [gtu state, right=of runnable] (running) {RUNNING};
    \node [gtu state, below=of running] (blocked) {BLOCKED};
    \node [gtu state, right=of running] (terminated) {TERMINATED};

    \path [gtu arrow] (new) -- (runnable);
    \path [gtu arrow] (runnable) -- node[font=\small] {CPU} (running);
    \path [gtu arrow] (running) edge [bend left] node[font=\small] {yield()} (runnable);
    \path [gtu arrow] (running) -- node[font=\small] {wait()} (blocked);
    \path [gtu arrow] (blocked) edge [bend left] node[font=\small] {notify()} (runnable);
    \path [gtu arrow] (running) -- (terminated);
\end{tikzpicture}
\captionof{figure}{Thread Life Cycle}
\end{center}

\begin{center}
\captionof{table}{Thread States}
\begin{tabulary}{\linewidth}{|L|L|}
\hline
\textbf{સ્થિતિ} & \textbf{વર્ણન} \\ \hline
NEW & Thread બન્યું પણ શરૂ નથી થયું \\ \hline
RUNNABLE & ચાલવા તૈયાર, CPU ની રાહમાં \\ \hline
RUNNING & હાલમાં execute થઈ રહ્યું છે \\ \hline
BLOCKED & resource અથવા sleep ની રાહમાં \\ \hline
TERMINATED & execution પૂર્ણ થયું \\ \hline
\end{tabulary}
\end{center}

\begin{lstlisting}[language=Java,caption={Thread Creation Example}]
// પદ્ધતિ 1: Thread વિસ્તરવું
class MyThread extends Thread {
    public void run() {
        System.out.println("Thread running");
    }
}

// પદ્ધતિ 2: Runnable implement કરવું
class MyTask implements Runnable {
    public void run() {
        System.out.println("Task running");
    }
}

class Main {
    public static void main(String[] args) {
        MyThread t1 = new MyThread();
        Thread t2 = new Thread(new MyTask());
        t1.start();
        t2.start();
    }
}
\end{lstlisting}
\end{solutionbox}

\begin{mnemonicbox}
\mnemonic{Thread Concurrent Execution}
\end{mnemonicbox}

\questionmarks{3(a OR)}{3}{java માં JVM નો હેતુ સમજાવો.}

\begin{solutionbox}
\textbf{JVM (Java Virtual Machine)}: runtime environment જે Java bytecode execute કરે છે અને platform independence પ્રદાન કરે છે.

\begin{center}
\captionof{table}{JVM Components}
\begin{tabulary}{\linewidth}{|L|L|}
\hline
\textbf{Component} & \textbf{હેતુ} \\ \hline
Class Loader & .class files ને memory માં લોડ કરે \\ \hline
Execution Engine & bytecode execute કરે \\ \hline
Memory Area & heap અને stack memory ને manage કરે \\ \hline
Garbage Collector & automatic memory management \\ \hline
\end{tabulary}
\end{center}

\begin{center}
\begin{tikzpicture}[node distance=1.5cm]
    \node [gtu block] (java) {Java Source (.java)};
    \node [gtu block, below=0.8cm of java] (compiler) {Compiler (javac)};
    \node [gtu block, below=0.8cm of compiler] (bytecode) {Bytecode (.class)};
    \node [gtu block, below=0.8cm of bytecode] (jvm) {JVM (Platform Specific)};
    
    \path [gtu arrow] (java) -- (compiler);
    \path [gtu arrow] (compiler) -- (bytecode);
    \path [gtu arrow] (bytecode) -- (jvm);
\end{tikzpicture}
\captionof{figure}{JVM Workflow}
\end{center}

\begin{itemize}
    \item \keyword{Platform Independence}: "એકવાર લખો, દરેક જગ્યાએ ચલાવો"
    \item \keyword{Memory Management}: automatic garbage collection
    \item \keyword{Security}: Bytecode verification
\end{itemize}
\end{solutionbox}

\begin{mnemonicbox}
\mnemonic{JVM Java Virtual Machine}
\end{mnemonicbox}

\questionmarks{3(b OR)}{4}{વ્યાખ્યાયિત કરો: પેકેજ. યોગ્ય ઉદાહરણ સાથે પેકેજ બનાવવા માટેના પગલાંઓ લખો.}

\begin{solutionbox}
\textbf{Package}: સંબંધિત classes અને interfaces નો સંગ્રહ, namespace અને access control પ્રદાન કરે છે.

\begin{center}
\captionof{table}{Package ના ફાયદા}
\begin{tabulary}{\linewidth}{|L|L|}
\hline
\textbf{ફાયદો} & \textbf{વર્ણન} \\ \hline
Namespace & નામની ટકરાર ટાળે \\ \hline
Access Control & બહેતર encapsulation \\ \hline
Organization & logical grouping \\ \hline
Reusability & maintain કરવું સરળ \\ \hline
\end{tabulary}
\end{center}

\textbf{Package બનાવવાના પગલાં:}
\begin{enumerate}
    \item \textbf{Package declare કરો} file ની ટોચે
    \item \textbf{Directory structure બનાવો} package name મુજબ
    \item \textbf{Compile કરો} package structure સાથે
    \item \textbf{Import કરો} અન્ય classes માં
\end{enumerate}

\begin{lstlisting}[language=Java,caption={Package Example}]
// File: com/company/utilities/Calculator.java
package com.company.utilities;

public class Calculator {
    public int add(int a, int b) {
        return a + b;
    }
}

// File: Main.java
import com.company.utilities.Calculator;

class Main {
    public static void main(String[] args) {
        Calculator calc = new Calculator();
        System.out.println(calc.add(5, 3));
    }
}
\end{lstlisting}
\end{solutionbox}

\begin{mnemonicbox}
\mnemonic{Package Class Group}
\end{mnemonicbox}

\questionmarks{3(c OR)}{7}{થ્રેડમાં Synchronization ને યોગ્ય ઉદાહરણ સાથે સમજાવો.}

\begin{solutionbox}
\textbf{Synchronization}: shared resources ના access ને multiple threads દ્વારા control કરવાની mechanism, data inconsistency ટાળવા માટે.

\begin{center}
\captionof{table}{Synchronization ના પ્રકારો}
\begin{tabulary}{\linewidth}{|L|L|L|}
\hline
\textbf{પ્રકાર} & \textbf{વર્ણન} & \textbf{ઉપયોગ} \\ \hline
Synchronized method & આખી method lock થાય & \code{synchronized void} \\ \hline
Synchronized block & specific block lock થાય & \code{synchronized(obj)} \\ \hline
Static synchronization & Class level locking & \code{static synchronized} \\ \hline
\end{tabulary}
\end{center}

\begin{center}
\begin{tikzpicture}[node distance=1.5cm, auto]
    \node [gtu state] (t1) {Thread 1};
    \node [gtu state, right=2cm of t1] (t2) {Thread 2};
    \node [gtu decision, below=1.5cm of t1] (lock) {Lock};
    \node [gtu block, below=1.5cm of lock] (res) {Shared Resource};
    
    \path [gtu arrow] (t1) -- (lock);
    \path [gtu arrow] (t2) -- node {Wait} (lock);
    \path [gtu arrow] (lock) -- (res);
\end{tikzpicture}
\captionof{figure}{Synchronization Mechanism}
\end{center}

\begin{lstlisting}[language=Java,caption={Synchronization Example}]
class Counter {
    private int count = 0;
    
    public synchronized void increment() {
        count++;
    }
    
    public void decrement() {
        synchronized(this) {
            count--;
        }
    }
}
\end{lstlisting}
\end{solutionbox}

\begin{mnemonicbox}
\mnemonic{Synchronization Prevents Race Conditions}
\end{mnemonicbox}

\questionmarks{4(a)}{3}{સ્ટ્રિંગ ક્લાસ અને સ્ટ્રિંગબફર ક્લાસ વચ્ચે તફાવત કરો.}

\begin{solutionbox}
\begin{center}
\captionof{table}{String vs StringBuffer}
\begin{tabulary}{\linewidth}{|L|L|L|}
\hline
\textbf{પાસું} & \textbf{String} & \textbf{StringBuffer} \\ \hline
Mutability & Immutable (બદલાતું નથી) & Mutable (બદલાય છે) \\ \hline
Performance & concatenation માટે ધીમું & concatenation માટે ઝડપું \\ \hline
Memory & દર વખતે નવું object બનાવે & હાલનું object modify કરે \\ \hline
Thread Safety & Thread safe & Thread safe \\ \hline
\end{tabulary}
\end{center}

\begin{lstlisting}[language=Java,caption={String vs StringBuffer}]
// String - Immutable
String s1 = "Hello";
s1 = s1 + " World";  // નવું String object બનાવે છે

// StringBuffer - Mutable
StringBuffer sb = new StringBuffer("Hello");
sb.append(" World");  // હાલનું object modify કરે છે
\end{lstlisting}
\end{solutionbox}

\begin{mnemonicbox}
\mnemonic{String Immutable, StringBuffer Mutable}
\end{mnemonicbox}

\questionmarks{4(b)}{4}{એરેની 10 સંખ્યાઓનો સરવાળો અને સરેરાશ મેળવવા માટે જાવા પ્રોગ્રામ લખો.}

\begin{solutionbox}
\begin{lstlisting}[language=Java,caption={Array Sum and Average}]
class ArraySum {
    public static void main(String[] args) {
        int[] numbers = {10, 20, 30, 40, 50, 60, 70, 80, 90, 100};
        int sum = 0;
        
        // સરવાળો Calculate
        for(int i = 0; i < numbers.length; i++) {
            sum += numbers[i];
        }
        
        // સરેરાશ Calculate
        double average = (double) sum / numbers.length;
        
        System.out.println("Sum: " + sum);
        System.out.println("Average: " + average);
    }
}
\end{lstlisting}

\begin{center}
\textbf{આઉટપુટ:}
\begin{verbatim}
Sum: 550
Average: 55.0
\end{verbatim}
\end{center}
\end{solutionbox}

\begin{mnemonicbox}
\mnemonic{Loop Sum Divide Average}
\end{mnemonicbox}

\questionmarks{4(c)}{7}{I) યોગ્ય ઉદાહરણ સાથે abstract class સમજાવો. II) યોગ્ય ઉદાહરણ સાથે final class સમજાવો.}

\begin{solutionbox}
\textbf{I) Abstract Class}: class જેનું instantiation થઈ શકતું નથી, abstract methods હોય છે.

\begin{center}
\captionof{table}{Abstract Class ની વિશેષતાઓ}
\begin{tabulary}{\linewidth}{|L|L|}
\hline
\textbf{વિશેષતા} & \textbf{વર્ણન} \\ \hline
Cannot instantiate & object બનાવી શકાતું નથી \\ \hline
Abstract methods & implementation વિનાની methods \\ \hline
Inheritance & Subclasses implement methods \\ \hline
\end{tabulary}
\end{center}

\begin{lstlisting}[language=Java,caption={Abstract Class Example}]
abstract class Shape {
    abstract void draw();
}

class Circle extends Shape {
    void draw() {
        System.out.println("Drawing Circle");
    }
}
\end{lstlisting}

\textbf{II) Final Class}: class જેનું extension થઈ શકતું નથી.

\begin{center}
\captionof{table}{Final Class ની વિશેષતાઓ}
\begin{tabulary}{\linewidth}{|L|L|}
\hline
\textbf{વિશેષતા} & \textbf{વર્ણન} \\ \hline
No inheritance & extend કરી શકાતું નથી \\ \hline
Security & modification અટકાવે છે \\ \hline
ઉદાહરણો & String, Integer, System \\ \hline
\end{tabulary}
\end{center}

\begin{lstlisting}[language=Java,caption={Final Class Example}]
final class FinalClass {
    void display() {
        System.out.println("This is final class");
    }
}
// class Sub extends FinalClass { } // ભૂલ
\end{lstlisting}
\end{solutionbox}

\begin{mnemonicbox}
\mnemonic{Abstract Incomplete, Final Complete}
\end{mnemonicbox}

\questionmarks{4(a OR)}{3}{જાવામાં Garbage Collection સમજાવો.}

\begin{solutionbox}
\textbf{Garbage Collection}: automatic memory management process જે heap memory માંથી unused objects ને remove કરે છે.

\begin{center}
\captionof{table}{GC ના ફાયદા}
\begin{tabulary}{\linewidth}{|L|L|}
\hline
\textbf{ફાયદો} & \textbf{વર્ણન} \\ \hline
Automatic & manual memory management નથી \\ \hline
Memory leak prevention & unreferenced objects દૂર કરે \\ \hline
Performance & memory usage optimize કરે \\ \hline
Safety & memory errors અટકાવે \\ \hline
\end{tabulary}
\end{center}

\begin{center}
\begin{tikzpicture}[node distance=1.5cm]
    \node [gtu block] (new) {Object created (new)};
    \node [gtu block, below=0.8cm of new] (use) {Object in use};
    \node [gtu block, below=0.8cm of use] (noref) {No references (Unreachable)};
    \node [gtu block, below=0.8cm of noref] (gc) {Garbage Collector removes};
    
    \path [gtu arrow] (new) -- (use);
    \path [gtu arrow] (use) -- (noref);
    \path [gtu arrow] (noref) -- (gc);
\end{tikzpicture}
\captionof{figure}{Garbage Collection Process}
\end{center}
\end{solutionbox}

\begin{mnemonicbox}
\mnemonic{GC Automatic Memory Cleanup}
\end{mnemonicbox}

\questionmarks{4(b OR)}{4}{'Divide by Zero' એરર માટે યુઝર ડિફાઇન્ડ એક્સેપ્શન હેન્ડલ કરવા માટે જાવા પ્રોગ્રામ લખો.}

\begin{solutionbox}
\begin{lstlisting}[language=Java,caption={User Defined Exception}]
class DivideByZeroException extends Exception {
    public DivideByZeroException(String message) {
        super(message);
    }
}

class Calculator {
    public static double divide(int a, int b) throws DivideByZeroException {
        if(b == 0) {
            throw new DivideByZeroException("શૂન્ય થી ભાગ ન આપી શકાય!");
        }
        return (double) a / b;
    }
}

class Main {
    public static void main(String[] args) {
        try {
            Calculator.divide(10, 0);
        } catch(DivideByZeroException e) {
            System.out.println("ભૂલ: " + e.getMessage());
        }
    }
}
\end{lstlisting}
\end{solutionbox}

\begin{mnemonicbox}
\mnemonic{Custom Exception Handle Error}
\end{mnemonicbox}

\questionmarks{4(c OR)}{7}{Multiple try block અને multiple catch block exception દર્શાવવા માટે જાવા પ્રોગ્રામ લખો.}

\begin{solutionbox}
\begin{lstlisting}[language=Java,caption={Multiple Try-Catch Blocks}]
class MultipleExceptionDemo {
    public static void main(String[] args) {
        // પહેલો try block
        try {
            int[] arr = {1, 2, 3};
            System.out.println(arr[5]); // IndexOutOfBounds
        } 
        catch(ArrayIndexOutOfBoundsException e) {
            System.out.println("Index ભૂલ: " + e.getMessage());
        }
        
        // બીજો try block
        try {
            String str = null;
            System.out.println(str.length()); // NullPointer
        }
        catch(NullPointerException e) {
            System.out.println("Null ભૂલ: " + e.getMessage());
        }
        
        // Multiple catch
        try {
            int result = 10 / 0;  // Arithmetic
        }
        catch(ArithmeticException e) {
            System.out.println("Math ભૂલ: " + e.getMessage());
        }
        catch(Exception e) {
            System.out.println("અન્ય ભૂલ: " + e.getMessage());
        }
        finally {
            System.out.println("પ્રોગ્રામ પૂર્ણ થયું");
        }
    }
}
\end{lstlisting}
\end{solutionbox}

\begin{mnemonicbox}
\mnemonic{Multiple Try Multiple Catch}
\end{mnemonicbox}

\questionmarks{5(a)}{3}{ફાઇલ બનાવવા અને આ ફાઇલ પર write operation કરવા માટે જાવામાં પ્રોગ્રામ લખો.}

\begin{solutionbox}
\begin{lstlisting}[language=Java,caption={File Write Example}]
import java.io.*;

class FileWriteDemo {
    public static void main(String[] args) {
        try {
            // ફાઇલ બનાવો
            File file = new File("demo.txt");
            
            // FileWriter object બનાવો
            FileWriter writer = new FileWriter(file);
            
            // ફાઇલમાં ડેટા લખો
            writer.write("નમસ્તે દુનિયા!\n");
            writer.write("આ Java ફાઇલ લેખન ડેમો છે.\n");
            
            writer.close();
            System.out.println("ફાઇલ સફળતાપૂર્વક બનાવવામાં આવી!");
            
        } catch(IOException e) {
            System.out.println("ભૂલ: " + e.getMessage());
        }
    }
}
\end{lstlisting}
\end{solutionbox}

\begin{mnemonicbox}
\mnemonic{File Writer Write Close}
\end{mnemonicbox}

\questionmarks{5(b)}{4}{Throw અને finally ને Exception Handling માં ઉદાહરણ સાથે સમજાવો.}

\begin{solutionbox}
\textbf{Throw}: keyword જેનો ઉપયોગ explicitly exception throw કરવા માટે થાય.
\textbf{Finally}: block જે exception આવે કે ન આવે હંમેશા execute થાય છે.

\begin{center}
\captionof{table}{Throw vs Finally}
\begin{tabulary}{\linewidth}{|L|L|L|}
\hline
\textbf{કીવર્ડ} & \textbf{હેતુ} & \textbf{ઉપયોગ} \\ \hline
throw & Explicitly exception throw કરે & \code{throw new Ex()} \\ \hline
finally & હંમેશા cleanup code execute કરે & \code{finally \{ \}} \\ \hline
\end{tabulary}
\end{center}

\begin{lstlisting}[language=Java,caption={Throw and Finally}]
class ThrowFinallyDemo {
    public static void checkAge(int age) throws Exception {
        if(age < 18) throw new Exception("અમાન્ય ઉંમર");
    }
    
    public static void main(String[] args) {
        try {
            checkAge(15);
        } catch(Exception e) {
            System.out.println("ભૂલ: " + e.getMessage());
        } finally {
            System.out.println("હંમેશા execute થાય છે");
        }
    }
}
\end{lstlisting}
\end{solutionbox}

\begin{mnemonicbox}
\mnemonic{Throw Exception, Finally Always}
\end{mnemonicbox}

\questionmarks{5(c)}{7}{વર્ણવો: પોલીમોર્ફિઝમ. જાવામાં યોગ્ય ઉદાહરણ સાથે રન ટાઇમ પોલીમોર્ફિઝમ સમજાવો.}

\begin{solutionbox}
\textbf{Polymorphism}: એક interface, અનેક implementations. Object તેના actual type પર આધાર રાખીને અલગ રીતે વર્તે છે.

\begin{center}
\captionof{table}{Polymorphism ના પ્રકારો}
\begin{tabulary}{\linewidth}{|L|L|L|}
\hline
\textbf{પ્રકાર} & \textbf{વર્ણન} & \textbf{ક્યારે નક્કી થાય} \\ \hline
Compile-time & Method overloading & Compilation વખતે \\ \hline
Run-time & Method overriding & Execution વખતે \\ \hline
\end{tabulary}
\end{center}

\begin{center}
\begin{tikzpicture}[node distance=1.5cm, auto]
    \node (A) {Animal Reference};
    \node [right=2cm of A, yshift=1cm] (B) {Dog Object -- Dog Sound};
    \node [right=2cm of A, yshift=-1cm] (C) {Cat Object -- Cat Sound};
    
    \draw [->] (A) -- (B);
    \draw [->] (A) -- (C);
\end{tikzpicture}
\captionof{figure}{Runtime Polymorphism}
\end{center}

\begin{lstlisting}[language=Java,caption={Runtime Polymorphism}]
class Animal {
    void makeSound() { System.out.println("પ્રાણી અવાજ કરે છે"); }
}

class Dog extends Animal {
    void makeSound() { System.out.println("કૂતરો ભસે છે"); }
}

class Cat extends Animal {
    void makeSound() { System.out.println("બિલાડી મ્યાં કરે છે"); }
}

class Main {
    public static void main(String[] args) {
        Animal a;
        
        a = new Dog();
        a.makeSound(); // કૂતરો ભસે છે
        
        a = new Cat();
        a.makeSound(); // બિલાડી મ્યાં કરે છે
    }
}
\end{lstlisting}
\end{solutionbox}

\begin{mnemonicbox}
\mnemonic{Polymorphism Many Forms Runtime}
\end{mnemonicbox}

\questionmarks{5(a OR)}{3}{જાવામાં એક પ્રોગ્રામ લખો જે બાઇટ બાય બાઇટ ફાઇલના કન્ટેન્ટ વાંચે અને તેને બીજી ફાઇલમાં કોપી કરે.}

\begin{solutionbox}
\begin{lstlisting}[language=Java,caption={Byte Stream File Copy}]
import java.io.*;

class FileCopyDemo {
    public static void main(String[] args) {
        try {
            FileInputStream input = new FileInputStream("in.txt");
            FileOutputStream output = new FileOutputStream("out.txt");
            
            int byteData;
            while((byteData = input.read()) != -1) {
                output.write(byteData);
            }
            
            input.close();
            output.close();
        } catch(IOException e) {
            System.out.println("ભૂલ: " + e.getMessage());
        }
    }
}
\end{lstlisting}
\end{solutionbox}

\begin{mnemonicbox}
\mnemonic{Read Byte Write Byte}
\end{mnemonicbox}

\questionmarks{5(b OR)}{4}{Java સાથે ઉપલબ્ધ વિવિધ I/O વર્ગો સમજાવો.}

\begin{solutionbox}
\begin{center}
\captionof{table}{Java I/O Classes}
\begin{tabulary}{\linewidth}{|L|L|L|}
\hline
\textbf{Class Type} & \textbf{Class Name} & \textbf{હેતુ} \\ \hline
Byte Stream & FileInputStream & ફાઇલથી bytes વાંચે \\ \hline
Byte Stream & FileOutputStream & ફાઇલમાં bytes લખે \\ \hline
Char Stream & FileReader & ફાઇલથી chars વાંચે \\ \hline
Char Stream & FileWriter & ફાઇલમાં chars લખે \\ \hline
Buffered & BufferedReader & કાર્યક્ષમ વાંચન \\ \hline
Buffered & BufferedWriter & કાર્યક્ષમ લખવુ \\ \hline
\end{tabulary}
\end{center}

\begin{center}
\begin{tikzpicture}[node distance=1.2cm]
    \node [gtu block] (io) {Java I/O};
    \node [gtu block, below left=1cm of io] (byte) {Byte Stream};
    \node [gtu block, below right=1cm of io] (char) {Character Stream};
    
    \node [gtu block, below=0.5cm of byte] (fis) {FileInputStream};
    \node [gtu block, below=0.5cm of fis] (fos) {FileOutputStream};
    
    \node [gtu block, below=0.5cm of char] (fr) {FileReader};
    \node [gtu block, below=0.5cm of fr] (fw) {FileWriter};
    
    \path [gtu arrow] (io) -- (byte);
    \path [gtu arrow] (io) -- (char);
    \path [gtu arrow] (byte) -- (fis);
    \path [gtu arrow] (byte) -- (fos);
    \path [gtu arrow] (char) -- (fr);
    \path [gtu arrow] (char) -- (fw);
\end{tikzpicture}
\captionof{figure}{I/O Class Hierarchy}
\end{center}
\end{solutionbox}

\begin{mnemonicbox}
\mnemonic{Byte Character Buffered Streams}
\end{mnemonicbox}

\questionmarks{5(c OR)}{7}{જાવા પ્રોગ્રામ લખો જે બે થ્રેડોને એક્ઝિક્યુટ કરે છે. એક થ્રેડ દર 3 સેકન્ડે "Java Programming" દર્શાવે છે, અને બીજો દર 6 સેકન્ડે "Semester - 4th IT" દર્શાવે છે.}

\begin{solutionbox}
\begin{lstlisting}[language=Java,caption={Thread Timing Example}]
class JavaThread extends Thread {
    public void run() {
        try {
            while(true) {
                System.out.println("Java Programming");
                Thread.sleep(3000);
            }
        } catch(InterruptedException e) {}
    }
}

class SemThread extends Thread {
    public void run() {
        try {
            while(true) {
                System.out.println("Semester - 4th IT");
                Thread.sleep(6000);
            }
        } catch(InterruptedException e) {}
    }
}

class Main {
    public static void main(String[] args) {
        new JavaThread().start();
        new SemThread().start();
    }
}
\end{lstlisting}
\end{solutionbox}

\begin{mnemonicbox}
\mnemonic{Two Threads Different Timing}
\end{mnemonicbox}

\end{document}
