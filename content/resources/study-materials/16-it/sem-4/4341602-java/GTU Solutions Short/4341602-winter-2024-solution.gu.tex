\documentclass{article}

% content/resources/templates/preamble.tex
\usepackage[margin=0.6in]{geometry}
\author{Milav Dabgar}
\usepackage{amsmath,amssymb,amsthm}
\usepackage{booktabs}
\usepackage{multirow}
\usepackage{xcolor}
\usepackage{tcolorbox}
\tcbuselibrary{breakable,skins}
\usepackage[colorlinks=true,linkcolor=blue]{hyperref}
\usepackage{titlesec}
\usepackage{enumitem}
\usepackage{tikz}
\usepackage{pgfplots}
\usepackage{circuitikz}
\usepackage[version=4]{mhchem}
\usepackage{longtable}
\usepackage{array}
\usepackage{float}
\usepackage{caption}
\usepackage{listings}

\lstset{
  basicstyle=\small\ttfamily,
  breaklines=true,
  breakatwhitespace=false,
  postbreak=\mbox{\textcolor{red}{$\hookrightarrow$}\space},
  float=false,
  numbers=left,
  numberstyle=\tiny\color{gray},
  numbersep=10pt,
  xleftmargin=2em,
  keywordstyle=\color{blue},
  commentstyle=\color{green!60!black},
  stringstyle=\color{purple},
  backgroundcolor=\color{gray!5},
  showstringspaces=false,
  tabsize=2,
  captionpos=b,
  keepspaces=true,
  columns=flexible
}

\pgfplotsset{compat=1.18}
\usetikzlibrary{shapes,arrows,positioning,calc,patterns,decorations.pathmorphing,decorations.markings,arrows.meta}

% Color scheme
\definecolor{headcolor}{RGB}{0,102,204}
\definecolor{keycolor}{RGB}{220,20,60}
\definecolor{solutioncolor}{RGB}{34,139,34}
\definecolor{mnemoniccolor}{RGB}{148,0,211}
\definecolor{codecolor}{RGB}{0,0,100}

% Spacing
\setlength{\parskip}{3pt}
\setlist[itemize]{nosep}
\setlist[enumerate]{nosep}

% Title formatting
\titleformat{\section}{\Large\bfseries\color{headcolor}}{\thesection}{1em}{}
\titleformat{\subsection}{\large\bfseries\color{headcolor}}{\thesubsection}{1em}{}

% Pandoc tightlist compatibility
\providecommand{\tightlist}{%
  \setlength{\itemsep}{0pt}\setlength{\parskip}{0pt}}

% Pandoc longtable compatibility
\newcounter{none}
\def\thenone{}


% content/resources/templates/gujarati-boxes.tex
\usepackage{fontspec}
\usepackage{polyglossia}

% Set Gujarati as main language (document is primarily in Gujarati)
% Note: gloss-gujarati.ldf doesn't exist in polyglossia, but it will use hyphenation patterns
\setdefaultlanguage{gujarati}
\setotherlanguage{english}

% Configure Gujarati font properly
% Use Language=Default to prevent polyglossia from trying to add language-specific features
% that don't exist for Gujarati, which causes "empty feature" warnings
\newfontfamily\gujaratifont[Script=Gujarati,AutoFakeBold=2.5,AutoFakeSlant=0.3]{Noto Sans Gujarati}
\setmainfont[Script=Gujarati,AutoFakeBold=2.5,AutoFakeSlant=0.3]{Noto Sans Gujarati}
% Use Noto Sans Gujarati for monospace to support Gujarati in text
\setmonofont[Scale=0.9]{Noto Sans Gujarati}

% Configure English to use the same font
\newfontfamily\englishfont[Script=Gujarati,AutoFakeBold=2.5,AutoFakeSlant=0.3]{Noto Sans Gujarati}

% Translations for polyglossia
\gappto\captionsgujarati{
  \renewcommand{\tablename}{કોષ્ટક}
  \renewcommand{\figurename}{આકૃતિ}
}

% Helper for TikZ nodes to ensure Gujarati font
\newcommand{\gu}[1]{{\gujaratifont #1}}

% Custom environments
\newtcolorbox{solutionbox}{
    breakable,
    enhanced,
    colback=solutioncolor!5!white,
    colframe=solutioncolor!75!black,
    fonttitle=\bfseries,
    title=જવાબ
}

\newtcolorbox{solutionboxnobreak}{
 colback=solutioncolor!5!white,
 colframe=solutioncolor!75!black,
 fonttitle=\bfseries,
 title=જવાબ
}

\newtcolorbox{keyformula}{
 breakable,
 enhanced,
 colback=keycolor!5!white,
 colframe=keycolor!75!black,
 fonttitle=\bfseries,
 title=રાસાયણિક સમીકરણ/સૂત્ર
}

\newtcolorbox{mnemonicbox}{
 breakable,
 enhanced,
 colback=mnemoniccolor!5!white,
 colframe=mnemoniccolor!75!black,
 fonttitle=\bfseries,
 title=મેમરી ટ્રીક
}


% Custom commands for GTU solutions
% This file defines semantic commands for consistent formatting

% Question command with automatic formatting
\newcommand{\question}[2]{%
  \section*{Question #1}%
  \textbf{#2}%
}

% OR question variant
\newcommand{\questionor}[2]{%
  \section*{Question #1 OR}%
  \textbf{#2}%
}

% Proper table environment with caption
\newenvironment{answertable}[1]{%
  \begin{table}[htbp]
  \centering
  \caption{#1}
}{%
  \end{table}
}

% Proper figure environment for diagrams
\newenvironment{answerdiagram}[1]{%
  \begin{figure}[htbp]
  \centering
  \caption{#1}
}{%
  \end{figure}
}

% Semantic markup for key terms
\newcommand{\keyword}[1]{\textbf{#1}}
\newcommand{\code}[1]{\texttt{#1}}
\newcommand{\classname}[1]{\texttt{#1}}
\newcommand{\methodname}[1]{\texttt{#1}}

% Proper quotation marks
\newcommand{\mnemonic}[1]{``#1''}


\title{Object Oriented Programming with JAVA (4341602) - Winter 2024 Solution}
\date{November 26, 2024}

\begin{document}
\maketitle

\questionmarks{1(અ)}{3}{OOP અને POP વચ્ચેનો તફાવત લખો.}

\begin{solutionbox}
\begin{center}
\captionof{table}{OOP vs POP}
\begin{tabulary}{\linewidth}{|L|L|L|}
\hline
\textbf{પાસાં} & \textbf{OOP} & \textbf{POP} \\ \hline
\textbf{અભિગમ} & બોટમ-અપ અભિગમ & ટોપ-ડાઉન અભિગમ \\ \hline
\textbf{ફોકસ} & ઓબ્જેક્ટ અને ક્લાસ & ફંક્શન અને પ્રોસીજર \\ \hline
\textbf{ડેટા સિક્યોરિટી} & એન્કેપ્સુલેશન દ્વારા ડેટા હાઇડિંગ & ડેટા હાઇડિંગ નથી \\ \hline
\textbf{પ્રોબ્લેમ સોલ્વિંગ} & સમસ્યાને ઓબ્જેક્ટમાં વિભાજિત કરો & સમસ્યાને ફંક્શનમાં વિભાજિત કરો \\ \hline
\end{tabulary}
\end{center}
\end{solutionbox}

\begin{mnemonicbox}
\mnemonic{Objects Bottom, Procedures Top}
\end{mnemonicbox}

\questionmarks{1(બ)}{4}{બાઇટ કોડ શું છે? JVM ને વિગતવાર સમજાવો.}

\begin{solutionbox}
\textbf{બાઇટ કોડ} (Byte Code): Java compiler દ્વારા સોર્સ કોડમાંથી જનરેટ થતો પ્લેટફોર્મ-ઇન્ડિપેન્ડન્ટ ઇન્ટરમીડિયેટ કોડ.

\begin{center}
\begin{tikzpicture}[auto, node distance=2cm]
    \node [gtu block] (source) {Java Source Code};
    \node [gtu block, right=of source] (compiler) {Java Compiler javac};
    \node [gtu block, below=of compiler] (bytecode) {Byte Code .class};
    \node [gtu block, left=of bytecode] (jvm) {JVM};
    \node [gtu block, left=of jvm] (machine) {Machine Code};

    \path [gtu arrow] (source) -- (compiler);
    \path [gtu arrow] (compiler) -- (bytecode);
    \path [gtu arrow] (bytecode) -- (jvm);
    \path [gtu arrow] (jvm) -- (machine);
\end{tikzpicture}
\end{center}

\textbf{JVM કોમ્પોનન્ટ્સ}:
\begin{itemize}
    \item \textbf{Class Loader}: .class ફાઇલોને મેમરીમાં લોડ કરે છે
    \item \textbf{Memory Area}: Heap, stack, method area સ્ટોરેજ
    \item \textbf{Execution Engine}: બાઇટકોડને ઇન્ટરપ્રેટ અને એક્ઝિક્યુટ કરે છે
    \item \textbf{Garbage Collector}: ઓટોમેટિક મેમરી મેનેજમેન્ટ
\end{itemize}
\end{solutionbox}

\begin{mnemonicbox}
\mnemonic{Byte Code Runs Everywhere}
\end{mnemonicbox}

\questionmarks{1(ક)}{7}{એરેના એલિમેન્ટ્સને ચડતા ક્રમમાં સૉર્ટ કરવા માટે જાવામાં પ્રોગ્રામ લખો}

\begin{solutionbox}
\begin{lstlisting}[language=Java,caption={Array Sort}]
import java.util.Arrays;

public class ArraySort {
    public static void main(String[] args) {
        int[] arr = {64, 34, 25, 12, 22, 11, 90};
        
        // Bubble Sort
        for(int i = 0; i < arr.length-1; i++) {
            for(int j = 0; j < arr.length-i-1; j++) {
                if(arr[j] > arr[j+1]) {
                    int temp = arr[j];
                    arr[j] = arr[j+1];
                    arr[j+1] = temp;
                }
            }
        }
        
        System.out.println("Sorted array: " + Arrays.toString(arr));
    }
}
\end{lstlisting}

\textbf{મુખ્ય મુદ્દાઓ}:
\begin{itemize}
    \item \textbf{Bubble Sort}: બાજુના એલિમેન્ટ્સની તુલના કરે છે
    \item \textbf{Time Complexity}: O(n\textsuperscript{2})
    \item \textbf{Space Complexity}: O(1)
\end{itemize}
\end{solutionbox}

\begin{mnemonicbox}
\mnemonic{Bubble Up The Smallest}
\end{mnemonicbox}

\questionmarks{1(ક OR)}{7}{કમાંડ લાઇન આર્ગ્યુમેન્ટ્સનો ઉપયોગ કરીને કોઈપણ દસ સંખ્યાઓમાંથી મહત્તમ શોધવા માટે જાવામાં પ્રોગ્રામ લખો.}

\begin{solutionbox}
\begin{lstlisting}[language=Java,caption={Find Maximum from Command Line}]
public class FindMaximum {
    public static void main(String[] args) {
        if(args.length != 10) {
            System.out.println("Please enter exactly 10 numbers");
            return;
        }
        
        int max = Integer.parseInt(args[0]);
        
        for(int i = 1; i < args.length; i++) {
            int num = Integer.parseInt(args[i]);
            if(num > max) {
                max = num;
            }
        }
        
        System.out.println("Maximum number: " + max);
    }
}
\end{lstlisting}

\textbf{મુખ્ય મુદ્દાઓ}:
\begin{itemize}
    \item \textbf{Command Line}: \code{args[]} array આર્ગ્યુમેન્ટ્સ સ્ટોર કરે છે
    \item \textbf{parseInt()}: સ્ટ્રિંગને ઇન્ટિજરમાં કન્વર્ટ કરે છે
    \item \textbf{Validation}: Array length ચેક કરો
\end{itemize}
\end{solutionbox}

\begin{mnemonicbox}
\mnemonic{Arguments Maximum Search}
\end{mnemonicbox}

\questionmarks{2(અ)}{3}{Wrapper ક્લાસ શું છે? ઉદાહરણ સાથે સમજાવો.}

\begin{solutionbox}
\textbf{Wrapper Class}: પ્રિમિટિવ ડેટા ટાઇપ્સને ઓબ્જેક્ટમાં કન્વર્ટ કરે છે.

\begin{center}
\captionof{table}{Wrapper Classes}
\begin{tabulary}{\linewidth}{|L|L|}
\hline
\textbf{Primitive} & \textbf{Wrapper Class} \\ \hline
int & Integer \\ \hline
char & Character \\ \hline
boolean & Boolean \\ \hline
double & Double \\ \hline
\end{tabulary}
\end{center}

\begin{lstlisting}[language=Java,caption={Wrapper Class Example}]
// Boxing
Integer obj = Integer.valueOf(10);
// Unboxing  
int value = obj.intValue();
\end{lstlisting}
\end{solutionbox}

\begin{mnemonicbox}
\mnemonic{Wrap Primitives Into Objects}
\end{mnemonicbox}

\questionmarks{2(બ)}{4}{જાવાના વિવિધ લક્ષણોની યાદી આપો. કોઈપણ બે સમજાવો.}

\begin{solutionbox}
\textbf{Java Features}:
\begin{itemize}
    \item \textbf{Simple}: સરળ syntax, pointers નથી
    \item \textbf{Platform Independent}: એકવાર લખો, દરેક જગ્યાએ ચલાવો
    \item \textbf{Object Oriented}: ઓબ્જેક્ટ અને ક્લાસ પર આધારિત
    \item \textbf{Secure}: explicit pointers નથી, bytecode verification
\end{itemize}

\textbf{વિગતવાર સમજૂતી}:
\begin{itemize}
    \item \textbf{Platform Independence}: Java bytecode JVM વાળા કોઈપણ પ્લેટફોર્મ પર ચાલે છે
    \item \textbf{Object Oriented}: inheritance, encapsulation, polymorphism, abstraction સપોર્ટ કરે છે
\end{itemize}
\end{solutionbox}

\begin{mnemonicbox}
\mnemonic{Simple Platform Object Security}
\end{mnemonicbox}

\questionmarks{2(ક)}{7}{ઓવરરાઇડિંગ પદ્ધતિ શું છે? ઉદાહરણ સાથે સમજાવો.}

\begin{solutionbox}
\textbf{Method Overriding}: ચાઇલ્ડ ક્લાસ પેરન્ટ ક્લાસની મેથડનું વિશિષ્ટ implementation પ્રદાન કરે છે.

\begin{lstlisting}[language=Java,caption={Method Overriding}]
class Animal {
    public void sound() {
        System.out.println("Animal makes sound");
    }
}

class Dog extends Animal {
    @Override
    public void sound() {
        System.out.println("Dog barks");
    }
}

public class Test {
    public static void main(String[] args) {
        Animal a = new Dog();
        a.sound(); // Output: Dog barks
    }
}
\end{lstlisting}

\textbf{મુખ્ય મુદ્દાઓ}:
\begin{itemize}
    \item \textbf{Runtime Polymorphism}: ઓબ્જેક્ટ ટાઇપના આધારે મેથડ કોલ થાય છે
    \item \textbf{@Override}: મેથડ ઓવરરાઇડિંગ માટે annotation
    \item \textbf{Dynamic Binding}: રનટાઇમ પર મેથડ રિઝોલ્યુશન
\end{itemize}
\end{solutionbox}

\begin{mnemonicbox}
\mnemonic{Child Changes Parent Method}
\end{mnemonicbox}

\questionmarks{2(અ OR)}{3}{જાવામાં Garbage collection સમજાવો.}

\begin{solutionbox}
\textbf{Garbage Collection}: ઓટોમેટિક મેમરી મેનેજમેન્ટ જે અનુપયોગી ઓબ્જેક્ટ્સને દૂર કરે છે.

\begin{center}
\begin{tikzpicture}[auto, node distance=1.5cm]
    \node [gtu block] (created) {Object Created};
    \node [gtu block, right=of created] (used) {Object Used};
    \node [gtu block, below=of used] (unref) {Object Unreferenced};
    \node [gtu block, left=of unref] (gc) {Garbage Collector};
    \node [gtu block, above=of gc] (freed) {Memory Freed};

    \path [gtu arrow] (created) -- (used);
    \path [gtu arrow] (used) -- (unref);
    \path [gtu arrow] (unref) -- (gc);
    \path [gtu arrow] (gc) -- (freed);
\end{tikzpicture}
\end{center}

\textbf{મુખ્ય મુદ્દાઓ}:
\begin{itemize}
    \item \textbf{Automatic}: મેન્યુઅલ મેમરી deallocation નથી
    \item \textbf{Mark and Sweep}: અનુપયોગી ઓબ્જેક્ટ્સને ઓળખે અને દૂર કરે છે
    \item \textbf{Heap Memory}: heap memory area પર કામ કરે છે
\end{itemize}
\end{solutionbox}

\begin{mnemonicbox}
\mnemonic{Auto Clean Unused Objects}
\end{mnemonicbox}

\questionmarks{2(બ OR)}{4}{static કીવર્ડ ઉદાહરણ સાથે સમજાવો.}

\begin{solutionbox}
\textbf{Static Keyword}: ઇન્સ્ટન્સને બદલે ક્લાસનું છે.

\begin{lstlisting}[language=Java,caption={Static Example}]
class Student {
    static String college = "GTU";  // Static variable
    String name;
    
    static void showCollege() {     // Static method
        System.out.println("College: " + college);
    }
}
\end{lstlisting}

\textbf{Static Features}:
\begin{itemize}
    \item \textbf{Memory}: ક્લાસ લોડિંગ ટાઇમે લોડ થાય છે
    \item \textbf{Access}: ઓબ્જેક્ટ વિના એક્સેસ કરી શકાય છે
    \item \textbf{Sharing}: બધા instances વચ્ચે શેર થાય છે
\end{itemize}
\end{solutionbox}

\begin{mnemonicbox}
\mnemonic{Class Level Memory Sharing}
\end{mnemonicbox}

\questionmarks{2(ક OR)}{7}{કન્સ્ટ્રક્ટર શું છે? કોપી કન્સ્ટ્રક્ટરને ઉદાહરણ સાથે સમજાવો.}

\begin{solutionbox}
\textbf{Constructor}: ઓબ્જેક્ટ્સને initialize કરવા માટેની વિશેષ મેથડ.

\begin{lstlisting}[language=Java,caption={Constructor Types}]
class Person {
    String name;
    int age;
    
    // Default constructor
    Person() {
        name = "Unknown";
        age = 0;
    }
    
    // Parameterized constructor
    Person(String n, int a) {
        name = n;
        age = a;
    }
    
    // Copy constructor
    Person(Person p) {
        name = p.name;
        age = p.age;
    }
}
\end{lstlisting}

\textbf{Constructor Types}:
\begin{itemize}
    \item \textbf{Default}: કોઈ પેરામીટર નથી
    \item \textbf{Parameterized}: પેરામીટર લે છે
    \item \textbf{Copy}: અસ્તિત્વમાં રહેલા ઓબ્જેક્ટમાંથી ઓબ્જેક્ટ બનાવે છે
\end{itemize}
\end{solutionbox}

\begin{mnemonicbox}
\mnemonic{Default Parameter Copy}
\end{mnemonicbox}

\questionmarks{3(અ)}{3}{Super કીવર્ડ ઉદાહરણ સાથે સમજાવો.}

\begin{solutionbox}
\textbf{Super Keyword}: પેરન્ટ ક્લાસના મેમ્બર્સનો રેફરન્સ આપે છે.

\begin{lstlisting}[language=Java,caption={Super Keyword}]
class Vehicle {
    String brand = "Generic";
}

class Car extends Vehicle {
    String brand = "Toyota";
    
    void display() {
        System.out.println("Child: " + brand);
        System.out.println("Parent: " + super.brand);
    }
}
\end{lstlisting}

\textbf{Super ના ઉપયોગો}:
\begin{itemize}
    \item \textbf{Variables}: પેરન્ટ ક્લાસના વેરિયેબલ્સ એક્સેસ કરવા
    \item \textbf{Methods}: પેરન્ટ ક્લાસની મેથડ્સ કોલ કરવા
    \item \textbf{Constructor}: પેરન્ટ ક્લાસનું કન્સ્ટ્રક્ટર કોલ કરવા
\end{itemize}
\end{solutionbox}

\begin{mnemonicbox}
\mnemonic{Super Calls Parent}
\end{mnemonicbox}

\questionmarks{3(બ)}{4}{ઇન્હેરિટન્સના વિવિધ પ્રકારોની યાદી આપો. મલ્ટિલેવલ ઇન્હેરિટન્સ સમજાવો.}

\begin{solutionbox}
\textbf{Inheritance Types}:
\begin{center}
\captionof{table}{Inheritance Types}
\begin{tabulary}{\linewidth}{|L|L|}
\hline
\textbf{પ્રકાર} & \textbf{વર્ણન} \\ \hline
Single & એક પેરન્ટ, એક ચાઇલ્ડ \\ \hline
Multilevel & ઇન્હેરિટન્સની ચેઇન \\ \hline
Hierarchical & એક પેરન્ટ, અનેક ચાઇલ્ડ \\ \hline
Multiple & અનેક પેરન્ટ (ઇન્ટરફેસ દ્વારા) \\ \hline
\end{tabulary}
\end{center}

\textbf{Multilevel Inheritance}:

\begin{lstlisting}[language=Java,caption={Multilevel Inheritance}]
class Animal {
    void eat() { System.out.println("Eating"); }
}

class Mammal extends Animal {
    void breathe() { System.out.println("Breathing"); }
}

class Dog extends Mammal {
    void bark() { System.out.println("Barking"); }
}
\end{lstlisting}
\end{solutionbox}

\begin{mnemonicbox}
\mnemonic{Single Multi Hierarchical Multiple}
\end{mnemonicbox}

\questionmarks{3(ક)}{7}{ઇન્ટરફેસ શું છે? ઉદાહરણ સાથે મલ્ટીપલ ઇન્હેરિટન્સ સમજાવો.}

\begin{solutionbox}
\textbf{Interface}: કોન્ટ્રાક્ટ જે વ્યાખ્યાયિત કરે છે કે ક્લાસ શું કરવું જોઈએ, કેવી રીતે નહીં.

\begin{lstlisting}[language=Java,caption={Multiple Inheritance with Interface}]
interface Flyable {
    void fly();
}

interface Swimmable {
    void swim();
}

class Duck implements Flyable, Swimmable {
    public void fly() {
        System.out.println("Duck is flying");
    }
    
    public void swim() {
        System.out.println("Duck is swimming");
    }
}
\end{lstlisting}

\textbf{Interface Features}:
\begin{itemize}
    \item \textbf{Multiple Inheritance}: ક્લાસ એકથી વધુ ઇન્ટરફેસને implement કરી શકે છે
    \item \textbf{Abstract Methods}: બધી મેથડ્સ બાય ડિફોલ્ટ abstract હોય છે
    \item \textbf{Constants}: બધા વેરિયેબલ્સ public, static, final હોય છે
\end{itemize}
\end{solutionbox}

\begin{mnemonicbox}
\mnemonic{Multiple Abstract Constants}
\end{mnemonicbox}

\questionmarks{3(અ OR)}{3}{final કીવર્ડ ઉદાહરણ સાથે સમજાવો.}

\begin{solutionbox}
\textbf{Final Keyword}: ફેરફાર, ઇન્હેરિટન્સ અથવા ઓવરરાઇડિંગને પ્રતિબંધિત કરે છે.

\begin{lstlisting}[language=Java,caption={Final Keyword}]
final class Math {           // ઇન્હેરિટ થઈ શકતું નથી
    final int PI = 3.14;     // બદલી શકાતું નથી
    
    final void calculate() { // ઓવરરાઇડ થઈ શકતું નથી
        System.out.println("Calculating");
    }
}
\end{lstlisting}

\textbf{Final ના ઉપયોગો}:
\begin{itemize}
    \item \textbf{Class}: એક્સટેન્ડ કરી શકાતું નથી
    \item \textbf{Method}: ઓવરરાઇડ કરી શકાતું નથી
    \item \textbf{Variable}: ફરીથી assign કરી શકાતું નથી
\end{itemize}
\end{solutionbox}

\begin{mnemonicbox}
\mnemonic{Final Stops Changes}
\end{mnemonicbox}

\questionmarks{3(બ OR)}{4}{જાવામાં વિવિધ access controls સમજાવો.}

\begin{solutionbox}
\textbf{Access Modifiers}:

\begin{center}
\captionof{table}{Access Modifiers}
\begin{tabulary}{\linewidth}{|L|C|C|C|C|}
\hline
\textbf{Modifier} & \textbf{Same Class} & \textbf{Same Package} & \textbf{Subclass} & \textbf{Diff Package} \\ \hline
public & \checkmark & \checkmark & \checkmark & \checkmark \\ \hline
protected & \checkmark & \checkmark & \checkmark & \ding{55} \\ \hline
default & \checkmark & \checkmark & \ding{55} & \ding{55} \\ \hline
private & \checkmark & \ding{55} & \ding{55} & \ding{55} \\ \hline
\end{tabulary}
\end{center}
\end{solutionbox}

\begin{mnemonicbox}
\mnemonic{Public Protected Default Private}
\end{mnemonicbox}

\questionmarks{3(ક OR)}{7}{પેકેજ શું છે? પેકેજ બનાવવા માટેના પગલાં લખો અને તેનું ઉદાહરણ આપો.}

\begin{solutionbox}
\textbf{Package}: સંબંધિત ક્લાસ અને ઇન્ટરફેસનું જૂથ.

\textbf{પેકેજ બનાવવાના પગલાં}:
\begin{enumerate}
    \item \textbf{Declare}: ટોચ પર \code{package} વિધાન વાપરો
    \item \textbf{Compile}: \code{javac -d . ClassName.java}
    \item \textbf{Run}: \code{java packagename.ClassName}
\end{enumerate}

\begin{lstlisting}[language=Java,caption={Package Example}]
// File: mypack/Calculator.java
package mypack;

public class Calculator {
    public int add(int a, int b) {
        return a + b;
    }
}

// File: Test.java
import mypack.Calculator;

public class Test {
    public static void main(String[] args) {
        Calculator calc = new Calculator();
        System.out.println(calc.add(5, 3));
    }
}
\end{lstlisting}

\textbf{Package ના ફાયદા}:
\begin{itemize}
    \item \textbf{Organization}: સંબંધિત ક્લાસને જૂથબદ્ધ કરે છે
    \item \textbf{Access Control}: પેકેજ-લેવલ સુરક્ષા
    \item \textbf{Namespace}: નામના સંઘર્ષ (naming conflicts) ને ટાળે છે
\end{itemize}
\end{solutionbox}

\begin{mnemonicbox}
\mnemonic{Declare Compile Run}
\end{mnemonicbox}

\questionmarks{4(અ)}{3}{યોગ્ય ઉદાહરણ સાથે થ્રેડ પ્રાયોરિટીઝ સમજાવો.}

\begin{solutionbox}
\textbf{Thread Priority}: થ્રેડ એક્ઝિક્યુશનનો ક્રમ નક્કી કરે છે (1-10 સ્કેલ).

\begin{lstlisting}[language=Java,caption={Thread Priority}]
class MyThread extends Thread {
    public void run() {
        System.out.println(getName() + " Priority: " + getPriority());
    }
}

public class ThreadPriorityExample {
    public static void main(String[] args) {
        MyThread t1 = new MyThread();
        MyThread t2 = new MyThread();
        
        t1.setPriority(Thread.MIN_PRIORITY);  // 1
        t2.setPriority(Thread.MAX_PRIORITY);  // 10
        
        t1.start();
        t2.start();
    }
}
\end{lstlisting}

\textbf{Priority Constants}:
\begin{itemize}
    \item \textbf{MIN\_PRIORITY}: 1
    \item \textbf{NORM\_PRIORITY}: 5  
    \item \textbf{MAX\_PRIORITY}: 10
\end{itemize}
\end{solutionbox}

\begin{mnemonicbox}
\mnemonic{Min Normal Max}
\end{mnemonicbox}

\questionmarks{4(બ)}{4}{થ્રેડ શું છે? થ્રેડ લાઈફ સાયકલ સમજાવો.}

\begin{solutionbox}
\textbf{Thread}: એકસાથે એક્ઝિક્યુશન માટે લાઇટવેઇટ પ્રોસેસ.

\begin{center}
\begin{tikzpicture}[auto, node distance=2cm]
    \node [gtu state] (new) {New};
    \node [gtu state, right=of new] (runnable) {Runnable};
    \node [gtu state, right=of runnable] (running) {Running};
    \node [gtu state, right=of running] (dead) {Dead};
    \node [gtu state, below=of running] (blocked) {Blocked};

    \path [gtu arrow] (new) -- node {start()} (runnable);
    \path [gtu arrow] (runnable) -- node {CPU allocation} (running);
    \path [gtu arrow] (running) -- node {complete} (dead);
    \path [gtu arrow] (running) edge[bend right] node[left] {wait/sleep} (blocked);
    \path [gtu arrow] (blocked) edge[bend right] node[right] {notify/timeout} (runnable);
\end{tikzpicture}
\end{center}

\textbf{Thread States}:
\begin{itemize}
    \item \textbf{New}: થ્રેડ બન્યો પણ શરૂ થયો નથી
    \item \textbf{Runnable}: રન થવા માટે તૈયાર
    \item \textbf{Running}: હાલમાં એક્ઝિક્યુટ થઈ રહ્યો છે
    \item \textbf{Blocked}: રિસોર્સ માટે રાહ જોઈ રહ્યો છે
    \item \textbf{Dead}: એક્ઝિક્યુશન પૂર્ણ થયું
\end{itemize}
\end{solutionbox}

\begin{mnemonicbox}
\mnemonic{New Runnable Running Blocked Dead}
\end{mnemonicbox}

\questionmarks{4(ક)}{7}{Runnable ઇન્ટરફેસનો અમલ કરીને મલ્ટીપલ થ્રેડ બનાવે તેવો જાવા પ્રોગ્રામ લખો.}

\begin{solutionbox}
\begin{lstlisting}[language=Java,caption={Multiple Threads}]
class MyRunnable implements Runnable {
    private String threadName;
    
    MyRunnable(String name) {
        threadName = name;
    }
    
    public void run() {
        for(int i = 1; i <= 5; i++) {
            System.out.println(threadName + " - Count: " + i);
            try {
                Thread.sleep(1000);
            } catch(InterruptedException e) {
                e.printStackTrace();
            }
        }
    }
}

public class MultipleThreads {
    public static void main(String[] args) {
        Thread t1 = new Thread(new MyRunnable("Thread-1"));
        Thread t2 = new Thread(new MyRunnable("Thread-2"));
        Thread t3 = new Thread(new MyRunnable("Thread-3"));
        
        t1.start();
        t2.start(); 
        t3.start();
    }
}
\end{lstlisting}

\textbf{મુખ્ય મુદ્દાઓ}:
\begin{itemize}
    \item \textbf{Runnable Interface}: Thread ક્લાસને extend કરવા કરતા વધુ સારું
    \item \textbf{Thread.sleep()}: થ્રેડ એક્ઝિક્યુશનને થોભાવે છે
    \item \textbf{Multiple Threads}: એકસાથે રન થાય છે (conucrently)
\end{itemize}
\end{solutionbox}

\begin{mnemonicbox}
\mnemonic{Implement Runnable Start Multiple}
\end{mnemonicbox}

\questionmarks{4(અ OR)}{3}{ચાર અલગ અલગ ઇનબિલ્ટ એક્સેપ્શનની યાદી આપો. કોઈપણ એક સમજાવો.}

\begin{solutionbox}
\textbf{Inbuilt Exceptions}:
\begin{itemize}
    \item \textbf{NullPointerException}: null ઓબ્જેક્ટ એક્સેસ કરવું
    \item \textbf{ArrayIndexOutOfBoundsException}: અમાન્ય એરે ઇન્ડેક્સ
    \item \textbf{ArithmeticException}: શૂન્ય વડે ભાગાકાર
    \item \textbf{NumberFormatException}: અમાન્ય નંબર ફોર્મેટ
\end{itemize}

\textbf{ArithmeticException}: જ્યારે એરિથમેટિક ઑપરેશન નિષ્ફળ જાય ત્યારે throw થાય છે.

\begin{lstlisting}[language=Java,caption={ArithmeticException}]
int result = 10 / 0; // Throws ArithmeticException
\end{lstlisting}
\end{solutionbox}

\begin{mnemonicbox}
\mnemonic{Null Array Arithmetic Number}
\end{mnemonicbox}

\questionmarks{4(બ OR)}{4}{Try અને Catch યોગ્ય ઉદાહરણ સાથે સમજાવો.}

\begin{solutionbox}
\textbf{Try-Catch}: એક્સેપ્શન હેન્ડલિંગ મિકેનિઝમ.

\begin{lstlisting}[language=Java,caption={Try Catch Example}]
public class TryCatchExample {
    public static void main(String[] args) {
        try {
            int[] arr = {1, 2, 3};
            System.out.println(arr[5]); // Index out of bounds
        }
        catch(ArrayIndexOutOfBoundsException e) {
            System.out.println("Array index error: " + e.getMessage());
        }
        finally {
            System.out.println("Always executed");
        }
    }
}
\end{lstlisting}

\textbf{Exception Handling Flow}:
\begin{itemize}
    \item \textbf{Try}: કોડ જે એક્સેપ્શન throw કરી શકે છે
    \item \textbf{Catch}: ચોક્કસ એક્સેપ્શન હેન્ડલ કરે છે
    \item \textbf{Finally}: હંમેશા એક્ઝિક્યુટ થાય છે
\end{itemize}
\end{solutionbox}

\begin{mnemonicbox}
\mnemonic{Try Catch Finally}
\end{mnemonicbox}

\questionmarks{4(ક OR)}{7}{Exception શું છે? Arithmetic Exception નો ઉપયોગ દર્શાવતો પ્રોગ્રામ લખો.}

\begin{solutionbox}
\textbf{Exception}: રનટાઇમ એરર જે સામાન્ય પ્રોગ્રામ ફ્લોને વિક્ષેપિત કરે છે.

\begin{lstlisting}[language=Java,caption={ArithmeticException Example}]
public class ArithmeticExceptionExample {
    public static void main(String[] args) {
        Scanner sc = new Scanner(System.in);
        
        try {
            System.out.print("Enter first number: ");
            int num1 = sc.nextInt();
            
            System.out.print("Enter second number: ");
            int num2 = sc.nextInt();
            
            int result = num1 / num2;
            System.out.println("Result: " + result);
        }
        catch(ArithmeticException e) {
            System.out.println("Error: Cannot divide by zero!");
        }
        catch(Exception e) {
            System.out.println("General error: " + e.getMessage());
        }
        finally {
            sc.close();
        }
    }
}
\end{lstlisting}

\textbf{Exception Types}:
\begin{itemize}
    \item \textbf{Checked}: Compile-time exceptions
    \item \textbf{Unchecked}: Runtime exceptions
    \item \textbf{Error}: System-level problems
\end{itemize}
\end{solutionbox}

\begin{mnemonicbox}
\mnemonic{Runtime Error Disrupts Flow}
\end{mnemonicbox}

\questionmarks{5(અ)}{3}{Java માં ArrayIndexOutOfBound Exception ઉદાહરણ સાથે સમજાવો.}

\begin{solutionbox}
\textbf{ArrayIndexOutOfBoundsException}: જ્યારે અમાન્ય એરે ઇન્ડેક્સ એક્સેસ કરવામાં આવે ત્યારે throw થાય છે.

\begin{lstlisting}[language=Java,caption={ArrayIndexOutOfBoundsException}]
public class ArrayIndexExample {
    public static void main(String[] args) {
        int[] numbers = {10, 20, 30};
        
        try {
            System.out.println(numbers[5]); // Invalid index
        }
        catch(ArrayIndexOutOfBoundsException e) {
            System.out.println("Invalid array index: " + e.getMessage());
        }
    }
}
\end{lstlisting}

\textbf{મુખ્ય મુદ્દાઓ}:
\begin{itemize}
    \item \textbf{Valid Range}: 0 to array.length-1
    \item \textbf{Negative Index}: આ પણ એક્સેપ્શન throw કરે છે
    \item \textbf{Runtime Exception}: Unchecked exception
\end{itemize}
\end{solutionbox}

\begin{mnemonicbox}
\mnemonic{Array Index Range Check}
\end{mnemonicbox}

\questionmarks{5(બ)}{4}{સ્ટ્રીમ ક્લાસીસના બેઝિક્સ સમજાવો.}

\begin{solutionbox}
\textbf{Stream Classes}: ઇનપુટ/આઉટપુટ ઓપરેશન્સ હેન્ડલ કરે છે.

\begin{center}
\captionof{table}{Stream Classes}
\begin{tabulary}{\linewidth}{|L|L|}
\hline
\textbf{Stream Type} & \textbf{Classes} \\ \hline
Byte Streams & InputStream, OutputStream \\ \hline
Character Streams & Reader, Writer \\ \hline
File Streams & FileInputStream, FileOutputStream \\ \hline
Buffered Streams & BufferedReader, BufferedWriter \\ \hline
\end{tabulary}
\end{center}

\begin{center}
\begin{tikzpicture}[auto, node distance=1.5cm]
    \node [gtu block] (stream) {Stream Classes};
    \node [gtu block, below left=of stream, text width=3cm] (byte) {Byte Streams};
    \node [gtu block, below right=of stream, text width=3cm] (char) {Character Streams};
    
    \node [gtu block, below=0.5cm of byte, text width=3cm] (input) {InputStream};
    \node [gtu block, below=0.5cm of input, text width=3cm] (output) {OutputStream};
    
    \node [gtu block, below=0.5cm of char, text width=3cm] (reader) {Reader};
    \node [gtu block, below=0.5cm of reader, text width=3cm] (writer) {Writer};

    \path [gtu arrow] (stream) -- (byte);
    \path [gtu arrow] (stream) -- (char);
    \path [gtu arrow] (byte) -- (input);
    \path [gtu arrow] (byte) -- (output);
    \path [gtu arrow] (char) -- (reader);
    \path [gtu arrow] (char) -- (writer);
\end{tikzpicture}
\end{center}

\textbf{Stream Features}:
\begin{itemize}
    \item \textbf{Sequential}: ડેટા ક્રમમાં વહે છે
    \item \textbf{One Direction}: કાં તો ઇનપુટ અથવા આઉટપુટ
    \item \textbf{Automatic}: નીચલા સ્તરની વિગતો (low-level details) હેન્ડલ કરે છે
\end{itemize}
\end{solutionbox}

\begin{mnemonicbox}
\mnemonic{Byte Character File Buffered}
\end{mnemonicbox}

\questionmarks{5(ક)}{7}{ટેક્સ્ટ ફાઇલ બનાવવા માટે જાવા પ્રોગ્રામ લખો અને ટેક્સ્ટ ફાઇલ પર રીડ ઑપરેશન કરો.}

\begin{solutionbox}
\begin{lstlisting}[language=Java,caption={File Create and Read}]
import java.io.*;

public class FileReadExample {
    public static void main(String[] args) {
        // Create and write to file
        try {
            FileWriter writer = new FileWriter("sample.txt");
            writer.write("Hello World!\n");
            writer.write("Java File Handling\n");
            writer.write("GTU Exam 2024");
            writer.close();
            System.out.println("File created successfully");
        }
        catch(IOException e) {
            System.out.println("Error creating file: " + e.getMessage());
        }
        
        // Read from file
        try {
            BufferedReader reader = new BufferedReader(new FileReader("sample.txt"));
            String line;
            
            System.out.println("\nFile contents:");
            while((line = reader.readLine()) != null) {
                System.out.println(line);
            }
            reader.close();
        }
        catch(IOException e) {
            System.out.println("Error reading file: " + e.getMessage());
        }
    }
}
\end{lstlisting}

\textbf{મુખ્ય મુદ્દાઓ}:
\begin{itemize}
    \item \textbf{FileWriter}: ફાઇલ બનાવે અને લખે છે
    \item \textbf{BufferedReader}: કાર્યક્ષમ વાંચન
    \item \textbf{Exception Handling}: IOException handle કરો
\end{itemize}
\end{solutionbox}

\begin{mnemonicbox}
\mnemonic{Create Write Read Close}
\end{mnemonicbox}

\questionmarks{5(અ OR)}{3}{Java માં Divide by Zero Exception ને ઉદાહરણ સાથે સમજાવો.}

\begin{solutionbox}
\textbf{ArithmeticException}: શૂન્યથી ભાગાકાર ઑપરેશન દરમિયાન throw થાય છે.

\begin{lstlisting}[language=Java,caption={Divide by Zero}]
public class DivideByZeroExample {
    public static void main(String[] args) {
        try {
            int a = 10;
            int b = 0;
            int result = a / b;  // Throws ArithmeticException
            System.out.println("Result: " + result);
        }
        catch(ArithmeticException e) {
            System.out.println("Cannot divide by zero: " + e.getMessage());
        }
    }
}
\end{lstlisting}

\textbf{મુખ્ય મુદ્દાઓ}:
\begin{itemize}
    \item \textbf{Integer Division}: માત્ર integer division by zero exception throw કરે છે
    \item \textbf{Floating Point}: floating point division માટે Infinity return કરે છે
\item \textbf{Runtime Exception}: Unchecked exception
\end{itemize}
\end{solutionbox}

\begin{mnemonicbox}
\mnemonic{Zero Division Arithmetic Error}
\end{mnemonicbox}

\questionmarks{5(બ OR)}{4}{java I/O પ્રક્રિયા સમજાવો.}

\begin{solutionbox}
\textbf{Java I/O Process}: ડેટા વાંચવા અને લખવાની પદ્ધતિ.

\begin{center}
\begin{tikzpicture}[auto, node distance=2cm]
    \node [gtu block, text width=3cm] (source) {Data Source};
    \node [gtu block, right=of source, text width=3cm] (input) {Input Stream};
    \node [gtu block, right=of input, text width=3cm] (prog) {Java Program};
    \node [gtu block, below=of prog, text width=3cm] (output) {Output Stream};
    \node [gtu block, left=of output, text width=3cm] (dest) {Data Destination};

    \path [gtu arrow] (source) -- (input);
    \path [gtu arrow] (input) -- (prog);
    \path [gtu arrow] (prog) -- (output);
    \path [gtu arrow] (output) -- (dest);
\end{tikzpicture}
\end{center}

\textbf{I/O Components}:
\begin{itemize}
    \item \textbf{Stream}: ડેટાનો ક્રમ
    \item \textbf{Buffer}: કાર્યક્ષમતા માટે અસ્થાયી સ્ટોરેજ
    \item \textbf{File}: સ્થાયી સ્ટોરેજ
    \item \textbf{Network}: દૂરસ્થ ડેટા ટ્રાન્સફર
\end{itemize}

\textbf{I/O Types}:
\begin{itemize}
    \item \textbf{Byte-oriented}: કાચો ડેટા (images, videos)
    \item \textbf{Character-oriented}: ટેક્સ્ટ ડેટા
    \item \textbf{Synchronous}: Blocking operations
    \item \textbf{Asynchronous}: Non-blocking operations
\end{itemize}
\end{solutionbox}

\begin{mnemonicbox}
\mnemonic{Stream Buffer File Network}
\end{mnemonicbox}

\questionmarks{5(ક OR)}{7}{ટેક્સ્ટ ફાઇલ બનાવવા માટે જાવા પ્રોગ્રામ લખો અને ટેક્સ્ટ ફાઇલ પર રાઇટ ઑપરેશન કરો.}

\begin{solutionbox}
\begin{lstlisting}[language=Java,caption={File Write Example}]
import java.io.*;
import java.util.Scanner;

public class FileWriteExample {
    public static void main(String[] args) {
        Scanner sc = new Scanner(System.in);
        
        try {
            // Create file with FileWriter
            FileWriter writer = new FileWriter("student.txt");
            
            System.out.println("વિદ્યાર્થીની વિગતો દાખલ કરો:");
            System.out.print("નામ: ");
            String name = sc.nextLine();
            
            System.out.print("રોલ નંબર: ");
            String rollNo = sc.nextLine();
            
            System.out.print("શાખા: ");
            String branch = sc.nextLine();
            
            // Write data to file
            writer.write("Student Information\n");
            writer.write("==================\n");
            writer.write("નામ: " + name + "\n");
            writer.write("રોલ નંબર: " + rollNo + "\n");
            writer.write("શાખા: " + branch + "\n");
            writer.write("તારીખ: " + new java.util.Date() + "\n");
            
            writer.close();
            System.out.println("\nડેટા સફળતાપૂર્વક ફાઇલમાં લખાયો!");
            
        }
        catch(IOException e) {
            System.out.println("ફાઇલમાં લખવામાં એરર: " + e.getMessage());
        }
        finally {
            sc.close();
        }
    }
}
\end{lstlisting}

\textbf{મુખ્ય મુદ્દાઓ}:
\begin{itemize}
    \item \textbf{FileWriter}: ફાઇલમાં character data લખે છે
    \item \textbf{BufferedWriter}: મોટા ડેટા માટે વધુ કાર્યક્ષમ
    \item \textbf{Auto-close}: automatic closing માટે try-with-resources વાપરો
\end{itemize}
\end{solutionbox}

\begin{mnemonicbox}
\mnemonic{Create Write Close Handle}
\end{mnemonicbox}

\end{document}



