\documentclass{article}

% content/resources/templates/preamble.tex
\usepackage[margin=0.6in]{geometry}
\author{Milav Dabgar}
\usepackage{amsmath,amssymb,amsthm}
\usepackage{booktabs}
\usepackage{multirow}
\usepackage{xcolor}
\usepackage{tcolorbox}
\tcbuselibrary{breakable,skins}
\usepackage[colorlinks=true,linkcolor=blue]{hyperref}
\usepackage{titlesec}
\usepackage{enumitem}
\usepackage{tikz}
\usepackage{pgfplots}
\usepackage{circuitikz}
\usepackage[version=4]{mhchem}
\usepackage{longtable}
\usepackage{array}
\usepackage{float}
\usepackage{caption}
\usepackage{listings}

\lstset{
  basicstyle=\small\ttfamily,
  breaklines=true,
  breakatwhitespace=false,
  postbreak=\mbox{\textcolor{red}{$\hookrightarrow$}\space},
  float=false,
  numbers=left,
  numberstyle=\tiny\color{gray},
  numbersep=10pt,
  xleftmargin=2em,
  keywordstyle=\color{blue},
  commentstyle=\color{green!60!black},
  stringstyle=\color{purple},
  backgroundcolor=\color{gray!5},
  showstringspaces=false,
  tabsize=2,
  captionpos=b,
  keepspaces=true,
  columns=flexible
}

\pgfplotsset{compat=1.18}
\usetikzlibrary{shapes,arrows,positioning,calc,patterns,decorations.pathmorphing,decorations.markings,arrows.meta}

% Color scheme
\definecolor{headcolor}{RGB}{0,102,204}
\definecolor{keycolor}{RGB}{220,20,60}
\definecolor{solutioncolor}{RGB}{34,139,34}
\definecolor{mnemoniccolor}{RGB}{148,0,211}
\definecolor{codecolor}{RGB}{0,0,100}

% Spacing
\setlength{\parskip}{3pt}
\setlist[itemize]{nosep}
\setlist[enumerate]{nosep}

% Title formatting
\titleformat{\section}{\Large\bfseries\color{headcolor}}{\thesection}{1em}{}
\titleformat{\subsection}{\large\bfseries\color{headcolor}}{\thesubsection}{1em}{}

% Pandoc tightlist compatibility
\providecommand{\tightlist}{%
  \setlength{\itemsep}{0pt}\setlength{\parskip}{0pt}}

% Pandoc longtable compatibility
\newcounter{none}
\def\thenone{}


% content/resources/templates/english-boxes.tex

% Custom environments
\newtcolorbox{solutionbox}{
 breakable,
 enhanced,
 colback=solutioncolor!5!white,
 colframe=solutioncolor!75!black,
 fonttitle=\bfseries,
 title=Solution
}

\newtcolorbox{solutionboxnobreak}{
 colback=solutioncolor!5!white,
 colframe=solutioncolor!75!black,
 fonttitle=\bfseries,
 title=Solution
}

\newtcolorbox{keyformula}{
 breakable,
 enhanced,
 colback=keycolor!5!white,
 colframe=keycolor!75!black,
 fonttitle=\bfseries,
 title=Key Formula
}

\newtcolorbox{mnemonicboxenv}{
 breakable,
 enhanced,
 colback=mnemoniccolor!5!white,
 colframe=mnemoniccolor!75!black,
 fonttitle=\bfseries,
 title=Mnemonic
}

\newcommand{\mnemonicbox}[1]{%
  \begin{mnemonicboxenv}
    #1
  \end{mnemonicboxenv}
}


% Custom commands for GTU solutions
% This file defines semantic commands for consistent formatting

% Question command with automatic formatting
\newcommand{\question}[2]{%
  \section*{Question #1}%
  \textbf{#2}%
}

% OR question variant
\newcommand{\questionor}[2]{%
  \section*{Question #1 OR}%
  \textbf{#2}%
}

% Proper table environment with caption
\newenvironment{answertable}[1]{%
  \begin{table}[htbp]
  \centering
  \caption{#1}
}{%
  \end{table}
}

% Proper figure environment for diagrams
\newenvironment{answerdiagram}[1]{%
  \begin{figure}[htbp]
  \centering
  \caption{#1}
}{%
  \end{figure}
}

% Semantic markup for key terms
\newcommand{\keyword}[1]{\textbf{#1}}
\newcommand{\code}[1]{\texttt{#1}}
\newcommand{\classname}[1]{\texttt{#1}}
\newcommand{\methodname}[1]{\texttt{#1}}

% Proper quotation marks
\newcommand{\mnemonic}[1]{``#1''}


\title{Object Oriented Programming with Java (4341602) - Winter 2023 Solution}
\date{January 19, 2024}

\begin{document}
\maketitle

\questionmarks{1(a)}{3}{List out basic concepts of oop. Explain any one in detail.}

\begin{solutionbox}
\begin{center}
\captionof{table}{Basic OOP Concepts}
\begin{tabulary}{\linewidth}{|L|L|}
\hline
\textbf{Basic OOP Concepts} & \textbf{Description} \\ \hline
\textbf{Class} & Blueprint for objects \\ \hline
\textbf{Object} & Instance of a class \\ \hline
\textbf{Encapsulation} & Data hiding mechanism \\ \hline
\textbf{Inheritance} & Acquiring properties from parent \\ \hline
\textbf{Polymorphism} & One interface, multiple forms \\ \hline
\textbf{Abstraction} & Hiding implementation details \\ \hline
\end{tabulary}
\end{center}

\textbf{Encapsulation} is the process of binding data and methods together within a class and hiding internal implementation from outside world. It provides data security by making variables private and accessing them through public methods.
\end{solutionbox}

\begin{mnemonicbox}
\mnemonic{CEO-IPA} (Class, Encapsulation, Object, Inheritance, Polymorphism, Abstraction)
\end{mnemonicbox}

\questionmarks{1(b)}{4}{Explain JVM in detail.}

\begin{solutionbox}
\begin{center}
\begin{tikzpicture}[auto, node distance=1.5cm]
    \node [gtu block] (source) {Java Source Code};
    \node [gtu block, right=of source] (compiler) {Java Compiler};
    \node [gtu block, right=of compiler] (bytecode) {Bytecode .class};
    \node [gtu block, below=of bytecode] (jvm) {JVM};
    \node [gtu block, left=of jvm] (code) {Machine Code};
    \node [gtu block, left=of code] (output) {Output};

    \path [gtu arrow] (source) -- (compiler);
    \path [gtu arrow] (compiler) -- (bytecode);
    \path [gtu arrow] (bytecode) -- (jvm);
    \path [gtu arrow] (jvm) -- (code);
    \path [gtu arrow] (code) -- (output);
\end{tikzpicture}
\end{center}

\textbf{JVM (Java Virtual Machine)} is a runtime environment that executes Java bytecode. It provides platform independence by converting bytecode to machine-specific code.

\begin{itemize}
    \item \textbf{Class Loader}: Loads class files into memory
    \item \textbf{Memory Management}: Handles heap and stack memory
    \item \textbf{Execution Engine}: Executes bytecode instructions
    \item \textbf{Garbage Collector}: Automatically manages memory
\end{itemize}
\end{solutionbox}

\begin{mnemonicbox}
\mnemonic{CMEG} (Class loader, Memory, Execution, Garbage collection)
\end{mnemonicbox}

\questionmarks{1(c)}{7}{Write a program in java to print Fibonacci series for n terms.}

\begin{solutionbox}
\begin{lstlisting}[language=Java,caption={Fibonacci Series Program}]
public class Fibonacci {
    public static void main(String[] args) {
        int n = 10, first = 0, second = 1;
        System.out.print("Fibonacci Series: " + first + " " + second);
        
        for(int i = 2; i < n; i++) {
            int next = first + second;
            System.out.print(" " + next);
            first = second;
            second = next;
        }
    }
}
\end{lstlisting}

\begin{itemize}
    \item \textbf{Logic}: Start with 0,1 and add previous two numbers
    \item \textbf{Loop}: Continues for n terms
    \item \textbf{Variables}: first, second, next for calculation
\end{itemize}
\end{solutionbox}

\begin{mnemonicbox}
\mnemonic{FSN} (First, Second, Next)
\end{mnemonicbox}

\questionmarks{1(c OR)}{7}{Write a program in java to find out minimum from any ten numbers using command line argument.}

\begin{solutionbox}
\begin{lstlisting}[language=Java,caption={Find Minimum using CommandLine Arguments}]
public class FindMinimum {
    public static void main(String[] args) {
        if(args.length != 10) {
            System.out.println("Please enter exactly 10 numbers");
            return;
        }
        
        int min = Integer.parseInt(args[0]);
        for(int i = 1; i < args.length; i++) {
            int num = Integer.parseInt(args[i]);
            if(num < min) {
                min = num;
            }
        }
        System.out.println("Minimum number: " + min);
    }
}
\end{lstlisting}

\begin{itemize}
    \item \textbf{Command Line}: \code{java FindMinimum 5 3 8 1 9 2 7 4 6 0}
    \item \textbf{Logic}: Compare each number with current minimum
    \item \textbf{Method}: \code{Integer.parseInt()} converts string to integer
\end{itemize}
\end{solutionbox}

\begin{mnemonicbox}
\mnemonic{CIM} (Check, Integer.parseInt, Minimum)
\end{mnemonicbox}

\questionmarks{2(a)}{3}{What is wrapper class? Explain with example.}

\begin{solutionbox}
\begin{center}
\captionof{table}{Wrapper Classes}
\begin{tabulary}{\linewidth}{|L|L|}
\hline
\textbf{Primitive} & \textbf{Wrapper Class} \\ \hline
int & Integer \\ \hline
char & Character \\ \hline
boolean & Boolean \\ \hline
double & Double \\ \hline
\end{tabulary}
\end{center}

\textbf{Wrapper classes} convert primitive data types into objects. They provide utility methods and enable primitives to be used in collections.

\textbf{Example}: \code{Integer obj = new Integer(25);} or \code{Integer obj = 25;} (autoboxing)
\end{solutionbox}

\begin{mnemonicbox}
\mnemonic{POC} (Primitive to Object Conversion)
\end{mnemonicbox}

\questionmarks{2(b)}{4}{List out different features of java. Explain any two.}

\begin{solutionbox}
\begin{center}
\captionof{table}{Java Features}
\begin{tabulary}{\linewidth}{|L|L|}
\hline
\textbf{Java Features} & \textbf{Description} \\ \hline
\textbf{Platform Independent} & Write once, run anywhere \\ \hline
\textbf{Object Oriented} & Everything is an object \\ \hline
\textbf{Simple} & Easy syntax, no pointers \\ \hline
\textbf{Secure} & Bytecode verification \\ \hline
\textbf{Robust} & Strong memory management \\ \hline
\textbf{Multithreaded} & Concurrent execution \\ \hline
\end{tabulary}
\end{center}

\textbf{Platform Independence}: Java source code compiles to bytecode which runs on any platform with JVM installed.

\textbf{Object Oriented}: Java follows OOP principles like encapsulation, inheritance, and polymorphism for better code organization.
\end{solutionbox}

\begin{mnemonicbox}
\mnemonic{POSSMR} (Platform, Object, Simple, Secure, Multithreaded, Robust)
\end{mnemonicbox}

\questionmarks{2(c)}{7}{What is method overload? Explain with example.}

\begin{solutionbox}
\textbf{Method Overloading} allows multiple methods with same name but different parameters in the same class.

\begin{lstlisting}[language=Java,caption={Method Overloading}]
class Calculator {
    public int add(int a, int b) {
        return a + b;
    }
    
    public double add(double a, double b) {
        return a + b;
    }
    
    public int add(int a, int b, int c) {
        return a + b + c;
    }
}
\end{lstlisting}

\begin{itemize}
    \item \textbf{Rules}: Different parameter types or number of parameters
    \item \textbf{Compile Time}: Decision made during compilation
    \item \textbf{Return Type}: Cannot be only difference
\end{itemize}
\end{solutionbox}

\begin{mnemonicbox}
\mnemonic{SNRT} (Same Name, different paRameters, compile Time)
\end{mnemonicbox}

\questionmarks{2(a OR)}{3}{Explain Garbage collection in java.}

\begin{solutionbox}
\begin{center}
\begin{tikzpicture}[node distance=0cm]
    \node [gtu block, text width=6cm, fill=blue!5] (method) {Method Area\\(Class Definitions)};
    \node [gtu block, text width=6cm, fill=green!5, above=of method] (stack) {Stack\\(Method Calls)};
    \node [gtu block, text width=6cm, fill=red!5, above=of stack] (heap) {Heap\\(Objects Stored Here)};
    
    \node [right=0.5cm of heap, align=left] {$\leftarrow$ Objects stored here};
    \node [right=0.5cm of stack, align=left] {$\leftarrow$ Method calls};
    \node [right=0.5cm of method, align=left] {$\leftarrow$ Class definitions};
\end{tikzpicture}
\end{center}

\textbf{Garbage Collection} automatically deallocates memory of unreferenced objects. JVM runs garbage collector periodically to free up heap memory.

\begin{itemize}
    \item \textbf{Automatic}: No manual memory management needed
    \item \textbf{Mark and Sweep}: Marks unreferenced objects, then removes them
\end{itemize}
\end{solutionbox}

\begin{mnemonicbox}
\mnemonic{ARMS} (Automatic Reference Management System)
\end{mnemonicbox}

\questionmarks{2(b OR)}{4}{Explain final keyword with example.}

\begin{solutionbox}
\begin{center}
\captionof{table}{Final Keyword Usage}
\begin{tabulary}{\linewidth}{|L|L|L|}
\hline
\textbf{Usage} & \textbf{Description} & \textbf{Example} \\ \hline
\textbf{final variable} & Cannot be changed & \code{final int x = 10;} \\ \hline
\textbf{final method} & Cannot be overridden & \code{final void display()} \\ \hline
\textbf{final class} & Cannot be inherited & \code{final class MyClass} \\ \hline
\end{tabulary}
\end{center}

\begin{lstlisting}[language=Java,caption={Final Keyword Example}]
final class FinalClass {
    final int value = 100;
    final void show() {
        System.out.println("Final method");
    }
}
\end{lstlisting}
\end{solutionbox}

\begin{mnemonicbox}
\mnemonic{VCM} (Variable constant, Class not inherited, Method not overridden)
\end{mnemonicbox}

\questionmarks{2(c OR)}{7}{What is constructor? Explain parameterized constructor with example.}

\begin{solutionbox}
\textbf{Constructor} is a special method that initializes objects when created. It has same name as class and no return type.

\begin{lstlisting}[language=Java,caption={Parameterized Constructor}]
class Student {
    String name;
    int age;
    
    // Parameterized Constructor
    public Student(String n, int a) {
        name = n;
        age = a;
    }
    
    public void display() {
        System.out.println("Name: " + name + ", Age: " + age);
    }
}

class Main {
    public static void main(String[] args) {
        Student s1 = new Student("John", 20);
        s1.display();
    }
}
\end{lstlisting}

\begin{itemize}
    \item \textbf{Purpose}: Initialize object with specific values
    \item \textbf{Parameters}: Accepts arguments to set initial state
    \item \textbf{Automatic}: Called automatically when object is created
\end{itemize}
\end{solutionbox}

\begin{mnemonicbox}
\mnemonic{SPA} (Same name, Parameters, Automatic call)
\end{mnemonicbox}

\questionmarks{3(a)}{3}{Explain super keyword with example.}

\begin{solutionbox}
\textbf{super keyword} refers to parent class members and constructor. It resolves naming conflicts between parent and child classes.

\begin{lstlisting}[language=Java,caption={Super Keyword Example}]
class Parent {
    int x = 10;
}
class Child extends Parent {
    int x = 20;
    void display() {
        System.out.println(super.x); // 10
        System.out.println(x);       // 20
    }
}
\end{lstlisting}

\begin{itemize}
    \item \textbf{super.variable}: Access parent class variable
    \item \textbf{super.method()}: Call parent class method
    \item \textbf{super()}: Call parent class constructor
\end{itemize}
\end{solutionbox}

\begin{mnemonicbox}
\mnemonic{VMC} (Variable, Method, Constructor)
\end{mnemonicbox}

\questionmarks{3(b)}{4}{List out different types of inheritance. Explain multilevel inheritance.}

\begin{solutionbox}
\begin{center}
\captionof{table}{Inheritance Types}
\begin{tabulary}{\linewidth}{|L|L|}
\hline
\textbf{Inheritance Types} & \textbf{Description} \\ \hline
\textbf{Single} & One parent, one child \\ \hline
\textbf{Multilevel} & Chain of inheritance \\ \hline
\textbf{Hierarchical} & One parent, multiple children \\ \hline
\textbf{Multiple} & Multiple parents (via interfaces) \\ \hline
\end{tabulary}
\end{center}

\begin{center}
\begin{tikzpicture}[auto, node distance=1.5cm]
    \node [gtu block] (animal) {Animal};
    \node [gtu block, right=of animal] (mammal) {Mammal};
    \node [gtu block, right=of mammal] (dog) {Dog};
    
    \path [gtu arrow] (animal) -- (mammal);
    \path [gtu arrow] (mammal) -- (dog);
\end{tikzpicture}
\end{center}

\textbf{Multilevel Inheritance}: Class inherits from another class which itself inherits from another class, forming a chain.

\begin{lstlisting}[language=Java,caption={Multilevel Inheritance}]
class Animal {
    void eat() { System.out.println("Eating"); }
}
class Mammal extends Animal {
    void walk() { System.out.println("Walking"); }
}
class Dog extends Mammal {
    void bark() { System.out.println("Barking"); }
}
\end{lstlisting}
\end{solutionbox}

\begin{mnemonicbox}
\mnemonic{SMHM} (Single, Multilevel, Hierarchical, Multiple)
\end{mnemonicbox}

\questionmarks{3(c)}{7}{What is interface? Explain multiple inheritance with example.}

\begin{solutionbox}
\textbf{Interface} is a contract that defines what methods a class must implement. It contains only abstract methods and constants.

\begin{lstlisting}[language=Java,caption={Multiple Inheritance with Interface}]
interface Flyable {
    void fly();
}

interface Swimmable {
    void swim();
}

class Duck implements Flyable, Swimmable {
    public void fly() {
        System.out.println("Duck is flying");
    }
    
    public void swim() {
        System.out.println("Duck is swimming");
    }
}
\end{lstlisting}

\textbf{Multiple Inheritance}: A class can implement multiple interfaces, achieving multiple inheritance of behavior.

\begin{itemize}
    \item \textbf{Abstract Methods}: All methods are abstract by default
    \item \textbf{Constants}: All variables are public, static, final
    \item \textbf{implements}: Keyword to implement interface
\end{itemize}
\end{solutionbox}

\begin{mnemonicbox}
\mnemonic{ACI} (Abstract methods, Constants, implements keyword)
\end{mnemonicbox}

\questionmarks{3(a OR)}{3}{Explain static keyword with example.}

\begin{solutionbox}
\textbf{static keyword} creates class-level members that belong to class rather than instances. Memory allocated once when class loads.

\begin{lstlisting}[language=Java,caption={Static Keyword}]
class Counter {
    static int count = 0;
    static void increment() {
        count++;
    }
}
\end{lstlisting}

\begin{itemize}
    \item \textbf{static variable}: Shared among all objects
    \item \textbf{static method}: Called without object creation
    \item \textbf{Memory}: Allocated in method area
\end{itemize}
\end{solutionbox}

\begin{mnemonicbox}
\mnemonic{SOM} (Shared, Object not needed, Method area)
\end{mnemonicbox}

\questionmarks{3(b OR)}{4}{Explain different access controls in Java.}

\begin{solutionbox}
\begin{center}
\captionof{table}{Access Modifiers}
\begin{tabulary}{\linewidth}{|L|C|C|C|C|}
\hline
\textbf{Access Modifier} & \textbf{Same Class} & \textbf{Same Package} & \textbf{Subclass} & \textbf{Diff Package} \\ \hline
\textbf{private} & \checkmark & \ding{55} & \ding{55} & \ding{55} \\ \hline
\textbf{default} & \checkmark & \checkmark & \ding{55} & \ding{55} \\ \hline
\textbf{protected} & \checkmark & \checkmark & \checkmark & \ding{55} \\ \hline
\textbf{public} & \checkmark & \checkmark & \checkmark & \checkmark \\ \hline
\end{tabulary}
\end{center}

\textbf{Access Control} determines visibility and accessibility of classes, methods, and variables.
\end{solutionbox}

\begin{mnemonicbox}
\mnemonic{PriDef ProPub} (Private, Default, Protected, Public)
\end{mnemonicbox}

\questionmarks{3(c OR)}{7}{What is package? Write steps to create a package and give example of it.}

\begin{solutionbox}
\textbf{Package} is a namespace that organizes related classes and interfaces. It provides access protection and namespace management.

\textbf{Steps to create package}:
\begin{enumerate}
    \item Use \code{package} statement at top of file
    \item Create directory structure matching package name
    \item Compile with \code{-d} option
    \item Import package in other files
\end{enumerate}

\begin{lstlisting}[language=Java,caption={Package creation and usage}]
// File: com/mycompany/MyClass.java
package com.mycompany;

public class MyClass {
    public void display() {
        System.out.println("Package example");
    }
}

// Using the package
import com.mycompany.MyClass;

class Main {
    public static void main(String[] args) {
        MyClass obj = new MyClass();
        obj.display();
    }
}
\end{lstlisting}

\textbf{Compilation}: \code{javac -d . MyClass.java}
\end{solutionbox}

\begin{mnemonicbox}
\mnemonic{PDCI} (Package statement, Directory, Compile, Import)
\end{mnemonicbox}

\questionmarks{4(a)}{3}{Explain thread priorities with suitable example.}

\begin{solutionbox}
\textbf{Thread Priority} determines execution order of threads. Java provides 10 priority levels from 1 (lowest) to 10 (highest).

\begin{lstlisting}[language=Java,caption={Thread Priority}]
class MyThread extends Thread {
    public void run() {
        System.out.println(getName() + " Priority: " + getPriority());
    }
}

class Main {
    public static void main(String[] args) {
        MyThread t1 = new MyThread();
        MyThread t2 = new MyThread();
        
        t1.setPriority(Thread.MIN_PRIORITY); // 1
        t2.setPriority(Thread.MAX_PRIORITY); // 10
        
        t1.start();
        t2.start();
    }
}
\end{lstlisting}

\textbf{Priority Constants}: MIN\_PRIORITY (1), NORM\_PRIORITY (5), MAX\_PRIORITY (10)
\end{solutionbox}

\begin{mnemonicbox}
\mnemonic{MNM} (MIN, NORM, MAX)
\end{mnemonicbox}

\questionmarks{4(b)}{4}{What is Thread? Explain Thread life cycle.}

\begin{solutionbox}
\begin{center}
\begin{tikzpicture}[auto, node distance=2cm]
    \node [gtu state] (new) {New};
    \node [gtu state, right=of new] (runnable) {Runnable};
    \node [gtu state, right=of runnable] (running) {Running};
    \node [gtu state, right=of running] (dead) {Dead};
    \node [gtu state, below=of running] (blocked) {Blocked};

    \path [gtu arrow] (new) -- node {start()} (runnable);
    \path [gtu arrow] (runnable) -- node {Scheduler} (running);
    \path [gtu arrow] (running) -- node {completes} (dead);
    \path [gtu arrow] (running) edge[bend right] node[left] {wait/sleep} (blocked);
    \path [gtu arrow] (blocked) edge[bend right] node[right] {notify/timeout} (runnable);
    \path [gtu arrow] (running) edge[bend left] node[above] {yield()} (runnable);
\end{tikzpicture}
\end{center}

\textbf{Thread} is a lightweight subprocess that enables concurrent execution within a program.

\textbf{Thread Life Cycle States}:
\begin{itemize}
    \item \textbf{New}: Thread created but not started
    \item \textbf{Runnable}: Ready to run, waiting for CPU
    \item \textbf{Running}: Currently executing
    \item \textbf{Blocked}: Waiting for resource or I/O
    \item \textbf{Dead}: Thread execution completed
\end{itemize}
\end{solutionbox}

\begin{mnemonicbox}
\mnemonic{NRRBD} (New, Runnable, Running, Blocked, Dead)
\end{mnemonicbox}

\questionmarks{4(c)}{7}{Write a program in java that create the multiple threads by implementing the Thread class.}

\begin{solutionbox}
\begin{lstlisting}[language=Java,caption={Multiple Threads}]
class MyThread extends Thread {
    private String threadName;
    
    public MyThread(String name) {
        threadName = name;
        setName(threadName);
    }
    
    public void run() {
        for(int i = 1; i <= 5; i++) {
            System.out.println(threadName + " - Count: " + i);
            try {
                Thread.sleep(1000);
            } catch(InterruptedException e) {
                System.out.println(threadName + " interrupted");
            }
        }
        System.out.println(threadName + " completed");
    }
}

class Main {
    public static void main(String[] args) {
        MyThread thread1 = new MyThread("Thread-1");
        MyThread thread2 = new MyThread("Thread-2");
        MyThread thread3 = new MyThread("Thread-3");
        
        thread1.start();
        thread2.start();
        thread3.start();
    }
}
\end{lstlisting}

\begin{itemize}
    \item \textbf{extends Thread}: Inherit Thread class functionality
    \item \textbf{Override run()}: Define thread execution logic
    \item \textbf{start()}: Begin thread execution
\end{itemize}
\end{solutionbox}

\begin{mnemonicbox}
\mnemonic{EOS} (Extends, Override run, Start method)
\end{mnemonicbox}

\questionmarks{4(a OR)}{3}{List four different inbuilt exceptions. Explain any one inbuilt exception.}

\begin{solutionbox}
\begin{center}
\captionof{table}{Inbuilt Exceptions}
\begin{tabulary}{\linewidth}{|L|L|}
\hline
\textbf{Inbuilt Exceptions} & \textbf{Description} \\ \hline
\textbf{NullPointerException} & Null reference access \\ \hline
\textbf{ArrayIndexOutOfBoundsException} & Invalid array index \\ \hline
\textbf{NumberFormatException} & Invalid number format \\ \hline
\textbf{ClassCastException} & Invalid type casting \\ \hline
\end{tabulary}
\end{center}

\textbf{NullPointerException} occurs when trying to access methods or variables of a null reference.

\begin{lstlisting}[language=Java,caption={NullPointerException}]
String str = null;
int length = str.length(); // Throws NullPointerException
\end{lstlisting}
\end{solutionbox}

\begin{mnemonicbox}
\mnemonic{NANC} (NullPointer, ArrayIndex, NumberFormat, ClassCast)
\end{mnemonicbox}

\questionmarks{4(b OR)}{4}{Explain multiple catch with suitable example.}

\begin{solutionbox}
\textbf{Multiple catch} blocks handle different types of exceptions that might occur in try block. Each catch handles specific exception type.

\begin{lstlisting}[language=Java,caption={Multiple Catch Blocks}]
class MultipleCatch {
    public static void main(String[] args) {
        try {
            int[] arr = {1, 2, 3};
            System.out.println(arr[5]); // ArrayIndexOutOfBoundsException
            int result = 10/0;          // ArithmeticException
        }
        catch(ArrayIndexOutOfBoundsException e) {
            System.out.println("Array index error: " + e.getMessage());
        }
        catch(ArithmeticException e) {
            System.out.println("Arithmetic error: " + e.getMessage());
        }
        catch(Exception e) {
            System.out.println("General error: " + e.getMessage());
        }
    }
}
\end{lstlisting}

\textbf{Order}: Specific exceptions first, general exceptions last
\end{solutionbox}

\begin{mnemonicbox}
\mnemonic{SGO} (Specific first, General last, Ordered)
\end{mnemonicbox}

\questionmarks{4(c OR)}{7}{What is Exception? Write a program that show the use of Arithmetic Exception.}

\begin{solutionbox}
\textbf{Exception} is an abnormal condition that disrupts normal program flow. It's an object representing an error condition.

\begin{lstlisting}[language=Java,caption={ArithmeticException Handling}]
class ArithmeticExceptionDemo {
    public static void main(String[] args) {
        int numerator = 100;
        int[] denominators = {5, 0, 2, 0, 10};
        
        for(int i = 0; i < denominators.length; i++) {
            try {
                int result = numerator / denominators[i];
                System.out.println(numerator + " / " + denominators[i] + " = " + result);
            }
            catch(ArithmeticException e) {
                System.out.println("Error: Cannot divide by zero!");
                System.out.println("Exception message: " + e.getMessage());
            }
        }
        
        System.out.println("Program continues after exception handling");
    }
}
\end{lstlisting}

\textbf{ArithmeticException} thrown when mathematical error occurs like division by zero.

\textbf{Exception Hierarchy}: Object $\to$ Throwable $\to$ Exception $\to$ RuntimeException $\to$ ArithmeticException
\end{solutionbox}

\begin{mnemonicbox}
\mnemonic{OTERRA} (Object, Throwable, Exception, RuntimeException, ArithmeticException)
\end{mnemonicbox}

\questionmarks{5(a)}{3}{Explain ArrayIndexOutOfBound Exception in Java with example.}

\begin{solutionbox}
\textbf{ArrayIndexOutOfBoundsException} occurs when accessing array element with invalid index (negative or >= array length).

\begin{lstlisting}[language=Java,caption={ArrayIndexOutOfBoundsException}]
class ArrayException {
    public static void main(String[] args) {
        int[] numbers = {10, 20, 30};
        
        try {
            System.out.println(numbers[5]); // Invalid index
        }
        catch(ArrayIndexOutOfBoundsException e) {
            System.out.println("Invalid array index: " + e.getMessage());
        }
    }
}
\end{lstlisting}

\begin{itemize}
    \item \textbf{Valid Range}: 0 to (length-1)
    \item \textbf{Runtime Exception}: Unchecked exception
    \item \textbf{Common Cause}: Loop condition errors
\end{itemize}
\end{solutionbox}

\begin{mnemonicbox}
\mnemonic{VRC} (Valid range, Runtime exception, Common in loops)
\end{mnemonicbox}

\questionmarks{5(b)}{4}{Explain basics of stream classes.}

\begin{solutionbox}
\begin{center}
\begin{tikzpicture}[auto, node distance=1.5cm]
    \node [gtu block] (stream) {Stream Classes};
    \node [gtu block, below left=of stream, text width=3cm] (byte) {Byte Streams};
    \node [gtu block, below right=of stream, text width=3cm] (char) {Character Streams};
    
    \node [gtu block, below=0.5cm of byte, text width=3cm] (input) {InputStream};
    \node [gtu block, below=0.5cm of input, text width=3cm] (output) {OutputStream};
    
    \node [gtu block, below=0.5cm of char, text width=3cm] (reader) {Reader};
    \node [gtu block, below=0.5cm of reader, text width=3cm] (writer) {Writer};

    \path [gtu arrow] (stream) -- (byte);
    \path [gtu arrow] (stream) -- (char);
    \path [gtu arrow] (byte) -- (input);
    \path [gtu arrow] (byte) -- (output);
    \path [gtu arrow] (char) -- (reader);
    \path [gtu arrow] (char) -- (writer);
\end{tikzpicture}
\end{center}

\textbf{Stream Classes} provide input/output operations for reading and writing data.

\begin{center}
\captionof{table}{Stream Classes}
\begin{tabulary}{\linewidth}{|L|L|L|}
\hline
\textbf{Stream Type} & \textbf{Purpose} & \textbf{Base Classes} \\ \hline
\textbf{Byte Streams} & Binary data & InputStream, OutputStream \\ \hline
\textbf{Character Streams} & Text data & Reader, Writer \\ \hline
\end{tabulary}
\end{center}

\begin{itemize}
    \item \textbf{Input Streams}: Read data from source
    \item \textbf{Output Streams}: Write data to destination
    \item \textbf{Buffered Streams}: Improve performance with buffering
\end{itemize}
\end{solutionbox}

\begin{mnemonicbox}
\mnemonic{BIOC} (Byte, Input/Output, Character streams)
\end{mnemonicbox}

\questionmarks{5(c)}{7}{Write a java program to create a text file and perform read operation on the text file.}

\begin{solutionbox}
\begin{lstlisting}[language=Java,caption={File Create and Read}]
import java.io.*;

class FileOperations {
    public static void main(String[] args) {
        // Create and write to file
        try {
            FileWriter writer = new FileWriter("sample.txt");
            writer.write("Hello World!\n");
            writer.write("This is Java file handling example.\n");
            writer.write("Learning Input/Output operations.");
            writer.close();
            System.out.println("File created and written successfully.");
        }
        catch(IOException e) {
            System.out.println("Error creating file: " + e.getMessage());
        }
        
        // Read from file
        try {
            FileReader reader = new FileReader("sample.txt");
            BufferedReader bufferedReader = new BufferedReader(reader);
            String line;
            
            System.out.println("\nFile contents:");
            while((line = bufferedReader.readLine()) != null) {
                System.out.println(line);
            }
            
            bufferedReader.close();
            reader.close();
        }
        catch(IOException e) {
            System.out.println("Error reading file: " + e.getMessage());
        }
    }
}
\end{lstlisting}

\begin{itemize}
    \item \textbf{FileWriter}: Creates and writes to text file
    \item \textbf{FileReader}: Reads from text file
    \item \textbf{BufferedReader}: Efficient line-by-line reading
\end{itemize}
\end{solutionbox}

\begin{mnemonicbox}
\mnemonic{WRB} (Writer creates, Reader reads, Buffered for efficiency)
\end{mnemonicbox}

\questionmarks{5(a OR)}{3}{Explain Divide by Zero Exception in Java with example.}

\begin{solutionbox}
\textbf{ArithmeticException (Divide by Zero)} occurs when integer is divided by zero. Floating-point division by zero returns Infinity.

\begin{lstlisting}[language=Java,caption={Divide By Zero}]
class DivideByZeroExample {
    public static void main(String[] args) {
        try {
            int result = 10 / 0; // Throws ArithmeticException
            System.out.println("Result: " + result);
        }
        catch(ArithmeticException e) {
            System.out.println("Cannot divide by zero!");
        }
        
        // Floating point division
        double floatResult = 10.0 / 0.0; // Returns Infinity
        System.out.println("Float result: " + floatResult);
    }
}
\end{lstlisting}

\begin{itemize}
    \item \textbf{Integer Division}: Throws ArithmeticException
    \item \textbf{Float Division}: Returns Infinity or NaN
\end{itemize}
\end{solutionbox}

\begin{mnemonicbox}
\mnemonic{IFI} (Integer throws exception, Float returns Infinity)
\end{mnemonicbox}

\questionmarks{5(b OR)}{4}{Explain java I/O process.}

\begin{solutionbox}
\begin{center}
\begin{tikzpicture}[auto, node distance=2cm]
    \node [gtu block, text width=3cm] (source) {Source\\(File, Keyboard, Network)};
    \node [gtu block, right=of source, text width=3cm] (stream) {Stream\\(Reader/Writer, IO Stream)};
    \node [gtu block, right=of stream, text width=3cm] (dest) {Destination\\(File, Screen, Network)};
    
    \path [gtu arrow] (source) -- (stream);
    \path [gtu arrow] (stream) -- (dest);
\end{tikzpicture}
\end{center}

\textbf{Java I/O Process} handles data transfer between program and external sources using streams.

\begin{center}
\captionof{table}{I/O Process Components}
\begin{tabulary}{\linewidth}{|L|L|}
\hline
\textbf{Component} & \textbf{Purpose} \\ \hline
\textbf{Source} & Data origin (file, keyboard, network) \\ \hline
\textbf{Stream} & Data pathway (byte/character streams) \\ \hline
\textbf{Destination} & Data target (file, screen, network) \\ \hline
\end{tabulary}
\end{center}

\textbf{Process Steps}:
\begin{enumerate}
    \item \textbf{Open Stream}: Create connection to source/destination
    \item \textbf{Process Data}: Read/write operations
    \item \textbf{Close Stream}: Release resources
\end{enumerate}
\end{solutionbox}

\begin{mnemonicbox}
\mnemonic{OPC} (Open, Process, Close)
\end{mnemonicbox}

\questionmarks{5(c OR)}{7}{Write a java program to display the content of a text file and perform append operation on the text file.}

\begin{solutionbox}
\begin{lstlisting}[language=Java,caption={File Append Operation}]
import java.io.*;

class FileAppendExample {
    public static void main(String[] args) {
        String fileName = "data.txt";
        
        // Create initial file content
        try {
            FileWriter writer = new FileWriter(fileName);
            writer.write("Initial content line 1\n");
            writer.write("Initial content line 2\n");
            writer.close();
            System.out.println("Initial file created.");
        }
        catch(IOException e) {
            System.out.println("Error creating file: " + e.getMessage());
        }
        
        // Display file content
        displayFileContent(fileName);
        
        // Append to file
        try {
            FileWriter appendWriter = new FileWriter(fileName, true); // true for append
            appendWriter.write("Appended line 1\n");
            appendWriter.write("Appended line 2\n");
            appendWriter.close();
            System.out.println("\nContent appended successfully.");
        }
        catch(IOException e) {
            System.out.println("Error appending to file: " + e.getMessage());
        }
        
        // Display updated content
        System.out.println("\nFile content after append:");
        displayFileContent(fileName);
    }
    
    static void displayFileContent(String fileName) {
        try {
            BufferedReader reader = new BufferedReader(new FileReader(fileName));
            String line;
            System.out.println("\nFile contents:");
            while((line = reader.readLine()) != null) {
                System.out.println(line);
            }
            reader.close();
        }
        catch(IOException e) {
            System.out.println("Error reading file: " + e.getMessage());
        }
    }
}
\end{lstlisting}

\begin{itemize}
    \item \textbf{FileWriter(filename, true)}: Append mode enabled
    \item \textbf{displayFileContent()}: Reusable method for reading
    \item \textbf{BufferedReader}: Efficient line reading
\end{itemize}
\end{solutionbox}

\begin{mnemonicbox}
\mnemonic{ARB} (Append mode, Reusable method, Buffered reading)
\end{mnemonicbox}

\end{document}
