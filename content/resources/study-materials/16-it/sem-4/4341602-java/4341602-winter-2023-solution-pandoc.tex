\documentclass[10pt,a4paper]{article}

% content/resources/templates/preamble.tex
\usepackage[margin=0.6in]{geometry}
\author{Milav Dabgar}
\usepackage{amsmath,amssymb,amsthm}
\usepackage{booktabs}
\usepackage{multirow}
\usepackage{xcolor}
\usepackage{tcolorbox}
\tcbuselibrary{breakable,skins}
\usepackage[colorlinks=true,linkcolor=blue]{hyperref}
\usepackage{titlesec}
\usepackage{enumitem}
\usepackage{tikz}
\usepackage{pgfplots}
\usepackage{circuitikz}
\usepackage[version=4]{mhchem}
\usepackage{longtable}
\usepackage{array}
\usepackage{float}
\usepackage{caption}
\usepackage{listings}

\lstset{
  basicstyle=\small\ttfamily,
  breaklines=true,
  breakatwhitespace=false,
  postbreak=\mbox{\textcolor{red}{$\hookrightarrow$}\space},
  float=false,
  numbers=left,
  numberstyle=\tiny\color{gray},
  numbersep=10pt,
  xleftmargin=2em,
  keywordstyle=\color{blue},
  commentstyle=\color{green!60!black},
  stringstyle=\color{purple},
  backgroundcolor=\color{gray!5},
  showstringspaces=false,
  tabsize=2,
  captionpos=b,
  keepspaces=true,
  columns=flexible
}

\pgfplotsset{compat=1.18}
\usetikzlibrary{shapes,arrows,positioning,calc,patterns,decorations.pathmorphing,decorations.markings,arrows.meta}

% Color scheme
\definecolor{headcolor}{RGB}{0,102,204}
\definecolor{keycolor}{RGB}{220,20,60}
\definecolor{solutioncolor}{RGB}{34,139,34}
\definecolor{mnemoniccolor}{RGB}{148,0,211}
\definecolor{codecolor}{RGB}{0,0,100}

% Spacing
\setlength{\parskip}{3pt}
\setlist[itemize]{nosep}
\setlist[enumerate]{nosep}

% Title formatting
\titleformat{\section}{\Large\bfseries\color{headcolor}}{\thesection}{1em}{}
\titleformat{\subsection}{\large\bfseries\color{headcolor}}{\thesubsection}{1em}{}

% Pandoc tightlist compatibility
\providecommand{\tightlist}{%
  \setlength{\itemsep}{0pt}\setlength{\parskip}{0pt}}

% Pandoc longtable compatibility
\newcounter{none}
\def\thenone{}


% content/resources/templates/english-boxes.tex
% This file is currently empty - it exists to maintain consistency with the import structure.
% Add custom environments here if needed in the future.


\begin{document}

\begin{center}
{\Huge\bfseries\color{headcolor} Subject Name Solutions}\\[5pt]
{\LARGE 4341602 -- Winter 2023}\\[3pt]
{\large Semester 1 Study Material}\\[3pt]
{\normalsize\textit{Detailed Solutions and Explanations}}
\end{center}

\vspace{10pt}

\subsection*{Question 1(a) [3 marks]}\label{q1a}

\textbf{List out basic concepts of oop. Explain any one in detail.}

\begin{solutionbox}

{\def\LTcaptype{none} % do not increment counter
\begin{longtable}[]{@{}ll@{}}
\toprule\noalign{}
Basic OOP Concepts & Description \\
\midrule\noalign{}
\endhead
\bottomrule\noalign{}
\endlastfoot
\textbf{Class} & Blueprint for objects \\
\textbf{Object} & Instance of a class \\
\textbf{Encapsulation} & Data hiding mechanism \\
\textbf{Inheritance} & Acquiring properties from parent \\
\textbf{Polymorphism} & One interface, multiple forms \\
\textbf{Abstraction} & Hiding implementation details \\
\end{longtable}
}

\textbf{Encapsulation} is the process of binding data and methods
together within a class and hiding internal implementation from outside
world. It provides data security by making variables private and
accessing them through public methods.

\end{solutionbox}
\begin{mnemonicbox}
``CEO-IPA'' (Class, Encapsulation, Object,
Inheritance, Polymorphism, Abstraction)

\end{mnemonicbox}
\subsection*{Question 1(b) [4 marks]}\label{q1b}

\textbf{Explain JVM in detail.}

\begin{solutionbox}

\begin{center}
\textbf{Mermaid Diagram (Code)}
\begin{verbatim}
{Shaded}
{Highlighting}[]
graph LR
    A[Java Source Code] {-{-}{} B[Java Compiler]}
    B {-{-}{} C[Bytecode .class]}
    C {-{-}{} D[JVM]}
    D {-{-}{} E[Machine Code]}
    E {-{-}{} F[Output]}
{Highlighting}
{Shaded}
\end{verbatim}
\end{center}

\textbf{JVM (Java Virtual Machine)} is a runtime environment that
executes Java bytecode. It provides platform independence by converting
bytecode to machine-specific code.

\begin{itemize}
\tightlist
\item
  \textbf{Class Loader}: Loads class files into memory
\item
  \textbf{Memory Management}: Handles heap and stack memory
\item
  \textbf{Execution Engine}: Executes bytecode instructions
\item
  \textbf{Garbage Collector}: Automatically manages memory
\end{itemize}

\end{solutionbox}
\begin{mnemonicbox}
``CMEG'' (Class loader, Memory, Execution, Garbage
collection)

\end{mnemonicbox}
\subsection*{Question 1(c) [7 marks]}\label{q1c}

\textbf{Write a program in java to print Fibonacci series for n terms.}

\begin{solutionbox}

\begin{verbatim}
public class Fibonacci \{
    public static void main(String[] args) \{
int

n = 10, first = 0, second = 1;

        System.out.print("Fibonacci Series: " + first + " " + second);
        
        for(int i = 2; i {} n; i++) \{
            int next = first + second;
            System.out.print(" " + next);
            first = second;
            second = next;
        \}
    \}
\}
\end{verbatim}

\begin{itemize}
\tightlist
\item
  \textbf{Logic}: Start with 0,1 and add previous two numbers
\item
  \textbf{Loop}: Continues for n terms
\item
  \textbf{Variables}: first, second, next for calculation
\end{itemize}

\end{solutionbox}
\begin{mnemonicbox}
``FSN'' (First, Second, Next)

\end{mnemonicbox}
\subsection*{Question 1(c OR) [7
marks]}\label{question-1c-or-7-marks}

\textbf{Write a program in java to find out minimum from any ten numbers
using command line argument.}

\begin{solutionbox}

\begin{verbatim}
public class FindMinimum \{
    public static void main(String[] args) \{
        if(args.length != 10) \{
            System.out.println("Please enter exactly 10 numbers");
            return;
        \}
        
        int min = Integer.parseInt(args[0]);
        for(int i = 1; i {} args.length; i++) \{
            int num = Integer.parseInt(args[i]);
            if(num {} min) \{
                min = num;
            \}
        \}
        System.out.println("Minimum number: " + min);
    \}
\}
\end{verbatim}

\begin{itemize}
\tightlist
\item
  \textbf{Command Line}: java FindMinimum 5 3 8 1 9 2 7 4 6 0
\item
  \textbf{Logic}: Compare each number with current minimum
\item
  \textbf{Method}: Integer.parseInt() converts string to integer
\end{itemize}

\end{solutionbox}
\begin{mnemonicbox}
``CIM'' (Check, Integer.parseInt, Minimum)

\end{mnemonicbox}
\subsection*{Question 2(a) [3 marks]}\label{q2a}

\textbf{What is wrapper class? Explain with example.}

\begin{solutionbox}

{\def\LTcaptype{none} % do not increment counter
\begin{longtable}[]{@{}ll@{}}
\toprule\noalign{}
Primitive & Wrapper Class \\
\midrule\noalign{}
\endhead
\bottomrule\noalign{}
\endlastfoot
int & Integer \\
char & Character \\
boolean & Boolean \\
double & Double \\
\end{longtable}
}

\textbf{Wrapper classes} convert primitive data types into objects. They
provide utility methods and enable primitives to be used in collections.

\textbf{Example}: \texttt{Integer\ obj\ =\ new\ Integer(25);} or
\texttt{Integer\ obj\ =\ 25;} (autoboxing)

\end{solutionbox}
\begin{mnemonicbox}
``POC'' (Primitive to Object Conversion)

\end{mnemonicbox}
\subsection*{Question 2(b) [4 marks]}\label{q2b}

\textbf{List out different features of java. Explain any two.}

\begin{solutionbox}

{\def\LTcaptype{none} % do not increment counter
\begin{longtable}[]{@{}ll@{}}
\toprule\noalign{}
Java Features & Description \\
\midrule\noalign{}
\endhead
\bottomrule\noalign{}
\endlastfoot
\textbf{Platform Independent} & Write once, run anywhere \\
\textbf{Object Oriented} & Everything is an object \\
\textbf{Simple} & Easy syntax, no pointers \\
\textbf{Secure} & Bytecode verification \\
\textbf{Robust} & Strong memory management \\
\textbf{Multithreaded} & Concurrent execution \\
\end{longtable}
}

\textbf{Platform Independence}: Java source code compiles to bytecode
which runs on any platform with JVM installed.

\textbf{Object Oriented}: Java follows OOP principles like
encapsulation, inheritance, and polymorphism for better code
organization.

\end{solutionbox}
\begin{mnemonicbox}
``POSSMR'' (Platform, Object, Simple, Secure,
Multithreaded, Robust)

\end{mnemonicbox}
\subsection*{Question 2(c) [7 marks]}\label{q2c}

\textbf{What is method overload? Explain with example.}

\begin{solutionbox}

\textbf{Method Overloading} allows multiple methods with same name but
different parameters in the same class.

\begin{verbatim}
class Calculator \{
    public int add(int a, int b) \{
        return a + b;
    \}
    
    public double add(double a, double b) \{
        return a + b;
    \}
    
    public int add(int a, int b, int c) \{
        return a + b + c;
    \}
\}
\end{verbatim}

\begin{itemize}
\tightlist
\item
  \textbf{Rules}: Different parameter types or number of parameters
\item
  \textbf{Compile Time}: Decision made during compilation
\item
  \textbf{Return Type}: Cannot be only difference
\end{itemize}

\end{solutionbox}
\begin{mnemonicbox}
``SNRT'' (Same Name, different paRameters, compile
Time)

\end{mnemonicbox}
\subsection*{Question 2(a OR) [3
marks]}\label{question-2a-or-3-marks}

\textbf{Explain Garbage collection in java.}

\begin{solutionbox}

\begin{verbatim}
Memory Areas:
┌─────────────┐
│    Heap     │  Objects stored here
├─────────────┤
│   Stack     │  Method calls
├─────────────┤
│   Method    │  Class definitions
│    Area     │
└─────────────┘
\end{verbatim}

\textbf{Garbage Collection} automatically deallocates memory of
unreferenced objects. JVM runs garbage collector periodically to free up
heap memory.

\begin{itemize}
\tightlist
\item
  \textbf{Automatic}: No manual memory management needed
\item
  \textbf{Mark and Sweep}: Marks unreferenced objects, then removes them
\end{itemize}

\end{solutionbox}
\begin{mnemonicbox}
``ARMS'' (Automatic Reference Management System)

\end{mnemonicbox}
\subsection*{Question 2(b OR) [4
marks]}\label{question-2b-or-4-marks}

\textbf{Explain final keyword with example.}

\begin{solutionbox}

{\def\LTcaptype{none} % do not increment counter
\begin{longtable}[]{@{}lll@{}}
\toprule\noalign{}
Usage & Description & Example \\
\midrule\noalign{}
\endhead
\bottomrule\noalign{}
\endlastfoot
\textbf{final variable} & Cannot be changed &
\texttt{final\ int\ x\ =\ 10;} \\
\textbf{final method} & Cannot be overridden &
\texttt{final\ void\ display()} \\
\textbf{final class} & Cannot be inherited &
\texttt{final\ class\ MyClass} \\
\end{longtable}
}

\textbf{Example}:

\begin{verbatim}
final class FinalClass \{
    final int value = 100;
    final void show() \{
        System.out.println("Final method");
    \}
\}
\end{verbatim}

\end{solutionbox}
\begin{mnemonicbox}
``VCM'' (Variable constant, Class not inherited,
Method not overridden)

\end{mnemonicbox}
\subsection*{Question 2(c OR) [7
marks]}\label{question-2c-or-7-marks}

\textbf{What is constructor? Explain parameterized constructor with
example.}

\begin{solutionbox}

\textbf{Constructor} is a special method that initializes objects when
created. It has same name as class and no return type.

\begin{verbatim}
class Student \{
    String name;
    int age;
    
    // Parameterized Constructor
    public Student(String n, int a) \{
        name = n;
        age = a;
    \}
    
    public void display() \{
        System.out.println("Name: " + name + ", Age: " + age);
    \}
\}

class Main \{
    public static void main(String[] args) \{
        Student s1 = new Student("John", 20);
        s1.display();
    \}
\}
\end{verbatim}

\begin{itemize}
\tightlist
\item
  \textbf{Purpose}: Initialize object with specific values
\item
  \textbf{Parameters}: Accepts arguments to set initial state
\item
  \textbf{Automatic}: Called automatically when object is created
\end{itemize}

\end{solutionbox}
\begin{mnemonicbox}
``SPA'' (Same name, Parameters, Automatic call)

\end{mnemonicbox}
\subsection*{Question 3(a) [3 marks]}\label{q3a}

\textbf{Explain super keyword with example.}

\begin{solutionbox}

\textbf{super keyword} refers to parent class members and constructor.
It resolves naming conflicts between parent and child classes.

\begin{verbatim}
class Parent \{
    int x = 10;
\}
class Child extends Parent \{
    int x = 20;
    void display() \{
        System.out.println(super.x); // 10
        System.out.println(x);       // 20
    \}
\}
\end{verbatim}

\begin{itemize}
\tightlist
\item
  \textbf{super.variable}: Access parent class variable
\item
  \textbf{super.method()}: Call parent class method
\item
  \textbf{super()}: Call parent class constructor
\end{itemize}

\end{solutionbox}
\begin{mnemonicbox}
``VMC'' (Variable, Method, Constructor)

\end{mnemonicbox}
\subsection*{Question 3(b) [4 marks]}\label{q3b}

\textbf{List out different types of inheritance. Explain multilevel
inheritance.}

\begin{solutionbox}

{\def\LTcaptype{none} % do not increment counter
\begin{longtable}[]{@{}ll@{}}
\toprule\noalign{}
Inheritance Types & Description \\
\midrule\noalign{}
\endhead
\bottomrule\noalign{}
\endlastfoot
\textbf{Single} & One parent, one child \\
\textbf{Multilevel} & Chain of inheritance \\
\textbf{Hierarchical} & One parent, multiple children \\
\textbf{Multiple} & Multiple parents (via interfaces) \\
\end{longtable}
}

\begin{center}
\textbf{Mermaid Diagram (Code)}
\begin{verbatim}
{Shaded}
{Highlighting}[]
graph LR
    A[Animal] {-{-}{} B[Mammal]}
    B {-{-}{} C[Dog]}
{Highlighting}
{Shaded}
\end{verbatim}
\end{center}

\textbf{Multilevel Inheritance}: Class inherits from another class which
itself inherits from another class, forming a chain.

\begin{verbatim}
class Animal \{
    void eat() \{ System.out.println("Eating"); \}
\}
class Mammal extends Animal \{
    void walk() \{ System.out.println("Walking"); \}
\}
class Dog extends Mammal \{
    void bark() \{ System.out.println("Barking"); \}
\}
\end{verbatim}

\end{solutionbox}
\begin{mnemonicbox}
``SMHM'' (Single, Multilevel, Hierarchical, Multiple)

\end{mnemonicbox}
\subsection*{Question 3(c) [7 marks]}\label{q3c}

\textbf{What is interface? Explain multiple inheritance with example.}

\begin{solutionbox}

\textbf{Interface} is a contract that defines what methods a class must
implement. It contains only abstract methods and constants.

\begin{verbatim}
interface Flyable \{
    void fly();
\}

interface Swimmable \{
    void swim();
\}

class Duck implements Flyable, Swimmable \{
    public void fly() \{
        System.out.println("Duck is flying");
    \}
    
    public void swim() \{
        System.out.println("Duck is swimming");
    \}
\}
\end{verbatim}

\textbf{Multiple Inheritance}: A class can implement multiple
interfaces, achieving multiple inheritance of behavior.

\begin{itemize}
\tightlist
\item
  \textbf{Abstract Methods}: All methods are abstract by default
\item
  \textbf{Constants}: All variables are public, static, final
\item
  \textbf{implements}: Keyword to implement interface
\end{itemize}

\end{solutionbox}
\begin{mnemonicbox}
``ACI'' (Abstract methods, Constants, implements
keyword)

\end{mnemonicbox}
\subsection*{Question 3(a OR) [3
marks]}\label{question-3a-or-3-marks}

\textbf{Explain static keyword with example.}

\begin{solutionbox}

\textbf{static keyword} creates class-level members that belong to class
rather than instances. Memory allocated once when class loads.

\begin{verbatim}
class Counter \{
    static int count = 0;
    static void increment() \{
        count++;
    \}
\}
\end{verbatim}

\begin{itemize}
\tightlist
\item
  \textbf{static variable}: Shared among all objects
\item
  \textbf{static method}: Called without object creation
\item
  \textbf{Memory}: Allocated in method area
\end{itemize}

\end{solutionbox}
\begin{mnemonicbox}
``SOM'' (Shared, Object not needed, Method area)

\end{mnemonicbox}
\subsection*{Question 3(b OR) [4
marks]}\label{question-3b-or-4-marks}

\textbf{Explain different access controls in Java.}

\begin{solutionbox}

{\def\LTcaptype{none} % do not increment counter
\begin{longtable}[]{@{}
  >{\raggedright\arraybackslash}p{(\linewidth - 8\tabcolsep) * \real{0.2254}}
  >{\raggedright\arraybackslash}p{(\linewidth - 8\tabcolsep) * \real{0.1690}}
  >{\raggedright\arraybackslash}p{(\linewidth - 8\tabcolsep) * \real{0.1972}}
  >{\raggedright\arraybackslash}p{(\linewidth - 8\tabcolsep) * \real{0.1408}}
  >{\raggedright\arraybackslash}p{(\linewidth - 8\tabcolsep) * \real{0.2676}}@{}}
\toprule\noalign{}
\begin{minipage}[b]{\linewidth}\raggedright
Access Modifier
\end{minipage} & \begin{minipage}[b]{\linewidth}\raggedright
Same Class
\end{minipage} & \begin{minipage}[b]{\linewidth}\raggedright
Same Package
\end{minipage} & \begin{minipage}[b]{\linewidth}\raggedright
Subclass
\end{minipage} & \begin{minipage}[b]{\linewidth}\raggedright
Different Package
\end{minipage} \\
\midrule\noalign{}
\endhead
\bottomrule\noalign{}
\endlastfoot
\textbf{private} & ✓ & ✗ & ✗ & ✗ \\
\textbf{default} & ✓ & ✓ & ✗ & ✗ \\
\textbf{protected} & ✓ & ✓ & ✓ & ✗ \\
\textbf{public} & ✓ & ✓ & ✓ & ✓ \\
\end{longtable}
}

\textbf{Access Control} determines visibility and accessibility of
classes, methods, and variables.

\end{solutionbox}
\begin{mnemonicbox}
``PriDef ProPub'' (Private, Default, Protected,
Public)

\end{mnemonicbox}
\subsection*{Question 3(c OR) [7
marks]}\label{question-3c-or-7-marks}

\textbf{What is package? Write steps to create a package and give
example of it.}

\begin{solutionbox}

\textbf{Package} is a namespace that organizes related classes and
interfaces. It provides access protection and namespace management.

\textbf{Steps to create package}:

\begin{enumerate}
\tightlist
\item
  Use \texttt{package} statement at top of file
\item
  Create directory structure matching package name
\item
  Compile with \texttt{-d} option
\item
  Import package in other files
\end{enumerate}

\begin{verbatim}
// File: com/mycompany/MyClass.java
package com.mycompany;

public class MyClass \{
    public void display() \{
        System.out.println("Package example");
    \}
\}

// Using the package
import com.mycompany.MyClass;

class Main \{
    public static void main(String[] args) \{
        MyClass obj = new MyClass();
        obj.display();
    \}
\}
\end{verbatim}

\textbf{Compilation}: \texttt{javac\ -d\ .\ MyClass.java}

\end{solutionbox}
\begin{mnemonicbox}
``PDCI'' (Package statement, Directory, Compile,
Import)

\end{mnemonicbox}
\subsection*{Question 4(a) [3 marks]}\label{q4a}

\textbf{Explain thread priorities with suitable example.}

\begin{solutionbox}

\textbf{Thread Priority} determines execution order of threads. Java
provides 10 priority levels from 1 (lowest) to 10 (highest).

\begin{verbatim}
class MyThread extends Thread \{
    public void run() \{
        System.out.println(getName() + " Priority: " + getPriority());
    \}
\}

class Main \{
    public static void main(String[] args) \{
        MyThread t1 = new MyThread();
        MyThread t2 = new MyThread();
        
        t1.setPriority(Thread.MIN\_PRIORITY); // 1
        t2.setPriority(Thread.MAX\_PRIORITY); // 10
        
        t1.start();
        t2.start();
    \}
\}
\end{verbatim}

\textbf{Priority Constants}: MIN\_PRIORITY (1), NORM\_PRIORITY (5),
MAX\_PRIORITY (10)

\end{solutionbox}
\begin{mnemonicbox}
``MNM'' (MIN, NORM, MAX)

\end{mnemonicbox}
\subsection*{Question 4(b) [4 marks]}\label{q4b}

\textbf{What is Thread? Explain Thread life cycle.}

\begin{solutionbox}

\begin{verbatim}
stateDiagram{-v2}
        direction LR
    [*] {-{-} New}
    New {-{-} Runnable : start()}
    Runnable {-{-} Running : Scheduler}
    Running {-{-} Blocked : wait/sleep}
    Blocked {-{-} Runnable : notify/timeout}
    Running {-{-} Dead : completes}
    Running {-{-} Runnable : yield()}
\end{verbatim}

\textbf{Thread} is a lightweight subprocess that enables concurrent
execution within a program.

\textbf{Thread Life Cycle States}:

\begin{itemize}
\tightlist
\item
  \textbf{New}: Thread created but not started
\item
  \textbf{Runnable}: Ready to run, waiting for CPU
\item
  \textbf{Running}: Currently executing
\item
  \textbf{Blocked}: Waiting for resource or I/O
\item
  \textbf{Dead}: Thread execution completed
\end{itemize}

\end{solutionbox}
\begin{mnemonicbox}
``NRRBD'' (New, Runnable, Running, Blocked, Dead)

\end{mnemonicbox}
\subsection*{Question 4(c) [7 marks]}\label{q4c}

\textbf{Write a program in java that create the multiple threads by
implementing the Thread class.}

\begin{solutionbox}

\begin{verbatim}
class MyThread extends Thread \{
    private String threadName;
    
    public MyThread(String name) \{
        threadName = name;
        setName(threadName);
    \}
    
    public void run() \{
for(int

i = 1; i {=} 5; i++) \{

            System.out.println(threadName + " {- Count: "} + i);
            try \{
                Thread.sleep(1000);
            \} catch(InterruptedException e) \{
                System.out.println(threadName + " interrupted");
            \}
        \}
        System.out.println(threadName + " completed");
    \}
\}

class Main \{
    public static void main(String[] args) \{
        MyThread thread1 = new MyThread("Thread{-1"});
        MyThread thread2 = new MyThread("Thread{-2"});
        MyThread thread3 = new MyThread("Thread{-3"});
        
        thread1.start();
        thread2.start();
        thread3.start();
    \}
\}
\end{verbatim}

\begin{itemize}
\tightlist
\item
  \textbf{extends Thread}: Inherit Thread class functionality
\item
  \textbf{Override run()}: Define thread execution logic
\item
  \textbf{start()}: Begin thread execution
\end{itemize}

\end{solutionbox}
\begin{mnemonicbox}
``EOS'' (Extends, Override run, Start method)

\end{mnemonicbox}
\subsection*{Question 4(a OR) [3
marks]}\label{question-4a-or-3-marks}

\textbf{List four different inbuilt exceptions. Explain any one inbuilt
exception.}

\begin{solutionbox}

{\def\LTcaptype{none} % do not increment counter
\begin{longtable}[]{@{}ll@{}}
\toprule\noalign{}
Inbuilt Exceptions & Description \\
\midrule\noalign{}
\endhead
\bottomrule\noalign{}
\endlastfoot
\textbf{NullPointerException} & Null reference access \\
\textbf{ArrayIndexOutOfBoundsException} & Invalid array index \\
\textbf{NumberFormatException} & Invalid number format \\
\textbf{ClassCastException} & Invalid type casting \\
\end{longtable}
}

\textbf{NullPointerException} occurs when trying to access methods or
variables of a null reference.

\begin{verbatim}
String str = null;
int length = str.length(); // Throws NullPointerException
\end{verbatim}

\end{solutionbox}
\begin{mnemonicbox}
``NANC'' (NullPointer, ArrayIndex, NumberFormat,
ClassCast)

\end{mnemonicbox}
\subsection*{Question 4(b OR) [4
marks]}\label{question-4b-or-4-marks}

\textbf{Explain multiple catch with suitable example.}

\begin{solutionbox}

\textbf{Multiple catch} blocks handle different types of exceptions that
might occur in try block. Each catch handles specific exception type.

\begin{verbatim}
class MultipleCatch \{
    public static void main(String[] args) \{
        try \{
            int[] arr = \{1, 2, 3\;}
            System.out.println(arr[5]); // ArrayIndexOutOfBoundsException
            int result = 10/0;          // ArithmeticException
        \}
        catch(ArrayIndexOutOfBoundsException e) \{
            System.out.println("Array index error: " + e.getMessage());
        \}
        catch(ArithmeticException e) \{
            System.out.println("Arithmetic error: " + e.getMessage());
        \}
        catch(Exception e) \{
            System.out.println("General error: " + e.getMessage());
        \}
    \}
\}
\end{verbatim}

\textbf{Order}: Specific exceptions first, general exceptions last

\end{solutionbox}
\begin{mnemonicbox}
``SGO'' (Specific first, General last, Ordered)

\end{mnemonicbox}
\subsection*{Question 4(c OR) [7
marks]}\label{question-4c-or-7-marks}

\textbf{What is Exception? Write a program that show the use of
Arithmetic Exception.}

\begin{solutionbox}

\textbf{Exception} is an abnormal condition that disrupts normal program
flow. It's an object representing an error condition.

\begin{verbatim}
class ArithmeticExceptionDemo \{
    public static void main(String[] args) \{
        int numerator = 100;
        int[] denominators = \{5, 0, 2, 0, 10\;}
        
        for(int i = 0; i {} denominators.length; i++) \{
            try \{
                int result = numerator / denominators[i];
                System.out.println(numerator + " / " + denominators[i] + " = " + result);
            \}
            catch(ArithmeticException e) \{
                System.out.println("Error: Cannot divide by zero!");
                System.out.println("Exception message: " + e.getMessage());
            \}
        \}
        
        System.out.println("Program continues after exception handling");
    \}
\}
\end{verbatim}

\textbf{ArithmeticException} thrown when mathematical error occurs like
division by zero.

\textbf{Exception Hierarchy}: Object \rightarrow Throwable \rightarrow Exception \rightarrow
RuntimeException \rightarrow ArithmeticException

\end{solutionbox}
\begin{mnemonicbox}
``OTERRA'' (Object, Throwable, Exception,
RuntimeException, ArithmeticException)

\end{mnemonicbox}
\subsection*{Question 5(a) [3 marks]}\label{q5a}

\textbf{Explain ArrayIndexOutOfBound Exception in Java with example.}

\begin{solutionbox}

\textbf{ArrayIndexOutOfBoundsException} occurs when accessing array
element with invalid index (negative or \textgreater= array length).

\begin{verbatim}
class ArrayException \{
    public static void main(String[] args) \{
        int[] numbers = \{10, 20, 30\;}
        
        try \{
            System.out.println(numbers[5]); // Invalid index
        \}
        catch(ArrayIndexOutOfBoundsException e) \{
            System.out.println("Invalid array index: " + e.getMessage());
        \}
    \}
\}
\end{verbatim}

\begin{itemize}
\tightlist
\item
  \textbf{Valid Range}: 0 to (length-1)
\item
  \textbf{Runtime Exception}: Unchecked exception
\item
  \textbf{Common Cause}: Loop condition errors
\end{itemize}

\end{solutionbox}
\begin{mnemonicbox}
``VRC'' (Valid range, Runtime exception, Common in
loops)

\end{mnemonicbox}
\subsection*{Question 5(b) [4 marks]}\label{q5b}

\textbf{Explain basics of stream classes.}

\begin{solutionbox}

\begin{center}
\textbf{Mermaid Diagram (Code)}
\begin{verbatim}
{Shaded}
{Highlighting}[]
graph TD
    A[Stream Classes] {-{-}{} B[Byte Streams]}
    A {-{-}{} C[Character Streams]}
    B {-{-}{} D[InputStream]}
    B {-{-}{} E[OutputStream]}
    C {-{-}{} F[Reader]}
    C {-{-}{} G[Writer]}
{Highlighting}
{Shaded}
\end{verbatim}
\end{center}

\textbf{Stream Classes} provide input/output operations for reading and
writing data.

{\def\LTcaptype{none} % do not increment counter
\begin{longtable}[]{@{}lll@{}}
\toprule\noalign{}
Stream Type & Purpose & Base Classes \\
\midrule\noalign{}
\endhead
\bottomrule\noalign{}
\endlastfoot
\textbf{Byte Streams} & Binary data & InputStream, OutputStream \\
\textbf{Character Streams} & Text data & Reader, Writer \\
\end{longtable}
}

\begin{itemize}
\tightlist
\item
  \textbf{Input Streams}: Read data from source
\item
  \textbf{Output Streams}: Write data to destination
\item
  \textbf{Buffered Streams}: Improve performance with buffering
\end{itemize}

\end{solutionbox}
\begin{mnemonicbox}
``BIOC'' (Byte, Input/Output, Character streams)

\end{mnemonicbox}
\subsection*{Question 5(c) [7 marks]}\label{q5c}

\textbf{Write a java program to create a text file and perform read
operation on the text file.}

\begin{solutionbox}

\begin{verbatim}
import java.io.*;

class FileOperations \{
    public static void main(String[] args) \{
        // Create and write to file
        try \{
            FileWriter writer = new FileWriter("sample.txt");
            writer.write("Hello World!{n}");
            writer.write("This is Java file handling example.{n}");
            writer.write("Learning Input/Output operations.");
            writer.close();
            System.out.println("File created and written successfully.");
        \}
        catch(IOException e) \{
            System.out.println("Error creating file: " + e.getMessage());
        \}
        
        // Read from file
        try \{
            FileReader reader = new FileReader("sample.txt");
            BufferedReader bufferedReader = new BufferedReader(reader);
            String line;
            
            System.out.println("{n}File contents:");
            while((line = bufferedReader.readLine()) != null) \{
                System.out.println(line);
            \}
            
            bufferedReader.close();
            reader.close();
        \}
        catch(IOException e) \{
            System.out.println("Error reading file: " + e.getMessage());
        \}
    \}
\}
\end{verbatim}

\begin{itemize}
\tightlist
\item
  \textbf{FileWriter}: Creates and writes to text file
\item
  \textbf{FileReader}: Reads from text file
\item
  \textbf{BufferedReader}: Efficient line-by-line reading
\end{itemize}

\end{solutionbox}
\begin{mnemonicbox}
``WRB'' (Writer creates, Reader reads, Buffered for
efficiency)

\end{mnemonicbox}
\subsection*{Question 5(a OR) [3
marks]}\label{question-5a-or-3-marks}

\textbf{Explain Divide by Zero Exception in Java with example.}

\begin{solutionbox}

\textbf{ArithmeticException (Divide by Zero)} occurs when integer is
divided by zero. Floating-point division by zero returns Infinity.

\begin{verbatim}
class DivideByZeroExample \{
    public static void main(String[] args) \{
        try \{
            int result = 10 / 0; // Throws ArithmeticException
            System.out.println("Result: " + result);
        \}
        catch(ArithmeticException e) \{
            System.out.println("Cannot divide by zero!");
        \}
        
        // Floating point division
        double floatResult = 10.0 / 0.0; // Returns Infinity
        System.out.println("Float result: " + floatResult);
    \}
\}
\end{verbatim}

\begin{itemize}
\tightlist
\item
  \textbf{Integer Division}: Throws ArithmeticException
\item
  \textbf{Float Division}: Returns Infinity or NaN
\end{itemize}

\end{solutionbox}
\begin{mnemonicbox}
``IFI'' (Integer throws exception, Float returns
Infinity)

\end{mnemonicbox}
\subsection*{Question 5(b OR) [4
marks]}\label{question-5b-or-4-marks}

\textbf{Explain java I/O process.}

\begin{solutionbox}

\begin{verbatim}
Java I/O Process:
┌─────────────┐    ┌─────────────┐    ┌─────────────┐
│   Source    │───▶│   Stream    │───▶│ Destination │
│  (File,     │    │  (Reader/   │    │ (File,      │
│  Keyboard,  │    │   Writer,   │    │  Screen,    │
│  Network)   │    │ Input/Output│    │  Network)   │
└─────────────┘    │   Stream)   │    └─────────────┘
                   └─────────────┘
\end{verbatim}

\textbf{Java I/O Process} handles data transfer between program and
external sources using streams.

{\def\LTcaptype{none} % do not increment counter
\begin{longtable}[]{@{}ll@{}}
\toprule\noalign{}
Component & Purpose \\
\midrule\noalign{}
\endhead
\bottomrule\noalign{}
\endlastfoot
\textbf{Source} & Data origin (file, keyboard, network) \\
\textbf{Stream} & Data pathway (byte/character streams) \\
\textbf{Destination} & Data target (file, screen, network) \\
\end{longtable}
}

\textbf{Process Steps}:

\begin{enumerate}
\tightlist
\item
  \textbf{Open Stream}: Create connection to source/destination
\item
  \textbf{Process Data}: Read/write operations
\item
  \textbf{Close Stream}: Release resources
\end{enumerate}

\end{solutionbox}
\begin{mnemonicbox}
``OPC'' (Open, Process, Close)

\end{mnemonicbox}
\subsection*{Question 5(c OR) [7
marks]}\label{question-5c-or-7-marks}

\textbf{Write a java program to display the content of a text file and
perform append operation on the text file.}

\begin{solutionbox}

\begin{verbatim}
import java.io.*;

class FileAppendExample \{
    public static void main(String[] args) \{
        String fileName = "data.txt";
        
        // Create initial file content
        try \{
            FileWriter writer = new FileWriter(fileName);
            writer.write("Initial content line 1{n}");
            writer.write("Initial content line 2{n}");
            writer.close();
            System.out.println("Initial file created.");
        \}
        catch(IOException e) \{
            System.out.println("Error creating file: " + e.getMessage());
        \}
        
        // Display file content
        displayFileContent(fileName);
        
        // Append to file
        try \{
            FileWriter appendWriter = new FileWriter(fileName, true); // true for append
            appendWriter.write("Appended line 1{n}");
            appendWriter.write("Appended line 2{n}");
            appendWriter.close();
            System.out.println("{n}Content appended successfully.");
        \}
        catch(IOException e) \{
            System.out.println("Error appending to file: " + e.getMessage());
        \}
        
        // Display updated content
        System.out.println("{n}File content after append:");
        displayFileContent(fileName);
    \}
    
    static void displayFileContent(String fileName) \{
        try \{
            BufferedReader reader = new BufferedReader(new FileReader(fileName));
            String line;
            System.out.println("{n}File contents:");
            while((line = reader.readLine()) != null) \{
                System.out.println(line);
            \}
            reader.close();
        \}
        catch(IOException e) \{
            System.out.println("Error reading file: " + e.getMessage());
        \}
    \}
\}
\end{verbatim}

\begin{itemize}
\tightlist
\item
  \textbf{FileWriter(filename, true)}: Append mode enabled
\item
  \textbf{displayFileContent()}: Reusable method for reading
\item
  \textbf{BufferedReader}: Efficient line reading
\end{itemize}

\end{solutionbox}
\begin{mnemonicbox}
``ARB'' (Append mode, Reusable method, Buffered
reading)

\end{mnemonicbox}

\end{document}
