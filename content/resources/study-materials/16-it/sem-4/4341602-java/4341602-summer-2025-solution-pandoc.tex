\documentclass[10pt,a4paper]{article}

% content/resources/templates/preamble.tex
\usepackage[margin=0.6in]{geometry}
\author{Milav Dabgar}
\usepackage{amsmath,amssymb,amsthm}
\usepackage{booktabs}
\usepackage{multirow}
\usepackage{xcolor}
\usepackage{tcolorbox}
\tcbuselibrary{breakable,skins}
\usepackage[colorlinks=true,linkcolor=blue]{hyperref}
\usepackage{titlesec}
\usepackage{enumitem}
\usepackage{tikz}
\usepackage{pgfplots}
\usepackage{circuitikz}
\usepackage[version=4]{mhchem}
\usepackage{longtable}
\usepackage{array}
\usepackage{float}
\usepackage{caption}
\usepackage{listings}

\lstset{
  basicstyle=\small\ttfamily,
  breaklines=true,
  breakatwhitespace=false,
  postbreak=\mbox{\textcolor{red}{$\hookrightarrow$}\space},
  float=false,
  numbers=left,
  numberstyle=\tiny\color{gray},
  numbersep=10pt,
  xleftmargin=2em,
  keywordstyle=\color{blue},
  commentstyle=\color{green!60!black},
  stringstyle=\color{purple},
  backgroundcolor=\color{gray!5},
  showstringspaces=false,
  tabsize=2,
  captionpos=b,
  keepspaces=true,
  columns=flexible
}

\pgfplotsset{compat=1.18}
\usetikzlibrary{shapes,arrows,positioning,calc,patterns,decorations.pathmorphing,decorations.markings,arrows.meta}

% Color scheme
\definecolor{headcolor}{RGB}{0,102,204}
\definecolor{keycolor}{RGB}{220,20,60}
\definecolor{solutioncolor}{RGB}{34,139,34}
\definecolor{mnemoniccolor}{RGB}{148,0,211}
\definecolor{codecolor}{RGB}{0,0,100}

% Spacing
\setlength{\parskip}{3pt}
\setlist[itemize]{nosep}
\setlist[enumerate]{nosep}

% Title formatting
\titleformat{\section}{\Large\bfseries\color{headcolor}}{\thesection}{1em}{}
\titleformat{\subsection}{\large\bfseries\color{headcolor}}{\thesubsection}{1em}{}

% Pandoc tightlist compatibility
\providecommand{\tightlist}{%
  \setlength{\itemsep}{0pt}\setlength{\parskip}{0pt}}

% Pandoc longtable compatibility
\newcounter{none}
\def\thenone{}


% content/resources/templates/english-boxes.tex
% This file is currently empty - it exists to maintain consistency with the import structure.
% Add custom environments here if needed in the future.


\begin{document}

\begin{center}
{\Huge\bfseries\color{headcolor} Subject Name Solutions}\\[5pt]
{\LARGE 4341602 -- Summer 2025}\\[3pt]
{\large Semester 1 Study Material}\\[3pt]
{\normalsize\textit{Detailed Solutions and Explanations}}
\end{center}

\vspace{10pt}

\subsection*{Question 1(a) [3 marks]}\label{q1a}

\textbf{Differentiate between Procedure Oriented Programming (POP) and
object-oriented programming (OOP).}

\begin{solutionbox}


{\def\LTcaptype{none} % do not increment counter
\begin{longtable}[]{@{}
  >{\raggedright\arraybackslash}p{(\linewidth - 4\tabcolsep) * \real{0.4444}}
  >{\raggedright\arraybackslash}p{(\linewidth - 4\tabcolsep) * \real{0.2778}}
  >{\raggedright\arraybackslash}p{(\linewidth - 4\tabcolsep) * \real{0.2778}}@{}}
\toprule\noalign{}
\begin{minipage}[b]{\linewidth}\raggedright
Aspect
\end{minipage} & \begin{minipage}[b]{\linewidth}\raggedright
POP
\end{minipage} & \begin{minipage}[b]{\linewidth}\raggedright
OOP
\end{minipage} \\
\midrule\noalign{}
\endhead
\bottomrule\noalign{}
\endlastfoot
\textbf{Approach} & Top-down approach & Bottom-up approach \\
\textbf{Focus} & Functions and procedures & Objects and classes \\
\textbf{Data Security} & Less secure, global data & More secure, data
encapsulation \\
\textbf{Problem Solving} & Divides into functions & Divides into
objects \\
\end{longtable}
}

\textbf{Key Points:}

\begin{itemize}
\tightlist
\item
  \textbf{POP}: Functions are primary building blocks
\item
  \textbf{OOP}: Objects contain both data and methods
\item
  \textbf{Reusability}: OOP provides better code reusability
\end{itemize}

\end{solutionbox}
\begin{mnemonicbox}
``POP Functions, OOP Objects''

\end{mnemonicbox}
\subsection*{Question 1(b) [4 marks]}\label{q1b}

\textbf{Enlist and explain the basic concepts of OOP.}

\begin{solutionbox}

\textbf{Basic OOP Concepts:}

\begin{itemize}
\tightlist
\item
  \textbf{Encapsulation}: Binding data and methods together in a class
\item
  \textbf{Inheritance}: Creating new classes from existing classes
\item
  \textbf{Polymorphism}: Same method name with different implementations
\item
  \textbf{Abstraction}: Hiding implementation details from user
\end{itemize}

\textbf{Benefits:}

\begin{itemize}
\tightlist
\item
  \textbf{Code Reusability}: Through inheritance and polymorphism
\item
  \textbf{Data Security}: Through encapsulation
\item
  \textbf{Easy Maintenance}: Modular approach
\end{itemize}

\end{solutionbox}
\begin{mnemonicbox}
``Every Intelligent Person Abstracts''

\end{mnemonicbox}
\subsection*{Question 1(c) [7 marks]}\label{q1c}

\textbf{Define Constructor. Enlist different types of Constructors and
explain any 2 of them with a proper example.}

\begin{solutionbox}

\textbf{Constructor Definition:} A constructor is a special method that
initializes objects when they are created. It has the same name as the
class and no return type.

\textbf{Types of Constructors:}

\begin{itemize}
\tightlist
\item
  \textbf{Default Constructor}: No parameters
\item
  \textbf{Parameterized Constructor}: Takes parameters
\item
  \textbf{Copy Constructor}: Creates object from another object
\item
  \textbf{Private Constructor}: Restricts object creation
\end{itemize}

\textbf{Code Example:}

\begin{verbatim}
class Student \{
    String name;
    int age;
    
    // Default Constructor
    public Student() \{
        name = "Unknown";
        age = 0;
    \}
    
    // Parameterized Constructor
    public Student(String n, int a) \{
        name = n;
        age = a;
    \}
\}

class Main \{
    public static void main(String[] args) \{
        Student s1 = new Student();        // Default
        Student s2 = new Student("John", 20); // Parameterized
    \}
\}
\end{verbatim}

\textbf{Key Features:}

\begin{itemize}
\tightlist
\item
  \textbf{Automatic Invocation}: Called automatically during object
  creation
\item
  \textbf{No Return Type}: Constructors don't have return type
\end{itemize}

\end{solutionbox}
\begin{mnemonicbox}
``Constructors Create Objects''

\end{mnemonicbox}
\subsection*{Question 1(c OR) [7
marks]}\label{question-1c-or-7-marks}

\textbf{Explain String class. Enlist different methods of String class
and explain any 3 of them with a proper example.}

\begin{solutionbox}

\textbf{String Class:} String class in Java represents immutable
character sequences. Once created, String objects cannot be modified.

\textbf{String Methods:}

{\def\LTcaptype{none} % do not increment counter
\begin{longtable}[]{@{}ll@{}}
\toprule\noalign{}
Method & Purpose \\
\midrule\noalign{}
\endhead
\bottomrule\noalign{}
\endlastfoot
length() & Returns string length \\
charAt(index) & Returns character at index \\
substring(start, end) & Extracts substring \\
indexOf(char) & Finds character position \\
toUpperCase() & Converts to uppercase \\
\end{longtable}
}

\textbf{Code Example:}

\begin{verbatim}
public class StringDemo \{
    public static void main(String[] args) \{
        String str = "Hello World";
        
        // length() method
        System.out.println("Length: " + str.length()); // 11
        
        // charAt() method
        System.out.println("Char at 0: " + str.charAt(0)); // H
        
        // substring() method
        System.out.println("Substring: " + str.substring(0, 5)); // Hello
    \}
\}
\end{verbatim}

\textbf{Key Points:}

\begin{itemize}
\tightlist
\item
  \textbf{Immutable}: String objects cannot be changed
\item
  \textbf{Memory Efficient}: String pool for storage
\end{itemize}

\end{solutionbox}
\begin{mnemonicbox}
``Strings Store Text''

\end{mnemonicbox}
\subsection*{Question 2(a) [3 marks]}\label{q2a}

\textbf{Define Garbage collection. Describe the importance of Garbage
collection in JAVA Programming.}

\begin{solutionbox}

\textbf{Garbage Collection Definition:} Automatic memory management
process that reclaims memory occupied by objects that are no longer
referenced.

\textbf{Importance:}

\begin{itemize}
\tightlist
\item
  \textbf{Automatic Memory Management}: No manual memory deallocation
  needed
\item
  \textbf{Prevents Memory Leaks}: Automatically frees unused memory
\item
  \textbf{Application Performance}: Optimizes memory usage
\end{itemize}

\textbf{Benefits:}

\begin{itemize}
\tightlist
\item
  \textbf{Programmer Productivity}: Focus on logic, not memory
  management
\item
  \textbf{Reliability}: Reduces crashes due to memory issues
\end{itemize}

\end{solutionbox}
\begin{mnemonicbox}
``Garbage Collector Cleans Memory''

\end{mnemonicbox}
\subsection*{Question 2(b) [4 marks]}\label{q2b}

\textbf{List down the four ways to make an object eligible for garbage
collection.}

\begin{solutionbox}

\textbf{Four Ways for GC Eligibility:}

{\def\LTcaptype{none} % do not increment counter
\begin{longtable}[]{@{}ll@{}}
\toprule\noalign{}
Method & Description \\
\midrule\noalign{}
\endhead
\bottomrule\noalign{}
\endlastfoot
\textbf{Nullifying Reference} & Set object reference to null \\
\textbf{Reassigning Reference} & Point reference to another object \\
\textbf{Anonymous Objects} & Create objects without reference \\
\textbf{Island of Isolation} & Objects refer only to each other \\
\end{longtable}
}

\textbf{Examples:}

\begin{itemize}
\tightlist
\item
  \textbf{Nullifying}: \texttt{obj\ =\ null;}
\item
  \textbf{Reassigning}: \texttt{obj1\ =\ obj2;}
\item
  \textbf{Anonymous}: \texttt{new\ Student();}
\item
  \textbf{Island}: Circular references with no external access
\end{itemize}

\end{solutionbox}
\begin{mnemonicbox}
``Null References Attract Islands''

\end{mnemonicbox}
\subsection*{Question 2(c) [7 marks]}\label{q2c}

\textbf{Write a Java Program to demonstrate a static block that gets
executed before main. Explain its significance.}

\begin{solutionbox}

\textbf{Code Example:}

\begin{verbatim}
public class StaticBlockDemo \{
    static int count;
    
    // Static block
    static \{
        System.out.println("Static block executed first");
        count = 10;
        System.out.println("Count initialized to: " + count);
    \}
    
    public static void main(String[] args) \{
        System.out.println("Main method started");
        System.out.println("Count value: " + count);
    \}
\}
\end{verbatim}

\textbf{Output:}

\begin{verbatim}
Static block executed first
Count initialized to: 10
Main method started
Count value: 10
\end{verbatim}

\textbf{Significance:}

\begin{itemize}
\tightlist
\item
  \textbf{Early Initialization}: Executes before main method
\item
  \textbf{Class Loading}: Runs when class is first loaded
\item
  \textbf{One-time Execution}: Executes only once per class
\end{itemize}

\textbf{Uses:}

\begin{itemize}
\tightlist
\item
  \textbf{Static Variable Initialization}: Initialize static variables
\item
  \textbf{Resource Loading}: Load configuration files
\end{itemize}

\end{solutionbox}
\begin{mnemonicbox}
``Static Blocks Start Before Main''

\end{mnemonicbox}
\subsection*{Question 2(a OR) [3
marks]}\label{question-2a-or-3-marks}

\textbf{Describe Minor/Incremental and Major/Full Garbage collection in
JAVA.}

\begin{solutionbox}

\textbf{Types of Garbage Collection:}

{\def\LTcaptype{none} % do not increment counter
\begin{longtable}[]{@{}lll@{}}
\toprule\noalign{}
Type & Description & Frequency \\
\midrule\noalign{}
\endhead
\bottomrule\noalign{}
\endlastfoot
\textbf{Minor GC} & Cleans young generation & Frequent \\
\textbf{Major GC} & Cleans old generation & Less frequent \\
\end{longtable}
}

\textbf{Minor GC:}

\begin{itemize}
\tightlist
\item
  \textbf{Target}: Young generation objects
\item
  \textbf{Speed}: Fast execution
\item
  \textbf{Impact}: Low application pause
\end{itemize}

\textbf{Major GC:}

\begin{itemize}
\tightlist
\item
  \textbf{Target}: Old generation objects
\item
  \textbf{Speed}: Slower execution
\item
  \textbf{Impact}: Higher application pause
\end{itemize}

\end{solutionbox}
\begin{mnemonicbox}
``Minor Frequent, Major Slow''

\end{mnemonicbox}
\subsection*{Question 2(b OR) [4
marks]}\label{question-2b-or-4-marks}

\textbf{Explicate the finalize() method in java with its advantages.}

\begin{solutionbox}

\textbf{finalize() Method:} Special method called by garbage collector
before object destruction for cleanup operations.

\textbf{Syntax:}

\begin{verbatim}
protected void finalize() throws Throwable \{
    // Cleanup code
\}
\end{verbatim}

\textbf{Advantages:}

\begin{itemize}
\tightlist
\item
  \textbf{Resource Cleanup}: Close files, database connections
\item
  \textbf{Memory Management}: Free native resources
\item
  \textbf{Safety Net}: Last chance for cleanup
\end{itemize}

\textbf{Example:}

\begin{verbatim}
class FileHandler \{
    protected void finalize() throws Throwable \{
        System.out.println("Cleanup before destruction");
        super.finalize();
    \}
\}
\end{verbatim}

\end{solutionbox}
\begin{mnemonicbox}
``Finalize Frees Resources''

\end{mnemonicbox}
\subsection*{Question 2(c OR) [7
marks]}\label{question-2c-or-7-marks}

\textbf{Explain the syntax of public static void main (String[]
args). Write a Java Program to print input taken as command line
argument.}

\begin{solutionbox}

\textbf{Main Method Syntax:}

\begin{verbatim}
public static void main(String[] args)
\end{verbatim}

\textbf{Explanation:}

\begin{itemize}
\tightlist
\item
  \textbf{public}: Accessible from anywhere
\item
  \textbf{static}: Can be called without object creation
\item
  \textbf{void}: No return value
\item
  \textbf{main}: Method name recognized by JVM
\item
  \textbf{String[] args}: Command line arguments array
\end{itemize}

\textbf{Code Example:}

\begin{verbatim}
public class CommandLineDemo \{
    public static void main(String[] args) \{
        System.out.println("Number of arguments: " + args.length);
        
        if(args.length {} 0) \{
            System.out.println("Command line arguments:");
            for(int i = 0; i {} args.length; i++) \{
                System.out.println("Arg " + i + ": " + args[i]);
            \}
        \} else \{
            System.out.println("No arguments provided");
        \}
    \}
\}
\end{verbatim}

\textbf{Execution:}

\begin{verbatim}
java CommandLineDemo Hello World 123
\end{verbatim}

\textbf{Output:}

\begin{verbatim}
Number of arguments: 3
Command line arguments:
Arg 0: Hello
Arg 1: World
Arg 2: 123
\end{verbatim}

\end{solutionbox}
\begin{mnemonicbox}
``Public Static Void Main Args''

\end{mnemonicbox}
\subsection*{Question 3(a) [3 marks]}\label{q3a}

\textbf{Enlist and Explain various Java access modifier(s).}

\begin{solutionbox}

\textbf{Java Access Modifiers:}

{\def\LTcaptype{none} % do not increment counter
\begin{longtable}[]{@{}lllll@{}}
\toprule\noalign{}
Modifier & Class & Package & Subclass & World \\
\midrule\noalign{}
\endhead
\bottomrule\noalign{}
\endlastfoot
\textbf{public} & ✓ & ✓ & ✓ & ✓ \\
\textbf{protected} & ✓ & ✓ & ✓ & ✗ \\
\textbf{default} & ✓ & ✓ & ✗ & ✗ \\
\textbf{private} & ✓ & ✗ & ✗ & ✗ \\
\end{longtable}
}

\textbf{Usage:}

\begin{itemize}
\tightlist
\item
  \textbf{public}: Accessible everywhere
\item
  \textbf{protected}: Accessible in package and subclasses
\item
  \textbf{default}: Package-level access only
\item
  \textbf{private}: Class-level access only
\end{itemize}

\end{solutionbox}
\begin{mnemonicbox}
``Public Protected Default Private''

\end{mnemonicbox}
\subsection*{Question 3(b) [4 marks]}\label{q3b}

\textbf{Describe interface in JAVA. Demonstrate inheritance of an
interface with an executable example.}

\begin{solutionbox}

\textbf{Interface in Java:} A contract that defines method signatures
without implementation. Classes implement interfaces to provide method
definitions.

\textbf{Interface Inheritance Example:}

\begin{verbatim}
// Parent interface
interface Animal \{
    void sound();
\}

// Child interface inheriting from Animal
interface Mammal extends Animal \{
    void walk();
\}

// Class implementing the child interface
class Dog implements Mammal \{
    public void sound() \{
        System.out.println("Dog barks");
    \}
    
    public void walk() \{
        System.out.println("Dog walks on four legs");
    \}
\}

class Main \{
    public static void main(String[] args) \{
        Dog d = new Dog();
        d.sound();
        d.walk();
    \}
\}
\end{verbatim}

\textbf{Key Features:}

\begin{itemize}
\tightlist
\item
  \textbf{Multiple Inheritance}: Interface supports multiple inheritance
\item
  \textbf{Contract}: Defines what class must implement
\end{itemize}

\end{solutionbox}
\begin{mnemonicbox}
``Interfaces Inherit Contracts''

\end{mnemonicbox}
\subsection*{Question 3(c) [7 marks]}\label{q3c}

\textbf{Define super keyword and demonstrate the use of super keyword
with an executable Java Program}

\begin{solutionbox}

\textbf{super Keyword:} References immediate parent class object. Used
to access parent class methods, variables, and constructors.

\textbf{Code Example:}

\begin{verbatim}
class Animal \{
    String name = "Animal";
    
    Animal(String type) \{
        System.out.println("Animal constructor: " + type);
    \}
    
    void sound() \{
        System.out.println("Animal makes sound");
    \}
\}

class Dog extends Animal \{
    String name = "Dog";
    
    Dog() \{
        super("Mammal");  // Call parent constructor
        System.out.println("Dog constructor");
    \}
    
    void sound() \{
        super.sound();    // Call parent method
        System.out.println("Dog barks");
    \}
    
    void display() \{
        System.out.println("Parent name: " + super.name);
        System.out.println("Child name: " + this.name);
    \}
\}

class Main \{
    public static void main(String[] args) \{
        Dog d = new Dog();
        d.sound();
        d.display();
    \}
\}
\end{verbatim}

\textbf{Uses of super:}

\begin{itemize}
\tightlist
\item
  \textbf{Constructor Call}: \texttt{super(parameters)}
\item
  \textbf{Method Call}: \texttt{super.methodName()}
\item
  \textbf{Variable Access}: \texttt{super.variableName}
\end{itemize}

\end{solutionbox}
\begin{mnemonicbox}
``Super Calls Parent''

\end{mnemonicbox}
\subsection*{Question 3(a OR) [3
marks]}\label{question-3a-or-3-marks}

\textbf{Explain package in JAVA with workable illustration.}

\begin{solutionbox}

\textbf{Package in Java:} A namespace that organizes related classes and
interfaces together. Provides access control and namespace management.

\textbf{Package Structure:}

\begin{verbatim}
com.company.project
├── model
│   └── Student.java
├── service
│   └── StudentService.java
└── Main.java
\end{verbatim}

\textbf{Example:}

\begin{verbatim}
// File: com/company/model/Student.java
package com.company.model;

public class Student \{
    private String name;
    public String getName() \{ return name; \}
    public void setName(String name) \{ this.name = name; \}
\}

// File: Main.java
import com.company.model.Student;

public class Main \{
    public static void main(String[] args) \{
        Student s = new Student();
        s.setName("John");
    \}
\}
\end{verbatim}

\textbf{Benefits:}

\begin{itemize}
\tightlist
\item
  \textbf{Organization}: Groups related classes
\item
  \textbf{Access Control}: Package-level access
\end{itemize}

\end{solutionbox}
\begin{mnemonicbox}
``Packages Organize Classes''

\end{mnemonicbox}
\subsection*{Question 3(b OR) [4
marks]}\label{question-3b-or-4-marks}

\textbf{Explain abstract and final keywords with a viable illustration.}

\begin{solutionbox}

\textbf{Keywords Explanation:}

{\def\LTcaptype{none} % do not increment counter
\begin{longtable}[]{@{}lll@{}}
\toprule\noalign{}
Keyword & Purpose & Usage \\
\midrule\noalign{}
\endhead
\bottomrule\noalign{}
\endlastfoot
\textbf{abstract} & Incomplete implementation & Classes and methods \\
\textbf{final} & Prevent modification & Classes, methods, variables \\
\end{longtable}
}

\textbf{Code Example:}

\begin{verbatim}
// Abstract class
abstract class Shape \{
    final double PI = 3.14;  // final variable
    
    abstract void draw();    // abstract method
    
    final void display() \{   // final method
        System.out.println("Displaying shape");
    \}
\}

// Final class
final class Circle extends Shape \{
    void draw() \{
        System.out.println("Drawing circle");
    \}
\}

// Cannot extend Circle class due to final
// class Oval extends Circle \{ \ // Error!}
\end{verbatim}

\textbf{Key Points:}

\begin{itemize}
\tightlist
\item
  \textbf{abstract}: Must be overridden in subclass
\item
  \textbf{final}: Cannot be overridden or extended
\end{itemize}

\end{solutionbox}
\begin{mnemonicbox}
``Abstract Allows, Final Forbids''

\end{mnemonicbox}
\subsection*{Question 3(c OR) [7
marks]}\label{question-3c-or-7-marks}

\textbf{State Dynamic Method Dispatch in Java Programming language
context. Construct an executable program demonstrating Dynamic Method
Dispatch.}

\begin{solutionbox}

\textbf{Dynamic Method Dispatch:} Runtime polymorphism where method call
is resolved during execution based on actual object type, not reference
type.

\textbf{Code Example:}

\begin{verbatim}
// Base class
class Animal \{
    void sound() \{
        System.out.println("Animal makes sound");
    \}
\}

// Derived classes
class Dog extends Animal \{
    void sound() \{
        System.out.println("Dog barks");
    \}
\}

class Cat extends Animal \{
    void sound() \{
        System.out.println("Cat meows");
    \}
\}

class DynamicDispatchDemo \{
    public static void main(String[] args) \{
        Animal ref;  // Reference variable
        
        // Runtime method resolution
        ref = new Dog();
        ref.sound();  // Calls Dog{s sound()}
        
        ref = new Cat();
        ref.sound();  // Calls Cat{s sound()}
        
        ref = new Animal();
        ref.sound();  // Calls Animal{s sound()}
    \}
\}
\end{verbatim}

\textbf{Output:}

\begin{verbatim}
Dog barks
Cat meows
Animal makes sound
\end{verbatim}

\textbf{Key Features:}

\begin{itemize}
\tightlist
\item
  \textbf{Runtime Resolution}: Method determined at runtime
\item
  \textbf{Polymorphism}: Same interface, different behavior
\item
  \textbf{Virtual Method Table}: JVM uses vtable for method lookup
\end{itemize}

\end{solutionbox}
\begin{mnemonicbox}
``Dynamic Dispatch Decides Runtime''

\end{mnemonicbox}
\subsection*{Question 4(a) [3 marks]}\label{q4a}

\textbf{Explain throw and finally keywords in Exception Handling.}

\begin{solutionbox}

\textbf{Exception Handling Keywords:}

{\def\LTcaptype{none} % do not increment counter
\begin{longtable}[]{@{}lll@{}}
\toprule\noalign{}
Keyword & Purpose & Usage \\
\midrule\noalign{}
\endhead
\bottomrule\noalign{}
\endlastfoot
\textbf{throw} & Manually throw exception &
\texttt{throw\ new\ Exception();} \\
\textbf{finally} & Always executed block & After try-catch \\
\end{longtable}
}

\textbf{Examples:}

\begin{verbatim}
// throw example
if(age {} 0) \{
    throw new IllegalArgumentException("Invalid age");
\}

// finally example
try \{
    // risky code
\} catch(Exception e) \{
    // handle exception
\} finally \{
    // cleanup code {- always executes}
\}
\end{verbatim}

\textbf{Key Points:}

\begin{itemize}
\tightlist
\item
  \textbf{throw}: Creates and throws exception explicitly
\item
  \textbf{finally}: Executes regardless of exception occurrence
\end{itemize}

\end{solutionbox}
\begin{mnemonicbox}
``Throw Creates, Finally Cleans''

\end{mnemonicbox}
\subsection*{Question 4(b) [4 marks]}\label{q4b}

\textbf{Write a program demonstrating try\ldots catch block in JAVA}

\begin{solutionbox}

\textbf{Code Example:}

\begin{verbatim}
public class TryCatchDemo \{
    public static void main(String[] args) \{
        try \{
            int[] arr = \{1, 2, 3\;}
            System.out.println("Array element: " + arr[5]); // Index out of bounds
            
            int result = 10 / 0; // Division by zero
            
        \} catch(ArrayIndexOutOfBoundsException e) \{
            System.out.println("Array index error: " + e.getMessage());
            
        \} catch(ArithmeticException e) \{
            System.out.println("Math error: " + e.getMessage());
            
        \} catch(Exception e) \{
            System.out.println("General error: " + e.getMessage());
        \}
        
        System.out.println("Program continues...");
    \}
\}
\end{verbatim}

\textbf{Output:}

\begin{verbatim}
Array index error: Index 5 out of bounds for length 3
Program continues...
\end{verbatim}

\textbf{Benefits:}

\begin{itemize}
\tightlist
\item
  \textbf{Exception Handling}: Graceful error management
\item
  \textbf{Program Continuity}: Program doesn't crash
\end{itemize}

\end{solutionbox}
\begin{mnemonicbox}
``Try Code, Catch Errors''

\end{mnemonicbox}
\subsection*{Question 4(c) [7 marks]}\label{q4c}

\textbf{Define ArrayIndexOutOfBoundsException Exception. Write a
workable JAVA program exhibiting it. Also mention input(s) which will
raise this Exception.}

\begin{solutionbox}

\textbf{ArrayIndexOutOfBoundsException:} Runtime exception thrown when
trying to access array element with invalid index (negative or
\textgreater= array length).

\textbf{Code Example:}

\begin{verbatim}
public class ArrayExceptionDemo \{
    public static void main(String[] args) \{
        int[] numbers = \{10, 20, 30, 40, 50\;} // Array size: 5
        
        try \{
            System.out.println("Array length: " + numbers.length);
            
            // Valid access
            System.out.println("Element at index 2: " + numbers[2]);
            
            // Invalid access {- will throw exception}
            System.out.println("Element at index 10: " + numbers[10]);
            
        \} catch(ArrayIndexOutOfBoundsException e) \{
            System.out.println("Exception caught: " + e.getMessage());
            System.out.println("Invalid index accessed!");
        \}
        
        System.out.println("Program completed successfully");
    \}
\}
\end{verbatim}

\textbf{Inputs that raise exception:}

\begin{itemize}
\tightlist
\item
  \textbf{Negative Index}: \texttt{arr[-1]}
\item
  \textbf{Index \textgreater= Length}: \texttt{arr[5]} for array of
  size 5
\item
  \textbf{Empty Array Access}: \texttt{arr[0]} for empty array
\end{itemize}

\textbf{Prevention:}

\begin{itemize}
\tightlist
\item
  \textbf{Bounds Checking}: Verify index before access
\item
  \textbf{Array Length}: Use \texttt{array.length} property
\end{itemize}

\end{solutionbox}
\begin{mnemonicbox}
``Array Bounds Break Programs''

\end{mnemonicbox}
\subsection*{Question 4(a OR) [3
marks]}\label{question-4a-or-3-marks}

\textbf{Draw and explain the life cycle of Thread in JAVA with example.}

\begin{solutionbox}

\textbf{Thread Life Cycle:}

\begin{verbatim}
stateDiagram{-v2}
        direction LR
    [*] {-{-} NEW}
    NEW {-{-} RUNNABLE : start()}
    RUNNABLE {-{-} RUNNING : CPU allocation}
    RUNNING {-{-} RUNNABLE : yield()}
    RUNNING {-{-} BLOCKED : wait for resource}
    BLOCKED {-{-} RUNNABLE : resource available}
    RUNNING {-{-} WAITING : wait()}
    WAITING {-{-} RUNNABLE : notify()}
    RUNNING {-{-} TIMED\_WAITING : sleep()}
    TIMED\_WAITING {-{-} RUNNABLE : timeout}
    RUNNING {-{-} TERMINATED : completion}
    TERMINATED {-{-} [*]}
\end{verbatim}

\textbf{States:}

\begin{itemize}
\tightlist
\item
  \textbf{NEW}: Thread created but not started
\item
  \textbf{RUNNABLE}: Ready to run or running
\item
  \textbf{BLOCKED}: Waiting for resource
\item
  \textbf{WAITING}: Waiting indefinitely
\item
  \textbf{TIMED\_WAITING}: Waiting for specific time
\item
  \textbf{TERMINATED}: Thread execution completed
\end{itemize}

\end{solutionbox}
\begin{mnemonicbox}
``New Runs, Blocks Wait, Terminates''

\end{mnemonicbox}
\subsection*{Question 4(b OR) [4
marks]}\label{question-4b-or-4-marks}

\textbf{Explain JAVA Optional class. Describe the OfNullable() method of
Optional class.}

\begin{solutionbox}

\textbf{Optional Class:} Container object that may or may not contain a
value. Helps avoid NullPointerException and makes code more readable.

\textbf{ofNullable() Method:} Returns Optional containing value if
non-null, otherwise returns empty Optional.

\textbf{Code Example:}

\begin{verbatim}
import java.util.Optional;

public class OptionalDemo \{
    public static void main(String[] args) \{
        String name1 = "John";
        String name2 = null;
        
        // ofNullable() examples
        Optional{}String{} opt1 = Optional.ofNullable(name1);
        Optional{}String{} opt2 = Optional.ofNullable(name2);
        
        System.out.println("opt1 present: " + opt1.isPresent()); // true
        System.out.println("opt2 present: " + opt2.isPresent()); // false
        
        // Safe value retrieval
        System.out.println("Name1: " + opt1.orElse("Unknown"));
        System.out.println("Name2: " + opt2.orElse("Unknown"));
    \}
\}
\end{verbatim}

\textbf{Benefits:}

\begin{itemize}
\tightlist
\item
  \textbf{Null Safety}: Prevents NullPointerException
\item
  \textbf{Readable Code}: Clear indication of optional values
\end{itemize}

\end{solutionbox}
\begin{mnemonicbox}
``Optional Offers Null Safety''

\end{mnemonicbox}
\subsection*{Question 4(c OR) [7
marks]}\label{question-4c-or-7-marks}

\textbf{Write a workable JAVA program showcasing nested try\ldots catch
block.}

\begin{solutionbox}

\textbf{Code Example:}

\begin{verbatim}
public class NestedTryCatchDemo \{
    public static void main(String[] args) \{
        try \{
            System.out.println("Outer try block started");
            
            int[] numbers = \{10, 20, 30\;}
            
            try \{
                System.out.println("Inner try block started");
                
                // This will cause ArrayIndexOutOfBoundsException
                System.out.println("Accessing index 5: " + numbers[5]);
                
                // This line won{t execute}
                int result = 100 / 0;
                
            \} catch(ArrayIndexOutOfBoundsException e) \{
                System.out.println("Inner catch: Array index error {- "} + e.getMessage());
                
                // Throwing new exception from inner catch
                throw new RuntimeException("Error in inner block");
            \}
            
            System.out.println("After inner try{-catch"});
            
        \} catch(RuntimeException e) \{
            System.out.println("Outer catch: Runtime error {- "} + e.getMessage());
            
        \} catch(Exception e) \{
            System.out.println("Outer catch: General error {- "} + e.getMessage());
            
        \} finally \{
            System.out.println("Outer finally: Cleanup operations");
        \}
        
        System.out.println("Program execution completed");
    \}
\}
\end{verbatim}

\textbf{Output:}

\begin{verbatim}
Outer try block started
Inner try block started
Inner catch: Array index error - Index 5 out of bounds for length 3
Outer catch: Runtime error - Error in inner block
Outer finally: Cleanup operations
Program execution completed
\end{verbatim}

\textbf{Key Features:}

\begin{itemize}
\tightlist
\item
  \textbf{Multiple Levels}: Inner and outer exception handling
\item
  \textbf{Exception Propagation}: Inner exceptions can be caught by
  outer blocks
\item
  \textbf{Specific Handling}: Different exceptions at different levels
\end{itemize}

\end{solutionbox}
\begin{mnemonicbox}
``Nested Try Catches Layers''

\end{mnemonicbox}
\subsection*{Question 5(a) [3 marks]}\label{q5a}

\textbf{Explain thread synchronization with an executable code in JAVA.}

\begin{solutionbox}

\textbf{Thread Synchronization:} Mechanism to control access to shared
resources by multiple threads to prevent data inconsistency and race
conditions.

\textbf{Code Example:}

\begin{verbatim}
class Counter \{
    private int count = 0;
    
    // Synchronized method
    public synchronized void increment() \{
        count++;
    \}
    
    public int getCount() \{
        return count;
    \}
\}

class SyncDemo extends Thread \{
    Counter counter;
    
    SyncDemo(Counter c) \{
        counter = c;
    \}
    
    public void run() \{
        for(int i = 0; i {} 1000; i++) \{
            counter.increment();
        \}
    \}
\}
\end{verbatim}

\textbf{Benefits:}

\begin{itemize}
\tightlist
\item
  \textbf{Data Consistency}: Prevents race conditions
\item
  \textbf{Thread Safety}: Safe access to shared resources
\end{itemize}

\end{solutionbox}
\begin{mnemonicbox}
``Synchronize Secures Shared Data''

\end{mnemonicbox}
\subsection*{Question 5(b) [4 marks]}\label{q5b}

\textbf{Enlist various stream classes in JAVA. Explain anyone with an
executable example.}

\begin{solutionbox}

\textbf{Stream Classes:}

{\def\LTcaptype{none} % do not increment counter
\begin{longtable}[]{@{}lll@{}}
\toprule\noalign{}
Class & Purpose & Type \\
\midrule\noalign{}
\endhead
\bottomrule\noalign{}
\endlastfoot
\textbf{FileInputStream} & Read bytes from file & Input \\
\textbf{FileOutputStream} & Write bytes to file & Output \\
\textbf{BufferedReader} & Buffered character reading & Input \\
\textbf{PrintWriter} & Formatted text output & Output \\
\end{longtable}
}

\textbf{FileInputStream Example:}

\begin{verbatim}
import java.io.*;

public class StreamDemo \{
    public static void main(String[] args) \{
        try \{
            // Create file and write data
            FileOutputStream fos = new FileOutputStream("test.txt");
            String data = "Hello World";
            fos.write(data.getBytes());
            fos.close();
            
            // Read file using FileInputStream
            FileInputStream fis = new FileInputStream("test.txt");
            int ch;
            while((ch = fis.read()) != {-}1) \{
                System.out.print((char)ch);
            \}
            fis.close();
            
        \} catch(IOException e) \{
            e.printStackTrace();
        \}
    \}
\}
\end{verbatim}

\textbf{Stream Features:}

\begin{itemize}
\tightlist
\item
  \textbf{Byte-oriented}: Handles binary data
\item
  \textbf{Character-oriented}: Handles text data
\end{itemize}

\end{solutionbox}
\begin{mnemonicbox}
``Streams Send Data''

\end{mnemonicbox}
\subsection*{Question 5(c) [7 marks]}\label{q5c}

\textbf{Write a JAVA program extending Thread class to display odd
numbers between given two integer numbers using thread.}

\begin{solutionbox}

\textbf{Code Example:}

\begin{verbatim}
class OddNumberThread extends Thread \{
    private int start;
    private int end;
    
    public OddNumberThread(int start, int end) \{
        this.start = start;
        this.end = end;
    \}
    
    @Override
    public void run() \{
        System.out.println("Thread started: " + Thread.currentThread().getName());
        System.out.println("Finding odd numbers between " + start + " and " + end);
        
for(int

i = start; i {=} end; i++) \{

            if(i \% 2 != 0) \{  // Check if number is odd
                System.out.println("Odd number: " + i);
                try \{
                    Thread.sleep(500);  // Pause for 500ms
                \} catch(InterruptedException e) \{
                    System.out.println("Thread interrupted");
                \}
            \}
        \}
        
        System.out.println("Thread completed: " + Thread.currentThread().getName());
    \}
\}

public class OddNumberDemo \{
    public static void main(String[] args) \{
        // Create thread objects
        OddNumberThread thread1 = new OddNumberThread(1, 10);
        OddNumberThread thread2 = new OddNumberThread(11, 20);
        
        // Set thread names
        thread1.setName("OddThread{-1"});
        thread2.setName("OddThread{-2"});
        
        // Start threads
        thread1.start();
        thread2.start();
        
        try \{
            // Wait for threads to complete
            thread1.join();
            thread2.join();
        \} catch(InterruptedException e) \{
            e.printStackTrace();
        \}
        
        System.out.println("All threads completed!");
    \}
\}
\end{verbatim}

\textbf{Output:}

\begin{verbatim}
Thread started: OddThread-1
Finding odd numbers between 1 and 10
Thread started: OddThread-2
Finding odd numbers between 11 and 20
Odd number: 1
Odd number: 11
Odd number: 3
Odd number: 13
...
\end{verbatim}

\textbf{Thread Features:}

\begin{itemize}
\tightlist
\item
  \textbf{Concurrent Execution}: Multiple threads run simultaneously
\item
  \textbf{Thread Extension}: Extends Thread class for custom behavior
\end{itemize}

\end{solutionbox}
\begin{mnemonicbox}
``Threads Take Turns''

\end{mnemonicbox}
\subsection*{Question 5(a OR) [3
marks]}\label{question-5a-or-3-marks}

\textbf{Explain join() and alive() methods of Thread class in JAVA.}

\begin{solutionbox}

\textbf{Thread Methods:}

{\def\LTcaptype{none} % do not increment counter
\begin{longtable}[]{@{}lll@{}}
\toprule\noalign{}
Method & Purpose & Return Type \\
\midrule\noalign{}
\endhead
\bottomrule\noalign{}
\endlastfoot
\textbf{join()} & Wait for thread completion & void \\
\textbf{isAlive()} & Check if thread is running & boolean \\
\end{longtable}
}

\textbf{Method Explanations:}

\begin{itemize}
\tightlist
\item
  \textbf{join()}: Current thread waits until the specified thread
  completes execution
\item
  \textbf{isAlive()}: Returns true if thread is still running, false if
  completed
\end{itemize}

\textbf{Code Example:}

\begin{verbatim}
class TestThread extends Thread \{
    public void run() \{
for(int

i = 1; i {=} 3; i++) \{

            System.out.println("Running: " + i);
            try \{ sleep(1000); \} catch(InterruptedException e) \{\}
        \}
    \}
\}

public class Main \{
    public static void main(String[] args) throws InterruptedException \{
        TestThread t = new TestThread();
        System.out.println("Before start: " + t.isAlive()); // false
        
        t.start();
        System.out.println("After start: " + t.isAlive()); // true
        
        t.join(); // Wait for completion
        System.out.println("After join: " + t.isAlive()); // false
    \}
\}
\end{verbatim}

\end{solutionbox}
\begin{mnemonicbox}
``Join Waits, Alive Checks''

\end{mnemonicbox}
\subsection*{Question 5(b OR) [4
marks]}\label{question-5b-or-4-marks}

\textbf{Define user-defined exceptions in JAVA. Write a program to show
user defined exception.}

\begin{solutionbox}

\textbf{User-defined Exceptions:} Custom exception classes created by
extending Exception class or its subclasses to handle specific
application errors.

\textbf{Code Example:}

\begin{verbatim}
// Custom exception class
class AgeValidationException extends Exception \{
    public AgeValidationException(String message) \{
        super(message);
    \}
\}

class Person \{
    private int age;
    
    public void setAge(int age) throws AgeValidationException \{
        if(age {} 0) \{
            throw new AgeValidationException("Age cannot be negative: " + age);
        \}
        if(age {} 150) \{
            throw new AgeValidationException("Age cannot exceed 150: " + age);
        \}
        this.age = age;
        System.out.println("Valid age set: " + age);
    \}
    
    public int getAge() \{
        return age;
    \}
\}

public class UserDefinedExceptionDemo \{
    public static void main(String[] args) \{
        Person person = new Person();
        
        try \{
            person.setAge(25);    // Valid age
            person.setAge({-}5);    // Invalid age {- throws exception}
            
        \} catch(AgeValidationException e) \{
            System.out.println("Custom Exception: " + e.getMessage());
        \}
        
        try \{
            person.setAge(200);   // Invalid age {- throws exception}
        \} catch(AgeValidationException e) \{
            System.out.println("Custom Exception: " + e.getMessage());
        \}
    \}
\}
\end{verbatim}

\textbf{Output:}

\begin{verbatim}
Valid age set: 25
Custom Exception: Age cannot be negative: -5
Custom Exception: Age cannot exceed 150: 200
\end{verbatim}

\textbf{Benefits:}

\begin{itemize}
\tightlist
\item
  \textbf{Specific Error Handling}: Handle application-specific errors
\item
  \textbf{Better Code Organization}: Separate exception logic
\end{itemize}

\end{solutionbox}
\begin{mnemonicbox}
``Custom Exceptions Catch Specific Errors''

\end{mnemonicbox}
\subsection*{Question 5(c OR) [7
marks]}\label{question-5c-or-7-marks}

\textbf{Write a JAVA program to copy content of file a.txt to b.txt.}

\begin{solutionbox}

\textbf{Code Example:}

\begin{verbatim}
import java.io.*;

public class FileCopyDemo \{
    public static void main(String[] args) \{
        String sourceFile = "a.txt";
        String targetFile = "b.txt";
        
        // Method 1: Using FileInputStream and FileOutputStream
        copyUsingStream(sourceFile, targetFile);
        
        // Method 2: Using BufferedReader and PrintWriter
        copyUsingBuffered(sourceFile, targetFile);
    \}
    
    // Method 1: Byte{-by{-}byte copy}
    public static void copyUsingStream(String source, String target) \{
        try \{
            // Create source file with sample data
            FileOutputStream createFile = new FileOutputStream(source);
            String data = "Hello World!{n}This is sample text.{n}Java File Operations.";
            createFile.write(data.getBytes());
            createFile.close();
            System.out.println("Source file created with sample data");
            
            // Copy file
            FileInputStream fis = new FileInputStream(source);
            FileOutputStream fos = new FileOutputStream(target);
            
            int ch;
            while((ch = fis.read()) != {-}1) \{
                fos.write(ch);
            \}
            
            fis.close();
            fos.close();
            System.out.println("File copied successfully using Stream");
            
        \} catch(IOException e) \{
            System.out.println("Error during file copy: " + e.getMessage());
        \}
    \}
    
    // Method 2: Line{-by{-}line copy with buffering}
    public static void copyUsingBuffered(String source, String target) \{
        try \{
            BufferedReader reader = new BufferedReader(new FileReader(source));
            PrintWriter writer = new PrintWriter(new FileWriter("buffered\_" + target));
            
            String line;
            while((line = reader.readLine()) != null) \{
                writer.println(line);
            \}
            
            reader.close();
            writer.close();
            System.out.println("File copied successfully using BufferedReader");
            
            // Display copied content
            displayFileContent("buffered\_" + target);
            
        \} catch(IOException e) \{
            System.out.println("Error during buffered copy: " + e.getMessage());
        \}
    \}
    
    // Helper method to display file content
    public static void displayFileContent(String filename) \{
        try \{
            System.out.println("{n}Content of " + filename + ":");
            BufferedReader reader = new BufferedReader(new FileReader(filename));
            String line;
            while((line = reader.readLine()) != null) \{
                System.out.println(line);
            \}
            reader.close();
            
        \} catch(IOException e) \{
            System.out.println("Error reading file: " + e.getMessage());
        \}
    \}
\}
\end{verbatim}

\textbf{Output:}

\begin{verbatim}
Source file created with sample data
File copied successfully using Stream
File copied successfully using BufferedReader

Content of buffered_b.txt:
Hello World!
This is sample text.
Java File Operations.
\end{verbatim}

\textbf{File Operations:}

\begin{itemize}
\tightlist
\item
  \textbf{FileInputStream/FileOutputStream}: Byte-level operations
\item
  \textbf{BufferedReader/PrintWriter}: Line-level operations with
  buffering
\item
  \textbf{Exception Handling}: Proper error management
\end{itemize}

\textbf{Key Features:}

\begin{itemize}
\tightlist
\item
  \textbf{Multiple Methods}: Different approaches for file copying
\item
  \textbf{Error Handling}: Try-catch blocks for IOException
\item
  \textbf{Resource Management}: Proper closing of file streams
\end{itemize}

\textbf{Best Practices:}

\begin{itemize}
\tightlist
\item
  \textbf{Close Streams}: Always close file streams after use
\item
  \textbf{Exception Handling}: Handle IOException properly
\item
  \textbf{Buffer Usage}: Use buffered streams for better performance
\end{itemize}

\end{solutionbox}
\begin{mnemonicbox}
``Files Flow From Source To Target''

\end{mnemonicbox}

\end{document}
