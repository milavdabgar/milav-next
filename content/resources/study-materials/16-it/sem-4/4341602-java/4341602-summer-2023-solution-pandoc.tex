\documentclass[10pt,a4paper]{article}

% content/resources/templates/preamble.tex
\usepackage[margin=0.6in]{geometry}
\author{Milav Dabgar}
\usepackage{amsmath,amssymb,amsthm}
\usepackage{booktabs}
\usepackage{multirow}
\usepackage{xcolor}
\usepackage{tcolorbox}
\tcbuselibrary{breakable,skins}
\usepackage[colorlinks=true,linkcolor=blue]{hyperref}
\usepackage{titlesec}
\usepackage{enumitem}
\usepackage{tikz}
\usepackage{pgfplots}
\usepackage{circuitikz}
\usepackage[version=4]{mhchem}
\usepackage{longtable}
\usepackage{array}
\usepackage{float}
\usepackage{caption}
\usepackage{listings}

\lstset{
  basicstyle=\small\ttfamily,
  breaklines=true,
  breakatwhitespace=false,
  postbreak=\mbox{\textcolor{red}{$\hookrightarrow$}\space},
  float=false,
  numbers=left,
  numberstyle=\tiny\color{gray},
  numbersep=10pt,
  xleftmargin=2em,
  keywordstyle=\color{blue},
  commentstyle=\color{green!60!black},
  stringstyle=\color{purple},
  backgroundcolor=\color{gray!5},
  showstringspaces=false,
  tabsize=2,
  captionpos=b,
  keepspaces=true,
  columns=flexible
}

\pgfplotsset{compat=1.18}
\usetikzlibrary{shapes,arrows,positioning,calc,patterns,decorations.pathmorphing,decorations.markings,arrows.meta}

% Color scheme
\definecolor{headcolor}{RGB}{0,102,204}
\definecolor{keycolor}{RGB}{220,20,60}
\definecolor{solutioncolor}{RGB}{34,139,34}
\definecolor{mnemoniccolor}{RGB}{148,0,211}
\definecolor{codecolor}{RGB}{0,0,100}

% Spacing
\setlength{\parskip}{3pt}
\setlist[itemize]{nosep}
\setlist[enumerate]{nosep}

% Title formatting
\titleformat{\section}{\Large\bfseries\color{headcolor}}{\thesection}{1em}{}
\titleformat{\subsection}{\large\bfseries\color{headcolor}}{\thesubsection}{1em}{}

% Pandoc tightlist compatibility
\providecommand{\tightlist}{%
  \setlength{\itemsep}{0pt}\setlength{\parskip}{0pt}}

% Pandoc longtable compatibility
\newcounter{none}
\def\thenone{}


% content/resources/templates/english-boxes.tex
% This file is currently empty - it exists to maintain consistency with the import structure.
% Add custom environments here if needed in the future.


\begin{document}

\begin{center}
{\Huge\bfseries\color{headcolor} Subject Name Solutions}\\[5pt]
{\LARGE 4341602 -- Summer 2023}\\[3pt]
{\large Semester 1 Study Material}\\[3pt]
{\normalsize\textit{Detailed Solutions and Explanations}}
\end{center}

\vspace{10pt}

\subsection*{Question 1(a) [3 marks]}\label{q1a}

\textbf{Differentiate between Procedure-Oriented Programming (POP) and
Object-Oriented Programming (OOP).}

\begin{solutionbox}


{\def\LTcaptype{none} % do not increment counter
\begin{longtable}[]{@{}
  >{\raggedright\arraybackslash}p{(\linewidth - 4\tabcolsep) * \real{0.4444}}
  >{\raggedright\arraybackslash}p{(\linewidth - 4\tabcolsep) * \real{0.2778}}
  >{\raggedright\arraybackslash}p{(\linewidth - 4\tabcolsep) * \real{0.2778}}@{}}
\toprule\noalign{}
\begin{minipage}[b]{\linewidth}\raggedright
Aspect
\end{minipage} & \begin{minipage}[b]{\linewidth}\raggedright
POP
\end{minipage} & \begin{minipage}[b]{\linewidth}\raggedright
OOP
\end{minipage} \\
\midrule\noalign{}
\endhead
\bottomrule\noalign{}
\endlastfoot
\textbf{Focus} & Functions/Procedures & Objects and Classes \\
\textbf{Data Security} & Less secure, global data & More secure, data
encapsulation \\
\textbf{Problem Solving} & Top-down approach & Bottom-up approach \\
\textbf{Code Reusability} & Limited & High through inheritance \\
\textbf{Examples} & C, Pascal & Java, C++, Python \\
\end{longtable}
}

\begin{itemize}
\tightlist
\item
  \textbf{POP}: Program divided into functions, data flows between
  functions
\item
  \textbf{OOP}: Program organized around objects that contain both data
  and methods
\end{itemize}

\end{solutionbox}
\begin{mnemonicbox}
``POP Functions, OOP Objects''

\end{mnemonicbox}
\begin{center}\rule{0.5\linewidth}{0.5pt}\end{center}

\subsection*{Question 1(b) [4 marks]}\label{q1b}

\textbf{Explain Super keyword in inheritance with suitable example.}

\begin{solutionbox}

\textbf{Super keyword} is used to access parent class members from child
class.


{\def\LTcaptype{none} % do not increment counter
\vspace{-5pt}
\captionof{table}{Super keyword uses}
\vspace{-10pt}
\begin{longtable}[]{@{}lll@{}}
\toprule\noalign{}
Use & Purpose & Example \\
\midrule\noalign{}
\endhead
\bottomrule\noalign{}
\endlastfoot
\textbf{super()} & Call parent constructor & super(name, age) \\
\textbf{super.method()} & Call parent method & super.display() \\
\textbf{super.variable} & Access parent variable & super.name \\
\end{longtable}
}

\textbf{Code Block:}

\begin{lstlisting}[language=Java]
class Animal {
    String name = "Animal";
    void eat() { System.out.println("Animal eats"); }
}

class Dog extends Animal {
    String name = "Dog";
    void eat() {
        super.eat(); // calls parent method
        System.out.println("Dog eats bones");
    }
    void display() {
        System.out.println(super.name); // prints "Animal"
    }
}
\end{lstlisting}

\end{solutionbox}
\begin{mnemonicbox}
``Super calls Parent''

\end{mnemonicbox}
\begin{center}\rule{0.5\linewidth}{0.5pt}\end{center}

\subsection*{Question 1(c) [7 marks]}\label{q1c}

\textbf{Define: Method Overriding. List out Rules for method overriding.
Write a java program that implements method overriding.}

\begin{solutionbox}

\textbf{Method Overriding}: Child class provides specific implementation
of parent class method with same signature.


{\def\LTcaptype{none} % do not increment counter
\vspace{-5pt}
\captionof{table}{Method Overriding Rules}
\vspace{-10pt}
\begin{longtable}[]{@{}ll@{}}
\toprule\noalign{}
Rule & Description \\
\midrule\noalign{}
\endhead
\bottomrule\noalign{}
\endlastfoot
\textbf{Same name} & Method name must be identical \\
\textbf{Same parameters} & Parameter list must match exactly \\
\textbf{IS-A relationship} & Must have inheritance \\
\textbf{Access modifier} & Cannot reduce visibility \\
\textbf{Return type} & Must be same or covariant \\
\end{longtable}
}

\textbf{Code Block:}

\begin{lstlisting}[language=Java]
class Shape {
    void draw() {
        System.out.println("Drawing a shape");
    }
}

class Circle extends Shape {
    @Override
    void draw() {
        System.out.println("Drawing a circle");
    }
}

class Main {
    public static void main(String[] args) {
        Shape s = new Circle();
        s.draw(); // Output: Drawing a circle
    }
}
\end{lstlisting}

\end{solutionbox}
\begin{mnemonicbox}
``Override Same Signature''

\end{mnemonicbox}
\begin{center}\rule{0.5\linewidth}{0.5pt}\end{center}

\subsection*{Question 1(c OR) [7
marks]}\label{question-1c-or-7-marks}

\textbf{Describe: Interface. Write a java program using interface to
demonstrate multiple inheritance.}

\begin{solutionbox}

\textbf{Interface}: Blueprint containing abstract methods and constants.
Classes implement interfaces to achieve multiple inheritance.


{\def\LTcaptype{none} % do not increment counter
\vspace{-5pt}
\captionof{table}{Interface Features}
\vspace{-10pt}
\begin{longtable}[]{@{}ll@{}}
\toprule\noalign{}
Feature & Description \\
\midrule\noalign{}
\endhead
\bottomrule\noalign{}
\endlastfoot
\textbf{Abstract methods} & No implementation (before Java 8) \\
\textbf{Constants} & All variables are public static final \\
\textbf{Multiple inheritance} & Class can implement multiple
interfaces \\
\textbf{Default methods} & Concrete methods (Java 8+) \\
\end{longtable}
}

\textbf{Code Block:}

\begin{lstlisting}[language=Java]
interface Flyable {
    void fly();
}

interface Swimmable {
    void swim();
}

class Duck implements Flyable, Swimmable {
    public void fly() {
        System.out.println("Duck flies");
    }
    
    public void swim() {
        System.out.println("Duck swims");
    }
}

class Main {
    public static void main(String[] args) {
        Duck d = new Duck();
        d.fly();
        d.swim();
    }
}
\end{lstlisting}

\end{solutionbox}
\begin{mnemonicbox}
``Interface Multiple Implementation''

\end{mnemonicbox}
\begin{center}\rule{0.5\linewidth}{0.5pt}\end{center}

\subsection*{Question 2(a) [3 marks]}\label{q2a}

\textbf{Explain the Java Program Structure with example.}

\begin{solutionbox}

\textbf{Java Program Structure} consists of package, imports, class
declaration, and main method.

\textbf{Diagram:}

\begin{lstlisting}
+------------------+
| Package statement|
+------------------+
| Import statements|
+------------------+
| Class declaration|
| +-------------+  |
| | Variables   |  |
| +-------------+  |
| | Methods     |  |
| +-------------+  |
+------------------+
\end{lstlisting}

\textbf{Code Block:}

\begin{lstlisting}[language=Java]
package com.example;        // Package
import java.util.*;         // Import

public class HelloWorld {   // Class
    static int count = 0;   // Variable
    
    public static void main(String[] args) { // Main method
        System.out.println("Hello World");
    }
}
\end{lstlisting}

\end{solutionbox}
\begin{mnemonicbox}
``Package Import Class Main''

\end{mnemonicbox}
\begin{center}\rule{0.5\linewidth}{0.5pt}\end{center}

\subsection*{Question 2(b) [4 marks]}\label{q2b}

\textbf{Explain static keyword with suitable example.}

\begin{solutionbox}

\textbf{Static keyword} belongs to class rather than instance. Memory
allocated once.


{\def\LTcaptype{none} % do not increment counter
\vspace{-5pt}
\captionof{table}{Static Uses}
\vspace{-10pt}
\begin{longtable}[]{@{}lll@{}}
\toprule\noalign{}
Type & Description & Example \\
\midrule\noalign{}
\endhead
\bottomrule\noalign{}
\endlastfoot
\textbf{Static variable} & Shared by all objects & static int count \\
\textbf{Static method} & Called without object & static void
display() \\
\textbf{Static block} & Executes before main & static \{ \} \\
\end{longtable}
}

\textbf{Code Block:}

\begin{lstlisting}[language=Java]
class Student {
    static String college = "GTU";  // static variable
    String name;
    
    static void showCollege() {     // static method
        System.out.println(college);
    }
    
    static {                        // static block
        System.out.println("Static block executed");
    }
}

class Main {
    public static void main(String[] args) {
        Student.showCollege(); // No object needed
    }
}
\end{lstlisting}

\end{solutionbox}
\begin{mnemonicbox}
``Static Shared by Class''

\end{mnemonicbox}
\begin{center}\rule{0.5\linewidth}{0.5pt}\end{center}

\subsection*{Question 2(c) [7 marks]}\label{q2c}

\textbf{Define: Constructor. List out types of it. Explain Parameterized
and copy constructor with suitable example.}

\begin{solutionbox}

\textbf{Constructor}: Special method to initialize objects, same name as
class, no return type.


{\def\LTcaptype{none} % do not increment counter
\vspace{-5pt}
\captionof{table}{Constructor Types}
\vspace{-10pt}
\begin{longtable}[]{@{}lll@{}}
\toprule\noalign{}
Type & Description & Example \\
\midrule\noalign{}
\endhead
\bottomrule\noalign{}
\endlastfoot
\textbf{Default} & No parameters & Student() \\
\textbf{Parameterized} & With parameters & Student(String name) \\
\textbf{Copy} & Creates copy of object & Student(Student s) \\
\end{longtable}
}

\textbf{Code Block:}

\begin{lstlisting}[language=Java]
class Student {
    String name;
    int age;
    
    // Parameterized constructor
    Student(String n, int a) {
        name = n;
        age = a;
    }
    
    // Copy constructor
    Student(Student s) {
        name = s.name;
        age = s.age;
    }
    
    void display() {
        System.out.println(name + " " + age);
    }
}

class Main {
    public static void main(String[] args) {
        Student s1 = new Student("John", 20);  // Parameterized
        Student s2 = new Student(s1);          // Copy
        s1.display();
        s2.display();
    }
}
\end{lstlisting}

\end{solutionbox}
\begin{mnemonicbox}
``Constructor Initializes Objects''

\end{mnemonicbox}
\begin{center}\rule{0.5\linewidth}{0.5pt}\end{center}

\subsection*{Question 2(a OR) [3
marks]}\label{question-2a-or-3-marks}

\textbf{Explain the Primitive Data Types and User Defined Data Types in
java.}

\begin{solutionbox}

\textbf{Primitive Data Types}: Built-in types provided by Java language.
\textbf{User Defined Types}: Custom types created by programmer using
classes.


{\def\LTcaptype{none} % do not increment counter
\vspace{-5pt}
\captionof{table}{Data Types}
\vspace{-10pt}
\begin{longtable}[]{@{}llll@{}}
\toprule\noalign{}
Category & Types & Size & Example \\
\midrule\noalign{}
\endhead
\bottomrule\noalign{}
\endlastfoot
\textbf{Primitive} & byte, short, int, long & 1,2,4,8 bytes & int x =
10; \\
\textbf{Primitive} & float, double & 4,8 bytes & double d = 3.14; \\
\textbf{Primitive} & char, boolean & 2,1 bytes & char c = `A'; \\
\textbf{User Defined} & Class, Interface, Array & Variable & Student
s; \\
\end{longtable}
}

\begin{itemize}
\tightlist
\item
  \textbf{Primitive}: Stored in stack, faster access
\item
  \textbf{User Defined}: Stored in heap, complex operations
\end{itemize}

\end{solutionbox}
\begin{mnemonicbox}
``Primitive Built-in, User Custom''

\end{mnemonicbox}
\begin{center}\rule{0.5\linewidth}{0.5pt}\end{center}

\subsection*{Question 2(b OR) [4
marks]}\label{question-2b-or-4-marks}

\textbf{Explain this keyword with suitable example.}

\begin{solutionbox}

\textbf{This keyword} refers to current object instance, used to
distinguish between instance and local variables.


{\def\LTcaptype{none} % do not increment counter
\vspace{-5pt}
\captionof{table}{This keyword uses}
\vspace{-10pt}
\begin{longtable}[]{@{}lll@{}}
\toprule\noalign{}
Use & Purpose & Example \\
\midrule\noalign{}
\endhead
\bottomrule\noalign{}
\endlastfoot
\textbf{this.variable} & Access instance variable & this.name = name; \\
\textbf{this.method()} & Call instance method & this.display(); \\
\textbf{this()} & Call constructor & this(name, age); \\
\end{longtable}
}

\textbf{Code Block:}

\begin{lstlisting}[language=Java]
class Student {
    String name;
    int age;
    
    Student(String name, int age) {
        this.name = name;    // this distinguishes
        this.age = age;      // instance from parameter
    }
    
    void setData(String name) {
        this.name = name;    // this refers to current object
    }
    
    void display() {
        System.out.println(this.name + " " + this.age);
    }
}
\end{lstlisting}

\end{solutionbox}
\begin{mnemonicbox}
``This Current Object''

\end{mnemonicbox}
\begin{center}\rule{0.5\linewidth}{0.5pt}\end{center}

\subsection*{Question 2(c OR) [7
marks]}\label{question-2c-or-7-marks}

\textbf{Define Inheritance. List out types of it. Explain multilevel and
hierarchical inheritance with suitable example.}

\begin{solutionbox}

\textbf{Inheritance}: Mechanism where child class acquires properties
and methods of parent class.


{\def\LTcaptype{none} % do not increment counter
\vspace{-5pt}
\captionof{table}{Inheritance Types}
\vspace{-10pt}
\begin{longtable}[]{@{}lll@{}}
\toprule\noalign{}
Type & Description & Structure \\
\midrule\noalign{}
\endhead
\bottomrule\noalign{}
\endlastfoot
\textbf{Single} & One parent, one child & A \rightarrow B \\
\textbf{Multilevel} & Chain of inheritance & A \rightarrow B \rightarrow C \\
\textbf{Hierarchical} & One parent, multiple children & A \rightarrow B, A \rightarrow C \\
\textbf{Multiple} & Multiple parents (via interfaces) & B,C \rightarrow A \\
\end{longtable}
}

\textbf{Diagram - Multilevel:}

\includegraphics[width=1\linewidth,height=\textheight,keepaspectratio]{mermaid-3d2aa48e.pdf}

\textbf{Code Block - Multilevel:}

\begin{lstlisting}[language=Java]
class Animal {
    void eat() { System.out.println("Animal eats"); }
}

class Mammal extends Animal {
    void breathe() { System.out.println("Mammal breathes"); }
}

class Dog extends Mammal {
    void bark() { System.out.println("Dog barks"); }
}
\end{lstlisting}

\textbf{Diagram - Hierarchical:}

\includegraphics[width=1\linewidth,height=\textheight,keepaspectratio]{mermaid-6bce1e95.pdf}

\textbf{Code Block - Hierarchical:}

\begin{lstlisting}[language=Java]
class Shape {
    void draw() { System.out.println("Drawing shape"); }
}

class Circle extends Shape {
    void drawCircle() { System.out.println("Drawing circle"); }
}

class Rectangle extends Shape {
    void drawRectangle() { System.out.println("Drawing rectangle"); }
}
\end{lstlisting}

\end{solutionbox}
\begin{mnemonicbox}
``Inheritance Shares Properties''

\end{mnemonicbox}
\begin{center}\rule{0.5\linewidth}{0.5pt}\end{center}

\subsection*{Question 3(a) [3 marks]}\label{q3a}

\textbf{Explain Type Conversion and Casting in java.}

\begin{solutionbox}

\textbf{Type Conversion}: Converting one data type to another.
\textbf{Casting}: Explicit type conversion by programmer.


{\def\LTcaptype{none} % do not increment counter
\vspace{-5pt}
\captionof{table}{Type Conversion}
\vspace{-10pt}
\begin{longtable}[]{@{}
  >{\raggedright\arraybackslash}p{(\linewidth - 4\tabcolsep) * \real{0.2143}}
  >{\raggedright\arraybackslash}p{(\linewidth - 4\tabcolsep) * \real{0.4643}}
  >{\raggedright\arraybackslash}p{(\linewidth - 4\tabcolsep) * \real{0.3214}}@{}}
\toprule\noalign{}
\begin{minipage}[b]{\linewidth}\raggedright
Type
\end{minipage} & \begin{minipage}[b]{\linewidth}\raggedright
Description
\end{minipage} & \begin{minipage}[b]{\linewidth}\raggedright
Example
\end{minipage} \\
\midrule\noalign{}
\endhead
\bottomrule\noalign{}
\endlastfoot
\textbf{Implicit (Widening)} & Automatic, smaller to larger & int to
double \\
\textbf{Explicit (Narrowing)} & Manual, larger to smaller & double to
int \\
\end{longtable}
}

\textbf{Code Block:}

\begin{lstlisting}[language=Java]
// Implicit conversion
int i = 10;
double d = i;        // int to double (automatic)

// Explicit casting
double x = 10.5;
int y = (int) x;     // double to int (manual)

// String conversion
String s = String.valueOf(i);    // int to String
int z = Integer.parseInt("123"); // String to int
\end{lstlisting}

\end{solutionbox}
\begin{mnemonicbox}
``Implicit Auto, Explicit Manual''

\end{mnemonicbox}
\begin{center}\rule{0.5\linewidth}{0.5pt}\end{center}

\subsection*{Question 3(b) [4 marks]}\label{q3b}

\textbf{Explain different visibility controls used in Java.}

\begin{solutionbox}

\textbf{Visibility Controls (Access Modifiers)}: Control access to
classes, methods, and variables.


{\def\LTcaptype{none} % do not increment counter
\vspace{-5pt}
\captionof{table}{Access Modifiers}
\vspace{-10pt}
\begin{longtable}[]{@{}lllll@{}}
\toprule\noalign{}
Modifier & Same Class & Same Package & Subclass & Different Package \\
\midrule\noalign{}
\endhead
\bottomrule\noalign{}
\endlastfoot
\textbf{private} & ✓ & ✗ & ✗ & ✗ \\
\textbf{default} & ✓ & ✓ & ✗ & ✗ \\
\textbf{protected} & ✓ & ✓ & ✓ & ✗ \\
\textbf{public} & ✓ & ✓ & ✓ & ✓ \\
\end{longtable}
}

\textbf{Code Block:}

\begin{lstlisting}[language=Java]
class Example {
    private int x = 10;      // Only within class
    int y = 20;              // Package level
    protected int z = 30;    // Package + subclass
    public int w = 40;       // Everywhere
    
    private void method1() { }    // Private method
    public void method2() { }     // Public method
}
\end{lstlisting}

\end{solutionbox}
\begin{mnemonicbox}
``Private Package Protected Public''

\end{mnemonicbox}
\begin{center}\rule{0.5\linewidth}{0.5pt}\end{center}

\subsection*{Question 3(c) [7 marks]}\label{q3c}

\textbf{Define: Thread. List different methods used to create Thread.
Explain Thread life cycle in detail.}

\begin{solutionbox}

\textbf{Thread}: Lightweight subprocess that allows concurrent execution
of multiple parts of program.


{\def\LTcaptype{none} % do not increment counter
\vspace{-5pt}
\captionof{table}{Thread Creation Methods}
\vspace{-10pt}
\begin{longtable}[]{@{}
  >{\raggedright\arraybackslash}p{(\linewidth - 4\tabcolsep) * \real{0.2667}}
  >{\raggedright\arraybackslash}p{(\linewidth - 4\tabcolsep) * \real{0.4333}}
  >{\raggedright\arraybackslash}p{(\linewidth - 4\tabcolsep) * \real{0.3000}}@{}}
\toprule\noalign{}
\begin{minipage}[b]{\linewidth}\raggedright
Method
\end{minipage} & \begin{minipage}[b]{\linewidth}\raggedright
Description
\end{minipage} & \begin{minipage}[b]{\linewidth}\raggedright
Example
\end{minipage} \\
\midrule\noalign{}
\endhead
\bottomrule\noalign{}
\endlastfoot
\textbf{Extending Thread} & Inherit Thread class & class MyThread
extends Thread \\
\textbf{Implementing Runnable} & Implement Runnable interface & class
MyTask implements Runnable \\
\end{longtable}
}

\textbf{Diagram: Thread Life Cycle}

\includegraphics[width=1\linewidth,height=\textheight,keepaspectratio]{mermaid-43721078.pdf}


{\def\LTcaptype{none} % do not increment counter
\vspace{-5pt}
\captionof{table}{Thread States}
\vspace{-10pt}
\begin{longtable}[]{@{}ll@{}}
\toprule\noalign{}
State & Description \\
\midrule\noalign{}
\endhead
\bottomrule\noalign{}
\endlastfoot
\textbf{NEW} & Thread created but not started \\
\textbf{RUNNABLE} & Ready to run, waiting for CPU \\
\textbf{RUNNING} & Currently executing \\
\textbf{BLOCKED} & Waiting for resource or sleep \\
\textbf{TERMINATED} & Execution completed \\
\end{longtable}
}

\textbf{Code Block:}

\begin{lstlisting}[language=Java]
// Method 1: Extending Thread
class MyThread extends Thread {
    public void run() {
        System.out.println("Thread running");
    }
}

// Method 2: Implementing Runnable
class MyTask implements Runnable {
    public void run() {
        System.out.println("Task running");
    }
}

class Main {
    public static void main(String[] args) {
        MyThread t1 = new MyThread();
        Thread t2 = new Thread(new MyTask());
        t1.start();
        t2.start();
    }
}
\end{lstlisting}

\end{solutionbox}
\begin{mnemonicbox}
``Thread Concurrent Execution''

\end{mnemonicbox}
\begin{center}\rule{0.5\linewidth}{0.5pt}\end{center}

\subsection*{Question 3(a OR) [3
marks]}\label{question-3a-or-3-marks}

\textbf{Explain the purpose of JVM in java.}

\begin{solutionbox}

\textbf{JVM (Java Virtual Machine)}: Runtime environment that executes
Java bytecode and provides platform independence.


{\def\LTcaptype{none} % do not increment counter
\vspace{-5pt}
\captionof{table}{JVM Components}
\vspace{-10pt}
\begin{longtable}[]{@{}ll@{}}
\toprule\noalign{}
Component & Purpose \\
\midrule\noalign{}
\endhead
\bottomrule\noalign{}
\endlastfoot
\textbf{Class Loader} & Loads .class files into memory \\
\textbf{Execution Engine} & Executes bytecode \\
\textbf{Memory Area} & Manages heap and stack memory \\
\textbf{Garbage Collector} & Automatic memory management \\
\end{longtable}
}

\textbf{Diagram:}

\begin{lstlisting}
+----------------+
| Java Source    |
| (.java)        |
+----------------+
        |
        v
+----------------+
| Java Compiler  |
| (javac)        |
+----------------+
        |
        v
+----------------+
| Bytecode       |
| (.class)       |
+----------------+
        |
        v
+----------------+
| JVM            |
| (Platform      |
|  Specific)     |
+----------------+
\end{lstlisting}

\begin{itemize}
\tightlist
\item
  \textbf{Platform Independence}: ``Write Once, Run Anywhere''
\item
  \textbf{Memory Management}: Automatic garbage collection
\item
  \textbf{Security}: Bytecode verification
\end{itemize}

\end{solutionbox}
\begin{mnemonicbox}
``JVM Java Virtual Machine''

\end{mnemonicbox}
\begin{center}\rule{0.5\linewidth}{0.5pt}\end{center}

\subsection*{Question 3(b OR) [4
marks]}\label{question-3b-or-4-marks}

\textbf{Define: Package. Write the steps to create a Package with
suitable example.}

\begin{solutionbox}

\textbf{Package}: Collection of related classes and interfaces grouped
together, providing namespace and access control.


{\def\LTcaptype{none} % do not increment counter
\vspace{-5pt}
\captionof{table}{Package Benefits}
\vspace{-10pt}
\begin{longtable}[]{@{}ll@{}}
\toprule\noalign{}
Benefit & Description \\
\midrule\noalign{}
\endhead
\bottomrule\noalign{}
\endlastfoot
\textbf{Namespace} & Avoid name conflicts \\
\textbf{Access Control} & Better encapsulation \\
\textbf{Organization} & Logical grouping \\
\textbf{Reusability} & Easy to maintain \\
\end{longtable}
}

\textbf{Steps to create Package:}

\begin{enumerate}
\tightlist
\item
  \textbf{Declare package} at top of file
\item
  \textbf{Create directory} structure matching package name
\item
  \textbf{Compile} with package structure
\item
  \textbf{Import} in other classes
\end{enumerate}

\textbf{Code Block:}

\begin{lstlisting}[language=Java]
// File: com/company/utilities/Calculator.java
package com.company.utilities;

public class Calculator {
    public int add(int a, int b) {
        return a + b;
    }
}

// File: Main.java
import com.company.utilities.Calculator;

class Main {
    public static void main(String[] args) {
        Calculator calc = new Calculator();
        System.out.println(calc.add(5, 3));
    }
}
\end{lstlisting}

\textbf{Directory Structure:}

\begin{lstlisting}
com/
  company/
    utilities/
      Calculator.class
Main.class
\end{lstlisting}

\end{solutionbox}
\begin{mnemonicbox}
``Package Groups Classes''

\end{mnemonicbox}
\begin{center}\rule{0.5\linewidth}{0.5pt}\end{center}

\subsection*{Question 3(c OR) [7
marks]}\label{question-3c-or-7-marks}

\textbf{Explain Synchronization in Thread with suitable example.}

\begin{solutionbox}

\textbf{Synchronization}: Mechanism to control access to shared
resources by multiple threads to avoid data inconsistency.


{\def\LTcaptype{none} % do not increment counter
\vspace{-5pt}
\captionof{table}{Synchronization Types}
\vspace{-10pt}
\begin{longtable}[]{@{}
  >{\raggedright\arraybackslash}p{(\linewidth - 4\tabcolsep) * \real{0.2222}}
  >{\raggedright\arraybackslash}p{(\linewidth - 4\tabcolsep) * \real{0.4815}}
  >{\raggedright\arraybackslash}p{(\linewidth - 4\tabcolsep) * \real{0.2963}}@{}}
\toprule\noalign{}
\begin{minipage}[b]{\linewidth}\raggedright
Type
\end{minipage} & \begin{minipage}[b]{\linewidth}\raggedright
Description
\end{minipage} & \begin{minipage}[b]{\linewidth}\raggedright
Usage
\end{minipage} \\
\midrule\noalign{}
\endhead
\bottomrule\noalign{}
\endlastfoot
\textbf{Synchronized method} & Entire method locked & synchronized void
method() \\
\textbf{Synchronized block} & Specific code block locked &
synchronized(object) \{ \} \\
\textbf{Static synchronization} & Class level locking & synchronized
static void method() \\
\end{longtable}
}

\textbf{Diagram: Without vs With Synchronization}

\includegraphics[width=1\linewidth,height=\textheight,keepaspectratio]{mermaid-6c035f46.pdf}

\textbf{Code Block:}

\begin{lstlisting}[language=Java]
class Counter {
    private int count = 0;
    
    // Synchronized method
    public synchronized void increment() {
        count++;
    }
    
    // Synchronized block
    public void decrement() {
        synchronized(this) {
            count--;
        }
    }
    
    public int getCount() {
        return count;
    }
}

class CounterThread extends Thread {
    Counter counter;
    
    CounterThread(Counter c) {
        counter = c;
    }
    
    public void run() {
        for(int i = 0; i < 1000; i++) {
            counter.increment();
        }
    }
}

class Main {
    public static void main(String[] args) throws InterruptedException {
        Counter c = new Counter();
        CounterThread t1 = new CounterThread(c);
        CounterThread t2 = new CounterThread(c);
        
        t1.start();
        t2.start();
        
        t1.join();
        t2.join();
        
        System.out.println("Final count: " + c.getCount());
    }
}
\end{lstlisting}

\end{solutionbox}
\begin{mnemonicbox}
``Synchronization Prevents Race Conditions''

\end{mnemonicbox}
\begin{center}\rule{0.5\linewidth}{0.5pt}\end{center}

\subsection*{Question 4(a) [3 marks]}\label{q4a}

\textbf{Differentiate between String class and StringBuffer class.}

\begin{solutionbox}


{\def\LTcaptype{none} % do not increment counter
\vspace{-5pt}
\captionof{table}{String vs StringBuffer}
\vspace{-10pt}
\begin{longtable}[]{@{}
  >{\raggedright\arraybackslash}p{(\linewidth - 4\tabcolsep) * \real{0.2667}}
  >{\raggedright\arraybackslash}p{(\linewidth - 4\tabcolsep) * \real{0.2667}}
  >{\raggedright\arraybackslash}p{(\linewidth - 4\tabcolsep) * \real{0.4667}}@{}}
\toprule\noalign{}
\begin{minipage}[b]{\linewidth}\raggedright
Aspect
\end{minipage} & \begin{minipage}[b]{\linewidth}\raggedright
String
\end{minipage} & \begin{minipage}[b]{\linewidth}\raggedright
StringBuffer
\end{minipage} \\
\midrule\noalign{}
\endhead
\bottomrule\noalign{}
\endlastfoot
\textbf{Mutability} & Immutable (cannot change) & Mutable (can
change) \\
\textbf{Performance} & Slower for concatenation & Faster for
concatenation \\
\textbf{Memory} & Creates new object each time & Modifies existing
object \\
\textbf{Thread Safety} & Thread safe & Thread safe \\
\textbf{Methods} & concat(), substring() & append(), insert(),
delete() \\
\end{longtable}
}

\textbf{Code Block:}

\begin{lstlisting}[language=Java]
// String - Immutable
String s1 = "Hello";
s1 = s1 + " World";  // Creates new String object

// StringBuffer - Mutable
StringBuffer sb = new StringBuffer("Hello");
sb.append(" World");  // Modifies existing object
\end{lstlisting}

\begin{itemize}
\tightlist
\item
  \textbf{String}: Use when content doesn't change frequently
\item
  \textbf{StringBuffer}: Use when frequent modifications needed
\end{itemize}

\end{solutionbox}
\begin{mnemonicbox}
``String Immutable, StringBuffer Mutable''

\end{mnemonicbox}
\begin{center}\rule{0.5\linewidth}{0.5pt}\end{center}

\subsection*{Question 4(b) [4 marks]}\label{q4b}

\textbf{Write a Java Program to find sum and average of 10 numbers of an
array.}

\begin{solutionbox}

\textbf{Code Block:}

\begin{lstlisting}[language=Java]
class ArraySum {
    public static void main(String[] args) {
        // Initialize array with 10 numbers
        int[] numbers = {10, 20, 30, 40, 50, 60, 70, 80, 90, 100};
        
        int sum = 0;
        
        // Calculate sum
        for(int i = 0; i < numbers.length; i++) {
            sum += numbers[i];
        }
        
        // Calculate average
        double average = (double) sum / numbers.length;
        
        // Display results
        System.out.println("Array elements: ");
        for(int num : numbers) {
            System.out.print(num + " ");
        }
        
        System.out.println("\nSum: " + sum);
        System.out.println("Average: " + average);
    }
}
\end{lstlisting}

\textbf{Output:}

\begin{lstlisting}
Array elements: 10 20 30 40 50 60 70 80 90 100
Sum: 550
Average: 55.0
\end{lstlisting}

\textbf{Logic Steps:}

\begin{enumerate}
\tightlist
\item
  \textbf{Initialize} array with 10 numbers
\item
  \textbf{Loop} through array to calculate sum
\item
  \textbf{Calculate} average = sum / length
\item
  \textbf{Display} results
\end{enumerate}

\end{solutionbox}
\begin{mnemonicbox}
``Loop Sum Divide Average''

\end{mnemonicbox}
\begin{center}\rule{0.5\linewidth}{0.5pt}\end{center}

\subsection*{Question 4(c) [7 marks]}\label{q4c}

\textbf{I) Explain abstract class with suitable example. II) Explain
final class with suitable example.}

\begin{solutionbox}

\textbf{I) Abstract Class}: Class that cannot be instantiated, contains
abstract methods that must be implemented by subclasses.


{\def\LTcaptype{none} % do not increment counter
\vspace{-5pt}
\captionof{table}{Abstract Class Features}
\vspace{-10pt}
\begin{longtable}[]{@{}ll@{}}
\toprule\noalign{}
Feature & Description \\
\midrule\noalign{}
\endhead
\bottomrule\noalign{}
\endlastfoot
\textbf{Cannot instantiate} & No object creation \\
\textbf{Abstract methods} & Methods without implementation \\
\textbf{Concrete methods} & Methods with implementation \\
\textbf{Inheritance} & Subclasses must implement abstract methods \\
\end{longtable}
}

\textbf{Code Block - Abstract Class:}

\begin{lstlisting}[language=Java]
abstract class Shape {
    String color;
    
    // Abstract method
    abstract void draw();
    
    // Concrete method
    void setColor(String c) {
        color = c;
    }
}

class Circle extends Shape {
    void draw() {
        System.out.println("Drawing Circle");
    }
}

class Main {
    public static void main(String[] args) {
        // Shape s = new Shape(); // Error: Cannot instantiate
        Circle c = new Circle();
        c.draw();
    }
}
\end{lstlisting}

\textbf{II) Final Class}: Class that cannot be extended (no inheritance
allowed).


{\def\LTcaptype{none} % do not increment counter
\vspace{-5pt}
\captionof{table}{Final Class Features}
\vspace{-10pt}
\begin{longtable}[]{@{}ll@{}}
\toprule\noalign{}
Feature & Description \\
\midrule\noalign{}
\endhead
\bottomrule\noalign{}
\endlastfoot
\textbf{No inheritance} & Cannot be extended \\
\textbf{Security} & Prevents modification \\
\textbf{Performance} & Better optimization \\
\textbf{Examples} & String, Integer, System \\
\end{longtable}
}

\textbf{Code Block - Final Class:}

\begin{lstlisting}[language=Java]
final class FinalClass {
    void display() {
        System.out.println("This is final class");
    }
}

// class SubClass extends FinalClass { } // Error: Cannot extend

class Main {
    public static void main(String[] args) {
        FinalClass obj = new FinalClass();
        obj.display();
    }
}
\end{lstlisting}

\end{solutionbox}
\begin{mnemonicbox}
``Abstract Incomplete, Final Complete''

\end{mnemonicbox}
\begin{center}\rule{0.5\linewidth}{0.5pt}\end{center}

\subsection*{Question 4(a OR) [3
marks]}\label{question-4a-or-3-marks}

\textbf{Explain Garbage Collection in Java.}

\begin{solutionbox}

\textbf{Garbage Collection}: Automatic memory management process that
removes unused objects from heap memory.


{\def\LTcaptype{none} % do not increment counter
\vspace{-5pt}
\captionof{table}{GC Benefits}
\vspace{-10pt}
\begin{longtable}[]{@{}ll@{}}
\toprule\noalign{}
Benefit & Description \\
\midrule\noalign{}
\endhead
\bottomrule\noalign{}
\endlastfoot
\textbf{Automatic} & No manual memory management \\
\textbf{Memory leak prevention} & Removes unreferenced objects \\
\textbf{Performance} & Optimizes memory usage \\
\textbf{Safety} & Prevents memory errors \\
\end{longtable}
}

\textbf{Diagram:}

\begin{lstlisting}
+------------------+
| Object created   |
| (new keyword)    |
+------------------+
        |
        v
+------------------+
| Object in use    |
| (has references) |
+------------------+
        |
        v
+------------------+
| No references    |
| (eligible for GC)|
+------------------+
        |
        v
+------------------+
| Garbage Collector|
| removes object   |
+------------------+
\end{lstlisting}

\begin{itemize}
\tightlist
\item
  \textbf{When occurs}: When heap memory is low or System.gc() called
\item
  \textbf{Process}: Mark and Sweep algorithm
\item
  \textbf{Cannot guarantee}: Exact timing of garbage collection
\end{itemize}

\end{solutionbox}
\begin{mnemonicbox}
``GC Automatic Memory Cleanup''

\end{mnemonicbox}
\begin{center}\rule{0.5\linewidth}{0.5pt}\end{center}

\subsection*{Question 4(b OR) [4
marks]}\label{question-4b-or-4-marks}

\textbf{Write a Java program to handle user defined exception for
`Divide by Zero' error.}

\begin{solutionbox}

\textbf{Code Block:}

\begin{lstlisting}[language=Java]
// User defined exception class
class DivideByZeroException extends Exception {
    public DivideByZeroException(String message) {
        super(message);
    }
}

class Calculator {
    public static double divide(int a, int b) throws DivideByZeroException {
        if(b == 0) {
            throw new DivideByZeroException("Cannot divide by zero!");
        }
        return (double) a / b;
    }
}

class Main {
    public static void main(String[] args) {
        try {
            int num1 = 10;
            int num2 = 0;
            
            double result = Calculator.divide(num1, num2);
            System.out.println("Result: " + result);
            
        } catch(DivideByZeroException e) {
            System.out.println("Error: " + e.getMessage());
        }
    }
}
\end{lstlisting}

\textbf{Output:}

\begin{lstlisting}
Error: Cannot divide by zero!
\end{lstlisting}

\textbf{Steps:}

\begin{enumerate}
\tightlist
\item
  \textbf{Create} custom exception class extending Exception
\item
  \textbf{Throw} exception when condition occurs
\item
  \textbf{Handle} exception with try-catch block
\end{enumerate}

\end{solutionbox}
\begin{mnemonicbox}
``Custom Exception Handle Error''

\end{mnemonicbox}
\begin{center}\rule{0.5\linewidth}{0.5pt}\end{center}

\subsection*{Question 4(c OR) [7
marks]}\label{question-4c-or-7-marks}

\textbf{Write a java program to demonstrate multiple try block and
multiple catch block exception.}

\begin{solutionbox}

\textbf{Code Block:}

\begin{lstlisting}[language=Java]
class MultipleExceptionDemo {
    public static void main(String[] args) {
        // First try block
        try {
            int[] arr = {1, 2, 3};
            System.out.println("Array element: " + arr[5]); // ArrayIndexOutOfBounds
        } 
        catch(ArrayIndexOutOfBoundsException e) {
            System.out.println("Array index error: " + e.getMessage());
        }
        catch(Exception e) {
            System.out.println("General exception: " + e.getMessage());
        }
        
        // Second try block
        try {
            String str = null;
            System.out.println("String length: " + str.length()); // NullPointer
        }
        catch(NullPointerException e) {
            System.out.println("Null pointer error: " + e.getMessage());
        }
        
        // Third try block with multiple catch
        try {
            int a = 10;
            int b = 0;
            int result = a / b;  // ArithmeticException
            
            String s = "abc";
            int num = Integer.parseInt(s);  // NumberFormatException
        }
        catch(ArithmeticException e) {
            System.out.println("Arithmetic error: " + e.getMessage());
        }
        catch(NumberFormatException e) {
            System.out.println("Number format error: " + e.getMessage());
        }
        catch(Exception e) {
            System.out.println("Other error: " + e.getMessage());
        }
        finally {
            System.out.println("Program completed");
        }
    }
}
\end{lstlisting}

\textbf{Output:}

\begin{lstlisting}
Array index error: Index 5 out of bounds for length 3
Null pointer error: null
Arithmetic error: / by zero
Program completed
\end{lstlisting}

\textbf{Features demonstrated:}

\begin{itemize}
\tightlist
\item
  \textbf{Multiple try blocks}: Each handles different operations
\item
  \textbf{Multiple catch blocks}: Each handles specific exception type
\item
  \textbf{Exception hierarchy}: General Exception catches all
\item
  \textbf{Finally block}: Always executes
\end{itemize}

\end{solutionbox}
\begin{mnemonicbox}
``Multiple Try Multiple Catch''

\end{mnemonicbox}
\begin{center}\rule{0.5\linewidth}{0.5pt}\end{center}

\subsection*{Question 5(a) [3 marks]}\label{q5a}

\textbf{Write a program in Java to create a file and perform write
operation on this file.}

\begin{solutionbox}

\textbf{Code Block:}

\begin{lstlisting}[language=Java]
import java.io.*;

class FileWriteDemo {
    public static void main(String[] args) {
        try {
            // Create file
            File file = new File("demo.txt");
            
            // Create FileWriter object
            FileWriter writer = new FileWriter(file);
            
            // Write data to file
            writer.write("Hello World!\n");
            writer.write("This is Java file writing demo.\n");
            writer.write("File created successfully.");
            
            // Close the writer
            writer.close();
            
            System.out.println("File created and data written successfully!");
            
        } catch(IOException e) {
            System.out.println("Error: " + e.getMessage());
        }
    }
}
\end{lstlisting}

\textbf{Steps:}

\begin{enumerate}
\tightlist
\item
  \textbf{Import} java.io package
\item
  \textbf{Create} File object with filename
\item
  \textbf{Create} FileWriter object
\item
  \textbf{Write} data using write() method
\item
  \textbf{Close} writer to save changes
\end{enumerate}

\end{solutionbox}
\begin{mnemonicbox}
``File Writer Write Close''

\end{mnemonicbox}
\begin{center}\rule{0.5\linewidth}{0.5pt}\end{center}

\subsection*{Question 5(b) [4 marks]}\label{q5b}

\textbf{Explain throw and finally in Exception Handling with example.}

\begin{solutionbox}

\textbf{Throw}: Keyword used to explicitly throw an exception.
\textbf{Finally}: Block that always executes regardless of exception
occurrence.


{\def\LTcaptype{none} % do not increment counter
\vspace{-5pt}
\captionof{table}{Throw vs Finally}
\vspace{-10pt}
\begin{longtable}[]{@{}lll@{}}
\toprule\noalign{}
Keyword & Purpose & Usage \\
\midrule\noalign{}
\endhead
\bottomrule\noalign{}
\endlastfoot
\textbf{throw} & Explicitly throw exception & throw new Exception() \\
\textbf{finally} & Always execute cleanup code & finally \{ \} \\
\end{longtable}
}

\textbf{Code Block:}

\begin{lstlisting}[language=Java]
class ThrowFinallyDemo {
    public static void checkAge(int age) throws Exception {
        if(age < 18) {
            throw new Exception("Age must be 18 or above");
        }
        System.out.println("Valid age: " + age);
    }
    
    public static void main(String[] args) {
        try {
            checkAge(15);  // Will throw exception
        }
        catch(Exception e) {
            System.out.println("Error: " + e.getMessage());
        }
        finally {
            System.out.println("Finally block always executes");
        }
    }
}
\end{lstlisting}

\textbf{Output:}

\begin{lstlisting}
Error: Age must be 18 or above
Finally block always executes
\end{lstlisting}

\begin{itemize}
\tightlist
\item
  \textbf{Throw}: Forces exception to occur
\item
  \textbf{Finally}: Cleanup code, closes resources
\end{itemize}

\end{solutionbox}
\begin{mnemonicbox}
``Throw Exception, Finally Always''

\end{mnemonicbox}
\begin{center}\rule{0.5\linewidth}{0.5pt}\end{center}

\subsection*{Question 5(c) [7 marks]}\label{q5c}

\textbf{Describe: Polymorphism. Explain run time polymorphism with
suitable example in java.}

\begin{solutionbox}

\textbf{Polymorphism}: One interface, multiple implementations. Object
behaves differently based on its actual type.


{\def\LTcaptype{none} % do not increment counter
\vspace{-5pt}
\captionof{table}{Polymorphism Types}
\vspace{-10pt}
\begin{longtable}[]{@{}lll@{}}
\toprule\noalign{}
Type & Description & When Decided \\
\midrule\noalign{}
\endhead
\bottomrule\noalign{}
\endlastfoot
\textbf{Compile-time} & Method overloading & At compilation \\
\textbf{Run-time} & Method overriding & At execution \\
\end{longtable}
}

\textbf{Run-time Polymorphism}: Method call resolved at runtime based on
actual object type.

\textbf{Diagram:}

\includegraphics[width=1\linewidth,height=\textheight,keepaspectratio]{mermaid-18b559a3.pdf}

\textbf{Code Block:}

\begin{lstlisting}[language=Java]
class Animal {
    void makeSound() {
        System.out.println("Animal makes sound");
    }
}

class Dog extends Animal {
    @Override
    void makeSound() {
        System.out.println("Dog barks");
    }
}

class Cat extends Animal {
    @Override
    void makeSound() {
        System.out.println("Cat meows");
    }
}

class Main {
    public static void main(String[] args) {
        Animal animal1 = new Dog();  // Upcasting
        Animal animal2 = new Cat();  // Upcasting
        
        animal1.makeSound();  // Output: Dog barks
        animal2.makeSound();  // Output: Cat meows
        
        // Array of animals
        Animal[] animals = {new Dog(), new Cat(), new Dog()};
        for(Animal a : animals) {
            a.makeSound();  // Dynamic method dispatch
        }
    }
}
\end{lstlisting}

\textbf{Output:}

\begin{lstlisting}
Dog barks
Cat meows
Dog barks
Cat meows
Dog barks
\end{lstlisting}

\textbf{Features:}

\begin{itemize}
\tightlist
\item
  \textbf{Dynamic Method Dispatch}: JVM decides which method to call at
  runtime
\item
  \textbf{Upcasting}: Child object referenced by parent reference
\item
  \textbf{Flexibility}: Same code works with different object types
\end{itemize}

\end{solutionbox}
\begin{mnemonicbox}
``Polymorphism Many Forms Runtime''

\end{mnemonicbox}
\begin{center}\rule{0.5\linewidth}{0.5pt}\end{center}

\subsection*{Question 5(a OR) [3
marks]}\label{question-5a-or-3-marks}

\textbf{Write a program in Java that read the content of a file byte by
byte and copy it into another file.}

\begin{solutionbox}

\textbf{Code Block:}

\begin{lstlisting}[language=Java]
import java.io.*;

class FileCopyDemo {
    public static void main(String[] args) {
        try {
            // Create input stream to read from source file
            FileInputStream input = new FileInputStream("source.txt");
            
            // Create output stream to write to destination file
            FileOutputStream output = new FileOutputStream("destination.txt");
            
            int byteData;
            
            // Read byte by byte and copy
            while((byteData = input.read()) != -1) {
                output.write(byteData);
            }
            
            // Close streams
            input.close();
            output.close();
            
            System.out.println("File copied successfully!");
            
        } catch(IOException e) {
            System.out.println("Error: " + e.getMessage());
        }
    }
}
\end{lstlisting}

\textbf{Steps:}

\begin{enumerate}
\tightlist
\item
  \textbf{Create} FileInputStream for reading
\item
  \textbf{Create} FileOutputStream for writing
\item
  \textbf{Read} byte by byte using read()
\item
  \textbf{Write} each byte using write()
\item
  \textbf{Close} both streams
\end{enumerate}

\end{solutionbox}
\begin{mnemonicbox}
``Read Byte Write Byte''

\end{mnemonicbox}
\begin{center}\rule{0.5\linewidth}{0.5pt}\end{center}

\subsection*{Question 5(b OR) [4
marks]}\label{question-5b-or-4-marks}

\textbf{Explain the different I/O Classes available with Java.}

\begin{solutionbox}


{\def\LTcaptype{none} % do not increment counter
\vspace{-5pt}
\captionof{table}{Java I/O Classes}
\vspace{-10pt}
\begin{longtable}[]{@{}lll@{}}
\toprule\noalign{}
Class Type & Class Name & Purpose \\
\midrule\noalign{}
\endhead
\bottomrule\noalign{}
\endlastfoot
\textbf{Byte Stream} & FileInputStream & Read bytes from file \\
\textbf{Byte Stream} & FileOutputStream & Write bytes to file \\
\textbf{Character Stream} & FileReader & Read characters from file \\
\textbf{Character Stream} & FileWriter & Write characters to file \\
\textbf{Buffered} & BufferedReader & Efficient character reading \\
\textbf{Buffered} & BufferedWriter & Efficient character writing \\
\end{longtable}
}

\textbf{Diagram: I/O Class Hierarchy}

\begin{lstlisting}
+------------------+
|    InputStream   |
+------------------+
        |
        +-- FileInputStream
        +-- BufferedInputStream
        
+------------------+
|   OutputStream   |
+------------------+
        |
        +-- FileOutputStream
        +-- BufferedOutputStream
        
+------------------+
|      Reader      |
+------------------+
        |
        +-- FileReader
        +-- BufferedReader
        
+------------------+
|      Writer      |
+------------------+
        |
        +-- FileWriter
        +-- BufferedWriter
\end{lstlisting}

\textbf{Code Example:}

\begin{lstlisting}[language=Java]
// Byte streams
FileInputStream fis = new FileInputStream("file.txt");
FileOutputStream fos = new FileOutputStream("output.txt");

// Character streams
FileReader fr = new FileReader("file.txt");
FileWriter fw = new FileWriter("output.txt");

// Buffered streams
BufferedReader br = new BufferedReader(new FileReader("file.txt"));
BufferedWriter bw = new BufferedWriter(new FileWriter("output.txt"));
\end{lstlisting}

\end{solutionbox}
\begin{mnemonicbox}
``Byte Character Buffered Streams''

\end{mnemonicbox}
\begin{center}\rule{0.5\linewidth}{0.5pt}\end{center}

\subsection*{Question 5(c OR) [7
marks]}\label{question-5c-or-7-marks}

\textbf{Write a java program that executes two threads. One thread
displays ``Java Programming'' every 3 seconds, and the other displays
``Semester - 4th IT'' every 6 seconds.(Create the threads by extending
the Thread class)}

\begin{solutionbox}

\textbf{Code Block:}

\begin{lstlisting}[language=Java]
class JavaThread extends Thread {
    public void run() {
        try {
            while(true) {
                System.out.println("Java Programming");
                Thread.sleep(3000);  // Sleep for 3 seconds
            }
        } catch(InterruptedException e) {
            System.out.println("JavaThread interrupted");
        }
    }
}

class SemesterThread extends Thread {
    public void run() {
        try {
            while(true) {
                System.out.println("Semester - 4th IT");
                Thread.sleep(6000);  // Sleep for 6 seconds
            }
        } catch(InterruptedException e) {
            System.out.println("SemesterThread interrupted");
        }
    }
}

class Main {
    public static void main(String[] args) {
        // Create thread objects
        JavaThread javaThread = new JavaThread();
        SemesterThread semesterThread = new SemesterThread();
        
        // Start both threads
        javaThread.start();
        semesterThread.start();
        
        // Let threads run for 20 seconds then stop
        try {
            Thread.sleep(20000);
            javaThread.interrupt();
            semesterThread.interrupt();
        } catch(InterruptedException e) {
            System.out.println("Main thread interrupted");
        }
    }
}
\end{lstlisting}

\textbf{Sample Output:}

\begin{lstlisting}
Java Programming
Semester - 4th IT
Java Programming
Java Programming
Semester - 4th IT
Java Programming
Java Programming
Semester - 4th IT
...
\end{lstlisting}

\textbf{Features:}

\begin{itemize}
\tightlist
\item
  \textbf{Two separate threads}: Each with different timing
\item
  \textbf{Thread.sleep()}: Pauses execution for specified milliseconds
\item
  \textbf{Concurrent execution}: Both threads run simultaneously
\item
  \textbf{Extending Thread class}: Override run() method
\end{itemize}

\textbf{Execution Pattern:}

\begin{itemize}
\tightlist
\item
  JavaThread: Displays every 3 seconds
\item
  SemesterThread: Displays every 6 seconds
\item
  Both run concurrently showing different timing
\end{itemize}

\end{solutionbox}
\begin{mnemonicbox}
``Two Threads Different Timing''

\end{mnemonicbox}

\end{document}
