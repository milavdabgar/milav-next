\documentclass{article}

% content/resources/templates/preamble.tex
\usepackage[margin=0.6in]{geometry}
\author{Milav Dabgar}
\usepackage{amsmath,amssymb,amsthm}
\usepackage{booktabs}
\usepackage{multirow}
\usepackage{xcolor}
\usepackage{tcolorbox}
\tcbuselibrary{breakable,skins}
\usepackage[colorlinks=true,linkcolor=blue]{hyperref}
\usepackage{titlesec}
\usepackage{enumitem}
\usepackage{tikz}
\usepackage{pgfplots}
\usepackage{circuitikz}
\usepackage[version=4]{mhchem}
\usepackage{longtable}
\usepackage{array}
\usepackage{float}
\usepackage{caption}
\usepackage{listings}

\lstset{
  basicstyle=\small\ttfamily,
  breaklines=true,
  breakatwhitespace=false,
  postbreak=\mbox{\textcolor{red}{$\hookrightarrow$}\space},
  float=false,
  numbers=left,
  numberstyle=\tiny\color{gray},
  numbersep=10pt,
  xleftmargin=2em,
  keywordstyle=\color{blue},
  commentstyle=\color{green!60!black},
  stringstyle=\color{purple},
  backgroundcolor=\color{gray!5},
  showstringspaces=false,
  tabsize=2,
  captionpos=b,
  keepspaces=true,
  columns=flexible
}

\pgfplotsset{compat=1.18}
\usetikzlibrary{shapes,arrows,positioning,calc,patterns,decorations.pathmorphing,decorations.markings,arrows.meta}

% Color scheme
\definecolor{headcolor}{RGB}{0,102,204}
\definecolor{keycolor}{RGB}{220,20,60}
\definecolor{solutioncolor}{RGB}{34,139,34}
\definecolor{mnemoniccolor}{RGB}{148,0,211}
\definecolor{codecolor}{RGB}{0,0,100}

% Spacing
\setlength{\parskip}{3pt}
\setlist[itemize]{nosep}
\setlist[enumerate]{nosep}

% Title formatting
\titleformat{\section}{\Large\bfseries\color{headcolor}}{\thesection}{1em}{}
\titleformat{\subsection}{\large\bfseries\color{headcolor}}{\thesubsection}{1em}{}

% Pandoc tightlist compatibility
\providecommand{\tightlist}{%
  \setlength{\itemsep}{0pt}\setlength{\parskip}{0pt}}

% Pandoc longtable compatibility
\newcounter{none}
\def\thenone{}


% content/resources/templates/english-boxes.tex

% Custom environments
\newtcolorbox{solutionbox}{
 breakable,
 enhanced,
 colback=solutioncolor!5!white,
 colframe=solutioncolor!75!black,
 fonttitle=\bfseries,
 title=Solution
}

\newtcolorbox{solutionboxnobreak}{
 colback=solutioncolor!5!white,
 colframe=solutioncolor!75!black,
 fonttitle=\bfseries,
 title=Solution
}

\newtcolorbox{keyformula}{
 breakable,
 enhanced,
 colback=keycolor!5!white,
 colframe=keycolor!75!black,
 fonttitle=\bfseries,
 title=Key Formula
}

\newtcolorbox{mnemonicboxenv}{
 breakable,
 enhanced,
 colback=mnemoniccolor!5!white,
 colframe=mnemoniccolor!75!black,
 fonttitle=\bfseries,
 title=Mnemonic
}

\newcommand{\mnemonicbox}[1]{%
  \begin{mnemonicboxenv}
    #1
  \end{mnemonicboxenv}
}


% Custom commands for GTU solutions
% This file defines semantic commands for consistent formatting

% Question command with automatic formatting
\newcommand{\question}[2]{%
  \section*{Question #1}%
  \textbf{#2}%
}

% OR question variant
\newcommand{\questionor}[2]{%
  \section*{Question #1 OR}%
  \textbf{#2}%
}

% Proper table environment with caption
\newenvironment{answertable}[1]{%
  \begin{table}[htbp]
  \centering
  \caption{#1}
}{%
  \end{table}
}

% Proper figure environment for diagrams
\newenvironment{answerdiagram}[1]{%
  \begin{figure}[htbp]
  \centering
  \caption{#1}
}{%
  \end{figure}
}

% Semantic markup for key terms
\newcommand{\keyword}[1]{\textbf{#1}}
\newcommand{\code}[1]{\texttt{#1}}
\newcommand{\classname}[1]{\texttt{#1}}
\newcommand{\methodname}[1]{\texttt{#1}}

% Proper quotation marks
\newcommand{\mnemonic}[1]{``#1''}


\title{Object Oriented Programming with JAVA (4341602) - Winter 2024 Solution}
\date{November 26, 2024}

\begin{document}
\maketitle

\questionmarks{1(a)}{3}{Write down the difference between oop and pop.}

\begin{solutionbox}
\begin{center}
\captionof{table}{OOP vs POP}
\begin{tabulary}{\linewidth}{|L|L|L|}
\hline
\textbf{Aspect} & \textbf{OOP} & \textbf{POP} \\ \hline
\textbf{Approach} & Bottom-up approach & Top-down approach \\ \hline
\textbf{Focus} & Objects and classes & Functions and procedures \\ \hline
\textbf{Data Security} & Data hiding through encapsulation & No data hiding \\ \hline
\textbf{Problem Solving} & Divide problem into objects & Divide problem into functions \\ \hline
\end{tabulary}
\end{center}
\end{solutionbox}

\begin{mnemonicbox}
\mnemonic{Objects Bottom, Procedures Top}
\end{mnemonicbox}

\questionmarks{1(b)}{4}{What is byte code? Explain JVM in detail.}

\begin{solutionbox}
\textbf{Byte Code}: Platform-independent intermediate code generated by Java compiler from source code.

\begin{center}
\begin{tikzpicture}[auto, node distance=2cm]
    \node [gtu block] (source) {Java Source Code};
    \node [gtu block, right=of source] (compiler) {Java Compiler javac};
    \node [gtu block, below=of compiler] (bytecode) {Byte Code .class};
    \node [gtu block, left=of bytecode] (jvm) {JVM};
    \node [gtu block, left=of jvm] (machine) {Machine Code};

    \path [gtu arrow] (source) -- (compiler);
    \path [gtu arrow] (compiler) -- (bytecode);
    \path [gtu arrow] (bytecode) -- (jvm);
    \path [gtu arrow] (jvm) -- (machine);
\end{tikzpicture}
\end{center}

\textbf{JVM Components}:
\begin{itemize}
    \item \textbf{Class Loader}: Loads .class files into memory
    \item \textbf{Memory Area}: Heap, stack, method area storage
    \item \textbf{Execution Engine}: Interprets and executes bytecode
    \item \textbf{Garbage Collector}: Automatic memory management
\end{itemize}
\end{solutionbox}

\begin{mnemonicbox}
\mnemonic{Byte Code Runs Everywhere}
\end{mnemonicbox}

\questionmarks{1(c)}{7}{Write a program in Java to sort the elements of an array in ascending order}

\begin{solutionbox}
\begin{lstlisting}[language=Java,caption={Array Sort}]
import java.util.Arrays;

public class ArraySort {
    public static void main(String[] args) {
        int[] arr = {64, 34, 25, 12, 22, 11, 90};
        
        // Bubble Sort
        for(int i = 0; i < arr.length-1; i++) {
            for(int j = 0; j < arr.length-i-1; j++) {
                if(arr[j] > arr[j+1]) {
                    int temp = arr[j];
                    arr[j] = arr[j+1];
                    arr[j+1] = temp;
                }
            }
        }
        
        System.out.println("Sorted array: " + Arrays.toString(arr));
    }
}
\end{lstlisting}

\textbf{Key Points}:
\begin{itemize}
    \item \textbf{Bubble Sort}: Compares adjacent elements
    \item \textbf{Time Complexity}: O(n\textsuperscript{2})
    \item \textbf{Space Complexity}: O(1)
\end{itemize}
\end{solutionbox}

\begin{mnemonicbox}
\mnemonic{Bubble Up The Smallest}
\end{mnemonicbox}

\questionmarks{1(c OR)}{7}{Write a program in java to find out maximum from any ten numbers using command line argument.}

\begin{solutionbox}
\begin{lstlisting}[language=Java,caption={Find Maximum from Command Line}]
public class FindMaximum {
    public static void main(String[] args) {
        if(args.length != 10) {
            System.out.println("Please enter exactly 10 numbers");
            return;
        }
        
        int max = Integer.parseInt(args[0]);
        
        for(int i = 1; i < args.length; i++) {
            int num = Integer.parseInt(args[i]);
            if(num > max) {
                max = num;
            }
        }
        
        System.out.println("Maximum number: " + max);
    }
}
\end{lstlisting}

\textbf{Key Points}:
\begin{itemize}
    \item \textbf{Command Line}: \code{args[]} array stores arguments
    \item \textbf{parseInt()}: Converts string to integer
    \item \textbf{Validation}: Check array length
\end{itemize}
\end{solutionbox}

\begin{mnemonicbox}
\mnemonic{Arguments Maximum Search}
\end{mnemonicbox}

\questionmarks{2(a)}{3}{What is wrapper class? Explain with example.}

\begin{solutionbox}
\textbf{Wrapper Class}: Converts primitive data types into objects.

\begin{center}
\captionof{table}{Wrapper Classes}
\begin{tabulary}{\linewidth}{|L|L|}
\hline
\textbf{Primitive} & \textbf{Wrapper Class} \\ \hline
int & Integer \\ \hline
char & Character \\ \hline
boolean & Boolean \\ \hline
double & Double \\ \hline
\end{tabulary}
\end{center}

\begin{lstlisting}[language=Java,caption={Wrapper Class Example}]
// Boxing
Integer obj = Integer.valueOf(10);
// Unboxing  
int value = obj.intValue();
\end{lstlisting}
\end{solutionbox}

\begin{mnemonicbox}
\mnemonic{Wrap Primitives Into Objects}
\end{mnemonicbox}

\questionmarks{2(b)}{4}{List out different features of java. Explain any two.}

\begin{solutionbox}
\textbf{Java Features}:
\begin{itemize}
    \item \textbf{Simple}: Easy syntax, no pointers
    \item \textbf{Platform Independent}: Write once, run anywhere
    \item \textbf{Object Oriented}: Based on objects and classes
    \item \textbf{Secure}: No explicit pointers, bytecode verification
\end{itemize}

\textbf{Detailed Explanation}:
\begin{itemize}
    \item \textbf{Platform Independence}: Java bytecode runs on any platform with JVM
    \item \textbf{Object Oriented}: Supports inheritance, encapsulation, polymorphism, abstraction
\end{itemize}
\end{solutionbox}

\begin{mnemonicbox}
\mnemonic{Simple Platform Object Security}
\end{mnemonicbox}

\questionmarks{2(c)}{7}{What is method overriding? Explain with example.}

\begin{solutionbox}
\textbf{Method Overriding}: Child class provides specific implementation of parent class method.

\begin{lstlisting}[language=Java,caption={Method Overriding}]
class Animal {
    public void sound() {
        System.out.println("Animal makes sound");
    }
}

class Dog extends Animal {
    @Override
    public void sound() {
        System.out.println("Dog barks");
    }
}

public class Test {
    public static void main(String[] args) {
        Animal a = new Dog();
        a.sound(); // Output: Dog barks
    }
}
\end{lstlisting}

\textbf{Key Points}:
\begin{itemize}
    \item \textbf{Runtime Polymorphism}: Method called based on object type
    \item \textbf{@Override}: Annotation for method overriding
    \item \textbf{Dynamic Binding}: Method resolution at runtime
\end{itemize}
\end{solutionbox}

\begin{mnemonicbox}
\mnemonic{Child Changes Parent Method}
\end{mnemonicbox}

\questionmarks{2(a OR)}{3}{Explain Garbage collection in java.}

\begin{solutionbox}
\textbf{Garbage Collection}: Automatic memory management that removes unused objects.

\begin{center}
\begin{tikzpicture}[auto, node distance=1.5cm]
    \node [gtu block] (created) {Object Created};
    \node [gtu block, right=of created] (used) {Object Used};
    \node [gtu block, below=of used] (unref) {Object Unreferenced};
    \node [gtu block, left=of unref] (gc) {Garbage Collector};
    \node [gtu block, above=of gc] (freed) {Memory Freed};

    \path [gtu arrow] (created) -- (used);
    \path [gtu arrow] (used) -- (unref);
    \path [gtu arrow] (unref) -- (gc);
    \path [gtu arrow] (gc) -- (freed);
\end{tikzpicture}
\end{center}

\textbf{Key Points}:
\begin{itemize}
    \item \textbf{Automatic}: No manual memory deallocation
    \item \textbf{Mark and Sweep}: Identifies and removes unused objects
    \item \textbf{Heap Memory}: Works on heap memory area
\end{itemize}
\end{solutionbox}

\begin{mnemonicbox}
\mnemonic{Auto Clean Unused Objects}
\end{mnemonicbox}

\questionmarks{2(b OR)}{4}{Explain static keyword with example.}

\begin{solutionbox}
\textbf{Static Keyword}: Belongs to class rather than instance.

\begin{lstlisting}[language=Java,caption={Static Example}]
class Student {
    static String college = "GTU";  // Static variable
    String name;
    
    static void showCollege() {     // Static method
        System.out.println("College: " + college);
    }
}
\end{lstlisting}

\textbf{Static Features}:
\begin{itemize}
    \item \textbf{Memory}: Loaded at class loading time
    \item \textbf{Access}: Can be accessed without object
    \item \textbf{Sharing}: Shared among all instances
\end{itemize}
\end{solutionbox}

\begin{mnemonicbox}
\mnemonic{Class Level Memory Sharing}
\end{mnemonicbox}

\questionmarks{2(c OR)}{7}{What is constructor? Explain copy constructor with example.}

\begin{solutionbox}
\textbf{Constructor}: Special method to initialize objects.

\begin{lstlisting}[language=Java,caption={Constructor Types}]
class Person {
    String name;
    int age;
    
    // Default constructor
    Person() {
        name = "Unknown";
        age = 0;
    }
    
    // Parameterized constructor
    Person(String n, int a) {
        name = n;
        age = a;
    }
    
    // Copy constructor
    Person(Person p) {
        name = p.name;
        age = p.age;
    }
}
\end{lstlisting}

\textbf{Constructor Types}:
\begin{itemize}
    \item \textbf{Default}: No parameters
    \item \textbf{Parameterized}: Takes parameters
    \item \textbf{Copy}: Creates object from existing object
\end{itemize}
\end{solutionbox}

\begin{mnemonicbox}
\mnemonic{Default Parameter Copy}
\end{mnemonicbox}

\questionmarks{3(a)}{3}{Explain super keyword with example.}

\begin{solutionbox}
\textbf{Super Keyword}: References parent class members.

\begin{lstlisting}[language=Java,caption={Super Keyword}]
class Vehicle {
    String brand = "Generic";
}

class Car extends Vehicle {
    String brand = "Toyota";
    
    void display() {
        System.out.println("Child: " + brand);
        System.out.println("Parent: " + super.brand);
    }
}
\end{lstlisting}

\textbf{Super Uses}:
\begin{itemize}
    \item \textbf{Variables}: Access parent class variables
    \item \textbf{Methods}: Call parent class methods
    \item \textbf{Constructor}: Call parent class constructor
\end{itemize}
\end{solutionbox}

\begin{mnemonicbox}
\mnemonic{Super Calls Parent}
\end{mnemonicbox}

\questionmarks{3(b)}{4}{List out different types of inheritance. Explain multilevel inheritance.}

\begin{solutionbox}
\textbf{Inheritance Types}:
\begin{center}
\captionof{table}{Inheritance Types}
\begin{tabulary}{\linewidth}{|L|L|}
\hline
\textbf{Type} & \textbf{Description} \\ \hline
Single & One parent, one child \\ \hline
Multilevel & Chain of inheritance \\ \hline
Hierarchical & One parent, multiple children \\ \hline
Multiple & Multiple parents (via interfaces) \\ \hline
\end{tabulary}
\end{center}

\textbf{Multilevel Inheritance}:

\begin{lstlisting}[language=Java,caption={Multilevel Inheritance}]
class Animal {
    void eat() { System.out.println("Eating"); }
}

class Mammal extends Animal {
    void breathe() { System.out.println("Breathing"); }
}

class Dog extends Mammal {
    void bark() { System.out.println("Barking"); }
}
\end{lstlisting}
\end{solutionbox}

\begin{mnemonicbox}
\mnemonic{Single Multi Hierarchical Multiple}
\end{mnemonicbox}

\questionmarks{3(c)}{7}{What is interface? Explain multiple inheritance with example.}

\begin{solutionbox}
\textbf{Interface}: Contract that defines what class must do, not how.

\begin{lstlisting}[language=Java,caption={Multiple Inheritance with Interface}]
interface Flyable {
    void fly();
}

interface Swimmable {
    void swim();
}

class Duck implements Flyable, Swimmable {
    public void fly() {
        System.out.println("Duck is flying");
    }
    
    public void swim() {
        System.out.println("Duck is swimming");
    }
}
\end{lstlisting}

\textbf{Interface Features}:
\begin{itemize}
    \item \textbf{Multiple Inheritance}: Class can implement multiple interfaces
    \item \textbf{Abstract Methods}: All methods are abstract by default
    \item \textbf{Constants}: All variables are public, static, final
\end{itemize}
\end{solutionbox}

\begin{mnemonicbox}
\mnemonic{Multiple Abstract Constants}
\end{mnemonicbox}

\questionmarks{3(a OR)}{3}{Explain final keyword with example.}

\begin{solutionbox}
\textbf{Final Keyword}: Restricts modification, inheritance, or overriding.

\begin{lstlisting}[language=Java,caption={Final Keyword}]
final class Math {           // Cannot be inherited
    final int PI = 3.14;     // Cannot be modified
    
    final void calculate() { // Cannot be overridden
        System.out.println("Calculating");
    }
}
\end{lstlisting}

\textbf{Final Uses}:
\begin{itemize}
    \item \textbf{Class}: Cannot be extended
    \item \textbf{Method}: Cannot be overridden
    \item \textbf{Variable}: Cannot be reassigned
\end{itemize}
\end{solutionbox}

\begin{mnemonicbox}
\mnemonic{Final Stops Changes}
\end{mnemonicbox}

\questionmarks{3(b OR)}{4}{Explain different access controls in Java.}

\begin{solutionbox}
\textbf{Access Modifiers}:

\begin{center}
\captionof{table}{Access Modifiers}
\begin{tabulary}{\linewidth}{|L|C|C|C|C|}
\hline
\textbf{Modifier} & \textbf{Same Class} & \textbf{Same Package} & \textbf{Subclass} & \textbf{Diff Package} \\ \hline
public & \checkmark & \checkmark & \checkmark & \checkmark \\ \hline
protected & \checkmark & \checkmark & \checkmark & \ding{55} \\ \hline
default & \checkmark & \checkmark & \ding{55} & \ding{55} \\ \hline
private & \checkmark & \ding{55} & \ding{55} & \ding{55} \\ \hline
\end{tabulary}
\end{center}
\end{solutionbox}

\begin{mnemonicbox}
\mnemonic{Public Protected Default Private}
\end{mnemonicbox}

\questionmarks{3(c OR)}{7}{What is package? Write steps to create a package and give example of it.}

\begin{solutionbox}
\textbf{Package}: Group of related classes and interfaces.

\textbf{Steps to Create Package}:
\begin{enumerate}
    \item \textbf{Declare}: Use \code{package} statement at top
    \item \textbf{Compile}: \code{javac -d . ClassName.java}
    \item \textbf{Run}: \code{java packagename.ClassName}
\end{enumerate}

\begin{lstlisting}[language=Java,caption={Package Example}]
// File: mypack/Calculator.java
package mypack;

public class Calculator {
    public int add(int a, int b) {
        return a + b;
    }
}

// File: Test.java
import mypack.Calculator;

public class Test {
    public static void main(String[] args) {
        Calculator calc = new Calculator();
        System.out.println(calc.add(5, 3));
    }
}
\end{lstlisting}

\textbf{Package Benefits}:
\begin{itemize}
    \item \textbf{Organization}: Groups related classes
    \item \textbf{Access Control}: Package-level protection
    \item \textbf{Namespace}: Avoids naming conflicts
\end{itemize}
\end{solutionbox}

\begin{mnemonicbox}
\mnemonic{Declare Compile Run}
\end{mnemonicbox}

\questionmarks{4(a)}{3}{Explain thread priorities with suitable example.}

\begin{solutionbox}
\textbf{Thread Priority}: Determines thread execution order (1-10 scale).

\begin{lstlisting}[language=Java,caption={Thread Priority}]
class MyThread extends Thread {
    public void run() {
        System.out.println(getName() + " Priority: " + getPriority());
    }
}

public class ThreadPriorityExample {
    public static void main(String[] args) {
        MyThread t1 = new MyThread();
        MyThread t2 = new MyThread();
        
        t1.setPriority(Thread.MIN_PRIORITY);  // 1
        t2.setPriority(Thread.MAX_PRIORITY);  // 10
        
        t1.start();
        t2.start();
    }
}
\end{lstlisting}

\textbf{Priority Constants}:
\begin{itemize}
    \item \textbf{MIN\_PRIORITY}: 1
    \item \textbf{NORM\_PRIORITY}: 5  
    \item \textbf{MAX\_PRIORITY}: 10
\end{itemize}
\end{solutionbox}

\begin{mnemonicbox}
\mnemonic{Min Normal Max}
\end{mnemonicbox}

\questionmarks{4(b)}{4}{What is Thread? Explain Thread life cycle.}

\begin{solutionbox}
\textbf{Thread}: Lightweight process for concurrent execution.

\begin{center}
\begin{tikzpicture}[auto, node distance=2cm]
    \node [gtu state] (new) {New};
    \node [gtu state, right=of new] (runnable) {Runnable};
    \node [gtu state, right=of runnable] (running) {Running};
    \node [gtu state, right=of running] (dead) {Dead};
    \node [gtu state, below=of running] (blocked) {Blocked};

    \path [gtu arrow] (new) -- node {start()} (runnable);
    \path [gtu arrow] (runnable) -- node {CPU allocation} (running);
    \path [gtu arrow] (running) -- node {complete} (dead);
    \path [gtu arrow] (running) edge[bend right] node[left] {wait/sleep} (blocked);
    \path [gtu arrow] (blocked) edge[bend right] node[right] {notify/timeout} (runnable);
\end{tikzpicture}
\end{center}

\textbf{Thread States}:
\begin{itemize}
    \item \textbf{New}: Thread created but not started
    \item \textbf{Runnable}: Ready to run
    \item \textbf{Running}: Currently executing
    \item \textbf{Blocked}: Waiting for resource
    \item \textbf{Dead}: Execution completed
\end{itemize}
\end{solutionbox}

\begin{mnemonicbox}
\mnemonic{New Runnable Running Blocked Dead}
\end{mnemonicbox}

\questionmarks{4(c)}{7}{Write a program in java that create the multiple threads by implementing the Runnable interface.}

\begin{solutionbox}
\begin{lstlisting}[language=Java,caption={Multiple Threads}]
class MyRunnable implements Runnable {
    private String threadName;
    
    MyRunnable(String name) {
        threadName = name;
    }
    
    public void run() {
        for(int i = 1; i <= 5; i++) {
            System.out.println(threadName + " - Count: " + i);
            try {
                Thread.sleep(1000);
            } catch(InterruptedException e) {
                e.printStackTrace();
            }
        }
    }
}

public class MultipleThreads {
    public static void main(String[] args) {
        Thread t1 = new Thread(new MyRunnable("Thread-1"));
        Thread t2 = new Thread(new MyRunnable("Thread-2"));
        Thread t3 = new Thread(new MyRunnable("Thread-3"));
        
        t1.start();
        t2.start(); 
        t3.start();
    }
}
\end{lstlisting}

\textbf{Key Points}:
\begin{itemize}
    \item \textbf{Runnable Interface}: Better than extending Thread class
    \item \textbf{Thread.sleep()}: Pauses thread execution
    \item \textbf{Multiple Threads}: Run concurrently
\end{itemize}
\end{solutionbox}

\begin{mnemonicbox}
\mnemonic{Implement Runnable Start Multiple}
\end{mnemonicbox}

\questionmarks{4(a OR)}{3}{List four different inbuilt exceptions. Explain any one inbuilt exception.}

\begin{solutionbox}
\textbf{Inbuilt Exceptions}:
\begin{itemize}
    \item \textbf{NullPointerException}: Accessing null object
    \item \textbf{ArrayIndexOutOfBoundsException}: Invalid array index
    \item \textbf{ArithmeticException}: Division by zero
    \item \textbf{NumberFormatException}: Invalid number format
\end{itemize}

\textbf{ArithmeticException}: Thrown when arithmetic operation fails.

\begin{lstlisting}[language=Java,caption={ArithmeticException}]
int result = 10 / 0; // Throws ArithmeticException
\end{lstlisting}
\end{solutionbox}

\begin{mnemonicbox}
\mnemonic{Null Array Arithmetic Number}
\end{mnemonicbox}

\questionmarks{4(b OR)}{4}{Explain Try and Catch with suitable example.}

\begin{solutionbox}
\textbf{Try-Catch}: Exception handling mechanism.

\begin{lstlisting}[language=Java,caption={Try Catch Example}]
public class TryCatchExample {
    public static void main(String[] args) {
        try {
            int[] arr = {1, 2, 3};
            System.out.println(arr[5]); // Index out of bounds
        }
        catch(ArrayIndexOutOfBoundsException e) {
            System.out.println("Array index error: " + e.getMessage());
        }
        finally {
            System.out.println("Always executed");
        }
    }
}
\end{lstlisting}

\textbf{Exception Handling Flow}:
\begin{itemize}
    \item \textbf{Try}: Code that may throw exception
    \item \textbf{Catch}: Handles specific exceptions
    \item \textbf{Finally}: Always executes
\end{itemize}
\end{solutionbox}

\begin{mnemonicbox}
\mnemonic{Try Catch Finally}
\end{mnemonicbox}

\questionmarks{4(c OR)}{7}{What is Exception? Write a program that show the use of Arithmetic Exception.}

\begin{solutionbox}
\textbf{Exception}: Runtime error that disrupts normal program flow.

\begin{lstlisting}[language=Java,caption={ArithmeticException Example}]
public class ArithmeticExceptionExample {
    public static void main(String[] args) {
        Scanner sc = new Scanner(System.in);
        
        try {
            System.out.print("Enter first number: ");
            int num1 = sc.nextInt();
            
            System.out.print("Enter second number: ");
            int num2 = sc.nextInt();
            
            int result = num1 / num2;
            System.out.println("Result: " + result);
        }
        catch(ArithmeticException e) {
            System.out.println("Error: Cannot divide by zero!");
        }
        catch(Exception e) {
            System.out.println("General error: " + e.getMessage());
        }
        finally {
            sc.close();
        }
    }
}
\end{lstlisting}

\textbf{Exception Types}:
\begin{itemize}
    \item \textbf{Checked}: Compile-time exceptions
    \item \textbf{Unchecked}: Runtime exceptions
    \item \textbf{Error}: System-level problems
\end{itemize}
\end{solutionbox}

\begin{mnemonicbox}
\mnemonic{Runtime Error Disrupts Flow}
\end{mnemonicbox}

\questionmarks{5(a)}{3}{Explain ArrayIndexOutOfBound Exception in Java with example.}

\begin{solutionbox}
\textbf{ArrayIndexOutOfBoundsException}: Thrown when accessing invalid array index.

\begin{lstlisting}[language=Java,caption={ArrayIndexOutOfBoundsException}]
public class ArrayIndexExample {
    public static void main(String[] args) {
        int[] numbers = {10, 20, 30};
        
        try {
            System.out.println(numbers[5]); // Invalid index
        }
        catch(ArrayIndexOutOfBoundsException e) {
            System.out.println("Invalid array index: " + e.getMessage());
        }
    }
}
\end{lstlisting}

\textbf{Key Points}:
\begin{itemize}
    \item \textbf{Valid Range}: 0 to array.length-1
    \item \textbf{Negative Index}: Also throws exception
    \item \textbf{Runtime Exception}: Unchecked exception
\end{itemize}
\end{solutionbox}

\begin{mnemonicbox}
\mnemonic{Array Index Range Check}
\end{mnemonicbox}

\questionmarks{5(b)}{4}{Explain basics of stream classes.}

\begin{solutionbox}
\textbf{Stream Classes}: Handle input/output operations.

\begin{center}
\captionof{table}{Stream Classes}
\begin{tabulary}{\linewidth}{|L|L|}
\hline
\textbf{Stream Type} & \textbf{Classes} \\ \hline
Byte Streams & InputStream, OutputStream \\ \hline
Character Streams & Reader, Writer \\ \hline
File Streams & FileInputStream, FileOutputStream \\ \hline
Buffered Streams & BufferedReader, BufferedWriter \\ \hline
\end{tabulary}
\end{center}

\begin{center}
\begin{tikzpicture}[auto, node distance=1.5cm]
    \node [gtu block] (stream) {Stream Classes};
    \node [gtu block, below left=of stream, text width=3cm] (byte) {Byte Streams};
    \node [gtu block, below right=of stream, text width=3cm] (char) {Character Streams};
    
    \node [gtu block, below=0.5cm of byte, text width=3cm] (input) {InputStream};
    \node [gtu block, below=0.5cm of input, text width=3cm] (output) {OutputStream};
    
    \node [gtu block, below=0.5cm of char, text width=3cm] (reader) {Reader};
    \node [gtu block, below=0.5cm of reader, text width=3cm] (writer) {Writer};

    \path [gtu arrow] (stream) -- (byte);
    \path [gtu arrow] (stream) -- (char);
    \path [gtu arrow] (byte) -- (input);
    \path [gtu arrow] (byte) -- (output);
    \path [gtu arrow] (char) -- (reader);
    \path [gtu arrow] (char) -- (writer);
\end{tikzpicture}
\end{center}

\textbf{Stream Features}:
\begin{itemize}
    \item \textbf{Sequential}: Data flows in sequence
    \item \textbf{One Direction}: Either input or output
    \item \textbf{Automatic}: Handles low-level details
\end{itemize}
\end{solutionbox}

\begin{mnemonicbox}
\mnemonic{Byte Character File Buffered}
\end{mnemonicbox}

\questionmarks{5(c)}{7}{Write a java program to create a text file and perform read operation on the text file.}

\begin{solutionbox}
\begin{lstlisting}[language=Java,caption={File Create and Read}]
import java.io.*;

public class FileReadExample {
    public static void main(String[] args) {
        // Create and write to file
        try {
            FileWriter writer = new FileWriter("sample.txt");
            writer.write("Hello World!\n");
            writer.write("Java File Handling\n");
            writer.write("GTU Exam 2024");
            writer.close();
            System.out.println("File created successfully");
        }
        catch(IOException e) {
            System.out.println("Error creating file: " + e.getMessage());
        }
        
        // Read from file
        try {
            BufferedReader reader = new BufferedReader(new FileReader("sample.txt"));
            String line;
            
            System.out.println("\nFile contents:");
            while((line = reader.readLine()) != null) {
                System.out.println(line);
            }
            reader.close();
        }
        catch(IOException e) {
            System.out.println("Error reading file: " + e.getMessage());
        }
    }
}
\end{lstlisting}

\textbf{Key Points}:
\begin{itemize}
    \item \textbf{FileWriter}: Creates and writes to file
    \item \textbf{BufferedReader}: Efficient reading
    \item \textbf{Exception Handling}: Handle IOException
\end{itemize}
\end{solutionbox}

\begin{mnemonicbox}
\mnemonic{Create Write Read Close}
\end{mnemonicbox}

\questionmarks{5(a OR)}{3}{Explain Divide by Zero Exception in Java with example.}

\begin{solutionbox}
\textbf{ArithmeticException}: Thrown during divide by zero operation.

\begin{lstlisting}[language=Java,caption={Divide by Zero}]
public class DivideByZeroExample {
    public static void main(String[] args) {
        try {
            int a = 10;
            int b = 0;
            int result = a / b;  // Throws ArithmeticException
            System.out.println("Result: " + result);
        }
        catch(ArithmeticException e) {
            System.out.println("Cannot divide by zero: " + e.getMessage());
        }
    }
}
\end{lstlisting}

\textbf{Key Points}:
\begin{itemize}
    \item \textbf{Integer Division}: Only integer division by zero throws exception
    \item \textbf{Floating Point}: Returns Infinity for floating point division
    \item \textbf{Runtime Exception}: Unchecked exception
\end{itemize}
\end{solutionbox}

\begin{mnemonicbox}
\mnemonic{Zero Division Arithmetic Error}
\end{mnemonicbox}

\questionmarks{5(b OR)}{4}{Explain java I/O process.}

\begin{solutionbox}
\textbf{Java I/O Process}: Mechanism for reading and writing data.

\begin{center}
\begin{tikzpicture}[auto, node distance=2cm]
    \node [gtu block, text width=3cm] (source) {Data Source};
    \node [gtu block, right=of source, text width=3cm] (input) {Input Stream};
    \node [gtu block, right=of input, text width=3cm] (prog) {Java Program};
    \node [gtu block, below=of prog, text width=3cm] (output) {Output Stream};
    \node [gtu block, left=of output, text width=3cm] (dest) {Data Destination};

    \path [gtu arrow] (source) -- (input);
    \path [gtu arrow] (input) -- (prog);
    \path [gtu arrow] (prog) -- (output);
    \path [gtu arrow] (output) -- (dest);
\end{tikzpicture}
\end{center}

\textbf{I/O Components}:
\begin{itemize}
    \item \textbf{Stream}: Sequence of data
    \item \textbf{Buffer}: Temporary storage for efficiency
    \item \textbf{File}: Persistent storage
    \item \textbf{Network}: Remote data transfer
\end{itemize}

\textbf{I/O Types}:
\begin{itemize}
    \item \textbf{Byte-oriented}: Raw data (images, videos)
    \item \textbf{Character-oriented}: Text data
    \item \textbf{Synchronous}: Blocking operations
    \item \textbf{Asynchronous}: Non-blocking operations
\end{itemize}
\end{solutionbox}

\begin{mnemonicbox}
\mnemonic{Stream Buffer File Network}
\end{mnemonicbox}

\questionmarks{5(c OR)}{7}{Write a java program to create a text file and perform write operation on the text file.}

\begin{solutionbox}
\begin{lstlisting}[language=Java,caption={File Write Example}]
import java.io.*;
import java.util.Scanner;

public class FileWriteExample {
    public static void main(String[] args) {
        Scanner sc = new Scanner(System.in);
        
        try {
            // Create file with FileWriter
            FileWriter writer = new FileWriter("student.txt");
            
            System.out.println("Enter student details:");
            System.out.print("Name: ");
            String name = sc.nextLine();
            
            System.out.print("Roll Number: ");
            String rollNo = sc.nextLine();
            
            System.out.print("Branch: ");
            String branch = sc.nextLine();
            
            // Write data to file
            writer.write("Student Information\n");
            writer.write("==================\n");
            writer.write("Name: " + name + "\n");
            writer.write("Roll Number: " + rollNo + "\n");
            writer.write("Branch: " + branch + "\n");
            writer.write("Date: " + new java.util.Date() + "\n");
            
            writer.close();
            System.out.println("\nData written to file successfully!");
            
        }
        catch(IOException e) {
            System.out.println("Error writing to file: " + e.getMessage());
        }
        finally {
            sc.close();
        }
    }
}
\end{lstlisting}

\textbf{Key Points}:
\begin{itemize}
    \item \textbf{FileWriter}: Writes character data to file
    \item \textbf{BufferedWriter}: More efficient for large data
    \item \textbf{Auto-close}: Use try-with-resources for automatic closing
\end{itemize}
\end{solutionbox}

\begin{mnemonicbox}
\mnemonic{Create Write Close Handle}
\end{mnemonicbox}

\end{document}


