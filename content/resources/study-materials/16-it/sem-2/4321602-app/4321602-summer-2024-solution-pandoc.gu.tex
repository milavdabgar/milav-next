\documentclass[10pt,a4paper]{article}

% content/resources/templates/preamble.tex
\usepackage[margin=0.6in]{geometry}
\author{Milav Dabgar}
\usepackage{amsmath,amssymb,amsthm}
\usepackage{booktabs}
\usepackage{multirow}
\usepackage{xcolor}
\usepackage{tcolorbox}
\tcbuselibrary{breakable,skins}
\usepackage[colorlinks=true,linkcolor=blue]{hyperref}
\usepackage{titlesec}
\usepackage{enumitem}
\usepackage{tikz}
\usepackage{pgfplots}
\usepackage{circuitikz}
\usepackage[version=4]{mhchem}
\usepackage{longtable}
\usepackage{array}
\usepackage{float}
\usepackage{caption}
\usepackage{listings}

\lstset{
  basicstyle=\small\ttfamily,
  breaklines=true,
  breakatwhitespace=false,
  postbreak=\mbox{\textcolor{red}{$\hookrightarrow$}\space},
  float=false,
  numbers=left,
  numberstyle=\tiny\color{gray},
  numbersep=10pt,
  xleftmargin=2em,
  keywordstyle=\color{blue},
  commentstyle=\color{green!60!black},
  stringstyle=\color{purple},
  backgroundcolor=\color{gray!5},
  showstringspaces=false,
  tabsize=2,
  captionpos=b,
  keepspaces=true,
  columns=flexible
}

\pgfplotsset{compat=1.18}
\usetikzlibrary{shapes,arrows,positioning,calc,patterns,decorations.pathmorphing,decorations.markings,arrows.meta}

% Color scheme
\definecolor{headcolor}{RGB}{0,102,204}
\definecolor{keycolor}{RGB}{220,20,60}
\definecolor{solutioncolor}{RGB}{34,139,34}
\definecolor{mnemoniccolor}{RGB}{148,0,211}
\definecolor{codecolor}{RGB}{0,0,100}

% Spacing
\setlength{\parskip}{3pt}
\setlist[itemize]{nosep}
\setlist[enumerate]{nosep}

% Title formatting
\titleformat{\section}{\Large\bfseries\color{headcolor}}{\thesection}{1em}{}
\titleformat{\subsection}{\large\bfseries\color{headcolor}}{\thesubsection}{1em}{}

% Pandoc tightlist compatibility
\providecommand{\tightlist}{%
  \setlength{\itemsep}{0pt}\setlength{\parskip}{0pt}}

% Pandoc longtable compatibility
\newcounter{none}
\def\thenone{}


% content/resources/templates/gujarati-boxes.tex
\usepackage{fontspec}
\usepackage{polyglossia}

% Set Gujarati as main language (document is primarily in Gujarati)
% Note: gloss-gujarati.ldf doesn't exist in polyglossia, but it will use hyphenation patterns
\setdefaultlanguage{gujarati}
\setotherlanguage{english}

% Configure Gujarati font properly
% Use Language=Default to prevent polyglossia from trying to add language-specific features
% that don't exist for Gujarati, which causes "empty feature" warnings
\newfontfamily\gujaratifont[Script=Gujarati,AutoFakeBold=2.5,AutoFakeSlant=0.3]{Noto Sans Gujarati}
\setmainfont[Script=Gujarati,AutoFakeBold=2.5,AutoFakeSlant=0.3]{Noto Sans Gujarati}
% Use Noto Sans Gujarati for monospace to support Gujarati in text
\setmonofont[Scale=0.9]{Noto Sans Gujarati}

% Configure English to use the same font
\newfontfamily\englishfont[Script=Gujarati,AutoFakeBold=2.5,AutoFakeSlant=0.3]{Noto Sans Gujarati}

% Translations for polyglossia
\gappto\captionsgujarati{
  \renewcommand{\tablename}{કોષ્ટક}
  \renewcommand{\figurename}{આકૃતિ}
}

% Helper for TikZ nodes to ensure Gujarati font
\newcommand{\gu}[1]{{\gujaratifont #1}}

% Custom environments
\newtcolorbox{solutionbox}{
    breakable,
    enhanced,
    colback=solutioncolor!5!white,
    colframe=solutioncolor!75!black,
    fonttitle=\bfseries,
    title=જવાબ
}

\newtcolorbox{solutionboxnobreak}{
 colback=solutioncolor!5!white,
 colframe=solutioncolor!75!black,
 fonttitle=\bfseries,
 title=જવાબ
}

\newtcolorbox{keyformula}{
 breakable,
 enhanced,
 colback=keycolor!5!white,
 colframe=keycolor!75!black,
 fonttitle=\bfseries,
 title=રાસાયણિક સમીકરણ/સૂત્ર
}

\newtcolorbox{mnemonicbox}{
 breakable,
 enhanced,
 colback=mnemoniccolor!5!white,
 colframe=mnemoniccolor!75!black,
 fonttitle=\bfseries,
 title=મેમરી ટ્રીક
}


\begin{document}

\begin{center}
{\Huge\bfseries\color{headcolor} Subject Name (Gujarati)}\\[5pt]
{\LARGE 4321602 -- Summer 2024}\\[3pt]
{\large Semester 1 Study Material}\\[3pt]
{\normalsize\textit{Detailed Solutions and Explanations}}
\end{center}

\vspace{10pt}

\subsection*{પ્રશ્ન 1(અ) [3
ગુણ]}\label{uxaaauxab0uxab6uxaa8-1uxa85-3-uxa97uxaa3}

\textbf{પાયથનમાં ટપલ અને લિસ્ટ વચ્ચેનો તફાવત લખો.}

\begin{solutionbox}

{\def\LTcaptype{none} % do not increment counter
\begin{longtable}[]{@{}
  >{\raggedright\arraybackslash}p{(\linewidth - 4\tabcolsep) * \real{0.4091}}
  >{\raggedright\arraybackslash}p{(\linewidth - 4\tabcolsep) * \real{0.3182}}
  >{\raggedright\arraybackslash}p{(\linewidth - 4\tabcolsep) * \real{0.2727}}@{}}
\toprule\noalign{}
\begin{minipage}[b]{\linewidth}\raggedright
લક્ષણ
\end{minipage} & \begin{minipage}[b]{\linewidth}\raggedright
ટપલ
\end{minipage} & \begin{minipage}[b]{\linewidth}\raggedright
લિસ્ટ
\end{minipage} \\
\midrule\noalign{}
\endhead
\bottomrule\noalign{}
\endlastfoot
\textbf{મ્યુટેબિલિટી} & ઇમ્યુટેબલ (બદલી શકાતું નથી) & મ્યુટેબલ (બદલી શકાય છે) \\
\textbf{સિન્ટેક્સ} & () સાથે બનાવાય છે & [] સાથે બનાવાય છે \\
\textbf{પ્રદર્શન} & ઝડપી & ધીમું \\
\textbf{મેથડ્સ} & મર્યાદિત મેથડ્સ (count, index) & ઘણી મેથડ્સ (append, remove,
વગેરે) \\
\end{longtable}
}

\begin{itemize}
\tightlist
\item
  \textbf{મેમરી કાર્યક્ષમ}: ટપલ લિસ્ટ કરતાં ઓછી મેમરી વાપરે છે
\item
  \textbf{ઉપયોગ}: સ્થિર ડેટા માટે ટપલ, ગતિશીલ ડેટા માટે લિસ્ટ
\end{itemize}

\end{solutionbox}
\begin{mnemonicbox}
``ટપલ ટાઇટ, લિસ્ટ લૂઝ''

\end{mnemonicbox}
\begin{center}\rule{0.5\linewidth}{0.5pt}\end{center}

\subsection*{પ્રશ્ન 1(બ) [4
ગુણ]}\label{uxaaauxab0uxab6uxaa8-1uxaac-4-uxa97uxaa3}

\textbf{સેટ સમજાવો અને પાયથનમાં સેટ કેવી રીતે બનાવાય છે?}

\begin{solutionbox}

\textbf{સેટ} એ પાયથનમાં અનોખા તત્વોનો અક્રમાંકિત સંગ્રહ છે.

\textbf{સેટ બનાવવાની રીતો:}

\begin{verbatim}
\# ખાલી સેટ
my\_set = set()

\# તત્વો સાથે સેટ
fruits = \{"apple", "banana", "orange"\}

\# લિસ્ટમાંથી સેટ
numbers = set([1, 2, 3, 4])
\end{verbatim}

\begin{itemize}
\tightlist
\item
  \textbf{અનોખા તત્વો}: ડુપ્લિકેટની મંજૂરી નથી
\item
  \textbf{અક્રમાંકિત}: તત્વોનો કોઈ ચોક્કસ ક્રમ નથી
\item
  \textbf{ઓપરેશન્સ}: યુનિયન, ઇન્ટરસેક્શન, ડિફરન્સ સપોર્ટેડ
\end{itemize}

\end{solutionbox}
\begin{mnemonicbox}
``સેટ સ્પેશિયલ - અનોખા અને અક્રમાંકિત''

\end{mnemonicbox}
\begin{center}\rule{0.5\linewidth}{0.5pt}\end{center}

\subsection*{પ્રશ્ન 1(ક) [7
ગુણ]}\label{uxaaauxab0uxab6uxaa8-1uxa95-7-uxa97uxaa3}

\textbf{પાયથનમાં ડિક્શનરી એટલે શું? બે ડિક્શનરીને નવી ડિક્શનરીમાં જોડવા માટેનો
પ્રોગ્રામ લખો.}

\begin{solutionbox}

\textbf{ડિક્શનરી} એ પાયથનમાં કી-વેલ્યુ પેર્સનો ક્રમાંકિત સંગ્રહ છે.

\textbf{પ્રોગ્રામ:}

\begin{verbatim}
\# બે ડિક્શનરીઓ
dict1 = \{1: 10, 2: 20\}
dict2 = \{3: 30, 4: 40\}

\# મેથડ 1: update() નો ઉપયોગ
result1 = dict1.copy()
result1.update(dict2)

\# મેથડ 2: ** ઓપરેટરનો ઉપયોગ
result2 = \{**dict1, **dict2\}

print("પરિણામ:", result2)
\# આઉટપુટ: \{1: 10, 2: 20, 3: 30, 4: 40\}
\end{verbatim}

\begin{itemize}
\tightlist
\item
  \textbf{કી-વેલ્યુ પેર્સ}: દરેક તત્વમાં કી અને વેલ્યુ હોય છે
\item
  \textbf{મ્યુટેબલ}: બનાવ્યા પછી બદલી શકાય છે
\item
  \textbf{ઝડપી એક્સેસ}: O(1) સરેરાશ સમય જટિલતા
\end{itemize}

\end{solutionbox}
\begin{mnemonicbox}
``ડિક્શનરી ડાયનેમિક કી-વેલ્યુ સ્ટોર છે''

\end{mnemonicbox}
\begin{center}\rule{0.5\linewidth}{0.5pt}\end{center}

\subsection*{પ્રશ્ન 1(ક) અથવા [7
ગુણ]}\label{uxaaauxab0uxab6uxaa8-1uxa95-uxa85uxaa5uxab5-7-uxa97uxaa3}

\textbf{પાયથનમાં લિસ્ટ એટલે શું? એક પ્રોગ્રામ લખો જે સૂચિમાંથી મહત્તમ અને ન્યૂનતમ નંબરો
શોધે.}

\begin{solutionbox}

\textbf{લિસ્ટ} એ પાયથનમાં તત્વોનો ક્રમાંકિત, મ્યુટેબલ સંગ્રહ છે.

\textbf{પ્રોગ્રામ:}

\begin{verbatim}
\# ઇનપુટ લિસ્ટ
numbers = [45, 12, 78, 23, 56, 89, 34]

\# મહત્તમ અને ન્યૂનતમ શોધો
maximum = max(numbers)
minimum = min(numbers)

print(f"મહત્તમ: \{maximum\}")
print(f"ન્યૂનતમ: \{minimum\}")

\# મેન્યુઅલ મેથડ
max\_val = numbers[0]
min\_val = numbers[0]
for num in numbers:
    if num {} max\_val:
        max\_val = num
    if num {} min\_val:
        min\_val = num
\end{verbatim}

\begin{itemize}
\tightlist
\item
  \textbf{ક્રમાંકિત}: તત્વો ઇન્સર્શન ઓર્ડર જાળવે છે
\item
  \textbf{ઇન્ડેક્સિંગ}: ઇન્ડેક્સ [0, 1, 2\ldots] વાપરીને એક્સેસ
\item
  \textbf{બિલ્ટ-ઇન ફંક્શન્સ}: min(), max(), len() ઉપલબ્ધ
\end{itemize}

\end{solutionbox}
\begin{mnemonicbox}
``લિસ્ટ લિનિયર અને ઇન્ડેક્સ્ડ છે''

\end{mnemonicbox}
\begin{center}\rule{0.5\linewidth}{0.5pt}\end{center}

\subsection*{પ્રશ્ન 2(અ) [3
ગુણ]}\label{uxaaauxab0uxab6uxaa8-2uxa85-3-uxa97uxaa3}

\textbf{નેસ્ટેડ ટપલને ઉદાહરણ સાથે સમજાવો.}

\begin{solutionbox}

\textbf{નેસ્ટેડ ટપલ} એ ટપલ છે જેમાં અન્ય ટપલ તત્વો તરીકે હોય છે.

\textbf{ઉદાહરણ:}

\begin{verbatim}
\# નેસ્ટેડ ટપલ
student\_data = (
    ("John", 85, "A"),
    ("Alice", 92, "A+"),
    ("Bob", 78, "B")
)

\# તત્વોને એક્સેસ કરવું
print(student\_data[0][1])  \# આઉટપુટ: 85
print(student\_data[1][0])  \# આઉટપુટ: Alice
\end{verbatim}

\begin{itemize}
\tightlist
\item
  \textbf{બહુ-પરિમાણીય}: ટપલની અંદર ટપલ
\item
  \textbf{ઇન્ડેક્સિંગ}: બહુવિધ ઇન્ડિસેસ [i][j] વાપરો
\item
  \textbf{ઇમ્યુટેબલ}: નેસ્ટેડ તત્વો બદલી શકાતા નથી
\end{itemize}

\end{solutionbox}
\begin{mnemonicbox}
``નેસ્ટેડ મતલબ ટપલની અંદર ટપલ''

\end{mnemonicbox}
\begin{center}\rule{0.5\linewidth}{0.5pt}\end{center}

\subsection*{પ્રશ્ન 2(બ) [4
ગુણ]}\label{uxaaauxab0uxab6uxaa8-2uxaac-4-uxa97uxaa3}

\textbf{રેન્ડમ મોડ્યુલ શું છે? ઉદાહરણ સાથે સમજાવો.}

\begin{solutionbox}

\textbf{રેન્ડમ મોડ્યુલ} રેન્ડમ નંબરો જનરેટ કરે છે અને રેન્ડમ ઓપરેશન્સ કરે છે.

\textbf{ઉદાહરણ:}

\begin{verbatim}
import random

\# રેન્ડમ ઇન્ટિજર
num = random.randint(1, 10)
print(f"રેન્ડમ નંબર: \{num\}")

\# લિસ્ટમાંથી રેન્ડમ પસંદગી
colors = ["લાલ", "નીલો", "લીલો"]
choice = random.choice(colors)
print(f"રેન્ડમ રંગ: \{choice\}")

\# રેન્ડમ ફ્લોટ
decimal = random.random()
print(f"રેન્ડમ દશાંશ: \{decimal\}")
\end{verbatim}

\begin{itemize}
\tightlist
\item
  \textbf{ઇમ્પોર્ટ જરૂરી}: import random
\item
  \textbf{વિવિધ ફંક્શન્સ}: randint(), choice(), random()
\item
  \textbf{ઉપયોગી}: ગેમ્સ, સિમ્યુલેશન, ટેસ્ટિંગ માટે
\end{itemize}

\end{solutionbox}
\begin{mnemonicbox}
``રેન્ડમ વસ્તુઓને અણધારી બનાવે છે''

\end{mnemonicbox}
\begin{center}\rule{0.5\linewidth}{0.5pt}\end{center}

\subsection*{પ્રશ્ન 2(ક) [7
ગુણ]}\label{uxaaauxab0uxab6uxaa8-2uxa95-7-uxa97uxaa3}

\textbf{પેકેજને ઇમ્પોર્ટ કરવાની વિવિધ રીતો સમજાવો. તેનું એક ઉદાહરણ આપો.}

\begin{solutionbox}

\textbf{ઇમ્પોર્ટ મેથડ્સ:}

{\def\LTcaptype{none} % do not increment counter
\begin{longtable}[]{@{}lll@{}}
\toprule\noalign{}
મેથડ & સિન્ટેક્સ & ઉપયોગ \\
\midrule\noalign{}
\endhead
\bottomrule\noalign{}
\endlastfoot
\textbf{નોર્મલ ઇમ્પોર્ટ} & \texttt{import\ package} & package.function() \\
\textbf{ફ્રોમ ઇમ્પોર્ટ} & \texttt{from\ package\ import\ function} &
function() \\
\textbf{બધું ઇમ્પોર્ટ} & \texttt{from\ package\ import\ *} & function() \\
\textbf{એલિયાસ ઇમ્પોર્ટ} & \texttt{import\ package\ as\ alias} &
alias.function() \\
\end{longtable}
}

\textbf{ઉદાહરણ:}

\begin{verbatim}
\# નોર્મલ ઇમ્પોર્ટ
import math
result1 = math.sqrt(16)

\# ફ્રોમ ઇમ્પોર્ટ
from math import sqrt
result2 = sqrt(16)

\# એલિયાસ સાથે ઇમ્પોર્ટ
import math as m
result3 = m.sqrt(16)

\# બધું ઇમ્પોર્ટ (ભલામણ નથી)
from math import *
result4 = sqrt(16)
\end{verbatim}

\begin{itemize}
\tightlist
\item
  \textbf{નેમસ્પેસ}: નોર્મલ ઇમ્પોર્ટ અલગ નેમસ્પેસ રાખે છે
\item
  \textbf{ડાયરેક્ટ એક્સેસ}: ફ્રોમ ઇમ્પોર્ટ ડાયરેક્ટ ફંક્શન કોલ કરવાની મંજૂરી આપે છે
\item
  \textbf{એલિયાસ}: સુવિધા માટે ટૂંકા નામો
\end{itemize}

\end{solutionbox}
\begin{mnemonicbox}
``ઇમ્પોર્ટ મેથડ્સ: નોર્મલ, ફ્રોમ, બધું, એલિયાસ''

\end{mnemonicbox}
\begin{center}\rule{0.5\linewidth}{0.5pt}\end{center}

\subsection*{પ્રશ્ન 2(અ) અથવા [3
ગુણ]}\label{uxaaauxab0uxab6uxaa8-2uxa85-uxa85uxaa5uxab5-3-uxa97uxaa3}

\textbf{પાયથનમાં ડિક્શનરીના ગુણધર્મો લખો.}

\begin{solutionbox}

\textbf{ડિક્શનરીના ગુણધર્મો:}

{\def\LTcaptype{none} % do not increment counter
\begin{longtable}[]{@{}ll@{}}
\toprule\noalign{}
ગુણધર્મ & વર્ણન \\
\midrule\noalign{}
\endhead
\bottomrule\noalign{}
\endlastfoot
\textbf{ક્રમાંકિત} & ઇન્સર્શન ઓર્ડર જાળવે છે (Python 3.7+) \\
\textbf{મ્યુટેબલ} & બનાવ્યા પછી બદલી શકાય છે \\
\textbf{કી-અનોખી} & ડુપ્લિકેટ કીઓની મંજૂરી નથી \\
\textbf{હેટેરોજીનિયસ} & કીઓ અને વેલ્યુઝ અલગ પ્રકારના હોઈ શકે \\
\end{longtable}
}

\begin{itemize}
\tightlist
\item
  \textbf{ઝડપી એક્સેસ}: O(1) સરેરાશ લુકઅપ ટાઇમ
\item
  \textbf{ડાયનેમિક સાઇઝ}: વધી અથવા ઘટી શકે છે
\item
  \textbf{કી પ્રતિબંધો}: કીઓ ઇમ્યુટેબલ હોવી જોઈએ
\end{itemize}

\end{solutionbox}
\begin{mnemonicbox}
``ડિક્શનરી ક્રમાંકિત, મ્યુટેબલ, અનોખી, હેટેરોજીનિયસ છે''

\end{mnemonicbox}
\begin{center}\rule{0.5\linewidth}{0.5pt}\end{center}

\subsection*{પ્રશ્ન 2(બ) અથવા [4
ગુણ]}\label{uxaaauxab0uxab6uxaa8-2uxaac-uxa85uxaa5uxab5-4-uxa97uxaa3}

\textbf{પાયથનમાં dir() ફંક્શન શું છે. ઉદાહરણ સાથે સમજાવો.}

\begin{solutionbox}

\textbf{dir() ફંક્શન} ઓબ્જેક્ટના બધા એટ્રિબ્યુટ્સ અને મેથડ્સ રિટર્ન કરે છે.

\textbf{ઉદાહરણ:}

\begin{verbatim}
\# સ્ટ્રિંગના બધા એટ્રિબ્યુટ્સ
text = "hello"
attributes = dir(text)
print(attributes[:5])  \# પ્રથમ 5 એટ્રિબ્યુટ્સ

\# ઉપલબ્ધ મેથડ્સ ચેક કરો
print("upper" in dir(text))  \# True

\# મોડ્યુલ્સ માટે
import math
math\_methods = dir(math)
print("sqrt" in math\_methods)  \# True

\# કસ્ટમ ઓબ્જેક્ટ્સ માટે
class MyClass:
    def my\_method(self):
        pass

obj = MyClass()
print(dir(obj))
\end{verbatim}

\begin{itemize}
\tightlist
\item
  \textbf{ઇન્ટ્રોસ્પેક્શન}: ઓબ્જેક્ટ પ્રોપર્ટીઝ તપાસે છે
\item
  \textbf{ડિબગિંગ}: ઉપલબ્ધ મેથડ્સ શોધવામાં મદદ કરે છે
\item
  \textbf{બધા ઓબ્જેક્ટ્સ}: કોઈપણ Python ઓબ્જેક્ટ સાથે કામ કરે છે
\end{itemize}

\end{solutionbox}
\begin{mnemonicbox}
``dir() ઓબ્જેક્ટ એટ્રિબ્યુટ્સની ડિરેક્ટરી બતાવે છે''

\end{mnemonicbox}
\begin{center}\rule{0.5\linewidth}{0.5pt}\end{center}

\subsection*{પ્રશ્ન 2(ક) અથવા [7
ગુણ]}\label{uxaaauxab0uxab6uxaa8-2uxa95-uxa85uxaa5uxab5-7-uxa97uxaa3}

\textbf{બે સંખ્યાઓનો સરવાળો શોધવા માટે મોડ્યુલને વ્યાખ્યાયિત કરવા માટે પ્રોગ્રામ
લખો. બીજા પ્રોગ્રામમાં મોડ્યુલ ઇમ્પોર્ટ કરો.}

\begin{solutionbox}

\textbf{મોડ્યુલ ફાઇલ (calculator.py):}

\begin{verbatim}
\# calculator.py
def add\_numbers(a, b):
    """બે સંખ્યાઓ ઉમેરવા માટેનું ફંક્શન"""
    return a + b

def multiply\_numbers(a, b):
    """બે સંખ્યાઓ ગુણવા માટેનું ફંક્શન"""
    return a * b

def get\_sum(num1, num2):
    """વૈકલ્પિક સમ ફંક્શન"""
    result = num1 + num2
    return result
\end{verbatim}

\textbf{મુખ્ય પ્રોગ્રામ:}

\begin{verbatim}
\# main.py
import calculator

\# મોડ્યુલનો ઉપયોગ
result1 = calculator.add\_numbers(10, 20)
print(f"સરવાળો: \{result1\}")

\# ફ્રોમ ઇમ્પોર્ટ
from calculator import get\_sum
result2 = get\_sum(15, 25)
print(f"ફ્રોમ ઇમ્પોર્ટ વાપરીને સરવાળો: \{result2\}")
\end{verbatim}

\begin{itemize}
\tightlist
\item
  \textbf{મોડ્યુલ બનાવટ}: ફંક્શન્સને .py ફાઇલમાં સેવ કરો
\item
  \textbf{ઇમ્પોર્ટ}: ઇમ્પોર્ટ સ્ટેટમેન્ટ વાપરીને એક્સેસ કરો
\item
  \textbf{કોડ પુનઃઉપયોગ}: એક જ મોડ્યુલને અનેક પ્રોગ્રામમાં વાપરો
\end{itemize}

\end{solutionbox}
\begin{mnemonicbox}
``મોડ્યુલ કોડને પુનઃઉપયોગી અને વ્યવસ્થિત બનાવે છે''

\end{mnemonicbox}
\begin{center}\rule{0.5\linewidth}{0.5pt}\end{center}

\subsection*{પ્રશ્ન 3(અ) [3
ગુણ]}\label{uxaaauxab0uxab6uxaa8-3uxa85-3-uxa97uxaa3}

\textbf{રનટાઇમ એરર અને લોજિકલ એરર શું છે. ઉદાહરણ સાથે સમજાવો.}

\begin{solutionbox}

{\def\LTcaptype{none} % do not increment counter
\begin{longtable}[]{@{}
  >{\raggedright\arraybackslash}p{(\linewidth - 4\tabcolsep) * \real{0.3636}}
  >{\raggedright\arraybackslash}p{(\linewidth - 4\tabcolsep) * \real{0.3636}}
  >{\raggedright\arraybackslash}p{(\linewidth - 4\tabcolsep) * \real{0.2727}}@{}}
\toprule\noalign{}
\begin{minipage}[b]{\linewidth}\raggedright
એરર પ્રકાર
\end{minipage} & \begin{minipage}[b]{\linewidth}\raggedright
વ્યાખ્યા
\end{minipage} & \begin{minipage}[b]{\linewidth}\raggedright
ઉદાહરણ
\end{minipage} \\
\midrule\noalign{}
\endhead
\bottomrule\noalign{}
\endlastfoot
\textbf{રનટાઇમ એરર} & પ્રોગ્રામ એક્ઝિક્યુશન દરમિયાન થાય છે & શૂન્ય વડે ભાગાકાર,
ફાઇલ ન મળે \\
\textbf{લોજિકલ એરર} & પ્રોગ્રામ ચાલે છે પણ ખોટો આઉટપુટ આપે છે & ખોટું ફોર્મ્યુલા,
ખોટી કન્ડિશન \\
\end{longtable}
}

\textbf{ઉદાહરણો:}

\begin{verbatim}
\# રનટાઇમ એરર
x = 10
y = 0
result = x / y  \# ZeroDivisionError

\# લોજિકલ એરર
def calculate\_area(radius):
    return 3.14 * radius  \# radius * radius હોવું જોઈએ
\end{verbatim}

\begin{itemize}
\tightlist
\item
  \textbf{રનટાઇમ}: પ્રોગ્રામ એક્ઝિક્યુશન ક્રેશ કરે છે
\item
  \textbf{લોજિકલ}: પ્રોગ્રામ ચાલુ રહે છે પણ ખોટું પરિણામ
\end{itemize}

\end{solutionbox}
\begin{mnemonicbox}
``રનટાઇમ ક્રેશ કરે, લોજિકલ કન્ફ્યુઝ કરે''

\end{mnemonicbox}
\begin{center}\rule{0.5\linewidth}{0.5pt}\end{center}

\subsection*{પ્રશ્ન 3(બ) [4
ગુણ]}\label{uxaaauxab0uxab6uxaa8-3uxaac-4-uxa97uxaa3}

\textbf{Except ક્લોઝના મુદ્દાઓ લખો અને તેને સમજાવો.}

\begin{solutionbox}

\textbf{Except ક્લોઝ} try-except બ્લોકમાં ચોક્કસ exceptions ને હેન્ડલ કરે છે.

\textbf{મુખ્ય મુદ્દાઓ:}

{\def\LTcaptype{none} % do not increment counter
\begin{longtable}[]{@{}ll@{}}
\toprule\noalign{}
લક્ષણ & વર્ણન \\
\midrule\noalign{}
\endhead
\bottomrule\noalign{}
\endlastfoot
\textbf{સિન્ટેક્સ} & \texttt{except\ ExceptionType:} \\
\textbf{બહુવિધ} & બહુવિધ except બ્લોક્સ હોઈ શકે \\
\textbf{જનરિક} & \texttt{except:} બધા exceptions પકડે છે \\
\textbf{વેરિયેબલ} & \texttt{except\ Exception\ as\ e:} એરર સ્ટોર કરે છે \\
\end{longtable}
}

\begin{verbatim}
try:
    number = int(input("નંબર દાખલ કરો: "))
    result = 10 / number
except ValueError:
    print("અયોગ્ય ઇનપુટ")
except ZeroDivisionError:
    print("શૂન્ય વડે ભાગાકાર કરી શકાતો નથી")
except Exception as e:
    print(f"એરર: \{e\}")
\end{verbatim}

\begin{itemize}
\tightlist
\item
  \textbf{સ્પેસિફિક હેન્ડલિંગ}: અલગ exceptions અલગ રીતે હેન્ડલ થાય
\item
  \textbf{એરર રિકવરી}: હેન્ડલિંગ પછી પ્રોગ્રામ ચાલુ રહે
\end{itemize}

\end{solutionbox}
\begin{mnemonicbox}
``Except પકડે છે અને એરર હેન્ડલ કરે છે''

\end{mnemonicbox}
\begin{center}\rule{0.5\linewidth}{0.5pt}\end{center}

\subsection*{પ્રશ્ન 3(ક) [7
ગુણ]}\label{uxaaauxab0uxab6uxaa8-3uxa95-7-uxa97uxaa3}

\textbf{Divide by zero Exception ને કેચ કરવા માટેનો પ્રોગ્રામ લખો. finally
બ્લોકનો ઉપયોગ કરો.}

\begin{solutionbox}

\begin{verbatim}
def safe\_division():
    try:
        \# યુઝરથી ઇનપુટ લો
        numerator = float(input("અંશ દાખલ કરો: "))
        denominator = float(input("હર દાખલ કરો: "))
        
        \# ભાગાકાર કરો
        result = numerator / denominator
        print(f"પરિણામ: \{numerator\} / \{denominator\} = \{result\}")
        
    except ZeroDivisionError:
        print("એરર: શૂન્ય વડે ભાગાકાર કરી શકાતો નથી!")
        print("કૃપા કરીને બિન{-શૂન્ય હર દાખલ કરો"})
        
    except ValueError:
        print("એરર: કૃપા કરીને માત્ર માન્ય નંબરો જ દાખલ કરો")
        
    except Exception as e:
        print(f"અનપેક્ષિત એરર આવી: \{e\}")
        
    finally:
        print("ભાગાકાર ઓપરેશન પૂર્ણ થયું")
        print("કેલ્ક્યુલેટર ઉપયોગ કરવા બદલ આભાર")

\# ફંક્શનને કોલ કરો
safe\_division()
\end{verbatim}

\begin{itemize}
\tightlist
\item
  \textbf{Try બ્લોક}: જોખમી કોડ સમાવે છે
\item
  \textbf{Except}: ZeroDivisionError ને સ્પેસિફિકલી હેન્ડલ કરે છે
\item
  \textbf{Finally}: exception હોય કે ન હોય હંમેશા એક્ઝિક્યુટ થાય છે
\end{itemize}

\end{solutionbox}
\begin{mnemonicbox}
``Try જોખમી કોડ, Except એરર હેન્ડલ કરે, Finally હંમેશા
ચાલે''

\end{mnemonicbox}
\begin{center}\rule{0.5\linewidth}{0.5pt}\end{center}

\subsection*{પ્રશ્ન 3(અ) અથવા [3
ગુણ]}\label{uxaaauxab0uxab6uxaa8-3uxa85-uxa85uxaa5uxab5-3-uxa97uxaa3}

\textbf{બિલ્ટ-ઇન exceptions શું છે અને તેના પ્રકારો લખો.}

\begin{solutionbox}

\textbf{બિલ્ટ-ઇન Exception પ્રકારો:}

{\def\LTcaptype{none} % do not increment counter
\begin{longtable}[]{@{}lll@{}}
\toprule\noalign{}
પ્રકાર & વર્ણન & ઉદાહરણ \\
\midrule\noalign{}
\endhead
\bottomrule\noalign{}
\endlastfoot
\textbf{ValueError} & ઓપરેશન માટે અયોગ્ય વેલ્યુ & int(``abc'') \\
\textbf{TypeError} & ખોટો ડેટા પ્રકાર & ``5'' + 5 \\
\textbf{IndexError} & ઇન્ડેક્સ રેન્જની બહાર & list[10] for 5-element
list \\
\textbf{KeyError} & ડિક્શનરીમાં કી ન મળે & dict[``missing\_key''] \\
\textbf{FileNotFoundError} & ફાઇલ અસ્તિત્વમાં નથી &
open(``missing.txt'') \\
\end{longtable}
}

\begin{verbatim}
\# ઉદાહરણો
try:
    int("hello")  \# ValueError
    "5" + 5       \# TypeError
    [1,2,3][5]    \# IndexError
except (ValueError, TypeError, IndexError) as e:
    print(f"એરર: \{type(e).\_\_name\_\_\}")
\end{verbatim}

\end{solutionbox}
\begin{mnemonicbox}
``Value, Type, Index, Key, File - સામાન્ય એરર
પ્રકારો''

\end{mnemonicbox}
\begin{center}\rule{0.5\linewidth}{0.5pt}\end{center}

\subsection*{પ્રશ્ન 3(બ) અથવા [4
ગુણ]}\label{uxaaauxab0uxab6uxaa8-3uxaac-uxa85uxaa5uxab5-4-uxa97uxaa3}

\textbf{સિન્ટેક્સ એરર સમજાવો અને આપણે તેને કેવી રીતે ઓળખી શકીએ? એક ઉદાહરણ આપો.}

\begin{solutionbox}

\textbf{સિન્ટેક્સ એરર} ત્યારે થાય છે જ્યારે Python ખોટા સિન્ટેક્સને કારણે કોડ parse
કરી શકતું નથી.

\textbf{ઓળખવાની રીતો:}

{\def\LTcaptype{none} % do not increment counter
\begin{longtable}[]{@{}ll@{}}
\toprule\noalign{}
મેથડ & વર્ણન \\
\midrule\noalign{}
\endhead
\bottomrule\noalign{}
\endlastfoot
\textbf{Python interpreter} & લાઇન નંબર સાથે એરર મેસેજ બતાવે છે \\
\textbf{IDE highlighting} & કોડ એડિટર્સ સિન્ટેક્સ એરર હાઇલાઇટ કરે છે \\
\textbf{Error message} & એરરનું ચોક્કસ સ્થાન બતાવે છે \\
\end{longtable}
}

\textbf{ઉદાહરણો:}

\begin{verbatim}
\# ગુમ થયેલો કોલન
if x {} 5
    print("વધારે")  \# SyntaxError

\# અમેળ કૌંસ
print("Hello"  \# SyntaxError

\# ખોટું indentation
def my\_function():
print("Hello")  \# IndentationError

\# અયોગ્ય વેરિયેબલ નામ
2variable = 10  \# SyntaxError
\end{verbatim}

\begin{itemize}
\tightlist
\item
  \textbf{ડિટેક્શન}: પ્રોગ્રામ એક્ઝિક્યુશન પહેલાં
\item
  \textbf{એરર મેસેજ}: લાઇન અને કેરેક્ટર પોઝિશન બતાવે છે
\item
  \textbf{સામાન્ય કારણો}: ગુમ કોલન, બ્રેકેટ્સ, ખોટું indentation
\end{itemize}

\end{solutionbox}
\begin{mnemonicbox}
``સિન્ટેક્સ એરર કોડને શરૂ થવાથી રોકે છે''

\end{mnemonicbox}
\begin{center}\rule{0.5\linewidth}{0.5pt}\end{center}

\subsection*{પ્રશ્ન 3(ક) અથવા [7
ગુણ]}\label{uxaaauxab0uxab6uxaa8-3uxa95-uxa85uxaa5uxab5-7-uxa97uxaa3}

\textbf{પાયથનમાં એક્સેપશન હેન્ડલિંગ શું છે? યોગ્ય ઉદાહરણ સાથે સમજાવો.}

\begin{solutionbox}

\textbf{Exception Handling} એ રનટાઇમ એરર્સને પ્રોગ્રામ ક્રેશ કર્યા વિના
gracefully હેન્ડલ કરવાની પદ્ધતિ છે.

\textbf{સ્ટ્રક્ચર:}

\begin{verbatim}
try:
    \# જોખમી કોડ
    pass
except SpecificException:
    \# સ્પેસિફિક એરર હેન્ડલ કરો
    pass
except Exception as e:
    \# અન્ય કોઈ એરર હેન્ડલ કરો
    pass
else:
    \# exception ન હોય તો ચાલે
    pass
finally:
    \# હંમેશા ચાલે
    pass
\end{verbatim}

\textbf{સંપૂર્ણ ઉદાહરણ:}

\begin{verbatim}
def file\_processor():
    filename = None
    try:
        filename = input("ફાઇલનામ દાખલ કરો: ")
        with open(filename, {r}) as file:
            content = file.read()
            numbers = [int(x) for x in content.split()]
            average = sum(numbers) / len(numbers)
            print(f"સરેરાશ: \{average\}")
            
    except FileNotFoundError:
        print(f"એરર: ફાઇલ {}\{filename\}{ ન મળી"})
        
    except ValueError:
        print("એરર: ફાઇલમાં બિન{-આંકડાકીય ડેટા છે"})
        
    except ZeroDivisionError:
        print("એરર: ફાઇલમાં કોઈ નંબરો ન મળ્યા")
        
    except Exception as e:
        print(f"અનપેક્ષિત એરર: \{e\}")
        
    else:
        print("ફાઇલ સફળતાપૂર્વક પ્રોસેસ થઈ")
        
    finally:
        print("ફાઇલ પ્રોસેસિંગ ઓપરેશન પૂર્ણ થયું")

\# ફંક્શન ચલાવો
file\_processor()
\end{verbatim}

\begin{itemize}
\tightlist
\item
  \textbf{Graceful handling}: એરર પછી પ્રોગ્રામ ચાલુ રહે છે
\item
  \textbf{Multiple exceptions}: અલગ એરર પ્રકારો અલગ રીતે હેન્ડલ થાય છે
\item
  \textbf{Else clause}: માત્ર exception ન હોય તો જ ચાલે છે
\item
  \textbf{Finally clause}: cleanup માટે હંમેશા એક્ઝિક્યુટ થાય છે
\end{itemize}

\end{solutionbox}
\begin{mnemonicbox}
``Try-Except-Else-Finally: સંપૂર્ણ એરર હેન્ડલિંગ''

\end{mnemonicbox}
\begin{center}\rule{0.5\linewidth}{0.5pt}\end{center}

\subsection*{પ્રશ્ન 4(અ) [3
ગુણ]}\label{uxaaauxab0uxab6uxaa8-4uxa85-3-uxa97uxaa3}

\textbf{ફાઇલમાં આપણે કેવા પ્રકારની વિવિધ ઓપરેશન કરી શકીએ છીએ?}

\begin{solutionbox}

\textbf{ફાઇલ ઓપરેશન્સ:}

{\def\LTcaptype{none} % do not increment counter
\begin{longtable}[]{@{}lll@{}}
\toprule\noalign{}
ઓપરેશન & વર્ણન & મેથડ \\
\midrule\noalign{}
\endhead
\bottomrule\noalign{}
\endlastfoot
\textbf{Read} & ફાઇલ કન્ટેન્ટ વાંચો & read(), readline(), readlines() \\
\textbf{Write} & ફાઇલમાં ડેટા લખો & write(), writelines() \\
\textbf{Append} & અંતમાં ડેટા ઉમેરો & `a' મોડ સાથે open \\
\textbf{Create} & નવી ફાઇલ બનાવો & `w' અથવા `x' મોડ સાથે open \\
\textbf{Delete} & ફાઇલ રીમૂવ કરો & os.remove() \\
\textbf{Seek} & ફાઇલ પોઇન્ટર ખસેડો & seek() \\
\end{longtable}
}

\begin{verbatim}
\# ઉદાહરણ ઓપરેશન્સ
with open({file.txt}, {w}) as f:
    f.write("Hello")  \# Write
    
with open({file.txt}, {r}) as f:
    content = f.read()  \# Read
\end{verbatim}

\end{solutionbox}
\begin{mnemonicbox}
``Read, Write, Append, Create, Delete, Seek''

\end{mnemonicbox}
\begin{center}\rule{0.5\linewidth}{0.5pt}\end{center}

\subsection*{પ્રશ્ન 4(બ) [4
ગુણ]}\label{uxaaauxab0uxab6uxaa8-4uxaac-4-uxa97uxaa3}

\textbf{ફાઇલ મોડ્સની યાદી આપો. કોઈપણ ચાર મોડનું વર્ણન લખો.}

\begin{solutionbox}

\textbf{ફાઇલ મોડ્સ:}

{\def\LTcaptype{none} % do not increment counter
\begin{longtable}[]{@{}lll@{}}
\toprule\noalign{}
મોડ & વર્ણન & હેતુ \\
\midrule\noalign{}
\endhead
\bottomrule\noalign{}
\endlastfoot
\textbf{`r'} & Read મોડ (default) & અસ્તિત્વમાં છે તે ફાઇલ વાંચો \\
\textbf{`w'} & Write મોડ & નવી બનાવો અથવા અસ્તિત્વમાં છે તેને overwrite કરો \\
\textbf{`a'} & Append મોડ & અસ્તિત્વમાં છે તે ફાઇલના અંતમાં ઉમેરો \\
\textbf{`x'} & Exclusive creation & નવી ફાઇલ બનાવો, અસ્તિત્વમાં હોય તો
fail \\
\textbf{`b'} & Binary મોડ & binary ફાઇલ્સ હેન્ડલ કરો \\
\textbf{`t'} & Text મોડ (default) & text ફાઇલ્સ હેન્ડલ કરો \\
\textbf{`+'} & Read અને write & બંને ઓપરેશન્સની મંજૂરી \\
\end{longtable}
}

\textbf{ચાર મોડનું વર્ણન:}

\begin{enumerate}
\tightlist
\item
  \textbf{`r' (Read)}: માત્ર વાંચવા માટે ફાઇલ ખોલે છે, ફાઇલ પોઇન્ટર શરૂઆતમાં
\item
  \textbf{`w' (Write)}: લખવા માટે ખોલે છે, ફાઇલ truncate કરે છે અથવા નવી બનાવે
  છે
\item
  \textbf{`a' (Append)}: લખવા માટે ખોલે છે, ફાઇલ પોઇન્ટર ફાઇલના અંતમાં
\item
  \textbf{`r+' (Read/Write)}: વાંચવા અને લખવા બંને માટે ખોલે છે
\end{enumerate}

\end{solutionbox}
\begin{mnemonicbox}
``Read, Write, Append, eXclusive - મુખ્ય ફાઇલ મોડ્સ''

\end{mnemonicbox}
\begin{center}\rule{0.5\linewidth}{0.5pt}\end{center}

\subsection*{પ્રશ્ન 4(ક) [7
ગુણ]}\label{uxaaauxab0uxab6uxaa8-4uxa95-7-uxa97uxaa3}

\textbf{ફાઇલમાંના બધા શબ્દોને સૉર્ટ કરવા માટે એક પ્રોગ્રામ લખો અને તેને લિસ્ટમાં
મૂકો.}

\begin{solutionbox}

\begin{verbatim}
def sort\_words\_from\_file():
    try:
        \# ફાઇલનામ ઇનપુટ
        filename = input("ફાઇલનામ દાખલ કરો: ")
        
        \# ફાઇલ કન્ટેન્ટ વાંચો
        with open(filename, {r}, encoding={utf{-}8}) as file:
            content = file.read()
        
        \# શબ્દોમાં વિભાજિત કરો અને સાફ કરો
        words = content.lower().split()
        
        \# Punctuation રીમૂવ કરો અને ખાલી સ્ટ્રિંગ્સ
        import string
        clean\_words = []
        for word in words:
            clean\_word = word.translate(str.maketrans({}, {}, string.punctuation))
            if clean\_word:  \# માત્ર બિન{-ખાલી શબ્દો ઉમેરો}
                clean\_words.append(clean\_word)
        
        \# શબ્દોને સૉર્ટ કરો
        sorted\_words = sorted(clean\_words)
        
        \# પરિણામો દર્શાવો
        print("સૉર્ટ થયેલા શબ્દો:")
        print(sorted\_words)
        
        \# નવી ફાઇલમાં સેવ કરો
        with open({sorted\_words.txt}, {w}, encoding={utf{-}8}) as output\_file:
            for word in sorted\_words:
                output\_file.write(word + {}{n}{})
        
        print(f"કુલ શબ્દો: \{len(sorted\_words)\}")
        print("સૉર્ટ થયેલા શબ્દો {sorted\_words.txt માં સેવ થયા"})
        
    except FileNotFoundError:
        print("એરર: ફાઇલ ન મળી")
    except Exception as e:
        print(f"એરર: \{e\}")

\# પ્રોગ્રામ ચલાવો
sort\_words\_from\_file()
\end{verbatim}

\begin{itemize}
\tightlist
\item
  \textbf{ફાઇલ રીડિંગ}: સંપૂર્ણ ફાઇલ કન્ટેન્ટ વાંચો
\item
  \textbf{શબ્દ પ્રોસેસિંગ}: શબ્દોને વિભાજિત, સાફ અને સૉર્ટ કરો
\item
  \textbf{લિસ્ટ બનાવટ}: સૉર્ટ થયેલા શબ્દોને લિસ્ટમાં સ્ટોર કરો
\end{itemize}

\end{solutionbox}
\begin{mnemonicbox}
``વાંચો, વિભાજિત કરો, સાફ કરો, સૉર્ટ કરો, સેવ કરો''

\end{mnemonicbox}
\begin{center}\rule{0.5\linewidth}{0.5pt}\end{center}

\subsection*{પ્રશ્ન 4(અ) અથવા [3
ગુણ]}\label{uxaaauxab0uxab6uxaa8-4uxa85-uxa85uxaa5uxab5-3-uxa97uxaa3}

\textbf{ફાઇલ હેન્ડલિંગ શું છે? ફાઇલ્સ હેન્ડલિંગ ઓપરેશનની યાદી બનાવો અને તેને સમજાવો.}

\begin{solutionbox}

\textbf{ફાઇલ હેન્ડલિંગ} એ ડેટાને કાયમી ધોરણે સ્ટોર અને retrieve કરવા માટે ફાઇલો
સાથે કામ કરવાની પ્રક્રિયા છે.

\textbf{ફાઇલ હેન્ડલિંગ ઓપરેશન્સ:}

{\def\LTcaptype{none} % do not increment counter
\begin{longtable}[]{@{}lll@{}}
\toprule\noalign{}
ઓપરેશન & ફંક્શન & વર્ણન \\
\midrule\noalign{}
\endhead
\bottomrule\noalign{}
\endlastfoot
\textbf{Open} & open() & ચોક્કસ મોડમાં ફાઇલ ખોલે છે \\
\textbf{Read} & read(), readline() & ફાઇલમાંથી ડેટા વાંચે છે \\
\textbf{Write} & write(), writelines() & ફાઇલમાં ડેટા લખે છે \\
\textbf{Close} & close() & ફાઇલ બંધ કરે છે અને resources મુક્ત કરે છે \\
\textbf{Seek} & seek() & ફાઇલ પોઇન્ટર પોઝિશન ખસેડે છે \\
\textbf{Tell} & tell() & વર્તમાન ફાઇલ પોઇન્ટર પોઝિશન રિટર્ન કરે છે \\
\end{longtable}
}

\begin{verbatim}
\# બેસિક ફાઇલ ઓપરેશન્સ
file = open({data.txt}, {w})  \# Open
file.write({Hello World})     \# Write
file.close()                  \# Close

file = open({data.txt}, {r})  \# વાંચવા માટે ખોલો
content = file.read()         \# Read
file.close()                  \# Close
\end{verbatim}

\end{solutionbox}
\begin{mnemonicbox}
``Open, Read, Write, Close - બેસિક ફાઇલ સાઇકલ''

\end{mnemonicbox}
\begin{center}\rule{0.5\linewidth}{0.5pt}\end{center}

\subsection*{પ્રશ્ન 4(બ) અથવા [4
ગુણ]}\label{uxaaauxab0uxab6uxaa8-4uxaac-uxa85uxaa5uxab5-4-uxa97uxaa3}

\textbf{ઉદાહરણ સાથે load() મેથડ સમજાવો.}

\begin{solutionbox}

\textbf{load() મેથડ} ફાઇલમાંથી ડેટાને deserialize કરવા માટે વપરાય છે (સામાન્ય
રીતે pickle મોડ્યુલ સાથે).

\textbf{Pickle load() ઉદાહરણ:}

\begin{verbatim}
import pickle

\# પ્રથમ, કંઈક ડેટા સેવ કરો
data\_to\_save = \{
    {name}: {John},
    {age}: 25,
    {scores}: [85, 92, 78]
\}

\# ડેટાને ફાઇલમાં સેવ કરો
with open({data.pkl}, {wb}) as file:
    pickle.dump(data\_to\_save, file)

\# ફાઇલમાંથી ડેટા લોડ કરો
with open({data.pkl}, {rb}) as file:
    loaded\_data = pickle.load(file)

print("લોડ થયેલો ડેટા:", loaded\_data)
print("નામ:", loaded\_data[{name}])
print("સ્કોર્સ:", loaded\_data[{scores}])
\end{verbatim}

\textbf{JSON load() ઉદાહરણ:}

\begin{verbatim}
import json

\# JSON ડેટા લોડ કરો
with open({config.json}, {r}) as file:
    config = json.load(file)
    
print("કન્ફિગરેશન:", config)
\end{verbatim}

\begin{itemize}
\tightlist
\item
  \textbf{Deserialization}: ફાઇલ ડેટાને પાછું Python objects માં કન્વર્ટ કરે છે
\item
  \textbf{Binary મોડ}: pickle ફાઇલ્સ માટે `rb' મોડ વાપરો
\item
  \textbf{Error handling}: FileNotFoundError હેન્ડલ કરો
\end{itemize}

\end{solutionbox}
\begin{mnemonicbox}
``load() ફાઇલ ડેટાને પાછું Python objects માં લાવે છે''

\end{mnemonicbox}
\begin{center}\rule{0.5\linewidth}{0.5pt}\end{center}

\subsection*{પ્રશ્ન 4(ક) અથવા [7
ગુણ]}\label{uxaaauxab0uxab6uxaa8-4uxa95-uxa85uxaa5uxab5-7-uxa97uxaa3}

\textbf{એક પ્રોગ્રામ લખો જે ટેક્સ્ટ ફાઇલને ઇનપુટ કરે. પ્રોગ્રામે ફાઇલમાંના તમામ યુનીક
શબ્દોને મૂળાક્ષરોના ક્રમમાં છાપવા જોઈએ.}

\begin{solutionbox}

\begin{verbatim}
def find\_unique\_words():
    try:
        \# યુઝરથી ફાઇલનામ લો
        filename = input("ટેક્સ્ટ ફાઇલનામ દાખલ કરો: ")
        
        \# ફાઇલ કન્ટેન્ટ વાંચો
        with open(filename, {r}, encoding={utf{-}8}) as file:
            content = file.read().lower()
        
        \# શબ્દો સાફ કરો અને એક્સ્ટ્રેક્ટ કરો
        import re
        import string
        
        \# Punctuation રીમૂવ કરો અને શબ્દોમાં વિભાજિત કરો
        words = re.findall(r{}{b}[a{-zA{-}Z]}+{b}{}, content.lower())
        
        \# યુનીક શબ્દો મેળવવા માટે set બનાવો
        unique\_words = set(words)
        
        \# સૉર્ટ થયેલી લિસ્ટમાં કન્વર્ટ કરો
        sorted\_unique\_words = sorted(list(unique\_words))
        
        \# પરિણામો દર્શાવો
        print("{n}મૂળાક્ષરોના ક્રમમાં યુનીક શબ્દો:")
        print("{-"} * 40)
        
        for i, word in enumerate(sorted\_unique\_words, 1):
            print(f"\{i:3d\}. \{word\}")
        
        print(f"{n}કુલ યુનીક શબ્દો: \{len(sorted\_unique\_words)\}")
        
        \# પરિણામો ફાઇલમાં સેવ કરો
        with open({unique\_words\_output.txt}, {w}, encoding={utf{-}8}) as output\_file:
            output\_file.write("મૂળાક્ષરોના ક્રમમાં યુનીક શબ્દો{n}")
            output\_file.write("=" * 40 + "{nn}")
            for word in sorted\_unique\_words:
                output\_file.write(word + {}{n}{})
        
        print("પરિણામો {unique\_words\_output.txt માં સેવ થયા"})
        
    except FileNotFoundError:
        print(f"એરર: ફાઇલ {}\{filename\}{ ન મળી"})
    except PermissionError:
        print("એરર: ફાઇલ વાંચવાની પરમિશન નકારાઈ")
    except Exception as e:
        print(f"અનપેક્ષિત એરર: \{e\}")

\# ઉદાહરણ ઉપયોગ
def create\_sample\_file():
    sample\_text = """
    Python એક શક્તિશાળી પ્રોગ્રામિંગ લેંગ્વેજ છે.
    Python શીખવામાં સરળ છે અને Python વર્સેટાઇલ છે.
    Python સાથે પ્રોગ્રામિંગ મજાદાર છે અને પ્રોગ્રામિંગ લાભદાયક છે.
    """
    
    with open({sample.txt}, {w}, encoding={utf{-}8}) as f:
        f.write(sample\_text)
    print("નમૂનો ફાઇલ {sample.txt બનાવવામાં આવી"})

\# નમૂનો બનાવો અને પ્રોગ્રામ ચલાવો
create\_sample\_file()
find\_unique\_words()
\end{verbatim}

\begin{itemize}
\tightlist
\item
  \textbf{Regular expressions}: માત્ર અક્ષરવાળા શબ્દો એક્સ્ટ્રેક્ટ કરે છે
\item
  \textbf{Set ડેટા સ્ટ્રક્ચર}: આપમેળે ડુપ્લિકેટ્સ રીમૂવ કરે છે
\item
  \textbf{Sorted ફંક્શન}: શબ્દોને મૂળાક્ષરોના ક્રમમાં ગોઠવે છે
\item
  \textbf{ફાઇલ આઉટપુટ}: ભાવિ સંદર્ભ માટે પરિણામો સેવ કરે છે
\end{itemize}

\end{solutionbox}
\begin{mnemonicbox}
``વાંચો, એક્સ્ટ્રેક્ટ કરો, યુનીક, સૉર્ટ, દર્શાવો''

\end{mnemonicbox}
\begin{center}\rule{0.5\linewidth}{0.5pt}\end{center}

\subsection*{પ્રશ્ન 5(અ) [3
ગુણ]}\label{uxaaauxab0uxab6uxaa8-5uxa85-3-uxa97uxaa3}

\textbf{નીચેના ટર્ટલ ફંક્શનને યોગ્ય ઉદાહરણ સાથે સમજાવો. (a) turn() (b) move().}

\begin{solutionbox}

\textbf{નોંધ}: સ્ટાન્ડર્ડ ટર્ટલ મોડ્યુલ \texttt{turn()} ને બદલે \texttt{left()},
\texttt{right()} અને \texttt{move()} ને બદલે \texttt{forward()},
\texttt{backward()} વાપરે છે.

\textbf{ટર્ટલ મૂવમેન્ટ ફંક્શન્સ:}

{\def\LTcaptype{none} % do not increment counter
\begin{longtable}[]{@{}lll@{}}
\toprule\noalign{}
ફંક્શન & હેતુ & ઉદાહરણ \\
\midrule\noalign{}
\endhead
\bottomrule\noalign{}
\endlastfoot
\textbf{left(angle)} & ડિગ્રીમાં ડાબે ફેરવો & turtle.left(90) \\
\textbf{right(angle)} & ડિગ્રીમાં જમણે ફેરવો & turtle.right(45) \\
\textbf{forward(distance)} & આગળ ખસો & turtle.forward(100) \\
\textbf{backward(distance)} & પાછળ ખસો & turtle.backward(50) \\
\end{longtable}
}

\begin{verbatim}
import turtle

\# ટર્ટલ બનાવો
t = turtle.Turtle()

\# ટર્ન ફંક્શન્સ
t.left(90)    \# ડાબે 90 ડિગ્રી ફેરવો
t.right(45)   \# જમણે 45 ડિગ્રી ફેરવો

\# મૂવ ફંક્શન્સ
t.forward(100)  \# આગળ 100 યુનિટ ખસો
t.backward(50)  \# પાછળ 50 યુનિટ ખસો

\# વિન્ડો ખુલ્લી રાખો
turtle.done()
\end{verbatim}

\end{solutionbox}
\begin{mnemonicbox}
``ટર્ન દિશા બદલે છે, મૂવ પોઝિશન બદલે છે''

\end{mnemonicbox}
\begin{center}\rule{0.5\linewidth}{0.5pt}\end{center}

\subsection*{પ્રશ્ન 5(બ) [4
ગુણ]}\label{uxaaauxab0uxab6uxaa8-5uxaac-4-uxa97uxaa3}

\textbf{ટર્ટલની દિશા બદલવાની વિવિધ ઇનબિલ્ટ પદ્ધતિઓ સમજાવો.}

\begin{solutionbox}

\textbf{દિશા કન્ટ્રોલ મેથડ્સ:}

{\def\LTcaptype{none} % do not increment counter
\begin{longtable}[]{@{}
  >{\raggedright\arraybackslash}p{(\linewidth - 4\tabcolsep) * \real{0.2667}}
  >{\raggedright\arraybackslash}p{(\linewidth - 4\tabcolsep) * \real{0.4333}}
  >{\raggedright\arraybackslash}p{(\linewidth - 4\tabcolsep) * \real{0.3000}}@{}}
\toprule\noalign{}
\begin{minipage}[b]{\linewidth}\raggedright
મેથડ
\end{minipage} & \begin{minipage}[b]{\linewidth}\raggedright
વર્ણન
\end{minipage} & \begin{minipage}[b]{\linewidth}\raggedright
ઉદાહરણ
\end{minipage} \\
\midrule\noalign{}
\endhead
\bottomrule\noalign{}
\endlastfoot
\textbf{left(angle)} & વામાવર્ત ફેરવો & turtle.left(90) \\
\textbf{right(angle)} & દક્ષિણાવર્ત ફેરવો & turtle.right(45) \\
\textbf{setheading(angle)} & ચોક્કસ દિશા સેટ કરો & turtle.setheading(0) \\
\textbf{towards(x, y)} & કોઓર્ડિનેટ્સ તરફ નિર્દેશ કરો &
turtle.setheading(turtle.towards(100, 100)) \\
\end{longtable}
}

\begin{verbatim}
import turtle

t = turtle.Turtle()

\# સંબંધિત ફેરવણું
t.left(90)        \# ડાબે 90^ ફેરવો
t.right(45)       \# જમણે 45^ ફેરવો

\# ચોક્કસ દિશા
t.setheading(0)   \# પૂર્વ તરફ નિર્દેશ કરો (0^)
t.setheading(90)  \# ઉત્તર તરફ નિર્દેશ કરો (90^)

\# ચોક્કસ પોઇન્ટ તરફ નિર્દેશ કરો
angle = t.towards(100, 100)
t.setheading(angle)
\end{verbatim}

\begin{itemize}
\tightlist
\item
  \textbf{સંબંધિત}: left() અને right() વર્તમાન દિશા બદલે છે
\item
  \textbf{ચોક્કસ}: setheading() ચોક્કસ દિશા સેટ કરે છે
\item
  \textbf{કોઓર્ડિનેટ-આધારિત}: towards() પોઇન્ટ તરફની દિશા ગણે છે
\end{itemize}

\end{solutionbox}
\begin{mnemonicbox}
``ડાબે-જમણે સંબંધિત, હેડિંગ ચોક્કસ, તરફ ગણતરી કરે''

\end{mnemonicbox}
\begin{center}\rule{0.5\linewidth}{0.5pt}\end{center}

\subsection*{પ્રશ્ન 5(ક) [7
ગુણ]}\label{uxaaauxab0uxab6uxaa8-5uxa95-7-uxa97uxaa3}

\textbf{ટર્ટલનો ઉપયોગ કરીને ચોરસ, લંબચોરસ અને વર્તુળ દોરવા માટેનો પ્રોગ્રામ લખો.}

\begin{solutionbox}

\begin{verbatim}
import turtle

def draw\_shapes():
    \# ટર્ટલ અને સ્ક્રીન બનાવો
    screen = turtle.Screen()
    screen.title("ટર્ટલ સાથે આકારો દોરવા")
    screen.bgcolor("white")
    screen.setup(800, 600)
    
    \# ટર્ટલ બનાવો
    pen = turtle.Turtle()
    pen.speed(3)
    pen.color("blue")
    
    \# ચોરસ દોરો
    pen.penup()
    pen.goto({-}200, 100)
    pen.pendown()
    pen.write("ચોરસ", font=("Arial", 12, "bold"))
    pen.goto({-}200, 50)
    
    for i in range(4):
        pen.forward(80)
        pen.right(90)
    
    \# લંબચોરસ દોરો
    pen.penup()
    pen.goto(0, 100)
    pen.pendown()
    pen.color("red")
    pen.write("લંબચોરસ", font=("Arial", 12, "bold"))
    pen.goto(0, 50)
    
    for i in range(2):
        pen.forward(120)  \# લંબાઈ
        pen.right(90)
        pen.forward(60)   \# પહોળાઈ
        pen.right(90)
    
    \# વર્તુળ દોરો
    pen.penup()
    pen.goto(200, 100)
    pen.pendown()
    pen.color("green")
    pen.write("વર્તુળ", font=("Arial", 12, "bold"))
    pen.goto(200, 50)
    
    pen.circle(40)  \# Radius = 40
    
    \# ટર્ટલ છુપાવો અને વિન્ડો ખુલ્લી રાખો
    pen.hideturtle()
    screen.exitonclick()

\# દરેક આકાર માટે વૈકલ્પિક ફંક્શન
def draw\_square(turtle\_obj, size):
    """આપેલા સાઇઝ સાથે ચોરસ દોરો"""
    for \_ in range(4):
        turtle\_obj.forward(size)
        turtle\_obj.right(90)

def draw\_rectangle(turtle\_obj, width, height):
    """આપેલા પરિમાણો સાથે લંબચોરસ દોરો"""
    for \_ in range(2):
        turtle\_obj.forward(width)
        turtle\_obj.right(90)
        turtle\_obj.forward(height)
        turtle\_obj.right(90)

def draw\_circle(turtle\_obj, radius):
    """આપેલા radius સાથે વર્તુળ દોરો"""
    turtle\_obj.circle(radius)

\# મુખ્ય પ્રોગ્રામ ચલાવો
draw\_shapes()
\end{verbatim}

\begin{itemize}
\tightlist
\item
  \textbf{ચોરસ}: 90^\circ ફેરવણું સાથે 4 સમાન બાજુઓ
\item
  \textbf{લંબચોરસ}: સમાન બાજુઓની 2 જોડી
\item
  \textbf{વર્તુળ}: radius સાથે બિલ્ટ-ઇન circle() મેથડ
\end{itemize}

\end{solutionbox}
\begin{mnemonicbox}
``ચોરસ: 4 સમાન બાજુ, લંબચોરસ: 2 જોડી, વર્તુળ: radius
મેથડ''

\end{mnemonicbox}
\begin{center}\rule{0.5\linewidth}{0.5pt}\end{center}

\subsection*{પ્રશ્ન 5(અ) અથવા [3
ગુણ]}\label{uxaaauxab0uxab6uxaa8-5uxa85-uxa85uxaa5uxab5-3-uxa97uxaa3}

\textbf{ટર્ટલમાં પેન કમાન્ડના વિવિધ પ્રકારો કયા છે? તે બધાને સમજાવો.}

\begin{solutionbox}

\textbf{પેન કન્ટ્રોલ કમાન્ડ્સ:}

{\def\LTcaptype{none} % do not increment counter
\begin{longtable}[]{@{}lll@{}}
\toprule\noalign{}
કમાન્ડ & હેતુ & ઉદાહરણ \\
\midrule\noalign{}
\endhead
\bottomrule\noalign{}
\endlastfoot
\textbf{penup()} & પેન ઉઠાવો (દોરવું નહીં) & turtle.penup() \\
\textbf{pendown()} & પેન નીચે મૂકો (દોરવાનું શરૂ કરો) & turtle.pendown() \\
\textbf{pensize(width)} & પેનની જાડાઈ સેટ કરો & turtle.pensize(5) \\
\textbf{pencolor(color)} & પેનનો રંગ સેટ કરો & turtle.pencolor(``red'') \\
\textbf{fillcolor(color)} & ભરવાનો રંગ સેટ કરો &
turtle.fillcolor(``blue'') \\
\textbf{begin\_fill()} & આકાર ભરવાનું શરૂ કરો & turtle.begin\_fill() \\
\textbf{end\_fill()} & આકાર ભરવાનું બંધ કરો & turtle.end\_fill() \\
\end{longtable}
}

\begin{verbatim}
import turtle

t = turtle.Turtle()

\# પેન કન્ટ્રોલ
t.penup()           \# પેન ઉઠાવો
t.goto(50, 50)      \# દોર્યા વિના ખસો
t.pendown()         \# પેન નીચે મૂકો
t.pensize(3)        \# જાડાઈ સેટ કરો
t.pencolor("red")   \# રંગ સેટ કરો
\end{verbatim}

\end{solutionbox}
\begin{mnemonicbox}
``Up-Down દોરવાનું કન્ટ્રોલ કરે, Size-Color દેખાવ કન્ટ્રોલ
કરે''

\end{mnemonicbox}
\begin{center}\rule{0.5\linewidth}{0.5pt}\end{center}

\subsection*{પ્રશ્ન 5(બ) અથવા [4
ગુણ]}\label{uxaaauxab0uxab6uxaa8-5uxaac-uxa85uxaa5uxab5-4-uxa97uxaa3}

\textbf{ટર્ટલનો ઉપયોગ કરીને વર્તુળ અને સ્ટારના આકાર દોરો અને તેમને લાલ રંગથી ભરો.}

\begin{solutionbox}

\begin{verbatim}
import turtle

def draw\_filled\_shapes():
    \# સ્ક્રીન સેટઅપ
    screen = turtle.Screen()
    screen.bgcolor("white")
    screen.title("ભરેલા વર્તુળ અને સ્ટાર")
    
    \# ટર્ટલ બનાવો
    artist = turtle.Turtle()
    artist.speed(5)
    
    \# ભરેલા વર્તુળ દોરો
    artist.penup()
    artist.goto({-}150, 0)
    artist.pendown()
    
    \# વર્તુળ માટે રંગો સેટ કરો
    artist.color("red", "red")  \# pen color, fill color
    artist.begin\_fill()
    artist.circle(50)
    artist.end\_fill()
    
    \# ભરેલા સ્ટાર દોરો
    artist.penup()
    artist.goto(100, 0)
    artist.pendown()
    
    \# સ્ટાર માટે રંગો સેટ કરો
    artist.color("red", "red")
    artist.begin\_fill()
    
    \# 5{-પોઇન્ટેડ સ્ટાર દોરો}
    for i in range(5):
        artist.forward(100)
        artist.right(144)
    
    artist.end\_fill()
    
    \# લેબલ્સ ઉમેરો
    artist.penup()
    artist.goto({-}180, {-}80)
    artist.color("black")
    artist.write("ભરેલા વર્તુળ", font=("Arial", 12, "bold"))
    
    artist.goto(70, {-}80)
    artist.write("ભરેલા સ્ટાર", font=("Arial", 12, "bold"))
    
    \# ટર્ટલ છુપાવો
    artist.hideturtle()
    screen.exitonclick()

\# પ્રોગ્રામ ચલાવો
draw\_filled\_shapes()
\end{verbatim}

\textbf{મુખ્ય મુદ્દાઓ:}

\begin{itemize}
\tightlist
\item
  \textbf{begin\_fill()}: આકાર ભરવાનું શરૂ કરો
\item
  \textbf{end\_fill()}: ભરવાનું પૂર્ણ કરો
\item
  \textbf{color()}: pen અને fill બંને રંગો સેટ કરો
\item
  \textbf{સ્ટાર angle}: 5-પોઇન્ટેડ સ્ટાર માટે 144^\circ
\end{itemize}

\end{solutionbox}
\begin{mnemonicbox}
``Begin fill, આકાર દોરો, End fill = ભરેલા આકાર''

\end{mnemonicbox}
\begin{center}\rule{0.5\linewidth}{0.5pt}\end{center}

\subsection*{પ્રશ્ન 5(ક) અથવા [7
ગુણ]}\label{uxaaauxab0uxab6uxaa8-5uxa95-uxa85uxaa5uxab5-7-uxa97uxaa3}

\textbf{ટર્ટલનો ઉપયોગ કરીને ભારતનો ઝંડો દોરવા માટેનો પ્રોગ્રામ લખો.}

\begin{solutionbox}

\begin{verbatim}
import turtle

def draw\_indian\_flag():
    \# સ્ક્રીન બનાવો
    screen = turtle.Screen()
    screen.bgcolor("white")
    screen.title("ભારતનો ઝંડો")
    screen.setup(800, 600)
    
    \# ટર્ટલ બનાવો
    flag = turtle.Turtle()
    flag.speed(5)
    flag.pensize(2)
    
    \# ઝંડાના પરિમાણો
    flag\_width = 300
    flag\_height = 200
    
    \# શરૂઆતની પોઝિશન
    start\_x = {-}150
    start\_y = 100
    
    \# ઝંડાનો દંડ દોરો
    flag.penup()
    flag.goto(start\_x {-} 20, start\_y + 50)
    flag.pendown()
    flag.color("brown")
    flag.pensize(8)
    flag.setheading(270)  \# નીચે તરફ નિર્દેશ કરો
    flag.forward(400)
    
    \# પેન રીસેટ કરો
    flag.pensize(2)
    flag.color("black")
    
    \# કેસરી લંબચોરસ (ઉપર)
    flag.penup()
    flag.goto(start\_x, start\_y)
    flag.pendown()
    flag.color("orange", "orange")
    flag.begin\_fill()
    flag.setheading(0)
    
    for \_ in range(2):
        flag.forward(flag\_width)
        flag.right(90)
        flag.forward(flag\_height // 3)
        flag.right(90)
    flag.end\_fill()
    
    \# સફેદ લંબચોરસ (મધ્ય)
    flag.penup()
    flag.goto(start\_x, start\_y {-} flag\_height // 3)
    flag.pendown()
    flag.color("black", "white")
    flag.begin\_fill()
    
    for \_ in range(2):
        flag.forward(flag\_width)
        flag.right(90)
        flag.forward(flag\_height // 3)
        flag.right(90)
    flag.end\_fill()
    
    \# લીલો લંબચોરસ (નીચે)
    flag.penup()
    flag.goto(start\_x, start\_y {-} 2 * flag\_height // 3)
    flag.pendown()
    flag.color("green", "green")
    flag.begin\_fill()
    
    for \_ in range(2):
        flag.forward(flag\_width)
        flag.right(90)
        flag.forward(flag\_height // 3)
        flag.right(90)
    flag.end\_fill()
    
    \# અશોક ચક્ર (ચક્ર) દોરો
    chakra\_center\_x = start\_x + flag\_width // 2
    chakra\_center\_y = start\_y {-} flag\_height // 2
    
    flag.penup()
    flag.goto(chakra\_center\_x, chakra\_center\_y {-} 30)
    flag.pendown()
    flag.color("navy")
    flag.pensize(3)
    
    \# બાહ્ય વર્તુળ દોરો
    flag.circle(30)
    
    \# તીલીઓ દોરો
    flag.penup()
    flag.goto(chakra\_center\_x, chakra\_center\_y)
    flag.pendown()
    
    for i in range(24):  \# અશોક ચક્રમાં 24 તીલીઓ
        flag.setheading(i * 15)  \# 360/24 = 15 ડિગ્રી
        flag.forward(30)
        flag.backward(30)
    
    \# અંદરનું વર્તુળ દોરો
    flag.penup()
    flag.goto(chakra\_center\_x, chakra\_center\_y {-} 5)
    flag.pendown()
    flag.circle(5)
    
    \# શીર્ષક ઉમેરો
    flag.penup()
    flag.goto({-}100, 200)
    flag.color("black")
    flag.write("ભારતનો ઝંડો", font=("Arial", 16, "bold"))
    
    \# ટર્ટલ છુપાવો
    flag.hideturtle()
    screen.exitonclick()

\# પ્રોગ્રામ ચલાવો
draw\_indian\_flag()
\end{verbatim}

\textbf{ઝંડાના ઘટકો:}

\begin{itemize}
\tightlist
\item
  \textbf{કેસરી}: બહાદુરી અને બલિદાન (ઉપર)
\item
  \textbf{સફેદ}: સત્ય અને શાંતિ (મધ્ય)
\item
  \textbf{લીલો}: શ્રદ્ધા અને વીરતા (નીચે)
\item
  \textbf{અશોક ચક્ર}: ઘેરા વાદળી રંગમાં 24-તીલીવાળું ચક્ર
\end{itemize}

\end{solutionbox}
\begin{mnemonicbox}
``કેસરી-સફેદ-લીલી પટ્ટીઓ 24-તીલીવાળા ચક્ર સાથે''

\end{mnemonicbox}

\end{document}
