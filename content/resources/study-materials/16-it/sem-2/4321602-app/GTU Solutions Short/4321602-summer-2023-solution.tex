\documentclass{article}

% content/resources/templates/preamble.tex
\usepackage[margin=0.6in]{geometry}
\author{Milav Dabgar}
\usepackage{amsmath,amssymb,amsthm}
\usepackage{booktabs}
\usepackage{multirow}
\usepackage{xcolor}
\usepackage{tcolorbox}
\tcbuselibrary{breakable,skins}
\usepackage[colorlinks=true,linkcolor=blue]{hyperref}
\usepackage{titlesec}
\usepackage{enumitem}
\usepackage{tikz}
\usepackage{pgfplots}
\usepackage{circuitikz}
\usepackage[version=4]{mhchem}
\usepackage{longtable}
\usepackage{array}
\usepackage{float}
\usepackage{caption}
\usepackage{listings}

\lstset{
  basicstyle=\small\ttfamily,
  breaklines=true,
  breakatwhitespace=false,
  postbreak=\mbox{\textcolor{red}{$\hookrightarrow$}\space},
  float=false,
  numbers=left,
  numberstyle=\tiny\color{gray},
  numbersep=10pt,
  xleftmargin=2em,
  keywordstyle=\color{blue},
  commentstyle=\color{green!60!black},
  stringstyle=\color{purple},
  backgroundcolor=\color{gray!5},
  showstringspaces=false,
  tabsize=2,
  captionpos=b,
  keepspaces=true,
  columns=flexible
}

\pgfplotsset{compat=1.18}
\usetikzlibrary{shapes,arrows,positioning,calc,patterns,decorations.pathmorphing,decorations.markings,arrows.meta}

% Color scheme
\definecolor{headcolor}{RGB}{0,102,204}
\definecolor{keycolor}{RGB}{220,20,60}
\definecolor{solutioncolor}{RGB}{34,139,34}
\definecolor{mnemoniccolor}{RGB}{148,0,211}
\definecolor{codecolor}{RGB}{0,0,100}

% Spacing
\setlength{\parskip}{3pt}
\setlist[itemize]{nosep}
\setlist[enumerate]{nosep}

% Title formatting
\titleformat{\section}{\Large\bfseries\color{headcolor}}{\thesection}{1em}{}
\titleformat{\subsection}{\large\bfseries\color{headcolor}}{\thesubsection}{1em}{}

% Pandoc tightlist compatibility
\providecommand{\tightlist}{%
  \setlength{\itemsep}{0pt}\setlength{\parskip}{0pt}}

% Pandoc longtable compatibility
\newcounter{none}
\def\thenone{}


% content/resources/templates/english-boxes.tex

% Custom environments
\newtcolorbox{solutionbox}{
 breakable,
 enhanced,
 colback=solutioncolor!5!white,
 colframe=solutioncolor!75!black,
 fonttitle=\bfseries,
 title=Solution
}

\newtcolorbox{solutionboxnobreak}{
 colback=solutioncolor!5!white,
 colframe=solutioncolor!75!black,
 fonttitle=\bfseries,
 title=Solution
}

\newtcolorbox{keyformula}{
 breakable,
 enhanced,
 colback=keycolor!5!white,
 colframe=keycolor!75!black,
 fonttitle=\bfseries,
 title=Key Formula
}

\newtcolorbox{mnemonicboxenv}{
 breakable,
 enhanced,
 colback=mnemoniccolor!5!white,
 colframe=mnemoniccolor!75!black,
 fonttitle=\bfseries,
 title=Mnemonic
}

\newcommand{\mnemonicbox}[1]{%
  \begin{mnemonicboxenv}
    #1
  \end{mnemonicboxenv}
}


% Custom commands for GTU solutions
% This file defines semantic commands for consistent formatting

% Question command with automatic formatting
\newcommand{\question}[2]{%
  \section*{Question #1}%
  \textbf{#2}%
}

% OR question variant
\newcommand{\questionor}[2]{%
  \section*{Question #1 OR}%
  \textbf{#2}%
}

% Proper table environment with caption
\newenvironment{answertable}[1]{%
  \begin{table}[htbp]
  \centering
  \caption{#1}
}{%
  \end{table}
}

% Proper figure environment for diagrams
\newenvironment{answerdiagram}[1]{%
  \begin{figure}[htbp]
  \centering
  \caption{#1}
}{%
  \end{figure}
}

% Semantic markup for key terms
\newcommand{\keyword}[1]{\textbf{#1}}
\newcommand{\code}[1]{\texttt{#1}}
\newcommand{\classname}[1]{\texttt{#1}}
\newcommand{\methodname}[1]{\texttt{#1}}

% Proper quotation marks
\newcommand{\mnemonic}[1]{``#1''}


\title{Advanced Python Programming (4321602) - Summer 2023 Solution}
\date{July 31, 2023}

\begin{document}
\maketitle

\section*{Question 1}

\questionmarks{1(a)}{3}{What is List? Write its characteristics and usage in Python.}
\begin{solutionbox}
    A \textbf{List} is an ordered collection of items (elements) that allows storing multiple values in a single variable. Lists are mutable and allow duplicate elements.

    \textbf{Characteristics:}
    \begin{center}
        \begin{tabulary}{\linewidth}{L L}
            \toprule
            \textbf{Feature} & \textbf{Description} \\
            \midrule
            \textbf{Ordered} & Elements have a defined order \\
            \textbf{Mutable} & Can be changed after creation \\
            \textbf{Indexed} & Accessed using index [0,1,2...] \\
            \textbf{Duplicates} & Allow duplicate values \\
            \bottomrule
        \end{tabulary}
    \end{center}

    \textbf{Usage in Python:}
    \begin{itemize}
        \item \textbf{Data Storage}: Storing related items.
        \item \textbf{Dynamic Arrays}: Resizable collection during runtime.
        \item \textbf{Iteration}: Easy looping through elements.
    \end{itemize}
    \begin{mnemonicbox}OMID - Ordered, Mutable, Indexed, Duplicates\end{mnemonicbox}
\end{solutionbox}

\questionmarks{1(b)}{4}{Explain String built-in functions in Python.}
\begin{solutionbox}
    String built-in functions help in efficiently manipulating and processing text data in Python programs.

    \textbf{Common String Functions:}
    \begin{center}
        \begin{tabulary}{\linewidth}{L L L}
            \toprule
            \textbf{Function} & \textbf{Purpose} & \textbf{Example} \\
            \midrule
            \textbf{upper()} & Convert to uppercase & \code{"hello".upper()} $\to$ "HELLO" \\
            \textbf{lower()} & Convert to lowercase & \code{"WORLD".lower()} $\to$ "world" \\
            \textbf{strip()} & Remove whitespace & \code{" hi ".strip()} $\to$ "hi" \\
            \textbf{split()} & Split into list & \code{"a,b".split(",")} $\to$ ['a','b'] \\
            \textbf{replace()} & Replace substring & \code{"cat".replace("c","b")} $\to$ "bat" \\
            \textbf{find()} & Find substring pos & \code{"hello".find("e")} $\to$ 1 \\
            \bottomrule
        \end{tabulary}
    \end{center}

    \textbf{Key Points:}
    \begin{itemize}
        \item \textbf{Immutable}: Original string remains unchanged.
        \item \textbf{Return Values}: Functions return new strings.
        \item \textbf{Case Sensitive}: Functions respect case.
    \end{itemize}
    \begin{mnemonicbox}ULSR-FR - Upper, Lower, Strip, Replace, Find, Replace\end{mnemonicbox}
\end{solutionbox}

\questionmarks{1(c)}{7}{Write how to add, remove element from set. Explain how POP differs from remove.}
\begin{solutionbox}
    \textbf{Sets} are unordered collections of unique elements.

    \textbf{Set Operations:}
    \begin{center}
        \begin{tabulary}{\linewidth}{L L L L}
            \toprule
            \textbf{Operation} & \textbf{Method} & \textbf{Syntax} & \textbf{Example} \\
            \midrule
            \textbf{Add} & add() & \code{set.add(e)} & \code{s.add(5)} \\
            \textbf{Remove} & remove() & \code{set.remove(e)} & \code{s.remove(3)} \\
            \textbf{Safe Remove} & discard() & \code{set.discard(e)} & \code{s.discard(7)} \\
            \textbf{Pop} & pop() & \code{set.pop()} & \code{s.pop()} \\
            \bottomrule
        \end{tabulary}
    \end{center}

    \textbf{Code Example:}
    \begin{lstlisting}[language=Python]
my_set = {1, 2, 3}
my_set.add(5)        # Add
my_set.remove(2)     # Remove specific
element = my_set.pop() # Remove random
    \end{lstlisting}

    \textbf{Difference POP vs REMOVE:}
    \begin{center}
        \begin{tabulary}{\linewidth}{L L L}
            \toprule
            \textbf{Aspect} & \textbf{pop()} & \textbf{remove()} \\
            \midrule
            \textbf{Target} & Random element & Specific element \\
            \textbf{Parameter} & No parameter & Requires element value \\
            \textbf{Return} & Returns removed element & Returns None \\
            \textbf{Error} & Error if set empty & Error if element not found \\
            \bottomrule
        \end{tabulary}
    \end{center}
    \begin{mnemonicbox}PRRE - Pop Random, Remove Exact\end{mnemonicbox}
\end{solutionbox}

\questionmarks{1(c) OR}{7}{List out built-in Dictionary functions. Write a program to demonstrate dictionary functions and operations.}
\begin{solutionbox}
    \textbf{Dictionary Functions:}
    \begin{center}
        \begin{tabulary}{\linewidth}{L L L}
            \toprule
            \textbf{Function} & \textbf{Purpose} & \textbf{Returns} \\
            \midrule
            \textbf{keys()} & Get all keys & dict\_keys object \\
            \textbf{values()} & Get all values & dict\_values object \\
            \textbf{items()} & Get key-value pairs & dict\_items object \\
            \textbf{get()} & Safe value retrieval & Value or None \\
            \textbf{pop()} & Remove and return value & Removed value \\
            \textbf{clear()} & Remove all items & None \\
            \textbf{update()} & Merge dictionaries & None \\
            \bottomrule
        \end{tabulary}
    \end{center}

    \textbf{Program Demonstration:}
    \begin{lstlisting}[language=Python]
# Dictionary Creation
student = {
    'name': 'John Doe',
    'age': 20,
    'course': 'IT'
}

# Demonstrating Functions
print("Keys:", list(student.keys()))
print("Values:", list(student.values()))

# Get specific value safely
grade = student.get('grade', 'Not Assigned')
print(f"Grade: {grade}")

# Update dictionary
student.update({'grade': 'A', 'city': 'Ahmedabad'})

# Remove item
age = student.pop('age')
print(f"Removed Age: {age}")

# Iterating
print("\nStudent Details:")
for key, value in student.items():
    print(f"{key}: {value}")
    \end{lstlisting}
    \begin{mnemonicbox}KVIGPCU - Keys, Values, Items, Get, Pop, Clear, Update\end{mnemonicbox}
\end{solutionbox}

\section*{Question 2}

\questionmarks{2(a)}{3}{Define Tuple and how it is created in Python.}
\begin{solutionbox}
    A \textbf{Tuple} is an ordered collection which is immutable (unchangeable).

    \textbf{Tuple Creation Methods:}
    \begin{center}
        \begin{tabulary}{\linewidth}{L L L}
            \toprule
            \textbf{Method} & \textbf{Syntax} & \textbf{Example} \\
            \midrule
            \textbf{Parentheses} & (item1, item2) & (1, 2, 3) \\
            \textbf{No Parentheses} & item1, item2 & 1, 2, 3 \\
            \textbf{Single Item} & (item,) & (5,) \\
            \textbf{Empty Tuple} & () & () \\
            \bottomrule
        \end{tabulary}
    \end{center}
    \begin{mnemonicbox}IOI - Immutable, Ordered, Indexed\end{mnemonicbox}
\end{solutionbox}

\questionmarks{2(b)}{4}{Explain advantages of Module.}
\begin{solutionbox}
    \textbf{Modules} are Python files containing functions, classes, and variables that can be imported.

    \textbf{Advantages:}
    \begin{center}
        \begin{tabulary}{\linewidth}{L L}
            \toprule
            \textbf{Advantage} & \textbf{Benefit} \\
            \midrule
            \textbf{Reusability} & Write once, use everywhere \\
            \textbf{Organization} & Break code into logical units \\
            \textbf{Namespace} & Avoids naming conflicts \\
            \textbf{Maintainability} & Easier to debug and update \\
            \bottomrule
        \end{tabulary}
    \end{center}
    \begin{mnemonicbox}RONM - Reusability, Organization, Namespace, Maintainability\end{mnemonicbox}
\end{solutionbox}

\questionmarks{2(c)}{7}{List out the steps to create a user defined package with proper example.}
\begin{solutionbox}
    A \textbf{package} is a directory containing multiple modules with a special \code{\_\_init\_\_.py} file.

    \textbf{Steps to Create Package:}
    \begin{center}
        \begin{tikzpicture}[node distance=2cm, auto]
            \node [gtu state, text width=3cm] (create) {Create Package Directory};
            \node [gtu state, right of=create, xshift=2cm, text width=3cm] (init) {Create \code{\_\_init\_\_.py}};
            \node [gtu state, right of=init, xshift=2cm, text width=3cm] (modules) {Create Module Files};
            \node [gtu state, below of=create, yshift=-1cm, text width=3cm] (funcs) {Write Functions};
            \node [gtu state, right of=funcs, xshift=2cm, text width=3cm] (use) {Import and Use};

            \draw [gtu arrow] (create) -- (init);
            \draw [gtu arrow] (init) -- (modules);
            \draw [gtu arrow] (modules) -- (funcs);
            \draw [gtu arrow] (funcs) -- (use);
        \end{tikzpicture}
    \end{center}

    \textbf{Step-by-Step Implementation:}
    \begin{enumerate}
        \item \textbf{Create Directory}: \code{mkdir mathtools}
        \item \textbf{Create \code{\_\_init\_\_.py}}:
        \begin{lstlisting}[language=Python]
# mathtools/__init__.py
print("MathTools package loaded")
        \end{lstlisting}
        \item \textbf{Create Module (basic.py)}:
        \begin{lstlisting}[language=Python]
# mathtools/basic.py
def add(a, b):
    return a + b
        \end{lstlisting}
        \item \textbf{Use Package}:
        \begin{lstlisting}[language=Python]
import mathtools.basic
result = mathtools.basic.add(5, 3)
print(result)  # Output: 8
        \end{lstlisting}
    \end{enumerate}

    \textbf{Key Requirements:}
    \begin{itemize}
        \item \textbf{Directory}: Package must be a directory.
        \item \textbf{\code{\_\_init\_\_.py}}: Required file (can be empty).
        \item \textbf{Import Path}: Python must find package in path.
    \end{itemize}
    \begin{mnemonicbox}DDMFU - Directory, Dunder-init, Modules, Functions, Use\end{mnemonicbox}
\end{solutionbox}

\questionmarks{2(a) OR}{3}{Write difference between Tuple and List.}
\begin{solutionbox}
    \textbf{Comparison:}
    \begin{center}
        \begin{tabulary}{\linewidth}{L L L}
            \toprule
            \textbf{Feature} & \textbf{Tuple} & \textbf{List} \\
            \midrule
            \textbf{Mutability} & Immutable (Fixed) & Mutable (Changeable) \\
            \textbf{Syntax} & Parentheses (1, 2) & Brackets [1, 2] \\
            \textbf{Performance} & Faster & Slower \\
            \textbf{Methods} & Limited & Many methods \\
            \textbf{Memory} & Less memory & More memory \\
            \bottomrule
        \end{tabulary}
    \end{center}
    \begin{mnemonicbox}TIF-LIM - Tuple Immutable Fixed, List Mutable Dynamic\end{mnemonicbox}
\end{solutionbox}

\questionmarks{2(b) OR}{4}{Explain concept of intra-package reference in Python.}
\begin{solutionbox}
    \textbf{Intra-package references} allow modules within the same package to refer to each other using relative imports.

    \textbf{Import Types:}
    \begin{center}
        \begin{tabulary}{\linewidth}{L L L}
            \toprule
            \textbf{Type} & \textbf{Syntax} & \textbf{Usage} \\
            \midrule
            \textbf{Absolute} & \code{from pkg.mod import fn} & Full path from root \\
            \textbf{Relative} & \code{from .mod import fn} & Same package \\
            \textbf{Parent} & \code{from ..mod import fn} & Parent package \\
            \bottomrule
        \end{tabulary}
    \end{center}

    \textbf{Example Structure:}
    \begin{lstlisting}[language=Python]
mypackage/
    __init__.py
    module1.py
    subpackage/
        __init__.py
        module2.py  # can import ..module1
    \end{lstlisting}
    \begin{mnemonicbox}RAP - Relative, Absolute, Parent imports\end{mnemonicbox}
\end{solutionbox}

\questionmarks{2(c) OR}{7}{What is module? Write a program to create a module to find area and circumference of circle. Import the module into program and call functions.}
\begin{solutionbox}
    \textbf{Module} is a Python file containing definitions and statements.

    \textbf{1. Circle Module (circle.py):}
    \begin{lstlisting}[language=Python]
import math

def area(radius):
    """Calculate area of circle"""
    return math.pi * radius * radius

def circumference(radius):
    """Calculate circumference of circle"""
    return 2 * math.pi * radius
    \end{lstlisting}

    \textbf{2. Main Program (main.py):}
    \begin{lstlisting}[language=Python]
import circle

# Get input
r = float(input("Enter radius: "))

# Call module functions
a = circle.area(r)
c = circle.circumference(r)

# Display results
print(f"Area: {a:.2f}")
print(f"Circumference: {c:.2f}")
    \end{lstlisting}
    \begin{mnemonicbox}IRUD - Import, Reuse, Use, Debug\end{mnemonicbox}
\end{solutionbox}

\section*{Question 3}

\questionmarks{3(a)}{3}{Explain types of errors in Python.}
\begin{solutionbox}
    \textbf{Errors} are issues in code that prevent execution or cause incorrect results.

    \textbf{Types of Errors:}
    \begin{center}
        \begin{tabulary}{\linewidth}{L L L}
            \toprule
            \textbf{Error Type} & \textbf{Description} & \textbf{Example} \\
            \midrule
            \textbf{Syntax Error} & Violation of language rules & Missing colon, typo \\
            \textbf{Runtime Error} & Error during execution & Division by zero \\
            \textbf{Logical Error} & Program runs but wrong output & Wrong formula \\
            \bottomrule
        \end{tabulary}
    \end{center}
    \begin{mnemonicbox}SRL - Syntax, Runtime, Logical\end{mnemonicbox}
\end{solutionbox}

\questionmarks{3(b)}{4}{Explain structure of try except.}
\begin{solutionbox}
    The \textbf{try-except} structure is used to handle runtime errors gracefully without crashing the program.

    \textbf{Basic Structure:}
    \begin{center}
        \begin{tikzpicture}[node distance=2.5cm, auto]
            \node [gtu state] (try) {try block};
            \node [gtu state, right of=try, xshift=2cm] (exec) {Code execution};
            \node [gtu decision, right of=exec, xshift=2cm] (error) {Error?};
            \node [gtu state, below of=error, yshift=-1cm] (except) {except block};
            \node [gtu state, right of=error, xshift=3cm] (cont) {Continue};

            \draw [gtu arrow] (try) -- (exec);
            \draw [gtu arrow] (exec) -- (error);
            \draw [gtu arrow] (error) -- node [near start] {No} (cont);
            \draw [gtu arrow] (error) -- node [near start] {Yes} (except);
            \draw [gtu arrow] (except) -| (cont);
        \end{tikzpicture}
    \end{center}

    \textbf{Syntax:}
    \begin{lstlisting}[language=Python]
try:
    # Code that might cause error
    risky_code()
except SomeError:
    # Code to handle error
    handle_error()
else:
    # Code if no error occurs
    success_code()
finally:
    # Code that always runs
    cleanup_code()
    \end{lstlisting}
    \begin{mnemonicbox}TEEF - Try, Except, Else, Finally\end{mnemonicbox}
\end{solutionbox}

\questionmarks{3(c)}{7}{Write a function marks\_result which takes two arguments of marks of English and Maths, generates error if any of the argument is less than 0.}
\begin{solutionbox}
    \textbf{Problem:} Create a custom exception handling scenario for mark validation.

    \textbf{Code Implementation:}
    \begin{lstlisting}[language=Python]
class InvalidMarksError(Exception):
    """Custom exception for invalid marks"""
    def __init__(self, subject, marks):
        super().__init__(f"Invalid {subject} marks: {marks}. Cannot be negative.")

def marks_result(english, maths):
    """Calculate result with validation"""
    # Validation logic
    if english < 0:
        raise InvalidMarksError("English", english)
    if maths < 0:
        raise InvalidMarksError("Mathematics", maths)
    
    # Also valid to check > 100
    if english > 100:
        raise InvalidMarksError("English", english)
    if maths > 100:
        raise InvalidMarksError("Mathematics", maths)

    total = english + maths
    percentage = (total / 200) * 100
    
    if percentage >= 50:
        status = 'Pass'
    else:
        status = 'Fail'
        
    return {
        'total': total, 
        'percentage': percentage, 
        'status': status
    }

# Testing
try:
    print(marks_result(80, 90))
    print(marks_result(80, -10))  # Will raise error
except InvalidMarksError as e:
    print(f"Error: {e}")
    \end{lstlisting}
    \begin{mnemonicbox}CVIR - Custom, Validate, Interactive, Robust\end{mnemonicbox}
\end{solutionbox}

\questionmarks{3(a) OR}{3}{List out built-in exceptions in Python (Any five).}
\begin{solutionbox}
    \textbf{Built-in Exceptions:}
    \begin{center}
        \begin{tabulary}{\linewidth}{L L L}
            \toprule
            \textbf{Exception} & \textbf{Cause} & \textbf{Example} \\
            \midrule
            \textbf{ValueError} & Invalid value type & \code{int("abc")} \\
            \textbf{TypeError} & Invalid operation/type & \code{"5"+5} \\
            \textbf{IndexError} & Index out of range & \code{list[10]} \\
            \textbf{KeyError} & Key not found & \code{dict["x"]} \\
            \textbf{ZeroDivisionError} & Division by zero & \code{10/0} \\
            \bottomrule
        \end{tabulary}
    \end{center}
    \begin{mnemonicbox}VTIKZ - ValueError, TypeError, IndexError, KeyError, ZeroDivisionError\end{mnemonicbox}
\end{solutionbox}

\questionmarks{3(b) OR}{4}{Write points on finally and explain with example.}
\begin{solutionbox}
    \textbf{Finally Block:} Code block that executes regardless of whether an exception occurs or not.

    \textbf{Characteristics:}
    \begin{itemize}
        \item \textbf{Always Executes}: Runs if try succeeds or fails.
        \item \textbf{Cleanup}: essential for closing files, network connections.
        \item \textbf{Placement}: Must be the last block in try-except structure.
    \end{itemize}

    \textbf{Example:}
    \begin{lstlisting}[language=Python]
try:
    file = open("data.txt", "r")
    # File operations
except FileNotFoundError:
    print("File not found error")
finally:
    print("Cleanup initiated")
    # Close file if it was opened
    if 'file' in locals():
        file.close()
    \end{lstlisting}
    \begin{mnemonicbox}ARGC - Always Runs, Resource Cleanup\end{mnemonicbox}
\end{solutionbox}

\questionmarks{3(c) OR}{7}{Write a program to catch divide by zero exception with finally clause.}
\begin{solutionbox}
    \textbf{Program:}
    \begin{lstlisting}[language=Python]
def safe_divide(a, b):
    try:
        print(f"Attempting to divide {a} by {b}")
        result = a / b
        print(f"Result: {result}")
    except ZeroDivisionError:
        print("Error: Cannot divide by zero!")
    except TypeError:
        print("Error: Inputs must be numbers!")
    else:
        print("Division successful")
    finally:
        print("Operation completed\n")

# Test Cases
safe_divide(10, 2)   # Successful
safe_divide(5, 0)    # ZeroDivisionError
safe_divide(10, "a") # TypeError
    \end{lstlisting}
    \begin{mnemonicbox}CFLIS - Comprehensive, Finally, Logging, Interactive, Statistics\end{mnemonicbox}
\end{solutionbox}

\section*{Question 4}

\questionmarks{4(a)}{3}{What is File Handling? List out File Handling Operations.}
\begin{solutionbox}
    \textbf{File Handling} is the mechanism to read from and write to files on the disk using Python.

    \textbf{Operations:}
    \begin{center}
        \begin{tabulary}{\linewidth}{L L L}
            \toprule
            \textbf{Operation} & \textbf{Purpose} & \textbf{Method} \\
            \midrule
            \textbf{Open} & Open file in mode & \code{open()} \\
            \textbf{Read} & Read content & \code{read()} \\
            \textbf{Write} & Write content & \code{write()} \\
            \textbf{Close} & Close file & \code{close()} \\
            \textbf{Seek} & Move cursor & \code{seek()} \\
            \bottomrule
        \end{tabulary}
    \end{center}
    \begin{mnemonicbox}ORWCST - Open, Read, Write, Close, Seek, Tell\end{mnemonicbox}
\end{solutionbox}

\questionmarks{4(b)}{4}{Explain Object Serialization.}
\begin{solutionbox}
    \textbf{Object Serialization} is the process of converting a Python object structure into a byte stream to store it or transmit it.

    \textbf{Implementation:}
    \begin{itemize}
        \item \textbf{Module}: \code{pickle} module is used.
        \item \textbf{Pickling}: Converting object to bytes (\code{dump}).
        \item \textbf{Unpickling}: Converting bytes back to object (\code{load}).
    \end{itemize}

    \textbf{Example:}
    \begin{lstlisting}[language=Python]
import pickle
data = {'a': 1, 'b': 2}
# Serialize
with open('data.pkl', 'wb') as f:
    pickle.dump(data, f)
# Deserialize
with open('data.pkl', 'rb') as f:
    loaded = pickle.load(f)
    \end{lstlisting}
    \begin{mnemonicbox}SPDT - Store, Persist, Data Transfer\end{mnemonicbox}
\end{solutionbox}

\questionmarks{4(c)}{7}{Write a program to count vowels stored in a file.}
\begin{solutionbox}
    \textbf{Program:}
    \begin{lstlisting}[language=Python]
def count_vowels(filename):
    vowels = 'aeiouAEIOU'
    count = 0
    try:
        with open(filename, 'r') as f:
            text = f.read()
            for char in text:
                if char in vowels:
                    count += 1
        print(f"Total characters: {len(text)}")
        print(f"Total Vowels: {count}")
    except FileNotFoundError:
        print("Error: File not found")

# Create test file
with open("test.txt", "w") as f:
    f.write("Hello World, Python is Awesome!")

count_vowels("test.txt")
    \end{lstlisting}
    \begin{mnemonicbox}FVESI - File Validation, Vowel Extraction, Statistics, Interactive\end{mnemonicbox}
\end{solutionbox}

\questionmarks{4(a) OR}{3}{How to open and close file? Give syntax.}
\begin{solutionbox}
    \textbf{Opening}: Uses \code{open()} function.
    \textbf{Closing}: Uses \code{close()} method.

    \textbf{Syntax and Modes:}
    \begin{itemize}
        \item 'r': Read (default)
        \item 'w': Write (overwrites)
        \item 'a': Append
    \end{itemize}

    \textbf{Code:}
    \begin{lstlisting}[language=Python]
# Manual Closing
f = open("file.txt", "mode")
# operations
f.close()

# Automatic Closing (Recommended)
with open("file.txt", "r") as f:
    data = f.read()
# Automatically closed here
    \end{lstlisting}
    \begin{mnemonicbox}ORWA - Open, Read, Write, Append modes\end{mnemonicbox}
\end{solutionbox}

\questionmarks{4(b) OR}{4}{What is Differentiate between Text file and Binary file?}
\begin{solutionbox}
    \textbf{Comparison:}
    \begin{center}
        \begin{tabulary}{\linewidth}{L L L}
            \toprule
            \textbf{Aspect} & \textbf{Text File} & \textbf{Binary File} \\
            \midrule
            \textbf{Content} & Human readable chars & Machine readable bytes \\
            \textbf{Mode} & 'r', 'w' & 'rb', 'wb' \\
            \textbf{Encoding} & ASCII/UTF-8 & None \\
            \textbf{Size} & Larger & Compact \\
            \bottomrule
        \end{tabulary}
    \end{center}
    \begin{mnemonicbox}TCEB - Text Character Encoding Bigger, Binary Compact Efficient\end{mnemonicbox}
\end{solutionbox}

\questionmarks{4(c) OR}{7}{Write a program to create a binary file to store Seat no and Name. Search any Seat no and display name if Seat No. found otherwise "Seat no not found".}
\begin{solutionbox}
    \textbf{Program:}
    \begin{lstlisting}[language=Python]
import pickle

def add_student(seat, name):
    record = {seat: name}
    with open("students.dat", "ab") as f:
        # Note: Appending pickle streams can be complex.
        # Ideally read all, update, write all.
        # Simplified for exam:
        pass

# Better approach: Manage dictionary
def manage_students():
    data = {}
    # Add records
    data[1] = "Ram"
    data[2] = "Shyam"
    
    # Save
    with open("students.dat", "wb") as f:
        pickle.dump(data, f)
        
    # Search
    search_seat = 1
    try:
        with open("students.dat", "rb") as f:
            loaded = pickle.load(f)
            if search_seat in loaded:
                print(f"Found: {loaded[search_seat]}")
            else:
                print("Seat no not found")
    except:
        print("Error reading file")

manage_students()
    \end{lstlisting}
    \begin{mnemonicbox}BSECH - Binary Storage, Search Efficiently, CRUD Handling\end{mnemonicbox}
\end{solutionbox}

\section*{Question 5}

\questionmarks{5(a)}{3}{What is Turtle and how is it used to draw objects?}
\begin{solutionbox}
    \textbf{Turtle} is a Python graphics module that provides a drawing canvas and a cursor (turtle) to create graphics programmatically.

    \textbf{Usage:}
    \begin{lstlisting}[language=Python]
import turtle
t = turtle.Turtle()
# Draw square
for i in range(4):
    t.forward(100)
    t.right(90)
    \end{lstlisting}
    \begin{mnemonicbox}CPTT - Canvas, Pen, Turtle, Teaching tool\end{mnemonicbox}
\end{solutionbox}

\questionmarks{5(b)}{4}{Explain Different ways to move turtle to another position.}
\begin{solutionbox}
    \textbf{Movement Methods:}
    \begin{center}
        \begin{tabulary}{\linewidth}{L L}
            \toprule
            \textbf{Method} & \textbf{Action} \\
            \midrule
            \textbf{forward(d)} & Move forward d units \\
            \textbf{backward(d)} & Move backward d units \\
            \textbf{goto(x,y)} & Move to coordinate (x,y) \\
            \textbf{setx(x)} & Change x coordinate \\
            \textbf{sety(y)} & Change y coordinate \\
            \bottomrule
        \end{tabulary}
    \end{center}
    \begin{mnemonicbox}FGPRS - Forward, Goto, Penup, Rotate, Set coordinates\end{mnemonicbox}
\end{solutionbox}

\questionmarks{5(c)}{7}{Explain how loops can be useful in turtle and provide an example.}
\begin{solutionbox}
    \textbf{Loops} allow repeating drawing commands to create patterns and shapes efficiently.

    \textbf{Example (Star Pattern):}
    \begin{lstlisting}[language=Python]
import turtle
t = turtle.Turtle()

# Draw a star using loop
for i in range(5):
    t.forward(100)
    t.right(144)
    \end{lstlisting}

    \textbf{Benefits:}
    \begin{itemize}
        \item Reduces code repetition.
        \item Easy to change size/sides.
        \item Creates complex geometric patterns.
    \end{itemize}
    \begin{mnemonicbox}LPDC - Loops, Patterns, DynamicGraphics, ComplexDesigns\end{mnemonicbox}
\end{solutionbox}

\questionmarks{5(a) OR}{3}{Explain Shape function in Turtle. How many types of shapes are their in turtle?}
\begin{solutionbox}
    \textbf{Shape function} changes the appearance of the turtle cursor.

    \textbf{Built-in Shapes:}
    \begin{itemize}
        \item "arrow"
        \item "turtle"
        \item "circle"
        \item "square"
        \item "triangle"
        \item "classic"
    \end{itemize}

    \textbf{Code:}
    \begin{lstlisting}[language=Python]
t.shape("turtle")
    \end{lstlisting}
    \begin{mnemonicbox}ATCSTC - Arrow, Turtle, Circle, Square, Triangle, Classic\end{mnemonicbox}
\end{solutionbox}

\questionmarks{5(b) OR}{4}{What are the various types of pen command in Turtle? Explain them.}
\begin{solutionbox}
    \textbf{Pen Commands:}
    \begin{itemize}
        \item \textbf{penup()}: Lifts pen, moves without drawing.
        \item \textbf{pendown()}: Lowers pen, moves with drawing.
        \item \textbf{pensize(w)}: Sets line width.
        \item \textbf{pencolor(c)}: Sets line color.
        \item \textbf{speed(s)}: Sets drawing speed.
    \end{itemize}
    \begin{mnemonicbox}SSCSF - State, Size, Color, Speed, Fill commands\end{mnemonicbox}
\end{solutionbox}

\questionmarks{5(c) OR}{7}{Write a program for draw an Indian Flag using Turtle.}
\begin{solutionbox}
    \textbf{Indian Flag Program:}
    \begin{lstlisting}[language=Python]
import turtle

def draw_rect(color, x, y, width, height):
    t.penup()
    t.goto(x, y)
    t.pendown()
    t.color(color)
    t.begin_fill()
    for _ in range(2):
        t.forward(width)
        t.right(90)
        t.forward(height)
        t.right(90)
    t.end_fill()

t = turtle.Turtle()
t.speed(5)
width = 300
height = 50

# Draw Stripes
draw_rect("orange", -150, 100, width, height)
draw_rect("white", -150, 50, width, height)
draw_rect("green", -150, 0, width, height)

# Draw Chakra
t.penup()
t.goto(0, 0)
t.pendown()
t.color("navy")
t.circle(25)
# Spokes
for i in range(24):
    t.penup()
    t.goto(0, 25)
    t.pendown()
    t.forward(25)
    t.backward(25)
    t.right(15)
    \end{lstlisting}
    \begin{mnemonicbox}SWACP - Stripes, White-chakra, Accurate, Colors, Proportional\end{mnemonicbox}
\end{solutionbox}

\end{document}
