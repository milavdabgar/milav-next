\documentclass{article}

% content/resources/templates/preamble.tex
\usepackage[margin=0.6in]{geometry}
\author{Milav Dabgar}
\usepackage{amsmath,amssymb,amsthm}
\usepackage{booktabs}
\usepackage{multirow}
\usepackage{xcolor}
\usepackage{tcolorbox}
\tcbuselibrary{breakable,skins}
\usepackage[colorlinks=true,linkcolor=blue]{hyperref}
\usepackage{titlesec}
\usepackage{enumitem}
\usepackage{tikz}
\usepackage{pgfplots}
\usepackage{circuitikz}
\usepackage[version=4]{mhchem}
\usepackage{longtable}
\usepackage{array}
\usepackage{float}
\usepackage{caption}
\usepackage{listings}

\lstset{
  basicstyle=\small\ttfamily,
  breaklines=true,
  breakatwhitespace=false,
  postbreak=\mbox{\textcolor{red}{$\hookrightarrow$}\space},
  float=false,
  numbers=left,
  numberstyle=\tiny\color{gray},
  numbersep=10pt,
  xleftmargin=2em,
  keywordstyle=\color{blue},
  commentstyle=\color{green!60!black},
  stringstyle=\color{purple},
  backgroundcolor=\color{gray!5},
  showstringspaces=false,
  tabsize=2,
  captionpos=b,
  keepspaces=true,
  columns=flexible
}

\pgfplotsset{compat=1.18}
\usetikzlibrary{shapes,arrows,positioning,calc,patterns,decorations.pathmorphing,decorations.markings,arrows.meta}

% Color scheme
\definecolor{headcolor}{RGB}{0,102,204}
\definecolor{keycolor}{RGB}{220,20,60}
\definecolor{solutioncolor}{RGB}{34,139,34}
\definecolor{mnemoniccolor}{RGB}{148,0,211}
\definecolor{codecolor}{RGB}{0,0,100}

% Spacing
\setlength{\parskip}{3pt}
\setlist[itemize]{nosep}
\setlist[enumerate]{nosep}

% Title formatting
\titleformat{\section}{\Large\bfseries\color{headcolor}}{\thesection}{1em}{}
\titleformat{\subsection}{\large\bfseries\color{headcolor}}{\thesubsection}{1em}{}

% Pandoc tightlist compatibility
\providecommand{\tightlist}{%
  \setlength{\itemsep}{0pt}\setlength{\parskip}{0pt}}

% Pandoc longtable compatibility
\newcounter{none}
\def\thenone{}


% content/resources/templates/english-boxes.tex

% Custom environments
\newtcolorbox{solutionbox}{
 breakable,
 enhanced,
 colback=solutioncolor!5!white,
 colframe=solutioncolor!75!black,
 fonttitle=\bfseries,
 title=Solution
}

\newtcolorbox{solutionboxnobreak}{
 colback=solutioncolor!5!white,
 colframe=solutioncolor!75!black,
 fonttitle=\bfseries,
 title=Solution
}

\newtcolorbox{keyformula}{
 breakable,
 enhanced,
 colback=keycolor!5!white,
 colframe=keycolor!75!black,
 fonttitle=\bfseries,
 title=Key Formula
}

\newtcolorbox{mnemonicboxenv}{
 breakable,
 enhanced,
 colback=mnemoniccolor!5!white,
 colframe=mnemoniccolor!75!black,
 fonttitle=\bfseries,
 title=Mnemonic
}

\newcommand{\mnemonicbox}[1]{%
  \begin{mnemonicboxenv}
    #1
  \end{mnemonicboxenv}
}


% Custom commands for GTU solutions
% This file defines semantic commands for consistent formatting

% Question command with automatic formatting
\newcommand{\question}[2]{%
  \section*{Question #1}%
  \textbf{#2}%
}

% OR question variant
\newcommand{\questionor}[2]{%
  \section*{Question #1 OR}%
  \textbf{#2}%
}

% Proper table environment with caption
\newenvironment{answertable}[1]{%
  \begin{table}[htbp]
  \centering
  \caption{#1}
}{%
  \end{table}
}

% Proper figure environment for diagrams
\newenvironment{answerdiagram}[1]{%
  \begin{figure}[htbp]
  \centering
  \caption{#1}
}{%
  \end{figure}
}

% Semantic markup for key terms
\newcommand{\keyword}[1]{\textbf{#1}}
\newcommand{\code}[1]{\texttt{#1}}
\newcommand{\classname}[1]{\texttt{#1}}
\newcommand{\methodname}[1]{\texttt{#1}}

% Proper quotation marks
\newcommand{\mnemonic}[1]{``#1''}


\title{Advanced Python Programming (4321602) - Summer 2024 Solution}
\date{June 21, 2024}

\begin{document}
\maketitle

\section*{Question 1}

\questionmarks{1(a)}{3}{Give the difference between Tuple and List in python.}
\begin{solutionbox}
    \textbf{Comparison:}
    \begin{center}
        \begin{tabulary}{\linewidth}{L L L}
            \toprule
            \textbf{Feature} & \textbf{Tuple} & \textbf{List} \\
            \midrule
            \textbf{Mutability} & Immutable (cannot be changed) & Mutable (can be changed) \\
            \textbf{Syntax} & Created using () & Created using [] \\
            \textbf{Performance} & Faster & Slower \\
            \textbf{Methods} & Limited methods (count, index) & Many methods (append, remove, etc.) \\
            \bottomrule
        \end{tabulary}
    \end{center}

    \begin{itemize}
        \item \textbf{Memory efficient}: Tuples use less memory than lists
        \item \textbf{Use case}: Tuples for fixed data, lists for dynamic data
    \end{itemize}
    \begin{mnemonicbox}Tuples are Tight, Lists are Loose\end{mnemonicbox}
\end{solutionbox}

\questionmarks{1(b)}{4}{Define Set and how is it created in python?}
\begin{solutionbox}
    \textbf{Set} is an unordered collection of unique elements in Python.

    \textbf{Creating Sets:}
    \begin{lstlisting}[language=Python]
# Empty set
my_set = set()

# Set with elements
fruits = {"apple", "banana", "orange"}

# From list
numbers = set([1, 2, 3, 4])
    \end{lstlisting}

    \begin{itemize}
        \item \textbf{Unique elements}: No duplicates allowed
        \item \textbf{Unordered}: Elements have no specific order
        \item \textbf{Operations}: Union, intersection, difference supported
    \end{itemize}
    \begin{mnemonicbox}Sets are Special - Unique and Unordered\end{mnemonicbox}
\end{solutionbox}

\questionmarks{1(c)}{7}{What is Dictionary in Python? Write a program to concatenate two dictionary into new one.}
\begin{solutionbox}
    \textbf{Dictionary} is an ordered collection of key-value pairs in Python.

    \textbf{Program:}
    \begin{lstlisting}[language=Python]
# Two dictionaries
dict1 = {1: 10, 2: 20}
dict2 = {3: 30, 4: 40}

# Method 1: Using update()
result1 = dict1.copy()
result1.update(dict2)

# Method 2: Using ** operator
result2 = {**dict1, **dict2}

print("Result:", result2)
# Output: {1: 10, 2: 20, 3: 30, 4: 40}
    \end{lstlisting}

    \begin{itemize}
        \item \textbf{Key-value pairs}: Each element has a key and value
        \item \textbf{Mutable}: Can be modified after creation
        \item \textbf{Fast access}: O(1) average time complexity
    \end{itemize}
    \begin{mnemonicbox}Dictionaries are Dynamic Key-Value stores\end{mnemonicbox}
\end{solutionbox}

\questionmarks{1(c) OR}{7}{What is a list in python? Write a program that finds maximum and minimum numbers from a list.}
\begin{solutionbox}
    \textbf{List} is an ordered, mutable collection of elements in Python.

    \textbf{Program:}
    \begin{lstlisting}[language=Python]
# Input list
numbers = [45, 12, 78, 23, 56, 89, 34]

# Find maximum and minimum
maximum = max(numbers)
minimum = min(numbers)

print(f"Maximum: {maximum}")
print(f"Minimum: {minimum}")

# Manual method
max_val = numbers[0]
min_val = numbers[0]
for num in numbers:
    if num > max_val:
        max_val = num
    if num < min_val:
        min_val = num
    \end{lstlisting}

    \begin{itemize}
        \item \textbf{Ordered}: Elements maintain insertion order
        \item \textbf{Indexing}: Accessed using index [0, 1, 2...]
        \item \textbf{Built-in functions}: min(), max(), len() available
    \end{itemize}
    \begin{mnemonicbox}Lists are Linear and Indexed\end{mnemonicbox}
\end{solutionbox}

\section*{Question 2}

\questionmarks{2(a)}{3}{Explain Nested Tuple with example.}
\begin{solutionbox}
    \textbf{Nested Tuple} is a tuple containing other tuples as elements.

    \textbf{Example:}
    \begin{lstlisting}[language=Python]
# Nested tuple
student_data = (
    ("John", 85, "A"),
    ("Alice", 92, "A+"),
    ("Bob", 78, "B")
)

# Accessing elements
print(student_data[0][1])  # Output: 85
print(student_data[1][0])  # Output: Alice
    \end{lstlisting}

    \begin{itemize}
        \item \textbf{Multi-dimensional}: Tuples within tuples
        \item \textbf{Indexing}: Use multiple indices [i][j]
        \item \textbf{Immutable}: Cannot change nested elements
    \end{itemize}
    \begin{mnemonicbox}Nested means Tuples inside Tuples\end{mnemonicbox}
\end{solutionbox}

\questionmarks{2(b)}{4}{What is random module? Explain with example.}
\begin{solutionbox}
    \textbf{Random module} generates random numbers and performs random operations.

    \textbf{Example:}
    \begin{lstlisting}[language=Python]
import random

# Random integer
num = random.randint(1, 10)
print(f"Random number: {num}")

# Random choice from list
colors = ["red", "blue", "green"]
choice = random.choice(colors)
print(f"Random color: {choice}")

# Random float
decimal = random.random()
print(f"Random decimal: {decimal}")
    \end{lstlisting}

    \begin{itemize}
        \item \textbf{Import required}: import random
        \item \textbf{Various functions}: randint(), choice(), random()
        \item \textbf{Useful for}: Games, simulations, testing
    \end{itemize}
    \begin{mnemonicbox}Random makes things Unpredictable\end{mnemonicbox}
\end{solutionbox}

\questionmarks{2(c)}{7}{Explain different ways of importing package. Give one example of it.}
\begin{solutionbox}
    \textbf{Import Methods:}
    \begin{center}
        \begin{tabulary}{\linewidth}{L L L}
            \toprule
            \textbf{Method} & \textbf{Syntax} & \textbf{Usage} \\
            \midrule
            \textbf{Normal import} & \code{import package} & package.function() \\
            \textbf{From import} & \code{from package import function} & function() \\
            \textbf{Import all} & \code{from package import *} & function() \\
            \textbf{Alias import} & \code{import package as alias} & alias.function() \\
            \bottomrule
        \end{tabulary}
    \end{center}

    \textbf{Example:}
    \begin{lstlisting}[language=Python]
# Normal import
import math
result1 = math.sqrt(16)

# From import
from math import sqrt
result2 = sqrt(16)

# Import with alias
import math as m
result3 = m.sqrt(16)

# Import all (not recommended)
from math import *
result4 = sqrt(16)
    \end{lstlisting}

    \begin{itemize}
        \item \textbf{Namespace}: Normal import keeps separate namespace
        \item \textbf{Direct access}: From import allows direct function call
        \item \textbf{Alias}: Shorter names for convenience
    \end{itemize}
    \begin{mnemonicbox}Import methods: Normal, From, All, Alias\end{mnemonicbox}
\end{solutionbox}

\questionmarks{2(a) OR}{3}{Write down the properties of dictionary in python.}
\begin{solutionbox}
    \textbf{Dictionary Properties:}
    \begin{center}
        \begin{tabulary}{\linewidth}{L L}
            \toprule
            \textbf{Property} & \textbf{Description} \\
            \midrule
            \textbf{Ordered} & Maintains insertion order (Python 3.7+) \\
            \textbf{Mutable} & Can be modified after creation \\
            \textbf{Key-unique} & No duplicate keys allowed \\
            \textbf{Heterogeneous} & Keys and values can be different types \\
            \bottomrule
        \end{tabulary}
    \end{center}

    \begin{itemize}
        \item \textbf{Fast access}: O(1) average lookup time
        \item \textbf{Dynamic size}: Can grow or shrink
        \item \textbf{Key restrictions}: Keys must be immutable
    \end{itemize}
    \begin{mnemonicbox}Dictionaries are Ordered, Mutable, Unique, Heterogeneous\end{mnemonicbox}
\end{solutionbox}

\questionmarks{2(b) OR}{4}{What is the dir() function in python. Explain with example.}
\begin{solutionbox}
    \textbf{dir() function} returns all attributes and methods of an object.

    \textbf{Example:}
    \begin{lstlisting}[language=Python]
# List all attributes of string
text = "hello"
attributes = dir(text)
print(attributes[:5])  # First 5 attributes

# Check available methods
print("upper" in dir(text))  # True

# For modules
import math
math_methods = dir(math)
print("sqrt" in math_methods)  # True

# For custom objects
class MyClass:
    def my_method(self):
        pass

obj = MyClass()
print(dir(obj))
    \end{lstlisting}

    \begin{itemize}
        \item \textbf{Introspection}: Examines object properties
        \item \textbf{Debugging}: Helps find available methods
        \item \textbf{All objects}: Works with any Python object
    \end{itemize}
    \begin{mnemonicbox}dir() shows Directory of object attributes\end{mnemonicbox}
\end{solutionbox}

\questionmarks{2(c) OR}{7}{Write a program to define module to find sum of two numbers. Import module to another program.}
\begin{solutionbox}
    \textbf{Module file (calculator.py):}
    \begin{lstlisting}[language=Python]
# calculator.py
def add_numbers(a, b):
    """Function to add two numbers"""
    return a + b

def multiply_numbers(a, b):
    """Function to multiply two numbers"""
    return a * b

def get_sum(num1, num2):
    """Alternative sum function"""
    result = num1 + num2
    return result
    \end{lstlisting}

    \textbf{Main program (main.py):}
    \begin{lstlisting}[language=Python]
# main.py
import calculator

# Using the module
result1 = calculator.add_numbers(10, 20)
print(f"Sum: {result1}")

# From import
from calculator import get_sum
result2 = get_sum(15, 25)
print(f"Sum using from import: {result2}")
    \end{lstlisting}

    \begin{itemize}
        \item \textbf{Module creation}: Save functions in .py file
        \item \textbf{Import}: Use import statement to access
        \item \textbf{Code reusability}: Use same module in multiple programs
    \end{itemize}
    \begin{mnemonicbox}Modules make code Reusable and Organized\end{mnemonicbox}
\end{solutionbox}

\section*{Question 3}

\questionmarks{3(a)}{3}{What is Runtime error and Logical error. Explain with example.}
\begin{solutionbox}
    \textbf{Comparison:}
    \begin{center}
        \begin{tabulary}{\linewidth}{L L L}
            \toprule
            \textbf{Error Type} & \textbf{Definition} & \textbf{Example} \\
            \midrule
            \textbf{Runtime Error} & Occurs during program execution & Division by zero, file not found \\
            \textbf{Logical Error} & Program runs but gives wrong output & Wrong formula, incorrect condition \\
            \bottomrule
        \end{tabulary}
    \end{center}

    \textbf{Examples:}
    \begin{lstlisting}[language=Python]
# Runtime Error
x = 10
y = 0
result = x / y  # ZeroDivisionError

# Logical Error
def calculate_area(radius):
    return 3.14 * radius  # Should be radius * radius
    \end{lstlisting}
    \begin{mnemonicbox}Runtime Crashes, Logical Confuses\end{mnemonicbox}
\end{solutionbox}

\questionmarks{3(b)}{4}{Write points on Except and explaining it.}
\begin{solutionbox}
    \textbf{Except clause} handles specific exceptions in try-except block.

    \textbf{Key Points:}
    \begin{center}
        \begin{tabulary}{\linewidth}{L L}
            \toprule
            \textbf{Feature} & \textbf{Description} \\
            \midrule
            \textbf{Syntax} & \code{except ExceptionType:} \\
            \textbf{Multiple} & Can have multiple except blocks \\
            \textbf{Generic} & \code{except:} catches all exceptions \\
            \textbf{Variable} & \code{except Exception as e:} stores error \\
            \bottomrule
        \end{tabulary}
    \end{center}

    \begin{lstlisting}[language=Python]
try:
    number = int(input("Enter number: "))
    result = 10 / number
except ValueError:
    print("Invalid input")
except ZeroDivisionError:
    print("Cannot divide by zero")
except Exception as e:
    print(f"Error: {e}")
    \end{lstlisting}
    \begin{mnemonicbox}Except Catches and Handles errors\end{mnemonicbox}
\end{solutionbox}

\questionmarks{3(c)}{7}{Write a program to catch Divide by zero Exception. Also use finally block.}
\begin{solutionbox}
    \textbf{Program:}
    \begin{lstlisting}[language=Python]
def safe_division():
    try:
        # Get input from user
        numerator = float(input("Enter numerator: "))
        denominator = float(input("Enter denominator: "))
        
        # Perform division
        result = numerator / denominator
        print(f"Result: {numerator} / {denominator} = {result}")
        
    except ZeroDivisionError:
        print("Error: Cannot divide by zero!")
        print("Please enter a non-zero denominator")
        
    except ValueError:
        print("Error: Please enter valid numbers only")
        
    except Exception as e:
        print(f"Unexpected error occurred: {e}")
        
    finally:
        print("Division operation completed")
        print("Thank you for using the calculator")

# Call the function
safe_division()
    \end{lstlisting}

    \begin{itemize}
        \item \textbf{Try block}: Contains risky code
        \item \textbf{Except}: Handles ZeroDivisionError specifically
        \item \textbf{Finally}: Always executes regardless of exception
    \end{itemize}
    \begin{mnemonicbox}Try risky code, Except handles errors, Finally always runs\end{mnemonicbox}
\end{solutionbox}

\questionmarks{3(a) OR}{3}{What are the built-in exceptions and gives its types.}
\begin{solutionbox}
    \textbf{Built-in Exception Types:}
    \begin{center}
        \begin{tabulary}{\linewidth}{L L L}
            \toprule
            \textbf{Type} & \textbf{Description} & \textbf{Example} \\
            \midrule
            \textbf{ValueError} & Invalid value for operation & \code{int("abc")} \\
            \textbf{TypeError} & Wrong data type & \code{"5" + 5} \\
            \textbf{IndexError} & Index out of range & \code{list[10]} \\
            \textbf{KeyError} & Key not found & \code{dict["missing"]} \\
            \textbf{FileNotFoundError} & File doesn't exist & \code{open("no.txt")} \\
            \bottomrule
        \end{tabulary}
    \end{center}
    \begin{mnemonicbox}Value, Type, Index, Key, File - common error types\end{mnemonicbox}
\end{solutionbox}

\questionmarks{3(b) OR}{4}{Explain Syntax error and how do we identify it? Give an example.}
\begin{solutionbox}
    \textbf{Syntax Error} occurs when Python cannot parse the code due to incorrect syntax.

    \textbf{Identification:}
    \begin{center}
        \begin{tabulary}{\linewidth}{L L}
            \toprule
            \textbf{Method} & \textbf{Description} \\
            \midrule
            \textbf{Python interpreter} & Shows error message with line number \\
            \textbf{IDE highlighting} & Code editors highlight syntax errors \\
            \textbf{Error message} & Points to exact location of error \\
            \bottomrule
        \end{tabulary}
    \end{center}

    \textbf{Examples:}
    \begin{lstlisting}[language=Python]
# Missing colon
if x > 5
    print("Greater")  # SyntaxError

# Unmatched parentheses
print("Hello"  # SyntaxError

# Incorrect indentation
def my_function():
print("Hello")  # IndentationError
    \end{lstlisting}
    \begin{mnemonicbox}Syntax errors Stop code from Starting\end{mnemonicbox}
\end{solutionbox}

\questionmarks{3(c) OR}{7}{What is Exception handling in python? Explain with proper example.}
\begin{solutionbox}
    \textbf{Exception Handling} is a mechanism to handle runtime errors gracefully without crashing the program.

    \textbf{Program:}
    \begin{lstlisting}[language=Python]
def file_processor():
    filename = None
    try:
        filename = input("Enter filename: ")
        with open(filename, 'r') as file:
            content = file.read()
            numbers = [int(x) for x in content.split()]
            average = sum(numbers) / len(numbers)
            print(f"Average: {average}")
            
    except FileNotFoundError:
        print(f"Error: File '{filename}' not found")
        
    except ValueError:
        print("Error: File contains non-numeric data")
        
    except ZeroDivisionError:
        print("Error: No numbers found in file")
        
    except Exception as e:
        print(f"Unexpected error: {e}")
        
    else:
        print("File processed successfully")
        
    finally:
        print("File processing operation completed")

# Run the function
file_processor()
    \end{lstlisting}
    \begin{mnemonicbox}Try-Except-Else-Finally: Complete error handling\end{mnemonicbox}
\end{solutionbox}

\section*{Question 4}

\questionmarks{4(a)}{3}{What kind of different operations we can perform in a file?}
\begin{solutionbox}
    \textbf{File Operations:}
    \begin{center}
        \begin{tabulary}{\linewidth}{L L L}
            \toprule
            \textbf{Operation} & \textbf{Description} & \textbf{Method} \\
            \midrule
            \textbf{Read} & Read file content & \code{read()}, \code{readline()} \\
            \textbf{Write} & Write data to file & \code{write()} \\
            \textbf{Append} & Add data to end & mode 'a' \\
            \textbf{Create} & Create new file & mode 'w', 'x' \\
            \textbf{Delete} & Remove file & \code{os.remove()} \\
            \textbf{Seek} & Move file pointer & \code{seek()} \\
            \bottomrule
        \end{tabulary}
    \end{center}
    \begin{mnemonicbox}Read, Write, Append, Create, Delete, Seek\end{mnemonicbox}
\end{solutionbox}

\questionmarks{4(b)}{4}{Give list of file modes. Write Description of any four mode.}
\begin{solutionbox}
    \textbf{File Modes:}
    \begin{center}
        \begin{tabulary}{\linewidth}{L L L}
            \toprule
            \textbf{Mode} & \textbf{Description} & \textbf{Purpose} \\
            \midrule
            \textbf{'r'} & Read mode & Read existing file (default) \\
            \textbf{'w'} & Write mode & Create/Overwrite file \\
            \textbf{'a'} & Append mode & Add to end of file \\
            \textbf{'x'} & Exclusive & Create new, fail if exists \\
            \textbf{'b'} & Binary mode & Binary files \\
            \textbf{'+'} & Read+Write & Update mode \\
            \bottomrule
        \end{tabulary}
    \end{center}
    \begin{mnemonicbox}Read, Write, Append, eXclusive - main file modes\end{mnemonicbox}
\end{solutionbox}

\questionmarks{4(c)}{7}{Write a program to sort all the words in a file and put it in list.}
\begin{solutionbox}
    \textbf{Program:}
    \begin{lstlisting}[language=Python]
def sort_words_from_file():
    try:
        # Input filename
        filename = input("Enter filename: ")
        
        # Read file content
        with open(filename, 'r') as file:
            content = file.read()
        
        # Split into words and clean them
        words = content.lower().split()
        
        # Remove punctuation and empty strings
        import string
        clean_words = []
        for word in words:
            clean_word = word.translate(str.maketrans('', '', string.punctuation))
            if clean_word:  # Add only non-empty words
                clean_words.append(clean_word)
        
        # Sort the words
        sorted_words = sorted(clean_words)
        
        # Display results
        print("Sorted words:")
        print(sorted_words)
        
        # Save to new file
        with open('sorted_words.txt', 'w') as output_file:
            for word in sorted_words:
                output_file.write(word + '\n')
                
        print(f"Total words: {len(sorted_words)}")
        
    except FileNotFoundError:
        print("Error: File not found")
    except Exception as e:
        print(f"Error: {e}")

sort_words_from_file()
    \end{lstlisting}
    \begin{mnemonicbox}Read, Split, Clean, Sort, Save\end{mnemonicbox}
\end{solutionbox}

\questionmarks{4(a) OR}{3}{What is file handling? List files handling operation and explain it.}
\begin{solutionbox}
    \textbf{File Handling} is the process of working with files to store and retrieve data permanently.

    \textbf{File Handling Operations:}
    \begin{center}
        \begin{tabulary}{\linewidth}{L L L}
            \toprule
            \textbf{Operation} & \textbf{Function} & \textbf{Description} \\
            \midrule
            \textbf{Open} & \code{open()} & Opens file in specified mode \\
            \textbf{Read} & \code{read()} & Reads data from file \\
            \textbf{Write} & \code{write()} & Writes data to file \\
            \textbf{Close} & \code{close()} & Closes file \\
            \textbf{Seek} & \code{seek()} & Moves file pointer \\
            \textbf{Tell} & \code{tell()} & Returns current position \\
            \bottomrule
        \end{tabulary}
    \end{center}
    \begin{mnemonicbox}Open, Read, Write, Close - basic file cycle\end{mnemonicbox}
\end{solutionbox}

\questionmarks{4(b) OR}{4}{Explain load() method with example.}
\begin{solutionbox}
    \textbf{load() method} is used to deserialize data from a file (usually with pickle module).

    \textbf{Example:}
    \begin{lstlisting}[language=Python]
import pickle

# First, save some data
data_to_save = {'name': 'John', 'scores': [85, 92, 78]}
with open('data.pkl', 'wb') as file:
    pickle.dump(data_to_save, file)

# Load data from file
with open('data.pkl', 'rb') as file:
    loaded_data = pickle.load(file)

print("Loaded data:", loaded_data)
    \end{lstlisting}

    \begin{itemize}
        \item \textbf{Deserialization}: Converts file data back to Python objects
        \item \textbf{Binary mode}: Use 'rb' mode for pickle files
    \end{itemize}
    \begin{mnemonicbox}load() brings file data back to Python objects\end{mnemonicbox}
\end{solutionbox}

\questionmarks{4(c) OR}{7}{Write a program that inputs a text file. The program should print all of the unique words in the file in alphabetical order.}
\begin{solutionbox}
    \textbf{Program:}
    \begin{lstlisting}[language=Python]
def find_unique_words():
    try:
        # Get filename
        filename = input("Enter text filename: ")
        
        # Read file content
        with open(filename, 'r', encoding='utf-8') as file:
            content = file.read().lower()
        
        # Clean and extract words
        import re
        words = re.findall(r'\b[a-zA-Z]+\b', content)
        
        # Create set (unique) and sort
        unique_words = sorted(list(set(words)))
        
        # Display results
        print("\nUnique words in alphabetical order:")
        for i, word in enumerate(unique_words, 1):
            print(f"{i:3d}. {word}")
        
        print(f"\nTotal unique words: {len(unique_words)}")
        
        # Save results
        with open('unique_words.txt', 'w') as f:
            for word in unique_words:
                f.write(word + '\n')
        
    except FileNotFoundError:
        print(f"Error: File '{filename}' not found")
    except Exception as e:
        print(f"Error: {e}")

find_unique_words()
    \end{lstlisting}
    \begin{mnemonicbox}Read, Extract, Unique, Sort, Display\end{mnemonicbox}
\end{solutionbox}

\section*{Question 5}

\questionmarks{5(a)}{3}{Explain the use of the following turtle function with an appropriate example. (a) turn() (b) move().}
\begin{solutionbox}
    \textbf{Note}: Standard turtle corresponds to \code{left/right} for turn and \code{forward/backward} for move.

    \textbf{Functions:}
    \begin{center}
        \begin{tabulary}{\linewidth}{L L L}
            \toprule
            \textbf{Function} & \textbf{Purpose} & \textbf{Example} \\
            \midrule
            \textbf{Turn} & Change direction & \code{turtle.left(90)} \\
            \textbf{Move} & Change position & \code{turtle.forward(100)} \\
            \bottomrule
        \end{tabulary}
    \end{center}

    \begin{lstlisting}[language=Python]
import turtle
t = turtle.Turtle()
t.forward(100)  # Move
t.left(90)      # Turn
    \end{lstlisting}
    \begin{mnemonicbox}Turn changes direction, Move changes position\end{mnemonicbox}
\end{solutionbox}

\questionmarks{5(b)}{4}{Explain the various inbuilt methods to change the direction of the turtle.}
\begin{solutionbox}
    \textbf{Direction Methods:}
    \begin{center}
        \begin{tabulary}{\linewidth}{L L L}
            \toprule
            \textbf{Method} & \textbf{Description} & \textbf{Example} \\
            \midrule
            \textbf{left(deg)} & Turn left (counter-clockwise) & \code{t.left(90)} \\
            \textbf{right(deg)} & Turn right (clockwise) & \code{t.right(45)} \\
            \textbf{setheading(deg)} & Set absolute angle & \code{t.setheading(0)} \\
            \textbf{towards(x,y)} & Angle towards point & \code{t.towards(0,0)} \\
            \bottomrule
        \end{tabulary}
    \end{center}
    \begin{mnemonicbox}Left-Right relative, Heading absolute, Towards calculates\end{mnemonicbox}
\end{solutionbox}

\questionmarks{5(c)}{7}{Write a program to draw square, rectangle and circle using turtle.}
\begin{solutionbox}
    \textbf{Program:}
    \begin{lstlisting}[language=Python]
import turtle

def draw_shapes():
    t = turtle.Turtle()
    t.speed(3)
    
    # Draw Square
    t.penup(); t.goto(-200, 50); t.pendown()
    t.write("Square")
    for _ in range(4):
        t.forward(80)
        t.right(90)
    
    # Draw Rectangle
    t.penup(); t.goto(0, 50); t.pendown()
    t.write("Rectangle")
    for _ in range(2):
        t.forward(120)  # Length
        t.right(90)
        t.forward(60)   # Width
        t.right(90)
    
    # Draw Circle
    t.penup(); t.goto(200, 50); t.pendown()
    t.write("Circle")
    t.circle(40)
    
    turtle.done()

draw_shapes()
    \end{lstlisting}
    \begin{mnemonicbox}Square: 4 equal sides, Rectangle: 2 pairs, Circle: radius method\end{mnemonicbox}
\end{solutionbox}

\questionmarks{5(a) OR}{3}{What are the various types of pen command in turtle? Explain them all.}
\begin{solutionbox}
    \textbf{Pen Commands:}
    \begin{center}
        \begin{tabulary}{\linewidth}{L L}
            \toprule
            \textbf{Command} & \textbf{Purpose} \\
            \midrule
            \textbf{penup()} & Lift pen (stop drawing) \\
            \textbf{pendown()} & Lower pen (start drawing) \\
            \textbf{pensize(w)} & Set line thickness \\
            \textbf{pencolor(c)} & Set line color \\
            \textbf{fillcolor(c)} & Set fill color \\
            \textbf{begin\_fill()} & Start filling shape \\
            \textbf{end\_fill()} & Stop filling shape \\
            \bottomrule
        \end{tabulary}
    \end{center}
    \begin{mnemonicbox}Up-Down controls drawing, Size-Color controls appearance\end{mnemonicbox}
\end{solutionbox}

\questionmarks{5(b) OR}{4}{Draw circle and star shapes using turtle and fill them with red color.}
\begin{solutionbox}
    \textbf{Program:}
    \begin{lstlisting}[language=Python]
import turtle

t = turtle.Turtle()
t.color("red", "red")  # Pen and Fill color

# Filled Circle
t.begin_fill()
t.circle(50)
t.end_fill()

t.penup(); t.forward(150); t.pendown()

# Filled Star
t.begin_fill()
for _ in range(5):
    t.forward(100)
    t.right(144)
t.end_fill()

turtle.done()
    \end{lstlisting}
    \begin{mnemonicbox}Begin fill, Draw shape, End fill\end{mnemonicbox}
\end{solutionbox}

\questionmarks{5(c) OR}{7}{Write a program to draw Indian Flag using turtle.}
\begin{solutionbox}
    \textbf{Indian Flag Program:}
    \begin{lstlisting}[language=Python]
import turtle

def draw_rect(color, x, y, width, height):
    t.penup(); t.goto(x, y); t.pendown()
    t.color(color)
    t.begin_fill()
    for _ in range(2):
        t.forward(width); t.right(90)
        t.forward(height); t.right(90)
    t.end_fill()

t = turtle.Turtle()
t.speed(5)
width = 300
height = 60

# Draw Stripes
draw_rect("orange", -150, 150, width, height)
draw_rect("white", -150, 90, width, height)
draw_rect("green", -150, 30, width, height)

# Draw Chakra
t.penup()
t.goto(0, 60)  # Center of white stripe
t.pendown()
t.color("navy")
t.circle(30)   # Outer circle

# Spokes
for i in range(24):
    t.penup(); t.goto(0, 90); t.pendown()
    t.setheading(i * 15)
    t.forward(30)

t.hideturtle()
turtle.done()
    \end{lstlisting}
    \begin{mnemonicbox}Saffron-White-Green stripes with 24-spoke Chakra\end{mnemonicbox}
\end{solutionbox}

\end{document}
