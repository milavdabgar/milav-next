\documentclass{article}

% content/resources/templates/preamble.tex
\usepackage[margin=0.6in]{geometry}
\author{Milav Dabgar}
\usepackage{amsmath,amssymb,amsthm}
\usepackage{booktabs}
\usepackage{multirow}
\usepackage{xcolor}
\usepackage{tcolorbox}
\tcbuselibrary{breakable,skins}
\usepackage[colorlinks=true,linkcolor=blue]{hyperref}
\usepackage{titlesec}
\usepackage{enumitem}
\usepackage{tikz}
\usepackage{pgfplots}
\usepackage{circuitikz}
\usepackage[version=4]{mhchem}
\usepackage{longtable}
\usepackage{array}
\usepackage{float}
\usepackage{caption}
\usepackage{listings}

\lstset{
  basicstyle=\small\ttfamily,
  breaklines=true,
  breakatwhitespace=false,
  postbreak=\mbox{\textcolor{red}{$\hookrightarrow$}\space},
  float=false,
  numbers=left,
  numberstyle=\tiny\color{gray},
  numbersep=10pt,
  xleftmargin=2em,
  keywordstyle=\color{blue},
  commentstyle=\color{green!60!black},
  stringstyle=\color{purple},
  backgroundcolor=\color{gray!5},
  showstringspaces=false,
  tabsize=2,
  captionpos=b,
  keepspaces=true,
  columns=flexible
}

\pgfplotsset{compat=1.18}
\usetikzlibrary{shapes,arrows,positioning,calc,patterns,decorations.pathmorphing,decorations.markings,arrows.meta}

% Color scheme
\definecolor{headcolor}{RGB}{0,102,204}
\definecolor{keycolor}{RGB}{220,20,60}
\definecolor{solutioncolor}{RGB}{34,139,34}
\definecolor{mnemoniccolor}{RGB}{148,0,211}
\definecolor{codecolor}{RGB}{0,0,100}

% Spacing
\setlength{\parskip}{3pt}
\setlist[itemize]{nosep}
\setlist[enumerate]{nosep}

% Title formatting
\titleformat{\section}{\Large\bfseries\color{headcolor}}{\thesection}{1em}{}
\titleformat{\subsection}{\large\bfseries\color{headcolor}}{\thesubsection}{1em}{}

% Pandoc tightlist compatibility
\providecommand{\tightlist}{%
  \setlength{\itemsep}{0pt}\setlength{\parskip}{0pt}}

% Pandoc longtable compatibility
\newcounter{none}
\def\thenone{}


% content/resources/templates/gujarati-boxes.tex
\usepackage{fontspec}
\usepackage{polyglossia}

% Set Gujarati as main language (document is primarily in Gujarati)
% Note: gloss-gujarati.ldf doesn't exist in polyglossia, but it will use hyphenation patterns
\setdefaultlanguage{gujarati}
\setotherlanguage{english}

% Configure Gujarati font properly
% Use Language=Default to prevent polyglossia from trying to add language-specific features
% that don't exist for Gujarati, which causes "empty feature" warnings
\newfontfamily\gujaratifont[Script=Gujarati,AutoFakeBold=2.5,AutoFakeSlant=0.3]{Noto Sans Gujarati}
\setmainfont[Script=Gujarati,AutoFakeBold=2.5,AutoFakeSlant=0.3]{Noto Sans Gujarati}
% Use Noto Sans Gujarati for monospace to support Gujarati in text
\setmonofont[Scale=0.9]{Noto Sans Gujarati}

% Configure English to use the same font
\newfontfamily\englishfont[Script=Gujarati,AutoFakeBold=2.5,AutoFakeSlant=0.3]{Noto Sans Gujarati}

% Translations for polyglossia
\gappto\captionsgujarati{
  \renewcommand{\tablename}{કોષ્ટક}
  \renewcommand{\figurename}{આકૃતિ}
}

% Helper for TikZ nodes to ensure Gujarati font
\newcommand{\gu}[1]{{\gujaratifont #1}}

% Custom environments
\newtcolorbox{solutionbox}{
    breakable,
    enhanced,
    colback=solutioncolor!5!white,
    colframe=solutioncolor!75!black,
    fonttitle=\bfseries,
    title=જવાબ
}

\newtcolorbox{solutionboxnobreak}{
 colback=solutioncolor!5!white,
 colframe=solutioncolor!75!black,
 fonttitle=\bfseries,
 title=જવાબ
}

\newtcolorbox{keyformula}{
 breakable,
 enhanced,
 colback=keycolor!5!white,
 colframe=keycolor!75!black,
 fonttitle=\bfseries,
 title=રાસાયણિક સમીકરણ/સૂત્ર
}

\newtcolorbox{mnemonicbox}{
 breakable,
 enhanced,
 colback=mnemoniccolor!5!white,
 colframe=mnemoniccolor!75!black,
 fonttitle=\bfseries,
 title=મેમરી ટ્રીક
}


% Custom commands for GTU solutions
% This file defines semantic commands for consistent formatting

% Question command with automatic formatting
\newcommand{\question}[2]{%
  \section*{Question #1}%
  \textbf{#2}%
}

% OR question variant
\newcommand{\questionor}[2]{%
  \section*{Question #1 OR}%
  \textbf{#2}%
}

% Proper table environment with caption
\newenvironment{answertable}[1]{%
  \begin{table}[htbp]
  \centering
  \caption{#1}
}{%
  \end{table}
}

% Proper figure environment for diagrams
\newenvironment{answerdiagram}[1]{%
  \begin{figure}[htbp]
  \centering
  \caption{#1}
}{%
  \end{figure}
}

% Semantic markup for key terms
\newcommand{\keyword}[1]{\textbf{#1}}
\newcommand{\code}[1]{\texttt{#1}}
\newcommand{\classname}[1]{\texttt{#1}}
\newcommand{\methodname}[1]{\texttt{#1}}

% Proper quotation marks
\newcommand{\mnemonic}[1]{``#1''}


\title{Advanced Python Programming (4321602) - Summer 2024 Solution}
\date{June 21, 2024}

\begin{document}
\maketitle

\section*{Question 1}

\questionmarks{1(a)}{3}{પાયથનમાં ટપલ અને લિસ્ટ વચ્ચેનો તફાવત લખો.}
\begin{solutionbox}
    \textbf{તફાવત:}
    \begin{center}
        \begin{tabulary}{\linewidth}{L L L}
            \toprule
            \textbf{લક્ષણ} & \textbf{ટપલ} & \textbf{લિસ્ટ} \\
            \midrule
            \textbf{મ્યુટેબિલિટી} & ઇમ્યુટેબલ (બદલી શકાતું નથી) & મ્યુટેબલ (બદલી શકાય છે) \\
            \textbf{સિન્ટેક્સ} & () સાથે બનાવાય છે & [] સાથે બનાવાય છે \\
            \textbf{પ્રદર્શન} & ઝડપી & ધીમું \\
            \textbf{મેથડ્સ} & મર્યાદિત મેથડ્સ (count, index) & ઘણી મેથડ્સ (append, remove, વગેરે) \\
            \bottomrule
        \end{tabulary}
    \end{center}

    \begin{itemize}
        \item \textbf{મેમરી કાર્યક્ષમ}: ટપલ લિસ્ટ કરતાં ઓછી મેમરી વાપરે છે
        \item \textbf{ઉપયોગ}: સ્થિર ડેટા માટે ટપલ, ગતિશીલ ડેટા માટે લિસ્ટ
    \end{itemize}
    \begin{mnemonicbox}ટપલ ટાઇટ, લિસ્ટ લૂઝ\end{mnemonicbox}
\end{solutionbox}

\questionmarks{1(b)}{4}{સેટ સમજાવો અને પાયથનમાં સેટ કેવી રીતે બનાવાય છે?}
\begin{solutionbox}
    \textbf{સેટ} એ પાયથનમાં અનોખા તત્વોનો અક્રમાંકિત સંગ્રહ છે.

    \textbf{સેટ બનાવવાની રીતો:}
    \begin{lstlisting}[language=Python]
# ખાલી સેટ
my_set = set()

# તત્વો સાથે સેટ
fruits = {"apple", "banana", "orange"}

# લિસ્ટમાંથી સેટ
numbers = set([1, 2, 3, 4])
    \end{lstlisting}

    \begin{itemize}
        \item \textbf{અનોખા તત્વો}: ડુપ્લિકેટની મંજૂરી નથી
        \item \textbf{અક્રમાંકિત}: તત્વોનો કોઈ ચોક્કસ ક્રમ નથી
        \item \textbf{ઓપરેશન્સ}: યુનિયન, ઇન્ટરસેક્શન, ડિફરન્સ સપોર્ટેડ
    \end{itemize}
    \begin{mnemonicbox}સેટ સ્પેશિયલ - અનોખા અને અક્રમાંકિત\end{mnemonicbox}
\end{solutionbox}

\questionmarks{1(c)}{7}{પાયથનમાં ડિક્શનરી એટલે શું? બે ડિક્શનરીને નવી ડિક્શનરીમાં જોડવા માટેનો પ્રોગ્રામ લખો.}
\begin{solutionbox}
    \textbf{ડિક્શનરી} એ પાયથનમાં કી-વેલ્યુ પેર્સનો ક્રમાંકિત સંગ્રહ છે.

    \textbf{પ્રોગ્રામ:}
    \begin{lstlisting}[language=Python]
# બે ડિક્શનરીઓ
dict1 = {1: 10, 2: 20}
dict2 = {3: 30, 4: 40}

# મેથડ 1: update() નો ઉપયોગ
result1 = dict1.copy()
result1.update(dict2)

# મેથડ 2: ** ઓપરેટરનો ઉપયોગ
result2 = {**dict1, **dict2}

print("પરિણામ:", result2)
# આઉટપુટ: {1: 10, 2: 20, 3: 30, 4: 40}
    \end{lstlisting}

    \begin{itemize}
        \item \textbf{કી-વેલ્યુ પેર્સ}: દરેક તત્વમાં કી અને વેલ્યુ હોય છે
        \item \textbf{મ્યુટેબલ}: બનાવ્યા પછી બદલી શકાય છે
        \item \textbf{ઝડપી એક્સેસ}: O(1) સરેરાશ સમય જટિલતા
    \end{itemize}
    \begin{mnemonicbox}ડિક્શનરી ડાયનેમિક કી-વેલ્યુ સ્ટોર છે\end{mnemonicbox}
\end{solutionbox}

\questionmarks{1(c) OR}{7}{પાયથનમાં લિસ્ટ એટલે શું? એક પ્રોગ્રામ લખો જે સૂચિમાંથી મહત્તમ અને ન્યૂનતમ નંબરો શોધે.}
\begin{solutionbox}
    \textbf{લિસ્ટ} એ પાયથનમાં તત્વોનો ક્રમાંકિત, મ્યુટેબલ સંગ્રહ છે.

    \textbf{પ્રોગ્રામ:}
    \begin{lstlisting}[language=Python]
# ઇનપુટ લિસ્ટ
numbers = [45, 12, 78, 23, 56, 89, 34]

# મહત્તમ અને ન્યૂનતમ શોધો
maximum = max(numbers)
minimum = min(numbers)

print(f"મહત્તમ: {maximum}")
print(f"ન્યૂનતમ: {minimum}")

# મેન્યુઅલ મેથડ
max_val = numbers[0]
min_val = numbers[0]
for num in numbers:
    if num > max_val:
        max_val = num
    if num < min_val:
        min_val = num
    \end{lstlisting}

    \begin{itemize}
        \item \textbf{ક્રમાંકિત}: તત્વો ઇન્સર્શન ઓર્ડર જાળવે છે
        \item \textbf{ઇન્ડેક્સિંગ}: ઇન્ડેક્સ [0, 1, 2...] વાપરીને એક્સેસ
        \item \textbf{બિલ્ટ-ઇન ફંક્શન્સ}: min(), max(), len() ઉપલબ્ધ
    \end{itemize}
    \begin{mnemonicbox}લિસ્ટ લિનિયર અને ઇન્ડેક્સ્ડ છે\end{mnemonicbox}
\end{solutionbox}

\section*{Question 2}

\questionmarks{2(a)}{3}{નેસ્ટેડ ટપલને ઉદાહરણ સાથે સમજાવો.}
\begin{solutionbox}
    \textbf{નેસ્ટેડ ટપલ} એ ટપલ છે જેમાં અન્ય ટપલ તત્વો તરીકે હોય છે.

    \textbf{ઉદાહરણ:}
    \begin{lstlisting}[language=Python]
# નેસ્ટેડ ટપલ
student_data = (
    ("John", 85, "A"),
    ("Alice", 92, "A+"),
    ("Bob", 78, "B")
)

# તત્વોને એક્સેસ કરવું
print(student_data[0][1])  # આઉટપુટ: 85
print(student_data[1][0])  # આઉટપુટ: Alice
    \end{lstlisting}

    \begin{itemize}
        \item \textbf{બહુ-પરિમાણીય}: ટપલની અંદર ટપલ
        \item \textbf{ઇન્ડેક્સિંગ}: બહુવિધ ઇન્ડિસેસ [i][j] વાપરો
        \item \textbf{ઇમ્યુટેબલ}: નેસ્ટેડ તત્વો બદલી શકાતા નથી
    \end{itemize}
    \begin{mnemonicbox}નેસ્ટેડ મતલબ ટપલની અંદર ટપલ\end{mnemonicbox}
\end{solutionbox}

\questionmarks{2(b)}{4}{રેન્ડમ મોડ્યુલ શું છે? ઉદાહરણ સાથે સમજાવો.}
\begin{solutionbox}
    \textbf{રેન્ડમ મોડ્યુલ} રેન્ડમ નંબરો જનરેટ કરે છે અને રેન્ડમ ઓપરેશન્સ કરે છે.

    \textbf{ઉદાહરણ:}
    \begin{lstlisting}[language=Python]
import random

# રેન્ડમ ઇન્ટિજર
num = random.randint(1, 10)
print(f"રેન્ડમ નંબર: {num}")

# લિસ્ટમાંથી રેન્ડમ પસંદગી
colors = ["red", "blue", "green"]
choice = random.choice(colors)
print(f"રેન્ડમ રંગ: {choice}")

# રેન્ડમ ફ્લોટ
decimal = random.random()
print(f"રેન્ડમ દશાંશ: {decimal}")
    \end{lstlisting}

    \begin{itemize}
        \item \textbf{ઇમ્પોર્ટ જરૂરી}: import random
        \item \textbf{વિવિધ ફંક્શન્સ}: randint(), choice(), random()
        \item \textbf{ઉપયોગી}: ગેમ્સ, સિમ્યુલેશન, ટેસ્ટિંગ માટે
    \end{itemize}
    \begin{mnemonicbox}રેન્ડમ વસ્તુઓને અણધારી બનાવે છે\end{mnemonicbox}
\end{solutionbox}

\questionmarks{2(c)}{7}{પેકેજને ઇમ્પોર્ટ કરવાની વિવિધ રીતો સમજાવો. તેનું એક ઉદાહરણ આપો.}
\begin{solutionbox}
    \textbf{ઇમ્પોર્ટ મેથડ્સ:}
    \begin{center}
        \begin{tabulary}{\linewidth}{L L L}
            \toprule
            \textbf{મેથડ} & \textbf{સિન્ટેક્સ} & \textbf{ઉપયોગ} \\
            \midrule
            \textbf{નોર્મલ ઇમ્પોર્ટ} & \code{import package} & package.function() \\
            \textbf{ફ્રોમ ઇમ્પોર્ટ} & \code{from package import function} & function() \\
            \textbf{બધું ઇમ્પોર્ટ} & \code{from package import *} & function() \\
            \textbf{એલિયાસ ઇમ્પોર્ટ} & \code{import package as alias} & alias.function() \\
            \bottomrule
        \end{tabulary}
    \end{center}

    \textbf{ઉદાહરણ:}
    \begin{lstlisting}[language=Python]
# નોર્મલ ઇમ્પોર્ટ
import math
result1 = math.sqrt(16)

# ફ્રોમ ઇમ્પોર્ટ
from math import sqrt
result2 = sqrt(16)

# એલિયાસ સાથે ઇમ્પોર્ટ
import math as m
result3 = m.sqrt(16)

# બધું ઇમ્પોર્ટ (ભલામણ નથી)
from math import *
result4 = sqrt(16)
    \end{lstlisting}

    \begin{itemize}
        \item \textbf{નેમસ્પેસ}: નોર્મલ ઇમ્પોર્ટ અલગ નેમસ્પેસ રાખે છે
        \item \textbf{ડાયરેક્ટ એક્સેસ}: ફ્રોમ ઇમ્પોર્ટ ડાયરેક્ટ ફંક્શન કોલ કરવાની મંજૂરી આપે છે
        \item \textbf{એલિયાસ}: સુવિધા માટે ટૂંકા નામો
    \end{itemize}
    \begin{mnemonicbox}ઇમ્પોર્ટ મેથડ્સ: નોર્મલ, ફ્રોમ, બધું, એલિયાસ\end{mnemonicbox}
\end{solutionbox}

\questionmarks{2(a) OR}{3}{પાયથનમાં ડિક્શનરીના ગુણધર્મો લખો.}
\begin{solutionbox}
    \textbf{ડિક્શનરીના ગુણધર્મો:}
    \begin{center}
        \begin{tabulary}{\linewidth}{L L}
            \toprule
            \textbf{ગુણધર્મ} & \textbf{વર્ણન} \\
            \midrule
            \textbf{ક્રમાંકિત} & ઇન્સર્શન ઓર્ડર જાળવે છે (Python 3.7+) \\
            \textbf{મ્યુટેબલ} & બનાવ્યા પછી બદલી શકાય છે \\
            \textbf{કી-અનોખી} & ડુપ્લિકેટ કીઓની મંજૂરી નથી \\
            \textbf{હેટેરોજીનિયસ} & કીઓ અને વેલ્યુઝ અલગ પ્રકારના હોઈ શકે \\
            \bottomrule
        \end{tabulary}
    \end{center}

    \begin{itemize}
        \item \textbf{ઝડપી એક્સેસ}: O(1) સરેરાશ લુકઅપ ટાઇમ
        \item \textbf{ડાયનેમિક સાઇઝ}: વધી અથવા ઘટી શકે છે
        \item \textbf{કી પ્રતિબંધો}: કીઓ ઇમ્યુટેબલ હોવી જોઈએ
    \end{itemize}
    \begin{mnemonicbox}ડિક્શનરી ક્રમાંકિત, મ્યુટેબલ, અનોખી, હેટેરોજીનિયસ છે\end{mnemonicbox}
\end{solutionbox}

\questionmarks{2(b) OR}{4}{પાયથનમાં dir() ફંક્શન શું છે. ઉદાહરણ સાથે સમજાવો.}
\begin{solutionbox}
    \textbf{dir() ફંક્શન} ઓબ્જેક્ટના બધા એટ્રિબ્યુટ્સ અને મેથડ્સ રિટર્ન કરે છે.

    \textbf{ઉદાહરણ:}
    \begin{lstlisting}[language=Python]
# સ્ટ્રિંગના બધા એટ્રિબ્યુટ્સ
text = "hello"
attributes = dir(text)
print(attributes[:5])  # પ્રથમ 5 એટ્રિબ્યુટ્સ

# ઉપલબ્ધ મેથડ્સ ચેક કરો
print("upper" in dir(text))  # True

# મોડ્યુલ્સ માટે
import math
math_methods = dir(math)
print("sqrt" in math_methods)  # True

# કસ્ટમ ઓબ્જેક્ટ્સ માટે
class MyClass:
    def my_method(self):
        pass

obj = MyClass()
print(dir(obj))
    \end{lstlisting}

    \begin{itemize}
        \item \textbf{ઇન્ટ્રોસ્પેક્શન}: ઓબ્જેક્ટ પ્રોપર્ટીઝ તપાસે છે
        \item \textbf{ડિબગિંગ}: ઉપલબ્ધ મેથડ્સ શોધવામાં મદદ કરે છે
        \item \textbf{બધા ઓબ્જેક્ટ્સ}: કોઈપણ Python ઓબ્જેક્ટ સાથે કામ કરે છે
    \end{itemize}
    \begin{mnemonicbox}dir() ઓબ્જેક્ટ એટ્રિબ્યુટ્સની ડિરેક્ટરી બતાવે છે\end{mnemonicbox}
\end{solutionbox}

\questionmarks{2(c) OR}{7}{બે સંખ્યાઓનો સરવાળો શોધવા માટે મોડ્યુલને વ્યાખ્યાયિત કરવા માટે પ્રોગ્રામ લખો. બીજા પ્રોગ્રામમાં મોડ્યુલ ઇમ્પોર્ટ કરો.}
\begin{solutionbox}
    \textbf{મોડ્યુલ ફાઇલ (calculator.py):}
    \begin{lstlisting}[language=Python]
# calculator.py
def add_numbers(a, b):
    """બે સંખ્યાઓ ઉમેરવા માટેનું ફંક્શન"""
    return a + b

def multiply_numbers(a, b):
    """બે સંખ્યાઓ ગુણવા માટેનું ફંક્શન"""
    return a * b

def get_sum(num1, num2):
    """વૈકલ્પિક સમ ફંક્શન"""
    result = num1 + num2
    return result
    \end{lstlisting}

    \textbf{મુખ્ય પ્રોગ્રામ (main.py):}
    \begin{lstlisting}[language=Python]
# main.py
import calculator

# મોડ્યુલનો ઉપયોગ
result1 = calculator.add_numbers(10, 20)
print(f"સરવાળો: {result1}")

# ફ્રોમ ઇમ્પોર્ટ
from calculator import get_sum
result2 = get_sum(15, 25)
print(f"ફ્રોમ ઇમ્પોર્ટ વાપરીને સરવાળો: {result2}")
    \end{lstlisting}

    \begin{itemize}
        \item \textbf{મોડ્યુલ બનાવટ}: ફંક્શન્સને .py ફાઇલમાં સેવ કરો
        \item \textbf{ઇમ્પોર્ટ}: ઇમ્પોર્ટ સ્ટેટમેન્ટ વાપરીને એક્સેસ કરો
        \item \textbf{કોડ પુનઃઉપયોગ}: એક જ મોડ્યુલને અનેક પ્રોગ્રામમાં વાપરો
    \end{itemize}
    \begin{mnemonicbox}મોડ્યુલ કોડને પુનઃઉપયોગી અને વ્યવસ્થિત બનાવે છે\end{mnemonicbox}
\end{solutionbox}

\section*{Question 3}

\questionmarks{3(a)}{3}{રનટાઇમ એરર અને લોજિકલ એરર શું છે. ઉદાહરણ સાથે સમજાવો.}
\begin{solutionbox}
    \textbf{તફાવત:}
    \begin{center}
        \begin{tabulary}{\linewidth}{L L L}
            \toprule
            \textbf{એરર પ્રકાર} & \textbf{વ્યાખ્યા} & \textbf{ઉદાહરણ} \\
            \midrule
            \textbf{રનટાઇમ એરર} & પ્રોગ્રામના અમલ દરમિયાન થાય છે & ઝીરો વડે ભાગાકાર, ફાઇલ ન મળવી \\
            \textbf{લોજિકલ એરર} & પ્રોગ્રામ ચાલે છે પણ ખોટું આઉટપુટ આપે છે & ખોટું સૂત્ર, ખોટી શરત \\
            \bottomrule
        \end{tabulary}
    \end{center}

    \textbf{ઉદાહરણો:}
    \begin{lstlisting}[language=Python]
# રનટાઇમ એરર
x = 10
y = 0
result = x / y  # ZeroDivisionError

# લોજિકલ એરર
def calculate_area(radius):
    return 3.14 * radius  # radius * radius હોવું જોઈએ
    \end{lstlisting}
    \begin{mnemonicbox}રનટાઇમ ક્રેશ કરે, લોજિકલ કન્ફ્યુઝ કરે\end{mnemonicbox}
\end{solutionbox}

\questionmarks{3(b)}{4}{Except પર પોઇન્ટ્સ લખો અને તેને સમજાવો.}
\begin{solutionbox}
    \textbf{Except ક્લોઝ} try-except બ્લોકમાં ચોક્કસ એક્સેપ્શન હેન્ડલ કરે છે.

    \textbf{મુખ્ય પોઇન્ટ્સ:}
    \begin{center}
        \begin{tabulary}{\linewidth}{L L}
            \toprule
            \textbf{લક્ષણ} & \textbf{વર્ણન} \\
            \midrule
            \textbf{સિન્ટેક્સ} & \code{except ExceptionType:} \\
            \textbf{બહુવિધ} & એકથી વધુ except બ્લોક હોઈ શકે છે \\
            \textbf{સામાન્ય} & \code{except:} બધા એક્સેપ્શન પકડે છે \\
            \textbf{વેરિએબલ} & \code{except Exception as e:} એરર સ્ટોર કરે છે \\
            \bottomrule
        \end{tabulary}
    \end{center}

    \begin{lstlisting}[language=Python]
try:
    number = int(input("નંબર દાખલ કરો: "))
    result = 10 / number
except ValueError:
    print("અમાન્ય ઇનપુટ")
except ZeroDivisionError:
    print("ઝીરો વડે ભાગી શકાતું નથી")
except Exception as e:
    print(f"એરર: {e}")
    \end{lstlisting}
    \begin{mnemonicbox}Except એરર પકડે છે અને હેન્ડલ કરે છે\end{mnemonicbox}
\end{solutionbox}

\questionmarks{3(c)}{7}{Divide by zero Exception પકડવા માટે પ્રોગ્રામ લખો. તેમજ finally બ્લોકનો ઉપયોગ કરો.}
\begin{solutionbox}
    \textbf{પ્રોગ્રામ:}
    \begin{lstlisting}[language=Python]
def safe_division():
    try:
        # યુઝર પાસેથી ઇનપુટ મેળવો
        numerator = float(input("અંશ દાખલ કરો: "))
        denominator = float(input("છેદ દાખલ કરો: "))
        
        # ભાગાકાર કરો
        result = numerator / denominator
        print(f"પરિણામ: {numerator} / {denominator} = {result}")
        
    except ZeroDivisionError:
        print("એરર: ઝીરો વડે ભાગી શકાતું નથી!")
        print("કૃપા કરીને નોન-ઝીરો છેદ દાખલ કરો")
        
    except ValueError:
        print("એરર: કૃપા કરીને માન્ય નંબરો જ દાખલ કરો")
        
    except Exception as e:
        print(f"અનપેક્ષિત એરર: {e}")
        
    finally:
        print("ભાગાકાર પ્રક્રિયા પૂર્ણ થઈ")
        print("કેલ્ક્યુલેટર વાપરવા બદલ આભાર")

# ફંક્શન કોલ કરો
safe_division()
    \end{lstlisting}

    \begin{itemize}
        \item \textbf{Try બ્લોક}: જોખમી કોડ ધરાવે છે
        \item \textbf{Except}: ખાસ કરીને ZeroDivisionError હેન્ડલ કરે છે
        \item \textbf{Finally}: એક્સેપ્શન ગમે તે હોય, હંમેશા એક્ઝિક્યુટ થાય છે
    \end{itemize}
    \begin{mnemonicbox}Try જોખમી કોડ, Except એરર હેન્ડલ કરે, Finally હંમેશા ચાલે\end{mnemonicbox}
\end{solutionbox}

\questionmarks{3(a) OR}{3}{બિલ્ટ-ઇન એક્સેપ્શન શું છે અને તેના પ્રકારો આપો.}
\begin{solutionbox}
    \textbf{બિલ્ટ-ઇન એક્સેપ્શન પ્રકારો:}
    \begin{center}
        \begin{tabulary}{\linewidth}{L L L}
            \toprule
            \textbf{પ્રકાર} & \textbf{વર્ણન} & \textbf{ઉદાહરણ} \\
            \midrule
            \textbf{ValueError} & ઓપરેશન માટે અમાન્ય વેલ્યુ & \code{int("abc")} \\
            \textbf{TypeError} & ખોટો ડેટા ટાઇપ & \code{"5" + 5} \\
            \textbf{IndexError} & ઇન્ડેક્સ રેન્જ બહાર & \code{list[10]} \\
            \textbf{KeyError} & કી મળી નથી & \code{dict["missing"]} \\
            \textbf{FileNotFoundError} & ફાઇલ અસ્તિત્વમાં નથી & \code{open("no.txt")} \\
            \bottomrule
        \end{tabulary}
    \end{center}
    \begin{mnemonicbox}Value, Type, Index, Key, File - સામાન્ય એરર પ્રકારો\end{mnemonicbox}
\end{solutionbox}

\questionmarks{3(b) OR}{4}{Syntax error સમજાવો અને આપણે તેને કેવી રીતે ઓળખી શકીએ? ઉદાહરણ આપો.}
\begin{solutionbox}
    \textbf{Syntax Error} ત્યારે થાય છે જ્યારે ખોટા સિન્ટેક્સને કારણે Python કોડ પાર્સ કરી શકતું નથી.

    \textbf{ઓળખ:}
    \begin{center}
        \begin{tabulary}{\linewidth}{L L}
            \toprule
            \textbf{પદ્ધતિ} & \textbf{વર્ણન} \\
            \midrule
            \textbf{Python interpreter} & લાઇન નંબર સાથે એરર મેસેજ બતાવે છે \\
            \textbf{IDE highlighting} & કોડ એડિટર્સ સિન્ટેક્સ એરર હાઇલાઇટ કરે છે \\
            \textbf{એરર મેસેજ} & એરરની ચોક્કસ જગ્યા દર્શાવે છે \\
            \bottomrule
        \end{tabulary}
    \end{center}

    \textbf{ઉદાહરણો:}
    \begin{lstlisting}[language=Python]
# કોલોન ખૂટે છે
if x > 5
    print("Greater")  # SyntaxError

# કૌંસ બંધ નથી
print("Hello"  # SyntaxError

# ખોટું ઇન્ડેન્ટેશન
def my_function():
print("Hello")  # IndentationError
    \end{lstlisting}
    \begin{mnemonicbox}Syntax એરર કોડને શરૂ થતા અટકાવે છે\end{mnemonicbox}
\end{solutionbox}

\questionmarks{3(c) OR}{7}{પાયથનમાં Exception handling શું છે? યોગ્ય ઉદાહરણ સાથે સમજાવો.}
\begin{solutionbox}
    \textbf{Exception Handling} એ પ્રોગ્રામ ક્રેશ થયા વગર રનટાઇમ એરરને ગ્રેસફુલી હેન્ડલ કરવાની પદ્ધતિ છે.

    \textbf{પ્રોગ્રામ:}
    \begin{lstlisting}[language=Python]
def file_processor():
    filename = None
    try:
        filename = input("ફાઇલનામ દાખલ કરો: ")
        with open(filename, 'r') as file:
            content = file.read()
            numbers = [int(x) for x in content.split()]
            average = sum(numbers) / len(numbers)
            print(f"સરેરાશ: {average}")
            
    except FileNotFoundError:
        print(f"એરર: ફાઇલ '{filename}' મળી નથી")
        
    except ValueError:
        print("એરર: ફાઇલમાં નોન-ન્યૂમેરિક ડેટા છે")
        
    except ZeroDivisionError:
        print("એરર: ફાઇલમાં કોઈ નંબરો મળ્યા નથી")
        
    except Exception as e:
        print(f"અનપેક્ષિત એરર: {e}")
        
    else:
        print("ફાઇલ સફળતાપૂર્વક પ્રોસેસ થઈ")
        
    finally:
        print("ફાઇલ પ્રોસેસિંગ ઓપરેશન પૂર્ણ થયું")

# ફંક્શન ચલાવો
file_processor()
    \end{lstlisting}
    \begin{mnemonicbox}Try-Except-Else-Finally: સંપૂર્ણ એરર હેન્ડલિંગ\end{mnemonicbox}
\end{solutionbox}

\section*{Question 4}

\questionmarks{4(a)}{3}{ફાઇલમાં આપણે કયા પ્રકારના વિવિધ ઓપરેશન્સ કરી શકીએ?}
\begin{solutionbox}
    \textbf{ફાઇલ ઓપરેશન્સ:}
    \begin{center}
        \begin{tabulary}{\linewidth}{L L L}
            \toprule
            \textbf{ઓપરેશન} & \textbf{વર્ણન} & \textbf{મેથડ} \\
            \midrule
            \textbf{Read} & ફાઇલ કન્ટેન્ટ વાંચો & \code{read()}, \code{readline()} \\
            \textbf{Write} & ફાઇલમાં ડેટા લખો & \code{write()} \\
            \textbf{Append} & અંતે ડેટા ઉમેરો & mode 'a' \\
            \textbf{Create} & નવી ફાઇલ બનાવો & mode 'w', 'x' \\
            \textbf{Delete} & ફાઇલ દૂર કરો & \code{os.remove()} \\
            \textbf{Seek} & ફાઇલ પોઇન્ટર ખસેડો & \code{seek()} \\
            \bottomrule
        \end{tabulary}
    \end{center}
    \begin{mnemonicbox}Read, Write, Append, Create, Delete, Seek\end{mnemonicbox}
\end{solutionbox}

\questionmarks{4(b)}{4}{ફાઇલ મોડ્સની યાદી આપો. કોઈપણ ચાર મોડનું વર્ણન લખો.}
\begin{solutionbox}
    \textbf{ફાઇલ મોડ્સ:}
    \begin{center}
        \begin{tabulary}{\linewidth}{L L L}
            \toprule
            \textbf{મોડ} & \textbf{વર્ણન} & \textbf{હેતુ} \\
            \midrule
            \textbf{'r'} & Read મોડ & અસ્તિત્વમાં રહેલી ફાઇલ વાંચો (ડિફોલ્ટ) \\
            \textbf{'w'} & Write મોડ & ફાઇલ બનાવો/ઓવરરાઇટ કરો \\
            \textbf{'a'} & Append મોડ & ફાઇલના અંતે ઉમેરો \\
            \textbf{'x'} & Exclusive & નવી બનાવો, જો અસ્તિત્વમાં હોય તો ફેલ \\
            \textbf{'b'} & Binary મોડ & બાઈનરી ફાઇલો \\
            \textbf{'+'} & Read+Write & અપડેટ મોડ \\
            \bottomrule
        \end{tabulary}
    \end{center}
    \begin{mnemonicbox}Read, Write, Append, eXclusive - મુખ્ય ફાઇલ મોડ્સ\end{mnemonicbox}
\end{solutionbox}

\questionmarks{4(c)}{7}{ફાઇલમાંના તમામ શબ્દોને સૉર્ટ કરવા અને તેને લિસ્ટમાં મૂકવા માટેનો પ્રોગ્રામ લખો.}
\begin{solutionbox}
    \textbf{પ્રોગ્રામ:}
    \begin{lstlisting}[language=Python]
def sort_words_from_file():
    try:
        # ઇનપુટ ફાઇલનામ
        filename = input("ફાઇલનામ દાખલ કરો: ")
        
        # ફાઇલ કન્ટેન્ટ વાંચો
        with open(filename, 'r') as file:
            content = file.read()
        
        # વિભાજિત કરો અને સાફ કરો
        words = content.lower().split()
        
        # Punctuation દૂર કરો
        import string
        clean_words = []
        for word in words:
            clean_word = word.translate(str.maketrans('', '', string.punctuation))
            if clean_word:  # માત્ર ખાલી ન હોય તેવા શબ્દો
                clean_words.append(clean_word)
        
        # શબ્દો સૉર્ટ કરો
        sorted_words = sorted(clean_words)
        
        # પરિણામો દર્શાવો
        print("સૉર્ટ થયેલા શબ્દો:")
        print(sorted_words)
        
        # નવી ફાઇલમાં સેવ કરો
        with open('sorted_words.txt', 'w') as output_file:
            for word in sorted_words:
                output_file.write(word + '\n')
                
        print(f"કુલ શબ્દો: {len(sorted_words)}")
        
    except FileNotFoundError:
        print("એરર: ફાઇલ મળી નથી")
    except Exception as e:
        print(f"એરર: {e}")

sort_words_from_file()
    \end{lstlisting}
    \begin{mnemonicbox}Read, Split, Clean, Sort, Save\end{mnemonicbox}
\end{solutionbox}

\questionmarks{4(a) OR}{3}{ફાઇલ હેન્ડલિંગ શું છે? ફાઇલ હેન્ડલિંગ ઓપરેશનની યાદી આપો અને સમજાવો.}
\begin{solutionbox}
    \textbf{ફાઇલ હેન્ડલિંગ} એ ડેટાને કાયમી ધોરણે સ્ટોર કરવા અને પુનઃપ્રાપ્ત કરવા માટે ફાઇલો સાથે કામ કરવાની પ્રક્રિયા છે.

    \textbf{ફાઇલ હેન્ડલિંગ ઓપરેશન્સ:}
    \begin{center}
        \begin{tabulary}{\linewidth}{L L L}
            \toprule
            \textbf{ઓપરેશન} & \textbf{ફંક્શન} & \textbf{વર્ણન} \\
            \midrule
            \textbf{Open} & \code{open()} & ચોક્કસ મોડમાં ફાઇલ ખોલે છે \\
            \textbf{Read} & \code{read()} & ફાઇલમાંથી ડેટા વાંચે છે \\
            \textbf{Write} & \code{write()} & ફાઇલમાં ડેટા લખે છે \\
            \textbf{Close} & \code{close()} & ફાઇલ બંધ કરે છે \\
            \textbf{Seek} & \code{seek()} & ફાઇલ પોઇન્ટર ખસેડે છે \\
            \textbf{Tell} & \code{tell()} & વર્તમાન પોઝિશન રિટર્ન કરે છે \\
            \bottomrule
        \end{tabulary}
    \end{center}
    \begin{mnemonicbox}Open, Read, Write, Close - બેઝિક ફાઇલ સાયકલ\end{mnemonicbox}
\end{solutionbox}

\questionmarks{4(b) OR}{4}{load() મેથડ ઉદાહરણ સાથે સમજાવો.}
\begin{solutionbox}
    \textbf{load() મેથડ} ફાઇલમાંથી ડેટાને ડીસીરિયલાઇઝ કરવા માટે વપરાય છે (સામાન્ય રીતે pickle મોડ્યુલ સાથે).

    \textbf{ઉદાહરણ:}
    \begin{lstlisting}[language=Python]
import pickle

# પહેલાં, ડેટા સેવ કરીએ
data_to_save = {'name': 'John', 'scores': [85, 92, 78]}
with open('data.pkl', 'wb') as file:
    pickle.dump(data_to_save, file)

# ફાઇલમાંથી ડેટા લોડ કરીએ
with open('data.pkl', 'rb') as file:
    loaded_data = pickle.load(file)

print("લોડ થયેલ ડેટા:", loaded_data)
    \end{lstlisting}

    \begin{itemize}
        \item \textbf{ડીસીરિયલાઇઝેશન}: ફાઇલ ડેટાને પાછું Python objects માં કન્વર્ટ કરે છે
        \item \textbf{Binary મોડ}: pickle ફાઇલ્સ માટે 'rb' મોડ વાપરો
    \end{itemize}
    \begin{mnemonicbox}load() ફાઇલ ડેટાને પાછું Python objects માં લાવે છે\end{mnemonicbox}
\end{solutionbox}

\questionmarks{4(c) OR}{7}{એક પ્રોગ્રામ લખો જે ટેક્સ્ટ ફાઇલને ઇનપુટ કરે. પ્રોગ્રામે ફાઇલમાંના તમામ યુનીક શબ્દોને મૂળાક્ષરોના ક્રમમાં છાપવા જોઈએ.}
\begin{solutionbox}
    \textbf{પ્રોગ્રામ:}
    \begin{lstlisting}[language=Python]
def find_unique_words():
    try:
        # ફાઇલનામ મેળવો
        filename = input("ટેક્સ્ટ ફાઇલનામ દાખલ કરો: ")
        
        # ફાઇલ કન્ટેન્ટ વાંચો
        with open(filename, 'r', encoding='utf-8') as file:
            content = file.read().lower()
        
        # સાફ કરો અને શબ્દો એક્સ્ટ્રેક્ટ કરો
        import re
        words = re.findall(r'\b[a-zA-Z]+\b', content)
        
        # સેટ (યુનીક) બનાવો અને સૉર્ટ કરો
        unique_words = sorted(list(set(words)))
        
        # પરિણામો દર્શાવો
        print("\nમૂળાક્ષરોના ક્રમમાં યુનીક શબ્દો:")
        for i, word in enumerate(unique_words, 1):
            print(f"{i:3d}. {word}")
        
        print(f"\nકુલ યુનીક શબ્દો: {len(unique_words)}")
        
        # પરિણામો સેવ કરો
        with open('unique_words.txt', 'w') as f:
            for word in unique_words:
                f.write(word + '\n')
        
    except FileNotFoundError:
        print(f"એરર: ફાઇલ '{filename}' મળી નથી")
    except Exception as e:
        print(f"એરર: {e}")

find_unique_words()
    \end{lstlisting}
    \begin{mnemonicbox}વાંચો, એક્સ્ટ્રેક્ટ કરો, યુનીક, સૉર્ટ, દર્શાવો\end{mnemonicbox}
\end{solutionbox}

\section*{Question 5}

\questionmarks{5(a)}{3}{નીચેના ટર્ટલ ફંક્શનને યોગ્ય ઉદાહરણ સાથે સમજાવો. (a) turn() (b) move().}
\begin{solutionbox}
    \textbf{નોંધ}: સ્ટાન્ડર્ડ ટર્ટલ \code{left/right} (turn) અને \code{forward/backward} (move) વાપરે છે.

    \textbf{ફંક્શન્સ:}
    \begin{center}
        \begin{tabulary}{\linewidth}{L L L}
            \toprule
            \textbf{ફંક્શન} & \textbf{હેતુ} & \textbf{ઉદાહરણ} \\
            \midrule
            \textbf{Turn} & દિશા બદલે છે & \code{turtle.left(90)} \\
            \textbf{Move} & પોઝિશન બદલે છે & \code{turtle.forward(100)} \\
            \bottomrule
        \end{tabulary}
    \end{center}

    \begin{lstlisting}[language=Python]
import turtle
t = turtle.Turtle()
t.forward(100)  # Move
t.left(90)      # Turn
    \end{lstlisting}
    \begin{mnemonicbox}Turn દિશા બદલે, Move પોઝિશન બદલે\end{mnemonicbox}
\end{solutionbox}

\questionmarks{5(b)}{4}{ટર્ટલની દિશા બદલવાની વિવિધ ઇનબિલ્ટ પદ્ધતિઓ સમજાવો.}
\begin{solutionbox}
    \textbf{દિશા મેથડ્સ:}
    \begin{center}
        \begin{tabulary}{\linewidth}{L L L}
            \toprule
            \textbf{મેથડ} & \textbf{વર્ણન} & \textbf{ઉદાહરણ} \\
            \midrule
            \textbf{left(deg)} & ડાબે ફેરવો (વામાવર્ત) & \code{t.left(90)} \\
            \textbf{right(deg)} & જમણે ફેરવો (દક્ષિણાવર્ત) & \code{t.right(45)} \\
            \textbf{setheading(deg)} & ચોક્કસ કોણ સેટ કરો & \code{t.setheading(0)} \\
            \textbf{towards(x,y)} & પોઇન્ટ તરફનો કોણ & \code{t.towards(0,0)} \\
            \bottomrule
        \end{tabulary}
    \end{center}
    \begin{mnemonicbox}Left-Right સંબંધિત, Heading ચોક્કસ, Towards ગણતરી કરે\end{mnemonicbox}
\end{solutionbox}

\questionmarks{5(c)}{7}{ટર્ટલનો ઉપયોગ કરીને ચોરસ, લંબચોરસ અને વર્તુળ દોરવા માટેનો પ્રોગ્રામ લખો.}
\begin{solutionbox}
    \textbf{પ્રોગ્રામ:}
    \begin{lstlisting}[language=Python]
import turtle

def draw_shapes():
    t = turtle.Turtle()
    t.speed(3)
    
    # ચોરસ દોરો
    t.penup(); t.goto(-200, 50); t.pendown()
    t.write("ચોરસ")
    for _ in range(4):
        t.forward(80)
        t.right(90)
    
    # લંબચોરસ દોરો
    t.penup(); t.goto(0, 50); t.pendown()
    t.write("લંબચોરસ")
    for _ in range(2):
        t.forward(120)  # લંબાઈ
        t.right(90)
        t.forward(60)   # પહોળાઈ
        t.right(90)
    
    # વર્તુળ દોરો
    t.penup(); t.goto(200, 50); t.pendown()
    t.write("વર્તુળ")
    t.circle(40)
    
    turtle.done()

draw_shapes()
    \end{lstlisting}
    \begin{mnemonicbox}ચોરસ: 4 સમાન બાજુ, લંબચોરસ: 2 જોડી, વર્તુળ: radius મેથડ\end{mnemonicbox}
\end{solutionbox}

\questionmarks{5(a) OR}{3}{ટર્ટલમાં પેન કમાન્ડના વિવિધ પ્રકારો કયા છે? તે બધાને સમજાવો.}
\begin{solutionbox}
    \textbf{પેન કમાન્ડ્સ:}
    \begin{center}
        \begin{tabulary}{\linewidth}{L L}
            \toprule
            \textbf{કમાન્ડ} & \textbf{હેતુ} \\
            \midrule
            \textbf{penup()} & પેન ઉઠાવો (દોરવાનું બંધ) \\
            \textbf{pendown()} & પેન નીચે મૂકો (દોરવાનું શરૂ) \\
            \textbf{pensize(w)} & લાઇન જાડાઈ સેટ કરો \\
            \textbf{pencolor(c)} & લાઇન કલર સેટ કરો \\
            \textbf{fillcolor(c)} & ભરવાનો કલર સેટ કરો \\
            \textbf{begin\_fill()} & ભરવાનું શરૂ કરો \\
            \textbf{end\_fill()} & ભરવાનું બંધ કરો \\
            \bottomrule
        \end{tabulary}
    \end{center}
    \begin{mnemonicbox}Up-Down દોરવાનું કન્ટ્રોલ કરે, Size-Color દેખાવ કન્ટ્રોલ કરે\end{mnemonicbox}
\end{solutionbox}

\questionmarks{5(b) OR}{4}{ટર્ટલનો ઉપયોગ કરીને વર્તુળ અને સ્ટારના આકાર દોરો અને તેમને લાલ રંગથી ભરો.}
\begin{solutionbox}
    \textbf{પ્રોગ્રામ:}
    \begin{lstlisting}[language=Python]
import turtle

t = turtle.Turtle()
t.color("red", "red")  # પેન અને ફિલ કલર

# ભરેલું વર્તુળ
t.begin_fill()
t.circle(50)
t.end_fill()

t.penup(); t.forward(150); t.pendown()

# ભરેલો સ્ટાર
t.begin_fill()
for _ in range(5):
    t.forward(100)
    t.right(144)
t.end_fill()

turtle.done()
    \end{lstlisting}
    \begin{mnemonicbox}Begin fill, આકાર દોરો, End fill\end{mnemonicbox}
\end{solutionbox}

\questionmarks{5(c) OR}{7}{ટર્ટલનો ઉપયોગ કરીને ભારતનો ઝંડો દોરવા માટેનો પ્રોગ્રામ લખો.}
\begin{solutionbox}
    \textbf{ભારતનો ઝંડો પ્રોગ્રામ:}
    \begin{lstlisting}[language=Python]
import turtle

def draw_rect(color, x, y, width, height):
    t.penup(); t.goto(x, y); t.pendown()
    t.color(color)
    t.begin_fill()
    for _ in range(2):
        t.forward(width); t.right(90)
        t.forward(height); t.right(90)
    t.end_fill()

t = turtle.Turtle()
t.speed(5)
width = 300
height = 60

# પટ્ટીઓ દોરો
draw_rect("orange", -150, 150, width, height)
draw_rect("white", -150, 90, width, height)
draw_rect("green", -150, 30, width, height)

# ચક્ર દોરો
t.penup()
t.goto(0, 60)  # સફેદ પટ્ટીનું કેન્દ્ર
t.pendown()
t.color("navy")
t.circle(30)   # બાહ્ય વર્તુળ

# તીલીઓ
for i in range(24):
    t.penup(); t.goto(0, 90); t.pendown()
    t.setheading(i * 15)
    t.forward(30)

t.hideturtle()
turtle.done()
    \end{lstlisting}
    \begin{mnemonicbox}કેસરી-સફેદ-લીલી પટ્ટીઓ 24-તીલીવાળા ચક્ર સાથે\end{mnemonicbox}
\end{solutionbox}

\end{document}
