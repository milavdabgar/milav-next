\documentclass{article}

% content/resources/templates/preamble.tex
\usepackage[margin=0.6in]{geometry}
\author{Milav Dabgar}
\usepackage{amsmath,amssymb,amsthm}
\usepackage{booktabs}
\usepackage{multirow}
\usepackage{xcolor}
\usepackage{tcolorbox}
\tcbuselibrary{breakable,skins}
\usepackage[colorlinks=true,linkcolor=blue]{hyperref}
\usepackage{titlesec}
\usepackage{enumitem}
\usepackage{tikz}
\usepackage{pgfplots}
\usepackage{circuitikz}
\usepackage[version=4]{mhchem}
\usepackage{longtable}
\usepackage{array}
\usepackage{float}
\usepackage{caption}
\usepackage{listings}

\lstset{
  basicstyle=\small\ttfamily,
  breaklines=true,
  breakatwhitespace=false,
  postbreak=\mbox{\textcolor{red}{$\hookrightarrow$}\space},
  float=false,
  numbers=left,
  numberstyle=\tiny\color{gray},
  numbersep=10pt,
  xleftmargin=2em,
  keywordstyle=\color{blue},
  commentstyle=\color{green!60!black},
  stringstyle=\color{purple},
  backgroundcolor=\color{gray!5},
  showstringspaces=false,
  tabsize=2,
  captionpos=b,
  keepspaces=true,
  columns=flexible
}

\pgfplotsset{compat=1.18}
\usetikzlibrary{shapes,arrows,positioning,calc,patterns,decorations.pathmorphing,decorations.markings,arrows.meta}

% Color scheme
\definecolor{headcolor}{RGB}{0,102,204}
\definecolor{keycolor}{RGB}{220,20,60}
\definecolor{solutioncolor}{RGB}{34,139,34}
\definecolor{mnemoniccolor}{RGB}{148,0,211}
\definecolor{codecolor}{RGB}{0,0,100}

% Spacing
\setlength{\parskip}{3pt}
\setlist[itemize]{nosep}
\setlist[enumerate]{nosep}

% Title formatting
\titleformat{\section}{\Large\bfseries\color{headcolor}}{\thesection}{1em}{}
\titleformat{\subsection}{\large\bfseries\color{headcolor}}{\thesubsection}{1em}{}

% Pandoc tightlist compatibility
\providecommand{\tightlist}{%
  \setlength{\itemsep}{0pt}\setlength{\parskip}{0pt}}

% Pandoc longtable compatibility
\newcounter{none}
\def\thenone{}


% content/resources/templates/gujarati-boxes.tex
\usepackage{fontspec}
\usepackage{polyglossia}

% Set Gujarati as main language (document is primarily in Gujarati)
% Note: gloss-gujarati.ldf doesn't exist in polyglossia, but it will use hyphenation patterns
\setdefaultlanguage{gujarati}
\setotherlanguage{english}

% Configure Gujarati font properly
% Use Language=Default to prevent polyglossia from trying to add language-specific features
% that don't exist for Gujarati, which causes "empty feature" warnings
\newfontfamily\gujaratifont[Script=Gujarati,AutoFakeBold=2.5,AutoFakeSlant=0.3]{Noto Sans Gujarati}
\setmainfont[Script=Gujarati,AutoFakeBold=2.5,AutoFakeSlant=0.3]{Noto Sans Gujarati}
% Use Noto Sans Gujarati for monospace to support Gujarati in text
\setmonofont[Scale=0.9]{Noto Sans Gujarati}

% Configure English to use the same font
\newfontfamily\englishfont[Script=Gujarati,AutoFakeBold=2.5,AutoFakeSlant=0.3]{Noto Sans Gujarati}

% Translations for polyglossia
\gappto\captionsgujarati{
  \renewcommand{\tablename}{કોષ્ટક}
  \renewcommand{\figurename}{આકૃતિ}
}

% Helper for TikZ nodes to ensure Gujarati font
\newcommand{\gu}[1]{{\gujaratifont #1}}

% Custom environments
\newtcolorbox{solutionbox}{
    breakable,
    enhanced,
    colback=solutioncolor!5!white,
    colframe=solutioncolor!75!black,
    fonttitle=\bfseries,
    title=જવાબ
}

\newtcolorbox{solutionboxnobreak}{
 colback=solutioncolor!5!white,
 colframe=solutioncolor!75!black,
 fonttitle=\bfseries,
 title=જવાબ
}

\newtcolorbox{keyformula}{
 breakable,
 enhanced,
 colback=keycolor!5!white,
 colframe=keycolor!75!black,
 fonttitle=\bfseries,
 title=રાસાયણિક સમીકરણ/સૂત્ર
}

\newtcolorbox{mnemonicbox}{
 breakable,
 enhanced,
 colback=mnemoniccolor!5!white,
 colframe=mnemoniccolor!75!black,
 fonttitle=\bfseries,
 title=મેમરી ટ્રીક
}


% Custom commands for GTU solutions
% This file defines semantic commands for consistent formatting

% Question command with automatic formatting
\newcommand{\question}[2]{%
  \section*{Question #1}%
  \textbf{#2}%
}

% OR question variant
\newcommand{\questionor}[2]{%
  \section*{Question #1 OR}%
  \textbf{#2}%
}

% Proper table environment with caption
\newenvironment{answertable}[1]{%
  \begin{table}[htbp]
  \centering
  \caption{#1}
}{%
  \end{table}
}

% Proper figure environment for diagrams
\newenvironment{answerdiagram}[1]{%
  \begin{figure}[htbp]
  \centering
  \caption{#1}
}{%
  \end{figure}
}

% Semantic markup for key terms
\newcommand{\keyword}[1]{\textbf{#1}}
\newcommand{\code}[1]{\texttt{#1}}
\newcommand{\classname}[1]{\texttt{#1}}
\newcommand{\methodname}[1]{\texttt{#1}}

% Proper quotation marks
\newcommand{\mnemonic}[1]{``#1''}


\title{Advanced Python Programming (4321602) - Summer 2023 Solution}
\date{July 31, 2023}

\begin{document}
\maketitle

\section*{Question 1}

\questionmarks{1(a)}{3}{લિસ્ટ શું છે? તેનો પાયથનમાં ઉપયોગ શું છે અને તેની લાક્ષણિકતાઓ લખો.}
\begin{solutionbox}
    \textbf{List} એ items (elements) નું ordered collection છે જે એક જ variable માં multiple values store કરી શકે છે. લિસ્ટ mutable છે અને duplicate elements ની મંજૂરી આપે છે.

    \textbf{લાક્ષણિકતાઓ:}
    \begin{center}
        \begin{tabulary}{\linewidth}{L L}
            \toprule
            \textbf{ફીચર} & \textbf{વર્ણન} \\
            \midrule
            \textbf{Ordered} & Elements નો ક્રમ નિર્ધારિત હોય છે \\
            \textbf{Mutable} & બનાવ્યા પછી બદલી શકાય છે \\
            \textbf{Indexed} & Index [0,1,2...] વાપરીને access કરી શકાય \\
            \textbf{Duplicates} & Duplicate values ની મંજૂરી છે \\
            \bottomrule
        \end{tabulary}
    \end{center}

    \textbf{પાયથનમાં ઉપયોગ:}
    \begin{itemize}
        \item \textbf{Data Storage}: સંબંધિત items નો સંગ્રહ.
        \item \textbf{Dynamic Arrays}: Runtime દરમિયાન size બદલી શકાય.
        \item \textbf{Iteration}: Elements માં આસાનીથી loop કરી શકાય.
    \end{itemize}
    \begin{mnemonicbox}OMID - Ordered, Mutable, Indexed, Duplicates\end{mnemonicbox}
\end{solutionbox}

\questionmarks{1(b)}{4}{પાયથનમાં String built-in functions સમજાવો.}
\begin{solutionbox}
    String built-in functions પાયથન પ્રોગ્રામમાં text data ને efficiently manipulate અને process કરવામાં મદદ કરે છે.

    \textbf{સામાન્ય String Functions:}
    \begin{center}
        \begin{tabulary}{\linewidth}{L L L}
            \toprule
            \textbf{Function} & \textbf{હેતુ} & \textbf{ઉદાહરણ} \\
            \midrule
            \textbf{upper()} & Uppercase માં convert કરે & \code{"hello".upper()} $\to$ "HELLO" \\
            \textbf{lower()} & Lowercase માં convert કરે & \code{"WORLD".lower()} $\to$ "world" \\
            \textbf{strip()} & Whitespace remove કરે & \code{" hi ".strip()} $\to$ "hi" \\
            \textbf{split()} & List માં split કરે & \code{"a,b".split(",")} $\to$ ['a','b'] \\
            \textbf{replace()} & Substring replace કરે & \code{"cat".replace("c","b")} $\to$ "bat" \\
            \textbf{find()} & Substring position શોધે & \code{"hello".find("e")} $\to$ 1 \\
            \bottomrule
        \end{tabulary}
    \end{center}

    \textbf{મુખ્ય મુદ્દાઓ:}
    \begin{itemize}
        \item \textbf{Immutable}: Original string અપરિવર્તિત રહે છે.
        \item \textbf{Return Values}: Functions નવી strings return કરે છે.
        \item \textbf{Case Sensitive}: Functions case ને ધ્યાનમાં રાખે છે.
    \end{itemize}
    \begin{mnemonicbox}ULSR-FR - Upper, Lower, Strip, Replace, Find, Replace\end{mnemonicbox}
\end{solutionbox}

\questionmarks{1(c)}{7}{સેટમાંથી કોઈ element કેવી રીતે ઉમેરવું, દૂર કરવું તે લખો. POP remove થી કઈ રીતે અલગ છે તે સમજાવો.}
\begin{solutionbox}
    \textbf{Sets} એ unique elements નો unordered collection છે.

    \textbf{Set Operations:}
    \begin{center}
        \begin{tabulary}{\linewidth}{L L L L}
            \toprule
            \textbf{Operation} & \textbf{Method} & \textbf{Syntax} & \textbf{ઉદાહરણ} \\
            \midrule
            \textbf{Add} & add() & \code{set.add(e)} & \code{s.add(5)} \\
            \textbf{Remove} & remove() & \code{set.remove(e)} & \code{s.remove(3)} \\
            \textbf{Safe Remove} & discard() & \code{set.discard(e)} & \code{s.discard(7)} \\
            \textbf{Pop} & pop() & \code{set.pop()} & \code{s.pop()} \\
            \bottomrule
        \end{tabulary}
    \end{center}

    \textbf{Code ઉદાહરણ:}
    \begin{lstlisting}[language=Python]
my_set = {1, 2, 3}
my_set.add(5)        # Add
my_set.remove(2)     # Remove specific
element = my_set.pop() # Remove random
    \end{lstlisting}

    \textbf{POP vs REMOVE તફાવત:}
    \begin{center}
        \begin{tabulary}{\linewidth}{L L L}
            \toprule
            \textbf{પાસું} & \textbf{pop()} & \textbf{remove()} \\
            \midrule
            \textbf{Target} & Random element & Specific element \\
            \textbf{Parameter} & જરૂરી નથી & Element value જરૂરી \\
            \textbf{Return} & Removed element & None \\
            \textbf{Error} & Set empty હોય તો error & Element ન મળે તો error \\
            \bottomrule
        \end{tabulary}
    \end{center}
    \begin{mnemonicbox}PRRE - Pop Random, Remove Exact\end{mnemonicbox}
\end{solutionbox}

\questionmarks{1(c) OR}{7}{બિલ્ટ-ઇન Dictionary functions ની યાદી લખો. Dictionary ના functions અને operations દર્શાવવા માટે પ્રોગ્રામ લખો.}
\begin{solutionbox}
    \textbf{Dictionary Functions:}
    \begin{center}
        \begin{tabulary}{\linewidth}{L L L}
            \toprule
            \textbf{Function} & \textbf{હેતુ} & \textbf{Return કરે છે} \\
            \midrule
            \textbf{keys()} & બધી keys મેળવે & dict\_keys object \\
            \textbf{values()} & બધી values મેળવે & dict\_values object \\
            \textbf{items()} & Key-value pairs મેળવે & dict\_items object \\
            \textbf{get()} & Safe value retrieval & Value અથવા None \\
            \textbf{pop()} & Remove કરીને value return કરે & Removed value \\
            \textbf{clear()} & બધી items remove કરે & None \\
            \textbf{update()} & Dictionaries merge કરે & None \\
            \bottomrule
        \end{tabulary}
    \end{center}

    \textbf{પ્રોગ્રામ ઉદાહરણ:}
    \begin{lstlisting}[language=Python]
student = {'name': 'John', 'age': 20}

# Operations
print(list(student.keys()))    # Keys
print(student.get('age'))      # Safe get
student.update({'grade': 'A'}) # Update
student.pop('age')             # Remove
    \end{lstlisting}
    \begin{mnemonicbox}KVIGPCU - Keys, Values, Items, Get, Pop, Clear, Update\end{mnemonicbox}
\end{solutionbox}

\section*{Question 2}

\questionmarks{2(a)}{3}{Tuple ની વ્યાખ્યા લખો અને તે કઈ રીતે પાયથનમાં બનાવાય?}
\begin{solutionbox}
    \textbf{Tuple} એ ordered collection છે જે immutable છે (બનાવ્યા પછી બદલી શકાતી નથી).

    \textbf{Tuple Creation Methods:}
    \begin{center}
        \begin{tabulary}{\linewidth}{L L L}
            \toprule
            \textbf{Method} & \textbf{Syntax} & \textbf{ઉદાહરણ} \\
            \midrule
            \textbf{Parentheses} & (item1, item2) & (1, 2, 3) \\
            \textbf{No Parentheses} & item1, item2 & 1, 2, 3 \\
            \textbf{Single Item} & (item,) & (5,) \\
            \textbf{Empty TC} & () & () \\
            \bottomrule
        \end{tabulary}
    \end{center}
    \begin{mnemonicbox}IOI - Immutable, Ordered, Indexed\end{mnemonicbox}
\end{solutionbox}

\questionmarks{2(b)}{4}{Module ના ફાયદાઓ સમજાવો.}
\begin{solutionbox}
    \textbf{Modules} એ Python files છે જેમાં functions, classes અને variables હોય છે.

    \textbf{ફાયદાઓ:}
    \begin{center}
        \begin{tabulary}{\linewidth}{L L}
            \toprule
            \textbf{ફાયદો} & \textbf{લાભ} \\
            \midrule
            \textbf{Reusability} & એક વાર લખો, ઘણી વાર વાપરો \\
            \textbf{Organization} & Code ને logical units માં વિભાજિત કરે \\
            \textbf{Namespace} & Naming conflicts ટાળે \\
            \textbf{Maintainability} & Easy debugging અને updates \\
            \bottomrule
        \end{tabulary}
    \end{center}
    \begin{mnemonicbox}RONM - Reusability, Organization, Namespace, Maintainability\end{mnemonicbox}
\end{solutionbox}

\questionmarks{2(c)}{7}{યોગ્ય ઉદાહરણ સાથે user defined package બનાવવા માટેના steps લખો.}
\begin{solutionbox}
    \textbf{Package} એ directory છે જેમાં multiple modules અને \code{\_\_init\_\_.py} હોય છે.

    \textbf{Package બનાવવાના Steps:}
    \begin{center}
        \begin{tikzpicture}[node distance=2cm, auto]
            \node [gtu state, text width=3cm, align=center] (create) {Package Directory બનાવો};
            \node [gtu state, right of=create, xshift=2cm, text width=3cm, align=center] (init) {\code{\_\_init\_\_.py} બનાવો};
            \node [gtu state, right of=init, xshift=2cm, text width=3cm, align=center] (modules) {Module Files બનાવો};
            \node [gtu state, below of=create, yshift=-1cm, text width=3cm, align=center] (funcs) {Functions લખો};
            \node [gtu state, right of=funcs, xshift=2cm, text width=3cm, align=center] (use) {Import અને વાપરો};

            \draw [gtu arrow] (create) -- (init);
            \draw [gtu arrow] (init) -- (modules);
            \draw [gtu arrow] (modules) -- (funcs);
            \draw [gtu arrow] (funcs) -- (use);
        \end{tikzpicture}
    \end{center}

    \textbf{Step-by-Step Implementation:}
    \begin{enumerate}
        \item \textbf{Create Directory}: \code{mkdir mathtools}
        \item \textbf{Create \code{\_\_init\_\_.py}}:
        \begin{lstlisting}[language=Python]
# mathtools/__init__.py
print("Package loaded")
        \end{lstlisting}
        \item \textbf{Create Module (basic.py)}:
        \begin{lstlisting}[language=Python]
def add(a, b): return a + b
        \end{lstlisting}
        \item \textbf{Use Package}:
        \begin{lstlisting}[language=Python]
import mathtools.basic
print(mathtools.basic.add(5, 3))
        \end{lstlisting}
    \end{enumerate}
    \begin{mnemonicbox}DDMFU - Directory, Dunder-init, Modules, Functions, Use\end{mnemonicbox}
\end{solutionbox}

\questionmarks{2(a) OR}{3}{Tuple અને List વચ્ચેનો તફાવત લખો.}
\begin{solutionbox}
    \textbf{તફાવત:}
    \begin{center}
        \begin{tabulary}{\linewidth}{L L L}
            \toprule
            \textbf{ફીચર} & \textbf{Tuple} & \textbf{List} \\
            \midrule
            \textbf{Mutability} & Immutable & Mutable \\
            \textbf{Syntax} & (1, 2, 3) & [1, 2, 3] \\
            \textbf{Performance} & ઝડપી & ધીમું \\
            \textbf{Methods} & મર્યાદિત & ઘણી methods \\
            \textbf{Memory} & ઓછી memory & વધુ memory \\
            \bottomrule
        \end{tabulary}
    \end{center}
    \begin{mnemonicbox}TIF-LIM - Tuple Immutable Fixed, List Mutable Dynamic\end{mnemonicbox}
\end{solutionbox}

\questionmarks{2(b) OR}{4}{પાયથનમાં intra-package reference concept સમજાવો.}
\begin{solutionbox}
    \textbf{Intra-package references} package અંદરના modules ને relative imports વાપરીને એકબીજાને import કરવાની મંજૂરી આપે છે.

    \textbf{Import પ્રકારો:}
    \begin{center}
        \begin{tabulary}{\linewidth}{L L L}
            \toprule
            \textbf{Type} & \textbf{Syntax} & \textbf{Usage} \\
            \midrule
            \textbf{Absolute} & \code{from pkg.mod import fn} & Full path \\
            \textbf{Relative} & \code{from .mod import fn} & Same package \\
            \textbf{Parent} & \code{from ..mod import fn} & Parent package \\
            \bottomrule
        \end{tabulary}
    \end{center}
    \begin{mnemonicbox}RAP - Relative, Absolute, Parent imports\end{mnemonicbox}
\end{solutionbox}

\questionmarks{2(c) OR}{7}{Module એટલે શું? વર્તુળનું ક્ષેત્રફળ અને પરિઘ શોધવા માટે module બનાવવાનો પ્રોગ્રામ લખો. આ module ને પ્રોગ્રામમાં import કરો અને તેમાંથી functions call કરો.}
\begin{solutionbox}
    \textbf{Module} એ Python file છે જેમાં functions અને variables હોય છે.

    \textbf{1. Circle Module (circle.py):}
    \begin{lstlisting}[language=Python]
import math

def area(r):
    return math.pi * r * r

def circumference(r):
    return 2 * math.pi * r
    \end{lstlisting}

    \textbf{2. Main Program (main.py):}
    \begin{lstlisting}[language=Python]
import circle

r = 5
print(f"Area: {circle.area(r):.2f}")
print(f"Circumference: {circle.circumference(r):.2f}")
    \end{lstlisting}
    \begin{mnemonicbox}IRUD - Import, Reuse, Use, Debug\end{mnemonicbox}
\end{solutionbox}

\section*{Question 3}

\questionmarks{3(a)}{3}{પાયથનમાં errors ના પ્રકારો સમજાવો.}
\begin{solutionbox}
    \textbf{Errors} ત્યારે આવે છે જ્યારે code properly execute ન થઈ શકે.

    \textbf{Error પ્રકારો:}
    \begin{center}
        \begin{tabulary}{\linewidth}{L L L}
            \toprule
            \textbf{Error Type} & \textbf{વર્ણન} & \textbf{ઉદાહરણ} \\
            \midrule
            \textbf{Syntax Error} & Code structure ખોટું & Colon, brackets ગુમ \\
            \textbf{Runtime Error} & Execution દરમિયાન error & Zero થી division \\
            \textbf{Logical Error} & ખોટું result & ખોટું formula \\
            \bottomrule
        \end{tabulary}
    \end{center}
    \begin{mnemonicbox}SRL - Syntax, Runtime, Logical\end{mnemonicbox}
\end{solutionbox}

\questionmarks{3(b)}{4}{try except નું structure સમજાવો.}
\begin{solutionbox}
    \textbf{Try-except} structure runtime errors ને gracefully handle કરે છે.

    \textbf{Basic Structure:}
    \begin{center}
        \begin{tikzpicture}[node distance=2.5cm, auto]
            \node [gtu state] (try) {try block};
            \node [gtu state, right of=try, xshift=2cm] (exec) {Code execution};
            \node [gtu decision, right of=exec, xshift=2cm] (error) {Error?};
            \node [gtu state, below of=error, yshift=-1cm] (except) {except block};
            \node [gtu state, right of=error, xshift=3cm] (cont) {Continue};

            \draw [gtu arrow] (try) -- (exec);
            \draw [gtu arrow] (exec) -- (error);
            \draw [gtu arrow] (error) -- node [near start] {No} (cont);
            \draw [gtu arrow] (error) -- node [near start] {Yes} (except);
            \draw [gtu arrow] (except) -| (cont);
        \end{tikzpicture}
    \end{center}

    \textbf{Syntax:}
    \begin{lstlisting}[language=Python]
try:
    # Risky code
    code()
except ErrorType:
    # Handle error
    handle()
finally:
    # Always execute
    cleanup()
    \end{lstlisting}
    \begin{mnemonicbox}TEEF - Try, Except, Else, Finally\end{mnemonicbox}
\end{solutionbox}

\questionmarks{3(c)}{7}{Marks Result માટે એક function બનાવો જેમાં English અને Maths marks ની બે arguments હોય, જો કોઈપણ argument નું value 0 કરતાં ઓછું હોય તો error generate કરાવો.}
\begin{solutionbox}
    \textbf{Implementation:}
    \begin{lstlisting}[language=Python]
class InvalidMarksError(Exception):
    def __init__(self, subject, marks):
        super().__init__(f"Invalid {subject} marks: {marks}")

def marks_result(english, maths):
    if english < 0: raise InvalidMarksError("English", english)
    if maths < 0: raise InvalidMarksError("Mathematics", maths)

    if english > 100: raise InvalidMarksError("English", english)
    if maths > 100: raise InvalidMarksError("Mathematics", maths)

    total = english + maths
    percentage = (total / 200) * 100

    if percentage >= 50: status = 'Pass'
    else: status = 'Fail'

    return {
        'total': total,
        'percentage': percentage,
        'status': status
    }
    \end{lstlisting}
    \begin{mnemonicbox}CVIR - Custom, Validate, Interactive, Robust\end{mnemonicbox}
\end{solutionbox}

\questionmarks{3(a) OR}{3}{પાયથનમાં built-in exceptions ની યાદી લખો (કોઈપણ પાંચ).}
\begin{solutionbox}
    \textbf{Built-in Exceptions:}
    \begin{center}
        \begin{tabulary}{\linewidth}{L L L}
            \toprule
            \textbf{Exception} & \textbf{કારણ} & \textbf{ઉદાહરણ} \\
            \midrule
            \textbf{ValueError} & અમાન્ય મૂલ્ય & \code{int("abc")} \\
            \textbf{TypeError} & ખોટો data type & \code{"5"+5} \\
            \textbf{IndexError} & Index range બહાર & \code{list[10]} \\
            \textbf{KeyError} & Key ન મળે & \code{dict["x"]} \\
            \textbf{ZeroDivisionError} & Zero થી ભાગાકાર & \code{10/0} \\
            \bottomrule
        \end{tabulary}
    \end{center}
    \begin{mnemonicbox}VTIKZ - ValueError, TypeError, IndexError, KeyError, ZeroDivisionError\end{mnemonicbox}
\end{solutionbox}

\questionmarks{3(b) OR}{4}{finally પર મુદ્દા લખો અને ઉદાહરણ સાથે સમજાવો.}
\begin{solutionbox}
    \textbf{Finally Block:} exception આવે કે ન આવે, હંમેશા execute થાય છે.

    \textbf{લાક્ષણિકતાઓ:}
    \begin{itemize}
        \item \textbf{Always Executes}: હંમેશા run થાય.
        \item \textbf{Cleanup}: Resources close કરવા માટે ઉત્તમ.
    \end{itemize}

    \textbf{ઉદાહરણ:}
    \begin{lstlisting}[language=Python]
try:
    file = open("data.txt", "r")
except:
    print("Error")
finally:
    print("Cleanup")
    file.close()
    \end{lstlisting}
    \begin{mnemonicbox}ARGC - Always Runs, Resource Cleanup\end{mnemonicbox}
\end{solutionbox}

\questionmarks{3(c) OR}{7}{Divide by Zero Exception ને finally clause સાથે catch કરતો પ્રોગ્રામ લખો.}
\begin{solutionbox}
    \textbf{Program:}
    \begin{lstlisting}[language=Python]
def safe_divide(a, b):
    try:
        print(f"Dividing {a} by {b}")
        result = a / b
        print(f"Result: {result}")
    except ZeroDivisionError:
        print("Error: Zero division not allowed!")
    except TypeError:
        print("Error: Numbers required!")
    finally:
        print("Operation completed")

safe_divide(10, 2)
safe_divide(5, 0)
    \end{lstlisting}
    \begin{mnemonicbox}CFLIS - Comprehensive, Finally, Logging, Interactive, Statistics\end{mnemonicbox}
\end{solutionbox}

\section*{Question 4}

\questionmarks{4(a)}{3}{File Handling શું છે? File Handling Operations ની યાદી આપો.}
\begin{solutionbox}
    \textbf{File Handling} એ files ને read, write અને manipulate કરવાની પ્રક્રિયા છે.

    \textbf{Operations:}
    \begin{center}
        \begin{tabulary}{\linewidth}{L L L}
            \toprule
            \textbf{Operation} & \textbf{હેતુ} & \textbf{Method} \\
            \midrule
            \textbf{Open} & File ખોલવા & \code{open()} \\
            \textbf{Read} & વાંચવા & \code{read()} \\
            \textbf{Write} & લખવા & \code{write()} \\
            \textbf{Close} & બંધ કરવા & \code{close()} \\
            \bottomrule
        \end{tabulary}
    \end{center}
    \begin{mnemonicbox}ORWCST - Open, Read, Write, Close, Seek, Tell\end{mnemonicbox}
\end{solutionbox}

\questionmarks{4(b)}{4}{Object Serialization સમજાવો.}
\begin{solutionbox}
    \textbf{Object Serialization} એ Python objects ને byte stream માં convert કરવાની પ્રક્રિયા છે.

    \textbf{Methods:}
    \begin{itemize}
        \item \textbf{Pickle}: Python specific binary format.
        \item \textbf{JSON}: Text format, web માટે ઉપયોગી.
    \end{itemize}

    \textbf{Pickle ઉદાહરણ:}
    \begin{lstlisting}[language=Python]
import pickle
data = [1, 2, 3]
# Serialize
with open('data.pkl', 'wb') as f:
    pickle.dump(data, f)
    \end{lstlisting}
    \begin{mnemonicbox}SPDT - Store, Persist, Data Transfer\end{mnemonicbox}
\end{solutionbox}

\questionmarks{4(c)}{7}{File માં રહેલા vowels ગણવાનો પ્રોગ્રામ લખો.}
\begin{solutionbox}
    \textbf{Vowel Counter:}
    \begin{lstlisting}[language=Python]
def count_vowels(filename):
    vowels = 'aeiouAEIOU'
    count = 0
    try:
        with open(filename, 'r') as f:
            text = f.read()
            for char in text:
                if char in vowels:
                    count += 1
        print(f"Total Vowels: {count}")
    except FileNotFoundError:
        print("File not found")

# Test
with open("test.txt", "w") as f:
    f.write("Hello World")
count_vowels("test.txt")
    \end{lstlisting}
    \begin{mnemonicbox}FVESI - File Validation, Vowel Extraction, Statistics, Interactive\end{mnemonicbox}
\end{solutionbox}

\questionmarks{4(a) OR}{3}{File કેવી રીતે open અને close કરવી? તેની syntax આપો.}
\begin{solutionbox}
    \textbf{Modes:} 'r' (Read), 'w' (Write), 'a' (Append).

    \textbf{Syntax:}
    \begin{lstlisting}[language=Python]
# Manual
f = open("file.txt", "r")
f.close()

# With statement (Recommended)
with open("file.txt", "r") as f:
    data = f.read()
    \end{lstlisting}
    \begin{mnemonicbox}ORWA - Open, Read, Write, Append modes\end{mnemonicbox}
\end{solutionbox}

\questionmarks{4(b) OR}{4}{Text file અને Binary file વચ્ચેનો તફાવત લખો.}
\begin{solutionbox}
    \textbf{તફાવત:}
    \begin{center}
        \begin{tabulary}{\linewidth}{L L L}
            \toprule
            \textbf{પાસું} & \textbf{Text File} & \textbf{Binary File} \\
            \midrule
            \textbf{Content} & Human readable chars & Bytes \\
            \textbf{Mode} & 'r', 'w' & 'rb', 'wb' \\
            \textbf{Encoding} & ASCII/UTF-8 & None \\
            \textbf{Size} & મોટી & નાની (compact) \\
            \bottomrule
        \end{tabulary}
    \end{center}
    \begin{mnemonicbox}TCEB - Text Character Encoding Bigger, Binary Compact Efficient\end{mnemonicbox}
\end{solutionbox}

\questionmarks{4(c) OR}{7}{Seat no અને Name store કરવા માટે binary file બનાવવાનો પ્રોગ્રામ લખો. Seat no થી search કરી name display કરો.}
\begin{solutionbox}
    \textbf{Program:}
    \begin{lstlisting}[language=Python]
import pickle

# Add record
def add(seat, name):
    record = {seat: name}
    with open("student.dat", "wb") as f:
        pickle.dump(record, f)

# Search
def search(seat):
    try:
        with open("student.dat", "rb") as f:
            d = pickle.load(f)
            if seat in d:
                print(f"Found: {d[seat]}")
            else:
                print("Not found")
    except:
        print("Error")

add(1, "Ram")
search(1)
    \end{lstlisting}
    \begin{mnemonicbox}BSECH - Binary Storage, Search Efficiently, CRUD Handling\end{mnemonicbox}
\end{solutionbox}

\section*{Question 5}

\questionmarks{5(a)}{3}{Turtle શું છે અને objects draw કરવા માટે તેનો ઉપયોગ કેવી રીતે થાય છે?}
\begin{solutionbox}
    \textbf{Turtle} એ Python graphics module છે જે drawing canvas અને cursor (turtle) પ્રદાન કરે છે.

    \textbf{ઉદાહરણ:}
    \begin{lstlisting}[language=Python]
import turtle
t = turtle.Turtle()
t.forward(100)
    \end{lstlisting}
    \begin{mnemonicbox}CPTT - Canvas, Pen, Turtle, Teaching tool\end{mnemonicbox}
\end{solutionbox}

\questionmarks{5(b)}{4}{Turtle ને બીજી position પર move કરવા માટેની વિવિધ રીતો સમજાવો.}
\begin{solutionbox}
    \textbf{Movement Methods:}
    \begin{center}
        \begin{tabulary}{\linewidth}{L L}
            \toprule
            \textbf{Method} & \textbf{કાર્ય} \\
            \midrule
            \textbf{forward(d)} & આગળ વધે \\
            \textbf{backward(d)} & પાછળ જાય \\
            \textbf{goto(x,y)} & (x,y) પર જાય \\
            \textbf{penup()} & Drawing બંધ કરે \\
            \textbf{pendown()} & Drawing શરૂ કરે \\
            \bottomrule
        \end{tabulary}
    \end{center}
    \begin{mnemonicbox}FGPRS - Forward, Goto, Penup, Rotate, Set coordinates\end{mnemonicbox}
\end{solutionbox}

\questionmarks{5(c)}{7}{Turtle માં loops કેવી રીતે ઉપયોગી છે તે સમજાવો અને ઉદાહરણ આપો.}
\begin{solutionbox}
    \textbf{Loops} નો ઉપયોગ patterns અને shapes ને efficiently draw કરવા માટે થાય છે.

    \textbf{ઉદાહરણ (Square Loop):}
    \begin{lstlisting}[language=Python]
import turtle
t = turtle.Turtle()

for i in range(4):
    t.forward(100)
    t.right(90)
    \end{lstlisting}
    \begin{mnemonicbox}LPDC - Loops, Patterns, DynamicGraphics, ComplexDesigns\end{mnemonicbox}
\end{solutionbox}

\questionmarks{5(a) OR}{3}{Turtle માં Shape function સમજાવો. કેટલા પ્રકારના shapes ઉપલબ્ધ છે?}
\begin{solutionbox}
    \textbf{Shape function} cursor નો દેખાવ બદલે છે.

    \textbf{Shapes:} Arrow, Turtle, Circle, Square, Triangle, Classic.
    \begin{lstlisting}[language=Python]
t.shape("turtle")
    \end{lstlisting}
    \begin{mnemonicbox}ATCSTC - Arrow, Turtle, Circle, Square, Triangle, Classic\end{mnemonicbox}
\end{solutionbox}

\questionmarks{5(b) OR}{4}{Turtle માં વિવિધ પ્રકારના pen commands સમજાવો.}
\begin{solutionbox}
    \textbf{Pen Commands:}
    \begin{itemize}
        \item \textbf{penup()}: Drawing અટકાવે.
        \item \textbf{pendown()}: Drawing શરૂ કરે.
        \item \textbf{pensize(w)}: Line જાડાઈ નક્કી કરે.
        \item \textbf{pencolor(c)}: Line કલર નક્કી કરે.
    \end{itemize}
\end{solutionbox}

\questionmarks{5(c) OR}{7}{Turtle નો ઉપયોગ કરીને ભારતીય ધ્વજ દોરવા માટેનો પ્રોગ્રામ લખો.}
\begin{solutionbox}
    \textbf{ભારતીય ધ્વજ Program:}
    \begin{lstlisting}[language=Python]
import turtle

def draw_rect(color, x, y, width, height):
    t.penup()
    t.goto(x, y)
    t.pendown()
    t.color(color)
    t.begin_fill()
    for _ in range(2):
        t.forward(width)
        t.right(90)
        t.forward(height)
        t.right(90)
    t.end_fill()

t = turtle.Turtle()
t.speed(5)
width = 300
height = 50

# પટ્ટીઓ દોરો
draw_rect("orange", -150, 100, width, height)
draw_rect("white", -150, 50, width, height)
draw_rect("green", -150, 0, width, height)

# ચક્ર દોરો
t.penup()
t.goto(0, 0)
t.pendown()
t.color("navy")
t.circle(25)
# Spokes
for i in range(24):
    t.penup()
    t.goto(0, 25)
    t.pendown()
    t.forward(25)
    t.backward(25)
    t.right(15)
    \end{lstlisting}
    \begin{mnemonicbox}SWACP - Stripes, White-chakra, Accurate, Colors, Proportional\end{mnemonicbox}
\end{solutionbox}

\end{document}
