\documentclass{article}

% content/resources/templates/preamble.tex
\usepackage[margin=0.6in]{geometry}
\author{Milav Dabgar}
\usepackage{amsmath,amssymb,amsthm}
\usepackage{booktabs}
\usepackage{multirow}
\usepackage{xcolor}
\usepackage{tcolorbox}
\tcbuselibrary{breakable,skins}
\usepackage[colorlinks=true,linkcolor=blue]{hyperref}
\usepackage{titlesec}
\usepackage{enumitem}
\usepackage{tikz}
\usepackage{pgfplots}
\usepackage{circuitikz}
\usepackage[version=4]{mhchem}
\usepackage{longtable}
\usepackage{array}
\usepackage{float}
\usepackage{caption}
\usepackage{listings}

\lstset{
  basicstyle=\small\ttfamily,
  breaklines=true,
  breakatwhitespace=false,
  postbreak=\mbox{\textcolor{red}{$\hookrightarrow$}\space},
  float=false,
  numbers=left,
  numberstyle=\tiny\color{gray},
  numbersep=10pt,
  xleftmargin=2em,
  keywordstyle=\color{blue},
  commentstyle=\color{green!60!black},
  stringstyle=\color{purple},
  backgroundcolor=\color{gray!5},
  showstringspaces=false,
  tabsize=2,
  captionpos=b,
  keepspaces=true,
  columns=flexible
}

\pgfplotsset{compat=1.18}
\usetikzlibrary{shapes,arrows,positioning,calc,patterns,decorations.pathmorphing,decorations.markings,arrows.meta}

% Color scheme
\definecolor{headcolor}{RGB}{0,102,204}
\definecolor{keycolor}{RGB}{220,20,60}
\definecolor{solutioncolor}{RGB}{34,139,34}
\definecolor{mnemoniccolor}{RGB}{148,0,211}
\definecolor{codecolor}{RGB}{0,0,100}

% Spacing
\setlength{\parskip}{3pt}
\setlist[itemize]{nosep}
\setlist[enumerate]{nosep}

% Title formatting
\titleformat{\section}{\Large\bfseries\color{headcolor}}{\thesection}{1em}{}
\titleformat{\subsection}{\large\bfseries\color{headcolor}}{\thesubsection}{1em}{}

% Pandoc tightlist compatibility
\providecommand{\tightlist}{%
  \setlength{\itemsep}{0pt}\setlength{\parskip}{0pt}}

% Pandoc longtable compatibility
\newcounter{none}
\def\thenone{}


% content/resources/templates/gujarati-boxes.tex
\usepackage{fontspec}
\usepackage{polyglossia}

% Set Gujarati as main language (document is primarily in Gujarati)
% Note: gloss-gujarati.ldf doesn't exist in polyglossia, but it will use hyphenation patterns
\setdefaultlanguage{gujarati}
\setotherlanguage{english}

% Configure Gujarati font properly
% Use Language=Default to prevent polyglossia from trying to add language-specific features
% that don't exist for Gujarati, which causes "empty feature" warnings
\newfontfamily\gujaratifont[Script=Gujarati,AutoFakeBold=2.5,AutoFakeSlant=0.3]{Noto Sans Gujarati}
\setmainfont[Script=Gujarati,AutoFakeBold=2.5,AutoFakeSlant=0.3]{Noto Sans Gujarati}
% Use Noto Sans Gujarati for monospace to support Gujarati in text
\setmonofont[Scale=0.9]{Noto Sans Gujarati}

% Configure English to use the same font
\newfontfamily\englishfont[Script=Gujarati,AutoFakeBold=2.5,AutoFakeSlant=0.3]{Noto Sans Gujarati}

% Translations for polyglossia
\gappto\captionsgujarati{
  \renewcommand{\tablename}{કોષ્ટક}
  \renewcommand{\figurename}{આકૃતિ}
}

% Helper for TikZ nodes to ensure Gujarati font
\newcommand{\gu}[1]{{\gujaratifont #1}}

% Custom environments
\newtcolorbox{solutionbox}{
    breakable,
    enhanced,
    colback=solutioncolor!5!white,
    colframe=solutioncolor!75!black,
    fonttitle=\bfseries,
    title=જવાબ
}

\newtcolorbox{solutionboxnobreak}{
 colback=solutioncolor!5!white,
 colframe=solutioncolor!75!black,
 fonttitle=\bfseries,
 title=જવાબ
}

\newtcolorbox{keyformula}{
 breakable,
 enhanced,
 colback=keycolor!5!white,
 colframe=keycolor!75!black,
 fonttitle=\bfseries,
 title=રાસાયણિક સમીકરણ/સૂત્ર
}

\newtcolorbox{mnemonicbox}{
 breakable,
 enhanced,
 colback=mnemoniccolor!5!white,
 colframe=mnemoniccolor!75!black,
 fonttitle=\bfseries,
 title=મેમરી ટ્રીક
}


% Custom commands for GTU solutions
% This file defines semantic commands for consistent formatting

% Question command with automatic formatting
\newcommand{\question}[2]{%
  \section*{Question #1}%
  \textbf{#2}%
}

% OR question variant
\newcommand{\questionor}[2]{%
  \section*{Question #1 OR}%
  \textbf{#2}%
}

% Proper table environment with caption
\newenvironment{answertable}[1]{%
  \begin{table}[htbp]
  \centering
  \caption{#1}
}{%
  \end{table}
}

% Proper figure environment for diagrams
\newenvironment{answerdiagram}[1]{%
  \begin{figure}[htbp]
  \centering
  \caption{#1}
}{%
  \end{figure}
}

% Semantic markup for key terms
\newcommand{\keyword}[1]{\textbf{#1}}
\newcommand{\code}[1]{\texttt{#1}}
\newcommand{\classname}[1]{\texttt{#1}}
\newcommand{\methodname}[1]{\texttt{#1}}

% Proper quotation marks
\newcommand{\mnemonic}[1]{``#1''}


\title{Advanced Python Programming (4321602) - Winter 2023 Solution}
\date{January 24, 2024}

\begin{document}
\maketitle

\section*{Question 1}

\questionmarks{1(a)}{3}{What is Dictionary? Explain with example.}
\begin{solutionbox}
    \textbf{Dictionary} એ Python માં key-value pairs નું collection છે જે mutable અને ordered છે.

    \textbf{Dictionary ના ગુણધર્મો:}
    \begin{center}
        \begin{tabulary}{\linewidth}{L L}
            \toprule
            \textbf{ગુણધર્મ} & \textbf{વર્ણન} \\
            \midrule
            \textbf{Mutable} & Values બદલી શકાય છે \\
            \textbf{Ordered} & Insertion order જળવાઈ રહે છે (Python 3.7+) \\
            \textbf{Indexed} & Keys દ્વારા access થાય છે \\
            \textbf{No Duplicates} & Duplicate keys allow નથી \\
            \bottomrule
        \end{tabulary}
    \end{center}

    \begin{lstlisting}[language=Python]
# Dictionary ઉદાહરણ
student = {
    "name": "Raj",
    "age": 20,
    "course": "IT"
}
print(student["name"])  # Output: Raj
    \end{lstlisting}

    \begin{itemize}
        \item \textbf{Key-Value Structure}: દરેક element પાસે key અને value હોય છે
        \item \textbf{Fast Access}: Data access O(1) time complexity માં થાય છે
        \item \textbf{Dynamic Size}: Runtime એ size વધી કે ઘટી શકે છે
    \end{itemize}
    \begin{mnemonicbox}Dictionary = Key Value Treasure\end{mnemonicbox}
\end{solutionbox}

\questionmarks{1(b)}{4}{Explain Tuple Built-in functions and methods.}
\begin{solutionbox}
    Tuple immutable હોવાથી તેની પાસે limited built-in methods છે.

    \textbf{Tuple મેથડ્સ:}
    \begin{center}
        \begin{tabulary}{\linewidth}{L L L}
            \toprule
            \textbf{મેથડ} & \textbf{વર્ણન} & \textbf{ઉદાહરણ} \\
            \midrule
            \textbf{count()} & Element કેટલી વાર આવે છે તે આપે & \code{t.count(5)} \\
            \textbf{index()} & Element નો પ્રથમ index આપે & \code{t.index('a')} \\
            \textbf{len()} & Tuple ની length આપે & \code{len(t)} \\
            \textbf{max()} & Maximum value આપે & \code{max(t)} \\
            \textbf{min()} & Minimum value આપે & \code{min(t)} \\
            \bottomrule
        \end{tabulary}
    \end{center}

    \begin{lstlisting}[language=Python]
# Tuple Methods ઉદાહરણ
numbers = (1, 2, 3, 2, 4, 2)
print(numbers.count(2))     # Output: 3
print(numbers.index(3))     # Output: 2
print(len(numbers))         # Output: 6
    \end{lstlisting}

    \begin{itemize}
        \item \textbf{Immutable Nature}: Methods tuple ને modify કરતા નથી
        \item \textbf{Return Values}: બધી methods નવી values return કરે છે
        \item \textbf{Type Conversion}: \code{tuple()} function નો ઉપયોગ list ને tuple માં ફેરવવા થાય છે
    \end{itemize}
    \begin{mnemonicbox}Count Index Length Max Min\end{mnemonicbox}
\end{solutionbox}

\questionmarks{1(c)}{7}{Write a python program to demonstrate set operations.}
\begin{solutionbox}
    Set operations mathematical set theory પર આધારિત છે.

    \textbf{Set Operations:}
    \begin{center}
        \begin{tabulary}{\linewidth}{L L L L}
            \toprule
            \textbf{Operation} & \textbf{Symbol} & \textbf{મેથડ} & \textbf{વર્ણન} \\
            \midrule
            \textbf{Union} & \code{|} & \code{union()} & બંને sets ના elements \\
            \textbf{Intersection} & \code{\&} & \code{intersection()} & Common elements \\
            \textbf{Difference} & \code{-} & \code{difference()} & પહેલા માંથી બીજા ના બાદ \\
            \textbf{Symmetric Difference} & \code{\^} & \code{symmetric\_difference()} & માત્ર unique elements \\
            \bottomrule
        \end{tabulary}
    \end{center}

    \begin{lstlisting}[language=Python]
# Set Operations પ્રોગ્રામ
set1 = {1, 2, 3, 4, 5}
set2 = {4, 5, 6, 7, 8}

print("Set 1:", set1)
print("Set 2:", set2)

# Union Operation
union_result = set1 | set2
print("Union:", union_result)

# Intersection Operation  
intersection_result = set1 & set2
print("Intersection:", intersection_result)

# Difference Operation
difference_result = set1 - set2
print("Difference:", difference_result)

# Symmetric Difference
sym_diff_result = set1 ^ set2
print("Symmetric Difference:", sym_diff_result)

# Subset and Superset
set3 = {1, 2}
print("શું set3 એ set1 નો subset છે?", set3.issubset(set1))
print("શું set1 એ set3 નો superset છે?", set1.issuperset(set3))
    \end{lstlisting}

    \begin{itemize}
        \item \textbf{Mathematical Operations}: Set theory ના operations implement કરે છે
        \item \textbf{Efficient Processing}: Duplicate elements automatically remove થાય છે
        \item \textbf{Boolean Results}: Subset/superset operations boolean return કરે છે
    \end{itemize}
    \begin{mnemonicbox}Union Intersection Difference Symmetric\end{mnemonicbox}
\end{solutionbox}

\questionmarks{1(c) OR}{7}{Write a python program to demonstrate the dictionaries functions and operations.}
\begin{solutionbox}
    Dictionary operations data manipulation માટે powerful tools પૂરા પાડે છે.

    \textbf{Dictionary મેથડ્સ:}
    \begin{center}
        \begin{tabulary}{\linewidth}{L L L}
            \toprule
            \textbf{મેથડ} & \textbf{વર્ણન} & \textbf{ઉદાહરણ} \\
            \midrule
            \textbf{keys()} & બધી keys આપે & \code{dict.keys()} \\
            \textbf{values()} & બધી values આપે & \code{dict.values()} \\
            \textbf{items()} & key-value pairs આપે & \code{dict.items()} \\
            \textbf{get()} & Safe value retrieval & \code{dict.get('key')} \\
            \textbf{update()} & Dictionary merge કરે & \code{dict.update()} \\
            \bottomrule
        \end{tabulary}
    \end{center}

    \begin{lstlisting}[language=Python]
# Dictionary Operations પ્રોગ્રામ
student_data = {
    "name": "Amit",
    "age": 21,
    "course": "IT",
    "semester": 2
}

print("Original Dictionary:", student_data)

# Values access કરવી
print("Student Name:", student_data.get("name"))
print("Student Age:", student_data["age"])

# નવી key-value pair ઉમેરવી
student_data["city"] = "Ahmedabad"
print("City ઉમેર્યા પછી:", student_data)

# Existing value update કરવી
student_data.update({"age": 22, "semester": 3})
print("Update પછી:", student_data)

# Dictionary methods
print("Keys:", list(student_data.keys()))
print("Values:", list(student_data.values()))
print("Items:", list(student_data.items()))

# Elements remove કરવા
removed_value = student_data.pop("semester")
print("Removed value:", removed_value)
print("Final Dictionary:", student_data)
    \end{lstlisting}

    \begin{itemize}
        \item \textbf{Dynamic Operations}: Keys અને values runtime એ add/remove કરી શકાય
        \item \textbf{Safe Access}: \code{get()} method KeyError થી બચાવે છે
        \item \textbf{Iteration Support}: \code{keys()}, \code{values()}, \code{items()} methods લૂપ્સ માટે ઉપયોગી છે
    \end{itemize}
    \begin{mnemonicbox}Get Keys Values Items Update Pop\end{mnemonicbox}
\end{solutionbox}

\section*{Question 2}

\questionmarks{2(a)}{3}{Distinguish between Tuple and List in Python.}
\begin{solutionbox}
    \textbf{Tuple vs List તફાવત:}
    \begin{center}
        \begin{tabulary}{\linewidth}{L L L}
            \toprule
            \textbf{વિશેષતા} & \textbf{Tuple} & \textbf{List} \\
            \midrule
            \textbf{Mutability} & Immutable (બદલી ન શકાય) & Mutable (બદલી શકાય) \\
            \textbf{Syntax} & Parentheses \code{()} & Square brackets \code{[]} \\
            \textbf{Performance} & Faster (ઝડપી) & Slower (ધીમું) \\
            \textbf{Memory} & ઓછી memory & વધુ memory \\
            \textbf{Methods} & Limited (count, index) & ઘણી methods ઉપલબ્ધ \\
            \textbf{Use Case} & Fixed data & Dynamic data \\
            \bottomrule
        \end{tabulary}
    \end{center}

    \begin{itemize}
        \item \textbf{Immutable Nature}: Tuple બન્યા પછી બદલી શકાતું નથી
        \item \textbf{Performance}: Tuple operations list કરતા ઝડપી છે
        \item \textbf{Memory Efficient}: Tuple ઓછી memory રોકે છે
    \end{itemize}
    \begin{mnemonicbox}Tuple Tight, List Light\end{mnemonicbox}
\end{solutionbox}

\questionmarks{2(b)}{4}{What is the dir() function in python? Explain with example.}
\begin{solutionbox}
    \code{dir()} એ એક built-in function છે જે object ના attributes અને methods નું list આપે છે.

    \textbf{dir() Function Features:}
    \begin{center}
        \begin{tabulary}{\linewidth}{L L}
            \toprule
            \textbf{Feature} & \textbf{વર્ણન} \\
            \midrule
            \textbf{Object Inspection} & Object ના attributes બતાવે છે \\
            \textbf{Method Discovery} & ઉપલબ્ધ methods નું list આપે છે \\
            \textbf{Namespace Exploration} & Current namespace ના variables બતાવે છે \\
            \textbf{Module Analysis} & Module ના contents explore કરે છે \\
            \bottomrule
        \end{tabulary}
    \end{center}

    \begin{lstlisting}[language=Python]
# dir() Function ઉદાહરણ
# String object માટે
text = "Hello"
string_methods = dir(text)
print("String methods:", string_methods[:5])

# List object માટે  
my_list = [1, 2, 3]
list_methods = dir(my_list)
print("List methods:", [m for m in list_methods if not m.startswith('_')][:5])

# Current namespace માટે
print("Current namespace:", dir()[:3])

# Built-in functions માટે
import math
print("Math module:", dir(math)[:5])
    \end{lstlisting}

    \begin{itemize}
        \item \textbf{Interactive Development}: Objects ની capabilities જાણવા માટે ઉપયોગી
        \item \textbf{Debugging Tool}: ઉપલબ્ધ methods શોધવા માટે
        \item \textbf{Learning Aid}: નવી libraries explore કરવા માટે મદદરૂપ
    \end{itemize}
    \begin{mnemonicbox}Dir = Directory of Methods\end{mnemonicbox}
\end{solutionbox}

\questionmarks{2(c)}{7}{Write a program to define a module to find the area and circumference of a circle. Import module to another program.}
\begin{solutionbox}
    Module approach code reusability અને organization સુધારે છે.

    \begin{center}
        \begin{tikzpicture}[
            node distance=4cm,
            auto,
            block/.style={
                rectangle,
                draw=blue!80!black,
                fill=blue!5,
                very thick,
                minimum width=3cm,
                minimum height=2cm,
                align=left,
                font=\small
            }
        ]
            % Nodes
            \node[block] (circle) {\textbf{circle.py} (Module)\\-\\area()\\-\\circumference\\-\\PI constant};
            \node[block, right of=circle] (main) {\textbf{main.py} (Program)\\-\\import circle\\-\\use functions};

            % Arrow
            \draw[->, very thick, blue!80!black] (circle) -- node[above] {Imported by} (main);
        \end{tikzpicture}
    \end{center}

    \textbf{File 1: circle.py (Module)}
    \begin{lstlisting}[language=Python]
# circle.py - Circle calculation module
import math

# Constants
PI = math.pi

def area(radius):
    """Circle નો area ગણો"""
    if radius < 0:
        return "Radius negative ન હોઈ શકે"
    return PI * radius * radius

def circumference(radius):
    """Circle નો circumference ગણો"""
    if radius < 0:
        return "Radius negative ન હોઈ શકે"
    return 2 * PI * radius

def display_info():
    """Module information display કરો"""
    print("Circle Module - Version 1.0")
    print("Functions: area(), circumference()")
    \end{lstlisting}

    \textbf{File 2: main.py (Main Program)}
    \begin{lstlisting}[language=Python]
# main.py - Circle module નો ઉપયોગ કરતો main program
import circle

# User પાસેથી radius લો
radius = float(input("Radius નાખો: "))

# Module functions નો ઉપયોગ કરીને ગણતરી
circle_area = circle.area(radius)
circle_circumference = circle.circumference(radius)

# પરિણામ દર્શાવો
print(f"Radius {radius} માટે:")
print(f"Area: {circle_area:.2f}")
print(f"Circumference: {circle_circumference:.2f}")

# Module info દર્શાવો
circle.display_info()
    \end{lstlisting}

    \begin{itemize}
        \item \textbf{Modular Design}: Functions ને separate file માં organize કરે છે
        \item \textbf{Reusability}: Module નો ઉપયોગ multiple programs માં થઈ શકે
        \item \textbf{Namespace Management}: Functions module prefix દ્વારા access થાય છે
    \end{itemize}
    \begin{mnemonicbox}Import Calculate Display\end{mnemonicbox}
\end{solutionbox}

\questionmarks{2(a) OR}{3}{Explain Nested Tuple with example.}
\begin{solutionbox}
    \textbf{Nested Tuple} એ બીજા tuples ને તની અંદર સમાવે છે, જે hierarchical structure બનાવે છે.

    \textbf{Nested Tuple Features:}
    \begin{center}
        \begin{tabulary}{\linewidth}{L L}
            \toprule
            \textbf{Feature} & \textbf{વર્ણન} \\
            \midrule
            \textbf{Multi-dimensional} & 2D અથવા 3D data structure \\
            \textbf{Immutable} & દરેક level પર immutable હોય છે \\
            \textbf{Indexing} & Multiple square brackets દ્વારા access થાય \\
            \textbf{Heterogeneous} & અલગ અલગ data types store કરી શકે \\
            \bottomrule
        \end{tabulary}
    \end{center}

    \begin{lstlisting}[language=Python]
# Nested Tuple ઉદાહરણ
student_records = (
    ("Raj", 20, ("IT", 2)),
    ("Priya", 19, ("CS", 1)), 
    ("Amit", 21, ("IT", 3))
)

# Nested elements access કરવા
print("First student:", student_records[0])
print("First student name:", student_records[0][0])
print("First student course:", student_records[0][2][0])

# Nested tuple માં iterate કરવું
for student in student_records:
    name, age, (course, semester) = student
    print(f"{name} - Age: {age}, Course: {course}, Sem: {semester}")
    \end{lstlisting}

    \begin{itemize}
        \item \textbf{Data Organization}: Related data ને group કરવા માટે ઉપયોગી
        \item \textbf{Immutable Structure}: Structure એકવાર બન્યા પછી બદલી શકાતું નથી
        \item \textbf{Efficient Access}: Fast index-based access
    \end{itemize}
    \begin{mnemonicbox}Nested = Tuple Inside Tuple\end{mnemonicbox}
\end{solutionbox}

\questionmarks{2(b) OR}{4}{What is PIP? Write the syntax to install and uninstall python packages.}
\begin{solutionbox}
    \textbf{PIP} (Pip Installs Packages) એ Python package installer છે જે PyPI પરથી packages download અને install કરે છે.

    \textbf{PIP Commands:}
    \begin{center}
        \begin{tabulary}{\linewidth}{L L L}
            \toprule
            \textbf{Command} & \textbf{Syntax} & \textbf{વર્ણન} \\
            \midrule
            \textbf{Install} & \code{pip install package} & Package install કરે \\
            \textbf{Uninstall} & \code{pip uninstall package} & Package remove કરે \\
            \textbf{List} & \code{pip list} & Installed packages બતાવે \\
            \textbf{Show} & \code{pip show package} & Package info બતાવે \\
            \textbf{Upgrade} & \code{pip install --upgrade pkg} & Package update કરે \\
            \bottomrule
        \end{tabulary}
    \end{center}

    \begin{lstlisting}[language=Python]
# PIP Command ઉદાહરણો (Terminal/Command Prompt માં run કરો)

# Package install કરવા
# pip install requests

# Specific version install કરવા
# pip install Django==3.2.0

# Package uninstall કરવા  
# pip uninstall numpy

# બધા installed packages જોવા
# pip list

# Package information જોવા
# pip show matplotlib

# Package upgrade કરવા
# pip install --upgrade pandas

# Requirements file માંથી install કરવા
# pip install -r requirements.txt
    \end{lstlisting}

    \begin{itemize}
        \item \textbf{Package Management}: Third-party libraries ને સરળતાથી manage કરે છે
        \item \textbf{Version Control}: Specific versions install કરી શકાય છે
        \item \textbf{Dependency Resolution}: જરૂરી dependencies automatically install થાય છે
    \end{itemize}
    \begin{mnemonicbox}PIP = Package Install Python\end{mnemonicbox}
\end{solutionbox}

\questionmarks{2(c) OR}{7}{Explain different ways of importing package. How are modules and packages connected to each other?}
\begin{solutionbox}
    Python માં અલગ અલગ રીતે import કરવાથી code organization અને namespace management માં મદદ મળે છે.

    \textbf{Import Methods:}
    \begin{center}
        \begin{tabulary}{\linewidth}{L L L}
            \toprule
            \textbf{Method} & \textbf{Syntax} & \textbf{ઉપયોગ} \\
            \midrule
            \textbf{Basic Import} & \code{import module} & Full module name જરૂરી \\
            \textbf{From Import} & \code{from module import function} & Direct function access \\
            \textbf{Alias Import} & \code{import module as alias} & Module માટે short name \\
            \textbf{Star Import} & \code{from module import *} & બધા functions import કરે \\
            \textbf{Package Import} & \code{from package import module} & Package માંથી import \\
            \bottomrule
        \end{tabulary}
    \end{center}

    \begin{lstlisting}[language=Python]
# અલગ અલગ Import Ways

# 1. Basic Import
import math
result = math.sqrt(16)

# 2. From Import
from math import sqrt, pi
result = sqrt(16)
area = pi * 5 * 5

# 3. Alias Import
import numpy as np
array = np.array([1, 2, 3])

# 4. Star Import (not recommended)
from math import *
result = cos(0)

# 5. Package Import
from mypackage import module1
from mypackage.subpackage import module3
    \end{lstlisting}

    \textbf{Module-Package Connection:}
    \begin{itemize}
        \item \textbf{Modules}: Single .py files જેમાં Python code હોય છે
        \item \textbf{Packages}: Directories જેમાં multiple modules અને \code{\_\_init\_\_.py} હોય છે
        \item \textbf{Namespace}: Packages hierarchical namespace structure બનાવે છે
        \item \textbf{\_\_init\_\_.py}: Directory ને package બનાવે છે અને imports control કરે છે
    \end{itemize}
    \begin{mnemonicbox}Import From As Star Package\end{mnemonicbox}
\end{solutionbox}

\section*{Question 3}

\questionmarks{3(a)}{3}{Describe Runtime Error and Syntax Error. Explain with example.}
\begin{solutionbox}
    \textbf{Error Types તફાવત:}
    \begin{center}
        \begin{tabulary}{\linewidth}{L L L L}
            \toprule
            \textbf{Error Type} & \textbf{ક્યારે થાય} & \textbf{Detection} & \textbf{ઉદાહરણ} \\
            \midrule
            \textbf{Syntax Error} & Code parsing વખતે & Execution પહેલાં & Colon ભૂલી જવું \\
            \textbf{Runtime Error} & Execution દરમ્યાન & Run થતી વખતે & Zero division \\
            \textbf{Logic Error} & હંમેશા & Execution પછી & ખોટું logic \\
            \bottomrule
        \end{tabulary}
    \end{center}

    \begin{lstlisting}[language=Python]
# Syntax Error ઉદાહરણ
# print("Hello World"  # Closing parenthesis નથી
# SyntaxError: unexpected EOF while parsing

# Runtime Error ઉદાહરણો
try:
    # ZeroDivisionError
    result = 10 / 0
except ZeroDivisionError:
    print("Zero વડે ભાગી શકાય નહિ")

try:
    # FileNotFoundError  
    file = open("nonexistent.txt", "r")
except FileNotFoundError:
    print("File મળી નથી")
    \end{lstlisting}

    \begin{itemize}
        \item \textbf{Syntax Errors}: Code run થાય તે પહેલાં પકડાઈ જાય છે
        \item \textbf{Runtime Errors}: Program execution દરમ્યાન આવે છે
        \item \textbf{Prevention}: Exception handling થી runtime errors handle કરી શકાય
    \end{itemize}
    \begin{mnemonicbox}Syntax Before, Runtime During\end{mnemonicbox}
\end{solutionbox}

\questionmarks{3(b)}{4}{What is Exception handling in Python? Explain with example.}
\begin{solutionbox}
    \textbf{Exception handling} એ runtime errors ને gracefully handle કરવા અને program crash થતો અટકાવવા માટે વપરાય છે.

    \textbf{Exception Handling Keywords:}
    \begin{center}
        \begin{tabulary}{\linewidth}{L L L}
            \toprule
            \textbf{Keyword} & \textbf{હેતુ} & \textbf{વર્ણન} \\
            \midrule
            \textbf{try} & જેમાં exception આવી શકે & Risk code block \\
            \textbf{except} & Exception handle કરવા & Error handling block \\
            \textbf{finally} & હંમેશા run થાય & Cleanup code \\
            \textbf{else} & જો exception ન આવે & Success code block \\
            \textbf{raise} & Manual exception raise કરવા & Custom error throwing \\
            \bottomrule
        \end{tabulary}
    \end{center}

    \begin{lstlisting}[language=Python]
# Exception Handling ઉદાહરણ
def safe_division(a, b):
    try:
        # Code જેમાં exception આવી શકે
        result = a / b
        print(f"ભાગાકાર સફળ: {result}")
        
    except ZeroDivisionError:
        # Specific exception handle કરો
        print("Error: Zero વડે ભાગી શકાય નહિ")
        result = None
        
    except TypeError:
        # Type errors handle કરો
        print("Error: Invalid data types")
        result = None
        
    else:
        # Exception ન આવે તો જ run થાય
        print("Division સફળતાપૂર્વક પૂર્ણ થયું")
        
    finally:
        # હંમેશા run થાય
        print("Division operation સમાપ્ત")
        
    return result

# Function test કરો
safe_division(10, 2)   # Normal case
safe_division(10, 0)   # Zero division
safe_division(10, "a") # Type error
    \end{lstlisting}

    \begin{itemize}
        \item \textbf{Error Prevention}: Program ને crash થતો અટકાવે છે
        \item \textbf{Graceful Handling}: User-friendly error messages આપે છે
        \item \textbf{Resource Management}: Finally block માં cleanup operations થાય છે
    \end{itemize}
    \begin{mnemonicbox}Try Except Finally Else Raise\end{mnemonicbox}
\end{solutionbox}

\questionmarks{3(c)}{7}{Create a function for division of two numbers, if the value of any argument is non-integer then raise the error or if second argument is 0 then raise the error.}
\begin{solutionbox}
    Custom exception handling function validation અને error control માટે મહત્વપૂર્ણ છે.

    \begin{lstlisting}[language=Python]
def safe_integer_division(num1, num2):
    """
    બે સંખ્યાઓનો ભાગાકાર validation સાથે
    જો arguments integer ન હોય તો TypeError raise કરો
    જો બીજી argument 0 હોય તો ZeroDivisionError raise કરો
    """
    
    # બંને arguments integers છે કે કેમ તે તપાસો
    if not isinstance(num1, int):
        raise TypeError(f"પહેલી argument integer હોવી જોઈએ, {type(num1).__name__} મળી")
    
    if not isinstance(num2, int):
        raise TypeError(f"બીજી argument integer હોવી જોઈએ, {type(num2).__name__} મળી")
    
    # Zero division તપાસો
    if num2 == 0:
        raise ZeroDivisionError("Zero વડે ભાગી શકાય નહિ")
    
    # ભાગાકાર કરો
    result = num1 / num2
    return result

# અલગ અલગ cases સાથે function test કરો
def test_division():
    test_cases = [
        (10, 2),      # Valid case
        (15, 3),      # Valid case  
        (10, 0),      # Zero division error
        (10.5, 2),    # Non-integer first argument
        (10, 2.5),    # Non-integer second argument
        ("10", 2),    # String argument
    ]
    
    for num1, num2 in test_cases:
        try:
            result = safe_integer_division(num1, num2)
            print(f"{num1} / {num2} = {result}")
            
        except TypeError as e:
            print(f"Type Error: {e}")
            
        except ZeroDivisionError as e:
            print(f"Zero Division Error: {e}")
            
        except Exception as e:
            print(f"Unexpected Error: {e}")
        
        print("-" * 40)

# Tests run કરો
test_division()
    \end{lstlisting}

    \begin{itemize}
        \item \textbf{Input Validation}: Arguments ના type અને value તપાસે છે
        \item \textbf{Custom Errors}: Specific exceptions raise કરે છે
        \item \textbf{Error Messages}: સ્પષ્ટ અને વર્ણનાત્મક error messages આપે છે
    \end{itemize}
    \begin{mnemonicbox}Validate Type, Check Zero, Divide Safe\end{mnemonicbox}
\end{solutionbox}

\questionmarks{3(a) OR}{3}{Describe any five built-in exceptions in Python.}
\begin{solutionbox}
    \textbf{Built-in Exceptions:}
    \begin{center}
        \begin{tabulary}{\linewidth}{L L L}
            \toprule
            \textbf{Exception} & \textbf{કારણ} & \textbf{ઉદાહરણ} \\
            \midrule
            \textbf{ValueError} & Operation માટે invalid value અપા ત્યારે & \code{int("abc")} \\
            \textbf{TypeError} & ખોટો data type હોય ત્યારે & \code{"5" + 5} \\
            \textbf{IndexError} & Index range ની બહાર હોય ત્યારે & \code{list[10]} \\
            \textbf{KeyError} & Dictionary key ન મળે ત્યારે & \code{dict["missing"]} \\
            \textbf{FileNotFoundError} & File અસ્તિત્વમાં ન હોય ત્યારે & \code{open("missing.txt")} \\
            \bottomrule
        \end{tabulary}
    \end{center}

    \begin{itemize}
        \item \textbf{Automatic Detection}: Python આ exceptions automatically raise કરે છે
        \item \textbf{Specific Handling}: દરેક exception નો specific હેતુ છે
        \item \textbf{Inheritance}: બધા exceptions BaseException class માંથી inherit થાય છે
    \end{itemize}
    \begin{mnemonicbox}Value Type Index Key File\end{mnemonicbox}
\end{solutionbox}

\questionmarks{3(b) OR}{4}{Explain try...except...else...finally block with example.}
\begin{solutionbox}
    \textbf{Exception Block Structure:}
    \begin{center}
        \begin{tikzpicture}[node distance=2cm, auto]
            \node (try) [rectangle, draw, fill=blue!10, text width=6em, text centered] {\textbf{try}\\Rise Code};
            \node (except) [rectangle, draw, fill=red!10, text width=6em, text centered, below left of=try, node distance=3cm] {\textbf{except}\\Handle Error};
            \node (else) [rectangle, draw, fill=green!10, text width=6em, text centered, below right of=try, node distance=3cm] {\textbf{else}\\No Error};
            \node (finally) [rectangle, draw, fill=yellow!10, text width=6em, text centered, below of=try, node distance=4cm] {\textbf{finally}\\Always Run};

            \draw [->] (try) -- node [left] {Error} (except);
            \draw [->] (try) -- node [right] {Success} (else);
            \draw [->] (except) -- (finally);
            \draw [->] (else) -- (finally);
        \end{tikzpicture}
    \end{center}

    \begin{lstlisting}[language=Python]
# Complete Exception Block ઉદાહરણ
def file_operation(filename):
    try:
        # File open કરવાનો પ્રયત્ન
        f = open(filename, "r")
        print("File સફળતાપૂર્વક open થઈ")
        
    except FileNotFoundError:
        # જો file ન મળે
        print("Error: File મળી નથી")
        
    else:
        # જો કોઈ error ન આવે તો run થાય
        print("File content વાંચી રહ્યા છીએ...")
        print(f.read())
        f.close()
        
    finally:
        # હંમેશા run થાય
        print("File operation process પૂર્ણ")

# Existing અને missing files સાથે test કરો
file_operation("existing.txt")
file_operation("missing.txt")
    \end{lstlisting}

    \begin{itemize}
        \item \textbf{Else Block}: માત્ર ત્યારે જ execute થાય જો try block માં exception ન આવે
        \item \textbf{Finally Block}: Error આવે કે ન આવે, હંમેશા execute થાય (cleanup માટે)
        \item \textbf{Complete Flow}: Execution ની બધી શક્યતાઓ cover કરે છે
    \end{itemize}
    \begin{mnemonicbox}Try Exception Else Finally\end{mnemonicbox}
\end{solutionbox}

\questionmarks{3(c) OR}{7}{Write a user defined exception that could be raised when the text entered by a user consists of less than 10 characters.}
\begin{solutionbox}
    User-defined exceptions custom validation logic implement કરવા દે છે.

    \begin{lstlisting}[language=Python]
class ShortTextError(Exception):
    """Text validation માટે custom exception"""
    def __init__(self, length):
        self.length = length
        self.message = f"Text ખૂબ નાનું છે ({length} chars). ઓછામાં ઓછા 10 જોઈએ."
        super().__init__(self.message)

def validate_text():
    while True:
        try:
            # User input લો
            text = input("Text દાખલ કરો (min 10 chars): ")
            
            # Length તપાસો
            if len(text) < 10:
                raise ShortTextError(len(text))
            
            print(f"Valid text સ્વીકારાયું: {text}")
            break
            
        except ShortTextError as e:
            print(f"Error: {e}")
            print("ફરી પ્રયાસ કરો.\n")
            
        except Exception as e:
            print(f"Unexpected error: {e}")

# Validation run કરો
print("--- Text Validation Program ---")
# validate_text() # Run કરવા માટે uncomment કરો
    \end{lstlisting}

    \begin{itemize}
        \item \textbf{Inheritance}: Custom exception class \code{Exception} માંથી inherit થાય છે
        \item \textbf{Raising}: \code{raise} keyword exception trigger કરવા વપરાય છે
        \item \textbf{Usage}: Domain-specific rules implement કરવામાં મદદ કરે છે
    \end{itemize}
    \begin{mnemonicbox}Class Inherit Raise Catch\end{mnemonicbox}
\end{solutionbox}

\section*{Question 4}

\questionmarks{4(a)}{3}{Write five points on difference between Text File and Binary File.}
\begin{solutionbox}
    \textbf{Text vs Binary File:}
    \begin{center}
        \begin{tabulary}{\linewidth}{L L L}
            \toprule
            \textbf{Feature} & \textbf{Text File} & \textbf{Binary File} \\
            \midrule
            \textbf{Content} & માનવ વાંચી શકે તેવા અક્ષરો & મશીન વાંચી શકે તેવો data (0s/1s) \\
            \textbf{Encoding} & Encoding વપરાય (ASCII/UTF-8) & Encoding વપરતું નથી \\
            \textbf{Extensions} & .txt, .py, .csv & .bin, .jpg, .exe \\
            \textbf{EOL} & Newline translation handle કરે & Translation કરતું નથી \\
            \textbf{Processing} & ધીમું processing & ઝડપી processing \\
            \bottomrule
        \end{tabulary}
    \end{center}

    \begin{itemize}
        \item \textbf{Readability}: Text files સાદા editors માં ખોલી શકાય છે
        \item \textbf{Portability}: Binary files bytes નો continuous stream છે
        \item \textbf{Usage}: Documents માટે Text, Images/Videos માટે Binary
    \end{itemize}
    \begin{mnemonicbox}Text Human, Binary Machine\end{mnemonicbox}
\end{solutionbox}

\questionmarks{4(b)}{4}{Write a program to read the data from a file and separate the uppercase character and lowercase character into two separate files.}
\begin{solutionbox}
    File processing માં વાંચવું, વિશ્લેષણ કરવું અને લખવું શામેલ છે.

    \begin{lstlisting}[language=Python]
def separate_case_chars(source_file):
    try:
        # Characters store કરવા માટે strings
        upper_chars = ""
        lower_chars = ""
        
        # Source file વાંચો
        with open(source_file, "r") as f:
            content = f.read()
            
            # દરેક character process કરો
            for char in content:
                if char.isupper():
                    upper_chars += char
                elif char.islower():
                    lower_chars += char
        
        # Uppercase characters લખો
        with open("uppercase.txt", "w") as f:
            f.write(upper_chars)
            
        # Lowercase characters લખો
        with open("lowercase.txt", "w") as f:
            f.write(lower_chars)
            
        print("Characters સફળતાપૂર્વક અલગ કર્યા!")
        
    except FileNotFoundError:
        print("Source file મળી નથી")

# પહેલા sample file બનાવો
with open("input.txt", "w") as f:
    f.write("Hello World! Python Programming.")

# Separation run કરો
separate_case_chars("input.txt")
    \end{lstlisting}

    \begin{itemize}
        \item \textbf{String Methods}: \code{isupper()} અને \code{islower()} case તપાસે છે
        \item \textbf{File Handling}: \code{with} statement safe file operations માટે વપરાય છે
        \item \textbf{Data Separation}: Content અલગ અલગ streams માં filter થાય છે
    \end{itemize}
    \begin{mnemonicbox}Read Loop Check Write\end{mnemonicbox}
\end{solutionbox}

\questionmarks{4(c)}{7}{Describe dump() and load() method. Explain with example.}
\begin{solutionbox}
    \code{dump()} અને \code{load()} એ \textbf{pickle} module ના ભાગ છે જે object serialization માટે વપરાય છે.

    \textbf{Pickle મેથડ્સ:}
    \begin{center}
        \begin{tabulary}{\linewidth}{L L L}
            \toprule
            \textbf{મેથડ} & \textbf{હેતુ} & \textbf{Mode} \\
            \midrule
            \textbf{dump(obj, file)} & Object ને file માં serialize કરે & Write Binary ('wb') \\
            \textbf{load(file)} & File માંથી object deserialize કરે & Read Binary ('rb') \\
            \bottomrule
        \end{tabulary}
    \end{center}

    \begin{lstlisting}[language=Python]
import pickle

# Serialize કરવા માટે data
student = {
    "name": "Rahul",
    "roll": 101,
    "marks": [85, 90, 88]
}

# DUMP ઉદાહરણ
try:
    with open("data.pkl", "wb") as f:
        pickle.dump(student, f)
    print("Data સફળતાપૂર્વક dump થયો")
except Exception as e:
    print(f"Error: {e}")

# LOAD ઉદાહરણ
try:
    with open("data.pkl", "rb") as f:
        loaded_data = pickle.load(f)
    print("Data સફળતાપૂર્વક load થયો:")
    print(loaded_data)
except Exception as e:
    print(f"Error: {e}")
    \end{lstlisting}

    \begin{itemize}
        \item \textbf{Serialization}: Object ને byte stream માં ફેરવવું (Pickling)
        \item \textbf{Deserialization}: Byte stream ને પાછું object માં ફેરવવું (Unpickling)
        \item \textbf{Persistence}: Objects ની state ને disk પર save કરવા દે છે
    \end{itemize}
    \begin{mnemonicbox}Dump Store, Load Restore\end{mnemonicbox}
\end{solutionbox}

\questionmarks{4(a) OR}{3}{List different types of file modes provided by python for file operations and explain their uses.}
\begin{solutionbox}
    \textbf{Python File Modes:}
    \begin{center}
        \begin{tabulary}{\linewidth}{L L L}
            \toprule
            \textbf{Mode} & \textbf{નામ} & \textbf{વર્ણન} \\
            \midrule
            \textbf{'r'} & Read & Default mode. વાંચવા માટે ખોલે. \\
            \textbf{'w'} & Write & લખવા માટે ખોલે. File overwrite કરે. \\
            \textbf{'a'} & Append & લખવા માટે ખોલે. અંતમાં ઉમેરે. \\
            \textbf{'x'} & Create & નવી file બનાવે. જે હોય તો fail થાય. \\
            \textbf{'b'} & Binary & Binary mode (દા.ત., 'rb', 'wb'). \\
            \textbf{'+'} & Update & Read અને Write (દા.ત., 'r+', 'w+'). \\
            \bottomrule
        \end{tabulary}
    \end{center}

    \begin{itemize}
        \item \textbf{Safety}: 'x' આકસ્મિક overwriting અટકાવે છે
        \item \textbf{Binary}: 'b' નો ઉપયોગ non-text files માટે થવો જોઈએ
        \item \textbf{Combination}: '+ ને અન્ય modes સાથે જોડી શકાય છે
    \end{itemize}
    \begin{mnemonicbox}Read Write Append Create\end{mnemonicbox}
\end{solutionbox}

\questionmarks{4(b) OR}{4}{Describe readline() and writeline() functions of the file.}
\begin{solutionbox}
    \textbf{નોંધ}: Python માં \code{writelines()} છે પણ \code{writeline()} નથી.

    \textbf{Line Operations:}
    \begin{center}
        \begin{tabulary}{\linewidth}{L L L}
            \toprule
            \textbf{Function} & \textbf{હેતુ} & \textbf{ઉદાહરણ} \\
            \midrule
            \textbf{readline()} & એક line વાંચે છે & \code{line = f.readline()} \\
            \textbf{readlines()} & બધી lines list માં વાંચે છે & \code{lines = f.readlines()} \\
            \textbf{writelines()} & Strings નું list લખે છે & \code{f.writelines(list)} \\
            \bottomrule
        \end{tabulary}
    \end{center}

    \begin{lstlisting}[language=Python]
# Lines લખવી
lines = ["First Line\n", "Second Line\n"]
with open("demo.txt", "w") as f:
    f.writelines(lines)

# Lines વાંચવી
with open("demo.txt", "r") as f:
    line1 = f.readline()
    print(f"Read 1: {line1.strip()}")
    
    line2 = f.readline()
    print(f"Read 2: {line2.strip()}")
    \end{lstlisting}

    \begin{itemize}
        \item \textbf{Sequential}: readline pointer ને line by line ખસેડે છે
        \item \textbf{List Support}: writelines iterable (list/tuple) accept કરે છે
        \item \textbf{Memory}: readline મોટી files માટે efficient છે
    \end{itemize}
    \begin{mnemonicbox}Read One, Write Many\end{mnemonicbox}
\end{solutionbox}

\questionmarks{4(c) OR}{7}{Write a python program to demonstrate seek() and tell() methods.}
\begin{solutionbox}
    \code{seek()} file pointer ખસેડે છે, \code{tell()} current position આપે છે.

    \begin{lstlisting}[language=Python]
# Seek અને Tell Demonstration
filename = "seek_demo.txt"

# File બનાવો
with open(filename, "w") as f:
    f.write("Hello Python World")

# Read operations
with open(filename, "r") as f:
    # શરૂઆતની position
    print(f"Start Position: {f.tell()}")  # 0
    
    # પહેલા 5 chars વાંચો
    print(f"Read: {f.read(5)}")          # Hello
    print(f"Current Position: {f.tell()}") # 5
    
    # શરૂઆતમાં seek કરો
    f.seek(0)
    print(f"After seek(0): {f.tell()}")   # 0
    
    # Specific position પર seek કરો
    f.seek(6)
    print(f"After seek(6): {f.read(6)}")  # Python
    
    # અંતે જાઓ (binary mode માં અથવા specific syntax સાથે)
    # f.seek(0, 2) અંતે લઈ જાય
    \end{lstlisting}

    \begin{center}
        \begin{tabulary}{\linewidth}{L L}
            \toprule
            \textbf{Method} & \textbf{Syntax} \\
            \midrule
            \textbf{tell()} & \code{pos = f.tell()} \\
            \textbf{seek()} & \code{f.seek(offset, whence)} \\
            \bottomrule
        \end{tabulary}
    \end{center}

    \begin{itemize}
        \item \textbf{Navigation}: Files માં random access ની સુવિધા આપે છે
        \item \textbf{Whence}: 0=Start, 1=Current, 2=End
        \item \textbf{Binary}: Relative seeking binary mode માં શ્રેષ્ઠ કામ કરે છે
    \end{itemize}
    \begin{mnemonicbox}Tell Position, Seek Location\end{mnemonicbox}
\end{solutionbox}

\section*{Question 5}

\questionmarks{5(a)}{3}{Draw the shape of circle and rectangle using turtle and fill with red color.}
\begin{solutionbox}
    \begin{lstlisting}[language=Python]
import turtle

t = turtle.Turtle()

# Fill color red set કરો
t.fillcolor("red")

# Circle દોરો
t.begin_fill()
t.circle(50)      # 50 radius વાળું circle
t.end_fill()

# નવી position પર જાઓ
t.penup()
t.goto(100, 0)
t.pendown()

# Rectangle દોરો
t.begin_fill()
for _ in range(2):
    t.forward(100) # લંબાઈ
    t.right(90)
    t.forward(50)  # પહોળાઈ
    t.right(90)
t.end_fill()

turtle.done()
    \end{lstlisting}

    \begin{itemize}
        \item \textbf{begin\_fill()}: Shape ભરવાનું શરૂ કરે છે
        \item \textbf{end\_fill()}: Shape ભરવાનું પૂર્ણ કરે છે
        \item \textbf{Geometry}: Circle અને loop-આધારિત rectangle drawing
    \end{itemize}
    \begin{mnemonicbox}Begin Fill End\end{mnemonicbox}
\end{solutionbox}

\questionmarks{5(b)}{4}{Explain various inbuilt methods for changing the direction of turtle.}
\begin{solutionbox}
    \textbf{Turtle Direction મેથડ્સ:}
    \begin{center}
        \begin{tabulary}{\linewidth}{L L L}
            \toprule
            \textbf{મેથડ} & \textbf{વર્ણન} & \textbf{ઉદાહરણ} \\
            \midrule
            \textbf{right(angle)} & જમણે વળો (angle) & \code{t.right(90)} \\
            \textbf{left(angle)} & ડાબે વળો (angle) & \code{t.left(45)} \\
            \textbf{setheading(angle)} & Absolute angle set કરો & \code{t.setheading(0)} \\
            \textbf{towards(x,y)} & Coordinates તરફ point કરો & \code{t.towards(0,0)} \\
            \bottomrule
        \end{tabulary}
    \end{center}

    \begin{lstlisting}[language=Python]
# Direction ઉદાહરણો
import turtle
t = turtle.Turtle()

t.forward(100)
t.right(90)       # 90 degrees જમણે
t.forward(50)
t.left(45)        # 45 degrees ડાબે
t.setheading(180) # પશ્ચિમ તરફ (180 degrees)
    \end{lstlisting}

    \begin{itemize}
        \item \textbf{Relative}: right/left વર્તમાન direction થી સાપેક્ષ છે
        \item \textbf{Absolute}: setheading 0-360 degree system વાપરે છે
        \item \textbf{Navigation}: જટિલ shapes દોરવા માટે જરૂરી
    \end{itemize}
    \begin{mnemonicbox}Right Left Heading Towards\end{mnemonicbox}
\end{solutionbox}

\questionmarks{5(c)}{7}{Write a python program to draw rainbow using turtle.}
\begin{solutionbox}
    Rainbow drawing માં VIBGYOR colors સાથે concentric semi-circles વપરાય છે.

    \begin{lstlisting}[language=Python]
import turtle

def draw_rainbow():
    # Setup screen
    wn = turtle.Screen()
    wn.bgcolor("skyblue")
    
    # Setup turtle
    t = turtle.Turtle()
    t.speed(5)
    t.pensize(10)
    
    # Rainbow colors (VIBGYOR reversed)
    colors = ['violet', 'indigo', 'blue', 'green', 
              'yellow', 'orange', 'red']
    
    # શરૂઆતના position parameters
    radius = 180
    
    # Arcs દોરો
    for color in colors:
        t.penup()
        t.goto(0, -50)     # Center bottom
        t.setheading(90)   # ઉપર તરફ
        t.right(90)        # જમણી તરફ (0 deg)
        t.forward(radius)  # Arc ની શરૂઆતમાં જાઓ
        t.left(90)         # ફરી ઉપર તરફ face કરો
        t.pendown()
        
        t.pencolor(color)
        t.circle(radius, 180) # Semi-circle દોરો
        
        radius -= 20      # આગલા color માટે radius ઘટાડો
    
    t.hideturtle()
    turtle.done()

draw_rainbow()
    \end{lstlisting}

    \begin{itemize}
        \item \textbf{Looping}: Colors ના list માં iterate કરે છે
        \item \textbf{Concentric}: દરેક અંદરના arc માટે radius ઘટે છે
        \item \textbf{Arc}: \code{circle(radius, 180)} અર્ધ વર્તુળ દોરે છે
    \end{itemize}
    \begin{mnemonicbox}Colors Loop Radius Circle\end{mnemonicbox}
\end{solutionbox}

\end{document}
