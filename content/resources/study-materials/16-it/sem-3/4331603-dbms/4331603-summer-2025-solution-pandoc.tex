\documentclass[10pt,a4paper]{article}

% content/resources/templates/preamble.tex
\usepackage[margin=0.6in]{geometry}
\author{Milav Dabgar}
\usepackage{amsmath,amssymb,amsthm}
\usepackage{booktabs}
\usepackage{multirow}
\usepackage{xcolor}
\usepackage{tcolorbox}
\tcbuselibrary{breakable,skins}
\usepackage[colorlinks=true,linkcolor=blue]{hyperref}
\usepackage{titlesec}
\usepackage{enumitem}
\usepackage{tikz}
\usepackage{pgfplots}
\usepackage{circuitikz}
\usepackage[version=4]{mhchem}
\usepackage{longtable}
\usepackage{array}
\usepackage{float}
\usepackage{caption}
\usepackage{listings}

\lstset{
  basicstyle=\small\ttfamily,
  breaklines=true,
  breakatwhitespace=false,
  postbreak=\mbox{\textcolor{red}{$\hookrightarrow$}\space},
  float=false,
  numbers=left,
  numberstyle=\tiny\color{gray},
  numbersep=10pt,
  xleftmargin=2em,
  keywordstyle=\color{blue},
  commentstyle=\color{green!60!black},
  stringstyle=\color{purple},
  backgroundcolor=\color{gray!5},
  showstringspaces=false,
  tabsize=2,
  captionpos=b,
  keepspaces=true,
  columns=flexible
}

\pgfplotsset{compat=1.18}
\usetikzlibrary{shapes,arrows,positioning,calc,patterns,decorations.pathmorphing,decorations.markings,arrows.meta}

% Color scheme
\definecolor{headcolor}{RGB}{0,102,204}
\definecolor{keycolor}{RGB}{220,20,60}
\definecolor{solutioncolor}{RGB}{34,139,34}
\definecolor{mnemoniccolor}{RGB}{148,0,211}
\definecolor{codecolor}{RGB}{0,0,100}

% Spacing
\setlength{\parskip}{3pt}
\setlist[itemize]{nosep}
\setlist[enumerate]{nosep}

% Title formatting
\titleformat{\section}{\Large\bfseries\color{headcolor}}{\thesection}{1em}{}
\titleformat{\subsection}{\large\bfseries\color{headcolor}}{\thesubsection}{1em}{}

% Pandoc tightlist compatibility
\providecommand{\tightlist}{%
  \setlength{\itemsep}{0pt}\setlength{\parskip}{0pt}}

% Pandoc longtable compatibility
\newcounter{none}
\def\thenone{}


% content/resources/templates/english-boxes.tex
% This file is currently empty - it exists to maintain consistency with the import structure.
% Add custom environments here if needed in the future.


\begin{document}

\begin{center}
{\Huge\bfseries\color{headcolor} Subject Name Solutions}\\[5pt]
{\LARGE 4331603 -- Summer 2025}\\[3pt]
{\large Semester 1 Study Material}\\[3pt]
{\normalsize\textit{Detailed Solutions and Explanations}}
\end{center}

\vspace{10pt}

\subsection*{Question 1(a) [3 marks]}\label{q1a}

\textbf{Define the following terms. 1) Metadata 2) Schema 3) Data
dictionary.}

\begin{solutionbox}


{\def\LTcaptype{none} % do not increment counter
\begin{longtable}[]{@{}
  >{\raggedright\arraybackslash}p{(\linewidth - 2\tabcolsep) * \real{0.3333}}
  >{\raggedright\arraybackslash}p{(\linewidth - 2\tabcolsep) * \real{0.6667}}@{}}
\toprule\noalign{}
\begin{minipage}[b]{\linewidth}\raggedright
Term
\end{minipage} & \begin{minipage}[b]{\linewidth}\raggedright
Definition
\end{minipage} \\
\midrule\noalign{}
\endhead
\bottomrule\noalign{}
\endlastfoot
\textbf{Metadata} & Data about data that describes structure, format,
and characteristics of database \\
\textbf{Schema} & Logical structure describing database organization and
relationships \\
\textbf{Data Dictionary} & Centralized repository storing information
about database elements \\
\end{longtable}
}

\begin{itemize}
\tightlist
\item
  \textbf{Metadata}: Information describing data characteristics and
  properties
\item
  \textbf{Schema}: Blueprint defining database structure and constraints
\item
  \textbf{Data Dictionary}: Catalog of all database objects and their
  attributes
\end{itemize}

\end{solutionbox}
\begin{mnemonicbox}
``MSD - My System Dictionary''

\end{mnemonicbox}
\subsection*{Question 1(b) [4 marks]}\label{q1b}

\textbf{Write down advantages of Database Management system.}

\begin{solutionbox}


{\def\LTcaptype{none} % do not increment counter
\begin{longtable}[]{@{}ll@{}}
\toprule\noalign{}
Advantage & Description \\
\midrule\noalign{}
\endhead
\bottomrule\noalign{}
\endlastfoot
\textbf{Data Independence} & Applications independent of data storage \\
\textbf{Data Integrity} & Maintains accuracy and consistency \\
\textbf{Security Control} & User authentication and authorization \\
\textbf{Concurrent Access} & Multiple users access simultaneously \\
\end{longtable}
}

\begin{itemize}
\tightlist
\item
  \textbf{Reduced Redundancy}: Eliminates duplicate data storage
\item
  \textbf{Centralized Control}: Single point of data management
\item
  \textbf{Data Sharing}: Multiple applications can use same data
\item
  \textbf{Backup Recovery}: Automatic data protection mechanisms
\end{itemize}

\end{solutionbox}
\begin{mnemonicbox}
``DISC-RCDB - Database Is Super Cool''

\end{mnemonicbox}
\subsection*{Question 1(c) [7 marks]}\label{q1c}

\textbf{Explain Responsibilities of DBA.}

\begin{solutionbox}


{\def\LTcaptype{none} % do not increment counter
\begin{longtable}[]{@{}ll@{}}
\toprule\noalign{}
Responsibility & Tasks \\
\midrule\noalign{}
\endhead
\bottomrule\noalign{}
\endlastfoot
\textbf{Database Design} & Create logical and physical structures \\
\textbf{Security Management} & Control user access and permissions \\
\textbf{Performance Tuning} & Optimize queries and database
operations \\
\textbf{Backup Recovery} & Ensure data protection and restoration \\
\textbf{User Management} & Create accounts and assign privileges \\
\end{longtable}
}

\begin{center}
\textbf{Mermaid Diagram (Code)}
\begin{verbatim}
{Shaded}
{Highlighting}[]
graph TD
    A[DBA Responsibilities] {-{-}{} B[Database Design]}
    A {-{-}{} C[Security Management]}
    A {-{-}{} D[Performance Tuning]}
    A {-{-}{} E[Backup \& Recovery]}
    A {-{-}{} F[User Management]}
    A {-{-}{} G[System Monitoring]}
{Highlighting}
{Shaded}
\end{verbatim}
\end{center}

\begin{itemize}
\tightlist
\item
  \textbf{Database Installation}: Setup and configure DBMS software
\item
  \textbf{Data Migration}: Transfer data between systems safely
\item
  \textbf{Documentation}: Maintain database schemas and procedures
\item
  \textbf{Monitoring}: Track system performance and resource usage
\item
  \textbf{Troubleshooting}: Resolve database issues and errors
\end{itemize}

\end{solutionbox}
\begin{mnemonicbox}
``DSPBU-DMT - DBA Solves Problems By Understanding
Database Management Tasks''

\end{mnemonicbox}
\subsection*{Question 1(c OR) [7
marks]}\label{question-1c-or-7-marks}

\textbf{What is data abstraction? Explain three level ANSI SPARC
architecture in detail.}

\begin{solutionbox}

\textbf{Data Abstraction}: Hiding complex database implementation
details from users while providing simplified interfaces.

\begin{center}
\textbf{Mermaid Diagram (Code)}
\begin{verbatim}
{Shaded}
{Highlighting}[]
graph LR
    A[External Level] {-{-}{} B[Conceptual Level]}
    B {-{-}{} C[Internal Level]}
    A1[User Views] {-{-}{} A}
    B1[Logical Schema] {-{-}{} B}
    C1[Physical Storage] {-{-}{} C}
{Highlighting}
{Shaded}
\end{verbatim}
\end{center}


{\def\LTcaptype{none} % do not increment counter
\begin{longtable}[]{@{}
  >{\raggedright\arraybackslash}p{(\linewidth - 4\tabcolsep) * \real{0.2593}}
  >{\raggedright\arraybackslash}p{(\linewidth - 4\tabcolsep) * \real{0.4815}}
  >{\raggedright\arraybackslash}p{(\linewidth - 4\tabcolsep) * \real{0.2593}}@{}}
\toprule\noalign{}
\begin{minipage}[b]{\linewidth}\raggedright
Level
\end{minipage} & \begin{minipage}[b]{\linewidth}\raggedright
Description
\end{minipage} & \begin{minipage}[b]{\linewidth}\raggedright
Users
\end{minipage} \\
\midrule\noalign{}
\endhead
\bottomrule\noalign{}
\endlastfoot
\textbf{External Level} & Individual user views and applications & End
Users \\
\textbf{Conceptual Level} & Complete logical database structure &
Database Designers \\
\textbf{Internal Level} & Physical storage and access methods & System
Programmers \\
\end{longtable}
}

\begin{itemize}
\tightlist
\item
  \textbf{External Level}: Multiple user views hiding complexity
\item
  \textbf{Conceptual Level}: Complete database schema without storage
  details
\item
  \textbf{Internal Level}: Physical file organization and indexing
\item
  \textbf{Data Independence}: Changes at one level don't affect others
\end{itemize}

\end{solutionbox}
\begin{mnemonicbox}
``ECI - Every Computer Implements''

\end{mnemonicbox}
\subsection*{Question 2(a) [3 marks]}\label{q2a}

\textbf{Differentiate Schema vs Instance}

\begin{solutionbox}


{\def\LTcaptype{none} % do not increment counter
\begin{longtable}[]{@{}
  >{\raggedright\arraybackslash}p{(\linewidth - 4\tabcolsep) * \real{0.3077}}
  >{\raggedright\arraybackslash}p{(\linewidth - 4\tabcolsep) * \real{0.3077}}
  >{\raggedright\arraybackslash}p{(\linewidth - 4\tabcolsep) * \real{0.3846}}@{}}
\toprule\noalign{}
\begin{minipage}[b]{\linewidth}\raggedright
Aspect
\end{minipage} & \begin{minipage}[b]{\linewidth}\raggedright
Schema
\end{minipage} & \begin{minipage}[b]{\linewidth}\raggedright
Instance
\end{minipage} \\
\midrule\noalign{}
\endhead
\bottomrule\noalign{}
\endlastfoot
\textbf{Definition} & Database structure blueprint & Actual data at
specific time \\
\textbf{Nature} & Static logical design & Dynamic data content \\
\textbf{Changes} & Rarely modified & Frequently updated \\
\end{longtable}
}

\begin{itemize}
\tightlist
\item
  \textbf{Schema}: Describes database organization and constraints
\item
  \textbf{Instance}: Snapshot of database content at particular moment
\item
  \textbf{Relationship}: Schema defines structure, instance contains
  data
\end{itemize}

\end{solutionbox}
\begin{mnemonicbox}
``SI - Structure vs Information''

\end{mnemonicbox}
\subsection*{Question 2(b) [4 marks]}\label{q2b}

\textbf{Explain Specialization with example.}

\begin{solutionbox}

\textbf{Specialization}: Process of creating subclasses from superclass
based on specific characteristics.

\begin{verbatim}
erDiagram
    EMPLOYEE \{
        int emp\_id
        string name
        float salary
    \}
    MANAGER \{
        string department
        int team\_size
    \}
    DEVELOPER \{
        string programming\_language
        string project
    \}
    
    EMPLOYEE ||{-{-}|| MANAGER : specializes}
    EMPLOYEE ||{-{-}|| DEVELOPER : specializes}
\end{verbatim}

\begin{itemize}
\tightlist
\item
  \textbf{Top-Down Approach}: From general entity to specific entities
\item
  \textbf{Inheritance}: Subclasses inherit superclass attributes
\item
  \textbf{Disjoint}: Manager and Developer are separate categories
\item
  \textbf{Example}: Employee specialized into Manager and Developer
\end{itemize}

\end{solutionbox}
\begin{mnemonicbox}
``STID - Specialization Takes Inheritance Down''

\end{mnemonicbox}
\subsection*{Question 2(c) [7 marks]}\label{q2c}

\textbf{What is ER diagram? Explain different symbols used in E-R
diagram with example.}

\begin{solutionbox}

\textbf{ER Diagram}: Graphical representation showing entities,
attributes, and relationships in database design.


{\def\LTcaptype{none} % do not increment counter
\begin{longtable}[]{@{}llll@{}}
\toprule\noalign{}
Symbol & Shape & Purpose & Example \\
\midrule\noalign{}
\endhead
\bottomrule\noalign{}
\endlastfoot
\textbf{Entity} & Rectangle & Real-world object & Student, Course \\
\textbf{Attribute} & Oval & Entity properties & Name, Age, ID \\
\textbf{Relationship} & Diamond & Entity connections & Enrolls, Takes \\
\textbf{Primary Key} & Underlined oval & Unique identifier &
Student\_ID \\
\end{longtable}
}

\begin{verbatim}
erDiagram
    STUDENT \{
        int student\_id PK
        string name
        string email
        date birth\_date
    \}
    COURSE \{
        string course\_id PK
        string course\_name
        int credits
    \}
    ENROLLMENT \{
        date enrollment\_date
        string grade
    \}
    
    STUDENT ||{-{-}o\{ ENROLLMENT : enrolls}
    COURSE ||{-{-}o\{ ENROLLMENT : includes}
\end{verbatim}

\begin{itemize}
\tightlist
\item
  \textbf{Entity Sets}: Collection of similar entities with same
  attributes
\item
  \textbf{Weak Entity}: Depends on strong entity for identification
\item
  \textbf{Cardinality}: Defines relationship participation (1:1, 1:M,
  M:N)
\item
  \textbf{Participation}: Total (double line) or Partial (single line)
\end{itemize}

\end{solutionbox}
\begin{mnemonicbox}
``EARP - Entities And Relationships Program''

\end{mnemonicbox}
\subsection*{Question 2(a OR) [3
marks]}\label{question-2a-or-3-marks}

\textbf{Differentiate DA vs DBA.}

\begin{solutionbox}


{\def\LTcaptype{none} % do not increment counter
\begin{longtable}[]{@{}
  >{\raggedright\arraybackslash}p{(\linewidth - 4\tabcolsep) * \real{0.1270}}
  >{\raggedright\arraybackslash}p{(\linewidth - 4\tabcolsep) * \real{0.3968}}
  >{\raggedright\arraybackslash}p{(\linewidth - 4\tabcolsep) * \real{0.4762}}@{}}
\toprule\noalign{}
\begin{minipage}[b]{\linewidth}\raggedright
Aspect
\end{minipage} & \begin{minipage}[b]{\linewidth}\raggedright
Data Administrator (DA)
\end{minipage} & \begin{minipage}[b]{\linewidth}\raggedright
Database Administrator (DBA)
\end{minipage} \\
\midrule\noalign{}
\endhead
\bottomrule\noalign{}
\endlastfoot
\textbf{Focus} & Data policies and standards & Technical database
operations \\
\textbf{Level} & Strategic planning & Operational implementation \\
\textbf{Scope} & Organization-wide data & Specific database systems \\
\end{longtable}
}

\begin{itemize}
\tightlist
\item
  \textbf{DA}: Manages data as organizational resource
\item
  \textbf{DBA}: Handles technical database maintenance and performance
\item
  \textbf{Collaboration}: DA sets policies, DBA implements them
\end{itemize}

\end{solutionbox}
\begin{mnemonicbox}
``DA-DBA: Design Authority - Database Builder
Administrator''

\end{mnemonicbox}
\subsection*{Question 2(b OR) [4
marks]}\label{question-2b-or-4-marks}

\textbf{Explain Generalization with example.}

\begin{solutionbox}

\textbf{Generalization}: Bottom-up process combining similar entities
into common superclass.

\begin{verbatim}
erDiagram
    VEHICLE \{
        string vehicle\_id
        string brand
        int year
        string color
    \}
    CAR \{
        int doors
        string fuel\_type
    \}
    MOTORCYCLE \{
        int engine\_cc
        string bike\_type
    \}
    
    VEHICLE ||{-{-}|| CAR : generalizes}
    VEHICLE ||{-{-}|| MOTORCYCLE : generalizes}
\end{verbatim}

\begin{itemize}
\tightlist
\item
  \textbf{Bottom-Up Approach}: From specific entities to general entity
\item
  \textbf{Common Attributes}: Shared properties moved to superclass
\item
  \textbf{Specialization Reverse}: Opposite of specialization process
\item
  \textbf{Example}: Car and Motorcycle generalized into Vehicle
\end{itemize}

\end{solutionbox}
\begin{mnemonicbox}
``GBCS - Generalization Brings Common Superclass''

\end{mnemonicbox}
\subsection*{Question 2(c OR) [7
marks]}\label{question-2c-or-7-marks}

\textbf{What is attribute? Explain different types of attributes with
example.}

\begin{solutionbox}

\textbf{Attribute}: Property or characteristic that describes an entity.


{\def\LTcaptype{none} % do not increment counter
\begin{longtable}[]{@{}lll@{}}
\toprule\noalign{}
Attribute Type & Description & Example \\
\midrule\noalign{}
\endhead
\bottomrule\noalign{}
\endlastfoot
\textbf{Simple} & Cannot be divided further & Age, Name \\
\textbf{Composite} & Can be subdivided & Address (Street, City, ZIP) \\
\textbf{Single-valued} & One value per entity & Student\_ID \\
\textbf{Multi-valued} & Multiple values possible & Phone\_numbers \\
\textbf{Derived} & Calculated from other attributes & Age from
Birth\_date \\
\end{longtable}
}

\begin{center}
\textbf{Mermaid Diagram (Code)}
\begin{verbatim}
{Shaded}
{Highlighting}[]
graph TD
    A[Attributes] {-{-}{} B[Simple]}
    A {-{-}{} C[Composite]}
    A {-{-}{} D[Single{-}valued]}
    A {-{-}{} E[Multi{-}valued]}
    A {-{-}{} F[Derived]}
    A {-{-}{} G[Key Attributes]}
    
    C {-{-}{} C1[Address: Street, City, ZIP]}
    E {-{-}{} E1[Phone: Mobile, Home, Work]}
    F {-{-}{} F1[Age calculated from DOB]}
{Highlighting}
{Shaded}
\end{verbatim}
\end{center}

\begin{itemize}
\tightlist
\item
  \textbf{Key Attribute}: Uniquely identifies entity instances
\item
  \textbf{Null Values}: Attributes that may have no value
\item
  \textbf{Default Values}: Predetermined values when not specified
\item
  \textbf{Constraints}: Rules governing attribute values
\end{itemize}

\end{solutionbox}
\begin{mnemonicbox}
``SCSMD-K - Simple Composite Single Multi Derived
Key''

\end{mnemonicbox}
\subsection*{Question 3(a) [3 marks]}\label{q3a}

\textbf{Explain the GRANT and REVOKE statement in SQL.}

\begin{solutionbox}


{\def\LTcaptype{none} % do not increment counter
\begin{longtable}[]{@{}
  >{\raggedright\arraybackslash}p{(\linewidth - 4\tabcolsep) * \real{0.3056}}
  >{\raggedright\arraybackslash}p{(\linewidth - 4\tabcolsep) * \real{0.2500}}
  >{\raggedright\arraybackslash}p{(\linewidth - 4\tabcolsep) * \real{0.4444}}@{}}
\toprule\noalign{}
\begin{minipage}[b]{\linewidth}\raggedright
Statement
\end{minipage} & \begin{minipage}[b]{\linewidth}\raggedright
Purpose
\end{minipage} & \begin{minipage}[b]{\linewidth}\raggedright
Syntax Example
\end{minipage} \\
\midrule\noalign{}
\endhead
\bottomrule\noalign{}
\endlastfoot
\textbf{GRANT} & Provides privileges to users & GRANT SELECT ON table TO
user \\
\textbf{REVOKE} & Removes user privileges & REVOKE INSERT ON table FROM
user \\
\end{longtable}
}

\begin{verbatim}
{-{-} Grant privileges}
GRANT SELECT, INSERT ON employees TO john;
GRANT ALL PRIVILEGES ON database TO admin;

{-{-} Revoke privileges  }
REVOKE DELETE ON employees FROM john;
REVOKE ALL ON database FROM user;
\end{verbatim}

\begin{itemize}
\tightlist
\item
  \textbf{Privileges}: SELECT, INSERT, UPDATE, DELETE, ALL
\item
  \textbf{Objects}: Tables, views, databases, procedures
\item
  \textbf{Security}: Controls data access and modification rights
\end{itemize}

\end{solutionbox}
\begin{mnemonicbox}
``GR - Grant Rights, Remove Rights''

\end{mnemonicbox}
\subsection*{Question 3(b) [4 marks]}\label{q3b}

\textbf{Explain following Character functions. 1) INITCAP 2) SUBSTR}

\begin{solutionbox}


{\def\LTcaptype{none} % do not increment counter
\begin{longtable}[]{@{}
  >{\raggedright\arraybackslash}p{(\linewidth - 6\tabcolsep) * \real{0.2703}}
  >{\raggedright\arraybackslash}p{(\linewidth - 6\tabcolsep) * \real{0.2432}}
  >{\raggedright\arraybackslash}p{(\linewidth - 6\tabcolsep) * \real{0.2432}}
  >{\raggedright\arraybackslash}p{(\linewidth - 6\tabcolsep) * \real{0.2432}}@{}}
\toprule\noalign{}
\begin{minipage}[b]{\linewidth}\raggedright
Function
\end{minipage} & \begin{minipage}[b]{\linewidth}\raggedright
Purpose
\end{minipage} & \begin{minipage}[b]{\linewidth}\raggedright
Syntax
\end{minipage} & \begin{minipage}[b]{\linewidth}\raggedright
Example
\end{minipage} \\
\midrule\noalign{}
\endhead
\bottomrule\noalign{}
\endlastfoot
\textbf{INITCAP} & Capitalizes first letter of each word &
INITCAP(string) & INITCAP(`hello world') = `Hello World' \\
\textbf{SUBSTR} & Extracts substring from string & SUBSTR(string, start,
length) & SUBSTR(`Database', 1, 4) = `Data' \\
\end{longtable}
}

\begin{verbatim}
{-{-} INITCAP examples}
SELECT INITCAP({database management}) FROM dual; {-{-} Database Management}
SELECT INITCAP({gtu university}) FROM dual; {-{-} Gtu University}

{-{-} SUBSTR examples  }
SELECT SUBSTR({Programming}, 1, 7) FROM dual; {-{-} Program}
SELECT SUBSTR({Database}, 5) FROM dual; {-{-} base}
\end{verbatim}

\begin{itemize}
\tightlist
\item
  \textbf{INITCAP}: Converts string to proper case format
\item
  \textbf{SUBSTR}: Parameters are string, starting position, optional
  length
\item
  \textbf{Usage}: Text formatting and string manipulation operations
\end{itemize}

\end{solutionbox}
\begin{mnemonicbox}
``IS - Initialize String, Split String''

\end{mnemonicbox}
\subsection*{Question 3(c) [7 marks]}\label{q3c}

\textbf{Consider following tables and write answers for the given
queries.} \textbf{stud\_master (enroll\_no, name, city, dept)}

\begin{solutionbox}

\begin{verbatim}
{-{-} 1. Display all student details who study in IT dept}
SELECT * FROM stud\_master 
WHERE dept = {IT};

{-{-} 2. Retrieve all information about name where name begins with p}
SELECT * FROM stud\_master 
WHERE name LIKE {p\%};

{-{-} 3. Insert new student to table}
INSERT INTO stud\_master (enroll\_no, name, city, dept) 
VALUES ({202501}, {John Smith}, {Mumbai}, {CS});

{-{-} 4. Add new column gender to table stud\_master}
ALTER TABLE stud\_master 
ADD gender VARCHAR(10);

{-{-} 5. Count number of rows for stud\_master table}
SELECT COUNT(*) FROM stud\_master;

{-{-} 6. Display all student details in descending order of enroll\_no}
SELECT * FROM stud\_master 
ORDER BY enroll\_no DESC;

{-{-} 7. Destroy table stud\_master along with data}
DROP TABLE stud\_master;
\end{verbatim}


{\def\LTcaptype{none} % do not increment counter
\begin{longtable}[]{@{}lll@{}}
\toprule\noalign{}
Query Type & SQL Command & Purpose \\
\midrule\noalign{}
\endhead
\bottomrule\noalign{}
\endlastfoot
\textbf{SELECT} & Retrieves data & Display records \\
\textbf{INSERT} & Adds new data & Create records \\
\textbf{ALTER} & Modifies structure & Add columns \\
\textbf{COUNT} & Aggregate function & Count rows \\
\end{longtable}
}

\end{solutionbox}
\begin{mnemonicbox}
``SIAC-DOC - SQL Is A Complete Database Operations
Collection''

\end{mnemonicbox}
\subsection*{Question 3(a OR) [3
marks]}\label{question-3a-or-3-marks}

\textbf{Explain equi join with example in SQL.}

\begin{solutionbox}

\textbf{Equi Join}: Join operation using equality condition to combine
tables based on common columns.

\begin{verbatim}
{-{-} Equi Join example}
SELECT s.name, c.course\_name
FROM students s, courses c
WHERE s.course\_id = c.course\_id;

{-{-} Using JOIN syntax}
SELECT s.name, c.course\_name  
FROM students s
JOIN courses c ON s.course\_id = c.course\_id;
\end{verbatim}

\begin{itemize}
\tightlist
\item
  \textbf{Equality Operator}: Uses = to match column values
\item
  \textbf{Common Columns}: Tables must have related attributes
\item
  \textbf{Result}: Combined data from multiple tables based on matches
\end{itemize}

\end{solutionbox}
\begin{mnemonicbox}
``EJ - Equal Join''

\end{mnemonicbox}
\subsection*{Question 3(b OR) [4
marks]}\label{question-3b-or-4-marks}

\textbf{Explain following Aggregate functions. 1) MAX 2) SUM}

\begin{solutionbox}


{\def\LTcaptype{none} % do not increment counter
\begin{longtable}[]{@{}llll@{}}
\toprule\noalign{}
Function & Purpose & Syntax & Example \\
\midrule\noalign{}
\endhead
\bottomrule\noalign{}
\endlastfoot
\textbf{MAX} & Returns maximum value & MAX(column) & MAX(salary) =
50000 \\
\textbf{SUM} & Returns total of values & SUM(column) & SUM(marks) =
450 \\
\end{longtable}
}

\begin{verbatim}
{-{-} MAX examples}
SELECT MAX(salary) FROM employees; {-{-} Highest salary}
SELECT MAX(age) FROM students; {-{-} Oldest student age}

{-{-} SUM examples}
SELECT SUM(credits) FROM courses; {-{-} Total credits}
SELECT SUM(price * quantity) FROM orders; {-{-} Total order value}
\end{verbatim}

\begin{itemize}
\tightlist
\item
  \textbf{Aggregate Functions}: Operate on multiple rows, return single
  value
\item
  \textbf{NULL Handling}: Ignore NULL values in calculations
\item
  \textbf{GROUP BY}: Can be used with grouping for category-wise results
\end{itemize}

\end{solutionbox}
\begin{mnemonicbox}
``MS - Maximum Sum''

\end{mnemonicbox}
\subsection*{Question 3(c OR) [7
marks]}\label{question-3c-or-7-marks}

\textbf{Write SQL queries for the following table:}
\textbf{PRODUCT\_Master: (prod\_no, prod\_name, profit, quantity,
sell\_price, cost\_price)}

\begin{solutionbox}

\begin{verbatim}
{-{-} 1. Create table PRODUCT\_Master}
CREATE TABLE PRODUCT\_Master (
    prod\_no VARCHAR(10) PRIMARY KEY,
    prod\_name VARCHAR(50),
    profit NUMBER(10,2),
    quantity NUMBER,
    sell\_price NUMBER(10,2),
    cost\_price NUMBER(10,2)
);

{-{-} 2. Insert one record in this table}
INSERT INTO PRODUCT\_Master VALUES 
({P001}, {Laptop}, 15000, 10, 45000, 30000);

{-{-} 3. Find product having profit greater than 20000}
SELECT * FROM PRODUCT\_Master 
WHERE profit {} 20000;

{-{-} 4. Delete product having quantity less than 5}
DELETE FROM PRODUCT\_Master 
WHERE quantity {} 5;

{-{-} 5. Add 2\% profit in product having sell price greater than 5000}
UPDATE PRODUCT\_Master 
SET profit = profit * 1.02 
WHERE sell\_price {} 5000;

{-{-} 6. Add new field total\_price to PRODUCT\_Master}
ALTER TABLE PRODUCT\_Master 
ADD total\_price NUMBER(10,2);

{-{-} 7. Find product name having no duplicate data}
SELECT DISTINCT prod\_name FROM PRODUCT\_Master;
\end{verbatim}

\end{solutionbox}
\begin{mnemonicbox}
``CIDFAUD - Create Insert Delete Find Add Update
Distinct''

\end{mnemonicbox}
\subsection*{Question 4(a) [3 marks]}\label{q4a}

\textbf{Explain fully functional dependency with example.}

\begin{solutionbox}

\textbf{Fully Functional Dependency}: Attribute is fully functionally
dependent if it depends on complete primary key, not on partial key.


{\def\LTcaptype{none} % do not increment counter
\begin{longtable}[]{@{}
  >{\raggedright\arraybackslash}p{(\linewidth - 4\tabcolsep) * \real{0.4211}}
  >{\raggedright\arraybackslash}p{(\linewidth - 4\tabcolsep) * \real{0.3421}}
  >{\raggedright\arraybackslash}p{(\linewidth - 4\tabcolsep) * \real{0.2368}}@{}}
\toprule\noalign{}
\begin{minipage}[b]{\linewidth}\raggedright
Dependency Type
\end{minipage} & \begin{minipage}[b]{\linewidth}\raggedright
Definition
\end{minipage} & \begin{minipage}[b]{\linewidth}\raggedright
Example
\end{minipage} \\
\midrule\noalign{}
\endhead
\bottomrule\noalign{}
\endlastfoot
\textbf{Full FD} & Depends on entire key & (Student\_ID, Course\_ID) \rightarrow
Grade \\
\textbf{Partial FD} & Depends on part of key & (Student\_ID, Course\_ID)
\rightarrow Student\_Name \\
\end{longtable}
}

\begin{verbatim}
Example: Student_Course(Student_ID, Course_ID, Student_Name, Grade)

Full FD: (Student_ID, Course_ID) \rightarrow Grade
Partial FD: Student_ID \rightarrow Student_Name
\end{verbatim}

\begin{itemize}
\tightlist
\item
  \textbf{Complete Key}: All attributes of composite primary key
  required
\item
  \textbf{Non-key Attribute}: Depends on full primary key combination
\item
  \textbf{2NF Requirement}: Eliminates partial dependencies
\end{itemize}

\end{solutionbox}
\begin{mnemonicbox}
``FFD - Full Function Dependency''

\end{mnemonicbox}
\subsection*{Question 4(b) [4 marks]}\label{q4b}

\textbf{Consider following relational schema \& give Relational Algebra
Expressions:} \textbf{Employee (Emp\_name, Emp\_id, birth\_date, Post,
salary)}

\begin{solutionbox}

\begin{verbatim}
(i) List all Employees having Post="Clerk"
σ(Post='Clerk')(Employee)

(ii) Find Emp_id and Emp_name having salary > 2000 and post='Manager'
π(Emp_id, Emp_name)(σ(salary>2000 \wedge Post='Manager')(Employee))
\end{verbatim}


{\def\LTcaptype{none} % do not increment counter
\begin{longtable}[]{@{}lll@{}}
\toprule\noalign{}
Symbol & Operation & Purpose \\
\midrule\noalign{}
\endhead
\bottomrule\noalign{}
\endlastfoot
\textbf{σ} & Selection & Filter rows based on condition \\
\textbf{π} & Projection & Select specific columns \\
\textbf{\wedge} & AND & Logical conjunction \\
\end{longtable}
}

\begin{itemize}
\tightlist
\item
  \textbf{Selection (σ)}: Chooses rows meeting specified conditions
\item
  \textbf{Projection (π)}: Selects required columns from result
\item
  \textbf{Combined Operations}: Can use multiple operations together
\end{itemize}

\end{solutionbox}
\begin{mnemonicbox}
``SPA - Select Project And''

\end{mnemonicbox}
\subsection*{Question 4(c) [7 marks]}\label{q4c}

\textbf{What are the criteria of 2NF? Find different functional
dependencies and normalize into 2NF.}

\begin{solutionbox}

\textbf{2NF Criteria}:

\begin{itemize}
\tightlist
\item
  Must be in 1NF
\item
  No partial functional dependencies on primary key
\end{itemize}

\textbf{Given Table}: Student\_Course(Student\_ID, Course\_ID,
Student\_Name, Course\_Name)

\textbf{Functional Dependencies}:

\begin{verbatim}
Student_ID \rightarrow Student_Name (Partial FD)
Course_ID \rightarrow Course_Name (Partial FD)
(Student_ID, Course_ID) \rightarrow (Student_Name, Course_Name) (Full FD)
\end{verbatim}

\textbf{2NF Normalization}:

\begin{verbatim}
{-{-} Table 1: Students}
Students(Student\_ID, Student\_Name)

{-{-} Table 2: Courses  }
Courses(Course\_ID, Course\_Name)

{-{-} Table 3: Enrollment}
Enrollment(Student\_ID, Course\_ID)
\end{verbatim}

\begin{verbatim}
erDiagram
    STUDENTS \{
        string Student\_ID PK
        string Student\_Name
    \}
    COURSES \{
        string Course\_ID PK
        string Course\_Name
    \}
    ENROLLMENT \{
        string Student\_ID PK,FK
        string Course\_ID PK,FK
    \}
    
    STUDENTS ||{-{-}o\{ ENROLLMENT : enrolls}
    COURSES ||{-{-}o\{ ENROLLMENT : includes}
\end{verbatim}

\end{solutionbox}
\begin{mnemonicbox}
``2NF - Two Normal Form removes partial
dependencies''

\end{mnemonicbox}
\subsection*{Question 4(a OR) [3
marks]}\label{question-4a-or-3-marks}

\textbf{Explain 3NF with example.}

\begin{solutionbox}

\textbf{3NF (Third Normal Form)}: Table in 2NF with no transitive
dependencies on primary key.


{\def\LTcaptype{none} % do not increment counter
\begin{longtable}[]{@{}lll@{}}
\toprule\noalign{}
Normal Form & Requirement & Eliminates \\
\midrule\noalign{}
\endhead
\bottomrule\noalign{}
\endlastfoot
\textbf{3NF} & In 2NF + No transitive dependencies & Transitive FD \\
\end{longtable}
}

\begin{verbatim}
Example: Employee(Emp_ID, Dept_ID, Dept_Name)

Transitive Dependency: Emp_ID \rightarrow Dept_ID \rightarrow Dept_Name

3NF Solution:
Employee(Emp_ID, Dept_ID)
Department(Dept_ID, Dept_Name)
\end{verbatim}

\begin{itemize}
\tightlist
\item
  \textbf{Transitive Dependency}: A \rightarrow B \rightarrow C where A is primary key
\item
  \textbf{Non-key to Non-key}: Dependency between non-key attributes
\item
  \textbf{Decomposition}: Split table to remove transitive dependencies
\end{itemize}

\end{solutionbox}
\begin{mnemonicbox}
``3NF - Third Normal Form removes Transitive
dependencies''

\end{mnemonicbox}
\subsection*{Question 4(b OR) [4
marks]}\label{question-4b-or-4-marks}

\textbf{Consider following Relational Schema and give Relational Algebra
Expression:} \textbf{Students (Name, SPI, DOB, Enrollment No)}

\begin{solutionbox}

\begin{verbatim}
(i) List all students whose SPI is greater than 7.0
σ(SPI > 7.0)(Students)

(ii) List name, DOB of student whose enrollment number is 007
π(Name, DOB)(σ(Enrollment_No = '007')(Students))
\end{verbatim}


{\def\LTcaptype{none} % do not increment counter
\begin{longtable}[]{@{}lll@{}}
\toprule\noalign{}
Query & Relational Algebra & Purpose \\
\midrule\noalign{}
\endhead
\bottomrule\noalign{}
\endlastfoot
\textbf{Filter} & σ(condition) & Select rows \\
\textbf{Project} & π(attributes) & Select columns \\
\end{longtable}
}

\begin{itemize}
\tightlist
\item
  \textbf{Selection First}: Apply conditions before projection
\item
  \textbf{Specific Value}: Use quotes for string literals
\item
  \textbf{Column Names}: Exact attribute names required
\end{itemize}

\end{solutionbox}
\begin{mnemonicbox}
``SPI-DOB: Select Project Information - Display
Output Better''

\end{mnemonicbox}
\subsection*{Question 4(c OR) [7
marks]}\label{question-4c-or-7-marks}

\textbf{What are criteria of 1NF? Normalize given table into 1NF with
two different techniques.}

\begin{solutionbox}

\textbf{1NF Criteria}:

\begin{itemize}
\tightlist
\item
  Each cell contains single atomic value
\item
  No repeating groups or arrays
\item
  Each row must be unique
\end{itemize}

\textbf{Given Table}:

{\def\LTcaptype{none} % do not increment counter
\begin{longtable}[]{@{}lll@{}}
\toprule\noalign{}
EnrollmentNo & Name & Subjects \\
\midrule\noalign{}
\endhead
\bottomrule\noalign{}
\endlastfoot
001 & DEF & Maths,Physics,Chemistry \\
002 & XYZ & History,Biology,English \\
\end{longtable}
}

\textbf{Technique 1 - Separate Rows}:

{\def\LTcaptype{none} % do not increment counter
\begin{longtable}[]{@{}lll@{}}
\toprule\noalign{}
EnrollmentNo & Name & Subject \\
\midrule\noalign{}
\endhead
\bottomrule\noalign{}
\endlastfoot
001 & DEF & Maths \\
001 & DEF & Physics \\
001 & DEF & Chemistry \\
002 & XYZ & History \\
002 & XYZ & Biology \\
002 & XYZ & English \\
\end{longtable}
}

\textbf{Technique 2 - Separate Tables}:

\begin{verbatim}
{-{-} Students Table}
Students(EnrollmentNo, Name)

{-{-} Subjects Table  }
Subjects(SubjectID, SubjectName)

{-{-} Student\_Subjects Table}
Student\_Subjects(EnrollmentNo, SubjectID)
\end{verbatim}

\end{solutionbox}
\begin{mnemonicbox}
``1NF - One Normal Form creates Atomic values''

\end{mnemonicbox}
\subsection*{Question 5(a) [3 marks]}\label{q5a}

\textbf{Explain ACID properties of transaction.}

\begin{solutionbox}


{\def\LTcaptype{none} % do not increment counter
\begin{longtable}[]{@{}
  >{\raggedright\arraybackslash}p{(\linewidth - 4\tabcolsep) * \real{0.3125}}
  >{\raggedright\arraybackslash}p{(\linewidth - 4\tabcolsep) * \real{0.4062}}
  >{\raggedright\arraybackslash}p{(\linewidth - 4\tabcolsep) * \real{0.2812}}@{}}
\toprule\noalign{}
\begin{minipage}[b]{\linewidth}\raggedright
Property
\end{minipage} & \begin{minipage}[b]{\linewidth}\raggedright
Description
\end{minipage} & \begin{minipage}[b]{\linewidth}\raggedright
Purpose
\end{minipage} \\
\midrule\noalign{}
\endhead
\bottomrule\noalign{}
\endlastfoot
\textbf{Atomicity} & All or nothing execution & Transaction
completeness \\
\textbf{Consistency} & Database remains valid & Data integrity \\
\textbf{Isolation} & Concurrent transactions independent & Avoid
interference \\
\textbf{Durability} & Committed changes permanent & Data persistence \\
\end{longtable}
}

\begin{itemize}
\tightlist
\item
  \textbf{Atomicity}: Transaction executes completely or not at all
\item
  \textbf{Consistency}: Database constraints maintained before/after
  transaction
\item
  \textbf{Isolation}: Transactions don't interfere with each other
\item
  \textbf{Durability}: Once committed, changes survive system failures
\end{itemize}

\end{solutionbox}
\begin{mnemonicbox}
``ACID - All Consistent Isolated Durable''

\end{mnemonicbox}
\subsection*{Question 5(b) [4 marks]}\label{q5b}

\textbf{Create following table with specification:} \textbf{STUDENT:
(stu\_id, stu\_name, Address, City, contact\_no, Branch\_name)}

\begin{solutionbox}

\begin{verbatim}
CREATE TABLE STUDENT (
    stu\_id VARCHAR(10) PRIMARY KEY,
    stu\_name VARCHAR(50) NOT NULL,
    Address VARCHAR(100),
    City VARCHAR(30),
    contact\_no NUMBER(10),
    Branch\_name VARCHAR(20) CHECK (Branch\_name IN ({IT}, {Computer}, {Electrical}, {Civil}))
);
\end{verbatim}


{\def\LTcaptype{none} % do not increment counter
\begin{longtable}[]{@{}lll@{}}
\toprule\noalign{}
Constraint & Purpose & Implementation \\
\midrule\noalign{}
\endhead
\bottomrule\noalign{}
\endlastfoot
\textbf{NOT NULL} & Mandatory field & stu\_name NOT NULL \\
\textbf{CHECK} & Value validation & Branch\_name IN (\ldots) \\
\end{longtable}
}

\begin{itemize}
\tightlist
\item
  \textbf{Primary Key}: stu\_id uniquely identifies each student
\item
  \textbf{NOT NULL}: stu\_name cannot be empty
\item
  \textbf{CHECK Constraint}: Branch\_name limited to specified values
\item
  \textbf{Data Types}: Appropriate sizes for each field
\end{itemize}

\end{solutionbox}
\begin{mnemonicbox}
``CNPD - Constraints Names Primary Datatypes''

\end{mnemonicbox}
\subsection*{Question 5(c) [7 marks]}\label{q5c}

\textbf{What is trigger? Write syntax to create trigger in oracle.
Create simple trigger.}

\begin{solutionbox}

\textbf{Trigger}: Special stored procedure that automatically executes
in response to database events.

\textbf{Oracle Trigger Syntax}:

\begin{verbatim}
CREATE [OR REPLACE] TRIGGER trigger\_name
\{BEFORE | AFTER | INSTEAD OF\ \{}INSERT | UPDATE | DELETE\}
ON table\_name
[FOR EACH ROW]
[WHEN condition]
DECLARE
    {-{-} Variable declarations}
BEGIN
    {-{-} Trigger logic}
END;
\end{verbatim}

\textbf{Simple Trigger Example}:

\begin{verbatim}
CREATE OR REPLACE TRIGGER display\_student\_trigger
BEFORE INSERT ON STUDENT
FOR EACH ROW
BEGIN
    DBMS\_OUTPUT.PUT\_LINE({Inserting student: } || :NEW.stu\_name || 
                        { with ID: } || :NEW.stu\_id);
END;
\end{verbatim}


{\def\LTcaptype{none} % do not increment counter
\begin{longtable}[]{@{}lll@{}}
\toprule\noalign{}
Trigger Type & When Executed & Purpose \\
\midrule\noalign{}
\endhead
\bottomrule\noalign{}
\endlastfoot
\textbf{BEFORE} & Before DML operation & Validation, modification \\
\textbf{AFTER} & After DML operation & Logging, auditing \\
\textbf{FOR EACH ROW} & Row-level trigger & Per row execution \\
\end{longtable}
}

\begin{itemize}
\tightlist
\item
  \textbf{:NEW}: References new values being inserted/updated
\item
  \textbf{:OLD}: References old values being deleted/updated
\item
  \textbf{Automatic Execution}: Fires automatically on specified events
\item
  \textbf{Business Logic}: Enforces complex business rules
\end{itemize}

\end{solutionbox}
\begin{mnemonicbox}
``TBA-FEN - Triggers Before After For Each New''

\end{mnemonicbox}
\subsection*{Question 5(a OR) [3
marks]}\label{question-5a-or-3-marks}

\textbf{Explain problems of concurrency control in transaction.}

\begin{solutionbox}


{\def\LTcaptype{none} % do not increment counter
\begin{longtable}[]{@{}
  >{\raggedright\arraybackslash}p{(\linewidth - 4\tabcolsep) * \real{0.2903}}
  >{\raggedright\arraybackslash}p{(\linewidth - 4\tabcolsep) * \real{0.4194}}
  >{\raggedright\arraybackslash}p{(\linewidth - 4\tabcolsep) * \real{0.2903}}@{}}
\toprule\noalign{}
\begin{minipage}[b]{\linewidth}\raggedright
Problem
\end{minipage} & \begin{minipage}[b]{\linewidth}\raggedright
Description
\end{minipage} & \begin{minipage}[b]{\linewidth}\raggedright
Example
\end{minipage} \\
\midrule\noalign{}
\endhead
\bottomrule\noalign{}
\endlastfoot
\textbf{Lost Update} & One transaction overwrites another's changes &
T1, T2 update same record \\
\textbf{Dirty Read} & Reading uncommitted data & T1 reads T2's
uncommitted changes \\
\textbf{Unrepeatable Read} & Same query returns different results & T1
reads, T2 updates, T1 reads again \\
\end{longtable}
}

\begin{itemize}
\tightlist
\item
  \textbf{Phantom Read}: New rows appear between queries in same
  transaction
\item
  \textbf{Deadlock}: Two transactions wait for each other's locks
\item
  \textbf{Inconsistent Analysis}: Reading data while it's being modified
\end{itemize}

\end{solutionbox}
\begin{mnemonicbox}
``LDU-PID - Lost Dirty Unrepeatable Phantom
Inconsistent Deadlock''

\end{mnemonicbox}
\subsection*{Question 5(b OR) [4
marks]}\label{question-5b-or-4-marks}

\textbf{Create following table with specification:} \textbf{STUDENT:
(stu\_id, stu\_name, Address, City, contact\_no, Branch\_name)}

\begin{solutionbox}

\begin{verbatim}
CREATE TABLE STUDENT (
    stu\_id VARCHAR(10) PRIMARY KEY CHECK (stu\_id LIKE {S\%}),
    stu\_name VARCHAR(50),
    Address VARCHAR(100),
    City VARCHAR(30),
    contact\_no NUMBER(10),
    Branch\_name VARCHAR(20)
);
\end{verbatim}


{\def\LTcaptype{none} % do not increment counter
\begin{longtable}[]{@{}lll@{}}
\toprule\noalign{}
Constraint & Implementation & Purpose \\
\midrule\noalign{}
\endhead
\bottomrule\noalign{}
\endlastfoot
\textbf{PRIMARY KEY} & stu\_id PRIMARY KEY & Unique identification \\
\textbf{CHECK} & stu\_id LIKE `S\%' & Must start with `S' \\
\end{longtable}
}

\begin{itemize}
\tightlist
\item
  \textbf{Primary Key}: stu\_id serves as unique identifier
\item
  \textbf{Pattern Check}: stu\_id must begin with letter `S'
\item
  \textbf{Data Types}: Appropriate field sizes and types
\item
  \textbf{Constraint Validation}: Database enforces rules automatically
\end{itemize}

\end{solutionbox}
\begin{mnemonicbox}
``PKC-ST - Primary Key Check Starts''

\end{mnemonicbox}
\subsection*{Question 5(c OR) [7
marks]}\label{question-5c-or-7-marks}

\textbf{What is Explicit cursor? Explain explicit cursor with simple
example.}

\begin{solutionbox}

\textbf{Explicit Cursor}: User-defined cursor for handling SELECT
statements that return multiple rows with programmatic control.

\textbf{Cursor Operations}:

\begin{verbatim}
{-{-} Declaration}
DECLARE
    CURSOR student\_cursor IS
        SELECT stu\_id, stu\_name FROM STUDENT WHERE city = {Ahmedabad};
    v\_id STUDENT.stu\_id\%TYPE;
    v\_name STUDENT.stu\_name\%TYPE;
BEGIN
    {-{-} Open cursor}
    OPEN student\_cursor;
    
    {-{-} Fetch data}
    LOOP
        FETCH student\_cursor INTO v\_id, v\_name;
        EXIT WHEN student\_cursor\%NOTFOUND;
        
        DBMS\_OUTPUT.PUT\_LINE({ID: } || v\_id || {, Name: } || v\_name);
    END LOOP;
    
    {-{-} Close cursor}
    CLOSE student\_cursor;
END;
\end{verbatim}


{\def\LTcaptype{none} % do not increment counter
\begin{longtable}[]{@{}lll@{}}
\toprule\noalign{}
Operation & Purpose & Syntax \\
\midrule\noalign{}
\endhead
\bottomrule\noalign{}
\endlastfoot
\textbf{DECLARE} & Define cursor & CURSOR name IS SELECT\ldots{} \\
\textbf{OPEN} & Initialize cursor & OPEN cursor\_name \\
\textbf{FETCH} & Retrieve data & FETCH cursor INTO variables \\
\textbf{CLOSE} & Release resources & CLOSE cursor\_name \\
\end{longtable}
}

\begin{center}
\textbf{Mermaid Diagram (Code)}
\begin{verbatim}
{Shaded}
{Highlighting}[]
graph LR
    A[DECLARE Cursor] {-{-}{} B[OPEN Cursor]}
    B {-{-}{} C[FETCH Data]}
    C {-{-}{} D\{More Rows?\}}
    D {-{-}{}|Yes| C}
    D {-{-}{}|No| E[CLOSE Cursor]}
{Highlighting}
{Shaded}
\end{verbatim}
\end{center}

\begin{itemize}
\tightlist
\item
  \textbf{Manual Control}: Programmer controls cursor operations
\item
  \textbf{Memory Management}: Must explicitly open and close
\item
  \textbf{Loop Processing}: Typically used with loops for multiple rows
\item
  \textbf{Cursor Attributes}: \%FOUND, \%NOTFOUND, \%ROWCOUNT
\end{itemize}

\end{solutionbox}
\begin{mnemonicbox}
``DOFC - Declare Open Fetch Close''

\end{mnemonicbox}

\end{document}
