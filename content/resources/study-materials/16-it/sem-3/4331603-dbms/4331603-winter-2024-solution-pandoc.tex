\documentclass[10pt,a4paper]{article}

% content/resources/templates/preamble.tex
\usepackage[margin=0.6in]{geometry}
\author{Milav Dabgar}
\usepackage{amsmath,amssymb,amsthm}
\usepackage{booktabs}
\usepackage{multirow}
\usepackage{xcolor}
\usepackage{tcolorbox}
\tcbuselibrary{breakable,skins}
\usepackage[colorlinks=true,linkcolor=blue]{hyperref}
\usepackage{titlesec}
\usepackage{enumitem}
\usepackage{tikz}
\usepackage{pgfplots}
\usepackage{circuitikz}
\usepackage[version=4]{mhchem}
\usepackage{longtable}
\usepackage{array}
\usepackage{float}
\usepackage{caption}
\usepackage{listings}

\lstset{
  basicstyle=\small\ttfamily,
  breaklines=true,
  breakatwhitespace=false,
  postbreak=\mbox{\textcolor{red}{$\hookrightarrow$}\space},
  float=false,
  numbers=left,
  numberstyle=\tiny\color{gray},
  numbersep=10pt,
  xleftmargin=2em,
  keywordstyle=\color{blue},
  commentstyle=\color{green!60!black},
  stringstyle=\color{purple},
  backgroundcolor=\color{gray!5},
  showstringspaces=false,
  tabsize=2,
  captionpos=b,
  keepspaces=true,
  columns=flexible
}

\pgfplotsset{compat=1.18}
\usetikzlibrary{shapes,arrows,positioning,calc,patterns,decorations.pathmorphing,decorations.markings,arrows.meta}

% Color scheme
\definecolor{headcolor}{RGB}{0,102,204}
\definecolor{keycolor}{RGB}{220,20,60}
\definecolor{solutioncolor}{RGB}{34,139,34}
\definecolor{mnemoniccolor}{RGB}{148,0,211}
\definecolor{codecolor}{RGB}{0,0,100}

% Spacing
\setlength{\parskip}{3pt}
\setlist[itemize]{nosep}
\setlist[enumerate]{nosep}

% Title formatting
\titleformat{\section}{\Large\bfseries\color{headcolor}}{\thesection}{1em}{}
\titleformat{\subsection}{\large\bfseries\color{headcolor}}{\thesubsection}{1em}{}

% Pandoc tightlist compatibility
\providecommand{\tightlist}{%
  \setlength{\itemsep}{0pt}\setlength{\parskip}{0pt}}

% Pandoc longtable compatibility
\newcounter{none}
\def\thenone{}


% content/resources/templates/english-boxes.tex
% This file is currently empty - it exists to maintain consistency with the import structure.
% Add custom environments here if needed in the future.


\begin{document}

\begin{center}
{\Huge\bfseries\color{headcolor} Subject Name Solutions}\\[5pt]
{\LARGE 4331603 -- Winter 2024}\\[3pt]
{\large Semester 1 Study Material}\\[3pt]
{\normalsize\textit{Detailed Solutions and Explanations}}
\end{center}

\vspace{10pt}

\subsection*{Question 1(a) [3 marks]}\label{q1a}

\textbf{Explain three-level database architecture.}

\begin{solutionbox}


{\def\LTcaptype{none} % do not increment counter
\begin{longtable}[]{@{}
  >{\raggedright\arraybackslash}p{(\linewidth - 4\tabcolsep) * \real{0.2414}}
  >{\raggedright\arraybackslash}p{(\linewidth - 4\tabcolsep) * \real{0.4483}}
  >{\raggedright\arraybackslash}p{(\linewidth - 4\tabcolsep) * \real{0.3103}}@{}}
\toprule\noalign{}
\begin{minipage}[b]{\linewidth}\raggedright
Level
\end{minipage} & \begin{minipage}[b]{\linewidth}\raggedright
Description
\end{minipage} & \begin{minipage}[b]{\linewidth}\raggedright
Purpose
\end{minipage} \\
\midrule\noalign{}
\endhead
\bottomrule\noalign{}
\endlastfoot
\textbf{External Level} & User views and application programs & Data
abstraction for users \\
\textbf{Conceptual Level} & Complete logical structure &
Organization-wide data view \\
\textbf{Internal Level} & Physical storage details & Storage and access
methods \\
\end{longtable}
}

\textbf{Diagram:}

\begin{center}
\textbf{Mermaid Diagram (Code)}
\begin{verbatim}
{Shaded}
{Highlighting}[]
graph LR
    A[External Level] {-{-}{} B[Conceptual Level]}
    B {-{-}{} C[Internal Level]}
    A1[User View 1] {-{-}{} A}
    A2[User View 2] {-{-}{} A}
    A3[User View n] {-{-}{} A}
    C {-{-}{} D[Physical Storage]}
{Highlighting}
{Shaded}
\end{verbatim}
\end{center}

\begin{itemize}
\tightlist
\item
  \textbf{External Level}: Individual user views and specific
  application requirements
\item
  \textbf{Conceptual Level}: Complete database schema without storage
  details\\
\item
  \textbf{Internal Level}: Physical storage structures and access paths
\end{itemize}

\end{solutionbox}
\begin{mnemonicbox}
``ECI - Every Computer Interface''

\end{mnemonicbox}
\subsection*{Question 1(b) [4 marks]}\label{q1b}

\textbf{Explain Total Participation and Partial Participation with
example.}

\begin{solutionbox}


{\def\LTcaptype{none} % do not increment counter
\begin{longtable}[]{@{}
  >{\raggedright\arraybackslash}p{(\linewidth - 6\tabcolsep) * \real{0.3878}}
  >{\raggedright\arraybackslash}p{(\linewidth - 6\tabcolsep) * \real{0.2449}}
  >{\raggedright\arraybackslash}p{(\linewidth - 6\tabcolsep) * \real{0.1837}}
  >{\raggedright\arraybackslash}p{(\linewidth - 6\tabcolsep) * \real{0.1837}}@{}}
\toprule\noalign{}
\begin{minipage}[b]{\linewidth}\raggedright
Participation Type
\end{minipage} & \begin{minipage}[b]{\linewidth}\raggedright
Definition
\end{minipage} & \begin{minipage}[b]{\linewidth}\raggedright
Symbol
\end{minipage} & \begin{minipage}[b]{\linewidth}\raggedright
Example
\end{minipage} \\
\midrule\noalign{}
\endhead
\bottomrule\noalign{}
\endlastfoot
\textbf{Total Participation} & Every entity must participate & Double
line & Student-Course enrollment \\
\textbf{Partial Participation} & Some entities may not participate &
Single line & Employee-Department management \\
\end{longtable}
}

\textbf{Diagram:}

\begin{verbatim}
erDiagram
    STUDENT ||{-{-}|| ENROLLMENT : "Total (must enroll)"}
    EMPLOYEE \|{-}{-}|| DEPARTMENT : "Partial (may not manage)"}
\end{verbatim}

\begin{itemize}
\tightlist
\item
  \textbf{Total Participation}: All students must be enrolled in at
  least one course
\item
  \textbf{Partial Participation}: Not all employees manage a department
\item
  \textbf{Double lines} indicate total participation constraints
\item
  \textbf{Single lines} show partial participation relationships
\end{itemize}

\end{solutionbox}
\begin{mnemonicbox}
``Total = Two lines, Partial = Plain line''

\end{mnemonicbox}
\subsection*{Question 1(c) [7 marks]}\label{q1c}

\textbf{Explain advantages of DBMS over file management systems.}

\begin{solutionbox}


{\def\LTcaptype{none} % do not increment counter
\begin{longtable}[]{@{}lll@{}}
\toprule\noalign{}
Advantage & File System & DBMS \\
\midrule\noalign{}
\endhead
\bottomrule\noalign{}
\endlastfoot
\textbf{Data Redundancy} & High duplication & Controlled redundancy \\
\textbf{Data Inconsistency} & Common problem & Data integrity
maintained \\
\textbf{Data Sharing} & Limited sharing & Concurrent access support \\
\textbf{Security} & File-level security & User-level access control \\
\textbf{Backup \& Recovery} & Manual process & Automatic mechanisms \\
\end{longtable}
}

\begin{itemize}
\tightlist
\item
  \textbf{Reduced Data Redundancy}: Eliminates duplicate data storage
  across applications
\item
  \textbf{Data Consistency}: Ensures uniform data across all
  applications
\item
  \textbf{Data Independence}: Applications independent of data structure
  changes
\item
  \textbf{Concurrent Access}: Multiple users can access data
  simultaneously
\item
  \textbf{Security Control}: User authentication and authorization
  mechanisms
\item
  \textbf{Backup and Recovery}: Automatic data protection and
  restoration
\item
  \textbf{Data Integrity}: Constraint enforcement maintains data quality
\end{itemize}

\end{solutionbox}
\begin{mnemonicbox}
``RDCCSBI - Really Don't Copy, Control, Secure,
Backup, Integrate''

\end{mnemonicbox}
\subsection*{Question 1(c OR) [7
marks]}\label{question-1c-or-7-marks}

\textbf{List out various data models. Explain any two in brief.}

\begin{solutionbox}

\textbf{Data Models List:}

\begin{itemize}
\tightlist
\item
  Hierarchical Data Model
\item
  Network Data Model\\
\item
  Relational Data Model
\item
  Object-Oriented Data Model
\item
  Entity-Relationship Model
\end{itemize}


{\def\LTcaptype{none} % do not increment counter
\begin{longtable}[]{@{}
  >{\raggedright\arraybackslash}p{(\linewidth - 6\tabcolsep) * \real{0.1556}}
  >{\raggedright\arraybackslash}p{(\linewidth - 6\tabcolsep) * \real{0.2444}}
  >{\raggedright\arraybackslash}p{(\linewidth - 6\tabcolsep) * \real{0.2667}}
  >{\raggedright\arraybackslash}p{(\linewidth - 6\tabcolsep) * \real{0.3333}}@{}}
\toprule\noalign{}
\begin{minipage}[b]{\linewidth}\raggedright
Model
\end{minipage} & \begin{minipage}[b]{\linewidth}\raggedright
Structure
\end{minipage} & \begin{minipage}[b]{\linewidth}\raggedright
Advantages
\end{minipage} & \begin{minipage}[b]{\linewidth}\raggedright
Disadvantages
\end{minipage} \\
\midrule\noalign{}
\endhead
\bottomrule\noalign{}
\endlastfoot
\textbf{Relational Model} & Tables with rows/columns & Simple, flexible
& Performance overhead \\
\textbf{Network Model} & Graph with records/links & Efficient navigation
& Complex structure \\
\end{longtable}
}

\textbf{Relational Data Model:}

\begin{itemize}
\tightlist
\item
  \textbf{Structure}: Data organized in tables (relations)
\item
  \textbf{Components}: Tuples (rows), attributes (columns), domains
\item
  \textbf{Operations}: Select, project, join operations available
\end{itemize}

\textbf{Network Data Model:}

\begin{itemize}
\tightlist
\item
  \textbf{Structure}: Graph-based with owner-member relationships
\item
  \textbf{Navigation}: Explicit links between record types
\item
  \textbf{Flexibility}: Many-to-many relationships supported naturally
\end{itemize}

\end{solutionbox}
\begin{mnemonicbox}
``HNROE - Have Network Relational Object Entity''

\end{mnemonicbox}
\subsection*{Question 2(a) [3 marks]}\label{q2a}

\textbf{Explain Mapping Cardinalities.}

\begin{solutionbox}


{\def\LTcaptype{none} % do not increment counter
\begin{longtable}[]{@{}
  >{\raggedright\arraybackslash}p{(\linewidth - 6\tabcolsep) * \real{0.2955}}
  >{\raggedright\arraybackslash}p{(\linewidth - 6\tabcolsep) * \real{0.2045}}
  >{\raggedright\arraybackslash}p{(\linewidth - 6\tabcolsep) * \real{0.2955}}
  >{\raggedright\arraybackslash}p{(\linewidth - 6\tabcolsep) * \real{0.2045}}@{}}
\toprule\noalign{}
\begin{minipage}[b]{\linewidth}\raggedright
Cardinality
\end{minipage} & \begin{minipage}[b]{\linewidth}\raggedright
Symbol
\end{minipage} & \begin{minipage}[b]{\linewidth}\raggedright
Description
\end{minipage} & \begin{minipage}[b]{\linewidth}\raggedright
Example
\end{minipage} \\
\midrule\noalign{}
\endhead
\bottomrule\noalign{}
\endlastfoot
\textbf{One-to-One} & 1:1 & Each entity relates to one other &
Person-Passport \\
\textbf{One-to-Many} & 1:M & One entity relates to many &
Department-Employee \\
\textbf{Many-to-One} & M:1 & Many entities relate to one &
Student-Course \\
\textbf{Many-to-Many} & M:N & Many relate to many & Student-Subject \\
\end{longtable}
}

\textbf{Diagram:}

\begin{verbatim}
erDiagram
    PERSON ||{-{-}|| PASSPORT : "1:1"}
    DEPARTMENT ||{-{-}o\{ EMPLOYEE : "1:M"}
    STUDENT {-}{-}|| COURSE : "M:1"}
    STUDENT \|{-}{-}|\{ SUBJECT : "M:N"}
\end{verbatim}

\begin{itemize}
\tightlist
\item
  \textbf{Cardinality constraints} define relationship participation
  limits
\item
  \textbf{Maximum cardinality} specifies upper bound of associations
\item
  \textbf{Helps in database design} and relationship modeling
\end{itemize}

\end{solutionbox}
\begin{mnemonicbox}
``OMOM - One, One-Many, One-Many, Many-Many''

\end{mnemonicbox}
\subsection*{Question 2(b) [4 marks]}\label{q2b}

\textbf{Explain Outer Join operation in Relational Algebra.}

\begin{solutionbox}


{\def\LTcaptype{none} % do not increment counter
\begin{longtable}[]{@{}
  >{\raggedright\arraybackslash}p{(\linewidth - 6\tabcolsep) * \real{0.2500}}
  >{\raggedright\arraybackslash}p{(\linewidth - 6\tabcolsep) * \real{0.2045}}
  >{\raggedright\arraybackslash}p{(\linewidth - 6\tabcolsep) * \real{0.2045}}
  >{\raggedright\arraybackslash}p{(\linewidth - 6\tabcolsep) * \real{0.3409}}@{}}
\toprule\noalign{}
\begin{minipage}[b]{\linewidth}\raggedright
Join Type
\end{minipage} & \begin{minipage}[b]{\linewidth}\raggedright
Symbol
\end{minipage} & \begin{minipage}[b]{\linewidth}\raggedright
Result
\end{minipage} & \begin{minipage}[b]{\linewidth}\raggedright
NULL Handling
\end{minipage} \\
\midrule\noalign{}
\endhead
\bottomrule\noalign{}
\endlastfoot
\textbf{Left Outer Join} & ⟕ & All left + matching right & NULLs for
unmatched right \\
\textbf{Right Outer Join} & ⟖ & All right + matching left & NULLs for
unmatched left \\
\textbf{Full Outer Join} & ⟗ & All from both tables & NULLs for
unmatched \\
\end{longtable}
}

\textbf{Example:}

\begin{verbatim}
EMPLOYEE ⟕ DEPARTMENT
- Includes all employees
- NULL values for employees without departments
\end{verbatim}

\begin{itemize}
\tightlist
\item
  \textbf{Preserves unmatched tuples} from specified relation(s)
\item
  \textbf{NULL values} fill missing attribute values
\item
  \textbf{Three types}: Left, Right, and Full outer joins
\item
  \textbf{Useful for reporting} incomplete data relationships
\end{itemize}

\end{solutionbox}
\begin{mnemonicbox}
``LRF - Left Right Full outer joins''

\end{mnemonicbox}
\subsection*{Question 2(c) [7 marks]}\label{q2c}

\textbf{Explain concept of Specialization and Generalization with
example.}

\begin{solutionbox}


{\def\LTcaptype{none} % do not increment counter
\begin{longtable}[]{@{}
  >{\raggedright\arraybackslash}p{(\linewidth - 6\tabcolsep) * \real{0.2368}}
  >{\raggedright\arraybackslash}p{(\linewidth - 6\tabcolsep) * \real{0.2895}}
  >{\raggedright\arraybackslash}p{(\linewidth - 6\tabcolsep) * \real{0.2368}}
  >{\raggedright\arraybackslash}p{(\linewidth - 6\tabcolsep) * \real{0.2368}}@{}}
\toprule\noalign{}
\begin{minipage}[b]{\linewidth}\raggedright
Concept
\end{minipage} & \begin{minipage}[b]{\linewidth}\raggedright
Direction
\end{minipage} & \begin{minipage}[b]{\linewidth}\raggedright
Process
\end{minipage} & \begin{minipage}[b]{\linewidth}\raggedright
Example
\end{minipage} \\
\midrule\noalign{}
\endhead
\bottomrule\noalign{}
\endlastfoot
\textbf{Specialization} & Top-Down & General to Specific & Vehicle \rightarrow
Car, Truck \\
\textbf{Generalization} & Bottom-Up & Specific to General & Car, Truck \rightarrow
Vehicle \\
\end{longtable}
}

\textbf{Diagram:}

\begin{verbatim}
erDiagram
    VEHICLE \{
        int vehicle\_id
        string make
        string model
    \}
    CAR \{
        int doors
        string fuel\_type
    \}
    TRUCK \{
        int payload
        string truck\_type
    \}
    
    VEHICLE ||{-{-}|| CAR : "ISA"}
    VEHICLE ||{-{-}|| TRUCK : "ISA"}
\end{verbatim}

\textbf{Specialization:}

\begin{itemize}
\tightlist
\item
  \textbf{Process}: Creating subclasses from superclass
\item
  \textbf{Inheritance}: Subclasses inherit all superclass attributes
\item
  \textbf{Additional attributes}: Subclasses have specific properties
\end{itemize}

\textbf{Generalization:}

\begin{itemize}
\tightlist
\item
  \textbf{Process}: Creating superclass from common subclass features
\item
  \textbf{Abstraction}: Identifies common attributes and relationships
\item
  \textbf{Simplification}: Reduces complexity through hierarchy
\end{itemize}

\end{solutionbox}
\begin{mnemonicbox}
``SG-TD-BU - Specialization General-To-Detail,
Bottom-Up''

\end{mnemonicbox}
\subsection*{Question 2(a OR) [3
marks]}\label{question-2a-or-3-marks}

\textbf{Explain different types of Keys in Relational Algebra.}

\begin{solutionbox}


{\def\LTcaptype{none} % do not increment counter
\begin{longtable}[]{@{}
  >{\raggedright\arraybackslash}p{(\linewidth - 6\tabcolsep) * \real{0.2326}}
  >{\raggedright\arraybackslash}p{(\linewidth - 6\tabcolsep) * \real{0.2791}}
  >{\raggedright\arraybackslash}p{(\linewidth - 6\tabcolsep) * \real{0.2791}}
  >{\raggedright\arraybackslash}p{(\linewidth - 6\tabcolsep) * \real{0.2093}}@{}}
\toprule\noalign{}
\begin{minipage}[b]{\linewidth}\raggedright
Key Type
\end{minipage} & \begin{minipage}[b]{\linewidth}\raggedright
Definition
\end{minipage} & \begin{minipage}[b]{\linewidth}\raggedright
Uniqueness
\end{minipage} & \begin{minipage}[b]{\linewidth}\raggedright
Example
\end{minipage} \\
\midrule\noalign{}
\endhead
\bottomrule\noalign{}
\endlastfoot
\textbf{Super Key} & Any attribute set that uniquely identifies & Yes &
\{ID, Name, Phone\} \\
\textbf{Candidate Key} & Minimal super key & Yes & \{ID\}, \{Email\} \\
\textbf{Primary Key} & Chosen candidate key & Yes & \{StudentID\} \\
\textbf{Foreign Key} & References primary key & No & \{DeptID\}
references Dept \\
\end{longtable}
}

\begin{itemize}
\tightlist
\item
  \textbf{Super Key}: Uniquely identifies tuples, may have extra
  attributes
\item
  \textbf{Candidate Key}: Minimal super key without redundant attributes
\item
  \textbf{Primary Key}: Selected candidate key for entity identification
\item
  \textbf{Foreign Key}: Establishes referential integrity between tables
\end{itemize}

\end{solutionbox}
\begin{mnemonicbox}
``SCPF - Super Candidate Primary Foreign''

\end{mnemonicbox}
\subsection*{Question 2(b OR) [4
marks]}\label{question-2b-or-4-marks}

\textbf{Explain types of attributes in ER-diagram with suitable
example.}

\begin{solutionbox}


{\def\LTcaptype{none} % do not increment counter
\begin{longtable}[]{@{}
  >{\raggedright\arraybackslash}p{(\linewidth - 6\tabcolsep) * \real{0.3404}}
  >{\raggedright\arraybackslash}p{(\linewidth - 6\tabcolsep) * \real{0.1915}}
  >{\raggedright\arraybackslash}p{(\linewidth - 6\tabcolsep) * \real{0.2766}}
  >{\raggedright\arraybackslash}p{(\linewidth - 6\tabcolsep) * \real{0.1915}}@{}}
\toprule\noalign{}
\begin{minipage}[b]{\linewidth}\raggedright
Attribute Type
\end{minipage} & \begin{minipage}[b]{\linewidth}\raggedright
Symbol
\end{minipage} & \begin{minipage}[b]{\linewidth}\raggedright
Description
\end{minipage} & \begin{minipage}[b]{\linewidth}\raggedright
Example
\end{minipage} \\
\midrule\noalign{}
\endhead
\bottomrule\noalign{}
\endlastfoot
\textbf{Simple} & Oval & Cannot be subdivided & Age, Name \\
\textbf{Composite} & Oval with sub-ovals & Can be subdivided & Address
(Street, City) \\
\textbf{Derived} & Dashed oval & Calculated from others & Age from
Birth\_Date \\
\textbf{Multi-valued} & Double oval & Multiple values &
Phone\_Numbers \\
\end{longtable}
}

\textbf{Diagram:}

\begin{verbatim}
    +{-{-}{-}{-}{-}{-}{-}{-}{-}{-}+}
    |   Name   |  {{-}{-} Simple}
    +{-{-}{-}{-}{-}{-}{-}{-}{-}{-}+}
    
    +{-{-}{-}{-}{-}{-}{-}{-}{-}{-}+}
    |  Address |  {{-}{-} Composite}
    +{-{-}{-}{-}+{-}{-}{-}{-}{-}+}
         |
    +{-{-}{-}{-}+{-}{-}{-}+{-}{-}{-}{-}{-}{-}+}
    | Street | City |
    +{-{-}{-}{-}{-}{-}{-}{-}+{-}{-}{-}{-}{-}{-}+}
    
    +{-{-}{-}{-}{-}{-}{-}{-}{-}{-}+}
    :|Phone\_No|:  {{-}{-} Multi{-}valued}
    +{-{-}{-}{-}{-}{-}{-}{-}{-}{-}+}
    
    +{-{-}{-}{-}{-}{-}{-}{-}{-}{-}+}
    :   Age    :  {{-}{-} Derived}
    +{-{-}{-}{-}{-}{-}{-}{-}{-}{-}+}
\end{verbatim}

\begin{itemize}
\tightlist
\item
  \textbf{Simple attributes} are atomic and indivisible
\item
  \textbf{Composite attributes} have meaningful sub-parts
\item
  \textbf{Derived attributes} computed from other attribute values\\
\item
  \textbf{Multi-valued attributes} store multiple values per entity
\end{itemize}

\end{solutionbox}
\begin{mnemonicbox}
``SCDM - Simple Composite Derived Multi-valued''

\end{mnemonicbox}
\subsection*{Question 2(c OR) [7
marks]}\label{question-2c-or-7-marks}

\textbf{Explain SELECT, PROJECT, UNION and SET-INTERSECTION operation
with suitable example.}

\begin{solutionbox}


{\def\LTcaptype{none} % do not increment counter
\begin{longtable}[]{@{}llll@{}}
\toprule\noalign{}
Operation & Symbol & Purpose & Example \\
\midrule\noalign{}
\endhead
\bottomrule\noalign{}
\endlastfoot
\textbf{SELECT} & σ & Filter rows & σ(salary \textgreater{}
50000)(Employee) \\
\textbf{PROJECT} & π & Select columns & π(name, age)(Employee) \\
\textbf{UNION} & \cup & Combine relations & R \cup S \\
\textbf{INTERSECTION} & \cap & Common tuples & R \cap S \\
\end{longtable}
}

\textbf{Examples:}

\textbf{SELECT Operation:}

\begin{verbatim}
σ(age > 25)(STUDENT)
- Returns students older than 25 years
\end{verbatim}

\textbf{PROJECT Operation:}

\begin{verbatim}
π(name, course)(STUDENT)  
- Returns only name and course columns
\end{verbatim}

\textbf{UNION Operation:}

\begin{verbatim}
SCIENCE_STUDENTS \cup ARTS_STUDENTS
- Combines students from both streams
\end{verbatim}

\textbf{INTERSECTION Operation:}

\begin{verbatim}
MALE_STUDENTS \cap SPORTS_STUDENTS
- Returns male students who play sports
\end{verbatim}

\end{solutionbox}
\begin{mnemonicbox}
``SPUI - Select Project Union Intersection''

\end{mnemonicbox}
\subsection*{Question 3(a) [3 marks]}\label{q3a}

\textbf{Differentiate Primary Key and Foreign Key constraint.}

\begin{solutionbox}


{\def\LTcaptype{none} % do not increment counter
\begin{longtable}[]{@{}lll@{}}
\toprule\noalign{}
Aspect & Primary Key & Foreign Key \\
\midrule\noalign{}
\endhead
\bottomrule\noalign{}
\endlastfoot
\textbf{Purpose} & Unique identification & Referential integrity \\
\textbf{NULL Values} & Not allowed & Allowed \\
\textbf{Uniqueness} & Must be unique & Can be duplicate \\
\textbf{Number per table} & Only one & Multiple allowed \\
\end{longtable}
}

\begin{itemize}
\tightlist
\item
  \textbf{Primary Key}: Ensures entity integrity within table
\item
  \textbf{Foreign Key}: Maintains referential integrity between tables
\item
  \textbf{Uniqueness}: Primary keys unique, foreign keys can repeat
\item
  \textbf{NULL handling}: Primary keys never NULL, foreign keys may be
  NULL
\end{itemize}

\end{solutionbox}
\begin{mnemonicbox}
``PU-FN - Primary Unique, Foreign Nullable''

\end{mnemonicbox}
\subsection*{Question 3(b) [4 marks]}\label{q3b}

\textbf{Explain DUAL table and SYSDATE with example.}

\begin{solutionbox}


{\def\LTcaptype{none} % do not increment counter
\begin{longtable}[]{@{}
  >{\raggedright\arraybackslash}p{(\linewidth - 6\tabcolsep) * \real{0.3143}}
  >{\raggedright\arraybackslash}p{(\linewidth - 6\tabcolsep) * \real{0.1714}}
  >{\raggedright\arraybackslash}p{(\linewidth - 6\tabcolsep) * \real{0.2571}}
  >{\raggedright\arraybackslash}p{(\linewidth - 6\tabcolsep) * \real{0.2571}}@{}}
\toprule\noalign{}
\begin{minipage}[b]{\linewidth}\raggedright
Component
\end{minipage} & \begin{minipage}[b]{\linewidth}\raggedright
Type
\end{minipage} & \begin{minipage}[b]{\linewidth}\raggedright
Purpose
\end{minipage} & \begin{minipage}[b]{\linewidth}\raggedright
Example
\end{minipage} \\
\midrule\noalign{}
\endhead
\bottomrule\noalign{}
\endlastfoot
\textbf{DUAL} & Virtual table & Test expressions & SELECT 2+3 FROM
DUAL \\
\textbf{SYSDATE} & System function & Current date/time & SELECT SYSDATE
FROM DUAL \\
\end{longtable}
}

\textbf{DUAL Table:}

\begin{itemize}
\tightlist
\item
  \textbf{Virtual table} with one row and one column
\item
  \textbf{Used for testing} expressions and functions
\item
  \textbf{Oracle-specific} pseudo table
\end{itemize}

\textbf{SYSDATE Function:}

\begin{itemize}
\tightlist
\item
  \textbf{Returns current} system date and time
\item
  \textbf{Automatic update} with system clock
\item
  \textbf{Date/time operations} supported
\end{itemize}

\textbf{Examples:}

\begin{verbatim}
SELECT SYSDATE FROM DUAL;
SELECT SYSDATE + 30 FROM DUAL;  {-{-} 30 days later}
\end{verbatim}

\end{solutionbox}
\begin{mnemonicbox}
``DT-ST - DUAL Testing, SYSDATE Time''

\end{mnemonicbox}
\subsection*{Question 3(c) [7 marks]}\label{q3c}

\textbf{Write SQL queries to use various numeric functions:}

\begin{solutionbox}


{\def\LTcaptype{none} % do not increment counter
\begin{longtable}[]{@{}llll@{}}
\toprule\noalign{}
Function & Purpose & SQL Query & Result \\
\midrule\noalign{}
\endhead
\bottomrule\noalign{}
\endlastfoot
\textbf{TRUNC} & Integer value &
\texttt{SELECT\ TRUNC(125.25)\ FROM\ DUAL;} & 125 \\
\textbf{ABS} & Absolute value & \texttt{SELECT\ ABS(-15)\ FROM\ DUAL;} &
15 \\
\textbf{CEIL} & Ceiling value &
\texttt{SELECT\ CEIL(55.65)\ FROM\ DUAL;} & 56 \\
\textbf{FLOOR} & Floor value &
\texttt{SELECT\ FLOOR(100.2)\ FROM\ DUAL;} & 100 \\
\end{longtable}
}

\textbf{SQL Queries:}

\begin{verbatim}
{-{-} (a) Display integer value of 125.25}
SELECT TRUNC(125.25) FROM DUAL;

{-{-} (b) Display absolute value of({-}15)  }
SELECT ABS({-}15) FROM DUAL;

{-{-} (c) Display ceil value of 55.65}
SELECT CEIL(55.65) FROM DUAL;

{-{-} (d) Display floor value of 100.2}
SELECT FLOOR(100.2) FROM DUAL;

{-{-} (e) Display the square root of 16}
SELECT SQRT(16) FROM DUAL;

{-{-} (f) Show value of e^{3}}
SELECT EXP(3) FROM DUAL;

{-{-} (g) Display result of 12 raised to 6}
SELECT POWER(12, 6) FROM DUAL;

{-{-} (h) Display result of 24 mod 2}
SELECT MOD(24, 2) FROM DUAL;

{-{-} (i) Show output of sign({-}25), sign(25), sign(0)}
SELECT SIGN({-}25), SIGN(25), SIGN(0) FROM DUAL;
\end{verbatim}

\end{solutionbox}
\begin{mnemonicbox}
``TACFSEPM - TRUNC ABS CEIL FLOOR SQRT EXP POWER
MOD''

\end{mnemonicbox}
\subsection*{Question 3(a OR) [3
marks]}\label{question-3a-or-3-marks}

\textbf{Explain Unique and Check Constraint with suitable example.}

\begin{solutionbox}


{\def\LTcaptype{none} % do not increment counter
\begin{longtable}[]{@{}llll@{}}
\toprule\noalign{}
Constraint & Purpose & Duplicates & Example \\
\midrule\noalign{}
\endhead
\bottomrule\noalign{}
\endlastfoot
\textbf{UNIQUE} & Prevent duplicates & Not allowed & Email address \\
\textbf{CHECK} & Validate data & Value restrictions & Age \textgreater{}
0 \\
\end{longtable}
}

\textbf{Examples:}

\begin{verbatim}
{-{-} UNIQUE Constraint}
CREATE TABLE Student (
    email VARCHAR(50) UNIQUE,
    phone VARCHAR(15) UNIQUE
);

{-{-} CHECK Constraint  }
CREATE TABLE Employee (
    age NUMBER CHECK (age {=} 18),
    salary NUMBER CHECK (salary {} 0)
);
\end{verbatim}

\begin{itemize}
\tightlist
\item
  \textbf{UNIQUE constraint} ensures no duplicate values in column
\item
  \textbf{CHECK constraint} validates data against specified conditions
\item
  \textbf{Multiple constraints} can be applied to single column
\end{itemize}

\end{solutionbox}
\begin{mnemonicbox}
``UC-DV - Unique no Copy, Check Validates''

\end{mnemonicbox}
\subsection*{Question 3(b OR) [4
marks]}\label{question-3b-or-4-marks}

\textbf{Explain structure of PL/SQL block.}

\begin{solutionbox}


{\def\LTcaptype{none} % do not increment counter
\begin{longtable}[]{@{}
  >{\raggedright\arraybackslash}p{(\linewidth - 6\tabcolsep) * \real{0.2432}}
  >{\raggedright\arraybackslash}p{(\linewidth - 6\tabcolsep) * \real{0.2703}}
  >{\raggedright\arraybackslash}p{(\linewidth - 6\tabcolsep) * \real{0.2432}}
  >{\raggedright\arraybackslash}p{(\linewidth - 6\tabcolsep) * \real{0.2432}}@{}}
\toprule\noalign{}
\begin{minipage}[b]{\linewidth}\raggedright
Section
\end{minipage} & \begin{minipage}[b]{\linewidth}\raggedright
Required
\end{minipage} & \begin{minipage}[b]{\linewidth}\raggedright
Purpose
\end{minipage} & \begin{minipage}[b]{\linewidth}\raggedright
Example
\end{minipage} \\
\midrule\noalign{}
\endhead
\bottomrule\noalign{}
\endlastfoot
\textbf{DECLARE} & Optional & Variable declarations & var\_name
VARCHAR2(20); \\
\textbf{BEGIN} & Mandatory & Executable statements & SELECT \ldots{}
INTO var; \\
\textbf{EXCEPTION} & Optional & Error handling & WHEN OTHERS THEN
\ldots{} \\
\textbf{END} & Mandatory & Block termination & END; \\
\end{longtable}
}

\textbf{Diagram:}

\begin{verbatim}
DECLARE
    -- Variable declarations
BEGIN  
    -- Executable statements
EXCEPTION
    -- Error handling
END;
\end{verbatim}

\begin{itemize}
\tightlist
\item
  \textbf{DECLARE section}: Variable and cursor declarations
\item
  \textbf{BEGIN-END}: Mandatory executable section
\item
  \textbf{EXCEPTION section}: Error handling routines
\item
  \textbf{Nested blocks}: PL/SQL blocks can be nested
\end{itemize}

\end{solutionbox}
\begin{mnemonicbox}
``DBE-E - Declare Begin Exception End''

\end{mnemonicbox}
\subsection*{Question 3(c OR) [7
marks]}\label{question-3c-or-7-marks}

\textbf{Consider the following table and solve queries:}

\begin{solutionbox}

\textbf{I) Create the BRANCH table:}

\begin{verbatim}
CREATE TABLE BRANCH (
    branchid VARCHAR2(10) PRIMARY KEY,
    branchname VARCHAR2(50) NOT NULL,
    address VARCHAR2(100)
);
\end{verbatim}

\textbf{II) Create the EMPLOYEE table:}

\begin{verbatim}
CREATE TABLE EMPLOYEE (
    empid VARCHAR2(10) PRIMARY KEY,
    name VARCHAR2(50) NOT NULL,
    post VARCHAR2(30),
    gender CHAR(1) CHECK (gender IN ({M}, {F})),
    birthdate DATE,
    salary NUMBER(10,2),
    branchid VARCHAR2(10),
    FOREIGN KEY (branchid) REFERENCES BRANCH(branchid)
);
\end{verbatim}

\textbf{III) Find employees in Ahmedabad branch:}

\begin{verbatim}
SELECT e.* FROM EMPLOYEE e, BRANCH b 
WHERE e.branchid = b.branchid 
AND b.branchname = {Ahmedabad};
\end{verbatim}

\textbf{IV) Find employees born in 1998:}

\begin{verbatim}
SELECT * FROM EMPLOYEE 
WHERE EXTRACT(YEAR FROM birthdate) = 1998;
\end{verbatim}

\textbf{V) Find female employees with salary \textgreater{} 5000:}

\begin{verbatim}
SELECT * FROM EMPLOYEE 
WHERE gender = {F} AND salary {} 5000;
\end{verbatim}

\textbf{VI) Find address where Ajay works:}

\begin{verbatim}
SELECT b.address FROM EMPLOYEE e, BRANCH b
WHERE e.branchid = b.branchid 
AND e.name = {Ajay};
\end{verbatim}

\end{solutionbox}
\begin{mnemonicbox}
``CBEFFA - Create Branch Employee Find Female
Address''

\end{mnemonicbox}
\subsection*{Question 4(a) [3 marks]}\label{q4a}

\textbf{Explain Referential Integrity with suitable example.}

\begin{solutionbox}


{\def\LTcaptype{none} % do not increment counter
\begin{longtable}[]{@{}
  >{\raggedright\arraybackslash}p{(\linewidth - 4\tabcolsep) * \real{0.2667}}
  >{\raggedright\arraybackslash}p{(\linewidth - 4\tabcolsep) * \real{0.4333}}
  >{\raggedright\arraybackslash}p{(\linewidth - 4\tabcolsep) * \real{0.3000}}@{}}
\toprule\noalign{}
\begin{minipage}[b]{\linewidth}\raggedright
Aspect
\end{minipage} & \begin{minipage}[b]{\linewidth}\raggedright
Description
\end{minipage} & \begin{minipage}[b]{\linewidth}\raggedright
Example
\end{minipage} \\
\midrule\noalign{}
\endhead
\bottomrule\noalign{}
\endlastfoot
\textbf{Definition} & Foreign key must reference existing primary key &
Employee.deptid \rightarrow Department.deptid \\
\textbf{Purpose} & Maintain data consistency & Prevent orphan records \\
\textbf{Actions} & CASCADE, SET NULL, RESTRICT & ON DELETE CASCADE \\
\end{longtable}
}

\textbf{Diagram:}

\begin{verbatim}
erDiagram
    DEPARTMENT \{
        int deptid PK
        string deptname
    \}
    EMPLOYEE \{
        int empid PK
        string name
        int deptid FK
    \}
    DEPARTMENT ||{-{-}o\{ EMPLOYEE : "references"}
\end{verbatim}

\begin{itemize}
\tightlist
\item
  \textbf{Referential integrity} ensures foreign key values exist in
  referenced table
\item
  \textbf{Orphan records} prevented by constraint enforcement
\item
  \textbf{Cascade operations} maintain consistency during
  updates/deletes
\end{itemize}

\end{solutionbox}
\begin{mnemonicbox}
``RIO - Referential Integrity prevents Orphans''

\end{mnemonicbox}
\subsection*{Question 4(b) [4 marks]}\label{q4b}

\textbf{Differentiate Partial and Full Functional Dependency.}

\begin{solutionbox}


{\def\LTcaptype{none} % do not increment counter
\begin{longtable}[]{@{}
  >{\raggedright\arraybackslash}p{(\linewidth - 6\tabcolsep) * \real{0.3333}}
  >{\raggedright\arraybackslash}p{(\linewidth - 6\tabcolsep) * \real{0.2353}}
  >{\raggedright\arraybackslash}p{(\linewidth - 6\tabcolsep) * \real{0.1765}}
  >{\raggedright\arraybackslash}p{(\linewidth - 6\tabcolsep) * \real{0.2549}}@{}}
\toprule\noalign{}
\begin{minipage}[b]{\linewidth}\raggedright
Dependency Type
\end{minipage} & \begin{minipage}[b]{\linewidth}\raggedright
Definition
\end{minipage} & \begin{minipage}[b]{\linewidth}\raggedright
Example
\end{minipage} & \begin{minipage}[b]{\linewidth}\raggedright
Requirement
\end{minipage} \\
\midrule\noalign{}
\endhead
\bottomrule\noalign{}
\endlastfoot
\textbf{Partial} & Depends on part of composite key & (StudentID,
CourseID) \rightarrow StudentName & Composite primary key \\
\textbf{Full} & Depends on entire key & (StudentID, CourseID) \rightarrow Grade &
Complete key needed \\
\end{longtable}
}

\textbf{Examples:}

\textbf{Partial Functional Dependency:}

\begin{verbatim}
(StudentID, CourseID) \rightarrow StudentName
StudentName depends only on StudentID, not CourseID
\end{verbatim}

\textbf{Full Functional Dependency:}

\begin{verbatim}
(StudentID, CourseID) \rightarrow Grade  
Grade depends on both StudentID and CourseID
\end{verbatim}

\begin{itemize}
\tightlist
\item
  \textbf{Partial dependency} causes data redundancy and anomalies
\item
  \textbf{Full dependency} required for proper normalization
\item
  \textbf{2NF eliminates} partial functional dependencies
\end{itemize}

\end{solutionbox}
\begin{mnemonicbox}
``PF-CF - Partial Few, Complete Full''

\end{mnemonicbox}
\subsection*{Question 4(c) [7 marks]}\label{q4c}

\textbf{Explain 3rd Normal Form with example.}

\begin{solutionbox}

\textbf{3rd Normal Form Requirements:}

\begin{enumerate}
\tightlist
\item
  Must be in 2NF
\item
  No transitive dependencies
\item
  Non-key attributes depend only on primary key
\end{enumerate}

\textbf{Table Before 3NF:}

{\def\LTcaptype{none} % do not increment counter
\begin{longtable}[]{@{}
  >{\raggedright\arraybackslash}p{(\linewidth - 10\tabcolsep) * \real{0.1447}}
  >{\raggedright\arraybackslash}p{(\linewidth - 10\tabcolsep) * \real{0.1711}}
  >{\raggedright\arraybackslash}p{(\linewidth - 10\tabcolsep) * \real{0.1316}}
  >{\raggedright\arraybackslash}p{(\linewidth - 10\tabcolsep) * \real{0.1579}}
  >{\raggedright\arraybackslash}p{(\linewidth - 10\tabcolsep) * \real{0.1842}}
  >{\raggedright\arraybackslash}p{(\linewidth - 10\tabcolsep) * \real{0.2105}}@{}}
\toprule\noalign{}
\begin{minipage}[b]{\linewidth}\raggedright
StudentID
\end{minipage} & \begin{minipage}[b]{\linewidth}\raggedright
StudentName
\end{minipage} & \begin{minipage}[b]{\linewidth}\raggedright
CourseID
\end{minipage} & \begin{minipage}[b]{\linewidth}\raggedright
CourseName
\end{minipage} & \begin{minipage}[b]{\linewidth}\raggedright
InstructorID
\end{minipage} & \begin{minipage}[b]{\linewidth}\raggedright
InstructorName
\end{minipage} \\
\midrule\noalign{}
\endhead
\bottomrule\noalign{}
\endlastfoot
S1 & John & C1 & Math & I1 & Dr.~Smith \\
S2 & Jane & C1 & Math & I1 & Dr.~Smith \\
\end{longtable}
}

\textbf{Problems:}

\begin{itemize}
\tightlist
\item
  \textbf{Transitive dependency}: StudentID \rightarrow CourseID \rightarrow InstructorName
\item
  \textbf{Update anomaly}: Instructor name change requires multiple
  updates
\item
  \textbf{Delete anomaly}: Removing student may lose instructor
  information
\end{itemize}

\textbf{3NF Solution:}

\textbf{STUDENT Table:}

{\def\LTcaptype{none} % do not increment counter
\begin{longtable}[]{@{}lll@{}}
\toprule\noalign{}
StudentID & StudentName & CourseID \\
\midrule\noalign{}
\endhead
\bottomrule\noalign{}
\endlastfoot
S1 & John & C1 \\
S2 & Jane & C1 \\
\end{longtable}
}

\textbf{COURSE Table:}

{\def\LTcaptype{none} % do not increment counter
\begin{longtable}[]{@{}lll@{}}
\toprule\noalign{}
CourseID & CourseName & InstructorID \\
\midrule\noalign{}
\endhead
\bottomrule\noalign{}
\endlastfoot
C1 & Math & I1 \\
\end{longtable}
}

\textbf{INSTRUCTOR Table:}

{\def\LTcaptype{none} % do not increment counter
\begin{longtable}[]{@{}ll@{}}
\toprule\noalign{}
InstructorID & InstructorName \\
\midrule\noalign{}
\endhead
\bottomrule\noalign{}
\endlastfoot
I1 & Dr.~Smith \\
\end{longtable}
}

\end{solutionbox}
\begin{mnemonicbox}
``3NF-NT - 3rd Normal Form No Transitives''

\end{mnemonicbox}
\subsection*{Question 4(a OR) [3
marks]}\label{question-4a-or-3-marks}

\textbf{Explain Importance of Normalization.}

\begin{solutionbox}


{\def\LTcaptype{none} % do not increment counter
\begin{longtable}[]{@{}
  >{\raggedright\arraybackslash}p{(\linewidth - 4\tabcolsep) * \real{0.2647}}
  >{\raggedright\arraybackslash}p{(\linewidth - 4\tabcolsep) * \real{0.4706}}
  >{\raggedright\arraybackslash}p{(\linewidth - 4\tabcolsep) * \real{0.2647}}@{}}
\toprule\noalign{}
\begin{minipage}[b]{\linewidth}\raggedright
Benefit
\end{minipage} & \begin{minipage}[b]{\linewidth}\raggedright
Problem Solved
\end{minipage} & \begin{minipage}[b]{\linewidth}\raggedright
Result
\end{minipage} \\
\midrule\noalign{}
\endhead
\bottomrule\noalign{}
\endlastfoot
\textbf{Reduce Redundancy} & Duplicate data & Storage efficiency \\
\textbf{Eliminate Anomalies} & Update/Insert/Delete issues & Data
consistency \\
\textbf{Improve Integrity} & Data inconsistency & Reliable
information \\
\end{longtable}
}

\begin{itemize}
\tightlist
\item
  \textbf{Data redundancy minimized} through proper table decomposition
\item
  \textbf{Update anomalies eliminated} by removing duplicate information
\item
  \textbf{Storage space optimized} through normalized structure
\item
  \textbf{Data integrity maintained} with referential constraints
\item
  \textbf{Maintenance simplified} with logical table organization
\end{itemize}

\end{solutionbox}
\begin{mnemonicbox}
``RESIM - Redundancy Eliminated, Storage Improved,
Maintenance''

\end{mnemonicbox}
\subsection*{Question 4(b OR) [4
marks]}\label{question-4b-or-4-marks}

\textbf{Differentiate Prime Attributes and Non-Prime Attributes.}

\begin{solutionbox}


{\def\LTcaptype{none} % do not increment counter
\begin{longtable}[]{@{}
  >{\raggedright\arraybackslash}p{(\linewidth - 6\tabcolsep) * \real{0.3721}}
  >{\raggedright\arraybackslash}p{(\linewidth - 6\tabcolsep) * \real{0.2791}}
  >{\raggedright\arraybackslash}p{(\linewidth - 6\tabcolsep) * \real{0.1395}}
  >{\raggedright\arraybackslash}p{(\linewidth - 6\tabcolsep) * \real{0.2093}}@{}}
\toprule\noalign{}
\begin{minipage}[b]{\linewidth}\raggedright
Attribute Type
\end{minipage} & \begin{minipage}[b]{\linewidth}\raggedright
Definition
\end{minipage} & \begin{minipage}[b]{\linewidth}\raggedright
Role
\end{minipage} & \begin{minipage}[b]{\linewidth}\raggedright
Example
\end{minipage} \\
\midrule\noalign{}
\endhead
\bottomrule\noalign{}
\endlastfoot
\textbf{Prime} & Part of candidate key & Key formation & StudentID,
CourseID \\
\textbf{Non-Prime} & Not part of any candidate key & Data storage &
StudentName, Grade \\
\end{longtable}
}

\textbf{Example:}

\begin{verbatim}
ENROLLMENT (StudentID, CourseID, Grade, Semester)
Candidate Key: (StudentID, CourseID)

Prime Attributes: StudentID, CourseID
Non-Prime Attributes: Grade, Semester
\end{verbatim}

\begin{itemize}
\tightlist
\item
  \textbf{Prime attributes} participate in candidate key formation
\item
  \textbf{Non-Prime attributes} provide additional entity information
\item
  \textbf{Functional dependencies} between these determine normal forms
\item
  \textbf{2NF requires} no partial dependencies of non-prime on prime
  attributes
\end{itemize}

\end{solutionbox}
\begin{mnemonicbox}
``PN-KD - Prime in Key, Non-prime for Data''

\end{mnemonicbox}
\subsection*{Question 4(c OR) [7
marks]}\label{question-4c-or-7-marks}

\textbf{Explain 2nd Normal Form with example.}

\begin{solutionbox}

\textbf{2nd Normal Form Requirements:}

\begin{enumerate}
\tightlist
\item
  Must be in 1NF
\item
  No partial functional dependencies
\item
  All non-key attributes fully depend on primary key
\end{enumerate}

\textbf{Table Before 2NF:}

{\def\LTcaptype{none} % do not increment counter
\begin{longtable}[]{@{}lllll@{}}
\toprule\noalign{}
StudentID & CourseID & StudentName & CourseName & Grade \\
\midrule\noalign{}
\endhead
\bottomrule\noalign{}
\endlastfoot
S1 & C1 & John & Math & A \\
S1 & C2 & John & Physics & B \\
S2 & C1 & Jane & Math & A \\
\end{longtable}
}

\textbf{Problems:}

\begin{itemize}
\tightlist
\item
  \textbf{Partial Dependencies}: StudentID \rightarrow StudentName, CourseID \rightarrow
  CourseName
\item
  \textbf{Update Anomaly}: Student name change requires multiple updates
\item
  \textbf{Insert Anomaly}: Cannot add course without student enrollment
\end{itemize}

\textbf{2NF Solution:}

\textbf{STUDENT Table:}

{\def\LTcaptype{none} % do not increment counter
\begin{longtable}[]{@{}ll@{}}
\toprule\noalign{}
StudentID & StudentName \\
\midrule\noalign{}
\endhead
\bottomrule\noalign{}
\endlastfoot
S1 & John \\
S2 & Jane \\
\end{longtable}
}

\textbf{COURSE Table:}

{\def\LTcaptype{none} % do not increment counter
\begin{longtable}[]{@{}ll@{}}
\toprule\noalign{}
CourseID & CourseName \\
\midrule\noalign{}
\endhead
\bottomrule\noalign{}
\endlastfoot
C1 & Math \\
C2 & Physics \\
\end{longtable}
}

\textbf{ENROLLMENT Table:}

{\def\LTcaptype{none} % do not increment counter
\begin{longtable}[]{@{}lll@{}}
\toprule\noalign{}
StudentID & CourseID & Grade \\
\midrule\noalign{}
\endhead
\bottomrule\noalign{}
\endlastfoot
S1 & C1 & A \\
S1 & C2 & B \\
S2 & C1 & A \\
\end{longtable}
}

\end{solutionbox}
\begin{mnemonicbox}
``2NF-FD - 2nd Normal Form Full Dependencies''

\end{mnemonicbox}
\subsection*{Question 5(a) [3 marks]}\label{q5a}

\textbf{Explain Transaction states with proper diagram.}

\begin{solutionbox}

\textbf{Diagram:}

\begin{verbatim}
stateDiagram{-v2}
  direction LR
    [*] {-{-} Active}
    Active {-{-} Partially\_Committed : commit}
    Active {-{-} Failed : abort/error}
    Partially\_Committed {-{-} Committed : write complete}
    Partially\_Committed {-{-} Failed : write failure}
    Failed {-{-} Aborted : rollback}
    Committed {-{-} [*]}
    Aborted {-{-} [*]}
\end{verbatim}


{\def\LTcaptype{none} % do not increment counter
\begin{longtable}[]{@{}lll@{}}
\toprule\noalign{}
State & Description & Next State \\
\midrule\noalign{}
\endhead
\bottomrule\noalign{}
\endlastfoot
\textbf{Active} & Transaction executing & Partially Committed/Failed \\
\textbf{Partially Committed} & Last statement executed &
Committed/Failed \\
\textbf{Committed} & Transaction successful & End \\
\textbf{Failed} & Cannot proceed normally & Aborted \\
\textbf{Aborted} & Transaction rolled back & End \\
\end{longtable}
}

\begin{itemize}
\tightlist
\item
  \textbf{Active state}: Transaction currently executing operations
\item
  \textbf{Partially committed}: All operations executed, waiting for
  commit
\item
  \textbf{Failed state}: Error occurred, transaction cannot continue
\end{itemize}

\end{solutionbox}
\begin{mnemonicbox}
``APCFA - Active Partial Commit Fail Abort''

\end{mnemonicbox}
\subsection*{Question 5(b) [4 marks]}\label{q5b}

\textbf{Explain any two DDL commands with a suitable example.}

\begin{solutionbox}


{\def\LTcaptype{none} % do not increment counter
\begin{longtable}[]{@{}
  >{\raggedright\arraybackslash}p{(\linewidth - 6\tabcolsep) * \real{0.2500}}
  >{\raggedright\arraybackslash}p{(\linewidth - 6\tabcolsep) * \real{0.2500}}
  >{\raggedright\arraybackslash}p{(\linewidth - 6\tabcolsep) * \real{0.2500}}
  >{\raggedright\arraybackslash}p{(\linewidth - 6\tabcolsep) * \real{0.2500}}@{}}
\toprule\noalign{}
\begin{minipage}[b]{\linewidth}\raggedright
Command
\end{minipage} & \begin{minipage}[b]{\linewidth}\raggedright
Purpose
\end{minipage} & \begin{minipage}[b]{\linewidth}\raggedright
Syntax
\end{minipage} & \begin{minipage}[b]{\linewidth}\raggedright
Example
\end{minipage} \\
\midrule\noalign{}
\endhead
\bottomrule\noalign{}
\endlastfoot
\textbf{CREATE} & Create database objects & CREATE TABLE & CREATE TABLE
Student(\ldots) \\
\textbf{ALTER} & Modify existing objects & ALTER TABLE & ALTER TABLE
Student ADD\ldots{} \\
\end{longtable}
}

\textbf{CREATE Command:}

\begin{verbatim}
CREATE TABLE EMPLOYEE (
    empid NUMBER(5) PRIMARY KEY,
    name VARCHAR2(50) NOT NULL,
    salary NUMBER(10,2),
    deptid NUMBER(3)
);
\end{verbatim}

\textbf{ALTER Command:}

\begin{verbatim}
{-{-} Add new column}
ALTER TABLE EMPLOYEE ADD phone VARCHAR2(15);

{-{-} Modify existing column}
ALTER TABLE EMPLOYEE MODIFY name VARCHAR2(100);

{-{-} Drop column}
ALTER TABLE EMPLOYEE DROP COLUMN phone;
\end{verbatim}

\begin{itemize}
\tightlist
\item
  \textbf{CREATE} establishes new database structures
\item
  \textbf{ALTER} modifies existing table definitions
\item
  \textbf{DDL commands} auto-commit changes
\item
  \textbf{Schema changes} affect data structure permanently
\end{itemize}

\end{solutionbox}
\begin{mnemonicbox}
``CA-NM - CREATE Adds, ALTER Modifies''

\end{mnemonicbox}
\subsection*{Question 5(c) [7 marks]}\label{q5c}

\textbf{Explain ACID Properties in detail.}

\begin{solutionbox}


{\def\LTcaptype{none} % do not increment counter
\begin{longtable}[]{@{}
  >{\raggedright\arraybackslash}p{(\linewidth - 6\tabcolsep) * \real{0.2500}}
  >{\raggedright\arraybackslash}p{(\linewidth - 6\tabcolsep) * \real{0.3000}}
  >{\raggedright\arraybackslash}p{(\linewidth - 6\tabcolsep) * \real{0.2250}}
  >{\raggedright\arraybackslash}p{(\linewidth - 6\tabcolsep) * \real{0.2250}}@{}}
\toprule\noalign{}
\begin{minipage}[b]{\linewidth}\raggedright
Property
\end{minipage} & \begin{minipage}[b]{\linewidth}\raggedright
Definition
\end{minipage} & \begin{minipage}[b]{\linewidth}\raggedright
Purpose
\end{minipage} & \begin{minipage}[b]{\linewidth}\raggedright
Example
\end{minipage} \\
\midrule\noalign{}
\endhead
\bottomrule\noalign{}
\endlastfoot
\textbf{Atomicity} & All or nothing execution & Transaction integrity &
Bank transfer \\
\textbf{Consistency} & Database remains valid & Data integrity & Balance
constraints \\
\textbf{Isolation} & Concurrent execution independence & Concurrency
control & Separate transactions \\
\textbf{Durability} & Committed changes permanent & Recovery guarantee &
Power failure survival \\
\end{longtable}
}

\textbf{Atomicity:}

\begin{itemize}
\tightlist
\item
  \textbf{All operations} in transaction execute completely or not at
  all
\item
  \textbf{Rollback mechanism} undoes partial changes on failure
\item
  \textbf{Example}: Bank transfer requires both debit and credit
  operations
\end{itemize}

\textbf{Consistency:}

\begin{itemize}
\tightlist
\item
  \textbf{Database state} remains valid before and after transaction
\item
  \textbf{Integrity constraints} maintained throughout execution
\item
  \textbf{Example}: Account balance never becomes negative
\end{itemize}

\textbf{Isolation:}

\begin{itemize}
\tightlist
\item
  \textbf{Concurrent transactions} do not interfere with each other
\item
  \textbf{Locking mechanisms} prevent interference
\item
  \textbf{Example}: Two users updating same account simultaneously
\end{itemize}

\textbf{Durability:}

\begin{itemize}
\tightlist
\item
  \textbf{Committed changes} survive system failures
\item
  \textbf{Write-ahead logging} ensures recovery capability
\item
  \textbf{Example}: Transaction survives power outage after commit
\end{itemize}

\end{solutionbox}
\begin{mnemonicbox}
``ACID - Atomicity Consistency Isolation Durability''

\end{mnemonicbox}
\subsection*{Question 5(a OR) [3
marks]}\label{question-5a-or-3-marks}

\textbf{What is two phase locking technique?}

\begin{solutionbox}


{\def\LTcaptype{none} % do not increment counter
\begin{longtable}[]{@{}
  >{\raggedright\arraybackslash}p{(\linewidth - 6\tabcolsep) * \real{0.1556}}
  >{\raggedright\arraybackslash}p{(\linewidth - 6\tabcolsep) * \real{0.1778}}
  >{\raggedright\arraybackslash}p{(\linewidth - 6\tabcolsep) * \real{0.2889}}
  >{\raggedright\arraybackslash}p{(\linewidth - 6\tabcolsep) * \real{0.3778}}@{}}
\toprule\noalign{}
\begin{minipage}[b]{\linewidth}\raggedright
Phase
\end{minipage} & \begin{minipage}[b]{\linewidth}\raggedright
Action
\end{minipage} & \begin{minipage}[b]{\linewidth}\raggedright
Description
\end{minipage} & \begin{minipage}[b]{\linewidth}\raggedright
Lock Operations
\end{minipage} \\
\midrule\noalign{}
\endhead
\bottomrule\noalign{}
\endlastfoot
\textbf{Growing Phase} & Acquire locks & Transaction obtains all needed
locks & LOCK only \\
\textbf{Shrinking Phase} & Release locks & Transaction releases locks
one by one & UNLOCK only \\
\end{longtable}
}

\textbf{Diagram:}

\begin{verbatim}
Number of Locks
      \^{}
      |     /{}
      |    /  {}
      |   /    {}
      |  /      {}
      | /        {}
      |/          {}
      +{-{-}{-}{-}{-}{-}{-}{-}{-}{-}{-}{-}}
     Growing  Shrinking
     Phase     Phase
        Time
\end{verbatim}

\begin{itemize}
\tightlist
\item
  \textbf{Two phases}: Growing (lock acquisition) and Shrinking (lock
  release)
\item
  \textbf{No lock upgrades} allowed after first unlock operation
\item
  \textbf{Prevents deadlocks} when properly implemented
\item
  \textbf{Serializability guarantee} for concurrent transactions
\end{itemize}

\end{solutionbox}
\begin{mnemonicbox}
``2PL-GS - Two Phase Locking Growing Shrinking''

\end{mnemonicbox}
\subsection*{Question 5(b OR) [4
marks]}\label{question-5b-or-4-marks}

\textbf{Explain any two DML commands with a suitable example.}

\begin{solutionbox}


{\def\LTcaptype{none} % do not increment counter
\begin{longtable}[]{@{}
  >{\raggedright\arraybackslash}p{(\linewidth - 6\tabcolsep) * \real{0.2500}}
  >{\raggedright\arraybackslash}p{(\linewidth - 6\tabcolsep) * \real{0.2500}}
  >{\raggedright\arraybackslash}p{(\linewidth - 6\tabcolsep) * \real{0.2500}}
  >{\raggedright\arraybackslash}p{(\linewidth - 6\tabcolsep) * \real{0.2500}}@{}}
\toprule\noalign{}
\begin{minipage}[b]{\linewidth}\raggedright
Command
\end{minipage} & \begin{minipage}[b]{\linewidth}\raggedright
Purpose
\end{minipage} & \begin{minipage}[b]{\linewidth}\raggedright
Syntax
\end{minipage} & \begin{minipage}[b]{\linewidth}\raggedright
Example
\end{minipage} \\
\midrule\noalign{}
\endhead
\bottomrule\noalign{}
\endlastfoot
\textbf{INSERT} & Add new records & INSERT INTO & INSERT INTO Student
VALUES\ldots{} \\
\textbf{UPDATE} & Modify existing records & UPDATE SET & UPDATE Student
SET name=\ldots{} \\
\end{longtable}
}

\textbf{INSERT Command:}

\begin{verbatim}
{-{-} Insert single record}
INSERT INTO EMPLOYEE (empid, name, salary, deptid)
VALUES (101, {John Smith}, 50000, 10);

{-{-} Insert multiple records}
INSERT INTO EMPLOYEE 
VALUES (102, {Jane Doe}, 45000, 20),
       (103, {Bob Wilson}, 55000, 10);
\end{verbatim}

\textbf{UPDATE Command:}

\begin{verbatim}
{-{-} Update single record}
UPDATE EMPLOYEE 
SET salary = 60000 
WHERE empid = 101;

{-{-} Update multiple records}
UPDATE EMPLOYEE 
SET salary = salary * 1.10 
WHERE deptid = 10;
\end{verbatim}

\begin{itemize}
\tightlist
\item
  \textbf{INSERT} adds new rows to table
\item
  \textbf{UPDATE} modifies existing row values
\item
  \textbf{WHERE clause} specifies update conditions
\item
  \textbf{DML commands} require explicit commit
\end{itemize}

\end{solutionbox}
\begin{mnemonicbox}
``IU-AM - INSERT Adds, UPDATE Modifies''

\end{mnemonicbox}
\subsection*{Question 5(c OR) [7
marks]}\label{question-5c-or-7-marks}

\textbf{List problems of concurrency control and explain any two in
detail.}

\begin{solutionbox}

\textbf{Concurrency Control Problems:}

\begin{enumerate}
\tightlist
\item
  Lost Update Problem
\item
  Dirty Read Problem\\
\item
  Unrepeatable Read Problem
\item
  Phantom Read Problem
\item
  Inconsistent Analysis Problem
\end{enumerate}


{\def\LTcaptype{none} % do not increment counter
\begin{longtable}[]{@{}
  >{\raggedright\arraybackslash}p{(\linewidth - 4\tabcolsep) * \real{0.2812}}
  >{\raggedright\arraybackslash}p{(\linewidth - 4\tabcolsep) * \real{0.4062}}
  >{\raggedright\arraybackslash}p{(\linewidth - 4\tabcolsep) * \real{0.3125}}@{}}
\toprule\noalign{}
\begin{minipage}[b]{\linewidth}\raggedright
Problem
\end{minipage} & \begin{minipage}[b]{\linewidth}\raggedright
Description
\end{minipage} & \begin{minipage}[b]{\linewidth}\raggedright
Solution
\end{minipage} \\
\midrule\noalign{}
\endhead
\bottomrule\noalign{}
\endlastfoot
\textbf{Lost Update} & One transaction overwrites another's changes &
Locking mechanisms \\
\textbf{Dirty Read} & Reading uncommitted data & Read committed
isolation \\
\end{longtable}
}

\textbf{Lost Update Problem:}

\begin{itemize}
\tightlist
\item
  \textbf{Scenario}: Two transactions read same data, modify it, and
  write back
\item
  \textbf{Example}:

  \begin{itemize}
  \tightlist
  \item
    T1 reads account balance: \$1000
  \item
    T2 reads account balance: \$1000\\
  \item
    T1 adds \$100, writes \$1100
  \item
    T2 subtracts \$50, writes \$950
  \item
    \textbf{Result}: T1's update lost, final balance incorrect
  \end{itemize}
\end{itemize}

\textbf{Dirty Read Problem:}

\begin{itemize}
\tightlist
\item
  \textbf{Scenario}: Transaction reads data modified by another
  uncommitted transaction
\item
  \textbf{Example}:

  \begin{itemize}
  \tightlist
  \item
    T1 updates account balance from \$1000 to \$1500
  \item
    T2 reads balance as \$1500 (uncommitted data)
  \item
    T1 fails and rolls back to \$1000
  \item
    \textbf{Result}: T2 used incorrect data for calculations
  \end{itemize}
\end{itemize}

\textbf{Solutions:}

\begin{itemize}
\tightlist
\item
  \textbf{Locking protocols}: Prevent simultaneous access to same data
\item
  \textbf{Isolation levels}: Control visibility of uncommitted changes
\item
  \textbf{Timestamp ordering}: Order transactions based on timestamps
\item
  \textbf{Multi-version concurrency}: Maintain multiple data versions
\end{itemize}

\end{solutionbox}
\begin{mnemonicbox}
``LDUI - Lost Dirty Unrepeatable Inconsistent''

\end{mnemonicbox}

\end{document}
