\documentclass[10pt,a4paper]{article}

% content/resources/templates/preamble.tex
\usepackage[margin=0.6in]{geometry}
\author{Milav Dabgar}
\usepackage{amsmath,amssymb,amsthm}
\usepackage{booktabs}
\usepackage{multirow}
\usepackage{xcolor}
\usepackage{tcolorbox}
\tcbuselibrary{breakable,skins}
\usepackage[colorlinks=true,linkcolor=blue]{hyperref}
\usepackage{titlesec}
\usepackage{enumitem}
\usepackage{tikz}
\usepackage{pgfplots}
\usepackage{circuitikz}
\usepackage[version=4]{mhchem}
\usepackage{longtable}
\usepackage{array}
\usepackage{float}
\usepackage{caption}
\usepackage{listings}

\lstset{
  basicstyle=\small\ttfamily,
  breaklines=true,
  breakatwhitespace=false,
  postbreak=\mbox{\textcolor{red}{$\hookrightarrow$}\space},
  float=false,
  numbers=left,
  numberstyle=\tiny\color{gray},
  numbersep=10pt,
  xleftmargin=2em,
  keywordstyle=\color{blue},
  commentstyle=\color{green!60!black},
  stringstyle=\color{purple},
  backgroundcolor=\color{gray!5},
  showstringspaces=false,
  tabsize=2,
  captionpos=b,
  keepspaces=true,
  columns=flexible
}

\pgfplotsset{compat=1.18}
\usetikzlibrary{shapes,arrows,positioning,calc,patterns,decorations.pathmorphing,decorations.markings,arrows.meta}

% Color scheme
\definecolor{headcolor}{RGB}{0,102,204}
\definecolor{keycolor}{RGB}{220,20,60}
\definecolor{solutioncolor}{RGB}{34,139,34}
\definecolor{mnemoniccolor}{RGB}{148,0,211}
\definecolor{codecolor}{RGB}{0,0,100}

% Spacing
\setlength{\parskip}{3pt}
\setlist[itemize]{nosep}
\setlist[enumerate]{nosep}

% Title formatting
\titleformat{\section}{\Large\bfseries\color{headcolor}}{\thesection}{1em}{}
\titleformat{\subsection}{\large\bfseries\color{headcolor}}{\thesubsection}{1em}{}

% Pandoc tightlist compatibility
\providecommand{\tightlist}{%
  \setlength{\itemsep}{0pt}\setlength{\parskip}{0pt}}

% Pandoc longtable compatibility
\newcounter{none}
\def\thenone{}


% content/resources/templates/gujarati-boxes.tex
\usepackage{fontspec}
\usepackage{polyglossia}

% Set Gujarati as main language (document is primarily in Gujarati)
% Note: gloss-gujarati.ldf doesn't exist in polyglossia, but it will use hyphenation patterns
\setdefaultlanguage{gujarati}
\setotherlanguage{english}

% Configure Gujarati font properly
% Use Language=Default to prevent polyglossia from trying to add language-specific features
% that don't exist for Gujarati, which causes "empty feature" warnings
\newfontfamily\gujaratifont[Script=Gujarati,AutoFakeBold=2.5,AutoFakeSlant=0.3]{Noto Sans Gujarati}
\setmainfont[Script=Gujarati,AutoFakeBold=2.5,AutoFakeSlant=0.3]{Noto Sans Gujarati}
% Use Noto Sans Gujarati for monospace to support Gujarati in text
\setmonofont[Scale=0.9]{Noto Sans Gujarati}

% Configure English to use the same font
\newfontfamily\englishfont[Script=Gujarati,AutoFakeBold=2.5,AutoFakeSlant=0.3]{Noto Sans Gujarati}

% Translations for polyglossia
\gappto\captionsgujarati{
  \renewcommand{\tablename}{કોષ્ટક}
  \renewcommand{\figurename}{આકૃતિ}
}

% Helper for TikZ nodes to ensure Gujarati font
\newcommand{\gu}[1]{{\gujaratifont #1}}

% Custom environments
\newtcolorbox{solutionbox}{
    breakable,
    enhanced,
    colback=solutioncolor!5!white,
    colframe=solutioncolor!75!black,
    fonttitle=\bfseries,
    title=જવાબ
}

\newtcolorbox{solutionboxnobreak}{
 colback=solutioncolor!5!white,
 colframe=solutioncolor!75!black,
 fonttitle=\bfseries,
 title=જવાબ
}

\newtcolorbox{keyformula}{
 breakable,
 enhanced,
 colback=keycolor!5!white,
 colframe=keycolor!75!black,
 fonttitle=\bfseries,
 title=રાસાયણિક સમીકરણ/સૂત્ર
}

\newtcolorbox{mnemonicbox}{
 breakable,
 enhanced,
 colback=mnemoniccolor!5!white,
 colframe=mnemoniccolor!75!black,
 fonttitle=\bfseries,
 title=મેમરી ટ્રીક
}


\begin{document}

\begin{center}
{\Huge\bfseries\color{headcolor} Subject Name (Gujarati)}\\[5pt]
{\LARGE 4331603 -- Winter 2023}\\[3pt]
{\large Semester 1 Study Material}\\[3pt]
{\normalsize\textit{Detailed Solutions and Explanations}}
\end{center}

\vspace{10pt}

\subsection*{પ્રશ્ન 1(a) [3
ગુણ]}\label{q1a}

\textbf{નીચેના શબ્દો વ્યાખ્યાયિત કરો: a). Data items b). Data dictionary
c).Meta data}

\begin{solutionbox}

{\def\LTcaptype{none} % do not increment counter
\begin{longtable}[]{@{}
  >{\raggedright\arraybackslash}p{(\linewidth - 2\tabcolsep) * \real{0.3333}}
  >{\raggedright\arraybackslash}p{(\linewidth - 2\tabcolsep) * \real{0.6667}}@{}}
\toprule\noalign{}
\begin{minipage}[b]{\linewidth}\raggedright
શબ્દ
\end{minipage} & \begin{minipage}[b]{\linewidth}\raggedright
વ્યાખ્યા
\end{minipage} \\
\midrule\noalign{}
\endhead
\bottomrule\noalign{}
\endlastfoot
\textbf{Data Items} & ડેટાના મૂળભૂત એકમો જે વધુ વિભાજન કરી શકાતા નથી. ડેટાબેઝ
ફીલ્ડ્સમાં સંગ્રહિત વ્યક્તિગત તથ્યો અથવા મૂલ્યો \\
\textbf{Data Dictionary} & ડેટાબેઝ સ્ટ્રક્ચર, ટેબલ્સ, કોલમ્સ અને સંબંધો વિશે મેટાડેટા
ધરાવતો કેન્દ્રીય ભંડાર \\
\textbf{Metadata} & ડેટા વિશેનો ડેટા જે ડેટાબેઝ એલિમેન્ટ્સની રચના, અવરોધો અને
ગુણધર્મોનું વર્ણન કરે છે \\
\end{longtable}
}

\end{solutionbox}
\begin{mnemonicbox}
``DDM - Data Dictionary Manages''

\end{mnemonicbox}
\begin{center}\rule{0.5\linewidth}{0.5pt}\end{center}

\subsection*{પ્રશ્ન 1(b) [4
ગુણ]}\label{q1b}

\textbf{ફાઇલ ઓરિએન્ટેડ સિસ્ટમના ગેરફાયદા સમજાવો.}

\begin{solutionbox}

{\def\LTcaptype{none} % do not increment counter
\begin{longtable}[]{@{}
  >{\raggedright\arraybackslash}p{(\linewidth - 2\tabcolsep) * \real{0.5185}}
  >{\raggedright\arraybackslash}p{(\linewidth - 2\tabcolsep) * \real{0.4815}}@{}}
\toprule\noalign{}
\begin{minipage}[b]{\linewidth}\raggedright
ગેરફાયદો
\end{minipage} & \begin{minipage}[b]{\linewidth}\raggedright
વિવરણ
\end{minipage} \\
\midrule\noalign{}
\endhead
\bottomrule\noalign{}
\endlastfoot
\textbf{ડેટા રીડન્ડન્સી} & બહુવિધ ફાઇલોમાં સમાન ડેટાનો સંગ્રહ, જે સ્ટોરેજનો બગાડ કરે
છે \\
\textbf{ડેટા અસંગતતા} & વિવિધ ફાઇલોમાં સમાન ડેટાના વિવિધ વર્ઝન \\
\textbf{ડેટા આઇસોલેશન} & બહુવિધ ફાઇલોમાં વિખરાયેલા ડેટાને એક્સેસ કરવામાં મુશ્કેલી \\
\textbf{સિક્યોરિટી સમસ્યાઓ} & મર્યાદિત એક્સેસ કંટ્રોલ અને સુરક્ષા મિકેનિઝમ \\
\end{longtable}
}

\end{solutionbox}
\begin{mnemonicbox}
``RDIS - Really Difficult Information System''

\end{mnemonicbox}
\begin{center}\rule{0.5\linewidth}{0.5pt}\end{center}

\subsection*{પ્રશ્ન 1(c) [7
ગુણ]}\label{q1c}

\textbf{DBA ની જવાબદારીઓનું વિગતવાર વર્ણન કરો.}

\begin{solutionbox}

{\def\LTcaptype{none} % do not increment counter
\begin{longtable}[]{@{}ll@{}}
\toprule\noalign{}
જવાબદારી & વિગતો \\
\midrule\noalign{}
\endhead
\bottomrule\noalign{}
\endlastfoot
\textbf{ડેટાબેઝ ડિઝાઇન} & લોજિકલ અને ફિઝિકલ ડેટાબેઝ સ્ટ્રક્ચર બનાવવું \\
\textbf{સિક્યોરિટી મેનેજમેન્ટ} & યુઝર એક્સેસ કંટ્રોલ અને ડેટા પ્રોટેક્શન લાગુ કરવું \\
\textbf{પર્ફોર્મન્સ મોનિટરિંગ} & ડેટાબેઝ પર્ફોર્મન્સ અને ક્વેરી એક્ઝિક્યુશન ઓપ્ટિમાઇઝ
કરવું \\
\textbf{બેકઅપ અને રિકવરી} & નિયમિત બેકઅપ દ્વારા ડેટા સેફ્ટી સુનિશ્ચિત કરવી \\
\textbf{યુઝર સપોર્ટ} & ડેટાબેઝ યુઝર્સને ટેકનિકલ સહાય પૂરી પાડવી \\
\textbf{સિસ્ટમ મેઇન્ટેનન્સ} & નિયમિત અપડેટ્સ, પેચેસ અને સિસ્ટમ ઓપ્ટિમાઇઝેશન \\
\end{longtable}
}

\begin{center}
\textbf{Mermaid Diagram (Code)}
\begin{verbatim}
{Shaded}
{Highlighting}[]
graph TD
    A[DBA જવાબદારીઓ] {-{-}{} B[ડિઝાઇન અને પ્લાનિંગ]}
    A {-{-}{} C[સિક્યોરિટી અને એક્સેસ]}
    A {-{-}{} D[પર્ફોર્મન્સ અને ઓપ્ટિમાઇઝેશન]}
    A {-{-}{} E[બેકઅપ અને રિકવરી]}
    A {-{-}{} F[યુઝર સપોર્ટ]}
    A {-{-}{} G[મેઇન્ટેનન્સ]}
{Highlighting}
{Shaded}
\end{verbatim}
\end{center}

\end{solutionbox}
\begin{mnemonicbox}
``DSPBUM - Database Specialists Provide Better User
Management''

\end{mnemonicbox}
\begin{center}\rule{0.5\linewidth}{0.5pt}\end{center}

\subsection*{પ્રશ્ન 1(c OR) [7
ગુણ]}\label{uxaaauxab0uxab6uxaa8-1c-or-7-uxa97uxaa3}

\textbf{Data abstraction ની વ્યાખ્યા આપો? DBMS નું ત્રિ સ્તરનું આર્કિટેક્ચર
સમજાવો.}

\begin{solutionbox}

\textbf{Data Abstraction}: યુઝર્સને માત્ર આવશ્યક ફીચર્સ દર્શાવતી વખતે જટિલ
implementation વિગતો છુપાવવાની પ્રક્રિયા.

{\def\LTcaptype{none} % do not increment counter
\begin{longtable}[]{@{}
  >{\raggedright\arraybackslash}p{(\linewidth - 4\tabcolsep) * \real{0.2414}}
  >{\raggedright\arraybackslash}p{(\linewidth - 4\tabcolsep) * \real{0.4483}}
  >{\raggedright\arraybackslash}p{(\linewidth - 4\tabcolsep) * \real{0.3103}}@{}}
\toprule\noalign{}
\begin{minipage}[b]{\linewidth}\raggedright
સ્તર
\end{minipage} & \begin{minipage}[b]{\linewidth}\raggedright
વિવરણ
\end{minipage} & \begin{minipage}[b]{\linewidth}\raggedright
હેતુ
\end{minipage} \\
\midrule\noalign{}
\endhead
\bottomrule\noalign{}
\endlastfoot
\textbf{External Level} & ડેટાબેઝનો યુઝર વ્યૂ & વ્યક્તિગત યુઝર પરસ્પેક્ટિવ્સ \\
\textbf{Conceptual Level} & સંપૂર્ણ ડેટાબેઝની લોજિકલ સ્ટ્રક્ચર & એકંદર ડેટાબેઝ
ઓર્ગેનાઇઝેશન \\
\textbf{Internal Level} & ફિઝિકલ સ્ટોરેજ વિગતો & ડેટા ખરેખર કેવી રીતે સ્ટોર થાય
છે \\
\end{longtable}
}

\begin{center}
\textbf{Mermaid Diagram (Code)}
\begin{verbatim}
{Shaded}
{Highlighting}[]
graph LR
    A[External Level{br/{}યુઝર વ્યૂઝ] {-}{-}{} B[Conceptual Level{}br/{}લોજિકલ સ્કીમા]}
    B {-{-}{} C[Internal Level{}br/{}ફિઝિકલ સ્કીમા]}
    
    A1[યુઝર 1 વ્યૂ] {-{-}{} A}
    A2[યુઝર 2 વ્યૂ] {-{-}{} A}
    A3[યુઝર 3 વ્યૂ] {-{-}{} A}
{Highlighting}
{Shaded}
\end{verbatim}
\end{center}

\end{solutionbox}
\begin{mnemonicbox}
``ECI - Every Computer Industry''

\end{mnemonicbox}
\begin{center}\rule{0.5\linewidth}{0.5pt}\end{center}

\subsection*{પ્રશ્ન 2(a) [3
ગુણ]}\label{q2a}

\textbf{નીચેના શબ્દો વ્યાખ્યાયિત કરો: a).Relationship set b).Participation
c).Candidate key}

\begin{solutionbox}

{\def\LTcaptype{none} % do not increment counter
\begin{longtable}[]{@{}
  >{\raggedright\arraybackslash}p{(\linewidth - 2\tabcolsep) * \real{0.3333}}
  >{\raggedright\arraybackslash}p{(\linewidth - 2\tabcolsep) * \real{0.6667}}@{}}
\toprule\noalign{}
\begin{minipage}[b]{\linewidth}\raggedright
શબ્દ
\end{minipage} & \begin{minipage}[b]{\linewidth}\raggedright
વ્યાખ્યા
\end{minipage} \\
\midrule\noalign{}
\endhead
\bottomrule\noalign{}
\endlastfoot
\textbf{Relationship Set} & એન્ટિટી સેટ્સ વચ્ચે સમાન પ્રકારના સંબંધોનો સંગ્રહ \\
\textbf{Participation} & અવરોધ જે સ્પષ્ટ કરે છે કે એન્ટિટી ઓકરન્સ સંબંધમાં ફરજિયાત
છે કે નહીં \\
\textbf{Candidate Key} & એટ્રિબ્યુટ્સનો ન્યૂનતમ સેટ જે એન્ટિટી સેટમાં દરેક એન્ટિટીને
અનન્ય રીતે ઓળખે છે \\
\end{longtable}
}

\end{solutionbox}
\begin{mnemonicbox}
``RPC - Relationship Participation Candidate''

\end{mnemonicbox}
\begin{center}\rule{0.5\linewidth}{0.5pt}\end{center}

\subsection*{પ્રશ્ન 2(b) [4
ગુણ]}\label{q2b}

\textbf{Generalization ઉદાહરણ સાથે સમજાવો.}

\begin{solutionbox}

\textbf{Generalization}: બોટમ-અપ અપ્રોચ જ્યાં નીચલા-સ્તરની એન્ટિટીઝના સામાન્ય
એટ્રિબ્યુટ્સને ઉચ્ચ-સ્તરની એન્ટિટીમાં જોડવામાં આવે છે.

{\def\LTcaptype{none} % do not increment counter
\begin{longtable}[]{@{}ll@{}}
\toprule\noalign{}
ખ્યાલ & વિવરણ \\
\midrule\noalign{}
\endhead
\bottomrule\noalign{}
\endlastfoot
\textbf{હેતુ} & સામાન્ય સુપરક્લાસ બનાવીને રીડન્ડન્સી ઘટાડવી \\
\textbf{દિશા} & બોટમ-અપ (વિશિષ્ટથી સામાન્ય) \\
\textbf{ઉદાહરણ} & Car, Truck, Bus \rightarrow Vehicle \\
\end{longtable}
}

\begin{verbatim}
graph BT
    A[Car] {-{-} D[Vehicle]}
    B[Truck] {-{-} D}
    C[Bus] {-{-} D}
    
    A1[Brand, Model, Fuel Type] {-{-} A}
    B1[Brand, Model, Load Capacity] {-{-} B}
    C1[Brand, Model, Seating Capacity] {-{-} C}
    D1[Vehicle\_ID, Brand, Model] {-{-} D}
\end{verbatim}

\end{solutionbox}
\begin{mnemonicbox}
``GBU - Generalization Builds Up''

\end{mnemonicbox}
\begin{center}\rule{0.5\linewidth}{0.5pt}\end{center}

\subsection*{પ્રશ્ન 2(c) [7
ગુણ]}\label{q2c}

\textbf{E-R Diagram ની વ્યાખ્યા આપો? E-R ડાયાગ્રામમાં વપરાતા વિવિધ Symbols ને
ઉદાહરણ સાથે સમજાવો.}

\begin{solutionbox}

\textbf{E-R Diagram}: ડેટાબેઝ ડિઝાઇનમાં એન્ટિટીઝ, એટ્રિબ્યુટ્સ અને સંબંધો દર્શાવતું
ગ્રાફિકલ પ્રતિનિધિત્વ.

{\def\LTcaptype{none} % do not increment counter
\begin{longtable}[]{@{}llll@{}}
\toprule\noalign{}
સિમ્બોલ & આકાર & ઉપયોગ & ઉદાહરણ \\
\midrule\noalign{}
\endhead
\bottomrule\noalign{}
\endlastfoot
\textbf{Entity} & લંબચોરસ & ઓબ્જેક્ટ્સનું પ્રતિનિધિત્વ & Student, Course \\
\textbf{Attribute} & અંડાકાર & એન્ટિટીઝના ગુણધર્મો & Name, Age, ID \\
\textbf{Relationship} & હીરા & એન્ટિટીઝ વચ્ચેના જોડાણો & Enrolls,
Teaches \\
\textbf{Primary Key} & અન્ડરલાઇન્ડ અંડાકાર & અનન્ય ઓળખકર્તા & Student\_ID \\
\textbf{Multivalued} & ડબલ અંડાકાર & બહુવિધ મૂલ્યો & Phone\_Numbers \\
\textbf{Derived} & ડેશ્ડ અંડાકાર & ગણતરી કરેલા એટ્રિબ્યુટ્સ & Age from DOB \\
\end{longtable}
}

\begin{verbatim}
erDiagram
    STUDENT \{
        int student\_id PK
        string name
        date birth\_date
        string email
    \}
    COURSE \{
        int course\_id PK
        string course\_name
        int credits
    \}
    STUDENT ||{-{-}o\{ ENROLLMENT : enrolls}
    COURSE ||{-{-}o\{ ENROLLMENT : "enrolled in"}
    ENROLLMENT \{
        int student\_id FK
        int course\_id FK
        date enrollment\_date
        string grade
    \}
\end{verbatim}

\end{solutionbox}
\begin{mnemonicbox}
``EARPM - Every Attribute Represents Proper
Meaning''

\end{mnemonicbox}
\begin{center}\rule{0.5\linewidth}{0.5pt}\end{center}

\subsection*{પ્રશ્ન 2(a OR) [3
ગુણ]}\label{uxaaauxab0uxab6uxaa8-2a-or-3-uxa97uxaa3}

\textbf{Relational Algebra ની વ્યાખ્યા આપો? Relational Algebra માં વિવિધ
કામગીરીની યાદી આપો?}

\begin{solutionbox}

\textbf{Relational Algebra}: રિલેશનલ ડેટાબેઝ ટેબલ્સને મેનિપ્યુલેટ કરવા માટેની
ઓપરેશન્સ સાથે ફોર્મલ ક્વેરી લેંગ્વેજ.

{\def\LTcaptype{none} % do not increment counter
\begin{longtable}[]{@{}
  >{\raggedright\arraybackslash}p{(\linewidth - 2\tabcolsep) * \real{0.5714}}
  >{\raggedright\arraybackslash}p{(\linewidth - 2\tabcolsep) * \real{0.4286}}@{}}
\toprule\noalign{}
\begin{minipage}[b]{\linewidth}\raggedright
ઓપરેશન પ્રકાર
\end{minipage} & \begin{minipage}[b]{\linewidth}\raggedright
ઓપરેશન્સ
\end{minipage} \\
\midrule\noalign{}
\endhead
\bottomrule\noalign{}
\endlastfoot
\textbf{મૂળભૂત ઓપરેશન્સ} & Select, Project, Union, Set Difference, Cartesian
Product \\
\textbf{વધારાની ઓપરેશન્સ} & Intersection, Join, Division, Rename \\
\end{longtable}
}

\end{solutionbox}
\begin{mnemonicbox}
``SPUDC-IJDR - Simple People Use Database Concepts''

\end{mnemonicbox}
\begin{center}\rule{0.5\linewidth}{0.5pt}\end{center}

\subsection*{પ્રશ્ન 2(b OR) [4
ગુણ]}\label{uxaaauxab0uxab6uxaa8-2b-or-4-uxa97uxaa3}

\textbf{Specialization ઉદાહરણ સાથે સમજાવો.}

\begin{solutionbox}

\textbf{Specialization}: ટોપ-ડાઉન અપ્રોચ જ્યાં ઉચ્ચ-સ્તરની એન્ટિટીને વિશિષ્ટ
નીચલા-સ્તરની એન્ટિટીઝમાં વિભાજિત કરવામાં આવે છે.

{\def\LTcaptype{none} % do not increment counter
\begin{longtable}[]{@{}ll@{}}
\toprule\noalign{}
ખ્યાલ & વિવરણ \\
\midrule\noalign{}
\endhead
\bottomrule\noalign{}
\endlastfoot
\textbf{હેતુ} & અનન્ય એટ્રિબ્યુટ્સ સાથે વિશિષ્ટ સબક્લાસીસ બનાવવી \\
\textbf{દિશા} & ટોપ-ડાઉન (સામાન્યથી વિશિષ્ટ) \\
\textbf{ઉદાહરણ} & Employee \rightarrow Manager, Clerk, Engineer \\
\end{longtable}
}

\begin{center}
\textbf{Mermaid Diagram (Code)}
\begin{verbatim}
{Shaded}
{Highlighting}[]
graph TD
    A[Employee{br/{}Emp\_ID, Name, Salary] {-}{-}{} B[Manager{}br/{}Department]}
    A {-{-}{} C[Clerk{}br/{}Typing\_Speed]}
    A {-{-}{} D[Engineer{}br/{}Specialization]}
{Highlighting}
{Shaded}
\end{verbatim}
\end{center}

\end{solutionbox}
\begin{mnemonicbox}
``STD - Specialization Top Down''

\end{mnemonicbox}
\begin{center}\rule{0.5\linewidth}{0.5pt}\end{center}

\subsection*{પ્રશ્ન 2(c OR) [7
ગુણ]}\label{uxaaauxab0uxab6uxaa8-2c-or-7-uxa97uxaa3}

\textbf{Attribute ની વ્યાખ્યા આપો? વિવિધ પ્રકારના Attribute ને ઉદાહરણ સાથે
સમજાવો.}

\begin{solutionbox}

\textbf{Attribute}: એન્ટિટીનું વર્ણન કરતી મિલકત અથવા લાક્ષણિકતા.

{\def\LTcaptype{none} % do not increment counter
\begin{longtable}[]{@{}lll@{}}
\toprule\noalign{}
એટ્રિબ્યુટ પ્રકાર & વિવરણ & ઉદાહરણ \\
\midrule\noalign{}
\endhead
\bottomrule\noalign{}
\endlastfoot
\textbf{Simple} & વધુ વિભાજન કરી શકાતું નથી & Age, Name \\
\textbf{Composite} & ઉપવિભાગ કરી શકાય છે & Address (Street, City,
State) \\
\textbf{Single-valued} & એક મૂલ્ય ધરાવે છે & SSN, Employee\_ID \\
\textbf{Multi-valued} & બહુવિધ મૂલ્યો હોઈ શકે છે & Phone\_Numbers, Skills \\
\textbf{Derived} & અન્ય એટ્રિબ્યુટ્સથી ગણતરી કરેલ & Age from Birth\_Date \\
\textbf{Key} & એન્ટિટીને અનન્ય રીતે ઓળખે છે & Student\_ID \\
\end{longtable}
}

\begin{center}
\textbf{Mermaid Diagram (Code)}
\begin{verbatim}
{Shaded}
{Highlighting}[]
graph TD
    A[Attributes] {-{-}{} B[Simple{}br/{}Age, Name]}
    A {-{-}{} C[Composite{}br/{}Address]}
    A {-{-}{} D[Multi{-}valued{}br/{}Phone Numbers]}
    A {-{-}{} E[Derived{}br/{}Age from DOB]}
    
    C {-{-}{} F[Street]}
    C {-{-}{} G[City]}
    C {-{-}{} H[State]}
{Highlighting}
{Shaded}
\end{verbatim}
\end{center}

\end{solutionbox}
\begin{mnemonicbox}
``SCSMDK - Simple Composite Single Multi Derived
Key''

\end{mnemonicbox}
\begin{center}\rule{0.5\linewidth}{0.5pt}\end{center}

\subsection*{પ્રશ્ન 3(a) [3
ગુણ]}\label{q3a}

\textbf{SQL માં GRANT અને REVOKE સ્ટેટમેન્ટ સમજાવો.}

\begin{solutionbox}

{\def\LTcaptype{none} % do not increment counter
\begin{longtable}[]{@{}
  >{\raggedright\arraybackslash}p{(\linewidth - 4\tabcolsep) * \real{0.3056}}
  >{\raggedright\arraybackslash}p{(\linewidth - 4\tabcolsep) * \real{0.2500}}
  >{\raggedright\arraybackslash}p{(\linewidth - 4\tabcolsep) * \real{0.4444}}@{}}
\toprule\noalign{}
\begin{minipage}[b]{\linewidth}\raggedright
સ્ટેટમેન્ટ
\end{minipage} & \begin{minipage}[b]{\linewidth}\raggedright
હેતુ
\end{minipage} & \begin{minipage}[b]{\linewidth}\raggedright
સિન્ટેક્સ ઉદાહરણ
\end{minipage} \\
\midrule\noalign{}
\endhead
\bottomrule\noalign{}
\endlastfoot
\textbf{GRANT} & યુઝર્સને વિશેષાધિકારો પ્રદાન કરે છે &
\texttt{GRANT\ SELECT\ ON\ table\ TO\ user} \\
\textbf{REVOKE} & યુઝર્સ પાસેથી વિશેષાધિકારો દૂર કરે છે &
\texttt{REVOKE\ SELECT\ ON\ table\ FROM\ user} \\
\end{longtable}
}

\textbf{સામાન્ય વિશેષાધિકારો}: SELECT, INSERT, UPDATE, DELETE, ALL

\end{solutionbox}
\begin{mnemonicbox}
``GR - Grant Removes (via REVOKE)''

\end{mnemonicbox}
\begin{center}\rule{0.5\linewidth}{0.5pt}\end{center}

\subsection*{પ્રશ્ન 3(b) [4
ગુણ]}\label{q3b}

\textbf{નીચેના Character function સમજાવો .1) INSTR 2) LENGTH}

\begin{solutionbox}

{\def\LTcaptype{none} % do not increment counter
\begin{longtable}[]{@{}
  >{\raggedright\arraybackslash}p{(\linewidth - 6\tabcolsep) * \real{0.2778}}
  >{\raggedright\arraybackslash}p{(\linewidth - 6\tabcolsep) * \real{0.2500}}
  >{\raggedright\arraybackslash}p{(\linewidth - 6\tabcolsep) * \real{0.2222}}
  >{\raggedright\arraybackslash}p{(\linewidth - 6\tabcolsep) * \real{0.2500}}@{}}
\toprule\noalign{}
\begin{minipage}[b]{\linewidth}\raggedright
ફંક્શન
\end{minipage} & \begin{minipage}[b]{\linewidth}\raggedright
હેતુ
\end{minipage} & \begin{minipage}[b]{\linewidth}\raggedright
સિન્ટેક્સ
\end{minipage} & \begin{minipage}[b]{\linewidth}\raggedright
ઉદાહરણ
\end{minipage} \\
\midrule\noalign{}
\endhead
\bottomrule\noalign{}
\endlastfoot
\textbf{INSTR} & સબસ્ટ્રિંગની સ્થિતિ શોધે છે &
\texttt{INSTR(string,\ substring)} &
\texttt{INSTR(\textquotesingle{}Hello\textquotesingle{},\ \textquotesingle{}e\textquotesingle{})}
2 રિટર્ન કરે છે \\
\textbf{LENGTH} & સ્ટ્રિંગની લંબાઈ રિટર્ન કરે છે & \texttt{LENGTH(string)} &
\texttt{LENGTH(\textquotesingle{}Hello\textquotesingle{})} 5 રિટર્ન કરે
છે \\
\end{longtable}
}

\end{solutionbox}
\begin{mnemonicbox}
``IL - INSTR Locates, LENGTH measures''

\end{mnemonicbox}
\begin{center}\rule{0.5\linewidth}{0.5pt}\end{center}

\subsection*{પ્રશ્ન 3(c) [7
ગુણ]}\label{q3c}

\textbf{નીચેના Table માટે SQL સ્ટેટમેન્ટ લખો:
Student(Enno,name,branch,sem,clgname,bdate)}

\begin{solutionbox}

\begin{verbatim}
{-{-} 1. Create a table Student}
CREATE TABLE Student (
    Enno VARCHAR(10) PRIMARY KEY,
    name VARCHAR(50),
    branch VARCHAR(20),
    sem INT,
    clgname VARCHAR(100),
    bdate DATE
);

{-{-} 2. Add a column mobno in Student table}
ALTER TABLE Student ADD mobno VARCHAR(15);

{-{-} 3. Insert one record in student table}
INSERT INTO Student VALUES 
({E001}, {Raj Patel}, {IT}, 3, {GTU College}, {2003{-}05{-}15}, {9876543210});

{-{-} 4. Find out list of students who have enrolled in "IT" branch}
SELECT * FROM Student WHERE branch = {IT};

{-{-} 5. Retrieve all information about student where name begin with a}
SELECT * FROM Student WHERE name LIKE {a\%};

{-{-} 6. Count the number of rows in student table}
SELECT COUNT(*) FROM Student;

{-{-} 7. Delete all record of student table}
DELETE FROM Student;
\end{verbatim}

\end{solutionbox}
\begin{mnemonicbox}
``CAIRSCD - Create Add Insert Retrieve Search Count
Delete''

\end{mnemonicbox}
\begin{center}\rule{0.5\linewidth}{0.5pt}\end{center}

\subsection*{પ્રશ્ન 3(a OR) [3
ગુણ]}\label{uxaaauxab0uxab6uxaa8-3a-or-3-uxa97uxaa3}

\textbf{SQL માં equi join ઉદાહરણ સાથે સમજાવો.}

\begin{solutionbox}

\textbf{Equi Join}: ટેબલ્સને જોડવા માટે સમતા શરતનો ઉપયોગ કરતી જોઇન ઓપરેશન.

{\def\LTcaptype{none} % do not increment counter
\begin{longtable}[]{@{}lll@{}}
\toprule\noalign{}
જોઇન પ્રકાર & શરત & પરિણામ \\
\midrule\noalign{}
\endhead
\bottomrule\noalign{}
\endlastfoot
\textbf{Equi Join} & Column1 = Column2 & બંને ટેબલ્સમાંથી મેચિંગ રો \\
\end{longtable}
}

\begin{verbatim}
{-{-} ઉદાહરણ}
SELECT s.name, c.course\_name 
FROM Student s, Course c 
WHERE s.course\_id = c.course\_id;
\end{verbatim}

\end{solutionbox}
\begin{mnemonicbox}
``EE - Equi Equals''

\end{mnemonicbox}
\begin{center}\rule{0.5\linewidth}{0.5pt}\end{center}

\subsection*{પ્રશ્ન 3(b OR) [4
ગુણ]}\label{uxaaauxab0uxab6uxaa8-3b-or-4-uxa97uxaa3}

\textbf{નીચેના Aggregate function સમજાવો .1) MAX 2) SUM}

\begin{solutionbox}

{\def\LTcaptype{none} % do not increment counter
\begin{longtable}[]{@{}llll@{}}
\toprule\noalign{}
ફંક્શન & હેતુ & સિન્ટેક્સ & ઉદાહરણ \\
\midrule\noalign{}
\endhead
\bottomrule\noalign{}
\endlastfoot
\textbf{MAX} & મહત્તમ મૂલ્ય રિટર્ન કરે છે & \texttt{MAX(column)} &
\texttt{MAX(salary)} \\
\textbf{SUM} & કુલ સરવાળો રિટર્ન કરે છે & \texttt{SUM(column)} &
\texttt{SUM(marks)} \\
\end{longtable}
}

\end{solutionbox}
\begin{mnemonicbox}
``MS - MAX Sum''

\end{mnemonicbox}
\begin{center}\rule{0.5\linewidth}{0.5pt}\end{center}

\subsection*{પ્રશ્ન 3(c OR) [7
ગુણ]}\label{uxaaauxab0uxab6uxaa8-3c-or-7-uxa97uxaa3}

\textbf{નીચેના Table માટે SQL સ્ટેટમેન્ટ લખો:
Employee(EmpID,Ename,DOB,Dept,Salary)}

\begin{solutionbox}

\begin{verbatim}
{-{-} 1. Create a table Employee}
CREATE TABLE Employee (
    EmpID VARCHAR(10) PRIMARY KEY,
    Ename VARCHAR(50),
    DOB DATE,
    Dept VARCHAR(30),
    Salary DECIMAL(10,2)
);

{-{-} 2. Find sum of salaries of all employee}
SELECT SUM(Salary) FROM Employee;

{-{-} 3. Insert one record in Employee table}
INSERT INTO Employee VALUES 
({E001}, {John Doe}, {1990{-}05{-}15}, {IT}, 35000);

{-{-} 4. Find names of employees who salary between 25000/{-} and 48000/{-}}
SELECT Ename FROM Employee WHERE Salary BETWEEN 25000 AND 48000;

{-{-} 5. Display detail of all employees in descending order of their DOB}
SELECT * FROM Employee ORDER BY DOB DESC;

{-{-} 6. List name of all employees whose name ends with a}
SELECT Ename FROM Employee WHERE Ename LIKE {\%a};

{-{-} 7. Find highest and least salaries of all employees}
SELECT MAX(Salary) AS Highest, MIN(Salary) AS Lowest FROM Employee;
\end{verbatim}

\end{solutionbox}
\begin{mnemonicbox}
``CSIDDHL - Create Sum Insert Display Display List
HighLow''

\end{mnemonicbox}
\begin{center}\rule{0.5\linewidth}{0.5pt}\end{center}

\subsection*{પ્રશ્ન 4(a) [3
ગુણ]}\label{q4a}

\textbf{નીચે દર્શાવેલ રિલેશનલ સ્કીમાનું ધ્યાન માં લઇ દરેક ક્વેરી માટે રિલેશનલ એલજીબ્રા
એક્સપ્રેશન લખો.}

\begin{solutionbox}

\begin{verbatim}
Student (Enrollment_No,Name,DOB,SPI)

i. σ(SPI > 7.0)(Student)
ii. π(Name)(σ(Enrollment_No = 007)(Student))
\end{verbatim}

\end{solutionbox}
\begin{mnemonicbox}
``SP - Select Project''

\end{mnemonicbox}
\begin{center}\rule{0.5\linewidth}{0.5pt}\end{center}

\subsection*{પ્રશ્ન 4(b) [4
ગુણ]}\label{q4b}

\textbf{Partial functional dependency ની ટૂંકી નોંધ લખો.}

\begin{solutionbox}

{\def\LTcaptype{none} % do not increment counter
\begin{longtable}[]{@{}
  >{\raggedright\arraybackslash}p{(\linewidth - 2\tabcolsep) * \real{0.4091}}
  >{\raggedright\arraybackslash}p{(\linewidth - 2\tabcolsep) * \real{0.5909}}@{}}
\toprule\noalign{}
\begin{minipage}[b]{\linewidth}\raggedright
ખ્યાલ
\end{minipage} & \begin{minipage}[b]{\linewidth}\raggedright
વિવરણ
\end{minipage} \\
\midrule\noalign{}
\endhead
\bottomrule\noalign{}
\endlastfoot
\textbf{વ્યાખ્યા} & Non-prime એટ્રિબ્યુટ કમ્પોઝિટ પ્રાઇમરી કીના ભાગ પર આધાર રાખે
છે \\
\textbf{ક્યાં જોવા મળે} & કમ્પોઝિટ પ્રાઇમરી કી વાળા ટેબલ્સમાં \\
\textbf{સમસ્યા} & રીડન્ડન્સી અને અપડેટ એનોમેલીઝ કારણભૂત \\
\textbf{સોલ્યુશન} & 2NF માં ડીકમ્પોઝ કરવું \\
\end{longtable}
}

\textbf{ઉદાહરણ}: Table(StudentID, CourseID, StudentName, CourseName) માં,
StudentName માત્ર StudentID પર આધાર રાખે છે (કીનો ભાગ).

\end{solutionbox}
\begin{mnemonicbox}
``PDPR - Partial Dependency Problems Resolved''

\end{mnemonicbox}
\begin{center}\rule{0.5\linewidth}{0.5pt}\end{center}

\subsection*{પ્રશ્ન 4(c) [7
ગુણ]}\label{q4c}

\textbf{Normalization ની જરૂરિયાત સમજાવો? ઉદાહરણ સાથે 2NF વિશે ચર્ચા કરો.}

\begin{solutionbox}

\textbf{Normalization ની જરૂરિયાત}:

{\def\LTcaptype{none} % do not increment counter
\begin{longtable}[]{@{}ll@{}}
\toprule\noalign{}
સમસ્યા & Normalization દ્વારા સોલ્યુશન \\
\midrule\noalign{}
\endhead
\bottomrule\noalign{}
\endlastfoot
\textbf{ડેટા રીડન્ડન્સી} & ડુપ્લિકેટ ડેટા દૂર કરે છે \\
\textbf{અપડેટ એનોમેલીઝ} & અસંગત અપડેટ્સ અટકાવે છે \\
\textbf{ઇન્સર્ટ એનોમેલીઝ} & સ્વતંત્ર ડેટા ઇન્સર્શનની મંજૂરી આપે છે \\
\textbf{ડિલીટ એનોમેલીઝ} & મહત્વપૂર્ણ ડેટાની હાનિ અટકાવે છે \\
\end{longtable}
}

\textbf{Second Normal Form (2NF)}:

\begin{itemize}
\tightlist
\item
  1NF માં હોવું જોઈએ
\item
  કોઈ આંશિક કાર્યાત્મક નિર્ભરતા નહીં
\end{itemize}

\textbf{ઉદાહરણ}:

\begin{verbatim}
2NF પહેલાં:
StudentCourse(StudentID, CourseID, StudentName, CourseName)

2NF પછી:
Student(StudentID, StudentName)
Course(CourseID, CourseName)
Enrollment(StudentID, CourseID)
\end{verbatim}

\end{solutionbox}
\begin{mnemonicbox}
``NUID2 - Normalization Unifies Important Data to
2NF''

\end{mnemonicbox}
\begin{center}\rule{0.5\linewidth}{0.5pt}\end{center}

\subsection*{પ્રશ્ન 4(a OR) [3
ગુણ]}\label{uxaaauxab0uxab6uxaa8-4a-or-3-uxa97uxaa3}

\textbf{નીચે દર્શાવેલ રિલેશનલ સ્કીમાનું ધ્યાન માં લઇ દરેક ક્વેરી માટે રિલેશનલ એલજીબ્રા
એક્સપ્રેશન લખો.}

\begin{solutionbox}

\begin{verbatim}
Student(Enno,name,age,address)

i. π(name)(σ(address = 'Surat')(Student))
ii. π(name)(σ(age > 30)(Student))
\end{verbatim}

\end{solutionbox}
\begin{center}\rule{0.5\linewidth}{0.5pt}\end{center}

\subsection*{પ્રશ્ન 4(b OR) [4
ગુણ]}\label{uxaaauxab0uxab6uxaa8-4b-or-4-uxa97uxaa3}

\textbf{1NF ની વ્યાખ્યા આપો? યોગ્ય ઉદાહરણ સાથે 1NF સમજાવો.}

\begin{solutionbox}

\textbf{First Normal Form (1NF)}: દરેક કૉલમ એટોમિક (અવિભાજ્ય) મૂલ્યો ધરાવે છે,
અને દરેક કૉલમ એક જ પ્રકારના મૂલ્યો ધરાવે છે.

{\def\LTcaptype{none} % do not increment counter
\begin{longtable}[]{@{}ll@{}}
\toprule\noalign{}
નિયમ & વિવરણ \\
\midrule\noalign{}
\endhead
\bottomrule\noalign{}
\endlastfoot
\textbf{એટોમિક મૂલ્યો} & એક સેલમાં બહુવિધ મૂલ્યો નહીં \\
\textbf{રિપીટિંગ ગ્રુપ્સ નહીં} & ડુપ્લિકેટ કૉલમ્સ નહીં \\
\textbf{અનન્ય રો} & દરેક રો અનન્ય હોવી જોઈએ \\
\end{longtable}
}

\textbf{ઉદાહરણ}:

\begin{verbatim}
1NF પહેલાં:
Student(ID, Name, Subjects)
1, John, Math,Science,English

1NF પછી:
Student(ID, Name, Subject)
1, John, Math
1, John, Science  
1, John, English
\end{verbatim}

\end{solutionbox}
\begin{mnemonicbox}
``ANU - Atomic No-repeat Unique''

\end{mnemonicbox}
\begin{center}\rule{0.5\linewidth}{0.5pt}\end{center}

\subsection*{પ્રશ્ન 4(c OR) [7
ગુણ]}\label{uxaaauxab0uxab6uxaa8-4c-or-7-uxa97uxaa3}

\textbf{Transitive Dependency ની વ્યાખ્યા આપો? યોગ્ય ઉદાહરણ સાથે 3NF
સમજાવો.}

\begin{solutionbox}

\textbf{Transitive Dependency}: Non-prime એટ્રિબ્યુટ પ્રાઇમરી કી પર સીધો
આધાર ન રાખીને બીજા non-prime એટ્રિબ્યુટ પર આધાર રાખે છે.

\textbf{Third Normal Form (3NF)}:

\begin{itemize}
\tightlist
\item
  2NF માં હોવું જોઈએ
\item
  કોઈ ટ્રાન્ઝિટિવ નિર્ભરતા નહીં
\end{itemize}

{\def\LTcaptype{none} % do not increment counter
\begin{longtable}[]{@{}ll@{}}
\toprule\noalign{}
3NF પહેલાં & 3NF પછી \\
\midrule\noalign{}
\endhead
\bottomrule\noalign{}
\endlastfoot
Student(ID, Name, DeptCode, DeptName) & Student(ID, Name, DeptCode) \\
DeptName, DeptCode પર આધાર રાખે છે & Department(DeptCode, DeptName) \\
\end{longtable}
}

\begin{center}
\textbf{Mermaid Diagram (Code)}
\begin{verbatim}
{Shaded}
{Highlighting}[]
graph LR
    A[Student\_ID] {-{-}{} B[DeptCode]}
    B {-{-}{} C[DeptName]}
    A {-.{-}{} C}
    
    D[3NF પછી:] 
    E[Student\_ID] {-{-}{} F[DeptCode]}
    G[DeptCode] {-{-}{} H[DeptName]}
{Highlighting}
{Shaded}
\end{verbatim}
\end{center}

\end{solutionbox}
\begin{mnemonicbox}
``T3ND - Transitive Third Normal Form No
Dependencies''

\end{mnemonicbox}
\begin{center}\rule{0.5\linewidth}{0.5pt}\end{center}

\subsection*{પ્રશ્ન 5(a) [3
ગુણ]}\label{q5a}

\textbf{Serializability ની વ્યાખ્યા આપો? Serializability ના નિયમો સમજાવો?}

\begin{solutionbox}

\textbf{Serializability}: સમાંતર ટ્રાન્ઝેક્શન એક્ઝિક્યુશન સીરિયલ એક્ઝિક્યુશનના સમાન
પરિણામ આપે તેની ખાતરી કરતી મિલકત.

{\def\LTcaptype{none} % do not increment counter
\begin{longtable}[]{@{}ll@{}}
\toprule\noalign{}
નિયમ & વિવરણ \\
\midrule\noalign{}
\endhead
\bottomrule\noalign{}
\endlastfoot
\textbf{Conflict Serializability} & વિવિધ ક્રમમાં કોઈ સંઘર્ષકારી ઓપરેશન્સ
નહીં \\
\textbf{View Serializability} & સીરિયલ શેડ્યૂલ જેવા જ રીડ-રાઇટ પેટર્ન \\
\end{longtable}
}

\end{solutionbox}
\begin{mnemonicbox}
``SCV - Serial Conflict View''

\end{mnemonicbox}
\begin{center}\rule{0.5\linewidth}{0.5pt}\end{center}

\subsection*{પ્રશ્ન 5(b) [4
ગુણ]}\label{q5b}

\textbf{Implicit Cursors ના Attribute સમજાવો.}

\begin{solutionbox}

{\def\LTcaptype{none} % do not increment counter
\begin{longtable}[]{@{}ll@{}}
\toprule\noalign{}
એટ્રિબ્યુટ & વિવરણ \\
\midrule\noalign{}
\endhead
\bottomrule\noalign{}
\endlastfoot
\textbf{\%FOUND} & TRUE જો છેલ્લા SQL એ ઓછામાં ઓછી એક રો પર અસર કરી \\
\textbf{\%NOTFOUND} & TRUE જો છેલ્લા SQL એ કોઈ રો પર અસર ન કરી \\
\textbf{\%ROWCOUNT} & છેલ્લા SQL દ્વારા પ્રભાવિત રોની સંખ્યા \\
\textbf{\%ISOPEN} & ઇમ્પ્લિસિટ કર્સર્સ માટે હંમેશા FALSE \\
\end{longtable}
}

\end{solutionbox}
\begin{mnemonicbox}
``FNRI - Found NotFound RowCount IsOpen''

\end{mnemonicbox}
\begin{center}\rule{0.5\linewidth}{0.5pt}\end{center}

\subsection*{પ્રશ્ન 5(c) [7
ગુણ]}\label{q5c}

\textbf{Two phase locking protocol ને યોગ્ય ઉદાહરણ સાથે સમજાવો.}

\begin{solutionbox}

\textbf{Two Phase Locking (2PL)}: બે તબક્કા દ્વારા serializability સુનિશ્ચિત
કરતો પ્રોટોકોલ.

{\def\LTcaptype{none} % do not increment counter
\begin{longtable}[]{@{}lll@{}}
\toprule\noalign{}
તબક્કો & વિવરણ & નિયમો \\
\midrule\noalign{}
\endhead
\bottomrule\noalign{}
\endlastfoot
\textbf{વૃદ્ધિ તબક્કો} & માત્ર લોક મેળવવા & લોક મેળવી શકે છે, છોડી શકતા નથી \\
\textbf{ઘટાડો તબક્કો} & માત્ર લોક છોડવા & લોક છોડી શકે છે, મેળવી શકતા નથી \\
\end{longtable}
}

\textbf{ઉદાહરણ}:

\begin{verbatim}
Transaction T1:
1. Lock(A) - વૃદ્ધિ
2. Lock(B) - વૃદ્ધિ  
3. Read(A), Write(A)
4. Unlock(A) - ઘટાડો
5. Read(B), Write(B)
6. Unlock(B) - ઘટાડો
\end{verbatim}

\begin{center}
\textbf{Mermaid Diagram (Code)}
\begin{verbatim}
{Shaded}
{Highlighting}[]
graph LR
    A[શરુઆત] {-{-}{} B[વૃદ્ધિ તબક્કો{}br/{}લોક મેળવવા]}
    B {-{-}{} C[લોક પોઇન્ટ{}br/{}મહત્તમ લોક્સ]}
    C {-{-}{} D[ઘટાડો તબક્કો{}br/{}લોક છોડવા]}
    D {-{-}{} E[સમાપ્ત]}
{Highlighting}
{Shaded}
\end{verbatim}
\end{center}

\end{solutionbox}
\begin{mnemonicbox}
``2PGS - Two Phase Growing Shrinking''

\end{mnemonicbox}
\begin{center}\rule{0.5\linewidth}{0.5pt}\end{center}

\subsection*{પ્રશ્ન 5(a OR) [3
ગુણ]}\label{uxaaauxab0uxab6uxaa8-5a-or-3-uxa97uxaa3}

\textbf{ટ્રાન્ઝેક્શનની ACID પ્રોપર્ટીસ સમજાવો.}

\begin{solutionbox}

{\def\LTcaptype{none} % do not increment counter
\begin{longtable}[]{@{}ll@{}}
\toprule\noalign{}
પ્રોપર્ટી & વિવરણ \\
\midrule\noalign{}
\endhead
\bottomrule\noalign{}
\endlastfoot
\textbf{Atomicity} & ટ્રાન્ઝેક્શન all-or-nothing છે \\
\textbf{Consistency} & ડેટાબેઝ વેલિડ સ્ટેટમાં રહે છે \\
\textbf{Isolation} & સમાંતર ટ્રાન્ઝેક્શન્સ દખલ કરતા નથી \\
\textbf{Durability} & કમિટ થયેલા ફેરફારો કાયમી છે \\
\end{longtable}
}

\end{solutionbox}
\begin{mnemonicbox}
``ACID - All Changes In Database''

\end{mnemonicbox}
\begin{center}\rule{0.5\linewidth}{0.5pt}\end{center}

\subsection*{પ્રશ્ન 5(b OR) [4
ગુણ]}\label{uxaaauxab0uxab6uxaa8-5b-or-4-uxa97uxaa3}

\textbf{Triggers ની વ્યાખ્યા આપો? ટ્રિગર્સના ફાયદા સમજાવો.}

\begin{solutionbox}

\textbf{Triggers}: ડેટાબેઝ ઇવેન્ટ્સના જવાબમાં આપોઆપ એક્ઝિક્યુટ થતી વિશેષ સ્ટોર્ડ
પ્રોસીજર્સ.

{\def\LTcaptype{none} % do not increment counter
\begin{longtable}[]{@{}ll@{}}
\toprule\noalign{}
ફાયદો & વિવરણ \\
\midrule\noalign{}
\endhead
\bottomrule\noalign{}
\endlastfoot
\textbf{આપોઆપ એક્ઝિક્યુશન} & સ્પષ્ટ કૉલ વિના ચાલે છે \\
\textbf{ડેટા ઇન્ટેગ્રિટી} & બિઝનેસ રૂલ્સ લાગુ કરે છે \\
\textbf{ઓડિટિંગ} & ડેટાબેઝ ફેરફારોને ટ્રેક કરે છે \\
\textbf{સિક્યોરિટી} & ડેટા એક્સેસ કંટ્રોલ કરે છે \\
\end{longtable}
}

\end{solutionbox}
\begin{mnemonicbox}
``ADAS - Automatic Data Auditing Security''

\end{mnemonicbox}
\begin{center}\rule{0.5\linewidth}{0.5pt}\end{center}

\subsection*{પ્રશ્ન 5(c OR) [7
ગુણ]}\label{uxaaauxab0uxab6uxaa8-5c-or-7-uxa97uxaa3}

\textbf{Problems of concurrency control ની યાદી બનાવો. કોઈપણ બેના યોગ્ય
ઉદાહરણ સાથે સમજાવો.}

\begin{solutionbox}

\textbf{Concurrency Control ની સમસ્યાઓ}:

{\def\LTcaptype{none} % do not increment counter
\begin{longtable}[]{@{}ll@{}}
\toprule\noalign{}
સમસ્યા & વિવરણ \\
\midrule\noalign{}
\endhead
\bottomrule\noalign{}
\endlastfoot
\textbf{Lost Update} & એક ટ્રાન્ઝેક્શનનું અપડેટ બીજાના દ્વારા ઓવરરાઇટ થાય છે \\
\textbf{Dirty Read} & અનકમિટેડ ડેટા વાંચવો \\
\textbf{Non-repeatable Read} & સમાન ટ્રાન્ઝેક્શનમાં વિવિધ મૂલ્યો વાંચવા \\
\textbf{Phantom Read} & રીડ્સ વચ્ચે નવી રો દેખાય છે \\
\end{longtable}
}

\textbf{ઉદાહરણ 1 - Lost Update}:

\begin{verbatim}
T1: Read(A=100)
T2: Read(A=100)  
T1:

A = A + 50 (A=150)

T2:

A = A + 30 (A=130) <- T1 નું અપડેટ ગુમ

T1: Write(A=150)
T2: Write(A=130) <- અંતિમ મૂલ્ય ખોટું
\end{verbatim}

\textbf{ઉદાહરણ 2 - Dirty Read}:

\begin{verbatim}
T1: Write(A=200) [કમિટ નથી]
T2: Read(A=200)  <- ડર્ટી રીડ
T1: Rollback     <- A પાછું મૂળ મૂલ્યે
T2: ખોટા મૂલ્ય સાથે ચાલુ રાખે છે
\end{verbatim}

\end{solutionbox}
\begin{mnemonicbox}
``LDNP - Lost Dirty Non-repeatable Phantom''

\end{mnemonicbox}

\end{document}
