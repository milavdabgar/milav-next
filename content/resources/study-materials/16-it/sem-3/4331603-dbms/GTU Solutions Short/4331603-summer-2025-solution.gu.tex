\documentclass{article}
% Adjust the relative path to point to the latex-templates directory

% content/resources/templates/preamble.tex
\usepackage[margin=0.6in]{geometry}
\author{Milav Dabgar}
\usepackage{amsmath,amssymb,amsthm}
\usepackage{booktabs}
\usepackage{multirow}
\usepackage{xcolor}
\usepackage{tcolorbox}
\tcbuselibrary{breakable,skins}
\usepackage[colorlinks=true,linkcolor=blue]{hyperref}
\usepackage{titlesec}
\usepackage{enumitem}
\usepackage{tikz}
\usepackage{pgfplots}
\usepackage{circuitikz}
\usepackage[version=4]{mhchem}
\usepackage{longtable}
\usepackage{array}
\usepackage{float}
\usepackage{caption}
\usepackage{listings}

\lstset{
  basicstyle=\small\ttfamily,
  breaklines=true,
  breakatwhitespace=false,
  postbreak=\mbox{\textcolor{red}{$\hookrightarrow$}\space},
  float=false,
  numbers=left,
  numberstyle=\tiny\color{gray},
  numbersep=10pt,
  xleftmargin=2em,
  keywordstyle=\color{blue},
  commentstyle=\color{green!60!black},
  stringstyle=\color{purple},
  backgroundcolor=\color{gray!5},
  showstringspaces=false,
  tabsize=2,
  captionpos=b,
  keepspaces=true,
  columns=flexible
}

\pgfplotsset{compat=1.18}
\usetikzlibrary{shapes,arrows,positioning,calc,patterns,decorations.pathmorphing,decorations.markings,arrows.meta}

% Color scheme
\definecolor{headcolor}{RGB}{0,102,204}
\definecolor{keycolor}{RGB}{220,20,60}
\definecolor{solutioncolor}{RGB}{34,139,34}
\definecolor{mnemoniccolor}{RGB}{148,0,211}
\definecolor{codecolor}{RGB}{0,0,100}

% Spacing
\setlength{\parskip}{3pt}
\setlist[itemize]{nosep}
\setlist[enumerate]{nosep}

% Title formatting
\titleformat{\section}{\Large\bfseries\color{headcolor}}{\thesection}{1em}{}
\titleformat{\subsection}{\large\bfseries\color{headcolor}}{\thesubsection}{1em}{}

% Pandoc tightlist compatibility
\providecommand{\tightlist}{%
  \setlength{\itemsep}{0pt}\setlength{\parskip}{0pt}}

% Pandoc longtable compatibility
\newcounter{none}
\def\thenone{}


% content/resources/templates/gujarati-boxes.tex
\usepackage{fontspec}
\usepackage{polyglossia}

% Set Gujarati as main language (document is primarily in Gujarati)
% Note: gloss-gujarati.ldf doesn't exist in polyglossia, but it will use hyphenation patterns
\setdefaultlanguage{gujarati}
\setotherlanguage{english}

% Configure Gujarati font properly
% Use Language=Default to prevent polyglossia from trying to add language-specific features
% that don't exist for Gujarati, which causes "empty feature" warnings
\newfontfamily\gujaratifont[Script=Gujarati,AutoFakeBold=2.5,AutoFakeSlant=0.3]{Noto Sans Gujarati}
\setmainfont[Script=Gujarati,AutoFakeBold=2.5,AutoFakeSlant=0.3]{Noto Sans Gujarati}
% Use Noto Sans Gujarati for monospace to support Gujarati in text
\setmonofont[Scale=0.9]{Noto Sans Gujarati}

% Configure English to use the same font
\newfontfamily\englishfont[Script=Gujarati,AutoFakeBold=2.5,AutoFakeSlant=0.3]{Noto Sans Gujarati}

% Translations for polyglossia
\gappto\captionsgujarati{
  \renewcommand{\tablename}{કોષ્ટક}
  \renewcommand{\figurename}{આકૃતિ}
}

% Helper for TikZ nodes to ensure Gujarati font
\newcommand{\gu}[1]{{\gujaratifont #1}}

% Custom environments
\newtcolorbox{solutionbox}{
    breakable,
    enhanced,
    colback=solutioncolor!5!white,
    colframe=solutioncolor!75!black,
    fonttitle=\bfseries,
    title=જવાબ
}

\newtcolorbox{solutionboxnobreak}{
 colback=solutioncolor!5!white,
 colframe=solutioncolor!75!black,
 fonttitle=\bfseries,
 title=જવાબ
}

\newtcolorbox{keyformula}{
 breakable,
 enhanced,
 colback=keycolor!5!white,
 colframe=keycolor!75!black,
 fonttitle=\bfseries,
 title=રાસાયણિક સમીકરણ/સૂત્ર
}

\newtcolorbox{mnemonicbox}{
 breakable,
 enhanced,
 colback=mnemoniccolor!5!white,
 colframe=mnemoniccolor!75!black,
 fonttitle=\bfseries,
 title=મેમરી ટ્રીક
}


% Custom commands for GTU solutions
% This file defines semantic commands for consistent formatting

% Question command with automatic formatting
\newcommand{\question}[2]{%
  \section*{Question #1}%
  \textbf{#2}%
}

% OR question variant
\newcommand{\questionor}[2]{%
  \section*{Question #1 OR}%
  \textbf{#2}%
}

% Proper table environment with caption
\newenvironment{answertable}[1]{%
  \begin{table}[htbp]
  \centering
  \caption{#1}
}{%
  \end{table}
}

% Proper figure environment for diagrams
\newenvironment{answerdiagram}[1]{%
  \begin{figure}[htbp]
  \centering
  \caption{#1}
}{%
  \end{figure}
}

% Semantic markup for key terms
\newcommand{\keyword}[1]{\textbf{#1}}
\newcommand{\code}[1]{\texttt{#1}}
\newcommand{\classname}[1]{\texttt{#1}}
\newcommand{\methodname}[1]{\texttt{#1}}

% Proper quotation marks
\newcommand{\mnemonic}[1]{``#1''}


\title{ડેટાબેઝ મેનેજમેન્ટ (4331603) - ઉનાળો 2025 ઉકેલ}
\date{May 15, 2025}

\begin{document}
\maketitle

\questionmarks{1(a)}{3}{નીચેના શબ્દોની વ્યાખ્યા આપો. 1) મેટાડેટા 2) સ્કીમા 3) ડેટા ડિક્શનરી.}

\begin{solutionbox}
\textbf{ટેબલ:}
\begin{center}
\captionof{table}{ડેટાબેઝ શબ્દો}
\begin{tabulary}{\linewidth}{|L|L|}
\hline
\textbf{શબ્દ} & \textbf{વ્યાખ્યા} \\ \hline
\textbf{મેટાડેટા} & ડેટા વિશેનો ડેટા જે ડેટાબેઝની રચના અને વિશેષતાઓ વર્ણવે છે \\ \hline
\textbf{સ્કીમા} & ડેટાબેઝના સંગઠન અને સંબંધોને દર્શાવતી તાર્કિક રચના \\ \hline
\textbf{ડેટા ડિક્શનરી} & ડેટાબેઝના તત્વો વિશેની માહિતી સંગ્રહિત કરતો કેન્દ્રીય ભંડાર \\ \hline
\end{tabulary}
\end{center}

\begin{itemize}
    \item \keyword{મેટાડેટા}: ડેટાની લાક્ષણિકતાઓ અને ગુણધર્મો વર્ણવતી માહિતી
    \item \keyword{સ્કીમા}: ડેટાબેઝની રચના અને મર્યાદાઓ વ્યાખ્યાયિત કરતો બ્લુપ્રિન્ટ
    \item \keyword{ડેટા ડિક્શનરી}: બધા ડેટાબેઝ ઓબ્જેક્ટ્સ અને તેમના ગુણધર્મોની કેટલોગ
\end{itemize}
\end{solutionbox}

\begin{mnemonicbox}
\mnemonic{MSD - My System Dictionary}
\end{mnemonicbox}

\questionmarks{1(b)}{4}{ડેટાબેઝ મેનેજમેન્ટ સિસ્ટમના ફાયદા લખો.}

\begin{solutionbox}
\textbf{ટેબલ:}
\begin{center}
\captionof{table}{DBMS ના ફાયદા}
\begin{tabulary}{\linewidth}{|L|L|}
\hline
\textbf{ફાયદો} & \textbf{વર્ણન} \\ \hline
\textbf{ડેટા સ્વતંત્રતા} & એપ્લિકેશન્સ ડેટા સ્ટોરેજથી સ્વતંત્ર \\ \hline
\textbf{ડેટા અખંડિતતા} & ડેટાની ચોકસાઈ અને સુસંગતતા જાળવે છે \\ \hline
\textbf{સુરક્ષા નિયંત્રણ} & વપરાશકર્તા પ્રમાણીકરણ અને અધિકરણ \\ \hline
\textbf{સમવર્તી પહોંચ} & અનેક વપરાશકર્તાઓ એકસાથે પહોંચ કરી શકે છે \\ \hline
\end{tabulary}
\end{center}

\begin{itemize}
    \item \keyword{ઘટેલી રીડન્ડન્સી}: ડુપ્લિકેટ ડેટા સ્ટોરેજ દૂર કરે છે
    \item \keyword{કેન્દ્રીકૃત નિયંત્રણ}: ડેટા મેનેજમેન્ટનું એક જ બિંદુ
    \item \keyword{ડેટા વહેંચણી}: અનેક એપ્લિકેશન્સ સમાન ડેટાનો ઉપયોગ કરી શકે છે
    \item \keyword{બેકઅપ પુનઃપ્રાપ્તિ}: આપોઆપ ડેટા સુરક્ષા પદ્ધતિઓ
\end{itemize}
\end{solutionbox}

\begin{mnemonicbox}
\mnemonic{DISC-RCDB - Database Is Super Cool}
\end{mnemonicbox}

\questionmarks{1(c)}{7}{DBA ની જવાબદારીઓ સમજાવો.}

\begin{solutionbox}
\textbf{ટેબલ:}
\begin{center}
\captionof{table}{DBA ની જવાબદારીઓ}
\begin{tabulary}{\linewidth}{|L|L|}
\hline
\textbf{જવાબદારી} & \textbf{કાર્યો} \\ \hline
\textbf{ડેટાબેઝ ડિઝાઈન} & તાર્કિક અને ભૌતિક રચનાઓ બનાવવી \\ \hline
\textbf{સુરક્ષા મેનેજમેન્ટ} & વપરાશકર્તા પહોંચ અને પરવાનગીઓનું નિયંત્રણ \\ \hline
\textbf{પર્ફોર્મન્સ ટ્યુનિંગ} & ક્વેરીઝ અને ડેટાબેઝ ઓપરેશન્સને ઓપ્ટિમાઈઝ કરવા \\ \hline
\textbf{બેકઅપ પુનઃપ્રાપ્તિ} & ડેટા સુરક્ષા અને પુનઃસ્થાપન સુનિશ્ચિત કરવું \\ \hline
\textbf{યુઝર મેનેજમેન્ટ} & એકાઉન્ટ બનાવવા અને વિશેષાધિકારો અસાઇન કરવા \\ \hline
\end{tabulary}
\end{center}

\begin{center}
\begin{tikzpicture}[node distance=1.5cm, auto, thick]
    \node [gtu root] (Root) {DBA જવાબદારીઓ};
    
    \node [gtu child, below left=1.5cm and 1cm of Root] (Sec) {સુરક્ષા મેનેજમેન્ટ};
    \node [gtu child, left=0.5cm of Sec] (Des) {ડેટાબેઝ ડિઝાઈન};
    
    \node [gtu child, below right=1.5cm and 1cm of Root] (Back) {બેકઅપ અને પુનઃપ્રાપ્તિ};
    \node [gtu child, right=0.5cm of Back] (User) {યુઝર મેનેજમેન્ટ};
    
    \node [gtu child, below=1.5cm of Root] (Perf) {પર્ફોર્મન્સ ટ્યુનિંગ};
    
    \draw [gtu arrow] (Root) -- (Des);
    \draw [gtu arrow] (Root) -- (Sec);
    \draw [gtu arrow] (Root) -- (Perf);
    \draw [gtu arrow] (Root) -- (Back);
    \draw [gtu arrow] (Root) -- (User);
\end{tikzpicture}
\captionof{figure}{DBA ની મુખ્ય જવાબદારીઓ}
\end{center}

\begin{itemize}
    \item \keyword{ડેટાબેઝ ઇન્સ્ટલેશન}: DBMS સોફ્ટવેર સેટઅપ અને કોન્ફિગર કરવું
    \item \keyword{ડેટા માઇગ્રેશન}: સિસ્ટમ્સ વચ્ચે ડેટાને સુરક્ષિત રીતે ટ્રાન્સફર કરવો
    \item \keyword{ડોક્યુમેન્ટેશન}: ડેટાબેઝ સ્કીમા અને પ્રક્રિયાઓ જાળવવી
    \item \keyword{મોનિટરિંગ}: સિસ્ટમ પર્ફોર્મન્સ અને રિસોર્સ વપરાશ ટ્રેક કરવું
    \item \keyword{ટ્રબલશૂટિંગ}: ડેટાબેઝ સમસ્યાઓ અને ભૂલો ઉકેલવી
\end{itemize}
\end{solutionbox}

\begin{mnemonicbox}
\mnemonic{DSPBU-DMT - DBA Solves Problems By Understanding Database Management Tasks}
\end{mnemonicbox}

\questionmarks{1(c OR)}{7}{ડેટા એબ્સ્ટ્રેક્શન શું છે? ત્રણ સ્તરની ANSI SPARC આર્કિટેક્ચરને વિગતવાર સમજાવો.}

\begin{solutionbox}
\textbf{ડેટા એબ્સ્ટ્રેક્શન}: વપરાશકર્તાઓથી જટિલ ડેટાબેઝ અમલીકરણ વિગતો છુપાવીને સરળ ઇન્ટરફેસ પ્રદાન કરવું.

\begin{center}
\begin{tikzpicture}[node distance=1.5cm, auto, thick]
    % Levels
    \node [gtu block, fill=red!10, minimum width=6cm] (Ext) {એક્સટર્નલ લેવલ (યુઝર વ્યૂઝ)};
    \node [gtu block, fill=blue!10, minimum width=6cm, below=1cm of Ext] (Con) {કોન્સેપ્ચુઅલ લેવલ (લોજિકલ સ્કીમા)};
    \node [gtu block, fill=green!10, minimum width=6cm, below=1cm of Con] (Int) {ઇન્ટર્નલ લેવલ (ફિઝિકલ સ્ટોરેજ)};
    
    % Users
    \node [above=0.8cm of Ext] (Users) {એન્ડ યુઝર્સ / એપ્લિકેશન્સ};
    
    % Connections
    \draw [gtu arrow] (Users) -- (Ext);
    \draw [gtu arrow] (Ext) -- (Con);
    \draw [gtu arrow] (Con) -- (Int);
    
    % Labels (Mapping)
    \node [right=0.5cm of Ext, font=\small] {એક્સટર્નલ-કોન્સેપ્ચુઅલ મેપિંગ};
    \node [right=0.5cm of Con, font=\small] {કોન્સેપ્ચુઅલ-ઇન્ટર્નલ મેપિંગ};
\end{tikzpicture}
\captionof{figure}{ત્રણ સ્તરની ANSI SPARC આર્કિટેક્ચર}
\end{center}

\textbf{ટેબલ:}
\begin{center}
\captionof{table}{આર્કિટેક્ચર સ્તરો}
\begin{tabulary}{\linewidth}{|L|L|L|}
\hline
\textbf{સ્તર} & \textbf{વર્ણન} & \textbf{વપરાશકર્તાઓ} \\ \hline
\textbf{એક્સટર્નલ લેવલ} & વ્યક્તિગત વપરાશકર્તા દૃશ્યો અને એપ્લિકેશન્સ & એન્ડ યુઝર્સ \\ \hline
\textbf{કોન્સેપ્ચુઅલ લેવલ} & સંપૂર્ણ તાર્કિક ડેટાબેઝ રચના & ડેટાબેઝ ડિઝાઇનર્સ \\ \hline
\textbf{ઇન્ટર્નલ લેવલ} & ભૌતિક સ્ટોરેજ અને એક્સેસ પદ્ધતિઓ & સિસ્ટમ પ્રોગ્રામર્સ \\ \hline
\end{tabulary}
\end{center}

\begin{itemize}
    \item \keyword{એક્સટર્નલ લેવલ}: જટિલતા છુપાવતા અનેક વપરાશકર્તા દૃશ્યો
    \item \keyword{કોન્સેપ્ચુઅલ લેવલ}: સ્ટોરેજ વિગતો વિના સંપૂર્ણ ડેટાબેઝ સ્કીમા
    \item \keyword{ઇન્ટર્નલ લેવલ}: ભૌતિક ફાઇલ સંગઠન અને ઇન્ડેક્સિંગ
    \item \keyword{ડેટા સ્વતંત્રતા}: એક સ્તરમાં ફેરફારો અન્યને અસર કરતા નથી
\end{itemize}
\end{solutionbox}

\begin{mnemonicbox}
\mnemonic{ECI - Every Computer Implements}
\end{mnemonicbox}

\questionmarks{2(a)}{3}{સ્કીમા અને ઇન્સ્ટન્સનો તફાવત સમજાવો}

\begin{solutionbox}
\textbf{ટેબલ:}
\begin{center}
\captionof{table}{સ્કીમા vs ઇન્સ્ટન્સ}
\begin{tabulary}{\linewidth}{|L|L|L|}
\hline
\textbf{પાસું} & \textbf{સ્કીમા} & \textbf{ઇન્સ્ટન્સ} \\ \hline
\textbf{વ્યાખ્યા} & ડેટાબેઝ રચનાનો બ્લુપ્રિન્ટ & ચોક્કસ સમયે વાસ્તવિક ડેટા \\ \hline
\textbf{પ્રકૃતિ} & સ્થિર તાર્કિક ડિઝાઇન & ડાયનામિક ડેટા સામગ્રી \\ \hline
\textbf{ફેરફારો} & ભાગ્યે જ સંશોધિત & વારંવાર અપડેટ \\ \hline
\end{tabulary}
\end{center}

\begin{itemize}
    \item \keyword{સ્કીમા}: ડેટાબેઝ સંગઠન અને મર્યાદાઓ વર્ણવે છે
    \item \keyword{ઇન્સ્ટન્સ}: ચોક્કસ ક્ષણે ડેટાબેઝ સામગ્રીનો સ્નેપશોટ
    \item \keyword{સંબંધ}: સ્કીમા રચના વ્યાખ્યાયિત કરે છે, ઇન્સ્ટન્સ ડેટા સમાવે છે
\end{itemize}
\end{solutionbox}

\begin{mnemonicbox}
\mnemonic{SI - Structure vs Information}
\end{mnemonicbox}

\questionmarks{2(b)}{4}{સ્પેશ્યલાઈઝેશન ઉદાહરણ સાથે સમજાવો.}

\begin{solutionbox}
\textbf{સ્પેશ્યલાઈઝેશન}: ચોક્કસ લાક્ષણિકતાઓના આધારે સુપરક્લાસમાંથી સબક્લાસ બનાવવાની પ્રક્રિયા.

\begin{center}
\begin{tikzpicture}[node distance=2cm, auto, thick]
    % Entities
    \node [gtu block] (Emp) {EMPLOYEE};
    
    \node [gtu decision, below=1cm of Emp] (IsA) {IS-A};
    
    \node [gtu block, below left=1.5cm and 1cm of IsA] (Mgr) {MANAGER};
    \node [gtu block, below right=1.5cm and 1cm of IsA] (Dev) {DEVELOPER};
    
    % Connections
    \draw [thick] (Emp) -- (IsA);
    \draw [thick] (IsA) -| (Mgr);
    \draw [thick] (IsA) -| (Dev);
    
    % Attributes (Sample)
    \node [ellipse, draw, right=0.5cm of Emp, font=\footnotesize] {Salary};
    \node [ellipse, draw, left=0.5cm of Mgr, font=\footnotesize] {TeamSize};
    \node [ellipse, draw, right=0.5cm of Dev, font=\footnotesize] {Project};
\end{tikzpicture}
\captionof{figure}{સ્પેશ્યલાઈઝેશન હાયરાર્કી}
\end{center}

\begin{itemize}
    \item \keyword{ટોપ-ડાઉન અપ્રોચ}: સામાન્ય એન્ટિટીથી ચોક્કસ એન્ટિટીઓ તરફ
    \item \keyword{ઇન્હેરિટન્સ}: સબક્લાસેસ સુપરક્લાસના ગુણધર્મો વારસામાં લે છે
    \item \keyword{ડિસજોઇન્ટ}: મેનેજર અને ડેવલપર અલગ કેટેગરી છે
    \item \keyword{ઉદાહરણ}: એમ્પ્લોયી મેનેજર અને ડેવલપરમાં વિશેષીકૃત
\end{itemize}
\end{solutionbox}

\begin{mnemonicbox}
\mnemonic{STID - Specialization Takes Inheritance Down}
\end{mnemonicbox}

\questionmarks{2(c)}{7}{ER ડાયાગ્રામ શું છે? ER ડાયાગ્રામમાં વપરાતા વિવિધ પ્રતીકોને ઉદાહરણ સાથે સમજાવો.}

\begin{solutionbox}
\textbf{ER ડાયાગ્રામ}: ડેટાબેઝ ડિઝાઇનમાં એન્ટિટીઝ, એટ્રિબ્યુટ્સ અને સંબંધો દર્શાવતી ગ્રાફિકલ પ્રતિનિધિત્વ.

\textbf{ટેબલ:}
\begin{center}
\captionof{table}{ER ડાયાગ્રામ પ્રતીકો}
\begin{tabulary}{\linewidth}{|L|L|L|L|}
\hline
\textbf{પ્રતીક} & \textbf{આકાર} & \textbf{હેતુ} & \textbf{ઉદાહરણ} \\ \hline
\textbf{એન્ટિટી} & લંબચોરસ & વાસ્તવિક વિશ્વનો ઓબ્જેક્ટ & Student, Course \\ \hline
\textbf{એટ્રિબ્યુટ} & અંડાકાર & એન્ટિટીના ગુણધર્મો & Name, Age, ID \\ \hline
\textbf{સંબંધ} & હીરો & એન્ટિટી કનેક્શન્સ & Enrolls, Takes \\ \hline
\textbf{પ્રાઇમરી કી} & અન્ડરલાઇન અંડાકાર & યુનિક આઇડેન્ટિફાયર & Student\_ID \\ \hline
\end{tabulary}
\end{center}

\begin{center}
\begin{tikzpicture}[node distance=2cm, auto, thick]
    % Entity: Student
    \node [gtu block] (Student) {Student};
    
    % Attributes
    \node [ellipse, draw, above left=1cm of Student] (Sid) {\underline{student\_id}};
    \node [ellipse, draw, above=1cm of Student] (Name) {name};
    \node [ellipse, draw, above right=1cm of Student] (Email) {email};
    
    % Entity: Course
    \node [gtu block, right=4cm of Student] (Course) {Course};
    \node [ellipse, draw, above=1cm of Course] (Cid) {\underline{course\_id}};
    
    % Relationship
    \node [gtu decision, between=Student and Course] (Enrolls) {Enrolls};
    
    % Edges
    \draw (Student) -- (Sid);
    \draw (Student) -- (Name);
    \draw (Student) -- (Email);
    \draw (Course) -- (Cid);
    
    \draw (Student) -- node[above] {M} (Enrolls);
    \draw (Enrolls) -- node[above] {N} (Course);
\end{tikzpicture}
\captionof{figure}{ER ડાયાગ્રામ ઉદાહરણ}
\end{center}

\begin{itemize}
    \item \keyword{એન્ટિટી સેટ્સ}: સમાન ગુણધર્મો ધરાવતી સમાન એન્ટિટીઝનો સંગ્રહ
    \item \keyword{વીક એન્ટિટી}: ઓળખ માટે સ્ટ્રોંગ એન્ટિટી પર આધારિત
    \item \keyword{કાર્ડિનાલિટી}: સંબંધ સહભાગિતા વ્યાખ્યાયિત કરે છે (1:1, 1:M, M:N)
    \item \keyword{પાર્ટિસિપેશન}: ટોટલ (ડબલ લાઇન) અથવા પાર્શિયલ (સિંગલ લાઇન)
\end{itemize}
\end{solutionbox}

\begin{mnemonicbox}
\mnemonic{EARP - Entities And Relationships Program}
\end{mnemonicbox}

\questionmarks{2(a OR)}{3}{DA અને DBA નો તફાવત સમજાવો.}

\begin{solutionbox}
\textbf{ટેબલ:}
\begin{center}
\captionof{table}{DA vs DBA}
\begin{tabulary}{\linewidth}{|L|L|L|}
\hline
\textbf{પાસું} & \textbf{ડેટા એડમિનિસ્ટ્રેટર (DA)} & \textbf{ડેટાબેઝ એડમિનિસ્ટ્રેટર (DBA)} \\ \hline
\textbf{ફોકસ} & ડેટા પોલિસીઝ અને સ્ટાન્ડર્ડ્સ & તકનીકી ડેટાબેઝ ઓપરેશન્સ \\ \hline
\textbf{સ્તર} & વ્યૂહાત્મક આયોજન & ઓપરેશનલ અમલીકરણ \\ \hline
\textbf{સ્કોપ} & સંસ્થા-વ્યાપી ડેટા & ચોક્કસ ડેટાબેઝ સિસ્ટમ્સ \\ \hline
\end{tabulary}
\end{center}

\begin{itemize}
    \item \keyword{DA}: સંસ્થાકીય સંસાધન તરીકે ડેટાનું સંચાલન કરે છે
    \item \keyword{DBA}: તકનીકી ડેટાબેઝ જાળવણી અને પર્ફોર્મન્સ સંભાળે છે
    \item \keyword{સહયોગ}: DA નીતિઓ સેટ કરે છે, DBA તેમને અમલમાં મૂકે છે
\end{itemize}
\end{solutionbox}

\begin{mnemonicbox}
\mnemonic{DA-DBA: Design Authority - Database Builder Administrator}
\end{mnemonicbox}

\questionmarks{2(b OR)}{4}{જનરલાઈઝેશન ઉદાહરણ સાથે સમજાવો.}

\begin{solutionbox}
\textbf{જનરલાઈઝેશન}: સમાન એન્ટિટીઝને સામાન્ય સુપરક્લાસમાં જોડવાની બોટમ-અપ પ્રક્રિયા.

\begin{center}
\begin{tikzpicture}[node distance=2cm, auto, thick]
    % Superclass
    \node [gtu block] (Vehicle) {VEHICLE};
    
    % Relationship
    \node [gtu decision, below=1cm of Vehicle] (IsA) {IS-A};
    
    % Subclasses
    \node [gtu block, below left=1.5cm and 1cm of IsA] (Car) {CAR};
    \node [gtu block, below right=1.5cm and 1cm of IsA] (Bike) {MOTORCYCLE};
    
    % Connections
    \draw [thick] (Vehicle) -- (IsA);
    \draw [thick] (IsA) -| (Car);
    \draw [thick] (IsA) -| (Bike);
    
    % Attributes
    \node [ellipse, draw, right=0.5cm of Vehicle, font=\footnotesize] {Brand};
    \node [ellipse, draw, left=0.5cm of Car, font=\footnotesize] {Doors};
    \node [ellipse, draw, right=0.5cm of Bike, font=\footnotesize] {EngineCC};
\end{tikzpicture}
\captionof{figure}{જનરલાઈઝેશન હાયરાર્કી}
\end{center}

\begin{itemize}
    \item \keyword{બોટમ-અપ અપ્રોચ}: ચોક્કસ એન્ટિટીઝથી સામાન્ય એન્ટિટી તરફ
    \item \keyword{કોમન એટ્રિબ્યુટ્સ}: સહેજ ગુણધર્મો સુપરક્લાસમાં ખસેડાય છે
    \item \keyword{સ્પેશ્યલાઈઝેશન રિવર્સ}: સ્પેશ્યલાઈઝેશન પ્રક્રિયાનું વિપરીત
    \item \keyword{ઉદાહરણ}: કાર અને મોટરસાઇકલ વાહનમાં સામાન્યીકૃત
\end{itemize}
\end{solutionbox}

\begin{mnemonicbox}
\mnemonic{GBCS - Generalization Brings Common Superclass}
\end{mnemonicbox}

\questionmarks{2(c OR)}{7}{એટ્રિબ્યુટ શું છે? વિવિધ પ્રકારના એટ્રિબ્યુટ્સ ઉદાહરણ સાથે સમજાવો.}

\begin{solutionbox}
\textbf{એટ્રિબ્યુટ}: એન્ટિટીનું વર્ણન કરતી ગુણવત્તા અથવા લાક્ષણિકતા.

\textbf{ટેબલ:}
\begin{center}
\captionof{table}{એટ્રિબ્યુટ પ્રકારો}
\begin{tabulary}{\linewidth}{|L|L|L|}
\hline
\textbf{એટ્રિબ્યુટ પ્રકાર} & \textbf{વર્ણન} & \textbf{ઉદાહરણ} \\ \hline
\textbf{સિમ્પલ} & વધુ વિભાજિત કરી શકાતું નથી & Age, Name \\ \hline
\textbf{કોમ્પોઝિટ} & ઉપવિભાગ કરી શકાય છે & Address (Street, City, ZIP) \\ \hline
\textbf{સિંગલ-વેલ્યુડ} & એન્ટિટી દીઠ એક મૂલ્ય & Student\_ID \\ \hline
\textbf{મલ્ટિ-વેલ્યુડ} & અનેક મૂલ્યો શક્ય & Phone\_numbers \\ \hline
\textbf{ડેરાઇવ્ડ} & અન્ય એટ્રિબ્યુટ્સમાંથી ગણાય છે & Age from Birth\_date \\ \hline
\end{tabulary}
\end{center}

\begin{center}
\begin{tikzpicture}[level/.style={sibling distance=25mm, level distance=20mm}, auto, thick]
    \node [gtu root] {એટ્રિબ્યુટ્સ}
        child {node [gtu child] {સિમ્પલ}
             child {node [gtu state, fill=white] {Name}}
        }
        child {node [gtu child] {કોમ્પોઝિટ}
             child {node [gtu state, fill=white] {Address}
                child {node [gtu state, fill=white] {City}}
                child {node [gtu state, fill=white] {ZIP}}
             }
        }
        child {node [gtu child] {મલ્ટિ-વેલ્યુડ}
             child {node [gtu state, double, fill=white] {Phone}}
        }
        child {node [gtu child] {ડેરાઇવ્ડ}
             child {node [gtu state, dashed, fill=white] {Age}}
        };
\end{tikzpicture}
\captionof{figure}{એટ્રિબ્યુટ્સના પ્રકાર}
\end{center}

\begin{itemize}
    \item \keyword{કી એટ્રિબ્યુટ}: એન્ટિટી ઇન્સ્ટન્સેસને યુનિકલી ઓળખે છે
    \item \keyword{નલ વેલ્યુઝ}: એટ્રિબ્યુટ્સ કે જેમાં કોઈ મૂલ્ય ન હોઈ શકે
    \item \keyword{ડિફોલ્ટ વેલ્યુઝ}: નિર્દિષ્ટ ન હોય ત્યારે પૂર્વનિર્ધારિત મૂલ્યો
    \item \keyword{કન્સ્ટ્રેઇન્ટ્સ}: એટ્રિબ્યુટ મૂલ્યોને સંચાલિત કરતા નિયમો
\end{itemize}
\end{solutionbox}

\begin{mnemonicbox}
\mnemonic{SCSMD-K - Simple Composite Single Multi Derived Key}
\end{mnemonicbox}

\questionmarks{3(a)}{3}{SQL માં GRANT અને REVOKE સ્ટેટમેન્ટ સમજાવો.}

\begin{solutionbox}
\textbf{ટેબલ:}
\begin{center}
\captionof{table}{GRANT અને REVOKE}
\begin{tabulary}{\linewidth}{|L|L|L|}
\hline
\textbf{સ્ટેટમેન્ટ} & \textbf{હેતુ} & \textbf{સિન્ટેક્સ ઉદાહરણ} \\ \hline
\textbf{GRANT} & વપરાશકર્તાઓને વિશેષાધિકારો પ્રદાન કરે છે & GRANT SELECT ON table TO user \\ \hline
\textbf{REVOKE} & વપરાશકર્તા વિશેષાધિકારો દૂર કરે છે & REVOKE INSERT ON table FROM user \\ \hline
\end{tabulary}
\end{center}

\begin{lstlisting}[language=SQL]
-- Grant privileges
GRANT SELECT, INSERT ON employees TO john;
GRANT ALL PRIVILEGES ON database TO admin;

-- Revoke privileges  
REVOKE DELETE ON employees FROM john;
REVOKE ALL ON database FROM user;
\end{lstlisting}

\begin{itemize}
    \item \keyword{વિશેષાધિકારો}: SELECT, INSERT, UPDATE, DELETE, ALL
    \item \keyword{ઓબ્જેક્ટ્સ}: ટેબલ્સ, વ્યૂઝ, ડેટાબેઝિસ, પ્રોસીજર્સ
    \item \keyword{સુરક્ષા}: ડેટા એક્સેસ અને મોડિફિકેશન રાઇટ્સનું નિયંત્રણ
\end{itemize}
\end{solutionbox}

\begin{mnemonicbox}
\mnemonic{GR - Grant Rights, Remove Rights}
\end{mnemonicbox}

\questionmarks{3(b)}{4}{નીચેના Character functions સમજાવો. 1) INITCAP 2) SUBSTR}

\begin{solutionbox}
\textbf{ટેબલ:}
\begin{center}
\captionof{table}{Character ફંક્શન્સ}
\begin{tabulary}{\linewidth}{|L|L|L|L|}
\hline
\textbf{ફંક્શન} & \textbf{હેતુ} & \textbf{સિન્ટેક્સ} & \textbf{ઉદાહરણ} \\ \hline
\textbf{INITCAP} & દરેક શબ્દનો પહેલો અક્ષર મોટો કરે છે & INITCAP(string) & INITCAP('hello world') = 'Hello World' \\ \hline
\textbf{SUBSTR} & સ્ટ્રિંગમાંથી સબસ્ટ્રિંગ કાઢે છે & SUBSTR(string, start, length) & SUBSTR('Database', 1, 4) = 'Data' \\ \hline
\end{tabulary}
\end{center}

\begin{lstlisting}[language=SQL]
-- INITCAP examples
SELECT INITCAP('database management') FROM dual; -- Database Management
SELECT INITCAP('gtu university') FROM dual; -- Gtu University

-- SUBSTR examples  
SELECT SUBSTR('Programming', 1, 7) FROM dual; -- Program
SELECT SUBSTR('Database', 5) FROM dual; -- base
\end{lstlisting}

\begin{itemize}
    \item \keyword{INITCAP}: સ્ટ્રિંગને યોગ્ય કેસ ફોર્મેટમાં કન્વર્ટ કરે છે
    \item \keyword{SUBSTR}: પેરામીટર્સ છે સ્ટ્રિંગ, શરૂઆતની સ્થિતિ, વૈકલ્પિક લંબાઈ
    \item \keyword{વપરાશ}: ટેક્સ્ટ ફોર્મેટિંગ અને સ્ટ્રિંગ મેનિપ્યુલેશન ઓપરેશન્સ
\end{itemize}
\end{solutionbox}

\begin{mnemonicbox}
\mnemonic{IS - Initialize String, Split String}
\end{mnemonicbox}

\questionmarks{3(c)}{7}{નીચે દશાવેલ ટેબલને ધ્યાનમાં લઈ આપેલ ક્વેરીઝના જવાબ લખો. stud\_master (enroll\_no, name, city, dept)}

\begin{solutionbox}
\begin{lstlisting}[language=SQL]
-- 1. IT dept માં અભ્યાસ કરતા બધા વિદ્યાર્થીઓની વિગતો દર્શાવો
SELECT * FROM stud_master 
WHERE dept = 'IT';

-- 2. 'p' થી શરૂ થતા નામ વિશેની બધી માહિતી મેળવો
SELECT * FROM stud_master 
WHERE name LIKE 'p%';

-- 3. ટેબલમાં નવો વિદ્યાર્થી દાખલ કરો
INSERT INTO stud_master (enroll_no, name, city, dept) 
VALUES ('202501', 'John Smith', 'Mumbai', 'CS');

-- 4. stud_master ટેબલમાં gender નામનું નવું કૉલમ ઉમેરો
ALTER TABLE stud_master 
ADD gender VARCHAR(10);

-- 5. stud_master ટેબલની પંક્તિઓની સંખ્યા ગણો
SELECT COUNT(*) FROM stud_master;

-- 6. enroll_no ના અવરોહી ક્રમમાં બધી વિદ્યાર્થી વિગતો દર્શાવો
SELECT * FROM stud_master 
ORDER BY enroll_no DESC;

-- 7. ડેટા સાથે stud_master ટેબલનો નાશ કરો
DROP TABLE stud_master;
\end{lstlisting}

\textbf{ટેબલ:}
\begin{center}
\captionof{table}{SQL ક્વેરીઝ}
\begin{tabulary}{\linewidth}{|L|L|L|}
\hline
\textbf{ક્વેરી પ્રકાર} & \textbf{SQL કમાન્ડ} & \textbf{હેતુ} \\ \hline
\textbf{SELECT} & ડેટા મેળવે છે & રેકોર્ડ્સ દર્શાવે છે \\ \hline
\textbf{INSERT} & નવો ડેટા ઉમેરે છે & રેકોર્ડ્સ બનાવે છે \\ \hline
\textbf{ALTER} & રચના સંશોધિત કરે છે & કૉલમ્સ ઉમેરે છે \\ \hline
\textbf{COUNT} & એગ્રિગેટ ફંક્શન & પંક્તિઓ ગણે છે \\ \hline
\end{tabulary}
\end{center}
\end{solutionbox}

\begin{mnemonicbox}
\mnemonic{SIAC-DOC - SQL Is A Complete Database Operations Collection}
\end{mnemonicbox}

\questionmarks{3(a OR)}{3}{SQL માં equi join ઉદાહરણ સાથે સમજાવો.}

\begin{solutionbox}
\textbf{Equi Join}: સમાન કૉલમ્સના આધારે ટેબલ્સને જોડવા માટે સમાનતા શરતનો ઉપયોગ કરતું જોઇન ઓપરેશન.

\begin{lstlisting}[language=SQL]
-- Equi Join ઉદાહરણ
SELECT s.name, c.course_name
FROM students s, courses c
WHERE s.course_id = c.course_id;

-- JOIN સિન્ટેક્સનો ઉપયોગ
SELECT s.name, c.course_name  
FROM students s
JOIN courses c ON s.course_id = c.course_id;
\end{lstlisting}

\begin{itemize}
    \item \keyword{સમાનતા ઓપરેટર}: કૉલમ મૂલ્યો મેચ કરવા માટે = નો ઉપયોગ
    \item \keyword{કોમન કૉલમ્સ}: ટેબલ્સમાં સંબંધિત એટ્રિબ્યુટ્સ હોવા જોઈએ
    \item \keyword{પરિણામ}: મેચના આધારે અનેક ટેબલ્સમાંથી સંયુક્ત ડેટા
\end{itemize}
\end{solutionbox}

\begin{mnemonicbox}
\mnemonic{EJ - Equal Join}
\end{mnemonicbox}

\questionmarks{3(b OR)}{4}{નીચેના Aggregate functions સમજાવો. 1) MAX 2) SUM}

\begin{solutionbox}
\textbf{ટેબલ:}
\begin{center}
\captionof{table}{Aggregate ફંક્શન્સ}
\begin{tabulary}{\linewidth}{|L|L|L|L|}
\hline
\textbf{ફંક્શન} & \textbf{હેતુ} & \textbf{સિન્ટેક્સ} & \textbf{ઉદાહરણ} \\ \hline
\textbf{MAX} & મહત્તમ મૂલ્ય પરત કરે છે & MAX(column) & MAX(salary) = 50000 \\ \hline
\textbf{SUM} & મૂલ્યોનો કુલ સરવાળો પરત કરે છે & SUM(column) & SUM(marks) = 450 \\ \hline
\end{tabulary}
\end{center}

\begin{lstlisting}[language=SQL]
-- MAX ઉદાહરણો
SELECT MAX(salary) FROM employees; -- સૌથી વધુ પગાર
SELECT MAX(age) FROM students; -- સૌથી જૂના વિદ્યાર્થીની ઉંમર

-- SUM ઉદાહરણો
SELECT SUM(credits) FROM courses; -- કુલ ક્રેડિટ્સ
SELECT SUM(price * quantity) FROM orders; -- કુલ ઓર્ડર મૂલ્ય
\end{lstlisting}

\begin{itemize}
    \item \keyword{એગ્રિગેટ ફંક્શન્સ}: અનેક પંક્તિઓ પર કામ કરે છે, એક મૂલ્ય પરત કરે છે
    \item \keyword{NULL હેન્ડલિંગ}: ગણતરીમાં NULL મૂલ્યોને અવગણે છે
    \item \keyword{GROUP BY}: કેટેગરી-વાઇઝ પરિણામો માટે ગ્રુપિંગ સાથે ઉપયોગ કરી શકાય છે
\end{itemize}
\end{solutionbox}

\begin{mnemonicbox}
\mnemonic{MS - Maximum Sum}
\end{mnemonicbox}

\questionmarks{3(c OR)}{7}{નીચે દશાવેલ ટેબલ માટે SQL ક્વેરીઝ લખો: PRODUCT\_Master: (prod\_no, prod\_name, profit, quantity, sell\_price, cost\_price)}

\begin{solutionbox}
\begin{lstlisting}[language=SQL]
-- 1. PRODUCT_Master ટેબલ બનાવો
CREATE TABLE PRODUCT_Master (
    prod_no VARCHAR(10) PRIMARY KEY,
    prod_name VARCHAR(50),
    profit NUMBER(10,2),
    quantity NUMBER,
    sell_price NUMBER(10,2),
    cost_price NUMBER(10,2)
);

-- 2. આ ટેબલમાં એક રેકોર્ડ દાખલ કરો
INSERT INTO PRODUCT_Master VALUES 
('P001', 'Laptop', 15000, 10, 45000, 30000);

-- 3. 20000 થી વધુ નફો ધરાવતા પ્રોડક્ટ શોધો
SELECT * FROM PRODUCT_Master 
WHERE profit > 20000;

-- 4. 5 થી ઓછી quantity ધરાવતા પ્રોડક્ટ ડિલીટ કરો
DELETE FROM PRODUCT_Master 
WHERE quantity < 5;

-- 5. 5000 થી વધુ sell_price ધરાવતા પ્રોડક્ટમાં 2% નફો ઉમેરો
UPDATE PRODUCT_Master 
SET profit = profit * 1.02 
WHERE sell_price > 5000;

-- 6. PRODUCT_Master માં total_price નામનું નવું ફીલ્ડ ઉમેરો
ALTER TABLE PRODUCT_Master 
ADD total_price NUMBER(10,2);

-- 7. કોઈ ડુપ્લિકેટ ડેટા વગર પ્રોડક્ટ નામ શોધો
SELECT DISTINCT prod_name FROM PRODUCT_Master;
\end{lstlisting}
\end{solutionbox}

\begin{mnemonicbox}
\mnemonic{CIDFAUD - Create Insert Delete Find Add Update Distinct}
\end{mnemonicbox}

\questionmarks{4(a)}{3}{fully functional dependency ઉદાહરણ સાથે સમજાવો.}

\begin{solutionbox}
\textbf{Fully Functional Dependency}: એટ્રિબ્યુટ સંપૂર્ણ રીતે ફંક્શનલ ડિપેન્ડન્ટ છે જો તે સંપૂર્ણ પ્રાઇમરી કી પર આધારિત હોય, આંશિક કી પર નહીં.

\textbf{ટેબલ:}
\begin{center}
\captionof{table}{ડિપેન્ડન્સી પ્રકાર}
\begin{tabulary}{\linewidth}{|L|L|L|}
\hline
\textbf{ડિપેન્ડન્સી પ્રકાર} & \textbf{વ્યાખ્યા} & \textbf{ઉદાહરણ} \\ \hline
\textbf{Full FD} & સંપૂર્ણ કી પર આધારિત & (Student\_ID, Course\_ID) $\rightarrow$ Grade \\ \hline
\textbf{Partial FD} & કીના ભાગ પર આધારિત & (Student\_ID, Course\_ID) $\rightarrow$ Student\_Name \\ \hline
\end{tabulary}
\end{center}

\begin{example}
ઉદાહરણ: Student\_Course(Student\_ID, Course\_ID, Student\_Name, Grade)

Full FD: (Student\_ID, Course\_ID) $\rightarrow$ Grade
Partial FD: Student\_ID $\rightarrow$ Student\_Name
\end{example}

\begin{itemize}
    \item \keyword{સંપૂર્ણ કી}: કોમ્પોઝિટ પ્રાઇમરી કીના બધા એટ્રિબ્યુટ્સ જરૂરી
    \item \keyword{નોન-કી એટ્રિબ્યુટ}: સંપૂર્ણ પ્રાઇમરી કી કોમ્બિનેશન પર આધારિત
    \item \keyword{2NF જરૂરિયાત}: આંશિક ડિપેન્ડન્સીઝ દૂર કરે છે
\end{itemize}
\end{solutionbox}

\begin{mnemonicbox}
\mnemonic{FFD - Full Function Dependency}
\end{mnemonicbox}

\questionmarks{4(b)}{4}{નીચે દશાવેલ રિલેશનલ સ્કીમાને ધ્યાનમાં લઈ રિલેશનલ અલજીબ્રા એક્સપ્રેશન આપો: Employee (Emp\_name, Emp\_id, birth\_date, Post, salary)}

\begin{solutionbox}
\begin{enumerate}
    \item Post="Clerk" ધરાવતા બધા કર્મચારીઓની યાદી બનાવો
    \begin{center}
    $\sigma_{Post='Clerk'}(Employee)$
    \end{center}
    
    \item salary > 2000 અને post='Manager' ધરાવતા Emp\_id અને Emp\_name શોધો
    \begin{center}
    $\pi_{Emp\_id, Emp\_name}(\sigma_{salary>2000 \land Post='Manager'}(Employee))$
    \end{center}
\end{enumerate}

\textbf{ટેબલ:}
\begin{center}
\captionof{table}{રિલેશનલ અલજીબ્રા પ્રતીકો}
\begin{tabulary}{\linewidth}{|C|L|L|}
\hline
\textbf{પ્રતીક} & \textbf{ઓપરેશન} & \textbf{હેતુ} \\ \hline
$\sigma$ & સિલેક્શન & શરત આધારિત પંક્તિઓ ફિલ્ટર કરે છે \\ \hline
$\pi$ & પ્રોજેક્શન & ચોક્કસ કૉલમ્સ પસંદ કરે છે \\ \hline
$\land$ & AND & તાર્કિક સંયોજન \\ \hline
\end{tabulary}
\end{center}

\begin{itemize}
    \item \keyword{સિલેક્શન ($\sigma$)}: નિર્દિષ્ટ શરતો પૂરી કરતી પંક્તિઓ પસંદ કરે છે
    \item \keyword{પ્રોજેક્શન ($\pi$)}: પરિણામમાંથી જરૂરી કૉલમ્સ પસંદ કરે છે
    \item \keyword{સંયુક્ત ઓપરેશન્સ}: અનેક ઓપરેશન્સનો એકસાથે ઉપયોગ કરી શકાય છે
\end{itemize}
\end{solutionbox}

\begin{mnemonicbox}
\mnemonic{SPA - Select Project And}
\end{mnemonicbox}

\questionmarks{4(c)}{7}{2NF ના ક્રાઇટેરિયા શું છે? આપેલ રિલેશનમાં વિવિધ ફંક્શનલ ડિપેન્ડન્સીઝ શોધો અને તેને 2NF માં નોર્મલાઈઝ કરો.}

\begin{solutionbox}
\textbf{2NF ક્રાઇટેરિયા}:
\begin{itemize}
    \item 1NF માં હોવું જોઈએ
    \item પ્રાઇમરી કી પર કોઈ આંશિક ફંક્શનલ ડિપેન્ડન્સીઝ ન હોવી જોઈએ
\end{itemize}

\textbf{આપેલ ટેબલ}: Student\_Course(Student\_ID, Course\_ID, Student\_Name, Course\_Name)

\textbf{ફંક્શનલ ડિપેન્ડન્સીઝ}:
\begin{itemize}
    \item Student\_ID $\rightarrow$ Student\_Name (Partial FD)
    \item Course\_ID $\rightarrow$ Course\_Name (Partial FD)
    \item (Student\_ID, Course\_ID) $\rightarrow$ (Student\_Name, Course\_Name) (Full FD)
\end{itemize}

\textbf{2NF નોર્મલાઈઝેશન}:

\begin{center}
\begin{tikzpicture}[node distance=2cm, auto, thick]
    % Tables
    \node [gtu block] (Students) {Students\\(Student\_ID, Student\_Name)};
    \node [gtu block, right=4cm of Students] (Courses) {Courses\\(Course\_ID, Course\_Name)};
    \node [gtu block, below right=1.5cm and 1cm of Students] (Enrollment) {Enrollment\\(Student\_ID, Course\_ID)};
    
    % Relationships
    \draw [gtu arrow] (Enrollment) -| node[near start, below] {FK} (Students);
    \draw [gtu arrow] (Enrollment) -| node[near start, below] {FK} (Courses);
\end{tikzpicture}
\captionof{figure}{2NF ડિકમ્પોઝિશન}
\end{center}

\begin{lstlisting}[language=SQL]
-- ટેબલ 1: Students
Students(Student_ID, Student_Name)

-- ટેબલ 2: Courses  
Courses(Course_ID, Course_Name)

-- ટેબલ 3: Enrollment
Enrollment(Student_ID, Course_ID)
\end{lstlisting}
\end{solutionbox}

\begin{mnemonicbox}
\mnemonic{2NF - Two Normal Form removes partial dependencies}
\end{mnemonicbox}

\questionmarks{4(a OR)}{3}{3NF ઉદાહરણ સાથે સમજાવો.}

\begin{solutionbox}
\textbf{3NF (Third Normal Form)}: 2NF માં હોય અને પ્રાઇમરી કી પર કોઈ ટ્રાન્ઝિટિવ ડિપેન્ડન્સીઝ ન હોય તેવું ટેબલ.

\textbf{ટેબલ:}
\begin{center}
\captionof{table}{3NF જરૂરિયાત}
\begin{tabulary}{\linewidth}{|L|L|L|}
\hline
\textbf{નોર્મલ ફોર્મ} & \textbf{જરૂરિયાત} & \textbf{નાબૂદ કરે છે} \\ \hline
\textbf{3NF} & 2NF માં + કોઈ ટ્રાન્ઝિટિવ ડિપેન્ડન્સીઝ નહીં & ટ્રાન્ઝિટિવ FD \\ \hline
\end{tabulary}
\end{center}

\begin{example}
ઉદાહરણ: Employee(Emp\_ID, Dept\_ID, Dept\_Name)

ટ્રાન્ઝિટિવ ડિપેન્ડન્સી: Emp\_ID $\rightarrow$ Dept\_ID $\rightarrow$ Dept\_Name

3NF ઉકેલ:
Employee(Emp\_ID, Dept\_ID)
Department(Dept\_ID, Dept\_Name)
\end{example}

\begin{itemize}
    \item \keyword{ટ્રાન્ઝિટિવ ડિપેન્ડન્સી}: A $\rightarrow$ B $\rightarrow$ C જ્યાં A પ્રાઇમરી કી છે
    \item \keyword{નોન-કી ટુ નોન-કી}: નોન-કી એટ્રિબ્યુટ્સ વચ્ચે ડિપેન્ડન્સી
    \item \keyword{ડિકમ્પોઝિશન}: ટ્રાન્ઝિટિવ ડિપેન્ડન્સીઝ દૂર કરવા માટે ટેબલ વિભાજિત કરવું
\end{itemize}
\end{solutionbox}

\begin{mnemonicbox}
\mnemonic{3NF - Third Normal Form removes Transitive dependencies}
\end{mnemonicbox}

\questionmarks{4(b OR)}{4}{નીચે દશાવેલ રિલેશનલ સ્કીમાને ધ્યાનમાં લઈ રિલેશનલ અલજીબ્રા એક્સપ્રેશન આપો: Students (Name, SPI, DOB, Enrollment No)}

\begin{solutionbox}
\begin{enumerate}
    \item SPI 7.0 થી વધુ હોય તેવા બધા વિદ્યાર્થીઓની યાદી બનાવો
    \begin{center}
    $\sigma_{SPI > 7.0}(Students)$
    \end{center}
    
    \item enrollment number 007 હોય તેવા વિદ્યાર્થીનું name, DOB દર્શાવો
    \begin{center}
    $\pi_{Name, DOB}(\sigma_{Enrollment\_No = '007'}(Students))$
    \end{center}
\end{enumerate}

\textbf{ટેબલ:}
\begin{center}
\captionof{table}{રિલેશનલ અલજીબ્રા}
\begin{tabulary}{\linewidth}{|L|L|L|}
\hline
\textbf{ક્વેરી} & \textbf{રિલેશનલ અલજીબ્રા} & \textbf{હેતુ} \\ \hline
\textbf{ફિલ્ટર} & $\sigma(condition)$ & પંક્તિઓ પસંદ કરે છે \\ \hline
\textbf{પ્રોજેક્ટ} & $\pi(attributes)$ & કૉલમ્સ પસંદ કરે છે \\ \hline
\end{tabulary}
\end{center}

\begin{itemize}
    \item \keyword{પહેલા સિલેક્શન}: પ્રોજેક્શન પહેલાં શરતો લાગુ કરો
    \item \keyword{ચોક્કસ મૂલ્ય}: સ્ટ્રિંગ લિટરલ્સ માટે ક્વોટ્સનો ઉપયોગ કરો
    \item \keyword{કૉલમ નામો}: ચોક્કસ એટ્રિબ્યુટ નામો જરૂરી
\end{itemize}
\end{solutionbox}

\begin{mnemonicbox}
\mnemonic{SPI-DOB: Select Project Information - Display Output Better}
\end{mnemonicbox}

\questionmarks{4(c OR)}{7}{1NF ના ક્રાઇટેરિયા શું છે? આપેલ ટેબલને બે અલગ અલગ પદ્ધતિથી 1NF માં નોર્મલાઇઝ કરો.}

\begin{solutionbox}
\textbf{1NF ક્રાઇટેરિયા}:
\begin{itemize}
    \item દરેક સેલમાં એક જ અણુ મૂલ્ય હોવું જોઈએ
    \item કોઈ પુનરાવર્તિત જૂથો અથવા એરેઝ નહીં
    \item દરેક પંક્તિ અનન્ય હોવી જોઈએ
\end{itemize}

\textbf{આપેલ ટેબલ}:
\begin{center}
\begin{tabulary}{\linewidth}{|L|L|L|}
\hline
\textbf{EnrollmentNo} & \textbf{Name} & \textbf{Subjects} \\ \hline
001 & DEF & Maths,Physics,Chemistry \\ \hline
002 & XYZ & History,Biology,English \\ \hline
\end{tabulary}
\end{center}

\textbf{પદ્ધતિ 1 - અલગ પંક્તિઓ}:
\begin{center}
\captionof{table}{1NF - અલગ પંક્તિઓ}
\begin{tabulary}{\linewidth}{|L|L|L|}
\hline
\textbf{EnrollmentNo} & \textbf{Name} & \textbf{Subject} \\ \hline
001 & DEF & Maths \\ \hline
001 & DEF & Physics \\ \hline
001 & DEF & Chemistry \\ \hline
002 & XYZ & History \\ \hline
002 & XYZ & Biology \\ \hline
002 & XYZ & English \\ \hline
\end{tabulary}
\end{center}

\textbf{પદ્ધતિ 2 - અલગ ટેબલ્સ}:
\begin{lstlisting}[language=SQL]
-- Students ટેબલ
Students(EnrollmentNo, Name)

-- Subjects ટેબલ  
Subjects(SubjectID, SubjectName)

-- Student_Subjects ટેબલ
Student_Subjects(EnrollmentNo, SubjectID)
\end{lstlisting}
\end{solutionbox}

\begin{mnemonicbox}
\mnemonic{1NF - One Normal Form creates Atomic values}
\end{mnemonicbox}

\questionmarks{5(a)}{3}{ટ્રાન્ઝેક્શનની ACID પ્રોપર્ટીઝ સમજાવો.}

\begin{solutionbox}
\textbf{ટેબલ:}
\begin{center}
\captionof{table}{ACID પ્રોપર્ટીઝ}
\begin{tabulary}{\linewidth}{|L|L|L|}
\hline
\textbf{પ્રોપર્ટી} & \textbf{વર્ણન} & \textbf{હેતુ} \\ \hline
\textbf{Atomicity} & સંપૂર્ણ અથવા કંઈ જ એક્ઝિક્યુશન & ટ્રાન્ઝેક્શન સંપૂર્ણતા \\ \hline
\textbf{Consistency} & ડેટાબેઝ માન્ય રહે છે & ડેટા અખંડિતતા \\ \hline
\textbf{Isolation} & સમવર્તી ટ્રાન્ઝેક્શન્સ સ્વતંત્ર & હસ્તક્ષેપ ટાળવો \\ \hline
\textbf{Durability} & કમિટ થયેલા ફેરફારો કાયમી & ડેટા સ્થિરતા \\ \hline
\end{tabulary}
\end{center}

\begin{itemize}
    \item \keyword{Atomicity}: ટ્રાન્ઝેક્શન સંપૂર્ણ રીતે એક્ઝિક્યુટ થાય અથવા બિલકુલ નહીં
    \item \keyword{Consistency}: ટ્રાન્ઝેક્શન પહેલાં/પછી ડેટાબેઝ કન્સ્ટ્રેઇન્ટ્સ જાળવાય છે
    \item \keyword{Isolation}: ટ્રાન્ઝેક્શન્સ એકબીજા સાથે હસ્તક્ષેપ કરતા નથી
    \item \keyword{Durability}: એકવાર કમિટ થયા પછી, ફેરફારો સિસ્ટમ ફેઇલ્યુર્સમાં ટકી રહે છે
\end{itemize}
\end{solutionbox}

\begin{mnemonicbox}
\mnemonic{ACID - All Consistent Isolated Durable}
\end{mnemonicbox}

\questionmarks{5(b)}{4}{નીચે દશાવેલ સ્પેસિફિકેશન મુજબ ટેબલ બનાવો: STUDENT: (stu\_id, stu\_name, Address, City, contact\_no, Branch\_name)}

\begin{solutionbox}
\begin{lstlisting}[language=SQL]
CREATE TABLE STUDENT (
    stu_id VARCHAR(10) PRIMARY KEY,
    stu_name VARCHAR(50) NOT NULL,
    Address VARCHAR(100),
    City VARCHAR(30),
    contact_no NUMBER(10),
    Branch_name VARCHAR(20) CHECK (Branch_name IN ('IT', 'Computer', 'Electrical', 'Civil'))
);
\end{lstlisting}

\textbf{ટેબલ:}
\begin{center}
\captionof{table}{ટેબલ કન્સ્ટ્રેઇન્ટ્સ}
\begin{tabulary}{\linewidth}{|L|L|L|}
\hline
\textbf{કન્સ્ટ્રેઇન્ટ} & \textbf{હેતુ} & \textbf{અમલીકરણ} \\ \hline
\textbf{NOT NULL} & ફરજિયાત ફીલ્ડ & stu\_name NOT NULL \\ \hline
\textbf{CHECK} & વેલ્યુ વેલિડેશન & Branch\_name IN (...) \\ \hline
\end{tabulary}
\end{center}

\begin{itemize}
    \item \keyword{પ્રાઇમરી કી}: stu\_id દરેક વિદ્યાર્થીને અનન્ય રીતે ઓળખે છે
    \item \keyword{NOT NULL}: stu\_name ખાલી હોઈ શકે નહીં
    \item \keyword{CHECK કન્સ્ટ્રેઇન્ટ}: Branch\_name નિર્દિષ્ટ મૂલ્યો સુધી મર્યાદિત
    \item \keyword{ડેટા ટાઇપ્સ}: દરેક ફીલ્ડ માટે યોગ્ય સાઇઝ
\end{itemize}
\end{solutionbox}

\begin{mnemonicbox}
\mnemonic{CNPD - Constraints Names Primary Datatypes}
\end{mnemonicbox}

\questionmarks{5(c)}{7}{ટ્રિગર શું છે? Oracle માં ટ્રિગર બનાવવા માટે સિન્ટેક્સ લખો. સિમ્પલ ટ્રિગર બનાવો.}

\begin{solutionbox}
\textbf{ટ્રિગર}: વિશેષ સ્ટોર્ડ પ્રોસીજર જે ડેટાબેઝ ઇવેન્ટ્સના પ્રતિભાવમાં આપોઆપ એક્ઝિક્યુટ થાય છે.

\textbf{Oracle ટ્રિગર સિન્ટેક્સ}:
\begin{lstlisting}[language=SQL]
CREATE [OR REPLACE] TRIGGER trigger_name
{BEFORE | AFTER | INSTEAD OF} {INSERT | UPDATE | DELETE}
ON table_name
[FOR EACH ROW]
[WHEN condition]
DECLARE
    -- Variable declarations
BEGIN
    -- Trigger logic
END;
\end{lstlisting}

\textbf{સિમ્પલ ટ્રિગર ઉદાહરણ}:
\begin{lstlisting}[language=SQL]
CREATE OR REPLACE TRIGGER display_student_trigger
BEFORE INSERT ON STUDENT
FOR EACH ROW
BEGIN
    DBMS_OUTPUT.PUT_LINE('Inserting student: ' || :NEW.stu_name || 
                        ' with ID: ' || :NEW.stu_id);
END;
\end{lstlisting}

\textbf{ટેબલ:}
\begin{center}
\captionof{table}{ટ્રિગર પ્રકારો}
\begin{tabulary}{\linewidth}{|L|L|L|}
\hline
\textbf{ટ્રિગર પ્રકાર} & \textbf{ક્યારે એક્ઝિક્યુટ થાય} & \textbf{હેતુ} \\ \hline
\textbf{BEFORE} & DML ઓપરેશન પહેલાં & વેલિડેશન, મોડિફિકેશન \\ \hline
\textbf{AFTER} & DML ઓપરેશન પછી & લોગિંગ, ઓડિટિંગ \\ \hline
\textbf{FOR EACH ROW} & રો-લેવલ ટ્રિગર & પ્રતિ પંક્તિ એક્ઝિક્યુશન \\ \hline
\end{tabulary}
\end{center}

\begin{itemize}
    \item \keyword{:NEW}: દાખલ/અપડેટ કરવામાં આવતા નવા મૂલ્યોનો સંદર્ભ
    \item \keyword{:OLD}: ડિલીટ/અપડેટ કરવામાં આવતા જૂના મૂલ્યોનો સંદર્ભ
    \item \keyword{આપોઆપ એક્ઝિક્યુશન}: નિર્દિષ્ષ્ટ ઇવેન્ટ્સ પર આપોઆપ ફાયર થાય છે
    \item \keyword{બિઝનેસ લોજિક}: જટિલ બિઝનેસ નિયમો લાગુ કરે છે
\end{itemize}
\end{solutionbox}

\begin{mnemonicbox}
\mnemonic{TBA-FEN - Triggers Before After For Each New}
\end{mnemonicbox}

\questionmarks{5(a OR)}{3}{ટ્રાન્ઝેક્શનમાં કન્કરન્સી કંટ્રોલના પ્રોબ્લેમ્સ સમજાવો.}

\begin{solutionbox}
\textbf{ટેબલ:}
\begin{center}
\captionof{table}{કન્કરન્સી કંટ્રોલ સમસ્યાઓ}
\begin{tabulary}{\linewidth}{|L|L|L|}
\hline
\textbf{સમસ્યા} & \textbf{વર્ણન} & \textbf{ઉદાહરણ} \\ \hline
\textbf{Lost Update} & એક ટ્રાન્ઝેક્શન બીજાના ફેરફારો પર લખે છે & T1, T2 સમાન રેકોર્ડ અપડેટ કરે છે \\ \hline
\textbf{Dirty Read} & અનકમિટ ડેટા વાંચવો & T1 T2 ના અનકમિટ ફેરફારો વાંચે છે \\ \hline
\textbf{Unrepeatable Read} & સમાન ક્વેરી અલગ પરિણામો આપે છે & T1 વાંચે, T2 અપડેટ કરે, T1 ફરી વાંચે \\ \hline
\end{tabulary}
\end{center}

\begin{itemize}
    \item \keyword{Phantom Read}: સમાન ટ્રાન્ઝેક્શનમાં ક્વેરીઝ વચ્ચે નવી પંક્તિઓ દેખાય છે
    \item \keyword{Deadlock}: બે ટ્રાન્ઝેક્શન્સ એકબીજાના લોક્સની રાહ જુએ છે
    \item \keyword{Inconsistent Analysis}: ડેટા સંશોધિત થતો હોય ત્યારે વાંચવો
\end{itemize}
\end{solutionbox}

\begin{mnemonicbox}
\mnemonic{LDU-PID - Lost Dirty Unrepeatable Phantom Inconsistent Deadlock}
\end{mnemonicbox}

\questionmarks{5(b OR)}{4}{નીચે દશાવેલ સ્પેસિફિકેશન મુજબ ટેબલ બનાવો: STUDENT: (stu\_id, stu\_name, Address, City, contact\_no, Branch\_name)}

\begin{solutionbox}
\begin{lstlisting}[language=SQL]
CREATE TABLE STUDENT (
    stu_id VARCHAR(10) PRIMARY KEY CHECK (stu_id LIKE 'S%'),
    stu_name VARCHAR(50),
    Address VARCHAR(100),
    City VARCHAR(30),
    contact_no NUMBER(10),
    Branch_name VARCHAR(20)
);
\end{lstlisting}

\textbf{ટેબલ:}
\begin{center}
\captionof{table}{ટેબલ કન્સ્ટ્રેઇન્ટ્સ}
\begin{tabulary}{\linewidth}{|L|L|L|}
\hline
\textbf{કન્સ્ટ્રેઇન્ટ} & \textbf{અમલીકરણ} & \textbf{હેતુ} \\ \hline
\textbf{PRIMARY KEY} & stu\_id PRIMARY KEY & અનન્ય ઓળખ \\ \hline
\textbf{CHECK} & stu\_id LIKE 'S\%' & 'S' થી શરૂ થવું જોઈએ \\ \hline
\end{tabulary}
\end{center}

\begin{itemize}
    \item \keyword{પ્રાઇમરી કી}: stu\_id અનન્ય આઇડેન્ટિફાયર તરીકે કામ કરે છે
    \item \keyword{પેટર્ન ચેક}: stu\_id અક્ષર 'S' થી શરૂ થવું જોઈએ
    \item \keyword{ડેટા ટાઇપ્સ}: યોગ્ય ફીલ્ડ સાઇઝ અને ટાઇપ્સ
    \item \keyword{કન્સ્ટ્રેઇન્ટ વેલિડેશન}: ડેટાબેઝ આપોઆપ નિયમો લાગુ કરે છે
\end{itemize}
\end{solutionbox}

\begin{mnemonicbox}
\mnemonic{PKC-ST - Primary Key Check Starts}
\end{mnemonicbox}

\questionmarks{5(c OR)}{7}{એક્સપ્લિસિટ કર્સર શું છે? એક્સપ્લિસિટ કર્સર ઉદાહરણ સાથે સમજાવો.}

\begin{solutionbox}
\textbf{એક્સપ્લિસિટ કર્સર}: અનેક પંક્તિઓ પરત કરતા SELECT સ્ટેટમેન્ટ્સ હેન્ડલ કરવા માટે પ્રોગ્રામેટિક કંટ્રોલ સાથે વપરાશકર્તા-વ્યાખ્યાયિત કર્સર.

\textbf{કર્સર ઓપરેશન્સ}:

\begin{center}
\begin{tikzpicture}[node distance=1.5cm, auto, thick]
    \node [gtu state] (Dec) {DECLARE Cursor};
    \node [gtu state, right=1.5cm of Dec] (Open) {OPEN Cursor};
    \node [gtu state, right=1.5cm of Open] (Fetch) {FETCH Data};
    \node [gtu decision, display, below=1cm of Fetch] (Check) {વધુ Rows?};
    \node [gtu state, left=1.5cm of Check] (Close) {CLOSE Cursor};
    
    \draw [gtu arrow] (Dec) -- (Open);
    \draw [gtu arrow] (Open) -- (Fetch);
    \draw [gtu arrow] (Fetch) -- (Check);
    \draw [gtu arrow] (Check) -- node[right] {હા} (Fetch.south);
    \draw [gtu arrow] (Check) -- node[above] {ના} (Close);
\end{tikzpicture}
\captionof{figure}{એક્સપ્લિસિટ કર્સર લાઈફસાયકલ}
\end{center}

\begin{lstlisting}[language=SQL]
-- ડિક્લેરેશન
DECLARE
    CURSOR student_cursor IS
        SELECT stu_id, stu_name FROM STUDENT WHERE city = 'Ahmedabad';
    v_id STUDENT.stu_id%TYPE;
    v_name STUDENT.stu_name%TYPE;
BEGIN
    -- કર્સર ઓપન કરો
    OPEN student_cursor;
    
    -- ડેટા ફેચ કરો
    LOOP
        FETCH student_cursor INTO v_id, v_name;
        EXIT WHEN student_cursor%NOTFOUND;
        
        DBMS_OUTPUT.PUT_LINE('ID: ' || v_id || ', Name: ' || v_name);
    END LOOP;
    
    -- કર્સર બંધ કરો
    CLOSE student_cursor;
END;
\end{lstlisting}

\textbf{ટેબલ:}
\begin{center}
\captionof{table}{કર્સર કમાન્ડ્સ}
\begin{tabulary}{\linewidth}{|L|L|L|}
\hline
\textbf{ઓપરેશન} & \textbf{હેતુ} & \textbf{સિન્ટેક્સ} \\ \hline
\textbf{DECLARE} & કર્સર ડિફાઇન કરવું & CURSOR name IS SELECT... \\ \hline
\textbf{OPEN} & કર્સર ઇનિશિયલાઇઝ કરવું & OPEN cursor\_name \\ \hline
\textbf{FETCH} & ડેટા મેળવવો & FETCH cursor INTO variables \\ \hline
\textbf{CLOSE} & રિસોર્સ છોડવા & CLOSE cursor\_name \\ \hline
\end{tabulary}
\end{center}

\begin{itemize}
    \item \keyword{મેન્યુઅલ કંટ્રોલ}: પ્રોગ્રામર કર્સર ઓપરેશન્સને નિયંત્રિત કરે છે
    \item \keyword{મેમરી મેનેજમેન્ટ}: સ્પષ્ટ રીતે ઓપન અને ક્લોઝ કરવું જોઈએ
    \item \keyword{લૂપ પ્રોસેસિંગ}: સામાન્ય રીતે અનેક પંક્તિઓ માટે લૂપ્સ સાથે ઉપયોગ થાય છે
    \item \keyword{કર્સર એટ્રિબ્યુટ્સ}: \%FOUND, \%NOTFOUND, \%ROWCOUNT
\end{itemize}
\end{solutionbox}

\begin{mnemonicbox}
\mnemonic{DOFC - Declare Open Fetch Close}
\end{mnemonicbox}

\end{document}
