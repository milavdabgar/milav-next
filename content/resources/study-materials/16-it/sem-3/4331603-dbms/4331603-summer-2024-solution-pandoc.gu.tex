\documentclass[10pt,a4paper]{article}

% content/resources/templates/preamble.tex
\usepackage[margin=0.6in]{geometry}
\author{Milav Dabgar}
\usepackage{amsmath,amssymb,amsthm}
\usepackage{booktabs}
\usepackage{multirow}
\usepackage{xcolor}
\usepackage{tcolorbox}
\tcbuselibrary{breakable,skins}
\usepackage[colorlinks=true,linkcolor=blue]{hyperref}
\usepackage{titlesec}
\usepackage{enumitem}
\usepackage{tikz}
\usepackage{pgfplots}
\usepackage{circuitikz}
\usepackage[version=4]{mhchem}
\usepackage{longtable}
\usepackage{array}
\usepackage{float}
\usepackage{caption}
\usepackage{listings}

\lstset{
  basicstyle=\small\ttfamily,
  breaklines=true,
  breakatwhitespace=false,
  postbreak=\mbox{\textcolor{red}{$\hookrightarrow$}\space},
  float=false,
  numbers=left,
  numberstyle=\tiny\color{gray},
  numbersep=10pt,
  xleftmargin=2em,
  keywordstyle=\color{blue},
  commentstyle=\color{green!60!black},
  stringstyle=\color{purple},
  backgroundcolor=\color{gray!5},
  showstringspaces=false,
  tabsize=2,
  captionpos=b,
  keepspaces=true,
  columns=flexible
}

\pgfplotsset{compat=1.18}
\usetikzlibrary{shapes,arrows,positioning,calc,patterns,decorations.pathmorphing,decorations.markings,arrows.meta}

% Color scheme
\definecolor{headcolor}{RGB}{0,102,204}
\definecolor{keycolor}{RGB}{220,20,60}
\definecolor{solutioncolor}{RGB}{34,139,34}
\definecolor{mnemoniccolor}{RGB}{148,0,211}
\definecolor{codecolor}{RGB}{0,0,100}

% Spacing
\setlength{\parskip}{3pt}
\setlist[itemize]{nosep}
\setlist[enumerate]{nosep}

% Title formatting
\titleformat{\section}{\Large\bfseries\color{headcolor}}{\thesection}{1em}{}
\titleformat{\subsection}{\large\bfseries\color{headcolor}}{\thesubsection}{1em}{}

% Pandoc tightlist compatibility
\providecommand{\tightlist}{%
  \setlength{\itemsep}{0pt}\setlength{\parskip}{0pt}}

% Pandoc longtable compatibility
\newcounter{none}
\def\thenone{}


% content/resources/templates/gujarati-boxes.tex
\usepackage{fontspec}
\usepackage{polyglossia}

% Set Gujarati as main language (document is primarily in Gujarati)
% Note: gloss-gujarati.ldf doesn't exist in polyglossia, but it will use hyphenation patterns
\setdefaultlanguage{gujarati}
\setotherlanguage{english}

% Configure Gujarati font properly
% Use Language=Default to prevent polyglossia from trying to add language-specific features
% that don't exist for Gujarati, which causes "empty feature" warnings
\newfontfamily\gujaratifont[Script=Gujarati,AutoFakeBold=2.5,AutoFakeSlant=0.3]{Noto Sans Gujarati}
\setmainfont[Script=Gujarati,AutoFakeBold=2.5,AutoFakeSlant=0.3]{Noto Sans Gujarati}
% Use Noto Sans Gujarati for monospace to support Gujarati in text
\setmonofont[Scale=0.9]{Noto Sans Gujarati}

% Configure English to use the same font
\newfontfamily\englishfont[Script=Gujarati,AutoFakeBold=2.5,AutoFakeSlant=0.3]{Noto Sans Gujarati}

% Translations for polyglossia
\gappto\captionsgujarati{
  \renewcommand{\tablename}{કોષ્ટક}
  \renewcommand{\figurename}{આકૃતિ}
}

% Helper for TikZ nodes to ensure Gujarati font
\newcommand{\gu}[1]{{\gujaratifont #1}}

% Custom environments
\newtcolorbox{solutionbox}{
    breakable,
    enhanced,
    colback=solutioncolor!5!white,
    colframe=solutioncolor!75!black,
    fonttitle=\bfseries,
    title=જવાબ
}

\newtcolorbox{solutionboxnobreak}{
 colback=solutioncolor!5!white,
 colframe=solutioncolor!75!black,
 fonttitle=\bfseries,
 title=જવાબ
}

\newtcolorbox{keyformula}{
 breakable,
 enhanced,
 colback=keycolor!5!white,
 colframe=keycolor!75!black,
 fonttitle=\bfseries,
 title=રાસાયણિક સમીકરણ/સૂત્ર
}

\newtcolorbox{mnemonicbox}{
 breakable,
 enhanced,
 colback=mnemoniccolor!5!white,
 colframe=mnemoniccolor!75!black,
 fonttitle=\bfseries,
 title=મેમરી ટ્રીક
}


\begin{document}

\begin{center}
{\Huge\bfseries\color{headcolor} Subject Name (Gujarati)}\\[5pt]
{\LARGE 4331603 -- Summer 2024}\\[3pt]
{\large Semester 1 Study Material}\\[3pt]
{\normalsize\textit{Detailed Solutions and Explanations}}
\end{center}

\vspace{10pt}

\subsection*{પ્રશ્ન 1(અ) [3
ગુણ]}\label{uxaaauxab0uxab6uxaa8-1uxa85-3-uxa97uxaa3}

\textbf{નીચેના શબ્દોને વ્યાખ્યાયિત કરો: 1. ડેટા 2. ઇન્ફોર્મેશન 3. મેટાડેટા}

\begin{solutionbox}

\textbf{ટેબલ: ડેટા વિ ઇન્ફોર્મેશન વિ મેટાડેટા}

{\def\LTcaptype{none} % do not increment counter
\begin{longtable}[]{@{}
  >{\raggedright\arraybackslash}p{(\linewidth - 4\tabcolsep) * \real{0.2222}}
  >{\raggedright\arraybackslash}p{(\linewidth - 4\tabcolsep) * \real{0.4444}}
  >{\raggedright\arraybackslash}p{(\linewidth - 4\tabcolsep) * \real{0.3333}}@{}}
\toprule\noalign{}
\begin{minipage}[b]{\linewidth}\raggedright
શબ્દ
\end{minipage} & \begin{minipage}[b]{\linewidth}\raggedright
વ્યાખ્યા
\end{minipage} & \begin{minipage}[b]{\linewidth}\raggedright
ઉદાહરણ
\end{minipage} \\
\midrule\noalign{}
\endhead
\bottomrule\noalign{}
\endlastfoot
\textbf{ડેટા} & કોઈ સંદર્ભ વગરના કાચા તથ્યો અને આંકડાઓ & ``25'', ``જોન'',
``મુંબઈ'' \\
\textbf{ઇન્ફોર્મેશન} & અર્થ અને સંદર્ભ સાથે પ્રોસેસ કરેલા ડેટા & ``જોન 25 વર્ષનો છે અને
મુંબઈમાં રહે છે'' \\
\textbf{મેટાડેટા} & ડેટા વિશેનો ડેટા જે સ્ટ્રક્ચર અને પ્રોપર્ટીઝ વર્ણવે છે & ``ઉંમર
ફીલ્ડ: Integer, મહત્તમ લંબાઈ: 3'' \\
\end{longtable}
}

\begin{itemize}
\tightlist
\item
  \textbf{ડેટા}: ઇન્ફોર્મેશન સિસ્ટમ્સના મૂળભૂત બિલ્ડિંગ બ્લોક્સ
\item
  \textbf{ઇન્ફોર્મેશન}: નિર્ણય લેવા માટે ડેટા પ્રોસેસિંગનું પરિણામ
\item
  \textbf{મેટાડેટા}: ડેટાબેસ ડિઝાઇન અને મેનેજમેન્ટ માટે જરૂરી
\end{itemize}

\end{solutionbox}
\begin{mnemonicbox}
``DIM - ડેટા મેટાડેટાનો ઉપયોગ કરીને ઇન્ફોર્મેશન આપે છે''

\end{mnemonicbox}
\begin{center}\rule{0.5\linewidth}{0.5pt}\end{center}

\subsection*{પ્રશ્ન 1(બ) [4
ગુણ]}\label{uxaaauxab0uxab6uxaa8-1uxaac-4-uxa97uxaa3}

\textbf{ફાઇલ સિસ્ટમ વિ ડેટાબેસ સિસ્ટમની તુલના કરો}

\begin{solutionbox}

\textbf{ટેબલ: ફાઇલ સિસ્ટમ વિ ડેટાબેસ સિસ્ટમ તુલના}

{\def\LTcaptype{none} % do not increment counter
\begin{longtable}[]{@{}lll@{}}
\toprule\noalign{}
પાસું & ફાઇલ સિસ્ટમ & ડેટાબેસ સિસ્ટમ \\
\midrule\noalign{}
\endhead
\bottomrule\noalign{}
\endlastfoot
\textbf{ડેટા સ્ટોરેજ} & દરેક એપ્લિકેશન માટે અલગ ફાઇલો & કેન્દ્રીકૃત સ્ટોરેજ \\
\textbf{ડેટા રિડન્ડન્સી} & ઉચ્ચ રિડન્ડન્સી & લઘુત્તમ રિડન્ડન્સી \\
\textbf{ડેટા સુસંગતતા} & નબળી સુસંગતતા & ઉચ્ચ સુસંગતતા \\
\textbf{ડેટા સિક્યોરિટી} & મર્યાદિત સિક્યોરિટી & એડવાન્સ સિક્યોરિટી ફીચર્સ \\
\textbf{એકસાથે એક્સેસ} & મર્યાદિત સપોર્ટ & સંપૂર્ણ એકસાથે સપોર્ટ \\
\textbf{ડેટા ઇન્ડિપેન્ડન્સ} & કોઈ ઇન્ડિપેન્ડન્સ નથી & ફિઝિકલ અને લોજિકલ
ઇન્ડિપેન્ડન્સ \\
\end{longtable}
}

\begin{itemize}
\tightlist
\item
  \textbf{ફાઇલ સિસ્ટમ}: સરળ પણ ડેટા ડુપ્લિકેશનની સમસ્યાઓ સાથે
\item
  \textbf{ડેટાબેસ સિસ્ટમ}: જટિલ પણ કાર્યક્ષમ ડેટા મેનેજમેન્ટ
\item
  \textbf{મુખ્ય ફાયદો}: DBMS ડેટા રિડન્ડન્સી અને અસુસંગતતા દૂર કરે છે
\end{itemize}

\end{solutionbox}
\begin{mnemonicbox}
``DBMS = ડેટા બેટર મેનેજ્ડ સિસ્ટમેટિકલી''

\end{mnemonicbox}
\begin{center}\rule{0.5\linewidth}{0.5pt}\end{center}

\subsection*{પ્રશ્ન 1(ક) [7
ગુણ]}\label{uxaaauxab0uxab6uxaa8-1uxa95-7-uxa97uxaa3}

\textbf{નેટવર્ક ડેટા મોડેલ દોરો અને સમજાવો}

\begin{solutionbox}

\textbf{ડાયાગ્રામ:}

\begin{verbatim}
    Owner 1
       |
    Set Type 1
    /    |    {}
Member1 Member2 Member3
   |       |       |
Set Type 2 Set Type 3 Set Type 4
   |       |       |
Member4 Member5 Member6
\end{verbatim}

\textbf{ટેબલ: નેટવર્ક મોડેલના ઘટકો}

{\def\LTcaptype{none} % do not increment counter
\begin{longtable}[]{@{}lll@{}}
\toprule\noalign{}
ઘટક & વર્ણન & ઉદાહરણ \\
\midrule\noalign{}
\endhead
\bottomrule\noalign{}
\endlastfoot
\textbf{રેકોર્ડ ટાઇપ} & એન્ટિટીનું પ્રતિનિધિત્વ & કર્મચારી, વિભાગ \\
\textbf{સેટ ટાઇપ} & રેકોર્ડ્સ વચ્ચેનો સંબંધ & કામ-કરે, મેનેજ-કરે \\
\textbf{ઓનર} & સંબંધમાં પેરેન્ટ રેકોર્ડ & વિભાગ (ઓનર) \\
\textbf{મેમ્બર} & સંબંધમાં ચાઇલ્ડ રેકોર્ડ & કર્મચારી (મેમ્બર) \\
\end{longtable}
}

\begin{itemize}
\tightlist
\item
  \textbf{ઓનર રેકોર્ડ}: સેટને નિયંત્રિત કરે છે અને અનેક મેમ્બર્સ હોઈ શકે છે
\item
  \textbf{મેમ્બર રેકોર્ડ}: એક અથવા વધુ સેટ્સનું સભ્ય છે
\item
  \textbf{સેટ ઓકરન્સ}: સેટ ટાઇપનું ઇન્સ્ટન્સ જે ઓનરને મેમ્બર્સ સાથે જોડે છે
\item
  \textbf{નેવિગેશન}: રેકોર્ડ એક્સેસ માટે પોઇન્ટર્સનો ઉપયોગ
\end{itemize}

\end{solutionbox}
\begin{mnemonicbox}
``નેટવર્ક = અનેક કનેક્શન્સ સાથેના નોડ્સ''

\end{mnemonicbox}
\begin{center}\rule{0.5\linewidth}{0.5pt}\end{center}

\subsection*{પ્રશ્ન 1(ક) અથવા [7
ગુણ]}\label{uxaaauxab0uxab6uxaa8-1uxa95-uxa85uxaa5uxab5-7-uxa97uxaa3}

\textbf{સ્કીમા શું છે? ઉદાહરણ સાથે સ્કીમાના વિવિધ પ્રકારો સમજાવો}

\begin{solutionbox}

\textbf{વ્યાખ્યા}: સ્કીમા એ ડેટાબેસનું લોજિકલ સ્ટ્રક્ચર અથવા બ્લુપ્રિન્ટ છે જે વ્યાખ્યાયિત
કરે છે કે ડેટા કેવી રીતે ગોઠવાયેલો છે.

\textbf{ડાયાગ્રામ:}

\begin{center}
\textbf{Mermaid Diagram (Code)}
\begin{verbatim}
{Shaded}
{Highlighting}[]
graph LR
    A[External Schema] {-{-}{} B[Conceptual Schema]}
    B {-{-}{} C[Internal Schema]}
    A {-{-}{} D[View 1]}
    A {-{-}{} E[View 2]}
    B {-{-}{} F[Logical Structure]}
    C {-{-}{} G[Physical Storage]}
{Highlighting}
{Shaded}
\end{verbatim}
\end{center}

\textbf{ટેબલ: સ્કીમાના પ્રકારો}

{\def\LTcaptype{none} % do not increment counter
\begin{longtable}[]{@{}
  >{\raggedright\arraybackslash}p{(\linewidth - 6\tabcolsep) * \real{0.3095}}
  >{\raggedright\arraybackslash}p{(\linewidth - 6\tabcolsep) * \real{0.1667}}
  >{\raggedright\arraybackslash}p{(\linewidth - 6\tabcolsep) * \real{0.3095}}
  >{\raggedright\arraybackslash}p{(\linewidth - 6\tabcolsep) * \real{0.2143}}@{}}
\toprule\noalign{}
\begin{minipage}[b]{\linewidth}\raggedright
સ્કીમા પ્રકાર
\end{minipage} & \begin{minipage}[b]{\linewidth}\raggedright
લેવલ
\end{minipage} & \begin{minipage}[b]{\linewidth}\raggedright
વર્ણન
\end{minipage} & \begin{minipage}[b]{\linewidth}\raggedright
ઉદાહરણ
\end{minipage} \\
\midrule\noalign{}
\endhead
\bottomrule\noalign{}
\endlastfoot
\textbf{એક્સટર્નલ સ્કીમા} & વ્યૂ લેવલ & ડેટાબેસનો યુઝર-સ્પેસિફિક વ્યૂ & શિક્ષકો માટે
વિદ્યાર્થીઓના ગ્રેડ્સનો વ્યૂ \\
\textbf{કોન્સેપ્ચુઅલ સ્કીમા} & લોજિકલ લેવલ & સંપૂર્ણ લોજિકલ સ્ટ્રક્ચર & બધા ટેબલ્સ,
સંબંધો, કન્સ્ટ્રેન્ટ્સ \\
\textbf{ઇન્ટર્નલ સ્કીમા} & ફિઝિકલ લેવલ & ફિઝિકલ સ્ટોરેજ સ્ટ્રક્ચર & ઇન્ડેક્સ ફાઇલો,
સ્ટોરેજ એલોકેશન \\
\end{longtable}
}

\begin{itemize}
\tightlist
\item
  \textbf{એક્સટર્નલ સ્કીમા}: યુઝર્સ માટે ડેટા ઇન્ડિપેન્ડન્સ પ્રદાન કરે છે
\item
  \textbf{કોન્સેપ્ચુઅલ સ્કીમા}: ડેટાબેસ ડિઝાઇનરનો સંપૂર્ણ વ્યૂ
\item
  \textbf{ઇન્ટર્નલ સ્કીમા}: ડેટાબેસ એડમિનિસ્ટ્રેટરનો ફિઝિકલ વ્યૂ
\end{itemize}

\end{solutionbox}
\begin{mnemonicbox}
``ECI - એક્સટર્નલ કોન્સેપ્ચુઅલ ઇન્ટર્નલ''

\end{mnemonicbox}
\begin{center}\rule{0.5\linewidth}{0.5pt}\end{center}

\subsection*{પ્રશ્ન 2(અ) [3
ગુણ]}\label{uxaaauxab0uxab6uxaa8-2uxa85-3-uxa97uxaa3}

\textbf{નીચેના શબ્દોને વ્યાખ્યાયિત કરો: 1. એન્ટિટી 2. એટ્રિબ્યુટ્સ 3. રિલેશનશિપ}

\begin{solutionbox}

\textbf{ટેબલ: ER મોડેલની મૂળભૂત કોન્સેપ્ટ્સ}

{\def\LTcaptype{none} % do not increment counter
\begin{longtable}[]{@{}
  >{\raggedright\arraybackslash}p{(\linewidth - 4\tabcolsep) * \real{0.2222}}
  >{\raggedright\arraybackslash}p{(\linewidth - 4\tabcolsep) * \real{0.4444}}
  >{\raggedright\arraybackslash}p{(\linewidth - 4\tabcolsep) * \real{0.3333}}@{}}
\toprule\noalign{}
\begin{minipage}[b]{\linewidth}\raggedright
શબ્દ
\end{minipage} & \begin{minipage}[b]{\linewidth}\raggedright
વ્યાખ્યા
\end{minipage} & \begin{minipage}[b]{\linewidth}\raggedright
ઉદાહરણ
\end{minipage} \\
\midrule\noalign{}
\endhead
\bottomrule\noalign{}
\endlastfoot
\textbf{એન્ટિટી} & સ્વતંત્ર અસ્તિત્વ ધરાવતો વાસ્તવિક વિશ્વનો ઓબ્જેક્ટ & વિદ્યાર્થી,
કોર્સ, શિક્ષક \\
\textbf{એટ્રિબ્યુટ્સ} & એન્ટિટીનું વર્ણન કરતા ગુણધર્મો & વિદ્યાર્થી: ID, નામ, ઉંમર \\
\textbf{રિલેશનશિપ} & બે અથવા વધુ એન્ટિટી વચ્ચેનો સંબંધ & વિદ્યાર્થી કોર્સમાં નોંધણી
કરે છે \\
\end{longtable}
}

\begin{itemize}
\tightlist
\item
  \textbf{એન્ટિટી}: ER ડાયાગ્રામમાં લંબચોરસ દ્વારા રજૂ થાય છે
\item
  \textbf{એટ્રિબ્યુટ્સ}: એન્ટિટીઓ સાથે જોડાયેલા અંડાકાર દ્વારા રજૂ થાય છે
\item
  \textbf{રિલેશનશિપ}: એન્ટિટીઓને જોડતા હીરા દ્વારા રજૂ થાય છે
\end{itemize}

\end{solutionbox}
\begin{mnemonicbox}
``EAR - એન્ટિટીના એટ્રિબ્યુટ્સ અને રિલેશનશિપ્સ છે''

\end{mnemonicbox}
\begin{center}\rule{0.5\linewidth}{0.5pt}\end{center}

\subsection*{પ્રશ્ન 2(બ) [4
ગુણ]}\label{uxaaauxab0uxab6uxaa8-2uxaac-4-uxa97uxaa3}

\textbf{ઉદાહરણ સાથે વીક એન્ટિટી સેટ્સનું વર્ણન કરો}

\begin{solutionbox}

\textbf{વ્યાખ્યા}: વીક એન્ટિટી એ એવી એન્ટિટી છે જે પોતાના એટ્રિબ્યુટ્સ દ્વારા અનન્ય
રીતે ઓળખાઈ શકતી નથી અને સ્ટ્રોંગ એન્ટિટી પર આધાર રાખે છે.

\textbf{ડાયાગ્રામ:}

\begin{verbatim}
+{-{-}{-}{-}{-}{-}{-}{-}{-}{-}+       +===========+       +{-}{-}{-}{-}{-}{-}{-}{-}{-}{-}+}
| Employee |{-{-}{-}{-}{-}{-}{-}| Dependent |{-}{-}{-}{-}{-}{-}{-}| Person   |}
|   (1)    |       |  (Weak)   |       |   (N)    |
+{-{-}{-}{-}{-}{-}{-}{-}{-}{-}+       +===========+       +{-}{-}{-}{-}{-}{-}{-}{-}{-}{-}+}
    emp\_id              name              dep\_name
                     (Partial Key)
\end{verbatim}

\textbf{ટેબલ: વીક વિ સ્ટ્રોંગ એન્ટિટી}

{\def\LTcaptype{none} % do not increment counter
\begin{longtable}[]{@{}lll@{}}
\toprule\noalign{}
પાસું & સ્ટ્રોંગ એન્ટિટી & વીક એન્ટિટી \\
\midrule\noalign{}
\endhead
\bottomrule\noalign{}
\endlastfoot
\textbf{પ્રાઇમરી કી} & પોતાની પ્રાઇમરી કી છે & કોઈ પ્રાઇમરી કી નથી \\
\textbf{અસ્તિત્વ} & સ્વતંત્ર અસ્તિત્વ & સ્ટ્રોંગ એન્ટિટી પર આધાર \\
\textbf{પ્રતિનિધિત્વ} & એક લંબચોરસ & ડબલ લંબચોરસ \\
\textbf{ઉદાહરણ} & કર્મચારી & કર્મચારીનો આશ્રિત \\
\end{longtable}
}

\begin{itemize}
\tightlist
\item
  \textbf{પાર્શિયલ કી}: એટ્રિબ્યુટ જે વીક એન્ટિટીને આંશિક રૂપે ઓળખે છે
\item
  \textbf{આઇડેન્ટિફાઇંગ રિલેશનશિપ}: વીક એન્ટિટીને સ્ટ્રોંગ એન્ટિટી સાથે જોડે છે
\item
  \textbf{ટોટલ પાર્ટિસિપેશન}: વીક એન્ટિટીએ સંબંધમાં સહભાગી થવું જ જોઈએ
\end{itemize}

\end{solutionbox}
\begin{mnemonicbox}
``વીક એન્ટિટીઓ આશ્રિત હોય છે''

\end{mnemonicbox}
\begin{center}\rule{0.5\linewidth}{0.5pt}\end{center}

\subsection*{પ્રશ્ન 2(ક) [7
ગુણ]}\label{uxaaauxab0uxab6uxaa8-2uxa95-7-uxa97uxaa3}

\textbf{યુનિવર્સિટી મેનેજમેન્ટ સિસ્ટમ માટે ER ડાયાગ્રામ દોરો}

\begin{solutionbox}

\textbf{ડાયાગ્રામ:}

\begin{verbatim}
erDiagram
    STUDENT \{
        int student\_id PK
        string name
        string email
        date birth\_date
        string address
    \}
    
    COURSE \{
        int course\_id PK
        string course\_name
        int credits
        string department
    \}
    
    TEACHER \{
        int teacher\_id PK
        string name
        string department
        string qualification
    \}
    
    ENROLLMENT \{
        int enrollment\_id PK
        date enrollment\_date
        char grade
    \}
    
    STUDENT ||{-{-}o\{ ENROLLMENT : enrolls}
    COURSE ||{-{-}o\{ ENROLLMENT : has}
    TEACHER ||{-{-}o\{ COURSE : teaches}
\end{verbatim}

\textbf{ટેબલ: એન્ટિટી રિલેશનશિપ્સ}

{\def\LTcaptype{none} % do not increment counter
\begin{longtable}[]{@{}
  >{\raggedright\arraybackslash}p{(\linewidth - 4\tabcolsep) * \real{0.3500}}
  >{\raggedright\arraybackslash}p{(\linewidth - 4\tabcolsep) * \real{0.3250}}
  >{\raggedright\arraybackslash}p{(\linewidth - 4\tabcolsep) * \real{0.3250}}@{}}
\toprule\noalign{}
\begin{minipage}[b]{\linewidth}\raggedright
રિલેશનશિપ
\end{minipage} & \begin{minipage}[b]{\linewidth}\raggedright
કાર્ડિનાલિટી
\end{minipage} & \begin{minipage}[b]{\linewidth}\raggedright
વર્ણન
\end{minipage} \\
\midrule\noalign{}
\endhead
\bottomrule\noalign{}
\endlastfoot
\textbf{વિદ્યાર્થી નોંધણી કરે કોર્સ} & M:N & અનેક વિદ્યાર્થીઓ અનેક કોર્સમાં નોંધણી
કરી શકે \\
\textbf{શિક્ષક શીખવે કોર્સ} & 1:N & એક શિક્ષક અનેક કોર્સ શીખવે છે \\
\textbf{કોર્સ છે નોંધણી} & 1:N & એક કોર્સમાં અનેક નોંધણીઓ છે \\
\end{longtable}
}

\begin{itemize}
\tightlist
\item
  \textbf{પ્રાથમિક એન્ટિટીઓ}: વિદ્યાર્થી, કોર્સ, શિક્ષક
\item
  \textbf{એસોસિએટિવ એન્ટિટી}: નોંધણી (M:N સંબંધ ઉકેલે છે)
\item
  \textbf{કી એટ્રિબ્યુટ્સ}: બધી એન્ટિટીઓમાં અનન્ય ઓળખકર્તા છે
\end{itemize}

\end{solutionbox}
\begin{mnemonicbox}
``યુનિવર્સિટી = વિદ્યાર્થીઓ શિક્ષકો પાસેથી કોર્સ લે છે''

\end{mnemonicbox}
\begin{center}\rule{0.5\linewidth}{0.5pt}\end{center}

\subsection*{પ્રશ્ન 2(અ) અથવા [3
ગુણ]}\label{uxaaauxab0uxab6uxaa8-2uxa85-uxa85uxaa5uxab5-3-uxa97uxaa3}

\textbf{નીચેના શબ્દોને વ્યાખ્યાયિત કરો: 1. પ્રાઇમરી કી 2. ફોરેન કી 3. કેન્ડિડેટ
કી}

\begin{solutionbox}

\textbf{ટેબલ: ડેટાબેસ કીઝ}

{\def\LTcaptype{none} % do not increment counter
\begin{longtable}[]{@{}
  >{\raggedright\arraybackslash}p{(\linewidth - 4\tabcolsep) * \real{0.3226}}
  >{\raggedright\arraybackslash}p{(\linewidth - 4\tabcolsep) * \real{0.3871}}
  >{\raggedright\arraybackslash}p{(\linewidth - 4\tabcolsep) * \real{0.2903}}@{}}
\toprule\noalign{}
\begin{minipage}[b]{\linewidth}\raggedright
કી પ્રકાર
\end{minipage} & \begin{minipage}[b]{\linewidth}\raggedright
વ્યાખ્યા
\end{minipage} & \begin{minipage}[b]{\linewidth}\raggedright
ઉદાહરણ
\end{minipage} \\
\midrule\noalign{}
\endhead
\bottomrule\noalign{}
\endlastfoot
\textbf{પ્રાઇમરી કી} & દરેક રેકોર્ડ માટે અનન્ય ઓળખકર્તા & વિદ્યાર્થી ટેબલમાં
Student\_ID \\
\textbf{ફોરેન કી} & બીજા ટેબલની પ્રાઇમરી કીનો સંદર્ભ & નોંધણી ટેબલમાં
Student\_ID \\
\textbf{કેન્ડિડેટ કી} & સંભવિત પ્રાઇમરી કી એટ્રિબ્યુટ & વિદ્યાર્થી ટેબલમાં Email,
ફોન \\
\end{longtable}
}

\begin{itemize}
\tightlist
\item
  \textbf{પ્રાઇમરી કી}: NULL હોઈ શકે નહીં અને અનન્ય હોવી જોઈએ
\item
  \textbf{ફોરેન કી}: રેફરન્શિયલ ઇન્ટેગ્રિટી જાળવે છે
\item
  \textbf{કેન્ડિડેટ કી}: વૈકલ્પિક અનન્ય ઓળખકર્તાઓ
\end{itemize}

\end{solutionbox}
\begin{mnemonicbox}
``PFC - પ્રાઇમરી ફોરેન કેન્ડિડેટ''

\end{mnemonicbox}
\begin{center}\rule{0.5\linewidth}{0.5pt}\end{center}

\subsection*{પ્રશ્ન 2(બ) અથવા [4
ગુણ]}\label{uxaaauxab0uxab6uxaa8-2uxaac-uxa85uxaa5uxab5-4-uxa97uxaa3}

\textbf{જનરલાઇઝેશન અને સ્પેશિયલાઇઝેશન પર ટૂંકી નોંધ લખો}

\begin{solutionbox}

\textbf{જનરલાઇઝેશન}: અનેક એન્ટિટીઓમાંથી સામાન્ય એટ્રિબ્યુટ્સ કાઢીને સામાન્ય એન્ટિટી
બનાવવાની પ્રક્રિયા.

\textbf{સ્પેશિયલાઇઝેશન}: વિશિષ્ટ લાક્ષણિકતાઓના આધારે એન્ટિટીના પેટા વર્ગો
વ્યાખ્યાયિત કરવાની પ્રક્રિયા.

\textbf{ડાયાગ્રામ:}

\begin{center}
\textbf{Mermaid Diagram (Code)}
\begin{verbatim}
{Shaded}
{Highlighting}[]
graph TD
    A[વ્યક્તિ] {-{-}{} B[વિદ્યાર્થી]}
    A {-{-}{} C[શિક્ષક]}
    A {-{-}{} D[સ્ટાફ]}
    B {-{-}{} E[અંડરગ્રેજ્યુએટ]}
    B {-{-}{} F[ગ્રેજ્યુએટ]}
{Highlighting}
{Shaded}
\end{verbatim}
\end{center}

\textbf{ટેબલ: જનરલાઇઝેશન વિ સ્પેશિયલાઇઝેશન}

{\def\LTcaptype{none} % do not increment counter
\begin{longtable}[]{@{}lll@{}}
\toprule\noalign{}
પાસું & જનરલાઇઝેશન & સ્પેશિયલાઇઝેશન \\
\midrule\noalign{}
\endhead
\bottomrule\noalign{}
\endlastfoot
\textbf{દિશા} & બોટમ-અપ અપ્રોચ & ટોપ-ડાઉન અપ્રોચ \\
\textbf{હેતુ} & રિડન્ડન્સી દૂર કરવી & વિશિષ્ટ એટ્રિબ્યુટ્સ ઉમેરવા \\
\textbf{પરિણામ} & સુપરક્લાસ બનાવટ & સબક્લાસ બનાવટ \\
\end{longtable}
}

\begin{itemize}
\tightlist
\item
  \textbf{ISA રિલેશનશિપ}: સુપરક્લાસ અને સબક્લાસ વચ્ચે ``Is-A'' સંબંધ
\item
  \textbf{ઇન્હેરિટન્સ}: સબક્લાસ સુપરક્લાસમાંથી એટ્રિબ્યુટ્સ વારસામાં લે છે
\end{itemize}

\end{solutionbox}
\begin{mnemonicbox}
``જનરલ ઉપર જાય, સ્પેશિયલ નીચે જાય''

\end{mnemonicbox}
\begin{center}\rule{0.5\linewidth}{0.5pt}\end{center}

\subsection*{પ્રશ્ન 2(ક) અથવા [7
ગુણ]}\label{uxaaauxab0uxab6uxaa8-2uxa95-uxa85uxaa5uxab5-7-uxa97uxaa3}

\textbf{ઉદાહરણ સાથે વિવિધ રિલેશનલ એલ્જીબ્રા ઓપરેશન સમજાવો}

\begin{solutionbox}

\textbf{ટેબલ: રિલેશનલ એલ્જીબ્રા ઓપરેશન્સ}

{\def\LTcaptype{none} % do not increment counter
\begin{longtable}[]{@{}llll@{}}
\toprule\noalign{}
ઓપરેશન & સિમ્બોલ & વર્ણન & ઉદાહરણ \\
\midrule\noalign{}
\endhead
\bottomrule\noalign{}
\endlastfoot
\textbf{સિલેક્ટ} & σ & શરત આધારે પંક્તિઓ પસંદ કરે &
σ(age\textgreater20)(Student) \\
\textbf{પ્રોજેક્ટ} & π & વિશિષ્ટ કૉલમ્સ પસંદ કરે & π(name,age)(Student) \\
\textbf{યુનિયન} & \cup & બે રિલેશન્સને જોડે & R \cup S \\
\textbf{ઇન્ટરસેક્શન} & \cap & રિલેશન્સમાંથી સામાન્ય ટ્યુપલ્સ & R \cap S \\
\textbf{ડિફરન્સ} & - & R માં છે પણ S માં નથી તે ટ્યુપલ્સ & R - S \\
\textbf{જોઇન} & ⋈ & સંબંધિત ટ્યુપલ્સને જોડે & Student ⋈ Enrollment \\
\end{longtable}
}

\textbf{ઉદાહરણ રિલેશન્સ:}

Student: (ID=1, Name=જોન, Age=20) Course: (CID=101, CName=DBMS,
Credits=3)

\begin{itemize}
\tightlist
\item
  \textbf{સિલેક્શન}: σ(Age\textgreater18)(Student) 18 વર્ષથી વધુ વયના
  વિદ્યાર્થીઓ રિટર્ન કરે
\item
  \textbf{પ્રોજેક્શન}: π(Name)(Student) માત્ર નામો રિટર્ન કરે
\item
  \textbf{જોઇન}: Student ⋈ Enrollment વિદ્યાર્થી અને નોંધણીનો ડેટા જોડે
\end{itemize}

\end{solutionbox}
\begin{mnemonicbox}
``SPUDIJ - સિલેક્ટ પ્રોજેક્ટ યુનિયન ડિફરન્સ ઇન્ટરસેક્શન જોઇન''

\end{mnemonicbox}
\begin{center}\rule{0.5\linewidth}{0.5pt}\end{center}

\subsection*{પ્રશ્ન 3(અ) [3
ગુણ]}\label{uxaaauxab0uxab6uxaa8-3uxa85-3-uxa97uxaa3}

\textbf{SQL માં ન્યુમેરિક ફંક્શન્સની યાદી આપો. કોઈપણ બે સમજાવો}

\begin{solutionbox}

\textbf{ટેબલ: SQL ન્યુમેરિક ફંક્શન્સ}

{\def\LTcaptype{none} % do not increment counter
\begin{longtable}[]{@{}lll@{}}
\toprule\noalign{}
ફંક્શન & હેતુ & ઉદાહરણ \\
\midrule\noalign{}
\endhead
\bottomrule\noalign{}
\endlastfoot
\textbf{ABS()} & એબ્સોલ્યુટ વેલ્યુ & ABS(-15) = 15 \\
\textbf{CEIL()} & વેલ્યુ \geq ની સૌથી નાની પૂર્ણાંક & CEIL(4.3) = 5 \\
\textbf{FLOOR()} & વેલ્યુ \leq ની સૌથી મોટી પૂર્ણાંક & FLOOR(4.7) = 4 \\
\textbf{ROUND()} & નિર્દિષ્ટ સ્થાને રાઉન્ડ કરે & ROUND(15.76, 1) = 15.8 \\
\textbf{SQRT()} & વર્ગમૂળ & SQRT(16) = 4 \\
\textbf{POWER()} & પાવર પર વધારો & POWER(2, 3) = 8 \\
\end{longtable}
}

\textbf{વિગતવાર ઉદાહરણો:}

\begin{itemize}
\tightlist
\item
  \textbf{ABS(number)}: એબ્સોલ્યુટ વેલ્યુ રિટર્ન કરે, નેગેટિવ સાઇન દૂર કરે
\item
  \textbf{ROUND(number, decimal\_places)}: નિર્દિષ્ટ દશાંશ સ્થાને નંબર રાઉન્ડ
  કરે
\end{itemize}

\end{solutionbox}
\begin{mnemonicbox}
``ગણિત ફંક્શન્સ નંબર્સને સરસ બનાવે''

\end{mnemonicbox}
\begin{center}\rule{0.5\linewidth}{0.5pt}\end{center}

\subsection*{પ્રશ્ન 3(બ) [4
ગુણ]}\label{uxaaauxab0uxab6uxaa8-3uxaac-4-uxa97uxaa3}

\textbf{ઉદાહરણ સાથે Having અને Order by Clause નું વર્ણન કરો}

\begin{solutionbox}

\textbf{HAVING Clause}: GROUP BY સાથે એગ્રીગેટ કન્ડિશન્સ આધારે ગ્રુપ્સ ફિલ્ટર કરવા
ઉપયોગ થાય.

\textbf{ORDER BY Clause}: પરિણામ સેટને ચડતા અથવા ઊતરતા ક્રમમાં સોર્ટ કરવા
ઉપયોગ થાય.

\textbf{ટેબલ: HAVING વિ WHERE}

{\def\LTcaptype{none} % do not increment counter
\begin{longtable}[]{@{}lll@{}}
\toprule\noalign{}
પાસું & WHERE & HAVING \\
\midrule\noalign{}
\endhead
\bottomrule\noalign{}
\endlastfoot
\textbf{ઉપયોગ} & વ્યક્તિગત પંક્તિઓ ફિલ્ટર કરે & ગ્રુપ કરેલા પરિણામો ફિલ્ટર કરે \\
\textbf{એગ્રીગેટ્સ સાથે} & ઉપયોગ કરી શકાતો નથી & એગ્રીગેટ ફંક્શન્સ ઉપયોગ કરી
શકે \\
\textbf{સ્થિતિ} & GROUP BY પહેલાં & GROUP BY પછી \\
\end{longtable}
}

\textbf{ઉદાહરણ:}

\begin{verbatim}
SELECT department, COUNT(*) as emp\_count
FROM employees 
WHERE salary {} 30000
GROUP BY department 
HAVING COUNT(*) {} 5
ORDER BY emp\_count DESC;
\end{verbatim}

\begin{itemize}
\tightlist
\item
  \textbf{WHERE}: 30000 થી વધુ પગાર ધરાવતા કર્મચારીઓ ફિલ્ટર કરે
\item
  \textbf{HAVING}: માત્ર 5 થી વધુ કર્મચારીઓ ધરાવતા વિભાગો બતાવે
\item
  \textbf{ORDER BY}: કર્મચારીઓની ગણતરી આધારે ઉતરતા ક્રમમાં સોર્ટ કરે
\end{itemize}

\end{solutionbox}
\begin{mnemonicbox}
``WHERE પંક્તિઓ ફિલ્ટર કરે, HAVING ગ્રુપ્સ ફિલ્ટર કરે, ORDER
BY પરિણામો સોર્ટ કરે''

\end{mnemonicbox}
\begin{center}\rule{0.5\linewidth}{0.5pt}\end{center}

\subsection*{પ્રશ્ન 3(ક) [7
ગુણ]}\label{uxaaauxab0uxab6uxaa8-3uxa95-7-uxa97uxaa3}

\textbf{Student\_ID, Stu\_Name, Stu\_Subject\_ID, Stu\_Marks, Stu\_Age
ફીલ્ડ્સ ધરાવતા student ટેબલ પર નીચેની queries perform કરો}

\begin{solutionbox}

\textbf{1. student ટેબલ બનાવવા માટે ક્વેરી:}

\begin{verbatim}
CREATE TABLE student (
    Student\_ID INT PRIMARY KEY,
    Stu\_Name VARCHAR(50),
    Stu\_Subject\_ID INT,
    Stu\_Marks INT,
    Stu\_Age INT
);
\end{verbatim}

\textbf{2. student ટેબલમાં રેકોર્ડ દાખલ કરવા માટે ક્વેરી:}

\begin{verbatim}
INSERT INTO student VALUES 
(1, {જોન}, 101, 85, 22),
(2, {મેરી}, 102, 90, 21);
\end{verbatim}

\textbf{3. લઘુત્તમ અને મહત્તમ ગુણ શોધો:}

\begin{verbatim}
SELECT MIN(Stu\_Marks) as Min\_Marks, 
       MAX(Stu\_Marks) as Max\_Marks 
FROM student;
\end{verbatim}

\textbf{4. 82 થી વધુ ગુણ અને 22 વર્ષ વયના વિદ્યાર્થીઓ:}

\begin{verbatim}
SELECT * FROM student 
WHERE Stu\_Marks {} 82 AND Stu\_Age = 22;
\end{verbatim}

\textbf{5. નામ `m' અક્ષરથી શરૂ થતા વિદ્યાર્થીઓ:}

\begin{verbatim}
SELECT * FROM student 
WHERE Stu\_Name LIKE {m\%};
\end{verbatim}

\textbf{6. સરેરાશ ગુણ શોધો:}

\begin{verbatim}
SELECT AVG(Stu\_Marks) as Average\_Marks 
FROM student;
\end{verbatim}

\textbf{7. Stu\_address કૉલમ ઉમેરો:}

\begin{verbatim}
ALTER TABLE student 
ADD Stu\_address VARCHAR(100);
\end{verbatim}

\end{solutionbox}
\begin{mnemonicbox}
``CRUD + એનાલિટિક્સ = સંપૂર્ણ ડેટાબેસ ઓપરેશન્સ''

\end{mnemonicbox}
\begin{center}\rule{0.5\linewidth}{0.5pt}\end{center}

\subsection*{પ્રશ્ન 3(અ) અથવા [3
ગુણ]}\label{uxaaauxab0uxab6uxaa8-3uxa85-uxa85uxaa5uxab5-3-uxa97uxaa3}

\textbf{ઉદાહરણ સાથે SQL માં વિવિધ ડેટ ફંક્શન વર્ણવો}

\begin{solutionbox}

\textbf{ટેબલ: SQL ડેટ ફંક્શન્સ}

{\def\LTcaptype{none} % do not increment counter
\begin{longtable}[]{@{}
  >{\raggedright\arraybackslash}p{(\linewidth - 4\tabcolsep) * \real{0.3571}}
  >{\raggedright\arraybackslash}p{(\linewidth - 4\tabcolsep) * \real{0.3214}}
  >{\raggedright\arraybackslash}p{(\linewidth - 4\tabcolsep) * \real{0.3214}}@{}}
\toprule\noalign{}
\begin{minipage}[b]{\linewidth}\raggedright
ફંક્શન
\end{minipage} & \begin{minipage}[b]{\linewidth}\raggedright
હેતુ
\end{minipage} & \begin{minipage}[b]{\linewidth}\raggedright
ઉદાહરણ
\end{minipage} \\
\midrule\noalign{}
\endhead
\bottomrule\noalign{}
\endlastfoot
\textbf{SYSDATE} & વર્તમાન સિસ્ટમ ડેટ & SYSDATE `2024-06-12' રિટર્ન કરે \\
\textbf{ADD\_MONTHS()} & ડેટમાં મહિનાઓ ઉમેરે & ADD\_MONTHS(`2024-01-15',
3) \\
\textbf{MONTHS\_BETWEEN()} & ડેટ્સ વચ્ચેના મહિનાઓ &
MONTHS\_BETWEEN(`2024-06-12', `2024-01-12') \\
\textbf{LAST\_DAY()} & મહિનાનો છેલ્લો દિવસ & LAST\_DAY(`2024-02-15') =
`2024-02-29' \\
\textbf{NEXT\_DAY()} & દિવસની આગલી ઘટના & NEXT\_DAY(`2024-06-12',
`FRIDAY') \\
\end{longtable}
}

\textbf{ઉદાહરણો:}

\begin{itemize}
\tightlist
\item
  \textbf{SYSDATE}: વર્તમાન સિસ્ટમ ડેટ અને ટાઇમ રિટર્ન કરે
\item
  \textbf{ADD\_MONTHS}: લોન ડ્યુ ડેટ્સ જેવી ભવિષ્યની તારીખો ગણવા માટે ઉપયોગી
\end{itemize}

\end{solutionbox}
\begin{mnemonicbox}
``ડેટ ફંક્શન્સ ટાઇમ મેનેજમેન્ટમાં મદદ કરે''

\end{mnemonicbox}
\begin{center}\rule{0.5\linewidth}{0.5pt}\end{center}

\subsection*{પ્રશ્ન 3(બ) અથવા [4
ગુણ]}\label{uxaaauxab0uxab6uxaa8-3uxaac-uxa85uxaa5uxab5-4-uxa97uxaa3}

\textbf{SQL માં કન્સ્ટ્રેન્ટ્સની સૂચિ બનાવો. ઉદાહરણ સાથે કોઈપણ બે સમજાવો}

\begin{solutionbox}

\textbf{ટેબલ: SQL કન્સ્ટ્રેન્ટ્સ}

{\def\LTcaptype{none} % do not increment counter
\begin{longtable}[]{@{}lll@{}}
\toprule\noalign{}
કન્સ્ટ્રેન્ટ & હેતુ & ઉદાહરણ \\
\midrule\noalign{}
\endhead
\bottomrule\noalign{}
\endlastfoot
\textbf{PRIMARY KEY} & અનન્ય ઓળખકર્તા & Student\_ID INT PRIMARY KEY \\
\textbf{FOREIGN KEY} & બીજા ટેબલનો સંદર્ભ & REFERENCES
Student(Student\_ID) \\
\textbf{NOT NULL} & null વેલ્યુઝ અટકાવે & Name VARCHAR(50) NOT NULL \\
\textbf{UNIQUE} & અનન્યતા સુનિશ્ચિત કરે & Email VARCHAR(100) UNIQUE \\
\textbf{CHECK} & ડેટા વેલિડેટ કરે & Age INT CHECK (Age \textgreater= 18) \\
\textbf{DEFAULT} & ડિફોલ્ટ વેલ્યુ & Status VARCHAR(10) DEFAULT `Active' \\
\end{longtable}
}

\textbf{વિગતવાર ઉદાહરણો:}

\textbf{PRIMARY KEY કન્સ્ટ્રેન્ટ:}

\begin{verbatim}
CREATE TABLE Student (
    Student\_ID INT PRIMARY KEY,
    Name VARCHAR(50)
);
\end{verbatim}

\textbf{CHECK કન્સ્ટ્રેન્ટ:}

\begin{verbatim}
CREATE TABLE Employee (
    Emp\_ID INT,
    Salary INT CHECK (Salary {} 0)
);
\end{verbatim}

\begin{itemize}
\tightlist
\item
  \textbf{PRIMARY KEY}: દરેક રેકોર્ડ અનન્ય ઓળખકર્તા છે તેની ખાતરી કરે
\item
  \textbf{CHECK}: ડેટા એન્ટ્રી દરમિયાન બિઝનેસ નિયમો વેલિડેટ કરે
\end{itemize}

\end{solutionbox}
\begin{mnemonicbox}
``કન્સ્ટ્રેન્ટ્સ ડેટા ક્વોલિટી કંટ્રોલ કરે''

\end{mnemonicbox}
\begin{center}\rule{0.5\linewidth}{0.5pt}\end{center}

\subsection*{પ્રશ્ન 3(ક) અથવા [7
ગુણ]}\label{uxaaauxab0uxab6uxaa8-3uxa95-uxa85uxaa5uxab5-7-uxa97uxaa3}

\textbf{ઉદાહરણ સાથે SQL માં વિવિધ પ્રકારના joins સમજાવો}

\begin{solutionbox}

\textbf{ટેબલ: SQL Joins ના પ્રકારો}

{\def\LTcaptype{none} % do not increment counter
\begin{longtable}[]{@{}
  >{\raggedright\arraybackslash}p{(\linewidth - 4\tabcolsep) * \real{0.3438}}
  >{\raggedright\arraybackslash}p{(\linewidth - 4\tabcolsep) * \real{0.4062}}
  >{\raggedright\arraybackslash}p{(\linewidth - 4\tabcolsep) * \real{0.2500}}@{}}
\toprule\noalign{}
\begin{minipage}[b]{\linewidth}\raggedright
Join પ્રકાર
\end{minipage} & \begin{minipage}[b]{\linewidth}\raggedright
વર્ણન
\end{minipage} & \begin{minipage}[b]{\linewidth}\raggedright
સિન્ટેક્સ
\end{minipage} \\
\midrule\noalign{}
\endhead
\bottomrule\noalign{}
\endlastfoot
\textbf{INNER JOIN} & બંને ટેબલમાંથી મેચિંગ રેકોર્ડ્સ રિટર્ન કરે & Table1 INNER
JOIN Table2 ON condition \\
\textbf{LEFT JOIN} & ડાબા ટેબલના બધા + જમણાના મેચિંગ રેકોર્ડ્સ & Table1 LEFT
JOIN Table2 ON condition \\
\textbf{RIGHT JOIN} & જમણા ટેબલના બધા + ડાબાના મેચિંગ રેકોર્ડ્સ & Table1 RIGHT
JOIN Table2 ON condition \\
\textbf{FULL OUTER JOIN} & બંને ટેબલના બધા રેકોર્ડ્સ & Table1 FULL OUTER JOIN
Table2 ON condition \\
\end{longtable}
}

\textbf{ઉદાહરણ ટેબલ્સ:} Students: (ID=1, Name=જોન), (ID=2, Name=મેરી)
Enrollments: (StudentID=1, Course=DBMS), (StudentID=3, Course=Java)

\textbf{INNER JOIN ઉદાહરણ:}

\begin{verbatim}
SELECT s.Name, e.Course 
FROM Students s 
INNER JOIN Enrollments e ON s.ID = e.StudentID;
\end{verbatim}

\emph{પરિણામ: માત્ર જોન DBMS કોર્સ સાથે}

\textbf{LEFT JOIN ઉદાહરણ:}

\begin{verbatim}
SELECT s.Name, e.Course 
FROM Students s 
LEFT JOIN Enrollments e ON s.ID = e.StudentID;
\end{verbatim}

\emph{પરિણામ: જોન-DBMS, મેરી-NULL}

\end{solutionbox}
\begin{mnemonicbox}
``JOIN સંબંધિત ટેબલ્સને જોડે છે''

\end{mnemonicbox}
\begin{center}\rule{0.5\linewidth}{0.5pt}\end{center}

\subsection*{પ્રશ્ન 4(અ) [3
ગુણ]}\label{uxaaauxab0uxab6uxaa8-4uxa85-3-uxa97uxaa3}

\textbf{SQL માં Grant અને Revoke કમાન્ડનું ઉદાહરણ આપો}

\begin{solutionbox}

\textbf{GRANT કમાન્ડ}: ડેટાબેસ ઓબ્જેક્ટ્સ પર યુઝર્સને વિશિષ્ટ વિશેષાધિકારો પ્રદાન કરે.

\textbf{REVOKE કમાન્ડ}: યુઝર્સમાંથી અગાઉ આપેલા વિશેષાધિકારો દૂર કરે.

\textbf{ટેબલ: સામાન્ય વિશેષાધિકારો}

{\def\LTcaptype{none} % do not increment counter
\begin{longtable}[]{@{}lll@{}}
\toprule\noalign{}
વિશેષાધિકાર & વર્ણન & ઉદાહરણ \\
\midrule\noalign{}
\endhead
\bottomrule\noalign{}
\endlastfoot
\textbf{SELECT} & ડેટા વાંચવો & GRANT SELECT ON Student TO user1 \\
\textbf{INSERT} & નવા રેકોર્ડ્સ ઉમેરવા & GRANT INSERT ON Student TO user1 \\
\textbf{UPDATE} & હાલના રેકોર્ડ્સ સુધારવા & GRANT UPDATE ON Student TO
user1 \\
\textbf{DELETE} & રેકોર્ડ્સ દૂર કરવા & GRANT DELETE ON Student TO user1 \\
\textbf{ALL} & બધા વિશેષાધિકારો & GRANT ALL ON Student TO user1 \\
\end{longtable}
}

\textbf{ઉદાહરણો:}

\begin{verbatim}
{-{-} SELECT વિશેષાધિકાર આપો}
GRANT SELECT ON Student TO john;

{-{-} INSERT વિશેષાધિકાર દૂર કરો  }
REVOKE INSERT ON Student FROM john;
\end{verbatim}

\begin{itemize}
\tightlist
\item
  \textbf{WITH GRANT OPTION}: યુઝરને બીજાઓને વિશેષાધિકારો આપવાની મંજૂરી
\item
  \textbf{CASCADE}: જેમને આ વિશેષાધિકારો મળ્યા છે તે બધામાંથી વિશેષાધિકારો દૂર કરે
\end{itemize}

\end{solutionbox}
\begin{mnemonicbox}
``GRANT અધિકારો આપે, REVOKE અધિકારો દૂર કરે''

\end{mnemonicbox}
\begin{center}\rule{0.5\linewidth}{0.5pt}\end{center}

\subsection*{પ્રશ્ન 4(બ) [4
ગુણ]}\label{uxaaauxab0uxab6uxaa8-4uxaac-4-uxa97uxaa3}

\textbf{SQL Views પર ટૂંકી નોંધ લખો}

\begin{solutionbox}

\textbf{વ્યાખ્યા}: વ્યૂ એ SQL સ્ટેટમેન્ટના પરિણામ આધારિત વર્ચ્યુઅલ ટેબલ છે જેમાં
વાસ્તવિક ટેબલની જેમ પંક્તિઓ અને કૉલમ્સ હોય છે.

\textbf{ટેબલ: વ્યૂની લાક્ષણિકતાઓ}

{\def\LTcaptype{none} % do not increment counter
\begin{longtable}[]{@{}
  >{\raggedright\arraybackslash}p{(\linewidth - 4\tabcolsep) * \real{0.2667}}
  >{\raggedright\arraybackslash}p{(\linewidth - 4\tabcolsep) * \real{0.4333}}
  >{\raggedright\arraybackslash}p{(\linewidth - 4\tabcolsep) * \real{0.3000}}@{}}
\toprule\noalign{}
\begin{minipage}[b]{\linewidth}\raggedright
પાસું
\end{minipage} & \begin{minipage}[b]{\linewidth}\raggedright
વર્ણન
\end{minipage} & \begin{minipage}[b]{\linewidth}\raggedright
ઉદાહરણ
\end{minipage} \\
\midrule\noalign{}
\endhead
\bottomrule\noalign{}
\endlastfoot
\textbf{વર્ચ્યુઅલ ટેબલ} & ફિઝિકલ રીતે ડેટા સ્ટોર કરતું નથી & CREATE VIEW
student\_view AS\ldots{} \\
\textbf{સિક્યોરિટી} & સંવેદનશીલ કૉલમ્સ છુપાવે & કર્મચારીઓમાંથી પગાર કૉલમ
છુપાવો \\
\textbf{સરળીકરણ} & જટિલ ક્વેરીઝ સરળ બનાવે & એક વ્યૂમાં અનેક ટેબલ્સ જોડો \\
\textbf{ડેટા ઇન્ડિપેન્ડન્સ} & મૂળ ટેબલમાં ફેરફારો યુઝર્સને અસર કરતા નથી & એપ્લિકેશન્સને
અસર કર્યા વિના ટેબલ સ્ટ્રક્ચર સુધારો \\
\end{longtable}
}

\textbf{ઉદાહરણ:}

\begin{verbatim}
CREATE VIEW active\_students AS
SELECT Student\_ID, Name, Age 
FROM Student 
WHERE Status = {Active};

{-{-} વ્યૂનો ઉપયોગ}
SELECT * FROM active\_students;
\end{verbatim}

\textbf{ફાયદાઓ:}

\begin{itemize}
\tightlist
\item
  \textbf{સિક્યોરિટી}: સંવેદનશીલ ડેટાની એક્સેસ મર્યાદિત કરે
\item
  \textbf{સરળતા}: અંતિમ યુઝર્સમાંથી જટિલ joins છુપાવે
\item
  \textbf{સુસંગતતા}: પ્રમાણિત ડેટા એક્સેસ
\end{itemize}

\end{solutionbox}
\begin{mnemonicbox}
``વ્યૂઝ એ ડેટાની વર્ચ્યુઅલ વિન્ડોઝ છે''

\end{mnemonicbox}
\begin{center}\rule{0.5\linewidth}{0.5pt}\end{center}

\subsection*{પ્રશ્ન 4(ક) [7
ગુણ]}\label{uxaaauxab0uxab6uxaa8-4uxa95-7-uxa97uxaa3}

\textbf{નોર્મલાઇઝેશન શું છે? ઉદાહરણ સાથે 2NF સમજાવો}

\begin{solutionbox}

\textbf{નોર્મલાઇઝેશન}: રિડન્ડન્સી ઘટાડવા અને મોટા ટેબલ્સને નાના સંબંધિત ટેબલ્સમાં
વિભાજિત કરીને ડેટા ઇન્ટેગ્રિટી સુધારવા માટે ડેટાબેસ ગોઠવવાની પ્રક્રિયા.

\textbf{2NF (સેકન્ડ નોર્મલ ફોર્મ)}:

\begin{itemize}
\tightlist
\item
  1NF માં હોવું જોઈએ
\item
  પાર્શિયલ ફંક્શનલ ડિપેન્ડન્સીઝ દૂર કરવી
\item
  નોન-કી એટ્રિબ્યુટ્સ સંપૂર્ણ પ્રાઇમરી કી પર આધાર રાખવા જોઈએ
\end{itemize}

\textbf{ઉદાહરણ - અનોર્મલાઇઝ્ડ ટેબલ:}

{\def\LTcaptype{none} % do not increment counter
\begin{longtable}[]{@{}lllll@{}}
\toprule\noalign{}
Student\_ID & Course\_ID & Student\_Name & Course\_Name & Instructor \\
\midrule\noalign{}
\endhead
\bottomrule\noalign{}
\endlastfoot
101 & C1 & જોન & DBMS & ડૉ. સ્મિથ \\
101 & C2 & જોન & Java & ડૉ. જોન્સ \\
102 & C1 & મેરી & DBMS & ડૉ. સ્મિથ \\
\end{longtable}
}

\textbf{સમસ્યાઓ:}

\begin{itemize}
\tightlist
\item
  Student\_Name માત્ર Student\_ID પર આધાર રાખે છે (પાર્શિયલ ડિપેન્ડન્સી)
\item
  Course\_Name અને Instructor માત્ર Course\_ID પર આધાર રાખે છે
\end{itemize}

\textbf{2NF પછી:}

\textbf{Student ટેબલ:}

{\def\LTcaptype{none} % do not increment counter
\begin{longtable}[]{@{}ll@{}}
\toprule\noalign{}
Student\_ID & Student\_Name \\
\midrule\noalign{}
\endhead
\bottomrule\noalign{}
\endlastfoot
101 & જોન \\
102 & મેરી \\
\end{longtable}
}

\textbf{Course ટેબલ:}

{\def\LTcaptype{none} % do not increment counter
\begin{longtable}[]{@{}lll@{}}
\toprule\noalign{}
Course\_ID & Course\_Name & Instructor \\
\midrule\noalign{}
\endhead
\bottomrule\noalign{}
\endlastfoot
C1 & DBMS & ડૉ. સ્મિથ \\
C2 & Java & ડૉ. જોન્સ \\
\end{longtable}
}

\textbf{Enrollment ટેબલ:}

{\def\LTcaptype{none} % do not increment counter
\begin{longtable}[]{@{}ll@{}}
\toprule\noalign{}
Student\_ID & Course\_ID \\
\midrule\noalign{}
\endhead
\bottomrule\noalign{}
\endlastfoot
101 & C1 \\
101 & C2 \\
102 & C1 \\
\end{longtable}
}

\textbf{ફાયદાઓ:}

\begin{itemize}
\tightlist
\item
  \textbf{રિડન્ડન્સી દૂર કરે}: વિદ્યાર્થીના નામ પુનરાવર્તન નથી
\item
  \textbf{સ્ટોરેજ ઘટાડે}: ઓછો ડુપ્લિકેટ ડેટા
\item
  \textbf{સુસંગતતા સુધારે}: વિદ્યાર્થીનું નામ એક જ જગ્યાએ અપડેટ કરો
\end{itemize}

\end{solutionbox}
\begin{mnemonicbox}
``2NF = કોઈ પાર્શિયલ ડિપેન્ડન્સીઝ નહીં''

\end{mnemonicbox}
\begin{center}\rule{0.5\linewidth}{0.5pt}\end{center}

\subsection*{પ્રશ્ન 4(અ) અથવા [3
ગુણ]}\label{uxaaauxab0uxab6uxaa8-4uxa85-uxa85uxaa5uxab5-3-uxa97uxaa3}

\textbf{SQL માં Group By Clause નું ઉદાહરણ આપો}

\begin{solutionbox}

\textbf{GROUP BY Clause}: નિર્દિષ્ટ કૉલમ્સમાં સમાન વેલ્યુઝ ધરાવતી પંક્તિઓને ગ્રુપ કરે
છે અને દરેક ગ્રુપ પર એગ્રીગેટ ફંક્શન્સની મંજૂરી આપે છે.

\textbf{ટેબલ: GROUP BY ઉપયોગ}

{\def\LTcaptype{none} % do not increment counter
\begin{longtable}[]{@{}lll@{}}
\toprule\noalign{}
હેતુ & ફંક્શન & ઉદાહરણ \\
\midrule\noalign{}
\endhead
\bottomrule\noalign{}
\endlastfoot
\textbf{ગણતરી} & COUNT() & વિભાગ દીઠ વિદ્યાર્થીઓની ગણતરી \\
\textbf{સરવાળો} & SUM() & વિભાગ દીઠ કુલ પગાર \\
\textbf{સરેરાશ} & AVG() & કોર્સ દીઠ સરેરાશ ગુણ \\
\textbf{મિન/મેક્સ શોધવું} & MIN()/MAX() & વિભાગ દીઠ સર્વોચ્ચ પગાર \\
\end{longtable}
}

\textbf{ઉદાહરણ:}

\begin{verbatim}
SELECT Department, COUNT(*) as Total\_Students, AVG(Marks) as Avg\_Marks
FROM Student 
GROUP BY Department;
\end{verbatim}

\textbf{પરિણામ:}

{\def\LTcaptype{none} % do not increment counter
\begin{longtable}[]{@{}lll@{}}
\toprule\noalign{}
Department & Total\_Students & Avg\_Marks \\
\midrule\noalign{}
\endhead
\bottomrule\noalign{}
\endlastfoot
IT & 25 & 78.5 \\
CS & 30 & 82.1 \\
\end{longtable}
}

\begin{itemize}
\tightlist
\item
  \textbf{ગ્રુપ્સ}: દરેક વિભાગ માટે અલગ ગ્રુપ બનાવે
\item
  \textbf{એગ્રીગેટ્સ}: દરેક ગ્રુપ માટે કાઉન્ટ અને સરેરાશ ગણે
\end{itemize}

\end{solutionbox}
\begin{mnemonicbox}
``GROUP BY સમરી રિપોર્ટ્સ બનાવે''

\end{mnemonicbox}
\begin{center}\rule{0.5\linewidth}{0.5pt}\end{center}

\subsection*{પ્રશ્ન 4(બ) અથવા [4
ગુણ]}\label{uxaaauxab0uxab6uxaa8-4uxaac-uxa85uxaa5uxab5-4-uxa97uxaa3}

\textbf{ઉદાહરણ સાથે SQL માં Set Operators નું વર્ણન કરો}

\begin{solutionbox}

\textbf{Set Operators}: બે અથવા વધુ SELECT સ્ટેટમેન્ટ્સના પરિણામોને જોડે છે.

\textbf{ટેબલ: SQL Set Operators}

{\def\LTcaptype{none} % do not increment counter
\begin{longtable}[]{@{}
  >{\raggedright\arraybackslash}p{(\linewidth - 6\tabcolsep) * \real{0.2222}}
  >{\raggedright\arraybackslash}p{(\linewidth - 6\tabcolsep) * \real{0.2889}}
  >{\raggedright\arraybackslash}p{(\linewidth - 6\tabcolsep) * \real{0.2889}}
  >{\raggedright\arraybackslash}p{(\linewidth - 6\tabcolsep) * \real{0.2000}}@{}}
\toprule\noalign{}
\begin{minipage}[b]{\linewidth}\raggedright
ઓપરેટર
\end{minipage} & \begin{minipage}[b]{\linewidth}\raggedright
વર્ણન
\end{minipage} & \begin{minipage}[b]{\linewidth}\raggedright
આવશ્યકતા
\end{minipage} & \begin{minipage}[b]{\linewidth}\raggedright
ઉદાહરણ
\end{minipage} \\
\midrule\noalign{}
\endhead
\bottomrule\noalign{}
\endlastfoot
\textbf{UNION} & પરિણામો જોડે, ડુપ્લિકેટ્સ દૂર કરે & સમાન કૉલમ સ્ટ્રક્ચર & SELECT
name FROM students UNION SELECT name FROM teachers \\
\textbf{UNION ALL} & પરિણામો જોડે, ડુપ્લિકેટ્સ રાખે & સમાન કૉલમ સ્ટ્રક્ચર &
SELECT name FROM students UNION ALL SELECT name FROM alumni \\
\textbf{INTERSECT} & સામાન્ય રેકોર્ડ્સ રિટર્ન કરે & સમાન કૉલમ સ્ટ્રક્ચર & SELECT
course FROM current\_courses INTERSECT SELECT course FROM
popular\_courses \\
\textbf{MINUS} & પહેલી ક્વેરીમાં છે પણ બીજીમાં નથી & સમાન કૉલમ સ્ટ્રક્ચર & SELECT
student\_id FROM enrolled MINUS SELECT student\_id FROM graduated \\
\end{longtable}
}

\textbf{ઉદાહરણ:}

\begin{verbatim}
{-{-} વિદ્યાર્થીઓ જે શિક્ષકો પણ છે}
SELECT name FROM students
INTERSECT
SELECT name FROM teachers;

{-{-} યુનિવર્સિટીના બધા લોકો}
SELECT name, {Student} as type FROM students
UNION
SELECT name, {Teacher} as type FROM teachers;
\end{verbatim}

\textbf{નિયમો:}

\begin{itemize}
\tightlist
\item
  \textbf{કૉલમ કાઉન્ટ}: બધી ક્વેરીઝમાં સમાન હોવી જોઈએ
\item
  \textbf{ડેટા ટાઇપ્સ}: અનુરૂપ કૉલમ્સમાં સુસંગત ટાઇપ્સ હોવા જોઈએ
\item
  \textbf{ઓર્ડર}: ORDER BY માત્ર અંતે ઉપયોગ કરી શકાય
\end{itemize}

\end{solutionbox}
\begin{mnemonicbox}
``Set operators ડેટાને યુનાઇટ, ઇન્ટરસેક્ટ અને સબ્ટ્રેક્ટ કરે''

\end{mnemonicbox}
\begin{center}\rule{0.5\linewidth}{0.5pt}\end{center}

\subsection*{પ્રશ્ન 4(ક) અથવા [7
ગુણ]}\label{uxaaauxab0uxab6uxaa8-4uxa95-uxa85uxaa5uxab5-7-uxa97uxaa3}

\textbf{નોર્મલાઇઝેશનના મહત્વને ન્યાયી ઠેરવો. ઉદાહરણ સાથે 1NF સમજાવો}

\begin{solutionbox}

\textbf{નોર્મલાઇઝેશનનું મહત્વ:}

\textbf{ટેબલ: નોર્મલાઇઝેશનના ફાયદાઓ}

{\def\LTcaptype{none} % do not increment counter
\begin{longtable}[]{@{}
  >{\raggedright\arraybackslash}p{(\linewidth - 4\tabcolsep) * \real{0.3000}}
  >{\raggedright\arraybackslash}p{(\linewidth - 4\tabcolsep) * \real{0.4333}}
  >{\raggedright\arraybackslash}p{(\linewidth - 4\tabcolsep) * \real{0.2667}}@{}}
\toprule\noalign{}
\begin{minipage}[b]{\linewidth}\raggedright
ફાયદો
\end{minipage} & \begin{minipage}[b]{\linewidth}\raggedright
વર્ણન
\end{minipage} & \begin{minipage}[b]{\linewidth}\raggedright
અસર
\end{minipage} \\
\midrule\noalign{}
\endhead
\bottomrule\noalign{}
\endlastfoot
\textbf{રિડન્ડન્સી દૂર કરે} & ડુપ્લિકેટ ડેટા સ્ટોરેજ ઘટાડે & સ્ટોરેજ સ્પેસ બચાવે \\
\textbf{એનોમલીઝ અટકાવે} & ઇન્સર્શન, ડિલીશન, અપડેટ સમસ્યાઓ ટાળે & ડેટા સુસંગતતા
જાળવે \\
\textbf{ઇન્ટેગ્રિટી સુધારે} & ડેટાની સચોટતા સુનિશ્ચિત કરે & વિશ્વસનીય ઇન્ફોર્મેશન
સિસ્ટમ \\
\textbf{લવચીક ડિઝાઇન} & સુધારવા અને વિસ્તારવામાં સરળ & બિઝનેસ ફેરફારોને અનુકૂળ \\
\end{longtable}
}

\textbf{1NF (ફર્સ્ટ નોર્મલ ફોર્મ)}:

\begin{itemize}
\tightlist
\item
  સમાન ટેબલમાંથી ડુપ્લિકેટ કૉલમ્સ દૂર કરો
\item
  સંબંધિત ડેટા માટે અલગ ટેબલ્સ બનાવો
\item
  દરેક સેલમાં એક વેલ્યુ હોય (એટોમિક વેલ્યુઝ)
\end{itemize}

\textbf{ઉદાહરણ - અનોર્મલાઇઝ્ડ ટેબલ:}

{\def\LTcaptype{none} % do not increment counter
\begin{longtable}[]{@{}lll@{}}
\toprule\noalign{}
Student\_ID & Name & Subjects \\
\midrule\noalign{}
\endhead
\bottomrule\noalign{}
\endlastfoot
101 & જોન & ગણિત, વિજ્ઞાન, અંગ્રેજી \\
102 & મેરી & વિજ્ઞાન, ઇતિહાસ \\
\end{longtable}
}

\textbf{સમસ્યાઓ:}

\begin{itemize}
\tightlist
\item
  Subjects કૉલમમાં અનેક વેલ્યુઝ છે
\item
  વિશિષ્ટ વિષયો શોધવા મુશ્કેલ
\item
  વિષયો ઉમેરવા/દૂર કરવામાં અપડેટ એનોમલીઝ
\end{itemize}

\textbf{1NF પછી:}

\textbf{Student ટેબલ:}

{\def\LTcaptype{none} % do not increment counter
\begin{longtable}[]{@{}ll@{}}
\toprule\noalign{}
Student\_ID & Name \\
\midrule\noalign{}
\endhead
\bottomrule\noalign{}
\endlastfoot
101 & જોન \\
102 & મેરી \\
\end{longtable}
}

\textbf{Student\_Subject ટેબલ:}

{\def\LTcaptype{none} % do not increment counter
\begin{longtable}[]{@{}ll@{}}
\toprule\noalign{}
Student\_ID & Subject \\
\midrule\noalign{}
\endhead
\bottomrule\noalign{}
\endlastfoot
101 & ગણિત \\
101 & વિજ્ઞાન \\
101 & અંગ્રેજી \\
102 & વિજ્ઞાન \\
102 & ઇતિહાસ \\
\end{longtable}
}

\textbf{ફાયદાઓ:}

\begin{itemize}
\tightlist
\item
  \textbf{એટોમિક વેલ્યુઝ}: દરેક સેલમાં એક વેલ્યુ
\item
  \textbf{લવચીક ક્વેરીઝ}: વિશિષ્ટ વિષયો અભ્યાસ કરતા વિદ્યાર્થીઓ સરળતાથી શોધો
\item
  \textbf{સરળ અપડેટ્સ}: બીજા ડેટાને અસર કર્યા વિના વિષયો ઉમેરો/દૂર કરો
\end{itemize}

\end{solutionbox}
\begin{mnemonicbox}
``1NF = એક સેલ દીઠ એક વેલ્યુ, કોઈ રિપીટિંગ ગ્રુપ્સ નહીં''

\end{mnemonicbox}
\begin{center}\rule{0.5\linewidth}{0.5pt}\end{center}

\subsection*{પ્રશ્ન 5(અ) [3
ગુણ]}\label{uxaaauxab0uxab6uxaa8-5uxa85-3-uxa97uxaa3}

\textbf{ટ્રાન્ઝેક્શન મેનેજમેન્ટમાં Serializability સમજાવો}

\begin{solutionbox}

\textbf{Serializability}: એ ગુણધર્મ છે જે સુનિશ્ચિત કરે છે કે ટ્રાન્ઝેક્શન્સનું એકસાથે
એક્ઝિક્યુશન તે ટ્રાન્ઝેક્શન્સના કોઈ સીરિયલ એક્ઝિક્યુશન જેવું જ પરિણામ આપે.

\textbf{ટેબલ: Serializability ના પ્રકારો}

{\def\LTcaptype{none} % do not increment counter
\begin{longtable}[]{@{}
  >{\raggedright\arraybackslash}p{(\linewidth - 4\tabcolsep) * \real{0.2222}}
  >{\raggedright\arraybackslash}p{(\linewidth - 4\tabcolsep) * \real{0.4815}}
  >{\raggedright\arraybackslash}p{(\linewidth - 4\tabcolsep) * \real{0.2963}}@{}}
\toprule\noalign{}
\begin{minipage}[b]{\linewidth}\raggedright
પ્રકાર
\end{minipage} & \begin{minipage}[b]{\linewidth}\raggedright
વર્ણન
\end{minipage} & \begin{minipage}[b]{\linewidth}\raggedright
પદ્ધતિ
\end{minipage} \\
\midrule\noalign{}
\endhead
\bottomrule\noalign{}
\endlastfoot
\textbf{Conflict Serializability} & કોન્ફ્લિક્ટિંગ ઓપરેશન્સ આધારિત & પ્રિસિડન્સ
ગ્રાફ \\
\textbf{View Serializability} & રીડ-રાઇટ પેટર્ન આધારિત & વ્યૂ ઇક્વિવેલન્સ \\
\end{longtable}
}

\textbf{ઉદાહરણ:} Transaction T1: R(A), W(A), R(B), W(B) Transaction T2:
R(A), W(A), R(B), W(B)

\textbf{સીરિયલ શેડ્યુલ:} T1 \rightarrow T2 અથવા T2 \rightarrow T1 \textbf{કોન્કરન્ટ શેડ્યુલ:}
ઇન્ટરલીવ્ડ ઓપરેશન્સ

\begin{itemize}
\tightlist
\item
  \textbf{કોન્ફ્લિક્ટ ઓપરેશન્સ}: સમાન ડેટા આઇટમ પરના ઓપરેશન્સ જ્યાં ઓછામાં ઓછું એક
  રાઇટ હોય
\item
  \textbf{સીરિયલાઇઝેબલ શેડ્યુલ}: કોઈ સીરિયલ શેડ્યુલ સમકક્ષ
\item
  \textbf{નોન-સીરિયલાઇઝેબલ}: અસુસંગત ડેટાબેસ સ્ટેટ તરફ દોરી શકે
\end{itemize}

\end{solutionbox}
\begin{mnemonicbox}
``Serializability ટ્રાન્ઝેક્શન કન્સિસ્ટન્સી સુનિશ્ચિત કરે''

\end{mnemonicbox}
\begin{center}\rule{0.5\linewidth}{0.5pt}\end{center}

\subsection*{પ્રશ્ન 5(બ) [4
ગુણ]}\label{uxaaauxab0uxab6uxaa8-5uxaac-4-uxa97uxaa3}

\textbf{ઉદાહરણ સાથે પાર્શિયલ ફંક્શનલ ડિપેન્ડન્સી નું વર્ણન કરો}

\begin{solutionbox}

\textbf{પાર્શિયલ ફંક્શનલ ડિપેન્ડન્સી}: જ્યારે કોઈ નોન-કી એટ્રિબ્યુટ કમ્પોઝિટ પ્રાઇમરી
કીના માત્ર એક ભાગ પર ફંક્શનલી ડિપેન્ડન્ટ હોય.

\textbf{ટેબલ: ફંક્શનલ ડિપેન્ડન્સીના પ્રકારો}

{\def\LTcaptype{none} % do not increment counter
\begin{longtable}[]{@{}
  >{\raggedright\arraybackslash}p{(\linewidth - 4\tabcolsep) * \real{0.2222}}
  >{\raggedright\arraybackslash}p{(\linewidth - 4\tabcolsep) * \real{0.4444}}
  >{\raggedright\arraybackslash}p{(\linewidth - 4\tabcolsep) * \real{0.3333}}@{}}
\toprule\noalign{}
\begin{minipage}[b]{\linewidth}\raggedright
પ્રકાર
\end{minipage} & \begin{minipage}[b]{\linewidth}\raggedright
વ્યાખ્યા
\end{minipage} & \begin{minipage}[b]{\linewidth}\raggedright
ઉદાહરણ
\end{minipage} \\
\midrule\noalign{}
\endhead
\bottomrule\noalign{}
\endlastfoot
\textbf{ફુલ ડિપેન્ડન્સી} & સંપૂર્ણ પ્રાઇમરી કી પર આધાર & (Student\_ID,
Course\_ID) \rightarrow Grade \\
\textbf{પાર્શિયલ ડિપેન્ડન્સી} & પ્રાઇમરી કીના ભાગ પર આધાર & (Student\_ID,
Course\_ID) \rightarrow Student\_Name \\
\end{longtable}
}

\textbf{ઉદાહરણ:} \textbf{Enrollment ટેબલ:} પ્રાઇમરી કી: (Student\_ID,
Course\_ID)

{\def\LTcaptype{none} % do not increment counter
\begin{longtable}[]{@{}lllll@{}}
\toprule\noalign{}
Student\_ID & Course\_ID & Student\_Name & Course\_Name & Grade \\
\midrule\noalign{}
\endhead
\bottomrule\noalign{}
\endlastfoot
101 & C1 & જોન & DBMS & A \\
101 & C2 & જોન & Java & B \\
\end{longtable}
}

\textbf{પાર્શિયલ ડિપેન્ડન્સીઝ:}

\begin{itemize}
\tightlist
\item
  Student\_ID \rightarrow Student\_Name (Student\_Name માત્ર Student\_ID પર આધાર
  રાખે)
\item
  Course\_ID \rightarrow Course\_Name (Course\_Name માત્ર Course\_ID પર આધાર રાખે)
\end{itemize}

\textbf{સમસ્યાઓ:}

\begin{itemize}
\tightlist
\item
  \textbf{અપડેટ એનોમલી}: વિદ્યાર્થીનું નામ બદલવા માટે અનેક અપડેટ્સ જરૂરી
\item
  \textbf{ઇન્સર્શન એનોમલી}: કોર્સમાં નોંધણી કર્યા વિના વિદ્યાર્થી ઉમેરી શકાતો નથી
\item
  \textbf{ડિલીશન એનોમલી}: નોંધણી ડિલીટ કરવાથી વિદ્યાર્થીની માહિતી ખોવાઈ શકે
\end{itemize}

\textbf{સોલ્યુશન}: પાર્શિયલ ડિપેન્ડન્સીઝ દૂર કરીને 2NF માં નોર્મલાઇઝ કરો

\end{solutionbox}
\begin{mnemonicbox}
``પાર્શિયલ ડિપેન્ડન્સી = કીનો ભાગ એટ્રિબ્યુટ નક્કી કરે''

\end{mnemonicbox}
\begin{center}\rule{0.5\linewidth}{0.5pt}\end{center}

\subsection*{પ્રશ્ન 5(ક) [7
ગુણ]}\label{uxaaauxab0uxab6uxaa8-5uxa95-7-uxa97uxaa3}

\textbf{ટ્રાન્ઝેક્શન મેનેજમેન્ટમાં ઉદાહરણ સાથે Locking Mechanism પર ટૂંકી નોંધ લખો}

\begin{solutionbox}

\textbf{Locking Mechanism}: કન્કરન્સી કંટ્રોલ ટેકનીક જે ટ્રાન્ઝેક્શન એક્ઝિક્યુશન
દરમિયાન ડેટા આઇટમ્સની એકસાથે એક્સેસ અટકાવે છે.

\textbf{ટેબલ: Locks ના પ્રકારો}

{\def\LTcaptype{none} % do not increment counter
\begin{longtable}[]{@{}
  >{\raggedright\arraybackslash}p{(\linewidth - 4\tabcolsep) * \real{0.3548}}
  >{\raggedright\arraybackslash}p{(\linewidth - 4\tabcolsep) * \real{0.4194}}
  >{\raggedright\arraybackslash}p{(\linewidth - 4\tabcolsep) * \real{0.2258}}@{}}
\toprule\noalign{}
\begin{minipage}[b]{\linewidth}\raggedright
Lock પ્રકાર
\end{minipage} & \begin{minipage}[b]{\linewidth}\raggedright
વર્ણન
\end{minipage} & \begin{minipage}[b]{\linewidth}\raggedright
ઉપયોગ
\end{minipage} \\
\midrule\noalign{}
\endhead
\bottomrule\noalign{}
\endlastfoot
\textbf{Shared Lock (S)} & અનેક ટ્રાન્ઝેક્શન્સ વાંચી શકે & રીડ ઓપરેશન્સ \\
\textbf{Exclusive Lock (X)} & માત્ર એક ટ્રાન્ઝેક્શન એક્સેસ કરી શકે & રાઇટ
ઓપરેશન્સ \\
\textbf{Intention Lock} & નિચલા લેવલે lock કરવાનો ઇરાદો દર્શાવે & હાયરાર્કિકલ
લોકિંગ \\
\end{longtable}
}

\textbf{Two-Phase Locking (2PL) પ્રોટોકોલ:}

\begin{enumerate}
\tightlist
\item
  \textbf{ગ્રોઇંગ ફેઝ}: locks એક્વાયર કરો, કોઈ lock રિલીઝ ન કરો
\item
  \textbf{શ્રિંકિંગ ફેઝ}: locks રિલીઝ કરો, નવા locks એક્વાયર ન કરો
\end{enumerate}

\textbf{ઉદાહરણ:}

\begin{verbatim}
Transaction T1: Read(A), Write(A), Read(B), Write(B)
Transaction T2: Read(A), Write(A), Read(C), Write(C)

T1: S-lock(A), Read(A), X-lock(A), Write(A), S-lock(B), Read(B), X-lock(B), Write(B), Unlock(A), Unlock(B)
T2: A માટે રાહ જુએ, S-lock(A), Read(A), X-lock(A), Write(A), S-lock(C), Read(C), X-lock(C), Write(C), Unlock(A), Unlock(C)
\end{verbatim}

\textbf{Lock Compatibility Matrix:}

{\def\LTcaptype{none} % do not increment counter
\begin{longtable}[]{@{}lll@{}}
\toprule\noalign{}
વર્તમાન/માંગેલ & S & X \\
\midrule\noalign{}
\endhead
\bottomrule\noalign{}
\endlastfoot
\textbf{S} & ✓ & ✗ \\
\textbf{X} & ✗ & ✗ \\
\end{longtable}
}

\textbf{સમસ્યાઓ:}

\begin{itemize}
\tightlist
\item
  \textbf{ડેડલોક}: બે ટ્રાન્ઝેક્શન્સ એકબીજાના locks માટે રાહ જુએ
\item
  \textbf{સ્ટાર્વેશન}: ટ્રાન્ઝેક્શન lock માટે અનંત રાહ જુએ
\end{itemize}

\textbf{સોલ્યુશન્સ:}

\begin{itemize}
\tightlist
\item
  \textbf{ડેડલોક ડિટેક્શન}: wait-for ગ્રાફનો ઉપયોગ
\item
  \textbf{ડેડલોક પ્રિવેન્શન}: ટાઇમસ્ટેમ્પ-આધારિત પ્રોટોકોલ્સ
\end{itemize}

\end{solutionbox}
\begin{mnemonicbox}
``Locking કોન્કરન્ટ કોન્ફ્લિક્ટ્સ અટકાવે''

\end{mnemonicbox}
\begin{center}\rule{0.5\linewidth}{0.5pt}\end{center}

\subsection*{પ્રશ્ન 5(અ) અથવા [3
ગુણ]}\label{uxaaauxab0uxab6uxaa8-5uxa85-uxa85uxaa5uxab5-3-uxa97uxaa3}

\textbf{ટ્રાન્ઝેક્શન મેનેજમેન્ટમાં ડેડલોક સમજાવો}

\begin{solutionbox}

\textbf{ડેડલોક}: એવી પરિસ્થિતિ જ્યાં બે અથવા વધુ ટ્રાન્ઝેક્શન્સ એકબીજાને locks રિલીઝ
કરવા માટે અનંત રાહ જુએ છે, ચક્રાકાર રાહની સ્થિતિ બનાવે છે.

\textbf{ટેબલ: ડેડલોકના ઘટકો}

{\def\LTcaptype{none} % do not increment counter
\begin{longtable}[]{@{}
  >{\raggedright\arraybackslash}p{(\linewidth - 4\tabcolsep) * \real{0.3333}}
  >{\raggedright\arraybackslash}p{(\linewidth - 4\tabcolsep) * \real{0.3939}}
  >{\raggedright\arraybackslash}p{(\linewidth - 4\tabcolsep) * \real{0.2727}}@{}}
\toprule\noalign{}
\begin{minipage}[b]{\linewidth}\raggedright
ઘટક
\end{minipage} & \begin{minipage}[b]{\linewidth}\raggedright
વર્ણન
\end{minipage} & \begin{minipage}[b]{\linewidth}\raggedright
ઉદાહરણ
\end{minipage} \\
\midrule\noalign{}
\endhead
\bottomrule\noalign{}
\endlastfoot
\textbf{મ્યુચ્યુઅલ એક્સક્લુઝન} & રિસોર્સ શેર કરી શકાતા નથી & એક્સક્લુઝિવ locks \\
\textbf{હોલ્ડ એન્ડ વેઇટ} & પ્રોસેસ રિસોર્સ પકડીને બીજાની રાહ જુએ & T1 A પકડે, B
ની રાહ જુએ \\
\textbf{નો પ્રીએમ્પ્શન} & રિસોર્સ બળજબરીથી છીનવી શકાતા નથી & Locks રદ કરી
શકાતા નથી \\
\textbf{સર્ક્યુલર વેઇટ} & પ્રોસેસોની ચક્રાકાર રાહની સાંકળ & T1\rightarrowT2\rightarrowT1 \\
\end{longtable}
}

\textbf{ઉદાહરણ:}

\begin{verbatim}
Transaction T1: Lock(A), Lock(B)
Transaction T2: Lock(B), Lock(A)

સમય 1: T1 ને Lock(A) મળે
સમય 2: T2 ને Lock(B) મળે 
સમય 3: T1 Lock(B) ની રાહ જુએ - T2 પાસે છે
સમય 4: T2 Lock(A) ની રાહ જુએ - T1 પાસે છે
પરિણામ: ડેડલોક!
\end{verbatim}

\textbf{ડિટેક્શન}: ચક્રો ઓળખવા માટે wait-for ગ્રાફનો ઉપયોગ \textbf{પ્રિવેન્શન}:
ટાઇમસ્ટેમ્પ ઓર્ડરિંગ અથવા wound-wait પ્રોટોકોલ્સનો ઉપયોગ

\end{solutionbox}
\begin{mnemonicbox}
``ડેડલોક = રિસોર્સ માટે ચક્રાકાર રાહ''

\end{mnemonicbox}
\begin{center}\rule{0.5\linewidth}{0.5pt}\end{center}

\subsection*{પ્રશ્ન 5(બ) અથવા [4
ગુણ]}\label{uxaaauxab0uxab6uxaa8-5uxaac-uxa85uxaa5uxab5-4-uxa97uxaa3}

\textbf{ઉદાહરણ સાથે ફુલ ફંક્શનલ ડિપેન્ડન્સી નું વર્ણન કરો}

\begin{solutionbox}

\textbf{ફુલ ફંક્શનલ ડિપેન્ડન્સી}: જ્યારે કોઈ નોન-કી એટ્રિબ્યુટ સંપૂર્ણ પ્રાઇમરી કી પર
ફંક્શનલી ડિપેન્ડન્ટ હોય (માત્ર તેના ભાગ પર નહીં).

\textbf{ટેબલ: ડિપેન્ડન્સી તુલના}

{\def\LTcaptype{none} % do not increment counter
\begin{longtable}[]{@{}
  >{\raggedright\arraybackslash}p{(\linewidth - 4\tabcolsep) * \real{0.2222}}
  >{\raggedright\arraybackslash}p{(\linewidth - 4\tabcolsep) * \real{0.4444}}
  >{\raggedright\arraybackslash}p{(\linewidth - 4\tabcolsep) * \real{0.3333}}@{}}
\toprule\noalign{}
\begin{minipage}[b]{\linewidth}\raggedright
પ્રકાર
\end{minipage} & \begin{minipage}[b]{\linewidth}\raggedright
વ્યાખ્યા
\end{minipage} & \begin{minipage}[b]{\linewidth}\raggedright
ઉદાહરણ
\end{minipage} \\
\midrule\noalign{}
\endhead
\bottomrule\noalign{}
\endlastfoot
\textbf{ફુલ ડિપેન્ડન્સી} & સંપૂર્ણ પ્રાઇમરી કી પર આધાર & (Student\_ID,
Course\_ID) \rightarrow Grade \\
\textbf{પાર્શિયલ ડિપેન્ડન્સી} & પ્રાઇમરી કીના ભાગ પર આધાર & (Student\_ID,
Course\_ID) \rightarrow Student\_Name \\
\end{longtable}
}

\textbf{ઉદાહરણ:} \textbf{Enrollment ટેબલ:} પ્રાઇમરી કી: (Student\_ID,
Course\_ID)

{\def\LTcaptype{none} % do not increment counter
\begin{longtable}[]{@{}llll@{}}
\toprule\noalign{}
Student\_ID & Course\_ID & Grade & Hours \\
\midrule\noalign{}
\endhead
\bottomrule\noalign{}
\endlastfoot
101 & C1 & A & 4 \\
101 & C2 & B & 3 \\
102 & C1 & B & 4 \\
\end{longtable}
}

\textbf{ફુલ ફંક્શનલ ડિપેન્ડન્સીઝ:}

\begin{itemize}
\tightlist
\item
  (Student\_ID, Course\_ID) \rightarrow Grade ✓
\item
  (Student\_ID, Course\_ID) \rightarrow Hours ✓
\end{itemize}

\textbf{સમજૂતી:}

\begin{itemize}
\tightlist
\item
  \textbf{Grade} Student\_ID અને Course\_ID બંને પર આધાર રાખે (વિશિષ્ટ
  વિદ્યાર્થી વિશિષ્ટ કોર્સમાં)
\item
  \textbf{Hours} પણ બંને પર આધાર રાખે (વિશિષ્ટ કોર્સમાં વિદ્યાર્થીના કલાકો)
\item
  માત્ર Student\_ID થી Grade નક્કી કરી શકાતો નથી
\item
  માત્ર Course\_ID થી Grade નક્કી કરી શકાતો નથી
\end{itemize}

\textbf{ફાયદાઓ:}

\begin{itemize}
\tightlist
\item
  \textbf{કોઈ અપડેટ એનોમલીઝ નહીં}: ફેરફારો માત્ર સંબંધિત રેકોર્ડ્સને અસર કરે
\item
  \textbf{યોગ્ય નોર્મલાઇઝેશન}: 2NF આવશ્યકતાઓને સપોર્ટ કરે
\item
  \textbf{ડેટા ઇન્ટેગ્રિટી}: સચોટ સંબંધો સુનિશ્ચિત કરે
\end{itemize}

\end{solutionbox}
\begin{mnemonicbox}
``ફુલ ડિપેન્ડન્સીને સંપૂર્ણ કીની જરૂર''

\end{mnemonicbox}
\begin{center}\rule{0.5\linewidth}{0.5pt}\end{center}

\subsection*{પ્રશ્ન 5(ક) અથવા [7
ગુણ]}\label{uxaaauxab0uxab6uxaa8-5uxa95-uxa85uxaa5uxab5-7-uxa97uxaa3}

\textbf{ઉદાહરણ સાથે ટ્રાન્ઝેક્શનના ACID ગુણધર્મો સમજાવો}

\begin{solutionbox}

\textbf{ACID ગુણધર્મો}: ડેટાબેસ ટ્રાન્ઝેક્શનની વિશ્વસનીયતાની બાંયધરી આપતા ચાર મૂળભૂત
ગુણધર્મો.

\textbf{ટેબલ: ACID ગુણધર્મો}

{\def\LTcaptype{none} % do not increment counter
\begin{longtable}[]{@{}
  >{\raggedright\arraybackslash}p{(\linewidth - 4\tabcolsep) * \real{0.3125}}
  >{\raggedright\arraybackslash}p{(\linewidth - 4\tabcolsep) * \real{0.4062}}
  >{\raggedright\arraybackslash}p{(\linewidth - 4\tabcolsep) * \real{0.2812}}@{}}
\toprule\noalign{}
\begin{minipage}[b]{\linewidth}\raggedright
ગુણધર્મ
\end{minipage} & \begin{minipage}[b]{\linewidth}\raggedright
વર્ણન
\end{minipage} & \begin{minipage}[b]{\linewidth}\raggedright
ઉદાહરણ
\end{minipage} \\
\midrule\noalign{}
\endhead
\bottomrule\noalign{}
\endlastfoot
\textbf{Atomicity} & બધું અથવા કશું નહીં એક્ઝિક્યુશન & બેંક ટ્રાન્સફર: ડેબિટ અને ક્રેડિટ
બંને થવા જોઈએ \\
\textbf{Consistency} & ડેટાબેસ વેલિડ સ્ટેટમાં રહે & એકાઉન્ટ બેલેન્સ નેગેટિવ ન હોઈ
શકે \\
\textbf{Isolation} & ટ્રાન્ઝેક્શન્સ એકબીજામાં દખલ ન કરે & કોન્કરન્ટ ટ્રાન્ઝેક્શન્સ
સીક્વન્શિયલ લાગે \\
\textbf{Durability} & કમિટ થયેલા ફેરફારો કાયમી રહે & સિસ્ટમ ક્રેશ પછી પણ ડેટા
બચે \\
\end{longtable}
}

\textbf{વિગતવાર ઉદાહરણો:}

\textbf{Atomicity ઉદાહરણ:}

\begin{verbatim}
BEGIN TRANSACTION;
UPDATE Account SET Balance = Balance {-} 1000 WHERE AccNo = {A001};
UPDATE Account SET Balance = Balance + 1000 WHERE AccNo = {A002};
COMMIT;
\end{verbatim}

\emph{જો કોઈ પણ અપડેટ નિષ્ફળ જાય તો સંપૂર્ણ ટ્રાન્ઝેક્શન રોલબેક થાય}

\textbf{Consistency ઉદાહરણ:}

\begin{verbatim}
{-{-} પહેલાં: A001 = 5000, A002 = 3000, કુલ = 8000}
{-{-} A001 થી A002 માં 1000 ટ્રાન્સફર}
{-{-} પછી: A001 = 4000, A002 = 4000, કુલ = 8000}
{-{-} સિસ્ટમમાં કુલ પૈસા સમાન રહે}
\end{verbatim}

\textbf{Isolation ઉદાહરણ:}

\begin{verbatim}
T1: Read(A=100),

A=A+50, Write(A=150)

T2: Read(A=100),

A=A*2, Write(A=200)

સીરિયલ પરિણામ:

A=300 અથવા

A=250

આઇસોલેટેડ એક્ઝિક્યુશન આમાંથી એક પરિણામ આપવો જોઈએ
\end{verbatim}

\textbf{Durability ઉદાહરણ:}

\begin{verbatim}
COMMIT એક્ઝિક્યુટ થયા પછી, સિસ્ટમ ક્રેશ થયા છતાં,
ટ્રાન્સફર થયેલ રકમ ડેસ્ટિનેશન એકાઉન્ટમાં રહે
\end{verbatim}

\textbf{અમલીકરણ:}

\begin{itemize}
\tightlist
\item
  \textbf{Atomicity}: ટ્રાન્ઝેક્શન લોગ્સ અને રોલબેકનો ઉપયોગ
\item
  \textbf{Consistency}: કન્સ્ટ્રેન્ટ્સ અને ટ્રિગર્સનો ઉપયોગ
\item
  \textbf{Isolation}: લોકિંગ મેકેનિઝમ્સનો ઉપયોગ
\item
  \textbf{Durability}: રાઇટ-અહેડ લોગિંગનો ઉપયોગ
\end{itemize}

\end{solutionbox}
\begin{mnemonicbox}
``ACID ટ્રાન્ઝેક્શન્સને વિશ્વસનીય રાખે''

\end{mnemonicbox}

\end{document}
