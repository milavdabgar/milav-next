\documentclass[10pt,a4paper]{article}

% content/resources/templates/preamble.tex
\usepackage[margin=0.6in]{geometry}
\author{Milav Dabgar}
\usepackage{amsmath,amssymb,amsthm}
\usepackage{booktabs}
\usepackage{multirow}
\usepackage{xcolor}
\usepackage{tcolorbox}
\tcbuselibrary{breakable,skins}
\usepackage[colorlinks=true,linkcolor=blue]{hyperref}
\usepackage{titlesec}
\usepackage{enumitem}
\usepackage{tikz}
\usepackage{pgfplots}
\usepackage{circuitikz}
\usepackage[version=4]{mhchem}
\usepackage{longtable}
\usepackage{array}
\usepackage{float}
\usepackage{caption}
\usepackage{listings}

\lstset{
  basicstyle=\small\ttfamily,
  breaklines=true,
  breakatwhitespace=false,
  postbreak=\mbox{\textcolor{red}{$\hookrightarrow$}\space},
  float=false,
  numbers=left,
  numberstyle=\tiny\color{gray},
  numbersep=10pt,
  xleftmargin=2em,
  keywordstyle=\color{blue},
  commentstyle=\color{green!60!black},
  stringstyle=\color{purple},
  backgroundcolor=\color{gray!5},
  showstringspaces=false,
  tabsize=2,
  captionpos=b,
  keepspaces=true,
  columns=flexible
}

\pgfplotsset{compat=1.18}
\usetikzlibrary{shapes,arrows,positioning,calc,patterns,decorations.pathmorphing,decorations.markings,arrows.meta}

% Color scheme
\definecolor{headcolor}{RGB}{0,102,204}
\definecolor{keycolor}{RGB}{220,20,60}
\definecolor{solutioncolor}{RGB}{34,139,34}
\definecolor{mnemoniccolor}{RGB}{148,0,211}
\definecolor{codecolor}{RGB}{0,0,100}

% Spacing
\setlength{\parskip}{3pt}
\setlist[itemize]{nosep}
\setlist[enumerate]{nosep}

% Title formatting
\titleformat{\section}{\Large\bfseries\color{headcolor}}{\thesection}{1em}{}
\titleformat{\subsection}{\large\bfseries\color{headcolor}}{\thesubsection}{1em}{}

% Pandoc tightlist compatibility
\providecommand{\tightlist}{%
  \setlength{\itemsep}{0pt}\setlength{\parskip}{0pt}}

% Pandoc longtable compatibility
\newcounter{none}
\def\thenone{}


% content/resources/templates/gujarati-boxes.tex
\usepackage{fontspec}
\usepackage{polyglossia}

% Set Gujarati as main language (document is primarily in Gujarati)
% Note: gloss-gujarati.ldf doesn't exist in polyglossia, but it will use hyphenation patterns
\setdefaultlanguage{gujarati}
\setotherlanguage{english}

% Configure Gujarati font properly
% Use Language=Default to prevent polyglossia from trying to add language-specific features
% that don't exist for Gujarati, which causes "empty feature" warnings
\newfontfamily\gujaratifont[Script=Gujarati,AutoFakeBold=2.5,AutoFakeSlant=0.3]{Noto Sans Gujarati}
\setmainfont[Script=Gujarati,AutoFakeBold=2.5,AutoFakeSlant=0.3]{Noto Sans Gujarati}
% Use Noto Sans Gujarati for monospace to support Gujarati in text
\setmonofont[Scale=0.9]{Noto Sans Gujarati}

% Configure English to use the same font
\newfontfamily\englishfont[Script=Gujarati,AutoFakeBold=2.5,AutoFakeSlant=0.3]{Noto Sans Gujarati}

% Translations for polyglossia
\gappto\captionsgujarati{
  \renewcommand{\tablename}{કોષ્ટક}
  \renewcommand{\figurename}{આકૃતિ}
}

% Helper for TikZ nodes to ensure Gujarati font
\newcommand{\gu}[1]{{\gujaratifont #1}}

% Custom environments
\newtcolorbox{solutionbox}{
    breakable,
    enhanced,
    colback=solutioncolor!5!white,
    colframe=solutioncolor!75!black,
    fonttitle=\bfseries,
    title=જવાબ
}

\newtcolorbox{solutionboxnobreak}{
 colback=solutioncolor!5!white,
 colframe=solutioncolor!75!black,
 fonttitle=\bfseries,
 title=જવાબ
}

\newtcolorbox{keyformula}{
 breakable,
 enhanced,
 colback=keycolor!5!white,
 colframe=keycolor!75!black,
 fonttitle=\bfseries,
 title=રાસાયણિક સમીકરણ/સૂત્ર
}

\newtcolorbox{mnemonicbox}{
 breakable,
 enhanced,
 colback=mnemoniccolor!5!white,
 colframe=mnemoniccolor!75!black,
 fonttitle=\bfseries,
 title=મેમરી ટ્રીક
}


\begin{document}

\begin{center}
{\Huge\bfseries\color{headcolor} Subject Name (Gujarati)}\\[5pt]
{\LARGE 4331603 -- Winter 2024}\\[3pt]
{\large Semester 1 Study Material}\\[3pt]
{\normalsize\textit{Detailed Solutions and Explanations}}
\end{center}

\vspace{10pt}

\subsection*{પ્રશ્ન 1(અ) [3
ગુણ]}\label{uxaaauxab0uxab6uxaa8-1uxa85-3-uxa97uxaa3}

\textbf{Three-level ડેટાબેઝ આર્કિટેક્ચરને સમજાવો.}

\begin{solutionbox}

\textbf{ટેબલ:}

{\def\LTcaptype{none} % do not increment counter
\begin{longtable}[]{@{}
  >{\raggedright\arraybackslash}p{(\linewidth - 4\tabcolsep) * \real{0.2414}}
  >{\raggedright\arraybackslash}p{(\linewidth - 4\tabcolsep) * \real{0.4483}}
  >{\raggedright\arraybackslash}p{(\linewidth - 4\tabcolsep) * \real{0.3103}}@{}}
\toprule\noalign{}
\begin{minipage}[b]{\linewidth}\raggedright
સ્તર
\end{minipage} & \begin{minipage}[b]{\linewidth}\raggedright
વર્ણન
\end{minipage} & \begin{minipage}[b]{\linewidth}\raggedright
હેતુ
\end{minipage} \\
\midrule\noalign{}
\endhead
\bottomrule\noalign{}
\endlastfoot
\textbf{External Level} & યુઝર વ્યુઝ અને એપ્લિકેશન પ્રોગ્રામ્સ & વપરાશકર્તાઓ માટે
ડેટા abstraction \\
\textbf{Conceptual Level} & સંપૂર્ણ લોજિકલ સ્ટ્રક્ચર & સંસ્થાવ્યાપી ડેટા દૃશ્ય \\
\textbf{Internal Level} & ભૌતિક સ્ટોરેજ વિગતો & સ્ટોરેજ અને access methods \\
\end{longtable}
}

\textbf{ડાયાગ્રામ:}

\begin{center}
\textbf{Mermaid Diagram (Code)}
\begin{verbatim}
{Shaded}
{Highlighting}[]
graph LR
    A[External Level] {-{-}{} B[Conceptual Level]}
    B {-{-}{} C[Internal Level]}
    A1[User View 1] {-{-}{} A}
    A2[User View 2] {-{-}{} A}
    A3[User View n] {-{-}{} A}
    C {-{-}{} D[Physical Storage]}
{Highlighting}
{Shaded}
\end{verbatim}
\end{center}

\begin{itemize}
\tightlist
\item
  \textbf{External Level}: વ્યક્તિગત યુઝર વ્યુઝ અને વિશિષ્ટ એપ્લિકેશન જરૂરિયાતો
\item
  \textbf{Conceptual Level}: સ્ટોરેજ વિગતો વિના સંપૂર્ણ ડેટાબેઝ schema\\
\item
  \textbf{Internal Level}: ભૌતિક સ્ટોરેજ સ્ટ્રક્ચર્સ અને access paths
\end{itemize}

\end{solutionbox}
\begin{mnemonicbox}
``ECI - Every Computer Interface''

\end{mnemonicbox}
\subsection*{પ્રશ્ન 1(બ) [4
ગુણ]}\label{uxaaauxab0uxab6uxaa8-1uxaac-4-uxa97uxaa3}

\textbf{ઉદાહરણ સાથે Total Participation અને Partial Participation સમજાવો.}

\begin{solutionbox}

\textbf{ટેબલ:}

{\def\LTcaptype{none} % do not increment counter
\begin{longtable}[]{@{}
  >{\raggedright\arraybackslash}p{(\linewidth - 6\tabcolsep) * \real{0.3878}}
  >{\raggedright\arraybackslash}p{(\linewidth - 6\tabcolsep) * \real{0.2449}}
  >{\raggedright\arraybackslash}p{(\linewidth - 6\tabcolsep) * \real{0.1837}}
  >{\raggedright\arraybackslash}p{(\linewidth - 6\tabcolsep) * \real{0.1837}}@{}}
\toprule\noalign{}
\begin{minipage}[b]{\linewidth}\raggedright
Participation Type
\end{minipage} & \begin{minipage}[b]{\linewidth}\raggedright
વ્યાખ્યા
\end{minipage} & \begin{minipage}[b]{\linewidth}\raggedright
પ્રતીક
\end{minipage} & \begin{minipage}[b]{\linewidth}\raggedright
ઉદાહરણ
\end{minipage} \\
\midrule\noalign{}
\endhead
\bottomrule\noalign{}
\endlastfoot
\textbf{Total Participation} & દરેક entity એ ભાગ લેવો જ જોઈએ & Double line
& Student-Course enrollment \\
\textbf{Partial Participation} & કેટલીક entities ભાગ ન પણ લઈ શકે & Single
line & Employee-Department management \\
\end{longtable}
}

\textbf{ડાયાગ્રામ:}

\begin{verbatim}
erDiagram
    STUDENT ||{-{-}|| ENROLLMENT : "Total (must enroll)"}
    EMPLOYEE \|{-}{-}|| DEPARTMENT : "Partial (may not manage)"}
\end{verbatim}

\begin{itemize}
\tightlist
\item
  \textbf{Total Participation}: તમામ વિદ્યાર્થીઓએ ઓછામાં ઓછા એક કોર્સમાં
  નોંધણી કરાવવી જ જોઈએ
\item
  \textbf{Partial Participation}: બધા કર્મચારીઓ department ને manage કરતા
  નથી
\item
  \textbf{Double lines} total participation constraints દર્શાવે છે
\item
  \textbf{Single lines} partial participation relationships બતાવે છે
\end{itemize}

\end{solutionbox}
\begin{mnemonicbox}
``Total = Two lines, Partial = Plain line''

\end{mnemonicbox}
\subsection*{પ્રશ્ન 1(ક) [7
ગુણ]}\label{uxaaauxab0uxab6uxaa8-1uxa95-7-uxa97uxaa3}

\textbf{ફાઇલ મેનેજમેન્ટ સિસ્ટમ પર DBMS ના ફાયદા સમજાવો.}

\begin{solutionbox}

\textbf{ટેબલ:}

{\def\LTcaptype{none} % do not increment counter
\begin{longtable}[]{@{}lll@{}}
\toprule\noalign{}
ફાયદો & File System & DBMS \\
\midrule\noalign{}
\endhead
\bottomrule\noalign{}
\endlastfoot
\textbf{Data Redundancy} & ઉચ્ચ duplication & નિયંત્રિત redundancy \\
\textbf{Data Inconsistency} & સામાન્ય સમસ્યા & ડેટા integrity જાળવાઈ રહે \\
\textbf{Data Sharing} & મર્યાદિત sharing & Concurrent access સપોર્ટ \\
\textbf{સિક્યુરિટી} & File-level security & User-level access control \\
\textbf{Backup \& Recovery} & Manual process & Automatic mechanisms \\
\end{longtable}
}

\begin{itemize}
\tightlist
\item
  \textbf{ઘટાડેલ ડેટા Redundancy}: એપ્લિકેશનોમાં duplicate ડેટા સ્ટોરેજ દૂર કરે છે
\item
  \textbf{ડેટા Consistency}: તમામ એપ્લિકેશનોમાં સમાન ડેટા સુનિશ્ચિત કરે છે
\item
  \textbf{ડેટા Independence}: એપ્લિકેશનો ડેટા structure ના ફેરફારોથી સ્વતંત્ર
\item
  \textbf{Concurrent Access}: અનેક યુઝર્સ એક સાથે ડેટા access કરી શકે છે
\item
  \textbf{સિક્યુરિટી કંટ્રોલ}: યુઝર authentication અને authorization
  mechanisms
\item
  \textbf{Backup અને Recovery}: આપોઆપ ડેટા સુરક્ષા અને પુનઃસ્થાપન
\item
  \textbf{ડેટા Integrity}: Constraint enforcement ડેટા ગુણવત્તા જાળવે છે
\end{itemize}

\end{solutionbox}
\begin{mnemonicbox}
``RDCCSBI - Really Don't Copy, Control, Secure,
Backup, Integrate''

\end{mnemonicbox}
\subsection*{પ્રશ્ન 1(ક OR) [7
ગુણ]}\label{uxaaauxab0uxab6uxaa8-1uxa95-or-7-uxa97uxaa3}

\textbf{વિવિધ ડેટા મોડેલ્સની યાદી બનાવો. કોઈપણ બેને ટૂંકમાં સમજાવો.}

\begin{solutionbox}

\textbf{ડેટા મોડેલ્સની યાદી:}

\begin{itemize}
\tightlist
\item
  Hierarchical Data Model
\item
  Network Data Model\\
\item
  Relational Data Model
\item
  Object-Oriented Data Model
\item
  Entity-Relationship Model
\end{itemize}

\textbf{ટેબલ:}

{\def\LTcaptype{none} % do not increment counter
\begin{longtable}[]{@{}
  >{\raggedright\arraybackslash}p{(\linewidth - 6\tabcolsep) * \real{0.1556}}
  >{\raggedright\arraybackslash}p{(\linewidth - 6\tabcolsep) * \real{0.2444}}
  >{\raggedright\arraybackslash}p{(\linewidth - 6\tabcolsep) * \real{0.2667}}
  >{\raggedright\arraybackslash}p{(\linewidth - 6\tabcolsep) * \real{0.3333}}@{}}
\toprule\noalign{}
\begin{minipage}[b]{\linewidth}\raggedright
મોડેલ
\end{minipage} & \begin{minipage}[b]{\linewidth}\raggedright
સ્ટ્રક્ચર
\end{minipage} & \begin{minipage}[b]{\linewidth}\raggedright
ફાયદા
\end{minipage} & \begin{minipage}[b]{\linewidth}\raggedright
ગેરફાયદા
\end{minipage} \\
\midrule\noalign{}
\endhead
\bottomrule\noalign{}
\endlastfoot
\textbf{Relational Model} & Tables with rows/columns & સરળ, લવચીક &
Performance overhead \\
\textbf{Network Model} & Graph with records/links & કુશળ navigation &
જટિલ સ્ટ્રક્ચર \\
\end{longtable}
}

\textbf{Relational Data Model:}

\begin{itemize}
\tightlist
\item
  \textbf{સ્ટ્રક્ચર}: ડેટા tables (relations) માં ગોઠવાયેલો
\item
  \textbf{ઘટકો}: Tuples (rows), attributes (columns), domains
\item
  \textbf{ઓપરેશન્સ}: Select, project, join operations ઉપલબ્ધ
\end{itemize}

\textbf{Network Data Model:}

\begin{itemize}
\tightlist
\item
  \textbf{સ્ટ્રક્ચર}: Owner-member relationships સાથે graph-based
\item
  \textbf{નેવિગેશન}: Record types વચ્ચે સ્પષ્ટ links
\item
  \textbf{લવચીકતા}: Many-to-many relationships કુદરતી રીતે સપોર્ટેડ
\end{itemize}

\end{solutionbox}
\begin{mnemonicbox}
``HNROE - Have Network Relational Object Entity''

\end{mnemonicbox}
\subsection*{પ્રશ્ન 2(અ) [3
ગુણ]}\label{uxaaauxab0uxab6uxaa8-2uxa85-3-uxa97uxaa3}

\textbf{મેપિંગ કાર્ડિનાલિટીઝ સમજાવો.}

\begin{solutionbox}

\textbf{ટેબલ:}

{\def\LTcaptype{none} % do not increment counter
\begin{longtable}[]{@{}
  >{\raggedright\arraybackslash}p{(\linewidth - 6\tabcolsep) * \real{0.2955}}
  >{\raggedright\arraybackslash}p{(\linewidth - 6\tabcolsep) * \real{0.2045}}
  >{\raggedright\arraybackslash}p{(\linewidth - 6\tabcolsep) * \real{0.2955}}
  >{\raggedright\arraybackslash}p{(\linewidth - 6\tabcolsep) * \real{0.2045}}@{}}
\toprule\noalign{}
\begin{minipage}[b]{\linewidth}\raggedright
Cardinality
\end{minipage} & \begin{minipage}[b]{\linewidth}\raggedright
પ્રતીક
\end{minipage} & \begin{minipage}[b]{\linewidth}\raggedright
વર્ણન
\end{minipage} & \begin{minipage}[b]{\linewidth}\raggedright
ઉદાહરણ
\end{minipage} \\
\midrule\noalign{}
\endhead
\bottomrule\noalign{}
\endlastfoot
\textbf{One-to-One} & 1:1 & દરેક entity એક બીજા સાથે સંબંધ ધરાવે &
Person-Passport \\
\textbf{One-to-Many} & 1:M & એક entity અનેક સાથે સંબંધ ધરાવે &
Department-Employee \\
\textbf{Many-to-One} & M:1 & અનેક entities એક સાથે સંબંધ ધરાવે &
Student-Course \\
\textbf{Many-to-Many} & M:N & અનેક entities અનેક સાથે સંબંધ &
Student-Subject \\
\end{longtable}
}

\textbf{ડાયાગ્રામ:}

\begin{verbatim}
erDiagram
    PERSON ||{-{-}|| PASSPORT : "1:1"}
    DEPARTMENT ||{-{-}o\{ EMPLOYEE : "1:M"}
    STUDENT {-}{-}|| COURSE : "M:1"}
    STUDENT \|{-}{-}|\{ SUBJECT : "M:N"}
\end{verbatim}

\begin{itemize}
\tightlist
\item
  \textbf{Cardinality constraints} relationship participation limits
  વ્યાખ્યાયિત કરે છે
\item
  \textbf{Maximum cardinality} associations ની ઉપરી મર્યાદા સ્પષ્ટ કરે છે
\item
  \textbf{ડેટાબેઝ ડિઝાઈન} અને relationship modeling માં મદદ કરે છે
\end{itemize}

\end{solutionbox}
\begin{mnemonicbox}
``OMOM - One, One-Many, One-Many, Many-Many''

\end{mnemonicbox}
\subsection*{પ્રશ્ન 2(બ) [4
ગુણ]}\label{uxaaauxab0uxab6uxaa8-2uxaac-4-uxa97uxaa3}

\textbf{Relational Algebra માં આઉટર જોઇન ઑપરેશન સમજાવો.}

\begin{solutionbox}

\textbf{ટેબલ:}

{\def\LTcaptype{none} % do not increment counter
\begin{longtable}[]{@{}
  >{\raggedright\arraybackslash}p{(\linewidth - 6\tabcolsep) * \real{0.2500}}
  >{\raggedright\arraybackslash}p{(\linewidth - 6\tabcolsep) * \real{0.2045}}
  >{\raggedright\arraybackslash}p{(\linewidth - 6\tabcolsep) * \real{0.2045}}
  >{\raggedright\arraybackslash}p{(\linewidth - 6\tabcolsep) * \real{0.3409}}@{}}
\toprule\noalign{}
\begin{minipage}[b]{\linewidth}\raggedright
Join Type
\end{minipage} & \begin{minipage}[b]{\linewidth}\raggedright
પ્રતીક
\end{minipage} & \begin{minipage}[b]{\linewidth}\raggedright
પરિણામ
\end{minipage} & \begin{minipage}[b]{\linewidth}\raggedright
NULL Handling
\end{minipage} \\
\midrule\noalign{}
\endhead
\bottomrule\noalign{}
\endlastfoot
\textbf{Left Outer Join} & ⟕ & બધા left + મેચિંગ right & અનમેચ્ડ right માટે
NULLs \\
\textbf{Right Outer Join} & ⟖ & બધા right + મેચિંગ left & અનમેચ્ડ left માટે
NULLs \\
\textbf{Full Outer Join} & ⟗ & બંને tables માંથી બધા & અનમેચ્ડ માટે NULLs \\
\end{longtable}
}

\textbf{ઉદાહરણ:}

\begin{verbatim}
EMPLOYEE ⟕ DEPARTMENT
- બધા employees શામેલ કરે છે
- Department વિના employees માટે NULL values
\end{verbatim}

\begin{itemize}
\tightlist
\item
  \textbf{અનમેચ્ડ tuples ને સાચવે છે} સ્પષ્ટ કરેલ relation(s) માંથી
\item
  \textbf{NULL values} ગુમ થયેલ attribute values ભરે છે
\item
  \textbf{ત્રણ પ્રકાર}: Left, Right, અને Full outer joins
\item
  \textbf{અધૂરા ડેટા relationships} ની reporting માટે ઉપયોગી
\end{itemize}

\end{solutionbox}
\begin{mnemonicbox}
``LRF - Left Right Full outer joins''

\end{mnemonicbox}
\subsection*{પ્રશ્ન 2(ક) [7
ગુણ]}\label{uxaaauxab0uxab6uxaa8-2uxa95-7-uxa97uxaa3}

\textbf{Specialization અને Generalization ની concept ના ઉદાહરણ સાથે
સમજાવો.}

\begin{solutionbox}

\textbf{ટેબલ:}

{\def\LTcaptype{none} % do not increment counter
\begin{longtable}[]{@{}
  >{\raggedright\arraybackslash}p{(\linewidth - 6\tabcolsep) * \real{0.2368}}
  >{\raggedright\arraybackslash}p{(\linewidth - 6\tabcolsep) * \real{0.2895}}
  >{\raggedright\arraybackslash}p{(\linewidth - 6\tabcolsep) * \real{0.2368}}
  >{\raggedright\arraybackslash}p{(\linewidth - 6\tabcolsep) * \real{0.2368}}@{}}
\toprule\noalign{}
\begin{minipage}[b]{\linewidth}\raggedright
Concept
\end{minipage} & \begin{minipage}[b]{\linewidth}\raggedright
દિશા
\end{minipage} & \begin{minipage}[b]{\linewidth}\raggedright
પ્રક્રિયા
\end{minipage} & \begin{minipage}[b]{\linewidth}\raggedright
ઉદાહરણ
\end{minipage} \\
\midrule\noalign{}
\endhead
\bottomrule\noalign{}
\endlastfoot
\textbf{Specialization} & Top-Down & સામાન્યથી વિશિષ્ટ & Vehicle \rightarrow Car,
Truck \\
\textbf{Generalization} & Bottom-Up & વિશિષ્ટથી સામાન્ય & Car, Truck \rightarrow
Vehicle \\
\end{longtable}
}

\textbf{ડાયાગ્રામ:}

\begin{verbatim}
erDiagram
    VEHICLE \{
        int vehicle\_id
        string make
        string model
    \}
    CAR \{
        int doors
        string fuel\_type
    \}
    TRUCK \{
        int payload
        string truck\_type
    \}
    
    VEHICLE ||{-{-}|| CAR : "ISA"}
    VEHICLE ||{-{-}|| TRUCK : "ISA"}
\end{verbatim}

\textbf{Specialization:}

\begin{itemize}
\tightlist
\item
  \textbf{પ્રક્રિયા}: Superclass માંથી subclasses બનાવવી
\item
  \textbf{વારસો}: Subclasses બધા superclass attributes વારસામાં મેળવે છે
\item
  \textbf{વધારાના attributes}: Subclasses ને વિશિષ્ટ ગુણધર્મો હોય છે
\end{itemize}

\textbf{Generalization:}

\begin{itemize}
\tightlist
\item
  \textbf{પ્રક્રિયા}: સામાન્ય subclass features માંથી superclass બનાવવું
\item
  \textbf{અમૂર્તીકરણ}: સામાન્ય attributes અને relationships ઓળખે છે
\item
  \textbf{સરળીકરણ}: Hierarchy દ્વારા જટિલતા ઘટાડે છે
\end{itemize}

\end{solutionbox}
\begin{mnemonicbox}
``SG-TD-BU - Specialization General-To-Detail,
Bottom-Up''

\end{mnemonicbox}
\subsection*{પ્રશ્ન 2(અ OR) [3
ગુણ]}\label{uxaaauxab0uxab6uxaa8-2uxa85-or-3-uxa97uxaa3}

\textbf{Relational Algebra માં keys ના વિવિધ પ્રકારો સમજાવો.}

\begin{solutionbox}

\textbf{ટેબલ:}

{\def\LTcaptype{none} % do not increment counter
\begin{longtable}[]{@{}
  >{\raggedright\arraybackslash}p{(\linewidth - 6\tabcolsep) * \real{0.2326}}
  >{\raggedright\arraybackslash}p{(\linewidth - 6\tabcolsep) * \real{0.2791}}
  >{\raggedright\arraybackslash}p{(\linewidth - 6\tabcolsep) * \real{0.2791}}
  >{\raggedright\arraybackslash}p{(\linewidth - 6\tabcolsep) * \real{0.2093}}@{}}
\toprule\noalign{}
\begin{minipage}[b]{\linewidth}\raggedright
Key Type
\end{minipage} & \begin{minipage}[b]{\linewidth}\raggedright
વ્યાખ્યા
\end{minipage} & \begin{minipage}[b]{\linewidth}\raggedright
અનન્યતા
\end{minipage} & \begin{minipage}[b]{\linewidth}\raggedright
ઉદાહરણ
\end{minipage} \\
\midrule\noalign{}
\endhead
\bottomrule\noalign{}
\endlastfoot
\textbf{Super Key} & કોઈપણ attribute set જે uniquely identifies કરે & હા &
\{ID, Name, Phone\} \\
\textbf{Candidate Key} & Minimal super key & હા & \{ID\}, \{Email\} \\
\textbf{Primary Key} & પસંદ કરેલ candidate key & હા & \{StudentID\} \\
\textbf{Foreign Key} & Primary key ને reference કરે છે & ના & \{DeptID\}
references Dept \\
\end{longtable}
}

\begin{itemize}
\tightlist
\item
  \textbf{Super Key}: Tuples ને uniquely identifies કરે છે, વધારાના
  attributes હોઈ શકે
\item
  \textbf{Candidate Key}: Redundant attributes વિના minimal super key
\item
  \textbf{Primary Key}: Entity identification માટે પસંદ કરેલ candidate key
\item
  \textbf{Foreign Key}: Tables વચ્ચે referential integrity સ્થાપિત કરે છે
\end{itemize}

\end{solutionbox}
\begin{mnemonicbox}
``SCPF - Super Candidate Primary Foreign''

\end{mnemonicbox}
\subsection*{પ્રશ્ન 2(બ OR) [4
ગુણ]}\label{uxaaauxab0uxab6uxaa8-2uxaac-or-4-uxa97uxaa3}

\textbf{યોગ્ય ઉદાહરણ સાથે ER-ડાયાગ્રામમાં attributes ના પ્રકારો સમજાવો.}

\begin{solutionbox}

\textbf{ટેબલ:}

{\def\LTcaptype{none} % do not increment counter
\begin{longtable}[]{@{}
  >{\raggedright\arraybackslash}p{(\linewidth - 6\tabcolsep) * \real{0.3404}}
  >{\raggedright\arraybackslash}p{(\linewidth - 6\tabcolsep) * \real{0.1915}}
  >{\raggedright\arraybackslash}p{(\linewidth - 6\tabcolsep) * \real{0.2766}}
  >{\raggedright\arraybackslash}p{(\linewidth - 6\tabcolsep) * \real{0.1915}}@{}}
\toprule\noalign{}
\begin{minipage}[b]{\linewidth}\raggedright
Attribute Type
\end{minipage} & \begin{minipage}[b]{\linewidth}\raggedright
પ્રતીક
\end{minipage} & \begin{minipage}[b]{\linewidth}\raggedright
વર્ણન
\end{minipage} & \begin{minipage}[b]{\linewidth}\raggedright
ઉદાહરણ
\end{minipage} \\
\midrule\noalign{}
\endhead
\bottomrule\noalign{}
\endlastfoot
\textbf{Simple} & Oval & વિભાજિત કરી શકાતા નથી & Age, Name \\
\textbf{Composite} & Oval with sub-ovals & વિભાજિત કરી શકાય છે & Address
(Street, City) \\
\textbf{Derived} & Dashed oval & બીજા attributes માંથી ગણતરી & Age from
Birth\_Date \\
\textbf{Multi-valued} & Double oval & અનેક values & Phone\_Numbers \\
\end{longtable}
}

\textbf{ડાયાગ્રામ:}

\begin{verbatim}
    +{-{-}{-}{-}{-}{-}{-}{-}{-}{-}+}
    |   Name   |  {{-}{-} Simple}
    +{-{-}{-}{-}{-}{-}{-}{-}{-}{-}+}
    
    +{-{-}{-}{-}{-}{-}{-}{-}{-}{-}+}
    |  Address |  {{-}{-} Composite}
    +{-{-}{-}{-}+{-}{-}{-}{-}{-}+}
         |
    +{-{-}{-}{-}+{-}{-}{-}+{-}{-}{-}{-}{-}{-}+}
    | Street | City |
    +{-{-}{-}{-}{-}{-}{-}{-}+{-}{-}{-}{-}{-}{-}+}
    
    +{-{-}{-}{-}{-}{-}{-}{-}{-}{-}+}
    :|Phone\_No|:  {{-}{-} Multi{-}valued}
    +{-{-}{-}{-}{-}{-}{-}{-}{-}{-}+}
    
    +{-{-}{-}{-}{-}{-}{-}{-}{-}{-}+}
    :   Age    :  {{-}{-} Derived}
    +{-{-}{-}{-}{-}{-}{-}{-}{-}{-}+}
\end{verbatim}

\begin{itemize}
\tightlist
\item
  \textbf{Simple attributes} atomic અને અવિભાજ્ય છે
\item
  \textbf{Composite attributes} ને અર્થપૂર્ણ ઉપ-ભાગો હોય છે
\item
  \textbf{Derived attributes} અન્ય attribute values માંથી computed છે\\
\item
  \textbf{Multi-valued attributes} entity દીઠ અનેક values સ્ટોર કરે છે
\end{itemize}

\end{solutionbox}
\begin{mnemonicbox}
``SCDM - Simple Composite Derived Multi-valued''

\end{mnemonicbox}
\subsection*{પ્રશ્ન 2(ક OR) [7
ગુણ]}\label{uxaaauxab0uxab6uxaa8-2uxa95-or-7-uxa97uxaa3}

\textbf{SELECT, PROJECT, UNION અને SET-INTERSECTION ઓપરેશનને યોગ્ય ઉદાહરણ
સાથે સમજાવો.}

\begin{solutionbox}

\textbf{ટેબલ:}

{\def\LTcaptype{none} % do not increment counter
\begin{longtable}[]{@{}llll@{}}
\toprule\noalign{}
Operation & પ્રતીક & હેતુ & ઉદાહરણ \\
\midrule\noalign{}
\endhead
\bottomrule\noalign{}
\endlastfoot
\textbf{SELECT} & σ & Rows filter કરવા & σ(salary \textgreater{}
50000)(Employee) \\
\textbf{PROJECT} & π & Columns પસંદ કરવા & π(name, age)(Employee) \\
\textbf{UNION} & \cup & Relations જોડવા & R \cup S \\
\textbf{INTERSECTION} & \cap & સામાન્ય tuples & R \cap S \\
\end{longtable}
}

\textbf{ઉદાહરણો:}

\textbf{SELECT Operation:}

\begin{verbatim}
σ(age > 25)(STUDENT)
- 25 વર્ષથી વધુ ઉંમરના students return કરે
\end{verbatim}

\textbf{PROJECT Operation:}

\begin{verbatim}
π(name, course)(STUDENT)  
- ફક્ત name અને course columns return કરે
\end{verbatim}

\textbf{UNION Operation:}

\begin{verbatim}
SCIENCE_STUDENTS \cup ARTS_STUDENTS
- બંને streams ના students જોડે છે
\end{verbatim}

\textbf{INTERSECTION Operation:}

\begin{verbatim}
MALE_STUDENTS \cap SPORTS_STUDENTS
- સ્પોર્ટ્સ રમતા પુરુષ વિદ્યાર્થીઓ return કરે
\end{verbatim}

\end{solutionbox}
\begin{mnemonicbox}
``SPUI - Select Project Union Intersection''

\end{mnemonicbox}
\subsection*{પ્રશ્ન 3(અ) [3
ગુણ]}\label{uxaaauxab0uxab6uxaa8-3uxa85-3-uxa97uxaa3}

\textbf{Primary Key અને Foreign Key constraint ને અલગ કરો.}

\begin{solutionbox}

\textbf{ટેબલ:}

{\def\LTcaptype{none} % do not increment counter
\begin{longtable}[]{@{}lll@{}}
\toprule\noalign{}
પાસું & Primary Key & Foreign Key \\
\midrule\noalign{}
\endhead
\bottomrule\noalign{}
\endlastfoot
\textbf{હેતુ} & Unique identification & Referential integrity \\
\textbf{NULL Values} & મંજૂર નથી & મંજૂર છે \\
\textbf{અનન્યતા} & Unique હોવી જ જોઈએ & Duplicate હોઈ શકે \\
\textbf{Table દીઠ સંખ્યા} & ફક્ત એક & અનેક મંજૂર \\
\end{longtable}
}

\begin{itemize}
\tightlist
\item
  \textbf{Primary Key}: Table ની અંદર entity integrity સુનિશ્ચિત કરે છે
\item
  \textbf{Foreign Key}: Tables વચ્ચે referential integrity જાળવે છે
\item
  \textbf{અનન્યતા}: Primary keys unique, foreign keys repeat થઈ શકે
\item
  \textbf{NULL handling}: Primary keys ક્યારેય NULL નથી, foreign keys NULL
  હોઈ શકે
\end{itemize}

\end{solutionbox}
\begin{mnemonicbox}
``PU-FN - Primary Unique, Foreign Nullable''

\end{mnemonicbox}
\subsection*{પ્રશ્ન 3(બ) [4
ગુણ]}\label{uxaaauxab0uxab6uxaa8-3uxaac-4-uxa97uxaa3}

\textbf{DUAL table અને SYSDATE ઉદાહરણ સાથે સમજાવો.}

\begin{solutionbox}

\textbf{ટેબલ:}

{\def\LTcaptype{none} % do not increment counter
\begin{longtable}[]{@{}
  >{\raggedright\arraybackslash}p{(\linewidth - 6\tabcolsep) * \real{0.3143}}
  >{\raggedright\arraybackslash}p{(\linewidth - 6\tabcolsep) * \real{0.1714}}
  >{\raggedright\arraybackslash}p{(\linewidth - 6\tabcolsep) * \real{0.2571}}
  >{\raggedright\arraybackslash}p{(\linewidth - 6\tabcolsep) * \real{0.2571}}@{}}
\toprule\noalign{}
\begin{minipage}[b]{\linewidth}\raggedright
ઘટક
\end{minipage} & \begin{minipage}[b]{\linewidth}\raggedright
પ્રકાર
\end{minipage} & \begin{minipage}[b]{\linewidth}\raggedright
હેતુ
\end{minipage} & \begin{minipage}[b]{\linewidth}\raggedright
ઉદાહરણ
\end{minipage} \\
\midrule\noalign{}
\endhead
\bottomrule\noalign{}
\endlastfoot
\textbf{DUAL} & Virtual table & Expressions ટેસ્ટ કરવા & SELECT 2+3 FROM
DUAL \\
\textbf{SYSDATE} & System function & વર્તમાન date/time & SELECT SYSDATE
FROM DUAL \\
\end{longtable}
}

\textbf{DUAL Table:}

\begin{itemize}
\tightlist
\item
  \textbf{Virtual table} એક row અને એક column સાથે
\item
  \textbf{ટેસ્ટિંગ માટે વપરાય છે} expressions અને functions
\item
  \textbf{Oracle-specific} pseudo table
\end{itemize}

\textbf{SYSDATE Function:}

\begin{itemize}
\tightlist
\item
  \textbf{વર્તમાન return કરે છે} system date અને time
\item
  \textbf{આપોઆપ અપડેટ} system clock સાથે
\item
  \textbf{Date/time operations} સપોર્ટેડ
\end{itemize}

\textbf{ઉદાહરણો:}

\begin{verbatim}
SELECT SYSDATE FROM DUAL;
SELECT SYSDATE + 30 FROM DUAL;  {-{-} 30 દિવસ પછી}
\end{verbatim}

\end{solutionbox}
\begin{mnemonicbox}
``DT-ST - DUAL Testing, SYSDATE Time''

\end{mnemonicbox}
\subsection*{પ્રશ્ન 3(ક) [7
ગુણ]}\label{uxaaauxab0uxab6uxaa8-3uxa95-7-uxa97uxaa3}

\textbf{વિવિધ numeric function નો ઉપયોગ કરવા માટે SQL પ્રશ્નો લખો:}

\begin{solutionbox}

\textbf{ટેબલ:}

{\def\LTcaptype{none} % do not increment counter
\begin{longtable}[]{@{}llll@{}}
\toprule\noalign{}
Function & હેતુ & SQL Query & પરિણામ \\
\midrule\noalign{}
\endhead
\bottomrule\noalign{}
\endlastfoot
\textbf{TRUNC} & Integer value &
\texttt{SELECT\ TRUNC(125.25)\ FROM\ DUAL;} & 125 \\
\textbf{ABS} & Absolute value & \texttt{SELECT\ ABS(-15)\ FROM\ DUAL;} &
15 \\
\textbf{CEIL} & Ceiling value &
\texttt{SELECT\ CEIL(55.65)\ FROM\ DUAL;} & 56 \\
\textbf{FLOOR} & Floor value &
\texttt{SELECT\ FLOOR(100.2)\ FROM\ DUAL;} & 100 \\
\end{longtable}
}

\textbf{SQL Queries:}

\begin{verbatim}
{-{-} (a) 125.25 નું integer મૂલ્ય દર્શાવો}
SELECT TRUNC(125.25) FROM DUAL;

{-{-} (b) ({-}15) નું absolute મૂલ્ય દર્શાવો}
SELECT ABS({-}15) FROM DUAL;

{-{-} (c) 55.65 ની ceil ની કિંમત દર્શાવો}
SELECT CEIL(55.65) FROM DUAL;

{-{-} (d) 100.2 નું floor મૂલ્ય દર્શાવો}
SELECT FLOOR(100.2) FROM DUAL;

{-{-} (e) 16 નું વર્ગમૂળ દર્શાવો}
SELECT SQRT(16) FROM DUAL;

{-{-} (f) e^{3} ની કિંમત બતાવો}
SELECT EXP(3) FROM DUAL;

{-{-} (g) 12 raised to 6 દર્શાવો}
SELECT POWER(12, 6) FROM DUAL;

{-{-} (h) 24 મોડ 2 નું પરિણામ દર્શાવો}
SELECT MOD(24, 2) FROM DUAL;

{-{-} (i) sign({-}25), sign(25), sign(0) નું આઉટપુટ બતાવો}
SELECT SIGN({-}25), SIGN(25), SIGN(0) FROM DUAL;
\end{verbatim}

\end{solutionbox}
\begin{mnemonicbox}
``TACFSEPM - TRUNC ABS CEIL FLOOR SQRT EXP POWER
MOD''

\end{mnemonicbox}
\subsection*{પ્રશ્ન 3(અ OR) [3
ગુણ]}\label{uxaaauxab0uxab6uxaa8-3uxa85-or-3-uxa97uxaa3}

\textbf{યોગ્ય ઉદાહરણ સાથે Unique અને Check સમજાવો.}

\begin{solutionbox}

\textbf{ટેબલ:}

{\def\LTcaptype{none} % do not increment counter
\begin{longtable}[]{@{}llll@{}}
\toprule\noalign{}
Constraint & હેતુ & Duplicates & ઉદાહરણ \\
\midrule\noalign{}
\endhead
\bottomrule\noalign{}
\endlastfoot
\textbf{UNIQUE} & Duplicates અટકાવવા & મંજૂર નથી & Email address \\
\textbf{CHECK} & ડેટા validate કરવા & Value restrictions & Age
\textgreater{} 0 \\
\end{longtable}
}

\textbf{ઉદાહરણો:}

\begin{verbatim}
{-{-} UNIQUE Constraint}
CREATE TABLE Student (
    email VARCHAR(50) UNIQUE,
    phone VARCHAR(15) UNIQUE
);

{-{-} CHECK Constraint  }
CREATE TABLE Employee (
    age NUMBER CHECK (age {=} 18),
    salary NUMBER CHECK (salary {} 0)
);
\end{verbatim}

\begin{itemize}
\tightlist
\item
  \textbf{UNIQUE constraint} column માં duplicate values અટકાવે છે
\item
  \textbf{CHECK constraint} specified conditions વિરુદ્ધ ડેટા validate કરે છે
\item
  \textbf{અનેક constraints} single column પર લાગુ કરી શકાય
\end{itemize}

\end{solutionbox}
\begin{mnemonicbox}
``UC-DV - Unique no Copy, Check Validates''

\end{mnemonicbox}
\subsection*{પ્રશ્ન 3(બ OR) [4
ગુણ]}\label{uxaaauxab0uxab6uxaa8-3uxaac-or-4-uxa97uxaa3}

\textbf{PL/SQL બ્લોકની રચના સમજાવો.}

\begin{solutionbox}

\textbf{ટેબલ:}

{\def\LTcaptype{none} % do not increment counter
\begin{longtable}[]{@{}
  >{\raggedright\arraybackslash}p{(\linewidth - 6\tabcolsep) * \real{0.2432}}
  >{\raggedright\arraybackslash}p{(\linewidth - 6\tabcolsep) * \real{0.2703}}
  >{\raggedright\arraybackslash}p{(\linewidth - 6\tabcolsep) * \real{0.2432}}
  >{\raggedright\arraybackslash}p{(\linewidth - 6\tabcolsep) * \real{0.2432}}@{}}
\toprule\noalign{}
\begin{minipage}[b]{\linewidth}\raggedright
વિભાગ
\end{minipage} & \begin{minipage}[b]{\linewidth}\raggedright
જરૂરી
\end{minipage} & \begin{minipage}[b]{\linewidth}\raggedright
હેતુ
\end{minipage} & \begin{minipage}[b]{\linewidth}\raggedright
ઉદાહરણ
\end{minipage} \\
\midrule\noalign{}
\endhead
\bottomrule\noalign{}
\endlastfoot
\textbf{DECLARE} & વૈકલ્પિક & Variable declarations & var\_name
VARCHAR2(20); \\
\textbf{BEGIN} & ફરજિયાત & Executable statements & SELECT \ldots{} INTO
var; \\
\textbf{EXCEPTION} & વૈકલ્પિક & Error handling & WHEN OTHERS THEN
\ldots{} \\
\textbf{END} & ફરજિયાત & Block termination & END; \\
\end{longtable}
}

\textbf{ડાયાગ્રામ:}

\begin{verbatim}
DECLARE
    -- Variable declarations
BEGIN  
    -- Executable statements
EXCEPTION
    -- Error handling
END;
\end{verbatim}

\begin{itemize}
\tightlist
\item
  \textbf{DECLARE section}: Variable અને cursor declarations
\item
  \textbf{BEGIN-END}: ફરજિયાત executable section
\item
  \textbf{EXCEPTION section}: Error handling routines
\item
  \textbf{Nested blocks}: PL/SQL blocks nested થઈ શકે છે
\end{itemize}

\end{solutionbox}
\begin{mnemonicbox}
``DBE-E - Declare Begin Exception End''

\end{mnemonicbox}
\subsection*{પ્રશ્ન 3(ક OR) [7
ગુણ]}\label{uxaaauxab0uxab6uxaa8-3uxa95-or-7-uxa97uxaa3}

\textbf{નીચેના tables ના પ્રમાણે પ્રશ્નો હલ કરો:}

\begin{solutionbox}

\textbf{I) PRIMARY KEY તરીકે branchId સાથે બ્રાન્ચ ટેબલ બનાવો:}

\begin{verbatim}
CREATE TABLE BRANCH (
    branchid VARCHAR2(10) PRIMARY KEY,
    branchname VARCHAR2(50) NOT NULL,
    address VARCHAR2(100)
);
\end{verbatim}

\textbf{II) EMPLOYEE ટેબલને પ્રાથમિક કી તરીકે empid સાથે બનાવો:}

\begin{verbatim}
CREATE TABLE EMPLOYEE (
    empid VARCHAR2(10) PRIMARY KEY,
    name VARCHAR2(50) NOT NULL,
    post VARCHAR2(30),
    gender CHAR(1) CHECK (gender IN ({M}, {F})),
    birthdate DATE,
    salary NUMBER(10,2),
    branchid VARCHAR2(10),
    FOREIGN KEY (branchid) REFERENCES BRANCH(branchid)
);
\end{verbatim}

\textbf{III) અમદાવાદ શાખામાં કામ કરતા તમામ કર્મચારીઓને શોધો:}

\begin{verbatim}
SELECT e.* FROM EMPLOYEE e, BRANCH b 
WHERE e.branchid = b.branchid 
AND b.branchname = {Ahmedabad};
\end{verbatim}

\textbf{IV) 1998 માં જન્મેલા તમામ કર્મચારીઓને શોધો:}

\begin{verbatim}
SELECT * FROM EMPLOYEE 
WHERE EXTRACT(YEAR FROM birthdate) = 1998;
\end{verbatim}

\textbf{V) 5000 થી વધુ પગાર ધરાવતા તમામ મહિલા કર્મચારીઓને શોધો:}

\begin{verbatim}
SELECT * FROM EMPLOYEE 
WHERE gender = {F} AND salary {} 5000;
\end{verbatim}

\textbf{VI) અજય જ્યાં કામ કરે છે તે શાખાનું સરનામું શોધો:}

\begin{verbatim}
SELECT b.address FROM EMPLOYEE e, BRANCH b
WHERE e.branchid = b.branchid 
AND e.name = {Ajay};
\end{verbatim}

\end{solutionbox}
\begin{mnemonicbox}
``CBEFFA - Create Branch Employee Find Female
Address''

\end{mnemonicbox}
\subsection*{પ્રશ્ન 4(અ) [3
ગુણ]}\label{uxaaauxab0uxab6uxaa8-4uxa85-3-uxa97uxaa3}

\textbf{યોગ્ય ઉદાહરણ સાથે રેફરન્શિયલ ઇન્ટિગ્રિટી સમજાવો.}

\begin{solutionbox}

\textbf{ટેબલ:}

{\def\LTcaptype{none} % do not increment counter
\begin{longtable}[]{@{}
  >{\raggedright\arraybackslash}p{(\linewidth - 4\tabcolsep) * \real{0.2667}}
  >{\raggedright\arraybackslash}p{(\linewidth - 4\tabcolsep) * \real{0.4333}}
  >{\raggedright\arraybackslash}p{(\linewidth - 4\tabcolsep) * \real{0.3000}}@{}}
\toprule\noalign{}
\begin{minipage}[b]{\linewidth}\raggedright
પાસું
\end{minipage} & \begin{minipage}[b]{\linewidth}\raggedright
વર્ણન
\end{minipage} & \begin{minipage}[b]{\linewidth}\raggedright
ઉદાહરણ
\end{minipage} \\
\midrule\noalign{}
\endhead
\bottomrule\noalign{}
\endlastfoot
\textbf{વ્યાખ્યા} & Foreign key એ વર્તમાન primary key ને reference કરવી જ
જોઈએ & Employee.deptid \rightarrow Department.deptid \\
\textbf{હેતુ} & ડેટા consistency જાળવવી & Orphan records અટકાવવા \\
\textbf{ક્રિયાઓ} & CASCADE, SET NULL, RESTRICT & ON DELETE CASCADE \\
\end{longtable}
}

\textbf{ડાયાગ્રામ:}

\begin{verbatim}
erDiagram
    DEPARTMENT \{
        int deptid PK
        string deptname
    \}
    EMPLOYEE \{
        int empid PK
        string name
        int deptid FK
    \}
    DEPARTMENT ||{-{-}o\{ EMPLOYEE : "references"}
\end{verbatim}

\begin{itemize}
\tightlist
\item
  \textbf{રેફરન્શિયલ ઇન્ટિગ્રિટી} સુનિશ્ચિત કરે છે કે foreign key values
  referenced table માં અસ્તિત્વ ધરાવે
\item
  \textbf{Orphan records} constraint enforcement દ્વારા અટકાવાય છે
\item
  \textbf{Cascade operations} updates/deletes દરમિયાન consistency જાળવે છે
\end{itemize}

\end{solutionbox}
\begin{mnemonicbox}
``RIO - Referential Integrity prevents Orphans''

\end{mnemonicbox}
\subsection*{પ્રશ્ન 4(બ) [4
ગુણ]}\label{uxaaauxab0uxab6uxaa8-4uxaac-4-uxa97uxaa3}

\textbf{આંશિક અને સંપૂર્ણ Functional Dependency ને અલગ કરો.}

\begin{solutionbox}

\textbf{ટેબલ:}

{\def\LTcaptype{none} % do not increment counter
\begin{longtable}[]{@{}
  >{\raggedright\arraybackslash}p{(\linewidth - 6\tabcolsep) * \real{0.3333}}
  >{\raggedright\arraybackslash}p{(\linewidth - 6\tabcolsep) * \real{0.2353}}
  >{\raggedright\arraybackslash}p{(\linewidth - 6\tabcolsep) * \real{0.1765}}
  >{\raggedright\arraybackslash}p{(\linewidth - 6\tabcolsep) * \real{0.2549}}@{}}
\toprule\noalign{}
\begin{minipage}[b]{\linewidth}\raggedright
Dependency Type
\end{minipage} & \begin{minipage}[b]{\linewidth}\raggedright
વ્યાખ્યા
\end{minipage} & \begin{minipage}[b]{\linewidth}\raggedright
ઉદાહરણ
\end{minipage} & \begin{minipage}[b]{\linewidth}\raggedright
જરૂરિયાત
\end{minipage} \\
\midrule\noalign{}
\endhead
\bottomrule\noalign{}
\endlastfoot
\textbf{Partial} & Composite key ના ભાગ પર આધાર & (StudentID, CourseID)
\rightarrow StudentName & Composite primary key \\
\textbf{Full} & સંપૂર્ણ key પર આધાર & (StudentID, CourseID) \rightarrow Grade &
Complete key needed \\
\end{longtable}
}

\textbf{ઉદાહરણો:}

\textbf{Partial Functional Dependency:}

\begin{verbatim}
(StudentID, CourseID) \rightarrow StudentName
StudentName ફક્ત StudentID પર આધાર રાખે છે, CourseID પર નહીં
\end{verbatim}

\textbf{Full Functional Dependency:}

\begin{verbatim}
(StudentID, CourseID) \rightarrow Grade  
Grade બંને StudentID અને CourseID પર આધાર રાખે છે
\end{verbatim}

\begin{itemize}
\tightlist
\item
  \textbf{Partial dependency} ડેટા redundancy અને anomalies ઉત્પન્ન કરે છે
\item
  \textbf{Full dependency} યોગ્ય normalization માટે જરૂરી છે
\item
  \textbf{2NF} partial functional dependencies દૂર કરે છે
\end{itemize}

\end{solutionbox}
\begin{mnemonicbox}
``PF-CF - Partial Few, Complete Full''

\end{mnemonicbox}
\subsection*{પ્રશ્ન 4(ક) [7
ગુણ]}\label{uxaaauxab0uxab6uxaa8-4uxa95-7-uxa97uxaa3}

\textbf{ઉદાહરણ સાથે તૃતીય Normal Form સમજાવો.}

\begin{solutionbox}

\textbf{3rd Normal Form જરૂરિયાતો:}

\begin{enumerate}
\tightlist
\item
  2NF માં હોવું જ જોઈએ
\item
  Transitive dependencies ન હોવી જોઈએ
\item
  Non-key attributes ફક્ત primary key પર આધાર રાખવા જોઈએ
\end{enumerate}

\textbf{3NF પહેલાંનું ટેબલ:}

{\def\LTcaptype{none} % do not increment counter
\begin{longtable}[]{@{}
  >{\raggedright\arraybackslash}p{(\linewidth - 10\tabcolsep) * \real{0.1447}}
  >{\raggedright\arraybackslash}p{(\linewidth - 10\tabcolsep) * \real{0.1711}}
  >{\raggedright\arraybackslash}p{(\linewidth - 10\tabcolsep) * \real{0.1316}}
  >{\raggedright\arraybackslash}p{(\linewidth - 10\tabcolsep) * \real{0.1579}}
  >{\raggedright\arraybackslash}p{(\linewidth - 10\tabcolsep) * \real{0.1842}}
  >{\raggedright\arraybackslash}p{(\linewidth - 10\tabcolsep) * \real{0.2105}}@{}}
\toprule\noalign{}
\begin{minipage}[b]{\linewidth}\raggedright
StudentID
\end{minipage} & \begin{minipage}[b]{\linewidth}\raggedright
StudentName
\end{minipage} & \begin{minipage}[b]{\linewidth}\raggedright
CourseID
\end{minipage} & \begin{minipage}[b]{\linewidth}\raggedright
CourseName
\end{minipage} & \begin{minipage}[b]{\linewidth}\raggedright
InstructorID
\end{minipage} & \begin{minipage}[b]{\linewidth}\raggedright
InstructorName
\end{minipage} \\
\midrule\noalign{}
\endhead
\bottomrule\noalign{}
\endlastfoot
S1 & John & C1 & Math & I1 & Dr.~Smith \\
S2 & Jane & C1 & Math & I1 & Dr.~Smith \\
\end{longtable}
}

\textbf{સમસ્યાઓ:}

\begin{itemize}
\tightlist
\item
  \textbf{Transitive dependency}: StudentID \rightarrow CourseID \rightarrow InstructorName
\item
  \textbf{Update anomaly}: Instructor name બદલવા માટે અનેક updates જરૂરી
\item
  \textbf{Delete anomaly}: Student દૂર કરવાથી instructor information
  ખોવાઈ શકે
\end{itemize}

\textbf{3NF સોલ્યુશન:}

\textbf{STUDENT ટેબલ:}

{\def\LTcaptype{none} % do not increment counter
\begin{longtable}[]{@{}lll@{}}
\toprule\noalign{}
StudentID & StudentName & CourseID \\
\midrule\noalign{}
\endhead
\bottomrule\noalign{}
\endlastfoot
S1 & John & C1 \\
S2 & Jane & C1 \\
\end{longtable}
}

\textbf{COURSE ટેબલ:}

{\def\LTcaptype{none} % do not increment counter
\begin{longtable}[]{@{}lll@{}}
\toprule\noalign{}
CourseID & CourseName & InstructorID \\
\midrule\noalign{}
\endhead
\bottomrule\noalign{}
\endlastfoot
C1 & Math & I1 \\
\end{longtable}
}

\textbf{INSTRUCTOR ટેબલ:}

{\def\LTcaptype{none} % do not increment counter
\begin{longtable}[]{@{}ll@{}}
\toprule\noalign{}
InstructorID & InstructorName \\
\midrule\noalign{}
\endhead
\bottomrule\noalign{}
\endlastfoot
I1 & Dr.~Smith \\
\end{longtable}
}

\end{solutionbox}
\begin{mnemonicbox}
``3NF-NT - 3rd Normal Form No Transitives''

\end{mnemonicbox}
\subsection*{પ્રશ્ન 4(અ OR) [3
ગુણ]}\label{uxaaauxab0uxab6uxaa8-4uxa85-or-3-uxa97uxaa3}

\textbf{નોર્મલાઇઝેશનનું મહત્વ સમજાવો.}

\begin{solutionbox}

\textbf{ટેબલ:}

{\def\LTcaptype{none} % do not increment counter
\begin{longtable}[]{@{}
  >{\raggedright\arraybackslash}p{(\linewidth - 4\tabcolsep) * \real{0.2647}}
  >{\raggedright\arraybackslash}p{(\linewidth - 4\tabcolsep) * \real{0.4706}}
  >{\raggedright\arraybackslash}p{(\linewidth - 4\tabcolsep) * \real{0.2647}}@{}}
\toprule\noalign{}
\begin{minipage}[b]{\linewidth}\raggedright
ફાયદો
\end{minipage} & \begin{minipage}[b]{\linewidth}\raggedright
હલ થતી સમસ્યા
\end{minipage} & \begin{minipage}[b]{\linewidth}\raggedright
પરિણામ
\end{minipage} \\
\midrule\noalign{}
\endhead
\bottomrule\noalign{}
\endlastfoot
\textbf{Redundancy ઘટાડવી} & Duplicate data & Storage efficiency \\
\textbf{Anomalies દૂર કરવી} & Update/Insert/Delete issues & Data
consistency \\
\textbf{Integrity સુધારવી} & Data inconsistency & વિશ્વસનીય માહિતી \\
\end{longtable}
}

\begin{itemize}
\tightlist
\item
  \textbf{ડેટા redundancy ઘટાડવી} યોગ્ય table decomposition દ્વારા
\item
  \textbf{Update anomalies દૂર કરવી} duplicate information દૂર કરીને
\item
  \textbf{Storage space ઓપ્ટિમાઇઝ કરવી} normalized structure દ્વારા
\item
  \textbf{ડેટા integrity જાળવવી} referential constraints સાથે
\item
  \textbf{Maintenance સરળ બનાવવી} logical table organization સાથે
\end{itemize}

\end{solutionbox}
\begin{mnemonicbox}
``RESIM - Redundancy Eliminated, Storage Improved,
Maintenance''

\end{mnemonicbox}
\subsection*{પ્રશ્ન 4(બ OR) [4
ગુણ]}\label{uxaaauxab0uxab6uxaa8-4uxaac-or-4-uxa97uxaa3}

\textbf{પ્રાઇમ એટ્રિબ્યુટ્સ અને નોન-પ્રાઇમ એટ્રિબ્યુટ્સને અલગ કરો.}

\begin{solutionbox}

\textbf{ટેબલ:}

{\def\LTcaptype{none} % do not increment counter
\begin{longtable}[]{@{}
  >{\raggedright\arraybackslash}p{(\linewidth - 6\tabcolsep) * \real{0.3721}}
  >{\raggedright\arraybackslash}p{(\linewidth - 6\tabcolsep) * \real{0.2791}}
  >{\raggedright\arraybackslash}p{(\linewidth - 6\tabcolsep) * \real{0.1395}}
  >{\raggedright\arraybackslash}p{(\linewidth - 6\tabcolsep) * \real{0.2093}}@{}}
\toprule\noalign{}
\begin{minipage}[b]{\linewidth}\raggedright
Attribute Type
\end{minipage} & \begin{minipage}[b]{\linewidth}\raggedright
વ્યાખ્યા
\end{minipage} & \begin{minipage}[b]{\linewidth}\raggedright
ભૂમિકા
\end{minipage} & \begin{minipage}[b]{\linewidth}\raggedright
ઉદાહરણ
\end{minipage} \\
\midrule\noalign{}
\endhead
\bottomrule\noalign{}
\endlastfoot
\textbf{Prime} & Candidate key નો ભાગ & Key formation & StudentID,
CourseID \\
\textbf{Non-Prime} & કોઈપણ candidate key નો ભાગ નથી & Data storage &
StudentName, Grade \\
\end{longtable}
}

\textbf{ઉદાહરણ:}

\begin{verbatim}
ENROLLMENT (StudentID, CourseID, Grade, Semester)
Candidate Key: (StudentID, CourseID)

Prime Attributes: StudentID, CourseID
Non-Prime Attributes: Grade, Semester
\end{verbatim}

\begin{itemize}
\tightlist
\item
  \textbf{Prime attributes} candidate key formation માં ભાગ લે છે
\item
  \textbf{Non-Prime attributes} વધારાની entity information પ્રદાન કરે છે
\item
  \textbf{Functional dependencies} આ વચ્ચે normal forms નક્કી કરે છે
\item
  \textbf{2NF જરૂરી છે} non-prime પર prime ની કોઈ partial dependencies ન
  હોવી
\end{itemize}

\end{solutionbox}
\begin{mnemonicbox}
``PN-KD - Prime in Key, Non-prime for Data''

\end{mnemonicbox}
\subsection*{પ્રશ્ન 4(ક OR) [7
ગુણ]}\label{uxaaauxab0uxab6uxaa8-4uxa95-or-7-uxa97uxaa3}

\textbf{2nd Normal Form ઉદાહરણ સાથે સમજાવો.}

\begin{solutionbox}

\textbf{2nd Normal Form જરૂરિયાતો:}

\begin{enumerate}
\tightlist
\item
  1NF માં હોવું જ જોઈએ
\item
  Partial functional dependencies ન હોવી જોઈએ
\item
  બધા non-key attributes primary key પર સંપૂર્ણ આધાર રાખવા જોઈએ
\end{enumerate}

\textbf{2NF પહેલાંનું ટેબલ:}

{\def\LTcaptype{none} % do not increment counter
\begin{longtable}[]{@{}lllll@{}}
\toprule\noalign{}
StudentID & CourseID & StudentName & CourseName & Grade \\
\midrule\noalign{}
\endhead
\bottomrule\noalign{}
\endlastfoot
S1 & C1 & John & Math & A \\
S1 & C2 & John & Physics & B \\
S2 & C1 & Jane & Math & A \\
\end{longtable}
}

\textbf{સમસ્યાઓ:}

\begin{itemize}
\tightlist
\item
  \textbf{Partial Dependencies}: StudentID \rightarrow StudentName, CourseID \rightarrow
  CourseName
\item
  \textbf{Update Anomaly}: Student name બદલવા માટે અનેક updates જરૂરી
\item
  \textbf{Insert Anomaly}: Student enrollment વિના course ઉમેરી શકાતો નથી
\end{itemize}

\textbf{2NF સોલ્યુશન:}

\textbf{STUDENT ટેબલ:}

{\def\LTcaptype{none} % do not increment counter
\begin{longtable}[]{@{}ll@{}}
\toprule\noalign{}
StudentID & StudentName \\
\midrule\noalign{}
\endhead
\bottomrule\noalign{}
\endlastfoot
S1 & John \\
S2 & Jane \\
\end{longtable}
}

\textbf{COURSE ટેબલ:}

{\def\LTcaptype{none} % do not increment counter
\begin{longtable}[]{@{}ll@{}}
\toprule\noalign{}
CourseID & CourseName \\
\midrule\noalign{}
\endhead
\bottomrule\noalign{}
\endlastfoot
C1 & Math \\
C2 & Physics \\
\end{longtable}
}

\textbf{ENROLLMENT ટેબલ:}

{\def\LTcaptype{none} % do not increment counter
\begin{longtable}[]{@{}lll@{}}
\toprule\noalign{}
StudentID & CourseID & Grade \\
\midrule\noalign{}
\endhead
\bottomrule\noalign{}
\endlastfoot
S1 & C1 & A \\
S1 & C2 & B \\
S2 & C1 & A \\
\end{longtable}
}

\end{solutionbox}
\begin{mnemonicbox}
``2NF-FD - 2nd Normal Form Full Dependencies''

\end{mnemonicbox}
\subsection*{પ્રશ્ન 5(અ) [3
ગુણ]}\label{uxaaauxab0uxab6uxaa8-5uxa85-3-uxa97uxaa3}

\textbf{ટ્રાન્ઝેક્શન સ્ટેટ્સને યોગ્ય ડાયાગ્રામ સાથે સમજાવો.}

\begin{solutionbox}

\textbf{ડાયાગ્રામ:}

\begin{verbatim}
stateDiagram{-v2}
  direction LR
    [*] {-{-} Active}
    Active {-{-} Partially\_Committed : commit}
    Active {-{-} Failed : abort/error}
    Partially\_Committed {-{-} Committed : write complete}
    Partially\_Committed {-{-} Failed : write failure}
    Failed {-{-} Aborted : rollback}
    Committed {-{-} [*]}
    Aborted {-{-} [*]}
\end{verbatim}

\textbf{ટેબલ:}

{\def\LTcaptype{none} % do not increment counter
\begin{longtable}[]{@{}
  >{\raggedright\arraybackslash}p{(\linewidth - 4\tabcolsep) * \real{0.2188}}
  >{\raggedright\arraybackslash}p{(\linewidth - 4\tabcolsep) * \real{0.4062}}
  >{\raggedright\arraybackslash}p{(\linewidth - 4\tabcolsep) * \real{0.3750}}@{}}
\toprule\noalign{}
\begin{minipage}[b]{\linewidth}\raggedright
સ્થિતિ
\end{minipage} & \begin{minipage}[b]{\linewidth}\raggedright
વર્ણન
\end{minipage} & \begin{minipage}[b]{\linewidth}\raggedright
આગળની સ્થિતિ
\end{minipage} \\
\midrule\noalign{}
\endhead
\bottomrule\noalign{}
\endlastfoot
\textbf{Active} & ટ્રાન્ઝેક્શન execute થઈ રહ્યું છે & Partially
Committed/Failed \\
\textbf{Partially Committed} & છેલ્લું statement execute થયું &
Committed/Failed \\
\textbf{Committed} & ટ્રાન્ઝેક્શન સફળ & End \\
\textbf{Failed} & સામાન્ય રીતે આગળ વધી શકતું નથી & Aborted \\
\textbf{Aborted} & ટ્રાન્ઝેક્શન rolled back & End \\
\end{longtable}
}

\begin{itemize}
\tightlist
\item
  \textbf{Active state}: ટ્રાન્ઝેક્શન હાલમાં operations execute કરી રહ્યું છે
\item
  \textbf{Partially committed}: બધા operations execute થયા, commit ની
  રાહ જોઈ રહ્યું છે
\item
  \textbf{Failed state}: error આવી, ટ્રાન્ઝેક્શન ચાલુ રાખી શકતું નથી
\end{itemize}

\end{solutionbox}
\begin{mnemonicbox}
``APCFA - Active Partial Commit Fail Abort''

\end{mnemonicbox}
\subsection*{પ્રશ્ન 5(બ) [4
ગુણ]}\label{uxaaauxab0uxab6uxaa8-5uxaac-4-uxa97uxaa3}

\textbf{કોઈપણ બે DDL commands ને યોગ્ય ઉદાહરણ સાથે સમજાવો.}

\begin{solutionbox}

\textbf{ટેબલ:}

{\def\LTcaptype{none} % do not increment counter
\begin{longtable}[]{@{}
  >{\raggedright\arraybackslash}p{(\linewidth - 6\tabcolsep) * \real{0.2500}}
  >{\raggedright\arraybackslash}p{(\linewidth - 6\tabcolsep) * \real{0.2500}}
  >{\raggedright\arraybackslash}p{(\linewidth - 6\tabcolsep) * \real{0.2500}}
  >{\raggedright\arraybackslash}p{(\linewidth - 6\tabcolsep) * \real{0.2500}}@{}}
\toprule\noalign{}
\begin{minipage}[b]{\linewidth}\raggedright
કમાન્ડ
\end{minipage} & \begin{minipage}[b]{\linewidth}\raggedright
હેતુ
\end{minipage} & \begin{minipage}[b]{\linewidth}\raggedright
Syntax
\end{minipage} & \begin{minipage}[b]{\linewidth}\raggedright
ઉદાહરણ
\end{minipage} \\
\midrule\noalign{}
\endhead
\bottomrule\noalign{}
\endlastfoot
\textbf{CREATE} & ડેટાબેઝ ઓબ્જેક્ટ્સ બનાવવા & CREATE TABLE & CREATE TABLE
Student(\ldots) \\
\textbf{ALTER} & વર્તમાન ઓબ્જેક્ટ્સ modify કરવા & ALTER TABLE & ALTER TABLE
Student ADD\ldots{} \\
\end{longtable}
}

\textbf{CREATE કમાન્ડ:}

\begin{verbatim}
CREATE TABLE EMPLOYEE (
    empid NUMBER(5) PRIMARY KEY,
    name VARCHAR2(50) NOT NULL,
    salary NUMBER(10,2),
    deptid NUMBER(3)
);
\end{verbatim}

\textbf{ALTER કમાન્ડ:}

\begin{verbatim}
{-{-} નવો column ઉમેરવો}
ALTER TABLE EMPLOYEE ADD phone VARCHAR2(15);

{-{-} વર્તમાન column modify કરવો}
ALTER TABLE EMPLOYEE MODIFY name VARCHAR2(100);

{-{-} Column drop કરવો}
ALTER TABLE EMPLOYEE DROP COLUMN phone;
\end{verbatim}

\begin{itemize}
\tightlist
\item
  \textbf{CREATE} નવા ડેટાબેઝ structures સ્થાપિત કરે છે
\item
  \textbf{ALTER} વર્તમાન table definitions modify કરે છે
\item
  \textbf{DDL commands} changes ને auto-commit કરે છે
\item
  \textbf{Schema changes} ડેટા structure ને કાયમી અસર કરે છે
\end{itemize}

\end{solutionbox}
\begin{mnemonicbox}
``CA-NM - CREATE Adds, ALTER Modifies''

\end{mnemonicbox}
\subsection*{પ્રશ્ન 5(ક) [7
ગુણ]}\label{uxaaauxab0uxab6uxaa8-5uxa95-7-uxa97uxaa3}

\textbf{ACID ગુણધર્મો વિગતવાર સમજાવો.}

\begin{solutionbox}

\textbf{ટેબલ:}

{\def\LTcaptype{none} % do not increment counter
\begin{longtable}[]{@{}
  >{\raggedright\arraybackslash}p{(\linewidth - 6\tabcolsep) * \real{0.2500}}
  >{\raggedright\arraybackslash}p{(\linewidth - 6\tabcolsep) * \real{0.3000}}
  >{\raggedright\arraybackslash}p{(\linewidth - 6\tabcolsep) * \real{0.2250}}
  >{\raggedright\arraybackslash}p{(\linewidth - 6\tabcolsep) * \real{0.2250}}@{}}
\toprule\noalign{}
\begin{minipage}[b]{\linewidth}\raggedright
ગુણધર્મ
\end{minipage} & \begin{minipage}[b]{\linewidth}\raggedright
વ્યાખ્યા
\end{minipage} & \begin{minipage}[b]{\linewidth}\raggedright
હેતુ
\end{minipage} & \begin{minipage}[b]{\linewidth}\raggedright
ઉદાહરણ
\end{minipage} \\
\midrule\noalign{}
\endhead
\bottomrule\noalign{}
\endlastfoot
\textbf{Atomicity} & બધું અથવા કંઈ નહીં execution & ટ્રાન્ઝેક્શન integrity &
Bank transfer \\
\textbf{Consistency} & ડેટાબેઝ valid રહે છે & ડેટા integrity & Balance
constraints \\
\textbf{Isolation} & Concurrent execution independence & Concurrency
control & અલગ transactions \\
\textbf{Durability} & Committed changes કાયમી & Recovery guarantee &
Power failure survival \\
\end{longtable}
}

\textbf{Atomicity:}

\begin{itemize}
\tightlist
\item
  \textbf{બધા operations} ટ્રાન્ઝેક્શનમાં સંપૂર્ણ execute થાય અથવા બિલકુલ ન થાય
\item
  \textbf{Rollback mechanism} failure પર partial changes undo કરે છે
\item
  \textbf{ઉદાહરણ}: Bank transfer માં debit અને credit બંને operations જરૂરી
\end{itemize}

\textbf{Consistency:}

\begin{itemize}
\tightlist
\item
  \textbf{ડેટાબેઝ state} ટ્રાન્ઝેક્શન પહેલાં અને પછી valid રહે છે
\item
  \textbf{Integrity constraints} execution દરમિયાન જાળવાય છે
\item
  \textbf{ઉદાહરણ}: Account balance ક્યારેય negative નથી થતું
\end{itemize}

\textbf{Isolation:}

\begin{itemize}
\tightlist
\item
  \textbf{Concurrent transactions} એકબીજા સાથે interference કરતા નથી
\item
  \textbf{Locking mechanisms} interference અટકાવે છે
\item
  \textbf{ઉદાહરણ}: બે યુઝર્સ એક સાથે same account update કરી રહ્યા છે
\end{itemize}

\textbf{Durability:}

\begin{itemize}
\tightlist
\item
  \textbf{Committed changes} system failures પછી પણ ટકે છે
\item
  \textbf{Write-ahead logging} recovery capability સુનિશ્ચિત કરે છે
\item
  \textbf{ઉદાહરણ}: Commit પછી power outage થતાં પણ ટ્રાન્ઝેક્શન ટકે છે
\end{itemize}

\end{solutionbox}
\begin{mnemonicbox}
``ACID - Atomicity Consistency Isolation
Durability''

\end{mnemonicbox}
\subsection*{પ્રશ્ન 5(અ OR) [3
ગુણ]}\label{uxaaauxab0uxab6uxaa8-5uxa85-or-3-uxa97uxaa3}

\textbf{two phase લોકિંગ ટેકનિક શું છે?}

\begin{solutionbox}

\textbf{ટેબલ:}

{\def\LTcaptype{none} % do not increment counter
\begin{longtable}[]{@{}
  >{\raggedright\arraybackslash}p{(\linewidth - 6\tabcolsep) * \real{0.1556}}
  >{\raggedright\arraybackslash}p{(\linewidth - 6\tabcolsep) * \real{0.1778}}
  >{\raggedright\arraybackslash}p{(\linewidth - 6\tabcolsep) * \real{0.2889}}
  >{\raggedright\arraybackslash}p{(\linewidth - 6\tabcolsep) * \real{0.3778}}@{}}
\toprule\noalign{}
\begin{minipage}[b]{\linewidth}\raggedright
ફેઝ
\end{minipage} & \begin{minipage}[b]{\linewidth}\raggedright
ક્રિયા
\end{minipage} & \begin{minipage}[b]{\linewidth}\raggedright
વર્ણન
\end{minipage} & \begin{minipage}[b]{\linewidth}\raggedright
Lock Operations
\end{minipage} \\
\midrule\noalign{}
\endhead
\bottomrule\noalign{}
\endlastfoot
\textbf{Growing Phase} & Locks મેળવવા & ટ્રાન્ઝેક્શન જરૂરી locks મેળવે છે & ફક્ત
LOCK \\
\textbf{Shrinking Phase} & Locks છોડવા & ટ્રાન્ઝેક્શન locks એક પછી એક છોડે છે
& ફક્ત UNLOCK \\
\end{longtable}
}

\textbf{ડાયાગ્રામ:}

\begin{verbatim}
Number of Locks
      \^{}
      |     /{}
      |    /  {}
      |   /    {}
      |  /      {}
      | /        {}
      |/          {}
      +{-{-}{-}{-}{-}{-}{-}{-}{-}{-}{-}{-}}
     Growing  Shrinking
     Phase     Phase
        Time
\end{verbatim}

\begin{itemize}
\tightlist
\item
  \textbf{બે ફેઝ}: Growing (lock acquisition) અને Shrinking (lock release)
\item
  \textbf{કોઈ lock upgrades} પ્રથમ unlock operation પછી મંજૂર નથી
\item
  \textbf{Deadlocks અટકાવે છે} જ્યારે યોગ્ય રીતે implemented હોય
\item
  \textbf{Serializability guarantee} concurrent transactions માટે
\end{itemize}

\end{solutionbox}
\begin{mnemonicbox}
``2PL-GS - Two Phase Locking Growing Shrinking''

\end{mnemonicbox}
\subsection*{પ્રશ્ન 5(બ OR) [4
ગુણ]}\label{uxaaauxab0uxab6uxaa8-5uxaac-or-4-uxa97uxaa3}

\textbf{કોઈપણ બે DML આદેશોને યોગ્ય ઉદાહરણ સાથે સમજાવો.}

\begin{solutionbox}

\textbf{ટેબલ:}

{\def\LTcaptype{none} % do not increment counter
\begin{longtable}[]{@{}
  >{\raggedright\arraybackslash}p{(\linewidth - 6\tabcolsep) * \real{0.2500}}
  >{\raggedright\arraybackslash}p{(\linewidth - 6\tabcolsep) * \real{0.2500}}
  >{\raggedright\arraybackslash}p{(\linewidth - 6\tabcolsep) * \real{0.2500}}
  >{\raggedright\arraybackslash}p{(\linewidth - 6\tabcolsep) * \real{0.2500}}@{}}
\toprule\noalign{}
\begin{minipage}[b]{\linewidth}\raggedright
કમાન્ડ
\end{minipage} & \begin{minipage}[b]{\linewidth}\raggedright
હેતુ
\end{minipage} & \begin{minipage}[b]{\linewidth}\raggedright
Syntax
\end{minipage} & \begin{minipage}[b]{\linewidth}\raggedright
ઉદાહરણ
\end{minipage} \\
\midrule\noalign{}
\endhead
\bottomrule\noalign{}
\endlastfoot
\textbf{INSERT} & નવા records ઉમેરવા & INSERT INTO & INSERT INTO Student
VALUES\ldots{} \\
\textbf{UPDATE} & વર્તમાન records modify કરવા & UPDATE SET & UPDATE
Student SET name=\ldots{} \\
\end{longtable}
}

\textbf{INSERT કમાન્ડ:}

\begin{verbatim}
{-{-} Single record insert કરવો}
INSERT INTO EMPLOYEE (empid, name, salary, deptid)
VALUES (101, {John Smith}, 50000, 10);

{-{-} Multiple records insert કરવા}
INSERT INTO EMPLOYEE 
VALUES (102, {Jane Doe}, 45000, 20),
       (103, {Bob Wilson}, 55000, 10);
\end{verbatim}

\textbf{UPDATE કમાન્ડ:}

\begin{verbatim}
{-{-} Single record update કરવો}
UPDATE EMPLOYEE 
SET salary = 60000 
WHERE empid = 101;

{-{-} Multiple records update કરવા}
UPDATE EMPLOYEE 
SET salary = salary * 1.10 
WHERE deptid = 10;
\end{verbatim}

\begin{itemize}
\tightlist
\item
  \textbf{INSERT} table માં નવા rows ઉમેરે છે
\item
  \textbf{UPDATE} વર્તમાન row values modify કરે છે
\item
  \textbf{WHERE clause} update conditions સ્પષ્ટ કરે છે
\item
  \textbf{DML commands} explicit commit જરૂરી છે
\end{itemize}

\end{solutionbox}
\begin{mnemonicbox}
``IU-AM - INSERT Adds, UPDATE Modifies''

\end{mnemonicbox}
\subsection*{પ્રશ્ન 5(ક OR) [7
ગુણ]}\label{uxaaauxab0uxab6uxaa8-5uxa95-or-7-uxa97uxaa3}

\textbf{concurrency control ની સમસ્યાઓની યાદી બનાવો અને કોઈપણ બેને વિગતવાર
સમજાવો.}

\begin{solutionbox}

\textbf{Concurrency Control સમસ્યાઓ:}

\begin{enumerate}
\tightlist
\item
  Lost Update Problem
\item
  Dirty Read Problem\\
\item
  Unrepeatable Read Problem
\item
  Phantom Read Problem
\item
  Inconsistent Analysis Problem
\end{enumerate}

\textbf{ટેબલ:}

{\def\LTcaptype{none} % do not increment counter
\begin{longtable}[]{@{}
  >{\raggedright\arraybackslash}p{(\linewidth - 4\tabcolsep) * \real{0.2812}}
  >{\raggedright\arraybackslash}p{(\linewidth - 4\tabcolsep) * \real{0.4062}}
  >{\raggedright\arraybackslash}p{(\linewidth - 4\tabcolsep) * \real{0.3125}}@{}}
\toprule\noalign{}
\begin{minipage}[b]{\linewidth}\raggedright
સમસ્યા
\end{minipage} & \begin{minipage}[b]{\linewidth}\raggedright
વર્ણન
\end{minipage} & \begin{minipage}[b]{\linewidth}\raggedright
ઉકેલ
\end{minipage} \\
\midrule\noalign{}
\endhead
\bottomrule\noalign{}
\endlastfoot
\textbf{Lost Update} & એક ટ્રાન્ઝેક્શન બીજાના changes overwrite કરે છે &
Locking mechanisms \\
\textbf{Dirty Read} & Uncommitted data વાંચવો & Read committed
isolation \\
\end{longtable}
}

\textbf{Lost Update સમસ્યા:}

\begin{itemize}
\tightlist
\item
  \textbf{સ્થિતિ}: બે ટ્રાન્ઝેક્શન same data વાંચે છે, modify કરે છે, અને પાછું લખે છે
\item
  \textbf{ઉદાહરણ}:

  \begin{itemize}
  \tightlist
  \item
    T1 account balance વાંચે છે: \$1000
  \item
    T2 account balance વાંચે છે: \$1000\\
  \item
    T1 \$100 ઉમેરે છે, \$1100 લખે છે
  \item
    T2 \$50 બાદ કરે છે, \$950 લખે છે
  \item
    \textbf{પરિણામ}: T1 નું update ખોવાઈ ગયું, અંતિમ balance ખોટું
  \end{itemize}
\end{itemize}

\textbf{Dirty Read સમસ્યા:}

\begin{itemize}
\tightlist
\item
  \textbf{સ્થિતિ}: ટ્રાન્ઝેક્શન બીજા uncommitted ટ્રાન્ઝેક્શન દ્વારા modified ડેટા
  વાંચે છે
\item
  \textbf{ઉદાહરણ}:

  \begin{itemize}
  \tightlist
  \item
    T1 account balance \$1000 થી \$1500 કરે છે
  \item
    T2 balance \$1500 તરીકે વાંચે છે (uncommitted data)
  \item
    T1 fail થાય છે અને \$1000 પર rollback કરે છે
  \item
    \textbf{પરિણામ}: T2 એ calculations માટે ખોટો ડેટા વાપર્યો
  \end{itemize}
\end{itemize}

\textbf{ઉકેલો:}

\begin{itemize}
\tightlist
\item
  \textbf{Locking protocols}: Same data ને simultaneous access અટકાવે છે
\item
  \textbf{Isolation levels}: Uncommitted changes ની visibility control
  કરે છે
\item
  \textbf{Timestamp ordering}: Timestamps ના આધારે transactions ને order
  કરે છે
\item
  \textbf{Multi-version concurrency}: અનેક ડેટા versions જાળવે છે
\end{itemize}

\end{solutionbox}
\begin{mnemonicbox}
``LDUI - Lost Dirty Unrepeatable Inconsistent''

\end{mnemonicbox}

\end{document}
