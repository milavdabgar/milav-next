\documentclass[10pt,a4paper]{article}

% content/resources/templates/preamble.tex
\usepackage[margin=0.6in]{geometry}
\author{Milav Dabgar}
\usepackage{amsmath,amssymb,amsthm}
\usepackage{booktabs}
\usepackage{multirow}
\usepackage{xcolor}
\usepackage{tcolorbox}
\tcbuselibrary{breakable,skins}
\usepackage[colorlinks=true,linkcolor=blue]{hyperref}
\usepackage{titlesec}
\usepackage{enumitem}
\usepackage{tikz}
\usepackage{pgfplots}
\usepackage{circuitikz}
\usepackage[version=4]{mhchem}
\usepackage{longtable}
\usepackage{array}
\usepackage{float}
\usepackage{caption}
\usepackage{listings}

\lstset{
  basicstyle=\small\ttfamily,
  breaklines=true,
  breakatwhitespace=false,
  postbreak=\mbox{\textcolor{red}{$\hookrightarrow$}\space},
  float=false,
  numbers=left,
  numberstyle=\tiny\color{gray},
  numbersep=10pt,
  xleftmargin=2em,
  keywordstyle=\color{blue},
  commentstyle=\color{green!60!black},
  stringstyle=\color{purple},
  backgroundcolor=\color{gray!5},
  showstringspaces=false,
  tabsize=2,
  captionpos=b,
  keepspaces=true,
  columns=flexible
}

\pgfplotsset{compat=1.18}
\usetikzlibrary{shapes,arrows,positioning,calc,patterns,decorations.pathmorphing,decorations.markings,arrows.meta}

% Color scheme
\definecolor{headcolor}{RGB}{0,102,204}
\definecolor{keycolor}{RGB}{220,20,60}
\definecolor{solutioncolor}{RGB}{34,139,34}
\definecolor{mnemoniccolor}{RGB}{148,0,211}
\definecolor{codecolor}{RGB}{0,0,100}

% Spacing
\setlength{\parskip}{3pt}
\setlist[itemize]{nosep}
\setlist[enumerate]{nosep}

% Title formatting
\titleformat{\section}{\Large\bfseries\color{headcolor}}{\thesection}{1em}{}
\titleformat{\subsection}{\large\bfseries\color{headcolor}}{\thesubsection}{1em}{}

% Pandoc tightlist compatibility
\providecommand{\tightlist}{%
  \setlength{\itemsep}{0pt}\setlength{\parskip}{0pt}}

% Pandoc longtable compatibility
\newcounter{none}
\def\thenone{}


% content/resources/templates/english-boxes.tex
% This file is currently empty - it exists to maintain consistency with the import structure.
% Add custom environments here if needed in the future.


\begin{document}

\begin{center}
{\Huge\bfseries\color{headcolor} Subject Name Solutions}\\[5pt]
{\LARGE 4331603 -- Summer 2024}\\[3pt]
{\large Semester 1 Study Material}\\[3pt]
{\normalsize\textit{Detailed Solutions and Explanations}}
\end{center}

\vspace{10pt}

\subsection*{Question 1(a) [3 marks]}\label{q1a}

\textbf{Define Following Terms: 1. Data 2. Information 3. Metadata}

\begin{solutionbox}


{\def\LTcaptype{none} % do not increment counter
\vspace{-5pt}
\captionof{table}{Data vs Information vs Metadata}
\vspace{-10pt}
\begin{longtable}[]{@{}
  >{\raggedright\arraybackslash}p{(\linewidth - 4\tabcolsep) * \real{0.2222}}
  >{\raggedright\arraybackslash}p{(\linewidth - 4\tabcolsep) * \real{0.4444}}
  >{\raggedright\arraybackslash}p{(\linewidth - 4\tabcolsep) * \real{0.3333}}@{}}
\toprule\noalign{}
\begin{minipage}[b]{\linewidth}\raggedright
Term
\end{minipage} & \begin{minipage}[b]{\linewidth}\raggedright
Definition
\end{minipage} & \begin{minipage}[b]{\linewidth}\raggedright
Example
\end{minipage} \\
\midrule\noalign{}
\endhead
\bottomrule\noalign{}
\endlastfoot
\textbf{Data} & Raw facts and figures without context & ``25'',
``John'', ``Mumbai'' \\
\textbf{Information} & Processed data with meaning and context & ``John
is 25 years old and lives in Mumbai'' \\
\textbf{Metadata} & Data about data describing structure and properties
& ``Age field: Integer, Max length: 3'' \\
\end{longtable}
}

\begin{itemize}
\tightlist
\item
  \textbf{Data}: Basic building blocks of information systems
\item
  \textbf{Information}: Result of data processing for decision making
\item
  \textbf{Metadata}: Essential for database design and management
\end{itemize}

\end{solutionbox}
\begin{mnemonicbox}
``DIM - Data gives Information using Metadata''

\end{mnemonicbox}
\begin{center}\rule{0.5\linewidth}{0.5pt}\end{center}

\subsection*{Question 1(b) [4 marks]}\label{q1b}

\textbf{Compare File System vs Database System}

\begin{solutionbox}


{\def\LTcaptype{none} % do not increment counter
\vspace{-5pt}
\captionof{table}{File System vs Database System Comparison}
\vspace{-10pt}
\begin{longtable}[]{@{}
  >{\raggedright\arraybackslash}p{(\linewidth - 4\tabcolsep) * \real{0.2105}}
  >{\raggedright\arraybackslash}p{(\linewidth - 4\tabcolsep) * \real{0.3421}}
  >{\raggedright\arraybackslash}p{(\linewidth - 4\tabcolsep) * \real{0.4474}}@{}}
\toprule\noalign{}
\begin{minipage}[b]{\linewidth}\raggedright
Aspect
\end{minipage} & \begin{minipage}[b]{\linewidth}\raggedright
File System
\end{minipage} & \begin{minipage}[b]{\linewidth}\raggedright
Database System
\end{minipage} \\
\midrule\noalign{}
\endhead
\bottomrule\noalign{}
\endlastfoot
\textbf{Data Storage} & Separate files for each application &
Centralized storage \\
\textbf{Data Redundancy} & High redundancy & Minimal redundancy \\
\textbf{Data Consistency} & Poor consistency & High consistency \\
\textbf{Data Security} & Limited security & Advanced security
features \\
\textbf{Concurrent Access} & Limited support & Full concurrent
support \\
\textbf{Data Independence} & No independence & Physical and logical
independence \\
\end{longtable}
}

\begin{itemize}
\tightlist
\item
  \textbf{File System}: Simple but with data duplication issues
\item
  \textbf{Database System}: Complex but efficient data management
\item
  \textbf{Main Advantage}: DBMS eliminates data redundancy and
  inconsistency
\end{itemize}

\end{solutionbox}
\begin{mnemonicbox}
``DBMS = Data Better Managed Systematically''

\end{mnemonicbox}
\begin{center}\rule{0.5\linewidth}{0.5pt}\end{center}

\subsection*{Question 1(c) [7 marks]}\label{q1c}

\textbf{Draw and Explain Network Data Model}

\begin{solutionbox}

\textbf{Diagram:}

\begin{verbatim}
    Owner 1
       |
    Set Type 1
    /    |    {}
Member1 Member2 Member3
   |       |       |
Set Type 2 Set Type 3 Set Type 4
   |       |       |
Member4 Member5 Member6
\end{verbatim}


{\def\LTcaptype{none} % do not increment counter
\vspace{-5pt}
\captionof{table}{Network Model Components}
\vspace{-10pt}
\begin{longtable}[]{@{}lll@{}}
\toprule\noalign{}
Component & Description & Example \\
\midrule\noalign{}
\endhead
\bottomrule\noalign{}
\endlastfoot
\textbf{Record Type} & Entity representation & Employee, Department \\
\textbf{Set Type} & Relationship between records & Works-In, Manages \\
\textbf{Owner} & Parent record in relationship & Department (owner) \\
\textbf{Member} & Child record in relationship & Employee (member) \\
\end{longtable}
}

\begin{itemize}
\tightlist
\item
  \textbf{Owner Record}: Controls the set and can have multiple members
\item
  \textbf{Member Record}: Belongs to one or more sets
\item
  \textbf{Set Occurrence}: Instance of set type linking owner to members
\item
  \textbf{Navigation}: Uses pointers for record access
\end{itemize}

\end{solutionbox}
\begin{mnemonicbox}
``Network = Nodes with Multiple Connections''

\end{mnemonicbox}
\begin{center}\rule{0.5\linewidth}{0.5pt}\end{center}

\subsection*{Question 1(c) OR [7
marks]}\label{q1c}

\textbf{What is Schema? Explain different types of Schema with example}

\begin{solutionbox}

\textbf{Definition}: Schema is the logical structure or blueprint of a
database that defines how data is organized.

\textbf{Diagram:}

\begin{center}
\textbf{Mermaid Diagram (Code)}
\begin{verbatim}
{Shaded}
{Highlighting}[]
graph LR
    A[External Schema] {-{-}{} B[Conceptual Schema]}
    B {-{-}{} C[Internal Schema]}
    A {-{-}{} D[View 1]}
    A {-{-}{} E[View 2]}
    B {-{-}{} F[Logical Structure]}
    C {-{-}{} G[Physical Storage]}
{Highlighting}
{Shaded}
\end{verbatim}
\end{center}


{\def\LTcaptype{none} % do not increment counter
\vspace{-5pt}
\captionof{table}{Types of Schema}
\vspace{-10pt}
\begin{longtable}[]{@{}
  >{\raggedright\arraybackslash}p{(\linewidth - 6\tabcolsep) * \real{0.3095}}
  >{\raggedright\arraybackslash}p{(\linewidth - 6\tabcolsep) * \real{0.1667}}
  >{\raggedright\arraybackslash}p{(\linewidth - 6\tabcolsep) * \real{0.3095}}
  >{\raggedright\arraybackslash}p{(\linewidth - 6\tabcolsep) * \real{0.2143}}@{}}
\toprule\noalign{}
\begin{minipage}[b]{\linewidth}\raggedright
Schema Type
\end{minipage} & \begin{minipage}[b]{\linewidth}\raggedright
Level
\end{minipage} & \begin{minipage}[b]{\linewidth}\raggedright
Description
\end{minipage} & \begin{minipage}[b]{\linewidth}\raggedright
Example
\end{minipage} \\
\midrule\noalign{}
\endhead
\bottomrule\noalign{}
\endlastfoot
\textbf{External Schema} & View Level & User-specific view of database &
Student grades view for teachers \\
\textbf{Conceptual Schema} & Logical Level & Complete logical structure
& All tables, relationships, constraints \\
\textbf{Internal Schema} & Physical Level & Physical storage structure &
Index files, storage allocation \\
\end{longtable}
}

\begin{itemize}
\tightlist
\item
  \textbf{External Schema}: Provides data independence for users
\item
  \textbf{Conceptual Schema}: Database designer's complete view
\item
  \textbf{Internal Schema}: Database administrator's physical view
\end{itemize}

\end{solutionbox}
\begin{mnemonicbox}
``ECI - External Conceptual Internal''

\end{mnemonicbox}
\begin{center}\rule{0.5\linewidth}{0.5pt}\end{center}

\subsection*{Question 2(a) [3 marks]}\label{q2a}

\textbf{Define Following Terms: 1. Entity 2. Attributes 3. Relationship}

\begin{solutionbox}


{\def\LTcaptype{none} % do not increment counter
\vspace{-5pt}
\captionof{table}{ER Model Basic Concepts}
\vspace{-10pt}
\begin{longtable}[]{@{}
  >{\raggedright\arraybackslash}p{(\linewidth - 4\tabcolsep) * \real{0.2222}}
  >{\raggedright\arraybackslash}p{(\linewidth - 4\tabcolsep) * \real{0.4444}}
  >{\raggedright\arraybackslash}p{(\linewidth - 4\tabcolsep) * \real{0.3333}}@{}}
\toprule\noalign{}
\begin{minipage}[b]{\linewidth}\raggedright
Term
\end{minipage} & \begin{minipage}[b]{\linewidth}\raggedright
Definition
\end{minipage} & \begin{minipage}[b]{\linewidth}\raggedright
Example
\end{minipage} \\
\midrule\noalign{}
\endhead
\bottomrule\noalign{}
\endlastfoot
\textbf{Entity} & Real-world object with independent existence &
Student, Course, Teacher \\
\textbf{Attributes} & Properties that describe an entity & Student: ID,
Name, Age \\
\textbf{Relationship} & Association between two or more entities &
Student ENROLLS IN Course \\
\end{longtable}
}

\begin{itemize}
\tightlist
\item
  \textbf{Entity}: Represented by rectangles in ER diagrams
\item
  \textbf{Attributes}: Represented by ovals connected to entities
\item
  \textbf{Relationship}: Represented by diamonds connecting entities
\end{itemize}

\end{solutionbox}
\begin{mnemonicbox}
``EAR - Entity has Attributes and Relationships''

\end{mnemonicbox}
\begin{center}\rule{0.5\linewidth}{0.5pt}\end{center}

\subsection*{Question 2(b) [4 marks]}\label{q2b}

\textbf{Describe Weak Entity Sets with example}

\begin{solutionbox}

\textbf{Definition}: Weak entity is an entity that cannot be uniquely
identified by its own attributes and depends on a strong entity.

\textbf{Diagram:}

\begin{verbatim}
+{-{-}{-}{-}{-}{-}{-}{-}{-}{-}+       +===========+       +{-}{-}{-}{-}{-}{-}{-}{-}{-}{-}+}
| Employee |{-{-}{-}{-}{-}{-}{-}| Dependent |{-}{-}{-}{-}{-}{-}{-}| Person   |}
|   (1)    |       |  (Weak)   |       |   (N)    |
+{-{-}{-}{-}{-}{-}{-}{-}{-}{-}+       +===========+       +{-}{-}{-}{-}{-}{-}{-}{-}{-}{-}+}
    emp\_id              name              dep\_name
                     (Partial Key)
\end{verbatim}


{\def\LTcaptype{none} % do not increment counter
\vspace{-5pt}
\captionof{table}{Weak vs Strong Entity}
\vspace{-10pt}
\begin{longtable}[]{@{}lll@{}}
\toprule\noalign{}
Aspect & Strong Entity & Weak Entity \\
\midrule\noalign{}
\endhead
\bottomrule\noalign{}
\endlastfoot
\textbf{Primary Key} & Has its own primary key & No primary key \\
\textbf{Existence} & Independent existence & Depends on strong entity \\
\textbf{Representation} & Single rectangle & Double rectangle \\
\textbf{Example} & Employee & Dependent of Employee \\
\end{longtable}
}

\begin{itemize}
\tightlist
\item
  \textbf{Partial Key}: Attribute that partially identifies weak entity
\item
  \textbf{Identifying Relationship}: Connects weak entity to strong
  entity
\item
  \textbf{Total Participation}: Weak entity must participate in
  relationship
\end{itemize}

\end{solutionbox}
\begin{mnemonicbox}
``Weak entities are DEPENDent''

\end{mnemonicbox}
\begin{center}\rule{0.5\linewidth}{0.5pt}\end{center}

\subsection*{Question 2(c) [7 marks]}\label{q2c}

\textbf{Draw ER Diagram for University Management System}

\begin{solutionbox}

\textbf{Diagram:}

\begin{verbatim}
erDiagram
    STUDENT \{
        int student\_id PK
        string name
        string email
        date birth\_date
        string address
    \}
    
    COURSE \{
        int course\_id PK
        string course\_name
        int credits
        string department
    \}
    
    TEACHER \{
        int teacher\_id PK
        string name
        string department
        string qualification
    \}
    
    ENROLLMENT \{
        int enrollment\_id PK
        date enrollment\_date
        char grade
    \}
    
    STUDENT ||{-{-}o\{ ENROLLMENT : enrolls}
    COURSE ||{-{-}o\{ ENROLLMENT : has}
    TEACHER ||{-{-}o\{ COURSE : teaches}
\end{verbatim}


{\def\LTcaptype{none} % do not increment counter
\vspace{-5pt}
\captionof{table}{Entity Relationships}
\vspace{-10pt}
\begin{longtable}[]{@{}
  >{\raggedright\arraybackslash}p{(\linewidth - 4\tabcolsep) * \real{0.3500}}
  >{\raggedright\arraybackslash}p{(\linewidth - 4\tabcolsep) * \real{0.3250}}
  >{\raggedright\arraybackslash}p{(\linewidth - 4\tabcolsep) * \real{0.3250}}@{}}
\toprule\noalign{}
\begin{minipage}[b]{\linewidth}\raggedright
Relationship
\end{minipage} & \begin{minipage}[b]{\linewidth}\raggedright
Cardinality
\end{minipage} & \begin{minipage}[b]{\linewidth}\raggedright
Description
\end{minipage} \\
\midrule\noalign{}
\endhead
\bottomrule\noalign{}
\endlastfoot
\textbf{Student ENROLLS Course} & M:N & Many students can enroll in many
courses \\
\textbf{Teacher TEACHES Course} & 1:N & One teacher teaches multiple
courses \\
\textbf{Course HAS Enrollment} & 1:N & One course has multiple
enrollments \\
\end{longtable}
}

\begin{itemize}
\tightlist
\item
  \textbf{Primary Entities}: Student, Course, Teacher
\item
  \textbf{Associative Entity}: Enrollment (resolves M:N relationship)
\item
  \textbf{Key Attributes}: All entities have unique identifier
\end{itemize}

\end{solutionbox}
\begin{mnemonicbox}
``University = Students Take Courses from Teachers''

\end{mnemonicbox}
\begin{center}\rule{0.5\linewidth}{0.5pt}\end{center}

\subsection*{Question 2(a) OR [3
marks]}\label{q2a}

\textbf{Define Following Terms: 1. Primary Key 2. Foreign Key 3.
Candidate Key}

\begin{solutionbox}


{\def\LTcaptype{none} % do not increment counter
\vspace{-5pt}
\captionof{table}{Database Keys}
\vspace{-10pt}
\begin{longtable}[]{@{}
  >{\raggedright\arraybackslash}p{(\linewidth - 4\tabcolsep) * \real{0.3226}}
  >{\raggedright\arraybackslash}p{(\linewidth - 4\tabcolsep) * \real{0.3871}}
  >{\raggedright\arraybackslash}p{(\linewidth - 4\tabcolsep) * \real{0.2903}}@{}}
\toprule\noalign{}
\begin{minipage}[b]{\linewidth}\raggedright
Key Type
\end{minipage} & \begin{minipage}[b]{\linewidth}\raggedright
Definition
\end{minipage} & \begin{minipage}[b]{\linewidth}\raggedright
Example
\end{minipage} \\
\midrule\noalign{}
\endhead
\bottomrule\noalign{}
\endlastfoot
\textbf{Primary Key} & Unique identifier for each record & Student\_ID
in Student table \\
\textbf{Foreign Key} & References primary key of another table &
Student\_ID in Enrollment table \\
\textbf{Candidate Key} & Potential primary key attribute & Email, Phone
in Student table \\
\end{longtable}
}

\begin{itemize}
\tightlist
\item
  \textbf{Primary Key}: Cannot be NULL and must be unique
\item
  \textbf{Foreign Key}: Maintains referential integrity
\item
  \textbf{Candidate Key}: Alternative unique identifiers
\end{itemize}

\end{solutionbox}
\begin{mnemonicbox}
``PFC - Primary Foreign Candidate''

\end{mnemonicbox}
\begin{center}\rule{0.5\linewidth}{0.5pt}\end{center}

\subsection*{Question 2(b) OR [4
marks]}\label{q2b}

\textbf{Write a Short note on Generalization and Specialization}

\begin{solutionbox}

\textbf{Generalization}: Process of extracting common attributes from
multiple entities to create a general entity.

\textbf{Specialization}: Process of defining subclasses of an entity
based on distinguishing characteristics.

\textbf{Diagram:}

\begin{center}
\textbf{Mermaid Diagram (Code)}
\begin{verbatim}
{Shaded}
{Highlighting}[]
graph TD
    A[Person] {-{-}{} B[Student]}
    A {-{-}{} C[Teacher]}
    A {-{-}{} D[Staff]}
    B {-{-}{} E[Undergraduate]}
    B {-{-}{} F[Graduate]}
{Highlighting}
{Shaded}
\end{verbatim}
\end{center}


{\def\LTcaptype{none} % do not increment counter
\vspace{-5pt}
\captionof{table}{Generalization vs Specialization}
\vspace{-10pt}
\begin{longtable}[]{@{}lll@{}}
\toprule\noalign{}
Aspect & Generalization & Specialization \\
\midrule\noalign{}
\endhead
\bottomrule\noalign{}
\endlastfoot
\textbf{Direction} & Bottom-up approach & Top-down approach \\
\textbf{Purpose} & Remove redundancy & Add specific attributes \\
\textbf{Result} & Superclass creation & Subclass creation \\
\end{longtable}
}

\begin{itemize}
\tightlist
\item
  \textbf{ISA Relationship}: ``Is-A'' relationship between superclass
  and subclass
\item
  \textbf{Inheritance}: Subclasses inherit attributes from superclass
\end{itemize}

\end{solutionbox}
\begin{mnemonicbox}
``General goes UP, Special goes DOWN''

\end{mnemonicbox}
\begin{center}\rule{0.5\linewidth}{0.5pt}\end{center}

\subsection*{Question 2(c) OR [7
marks]}\label{q2c}

\textbf{Explain different Relational Algebra operation with example}

\begin{solutionbox}


{\def\LTcaptype{none} % do not increment counter
\vspace{-5pt}
\captionof{table}{Relational Algebra Operations}
\vspace{-10pt}
\begin{longtable}[]{@{}
  >{\raggedright\arraybackslash}p{(\linewidth - 6\tabcolsep) * \real{0.2683}}
  >{\raggedright\arraybackslash}p{(\linewidth - 6\tabcolsep) * \real{0.1951}}
  >{\raggedright\arraybackslash}p{(\linewidth - 6\tabcolsep) * \real{0.3171}}
  >{\raggedright\arraybackslash}p{(\linewidth - 6\tabcolsep) * \real{0.2195}}@{}}
\toprule\noalign{}
\begin{minipage}[b]{\linewidth}\raggedright
Operation
\end{minipage} & \begin{minipage}[b]{\linewidth}\raggedright
Symbol
\end{minipage} & \begin{minipage}[b]{\linewidth}\raggedright
Description
\end{minipage} & \begin{minipage}[b]{\linewidth}\raggedright
Example
\end{minipage} \\
\midrule\noalign{}
\endhead
\bottomrule\noalign{}
\endlastfoot
\textbf{Select} & σ & Selects rows based on condition &
σ(age\textgreater20)(Student) \\
\textbf{Project} & π & Selects specific columns &
π(name,age)(Student) \\
\textbf{Union} & \cup & Combines two relations & R \cup S \\
\textbf{Intersection} & \cap & Common tuples from relations & R \cap S \\
\textbf{Difference} & - & Tuples in R but not in S & R - S \\
\textbf{Join} & ⋈ & Combines related tuples & Student ⋈ Enrollment \\
\end{longtable}
}

\textbf{Example Relations:}

Student: (ID=1, Name=John, Age=20) Course: (CID=101, CName=DBMS,
Credits=3)

\begin{itemize}
\tightlist
\item
  \textbf{Selection}: σ(Age\textgreater18)(Student) returns students
  above 18
\item
  \textbf{Projection}: π(Name)(Student) returns only names
\item
  \textbf{Join}: Student ⋈ Enrollment combines student and enrollment
  data
\end{itemize}

\end{solutionbox}
\begin{mnemonicbox}
``SPUDIJ - Select Project Union Difference
Intersection Join''

\end{mnemonicbox}
\begin{center}\rule{0.5\linewidth}{0.5pt}\end{center}

\subsection*{Question 3(a) [3 marks]}\label{q3a}

\textbf{List out Numeric Functions in SQL. Explain any Two}

\begin{solutionbox}


{\def\LTcaptype{none} % do not increment counter
\vspace{-5pt}
\captionof{table}{SQL Numeric Functions}
\vspace{-10pt}
\begin{longtable}[]{@{}lll@{}}
\toprule\noalign{}
Function & Purpose & Example \\
\midrule\noalign{}
\endhead
\bottomrule\noalign{}
\endlastfoot
\textbf{ABS()} & Absolute value & ABS(-15) = 15 \\
\textbf{CEIL()} & Smallest integer \geq value & CEIL(4.3) = 5 \\
\textbf{FLOOR()} & Largest integer \leq value & FLOOR(4.7) = 4 \\
\textbf{ROUND()} & Round to specified places & ROUND(15.76, 1) = 15.8 \\
\textbf{SQRT()} & Square root & SQRT(16) = 4 \\
\textbf{POWER()} & Raise to power & POWER(2, 3) = 8 \\
\end{longtable}
}

\textbf{Detailed Examples:}

\begin{itemize}
\tightlist
\item
  \textbf{ABS(number)}: Returns absolute value, removing negative sign
\item
  \textbf{ROUND(number, decimal\_places)}: Rounds number to specified
  decimal places
\end{itemize}

\end{solutionbox}
\begin{mnemonicbox}
``Math functions make Numbers Nice''

\end{mnemonicbox}
\begin{center}\rule{0.5\linewidth}{0.5pt}\end{center}

\subsection*{Question 3(b) [4 marks]}\label{q3b}

\textbf{Describe Having and Order by Clause with example}

\begin{solutionbox}

\textbf{HAVING Clause}: Used with GROUP BY to filter groups based on
aggregate conditions.

\textbf{ORDER BY Clause}: Used to sort result set in ascending or
descending order.


{\def\LTcaptype{none} % do not increment counter
\vspace{-5pt}
\captionof{table}{HAVING vs WHERE}
\vspace{-10pt}
\begin{longtable}[]{@{}lll@{}}
\toprule\noalign{}
Aspect & WHERE & HAVING \\
\midrule\noalign{}
\endhead
\bottomrule\noalign{}
\endlastfoot
\textbf{Usage} & Filters individual rows & Filters grouped results \\
\textbf{With Aggregates} & Cannot use & Can use aggregate functions \\
\textbf{Position} & Before GROUP BY & After GROUP BY \\
\end{longtable}
}

\textbf{Example:}

\begin{verbatim}
SELECT department, COUNT(*) as emp\_count
FROM employees 
WHERE salary {} 30000
GROUP BY department 
HAVING COUNT(*) {} 5
ORDER BY emp\_count DESC;
\end{verbatim}

\begin{itemize}
\tightlist
\item
  \textbf{WHERE}: Filters employees with salary \textgreater{} 30000
\item
  \textbf{HAVING}: Shows only departments with more than 5 employees
\item
  \textbf{ORDER BY}: Sorts by employee count in descending order
\end{itemize}

\end{solutionbox}
\begin{mnemonicbox}
``WHERE filters rows, HAVING filters groups, ORDER BY
sorts results''

\end{mnemonicbox}
\begin{center}\rule{0.5\linewidth}{0.5pt}\end{center}

\subsection*{Question 3(c) [7 marks]}\label{q3c}

\textbf{Perform the following Query on the table student having the
fields Student\_ID, Stu\_Name, Stu\_Subject\_ID, Stu\_Marks, Stu\_Age in
SQL}

\begin{solutionbox}

\textbf{1. Create student table:}

\begin{verbatim}
CREATE TABLE student (
    Student\_ID INT PRIMARY KEY,
    Stu\_Name VARCHAR(50),
    Stu\_Subject\_ID INT,
    Stu\_Marks INT,
    Stu\_Age INT
);
\end{verbatim}

\textbf{2. Insert record in student table:}

\begin{verbatim}
INSERT INTO student VALUES 
(1, {John}, 101, 85, 22),
(2, {Mary}, 102, 90, 21);
\end{verbatim}

\textbf{3. Find minimum and maximum marks:}

\begin{verbatim}
SELECT MIN(Stu\_Marks) as Min\_Marks, 
       MAX(Stu\_Marks) as Max\_Marks 
FROM student;
\end{verbatim}

\textbf{4. Students with marks \textgreater{} 82 and age = 22:}

\begin{verbatim}
SELECT * FROM student 
WHERE Stu\_Marks {} 82 AND Stu\_Age = 22;
\end{verbatim}

\textbf{5. Students whose name begins with `m':}

\begin{verbatim}
SELECT * FROM student 
WHERE Stu\_Name LIKE {m\%};
\end{verbatim}

\textbf{6. Find average marks:}

\begin{verbatim}
SELECT AVG(Stu\_Marks) as Average\_Marks 
FROM student;
\end{verbatim}

\textbf{7. Add Stu\_address column:}

\begin{verbatim}
ALTER TABLE student 
ADD Stu\_address VARCHAR(100);
\end{verbatim}

\end{solutionbox}
\begin{mnemonicbox}
``CRUD + Analytics = Complete Database Operations''

\end{mnemonicbox}
\begin{center}\rule{0.5\linewidth}{0.5pt}\end{center}

\subsection*{Question 3(a) OR [3
marks]}\label{q3a}

\textbf{Describe different date function in SQL with example}

\begin{solutionbox}


{\def\LTcaptype{none} % do not increment counter
\vspace{-5pt}
\captionof{table}{SQL Date Functions}
\vspace{-10pt}
\begin{longtable}[]{@{}
  >{\raggedright\arraybackslash}p{(\linewidth - 4\tabcolsep) * \real{0.3571}}
  >{\raggedright\arraybackslash}p{(\linewidth - 4\tabcolsep) * \real{0.3214}}
  >{\raggedright\arraybackslash}p{(\linewidth - 4\tabcolsep) * \real{0.3214}}@{}}
\toprule\noalign{}
\begin{minipage}[b]{\linewidth}\raggedright
Function
\end{minipage} & \begin{minipage}[b]{\linewidth}\raggedright
Purpose
\end{minipage} & \begin{minipage}[b]{\linewidth}\raggedright
Example
\end{minipage} \\
\midrule\noalign{}
\endhead
\bottomrule\noalign{}
\endlastfoot
\textbf{SYSDATE} & Current system date & SYSDATE returns `2024-06-12' \\
\textbf{ADD\_MONTHS()} & Add months to date & ADD\_MONTHS(`2024-01-15',
3) \\
\textbf{MONTHS\_BETWEEN()} & Months between dates &
MONTHS\_BETWEEN(`2024-06-12', `2024-01-12') \\
\textbf{LAST\_DAY()} & Last day of month & LAST\_DAY(`2024-02-15') =
`2024-02-29' \\
\textbf{NEXT\_DAY()} & Next occurrence of day & NEXT\_DAY(`2024-06-12',
`FRIDAY') \\
\end{longtable}
}

\textbf{Examples:}

\begin{itemize}
\tightlist
\item
  \textbf{SYSDATE}: Returns current system date and time
\item
  \textbf{ADD\_MONTHS}: Useful for calculating future dates like loan
  due dates
\end{itemize}

\end{solutionbox}
\begin{mnemonicbox}
``Date functions help with Time Management''

\end{mnemonicbox}
\begin{center}\rule{0.5\linewidth}{0.5pt}\end{center}

\subsection*{Question 3(b) OR [4
marks]}\label{q3b}

\textbf{List out Constraints in SQL. Explain any two with example}

\begin{solutionbox}


{\def\LTcaptype{none} % do not increment counter
\vspace{-5pt}
\captionof{table}{SQL Constraints}
\vspace{-10pt}
\begin{longtable}[]{@{}
  >{\raggedright\arraybackslash}p{(\linewidth - 4\tabcolsep) * \real{0.4000}}
  >{\raggedright\arraybackslash}p{(\linewidth - 4\tabcolsep) * \real{0.3000}}
  >{\raggedright\arraybackslash}p{(\linewidth - 4\tabcolsep) * \real{0.3000}}@{}}
\toprule\noalign{}
\begin{minipage}[b]{\linewidth}\raggedright
Constraint
\end{minipage} & \begin{minipage}[b]{\linewidth}\raggedright
Purpose
\end{minipage} & \begin{minipage}[b]{\linewidth}\raggedright
Example
\end{minipage} \\
\midrule\noalign{}
\endhead
\bottomrule\noalign{}
\endlastfoot
\textbf{PRIMARY KEY} & Unique identifier & Student\_ID INT PRIMARY
KEY \\
\textbf{FOREIGN KEY} & References another table & REFERENCES
Student(Student\_ID) \\
\textbf{NOT NULL} & Prevents null values & Name VARCHAR(50) NOT NULL \\
\textbf{UNIQUE} & Ensures uniqueness & Email VARCHAR(100) UNIQUE \\
\textbf{CHECK} & Validates data & Age INT CHECK (Age \textgreater=
18) \\
\textbf{DEFAULT} & Default value & Status VARCHAR(10) DEFAULT
`Active' \\
\end{longtable}
}

\textbf{Detailed Examples:}

\textbf{PRIMARY KEY Constraint:}

\begin{verbatim}
CREATE TABLE Student (
    Student\_ID INT PRIMARY KEY,
    Name VARCHAR(50)
);
\end{verbatim}

\textbf{CHECK Constraint:}

\begin{verbatim}
CREATE TABLE Employee (
    Emp\_ID INT,
    Salary INT CHECK (Salary {} 0)
);
\end{verbatim}

\begin{itemize}
\tightlist
\item
  \textbf{PRIMARY KEY}: Ensures each record has unique identifier
\item
  \textbf{CHECK}: Validates business rules during data entry
\end{itemize}

\end{solutionbox}
\begin{mnemonicbox}
``Constraints Control Data Quality''

\end{mnemonicbox}
\begin{center}\rule{0.5\linewidth}{0.5pt}\end{center}

\subsection*{Question 3(c) OR [7
marks]}\label{q3c}

\textbf{Explain different types of joins with example in SQL}

\begin{solutionbox}


{\def\LTcaptype{none} % do not increment counter
\vspace{-5pt}
\captionof{table}{Types of SQL Joins}
\vspace{-10pt}
\begin{longtable}[]{@{}
  >{\raggedright\arraybackslash}p{(\linewidth - 4\tabcolsep) * \real{0.3438}}
  >{\raggedright\arraybackslash}p{(\linewidth - 4\tabcolsep) * \real{0.4062}}
  >{\raggedright\arraybackslash}p{(\linewidth - 4\tabcolsep) * \real{0.2500}}@{}}
\toprule\noalign{}
\begin{minipage}[b]{\linewidth}\raggedright
Join Type
\end{minipage} & \begin{minipage}[b]{\linewidth}\raggedright
Description
\end{minipage} & \begin{minipage}[b]{\linewidth}\raggedright
Syntax
\end{minipage} \\
\midrule\noalign{}
\endhead
\bottomrule\noalign{}
\endlastfoot
\textbf{INNER JOIN} & Returns matching records from both tables & Table1
INNER JOIN Table2 ON condition \\
\textbf{LEFT JOIN} & All records from left table + matching from right &
Table1 LEFT JOIN Table2 ON condition \\
\textbf{RIGHT JOIN} & All records from right table + matching from left
& Table1 RIGHT JOIN Table2 ON condition \\
\textbf{FULL OUTER JOIN} & All records from both tables & Table1 FULL
OUTER JOIN Table2 ON condition \\
\end{longtable}
}

\textbf{Example Tables:} Students: (ID=1, Name=John), (ID=2, Name=Mary)
Enrollments: (StudentID=1, Course=DBMS), (StudentID=3, Course=Java)

\textbf{INNER JOIN Example:}

\begin{verbatim}
SELECT s.Name, e.Course 
FROM Students s 
INNER JOIN Enrollments e ON s.ID = e.StudentID;
\end{verbatim}

\emph{Result: Only John with DBMS course}

\textbf{LEFT JOIN Example:}

\begin{verbatim}
SELECT s.Name, e.Course 
FROM Students s 
LEFT JOIN Enrollments e ON s.ID = e.StudentID;
\end{verbatim}

\emph{Result: John-DBMS, Mary-NULL}

\end{solutionbox}
\begin{mnemonicbox}
``JOIN connects Related Tables''

\end{mnemonicbox}
\begin{center}\rule{0.5\linewidth}{0.5pt}\end{center}

\subsection*{Question 4(a) [3 marks]}\label{q4a}

\textbf{Give an example of Grant and Revoke command in SQL}

\begin{solutionbox}

\textbf{GRANT Command}: Provides specific privileges to users on
database objects.

\textbf{REVOKE Command}: Removes previously granted privileges from
users.


{\def\LTcaptype{none} % do not increment counter
\vspace{-5pt}
\captionof{table}{Common Privileges}
\vspace{-10pt}
\begin{longtable}[]{@{}
  >{\raggedright\arraybackslash}p{(\linewidth - 4\tabcolsep) * \real{0.3333}}
  >{\raggedright\arraybackslash}p{(\linewidth - 4\tabcolsep) * \real{0.3939}}
  >{\raggedright\arraybackslash}p{(\linewidth - 4\tabcolsep) * \real{0.2727}}@{}}
\toprule\noalign{}
\begin{minipage}[b]{\linewidth}\raggedright
Privilege
\end{minipage} & \begin{minipage}[b]{\linewidth}\raggedright
Description
\end{minipage} & \begin{minipage}[b]{\linewidth}\raggedright
Example
\end{minipage} \\
\midrule\noalign{}
\endhead
\bottomrule\noalign{}
\endlastfoot
\textbf{SELECT} & Read data & GRANT SELECT ON Student TO user1 \\
\textbf{INSERT} & Add new records & GRANT INSERT ON Student TO user1 \\
\textbf{UPDATE} & Modify existing records & GRANT UPDATE ON Student TO
user1 \\
\textbf{DELETE} & Remove records & GRANT DELETE ON Student TO user1 \\
\textbf{ALL} & All privileges & GRANT ALL ON Student TO user1 \\
\end{longtable}
}

\textbf{Examples:}

\begin{verbatim}
{-{-} Grant SELECT privilege}
GRANT SELECT ON Student TO john;

{-{-} Revoke INSERT privilege  }
REVOKE INSERT ON Student FROM john;
\end{verbatim}

\begin{itemize}
\tightlist
\item
  \textbf{WITH GRANT OPTION}: Allows user to grant privileges to others
\item
  \textbf{CASCADE}: Revokes privileges from all users who received them
\end{itemize}

\end{solutionbox}
\begin{mnemonicbox}
``GRANT gives rights, REVOKE removes rights''

\end{mnemonicbox}
\begin{center}\rule{0.5\linewidth}{0.5pt}\end{center}

\subsection*{Question 4(b) [4 marks]}\label{q4b}

\textbf{Write a short note on SQL Views}

\begin{solutionbox}

\textbf{Definition}: A view is a virtual table based on the result of an
SQL statement containing rows and columns like a real table.


{\def\LTcaptype{none} % do not increment counter
\vspace{-5pt}
\captionof{table}{View Characteristics}
\vspace{-10pt}
\begin{longtable}[]{@{}
  >{\raggedright\arraybackslash}p{(\linewidth - 4\tabcolsep) * \real{0.2667}}
  >{\raggedright\arraybackslash}p{(\linewidth - 4\tabcolsep) * \real{0.4333}}
  >{\raggedright\arraybackslash}p{(\linewidth - 4\tabcolsep) * \real{0.3000}}@{}}
\toprule\noalign{}
\begin{minipage}[b]{\linewidth}\raggedright
Aspect
\end{minipage} & \begin{minipage}[b]{\linewidth}\raggedright
Description
\end{minipage} & \begin{minipage}[b]{\linewidth}\raggedright
Example
\end{minipage} \\
\midrule\noalign{}
\endhead
\bottomrule\noalign{}
\endlastfoot
\textbf{Virtual Table} & Does not store data physically & CREATE VIEW
student\_view AS\ldots{} \\
\textbf{Security} & Hides sensitive columns & Hide salary column from
employees \\
\textbf{Simplification} & Simplifies complex queries & Join multiple
tables in single view \\
\textbf{Data Independence} & Changes in base tables don't affect users &
Modify table structure without affecting applications \\
\end{longtable}
}

\textbf{Example:}

\begin{verbatim}
CREATE VIEW active\_students AS
SELECT Student\_ID, Name, Age 
FROM Student 
WHERE Status = {Active};

{-{-} Using the view}
SELECT * FROM active\_students;
\end{verbatim}

\textbf{Advantages:}

\begin{itemize}
\tightlist
\item
  \textbf{Security}: Restrict access to sensitive data
\item
  \textbf{Simplicity}: Hide complex joins from end users
\item
  \textbf{Consistency}: Standardized data access
\end{itemize}

\end{solutionbox}
\begin{mnemonicbox}
``Views are Virtual Windows to Data''

\end{mnemonicbox}
\begin{center}\rule{0.5\linewidth}{0.5pt}\end{center}

\subsection*{Question 4(c) [7 marks]}\label{q4c}

\textbf{What is Normalization? Explain 2NF with example}

\begin{solutionbox}

\textbf{Normalization}: Process of organizing database to reduce
redundancy and improve data integrity by dividing large tables into
smaller related tables.

\textbf{2NF (Second Normal Form)}:

\begin{itemize}
\tightlist
\item
  Must be in 1NF
\item
  Remove partial functional dependencies
\item
  Non-key attributes must depend on entire primary key
\end{itemize}

\textbf{Example - Unnormalized Table:}

{\def\LTcaptype{none} % do not increment counter
\begin{longtable}[]{@{}lllll@{}}
\toprule\noalign{}
Student\_ID & Course\_ID & Student\_Name & Course\_Name & Instructor \\
\midrule\noalign{}
\endhead
\bottomrule\noalign{}
\endlastfoot
101 & C1 & John & DBMS & Dr.~Smith \\
101 & C2 & John & Java & Dr.~Jones \\
102 & C1 & Mary & DBMS & Dr.~Smith \\
\end{longtable}
}

\textbf{Problems:}

\begin{itemize}
\tightlist
\item
  Student\_Name depends only on Student\_ID (partial dependency)
\item
  Course\_Name and Instructor depend only on Course\_ID
\end{itemize}

\textbf{After 2NF:}

\textbf{Student Table:}

{\def\LTcaptype{none} % do not increment counter
\begin{longtable}[]{@{}ll@{}}
\toprule\noalign{}
Student\_ID & Student\_Name \\
\midrule\noalign{}
\endhead
\bottomrule\noalign{}
\endlastfoot
101 & John \\
102 & Mary \\
\end{longtable}
}

\textbf{Course Table:}

{\def\LTcaptype{none} % do not increment counter
\begin{longtable}[]{@{}lll@{}}
\toprule\noalign{}
Course\_ID & Course\_Name & Instructor \\
\midrule\noalign{}
\endhead
\bottomrule\noalign{}
\endlastfoot
C1 & DBMS & Dr.~Smith \\
C2 & Java & Dr.~Jones \\
\end{longtable}
}

\textbf{Enrollment Table:}

{\def\LTcaptype{none} % do not increment counter
\begin{longtable}[]{@{}ll@{}}
\toprule\noalign{}
Student\_ID & Course\_ID \\
\midrule\noalign{}
\endhead
\bottomrule\noalign{}
\endlastfoot
101 & C1 \\
101 & C2 \\
102 & C1 \\
\end{longtable}
}

\textbf{Benefits:}

\begin{itemize}
\tightlist
\item
  \textbf{Eliminates Redundancy}: Student names not repeated
\item
  \textbf{Reduces Storage}: Less duplicate data
\item
  \textbf{Improves Consistency}: Update student name in one place
\end{itemize}

\end{solutionbox}
\begin{mnemonicbox}
``2NF = No Partial Dependencies''

\end{mnemonicbox}
\begin{center}\rule{0.5\linewidth}{0.5pt}\end{center}

\subsection*{Question 4(a) OR [3
marks]}\label{q4a}

\textbf{Give an example of Group By Clause in SQL}

\begin{solutionbox}

\textbf{GROUP BY Clause}: Groups rows with same values in specified
columns and allows aggregate functions on each group.


{\def\LTcaptype{none} % do not increment counter
\vspace{-5pt}
\captionof{table}{GROUP BY Usage}
\vspace{-10pt}
\begin{longtable}[]{@{}lll@{}}
\toprule\noalign{}
Purpose & Function & Example \\
\midrule\noalign{}
\endhead
\bottomrule\noalign{}
\endlastfoot
\textbf{Counting} & COUNT() & Count students per department \\
\textbf{Summing} & SUM() & Total salary per department \\
\textbf{Averaging} & AVG() & Average marks per course \\
\textbf{Finding Min/Max} & MIN()/MAX() & Highest salary per
department \\
\end{longtable}
}

\textbf{Example:}

\begin{verbatim}
SELECT Department, COUNT(*) as Total\_Students, AVG(Marks) as Avg\_Marks
FROM Student 
GROUP BY Department;
\end{verbatim}

\textbf{Result:}

{\def\LTcaptype{none} % do not increment counter
\begin{longtable}[]{@{}lll@{}}
\toprule\noalign{}
Department & Total\_Students & Avg\_Marks \\
\midrule\noalign{}
\endhead
\bottomrule\noalign{}
\endlastfoot
IT & 25 & 78.5 \\
CS & 30 & 82.1 \\
\end{longtable}
}

\begin{itemize}
\tightlist
\item
  \textbf{Groups}: Creates separate groups for each department
\item
  \textbf{Aggregates}: Calculates count and average for each group
\end{itemize}

\end{solutionbox}
\begin{mnemonicbox}
``GROUP BY creates Summary Reports''

\end{mnemonicbox}
\begin{center}\rule{0.5\linewidth}{0.5pt}\end{center}

\subsection*{Question 4(b) OR [4
marks]}\label{q4b}

\textbf{Describe Set Operators in SQL with example}

\begin{solutionbox}

\textbf{Set Operators}: Combine results from two or more SELECT
statements.


{\def\LTcaptype{none} % do not increment counter
\vspace{-5pt}
\captionof{table}{SQL Set Operators}
\vspace{-10pt}
\begin{longtable}[]{@{}
  >{\raggedright\arraybackslash}p{(\linewidth - 6\tabcolsep) * \real{0.2222}}
  >{\raggedright\arraybackslash}p{(\linewidth - 6\tabcolsep) * \real{0.2889}}
  >{\raggedright\arraybackslash}p{(\linewidth - 6\tabcolsep) * \real{0.2889}}
  >{\raggedright\arraybackslash}p{(\linewidth - 6\tabcolsep) * \real{0.2000}}@{}}
\toprule\noalign{}
\begin{minipage}[b]{\linewidth}\raggedright
Operator
\end{minipage} & \begin{minipage}[b]{\linewidth}\raggedright
Description
\end{minipage} & \begin{minipage}[b]{\linewidth}\raggedright
Requirement
\end{minipage} & \begin{minipage}[b]{\linewidth}\raggedright
Example
\end{minipage} \\
\midrule\noalign{}
\endhead
\bottomrule\noalign{}
\endlastfoot
\textbf{UNION} & Combines results, removes duplicates & Same column
structure & SELECT name FROM students UNION SELECT name FROM teachers \\
\textbf{UNION ALL} & Combines results, keeps duplicates & Same column
structure & SELECT name FROM students UNION ALL SELECT name FROM
alumni \\
\textbf{INTERSECT} & Returns common records & Same column structure &
SELECT course FROM current\_courses INTERSECT SELECT course FROM
popular\_courses \\
\textbf{MINUS} & Records in first query but not second & Same column
structure & SELECT student\_id FROM enrolled MINUS SELECT student\_id
FROM graduated \\
\end{longtable}
}

\textbf{Example:}

\begin{verbatim}
{-{-} Students who are also teachers}
SELECT name FROM students
INTERSECT
SELECT name FROM teachers;

{-{-} All people in university}
SELECT name, {Student} as type FROM students
UNION
SELECT name, {Teacher} as type FROM teachers;
\end{verbatim}

\textbf{Rules:}

\begin{itemize}
\tightlist
\item
  \textbf{Column Count}: Must be same in all queries
\item
  \textbf{Data Types}: Corresponding columns must have compatible types
\item
  \textbf{Order}: ORDER BY can only be used at the end
\end{itemize}

\end{solutionbox}
\begin{mnemonicbox}
``Set operators Unite, Intersect, and Subtract data''

\end{mnemonicbox}
\begin{center}\rule{0.5\linewidth}{0.5pt}\end{center}

\subsection*{Question 4(c) OR [7
marks]}\label{q4c}

\textbf{Justify the importance of Normalization. Explain 1NF with
example}

\begin{solutionbox}

\textbf{Importance of Normalization:}


{\def\LTcaptype{none} % do not increment counter
\vspace{-5pt}
\captionof{table}{Benefits of Normalization}
\vspace{-10pt}
\begin{longtable}[]{@{}
  >{\raggedright\arraybackslash}p{(\linewidth - 4\tabcolsep) * \real{0.3000}}
  >{\raggedright\arraybackslash}p{(\linewidth - 4\tabcolsep) * \real{0.4333}}
  >{\raggedright\arraybackslash}p{(\linewidth - 4\tabcolsep) * \real{0.2667}}@{}}
\toprule\noalign{}
\begin{minipage}[b]{\linewidth}\raggedright
Benefit
\end{minipage} & \begin{minipage}[b]{\linewidth}\raggedright
Description
\end{minipage} & \begin{minipage}[b]{\linewidth}\raggedright
Impact
\end{minipage} \\
\midrule\noalign{}
\endhead
\bottomrule\noalign{}
\endlastfoot
\textbf{Eliminates Redundancy} & Reduces duplicate data storage & Saves
storage space \\
\textbf{Prevents Anomalies} & Avoids insertion, deletion, update
problems & Maintains data consistency \\
\textbf{Improves Integrity} & Ensures data accuracy & Reliable
information system \\
\textbf{Flexible Design} & Easy to modify and extend & Adaptable to
business changes \\
\end{longtable}
}

\textbf{1NF (First Normal Form)}:

\begin{itemize}
\tightlist
\item
  Eliminate duplicate columns from same table
\item
  Create separate tables for related data
\item
  Each cell contains single value (atomic values)
\end{itemize}

\textbf{Example - Unnormalized Table:}

{\def\LTcaptype{none} % do not increment counter
\begin{longtable}[]{@{}lll@{}}
\toprule\noalign{}
Student\_ID & Name & Subjects \\
\midrule\noalign{}
\endhead
\bottomrule\noalign{}
\endlastfoot
101 & John & Math, Science, English \\
102 & Mary & Science, History \\
\end{longtable}
}

\textbf{Problems:}

\begin{itemize}
\tightlist
\item
  Subjects column contains multiple values
\item
  Difficult to query specific subjects
\item
  Update anomalies when adding/removing subjects
\end{itemize}

\textbf{After 1NF:}

\textbf{Student Table:}

{\def\LTcaptype{none} % do not increment counter
\begin{longtable}[]{@{}ll@{}}
\toprule\noalign{}
Student\_ID & Name \\
\midrule\noalign{}
\endhead
\bottomrule\noalign{}
\endlastfoot
101 & John \\
102 & Mary \\
\end{longtable}
}

\textbf{Student\_Subject Table:}

{\def\LTcaptype{none} % do not increment counter
\begin{longtable}[]{@{}ll@{}}
\toprule\noalign{}
Student\_ID & Subject \\
\midrule\noalign{}
\endhead
\bottomrule\noalign{}
\endlastfoot
101 & Math \\
101 & Science \\
101 & English \\
102 & Science \\
102 & History \\
\end{longtable}
}

\textbf{Benefits:}

\begin{itemize}
\tightlist
\item
  \textbf{Atomic Values}: Each cell contains single value
\item
  \textbf{Flexible Queries}: Easy to find students studying specific
  subjects
\item
  \textbf{Easy Updates}: Add/remove subjects without affecting other
  data
\end{itemize}

\end{solutionbox}
\begin{mnemonicbox}
``1NF = One value per cell, No repeating groups''

\end{mnemonicbox}
\begin{center}\rule{0.5\linewidth}{0.5pt}\end{center}

\subsection*{Question 5(a) [3 marks]}\label{q5a}

\textbf{Explain Serializability in Transaction Management}

\begin{solutionbox}

\textbf{Serializability}: Property that ensures concurrent execution of
transactions produces same result as some serial execution of those
transactions.


{\def\LTcaptype{none} % do not increment counter
\vspace{-5pt}
\captionof{table}{Types of Serializability}
\vspace{-10pt}
\begin{longtable}[]{@{}
  >{\raggedright\arraybackslash}p{(\linewidth - 4\tabcolsep) * \real{0.2222}}
  >{\raggedright\arraybackslash}p{(\linewidth - 4\tabcolsep) * \real{0.4815}}
  >{\raggedright\arraybackslash}p{(\linewidth - 4\tabcolsep) * \real{0.2963}}@{}}
\toprule\noalign{}
\begin{minipage}[b]{\linewidth}\raggedright
Type
\end{minipage} & \begin{minipage}[b]{\linewidth}\raggedright
Description
\end{minipage} & \begin{minipage}[b]{\linewidth}\raggedright
Method
\end{minipage} \\
\midrule\noalign{}
\endhead
\bottomrule\noalign{}
\endlastfoot
\textbf{Conflict Serializability} & Based on conflicting operations &
Precedence graph \\
\textbf{View Serializability} & Based on read-write patterns & View
equivalence \\
\end{longtable}
}

\textbf{Example:} Transaction T1: R(A), W(A), R(B), W(B) Transaction T2:
R(A), W(A), R(B), W(B)

\textbf{Serial Schedule:} T1 \rightarrow T2 or T2 \rightarrow T1 \textbf{Concurrent
Schedule:} Interleaved operations

\begin{itemize}
\tightlist
\item
  \textbf{Conflict Operations}: Operations on same data item where at
  least one is write
\item
  \textbf{Serializable Schedule}: Equivalent to some serial schedule
\item
  \textbf{Non-serializable}: May lead to inconsistent database state
\end{itemize}

\end{solutionbox}
\begin{mnemonicbox}
``Serializability ensures Transaction Consistency''

\end{mnemonicbox}
\begin{center}\rule{0.5\linewidth}{0.5pt}\end{center}

\subsection*{Question 5(b) [4 marks]}\label{q5b}

\textbf{Describe Partial Functional Dependency with example}

\begin{solutionbox}

\textbf{Partial Functional Dependency}: When a non-key attribute is
functionally dependent on only part of a composite primary key.


{\def\LTcaptype{none} % do not increment counter
\vspace{-5pt}
\captionof{table}{Functional Dependency Types}
\vspace{-10pt}
\begin{longtable}[]{@{}
  >{\raggedright\arraybackslash}p{(\linewidth - 4\tabcolsep) * \real{0.2222}}
  >{\raggedright\arraybackslash}p{(\linewidth - 4\tabcolsep) * \real{0.4444}}
  >{\raggedright\arraybackslash}p{(\linewidth - 4\tabcolsep) * \real{0.3333}}@{}}
\toprule\noalign{}
\begin{minipage}[b]{\linewidth}\raggedright
Type
\end{minipage} & \begin{minipage}[b]{\linewidth}\raggedright
Definition
\end{minipage} & \begin{minipage}[b]{\linewidth}\raggedright
Example
\end{minipage} \\
\midrule\noalign{}
\endhead
\bottomrule\noalign{}
\endlastfoot
\textbf{Full Dependency} & Depends on entire primary key & (Student\_ID,
Course\_ID) \rightarrow Grade \\
\textbf{Partial Dependency} & Depends on part of primary key &
(Student\_ID, Course\_ID) \rightarrow Student\_Name \\
\end{longtable}
}

\textbf{Example:} \textbf{Enrollment Table:} Primary Key: (Student\_ID,
Course\_ID)

{\def\LTcaptype{none} % do not increment counter
\begin{longtable}[]{@{}lllll@{}}
\toprule\noalign{}
Student\_ID & Course\_ID & Student\_Name & Course\_Name & Grade \\
\midrule\noalign{}
\endhead
\bottomrule\noalign{}
\endlastfoot
101 & C1 & John & DBMS & A \\
101 & C2 & John & Java & B \\
\end{longtable}
}

\textbf{Partial Dependencies:}

\begin{itemize}
\tightlist
\item
  Student\_ID \rightarrow Student\_Name (Student\_Name depends only on
  Student\_ID)
\item
  Course\_ID \rightarrow Course\_Name (Course\_Name depends only on Course\_ID)
\end{itemize}

\textbf{Problems:}

\begin{itemize}
\tightlist
\item
  \textbf{Update Anomaly}: Changing student name requires multiple
  updates
\item
  \textbf{Insertion Anomaly}: Cannot add student without enrolling in
  course
\item
  \textbf{Deletion Anomaly}: Deleting enrollment may lose student
  information
\end{itemize}

\textbf{Solution}: Normalize to 2NF by removing partial dependencies

\end{solutionbox}
\begin{mnemonicbox}
``Partial dependency = Part of key determines
attribute''

\end{mnemonicbox}
\begin{center}\rule{0.5\linewidth}{0.5pt}\end{center}

\subsection*{Question 5(c) [7 marks]}\label{q5c}

\textbf{Write a Short note on Locking Mechanism with example in
Transaction Management}

\begin{solutionbox}

\textbf{Locking Mechanism}: Concurrency control technique that prevents
simultaneous access to data items during transaction execution.


{\def\LTcaptype{none} % do not increment counter
\vspace{-5pt}
\captionof{table}{Types of Locks}
\vspace{-10pt}
\begin{longtable}[]{@{}
  >{\raggedright\arraybackslash}p{(\linewidth - 4\tabcolsep) * \real{0.3548}}
  >{\raggedright\arraybackslash}p{(\linewidth - 4\tabcolsep) * \real{0.4194}}
  >{\raggedright\arraybackslash}p{(\linewidth - 4\tabcolsep) * \real{0.2258}}@{}}
\toprule\noalign{}
\begin{minipage}[b]{\linewidth}\raggedright
Lock Type
\end{minipage} & \begin{minipage}[b]{\linewidth}\raggedright
Description
\end{minipage} & \begin{minipage}[b]{\linewidth}\raggedright
Usage
\end{minipage} \\
\midrule\noalign{}
\endhead
\bottomrule\noalign{}
\endlastfoot
\textbf{Shared Lock (S)} & Multiple transactions can read & Read
operations \\
\textbf{Exclusive Lock (X)} & Only one transaction can access & Write
operations \\
\textbf{Intention Lock} & Indicates intent to lock at lower level &
Hierarchical locking \\
\end{longtable}
}

\textbf{Two-Phase Locking (2PL) Protocol:}

\begin{enumerate}
\tightlist
\item
  \textbf{Growing Phase}: Acquire locks, cannot release any lock
\item
  \textbf{Shrinking Phase}: Release locks, cannot acquire new locks
\end{enumerate}

\textbf{Example:}

\begin{verbatim}
Transaction T1: Read(A), Write(A), Read(B), Write(B)
Transaction T2: Read(A), Write(A), Read(C), Write(C)

T1: S-lock(A), Read(A), X-lock(A), Write(A), S-lock(B), Read(B), X-lock(B), Write(B), Unlock(A), Unlock(B)
T2: Wait for A, S-lock(A), Read(A), X-lock(A), Write(A), S-lock(C), Read(C), X-lock(C), Write(C), Unlock(A), Unlock(C)
\end{verbatim}

\textbf{Lock Compatibility Matrix:}

{\def\LTcaptype{none} % do not increment counter
\begin{longtable}[]{@{}lll@{}}
\toprule\noalign{}
Current/Requested & S & X \\
\midrule\noalign{}
\endhead
\bottomrule\noalign{}
\endlastfoot
\textbf{S} & ✓ & ✗ \\
\textbf{X} & ✗ & ✗ \\
\end{longtable}
}

\textbf{Problems:}

\begin{itemize}
\tightlist
\item
  \textbf{Deadlock}: Two transactions waiting for each other's locks
\item
  \textbf{Starvation}: Transaction waits indefinitely for lock
\end{itemize}

\textbf{Solutions:}

\begin{itemize}
\tightlist
\item
  \textbf{Deadlock Detection}: Use wait-for graph
\item
  \textbf{Deadlock Prevention}: Timestamp-based protocols
\end{itemize}

\end{solutionbox}
\begin{mnemonicbox}
``Locking prevents Concurrent Conflicts''

\end{mnemonicbox}
\begin{center}\rule{0.5\linewidth}{0.5pt}\end{center}

\subsection*{Question 5(a) OR [3
marks]}\label{q5a}

\textbf{Explain Deadlock in Transaction Management}

\begin{solutionbox}

\textbf{Deadlock}: Situation where two or more transactions are waiting
indefinitely for each other to release locks, creating a circular wait
condition.


{\def\LTcaptype{none} % do not increment counter
\vspace{-5pt}
\captionof{table}{Deadlock Components}
\vspace{-10pt}
\begin{longtable}[]{@{}
  >{\raggedright\arraybackslash}p{(\linewidth - 4\tabcolsep) * \real{0.3333}}
  >{\raggedright\arraybackslash}p{(\linewidth - 4\tabcolsep) * \real{0.3939}}
  >{\raggedright\arraybackslash}p{(\linewidth - 4\tabcolsep) * \real{0.2727}}@{}}
\toprule\noalign{}
\begin{minipage}[b]{\linewidth}\raggedright
Component
\end{minipage} & \begin{minipage}[b]{\linewidth}\raggedright
Description
\end{minipage} & \begin{minipage}[b]{\linewidth}\raggedright
Example
\end{minipage} \\
\midrule\noalign{}
\endhead
\bottomrule\noalign{}
\endlastfoot
\textbf{Mutual Exclusion} & Resources cannot be shared & Exclusive
locks \\
\textbf{Hold and Wait} & Process holds resources while waiting & T1
holds A, waits for B \\
\textbf{No Preemption} & Resources cannot be forcibly taken & Locks
cannot be revoked \\
\textbf{Circular Wait} & Circular chain of waiting processes &
T1\rightarrowT2\rightarrowT1 \\
\end{longtable}
}

\textbf{Example:}

\begin{verbatim}
Transaction T1: Lock(A), Lock(B)
Transaction T2: Lock(B), Lock(A)

Time 1: T1 gets Lock(A)
Time 2: T2 gets Lock(B) 
Time 3: T1 waits for Lock(B) - held by T2
Time 4: T2 waits for Lock(A) - held by T1
Result: DEADLOCK!
\end{verbatim}

\textbf{Detection}: Use wait-for graph to identify cycles
\textbf{Prevention}: Use timestamp ordering or wound-wait protocols

\end{solutionbox}
\begin{mnemonicbox}
``Deadlock = Circular Waiting for Resources''

\end{mnemonicbox}
\begin{center}\rule{0.5\linewidth}{0.5pt}\end{center}

\subsection*{Question 5(b) OR [4
marks]}\label{q5b}

\textbf{Describe Full Functional Dependency with example}

\begin{solutionbox}

\textbf{Full Functional Dependency}: A non-key attribute is functionally
dependent on the entire primary key (not just part of it).


{\def\LTcaptype{none} % do not increment counter
\vspace{-5pt}
\captionof{table}{Dependency Comparison}
\vspace{-10pt}
\begin{longtable}[]{@{}
  >{\raggedright\arraybackslash}p{(\linewidth - 4\tabcolsep) * \real{0.2222}}
  >{\raggedright\arraybackslash}p{(\linewidth - 4\tabcolsep) * \real{0.4444}}
  >{\raggedright\arraybackslash}p{(\linewidth - 4\tabcolsep) * \real{0.3333}}@{}}
\toprule\noalign{}
\begin{minipage}[b]{\linewidth}\raggedright
Type
\end{minipage} & \begin{minipage}[b]{\linewidth}\raggedright
Definition
\end{minipage} & \begin{minipage}[b]{\linewidth}\raggedright
Example
\end{minipage} \\
\midrule\noalign{}
\endhead
\bottomrule\noalign{}
\endlastfoot
\textbf{Full Dependency} & Depends on complete primary key &
(Student\_ID, Course\_ID) \rightarrow Grade \\
\textbf{Partial Dependency} & Depends on part of primary key &
(Student\_ID, Course\_ID) \rightarrow Student\_Name \\
\end{longtable}
}

\textbf{Example:} \textbf{Enrollment Table:} Primary Key: (Student\_ID,
Course\_ID)

{\def\LTcaptype{none} % do not increment counter
\begin{longtable}[]{@{}llll@{}}
\toprule\noalign{}
Student\_ID & Course\_ID & Grade & Hours \\
\midrule\noalign{}
\endhead
\bottomrule\noalign{}
\endlastfoot
101 & C1 & A & 4 \\
101 & C2 & B & 3 \\
102 & C1 & B & 4 \\
\end{longtable}
}

\textbf{Full Functional Dependencies:}

\begin{itemize}
\tightlist
\item
  (Student\_ID, Course\_ID) \rightarrow Grade ✓
\item
  (Student\_ID, Course\_ID) \rightarrow Hours ✓
\end{itemize}

\textbf{Explanation:}

\begin{itemize}
\tightlist
\item
  \textbf{Grade} depends on both Student\_ID AND Course\_ID (specific
  student in specific course)
\item
  \textbf{Hours} also depends on both (student's hours in specific
  course)
\item
  Cannot determine Grade from Student\_ID alone
\item
  Cannot determine Grade from Course\_ID alone
\end{itemize}

\textbf{Benefits:}

\begin{itemize}
\tightlist
\item
  \textbf{No Update Anomalies}: Changes affect only relevant records
\item
  \textbf{Proper Normalization}: Supports 2NF requirements
\item
  \textbf{Data Integrity}: Ensures accurate relationships
\end{itemize}

\end{solutionbox}
\begin{mnemonicbox}
``Full dependency needs Complete Key''

\end{mnemonicbox}
\begin{center}\rule{0.5\linewidth}{0.5pt}\end{center}

\subsection*{Question 5(c) OR [7
marks]}\label{q5c}

\textbf{Explain ACID Properties of Transaction with example}

\begin{solutionbox}

\textbf{ACID Properties}: Four fundamental properties that guarantee
database transaction reliability.


{\def\LTcaptype{none} % do not increment counter
\vspace{-5pt}
\captionof{table}{ACID Properties}
\vspace{-10pt}
\begin{longtable}[]{@{}
  >{\raggedright\arraybackslash}p{(\linewidth - 4\tabcolsep) * \real{0.3125}}
  >{\raggedright\arraybackslash}p{(\linewidth - 4\tabcolsep) * \real{0.4062}}
  >{\raggedright\arraybackslash}p{(\linewidth - 4\tabcolsep) * \real{0.2812}}@{}}
\toprule\noalign{}
\begin{minipage}[b]{\linewidth}\raggedright
Property
\end{minipage} & \begin{minipage}[b]{\linewidth}\raggedright
Description
\end{minipage} & \begin{minipage}[b]{\linewidth}\raggedright
Example
\end{minipage} \\
\midrule\noalign{}
\endhead
\bottomrule\noalign{}
\endlastfoot
\textbf{Atomicity} & All or nothing execution & Bank transfer: both
debit and credit must happen \\
\textbf{Consistency} & Database remains in valid state & Account balance
cannot be negative \\
\textbf{Isolation} & Transactions don't interfere & Concurrent
transactions appear sequential \\
\textbf{Durability} & Committed changes are permanent & Data survives
system crashes \\
\end{longtable}
}

\textbf{Detailed Examples:}

\textbf{Atomicity Example:}

\begin{verbatim}
BEGIN TRANSACTION;
UPDATE Account SET Balance = Balance {-} 1000 WHERE AccNo = {A001};
UPDATE Account SET Balance = Balance + 1000 WHERE AccNo = {A002};
COMMIT;
\end{verbatim}

\emph{If either update fails, entire transaction is rolled back}

\textbf{Consistency Example:}

\begin{verbatim}
{-{-} Before: A001 = 5000, A002 = 3000, Total = 8000}
{-{-} Transfer 1000 from A001 to A002}
{-{-} After: A001 = 4000, A002 = 4000, Total = 8000}
{-{-} Total money in system remains constant}
\end{verbatim}

\textbf{Isolation Example:}

\begin{verbatim}
T1: Read(A=100),

A=A+50, Write(A=150)

T2: Read(A=100),

A=A*2, Write(A=200)

Serial Result:

A=300 or

A=250

Isolated execution must produce one of these results
\end{verbatim}

\textbf{Durability Example:}

\begin{verbatim}
After COMMIT is executed, even if system crashes,
the transferred amount remains in destination account
\end{verbatim}

\textbf{Implementation:}

\begin{itemize}
\tightlist
\item
  \textbf{Atomicity}: Using transaction logs and rollback
\item
  \textbf{Consistency}: Using constraints and triggers
\item
  \textbf{Isolation}: Using locking mechanisms
\item
  \textbf{Durability}: Using write-ahead logging
\end{itemize}

\end{solutionbox}
\begin{mnemonicbox}
``ACID keeps Transactions Reliable''

\end{mnemonicbox}

\end{document}
