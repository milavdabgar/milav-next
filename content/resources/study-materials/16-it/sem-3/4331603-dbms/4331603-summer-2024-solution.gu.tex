\documentclass{article}
% Adjust the relative path to point to the latex-templates directory

% content/resources/templates/preamble.tex
\usepackage[margin=0.6in]{geometry}
\author{Milav Dabgar}
\usepackage{amsmath,amssymb,amsthm}
\usepackage{booktabs}
\usepackage{multirow}
\usepackage{xcolor}
\usepackage{tcolorbox}
\tcbuselibrary{breakable,skins}
\usepackage[colorlinks=true,linkcolor=blue]{hyperref}
\usepackage{titlesec}
\usepackage{enumitem}
\usepackage{tikz}
\usepackage{pgfplots}
\usepackage{circuitikz}
\usepackage[version=4]{mhchem}
\usepackage{longtable}
\usepackage{array}
\usepackage{float}
\usepackage{caption}
\usepackage{listings}

\lstset{
  basicstyle=\small\ttfamily,
  breaklines=true,
  breakatwhitespace=false,
  postbreak=\mbox{\textcolor{red}{$\hookrightarrow$}\space},
  float=false,
  numbers=left,
  numberstyle=\tiny\color{gray},
  numbersep=10pt,
  xleftmargin=2em,
  keywordstyle=\color{blue},
  commentstyle=\color{green!60!black},
  stringstyle=\color{purple},
  backgroundcolor=\color{gray!5},
  showstringspaces=false,
  tabsize=2,
  captionpos=b,
  keepspaces=true,
  columns=flexible
}

\pgfplotsset{compat=1.18}
\usetikzlibrary{shapes,arrows,positioning,calc,patterns,decorations.pathmorphing,decorations.markings,arrows.meta}

% Color scheme
\definecolor{headcolor}{RGB}{0,102,204}
\definecolor{keycolor}{RGB}{220,20,60}
\definecolor{solutioncolor}{RGB}{34,139,34}
\definecolor{mnemoniccolor}{RGB}{148,0,211}
\definecolor{codecolor}{RGB}{0,0,100}

% Spacing
\setlength{\parskip}{3pt}
\setlist[itemize]{nosep}
\setlist[enumerate]{nosep}

% Title formatting
\titleformat{\section}{\Large\bfseries\color{headcolor}}{\thesection}{1em}{}
\titleformat{\subsection}{\large\bfseries\color{headcolor}}{\thesubsection}{1em}{}

% Pandoc tightlist compatibility
\providecommand{\tightlist}{%
  \setlength{\itemsep}{0pt}\setlength{\parskip}{0pt}}

% Pandoc longtable compatibility
\newcounter{none}
\def\thenone{}


% content/resources/templates/gujarati-boxes.tex
\usepackage{fontspec}
\usepackage{polyglossia}

% Set Gujarati as main language (document is primarily in Gujarati)
% Note: gloss-gujarati.ldf doesn't exist in polyglossia, but it will use hyphenation patterns
\setdefaultlanguage{gujarati}
\setotherlanguage{english}

% Configure Gujarati font properly
% Use Language=Default to prevent polyglossia from trying to add language-specific features
% that don't exist for Gujarati, which causes "empty feature" warnings
\newfontfamily\gujaratifont[Script=Gujarati,AutoFakeBold=2.5,AutoFakeSlant=0.3]{Noto Sans Gujarati}
\setmainfont[Script=Gujarati,AutoFakeBold=2.5,AutoFakeSlant=0.3]{Noto Sans Gujarati}
% Use Noto Sans Gujarati for monospace to support Gujarati in text
\setmonofont[Scale=0.9]{Noto Sans Gujarati}

% Configure English to use the same font
\newfontfamily\englishfont[Script=Gujarati,AutoFakeBold=2.5,AutoFakeSlant=0.3]{Noto Sans Gujarati}

% Translations for polyglossia
\gappto\captionsgujarati{
  \renewcommand{\tablename}{કોષ્ટક}
  \renewcommand{\figurename}{આકૃતિ}
}

% Helper for TikZ nodes to ensure Gujarati font
\newcommand{\gu}[1]{{\gujaratifont #1}}

% Custom environments
\newtcolorbox{solutionbox}{
    breakable,
    enhanced,
    colback=solutioncolor!5!white,
    colframe=solutioncolor!75!black,
    fonttitle=\bfseries,
    title=જવાબ
}

\newtcolorbox{solutionboxnobreak}{
 colback=solutioncolor!5!white,
 colframe=solutioncolor!75!black,
 fonttitle=\bfseries,
 title=જવાબ
}

\newtcolorbox{keyformula}{
 breakable,
 enhanced,
 colback=keycolor!5!white,
 colframe=keycolor!75!black,
 fonttitle=\bfseries,
 title=રાસાયણિક સમીકરણ/સૂત્ર
}

\newtcolorbox{mnemonicbox}{
 breakable,
 enhanced,
 colback=mnemoniccolor!5!white,
 colframe=mnemoniccolor!75!black,
 fonttitle=\bfseries,
 title=મેમરી ટ્રીક
}


% Custom commands for GTU solutions
% This file defines semantic commands for consistent formatting

% Question command with automatic formatting
\newcommand{\question}[2]{%
  \section*{Question #1}%
  \textbf{#2}%
}

% OR question variant
\newcommand{\questionor}[2]{%
  \section*{Question #1 OR}%
  \textbf{#2}%
}

% Proper table environment with caption
\newenvironment{answertable}[1]{%
  \begin{table}[htbp]
  \centering
  \caption{#1}
}{%
  \end{table}
}

% Proper figure environment for diagrams
\newenvironment{answerdiagram}[1]{%
  \begin{figure}[htbp]
  \centering
  \caption{#1}
}{%
  \end{figure}
}

% Semantic markup for key terms
\newcommand{\keyword}[1]{\textbf{#1}}
\newcommand{\code}[1]{\texttt{#1}}
\newcommand{\classname}[1]{\texttt{#1}}
\newcommand{\methodname}[1]{\texttt{#1}}

% Proper quotation marks
\newcommand{\mnemonic}[1]{``#1''}


\newcommand{\xmark}{\ding{55}}

\tikzset{
    gtu database/.style={cylinder, shape border rotate=90, draw, fill=yellow!10, aspect=0.25, align=center}
}

\title{Database Management (4331603) - Summer 2024 Solution (Gujarati)}
\date{June 12, 2024}

\begin{document}
\maketitle


\questionmarks{1(અ)}{3}{નીચેના શબ્દોને વ્યાખ્યાયિત કરો: 1. ડેટા 2. ઇન્ફોર્મેશન 3. મેટાડેટા}

\begin{solutionbox}
\begin{center}
\captionof{table}{ડેટા વિ ઇન્ફોર્મેશન વિ મેટાડેટા}
\begin{tabulary}{\linewidth}{|L|L|L|}
\hline
\textbf{શબ્દ} & \textbf{વ્યાખ્યા} & \textbf{ઉદાહરણ} \\ \hline
\textbf{ડેટા} & કોઈ સંદર્ભ વગરના કાચા તથ્યો અને આંકડાઓ & "25", "જોન", "મુંબઈ" \\ \hline
\textbf{ઇન્ફોર્મેશન} & અર્થ અને સંદર્ભ સાથે પ્રોસેસ કરેલા ડેટા & "જોન 25 વર્ષનો છે અને મુંબઈમાં રહે છે" \\ \hline
\textbf{મેટાડેટા} & ડેટા વિશેનો ડેટા જે સ્ટ્રક્ચર અને પ્રોપર્ટીઝ વર્ણવે છે & "ઉંમર ફીલ્ડ: Integer, મહત્તમ લંબાઈ: 3" \\ \hline
\end{tabulary}
\end{center}

\begin{itemize}
    \item \keyword{ડેટા}: ઇન્ફોર્મેશન સિસ્ટમ્સના મૂળભૂત બિલ્ડિંગ બ્લોક્સ
    \item \keyword{ઇન્ફોર્મેશન}: નિર્ણય લેવા માટે ડેટા પ્રોસેસિંગનું પરિણામ
    \item \keyword{મેટાડેટા}: ડેટાબેસ ડિઝાઇન અને મેનેજમેન્ટ માટે જરૂરી
\end{itemize}
\end{solutionbox}

\begin{mnemonicbox}
\mnemonic{DIM - ડેટા મેટાડેટાનો ઉપયોગ કરીને ઇન્ફોર્મેશન આપે છે}
\end{mnemonicbox}

\questionmarks{1(બ)}{4}{ફાઇલ સિસ્ટમ વિ ડેટાબેસ સિસ્ટમની તુલના કરો}

\begin{solutionbox}
\begin{center}
\captionof{table}{ફાઇલ સિસ્ટમ વિ ડેટાબેસ સિસ્ટમ તુલના}
\begin{tabulary}{\linewidth}{|L|L|L|}
\hline
\textbf{પાસું} & \textbf{ફાઇલ સિસ્ટમ} & \textbf{ડેટાબેસ સિસ્ટમ} \\ \hline
\textbf{ડેટા સ્ટોરેજ} & દરેક એપ્લિકેશન માટે અલગ ફાઇલો & કેન્દ્રીકૃત સ્ટોરેજ \\ \hline
\textbf{ડેટા રિડન્ડન્સી} & ઉચ્ચ રિડન્ડન્સી & લઘુત્તમ રિડન્ડન્સી \\ \hline
\textbf{ડેટા સુસંગતતા} & નબળી સુસંગતતા & ઉચ્ચ સુસંગતતા \\ \hline
\textbf{ડેટા સિક્યોરિટી} & મર્યાદિત સિક્યોરિટી & એડવાન્સ સિક્યોરિટી ફીચર્સ \\ \hline
\textbf{એકસાથે એક્સેસ} & મર્યાદિત સપોર્ટ & સંપૂર્ણ એકસાથે સપોર્ટ \\ \hline
\textbf{ડેટા ઇન્ડિપેન્ડન્સ} & કોઈ ઇન્ડિપેન્ડન્સ નથી & ફિઝિકલ અને લોજિકલ ઇન્ડિપેન્ડન્સ \\ \hline
\end{tabulary}
\end{center}

\begin{itemize}
    \item \keyword{ફાઇલ સિસ્ટમ}: સરળ પણ ડેટા ડુપ્લિકેશનની સમસ્યાઓ સાથે
    \item \keyword{ડેટાબેસ સિસ્ટમ}: જટિલ પણ કાર્યક્ષમ ડેટા મેનેજમેન્ટ
    \item \keyword{મુખ્ય ફાયદો}: DBMS ડેટા રિડન્ડન્સી અને અસુસંગતતા દૂર કરે છે
\end{itemize}
\end{solutionbox}

\begin{mnemonicbox}
\mnemonic{DBMS = ડેટા બેટર મેનેજ્ડ સિસ્ટમેટિકલી}
\end{mnemonicbox}

\questionmarks{1(ક)}{7}{નેટવર્ક ડેટા મોડેલ દોરો અને સમજાવો}

\begin{solutionbox}
\begin{center}
\begin{tikzpicture}[node distance=1.5cm, auto]
    % Owner
    \node [gtu block] (O1) {Owner 1};
    
    % Set Type 1
    \node [gtu state, below=1cm of O1] (ST1) {Set Type 1};
    
    % Members
    \node [gtu block, below left=1.5cm and 1cm of ST1] (M1) {Member 1};
    \node [gtu block, below=1.5cm of ST1] (M2) {Member 2};
    \node [gtu block, below right=1.5cm and 1cm of ST1] (M3) {Member 3};
    
    % Set Types for Members
    \node [gtu state, below=0.8cm of M1] (ST2) {Set Type 2};
    \node [gtu state, below=0.8cm of M2] (ST3) {Set Type 3};
    \node [gtu state, below=0.8cm of M3] (ST4) {Set Type 4};
    
    % Child Members
    \node [gtu block, below=0.8cm of ST2] (M4) {Member 4};
    \node [gtu block, below=0.8cm of ST3] (M5) {Member 5};
    \node [gtu block, below=0.8cm of ST4] (M6) {Member 6};

    % Connections
    \draw [gtu arrow] (O1) -- (ST1);
    \draw [gtu arrow] (ST1) -- (M1);
    \draw [gtu arrow] (ST1) -- (M2);
    \draw [gtu arrow] (ST1) -- (M3);
    
    \draw [gtu arrow] (M1) -- (ST2);
    \draw [gtu arrow] (ST2) -- (M4);
    
    \draw [gtu arrow] (M2) -- (ST3);
    \draw [gtu arrow] (ST3) -- (M5);
    
    \draw [gtu arrow] (M3) -- (ST4);
    \draw [gtu arrow] (ST4) -- (M6);
    
\end{tikzpicture}
\captionof{figure}{નેટવર્ક ડેટા મોડેલ સ્ટ્રક્ચર}
\end{center}

\begin{center}
\captionof{table}{નેટવર્ક મોડેલના ઘટકો}
\begin{tabulary}{\linewidth}{|L|L|L|}
\hline
\textbf{ઘટક} & \textbf{વર્ણન} & \textbf{ઉદાહરણ} \\ \hline
\textbf{રેકોર્ડ ટાઇપ} & એન્ટિટીનું પ્રતિનિધિત્વ & કર્મચારી, વિભાગ \\ \hline
\textbf{સેટ ટાઇપ} & રેકોર્ડ્સ વચ્ચેનો સંબંધ & કામ-કરે, મેનેજ-કરે \\ \hline
\textbf{ઓનર} & સંબંધમાં પેરેન્ટ રેકોર્ડ & વિભાગ (ઓનર) \\ \hline
\textbf{મેમ્બર} & સંબંધમાં ચાઇલ્ડ રેકોર્ડ & કર્મચારી (મેમ્બર) \\ \hline
\end{tabulary}
\end{center}

\begin{itemize}
    \item \keyword{ઓનર રેકોર્ડ}: સેટને નિયંત્રિત કરે છે અને અનેક મેમ્બર્સ હોઈ શકે છે
    \item \keyword{મેમ્બર રેકોર્ડ}: એક અથવા વધુ સેટ્સનું સભ્ય છે
    \item \keyword{સેટ ઓકરન્સ}: સેટ ટાઇપનું ઇન્સ્ટન્સ જે ઓનરને મેમ્બર્સ સાથે જોડે છે
    \item \keyword{નેવિગેશન}: રેકોર્ડ એક્સેસ માટે પોઇન્ટર્સનો ઉપયોગ
\end{itemize}
\end{solutionbox}

\begin{mnemonicbox}
\mnemonic{નેટવર્ક = અનેક કનેક્શન્સ સાથેના નોડ્સ}
\end{mnemonicbox}

\questionmarks{1(ક) અથવા}{7}{સ્કીમા શું છે? ઉદાહરણ સાથે સ્કીમાના વિવિધ પ્રકારો સમજાવો}

\begin{solutionbox}
\textbf{વ્યાખ્યા}: સ્કીમા એ ડેટાબેસનું લોજિકલ સ્ટ્રક્ચર અથવા બ્લુપ્રિન્ટ છે જે વ્યાખ્યાયિત કરે છે કે ડેટા કેવી રીતે ગોઠવાયેલો છે.

\begin{center}
\begin{tikzpicture}[node distance=1.5cm, auto]
    % Nodes
    \node [gtu block] (Ext) {External Schema (View Level)};
    \node [gtu block, below=1cm of Ext] (Con) {Conceptual Schema (Logical Level)};
    \node [gtu block, below=1cm of Con] (Int) {Internal Schema (Physical Level)};
    
    % Views attached to External
    \node [gtu state, above left=1cm and 0.5cm of Ext] (V1) {View 1};
    \node [gtu state, above right=1cm and 0.5cm of Ext] (V2) {View 2};
    
    % Physical Storage
    \node [gtu database, below=1cm of Int] (Phy) {Physical Storage};

    % Connections
    \draw [gtu arrow] (V1) -- (Ext);
    \draw [gtu arrow] (V2) -- (Ext);
    \draw [gtu arrow] (Ext) -- (Con);
    \draw [gtu arrow] (Con) -- (Int);
    \draw [gtu arrow] (Int) -- (Phy);
\end{tikzpicture}
\captionof{figure}{ત્રણ-સ્કીમા આર્કિટેક્ચર}
\end{center}

\begin{center}
\captionof{table}{સ્કીમાના પ્રકારો}
\begin{tabulary}{\linewidth}{|L|L|L|L|}
\hline
\textbf{સ્કીમા પ્રકાર} & \textbf{લેવલ} & \textbf{વર્ણન} & \textbf{ઉદાહરણ} \\ \hline
\textbf{એક્સટર્નલ સ્કીમા} & વ્યૂ લેવલ & ડેટાબેસનો યુઝર-સ્પેસિફિક વ્યૂ & શિક્ષકો માટે વિદ્યાર્થીઓના ગ્રેડ્સનો વ્યૂ \\ \hline
\textbf{કોન્સેપ્ચુઅલ સ્કીમા} & લોજિકલ લેવલ & સંપૂર્ણ લોજિકલ સ્ટ્રક્ચર & બધા ટેબલ્સ, સંબંધો, કન્સ્ટ્રેન્ટ્સ \\ \hline
\textbf{ઇન્ટર્નલ સ્કીમા} & ફિઝિકલ લેવલ & ફિઝિકલ સ્ટોરેજ સ્ટ્રક્ચર & ઇન્ડેક્સ ફાઇલો, સ્ટોરેજ એલોકેશન \\ \hline
\end{tabulary}
\end{center}

\begin{itemize}
    \item \keyword{એક્સટર્નલ સ્કીમા}: યુઝર્સ માટે ડેટા ઇન્ડિપેન્ડન્સ પ્રદાન કરે છે
    \item \keyword{કોન્સેપ્ચુઅલ સ્કીમા}: ડેટાબેસ ડિઝાઇનરનો સંપૂર્ણ વ્યૂ
    \item \keyword{ઇન્ટર્નલ સ્કીમા}: ડેટાબેસ એડમિનિસ્ટ્રેટરનો ફિઝિકલ વ્યૂ
\end{itemize}
\end{solutionbox}

\begin{mnemonicbox}
\mnemonic{ECI - એક્સટર્નલ કોન્સેપ્ચુઅલ ઇન્ટર્નલ}
\end{mnemonicbox}


\questionmarks{2(અ)}{3}{નીચેના શબ્દોને વ્યાખ્યાયિત કરો: 1. એન્ટિટી 2. એટ્રિબ્યુટ્સ 3. રિલેશનશિપ}

\begin{solutionbox}
\begin{center}
\captionof{table}{ER મોડેલની મૂળભૂત કોન્સેપ્ટ્સ}
\begin{tabulary}{\linewidth}{|L|L|L|}
\hline
\textbf{શબ્દ} & \textbf{વ્યાખ્યા} & \textbf{ઉદાહરણ} \\ \hline
\textbf{એન્ટિટી} & સ્વતંત્ર અસ્તિત્વ ધરાવતો વાસ્તવિક વિશ્વનો ઓબ્જેક્ટ & વિદ્યાર્થી, કોર્સ, શિક્ષક \\ \hline
\textbf{એટ્રિબ્યુટ્સ} & એન્ટિટીનું વર્ણન કરતા ગુણધર્મો & વિદ્યાર્થી: ID, નામ, ઉંમર \\ \hline
\textbf{રિલેશનશિપ} & બે અથવા વધુ એન્ટિટી વચ્ચેનો સંબંધ & વિદ્યાર્થી કોર્સમાં નોંધણી કરે છે \\ \hline
\end{tabulary}
\end{center}

\begin{itemize}
    \item \keyword{એન્ટિટી}: ER ડાયાગ્રામમાં લંબચોરસ દ્વારા રજૂ થાય છે
    \item \keyword{એટ્રિબ્યુટ્સ}: એન્ટિટીઓ સાથે જોડાયેલા અંડાકાર દ્વારા રજૂ થાય છે
    \item \keyword{રિલેશનશિપ}: એન્ટિટીઓને જોડતા હીરા દ્વારા રજૂ થાય છે
\end{itemize}
\end{solutionbox}

\begin{mnemonicbox}
\mnemonic{EAR - એન્ટિટીના એટ્રિબ્યુટ્સ અને રિલેશનશિપ્સ છે}
\end{mnemonicbox}

\questionmarks{2(બ)}{4}{ઉદાહરણ સાથે વીક એન્ટિટી સેટ્સનું વર્ણન કરો}

\begin{solutionbox}
\textbf{વ્યાખ્યા}: વીક એન્ટિટી એ એવી એન્ટિટી છે જે પોતાના એટ્રિબ્યુટ્સ દ્વારા અનન્ય રીતે ઓળખાઈ શકતી નથી અને સ્ટ્રોંગ એન્ટિટી પર આધાર રાખે છે.

\begin{center}
\begin{tikzpicture}[node distance=2cm, auto, thick]
    % Strong Entity
    \node [gtu block] (Emp) {Employee};
    \node [ellipse, draw, above left=0.5cm of Emp] (EmpId) {\underline{emp\_id}};
    \draw (Emp) -- (EmpId);

    % Weak Entity
    \node [gtu block, double, right=4cm of Emp] (Dep) {Dependent};
    \node [ellipse, draw, dashed, above right=0.5cm of Dep] (Name) {name};
    \draw (Dep) -- (Name);

    % Relationship
    \node [gtu decision, double, between=Emp and Dep] (Rel) {Has};
    
    % Connections
    \draw (Emp) -- (Rel);
    \draw [double] (Rel) -- (Dep);

    % Person (for completeness if in goat, but standard example usually connects Employee-Dependent)
    % MDX goat shows Employee -- Dependent -- Person. 
    % Let's stick to standard Employee-Dependent weak entity representation.
    % The MDX goat has: Employee -- Dependent -- Person. 
    % Actually looking at goat: Employee -- Dependent -- Person. 
    % Usually Dependent is weak on Employee. Person might be another entity. 
    % I'll use the diagram I created for English which is likely Employee-Dependent.
\end{tikzpicture}
\captionof{figure}{વીક એન્ટિટી સેટ ઉદાહરણ}
\end{center}

\begin{center}
\captionof{table}{વીક વિ સ્ટ્રોંગ એન્ટિટી}
\begin{tabulary}{\linewidth}{|L|L|L|}
\hline
\textbf{પાસું} & \textbf{સ્ટ્રોંગ એન્ટિટી} & \textbf{વીક એન્ટિટી} \\ \hline
\textbf{પ્રાઇમરી કી} & પોતાની પ્રાઇમરી કી છે & કોઈ પ્રાઇમરી કી નથી \\ \hline
\textbf{અસ્તિત્વ} & સ્વતંત્ર અસ્તિત્વ & સ્ટ્રોંગ એન્ટિટી પર આધાર \\ \hline
\textbf{પ્રતિનિધિત્વ} & એક લંબચોરસ & ડબલ લંબચોરસ \\ \hline
\textbf{ઉદાહરણ} & કર્મચારી & કર્મચારીનો આશ્રિત \\ \hline
\end{tabulary}
\end{center}

\begin{itemize}
    \item \keyword{પાર્શિયલ કી}: એટ્રિબ્યુટ જે વીક એન્ટિટીને આંશિક રૂપે ઓળખે છે
    \item \keyword{આઇડેન્ટિફાઇંગ રિલેશનશિપ}: વીક એન્ટિટીને સ્ટ્રોંગ એન્ટિટી સાથે જોડે છે
    \item \keyword{ટોટલ પાર્ટિસિપેશન}: વીક એન્ટિટીએ સંબંધમાં સહભાગી થવું જ જોઈએ
\end{itemize}
\end{solutionbox}

\begin{mnemonicbox}
\mnemonic{વીક એન્ટિટીઓ આશ્રિત હોય છે}
\end{mnemonicbox}

\questionmarks{2(ક)}{7}{યુનિવર્સિટી મેનેજમેન્ટ સિસ્ટમ માટે ER ડાયાગ્રામ દોરો}

\begin{solutionbox}
\begin{center}
\begin{tikzpicture}[node distance=2.5cm, auto, thick]
    % Entities
    \node [gtu block] (Student) {Student};
    \node [gtu decision, right=2.5cm of Student] (Enroll) {Enrolls};
    \node [gtu block, right=2.5cm of Enroll] (Course) {Course};
    \node [gtu decision, below=2cm of Course] (Teaches) {Teaches};
    \node [gtu block, left=2.5cm of Teaches] (Teacher) {Teacher};

    % Attributes for Student
    \node [ellipse, draw, above=0.5cm of Student] (Sid) {\underline{student\_id}};
    \node [ellipse, draw, left=0.5cm of Student] (Sname) {name};
    \draw (Student) -- (Sid);
    \draw (Student) -- (Sname);

    % Attributes for Course
    \node [ellipse, draw, above=0.5cm of Course] (Cid) {\underline{course\_id}};
    \node [ellipse, draw, right=0.5cm of Course] (Cname) {name};
    \draw (Course) -- (Cid);
    \draw (Course) -- (Cname);

    % Attributes for Teacher
    \node [ellipse, draw, below=0.5cm of Teacher] (Tid) {\underline{teacher\_id}};
    \node [ellipse, draw, left=0.5cm of Teacher] (Tname) {name};
    \draw (Teacher) -- (Tid);
    \draw (Teacher) -- (Tname);
    
    % Connections
    \draw (Student) -- node[above] {M} (Enroll);
    \draw (Enroll) -- node[above] {N} (Course);
    \draw (Teacher) -- node[above] {1} (Teaches);
    \draw (Teaches) -- node[above] {N} (Course);
    
\end{tikzpicture}
\captionof{figure}{યુનિવર્સિટી ER ડાયાગ્રામ}
\end{center}

\begin{center}
\captionof{table}{એન્ટિટી રિલેશનશિપ્સ}
\begin{tabulary}{\linewidth}{|L|L|L|}
\hline
\textbf{રિલેશનશિપ} & \textbf{કાર્ડિનાલિટી} & \textbf{વર્ણન} \\ \hline
\textbf{વિદ્યાર્થી નોંધણી કરે કોર્સ} & M:N & અનેક વિદ્યાર્થીઓ અનેક કોર્સમાં નોંધણી કરી શકે \\ \hline
\textbf{શિક્ષક શીખવે કોર્સ} & 1:N & એક શિક્ષક અનેક કોર્સ શીખવે છે \\ \hline
\textbf{કોર્સ છે નોંધણી} & 1:N & એક કોર્સમાં અનેક નોંધણીઓ છે \\ \hline
\end{tabulary}
\end{center}

\begin{itemize}
    \item \keyword{પ્રાથમિક એન્ટિટીઓ}: વિદ્યાર્થી, કોર્સ, શિક્ષક
    \item \keyword{એસોસિએટિવ એન્ટિટી}: નોંધણી (M:N સંબંધ ઉકેલે છે)
    \item \keyword{કી એટ્રિબ્યુટ્સ}: બધી એન્ટિટીઓમાં અનન્ય ઓળખકર્તા છે
\end{itemize}
\end{solutionbox}

\begin{mnemonicbox}
\mnemonic{યુનિવર્સિટી = વિદ્યાર્થીઓ શિક્ષકો પાસેથી કોર્સ લે છે}
\end{mnemonicbox}

\questionmarks{2(અ) અથવા}{3}{નીચેના શબ્દોને વ્યાખ્યાયિત કરો: 1. પ્રાઇમરી કી 2. ફોરેન કી 3. કેન્ડિડેટ કી}

\begin{solutionbox}
\begin{center}
\captionof{table}{ડેટાબેસ કીઝ}
\begin{tabulary}{\linewidth}{|L|L|L|}
\hline
\textbf{કી પ્રકાર} & \textbf{વ્યાખ્યા} & \textbf{ઉદાહરણ} \\ \hline
\textbf{પ્રાઇમરી કી} & દરેક રેકોર્ડ માટે અનન્ય ઓળખકર્તા & વિદ્યાર્થી ટેબલમાં Student\_ID \\ \hline
\textbf{ફોરેન કી} & બીજા ટેબલની પ્રાઇમરી કીનો સંદર્ભ & નોંધણી ટેબલમાં Student\_ID \\ \hline
\textbf{કેન્ડિડેટ કી} & સંભવિત પ્રાઇમરી કી એટ્રિબ્યુટ & વિદ્યાર્થી ટેબલમાં Email, ફોન \\ \hline
\end{tabulary}
\end{center}

\begin{itemize}
    \item \keyword{પ્રાઇમરી કી}: NULL હોઈ શકે નહીં અને અનન્ય હોવી જોઈએ
    \item \keyword{ફોરેન કી}: રેફરન્શિયલ ઇન્ટેગ્રિટી જાળવે છે
    \item \keyword{કેન્ડિડેટ કી}: વૈકલ્પિક અનન્ય ઓળખકર્તાઓ
\end{itemize}
\end{solutionbox}

\begin{mnemonicbox}
\mnemonic{PFC - પ્રાઇમરી ફોરેન કેન્ડિડેટ}
\end{mnemonicbox}

\questionmarks{2(બ) અથવા}{4}{જનરલાઇઝેશન અને સ્પેશિયલાઇઝેશન પર ટૂંકી નોંધ લખો}

\begin{solutionbox}
\textbf{જનરલાઇઝેશન}: અનેક એન્ટિટીઓમાંથી સામાન્ય એટ્રિબ્યુટ્સ કાઢીને સામાન્ય એન્ટિટી બનાવવાની પ્રક્રિયા.

\textbf{સ્પેશિયલાઇઝેશન}: વિશિષ્ટ લાક્ષણિકતાઓના આધારે એન્ટિટીના પેટા વર્ગો વ્યાખ્યાયિત કરવાની પ્રક્રિયા.

\begin{center}
\begin{tikzpicture}[gtu tree, level 1/.style={sibling distance=4cm}, level 2/.style={sibling distance=2cm}]
    % Person (Top)
    \node [gtu root] {વ્યક્તિ (Person)} % Translated
        child {node [gtu child] {વિદ્યાર્થી (Student)} % Translated
            child {node [gtu block, fill=orange!10] {UG}} 
            child {node [gtu block, fill=orange!10] {PG}}
        }
        child {node [gtu child] {શિક્ષક (Teacher)}}
        child {node [gtu child] {સ્ટાફ (Staff)}};
\end{tikzpicture}
\captionof{figure}{જનરલાઇઝેશન અને સ્પેશિયલાઇઝેશન હાયરાર્કી}
\end{center}

\begin{center}
\captionof{table}{જનરલાઇઝેશન વિ સ્પેશિયલાઇઝેશન}
\begin{tabulary}{\linewidth}{|L|L|L|}
\hline
\textbf{પાસું} & \textbf{જનરલાઇઝેશન} & \textbf{સ્પેશિયલાઇઝેશન} \\ \hline
\textbf{દિશા} & બોટમ-અપ અપ્રોચ & ટોપ-ડાઉન અપ્રોચ \\ \hline
\textbf{હેતુ} & રિડન્ડન્સી દૂર કરવી & વિશિષ્ટ એટ્રિબ્યુટ્સ ઉમેરવા \\ \hline
\textbf{પરિણામ} & સુપરક્લાસ બનાવટ & સબક્લાસ બનાવટ \\ \hline
\end{tabulary}
\end{center}

\begin{itemize}
    \item \keyword{ISA રિલેશનશિપ}: સુપરક્લાસ અને સબક્લાસ વચ્ચે "Is-A" સંબંધ
    \item \keyword{ઇન્હેરિટન્સ}: સબક્લાસ સુપરક્લાસમાંથી એટ્રિબ્યુટ્સ વારસામાં લે છે
\end{itemize}
\end{solutionbox}

\begin{mnemonicbox}
\mnemonic{જનરલ ઉપર જાય, સ્પેશિયલ નીચે જાય}
\end{mnemonicbox}

\questionmarks{2(ક) અથવા}{7}{ઉદાહરણ સાથે વિવિધ રિલેશનલ એલ્જીબ્રા ઓપરેશન સમજાવો}

\begin{solutionbox}
\begin{center}
\captionof{table}{રિલેશનલ એલ્જીબ્રા ઓપરેશન્સ}
\begin{tabulary}{\linewidth}{|L|C|L|L|}
\hline
\textbf{ઓપરેશન} & \textbf{સિમ્બોલ} & \textbf{વર્ણન} & \textbf{ઉદાહરણ} \\ \hline
\textbf{સિલેક્ટ} & $\sigma$ & શરત આધારે પંક્તિઓ પસંદ કરે & $\sigma_{age>20}(Student)$ \\ \hline
\textbf{પ્રોજેક્ટ} & $\pi$ & વિશિષ્ટ કૉલમ્સ પસંદ કરે & $\pi_{name,age}(Student)$ \\ \hline
\textbf{યુનિયન} & $\cup$ & બે રિલેશન્સને જોડે & $R \cup S$ \\ \hline
\textbf{ઇન્ટરસેક્શન} & $\cap$ & રિલેશન્સમાંથી સામાન્ય ટ્યુપલ્સ & $R \cap S$ \\ \hline
\textbf{ડિફરન્સ} & $-$ & R માં છે પણ S માં નથી તે ટ્યુપલ્સ & $R - S$ \\ \hline
\textbf{જોઇન} & $\bowtie$ & સંબંધિત ટ્યુપલ્સને જોડે & $Student \bowtie Enroll$ \\ \hline
\end{tabulary}
\end{center}

\textbf{ઉદાહરણ રિલેશન્સ:}
Student: (ID=1, Name=જોન, Age=20)\\
Course: (CID=101, CName=DBMS, Credits=3)

\begin{itemize}
    \item \keyword{સિલેક્શન}: $\sigma_{Age>18}(Student)$ 18 વર્ષથી વધુ વયના વિદ્યાર્થીઓ રિટર્ન કરે
    \item \keyword{પ્રોજેક્શન}: $\pi_{Name}(Student)$ માત્ર નામો રિટર્ન કરે
    \item \keyword{જોઇન}: $Student \bowtie Enrollment$ વિદ્યાર્થી અને નોંધણીનો ડેટા જોડે
\end{itemize}
\end{solutionbox}

\begin{mnemonicbox}
\mnemonic{SPUDIJ - સિલેક્ટ પ્રોજેક્ટ યુનિયન ડિફરન્સ ઇન્ટરસેક્શન જોઇન}
\end{mnemonicbox}



\questionmarks{3(અ)}{3}{SQL માં ન્યુમેરિક ફંક્શન્સની યાદી આપો. કોઈપણ બે સમજાવો}

\begin{solutionbox}
\begin{center}
\captionof{table}{SQL ન્યુમેરિક ફંક્શન્સ}
\begin{tabulary}{\linewidth}{|L|L|L|}
\hline
\textbf{ફંક્શન} & \textbf{હેતુ} & \textbf{ઉદાહરણ} \\ \hline
\textbf{ABS()} & એબ્સોલ્યુટ વેલ્યુ & ABS(-15) = 15 \\ \hline
\textbf{CEIL()} & વેલ્યુ $\ge$ ની સૌથી નાની પૂર્ણાંક & CEIL(4.3) = 5 \\ \hline
\textbf{FLOOR()} & વેલ્યુ $\le$ ની સૌથી મોટી પૂર્ણાંક & FLOOR(4.7) = 4 \\ \hline
\textbf{ROUND()} & નિર્દિષ્ટ સ્થાને રાઉન્ડ કરે & ROUND(15.76, 1) = 15.8 \\ \hline
\textbf{SQRT()} & વર્ગમૂળ & SQRT(16) = 4 \\ \hline
\textbf{POWER()} & પાવર પર વધારો & POWER(2, 3) = 8 \\ \hline
\end{tabulary}
\end{center}

\textbf{વિગતવાર ઉદાહરણો:}
\begin{itemize}
    \item \code{ABS(number)}: એબ્સોલ્યુટ વેલ્યુ રિટર્ન કરે, નેગેટિવ સાઇન દૂર કરે
    \item \code{ROUND(number, decimal\_places)}: નિર્દિષ્ટ દશાંશ સ્થાને નંબર રાઉન્ડ કરે
\end{itemize}
\end{solutionbox}

\begin{mnemonicbox}
\mnemonic{ગણિત ફંક્શન્સ નંબર્સને સરસ બનાવે}
\end{mnemonicbox}

\questionmarks{3(બ)}{4}{ઉદાહરણ સાથે Having અને Order by Clause નું વર્ણન કરો}

\begin{solutionbox}
\textbf{HAVING Clause}: GROUP BY સાથે એગ્રીગેટ કન્ડિશન્સ આધારે ગ્રુપ્સ ફિલ્ટર કરવા ઉપયોગ થાય.

\textbf{ORDER BY Clause}: પરિણામ સેટને ચડતા અથવા ઊતરતા ક્રમમાં સોર્ટ કરવા ઉપયોગ થાય.

\begin{center}
\captionof{table}{HAVING વિ WHERE}
\begin{tabulary}{\linewidth}{|L|L|L|}
\hline
\textbf{પાસું} & \textbf{WHERE} & \textbf{HAVING} \\ \hline
\textbf{ઉપયોગ} & વ્યક્તિગત પંક્તિઓ ફિલ્ટર કરે & ગ્રુપ કરેલા પરિણામો ફિલ્ટર કરે \\ \hline
\textbf{એગ્રીગેટ્સ સાથે} & ઉપયોગ કરી શકાતો નથી & એગ્રીગેટ ફંક્શન્સ ઉપયોગ કરી શકે \\ \hline
\textbf{સ્થિતિ} & GROUP BY પહેલાં & GROUP BY પછી \\ \hline
\end{tabulary}
\end{center}

\textbf{ઉદાહરણ:}
\begin{lstlisting}[language=SQL]
SELECT department, COUNT(*) as emp_count
FROM employees 
WHERE salary > 30000
GROUP BY department 
HAVING COUNT(*) > 5
ORDER BY emp_count DESC;
\end{lstlisting}

\begin{itemize}
    \item \keyword{WHERE}: 30000 થી વધુ પગાર ધરાવતા કર્મચારીઓ ફિલ્ટર કરે
    \item \keyword{HAVING}: માત્ર 5 થી વધુ કર્મચારીઓ ધરાવતા વિભાગો બતાવે
    \item \keyword{ORDER BY}: કર્મચારીઓની ગણતરી આધારે ઉતરતા ક્રમમાં સોર્ટ કરે
\end{itemize}
\end{solutionbox}

\begin{mnemonicbox}
\mnemonic{WHERE પંક્તિઓ ફિલ્ટર કરે, HAVING ગ્રુપ્સ ફિલ્ટર કરે, ORDER BY પરિણામો સોર્ટ કરે}
\end{mnemonicbox}

\questionmarks{3(ક)}{7}{Student\_ID, Stu\_Name, Stu\_Subject\_ID, Stu\_Marks, Stu\_Age ફીલ્ડ્સ ધરાવતા student ટેબલ પર નીચેની queries perform કરો}

\begin{solutionbox}
\textbf{1. student ટેબલ બનાવવા માટે ક્વેરી:}
\begin{lstlisting}[language=SQL]
CREATE TABLE student (
    Student_ID INT PRIMARY KEY,
    Stu_Name VARCHAR(50),
    Stu_Subject_ID INT,
    Stu_Marks INT,
    Stu_Age INT
);
\end{lstlisting}

\textbf{2. student ટેબલમાં રેકોર્ડ દાખલ કરવા માટે ક્વેરી:}
\begin{lstlisting}[language=SQL]
INSERT INTO student VALUES 
(1, 'John', 101, 85, 22), -- ગુજરાતી નામો વાપરી શકાય પણ કોડમાં અંગ્રેજી શ્રેષ્ઠ
(2, 'Mary', 102, 90, 21);
\end{lstlisting}

\textbf{3. લઘુત્તમ અને મહત્તમ ગુણ શોધો:}
\begin{lstlisting}[language=SQL]
SELECT MIN(Stu_Marks) as Min_Marks, 
       MAX(Stu_Marks) as Max_Marks 
FROM student;
\end{lstlisting}

\textbf{4. 82 થી વધુ ગુણ અને 22 વર્ષ વયના વિદ્યાર્થીઓ:}
\begin{lstlisting}[language=SQL]
SELECT * FROM student 
WHERE Stu_Marks > 82 AND Stu_Age = 22;
\end{lstlisting}

\textbf{5. નામ 'm' અક્ષરથી શરૂ થતા વિદ્યાર્થીઓ:}
\begin{lstlisting}[language=SQL]
SELECT * FROM student 
WHERE Stu_Name LIKE 'm%';
\end{lstlisting}

\textbf{6. સરેરાશ ગુણ શોધો:}
\begin{lstlisting}[language=SQL]
SELECT AVG(Stu_Marks) as Average_Marks 
FROM student;
\end{lstlisting}

\textbf{7. Stu\_address કૉલમ ઉમેરો:}
\begin{lstlisting}[language=SQL]
ALTER TABLE student 
ADD Stu_address VARCHAR(100);
\end{lstlisting}
\end{solutionbox}

\begin{mnemonicbox}
\mnemonic{CRUD + એનાલિટિક્સ = સંપૂર્ણ ડેટાબેસ ઓપરેશન્સ}
\end{mnemonicbox}

\questionmarks{3(અ) અથવા}{3}{ઉદાહરણ સાથે SQL માં વિવિધ ડેટ ફંક્શન વર્ણવો}

\begin{solutionbox}
\begin{center}
\captionof{table}{SQL ડેટ ફંક્શન્સ}
\begin{tabulary}{\linewidth}{|L|L|L|}
\hline
\textbf{ફંક્શન} & \textbf{હેતુ} & \textbf{ઉદાહરણ} \\ \hline
\textbf{SYSDATE} & વર્તમાન સિસ્ટમ ડેટ & SYSDATE '2024-06-12' રિટર્ન કરે \\ \hline
\textbf{ADD\_MONTHS()} & ડેટમાં મહિનાઓ ઉમેરે & ADD\_MONTHS('2024-01-15', 3) \\ \hline
\textbf{MONTHS\_BETWEEN()} & ડેટ્સ વચ્ચેના મહિનાઓ & MONTHS\_BETWEEN('2024-06-12', '2024-01-12') \\ \hline
\textbf{LAST\_DAY()} & મહિનાનો છેલ્લો દિવસ & LAST\_DAY('2024-02-15') = '2024-02-29' \\ \hline
\textbf{NEXT\_DAY()} & દિવસની આગલી ઘટના & NEXT\_DAY('2024-06-12', 'FRIDAY') \\ \hline
\end{tabulary}
\end{center}

\begin{itemize}
    \item \code{SYSDATE}: વર્તમાન સિસ્ટમ ડેટ અને ટાઇમ રિટર્ન કરે
    \item \code{ADD\_MONTHS}: લોન ડ્યુ ડેટ્સ જેવી ભવિષ્યની તારીખો ગણવા માટે ઉપયોગી
\end{itemize}
\end{solutionbox}

\begin{mnemonicbox}
\mnemonic{ડેટ ફંક્શન્સ ટાઇમ મેનેજમેન્ટમાં મદદ કરે}
\end{mnemonicbox}

\questionmarks{3(બ) અથવા}{4}{SQL માં કન્સ્ટ્રેન્ટ્સની સૂચિ બનાવો. ઉદાહરણ સાથે કોઈપણ બે સમજાવો}

\begin{solutionbox}
\begin{center}
\captionof{table}{SQL કન્સ્ટ્રેન્ટ્સ}
\begin{tabulary}{\linewidth}{|L|L|L|}
\hline
\textbf{કન્સ્ટ્રેન્ટ} & \textbf{હેતુ} & \textbf{ઉદાહરણ} \\ \hline
\textbf{PRIMARY KEY} & અનન્ય ઓળખકર્તા & Student\_ID INT PRIMARY KEY \\ \hline
\textbf{FOREIGN KEY} & બીજા ટેબલનો સંદર્ભ & REFERENCES Student(Student\_ID) \\ \hline
\textbf{NOT NULL} & null વેલ્યુઝ અટકાવે & Name VARCHAR(50) NOT NULL \\ \hline
\textbf{UNIQUE} & અનન્યતા સુનિશ્ચિત કરે & Email VARCHAR(100) UNIQUE \\ \hline
\textbf{CHECK} & ડેટા વેલિડેટ કરે & Age INT CHECK (Age >= 18) \\ \hline
\textbf{DEFAULT} & ડિફોલ્ટ વેલ્યુ & Status VARCHAR(10) DEFAULT 'Active' \\ \hline
\end{tabulary}
\end{center}

\textbf{વિગતવાર ઉદાહરણો:}

\textbf{1. PRIMARY KEY કન્સ્ટ્રેન્ટ:}
\begin{lstlisting}[language=SQL]
CREATE TABLE Student (
    Student_ID INT PRIMARY KEY,
    Name VARCHAR(50)
);
\end{lstlisting}

\textbf{2. CHECK કન્સ્ટ્રેન્ટ:}
\begin{lstlisting}[language=SQL]
CREATE TABLE Employee (
    Emp_ID INT,
    Salary INT CHECK (Salary > 0)
);
\end{lstlisting}

\begin{itemize}
    \item \keyword{PRIMARY KEY}: દરેક રેકોર્ડ અનન્ય ઓળખકર્તા છે તેની ખાતરી કરે
    \item \keyword{CHECK}: ડેટા એન્ટ્રી દરમિયાન બિઝનેસ નિયમો વેલિડેટ કરે
\end{itemize}
\end{solutionbox}

\begin{mnemonicbox}
\mnemonic{કન્સ્ટ્રેન્ટ્સ ડેટા ક્વોલિટી કંટ્રોલ કરે}
\end{mnemonicbox}

\questionmarks{3(ક) અથવા}{7}{ઉદાહરણ સાથે SQL માં વિવિધ પ્રકારના joins સમજાવો}

\begin{solutionbox}
\begin{center}
\captionof{table}{SQL Joins ના પ્રકારો}
\begin{tabulary}{\linewidth}{|L|L|L|}
\hline
\textbf{Join પ્રકાર} & \textbf{વર્ણન} & \textbf{સિન્ટેક્સ} \\ \hline
\textbf{INNER JOIN} & બંને ટેબલમાંથી મેચિંગ રેકોર્ડ્સ રિટર્ન કરે & Table1 INNER JOIN Table2 ON condition \\ \hline
\textbf{LEFT JOIN} & ડાબા ટેબલના બધા + જમણાના મેચિંગ રેકોર્ડ્સ & Table1 LEFT JOIN Table2 ON condition \\ \hline
\textbf{RIGHT JOIN} & જમણા ટેબલના બધા + ડાબાના મેચિંગ રેકોર્ડ્સ & Table1 RIGHT JOIN Table2 ON condition \\ \hline
\textbf{FULL OUTER JOIN} & બંને ટેબલના બધા રેકોર્ડ્સ & Table1 FULL OUTER JOIN Table2 ON condition \\ \hline
\end{tabulary}
\end{center}

\textbf{ઉદાહરણ ટેબલ્સ:}
Students: (ID=1, Name=John), (ID=2, Name=Mary)
Enrollments: (StudentID=1, Course=DBMS), (StudentID=3, Course=Java)

\textbf{INNER JOIN ઉદાહરણ:}
\begin{lstlisting}[language=SQL]
SELECT s.Name, e.Course 
FROM Students s 
INNER JOIN Enrollments e ON s.ID = e.StudentID;
\end{lstlisting}
\textit{પરિણામ: માત્ર John DBMS કોર્સ સાથે}

\textbf{LEFT JOIN ઉદાહરણ:}
\begin{lstlisting}[language=SQL]
SELECT s.Name, e.Course 
FROM Students s 
LEFT JOIN Enrollments e ON s.ID = e.StudentID;
\end{lstlisting}
\textit{પરિણામ: John-DBMS, Mary-NULL}
\end{solutionbox}

\begin{mnemonicbox}
\mnemonic{JOIN સંબંધિત ટેબલ્સને જોડે છે}
\end{mnemonicbox}

\questionmarks{4(અ)}{3}{SQL માં Grant અને Revoke કમાન્ડનું ઉદાહરણ આપો}

\begin{solutionbox}
\textbf{GRANT કમાન્ડ}: ડેટાબેસ ઓબ્જેક્ટ્સ પર યુઝર્સને વિશિષ્ટ વિશેષાધિકારો પ્રદાન કરે.

\textbf{REVOKE કમાન્ડ}: યુઝર્સમાંથી અગાઉ આપેલા વિશેષાધિકારો દૂર કરે.

\begin{center}
\captionof{table}{સામાન્ય વિશેષાધિકારો}
\begin{tabulary}{\linewidth}{|L|L|L|}
\hline
\textbf{વિશેષાધિકાર} & \textbf{વર્ણન} & \textbf{ઉદાહરણ} \\ \hline
\textbf{SELECT} & ડેટા વાંચવો & GRANT SELECT ON Student TO user1 \\ \hline
\textbf{INSERT} & નવા રેકોર્ડ્સ ઉમેરવા & GRANT INSERT ON Student TO user1 \\ \hline
\textbf{UPDATE} & હાલના રેકોર્ડ્સ સુધારવા & GRANT UPDATE ON Student TO user1 \\ \hline
\textbf{DELETE} & રેકોર્ડ્સ દૂર કરવા & GRANT DELETE ON Student TO user1 \\ \hline
\textbf{ALL} & બધા વિશેષાધિકારો & GRANT ALL ON Student TO user1 \\ \hline
\end{tabulary}
\end{center}

\textbf{ઉદાહરણો:}
\begin{lstlisting}[language=SQL]
-- Grant SELECT privilege
GRANT SELECT ON Student TO john;

-- Revoke INSERT privilege
REVOKE INSERT ON Student FROM john;
\end{lstlisting}

\begin{itemize}
    \item \keyword{WITH GRANT OPTION}: યુઝરને બીજાઓને વિશેષાધિકારો આપવાની મંજૂરી
    \item \keyword{CASCADE}: જેમને આ વિશેષાધિકારો મળ્યા છે તે બધામાંથી વિશેષાધિકારો દૂર કરે
\end{itemize}
\end{solutionbox}

\begin{mnemonicbox}
\mnemonic{GRANT અધિકારો આપે, REVOKE અધિકારો દૂર કરે}
\end{mnemonicbox}

\questionmarks{4(બ)}{4}{SQL Views પર ટૂંકી નોંધ લખો}

\begin{solutionbox}
\textbf{વ્યાખ્યા}: વ્યૂ એ SQL સ્ટેટમેન્ટના પરિણામ આધારિત વર્ચ્યુઅલ ટેબલ છે જેમાં વાસ્તવિક ટેબલની જેમ પંક્તિઓ અને કૉલમ્સ હોય છે.

\begin{center}
\captionof{table}{વ્યૂની લાક્ષણિકતાઓ}
\begin{tabulary}{\linewidth}{|L|L|L|}
\hline
\textbf{પાસું} & \textbf{વર્ણન} & \textbf{ઉદાહરણ} \\ \hline
\textbf{વર્ચ્યુઅલ ટેબલ} & ફિઝિકલ રીતે ડેટા સ્ટોર કરતું નથી & CREATE VIEW student\_view AS... \\ \hline
\textbf{સિક્યોરિટી} & સંવેદનશીલ કૉલમ્સ છુપાવે & કર્મચારીઓમાંથી પગાર કૉલમ છુપાવો \\ \hline
\textbf{સરળીકરણ} & જટિલ ક્વેરીઝ સરળ બનાવે & એક વ્યૂમાં અનેક ટેબલ્સ જોડો \\ \hline
\textbf{ડેટા ઇન્ડિપેન્ડન્સ} & મૂળ ટેબલમાં ફેરફારો યુઝર્સને અસર કરતા નથી & એપ્લિકેશન્સને અસર કર્યા વિના ટેબલ સ્ટ્રક્ચર સુધારો \\ \hline
\end{tabulary}
\end{center}

\textbf{ઉદાહરણ:}
\begin{lstlisting}[language=SQL]
CREATE VIEW active_students AS
SELECT Student_ID, Name, Age
FROM Student
WHERE Status = 'Active';

-- Using the View
SELECT * FROM active_students;
\end{lstlisting}

\begin{itemize}
    \item \keyword{સિક્યોરિટી}: સંવેદનશીલ ડેટાની એક્સેસ મર્યાદિત કરે
    \item \keyword{સરળતા}: અંતિમ યુઝર્સમાંથી જટિલ joins છુપાવે
    \item \keyword{સુસંગતતા}: પ્રમાણિત ડેટા એક્સેસ
\end{itemize}
\end{solutionbox}

\begin{mnemonicbox}
\mnemonic{વ્યૂઝ એ ડેટાની વર્ચ્યુઅલ વિન્ડોઝ છે}
\end{mnemonicbox}

\questionmarks{4(ક)}{7}{નોર્મલાઇઝેશન શું છે? ઉદાહરણ સાથે 2NF સમજાવો}

\begin{solutionbox}
\textbf{નોર્મલાઇઝેશન}: રિડન્ડન્સી ઘટાડવા અને મોટા ટેબલ્સને નાના સંબંધિત ટેબલ્સમાં વિભાજિત કરીને ડેટા ઇન્ટેગ્રિટી સુધારવા માટે ડેટાબેસ ગોઠવવાની પ્રક્રિયા.

\textbf{2NF (સેકન્ડ નોર્મલ ફોર્મ)}:
\begin{itemize}
    \item 1NF માં હોવું જોઈએ
    \item પાર્શિયલ ફંક્શનલ ડિપેન્ડન્સીઝ દૂર કરવી
    \item નોન-કી એટ્રિબ્યુટ્સ સંપૂર્ણ પ્રાઇમરી કી પર આધાર રાખવા જોઈએ
\end{itemize}

\textbf{ઉદાહરણ - અનોર્મલાઇઝ્ડ ટેબલ:}
\begin{center}
\begin{tabulary}{\linewidth}{|L|L|L|L|L|}
\hline
\textbf{Student\_ID} & \textbf{Course\_ID} & \textbf{Student\_Name} & \textbf{Course\_Name} & \textbf{Instructor} \\ \hline
101 & C1 & John & DBMS & Dr. Smith \\ \hline
101 & C2 & John & Java & Dr. Jones \\ \hline
102 & C1 & Mary & DBMS & Dr. Smith \\ \hline
\end{tabulary}
\end{center}

\textbf{સમસ્યાઓ:}
\begin{itemize}
    \item Student\_Name માત્ર Student\_ID પર આધાર રાખે છે (પાર્શિયલ ડિપેન્ડન્સી)
    \item Course\_Name અને Instructor માત્ર Course\_ID પર આધાર રાખે છે
\end{itemize}

\textbf{2NF પછી:}

\textbf{Student ટેબલ:}
\begin{center}
\begin{tabulary}{\linewidth}{|L|L|}
\hline
\textbf{Student\_ID} & \textbf{Student\_Name} \\ \hline
101 & John \\ \hline
102 & Mary \\ \hline
\end{tabulary}
\end{center}

\textbf{Course ટેબલ:}
\begin{center}
\begin{tabulary}{\linewidth}{|L|L|L|}
\hline
\textbf{Course\_ID} & \textbf{Course\_Name} & \textbf{Instructor} \\ \hline
C1 & DBMS & Dr. Smith \\ \hline
C2 & Java & Dr. Jones \\ \hline
\end{tabulary}
\end{center}

\textbf{Enrollment ટેબલ:}
\begin{center}
\begin{tabulary}{\linewidth}{|L|L|}
\hline
\textbf{Student\_ID} & \textbf{Course\_ID} \\ \hline
101 & C1 \\ \hline
101 & C2 \\ \hline
102 & C1 \\ \hline
\end{tabulary}
\end{center}

\begin{itemize}
    \item \keyword{રિડન્ડન્સી દૂર કરે}: વિદ્યાર્થીના નામ પુનરાવર્તન નથી
    \item \keyword{સ્ટોરેજ ઘટાડે}: ઓછો ડુપ્લિકેટ ડેટા
    \item \keyword{સુસંગતતા સુધારે}: વિદ્યાર્થીનું નામ એક જ જગ્યાએ અપડેટ કરો
\end{itemize}
\end{solutionbox}

\begin{mnemonicbox}
\mnemonic{2NF = કોઈ પાર્શિયલ ડિપેન્ડન્સીઝ નહીં}
\end{mnemonicbox}

\questionmarks{4(અ) અથવા}{3}{SQL માં Group By Clause નું ઉદાહરણ આપો}

\begin{solutionbox}
\textbf{GROUP BY Clause}: નિર્દિષ્ટ કૉલમ્સમાં સમાન વેલ્યુઝ ધરાવતી પંક્તિઓને ગ્રુપ કરે છે અને દરેક ગ્રુપ પર એગ્રીગેટ ફંક્શન્સની મંજૂરી આપે છે.

\begin{center}
\captionof{table}{GROUP BY ઉપયોગ}
\begin{tabulary}{\linewidth}{|L|L|L|}
\hline
\textbf{હેતુ} & \textbf{ફંક્શન} & \textbf{ઉદાહરણ} \\ \hline
\textbf{ગણતરી} & COUNT() & વિભાગ દીઠ વિદ્યાર્થીઓની ગણતરી \\ \hline
\textbf{સરવાળો} & SUM() & વિભાગ દીઠ કુલ પગાર \\ \hline
\textbf{સરેરાશ} & AVG() & કોર્સ દીઠ સરેરાશ ગુણ \\ \hline
\textbf{મિન/મેક્સ શોધવું} & MIN()/MAX() & વિભાગ દીઠ સર્વોચ્ચ પગાર \\ \hline
\end{tabulary}
\end{center}

\textbf{ઉદાહરણ:}
\begin{lstlisting}[language=SQL]
SELECT Department, COUNT(*) as Total_Students, AVG(Marks) as Avg_Marks
FROM Student
GROUP BY Department;
\end{lstlisting}

\textbf{પરિણામ:}
\begin{center}
\begin{tabulary}{\linewidth}{|L|L|L|}
\hline
\textbf{Department} & \textbf{Total\_Students} & \textbf{Avg\_Marks} \\ \hline
IT & 25 & 78.5 \\ \hline
CS & 30 & 82.1 \\ \hline
\end{tabulary}
\end{center}

\begin{itemize}
    \item \keyword{ગ્રુપ્સ}: દરેક વિભાગ માટે અલગ ગ્રુપ બનાવે
    \item \keyword{એગ્રીગેટ્સ}: દરેક ગ્રુપ માટે કાઉન્ટ અને સરેરાશ ગણે
\end{itemize}
\end{solutionbox}

\begin{mnemonicbox}
\mnemonic{GROUP BY સમરી રિપોર્ટ્સ બનાવે}
\end{mnemonicbox}

\questionmarks{4(બ) અથવા}{4}{ઉદાહરણ સાથે SQL માં Set Operators નું વર્ણન કરો}

\begin{solutionbox}
\textbf{Set Operators}: બે અથવા વધુ SELECT સ્ટેટમેન્ટ્સના પરિણામોને જોડે છે.

\begin{center}
\captionof{table}{SQL Set Operators}
\begin{tabulary}{\linewidth}{|L|L|L|L|}
\hline
\textbf{ઓપરેટર} & \textbf{વર્ણન} & \textbf{આવશ્યકતા} & \textbf{ઉદાહરણ} \\ \hline
\textbf{UNION} & પરિણામો જોડે, ડુપ્લિકેટ્સ દૂર કરે & સમાન કૉલમ સ્ટ્રક્ચર & SELECT name FROM students UNION SELECT name FROM teachers \\ \hline
\textbf{UNION ALL} & પરિણામો જોડે, ડુપ્લિકેટ્સ રાખે & સમાન કૉલમ સ્ટ્રક્ચર & SELECT name FROM students UNION ALL SELECT name FROM alumni \\ \hline
\textbf{INTERSECT} & સામાન્ય રેકોર્ડ્સ રિટર્ન કરે & સમાન કૉલમ સ્ટ્રક્ચર & SELECT course FROM current\_courses INTERSECT SELECT course FROM popular \\ \hline
\textbf{MINUS} & પહેલી ક્વેરીમાં છે પણ બીજીમાં નથી & સમાન કૉલમ સ્ટ્રક્ચર & SELECT id FROM enrolled MINUS SELECT id FROM graduated \\ \hline
\end{tabulary}
\end{center}

\textbf{ઉદાહરણ:}
\begin{lstlisting}[language=SQL]
-- Students who are also Teachers
SELECT name FROM students
INTERSECT
SELECT name FROM teachers;

-- Everyone in University
SELECT name, 'Student' as type FROM students
UNION
SELECT name, 'Teacher' as type FROM teachers;
\end{lstlisting}

\begin{itemize}
    \item \keyword{કૉલમ કાઉન્ટ}: બધી ક્વેરીઝમાં સમાન હોવી જોઈએ
    \item \keyword{ડેટા ટાઇપ્સ}: અનુરૂપ કૉલમ્સમાં સુસંગત ટાઇપ્સ હોવા જોઈએ
\end{itemize}
\end{solutionbox}

\begin{mnemonicbox}
\mnemonic{Set operators ડેટાને યુનાઇટ, ઇન્ટરસેક્ટ અને સબ્ટ્રેક્ટ કરે}
\end{mnemonicbox}

\questionmarks{4(ક) અથવા}{7}{નોર્મલાઇઝેશનના મહત્વને ન્યાયી ઠેરવો. ઉદાહરણ સાથે 1NF સમજાવો}

\begin{solutionbox}
\textbf{નોર્મલાઇઝેશનનું મહત્વ:}
\begin{center}
\captionof{table}{નોર્મલાઇઝેશનના ફાયદાઓ}
\begin{tabulary}{\linewidth}{|L|L|L|}
\hline
\textbf{ફાયદો} & \textbf{વર્ણન} & \textbf{અસર} \\ \hline
\textbf{રિડન્ડન્સી દૂર કરે} & ડુપ્લિકેટ ડેટા સ્ટોરેજ ઘટાડે & સ્ટોરેજ સ્પેસ બચાવે \\ \hline
\textbf{એનોમલીઝ અટકાવે} & ઇન્સર્શન, ડિલીશન, અપડેટ સમસ્યાઓ ટાળે & ડેટા સુસંગતતા જાળવે \\ \hline
\textbf{ઇન્ટેગ્રિટી સુધારે} & ડેટાની સચોટતા સુનિશ્ચિત કરે & વિશ્વસનીય ઇન્ફોર્મેશન સિસ્ટમ \\ \hline
\textbf{લવચીક ડિઝાઇન} & સુધારવા અને વિસ્તારવામાં સરળ & બિઝનેસ ફેરફારોને અનુકૂળ \\ \hline
\end{tabulary}
\end{center}

\textbf{1NF (ફર્સ્ટ નોર્મલ ફોર્મ)}:
\begin{itemize}
    \item સમાન ટેબલમાંથી ડુપ્લિકેટ કૉલમ્સ દૂર કરો
    \item સંબંધિત ડેટા માટે અલગ ટેબલ્સ બનાવો
    \item દરેક સેલમાં એક વેલ્યુ હોય (એટોમિક વેલ્યુઝ)
\end{itemize}

\textbf{ઉદાહરણ - અનોર્મલાઇઝ્ડ ટેબલ:}
\begin{center}
\begin{tabulary}{\linewidth}{|L|L|L|}
\hline
\textbf{Student\_ID} & \textbf{Name} & \textbf{Subjects} \\ \hline
101 & John & Maths, Science, English \\ \hline
102 & Mary & Science, History \\ \hline
\end{tabulary}
\end{center}

\textbf{સમસ્યાઓ:} Subjects કૉલમમાં અનેક વેલ્યુઝ છે.

\textbf{1NF પછી:}

\textbf{Student ટેબલ:}
\begin{center}
\begin{tabulary}{\linewidth}{|L|L|}
\hline
\textbf{Student\_ID} & \textbf{Name} \\ \hline
101 & John \\ \hline
102 & Mary \\ \hline
\end{tabulary}
\end{center}

\textbf{Student\_Subject ટેબલ:}
\begin{center}
\begin{tabulary}{\linewidth}{|L|L|}
\hline
\textbf{Student\_ID} & \textbf{Subject} \\ \hline
101 & Maths \\ \hline
101 & Science \\ \hline
101 & English \\ \hline
102 & Science \\ \hline
102 & History \\ \hline
\end{tabulary}
\end{center}

\begin{itemize}
    \item \keyword{એટોમિક વેલ્યુઝ}: દરેક સેલમાં એક વેલ્યુ
    \item \keyword{લવચીક ક્વેરીઝ}: વિશિષ્ટ વિષયો અભ્યાસ કરતા વિદ્યાર્થીઓ સરળતાથી શોધો
    \item \keyword{સરળ અપડેટ્સ}: બીજા ડેટાને અસર કર્યા વિના વિષયો ઉમેરો/દૂર કરો
\end{itemize}
\end{solutionbox}

\begin{mnemonicbox}
\mnemonic{1NF = એક સેલ દીઠ એક વેલ્યુ, કોઈ રિપીટિંગ ગ્રુપ્સ નહીં}
\end{mnemonicbox}


\questionmarks{5(અ)}{3}{ટ્રાન્ઝેક્શન મેનેજમેન્ટમાં Serializability સમજાવો}

\begin{solutionbox}
\textbf{Serializability}: એ ગુણધર્મ છે જે સુનિશ્ચિત કરે છે કે ટ્રાન્ઝેક્શન્સનું એકસાથે એક્ઝિક્યુશન તે ટ્રાન્ઝેક્શન્સના કોઈ સીરિયલ એક્ઝિક્યુશન જેવું જ પરિણામ આપે.

\begin{center}
\captionof{table}{Serializability ના પ્રકારો}
\begin{tabulary}{\linewidth}{|L|L|L|}
\hline
\textbf{પ્રકાર} & \textbf{વર્ણન} & \textbf{પદ્ધતિ} \\ \hline
\textbf{Conflict Serializability} & કોન્ફ્લિક્ટિંગ ઓપરેશન્સ આધારિત & પ્રિસિડન્સ ગ્રાફ \\ \hline
\textbf{View Serializability} & રીડ-રાઇટ પેટર્ન આધારિત & વ્યૂ ઇક્વિવેલન્સ \\ \hline
\end{tabulary}
\end{center}

\textbf{ઉદાહરણ:}
Transaction T1: R(A), W(A), R(B), W(B) \\
Transaction T2: R(A), W(A), R(B), W(B)

\begin{itemize}
    \item \keyword{સીરિયલ શેડ્યુલ}: T1 $\rightarrow$ T2 અથવા T2 $\rightarrow$ T1
    \item \keyword{કોન્ફ્લિક્ટ ઓપરેશન્સ}: સમાન ડેટા આઇટમ પરના ઓપરેશન્સ જ્યાં ઓછામાં ઓછું એક રાઇટ હોય
    \item \keyword{સીરિયલાઇઝેબલ શેડ્યુલ}: કોઈ સીરિયલ શેડ્યુલ સમકક્ષ
\end{itemize}
\end{solutionbox}

\begin{mnemonicbox}
\mnemonic{Serializability ટ્રાન્ઝેક્શન કન્સિસ્ટન્સી સુનિશ્ચિત કરે}
\end{mnemonicbox}

\questionmarks{5(બ)}{4}{ઉદાહરણ સાથે પાર્શિયલ ફંક્શનલ ડિપેન્ડન્સી નું વર્ણન કરો}

\begin{solutionbox}
\textbf{પાર્શિયલ ફંક્શનલ ડિપેન્ડન્સી}: જ્યારે કોઈ નોન-કી એટ્રિબ્યુટ કમ્પોઝિટ પ્રાઇમરી કીના માત્ર એક ભાગ પર ફંક્શનલી ડિપેન્ડન્ટ હોય.

\begin{center}
\captionof{table}{ફંક્શનલ ડિપેન્ડન્સીના પ્રકારો}
\begin{tabulary}{\linewidth}{|L|L|L|}
\hline
\textbf{પ્રકાર} & \textbf{વ્યાખ્યા} & \textbf{ઉદાહરણ} \\ \hline
\textbf{ફુલ ડિપેન્ડન્સી} & સંપૂર્ણ પ્રાઇમરી કી પર આધાર & (Student\_ID, Course\_ID) $\rightarrow$ Grade \\ \hline
\textbf{પાર્શિયલ ડિપેન્ડન્સી} & પ્રાઇમરી કીના ભાગ પર આધાર & (Student\_ID, Course\_ID) $\rightarrow$ Student\_Name \\ \hline
\end{tabulary}
\end{center}

\textbf{ઉદાહરણ - Enrollment ટેબલ:}
\begin{center}
\begin{tabulary}{\linewidth}{|L|L|L|L|L|}
\hline
\textbf{Student\_ID} & \textbf{Course\_ID} & \textbf{Student\_Name} & \textbf{Course\_Name} & \textbf{Grade} \\ \hline
101 & C1 & John & DBMS & A \\ \hline
101 & C2 & John & Java & B \\ \hline
\end{tabulary}
\end{center}

\textbf{પાર્શિયલ ડિપેન્ડન્સીઝ:}
\begin{itemize}
    \item Student\_ID $\rightarrow$ Student\_Name (Student\_Name માત્ર Student\_ID પર આધાર રાખે)
    \item Course\_ID $\rightarrow$ Course\_Name (Course\_Name માત્ર Course\_ID પર આધાર રાખે)
\end{itemize}

\textbf{નોર્મલાઇઝેશન}: પાર્શિયલ ડિપેન્ડન્સીઝ દૂર કરીને 2NF માં ફેરવો.
\end{solutionbox}

\begin{mnemonicbox}
\mnemonic{પાર્શિયલ ડિપેન્ડન્સી = કીનો ભાગ એટ્રિબ્યુટ નક્કી કરે}
\end{mnemonicbox}

\questionmarks{5(ક)}{7}{ટ્રાન્ઝેક્શન મેનેજમેન્ટમાં ઉદાહરણ સાથે Locking Mechanism પર ટૂંકી નોંધ લખો}

\begin{solutionbox}
\textbf{Locking Mechanism}: કન્કરન્સી કંટ્રોલ ટેકનીક જે ટ્રાન્ઝેક્શન એક્ઝિક્યુશન દરમિયાન ડેટા આઇટમ્સની એકસાથે એક્સેસ અટકાવે છે.

\begin{center}
\captionof{table}{Locks ના પ્રકારો}
\begin{tabulary}{\linewidth}{|L|L|L|}
\hline
\textbf{Lock પ્રકાર} & \textbf{વર્ણન} & \textbf{ઉપયોગ} \\ \hline
\textbf{Shared Lock (S)} & અનેક ટ્રાન્ઝેક્શન્સ વાંચી શકે & રીડ ઓપરેશન્સ \\ \hline
\textbf{Exclusive Lock (X)} & માત્ર એક ટ્રાન્ઝેક્શન એક્સેસ કરી શકે & રાઇટ ઓપરેશન્સ \\ \hline
\textbf{Intention Lock} & નિચલા લેવલે lock કરવાનો ઇરાદો દર્શાવે & હાયરાર્કિકલ લોકિંગ \\ \hline
\end{tabulary}
\end{center}

\textbf{Two-Phase Locking (2PL) પ્રોટોકોલ:}
\begin{enumerate}
    \item \textbf{ગ્રોઇંગ ફેઝ}: locks એક્વાયર કરો, કોઈ lock રિલીઝ ન કરો
    \item \textbf{શ્રિંકિંગ ફેઝ}: locks રિલીઝ કરો, નવા locks એક્વાયર ન કરો
\end{enumerate}

\textbf{Lock Compatibility Matrix:}
\begin{center}
\begin{tabulary}{\linewidth}{|L|C|C|}
\hline
\textbf{વર્તમાન/માંગેલ} & \textbf{S} & \textbf{X} \\ \hline
\textbf{S} & \checkmark & \xmark \\ \hline
\textbf{X} & \xmark & \xmark \\ \hline
\end{tabulary}
\end{center}

\begin{itemize}
    \item \keyword{ડેડલોક}: બે ટ્રાન્ઝેક્શન્સ એકબીજાના locks માટે રાહ જુએ
    \item \keyword{સ્ટાર્વેશન}: ટ્રાન્ઝેક્શન lock માટે અનંત રાહ જુએ
\end{itemize}
\end{solutionbox}

\begin{mnemonicbox}
\mnemonic{Locking કોન્કરન્ટ કોન્ફ્લિક્ટ્સ અટકાવે}
\end{mnemonicbox}

\questionmarks{5(અ) અથવા}{3}{ટ્રાન્ઝેક્શન મેનેજમેન્ટમાં ડેડલોક સમજાવો}

\begin{solutionbox}
\textbf{ડેડલોક}: એવી પરિસ્થિતિ જ્યાં બે અથવા વધુ ટ્રાન્ઝેક્શન્સ એકબીજાને locks રિલીઝ કરવા માટે અનંત રાહ જુએ છે, ચક્રાકાર રાહની સ્થિતિ બનાવે છે.

\begin{center}
\captionof{table}{ડેડલોકના ઘટકો}
\begin{tabulary}{\linewidth}{|L|L|L|}
\hline
\textbf{ઘટક} & \textbf{વર્ણન} & \textbf{ઉદાહરણ} \\ \hline
\textbf{મ્યુચ્યુઅલ એક્સક્લુઝન} & રિસોર્સ શેર કરી શકાતા નથી & એક્સક્લુઝિવ locks \\ \hline
\textbf{હોલ્ડ એન્ડ વેઇટ} & પ્રોસેસ રિસોર્સ પકડીને બીજાની રાહ જુએ & T1 A પકડે, B ની રાહ જુએ \\ \hline
\textbf{નો પ્રીએમ્પ્શન} & રિસોર્સ બળજબરીથી છીનવી શકાતા નથી & Locks રદ કરી શકાતા નથી \\ \hline
\textbf{સર્ક્યુલર વેઇટ} & પ્રોસેસોની ચક્રાકાર રાહની સાંકળ & T1 $\rightarrow$ T2 $\rightarrow$ T1 \\ \hline
\end{tabulary}
\end{center}

\textbf{ઉદાહરણ:}
\begin{lstlisting}
Transaction T1: Lock(A), Lock(B)
Transaction T2: Lock(B), Lock(A)

1: T1 gets Lock(A)
2: T2 gets Lock(B)
3: T1 waits for Lock(B) (Held by T2)
4: T2 waits for Lock(A) (Held by T1)
Result: DEADLOCK
\end{lstlisting}

\begin{itemize}
    \item \keyword{ડિટેક્શન}: ચક્રો ઓળખવા માટે wait-for ગ્રાફનો ઉપયોગ
    \item \keyword{પ્રિવેન્શન}: ટાઇમસ્ટેમ્પ ઓર્ડરિંગ અથવા wound-wait પ્રોટોકોલ્સનો ઉપયોગ
\end{itemize}
\end{solutionbox}

\begin{mnemonicbox}
\mnemonic{ડેડલોક = રિસોર્સ માટે ચક્રાકાર રાહ}
\end{mnemonicbox}

\questionmarks{5(બ) અથવા}{4}{ઉદાહરણ સાથે ફુલ ફંક્શનલ ડિપેન્ડન્સી નું વર્ણન કરો}

\begin{solutionbox}
\textbf{ફુલ ફંક્શનલ ડિપેન્ડન્સી}: જ્યારે કોઈ નોન-કી એટ્રિબ્યુટ સંપૂર્ણ પ્રાઇમરી કી પર ફંક્શનલી ડિપેન્ડન્ટ હોય (માત્ર તેના ભાગ પર નહીં).

\begin{center}
\captionof{table}{ડિપેન્ડન્સી તુલના}
\begin{tabulary}{\linewidth}{|L|L|L|}
\hline
\textbf{પ્રકાર} & \textbf{વ્યાખ્યા} & \textbf{ઉદાહરણ} \\ \hline
\textbf{ફુલ ડિપેન્ડન્સી} & સંપૂર્ણ પ્રાઇમરી કી પર આધાર & (Student\_ID, Course\_ID) $\rightarrow$ Grade \\ \hline
\textbf{પાર્શિયલ ડિપેન્ડન્સી} & પ્રાઇમરી કીના ભાગ પર આધાર & (Student\_ID, Course\_ID) $\rightarrow$ Student\_Name \\ \hline
\end{tabulary}
\end{center}

\textbf{ઉદાહરણ - Enrollment ટેબલ:}
\begin{center}
\begin{tabulary}{\linewidth}{|L|L|L|L|}
\hline
\textbf{Student\_ID} & \textbf{Course\_ID} & \textbf{Grade} & \textbf{Hours} \\ \hline
101 & C1 & A & 4 \\ \hline
101 & C2 & B & 3 \\ \hline
\end{tabulary}
\end{center}

\textbf{ફુલ ફંક્શનલ ડિપેન્ડન્સીઝ:}
\begin{itemize}
    \item (Student\_ID, Course\_ID) $\rightarrow$ Grade \checkmark
    \item (Student\_ID, Course\_ID) $\rightarrow$ Hours \checkmark
\end{itemize}

\textbf{સમજૂતી:}
\begin{itemize}
    \item \keyword{Grade} Student\_ID અને Course\_ID બંને પર આધાર રાખે (વિશિષ્ટ વિદ્યાર્થી વિશિષ્ટ કોર્સમાં)
    \item માત્ર Student\_ID અથવા માત્ર Course\_ID થી Grade નક્કી કરી શકાતો નથી
\end{itemize}
\end{solutionbox}

\begin{mnemonicbox}
\mnemonic{ફુલ ડિપેન્ડન્સીને સંપૂર્ણ કીની જરૂર}
\end{mnemonicbox}

\questionmarks{5(ક) અથવા}{7}{ઉદાહરણ સાથે ટ્રાન્ઝેક્શનના ACID ગુણધર્મો સમજાવો}

\begin{solutionbox}
\textbf{ACID ગુણધર્મો}: ડેટાબેસ ટ્રાન્ઝેક્શનની વિશ્વસનીયતાની બાંયધરી આપતા ચાર મૂળભૂત ગુણધર્મો.

\begin{center}
\captionof{table}{ACID ગુણધર્મો}
\begin{tabulary}{\linewidth}{|L|L|L|}
\hline
\textbf{ગુણધર્મ} & \textbf{વર્ણન} & \textbf{ઉદાહરણ} \\ \hline
\textbf{Atomicity} & બધું અથવા કશું નહીં એક્ઝિક્યુશન & બેંક ટ્રાન્સફર દ.ખ: ડેબિટ અને ક્રેડિટ બંને થવા જોઈએ \\ \hline
\textbf{Consistency} & ડેટાબેસ વેલિડ સ્ટેટમાં રહે & એકાઉન્ટ બેલેન્સ નેગેટિવ ન હોઈ શકે \\ \hline
\textbf{Isolation} & ટ્રાન્ઝેક્શન્સ એકબીજામાં દખલ ન કરે & કોન્કરન્ટ ટ્રાન્ઝેક્શન્સ સીક્વન્શિયલ લાગે \\ \hline
\textbf{Durability} & કમિટ થયેલા ફેરફારો કાયમી રહે & સિસ્ટમ ક્રેશ પછી પણ ડેટા બચે \\ \hline
\end{tabulary}
\end{center}

\textbf{વિગતવાર ઉદાહરણો:}

\textbf{Atomicity:}
\begin{lstlisting}[language=SQL]
BEGIN TRANSACTION;
UPDATE Account SET Balance = Balance - 1000 WHERE AccNo = 'A001';
UPDATE Account SET Balance = Balance + 1000 WHERE AccNo = 'A002';
COMMIT;
\end{lstlisting}
\textit{જો કોઈ પણ અપડેટ નિષ્ફળ જાય તો સંપૂર્ણ ટ્રાન્ઝેક્શન રોલબેક થાય}

\textbf{Consistency:} સિસ્ટમમાં કુલ પૈસા ટ્રાન્સફર પહેલા અને પછી સમાન રહે.

\textbf{Isolation:}
\begin{lstlisting}
T1: Read(A=100), A=A+50, Write(A=150)
T2: Read(A=100), A=A*2, Write(A=200)
Serialized result: A=300 or A=250
\end{lstlisting}

\textbf{Durability:} COMMIT એક્ઝિક્યુટ થયા પછી, સિસ્ટમ ક્રેશ થયા છતાં, ટ્રાન્સફર થયેલ રકમ સુરક્ષિત રહે.

\begin{itemize}
    \item \keyword{Atomicity}: ટ્રાન્ઝેક્શન લોગ દ.ખ
    \item \keyword{Consistency}: કન્સ્ટ્રેન્ટ્સ
    \item \keyword{Isolation}: લોકિંગ
    \item \keyword{Durability}: રાઇટ-અહેડ લોગિંગ
\end{itemize}
\end{solutionbox}

\begin{mnemonicbox}
\mnemonic{ACID ટ્રાન્ઝેક્શન્સને વિશ્વસનીય રાખે}
\end{mnemonicbox}

\end{document}
