\documentclass[10pt,a4paper]{article}

% content/resources/templates/preamble.tex
\usepackage[margin=0.6in]{geometry}
\author{Milav Dabgar}
\usepackage{amsmath,amssymb,amsthm}
\usepackage{booktabs}
\usepackage{multirow}
\usepackage{xcolor}
\usepackage{tcolorbox}
\tcbuselibrary{breakable,skins}
\usepackage[colorlinks=true,linkcolor=blue]{hyperref}
\usepackage{titlesec}
\usepackage{enumitem}
\usepackage{tikz}
\usepackage{pgfplots}
\usepackage{circuitikz}
\usepackage[version=4]{mhchem}
\usepackage{longtable}
\usepackage{array}
\usepackage{float}
\usepackage{caption}
\usepackage{listings}

\lstset{
  basicstyle=\small\ttfamily,
  breaklines=true,
  breakatwhitespace=false,
  postbreak=\mbox{\textcolor{red}{$\hookrightarrow$}\space},
  float=false,
  numbers=left,
  numberstyle=\tiny\color{gray},
  numbersep=10pt,
  xleftmargin=2em,
  keywordstyle=\color{blue},
  commentstyle=\color{green!60!black},
  stringstyle=\color{purple},
  backgroundcolor=\color{gray!5},
  showstringspaces=false,
  tabsize=2,
  captionpos=b,
  keepspaces=true,
  columns=flexible
}

\pgfplotsset{compat=1.18}
\usetikzlibrary{shapes,arrows,positioning,calc,patterns,decorations.pathmorphing,decorations.markings,arrows.meta}

% Color scheme
\definecolor{headcolor}{RGB}{0,102,204}
\definecolor{keycolor}{RGB}{220,20,60}
\definecolor{solutioncolor}{RGB}{34,139,34}
\definecolor{mnemoniccolor}{RGB}{148,0,211}
\definecolor{codecolor}{RGB}{0,0,100}

% Spacing
\setlength{\parskip}{3pt}
\setlist[itemize]{nosep}
\setlist[enumerate]{nosep}

% Title formatting
\titleformat{\section}{\Large\bfseries\color{headcolor}}{\thesection}{1em}{}
\titleformat{\subsection}{\large\bfseries\color{headcolor}}{\thesubsection}{1em}{}

% Pandoc tightlist compatibility
\providecommand{\tightlist}{%
  \setlength{\itemsep}{0pt}\setlength{\parskip}{0pt}}

% Pandoc longtable compatibility
\newcounter{none}
\def\thenone{}


% content/resources/templates/gujarati-boxes.tex
\usepackage{fontspec}
\usepackage{polyglossia}

% Set Gujarati as main language (document is primarily in Gujarati)
% Note: gloss-gujarati.ldf doesn't exist in polyglossia, but it will use hyphenation patterns
\setdefaultlanguage{gujarati}
\setotherlanguage{english}

% Configure Gujarati font properly
% Use Language=Default to prevent polyglossia from trying to add language-specific features
% that don't exist for Gujarati, which causes "empty feature" warnings
\newfontfamily\gujaratifont[Script=Gujarati,AutoFakeBold=2.5,AutoFakeSlant=0.3]{Noto Sans Gujarati}
\setmainfont[Script=Gujarati,AutoFakeBold=2.5,AutoFakeSlant=0.3]{Noto Sans Gujarati}
% Use Noto Sans Gujarati for monospace to support Gujarati in text
\setmonofont[Scale=0.9]{Noto Sans Gujarati}

% Configure English to use the same font
\newfontfamily\englishfont[Script=Gujarati,AutoFakeBold=2.5,AutoFakeSlant=0.3]{Noto Sans Gujarati}

% Translations for polyglossia
\gappto\captionsgujarati{
  \renewcommand{\tablename}{કોષ્ટક}
  \renewcommand{\figurename}{આકૃતિ}
}

% Helper for TikZ nodes to ensure Gujarati font
\newcommand{\gu}[1]{{\gujaratifont #1}}

% Custom environments
\newtcolorbox{solutionbox}{
    breakable,
    enhanced,
    colback=solutioncolor!5!white,
    colframe=solutioncolor!75!black,
    fonttitle=\bfseries,
    title=જવાબ
}

\newtcolorbox{solutionboxnobreak}{
 colback=solutioncolor!5!white,
 colframe=solutioncolor!75!black,
 fonttitle=\bfseries,
 title=જવાબ
}

\newtcolorbox{keyformula}{
 breakable,
 enhanced,
 colback=keycolor!5!white,
 colframe=keycolor!75!black,
 fonttitle=\bfseries,
 title=રાસાયણિક સમીકરણ/સૂત્ર
}

\newtcolorbox{mnemonicbox}{
 breakable,
 enhanced,
 colback=mnemoniccolor!5!white,
 colframe=mnemoniccolor!75!black,
 fonttitle=\bfseries,
 title=મેમરી ટ્રીક
}


\begin{document}

\begin{center}
{\Huge\bfseries\color{headcolor} Subject Name (Gujarati)}\\[5pt]
{\LARGE 4331603 -- Summer 2025}\\[3pt]
{\large Semester 1 Study Material}\\[3pt]
{\normalsize\textit{Detailed Solutions and Explanations}}
\end{center}

\vspace{10pt}

\subsection*{પ્રશ્ન 1(a) [3
ગુણ]}\label{q1a}

\textbf{નીચેના શબ્દોની વ્યાખ્યા આપો. 1) મેટાડેટા 2) સ્કીમા 3) ડેટા ડિક્શનરી.}

\begin{solutionbox}

\textbf{ટેબલ:}

{\def\LTcaptype{none} % do not increment counter
\begin{longtable}[]{@{}ll@{}}
\toprule\noalign{}
શબ્દ & વ્યાખ્યા \\
\midrule\noalign{}
\endhead
\bottomrule\noalign{}
\endlastfoot
\textbf{મેટાડેટા} & ડેટા વિશેનો ડેટા જે ડેટાબેઝની રચના અને વિશેષતાઓ વર્ણવે છે \\
\textbf{સ્કીમા} & ડેટાબેઝના સંગઠન અને સંબંધોને દર્શાવતી તાર્કિક રચના \\
\textbf{ડેટા ડિક્શનરી} & ડેટાબેઝના તત્વો વિશેની માહિતી સંગ્રહિત કરતો કેન્દ્રીય
ભંડાર \\
\end{longtable}
}

\begin{itemize}
\tightlist
\item
  \textbf{મેટાડેટા}: ડેટાની લાક્ષણિકતાઓ અને ગુણધર્મો વર્ણવતી માહિતી
\item
  \textbf{સ્કીમા}: ડેટાબેઝની રચના અને મર્યાદાઓ વ્યાખ્યાયિત કરતો બ્લુપ્રિન્ટ
\item
  \textbf{ડેટા ડિક્શનરી}: બધા ડેટાબેઝ ઓબ્જેક્ટ્સ અને તેમના ગુણધર્મોની કેટલોગ
\end{itemize}

\end{solutionbox}
\begin{mnemonicbox}
``MSD - My System Dictionary''

\end{mnemonicbox}
\subsection*{પ્રશ્ન 1(b) [4
ગુણ]}\label{q1b}

\textbf{ડેટાબેઝ મેનેજમેન્ટ સિસ્ટમના ફાયદા લખો.}

\begin{solutionbox}

\textbf{ટેબલ:}

{\def\LTcaptype{none} % do not increment counter
\begin{longtable}[]{@{}ll@{}}
\toprule\noalign{}
ફાયદો & વર્ણન \\
\midrule\noalign{}
\endhead
\bottomrule\noalign{}
\endlastfoot
\textbf{ડેટા સ્વતંત્રતા} & એપ્લિકેશન્સ ડેટા સ્ટોરેજથી સ્વતંત્ર \\
\textbf{ડેટા અખંડિતતા} & ડેટાની ચોકસાઈ અને સુસંગતતા જાળવે છે \\
\textbf{સુરક્ષા નિયંત્રણ} & વપરાશકર્તા પ્રમાણીકરણ અને અધિકરણ \\
\textbf{સમવર્તી પહોંચ} & અનેક વપરાશકર્તાઓ એકસાથે પહોંચ કરી શકે છે \\
\end{longtable}
}

\begin{itemize}
\tightlist
\item
  \textbf{ઘટેલી રીડન્ડન્સી}: ડુપ્લિકેટ ડેટા સ્ટોરેજ દૂર કરે છે
\item
  \textbf{કેન્દ્રીકૃત નિયંત્રણ}: ડેટા મેનેજમેન્ટનું એક જ બિંદુ
\item
  \textbf{ડેટા વહેંચણી}: અનેક એપ્લિકેશન્સ સમાન ડેટાનો ઉપયોગ કરી શકે છે
\item
  \textbf{બેકઅપ પુનઃપ્રાપ્તિ}: આપોઆપ ડેટા સુરક્ષા પદ્ધતિઓ
\end{itemize}

\end{solutionbox}
\begin{mnemonicbox}
``DISC-RCDB - Database Is Super Cool''

\end{mnemonicbox}
\subsection*{પ્રશ્ન 1(c) [7
ગુણ]}\label{q1c}

\textbf{DBA ની જવાબદારીઓ સમજાવો.}

\begin{solutionbox}

\textbf{ટેબલ:}

{\def\LTcaptype{none} % do not increment counter
\begin{longtable}[]{@{}ll@{}}
\toprule\noalign{}
જવાબદારી & કાર્યો \\
\midrule\noalign{}
\endhead
\bottomrule\noalign{}
\endlastfoot
\textbf{ડેટાબેઝ ડિઝાઈન} & તાર્કિક અને ભૌતિક રચનાઓ બનાવવી \\
\textbf{સુરક્ષા મેનેજમેન્ટ} & વપરાશકર્તા પહોંચ અને પરવાનગીઓનું નિયંત્રણ \\
\textbf{પર્ફોર્મન્સ ટ્યુનિંગ} & ક્વેરીઝ અને ડેટાબેઝ ઓપરેશન્સને ઓપ્ટિમાઈઝ કરવા \\
\textbf{બેકઅપ પુનઃપ્રાપ્તિ} & ડેટા સુરક્ષા અને પુનઃસ્થાપન સુનિશ્ચિત કરવું \\
\textbf{યુઝર મેનેજમેન્ટ} & એકાઉન્ટ બનાવવા અને વિશેષાધિકારો અસાઇન કરવા \\
\end{longtable}
}

\begin{center}
\textbf{Mermaid Diagram (Code)}
\begin{verbatim}
{Shaded}
{Highlighting}[]
graph TD
    A[DBA જવાબદારીઓ] {-{-}{} B[ડેટાબેઝ ડિઝાઈન]}
    A {-{-}{} C[સુરક્ષા મેનેજમેન્ટ]}
    A {-{-}{} D[પર્ફોર્મન્સ ટ્યુનિંગ]}
    A {-{-}{} E[બેકઅપ અને પુનઃપ્રાપ્તિ]}
    A {-{-}{} F[યુઝર મેનેજમેન્ટ]}
    A {-{-}{} G[સિસ્ટમ મોનિટરિંગ]}
{Highlighting}
{Shaded}
\end{verbatim}
\end{center}

\begin{itemize}
\tightlist
\item
  \textbf{ડેટાબેઝ ઇન્સ્ટલેશન}: DBMS સોફ્ટવેર સેટઅપ અને કોન્ફિગર કરવું
\item
  \textbf{ડેટા માઇગ્રેશન}: સિસ્ટમ્સ વચ્ચે ડેટાને સુરક્ષિત રીતે ટ્રાન્સફર કરવો
\item
  \textbf{ડોક્યુમેન્ટેશન}: ડેટાબેઝ સ્કીમા અને પ્રક્રિયાઓ જાળવવી
\item
  \textbf{મોનિટરિંગ}: સિસ્ટમ પર્ફોર્મન્સ અને રિસોર્સ વપરાશ ટ્રેક કરવું
\item
  \textbf{ટ્રબલશૂટિંગ}: ડેટાબેઝ સમસ્યાઓ અને ભૂલો ઉકેલવી
\end{itemize}

\end{solutionbox}
\begin{mnemonicbox}
``DSPBU-DMT - DBA Solves Problems By Understanding
Database Management Tasks''

\end{mnemonicbox}
\subsection*{પ્રશ્ન 1(c OR) [7
ગુણ]}\label{uxaaauxab0uxab6uxaa8-1c-or-7-uxa97uxaa3}

\textbf{ડેટા એબ્સ્ટ્રેક્શન શું છે? ત્રણ સ્તરની ANSI SPARC આર્કિટેક્ચરને વિગતવાર
સમજાવો.}

\begin{solutionbox}

\textbf{ડેટા એબ્સ્ટ્રેક્શન}: વપરાશકર્તાઓથી જટિલ ડેટાબેઝ અમલીકરણ વિગતો છુપાવીને સરળ
ઇન્ટરફેસ પ્રદાન કરવું.

\begin{center}
\textbf{Mermaid Diagram (Code)}
\begin{verbatim}
{Shaded}
{Highlighting}[]
graph LR
    A[એક્સટર્નલ લેવલ] {-{-}{} B[કોન્સેપ્ચુઅલ લેવલ]}
    B {-{-}{} C[ઇન્ટર્નલ લેવલ]}
    A1[યુઝર વ્યૂઝ] {-{-}{} A}
    B1[લોજિકલ સ્કીમા] {-{-}{} B}
    C1[ફિઝિકલ સ્ટોરેજ] {-{-}{} C}
{Highlighting}
{Shaded}
\end{verbatim}
\end{center}

\textbf{ટેબલ:}

{\def\LTcaptype{none} % do not increment counter
\begin{longtable}[]{@{}lll@{}}
\toprule\noalign{}
સ્તર & વર્ણન & વપરાશકર્તાઓ \\
\midrule\noalign{}
\endhead
\bottomrule\noalign{}
\endlastfoot
\textbf{એક્સટર્નલ લેવલ} & વ્યક્તિગત વપરાશકર્તા દૃશ્યો અને એપ્લિકેશન્સ & એન્ડ યુઝર્સ \\
\textbf{કોન્સેપ્ચુઅલ લેવલ} & સંપૂર્ણ તાર્કિક ડેટાબેઝ રચના & ડેટાબેઝ ડિઝાઇનર્સ \\
\textbf{ઇન્ટર્નલ લેવલ} & ભૌતિક સ્ટોરેજ અને એક્સેસ પદ્ધતિઓ & સિસ્ટમ પ્રોગ્રામર્સ \\
\end{longtable}
}

\begin{itemize}
\tightlist
\item
  \textbf{એક્સટર્નલ લેવલ}: જટિલતા છુપાવતા અનેક વપરાશકર્તા દૃશ્યો
\item
  \textbf{કોન્સેપ્ચુઅલ લેવલ}: સ્ટોરેજ વિગતો વિના સંપૂર્ણ ડેટાબેઝ સ્કીમા
\item
  \textbf{ઇન્ટર્નલ લેવલ}: ભૌતિક ફાઇલ સંગઠન અને ઇન્ડેક્સિંગ
\item
  \textbf{ડેટા સ્વતંત્રતા}: એક સ્તરમાં ફેરફારો અન્યને અસર કરતા નથી
\end{itemize}

\end{solutionbox}
\begin{mnemonicbox}
``ECI - Every Computer Implements''

\end{mnemonicbox}
\subsection*{પ્રશ્ન 2(a) [3
ગુણ]}\label{q2a}

\textbf{સ્કીમા અને ઇન્સ્ટન્સનો તફાવત સમજાવો}

\begin{solutionbox}

\textbf{ટેબલ:}

{\def\LTcaptype{none} % do not increment counter
\begin{longtable}[]{@{}lll@{}}
\toprule\noalign{}
પાસું & સ્કીમા & ઇન્સ્ટન્સ \\
\midrule\noalign{}
\endhead
\bottomrule\noalign{}
\endlastfoot
\textbf{વ્યાખ્યા} & ડેટાબેઝ રચનાનો બ્લુપ્રિન્ટ & ચોક્કસ સમયે વાસ્તવિક ડેટા \\
\textbf{પ્રકૃતિ} & સ્થિર તાર્કિક ડિઝાઇન & ડાયનામિક ડેટા સામગ્રી \\
\textbf{ફેરફારો} & ભાગ્યે જ સંશોધિત & વારંવાર અપડેટ \\
\end{longtable}
}

\begin{itemize}
\tightlist
\item
  \textbf{સ્કીમા}: ડેટાબેઝ સંગઠન અને મર્યાદાઓ વર્ણવે છે
\item
  \textbf{ઇન્સ્ટન્સ}: ચોક્કસ ક્ષણે ડેટાબેઝ સામગ્રીનો સ્નેપશોટ
\item
  \textbf{સંબંધ}: સ્કીમા રચના વ્યાખ્યાયિત કરે છે, ઇન્સ્ટન્સ ડેટા સમાવે છે
\end{itemize}

\end{solutionbox}
\begin{mnemonicbox}
``SI - Structure vs Information''

\end{mnemonicbox}
\subsection*{પ્રશ્ન 2(b) [4
ગુણ]}\label{q2b}

\textbf{સ્પેશ્યલાઈઝેશન ઉદાહરણ સાથે સમજાવો.}

\begin{solutionbox}

\textbf{સ્પેશ્યલાઈઝેશન}: ચોક્કસ લાક્ષણિકતાઓના આધારે સુપરક્લાસમાંથી સબક્લાસ બનાવવાની
પ્રક્રિયા.

\begin{verbatim}
erDiagram
    EMPLOYEE \{
        int emp\_id
        string name
        float salary
    \}
    MANAGER \{
        string department
        int team\_size
    \}
    DEVELOPER \{
        string programming\_language
        string project
    \}
    
    EMPLOYEE ||{-{-}|| MANAGER : specializes}
    EMPLOYEE ||{-{-}|| DEVELOPER : specializes}
\end{verbatim}

\begin{itemize}
\tightlist
\item
  \textbf{ટોપ-ડાઉન અપ્રોચ}: સામાન્ય એન્ટિટીથી ચોક્કસ એન્ટિટીઓ તરફ
\item
  \textbf{ઇન્હેરિટન્સ}: સબક્લાસેસ સુપરક્લાસના ગુણધર્મો વારસામાં લે છે
\item
  \textbf{ડિસજોઇન્ટ}: મેનેજર અને ડેવલપર અલગ કેટેગરી છે
\item
  \textbf{ઉદાહરણ}: એમ્પ્લોયી મેનેજર અને ડેવલપરમાં વિશેષીકૃત
\end{itemize}

\end{solutionbox}
\begin{mnemonicbox}
``STID - Specialization Takes Inheritance Down''

\end{mnemonicbox}
\subsection*{પ્રશ્ન 2(c) [7
ગુણ]}\label{q2c}

\textbf{ER ડાયાગ્રામ શું છે? ER ડાયાગ્રામમાં વપરાતા વિવિધ પ્રતીકોને ઉદાહરણ સાથે
સમજાવો.}

\begin{solutionbox}

\textbf{ER ડાયાગ્રામ}: ડેટાબેઝ ડિઝાઇનમાં એન્ટિટીઝ, એટ્રિબ્યુટ્સ અને સંબંધો દર્શાવતી
ગ્રાફિકલ પ્રતિનિધિત્વ.

\textbf{ટેબલ:}

{\def\LTcaptype{none} % do not increment counter
\begin{longtable}[]{@{}llll@{}}
\toprule\noalign{}
પ્રતીક & આકાર & હેતુ & ઉદાહરણ \\
\midrule\noalign{}
\endhead
\bottomrule\noalign{}
\endlastfoot
\textbf{એન્ટિટી} & લંબચોરસ & વાસ્તવિક વિશ્વનો ઓબ્જેક્ટ & Student, Course \\
\textbf{એટ્રિબ્યુટ} & અંડાકાર & એન્ટિટીના ગુણધર્મો & Name, Age, ID \\
\textbf{સંબંધ} & હીરો & એન્ટિટી કનેક્શન્સ & Enrolls, Takes \\
\textbf{પ્રાઇમરી કી} & અન્ડરલાઇન અંડાકાર & યુનિક આઇડેન્ટિફાયર & Student\_ID \\
\end{longtable}
}

\begin{verbatim}
erDiagram
    STUDENT \{
        int student\_id PK
        string name
        string email
        date birth\_date
    \}
    COURSE \{
        string course\_id PK
        string course\_name
        int credits
    \}
    ENROLLMENT \{
        date enrollment\_date
        string grade
    \}
    
    STUDENT ||{-{-}o\{ ENROLLMENT : enrolls}
    COURSE ||{-{-}o\{ ENROLLMENT : includes}
\end{verbatim}

\begin{itemize}
\tightlist
\item
  \textbf{એન્ટિટી સેટ્સ}: સમાન ગુણધર્મો ધરાવતી સમાન એન્ટિટીઝનો સંગ્રહ
\item
  \textbf{વીક એન્ટિટી}: ઓળખ માટે સ્ટ્રોંગ એન્ટિટી પર આધારિત
\item
  \textbf{કાર્ડિનાલિટી}: સંબંધ સહભાગિતા વ્યાખ્યાયિત કરે છે (1:1, 1:M, M:N)
\item
  \textbf{પાર્ટિસિપેશન}: ટોટલ (ડબલ લાઇન) અથવા પાર્શિયલ (સિંગલ લાઇન)
\end{itemize}

\end{solutionbox}
\begin{mnemonicbox}
``EARP - Entities And Relationships Program''

\end{mnemonicbox}
\subsection*{પ્રશ્ન 2(a OR) [3
ગુણ]}\label{uxaaauxab0uxab6uxaa8-2a-or-3-uxa97uxaa3}

\textbf{DA અને DBA નો તફાવત સમજાવો.}

\begin{solutionbox}

\textbf{ટેબલ:}

{\def\LTcaptype{none} % do not increment counter
\begin{longtable}[]{@{}lll@{}}
\toprule\noalign{}
પાસું & ડેટા એડમિનિસ્ટ્રેટર (DA) & ડેટાબેઝ એડમિનિસ્ટ્રેટર (DBA) \\
\midrule\noalign{}
\endhead
\bottomrule\noalign{}
\endlastfoot
\textbf{ફોકસ} & ડેટા પોલિસીઝ અને સ્ટાન્ડર્ડ્સ & તકનીકી ડેટાબેઝ ઓપરેશન્સ \\
\textbf{સ્તર} & વ્યૂહાત્મક આયોજન & ઓપરેશનલ અમલીકરણ \\
\textbf{સ્કોપ} & સંસ્થા-વ્યાપી ડેટા & ચોક્કસ ડેટાબેઝ સિસ્ટમ્સ \\
\end{longtable}
}

\begin{itemize}
\tightlist
\item
  \textbf{DA}: સંસ્થાકીય સંસાધન તરીકે ડેટાનું સંચાલન કરે છે
\item
  \textbf{DBA}: તકનીકી ડેટાબેઝ જાળવણી અને પર્ફોર્મન્સ સંભાળે છે
\item
  \textbf{સહયોગ}: DA નીતિઓ સેટ કરે છે, DBA તેમને અમલમાં મૂકે છે
\end{itemize}

\end{solutionbox}
\begin{mnemonicbox}
``DA-DBA: Design Authority - Database Builder
Administrator''

\end{mnemonicbox}
\subsection*{પ્રશ્ન 2(b OR) [4
ગુણ]}\label{uxaaauxab0uxab6uxaa8-2b-or-4-uxa97uxaa3}

\textbf{જનરલાઈઝેશન ઉદાહરણ સાથે સમજાવો.}

\begin{solutionbox}

\textbf{જનરલાઈઝેશન}: સમાન એન્ટિટીઝને સામાન્ય સુપરક્લાસમાં જોડવાની બોટમ-અપ
પ્રક્રિયા.

\begin{verbatim}
erDiagram
    VEHICLE \{
        string vehicle\_id
        string brand
        int year
        string color
    \}
    CAR \{
        int doors
        string fuel\_type
    \}
    MOTORCYCLE \{
        int engine\_cc
        string bike\_type
    \}
    
    VEHICLE ||{-{-}|| CAR : generalizes}
    VEHICLE ||{-{-}|| MOTORCYCLE : generalizes}
\end{verbatim}

\begin{itemize}
\tightlist
\item
  \textbf{બોટમ-અપ અપ્રોચ}: ચોક્કસ એન્ટિટીઝથી સામાન્ય એન્ટિટી તરફ
\item
  \textbf{કોમન એટ્રિબ્યુટ્સ}: સહેજ ગુણધર્મો સુપરક્લાસમાં ખસેડાય છે
\item
  \textbf{સ્પેશ્યલાઈઝેશન રિવર્સ}: સ્પેશ્યલાઈઝેશન પ્રક્રિયાનું વિપરીત
\item
  \textbf{ઉદાહરણ}: કાર અને મોટરસાઇકલ વાહનમાં સામાન્યીકૃત
\end{itemize}

\end{solutionbox}
\begin{mnemonicbox}
``GBCS - Generalization Brings Common Superclass''

\end{mnemonicbox}
\subsection*{પ્રશ્ન 2(c OR) [7
ગુણ]}\label{uxaaauxab0uxab6uxaa8-2c-or-7-uxa97uxaa3}

\textbf{એટ્રિબ્યુટ શું છે? વિવિધ પ્રકારના એટ્રિબ્યુટ્સ ઉદાહરણ સાથે સમજાવો.}

\begin{solutionbox}

\textbf{એટ્રિબ્યુટ}: એન્ટિટીનું વર્ણન કરતી ગુણવત્તા અથવા લાક્ષણિકતા.

\textbf{ટેબલ:}

{\def\LTcaptype{none} % do not increment counter
\begin{longtable}[]{@{}lll@{}}
\toprule\noalign{}
એટ્રિબ્યુટ પ્રકાર & વર્ણન & ઉદાહરણ \\
\midrule\noalign{}
\endhead
\bottomrule\noalign{}
\endlastfoot
\textbf{સિમ્પલ} & વધુ વિભાજિત કરી શકાતું નથી & Age, Name \\
\textbf{કોમ્પોઝિટ} & ઉપવિભાગ કરી શકાય છે & Address (Street, City, ZIP) \\
\textbf{સિંગલ-વેલ્યુડ} & એન્ટિટી દીઠ એક મૂલ્ય & Student\_ID \\
\textbf{મલ્ટિ-વેલ્યુડ} & અનેક મૂલ્યો શક્ય & Phone\_numbers \\
\textbf{ડેરાઇવ્ડ} & અન્ય એટ્રિબ્યુટ્સમાંથી ગણાય છે & Age from Birth\_date \\
\end{longtable}
}

\begin{center}
\textbf{Mermaid Diagram (Code)}
\begin{verbatim}
{Shaded}
{Highlighting}[]
graph TD
    A[એટ્રિબ્યુટ્સ] {-{-}{} B[સિમ્પલ]}
    A {-{-}{} C[કોમ્પોઝિટ]}
    A {-{-}{} D[સિંગલ{-}વેલ્યુડ]}
    A {-{-}{} E[મલ્ટિ{-}વેલ્યુડ]}
    A {-{-}{} F[ડેરાઇવ્ડ]}
    A {-{-}{} G[કી એટ્રિબ્યુટ્સ]}
    
    C {-{-}{} C1[Address: Street, City, ZIP]}
    E {-{-}{} E1[Phone: Mobile, Home, Work]}
    F {-{-}{} F1[Age calculated from DOB]}
{Highlighting}
{Shaded}
\end{verbatim}
\end{center}

\begin{itemize}
\tightlist
\item
  \textbf{કી એટ્રિબ્યુટ}: એન્ટિટી ઇન્સ્ટન્સેસને યુનિકલી ઓળખે છે
\item
  \textbf{નલ વેલ્યુઝ}: એટ્રિબ્યુટ્સ કે જેમાં કોઈ મૂલ્ય ન હોઈ શકે
\item
  \textbf{ડિફોલ્ટ વેલ્યુઝ}: નિર્દિષ્ટ ન હોય ત્યારે પૂર્વનિર્ધારિત મૂલ્યો
\item
  \textbf{કન્સ્ટ્રેઇન્ટ્સ}: એટ્રિબ્યુટ મૂલ્યોને સંચાલિત કરતા નિયમો
\end{itemize}

\end{solutionbox}
\begin{mnemonicbox}
``SCSMD-K - Simple Composite Single Multi Derived
Key''

\end{mnemonicbox}
\subsection*{પ્રશ્ન 3(a) [3
ગુણ]}\label{q3a}

\textbf{SQL માં GRANT અને REVOKE સ્ટેટમેન્ટ સમજાવો.}

\begin{solutionbox}

\textbf{ટેબલ:}

{\def\LTcaptype{none} % do not increment counter
\begin{longtable}[]{@{}
  >{\raggedright\arraybackslash}p{(\linewidth - 4\tabcolsep) * \real{0.3333}}
  >{\raggedright\arraybackslash}p{(\linewidth - 4\tabcolsep) * \real{0.1515}}
  >{\raggedright\arraybackslash}p{(\linewidth - 4\tabcolsep) * \real{0.5152}}@{}}
\toprule\noalign{}
\begin{minipage}[b]{\linewidth}\raggedright
સ્ટેટમેન્ટ
\end{minipage} & \begin{minipage}[b]{\linewidth}\raggedright
હેતુ
\end{minipage} & \begin{minipage}[b]{\linewidth}\raggedright
સિન્ટેક્સ ઉદાહરણ
\end{minipage} \\
\midrule\noalign{}
\endhead
\bottomrule\noalign{}
\endlastfoot
\textbf{GRANT} & વપરાશકર્તાઓને વિશેષાધિકારો પ્રદાન કરે છે & GRANT SELECT ON
table TO user \\
\textbf{REVOKE} & વપરાશકર્તા વિશેષાધિકારો દૂર કરે છે & REVOKE INSERT ON table
FROM user \\
\end{longtable}
}

\begin{verbatim}
{-{-} Grant privileges}
GRANT SELECT, INSERT ON employees TO john;
GRANT ALL PRIVILEGES ON database TO admin;

{-{-} Revoke privileges  }
REVOKE DELETE ON employees FROM john;
REVOKE ALL ON database FROM user;
\end{verbatim}

\begin{itemize}
\tightlist
\item
  \textbf{વિશેષાધિકારો}: SELECT, INSERT, UPDATE, DELETE, ALL
\item
  \textbf{ઓબ્જેક્ટ્સ}: ટેબલ્સ, વ્યૂઝ, ડેટાબેઝિસ, પ્રોસીજર્સ
\item
  \textbf{સુરક્ષા}: ડેટા એક્સેસ અને મોડિફિકેશન રાઇટ્સનું નિયંત્રણ
\end{itemize}

\end{solutionbox}
\begin{mnemonicbox}
``GR - Grant Rights, Remove Rights''

\end{mnemonicbox}
\subsection*{પ્રશ્ન 3(b) [4
ગુણ]}\label{q3b}

\textbf{નીચેના Character functions સમજાવો. 1) INITCAP 2) SUBSTR}

\begin{solutionbox}

\textbf{ટેબલ:}

{\def\LTcaptype{none} % do not increment counter
\begin{longtable}[]{@{}
  >{\raggedright\arraybackslash}p{(\linewidth - 6\tabcolsep) * \real{0.2581}}
  >{\raggedright\arraybackslash}p{(\linewidth - 6\tabcolsep) * \real{0.1613}}
  >{\raggedright\arraybackslash}p{(\linewidth - 6\tabcolsep) * \real{0.2903}}
  >{\raggedright\arraybackslash}p{(\linewidth - 6\tabcolsep) * \real{0.2903}}@{}}
\toprule\noalign{}
\begin{minipage}[b]{\linewidth}\raggedright
ફંક્શન
\end{minipage} & \begin{minipage}[b]{\linewidth}\raggedright
હેતુ
\end{minipage} & \begin{minipage}[b]{\linewidth}\raggedright
સિન્ટેક્સ
\end{minipage} & \begin{minipage}[b]{\linewidth}\raggedright
ઉદાહરણ
\end{minipage} \\
\midrule\noalign{}
\endhead
\bottomrule\noalign{}
\endlastfoot
\textbf{INITCAP} & દરેક શબ્દનો પહેલો અક્ષર મોટો કરે છે & INITCAP(string) &
INITCAP(`hello world') = `Hello World' \\
\textbf{SUBSTR} & સ્ટ્રિંગમાંથી સબસ્ટ્રિંગ કાઢે છે & SUBSTR(string, start,
length) & SUBSTR(`Database', 1, 4) = `Data' \\
\end{longtable}
}

\begin{verbatim}
{-{-} INITCAP examples}
SELECT INITCAP({database management}) FROM dual; {-{-} Database Management}
SELECT INITCAP({gtu university}) FROM dual; {-{-} Gtu University}

{-{-} SUBSTR examples  }
SELECT SUBSTR({Programming}, 1, 7) FROM dual; {-{-} Program}
SELECT SUBSTR({Database}, 5) FROM dual;

{-{-} base}
\end{verbatim}

\begin{itemize}
\tightlist
\item
  \textbf{INITCAP}: સ્ટ્રિંગને યોગ્ય કેસ ફોર્મેટમાં કન્વર્ટ કરે છે
\item
  \textbf{SUBSTR}: પેરામીટર્સ છે સ્ટ્રિંગ, શરૂઆતની સ્થિતિ, વૈકલ્પિક લંબાઈ
\item
  \textbf{વપરાશ}: ટેક્સ્ટ ફોર્મેટિંગ અને સ્ટ્રિંગ મેનિપ્યુલેશન ઓપરેશન્સ
\end{itemize}

\end{solutionbox}
\begin{mnemonicbox}
``IS - Initialize String, Split String''

\end{mnemonicbox}
\subsection*{પ્રશ્ન 3(c) [7
ગુણ]}\label{q3c}

\textbf{નીચે દશાવેલ ટેબલને ધ્યાનમાં લઈ આપેલ ક્વેરીઝના જવાબ લખો.}
\textbf{stud\_master (enroll\_no, name, city, dept)}

\begin{solutionbox}

\begin{verbatim}
{-{-} 1. IT dept માં અભ્યાસ કરતા બધા વિદ્યાર્થીઓની વિગતો દર્શાવો}
SELECT * FROM stud\_master 
WHERE dept = {IT};

{-{-} 2. p થી શરૂ થતા નામ વિશેની બધી માહિતી મેળવો}
SELECT * FROM stud\_master 
WHERE name LIKE {p\%};

{-{-} 3. ટેબલમાં નવો વિદ્યાર્થી દાખલ કરો}
INSERT INTO stud\_master (enroll\_no, name, city, dept) 
VALUES ({202501}, {John Smith}, {Mumbai}, {CS});

{-{-} 4. stud\_master ટેબલમાં gender નામનું નવું કૉલમ ઉમેરો}
ALTER TABLE stud\_master 
ADD gender VARCHAR(10);

{-{-} 5. stud\_master ટેબલની પંક્તિઓની સંખ્યા ગણો}
SELECT COUNT(*) FROM stud\_master;

{-{-} 6. enroll\_no ના અવરોહી ક્રમમાં બધી વિદ્યાર્થી વિગતો દર્શાવો}
SELECT * FROM stud\_master 
ORDER BY enroll\_no DESC;

{-{-} 7. ડેટા સાથે stud\_master ટેબલનો નાશ કરો}
DROP TABLE stud\_master;
\end{verbatim}

\textbf{ટેબલ:}

{\def\LTcaptype{none} % do not increment counter
\begin{longtable}[]{@{}lll@{}}
\toprule\noalign{}
ક્વેરી પ્રકાર & SQL કમાન્ડ & હેતુ \\
\midrule\noalign{}
\endhead
\bottomrule\noalign{}
\endlastfoot
\textbf{SELECT} & ડેટા મેળવે છે & રેકોર્ડ્સ દર્શાવે છે \\
\textbf{INSERT} & નવો ડેટા ઉમેરે છે & રેકોર્ડ્સ બનાવે છે \\
\textbf{ALTER} & રચના સંશોધિત કરે છે & કૉલમ્સ ઉમેરે છે \\
\textbf{COUNT} & એગ્રિગેટ ફંક્શન & પંક્તિઓ ગણે છે \\
\end{longtable}
}

\end{solutionbox}
\begin{mnemonicbox}
``SIAC-DOC - SQL Is A Complete Database Operations
Collection''

\end{mnemonicbox}
\subsection*{પ્રશ્ન 3(a OR) [3
ગુણ]}\label{uxaaauxab0uxab6uxaa8-3a-or-3-uxa97uxaa3}

\textbf{SQL માં equi join ઉદાહરણ સાથે સમજાવો.}

\begin{solutionbox}

\textbf{Equi Join}: સમાન કૉલમ્સના આધારે ટેબલ્સને જોડવા માટે સમાનતા શરતનો ઉપયોગ
કરતું જોઇન ઓપરેશન.

\begin{verbatim}
{-{-} Equi Join ઉદાહરણ}
SELECT s.name, c.course\_name
FROM students s, courses c
WHERE s.course\_id = c.course\_id;

{-{-} JOIN સિન્ટેક્સનો ઉપયોગ}
SELECT s.name, c.course\_name  
FROM students s
JOIN courses c ON s.course\_id = c.course\_id;
\end{verbatim}

\begin{itemize}
\tightlist
\item
  \textbf{સમાનતા ઓપરેટર}: કૉલમ મૂલ્યો મેચ કરવા માટે = નો ઉપયોગ
\item
  \textbf{કોમન કૉલમ્સ}: ટેબલ્સમાં સંબંધિત એટ્રિબ્યુટ્સ હોવા જોઈએ
\item
  \textbf{પરિણામ}: મેચના આધારે અનેક ટેબલ્સમાંથી સંયુક્ત ડેટા
\end{itemize}

\end{solutionbox}
\begin{mnemonicbox}
``EJ - Equal Join''

\end{mnemonicbox}
\subsection*{પ્રશ્ન 3(b OR) [4
ગુણ]}\label{uxaaauxab0uxab6uxaa8-3b-or-4-uxa97uxaa3}

\textbf{નીચેના Aggregate functions સમજાવો. 1) MAX 2) SUM}

\begin{solutionbox}

\textbf{ટેબલ:}

{\def\LTcaptype{none} % do not increment counter
\begin{longtable}[]{@{}llll@{}}
\toprule\noalign{}
ફંક્શન & હેતુ & સિન્ટેક્સ & ઉદાહરણ \\
\midrule\noalign{}
\endhead
\bottomrule\noalign{}
\endlastfoot
\textbf{MAX} & મહત્તમ મૂલ્ય પરત કરે છે & MAX(column) & MAX(salary) = 50000 \\
\textbf{SUM} & મૂલ્યોનો કુલ સરવાળો પરત કરે છે & SUM(column) & SUM(marks) =
450 \\
\end{longtable}
}

\begin{verbatim}
{-{-} MAX ઉદાહરણો}
SELECT MAX(salary) FROM employees; {-{-} સૌથી વધુ પગાર}
SELECT MAX(age) FROM students; {-{-} સૌથી જૂના વિદ્યાર્થીની ઉંમર}

{-{-} SUM ઉદાહરણો}
SELECT SUM(credits) FROM courses; {-{-} કુલ ક્રેડિટ્સ}
SELECT SUM(price * quantity) FROM orders; {-{-} કુલ ઓર્ડર મૂલ્ય}
\end{verbatim}

\begin{itemize}
\tightlist
\item
  \textbf{એગ્રિગેટ ફંક્શન્સ}: અનેક પંક્તિઓ પર કામ કરે છે, એક મૂલ્ય પરત કરે છે
\item
  \textbf{NULL હેન્ડલિંગ}: ગણતરીમાં NULL મૂલ્યોને અવગણે છે
\item
  \textbf{GROUP BY}: કેટેગરી-વાઇઝ પરિણામો માટે ગ્રુપિંગ સાથે ઉપયોગ કરી શકાય છે
\end{itemize}

\end{solutionbox}
\begin{mnemonicbox}
``MS - Maximum Sum''

\end{mnemonicbox}
\subsection*{પ્રશ્ન 3(c OR) [7
ગુણ]}\label{uxaaauxab0uxab6uxaa8-3c-or-7-uxa97uxaa3}

\textbf{નીચે દશાવેલ ટેબલ માટે SQL ક્વેરીઝ લખો:} \textbf{PRODUCT\_Master:
(prod\_no, prod\_name, profit, quantity, sell\_price, cost\_price)}

\begin{solutionbox}

\begin{verbatim}
{-{-} 1. PRODUCT\_Master ટેબલ બનાવો}
CREATE TABLE PRODUCT\_Master (
    prod\_no VARCHAR(10) PRIMARY KEY,
    prod\_name VARCHAR(50),
    profit NUMBER(10,2),
    quantity NUMBER,
    sell\_price NUMBER(10,2),
    cost\_price NUMBER(10,2)
);

{-{-} 2. આ ટેબલમાં એક રેકોર્ડ દાખલ કરો}
INSERT INTO PRODUCT\_Master VALUES 
({P001}, {Laptop}, 15000, 10, 45000, 30000);

{-{-} 3. 20000 થી વધુ નફો ધરાવતા પ્રોડક્ટ શોધો}
SELECT * FROM PRODUCT\_Master 
WHERE profit {} 20000;

{-{-} 4. 5 થી ઓછી quantity ધરાવતા પ્રોડક્ટ ડિલીટ કરો}
DELETE FROM PRODUCT\_Master 
WHERE quantity {} 5;

{-{-} 5. 5000 થી વધુ sell\_price ધરાવતા પ્રોડક્ટમાં 2\% નફો ઉમેરો}
UPDATE PRODUCT\_Master 
SET profit = profit * 1.02 
WHERE sell\_price {} 5000;

{-{-} 6. PRODUCT\_Master માં total\_price નામનું નવું ફીલ્ડ ઉમેરો}
ALTER TABLE PRODUCT\_Master 
ADD total\_price NUMBER(10,2);

{-{-} 7. કોઈ ડુપ્લિકેટ ડેટા વગર પ્રોડક્ટ નામ શોધો}
SELECT DISTINCT prod\_name FROM PRODUCT\_Master;
\end{verbatim}

\end{solutionbox}
\begin{mnemonicbox}
``CIDFAUD - Create Insert Delete Find Add Update
Distinct''

\end{mnemonicbox}
\subsection*{પ્રશ્ન 4(a) [3
ગુણ]}\label{q4a}

\textbf{fully functional dependency ઉદાહરણ સાથે સમજાવો.}

\begin{solutionbox}

\textbf{Fully Functional Dependency}: એટ્રિબ્યુટ સંપૂર્ણ રીતે ફંક્શનલ ડિપેન્ડન્ટ છે
જો તે સંપૂર્ણ પ્રાઇમરી કી પર આધારિત હોય, આંશિક કી પર નહીં.

\textbf{ટેબલ:}

{\def\LTcaptype{none} % do not increment counter
\begin{longtable}[]{@{}
  >{\raggedright\arraybackslash}p{(\linewidth - 4\tabcolsep) * \real{0.5135}}
  >{\raggedright\arraybackslash}p{(\linewidth - 4\tabcolsep) * \real{0.2432}}
  >{\raggedright\arraybackslash}p{(\linewidth - 4\tabcolsep) * \real{0.2432}}@{}}
\toprule\noalign{}
\begin{minipage}[b]{\linewidth}\raggedright
ડિપેન્ડન્સી પ્રકાર
\end{minipage} & \begin{minipage}[b]{\linewidth}\raggedright
વ્યાખ્યા
\end{minipage} & \begin{minipage}[b]{\linewidth}\raggedright
ઉદાહરણ
\end{minipage} \\
\midrule\noalign{}
\endhead
\bottomrule\noalign{}
\endlastfoot
\textbf{Full FD} & સંપૂર્ણ કી પર આધારિત & (Student\_ID, Course\_ID) \rightarrow
Grade \\
\textbf{Partial FD} & કીના ભાગ પર આધારિત & (Student\_ID, Course\_ID) \rightarrow
Student\_Name \\
\end{longtable}
}

\begin{verbatim}
ઉદાહરણ: Student_Course(Student_ID, Course_ID, Student_Name, Grade)

Full FD: (Student_ID, Course_ID) \rightarrow Grade
Partial FD: Student_ID \rightarrow Student_Name
\end{verbatim}

\begin{itemize}
\tightlist
\item
  \textbf{સંપૂર્ણ કી}: કોમ્પોઝિટ પ્રાઇમરી કીના બધા એટ્રિબ્યુટ્સ જરૂરી
\item
  \textbf{નોન-કી એટ્રિબ્યુટ}: સંપૂર્ણ પ્રાઇમરી કી કોમ્બિનેશન પર આધારિત
\item
  \textbf{2NF જરૂરિયાત}: આંશિક ડિપેન્ડન્સીઝ દૂર કરે છે
\end{itemize}

\end{solutionbox}
\begin{mnemonicbox}
``FFD - Full Function Dependency''

\end{mnemonicbox}
\subsection*{પ્રશ્ન 4(b) [4
ગુણ]}\label{q4b}

\textbf{નીચે દશાવેલ રિલેશનલ સ્કીમાને ધ્યાનમાં લઈ રિલેશનલ અલજીબ્રા એક્સપ્રેશન આપો:}
\textbf{Employee (Emp\_name, Emp\_id, birth\_date, Post, salary)}

\begin{solutionbox}

\begin{verbatim}
(i) Post="Clerk" ધરાવતા બધા કર્મચારીઓની યાદી બનાવો
σ(Post='Clerk')(Employee)

(ii) salary > 2000 અને post='Manager' ધરાવતા Emp_id અને Emp_name શોધો
π(Emp_id, Emp_name)(σ(salary>2000 \wedge Post='Manager')(Employee))
\end{verbatim}

\textbf{ટેબલ:}

{\def\LTcaptype{none} % do not increment counter
\begin{longtable}[]{@{}lll@{}}
\toprule\noalign{}
પ્રતીક & ઓપરેશન & હેતુ \\
\midrule\noalign{}
\endhead
\bottomrule\noalign{}
\endlastfoot
\textbf{σ} & સિલેક્શન & શરત આધારિત પંક્તિઓ ફિલ્ટર કરે છે \\
\textbf{π} & પ્રોજેક્શન & ચોક્કસ કૉલમ્સ પસંદ કરે છે \\
\textbf{\wedge} & AND & તાર્કિક સંયોજન \\
\end{longtable}
}

\begin{itemize}
\tightlist
\item
  \textbf{સિલેક્શન (σ)}: નિર્દિષ્ટ શરતો પૂરી કરતી પંક્તિઓ પસંદ કરે છે
\item
  \textbf{પ્રોજેક્શન (π)}: પરિણામમાંથી જરૂરી કૉલમ્સ પસંદ કરે છે
\item
  \textbf{સંયુક્ત ઓપરેશન્સ}: અનેક ઓપરેશન્સનો એકસાથે ઉપયોગ કરી શકાય છે
\end{itemize}

\end{solutionbox}
\begin{mnemonicbox}
``SPA - Select Project And''

\end{mnemonicbox}
\subsection*{પ્રશ્ન 4(c) [7
ગુણ]}\label{q4c}

\textbf{2NF ના ક્રાઇટેરિયા શું છે? આપેલ રિલેશનમાં વિવિધ ફંક્શનલ ડિપેન્ડન્સીઝ શોધો અને
તેને 2NF માં નોર્મલાઈઝ કરો.}

\begin{solutionbox}

\textbf{2NF ક્રાઇટેરિયા}:

\begin{itemize}
\tightlist
\item
  1NF માં હોવું જોઈએ
\item
  પ્રાઇમરી કી પર કોઈ આંશિક ફંક્શનલ ડિપેન્ડન્સીઝ ન હોવી જોઈએ
\end{itemize}

\textbf{આપેલ ટેબલ}: Student\_Course(Student\_ID, Course\_ID,
Student\_Name, Course\_Name)

\textbf{ફંક્શનલ ડિપેન્ડન્સીઝ}:

\begin{verbatim}
Student_ID \rightarrow Student_Name (Partial FD)
Course_ID \rightarrow Course_Name (Partial FD)
(Student_ID, Course_ID) \rightarrow (Student_Name, Course_Name) (Full FD)
\end{verbatim}

\textbf{2NF નોર્મલાઈઝેશન}:

\begin{verbatim}
{-{-} ટેબલ 1: Students}
Students(Student\_ID, Student\_Name)

{-{-} ટેબલ 2: Courses  }
Courses(Course\_ID, Course\_Name)

{-{-} ટેબલ 3: Enrollment}
Enrollment(Student\_ID, Course\_ID)
\end{verbatim}

\begin{verbatim}
erDiagram
    STUDENTS \{
        string Student\_ID PK
        string Student\_Name
    \}
    COURSES \{
        string Course\_ID PK
        string Course\_Name
    \}
    ENROLLMENT \{
        string Student\_ID PK,FK
        string Course\_ID PK,FK
    \}
    
    STUDENTS ||{-{-}o\{ ENROLLMENT : enrolls}
    COURSES ||{-{-}o\{ ENROLLMENT : includes}
\end{verbatim}

\end{solutionbox}
\begin{mnemonicbox}
``2NF - Two Normal Form removes partial
dependencies''

\end{mnemonicbox}
\subsection*{પ્રશ્ન 4(a OR) [3
ગુણ]}\label{uxaaauxab0uxab6uxaa8-4a-or-3-uxa97uxaa3}

\textbf{3NF ઉદાહરણ સાથે સમજાવો.}

\begin{solutionbox}

\textbf{3NF (Third Normal Form)}: 2NF માં હોય અને પ્રાઇમરી કી પર કોઈ
ટ્રાન્ઝિટિવ ડિપેન્ડન્સીઝ ન હોય તેવું ટેબલ.

\textbf{ટેબલ:}

{\def\LTcaptype{none} % do not increment counter
\begin{longtable}[]{@{}lll@{}}
\toprule\noalign{}
નોર્મલ ફોર્મ & જરૂરિયાત & નાબૂદ કરે છે \\
\midrule\noalign{}
\endhead
\bottomrule\noalign{}
\endlastfoot
\textbf{3NF} & 2NF માં + કોઈ ટ્રાન્ઝિટિવ ડિપેન્ડન્સીઝ નહીં & ટ્રાન્ઝિટિવ FD \\
\end{longtable}
}

\begin{verbatim}
ઉદાહરણ: Employee(Emp_ID, Dept_ID, Dept_Name)

ટ્રાન્ઝિટિવ ડિપેન્ડન્સી: Emp_ID \rightarrow Dept_ID \rightarrow Dept_Name

3NF ઉકેલ:
Employee(Emp_ID, Dept_ID)
Department(Dept_ID, Dept_Name)
\end{verbatim}

\begin{itemize}
\tightlist
\item
  \textbf{ટ્રાન્ઝિટિવ ડિપેન્ડન્સી}: A \rightarrow B \rightarrow C જ્યાં A પ્રાઇમરી કી છે
\item
  \textbf{નોન-કી ટુ નોન-કી}: નોન-કી એટ્રિબ્યુટ્સ વચ્ચે ડિપેન્ડન્સી
\item
  \textbf{ડિકમ્પોઝિશન}: ટ્રાન્ઝિટિવ ડિપેન્ડન્સીઝ દૂર કરવા માટે ટેબલ વિભાજિત કરવું
\end{itemize}

\end{solutionbox}
\begin{mnemonicbox}
``3NF - Third Normal Form removes Transitive
dependencies''

\end{mnemonicbox}
\subsection*{પ્રશ્ન 4(b OR) [4
ગુણ]}\label{uxaaauxab0uxab6uxaa8-4b-or-4-uxa97uxaa3}

\textbf{નીચે દશાવેલ રિલેશનલ સ્કીમાને ધ્યાનમાં લઈ રિલેશનલ અલજીબ્રા એક્સપ્રેશન આપો:}
\textbf{Students (Name, SPI, DOB, Enrollment No)}

\begin{solutionbox}

\begin{verbatim}
(i) SPI 7.0 થી વધુ હોય તેવા બધા વિદ્યાર્થીઓની યાદી બનાવો
σ(SPI > 7.0)(Students)

(ii) enrollment number 007 હોય તેવા વિદ્યાર્થીનું name, DOB દર્શાવો
π(Name, DOB)(σ(Enrollment_No = '007')(Students))
\end{verbatim}

\textbf{ટેબલ:}

{\def\LTcaptype{none} % do not increment counter
\begin{longtable}[]{@{}lll@{}}
\toprule\noalign{}
ક્વેરી & રિલેશનલ અલજીબ્રા & હેતુ \\
\midrule\noalign{}
\endhead
\bottomrule\noalign{}
\endlastfoot
\textbf{ફિલ્ટર} & σ(condition) & પંક્તિઓ પસંદ કરે છે \\
\textbf{પ્રોજેક્ટ} & π(attributes) & કૉલમ્સ પસંદ કરે છે \\
\end{longtable}
}

\begin{itemize}
\tightlist
\item
  \textbf{પહેલા સિલેક્શન}: પ્રોજેક્શન પહેલાં શરતો લાગુ કરો
\item
  \textbf{ચોક્કસ મૂલ્ય}: સ્ટ્રિંગ લિટરલ્સ માટે ક્વોટ્સનો ઉપયોગ કરો
\item
  \textbf{કૉલમ નામો}: ચોક્કસ એટ્રિબ્યુટ નામો જરૂરી
\end{itemize}

\end{solutionbox}
\begin{mnemonicbox}
``SPI-DOB: Select Project Information - Display
Output Better''

\end{mnemonicbox}
\subsection*{પ્રશ્ન 4(c OR) [7
ગુણ]}\label{uxaaauxab0uxab6uxaa8-4c-or-7-uxa97uxaa3}

\textbf{1NF ના ક્રાઇટેરિયા શું છે? આપેલ ટેબલને બે અલગ અલગ પદ્ધતિથી 1NF માં નોર્મલાઇઝ
કરો.}

\begin{solutionbox}

\textbf{1NF ક્રાઇટેરિયા}:

\begin{itemize}
\tightlist
\item
  દરેક સેલમાં એક જ અણુ મૂલ્ય હોવું જોઈએ
\item
  કોઈ પુનરાવર્તિત જૂથો અથવા એરેઝ નહીં
\item
  દરેક પંક્તિ અનન્ય હોવી જોઈએ
\end{itemize}

\textbf{આપેલ ટેબલ}:

{\def\LTcaptype{none} % do not increment counter
\begin{longtable}[]{@{}lll@{}}
\toprule\noalign{}
EnrollmentNo & Name & Subjects \\
\midrule\noalign{}
\endhead
\bottomrule\noalign{}
\endlastfoot
001 & DEF & Maths,Physics,Chemistry \\
002 & XYZ & History,Biology,English \\
\end{longtable}
}

\textbf{પદ્ધતિ 1 - અલગ પંક્તિઓ}:

{\def\LTcaptype{none} % do not increment counter
\begin{longtable}[]{@{}lll@{}}
\toprule\noalign{}
EnrollmentNo & Name & Subject \\
\midrule\noalign{}
\endhead
\bottomrule\noalign{}
\endlastfoot
001 & DEF & Maths \\
001 & DEF & Physics \\
001 & DEF & Chemistry \\
002 & XYZ & History \\
002 & XYZ & Biology \\
002 & XYZ & English \\
\end{longtable}
}

\textbf{પદ્ધતિ 2 - અલગ ટેબલ્સ}:

\begin{verbatim}
{-{-} Students ટેબલ}
Students(EnrollmentNo, Name)

{-{-} Subjects ટેબલ  }
Subjects(SubjectID, SubjectName)

{-{-} Student\_Subjects ટેબલ}
Student\_Subjects(EnrollmentNo, SubjectID)
\end{verbatim}

\end{solutionbox}
\begin{mnemonicbox}
``1NF - One Normal Form creates Atomic values''

\end{mnemonicbox}
\subsection*{પ્રશ્ન 5(a) [3
ગુણ]}\label{q5a}

\textbf{ટ્રાન્ઝેક્શનની ACID પ્રોપર્ટીઝ સમજાવો.}

\begin{solutionbox}

\textbf{ટેબલ:}

{\def\LTcaptype{none} % do not increment counter
\begin{longtable}[]{@{}lll@{}}
\toprule\noalign{}
પ્રોપર્ટી & વર્ણન & હેતુ \\
\midrule\noalign{}
\endhead
\bottomrule\noalign{}
\endlastfoot
\textbf{Atomicity} & સંપૂર્ણ અથવા કંઈ જ એક્ઝિક્યુશન & ટ્રાન્ઝેક્શન સંપૂર્ણતા \\
\textbf{Consistency} & ડેટાબેઝ માન્ય રહે છે & ડેટા અખંડિતતા \\
\textbf{Isolation} & સમવર્તી ટ્રાન્ઝેક્શન્સ સ્વતંત્ર & હસ્તક્ષેપ ટાળવો \\
\textbf{Durability} & કમિટ થયેલા ફેરફારો કાયમી & ડેટા સ્થિરતા \\
\end{longtable}
}

\begin{itemize}
\tightlist
\item
  \textbf{Atomicity}: ટ્રાન્ઝેક્શન સંપૂર્ણ રીતે એક્ઝિક્યુટ થાય અથવા બિલકુલ નહીં
\item
  \textbf{Consistency}: ટ્રાન્ઝેક્શન પહેલાં/પછી ડેટાબેઝ કન્સ્ટ્રેઇન્ટ્સ જાળવાય છે
\item
  \textbf{Isolation}: ટ્રાન્ઝેક્શન્સ એકબીજા સાથે હસ્તક્ષેપ કરતા નથી
\item
  \textbf{Durability}: એકવાર કમિટ થયા પછી, ફેરફારો સિસ્ટમ ફેઇલ્યુર્સમાં ટકી રહે
  છે
\end{itemize}

\end{solutionbox}
\begin{mnemonicbox}
``ACID - All Consistent Isolated Durable''

\end{mnemonicbox}
\subsection*{પ્રશ્ન 5(b) [4
ગુણ]}\label{q5b}

\textbf{નીચે દશાવેલ સ્પેસિફિકેશન મુજબ ટેબલ બનાવો:} \textbf{STUDENT: (stu\_id,
stu\_name, Address, City, contact\_no, Branch\_name)}

\begin{solutionbox}

\begin{verbatim}
CREATE TABLE STUDENT (
    stu\_id VARCHAR(10) PRIMARY KEY,
    stu\_name VARCHAR(50) NOT NULL,
    Address VARCHAR(100),
    City VARCHAR(30),
    contact\_no NUMBER(10),
    Branch\_name VARCHAR(20) CHECK (Branch\_name IN ({IT}, {Computer}, {Electrical}, {Civil}))
);
\end{verbatim}

\textbf{ટેબલ:}

{\def\LTcaptype{none} % do not increment counter
\begin{longtable}[]{@{}lll@{}}
\toprule\noalign{}
કન્સ્ટ્રેઇન્ટ & હેતુ & અમલીકરણ \\
\midrule\noalign{}
\endhead
\bottomrule\noalign{}
\endlastfoot
\textbf{NOT NULL} & ફરજિયાત ફીલ્ડ & stu\_name NOT NULL \\
\textbf{CHECK} & વેલ્યુ વેલિડેશન & Branch\_name IN (\ldots) \\
\end{longtable}
}

\begin{itemize}
\tightlist
\item
  \textbf{પ્રાઇમરી કી}: stu\_id દરેક વિદ્યાર્થીને અનન્ય રીતે ઓળખે છે
\item
  \textbf{NOT NULL}: stu\_name ખાલી હોઈ શકે નહીં
\item
  \textbf{CHECK કન્સ્ટ્રેઇન્ટ}: Branch\_name નિર્દિષ્ટ મૂલ્યો સુધી મર્યાદિત
\item
  \textbf{ડેટા ટાઇપ્સ}: દરેક ફીલ્ડ માટે યોગ્ય સાઇઝ
\end{itemize}

\end{solutionbox}
\begin{mnemonicbox}
``CNPD - Constraints Names Primary Datatypes''

\end{mnemonicbox}
\subsection*{પ્રશ્ન 5(c) [7
ગુણ]}\label{q5c}

\textbf{ટ્રિગર શું છે? Oracle માં ટ્રિગર બનાવવા માટે સિન્ટેક્સ લખો. સિમ્પલ ટ્રિગર
બનાવો.}

\begin{solutionbox}

\textbf{ટ્રિગર}: વિશેષ સ્ટોર્ડ પ્રોસીજર જે ડેટાબેઝ ઇવેન્ટ્સના પ્રતિભાવમાં આપોઆપ
એક્ઝિક્યુટ થાય છે.

\textbf{Oracle ટ્રિગર સિન્ટેક્સ}:

\begin{verbatim}
CREATE [OR REPLACE] TRIGGER trigger\_name
\{BEFORE | AFTER | INSTEAD OF\ \{}INSERT | UPDATE | DELETE\}
ON table\_name
[FOR EACH ROW]
[WHEN condition]
DECLARE
    {-{-} Variable declarations}
BEGIN
    {-{-} Trigger logic}
END;
\end{verbatim}

\textbf{સિમ્પલ ટ્રિગર ઉદાહરણ}:

\begin{verbatim}
CREATE OR REPLACE TRIGGER display\_student\_trigger
BEFORE INSERT ON STUDENT
FOR EACH ROW
BEGIN
    DBMS\_OUTPUT.PUT\_LINE({Inserting student: } || :NEW.stu\_name || 
                        { with ID: } || :NEW.stu\_id);
END;
\end{verbatim}

\textbf{ટેબલ:}

{\def\LTcaptype{none} % do not increment counter
\begin{longtable}[]{@{}lll@{}}
\toprule\noalign{}
ટ્રિગર પ્રકાર & ક્યારે એક્ઝિક્યુટ થાય & હેતુ \\
\midrule\noalign{}
\endhead
\bottomrule\noalign{}
\endlastfoot
\textbf{BEFORE} & DML ઓપરેશન પહેલાં & વેલિડેશન, મોડિફિકેશન \\
\textbf{AFTER} & DML ઓપરેશન પછી & લોગિંગ, ઓડિટિંગ \\
\textbf{FOR EACH ROW} & રો-લેવલ ટ્રિગર & પ્રતિ પંક્તિ એક્ઝિક્યુશન \\
\end{longtable}
}

\begin{itemize}
\tightlist
\item
  \textbf{:NEW}: દાખલ/અપડેટ કરવામાં આવતા નવા મૂલ્યોનો સંદર્ભ
\item
  \textbf{:OLD}: ડિલીટ/અપડેટ કરવામાં આવતા જૂના મૂલ્યોનો સંદર્ભ
\item
  \textbf{આપોઆપ એક્ઝિક્યુશન}: નિર્દિષ્ષ્ટ ઇવેન્ટ્સ પર આપોઆપ ફાયર થાય છે
\item
  \textbf{બિઝનેસ લોજિક}: જટિલ બિઝનેસ નિયમો લાગુ કરે છે
\end{itemize}

\end{solutionbox}
\begin{mnemonicbox}
``TBA-FEN - Triggers Before After For Each New''

\end{mnemonicbox}
\subsection*{પ્રશ્ન 5(a OR) [3
ગુણ]}\label{uxaaauxab0uxab6uxaa8-5a-or-3-uxa97uxaa3}

\textbf{ટ્રાન્ઝેક્શનમાં કન્કરન્સી કંટ્રોલના પ્રોબ્લેમ્સ સમજાવો.}

\begin{solutionbox}

\textbf{ટેબલ:}

{\def\LTcaptype{none} % do not increment counter
\begin{longtable}[]{@{}
  >{\raggedright\arraybackslash}p{(\linewidth - 4\tabcolsep) * \real{0.3333}}
  >{\raggedright\arraybackslash}p{(\linewidth - 4\tabcolsep) * \real{0.2917}}
  >{\raggedright\arraybackslash}p{(\linewidth - 4\tabcolsep) * \real{0.3750}}@{}}
\toprule\noalign{}
\begin{minipage}[b]{\linewidth}\raggedright
સમસ્યા
\end{minipage} & \begin{minipage}[b]{\linewidth}\raggedright
વર્ણન
\end{minipage} & \begin{minipage}[b]{\linewidth}\raggedright
ઉદાહરણ
\end{minipage} \\
\midrule\noalign{}
\endhead
\bottomrule\noalign{}
\endlastfoot
\textbf{Lost Update} & એક ટ્રાન્ઝેક્શન બીજાના ફેરફારો પર લખે છે & T1, T2 સમાન
રેકોર્ડ અપડેટ કરે છે \\
\textbf{Dirty Read} & અનકમિટ ડેટા વાંચવો & T1 T2 ના અનકમિટ ફેરફારો વાંચે છે \\
\textbf{Unrepeatable Read} & સમાન ક્વેરી અલગ પરિણામો આપે છે & T1 વાંચે, T2
અપડેટ કરે, T1 ફરી વાંચે \\
\end{longtable}
}

\begin{itemize}
\tightlist
\item
  \textbf{Phantom Read}: સમાન ટ્રાન્ઝેક્શનમાં ક્વેરીઝ વચ્ચે નવી પંક્તિઓ દેખાય છે
\item
  \textbf{Deadlock}: બે ટ્રાન્ઝેક્શન્સ એકબીજાના લોક્સની રાહ જુએ છે
\item
  \textbf{Inconsistent Analysis}: ડેટા સંશોધિત થતો હોય ત્યારે વાંચવો
\end{itemize}

\end{solutionbox}
\begin{mnemonicbox}
``LDU-PID - Lost Dirty Unrepeatable Phantom
Inconsistent Deadlock''

\end{mnemonicbox}
\subsection*{પ્રશ્ન 5(b OR) [4
ગુણ]}\label{uxaaauxab0uxab6uxaa8-5b-or-4-uxa97uxaa3}

\textbf{નીચે દશાવેલ સ્પેસિફિકેશન મુજબ ટેબલ બનાવો:} \textbf{STUDENT: (stu\_id,
stu\_name, Address, City, contact\_no, Branch\_name)}

\begin{solutionbox}

\begin{verbatim}
CREATE TABLE STUDENT (
    stu\_id VARCHAR(10) PRIMARY KEY CHECK (stu\_id LIKE {S\%}),
    stu\_name VARCHAR(50),
    Address VARCHAR(100),
    City VARCHAR(30),
    contact\_no NUMBER(10),
    Branch\_name VARCHAR(20)
);
\end{verbatim}

\textbf{ટેબલ:}

{\def\LTcaptype{none} % do not increment counter
\begin{longtable}[]{@{}lll@{}}
\toprule\noalign{}
કન્સ્ટ્રેઇન્ટ & અમલીકરણ & હેતુ \\
\midrule\noalign{}
\endhead
\bottomrule\noalign{}
\endlastfoot
\textbf{PRIMARY KEY} & stu\_id PRIMARY KEY & અનન્ય ઓળખ \\
\textbf{CHECK} & stu\_id LIKE `S\%' & `S' થી શરૂ થવું જોઈએ \\
\end{longtable}
}

\begin{itemize}
\tightlist
\item
  \textbf{પ્રાઇમરી કી}: stu\_id અનન્ય આઇડેન્ટિફાયર તરીકે કામ કરે છે
\item
  \textbf{પેટર્ન ચેક}: stu\_id અક્ષર `S' થી શરૂ થવું જોઈએ
\item
  \textbf{ડેટા ટાઇપ્સ}: યોગ્ય ફીલ્ડ સાઇઝ અને ટાઇપ્સ
\item
  \textbf{કન્સ્ટ્રેઇન્ટ વેલિડેશન}: ડેટાબેઝ આપોઆપ નિયમો લાગુ કરે છે
\end{itemize}

\end{solutionbox}
\begin{mnemonicbox}
``PKC-ST - Primary Key Check Starts''

\end{mnemonicbox}
\subsection*{પ્રશ્ન 5(c OR) [7
ગુણ]}\label{uxaaauxab0uxab6uxaa8-5c-or-7-uxa97uxaa3}

\textbf{એક્સપ્લિસિટ કર્સર શું છે? એક્સપ્લિસિટ કર્સર ઉદાહરણ સાથે સમજાવો.}

\begin{solutionbox}

\textbf{એક્સપ્લિસિટ કર્સર}: અનેક પંક્તિઓ પરત કરતા SELECT સ્ટેટમેન્ટ્સ હેન્ડલ કરવા માટે
પ્રોગ્રામેટિક કંટ્રોલ સાથે વપરાશકર્તા-વ્યાખ્યાયિત કર્સર.

\textbf{કર્સર ઓપરેશન્સ}:

\begin{verbatim}
{-{-} ડિક્લેરેશન}
DECLARE
    CURSOR student\_cursor IS
        SELECT stu\_id, stu\_name FROM STUDENT WHERE city = {Ahmedabad};
    v\_id STUDENT.stu\_id\%TYPE;
    v\_name STUDENT.stu\_name\%TYPE;
BEGIN
    {-{-} કર્સર ઓપન કરો}
    OPEN student\_cursor;
    
    {-{-} ડેટા ફેચ કરો}
    LOOP
        FETCH student\_cursor INTO v\_id, v\_name;
        EXIT WHEN student\_cursor\%NOTFOUND;
        
        DBMS\_OUTPUT.PUT\_LINE({ID: } || v\_id || {, Name: } || v\_name);
    END LOOP;
    
    {-{-} કર્સર બંધ કરો}
    CLOSE student\_cursor;
END;
\end{verbatim}

\textbf{ટેબલ:}

{\def\LTcaptype{none} % do not increment counter
\begin{longtable}[]{@{}lll@{}}
\toprule\noalign{}
ઓપરેશન & હેતુ & સિન્ટેક્સ \\
\midrule\noalign{}
\endhead
\bottomrule\noalign{}
\endlastfoot
\textbf{DECLARE} & કર્સર ડિફાઇન કરવું & CURSOR name IS SELECT\ldots{} \\
\textbf{OPEN} & કર્સર ઇનિશિયલાઇઝ કરવું & OPEN cursor\_name \\
\textbf{FETCH} & ડેટા મેળવવો & FETCH cursor INTO variables \\
\textbf{CLOSE} & રિસોર્સ છોડવા & CLOSE cursor\_name \\
\end{longtable}
}

\begin{center}
\textbf{Mermaid Diagram (Code)}
\begin{verbatim}
{Shaded}
{Highlighting}[]
graph LR
    A[DECLARE Cursor] {-{-}{} B[OPEN Cursor]}
    B {-{-}{} C[FETCH Data]}
    C {-{-}{} D\{વધુ Rows?\}}
    D {-{-}{}|હા| C}
    D {-{-}{}|ના| E[CLOSE Cursor]}
{Highlighting}
{Shaded}
\end{verbatim}
\end{center}

\begin{itemize}
\tightlist
\item
  \textbf{મેન્યુઅલ કંટ્રોલ}: પ્રોગ્રામર કર્સર ઓપરેશન્સને નિયંત્રિત કરે છે
\item
  \textbf{મેમરી મેનેજમેન્ટ}: સ્પષ્ટ રીતે ઓપન અને ક્લોઝ કરવું જોઈએ
\item
  \textbf{લૂપ પ્રોસેસિંગ}: સામાન્ય રીતે અનેક પંક્તિઓ માટે લૂપ્સ સાથે ઉપયોગ થાય છે
\item
  \textbf{કર્સર એટ્રિબ્યુટ્સ}: \%FOUND, \%NOTFOUND, \%ROWCOUNT
\end{itemize}

\end{solutionbox}
\begin{mnemonicbox}
``DOFC - Declare Open Fetch Close''

\end{mnemonicbox}

\end{document}
