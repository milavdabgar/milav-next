\documentclass{article}

% content/resources/templates/preamble.tex
\usepackage[margin=0.6in]{geometry}
\author{Milav Dabgar}
\usepackage{amsmath,amssymb,amsthm}
\usepackage{booktabs}
\usepackage{multirow}
\usepackage{xcolor}
\usepackage{tcolorbox}
\tcbuselibrary{breakable,skins}
\usepackage[colorlinks=true,linkcolor=blue]{hyperref}
\usepackage{titlesec}
\usepackage{enumitem}
\usepackage{tikz}
\usepackage{pgfplots}
\usepackage{circuitikz}
\usepackage[version=4]{mhchem}
\usepackage{longtable}
\usepackage{array}
\usepackage{float}
\usepackage{caption}
\usepackage{listings}

\lstset{
  basicstyle=\small\ttfamily,
  breaklines=true,
  breakatwhitespace=false,
  postbreak=\mbox{\textcolor{red}{$\hookrightarrow$}\space},
  float=false,
  numbers=left,
  numberstyle=\tiny\color{gray},
  numbersep=10pt,
  xleftmargin=2em,
  keywordstyle=\color{blue},
  commentstyle=\color{green!60!black},
  stringstyle=\color{purple},
  backgroundcolor=\color{gray!5},
  showstringspaces=false,
  tabsize=2,
  captionpos=b,
  keepspaces=true,
  columns=flexible
}

\pgfplotsset{compat=1.18}
\usetikzlibrary{shapes,arrows,positioning,calc,patterns,decorations.pathmorphing,decorations.markings,arrows.meta}

% Color scheme
\definecolor{headcolor}{RGB}{0,102,204}
\definecolor{keycolor}{RGB}{220,20,60}
\definecolor{solutioncolor}{RGB}{34,139,34}
\definecolor{mnemoniccolor}{RGB}{148,0,211}
\definecolor{codecolor}{RGB}{0,0,100}

% Spacing
\setlength{\parskip}{3pt}
\setlist[itemize]{nosep}
\setlist[enumerate]{nosep}

% Title formatting
\titleformat{\section}{\Large\bfseries\color{headcolor}}{\thesection}{1em}{}
\titleformat{\subsection}{\large\bfseries\color{headcolor}}{\thesubsection}{1em}{}

% Pandoc tightlist compatibility
\providecommand{\tightlist}{%
  \setlength{\itemsep}{0pt}\setlength{\parskip}{0pt}}

% Pandoc longtable compatibility
\newcounter{none}
\def\thenone{}


% content/resources/templates/gujarati-boxes.tex
\usepackage{fontspec}
\usepackage{polyglossia}

% Set Gujarati as main language (document is primarily in Gujarati)
% Note: gloss-gujarati.ldf doesn't exist in polyglossia, but it will use hyphenation patterns
\setdefaultlanguage{gujarati}
\setotherlanguage{english}

% Configure Gujarati font properly
% Use Language=Default to prevent polyglossia from trying to add language-specific features
% that don't exist for Gujarati, which causes "empty feature" warnings
\newfontfamily\gujaratifont[Script=Gujarati,AutoFakeBold=2.5,AutoFakeSlant=0.3]{Noto Sans Gujarati}
\setmainfont[Script=Gujarati,AutoFakeBold=2.5,AutoFakeSlant=0.3]{Noto Sans Gujarati}
% Use Noto Sans Gujarati for monospace to support Gujarati in text
\setmonofont[Scale=0.9]{Noto Sans Gujarati}

% Configure English to use the same font
\newfontfamily\englishfont[Script=Gujarati,AutoFakeBold=2.5,AutoFakeSlant=0.3]{Noto Sans Gujarati}

% Translations for polyglossia
\gappto\captionsgujarati{
  \renewcommand{\tablename}{કોષ્ટક}
  \renewcommand{\figurename}{આકૃતિ}
}

% Helper for TikZ nodes to ensure Gujarati font
\newcommand{\gu}[1]{{\gujaratifont #1}}

% Custom environments
\newtcolorbox{solutionbox}{
    breakable,
    enhanced,
    colback=solutioncolor!5!white,
    colframe=solutioncolor!75!black,
    fonttitle=\bfseries,
    title=જવાબ
}

\newtcolorbox{solutionboxnobreak}{
 colback=solutioncolor!5!white,
 colframe=solutioncolor!75!black,
 fonttitle=\bfseries,
 title=જવાબ
}

\newtcolorbox{keyformula}{
 breakable,
 enhanced,
 colback=keycolor!5!white,
 colframe=keycolor!75!black,
 fonttitle=\bfseries,
 title=રાસાયણિક સમીકરણ/સૂત્ર
}

\newtcolorbox{mnemonicbox}{
 breakable,
 enhanced,
 colback=mnemoniccolor!5!white,
 colframe=mnemoniccolor!75!black,
 fonttitle=\bfseries,
 title=મેમરી ટ્રીક
}


% Custom commands for GTU solutions
% This file defines semantic commands for consistent formatting

% Question command with automatic formatting
\newcommand{\question}[2]{%
  \section*{Question #1}%
  \textbf{#2}%
}

% OR question variant
\newcommand{\questionor}[2]{%
  \section*{Question #1 OR}%
  \textbf{#2}%
}

% Proper table environment with caption
\newenvironment{answertable}[1]{%
  \begin{table}[htbp]
  \centering
  \caption{#1}
}{%
  \end{table}
}

% Proper figure environment for diagrams
\newenvironment{answerdiagram}[1]{%
  \begin{figure}[htbp]
  \centering
  \caption{#1}
}{%
  \end{figure}
}

% Semantic markup for key terms
\newcommand{\keyword}[1]{\textbf{#1}}
\newcommand{\code}[1]{\texttt{#1}}
\newcommand{\classname}[1]{\texttt{#1}}
\newcommand{\methodname}[1]{\texttt{#1}}

% Proper quotation marks
\newcommand{\mnemonic}[1]{``#1''}

\usetikzlibrary{mindmap, trees}

\title{Fundamentals of Software Development (4331604) - શિયાળું 2023 ઉકેલ}
\date{January 20, 2024}

\begin{document}
\maketitle

\questionmarks{1(a)}{3}{Define Software and explain its characteristics.}

\begin{solutionbox}
સૉફ્ટવેર એ પ્રોગ્રામ્સ, સૂચનાઓ અને દસ્તાવેજીકરણનો સંગ્રહ છે જે કમ્પ્યુટર સિસ્ટમ પર કાર્યો કરે છે.

\textbf{મુખ્ય લક્ષણો:}
\begin{center}
\captionof{table}{સૉફ્ટવેર લક્ષણો}
\begin{tabulary}{\linewidth}{|L|L|}
\hline
\textbf{લક્ષણ} & \textbf{વર્ણન} \\ \hline
\textbf{અમૂર્ત} & શારીરિક રીતે સ્પર્શ કરી શકાતું નથી \\ \hline
\textbf{તાર્કિક} & વ્યવસ્થિત અભિગમ દ્વારા બનાવાયેલ \\ \hline
\textbf{ઉત્પાદિત} & પરંપરાગત રીતે ઉત્પન્ન નહીં, વિકસિત \\ \hline
\textbf{જટિલ} & અંતર્ગત જટિલ માળખું ધરાવે છે \\ \hline
\end{tabulary}
\end{center}
\end{solutionbox}

\begin{mnemonicbox}
\mnemonic{અમૂર્ત તાર્કિક ઉત્પાદન જટિલતા}
\end{mnemonicbox}

\questionmarks{1(b)}{4}{Write a note on Software engineering – A layered technology.}

\begin{solutionbox}
Software Engineering એ સ્તરીય ટેકનોલોજી તરીકે માળખાકીય છે જ્યાં દરેક સ્તર આગામી સ્તરને આધાર આપે છે.

\textbf{સ્તરીય માળખું:}
\begin{center}
\begin{tikzpicture}[node distance=0cm, outer sep=0pt]
    \node (Tools) [gtu block, minimum width=8cm, fill=blue!10] {Tools};
    \node (Methods) [gtu block, minimum width=8cm, fill=blue!20, above=of Tools] {Methods};
    \node (Process) [gtu block, minimum width=8cm, fill=blue!30, above=of Methods] {Process};
    \node (Quality) [gtu block, minimum width=8cm, fill=blue!40, above=of Process] {Quality Focus - પાયો};
\end{tikzpicture}
\captionof{figure}{Software Engineering Layers}
\end{center}

\begin{center}
\captionof{table}{સ્તર વર્ણન}
\begin{tabulary}{\linewidth}{|L|L|L|}
\hline
\textbf{સ્તર} & \textbf{હેતુ} & \textbf{વર્ણન} \\ \hline
\textbf{Quality Focus} & પાયો & ગુણવત્તાયુક્ત ઉત્પાદનો પહોંચાડવાનો ભાર \\ \hline
\textbf{Process} & ફ્રેમવર્ક & સૉફ્ટવેર ડેવલપમેન્ટ કેવી રીતે કરવું તે નક્કી કરે છે \\ \hline
\textbf{Methods} & તકનીકો & પ્રવૃત્તિઓ કરવાની વિશિષ્ટ પદ્ધતિઓ \\ \hline
\textbf{Tools} & સ્વચાલન & પદ્ધતિઓને આધાર આપતું સૉફ્ટવેર \\ \hline
\end{tabulary}
\end{center}
\end{solutionbox}

\begin{mnemonicbox}
\mnemonic{ટૂલ્સ મેથડ્સ પ્રોસેસ ક્વોલિટી}
\end{mnemonicbox}

\questionmarks{1(c)}{7}{Explain Software Process framework and umbrella activities.}

\begin{solutionbox}
સૉફ્ટવેર પ્રોસેસ ફ્રેમવર્ક સૉફ્ટવેર ડેવલપમેન્ટ માટે મુખ્ય પ્રવૃત્તિઓ અને umbrella પ્રવૃત્તિઓ સાથે માળખું પ્રદાન કરે છે.

\textbf{ફ્રેમવર્ક પ્રવૃત્તિઓ:}
\begin{center}
\captionof{table}{ફ્રેમવર્ક પ્રવૃત્તિઓ}
\begin{tabulary}{\linewidth}{|L|L|L|}
\hline
\textbf{પ્રવૃત્તિ} & \textbf{હેતુ} & \textbf{મુખ્ય કાર્યો} \\ \hline
\textbf{Communication} & આવશ્યકતાઓ સમજવી & હિસ્સેદારો સાથે વાતચીત, આવશ્યકતા એકત્રીકરણ \\ \hline
\textbf{Planning} & રોડમેપ બનાવવો & અંદાજ, શેડ્યૂલિંગ, જોખમ મૂલ્યાંકન \\ \hline
\textbf{Modeling} & બ્લુપ્રિન્ટ બનાવવા & વિશ્લેષણ અને ડિઝાઇન મોડલ્સ \\ \hline
\textbf{Construction} & સૉફ્ટવેર બનાવવું & કોડિંગ અને ટેસ્ટિંગ \\ \hline
\textbf{Deployment} & વપરાશકર્તાઓને પહોંચાડવું & ઇન્સ્ટોલેશન, સપોર્ટ, ફીડબેક \\ \hline
\end{tabulary}
\end{center}

\textbf{Umbrella પ્રવૃત્તિઓ:}
\begin{itemize}
    \item \keyword{Software project tracking}: પ્રગતિ નિરીક્ષણ અને ગુણવત્તા નિયંત્રણ
    \item \keyword{Risk management}: સંભવિત સમસ્યાઓ ઓળખવી અને ઘટાડવી
    \item \keyword{Quality assurance}: ધોરણો પૂરા થાય તેની ખાતરી કરવી
    \item \keyword{Configuration management}: ફેરફારોને વ્યવસ્થિત રીતે નિયંત્રિત કરવા
    \item \keyword{Work product preparation}: ડિલિવરેબલ દસ્તાવેજો બનાવવા
\end{itemize}

\begin{center}
\begin{tikzpicture}[node distance=1.2cm, auto, every node/.style={gtu block, align=center, font=\footnotesize, minimum width=1.8cm}]
    \node (Comm) {Communication};
    \node [right=0.5cm of Comm] (Plan) {Planning};
    \node [right=0.5cm of Plan] (Mod) {Modeling};
    \node [right=0.5cm of Mod] (Const) {Construction};
    \node [right=0.5cm of Const] (Dep) {Deployment};
    
    \node [below=1cm of Mod, minimum width=10cm, fill=orange!10] (Umb) {Umbrella Activities};

    \path [gtu arrow] (Comm) -- (Plan);
    \path [gtu arrow] (Plan) -- (Mod);
    \path [gtu arrow] (Mod) -- (Const);
    \path [gtu arrow] (Const) -- (Dep);
    
    \draw [dashed, ->] (Umb) -- (Comm);
    \draw [dashed, ->] (Umb) -- (Plan);
    \draw [dashed, ->] (Umb) -- (Mod);
    \draw [dashed, ->] (Umb) -- (Const);
    \draw [dashed, ->] (Umb) -- (Dep);
\end{tikzpicture}
\captionof{figure}{Process Framework}
\end{center}
\end{solutionbox}

\begin{mnemonicbox}
\mnemonic{કોમ્યુનિકેશન પ્લાનિંગ મોડલિંગ કન્સ્ટ્રક્શન ડિપ્લોયમેન્ટ}
\mnemonic{ટ્રેક રિસ્ક ક્વોલિટી કન્ફિગરેશન વર્ક}
\end{mnemonicbox}

\questionmarks{1(c OR)}{7}{Define SDLC and explain each phase.}

\begin{solutionbox}
SDLC (Software Development Life Cycle) એ સૉફ્ટવેર એપ્લિકેશન્સ વિકસાવવા માટેની વ્યવસ્થિત પ્રક્રિયા છે.

\textbf{SDLC તબક્કાઓ:}
\begin{center}
\captionof{table}{SDLC તબક્કાઓ}
\begin{tabulary}{\linewidth}{|L|L|L|L|}
\hline
\textbf{તબક્કો} & \textbf{હેતુ} & \textbf{મુખ્ય પ્રવૃત્તિઓ} & \textbf{ડિલિવરેબલ્સ} \\ \hline
\textbf{Planning} & અવકાશ નક્કી કરવો & શક્યતા અભ્યાસ, સંસાધન ફાળવણી & પ્રોજેક્ટ પ્લાન \\ \hline
\textbf{Analysis} & આવશ્યકતાઓ એકત્રિત કરવી & આવશ્યકતા સંગ્રહ, દસ્તાવેજીકરણ & SRS દસ્તાવેજ \\ \hline
\textbf{Design} & આર્કિટેક્ચર બનાવવું & સિસ્ટમ ડિઝાઇન, ડેટાબેસ ડિઝાઇન & ડિઝાઇન દસ્તાવેજો \\ \hline
\textbf{Implementation} & કોડ લખવો & પ્રોગ્રામિંગ, યુનિટ ટેસ્ટિંગ & સોર્સ કોડ \\ \hline
\textbf{Testing} & ગુણવત્તા ચકાસવી & સિસ્ટમ ટેસ્ટિંગ, બગ ફિક્સિંગ & ટેસ્ટ રિપોર્ટ્સ \\ \hline
\textbf{Deployment} & સૉફ્ટવેર રિલીઝ કરવું & ઇન્સ્ટોલેશન, યુઝર ટ્રેનિંગ & લાઇવ સિસ્ટમ \\ \hline
\textbf{Maintenance} & ચાલુ સપોર્ટ & બગ ફિક્સ, એન્હાન્સમેન્ટ્સ & અપડેટેડ સિસ્ટમ \\ \hline
\end{tabulary}
\end{center}

\begin{center}
\begin{tikzpicture}[node distance=1.5cm, auto, every node/.style={gtu block, align=center, font=\small}]
    \node (Plan) {Planning};
    \node [right=of Plan] (Ana) {Analysis};
    \node [right=of Ana] (Des) {Design};
    \node [below=of Des] (Imp) {Implementation};
    \node [left=of Imp] (Test) {Testing};
    \node [left=of Test] (Dep) {Deployment};
    \node [below=of Dep] (Maint) {Maintenance};

    \path [gtu arrow] (Plan) -- (Ana);
    \path [gtu arrow] (Ana) -- (Des);
    \path [gtu arrow] (Des) -- (Imp);
    \path [gtu arrow] (Imp) -- (Test);
    \path [gtu arrow] (Test) -- (Dep);
    \path [gtu arrow] (Dep) -- (Maint);
\end{tikzpicture}
\captionof{figure}{SDLC Phases}
\end{center}
\end{solutionbox}

\begin{mnemonicbox}
\mnemonic{પ્લાન એનાલિસિસ ડિઝાઇન ઇમ્પ્લિમેન્ટેશન ટેસ્ટિંગ ડિપ્લોયમેન્ટ મેઇન્ટેનન્સ}
\end{mnemonicbox}

\questionmarks{2(a)}{3}{Describe advantage disadvantage of prototype model.}

\begin{solutionbox}
\textbf{Prototype Model વિશ્લેષણ:}
\begin{center}
\captionof{table}{Prototype Model ફાયદા/નુકસાન}
\begin{tabulary}{\linewidth}{|L|L|}
\hline
\textbf{ફાયદા} & \textbf{નુકસાન} \\ \hline
\textbf{વહેલો ફીડબેક} વપરાશકર્તાઓ તરફથી & \textbf{સમય વાપરતું} ડેવલપમેન્ટ પ્રોસેસ \\ \hline
\textbf{ઓછું જોખમ} નિષ્ફળતાનું & \textbf{ખર્ચમાં વધારો} પુનરાવર્તન કારણે \\ \hline
\textbf{બહેતર સમજ} આવશ્યકતાઓની & \textbf{Scope creep} થઈ શકે છે \\ \hline
\end{tabulary}
\end{center}
\end{solutionbox}

\begin{mnemonicbox}
\mnemonic{વહેલો ઓછું બહેતર વિરુદ્ધ સમય ખર્ચ સ્કોપ}
\end{mnemonicbox}

\questionmarks{2(b)}{4}{Explain Prototyping Model and justify when to use with example.}

\begin{solutionbox}
Prototyping Model વિકાસ પ્રક્રિયાની શરૂઆતમાં સૉફ્ટવેરનું કાર્યશીલ મોડલ બનાવે છે.

\textbf{ક્યારે ઉપયોગ કરવો:}
\begin{center}
\captionof{table}{ઉપયોગના ઉદાહરણો}
\begin{tabulary}{\linewidth}{|L|L|L|}
\hline
\textbf{સ્થિતિ} & \textbf{ઉદાહરણ} & \textbf{જસ્ટિફિકેશન} \\ \hline
\textbf{અસ્પષ્ટ આવશ્યકતાઓ} & ઓનલાઇન શોપિંગ કાર્ટ & યુઝર ઇન્ટરફેસને સુધારવાની જરૂર \\ \hline
\textbf{નવી ટેકનોલોજી} & મોબાઇલ બેંકિંગ એપ & શક્યતા પરીક્ષણ જરૂરી \\ \hline
\textbf{યુઝર ઇન્ટરેક્શન જટિલ} & ગેમિંગ એપ્લિકેશન & યુઝર અનુભવ ચકાસણી જરૂરી \\ \hline
\end{tabulary}
\end{center}

\begin{center}
\begin{tikzpicture}[node distance=1.2cm, auto, every node/.style={gtu block, align=center, font=\footnotesize}]
    \node (Req) {Requirements};
    \node [right=of Req] (Des) {Quick Design};
    \node [right=of Des] (Build) {Build Prototype};
    \node [below=of Build] (User) {User Evaluation};
    \node [left=of User] (Sat) {Satisfied?};
    \node [left=of Sat] (Final) {Final System};

    \path [gtu arrow] (Req) -- (Des);
    \path [gtu arrow] (Des) -- (Build);
    \path [gtu arrow] (Build) -- (User);
    \path [gtu arrow] (User) -- (Sat);
    \path [gtu arrow] (Sat) -- node[above]{No} (Des);
    \path [gtu arrow] (Sat) -- node[above]{Yes} (Final);
\end{tikzpicture}
\captionof{figure}{Prototyping Process}
\end{center}
\end{solutionbox}

\begin{mnemonicbox}
\mnemonic{આવશ્યકતા ઝડપી બિલ્ડ યુઝર સંતુષ્ટ ફાઇનલ}
\end{mnemonicbox}

\questionmarks{2(c)}{7}{Sketch and discuss (I) Waterfall model \& (II) Incremental Model.}

\begin{solutionbox}
\textbf{(I) Waterfall Model:}
રેખીય ક્રમિક અભિગમ જ્યાં દરેક તબક્કો આગલા તબક્કા પહેલાં પૂર્ણ થવો જોઈએ.

\begin{center}
\begin{tikzpicture}[node distance=1.5cm, auto, every node/.style={gtu block, align=center, minimum width=2.5cm}]
    \node (Req) {Requirements\\Analysis};
    \node [below right=0.5cm and 0.5cm of Req] (Des) {System\\Design};
    \node [below right=0.5cm and 0.5cm of Des] (Imp) {Implementation};
    \node [below right=0.5cm and 0.5cm of Imp] (Test) {Testing};
    \node [below right=0.5cm and 0.5cm of Test] (Dep) {Deployment};
    \node [below right=0.5cm and 0.5cm of Dep] (Main) {Maintenance};

    \path [gtu arrow] (Req) -| (Des);
    \path [gtu arrow] (Des) -| (Imp);
    \path [gtu arrow] (Imp) -| (Test);
    \path [gtu arrow] (Test) -| (Dep);
    \path [gtu arrow] (Dep) -| (Main);
\end{tikzpicture}
\captionof{figure}{Waterfall Model}
\end{center}

\begin{center}
\captionof{table}{Waterfall લક્ષણો}
\begin{tabulary}{\linewidth}{|L|L|}
\hline
\textbf{લક્ષણો} & \textbf{વર્ણન} \\ \hline
\textbf{ક્રમિક} & એક સમયે એક તબક્કો \\ \hline
\textbf{દસ્તાવેજીકરણ આધારિત} & ભારે દસ્તાવેજીકરણ \\ \hline
\textbf{યોગ્ય} & સ્પષ્ટ આવશ્યકતાઓ માટે \\ \hline
\end{tabulary}
\end{center}

\textbf{(II) Incremental Model:}
નાના increments માં વિકાસ જ્યાં દરેક increment કાર્યક્ષમતા ઉમેરે છે.

\begin{center}
\begin{tikzpicture}[node distance=1.2cm, auto, every node/.style={gtu block, align=center, font=\footnotesize}]
    \node (Inc1) {Increment 1};
    \node [right=of Inc1] (Inc2) {Increment 2};
    \node [right=of Inc2] (IncN) {Increment N};
    \node [below=of Inc2] (Final) {Final Product};

    \node [above=0.5cm of Inc1, gtu state] (Des1) {Design};
    \node [above=0.5cm of Inc2, gtu state] (Des2) {Design};
    \node [above=0.5cm of IncN, gtu state] (DesN) {Design};
    
    \draw [->] (Des1) -- (Inc1);
    \draw [->] (Des2) -- (Inc2);
    \draw [->] (DesN) -- (IncN);
    
    \draw [->] (Inc1) -- (Final);
    \draw [->] (Inc2) -- (Final);
    \draw [->] (IncN) -- (Final);
\end{tikzpicture}
\captionof{figure}{Incremental Model Concept}
\end{center}

\begin{center}
\captionof{table}{ઉપયોગ સરખામણી}
\begin{tabulary}{\linewidth}{|L|L|L|}
\hline
\textbf{લક્ષણ} & \textbf{Waterfall} & \textbf{Incremental} \\ \hline
\textbf{લવચીકતા} & ઓછી & વધુ \\ \hline
\textbf{જોખમ} & વધુ & ઓછું \\ \hline
\textbf{ડિલિવરી} & પ્રોજેક્ટના અંતે & બહુવિધ ડિલિવરીઓ \\ \hline
\end{tabulary}
\end{center}
\end{solutionbox}

\begin{mnemonicbox}
\mnemonic{વોટર એકવાર પડે, ઇન્ક્રિમેન્ટ બહુવિધ બનાવે}
\end{mnemonicbox}

\questionmarks{2(a OR)}{3}{Describe advantage and disadvantage of Incremental Model.}

\begin{solutionbox}
\textbf{Incremental Model વિશ્લેષણ:}
\begin{center}
\captionof{table}{Incremental Model ફાયદા/નુકસાન}
\begin{tabulary}{\linewidth}{|L|L|}
\hline
\textbf{ફાયદા} & \textbf{નુકસાન} \\ \hline
\textbf{વહેલી ડિલિવરી} કાર્યશીલ સૉફ્ટવેરની & \textbf{કુલ ખર્ચ} વધુ હોઈ શકે \\ \hline
\textbf{સરળ ટેસ્ટિંગ} નાના increments ની & \textbf{સિસ્ટમ આર્કિટેક્ચર} સમસ્યાઓ \\ \hline
\textbf{ઓછું જોખમ} વહેલા ફીડબેક દ્વારા & \textbf{મેનેજમેન્ટ જટિલતા} વધે છે \\ \hline
\end{tabulary}
\end{center}
\end{solutionbox}

\begin{mnemonicbox}
\mnemonic{વહેલી સરળ ઓછું વિરુદ્ધ કુલ સિસ્ટમ મેનેજમેન્ટ}
\end{mnemonicbox}

\questionmarks{2(b OR)}{4}{Write concept of Rapid Application Development (RAD) and explain it.}

\begin{solutionbox}
RAD યોજના અને ટેસ્ટિંગ કરતાં ઝડપી prototyping અને ત્વરિત ફીડબેક પર ભાર મૂકે છે.

\textbf{RAD ઘટકો:}
\begin{center}
\captionof{table}{RAD પ્રક્રિયા}
\begin{tabulary}{\linewidth}{|L|L|L|L|}
\hline
\textbf{તબક્કો} & \textbf{અવધિ} & \textbf{પ્રવૃત્તિઓ} & \textbf{આઉટપુટ} \\ \hline
\textbf{Business Modeling} & ટૂંકી & માહિતી પ્રવાહ નક્કી કરવો & બિઝનેસ આવશ્યકતાઓ \\ \hline
\textbf{Data Modeling} & ટૂંકી & ડેટા ઓબ્જેક્ટ્સ નક્કી કરવા & ડેટા મોડલ્સ \\ \hline
\textbf{Process Modeling} & ટૂંકી & પ્રોસેસિંગ functions નક્કી કરવા & પ્રોસેસ વર્ણનો \\ \hline
\textbf{Application Generation} & ટૂંકી & ટૂલ્સ વાપરીને બનાવવું & કાર્યશીલ એપ્લિકેશન \\ \hline
\textbf{Testing \& Turnover} & ટૂંકી & ટેસ્ટ અને ડિલિવર કરવું & ફાઇનલ સિસ્ટમ \\ \hline
\end{tabulary}
\end{center}

\begin{center}
\begin{tikzpicture}[node distance=1.2cm, auto, every node/.style={gtu state, align=center, font=\small}]
    \node (Bus) {Business\\Modeling};
    \node [right=of Bus] (Data) {Data\\Modeling};
    \node [right=of Data] (Proc) {Process\\Modeling};
    \node [below=of Proc] (App) {Application\\Generation};
    \node [left=of App] (Test) {Testing \&\\Turnover};

    \path [gtu arrow] (Bus) -- (Data);
    \path [gtu arrow] (Data) -- (Proc);
    \path [gtu arrow] (Proc) -- (App);
    \path [gtu arrow] (App) -- (Test);
\end{tikzpicture}
\captionof{figure}{RAD Flow}
\end{center}
\end{solutionbox}

\begin{mnemonicbox}
\mnemonic{બિઝનેસ ડેટા પ્રોસેસ એપ્લિકેશન ટેસ્ટિંગ}
\end{mnemonicbox}

\questionmarks{2(c OR)}{7}{Design and describe Spiral Model and give advantage and disadvantage.}

\begin{solutionbox}
Spiral Model પુનરાવર્તક વિકાસને વ્યવસ્થિત જોખમ વિશ્લેષણ સાથે જોડે છે.

\begin{center}
\begin{tikzpicture}[node distance=2.5cm, auto, every node/.style={gtu state, align=center, font=\small, minimum size=2.5cm}]
    \node (Plan) at (0,3) {Planning};
    \node (Risk) at (3,3) {Risk\\Analysis};
    \node (Eng) at (3,0) {Engineering};
    \node (Eval) at (0,0) {Evaluation};

    \path [gtu arrow] (Plan) -- (Risk);
    \path [gtu arrow] (Risk) -- (Eng);
    \path [gtu arrow] (Eng) -- (Eval);
    \path [gtu arrow] (Eval) -- (Plan);
    
    \node at (1.5,1.5) [font=\large\bfseries] {Spiral};
\end{tikzpicture}
\captionof{figure}{Spiral Model Quadrants}
\end{center}

\textbf{Spiral ચતુર્થાંશ:}
\begin{center}
\captionof{table}{ચતુર્થાંશ વિગતો}
\begin{tabulary}{\linewidth}{|L|L|L|}
\hline
\textbf{ચતુર્થાંશ} & \textbf{પ્રવૃત્તિ} & \textbf{હેતુ} \\ \hline
\textbf{Planning} & લક્ષ્ય સેટિંગ & આવશ્યકતાઓ અને અવરોધો નક્કી કરવા \\ \hline
\textbf{Risk Analysis} & જોખમ મૂલ્યાંકન & જોખમો ઓળખવા અને ઉકેલવા \\ \hline
\textbf{Engineering} & વિકાસ & ઉત્પાદન બનાવવું અને ટેસ્ટ કરવું \\ \hline
\textbf{Evaluation} & ગ્રાહક મૂલ્યાંકન & પરિણામો મૂલ્યાંકન અને આગલા iteration ની યોજના \\ \hline
\end{tabulary}
\end{center}

\textbf{ફાયદા વિરુદ્ધ નુકસાન:}
\begin{center}
\captionof{table}{Spiral ફાયદા/નુકસાન}
\begin{tabulary}{\linewidth}{|L|L|}
\hline
\textbf{ફાયદા} & \textbf{નુકસાન} \\ \hline
\textbf{ઉચ્ચ જોખમ પ્રોજેક્ટ્સ} સારી રીતે હેન્ડલ થાય & \textbf{જટિલ મેનેજમેન્ટ} જરૂરી \\ \hline
\textbf{મોટી} એપ્લિકેશન્સ માટે સારું & \textbf{નાના પ્રોજેક્ટ્સ માટે મોંઘું} \\ \hline
\textbf{ગ્રાહક સામેલ} આખા દરમિયાન & \textbf{જોખમ વિશ્લેષણ કુશળતા} જરૂરી \\ \hline
\end{tabulary}
\end{center}
\end{solutionbox}

\begin{mnemonicbox}
\mnemonic{પ્લાન રિસ્ક એન્જિનિયર ઇવેલ્યુએટ + ઉચ્ચ સારું ગ્રાહક વિરુદ્ધ જટિલ મોંઘું જોખમ}
\end{mnemonicbox}

\questionmarks{3(a)}{3}{Illustrate importance of SRS}

\begin{solutionbox}
SRS (Software Requirements Specification) એ સૉફ્ટવેર ડેવલપમેન્ટ માટે મહત્વપૂર્ણ પાયાનું દસ્તાવેજ છે.

\textbf{મહત્વ કોષ્ટક:}
\begin{center}
\captionof{table}{SRS મહત્વ}
\begin{tabulary}{\linewidth}{|L|L|L|}
\hline
\textbf{પાસું} & \textbf{મહત્વ} & \textbf{ફાયદો} \\ \hline
\textbf{કોમ્યુનિકેશન} & હિસ્સેદારોની સમજ & સ્પષ્ટ અપેક્ષાઓ \\ \hline
\textbf{કરાર} & કાનૂની સમજૂતી & વિવાદ નિરાકરણ \\ \hline
\textbf{ટેસ્ટિંગ આધાર} & ચકાસણી માપદંડ & ગુણવત્તા ખાતરી \\ \hline
\end{tabulary}
\end{center}
\end{solutionbox}

\begin{mnemonicbox}
\mnemonic{કોમ્યુનિકેશન કરાર ટેસ્ટિંગ}
\end{mnemonicbox}

\questionmarks{3(b)}{4}{Specify characteristics of good \& bad SRS}

\begin{solutionbox}
\textbf{SRS ગુણવત્તા લક્ષણો:}
\begin{center}
\captionof{table}{સારો vs ખરાબ SRS}
\begin{tabulary}{\linewidth}{|L|L|}
\hline
\textbf{સારો SRS} & \textbf{ખરાબ SRS} \\ \hline
\textbf{સંપૂર્ણ} - બધી આવશ્યકતાઓ આવરી લેવાયેલ & \textbf{અધૂરો} - આવશ્યકતાઓ ખૂટે છે \\ \hline
\textbf{સુસંગત} - કોઈ વિરોધાભાસ નથી & \textbf{અસંગત} - વિરોધી નિવેદનો \\ \hline
\textbf{અસ્પષ્ટ નહીં} - સ્પષ્ટ અર્થ & \textbf{અસ્પષ્ટ} - બહુવિધ અર્થઘટન \\ \hline
\textbf{ચકાસી શકાય તેવું} - ટેસ્ટ કરી શકાય & \textbf{ચકાસી ન શકાય} - વેલિડેટ કરી શકાતું નથી \\ \hline
\end{tabulary}
\end{center}

\textbf{વધારાના સારા લક્ષણો:}
\begin{itemize}
    \item \keyword{સુધારી શકાય તેવું}: બદલવું અને જાળવવું સરળ
    \item \keyword{ટ્રેસેબલ}: સ્રોત અને ડિઝાઇન સાથે લિંક
\end{itemize}

\begin{center}
\begin{tikzpicture}[
    level 1/.style={sibling distance=4cm},
    level 2/.style={sibling distance=2cm},
    every node/.style={gtu block, align=center, font=\small}
]
    \node {SRS Quality}
        child { node {Good SRS}
            child { node {Complete} }
            child { node {Consistent} }
            child { node {Unambiguous} }
            child { node {Verifiable} }
        }
        child { node {Bad SRS}
            child { node {Incomplete} }
            child { node {Inconsistent} }
            child { node {Ambiguous} }
            child { node {Unverifiable} }
        };
\end{tikzpicture}
\captionof{figure}{SRS લક્ષણો}
\end{center}
\end{solutionbox}

\begin{mnemonicbox}
\mnemonic{સંપૂર્ણ સુસંગત અસ્પષ્ટ-ન ચકાસી-શકાય વિરુદ્ધ અધૂરો અસંગત અસ્પષ્ટ ચકાસી-ન-શકાય}
\end{mnemonicbox}

\questionmarks{3(c)}{7}{Classify Types of Requirements in SRS}

\begin{solutionbox}
સૉફ્ટવેર આવશ્યકતાઓને બે મુખ્ય શ્રેણીઓમાં વર્ગીકૃત કરવામાં આવે છે.

\textbf{(i) Functional Requirements:}
સિસ્ટમે શું કરવું જોઈએ તે નક્કી કરે છે - વિશિષ્ટ વર્તણૂકો અને કાર્યો.
\begin{center}
\captionof{table}{Functional Requirements}
\begin{tabulary}{\linewidth}{|L|L|L|}
\hline
\textbf{પ્રકાર} & \textbf{વર્ણન} & \textbf{ઉદાહરણ} \\ \hline
\textbf{બિઝનેસ નિયમો} & મુખ્ય બિઝનેસ લોજિક & "આવકના સ્લેબ મુજબ ટેક્સ ગણતરી કરવી" \\ \hline
\textbf{યુઝર એક્શન્સ} & સિસ્ટમ પ્રતિભાવો & "યુઝરનેમ/પાસવર્ડ સાથે લોગિન" \\ \hline
\textbf{ડેટા પ્રોસેસિંગ} & માહિતી હેન્ડલિંગ & "માસિક વેચાણ રિપોર્ટ જનરેટ કરવી" \\ \hline
\textbf{એક્સટર્નલ ઇન્ટરફેસ} & સિસ્ટમ ક્રિયાપ્રતિક્રિયાઓ & "પેમેન્ટ ગેટવે સાથે કનેક્ટ કરવું" \\ \hline
\end{tabulary}
\end{center}

\textbf{(ii) Non-functional Requirements:}
સિસ્ટમે કેવી રીતે પ્રદર્શન કરવું જોઈએ તે નક્કી કરે છે - ગુણવત્તા લક્ષણો અને મર્યાદાઓ.
\begin{center}
\captionof{table}{Non-functional Requirements}
\begin{tabulary}{\linewidth}{|L|L|L|L|}
\hline
\textbf{શ્રેણી} & \textbf{આવશ્યકતા} & \textbf{ઉદાહરણ} & \textbf{માપદંડ} \\ \hline
\textbf{પ્રદર્શન} & પ્રતિભાવ સમય & "પેજ લોડ < 3 સેકન્ડ" & સમય મેટ્રિક્સ \\ \hline
\textbf{સુરક્ષા} & ડેટા સુરક્ષા & "યુઝર પાસવર્ડ એન્ક્રિપ્ટ કરવા" & સુરક્ષા ધોરણો \\ \hline
\textbf{વિશ્વસનીયતા} & સિસ્ટમ અપટાઇમ & "99.9\% ઉપલબ્ધતા" & નિષ્ફળતા દરો \\ \hline
\textbf{ઉપયોગિતા} & યુઝર અનુભવ & "ચેકઆઉટ માટે મહત્તમ 3 ક્લિક" & યુઝર મેટ્રિક્સ \\ \hline
\textbf{સ્કેલેબિલિટી} & વૃદ્ધિ ક્ષમતા & "10,000 યુઝર્સ સપોર્ટ કરવા" & લોડ ક્ષમતા \\ \hline
\end{tabulary}
\end{center}

\begin{center}
\begin{tikzpicture}[
    mindmap,
    concept color=blue!30,
    every node/.style={concept, execute at begin node=\hskip0pt},
    root concept/.append style={concept, color=blue!50, minimum size=3cm, font=\bfseries},
    level 1 concept/.append style={level distance=4.5cm, sibling angle=90, color=blue!20}
]
    \node [root concept] {Requirements}
        child { node {Functional} 
            child { node {Business\\Rules} }
            child { node {User\\Actions} }
        }
        child { node {Non-Functional} 
            child { node {Performance} }
            child { node {Security} }
            child { node {Usability} }
        };
\end{tikzpicture}
\captionof{figure}{Requirement Types}
\end{center}
\end{solutionbox}

\begin{mnemonicbox}
\mnemonic{Functional = શું, Non-Functional = કેવી રીતે}
\end{mnemonicbox}

\questionmarks{3(a OR)}{3}{Describe skill to manage software projects}

\begin{solutionbox}
પ્રોજેક્ટ મેનેજમેન્ટ માટે સફળ સૉફ્ટવેર ડિલિવરી માટે વિવિધ કુશળતાઓની જરૂર છે.

\textbf{આવશ્યક કુશળતાઓ:}
\begin{center}
\captionof{table}{PM Skills}
\begin{tabulary}{\linewidth}{|L|L|L|}
\hline
\textbf{કુશળતા શ્રેણી} & \textbf{વર્ણન} & \textbf{ઉપયોગ} \\ \hline
\textbf{ટેકનિકલ} & ટેકનોલોજીની સમજ & આર્કિટેક્ચર નિર્ણયો \\ \hline
\textbf{નેતૃત્વ} & ટીમ પ્રેરણા & સંઘર્ષ નિરાકરણ \\ \hline
\textbf{કોમ્યુનિકેશન} & હિસ્સેદાર ક્રિયાપ્રતિક્રિયા & સ્થિતિ રિપોર્ટિંગ \\ \hline
\end{tabulary}
\end{center}
\end{solutionbox}

\begin{mnemonicbox}
\mnemonic{ટેકનિકલ નેતૃત્વ કોમ્યુનિકેશન}
\end{mnemonicbox}

\questionmarks{3(b OR)}{4}{Briefly give the Responsibility of software project Manager.}

\begin{solutionbox}
સૉફ્ટવેર પ્રોજેક્ટ મેનેજર સમગ્ર પ્રોજેક્ટ લાઇફસાઇકલની દેખરેખ રાખે છે અને સફળ ડિલિવરી સુનિશ્ચિત કરે છે.

\textbf{મુખ્ય જવાબદારીઓ:}
\begin{center}
\captionof{table}{PM જવાબદારીઓ}
\begin{tabulary}{\linewidth}{|L|L|L|}
\hline
\textbf{ક્ષેત્ર} & \textbf{જવાબદારી} & \textbf{પ્રવૃત્તિઓ} \\ \hline
\textbf{પ્લાનિંગ} & પ્રોજેક્ટ રોડમેપ & શેડ્યૂલ, બજેટ, સંસાધન ફાળવણી \\ \hline
\textbf{એક્ઝિક્યુશન} & ટીમ સંકલન & કાર્ય સોંપણી, પ્રગતિ નિરીક્ષણ \\ \hline
\textbf{ગુણવત્તા} & ધોરણ પાલન & કોડ રિવ્યુ, ટેસ્ટિંગ દેખરેખ \\ \hline
\textbf{કોમ્યુનિકેશન} & હિસ્સેદાર અપડેટ્સ & સ્થિતિ રિપોર્ટ્સ, જોખમ કોમ્યુનિકેશન \\ \hline
\end{tabulary}
\end{center}

\textbf{વધારાની ફરજો:}
\begin{itemize}
    \item \keyword{જોખમ વ્યવસ્થાપન}: પ્રોજેક્ટ જોખમો ઓળખવા અને ઘટાડવા
    \item \keyword{ટીમ ડેવલપમેન્ટ}: ટીમ સભ્યોને માર્ગદર્શન અને સંઘર્ષ નિરાકરણ
\end{itemize}

\begin{center}
\begin{tikzpicture}[node distance=1.5cm, auto, every node/.style={gtu block, align=center, font=\small}]
    \node (PM) [fill=yellow!20] {Project Manager};
    \node [above=of PM] (Plan) {Planning};
    \node [right=of PM] (Exec) {Execution};
    \node [below=of PM] (Qual) {Quality};
    \node [left=of PM] (Comm) {Communication};

    \path [gtu arrow] (PM) -- (Plan);
    \path [gtu arrow] (PM) -- (Exec);
    \path [gtu arrow] (PM) -- (Qual);
    \path [gtu arrow] (PM) -- (Comm);
\end{tikzpicture}
\captionof{figure}{PM Functions}
\end{center}
\end{solutionbox}

\begin{mnemonicbox}
\mnemonic{પ્લાન એક્ઝિક્યુટ ગુણવત્તા કોમ્યુનિકેટ જોખમ ટીમ}
\end{mnemonicbox}

\questionmarks{3(c OR)}{7}{Compare PERT chart – Gantt chart side by side.}

\begin{solutionbox}
બંને ચાર્ટ પ્રોજેક્ટ મેનેજમેન્ટ ટૂલ્સ છે પરંતુ વિવિધ હેતુઓ સેવે છે અને અલગ લક્ષણો ધરાવે છે.

\textbf{વિગતવાર સરખામણી:}
\begin{center}
\captionof{table}{PERT vs Gantt}
\begin{tabulary}{\linewidth}{|L|L|L|}
\hline
\textbf{પાસું} & \textbf{PERT Chart} & \textbf{Gantt Chart} \\ \hline
\textbf{હેતુ} & કાર્ય અવલંબન દર્શાવવું & પ્રોજેક્ટ ટાઇમલાઇન બતાવવું \\ \hline
\textbf{માળખું} & નેટવર્ક ડાયાગ્રામ & બાર ચાર્ટ \\ \hline
\textbf{ધ્યાન} & ક્રિટિકલ પાથ વિશ્લેષણ & શેડ્યૂલ વિઝ્યુઅલાઇઝેશન \\ \hline
\textbf{સમય પ્રદર્શન} & અંદાજિત અવધિ & વાસ્તવિક તારીખો \\ \hline
\textbf{અવલંબન} & સ્પષ્ટ તીરો & ગર્ભિત જોડાણો \\ \hline
\textbf{શ્રેષ્ઠ} & જટિલ પ્રોજેક્ટ્સ & સરળ શેડ્યૂલિંગ \\ \hline
\end{tabulary}
\end{center}

\textbf{વિઝ્યુઅલ રિપ્રેઝન્ટેશન:}
\begin{center}
\begin{tikzpicture}[node distance=1.5cm, auto, every node/.style={circle, draw, font=\small}]
    \node (A) {A};
    \node (B) [below=of A] {B};
    \node (C) [right=of A] {C};
    \node (D) [right=of C] {D};
    
    \path [gtu arrow] (A) -- (C);
    \path [gtu arrow] (B) -- (C);
    \path [gtu arrow] (C) -- (D);
\end{tikzpicture}
\captionof{figure}{PERT Chart ખ્યાલ}
\end{center}

\begin{center}
\begin{tikzpicture}
    % Gantt chart approximate
    \draw[->] (0,0) -- (6,0) node[right] {Time};
    \draw[->] (0,0) -- (0,3) node[above] {Tasks};
    
    \draw[fill=blue!30] (0.5, 2.5) rectangle (2.5, 2.8) node[midway] {Task A};
    \draw[fill=blue!30] (0.5, 2.0) rectangle (2.0, 2.3) node[midway] {Task B};
    \draw[fill=blue!30] (2.5, 1.5) rectangle (4.5, 1.8) node[midway] {Task C};
    \draw[fill=blue!30] (4.5, 1.0) rectangle (5.5, 1.3) node[midway] {Task D};
\end{tikzpicture}
\captionof{figure}{Gantt Chart ખ્યાલ}
\end{center}

\textbf{ક્યારે ઉપયોગ કરવો:}
\begin{center}
\captionof{table}{ઉપયોગ માર્ગદર્શિકા}
\begin{tabulary}{\linewidth}{|L|L|L|}
\hline
\textbf{સ્થિતિ} & \textbf{PERT} & \textbf{Gantt} \\ \hline
\textbf{પ્રોજેક્ટ પ્રકાર} & સંશોધન અને વિકાસ & બાંધકામ, સૉફ્ટવેર \\ \hline
\textbf{અનિશ્ચિતતા} & ઉચ્ચ અનિશ્ચિતતા & સ્પષ્ટ કાર્યો \\ \hline
\textbf{પ્રેક્ષકો} & ટેકનિકલ ટીમ & મેનેજમેન્ટ, ક્લાયન્ટ્સ \\ \hline
\end{tabulary}
\end{center}

\textbf{ફાયદાઓની સરખામણી:}
\begin{itemize}
    \item \textbf{PERT}: ક્રિટિકલ પાથ, લવચીક સમય, જોખમ વિશ્લેષણ
    \item \textbf{Gantt}: સમજવામાં સરળ, પ્રગતિ ટ્રેકિંગ, સંસાધન ફાળવણી
\end{itemize}
\end{solutionbox}

\begin{mnemonicbox}
\mnemonic{PERT = પાથ, Gantt = બાર્સ}
\end{mnemonicbox}

\questionmarks{4(a)}{3}{Give steps of Project Monitoring and control process}

\begin{solutionbox}
પ્રોજેક્ટ મોનિટરિંગ વ્યવસ્થિત નિરીક્ષણ અને સુધારાત્મક ક્રિયાઓ દ્વારા પ્રોજેક્ટ ટ્રેક પર રહે તેની ખાતરી કરે છે.

\textbf{મોનિટરિંગ પગલાં:}
\begin{center}
\captionof{table}{પ્રક્રિયાના પગલાં}
\begin{tabulary}{\linewidth}{|L|L|L|}
\hline
\textbf{પગલું} & \textbf{પ્રવૃત્તિ} & \textbf{હેતુ} \\ \hline
\textbf{પ્રગતિ ટ્રેક કરવી} & વાસ્તવિક વિરુદ્ધ આયોજિત માપવું & વિચલનો ઓળખવા \\ \hline
\textbf{ગુણવત્તા મૂલ્યાંકન} & ડિલિવરેબલ્સ સમીક્ષા & ધોરણો સુનિશ્ચિત કરવા \\ \hline
\textbf{પગલાં લેવા} & સુધારાઓ લાગુ કરવા & સંરેખણ જાળવવા \\ \hline
\end{tabulary}
\end{center}
\end{solutionbox}

\begin{mnemonicbox}
\mnemonic{ટ્રેક મૂલ્યાંકન પગલાં}
\end{mnemonicbox}

\questionmarks{4(b)}{4}{Discuss i)Risk Assessment ii)Risk Mitigation}

\begin{solutionbox}
\textbf{(i) Risk Assessment:}
સંભવિત પ્રોજેક્ટ જોખમો ઓળખવા અને મૂલ્યાંકન કરવાની પ્રક્રિયા.
\begin{center}
\captionof{table}{મૂલ્યાંકન ઘટકો}
\begin{tabulary}{\linewidth}{|L|L|L|}
\hline
\textbf{મૂલ્યાંકન પ્રકાર} & \textbf{પદ્ધતિ} & \textbf{આઉટપુટ} \\ \hline
\textbf{જોખમ ઓળખ} & બ્રેઇનસ્ટોર્મિંગ, ચેકલિસ્ટ્સ & જોખમ સૂચિ \\ \hline
\textbf{જોખમ વિશ્લેષણ} & સંભાવના $\times$ પ્રભાવ & જોખમ પ્રાથમિકતા \\ \hline
\textbf{જોખમ મૂલ્યાંકન} & જોખમ મેટ્રિક્સ & કાર્ય પ્રાથમિકતાઓ \\ \hline
\end{tabulary}
\end{center}

\textbf{(ii) Risk Mitigation:}
જોખમની અસર અને સંભાવના ઘટાડવાની વ્યૂહરચનાઓ.
\begin{center}
\captionof{table}{ઘટાડવાની વ્યૂહરચનાઓ}
\begin{tabulary}{\linewidth}{|L|L|L|}
\hline
\textbf{વ્યૂહરચના} & \textbf{વર્ણન} & \textbf{ઉદાહરણ} \\ \hline
\textbf{ટાળવું} & જોખમ સ્રોત દૂર કરવો & ટેકનોલોજી બદલવી \\ \hline
\textbf{ઘટાડવું} & અસર ઓછી કરવી & ટેસ્ટિંગ ઉમેરવું \\ \hline
\textbf{ટ્રાન્સફર કરવું} & અન્યને જોખમ સ્થાનાંતરિત કરવું & વીમો, આઉટસોર્સિંગ \\ \hline
\textbf{સ્વીકારવું} & જોખમ સાથે જીવવું & કન્ટિન્જન્સી પ્લાનિંગ \\ \hline
\end{tabulary}
\end{center}
\end{solutionbox}

\begin{mnemonicbox}
\mnemonic{ટાળો ઘટાડો ટ્રાન્સફર સ્વીકારો}
\end{mnemonicbox}

\questionmarks{4(c)}{7}{Define project risk and how Manage Risk Management it.}

\begin{solutionbox}
પ્રોજેક્ટ જોખમ એ અનિશ્ચિત ઘટના છે જે, જો થાય તો, પ્રોજેક્ટ લક્ષ્યો પર સકારાત્મક અથવા નકારાત્મક અસર કરે છે.

\textbf{જોખમ લક્ષણો:}
\begin{center}
\captionof{table}{લક્ષણો}
\begin{tabulary}{\linewidth}{|L|L|L|}
\hline
\textbf{લક્ષણ} & \textbf{વર્ણન} & \textbf{ઉદાહરણ} \\ \hline
\textbf{અનિશ્ચિતતા} & થઈ શકે અથવા ન પણ થાય & ટેકનોલોજી નિષ્ફળતા \\ \hline
\textbf{પ્રભાવ} & પ્રોજેક્ટ પેરામીટર્સને અસર કરે & ખર્ચ, શેડ્યૂલ, ગુણવત્તા \\ \hline
\textbf{સંભાવના} & થવાની શક્યતા & 30\% વિલંબની તક \\ \hline
\end{tabulary}
\end{center}

\textbf{જોખમ વ્યવસ્થાપન પ્રક્રિયા:}
\begin{center}
\begin{tikzpicture}[node distance=1.5cm, auto, every node/.style={gtu state, align=center, font=\footnotesize}]
    \node (Ident) {Identify};
    \node [right=of Ident] (Assess) {Assess};
    \node [right=of Assess] (Prior) {Prioritize};
    \node [below=of Prior] (Resp) {Response};
    \node [left=of Resp] (Mon) {Monitor};
    \node [left=of Mon] (Cont) {Control};

    \path [gtu arrow] (Ident) -- (Assess);
    \path [gtu arrow] (Assess) -- (Prior);
    \path [gtu arrow] (Prior) -- (Resp);
    \path [gtu arrow] (Resp) -- (Mon);
    \path [gtu arrow] (Mon) -- (Cont);
    \path [gtu arrow] (Cont) -- (Ident);
\end{tikzpicture}
\captionof{figure}{Management Loop}
\end{center}

\textbf{જોખમ વ્યવસ્થાપન પગલાં:}
\begin{center}
\captionof{table}{પ્રક્રિયા વિગતો}
\begin{tabulary}{\linewidth}{|L|L|L|L|}
\hline
\textbf{પગલું} & \textbf{પ્રવૃત્તિઓ} & \textbf{ટૂલ્સ} & \textbf{આઉટપુટ} \\ \hline
\textbf{જોખમ ઓળખ} & બ્રેઇનસ્ટોર્મિંગ & ચેકલિસ્ટ્સ, SWOT & જોખમ રજિસ્ટર \\ \hline
\textbf{જોખમ મૂલ્યાંકન} & સંભાવના/પ્રભાવ વિશ્લેષણ & જોખમ મેટ્રિક્સ & જોખમ રેટિંગ્સ \\ \hline
\textbf{જોખમ પ્રતિભાવ} & વ્યૂહરચના વિકસાવવી & પ્રતિભાવ ટેમ્પ્લેટ્સ & કાર્ય યોજનાઓ \\ \hline
\textbf{જોખમ મોનિટરિંગ} & સૂચકો ટ્રેક કરવા & ડેશબોર્ડ્સ & સ્થિતિ રિપોર્ટ્સ \\ \hline
\end{tabulary}
\end{center}

\textbf{જોખમ પ્રતિભાવ વ્યૂહરચનાઓ:}
\begin{itemize}
    \item \keyword{નકારાત્મક જોખમો}: ટાળવું, ટ્રાન્સફર કરવું, ઘટાડવું, સ્વીકારવું
    \item \keyword{સકારાત્મક જોખમો}: શોષણ કરવું, શેર કરવું, વધારવું, સ્વીકારવું
\end{itemize}
\end{solutionbox}

\begin{mnemonicbox}
\mnemonic{ઓળખો મૂલ્યાંકન પ્રતિભાવ મોનિટર + ટાળો ટ્રાન્સફર ઘટાડો સ્વીકારો}
\end{mnemonicbox}

\questionmarks{4(a OR)}{3}{Describe Software design process and explain Design methodologies.}

\begin{solutionbox}
સૉફ્ટવેર ડિઝાઇન આવશ્યકતાઓને વ્યવસ્થિત અભિગમ દ્વારા અમલીકરણ માટે બ્લુપ્રિન્ટમાં રૂપાંતરિત કરે છે.

\textbf{ડિઝાઇન પ્રક્રિયા:}
\begin{center}
\captionof{table}{પ્રક્રિયા માળખું}
\begin{tabulary}{\linewidth}{|L|L|L|}
\hline
\textbf{તબક્કો} & \textbf{પ્રવૃત્તિ} & \textbf{આઉટપુટ} \\ \hline
\textbf{વિશ્લેષણ} & આવશ્યકતાઓ સમજવી & સમસ્યા વ્યાખ્યા \\ \hline
\textbf{આર્કિટેક્ચર} & ઉચ્ચ-સ્તરીય માળખું & સિસ્ટમ આર્કિટેક્ચર \\ \hline
\textbf{વિગતવાર ડિઝાઇન} & ઘટક સ્પષ્ટીકરણ & ડિઝાઇન દસ્તાવેજો \\ \hline
\end{tabulary}
\end{center}
\end{solutionbox}

\begin{mnemonicbox}
\mnemonic{વિશ્લેષણ આર્કિટેક્ચર વિગત}
\end{mnemonicbox}

\questionmarks{4(b OR)}{4}{Compare Cohesion and Coupling side by side.}

\begin{solutionbox}
બંને ખ્યાલો મોડ્યુલ ડિઝાઇન ગુણવત્તા માપે છે પરંતુ વિવિધ પાસાઓ પર ધ્યાન કેન્દ્રિત કરે છે.

\textbf{વ્યાપક સરખામણી:}
\begin{center}
\captionof{table}{Cohesion vs Coupling}
\begin{tabulary}{\linewidth}{|L|L|L|}
\hline
\textbf{પાસું} & \textbf{Cohesion} & \textbf{Coupling} \\ \hline
\textbf{વ્યાખ્યા} & મોડ્યુલની અંદર સંબંધની ડિગ્રી & મોડ્યુલો વચ્ચે પરસ્પર નિર્ભરતાની ડિગ્રી \\ \hline
\textbf{લક્ષ્ય} & ઉચ્ચ cohesion ઇચ્છનીય & નીચું coupling ઇચ્છનીય \\ \hline
\textbf{ધ્યાન} & આંતરિક મોડ્યુલ માળખું & આંતર-મોડ્યુલ સંબંધો \\ \hline
\textbf{ગુણવત્તા} & મજબૂત = બહેતર & નબળું = બહેતર \\ \hline
\end{tabulary}
\end{center}

\textbf{પ્રકારોની સરખામણી (શ્રેષ્ઠથી ખરાબ):}
\begin{itemize}
    \item \keyword{Cohesion}: Functional, Sequential, Communicational, Procedural, Temporal, Logical, Coincidental
    \item \keyword{Coupling}: Data, Stamp, Control, External, Common, Content
\end{itemize}

\textbf{ડિઝાઇન પર પ્રભાવ:}
ઉચ્ચ cohesion અને નીચું coupling વધુ સારી જાળવણીક્ષમતા, પુનઃઉપયોગ, અને ટેસ્ટિંગ તરફ દોરી જાય છે.
\end{solutionbox}

\begin{mnemonicbox}
\mnemonic{Cohesion = અંદર મજબૂત, Coupling = વચ્ચે નબળું}
\end{mnemonicbox}

\questionmarks{4(c OR)}{7}{Sketch Data Flow Diagram with levels and explain.}

\begin{solutionbox}
ડેટા ફ્લો ડાયાગ્રામ (DFD) ગ્રાફિકલ નોટેશન વાપરીને સિસ્ટમ દ્વારા ડેટા કેવી રીતે ચાલે છે તે બતાવે છે અને વિગતના બહુવિધ સ્તરો ધરાવે છે.

\textbf{DFD સ્તરો:}
\begin{center}
\begin{tikzpicture}[node distance=1.5cm, auto, every node/.style={gtu block, align=center, font=\footnotesize}]
    \node (L0) {Level 0\\Context Diagram};
    \node [right=of L0] (L1) {Level 1\\Major Processes};
    \node [right=of L1] (L2) {Level 2\\Sub-processes};
    \node [right=of L2] (L3) {Level 3\\Detailed};

    \path [gtu arrow] (L0) -- (L1);
    \path [gtu arrow] (L1) -- (L2);
    \path [gtu arrow] (L2) -- (L3);
\end{tikzpicture}
\captionof{figure}{DFD Levels}
\end{center}

\textbf{સ્તર વર્ણનો:}
\begin{center}
\captionof{table}{સ્તર વિગતો}
\begin{tabulary}{\linewidth}{|L|L|L|L|}
\hline
\textbf{સ્તર} & \textbf{અવકાશ} & \textbf{હેતુ} & \textbf{વિગત} \\ \hline
\textbf{Level 0} & સંપૂર્ણ સિસ્ટમ & સિસ્ટમ સીમા & એક પ્રોસેસ \\ \hline
\textbf{Level 1} & મુખ્ય કાર્યો & ઉચ્ચ-સ્તરીય પ્રોસેસો & 5-7 પ્રોસેસો \\ \hline
\textbf{Level 2} & ઉપ-કાર્યો & પ્રોસેસ વિભાજન & વિગતવાર દૃશ્ય \\ \hline
\textbf{Level 3+} & બારીક વિગતો & અમલીકરણ સ્તર & ખૂબ જ વિશિષ્ટ \\ \hline
\end{tabulary}
\end{center}

\textbf{ઉદાહરણ - Level 1 DFD:}
\begin{center}
\begin{tikzpicture}[node distance=1.5cm, auto, every node/.style={font=\footnotesize}]
    \node (Stu) [draw, rectangle] {Student};
    \node (P1) [draw, circle, right=of Stu] {1.0 Register};
    \node (DB) [draw, cylinder, shape border rotate=90, aspect=0.25, right=of P1] {Student DB};
    \node (P2) [draw, circle, right=of DB] {2.0 Report};
    \node (Adm) [draw, rectangle, right=of P2] {Admin};

    \path [gtu arrow] (Stu) -- (P1);
    \path [gtu arrow] (P1) -- (DB);
    \path [gtu arrow] (DB) -- (P2);
    \path [gtu arrow] (P2) -- (Adm);
\end{tikzpicture}
\captionof{figure}{Level 1 ઉદાહરણ}
\end{center}

\textbf{ફાયદા:}
અમૂર્તતા, વિઘટન, ચકાસણી.
\end{solutionbox}

\begin{mnemonicbox}
\mnemonic{Context મુખ્ય ઉપ બારીક + પ્રોસેસ એન્ટિટી સ્ટોર ફ્લો}
\end{mnemonicbox}

\questionmarks{5(a)}{3}{Give Characteristics of good UI.}

\begin{solutionbox}
સારો યુઝર ઇન્ટરફેસ ડિઝાઇન સૉફ્ટવેર સિસ્ટમ સાથે અસરકારક યુઝર ક્રિયાપ્રતિક્રિયા સુનિશ્ચિત કરે છે.

\textbf{UI લાક્ષણિકતાઓ:}
\begin{center}
\captionof{table}{મુખ્ય લક્ષણો}
\begin{tabulary}{\linewidth}{|L|L|L|}
\hline
\textbf{લાક્ષણિકતા} & \textbf{વર્ણન} & \textbf{ફાયદો} \\ \hline
\textbf{સરળ} & સમજવામાં સરળ & શીખવાની વળાંક ઘટાડે છે \\ \hline
\textbf{સુસંગત} & એકસમાન વર્તન & અનુમાનિત ક્રિયાપ્રતિક્રિયા \\ \hline
\textbf{પ્રતિસાદશીલ} & ઝડપી ફીડબેક & યુઝર સંતુષ્ટતા \\ \hline
\end{tabulary}
\end{center}
\end{solutionbox}

\begin{mnemonicbox}
\mnemonic{સરળ સુસંગત પ્રતિસાદશીલ}
\end{mnemonicbox}

\questionmarks{5(b)}{4}{Briefly explain Unit testing}

\begin{solutionbox}
યુનિટ ટેસ્ટિંગ સાચી કાર્યક્ષમતા સુનિશ્ચિત કરવા માટે વ્યક્તિગત સૉફ્ટવેર ઘટકોને અલગતામાં ચકાસે છે.

\textbf{યુનિટ ટેસ્ટિંગ ઝાંખી:}
\begin{center}
\captionof{table}{ટેસ્ટિંગ અવકાશ}
\begin{tabulary}{\linewidth}{|L|L|L|}
\hline
\textbf{પાસું} & \textbf{વર્ણન} & \textbf{હેતુ} \\ \hline
\textbf{અવકાશ} & વ્યક્તિગત મોડ્યુલ્સ & ઘટક ચકાસણી \\ \hline
\textbf{અલગતા} & અલગતામાં ટેસ્ટ & સ્વતંત્ર ચકાસણી \\ \hline
\textbf{સ્વચાલન} & સ્વચાલિત અમલીકરણ & કાર્યક્ષમ ટેસ્ટિંગ \\ \hline
\textbf{વહેલી શોધ} & વહેલો બગ શોધ & ખર્ચ-અસરકારક \\ \hline
\end{tabulary}
\end{center}

\begin{center}
\begin{tikzpicture}[node distance=1.2cm, auto, every node/.style={gtu block, align=center, font=\small}]
    \node (Test) {Write Test Cases};
    \node [right=of Test] (Exec) {Execute Tests};
    \node [right=of Exec] (Res) {Analyze Results};
    \node [right=of Res] (Fix) {Fix Defects};
    
    \path [gtu arrow] (Test) -- (Exec);
    \path [gtu arrow] (Exec) -- (Res);
    \path [gtu arrow] (Res) -- (Fix);
    \path [gtu arrow] (Fix) to [bend left] (Exec);
\end{tikzpicture}
\captionof{figure}{Unit Testing Cycle}
\end{center}

\textbf{ફાયદા:}
વહેલી બગ શોધ, કોડ ગુણવત્તા સુધારણા, રિગ્રેશન ટેસ્ટિંગ.
\end{solutionbox}

\begin{mnemonicbox}
\mnemonic{અવકાશ અલગતા સ્વચાલન વહેલી}
\end{mnemonicbox}

\questionmarks{5(c)}{7}{Draw activity diagrams of the train reservation system, explain each step.}

\begin{solutionbox}
Activity Diagram યુઝર વિનંતીથી ટિકિટ પુષ્ટિ સુધી ટ્રેન રિઝર્વેશન સિસ્ટમનો વર્કફ્લો બતાવે છે.

\begin{center}
\begin{tikzpicture}[node distance=1.5cm, auto, every node/.style={gtu block, align=center, font=\footnotesize}]
    \node (Start) [gtu start] {Start};
    \node (Login) [below=0.8cm of Start] {User Login};
    \node (Cred) [gtu decision, below=0.8cm of Login] {Valid?};
    \node (Search) [below=0.8cm of Cred] {Search Trains};
    \node (Select) [below=0.8cm of Search] {Select Train};
    \node (Details) [right=of Search] {Enter Details};
    \node (Review) [right=of Details] {Review Booking};
    \node (Pay) [below=0.8cm of Review] {Process Payment};
    \node (Success) [gtu decision, below=0.8cm of Pay] {Success?};
    \node (Gen) [below=0.8cm of Success] {Generate Ticket};
    \node (End) [gtu stop, below=0.8cm of Gen] {End};

    \path [gtu arrow] (Start) -- (Login);
    \path [gtu arrow] (Login) -- (Cred);
    \path [gtu arrow] (Cred) -- node[right] {Yes} (Search);
    \path [gtu arrow] (Cred) to [bend right] node[left] {No} (Login);
    \path [gtu arrow] (Search) -- (Select);
    \path [gtu arrow] (Select) -| (Details);
    \path [gtu arrow] (Details) -- (Review);
    \path [gtu arrow] (Review) -- (Pay);
    \path [gtu arrow] (Pay) -- (Success);
    \path [gtu arrow] (Success) -- node[right] {Yes} (Gen);
    \path [gtu arrow] (Success) to [bend right] node[left] {No} (Pay);
    \path [gtu arrow] (Gen) -- (End);
\end{tikzpicture}
\captionof{figure}{Train Reservation Activity}
\end{center}

\textbf{પગલા-દર-પગલાની સમજૂતી:}
\begin{itemize}
    \item \textbf{Login}: યુઝર ઓથેન્ટિકેશન.
    \item \textbf{Search}: રૂટ/તારીખ માટે ટ્રેન શોધવી.
    \item \textbf{Selection}: ટ્રેન અને સીટ પસંદ કરવી.
    \item \textbf{Details}: પેસેન્જર માહિતી દાખલ કરવી.
    \item \textbf{Payment}: ટ્રાન્ઝેક્શન હેન્ડલ કરવું.
    \item \textbf{Ticket}: ટિકિટ જનરેટ અને પુષ્ટિ મોકલવી.
\end{itemize}
\end{solutionbox}

\begin{mnemonicbox}
\mnemonic{લોગિન સર્ચ સિલેક્ટ ચૂઝ એન્ટર રિવ્યુ પે જનરેટ સેન્ડ}
\end{mnemonicbox}

\questionmarks{5(a OR)}{3}{Compare Verification, Validation side by side.}

\begin{solutionbox}
બંને ગુણવત્તા ખાતરીની પ્રવૃત્તિઓ છે પરંતુ વિવિધ પાસાઓ પર ધ્યાન કેન્દ્રિત કરે છે.

\textbf{Verification વિરુદ્ધ Validation:}
\begin{center}
\captionof{table}{સરખામણી}
\begin{tabulary}{\linewidth}{|L|L|L|}
\hline
\textbf{પાસું} & \textbf{Verification} & \textbf{Validation} \\ \hline
\textbf{પ્રશ્ન} & "શું આપણે સાચું બનાવી રહ્યા છીએ?" & "શું આપણે સાચી વસ્તુ બનાવી રહ્યા છીએ?" \\ \hline
\textbf{ધ્યાન} & પ્રક્રિયાની સાચકી & ઉત્પાદનની સાચકી \\ \hline
\textbf{પદ્ધતિ} & સમીક્ષાઓ, નિરીક્ષણો & ટેસ્ટિંગ, યુઝર ફીડબેક \\ \hline
\end{tabulary}
\end{center}
\end{solutionbox}

\begin{mnemonicbox}
\mnemonic{Verification = સાચી પ્રક્રિયા, Validation = સાચું ઉત્પાદન}
\end{mnemonicbox}

\questionmarks{5(b OR)}{4}{Define Testing describe any two testing type.}

\begin{solutionbox}
ટેસ્ટિંગ એ ભૂલો શોધવા અને તે આવશ્યકતાઓ પૂરી કરે છે તે ચકાસવા માટે સૉફ્ટવેરનું મૂલ્યાંકન કરવાની પ્રક્રિયા છે.

\textbf{બે ટેસ્ટિંગ પ્રકારો:}
\begin{center}
\captionof{table}{Black Box vs White Box}
\begin{tabulary}{\linewidth}{|L|L|L|}
\hline
\textbf{પાસું} & \textbf{Black Box} & \textbf{White Box} \\ \hline
\textbf{અભિગમ} & આંતરિક માળખું જાણ્યા વિના & કોડ માળખા જાણ સાથે \\ \hline
\textbf{ધ્યાન} & કાર્યાત્મક આવશ્યકતાઓ & આંતરિક તર્ક \\ \hline
\textbf{ટેસ્ટર} & યુઝર સ્વીકૃતિ & ડેવલપર યુનિટ ટેસ્ટિંગ \\ \hline
\end{tabulary}
\end{center}
\end{solutionbox}

\begin{mnemonicbox}
\mnemonic{Black = બાહ્ય, White = આંતરિક}
\end{mnemonicbox}

\questionmarks{5(c OR)}{7}{Describe each Coding standards and guidelines.}

\begin{solutionbox}
કોડિંગ સ્ટાન્ડર્ડ્સ એ સુસંગત, જાળવી શકાય તેવા કોડ લખવા માટેના નિયમો અને પરંપરાઓનો સમૂહ છે.

\textbf{મુખ્ય શ્રેણીઓ:}
\begin{enumerate}
    \item \textbf{નામકરણ પરંપરાઓ}: વેરિએબલ્સ માટે camelCase, ક્લાસીસ માટે PascalCase.
    \item \textbf{કોડ માળખું}: સુસંગત ઇન્ડેન્ટેશન, લાઇન લંબાઈ મર્યાદા.
    \item \textbf{ગોઠવણી}: એક જવાબદારી, નાના ફંક્શન્સ.
    \item \textbf{દસ્તાવેજીકરણ}: હેડર કોમેન્ટ્સ, અર્થપૂર્ણ ઇનલાઇન કોમેન્ટ્સ.
    \item \textbf{એરર હેન્ડલિંગ}: ગ્રેસફુલ એક્સેપ્શન હેન્ડલિંગ, લોગિંગ.
    \item \textbf{પર્ફોર્મન્સ}: મેમરી લીક્સ ટાળવા, કાર્યક્ષમ એલ્ગોરિધમ્સ.
\end{enumerate}

\begin{center}
\begin{tikzpicture}[
    mindmap,
    concept color=purple!30,
    every node/.style={concept, execute at begin node=\hskip0pt},
    root concept/.append style={concept, color=purple!50, minimum size=3cm, font=\bfseries},
    level 1 concept/.append style={level distance=4cm, sibling angle=60, color=purple!20}
]
    \node [root concept] {Coding\\Standards}
        child { node {Naming} }
        child { node {Structure} }
        child { node {Organization} }
        child { node {Documentation} }
        child { node {Error\\Handling} }
        child { node {Performance} };
\end{tikzpicture}
\captionof{figure}{Standard Categories}
\end{center}

\textbf{ફાયદા:}
બહેતર વાંચી શકાય તેવું, સુસંગતતા, જાળવણીક્ષમતા, અને ગુણવત્તા.
\end{solutionbox}

\begin{mnemonicbox}
\mnemonic{નામ માળખું ગોઠવણી દસ્તાવેજ હેન્ડલ પર્ફોર્મ રિવ્યુ}
\end{mnemonicbox}

\end{document}
