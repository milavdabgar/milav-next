\documentclass[10pt,a4paper]{article}

% content/resources/templates/preamble.tex
\usepackage[margin=0.6in]{geometry}
\author{Milav Dabgar}
\usepackage{amsmath,amssymb,amsthm}
\usepackage{booktabs}
\usepackage{multirow}
\usepackage{xcolor}
\usepackage{tcolorbox}
\tcbuselibrary{breakable,skins}
\usepackage[colorlinks=true,linkcolor=blue]{hyperref}
\usepackage{titlesec}
\usepackage{enumitem}
\usepackage{tikz}
\usepackage{pgfplots}
\usepackage{circuitikz}
\usepackage[version=4]{mhchem}
\usepackage{longtable}
\usepackage{array}
\usepackage{float}
\usepackage{caption}
\usepackage{listings}

\lstset{
  basicstyle=\small\ttfamily,
  breaklines=true,
  breakatwhitespace=false,
  postbreak=\mbox{\textcolor{red}{$\hookrightarrow$}\space},
  float=false,
  numbers=left,
  numberstyle=\tiny\color{gray},
  numbersep=10pt,
  xleftmargin=2em,
  keywordstyle=\color{blue},
  commentstyle=\color{green!60!black},
  stringstyle=\color{purple},
  backgroundcolor=\color{gray!5},
  showstringspaces=false,
  tabsize=2,
  captionpos=b,
  keepspaces=true,
  columns=flexible
}

\pgfplotsset{compat=1.18}
\usetikzlibrary{shapes,arrows,positioning,calc,patterns,decorations.pathmorphing,decorations.markings,arrows.meta}

% Color scheme
\definecolor{headcolor}{RGB}{0,102,204}
\definecolor{keycolor}{RGB}{220,20,60}
\definecolor{solutioncolor}{RGB}{34,139,34}
\definecolor{mnemoniccolor}{RGB}{148,0,211}
\definecolor{codecolor}{RGB}{0,0,100}

% Spacing
\setlength{\parskip}{3pt}
\setlist[itemize]{nosep}
\setlist[enumerate]{nosep}

% Title formatting
\titleformat{\section}{\Large\bfseries\color{headcolor}}{\thesection}{1em}{}
\titleformat{\subsection}{\large\bfseries\color{headcolor}}{\thesubsection}{1em}{}

% Pandoc tightlist compatibility
\providecommand{\tightlist}{%
  \setlength{\itemsep}{0pt}\setlength{\parskip}{0pt}}

% Pandoc longtable compatibility
\newcounter{none}
\def\thenone{}


% content/resources/templates/gujarati-boxes.tex
\usepackage{fontspec}
\usepackage{polyglossia}

% Set Gujarati as main language (document is primarily in Gujarati)
% Note: gloss-gujarati.ldf doesn't exist in polyglossia, but it will use hyphenation patterns
\setdefaultlanguage{gujarati}
\setotherlanguage{english}

% Configure Gujarati font properly
% Use Language=Default to prevent polyglossia from trying to add language-specific features
% that don't exist for Gujarati, which causes "empty feature" warnings
\newfontfamily\gujaratifont[Script=Gujarati,AutoFakeBold=2.5,AutoFakeSlant=0.3]{Noto Sans Gujarati}
\setmainfont[Script=Gujarati,AutoFakeBold=2.5,AutoFakeSlant=0.3]{Noto Sans Gujarati}
% Use Noto Sans Gujarati for monospace to support Gujarati in text
\setmonofont[Scale=0.9]{Noto Sans Gujarati}

% Configure English to use the same font
\newfontfamily\englishfont[Script=Gujarati,AutoFakeBold=2.5,AutoFakeSlant=0.3]{Noto Sans Gujarati}

% Translations for polyglossia
\gappto\captionsgujarati{
  \renewcommand{\tablename}{કોષ્ટક}
  \renewcommand{\figurename}{આકૃતિ}
}

% Helper for TikZ nodes to ensure Gujarati font
\newcommand{\gu}[1]{{\gujaratifont #1}}

% Custom environments
\newtcolorbox{solutionbox}{
    breakable,
    enhanced,
    colback=solutioncolor!5!white,
    colframe=solutioncolor!75!black,
    fonttitle=\bfseries,
    title=જવાબ
}

\newtcolorbox{solutionboxnobreak}{
 colback=solutioncolor!5!white,
 colframe=solutioncolor!75!black,
 fonttitle=\bfseries,
 title=જવાબ
}

\newtcolorbox{keyformula}{
 breakable,
 enhanced,
 colback=keycolor!5!white,
 colframe=keycolor!75!black,
 fonttitle=\bfseries,
 title=રાસાયણિક સમીકરણ/સૂત્ર
}

\newtcolorbox{mnemonicbox}{
 breakable,
 enhanced,
 colback=mnemoniccolor!5!white,
 colframe=mnemoniccolor!75!black,
 fonttitle=\bfseries,
 title=મેમરી ટ્રીક
}


\begin{document}

\begin{center}
{\Huge\bfseries\color{headcolor} Subject Name (Gujarati)}\\[5pt]
{\LARGE 4331604 -- Summer 2025}\\[3pt]
{\large Semester 1 Study Material}\\[3pt]
{\normalsize\textit{Detailed Solutions and Explanations}}
\end{center}

\vspace{10pt}

\subsection*{પ્રશ્ન ૧(અ) [3
ગુણ]}\label{uxaaauxab0uxab6uxaa8-uxae7uxa85-3-uxa97uxaa3}

\textbf{સોફ્ટવેરની IEEE વ્યાખ્યા આપો. એપ્લીકેશન અને સિસ્ટમ સોફ્ટવેરનું એક એક ઉદાહરણ
આપો.}

\begin{solutionbox}

\textbf{IEEE વ્યાખ્યા}: સોફ્ટવેર એ કમ્પ્યુટર પ્રોગ્રામ્સ, પ્રક્રિયાઓ, નિયમો અને સંબંધિત
દસ્તાવેજીકરણ અને ડેટાનો સંગ્રહ છે.

\textbf{ઉદાહરણો}:

{\def\LTcaptype{none} % do not increment counter
\begin{longtable}[]{@{}
  >{\raggedright\arraybackslash}p{(\linewidth - 4\tabcolsep) * \real{0.4545}}
  >{\raggedright\arraybackslash}p{(\linewidth - 4\tabcolsep) * \real{0.2727}}
  >{\raggedright\arraybackslash}p{(\linewidth - 4\tabcolsep) * \real{0.2727}}@{}}
\toprule\noalign{}
\begin{minipage}[b]{\linewidth}\raggedright
સોફ્ટવેર પ્રકાર
\end{minipage} & \begin{minipage}[b]{\linewidth}\raggedright
ઉદાહરણ
\end{minipage} & \begin{minipage}[b]{\linewidth}\raggedright
હેતુ
\end{minipage} \\
\midrule\noalign{}
\endhead
\bottomrule\noalign{}
\endlastfoot
\textbf{એપ્લીકેશન સોફ્ટવેર} & Microsoft Word & વર્ડ પ્રોસેસિંગ અને ડોક્યુમેન્ટ
બનાવવા \\
\textbf{સિસ્ટમ સોફ્ટવેર} & Windows 10 & હાર્ડવેર સંસાધનોનું સંચાલન કરતું ઓપરેટિંગ
સિસ્ટમ \\
\end{longtable}
}

\begin{itemize}
\tightlist
\item
  \textbf{એપ્લીકેશન સોફ્ટવેર}: અંતિમ વપરાશકર્તાઓ માટે ચોક્કસ કાર્યો પૂર્ણ કરવા માટે
  ડિઝાઇન કરેલા પ્રોગ્રામ્સ
\item
  \textbf{સિસ્ટમ સોફ્ટવેર}: કમ્પ્યુટર હાર્ડવેરનું સંચાલન અને સંચાલન કરતા પ્રોગ્રામ્સ
\end{itemize}

\end{solutionbox}
\begin{mnemonicbox}
``Apps મદદ કરે Users ને, Systems મદદ કરે Hardware ને''

\end{mnemonicbox}
\begin{center}\rule{0.5\linewidth}{0.5pt}\end{center}

\subsection*{પ્રશ્ન ૧(બ) [4
ગુણ]}\label{uxaaauxab0uxab6uxaa8-uxae7uxaac-4-uxa97uxaa3}

\textbf{ડેટા ડિક્શનરી પર ટૂંકનોંધ લખો.}

\begin{solutionbox}

ડેટા ડિક્શનરી એ સિસ્ટમમાં વપરાતા ડેટા તત્વોની વ્યાખ્યાઓ અને લક્ષણો ધરાવતો કેન્દ્રીયકૃત
ભંડાર છે.

\textbf{ઘટકો સારણી}:

{\def\LTcaptype{none} % do not increment counter
\begin{longtable}[]{@{}ll@{}}
\toprule\noalign{}
ઘટક & વર્ણન \\
\midrule\noalign{}
\endhead
\bottomrule\noalign{}
\endlastfoot
\textbf{ડેટા નામ} & ડેટા તત્વ માટે અનન્ય ઓળખકર્તા \\
\textbf{ઉપનામો} & વપરાયેલા વૈકલ્પિક નામો \\
\textbf{વર્ણન} & હેતુ અને અર્થ \\
\textbf{ડેટા પ્રકાર} & ફોર્મેટ (integer, string, વગેરે) \\
\textbf{લંબાઈ} & સાઇઝ મર્યાદાઓ \\
\textbf{મૂલ્યો} & માન્ય શ્રેણી અથવા સેટ \\
\end{longtable}
}

\begin{itemize}
\tightlist
\item
  \textbf{હેતુ}: ડેવલપમેન્ટ ટીમમાં ડેટા ઉપયોગમાં સુસંગતતા સુનિશ્ચિત કરે છે
\item
  \textbf{ફાયદા}: અસ્પષ્ટતા ઘટાડે છે, સંચાર સુધારે છે, ડેટા વ્યાખ્યાઓનું પ્રમાણીકરણ કરે
  છે
\item
  \textbf{ઉપયોગ}: સિસ્ટમ ડિઝાઇન અને ડેટાબેઝ બનાવવા દરમિયાન સંદર્ભિત
\end{itemize}

\end{solutionbox}
\begin{mnemonicbox}
``Dictionary ડેટાને સ્પષ્ટ રીતે વ્યાખ્યાયિત કરે છે''

\end{mnemonicbox}
\begin{center}\rule{0.5\linewidth}{0.5pt}\end{center}

\subsection*{પ્રશ્ન ૧(ક) [7
ગુણ]}\label{uxaaauxab0uxab6uxaa8-uxae7uxa95-7-uxa97uxaa3}

\textbf{પ્રોટોટાઇપ મોડેલ આકૃતિ સહિત સમજાવો.}

\begin{solutionbox}

પ્રોટોટાઇપ મોડેલ એ પુનરાવર્તક અભિગમ છે જ્યાં આવશ્યકતાઓને વધુ સારી રીતે સમજવા માટે
વહેલું કામકાજનું મોડેલ બનાવવામાં આવે છે.

\textbf{ડાયાગ્રામ}:

\begin{center}
\textbf{Mermaid Diagram (Code)}
\begin{verbatim}
{Shaded}
{Highlighting}[]
graph LR
    A[આવશ્યકતા સંગ્રહ] {-{-}{} B[ઝડપી ડિઝાઇન]}
    B {-{-}{} C[પ્રોટોટાઇપ બનાવો]}
    C {-{-}{} D[વપરાશકર્તા મૂલ્યાંકન]}
    D {-{-}{} E\{વપરાશકર્તા સંતુષ્ટ?\}}
    E {-{-}{}|ના| F[આવશ્યકતાઓ શુદ્ધ કરો]}
    F {-{-}{} B}
    E {-{-}{}|હા| G[અંતિમ સિસ્ટમ ડેવલપમેન્ટ]}
    G {-{-}{} H[ટેસ્ટિંગ અને મેઇન્ટેનન્સ]}
{Highlighting}
{Shaded}
\end{verbatim}
\end{center}

\textbf{લક્ષણો}:

{\def\LTcaptype{none} % do not increment counter
\begin{longtable}[]{@{}lll@{}}
\toprule\noalign{}
તબક્કો & પ્રવૃત્તિ & આઉટપુટ \\
\midrule\noalign{}
\endhead
\bottomrule\noalign{}
\endlastfoot
\textbf{ઝડપી ડિઝાઇન} & મૂળભૂત આર્કિટેક્ચર & પ્રારંભિક ડિઝાઇન \\
\textbf{પ્રોટોટાઇપ બિલ્ડ} & કામકાજનું મોડેલ & પરીક્ષણયોગ્ય સિસ્ટમ \\
\textbf{વપરાશકર્તા મૂલ્યાંકન} & ફીડબેક સંગ્રહ & આવશ્યકતાઓનું શુદ્ધીકરણ \\
\end{longtable}
}

\begin{itemize}
\tightlist
\item
  \textbf{ફાયદા}: વહેલું વપરાશકર્તા ફીડબેક, ઓછું ડેવલપમેન્ટ જોખમ, આવશ્યકતાઓની વધુ
  સારી સમજ
\item
  \textbf{ગેરફાયદા}: અપર્યાપ્ત વિશ્લેષણ તરફ દોરી શકે છે, ગ્રાહક પ્રોટોટાઇપને અંતિમ
  ઉત્પાદન તરીકે અપેક્ષા કરે છે
\item
  \textbf{શ્રેષ્ઠ માટે}: અસ્પષ્ટ આવશ્યકતાઓ સાથેના પ્રોજેક્ટ્સ
\end{itemize}

\end{solutionbox}
\begin{mnemonicbox}
``Prototype શક્યતાઓ સાબિત કરે છે''

\end{mnemonicbox}
\begin{center}\rule{0.5\linewidth}{0.5pt}\end{center}

\subsection*{પ્રશ્ન ૧(ક) અથવા [7
ગુણ]}\label{uxaaauxab0uxab6uxaa8-uxae7uxa95-uxa85uxaa5uxab5-7-uxa97uxaa3}

\textbf{RAD મોડેલ ફાયદા અને ગેરફાયદા સાથે સમજાવો.}

\begin{solutionbox}

RAD (Rapid Application Development) પ્રોટોટાઇપિંગ અને પુનરાવર્તક ડેવલપમેન્ટ
દ્વારા ઝડપી ડેવલપમેન્ટ પર ભાર મૂકે છે.

\textbf{RAD તબક્કાઓ}:

\begin{center}
\textbf{Mermaid Diagram (Code)}
\begin{verbatim}
{Shaded}
{Highlighting}[]
graph LR
    A[બિઝનેસ મોડેલિંગ] {-{-}{} B[ડેટા મોડેલિંગ]}
    B {-{-}{} C[પ્રક્રિયા મોડેલિંગ]}
    C {-{-}{} D[એપ્લીકેશન જનરેશન]}
    D {-{-}{} E[ટેસ્ટિંગ અને ટર્નઓવર]}
{Highlighting}
{Shaded}
\end{verbatim}
\end{center}

\textbf{ફાયદા વિ ગેરફાયદા}:

{\def\LTcaptype{none} % do not increment counter
\begin{longtable}[]{@{}ll@{}}
\toprule\noalign{}
ફાયદા & ગેરફાયદા \\
\midrule\noalign{}
\endhead
\bottomrule\noalign{}
\endlastfoot
\textbf{ઝડપી ડેવલપમેન્ટ} & \textbf{કુશળ ડેવલપર્સની જરૂર} \\
\textbf{વહેલો વપરાશકર્તા સંડોવણી} & \textbf{મોટા પ્રોજેક્ટ્સ માટે યોગ્ય નથી} \\
\textbf{ઓછો ખર્ચ} & \textbf{વપરાશકર્તાની પ્રતિબદ્ધતા જરૂરી} \\
\textbf{વધુ સારી ગુણવત્તા} & \textbf{સંચાલિત ન હોય તો તકનીકી જોખમો} \\
\end{longtable}
}

\begin{itemize}
\tightlist
\item
  \textbf{મુખ્ય વિશેષતા}: સ્વયંસંચાલિત સાધનો અને 4GL પ્રોગ્રામિંગનો ઉપયોગ
\item
  \textbf{સમયસીમા}: સામાન્ય રીતે ડેવલપમેન્ટ માટે 60-90 દિવસ
\item
  \textbf{ટીમ}: નાની, અનુભવી ડેવલપમેન્ટ ટીમો
\end{itemize}

\end{solutionbox}
\begin{mnemonicbox}
``RAD ઝડપથી ડેવલપમેન્ટને ઝડપી બનાવે છે''

\end{mnemonicbox}
\begin{center}\rule{0.5\linewidth}{0.5pt}\end{center}

\subsection*{પ્રશ્ન ૨(અ) [3
ગુણ]}\label{uxaaauxab0uxab6uxaa8-uxae8uxa85-3-uxa97uxaa3}

\textbf{પૂર્ણ નામ આપો: SQA, FTR, RAD, BVA, GUI, DFD}

\begin{solutionbox}

{\def\LTcaptype{none} % do not increment counter
\begin{longtable}[]{@{}ll@{}}
\toprule\noalign{}
સંક્ષિપ્ત શબ્દ & પૂર્ણ નામ \\
\midrule\noalign{}
\endhead
\bottomrule\noalign{}
\endlastfoot
\textbf{SQA} & Software Quality Assurance \\
\textbf{FTR} & Formal Technical Review \\
\textbf{RAD} & Rapid Application Development \\
\textbf{BVA} & Boundary Value Analysis \\
\textbf{GUI} & Graphical User Interface \\
\textbf{DFD} & Data Flow Diagram \\
\end{longtable}
}

\end{solutionbox}
\begin{mnemonicbox}
``Software Quality And Formal Technical Reviews
Rapidly Analyze Development, Boundary Value Analysis Guides User
Interface, Data Flow Diagrams''

\end{mnemonicbox}
\begin{center}\rule{0.5\linewidth}{0.5pt}\end{center}

\subsection*{પ્રશ્ન ૨(બ) [4
ગુણ]}\label{uxaaauxab0uxab6uxaa8-uxae8uxaac-4-uxa97uxaa3}

\textbf{Agile મેથોડોલોજીની વ્યાખ્યા આપો. તેના સિદ્ધાંતો સમજાવો.}

\begin{solutionbox}

\textbf{વ્યાખ્યા}: Agile એ પુનરાવર્તક સોફ્ટવેર ડેવલપમેન્ટ અભિગમ છે જે સહયોગ,
લવચીકતા અને કામકાજના સોફ્ટવેરની ઝડપી ડિલિવરી પર ભાર મૂકે છે.

\textbf{મુખ્ય Agile સિદ્ધાંતો}:

{\def\LTcaptype{none} % do not increment counter
\begin{longtable}[]{@{}ll@{}}
\toprule\noalign{}
સિદ્ધાંત & વર્ણન \\
\midrule\noalign{}
\endhead
\bottomrule\noalign{}
\endlastfoot
\textbf{પ્રક્રિયાઓ કરતા વ્યક્તિઓ} & લોકો અને સંચાર પ્રાથમિકતા છે \\
\textbf{દસ્તાવેજીકરણ કરતા કામકાજનું સોફ્ટવેર} & કાર્યાત્મક સોફ્ટવેર પ્રાથમિક માપદંડ
છે \\
\textbf{ગ્રાહક સહયોગ} & સતત ગ્રાહક સંડોવણી \\
\textbf{પરિવર્તનનો જવાબ} & કઠોર યોજનાઓ કરતા અનુકૂલનક્ષમતા \\
\end{longtable}
}

\begin{itemize}
\tightlist
\item
  \textbf{પુનરાવર્તન લંબાઈ}: સામાન્ય રીતે 2-4 અઠવાડિયા (sprints)
\item
  \textbf{ડિલિવરી}: વારંવાર કામકાજના સોફ્ટવેર રિલીઝ
\item
  \textbf{ટીમ માળખું}: ક્રોસ-ફંક્શનલ, સ્વ-સંગઠિત ટીમો
\end{itemize}

\end{solutionbox}
\begin{mnemonicbox}
``Agile અનુકૂલન કરે છે અને આગળ વધે છે''

\end{mnemonicbox}
\begin{center}\rule{0.5\linewidth}{0.5pt}\end{center}

\subsection*{પ્રશ્ન ૨(ક) [7
ગુણ]}\label{uxaaauxab0uxab6uxaa8-uxae8uxa95-7-uxa97uxaa3}

\textbf{XP મોડેલ ફાયદા અને ગેરફાયદા સાથે સમજાવો.}

\begin{solutionbox}

XP (Extreme Programming) એ agile પદ્ધતિ છે જે એન્જિનિયરિંગ પ્રેક્ટિસ અને ગ્રાહક
સંતોષ પર ભાર મૂકે છે.

\textbf{XP પ્રેક્ટિસીસ}:

\begin{verbatim}
mindmap
  root((XP પ્રેક્ટિસીસ))
    પ્લાનિંગ ગેમ
    નાની રિલીઝીસ
    પેર પ્રોગ્રામિંગ
    ટેસ્ટ{-ડ્રિવન ડેવલપમેન્ટ}
    સતત ઇન્ટિગ્રેશન
    રિફેક્ટરિંગ
    સિમ્પલ ડિઝાઇન
    કલેક્ટિવ કોડ ઓનરશિપ
\end{verbatim}

\textbf{ફાયદા અને ગેરફાયદા}:

{\def\LTcaptype{none} % do not increment counter
\begin{longtable}[]{@{}ll@{}}
\toprule\noalign{}
ફાયદા & ગેરફાયદા \\
\midrule\noalign{}
\endhead
\bottomrule\noalign{}
\endlastfoot
\textbf{ઉચ્ચ કોડ ગુણવત્તા} & \textbf{અનુભવી પ્રોગ્રામર્સની જરૂર} \\
\textbf{ઝડપી ફીડબેક} & \textbf{ગ્રાહક ઉપલબ્ધ હોવા જોઈએ} \\
\textbf{ઓછા બગ્સ} & \textbf{કોડ-કેન્દ્રિત, ઓછા દસ્તાવેજીકરણ} \\
\textbf{લવચીકતા} & \textbf{ખર્ચનો અંદાજ લગાવવો મુશ્કેલ} \\
\end{longtable}
}

\begin{itemize}
\tightlist
\item
  \textbf{મુખ્ય પ્રેક્ટિસ}: પેર પ્રોગ્રામિંગ કોડ ગુણવત્તા સુનિશ્ચિત કરે છે
\item
  \textbf{ટેસ્ટિંગ}: સ્વયંસંચાલિત ટેસ્ટિંગ સાથે ટેસ્ટ-ફર્સ્ટ અભિગમ
\item
  \textbf{ગ્રાહકની ભૂમિકા}: સતત ફીડબેક પ્રદાન કરતો ઓન-સાઇટ ગ્રાહક
\end{itemize}

\end{solutionbox}
\begin{mnemonicbox}
``eXtreme Programming પ્રેક્ટિસીસ દ્વારા શ્રેષ્ઠતા હાંસલ કરે
છે''

\end{mnemonicbox}
\begin{center}\rule{0.5\linewidth}{0.5pt}\end{center}

\subsection*{પ્રશ્ન ૨(અ) અથવા [3
ગુણ]}\label{uxaaauxab0uxab6uxaa8-uxae8uxa85-uxa85uxaa5uxab5-3-uxa97uxaa3}

\textbf{બ્લેક બોક્સ ટેસ્ટિંગની વ્યાખ્યા આપો. તેની બે પદ્ધતિઓના નામ આપો.}

\begin{solutionbox}

\textbf{વ્યાખ્યા}: બ્લેક બોક્સ ટેસ્ટિંગ આંતરિક કોડ માળખાના જ્ઞાન વિના સોફ્ટવેરની
કાર્યક્ષમતાની તપાસ કરે છે, ઇનપુટ-આઉટપુટ વર્તન પર ધ્યાન કેન્દ્રિત કરે છે.

\textbf{બ્લેક બોક્સ ટેસ્ટિંગ પદ્ધતિઓ}:

{\def\LTcaptype{none} % do not increment counter
\begin{longtable}[]{@{}ll@{}}
\toprule\noalign{}
પદ્ધતિ & વર્ણન \\
\midrule\noalign{}
\endhead
\bottomrule\noalign{}
\endlastfoot
\textbf{Equivalence Partitioning} & ઇનપુટને માન્ય/અમાન્ય વર્ગોમાં વિભાજિત કરે
છે \\
\textbf{Boundary Value Analysis} & ઇનપુટ સીમાઓ પર મૂલ્યોનું પરીક્ષણ કરે છે \\
\end{longtable}
}

\begin{itemize}
\tightlist
\item
  \textbf{અભિગમ}: આવશ્યકતાઓ અને સ્પેસિફિકેશન આધારિત પરીક્ષણ
\item
  \textbf{ટેસ્ટર જ્ઞાન}: આંતરિક કોડ જ્ઞાનની જરૂર નથી
\item
  \textbf{ફોકસ}: બાહ્ય વર્તન અને કાર્યક્ષમતા
\end{itemize}

\end{solutionbox}
\begin{mnemonicbox}
``Black Box વર્તન આધારિત છે''

\end{mnemonicbox}
\begin{center}\rule{0.5\linewidth}{0.5pt}\end{center}

\subsection*{પ્રશ્ન ૨(બ) અથવા [4
ગુણ]}\label{uxaaauxab0uxab6uxaa8-uxae8uxaac-uxa85uxaa5uxab5-4-uxa97uxaa3}

\textbf{CLI નું પૂર્ણ નામ આપો. CLI ને ટૂંકમાં સમજાવો.}

\begin{solutionbox}

\textbf{CLI}: Command Line Interface

\textbf{CLI લક્ષણો}:

{\def\LTcaptype{none} % do not increment counter
\begin{longtable}[]{@{}ll@{}}
\toprule\noalign{}
પાસાં & વર્ણન \\
\midrule\noalign{}
\endhead
\bottomrule\noalign{}
\endlastfoot
\textbf{ઇનપુટ પદ્ધતિ} & વપરાશકર્તા દ્વારા ટાઇપ કરેલા ટેક્સ્ટ કમાન્ડ્સ \\
\textbf{આઉટપુટ} & ટેક્સ્ટ-આધારિત પ્રતિસાદો \\
\textbf{નેવિગેશન} & ફાઇલ/ડાયરેક્ટરી ઓપરેશન માટે કમાન્ડ્સ \\
\textbf{કાર્યક્ષમતા} & અનુભવી વપરાશકર્તાઓ માટે ઝડપી \\
\end{longtable}
}

\begin{itemize}
\tightlist
\item
  \textbf{ફાયદા}: ઝડપી એક્ઝિક્યુશન, ઓછો મેમરી ઉપયોગ, સ્ક્રિપ્ટેબલ
\item
  \textbf{ગેરફાયદા}: કમાન્ડ્સ શીખવાની જરૂર, શરૂઆતીઓ માટે વપરાશકર્તા-મિત્ર નથી
\item
  \textbf{ઉદાહરણો}: Windows Command Prompt, Linux Terminal, DOS
\end{itemize}

\end{solutionbox}
\begin{mnemonicbox}
``Commands ઇન્ટરેક્શનને લીડ કરે છે''

\end{mnemonicbox}
\begin{center}\rule{0.5\linewidth}{0.5pt}\end{center}

\subsection*{પ્રશ્ન ૨(ક) અથવા [7
ગુણ]}\label{uxaaauxab0uxab6uxaa8-uxae8uxa95-uxa85uxaa5uxab5-7-uxa97uxaa3}

\textbf{સ્વચ્છ આકૃતિ સાથે વોટરફોલ મોડેલ સમજાવો.}

\begin{solutionbox}

વોટરફોલ મોડેલ એ રેખીય ક્રમિક અભિગમ છે જ્યાં પછીના તબક્કામાં જતા પહેલા દરેક તબક્કો
પૂર્ણ કરવો આવશ્યક છે.

\textbf{વોટરફોલ મોડેલ ડાયાગ્રામ}:

\begin{center}
\textbf{Mermaid Diagram (Code)}
\begin{verbatim}
{Shaded}
{Highlighting}[]
graph LR
    A[આવશ્યકતા વિશ્લેષણ] {-{-}{} B[સિસ્ટમ ડિઝાઇન]}
    B {-{-}{} C[અમલીકરણ]}
    C {-{-}{} D[ઇન્ટિગ્રેશન અને ટેસ્ટિંગ]}
    D {-{-}{} E[ડિપ્લોયમેન્ટ]}
    E {-{-}{} F[મેઇન્ટેનન્સ]}
    
    style A fill:\#e1f5fe
    style B fill:\#f3e5f5
    style C fill:\#fff3e0
    style D fill:\#f1f8e9
    style E fill:\#fce4ec
    style F fill:\#fff8e1
{Highlighting}
{Shaded}
\end{verbatim}
\end{center}

\textbf{તબક્કાની વિગતો}:

{\def\LTcaptype{none} % do not increment counter
\begin{longtable}[]{@{}lll@{}}
\toprule\noalign{}
તબક્કો & પ્રવૃત્તિઓ & ડિલિવરેબલ્સ \\
\midrule\noalign{}
\endhead
\bottomrule\noalign{}
\endlastfoot
\textbf{આવશ્યકતાઓ} & જરૂરિયાતો એકત્રિત અને દસ્તાવેજીકરણ & SRS દસ્તાવેજ \\
\textbf{ડિઝાઇન} & સિસ્ટમ આર્કિટેક્ચર & ડિઝાઇન દસ્તાવેજો \\
\textbf{અમલીકરણ} & કોડ ડેવલપમેન્ટ & સોર્સ કોડ \\
\textbf{ટેસ્ટિંગ} & કાર્યક્ષમતા ચકાસો & ટેસ્ટ રિપોર્ટ્સ \\
\textbf{ડિપ્લોયમેન્ટ} & સિસ્ટમ ઇન્સ્ટોલેશન & કામકાજનું સિસ્ટમ \\
\textbf{મેઇન્ટેનન્સ} & બગ ફિક્સ, અપડેટ્સ & અપડેટેડ સિસ્ટમ \\
\end{longtable}
}

\begin{itemize}
\tightlist
\item
  \textbf{ફાયદા}: સરળ, સંચાલન કરવા સરળ, સારી રીતે દસ્તાવેજીકૃત
\item
  \textbf{ગેરફાયદા}: અનમ્ય, મોડું ટેસ્ટિંગ, પરિવર્તનોને સમાવવા મુશ્કેલ
\end{itemize}

\end{solutionbox}
\begin{mnemonicbox}
``પાણી હંમેશા નીચે તરફ વહે છે''

\end{mnemonicbox}
\begin{center}\rule{0.5\linewidth}{0.5pt}\end{center}

\subsection*{પ્રશ્ન ૩(અ) [3
ગુણ]}\label{uxaaauxab0uxab6uxaa8-uxae9uxa85-3-uxa97uxaa3}

\textbf{એક શબ્દમાં જવાબ આપો:}

\begin{solutionbox}

{\def\LTcaptype{none} % do not increment counter
\begin{longtable}[]{@{}ll@{}}
\toprule\noalign{}
પ્રશ્ન & જવાબ \\
\midrule\noalign{}
\endhead
\bottomrule\noalign{}
\endlastfoot
\textbf{સૌથી નાનું કોહેશન} & Coincidental \\
\textbf{સૌથી મોટું કપલિંગ} & Content \\
\textbf{ક્રીટીકલ એકટીવીટીનો ફાજલ સમય} & Zero \\
\end{longtable}
}

\end{solutionbox}
\begin{mnemonicbox}
``Coincidental કોહેશન, Content કપલિંગ, Critical શૂન્ય''

\end{mnemonicbox}
\begin{center}\rule{0.5\linewidth}{0.5pt}\end{center}

\subsection*{પ્રશ્ન ૩(બ) [4
ગુણ]}\label{uxaaauxab0uxab6uxaa8-uxae9uxaac-4-uxa97uxaa3}

\textbf{કપલિંગનું વર્ગીકરણ સમજાવો.}

\begin{solutionbox}

કપલિંગ મોડ્યુલો વચ્ચે પરસ્પર નિર્ભરતાને માપે છે. જાળવણીક્ષમતા માટે ઓછું કપલિંગ વધુ સારું
છે.

\textbf{કપલિંગ પ્રકારો (શ્રેષ્ઠથી ખરાબ સુધી)}:

{\def\LTcaptype{none} % do not increment counter
\begin{longtable}[]{@{}lll@{}}
\toprule\noalign{}
પ્રકાર & વર્ણન & ઉદાહરણ \\
\midrule\noalign{}
\endhead
\bottomrule\noalign{}
\endlastfoot
\textbf{ડેટા} & પેરામીટર્સ પાસ કરવામાં આવે છે & પેરામીટર્સ સાથે મેથડ કોલ્સ \\
\textbf{સ્ટેમ્પ} & ડેટા સ્ટ્રક્ચર પાસ કરવામાં આવે છે & ઓબ્જેક્ટ્સ/રેકોર્ડ્સ પાસ કરવા \\
\textbf{કંટ્રોલ} & કંટ્રોલ માહિતી પાસ કરવામાં આવે છે & ફ્લેગ્સ/સ્વિચીસ પાસ કરવા \\
\textbf{એક્સટર્નલ} & બાહ્ય ડેટા સંદર્ભ & ગ્લોબલ વેરિયેબલ્સ \\
\textbf{કોમન} & શેર્ડ ડેટા એરિયા & કોમન મેમરી બ્લોક્સ \\
\textbf{કન્ટેન્ટ} & આંતરિક બાબતોમાં સીધો પ્રવેશ & બીજા મોડ્યુલના ડેટાને બદલવા \\
\end{longtable}
}

\begin{itemize}
\tightlist
\item
  \textbf{શ્રેષ્ઠ પ્રેક્ટિસ}: ડેટા કપલિંગનો લક્ષ્ય રાખો
\item
  \textbf{ટાળો}: કન્ટેન્ટ અને કોમન કપલિંગ
\item
  \textbf{ડિઝાઇન ધ્યેય}: મોડ્યુલો વચ્ચેની નિર્ભરતાઓ ઘટાડો
\end{itemize}

\end{solutionbox}
\begin{mnemonicbox}
``Data Stamps Control External Common Content''

\end{mnemonicbox}
\begin{center}\rule{0.5\linewidth}{0.5pt}\end{center}

\subsection*{પ્રશ્ન ૩(ક) [7
ગુણ]}\label{uxaaauxab0uxab6uxaa8-uxae9uxa95-7-uxa97uxaa3}

\textbf{નીચેના પદોની વ્યાખ્યા આપો (ફક્ત પૂર્ણ નામ ન આપવું):}

\begin{solutionbox}

{\def\LTcaptype{none} % do not increment counter
\begin{longtable}[]{@{}
  >{\raggedright\arraybackslash}p{(\linewidth - 2\tabcolsep) * \real{0.3333}}
  >{\raggedright\arraybackslash}p{(\linewidth - 2\tabcolsep) * \real{0.6667}}@{}}
\toprule\noalign{}
\begin{minipage}[b]{\linewidth}\raggedright
પદ
\end{minipage} & \begin{minipage}[b]{\linewidth}\raggedright
વ્યાખ્યા
\end{minipage} \\
\midrule\noalign{}
\endhead
\bottomrule\noalign{}
\endlastfoot
\textbf{UI} & User Interface - વપરાશકર્તાઓ સોફ્ટવેર સિસ્ટમ્સ સાથે
ક્રિયાપ્રતિક્રિયા કરવાનું સાધન \\
\textbf{SE} & Software Engineering - એન્જિનિયરિંગ સિદ્ધાંતોનો ઉપયોગ કરીને
સોફ્ટવેર ડેવલપમેન્ટ માટે વ્યવસ્થિત અભિગમ \\
\textbf{PMC} & Project Management and Control - સોફ્ટવેર પ્રોજેક્ટ્સનું આયોજન,
મોનિટરિંગ અને નિયંત્રણ \\
\textbf{SDLC} & Software Development Life Cycle - વિભાવનાથી મેઇન્ટેનન્સ સુધી
સોફ્ટવેર ડેવલપમેન્ટમાં સંડોવાયેલા તબક્કાઓ \\
\textbf{Verification} & સોફ્ટવેર નિર્દિષ્ટ આવશ્યકતાઓ અને ડિઝાઇનને પૂર્ણ કરે છે કે કેમ
તે તપાસવાની પ્રક્રિયા \\
\textbf{Validation} & સોફ્ટવેર વપરાશકર્તાની જરૂરિયાતો અને હેતુપૂર્ણ હેતુને પૂર્ણ કરે છે
કે કેમ તે તપાસવાની પ્રક્રિયા \\
\textbf{SRS} & Software Requirements Specification - સોફ્ટવેર કાર્યક્ષમતા અને
મર્યાદાઓનું વર્ણન કરતો વિસ્તૃત દસ્તાવેજ \\
\end{longtable}
}

\begin{itemize}
\tightlist
\item
  \textbf{Verification}: ``શું આપણે ઉત્પાદન સાચી રીતે બનાવી રહ્યા છીએ?''
\item
  \textbf{Validation}: ``શું આપણે સાચો ઉત્પાદન બનાવી રહ્યા છીએ?''
\item
  \textbf{મુખ્ય તફાવત}: Verification સ્પેસિફિકેશન તપાસે છે, Validation
  વપરાશકર્તાની સંતોષ તપાસે છે
\end{itemize}

\end{solutionbox}
\begin{mnemonicbox}
``Users ઇન્ટરેક્ટ કરે છે, Software Engineers પ્લાન કરે છે,
Projects મેનેજ કરે છે, Cycles ડિફાઇન કરે છે, Verification આવશ્યકતાઓ તપાસે છે,
Validation સંતોષ તપાસે છે, Requirements સોફ્ટવેર સ્પેસિફાય કરે છે''

\end{mnemonicbox}
\begin{center}\rule{0.5\linewidth}{0.5pt}\end{center}

\subsection*{પ્રશ્ન ૩(અ) અથવા [3
ગુણ]}\label{uxaaauxab0uxab6uxaa8-uxae9uxa85-uxa85uxaa5uxab5-3-uxa97uxaa3}

\textbf{મેન આધારિત UI ફાયદા અને ગેરફાયદા સાથે સમજાવો.}

\begin{solutionbox}

મેન-આધારિત UI વપરાશકર્તા પસંદગી માટે વિકલ્પોને હાયરાર્કિકલ મેન્યુમાં રજૂ કરે છે.

\textbf{ફાયદા વિ ગેરફાયદા}:

{\def\LTcaptype{none} % do not increment counter
\begin{longtable}[]{@{}ll@{}}
\toprule\noalign{}
ફાયદા & ગેરફાયદા \\
\midrule\noalign{}
\endhead
\bottomrule\noalign{}
\endlastfoot
\textbf{શીખવા માટે સરળ} & \textbf{નિષ્ણાતો માટે ધીમું} \\
\textbf{ભૂલો ઘટાડે છે} & \textbf{મર્યાદિત લવચીકતા} \\
\textbf{સ્વ-સ્પષ્ટીકરણ} & \textbf{સ્ક્રીન સ્પેસનો વપરાશ} \\
\end{longtable}
}

\begin{itemize}
\tightlist
\item
  \textbf{માળખું}: વિકલ્પોનું હાયરાર્કિકલ સંગઠન
\item
  \textbf{નેવિગેશન}: પોઇન્ટ-એન્ડ-ક્લિક અથવા કીબોર્ડ શોર્ટકટ્સ
\item
  \textbf{શ્રેષ્ઠ માટે}: સારી રીતે વ્યાખ્યાયિત કાર્યો સાથેની એપ્લીકેશન્સ
\end{itemize}

\end{solutionbox}
\begin{mnemonicbox}
``Menus પસંદગીઓને સ્પષ્ટ બનાવે છે''

\end{mnemonicbox}
\begin{center}\rule{0.5\linewidth}{0.5pt}\end{center}

\subsection*{પ્રશ્ન ૩(બ) અથવા [4
ગુણ]}\label{uxaaauxab0uxab6uxaa8-uxae9uxaac-uxa85uxaa5uxab5-4-uxa97uxaa3}

\textbf{કોહેશનનું વર્ગીકરણ સમજાવો.}

\begin{solutionbox}

કોહેશન મોડ્યુલની અંદર તત્વો કેટલા નજીકથી સંબંધિત છે તે માપે છે. ઉચ્ચ કોહેશન વધુ સારું છે.

\textbf{કોહેશન પ્રકારો (શ્રેષ્ઠથી ખરાબ સુધી)}:

{\def\LTcaptype{none} % do not increment counter
\begin{longtable}[]{@{}ll@{}}
\toprule\noalign{}
પ્રકાર & વર્ણન \\
\midrule\noalign{}
\endhead
\bottomrule\noalign{}
\endlastfoot
\textbf{ફંક્શનલ} & એક, સારી રીતે વ્યાખ્યાયિત કાર્ય \\
\textbf{સિક્વેન્શિયલ} & એક તત્વનું આઉટપુટ આગળના તત્વને ફીડ કરે છે \\
\textbf{કમ્યુનિકેશનલ} & તત્વો સમાન ડેટા પર કામ કરે છે \\
\textbf{પ્રોસીડ્યુરલ} & તત્વો અમલીકરણ ક્રમને અનુસરે છે \\
\textbf{ટેમ્પોરલ} & તત્વો સમાન સમયે અમલમાં મૂકાય છે \\
\textbf{લોજિકલ} & તત્વો સમાન કાર્યો કરે છે \\
\textbf{કોઇન્સિડેન્ટલ} & તત્વો રેન્ડમ રીતે ગ્રુપ કરવામાં આવ્યા છે \\
\end{longtable}
}

\begin{itemize}
\tightlist
\item
  \textbf{ધ્યેય}: ફંક્શનલ કોહેશન હાંસલ કરો
\item
  \textbf{ડિઝાઇન સિદ્ધાંત}: દરેક મોડ્યુલની એક જ જવાબદારી હોવી જોઈએ
\item
  \textbf{માપદંડ}: ઉચ્ચ કોહેશન = વધુ સારું ડિઝાઇન
\end{itemize}

\end{solutionbox}
\begin{mnemonicbox}
``Functional Sequences Communicate Procedures
Temporally through Logical Coincidence''

\end{mnemonicbox}
\begin{center}\rule{0.5\linewidth}{0.5pt}\end{center}

\subsection*{પ્રશ્ન ૩(ક) અથવા [7
ગુણ]}\label{uxaaauxab0uxab6uxaa8-uxae9uxa95-uxa85uxaa5uxab5-7-uxa97uxaa3}

\textbf{રિસ્કની વ્યાખ્યા આપો. રિસ્ક મેનેજમેન્ટ સમજાવો.}

\begin{solutionbox}

\textbf{રિસ્ક વ્યાખ્યા}: સોફ્ટવેર ડેવલપમેન્ટ દરમિયાન થઈ શકતી સંભવિત સમસ્યા, જે
પ્રોજેક્ટની સફળતા પર નકારાત્મક અસર કરે છે.

\textbf{રિસ્ક મેનેજમેન્ટ પ્રક્રિયા}:

\begin{center}
\textbf{Mermaid Diagram (Code)}
\begin{verbatim}
{Shaded}
{Highlighting}[]
graph LR
    A[રિસ્ક ઓળખ] {-{-}{} B[રિસ્ક મૂલ્યાંકન]}
    B {-{-}{} C[રિસ્ક પ્રાથમિકતા]}
    C {-{-}{} D[રિસ્ક ઘટાડો]}
    D {-{-}{} E[રિસ્ક મોનિટરિંગ]}
    E {-{-}{} A}
{Highlighting}
{Shaded}
\end{verbatim}
\end{center}

\textbf{રિસ્ક મેનેજમેન્ટ પ્રવૃત્તિઓ}:

{\def\LTcaptype{none} % do not increment counter
\begin{longtable}[]{@{}lll@{}}
\toprule\noalign{}
પ્રવૃત્તિ & વર્ણન & આઉટપુટ \\
\midrule\noalign{}
\endhead
\bottomrule\noalign{}
\endlastfoot
\textbf{ઓળખ} & સંભવિત સમસ્યાઓ શોધો & રિસ્ક યાદી \\
\textbf{મૂલ્યાંકન} & સંભાવના અને અસરનું વિશ્લેષણ & રિસ્ક વિશ્લેષણ \\
\textbf{પ્રાથમિકતા} & મહત્વ પ્રમાણે રિસ્ક રેન્ક કરો & પ્રાથમિકતા મેટ્રિક્સ \\
\textbf{ઘટાડો} & રિસ્ક પ્રતિસાદ આયોજન & ઘટાડવાની વ્યૂહરચનાઓ \\
\textbf{મોનિટરિંગ} & રિસ્ક સ્થિતિ ટ્રેક કરો & અપડેટેડ રિસ્ક સ્થિતિ \\
\end{longtable}
}

\begin{itemize}
\tightlist
\item
  \textbf{રિસ્ક પ્રકારો}: તકનીકી, પ્રોજેક્ટ, બિઝનેસ રિસ્ક
\item
  \textbf{વ્યૂહરચનાઓ}: ટાળો, ટ્રાન્સફર કરો, ઘટાડો, સ્વીકારો
\item
  \textbf{સાધનો}: રિસ્ક મેટ્રિસેસ, સંભાવના-અસર ચાર્ટ્સ
\end{itemize}

\end{solutionbox}
\begin{mnemonicbox}
``Risk ને સાવચેતીભર્યા આયોજનની જરૂર છે''

\end{mnemonicbox}
\begin{center}\rule{0.5\linewidth}{0.5pt}\end{center}

\subsection*{પ્રશ્ન ૪(અ) [3
ગુણ]}\label{uxaaauxab0uxab6uxaa8-uxaeauxa85-3-uxa97uxaa3}

\textbf{વ્યાખ્યા આપો: Error, Failure, Test case}

\begin{solutionbox}

{\def\LTcaptype{none} % do not increment counter
\begin{longtable}[]{@{}
  >{\raggedright\arraybackslash}p{(\linewidth - 2\tabcolsep) * \real{0.3333}}
  >{\raggedright\arraybackslash}p{(\linewidth - 2\tabcolsep) * \real{0.6667}}@{}}
\toprule\noalign{}
\begin{minipage}[b]{\linewidth}\raggedright
પદ
\end{minipage} & \begin{minipage}[b]{\linewidth}\raggedright
વ્યાખ્યા
\end{minipage} \\
\midrule\noalign{}
\endhead
\bottomrule\noalign{}
\endlastfoot
\textbf{Error} & સોફ્ટવેર ડેવલપમેન્ટ પ્રક્રિયા દરમિયાન થયેલી માનવીય ભૂલ \\
\textbf{Failure} & અપેક્ષિત પરિણામોથી સોફ્ટવેર વર્તનનું વિચલન \\
\textbf{Test case} & ચોક્કસ કાર્યક્ષમતા અથવા સિસ્ટમ આવશ્યકતાને ચકાસવા માટેની
શરતોનો સેટ \\
\end{longtable}
}

\begin{itemize}
\tightlist
\item
  \textbf{સંબંધ}: Error દોષ તરફ દોરી જાય છે, દોષ નિષ્ફળતાનું કારણ બને છે
\item
  \textbf{Error સ્રોત}: ડેવલપરની ભૂલો, આવશ્યકતાઓની ગેરસમજ
\item
  \textbf{Test case ઘટકો}: ઇનપુટ, અપેક્ષિત આઉટપુટ, અમલીકરણ પગલાં
\end{itemize}

\end{solutionbox}
\begin{mnemonicbox}
``Errors નિષ્ફળતાનું કારણ બને છે, Tests સમસ્યાઓ પકડે છે''

\end{mnemonicbox}
\begin{center}\rule{0.5\linewidth}{0.5pt}\end{center}

\subsection*{પ્રશ્ન ૪(બ) [4
ગુણ]}\label{uxaaauxab0uxab6uxaa8-uxaeauxaac-4-uxa97uxaa3}

\textbf{ATM સિસ્ટમની કોઈ પણ છ ફંક્શનલ રિક્વાયરમેન્ટ ઓળખો.}

\begin{solutionbox}

\textbf{ATM સિસ્ટમ ફંક્શનલ રિક્વાયરમેન્ટ્સ}:

{\def\LTcaptype{none} % do not increment counter
\begin{longtable}[]{@{}ll@{}}
\toprule\noalign{}
રિક્વાયરમેન્ટ & વર્ણન \\
\midrule\noalign{}
\endhead
\bottomrule\noalign{}
\endlastfoot
\textbf{વપરાશકર્તા ઓથેન્ટિકેશન} & એકાઉન્ટ પ્રવેશ માટે PIN વેરિફિકેશન \\
\textbf{બેલેન્સ ઇન્ક્વાયરી} & વર્તમાન એકાઉન્ટ બેલેન્સ પ્રદર્શિત કરો \\
\textbf{કેશ વિથડ્રોવલ} & વિનંતી કરેલ કેશ રકમ વિતરિત કરો \\
\textbf{ફંડ ટ્રાન્સફર} & એકાઉન્ટ્સ વચ્ચે પૈસા ટ્રાન્સફર કરો \\
\textbf{ટ્રાન્ઝેક્શન હિસ્ટરી} & તાજેતરના ટ્રાન્ઝેક્શન રેકોર્ડ્સ બતાવો \\
\textbf{PIN ચેન્જ} & વપરાશકર્તાઓને PIN બદલવાની મંજૂરી આપો \\
\end{longtable}
}

\begin{itemize}
\tightlist
\item
  \textbf{સિક્યોરિટી}: બધા ટ્રાન્ઝેક્શન્સ માટે ઓથેન્ટિકેશન જરૂરી
\item
  \textbf{વેલિડેશન}: વિથડ્રોવલ પહેલા પર્યાપ્ત બેલેન્સ તપાસો
\item
  \textbf{લોગિંગ}: ઓડિટ માટે બધા ટ્રાન્ઝેક્શન્સ રેકોર્ડ કરો
\end{itemize}

\end{solutionbox}
\begin{mnemonicbox}
``ATMs ઓથેન્ટિકેટ કરે છે, બેલેન્સ કરે છે, કેશ કરે છે, ટ્રાન્સફર કરે
છે, હિસ્ટરી કરે છે, PIN કરે છે''

\end{mnemonicbox}
\begin{center}\rule{0.5\linewidth}{0.5pt}\end{center}

\subsection*{પ્રશ્ન ૪(ક) [7
ગુણ]}\label{uxaaauxab0uxab6uxaa8-uxaeauxa95-7-uxa97uxaa3}

\textbf{એક્ટિવિટી નેટવર્ક ડાયાગ્રામનો ઉપયોગ જણાવો. નીચેની સિસ્ટમ માટે એક્ટિવિટી
નેટવર્ક ડાયાગ્રામ બનાવો અને તેના માટે ક્રિટિકલ પાથ શોધો.}

\begin{solutionbox}

\textbf{એક્ટિવિટી નેટવર્ક ડાયાગ્રામના ઉપયોગો}:

\begin{itemize}
\tightlist
\item
  \textbf{પ્રોજેક્ટ શેડ્યુલિંગ}: પ્રોજેક્ટ ટાઇમલાઇન નક્કી કરો
\item
  \textbf{ક્રિટિકલ પાથ ઓળખ}: લાંબામાં લાંબો પાથ શોધો જે લઘુત્તમ પ્રોજેક્ટ અવધિ
  નક્કી કરે છે
\item
  \textbf{રિસોર્સ પ્લાનિંગ}: રિસોર્સ ફાળવણીને ઓપ્ટિમાઇઝ કરો
\end{itemize}

\textbf{એક્ટિવિટી નેટવર્ક ડાયાગ્રામ}:

\begin{verbatim}
    A(2) {-{-}{-}{-}{-} C(2) {-}{-}{-}{-}{-} E(4) {-}{-}{-}{-}{-} G(5) {-}{-}{-}{-}{-} H(2)}
             /           {           /}
    B(3) {-{-}{-}+             +{-}{-} D(4) +}
                              |
                              F(3)
\end{verbatim}

\textbf{ક્રિટિકલ પાથ વિશ્લેષણ}:

{\def\LTcaptype{none} % do not increment counter
\begin{longtable}[]{@{}llll@{}}
\toprule\noalign{}
પાથ & એક્ટિવિટીઝ & અવધિ & ક્રિટિકલ? \\
\midrule\noalign{}
\endhead
\bottomrule\noalign{}
\endlastfoot
\textbf{A-C-E-G-H} & A\rightarrowC\rightarrowE\rightarrowG\rightarrowH & 2+2+4+5+2 = 15 & ના \\
\textbf{B-C-E-G-H} & B\rightarrowC\rightarrowE\rightarrowG\rightarrowH & 3+2+4+5+2 = 16 & \textbf{હા} \\
\textbf{A-C-D-G-H} & A\rightarrowC\rightarrowD\rightarrowG\rightarrowH & 2+2+4+5+2 = 15 & ના \\
\end{longtable}
}

\textbf{ક્રિટિકલ પાથ}: B\rightarrowC\rightarrowE\rightarrowG\rightarrowH (16 દિવસ) \textbf{પ્રોજેક્ટ અવધિ}: 16 દિવસ

\end{solutionbox}
\begin{mnemonicbox}
``Networks પ્રોજેક્ટ પાથ્સને નેવિગેટ કરે છે''

\end{mnemonicbox}
\begin{center}\rule{0.5\linewidth}{0.5pt}\end{center}

\subsection*{પ્રશ્ન ૪(અ) અથવા [3
ગુણ]}\label{uxaaauxab0uxab6uxaa8-uxaeauxa85-uxa85uxaa5uxab5-3-uxa97uxaa3}

\textbf{રિક્વાયરમેન્ટ સંગ્રહ કરવાની કોઈ પણ ત્રણ પ્રક્રિયાઓ સમજાવો.}

\begin{solutionbox}

\textbf{રિક્વાયરમેન્ટ સંગ્રહ પ્રવૃત્તિઓ}:

{\def\LTcaptype{none} % do not increment counter
\begin{longtable}[]{@{}
  >{\raggedright\arraybackslash}p{(\linewidth - 4\tabcolsep) * \real{0.3226}}
  >{\raggedright\arraybackslash}p{(\linewidth - 4\tabcolsep) * \real{0.4194}}
  >{\raggedright\arraybackslash}p{(\linewidth - 4\tabcolsep) * \real{0.2581}}@{}}
\toprule\noalign{}
\begin{minipage}[b]{\linewidth}\raggedright
પ્રવૃત્તિ
\end{minipage} & \begin{minipage}[b]{\linewidth}\raggedright
વર્ણન
\end{minipage} & \begin{minipage}[b]{\linewidth}\raggedright
આઉટપુટ
\end{minipage} \\
\midrule\noalign{}
\endhead
\bottomrule\noalign{}
\endlastfoot
\textbf{સ્ટેકહોલ્ડર ઇન્ટરવ્યુ} & વપરાશકર્તાઓ અને ક્લાયન્ટ્સ સાથે સીધી ચર્ચા & ઇન્ટરવ્યુ
નોંધો, રિક્વાયરમેન્ટ્સ યાદી \\
\textbf{પ્રશ્નાવલીઓ} & મોટા વપરાશકર્તા જૂથો માટે માળખાગત પ્રશ્નો & સર્વે
પ્રતિસાદો, આંકડાકીય ડેટા \\
\textbf{દસ્તાવેજ વિશ્લેષણ} & હાલની સિસ્ટમ દસ્તાવેજીકરણની સમીક્ષા & વર્તમાન
સિસ્ટમની સમજ \\
\end{longtable}
}

\begin{itemize}
\tightlist
\item
  \textbf{હેતુ}: વપરાશકર્તાની જરૂરિયાતો અને સિસ્ટમ અપેક્ષાઓ સમજવા
\item
  \textbf{સહભાગીઓ}: વપરાશકર્તાઓ, ગ્રાહકો, ડોમેઇન નિષ્ણાતો, ડેવલપર્સ
\item
  \textbf{દસ્તાવેજીકરણ}: બધા ધોરણો SRS દસ્તાવેજમાં રેકોર્ડ કરવામાં આવે છે
\end{itemize}

\end{solutionbox}
\begin{mnemonicbox}
``Interviews, Questions, Documents રિક્વાયરમેન્ટ્સ
એકત્રિત કરે છે''

\end{mnemonicbox}
\begin{center}\rule{0.5\linewidth}{0.5pt}\end{center}

\subsection*{પ્રશ્ન ૪(બ) અથવા [4
ગુણ]}\label{uxaaauxab0uxab6uxaa8-uxaeauxaac-uxa85uxaa5uxab5-4-uxa97uxaa3}

\textbf{Bank ATM સિસ્ટમ માટે યુઝ કેસ ડાયાગ્રામ દોરો.}

\begin{solutionbox}

\textbf{ATM યુઝ કેસ ડાયાગ્રામ}:

\begin{verbatim}
graph TB
    Customer((Customer))
    Admin((Admin))
    Bank[Bank System]
    
    Customer {-{-} UC1[Check Balance]}
    Customer {-{-} UC2[Withdraw Cash]}
    Customer {-{-} UC3[Transfer Funds]}
    Customer {-{-} UC4[Change PIN]}
    Customer {-{-} UC5[Print Receipt]}
    
    Admin {-{-} UC6[Load Cash]}
    Admin {-{-} UC7[View Logs]}
    Admin {-{-} UC8[Maintenance]}
    
    UC1 {-.{-} Bank}
    UC2 {-.{-} Bank}
    UC3 {-.{-} Bank}
    UC4 {-.{-} Bank}
\end{verbatim}

\textbf{યુઝ કેસ વિગતો}:

{\def\LTcaptype{none} % do not increment counter
\begin{longtable}[]{@{}ll@{}}
\toprule\noalign{}
એક્ટર & યુઝ કેસીસ \\
\midrule\noalign{}
\endhead
\bottomrule\noalign{}
\endlastfoot
\textbf{કસ્ટમર} & Check Balance, Withdraw Cash, Transfer Funds, Change
PIN \\
\textbf{એડમિન} & Load Cash, View Logs, System Maintenance \\
\textbf{બેંક સિસ્ટમ} & Validate accounts, Process transactions \\
\end{longtable}
}

\end{solutionbox}
\begin{mnemonicbox}
``Customers ATMs વાપરે છે, Admins સિસ્ટમ્સ મેઇન્ટેઇન કરે છે''

\end{mnemonicbox}
\begin{center}\rule{0.5\linewidth}{0.5pt}\end{center}

\subsection*{પ્રશ્ન ૪(ક) અથવા [7
ગુણ]}\label{uxaaauxab0uxab6uxaa8-uxaeauxa95-uxa85uxaa5uxab5-7-uxa97uxaa3}

\textbf{સ્પાઇરલ મોડેલ આકૃતિ સહિત સમજાવો.}

\begin{solutionbox}

\textbf{સ્પાઇરલ મોડેલ ડાયાગ્રામ}:

\begin{verbatim}
graph TB
    subgraph "સ્પાઇરલ મોડેલ"
        A[આયોજન] {-{-} B[રિસ્ક વિશ્લેષણ]}
        B {-{-} C[એન્જિનિયરિંગ]}
        C {-{-} D[ગ્રાહક મૂલ્યાંકન]}
        D {-{-} A}
        
        A1[પ્લાન 1] {-{-} B1[રિસ્ક 1]}
        B1 {-{-} C1[કોડ 1]}
        C1 {-{-} D1[ટેસ્ટ 1]}
        D1 {-{-} A2[પ્લાન 2]}
        A2 {-{-} B2[રિસ્ક 2]}
        B2 {-{-} C2[કોડ 2]}
        C2 {-{-} D2[ટેસ્ટ 2]}
    end
\end{verbatim}

\textbf{સ્પાઇરલ મોડેલ લક્ષણો}:

{\def\LTcaptype{none} % do not increment counter
\begin{longtable}[]{@{}lll@{}}
\toprule\noalign{}
ક્વાડ્રન્ટ & પ્રવૃત્તિ & હેતુ \\
\midrule\noalign{}
\endhead
\bottomrule\noalign{}
\endlastfoot
\textbf{આયોજન} & ઉદ્દેશ્યો, વિકલ્પો નક્કી કરો & પુનરાવર્તન માટે ધ્યેયો સેટ કરો \\
\textbf{રિસ્ક વિશ્લેષણ} & રિસ્ક ઓળખો અને ઉકેલો & પ્રોજેક્ટ રિસ્ક ઘટાડો \\
\textbf{એન્જિનિયરિંગ} & ઉત્પાદન વિકસાવો અને ટેસ્ટ કરો & કામકાજનું સોફ્ટવેર
બનાવો \\
\textbf{મૂલ્યાંકન} & ગ્રાહક મૂલ્યાંકન & વપરાશકર્તા ફીડબેક મેળવો \\
\end{longtable}
}

\begin{itemize}
\tightlist
\item
  \textbf{મુખ્ય વિશેષતા}: પુનરાવર્તક ડેવલપમેન્ટ સાથે રિસ્ક-ડ્રિવન અભિગમ
\item
  \textbf{શ્રેષ્ઠ માટે}: મોટા, જટિલ, ઉચ્ચ-રિસ્ક પ્રોજેક્ટ્સ
\item
  \textbf{ફાયદા}: રિસ્ક મેનેજમેન્ટ, લવચીક, વૃદ્ધિશીલ ડેવલપમેન્ટ
\item
  \textbf{ગેરફાયદા}: જટિલ મેનેજમેન્ટ, મોંઘું, રિસ્ક નિપુણતાની જરૂર
\end{itemize}

\end{solutionbox}
\begin{mnemonicbox}
``Spirals પ્લાન કરે છે, રિસ્ક કરે છે, એન્જિનિયર કરે છે,
મૂલ્યાંકન કરે છે''

\end{mnemonicbox}
\begin{center}\rule{0.5\linewidth}{0.5pt}\end{center}

\subsection*{પ્રશ્ન ૫(અ) [3
ગુણ]}\label{uxaaauxab0uxab6uxaa8-uxaebuxa85-3-uxa97uxaa3}

\textbf{સાચું છે કે ખોટું તે જણાવો.}

\begin{solutionbox}

{\def\LTcaptype{none} % do not increment counter
\begin{longtable}[]{@{}
  >{\raggedright\arraybackslash}p{(\linewidth - 4\tabcolsep) * \real{0.3438}}
  >{\raggedright\arraybackslash}p{(\linewidth - 4\tabcolsep) * \real{0.2500}}
  >{\raggedright\arraybackslash}p{(\linewidth - 4\tabcolsep) * \real{0.4062}}@{}}
\toprule\noalign{}
\begin{minipage}[b]{\linewidth}\raggedright
વિધાન
\end{minipage} & \begin{minipage}[b]{\linewidth}\raggedright
જવાબ
\end{minipage} & \begin{minipage}[b]{\linewidth}\raggedright
સ્પષ્ટીકરણ
\end{minipage} \\
\midrule\noalign{}
\endhead
\bottomrule\noalign{}
\endlastfoot
\textbf{એક્ટિવિટી નેટવર્ક ડાયાગ્રામ ક્રિટિકલ પાથ નક્કી કરવા વપરાય છે} &
\textbf{સાચું} & એક્ટિવિટી નેટવર્કનો પ્રાથમિક હેતુ \\
\textbf{CPM માં સૌથી નાનો પાથ ક્રિટિકલ પાથ છે} & \textbf{ખોટું} & લાંબામાં લાંબો
પાથ ક્રિટિકલ પાથ છે \\
\textbf{રિસ્ક આવોઇડન્સ એ રિસ્ક ઉકેલવાની શ્રેષ્ઠ તકનીક છે} & \textbf{ખોટું} & શ્રેષ્ઠ
તકનીક રિસ્ક પ્રકાર પર આધારિત છે \\
\end{longtable}
}

\begin{itemize}
\tightlist
\item
  \textbf{ક્રિટિકલ પાથ}: પ્રોજેક્ટ નેટવર્કમાં લાંબામાં લાંબો અવધિનો પાથ
\item
  \textbf{CPM}: ક્રિટિકલ પાથ મેથડ પ્રોજેક્ટ બોટલનેક ઓળખે છે
\item
  \textbf{રિસ્ક વ્યૂહરચનાઓ}: ટાળો, ટ્રાન્સફર કરો, ઘટાડો, સ્વીકારો (પસંદગી સંદર્ભ
  પર આધારિત છે)
\end{itemize}

\end{solutionbox}
\begin{mnemonicbox}
``સાચા નેટવર્ક્સ, ખોટા નાના, ખોટા શ્રેષ્ઠ''

\end{mnemonicbox}
\begin{center}\rule{0.5\linewidth}{0.5pt}\end{center}

\subsection*{પ્રશ્ન ૫(બ) [4
ગુણ]}\label{uxaaauxab0uxab6uxaa8-uxaebuxaac-4-uxa97uxaa3}

\textbf{પ્રણાલીગત અને એજાઇલ માર્ગ વચ્ચેના તફાવતને ઓળખો.}

\begin{solutionbox}

\textbf{પ્રણાલીગત વિ એજાઇલ તુલના}:

{\def\LTcaptype{none} % do not increment counter
\begin{longtable}[]{@{}lll@{}}
\toprule\noalign{}
પાસું & પ્રણાલીગત & એજાઇલ \\
\midrule\noalign{}
\endhead
\bottomrule\noalign{}
\endlastfoot
\textbf{આયોજન} & વ્યાપક અગાઉનું આયોજન & અનુકૂલનશીલ આયોજન \\
\textbf{દસ્તાવેજીકરણ} & ભારે દસ્તાવેજીકરણ & ન્યૂનતમ દસ્તાવેજીકરણ \\
\textbf{ગ્રાહક સંડોવણી} & આવશ્યકતા તબક્કા સુધી મર્યાદિત & સતત સંડોવણી \\
\textbf{પરિવર્તન હેન્ડલિંગ} & મુશ્કેલ અને મોંઘું & પરિવર્તનને અપનાવે છે \\
\textbf{ડિલિવરી} & એક અંતિમ ડિલિવરી & વારંવાર વૃદ્ધિશીલ ડિલિવરી \\
\textbf{પ્રક્રિયા} & પ્રક્રિયા-સંચાલિત & લોકો-સંચાલિત \\
\end{longtable}
}

\begin{itemize}
\tightlist
\item
  \textbf{પ્રણાલીગત}: અનુમાનિત, ક્રમિક અભિગમ
\item
  \textbf{એજાઇલ}: અનુકૂલનશીલ, પુનરાવર્તક અભિગમ
\item
  \textbf{લવચીકતા}: એજાઇલ બદલાતી આવશ્યકતાઓ માટે વધુ પ્રતિસાદશીલ
\end{itemize}

\end{solutionbox}
\begin{mnemonicbox}
``પ્રણાલીગત ભારે આયોજન કરે છે, એજાઇલ હળવું અનુકૂલન કરે છે''

\end{mnemonicbox}
\begin{center}\rule{0.5\linewidth}{0.5pt}\end{center}

\subsection*{પ્રશ્ન ૫(ક) [7
ગુણ]}\label{uxaaauxab0uxab6uxaa8-uxaebuxa95-7-uxa97uxaa3}

\textbf{યુનિટ ટેસ્ટિંગની વ્યાખ્યા આપો. તેની આકૃતિ દોરો. તેની પ્રક્રિયા સમજાવો.}

\begin{solutionbox}

\textbf{યુનિટ ટેસ્ટિંગ વ્યાખ્યા}: ડિઝાઇન સ્પેસિફિકેશન અનુસાર તેઓ યોગ્ય રીતે કાર્ય કરે છે
તે ચકાસવા માટે વ્યક્તિગત સોફ્ટવેર ઘટકો અથવા મોડ્યુલોનું અલગથી પરીક્ષણ.

\textbf{યુનિટ ટેસ્ટિંગ પ્રક્રિયા}:

\begin{center}
\textbf{Mermaid Diagram (Code)}
\begin{verbatim}
{Shaded}
{Highlighting}[]
graph LR
    A[યુનિટ પસંદ કરો] {-{-}{} B[ટેસ્ટ કેસીસ ડિઝાઇન કરો]}
    B {-{-}{} C[ટેસ્ટ એન્વાયરનમેન્ટ સેટ કરો]}
    C {-{-}{} D[ટેસ્ટ એક્ઝિક્યુટ કરો]}
    D {-{-}{} E[પરિણામો રેકોર્ડ કરો]}
    E {-{-}{} F\{બધા ટેસ્ટ પાસ?\}}
    F {-{-}{}|ના| G[ડીબગ અને ફિક્સ]}
    G {-{-}{} D}
    F {-{-}{}|હા| H[યુનિટ મંજૂર]}
{Highlighting}
{Shaded}
\end{verbatim}
\end{center}

\textbf{યુનિટ ટેસ્ટિંગ પ્રક્રિયા પગલાં}:

{\def\LTcaptype{none} % do not increment counter
\begin{longtable}[]{@{}lll@{}}
\toprule\noalign{}
પગલું & પ્રવૃત્તિ & હેતુ \\
\midrule\noalign{}
\endhead
\bottomrule\noalign{}
\endlastfoot
\textbf{ટેસ્ટ પ્લાનિંગ} & ટેસ્ટ કરવાના યુનિટ ઓળખો & ટેસ્ટિંગ સ્કોપ વ્યાખ્યાયિત કરો \\
\textbf{ટેસ્ટ ડિઝાઇન} & ટેસ્ટ કેસીસ બનાવો & બધા કોડ પાથ આવરો \\
\textbf{ટેસ્ટ સેટઅપ} & ટેસ્ટ એન્વાયરનમેન્ટ તૈયાર કરો & ટેસ્ટ હેઠળના યુનિટને અલગ કરો \\
\textbf{ટેસ્ટ એક્ઝિક્યુશન} & ટેસ્ટ કેસીસ ચલાવો & યુનિટ વર્તન ચકાસો \\
\textbf{પરિણામ વિશ્લેષણ} & પરિણામોનું મૂલ્યાંકન કરો & ખામીઓ ઓળખો \\
\textbf{ખામી સુધારણા} & મળેલી સમસ્યાઓ સુધારો & યુનિટ ગુણવત્તા સુનિશ્ચિત કરો \\
\end{longtable}
}

\begin{itemize}
\tightlist
\item
  \textbf{ફાયદા}: વહેલું ખામી શોધ, સરળ ડીબગિંગ, સુધારેલી કોડ ગુણવત્તા
\item
  \textbf{સાધનો}: JUnit, NUnit, સ્વયંસંચાલિત ટેસ્ટિંગ ફ્રેમવર્ક્સ
\item
  \textbf{કવરેજ}: ઉચ્ચ કોડ કવરેજનો લક્ષ્ય રાખો (સ્ટેટમેન્ટ્સ, બ્રાન્ચીસ, પાથ્સ)
\end{itemize}

\end{solutionbox}
\begin{mnemonicbox}
``યુનિટ્સ વ્યક્તિગત ઘટકોનું સંપૂર્ણ પરીક્ષણ કરે છે''

\end{mnemonicbox}
\begin{center}\rule{0.5\linewidth}{0.5pt}\end{center}

\subsection*{પ્રશ્ન ૫(અ) અથવા [3
ગુણ]}\label{uxaaauxab0uxab6uxaa8-uxaebuxa85-uxa85uxaa5uxab5-3-uxa97uxaa3}

\textbf{પૂર્ણ નામ આપો.}

\begin{solutionbox}

{\def\LTcaptype{none} % do not increment counter
\begin{longtable}[]{@{}ll@{}}
\toprule\noalign{}
સંક્ષિપ્ત શબ્દ & પૂર્ણ નામ \\
\midrule\noalign{}
\endhead
\bottomrule\noalign{}
\endlastfoot
\textbf{AOA} & Activity On Arrow \\
\textbf{PERT} & Program Evaluation and Review Technique \\
\textbf{EVA} & Earned Value Analysis \\
\textbf{CPM} & Critical Path Method \\
\textbf{WBS} & Work Breakdown Structure \\
\textbf{PMC} & Project Management and Control \\
\end{longtable}
}

\end{solutionbox}
\begin{mnemonicbox}
``Activities On Arrows, Programs Evaluate Review
Techniques, Earned Values Analyzed, Critical Paths Managed, Work Broken
Structured, Projects Managed Controlled''

\end{mnemonicbox}
\begin{center}\rule{0.5\linewidth}{0.5pt}\end{center}

\subsection*{પ્રશ્ન ૫(બ) અથવા [4
ગુણ]}\label{uxaaauxab0uxab6uxaa8-uxaebuxaac-uxa85uxaa5uxab5-4-uxa97uxaa3}

\textbf{કોડ ઇન્સ્પેક્શન સમજાવો.}

\begin{solutionbox}

કોડ ઇન્સ્પેક્શન એ ખામીઓ ઓળખવા અને ગુણવત્તા ધોરણો સુનિશ્ચિત કરવા માટે ટીમ સભ્યો
દ્વારા સોર્સ કોડની વ્યવસ્થિત તપાસ છે.

\textbf{કોડ ઇન્સ્પેક્શન પ્રક્રિયા}:

{\def\LTcaptype{none} % do not increment counter
\begin{longtable}[]{@{}lll@{}}
\toprule\noalign{}
તબક્કો & પ્રવૃત્તિ & સહભાગીઓ \\
\midrule\noalign{}
\endhead
\bottomrule\noalign{}
\endlastfoot
\textbf{આયોજન} & ઇન્સ્પેક્શન મીટિંગ શેડ્યુલ કરો & મોડરેટર \\
\textbf{તૈયારી} & વ્યક્તિગત રીતે કોડ સમીક્ષા કરો & બધા ઇન્સ્પેક્ટર્સ \\
\textbf{ઇન્સ્પેક્શન મીટિંગ} & ધોરણોની ચર્ચા કરો & ટીમ સભ્યો \\
\textbf{રીવર્ક} & ઓળખાયેલી સમસ્યાઓ સુધારો & લેખક \\
\textbf{ફોલો-અપ} & સુધારણાઓ ચકાસો & મોડરેટર \\
\end{longtable}
}

\begin{itemize}
\tightlist
\item
  \textbf{ફાયદા}: વહેલું ખામી શોધ, જ્ઞાન શેરિંગ, સુધારેલી કોડ ગુણવત્તા
\item
  \textbf{ભૂમિકાઓ}: લેખક, મોડરેટર, રિવ્યુઅર્સ, રેકોર્ડર
\item
  \textbf{ફોકસ વિસ્તારો}: લોજિક એરર્સ, કોડિંગ સ્ટાન્ડર્ડ્સ, જાળવણીક્ષમતા
\end{itemize}

\end{solutionbox}
\begin{mnemonicbox}
``Inspections કોડ ગુણવત્તા સુધારે છે''

\end{mnemonicbox}
\begin{center}\rule{0.5\linewidth}{0.5pt}\end{center}

\subsection*{પ્રશ્ન ૫(ક) અથવા [7
ગુણ]}\label{uxaaauxab0uxab6uxaa8-uxaebuxa95-uxa85uxaa5uxab5-7-uxa97uxaa3}

\textbf{વ્હાઇટ બોક્સ ટેસ્ટિંગ મેથડની વ્યાખ્યા આપો. જુદી જુદી વ્હાઇટ બોક્સ ટેસ્ટિંગ મેથડ
સમજાવો.}

\begin{solutionbox}

\textbf{વ્હાઇટ બોક્સ ટેસ્ટિંગ વ્યાખ્યા}: ટેસ્ટિંગ પદ્ધતિ જે સંપૂર્ણ કવરેજ સુનિશ્ચિત કરવા
માટે આંતરિક કોડ માળખું, લોજિક પાથ્સ અને અમલીકરણ વિગતોની તપાસ કરે છે.

\textbf{વ્હાઇટ બોક્સ ટેસ્ટિંગ પદ્ધતિઓ}:

{\def\LTcaptype{none} % do not increment counter
\begin{longtable}[]{@{}lll@{}}
\toprule\noalign{}
પદ્ધતિ & વર્ણન & કવરેજ ફોકસ \\
\midrule\noalign{}
\endhead
\bottomrule\noalign{}
\endlastfoot
\textbf{સ્ટેટમેન્ટ કવરેજ} & દરેક સ્ટેટમેન્ટ એક્ઝિક્યુટ કરો & બધી કોડ લાઇન્સ \\
\textbf{બ્રાન્ચ કવરેજ} & બધા નિર્ણય પરિણામોનું પરીક્ષણ કરો & If-else શરતો \\
\textbf{પાથ કવરેજ} & બધા સંભવિત પાથ્સ એક્ઝિક્યુટ કરો & સંપૂર્ણ એક્ઝિક્યુશન ફ્લો \\
\textbf{કન્ડિશન કવરેજ} & બધા કન્ડિશન કોમ્બિનેશનનું પરીક્ષણ કરો & Boolean
એક્સપ્રેશન્સ \\
\end{longtable}
}

\textbf{ટેસ્ટિંગ તકનીકો}:

\begin{verbatim}
mindmap
  root((વ્હાઇટ બોક્સ ટેસ્ટિંગ))
    સ્ટેટમેન્ટ ટેસ્ટિંગ
      લાઇન કવરેજ
      કોડ એક્ઝિક્યુશન
    બ્રાન્ચ ટેસ્ટિંગ
      નિર્ણય પોઇન્ટ્સ
      True/False પાથ્સ
    પાથ ટેસ્ટિંગ
      બધા રૂટ્સ
      લૂપ ટેસ્ટિંગ
    કન્ડિશન ટેસ્ટિંગ
      Boolean લોજિક
      બહુવિધ કન્ડિશન્સ
\end{verbatim}

\textbf{કવરેજ વિશ્લેષણ}:

{\def\LTcaptype{none} % do not increment counter
\begin{longtable}[]{@{}
  >{\raggedright\arraybackslash}p{(\linewidth - 4\tabcolsep) * \real{0.3793}}
  >{\raggedright\arraybackslash}p{(\linewidth - 4\tabcolsep) * \real{0.3103}}
  >{\raggedright\arraybackslash}p{(\linewidth - 4\tabcolsep) * \real{0.3103}}@{}}
\toprule\noalign{}
\begin{minipage}[b]{\linewidth}\raggedright
તકનીક
\end{minipage} & \begin{minipage}[b]{\linewidth}\raggedright
ફોર્મ્યુલા
\end{minipage} & \begin{minipage}[b]{\linewidth}\raggedright
હેતુ
\end{minipage} \\
\midrule\noalign{}
\endhead
\bottomrule\noalign{}
\endlastfoot
\textbf{સ્ટેટમેન્ટ} & એક્ઝિક્યુટેડ સ્ટેટમેન્ટ્સ / કુલ સ્ટેટમેન્ટ્સ & બધા કોડ ચાલે તે સુનિશ્ચિત
કરો \\
\textbf{બ્રાન્ચ} & ટેસ્ટેડ બ્રાન્ચીસ / કુલ બ્રાન્ચીસ & બધા નિર્ણયો આવરો \\
\textbf{પાથ} & ટેસ્ટેડ પાથ્સ / કુલ પાથ્સ & સંપૂર્ણ ફ્લો કવરેજ \\
\end{longtable}
}

\begin{itemize}
\tightlist
\item
  \textbf{સાધનો}: કોડ કવરેજ વિશ્લેષકો, ડીબગિંગ સાધનો
\item
  \textbf{ફાયદા}: સંપૂર્ણ ટેસ્ટિંગ, મૃત કોડ ઓળખે છે, ગુણવત્તા સુનિશ્ચિત કરે છે
\item
  \textbf{ગેરફાયદા}: કોડ જ્ઞાનની જરૂર, સમય લેતું, આવશ્યકતા ગેપ્સ ચૂકી શકે છે
\end{itemize}

\end{solutionbox}
\begin{mnemonicbox}
``વ્હાઇટ બોક્સ કોડ માળખાની અંદર જુએ છે''

\end{mnemonicbox}

\end{document}
