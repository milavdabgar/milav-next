\documentclass[10pt,a4paper]{article}

% content/resources/templates/preamble.tex
\usepackage[margin=0.6in]{geometry}
\author{Milav Dabgar}
\usepackage{amsmath,amssymb,amsthm}
\usepackage{booktabs}
\usepackage{multirow}
\usepackage{xcolor}
\usepackage{tcolorbox}
\tcbuselibrary{breakable,skins}
\usepackage[colorlinks=true,linkcolor=blue]{hyperref}
\usepackage{titlesec}
\usepackage{enumitem}
\usepackage{tikz}
\usepackage{pgfplots}
\usepackage{circuitikz}
\usepackage[version=4]{mhchem}
\usepackage{longtable}
\usepackage{array}
\usepackage{float}
\usepackage{caption}
\usepackage{listings}

\lstset{
  basicstyle=\small\ttfamily,
  breaklines=true,
  breakatwhitespace=false,
  postbreak=\mbox{\textcolor{red}{$\hookrightarrow$}\space},
  float=false,
  numbers=left,
  numberstyle=\tiny\color{gray},
  numbersep=10pt,
  xleftmargin=2em,
  keywordstyle=\color{blue},
  commentstyle=\color{green!60!black},
  stringstyle=\color{purple},
  backgroundcolor=\color{gray!5},
  showstringspaces=false,
  tabsize=2,
  captionpos=b,
  keepspaces=true,
  columns=flexible
}

\pgfplotsset{compat=1.18}
\usetikzlibrary{shapes,arrows,positioning,calc,patterns,decorations.pathmorphing,decorations.markings,arrows.meta}

% Color scheme
\definecolor{headcolor}{RGB}{0,102,204}
\definecolor{keycolor}{RGB}{220,20,60}
\definecolor{solutioncolor}{RGB}{34,139,34}
\definecolor{mnemoniccolor}{RGB}{148,0,211}
\definecolor{codecolor}{RGB}{0,0,100}

% Spacing
\setlength{\parskip}{3pt}
\setlist[itemize]{nosep}
\setlist[enumerate]{nosep}

% Title formatting
\titleformat{\section}{\Large\bfseries\color{headcolor}}{\thesection}{1em}{}
\titleformat{\subsection}{\large\bfseries\color{headcolor}}{\thesubsection}{1em}{}

% Pandoc tightlist compatibility
\providecommand{\tightlist}{%
  \setlength{\itemsep}{0pt}\setlength{\parskip}{0pt}}

% Pandoc longtable compatibility
\newcounter{none}
\def\thenone{}


% content/resources/templates/gujarati-boxes.tex
\usepackage{fontspec}
\usepackage{polyglossia}

% Set Gujarati as main language (document is primarily in Gujarati)
% Note: gloss-gujarati.ldf doesn't exist in polyglossia, but it will use hyphenation patterns
\setdefaultlanguage{gujarati}
\setotherlanguage{english}

% Configure Gujarati font properly
% Use Language=Default to prevent polyglossia from trying to add language-specific features
% that don't exist for Gujarati, which causes "empty feature" warnings
\newfontfamily\gujaratifont[Script=Gujarati,AutoFakeBold=2.5,AutoFakeSlant=0.3]{Noto Sans Gujarati}
\setmainfont[Script=Gujarati,AutoFakeBold=2.5,AutoFakeSlant=0.3]{Noto Sans Gujarati}
% Use Noto Sans Gujarati for monospace to support Gujarati in text
\setmonofont[Scale=0.9]{Noto Sans Gujarati}

% Configure English to use the same font
\newfontfamily\englishfont[Script=Gujarati,AutoFakeBold=2.5,AutoFakeSlant=0.3]{Noto Sans Gujarati}

% Translations for polyglossia
\gappto\captionsgujarati{
  \renewcommand{\tablename}{કોષ્ટક}
  \renewcommand{\figurename}{આકૃતિ}
}

% Helper for TikZ nodes to ensure Gujarati font
\newcommand{\gu}[1]{{\gujaratifont #1}}

% Custom environments
\newtcolorbox{solutionbox}{
    breakable,
    enhanced,
    colback=solutioncolor!5!white,
    colframe=solutioncolor!75!black,
    fonttitle=\bfseries,
    title=જવાબ
}

\newtcolorbox{solutionboxnobreak}{
 colback=solutioncolor!5!white,
 colframe=solutioncolor!75!black,
 fonttitle=\bfseries,
 title=જવાબ
}

\newtcolorbox{keyformula}{
 breakable,
 enhanced,
 colback=keycolor!5!white,
 colframe=keycolor!75!black,
 fonttitle=\bfseries,
 title=રાસાયણિક સમીકરણ/સૂત્ર
}

\newtcolorbox{mnemonicbox}{
 breakable,
 enhanced,
 colback=mnemoniccolor!5!white,
 colframe=mnemoniccolor!75!black,
 fonttitle=\bfseries,
 title=મેમરી ટ્રીક
}


\begin{document}

\begin{center}
{\Huge\bfseries\color{headcolor} Subject Name (Gujarati)}\\[5pt]
{\LARGE 4331604 -- Winter 2023}\\[3pt]
{\large Semester 1 Study Material}\\[3pt]
{\normalsize\textit{Detailed Solutions and Explanations}}
\end{center}

\vspace{10pt}

\subsection*{પ્રશ્ન 1(અ) [3
ગુણ]}\label{uxaaauxab0uxab6uxaa8-1uxa85-3-uxa97uxaa3}

\textbf{સૉફ્ટવેર ની વ્યાખ્યા આપો. તેમજ એના લક્ષણો સમજાવો}

\begin{solutionbox}

સૉફ્ટવેર એ પ્રોગ્રામ્સ, સૂચનાઓ અને દસ્તાવેજીકરણનો સંગ્રહ છે જે કમ્પ્યુટર સિસ્ટમ પર કાર્યો
કરે છે.

\textbf{મુખ્ય લક્ષણો:}

{\def\LTcaptype{none} % do not increment counter
\begin{longtable}[]{@{}ll@{}}
\toprule\noalign{}
લક્ષણ & વર્ણન \\
\midrule\noalign{}
\endhead
\bottomrule\noalign{}
\endlastfoot
\textbf{અમૂર્ત} & શારીરિક રીતે સ્પર્શ કરી શકાતું નથી \\
\textbf{તાર્કિક} & વ્યવસ્થિત અભિગમ દ્વારા બનાવાયેલ \\
\textbf{ઉત્પાદિત} & પરંપરાગત રીતે ઉત્પન્ન નહીં, વિકસિત \\
\textbf{જટિલ} & અંતર્ગત જટિલ માળખું ધરાવે છે \\
\end{longtable}
}

\end{solutionbox}
\begin{mnemonicbox}
``અમૂર્ત તાર્કિક ઉત્પાદન જટિલતા''

\end{mnemonicbox}
\subsection*{પ્રશ્ન 1(બ) [4
ગુણ]}\label{uxaaauxab0uxab6uxaa8-1uxaac-4-uxa97uxaa3}

\textbf{Software engineering- A layered technology વિશેનો નોંધ લખો.}

\begin{solutionbox}

Software Engineering એ સ્તરીય ટેકનોલોજી તરીકે માળખાકીય છે જ્યાં દરેક સ્તર આગામી
સ્તરને આધાર આપે છે.

\textbf{સ્તરીય માળખું:}

{\def\LTcaptype{none} % do not increment counter
\begin{longtable}[]{@{}lll@{}}
\toprule\noalign{}
સ્તર & હેતુ & વર્ણન \\
\midrule\noalign{}
\endhead
\bottomrule\noalign{}
\endlastfoot
\textbf{Quality Focus} & પાયો & ગુણવત્તાયુક્ત ઉત્પાદનો પહોંચાડવાનો ભાર \\
\textbf{Process} & ફ્રેમવર્ક & સૉફ્ટવેર ડેવલપમેન્ટ કેવી રીતે કરવું તે નક્કી કરે છે \\
\textbf{Methods} & તકનીકો & પ્રવૃત્તિઓ કરવાની વિશિષ્ટ પદ્ધતિઓ \\
\textbf{Tools} & સ્વચાલન & પદ્ધતિઓને આધાર આપતું સૉફ્ટવેર \\
\end{longtable}
}

\begin{center}
\textbf{Mermaid Diagram (Code)}
\begin{verbatim}
{Shaded}
{Highlighting}[]
graph LR
    A[Tools] {-{-}{} B[Methods]}
    B {-{-}{} C[Process]}
    C {-{-}{} D[Quality Focus {-} પાયો]}
{Highlighting}
{Shaded}
\end{verbatim}
\end{center}

\end{solutionbox}
\begin{mnemonicbox}
``ટૂલ્સ મેથડ્સ પ્રોસેસ ક્વોલિટી''

\end{mnemonicbox}
\subsection*{પ્રશ્ન 1(ક) [7
ગુણ]}\label{uxaaauxab0uxab6uxaa8-1uxa95-7-uxa97uxaa3}

\textbf{સૉફ્ટવેર પ્રોસેસ ફ્રેમવર્ક તેમજ umbrella એક્ટિવિટી સમજાવો.}

\begin{solutionbox}

સૉફ્ટવેર પ્રોસેસ ફ્રેમવર્ક સૉફ્ટવેર ડેવલપમેન્ટ માટે મુખ્ય પ્રવૃત્તિઓ અને umbrella પ્રવૃત્તિઓ
સાથે માળખું પ્રદાન કરે છે.

\textbf{ફ્રેમવર્ક પ્રવૃત્તિઓ:}

{\def\LTcaptype{none} % do not increment counter
\begin{longtable}[]{@{}
  >{\raggedright\arraybackslash}p{(\linewidth - 4\tabcolsep) * \real{0.3333}}
  >{\raggedright\arraybackslash}p{(\linewidth - 4\tabcolsep) * \real{0.3333}}
  >{\raggedright\arraybackslash}p{(\linewidth - 4\tabcolsep) * \real{0.3333}}@{}}
\toprule\noalign{}
\begin{minipage}[b]{\linewidth}\raggedright
પ્રવૃત્તિ
\end{minipage} & \begin{minipage}[b]{\linewidth}\raggedright
હેતુ
\end{minipage} & \begin{minipage}[b]{\linewidth}\raggedright
મુખ્ય કાર્યો
\end{minipage} \\
\midrule\noalign{}
\endhead
\bottomrule\noalign{}
\endlastfoot
\textbf{Communication} & આવશ્યકતાઓ સમજવી & હિસ્સેદારો સાથે વાતચીત, આવશ્યકતા
એકત્રીકરણ \\
\textbf{Planning} & રોડમેપ બનાવવો & અંદાજ, શેડ્યૂલિંગ, જોખમ મૂલ્યાંકન \\
\textbf{Modeling} & બ્લુપ્રિન્ટ બનાવવા & વિશ્લેષણ અને ડિઝાઇન મોડલ્સ \\
\textbf{Construction} & સૉફ્ટવેર બનાવવું & કોડિંગ અને ટેસ્ટિંગ \\
\textbf{Deployment} & વપરાશકર્તાઓને પહોંચાડવું & ઇન્સ્ટોલેશન, સપોર્ટ, ફીડબેક \\
\end{longtable}
}

\textbf{Umbrella પ્રવૃત્તિઓ:}

\begin{itemize}
\tightlist
\item
  \textbf{Software project tracking}: પ્રગતિ નિરીક્ષણ અને ગુણવત્તા નિયંત્રણ
\item
  \textbf{Risk management}: સંભવિત સમસ્યાઓ ઓળખવી અને ઘટાડવી
\item
  \textbf{Quality assurance}: ધોરણો પૂરા થાય તેની ખાતરી કરવી
\item
  \textbf{Configuration management}: ફેરફારોને વ્યવસ્થિત રીતે નિયંત્રિત કરવા
\item
  \textbf{Work product preparation}: ડિલિવરેબલ દસ્તાવેજો બનાવવા
\end{itemize}

\begin{center}
\textbf{Mermaid Diagram (Code)}
\begin{verbatim}
{Shaded}
{Highlighting}[]
graph LR
    A[Communication] {-{-}{} B[Planning]}
    B {-{-}{} C[Modeling]}
    C {-{-}{} D[Construction]}
    D {-{-}{} E[Deployment]}
    F[Umbrella Activities] {-.{-}{} A}
    F {-.{-}{} B}
    F {-.{-}{} C}
    F {-.{-}{} D}
    F {-.{-}{} E}
{Highlighting}
{Shaded}
\end{verbatim}
\end{center}

\end{solutionbox}
\begin{mnemonicbox}
``કોમ્યુનિકેશન પ્લાનિંગ મોડલિંગ કન્સ્ટ્રક્શન ડિપ્લોયમેન્ટ'' +
``ટ્રેક રિસ્ક ક્વોલિટી કન્ફિગરેશન વર્ક''

\end{mnemonicbox}
\subsection*{પ્રશ્ન 1(ક) અથવા [7
ગુણ]}\label{uxaaauxab0uxab6uxaa8-1uxa95-uxa85uxaa5uxab5-7-uxa97uxaa3}

\textbf{SDLC ની વ્યાખ્યા આપો. તેમજ દરેક તબક્કા સમજાવો.}

\begin{solutionbox}

SDLC (Software Development Life Cycle) એ સૉફ્ટવેર એપ્લિકેશન્સ વિકસાવવા માટેની
વ્યવસ્થિત પ્રક્રિયા છે.

\textbf{SDLC તબક્કાઓ:}

{\def\LTcaptype{none} % do not increment counter
\begin{longtable}[]{@{}
  >{\raggedright\arraybackslash}p{(\linewidth - 6\tabcolsep) * \real{0.2500}}
  >{\raggedright\arraybackslash}p{(\linewidth - 6\tabcolsep) * \real{0.2500}}
  >{\raggedright\arraybackslash}p{(\linewidth - 6\tabcolsep) * \real{0.2500}}
  >{\raggedright\arraybackslash}p{(\linewidth - 6\tabcolsep) * \real{0.2500}}@{}}
\toprule\noalign{}
\begin{minipage}[b]{\linewidth}\raggedright
તબક્કો
\end{minipage} & \begin{minipage}[b]{\linewidth}\raggedright
હેતુ
\end{minipage} & \begin{minipage}[b]{\linewidth}\raggedright
મુખ્ય પ્રવૃત્તિઓ
\end{minipage} & \begin{minipage}[b]{\linewidth}\raggedright
ડિલિવરેબલ્સ
\end{minipage} \\
\midrule\noalign{}
\endhead
\bottomrule\noalign{}
\endlastfoot
\textbf{Planning} & અવકાશ નક્કી કરવો & શક્યતા અભ્યાસ, સંસાધન ફાળવણી & પ્રોજેક્ટ
પ્લાન \\
\textbf{Analysis} & આવશ્યકતાઓ એકત્રિત કરવી & આવશ્યકતા સંગ્રહ, દસ્તાવેજીકરણ &
SRS દસ્તાવેજ \\
\textbf{Design} & આર્કિટેક્ચર બનાવવું & સિસ્ટમ ડિઝાઇન, ડેટાબેસ ડિઝાઇન & ડિઝાઇન
દસ્તાવેજો \\
\textbf{Implementation} & કોડ લખવો & પ્રોગ્રામિંગ, યુનિટ ટેસ્ટિંગ & સોર્સ કોડ \\
\textbf{Testing} & ગુણવત્તા ચકાસવી & સિસ્ટમ ટેસ્ટિંગ, બગ ફિક્સિંગ & ટેસ્ટ
રિપોર્ટ્સ \\
\textbf{Deployment} & સૉફ્ટવેર રિલીઝ કરવું & ઇન્સ્ટોલેશન, યુઝર ટ્રેનિંગ & લાઇવ
સિસ્ટમ \\
\textbf{Maintenance} & ચાલુ સપોર્ટ & બગ ફિક્સ, એન્હાન્સમેન્ટ્સ & અપડેટેડ સિસ્ટમ \\
\end{longtable}
}

\begin{center}
\textbf{Mermaid Diagram (Code)}
\begin{verbatim}
{Shaded}
{Highlighting}[]
graph LR
    A[Planning] {-{-}{} B[Analysis]}
    B {-{-}{} C[Design]}
    C {-{-}{} D[Implementation]}
    D {-{-}{} E[Testing]}
    E {-{-}{} F[Deployment]}
    F {-{-}{} G[Maintenance]}
{Highlighting}
{Shaded}
\end{verbatim}
\end{center}

\end{solutionbox}
\begin{mnemonicbox}
``પ્લાન એનાલિસિસ ડિઝાઇન ઇમ્પ્લિમેન્ટેશન ટેસ્ટિંગ ડિપ્લોયમેન્ટ
મેઇન્ટેનન્સ''

\end{mnemonicbox}
\subsection*{પ્રશ્ન 2(અ) [3
ગુણ]}\label{uxaaauxab0uxab6uxaa8-2uxa85-3-uxa97uxaa3}

\textbf{Prototype model ના ફાયદા તેમજ નુકશાન વર્ણન કરો.}

\begin{solutionbox}

\textbf{Prototype Model વિશ્લેષણ:}

{\def\LTcaptype{none} % do not increment counter
\begin{longtable}[]{@{}ll@{}}
\toprule\noalign{}
ફાયદા & નુકસાન \\
\midrule\noalign{}
\endhead
\bottomrule\noalign{}
\endlastfoot
\textbf{વહેલો ફીડબેક} વપરાશકર્તાઓ તરફથી & \textbf{સમય વાપરતું} ડેવલપમેન્ટ
પ્રોસેસ \\
\textbf{ઓછું જોખમ} નિષ્ફળતાનું & \textbf{ખર્ચમાં વધારો} પુનરાવર્તન કારણે \\
\textbf{બહેતર સમજ} આવશ્યકતાઓની & \textbf{Scope creep} થઈ શકે છે \\
\end{longtable}
}

\end{solutionbox}
\begin{mnemonicbox}
``વહેલો ઓછું બહેતર'' વિરુદ્ધ ``સમય ખર્ચ સ્કોપ''

\end{mnemonicbox}
\subsection*{પ્રશ્ન 2(બ) [4
ગુણ]}\label{uxaaauxab0uxab6uxaa8-2uxaac-4-uxa97uxaa3}

\textbf{Prototyping મૉડલ સમજાવો, એ મૉડલ ક્યારે ઉપયોગમાં લઈ શકાય.}

\begin{solutionbox}

Prototyping Model વિકાસ પ્રક્રિયાની શરૂઆતમાં સૉફ્ટવેરનું કાર્યશીલ મોડલ બનાવે છે.

\textbf{ક્યારે ઉપયોગ કરવો:}

{\def\LTcaptype{none} % do not increment counter
\begin{longtable}[]{@{}lll@{}}
\toprule\noalign{}
સ્થિતિ & ઉદાહરણ & જસ્ટિફિકેશન \\
\midrule\noalign{}
\endhead
\bottomrule\noalign{}
\endlastfoot
\textbf{અસ્પષ્ટ આવશ્યકતાઓ} & ઓનલાઇન શોપિંગ કાર્ટ & યુઝર ઇન્ટરફેસને સુધારવાની
જરૂર \\
\textbf{નવી ટેકનોલોજી} & મોબાઇલ બેંકિંગ એપ & શક્યતા પરીક્ષણ જરૂરી \\
\textbf{યુઝર ઇન્ટરેક્શન જટિલ} & ગેમિંગ એપ્લિકેશન & યુઝર અનુભવ ચકાસણી જરૂરી \\
\end{longtable}
}

\begin{center}
\textbf{Mermaid Diagram (Code)}
\begin{verbatim}
{Shaded}
{Highlighting}[]
graph LR
    A[Requirements] {-{-}{} B[Quick Design]}
    B {-{-}{} C[Build Prototype]}
    C {-{-}{} D[User Evaluation]}
    D {-{-}{} E\{Satisfied?\}}
    E {-{-}{}|ના| B}
    E {-{-}{}|હા| F[Final System]}
{Highlighting}
{Shaded}
\end{verbatim}
\end{center}

\end{solutionbox}
\begin{mnemonicbox}
``આવશ્યકતા ઝડપી બિલ્ડ યુઝર સંતુષ્ટ ફાઇનલ''

\end{mnemonicbox}
\subsection*{પ્રશ્ન 2(ક) [7
ગુણ]}\label{uxaaauxab0uxab6uxaa8-2uxa95-7-uxa97uxaa3}

\textbf{આકૃતિ બનાવી સમજાવો (I) Waterfall model \& (II) Incremental
Model.}

\begin{solutionbox}

\textbf{(I) Waterfall Model:}

રેખીય ક્રમિક અભિગમ જ્યાં દરેક તબક્કો આગલા તબક્કા પહેલાં પૂર્ણ થવો જોઈએ.

\begin{center}
\textbf{Mermaid Diagram (Code)}
\begin{verbatim}
{Shaded}
{Highlighting}[]
graph LR
    A[Requirements Analysis] {-{-}{} B[System Design]}
    B {-{-}{} C[Implementation]}
    C {-{-}{} D[Testing]}
    D {-{-}{} E[Deployment]}
    E {-{-}{} F[Maintenance]}
{Highlighting}
{Shaded}
\end{verbatim}
\end{center}

{\def\LTcaptype{none} % do not increment counter
\begin{longtable}[]{@{}ll@{}}
\toprule\noalign{}
લક્ષણો & વર્ણન \\
\midrule\noalign{}
\endhead
\bottomrule\noalign{}
\endlastfoot
\textbf{ક્રમિક} & એક સમયે એક તબક્કો \\
\textbf{દસ્તાવેજીકરણ આધારિત} & ભારે દસ્તાવેજીકરણ \\
\textbf{યોગ્ય} & સ્પષ્ટ આવશ્યકતાઓ માટે \\
\end{longtable}
}

\textbf{(II) Incremental Model:}

નાના increments માં વિકાસ જ્યાં દરેક increment કાર્યક્ષમતા ઉમેરે છે.

\begin{center}
\textbf{Mermaid Diagram (Code)}
\begin{verbatim}
{Shaded}
{Highlighting}[]
graph LR
    A[Analysis] {-{-}{} B[Design]}
    B {-{-}{} C[Code]}
    C {-{-}{} D[Test]}
    D {-{-}{} E[Increment 1]}
    
    F[Analysis] {-{-}{} G[Design]}
    G {-{-}{} H[Code]}
    H {-{-}{} I[Test]}
    I {-{-}{} J[Increment 2]}
    
    E {-{-}{} K[Final Product]}
    J {-{-}{} K}
{Highlighting}
{Shaded}
\end{verbatim}
\end{center}

{\def\LTcaptype{none} % do not increment counter
\begin{longtable}[]{@{}lll@{}}
\toprule\noalign{}
લક્ષણ & Waterfall & Incremental \\
\midrule\noalign{}
\endhead
\bottomrule\noalign{}
\endlastfoot
\textbf{લવચીકતા} & ઓછી & વધુ \\
\textbf{જોખમ} & વધુ & ઓછું \\
\textbf{ડિલિવરી} & પ્રોજેક્ટના અંતે & બહુવિધ ડિલિવરીઓ \\
\end{longtable}
}

\end{solutionbox}
\begin{mnemonicbox}
``વોટર એકવાર પડે, ઇન્ક્રિમેન્ટ બહુવિધ બનાવે''

\end{mnemonicbox}
\subsection*{પ્રશ્ન 2(અ) અથવા [3
ગુણ]}\label{uxaaauxab0uxab6uxaa8-2uxa85-uxa85uxaa5uxab5-3-uxa97uxaa3}

\textbf{Incremental Model ના ફાયદા તેમજ નુકશાન વર્ણન કરો.}

\begin{solutionbox}

\textbf{Incremental Model વિશ્લેષણ:}

{\def\LTcaptype{none} % do not increment counter
\begin{longtable}[]{@{}ll@{}}
\toprule\noalign{}
ફાયદા & નુકસાન \\
\midrule\noalign{}
\endhead
\bottomrule\noalign{}
\endlastfoot
\textbf{વહેલી ડિલિવરી} કાર્યશીલ સૉફ્ટવેરની & \textbf{કુલ ખર્ચ} વધુ હોઈ શકે \\
\textbf{સરળ ટેસ્ટિંગ} નાના increments ની & \textbf{સિસ્ટમ આર્કિટેક્ચર}
સમસ્યાઓ \\
\textbf{ઓછું જોખમ} વહેલા ફીડબેક દ્વારા & \textbf{મેનેજમેન્ટ જટિલતા} વધે છે \\
\end{longtable}
}

\end{solutionbox}
\begin{mnemonicbox}
``વહેલી સરળ ઓછું'' વિરુદ્ધ ``કુલ સિસ્ટમ મેનેજમેન્ટ''

\end{mnemonicbox}
\subsection*{પ્રશ્ન 2(બ) અથવા [4
ગુણ]}\label{uxaaauxab0uxab6uxaa8-2uxaac-uxa85uxaa5uxab5-4-uxa97uxaa3}

\textbf{Rapid Application Development (RAD) નો ખ્યાલ આપો સમજાવો.}

\begin{solutionbox}

RAD યોજના અને ટેસ્ટિંગ કરતાં ઝડપી prototyping અને ત્વરિત ફીડબેક પર ભાર મૂકે છે.

\textbf{RAD ઘટકો:}

{\def\LTcaptype{none} % do not increment counter
\begin{longtable}[]{@{}
  >{\raggedright\arraybackslash}p{(\linewidth - 6\tabcolsep) * \real{0.2500}}
  >{\raggedright\arraybackslash}p{(\linewidth - 6\tabcolsep) * \real{0.2500}}
  >{\raggedright\arraybackslash}p{(\linewidth - 6\tabcolsep) * \real{0.2500}}
  >{\raggedright\arraybackslash}p{(\linewidth - 6\tabcolsep) * \real{0.2500}}@{}}
\toprule\noalign{}
\begin{minipage}[b]{\linewidth}\raggedright
તબક્કો
\end{minipage} & \begin{minipage}[b]{\linewidth}\raggedright
અવધિ
\end{minipage} & \begin{minipage}[b]{\linewidth}\raggedright
પ્રવૃત્તિઓ
\end{minipage} & \begin{minipage}[b]{\linewidth}\raggedright
આઉટપુટ
\end{minipage} \\
\midrule\noalign{}
\endhead
\bottomrule\noalign{}
\endlastfoot
\textbf{Business Modeling} & ટૂંકી & માહિતી પ્રવાહ નક્કી કરવો & બિઝનેસ
આવશ્યકતાઓ \\
\textbf{Data Modeling} & ટૂંકી & ડેટા ઓબ્જેક્ટ્સ નક્કી કરવા & ડેટા મોડલ્સ \\
\textbf{Process Modeling} & ટૂંકી & પ્રોસેસિંગ functions નક્કી કરવા & પ્રોસેસ
વર્ણનો \\
\textbf{Application Generation} & ટૂંકી & ટૂલ્સ વાપરીને બનાવવું & કાર્યશીલ
એપ્લિકેશન \\
\textbf{Testing \& Turnover} & ટૂંકી & ટેસ્ટ અને ડિલિવર કરવું & ફાઇનલ સિસ્ટમ \\
\end{longtable}
}

\begin{center}
\textbf{Mermaid Diagram (Code)}
\begin{verbatim}
{Shaded}
{Highlighting}[]
graph LR
    A[Business Modeling] {-{-}{} B[Data Modeling]}
    B {-{-}{} C[Process Modeling]}
    C {-{-}{} D[Application Generation]}
    D {-{-}{} E[Testing \& Turnover]}
{Highlighting}
{Shaded}
\end{verbatim}
\end{center}

\end{solutionbox}
\begin{mnemonicbox}
``બિઝનેસ ડેટા પ્રોસેસ એપ્લિકેશન ટેસ્ટિંગ''

\end{mnemonicbox}
\subsection*{પ્રશ્ન 2(ક) અથવા [7
ગુણ]}\label{uxaaauxab0uxab6uxaa8-2uxa95-uxa85uxaa5uxab5-7-uxa97uxaa3}

\textbf{Spiral Model ની આકૃતિ બનાવી સમજાવો. તેમજ ફાયદા અને નુકશાન વર્ણન કરો.}

\begin{solutionbox}

Spiral Model પુનરાવર્તક વિકાસને વ્યવસ્થિત જોખમ વિશ્લેષણ સાથે જોડે છે.

\begin{center}
\textbf{Mermaid Diagram (Code)}
\begin{verbatim}
{Shaded}
{Highlighting}[]
graph LR
    A[Planning] {-{-}{} B[Risk Analysis]}
    B {-{-}{} C[Engineering]}
    C {-{-}{} D[Evaluation]}
    D {-{-}{} A}
    
    E[Determine Objectives] {-{-}{} F[Identify Risks]}
    F {-{-}{} G[Develop \& Test]}
    G {-{-}{} H[Plan Next Iteration]}
    H {-{-}{} E}
{Highlighting}
{Shaded}
\end{verbatim}
\end{center}

\textbf{Spiral ચતુર્થાંશ:}

{\def\LTcaptype{none} % do not increment counter
\begin{longtable}[]{@{}
  >{\raggedright\arraybackslash}p{(\linewidth - 4\tabcolsep) * \real{0.3333}}
  >{\raggedright\arraybackslash}p{(\linewidth - 4\tabcolsep) * \real{0.3333}}
  >{\raggedright\arraybackslash}p{(\linewidth - 4\tabcolsep) * \real{0.3333}}@{}}
\toprule\noalign{}
\begin{minipage}[b]{\linewidth}\raggedright
ચતુર્થાંશ
\end{minipage} & \begin{minipage}[b]{\linewidth}\raggedright
પ્રવૃત્તિ
\end{minipage} & \begin{minipage}[b]{\linewidth}\raggedright
હેતુ
\end{minipage} \\
\midrule\noalign{}
\endhead
\bottomrule\noalign{}
\endlastfoot
\textbf{Planning} & લક્ષ્ય સેટિંગ & આવશ્યકતાઓ અને અવરોધો નક્કી કરવા \\
\textbf{Risk Analysis} & જોખમ મૂલ્યાંકન & જોખમો ઓળખવા અને ઉકેલવા \\
\textbf{Engineering} & વિકાસ & ઉત્પાદન બનાવવું અને ટેસ્ટ કરવું \\
\textbf{Evaluation} & ગ્રાહક મૂલ્યાંકન & પરિણામો મૂલ્યાંકન અને આગલા iteration ની
યોજના \\
\end{longtable}
}

\textbf{ફાયદા વિરુદ્ધ નુકસાન:}

{\def\LTcaptype{none} % do not increment counter
\begin{longtable}[]{@{}ll@{}}
\toprule\noalign{}
ફાયદા & નુકસાન \\
\midrule\noalign{}
\endhead
\bottomrule\noalign{}
\endlastfoot
\textbf{ઉચ્ચ જોખમ પ્રોજેક્ટ્સ} સારી રીતે હેન્ડલ થાય & \textbf{જટિલ મેનેજમેન્ટ}
જરૂરી \\
\textbf{મોટી} એપ્લિકેશન્સ માટે સારું & \textbf{નાના પ્રોજેક્ટ્સ માટે મોંઘું} \\
\textbf{ગ્રાહક સામેલ} આખા દરમિયાન & \textbf{જોખમ વિશ્લેષણ કુશળતા} જરૂરી \\
\end{longtable}
}

\end{solutionbox}
\begin{mnemonicbox}
``પ્લાન રિસ્ક એન્જિનિયર ઇવેલ્યુએટ'' + ``ઉચ્ચ સારું ગ્રાહક''
વિરુદ્ધ ``જટિલ મોંઘું જોખમ''

\end{mnemonicbox}
\subsection*{પ્રશ્ન 3(અ) [3
ગુણ]}\label{uxaaauxab0uxab6uxaa8-3uxa85-3-uxa97uxaa3}

\textbf{SRS ના મહત્વ દર્શાવો}

\begin{solutionbox}

SRS (Software Requirements Specification) એ સૉફ્ટવેર ડેવલપમેન્ટ માટે મહત્વપૂર્ણ
પાયાનું દસ્તાવેજ છે.

\textbf{મહત્વ કોષ્ટક:}

{\def\LTcaptype{none} % do not increment counter
\begin{longtable}[]{@{}lll@{}}
\toprule\noalign{}
પાસું & મહત્વ & ફાયદો \\
\midrule\noalign{}
\endhead
\bottomrule\noalign{}
\endlastfoot
\textbf{કોમ્યુનિકેશન} & હિસ્સેદારોની સમજ & સ્પષ્ટ અપેક્ષાઓ \\
\textbf{કરાર} & કાનૂની સમજૂતી & વિવાદ નિરાકરણ \\
\textbf{ટેસ્ટિંગ આધાર} & ચકાસણી માપદંડ & ગુણવત્તા ખાતરી \\
\end{longtable}
}

\end{solutionbox}
\begin{mnemonicbox}
``કોમ્યુનિકેશન કરાર ટેસ્ટિંગ''

\end{mnemonicbox}
\subsection*{પ્રશ્ન 3(બ) [4
ગુણ]}\label{uxaaauxab0uxab6uxaa8-3uxaac-4-uxa97uxaa3}

\textbf{સારા અને ખરાબ SRS ના લક્ષણો સમજાવો}

\begin{solutionbox}

\textbf{SRS ગુણવત્તા લક્ષણો:}

{\def\LTcaptype{none} % do not increment counter
\begin{longtable}[]{@{}
  >{\raggedright\arraybackslash}p{(\linewidth - 2\tabcolsep) * \real{0.5000}}
  >{\raggedright\arraybackslash}p{(\linewidth - 2\tabcolsep) * \real{0.5000}}@{}}
\toprule\noalign{}
\begin{minipage}[b]{\linewidth}\raggedright
સારો SRS
\end{minipage} & \begin{minipage}[b]{\linewidth}\raggedright
ખરાબ SRS
\end{minipage} \\
\midrule\noalign{}
\endhead
\bottomrule\noalign{}
\endlastfoot
\textbf{સંપૂર્ણ} - બધી આવશ્યકતાઓ આવરી લેવાયેલ & \textbf{અધૂરો} - આવશ્યકતાઓ ખૂટે
છે \\
\textbf{સુસંગત} - કોઈ વિરોધાભાસ નથી & \textbf{અસંગત} - વિરોધી નિવેદનો \\
\textbf{અસ્પષ્ટ નહીં} - સ્પષ્ટ અર્થ & \textbf{અસ્પષ્ટ} - બહુવિધ અર્થઘટન \\
\textbf{ચકાસી શકાય તેવું} - ટેસ્ટ કરી શકાય & \textbf{ચકાસી ન શકાય} - વેલિડેટ
કરી શકાતું નથી \\
\end{longtable}
}

\textbf{વધારાના સારા લક્ષણો:}

\begin{itemize}
\tightlist
\item
  \textbf{સુધારી શકાય તેવું}: બદલવું અને જાળવવું સરળ
\item
  \textbf{ટ્રેસેબલ}: સ્રોત અને ડિઝાઇન સાથે લિંક
\end{itemize}

\begin{center}
\textbf{Mermaid Diagram (Code)}
\begin{verbatim}
{Shaded}
{Highlighting}[]
graph TD
    A[સારો SRS] {-{-}{} B[સંપૂર્ણ]}
    A {-{-}{} C[સુસંગત]}
    A {-{-}{} D[અસ્પષ્ટ નહીં]}
    A {-{-}{} E[ચકાસી શકાય તેવું]}
    
    F[ખરાબ SRS] {-{-}{} G[અધૂરો]}
    F {-{-}{} H[અસંગત]}
    F {-{-}{} I[અસ્પષ્ટ]}
    F {-{-}{} J[ચકાસી ન શકાય]}
{Highlighting}
{Shaded}
\end{verbatim}
\end{center}

\end{solutionbox}
\begin{mnemonicbox}
``સંપૂર્ણ સુસંગત અસ્પષ્ટ-ન ચકાસી-શકાય'' વિરુદ્ધ ``અધૂરો અસંગત
અસ્પષ્ટ ચકાસી-ન-શકાય''

\end{mnemonicbox}
\subsection*{પ્રશ્ન 3(ક) [7
ગુણ]}\label{uxaaauxab0uxab6uxaa8-3uxa95-7-uxa97uxaa3}

\textbf{નીચે આપેલ નું વર્ગીકરણ/વર્ણન કરો. i) Functional Requirements ii)
Non-functional Requirements}

\begin{solutionbox}

સૉફ્ટવેર આવશ્યકતાઓને બે મુખ્ય શ્રેણીઓમાં વર્ગીકૃત કરવામાં આવે છે.

\textbf{(i) Functional Requirements:}

સિસ્ટમે શું કરવું જોઈએ તે નક્કી કરે છે - વિશિષ્ટ વર્તણૂકો અને કાર્યો.

{\def\LTcaptype{none} % do not increment counter
\begin{longtable}[]{@{}lll@{}}
\toprule\noalign{}
પ્રકાર & વર્ણન & ઉદાહરણ \\
\midrule\noalign{}
\endhead
\bottomrule\noalign{}
\endlastfoot
\textbf{બિઝનેસ નિયમો} & મુખ્ય બિઝનેસ લોજિક & ``આવકના સ્લેબ મુજબ ટેક્સ ગણતરી
કરવી'' \\
\textbf{યુઝર એક્શન્સ} & સિસ્ટમ પ્રતિભાવો & ``યુઝરનેમ/પાસવર્ડ સાથે લોગિન'' \\
\textbf{ડેટા પ્રોસેસિંગ} & માહિતી હેન્ડલિંગ & ``માસિક વેચાણ રિપોર્ટ જનરેટ
કરવી'' \\
\textbf{એક્સટર્નલ ઇન્ટરફેસ} & સિસ્ટમ ક્રિયાપ્રતિક્રિયાઓ & ``પેમેન્ટ ગેટવે સાથે કનેક્ટ
કરવું'' \\
\end{longtable}
}

\textbf{(ii) Non-functional Requirements:}

સિસ્ટમે કેવી રીતે પ્રદર્શન કરવું જોઈએ તે નક્કી કરે છે - ગુણવત્તા લક્ષણો અને મર્યાદાઓ.

{\def\LTcaptype{none} % do not increment counter
\begin{longtable}[]{@{}llll@{}}
\toprule\noalign{}
શ્રેણી & આવશ્યકતા & ઉદાહરણ & માપદંડ \\
\midrule\noalign{}
\endhead
\bottomrule\noalign{}
\endlastfoot
\textbf{પ્રદર્શન} & પ્રતિભાવ સમય & ``પેજ લોડ \textless{} 3 સેકન્ડ'' & સમય
મેટ્રિક્સ \\
\textbf{સુરક્ષા} & ડેટા સુરક્ષા & ``યુઝર પાસવર્ડ એન્ક્રિપ્ટ કરવા'' & સુરક્ષા
ધોરણો \\
\textbf{વિશ્વસનીયતા} & સિસ્ટમ અપટાઇમ & ``99.9\% ઉપલબ્ધતા'' & નિષ્ફળતા દરો \\
\textbf{ઉપયોગિતા} & યુઝર અનુભવ & ``ચેકઆઉટ માટે મહત્તમ 3 ક્લિક'' & યુઝર
મેટ્રિક્સ \\
\textbf{સ્કેલેબિલિટી} & વૃદ્ધિ ક્ષમતા & ``10,000 યુઝર્સ સપોર્ટ કરવા'' & લોડ
ક્ષમતા \\
\end{longtable}
}

\begin{center}
\textbf{Mermaid Diagram (Code)}
\begin{verbatim}
{Shaded}
{Highlighting}[]
graph TD
    A[આવશ્યકતાઓ] {-{-}{} B[Functional]}
    A {-{-}{} C[Non{-}Functional]}
    
    B {-{-}{} D[બિઝનેસ નિયમો]}
    B {-{-}{} E[યુઝર એક્શન્સ]}
    B {-{-}{} F[ડેટા પ્રોસેસિંગ]}
    B {-{-}{} G[એક્સટર્નલ ઇન્ટરફેસ]}
    
    C {-{-}{} H[પ્રદર્શન]}
    C {-{-}{} I[સુરક્ષા]}
    C {-{-}{} J[વિશ્વસનીયતા]}
    C {-{-}{} K[ઉપયોગિતા]}
    C {-{-}{} L[સ્કેલેબિલિટી]}
{Highlighting}
{Shaded}
\end{verbatim}
\end{center}

\textbf{સરખામણી કોષ્ટક:}

{\def\LTcaptype{none} % do not increment counter
\begin{longtable}[]{@{}lll@{}}
\toprule\noalign{}
પાસું & Functional & Non-Functional \\
\midrule\noalign{}
\endhead
\bottomrule\noalign{}
\endlastfoot
\textbf{ધ્યાન} & સિસ્ટમ શું કરે છે & સિસ્ટમ કેવી રીતે પ્રદર્શન કરે છે \\
\textbf{ટેસ્ટિંગ} & Black-box testing & Performance testing \\
\textbf{દસ્તાવેજીકરણ} & Use cases & ગુણવત્તા મેટ્રિક્સ \\
\end{longtable}
}

\end{solutionbox}
\begin{mnemonicbox}
``Functional = શું, Non-Functional = કેવી રીતે''

\end{mnemonicbox}
\subsection*{પ્રશ્ન 3(અ) અથવા [3
ગુણ]}\label{uxaaauxab0uxab6uxaa8-3uxa85-uxa85uxaa5uxab5-3-uxa97uxaa3}

\textbf{Software projects ની વ્યવસ્થા કરવાની કુશળતાનું વર્ણન કરો.}

\begin{solutionbox}

પ્રોજેક્ટ મેનેજમેન્ટ માટે સફળ સૉફ્ટવેર ડિલિવરી માટે વિવિધ કુશળતાઓની જરૂર છે.

\textbf{આવશ્યક કુશળતાઓ:}

{\def\LTcaptype{none} % do not increment counter
\begin{longtable}[]{@{}lll@{}}
\toprule\noalign{}
કુશળતા શ્રેણી & વર્ણન & ઉપયોગ \\
\midrule\noalign{}
\endhead
\bottomrule\noalign{}
\endlastfoot
\textbf{ટેકનિકલ} & ટેકનોલોજીની સમજ & આર્કિટેક્ચર નિર્ણયો \\
\textbf{નેતૃત્વ} & ટીમ પ્રેરણા & સંઘર્ષ નિરાકરણ \\
\textbf{કોમ્યુનિકેશન} & હિસ્સેદાર ક્રિયાપ્રતિક્રિયા & સ્થિતિ રિપોર્ટિંગ \\
\end{longtable}
}

\end{solutionbox}
\begin{mnemonicbox}
``ટેકનિકલ નેતૃત્વ કોમ્યુનિકેશન''

\end{mnemonicbox}
\subsection*{પ્રશ્ન 3(બ) અથવા [4
ગુણ]}\label{uxaaauxab0uxab6uxaa8-3uxaac-uxa85uxaa5uxab5-4-uxa97uxaa3}

\textbf{ટૂંકમાં સૉફ્ટવેર પ્રોજેક્ટ મેનેજરની જવાબદારી આપો.}

\begin{solutionbox}

સૉફ્ટવેર પ્રોજેક્ટ મેનેજર સમગ્ર પ્રોજેક્ટ લાઇફસાઇકલની દેખરેખ રાખે છે અને સફળ ડિલિવરી
સુનિશ્ચિત કરે છે.

\textbf{મુખ્ય જવાબદારીઓ:}

{\def\LTcaptype{none} % do not increment counter
\begin{longtable}[]{@{}lll@{}}
\toprule\noalign{}
ક્ષેત્ર & જવાબદારી & પ્રવૃત્તિઓ \\
\midrule\noalign{}
\endhead
\bottomrule\noalign{}
\endlastfoot
\textbf{પ્લાનિંગ} & પ્રોજેક્ટ રોડમેપ & શેડ્યૂલ, બજેટ, સંસાધન ફાળવણી \\
\textbf{એક્ઝિક્યુશન} & ટીમ સંકલન & કાર્ય સોંપણી, પ્રગતિ નિરીક્ષણ \\
\textbf{ગુણવત્તા} & ધોરણ પાલન & કોડ રિવ્યુ, ટેસ્ટિંગ દેખરેખ \\
\textbf{કોમ્યુનિકેશન} & હિસ્સેદાર અપડેટ્સ & સ્થિતિ રિપોર્ટ્સ, જોખમ કોમ્યુનિકેશન \\
\end{longtable}
}

\textbf{વધારાની ફરજો:}

\begin{itemize}
\tightlist
\item
  \textbf{જોખમ વ્યવસ્થાપન}: પ્રોજેક્ટ જોખમો ઓળખવા અને ઘટાડવા
\item
  \textbf{ટીમ ડેવલપમેન્ટ}: ટીમ સભ્યોને માર્ગદર્શન અને સંઘર્ષ નિરાકરણ
\end{itemize}

\begin{center}
\textbf{Mermaid Diagram (Code)}
\begin{verbatim}
{Shaded}
{Highlighting}[]
graph TD
    A[પ્રોજેક્ટ મેનેજર] {-{-}{} B[પ્લાનિંગ]}
    A {-{-}{} C[એક્ઝિક્યુશન]}
    A {-{-}{} D[ગુણવત્તા]}
    A {-{-}{} E[કોમ્યુનિકેશન]}
    A {-{-}{} F[જોખમ વ્યવસ્થાપન]}
    A {-{-}{} G[ટીમ ડેવલપમેન્ટ]}
{Highlighting}
{Shaded}
\end{verbatim}
\end{center}

\end{solutionbox}
\begin{mnemonicbox}
``પ્લાન એક્ઝિક્યુટ ગુણવત્તા કોમ્યુનિકેટ જોખમ ટીમ''

\end{mnemonicbox}
\subsection*{પ્રશ્ન 3(ક) અથવા [7
ગુણ]}\label{uxaaauxab0uxab6uxaa8-3uxa95-uxa85uxaa5uxab5-7-uxa97uxaa3}

\textbf{PERT chart -- Gantt chart ની સરખામણી સામ સામે કરો.}

\begin{solutionbox}

બંને ચાર્ટ પ્રોજેક્ટ મેનેજમેન્ટ ટૂલ્સ છે પરંતુ વિવિધ હેતુઓ સેવે છે અને અલગ લક્ષણો ધરાવે છે.

\textbf{વિગતવાર સરખામણી:}

{\def\LTcaptype{none} % do not increment counter
\begin{longtable}[]{@{}lll@{}}
\toprule\noalign{}
પાસું & PERT Chart & Gantt Chart \\
\midrule\noalign{}
\endhead
\bottomrule\noalign{}
\endlastfoot
\textbf{હેતુ} & કાર્ય અવલંબન દર્શાવવું & પ્રોજેક્ટ ટાઇમલાઇન બતાવવું \\
\textbf{માળખું} & નેટવર્ક ડાયાગ્રામ & બાર ચાર્ટ \\
\textbf{ધ્યાન} & ક્રિટિકલ પાથ વિશ્લેષણ & શેડ્યૂલ વિઝ્યુઅલાઇઝેશન \\
\textbf{સમય પ્રદર્શન} & અંદાજિત અવધિ & વાસ્તવિક તારીખો \\
\textbf{અવલંબન} & સ્પષ્ટ તીરો & ગર્ભિત જોડાણો \\
\textbf{શ્રેષ્ઠ} & જટિલ પ્રોજેક્ટ્સ & સરળ શેડ્યૂલિંગ \\
\end{longtable}
}

\textbf{વિઝ્યુઅલ રિપ્રેઝન્ટેશન:}

\begin{center}
\textbf{Mermaid Diagram (Code)}
\begin{verbatim}
{Shaded}
{Highlighting}[]
graph TD
    subgraph "PERT Chart"
      direction LR
        A[Task A] {-{-}{} C[Task C]}
        B[Task B] {-{-}{} C}
        C {-{-}{} D[Task D]}
    end
{Highlighting}
{Shaded}
\end{verbatim}
\end{center}

\begin{verbatim}
gantt
    title Gantt Chart
    dateFormat  YYYY{-MM{-}DD}
    section Development
    Task A     :a1, 2024{-01{-}01, 3d}
    Task B     :a2, 2024{-01{-}01, 2d}
    Task C     :a3, after a1 a2, 4d
    Task D     :a4, after a3, 2d
\end{verbatim}

\textbf{ક્યારે ઉપયોગ કરવો:}

{\def\LTcaptype{none} % do not increment counter
\begin{longtable}[]{@{}lll@{}}
\toprule\noalign{}
સ્થિતિ & PERT & Gantt \\
\midrule\noalign{}
\endhead
\bottomrule\noalign{}
\endlastfoot
\textbf{પ્રોજેક્ટ પ્રકાર} & સંશોધન અને વિકાસ & બાંધકામ, સૉફ્ટવેર \\
\textbf{અનિશ્ચિતતા} & ઉચ્ચ અનિશ્ચિતતા & સ્પષ્ટ કાર્યો \\
\textbf{પ્રેક્ષકો} & ટેકનિકલ ટીમ & મેનેજમેન્ટ, ક્લાયન્ટ્સ \\
\end{longtable}
}

\textbf{ફાયદાઓની સરખામણી:}

{\def\LTcaptype{none} % do not increment counter
\begin{longtable}[]{@{}ll@{}}
\toprule\noalign{}
PERT ફાયદા & Gantt ફાયદા \\
\midrule\noalign{}
\endhead
\bottomrule\noalign{}
\endlastfoot
\textbf{ક્રિટિકલ પાથ} ઓળખ & \textbf{સમજવામાં સરળ} વિઝ્યુઅલી \\
\textbf{લવચીક સમય} અંદાજ & \textbf{પ્રગતિ ટ્રેકિંગ} ક્ષમતા \\
\textbf{જોખમ વિશ્લેષણ} સપોર્ટ & \textbf{સંસાધન ફાળવણી} પ્રદર્શન \\
\end{longtable}
}

\end{solutionbox}
\begin{mnemonicbox}
``PERT = પાથ, Gantt = બાર્સ''

\end{mnemonicbox}
\subsection*{પ્રશ્ન 4(અ) [3
ગુણ]}\label{uxaaauxab0uxab6uxaa8-4uxa85-3-uxa97uxaa3}

\textbf{પ્રોજેક્ટ મોનિટરિંગ અને નિયંત્રણ પ્રક્રિયાના પગલાં આપો}

\begin{solutionbox}

પ્રોજેક્ટ મોનિટરિંગ વ્યવસ્થિત નિરીક્ષણ અને સુધારાત્મક ક્રિયાઓ દ્વારા પ્રોજેક્ટ ટ્રેક પર
રહે તેની ખાતરી કરે છે.

\textbf{મોનિટરિંગ પગલાં:}

{\def\LTcaptype{none} % do not increment counter
\begin{longtable}[]{@{}lll@{}}
\toprule\noalign{}
પગલું & પ્રવૃત્તિ & હેતુ \\
\midrule\noalign{}
\endhead
\bottomrule\noalign{}
\endlastfoot
\textbf{પ્રગતિ ટ્રેક કરવી} & વાસ્તવિક વિરુદ્ધ આયોજિત માપવું & વિચલનો ઓળખવા \\
\textbf{ગુણવત્તા મૂલ્યાંકન} & ડિલિવરેબલ્સ સમીક્ષા & ધોરણો સુનિશ્ચિત કરવા \\
\textbf{પગલાં લેવા} & સુધારાઓ લાગુ કરવા & સંરેખણ જાળવવા \\
\end{longtable}
}

\end{solutionbox}
\begin{mnemonicbox}
``ટ્રેક મૂલ્યાંકન પગલાં''

\end{mnemonicbox}
\subsection*{પ્રશ્ન 4(બ) [4
ગુણ]}\label{uxaaauxab0uxab6uxaa8-4uxaac-4-uxa97uxaa3}

\textbf{ચર્ચા કરો i)Risk Assessment ii)Risk Mitigation}

\begin{solutionbox}

\textbf{(i) Risk Assessment:}

સંભવિત પ્રોજેક્ટ જોખમો ઓળખવા અને મૂલ્યાંકન કરવાની પ્રક્રિયા.

{\def\LTcaptype{none} % do not increment counter
\begin{longtable}[]{@{}lll@{}}
\toprule\noalign{}
મૂલ્યાંકન પ્રકાર & પદ્ધતિ & આઉટપુટ \\
\midrule\noalign{}
\endhead
\bottomrule\noalign{}
\endlastfoot
\textbf{જોખમ ઓળખ} & બ્રેઇનસ્ટોર્મિંગ, ચેકલિસ્ટ્સ & જોખમ સૂચિ \\
\textbf{જોખમ વિશ્લેષણ} & સંભાવના \times પ્રભાવ & જોખમ પ્રાથમિકતા \\
\textbf{જોખમ મૂલ્યાંકન} & જોખમ મેટ્રિક્સ & કાર્ય પ્રાથમિકતાઓ \\
\end{longtable}
}

\textbf{(ii) Risk Mitigation:}

જોખમની અસર અને સંભાવના ઘટાડવાની વ્યૂહરચનાઓ.

{\def\LTcaptype{none} % do not increment counter
\begin{longtable}[]{@{}lll@{}}
\toprule\noalign{}
વ્યૂહરચના & વર્ણન & ઉદાહરણ \\
\midrule\noalign{}
\endhead
\bottomrule\noalign{}
\endlastfoot
\textbf{ટાળવું} & જોખમ સ્રોત દૂર કરવો & ટેકનોલોજી બદલવી \\
\textbf{ઘટાડવું} & અસર ઓછી કરવી & ટેસ્ટિંગ ઉમેરવું \\
\textbf{ટ્રાન્સફર કરવું} & અન્યને જોખમ સ્થાનાંતરિત કરવું & વીમો, આઉટસોર્સિંગ \\
\textbf{સ્વીકારવું} & જોખમ સાથે જીવવું & કન્ટિન્જન્સી પ્લાનિંગ \\
\end{longtable}
}

\end{solutionbox}
\begin{mnemonicbox}
``ટાળો ઘટાડો ટ્રાન્સફર સ્વીકારો''

\end{mnemonicbox}
\subsection*{પ્રશ્ન 4(ક) [7
ગુણ]}\label{uxaaauxab0uxab6uxaa8-4uxa95-7-uxa97uxaa3}

\textbf{પ્રોજેક્ટ જોખમ વ્યાખ્યાયિત કરો અને જોખમ વ્યવસ્થાપન કેવી રીતે સંચાલિત કરશો?}

\begin{solutionbox}

પ્રોજેક્ટ જોખમ એ અનિશ્ચિત ઘટના છે જે, જો થાય તો, પ્રોજેક્ટ લક્ષ્યો પર સકારાત્મક અથવા
નકારાત્મક અસર કરે છે.

\textbf{જોખમ લક્ષણો:}

{\def\LTcaptype{none} % do not increment counter
\begin{longtable}[]{@{}lll@{}}
\toprule\noalign{}
લક્ષણ & વર્ણન & ઉદાહરણ \\
\midrule\noalign{}
\endhead
\bottomrule\noalign{}
\endlastfoot
\textbf{અનિશ્ચિતતા} & થઈ શકે અથવા ન પણ થાય & ટેકનોલોજી નિષ્ફળતા \\
\textbf{પ્રભાવ} & પ્રોજેક્ટ પેરામીટર્સને અસર કરે & ખર્ચ, શેડ્યૂલ, ગુણવત્તા \\
\textbf{સંભાવના} & થવાની શક્યતા & 30\% વિલંબની તક \\
\end{longtable}
}

\textbf{જોખમ વ્યવસ્થાપન પ્રક્રિયા:}

\begin{center}
\textbf{Mermaid Diagram (Code)}
\begin{verbatim}
{Shaded}
{Highlighting}[]
graph LR
    A[જોખમ ઓળખ] {-{-}{} B[જોખમ મૂલ્યાંકન]}
    B {-{-}{} C[જોખમ પ્રાથમિકતા]}
    C {-{-}{} D[જોખમ પ્રતિભાવ યોજના]}
    D {-{-}{} E[જોખમ મોનિટરિંગ]}
    E {-{-}{} F[જોખમ નિયંત્રણ]}
    F {-{-}{} A}
{Highlighting}
{Shaded}
\end{verbatim}
\end{center}

\textbf{જોખમ વ્યવસ્થાપન પગલાં:}

{\def\LTcaptype{none} % do not increment counter
\begin{longtable}[]{@{}
  >{\raggedright\arraybackslash}p{(\linewidth - 6\tabcolsep) * \real{0.2500}}
  >{\raggedright\arraybackslash}p{(\linewidth - 6\tabcolsep) * \real{0.2500}}
  >{\raggedright\arraybackslash}p{(\linewidth - 6\tabcolsep) * \real{0.2500}}
  >{\raggedright\arraybackslash}p{(\linewidth - 6\tabcolsep) * \real{0.2500}}@{}}
\toprule\noalign{}
\begin{minipage}[b]{\linewidth}\raggedright
પગલું
\end{minipage} & \begin{minipage}[b]{\linewidth}\raggedright
પ્રવૃત્તિઓ
\end{minipage} & \begin{minipage}[b]{\linewidth}\raggedright
ટૂલ્સ
\end{minipage} & \begin{minipage}[b]{\linewidth}\raggedright
આઉટપુટ
\end{minipage} \\
\midrule\noalign{}
\endhead
\bottomrule\noalign{}
\endlastfoot
\textbf{જોખમ ઓળખ} & બ્રેઇનસ્ટોર્મિંગ, ઇન્ટરવ્યુ & ચેકલિસ્ટ્સ, SWOT & જોખમ રજિસ્ટર \\
\textbf{જોખમ મૂલ્યાંકન} & સંભાવના અને પ્રભાવ વિશ્લેષણ & જોખમ મેટ્રિક્સ & જોખમ
રેટિંગ્સ \\
\textbf{જોખમ પ્રતિભાવ} & ઘટાડવાની વ્યૂહરચના વિકસાવવી & પ્રતિભાવ ટેમ્પ્લેટ્સ &
કાર્ય યોજનાઓ \\
\textbf{જોખમ મોનિટરિંગ} & જોખમ સૂચકો ટ્રેક કરવા & ડેશબોર્ડ્સ & સ્થિતિ રિપોર્ટ્સ \\
\end{longtable}
}

\textbf{જોખમ શ્રેણીઓ:}

{\def\LTcaptype{none} % do not increment counter
\begin{longtable}[]{@{}lll@{}}
\toprule\noalign{}
શ્રેણી & ઉદાહરણો & ઘટાડવાનો અભિગમ \\
\midrule\noalign{}
\endhead
\bottomrule\noalign{}
\endlastfoot
\textbf{ટેકનિકલ} & ટેકનોલોજી અપ્રચલિતતા & પ્રૂફ ઓફ કન્સેપ્ટ \\
\textbf{પ્રોજેક્ટ} & સંસાધન અનુપલબ્ધતા & સંસાધન આયોજન \\
\textbf{બિઝનેસ} & બજાર ફેરફારો & હિસ્સેદાર સંલગ્નતા \\
\textbf{બાહ્ય} & નિયમનકારી ફેરફારો & કાનૂની સલાહ \\
\end{longtable}
}

\textbf{જોખમ પ્રતિભાવ વ્યૂહરચનાઓ:}

\begin{itemize}
\tightlist
\item
  \textbf{નકારાત્મક જોખમો (ધમકીઓ)}: ટાળવું, ટ્રાન્સફર કરવું, ઘટાડવું, સ્વીકારવું
\item
  \textbf{સકારાત્મક જોખમો (તકો)}: શોષણ કરવું, શેર કરવું, વધારવું, સ્વીકારવું
\end{itemize}

\end{solutionbox}
\begin{mnemonicbox}
``ઓળખો મૂલ્યાંકન પ્રતિભાવ મોનિટર'' + ``ટાળો ટ્રાન્સફર
ઘટાડો સ્વીકારો''

\end{mnemonicbox}
\subsection*{પ્રશ્ન 4(અ) અથવા [3
ગુણ]}\label{uxaaauxab0uxab6uxaa8-4uxa85-uxa85uxaa5uxab5-3-uxa97uxaa3}

\textbf{સૉફ્ટવેર ડિઝાઇન પ્રક્રિયાનું વર્ણન કરો અને ડિઝાઇન પદ્ધતિઓ સમજાવો}

\begin{solutionbox}

સૉફ્ટવેર ડિઝાઇન આવશ્યકતાઓને વ્યવસ્થિત અભિગમ દ્વારા અમલીકરણ માટે બ્લુપ્રિન્ટમાં
રૂપાંતરિત કરે છે.

\textbf{ડિઝાઇન પ્રક્રિયા:}

{\def\LTcaptype{none} % do not increment counter
\begin{longtable}[]{@{}lll@{}}
\toprule\noalign{}
તબક્કો & પ્રવૃત્તિ & આઉટપુટ \\
\midrule\noalign{}
\endhead
\bottomrule\noalign{}
\endlastfoot
\textbf{વિશ્લેષણ} & આવશ્યકતાઓ સમજવી & સમસ્યા વ્યાખ્યા \\
\textbf{આર્કિટેક્ચર} & ઉચ્ચ-સ્તરીય માળખું & સિસ્ટમ આર્કિટેક્ચર \\
\textbf{વિગતવાર ડિઝાઇન} & ઘટક સ્પષ્ટીકરણ & ડિઝાઇન દસ્તાવેજો \\
\end{longtable}
}

\end{solutionbox}
\begin{mnemonicbox}
``વિશ્લેષણ આર્કિટેક્ચર વિગત''

\end{mnemonicbox}
\subsection*{પ્રશ્ન 4(બ) અથવા [4
ગુણ]}\label{uxaaauxab0uxab6uxaa8-4uxaac-uxa85uxaa5uxab5-4-uxa97uxaa3}

\textbf{Cohesion and Coupling ની સરખામણી સામ સામે કરો.}

\begin{solutionbox}

બંને ખ્યાલો મોડ્યુલ ડિઝાઇન ગુણવત્તા માપે છે પરંતુ વિવિધ પાસાઓ પર ધ્યાન કેન્દ્રિત કરે છે.

\textbf{વ્યાપક સરખામણી:}

{\def\LTcaptype{none} % do not increment counter
\begin{longtable}[]{@{}
  >{\raggedright\arraybackslash}p{(\linewidth - 4\tabcolsep) * \real{0.3333}}
  >{\raggedright\arraybackslash}p{(\linewidth - 4\tabcolsep) * \real{0.3333}}
  >{\raggedright\arraybackslash}p{(\linewidth - 4\tabcolsep) * \real{0.3333}}@{}}
\toprule\noalign{}
\begin{minipage}[b]{\linewidth}\raggedright
પાસું
\end{minipage} & \begin{minipage}[b]{\linewidth}\raggedright
Cohesion
\end{minipage} & \begin{minipage}[b]{\linewidth}\raggedright
Coupling
\end{minipage} \\
\midrule\noalign{}
\endhead
\bottomrule\noalign{}
\endlastfoot
\textbf{વ્યાખ્યા} & મોડ્યુલની અંદર સંબંધની ડિગ્રી & મોડ્યુલો વચ્ચે પરસ્પર નિર્ભરતાની
ડિગ્રી \\
\textbf{લક્ષ્ય} & ઉચ્ચ cohesion ઇચ્છનીય & નીચું coupling ઇચ્છનીય \\
\textbf{ધ્યાન} & આંતરિક મોડ્યુલ માળખું & આંતર-મોડ્યુલ સંબંધો \\
\textbf{ગુણવત્તા સૂચક} & મજબૂત = બહેતર & નબળું = બહેતર \\
\end{longtable}
}

\textbf{પ્રકારોની સરખામણી:}

{\def\LTcaptype{none} % do not increment counter
\begin{longtable}[]{@{}ll@{}}
\toprule\noalign{}
Cohesion પ્રકારો (શ્રેષ્ઠથી ખરાબ) & Coupling પ્રકારો (શ્રેષ્ઠથી ખરાબ) \\
\midrule\noalign{}
\endhead
\bottomrule\noalign{}
\endlastfoot
\textbf{Functional} - એક હેતુ & \textbf{Data} - સરળ ડેટા શેરિંગ \\
\textbf{Sequential} - આઉટપુટ\rightarrowઇનપુટ & \textbf{Stamp} - ડેટા સ્ટ્રક્ચર શેરિંગ \\
\textbf{Communicational} - સમાન ડેટા & \textbf{Control} - નિયંત્રણ
માહિતી \\
\textbf{Procedural} - ક્રમિક અમલીકરણ & \textbf{External} - બાહ્ય
નિર્ભરતા \\
\textbf{Temporal} - સમાન સમય & \textbf{Common} - વૈશ્વિક ડેટા \\
\textbf{Logical} - સમાન કાર્યો & \textbf{Content} - આંતરિક ડેટા પ્રવેશ \\
\textbf{Coincidental} - કોઈ સંબંધ નથી & \\
\end{longtable}
}

\textbf{ડિઝાઇન પર પ્રભાવ:}

{\def\LTcaptype{none} % do not increment counter
\begin{longtable}[]{@{}lll@{}}
\toprule\noalign{}
પરિબળ & ઉચ્ચ Cohesion & નીચું Coupling \\
\midrule\noalign{}
\endhead
\bottomrule\noalign{}
\endlastfoot
\textbf{જાળવણીક્ષમતા} & સુધારવામાં સરળ & સ્વતંત્ર ફેરફારો \\
\textbf{પુનઃઉપયોગ} & સ્વ-સમાયેલ મોડ્યુલ્સ & લવચીક એકીકરણ \\
\textbf{ટેસ્ટિંગ} & કેન્દ્રિત ટેસ્ટ કેસ & અલગ ટેસ્ટિંગ \\
\end{longtable}
}

\end{solutionbox}
\begin{mnemonicbox}
``Cohesion = અંદર મજબૂત, Coupling = વચ્ચે નબળું''

\end{mnemonicbox}
\subsection*{પ્રશ્ન 4(ક) અથવા [7
ગુણ]}\label{uxaaauxab0uxab6uxaa8-4uxa95-uxa85uxaa5uxab5-7-uxa97uxaa3}

\textbf{સ્તરો સાથે ડેટા-ફ્લો ડાયાગ્રામ સ્કેચ કરો અને સમજાવો}

\begin{solutionbox}

ડેટા ફ્લો ડાયાગ્રામ (DFD) ગ્રાફિકલ નોટેશન વાપરીને સિસ્ટમ દ્વારા ડેટા કેવી રીતે ચાલે
છે તે બતાવે છે અને વિગતના બહુવિધ સ્તરો ધરાવે છે.

\textbf{DFD સિમ્બોલ્સ:}

{\def\LTcaptype{none} % do not increment counter
\begin{longtable}[]{@{}lll@{}}
\toprule\noalign{}
સિમ્બોલ & રજૂઆત & વર્ણન \\
\midrule\noalign{}
\endhead
\bottomrule\noalign{}
\endlastfoot
\textbf{વર્તુળ/બબલ} & પ્રોસેસ & ઇનપુટને આઉટપુટમાં રૂપાંતરિત કરે છે \\
\textbf{લંબચોરસ} & બાહ્ય એન્ટિટી & સ્રોત અથવા ગંતવ્ય \\
\textbf{ખુલ્લો લંબચોરસ} & ડેટા સ્ટોર & ડેટાનો ભંડાર \\
\textbf{તીર} & ડેટા ફ્લો & ડેટાની હિલચાલ \\
\end{longtable}
}

\textbf{DFD સ્તરો:}

\begin{center}
\textbf{Mermaid Diagram (Code)}
\begin{verbatim}
{Shaded}
{Highlighting}[]
graph LR
    A[Context Diagram Level 0] {-{-}{} B[Level 1 DFD]}
    B {-{-}{} C[Level 2 DFD]}
    C {-{-}{} D[Level 3 DFD]}
    
    E[એક પ્રોસેસ] {-{-}{} F[મુખ્ય પ્રોસેસો]}
    F {-{-}{} G[ઉપ{-}પ્રોસેસો]}
    G {-{-}{} H[વિગતવાર પ્રોસેસો]}
{Highlighting}
{Shaded}
\end{verbatim}
\end{center}

\textbf{સ્તર વર્ણનો:}

{\def\LTcaptype{none} % do not increment counter
\begin{longtable}[]{@{}llll@{}}
\toprule\noalign{}
સ્તર & અવકાશ & હેતુ & વિગત \\
\midrule\noalign{}
\endhead
\bottomrule\noalign{}
\endlastfoot
\textbf{Level 0 (Context)} & સંપૂર્ણ સિસ્ટમ & સિસ્ટમ સીમા & એક પ્રોસેસ \\
\textbf{Level 1} & મુખ્ય કાર્યો & ઉચ્ચ-સ્તરીય પ્રોસેસો & 5-7 પ્રોસેસો \\
\textbf{Level 2} & ઉપ-કાર્યો & પ્રોસેસ વિભાજન & વિગતવાર દૃશ્ય \\
\textbf{Level 3+} & બારીક વિગતો & અમલીકરણ સ્તર & ખૂબ જ વિશિષ્ટ \\
\end{longtable}
}

\textbf{ઉદાહરણ - વિદ્યાર્થી માહિતી સિસ્ટમ:}

\textbf{Level 0 (Context Diagram):}

\begin{verbatim}
[વિદ્યાર્થી] \rightarrow વિદ્યાર્થી માહિતી \rightarrow [વિદ્યાર્થી સિસ્ટમ] \rightarrow રિપોર્ટ્સ \rightarrow [એડમિન]
\end{verbatim}

\textbf{Level 1 DFD:}

\begin{center}
\textbf{Mermaid Diagram (Code)}
\begin{verbatim}
{Shaded}
{Highlighting}[]
graph LR
    A[વિદ્યાર્થી] {-{-}{} B[1.0 વિદ્યાર્થી નોંધણી]}
    B {-{-}{} C[વિદ્યાર્થી ડેટાબેસ]}
    C {-{-}{} D[2.0 રિપોર્ટ જનરેટ કરવી]}
    D {-{-}{} E[એડમિન]}
    F[શિક્ષક] {-{-}{} G[3.0 ગ્રેડ અપડેટ કરવા]}
    G {-{-}{} C}
{Highlighting}
{Shaded}
\end{verbatim}
\end{center}

\textbf{બેલેન્સિંગ નિયમો:}

\begin{itemize}
\tightlist
\item
  \textbf{ડેટા સંરક્ષણ}: દરેક સ્તરે ઇનપુટ = આઉટપુટ
\item
  \textbf{પ્રોસેસ નંબરિંગ}: સ્તરીય નંબરિંગ સિસ્ટમ
\item
  \textbf{બાહ્ય એન્ટિટીઓ}: બધા સ્તરો પર સમાન
\end{itemize}

\textbf{સ્તરીય DFDs ના ફાયદા:}

{\def\LTcaptype{none} % do not increment counter
\begin{longtable}[]{@{}lll@{}}
\toprule\noalign{}
ફાયદો & વર્ણન & લાભ \\
\midrule\noalign{}
\endhead
\bottomrule\noalign{}
\endlastfoot
\textbf{અમૂર્તતા} & જટિલતા છુપાવવી & સરળ સમજ \\
\textbf{વિઘટન} & પ્રોસેસો તોડવા & મેનેજેબલ ભાગો \\
\textbf{ચકાસણી} & પૂર્ણતા તપાસવી & ગુણવત્તા ખાતરી \\
\end{longtable}
}

\end{solutionbox}
\begin{mnemonicbox}
``Context મુખ્ય ઉપ બારીક'' + ``પ્રોસેસ એન્ટિટી સ્ટોર ફ્લો''

\end{mnemonicbox}
\subsection*{પ્રશ્ન 5(અ) [3
ગુણ]}\label{uxaaauxab0uxab6uxaa8-5uxa85-3-uxa97uxaa3}

\textbf{સારા UI ની લાક્ષણિકતાઓ આપો.}

\begin{solutionbox}

સારો યુઝર ઇન્ટરફેસ ડિઝાઇન સૉફ્ટવેર સિસ્ટમ સાથે અસરકારક યુઝર ક્રિયાપ્રતિક્રિયા
સુનિશ્ચિત કરે છે.

\textbf{UI લાક્ષણિકતાઓ:}

{\def\LTcaptype{none} % do not increment counter
\begin{longtable}[]{@{}lll@{}}
\toprule\noalign{}
લાક્ષણિકતા & વર્ણન & ફાયદો \\
\midrule\noalign{}
\endhead
\bottomrule\noalign{}
\endlastfoot
\textbf{સરળ} & સમજવામાં સરળ & શીખવાની વળાંક ઘટાડે છે \\
\textbf{સુસંગત} & એકસમાન વર્તન & અનુમાનિત ક્રિયાપ્રતિક્રિયા \\
\textbf{પ્રતિસાદશીલ} & ઝડપી ફીડબેક & યુઝર સંતુષ્ટતા \\
\end{longtable}
}

\end{solutionbox}
\begin{mnemonicbox}
``સરળ સુસંગત પ્રતિસાદશીલ''

\end{mnemonicbox}
\subsection*{પ્રશ્ન 5(બ) [4
ગુણ]}\label{uxaaauxab0uxab6uxaa8-5uxaac-4-uxa97uxaa3}

\textbf{સંક્ષિપ્તમાં Unit testing સમજાવો}

\begin{solutionbox}

યુનિટ ટેસ્ટિંગ સાચી કાર્યક્ષમતા સુનિશ્ચિત કરવા માટે વ્યક્તિગત સૉફ્ટવેર ઘટકોને અલગતામાં
ચકાસે છે.

\textbf{યુનિટ ટેસ્ટિંગ ઝાંખી:}

{\def\LTcaptype{none} % do not increment counter
\begin{longtable}[]{@{}lll@{}}
\toprule\noalign{}
પાસું & વર્ણન & હેતુ \\
\midrule\noalign{}
\endhead
\bottomrule\noalign{}
\endlastfoot
\textbf{અવકાશ} & વ્યક્તિગત મોડ્યુલ્સ/ફંક્શન્સ & ઘટક ચકાસણી \\
\textbf{અલગતા} & અલગતામાં ટેસ્ટ & સ્વતંત્ર ચકાસણી \\
\textbf{સ્વચાલન} & સ્વચાલિત ટેસ્ટ અમલીકરણ & કાર્યક્ષમ ટેસ્ટિંગ \\
\textbf{વહેલી શોધ} & વહેલો બગ શોધ & ખર્ચ-અસરકારક ડિબગિંગ \\
\end{longtable}
}

\textbf{ટેસ્ટિંગ પ્રક્રિયા:}

\begin{center}
\textbf{Mermaid Diagram (Code)}
\begin{verbatim}
{Shaded}
{Highlighting}[]
graph LR
    A[ટેસ્ટ કેસ લખવા] {-{-}{} B[ટેસ્ટ અમલ કરવા]}
    B {-{-}{} C[પરિણામો વિશ્લેષણ]}
    C {-{-}{} D[ખામીઓ સુધારવી]}
    D {-{-}{} B}
{Highlighting}
{Shaded}
\end{verbatim}
\end{center}

\textbf{ફાયદા:}

\begin{itemize}
\tightlist
\item
  \textbf{વહેલી બગ શોધ} સુધારવાનો ખર્ચ ઘટાડે છે
\item
  \textbf{કોડ ગુણવત્તા} ટેસ્ટિંગ શિસ્ત દ્વારા સુધારણા
\item
  \textbf{રિગ્રેશન ટેસ્ટિંગ} ભાવિ ભંગાણ અટકાવે છે
\end{itemize}

\end{solutionbox}
\begin{mnemonicbox}
``અવકાશ અલગતા સ્વચાલન વહેલી''

\end{mnemonicbox}
\subsection*{પ્રશ્ન 5(ક) [7
ગુણ]}\label{uxaaauxab0uxab6uxaa8-5uxa95-7-uxa97uxaa3}

\textbf{ટ્રેન રિઝર્વેશન સિસ્ટમની activity diagrams બનાવો, દરેક પગલું સમજાવો.}

\begin{solutionbox}

Activity Diagram યુઝર વિનંતીથી ટિકિટ પુષ્ટિ સુધી ટ્રેન રિઝર્વેશન સિસ્ટમનો વર્કફ્લો
બતાવે છે.

\begin{center}
\textbf{Mermaid Diagram (Code)}
\begin{verbatim}
{Shaded}
{Highlighting}[]
graph TD
    A[શરૂઆત] {-{-}{} B[યુઝર લોગિન]}
    B {-{-}{} C\{વૈધ ઓળખપત્રો?\}}
    C {-{-}{}|ના| B}
    C {-{-}{}|હા| D[ટ્રેન સર્ચ કરવી]}
    D {-{-}{} E[ટ્રેન પસંદ કરવી]}
    E {-{-}{} F[સીટ પસંદ કરવી]}
    F {-{-}{} G\{સીટ ઉપલબ્ધ?\}}
    G {-{-}{}|ના| F}
    G {-{-}{}|હા| H[પેસેન્જર વિગતો દાખલ કરવી]}
    H {-{-}{} I[બુકિંગ સમીક્ષા]}
    I {-{-}{} J\{બુકિંગ કન્ફર્મ કરવું?\}}
    J {-{-}{}|ના| D}
    J {-{-}{}|હા| K[પેમેન્ટ પ્રોસેસ કરવું]}
    K {-{-}{} L\{પેમેન્ટ સફળ?\}}
    L {-{-}{}|ના| K}
    L {-{-}{}|હા| M[ટિકિટ જનરેટ કરવું]}
    M {-{-}{} N[પુષ્ટિ મોકલવી]}
    N {-{-}{} O[અંત]}
{Highlighting}
{Shaded}
\end{verbatim}
\end{center}

\textbf{પગલા-દર-પગલાની સમજૂતી:}

{\def\LTcaptype{none} % do not increment counter
\begin{longtable}[]{@{}llll@{}}
\toprule\noalign{}
પગલું & પ્રવૃત્તિ & વર્ણન & નિર્ણય બિંદુઓ \\
\midrule\noalign{}
\endhead
\bottomrule\noalign{}
\endlastfoot
\textbf{1} & યુઝર લોગિન & યુઝર ઓળખપત્રો ચકાસવા & વૈધ/અવૈધ \\
\textbf{2} & ટ્રેન સર્ચ કરવી & રૂટ/તારીખ માટે ઉપલબ્ધ ટ્રેન શોધવી & પરિણામો
મળ્યા \\
\textbf{3} & ટ્રેન પસંદ કરવી & વિશિષ્ટ ટ્રેન પસંદ કરવી & ટ્રેન પસંદગી \\
\textbf{4} & સીટ પસંદ કરવી & સીટ પસંદગીઓ પસંદ કરવી & ઉપલબ્ધતા તપાસ \\
\textbf{5} & વિગતો દાખલ કરવી & પેસેન્જર માહિતી પ્રદાન કરવી & ડેટા ચકાસણી \\
\textbf{6} & બુકિંગ સમીક્ષા & બુકિંગ વિગતો પુષ્ટિ કરવી & યુઝર પુષ્ટિ \\
\textbf{7} & પેમેન્ટ પ્રોસેસ કરવું & પેમેન્ટ ટ્રાન્ઝેક્શન હેન્ડલ કરવું & સફળ/નિષ્ફળ \\
\textbf{8} & ટિકિટ જનરેટ કરવું** & ટિકિટ દસ્તાવેજ બનાવવું & ટિકિટ બનાવટ \\
\textbf{9} & પુષ્ટિ મોકલવી & યુઝરને પુષ્ટિ પહોંચાડવી & પ્રક્રિયા પૂર્ણ \\
\end{longtable}
}

\textbf{પ્રવૃત્તિ પ્રકારો:}

{\def\LTcaptype{none} % do not increment counter
\begin{longtable}[]{@{}llll@{}}
\toprule\noalign{}
પ્રકાર & સિમ્બોલ & હેતુ & ઉદાહરણો \\
\midrule\noalign{}
\endhead
\bottomrule\noalign{}
\endlastfoot
\textbf{એક્શન} & ગોળાકાર લંબચોરસ & પ્રવૃત્તિ કરવી & ટ્રેન સર્ચ કરવી \\
\textbf{નિર્ણય} & હીરો & પાથ પસંદ કરવો & વૈધ ઓળખપત્રો? \\
\textbf{શરૂઆત/અંત} & વર્તુળ & શરૂઆત/સમાપ્તિ & શરૂઆત, અંત \\
\textbf{ફ્લો} & તીર & ક્રમ બતાવવો & પ્રોસેસ ફ્લો \\
\end{longtable}
}

\textbf{સમાંતર પ્રવૃત્તિઓ:}

\begin{itemize}
\tightlist
\item
  પેમેન્ટ પ્રોસેસિંગ અને સીટ રિઝર્વેશન એકસાથે થઈ શકે છે
\item
  પુષ્ટિ ઇમેઇલ અને SMS સમાંતરમાં મોકલી શકાય છે
\end{itemize}

\textbf{એક્સેપ્શન હેન્ડલિંગ:}

\begin{itemize}
\tightlist
\item
  \textbf{લોગિન નિષ્ફળતા}: લોગિન સ્ક્રીન પર પાછા ફરવું
\item
  \textbf{કોઈ સીટ નથી}: વિવિધ સીટ પસંદગીની મંજૂરી
\item
  \textbf{પેમેન્ટ નિષ્ફળતા}: પેમેન્ટ વિકલ્પો ફરી પ્રયાસ
\item
  \textbf{સિસ્ટમ એરર}: એરર મેસેજ બતાવવો અને ફરી શરૂ કરવું
\end{itemize}

\end{solutionbox}
\begin{mnemonicbox}
``લોગિન સર્ચ સિલેક્ટ ચૂઝ એન્ટર રિવ્યુ પે જનરેટ સેન્ડ''

\end{mnemonicbox}
\subsection*{પ્રશ્ન 5(અ) અથવા [3
ગુણ]}\label{uxaaauxab0uxab6uxaa8-5uxa85-uxa85uxaa5uxab5-3-uxa97uxaa3}

\textbf{Verification, Validation ની સરખામણી સામ સામે કરો.}

\begin{solutionbox}

બંને ગુણવત્તા ખાતરીની પ્રવૃત્તિઓ છે પરંતુ સાચકીના વિવિધ પાસાઓ પર ધ્યાન કેન્દ્રિત કરે છે.

\textbf{Verification વિરુદ્ધ Validation:}

{\def\LTcaptype{none} % do not increment counter
\begin{longtable}[]{@{}
  >{\raggedright\arraybackslash}p{(\linewidth - 4\tabcolsep) * \real{0.3333}}
  >{\raggedright\arraybackslash}p{(\linewidth - 4\tabcolsep) * \real{0.3333}}
  >{\raggedright\arraybackslash}p{(\linewidth - 4\tabcolsep) * \real{0.3333}}@{}}
\toprule\noalign{}
\begin{minipage}[b]{\linewidth}\raggedright
પાસું
\end{minipage} & \begin{minipage}[b]{\linewidth}\raggedright
Verification
\end{minipage} & \begin{minipage}[b]{\linewidth}\raggedright
Validation
\end{minipage} \\
\midrule\noalign{}
\endhead
\bottomrule\noalign{}
\endlastfoot
\textbf{પ્રશ્ન} & ``શું આપણે સાચું બનાવી રહ્યા છીએ?'' & ``શું આપણે સાચી વસ્તુ બનાવી
રહ્યા છીએ?'' \\
\textbf{ધ્યાન} & પ્રક્રિયાની સાચકી & ઉત્પાદનની સાચકી \\
\textbf{પદ્ધતિ} & સમીક્ષાઓ, નિરીક્ષણો & ટેસ્ટિંગ, યુઝર ફીડબેક \\
\end{longtable}
}

\end{solutionbox}
\begin{mnemonicbox}
``Verification = સાચી પ્રક્રિયા, Validation = સાચું
ઉત્પાદન''

\end{mnemonicbox}
\subsection*{પ્રશ્ન 5(બ) અથવા [4
ગુણ]}\label{uxaaauxab0uxab6uxaa8-5uxaac-uxa85uxaa5uxab5-4-uxa97uxaa3}

\textbf{Testing ની વ્યાખ્યા સમજાવો કોઈપણ બે Testing ના પ્રકારનું વર્ણન કરો}

\begin{solutionbox}

ટેસ્ટિંગ એ ભૂલો શોધવા અને તે આવશ્યકતાઓ પૂરી કરે છે તે ચકાસવા માટે સૉફ્ટવેરનું મૂલ્યાંકન
કરવાની પ્રક્રિયા છે.

\textbf{ટેસ્ટિંગ વ્યાખ્યા:} ખામીઓ શોધવા અને કાર્યક્ષમતા ચકાસવા માટે સૉફ્ટવેરની
વ્યવસ્થિત તપાસ.

\textbf{બે ટેસ્ટિંગ પ્રકારો:}

\textbf{(1) Black Box Testing:}

{\def\LTcaptype{none} % do not increment counter
\begin{longtable}[]{@{}lll@{}}
\toprule\noalign{}
પાસું & વર્ણન & ઉદાહરણ \\
\midrule\noalign{}
\endhead
\bottomrule\noalign{}
\endlastfoot
\textbf{અભિગમ} & આંતરિક માળખું જાણ્યા વિના ટેસ્ટ & ઇનપુટ/આઉટપુટ ટેસ્ટિંગ \\
\textbf{ધ્યાન} & કાર્યાત્મક આવશ્યકતાઓ & લોગિન ચકાસણી \\
\textbf{તકનીક} & સમકક્ષતા વિભાજન & વૈધ/અવૈધ ઇનપુટ્સ \\
\textbf{ટેસ્ટર} & બાહ્ય દૃષ્ટિકોણ & યુઝર સ્વીકૃતિ \\
\end{longtable}
}

\textbf{(2) White Box Testing:}

{\def\LTcaptype{none} % do not increment counter
\begin{longtable}[]{@{}lll@{}}
\toprule\noalign{}
પાસું & વર્ણન & ઉદાહરણ \\
\midrule\noalign{}
\endhead
\bottomrule\noalign{}
\endlastfoot
\textbf{અભિગમ} & કોડ માળખાના જ્ઞાન સાથે ટેસ્ટ & પાથ કવરેજ \\
\textbf{ધ્યાન} & આંતરિક તર્ક & કોડ શાખાઓ \\
\textbf{તકનીક} & સ્ટેટમેન્ટ કવરેજ & બધી લાઇનો અમલ \\
\textbf{ટેસ્ટર} & ડેવલપર દૃષ્ટિકોણ & યુનિટ ટેસ્ટિંગ \\
\end{longtable}
}

\textbf{સરખામણી:}

{\def\LTcaptype{none} % do not increment counter
\begin{longtable}[]{@{}lll@{}}
\toprule\noalign{}
પરિબળ & Black Box & White Box \\
\midrule\noalign{}
\endhead
\bottomrule\noalign{}
\endlastfoot
\textbf{જ્ઞાન} & કોડ જ્ઞાન નથી & સંપૂર્ણ કોડ જ્ઞાન \\
\textbf{કવરેજ} & કાર્યાત્મક કવરેજ & માળખાકીય કવરેજ \\
\textbf{સ્તર} & સિસ્ટમ સ્તર & યુનિટ સ્તર \\
\end{longtable}
}

\end{solutionbox}
\begin{mnemonicbox}
``Black = બાહ્ય, White = આંતરિક''

\end{mnemonicbox}
\subsection*{પ્રશ્ન 5(ક) અથવા [7
ગુણ]}\label{uxaaauxab0uxab6uxaa8-5uxa95-uxa85uxaa5uxab5-7-uxa97uxaa3}

\textbf{દરેક Coding standards અને માર્ગદર્શિકાઓનું વર્ણન કરો.}

\begin{solutionbox}

કોડિંગ સ્ટાન્ડર્ડ્સ એ સુસંગત, જાળવી શકાય તેવા અને વાંચી શકાય તેવા કોડ લખવા માટેના
નિયમો અને પરંપરાઓનો સમૂહ છે.

\textbf{કોડિંગ સ્ટાન્ડર્ડ્સનો હેતુ:}

{\def\LTcaptype{none} % do not increment counter
\begin{longtable}[]{@{}lll@{}}
\toprule\noalign{}
ફાયદો & વર્ણન & પ્રભાવ \\
\midrule\noalign{}
\endhead
\bottomrule\noalign{}
\endlastfoot
\textbf{વાંચી શકાય તેવું} & સમજવામાં સરળ કોડ & ઝડપી મેઇન્ટેનન્સ \\
\textbf{સુસંગતતા} & એકસમાન કોડિંગ શૈલી & ટીમ સહયોગ \\
\textbf{જાળવણીક્ષમતા} & સુધારવામાં સરળ & ઘટેલા ખર્ચ \\
\textbf{ગુણવત્તા} & ઓછી ખામીઓ & વિશ્વસનીય સૉફ્ટવેર \\
\end{longtable}
}

\textbf{મુખ્ય કોડિંગ સ્ટાન્ડર્ડ્સ શ્રેણીઓ:}

\textbf{(1) નામકરણ પરંપરાઓ:}

{\def\LTcaptype{none} % do not increment counter
\begin{longtable}[]{@{}
  >{\raggedright\arraybackslash}p{(\linewidth - 6\tabcolsep) * \real{0.2500}}
  >{\raggedright\arraybackslash}p{(\linewidth - 6\tabcolsep) * \real{0.2500}}
  >{\raggedright\arraybackslash}p{(\linewidth - 6\tabcolsep) * \real{0.2500}}
  >{\raggedright\arraybackslash}p{(\linewidth - 6\tabcolsep) * \real{0.2500}}@{}}
\toprule\noalign{}
\begin{minipage}[b]{\linewidth}\raggedright
તત્વ
\end{minipage} & \begin{minipage}[b]{\linewidth}\raggedright
સ્ટાન્ડર્ડ
\end{minipage} & \begin{minipage}[b]{\linewidth}\raggedright
ઉદાહરણ
\end{minipage} & \begin{minipage}[b]{\linewidth}\raggedright
હેતુ
\end{minipage} \\
\midrule\noalign{}
\endhead
\bottomrule\noalign{}
\endlastfoot
\textbf{વેરિએબલ્સ} & camelCase & userName, totalAmount & સ્પષ્ટ ઓળખ \\
\textbf{કોન્સ્ટન્ટ્સ} & UPPER\_CASE & MAX\_SIZE, DEFAULT\_VALUE & કોન્સ્ટન્ટ્સને
અલગ પાડવા \\
\textbf{ફંક્શન્સ} & વર્ણનાત્મક ક્રિયાપદ & calculateTax(), validateInput() &
ક્રિયા સ્પષ્ટતા \\
\textbf{ક્લાસીસ} & PascalCase & CustomerAccount, OrderManager & પ્રકાર
ઓળખ \\
\end{longtable}
}

\textbf{(2) કોડ માળખું:}

{\def\LTcaptype{none} % do not increment counter
\begin{longtable}[]{@{}llll@{}}
\toprule\noalign{}
પાસું & માર્ગદર્શિકા & ઉદાહરણ & ફાયદો \\
\midrule\noalign{}
\endhead
\bottomrule\noalign{}
\endlastfoot
\textbf{ઇન્ડેન્ટેશન} & સુસંગત અંતર & 4 સ્પેસ અથવા 1 ટેબ & વિઝ્યુઅલ હાયરાર્કી \\
\textbf{લાઇનની લંબાઈ} & મહત્તમ 80-120 અક્ષરો & લાંબી લાઇનો તોડવી & સ્ક્રીન
વાંચન \\
\textbf{બ્રેસીસ} & ઓપનિંગ બ્રેસ શૈલી & સમાન લાઇન વિરુદ્ધ નવી લાઇન & સુસંગતતા \\
\textbf{કોમેન્ટ્સ} & અર્થપૂર્ણ વર્ણનો & // ટેક્સ રકમ ગણતરી & કોડ દસ્તાવેજીકરણ \\
\end{longtable}
}

\textbf{(3) કોડ ગોઠવણી:}

\begin{center}
\textbf{Mermaid Diagram (Code)}
\begin{verbatim}
{Shaded}
{Highlighting}[]
graph TD
    A[કોડ ગોઠવણી] {-{-}{} B[ફાઇલ માળખું]}
    A {-{-}{} C[ફંક્શન સાઇઝ]}
    A {-{-}{} D[ક્લાસ ડિઝાઇન]}
    
    B {-{-}{} E[એક જવાબદારી]}
    C {-{-}{} F[નાના ફંક્શન્સ]}
    D {-{-}{} G[સ્પષ્ટ ઇન્ટરફેસ]}
{Highlighting}
{Shaded}
\end{verbatim}
\end{center}

{\def\LTcaptype{none} % do not increment counter
\begin{longtable}[]{@{}
  >{\raggedright\arraybackslash}p{(\linewidth - 6\tabcolsep) * \real{0.2500}}
  >{\raggedright\arraybackslash}p{(\linewidth - 6\tabcolsep) * \real{0.2500}}
  >{\raggedright\arraybackslash}p{(\linewidth - 6\tabcolsep) * \real{0.2500}}
  >{\raggedright\arraybackslash}p{(\linewidth - 6\tabcolsep) * \real{0.2500}}@{}}
\toprule\noalign{}
\begin{minipage}[b]{\linewidth}\raggedright
સિદ્ધાંત
\end{minipage} & \begin{minipage}[b]{\linewidth}\raggedright
માર્ગદર્શિકા
\end{minipage} & \begin{minipage}[b]{\linewidth}\raggedright
મર્યાદા
\end{minipage} & \begin{minipage}[b]{\linewidth}\raggedright
ફાયદો
\end{minipage} \\
\midrule\noalign{}
\endhead
\bottomrule\noalign{}
\endlastfoot
\textbf{ફાઇલ ગોઠવણી} & એક ફાઇલમાં એક ક્લાસ & સંબંધિત ફંક્શન્સ ગ્રુપ કરેલા & સરળ
નેવિગેશન \\
\textbf{ફંક્શનની લંબાઈ} & ફંક્શન્સ નાના રાખવા & મહત્તમ 20-30 લાઇન & બહેતર
ટેસ્ટિંગ \\
\textbf{ક્લાસ સાઇઝ} & એક જવાબદારી & કેન્દ્રિત હેતુ & જાળવણીક્ષમતા \\
\textbf{મોડ્યુલ કપ્લિંગ} & નિર્ભરતા ઓછી કરવી & લૂઝ કપ્લિંગ & લવચીકતા \\
\end{longtable}
}

\textbf{(4) દસ્તાવેજીકરણ સ્ટાન્ડર્ડ્સ:}

{\def\LTcaptype{none} % do not increment counter
\begin{longtable}[]{@{}
  >{\raggedright\arraybackslash}p{(\linewidth - 6\tabcolsep) * \real{0.2500}}
  >{\raggedright\arraybackslash}p{(\linewidth - 6\tabcolsep) * \real{0.2500}}
  >{\raggedright\arraybackslash}p{(\linewidth - 6\tabcolsep) * \real{0.2500}}
  >{\raggedright\arraybackslash}p{(\linewidth - 6\tabcolsep) * \real{0.2500}}@{}}
\toprule\noalign{}
\begin{minipage}[b]{\linewidth}\raggedright
પ્રકાર
\end{minipage} & \begin{minipage}[b]{\linewidth}\raggedright
ફોર્મેટ
\end{minipage} & \begin{minipage}[b]{\linewidth}\raggedright
સામગ્રી
\end{minipage} & \begin{minipage}[b]{\linewidth}\raggedright
ઉદાહરણ
\end{minipage} \\
\midrule\noalign{}
\endhead
\bottomrule\noalign{}
\endlastfoot
\textbf{હેડર કોમેન્ટ્સ} & ફાઇલ વર્ણન & હેતુ, લેખક, તારીખ &
\texttt{//\ ગ્રાહક\ વ્યવસ્થાપન\ મોડ્યુલ} \\
\textbf{ફંક્શન કોમેન્ટ્સ} & પેરામીટર વર્ણન & ઇનપુટ/આઉટપુટ સ્પેક્સ &
\texttt{@param\ userId\ -\ યુનીક\ ઓળખ} \\
\textbf{ઇનલાઇન કોમેન્ટ્સ} & જટિલ તર્ક & શા માટે, શું નહીં &
\texttt{//\ પર્ફોર્મન્સ\ માટે\ બાઇનરી\ સર્ચ} \\
\textbf{API દસ્તાવેજીકરણ} & પબ્લિક ઇન્ટરફેસ & ઉપયોગ ઉદાહરણો & મેથડ સહી \\
\end{longtable}
}

\textbf{(5) એરર હેન્ડલિંગ:}

{\def\LTcaptype{none} % do not increment counter
\begin{longtable}[]{@{}
  >{\raggedright\arraybackslash}p{(\linewidth - 6\tabcolsep) * \real{0.2500}}
  >{\raggedright\arraybackslash}p{(\linewidth - 6\tabcolsep) * \real{0.2500}}
  >{\raggedright\arraybackslash}p{(\linewidth - 6\tabcolsep) * \real{0.2500}}
  >{\raggedright\arraybackslash}p{(\linewidth - 6\tabcolsep) * \real{0.2500}}@{}}
\toprule\noalign{}
\begin{minipage}[b]{\linewidth}\raggedright
પ્રેક્ટિસ
\end{minipage} & \begin{minipage}[b]{\linewidth}\raggedright
વર્ણન
\end{minipage} & \begin{minipage}[b]{\linewidth}\raggedright
ઉદાહરણ
\end{minipage} & \begin{minipage}[b]{\linewidth}\raggedright
હેતુ
\end{minipage} \\
\midrule\noalign{}
\endhead
\bottomrule\noalign{}
\endlastfoot
\textbf{એક્સેપ્શન હેન્ડલિંગ} & try-catch બ્લોક્સ વાપરવા &
\texttt{try\ \{\ ...\ \}\ catch\ (Exception\ e)} & ગ્રેસફુલ ફેઇલ્યુર \\
\textbf{એરર મેસેજ} & અર્થપૂર્ણ સંદેશા & ``અવૈધ ઇમેઇલ ફોર્મેટ'' & યુઝર માર્ગદર્શન \\
\textbf{લોગિંગ} & એરર વિગતો રેકોર્ડ કરવી &
\texttt{log.error("ડેટાબેસ\ કનેક્શન\ નિષ્ફળ")} & ડિબગિંગ સપોર્ટ \\
\textbf{વેલિડેશન} & ઇનપુટ તપાસ & null વેલ્યુઝ તપાસવી & એરર અટકાવવા \\
\end{longtable}
}

\textbf{(6) પર્ફોર્મન્સ માર્ગદર્શિકાઓ:}

{\def\LTcaptype{none} % do not increment counter
\begin{longtable}[]{@{}
  >{\raggedright\arraybackslash}p{(\linewidth - 6\tabcolsep) * \real{0.2500}}
  >{\raggedright\arraybackslash}p{(\linewidth - 6\tabcolsep) * \real{0.2500}}
  >{\raggedright\arraybackslash}p{(\linewidth - 6\tabcolsep) * \real{0.2500}}
  >{\raggedright\arraybackslash}p{(\linewidth - 6\tabcolsep) * \real{0.2500}}@{}}
\toprule\noalign{}
\begin{minipage}[b]{\linewidth}\raggedright
ક્ષેત્ર
\end{minipage} & \begin{minipage}[b]{\linewidth}\raggedright
સ્ટાન્ડર્ડ
\end{minipage} & \begin{minipage}[b]{\linewidth}\raggedright
ઉદાહરણ
\end{minipage} & \begin{minipage}[b]{\linewidth}\raggedright
પ્રભાવ
\end{minipage} \\
\midrule\noalign{}
\endhead
\bottomrule\noalign{}
\endlastfoot
\textbf{મેમરી ઉપયોગ} & મેમરી લીક્સ ટાળવા & રિસોર્સીસ બંધ કરવા & સિસ્ટમ
સ્થિરતા \\
\textbf{એલ્ગોરિધમ પસંદગી} & કાર્યક્ષમ એલ્ગોરિધમ્સ & યોગ્ય ડેટા સ્ટ્રક્ચર વાપરવા &
પ્રતિભાવ સમય \\
\textbf{ડેટાબેસ એક્સેસ} & ક્વેરીઝ ઓછી કરવી & કનેક્શન પૂલિંગ વાપરવું & સ્કેલેબિલિટી \\
\textbf{કોડ ઓપ્ટિમાઇઝેશન} & અકાળ ઓપ્ટિમાઇઝેશન ટાળવું & ઓપ્ટિમાઇઝ કરતા પહેલા
પ્રોફાઇલ & જાળવણીક્ષમતા \\
\end{longtable}
}

\textbf{કોડ રિવ્યુ સ્ટાન્ડર્ડ્સ:}

\begin{center}
\textbf{Mermaid Diagram (Code)}
\begin{verbatim}
{Shaded}
{Highlighting}[]
graph LR
    A[કોડ લખાયેલો] {-{-}{} B[સેલ્ફ રિવ્યુ]}
    B {-{-}{} C[પીઅર રિવ્યુ]}
    C {-{-}{} D[ટીમ લીડ રિવ્યુ]}
    D {-{-}{} E[કોડ મંજૂર]}
    E {-{-}{} F[મેઇન માં મર્જ]}
{Highlighting}
{Shaded}
\end{verbatim}
\end{center}

\textbf{રિવ્યુ ચેકલિસ્ટ:}

{\def\LTcaptype{none} % do not increment counter
\begin{longtable}[]{@{}lll@{}}
\toprule\noalign{}
શ્રેણી & તપાસ વસ્તુઓ & હેતુ \\
\midrule\noalign{}
\endhead
\bottomrule\noalign{}
\endlastfoot
\textbf{કાર્યક્ષમતા} & આવશ્યકતાઓ પૂરી, એજ કેસ હેન્ડલ કરેલા & શુદ્ધતા \\
\textbf{સ્ટાન્ડર્ડ્સ} & નામકરણ, ફોર્મેટિંગ, દસ્તાવેજીકરણ & સુસંગતતા \\
\textbf{સુરક્ષા} & ઇનપુટ વેલિડેશન, ઓથેન્ટિકેશન & સુરક્ષા \\
\textbf{પર્ફોર્મન્સ} & કાર્યક્ષમ એલ્ગોરિધમ્સ, રિસોર્સ ઉપયોગ & સ્કેલેબિલિટી \\
\end{longtable}
}

\textbf{સ્ટાન્ડર્ડ્સ અનુસરવાના ફાયદા:}

{\def\LTcaptype{none} % do not increment counter
\begin{longtable}[]{@{}lll@{}}
\toprule\noalign{}
ફાયદો & વર્ણન & લાંબા ગાળાનો પ્રભાવ \\
\midrule\noalign{}
\endhead
\bottomrule\noalign{}
\endlastfoot
\textbf{ટીમ પ્રોડક્ટિવિટી} & ઝડપી વિકાસ & ડેવલપમેન્ટ સમય ઘટાડ્યો \\
\textbf{કોડ ગુણવત્તા} & ઓછા બગ્સ & નીચા મેઇન્ટેનન્સ ખર્ચ \\
\textbf{જ્ઞાન ટ્રાન્સફર} & સરળ સમજ & સરળ ટીમ ટ્રાન્ઝિશન \\
\textbf{ટૂલ સપોર્ટ} & બહેતર IDE સપોર્ટ & વધારેલ ડેવલપમેન્ટ અનુભવ \\
\end{longtable}
}

\textbf{અમલીકરણ વ્યૂહરચના:}

\begin{enumerate}
\tightlist
\item
  \textbf{માર્ગદર્શિકાઓ સ્થાપિત કરવી}: ટીમ-વિશિષ્ટ કોડિંગ સ્ટાન્ડર્ડ્સ દસ્તાવેજ
  બનાવવો
\item
  \textbf{ટૂલ ઇન્ટિગ્રેશન}: સ્વચાલિત ફોર્મેટિંગ અને લિન્ટિંગ ટૂલ્સ વાપરવા
\item
  \textbf{તાલીમ}: કોડિંગ શ્રેષ્ઠ પ્રેક્ટિસિસ પર વર્કશોપ્સ કરાવવી
\item
  \textbf{અમલીકરણ}: કોડ રિવ્યુ પ્રક્રિયામાં સ્ટાન્ડર્ડ્સ સામેલ કરવા
\item
  \textbf{સતત સુધારણા}: ટીમ ફીડબેક પર આધારિત નિયમિત અપડેટ્સ
\end{enumerate}

\textbf{લોકપ્રિય કોડિંગ સ્ટાન્ડર્ડ્સ:}

{\def\LTcaptype{none} % do not increment counter
\begin{longtable}[]{@{}llll@{}}
\toprule\noalign{}
ભાષા & સ્ટાન્ડર્ડ & સંસ્થા & ધ્યાન \\
\midrule\noalign{}
\endhead
\bottomrule\noalign{}
\endlastfoot
\textbf{Java} & Google Java Style & Google & વ્યાપક માર્ગદર્શિકાઓ \\
\textbf{Python} & PEP 8 & Python Software Foundation & Pythonic કોડ \\
\textbf{JavaScript} & Airbnb Style & Airbnb & આધુનિક JS પ્રેક્ટિસિસ \\
\textbf{C\#} & Microsoft Guidelines & Microsoft & .NET ઇકોસિસ્ટમ \\
\end{longtable}
}

\end{solutionbox}
\begin{mnemonicbox}
``નામ માળખું ગોઠવણી દસ્તાવેજ હેન્ડલ પર્ફોર્મ રિવ્યુ''

\end{mnemonicbox}

\end{document}
