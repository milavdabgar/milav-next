\documentclass[10pt,a4paper]{article}

% content/resources/templates/preamble.tex
\usepackage[margin=0.6in]{geometry}
\author{Milav Dabgar}
\usepackage{amsmath,amssymb,amsthm}
\usepackage{booktabs}
\usepackage{multirow}
\usepackage{xcolor}
\usepackage{tcolorbox}
\tcbuselibrary{breakable,skins}
\usepackage[colorlinks=true,linkcolor=blue]{hyperref}
\usepackage{titlesec}
\usepackage{enumitem}
\usepackage{tikz}
\usepackage{pgfplots}
\usepackage{circuitikz}
\usepackage[version=4]{mhchem}
\usepackage{longtable}
\usepackage{array}
\usepackage{float}
\usepackage{caption}
\usepackage{listings}

\lstset{
  basicstyle=\small\ttfamily,
  breaklines=true,
  breakatwhitespace=false,
  postbreak=\mbox{\textcolor{red}{$\hookrightarrow$}\space},
  float=false,
  numbers=left,
  numberstyle=\tiny\color{gray},
  numbersep=10pt,
  xleftmargin=2em,
  keywordstyle=\color{blue},
  commentstyle=\color{green!60!black},
  stringstyle=\color{purple},
  backgroundcolor=\color{gray!5},
  showstringspaces=false,
  tabsize=2,
  captionpos=b,
  keepspaces=true,
  columns=flexible
}

\pgfplotsset{compat=1.18}
\usetikzlibrary{shapes,arrows,positioning,calc,patterns,decorations.pathmorphing,decorations.markings,arrows.meta}

% Color scheme
\definecolor{headcolor}{RGB}{0,102,204}
\definecolor{keycolor}{RGB}{220,20,60}
\definecolor{solutioncolor}{RGB}{34,139,34}
\definecolor{mnemoniccolor}{RGB}{148,0,211}
\definecolor{codecolor}{RGB}{0,0,100}

% Spacing
\setlength{\parskip}{3pt}
\setlist[itemize]{nosep}
\setlist[enumerate]{nosep}

% Title formatting
\titleformat{\section}{\Large\bfseries\color{headcolor}}{\thesection}{1em}{}
\titleformat{\subsection}{\large\bfseries\color{headcolor}}{\thesubsection}{1em}{}

% Pandoc tightlist compatibility
\providecommand{\tightlist}{%
  \setlength{\itemsep}{0pt}\setlength{\parskip}{0pt}}

% Pandoc longtable compatibility
\newcounter{none}
\def\thenone{}


% content/resources/templates/gujarati-boxes.tex
\usepackage{fontspec}
\usepackage{polyglossia}

% Set Gujarati as main language (document is primarily in Gujarati)
% Note: gloss-gujarati.ldf doesn't exist in polyglossia, but it will use hyphenation patterns
\setdefaultlanguage{gujarati}
\setotherlanguage{english}

% Configure Gujarati font properly
% Use Language=Default to prevent polyglossia from trying to add language-specific features
% that don't exist for Gujarati, which causes "empty feature" warnings
\newfontfamily\gujaratifont[Script=Gujarati,AutoFakeBold=2.5,AutoFakeSlant=0.3]{Noto Sans Gujarati}
\setmainfont[Script=Gujarati,AutoFakeBold=2.5,AutoFakeSlant=0.3]{Noto Sans Gujarati}
% Use Noto Sans Gujarati for monospace to support Gujarati in text
\setmonofont[Scale=0.9]{Noto Sans Gujarati}

% Configure English to use the same font
\newfontfamily\englishfont[Script=Gujarati,AutoFakeBold=2.5,AutoFakeSlant=0.3]{Noto Sans Gujarati}

% Translations for polyglossia
\gappto\captionsgujarati{
  \renewcommand{\tablename}{કોષ્ટક}
  \renewcommand{\figurename}{આકૃતિ}
}

% Helper for TikZ nodes to ensure Gujarati font
\newcommand{\gu}[1]{{\gujaratifont #1}}

% Custom environments
\newtcolorbox{solutionbox}{
    breakable,
    enhanced,
    colback=solutioncolor!5!white,
    colframe=solutioncolor!75!black,
    fonttitle=\bfseries,
    title=જવાબ
}

\newtcolorbox{solutionboxnobreak}{
 colback=solutioncolor!5!white,
 colframe=solutioncolor!75!black,
 fonttitle=\bfseries,
 title=જવાબ
}

\newtcolorbox{keyformula}{
 breakable,
 enhanced,
 colback=keycolor!5!white,
 colframe=keycolor!75!black,
 fonttitle=\bfseries,
 title=રાસાયણિક સમીકરણ/સૂત્ર
}

\newtcolorbox{mnemonicbox}{
 breakable,
 enhanced,
 colback=mnemoniccolor!5!white,
 colframe=mnemoniccolor!75!black,
 fonttitle=\bfseries,
 title=મેમરી ટ્રીક
}


\begin{document}

\begin{center}
{\Huge\bfseries\color{headcolor} Subject Name (Gujarati)}\\[5pt]
{\LARGE 4331604 -- Winter 2024}\\[3pt]
{\large Semester 1 Study Material}\\[3pt]
{\normalsize\textit{Detailed Solutions and Explanations}}
\end{center}

\vspace{10pt}

\subsection*{પ્રશ્ન 1(અ) [3
ગુણ]}\label{uxaaauxab0uxab6uxaa8-1uxa85-3-uxa97uxaa3}

\textbf{સ્ક્રમ મોડેલ શું છે? એના વિશે લખો.}

\begin{solutionbox}

સ્ક્રમ એક \textbf{એજાઇલ ફ્રેમવર્ક} છે જે પુનરાવર્તક અને વધારાની પદ્ધતિઓ દ્વારા
સોફ્ટવેર ડેવલપમેન્ટ પ્રોજેક્ટ્સનું સંચાલન કરે છે.

{\def\LTcaptype{none} % do not increment counter
\begin{longtable}[]{@{}
  >{\raggedright\arraybackslash}p{(\linewidth - 2\tabcolsep) * \real{0.5000}}
  >{\raggedright\arraybackslash}p{(\linewidth - 2\tabcolsep) * \real{0.5000}}@{}}
\toprule\noalign{}
\begin{minipage}[b]{\linewidth}\raggedright
પાસાં
\end{minipage} & \begin{minipage}[b]{\linewidth}\raggedright
વર્ણન
\end{minipage} \\
\midrule\noalign{}
\endhead
\bottomrule\noalign{}
\endlastfoot
\textbf{ફ્રેમવર્ક પ્રકાર} & એજાઇલ મેથડોલોજી \\
\textbf{સ્પ્રિન્ટ અવધિ} & સામાન્ય રીતે 2-4 અઠવાડિયા \\
\textbf{ટીમ સાઇઝ} & 5-9 સભ્યો \\
\textbf{મુખ્ય સમારંભો} & ડેઇલી સ્ટેન્ડઅપ્સ, સ્પ્રિન્ટ પ્લાનિંગ, સ્પ્રિન્ટ રિવ્યુ,
રિટ્રોસ્પેક્ટિવ \\
\end{longtable}
}

\textbf{મુખ્ય લક્ષણો:}

\begin{itemize}
\tightlist
\item
  \textbf{પ્રોડક્ટ ઓનર}: જરૂરિયાતો અને પ્રાધાન્યતાઓ નક્કી કરે છે
\item
  \textbf{સ્ક્રમ માસ્ટર}: પ્રક્રિયાને સુવિધા આપે છે અને અવરોધો દૂર કરે છે
\item
  \textbf{ડેવલપમેન્ટ ટીમ}: પ્રોડક્ટ બનાવતી ક્રોસ-ફંક્શનલ ટીમ
\end{itemize}

\end{solutionbox}
\begin{mnemonicbox}
``SPIR'' - Sprint, Product owner, Incremental
delivery, Review

\end{mnemonicbox}
\begin{center}\rule{0.5\linewidth}{0.5pt}\end{center}

\subsection*{પ્રશ્ન 1(બ) [4
ગુણ]}\label{uxaaauxab0uxab6uxaa8-1uxaac-4-uxa97uxaa3}

\textbf{સોફ્ટવેરની વ્યાખ્યા આપો અને સોફ્ટવેરની કેરેક્ટરિસ્ટિક સમજાવો.}

\begin{solutionbox}

\textbf{સોફ્ટવેરની વ્યાખ્યા}: કમ્પ્યુટર સિસ્ટમ પર કાર્યો કરતા કમ્પ્યુટર પ્રોગ્રામ્સ,
પ્રક્રિયાઓ અને ડોક્યુમેન્ટેશનનો સંગ્રહ.

{\def\LTcaptype{none} % do not increment counter
\begin{longtable}[]{@{}ll@{}}
\toprule\noalign{}
લક્ષણ & વર્ણન \\
\midrule\noalign{}
\endhead
\bottomrule\noalign{}
\endlastfoot
\textbf{અસ્પષ્ટ} & શારીરિક રીતે સ્પર્શ કરી શકાતું નથી \\
\textbf{શારીરિક ઘસારો નહીં} & સમય સાથે બગડતું નથી \\
\textbf{કસ્ટમ બિલ્ટ} & ચોક્કસ જરૂરિયાતો માટે વિકસાવવામાં આવે છે \\
\textbf{મોંઘું} & વિકાસ અને જાળવણીની ઊંચી કિંમત \\
\end{longtable}
}

\textbf{મુખ્ય મુદ્દાઓ:}

\begin{itemize}
\tightlist
\item
  \textbf{લોજિકલ પ્રોડક્ટ}: સૂચનાઓ અને ડેટાથી બનેલું
\item
  \textbf{એન્જિનીયર્ડ}: વ્યવસ્થિત વિકાસ પ્રક્રિયાને અનુસરે છે
\item
  \textbf{કોમ્પ્લેક્સ}: બહુવિધ પરસ્પર જોડાયેલા કાર્યોને હેન્ડલ કરે છે
\item
  \textbf{જાળવી શકાય તેવું}: ફેરફાર અને અપડેટ કરી શકાય છે
\end{itemize}

\end{solutionbox}
\begin{mnemonicbox}
``IELM'' - Intangible, Engineered, Logical,
Maintainable

\end{mnemonicbox}
\begin{center}\rule{0.5\linewidth}{0.5pt}\end{center}

\subsection*{પ્રશ્ન 1(ક) [7
ગુણ]}\label{uxaaauxab0uxab6uxaa8-1uxa95-7-uxa97uxaa3}

\textbf{વોટરફોલ મોડેલ ડાયાગ્રામ સાથે સમજાવો.}

\begin{solutionbox}

\textbf{વોટરફોલ મોડેલ} એક લીનિયર સિક્વેન્શિયલ સોફ્ટવેર ડેવલપમેન્ટ અભિગમ છે જ્યાં દરેક
તબક્કો પૂર્ણ થયા પછી આગળનો શરૂ થાય છે.

\begin{verbatim}
flowchart LR
    A[Requirements Analysis] {-{-} B[System Design]}
    B {-{-} C[Implementation]}
    C {-{-} D[Testing]}
    D {-{-} E[Deployment]}
    E {-{-} F[Maintenance]}
    
    style A fill:\#e1f5fe
    style B fill:\#f3e5f5
    style C fill:\#e8f5e8
    style D fill:\#fff3e0
    style E fill:\#fce4ec
    style F fill:\#f1f8e9
\end{verbatim}

{\def\LTcaptype{none} % do not increment counter
\begin{longtable}[]{@{}lll@{}}
\toprule\noalign{}
તબક્કો & પ્રવૃત્તિઓ & આઉટપુટ \\
\midrule\noalign{}
\endhead
\bottomrule\noalign{}
\endlastfoot
\textbf{આવશ્યકતાઓ} & જરૂરિયાતો ભેગી કરવી અને દસ્તાવેજીકરણ & SRS ડોક્યુમેન્ટ \\
\textbf{ડિઝાઇન} & સિસ્ટમ આર્કિટેક્ચર આયોજન & ડિઝાઇન સ્પેસિફિકેશન્સ \\
\textbf{ઇમ્પ્લિમેન્ટેશન} & વાસ્તવિક કોડિંગ & સોર્સ કોડ \\
\textbf{ટેસ્ટિંગ} & વેરિફિકેશન અને વેલિડેશન & ટેસ્ટ રિપોર્ટ્સ \\
\textbf{ડિપ્લોયમેન્ટ} & ક્લાયન્ટ સાઇટ પર ઇન્સ્ટોલેશન & કાર્યકારી સિસ્ટમ \\
\textbf{મેન્ટેનન્સ} & બગ ફિક્સ અને અપડેટ્સ & અપડેટેડ સિસ્ટમ \\
\end{longtable}
}

\textbf{ફાયદાઓ:}

\begin{itemize}
\tightlist
\item
  \textbf{સમજવામાં સરળ} અને અમલીકરણ
\item
  \textbf{સારી રીતે દસ્તાવેજીકૃત} તબક્કાઓ
\item
  \textbf{સરળ પ્રોજેક્ટ મેનેજમેન્ટ} સ્પષ્ટ માઇલસ્ટોન્સ સાથે
\end{itemize}

\textbf{નુકસાનો:}

\begin{itemize}
\tightlist
\item
  \textbf{બદલાવ માટે કોઈ લવચીકતા} નથી
\item
  \textbf{મોડું ટેસ્ટિંગ} સમસ્યાઓની મોડી શોધ
\item
  \textbf{કોમ્પ્લેક્સ પ્રોજેક્ટ્સ} માટે યોગ્ય નથી
\end{itemize}

\end{solutionbox}
\begin{mnemonicbox}
``RSITDM'' - Requirements, System design,
Implementation, Testing, Deployment, Maintenance

\end{mnemonicbox}
\begin{center}\rule{0.5\linewidth}{0.5pt}\end{center}

\subsection*{પ્રશ્ન 1(ક) OR [7
ગુણ]}\label{uxaaauxab0uxab6uxaa8-1uxa95-or-7-uxa97uxaa3}

\textbf{સ્પાઇરલ મોડેલ ડાયાગ્રામ સાથે સમજાવો.}

\begin{solutionbox}

\textbf{સ્પાઇરલ મોડેલ} પુનરાવર્તક વિકાસને વ્યવસ્થિત જોખમ મૂલ્યાંકન સાથે જોડે છે, દરેક
પુનરાવર્તનમાં જોખમ વિશ્લેષણ પર ભાર મૂકે છે.

\begin{center}
\textbf{Mermaid Diagram (Code)}
\begin{verbatim}
{Shaded}
{Highlighting}[]
graph LR
    A[Planning] {-{-}{} B[Risk Analysis]}
    B {-{-}{} C[Engineering]}
    C {-{-}{} D[Customer Evaluation]}
    D {-{-}{} A}
    
    E[Risk Assessment] {-.{-}{} B}
    F[Prototype Development] {-.{-}{} C}
    G[Customer Feedback] {-.{-}{} D}
    
    style A fill:\#e3f2fd
    style B fill:\#ffebee
    style C fill:\#e8f5e8
    style D fill:\#fff8e1
{Highlighting}
{Shaded}
\end{verbatim}
\end{center}

{\def\LTcaptype{none} % do not increment counter
\begin{longtable}[]{@{}lll@{}}
\toprule\noalign{}
ચતુર્થાંશ & પ્રવૃત્તિ & હેતુ \\
\midrule\noalign{}
\endhead
\bottomrule\noalign{}
\endlastfoot
\textbf{પ્લાનિંગ} & આવશ્યકતા ભેગી કરવી & ઉદ્દેશ્યો નક્કી કરવા \\
\textbf{રિસ્ક એનાલિસિસ} & જોખમો ઓળખવા અને ઉકેલવા & અનિશ્ચિતતા ઘટાડવા \\
\textbf{એન્જિનીયરિંગ} & વિકાસ અને ટેસ્ટિંગ & કાર્યકારી સોફ્ટવેર બનાવવા \\
\textbf{ઇવેલ્યુએશન} & ગ્રાહક મૂલ્યાંકન & આગળના પુનરાવર્તન માટે પ્રતિસાદ \\
\end{longtable}
}

\textbf{મુખ્ય લક્ષણો:}

\begin{itemize}
\tightlist
\item
  \textbf{જોખમ-સંચાલિત અભિગમ} પ્રારંભિક જોખમ ઓળખ સાથે
\item
  \textbf{ગ્રાહકની સંડોવણી} સાથે પુનરાવર્તક વિકાસ
\item
  \textbf{દરેક સ્પાઇરલમાં} પ્રોટોટાઇપિંગ
\item
  \textbf{મોટા અને જટિલ} પ્રોજેક્ટ્સ માટે યોગ્ય
\end{itemize}

\textbf{ફાયદાઓ:}

\begin{itemize}
\tightlist
\item
  \textbf{પ્રારંભિક જોખમ શોધ} અને ઘટાડો
\item
  \textbf{સમગ્ર વિકાસ} દરમિયાન ગ્રાહકની સંડોવણી
\item
  \textbf{બદલાવોને સમાવવા} માટે લવચીક
\end{itemize}

\textbf{નુકસાનો:}

\begin{itemize}
\tightlist
\item
  \textbf{જોખમ વિશ્લેષણ} કારણે જટિલ મેનેજમેન્ટ
\item
  \textbf{નાના પ્રોજેક્ટ્સ} માટે મોંઘું
\item
  \textbf{જોખમ મૂલ્યાંકનમાં} નિપુણતાની જરૂર
\end{itemize}

\end{solutionbox}
\begin{mnemonicbox}
``PRICE'' - Planning, Risk analysis, Iterative,
Customer evaluation, Engineering

\end{mnemonicbox}
\begin{center}\rule{0.5\linewidth}{0.5pt}\end{center}

\subsection*{પ્રશ્ન 2(અ) [3
ગુણ]}\label{uxaaauxab0uxab6uxaa8-2uxa85-3-uxa97uxaa3}

\textbf{કઈ પરિસ્થિતિમાં પ્રોટોટાઇપ મોડેલ ઉપયોગ થાય છે?}

\begin{solutionbox}

\textbf{પ્રોટોટાઇપ મોડેલ} એ વખતે ઉપયોગ થાય છે જ્યારે આવશ્યકતાઓ અસ્પષ્ટ હોય અથવા
શક્યતાનું પ્રદર્શન કરવું મહત્વપૂર્ણ હોય.

{\def\LTcaptype{none} % do not increment counter
\begin{longtable}[]{@{}ll@{}}
\toprule\noalign{}
પરિસ્થિતિ & ઉપયોગ \\
\midrule\noalign{}
\endhead
\bottomrule\noalign{}
\endlastfoot
\textbf{અસ્પષ્ટ આવશ્યકતાઓ} & જ્યારે વપરાશકર્તાની જરૂરિયાતો સારી રીતે નિર્ધારિત
નથી \\
\textbf{નવી ટેક્નોલોજી} & નવા ટૂલ્સ/પ્લેટફોર્મની શક્યતા ચકાસવી \\
\textbf{યુઝર ઇન્ટરફેસ} & જટિલ UI/UX સિસ્ટમ્સ ડિઝાઇન કરવા \\
\textbf{હાઇ રિસ્ક પ્રોજેક્ટ્સ} & શરૂઆતમાં અનિશ્ચિતતા ઘટાડવા \\
\end{longtable}
}

\textbf{ચોક્કસ ઉપયોગના કેસેસ:}

\begin{itemize}
\tightlist
\item
  \textbf{વેબ એપ્લિકેશન્સ} જટિલ વપરાશકર્તા ક્રિયાપ્રતિક્રિયાઓ સાથે
\item
  \textbf{રિયલ-ટાઇમ સિસ્ટમ્સ} પ્રદર્શન વેલિડેશનની જરૂર
\item
  \textbf{AI/ML પ્રોજેક્ટ્સ} પ્રાયોગિક અલ્ગોરિધમ્સ સાથે
\end{itemize}

\end{solutionbox}
\begin{mnemonicbox}
``UNIT'' - Unclear requirements, New technology,
Interface design, Testing feasibility

\end{mnemonicbox}
\begin{center}\rule{0.5\linewidth}{0.5pt}\end{center}

\subsection*{પ્રશ્ન 2(બ) [4
ગુણ]}\label{uxaaauxab0uxab6uxaa8-2uxaac-4-uxa97uxaa3}

\textbf{રિક્વાયરમેન્ટ ગેધરિંગ વિગતમાં સમજાવો.}

\begin{solutionbox}

\textbf{રિક્વાયરમેન્ટ ગેધરિંગ} એ સ્ટેકહોલ્ડર્સ પાસેથી સોફ્ટવેર આવશ્યકતાઓ એકત્રિત
કરવાની, વિશ્લેષણ કરવાની અને દસ્તાવેજીકરણ કરવાની પ્રક્રિયા છે.

{\def\LTcaptype{none} % do not increment counter
\begin{longtable}[]{@{}lll@{}}
\toprule\noalign{}
તકનીક & વર્ણન & ક્યારે ઉપયોગ કરવો \\
\midrule\noalign{}
\endhead
\bottomrule\noalign{}
\endlastfoot
\textbf{ઇન્ટરવ્યુ} & એક-પર-એક ચર્ચાઓ & વિગતવાર આવશ્યકતાઓ \\
\textbf{પ્રશ્નાવલીઓ} & સંરચિત સર્વેક્ષણો & મોટા વપરાશકર્તા જૂથો \\
\textbf{અવલોકન} & વર્તમાન પ્રક્રિયાઓ જોવી & વર્કફ્લો સમજવા \\
\textbf{વર્કશોપ્સ} & જૂથ સત્રો & સહયોગી આવશ્યકતાઓ \\
\end{longtable}
}

\textbf{પ્રક્રિયાના પગલાં:}

\begin{itemize}
\tightlist
\item
  \textbf{સ્ટેકહોલ્ડર ઓળખ}: તમામ સંબંધિત પક્ષો શોધવા
\item
  \textbf{માહિતી સંગ્રહ}: વિવિધ એકત્રીકરણ તકનીકોનો ઉપયોગ
\item
  \textbf{વિશ્લેષણ}: આવશ્યકતાઓને પ્રાધાન્ય અને વર્ગીકૃત કરવી
\item
  \textbf{દસ્તાવેજીકરણ}: ઔપચારિક આવશ્યકતા સ્પેસિફિકેશન્સ બનાવવા
\end{itemize}

\textbf{પડકારો:}

\begin{itemize}
\tightlist
\item
  \textbf{વિકાસ દરમિયાન} બદલાતી આવશ્યકતાઓ
\item
  \textbf{સ્ટેકહોલ્ડર્સ વચ્ચે} કોમ્યુનિકેશન ગેપ
\item
  \textbf{વપરાશકર્તાઓ પાસેથી} અધૂરી માહિતી
\end{itemize}

\end{solutionbox}
\begin{mnemonicbox}
``IQOW'' - Interviews, Questionnaires, Observation,
Workshops

\end{mnemonicbox}
\begin{center}\rule{0.5\linewidth}{0.5pt}\end{center}

\subsection*{પ્રશ્ન 2(ક) [7
ગુણ]}\label{uxaaauxab0uxab6uxaa8-2uxa95-7-uxa97uxaa3}

\textbf{સોફ્ટવેર પ્રોજેક્ટ મેનેજરની જવાબદારીની ચર્ચા કરો.}

\begin{solutionbox}

\textbf{સોફ્ટવેર પ્રોજેક્ટ મેનેજર} સફળ પ્રોજેક્ટ ડિલિવરી સુનિશ્ચિત કરતા સમગ્ર સોફ્ટવેર
ડેવલપમેન્ટ લાઇફસાઇકલની દેખરેખ કરે છે.

{\def\LTcaptype{none} % do not increment counter
\begin{longtable}[]{@{}lll@{}}
\toprule\noalign{}
જવાબદારી વિસ્તાર & મુખ્ય કાર્યો & જરૂરી કુશળતા \\
\midrule\noalign{}
\endhead
\bottomrule\noalign{}
\endlastfoot
\textbf{પ્લાનિંગ} & પ્રોજેક્ટ શેડ્યુલિંગ, રિસોર્સ એલોકેશન & વ્યૂહાત્મક વિચારણા \\
\textbf{ટીમ મેનેજમેન્ટ} & ટીમ સંકલન, પ્રેરણા & નેતૃત્વ \\
\textbf{રિસ્ક મેનેજમેન્ટ} & જોખમ ઓળખ, ઘટાડાની વ્યૂહરચના & સમસ્યા-નિરાકરણ \\
\textbf{કોમ્યુનિકેશન} & સ્ટેકહોલ્ડર સંકલન, રિપોર્ટિંગ & વાતચીત કુશળતા \\
\textbf{ગુણવત્તા ખાતરી} & પ્રક્રિયા અનુપાલન, ડિલિવરેબલ ગુણવત્તા & વિગતો પર
ધ્યાન \\
\end{longtable}
}

\textbf{વિગતવાર જવાબદારીઓ:}

\textbf{પ્રોજેક્ટ પ્લાનિંગ:}

\begin{itemize}
\tightlist
\item
  \textbf{વર્ક બ્રેકડાઉન સ્ટ્રક્ચર} બનાવવું
\item
  \textbf{ટાઇમલાઇન એસ્ટિમેશન} અને શેડ્યુલિંગ
\item
  \textbf{રિસોર્સ એલોકેશન} અને બજેટ મેનેજમેન્ટ
\end{itemize}

\textbf{ટીમ લીડરશિપ:}

\begin{itemize}
\tightlist
\item
  \textbf{ટીમ બિલ્ડિંગ} અને પ્રેરણા
\item
  \textbf{ટીમ સભ્યો વચ્ચે} સંઘર્ષ નિરાકરણ
\item
  \textbf{પ્રદર્શન મોનિટરિંગ} અને પ્રતિસાદ
\end{itemize}

\textbf{સ્ટેકહોલ્ડર મેનેજમેન્ટ:}

\begin{itemize}
\tightlist
\item
  \textbf{ક્લાયન્ટ કોમ્યુનિકેશન} અને અપેક્ષા મેનેજમેન્ટ
\item
  \textbf{સિનિયર મેનેજમેન્ટને} પ્રગતિ રિપોર્ટિંગ
\item
  \textbf{ચેન્જ રિક્વેસ્ટ} હેન્ડલિંગ અને મંજૂરી
\end{itemize}

\textbf{રિસ્ક અને ક્વોલિટી મેનેજમેન્ટ:}

\begin{itemize}
\tightlist
\item
  \textbf{રિસ્ક એસેસમેન્ટ} અને આકસ્મિક આયોજન
\item
  \textbf{ગુણવત્તા ધોરણો} અમલીકરણ
\item
  \textbf{પ્રક્રિયા સુધારણા} અમલીકરણ
\end{itemize}

\textbf{આવશ્યક કુશળતા:}

\begin{itemize}
\tightlist
\item
  \textbf{સોફ્ટવેર ડેવલપમેન્ટનું} તકનીકી જ્ઞાન
\item
  \textbf{પ્રોજેક્ટ મેનેજમેન્ટ} મેથડોલોજીઝ (Agile, Waterfall)
\item
  \textbf{વિવિધ સ્ટેકહોલ્ડર્સ} માટે કોમ્યુનિકેશન સ્કિલ્સ
\item
  \textbf{સમસ્યા-નિરાકરણ} અને નિર્ણય લેવાની ક્ષમતા
\end{itemize}

\end{solutionbox}
\begin{mnemonicbox}
``PLACE'' - Planning, Leadership, Assessment,
Communication, Execution

\end{mnemonicbox}
\begin{center}\rule{0.5\linewidth}{0.5pt}\end{center}

\subsection*{પ્રશ્ન 2(અ) OR [3
ગુણ]}\label{uxaaauxab0uxab6uxaa8-2uxa85-or-3-uxa97uxaa3}

\textbf{ગાંટ ચાર્ટ અને પર્ટ ચાર્ટનો તફાવત લખો.}

\begin{solutionbox}

{\def\LTcaptype{none} % do not increment counter
\begin{longtable}[]{@{}lll@{}}
\toprule\noalign{}
પાસું & ગાંટ ચાર્ટ & પર્ટ ચાર્ટ \\
\midrule\noalign{}
\endhead
\bottomrule\noalign{}
\endlastfoot
\textbf{હેતુ} & કાર્યોની વિઝ્યુઅલ ટાઇમલાઇન & નિર્ભરતાનું નેટવર્ક વિશ્લેષણ \\
\textbf{ફોર્મેટ} & હોરિઝોન્ટલ બાર ચાર્ટ & નોડ્સ સાથે નેટવર્ક ડાયાગ્રામ \\
\textbf{સમય ફોકસ} & અવધિ અને તારીખો બતાવે છે & ક્રિટિકલ પાથ અને સ્લેક ટાઇમ \\
\textbf{જટિલતા} & સમજવામાં સરળ & વધુ જટિલ વિશ્લેષણ \\
\textbf{શ્રેષ્ઠ માટે} & પ્રોજેક્ટ શેડ્યુલિંગ & સમય ઑપ્ટિમાઇઝેશન \\
\end{longtable}
}

\textbf{મુખ્ય તફાવતો:}

\begin{itemize}
\tightlist
\item
  \textbf{ગાંટ}: બતાવે છે \textbf{કાર્યો ક્યારે થાય છે}
\item
  \textbf{પર્ટ}: બતાવે છે \textbf{કાર્ય સંબંધો} અને ક્રિટિકલ પાથ
\end{itemize}

\end{solutionbox}
\begin{mnemonicbox}
``GT vs PT'' - Gantt Timeline vs PERT dependencies

\end{mnemonicbox}
\begin{center}\rule{0.5\linewidth}{0.5pt}\end{center}

\subsection*{પ્રશ્ન 2(બ) OR [4
ગુણ]}\label{uxaaauxab0uxab6uxaa8-2uxaac-or-4-uxa97uxaa3}

\textbf{RAD, SDLC, XP model અને SRS નું પૂરું નામ લખો.}

\begin{solutionbox}

{\def\LTcaptype{none} % do not increment counter
\begin{longtable}[]{@{}
  >{\raggedright\arraybackslash}p{(\linewidth - 4\tabcolsep) * \real{0.2800}}
  >{\raggedright\arraybackslash}p{(\linewidth - 4\tabcolsep) * \real{0.4400}}
  >{\raggedright\arraybackslash}p{(\linewidth - 4\tabcolsep) * \real{0.2800}}@{}}
\toprule\noalign{}
\begin{minipage}[b]{\linewidth}\raggedright
સંક્ષેપ
\end{minipage} & \begin{minipage}[b]{\linewidth}\raggedright
પૂરું નામ
\end{minipage} & \begin{minipage}[b]{\linewidth}\raggedright
વર્ણન
\end{minipage} \\
\midrule\noalign{}
\endhead
\bottomrule\noalign{}
\endlastfoot
\textbf{RAD} & Rapid Application Development & ઝડપી પ્રોટોટાઇપિંગ
મેથડોલોજી \\
\textbf{SDLC} & Software Development Life Cycle & સંપૂર્ણ વિકાસ પ્રક્રિયા \\
\textbf{XP} & Extreme Programming & એજાઇલ ડેવલપમેન્ટ મેથડોલોજી \\
\textbf{SRS} & Software Requirement Specification & ઔપચારિક આવશ્યકતા
દસ્તાવેજ \\
\end{longtable}
}

\textbf{સંક્ષિપ્ત સમજૂતીઓ:}

\begin{itemize}
\tightlist
\item
  \textbf{RAD}: \textbf{ઝડપી પ્રોટોટાઇપિંગ} અને પુનરાવર્તક વિકાસ પર ફોકસ
\item
  \textbf{SDLC}: સોફ્ટવેર ડેવલપમેન્ટ તબક્કાઓ માટે \textbf{વ્યવસ્થિત અભિગમ}
\item
  \textbf{XP}: કોડિંગ પ્રેક્ટિસ પર ભાર મૂકતી \textbf{એજાઇલ મેથડોલોજી}
\item
  \textbf{SRS}: કાર્યાત્મક અને બિન-કાર્યાત્મક આવશ્યકતાઓનું \textbf{વિગતવાર
  દસ્તાવેજીકરણ}
\end{itemize}

\end{solutionbox}
\begin{mnemonicbox}
``RSXS'' - RAD, SDLC, XP, SRS

\end{mnemonicbox}
\begin{center}\rule{0.5\linewidth}{0.5pt}\end{center}

\subsection*{પ્રશ્ન 2(ક) OR [7
ગુણ]}\label{uxaaauxab0uxab6uxaa8-2uxa95-or-7-uxa97uxaa3}

\textbf{WBS વિગતમાં સમજાવો.}

\begin{solutionbox}

\textbf{વર્ક બ્રેકડાઉન સ્ટ્રક્ચર (WBS)} પ્રોજેક્ટ કાર્યનું વંશવેલો વિઘટન છે જે નાના,
વ્યવસ્થાપન યોગ્ય ઘટકોમાં વિભાજિત કરે છે.

\begin{center}
\textbf{Mermaid Diagram (Code)}
\begin{verbatim}
{Shaded}
{Highlighting}[]
graph TD
    A[Software Project] {-{-}{} B[Analysis Phase]}
    A {-{-}{} C[Design Phase]}
    A {-{-}{} D[Implementation Phase]}
    A {-{-}{} E[Testing Phase]}
    
    B {-{-}{} B1[Requirement Gathering]}
    B {-{-}{} B2[SRS Documentation]}
    
    C {-{-}{} C1[System Design]}
    C {-{-}{} C2[Database Design]}
    C {-{-}{} C3[UI Design]}
    
    D {-{-}{} D1[Module Development]}
    D {-{-}{} D2[Code Review]}
    D {-{-}{} D3[Integration]}
    
    E {-{-}{} E1[Unit Testing]}
    E {-{-}{} E2[System Testing]}
    E {-{-}{} E3[User Acceptance Testing]}
{Highlighting}
{Shaded}
\end{verbatim}
\end{center}

{\def\LTcaptype{none} % do not increment counter
\begin{longtable}[]{@{}lll@{}}
\toprule\noalign{}
WBS લેવલ & વર્ણન & ઉદાહરણ \\
\midrule\noalign{}
\endhead
\bottomrule\noalign{}
\endlastfoot
\textbf{લેવલ 1} & મુખ્ય પ્રોજેક્ટ તબક્કાઓ & વિશ્લેષણ, ડિઝાઇન, અમલીકરણ \\
\textbf{લેવલ 2} & મુખ્ય ડિલિવરેબલ્સ & SRS, ડિઝાઇન ડોક્સ, કોડ મોડ્યુલ્સ \\
\textbf{લેવલ 3} & વર્ક પેકેજીસ & ચોક્કસ કાર્યો અને પ્રવૃત્તિઓ \\
\textbf{લેવલ 4} & વ્યક્તિગત પ્રવૃત્તિઓ & વિગતવાર કાર્ય વિઘટન \\
\end{longtable}
}

\textbf{WBS ના ફાયદાઓ:}

\begin{itemize}
\tightlist
\item
  \textbf{સ્પષ્ટ પ્રોજેક્ટ સ્કોપ} વ્યાખ્યા
\item
  \textbf{સમય અને સંસાધનોનું} બહેતર અંદાજ
\item
  \textbf{સુધારેલ કાર્ય સોંપણી} અને જવાબદારી
\item
  \textbf{વર્ધિત પ્રગતિ ટ્રેકિંગ} અને નિયંત્રણ
\end{itemize}

\textbf{WBS બનાવવાની પ્રક્રિયા:}

\begin{itemize}
\tightlist
\item
  \textbf{પ્રોજેક્ટ સ્કોપમાંથી} મુખ્ય ડિલિવરેબલ્સ ઓળખવા
\item
  \textbf{ડિલિવરેબલ્સને} નાના ઘટકોમાં વિઘટન કરવા
\item
  \textbf{વર્ક પેકેજીસ} વ્યવસ્થાપન યોગ્ય થાય ત્યાં સુધી ભંગાણ ચાલુ રાખવું
\item
  \textbf{દરેક વર્ક પેકેજ} માટે જવાબદારીઓ સોંપવી
\end{itemize}

\textbf{મુખ્ય સિદ્ધાંતો:}

\begin{itemize}
\tightlist
\item
  \textbf{100\% નિયમ}: WBS માં તમામ પ્રોજેક્ટ કાર્ય સામેલ છે
\item
  \textbf{પરસ્પર વિશિષ્ટ}: ઘટકો વચ્ચે કોઈ ઓવરલેપ નથી
\item
  \textbf{વ્યવસ્થાપન યોગ્ય સાઇઝ}: વર્ક પેકેજીસ 8-80 કલાકની હોવી જોઈએ
\end{itemize}

\end{solutionbox}
\begin{mnemonicbox}
``DEBT'' - Decompose, Estimate, Breakdown, Track

\end{mnemonicbox}
\begin{center}\rule{0.5\linewidth}{0.5pt}\end{center}

\subsection*{પ્રશ્ન 3(અ) [3
ગુણ]}\label{uxaaauxab0uxab6uxaa8-3uxa85-3-uxa97uxaa3}

\textbf{ઇન્ક્રિમેન્ટલ મોડેલનો ડાયાગ્રામ દોરો.}

\begin{solutionbox}

\textbf{ઇન્ક્રિમેન્ટલ મોડેલ} સોફ્ટવેરને વધારાઓમાં વિકસાવે છે, દરેક વધારો પાછલા
વર્ઝનમાં કાર્યક્ષમતા ઉમેરે છે.

\begin{center}
\textbf{Mermaid Diagram (Code)}
\begin{verbatim}
{Shaded}
{Highlighting}[]
graph LR
    A[Requirements Analysis] {-{-}{} B[System Design]}
    B {-{-}{} C1[Increment 1]}
    B {-{-}{} C2[Increment 2]}
    B {-{-}{} C3[Increment 3]}
    
    C1 {-{-}{} D1[Design  Code  Test]}
    C2 {-{-}{} D2[Design  Code  Test]}
    C3 {-{-}{} D3[Design  Code  Test]}
    
    D1 {-{-}{} E1[Release 1]}
    D2 {-{-}{} E2[Release 2]}
    D3 {-{-}{} E3[Release 3]}
    
    E1 {-{-}{} F[Final Product]}
    E2 {-{-}{} F}
    E3 {-{-}{} F}
    
    style A fill:\#e3f2fd
    style B fill:\#f3e5f5
    style C1 fill:\#e8f5e8
    style C2 fill:\#fff3e0
    style C3 fill:\#fce4ec
{Highlighting}
{Shaded}
\end{verbatim}
\end{center}

\textbf{મુખ્ય લક્ષણો:}

\begin{itemize}
\tightlist
\item
  \textbf{કોર કાર્યક્ષમતા} પહેલા ડિલિવર કરવામાં આવે છે
\item
  \textbf{વધારાની સુવિધાઓ} ક્રમિક રીતે ઉમેરવામાં આવે છે
\item
  \textbf{કાર્યકારી સોફ્ટવેર} શરૂઆતમાં ઉપલબ્ધ
\end{itemize}

\end{solutionbox}
\begin{mnemonicbox}
``IRA'' - Incremental, Release, Add features

\end{mnemonicbox}
\begin{center}\rule{0.5\linewidth}{0.5pt}\end{center}

\subsection*{પ્રશ્ન 3(બ) [4
ગુણ]}\label{uxaaauxab0uxab6uxaa8-3uxaac-4-uxa97uxaa3}

\textbf{ફંક્શનલ અને નોન-ફંક્શનલ રિક્વાયરમેન્ટનો તફાવત લખો.}

\begin{solutionbox}

{\def\LTcaptype{none} % do not increment counter
\begin{longtable}[]{@{}lll@{}}
\toprule\noalign{}
પાસું & ફંક્શનલ રિક્વાયરમેન્ટ્સ & નોન-ફંક્શનલ રિક્વાયરમેન્ટ્સ \\
\midrule\noalign{}
\endhead
\bottomrule\noalign{}
\endlastfoot
\textbf{વ્યાખ્યા} & સિસ્ટમે શું કરવું જોઈએ & સિસ્ટમે કેવી રીતે કામ કરવું જોઈએ \\
\textbf{ફોકસ} & સિસ્ટમ વર્તન અને સુવિધાઓ & સિસ્ટમ ગુણવત્તા લક્ષણો \\
\textbf{ઉદાહરણો} & લોગિન, ડેટા પ્રોસેસિંગ, રિપોર્ટ્સ & પ્રદર્શન, સુરક્ષા,
ઉપયોગિતા \\
\textbf{ટેસ્ટિંગ} & ફંક્શનલ ટેસ્ટિંગ & પ્રદર્શન, સુરક્ષા ટેસ્ટિંગ \\
\textbf{દસ્તાવેજીકરણ} & યુઝ કેસીસ, યુઝર સ્ટોરીઝ & ગુણવત્તા મેટ્રિક્સ, મર્યાદાઓ \\
\end{longtable}
}

\textbf{વિગતવાર તુલના:}

\textbf{ફંક્શનલ રિક્વાયરમેન્ટ્સ:}

\begin{itemize}
\tightlist
\item
  \textbf{યુઝર ઓથેન્ટિકેશન} અને ઓથોરાઇઝેશન
\item
  \textbf{ડેટા પ્રોસેસિંગ} અને ગણતરીઓ
\item
  \textbf{રિપોર્ટ જનરેશન} અને એક્સપોર્ટ સુવિધાઓ
\item
  \textbf{બિઝનેસ લોજિક} અમલીકરણ
\end{itemize}

\textbf{નોન-ફંક્શનલ રિક્વાયરમેન્ટ્સ:}

\begin{itemize}
\tightlist
\item
  \textbf{પ્રદર્શન}: પ્રતિસાદ સમય, થ્રુપુટ
\item
  \textbf{સુરક્ષા}: ડેટા એન્ક્રિપ્શન, એક્સેસ કંટ્રોલ
\item
  \textbf{ઉપયોગિતા}: યુઝર ઇન્ટરફેસ ડિઝાઇન, પહોંચ
\item
  \textbf{વિશ્વસનીયતા}: સિસ્ટમ ઉપલબ્ધતા, ફોલ્ટ સહનશીલતા
\end{itemize}

\textbf{લાઇબ્રેરી સિસ્ટમ માટે ઉદાહરણો:}

\begin{itemize}
\tightlist
\item
  \textbf{ફંક્શનલ}: પુસ્તક શોધ, પુસ્તક ઇશ્યુ/રિટર્ન, દંડ ગણતરી
\item
  \textbf{નોન-ફંક્શનલ}: \textless2 સેકંડમાં શોધ પરિણામો, 99.9\% અપટાઇમ, SSL
  એન્ક્રિપ્શન
\end{itemize}

\end{solutionbox}
\begin{mnemonicbox}
``FW vs NH'' - Functional What vs Non-functional How

\end{mnemonicbox}
\begin{center}\rule{0.5\linewidth}{0.5pt}\end{center}

\subsection*{પ્રશ્ન 3(ક) [7
ગુણ]}\label{uxaaauxab0uxab6uxaa8-3uxa95-7-uxa97uxaa3}

\textbf{ડીએફડી ઉદાહરણ સાથે સમજાવો.}

\begin{solutionbox}

\textbf{ડેટા ફ્લો ડાયાગ્રામ (DFD)} પ્રોસેસીસ, ડેટા સ્ટોર્સ, બાહ્ય એન્ટિટીઝ અને ડેટા
ફ્લોઝનો ઉપયોગ કરીને સિસ્ટમ દ્વારા ડેટા ફ્લોનું ગ્રાફિકલ પ્રતિનિધિત્વ છે.

\textbf{DFD સિમ્બોલ્સ:}

{\def\LTcaptype{none} % do not increment counter
\begin{longtable}[]{@{}lll@{}}
\toprule\noalign{}
સિમ્બોલ & નામ & હેતુ \\
\midrule\noalign{}
\endhead
\bottomrule\noalign{}
\endlastfoot
વર્તુળ/અંડાકાર & પ્રોસેસ & ડેટા રૂપાંતરણ \\
લંબચોરસ & બાહ્ય એન્ટિટી & ડેટા સ્રોત/ગંતવ્ય \\
ખુલ્લો લંબચોરસ & ડેટા સ્ટોર & ડેટા સંગ્રહ \\
તીર & ડેટા ફ્લો & ડેટા હિલચાલની દિશા \\
\end{longtable}
}

\textbf{ઉદાહરણ: લાઇબ્રેરી મેનેજમેન્ટ સિસ્ટમ}

\begin{verbatim}
flowchart TD
    A[Student] {-{-}|Book Request| B((Search Books))}
    B {-{-}|Book Details| A}
    B {-{-}|Query| C[(Book Database)]}
    C {-{-}|Book Info| B}
    
    A {-{-}|Issue Request| D((Issue Book))}
    D {-{-}|Issue Details| E[(Issue Records)]}
    D {-{-}|Confirmation| A}
    
    F[Librarian] {-{-}|Book Return| G((Return Book))}
    G {-{-}|Update Status| C}
    G {-{-}|Update Records| E}
    G {-{-}|Receipt| F}
\end{verbatim}

\textbf{DFD લેવલ્સ:}

\textbf{કોન્ટેક્સ્ટ ડાયાગ્રામ (લેવલ 0):}

\begin{itemize}
\tightlist
\item
  \textbf{એક પ્રોસેસ} જે સમગ્ર સિસ્ટમનું પ્રતિનિધિત્વ કરે છે
\item
  \textbf{બાહ્ય એન્ટિટીઝ} અને મુખ્ય ડેટા ફ્લોઝ
\item
  \textbf{સિસ્ટમ બાઉન્ડ્રીઝનું} ઉચ્ચ-સ્તરનું વિહંગાવલોકન
\end{itemize}

\textbf{લેવલ 1 DFD:}

\begin{itemize}
\tightlist
\item
  \textbf{સિસ્ટમની મુખ્ય} પ્રોસેસીસ
\item
  \textbf{ડેટા સ્ટોર્સ} અને તેમની ક્રિયાપ્રતિક્રિયાઓ
\item
  \textbf{પ્રોસેસીસ વચ્ચે} વિગતવાર ડેટા ફ્લોઝ
\end{itemize}

\textbf{લેવલ 2 અને તે પછી:}

\begin{itemize}
\tightlist
\item
  \textbf{જટિલ પ્રોસેસીસનું} વિઘટન
\item
  \textbf{વધુ વિગતવાર} ડેટા રૂપાંતરણો
\item
  \textbf{નીચલા-સ્તરની} પ્રોસેસ સ્પેસિફિકેશન્સ
\end{itemize}

\textbf{DFD નિયમો:}

\begin{itemize}
\tightlist
\item
  \textbf{પ્રોસેસ નામકરણ}: ક્રિયાપદ + ઑબ્જેક્ટ ઉપયોગ કરો (જેમ કે ``વપરાશકર્તાને
  માન્ય કરો'')
\item
  \textbf{ડેટા ફ્લો નામકરણ}: સંજ્ઞા શબ્દસમૂહોનો ઉપયોગ કરો (જેમ કે ``વપરાશકર્તા
  વિગતો'')
\item
  \textbf{બેલેન્સિંગ}: લેવલ્સ વચ્ચે ઇનપુટ/આઉટપુટ મેચ થવા જોઈએ
\item
  \textbf{બાહ્ય એન્ટિટીઝ વચ્ચે} કોઈ સીધા કનેક્શન્સ નહીં
\end{itemize}

\textbf{ફાયદાઓ:}

\begin{itemize}
\tightlist
\item
  \textbf{સ્ટેકહોલ્ડર્સ સાથે} સ્પષ્ટ કોમ્યુનિકેશન
\item
  \textbf{સિસ્ટમ બાઉન્ડ્રી} ઓળખ
\item
  \textbf{પ્રોસેસ વિશ્લેષણ} અને ઑપ્ટિમાઇઝેશન
\item
  \textbf{સિસ્ટમ ડિઝાઇન} માટે દસ્તાવેજીકરણ
\end{itemize}

\end{solutionbox}
\begin{mnemonicbox}
``PEDS'' - Process, External entity, Data store,
Data flow

\end{mnemonicbox}
\begin{center}\rule{0.5\linewidth}{0.5pt}\end{center}

\subsection*{પ્રશ્ન 3(અ) OR [3
ગુણ]}\label{uxaaauxab0uxab6uxaa8-3uxa85-or-3-uxa97uxaa3}

\textbf{ડિઝાઇન એક્ટિવિટીનું ક્લાસિફિકેશન લખો.}

\begin{solutionbox}

\textbf{ડિઝાઇન એક્ટિવિટીઓ} સોફ્ટવેર ડેવલપમેન્ટમાં તેમના સ્કોપ અને હેતુના આધારે વર્ગીકૃત
કરવામાં આવે છે.

{\def\LTcaptype{none} % do not increment counter
\begin{longtable}[]{@{}lll@{}}
\toprule\noalign{}
વર્ગીકરણ & પ્રવૃત્તિઓ & હેતુ \\
\midrule\noalign{}
\endhead
\bottomrule\noalign{}
\endlastfoot
\textbf{સિસ્ટમ ડિઝાઇન} & આર્કિટેક્ચર, મોડ્યુલ્સ, ઇન્ટરફેસીસ & ઉચ્ચ-સ્તરનું માળખું \\
\textbf{વિગતવાર ડિઝાઇન} & અલ્ગોરિધમ્સ, ડેટા સ્ટ્રક્ચર્સ & અમલીકરણની વિગતો \\
\textbf{ઇન્ટરફેસ ડિઝાઇન} & UI/UX, API સ્પેસિફિકેશન્સ & યુઝર ઇન્ટરેક્શન \\
\textbf{ડેટાબેઝ ડિઝાઇન} & સ્કીમા, સંબંધો, ઑપ્ટિમાઇઝેશન & ડેટા મેનેજમેન્ટ \\
\end{longtable}
}

\textbf{મુખ્ય ડિઝાઇન એક્ટિવિટીઓ:}

\begin{itemize}
\tightlist
\item
  \textbf{આર્કિટેક્ચરલ ડિઝાઇન}: સમગ્ર સિસ્ટમ માળખું
\item
  \textbf{કોમ્પોનેન્ટ ડિઝાઇન}: વ્યક્તિગત મોડ્યુલ સ્પેસિફિકેશન્સ
\item
  \textbf{ડેટા ડિઝાઇન}: ડેટાબેઝ અને ફાઇલ સ્ટ્રક્ચર્સ
\end{itemize}

\end{solutionbox}
\begin{mnemonicbox}
``ACID'' - Architectural, Component, Interface, Data
design

\end{mnemonicbox}
\begin{center}\rule{0.5\linewidth}{0.5pt}\end{center}

\subsection*{પ્રશ્ન 3(બ) OR [4
ગુણ]}\label{uxaaauxab0uxab6uxaa8-3uxaac-or-4-uxa97uxaa3}

\textbf{સારા SRS ની કેરેક્ટરિસ્ટિક લખો.}

\begin{solutionbox}

\textbf{સારું SRS (સોફ્ટવેર રિક્વાયરમેન્ટ સ્પેસિફિકેશન)} દસ્તાવેજમાં અસરકારક
કોમ્યુનિકેશન અને ડેવલપમેન્ટ માટે ચોક્કસ લક્ષણો હોવા જોઈએ.

{\def\LTcaptype{none} % do not increment counter
\begin{longtable}[]{@{}lll@{}}
\toprule\noalign{}
લક્ષણ & વર્ણન & ફાયદો \\
\midrule\noalign{}
\endhead
\bottomrule\noalign{}
\endlastfoot
\textbf{સંપૂર્ણ} & તમામ આવશ્યકતાઓ સામેલ & કોઈ ગુમ થતી કાર્યક્ષમતા નથી \\
\textbf{સુસંગત} & કોઈ વિરોધાભાસી આવશ્યકતાઓ નથી & સ્પષ્ટ સમજ \\
\textbf{અસ્પષ્ટ નથી} & એક જ અર્થઘટન શક્ય & મૂંઝવણ ઘટાડવી \\
\textbf{ચકાસી શકાય તેવું} & આવશ્યકતાઓનું ટેસ્ટ કરી શકાય & ગુણવત્તા ખાતરી \\
\textbf{બદલી શકાય તેવું} & અપડેટ અને જાળવણી સરળ & અનુકૂલનક્ષમતા \\
\textbf{ટ્રેસેબલ} & આવશ્યકતાઓને ટ્રેક કરી શકાય & ચેન્જ મેનેજમેન્ટ \\
\end{longtable}
}

\textbf{વિગતવાર લક્ષણો:}

\textbf{સંપૂર્ણતા:}

\begin{itemize}
\tightlist
\item
  \textbf{તમામ કાર્યાત્મક} આવશ્યકતાઓ સ્પષ્ટ
\item
  \textbf{તમામ બિન-કાર્યાત્મક} આવશ્યકતાઓ સામેલ
\item
  \textbf{તમામ ઇન્ટરફેસીસ} અને મર્યાદાઓ દસ્તાવેજીકૃત
\end{itemize}

\textbf{સુસંગતતા:}

\begin{itemize}
\tightlist
\item
  \textbf{કોઈ વિરોધી} આવશ્યકતાઓ નથી
\item
  \textbf{સમગ્ર દસ્તાવેજમાં} એકસમાન પરિભાષા
\item
  \textbf{સુસંગત ફોર્મેટિંગ} અને માળખું
\end{itemize}

\textbf{ચકાસણીયોગ્યતા:}

\begin{itemize}
\tightlist
\item
  \textbf{સ્પષ્ટ માપદંડો} સાથે ટેસ્ટ કરી શકાય તેવી આવશ્યકતાઓ
\item
  \textbf{માપી શકાય તેવા} ગુણવત્તા લક્ષણો
\item
  \textbf{ઉદ્દેશ્ય સફળતા} માપદંડો વ્યાખ્યાયિત
\end{itemize}

\end{solutionbox}
\begin{mnemonicbox}
``CCUMVT'' - Complete, Consistent, Unambiguous,
Modifiable, Verifiable, Traceable

\end{mnemonicbox}
\begin{center}\rule{0.5\linewidth}{0.5pt}\end{center}

\subsection*{પ્રશ્ન 3(ક) OR [7
ગુણ]}\label{uxaaauxab0uxab6uxaa8-3uxa95-or-7-uxa97uxaa3}

\textbf{વ્હાઇટ બોક્સ ટેસ્ટિંગ સમજાવો.}

\begin{solutionbox}

\textbf{વ્હાઇટ બોક્સ ટેસ્ટિંગ} એ ટેસ્ટિંગ પદ્ધતિ છે જે સોફ્ટવેર એપ્લિકેશન્સના આંતરિક
માળખા, કોડ અને લોજિકની તપાસ કરે છે.

{\def\LTcaptype{none} % do not increment counter
\begin{longtable}[]{@{}
  >{\raggedright\arraybackslash}p{(\linewidth - 2\tabcolsep) * \real{0.4615}}
  >{\raggedright\arraybackslash}p{(\linewidth - 2\tabcolsep) * \real{0.5385}}@{}}
\toprule\noalign{}
\begin{minipage}[b]{\linewidth}\raggedright
પાસું
\end{minipage} & \begin{minipage}[b]{\linewidth}\raggedright
વર્ણન
\end{minipage} \\
\midrule\noalign{}
\endhead
\bottomrule\noalign{}
\endlastfoot
\textbf{અન્ય નામથી ઓળખાય} & સ્ટ્રક્ચરલ ટેસ્ટિંગ, ગ્લાસ બોક્સ ટેસ્ટિંગ, ક્લિયર બોક્સ
ટેસ્ટિંગ \\
\textbf{એક્સેસ લેવલ} & સોર્સ કોડ અને આંતરિક માળખાની સંપૂર્ણ પહોંચ \\
\textbf{ફોકસ} & કોડ કવરેજ, લોજિક પાથ્સ, આંતરિક ડેટા સ્ટ્રક્ચર્સ \\
\textbf{ટેસ્ટરનું જ્ઞાન} & પ્રોગ્રામિંગ જ્ઞાનની જરૂર \\
\end{longtable}
}

\textbf{વ્હાઇટ બોક્સ ટેસ્ટિંગ તકનીકો:}

\begin{center}
\textbf{Mermaid Diagram (Code)}
\begin{verbatim}
{Shaded}
{Highlighting}[]
graph TD
    A[White Box Testing] {-{-}{} B[Statement Coverage]}
    A {-{-}{} C[Branch Coverage]}
    A {-{-}{} D[Path Coverage]}
    A {-{-}{} E[Condition Coverage]}
    
    B {-{-}{} B1[દરેક સ્ટેટમેન્ટ એક્ઝિક્યુટ કરો]}
    C {-{-}{} C1[તમામ ડિસિઝન પોઇન્ટ્સ ટેસ્ટ કરો]}
    D {-{-}{} D1[તમામ શક્ય પાથ્સ ટેસ્ટ કરો]}
    E {-{-}{} E1[તમામ લોજિકલ કંડિશન્સ ટેસ્ટ કરો]}
{Highlighting}
{Shaded}
\end{verbatim}
\end{center}

\textbf{કવરેજ પ્રકારો:}

{\def\LTcaptype{none} % do not increment counter
\begin{longtable}[]{@{}
  >{\raggedright\arraybackslash}p{(\linewidth - 4\tabcolsep) * \real{0.4333}}
  >{\raggedright\arraybackslash}p{(\linewidth - 4\tabcolsep) * \real{0.3333}}
  >{\raggedright\arraybackslash}p{(\linewidth - 4\tabcolsep) * \real{0.2333}}@{}}
\toprule\noalign{}
\begin{minipage}[b]{\linewidth}\raggedright
કવરેજ પ્રકાર
\end{minipage} & \begin{minipage}[b]{\linewidth}\raggedright
ફોર્મ્યુલા
\end{minipage} & \begin{minipage}[b]{\linewidth}\raggedright
વર્ણન
\end{minipage} \\
\midrule\noalign{}
\endhead
\bottomrule\noalign{}
\endlastfoot
\textbf{સ્ટેટમેન્ટ કવરેજ} & (એક્ઝિક્યુટેડ સ્ટેટમેન્ટ્સ / કુલ સ્ટેટમેન્ટ્સ) \times 100\% & કોડની
દરેક લાઇન ટેસ્ટ કરે છે \\
\textbf{બ્રાન્ચ કવરેજ} & (એક્ઝિક્યુટેડ બ્રાન્ચ / કુલ બ્રાન્ચ) \times 100\% & તમામ ડિસિઝન
આઉટકમ્સ ટેસ્ટ કરે છે \\
\textbf{પાથ કવરેજ} & (એક્ઝિક્યુટેડ પાથ્સ / કુલ પાથ્સ) \times 100\% & તમામ એક્ઝિક્યુશન
પાથ્સ ટેસ્ટ કરે છે \\
\textbf{કંડિશન કવરેજ} & (ટેસ્ટેડ કંડિશન્સ / કુલ કંડિશન્સ) \times 100\% & તમામ લોજિકલ
કંડિશન્સ ટેસ્ટ કરે છે \\
\end{longtable}
}

\textbf{ફાયદાઓ:}

\begin{itemize}
\tightlist
\item
  \textbf{કોડ લોજિકનું} સંપૂર્ણ ટેસ્ટિંગ
\item
  \textbf{ડેવલપમેન્ટમાં} પ્રારંભિક ડિફેક્ટ શોધ
\item
  \textbf{કોડ ઑપ્ટિમાઇઝેશન} તકોની ઓળખ
\item
  \textbf{સંપૂર્ણ કોડ કવરેજ} શક્ય
\end{itemize}

\textbf{નુકસાનો:}

\begin{itemize}
\tightlist
\item
  \textbf{મોંઘી અને સમય લેતી} પ્રક્રિયા
\item
  \textbf{ટેસ્ટર્સ પાસેથી} પ્રોગ્રામિંગ સ્કિલ્સની જરૂર
\item
  \textbf{રિક્વાયરમેન્ટ-સંબંધિત} ડિફેક્ટ્સ ચૂકી શકે છે
\item
  \textbf{મોટી એપ્લિકેશન્સ} માટે જટિલ
\end{itemize}

\textbf{ઉપયોગમાં લેવાતા ટૂલ્સ:}

\begin{itemize}
\tightlist
\item
  \textbf{કોડ કવરેજ ટૂલ્સ} (JaCoCo, gcov)
\item
  \textbf{સ્ટેટિક એનાલિસિસ ટૂલ્સ} (SonarQube)
\item
  \textbf{યુનિટ ટેસ્ટિંગ ફ્રેમવર્ક્સ} (JUnit, NUnit)
\end{itemize}

\textbf{ઉદાહરણ ટેસ્ટ કેસીસ:}

\begin{verbatim}
// ટેસ્ટ કરવાનું ફંક્શન
function calculateGrade(marks) \{
    if (marks {=} 90) return {A};
    else if (marks {=} 80) return {B};
    else if (marks {=} 70) return {C};
    else return {F};
\}

// 100\% બ્રાન્ચ કવરેજ માટે વ્હાઇટ બોક્સ ટેસ્ટ કેસીસ
// ટેસ્ટ 1: marks = 95 (A ગ્રેડ પાથ)
// ટેસ્ટ 2: marks = 85 (B ગ્રેડ પાથ)  
// ટેસ્ટ 3: marks = 75 (C ગ્રેડ પાથ)
// ટેસ્ટ 4: marks = 65 (F ગ્રેડ પાથ)
\end{verbatim}

\end{solutionbox}
\begin{mnemonicbox}
``SBPC'' - Statement, Branch, Path, Condition
coverage

\end{mnemonicbox}
\begin{center}\rule{0.5\linewidth}{0.5pt}\end{center}

\subsection*{પ્રશ્ન 4(અ) [3
ગુણ]}\label{uxaaauxab0uxab6uxaa8-4uxa85-3-uxa97uxaa3}

\textbf{RAD મોડેલનું મહત્વ સમજાવો.}

\begin{solutionbox}

\textbf{RAD (રેપિડ એપ્લિકેશન ડેવલપમેન્ટ)} મોડેલ પ્રોટોટાઇપિંગ અને પુનરાવર્તક ડિઝાઇન
દ્વારા ઝડપી વિકાસ પર ભાર મૂકે છે.

{\def\LTcaptype{none} % do not increment counter
\begin{longtable}[]{@{}
  >{\raggedright\arraybackslash}p{(\linewidth - 4\tabcolsep) * \real{0.2692}}
  >{\raggedright\arraybackslash}p{(\linewidth - 4\tabcolsep) * \real{0.3077}}
  >{\raggedright\arraybackslash}p{(\linewidth - 4\tabcolsep) * \real{0.4231}}@{}}
\toprule\noalign{}
\begin{minipage}[b]{\linewidth}\raggedright
મહત્વ
\end{minipage} & \begin{minipage}[b]{\linewidth}\raggedright
ફાયદો
\end{minipage} & \begin{minipage}[b]{\linewidth}\raggedright
એપ્લિકેશન
\end{minipage} \\
\midrule\noalign{}
\endhead
\bottomrule\noalign{}
\endlastfoot
\textbf{ઝડપી ડેવલપમેન્ટ} & માર્કેટ-ટૂ-ટાઇમ ઘટાડો & બિઝનેસ એપ્લિકેશન્સ \\
\textbf{યુઝર ઇન્વોલ્વમેન્ટ} & આવશ્યકતાઓની બહેતર સમજ & ઇન્ટરેક્ટિવ સિસ્ટમ્સ \\
\textbf{પ્રોટોટાઇપ-આધારિત} & પ્રારંભિક પ્રતિસાદ અને વેલિડેશન & UI-ઇન્ટેન્સિવ
એપ્લિકેશન્સ \\
\textbf{કોમ્પોનેન્ટ રીયુઝ} & કિંમત ઘટાડો અને કાર્યક્ષમતા & એન્ટરપ્રાઇઝ એપ્લિકેશન્સ \\
\end{longtable}
}

\textbf{મુખ્ય ફાયદાઓ:}

\begin{itemize}
\tightlist
\item
  \textbf{કાર્યકારી પ્રોટોટાઇપ્સની} ઝડપી ડિલિવરી
\item
  \textbf{ડેવલપમેન્ટ સમય} અને ખર્ચમાં ઘટાડો
\item
  \textbf{સંડોવણી દ્વારા} ઉચ્ચ યુઝર સંતોષ
\item
  \textbf{ડેવલપમેન્ટ દરમિયાન} બદલાવો માટે લવચીક
\end{itemize}

\textbf{RAD ક્યારે ઉપયોગ કરવું:}

\begin{itemize}
\tightlist
\item
  \textbf{સારી રીતે વ્યાખ્યાયિત} બિઝનેસ આવશ્યકતાઓ
\item
  \textbf{અનુભવી ડેવલપમેન્ટ} ટીમ ઉપલબ્ધ
\item
  \textbf{મોડ્યુલર સિસ્ટમ} આર્કિટેક્ચર શક્ય
\end{itemize}

\end{solutionbox}
\begin{mnemonicbox}
``FUPR'' - Fast, User involvement, Prototype-based,
Reusable components

\end{mnemonicbox}
\begin{center}\rule{0.5\linewidth}{0.5pt}\end{center}

\subsection*{પ્રશ્ન 4(બ) [4
ગુણ]}\label{uxaaauxab0uxab6uxaa8-4uxaac-4-uxa97uxaa3}

\textbf{કોડ ઇન્સ્પેક્શન સમજાવો.}

\begin{solutionbox}

\textbf{કોડ ઇન્સ્પેક્શન} સોર્સ કોડની વ્યવસ્થિત તપાસ છે જે ડિફેક્ટ્સ ઓળખવા, ગુણવત્તા
સુધારવા અને ધોરણોનું પાલન સુનિશ્ચિત કરવા માટે કરવામાં આવે છે.

{\def\LTcaptype{none} % do not increment counter
\begin{longtable}[]{@{}
  >{\raggedright\arraybackslash}p{(\linewidth - 6\tabcolsep) * \real{0.1935}}
  >{\raggedright\arraybackslash}p{(\linewidth - 6\tabcolsep) * \real{0.2258}}
  >{\raggedright\arraybackslash}p{(\linewidth - 6\tabcolsep) * \real{0.3548}}
  >{\raggedright\arraybackslash}p{(\linewidth - 6\tabcolsep) * \real{0.2258}}@{}}
\toprule\noalign{}
\begin{minipage}[b]{\linewidth}\raggedright
પ્રકાર
\end{minipage} & \begin{minipage}[b]{\linewidth}\raggedright
વર્ણન
\end{minipage} & \begin{minipage}[b]{\linewidth}\raggedright
સહભાગીઓ
\end{minipage} & \begin{minipage}[b]{\linewidth}\raggedright
અવધિ
\end{minipage} \\
\midrule\noalign{}
\endhead
\bottomrule\noalign{}
\endlastfoot
\textbf{ફોર્મલ ઇન્સ્પેક્શન} & નિર્ધારિત ભૂમિકાઓ સાથે સંરચિત પ્રક્રિયા & 3-6 લોકો &
2-4 કલાક \\
\textbf{વોકથ્રુ} & લેખક-આગેવાની વાળું રિવ્યુ સત્ર & 2-7 લોકો & 1-2 કલાક \\
\textbf{પીઅર રિવ્યુ} & અનૌપચારિક સાથીદાર રિવ્યુ & 2-3 લોકો & 30-60 મિનિટ \\
\textbf{ટૂલ-આધારિત રિવ્યુ} & ઓટોમેટેડ કોડ એનાલિસિસ & વ્યક્તિગત & વિવિધ \\
\end{longtable}
}

\textbf{કોડ ઇન્સ્પેક્શન પ્રક્રિયા:}

\begin{itemize}
\tightlist
\item
  \textbf{આયોજન}: કોડ પસંદ કરવો, ભૂમિકાઓ સોંપવી, મીટિંગ શેડ્યુલ કરવી
\item
  \textbf{ઓવરવ્યુ}: લેખક કોડનો હેતુ અને ડિઝાઇન સમજાવે છે
\item
  \textbf{તૈયારી}: રિવ્યુઅર્સ વ્યક્તિગત રીતે કોડનો અભ્યાસ કરે છે
\item
  \textbf{ઇન્સ્પેક્શન મીટિંગ}: વ્યવસ્થિત ડિફેક્ટ ઓળખ
\item
  \textbf{રીવર્ક}: લેખક ઓળખાયેલ સમસ્યાઓ ઠીક કરે છે
\item
  \textbf{ફોલો-અપ}: ડિફેક્ટ રિઝોલ્યુશનની ચકાસણી
\end{itemize}

\textbf{ફાયદાઓ:}

\begin{itemize}
\tightlist
\item
  \textbf{ટેસ્ટિંગ પહેલાં} પ્રારંભિક ડિફેક્ટ શોધ
\item
  \textbf{ટીમ સભ્યો વચ્ચે} નોલેજ શેરિંગ
\item
  \textbf{કોડ ગુણવત્તા સુધારણા} અને માનકીકરણ
\item
  \textbf{મેન્ટેનન્સ ખર્ચમાં} ઘટાડો
\end{itemize}

\end{solutionbox}
\begin{mnemonicbox}
``FWPT'' - Formal, Walkthrough, Peer review,
Tool-based

\end{mnemonicbox}
\begin{center}\rule{0.5\linewidth}{0.5pt}\end{center}

\subsection*{પ્રશ્ન 4(ક) [7
ગુણ]}\label{uxaaauxab0uxab6uxaa8-4uxa95-7-uxa97uxaa3}

\textbf{કોહેશન કલાસિફિકેશન સાથે સમજાવો.}

\begin{solutionbox}

\textbf{કોહેશન} માપે છે કે એક મોડ્યુલની જવાબદારીઓ કેટલી નજીકથી સંબંધિત અને કેન્દ્રિત
છે. ઊંચું કોહેશન વધુ સારું મોડ્યુલ ડિઝાઇન દર્શાવે છે.

\textbf{કોહેશન પ્રકારો (શ્રેષ્ઠથી ખરાબ ક્રમમાં):}

{\def\LTcaptype{none} % do not increment counter
\begin{longtable}[]{@{}
  >{\raggedright\arraybackslash}p{(\linewidth - 6\tabcolsep) * \real{0.2000}}
  >{\raggedright\arraybackslash}p{(\linewidth - 6\tabcolsep) * \real{0.2333}}
  >{\raggedright\arraybackslash}p{(\linewidth - 6\tabcolsep) * \real{0.3000}}
  >{\raggedright\arraybackslash}p{(\linewidth - 6\tabcolsep) * \real{0.2667}}@{}}
\toprule\noalign{}
\begin{minipage}[b]{\linewidth}\raggedright
પ્રકાર
\end{minipage} & \begin{minipage}[b]{\linewidth}\raggedright
વર્ણન
\end{minipage} & \begin{minipage}[b]{\linewidth}\raggedright
ઉદાહરણ
\end{minipage} & \begin{minipage}[b]{\linewidth}\raggedright
મજબૂતાઈ
\end{minipage} \\
\midrule\noalign{}
\endhead
\bottomrule\noalign{}
\endlastfoot
\textbf{ફંક્શનલ} & એક, સારી રીતે વ્યાખ્યાયિત કાર્ય & ટેક્સ રકમ ગણતરી & સર્વોચ્ચ \\
\textbf{સિક્વેન્શિયલ} & એક એલિમેન્ટનું આઉટપુટ આગળના માટે ઇનપુટ & વાંચો\rightarrowપ્રોસેસ
કરો\rightarrowલખો ડેટા & ઊંચું \\
\textbf{કોમ્યુનિકેશનલ} & એલિમેન્ટ્સ સમાન ડેટા પર કામ કરે છે & ગ્રાહક રેકોર્ડ અપડેટ &
ઊંચું \\
\textbf{પ્રોસીજરલ} & એલિમેન્ટ્સ એક્ઝિક્યુશન સિક્વેન્સ અનુસરે છે &
ઇનિશિયલાઇઝ\rightarrowપ્રોસેસ\rightarrowક્લીનઅપ & મધ્યમ \\
\textbf{ટેમ્પોરલ} & એલિમેન્ટ્સ સમાન સમયે એક્ઝિક્યુટ થાય છે & સિસ્ટમ સ્ટાર્ટઅપ રૂટીન્સ &
મધ્યમ \\
\textbf{લોજિકલ} & સમાન લોજિકલ ફંક્શન્સ ગૃપ કરેલા & તમામ ઇનપુટ/આઉટપુટ ઓપરેશન્સ &
નીચું \\
\textbf{કોઇન્સિડેન્ટલ} & કોઈ અર્થપૂર્ણ સંબંધ નથી & રેન્ડમ યુટિલિટી ફંક્શન્સ & સૌથી
નીચું \\
\end{longtable}
}

\textbf{વિગતવાર વર્ગીકરણ:}

\begin{center}
\textbf{Mermaid Diagram (Code)}
\begin{verbatim}
{Shaded}
{Highlighting}[]
graph TD
    A[કોહેશન પ્રકારો] {-{-}{} B[ફંક્શનલ {-} શ્રેષ્ઠ]}
    A {-{-}{} C[સિક્વેન્શિયલ]}
    A {-{-}{} D[કોમ્યુનિકેશનલ]}
    A {-{-}{} E[પ્રોસીજરલ]}
    A {-{-}{} F[ટેમ્પોરલ]}
    A {-{-}{} G[લોજિકલ]}
    A {-{-}{} H[કોઇન્સિડેન્ટલ {-} ખરાબ]}
    
    style B fill:\#4caf50
    style C fill:\#8bc34a
    style D fill:\#cddc39
    style E fill:\#ffeb3b
    style F fill:\#ff9800
    style G fill:\#ff5722
    style H fill:\#f44336
{Highlighting}
{Shaded}
\end{verbatim}
\end{center}

\textbf{ફંક્શનલ કોહેશન (શ્રેષ્ઠ):}

\begin{itemize}
\tightlist
\item
  \textbf{સિંગલ રિસ્પોન્સિબિલિટી} સિદ્ધાંત
\item
  \textbf{ઉદાહરણ}: \texttt{calculateInterest()} - ફક્ત વ્યાજ ગણતરી કરે છે
\item
  \textbf{ફાયદાઓ}: સમજવા, ટેસ્ટ કરવા અને જાળવવા આસાન
\end{itemize}

\textbf{સિક્વેન્શિયલ કોહેશન:}

\begin{itemize}
\tightlist
\item
  \textbf{ડેટા ફ્લો} એક એલિમેન્ટથી આગળના માટે
\item
  \textbf{ઉદાહરણ}:
  \texttt{readFile()\ \rightarrow\ parseData()\ \rightarrow\ generateReport()}
\item
  \textbf{પ્રોસેસિંગ પાઇપલાઇન્સ} માટે સારું ડિઝાઇન
\end{itemize}

\textbf{કોમ્યુનિકેશનલ કોહેશન:}

\begin{itemize}
\tightlist
\item
  \textbf{સમાન ડેટા સ્ટ્રક્ચર} મેનિપ્યુલેશન
\item
  \textbf{ઉદાહરણ}: ગ્રાહક રેકોર્ડના તમામ ફીલ્ડ અપડેટ કરતું મોડ્યુલ
\item
  \textbf{ડેટા-કેન્દ્રિત ઓપરેશન્સ} માટે વાજબી ડિઝાઇન
\end{itemize}

\textbf{પ્રોસીજરલ કોહેશન:}

\begin{itemize}
\tightlist
\item
  \textbf{કંટ્રોલ ફ્લો} સંબંધ
\item
  \textbf{ઉદાહરણ}: ચોક્કસ ક્રમમાં ઇનિશિયલાઇઝેશન સિક્વેન્સ
\item
  \textbf{પ્રોસીજરલ ઓપરેશન્સ} માટે સ્વીકાર્ય
\end{itemize}

\textbf{ટેમ્પોરલ કોહેશન:}

\begin{itemize}
\tightlist
\item
  \textbf{સમય-આધારિત} સંબંધ
\item
  \textbf{ઉદાહરણ}: સિસ્ટમ સ્ટાર્ટઅપ અથવા શટડાઉન રૂટીન્સ
\item
  \textbf{મધ્યમ ગુણવત્તા} ડિઝાઇન
\end{itemize}

\textbf{લોજિકલ કોહેશન:}

\begin{itemize}
\tightlist
\item
  \textbf{સમાન ફંક્શન્સ} એકસાથે ગૃપ કરેલા
\item
  \textbf{ઉદાહરણ}: એક મોડ્યુલમાં તમામ મેથેમેટિકલ ફંક્શન્સ
\item
  \textbf{ખરાબ ડિઝાઇન} - જાળવવું મુશ્કેલ
\end{itemize}

\textbf{કોઇન્સિડેન્ટલ કોહેશન (ખરાબ):}

\begin{itemize}
\tightlist
\item
  \textbf{એલિમેન્ટ્સ વચ્ચે કોઈ} લોજિકલ સંબંધ નથી
\item
  \textbf{ઉદાહરણ}: મિસેલેનિયસ યુટિલિટી ફંક્શન્સ
\item
  \textbf{આને ટાળો} - મેન્ટેનન્સ નાઇટમેર બનાવે છે
\end{itemize}

\textbf{ઊંચા કોહેશનના ફાયદાઓ:}

\begin{itemize}
\tightlist
\item
  \textbf{સરળ મેન્ટેનન્સ} અને ડીબગિંગ
\item
  \textbf{મોડ્યુલ્સની વધુ સારી} પુનઃઉપયોગિતા
\item
  \textbf{સુધારેલી ટેસ્ટેબિલિટી} અને વિશ્વસનીયતા
\item
  \textbf{સ્પષ્ટ કોડ} સમજ
\end{itemize}

\textbf{ઊંચું કોહેશન કેવી રીતે પ્રાપ્ત કરવું:}

\begin{itemize}
\tightlist
\item
  \textbf{સિંગલ રિસ્પોન્સિબિલિટી પ્રિન્સિપલ}: બદલાવનું એક કારણ
\item
  \textbf{સ્પષ્ટ મોડ્યુલ હેતુ}: સારી રીતે વ્યાખ્યાયિત કાર્યક્ષમતા
\item
  \textbf{મિનિમલ ઇન્ટરફેસીસ}: બાહ્ય નિર્ભરતા ઘટાડવી
\item
  \textbf{લોજિકલ ગૃપિંગ}: સંબંધિત ફંક્શન્સ એકસાથે
\end{itemize}

\end{solutionbox}
\begin{mnemonicbox}
``FSCPTLC'' - Functional, Sequential,
Communicational, Procedural, Temporal, Logical, Coincidental

\end{mnemonicbox}
\begin{center}\rule{0.5\linewidth}{0.5pt}\end{center}

\subsection*{પ્રશ્ન 4(અ) OR [3
ગુણ]}\label{uxaaauxab0uxab6uxaa8-4uxa85-or-3-uxa97uxaa3}

\textbf{સોફ્ટવેર વેર આઉટ થતું નથી.}

\begin{solutionbox}

\textbf{સોફ્ટવેર વેર આઉટ થતું નથી} એનો અર્થ છે કે સોફ્ટવેર હાર્ડવેર કોમ્પોનેન્ટ્સની જેમ
સમય સાથે શારીરિક રીતે બગડતું નથી.

{\def\LTcaptype{none} % do not increment counter
\begin{longtable}[]{@{}lll@{}}
\toprule\noalign{}
પાસું & હાર્ડવેર & સોફ્ટવેર \\
\midrule\noalign{}
\endhead
\bottomrule\noalign{}
\endlastfoot
\textbf{શારીરિક ડિગ્રેડેશન} & કોમ્પોનેન્ટ્સ ઘસાઈ જાય છે & કોઈ શારીરિક ડિગ્રેડેશન
નથી \\
\textbf{વય અસર} & પ્રદર્શન ઘટે છે & પ્રદર્શન સ્થિર રહે છે \\
\textbf{ફેઇલ્યુર પેટર્ન} & વધતો ફેઇલ્યુર રેટ & સ્થિર ફેઇલ્યુર રેટ \\
\textbf{મેન્ટેનન્સ} & ઘસાયેલા ભાગો બદલવા & ફક્ત લોજિકલ એરર્સ ઠીક કરવા \\
\end{longtable}
}

\textbf{મુખ્ય મુદ્દાઓ:}

\begin{itemize}
\tightlist
\item
  \textbf{કોઈ મિકેનિકલ ભાગો} ઘસાવા માટે નથી
\item
  \textbf{લોજિકલ એરર્સ} સમય સાથે વધતા નથી
\item
  \textbf{પ્રદર્શન ડિગ્રેડેશન} વાતાવરણીય ફેરફારોને કારણે, વય કારણે નહીં
\item
  \textbf{ફેઇલ્યુર થાય છે} ડિઝાઇન ખામીઓને કારણે, ઘસારાને કારણે નહીં
\end{itemize}

\textbf{આ કેમ મહત્વપૂર્ણ છે:}

\begin{itemize}
\tightlist
\item
  \textbf{અલગ મેન્ટેનન્સ} અભિગમની જરૂર
\item
  \textbf{બદલી કરવાને બદલે} અપડેટ્સ પર ફોકસ
\item
  \textbf{લોન્જેવિટી પ્લાનિંગ} હાર્ડવેરથી અલગ
\end{itemize}

\end{solutionbox}
\begin{mnemonicbox}
``NLPF'' - No physical parts, Logical errors,
Performance constant, Failures from design

\end{mnemonicbox}
\begin{center}\rule{0.5\linewidth}{0.5pt}\end{center}

\subsection*{પ્રશ્ન 4(બ) OR [4
ગુણ]}\label{uxaaauxab0uxab6uxaa8-4uxaac-or-4-uxa97uxaa3}

\textbf{યુઝ કેસ ડાયાગ્રામ સમજાવો.}

\begin{solutionbox}

\textbf{યુઝ કેસ ડાયાગ્રામ} એક UML વર્તણૂકીય ડાયાગ્રામ છે જે એક્ટર્સ અને યુઝ કેસીસ
વચ્ચેની ક્રિયાપ્રતિક્રિયાઓ દ્વારા યુઝરના પરસ્પેક્ટિવથી સિસ્ટમ કાર્યક્ષમતા બતાવે છે.

{\def\LTcaptype{none} % do not increment counter
\begin{longtable}[]{@{}lll@{}}
\toprule\noalign{}
કોમ્પોનેન્ટ & સિમ્બોલ & વર્ણન \\
\midrule\noalign{}
\endhead
\bottomrule\noalign{}
\endlastfoot
\textbf{એક્ટર} & સ્ટિક ફિગર & સિસ્ટમ સાથે ક્રિયાપ્રતિક્રિયા કરતી બાહ્ય એન્ટિટી \\
\textbf{યુઝ કેસ} & અંડાકાર & સિસ્ટમ ફંક્શન અથવા સેવા \\
\textbf{સિસ્ટમ બાઉન્ડ્રી} & લંબચોરસ & સિસ્ટમ સ્કોપ વ્યાખ્યા \\
\textbf{સંબંધો} & લાઇન્સ/એરો & કોમ્પોનેન્ટ્સ વચ્ચે એસોસિએશન્સ \\
\end{longtable}
}

\textbf{યુઝ કેસ ડાયાગ્રામ એલિમેન્ટ્સ:}

\begin{center}
\textbf{Mermaid Diagram (Code)}
\begin{verbatim}
{Shaded}
{Highlighting}[]
graph TD
    subgraph "Library Management System"
      direction LR
        UC1((Search Books))
        UC2((Borrow Book))
        UC3((Return Book))
        UC4((Manage Catalog))
        UC5((Generate Reports))
    end
    
    A1[Student] {-{-}{} UC1}
    A1 {-{-}{} UC2}
    A1 {-{-}{} UC3}
    
    A2[Librarian] {-{-}{} UC4}
    A2 {-{-}{} UC5}
    A2 {-{-}{} UC1}
    
    UC2 {-.{-}{} UC1}
    UC3 {-.{-}{} UC1}
{Highlighting}
{Shaded}
\end{verbatim}
\end{center}

\textbf{સંબંધ પ્રકારો:}

\begin{itemize}
\tightlist
\item
  \textbf{એસોસિએશન}: એક્ટર યુઝ કેસમાં ભાગ લે છે
\item
  \textbf{ઇન્ક્લુડ}: યુઝ કેસ હંમેશા બીજા યુઝ કેસનો સમાવેશ કરે છે
\item
  \textbf{એક્સ્ટેન્ડ}: યુઝ કેસ શરતી રીતે બીજાનું વિસ્તરણ કરે છે
\item
  \textbf{જનરલાઇઝેશન}: એક્ટર્સ/યુઝ કેસીસ વચ્ચે ઇન્હેરિટન્સ
\end{itemize}

\textbf{ફાયદાઓ:}

\begin{itemize}
\tightlist
\item
  \textbf{સ્પષ્ટ સિસ્ટમ સ્કોપ} વ્યાખ્યા
\item
  \textbf{યુઝર રિક્વાયરમેન્ટ્સનું} વિઝ્યુઅલાઇઝેશન
\item
  \textbf{સ્ટેકહોલ્ડર્સ સાથે} કોમ્યુનિકેશન ટૂલ
\item
  \textbf{ટેસ્ટ કેસ} બનાવવાનો આધાર
\end{itemize}

\end{solutionbox}
\begin{mnemonicbox}
``AUSB'' - Actor, Use case, System boundary,
Relationships

\end{mnemonicbox}
\begin{center}\rule{0.5\linewidth}{0.5pt}\end{center}

\subsection*{પ્રશ્ન 4(ક) OR [7
ગુણ]}\label{uxaaauxab0uxab6uxaa8-4uxa95-or-7-uxa97uxaa3}

\textbf{બ્લેક બોક્સ ટેસ્ટિંગ સમજાવો.}

\begin{solutionbox}

\textbf{બ્લેક બોક્સ ટેસ્ટિંગ} એ ટેસ્ટિંગ પદ્ધતિ છે જે આંતરિક કોડ સ્ટ્રક્ચર અથવા
અમલીકરણની વિગતોના જ્ઞાન વિના સોફ્ટવેર કાર્યક્ષમતાની તપાસ કરે છે.

{\def\LTcaptype{none} % do not increment counter
\begin{longtable}[]{@{}
  >{\raggedright\arraybackslash}p{(\linewidth - 2\tabcolsep) * \real{0.4615}}
  >{\raggedright\arraybackslash}p{(\linewidth - 2\tabcolsep) * \real{0.5385}}@{}}
\toprule\noalign{}
\begin{minipage}[b]{\linewidth}\raggedright
પાસું
\end{minipage} & \begin{minipage}[b]{\linewidth}\raggedright
વર્ણન
\end{minipage} \\
\midrule\noalign{}
\endhead
\bottomrule\noalign{}
\endlastfoot
\textbf{અન્ય નામથી ઓળખાય} & ફંક્શનલ ટેસ્ટિંગ, બિહેવિયરલ ટેસ્ટિંગ, સ્પેસિફિકેશન-આધારિત
ટેસ્ટિંગ \\
\textbf{એક્સેસ લેવલ} & સોર્સ કોડ અથવા આંતરિક માળખાની કોઈ પહોંચ નથી \\
\textbf{ફોકસ} & ઇનપુટ-આઉટપુટ વર્તન, કાર્યાત્મક આવશ્યકતાઓ \\
\textbf{ટેસ્ટરનું જ્ઞાન} & ડોમેઇન જ્ઞાનની જરૂર, પ્રોગ્રામિંગ નહીં \\
\end{longtable}
}

\textbf{બ્લેક બોક્સ ટેસ્ટિંગ તકનીકો:}

\begin{center}
\textbf{Mermaid Diagram (Code)}
\begin{verbatim}
{Shaded}
{Highlighting}[]
graph TD
    A[Black Box Testing] {-{-}{} B[Equivalence Partitioning]}
    A {-{-}{} C[Boundary Value Analysis]}
    A {-{-}{} D[Decision Table Testing]}
    A {-{-}{} E[State Transition Testing]}
    
    B {-{-}{} B1[વેલિડ/ઇનવેલિડ ઇનપુટ ક્લાસીસ]}
    C {-{-}{} C1[બાઉન્ડ્રી કંડિશન્સ ટેસ્ટ કરો]}
    D {-{-}{} D1[જટિલ બિઝનેસ રૂલ્સ]}
    E {-{-}{} E1[સ્ટેટ{-}આધારિત વર્તન]}
{Highlighting}
{Shaded}
\end{verbatim}
\end{center}

\textbf{ટેસ્ટિંગ તકનીકો:}

{\def\LTcaptype{none} % do not increment counter
\begin{longtable}[]{@{}
  >{\raggedright\arraybackslash}p{(\linewidth - 4\tabcolsep) * \real{0.3043}}
  >{\raggedright\arraybackslash}p{(\linewidth - 4\tabcolsep) * \real{0.3043}}
  >{\raggedright\arraybackslash}p{(\linewidth - 4\tabcolsep) * \real{0.3913}}@{}}
\toprule\noalign{}
\begin{minipage}[b]{\linewidth}\raggedright
તકનીક
\end{minipage} & \begin{minipage}[b]{\linewidth}\raggedright
વર્ણન
\end{minipage} & \begin{minipage}[b]{\linewidth}\raggedright
ઉદાહરણ
\end{minipage} \\
\midrule\noalign{}
\endhead
\bottomrule\noalign{}
\endlastfoot
\textbf{ઇક્વિવેલન્સ પાર્ટિશનિંગ} & ઇનપુટ્સને વેલિડ/ઇનવેલિડ ગૃપોમાં વહેંચવા & વય: 0-17,
18-60, 60+ \\
\textbf{બાઉન્ડ્રી વેલ્યુ એનાલિસિસ} & ઇનપુટ રેન્જીસની બાઉન્ડ્રીઝ પર ટેસ્ટ & 17, 18,
60, 61 પર ટેસ્ટ \\
\textbf{ડિસિઝન ટેબલ} & કંડિશન્સના કોમ્બિનેશન ટેસ્ટ & વેલિડ/ઇનવેલિડ યુઝર/પાસવર્ડ સાથે
લોગિન \\
\textbf{સ્ટેટ ટ્રાન્ઝિશન} & સ્ટેટ ચેન્જીસ ટેસ્ટ & ATM સ્ટેટ્સ: આઇડલ\rightarrowકાર્ડ ઇન્સર્ટ\rightarrowPIN
એન્ટ્રી \\
\end{longtable}
}

\textbf{ટેસ્ટ કેસ ડિઝાઇન ઉદાહરણ:}

\begin{verbatim}
ફંક્શન: લોગિન વેલિડેશન
ઇનપુટ્સ: યુઝરનેમ, પાસવર્ડ

વેલિડ ઇક્વિવેલન્સ ક્લાસીસ:
- યુઝરનેમ: 5-20 કેરેક્ટર્સ, આલ્ફાન્યુમેરિક
- પાસવર્ડ: 8-15 કેરેક્ટર્સ, સ્પેશિયલ કેરેક્ટર્સ મંજૂર

ઇનવેલિડ ઇક્વિવેલન્સ ક્લાસીસ:
- યુઝરનેમ: <5 અથવા >20 કેરેક્ટર્સ, સ્પેશિયલ કેરેક્ટર્સ
- પાસવર્ડ: <8 અથવા >15 કેરેક્ટર્સ, સ્પેસીસ

ટેસ્ટ કરવાની બાઉન્ડ્રી વેલ્યુઝ:
- યુઝરનેમ: 4, 5, 20, 21
- પાસવર્ડ: 7, 8, 15, 16
\end{verbatim}

\textbf{ફાયદાઓ:}

\begin{itemize}
\tightlist
\item
  \textbf{ટેસ્ટર્સ માટે} પ્રોગ્રામિંગ જ્ઞાનની જરૂર નથી
\item
  \textbf{યુઝર પરસ્પેક્ટિવ} ટેસ્ટિંગ અભિગમ
\item
  \textbf{રિક્વાયરમેન્ટ્સનું} સ્વતંત્ર વેરિફિકેશન
\item
  \textbf{મોટી એપ્લિકેશન્સ} માટે અસરકારક
\end{itemize}

\textbf{નુકસાનો:}

\begin{itemize}
\tightlist
\item
  \textbf{કોડ કવરેજની} મર્યાદિત દૃશ્યતા
\item
  \textbf{બિનઉપયોગી કોડ પાથ્સ} ઓળખી શકતું નથી
\item
  \textbf{સ્પેસિફિકેશન્સ વિના} ટેસ્ટ કેસીસ ડિઝાઇન કરવા મુશ્કેલ
\item
  \textbf{કોડમાં લોજિકલ એરર્સ} ચૂકી શકે છે
\end{itemize}

\textbf{બ્લેક બોક્સ ટેસ્ટિંગના પ્રકારો:}

\begin{itemize}
\tightlist
\item
  \textbf{ફંક્શનલ ટેસ્ટિંગ}: ફીચર વેરિફિકેશન
\item
  \textbf{ઇન્ટિગ્રેશન ટેસ્ટિંગ}: મોડ્યુલ ઇન્ટરેક્શન ટેસ્ટિંગ
\item
  \textbf{સિસ્ટમ ટેસ્ટિંગ}: સંપૂર્ણ સિસ્ટમ વેલિડેશન
\item
  \textbf{એક્સેપ્ટન્સ ટેસ્ટિંગ}: યુઝર રિક્વાયરમેન્ટ વેરિફિકેશન
\end{itemize}

\textbf{ઉપયોગમાં લેવાતા ટૂલ્સ:}

\begin{itemize}
\tightlist
\item
  \textbf{ટેસ્ટ મેનેજમેન્ટ ટૂલ્સ} (TestRail, Zephyr)
\item
  \textbf{ઓટોમેશન ટૂલ્સ} (Selenium, QTP)
\item
  \textbf{ડિફેક્ટ ટ્રેકિંગ ટૂલ્સ} (Jira, Bugzilla)
\end{itemize}

\textbf{ક્યારે ઉપયોગ કરવું:}

\begin{itemize}
\tightlist
\item
  \textbf{રિક્વાયરમેન્ટ્સ-આધારિત} ટેસ્ટિંગ
\item
  \textbf{યુઝર એક્સેપ્ટન્સ} ટેસ્ટિંગ
\item
  \textbf{સિસ્ટમ ઇન્ટિગ્રેશન} ટેસ્ટિંગ
\item
  \textbf{ફેરફારો પછી} રિગ્રેશન ટેસ્ટિંગ
\end{itemize}

\end{solutionbox}
\begin{mnemonicbox}
``EBDS'' - Equivalence, Boundary, Decision table,
State transition

\end{mnemonicbox}
\begin{center}\rule{0.5\linewidth}{0.5pt}\end{center}

\subsection*{પ્રશ્ન 5(અ) [3
ગુણ]}\label{uxaaauxab0uxab6uxaa8-5uxa85-3-uxa97uxaa3}

\textbf{વેરિફિકેશન અને વેલિડેશનનો તફાવત લખો.}

\begin{solutionbox}

{\def\LTcaptype{none} % do not increment counter
\begin{longtable}[]{@{}
  >{\raggedright\arraybackslash}p{(\linewidth - 4\tabcolsep) * \real{0.2000}}
  >{\raggedright\arraybackslash}p{(\linewidth - 4\tabcolsep) * \real{0.4333}}
  >{\raggedright\arraybackslash}p{(\linewidth - 4\tabcolsep) * \real{0.3667}}@{}}
\toprule\noalign{}
\begin{minipage}[b]{\linewidth}\raggedright
પાસું
\end{minipage} & \begin{minipage}[b]{\linewidth}\raggedright
વેરિફિકેશન
\end{minipage} & \begin{minipage}[b]{\linewidth}\raggedright
વેલિડેશન
\end{minipage} \\
\midrule\noalign{}
\endhead
\bottomrule\noalign{}
\endlastfoot
\textbf{વ્યાખ્યા} & ``અમે યોગ્ય પ્રોડક્ટ બનાવી રહ્યા છીએ?'' & ``અમે યોગ્ય પ્રોડક્ટ
બનાવી રહ્યા છીએ?'' \\
\textbf{ફોકસ} & પ્રક્રિયા અનુપાલન & પ્રોડક્ટ યોગ્યતા \\
\textbf{ક્યારે} & ડેવલપમેન્ટ દરમિયાન & ડેવલપમેન્ટ પછી \\
\textbf{પદ્ધતિ} & રિવ્યુઝ, ઇન્સ્પેક્શન્સ, વોકથ્રુઝ & વાસ્તવિક ડેટા સાથે ટેસ્ટિંગ \\
\textbf{ખર્ચ} & ડિફેક્ટ શોધનો નીચો ખર્ચ & ડિફેક્ટ શોધનો ઊંચો ખર્ચ \\
\end{longtable}
}

\textbf{મુખ્ય તફાવતો:}

\begin{itemize}
\tightlist
\item
  \textbf{વેરિફિકેશન}: \textbf{સ્પેસિફિકેશન્સ} સામે ચકાસે છે
\item
  \textbf{વેલિડેશન}: \textbf{યુઝર જરૂરિયાતો} સામે ચકાસે છે
\item
  \textbf{વેરિફિકેશન}: \textbf{સ્ટેટિક ટેસ્ટિંગ} પદ્ધતિઓ
\item
  \textbf{વેલિડેશન}: \textbf{ડાયનેમિક ટેસ્ટિંગ} પદ્ધતિઓ
\end{itemize}

\textbf{ઉદાહરણો:}

\begin{itemize}
\tightlist
\item
  \textbf{વેરિફિકેશન}: કોડ રિવ્યુ, ડિઝાઇન રિવ્યુ, SRS રિવ્યુ
\item
  \textbf{વેલિડેશન}: યુનિટ ટેસ્ટિંગ, ઇન્ટિગ્રેશન ટેસ્ટિંગ, સિસ્ટમ ટેસ્ટિંગ
\end{itemize}

\end{solutionbox}
\begin{mnemonicbox}
``VR vs VT'' - Verification Reviews vs Validation
Testing

\end{mnemonicbox}
\begin{center}\rule{0.5\linewidth}{0.5pt}\end{center}

\subsection*{પ્રશ્ન 5(બ) [4
ગુણ]}\label{uxaaauxab0uxab6uxaa8-5uxaac-4-uxa97uxaa3}

\textbf{એસઆરએસ સમજાવો.}

\begin{solutionbox}

\textbf{SRS (સોફ્ટવેર રિક્વાયરમેન્ટ સ્પેસિફિકેશન)} એ સોફ્ટવેર સિસ્ટમની કાર્યાત્મક અને
બિન-કાર્યાત્મક આવશ્યકતાઓનું વર્ણન કરતું વિગતવાર દસ્તાવેજ છે.

{\def\LTcaptype{none} % do not increment counter
\begin{longtable}[]{@{}
  >{\raggedright\arraybackslash}p{(\linewidth - 4\tabcolsep) * \real{0.5000}}
  >{\raggedright\arraybackslash}p{(\linewidth - 4\tabcolsep) * \real{0.2917}}
  >{\raggedright\arraybackslash}p{(\linewidth - 4\tabcolsep) * \real{0.2083}}@{}}
\toprule\noalign{}
\begin{minipage}[b]{\linewidth}\raggedright
કોમ્પોનેન્ટ
\end{minipage} & \begin{minipage}[b]{\linewidth}\raggedright
વર્ણન
\end{minipage} & \begin{minipage}[b]{\linewidth}\raggedright
હેતુ
\end{minipage} \\
\midrule\noalign{}
\endhead
\bottomrule\noalign{}
\endlastfoot
\textbf{પરિચય} & સિસ્ટમ ઓવરવ્યુ અને સ્કોપ & કોન્ટેક્સ્ટ સેટિંગ \\
\textbf{કાર્યાત્મક આવશ્યકતાઓ} & સિસ્ટમે શું કરવું જોઈએ & ફીચર સ્પેસિફિકેશન \\
\textbf{બિન-કાર્યાત્મક આવશ્યકતાઓ} & સિસ્ટમે કેવી રીતે કામ કરવું જોઈએ & ગુણવત્તા
લક્ષણો \\
\textbf{મર્યાદાઓ} & સીમાઓ અને પ્રતિબંધો & બાઉન્ડ્રી વ્યાખ્યા \\
\end{longtable}
}

\textbf{SRS માળખું:}

\begin{itemize}
\tightlist
\item
  \textbf{સિસ્ટમ હેતુ}: સિસ્ટમની જરૂર શા માટે છે
\item
  \textbf{સિસ્ટમ સ્કોપ}: સિસ્ટમ શું કરશે અને શું કરશે નહીં
\item
  \textbf{વ્યાખ્યાઓ}: તકનીકી શબ્દો અને સંક્ષેપો
\item
  \textbf{યુઝર આવશ્યકતાઓ}: ઉચ્ચ-સ્તરીય યુઝર જરૂરિયાતો
\item
  \textbf{સિસ્ટમ આવશ્યકતાઓ}: વિગતવાર તકનીકી સ્પેસિફિકેશન્સ
\end{itemize}

\textbf{SRS નું મહત્વ:}

\begin{itemize}
\tightlist
\item
  \textbf{સ્ટેકહોલ્ડર્સ વચ્ચે} કોમ્યુનિકેશન ટૂલ
\item
  \textbf{ટેસ્ટિંગ અને વેલિડેશન} માટે બેઝલાઇન
\item
  \textbf{ક્લાયન્ટ અને ડેવલપર} વચ્ચે કોન્ટ્રાક્ટ આધાર
\item
  \textbf{ચેન્જ મેનેજમેન્ટ} રેફરન્સ દસ્તાવેજ
\end{itemize}

\textbf{SRS ના વપરાશકર્તાઓ:}

\begin{itemize}
\tightlist
\item
  \textbf{ડેવલપર્સ}: અમલીકરણ માર્ગદર્શન
\item
  \textbf{ટેસ્ટર્સ}: ટેસ્ટ કેસ બનાવવા
\item
  \textbf{પ્રોજેક્ટ મેનેજર્સ}: આયોજન અને ટ્રેકિંગ
\item
  \textbf{ક્લાયન્ટ્સ}: આવશ્યકતા વેરિફિકેશન
\end{itemize}

\end{solutionbox}
\begin{mnemonicbox}
``IFNC'' - Introduction, Functional, Non-functional,
Constraints

\end{mnemonicbox}
\begin{center}\rule{0.5\linewidth}{0.5pt}\end{center}

\subsection*{પ્રશ્ન 5(ક) [7
ગુણ]}\label{uxaaauxab0uxab6uxaa8-5uxa95-7-uxa97uxaa3}

\textbf{રિસ્ક મેનેજમેન્ટ સમજાવો.}

\begin{solutionbox}

\textbf{રિસ્ક મેનેજમેન્ટ} પ્રોજેક્ટની સફળતા પર તેમની અસરને ઘટાડવા માટે પ્રોજેક્ટ
જોખમોને ઓળખવા, વિશ્લેષણ કરવા અને તેનો પ્રતિસાદ આપવાની વ્યવસ્થિત પ્રક્રિયા છે.

\textbf{રિસ્ક મેનેજમેન્ટ પ્રક્રિયા:}

\begin{verbatim}
flowchart LR
    A[જોખમ ઓળખ] {-{-} B[જોખમ વિશ્લેષણ]}
    B {-{-} C[જોખમ મૂલ્યાંકન]}
    C {-{-} D[જોખમ ઘટાડો]}
    D {-{-} E[જોખમ મોનિટરિંગ]}
    E {-{-} A}
    
    style A fill:\#ffebee
    style B fill:\#e8f5e8
    style C fill:\#fff3e0
    style D fill:\#e3f2fd
    style E fill:\#f3e5f5
\end{verbatim}

{\def\LTcaptype{none} % do not increment counter
\begin{longtable}[]{@{}lll@{}}
\toprule\noalign{}
તબક્કો & પ્રવૃત્તિઓ & આઉટપુટ \\
\midrule\noalign{}
\endhead
\bottomrule\noalign{}
\endlastfoot
\textbf{ઓળખ} & બ્રેઇનસ્ટોર્મિંગ, ચેકલિસ્ટ્સ, એક્સપર્ટ જજમેન્ટ & રિસ્ક રજિસ્ટર \\
\textbf{વિશ્લેષણ} & સંભાવના અને અસર મૂલ્યાંકન & રિસ્ક મેટ્રિક્સ \\
\textbf{મૂલ્યાંકન} & જોખમ પ્રાધાન્ય અને રેન્કિંગ & રિસ્ક પ્રાધાન્ય સૂચિ \\
\textbf{ઘટાડો} & પ્રતિસાદ વ્યૂહરચના વિકાસ & ઘટાડાની યોજનાઓ \\
\textbf{મોનિટરિંગ} & જોખમો અને ઘટાડાની અસરકારકતા ટ્રેક કરવી & સ્થિતિ
રિપોર્ટ્સ \\
\end{longtable}
}

\textbf{જોખમ શ્રેણીઓ:}

\textbf{પ્રોજેક્ટ રિસ્ક્સ:}

\begin{itemize}
\tightlist
\item
  \textbf{શેડ્યુલ વિલંબ} રિસોર્સ અનુપલબ્ધતાને કારણે
\item
  \textbf{બજેટ ઓવરરન} સ્કોપ ફેરફારોથી
\item
  \textbf{ટીમ ટર્નઓવર} ઉત્પાદકતાને અસર કરે છે
\item
  \textbf{કોમ્યુનિકેશન ગેપ્સ} સ્ટેકહોલ્ડર્સ વચ્ચે
\end{itemize}

\textbf{તકનીકી રિસ્ક્સ:}

\begin{itemize}
\tightlist
\item
  \textbf{ટેક્નોલોજી જટિલતા} ટીમ સ્કિલ્સ કરતાં વધારે
\item
  \textbf{હાલની સિસ્ટમ્સ સાથે} ઇન્ટિગ્રેશન પડકારો
\item
  \textbf{લોડ કંડિશન્સ હેઠળ} પ્રદર્શન સમસ્યાઓ
\item
  \textbf{ડિઝાઇનમાં} સિક્યોરિટી વલ્નરેબિલિટીઝ
\end{itemize}

\textbf{બિઝનેસ રિસ્ક્સ:}

\begin{itemize}
\tightlist
\item
  \textbf{માર્કેટ કંડિશન્સથી} બદલાતી આવશ્યકતાઓ
\item
  \textbf{કોમ્પિટિશન} સમાન પ્રોડક્ટ્સ રિલીઝ કરે છે
\item
  \textbf{રેગ્યુલેટરી ફેરફારો} કોમ્પ્લાયન્સને અસર કરે છે
\item
  \textbf{પ્રાધાન્યતાઓ પર} સ્ટેકહોલ્ડર સંઘર્ષો
\end{itemize}

\textbf{રિસ્ક રિસ્પોન્સ વ્યૂહરચનાઓ:}

{\def\LTcaptype{none} % do not increment counter
\begin{longtable}[]{@{}
  >{\raggedright\arraybackslash}p{(\linewidth - 6\tabcolsep) * \real{0.2273}}
  >{\raggedright\arraybackslash}p{(\linewidth - 6\tabcolsep) * \real{0.1591}}
  >{\raggedright\arraybackslash}p{(\linewidth - 6\tabcolsep) * \real{0.4091}}
  >{\raggedright\arraybackslash}p{(\linewidth - 6\tabcolsep) * \real{0.2045}}@{}}
\toprule\noalign{}
\begin{minipage}[b]{\linewidth}\raggedright
વ્યૂહરચના
\end{minipage} & \begin{minipage}[b]{\linewidth}\raggedright
વર્ણન
\end{minipage} & \begin{minipage}[b]{\linewidth}\raggedright
ક્યારે ઉપયોગ કરવો
\end{minipage} & \begin{minipage}[b]{\linewidth}\raggedright
ઉદાહરણ
\end{minipage} \\
\midrule\noalign{}
\endhead
\bottomrule\noalign{}
\endlastfoot
\textbf{સ્વીકાર} & જોખમ સ્વીકારવું, કોઈ ક્રિયા નહીં & નીચી અસરવાળા જોખમો &
નાના UI ફેરફારો \\
\textbf{ટાળવું} & જોખમ સ્રોત દૂર કરવો & ઊંચી અસર, ટાળી શકાય તેવા & ટેક્નોલોજી
બદલવી \\
\textbf{ઘટાડવું} & સંભાવના/અસર ઘટાડવી & વ્યવસ્થાપન યોગ્ય જોખમો & વધારાનું
ટેસ્ટિંગ \\
\textbf{ટ્રાન્સફર કરવું} & જોખમ થર્ડ પાર્ટીને શિફ્ટ કરવું & વિશેષીકૃત જોખમો &
ઇન્શ્યોરન્સ, આઉટસોર્સિંગ \\
\end{longtable}
}

\textbf{રિસ્ક એસેસમેન્ટ મેટ્રિક્સ:}

{\def\LTcaptype{none} % do not increment counter
\begin{longtable}[]{@{}llll@{}}
\toprule\noalign{}
સંભાવના/અસર & નીચી & મધ્યમ & ઊંચી \\
\midrule\noalign{}
\endhead
\bottomrule\noalign{}
\endlastfoot
\textbf{ઊંચી} & મધ્યમ & ઊંચી & ક્રિટિકલ \\
\textbf{મધ્યમ} & નીચી & મધ્યમ & ઊંચી \\
\textbf{નીચી} & ખૂબ નીચી & નીચી & મધ્યમ \\
\end{longtable}
}

\textbf{રિસ્ક મિટિગેશન તકનીકો:}

\begin{itemize}
\tightlist
\item
  \textbf{પ્રોટોટાઇપિંગ} તકનીકી અનિશ્ચિતતા ઘટાડવા માટે
\item
  \textbf{સ્ટાફ ટ્રેનિંગ} સ્કિલ ગેપ્સને સંબોધવા માટે
\item
  \textbf{નિયમિત રિવ્યુઝ} વહેલી સમસ્યાઓ પકડવા માટે
\item
  \textbf{કોન્ટિન્જન્સી પ્લાનિંગ} ક્રિટિકલ સ્થિતિઓ માટે
\end{itemize}

\textbf{રિસ્ક મેનેજમેન્ટના ફાયદાઓ:}

\begin{itemize}
\tightlist
\item
  \textbf{પ્રોએક્ટિવ સમસ્યા} નિવારણ
\item
  \textbf{રિસ્ક જાગૃતિ સાથે} બહેતર નિર્ણય લેવો
\item
  \textbf{પ્રોજેક્ટની સફળતાના} સુધારેલા દરો
\item
  \textbf{પ્રોજેક્ટ ડિલિવરીમાં} સ્ટેકહોલ્ડર આત્મવિશ્વાસ
\end{itemize}

\textbf{રિસ્ક મોનિટરિંગ પ્રવૃત્તિઓ:}

\begin{itemize}
\tightlist
\item
  \textbf{નિયમિત રિસ્ક રિવ્યુઝ} અને અપડેટ્સ
\item
  \textbf{પ્રારંભિક ચેતવણી} માટે રિસ્ક ટ્રિગર મોનિટરિંગ
\item
  \textbf{મિટિગેશન પ્લાન} પ્રગતિ ટ્રેકિંગ
\item
  \textbf{પ્રોજેક્ટ વિકસિત થાય} તેમ નવા રિસ્કની ઓળખ
\end{itemize}

\textbf{રિસ્ક મેનેજમેન્ટ માટે ટૂલ્સ:}

\begin{itemize}
\tightlist
\item
  \textbf{રિસ્ક રજિસ્ટર્સ} અને ડેટાબેસીસ
\item
  \textbf{રિસ્ક એસેસમેન્ટ} મેટ્રિસીસ
\item
  \textbf{ક્વોન્ટિટેટિવ એનાલિસિસ} માટે મોન્ટે કાર્લો સિમ્યુલેશન
\item
  \textbf{એક્સપર્ટ જજમેન્ટ} અને ઐતિહાસિક ડેટા
\end{itemize}

\textbf{મુખ્ય સફળતા પરિબળો:}

\begin{itemize}
\tightlist
\item
  \textbf{રિસ્ક પ્રક્રિયાઓ} માટે મેનેજમેન્ટ કમિટમેન્ટ
\item
  \textbf{ટીમ જાગૃતિ} અને સહભાગિતા
\item
  \textbf{રિસ્ક વિશે} નિયમિત કોમ્યુનિકેશન
\item
  \textbf{પ્રોજેક્ટ મેનેજમેન્ટ પ્રક્રિયાઓ} સાથે ઇન્ટિગ્રેશન
\end{itemize}

\end{solutionbox}
\begin{mnemonicbox}
``IATMM'' - Identify, Analyze, Assess, Treat,
Monitor risks

\end{mnemonicbox}
\begin{center}\rule{0.5\linewidth}{0.5pt}\end{center}

\subsection*{પ્રશ્ન 5(અ) OR [3
ગુણ]}\label{uxaaauxab0uxab6uxaa8-5uxa85-or-3-uxa97uxaa3}

\textbf{હોસ્ટેલ મેનેજમેન્ટ માટેની કોઈપણ ફંક્શનલ રિક્વાયરમેન્ટ નેલિસ્ટ આઉટ કરો.}

\begin{solutionbox}

\textbf{હોસ્ટેલ મેનેજમેન્ટ સિસ્ટમ} માટે \textbf{ફંક્શનલ રિક્વાયરમેન્ટ્સ} હોસ્ટેલ ઓપરેશન્સનું
અસરકારક સંચાલન કરવા માટે સિસ્ટમે શું કરવું જોઈએ તે વ્યાખ્યાયિત કરે છે.

{\def\LTcaptype{none} % do not increment counter
\begin{longtable}[]{@{}ll@{}}
\toprule\noalign{}
મોડ્યુલ & ફંક્શનલ રિક્વાયરમેન્ટ્સ \\
\midrule\noalign{}
\endhead
\bottomrule\noalign{}
\endlastfoot
\textbf{સ્ટુડન્ટ મેનેજમેન્ટ} & વિદ્યાર્થીઓની નોંધણી, રૂમ એસાઇનમેન્ટ, પ્રોફાઇલ
જાળવણી \\
\textbf{રૂમ મેનેજમેન્ટ} & રૂમ એલોકેશન, ઉપલબ્ધતા ટ્રેકિંગ, મેન્ટેનન્સ \\
\textbf{ફી મેનેજમેન્ટ} & ફી ગણતરી, પેમેન્ટ પ્રોસેસિંગ, રસીદ જનરેશન \\
\textbf{વિઝિટર મેનેજમેન્ટ} & વિઝિટર રજિસ્ટ્રેશન, એન્ટ્રી/એક્ઝિટ ટ્રેકિંગ, મંજૂરી \\
\end{longtable}
}

\textbf{વિગતવાર ફંક્શનલ રિક્વાયરમેન્ટ્સ:}

\textbf{સ્ટુડન્ટ મોડ્યુલ:}

\begin{itemize}
\tightlist
\item
  \textbf{વિદ્યાર્થી નોંધણી} વ્યક્તિગત વિગતો સાથે
\item
  \textbf{રૂમ એસાઇનમેન્ટ} ઉપલબ્ધતાના આધારે
\item
  \textbf{વિદ્યાર્થી પ્રોફાઇલ} મેનેજમેન્ટ અને અપડેટ્સ
\end{itemize}

\textbf{એડમિનિસ્ટ્રેટિવ મોડ્યુલ:}

\begin{itemize}
\tightlist
\item
  \textbf{સ્ટાફ મેનેજમેન્ટ} અને ભૂમિકા સોંપણી
\item
  \textbf{કબજો અને ફાઇનાન્સ} માટે રિપોર્ટ જનરેશન
\item
  \textbf{ફરિયાદ મેનેજમેન્ટ} અને રિઝોલ્યુશન ટ્રેકિંગ
\end{itemize}

\textbf{સિક્યોરિટી મોડ્યુલ:}

\begin{itemize}
\tightlist
\item
  \textbf{વિવિધ યુઝર પ્રકારો} માટે એક્સેસ કંટ્રોલ
\item
  \textbf{વિઝિટર લોગિંગ} અને મંજૂરી સિસ્ટમ
\item
  \textbf{ઇમર્જન્સી કોન્ટેક્ટ} મેનેજમેન્ટ
\end{itemize}

\end{solutionbox}
\begin{mnemonicbox}
``SRFV'' - Student, Room, Fee, Visitor management

\end{mnemonicbox}
\begin{center}\rule{0.5\linewidth}{0.5pt}\end{center}

\subsection*{પ્રશ્ન 5(બ) OR [4
ગુણ]}\label{uxaaauxab0uxab6uxaa8-5uxaac-or-4-uxa97uxaa3}

\textbf{એજાઇલ પ્રોસેસ સમજાવો.}

\begin{solutionbox}

\textbf{એજાઇલ પ્રોસેસ} એ પુનરાવર્તક અને વધારાશીલ સોફ્ટવેર ડેવલપમેન્ટ અભિગમ છે જે
સહયોગ, લવચીકતા અને ગ્રાહક સંતોષ પર ભાર મૂકે છે.

{\def\LTcaptype{none} % do not increment counter
\begin{longtable}[]{@{}
  >{\raggedright\arraybackslash}p{(\linewidth - 4\tabcolsep) * \real{0.5000}}
  >{\raggedright\arraybackslash}p{(\linewidth - 4\tabcolsep) * \real{0.2333}}
  >{\raggedright\arraybackslash}p{(\linewidth - 4\tabcolsep) * \real{0.2667}}@{}}
\toprule\noalign{}
\begin{minipage}[b]{\linewidth}\raggedright
એજાઇલ સિદ્ધાંત
\end{minipage} & \begin{minipage}[b]{\linewidth}\raggedright
વર્ણન
\end{minipage} & \begin{minipage}[b]{\linewidth}\raggedright
ફાયદો
\end{minipage} \\
\midrule\noalign{}
\endhead
\bottomrule\noalign{}
\endlastfoot
\textbf{ગ્રાહક સહયોગ} & સતત ગ્રાહક સંડોવણી & આવશ્યકતાઓની બહેતર સમજ \\
\textbf{કાર્યકારી સોફ્ટવેર} & વારંવાર કાર્યાત્મક સોફ્ટવેર ડિલિવર કરવું & પ્રારંભિક
વેલ્યુ ડિલિવરી \\
\textbf{બદલાવનો પ્રતિસાદ} & બદલાતી આવશ્યકતાઓને અનુકૂળ થવું & માર્કેટ
રિસ્પોન્સિવનેસ \\
\textbf{વ્યક્તિઓ અને ક્રિયાપ્રતિક્રિયાઓ} & પ્રક્રિયાઓ અને ટૂલ્સ કરતાં લોકો & બહેતર
ટીમ ડાયનેમિક્સ \\
\end{longtable}
}

\textbf{એજાઇલ વેલ્યુઝ:}

\begin{itemize}
\tightlist
\item
  \textbf{વ્યક્તિઓ અને ક્રિયાપ્રતિક્રિયાઓ} પ્રક્રિયાઓ અને ટૂલ્સ કરતાં
\item
  \textbf{કાર્યકારી સોફ્ટવેર} વ્યાપક દસ્તાવેજીકરણ કરતાં
\item
  \textbf{ગ્રાહક સહયોગ} કોન્ટ્રાક્ટ વાટાઘાટો કરતાં
\item
  \textbf{બદલાવનો પ્રતિસાદ} યોજનાને અનુસરવા કરતાં
\end{itemize}

\textbf{એજાઇલ પ્રેક્ટિસીસ:}

\begin{itemize}
\tightlist
\item
  \textbf{ટૂંકા પુનરાવર્તનો} (1-4 અઠવાડિયા)
\item
  \textbf{ટીમ સંકલન} માટે ડેઇલી સ્ટેન્ડઅપ્સ
\item
  \textbf{સ્પ્રિન્ટ પ્લાનિંગ} અને રિવ્યુ મીટિંગ્સ
\item
  \textbf{સતત ઇન્ટિગ્રેશન} અને ટેસ્ટિંગ
\end{itemize}

\textbf{ફાયદાઓ:}

\begin{itemize}
\tightlist
\item
  \textbf{કાર્યકારી સોફ્ટવેરની} ઝડપી ડિલિવરી
\item
  \textbf{સતત ટેસ્ટિંગ} દ્વારા બહેતર ગુણવત્તા
\item
  \textbf{સુધારેલ સ્ટેકહોલ્ડર} સંતોષ
\item
  \textbf{ફેરફારોને હેન્ડલ} કરવાની લવચીકતા
\end{itemize}

\end{solutionbox}
\begin{mnemonicbox}
``CWRI'' - Customer collaboration, Working software,
Responding to change, Individuals

\end{mnemonicbox}
\begin{center}\rule{0.5\linewidth}{0.5pt}\end{center}

\subsection*{પ્રશ્ન 5(ક) OR [7
ગુણ]}\label{uxaaauxab0uxab6uxaa8-5uxa95-or-7-uxa97uxaa3}

\textbf{સોફ્ટવેર એન્જિનીયરિંગ - લેયર્ડ એપ્રોચ સમજાવો.}

\begin{solutionbox}

\textbf{સોફ્ટવેર એન્જિનીયરિંગ - લેયર્ડ એપ્રોચ} સોફ્ટવેર એન્જિનીયરિંગને બહુવિધ પરસ્પર
જોડાયેલા સ્તરો સાથે સંરચિત પદ્ધતિ તરીકે રજૂ કરે છે, દરેક નીચલા સ્તરોના પાયા પર બનાવે
છે.

\textbf{લેયર્ડ આર્કિટેક્ચર:}

\begin{center}
\textbf{Mermaid Diagram (Code)}
\begin{verbatim}
{Shaded}
{Highlighting}[]
graph LR
    A[ગુણવત્તા ફોકસ] {-{-}{} B[પ્રોસેસ લેયર]}
    B {-{-}{} C[મેથડ્સ લેયર]}
    C {-{-}{} D[ટૂલ્સ લેયર]}
    
    style A fill:\#4caf50
    style B fill:\#2196f3
    style C fill:\#ff9800
    style D fill:\#9c27b0
{Highlighting}
{Shaded}
\end{verbatim}
\end{center}

{\def\LTcaptype{none} % do not increment counter
\begin{longtable}[]{@{}
  >{\raggedright\arraybackslash}p{(\linewidth - 6\tabcolsep) * \real{0.2222}}
  >{\raggedright\arraybackslash}p{(\linewidth - 6\tabcolsep) * \real{0.2593}}
  >{\raggedright\arraybackslash}p{(\linewidth - 6\tabcolsep) * \real{0.1852}}
  >{\raggedright\arraybackslash}p{(\linewidth - 6\tabcolsep) * \real{0.3333}}@{}}
\toprule\noalign{}
\begin{minipage}[b]{\linewidth}\raggedright
લેયર
\end{minipage} & \begin{minipage}[b]{\linewidth}\raggedright
વર્ણન
\end{minipage} & \begin{minipage}[b]{\linewidth}\raggedright
હેતુ
\end{minipage} & \begin{minipage}[b]{\linewidth}\raggedright
ઉદાહરણો
\end{minipage} \\
\midrule\noalign{}
\endhead
\bottomrule\noalign{}
\endlastfoot
\textbf{ગુણવત્તા ફોકસ} & ગુણવત્તા પર ભાર મૂકતો પાયો & ગ્રાહક સંતોષ સુનિશ્ચિત કરે છે
& ગુણવત્તા ધોરણો, મેટ્રિક્સ \\
\textbf{પ્રોસેસ} & સોફ્ટવેર ડેવલપમેન્ટ માટે ફ્રેમવર્ક & માળખું અને નિયંત્રણ પ્રદાન કરે છે &
SDLC મોડલ્સ, પ્રોજેક્ટ મેનેજમેન્ટ \\
\textbf{મેથડ્સ} & તકનીકી અભિગમો અને તકનીકો & ડેવલપમેન્ટ પ્રવૃત્તિઓને માર્ગદર્શન આપે છે
& વિશ્લેષણ, ડિઝાઇન, ટેસ્ટિંગ પદ્ધતિઓ \\
\textbf{ટૂલ્સ} & પદ્ધતિઓ માટે ઓટોમેટેડ સપોર્ટ & ઉત્પાદકતા વધારે છે & IDE, ટેસ્ટિંગ
ટૂલ્સ, CASE ટૂલ્સ \\
\end{longtable}
}

\textbf{વિગતવાર લેયર વિશ્લેષણ:}

\textbf{ગુણવત્તા ફોકસ (પાયાનો સ્તર):}

\begin{itemize}
\tightlist
\item
  \textbf{સોફ્ટવેર એન્જિનીયરિંગ અભિગમનો} પાયો
\item
  \textbf{તમામ પ્રવૃત્તિઓમાં} ગુણવત્તા પ્રત્યે પ્રતિબદ્ધતા
\item
  \textbf{પ્રાથમિક લક્ષ્ય} તરીકે ગ્રાહક સંતોષ
\item
  \textbf{સતત સુધારણા} માનસિકતા
\item
  \textbf{ગુણવત્તા લક્ષણો}: યોગ્યતા, વિશ્વસનીયતા, કાર્યક્ષમતા, જાળવણીયોગ્યતા
\end{itemize}

\textbf{પ્રોસેસ લેયર:}

\begin{itemize}
\tightlist
\item
  \textbf{અસરકારક ડિલિવરી} માટે ફ્રેમવર્ક વ્યાખ્યાયિત કરે છે
\item
  \textbf{તકનીકી પદ્ધતિઓ} માટે સંદર્ભ સ્થાપિત કરે છે
\item
  \textbf{મુખ્ય એલિમેન્ટ્સ}: કોમ્યુનિકેશન, આયોજન, મોડેલિંગ, બાંધકામ, જમાવટ
\item
  \textbf{પ્રોસેસ મોડલ્સ}: વોટરફોલ, એજાઇલ, સ્પાઇરલ, ઇન્ક્રિમેન્ટલ
\item
  \textbf{મેનેજમેન્ટ પ્રવૃત્તિઓ}: પ્રોજેક્ટ પ્લાનિંગ, ટ્રેકિંગ, રિસ્ક મેનેજમેન્ટ
\end{itemize}

\textbf{મેથડ્સ લેયર:}

\begin{itemize}
\tightlist
\item
  \textbf{સોફ્ટવેર બનાવવા} માટે તકનીકી જ્ઞાન
\item
  \textbf{વ્યાપક કાર્યોની} શ્રેણીને સમાવે છે
\item
  \textbf{કોમ્યુનિકેશન પદ્ધતિઓ}: આવશ્યકતા એલિસિટેશન, વિશ્લેષણ
\item
  \textbf{પ્લાનિંગ પદ્ધતિઓ}: પ્રોજેક્ટ એસ્ટિમેશન, શેડ્યુલિંગ
\item
  \textbf{મોડેલિંગ પદ્ધતિઓ}: વિશ્લેષણ અને ડિઝાઇન તકનીકો
\item
  \textbf{કન્સ્ટ્રક્શન પદ્ધતિઓ}: કોડિંગ ધોરણો, ટેસ્ટિંગ વ્યૂહરચનાઓ
\item
  \textbf{ડિપ્લોયમેન્ટ પદ્ધતિઓ}: ડિલિવરી, સપોર્ટ, પ્રતિસાદ
\end{itemize}

\textbf{ટૂલ્સ લેયર:}

\begin{itemize}
\tightlist
\item
  \textbf{ઓટોમેટેડ અથવા સેમી-ઓટોમેટેડ} સપોર્ટ
\item
  \textbf{કાર્યક્ષમતા વધારે છે} અને એરર્સ ઘટાડે છે
\item
  \textbf{ટૂલ કેટેગરીઝ}:

  \begin{itemize}
  \tightlist
  \item
    \textbf{ડેવલપમેન્ટ એનવાયરનમેન્ટ્સ}: IDE, કમ્પાઇલર્સ
  \item
    \textbf{વિશ્લેષણ અને ડિઝાઇન ટૂલ્સ}: UML ટૂલ્સ, CASE ટૂલ્સ
  \item
    \textbf{ટેસ્ટિંગ ટૂલ્સ}: યુનિટ ટેસ્ટિંગ, ઓટોમેશન ફ્રેમવર્ક્સ
  \item
    \textbf{પ્રોજેક્ટ મેનેજમેન્ટ ટૂલ્સ}: શેડ્યુલિંગ, ટ્રેકિંગ સોફ્ટવેર
  \end{itemize}
\end{itemize}

\textbf{લેયર્સ વચ્ચે ક્રિયાપ્રતિક્રિયાઓ:}

\textbf{ગુણવત્તા \leftrightarrow પ્રોસેસ:}

\begin{itemize}
\tightlist
\item
  ગુણવત્તા ફોકસ \textbf{પ્રોસેસ પસંદગીને} ચલાવે છે
\item
  પ્રોસેસ \textbf{ગુણવત્તા ડિલિવરી} સુનિશ્ચિત કરે છે
\end{itemize}

\textbf{પ્રોસેસ \leftrightarrow મેથડ્સ:}

\begin{itemize}
\tightlist
\item
  પ્રોસેસ મેથડ્સ માટે \textbf{સંદર્ભ પ્રદાન} કરે છે
\item
  મેથડ્સ \textbf{પ્રોસેસ પ્રવૃત્તિઓ} અમલમાં મૂકે છે
\end{itemize}

\textbf{મેથડ્સ \leftrightarrow ટૂલ્સ:}

\begin{itemize}
\tightlist
\item
  મેથડ્સ \textbf{શું કરવું} તે વ્યાખ્યાયિત કરે છે
\item
  ટૂલ્સ \textbf{કેવી રીતે કરવું} કાર્યક્ષમતાથી તે પ્રદાન કરે છે
\end{itemize}

\textbf{લેયર્ડ અભિગમના ફાયદાઓ:}

\begin{itemize}
\tightlist
\item
  \textbf{સોફ્ટવેર ડેવલપમેન્ટ} માટે વ્યવસ્થિત પદ્ધતિ
\item
  \textbf{નાનાથી મોટા પ્રોજેક્ટ્સ} માટે સ્કેલેબિલિટી
\item
  \textbf{ટૂલ્સ અને પદ્ધતિઓને} અનુકૂળ કરવાની લવચીકતા
\item
  \textbf{દરેક સ્તરે} ગુણવત્તા ખાતરી
\item
  \textbf{સંરચિત અભિગમ} દ્વારા જોખમ ઘટાડો
\end{itemize}

\textbf{અમલીકરણ વ્યૂહરચના:}

\begin{itemize}
\tightlist
\item
  \textbf{ગુણવત્તા ફોકસ} સ્થાપના સાથે શરૂઆત
\item
  \textbf{પ્રોજેક્ટ સંદર્ભ} માટે યોગ્ય પ્રોસેસ પસંદ કરવી
\item
  \textbf{પ્રોસેસ આવશ્યકતાઓને} મેચ કરતી પદ્ધતિઓ પસંદ કરવી
\item
  \textbf{પસંદ કરેલી પદ્ધતિઓને} સપોર્ટ કરતા ટૂલ્સનું ઇન્ટિગ્રેશન
\item
  \textbf{સતત મૂલ્યાંકન} અને સુધારણા
\end{itemize}

\textbf{મુખ્ય સફળતા પરિબળો:}

\begin{itemize}
\tightlist
\item
  \textbf{ગુણવત્તા પ્રત્યે} મેનેજમેન્ટ કમિટમેન્ટ
\item
  \textbf{પદ્ધતિઓ અને ટૂલ્સ} પર ટીમ ટ્રેનિંગ
\item
  \textbf{પ્રોસેસ પાલન} અને શિસ્ત
\item
  \textbf{ટૂલ ઇન્ટિગ્રેશન} અને માનકીકરણ
\item
  \textbf{સતત સુધારણા} સંસ્કૃતિ
\end{itemize}

\textbf{વાસ્તવિક વિશ્વ એપ્લિકેશન:}

\begin{itemize}
\tightlist
\item
  \textbf{મોટી સંસ્થાઓ}: સંપૂર્ણ લેયર અમલીકરણ
\item
  \textbf{નાની ટીમો}: સરળીકૃત પરંતુ સુસંગત અભિગમ
\item
  \textbf{પ્રોજેક્ટ-સ્પેસિફિક}: તૈયાર કરેલ લેયર પસંદગી
\item
  \textbf{ઇન્ડસ્ટ્રી સ્ટાન્ડર્ડ્સ}: ગુણવત્તા ફ્રેમવર્ક્સ સાથે અનુપાલન
\end{itemize}

\end{solutionbox}
\begin{mnemonicbox}
``QPMT'' - Quality focus, Process, Methods, Tools
(નીચેથી ઉપર)

\end{mnemonicbox}

\end{document}
