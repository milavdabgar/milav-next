\documentclass[10pt,a4paper]{article}

% content/resources/templates/preamble.tex
\usepackage[margin=0.6in]{geometry}
\author{Milav Dabgar}
\usepackage{amsmath,amssymb,amsthm}
\usepackage{booktabs}
\usepackage{multirow}
\usepackage{xcolor}
\usepackage{tcolorbox}
\tcbuselibrary{breakable,skins}
\usepackage[colorlinks=true,linkcolor=blue]{hyperref}
\usepackage{titlesec}
\usepackage{enumitem}
\usepackage{tikz}
\usepackage{pgfplots}
\usepackage{circuitikz}
\usepackage[version=4]{mhchem}
\usepackage{longtable}
\usepackage{array}
\usepackage{float}
\usepackage{caption}
\usepackage{listings}

\lstset{
  basicstyle=\small\ttfamily,
  breaklines=true,
  breakatwhitespace=false,
  postbreak=\mbox{\textcolor{red}{$\hookrightarrow$}\space},
  float=false,
  numbers=left,
  numberstyle=\tiny\color{gray},
  numbersep=10pt,
  xleftmargin=2em,
  keywordstyle=\color{blue},
  commentstyle=\color{green!60!black},
  stringstyle=\color{purple},
  backgroundcolor=\color{gray!5},
  showstringspaces=false,
  tabsize=2,
  captionpos=b,
  keepspaces=true,
  columns=flexible
}

\pgfplotsset{compat=1.18}
\usetikzlibrary{shapes,arrows,positioning,calc,patterns,decorations.pathmorphing,decorations.markings,arrows.meta}

% Color scheme
\definecolor{headcolor}{RGB}{0,102,204}
\definecolor{keycolor}{RGB}{220,20,60}
\definecolor{solutioncolor}{RGB}{34,139,34}
\definecolor{mnemoniccolor}{RGB}{148,0,211}
\definecolor{codecolor}{RGB}{0,0,100}

% Spacing
\setlength{\parskip}{3pt}
\setlist[itemize]{nosep}
\setlist[enumerate]{nosep}

% Title formatting
\titleformat{\section}{\Large\bfseries\color{headcolor}}{\thesection}{1em}{}
\titleformat{\subsection}{\large\bfseries\color{headcolor}}{\thesubsection}{1em}{}

% Pandoc tightlist compatibility
\providecommand{\tightlist}{%
  \setlength{\itemsep}{0pt}\setlength{\parskip}{0pt}}

% Pandoc longtable compatibility
\newcounter{none}
\def\thenone{}


% content/resources/templates/english-boxes.tex
% This file is currently empty - it exists to maintain consistency with the import structure.
% Add custom environments here if needed in the future.


\begin{document}

\begin{center}
{\Huge\bfseries\color{headcolor} Subject Name Solutions}\\[5pt]
{\LARGE 4331604 -- Winter 2024}\\[3pt]
{\large Semester 1 Study Material}\\[3pt]
{\normalsize\textit{Detailed Solutions and Explanations}}
\end{center}

\vspace{10pt}

\subsection*{Question 1(a) [3 marks]}\label{q1a}

\textbf{What is Scrum model? Write about it.}

\begin{solutionbox}

Scrum is an \textbf{agile framework} for managing software development
projects through iterative and incremental practices.

{\def\LTcaptype{none} % do not increment counter
\begin{longtable}[]{@{}
  >{\raggedright\arraybackslash}p{(\linewidth - 2\tabcolsep) * \real{0.3810}}
  >{\raggedright\arraybackslash}p{(\linewidth - 2\tabcolsep) * \real{0.6190}}@{}}
\toprule\noalign{}
\begin{minipage}[b]{\linewidth}\raggedright
Aspect
\end{minipage} & \begin{minipage}[b]{\linewidth}\raggedright
Description
\end{minipage} \\
\midrule\noalign{}
\endhead
\bottomrule\noalign{}
\endlastfoot
\textbf{Framework Type} & Agile methodology \\
\textbf{Sprint Duration} & 2-4 weeks typically \\
\textbf{Team Size} & 5-9 members \\
\textbf{Key Ceremonies} & Daily standups, Sprint planning, Sprint
review, Retrospective \\
\end{longtable}
}

\textbf{Key Features:}

\begin{itemize}
\tightlist
\item
  \textbf{Product Owner}: Defines requirements and priorities
\item
  \textbf{Scrum Master}: Facilitates process and removes obstacles\\
\item
  \textbf{Development Team}: Cross-functional team building the product
\end{itemize}

\end{solutionbox}
\begin{mnemonicbox}
``SPIR'' - Sprint, Product owner, Incremental
delivery, Review

\end{mnemonicbox}
\begin{center}\rule{0.5\linewidth}{0.5pt}\end{center}

\subsection*{Question 1(b) [4 marks]}\label{q1b}

\textbf{Define Software and Explain Software Characteristics.}

\begin{solutionbox}

\textbf{Software Definition}: A collection of computer programs,
procedures, and documentation that performs tasks on a computer system.

{\def\LTcaptype{none} % do not increment counter
\begin{longtable}[]{@{}ll@{}}
\toprule\noalign{}
Characteristic & Description \\
\midrule\noalign{}
\endhead
\bottomrule\noalign{}
\endlastfoot
\textbf{Intangible} & Cannot be touched physically \\
\textbf{No Physical Wear} & Doesn't deteriorate with time \\
\textbf{Custom Built} & Developed for specific requirements \\
\textbf{Expensive} & High development and maintenance costs \\
\end{longtable}
}

\textbf{Key Points:}

\begin{itemize}
\tightlist
\item
  \textbf{Logical Product}: Made of instructions and data
\item
  \textbf{Engineered}: Follows systematic development process
\item
  \textbf{Complex}: Handles multiple interconnected functions
\item
  \textbf{Maintainable}: Can be modified and updated
\end{itemize}

\end{solutionbox}
\begin{mnemonicbox}
``IELM'' - Intangible, Engineered, Logical,
Maintainable

\end{mnemonicbox}
\begin{center}\rule{0.5\linewidth}{0.5pt}\end{center}

\subsection*{Question 1(c) [7 marks]}\label{q1c}

\textbf{Explain Waterfall Model with diagram.}

\begin{solutionbox}

The \textbf{Waterfall Model} is a linear sequential software development
approach where each phase must be completed before the next begins.

\begin{verbatim}
flowchart LR
    A[Requirements Analysis] {-{-} B[System Design]}
    B {-{-} C[Implementation]}
    C {-{-} D[Testing]}
    D {-{-} E[Deployment]}
    E {-{-} F[Maintenance]}
    
    style A fill:\#e1f5fe
    style B fill:\#f3e5f5
    style C fill:\#e8f5e8
    style D fill:\#fff3e0
    style E fill:\#fce4ec
    style F fill:\#f1f8e9
\end{verbatim}

{\def\LTcaptype{none} % do not increment counter
\begin{longtable}[]{@{}lll@{}}
\toprule\noalign{}
Phase & Activities & Output \\
\midrule\noalign{}
\endhead
\bottomrule\noalign{}
\endlastfoot
\textbf{Requirements} & Gather and document needs & SRS Document \\
\textbf{Design} & System architecture planning & Design specs \\
\textbf{Implementation} & Actual coding & Source code \\
\textbf{Testing} & Verification and validation & Test reports \\
\textbf{Deployment} & Installation at client site & Working system \\
\textbf{Maintenance} & Bug fixes and updates & Updated system \\
\end{longtable}
}

\textbf{Advantages:}

\begin{itemize}
\tightlist
\item
  \textbf{Simple to understand} and implement
\item
  \textbf{Well-documented} phases
\item
  \textbf{Easy project management} with clear milestones
\end{itemize}

\textbf{Disadvantages:}

\begin{itemize}
\tightlist
\item
  \textbf{No flexibility} for requirement changes
\item
  \textbf{Late testing} discovery of issues
\item
  \textbf{Not suitable} for complex projects
\end{itemize}

\end{solutionbox}
\begin{mnemonicbox}
``RSITDM'' - Requirements, System design,
Implementation, Testing, Deployment, Maintenance

\end{mnemonicbox}
\begin{center}\rule{0.5\linewidth}{0.5pt}\end{center}

\subsection*{Question 1(c) OR [7
marks]}\label{q1c}

\textbf{Explain Spiral Model with diagram.}

\begin{solutionbox}

The \textbf{Spiral Model} combines iterative development with systematic
risk assessment, emphasizing risk analysis in each iteration.

\begin{center}
\textbf{Mermaid Diagram (Code)}
\begin{verbatim}
{Shaded}
{Highlighting}[]
graph TD
    A[Planning] {-{-}{} B[Risk Analysis]}
    B {-{-}{} C[Engineering]}
    C {-{-}{} D[Customer Evaluation]}
    D {-{-}{} A}
    
    E[Risk Assessment] {-.{-}{} B}
    F[Prototype Development] {-.{-}{} C}
    G[Customer Feedback] {-.{-}{} D}
    
    style A fill:\#e3f2fd
    style B fill:\#ffebee
    style C fill:\#e8f5e8
    style D fill:\#fff8e1
{Highlighting}
{Shaded}
\end{verbatim}
\end{center}

{\def\LTcaptype{none} % do not increment counter
\begin{longtable}[]{@{}
  >{\raggedright\arraybackslash}p{(\linewidth - 4\tabcolsep) * \real{0.3448}}
  >{\raggedright\arraybackslash}p{(\linewidth - 4\tabcolsep) * \real{0.3448}}
  >{\raggedright\arraybackslash}p{(\linewidth - 4\tabcolsep) * \real{0.3103}}@{}}
\toprule\noalign{}
\begin{minipage}[b]{\linewidth}\raggedright
Quadrant
\end{minipage} & \begin{minipage}[b]{\linewidth}\raggedright
Activity
\end{minipage} & \begin{minipage}[b]{\linewidth}\raggedright
Purpose
\end{minipage} \\
\midrule\noalign{}
\endhead
\bottomrule\noalign{}
\endlastfoot
\textbf{Planning} & Requirement gathering & Define objectives \\
\textbf{Risk Analysis} & Identify and resolve risks & Minimize
uncertainties \\
\textbf{Engineering} & Development and testing & Build working
software \\
\textbf{Evaluation} & Customer assessment & Get feedback for next
iteration \\
\end{longtable}
}

\textbf{Key Features:}

\begin{itemize}
\tightlist
\item
  \textbf{Risk-driven approach} with early risk identification
\item
  \textbf{Iterative development} with customer involvement
\item
  \textbf{Prototyping} in each spiral
\item
  \textbf{Suitable for large} and complex projects
\end{itemize}

\textbf{Advantages:}

\begin{itemize}
\tightlist
\item
  \textbf{Early risk detection} and mitigation
\item
  \textbf{Customer involvement} throughout development
\item
  \textbf{Flexible} to accommodate changes
\end{itemize}

\textbf{Disadvantages:}

\begin{itemize}
\tightlist
\item
  \textbf{Complex management} due to risk analysis
\item
  \textbf{Expensive} for small projects
\item
  \textbf{Requires expertise} in risk assessment
\end{itemize}

\end{solutionbox}
\begin{mnemonicbox}
``PRICE'' - Planning, Risk analysis, Iterative,
Customer evaluation, Engineering

\end{mnemonicbox}
\begin{center}\rule{0.5\linewidth}{0.5pt}\end{center}

\subsection*{Question 2(a) [3 marks]}\label{q2a}

\textbf{In which situation prototype model is used?}

\begin{solutionbox}

The \textbf{Prototype Model} is used when requirements are unclear or
when demonstrating feasibility is crucial.

{\def\LTcaptype{none} % do not increment counter
\begin{longtable}[]{@{}ll@{}}
\toprule\noalign{}
Situation & Application \\
\midrule\noalign{}
\endhead
\bottomrule\noalign{}
\endlastfoot
\textbf{Unclear Requirements} & When user needs are not well-defined \\
\textbf{New Technology} & Testing feasibility of new tools/platforms \\
\textbf{User Interface} & Designing complex UI/UX systems \\
\textbf{High Risk Projects} & Reducing uncertainties early \\
\end{longtable}
}

\textbf{Specific Use Cases:}

\begin{itemize}
\tightlist
\item
  \textbf{Web applications} with complex user interactions
\item
  \textbf{Real-time systems} requiring performance validation
\item
  \textbf{AI/ML projects} with experimental algorithms
\end{itemize}

\end{solutionbox}
\begin{mnemonicbox}
``UNIT'' - Unclear requirements, New technology,
Interface design, Testing feasibility

\end{mnemonicbox}
\begin{center}\rule{0.5\linewidth}{0.5pt}\end{center}

\subsection*{Question 2(b) [4 marks]}\label{q2b}

\textbf{Explain requirement gathering in detail.}

\begin{solutionbox}

\textbf{Requirement Gathering} is the process of collecting, analyzing,
and documenting software requirements from stakeholders.

{\def\LTcaptype{none} % do not increment counter
\begin{longtable}[]{@{}
  >{\raggedright\arraybackslash}p{(\linewidth - 4\tabcolsep) * \real{0.2973}}
  >{\raggedright\arraybackslash}p{(\linewidth - 4\tabcolsep) * \real{0.3514}}
  >{\raggedright\arraybackslash}p{(\linewidth - 4\tabcolsep) * \real{0.3514}}@{}}
\toprule\noalign{}
\begin{minipage}[b]{\linewidth}\raggedright
Technique
\end{minipage} & \begin{minipage}[b]{\linewidth}\raggedright
Description
\end{minipage} & \begin{minipage}[b]{\linewidth}\raggedright
When to Use
\end{minipage} \\
\midrule\noalign{}
\endhead
\bottomrule\noalign{}
\endlastfoot
\textbf{Interviews} & One-on-one discussions & Detailed requirements \\
\textbf{Questionnaires} & Structured surveys & Large user groups \\
\textbf{Observation} & Watching current processes & Understanding
workflows \\
\textbf{Workshops} & Group sessions & Collaborative requirements \\
\end{longtable}
}

\textbf{Process Steps:}

\begin{itemize}
\tightlist
\item
  \textbf{Stakeholder Identification}: Find all relevant parties
\item
  \textbf{Information Collection}: Use various gathering techniques
\item
  \textbf{Analysis}: Prioritize and categorize requirements
\item
  \textbf{Documentation}: Create formal requirement specifications
\end{itemize}

\textbf{Challenges:}

\begin{itemize}
\tightlist
\item
  \textbf{Changing requirements} during development
\item
  \textbf{Communication gaps} between stakeholders
\item
  \textbf{Incomplete information} from users
\end{itemize}

\end{solutionbox}
\begin{mnemonicbox}
``IQOW'' - Interviews, Questionnaires, Observation,
Workshops

\end{mnemonicbox}
\begin{center}\rule{0.5\linewidth}{0.5pt}\end{center}

\subsection*{Question 2(c) [7 marks]}\label{q2c}

\textbf{Discuss the responsibilities of software project manager.}

\begin{solutionbox}

A \textbf{Software Project Manager} oversees the entire software
development lifecycle ensuring successful project delivery.

{\def\LTcaptype{none} % do not increment counter
\begin{longtable}[]{@{}
  >{\raggedright\arraybackslash}p{(\linewidth - 4\tabcolsep) * \real{0.4043}}
  >{\raggedright\arraybackslash}p{(\linewidth - 4\tabcolsep) * \real{0.2340}}
  >{\raggedright\arraybackslash}p{(\linewidth - 4\tabcolsep) * \real{0.3617}}@{}}
\toprule\noalign{}
\begin{minipage}[b]{\linewidth}\raggedright
Responsibility Area
\end{minipage} & \begin{minipage}[b]{\linewidth}\raggedright
Key Tasks
\end{minipage} & \begin{minipage}[b]{\linewidth}\raggedright
Skills Required
\end{minipage} \\
\midrule\noalign{}
\endhead
\bottomrule\noalign{}
\endlastfoot
\textbf{Planning} & Project scheduling, resource allocation & Strategic
thinking \\
\textbf{Team Management} & Team coordination, motivation & Leadership \\
\textbf{Risk Management} & Risk identification, mitigation strategies &
Problem-solving \\
\textbf{Communication} & Stakeholder coordination, reporting &
Communication skills \\
\textbf{Quality Assurance} & Process compliance, deliverable quality &
Attention to detail \\
\end{longtable}
}

\textbf{Detailed Responsibilities:}

\textbf{Project Planning:}

\begin{itemize}
\tightlist
\item
  \textbf{Work Breakdown Structure} creation
\item
  \textbf{Timeline estimation} and scheduling
\item
  \textbf{Resource allocation} and budget management
\end{itemize}

\textbf{Team Leadership:}

\begin{itemize}
\tightlist
\item
  \textbf{Team building} and motivation
\item
  \textbf{Conflict resolution} between team members
\item
  \textbf{Performance monitoring} and feedback
\end{itemize}

\textbf{Stakeholder Management:}

\begin{itemize}
\tightlist
\item
  \textbf{Client communication} and expectation management
\item
  \textbf{Progress reporting} to senior management
\item
  \textbf{Change request} handling and approval
\end{itemize}

\textbf{Risk and Quality Management:}

\begin{itemize}
\tightlist
\item
  \textbf{Risk assessment} and contingency planning
\item
  \textbf{Quality standards} enforcement
\item
  \textbf{Process improvement} implementation
\end{itemize}

\textbf{Essential Skills:}

\begin{itemize}
\tightlist
\item
  \textbf{Technical knowledge} of software development
\item
  \textbf{Project management} methodologies (Agile, Waterfall)
\item
  \textbf{Communication skills} for diverse stakeholders
\item
  \textbf{Problem-solving} and decision-making abilities
\end{itemize}

\end{solutionbox}
\begin{mnemonicbox}
``PLACE'' - Planning, Leadership, Assessment,
Communication, Execution

\end{mnemonicbox}
\begin{center}\rule{0.5\linewidth}{0.5pt}\end{center}

\subsection*{Question 2(a) OR [3
marks]}\label{q2a}

\textbf{Difference between GANTT chart and PERT chart.}

\begin{solutionbox}

{\def\LTcaptype{none} % do not increment counter
\begin{longtable}[]{@{}
  >{\raggedright\arraybackslash}p{(\linewidth - 4\tabcolsep) * \real{0.2353}}
  >{\raggedright\arraybackslash}p{(\linewidth - 4\tabcolsep) * \real{0.3824}}
  >{\raggedright\arraybackslash}p{(\linewidth - 4\tabcolsep) * \real{0.3824}}@{}}
\toprule\noalign{}
\begin{minipage}[b]{\linewidth}\raggedright
Aspect
\end{minipage} & \begin{minipage}[b]{\linewidth}\raggedright
GANTT Chart
\end{minipage} & \begin{minipage}[b]{\linewidth}\raggedright
PERT Chart
\end{minipage} \\
\midrule\noalign{}
\endhead
\bottomrule\noalign{}
\endlastfoot
\textbf{Purpose} & Visual timeline of tasks & Network analysis of
dependencies \\
\textbf{Format} & Horizontal bar chart & Network diagram with nodes \\
\textbf{Time Focus} & Shows duration and dates & Shows critical path and
slack time \\
\textbf{Complexity} & Simple to understand & More complex analysis \\
\textbf{Best For} & Project scheduling & Time optimization \\
\end{longtable}
}

\textbf{Key Differences:}

\begin{itemize}
\tightlist
\item
  \textbf{GANTT}: Shows \textbf{when tasks happen}
\item
  \textbf{PERT}: Shows \textbf{task relationships} and critical path
\end{itemize}

\end{solutionbox}
\begin{mnemonicbox}
``GT vs PT'' - Gantt Timeline vs PERT dependencies

\end{mnemonicbox}
\begin{center}\rule{0.5\linewidth}{0.5pt}\end{center}

\subsection*{Question 2(b) OR [4
marks]}\label{q2b}

\textbf{Give the Full Form of: RAD, SDLC, XP model and SRS.}

\begin{solutionbox}

{\def\LTcaptype{none} % do not increment counter
\begin{longtable}[]{@{}
  >{\raggedright\arraybackslash}p{(\linewidth - 4\tabcolsep) * \real{0.2727}}
  >{\raggedright\arraybackslash}p{(\linewidth - 4\tabcolsep) * \real{0.3333}}
  >{\raggedright\arraybackslash}p{(\linewidth - 4\tabcolsep) * \real{0.3939}}@{}}
\toprule\noalign{}
\begin{minipage}[b]{\linewidth}\raggedright
Acronym
\end{minipage} & \begin{minipage}[b]{\linewidth}\raggedright
Full Form
\end{minipage} & \begin{minipage}[b]{\linewidth}\raggedright
Description
\end{minipage} \\
\midrule\noalign{}
\endhead
\bottomrule\noalign{}
\endlastfoot
\textbf{RAD} & Rapid Application Development & Fast prototyping
methodology \\
\textbf{SDLC} & Software Development Life Cycle & Complete development
process \\
\textbf{XP} & Extreme Programming & Agile development methodology \\
\textbf{SRS} & Software Requirement Specification & Formal requirement
document \\
\end{longtable}
}

\textbf{Brief Explanations:}

\begin{itemize}
\tightlist
\item
  \textbf{RAD}: Focuses on \textbf{rapid prototyping} and iterative
  development
\item
  \textbf{SDLC}: \textbf{Systematic approach} to software development
  phases
\item
  \textbf{XP}: \textbf{Agile methodology} emphasizing coding practices
\item
  \textbf{SRS}: \textbf{Detailed documentation} of functional and
  non-functional requirements
\end{itemize}

\end{solutionbox}
\begin{mnemonicbox}
``RSXS'' - RAD, SDLC, XP, SRS

\end{mnemonicbox}
\begin{center}\rule{0.5\linewidth}{0.5pt}\end{center}

\subsection*{Question 2(c) OR [7
marks]}\label{q2c}

\textbf{Explain WBS in Detail.}

\begin{solutionbox}

\textbf{Work Breakdown Structure (WBS)} is a hierarchical decomposition
of project work into smaller, manageable components.

\begin{center}
\textbf{Mermaid Diagram (Code)}
\begin{verbatim}
{Shaded}
{Highlighting}[]
graph TD
    A[Software Project] {-{-}{} B[Analysis Phase]}
    A {-{-}{} C[Design Phase]}
    A {-{-}{} D[Implementation Phase]}
    A {-{-}{} E[Testing Phase]}
    
    B {-{-}{} B1[Requirement Gathering]}
    B {-{-}{} B2[SRS Documentation]}
    
    C {-{-}{} C1[System Design]}
    C {-{-}{} C2[Database Design]}
    C {-{-}{} C3[UI Design]}
    
    D {-{-}{} D1[Module Development]}
    D {-{-}{} D2[Code Review]}
    D {-{-}{} D3[Integration]}
    
    E {-{-}{} E1[Unit Testing]}
    E {-{-}{} E2[System Testing]}
    E {-{-}{} E3[User Acceptance Testing]}
{Highlighting}
{Shaded}
\end{verbatim}
\end{center}

{\def\LTcaptype{none} % do not increment counter
\begin{longtable}[]{@{}
  >{\raggedright\arraybackslash}p{(\linewidth - 4\tabcolsep) * \real{0.3333}}
  >{\raggedright\arraybackslash}p{(\linewidth - 4\tabcolsep) * \real{0.3939}}
  >{\raggedright\arraybackslash}p{(\linewidth - 4\tabcolsep) * \real{0.2727}}@{}}
\toprule\noalign{}
\begin{minipage}[b]{\linewidth}\raggedright
WBS Level
\end{minipage} & \begin{minipage}[b]{\linewidth}\raggedright
Description
\end{minipage} & \begin{minipage}[b]{\linewidth}\raggedright
Example
\end{minipage} \\
\midrule\noalign{}
\endhead
\bottomrule\noalign{}
\endlastfoot
\textbf{Level 1} & Major project phases & Analysis, Design,
Implementation \\
\textbf{Level 2} & Major deliverables & SRS, Design docs, Code
modules \\
\textbf{Level 3} & Work packages & Specific tasks and activities \\
\textbf{Level 4} & Individual activities & Detailed task breakdown \\
\end{longtable}
}

\textbf{Benefits of WBS:}

\begin{itemize}
\tightlist
\item
  \textbf{Clear project scope} definition
\item
  \textbf{Better estimation} of time and resources
\item
  \textbf{Improved task assignment} and accountability
\item
  \textbf{Enhanced progress tracking} and control
\end{itemize}

\textbf{WBS Creation Process:}

\begin{itemize}
\tightlist
\item
  \textbf{Identify major deliverables} from project scope
\item
  \textbf{Decompose deliverables} into smaller components
\item
  \textbf{Continue breakdown} until work packages are manageable
\item
  \textbf{Assign responsibilities} for each work package
\end{itemize}

\textbf{Key Principles:}

\begin{itemize}
\tightlist
\item
  \textbf{100\% Rule}: WBS includes all project work
\item
  \textbf{Mutually Exclusive}: No overlap between components
\item
  \textbf{Manageable Size}: Work packages should be 8-80 hours
\end{itemize}

\end{solutionbox}
\begin{mnemonicbox}
``DEBT'' - Decompose, Estimate, Breakdown, Track

\end{mnemonicbox}
\begin{center}\rule{0.5\linewidth}{0.5pt}\end{center}

\subsection*{Question 3(a) [3 marks]}\label{q3a}

\textbf{Draw the diagram of Incremental Model.}

\begin{solutionbox}

The \textbf{Incremental Model} develops software in increments, with
each increment adding functionality to the previous versions.

\begin{center}
\textbf{Mermaid Diagram (Code)}
\begin{verbatim}
{Shaded}
{Highlighting}[]
graph LR
    A[Requirements Analysis] {-{-}{} B[System Design]}
    B {-{-}{} C1[Increment 1]}
    B {-{-}{} C2[Increment 2]}
    B {-{-}{} C3[Increment 3]}
    
    C1 {-{-}{} D1[Design  Code  Test]}
    C2 {-{-}{} D2[Design  Code  Test]}
    C3 {-{-}{} D3[Design  Code  Test]}
    
    D1 {-{-}{} E1[Release 1]}
    D2 {-{-}{} E2[Release 2]}
    D3 {-{-}{} E3[Release 3]}
    
    E1 {-{-}{} F[Final Product]}
    E2 {-{-}{} F}
    E3 {-{-}{} F}
    
    style A fill:\#e3f2fd
    style B fill:\#f3e5f5
    style C1 fill:\#e8f5e8
    style C2 fill:\#fff3e0
    style C3 fill:\#fce4ec
{Highlighting}
{Shaded}
\end{verbatim}
\end{center}

\textbf{Key Features:}

\begin{itemize}
\tightlist
\item
  \textbf{Core functionality} delivered first
\item
  \textbf{Additional features} added incrementally
\item
  \textbf{Working software} available early
\end{itemize}

\end{solutionbox}
\begin{mnemonicbox}
``IRA'' - Incremental, Release, Add features

\end{mnemonicbox}
\begin{center}\rule{0.5\linewidth}{0.5pt}\end{center}

\subsection*{Question 3(b) [4 marks]}\label{q3b}

\textbf{Difference between functional and non-functional requirements}

\begin{solutionbox}

{\def\LTcaptype{none} % do not increment counter
\begin{longtable}[]{@{}
  >{\raggedright\arraybackslash}p{(\linewidth - 4\tabcolsep) * \real{0.1333}}
  >{\raggedright\arraybackslash}p{(\linewidth - 4\tabcolsep) * \real{0.4000}}
  >{\raggedright\arraybackslash}p{(\linewidth - 4\tabcolsep) * \real{0.4667}}@{}}
\toprule\noalign{}
\begin{minipage}[b]{\linewidth}\raggedright
Aspect
\end{minipage} & \begin{minipage}[b]{\linewidth}\raggedright
Functional Requirements
\end{minipage} & \begin{minipage}[b]{\linewidth}\raggedright
Non-Functional Requirements
\end{minipage} \\
\midrule\noalign{}
\endhead
\bottomrule\noalign{}
\endlastfoot
\textbf{Definition} & What the system should do & How the system should
perform \\
\textbf{Focus} & System behavior and features & System quality
attributes \\
\textbf{Examples} & Login, data processing, reports & Performance,
security, usability \\
\textbf{Testing} & Functional testing & Performance, security testing \\
\textbf{Documentation} & Use cases, user stories & Quality metrics,
constraints \\
\end{longtable}
}

\textbf{Detailed Comparison:}

\textbf{Functional Requirements:}

\begin{itemize}
\tightlist
\item
  \textbf{User authentication} and authorization
\item
  \textbf{Data processing} and calculations
\item
  \textbf{Report generation} and export features
\item
  \textbf{Business logic} implementation
\end{itemize}

\textbf{Non-Functional Requirements:}

\begin{itemize}
\tightlist
\item
  \textbf{Performance}: Response time, throughput
\item
  \textbf{Security}: Data encryption, access control
\item
  \textbf{Usability}: User interface design, accessibility
\item
  \textbf{Reliability}: System availability, fault tolerance
\end{itemize}

\textbf{Examples for Library System:}

\begin{itemize}
\tightlist
\item
  \textbf{Functional}: Book search, issue/return books, fine calculation
\item
  \textbf{Non-Functional}: Search results in \textless2 seconds, 99.9\%
  uptime, SSL encryption
\end{itemize}

\end{solutionbox}
\begin{mnemonicbox}
``FW vs NH'' - Functional What vs Non-functional How

\end{mnemonicbox}
\begin{center}\rule{0.5\linewidth}{0.5pt}\end{center}

\subsection*{Question 3(c) [7 marks]}\label{q3c}

\textbf{Explain DFD with example.}

\begin{solutionbox}

\textbf{Data Flow Diagram (DFD)} is a graphical representation showing
data flow through a system using processes, data stores, external
entities, and data flows.

\textbf{DFD Symbols:}

{\def\LTcaptype{none} % do not increment counter
\begin{longtable}[]{@{}lll@{}}
\toprule\noalign{}
Symbol & Name & Purpose \\
\midrule\noalign{}
\endhead
\bottomrule\noalign{}
\endlastfoot
Circle/Oval & Process & Data transformation \\
Rectangle & External Entity & Data source/destination \\
Open Rectangle & Data Store & Data storage \\
Arrow & Data Flow & Data movement direction \\
\end{longtable}
}

\textbf{Example: Library Management System}

\begin{verbatim}
flowchart TD
    A[Student] {-{-}|Book Request| B((Search Books))}
    B {-{-}|Book Details| A}
    B {-{-}|Query| C[(Book Database)]}
    C {-{-}|Book Info| B}
    
    A {-{-}|Issue Request| D((Issue Book))}
    D {-{-}|Issue Details| E[(Issue Records)]}
    D {-{-}|Confirmation| A}
    
    F[Librarian] {-{-}|Book Return| G((Return Book))}
    G {-{-}|Update Status| C}
    G {-{-}|Update Records| E}
    G {-{-}|Receipt| F}
\end{verbatim}

\textbf{DFD Levels:}

\textbf{Context Diagram (Level 0):}

\begin{itemize}
\tightlist
\item
  \textbf{Single process} representing entire system
\item
  \textbf{External entities} and major data flows
\item
  \textbf{High-level overview} of system boundaries
\end{itemize}

\textbf{Level 1 DFD:}

\begin{itemize}
\tightlist
\item
  \textbf{Major processes} of the system
\item
  \textbf{Data stores} and their interactions
\item
  \textbf{Detailed data flows} between processes
\end{itemize}

\textbf{Level 2 and beyond:}

\begin{itemize}
\tightlist
\item
  \textbf{Decomposition} of complex processes
\item
  \textbf{More detailed} data transformations
\item
  \textbf{Lower-level} process specifications
\end{itemize}

\textbf{DFD Rules:}

\begin{itemize}
\tightlist
\item
  \textbf{Process naming}: Use verb + object (e.g., ``Validate User'')
\item
  \textbf{Data flow naming}: Use noun phrases (e.g., ``User Details'')
\item
  \textbf{Balancing}: Input/output must match between levels
\item
  \textbf{No direct connections} between external entities
\end{itemize}

\textbf{Benefits:}

\begin{itemize}
\tightlist
\item
  \textbf{Clear communication} with stakeholders
\item
  \textbf{System boundary} identification
\item
  \textbf{Process analysis} and optimization
\item
  \textbf{Documentation} for system design
\end{itemize}

\end{solutionbox}
\begin{mnemonicbox}
``PEDS'' - Process, External entity, Data store, Data
flow

\end{mnemonicbox}
\begin{center}\rule{0.5\linewidth}{0.5pt}\end{center}

\subsection*{Question 3(a) OR [3
marks]}\label{q3a}

\textbf{Write classification of design activities.}

\begin{solutionbox}

\textbf{Design Activities} are classified based on their scope and
purpose in software development.

{\def\LTcaptype{none} % do not increment counter
\begin{longtable}[]{@{}
  >{\raggedright\arraybackslash}p{(\linewidth - 4\tabcolsep) * \real{0.4324}}
  >{\raggedright\arraybackslash}p{(\linewidth - 4\tabcolsep) * \real{0.3243}}
  >{\raggedright\arraybackslash}p{(\linewidth - 4\tabcolsep) * \real{0.2432}}@{}}
\toprule\noalign{}
\begin{minipage}[b]{\linewidth}\raggedright
Classification
\end{minipage} & \begin{minipage}[b]{\linewidth}\raggedright
Activities
\end{minipage} & \begin{minipage}[b]{\linewidth}\raggedright
Purpose
\end{minipage} \\
\midrule\noalign{}
\endhead
\bottomrule\noalign{}
\endlastfoot
\textbf{System Design} & Architecture, modules, interfaces & High-level
structure \\
\textbf{Detailed Design} & Algorithms, data structures & Implementation
details \\
\textbf{Interface Design} & UI/UX, API specifications & User
interaction \\
\textbf{Database Design} & Schema, relationships, optimization & Data
management \\
\end{longtable}
}

\textbf{Key Design Activities:}

\begin{itemize}
\tightlist
\item
  \textbf{Architectural Design}: Overall system structure
\item
  \textbf{Component Design}: Individual module specifications
\item
  \textbf{Data Design}: Database and file structures
\end{itemize}

\end{solutionbox}
\begin{mnemonicbox}
``ACID'' - Architectural, Component, Interface, Data
design

\end{mnemonicbox}
\begin{center}\rule{0.5\linewidth}{0.5pt}\end{center}

\subsection*{Question 3(b) OR [4
marks]}\label{q3b}

\textbf{Explain characteristics of good SRS.}

\begin{solutionbox}

A \textbf{good SRS (Software Requirement Specification)} document should
possess specific characteristics for effective communication and
development.

{\def\LTcaptype{none} % do not increment counter
\begin{longtable}[]{@{}lll@{}}
\toprule\noalign{}
Characteristic & Description & Benefit \\
\midrule\noalign{}
\endhead
\bottomrule\noalign{}
\endlastfoot
\textbf{Complete} & All requirements included & No missing
functionality \\
\textbf{Consistent} & No contradictory requirements & Clear
understanding \\
\textbf{Unambiguous} & Single interpretation possible & Reduced
confusion \\
\textbf{Verifiable} & Requirements can be tested & Quality assurance \\
\textbf{Modifiable} & Easy to update and maintain & Adaptability \\
\textbf{Traceable} & Requirements can be tracked & Change management \\
\end{longtable}
}

\textbf{Detailed Characteristics:}

\textbf{Completeness:}

\begin{itemize}
\tightlist
\item
  \textbf{All functional} requirements specified
\item
  \textbf{All non-functional} requirements included
\item
  \textbf{All interfaces} and constraints documented
\end{itemize}

\textbf{Consistency:}

\begin{itemize}
\tightlist
\item
  \textbf{No conflicting} requirements
\item
  \textbf{Uniform terminology} throughout document
\item
  \textbf{Consistent formatting} and structure
\end{itemize}

\textbf{Verifiability:}

\begin{itemize}
\tightlist
\item
  \textbf{Testable requirements} with clear criteria
\item
  \textbf{Measurable quality} attributes
\item
  \textbf{Objective success} criteria defined
\end{itemize}

\end{solutionbox}
\begin{mnemonicbox}
``CCUMVT'' - Complete, Consistent, Unambiguous,
Modifiable, Verifiable, Traceable

\end{mnemonicbox}
\begin{center}\rule{0.5\linewidth}{0.5pt}\end{center}

\subsection*{Question 3(c) OR [7
marks]}\label{q3c}

\textbf{Explain White box Testing.}

\begin{solutionbox}

\textbf{White Box Testing} is a testing method that examines the
internal structure, code, and logic of software applications.

{\def\LTcaptype{none} % do not increment counter
\begin{longtable}[]{@{}
  >{\raggedright\arraybackslash}p{(\linewidth - 2\tabcolsep) * \real{0.3810}}
  >{\raggedright\arraybackslash}p{(\linewidth - 2\tabcolsep) * \real{0.6190}}@{}}
\toprule\noalign{}
\begin{minipage}[b]{\linewidth}\raggedright
Aspect
\end{minipage} & \begin{minipage}[b]{\linewidth}\raggedright
Description
\end{minipage} \\
\midrule\noalign{}
\endhead
\bottomrule\noalign{}
\endlastfoot
\textbf{Also Known As} & Structural testing, Glass box testing, Clear
box testing \\
\textbf{Access Level} & Full access to source code and internal
structure \\
\textbf{Focus} & Code coverage, logic paths, internal data structures \\
\textbf{Tester Knowledge} & Programming knowledge required \\
\end{longtable}
}

\textbf{White Box Testing Techniques:}

\begin{center}
\textbf{Mermaid Diagram (Code)}
\begin{verbatim}
{Shaded}
{Highlighting}[]
graph TD
    A[White Box Testing] {-{-}{} B[Statement Coverage]}
    A {-{-}{} C[Branch Coverage]}
    A {-{-}{} D[Path Coverage]}
    A {-{-}{} E[Condition Coverage]}
    
    B {-{-}{} B1[Execute every statement]}
    C {-{-}{} C1[Test all decision points]}
    D {-{-}{} D1[Test all possible paths]}
    E {-{-}{} E1[Test all logical conditions]}
{Highlighting}
{Shaded}
\end{verbatim}
\end{center}

\textbf{Coverage Types:}

{\def\LTcaptype{none} % do not increment counter
\begin{longtable}[]{@{}
  >{\raggedright\arraybackslash}p{(\linewidth - 4\tabcolsep) * \real{0.4054}}
  >{\raggedright\arraybackslash}p{(\linewidth - 4\tabcolsep) * \real{0.2432}}
  >{\raggedright\arraybackslash}p{(\linewidth - 4\tabcolsep) * \real{0.3514}}@{}}
\toprule\noalign{}
\begin{minipage}[b]{\linewidth}\raggedright
Coverage Type
\end{minipage} & \begin{minipage}[b]{\linewidth}\raggedright
Formula
\end{minipage} & \begin{minipage}[b]{\linewidth}\raggedright
Description
\end{minipage} \\
\midrule\noalign{}
\endhead
\bottomrule\noalign{}
\endlastfoot
\textbf{Statement Coverage} & (Executed statements / Total statements) \times
100\% & Tests every line of code \\
\textbf{Branch Coverage} & (Executed branches / Total branches) \times 100\%
& Tests all decision outcomes \\
\textbf{Path Coverage} & (Executed paths / Total paths) \times 100\% & Tests
all execution paths \\
\textbf{Condition Coverage} & (Tested conditions / Total conditions) \times
100\% & Tests all logical conditions \\
\end{longtable}
}

\textbf{Advantages:}

\begin{itemize}
\tightlist
\item
  \textbf{Thorough testing} of code logic
\item
  \textbf{Early defect detection} in development
\item
  \textbf{Code optimization} opportunities identification
\item
  \textbf{Complete code coverage} possible
\end{itemize}

\textbf{Disadvantages:}

\begin{itemize}
\tightlist
\item
  \textbf{Expensive and time-consuming} process
\item
  \textbf{Requires programming skills} from testers
\item
  \textbf{May miss} requirement-related defects
\item
  \textbf{Complex for large} applications
\end{itemize}

\textbf{Tools Used:}

\begin{itemize}
\tightlist
\item
  \textbf{Code coverage tools} (JaCoCo, gcov)
\item
  \textbf{Static analysis tools} (SonarQube)
\item
  \textbf{Unit testing frameworks} (JUnit, NUnit)
\end{itemize}

\textbf{Example Test Cases:}

\begin{verbatim}
// Function to test
function calculateGrade(marks) \{
    if (marks {=} 90) return {A};
    else if (marks {=} 80) return {B};
    else if (marks {=} 70) return {C};
    else return {F};
\}

// White box test cases for 100\% branch coverage
// Test 1: marks = 95 (A grade path)
// Test 2: marks = 85 (B grade path)  
// Test 3: marks = 75 (C grade path)
// Test 4: marks = 65 (F grade path)
\end{verbatim}

\end{solutionbox}
\begin{mnemonicbox}
``SBPC'' - Statement, Branch, Path, Condition
coverage

\end{mnemonicbox}
\begin{center}\rule{0.5\linewidth}{0.5pt}\end{center}

\subsection*{Question 4(a) [3 marks]}\label{q4a}

\textbf{Importance of RAD model.}

\begin{solutionbox}

\textbf{RAD (Rapid Application Development)} model emphasizes quick
development through prototyping and iterative design.

{\def\LTcaptype{none} % do not increment counter
\begin{longtable}[]{@{}
  >{\raggedright\arraybackslash}p{(\linewidth - 4\tabcolsep) * \real{0.3529}}
  >{\raggedright\arraybackslash}p{(\linewidth - 4\tabcolsep) * \real{0.2647}}
  >{\raggedright\arraybackslash}p{(\linewidth - 4\tabcolsep) * \real{0.3824}}@{}}
\toprule\noalign{}
\begin{minipage}[b]{\linewidth}\raggedright
Importance
\end{minipage} & \begin{minipage}[b]{\linewidth}\raggedright
Benefit
\end{minipage} & \begin{minipage}[b]{\linewidth}\raggedright
Application
\end{minipage} \\
\midrule\noalign{}
\endhead
\bottomrule\noalign{}
\endlastfoot
\textbf{Fast Development} & Reduced time-to-market & Business
applications \\
\textbf{User Involvement} & Better requirement understanding &
Interactive systems \\
\textbf{Prototype-based} & Early feedback and validation & UI-intensive
applications \\
\textbf{Component Reuse} & Cost reduction and efficiency & Enterprise
applications \\
\end{longtable}
}

\textbf{Key Benefits:}

\begin{itemize}
\tightlist
\item
  \textbf{Quick delivery} of working prototypes
\item
  \textbf{Reduced development} time and costs
\item
  \textbf{High user satisfaction} through involvement
\item
  \textbf{Flexible to changes} during development
\end{itemize}

\textbf{When to Use RAD:}

\begin{itemize}
\tightlist
\item
  \textbf{Well-defined business} requirements
\item
  \textbf{Experienced development} team available
\item
  \textbf{Modular system} architecture possible
\end{itemize}

\end{solutionbox}
\begin{mnemonicbox}
``FUPR'' - Fast, User involvement, Prototype-based,
Reusable components

\end{mnemonicbox}
\begin{center}\rule{0.5\linewidth}{0.5pt}\end{center}

\subsection*{Question 4(b) [4 marks]}\label{q4b}

\textbf{Explain code inspection.}

\begin{solutionbox}

\textbf{Code Inspection} is a systematic examination of source code to
identify defects, improve quality, and ensure compliance with standards.

{\def\LTcaptype{none} % do not increment counter
\begin{longtable}[]{@{}
  >{\raggedright\arraybackslash}p{(\linewidth - 6\tabcolsep) * \real{0.1395}}
  >{\raggedright\arraybackslash}p{(\linewidth - 6\tabcolsep) * \real{0.3023}}
  >{\raggedright\arraybackslash}p{(\linewidth - 6\tabcolsep) * \real{0.3256}}
  >{\raggedright\arraybackslash}p{(\linewidth - 6\tabcolsep) * \real{0.2326}}@{}}
\toprule\noalign{}
\begin{minipage}[b]{\linewidth}\raggedright
Type
\end{minipage} & \begin{minipage}[b]{\linewidth}\raggedright
Description
\end{minipage} & \begin{minipage}[b]{\linewidth}\raggedright
Participants
\end{minipage} & \begin{minipage}[b]{\linewidth}\raggedright
Duration
\end{minipage} \\
\midrule\noalign{}
\endhead
\bottomrule\noalign{}
\endlastfoot
\textbf{Formal Inspection} & Structured process with defined roles & 3-6
people & 2-4 hours \\
\textbf{Walkthrough} & Author-led review session & 2-7 people & 1-2
hours \\
\textbf{Peer Review} & Informal colleague review & 2-3 people & 30-60
minutes \\
\textbf{Tool-based Review} & Automated code analysis & Individual &
Varies \\
\end{longtable}
}

\textbf{Code Inspection Process:}

\begin{itemize}
\tightlist
\item
  \textbf{Planning}: Select code, assign roles, schedule meeting
\item
  \textbf{Overview}: Author explains code purpose and design
\item
  \textbf{Preparation}: Reviewers study code individually
\item
  \textbf{Inspection Meeting}: Systematic defect identification
\item
  \textbf{Rework}: Author fixes identified issues
\item
  \textbf{Follow-up}: Verify defect resolution
\end{itemize}

\textbf{Benefits:}

\begin{itemize}
\tightlist
\item
  \textbf{Early defect detection} before testing
\item
  \textbf{Knowledge sharing} among team members
\item
  \textbf{Code quality improvement} and standardization
\item
  \textbf{Reduced maintenance} costs
\end{itemize}

\end{solutionbox}
\begin{mnemonicbox}
``FWPT'' - Formal, Walkthrough, Peer review,
Tool-based

\end{mnemonicbox}
\begin{center}\rule{0.5\linewidth}{0.5pt}\end{center}

\subsection*{Question 4(c) [7 marks]}\label{q4c}

\textbf{Explain cohesion with its classification.}

\begin{solutionbox}

\textbf{Cohesion} measures how closely related and focused the
responsibilities of a single module are. Higher cohesion indicates
better module design.

\textbf{Cohesion Types (Ranked from Best to Worst):}

{\def\LTcaptype{none} % do not increment counter
\begin{longtable}[]{@{}
  >{\raggedright\arraybackslash}p{(\linewidth - 6\tabcolsep) * \real{0.1579}}
  >{\raggedright\arraybackslash}p{(\linewidth - 6\tabcolsep) * \real{0.3421}}
  >{\raggedright\arraybackslash}p{(\linewidth - 6\tabcolsep) * \real{0.2368}}
  >{\raggedright\arraybackslash}p{(\linewidth - 6\tabcolsep) * \real{0.2632}}@{}}
\toprule\noalign{}
\begin{minipage}[b]{\linewidth}\raggedright
Type
\end{minipage} & \begin{minipage}[b]{\linewidth}\raggedright
Description
\end{minipage} & \begin{minipage}[b]{\linewidth}\raggedright
Example
\end{minipage} & \begin{minipage}[b]{\linewidth}\raggedright
Strength
\end{minipage} \\
\midrule\noalign{}
\endhead
\bottomrule\noalign{}
\endlastfoot
\textbf{Functional} & Single, well-defined task & Calculate tax amount &
Highest \\
\textbf{Sequential} & Output of one element feeds next &
Read\rightarrowProcess\rightarrowWrite data & High \\
\textbf{Communicational} & Elements operate on same data & Update
customer record & High \\
\textbf{Procedural} & Elements follow execution sequence &
Initialize\rightarrowProcess\rightarrowCleanup & Medium \\
\textbf{Temporal} & Elements executed at same time & System startup
routines & Medium \\
\textbf{Logical} & Similar logical functions grouped & All input/output
operations & Low \\
\textbf{Coincidental} & No meaningful relationship & Random utility
functions & Lowest \\
\end{longtable}
}

\textbf{Detailed Classification:}

\begin{center}
\textbf{Mermaid Diagram (Code)}
\begin{verbatim}
{Shaded}
{Highlighting}[]
graph TD
    A[Cohesion Types] {-{-}{} B[Functional {-} Best]}
    A {-{-}{} C[Sequential]}
    A {-{-}{} D[Communicational]}
    A {-{-}{} E[Procedural]}
    A {-{-}{} F[Temporal]}
    A {-{-}{} G[Logical]}
    A {-{-}{} H[Coincidental {-} Worst]}
    
    style B fill:\#4caf50
    style C fill:\#8bc34a
    style D fill:\#cddc39
    style E fill:\#ffeb3b
    style F fill:\#ff9800
    style G fill:\#ff5722
    style H fill:\#f44336
{Highlighting}
{Shaded}
\end{verbatim}
\end{center}

\textbf{Functional Cohesion (Best):}

\begin{itemize}
\tightlist
\item
  \textbf{Single responsibility} principle
\item
  \textbf{Example}: \texttt{calculateInterest()} - only calculates
  interest
\item
  \textbf{Benefits}: Easy to understand, test, and maintain
\end{itemize}

\textbf{Sequential Cohesion:}

\begin{itemize}
\tightlist
\item
  \textbf{Data flows} from one element to next
\item
  \textbf{Example}:
  \texttt{readFile()\ \rightarrow\ parseData()\ \rightarrow\ generateReport()}
\item
  \textbf{Good design} for processing pipelines
\end{itemize}

\textbf{Communicational Cohesion:}

\begin{itemize}
\tightlist
\item
  \textbf{Same data structure} manipulation
\item
  \textbf{Example}: Module updating all fields of customer record
\item
  \textbf{Reasonable design} for data-centric operations
\end{itemize}

\textbf{Procedural Cohesion:}

\begin{itemize}
\tightlist
\item
  \textbf{Control flow} relationship
\item
  \textbf{Example}: Initialization sequence in specific order
\item
  \textbf{Acceptable} for procedural operations
\end{itemize}

\textbf{Temporal Cohesion:}

\begin{itemize}
\tightlist
\item
  \textbf{Time-based} relationship
\item
  \textbf{Example}: System startup or shutdown routines
\item
  \textbf{Moderate quality} design
\end{itemize}

\textbf{Logical Cohesion:}

\begin{itemize}
\tightlist
\item
  \textbf{Similar functions} grouped together
\item
  \textbf{Example}: All mathematical functions in one module
\item
  \textbf{Poor design} - difficult to maintain
\end{itemize}

\textbf{Coincidental Cohesion (Worst):}

\begin{itemize}
\tightlist
\item
  \textbf{No logical relationship} between elements
\item
  \textbf{Example}: Miscellaneous utility functions
\item
  \textbf{Avoid this} - creates maintenance nightmares
\end{itemize}

\textbf{Benefits of High Cohesion:}

\begin{itemize}
\tightlist
\item
  \textbf{Easier maintenance} and debugging
\item
  \textbf{Better reusability} of modules
\item
  \textbf{Improved testability} and reliability
\item
  \textbf{Clearer code} understanding
\end{itemize}

\textbf{How to Achieve High Cohesion:}

\begin{itemize}
\tightlist
\item
  \textbf{Single Responsibility Principle}: One reason to change
\item
  \textbf{Clear module purpose}: Well-defined functionality
\item
  \textbf{Minimal interfaces}: Reduce external dependencies
\item
  \textbf{Logical grouping}: Related functions together
\end{itemize}

\end{solutionbox}
\begin{mnemonicbox}
``FSCPTLC'' - Functional, Sequential,
Communicational, Procedural, Temporal, Logical, Coincidental

\end{mnemonicbox}
\begin{center}\rule{0.5\linewidth}{0.5pt}\end{center}

\subsection*{Question 4(a) OR [3
marks]}\label{q4a}

\textbf{Software doesn't wear out.}

\begin{solutionbox}

\textbf{Software doesn't wear out} means software doesn't deteriorate
physically like hardware components do over time.

{\def\LTcaptype{none} % do not increment counter
\begin{longtable}[]{@{}
  >{\raggedright\arraybackslash}p{(\linewidth - 4\tabcolsep) * \real{0.2857}}
  >{\raggedright\arraybackslash}p{(\linewidth - 4\tabcolsep) * \real{0.3571}}
  >{\raggedright\arraybackslash}p{(\linewidth - 4\tabcolsep) * \real{0.3571}}@{}}
\toprule\noalign{}
\begin{minipage}[b]{\linewidth}\raggedright
Aspect
\end{minipage} & \begin{minipage}[b]{\linewidth}\raggedright
Hardware
\end{minipage} & \begin{minipage}[b]{\linewidth}\raggedright
Software
\end{minipage} \\
\midrule\noalign{}
\endhead
\bottomrule\noalign{}
\endlastfoot
\textbf{Physical Degradation} & Components wear out & No physical
degradation \\
\textbf{Age Effect} & Performance decreases & Performance remains
constant \\
\textbf{Failure Pattern} & Increasing failure rate & Constant failure
rate \\
\textbf{Maintenance} & Replace worn parts & Fix logical errors only \\
\end{longtable}
}

\textbf{Key Points:}

\begin{itemize}
\tightlist
\item
  \textbf{No mechanical parts} to wear out
\item
  \textbf{Logical errors} don't increase with time
\item
  \textbf{Performance degradation} due to environment changes, not aging
\item
  \textbf{Failures occur} due to design flaws, not wear
\end{itemize}

\textbf{Why This Matters:}

\begin{itemize}
\tightlist
\item
  \textbf{Different maintenance} approach needed
\item
  \textbf{Focus on updates} rather than replacement
\item
  \textbf{Longevity planning} differs from hardware
\end{itemize}

\end{solutionbox}
\begin{mnemonicbox}
``NLPF'' - No physical parts, Logical errors,
Performance constant, Failures from design

\end{mnemonicbox}
\begin{center}\rule{0.5\linewidth}{0.5pt}\end{center}

\subsection*{Question 4(b) OR [4
marks]}\label{q4b}

\textbf{Explain use-case diagram.}

\begin{solutionbox}

\textbf{Use-case Diagram} is a UML behavioral diagram showing system
functionality from user's perspective through interactions between
actors and use cases.

{\def\LTcaptype{none} % do not increment counter
\begin{longtable}[]{@{}lll@{}}
\toprule\noalign{}
Component & Symbol & Description \\
\midrule\noalign{}
\endhead
\bottomrule\noalign{}
\endlastfoot
\textbf{Actor} & Stick figure & External entity interacting with
system \\
\textbf{Use Case} & Oval & System function or service \\
\textbf{System Boundary} & Rectangle & System scope definition \\
\textbf{Relationships} & Lines/Arrows & Associations between
components \\
\end{longtable}
}

\textbf{Use-case Diagram Elements:}

\begin{center}
\textbf{Mermaid Diagram (Code)}
\begin{verbatim}
{Shaded}
{Highlighting}[]
graph TD
    subgraph "Library Management System"
      direction LR
        UC1((Search Books))
        UC2((Borrow Book))
        UC3((Return Book))
        UC4((Manage Catalog))
        UC5((Generate Reports))
    end
    
    A1[Student] {-{-}{} UC1}
    A1 {-{-}{} UC2}
    A1 {-{-}{} UC3}
    
    A2[Librarian] {-{-}{} UC4}
    A2 {-{-}{} UC5}
    A2 {-{-}{} UC1}
    
    UC2 {-.{-}{} UC1}
    UC3 {-.{-}{} UC1}
{Highlighting}
{Shaded}
\end{verbatim}
\end{center}

\textbf{Relationship Types:}

\begin{itemize}
\tightlist
\item
  \textbf{Association}: Actor participates in use case
\item
  \textbf{Include}: Use case always includes another use case
\item
  \textbf{Extend}: Use case conditionally extends another
\item
  \textbf{Generalization}: Inheritance between actors/use cases
\end{itemize}

\textbf{Benefits:}

\begin{itemize}
\tightlist
\item
  \textbf{Clear system scope} definition
\item
  \textbf{User requirements} visualization
\item
  \textbf{Communication tool} with stakeholders
\item
  \textbf{Test case} derivation basis
\end{itemize}

\end{solutionbox}
\begin{mnemonicbox}
``AUSB'' - Actor, Use case, System boundary,
Relationships

\end{mnemonicbox}
\begin{center}\rule{0.5\linewidth}{0.5pt}\end{center}

\subsection*{Question 4(c) OR [7
marks]}\label{q4c}

\textbf{Explain Black box Testing.}

\begin{solutionbox}

\textbf{Black Box Testing} is a testing method that examines software
functionality without knowledge of internal code structure or
implementation details.

{\def\LTcaptype{none} % do not increment counter
\begin{longtable}[]{@{}
  >{\raggedright\arraybackslash}p{(\linewidth - 2\tabcolsep) * \real{0.3810}}
  >{\raggedright\arraybackslash}p{(\linewidth - 2\tabcolsep) * \real{0.6190}}@{}}
\toprule\noalign{}
\begin{minipage}[b]{\linewidth}\raggedright
Aspect
\end{minipage} & \begin{minipage}[b]{\linewidth}\raggedright
Description
\end{minipage} \\
\midrule\noalign{}
\endhead
\bottomrule\noalign{}
\endlastfoot
\textbf{Also Known As} & Functional testing, Behavioral testing,
Specification-based testing \\
\textbf{Access Level} & No access to source code or internal
structure \\
\textbf{Focus} & Input-output behavior, functional requirements \\
\textbf{Tester Knowledge} & Domain knowledge required, not
programming \\
\end{longtable}
}

\textbf{Black Box Testing Techniques:}

\begin{center}
\textbf{Mermaid Diagram (Code)}
\begin{verbatim}
{Shaded}
{Highlighting}[]
graph TD
    A[Black Box Testing] {-{-}{} B[Equivalence Partitioning]}
    A {-{-}{} C[Boundary Value Analysis]}
    A {-{-}{} D[Decision Table Testing]}
    A {-{-}{} E[State Transition Testing]}
    
    B {-{-}{} B1[Valid/Invalid input classes]}
    C {-{-}{} C1[Test boundary conditions]}
    D {-{-}{} D1[Complex business rules]}
    E {-{-}{} E1[State{-}dependent behavior]}
{Highlighting}
{Shaded}
\end{verbatim}
\end{center}

\textbf{Testing Techniques:}

{\def\LTcaptype{none} % do not increment counter
\begin{longtable}[]{@{}
  >{\raggedright\arraybackslash}p{(\linewidth - 4\tabcolsep) * \real{0.3333}}
  >{\raggedright\arraybackslash}p{(\linewidth - 4\tabcolsep) * \real{0.3939}}
  >{\raggedright\arraybackslash}p{(\linewidth - 4\tabcolsep) * \real{0.2727}}@{}}
\toprule\noalign{}
\begin{minipage}[b]{\linewidth}\raggedright
Technique
\end{minipage} & \begin{minipage}[b]{\linewidth}\raggedright
Description
\end{minipage} & \begin{minipage}[b]{\linewidth}\raggedright
Example
\end{minipage} \\
\midrule\noalign{}
\endhead
\bottomrule\noalign{}
\endlastfoot
\textbf{Equivalence Partitioning} & Divide inputs into valid/invalid
groups & Age: 0-17, 18-60, 60+ \\
\textbf{Boundary Value Analysis} & Test at boundaries of input ranges &
Test at 17, 18, 60, 61 \\
\textbf{Decision Table} & Test combinations of conditions & Login with
valid/invalid user/password \\
\textbf{State Transition} & Test state changes & ATM states: Idle\rightarrowCard
inserted\rightarrowPIN entry \\
\end{longtable}
}

\textbf{Test Case Design Example:}

\begin{verbatim}
Function: Login validation
Inputs: Username, Password
Valid equivalence classes:
- Username: 5-20 characters, alphanumeric
- Password: 8-15 characters, special chars allowed

Invalid equivalence classes:
- Username: <5 or >20 characters, special chars
- Password: <8 or >15 characters, spaces

Boundary values to test:
- Username: 4, 5, 20, 21
- Password: 7, 8, 15, 16
\end{verbatim}

\textbf{Advantages:}

\begin{itemize}
\tightlist
\item
  \textbf{No programming knowledge} required for testers
\item
  \textbf{User perspective} testing approach
\item
  \textbf{Independent verification} of requirements
\item
  \textbf{Effective for} large applications
\end{itemize}

\textbf{Disadvantages:}

\begin{itemize}
\tightlist
\item
  \textbf{Limited code coverage} visibility
\item
  \textbf{Cannot identify} unused code paths
\item
  \textbf{Difficult to design} test cases without specifications
\item
  \textbf{May miss} logical errors in code
\end{itemize}

\textbf{Types of Black Box Testing:}

\begin{itemize}
\tightlist
\item
  \textbf{Functional Testing}: Feature verification
\item
  \textbf{Integration Testing}: Module interaction testing
\item
  \textbf{System Testing}: Complete system validation
\item
  \textbf{Acceptance Testing}: User requirement verification
\end{itemize}

\textbf{Tools Used:}

\begin{itemize}
\tightlist
\item
  \textbf{Test management tools} (TestRail, Zephyr)
\item
  \textbf{Automation tools} (Selenium, QTP)
\item
  \textbf{Defect tracking tools} (Jira, Bugzilla)
\end{itemize}

\textbf{When to Use:}

\begin{itemize}
\tightlist
\item
  \textbf{Requirements-based} testing
\item
  \textbf{User acceptance} testing
\item
  \textbf{System integration} testing
\item
  \textbf{Regression testing} after changes
\end{itemize}

\end{solutionbox}
\begin{mnemonicbox}
``EBDS'' - Equivalence, Boundary, Decision table,
State transition

\end{mnemonicbox}
\begin{center}\rule{0.5\linewidth}{0.5pt}\end{center}

\subsection*{Question 5(a) [3 marks]}\label{q5a}

\textbf{Difference between verification and validation.}

\begin{solutionbox}

{\def\LTcaptype{none} % do not increment counter
\begin{longtable}[]{@{}
  >{\raggedright\arraybackslash}p{(\linewidth - 4\tabcolsep) * \real{0.2353}}
  >{\raggedright\arraybackslash}p{(\linewidth - 4\tabcolsep) * \real{0.4118}}
  >{\raggedright\arraybackslash}p{(\linewidth - 4\tabcolsep) * \real{0.3529}}@{}}
\toprule\noalign{}
\begin{minipage}[b]{\linewidth}\raggedright
Aspect
\end{minipage} & \begin{minipage}[b]{\linewidth}\raggedright
Verification
\end{minipage} & \begin{minipage}[b]{\linewidth}\raggedright
Validation
\end{minipage} \\
\midrule\noalign{}
\endhead
\bottomrule\noalign{}
\endlastfoot
\textbf{Definition} & ``Are we building the product right?'' & ``Are we
building the right product?'' \\
\textbf{Focus} & Process compliance & Product correctness \\
\textbf{When} & During development & After development \\
\textbf{Method} & Reviews, inspections, walkthroughs & Testing with
actual data \\
\textbf{Cost} & Lower cost of defect detection & Higher cost of defect
detection \\
\end{longtable}
}

\textbf{Key Differences:}

\begin{itemize}
\tightlist
\item
  \textbf{Verification}: Checks against \textbf{specifications}
\item
  \textbf{Validation}: Checks against \textbf{user needs}
\item
  \textbf{Verification}: \textbf{Static testing} methods
\item
  \textbf{Validation}: \textbf{Dynamic testing} methods
\end{itemize}

\textbf{Examples:}

\begin{itemize}
\tightlist
\item
  \textbf{Verification}: Code review, design review, SRS review
\item
  \textbf{Validation}: Unit testing, integration testing, system testing
\end{itemize}

\end{solutionbox}
\begin{mnemonicbox}
``VR vs VT'' - Verification Reviews vs Validation
Testing

\end{mnemonicbox}
\begin{center}\rule{0.5\linewidth}{0.5pt}\end{center}

\subsection*{Question 5(b) [4 marks]}\label{q5b}

\textbf{Explain SRS.}

\begin{solutionbox}

\textbf{SRS (Software Requirement Specification)} is a detailed document
describing the functional and non-functional requirements of a software
system.

{\def\LTcaptype{none} % do not increment counter
\begin{longtable}[]{@{}
  >{\raggedright\arraybackslash}p{(\linewidth - 4\tabcolsep) * \real{0.3333}}
  >{\raggedright\arraybackslash}p{(\linewidth - 4\tabcolsep) * \real{0.3939}}
  >{\raggedright\arraybackslash}p{(\linewidth - 4\tabcolsep) * \real{0.2727}}@{}}
\toprule\noalign{}
\begin{minipage}[b]{\linewidth}\raggedright
Component
\end{minipage} & \begin{minipage}[b]{\linewidth}\raggedright
Description
\end{minipage} & \begin{minipage}[b]{\linewidth}\raggedright
Purpose
\end{minipage} \\
\midrule\noalign{}
\endhead
\bottomrule\noalign{}
\endlastfoot
\textbf{Introduction} & System overview and scope & Context setting \\
\textbf{Functional Requirements} & What system should do & Feature
specification \\
\textbf{Non-functional Requirements} & How system should perform &
Quality attributes \\
\textbf{Constraints} & Limitations and restrictions & Boundary
definition \\
\end{longtable}
}

\textbf{SRS Structure:}

\begin{itemize}
\tightlist
\item
  \textbf{System Purpose}: Why the system is needed
\item
  \textbf{System Scope}: What the system will and won't do
\item
  \textbf{Definitions}: Technical terms and acronyms
\item
  \textbf{User Requirements}: High-level user needs
\item
  \textbf{System Requirements}: Detailed technical specifications
\end{itemize}

\textbf{Importance of SRS:}

\begin{itemize}
\tightlist
\item
  \textbf{Communication tool} between stakeholders
\item
  \textbf{Baseline for testing} and validation
\item
  \textbf{Contract basis} between client and developer
\item
  \textbf{Change management} reference document
\end{itemize}

\textbf{Users of SRS:}

\begin{itemize}
\tightlist
\item
  \textbf{Developers}: Implementation guidance
\item
  \textbf{Testers}: Test case creation
\item
  \textbf{Project Managers}: Planning and tracking
\item
  \textbf{Clients}: Requirement verification
\end{itemize}

\end{solutionbox}
\begin{mnemonicbox}
``IFNC'' - Introduction, Functional, Non-functional,
Constraints

\end{mnemonicbox}
\begin{center}\rule{0.5\linewidth}{0.5pt}\end{center}

\subsection*{Question 5(c) [7 marks]}\label{q5c}

\textbf{Explain Risk Management.}

\begin{solutionbox}

\textbf{Risk Management} is the systematic process of identifying,
analyzing, and responding to project risks to minimize their impact on
project success.

\textbf{Risk Management Process:}

\begin{verbatim}
flowchart LR
    A[Risk Identification] {-{-} B[Risk Analysis]}
    B {-{-} C[Risk Assessment]}
    C {-{-} D[Risk Mitigation]}
    D {-{-} E[Risk Monitoring]}
    E {-{-} A}
    
    style A fill:\#ffebee
    style B fill:\#e8f5e8
    style C fill:\#fff3e0
    style D fill:\#e3f2fd
    style E fill:\#f3e5f5
\end{verbatim}

{\def\LTcaptype{none} % do not increment counter
\begin{longtable}[]{@{}
  >{\raggedright\arraybackslash}p{(\linewidth - 4\tabcolsep) * \real{0.2593}}
  >{\raggedright\arraybackslash}p{(\linewidth - 4\tabcolsep) * \real{0.4444}}
  >{\raggedright\arraybackslash}p{(\linewidth - 4\tabcolsep) * \real{0.2963}}@{}}
\toprule\noalign{}
\begin{minipage}[b]{\linewidth}\raggedright
Phase
\end{minipage} & \begin{minipage}[b]{\linewidth}\raggedright
Activities
\end{minipage} & \begin{minipage}[b]{\linewidth}\raggedright
Output
\end{minipage} \\
\midrule\noalign{}
\endhead
\bottomrule\noalign{}
\endlastfoot
\textbf{Identification} & Brainstorming, checklists, expert judgment &
Risk register \\
\textbf{Analysis} & Probability and impact assessment & Risk matrix \\
\textbf{Assessment} & Risk prioritization and ranking & Risk priority
list \\
\textbf{Mitigation} & Response strategy development & Mitigation
plans \\
\textbf{Monitoring} & Track risks and mitigation effectiveness & Status
reports \\
\end{longtable}
}

\textbf{Risk Categories:}

\textbf{Project Risks:}

\begin{itemize}
\tightlist
\item
  \textbf{Schedule delays} due to resource unavailability
\item
  \textbf{Budget overruns} from scope changes
\item
  \textbf{Team turnover} affecting productivity
\item
  \textbf{Communication gaps} between stakeholders
\end{itemize}

\textbf{Technical Risks:}

\begin{itemize}
\tightlist
\item
  \textbf{Technology complexity} exceeding team skills
\item
  \textbf{Integration challenges} with existing systems
\item
  \textbf{Performance issues} under load conditions
\item
  \textbf{Security vulnerabilities} in design
\end{itemize}

\textbf{Business Risks:}

\begin{itemize}
\tightlist
\item
  \textbf{Changing requirements} from market conditions
\item
  \textbf{Competition} releasing similar products
\item
  \textbf{Regulatory changes} affecting compliance
\item
  \textbf{Stakeholder conflicts} on priorities
\end{itemize}

\textbf{Risk Response Strategies:}

{\def\LTcaptype{none} % do not increment counter
\begin{longtable}[]{@{}
  >{\raggedright\arraybackslash}p{(\linewidth - 6\tabcolsep) * \real{0.2222}}
  >{\raggedright\arraybackslash}p{(\linewidth - 6\tabcolsep) * \real{0.2889}}
  >{\raggedright\arraybackslash}p{(\linewidth - 6\tabcolsep) * \real{0.2889}}
  >{\raggedright\arraybackslash}p{(\linewidth - 6\tabcolsep) * \real{0.2000}}@{}}
\toprule\noalign{}
\begin{minipage}[b]{\linewidth}\raggedright
Strategy
\end{minipage} & \begin{minipage}[b]{\linewidth}\raggedright
Description
\end{minipage} & \begin{minipage}[b]{\linewidth}\raggedright
When to Use
\end{minipage} & \begin{minipage}[b]{\linewidth}\raggedright
Example
\end{minipage} \\
\midrule\noalign{}
\endhead
\bottomrule\noalign{}
\endlastfoot
\textbf{Accept} & Acknowledge risk, no action & Low impact risks & Minor
UI changes \\
\textbf{Avoid} & Eliminate risk source & High impact, avoidable & Change
technology \\
\textbf{Mitigate} & Reduce probability/impact & Manageable risks &
Additional testing \\
\textbf{Transfer} & Shift risk to third party & Specialized risks &
Insurance, outsourcing \\
\end{longtable}
}

\textbf{Risk Assessment Matrix:}

{\def\LTcaptype{none} % do not increment counter
\begin{longtable}[]{@{}llll@{}}
\toprule\noalign{}
Probability/Impact & Low & Medium & High \\
\midrule\noalign{}
\endhead
\bottomrule\noalign{}
\endlastfoot
\textbf{High} & Medium & High & Critical \\
\textbf{Medium} & Low & Medium & High \\
\textbf{Low} & Very Low & Low & Medium \\
\end{longtable}
}

\textbf{Risk Mitigation Techniques:}

\begin{itemize}
\tightlist
\item
  \textbf{Prototyping} to reduce technical uncertainty
\item
  \textbf{Staff training} to address skill gaps
\item
  \textbf{Regular reviews} to catch issues early
\item
  \textbf{Contingency planning} for critical scenarios
\end{itemize}

\textbf{Benefits of Risk Management:}

\begin{itemize}
\tightlist
\item
  \textbf{Proactive problem} prevention
\item
  \textbf{Better decision} making with risk awareness
\item
  \textbf{Improved project} success rates
\item
  \textbf{Stakeholder confidence} in project delivery
\end{itemize}

\textbf{Risk Monitoring Activities:}

\begin{itemize}
\tightlist
\item
  \textbf{Regular risk reviews} and updates
\item
  \textbf{Risk trigger monitoring} for early warning
\item
  \textbf{Mitigation plan} progress tracking
\item
  \textbf{New risk identification} as project evolves
\end{itemize}

\textbf{Tools for Risk Management:}

\begin{itemize}
\tightlist
\item
  \textbf{Risk registers} and databases
\item
  \textbf{Risk assessment} matrices
\item
  \textbf{Monte Carlo} simulation for quantitative analysis
\item
  \textbf{Expert judgment} and historical data
\end{itemize}

\textbf{Key Success Factors:}

\begin{itemize}
\tightlist
\item
  \textbf{Management commitment} to risk processes
\item
  \textbf{Team awareness} and participation
\item
  \textbf{Regular communication} about risks
\item
  \textbf{Integration} with project management processes
\end{itemize}

\end{solutionbox}
\begin{mnemonicbox}
``IATMM'' - Identify, Analyze, Assess, Treat, Monitor
risks

\end{mnemonicbox}
\begin{center}\rule{0.5\linewidth}{0.5pt}\end{center}

\subsection*{Question 5(a) OR [3
marks]}\label{q5a}

\textbf{List out any functional requirements for Hostel management
system.}

\begin{solutionbox}

\textbf{Functional Requirements} for Hostel Management System define
what the system should do to manage hostel operations effectively.

{\def\LTcaptype{none} % do not increment counter
\begin{longtable}[]{@{}
  >{\raggedright\arraybackslash}p{(\linewidth - 2\tabcolsep) * \real{0.2500}}
  >{\raggedright\arraybackslash}p{(\linewidth - 2\tabcolsep) * \real{0.7500}}@{}}
\toprule\noalign{}
\begin{minipage}[b]{\linewidth}\raggedright
Module
\end{minipage} & \begin{minipage}[b]{\linewidth}\raggedright
Functional Requirements
\end{minipage} \\
\midrule\noalign{}
\endhead
\bottomrule\noalign{}
\endlastfoot
\textbf{Student Management} & Register students, assign rooms, maintain
profiles \\
\textbf{Room Management} & Room allocation, availability tracking,
maintenance \\
\textbf{Fee Management} & Fee calculation, payment processing, receipt
generation \\
\textbf{Visitor Management} & Visitor registration, entry/exit tracking,
approval \\
\end{longtable}
}

\textbf{Detailed Functional Requirements:}

\textbf{Student Module:}

\begin{itemize}
\tightlist
\item
  \textbf{Student registration} with personal details
\item
  \textbf{Room assignment} based on availability
\item
  \textbf{Student profile} management and updates
\end{itemize}

\textbf{Administrative Module:}

\begin{itemize}
\tightlist
\item
  \textbf{Staff management} and role assignment
\item
  \textbf{Report generation} for occupancy and finances
\item
  \textbf{Complaint management} and resolution tracking
\end{itemize}

\textbf{Security Module:}

\begin{itemize}
\tightlist
\item
  \textbf{Access control} for different user types
\item
  \textbf{Visitor logging} and approval system
\item
  \textbf{Emergency contact} management
\end{itemize}

\end{solutionbox}
\begin{mnemonicbox}
``SRFV'' - Student, Room, Fee, Visitor management

\end{mnemonicbox}
\begin{center}\rule{0.5\linewidth}{0.5pt}\end{center}

\subsection*{Question 5(b) OR [4
marks]}\label{q5b}

\textbf{Explain Agile process.}

\begin{solutionbox}

\textbf{Agile Process} is an iterative and incremental software
development approach emphasizing collaboration, flexibility, and
customer satisfaction.

{\def\LTcaptype{none} % do not increment counter
\begin{longtable}[]{@{}
  >{\raggedright\arraybackslash}p{(\linewidth - 4\tabcolsep) * \real{0.4211}}
  >{\raggedright\arraybackslash}p{(\linewidth - 4\tabcolsep) * \real{0.3421}}
  >{\raggedright\arraybackslash}p{(\linewidth - 4\tabcolsep) * \real{0.2368}}@{}}
\toprule\noalign{}
\begin{minipage}[b]{\linewidth}\raggedright
Agile Principle
\end{minipage} & \begin{minipage}[b]{\linewidth}\raggedright
Description
\end{minipage} & \begin{minipage}[b]{\linewidth}\raggedright
Benefit
\end{minipage} \\
\midrule\noalign{}
\endhead
\bottomrule\noalign{}
\endlastfoot
\textbf{Customer Collaboration} & Continuous customer involvement &
Better requirement understanding \\
\textbf{Working Software} & Deliver functional software frequently &
Early value delivery \\
\textbf{Responding to Change} & Adapt to changing requirements & Market
responsiveness \\
\textbf{Individuals and Interactions} & People over processes and tools
& Better team dynamics \\
\end{longtable}
}

\textbf{Agile Values:}

\begin{itemize}
\tightlist
\item
  \textbf{Individuals and interactions} over processes and tools
\item
  \textbf{Working software} over comprehensive documentation
\item
  \textbf{Customer collaboration} over contract negotiation
\item
  \textbf{Responding to change} over following a plan
\end{itemize}

\textbf{Agile Practices:}

\begin{itemize}
\tightlist
\item
  \textbf{Short iterations} (1-4 weeks)
\item
  \textbf{Daily standups} for team coordination
\item
  \textbf{Sprint planning} and review meetings
\item
  \textbf{Continuous integration} and testing
\end{itemize}

\textbf{Benefits:}

\begin{itemize}
\tightlist
\item
  \textbf{Faster delivery} of working software
\item
  \textbf{Better quality} through continuous testing
\item
  \textbf{Improved stakeholder} satisfaction
\item
  \textbf{Flexibility} to handle changes
\end{itemize}

\end{solutionbox}
\begin{mnemonicbox}
``CWRI'' - Customer collaboration, Working software,
Responding to change, Individuals

\end{mnemonicbox}
\begin{center}\rule{0.5\linewidth}{0.5pt}\end{center}

\subsection*{Question 5(c) OR [7
marks]}\label{q5c}

\textbf{Explain Software Engineering - A layered approach}

\begin{solutionbox}

\textbf{Software Engineering - A Layered Approach} represents software
engineering as a structured methodology with multiple interconnected
layers, each building upon the foundation of lower layers.

\textbf{Layered Architecture:}

\begin{center}
\textbf{Mermaid Diagram (Code)}
\begin{verbatim}
{Shaded}
{Highlighting}[]
graph LR
    A[Quality Focus] {-{-}{} B[Process Layer]}
    B {-{-}{} C[Methods Layer]}
    C {-{-}{} D[Tools Layer]}
    
    style A fill:\#4caf50
    style B fill:\#2196f3
    style C fill:\#ff9800
    style D fill:\#9c27b0
{Highlighting}
{Shaded}
\end{verbatim}
\end{center}

{\def\LTcaptype{none} % do not increment counter
\begin{longtable}[]{@{}
  >{\raggedright\arraybackslash}p{(\linewidth - 6\tabcolsep) * \real{0.1795}}
  >{\raggedright\arraybackslash}p{(\linewidth - 6\tabcolsep) * \real{0.3333}}
  >{\raggedright\arraybackslash}p{(\linewidth - 6\tabcolsep) * \real{0.2308}}
  >{\raggedright\arraybackslash}p{(\linewidth - 6\tabcolsep) * \real{0.2564}}@{}}
\toprule\noalign{}
\begin{minipage}[b]{\linewidth}\raggedright
Layer
\end{minipage} & \begin{minipage}[b]{\linewidth}\raggedright
Description
\end{minipage} & \begin{minipage}[b]{\linewidth}\raggedright
Purpose
\end{minipage} & \begin{minipage}[b]{\linewidth}\raggedright
Examples
\end{minipage} \\
\midrule\noalign{}
\endhead
\bottomrule\noalign{}
\endlastfoot
\textbf{Quality Focus} & Foundation emphasizing quality & Ensures
customer satisfaction & Quality standards, metrics \\
\textbf{Process} & Framework for software development & Provides
structure and control & SDLC models, project management \\
\textbf{Methods} & Technical approaches and techniques & Guides
development activities & Analysis, design, testing methods \\
\textbf{Tools} & Automated support for methods & Increases productivity
& IDEs, testing tools, CASE tools \\
\end{longtable}
}

\textbf{Detailed Layer Analysis:}

\textbf{Quality Focus (Foundation Layer):}

\begin{itemize}
\tightlist
\item
  \textbf{Bedrock of software engineering} approach
\item
  \textbf{Commitment to quality} in all activities
\item
  \textbf{Customer satisfaction} as primary goal
\item
  \textbf{Continuous improvement} mindset
\item
  \textbf{Quality characteristics}: Correctness, reliability,
  efficiency, maintainability
\end{itemize}

\textbf{Process Layer:}

\begin{itemize}
\tightlist
\item
  \textbf{Defines framework} for effective delivery
\item
  \textbf{Establishes context} for technical methods
\item
  \textbf{Key elements}: Communication, planning, modeling,
  construction, deployment
\item
  \textbf{Process models}: Waterfall, Agile, Spiral, Incremental
\item
  \textbf{Management activities}: Project planning, tracking, risk
  management
\end{itemize}

\textbf{Methods Layer:}

\begin{itemize}
\tightlist
\item
  \textbf{Technical knowledge} for building software
\item
  \textbf{Encompasses broad array} of tasks
\item
  \textbf{Communication methods}: Requirement elicitation, analysis
\item
  \textbf{Planning methods}: Project estimation, scheduling
\item
  \textbf{Modeling methods}: Analysis and design techniques
\item
  \textbf{Construction methods}: Coding standards, testing strategies
\item
  \textbf{Deployment methods}: Delivery, support, feedback
\end{itemize}

\textbf{Tools Layer:}

\begin{itemize}
\tightlist
\item
  \textbf{Automated or semi-automated} support
\item
  \textbf{Increases efficiency} and reduces errors
\item
  \textbf{Tool categories}:

  \begin{itemize}
  \tightlist
  \item
    \textbf{Development environments}: IDEs, compilers
  \item
    \textbf{Analysis and design tools}: UML tools, CASE tools
  \item
    \textbf{Testing tools}: Unit testing, automation frameworks
  \item
    \textbf{Project management tools}: Scheduling, tracking software
  \end{itemize}
\end{itemize}

\textbf{Interactions Between Layers:}

\textbf{Quality \leftrightarrow Process:}

\begin{itemize}
\tightlist
\item
  Quality focus \textbf{drives process} selection
\item
  Process \textbf{ensures quality} delivery
\end{itemize}

\textbf{Process \leftrightarrow Methods:}

\begin{itemize}
\tightlist
\item
  Process \textbf{provides context} for methods
\item
  Methods \textbf{implement process} activities
\end{itemize}

\textbf{Methods \leftrightarrow Tools:}

\begin{itemize}
\tightlist
\item
  Methods \textbf{define what} needs to be done
\item
  Tools \textbf{provide how} to do it efficiently
\end{itemize}

\textbf{Benefits of Layered Approach:}

\begin{itemize}
\tightlist
\item
  \textbf{Systematic methodology} for software development
\item
  \textbf{Scalability} from small to large projects
\item
  \textbf{Flexibility} to adapt tools and methods
\item
  \textbf{Quality assurance} at every level
\item
  \textbf{Risk reduction} through structured approach
\end{itemize}

\textbf{Implementation Strategy:}

\begin{itemize}
\tightlist
\item
  \textbf{Start with quality focus} establishment
\item
  \textbf{Select appropriate process} for project context
\item
  \textbf{Choose methods} matching process requirements
\item
  \textbf{Integrate tools} supporting selected methods
\item
  \textbf{Continuous evaluation} and improvement
\end{itemize}

\textbf{Key Success Factors:}

\begin{itemize}
\tightlist
\item
  \textbf{Management commitment} to quality
\item
  \textbf{Team training} on methods and tools
\item
  \textbf{Process adherence} and discipline
\item
  \textbf{Tool integration} and standardization
\item
  \textbf{Continuous improvement} culture
\end{itemize}

\textbf{Real-world Application:}

\begin{itemize}
\tightlist
\item
  \textbf{Large organizations}: Complete layer implementation
\item
  \textbf{Small teams}: Simplified but consistent approach
\item
  \textbf{Project-specific}: Tailored layer selection
\item
  \textbf{Industry standards}: Compliance with quality frameworks
\end{itemize}

\end{solutionbox}
\begin{mnemonicbox}
``QPMT'' - Quality focus, Process, Methods, Tools
(from bottom to top)

\end{mnemonicbox}

\end{document}
