\documentclass[10pt,a4paper]{article}

% content/resources/templates/preamble.tex
\usepackage[margin=0.6in]{geometry}
\author{Milav Dabgar}
\usepackage{amsmath,amssymb,amsthm}
\usepackage{booktabs}
\usepackage{multirow}
\usepackage{xcolor}
\usepackage{tcolorbox}
\tcbuselibrary{breakable,skins}
\usepackage[colorlinks=true,linkcolor=blue]{hyperref}
\usepackage{titlesec}
\usepackage{enumitem}
\usepackage{tikz}
\usepackage{pgfplots}
\usepackage{circuitikz}
\usepackage[version=4]{mhchem}
\usepackage{longtable}
\usepackage{array}
\usepackage{float}
\usepackage{caption}
\usepackage{listings}

\lstset{
  basicstyle=\small\ttfamily,
  breaklines=true,
  breakatwhitespace=false,
  postbreak=\mbox{\textcolor{red}{$\hookrightarrow$}\space},
  float=false,
  numbers=left,
  numberstyle=\tiny\color{gray},
  numbersep=10pt,
  xleftmargin=2em,
  keywordstyle=\color{blue},
  commentstyle=\color{green!60!black},
  stringstyle=\color{purple},
  backgroundcolor=\color{gray!5},
  showstringspaces=false,
  tabsize=2,
  captionpos=b,
  keepspaces=true,
  columns=flexible
}

\pgfplotsset{compat=1.18}
\usetikzlibrary{shapes,arrows,positioning,calc,patterns,decorations.pathmorphing,decorations.markings,arrows.meta}

% Color scheme
\definecolor{headcolor}{RGB}{0,102,204}
\definecolor{keycolor}{RGB}{220,20,60}
\definecolor{solutioncolor}{RGB}{34,139,34}
\definecolor{mnemoniccolor}{RGB}{148,0,211}
\definecolor{codecolor}{RGB}{0,0,100}

% Spacing
\setlength{\parskip}{3pt}
\setlist[itemize]{nosep}
\setlist[enumerate]{nosep}

% Title formatting
\titleformat{\section}{\Large\bfseries\color{headcolor}}{\thesection}{1em}{}
\titleformat{\subsection}{\large\bfseries\color{headcolor}}{\thesubsection}{1em}{}

% Pandoc tightlist compatibility
\providecommand{\tightlist}{%
  \setlength{\itemsep}{0pt}\setlength{\parskip}{0pt}}

% Pandoc longtable compatibility
\newcounter{none}
\def\thenone{}


% content/resources/templates/english-boxes.tex
% This file is currently empty - it exists to maintain consistency with the import structure.
% Add custom environments here if needed in the future.


\begin{document}

\begin{center}
{\Huge\bfseries\color{headcolor} Subject Name Solutions}\\[5pt]
{\LARGE 4331604 -- Winter 2023}\\[3pt]
{\large Semester 1 Study Material}\\[3pt]
{\normalsize\textit{Detailed Solutions and Explanations}}
\end{center}

\vspace{10pt}

\subsection*{Question 1(a) [3 marks]}\label{q1a}

\textbf{Define Software and explain its characteristics.}

\begin{solutionbox}

Software is a collection of programs, instructions, and documentation
that performs tasks on a computer system.

\textbf{Key Characteristics:}

{\def\LTcaptype{none} % do not increment counter
\begin{longtable}[]{@{}ll@{}}
\toprule\noalign{}
Characteristic & Description \\
\midrule\noalign{}
\endhead
\bottomrule\noalign{}
\endlastfoot
\textbf{Intangible} & Cannot be touched physically \\
\textbf{Logical} & Created through systematic approach \\
\textbf{Manufactured} & Developed, not produced traditionally \\
\textbf{Complex} & Has intricate internal structure \\
\end{longtable}
}

\end{solutionbox}
\begin{mnemonicbox}
``In Logic, Manufacturing Creates'' (Intangible,
Logical, Manufactured, Complex)

\end{mnemonicbox}
\subsection*{Question 1(b) [4 marks]}\label{q1b}

\textbf{Write a note on Software engineering -- A layered technology.}

\begin{solutionbox}

Software engineering is structured as a layered technology with each
layer supporting the next.

\textbf{Layered Structure:}

{\def\LTcaptype{none} % do not increment counter
\begin{longtable}[]{@{}
  >{\raggedright\arraybackslash}p{(\linewidth - 4\tabcolsep) * \real{0.3333}}
  >{\raggedright\arraybackslash}p{(\linewidth - 4\tabcolsep) * \real{0.3333}}
  >{\raggedright\arraybackslash}p{(\linewidth - 4\tabcolsep) * \real{0.3333}}@{}}
\toprule\noalign{}
\begin{minipage}[b]{\linewidth}\raggedright
Layer
\end{minipage} & \begin{minipage}[b]{\linewidth}\raggedright
Purpose
\end{minipage} & \begin{minipage}[b]{\linewidth}\raggedright
Description
\end{minipage} \\
\midrule\noalign{}
\endhead
\bottomrule\noalign{}
\endlastfoot
\textbf{Quality Focus} & Foundation & Emphasis on delivering quality
products \\
\textbf{Process} & Framework & Defines how software development is
done \\
\textbf{Methods} & Techniques & Specific ways to perform activities \\
\textbf{Tools} & Automation & Software that supports methods \\
\end{longtable}
}

\begin{center}
\textbf{Mermaid Diagram (Code)}
\begin{verbatim}
{Shaded}
{Highlighting}[]
graph LR
    A[Tools] {-{-}{} B[Methods]}
    B {-{-}{} C[Process]}
    C {-{-}{} D[Quality Focus {-} Foundation]}
{Highlighting}
{Shaded}
\end{verbatim}
\end{center}

\end{solutionbox}
\begin{mnemonicbox}
``Tools Make Process Quality'' (Tools, Methods,
Process, Quality)

\end{mnemonicbox}
\subsection*{Question 1(c) [7 marks]}\label{q1c}

\textbf{Explain Software Process framework and umbrella activities.}

\begin{solutionbox}

Software Process Framework provides structure for software development
with core activities and umbrella activities.

\textbf{Framework Activities:}

{\def\LTcaptype{none} % do not increment counter
\begin{longtable}[]{@{}
  >{\raggedright\arraybackslash}p{(\linewidth - 4\tabcolsep) * \real{0.3333}}
  >{\raggedright\arraybackslash}p{(\linewidth - 4\tabcolsep) * \real{0.3333}}
  >{\raggedright\arraybackslash}p{(\linewidth - 4\tabcolsep) * \real{0.3333}}@{}}
\toprule\noalign{}
\begin{minipage}[b]{\linewidth}\raggedright
Activity
\end{minipage} & \begin{minipage}[b]{\linewidth}\raggedright
Purpose
\end{minipage} & \begin{minipage}[b]{\linewidth}\raggedright
Key Tasks
\end{minipage} \\
\midrule\noalign{}
\endhead
\bottomrule\noalign{}
\endlastfoot
\textbf{Communication} & Understand requirements & Stakeholder
interaction, requirement gathering \\
\textbf{Planning} & Create roadmap & Estimation, scheduling, risk
assessment \\
\textbf{Modeling} & Create blueprints & Analysis and design models \\
\textbf{Construction} & Build software & Coding and testing \\
\textbf{Deployment} & Deliver to users & Installation, support,
feedback \\
\end{longtable}
}

\textbf{Umbrella Activities:}

\begin{itemize}
\tightlist
\item
  \textbf{Software project tracking}: Monitor progress and control
  quality
\item
  \textbf{Risk management}: Identify and mitigate potential problems
\item
  \textbf{Quality assurance}: Ensure standards are met
\item
  \textbf{Configuration management}: Control changes systematically
\item
  \textbf{Work product preparation}: Create deliverable documents
\end{itemize}

\begin{center}
\textbf{Mermaid Diagram (Code)}
\begin{verbatim}
{Shaded}
{Highlighting}[]
graph LR
    A[Communication] {-{-}{} B[Planning]}
    B {-{-}{} C[Modeling]}
    C {-{-}{} D[Construction]}
    D {-{-}{} E[Deployment]}
    F[Umbrella Activities] {-.{-}{} A}
    F {-.{-}{} B}
    F {-.{-}{} C}
    F {-.{-}{} D}
    F {-.{-}{} E}
{Highlighting}
{Shaded}
\end{verbatim}
\end{center}

\end{solutionbox}
\begin{mnemonicbox}
``Can People Model Construction Daily'' + ``Track
Risk Quality Configuration Work''

\end{mnemonicbox}
\subsection*{Question 1(c OR) [7
marks]}\label{question-1c-or-7-marks}

\textbf{Define SDLC and explain each phase.}

\begin{solutionbox}

SDLC (Software Development Life Cycle) is a systematic process for
developing software applications.

\textbf{SDLC Phases:}

{\def\LTcaptype{none} % do not increment counter
\begin{longtable}[]{@{}
  >{\raggedright\arraybackslash}p{(\linewidth - 6\tabcolsep) * \real{0.2500}}
  >{\raggedright\arraybackslash}p{(\linewidth - 6\tabcolsep) * \real{0.2500}}
  >{\raggedright\arraybackslash}p{(\linewidth - 6\tabcolsep) * \real{0.2500}}
  >{\raggedright\arraybackslash}p{(\linewidth - 6\tabcolsep) * \real{0.2500}}@{}}
\toprule\noalign{}
\begin{minipage}[b]{\linewidth}\raggedright
Phase
\end{minipage} & \begin{minipage}[b]{\linewidth}\raggedright
Purpose
\end{minipage} & \begin{minipage}[b]{\linewidth}\raggedright
Key Activities
\end{minipage} & \begin{minipage}[b]{\linewidth}\raggedright
Deliverables
\end{minipage} \\
\midrule\noalign{}
\endhead
\bottomrule\noalign{}
\endlastfoot
\textbf{Planning} & Define scope & Feasibility study, resource
allocation & Project plan \\
\textbf{Analysis} & Gather requirements & Requirement collection,
documentation & SRS document \\
\textbf{Design} & Create architecture & System design, database design &
Design documents \\
\textbf{Implementation} & Write code & Programming, unit testing &
Source code \\
\textbf{Testing} & Verify quality & System testing, bug fixing & Test
reports \\
\textbf{Deployment} & Release software & Installation, user training &
Live system \\
\textbf{Maintenance} & Ongoing support & Bug fixes, enhancements &
Updated system \\
\end{longtable}
}

\begin{center}
\textbf{Mermaid Diagram (Code)}
\begin{verbatim}
{Shaded}
{Highlighting}[]
graph LR
    A[Planning] {-{-}{} B[Analysis]}
    B {-{-}{} C[Design]}
    C {-{-}{} D[Implementation]}
    D {-{-}{} E[Testing]}
    E {-{-}{} F[Deployment]}
    F {-{-}{} G[Maintenance]}
{Highlighting}
{Shaded}
\end{verbatim}
\end{center}

\end{solutionbox}
\begin{mnemonicbox}
``Please Analyze Design Implementation Testing
Deployment Maintenance''

\end{mnemonicbox}
\subsection*{Question 2(a) [3 marks]}\label{q2a}

\textbf{Describe advantage disadvantage of prototype model.}

\begin{solutionbox}

\textbf{Prototype Model Analysis:}

{\def\LTcaptype{none} % do not increment counter
\begin{longtable}[]{@{}
  >{\raggedright\arraybackslash}p{(\linewidth - 2\tabcolsep) * \real{0.5000}}
  >{\raggedright\arraybackslash}p{(\linewidth - 2\tabcolsep) * \real{0.5000}}@{}}
\toprule\noalign{}
\begin{minipage}[b]{\linewidth}\raggedright
Advantages
\end{minipage} & \begin{minipage}[b]{\linewidth}\raggedright
Disadvantages
\end{minipage} \\
\midrule\noalign{}
\endhead
\bottomrule\noalign{}
\endlastfoot
\textbf{Early feedback} from users & \textbf{Time consuming} development
process \\
\textbf{Reduced risk} of failure & \textbf{Cost increase} due to
iterations \\
\textbf{Better understanding} of requirements & \textbf{Scope creep} may
occur \\
\end{longtable}
}

\end{solutionbox}
\begin{mnemonicbox}
``Early Reduced Better'' vs ``Time Cost Scope''

\end{mnemonicbox}
\subsection*{Question 2(b) [4 marks]}\label{q2b}

\textbf{Explain Prototyping Model and justify when to use with example.}

\begin{solutionbox}

Prototyping Model creates working model of software early in development
process.

\textbf{When to Use:}

{\def\LTcaptype{none} % do not increment counter
\begin{longtable}[]{@{}
  >{\raggedright\arraybackslash}p{(\linewidth - 4\tabcolsep) * \real{0.3333}}
  >{\raggedright\arraybackslash}p{(\linewidth - 4\tabcolsep) * \real{0.3333}}
  >{\raggedright\arraybackslash}p{(\linewidth - 4\tabcolsep) * \real{0.3333}}@{}}
\toprule\noalign{}
\begin{minipage}[b]{\linewidth}\raggedright
Situation
\end{minipage} & \begin{minipage}[b]{\linewidth}\raggedright
Example
\end{minipage} & \begin{minipage}[b]{\linewidth}\raggedright
Justification
\end{minipage} \\
\midrule\noalign{}
\endhead
\bottomrule\noalign{}
\endlastfoot
\textbf{Unclear requirements} & Online shopping cart & User interface
needs refinement \\
\textbf{New technology} & Mobile banking app & Feasibility testing
required \\
\textbf{User interaction critical} & Gaming application & User
experience validation needed \\
\end{longtable}
}

\begin{center}
\textbf{Mermaid Diagram (Code)}
\begin{verbatim}
{Shaded}
{Highlighting}[]
graph LR
    A[Requirements] {-{-}{} B[Quick Design]}
    B {-{-}{} C[Build Prototype]}
    C {-{-}{} D[User Evaluation]}
    D {-{-}{} E\{Satisfied?\}}
    E {-{-}{}|No| B}
    E {-{-}{}|Yes| F[Final System]}
{Highlighting}
{Shaded}
\end{verbatim}
\end{center}

\end{solutionbox}
\begin{mnemonicbox}
``Requirements Quick Build User Satisfied Final''

\end{mnemonicbox}
\subsection*{Question 2(c) [7 marks]}\label{q2c}

\textbf{Sketch and discuss (I) Waterfall model \& (II) Incremental
Model.}

\begin{solutionbox}

\textbf{(I) Waterfall Model:}

Linear sequential approach where each phase must complete before next
begins.

\begin{center}
\textbf{Mermaid Diagram (Code)}
\begin{verbatim}
{Shaded}
{Highlighting}[]
graph LR
    A[Requirements Analysis] {-{-}{} B[System Design]}
    B {-{-}{} C[Implementation]}
    C {-{-}{} D[Testing]}
    D {-{-}{} E[Deployment]}
    E {-{-}{} F[Maintenance]}
{Highlighting}
{Shaded}
\end{verbatim}
\end{center}

{\def\LTcaptype{none} % do not increment counter
\begin{longtable}[]{@{}ll@{}}
\toprule\noalign{}
Characteristics & Description \\
\midrule\noalign{}
\endhead
\bottomrule\noalign{}
\endlastfoot
\textbf{Sequential} & One phase at a time \\
\textbf{Documentation driven} & Heavy documentation \\
\textbf{Suitable for} & Well-defined requirements \\
\end{longtable}
}

\textbf{(II) Incremental Model:}

Development in small increments with each increment adding
functionality.

\begin{center}
\textbf{Mermaid Diagram (Code)}
\begin{verbatim}
{Shaded}
{Highlighting}[]
graph LR
    A[Analysis] {-{-}{} B[Design]}
    B {-{-}{} C[Code]}
    C {-{-}{} D[Test]}
    D {-{-}{} E[Increment 1]}
    
    F[Analysis] {-{-}{} G[Design]}
    G {-{-}{} H[Code]}
    H {-{-}{} I[Test]}
    I {-{-}{} J[Increment 2]}
    
    E {-{-}{} K[Final Product]}
    J {-{-}{} K}
{Highlighting}
{Shaded}
\end{verbatim}
\end{center}

{\def\LTcaptype{none} % do not increment counter
\begin{longtable}[]{@{}lll@{}}
\toprule\noalign{}
Feature & Waterfall & Incremental \\
\midrule\noalign{}
\endhead
\bottomrule\noalign{}
\endlastfoot
\textbf{Flexibility} & Low & High \\
\textbf{Risk} & High & Low \\
\textbf{Delivery} & End of project & Multiple deliveries \\
\end{longtable}
}

\end{solutionbox}
\begin{mnemonicbox}
``Water Falls Once, Increments Build Multiple''

\end{mnemonicbox}
\subsection*{Question 2(a OR) [3
marks]}\label{question-2a-or-3-marks}

\textbf{Describe advantage and disadvantage of Incremental Model.}

\begin{solutionbox}

\textbf{Incremental Model Analysis:}

{\def\LTcaptype{none} % do not increment counter
\begin{longtable}[]{@{}
  >{\raggedright\arraybackslash}p{(\linewidth - 2\tabcolsep) * \real{0.5000}}
  >{\raggedright\arraybackslash}p{(\linewidth - 2\tabcolsep) * \real{0.5000}}@{}}
\toprule\noalign{}
\begin{minipage}[b]{\linewidth}\raggedright
Advantages
\end{minipage} & \begin{minipage}[b]{\linewidth}\raggedright
Disadvantages
\end{minipage} \\
\midrule\noalign{}
\endhead
\bottomrule\noalign{}
\endlastfoot
\textbf{Early delivery} of working software & \textbf{Total cost} may be
higher \\
\textbf{Easier testing} of small increments & \textbf{System
architecture} issues \\
\textbf{Reduced risk} through early feedback & \textbf{Management
complexity} increases \\
\end{longtable}
}

\end{solutionbox}
\begin{mnemonicbox}
``Early Easier Reduced'' vs ``Total System
Management''

\end{mnemonicbox}
\subsection*{Question 2(b OR) [4
marks]}\label{question-2b-or-4-marks}

\textbf{Write concept of Rapid Application Development (RAD) and explain
it.}

\begin{solutionbox}

RAD emphasizes rapid prototyping and quick feedback over planning and
testing.

\textbf{RAD Components:}

{\def\LTcaptype{none} % do not increment counter
\begin{longtable}[]{@{}
  >{\raggedright\arraybackslash}p{(\linewidth - 6\tabcolsep) * \real{0.2500}}
  >{\raggedright\arraybackslash}p{(\linewidth - 6\tabcolsep) * \real{0.2500}}
  >{\raggedright\arraybackslash}p{(\linewidth - 6\tabcolsep) * \real{0.2500}}
  >{\raggedright\arraybackslash}p{(\linewidth - 6\tabcolsep) * \real{0.2500}}@{}}
\toprule\noalign{}
\begin{minipage}[b]{\linewidth}\raggedright
Phase
\end{minipage} & \begin{minipage}[b]{\linewidth}\raggedright
Duration
\end{minipage} & \begin{minipage}[b]{\linewidth}\raggedright
Activities
\end{minipage} & \begin{minipage}[b]{\linewidth}\raggedright
Output
\end{minipage} \\
\midrule\noalign{}
\endhead
\bottomrule\noalign{}
\endlastfoot
\textbf{Business Modeling} & Short & Define information flow & Business
requirements \\
\textbf{Data Modeling} & Short & Define data objects & Data models \\
\textbf{Process Modeling} & Short & Define processing functions &
Process descriptions \\
\textbf{Application Generation} & Short & Use tools to create & Working
application \\
\textbf{Testing \& Turnover} & Short & Test and deliver & Final
system \\
\end{longtable}
}

\begin{center}
\textbf{Mermaid Diagram (Code)}
\begin{verbatim}
{Shaded}
{Highlighting}[]
graph LR
    A[Business Modeling] {-{-}{} B[Data Modeling]}
    B {-{-}{} C[Process Modeling]}
    C {-{-}{} D[Application Generation]}
    D {-{-}{} E[Testing \& Turnover]}
{Highlighting}
{Shaded}
\end{verbatim}
\end{center}

\end{solutionbox}
\begin{mnemonicbox}
``Business Data Process Application Testing''

\end{mnemonicbox}
\subsection*{Question 2(c OR) [7
marks]}\label{question-2c-or-7-marks}

\textbf{Design and describe Spiral Model and give advantage and
disadvantage.}

\begin{solutionbox}

Spiral Model combines iterative development with systematic risk
analysis.

\begin{center}
\textbf{Mermaid Diagram (Code)}
\begin{verbatim}
{Shaded}
{Highlighting}[]
graph LR
    A[Planning] {-{-}{} B[Risk Analysis]}
    B {-{-}{} C[Engineering]}
    C {-{-}{} D[Evaluation]}
    D {-{-}{} A}
    
    E[Determine Objectives] {-{-}{} F[Identify Risks]}
    F {-{-}{} G[Develop \& Test]}
    G {-{-}{} H[Plan Next Iteration]}
    H {-{-}{} E}
{Highlighting}
{Shaded}
\end{verbatim}
\end{center}

\textbf{Spiral Quadrants:}

{\def\LTcaptype{none} % do not increment counter
\begin{longtable}[]{@{}
  >{\raggedright\arraybackslash}p{(\linewidth - 4\tabcolsep) * \real{0.3333}}
  >{\raggedright\arraybackslash}p{(\linewidth - 4\tabcolsep) * \real{0.3333}}
  >{\raggedright\arraybackslash}p{(\linewidth - 4\tabcolsep) * \real{0.3333}}@{}}
\toprule\noalign{}
\begin{minipage}[b]{\linewidth}\raggedright
Quadrant
\end{minipage} & \begin{minipage}[b]{\linewidth}\raggedright
Activity
\end{minipage} & \begin{minipage}[b]{\linewidth}\raggedright
Purpose
\end{minipage} \\
\midrule\noalign{}
\endhead
\bottomrule\noalign{}
\endlastfoot
\textbf{Planning} & Objective setting & Define requirements and
constraints \\
\textbf{Risk Analysis} & Risk assessment & Identify and resolve risks \\
\textbf{Engineering} & Development & Build and test the product \\
\textbf{Evaluation} & Customer assessment & Evaluate results and plan
next iteration \\
\end{longtable}
}

\textbf{Advantages vs Disadvantages:}

{\def\LTcaptype{none} % do not increment counter
\begin{longtable}[]{@{}
  >{\raggedright\arraybackslash}p{(\linewidth - 2\tabcolsep) * \real{0.5000}}
  >{\raggedright\arraybackslash}p{(\linewidth - 2\tabcolsep) * \real{0.5000}}@{}}
\toprule\noalign{}
\begin{minipage}[b]{\linewidth}\raggedright
Advantages
\end{minipage} & \begin{minipage}[b]{\linewidth}\raggedright
Disadvantages
\end{minipage} \\
\midrule\noalign{}
\endhead
\bottomrule\noalign{}
\endlastfoot
\textbf{High risk projects} handled well & \textbf{Complex management}
required \\
\textbf{Good for large} applications & \textbf{Expensive} for small
projects \\
\textbf{Customer involved} throughout & \textbf{Risk analysis expertise}
needed \\
\end{longtable}
}

\end{solutionbox}
\begin{mnemonicbox}
``Plan Risk Engineer Evaluate'' + ``High Good
Customer'' vs ``Complex Expensive Risk''

\end{mnemonicbox}
\subsection*{Question 3(a) [3 marks]}\label{q3a}

\textbf{Illustrate importance of SRS}

\begin{solutionbox}

SRS (Software Requirements Specification) is crucial foundation document
for software development.

\textbf{Importance Table:}

{\def\LTcaptype{none} % do not increment counter
\begin{longtable}[]{@{}lll@{}}
\toprule\noalign{}
Aspect & Importance & Benefit \\
\midrule\noalign{}
\endhead
\bottomrule\noalign{}
\endlastfoot
\textbf{Communication} & Stakeholder understanding & Clear
expectations \\
\textbf{Contract} & Legal agreement & Dispute resolution \\
\textbf{Testing basis} & Validation criteria & Quality assurance \\
\end{longtable}
}

\end{solutionbox}
\begin{mnemonicbox}
``Communication Contract Testing''

\end{mnemonicbox}
\subsection*{Question 3(b) [4 marks]}\label{q3b}

\textbf{Specify characteristics of good \& bad SRS}

\begin{solutionbox}

\textbf{SRS Quality Characteristics:}

{\def\LTcaptype{none} % do not increment counter
\begin{longtable}[]{@{}
  >{\raggedright\arraybackslash}p{(\linewidth - 2\tabcolsep) * \real{0.5000}}
  >{\raggedright\arraybackslash}p{(\linewidth - 2\tabcolsep) * \real{0.5000}}@{}}
\toprule\noalign{}
\begin{minipage}[b]{\linewidth}\raggedright
Good SRS
\end{minipage} & \begin{minipage}[b]{\linewidth}\raggedright
Bad SRS
\end{minipage} \\
\midrule\noalign{}
\endhead
\bottomrule\noalign{}
\endlastfoot
\textbf{Complete} - All requirements covered & \textbf{Incomplete} -
Missing requirements \\
\textbf{Consistent} - No contradictions & \textbf{Inconsistent} -
Conflicting statements \\
\textbf{Unambiguous} - Clear meaning & \textbf{Ambiguous} - Multiple
interpretations \\
\textbf{Verifiable} - Can be tested & \textbf{Unverifiable} - Cannot be
validated \\
\end{longtable}
}

\textbf{Additional Good Characteristics:}

\begin{itemize}
\tightlist
\item
  \textbf{Modifiable}: Easy to change and maintain
\item
  \textbf{Traceable}: Links to source and design
\end{itemize}

\begin{center}
\textbf{Mermaid Diagram (Code)}
\begin{verbatim}
{Shaded}
{Highlighting}[]
graph TD
    A[Good SRS] {-{-}{} B[Complete]}
    A {-{-}{} C[Consistent]}
    A {-{-}{} D[Unambiguous]}
    A {-{-}{} E[Verifiable]}
    
    F[Bad SRS] {-{-}{} G[Incomplete]}
    F {-{-}{} H[Inconsistent]}
    F {-{-}{} I[Ambiguous]}
    F {-{-}{} J[Unverifiable]}
{Highlighting}
{Shaded}
\end{verbatim}
\end{center}

\end{solutionbox}
\begin{mnemonicbox}
``Complete Consistent Unambiguous Verifiable'' vs
``Incomplete Inconsistent Ambiguous Unverifiable''

\end{mnemonicbox}
\subsection*{Question 3(c) [7 marks]}\label{q3c}

\textbf{Classify Types of Requirements in SRS}

\begin{solutionbox}

Software requirements are classified into two main categories.

\textbf{(i) Functional Requirements:}

Define what the system should do - specific behaviors and functions.

{\def\LTcaptype{none} % do not increment counter
\begin{longtable}[]{@{}
  >{\raggedright\arraybackslash}p{(\linewidth - 4\tabcolsep) * \real{0.3333}}
  >{\raggedright\arraybackslash}p{(\linewidth - 4\tabcolsep) * \real{0.3333}}
  >{\raggedright\arraybackslash}p{(\linewidth - 4\tabcolsep) * \real{0.3333}}@{}}
\toprule\noalign{}
\begin{minipage}[b]{\linewidth}\raggedright
Type
\end{minipage} & \begin{minipage}[b]{\linewidth}\raggedright
Description
\end{minipage} & \begin{minipage}[b]{\linewidth}\raggedright
Example
\end{minipage} \\
\midrule\noalign{}
\endhead
\bottomrule\noalign{}
\endlastfoot
\textbf{Business Rules} & Core business logic & ``Calculate tax based on
income bracket'' \\
\textbf{User Actions} & System responses & ``Login with
username/password'' \\
\textbf{Data Processing} & Information handling & ``Generate monthly
sales report'' \\
\textbf{External Interfaces} & System interactions & ``Connect to
payment gateway'' \\
\end{longtable}
}

\textbf{(ii) Non-functional Requirements:}

Define how the system should perform - quality attributes and
constraints.

{\def\LTcaptype{none} % do not increment counter
\begin{longtable}[]{@{}
  >{\raggedright\arraybackslash}p{(\linewidth - 6\tabcolsep) * \real{0.2500}}
  >{\raggedright\arraybackslash}p{(\linewidth - 6\tabcolsep) * \real{0.2500}}
  >{\raggedright\arraybackslash}p{(\linewidth - 6\tabcolsep) * \real{0.2500}}
  >{\raggedright\arraybackslash}p{(\linewidth - 6\tabcolsep) * \real{0.2500}}@{}}
\toprule\noalign{}
\begin{minipage}[b]{\linewidth}\raggedright
Category
\end{minipage} & \begin{minipage}[b]{\linewidth}\raggedright
Requirement
\end{minipage} & \begin{minipage}[b]{\linewidth}\raggedright
Example
\end{minipage} & \begin{minipage}[b]{\linewidth}\raggedright
Measurement
\end{minipage} \\
\midrule\noalign{}
\endhead
\bottomrule\noalign{}
\endlastfoot
\textbf{Performance} & Response time & ``Page load \textless{} 3
seconds'' & Time metrics \\
\textbf{Security} & Data protection & ``Encrypt user passwords'' &
Security standards \\
\textbf{Reliability} & System uptime & ``99.9\% availability'' & Failure
rates \\
\textbf{Usability} & User experience & ``Max 3 clicks to checkout'' &
User metrics \\
\textbf{Scalability} & Growth capacity & ``Support 10,000 users'' & Load
capacity \\
\end{longtable}
}

\begin{center}
\textbf{Mermaid Diagram (Code)}
\begin{verbatim}
{Shaded}
{Highlighting}[]
graph TD
    A[Requirements] {-{-}{} B[Functional]}
    A {-{-}{} C[Non{-}Functional]}
    
    B {-{-}{} D[Business Rules]}
    B {-{-}{} E[User Actions]}
    B {-{-}{} F[Data Processing]}
    B {-{-}{} G[External Interfaces]}
    
    C {-{-}{} H[Performance]}
    C {-{-}{} I[Security]}
    C {-{-}{} J[Reliability]}
    C {-{-}{} K[Usability]}
    C {-{-}{} L[Scalability]}
{Highlighting}
{Shaded}
\end{verbatim}
\end{center}

\textbf{Comparison Table:}

{\def\LTcaptype{none} % do not increment counter
\begin{longtable}[]{@{}lll@{}}
\toprule\noalign{}
Aspect & Functional & Non-Functional \\
\midrule\noalign{}
\endhead
\bottomrule\noalign{}
\endlastfoot
\textbf{Focus} & What system does & How system performs \\
\textbf{Testing} & Black-box testing & Performance testing \\
\textbf{Documentation} & Use cases & Quality metrics \\
\end{longtable}
}

\end{solutionbox}
\begin{mnemonicbox}
``Functional = What, Non-Functional = How''

\end{mnemonicbox}
\subsection*{Question 3(a OR) [3
marks]}\label{question-3a-or-3-marks}

\textbf{Describe skill to manage software projects}

\begin{solutionbox}

Project management requires diverse skill set for successful software
delivery.

\textbf{Essential Skills:}

{\def\LTcaptype{none} % do not increment counter
\begin{longtable}[]{@{}lll@{}}
\toprule\noalign{}
Skill Category & Description & Application \\
\midrule\noalign{}
\endhead
\bottomrule\noalign{}
\endlastfoot
\textbf{Technical} & Understanding technology & Architecture
decisions \\
\textbf{Leadership} & Team motivation & Conflict resolution \\
\textbf{Communication} & Stakeholder interaction & Status reporting \\
\end{longtable}
}

\end{solutionbox}
\begin{mnemonicbox}
``Technical Leadership Communication''

\end{mnemonicbox}
\subsection*{Question 3(b OR) [4
marks]}\label{question-3b-or-4-marks}

\textbf{Briefly give the Responsibility of software project Manager.}

\begin{solutionbox}

Software Project Manager oversees entire project lifecycle and ensures
successful delivery.

\textbf{Key Responsibilities:}

{\def\LTcaptype{none} % do not increment counter
\begin{longtable}[]{@{}
  >{\raggedright\arraybackslash}p{(\linewidth - 4\tabcolsep) * \real{0.3333}}
  >{\raggedright\arraybackslash}p{(\linewidth - 4\tabcolsep) * \real{0.3333}}
  >{\raggedright\arraybackslash}p{(\linewidth - 4\tabcolsep) * \real{0.3333}}@{}}
\toprule\noalign{}
\begin{minipage}[b]{\linewidth}\raggedright
Area
\end{minipage} & \begin{minipage}[b]{\linewidth}\raggedright
Responsibility
\end{minipage} & \begin{minipage}[b]{\linewidth}\raggedright
Activities
\end{minipage} \\
\midrule\noalign{}
\endhead
\bottomrule\noalign{}
\endlastfoot
\textbf{Planning} & Project roadmap & Schedule, budget, resource
allocation \\
\textbf{Execution} & Team coordination & Task assignment, progress
monitoring \\
\textbf{Quality} & Standard compliance & Code reviews, testing
oversight \\
\textbf{Communication} & Stakeholder updates & Status reports, risk
communication \\
\end{longtable}
}

\textbf{Additional Duties:}

\begin{itemize}
\tightlist
\item
  \textbf{Risk Management}: Identify and mitigate project risks
\item
  \textbf{Team Development}: Mentor team members and resolve conflicts
\end{itemize}

\begin{center}
\textbf{Mermaid Diagram (Code)}
\begin{verbatim}
{Shaded}
{Highlighting}[]
graph TD
    A[Project Manager] {-{-}{} B[Planning]}
    A {-{-}{} C[Execution]}
    A {-{-}{} D[Quality]}
    A {-{-}{} E[Communication]}
    A {-{-}{} F[Risk Management]}
    A {-{-}{} G[Team Development]}
{Highlighting}
{Shaded}
\end{verbatim}
\end{center}

\end{solutionbox}
\begin{mnemonicbox}
``Plan Execute Quality Communicate Risk Team''

\end{mnemonicbox}
\subsection*{Question 3(c OR) [7
marks]}\label{question-3c-or-7-marks}

\textbf{Compare PERT chart -- Gantt chart side by side.}

\begin{solutionbox}

Both charts are project management tools but serve different purposes
and have distinct characteristics.

\textbf{Detailed Comparison:}

{\def\LTcaptype{none} % do not increment counter
\begin{longtable}[]{@{}lll@{}}
\toprule\noalign{}
Aspect & PERT Chart & Gantt Chart \\
\midrule\noalign{}
\endhead
\bottomrule\noalign{}
\endlastfoot
\textbf{Purpose} & Show task dependencies & Show project timeline \\
\textbf{Structure} & Network diagram & Bar chart \\
\textbf{Focus} & Critical path analysis & Schedule visualization \\
\textbf{Time Display} & Estimated durations & Actual dates \\
\textbf{Dependencies} & Explicit arrows & Implicit connections \\
\textbf{Best For} & Complex projects & Simple scheduling \\
\end{longtable}
}

\textbf{Visual Representation:}

\begin{center}
\textbf{Mermaid Diagram (Code)}
\begin{verbatim}
{Shaded}
{Highlighting}[]
graph TD
    subgraph "PERT Chart"
      direction LR
        A[Task A] {-{-}{} C[Task C]}
        B[Task B] {-{-}{} C}
        C {-{-}{} D[Task D]}
    end
{Highlighting}
{Shaded}
\end{verbatim}
\end{center}

\begin{verbatim}
gantt
    title Gantt Chart
    dateFormat  YYYY{-MM{-}DD}
    section Development
    Task A     :a1, 2024{-01{-}01, 3d}
    Task B     :a2, 2024{-01{-}01, 2d}
    Task C     :a3, after a1 a2, 4d
    Task D     :a4, after a3, 2d
\end{verbatim}

\textbf{When to Use:}

{\def\LTcaptype{none} % do not increment counter
\begin{longtable}[]{@{}lll@{}}
\toprule\noalign{}
Scenario & PERT & Gantt \\
\midrule\noalign{}
\endhead
\bottomrule\noalign{}
\endlastfoot
\textbf{Project Type} & Research \& Development & Construction,
Software \\
\textbf{Uncertainty} & High uncertainty & Well-defined tasks \\
\textbf{Audience} & Technical team & Management, Clients \\
\end{longtable}
}

\textbf{Advantages Comparison:}

{\def\LTcaptype{none} % do not increment counter
\begin{longtable}[]{@{}ll@{}}
\toprule\noalign{}
PERT Advantages & Gantt Advantages \\
\midrule\noalign{}
\endhead
\bottomrule\noalign{}
\endlastfoot
\textbf{Critical path} identification & \textbf{Easy to understand}
visually \\
\textbf{Flexible timing} estimates & \textbf{Progress tracking}
capability \\
\textbf{Risk analysis} support & \textbf{Resource allocation} display \\
\end{longtable}
}

\end{solutionbox}
\begin{mnemonicbox}
``PERT = Path, Gantt = Bars''

\end{mnemonicbox}
\subsection*{Question 4(a) [3 marks]}\label{q4a}

\textbf{Give steps of Project Monitoring and control process}

\begin{solutionbox}

Project monitoring ensures project stays on track through systematic
observation and corrective actions.

\textbf{Monitoring Steps:}

{\def\LTcaptype{none} % do not increment counter
\begin{longtable}[]{@{}lll@{}}
\toprule\noalign{}
Step & Activity & Purpose \\
\midrule\noalign{}
\endhead
\bottomrule\noalign{}
\endlastfoot
\textbf{Track Progress} & Measure actual vs planned & Identify
deviations \\
\textbf{Assess Quality} & Review deliverables & Ensure standards \\
\textbf{Take Action} & Implement corrections & Maintain alignment \\
\end{longtable}
}

\end{solutionbox}
\begin{mnemonicbox}
``Track Assess Take''

\end{mnemonicbox}
\subsection*{Question 4(b) [4 marks]}\label{q4b}

\textbf{Discuss i)Risk Assessment ii)Risk Mitigation}

\begin{solutionbox}

\textbf{(i) Risk Assessment:}

Process of identifying and evaluating potential project risks.

{\def\LTcaptype{none} % do not increment counter
\begin{longtable}[]{@{}lll@{}}
\toprule\noalign{}
Assessment Type & Method & Output \\
\midrule\noalign{}
\endhead
\bottomrule\noalign{}
\endlastfoot
\textbf{Risk Identification} & Brainstorming, checklists & Risk list \\
\textbf{Risk Analysis} & Probability \times Impact & Risk priority \\
\textbf{Risk Evaluation} & Risk matrix & Action priorities \\
\end{longtable}
}

\textbf{(ii) Risk Mitigation:}

Strategies to reduce risk impact and probability.

{\def\LTcaptype{none} % do not increment counter
\begin{longtable}[]{@{}lll@{}}
\toprule\noalign{}
Strategy & Description & Example \\
\midrule\noalign{}
\endhead
\bottomrule\noalign{}
\endlastfoot
\textbf{Avoidance} & Eliminate risk source & Change technology \\
\textbf{Reduction} & Minimize impact & Add testing \\
\textbf{Transfer} & Shift risk to others & Insurance, outsourcing \\
\textbf{Acceptance} & Live with risk & Contingency planning \\
\end{longtable}
}

\end{solutionbox}
\begin{mnemonicbox}
``Avoid Reduce Transfer Accept''

\end{mnemonicbox}
\subsection*{Question 4(c) [7 marks]}\label{q4c}

\textbf{Define project risk and how Manage Risk Management it.}

\begin{solutionbox}

Project Risk is an uncertain event that, if occurs, has positive or
negative effect on project objectives.

\textbf{Risk Characteristics:}

{\def\LTcaptype{none} % do not increment counter
\begin{longtable}[]{@{}lll@{}}
\toprule\noalign{}
Characteristic & Description & Example \\
\midrule\noalign{}
\endhead
\bottomrule\noalign{}
\endlastfoot
\textbf{Uncertainty} & May or may not occur & Technology failure \\
\textbf{Impact} & Affects project parameters & Cost, schedule,
quality \\
\textbf{Probability} & Likelihood of occurrence & 30\% chance of
delay \\
\end{longtable}
}

\textbf{Risk Management Process:}

\begin{center}
\textbf{Mermaid Diagram (Code)}
\begin{verbatim}
{Shaded}
{Highlighting}[]
graph LR
    A[Risk Identification] {-{-}{} B[Risk Assessment]}
    B {-{-}{} C[Risk Prioritization]}
    C {-{-}{} D[Risk Response Planning]}
    D {-{-}{} E[Risk Monitoring]}
    E {-{-}{} F[Risk Control]}
    F {-{-}{} A}
{Highlighting}
{Shaded}
\end{verbatim}
\end{center}

\textbf{Risk Management Steps:}

{\def\LTcaptype{none} % do not increment counter
\begin{longtable}[]{@{}
  >{\raggedright\arraybackslash}p{(\linewidth - 6\tabcolsep) * \real{0.2500}}
  >{\raggedright\arraybackslash}p{(\linewidth - 6\tabcolsep) * \real{0.2500}}
  >{\raggedright\arraybackslash}p{(\linewidth - 6\tabcolsep) * \real{0.2500}}
  >{\raggedright\arraybackslash}p{(\linewidth - 6\tabcolsep) * \real{0.2500}}@{}}
\toprule\noalign{}
\begin{minipage}[b]{\linewidth}\raggedright
Step
\end{minipage} & \begin{minipage}[b]{\linewidth}\raggedright
Activities
\end{minipage} & \begin{minipage}[b]{\linewidth}\raggedright
Tools
\end{minipage} & \begin{minipage}[b]{\linewidth}\raggedright
Output
\end{minipage} \\
\midrule\noalign{}
\endhead
\bottomrule\noalign{}
\endlastfoot
\textbf{Risk Identification} & Brainstorming, interviews & Checklists,
SWOT & Risk register \\
\textbf{Risk Assessment} & Probability and impact analysis & Risk matrix
& Risk ratings \\
\textbf{Risk Response} & Develop mitigation strategies & Response
templates & Action plans \\
\textbf{Risk Monitoring} & Track risk indicators & Dashboards & Status
reports \\
\end{longtable}
}

\textbf{Risk Categories:}

{\def\LTcaptype{none} % do not increment counter
\begin{longtable}[]{@{}lll@{}}
\toprule\noalign{}
Category & Examples & Mitigation Approach \\
\midrule\noalign{}
\endhead
\bottomrule\noalign{}
\endlastfoot
\textbf{Technical} & Technology obsolescence & Proof of concept \\
\textbf{Project} & Resource unavailability & Resource planning \\
\textbf{Business} & Market changes & Stakeholder engagement \\
\textbf{External} & Regulatory changes & Legal consultation \\
\end{longtable}
}

\textbf{Risk Response Strategies:}

\begin{itemize}
\tightlist
\item
  \textbf{Negative Risks (Threats)}: Avoid, Transfer, Mitigate, Accept
\item
  \textbf{Positive Risks (Opportunities)}: Exploit, Share, Enhance,
  Accept
\end{itemize}

\end{solutionbox}
\begin{mnemonicbox}
``Identify Assess Respond Monitor'' + ``Avoid
Transfer Mitigate Accept''

\end{mnemonicbox}
\subsection*{Question 4(a OR) [3
marks]}\label{question-4a-or-3-marks}

\textbf{Describe Software design process and explain Design
methodologies.}

\begin{solutionbox}

Software design transforms requirements into blueprint for
implementation through systematic approach.

\textbf{Design Process:}

{\def\LTcaptype{none} % do not increment counter
\begin{longtable}[]{@{}lll@{}}
\toprule\noalign{}
Phase & Activity & Output \\
\midrule\noalign{}
\endhead
\bottomrule\noalign{}
\endlastfoot
\textbf{Analysis} & Understand requirements & Problem definition \\
\textbf{Architecture} & High-level structure & System architecture \\
\textbf{Detailed Design} & Component specification & Design documents \\
\end{longtable}
}

\end{solutionbox}
\begin{mnemonicbox}
``Analysis Architecture Detail''

\end{mnemonicbox}
\subsection*{Question 4(b OR) [4
marks]}\label{question-4b-or-4-marks}

\textbf{Compare Cohesion and Coupling side by side.}

\begin{solutionbox}

Both concepts measure module design quality but focus on different
aspects.

\textbf{Comprehensive Comparison:}

{\def\LTcaptype{none} % do not increment counter
\begin{longtable}[]{@{}
  >{\raggedright\arraybackslash}p{(\linewidth - 4\tabcolsep) * \real{0.3333}}
  >{\raggedright\arraybackslash}p{(\linewidth - 4\tabcolsep) * \real{0.3333}}
  >{\raggedright\arraybackslash}p{(\linewidth - 4\tabcolsep) * \real{0.3333}}@{}}
\toprule\noalign{}
\begin{minipage}[b]{\linewidth}\raggedright
Aspect
\end{minipage} & \begin{minipage}[b]{\linewidth}\raggedright
Cohesion
\end{minipage} & \begin{minipage}[b]{\linewidth}\raggedright
Coupling
\end{minipage} \\
\midrule\noalign{}
\endhead
\bottomrule\noalign{}
\endlastfoot
\textbf{Definition} & Degree of relatedness within module & Degree of
interdependence between modules \\
\textbf{Goal} & High cohesion desired & Low coupling desired \\
\textbf{Focus} & Internal module structure & Inter-module
relationships \\
\textbf{Quality Indicator} & Stronger = Better & Weaker = Better \\
\end{longtable}
}

\textbf{Types Comparison:}

{\def\LTcaptype{none} % do not increment counter
\begin{longtable}[]{@{}
  >{\raggedright\arraybackslash}p{(\linewidth - 2\tabcolsep) * \real{0.5000}}
  >{\raggedright\arraybackslash}p{(\linewidth - 2\tabcolsep) * \real{0.5000}}@{}}
\toprule\noalign{}
\begin{minipage}[b]{\linewidth}\raggedright
Cohesion Types (Best to Worst)
\end{minipage} & \begin{minipage}[b]{\linewidth}\raggedright
Coupling Types (Best to Worst)
\end{minipage} \\
\midrule\noalign{}
\endhead
\bottomrule\noalign{}
\endlastfoot
\textbf{Functional} - Single purpose & \textbf{Data} - Simple data
sharing \\
\textbf{Sequential} - Output\rightarrowInput & \textbf{Stamp} - Data structure
sharing \\
\textbf{Communicational} - Same data & \textbf{Control} - Control
information \\
\textbf{Procedural} - Sequential execution & \textbf{External} -
External dependencies \\
\textbf{Temporal} - Same time & \textbf{Common} - Global data \\
\textbf{Logical} - Similar functions & \textbf{Content} - Internal data
access \\
\textbf{Coincidental} - No relation & \\
\end{longtable}
}

\textbf{Impact on Design:}

{\def\LTcaptype{none} % do not increment counter
\begin{longtable}[]{@{}lll@{}}
\toprule\noalign{}
Factor & High Cohesion & Low Coupling \\
\midrule\noalign{}
\endhead
\bottomrule\noalign{}
\endlastfoot
\textbf{Maintainability} & Easy to modify & Independent changes \\
\textbf{Reusability} & Self-contained modules & Flexible integration \\
\textbf{Testing} & Focused test cases & Isolated testing \\
\end{longtable}
}

\end{solutionbox}
\begin{mnemonicbox}
``Cohesion = Inside Strong, Coupling = Between Weak''

\end{mnemonicbox}
\subsection*{Question 4(c OR) [7
marks]}\label{question-4c-or-7-marks}

\textbf{Sketch Data Flow Diagram with levels and explain.}

\begin{solutionbox}

Data Flow Diagram (DFD) shows how data moves through system using
graphical notation with multiple levels of detail.

\textbf{DFD Symbols:}

{\def\LTcaptype{none} % do not increment counter
\begin{longtable}[]{@{}lll@{}}
\toprule\noalign{}
Symbol & Represents & Description \\
\midrule\noalign{}
\endhead
\bottomrule\noalign{}
\endlastfoot
\textbf{Circle/Bubble} & Process & Transforms input to output \\
\textbf{Rectangle} & External Entity & Source or destination \\
\textbf{Open Rectangle} & Data Store & Repository of data \\
\textbf{Arrow} & Data Flow & Movement of data \\
\end{longtable}
}

\textbf{DFD Levels:}

\begin{center}
\textbf{Mermaid Diagram (Code)}
\begin{verbatim}
{Shaded}
{Highlighting}[]
graph LR
    A[Context Diagram Level 0] {-{-}{} B[Level 1 DFD]}
    B {-{-}{} C[Level 2 DFD]}
    C {-{-}{} D[Level 3 DFD]}
    
    E[Single Process] {-{-}{} F[Major Processes]}
    F {-{-}{} G[Sub{-}processes]}
    G {-{-}{} H[Detailed Processes]}
{Highlighting}
{Shaded}
\end{verbatim}
\end{center}

\textbf{Level Descriptions:}

{\def\LTcaptype{none} % do not increment counter
\begin{longtable}[]{@{}
  >{\raggedright\arraybackslash}p{(\linewidth - 6\tabcolsep) * \real{0.2500}}
  >{\raggedright\arraybackslash}p{(\linewidth - 6\tabcolsep) * \real{0.2500}}
  >{\raggedright\arraybackslash}p{(\linewidth - 6\tabcolsep) * \real{0.2500}}
  >{\raggedright\arraybackslash}p{(\linewidth - 6\tabcolsep) * \real{0.2500}}@{}}
\toprule\noalign{}
\begin{minipage}[b]{\linewidth}\raggedright
Level
\end{minipage} & \begin{minipage}[b]{\linewidth}\raggedright
Scope
\end{minipage} & \begin{minipage}[b]{\linewidth}\raggedright
Purpose
\end{minipage} & \begin{minipage}[b]{\linewidth}\raggedright
Detail
\end{minipage} \\
\midrule\noalign{}
\endhead
\bottomrule\noalign{}
\endlastfoot
\textbf{Level 0 (Context)} & Entire system & System boundary & Single
process \\
\textbf{Level 1} & Major functions & High-level processes & 5-7
processes \\
\textbf{Level 2} & Sub-functions & Process breakdown & Detailed view \\
\textbf{Level 3+} & Fine details & Implementation level & Very
specific \\
\end{longtable}
}

\textbf{Example - Student Information System:}

\textbf{Level 0 (Context Diagram):}

\begin{verbatim}
[Student] \rightarrow Student Info \rightarrow [Student System] \rightarrow Reports \rightarrow [Admin]
\end{verbatim}

\textbf{Level 1 DFD:}

\begin{center}
\textbf{Mermaid Diagram (Code)}
\begin{verbatim}
{Shaded}
{Highlighting}[]
graph LR
    A[Student] {-{-}{} B[1.0 Register Student]}
    B {-{-}{} C[Student Database]}
    C {-{-}{} D[2.0 Generate Reports]}
    D {-{-}{} E[Admin]}
    F[Teacher] {-{-}{} G[3.0 Update Grades]}
    G {-{-}{} C}
{Highlighting}
{Shaded}
\end{verbatim}
\end{center}

\textbf{Balancing Rules:}

\begin{itemize}
\tightlist
\item
  \textbf{Data Conservation}: Input = Output at each level
\item
  \textbf{Process Numbering}: Hierarchical numbering system
\item
  \textbf{External Entities}: Same at all levels
\end{itemize}

\textbf{Benefits of Leveled DFDs:}

{\def\LTcaptype{none} % do not increment counter
\begin{longtable}[]{@{}lll@{}}
\toprule\noalign{}
Benefit & Description & Advantage \\
\midrule\noalign{}
\endhead
\bottomrule\noalign{}
\endlastfoot
\textbf{Abstraction} & Hide complexity & Easy understanding \\
\textbf{Decomposition} & Break down processes & Manageable chunks \\
\textbf{Verification} & Check completeness & Quality assurance \\
\end{longtable}
}

\end{solutionbox}
\begin{mnemonicbox}
``Context Major Sub Fine'' + ``Process Entity Store
Flow''

\end{mnemonicbox}
\subsection*{Question 5(a) [3 marks]}\label{q5a}

\textbf{Give Characteristics of good UI.}

\begin{solutionbox}

Good User Interface design ensures effective user interaction with
software system.

\textbf{UI Characteristics:}

{\def\LTcaptype{none} % do not increment counter
\begin{longtable}[]{@{}lll@{}}
\toprule\noalign{}
Characteristic & Description & Benefit \\
\midrule\noalign{}
\endhead
\bottomrule\noalign{}
\endlastfoot
\textbf{Simple} & Easy to understand & Reduced learning curve \\
\textbf{Consistent} & Uniform behavior & Predictable interaction \\
\textbf{Responsive} & Quick feedback & User satisfaction \\
\end{longtable}
}

\end{solutionbox}
\begin{mnemonicbox}
``Simple Consistent Responsive''

\end{mnemonicbox}
\subsection*{Question 5(b) [4 marks]}\label{q5b}

\textbf{Briefly explain Unit testing}

\begin{solutionbox}

Unit Testing verifies individual software components in isolation to
ensure correct functionality.

\textbf{Unit Testing Overview:}

{\def\LTcaptype{none} % do not increment counter
\begin{longtable}[]{@{}lll@{}}
\toprule\noalign{}
Aspect & Description & Purpose \\
\midrule\noalign{}
\endhead
\bottomrule\noalign{}
\endlastfoot
\textbf{Scope} & Individual modules/functions & Component
verification \\
\textbf{Isolation} & Test in isolation & Independent validation \\
\textbf{Automation} & Automated test execution & Efficient testing \\
\textbf{Early Detection} & Find bugs early & Cost-effective debugging \\
\end{longtable}
}

\textbf{Testing Process:}

\begin{center}
\textbf{Mermaid Diagram (Code)}
\begin{verbatim}
{Shaded}
{Highlighting}[]
graph LR
    A[Write Test Cases] {-{-}{} B[Execute Tests]}
    B {-{-}{} C[Analyze Results]}
    C {-{-}{} D[Fix Defects]}
    D {-{-}{} B}
{Highlighting}
{Shaded}
\end{verbatim}
\end{center}

\textbf{Benefits:}

\begin{itemize}
\tightlist
\item
  \textbf{Early bug detection} reduces fixing costs
\item
  \textbf{Code quality} improvement through testing discipline
\item
  \textbf{Regression testing} prevents future breaks
\end{itemize}

\end{solutionbox}
\begin{mnemonicbox}
``Scope Isolation Automation Early''

\end{mnemonicbox}
\subsection*{Question 5(c) [7 marks]}\label{q5c}

\textbf{Draw activity diagrams of the train reservation system, explain
each step.}

\begin{solutionbox}

Activity Diagram shows workflow of train reservation system from user
request to ticket confirmation.

\begin{center}
\textbf{Mermaid Diagram (Code)}
\begin{verbatim}
{Shaded}
{Highlighting}[]
graph TD
    A[Start] {-{-}{} B[User Login]}
    B {-{-}{} C\{Valid Credentials?\}}
    C {-{-}{}|No| B}
    C {-{-}{}|Yes| D[Search Trains]}
    D {-{-}{} E[Select Train]}
    E {-{-}{} F[Choose Seats]}
    F {-{-}{} G\{Seats Available?\}}
    G {-{-}{}|No| F}
    G {-{-}{}|Yes| H[Enter Passenger Details]}
    H {-{-}{} I[Review Booking]}
    I {-{-}{} J\{Confirm Booking?\}}
    J {-{-}{}|No| D}
    J {-{-}{}|Yes| K[Process Payment]}
    K {-{-}{} L\{Payment Success?\}}
    L {-{-}{}|No| K}
    L {-{-}{}|Yes| M[Generate Ticket]}
    M {-{-}{} N[Send Confirmation]}
    N {-{-}{} O[End]}
{Highlighting}
{Shaded}
\end{verbatim}
\end{center}

\textbf{Step-by-Step Explanation:}

{\def\LTcaptype{none} % do not increment counter
\begin{longtable}[]{@{}
  >{\raggedright\arraybackslash}p{(\linewidth - 6\tabcolsep) * \real{0.2500}}
  >{\raggedright\arraybackslash}p{(\linewidth - 6\tabcolsep) * \real{0.2500}}
  >{\raggedright\arraybackslash}p{(\linewidth - 6\tabcolsep) * \real{0.2500}}
  >{\raggedright\arraybackslash}p{(\linewidth - 6\tabcolsep) * \real{0.2500}}@{}}
\toprule\noalign{}
\begin{minipage}[b]{\linewidth}\raggedright
Step
\end{minipage} & \begin{minipage}[b]{\linewidth}\raggedright
Activity
\end{minipage} & \begin{minipage}[b]{\linewidth}\raggedright
Description
\end{minipage} & \begin{minipage}[b]{\linewidth}\raggedright
Decision Points
\end{minipage} \\
\midrule\noalign{}
\endhead
\bottomrule\noalign{}
\endlastfoot
\textbf{1} & User Login & Authenticate user credentials &
Valid/Invalid \\
\textbf{2} & Search Trains & Find available trains for route/date &
Results found \\
\textbf{3} & Select Train & Choose specific train & Train selection \\
\textbf{4} & Choose Seats & Select seat preferences & Availability
check \\
\textbf{5} & Enter Details & Provide passenger information & Data
validation \\
\textbf{6} & Review Booking & Confirm booking details & User
confirmation \\
\textbf{7} & Process Payment & Handle payment transaction &
Success/Failure \\
\textbf{8} & Generate Ticket & Create ticket document & Ticket
creation \\
\textbf{9} & Send Confirmation & Deliver confirmation to user & Process
complete \\
\end{longtable}
}

\textbf{Activity Types:}

{\def\LTcaptype{none} % do not increment counter
\begin{longtable}[]{@{}llll@{}}
\toprule\noalign{}
Type & Symbol & Purpose & Examples \\
\midrule\noalign{}
\endhead
\bottomrule\noalign{}
\endlastfoot
\textbf{Action} & Rounded Rectangle & Perform activity & Search
Trains \\
\textbf{Decision} & Diamond & Choose path & Valid Credentials? \\
\textbf{Start/End} & Circle & Begin/Terminate & Start, End \\
\textbf{Flow} & Arrow & Show sequence & Process flow \\
\end{longtable}
}

\textbf{Parallel Activities:}

\begin{itemize}
\tightlist
\item
  Payment processing and seat reservation can occur simultaneously
\item
  Confirmation email and SMS can be sent in parallel
\end{itemize}

\textbf{Exception Handling:}

\begin{itemize}
\tightlist
\item
  \textbf{Login Failure}: Return to login screen
\item
  \textbf{No Seats}: Allow different seat selection
\item
  \textbf{Payment Failure}: Retry payment options
\item
  \textbf{System Error}: Show error message and restart
\end{itemize}

\end{solutionbox}
\begin{mnemonicbox}
``Login Search Select Choose Enter Review Pay
Generate Send''

\end{mnemonicbox}
\subsection*{Question 5(a OR) [3
marks]}\label{question-5a-or-3-marks}

\textbf{Compare Verification, Validation side by side.}

\begin{solutionbox}

Both are quality assurance activities but focus on different aspects of
correctness.

\textbf{Verification vs Validation:}

{\def\LTcaptype{none} % do not increment counter
\begin{longtable}[]{@{}
  >{\raggedright\arraybackslash}p{(\linewidth - 4\tabcolsep) * \real{0.3333}}
  >{\raggedright\arraybackslash}p{(\linewidth - 4\tabcolsep) * \real{0.3333}}
  >{\raggedright\arraybackslash}p{(\linewidth - 4\tabcolsep) * \real{0.3333}}@{}}
\toprule\noalign{}
\begin{minipage}[b]{\linewidth}\raggedright
Aspect
\end{minipage} & \begin{minipage}[b]{\linewidth}\raggedright
Verification
\end{minipage} & \begin{minipage}[b]{\linewidth}\raggedright
Validation
\end{minipage} \\
\midrule\noalign{}
\endhead
\bottomrule\noalign{}
\endlastfoot
\textbf{Question} & ``Are we building right?'' & ``Are we building right
thing?'' \\
\textbf{Focus} & Process correctness & Product correctness \\
\textbf{Method} & Reviews, inspections & Testing, user feedback \\
\end{longtable}
}

\end{solutionbox}
\begin{mnemonicbox}
``Verification = Right Process, Validation = Right
Product''

\end{mnemonicbox}
\subsection*{Question 5(b OR) [4
marks]}\label{question-5b-or-4-marks}

\textbf{Define Testing describe any two testing type.}

\begin{solutionbox}

Testing is process of evaluating software to detect errors and ensure it
meets requirements.

\textbf{Testing Definition:} Systematic examination of software to find
defects and verify functionality.

\textbf{Two Testing Types:}

\textbf{(1) Black Box Testing:}

{\def\LTcaptype{none} % do not increment counter
\begin{longtable}[]{@{}
  >{\raggedright\arraybackslash}p{(\linewidth - 4\tabcolsep) * \real{0.3333}}
  >{\raggedright\arraybackslash}p{(\linewidth - 4\tabcolsep) * \real{0.3333}}
  >{\raggedright\arraybackslash}p{(\linewidth - 4\tabcolsep) * \real{0.3333}}@{}}
\toprule\noalign{}
\begin{minipage}[b]{\linewidth}\raggedright
Aspect
\end{minipage} & \begin{minipage}[b]{\linewidth}\raggedright
Description
\end{minipage} & \begin{minipage}[b]{\linewidth}\raggedright
Example
\end{minipage} \\
\midrule\noalign{}
\endhead
\bottomrule\noalign{}
\endlastfoot
\textbf{Approach} & Test without knowing internal structure &
Input/Output testing \\
\textbf{Focus} & Functional requirements & Login validation \\
\textbf{Technique} & Equivalence partitioning & Valid/Invalid inputs \\
\textbf{Tester} & External perspective & User acceptance \\
\end{longtable}
}

\textbf{(2) White Box Testing:}

{\def\LTcaptype{none} % do not increment counter
\begin{longtable}[]{@{}lll@{}}
\toprule\noalign{}
Aspect & Description & Example \\
\midrule\noalign{}
\endhead
\bottomrule\noalign{}
\endlastfoot
\textbf{Approach} & Test with knowledge of code structure & Path
coverage \\
\textbf{Focus} & Internal logic & Code branches \\
\textbf{Technique} & Statement coverage & All lines executed \\
\textbf{Tester} & Developer perspective & Unit testing \\
\end{longtable}
}

\textbf{Comparison:}

{\def\LTcaptype{none} % do not increment counter
\begin{longtable}[]{@{}lll@{}}
\toprule\noalign{}
Factor & Black Box & White Box \\
\midrule\noalign{}
\endhead
\bottomrule\noalign{}
\endlastfoot
\textbf{Knowledge} & No code knowledge & Full code knowledge \\
\textbf{Coverage} & Functional coverage & Structural coverage \\
\textbf{Level} & System level & Unit level \\
\end{longtable}
}

\end{solutionbox}
\begin{mnemonicbox}
``Black = External, White = Internal''

\end{mnemonicbox}
\subsection*{Question 5(c OR) [7
marks]}\label{question-5c-or-7-marks}

\textbf{Describe each Coding standards and guidelines.}

\begin{solutionbox}

Coding Standards are set of rules and conventions for writing
consistent, maintainable, and readable code.

\textbf{Purpose of Coding Standards:}

{\def\LTcaptype{none} % do not increment counter
\begin{longtable}[]{@{}lll@{}}
\toprule\noalign{}
Benefit & Description & Impact \\
\midrule\noalign{}
\endhead
\bottomrule\noalign{}
\endlastfoot
\textbf{Readability} & Easy to understand code & Faster maintenance \\
\textbf{Consistency} & Uniform coding style & Team collaboration \\
\textbf{Maintainability} & Easy to modify & Reduced costs \\
\textbf{Quality} & Fewer defects & Reliable software \\
\end{longtable}
}

\textbf{Major Coding Standards Categories:}

\textbf{(1) Naming Conventions:}

{\def\LTcaptype{none} % do not increment counter
\begin{longtable}[]{@{}
  >{\raggedright\arraybackslash}p{(\linewidth - 6\tabcolsep) * \real{0.2500}}
  >{\raggedright\arraybackslash}p{(\linewidth - 6\tabcolsep) * \real{0.2500}}
  >{\raggedright\arraybackslash}p{(\linewidth - 6\tabcolsep) * \real{0.2500}}
  >{\raggedright\arraybackslash}p{(\linewidth - 6\tabcolsep) * \real{0.2500}}@{}}
\toprule\noalign{}
\begin{minipage}[b]{\linewidth}\raggedright
Element
\end{minipage} & \begin{minipage}[b]{\linewidth}\raggedright
Standard
\end{minipage} & \begin{minipage}[b]{\linewidth}\raggedright
Example
\end{minipage} & \begin{minipage}[b]{\linewidth}\raggedright
Purpose
\end{minipage} \\
\midrule\noalign{}
\endhead
\bottomrule\noalign{}
\endlastfoot
\textbf{Variables} & camelCase & userName, totalAmount & Clear
identification \\
\textbf{Constants} & UPPER\_CASE & MAX\_SIZE, DEFAULT\_VALUE &
Distinguish constants \\
\textbf{Functions} & descriptive verbs & calculateTax(), validateInput()
& Action clarity \\
\textbf{Classes} & PascalCase & CustomerAccount, OrderManager & Type
identification \\
\end{longtable}
}

\textbf{(2) Code Structure:}

{\def\LTcaptype{none} % do not increment counter
\begin{longtable}[]{@{}
  >{\raggedright\arraybackslash}p{(\linewidth - 6\tabcolsep) * \real{0.2500}}
  >{\raggedright\arraybackslash}p{(\linewidth - 6\tabcolsep) * \real{0.2500}}
  >{\raggedright\arraybackslash}p{(\linewidth - 6\tabcolsep) * \real{0.2500}}
  >{\raggedright\arraybackslash}p{(\linewidth - 6\tabcolsep) * \real{0.2500}}@{}}
\toprule\noalign{}
\begin{minipage}[b]{\linewidth}\raggedright
Aspect
\end{minipage} & \begin{minipage}[b]{\linewidth}\raggedright
Guideline
\end{minipage} & \begin{minipage}[b]{\linewidth}\raggedright
Example
\end{minipage} & \begin{minipage}[b]{\linewidth}\raggedright
Benefit
\end{minipage} \\
\midrule\noalign{}
\endhead
\bottomrule\noalign{}
\endlastfoot
\textbf{Indentation} & Consistent spacing & 4 spaces or 1 tab & Visual
hierarchy \\
\textbf{Line Length} & Maximum 80-120 chars & Break long lines & Screen
readability \\
\textbf{Braces} & Opening brace style & Same line vs new line &
Consistency \\
\textbf{Comments} & Meaningful descriptions & // Calculate tax amount &
Code documentation \\
\end{longtable}
}

\textbf{(3) Code Organization:}

\begin{center}
\textbf{Mermaid Diagram (Code)}
\begin{verbatim}
{Shaded}
{Highlighting}[]
graph TD
    A[Code Organization] {-{-}{} B[File Structure]}
    A {-{-}{} C[Function Size]}
    A {-{-}{} D[Class Design]}
    
    B {-{-}{} E[Single Responsibility]}
    C {-{-}{} F[Small Functions]}
    D {-{-}{} G[Clear Interfaces]}
{Highlighting}
{Shaded}
\end{verbatim}
\end{center}

{\def\LTcaptype{none} % do not increment counter
\begin{longtable}[]{@{}
  >{\raggedright\arraybackslash}p{(\linewidth - 6\tabcolsep) * \real{0.2500}}
  >{\raggedright\arraybackslash}p{(\linewidth - 6\tabcolsep) * \real{0.2500}}
  >{\raggedright\arraybackslash}p{(\linewidth - 6\tabcolsep) * \real{0.2500}}
  >{\raggedright\arraybackslash}p{(\linewidth - 6\tabcolsep) * \real{0.2500}}@{}}
\toprule\noalign{}
\begin{minipage}[b]{\linewidth}\raggedright
Principle
\end{minipage} & \begin{minipage}[b]{\linewidth}\raggedright
Guideline
\end{minipage} & \begin{minipage}[b]{\linewidth}\raggedright
Limit
\end{minipage} & \begin{minipage}[b]{\linewidth}\raggedright
Benefit
\end{minipage} \\
\midrule\noalign{}
\endhead
\bottomrule\noalign{}
\endlastfoot
\textbf{File Organization} & One class per file & Related functions
grouped & Easy navigation \\
\textbf{Function Length} & Keep functions small & 20-30 lines max &
Better testing \\
\textbf{Class Size} & Single responsibility & Focused purpose &
Maintainability \\
\textbf{Module Coupling} & Minimize dependencies & Loose coupling &
Flexibility \\
\end{longtable}
}

\textbf{(4) Documentation Standards:}

{\def\LTcaptype{none} % do not increment counter
\begin{longtable}[]{@{}
  >{\raggedright\arraybackslash}p{(\linewidth - 6\tabcolsep) * \real{0.2500}}
  >{\raggedright\arraybackslash}p{(\linewidth - 6\tabcolsep) * \real{0.2500}}
  >{\raggedright\arraybackslash}p{(\linewidth - 6\tabcolsep) * \real{0.2500}}
  >{\raggedright\arraybackslash}p{(\linewidth - 6\tabcolsep) * \real{0.2500}}@{}}
\toprule\noalign{}
\begin{minipage}[b]{\linewidth}\raggedright
Type
\end{minipage} & \begin{minipage}[b]{\linewidth}\raggedright
Format
\end{minipage} & \begin{minipage}[b]{\linewidth}\raggedright
Content
\end{minipage} & \begin{minipage}[b]{\linewidth}\raggedright
Example
\end{minipage} \\
\midrule\noalign{}
\endhead
\bottomrule\noalign{}
\endlastfoot
\textbf{Header Comments} & File description & Purpose, author, date &
\texttt{//\ Customer\ management\ module} \\
\textbf{Function Comments} & Parameter description & Input/output specs
& \texttt{@param\ userId\ -\ unique\ identifier} \\
\textbf{Inline Comments} & Complex logic & Why, not what &
\texttt{//\ Using\ binary\ search\ for\ performance} \\
\textbf{API Documentation} & Public interfaces & Usage examples & Method
signatures \\
\end{longtable}
}

\textbf{(5) Error Handling:}

{\def\LTcaptype{none} % do not increment counter
\begin{longtable}[]{@{}
  >{\raggedright\arraybackslash}p{(\linewidth - 6\tabcolsep) * \real{0.2500}}
  >{\raggedright\arraybackslash}p{(\linewidth - 6\tabcolsep) * \real{0.2500}}
  >{\raggedright\arraybackslash}p{(\linewidth - 6\tabcolsep) * \real{0.2500}}
  >{\raggedright\arraybackslash}p{(\linewidth - 6\tabcolsep) * \real{0.2500}}@{}}
\toprule\noalign{}
\begin{minipage}[b]{\linewidth}\raggedright
Practice
\end{minipage} & \begin{minipage}[b]{\linewidth}\raggedright
Description
\end{minipage} & \begin{minipage}[b]{\linewidth}\raggedright
Example
\end{minipage} & \begin{minipage}[b]{\linewidth}\raggedright
Purpose
\end{minipage} \\
\midrule\noalign{}
\endhead
\bottomrule\noalign{}
\endlastfoot
\textbf{Exception Handling} & Use try-catch blocks &
\texttt{try\ \{\ ...\ \}\ catch\ (Exception\ e)} & Graceful failure \\
\textbf{Error Messages} & Meaningful messages & ``Invalid email format''
& User guidance \\
\textbf{Logging} & Record error details &
\texttt{log.error("Database\ connection\ failed")} & Debugging
support \\
\textbf{Validation} & Input checking & Check null values & Prevent
errors \\
\end{longtable}
}

\textbf{(6) Performance Guidelines:}

{\def\LTcaptype{none} % do not increment counter
\begin{longtable}[]{@{}
  >{\raggedright\arraybackslash}p{(\linewidth - 6\tabcolsep) * \real{0.2500}}
  >{\raggedright\arraybackslash}p{(\linewidth - 6\tabcolsep) * \real{0.2500}}
  >{\raggedright\arraybackslash}p{(\linewidth - 6\tabcolsep) * \real{0.2500}}
  >{\raggedright\arraybackslash}p{(\linewidth - 6\tabcolsep) * \real{0.2500}}@{}}
\toprule\noalign{}
\begin{minipage}[b]{\linewidth}\raggedright
Area
\end{minipage} & \begin{minipage}[b]{\linewidth}\raggedright
Standard
\end{minipage} & \begin{minipage}[b]{\linewidth}\raggedright
Example
\end{minipage} & \begin{minipage}[b]{\linewidth}\raggedright
Impact
\end{minipage} \\
\midrule\noalign{}
\endhead
\bottomrule\noalign{}
\endlastfoot
\textbf{Memory Usage} & Avoid memory leaks & Close resources & System
stability \\
\textbf{Algorithm Choice} & Efficient algorithms & Use appropriate data
structures & Response time \\
\textbf{Database Access} & Minimize queries & Use connection pooling &
Scalability \\
\textbf{Code Optimization} & Avoid premature optimization & Profile
before optimizing & Maintainability \\
\end{longtable}
}

\textbf{Code Review Standards:}

\begin{center}
\textbf{Mermaid Diagram (Code)}
\begin{verbatim}
{Shaded}
{Highlighting}[]
graph LR
    A[Code Written] {-{-}{} B[Self Review]}
    B {-{-}{} C[Peer Review]}
    C {-{-}{} D[Team Lead Review]}
    D {-{-}{} E[Code Approved]}
    E {-{-}{} F[Merge to Main]}
{Highlighting}
{Shaded}
\end{verbatim}
\end{center}

\textbf{Review Checklist:}

{\def\LTcaptype{none} % do not increment counter
\begin{longtable}[]{@{}
  >{\raggedright\arraybackslash}p{(\linewidth - 4\tabcolsep) * \real{0.3333}}
  >{\raggedright\arraybackslash}p{(\linewidth - 4\tabcolsep) * \real{0.3333}}
  >{\raggedright\arraybackslash}p{(\linewidth - 4\tabcolsep) * \real{0.3333}}@{}}
\toprule\noalign{}
\begin{minipage}[b]{\linewidth}\raggedright
Category
\end{minipage} & \begin{minipage}[b]{\linewidth}\raggedright
Check Items
\end{minipage} & \begin{minipage}[b]{\linewidth}\raggedright
Purpose
\end{minipage} \\
\midrule\noalign{}
\endhead
\bottomrule\noalign{}
\endlastfoot
\textbf{Functionality} & Requirements met, edge cases handled &
Correctness \\
\textbf{Standards} & Naming, formatting, documentation & Consistency \\
\textbf{Security} & Input validation, authentication & Safety \\
\textbf{Performance} & Efficient algorithms, resource usage &
Scalability \\
\end{longtable}
}

\textbf{Benefits of Following Standards:}

{\def\LTcaptype{none} % do not increment counter
\begin{longtable}[]{@{}
  >{\raggedright\arraybackslash}p{(\linewidth - 4\tabcolsep) * \real{0.3333}}
  >{\raggedright\arraybackslash}p{(\linewidth - 4\tabcolsep) * \real{0.3333}}
  >{\raggedright\arraybackslash}p{(\linewidth - 4\tabcolsep) * \real{0.3333}}@{}}
\toprule\noalign{}
\begin{minipage}[b]{\linewidth}\raggedright
Benefit
\end{minipage} & \begin{minipage}[b]{\linewidth}\raggedright
Description
\end{minipage} & \begin{minipage}[b]{\linewidth}\raggedright
Long-term Impact
\end{minipage} \\
\midrule\noalign{}
\endhead
\bottomrule\noalign{}
\endlastfoot
\textbf{Team Productivity} & Faster development & Reduced development
time \\
\textbf{Code Quality} & Fewer bugs & Lower maintenance costs \\
\textbf{Knowledge Transfer} & Easy understanding & Smooth team
transitions \\
\textbf{Tool Support} & Better IDE support & Enhanced development
experience \\
\end{longtable}
}

\textbf{Implementation Strategy:}

\begin{enumerate}
\tightlist
\item
  \textbf{Establish Guidelines}: Create team-specific coding standards
  document
\item
  \textbf{Tool Integration}: Use automated formatting and linting tools
\item
  \textbf{Training}: Conduct workshops on coding best practices
\item
  \textbf{Enforcement}: Include standards in code review process
\item
  \textbf{Continuous Improvement}: Regular updates based on team
  feedback
\end{enumerate}

\textbf{Popular Coding Standards:}

{\def\LTcaptype{none} % do not increment counter
\begin{longtable}[]{@{}llll@{}}
\toprule\noalign{}
Language & Standard & Organization & Focus \\
\midrule\noalign{}
\endhead
\bottomrule\noalign{}
\endlastfoot
\textbf{Java} & Google Java Style & Google & Comprehensive guidelines \\
\textbf{Python} & PEP 8 & Python Software Foundation & Pythonic code \\
\textbf{JavaScript} & Airbnb Style & Airbnb & Modern JS practices \\
\textbf{C\#} & Microsoft Guidelines & Microsoft & .NET ecosystem \\
\end{longtable}
}

\end{solutionbox}
\begin{mnemonicbox}
``Name Structure Organize Document Handle Perform
Review''

\end{mnemonicbox}

\end{document}
