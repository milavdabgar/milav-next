\documentclass{article}

% content/resources/templates/preamble.tex
\usepackage[margin=0.6in]{geometry}
\author{Milav Dabgar}
\usepackage{amsmath,amssymb,amsthm}
\usepackage{booktabs}
\usepackage{multirow}
\usepackage{xcolor}
\usepackage{tcolorbox}
\tcbuselibrary{breakable,skins}
\usepackage[colorlinks=true,linkcolor=blue]{hyperref}
\usepackage{titlesec}
\usepackage{enumitem}
\usepackage{tikz}
\usepackage{pgfplots}
\usepackage{circuitikz}
\usepackage[version=4]{mhchem}
\usepackage{longtable}
\usepackage{array}
\usepackage{float}
\usepackage{caption}
\usepackage{listings}

\lstset{
  basicstyle=\small\ttfamily,
  breaklines=true,
  breakatwhitespace=false,
  postbreak=\mbox{\textcolor{red}{$\hookrightarrow$}\space},
  float=false,
  numbers=left,
  numberstyle=\tiny\color{gray},
  numbersep=10pt,
  xleftmargin=2em,
  keywordstyle=\color{blue},
  commentstyle=\color{green!60!black},
  stringstyle=\color{purple},
  backgroundcolor=\color{gray!5},
  showstringspaces=false,
  tabsize=2,
  captionpos=b,
  keepspaces=true,
  columns=flexible
}

\pgfplotsset{compat=1.18}
\usetikzlibrary{shapes,arrows,positioning,calc,patterns,decorations.pathmorphing,decorations.markings,arrows.meta}

% Color scheme
\definecolor{headcolor}{RGB}{0,102,204}
\definecolor{keycolor}{RGB}{220,20,60}
\definecolor{solutioncolor}{RGB}{34,139,34}
\definecolor{mnemoniccolor}{RGB}{148,0,211}
\definecolor{codecolor}{RGB}{0,0,100}

% Spacing
\setlength{\parskip}{3pt}
\setlist[itemize]{nosep}
\setlist[enumerate]{nosep}

% Title formatting
\titleformat{\section}{\Large\bfseries\color{headcolor}}{\thesection}{1em}{}
\titleformat{\subsection}{\large\bfseries\color{headcolor}}{\thesubsection}{1em}{}

% Pandoc tightlist compatibility
\providecommand{\tightlist}{%
  \setlength{\itemsep}{0pt}\setlength{\parskip}{0pt}}

% Pandoc longtable compatibility
\newcounter{none}
\def\thenone{}


% content/resources/templates/gujarati-boxes.tex
\usepackage{fontspec}
\usepackage{polyglossia}

% Set Gujarati as main language (document is primarily in Gujarati)
% Note: gloss-gujarati.ldf doesn't exist in polyglossia, but it will use hyphenation patterns
\setdefaultlanguage{gujarati}
\setotherlanguage{english}

% Configure Gujarati font properly
% Use Language=Default to prevent polyglossia from trying to add language-specific features
% that don't exist for Gujarati, which causes "empty feature" warnings
\newfontfamily\gujaratifont[Script=Gujarati,AutoFakeBold=2.5,AutoFakeSlant=0.3]{Noto Sans Gujarati}
\setmainfont[Script=Gujarati,AutoFakeBold=2.5,AutoFakeSlant=0.3]{Noto Sans Gujarati}
% Use Noto Sans Gujarati for monospace to support Gujarati in text
\setmonofont[Scale=0.9]{Noto Sans Gujarati}

% Configure English to use the same font
\newfontfamily\englishfont[Script=Gujarati,AutoFakeBold=2.5,AutoFakeSlant=0.3]{Noto Sans Gujarati}

% Translations for polyglossia
\gappto\captionsgujarati{
  \renewcommand{\tablename}{કોષ્ટક}
  \renewcommand{\figurename}{આકૃતિ}
}

% Helper for TikZ nodes to ensure Gujarati font
\newcommand{\gu}[1]{{\gujaratifont #1}}

% Custom environments
\newtcolorbox{solutionbox}{
    breakable,
    enhanced,
    colback=solutioncolor!5!white,
    colframe=solutioncolor!75!black,
    fonttitle=\bfseries,
    title=જવાબ
}

\newtcolorbox{solutionboxnobreak}{
 colback=solutioncolor!5!white,
 colframe=solutioncolor!75!black,
 fonttitle=\bfseries,
 title=જવાબ
}

\newtcolorbox{keyformula}{
 breakable,
 enhanced,
 colback=keycolor!5!white,
 colframe=keycolor!75!black,
 fonttitle=\bfseries,
 title=રાસાયણિક સમીકરણ/સૂત્ર
}

\newtcolorbox{mnemonicbox}{
 breakable,
 enhanced,
 colback=mnemoniccolor!5!white,
 colframe=mnemoniccolor!75!black,
 fonttitle=\bfseries,
 title=મેમરી ટ્રીક
}


% Custom commands for GTU solutions
% This file defines semantic commands for consistent formatting

% Question command with automatic formatting
\newcommand{\question}[2]{%
  \section*{Question #1}%
  \textbf{#2}%
}

% OR question variant
\newcommand{\questionor}[2]{%
  \section*{Question #1 OR}%
  \textbf{#2}%
}

% Proper table environment with caption
\newenvironment{answertable}[1]{%
  \begin{table}[htbp]
  \centering
  \caption{#1}
}{%
  \end{table}
}

% Proper figure environment for diagrams
\newenvironment{answerdiagram}[1]{%
  \begin{figure}[htbp]
  \centering
  \caption{#1}
}{%
  \end{figure}
}

% Semantic markup for key terms
\newcommand{\keyword}[1]{\textbf{#1}}
\newcommand{\code}[1]{\texttt{#1}}
\newcommand{\classname}[1]{\texttt{#1}}
\newcommand{\methodname}[1]{\texttt{#1}}

% Proper quotation marks
\newcommand{\mnemonic}[1]{``#1''}

\usetikzlibrary{mindmap}

\title{Fundamentals of Software Development (4331604) - Summer 2025 Solution}
\date{May 17, 2025}

\begin{document}
\maketitle

\questionmarks{1(a)}{3}{Give IEEE definition of software. Write one example of each for application and system software.}

\begin{solutionbox}
\textbf{IEEE વ્યાખ્યા}: સોફ્ટવેર એ કમ્પ્યુટર પ્રોગ્રામ્સ, પ્રક્રિયાઓ, નિયમો અને સંલગ્ન દસ્તાવેજીકરણ અને ડેટાનો સંગ્રહ છે.

\textbf{ઉદાહરણો}:

\begin{center}
\captionof{table}{સોફ્ટવેર પ્રકારો અને ઉદાહરણો}
\begin{tabulary}{\linewidth}{|L|L|L|}
\hline
\textbf{સોફ્ટવેર પ્રકાર} & \textbf{ઉદાહરણ} & \textbf{હેતુ} \\ \hline
\textbf{એપ્લિકેશન સોફ્ટવેર} & Microsoft Word & વર્ડ પ્રોસેસિંગ અને દસ્તાવેજ બનાવટ \\ \hline
\textbf{સિસ્ટમ સોફ્ટવેર} & Windows 10 & હાર્ડવેર સંસાધનોનું સંચાલન કરતી ઓપરેટિંગ સિસ્ટમ \\ \hline
\end{tabulary}
\end{center}

\begin{itemize}
    \item \keyword{એપ્લિકેશન સોફ્ટવેર}: ચોક્કસ કાર્યો પૂર્ણ કરવા માટે અંતિમ વપરાશકર્તાઓ માટે રચાયેલ પ્રોગ્રામ્સ
    \item \keyword{સિસ્ટમ સોફ્ટવેર}: કમ્પ્યુટર હાર્ડવેરનું સંચાલન અને સંચાલન કરતા પ્રોગ્રામ્સ
\end{itemize}
\end{solutionbox}

\begin{mnemonicbox}
\mnemonic{Apps help Users, Systems help Hardware: તફાવત યાદ રાખવા માટે મેમરી ટ્રીક.}
\end{mnemonicbox}

\questionmarks{1(b)}{4}{Write a short note on data dictionary.}

\begin{solutionbox}
ડેટા ડिक्शनરી એ એક કેન્દ્રિય રિપોઝીટરી છે જેમાં સિસ્ટમમાં વપરાતા ડેટા તત્વોની વ્યાખ્યાઓ અને લાક્ષણિકતાઓ હોય છે.

\begin{center}
\captionof{table}{ડેટા ડિક્શનરીના ઘટકો}
\begin{tabulary}{\linewidth}{|L|L|}
\hline
\textbf{ઘટક} & \textbf{વર્ણન} \\ \hline
\textbf{ડેટા નામ} & ડેટા તત્વ માટે અનન્ય ઓળખકર્તા \\ \hline
\textbf{ઉપનામો (Aliases)} & વપરાયેલ વૈકલ્પિક નામો \\ \hline
\textbf{વર્ણન} & હેતુ અને અર્થ \\ \hline
\textbf{ડેટા પ્રકાર} & ફોર્મેટ (પૂર્ણાંક, સ્ટ્રિંગ, વગેરે) \\ \hline
\textbf{લંબાઈ} & કદ મર્યાદાઓ \\ \hline
\textbf{મૂલ્યો} & માન્ય શ્રેણી અથવા સેટ \\ \hline
\end{tabulary}
\end{center}

\begin{itemize}
    \item \keyword{હેતુ}: વિકાસ ટીમમાં ડેટાના ઉપયોગમાં સુસંગતતા સુનિશ્ચિત કરે છે
    \item \keyword{ફાયદા}: અસ્પષ્ટતા ઘટાડે છે, સંચાર સુધારે છે, ડેટા વ્યાખ્યાઓને પ્રમાણિત કરે છે
    \item \keyword{ઉપયોગ}: સિસ્ટમ ડિઝાઇન અને ડેટાબેઝ બનાવટ દરમિયાન સંદર્ભિત
\end{itemize}
\end{solutionbox}

\begin{mnemonicbox}
\mnemonic{Dictionary Defines Data Clearly: ડેટા ડિક્શનરીનો હેતુ.}
\end{mnemonicbox}

\questionmarks{1(c)}{7}{Explain prototype model with figure.}

\begin{solutionbox}
પ્રોટોટાઇપ મોડેલ એક પુનરાવર્તિત અભિગમ છે જ્યાં જરૂરિયાતોને વધુ સારી રીતે સમજવા માટે પ્રારંભિક તબક્કે કાર્યકારી મોડેલ બનાવવામાં આવે છે.

\begin{center}
\begin{tikzpicture}[node distance=1.5cm, auto, every node/.style={gtu block, align=center, font=\small}]
    \node (Req) {જરૂરિયાત\\એકત્રીકરણ};
    \node [right=of Req] (Des) {ઝડપી\\ડિઝાઇન};
    \node [right=of Des] (Build) {પ્રોટોટાઇપ\\બનાવો};
    \node [below=of Build] (Eval) {વપરાશકર્તા\\મૂલ્યાંકન};
    \node [below=of Des] (Sat) {વપરાશકર્તા\\સંતુષ્ટ?};
    \node [left=of Sat] (Refine) {જરૂરિયાતોમાં\\સુધારો};
    \node [right=of Sat] (Final) {અંતિમ સિસ્ટમ\\વિકાસ};
    \node [right=of Final] (Test) {ટેસ્ટિંગ અને\\જાળવણી};

    \path [gtu arrow] (Req) -- (Des);
    \path [gtu arrow] (Des) -- (Build);
    \path [gtu arrow] (Build) -- (Eval);
    \path [gtu arrow] (Eval) -- (Sat);
    \path [gtu arrow] (Sat) -- node[above]{ના} (Refine);
    \path [gtu arrow] (Refine) |- (Des);
    \path [gtu arrow] (Sat) -- node[above]{હા} (Final);
    \path [gtu arrow] (Final) -- (Test);
\end{tikzpicture}
\captionof{figure}{પ્રોટોટાઇપ મોડેલ}
\end{center}

\textbf{લાક્ષણિકતાઓ}:
\begin{center}
\captionof{table}{પ્રોટોટાઇપ મોડેલ તબક્કાઓ}
\begin{tabulary}{\linewidth}{|L|L|L|}
\hline
\textbf{તબક્કો} & \textbf{પ્રવૃત્તિ} & \textbf{આઉટપુટ} \\ \hline
\textbf{ઝડપી ડિઝાઇન} & મૂળભૂત આર્કિટેક્ચર & પ્રારંભિક ડિઝાઇન \\ \hline
\textbf{પ્રોટોટાઇપ બિલ્ડ} & કાર્યકારી મોડેલ & ટેસ્ટ કરી શકાય તેવી સિસ્ટમ \\ \hline
\textbf{વપરાશકર્તા મૂલ્યાંકન} & પ્રતિસાદ સંગ્રહ & જરૂરિયાતોમાં સુધારો \\ \hline
\end{tabulary}
\end{center}

\begin{itemize}
    \item \keyword{ફાયદા}: પ્રારંભિક વપરાશકર્તા પ્રતિસાદ, ઘટાડેલું વિકાસ જોખમ, વધુ સારી જરૂરિયાત સમજ
    \item \keyword{ગેરફાયદા}: અપૂરતું વિશ્લેષણ તરફ દોરી શકે છે, ગ્રાહક પ્રોટોટાઇપને અંતિમ ઉત્પાદન તરીકે અપેક્ષા રાખે છે
    \item \keyword{શ્રેષ્ઠ માટે}: અસ્પષ્ટ જરૂરિયાતોવાળા પ્રોજેક્ટ્સ
\end{itemize}
\end{solutionbox}

\begin{mnemonicbox}
\mnemonic{Prototype Proves Possibilities: પ્રોટોટાઇપિંગનો મુખ્ય લાભ.}
\end{mnemonicbox}

\questionmarks{1(c OR)}{7}{Explain RAD model with advantages and disadvantages.}

\begin{solutionbox}
RAD (રેપિડ એપ્લિકેશન ડેવલપમેન્ટ) પ્રોટોટાઇપિંગ અને પુનરાવર્તિત વિકાસ દ્વારા ઝડપી વિકાસ પર ભાર મૂકે છે.

\begin{center}
\begin{tikzpicture}[node distance=1.5cm, auto, every node/.style={gtu state, align=center, font=\small}]
    \node (Bus) {બિઝનેસ\\મોડેલિંગ};
    \node [right=of Bus] (Data) {ડેટા\\મોડેલિંગ};
    \node [right=of Data] (Proc) {પ્રોસેસ\\મોડેલિંગ};
    \node [below=of Proc] (App) {એપ્લિકેશન\\જનરેશન};
    \node [left=of App] (Test) {ટેસ્ટિંગ અને\\ટર્નઓવર};

    \path [gtu arrow] (Bus) -- (Data);
    \path [gtu arrow] (Data) -- (Proc);
    \path [gtu arrow] (Proc) -- (App);
    \path [gtu arrow] (App) -- (Test);
\end{tikzpicture}
\captionof{figure}{RAD મોડેલ તબક્કાઓ}
\end{center}

\begin{center}
\captionof{table}{RAD ફાયદા વિ. ગેરફાયદા}
\begin{tabulary}{\linewidth}{|L|L|}
\hline
\textbf{ફાયદા} & \textbf{ગેરફાયદા} \\ \hline
\textbf{ઝડપી વિકાસ} & \textbf{કુશળ ડેવલપર્સની જરૂર} \\ \hline
\textbf{પ્રારંભિક વપરાશકર્તા સંડોવણી} & \textbf{મોટા પ્રોજેક્ટ્સ માટે યોગ્ય નથી} \\ \hline
\textbf{ઘટાડેલો ખર્ચ} & \textbf{વપરાશકર્તા પ્રતિબદ્ધતાની જરૂર} \\ \hline
\textbf{વધુ સારી ગુણવત્તા} & \textbf{જો મેનેજ ન કરવામાં આવે તો તકનીકી જોખમો} \\ \hline
\end{tabulary}
\end{center}

\begin{itemize}
    \item \keyword{મુખ્ય લક્ષણ}: સ્વચાલિત સાધનો અને 4GL પ્રોગ્રામિંગનો ઉપયોગ કરે છે
    \item \keyword{સમયરેખા}: સામાન્ય રીતે વિકાસ માટે 60-90 દિવસ
    \item \keyword{ટીમ}: નાની, અનુભવી વિકાસ ટીમો
\end{itemize}
\end{solutionbox}

\begin{mnemonicbox}
\mnemonic{RAD Rapidly Accelerates Development: RAD નો મુખ્ય ખ્યાલ.}
\end{mnemonicbox}

\questionmarks{2(a)}{3}{Give the full form of following: SQA, FTR, RAD, BVA, GUI, DFD}

\begin{solutionbox}
\begin{center}
\captionof{table}{સંક્ષેપો અને પૂર્ણ સ્વરૂપો}
\begin{tabulary}{\linewidth}{|L|L|}
\hline
\textbf{સંક્ષેપ} & \textbf{પૂર્ણ સ્વરૂપ} \\ \hline
\textbf{SQA} & Software Quality Assurance \\ \hline
\textbf{FTR} & Formal Technical Review \\ \hline
\textbf{RAD} & Rapid Application Development \\ \hline
\textbf{BVA} & Boundary Value Analysis \\ \hline
\textbf{GUI} & Graphical User Interface \\ \hline
\textbf{DFD} & Data Flow Diagram \\ \hline
\end{tabulary}
\end{center}
\end{solutionbox}

\begin{mnemonicbox}
\mnemonic{Software Quality And Formal Technical Reviews Rapidly Analyze Development, Boundary Value Analysis Guides User Interface, Data Flow Diagrams}
\end{mnemonicbox}

\questionmarks{2(b)}{4}{Define agile methodology. Discuss agile principles.}

\begin{solutionbox}
\textbf{વ્યાખ્યા}: Agile એ પુનરાવર્તિત સોફ્ટવેર વિકાસ અભિગમ છે જે સહયોગ, સુગમતા અને કાર્યકારી સોફ્ટવેરની ઝડપી ડિલિવરી પર ભાર મૂકે છે.

\textbf{મુખ્ય Agile સિદ્ધાંતો}:
\begin{center}
\captionof{table}{Agile સિદ્ધાંતો}
\begin{tabulary}{\linewidth}{|L|L|}
\hline
\textbf{સિદ્ધાંત} & \textbf{વર્ણન} \\ \hline
\textbf{પ્રક્રિયાઓ કરતાં વ્યક્તિઓ} & લોકો અને સંચાર પ્રાથમિકતા છે \\ \hline
\textbf{દસ્તાવેજીકરણ કરતાં કાર્યકારી સોફ્ટવેર} & કાર્યકારી સોફ્ટવેર એ પ્રાથમિક માપદંડ છે \\ \hline
\textbf{ગ્રાહક સહયોગ} & સતત ગ્રાહક સંડોવણી \\ \hline
\textbf{ફેરફારોને પ્રતિસાદ} & કઠોર યોજનાઓ કરતાં અનુકૂલનક્ષમતા \\ \hline
\end{tabulary}
\end{center}

\begin{itemize}
    \item \keyword{પુનરાવર્તન કાળ}: સામાન્ય રીતે 2-4 અઠવાડિયા (sprints)
    \item \keyword{ડિલિવરી}: વારંવાર કાર્યકારી સોફ્ટવેર રિલીઝ
    \item \keyword{ટીમ માળખું}: ક્રોસ-ફંક્શનલ, સ્વ-વ્યવસ્થિત ટીમો
\end{itemize}
\end{solutionbox}

\begin{mnemonicbox}
\mnemonic{Agile Adapts And Advances: Agile મુખ્ય ફિલસૂફી.}
\end{mnemonicbox}

\questionmarks{2(c)}{7}{Explain XP model with its advantages and disadvantages.}

\begin{solutionbox}
XP (એક્સ્ટ્રીમ પ્રોગ્રામિંગ) એ એજાઇલ મેથડોલોજી છે જે એન્જિનિયરિંગ પ્રેક્ટિસ અને ગ્રાહક સંતોષ પર ભાર મૂકે છે.

\textbf{XP પ્રેક્ટિસ}:
\begin{center}
\begin{tikzpicture}[
    mindmap,
    concept color=blue!30,
    every node/.style={concept, execute at begin node=\hskip0pt},
    root concept/.append style={concept, color=blue!50, minimum size=3cm, font=\bfseries},
    level 1 concept/.append style={level distance=4.5cm, sibling angle=45, color=blue!20}
]
    \node [root concept] {XP પ્રેક્ટિસ}
        child { node {પ્લાનિંગ ગેમ} }
        child { node {નાના રિલીઝ} }
        child { node {પેર પ્રોગ્રામિંગ} }
        child { node {ટેસ્ટ-ડ્રિવન ડેવલપમેન્ટ} }
        child { node {કન્ટિન્યુઅસ ઇન્ટિગ્રેશન} }
        child { node {રિફેક્ટરિંગ} }
        child { node {સરળ ડિઝાઇન} }
        child { node {સામૂહિક માલિકી} };
\end{tikzpicture}
\captionof{figure}{એક્સ્ટ્રીમ પ્રોગ્રામિંગ પ્રેક્ટિસ}
\end{center}

\textbf{ફાયદા અને ગેરફાયદા}:
\begin{center}
\captionof{table}{XP ફાયદા અને ગેરફાયદા}
\begin{tabulary}{\linewidth}{|L|L|}
\hline
\textbf{ફાયદા} & \textbf{ગેરફાયદા} \\ \hline
\textbf{ઉચ્ચ કોડ ગુણવત્તા} & \textbf{અનુભવી પ્રોગ્રામરોની જરૂર} \\ \hline
\textbf{ઝડપી પ્રતિસાદ} & \textbf{ગ્રાહક ઉપલબ્ધ હોવો જોઈએ} \\ \hline
\textbf{ઘટાડેલા બગ્સ} & \textbf{કોડ-કેન્દ્રિત, ઓછું દસ્તાવેજીકરણ} \\ \hline
\textbf{સુગમતા} & \textbf{ખર્ચ અંદાજ કાઢવો મુશ્કેલ} \\ \hline
\end{tabulary}
\end{center}

\begin{itemize}
    \item \keyword{મુખ્ય પ્રથા}: પેર પ્રોગ્રામિંગ કોડની ગુણવત્તા સુનિશ્ચિત કરે છે
    \item \keyword{ટેસ્ટિંગ}: સ્વચાલિત ટેસ્ટિંગ સાથે ટેસ્ટ-ફર્સ્ટ અભિગમ
    \item \keyword{ગ્રાહક ભૂમિકા}: ઓન-સાઇટ ગ્રાહક સતત પ્રતિસાદ પ્રદાન કરે છે
\end{itemize}
\end{solutionbox}

\begin{mnemonicbox}
\mnemonic{eXtreme Programming eXcels through Practices: XP ચોક્કસ પ્રથાઓ પર આધાર રાખે છે.}
\end{mnemonicbox}

\questionmarks{2(a OR)}{3}{Define black box testing. Give at least two names of black box testing method.}

\begin{solutionbox}
\textbf{વ્યાખ્યા}: બ્લેક બોક્સ ટેસ્ટિંગ આંતરિક કોડ સ્ટ્રક્ચરના જ્ઞાન વિના સોફ્ટવેર કાર્યક્ષમતાની તપાસ કરે છે, જે ઇનપુટ-આઉટપુટ વર્તણૂક પર ધ્યાન કેન્દ્રિત કરે છે.

\textbf{બ્લેક બોક્સ ટેસ્ટિંગ પદ્ધતિઓ}:
\begin{center}
\captionof{table}{બ્લેક બોક્સ પદ્ધતિઓ}
\begin{tabulary}{\linewidth}{|L|L|}
\hline
\textbf{પદ્ધતિ} & \textbf{વર્ણન} \\ \hline
\textbf{ઇક્વિવેલન્સ પાર્ટીશનિંગ} & ઇનપુટને માન્ય/અમાન્ય વર્ગોમાં વિભાજિત કરે છે \\ \hline
\textbf{બાઉન્ડ્રી વેલ્યુ એનાલિસિસ} & ઇનપુટ સીમાઓ પર મૂલ્યોનું પરીક્ષણ કરે છે \\ \hline
\end{tabulary}
\end{center}

\begin{itemize}
    \item \keyword{અભિગમ}: જરૂરિયાતો અને સ્પષ્ટીકરણો પર આધારિત પરીક્ષણો
    \item \keyword{ટેસ્ટર જ્ઞાન}: આંતરિક કોડ જ્ઞાન જરૂરી નથી
    \item \keyword{ફોકસ}: બાહ્ય વર્તણૂક અને કાર્યક્ષમતા
\end{itemize}
\end{solutionbox}

\begin{mnemonicbox}
\mnemonic{Black Box Behavior Based: ટેસ્ટિંગ ફોકસ.}
\end{mnemonicbox}

\questionmarks{2(b OR)}{4}{Give the full form of CLI. Explain CLI in brief.}

\begin{solutionbox}
\textbf{CLI}: Command Line Interface

\textbf{CLI લાક્ષણિકતાઓ}:
\begin{center}
\captionof{table}{CLI સુવિધાઓ}
\begin{tabulary}{\linewidth}{|L|L|}
\hline
\textbf{પાસું} & \textbf{વર્ણન} \\ \hline
\textbf{ઇનપુટ પદ્ધતિ} & વપરાશકર્તા દ્વારા ટાઇપ કરેલ ટેક્સ્ટ આદેશો \\ \hline
\textbf{આઉટપુટ} & ટેક્સ્ટ-આધારિત પ્રતિસાદ \\ \hline
\textbf{નેવિગેશન} & ફાઇલ/ડિરેક્ટરી કામગીરી માટે આદેશો \\ \hline
\textbf{કાર્યક્ષમતા} & અનુભવી વપરાશકર્તાઓ માટે ઝડપી \\ \hline
\end{tabulary}
\end{center}

\begin{itemize}
    \item \keyword{ફાયદા}: ઝડપી અમલીકરણ, ઓછો મેમરી વપરાશ, સ્ક્રિપ્ટેબલ
    \item \keyword{ગેરફાયદા}: આદેશો શીખવાની જરૂર, નવા નિશાળીયા માટે વપરાશકર્તા મૈત્રીપૂર્ણ નથી
    \item \keyword{ઉદાહરણો}: Windows Command Prompt, Linux Terminal, DOS
\end{itemize}
\end{solutionbox}

\begin{mnemonicbox}
\mnemonic{Commands Lead Interaction: CLI ક્રિયાપ્રતિક્રિયા મોડેલ.}
\end{mnemonicbox}

\questionmarks{2(c OR)}{7}{Explain waterfall model with neat figure.}

\begin{solutionbox}
વોટરફોલ મોડેલ એ એક રેખીય ક્રમિક અભિગમ છે જ્યાં આગળના તબક્કામાં જતા પહેલા દરેક તબક્કો પૂર્ણ થવો આવશ્યક છે.

\begin{center}
\begin{tikzpicture}[node distance=1.5cm, auto, every node/.style={gtu block, align=center, minimum width=2.5cm}]
    \node (Req) {જરૂરિયાત\\વિશ્લેષણ};
    \node [below right=0.5cm and 0.5cm of Req] (Des) {સિસ્ટમ\\ડિઝાઇન};
    \node [below right=0.5cm and 0.5cm of Des] (Imp) {અમલીકરણ};
    \node [below right=0.5cm and 0.5cm of Imp] (Test) {ઇન્ટિગ્રેશન અને\\ટેસ્ટિંગ};
    \node [below right=0.5cm and 0.5cm of Test] (Dep) {ડિપ્લોયમેન્ટ};
    \node [below right=0.5cm and 0.5cm of Dep] (Main) {જાળવણી};

    \path [gtu arrow] (Req) -| (Des);
    \path [gtu arrow] (Des) -| (Imp);
    \path [gtu arrow] (Imp) -| (Test);
    \path [gtu arrow] (Test) -| (Dep);
    \path [gtu arrow] (Dep) -| (Main);
\end{tikzpicture}
\captionof{figure}{વોટરફોલ મોડેલ}
\end{center}

\textbf{તબક્કાવાર વિગતો}:
\begin{center}
\captionof{table}{વોટરફોલ મોડેલ તબક્કાઓ}
\begin{tabulary}{\linewidth}{|L|L|L|}
\hline
\textbf{તબક્કો} & \textbf{પ્રવૃત્તિઓ} & \textbf{ડિલિવરેબલ્સ} \\ \hline
\textbf{જરૂરિયાતો} & જરૂરિયાતો એકત્રિત અને દસ્તાવેજીકરણ & SRS દસ્તાવેજ \\ \hline
\textbf{ડિઝાઇન} & સિસ્ટમ આર્કિટેક્ચર & ડિઝાઇન દસ્તાવેજો \\ \hline
\textbf{અમલીકરણ} & કોડ વિકાસ & સોર્સ કોડ \\ \hline
\textbf{ટેસ્ટિંગ} & કાર્યક્ષમતા ચકાસણી & ટેસ્ટ રિપોર્ટ્સ \\ \hline
\textbf{ડિપ્લોયમેન્ટ} & સિસ્ટમ ઇન્સ્ટોલેશન & કાર્યકારી સિસ્ટમ \\ \hline
\textbf{જાળવણી} & બગ ફિક્સ, અપડેટ્સ & અપડેટેડ સિસ્ટમ \\ \hline
\end{tabulary}
\end{center}

\begin{itemize}
    \item \keyword{ફાયદા}: સરળ, મેનેજ કરવા માટે સરળ, સારી રીતે દસ્તાવેજીકૃત
    \item \keyword{ગેરફાયદા}: અક્કડ, મોડું ટેસ્ટિંગ, ફેરફારોને સમાવવા મુશ્કેલ
\end{itemize}
\end{solutionbox}

\begin{mnemonicbox}
\mnemonic{Water Always Flows Downward: વોટરફોલની ક્રમિક પ્રકૃતિ.}
\end{mnemonicbox}

\questionmarks{3(a)}{3}{Give one word answer:}

\begin{solutionbox}
\begin{center}
\captionof{table}{એક શબ્દના જવાબો}
\begin{tabulary}{\linewidth}{|L|L|}
\hline
\textbf{પ્રશ્ન} & \textbf{જવાબ} \\ \hline
\textbf{સૌથી ઓછું કોહેશન છે} & કોઇન્સિડેન્ટલ (Coincidental) \\ \hline
\textbf{સૌથી વધુ કપલિંગ છે} & કન્ટેન્ટ (Content) \\ \hline
\textbf{જટિલ પ્રવૃત્તિનો સ્લેક ટાઇમ છે} & શૂન્ય (Zero) \\ \hline
\end{tabulary}
\end{center}
\end{solutionbox}

\begin{mnemonicbox}
\mnemonic{Coincidental Cohesion, Content Coupling, Critical Zero}
\end{mnemonicbox}

\questionmarks{3(b)}{4}{Explain classification of coupling.}

\begin{solutionbox}
કપલિંગ મોડ્યુલો વચ્ચે પરસ્પર નિર્ભરતાને માપે છે. જાળવણી માટે ઓછું કપલિંગ વધુ સારું છે.

\textbf{કપલિંગ પ્રકારો (શ્રેષ્ઠ થી ખરાબ)}:
\begin{center}
\captionof{table}{કપલિંગ પ્રકારો}
\begin{tabulary}{\linewidth}{|L|L|L|}
\hline
\textbf{પ્રકાર} & \textbf{વર્ણન} & \textbf{ઉદાહરણ} \\ \hline
\textbf{Data} & પરિમાણો પસાર થાય છે & પરિમાણો સાથે પદ્ધતિ કોલ્સ \\ \hline
\textbf{Stamp} & ડેટા સ્ટ્રક્ચર પસાર થાય છે & ઓબ્જેક્ટ્સ/રેકોર્ડ્સ પસાર કરવા \\ \hline
\textbf{Control} & નિયંત્રણ માહિતી પસાર થાય છે & ફ્લેગ્સ/સ્વીચો પસાર થાય છે \\ \hline
\textbf{External} & બાહ્ય ડેટા સંદર્ભ & ગ્લોબલ વેરિયેબલ્સ \\ \hline
\textbf{Common} & શેર કરેલ ડેટા વિસ્તાર & સામાન્ય મેમરી બ્લોક્સ \\ \hline
\textbf{Content} & આંતરિક ભાગમાં સીધો પ્રવેશ & અન્ય મોડ્યુલના ડેટામાં ફેરફાર \\ \hline
\end{tabulary}
\end{center}

\begin{itemize}
    \item \keyword{શ્રેષ્ઠ પ્રથા}: ડેટા કપલિંગ માટે લક્ષ્ય રાખો
    \item \keyword{ટાળો}: કન્ટેન્ટ અને કોમન કપલિંગ
    \item \keyword{ડિઝાઇન ધ્યેય}: મોડ્યુલો વચ્ચે નિર્ભરતા ઘટાડો
\end{itemize}
\end{solutionbox}

\begin{mnemonicbox}
\mnemonic{Data Stamps Control External Common Content: કપલિંગ કડકતાનો ક્રમ.}
\end{mnemonicbox}

\questionmarks{3(c)}{7}{Define following terms (don't just give the full form):}

\begin{solutionbox}
\begin{center}
\captionof{table}{સોફ્ટવેર વ્યાખ્યાઓ}
\begin{tabulary}{\linewidth}{|L|L|}
\hline
\textbf{શબ્દ} & \textbf{વ્યાખ્યા} \\ \hline
\textbf{UI} & User Interface - સાધન જેના દ્વારા વપરાશકર્તાઓ સોફ્ટવેર સિસ્ટમો સાથે ક્રિયાપ્રતિક્રિયા કરે છે \\ \hline
\textbf{SE} & Software Engineering - એન્જિનિયરિંગ સિદ્ધાંતોનો ઉપયોગ કરીને સોફ્ટવેર વિકાસ માટે વ્યવસ્થિત અભિગમ \\ \hline
\textbf{PMC} & Project Management and Control - સોફ્ટવેર પ્રોજેક્ટ્સનું આયોજન, દેખરેખ અને નિયંત્રણ \\ \hline
\textbf{SDLC} & Software Development Life Cycle - વિભાવનાથી જાળવણી સુધીના સોફ્ટવેર વિકાસમાં સામેલ તબક્કાઓ \\ \hline
\textbf{Verification} & સોફ્ટવેર નિર્દિષ્ટ જરૂરિયાતો અને ડિઝાઇનને પૂર્ણ કરે છે કે કેમ તે તપાસવાની પ્રક્રિયા \\ \hline
\textbf{Validation} & સોફ્ટવેર વપરાશકર્તાની જરૂરિયાતો અને હેતુપૂર્વકના હેતુને પૂર્ણ કરે છે કે કેમ તે તપાસવાની પ્રક્રિયા \\ \hline
\textbf{SRS} & Software Requirements Specification - સોફ્ટવેર કાર્યક્ષમતા અને મર્યાદાઓનું વર્ણન કરતો વિગતવાર દસ્તાવેજ \\ \hline
\end{tabulary}
\end{center}

\begin{itemize}
    \item \keyword{Verification}: "શું આપણે યોગ્ય રીતે ઉત્પાદન બનાવી રહ્યા છીએ?"
    \item \keyword{Validation}: "શું આપણે યોગ્ય ઉત્પાદન બનાવી રહ્યા છીએ?"
    \item \keyword{મુખ્ય તફાવત}: વેરિફિકેશન સ્પષ્ટીકરણો તપાસે છે, વેલિડેશન વપરાશકર્તા સંતોષ તપાસે છે
\end{itemize}
\end{solutionbox}

\begin{mnemonicbox}
\mnemonic{Users Interact, Software Engineers Plan, Managing Cycles, Specifications Define, Verification checks Requirements, Validation checks Satisfaction, Requirements Specify Software}
\end{mnemonicbox}

\questionmarks{3(a OR)}{3}{Explain menu based UI with advantages and disadvantages.}

\begin{solutionbox}
મેનુ-આધારિત UI વપરાશકર્તા પસંદગી માટે શ્રેણીબદ્ધ મેનુમાં વિકલ્પો રજૂ કરે છે.

\textbf{ફાયદા વિ. ગેરફાયદા}:
\begin{center}
\captionof{table}{મેનુ UI ફાયદા અને ગેરફાયદા}
\begin{tabulary}{\linewidth}{|L|L|}
\hline
\textbf{ફાયદા} & \textbf{ગેરફાયદા} \\ \hline
\textbf{શીખવા માટે સરળ} & \textbf{નિષ્ણાતો માટે ધીમું} \\ \hline
\textbf{ભૂલો ઘટાડે છે} & \textbf{મર્યાદિત સુગમતા} \\ \hline
\textbf{સ્વ-સ્પષ્ટીકરણાત્મક} & \textbf{સ્ક્રીન સ્પેસ વપરાશ} \\ \hline
\end{tabulary}
\end{center}

\begin{itemize}
    \item \keyword{માળખું}: વિકલ્પોનું શ્રેણીબદ્ધ સંગઠન
    \item \keyword{નેવિગેશન}: પોઇન્ટ-એન્ડ-ક્લિક અથવા કીબોર્ડ શોર્ટકટ્સ
    \item \keyword{શ્રેષ્ઠ માટે}: સારી રીતે વ્યાખ્યાયિત કાર્યો વાળી એપ્લિકેશન્સ
\end{itemize}
\end{solutionbox}

\begin{mnemonicbox}
\mnemonic{Menus Make Choices Clear: મેનુ UI નો લાભ.}
\end{mnemonicbox}

\questionmarks{3(b OR)}{4}{Explain classification of cohesion.}

\begin{solutionbox}
કોહેશન માપે છે કે મોડ્યુલની અંદરના તત્વો કેટલા નજીકથી સંબંધિત છે. ઉચ્ચ કોહેશન વધુ સારું છે.

\textbf{કોહેશન પ્રકારો (શ્રેષ્ઠ થી ખરાબ)}:
\begin{center}
\captionof{table}{કોહેશન પ્રકારો}
\begin{tabulary}{\linewidth}{|L|L|}
\hline
\textbf{પ્રકાર} & \textbf{વર્ણન} \\ \hline
\textbf{Functional} & એકલ, સારી રીતે વ્યાખ્યાયિત કાર્ય \\ \hline
\textbf{Sequential} & એક ઘટકનું આઉટપુટ આગામીને ફીડ કરે છે \\ \hline
\textbf{Communicational} & તત્વો સમાન ડેટા પર કામ કરે છે \\ \hline
\textbf{Procedural} & તત્વો અમલીકરણ ક્રમને અનુસરે છે \\ \hline
\textbf{Temporal} & તત્વો તે જ સમયે ચલાવવામાં આવે છે \\ \hline
\textbf{Logical} & તત્વો સમાન કાર્યો કરે છે \\ \hline
\textbf{Coincidental} & તત્વો અવ્યવસ્થિત રીતે જૂથબદ્ધ \\ \hline
\end{tabulary}
\end{center}

\begin{itemize}
    \item \keyword{ધ્યેય}: ફંક્શનલ કોહેશન પ્રાપ્ત કરો
    \item \keyword{ડિઝાઇન સિદ્ધાંત}: દરેક મોડ્યુલની એક જ જવાબદારી હોવી જોઈએ
    \item \keyword{માપન}: ઉચ્ચ કોહેશન = વધુ સારી ડિઝાઇન
\end{itemize}
\end{solutionbox}

\begin{mnemonicbox}
\mnemonic{Functional Sequences Communicate Procedures Temporally through Logical Coincidence: કોહેશન મજબૂતાઈનો ક્રમ.}
\end{mnemonicbox}

\questionmarks{3(c OR)}{7}{Define risk. Explain risk management.}

\begin{solutionbox}
\textbf{જોખમ વ્યાખ્યા}: સંભવિત સમસ્યા જે સોફ્ટવેર વિકાસ દરમિયાન આવી શકે છે, જે પ્રોજેક્ટની સફળતા પર નકારાત્મક અસર કરે છે.

\begin{center}
\begin{tikzpicture}[node distance=1.5cm, auto, every node/.style={gtu state, align=center, font=\small}]
    \node (Ident) {જોખમ\\ઓળખ};
    \node [right=of Ident] (Assess) {જોખમ\\મૂલ્યાંકન};
    \node [right=of Assess] (Prior) {જોખમ\\પ્રાધાન્યતા};
    \node [below=of Prior] (Mit) {જોખમ\\ઘટાડો};
    \node [left=of Mit] (Mon) {જોખમ\\દેખરેખ};

    \path [gtu arrow] (Ident) -- (Assess);
    \path [gtu arrow] (Assess) -- (Prior);
    \path [gtu arrow] (Prior) -- (Mit);
    \path [gtu arrow] (Mit) -- (Mon);
    \path [gtu arrow] (Mon) -| (Ident);
\end{tikzpicture}
\captionof{figure}{જોખમ વ્યવસ્થાપન પ્રક્રિયા}
\end{center}

\textbf{જોખમ વ્યવસ્થાપન પ્રવૃત્તિઓ}:
\begin{center}
\captionof{table}{જોખમ પ્રવૃત્તિઓ}
\begin{tabulary}{\linewidth}{|L|L|L|}
\hline
\textbf{પ્રવૃત્તિ} & \textbf{વર્ણન} & \textbf{આઉટપુટ} \\ \hline
\textbf{ઓળખ} & સંભવિત સમસ્યાઓ શોધો & જોખમ સૂચિ \\ \hline
\textbf{મૂલ્યાંકન} & સંભાવના અને અસરનું વિશ્લેષણ કરો & જોખમ વિશ્લેષણ \\ \hline
\textbf{પ્રાધાન્યતા} & મહત્વ દ્વારા જોખમોને રેન્ક કરો & પ્રાયોરિટી મેટ્રિક્સ \\ \hline
\textbf{ઘટાડો} & જોખમ પ્રતિભાવોની યોજના બનાવો & મિટિગેશન વ્યૂહરચના \\ \hline
\textbf{દેખરેખ} & જોખમ સ્થિતિ ટ્રૅક કરો & અપડેટ થયેલ જોખમ સ્થિતિ \\ \hline
\end{tabulary}
\end{center}

\begin{itemize}
    \item \keyword{જોખમ પ્રકારો}: ટેકનિકલ, પ્રોજેક્ટ, બિઝનેસ જોખમો
    \item \keyword{વ્યૂહરચના}: ટાળો, સ્થાનાંતરિત કરો, ઓછું કરો, સ્વીકારો
    \item \keyword{સાધનો}: જોખમ મેટ્રિસીસ, સંભાવના-અસર ચાર્ટ
\end{itemize}
\end{solutionbox}

\begin{mnemonicbox}
\mnemonic{Risk Requires Careful Planning: જોખમ વ્યવસ્થાપનનું મહત્વ.}
\end{mnemonicbox}

\questionmarks{4(a)}{3}{Define: Error, Failure, Test case}

\begin{solutionbox}
\begin{center}
\captionof{table}{ટેસ્ટિંગ વ્યાખ્યાઓ}
\begin{tabulary}{\linewidth}{|L|L|}
\hline
\textbf{શબ્દ} & \textbf{વ્યાખ્યા} \\ \hline
\textbf{Error} & સોફ્ટવેર ડેવલપમેન્ટ પ્રક્રિયા દરમિયાન માનવીય ભૂલ \\ \hline
\textbf{Failure} & અપેક્ષિત પરિણામોમાંથી સોફ્ટવેર વર્તણૂકનું વિચલન \\ \hline
\textbf{Test case} & ચોક્કસ કાર્યક્ષમતા અથવા સિસ્ટમ જરૂરિયાતને ચકાસવા માટે શરતોનો સમૂહ \\ \hline
\end{tabulary}
\end{center}

\begin{itemize}
    \item \keyword{સંબંધ}: ભૂલ ખામી તરફ દોરી જાય છે, ખામી નિષ્ફળતાનું કારણ બને છે
    \item \keyword{ભૂલ સ્ત્રોત}: ડેવલપર ભૂલો, જરૂરિયાતોમાં ગેરસમજ
    \item \keyword{ટેસ્ટ કેસ ઘટકો}: ઇનપુટ, અપેક્ષિત આઉટપુટ, અમલીકરણ પગલાં
\end{itemize}
\end{solutionbox}

\begin{mnemonicbox}
\mnemonic{Errors Cause Failures, Tests Catch Problems: કારણ અને અસરની સાંકળ.}
\end{mnemonicbox}

\questionmarks{4(b)}{4}{Identify any six functional requirements of ATM system.}

\begin{solutionbox}
\textbf{ATM સિસ્ટમ કાર્યકારી જરૂરિયાતો}:
\begin{center}
\captionof{table}{ATM જરૂરિયાતો}
\begin{tabulary}{\linewidth}{|L|L|}
\hline
\textbf{જરૂરિયાત} & \textbf{વર્ણન} \\ \hline
\textbf{વપરાશકર્તા ઓથેન્ટિકેશન} & એકાઉન્ટ એક્સેસ માટે PIN વેરિફિકેશન \\ \hline
\textbf{બેલેન્સ ઇન્ક્વાયરી} & વર્તમાન એકાઉન્ટ બેલેન્સ પ્રદર્શિત કરો \\ \hline
\textbf{રોકડ ઉપાડ} & વિનંતી કરેલ રોકડ રકમ વિતરિત કરો \\ \hline
\textbf{ફંડ ટ્રાન્સફર} & એકાઉન્ટ્સ વચ્ચે પૈસા ટ્રાન્સફર કરો \\ \hline
\textbf{ટ્રાન્ઝેક્શન હિસ્ટ્રી} & તાજેતરના ટ્રાન્ઝેક્શન રેકોર્ડ્સ બતાવો \\ \hline
\textbf{PIN ફેરફાર} & વપરાશકર્તાઓને PIN બદલવાની મંજૂરી આપો \\ \hline
\end{tabulary}
\end{center}

\begin{itemize}
    \item \keyword{સુરક્ષા}: તમામ વ્યવહારો માટે ઓથેન્ટિકેશનની જરૂર છે
    \item \keyword{વેલિડેશન}: ઉપાડ પહેલાં પૂરતું બેલેન્સ તપાસો
    \item \keyword{લોગિંગ}: ઓડિટ માટે તમામ વ્યવહારો રેકોર્ડ કરો
\end{itemize}
\end{solutionbox}

\begin{mnemonicbox}
\mnemonic{ATMs Authenticate, Balance, Cash, Transfer, History, PIN: મુખ્ય ATM કાર્યો.}
\end{mnemonicbox}

\questionmarks{4(c)}{7}{State the use of activity network diagram. Develop activity network diagram for the following system and find the critical path for the same.}

\begin{solutionbox}
\textbf{એક્ટિવિટી નેટવર્ક ડાયાગ્રામ ઉપયોગો}:
\begin{itemize}
    \item \keyword{પ્રોજેક્ટ શિડ્યુલિંગ}: પ્રોજેક્ટ સમયરેખા નક્કી કરો
    \item \keyword{ક્રિટિકલ પાથ ઓળખ}: ન્યૂનતમ પ્રોજેક્ટ અવધિ નક્કી કરતો સૌથી લાંબો માર્ગ શોધો
    \item \keyword{સંસાધન આયોજન}: સંસાધન ફાળવણીને શ્રેષ્ઠ બનાવો
\end{itemize}

\textbf{એક્ટિવિટી નેટવર્ક ડાયાગ્રામ}:
\begin{center}
\begin{tikzpicture}[node distance=1.5cm, auto, every node/.style={circle, draw, font=\small}]
    \node (A) {A:2};
    \node [below left=1cm and 0.5cm of A] (B) {B:3};
    \node [right=1.5cm of A] (C) {C:2};
    \node [below right=1cm and 0.5cm of C] (D) {D:4};
    \node [right=1.5cm of C] (E) {E:4};
    \node [below=1cm of D] (F) {F:3};
    \node [right=1.5cm of E] (G) {G:5};
    \node [right=1.5cm of G] (H) {H:2};

    % Connections based on the ASCII art
    \path [gtu arrow] (A) -- (C);
    \path [gtu arrow] (B) -- (C);
    \path [gtu arrow] (B) -- (D);
    \path [gtu arrow] (C) -- (E);
    \path [gtu arrow] (C) -- (D);
    \path [gtu arrow] (D) -- (G);
    \path [gtu arrow] (E) -- (G);
    \path [gtu arrow] (G) -- (H);
    \path [gtu arrow] (F) -- (D); 
\end{tikzpicture}
\captionof{figure}{એક્ટિવિટી નેટવર્ક ડાયાગ્રામ}
\end{center}

\textbf{ક્રિટિકલ પાથ વિશ્લેષણ}:
\begin{center}
\captionof{table}{પાથ વિશ્લેષણ}
\begin{tabulary}{\linewidth}{|L|L|L|L|}
\hline
\textbf{પાથ} & \textbf{પ્રવૃત્તિઓ} & \textbf{અવધિ} & \textbf{ક્રિટિકલ?} \\ \hline
\textbf{A-C-E-G-H} & A$\to$C$\to$E$\to$G$\to$H & 2+2+4+5+2 = 15 & ના \\ \hline
\textbf{B-C-E-G-H} & B$\to$C$\to$E$\to$G$\to$H & 3+2+4+5+2 = 16 & \textbf{હા} \\ \hline
\textbf{A-C-D-G-H} & A$\to$C$\to$D$\to$G$\to$H & 2+2+4+5+2 = 15 & ના \\ \hline
\end{tabulary}
\end{center}

\begin{itemize}
    \item \textbf{ક્રિટિકલ પાથ}: B$\to$C$\to$E$\to$G$\to$H (16 દિવસ)
    \item \textbf{પ્રોજેક્ટ અવધિ}: 16 દિવસ
\end{itemize}
\end{solutionbox}

\begin{mnemonicbox}
\mnemonic{Networks Navigate Project Paths: નેટવર્ક ડાયાગ્રામ્સનું મહત્વ.}
\end{mnemonicbox}

\questionmarks{4(a OR)}{3}{Explain any three requirement gathering activities.}

\begin{solutionbox}
\textbf{જરૂરિયાત એકત્રીકરણ પ્રવૃત્તિઓ}:
\begin{center}
\captionof{table}{એકત્રીકરણ પ્રવૃત્તિઓ}
\begin{tabulary}{\linewidth}{|L|L|L|}
\hline
\textbf{પ્રવૃત્તિ} & \textbf{વર્ણન} & \textbf{આઉટપુટ} \\ \hline
\textbf{સ્ટેકહોલ્ડર ઇન્ટરવ્યુ} & વપરાશકર્તાઓ અને ગ્રાહકો સાથે સીધી ચર્ચા & ઇન્ટરવ્યુ નોંધો, જરૂરિયાતોની સૂચિ \\ \hline
\textbf{પ્રશ્નાવલિ} & મોટા વપરાશકર્તા જૂથો માટે માળખાગત પ્રશ્નો & સર્વે જવાબો, આંકડાકીય ડેટા \\ \hline
\textbf{દસ્તાવેજ વિશ્લેષણ} & હાલના સિસ્ટમ દસ્તાવેજીકરણની સમીક્ષા & વર્તમાન સિસ્ટમ સમજ \\ \hline
\end{tabulary}
\end{center}

\begin{itemize}
    \item \keyword{હેતુ}: વપરાશકર્તાની જરૂરિયાતો અને સિસ્ટમ અપેક્ષાઓને સમજવી
    \item \keyword{સહભાગીઓ}: વપરાશકર્તાઓ, ગ્રાહકો, ડોમેન નિષ્ણાતો, ડેવલપર્સ
    \item \keyword{દસ્તાવેજીકરણ}: SRS દસ્તાવેજમાં રેકોર્ડ થયેલ તમામ તારણો
\end{itemize}
\end{solutionbox}

\begin{mnemonicbox}
\mnemonic{Interviews, Questions, Documents Gather Requirements: એકત્રીકરણ માટેની તકનીકો.}
\end{mnemonicbox}

\questionmarks{4(b OR)}{4}{Develop use case diagram for Bank ATM system.}

\begin{solutionbox}
\textbf{ATM યુઝ કેસ ડાયાગ્રામ}:
\begin{center}
\begin{tikzpicture}[
    actor/.style={shape=circle, draw, align=center, minimum size=0.8cm},
    usecase/.style={shape=ellipse, draw, align=center, font=\footnotesize},
    edge/.style={->, >=stealth}
]
    % Actors
    \node[actor] (Cust) at (0, 0) {Cust};
    \node[actor] (Admin) at (8, 0) {Admin};
    \node[gtu block] (Bank) at (4, -5) {Bank\\System};

    % Customer Use Cases
    \node[usecase] (Check) at (3, 2) {Check\\Balance};
    \node[usecase] (Withdraw) at (3, 1) {Withdraw\\Cash};
    \node[usecase] (Transfer) at (3, 0) {Transfer\\Funds};
    \node[usecase] (Pin) at (4, -1.5) {Change\\PIN};
    \node[usecase] (Receipt) at (4, -3) {Print\\Receipt};
    
    % Admin Use Cases
    \node[usecase] (Load) at (5, 2) {Load\\Cash};
    \node[usecase] (Logs) at (5, 1) {View\\Logs};
    \node[usecase] (Maint) at (5, 0) {Maintenance};

    % Connections
    \draw (Cust) -- (Check);
    \draw (Cust) -- (Withdraw);
    \draw (Cust) -- (Transfer);
    \draw (Cust) -- (Pin);
    \draw (Cust) -- (Receipt);
    
    \draw (Admin) -- (Load);
    \draw (Admin) -- (Logs);
    \draw (Admin) -- (Maint);
    
    \draw[edge, dashed] (Check) -- (Bank);
    \draw[edge, dashed] (Withdraw) -- (Bank);
    \draw[edge, dashed] (Transfer) -- (Bank);
    \draw[edge, dashed] (Pin) -- (Bank);
\end{tikzpicture}
\captionof{figure}{ATM યુઝ કેસ ડાયાગ્રામ}
\end{center}

\textbf{યુઝ કેસ વિગતો}:
\begin{center}
\captionof{table}{યુઝ કેસ સારાંશ}
\begin{tabulary}{\linewidth}{|L|L|}
\hline
\textbf{એક્ટર} & \textbf{યુઝ કેસો} \\ \hline
\textbf{ગ્રાહક} & બેલેન્સ તપાસો, રોકડ ઉપાડો, ફંડ ટ્રાન્સફર કરો, PIN બદલો \\ \hline
\textbf{એડમિન} & રોકડ લોડ કરો, લોગ જુઓ, સિસ્ટમ જાળવણી \\ \hline
\textbf{બેંક સિસ્ટમ} & એકાઉન્ટ્સ માન્ય કરો, વ્યવહારો પર પ્રક્રિયા કરો \\ \hline
\end{tabulary}
\end{center}
\end{solutionbox}

\begin{mnemonicbox}
\mnemonic{Customers Use ATMs, Admins Maintain Systems: એક્ટર્સ અને ભૂમિકાઓ.}
\end{mnemonicbox}

\questionmarks{4(c OR)}{7}{Draw the figure of spiral model. Explain it in brief.}

\begin{solutionbox}
\begin{center}
\begin{tikzpicture}[node distance=2cm, auto, every node/.style={gtu state, align=center, font=\small, minimum size=2cm}]
    \node (Plan) at (0,2) {પ્લાનિંગ};
    \node (Risk) at (2,2) {જોખમ\\વિશ્લેષણ};
    \node (Eng) at (2,0) {એન્જિનિયરિંગ};
    \node (Eval) at (0,0) {ગ્રાહક\\મૂલ્યાંકન};

    \path [gtu arrow] (Plan) -- (Risk);
    \path [gtu arrow] (Risk) -- (Eng);
    \path [gtu arrow] (Eng) -- (Eval);
    \path [gtu arrow] (Eval) -- (Plan);
    
    \node at (1,1) [font=\large\bfseries] {Spiral};
\end{tikzpicture}
\captionof{figure}{સ્પાઇરલ મોડેલ ચતુર્થાંશ}
\end{center}

\textbf{સ્પાઇરલ મોડેલ લાક્ષણિકતાઓ}:
\begin{center}
\captionof{table}{સ્પાઇરલ ચતુર્થાંશ}
\begin{tabulary}{\linewidth}{|L|L|L|}
\hline
\textbf{ચતુર્થાંશ} & \textbf{પ્રવૃત્તિ} & \textbf{હેતુ} \\ \hline
\textbf{પ્લાનિંગ} & ઉદ્દેશ્યો, વિકલ્પો વ્યાખ્યાયિત કરો & પુનરાવર્તન માટે લક્ષ્યો સેટ કરો \\ \hline
\textbf{જોખમ વિશ્લેષણ} & જોખમો ઓળખો અને ઉકેલો & પ્રોજેક્ટ જોખમો ઓછા કરો \\ \hline
\textbf{એન્જિનિયરિંગ} & ઉત્પાદન વિકસાવો અને પરીક્ષણ કરો & કાર્યકારી સોફ્ટવેર બનાવો \\ \hline
\textbf{મૂલ્યાંકન} & ગ્રાહક આકારણી & વપરાશકર્તા પ્રતિસાદ મેળવો \\ \hline
\end{tabulary}
\end{center}

\begin{itemize}
    \item \keyword{મુખ્ય લક્ષણ}: પુનરાવર્તિત વિકાસ સાથે જોખમ-સંચાલિત અભિગમ
    \item \keyword{શ્રેષ્ઠ માટે}: મોટા, જટિલ, ઉચ્ચ જોખમવાળા પ્રોજેક્ટ્સ
    \item \keyword{ફાયદા}: જોખમ વ્યવસ્થાપન, લવચીક, વધતો વિકાસ
    \item \keyword{ગેરફાયદા}: જટિલ સંચાલન, ખર્ચાળ, જોખમ નિપુણતાની જરૂર છે
\end{itemize}
\end{solutionbox}

\begin{mnemonicbox}
\mnemonic{Spirals Plan, Risk, Engineer, Evaluate: ચાર ચતુર્થાંશ.}
\end{mnemonicbox}

\questionmarks{5(a)}{3}{State TRUE or FALSE for the following.}

\begin{solutionbox}
\begin{center}
\captionof{table}{સાચું કે ખોટું}
\begin{tabulary}{\linewidth}{|L|c|L|}
\hline
\textbf{નિવેદન} & \textbf{જવાબ} & \textbf{સમજૂતી} \\ \hline
\textbf{Activity network diagram used to determine critical path} & \textbf{TRUE} & પ્રવૃત્તિ નેટવર્ક્સનો પ્રાથમિક હેતુ \\ \hline
\textbf{In CPM, the shortest path is the critical path} & \textbf{FALSE} & સૌથી લાંબો રસ્તો જટિલ માર્ગ છે \\ \hline
\textbf{Risk avoidance is the best technique to solve risks} & \textbf{FALSE} & શ્રેષ્ઠ તકનીક જોખમ પ્રકાર પર આધારિત છે \\ \hline
\end{tabulary}
\end{center}
\end{solutionbox}

\begin{mnemonicbox}
\mnemonic{True Networks, False Shortest, False Best}
\end{mnemonicbox}

\questionmarks{5(b)}{4}{Identify the differences between traditional model approach and agile approach. (at least 4 differences)}

\begin{solutionbox}
\textbf{પરંપરાગત વિ. Agile સરખામણી}:
\begin{center}
\captionof{table}{પરંપરાગત વિ. Agile}
\begin{tabulary}{\linewidth}{|L|L|L|}
\hline
\textbf{પાસું} & \textbf{પરંપરાગત} & \textbf{Agile} \\ \hline
\textbf{પ્લાનિંગ} & વ્યાપક અપફ્રન્ટ પ્લાનિંગ & અનુકૂલનશીલ આયોજન \\ \hline
\textbf{દસ્તાવેજીકરણ} & ભારે દસ્તાવેજીકરણ & ન્યૂનતમ દસ્તાવેજીકરણ \\ \hline
\textbf{ગ્રાહક સંડોવણી} & જરૂરિયાતોના તબક્કા સુધી મર્યાદિત & સતત સંડોવણી \\ \hline
\textbf{ફેરફાર હેન્ડલિંગ} & મુશ્કેલ અને ખર્ચાળ & ફેરફાર સ્વીકારે છે \\ \hline
\textbf{ડિલિવરી} & સિંગલ ફાઇનલ ડિલિવરી & વારંવાર ઇન્ક્રિમેન્ટલ ડિલિવરી \\ \hline
\textbf{પ્રક્રિયા} & પ્રક્રિયા-સંચાલિત & લોકો-સંચાલિત \\ \hline
\end{tabulary}
\end{center}

\begin{itemize}
    \item \keyword{પરંપરાગત}: આગાહીયુક્ત, ક્રમિક અભિગમ
    \item \keyword{Agile}: અનુકૂલનશીલ, પુનરાવર્તિત અભિગમ
    \item \keyword{સુગમતા}: Agile બદલાતી જરૂરિયાતો માટે વધુ પ્રતિભાવશીલ
\end{itemize}
\end{solutionbox}

\begin{mnemonicbox}
\mnemonic{Traditional Plans Heavy, Agile Adapts Light: મુખ્ય તફાવત ફિલસૂફી.}
\end{mnemonicbox}

\questionmarks{5(c)}{7}{Define unit testing. Draw the figure of it. Explain the process of unit testing.}

\begin{solutionbox}
\textbf{યુનિટ ટેસ્ટિંગ વ્યાખ્યા}: વ્યક્તિગત સોફ્ટવેર ઘટકો અથવા મોડ્યુલોનું અલગતામાં પરીક્ષણ કરવું જેથી તેઓ ડિઝાઇન સ્પષ્ટીકરણો અનુસાર યોગ્ય રીતે કાર્ય કરે છે તેની ખાતરી કરી શકાય.

\begin{center}
\begin{tikzpicture}[node distance=1.5cm, auto, every node/.style={gtu block, align=center, font=\small}]
    \node (Sel) {યુનિટ\\પસંદ કરો};
    \node [right=of Sel] (Des) {ડિઝાઇન\\ટેસ્ટ કેસીસ};
    \node [right=of Des] (Set) {સેટઅપ\\Env};
    \node [below=of Set] (Exec) {ચલાવો\\ટેસ્ટ્સ};
    \node [left=of Exec] (Rec) {રેકોર્ડ\\પરિણામો};
    \node [left=of Rec] (Debug) {ડીબગ અને\\ફિક્સ};
    \node [below=of Exec] (Approve) {યુનિટ\\મંજૂર};

    \path [gtu arrow] (Sel) -- (Des);
    \path [gtu arrow] (Des) -- (Set);
    \path [gtu arrow] (Set) -- (Exec);
    \path [gtu arrow] (Exec) -- (Rec);
    \path [gtu arrow] (Rec) -- (Debug);
    \path [gtu arrow] (Debug) -- (Exec);
    \path [gtu arrow] (Rec) -- node[right]{પાસ} (Approve);
\end{tikzpicture}
\captionof{figure}{યુનિટ ટેસ્ટિંગ પ્રક્રિયા}
\end{center}

\textbf{યુનિટ ટેસ્ટિંગ પ્રક્રિયાના પગલાં}:
\begin{center}
\captionof{table}{યુનિટ ટેસ્ટિંગ પગલાં}
\begin{tabulary}{\linewidth}{|L|L|L|}
\hline
\textbf{પગલું} & \textbf{પ્રવૃત્તિ} & \textbf{હેતુ} \\ \hline
\textbf{ટેસ્ટ પ્લાનિંગ} & પરીક્ષણ કરવા માટે એકમો ઓળખો & પરીક્ષણ અવકાશ વ્યાખ્યાયિત કરો \\ \hline
\textbf{ટેસ્ટ ડિઝાઇન} & ટેસ્ટ કેસ બનાવો & બધા કોડ પાથ આવરી લો \\ \hline
\textbf{ટેસ્ટ સેટઅપ} & ટેસ્ટ પર્યાવરણ તૈયાર કરો & ટેસ્ટ હેઠળના એકમને અલગ કરો \\ \hline
\textbf{ટેસ્ટ અમલીકરણ} & ટેસ્ટ કેસ ચલાવો & યુનિટ વર્તણૂક ચકાસો \\ \hline
\textbf{પરિણામ વિશ્લેષણ} & પરિણામોનું મૂલ્યાંકન કરો & ખામીઓ ઓળખો \\ \hline
\textbf{ખામી ફિક્સિંગ} & મળી આવેલી સમસ્યાઓ સુધારો & યુનિટ ગુણવત્તા ખાતરી કરો \\ \hline
\end{tabulary}
\end{center}

\begin{itemize}
    \item \keyword{ફાયદા}: પ્રારંભિક ખામી શોધ, સરળ ડીબગીંગ, સુધારેલી કોડ ગુણવત્તા
    \item \keyword{સાધનો}: JUnit, NUnit, ઓટોમેટેડ ટેસ્ટિંગ ફ્રેમવર્ક
    \item \keyword{કવરેજ}: ઉચ્ચ કોડ કવરેજ માટે લક્ષ્ય રાખો (સ્ટેટમેન્ટ્સ, બ્રાન્ચીસ, પાથ્સ)
\end{itemize}
\end{solutionbox}

\begin{mnemonicbox}
\mnemonic{Units Test Individual Components Thoroughly: યુનિટ ટેસ્ટિંગનો અર્થ.}
\end{mnemonicbox}

\questionmarks{5(a OR)}{3}{Give the full form of the following.}

\begin{solutionbox}
\begin{center}
\captionof{table}{પૂર્ણ સ્વરૂપો}
\begin{tabulary}{\linewidth}{|L|L|}
\hline
\textbf{સંક્ષેપ} & \textbf{પૂર્ણ સ્વરૂપ} \\ \hline
\textbf{AOA} & Activity On Arrow \\ \hline
\textbf{PERT} & Program Evaluation and Review Technique \\ \hline
\textbf{EVA} & Earned Value Analysis \\ \hline
\textbf{CPM} & Critical Path Method \\ \hline
\textbf{WBS} & Work Breakdown Structure \\ \hline
\textbf{PMC} & Project Management and Control \\ \hline
\end{tabulary}
\end{center}
\end{solutionbox}

\begin{mnemonicbox}
\mnemonic{Activities On Arrows, Programs Evaluate Review Techniques, Earned Values Analyzed, Critical Paths Managed, Work Broken Structured, Projects Managed Controlled}
\end{mnemonicbox}

\questionmarks{5(b OR)}{4}{Explain code inspection.}

\begin{solutionbox}
કોડ ઈન્સ્પેક્શન એ ખામીઓને ઓળખવા અને ગુણવત્તાના ધોરણોની ખાતરી કરવા માટે ટીમના સભ્યો દ્વારા સોર્સ કોડની વ્યવસ્થિત તપાસ છે.

\begin{center}
\captionof{table}{ઈન્સ્પેક્શન પ્રક્રિયા}
\begin{tabulary}{\linewidth}{|L|L|L|}
\hline
\textbf{તબક્કો} & \textbf{પ્રવૃત્તિ} & \textbf{સહભાગીઓ} \\ \hline
\textbf{પ્લાનિંગ} & ઈન્સ્પેક્શન મીટિંગ સુનિશ્ચિત કરો & મોડરેટર \\ \hline
\textbf{તૈયારી} & વ્યક્તિગત રીતે કોડની સમીક્ષા કરો & બધા નિરીક્ષકો \\ \hline
\textbf{ઈન્સ્પેક્શન મીટિંગ} & તારણોની ચર્ચા કરો & ટીમના સભ્યો \\ \hline
\textbf{રીવર્ક} & ઓળખાયેલ સમસ્યાઓ સુધારો & લેખક \\ \hline
\textbf{ફોલો-અપ} & સુધારાઓ ચકાસો & મોડરેટર \\ \hline
\end{tabulary}
\end{center}

\begin{itemize}
    \item \keyword{ફાયદા}: પ્રારંભિક ખામી શોધ, જ્ઞાન વહેંચણી, સુધારેલી કોડ ગુણવત્તા
    \item \keyword{ભૂમિકાઓ}: લેખક, મોડરેટર, સમીક્ષકો, રેકોર્ડર
    \item \keyword{ફોકસ વિસ્તારો}: લોજિક ભૂલો, કોડિંગ ધોરણો, જાળવણીક્ષમતા
\end{itemize}
\end{solutionbox}

\begin{mnemonicbox}
\mnemonic{Inspections Improve Code Quality: ઈન્સ્પેક્શનનો હેતુ.}
\end{mnemonicbox}

\questionmarks{5(c OR)}{7}{Define white box testing method. Explain different white box testing methods.}

\begin{solutionbox}
\textbf{વ્હાઇટ બોક્સ ટેસ્ટિંગ વ્યાખ્યા}: ટેસ્ટિંગ પદ્ધતિ જે સંપૂર્ણ કવરેજ સુનિશ્ચિત કરવા માટે આંતરિક કોડ સ્ટ્રક્ચર, લોજિક પાથ અને અમલીકરણ વિગતોની તપાસ કરે છે.

\textbf{વ્હાઇટ બોક્સ ટેસ્ટિંગ પદ્ધતિઓ}:
\begin{center}
\captionof{table}{વ્હાઇટ બોક્સ પદ્ધતિઓ}
\begin{tabulary}{\linewidth}{|L|L|L|}
\hline
\textbf{પદ્ધતિ} & \textbf{વર્ણન} & \textbf{કવરેજ ફોકસ} \\ \hline
\textbf{સ્ટેટમેન્ટ કવરેજ} & દરેક નિવેદનને અમલમાં મૂકો & બધા કોડ લાઇન \\ \hline
\textbf{બ્રાન્ચ કવરેજ} & તમામ નિર્ણય પરિણામોનું પરીક્ષણ કરો & જો-તો શરતો \\ \hline
\textbf{પાથ કવરેજ} & તમામ સંભવિત માર્ગો ચલાવો & સંપૂર્ણ અમલીકરણ પ્રવાહ \\ \hline
\textbf{કન્ડિશન કવરેજ} & તમામ શરત સંયોજનોનું પરીક્ષણ કરો & બુલિયન અભિવ્યક્તિઓ \\ \hline
\end{tabulary}
\end{center}

\begin{center}
\begin{tikzpicture}[
    mindmap,
    concept color=green!30,
    every node/.style={concept, execute at begin node=\hskip0pt},
    root concept/.append style={concept, color=green!50, minimum size=3cm, font=\bfseries},
    level 1 concept/.append style={level distance=3.5cm, sibling angle=90, color=green!20}
]
    \node [root concept] {વ્હાઇટ બોક્સ\\ટેસ્ટિંગ}
        child { node {સ્ટેટમેન્ટ\\ટેસ્ટિંગ} }
        child { node {બ્રાન્ચ\\ટેસ્ટિંગ} }
        child { node {પાથ\\ટેસ્ટિંગ} }
        child { node {કન્ડિશન\\ટેસ્ટિંગ} };
\end{tikzpicture}
\captionof{figure}{વ્હાઇટ બોક્સ ટેસ્ટિંગ તકનીકો}
\end{center}

\begin{itemize}
    \item \keyword{કવરેજ વિશ્લેષણ}: ટેસ્ટિંગની અસરકારકતા માપે છે (દા.ત., એક્ઝિક્યુટેડ સ્ટેટમેન્ટ્સ / કુલ સ્ટેટમેન્ટ્સ)
    \item \keyword{ફાયદા}: સંપૂર્ણ ટેસ્ટિંગ, મૃત કોડ ઓળખે છે, ગુણવત્તા સુનિશ્ચિત કરે છે
    \item \keyword{ગેરફાયદા}: કોડ જ્ઞાનની જરૂર, સમય લેતું
\end{itemize}
\end{solutionbox}

\begin{mnemonicbox}
\mnemonic{White Box Sees Inside Code Structure: મુખ્ય ખ્યાલ.}
\end{mnemonicbox}

\end{document}
