\documentclass[10pt,a4paper]{article}

% content/resources/templates/preamble.tex
\usepackage[margin=0.6in]{geometry}
\author{Milav Dabgar}
\usepackage{amsmath,amssymb,amsthm}
\usepackage{booktabs}
\usepackage{multirow}
\usepackage{xcolor}
\usepackage{tcolorbox}
\tcbuselibrary{breakable,skins}
\usepackage[colorlinks=true,linkcolor=blue]{hyperref}
\usepackage{titlesec}
\usepackage{enumitem}
\usepackage{tikz}
\usepackage{pgfplots}
\usepackage{circuitikz}
\usepackage[version=4]{mhchem}
\usepackage{longtable}
\usepackage{array}
\usepackage{float}
\usepackage{caption}
\usepackage{listings}

\lstset{
  basicstyle=\small\ttfamily,
  breaklines=true,
  breakatwhitespace=false,
  postbreak=\mbox{\textcolor{red}{$\hookrightarrow$}\space},
  float=false,
  numbers=left,
  numberstyle=\tiny\color{gray},
  numbersep=10pt,
  xleftmargin=2em,
  keywordstyle=\color{blue},
  commentstyle=\color{green!60!black},
  stringstyle=\color{purple},
  backgroundcolor=\color{gray!5},
  showstringspaces=false,
  tabsize=2,
  captionpos=b,
  keepspaces=true,
  columns=flexible
}

\pgfplotsset{compat=1.18}
\usetikzlibrary{shapes,arrows,positioning,calc,patterns,decorations.pathmorphing,decorations.markings,arrows.meta}

% Color scheme
\definecolor{headcolor}{RGB}{0,102,204}
\definecolor{keycolor}{RGB}{220,20,60}
\definecolor{solutioncolor}{RGB}{34,139,34}
\definecolor{mnemoniccolor}{RGB}{148,0,211}
\definecolor{codecolor}{RGB}{0,0,100}

% Spacing
\setlength{\parskip}{3pt}
\setlist[itemize]{nosep}
\setlist[enumerate]{nosep}

% Title formatting
\titleformat{\section}{\Large\bfseries\color{headcolor}}{\thesection}{1em}{}
\titleformat{\subsection}{\large\bfseries\color{headcolor}}{\thesubsection}{1em}{}

% Pandoc tightlist compatibility
\providecommand{\tightlist}{%
  \setlength{\itemsep}{0pt}\setlength{\parskip}{0pt}}

% Pandoc longtable compatibility
\newcounter{none}
\def\thenone{}


% content/resources/templates/english-boxes.tex
% This file is currently empty - it exists to maintain consistency with the import structure.
% Add custom environments here if needed in the future.


\begin{document}

\begin{center}
{\Huge\bfseries\color{headcolor} Subject Name Solutions}\\[5pt]
{\LARGE 4331604 -- Summer 2025}\\[3pt]
{\large Semester 1 Study Material}\\[3pt]
{\normalsize\textit{Detailed Solutions and Explanations}}
\end{center}

\vspace{10pt}

\subsection*{Question 1(a) [3 marks]}\label{q1a}

\textbf{Give IEEE definition of software. Write one example of each for
application and system software.}

\begin{solutionbox}

\textbf{IEEE Definition}: Software is a collection of computer programs,
procedures, rules, and associated documentation and data.

\textbf{Examples}:

{\def\LTcaptype{none} % do not increment counter
\begin{longtable}[]{@{}
  >{\raggedright\arraybackslash}p{(\linewidth - 4\tabcolsep) * \real{0.4545}}
  >{\raggedright\arraybackslash}p{(\linewidth - 4\tabcolsep) * \real{0.2727}}
  >{\raggedright\arraybackslash}p{(\linewidth - 4\tabcolsep) * \real{0.2727}}@{}}
\toprule\noalign{}
\begin{minipage}[b]{\linewidth}\raggedright
Software Type
\end{minipage} & \begin{minipage}[b]{\linewidth}\raggedright
Example
\end{minipage} & \begin{minipage}[b]{\linewidth}\raggedright
Purpose
\end{minipage} \\
\midrule\noalign{}
\endhead
\bottomrule\noalign{}
\endlastfoot
\textbf{Application Software} & Microsoft Word & Word processing and
document creation \\
\textbf{System Software} & Windows 10 & Operating system managing
hardware resources \\
\end{longtable}
}

\begin{itemize}
\tightlist
\item
  \textbf{Application software}: Programs designed for end-users to
  accomplish specific tasks
\item
  \textbf{System software}: Programs that manage and operate computer
  hardware
\end{itemize}

\end{solutionbox}
\begin{mnemonicbox}
``Apps help Users, Systems help Hardware''

\end{mnemonicbox}
\begin{center}\rule{0.5\linewidth}{0.5pt}\end{center}

\subsection*{Question 1(b) [4 marks]}\label{q1b}

\textbf{Write a short note on data dictionary.}

\begin{solutionbox}

Data dictionary is a centralized repository containing definitions and
characteristics of data elements used in a system.

\textbf{Components Table}:

{\def\LTcaptype{none} % do not increment counter
\begin{longtable}[]{@{}ll@{}}
\toprule\noalign{}
Component & Description \\
\midrule\noalign{}
\endhead
\bottomrule\noalign{}
\endlastfoot
\textbf{Data Name} & Unique identifier for data element \\
\textbf{Aliases} & Alternative names used \\
\textbf{Description} & Purpose and meaning \\
\textbf{Data Type} & Format (integer, string, etc.) \\
\textbf{Length} & Size constraints \\
\textbf{Values} & Valid range or set \\
\end{longtable}
}

\begin{itemize}
\tightlist
\item
  \textbf{Purpose}: Ensures consistency in data usage across development
  team
\item
  \textbf{Benefits}: Reduces ambiguity, improves communication,
  standardizes data definitions
\item
  \textbf{Usage}: Referenced during system design and database creation
\end{itemize}

\end{solutionbox}
\begin{mnemonicbox}
``Dictionary Defines Data Clearly''

\end{mnemonicbox}
\begin{center}\rule{0.5\linewidth}{0.5pt}\end{center}

\subsection*{Question 1(c) [7 marks]}\label{q1c}

\textbf{Explain prototype model with figure.}

\begin{solutionbox}

Prototype model is an iterative approach where a working model is built
early to understand requirements better.

\textbf{Diagram}:

\begin{center}
\textbf{Mermaid Diagram (Code)}
\begin{verbatim}
{Shaded}
{Highlighting}[]
graph LR
    A[Requirement Gathering] {-{-}{} B[Quick Design]}
    B {-{-}{} C[Build Prototype]}
    C {-{-}{} D[User Evaluation]}
    D {-{-}{} E\{User Satisfied?\}}
    E {-{-}{}|No| F[Refine Requirements]}
    F {-{-}{} B}
    E {-{-}{}|Yes| G[Final System Development]}
    G {-{-}{} H[Testing \& Maintenance]}
{Highlighting}
{Shaded}
\end{verbatim}
\end{center}

\textbf{Characteristics}:

{\def\LTcaptype{none} % do not increment counter
\begin{longtable}[]{@{}lll@{}}
\toprule\noalign{}
Phase & Activity & Output \\
\midrule\noalign{}
\endhead
\bottomrule\noalign{}
\endlastfoot
\textbf{Quick Design} & Basic architecture & Initial design \\
\textbf{Prototype Build} & Working model & Testable system \\
\textbf{User Evaluation} & Feedback collection & Requirements
refinement \\
\end{longtable}
}

\begin{itemize}
\tightlist
\item
  \textbf{Advantages}: Early user feedback, reduced development risk,
  better requirement understanding
\item
  \textbf{Disadvantages}: May lead to inadequate analysis, customer
  expects prototype as final product
\item
  \textbf{Best for}: Projects with unclear requirements
\end{itemize}

\end{solutionbox}
\begin{mnemonicbox}
``Prototype Proves Possibilities''

\end{mnemonicbox}
\begin{center}\rule{0.5\linewidth}{0.5pt}\end{center}

\subsection*{Question 1(c) OR [7
marks]}\label{q1c}

\textbf{Explain RAD model with advantages and disadvantages.}

\begin{solutionbox}

RAD (Rapid Application Development) emphasizes quick development through
prototyping and iterative development.

\textbf{RAD Phases}:

\begin{center}
\textbf{Mermaid Diagram (Code)}
\begin{verbatim}
{Shaded}
{Highlighting}[]
graph LR
    A[Business Modeling] {-{-}{} B[Data Modeling]}
    B {-{-}{} C[Process Modeling]}
    C {-{-}{} D[Application Generation]}
    D {-{-}{} E[Testing \& Turnover]}
{Highlighting}
{Shaded}
\end{verbatim}
\end{center}

\textbf{Advantages vs Disadvantages}:

{\def\LTcaptype{none} % do not increment counter
\begin{longtable}[]{@{}ll@{}}
\toprule\noalign{}
Advantages & Disadvantages \\
\midrule\noalign{}
\endhead
\bottomrule\noalign{}
\endlastfoot
\textbf{Faster development} & \textbf{Requires skilled developers} \\
\textbf{Early user involvement} & \textbf{Not suitable for large
projects} \\
\textbf{Reduced costs} & \textbf{Requires user commitment} \\
\textbf{Better quality} & \textbf{Technical risks if not managed} \\
\end{longtable}
}

\begin{itemize}
\tightlist
\item
  \textbf{Key feature}: Uses automated tools and 4GL programming
\item
  \textbf{Timeline}: Typically 60-90 days for development
\item
  \textbf{Team}: Small, experienced development teams
\end{itemize}

\end{solutionbox}
\begin{mnemonicbox}
``RAD Rapidly Accelerates Development''

\end{mnemonicbox}
\begin{center}\rule{0.5\linewidth}{0.5pt}\end{center}

\subsection*{Question 2(a) [3 marks]}\label{q2a}

\textbf{Give the full form of following: SQA, FTR, RAD, BVA, GUI, DFD}

\begin{solutionbox}

{\def\LTcaptype{none} % do not increment counter
\begin{longtable}[]{@{}ll@{}}
\toprule\noalign{}
Abbreviation & Full Form \\
\midrule\noalign{}
\endhead
\bottomrule\noalign{}
\endlastfoot
\textbf{SQA} & Software Quality Assurance \\
\textbf{FTR} & Formal Technical Review \\
\textbf{RAD} & Rapid Application Development \\
\textbf{BVA} & Boundary Value Analysis \\
\textbf{GUI} & Graphical User Interface \\
\textbf{DFD} & Data Flow Diagram \\
\end{longtable}
}

\end{solutionbox}
\begin{mnemonicbox}
``Software Quality And Formal Technical Reviews
Rapidly Analyze Development, Boundary Value Analysis Guides User
Interface, Data Flow Diagrams''

\end{mnemonicbox}
\begin{center}\rule{0.5\linewidth}{0.5pt}\end{center}

\subsection*{Question 2(b) [4 marks]}\label{q2b}

\textbf{Define agile methodology. Discuss agile principles.}

\begin{solutionbox}

\textbf{Definition}: Agile is an iterative software development approach
emphasizing collaboration, flexibility, and rapid delivery of working
software.

\textbf{Core Agile Principles}:

{\def\LTcaptype{none} % do not increment counter
\begin{longtable}[]{@{}
  >{\raggedright\arraybackslash}p{(\linewidth - 2\tabcolsep) * \real{0.4583}}
  >{\raggedright\arraybackslash}p{(\linewidth - 2\tabcolsep) * \real{0.5417}}@{}}
\toprule\noalign{}
\begin{minipage}[b]{\linewidth}\raggedright
Principle
\end{minipage} & \begin{minipage}[b]{\linewidth}\raggedright
Description
\end{minipage} \\
\midrule\noalign{}
\endhead
\bottomrule\noalign{}
\endlastfoot
\textbf{Individuals over processes} & People and communication are
priority \\
\textbf{Working software over documentation} & Functional software is
primary measure \\
\textbf{Customer collaboration} & Continuous customer involvement \\
\textbf{Responding to change} & Adaptability over rigid plans \\
\end{longtable}
}

\begin{itemize}
\tightlist
\item
  \textbf{Iteration length}: Typically 2-4 weeks (sprints)
\item
  \textbf{Delivery}: Frequent working software releases
\item
  \textbf{Team structure}: Cross-functional, self-organizing teams
\end{itemize}

\end{solutionbox}
\begin{mnemonicbox}
``Agile Adapts And Advances''

\end{mnemonicbox}
\begin{center}\rule{0.5\linewidth}{0.5pt}\end{center}

\subsection*{Question 2(c) [7 marks]}\label{q2c}

\textbf{Explain XP model with its advantages and disadvantages.}

\begin{solutionbox}

XP (Extreme Programming) is an agile methodology emphasizing engineering
practices and customer satisfaction.

\textbf{XP Practices}:

\begin{verbatim}
mindmap
  root((XP Practices))
    Planning Game
    Small Releases
    Pair Programming
    Test{-Driven Development}
    Continuous Integration
    Refactoring
    Simple Design
    Collective Code Ownership
\end{verbatim}

\textbf{Advantages and Disadvantages}:

{\def\LTcaptype{none} % do not increment counter
\begin{longtable}[]{@{}ll@{}}
\toprule\noalign{}
Advantages & Disadvantages \\
\midrule\noalign{}
\endhead
\bottomrule\noalign{}
\endlastfoot
\textbf{High code quality} & \textbf{Requires experienced
programmers} \\
\textbf{Rapid feedback} & \textbf{Customer must be available} \\
\textbf{Reduced bugs} & \textbf{Code-focused, less documentation} \\
\textbf{Flexibility} & \textbf{Difficult to estimate costs} \\
\end{longtable}
}

\begin{itemize}
\tightlist
\item
  \textbf{Key practice}: Pair programming ensures code quality
\item
  \textbf{Testing}: Test-first approach with automated testing
\item
  \textbf{Customer role}: On-site customer provides continuous feedback
\end{itemize}

\end{solutionbox}
\begin{mnemonicbox}
``eXtreme Programming eXcels through Practices''

\end{mnemonicbox}
\begin{center}\rule{0.5\linewidth}{0.5pt}\end{center}

\subsection*{Question 2(a) OR [3
marks]}\label{q2a}

\textbf{Define black box testing. Give at least two names of black box
testing method.}

\begin{solutionbox}

\textbf{Definition}: Black box testing examines software functionality
without knowledge of internal code structure, focusing on input-output
behavior.

\textbf{Black Box Testing Methods}:

{\def\LTcaptype{none} % do not increment counter
\begin{longtable}[]{@{}
  >{\raggedright\arraybackslash}p{(\linewidth - 2\tabcolsep) * \real{0.3810}}
  >{\raggedright\arraybackslash}p{(\linewidth - 2\tabcolsep) * \real{0.6190}}@{}}
\toprule\noalign{}
\begin{minipage}[b]{\linewidth}\raggedright
Method
\end{minipage} & \begin{minipage}[b]{\linewidth}\raggedright
Description
\end{minipage} \\
\midrule\noalign{}
\endhead
\bottomrule\noalign{}
\endlastfoot
\textbf{Equivalence Partitioning} & Divides input into valid/invalid
classes \\
\textbf{Boundary Value Analysis} & Tests values at input boundaries \\
\end{longtable}
}

\begin{itemize}
\tightlist
\item
  \textbf{Approach}: Tests based on requirements and specifications
\item
  \textbf{Tester knowledge}: No internal code knowledge required
\item
  \textbf{Focus}: External behavior and functionality
\end{itemize}

\end{solutionbox}
\begin{mnemonicbox}
``Black Box Behavior Based''

\end{mnemonicbox}
\begin{center}\rule{0.5\linewidth}{0.5pt}\end{center}

\subsection*{Question 2(b) OR [4
marks]}\label{q2b}

\textbf{Give the full form of CLI. Explain CLI in brief.}

\begin{solutionbox}

\textbf{CLI}: Command Line Interface

\textbf{CLI Characteristics}:

{\def\LTcaptype{none} % do not increment counter
\begin{longtable}[]{@{}ll@{}}
\toprule\noalign{}
Aspect & Description \\
\midrule\noalign{}
\endhead
\bottomrule\noalign{}
\endlastfoot
\textbf{Input method} & Text commands typed by user \\
\textbf{Output} & Text-based responses \\
\textbf{Navigation} & Commands for file/directory operations \\
\textbf{Efficiency} & Faster for experienced users \\
\end{longtable}
}

\begin{itemize}
\tightlist
\item
  \textbf{Advantages}: Fast execution, less memory usage, scriptable
\item
  \textbf{Disadvantages}: Requires learning commands, not user-friendly
  for beginners
\item
  \textbf{Examples}: Windows Command Prompt, Linux Terminal, DOS
\end{itemize}

\end{solutionbox}
\begin{mnemonicbox}
``Commands Lead Interaction''

\end{mnemonicbox}
\begin{center}\rule{0.5\linewidth}{0.5pt}\end{center}

\subsection*{Question 2(c) OR [7
marks]}\label{q2c}

\textbf{Explain waterfall model with neat figure.}

\begin{solutionbox}

Waterfall model is a linear sequential approach where each phase must be
completed before moving to the next.

\textbf{Waterfall Model Diagram}:

\begin{center}
\textbf{Mermaid Diagram (Code)}
\begin{verbatim}
{Shaded}
{Highlighting}[]
graph LR
    A[Requirement Analysis] {-{-}{} B[System Design]}
    B {-{-}{} C[Implementation]}
    C {-{-}{} D[Integration \& Testing]}
    D {-{-}{} E[Deployment]}
    E {-{-}{} F[Maintenance]}
    
    style A fill:\#e1f5fe
    style B fill:\#f3e5f5
    style C fill:\#fff3e0
    style D fill:\#f1f8e9
    style E fill:\#fce4ec
    style F fill:\#fff8e1
{Highlighting}
{Shaded}
\end{verbatim}
\end{center}

\textbf{Phase Details}:

{\def\LTcaptype{none} % do not increment counter
\begin{longtable}[]{@{}lll@{}}
\toprule\noalign{}
Phase & Activities & Deliverables \\
\midrule\noalign{}
\endhead
\bottomrule\noalign{}
\endlastfoot
\textbf{Requirements} & Gather and document needs & SRS document \\
\textbf{Design} & System architecture & Design documents \\
\textbf{Implementation} & Code development & Source code \\
\textbf{Testing} & Verify functionality & Test reports \\
\textbf{Deployment} & System installation & Working system \\
\textbf{Maintenance} & Bug fixes, updates & Updated system \\
\end{longtable}
}

\begin{itemize}
\tightlist
\item
  \textbf{Advantages}: Simple, easy to manage, well-documented
\item
  \textbf{Disadvantages}: Inflexible, late testing, difficult to
  accommodate changes
\end{itemize}

\end{solutionbox}
\begin{mnemonicbox}
``Water Always Flows Downward''

\end{mnemonicbox}
\begin{center}\rule{0.5\linewidth}{0.5pt}\end{center}

\subsection*{Question 3(a) [3 marks]}\label{q3a}

\textbf{Give one word answer:}

\begin{solutionbox}

{\def\LTcaptype{none} % do not increment counter
\begin{longtable}[]{@{}ll@{}}
\toprule\noalign{}
Question & Answer \\
\midrule\noalign{}
\endhead
\bottomrule\noalign{}
\endlastfoot
\textbf{Lowest cohesion is} & Coincidental \\
\textbf{Highest coupling is} & Content \\
\textbf{Slack time of critical activity is} & Zero \\
\end{longtable}
}

\end{solutionbox}
\begin{mnemonicbox}
``Coincidental Cohesion, Content Coupling, Critical
Zero''

\end{mnemonicbox}
\begin{center}\rule{0.5\linewidth}{0.5pt}\end{center}

\subsection*{Question 3(b) [4 marks]}\label{q3b}

\textbf{Explain classification of coupling.}

\begin{solutionbox}

Coupling measures interdependence between modules. Lower coupling is
better for maintainability.

\textbf{Coupling Types (Best to Worst)}:

{\def\LTcaptype{none} % do not increment counter
\begin{longtable}[]{@{}
  >{\raggedright\arraybackslash}p{(\linewidth - 4\tabcolsep) * \real{0.2143}}
  >{\raggedright\arraybackslash}p{(\linewidth - 4\tabcolsep) * \real{0.4643}}
  >{\raggedright\arraybackslash}p{(\linewidth - 4\tabcolsep) * \real{0.3214}}@{}}
\toprule\noalign{}
\begin{minipage}[b]{\linewidth}\raggedright
Type
\end{minipage} & \begin{minipage}[b]{\linewidth}\raggedright
Description
\end{minipage} & \begin{minipage}[b]{\linewidth}\raggedright
Example
\end{minipage} \\
\midrule\noalign{}
\endhead
\bottomrule\noalign{}
\endlastfoot
\textbf{Data} & Parameters passed & Method calls with parameters \\
\textbf{Stamp} & Data structure passed & Passing objects/records \\
\textbf{Control} & Control information passed & Flags/switches passed \\
\textbf{External} & External data reference & Global variables \\
\textbf{Common} & Shared data area & Common memory blocks \\
\textbf{Content} & Direct access to internals & Modifying another
module's data \\
\end{longtable}
}

\begin{itemize}
\tightlist
\item
  \textbf{Best practice}: Aim for data coupling
\item
  \textbf{Avoid}: Content and common coupling
\item
  \textbf{Design goal}: Minimize dependencies between modules
\end{itemize}

\end{solutionbox}
\begin{mnemonicbox}
``Data Stamps Control External Common Content''

\end{mnemonicbox}
\begin{center}\rule{0.5\linewidth}{0.5pt}\end{center}

\subsection*{Question 3(c) [7 marks]}\label{q3c}

\textbf{Define following terms (don't just give the full form):}

\begin{solutionbox}

{\def\LTcaptype{none} % do not increment counter
\begin{longtable}[]{@{}
  >{\raggedright\arraybackslash}p{(\linewidth - 2\tabcolsep) * \real{0.3333}}
  >{\raggedright\arraybackslash}p{(\linewidth - 2\tabcolsep) * \real{0.6667}}@{}}
\toprule\noalign{}
\begin{minipage}[b]{\linewidth}\raggedright
Term
\end{minipage} & \begin{minipage}[b]{\linewidth}\raggedright
Definition
\end{minipage} \\
\midrule\noalign{}
\endhead
\bottomrule\noalign{}
\endlastfoot
\textbf{UI} & User Interface - the means by which users interact with
software systems \\
\textbf{SE} & Software Engineering - systematic approach to software
development using engineering principles \\
\textbf{PMC} & Project Management and Control - planning, monitoring,
and controlling software projects \\
\textbf{SDLC} & Software Development Life Cycle - phases involved in
software development from conception to maintenance \\
\textbf{Verification} & Process of checking if software meets specified
requirements and design \\
\textbf{Validation} & Process of checking if software meets user needs
and intended purpose \\
\textbf{SRS} & Software Requirements Specification - detailed document
describing software functionality and constraints \\
\end{longtable}
}

\begin{itemize}
\tightlist
\item
  \textbf{Verification}: ``Are we building the product right?''
\item
  \textbf{Validation}: ``Are we building the right product?''
\item
  \textbf{Key difference}: Verification checks specifications,
  Validation checks user satisfaction
\end{itemize}

\end{solutionbox}
\begin{mnemonicbox}
``Users Interact, Software Engineers Plan, Managing
Cycles, Specifications Define, Verification checks Requirements,
Validation checks Satisfaction, Requirements Specify Software''

\end{mnemonicbox}
\begin{center}\rule{0.5\linewidth}{0.5pt}\end{center}

\subsection*{Question 3(a) OR [3
marks]}\label{q3a}

\textbf{Explain menu based UI with advantages and disadvantages.}

\begin{solutionbox}

Menu-based UI presents options in hierarchical menus for user selection.

\textbf{Advantages vs Disadvantages}:

{\def\LTcaptype{none} % do not increment counter
\begin{longtable}[]{@{}ll@{}}
\toprule\noalign{}
Advantages & Disadvantages \\
\midrule\noalign{}
\endhead
\bottomrule\noalign{}
\endlastfoot
\textbf{Easy to learn} & \textbf{Slower for experts} \\
\textbf{Reduces errors} & \textbf{Limited flexibility} \\
\textbf{Self-explanatory} & \textbf{Screen space consumption} \\
\end{longtable}
}

\begin{itemize}
\tightlist
\item
  \textbf{Structure}: Hierarchical organization of options
\item
  \textbf{Navigation}: Point-and-click or keyboard shortcuts
\item
  \textbf{Best for}: Applications with well-defined functions
\end{itemize}

\end{solutionbox}
\begin{mnemonicbox}
``Menus Make Choices Clear''

\end{mnemonicbox}
\begin{center}\rule{0.5\linewidth}{0.5pt}\end{center}

\subsection*{Question 3(b) OR [4
marks]}\label{q3b}

\textbf{Explain classification of cohesion.}

\begin{solutionbox}

Cohesion measures how closely related elements within a module are.
Higher cohesion is better.

\textbf{Cohesion Types (Best to Worst)}:

{\def\LTcaptype{none} % do not increment counter
\begin{longtable}[]{@{}ll@{}}
\toprule\noalign{}
Type & Description \\
\midrule\noalign{}
\endhead
\bottomrule\noalign{}
\endlastfoot
\textbf{Functional} & Single, well-defined task \\
\textbf{Sequential} & Output of one element feeds next \\
\textbf{Communicational} & Elements work on same data \\
\textbf{Procedural} & Elements follow execution sequence \\
\textbf{Temporal} & Elements executed at same time \\
\textbf{Logical} & Elements perform similar functions \\
\textbf{Coincidental} & Elements randomly grouped \\
\end{longtable}
}

\begin{itemize}
\tightlist
\item
  \textbf{Goal}: Achieve functional cohesion
\item
  \textbf{Design principle}: Each module should have single
  responsibility
\item
  \textbf{Measurement}: Higher cohesion = better design
\end{itemize}

\end{solutionbox}
\begin{mnemonicbox}
``Functional Sequences Communicate Procedures
Temporally through Logical Coincidence''

\end{mnemonicbox}
\begin{center}\rule{0.5\linewidth}{0.5pt}\end{center}

\subsection*{Question 3(c) OR [7
marks]}\label{q3c}

\textbf{Define risk. Explain risk management.}

\begin{solutionbox}

\textbf{Risk Definition}: Potential problem that may occur during
software development, causing negative impact on project success.

\textbf{Risk Management Process}:

\begin{center}
\textbf{Mermaid Diagram (Code)}
\begin{verbatim}
{Shaded}
{Highlighting}[]
graph LR
    A[Risk Identification] {-{-}{} B[Risk Assessment]}
    B {-{-}{} C[Risk Prioritization]}
    C {-{-}{} D[Risk Mitigation]}
    D {-{-}{} E[Risk Monitoring]}
    E {-{-}{} A}
{Highlighting}
{Shaded}
\end{verbatim}
\end{center}

\textbf{Risk Management Activities}:

{\def\LTcaptype{none} % do not increment counter
\begin{longtable}[]{@{}lll@{}}
\toprule\noalign{}
Activity & Description & Output \\
\midrule\noalign{}
\endhead
\bottomrule\noalign{}
\endlastfoot
\textbf{Identification} & Find potential problems & Risk list \\
\textbf{Assessment} & Analyze probability and impact & Risk analysis \\
\textbf{Prioritization} & Rank risks by importance & Priority matrix \\
\textbf{Mitigation} & Plan risk responses & Mitigation strategies \\
\textbf{Monitoring} & Track risk status & Updated risk status \\
\end{longtable}
}

\begin{itemize}
\tightlist
\item
  \textbf{Risk types}: Technical, Project, Business risks
\item
  \textbf{Strategies}: Avoid, Transfer, Mitigate, Accept
\item
  \textbf{Tools}: Risk matrices, probability-impact charts
\end{itemize}

\end{solutionbox}
\begin{mnemonicbox}
``Risk Requires Careful Planning''

\end{mnemonicbox}
\begin{center}\rule{0.5\linewidth}{0.5pt}\end{center}

\subsection*{Question 4(a) [3 marks]}\label{q4a}

\textbf{Define: Error, Failure, Test case}

\begin{solutionbox}

{\def\LTcaptype{none} % do not increment counter
\begin{longtable}[]{@{}
  >{\raggedright\arraybackslash}p{(\linewidth - 2\tabcolsep) * \real{0.3333}}
  >{\raggedright\arraybackslash}p{(\linewidth - 2\tabcolsep) * \real{0.6667}}@{}}
\toprule\noalign{}
\begin{minipage}[b]{\linewidth}\raggedright
Term
\end{minipage} & \begin{minipage}[b]{\linewidth}\raggedright
Definition
\end{minipage} \\
\midrule\noalign{}
\endhead
\bottomrule\noalign{}
\endlastfoot
\textbf{Error} & Human mistake made during software development
process \\
\textbf{Failure} & Deviation of software behavior from expected
results \\
\textbf{Test case} & Set of conditions to verify specific functionality
or system requirement \\
\end{longtable}
}

\begin{itemize}
\tightlist
\item
  \textbf{Relationship}: Error leads to defect, defect causes failure
\item
  \textbf{Error source}: Developer mistakes, misunderstanding
  requirements
\item
  \textbf{Test case components}: Input, expected output, execution steps
\end{itemize}

\end{solutionbox}
\begin{mnemonicbox}
``Errors Cause Failures, Tests Catch Problems''

\end{mnemonicbox}
\begin{center}\rule{0.5\linewidth}{0.5pt}\end{center}

\subsection*{Question 4(b) [4 marks]}\label{q4b}

\textbf{Identify any six functional requirements of ATM system.}

\begin{solutionbox}

\textbf{ATM System Functional Requirements}:

{\def\LTcaptype{none} % do not increment counter
\begin{longtable}[]{@{}ll@{}}
\toprule\noalign{}
Requirement & Description \\
\midrule\noalign{}
\endhead
\bottomrule\noalign{}
\endlastfoot
\textbf{User Authentication} & PIN verification for account access \\
\textbf{Balance Inquiry} & Display current account balance \\
\textbf{Cash Withdrawal} & Dispense requested cash amount \\
\textbf{Fund Transfer} & Transfer money between accounts \\
\textbf{Transaction History} & Show recent transaction records \\
\textbf{PIN Change} & Allow users to modify PIN \\
\end{longtable}
}

\begin{itemize}
\tightlist
\item
  \textbf{Security}: All transactions require authentication
\item
  \textbf{Validation}: Check sufficient balance before withdrawal
\item
  \textbf{Logging}: Record all transactions for audit
\end{itemize}

\end{solutionbox}
\begin{mnemonicbox}
``ATMs Authenticate, Balance, Cash, Transfer,
History, PIN''

\end{mnemonicbox}
\begin{center}\rule{0.5\linewidth}{0.5pt}\end{center}

\subsection*{Question 4(c) [7 marks]}\label{q4c}

\textbf{State the use of activity network diagram. Develop activity
network diagram for the following system and find the critical path for
the same.}

\begin{solutionbox}

\textbf{Activity Network Diagram Uses}:

\begin{itemize}
\tightlist
\item
  \textbf{Project scheduling}: Determine project timeline
\item
  \textbf{Critical path identification}: Find longest path determining
  minimum project duration
\item
  \textbf{Resource planning}: Optimize resource allocation
\end{itemize}

\textbf{Activity Network Diagram}:

\begin{verbatim}
    A(2) {-{-}{-}{-}{-} C(2) {-}{-}{-}{-}{-} E(4) {-}{-}{-}{-}{-} G(5) {-}{-}{-}{-}{-} H(2)}
             /           {           /}
    B(3) {-{-}{-}+             +{-}{-} D(4) +}
                              |
                              F(3)
\end{verbatim}

\textbf{Critical Path Analysis}:

{\def\LTcaptype{none} % do not increment counter
\begin{longtable}[]{@{}llll@{}}
\toprule\noalign{}
Path & Activities & Duration & Critical? \\
\midrule\noalign{}
\endhead
\bottomrule\noalign{}
\endlastfoot
\textbf{A-C-E-G-H} & A\rightarrowC\rightarrowE\rightarrowG\rightarrowH & 2+2+4+5+2 = 15 & No \\
\textbf{B-C-E-G-H} & B\rightarrowC\rightarrowE\rightarrowG\rightarrowH & 3+2+4+5+2 = 16 & \textbf{Yes} \\
\textbf{A-C-D-G-H} & A\rightarrowC\rightarrowD\rightarrowG\rightarrowH & 2+2+4+5+2 = 15 & No \\
\end{longtable}
}

\textbf{Critical Path}: B\rightarrowC\rightarrowE\rightarrowG\rightarrowH (16 days) \textbf{Project Duration}:
16 days

\end{solutionbox}
\begin{mnemonicbox}
``Networks Navigate Project Paths''

\end{mnemonicbox}
\begin{center}\rule{0.5\linewidth}{0.5pt}\end{center}

\subsection*{Question 4(a) OR [3
marks]}\label{q4a}

\textbf{Explain any three requirement gathering activities.}

\begin{solutionbox}

\textbf{Requirement Gathering Activities}:

{\def\LTcaptype{none} % do not increment counter
\begin{longtable}[]{@{}
  >{\raggedright\arraybackslash}p{(\linewidth - 4\tabcolsep) * \real{0.3226}}
  >{\raggedright\arraybackslash}p{(\linewidth - 4\tabcolsep) * \real{0.4194}}
  >{\raggedright\arraybackslash}p{(\linewidth - 4\tabcolsep) * \real{0.2581}}@{}}
\toprule\noalign{}
\begin{minipage}[b]{\linewidth}\raggedright
Activity
\end{minipage} & \begin{minipage}[b]{\linewidth}\raggedright
Description
\end{minipage} & \begin{minipage}[b]{\linewidth}\raggedright
Output
\end{minipage} \\
\midrule\noalign{}
\endhead
\bottomrule\noalign{}
\endlastfoot
\textbf{Stakeholder Interviews} & Direct discussion with users and
clients & Interview notes, requirements list \\
\textbf{Questionnaires} & Structured questions for large user groups &
Survey responses, statistical data \\
\textbf{Document Analysis} & Review existing system documentation &
Current system understanding \\
\end{longtable}
}

\begin{itemize}
\tightlist
\item
  \textbf{Purpose}: Understand user needs and system expectations
\item
  \textbf{Participants}: Users, customers, domain experts, developers
\item
  \textbf{Documentation}: All findings recorded in SRS document
\end{itemize}

\end{solutionbox}
\begin{mnemonicbox}
``Interviews, Questions, Documents Gather
Requirements''

\end{mnemonicbox}
\begin{center}\rule{0.5\linewidth}{0.5pt}\end{center}

\subsection*{Question 4(b) OR [4
marks]}\label{q4b}

\textbf{Develop use case diagram for Bank ATM system.}

\begin{solutionbox}

\textbf{ATM Use Case Diagram}:

\begin{verbatim}
graph TB
    Customer((Customer))
    Admin((Admin))
    Bank[Bank System]
    
    Customer {-{-} UC1[Check Balance]}
    Customer {-{-} UC2[Withdraw Cash]}
    Customer {-{-} UC3[Transfer Funds]}
    Customer {-{-} UC4[Change PIN]}
    Customer {-{-} UC5[Print Receipt]}
    
    Admin {-{-} UC6[Load Cash]}
    Admin {-{-} UC7[View Logs]}
    Admin {-{-} UC8[Maintenance]}
    
    UC1 {-.{-} Bank}
    UC2 {-.{-} Bank}
    UC3 {-.{-} Bank}
    UC4 {-.{-} Bank}
\end{verbatim}

\textbf{Use Case Details}:

{\def\LTcaptype{none} % do not increment counter
\begin{longtable}[]{@{}
  >{\raggedright\arraybackslash}p{(\linewidth - 2\tabcolsep) * \real{0.3889}}
  >{\raggedright\arraybackslash}p{(\linewidth - 2\tabcolsep) * \real{0.6111}}@{}}
\toprule\noalign{}
\begin{minipage}[b]{\linewidth}\raggedright
Actor
\end{minipage} & \begin{minipage}[b]{\linewidth}\raggedright
Use Cases
\end{minipage} \\
\midrule\noalign{}
\endhead
\bottomrule\noalign{}
\endlastfoot
\textbf{Customer} & Check Balance, Withdraw Cash, Transfer Funds, Change
PIN \\
\textbf{Admin} & Load Cash, View Logs, System Maintenance \\
\textbf{Bank System} & Validate accounts, Process transactions \\
\end{longtable}
}

\end{solutionbox}
\begin{mnemonicbox}
``Customers Use ATMs, Admins Maintain Systems''

\end{mnemonicbox}
\begin{center}\rule{0.5\linewidth}{0.5pt}\end{center}

\subsection*{Question 4(c) OR [7
marks]}\label{q4c}

\textbf{Draw the figure of spiral model. Explain it in brief.}

\begin{solutionbox}

\textbf{Spiral Model Diagram}:

\begin{verbatim}
graph TB
    subgraph "Spiral Model"
        A[Planning] {-{-} B[Risk Analysis]}
        B {-{-} C[Engineering]}
        C {-{-} D[Customer Evaluation]}
        D {-{-} A}
        
        A1[Plan 1] {-{-} B1[Risk 1]}
        B1 {-{-} C1[Code 1]}
        C1 {-{-} D1[Test 1]}
        D1 {-{-} A2[Plan 2]}
        A2 {-{-} B2[Risk 2]}
        B2 {-{-} C2[Code 2]}
        C2 {-{-} D2[Test 2]}
    end
\end{verbatim}

\textbf{Spiral Model Characteristics}:

{\def\LTcaptype{none} % do not increment counter
\begin{longtable}[]{@{}
  >{\raggedright\arraybackslash}p{(\linewidth - 4\tabcolsep) * \real{0.3448}}
  >{\raggedright\arraybackslash}p{(\linewidth - 4\tabcolsep) * \real{0.3448}}
  >{\raggedright\arraybackslash}p{(\linewidth - 4\tabcolsep) * \real{0.3103}}@{}}
\toprule\noalign{}
\begin{minipage}[b]{\linewidth}\raggedright
Quadrant
\end{minipage} & \begin{minipage}[b]{\linewidth}\raggedright
Activity
\end{minipage} & \begin{minipage}[b]{\linewidth}\raggedright
Purpose
\end{minipage} \\
\midrule\noalign{}
\endhead
\bottomrule\noalign{}
\endlastfoot
\textbf{Planning} & Define objectives, alternatives & Set goals for
iteration \\
\textbf{Risk Analysis} & Identify and resolve risks & Minimize project
risks \\
\textbf{Engineering} & Develop and test product & Create working
software \\
\textbf{Evaluation} & Customer assessment & Get user feedback \\
\end{longtable}
}

\begin{itemize}
\tightlist
\item
  \textbf{Key feature}: Risk-driven approach with iterative development
\item
  \textbf{Best for}: Large, complex, high-risk projects
\item
  \textbf{Advantages}: Risk management, flexible, incremental
  development
\item
  \textbf{Disadvantages}: Complex management, expensive, requires risk
  expertise
\end{itemize}

\end{solutionbox}
\begin{mnemonicbox}
``Spirals Plan, Risk, Engineer, Evaluate''

\end{mnemonicbox}
\begin{center}\rule{0.5\linewidth}{0.5pt}\end{center}

\subsection*{Question 5(a) [3 marks]}\label{q5a}

\textbf{State TRUE or FALSE for the following.}

\begin{solutionbox}

{\def\LTcaptype{none} % do not increment counter
\begin{longtable}[]{@{}
  >{\raggedright\arraybackslash}p{(\linewidth - 4\tabcolsep) * \real{0.3438}}
  >{\raggedright\arraybackslash}p{(\linewidth - 4\tabcolsep) * \real{0.2500}}
  >{\raggedright\arraybackslash}p{(\linewidth - 4\tabcolsep) * \real{0.4062}}@{}}
\toprule\noalign{}
\begin{minipage}[b]{\linewidth}\raggedright
Statement
\end{minipage} & \begin{minipage}[b]{\linewidth}\raggedright
Answer
\end{minipage} & \begin{minipage}[b]{\linewidth}\raggedright
Explanation
\end{minipage} \\
\midrule\noalign{}
\endhead
\bottomrule\noalign{}
\endlastfoot
\textbf{Activity network diagram used to determine critical path} &
\textbf{TRUE} & Primary purpose of activity networks \\
\textbf{In CPM, the shortest path is the critical path} & \textbf{FALSE}
& Longest path is critical path \\
\textbf{Risk avoidance is the best technique to solve risks} &
\textbf{FALSE} & Best technique depends on risk type \\
\end{longtable}
}

\begin{itemize}
\tightlist
\item
  \textbf{Critical path}: Longest duration path in project network
\item
  \textbf{CPM}: Critical Path Method identifies project bottlenecks
\item
  \textbf{Risk strategies}: Avoid, Transfer, Mitigate, Accept (choice
  depends on context)
\end{itemize}

\end{solutionbox}
\begin{mnemonicbox}
``True Networks, False Shortest, False Best''

\end{mnemonicbox}
\begin{center}\rule{0.5\linewidth}{0.5pt}\end{center}

\subsection*{Question 5(b) [4 marks]}\label{q5b}

\textbf{Identify the differences between traditional model approach and
agile approach. (at least 4 differences)}

\begin{solutionbox}

\textbf{Traditional vs Agile Comparison}:

{\def\LTcaptype{none} % do not increment counter
\begin{longtable}[]{@{}
  >{\raggedright\arraybackslash}p{(\linewidth - 4\tabcolsep) * \real{0.2857}}
  >{\raggedright\arraybackslash}p{(\linewidth - 4\tabcolsep) * \real{0.4643}}
  >{\raggedright\arraybackslash}p{(\linewidth - 4\tabcolsep) * \real{0.2500}}@{}}
\toprule\noalign{}
\begin{minipage}[b]{\linewidth}\raggedright
Aspect
\end{minipage} & \begin{minipage}[b]{\linewidth}\raggedright
Traditional
\end{minipage} & \begin{minipage}[b]{\linewidth}\raggedright
Agile
\end{minipage} \\
\midrule\noalign{}
\endhead
\bottomrule\noalign{}
\endlastfoot
\textbf{Planning} & Extensive upfront planning & Adaptive planning \\
\textbf{Documentation} & Heavy documentation & Minimal documentation \\
\textbf{Customer involvement} & Limited to requirements phase &
Continuous involvement \\
\textbf{Change handling} & Difficult and expensive & Embraces change \\
\textbf{Delivery} & Single final delivery & Frequent incremental
delivery \\
\textbf{Process} & Process-driven & People-driven \\
\end{longtable}
}

\begin{itemize}
\tightlist
\item
  \textbf{Traditional}: Predictive, sequential approach
\item
  \textbf{Agile}: Adaptive, iterative approach
\item
  \textbf{Flexibility}: Agile more responsive to changing requirements
\end{itemize}

\end{solutionbox}
\begin{mnemonicbox}
``Traditional Plans Heavy, Agile Adapts Light''

\end{mnemonicbox}
\begin{center}\rule{0.5\linewidth}{0.5pt}\end{center}

\subsection*{Question 5(c) [7 marks]}\label{q5c}

\textbf{Define unit testing. Draw the figure of it. Explain the process
of unit testing.}

\begin{solutionbox}

\textbf{Unit Testing Definition}: Testing individual software components
or modules in isolation to verify they function correctly according to
design specifications.

\textbf{Unit Testing Process}:

\begin{center}
\textbf{Mermaid Diagram (Code)}
\begin{verbatim}
{Shaded}
{Highlighting}[]
graph LR
    A[Select Unit] {-{-}{} B[Design Test Cases]}
    B {-{-}{} C[Set Up Test Environment]}
    C {-{-}{} D[Execute Tests]}
    D {-{-}{} E[Record Results]}
    E {-{-}{} F\{All Tests Pass?\}}
    F {-{-}{}|No| G[Debug and Fix]}
    G {-{-}{} D}
    F {-{-}{}|Yes| H[Unit Approved]}
{Highlighting}
{Shaded}
\end{verbatim}
\end{center}

\textbf{Unit Testing Process Steps}:

{\def\LTcaptype{none} % do not increment counter
\begin{longtable}[]{@{}lll@{}}
\toprule\noalign{}
Step & Activity & Purpose \\
\midrule\noalign{}
\endhead
\bottomrule\noalign{}
\endlastfoot
\textbf{Test Planning} & Identify units to test & Define testing
scope \\
\textbf{Test Design} & Create test cases & Cover all code paths \\
\textbf{Test Setup} & Prepare test environment & Isolate unit under
test \\
\textbf{Test Execution} & Run test cases & Verify unit behavior \\
\textbf{Result Analysis} & Evaluate outcomes & Identify defects \\
\textbf{Defect Fixing} & Correct found issues & Ensure unit quality \\
\end{longtable}
}

\begin{itemize}
\tightlist
\item
  \textbf{Benefits}: Early defect detection, easier debugging, improved
  code quality
\item
  \textbf{Tools}: JUnit, NUnit, automated testing frameworks
\item
  \textbf{Coverage}: Aim for high code coverage (statements, branches,
  paths)
\end{itemize}

\end{solutionbox}
\begin{mnemonicbox}
``Units Test Individual Components Thoroughly''

\end{mnemonicbox}
\begin{center}\rule{0.5\linewidth}{0.5pt}\end{center}

\subsection*{Question 5(a) OR [3
marks]}\label{q5a}

\textbf{Give the full form of the following.}

\begin{solutionbox}

{\def\LTcaptype{none} % do not increment counter
\begin{longtable}[]{@{}ll@{}}
\toprule\noalign{}
Abbreviation & Full Form \\
\midrule\noalign{}
\endhead
\bottomrule\noalign{}
\endlastfoot
\textbf{AOA} & Activity On Arrow \\
\textbf{PERT} & Program Evaluation and Review Technique \\
\textbf{EVA} & Earned Value Analysis \\
\textbf{CPM} & Critical Path Method \\
\textbf{WBS} & Work Breakdown Structure \\
\textbf{PMC} & Project Management and Control \\
\end{longtable}
}

\end{solutionbox}
\begin{mnemonicbox}
``Activities On Arrows, Programs Evaluate Review
Techniques, Earned Values Analyzed, Critical Paths Managed, Work Broken
Structured, Projects Managed Controlled''

\end{mnemonicbox}
\begin{center}\rule{0.5\linewidth}{0.5pt}\end{center}

\subsection*{Question 5(b) OR [4
marks]}\label{q5b}

\textbf{Explain code inspection.}

\begin{solutionbox}

Code inspection is a systematic examination of source code by team
members to identify defects and ensure quality standards.

\textbf{Code Inspection Process}:

{\def\LTcaptype{none} % do not increment counter
\begin{longtable}[]{@{}lll@{}}
\toprule\noalign{}
Phase & Activity & Participants \\
\midrule\noalign{}
\endhead
\bottomrule\noalign{}
\endlastfoot
\textbf{Planning} & Schedule inspection meeting & Moderator \\
\textbf{Preparation} & Review code individually & All inspectors \\
\textbf{Inspection Meeting} & Discuss findings & Team members \\
\textbf{Rework} & Fix identified issues & Author \\
\textbf{Follow-up} & Verify corrections & Moderator \\
\end{longtable}
}

\begin{itemize}
\tightlist
\item
  \textbf{Benefits}: Early defect detection, knowledge sharing, improved
  code quality
\item
  \textbf{Roles}: Author, Moderator, Reviewers, Recorder
\item
  \textbf{Focus areas}: Logic errors, coding standards, maintainability
\end{itemize}

\end{solutionbox}
\begin{mnemonicbox}
``Inspections Improve Code Quality''

\end{mnemonicbox}
\begin{center}\rule{0.5\linewidth}{0.5pt}\end{center}

\subsection*{Question 5(c) OR [7
marks]}\label{q5c}

\textbf{Define white box testing method. Explain different white box
testing methods.}

\begin{solutionbox}

\textbf{White Box Testing Definition}: Testing method that examines
internal code structure, logic paths, and implementation details to
ensure thorough coverage.

\textbf{White Box Testing Methods}:

{\def\LTcaptype{none} % do not increment counter
\begin{longtable}[]{@{}
  >{\raggedright\arraybackslash}p{(\linewidth - 4\tabcolsep) * \real{0.2162}}
  >{\raggedright\arraybackslash}p{(\linewidth - 4\tabcolsep) * \real{0.3514}}
  >{\raggedright\arraybackslash}p{(\linewidth - 4\tabcolsep) * \real{0.4324}}@{}}
\toprule\noalign{}
\begin{minipage}[b]{\linewidth}\raggedright
Method
\end{minipage} & \begin{minipage}[b]{\linewidth}\raggedright
Description
\end{minipage} & \begin{minipage}[b]{\linewidth}\raggedright
Coverage Focus
\end{minipage} \\
\midrule\noalign{}
\endhead
\bottomrule\noalign{}
\endlastfoot
\textbf{Statement Coverage} & Execute every statement & All code
lines \\
\textbf{Branch Coverage} & Test all decision outcomes & If-else
conditions \\
\textbf{Path Coverage} & Execute all possible paths & Complete execution
flows \\
\textbf{Condition Coverage} & Test all condition combinations & Boolean
expressions \\
\end{longtable}
}

\textbf{Testing Techniques}:

\begin{verbatim}
mindmap
  root((White Box Testing))
    Statement Testing
      Line Coverage
      Code Execution
    Branch Testing
      Decision Points
      True/False Paths
    Path Testing
      All Routes
      Loop Testing
    Condition Testing
      Boolean Logic
      Multiple Conditions
\end{verbatim}

\textbf{Coverage Analysis}:

{\def\LTcaptype{none} % do not increment counter
\begin{longtable}[]{@{}
  >{\raggedright\arraybackslash}p{(\linewidth - 4\tabcolsep) * \real{0.3793}}
  >{\raggedright\arraybackslash}p{(\linewidth - 4\tabcolsep) * \real{0.3103}}
  >{\raggedright\arraybackslash}p{(\linewidth - 4\tabcolsep) * \real{0.3103}}@{}}
\toprule\noalign{}
\begin{minipage}[b]{\linewidth}\raggedright
Technique
\end{minipage} & \begin{minipage}[b]{\linewidth}\raggedright
Formula
\end{minipage} & \begin{minipage}[b]{\linewidth}\raggedright
Purpose
\end{minipage} \\
\midrule\noalign{}
\endhead
\bottomrule\noalign{}
\endlastfoot
\textbf{Statement} & Executed statements / Total statements & Ensure all
code runs \\
\textbf{Branch} & Tested branches / Total branches & Cover all
decisions \\
\textbf{Path} & Tested paths / Total paths & Complete flow coverage \\
\end{longtable}
}

\begin{itemize}
\tightlist
\item
  \textbf{Tools}: Code coverage analyzers, debugging tools
\item
  \textbf{Advantages}: Thorough testing, identifies dead code, ensures
  quality
\item
  \textbf{Disadvantages}: Requires code knowledge, time-consuming, may
  miss requirement gaps
\end{itemize}

\end{solutionbox}
\begin{mnemonicbox}
``White Box Sees Inside Code Structure''

\end{mnemonicbox}

\end{document}
