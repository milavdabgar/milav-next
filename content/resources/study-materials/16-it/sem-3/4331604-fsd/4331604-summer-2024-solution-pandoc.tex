\documentclass[10pt,a4paper]{article}

% content/resources/templates/preamble.tex
\usepackage[margin=0.6in]{geometry}
\author{Milav Dabgar}
\usepackage{amsmath,amssymb,amsthm}
\usepackage{booktabs}
\usepackage{multirow}
\usepackage{xcolor}
\usepackage{tcolorbox}
\tcbuselibrary{breakable,skins}
\usepackage[colorlinks=true,linkcolor=blue]{hyperref}
\usepackage{titlesec}
\usepackage{enumitem}
\usepackage{tikz}
\usepackage{pgfplots}
\usepackage{circuitikz}
\usepackage[version=4]{mhchem}
\usepackage{longtable}
\usepackage{array}
\usepackage{float}
\usepackage{caption}
\usepackage{listings}

\lstset{
  basicstyle=\small\ttfamily,
  breaklines=true,
  breakatwhitespace=false,
  postbreak=\mbox{\textcolor{red}{$\hookrightarrow$}\space},
  float=false,
  numbers=left,
  numberstyle=\tiny\color{gray},
  numbersep=10pt,
  xleftmargin=2em,
  keywordstyle=\color{blue},
  commentstyle=\color{green!60!black},
  stringstyle=\color{purple},
  backgroundcolor=\color{gray!5},
  showstringspaces=false,
  tabsize=2,
  captionpos=b,
  keepspaces=true,
  columns=flexible
}

\pgfplotsset{compat=1.18}
\usetikzlibrary{shapes,arrows,positioning,calc,patterns,decorations.pathmorphing,decorations.markings,arrows.meta}

% Color scheme
\definecolor{headcolor}{RGB}{0,102,204}
\definecolor{keycolor}{RGB}{220,20,60}
\definecolor{solutioncolor}{RGB}{34,139,34}
\definecolor{mnemoniccolor}{RGB}{148,0,211}
\definecolor{codecolor}{RGB}{0,0,100}

% Spacing
\setlength{\parskip}{3pt}
\setlist[itemize]{nosep}
\setlist[enumerate]{nosep}

% Title formatting
\titleformat{\section}{\Large\bfseries\color{headcolor}}{\thesection}{1em}{}
\titleformat{\subsection}{\large\bfseries\color{headcolor}}{\thesubsection}{1em}{}

% Pandoc tightlist compatibility
\providecommand{\tightlist}{%
  \setlength{\itemsep}{0pt}\setlength{\parskip}{0pt}}

% Pandoc longtable compatibility
\newcounter{none}
\def\thenone{}


% content/resources/templates/english-boxes.tex
% This file is currently empty - it exists to maintain consistency with the import structure.
% Add custom environments here if needed in the future.


\begin{document}

\begin{center}
{\Huge\bfseries\color{headcolor} Subject Name Solutions}\\[5pt]
{\LARGE 4331604 -- Summer 2024}\\[3pt]
{\large Semester 1 Study Material}\\[3pt]
{\normalsize\textit{Detailed Solutions and Explanations}}
\end{center}

\vspace{10pt}

\subsection*{Question 1(a) [3 marks]}\label{q1a}

\textbf{Explain software engineering layered approach.}

\begin{solutionbox}

Software engineering follows a layered approach with four fundamental
layers working together to create quality software products.


{\def\LTcaptype{none} % do not increment counter
\vspace{-5pt}
\captionof{table}{Software Engineering Layered Approach}
\vspace{-10pt}
\begin{longtable}[]{@{}
  >{\raggedright\arraybackslash}p{(\linewidth - 4\tabcolsep) * \real{0.2414}}
  >{\raggedright\arraybackslash}p{(\linewidth - 4\tabcolsep) * \real{0.4483}}
  >{\raggedright\arraybackslash}p{(\linewidth - 4\tabcolsep) * \real{0.3103}}@{}}
\toprule\noalign{}
\begin{minipage}[b]{\linewidth}\raggedright
Layer
\end{minipage} & \begin{minipage}[b]{\linewidth}\raggedright
Description
\end{minipage} & \begin{minipage}[b]{\linewidth}\raggedright
Purpose
\end{minipage} \\
\midrule\noalign{}
\endhead
\bottomrule\noalign{}
\endlastfoot
\textbf{Quality Focus} & Foundation layer emphasizing continuous
improvement & Ensures defect-free products \\
\textbf{Process} & Defines framework of activities and tasks & Provides
systematic development approach \\
\textbf{Methods} & Technical procedures for analysis, design, coding,
testing & Offers ``how-to'' guidance \\
\textbf{Tools} & Automated support for process and methods & Provides
efficiency and consistency \\
\end{longtable}
}

\begin{itemize}
\tightlist
\item
  \textbf{Quality Focus}: Forms the foundation ensuring customer
  satisfaction
\item
  \textbf{Process Layer}: Defines workflow and project management
  activities
\item
  \textbf{Methods Layer}: Provides technical approach for each
  development phase
\item
  \textbf{Tools Layer}: Supports automation and integration
\end{itemize}

\end{solutionbox}
\begin{mnemonicbox}
``Quality Processes Make Tools'' - Remember the four
layers from bottom to top.

\end{mnemonicbox}
\begin{center}\rule{0.5\linewidth}{0.5pt}\end{center}

\subsection*{Question 1(b) [4 marks]}\label{q1b}

\textbf{Explain Iterative waterfall model.}

\begin{solutionbox}

The Iterative Waterfall Model combines the structured approach of
waterfall with feedback loops for improvement and error correction.

\begin{center}
\textbf{Mermaid Diagram (Code)}
\begin{verbatim}
{Shaded}
{Highlighting}[]
graph LR
    A[Requirements Analysis] {-{-}{} B[System Design]}
    B {-{-}{} C[Implementation]}
    C {-{-}{} D[Integration \& Testing]}
    D {-{-}{} E[Deployment]}
    E {-{-}{} F[Maintenance]}
    B {-.{-}{}|Feedback| A}
    C {-.{-}{}|Feedback| B}
    D {-.{-}{}|Feedback| C}
    E {-.{-}{}|Feedback| D}
    F {-.{-}{}|Feedback| E}
{Highlighting}
{Shaded}
\end{verbatim}
\end{center}

\textbf{Key Features:}

\begin{itemize}
\tightlist
\item
  \textbf{Sequential phases}: Each phase completed before next begins
\item
  \textbf{Feedback loops}: Allow return to previous phases for
  corrections
\item
  \textbf{Documentation driven}: Heavy emphasis on documentation at each
  phase
\item
  \textbf{Error correction}: Issues identified in later phases can be
  fixed
\end{itemize}

\end{solutionbox}
\begin{mnemonicbox}
``Water Falls Back Up'' - Sequential flow with upward
feedback capability.

\end{mnemonicbox}
\begin{center}\rule{0.5\linewidth}{0.5pt}\end{center}

\subsection*{Question 1(c) [7 marks]}\label{q1c}

\textbf{Explain Agile Model and Agile Principles.}

\begin{solutionbox}

Agile is an iterative software development methodology emphasizing
collaboration, customer feedback, and rapid delivery of working
software.


{\def\LTcaptype{none} % do not increment counter
\vspace{-5pt}
\captionof{table}{Agile Values vs Traditional Approach}
\vspace{-10pt}
\begin{longtable}[]{@{}ll@{}}
\toprule\noalign{}
Agile Values & Traditional Approach \\
\midrule\noalign{}
\endhead
\bottomrule\noalign{}
\endlastfoot
Individuals and interactions & Processes and tools \\
Working software & Comprehensive documentation \\
Customer collaboration & Contract negotiation \\
Responding to change & Following a plan \\
\end{longtable}
}

\textbf{Core Agile Principles:}

\begin{itemize}
\tightlist
\item
  \textbf{Customer satisfaction}: Deliver valuable software early and
  continuously
\item
  \textbf{Welcome change}: Embrace changing requirements even late in
  development
\item
  \textbf{Frequent delivery}: Deliver working software frequently (weeks
  rather than months)
\item
  \textbf{Collaboration}: Business people and developers work together
  daily
\item
  \textbf{Motivated individuals}: Build projects around motivated people
\item
  \textbf{Face-to-face conversation}: Most efficient method of
  communication
\item
  \textbf{Working software}: Primary measure of progress
\item
  \textbf{Sustainable development}: Maintain constant pace indefinitely
\item
  \textbf{Technical excellence}: Continuous attention to good design
\item
  \textbf{Simplicity}: Art of maximizing work not done
\item
  \textbf{Self-organizing teams}: Best requirements emerge from
  self-organizing teams
\item
  \textbf{Regular reflection}: Team reflects and adjusts behavior
  regularly
\end{itemize}

\textbf{Diagram: Agile Development Cycle}

\begin{center}
\textbf{Mermaid Diagram (Code)}
\begin{verbatim}
{Shaded}
{Highlighting}[]
graph LR
    A[Planning] {-{-}{} B[Design]}
    B {-{-}{} C[Coding]}
    C {-{-}{} D[Testing]}
    D {-{-}{} E[Review]}
    E {-{-}{} A}
    E {-{-}{} F[Release]}
{Highlighting}
{Shaded}
\end{verbatim}
\end{center}

\end{solutionbox}
\begin{mnemonicbox}
``Customer Change Frequently Collaborates'' - Core
agile principles focus.

\end{mnemonicbox}
\begin{center}\rule{0.5\linewidth}{0.5pt}\end{center}

\subsection*{Question 1(c OR) [7
marks]}\label{question-1c-or-7-marks}

\textbf{Write a short note on Scrum.}

\begin{solutionbox}

Scrum is an agile framework for managing software development with
emphasis on team collaboration and iterative progress.


{\def\LTcaptype{none} % do not increment counter
\vspace{-5pt}
\captionof{table}{Scrum Roles and Responsibilities}
\vspace{-10pt}
\begin{longtable}[]{@{}
  >{\raggedright\arraybackslash}p{(\linewidth - 4\tabcolsep) * \real{0.1500}}
  >{\raggedright\arraybackslash}p{(\linewidth - 4\tabcolsep) * \real{0.4500}}
  >{\raggedright\arraybackslash}p{(\linewidth - 4\tabcolsep) * \real{0.4000}}@{}}
\toprule\noalign{}
\begin{minipage}[b]{\linewidth}\raggedright
Role
\end{minipage} & \begin{minipage}[b]{\linewidth}\raggedright
Responsibilities
\end{minipage} & \begin{minipage}[b]{\linewidth}\raggedright
Key Activities
\end{minipage} \\
\midrule\noalign{}
\endhead
\bottomrule\noalign{}
\endlastfoot
\textbf{Product Owner} & Defines product features and priorities &
Manages product backlog \\
\textbf{Scrum Master} & Facilitates process and removes obstacles &
Conducts ceremonies \\
\textbf{Development Team} & Creates working software & Self-organizing
and cross-functional \\
\end{longtable}
}

\textbf{Scrum Events:}

\begin{itemize}
\tightlist
\item
  \textbf{Sprint}: 1-4 week iteration producing potentially shippable
  product
\item
  \textbf{Sprint Planning}: Team plans work for upcoming sprint
\item
  \textbf{Daily Scrum}: 15-minute daily synchronization meeting
\item
  \textbf{Sprint Review}: Demonstrate completed work to stakeholders
\item
  \textbf{Sprint Retrospective}: Team reflects on process improvements
\end{itemize}

\textbf{Scrum Artifacts:}

\begin{itemize}
\tightlist
\item
  \textbf{Product Backlog}: Prioritized list of features
\item
  \textbf{Sprint Backlog}: Items selected for current sprint
\item
  \textbf{Increment}: Working product at sprint end
\end{itemize}

\textbf{Diagram: Scrum Process Flow}

\begin{center}
\textbf{Mermaid Diagram (Code)}
\begin{verbatim}
{Shaded}
{Highlighting}[]
graph LR
    A[Product Backlog] {-{-}{} B[Sprint Planning]}
    B {-{-}{} C[Sprint Backlog]}
    C {-{-}{} D[Daily Scrum]}
    D {-{-}{} E[Sprint Review]}
    E {-{-}{} F[Sprint Retrospective]}
    F {-{-}{} B}
    E {-{-}{} G[Product Increment]}
{Highlighting}
{Shaded}
\end{verbatim}
\end{center}

\end{solutionbox}
\begin{mnemonicbox}
``Product Sprints Daily Reviews'' - Key scrum
elements sequence.

\end{mnemonicbox}
\begin{center}\rule{0.5\linewidth}{0.5pt}\end{center}

\subsection*{Question 2(a) [3 marks]}\label{q2a}

\textbf{If you have to develop a word processing software product, what
process models will you choose? Justify your answer.}

\begin{solutionbox}

For word processing software development, I would choose the
\textbf{Incremental Model} as the most suitable process model.

\textbf{Justification:}

\begin{itemize}
\tightlist
\item
  \textbf{Complex functionality}: Word processors have numerous features
  (editing, formatting, spell-check) that can be developed incrementally
\item
  \textbf{User feedback}: Early increments allow user testing and
  feedback incorporation
\item
  \textbf{Risk management}: Core features delivered first, advanced
  features added later
\item
  \textbf{Market advantage}: Basic version can be released early to gain
  market presence
\end{itemize}

\textbf{Development Increments:}

\begin{enumerate}
\tightlist
\item
  \textbf{Increment 1}: Basic text editing and file operations
\item
  \textbf{Increment 2}: Formatting and font management
\item
  \textbf{Increment 3}: Advanced features (spell-check, templates)
\end{enumerate}

\end{solutionbox}
\begin{mnemonicbox}
``Word Processing Increments User Feedback'' -
Incremental approach suits complex software.

\end{mnemonicbox}
\begin{center}\rule{0.5\linewidth}{0.5pt}\end{center}

\subsection*{Question 2(b) [4 marks]}\label{q2b}

\textbf{Explain characteristics of good SRS.}

\begin{solutionbox}

A good Software Requirements Specification (SRS) document must possess
specific characteristics to ensure successful software development.


{\def\LTcaptype{none} % do not increment counter
\vspace{-5pt}
\captionof{table}{Characteristics of Good SRS}
\vspace{-10pt}
\begin{longtable}[]{@{}
  >{\raggedright\arraybackslash}p{(\linewidth - 4\tabcolsep) * \real{0.3902}}
  >{\raggedright\arraybackslash}p{(\linewidth - 4\tabcolsep) * \real{0.3171}}
  >{\raggedright\arraybackslash}p{(\linewidth - 4\tabcolsep) * \real{0.2927}}@{}}
\toprule\noalign{}
\begin{minipage}[b]{\linewidth}\raggedright
Characteristic
\end{minipage} & \begin{minipage}[b]{\linewidth}\raggedright
Description
\end{minipage} & \begin{minipage}[b]{\linewidth}\raggedright
Importance
\end{minipage} \\
\midrule\noalign{}
\endhead
\bottomrule\noalign{}
\endlastfoot
\textbf{Complete} & Contains all necessary requirements & Prevents scope
creep \\
\textbf{Consistent} & No conflicting requirements & Avoids
implementation confusion \\
\textbf{Unambiguous} & Clear and precise language & Single
interpretation possible \\
\textbf{Verifiable} & Requirements can be tested & Enables validation \\
\textbf{Modifiable} & Easy to change and maintain & Supports requirement
evolution \\
\textbf{Traceable} & Requirements linked to sources & Impact analysis
possible \\
\end{longtable}
}

\textbf{Additional Characteristics:}

\begin{itemize}
\tightlist
\item
  \textbf{Feasible}: Technically and economically achievable
\item
  \textbf{Necessary}: Each requirement serves a purpose
\item
  \textbf{Prioritized}: Requirements ranked by importance
\item
  \textbf{Testable}: Specific criteria for verification
\end{itemize}

\end{solutionbox}
\begin{mnemonicbox}
``Complete Consistent Unambiguous Verifiable'' - Core
SRS quality attributes.

\end{mnemonicbox}
\begin{center}\rule{0.5\linewidth}{0.5pt}\end{center}

\subsection*{Question 2(c) [7 marks]}\label{q2c}

\textbf{Explain functional and non-functional requirements for an ATM
software.}

\begin{solutionbox}

ATM software requirements are categorized into functional (what system
does) and non-functional (how system performs) requirements.


{\def\LTcaptype{none} % do not increment counter
\vspace{-5pt}
\captionof{table}{ATM Functional Requirements}
\vspace{-10pt}
\begin{longtable}[]{@{}
  >{\raggedright\arraybackslash}p{(\linewidth - 4\tabcolsep) * \real{0.3125}}
  >{\raggedright\arraybackslash}p{(\linewidth - 4\tabcolsep) * \real{0.4062}}
  >{\raggedright\arraybackslash}p{(\linewidth - 4\tabcolsep) * \real{0.2812}}@{}}
\toprule\noalign{}
\begin{minipage}[b]{\linewidth}\raggedright
Function
\end{minipage} & \begin{minipage}[b]{\linewidth}\raggedright
Description
\end{minipage} & \begin{minipage}[b]{\linewidth}\raggedright
Example
\end{minipage} \\
\midrule\noalign{}
\endhead
\bottomrule\noalign{}
\endlastfoot
\textbf{Authentication} & User login and verification & PIN validation,
card reading \\
\textbf{Account Operations} & Basic banking transactions & Balance
inquiry, cash withdrawal \\
\textbf{Transaction Processing} & Money transfer and deposits &
Account-to-account transfer \\
\textbf{Receipt Generation} & Transaction documentation & Print
transaction receipts \\
\textbf{Session Management} & User session control & Timeout, logout
functionality \\
\end{longtable}
}


{\def\LTcaptype{none} % do not increment counter
\vspace{-5pt}
\captionof{table}{ATM Non-Functional Requirements}
\vspace{-10pt}
\begin{longtable}[]{@{}lll@{}}
\toprule\noalign{}
Category & Requirement & Specification \\
\midrule\noalign{}
\endhead
\bottomrule\noalign{}
\endlastfoot
\textbf{Performance} & Response time & Maximum 3 seconds per
transaction \\
\textbf{Security} & Data protection & 256-bit encryption for all data \\
\textbf{Reliability} & System availability & 99.9\% uptime
requirement \\
\textbf{Usability} & User interface & Simple interface for all age
groups \\
\textbf{Scalability} & Load handling & Support 1000 concurrent users \\
\end{longtable}
}

\textbf{Functional Requirements Details:}

\begin{itemize}
\tightlist
\item
  \textbf{Cash Withdrawal}: Dispense cash after successful
  authentication
\item
  \textbf{Balance Inquiry}: Display current account balance
\item
  \textbf{PIN Change}: Allow users to update their PIN
\item
  \textbf{Mini Statement}: Provide last 10 transactions
\end{itemize}

\textbf{Non-Functional Requirements Details:}

\begin{itemize}
\tightlist
\item
  \textbf{Security}: Multi-factor authentication, transaction logging
\item
  \textbf{Performance}: Fast transaction processing, minimal wait time
\item
  \textbf{Availability}: 24/7 operation with minimal downtime
\item
  \textbf{Maintainability}: Easy software updates and hardware
  maintenance
\end{itemize}

\end{solutionbox}
\begin{mnemonicbox}
``Functions Work, Quality Matters'' - Functional vs
non-functional distinction.

\end{mnemonicbox}
\begin{center}\rule{0.5\linewidth}{0.5pt}\end{center}

\subsection*{Question 2(a OR) [3
marks]}\label{question-2a-or-3-marks}

\textbf{Explain Incremental Model with diagram.}

\begin{solutionbox}

The Incremental Model develops software in small, manageable portions
called increments, with each increment adding new functionality to the
existing system.

\textbf{Diagram: Incremental Model}

\begin{center}
\textbf{Mermaid Diagram (Code)}
\begin{verbatim}
{Shaded}
{Highlighting}[]
graph LR
    A[Requirements] {-{-}{} B[Increment 1]}
    A {-{-}{} C[Increment 2]}
    A {-{-}{} D[Increment 3]}
    
    B {-{-}{} B1[Analysis]}
    B1 {-{-}{} B2[Design]}
    B2 {-{-}{} B3[Code]}
    B3 {-{-}{} B4[Test]}
    B4 {-{-}{} B5[Release 1]}
    
    C {-{-}{} C1[Analysis]}
    C1 {-{-}{} C2[Design]}
    C2 {-{-}{} C3[Code]}
    C3 {-{-}{} C4[Test]}
    C4 {-{-}{} C5[Release 2]}
    
    D {-{-}{} D1[Analysis]}
    D1 {-{-}{} D2[Design]}
    D2 {-{-}{} D3[Code]}
    D3 {-{-}{} D4[Test]}
    D4 {-{-}{} D5[Final Release]}
{Highlighting}
{Shaded}
\end{verbatim}
\end{center}

\textbf{Key Features:}

\begin{itemize}
\tightlist
\item
  \textbf{Parallel development}: Multiple increments developed
  simultaneously
\item
  \textbf{Early delivery}: Working software available after first
  increment
\item
  \textbf{Risk reduction}: Core functionality delivered first
\end{itemize}

\end{solutionbox}
\begin{mnemonicbox}
``Increments Build Upon Previous'' - Each increment
adds to existing functionality.

\end{mnemonicbox}
\begin{center}\rule{0.5\linewidth}{0.5pt}\end{center}

\subsection*{Question 2(b OR) [4
marks]}\label{question-2b-or-4-marks}

\textbf{Differentiate between functional and non-functional
requirements.}

\begin{solutionbox}


{\def\LTcaptype{none} % do not increment counter
\vspace{-5pt}
\captionof{table}{Functional vs Non-Functional Requirements}
\vspace{-10pt}
\begin{longtable}[]{@{}
  >{\raggedright\arraybackslash}p{(\linewidth - 4\tabcolsep) * \real{0.1333}}
  >{\raggedright\arraybackslash}p{(\linewidth - 4\tabcolsep) * \real{0.4000}}
  >{\raggedright\arraybackslash}p{(\linewidth - 4\tabcolsep) * \real{0.4667}}@{}}
\toprule\noalign{}
\begin{minipage}[b]{\linewidth}\raggedright
Aspect
\end{minipage} & \begin{minipage}[b]{\linewidth}\raggedright
Functional Requirements
\end{minipage} & \begin{minipage}[b]{\linewidth}\raggedright
Non-Functional Requirements
\end{minipage} \\
\midrule\noalign{}
\endhead
\bottomrule\noalign{}
\endlastfoot
\textbf{Definition} & What the system does & How the system performs \\
\textbf{Focus} & System behavior and features & System quality
attributes \\
\textbf{Testing} & Black-box testing & Performance and stress testing \\
\textbf{Documentation} & Use cases, user stories & Quality metrics,
constraints \\
\textbf{Examples} & Login, search, calculate & Speed, security,
usability \\
\textbf{Verification} & Functional testing & Non-functional testing \\
\textbf{Change Impact} & Feature modification & Performance tuning \\
\textbf{User Visibility} & Directly visible to users & Indirectly
experienced \\
\end{longtable}
}

\textbf{Functional Requirements Characteristics:}

\begin{itemize}
\tightlist
\item
  \textbf{Behavior-focused}: Define system actions and responses
\item
  \textbf{Feature-specific}: Each requirement describes a specific
  capability
\item
  \textbf{User-driven}: Based on user needs and business processes
\end{itemize}

\textbf{Non-Functional Requirements Characteristics:}

\begin{itemize}
\tightlist
\item
  \textbf{Quality-focused}: Define performance and quality standards
\item
  \textbf{System-wide}: Apply to entire system rather than specific
  features
\item
  \textbf{Constraint-driven}: Set limits and boundaries for system
  operation
\end{itemize}

\end{solutionbox}
\begin{mnemonicbox}
``Functions Do, Quality Shows'' - Functional
requirements define actions, non-functional define quality.

\end{mnemonicbox}
\begin{center}\rule{0.5\linewidth}{0.5pt}\end{center}

\subsection*{Question 2(c OR) [7
marks]}\label{question-2c-or-7-marks}

\textbf{Write a short note on Requirements Analysis.}

\begin{solutionbox}

Requirements Analysis is the process of studying user needs and defining
system requirements to understand what the software system should
accomplish.


{\def\LTcaptype{none} % do not increment counter
\vspace{-5pt}
\captionof{table}{Requirements Analysis Process}
\vspace{-10pt}
\begin{longtable}[]{@{}
  >{\raggedright\arraybackslash}p{(\linewidth - 4\tabcolsep) * \real{0.2121}}
  >{\raggedright\arraybackslash}p{(\linewidth - 4\tabcolsep) * \real{0.3636}}
  >{\raggedright\arraybackslash}p{(\linewidth - 4\tabcolsep) * \real{0.4242}}@{}}
\toprule\noalign{}
\begin{minipage}[b]{\linewidth}\raggedright
Phase
\end{minipage} & \begin{minipage}[b]{\linewidth}\raggedright
Activities
\end{minipage} & \begin{minipage}[b]{\linewidth}\raggedright
Deliverables
\end{minipage} \\
\midrule\noalign{}
\endhead
\bottomrule\noalign{}
\endlastfoot
\textbf{Elicitation} & Gather requirements from stakeholders &
Requirement lists, interviews \\
\textbf{Analysis} & Study and understand requirements & Requirement
models, prototypes \\
\textbf{Specification} & Document requirements formally & SRS document,
use cases \\
\textbf{Validation} & Verify requirements correctness & Validated
requirements \\
\end{longtable}
}

\textbf{Requirements Elicitation Techniques:}

\begin{itemize}
\tightlist
\item
  \textbf{Interviews}: One-on-one discussions with stakeholders
\item
  \textbf{Questionnaires}: Structured surveys for large user groups
\item
  \textbf{Observation}: Studying current work processes
\item
  \textbf{Workshops}: Group sessions for requirement gathering
\item
  \textbf{Prototyping}: Building preliminary versions for feedback
\end{itemize}

\textbf{Analysis Activities:}

\begin{itemize}
\tightlist
\item
  \textbf{Requirement prioritization}: Ranking requirements by
  importance
\item
  \textbf{Feasibility study}: Assessing technical and economic viability
\item
  \textbf{Conflict resolution}: Resolving contradictory requirements
\item
  \textbf{Requirement modeling}: Creating visual representations
\end{itemize}

\textbf{Validation Techniques:}

\begin{itemize}
\tightlist
\item
  \textbf{Requirement reviews}: Formal examination of documented
  requirements
\item
  \textbf{Prototyping}: Building models to validate understanding
\item
  \textbf{Test case generation}: Creating tests from requirements
\end{itemize}

\textbf{Challenges in Requirements Analysis:}

\begin{itemize}
\tightlist
\item
  \textbf{Changing requirements}: Stakeholder needs evolve over time
\item
  \textbf{Communication gaps}: Misunderstanding between users and
  developers
\item
  \textbf{Incomplete requirements}: Missing or vague specifications
\item
  \textbf{Conflicting stakeholder needs}: Different user groups have
  different priorities
\end{itemize}

\end{solutionbox}
\begin{mnemonicbox}
``Every Analysis Specification Validates'' - Key
phases of requirements analysis.

\end{mnemonicbox}
\begin{center}\rule{0.5\linewidth}{0.5pt}\end{center}

\subsection*{Question 3(a) [3 marks]}\label{q3a}

\textbf{Explain Gantt Chart.}

\begin{solutionbox}

A Gantt Chart is a visual project management tool that displays project
tasks against a timeline, showing task duration, dependencies, and
progress.


{\def\LTcaptype{none} % do not increment counter
\vspace{-5pt}
\captionof{table}{Gantt Chart Components}
\vspace{-10pt}
\begin{longtable}[]{@{}
  >{\raggedright\arraybackslash}p{(\linewidth - 4\tabcolsep) * \real{0.3333}}
  >{\raggedright\arraybackslash}p{(\linewidth - 4\tabcolsep) * \real{0.3939}}
  >{\raggedright\arraybackslash}p{(\linewidth - 4\tabcolsep) * \real{0.2727}}@{}}
\toprule\noalign{}
\begin{minipage}[b]{\linewidth}\raggedright
Component
\end{minipage} & \begin{minipage}[b]{\linewidth}\raggedright
Description
\end{minipage} & \begin{minipage}[b]{\linewidth}\raggedright
Purpose
\end{minipage} \\
\midrule\noalign{}
\endhead
\bottomrule\noalign{}
\endlastfoot
\textbf{Tasks} & Project activities listed vertically & Shows work
breakdown \\
\textbf{Timeline} & Horizontal time scale & Displays project duration \\
\textbf{Bars} & Horizontal bars showing task duration & Visual task
representation \\
\textbf{Dependencies} & Lines connecting related tasks & Shows task
relationships \\
\textbf{Milestones} & Key project checkpoints & Marks important
events \\
\end{longtable}
}

\textbf{Diagram: Sample Gantt Chart}

\begin{verbatim}
Task Name       | Week 1 | Week 2 | Week 3 | Week 4 |
Requirements    |████████|        |        |        |
Design          |        |████████|████████|        |
Coding          |        |        |████████|████████|
Testing         |        |        |        |████████|
\end{verbatim}

\textbf{Benefits:}

\begin{itemize}
\tightlist
\item
  \textbf{Visual clarity}: Easy to understand project timeline
\item
  \textbf{Progress tracking}: Shows completed vs remaining work
\item
  \textbf{Resource planning}: Helps allocate resources effectively
\end{itemize}

\end{solutionbox}
\begin{mnemonicbox}
``Gantt Graphs Timeline Tasks'' - Visual timeline
representation of project tasks.

\end{mnemonicbox}
\begin{center}\rule{0.5\linewidth}{0.5pt}\end{center}

\subsection*{Question 3(b) [4 marks]}\label{q3b}

\textbf{Write in brief: Responsibilities and skills of software project
manager.}

\begin{solutionbox}

A software project manager oversees the entire software development
lifecycle, ensuring projects are completed on time, within budget, and
meet quality standards.


{\def\LTcaptype{none} % do not increment counter
\vspace{-5pt}
\captionof{table}{Project Manager Responsibilities}
\vspace{-10pt}
\begin{longtable}[]{@{}
  >{\raggedright\arraybackslash}p{(\linewidth - 4\tabcolsep) * \real{0.2273}}
  >{\raggedright\arraybackslash}p{(\linewidth - 4\tabcolsep) * \real{0.4091}}
  >{\raggedright\arraybackslash}p{(\linewidth - 4\tabcolsep) * \real{0.3636}}@{}}
\toprule\noalign{}
\begin{minipage}[b]{\linewidth}\raggedright
Category
\end{minipage} & \begin{minipage}[b]{\linewidth}\raggedright
Responsibilities
\end{minipage} & \begin{minipage}[b]{\linewidth}\raggedright
Key Activities
\end{minipage} \\
\midrule\noalign{}
\endhead
\bottomrule\noalign{}
\endlastfoot
\textbf{Planning} & Project scope and timeline definition & WBS
creation, scheduling \\
\textbf{Resource Management} & Team allocation and coordination & Staff
assignment, skill matching \\
\textbf{Risk Management} & Identify and mitigate project risks & Risk
assessment, contingency planning \\
\textbf{Communication} & Stakeholder coordination & Status reporting,
meetings \\
\textbf{Quality Assurance} & Ensure deliverable quality & Review
processes, standards \\
\end{longtable}
}

\textbf{Essential Skills:}

\begin{itemize}
\tightlist
\item
  \textbf{Technical skills}: Understanding of software development
  processes
\item
  \textbf{Leadership skills}: Team motivation and guidance
\item
  \textbf{Communication skills}: Effective stakeholder interaction
\item
  \textbf{Problem-solving skills}: Quick issue resolution
\item
  \textbf{Time management}: Efficient task prioritization
\end{itemize}

\textbf{Key Responsibilities:}

\begin{itemize}
\tightlist
\item
  \textbf{Project planning}: Define scope, timeline, and resources
\item
  \textbf{Team coordination}: Manage development team activities
\item
  \textbf{Stakeholder management}: Maintain client and sponsor
  relationships
\item
  \textbf{Risk mitigation}: Identify and address potential problems
\end{itemize}

\end{solutionbox}
\begin{mnemonicbox}
``Managers Plan Resources Risks Communication'' -
Core responsibilities of project managers.

\end{mnemonicbox}
\begin{center}\rule{0.5\linewidth}{0.5pt}\end{center}

\subsection*{Question 3(c) [7 marks]}\label{q3c}

\textbf{Write a short note on Risk Management.}

\begin{solutionbox}

Risk Management is the systematic process of identifying, analyzing, and
responding to project risks that could impact software development
success.


{\def\LTcaptype{none} % do not increment counter
\vspace{-5pt}
\captionof{table}{Risk Management Process}
\vspace{-10pt}
\begin{longtable}[]{@{}
  >{\raggedright\arraybackslash}p{(\linewidth - 6\tabcolsep) * \real{0.1707}}
  >{\raggedright\arraybackslash}p{(\linewidth - 6\tabcolsep) * \real{0.2927}}
  >{\raggedright\arraybackslash}p{(\linewidth - 6\tabcolsep) * \real{0.2927}}
  >{\raggedright\arraybackslash}p{(\linewidth - 6\tabcolsep) * \real{0.2439}}@{}}
\toprule\noalign{}
\begin{minipage}[b]{\linewidth}\raggedright
Phase
\end{minipage} & \begin{minipage}[b]{\linewidth}\raggedright
Activities
\end{minipage} & \begin{minipage}[b]{\linewidth}\raggedright
Techniques
\end{minipage} & \begin{minipage}[b]{\linewidth}\raggedright
Outcomes
\end{minipage} \\
\midrule\noalign{}
\endhead
\bottomrule\noalign{}
\endlastfoot
\textbf{Risk Identification} & Find potential risks & Brainstorming,
checklists & Risk register \\
\textbf{Risk Analysis} & Assess probability and impact & Risk matrices,
scoring & Prioritized risks \\
\textbf{Risk Planning} & Develop response strategies & Mitigation,
avoidance & Risk response plans \\
\textbf{Risk Monitoring} & Track and control risks & Regular reviews &
Updated risk status \\
\end{longtable}
}

\textbf{Types of Software Project Risks:}

\textbf{Technical Risks:}

\begin{itemize}
\tightlist
\item
  \textbf{Technology uncertainty}: New or unproven technologies
\item
  \textbf{Performance issues}: System not meeting performance
  requirements
\item
  \textbf{Integration problems}: Difficulty combining system components
\end{itemize}

\textbf{Project Risks:}

\begin{itemize}
\tightlist
\item
  \textbf{Schedule delays}: Tasks taking longer than estimated
\item
  \textbf{Resource constraints}: Insufficient staff or budget
\item
  \textbf{Scope creep}: Uncontrolled requirement changes
\end{itemize}

\textbf{Business Risks:}

\begin{itemize}
\tightlist
\item
  \textbf{Market changes}: Shifting business requirements
\item
  \textbf{Competition}: Competitive products affecting project value
\item
  \textbf{Regulatory changes}: New compliance requirements
\end{itemize}

\textbf{Risk Response Strategies:}

\begin{itemize}
\tightlist
\item
  \textbf{Risk Avoidance}: Eliminate risk by changing project approach
\item
  \textbf{Risk Mitigation}: Reduce probability or impact of risk
\item
  \textbf{Risk Transfer}: Shift risk to third party (insurance,
  outsourcing)
\item
  \textbf{Risk Acceptance}: Accept risk and develop contingency plans
\end{itemize}

\textbf{Risk Monitoring Techniques:}

\begin{itemize}
\tightlist
\item
  \textbf{Regular risk reviews}: Periodic assessment of risk status
\item
  \textbf{Risk metrics}: Quantitative measures of risk exposure
\item
  \textbf{Early warning indicators}: Signals of emerging risks
\end{itemize}

\end{solutionbox}
\begin{mnemonicbox}
``Identify Analyze Plan Monitor'' - Four phases of
risk management process.

\end{mnemonicbox}
\begin{center}\rule{0.5\linewidth}{0.5pt}\end{center}

\subsection*{Question 3(a OR) [3
marks]}\label{question-3a-or-3-marks}

\textbf{Explain WBS with example.}

\begin{solutionbox}

Work Breakdown Structure (WBS) is a hierarchical decomposition of
project work into smaller, manageable components that can be easily
estimated, assigned, and tracked.

\textbf{Diagram: WBS Example for E-commerce Website}

\begin{center}
\textbf{Mermaid Diagram (Code)}
\begin{verbatim}
{Shaded}
{Highlighting}[]
graph TD
    A[E{-commerce Website] {-}{-}{} B[Frontend Development]}
    A {-{-}{} C[Backend Development]}
    A {-{-}{} D[Testing]}
    A {-{-}{} E[Deployment]}
    
    B {-{-}{} B1[User Interface]}
    B {-{-}{} B2[Shopping Cart]}
    B {-{-}{} B3[Payment Gateway]}
    
    C {-{-}{} C1[Database Design]}
    C {-{-}{} C2[User Management]}
    C {-{-}{} C3[Order Processing]}
    
    D {-{-}{} D1[Unit Testing]}
    D {-{-}{} D2[Integration Testing]}
    D {-{-}{} D3[User Acceptance Testing]}
{Highlighting}
{Shaded}
\end{verbatim}
\end{center}

\textbf{WBS Characteristics:}

\begin{itemize}
\tightlist
\item
  \textbf{Hierarchical structure}: Top-down breakdown of project scope
\item
  \textbf{100\% rule}: WBS includes 100\% of work defined by project
  scope
\item
  \textbf{Mutually exclusive}: No overlap between WBS elements
\end{itemize}

\end{solutionbox}
\begin{mnemonicbox}
``Work Breaks Small'' - Breaking work into smaller
manageable pieces.

\end{mnemonicbox}
\begin{center}\rule{0.5\linewidth}{0.5pt}\end{center}

\subsection*{Question 3(b OR) [4
marks]}\label{question-3b-or-4-marks}

\textbf{Explain Project monitoring and control.}

\begin{solutionbox}

Project monitoring and control involves tracking project progress,
comparing actual performance against planned performance, and taking
corrective actions when necessary.


{\def\LTcaptype{none} % do not increment counter
\vspace{-5pt}
\captionof{table}{Monitoring and Control Activities}
\vspace{-10pt}
\begin{longtable}[]{@{}
  >{\raggedright\arraybackslash}p{(\linewidth - 4\tabcolsep) * \real{0.2439}}
  >{\raggedright\arraybackslash}p{(\linewidth - 4\tabcolsep) * \real{0.3171}}
  >{\raggedright\arraybackslash}p{(\linewidth - 4\tabcolsep) * \real{0.4390}}@{}}
\toprule\noalign{}
\begin{minipage}[b]{\linewidth}\raggedright
Activity
\end{minipage} & \begin{minipage}[b]{\linewidth}\raggedright
Description
\end{minipage} & \begin{minipage}[b]{\linewidth}\raggedright
Tools/Techniques
\end{minipage} \\
\midrule\noalign{}
\endhead
\bottomrule\noalign{}
\endlastfoot
\textbf{Progress Tracking} & Monitor task completion & Gantt charts,
dashboards \\
\textbf{Performance Measurement} & Compare actual vs planned & Earned
value analysis \\
\textbf{Quality Control} & Ensure deliverable quality & Reviews,
testing \\
\textbf{Risk Monitoring} & Track identified risks & Risk registers,
reports \\
\textbf{Change Control} & Manage scope changes & Change request
process \\
\end{longtable}
}

\textbf{Key Monitoring Metrics:}

\begin{itemize}
\tightlist
\item
  \textbf{Schedule performance}: Tasks completed on time
\item
  \textbf{Cost performance}: Budget utilization and variance
\item
  \textbf{Quality metrics}: Defect rates, customer satisfaction
\item
  \textbf{Resource utilization}: Team productivity and efficiency
\end{itemize}

\textbf{Control Actions:}

\begin{itemize}
\tightlist
\item
  \textbf{Corrective actions}: Address performance deviations
\item
  \textbf{Preventive actions}: Avoid potential problems
\item
  \textbf{Change management}: Handle scope modifications
\end{itemize}

\end{solutionbox}
\begin{mnemonicbox}
``Monitor Progress Performance Quality'' - Key areas
of project monitoring.

\end{mnemonicbox}
\begin{center}\rule{0.5\linewidth}{0.5pt}\end{center}

\subsection*{Question 3(c OR) [7
marks]}\label{question-3c-or-7-marks}

\textbf{Explain Critical Path Method (CPM) with a suitable example.}

\begin{solutionbox}

Critical Path Method (CPM) is a project management technique that
identifies the longest sequence of dependent tasks and determines the
minimum project completion time.


{\def\LTcaptype{none} % do not increment counter
\vspace{-5pt}
\captionof{table}{Sample Project Tasks}
\vspace{-10pt}
\begin{longtable}[]{@{}lll@{}}
\toprule\noalign{}
Task & Duration (Days) & Predecessors \\
\midrule\noalign{}
\endhead
\bottomrule\noalign{}
\endlastfoot
A - Requirements & 5 & - \\
B - Design & 8 & A \\
C - Database Setup & 6 & A \\
D - Frontend Coding & 10 & B \\
E - Backend Coding & 12 & B, C \\
F - Integration & 4 & D, E \\
G - Testing & 6 & F \\
\end{longtable}
}

\textbf{Diagram: CPM Network}

\begin{center}
\textbf{Mermaid Diagram (Code)}
\begin{verbatim}
{Shaded}
{Highlighting}[]
graph LR
    A[A:5] {-{-}{} B[B:8]}
    A {-{-}{} C[C:6]}
    B {-{-}{} D[D:10]}
    B {-{-}{} E[E:12]}
    C {-{-}{} E}
    D {-{-}{} F[F:4]}
    E {-{-}{} F}
    F {-{-}{} G[G:6]}
{Highlighting}
{Shaded}
\end{verbatim}
\end{center}

\textbf{Critical Path Calculation:}

\begin{itemize}
\tightlist
\item
  \textbf{Path 1}: A \rightarrow B \rightarrow D \rightarrow F \rightarrow G = 5 + 8 + 10 + 4 + 6 = 33 days
\item
  \textbf{Path 2}: A \rightarrow B \rightarrow E \rightarrow F \rightarrow G = 5 + 8 + 12 + 4 + 6 = 35 days
  (Critical Path)
\item
  \textbf{Path 3}: A \rightarrow C \rightarrow E \rightarrow F \rightarrow G = 5 + 6 + 12 + 4 + 6 = 33 days
\end{itemize}

\textbf{CPM Benefits:}

\begin{itemize}
\tightlist
\item
  \textbf{Project duration}: Determines minimum completion time
\item
  \textbf{Critical activities}: Identifies tasks that cannot be delayed
\item
  \textbf{Float calculation}: Shows available slack time for
  non-critical tasks
\item
  \textbf{Resource optimization}: Helps allocate resources efficiently
\end{itemize}

\textbf{CPM Steps:}

\begin{enumerate}
\tightlist
\item
  \textbf{Activity identification}: List all project activities
\item
  \textbf{Dependency mapping}: Determine task relationships
\item
  \textbf{Duration estimation}: Estimate time for each activity
\item
  \textbf{Network construction}: Create project network diagram
\item
  \textbf{Critical path calculation}: Find longest path through network
\end{enumerate}

\textbf{Float Types:}

\begin{itemize}
\tightlist
\item
  \textbf{Total Float}: Maximum delay without affecting project
  completion
\item
  \textbf{Free Float}: Delay without affecting successor activities
\item
  \textbf{Independent Float}: Delay without affecting predecessors or
  successors
\end{itemize}

\end{solutionbox}
\begin{mnemonicbox}
``Critical Paths Minimize Project Duration'' - CPM
finds longest path determining minimum time.

\end{mnemonicbox}
\begin{center}\rule{0.5\linewidth}{0.5pt}\end{center}

\subsection*{Question 4(a) [3 marks]}\label{q4a}

\textbf{Write a note on classification of design activities.}

\begin{solutionbox}

Software design activities are systematically classified to organize the
design process and ensure comprehensive system development.


{\def\LTcaptype{none} % do not increment counter
\vspace{-5pt}
\captionof{table}{Classification of Design Activities}
\vspace{-10pt}
\begin{longtable}[]{@{}
  >{\raggedright\arraybackslash}p{(\linewidth - 4\tabcolsep) * \real{0.4000}}
  >{\raggedright\arraybackslash}p{(\linewidth - 4\tabcolsep) * \real{0.3000}}
  >{\raggedright\arraybackslash}p{(\linewidth - 4\tabcolsep) * \real{0.3000}}@{}}
\toprule\noalign{}
\begin{minipage}[b]{\linewidth}\raggedright
Classification
\end{minipage} & \begin{minipage}[b]{\linewidth}\raggedright
Activities
\end{minipage} & \begin{minipage}[b]{\linewidth}\raggedright
Focus Area
\end{minipage} \\
\midrule\noalign{}
\endhead
\bottomrule\noalign{}
\endlastfoot
\textbf{Architectural Design} & System structure, components &
High-level organization \\
\textbf{Interface Design} & User interface, system interfaces &
Interaction design \\
\textbf{Component Design} & Module details, algorithms & Low-level
implementation \\
\textbf{Data Design} & Database, data structures & Data organization \\
\end{longtable}
}

\textbf{Design Activity Levels:}

\begin{itemize}
\tightlist
\item
  \textbf{System Level}: Overall system architecture and major
  components
\item
  \textbf{Subsystem Level}: Individual subsystem design and interfaces
\item
  \textbf{Component Level}: Detailed module design and algorithms
\end{itemize}

\textbf{Design Approaches:}

\begin{itemize}
\tightlist
\item
  \textbf{Top-down design}: Start with high-level and decompose
\item
  \textbf{Bottom-up design}: Build from individual components upward
\end{itemize}

\end{solutionbox}
\begin{mnemonicbox}
``Architects Interface Components Data'' - Four main
design activity classifications.

\end{mnemonicbox}
\begin{center}\rule{0.5\linewidth}{0.5pt}\end{center}

\subsection*{Question 4(b) [4 marks]}\label{q4b}

\textbf{Define Coupling. Explain its classification.}

\begin{solutionbox}

Coupling refers to the degree of interdependence between software
modules. Lower coupling indicates better software design with more
maintainable and flexible code.


{\def\LTcaptype{none} % do not increment counter
\vspace{-5pt}
\captionof{table}{Types of Coupling (Loosest to Tightest)}
\vspace{-10pt}
\begin{longtable}[]{@{}
  >{\raggedright\arraybackslash}p{(\linewidth - 4\tabcolsep) * \real{0.4054}}
  >{\raggedright\arraybackslash}p{(\linewidth - 4\tabcolsep) * \real{0.3514}}
  >{\raggedright\arraybackslash}p{(\linewidth - 4\tabcolsep) * \real{0.2432}}@{}}
\toprule\noalign{}
\begin{minipage}[b]{\linewidth}\raggedright
Coupling Type
\end{minipage} & \begin{minipage}[b]{\linewidth}\raggedright
Description
\end{minipage} & \begin{minipage}[b]{\linewidth}\raggedright
Example
\end{minipage} \\
\midrule\noalign{}
\endhead
\bottomrule\noalign{}
\endlastfoot
\textbf{Data Coupling} & Modules communicate through parameters &
Function calls with simple parameters \\
\textbf{Stamp Coupling} & Modules share composite data structure &
Passing record/structure as parameter \\
\textbf{Control Coupling} & One module controls another's execution &
Passing control flags \\
\textbf{External Coupling} & Modules depend on external format & Shared
file format or protocol \\
\textbf{Common Coupling} & Modules share global data & Global variables
access \\
\textbf{Content Coupling} & One module modifies another's data & Direct
access to another module's data \\
\end{longtable}
}

\textbf{Coupling Characteristics:}

\begin{itemize}
\tightlist
\item
  \textbf{Data coupling}: Best type - minimal interdependence
\item
  \textbf{Stamp coupling}: Acceptable - shared data structures
\item
  \textbf{Control coupling}: Moderate - control information passed
\item
  \textbf{Content coupling}: Worst type - high interdependence
\end{itemize}

\textbf{Benefits of Loose Coupling:}

\begin{itemize}
\tightlist
\item
  \textbf{Maintainability}: Easier to modify individual modules
\item
  \textbf{Reusability}: Modules can be used in different contexts
\item
  \textbf{Testability}: Modules can be tested independently
\end{itemize}

\end{solutionbox}
\begin{mnemonicbox}
``Data Stamp Control External Common Content'' -
Coupling types from loose to tight.

\end{mnemonicbox}
\begin{center}\rule{0.5\linewidth}{0.5pt}\end{center}

\subsection*{Question 4(c) [7 marks]}\label{q4c}

\textbf{Draw a use case diagram for online shopping web application.}

\begin{solutionbox}

A use case diagram shows the functional requirements of an online
shopping system by illustrating actors and their interactions with the
system.

\textbf{Diagram: Online Shopping Use Case Diagram}

\begin{center}
\textbf{Mermaid Diagram (Code)}
\begin{verbatim}
{Shaded}
{Highlighting}[]
graph TD
    Customer((Customer))
    Admin((Admin))
    PaymentSystem((Payment System))
    
    Customer {-{-}{} UC1[Browse Products]}
    Customer {-{-}{} UC2[Search Products]}
    Customer {-{-}{} UC3[Add to Cart]}
    Customer {-{-}{} UC4[View Cart]}
    Customer {-{-}{} UC5[Checkout]}
    Customer {-{-}{} UC6[Make Payment]}
    Customer {-{-}{} UC7[Track Order]}
    Customer {-{-}{} UC8[Register Account]}
    Customer {-{-}{} UC9[Login/Logout]}
    Customer {-{-}{} UC10[View Order History]}
    
    Admin {-{-}{} UC11[Manage Products]}
    Admin {-{-}{} UC12[Manage Categories]}
    Admin {-{-}{} UC13[Process Orders]}
    Admin {-{-}{} UC14[Generate Reports]}
    Admin {-{-}{} UC15[Manage Users]}
    
    UC6 {-{-}{} PaymentSystem}
    
    UC5 {-.{-}{}|includes| UC3}
    UC5 {-.{-}{}|includes| UC6}
    UC11 {-.{-}{}|extends| UC16[Update Inventory]}
{Highlighting}
{Shaded}
\end{verbatim}
\end{center}

\textbf{Key Use Cases Explained:}

\textbf{Customer Use Cases:}

\begin{itemize}
\tightlist
\item
  \textbf{Browse Products}: View available products by category
\item
  \textbf{Search Products}: Find specific products using keywords
\item
  \textbf{Shopping Cart}: Add, remove, and modify cart items
\item
  \textbf{Checkout Process}: Complete purchase with shipping details
\item
  \textbf{Payment Processing}: Handle secure payment transactions
\item
  \textbf{Order Management}: Track orders and view purchase history
\end{itemize}

\textbf{Admin Use Cases:}

\begin{itemize}
\tightlist
\item
  \textbf{Product Management}: Add, edit, delete products and categories
\item
  \textbf{Order Processing}: Manage order fulfillment and shipping
\item
  \textbf{User Management}: Handle customer accounts and permissions
\item
  \textbf{Reporting}: Generate sales and inventory reports
\end{itemize}

\textbf{System Relationships:}

\begin{itemize}
\tightlist
\item
  \textbf{Include}: Mandatory sub-use cases (checkout includes payment)
\item
  \textbf{Extend}: Optional extensions (inventory update extends product
  management)
\item
  \textbf{Inheritance}: Specialized actor behaviors
\end{itemize}

\textbf{Actors:}

\begin{itemize}
\tightlist
\item
  \textbf{Primary Actors}: Customer, Admin (initiate use cases)
\item
  \textbf{Secondary Actors}: Payment System (respond to system requests)
\end{itemize}

\end{solutionbox}
\begin{mnemonicbox}
``Customers Browse Buy, Admins Manage Monitor'' -
Core use case categories.

\end{mnemonicbox}
\begin{center}\rule{0.5\linewidth}{0.5pt}\end{center}

\subsection*{Question 4(a OR) [3
marks]}\label{question-4a-or-3-marks}

\textbf{Explain the characteristics of good UI.}

\begin{solutionbox}

Good User Interface (UI) design ensures effective user interaction with
software systems through intuitive and user-friendly design principles.


{\def\LTcaptype{none} % do not increment counter
\vspace{-5pt}
\captionof{table}{Characteristics of Good UI}
\vspace{-10pt}
\begin{longtable}[]{@{}
  >{\raggedright\arraybackslash}p{(\linewidth - 4\tabcolsep) * \real{0.4211}}
  >{\raggedright\arraybackslash}p{(\linewidth - 4\tabcolsep) * \real{0.3421}}
  >{\raggedright\arraybackslash}p{(\linewidth - 4\tabcolsep) * \real{0.2368}}@{}}
\toprule\noalign{}
\begin{minipage}[b]{\linewidth}\raggedright
Characteristic
\end{minipage} & \begin{minipage}[b]{\linewidth}\raggedright
Description
\end{minipage} & \begin{minipage}[b]{\linewidth}\raggedright
Example
\end{minipage} \\
\midrule\noalign{}
\endhead
\bottomrule\noalign{}
\endlastfoot
\textbf{Consistency} & Uniform design across application & Same button
styles throughout \\
\textbf{Simplicity} & Easy to understand and use & Minimal, clean
interface \\
\textbf{Visibility} & Important elements clearly visible & Key actions
prominently displayed \\
\textbf{Feedback} & System responds to user actions & Progress bars,
confirmations \\
\textbf{Error Prevention} & Prevents user mistakes & Input validation,
confirmations \\
\textbf{Flexibility} & Accommodates different user needs & Customizable
interfaces \\
\end{longtable}
}

\textbf{UI Design Principles:}

\begin{itemize}
\tightlist
\item
  \textbf{User-centered}: Design focused on user needs and goals
\item
  \textbf{Accessibility}: Usable by people with different abilities
\item
  \textbf{Efficiency}: Minimizes steps to complete tasks
\end{itemize}

\end{solutionbox}
\begin{mnemonicbox}
``Consistent Simple Visible Feedback'' - Core UI
design characteristics.

\end{mnemonicbox}
\begin{center}\rule{0.5\linewidth}{0.5pt}\end{center}

\subsection*{Question 4(b OR) [4
marks]}\label{question-4b-or-4-marks}

\textbf{Define Cohesion. Explain its classification.}

\begin{solutionbox}

Cohesion refers to how closely related and focused the responsibilities
of a single module are. High cohesion indicates well-designed modules
with related functionality.


{\def\LTcaptype{none} % do not increment counter
\vspace{-5pt}
\captionof{table}{Types of Cohesion (Weakest to Strongest)}
\vspace{-10pt}
\begin{longtable}[]{@{}
  >{\raggedright\arraybackslash}p{(\linewidth - 4\tabcolsep) * \real{0.4054}}
  >{\raggedright\arraybackslash}p{(\linewidth - 4\tabcolsep) * \real{0.3514}}
  >{\raggedright\arraybackslash}p{(\linewidth - 4\tabcolsep) * \real{0.2432}}@{}}
\toprule\noalign{}
\begin{minipage}[b]{\linewidth}\raggedright
Cohesion Type
\end{minipage} & \begin{minipage}[b]{\linewidth}\raggedright
Description
\end{minipage} & \begin{minipage}[b]{\linewidth}\raggedright
Example
\end{minipage} \\
\midrule\noalign{}
\endhead
\bottomrule\noalign{}
\endlastfoot
\textbf{Coincidental} & Elements grouped arbitrarily & Utility module
with unrelated functions \\
\textbf{Logical} & Elements perform similar logical functions & All
input/output operations \\
\textbf{Temporal} & Elements executed at same time & System
initialization module \\
\textbf{Procedural} & Elements follow specific sequence & Sequential
processing steps \\
\textbf{Communicational} & Elements operate on same data & Module
processing same record \\
\textbf{Sequential} & Output of one element is input to next & Data
transformation pipeline \\
\textbf{Functional} & All elements contribute to single task & Calculate
employee salary \\
\end{longtable}
}

\textbf{Cohesion Characteristics:}

\begin{itemize}
\tightlist
\item
  \textbf{Functional cohesion}: Best type - single, well-defined purpose
\item
  \textbf{Sequential cohesion}: Good - data flows through module
\item
  \textbf{Communicational cohesion}: Acceptable - operates on same data
\item
  \textbf{Coincidental cohesion}: Worst type - no logical relationship
\end{itemize}

\textbf{Benefits of High Cohesion:}

\begin{itemize}
\tightlist
\item
  \textbf{Maintainability}: Easier to understand and modify
\item
  \textbf{Reliability}: Less likely to have errors
\item
  \textbf{Reusability}: Single-purpose modules more reusable
\end{itemize}

\end{solutionbox}
\begin{mnemonicbox}
``Coincidental Logical Temporal Procedural
Communicational Sequential Functional'' - Cohesion types from weak to
strong.

\end{mnemonicbox}
\begin{center}\rule{0.5\linewidth}{0.5pt}\end{center}

\subsection*{Question 4(c OR) [7
marks]}\label{question-4c-or-7-marks}

\textbf{Draw context diagram for library system.}

\begin{solutionbox}

A context diagram shows the library system as a single process with its
external entities and data flows, providing a high-level view of system
boundaries.

\textbf{Diagram: Library System Context Diagram}

\begin{center}
\textbf{Mermaid Diagram (Code)}
\begin{verbatim}
{Shaded}
{Highlighting}[]
graph TD
    Student((Student))
    Librarian((Librarian))
    Administrator((Administrator))
    Publisher((Publisher))
    
    LibrarySystem[Library Management System]
    
    Student {-{-}{}|Book Request| LibrarySystem}
    Student {-{-}{}|Return Request| LibrarySystem}
    LibrarySystem {-{-}{}|Book Details| Student}
    LibrarySystem {-{-}{}|Due Date Notice| Student}
    
    Librarian {-{-}{}|Issue/Return Books| LibrarySystem}
    Librarian {-{-}{}|Search Books| LibrarySystem}
    LibrarySystem {-{-}{}|Book Status| Librarian}
    LibrarySystem {-{-}{}|Member Details| Librarian}
    
    Administrator {-{-}{}|Add/Remove Books| LibrarySystem}
    Administrator {-{-}{}|Manage Members| LibrarySystem}
    LibrarySystem {-{-}{}|System Reports| Administrator}
    LibrarySystem {-{-}{}|Overdue Reports| Administrator}
    
    Publisher {-{-}{}|Book Catalog| LibrarySystem}
    LibrarySystem {-{-}{}|Purchase Orders| Publisher}
{Highlighting}
{Shaded}
\end{verbatim}
\end{center}

\textbf{External Entities:}

\textbf{Student (Library Member):}

\begin{itemize}
\tightlist
\item
  \textbf{Inputs}: Book search requests, reservation requests, return
  notifications
\item
  \textbf{Outputs}: Book availability information, due dates, fine
  details
\end{itemize}

\textbf{Librarian:}

\begin{itemize}
\tightlist
\item
  \textbf{Inputs}: Book issue/return transactions, member verification
\item
  \textbf{Outputs}: Book status updates, member information, transaction
  confirmations
\end{itemize}

\textbf{Administrator:}

\begin{itemize}
\tightlist
\item
  \textbf{Inputs}: New book additions, member management, system
  configuration
\item
  \textbf{Outputs}: System reports, statistics, overdue notifications
\end{itemize}

\textbf{Publisher/Supplier:}

\begin{itemize}
\tightlist
\item
  \textbf{Inputs}: Book catalogs, availability updates
\item
  \textbf{Outputs}: Purchase orders, procurement requests
\end{itemize}

\textbf{Data Flows:}

\begin{itemize}
\tightlist
\item
  \textbf{Book Information}: Details about books, availability, location
\item
  \textbf{Member Data}: Student/faculty information, borrowing history
\item
  \textbf{Transaction Records}: Issue/return details, fine calculations
\item
  \textbf{Reports}: Usage statistics, overdue lists, inventory reports
\end{itemize}

\textbf{System Boundary:} The context diagram clearly defines what is
inside the library system (book management, member management,
transaction processing) and what is outside (external entities like
students, staff, and suppliers).

\textbf{Key Data Stores (Internal to System):}

\begin{itemize}
\tightlist
\item
  Book catalog database
\item
  Member information database
\item
  Transaction history database
\item
  Fine and payment records
\end{itemize}

\end{solutionbox}
\begin{mnemonicbox}
``Students Librarians Admins Publishers'' - Four main
external entities interacting with library system.

\end{mnemonicbox}
\begin{center}\rule{0.5\linewidth}{0.5pt}\end{center}

\subsection*{Question 5(a) [3 marks]}\label{q5a}

\textbf{Differentiate verification and validation.}

\begin{solutionbox}

Verification and validation are two complementary quality assurance
processes that ensure software meets requirements and user needs.


{\def\LTcaptype{none} % do not increment counter
\vspace{-5pt}
\captionof{table}{Verification vs Validation}
\vspace{-10pt}
\begin{longtable}[]{@{}
  >{\raggedright\arraybackslash}p{(\linewidth - 4\tabcolsep) * \real{0.2353}}
  >{\raggedright\arraybackslash}p{(\linewidth - 4\tabcolsep) * \real{0.4118}}
  >{\raggedright\arraybackslash}p{(\linewidth - 4\tabcolsep) * \real{0.3529}}@{}}
\toprule\noalign{}
\begin{minipage}[b]{\linewidth}\raggedright
Aspect
\end{minipage} & \begin{minipage}[b]{\linewidth}\raggedright
Verification
\end{minipage} & \begin{minipage}[b]{\linewidth}\raggedright
Validation
\end{minipage} \\
\midrule\noalign{}
\endhead
\bottomrule\noalign{}
\endlastfoot
\textbf{Question} & Are we building the product right? & Are we building
the right product? \\
\textbf{Focus} & Process and standards compliance & Product meets user
needs \\
\textbf{When} & Throughout development & After product completion \\
\textbf{Methods} & Reviews, inspections, walkthroughs & Testing, user
acceptance \\
\textbf{Cost} & Lower cost of defect detection & Higher cost but
essential \\
\textbf{Objective} & Ensure conformance to specifications & Ensure
fitness for use \\
\end{longtable}
}

\textbf{Verification Activities:}

\begin{itemize}
\tightlist
\item
  \textbf{Code reviews}: Checking code against coding standards
\item
  \textbf{Design reviews}: Ensuring design meets requirements
\item
  \textbf{Document reviews}: Verifying documentation completeness
\end{itemize}

\textbf{Validation Activities:}

\begin{itemize}
\tightlist
\item
  \textbf{System testing}: Testing complete integrated system
\item
  \textbf{User acceptance testing}: End-user validation of functionality
\item
  \textbf{Performance testing}: Validating system performance
  requirements
\end{itemize}

\end{solutionbox}
\begin{mnemonicbox}
``Verification Verifies Process, Validation Validates
Product'' - Key distinction between the two.

\end{mnemonicbox}
\begin{center}\rule{0.5\linewidth}{0.5pt}\end{center}

\subsection*{Question 5(b) [4 marks]}\label{q5b}

\textbf{Explain Code Review.}

\begin{solutionbox}

Code Review is a systematic examination of source code by developers
other than the author to identify defects, improve code quality, and
ensure adherence to coding standards.


{\def\LTcaptype{none} % do not increment counter
\vspace{-5pt}
\captionof{table}{Types of Code Review}
\vspace{-10pt}
\begin{longtable}[]{@{}
  >{\raggedright\arraybackslash}p{(\linewidth - 6\tabcolsep) * \real{0.1364}}
  >{\raggedright\arraybackslash}p{(\linewidth - 6\tabcolsep) * \real{0.2955}}
  >{\raggedright\arraybackslash}p{(\linewidth - 6\tabcolsep) * \real{0.3182}}
  >{\raggedright\arraybackslash}p{(\linewidth - 6\tabcolsep) * \real{0.2500}}@{}}
\toprule\noalign{}
\begin{minipage}[b]{\linewidth}\raggedright
Type
\end{minipage} & \begin{minipage}[b]{\linewidth}\raggedright
Description
\end{minipage} & \begin{minipage}[b]{\linewidth}\raggedright
Participants
\end{minipage} & \begin{minipage}[b]{\linewidth}\raggedright
Formality
\end{minipage} \\
\midrule\noalign{}
\endhead
\bottomrule\noalign{}
\endlastfoot
\textbf{Code Walkthrough} & Author explains code to reviewers & Author +
2-3 reviewers & Informal \\
\textbf{Code Inspection} & Formal systematic examination & Moderator,
author, reviewers & Formal \\
\textbf{Peer Review} & Colleague reviews code changes & 1-2 peer
developers & Semi-formal \\
\textbf{Tool-Assisted Review} & Automated tools assist review & Author +
automated tools & Variable \\
\end{longtable}
}

\textbf{Code Review Process:}

\begin{enumerate}
\tightlist
\item
  \textbf{Preparation}: Author prepares code and documentation
\item
  \textbf{Review Meeting}: Team examines code systematically
\item
  \textbf{Defect Logging}: Issues and improvements documented
\item
  \textbf{Follow-up}: Author addresses identified issues
\item
  \textbf{Re-review}: Verification of fixes if necessary
\end{enumerate}

\textbf{Review Criteria:}

\begin{itemize}
\tightlist
\item
  \textbf{Functionality}: Code performs intended operations correctly
\item
  \textbf{Standards Compliance}: Follows coding conventions and
  guidelines
\item
  \textbf{Maintainability}: Code is readable and well-documented
\item
  \textbf{Performance}: Efficient algorithms and resource usage
\end{itemize}

\textbf{Benefits:}

\begin{itemize}
\tightlist
\item
  \textbf{Defect Detection}: Early identification of bugs and issues
\item
  \textbf{Knowledge Sharing}: Team learns from each other's code
\item
  \textbf{Quality Improvement}: Consistent coding standards across team
\end{itemize}

\end{solutionbox}
\begin{mnemonicbox}
``Reviews Reveal Errors Early'' - Code reviews catch
defects before testing.

\end{mnemonicbox}
\begin{center}\rule{0.5\linewidth}{0.5pt}\end{center}

\subsection*{Question 5(c) [7 marks]}\label{q5c}

\textbf{Write a short note on White Box Testing.}

\begin{solutionbox}

White Box Testing is a software testing technique that examines the
internal structure, design, and coding of an application to verify
input-output flow and improve design and usability.


{\def\LTcaptype{none} % do not increment counter
\vspace{-5pt}
\captionof{table}{White Box Testing Techniques}
\vspace{-10pt}
\begin{longtable}[]{@{}
  >{\raggedright\arraybackslash}p{(\linewidth - 4\tabcolsep) * \real{0.2558}}
  >{\raggedright\arraybackslash}p{(\linewidth - 4\tabcolsep) * \real{0.3023}}
  >{\raggedright\arraybackslash}p{(\linewidth - 4\tabcolsep) * \real{0.4419}}@{}}
\toprule\noalign{}
\begin{minipage}[b]{\linewidth}\raggedright
Technique
\end{minipage} & \begin{minipage}[b]{\linewidth}\raggedright
Description
\end{minipage} & \begin{minipage}[b]{\linewidth}\raggedright
Coverage Criteria
\end{minipage} \\
\midrule\noalign{}
\endhead
\bottomrule\noalign{}
\endlastfoot
\textbf{Statement Coverage} & Execute every statement & All statements
executed at least once \\
\textbf{Branch Coverage} & Test all decision points & All branches
(true/false) covered \\
\textbf{Path Coverage} & Test all possible paths & All independent paths
executed \\
\textbf{Condition Coverage} & Test all conditions & All boolean
conditions tested \\
\end{longtable}
}

\textbf{White Box Testing Process:}

\begin{center}
\textbf{Mermaid Diagram (Code)}
\begin{verbatim}
{Shaded}
{Highlighting}[]
graph LR
    A[Code Analysis] {-{-}{} B[Test Case Design]}
    B {-{-}{} C[Test Execution]}
    C {-{-}{} D[Coverage Analysis]}
    D {-{-}{} E[Report Generation]}
{Highlighting}
{Shaded}
\end{verbatim}
\end{center}

\textbf{Coverage Types Explained:}

\textbf{Statement Coverage:}

\begin{itemize}
\tightlist
\item
  Ensures every line of code is executed at least once
\item
  Formula: (Statements Executed / Total Statements) \times 100
\item
  Minimum level of testing required
\end{itemize}

\textbf{Branch Coverage:}

\begin{itemize}
\tightlist
\item
  Tests all decision points (if-else, switch-case)
\item
  Ensures both true and false conditions are tested
\item
  More thorough than statement coverage
\end{itemize}

\textbf{Path Coverage:}

\begin{itemize}
\tightlist
\item
  Tests all possible execution paths through code
\item
  Most comprehensive but often impractical for complex programs
\item
  Uses cyclomatic complexity to determine paths
\end{itemize}

\textbf{Condition Coverage:}

\begin{itemize}
\tightlist
\item
  Tests all boolean sub-expressions individually
\item
  Ensures each condition evaluates to both true and false
\item
  Important for complex conditional statements
\end{itemize}

\textbf{White Box Testing Tools:}

\begin{itemize}
\tightlist
\item
  \textbf{Static Analysis Tools}: Examine code without execution
\item
  \textbf{Dynamic Analysis Tools}: Monitor code during execution
\item
  \textbf{Coverage Tools}: Measure test coverage percentage
\item
  \textbf{Profiling Tools}: Analyze performance characteristics
\end{itemize}

\textbf{Advantages:}

\begin{itemize}
\tightlist
\item
  \textbf{Thorough Testing}: Examines all code paths and logic
\item
  \textbf{Early Defect Detection}: Finds errors during development
\item
  \textbf{Optimization}: Identifies unused code and inefficiencies
\item
  \textbf{Security Testing}: Reveals potential security vulnerabilities
\end{itemize}

\textbf{Disadvantages:}

\begin{itemize}
\tightlist
\item
  \textbf{Time Consuming}: Requires detailed code knowledge
\item
  \textbf{Expensive}: Needs skilled testers familiar with code
\item
  \textbf{Limited Scope}: May miss integration and system-level issues
\item
  \textbf{Maintenance}: Test cases need updates with code changes
\end{itemize}

\textbf{White Box vs Black Box:}

\begin{itemize}
\tightlist
\item
  \textbf{White Box}: Internal structure focus, code-based testing
\item
  \textbf{Black Box}: Functional behavior focus, specification-based
  testing
\item
  \textbf{Complementary}: Both approaches needed for comprehensive
  testing
\end{itemize}

\textbf{Test Case Design Guidelines:}

\begin{itemize}
\tightlist
\item
  \textbf{Boundary Testing}: Test edge cases and limits
\item
  \textbf{Loop Testing}: Verify loop conditions and iterations
\item
  \textbf{Data Flow Testing}: Follow variable definitions and usage
\item
  \textbf{Control Flow Testing}: Test decision logic and branches
\end{itemize}

\end{solutionbox}
\begin{mnemonicbox}
``White Box Sees Inside Structure'' - Internal code
structure testing approach.

\end{mnemonicbox}
\begin{center}\rule{0.5\linewidth}{0.5pt}\end{center}

\subsection*{Question 5(a OR) [3
marks]}\label{question-5a-or-3-marks}

\textbf{List out various coding standards and guidelines.}

\begin{solutionbox}

Coding standards and guidelines ensure consistent, readable, and
maintainable code across development teams and projects.


{\def\LTcaptype{none} % do not increment counter
\vspace{-5pt}
\captionof{table}{Coding Standards Categories}
\vspace{-10pt}
\begin{longtable}[]{@{}
  >{\raggedright\arraybackslash}p{(\linewidth - 4\tabcolsep) * \real{0.3226}}
  >{\raggedright\arraybackslash}p{(\linewidth - 4\tabcolsep) * \real{0.3548}}
  >{\raggedright\arraybackslash}p{(\linewidth - 4\tabcolsep) * \real{0.3226}}@{}}
\toprule\noalign{}
\begin{minipage}[b]{\linewidth}\raggedright
Category
\end{minipage} & \begin{minipage}[b]{\linewidth}\raggedright
Standards
\end{minipage} & \begin{minipage}[b]{\linewidth}\raggedright
Examples
\end{minipage} \\
\midrule\noalign{}
\endhead
\bottomrule\noalign{}
\endlastfoot
\textbf{Naming Conventions} & Variable, function, class naming &
camelCase, PascalCase \\
\textbf{Code Structure} & Indentation, spacing, brackets & 4-space
indentation \\
\textbf{Documentation} & Comments, function headers & Inline comments,
API docs \\
\textbf{Error Handling} & Exception handling, logging & Try-catch
blocks \\
\end{longtable}
}

\textbf{Common Coding Guidelines:}

\begin{itemize}
\tightlist
\item
  \textbf{Meaningful names}: Use descriptive variable and function names
\item
  \textbf{Consistent indentation}: Use consistent spacing (2 or 4
  spaces)
\item
  \textbf{Comment code}: Explain complex logic and business rules
\item
  \textbf{Function size}: Keep functions small and focused
\item
  \textbf{Error handling}: Implement proper exception handling
\end{itemize}

\textbf{Language-Specific Standards:}

\begin{itemize}
\tightlist
\item
  \textbf{Java}: Oracle Java Code Conventions
\item
  \textbf{Python}: PEP 8 Style Guide
\item
  \textbf{JavaScript}: Airbnb JavaScript Style Guide
\item
  \textbf{C++}: Google C++ Style Guide
\end{itemize}

\end{solutionbox}
\begin{mnemonicbox}
``Names Structure Documentation Errors'' - Four main
coding standard categories.

\end{mnemonicbox}
\begin{center}\rule{0.5\linewidth}{0.5pt}\end{center}

\subsection*{Question 5(b OR) [4
marks]}\label{question-5b-or-4-marks}

\textbf{Explain Test cases and Test suite with example.}

\begin{solutionbox}

Test cases are specific conditions under which a tester determines
whether a software application is working correctly, while a test suite
is a collection of related test cases.


{\def\LTcaptype{none} % do not increment counter
\vspace{-5pt}
\captionof{table}{Test Case vs Test Suite}
\vspace{-10pt}
\begin{longtable}[]{@{}lll@{}}
\toprule\noalign{}
Aspect & Test Case & Test Suite \\
\midrule\noalign{}
\endhead
\bottomrule\noalign{}
\endlastfoot
\textbf{Definition} & Single test scenario & Collection of test cases \\
\textbf{Scope} & Specific functionality & Related functionalities \\
\textbf{Execution} & Individual test & Group execution \\
\textbf{Management} & Single test management & Batch management \\
\end{longtable}
}

\textbf{Test Case Components:}

\begin{itemize}
\tightlist
\item
  \textbf{Test Case ID}: Unique identifier (TC\_001)
\item
  \textbf{Test Description}: What is being tested
\item
  \textbf{Preconditions}: Setup requirements
\item
  \textbf{Test Steps}: Step-by-step procedure
\item
  \textbf{Expected Result}: Expected outcome
\item
  \textbf{Actual Result}: Observed outcome
\item
  \textbf{Status}: Pass/Fail/Blocked
\end{itemize}

\textbf{Example Test Case:}

\begin{verbatim}
Test Case ID: TC_LOGIN_001
Description: Verify user login with valid credentials
Preconditions: User account exists in system
Test Steps:
1. Navigate to login page
2. Enter valid username
3. Enter valid password
4. Click Login button
Expected Result: User redirected to dashboard
Actual Result: [To be filled during execution]
Status: [Pass/Fail]
\end{verbatim}

\textbf{Test Suite Example:}

\begin{itemize}
\tightlist
\item
  \textbf{Login Test Suite}: Contains all login-related test cases

  \begin{itemize}
  \tightlist
  \item
    TC\_LOGIN\_001: Valid login
  \item
    TC\_LOGIN\_002: Invalid username
  \item
    TC\_LOGIN\_003: Invalid password
  \item
    TC\_LOGIN\_004: Empty fields
  \end{itemize}
\end{itemize}

\end{solutionbox}
\begin{mnemonicbox}
``Cases Test Functions, Suites Group Cases'' -
Individual vs collection relationship.

\end{mnemonicbox}
\begin{center}\rule{0.5\linewidth}{0.5pt}\end{center}

\subsection*{Question 5(c OR) [7
marks]}\label{question-5c-or-7-marks}

\textbf{Write a short note on Black Box Testing.}

\begin{solutionbox}

Black Box Testing is a software testing method that examines
functionality without knowledge of internal code structure, focusing on
input-output behavior and requirement compliance.


{\def\LTcaptype{none} % do not increment counter
\vspace{-5pt}
\captionof{table}{Black Box Testing Techniques}
\vspace{-10pt}
\begin{longtable}[]{@{}
  >{\raggedright\arraybackslash}p{(\linewidth - 4\tabcolsep) * \real{0.2973}}
  >{\raggedright\arraybackslash}p{(\linewidth - 4\tabcolsep) * \real{0.3514}}
  >{\raggedright\arraybackslash}p{(\linewidth - 4\tabcolsep) * \real{0.3514}}@{}}
\toprule\noalign{}
\begin{minipage}[b]{\linewidth}\raggedright
Technique
\end{minipage} & \begin{minipage}[b]{\linewidth}\raggedright
Description
\end{minipage} & \begin{minipage}[b]{\linewidth}\raggedright
Application
\end{minipage} \\
\midrule\noalign{}
\endhead
\bottomrule\noalign{}
\endlastfoot
\textbf{Equivalence Partitioning} & Divide inputs into equivalent groups
& Input validation testing \\
\textbf{Boundary Value Analysis} & Test edge values and boundaries &
Range and limit testing \\
\textbf{Decision Table Testing} & Test combinations of conditions &
Complex business logic \\
\textbf{State Transition Testing} & Test state changes & Workflow and
status testing \\
\textbf{Use Case Testing} & Test user scenarios & End-to-end
functionality \\
\end{longtable}
}

\textbf{Black Box Testing Process:}

\begin{center}
\textbf{Mermaid Diagram (Code)}
\begin{verbatim}
{Shaded}
{Highlighting}[]
graph LR
    A[Requirement Analysis] {-{-}{} B[Test Case Design]}
    B {-{-}{} C[Test Data Preparation]}
    C {-{-}{} D[Test Execution]}
    D {-{-}{} E[Result Analysis]}
{Highlighting}
{Shaded}
\end{verbatim}
\end{center}

\textbf{Testing Techniques Explained:}

\textbf{Equivalence Partitioning:}

\begin{itemize}
\tightlist
\item
  Divides input domain into classes of equivalent data
\item
  One test case from each partition represents entire class
\item
  Reduces number of test cases while maintaining coverage
\item
  Example: Age input (0-17: Minor, 18-65: Adult, 65+: Senior)
\end{itemize}

\textbf{Boundary Value Analysis:}

\begin{itemize}
\tightlist
\item
  Tests values at boundaries of equivalence partitions
\item
  Focuses on edge cases where errors commonly occur
\item
  Tests minimum, maximum, and just inside/outside boundaries
\item
  Example: For range 1-100, test: 0, 1, 2, 99, 100, 101
\end{itemize}

\textbf{Decision Table Testing:}

\begin{itemize}
\tightlist
\item
  Represents complex business rules in tabular format
\item
  Shows all possible combinations of conditions and actions
\item
  Ensures complete coverage of business logic scenarios
\item
  Useful for systems with multiple interacting conditions
\end{itemize}

\textbf{State Transition Testing:}

\begin{itemize}
\tightlist
\item
  Models system behavior as states and transitions
\item
  Tests valid and invalid state changes
\item
  Verifies system handles state transitions correctly
\item
  Example: Order states (Pending \rightarrow Processing \rightarrow Shipped \rightarrow Delivered)
\end{itemize}

\textbf{Use Case Testing:}

\begin{itemize}
\tightlist
\item
  Based on user scenarios and use cases
\item
  Tests complete business workflows end-to-end
\item
  Focuses on user perspective and real-world usage
\item
  Validates system meets user requirements
\end{itemize}

\textbf{Black Box Testing Levels:}

\begin{itemize}
\tightlist
\item
  \textbf{Unit Testing}: Individual component functionality
\item
  \textbf{Integration Testing}: Component interaction testing
\item
  \textbf{System Testing}: Complete system functionality
\item
  \textbf{Acceptance Testing}: User requirement validation
\end{itemize}

\textbf{Advantages:}

\begin{itemize}
\tightlist
\item
  \textbf{User Perspective}: Tests from end-user viewpoint
\item
  \textbf{No Code Knowledge}: Testers don't need programming skills
\item
  \textbf{Unbiased Testing}: Not influenced by code implementation
\item
  \textbf{Early Testing}: Can start with requirements specification
\end{itemize}

\textbf{Disadvantages:}

\begin{itemize}
\tightlist
\item
  \textbf{Limited Coverage}: May miss internal logic errors
\item
  \textbf{Inefficient}: Difficult to identify all possible inputs
\item
  \textbf{Redundant Testing}: May duplicate test scenarios
\item
  \textbf{Blind Testing}: Cannot target specific code areas
\end{itemize}

\textbf{Test Data Design:}

\begin{itemize}
\tightlist
\item
  \textbf{Valid Inputs}: Test normal operational conditions
\item
  \textbf{Invalid Inputs}: Test error handling capabilities
\item
  \textbf{Edge Cases}: Test boundary conditions and limits
\item
  \textbf{Stress Inputs}: Test system under extreme conditions
\end{itemize}

\textbf{Black Box vs White Box Comparison:}

\begin{itemize}
\tightlist
\item
  \textbf{Black Box}: External behavior, specification-based
\item
  \textbf{White Box}: Internal structure, code-based
\item
  \textbf{Gray Box}: Combination of both approaches
\item
  \textbf{Complementary}: Both needed for thorough testing
\end{itemize}

\end{solutionbox}
\begin{mnemonicbox}
``Black Box Behavior Based'' - Focus on external
functionality without internal knowledge.

\end{mnemonicbox}

\end{document}
