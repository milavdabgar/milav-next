\documentclass[10pt,a4paper]{article}

% content/resources/templates/preamble.tex
\usepackage[margin=0.6in]{geometry}
\author{Milav Dabgar}
\usepackage{amsmath,amssymb,amsthm}
\usepackage{booktabs}
\usepackage{multirow}
\usepackage{xcolor}
\usepackage{tcolorbox}
\tcbuselibrary{breakable,skins}
\usepackage[colorlinks=true,linkcolor=blue]{hyperref}
\usepackage{titlesec}
\usepackage{enumitem}
\usepackage{tikz}
\usepackage{pgfplots}
\usepackage{circuitikz}
\usepackage[version=4]{mhchem}
\usepackage{longtable}
\usepackage{array}
\usepackage{float}
\usepackage{caption}
\usepackage{listings}

\lstset{
  basicstyle=\small\ttfamily,
  breaklines=true,
  breakatwhitespace=false,
  postbreak=\mbox{\textcolor{red}{$\hookrightarrow$}\space},
  float=false,
  numbers=left,
  numberstyle=\tiny\color{gray},
  numbersep=10pt,
  xleftmargin=2em,
  keywordstyle=\color{blue},
  commentstyle=\color{green!60!black},
  stringstyle=\color{purple},
  backgroundcolor=\color{gray!5},
  showstringspaces=false,
  tabsize=2,
  captionpos=b,
  keepspaces=true,
  columns=flexible
}

\pgfplotsset{compat=1.18}
\usetikzlibrary{shapes,arrows,positioning,calc,patterns,decorations.pathmorphing,decorations.markings,arrows.meta}

% Color scheme
\definecolor{headcolor}{RGB}{0,102,204}
\definecolor{keycolor}{RGB}{220,20,60}
\definecolor{solutioncolor}{RGB}{34,139,34}
\definecolor{mnemoniccolor}{RGB}{148,0,211}
\definecolor{codecolor}{RGB}{0,0,100}

% Spacing
\setlength{\parskip}{3pt}
\setlist[itemize]{nosep}
\setlist[enumerate]{nosep}

% Title formatting
\titleformat{\section}{\Large\bfseries\color{headcolor}}{\thesection}{1em}{}
\titleformat{\subsection}{\large\bfseries\color{headcolor}}{\thesubsection}{1em}{}

% Pandoc tightlist compatibility
\providecommand{\tightlist}{%
  \setlength{\itemsep}{0pt}\setlength{\parskip}{0pt}}

% Pandoc longtable compatibility
\newcounter{none}
\def\thenone{}


% content/resources/templates/gujarati-boxes.tex
\usepackage{fontspec}
\usepackage{polyglossia}

% Set Gujarati as main language (document is primarily in Gujarati)
% Note: gloss-gujarati.ldf doesn't exist in polyglossia, but it will use hyphenation patterns
\setdefaultlanguage{gujarati}
\setotherlanguage{english}

% Configure Gujarati font properly
% Use Language=Default to prevent polyglossia from trying to add language-specific features
% that don't exist for Gujarati, which causes "empty feature" warnings
\newfontfamily\gujaratifont[Script=Gujarati,AutoFakeBold=2.5,AutoFakeSlant=0.3]{Noto Sans Gujarati}
\setmainfont[Script=Gujarati,AutoFakeBold=2.5,AutoFakeSlant=0.3]{Noto Sans Gujarati}
% Use Noto Sans Gujarati for monospace to support Gujarati in text
\setmonofont[Scale=0.9]{Noto Sans Gujarati}

% Configure English to use the same font
\newfontfamily\englishfont[Script=Gujarati,AutoFakeBold=2.5,AutoFakeSlant=0.3]{Noto Sans Gujarati}

% Translations for polyglossia
\gappto\captionsgujarati{
  \renewcommand{\tablename}{કોષ્ટક}
  \renewcommand{\figurename}{આકૃતિ}
}

% Helper for TikZ nodes to ensure Gujarati font
\newcommand{\gu}[1]{{\gujaratifont #1}}

% Custom environments
\newtcolorbox{solutionbox}{
    breakable,
    enhanced,
    colback=solutioncolor!5!white,
    colframe=solutioncolor!75!black,
    fonttitle=\bfseries,
    title=જવાબ
}

\newtcolorbox{solutionboxnobreak}{
 colback=solutioncolor!5!white,
 colframe=solutioncolor!75!black,
 fonttitle=\bfseries,
 title=જવાબ
}

\newtcolorbox{keyformula}{
 breakable,
 enhanced,
 colback=keycolor!5!white,
 colframe=keycolor!75!black,
 fonttitle=\bfseries,
 title=રાસાયણિક સમીકરણ/સૂત્ર
}

\newtcolorbox{mnemonicbox}{
 breakable,
 enhanced,
 colback=mnemoniccolor!5!white,
 colframe=mnemoniccolor!75!black,
 fonttitle=\bfseries,
 title=મેમરી ટ્રીક
}


\begin{document}

\begin{center}
{\Huge\bfseries\color{headcolor} Subject Name (Gujarati)}\\[5pt]
{\LARGE 4331602 -- Summer 2024}\\[3pt]
{\large Semester 1 Study Material}\\[3pt]
{\normalsize\textit{Detailed Solutions and Explanations}}
\end{center}

\vspace{10pt}

\subsection*{પ્રશ્ન 1(અ) [3
ગુણ]}\label{uxaaauxab0uxab6uxaa8-1uxa85-3-uxa97uxaa3}

\textbf{ઓપરેટિંગ સિસ્ટમ વ્યાખ્યાયિત કરો અને તેનું લક્ષ્ય આપો.}

\begin{solutionbox}

\textbf{ઓપરેટિંગ સિસ્ટમ વ્યાખ્યા}: એક પ્રોગ્રામ જે કમ્પ્યુટર હાર્ડવેર અને યુઝર વચ્ચે
ઇન્ટરફેસ તરીકે કામ કરે છે, સિસ્ટમ રિસોર્સિસ મેનેજ કરે છે અને પ્રોગ્રામ એક્ઝિક્યુશન કંટ્રોલ
કરે છે.

\textbf{ઓપરેટિંગ સિસ્ટમના લક્ષ્યો}:

{\def\LTcaptype{none} % do not increment counter
\begin{longtable}[]{@{}ll@{}}
\toprule\noalign{}
લક્ષ્ય & વર્ણન \\
\midrule\noalign{}
\endhead
\bottomrule\noalign{}
\endlastfoot
\textbf{રિસોર્સ મેનેજમેન્ટ} & CPU, મેમરી, I/O ડિવાઇસિસનું કાર્યક્ષમ ફાળવણી \\
\textbf{યુઝર સુવિધા} & વાપરવામાં સરળ ઇન્ટરફેસ પ્રદાન કરવું \\
\textbf{સિસ્ટમ પ્રોટેક્શન} & અનધિકૃત પહોંચથી સિસ્ટમને સુરક્ષિત કરવું \\
\end{longtable}
}

\end{solutionbox}
\begin{mnemonicbox}
``RUS'' - Resource management, User convenience,
System protection

\end{mnemonicbox}
\begin{center}\rule{0.5\linewidth}{0.5pt}\end{center}

\subsection*{પ્રશ્ન 1(બ) [4
ગુણ]}\label{uxaaauxab0uxab6uxaa8-1uxaac-4-uxa97uxaa3}

\textbf{કમ્પ્યુટર સિસ્ટમના ઘટકો નામ આપો અને ઓપરેટિંગ સિસ્ટમની જરૂરિયાત સમજાવો.}

\begin{solutionbox}

\textbf{કમ્પ્યુટર સિસ્ટમ ઘટકો}:

\begin{center}
\textbf{Mermaid Diagram (Code)}
\begin{verbatim}
{Shaded}
{Highlighting}[]
graph TD
    A[કમ્પ્યુટર સિસ્ટમ] {-{-}{} B[હાર્ડવેર]}
    A {-{-}{} C[ઓપરેટિંગ સિસ્ટમ]}
    A {-{-}{} D[એપ્લિકેશન પ્રોગ્રામ્સ]}
    A {-{-}{} E[યુઝર્સ]}
    
    B {-{-}{} F[CPU]}
    B {-{-}{} G[મેમરી]}
    B {-{-}{} H[I/O ડિવાઇસિસ]}
{Highlighting}
{Shaded}
\end{verbatim}
\end{center}

\textbf{ઓપરેટિંગ સિસ્ટમની જરૂરિયાત}:

\begin{itemize}
\tightlist
\item
  \textbf{રિસોર્સ મેનેજર}: હાર્ડવેર ફાળવણી કંટ્રોલ કરે છે
\item
  \textbf{ઇન્ટરફેસ પ્રદાતા}: યુઝર અને હાર્ડવેર વચ્ચે સરળ કમ્યુનિકેશન
\item
  \textbf{સિક્યોરિટી}: સિસ્ટમને ખતરાઓથી બચાવે છે
\item
  \textbf{એરર હેન્ડલિંગ}: સિસ્ટમ એરર્સને કાર્યક્ષમ રીતે મેનેજ કરે છે
\end{itemize}

\end{solutionbox}
\begin{mnemonicbox}
``RISE'' - Resource management, Interface, Security,
Error handling

\end{mnemonicbox}
\begin{center}\rule{0.5\linewidth}{0.5pt}\end{center}

\subsection*{પ્રશ્ન 1(ક) [7
ગુણ]}\label{uxaaauxab0uxab6uxaa8-1uxa95-7-uxa97uxaa3}

\textbf{નીચે ઓપરેટિંગ સિસ્ટમના પ્રકારો સમજાવો.}

\begin{solutionbox}

\textbf{I. Batch Operating System}

{\def\LTcaptype{none} % do not increment counter
\begin{longtable}[]{@{}ll@{}}
\toprule\noalign{}
લક્ષણ & વર્ણન \\
\midrule\noalign{}
\endhead
\bottomrule\noalign{}
\endlastfoot
\textbf{પ્રોસેસિંગ} & યુઝર ઇન્ટરેક્શન વિના બેચમાં જોબ્સ પ્રોસેસ કરે છે \\
\textbf{કાર્યક્ષમતા} & ઊંચું throughput, ઓછું યુઝર ઇન્ટરેક્શન \\
\textbf{ઉદાહરણ} & IBM મેઇનફ્રેમ્સ \\
\end{longtable}
}

\textbf{II. Multiprogramming Operating System}

{\def\LTcaptype{none} % do not increment counter
\begin{longtable}[]{@{}ll@{}}
\toprule\noalign{}
લક્ષણ & વર્ણન \\
\midrule\noalign{}
\endhead
\bottomrule\noalign{}
\endlastfoot
\textbf{કોન્સેપ્ટ} & મેમરીમાં એકસાથે બહુવિધ પ્રોગ્રામ્સ \\
\textbf{CPU ઉપયોગ} & વધુ સારું CPU utilization \\
\textbf{ફાયદો} & idle time ઘટાડે છે \\
\end{longtable}
}

\textbf{III. Time Sharing Operating System}

{\def\LTcaptype{none} % do not increment counter
\begin{longtable}[]{@{}ll@{}}
\toprule\noalign{}
લક્ષણ & વર્ણન \\
\midrule\noalign{}
\endhead
\bottomrule\noalign{}
\endlastfoot
\textbf{Time Slices} & યુઝર્સ વચ્ચે CPU time વહેંચાયેલું \\
\textbf{Response} & ઝડપી response time \\
\textbf{ઉદાહરણ} & Unix, Linux \\
\end{longtable}
}

\end{solutionbox}
\begin{mnemonicbox}
``BMT'' - Batch (કોઈ ઇન્ટરેક્શન નહીં), Multiprogramming
(ઘણા પ્રોગ્રામ્સ), Time-sharing (time slices)

\end{mnemonicbox}
\begin{center}\rule{0.5\linewidth}{0.5pt}\end{center}

\subsection*{પ્રશ્ન 1(ક) OR [7
ગુણ]}\label{uxaaauxab0uxab6uxaa8-1uxa95-or-7-uxa97uxaa3}

\textbf{Linux આર્કિટેક્ચર અને લક્ષણો તેના ઘટકો સાથે સમજાવો.}

\begin{solutionbox}

\textbf{Linux આર્કિટેક્ચર}:

\begin{center}
\textbf{Mermaid Diagram (Code)}
\begin{verbatim}
{Shaded}
{Highlighting}[]
graph TD
    A[યુઝર એપ્લિકેશન્સ] {-{-}{} B[સિસ્ટમ લાઇબ્રેરીઝ]}
    B {-{-}{} C[સિસ્ટમ કોલ ઇન્ટરફેસ]}
    C {-{-}{} D[Linux કર્નલ]}
    D {-{-}{} E[હાર્ડવેર]}
    
    D {-{-}{} F[પ્રોસેસ મેનેજમેન્ટ]}
    D {-{-}{} G[મેમરી મેનેજમેન્ટ]}
    D {-{-}{} H[ફાઇલ સિસ્ટમ]}
    D {-{-}{} I[ડિવાઇસ ડ્રાઇવર્સ]}
{Highlighting}
{Shaded}
\end{verbatim}
\end{center}

\textbf{Linux લક્ષણો}:

{\def\LTcaptype{none} % do not increment counter
\begin{longtable}[]{@{}ll@{}}
\toprule\noalign{}
લક્ષણ & વર્ણન \\
\midrule\noalign{}
\endhead
\bottomrule\noalign{}
\endlastfoot
\textbf{ઓપન સોર્સ} & મફત અને સુધારી શકાય તેવું \\
\textbf{મલ્ટિયુઝર} & એકસાથે બહુવિધ યુઝર્સ \\
\textbf{મલ્ટિટાસ્કિંગ} & એકસાથે બહુવિધ પ્રોસેસિસ \\
\textbf{પોર્ટેબલ} & વિવિધ હાર્ડવેર પર ચાલે છે \\
\end{longtable}
}

\textbf{ઘટકો}:

\begin{itemize}
\tightlist
\item
  \textbf{કર્નલ}: ઓપરેટિંગ સિસ્ટમનો મુખ્ય ભાગ
\item
  \textbf{શેલ}: કમાન્ડ interpreter
\item
  \textbf{ફાઇલ સિસ્ટમ}: ડેટા સ્ટોરેજ ઓર્ગેનાઇઝ કરે છે
\end{itemize}

\end{solutionbox}
\begin{mnemonicbox}
``COMP'' - Core (કર્નલ), Open source, Multiuser,
Portable

\end{mnemonicbox}
\begin{center}\rule{0.5\linewidth}{0.5pt}\end{center}

\subsection*{પ્રશ્ન 2(અ) [3
ગુણ]}\label{uxaaauxab0uxab6uxaa8-2uxa85-3-uxa97uxaa3}

\textbf{પ્રક્રિયા નિયંત્રણ બ્લોકનું વર્ણન કરો. અને વ્યાખ્યાયિત કરો (1) PID (2)
stack pointer (3) program counter}

\begin{solutionbox}

\textbf{Process Control Block (PCB)}: OS મેનેજમેન્ટ માટે પ્રોસેસ ઇન્ફર્મેશન ધરાવતું
ડેટા સ્ટ્રક્ચર.

\textbf{વ્યાખ્યાઓ}:

{\def\LTcaptype{none} % do not increment counter
\begin{longtable}[]{@{}ll@{}}
\toprule\noalign{}
શબ્દ & વ્યાખ્યા \\
\midrule\noalign{}
\endhead
\bottomrule\noalign{}
\endlastfoot
\textbf{PID} & Process Identifier - દરેક પ્રોસેસ માટે અનન્ય નંબર \\
\textbf{Stack Pointer} & પ્રોસેસ સ્ટેકની ટોપ તરફ પોઇન્ટ કરે છે \\
\textbf{Program Counter} & આગલી instruction નું address ધરાવે છે \\
\end{longtable}
}

\end{solutionbox}
\begin{mnemonicbox}
``PSP'' - PID (identifier), Stack pointer (ટોપ),
Program counter (આગલું)

\end{mnemonicbox}
\begin{center}\rule{0.5\linewidth}{0.5pt}\end{center}

\subsection*{પ્રશ્ન 2(બ) [4
ગુણ]}\label{uxaaauxab0uxab6uxaa8-2uxaac-4-uxa97uxaa3}

\textbf{પ્રક્રિયા મોડલ અને પ્રક્રિયા સ્થિતિઓનું વર્ણન કરો}

\begin{solutionbox}

\textbf{પ્રોસેસ મોડલ}: OS દ્વારા પ્રોસેસિસ કેવી રીતે મેનેજ થાય છે તેનું કોન્સેપ્ચ્યુઅલ
રિપ્રેઝેન્ટેશન.

\textbf{પ્રોસેસ સ્થિતિઓ}:

\begin{verbatim}
stateDiagram{-v2}
  direction LR
    [*] {-{-} New}
    New {-{-} Ready}
    Ready {-{-} Running}
    Running {-{-} Waiting}
    Running {-{-} Ready}
    Waiting {-{-} Ready}
    Running {-{-} Terminated}
    Terminated {-{-} [*]}
\end{verbatim}

{\def\LTcaptype{none} % do not increment counter
\begin{longtable}[]{@{}ll@{}}
\toprule\noalign{}
સ્થિતિ & વર્ણન \\
\midrule\noalign{}
\endhead
\bottomrule\noalign{}
\endlastfoot
\textbf{New} & પ્રોસેસ બનાવાઈ રહ્યું છે \\
\textbf{Ready} & CPU માટે રાહ જોઈ રહ્યું છે \\
\textbf{Running} & instructions એક્ઝિક્યુટ કરી રહ્યું છે \\
\textbf{Waiting} & I/O માટે રાહ જોઈ રહ્યું છે \\
\textbf{Terminated} & પ્રોસેસ સમાપ્ત થયું \\
\end{longtable}
}

\end{solutionbox}
\begin{mnemonicbox}
``NRRWT'' - New, Ready, Running, Waiting, Terminated

\end{mnemonicbox}
\begin{center}\rule{0.5\linewidth}{0.5pt}\end{center}

\subsection*{પ્રશ્ન 2(ક) [7
ગુણ]}\label{uxaaauxab0uxab6uxaa8-2uxa95-7-uxa97uxaa3}

\textbf{શેડ્યુલિંગ અલ્ગોરિધમનું વર્ણન કરો:(I) First Come First Serve,(II)
Shortest Job First}

\begin{solutionbox}

\textbf{I. First Come First Serve (FCFS)}

{\def\LTcaptype{none} % do not increment counter
\begin{longtable}[]{@{}
  >{\raggedright\arraybackslash}p{(\linewidth - 8\tabcolsep) * \real{0.1304}}
  >{\raggedright\arraybackslash}p{(\linewidth - 8\tabcolsep) * \real{0.2029}}
  >{\raggedright\arraybackslash}p{(\linewidth - 8\tabcolsep) * \real{0.1739}}
  >{\raggedright\arraybackslash}p{(\linewidth - 8\tabcolsep) * \real{0.2464}}
  >{\raggedright\arraybackslash}p{(\linewidth - 8\tabcolsep) * \real{0.2464}}@{}}
\toprule\noalign{}
\begin{minipage}[b]{\linewidth}\raggedright
પ્રોસેસ
\end{minipage} & \begin{minipage}[b]{\linewidth}\raggedright
આગમન સમય
\end{minipage} & \begin{minipage}[b]{\linewidth}\raggedright
Burst Time
\end{minipage} & \begin{minipage}[b]{\linewidth}\raggedright
પૂર્ણતા સમય
\end{minipage} & \begin{minipage}[b]{\linewidth}\raggedright
Turnaround Time
\end{minipage} \\
\midrule\noalign{}
\endhead
\bottomrule\noalign{}
\endlastfoot
P1 & 0 & 4 & 4 & 4 \\
P2 & 1 & 3 & 7 & 6 \\
P3 & 2 & 2 & 9 & 7 \\
\end{longtable}
}

\textbf{સરેરાશ Turnaround Time} = (4+6+7)/3 = 5.67

\textbf{II. Shortest Job First (SJF)}

{\def\LTcaptype{none} % do not increment counter
\begin{longtable}[]{@{}
  >{\raggedright\arraybackslash}p{(\linewidth - 8\tabcolsep) * \real{0.1304}}
  >{\raggedright\arraybackslash}p{(\linewidth - 8\tabcolsep) * \real{0.2029}}
  >{\raggedright\arraybackslash}p{(\linewidth - 8\tabcolsep) * \real{0.1739}}
  >{\raggedright\arraybackslash}p{(\linewidth - 8\tabcolsep) * \real{0.2464}}
  >{\raggedright\arraybackslash}p{(\linewidth - 8\tabcolsep) * \real{0.2464}}@{}}
\toprule\noalign{}
\begin{minipage}[b]{\linewidth}\raggedright
પ્રોસેસ
\end{minipage} & \begin{minipage}[b]{\linewidth}\raggedright
આગમન સમય
\end{minipage} & \begin{minipage}[b]{\linewidth}\raggedright
Burst Time
\end{minipage} & \begin{minipage}[b]{\linewidth}\raggedright
પૂર્ણતા સમય
\end{minipage} & \begin{minipage}[b]{\linewidth}\raggedright
Turnaround Time
\end{minipage} \\
\midrule\noalign{}
\endhead
\bottomrule\noalign{}
\endlastfoot
P3 & 2 & 2 & 4 & 2 \\
P2 & 1 & 3 & 7 & 6 \\
P1 & 0 & 4 & 11 & 11 \\
\end{longtable}
}

\textbf{સરેરાશ Turnaround Time} = (2+6+11)/3 = 6.33

\end{solutionbox}
\begin{mnemonicbox}
``FS'' - FCFS (પહેલા ક્રમ), SJF (સૌથી ટૂંકું પહેલા)

\end{mnemonicbox}
\begin{center}\rule{0.5\linewidth}{0.5pt}\end{center}

\subsection*{પ્રશ્ન 2(અ) OR [3
ગુણ]}\label{uxaaauxab0uxab6uxaa8-2uxa85-or-3-uxa97uxaa3}

\textbf{વ્યાખ્યાયિત કરો Race condition, Mutual Exclusion}

\begin{solutionbox}

{\def\LTcaptype{none} % do not increment counter
\begin{longtable}[]{@{}
  >{\raggedright\arraybackslash}p{(\linewidth - 2\tabcolsep) * \real{0.3333}}
  >{\raggedright\arraybackslash}p{(\linewidth - 2\tabcolsep) * \real{0.6667}}@{}}
\toprule\noalign{}
\begin{minipage}[b]{\linewidth}\raggedright
શબ્દ
\end{minipage} & \begin{minipage}[b]{\linewidth}\raggedright
વ્યાખ્યા
\end{minipage} \\
\midrule\noalign{}
\endhead
\bottomrule\noalign{}
\endlastfoot
\textbf{Race Condition} & બહુવિધ પ્રોસેસિસ એકસાથે shared data એક્સેસ કરે છે જેથી
inconsistent પરિણામો આવે છે \\
\textbf{Mutual Exclusion} & એક સમયે માત્ર એક પ્રોસેસ critical section એક્સેસ
કરી શકે છે \\
\end{longtable}
}

\textbf{ઉદાહરણ}: બે પ્રોસેસિસ એકજ બેંક એકાઉન્ટ બેલેન્સ અપડેટ કરી રહ્યા છે.

\end{solutionbox}
\begin{mnemonicbox}
``RM'' - Race (એકસાથે એક્સેસ), Mutual (એક સમયે એક)

\end{mnemonicbox}
\begin{center}\rule{0.5\linewidth}{0.5pt}\end{center}

\subsection*{પ્રશ્ન 2(બ) OR [4
ગુણ]}\label{uxaaauxab0uxab6uxaa8-2uxaac-or-4-uxa97uxaa3}

\textbf{વ્યાખ્યાયિત કરો Throughput, Turnaround Time, Waiting Time,
Response Time}

\begin{solutionbox}

{\def\LTcaptype{none} % do not increment counter
\begin{longtable}[]{@{}ll@{}}
\toprule\noalign{}
શબ્દ & વ્યાખ્યા \\
\midrule\noalign{}
\endhead
\bottomrule\noalign{}
\endlastfoot
\textbf{Throughput} & એકમ સમયમાં પૂર્ણ થયેલ પ્રોસેસિસની સંખ્યા \\
\textbf{Turnaround Time} & submission થી completion સુધીનો કુલ સમય \\
\textbf{Waiting Time} & ready queue માં રાહ જોવાનો સમય \\
\textbf{Response Time} & submission થી પહેલા response સુધીનો સમય \\
\end{longtable}
}

\textbf{ફોર્મ્યુલા ટેબલ}:

{\def\LTcaptype{none} % do not increment counter
\begin{longtable}[]{@{}ll@{}}
\toprule\noalign{}
મેટ્રિક & ફોર્મ્યુલા \\
\midrule\noalign{}
\endhead
\bottomrule\noalign{}
\endlastfoot
Turnaround Time & Completion Time - Arrival Time \\
Waiting Time & Turnaround Time - Burst Time \\
Response Time & First CPU Time - Arrival Time \\
\end{longtable}
}

\end{solutionbox}
\begin{mnemonicbox}
``TTWR'' - Throughput, Turnaround, Waiting, Response

\end{mnemonicbox}
\begin{center}\rule{0.5\linewidth}{0.5pt}\end{center}

\subsection*{પ્રશ્ન 2(ક) OR [7
ગુણ]}\label{uxaaauxab0uxab6uxaa8-2uxa95-or-7-uxa97uxaa3}

\textbf{રાઉન્ડ રોબિન અલ્ગોરિધમ ઉદાહરણ સાથે સમજાવો.}

\begin{solutionbox}

\textbf{રાઉન્ડ રોબિન}: દરેક પ્રોસેસને સમાન CPU time slice (quantum) મળે છે.

\textbf{ઉદાહરણ} (Time Quantum = 2):

{\def\LTcaptype{none} % do not increment counter
\begin{longtable}[]{@{}ll@{}}
\toprule\noalign{}
પ્રોસેસ & Burst Time \\
\midrule\noalign{}
\endhead
\bottomrule\noalign{}
\endlastfoot
P1 & 5 \\
P2 & 3 \\
P3 & 4 \\
\end{longtable}
}

\textbf{એક્ઝિક્યુશન ટાઇમલાઇન}:

\begin{verbatim}
0{-{-}{-}{-}2{-}{-}{-}{-}4{-}{-}{-}{-}6{-}{-}{-}{-}8{-}{-}{-}{-}10{-}{-}{-}12}
 P1   P2   P3   P1   P3   P1
\end{verbatim}

{\def\LTcaptype{none} % do not increment counter
\begin{longtable}[]{@{}lll@{}}
\toprule\noalign{}
પ્રોસેસ & પૂર્ણતા સમય & Turnaround Time \\
\midrule\noalign{}
\endhead
\bottomrule\noalign{}
\endlastfoot
P1 & 12 & 12 \\
P2 & 6 & 6 \\
P3 & 10 & 10 \\
\end{longtable}
}

\textbf{સરેરાશ Turnaround Time} = (12+6+10)/3 = 9.33

\textbf{ફાયદાઓ}:

\begin{itemize}
\tightlist
\item
  \textbf{ન્યાયસંગત}: બધા પ્રોસેસિસને સમાન સમય
\item
  \textbf{રિસ્પોન્સિવ}: ઇન્ટરેક્ટિવ સિસ્ટમ્સ માટે સારું
\end{itemize}

\end{solutionbox}
\begin{mnemonicbox}
``RR-FE'' - Round Robin આપે છે Fair અને Equal સમય

\end{mnemonicbox}
\begin{center}\rule{0.5\linewidth}{0.5pt}\end{center}

\subsection*{પ્રશ્ન 3(અ) [3
ગુણ]}\label{uxaaauxab0uxab6uxaa8-3uxa85-3-uxa97uxaa3}

\textbf{ફાઇલ એક્સેસ પદ્ધતિઓનો પ્રકાર આપો}

\begin{solutionbox}

{\def\LTcaptype{none} % do not increment counter
\begin{longtable}[]{@{}ll@{}}
\toprule\noalign{}
એક્સેસ પદ્ધતિ & વર્ણન \\
\midrule\noalign{}
\endhead
\bottomrule\noalign{}
\endlastfoot
\textbf{Sequential} & શરૂઆતથી ક્રમમાં read/write \\
\textbf{Direct} & કોઈ પણ record ને સીધું એક્સેસ \\
\textbf{Indexed} & records શોધવા માટે index ઉપયોગ \\
\end{longtable}
}

\end{solutionbox}
\begin{mnemonicbox}
``SDI'' - Sequential (ક્રમ), Direct (કોઈ પણ), Indexed
(index)

\end{mnemonicbox}
\begin{center}\rule{0.5\linewidth}{0.5pt}\end{center}

\subsection*{પ્રશ્ન 3(બ) [4
ગુણ]}\label{uxaaauxab0uxab6uxaa8-3uxaac-4-uxa97uxaa3}

\textbf{ડેડલોક લાક્ષણિકતાઓ આપો અને Deadlock Prevention, Deadlock Avoidance
વર્ણન કરો}

\begin{solutionbox}

\textbf{ડેડલોક લાક્ષણિકતાઓ}:

{\def\LTcaptype{none} % do not increment counter
\begin{longtable}[]{@{}ll@{}}
\toprule\noalign{}
શરત & વર્ણન \\
\midrule\noalign{}
\endhead
\bottomrule\noalign{}
\endlastfoot
\textbf{Mutual Exclusion} & રિસોર્સિસ શેર કરી શકાતા નથી \\
\textbf{Hold and Wait} & પ્રોસેસ રિસોર્સ પકડીને રાહ જુએ છે \\
\textbf{No Preemption} & રિસોર્સિસ બળજબરીથી લઈ શકાતા નથી \\
\textbf{Circular Wait} & રાહ જોતા પ્રોસેસિસનો ગોળાકાર chain \\
\end{longtable}
}

\textbf{Deadlock Prevention}: ચાર શરતોમાંથી કોઈ એક દૂર કરો.

\textbf{Deadlock Avoidance}: unsafe states ટાળવા માટે Banker's algorithm
જેવા અલ્ગોરિધમ ઉપયોગ કરો.

\end{solutionbox}
\begin{mnemonicbox}
``MHNC'' - Mutual exclusion, Hold and wait, No
preemption, Circular wait

\end{mnemonicbox}
\begin{center}\rule{0.5\linewidth}{0.5pt}\end{center}

\subsection*{પ્રશ્ન 3(ક) [7
ગુણ]}\label{uxaaauxab0uxab6uxaa8-3uxa95-7-uxa97uxaa3}

\textbf{ફાઈલ ફાળવણી પદ્ધતિઓ લગતી, લિંક્ડ, અનુક્રમિત સમજાવો}

\begin{solutionbox}

\textbf{ફાઈલ ફાળવણી પદ્ધતિઓ}:

{\def\LTcaptype{none} % do not increment counter
\begin{longtable}[]{@{}
  >{\raggedright\arraybackslash}p{(\linewidth - 6\tabcolsep) * \real{0.1667}}
  >{\raggedright\arraybackslash}p{(\linewidth - 6\tabcolsep) * \real{0.2708}}
  >{\raggedright\arraybackslash}p{(\linewidth - 6\tabcolsep) * \real{0.2500}}
  >{\raggedright\arraybackslash}p{(\linewidth - 6\tabcolsep) * \real{0.3125}}@{}}
\toprule\noalign{}
\begin{minipage}[b]{\linewidth}\raggedright
પદ્ધતિ
\end{minipage} & \begin{minipage}[b]{\linewidth}\raggedright
વર્ણન
\end{minipage} & \begin{minipage}[b]{\linewidth}\raggedright
ફાયદાઓ
\end{minipage} & \begin{minipage}[b]{\linewidth}\raggedright
નુકસાન
\end{minipage} \\
\midrule\noalign{}
\endhead
\bottomrule\noalign{}
\endlastfoot
\textbf{Contiguous} & Sequential blocks & ઝડપી એક્સેસ & External
fragmentation \\
\textbf{Linked} & પોઇન્ટર્સ સાથે વિખરાયેલા blocks & કોઈ fragmentation નહીં &
ધીમું random access \\
\textbf{Indexed} & Index block માં addresses & ઝડપી random access &
વધારાનું overhead \\
\end{longtable}
}

\textbf{Contiguous Allocation}:

\begin{verbatim}
File A: [1][2][3][4][5]
\end{verbatim}

\textbf{Linked Allocation}:

\begin{verbatim}
File A: [1][7][3][9]
\end{verbatim}

\textbf{Indexed Allocation}:

\begin{verbatim}
Index Block: [1,3,7,9,12]
File blocks: [1][3][7][9][12]
\end{verbatim}

\end{solutionbox}
\begin{mnemonicbox}
``CLI'' - Contiguous (એકસાથે), Linked (પોઇન્ટર્સ),
Indexed (index block)

\end{mnemonicbox}
\begin{center}\rule{0.5\linewidth}{0.5pt}\end{center}

\subsection*{પ્રશ્ન 3(અ) OR [3
ગુણ]}\label{uxaaauxab0uxab6uxaa8-3uxa85-or-3-uxa97uxaa3}

\textbf{Linux ફાઈલ સિસ્ટમ સ્ટ્રક્ચરની સમજણ આપો.}

\begin{solutionbox}

\textbf{Linux ફાઈલ સિસ્ટમ હાયરાર્કી}:

\begin{verbatim}
/
├── bin/     (સિસ્ટમ binaries)
├── etc/     (કન્ફિગરેશન ફાઈલો)
├── home/    (યુઝર ડિરેક્ટરીઝ)
├── var/     (Variable ડેટા)
├── usr/     (યુઝર પ્રોગ્રામ્સ)
└── tmp/     (Temporary ફાઈલો)
\end{verbatim}

{\def\LTcaptype{none} % do not increment counter
\begin{longtable}[]{@{}ll@{}}
\toprule\noalign{}
ડિરેક્ટરી & હેતુ \\
\midrule\noalign{}
\endhead
\bottomrule\noalign{}
\endlastfoot
\textbf{/bin} & આવશ્યક સિસ્ટમ binaries \\
\textbf{/etc} & સિસ્ટમ કન્ફિગરેશન ફાઈલો \\
\textbf{/home} & યુઝર home ડિરેક્ટરીઝ \\
\end{longtable}
}

\end{solutionbox}
\begin{mnemonicbox}
``BEH'' - Bin (binaries), Etc (config), Home (યુઝર્સ)

\end{mnemonicbox}
\begin{center}\rule{0.5\linewidth}{0.5pt}\end{center}

\subsection*{પ્રશ્ન 3(બ) OR [4
ગુણ]}\label{uxaaauxab0uxab6uxaa8-3uxaac-or-4-uxa97uxaa3}

\textbf{ઉદાહરણ સાથે Critical Section and Semaphore સમજાવો.}

\begin{solutionbox}

\textbf{Critical Section}: shared resources એક્સેસ કરતો કોડ segment.

\textbf{Semaphore}: counter variable ઉપયોગ કરતું synchronization tool.

\textbf{ઉદાહરણ}:

\begin{verbatim}
\# Binary Semaphore
wait(S):
  while S {}= 0 do nothing
  S = S {-} 1

signal(S):
  S = S + 1
\end{verbatim}

\textbf{Critical Section સ્ટ્રક્ચર}:

{\def\LTcaptype{none} % do not increment counter
\begin{longtable}[]{@{}ll@{}}
\toprule\noalign{}
Section & વર્ણન \\
\midrule\noalign{}
\endhead
\bottomrule\noalign{}
\endlastfoot
\textbf{Entry} & પરવાનગી માંગવી \\
\textbf{Critical} & Shared resource એક્સેસ કરવું \\
\textbf{Exit} & પરવાનગી છોડવી \\
\textbf{Remainder} & બીજો કોડ \\
\end{longtable}
}

\end{solutionbox}
\begin{mnemonicbox}
``ECER'' - Entry, Critical, Exit, Remainder

\end{mnemonicbox}
\begin{center}\rule{0.5\linewidth}{0.5pt}\end{center}

\subsection*{પ્રશ્ન 3(ક) OR [7
ગુણ]}\label{uxaaauxab0uxab6uxaa8-3uxa95-or-7-uxa97uxaa3}

\textbf{ડેડલોક ટાળો, ડેડલોક શોધ,અને પ્રોસેસ પુનઃપ્રાપ્તિ વ્યાખ્યાયિત કરો અને
સમજાવો}

\begin{solutionbox}

\textbf{Deadlock Avoidance}:

\begin{itemize}
\tightlist
\item
  \textbf{Banker's Algorithm} ઉપયોગ કરો
\item
  resource allocation safe state તરફ લઈ જાય છે કે નહીં તે ચેક કરો
\end{itemize}

\textbf{Deadlock Detection}:

\begin{itemize}
\tightlist
\item
  \textbf{Wait-for Graph} ઉપયોગ કરીને નિયમિત deadlock ચેક કરો
\end{itemize}

\textbf{Deadlock Recovery પદ્ધતિઓ}:

{\def\LTcaptype{none} % do not increment counter
\begin{longtable}[]{@{}ll@{}}
\toprule\noalign{}
પદ્ધતિ & વર્ણન \\
\midrule\noalign{}
\endhead
\bottomrule\noalign{}
\endlastfoot
\textbf{Process Termination} & Deadlocked પ્રોસેસિસને kill કરો \\
\textbf{Resource Preemption} & પ્રોસેસિસમાંથી resources લો \\
\textbf{Rollback} & અગાઉની safe state પર પાછા જાઓ \\
\end{longtable}
}

\textbf{Banker's Algorithm સ્ટેપ્સ}:

\begin{enumerate}
\tightlist
\item
  request \leq available resources છે કે નહીં ચેક કરો
\item
  allocation simulate કરો
\item
  safe state અસ્તિત્વ ચેક કરો
\end{enumerate}

\textbf{Wait-for Graph}:

\begin{center}
\textbf{Mermaid Diagram (Code)}
\begin{verbatim}
{Shaded}
{Highlighting}[]
graph LR
    P1 {-{-}{} P2}
    P2 {-{-}{} P3}
    P3 {-{-}{} P1}
{Highlighting}
{Shaded}
\end{verbatim}
\end{center}

\end{solutionbox}
\begin{mnemonicbox}
``ADR-BWT'' - Avoidance (Banker's), Detection
(Wait-for), Recovery (Terminate)

\end{mnemonicbox}
\begin{center}\rule{0.5\linewidth}{0.5pt}\end{center}

\subsection*{પ્રશ્ન 4(અ) [3
ગુણ]}\label{uxaaauxab0uxab6uxaa8-4uxa85-3-uxa97uxaa3}

\textbf{શા માટે ફાઈલ પ્રોટેક્શનની જરૂર છે સમજાવો?}

\begin{solutionbox}

\textbf{ફાઈલ પ્રોટેક્શનની જરૂરિયાત}:

{\def\LTcaptype{none} % do not increment counter
\begin{longtable}[]{@{}ll@{}}
\toprule\noalign{}
કારણ & વર્ણન \\
\midrule\noalign{}
\endhead
\bottomrule\noalign{}
\endlastfoot
\textbf{ગોપનીયતા} & વ્યક્તિગત ડેટાનું રક્ષણ \\
\textbf{સિક્યોરિટી} & અનધિકૃત એક્સેસ અટકાવવું \\
\textbf{અખંડતા} & ડેટા consistency જાળવવી \\
\end{longtable}
}

\textbf{પ્રોટેક્શન મેકેનિઝમ્સ}:

\begin{itemize}
\tightlist
\item
  \textbf{Access Control Lists (ACL)}
\item
  \textbf{ફાઈલ Permissions} (Read, Write, Execute)
\item
  \textbf{યુઝર Authentication}
\end{itemize}

\end{solutionbox}
\begin{mnemonicbox}
``PSI'' - Privacy, Security, Integrity

\end{mnemonicbox}
\begin{center}\rule{0.5\linewidth}{0.5pt}\end{center}

\subsection*{પ્રશ્ન 4(બ) [4
ગુણ]}\label{uxaaauxab0uxab6uxaa8-4uxaac-4-uxa97uxaa3}

\textbf{Program threats, System threats નું વર્ણન કરો}

\begin{solutionbox}

\textbf{Program Threats}:

{\def\LTcaptype{none} % do not increment counter
\begin{longtable}[]{@{}ll@{}}
\toprule\noalign{}
ખતરો & વર્ણન \\
\midrule\noalign{}
\endhead
\bottomrule\noalign{}
\endlastfoot
\textbf{Virus} & Self-replicating દુર્ભાવનાપૂર્ણ કોડ \\
\textbf{Worm} & નેટવર્ક પર ફેલાતા malware \\
\textbf{Trojan Horse} & છૂપાયેલ દુર્ભાવનાપૂર્ણ પ્રોગ્રામ \\
\end{longtable}
}

\textbf{System Threats}:

{\def\LTcaptype{none} % do not increment counter
\begin{longtable}[]{@{}ll@{}}
\toprule\noalign{}
ખતરો & વર્ણન \\
\midrule\noalign{}
\endhead
\bottomrule\noalign{}
\endlastfoot
\textbf{Denial of Service} & સિસ્ટમ resources ભરાવી દેવા \\
\textbf{Port Scanning} & vulnerable services શોધવી \\
\textbf{Man-in-Middle} & communications intercept કરવા \\
\end{longtable}
}

\textbf{પ્રોટેક્શન પદ્ધતિઓ}:

\begin{itemize}
\tightlist
\item
  \textbf{એન્ટિવાયરસ સોફ્ટવેર}
\item
  \textbf{ફાયરવોલ્સ}
\item
  \textbf{નિયમિત અપડેટ્સ}
\end{itemize}

\end{solutionbox}
\begin{mnemonicbox}
``VWT-DPM'' - Virus, Worm, Trojan; DoS, Port scan,
Man-in-middle

\end{mnemonicbox}
\begin{center}\rule{0.5\linewidth}{0.5pt}\end{center}

\subsection*{પ્રશ્ન 4(ક) [7
ગુણ]}\label{uxaaauxab0uxab6uxaa8-4uxa95-7-uxa97uxaa3}

\textbf{સંક્ષિપ્તમાં ઓપરેટિંગ સિસ્ટમ સુરક્ષા નીતિઓ અને પ્રક્રિયાઓની વિગતો આપો}

\begin{solutionbox}

\textbf{સિક્યોરિટી નીતિઓ}:

{\def\LTcaptype{none} % do not increment counter
\begin{longtable}[]{@{}ll@{}}
\toprule\noalign{}
નીતિ પ્રકાર & વર્ણન \\
\midrule\noalign{}
\endhead
\bottomrule\noalign{}
\endlastfoot
\textbf{Access Control} & કોણ કયા resources એક્સેસ કરી શકે \\
\textbf{Authentication} & યુઝર identity verify કરવી \\
\textbf{Authorization} & યુઝર permissions નક્કી કરવી \\
\textbf{Audit} & પ્રવૃત્તિઓ monitor અને log કરવી \\
\end{longtable}
}

\textbf{સિક્યોરિટી પ્રક્રિયાઓ}:

\begin{verbatim}
flowchart LR
    A[યુઝર લોગિન] {-{-} B[Authentication]}
    B {-{-} C[Authorization ચેક]}
    C {-{-} D[Resource એક્સેસ]}
    D {-{-} E[Activity Logging]}
    E {-{-} F[Audit Review]}
\end{verbatim}

\textbf{અમલીકરણ સ્ટેપ્સ}:

\begin{enumerate}
\tightlist
\item
  \textbf{યુઝર Registration} અને credential સેટઅપ
\item
  \textbf{Multi-factor Authentication}
\item
  \textbf{Role-based Access Control}
\item
  \textbf{નિયમિત Security Audits}
\end{enumerate}

\textbf{સામાન્ય સિક્યોરિટી પગલાં}:

\begin{itemize}
\tightlist
\item
  \textbf{Password નીતિઓ}
\item
  \textbf{Encryption}
\item
  \textbf{Backup પ્રક્રિયાઓ}
\item
  \textbf{Incident Response યોજનાઓ}
\end{itemize}

\end{solutionbox}
\begin{mnemonicbox}
``AAAA'' - Access control, Authentication,
Authorization, Audit

\end{mnemonicbox}
\begin{center}\rule{0.5\linewidth}{0.5pt}\end{center}

\subsection*{પ્રશ્ન 4(અ) OR [3
ગુણ]}\label{uxaaauxab0uxab6uxaa8-4uxa85-or-3-uxa97uxaa3}

\textbf{Authentication and Authorization સમજણ આપો}

\begin{solutionbox}

{\def\LTcaptype{none} % do not increment counter
\begin{longtable}[]{@{}lll@{}}
\toprule\noalign{}
શબ્દ & વ્યાખ્યા & ઉદાહરણ \\
\midrule\noalign{}
\endhead
\bottomrule\noalign{}
\endlastfoot
\textbf{Authentication} & યુઝર identity verify કરવી &
Username/password \\
\textbf{Authorization} & એક્સેસ અધિકારો નક્કી કરવા & ફાઈલ permissions \\
\end{longtable}
}

\textbf{Authentication પદ્ધતિઓ}:

\begin{itemize}
\tightlist
\item
  \textbf{Password-based}
\item
  \textbf{Biometric}
\item
  \textbf{Token-based}
\end{itemize}

\end{solutionbox}
\begin{mnemonicbox}
``AA'' - Authentication (તમે કોણ છો), Authorization
(તમે શું કરી શકો છો)

\end{mnemonicbox}
\begin{center}\rule{0.5\linewidth}{0.5pt}\end{center}

\subsection*{પ્રશ્ન 4(બ) OR [4
ગુણ]}\label{uxaaauxab0uxab6uxaa8-4uxaac-or-4-uxa97uxaa3}

\textbf{ઓપરેટિંગ સિસ્ટમ સુરક્ષા નીતિઓ અને પ્રક્રિયાઓ સમજાવો}

\begin{solutionbox}

\textbf{સિક્યોરિટી નીતિઓ ફ્રેમવર્ક}:

{\def\LTcaptype{none} % do not increment counter
\begin{longtable}[]{@{}ll@{}}
\toprule\noalign{}
ઘટક & હેતુ \\
\midrule\noalign{}
\endhead
\bottomrule\noalign{}
\endlastfoot
\textbf{યુઝર મેનેજમેન્ટ} & યુઝર એકાઉન્ટ્સ કંટ્રોલ કરવા \\
\textbf{ડેટા પ્રોટેક્શન} & સંવેદનશીલ માહિતી સુરક્ષિત કરવી \\
\textbf{નેટવર્ક સિક્યોરિટી} & કમ્યુનિકેશન્સ સુરક્ષિત કરવા \\
\textbf{સિસ્ટમ મોનિટરિંગ} & ખતરાઓ શોધવા \\
\end{longtable}
}

\textbf{અમલીકરણ પ્રક્રિયાઓ}:

\begin{enumerate}
\tightlist
\item
  \textbf{જોખમ મૂલ્યાંકન}
\item
  \textbf{નીતિ વિકાસ}
\item
  \textbf{તાલીમ કાર્યક્રમો}
\item
  \textbf{નિયમિત સમીક્ષાઓ}
\end{enumerate}

\end{solutionbox}
\begin{mnemonicbox}
``UDNS'' - User management, Data protection, Network
security, System monitoring

\end{mnemonicbox}
\begin{center}\rule{0.5\linewidth}{0.5pt}\end{center}

\subsection*{પ્રશ્ન 4(ક) OR [7
ગુણ]}\label{uxaaauxab0uxab6uxaa8-4uxa95-or-7-uxa97uxaa3}

\textbf{ઑપરેટિંગ સિસ્ટમમાં સુરક્ષા પગલાંની વિગતો આપો.}

\begin{solutionbox}

\textbf{વ્યાપક સિક્યોરિટી પગલાં}:

{\def\LTcaptype{none} % do not increment counter
\begin{longtable}[]{@{}ll@{}}
\toprule\noalign{}
સ્તર & સિક્યોરિટી પગલાં \\
\midrule\noalign{}
\endhead
\bottomrule\noalign{}
\endlastfoot
\textbf{Physical} & સર્વર રૂમ એક્સેસ, biometric locks \\
\textbf{Network} & Firewalls, VPN, intrusion detection \\
\textbf{System} & Antivirus, patches, access controls \\
\textbf{Application} & Input validation, secure coding \\
\textbf{Data} & Encryption, backup, integrity checks \\
\end{longtable}
}

\textbf{Access Control Matrix}:

{\def\LTcaptype{none} % do not increment counter
\begin{longtable}[]{@{}llll@{}}
\toprule\noalign{}
યુઝર/રોલ & ફાઈલ A & ફાઈલ B & પ્રિન્ટર \\
\midrule\noalign{}
\endhead
\bottomrule\noalign{}
\endlastfoot
Admin & RWX & RWX & RWX \\
User1 & RW- & R-- & -W- \\
Guest & R-- & --- & --- \\
\end{longtable}
}

\textbf{સિક્યોરિટી અમલીકરણ સમયસીમા}:

\begin{verbatim}
gantt
    title સિક્યોરિટી અમલીકરણ
    dateFormat  YYYY{-MM{-}DD}
    section તબક્કો 1
    જોખમ મૂલ્યાંકન    :2024{-01{-}01, 30d}
    નીતિ વિકાસ :2024{-01{-}15, 45d}
    section તબક્કો 2
    સિસ્ટમ હાર્ડેનિંગ   :2024{-02{-}01, 60d}
    તાલીમ કાર્યક્રમ   :2024{-02{-}15, 30d}
\end{verbatim}

\textbf{મોનિટરિંગ ટૂલ્સ}:

\begin{itemize}
\tightlist
\item
  \textbf{લોગ એનાલિસિસ}
\item
  \textbf{Intrusion Detection Systems}
\item
  \textbf{Vulnerability Scanners}
\end{itemize}

\end{solutionbox}
\begin{mnemonicbox}
``PNSAD'' - Physical, Network, System, Application,
Data security

\end{mnemonicbox}
\begin{center}\rule{0.5\linewidth}{0.5pt}\end{center}

\subsection*{પ્રશ્ન 5(અ) [3
ગુણ]}\label{uxaaauxab0uxab6uxaa8-5uxa85-3-uxa97uxaa3}

\textbf{calendar, date ના પાંચ મૂળભૂત કમાંડ સમજાવો}

\begin{solutionbox}

\textbf{મૂળભૂત Linux કમાંડ્સ}:

{\def\LTcaptype{none} % do not increment counter
\begin{longtable}[]{@{}lll@{}}
\toprule\noalign{}
કમાંડ & કાર્ય & ઉદાહરણ \\
\midrule\noalign{}
\endhead
\bottomrule\noalign{}
\endlastfoot
\texttt{cal} & કેલેન્ડર દર્શાવવું & \texttt{cal\ 2024} \\
\texttt{date} & વર્તમાન તારીખ/સમય બતાવવો & \texttt{date\ +\%d/\%m/\%Y} \\
\texttt{who} & લોગ-ઇન યુઝર્સ બતાવવા & \texttt{who} \\
\texttt{pwd} & વર્કિંગ ડિરેક્ટરી પ્રિન્ટ કરવી & \texttt{pwd} \\
\texttt{clear} & સ્ક્રીન સાફ કરવી & \texttt{clear} \\
\end{longtable}
}

\textbf{કમાંડ ઉદાહરણો}:

\begin{verbatim}
\# ચોક્કસ મહિના માટે કેલેન્ડર દર્શાવવો
cal 6 2024

\# તારીખ આઉટપુટ ફોર્મેટ કરવો
date "+\%A, \%B \%d, \%Y"
\end{verbatim}

\end{solutionbox}
\begin{mnemonicbox}
``CDWPC'' - Cal, Date, Who, Pwd, Clear

\end{mnemonicbox}
\begin{center}\rule{0.5\linewidth}{0.5pt}\end{center}

\subsection*{પ્રશ્ન 5(બ) [4
ગુણ]}\label{uxaaauxab0uxab6uxaa8-5uxaac-4-uxa97uxaa3}

\textbf{Linux ફાઈલ અને ડિરેક્ટરી કમાંડ સમજાવો: ls, cat, mkdir, rmdir, pwd,}

\begin{solutionbox}

\textbf{ફાઈલ અને ડિરેક્ટરી કમાંડ્સ}:

{\def\LTcaptype{none} % do not increment counter
\begin{longtable}[]{@{}llll@{}}
\toprule\noalign{}
કમાંડ & કાર્ય & Syntax & ઉદાહરણ \\
\midrule\noalign{}
\endhead
\bottomrule\noalign{}
\endlastfoot
\texttt{ls} & ડિરેક્ટરી contents લિસ્ટ કરવા &
\texttt{ls\ [options]\ [path]} & \texttt{ls\ -la} \\
\texttt{cat} & ફાઈલ content દર્શાવવો & \texttt{cat\ filename} &
\texttt{cat\ file.txt} \\
\texttt{mkdir} & ડિરેક્ટરી બનાવવી & \texttt{mkdir\ dirname} &
\texttt{mkdir\ newdir} \\
\texttt{rmdir} & ખાલી ડિરેક્ટરી દૂર કરવી & \texttt{rmdir\ dirname} &
\texttt{rmdir\ olddir} \\
\texttt{pwd} & વર્કિંગ ડિરેક્ટરી પ્રિન્ટ કરવી & \texttt{pwd} & \texttt{pwd} \\
\end{longtable}
}

\textbf{ઉપયોગ ઉદાહરણો}:

\begin{verbatim}
\# વિગતો સાથે ફાઈલો લિસ્ટ કરવી
ls {-l} /home/user

\# બહુવિધ ડિરેક્ટરીઝ બનાવવી
mkdir {-p} dir1/dir2/dir3

\# લાઈન નંબર્સ સાથે ફાઈલ દર્શાવવી
cat {-n} document.txt
\end{verbatim}

\textbf{સામાન્ય વિકલ્પો}:

\begin{itemize}
\tightlist
\item
  \texttt{ls\ -l}: લાંબો ફોર્મેટ
\item
  \texttt{ls\ -a}: છુપાયેલી ફાઈલો બતાવવી
\item
  \texttt{mkdir\ -p}: parent ડિરેક્ટરીઝ બનાવવી
\end{itemize}

\end{solutionbox}
\begin{mnemonicbox}
``LCMRP'' - List, Cat, Mkdir, Rmdir, Pwd

\end{mnemonicbox}
\begin{center}\rule{0.5\linewidth}{0.5pt}\end{center}

\subsection*{પ્રશ્ન 5(ક) [7
ગુણ]}\label{uxaaauxab0uxab6uxaa8-5uxa95-7-uxa97uxaa3}

\textbf{નિયંત્રણ નિવેદનો સમજો અને ઉપયોગ કરી શેલ સ્ક્રિપ્ટ લખો: ત્રણ સંખ્યાઓમાંથી
મહત્તમ સંખ્યા શોધવા માટે શેલ સ્ક્રિપ્ટ લખો.}

\begin{solutionbox}

\textbf{ત્રણ સંખ્યાઓમાંથી મહત્તમ માટે શેલ સ્ક્રિપ્ટ}:

\begin{verbatim}
\#!/bin/bash
\# ત્રણ સંખ્યાઓમાંથી મહત્તમ શોધવા માટે સ્ક્રિપ્ટ

echo "ત્રણ સંખ્યાઓ દાખલ કરો:"
read {-p} "પહેલી સંખ્યા: " num1
read {-p} "બીજી સંખ્યા: " num2
read {-p} "ત્રીજી સંખ્યા: " num3

\# પદ્ધતિ 1: if{-elif{-}else ઉપયોગ કરીને}
if [ $num1 {-ge} $num2 ] \&\& [ $num1 {-ge} $num3 ]; then
    max=$num1
elif [ $num2 {-ge} $num1 ] \&\& [ $num2 {-ge} $num3 ]; then
    max=$num2
else
    max=$num3
fi

echo "મહત્તમ સંખ્યા છે: $max"

\# પદ્ધતિ 2: nested if ઉપયોગ કરીને
if [ $num1 {-gt} $num2 ]; then
    if [ $num1 {-gt} $num3 ]; then
        echo "મહત્તમ: $num1"
    else
        echo "મહત્તમ: $num3"
    fi
else
    if [ $num2 {-gt} $num3 ]; then
        echo "મહત્તમ: $num2"
    else
        echo "મહત્તમ: $num3"
    fi
fi
\end{verbatim}

\textbf{ઉપયોગમાં લેવાયેલા Control Statements}:

{\def\LTcaptype{none} % do not increment counter
\begin{longtable}[]{@{}ll@{}}
\toprule\noalign{}
Statement & હેતુ \\
\midrule\noalign{}
\endhead
\bottomrule\noalign{}
\endlastfoot
\texttt{if-elif-else} & બહુવિધ condition ચેકિંગ \\
\texttt{read} & યુઝર input \\
\texttt{echo} & આઉટપુટ દર્શાવવો \\
Comparison operators & \texttt{-ge}, \texttt{-gt}, \texttt{-lt} \\
\end{longtable}
}

\textbf{Comparison Operators}:

\begin{itemize}
\tightlist
\item
  \texttt{-eq}: બરાબર
\item
  \texttt{-ne}: બરાબર નહીં
\item
  \texttt{-gt}: કરતાં મોટું
\item
  \texttt{-ge}: કરતાં મોટું અથવા સમાન
\item
  \texttt{-lt}: કરતાં નાનું
\item
  \texttt{-le}: કરતાં નાનું અથવા સમાન
\end{itemize}

\end{solutionbox}
\begin{mnemonicbox}
``IER'' - If (condition), Echo (આઉટપુટ), Read (ઇનપુટ)

\end{mnemonicbox}
\begin{center}\rule{0.5\linewidth}{0.5pt}\end{center}

\subsection*{પ્રશ્ન 5(અ) OR [3
ગુણ]}\label{uxaaauxab0uxab6uxaa8-5uxa85-or-3-uxa97uxaa3}

\textbf{top, ps, kill Linux પ્રોસેસ કમાન્ડ શું છે}

\begin{solutionbox}

\textbf{Linux પ્રોસેસ કમાંડ્સ}:

{\def\LTcaptype{none} % do not increment counter
\begin{longtable}[]{@{}lll@{}}
\toprule\noalign{}
કમાંડ & કાર્ય & ઉપયોગ \\
\midrule\noalign{}
\endhead
\bottomrule\noalign{}
\endlastfoot
\texttt{top} & ચાલતી પ્રોસેસિસ દર્શાવવી & \texttt{top} \\
\texttt{ps} & પ્રોસેસ સ્ટેટસ બતાવવો & \texttt{ps\ aux} \\
\texttt{kill} & પ્રોસેસ બંધ કરવી & \texttt{kill\ PID} \\
\end{longtable}
}

\textbf{કમાંડ વિગતો}:

\textbf{top કમાંડ}:

\begin{itemize}
\tightlist
\item
  Real-time પ્રોસેસ માહિતી બતાવે છે
\item
  CPU અને મેમરી ઉપયોગ
\item
  Load average
\end{itemize}

\textbf{ps કમાંડ વિકલ્પો}:

\begin{itemize}
\tightlist
\item
  \texttt{ps\ aux}: વિગતો સાથે બધી પ્રોસેસિસ
\item
  \texttt{ps\ -ef}: સંપૂર્ણ ફોર્મેટ લિસ્ટિંગ
\end{itemize}

\textbf{kill કમાંડ}:

\begin{itemize}
\tightlist
\item
  \texttt{kill\ -9\ PID}: પ્રોસેસને ફોર્સ kill કરવી
\item
  \texttt{killall\ process\_name}: નામ દ્વારા kill કરવી
\end{itemize}

\end{solutionbox}
\begin{mnemonicbox}
``TPK'' - Top (real-time), Ps (સ્ટેટસ), Kill (બંધ કરવી)

\end{mnemonicbox}
\begin{center}\rule{0.5\linewidth}{0.5pt}\end{center}

\subsection*{પ્રશ્ન 5(બ) OR [4
ગુણ]}\label{uxaaauxab0uxab6uxaa8-5uxaac-or-4-uxa97uxaa3}

\textbf{Linux ફાઈલ અને ડિરેક્ટરી કમાંડ સમજાવો: rm, mv, split, diff, grep}

\begin{solutionbox}

\textbf{અદ્યતન ફાઈલ કમાંડ્સ}:

{\def\LTcaptype{none} % do not increment counter
\begin{longtable}[]{@{}
  >{\raggedright\arraybackslash}p{(\linewidth - 6\tabcolsep) * \real{0.2500}}
  >{\raggedright\arraybackslash}p{(\linewidth - 6\tabcolsep) * \real{0.2778}}
  >{\raggedright\arraybackslash}p{(\linewidth - 6\tabcolsep) * \real{0.2222}}
  >{\raggedright\arraybackslash}p{(\linewidth - 6\tabcolsep) * \real{0.2500}}@{}}
\toprule\noalign{}
\begin{minipage}[b]{\linewidth}\raggedright
કમાંડ
\end{minipage} & \begin{minipage}[b]{\linewidth}\raggedright
કાર્ય
\end{minipage} & \begin{minipage}[b]{\linewidth}\raggedright
Syntax
\end{minipage} & \begin{minipage}[b]{\linewidth}\raggedright
ઉદાહરણ
\end{minipage} \\
\midrule\noalign{}
\endhead
\bottomrule\noalign{}
\endlastfoot
\texttt{rm} & ફાઈલો/ડિરેક્ટરીઝ દૂર કરવી & \texttt{rm\ [options]\ file}
& \texttt{rm\ -rf\ folder} \\
\texttt{mv} & ફાઈલો ખસેડવી/નામ બદલવું & \texttt{mv\ source\ dest} &
\texttt{mv\ old.txt\ new.txt} \\
\texttt{split} & મોટી ફાઈલો વિભાજિત કરવી &
\texttt{split\ -l\ lines\ file} & \texttt{split\ -l\ 100\ data.txt} \\
\texttt{diff} & ફાઈલો તુલના કરવી & \texttt{diff\ file1\ file2} &
\texttt{diff\ old.txt\ new.txt} \\
\texttt{grep} & ટેક્સ્ટ પેટર્ન શોધવા & \texttt{grep\ pattern\ file} &
\texttt{grep\ "error"\ log.txt} \\
\end{longtable}
}

\textbf{ઉપયોગ ઉદાહરણો}:

\begin{verbatim}
\# ડિરેક્ટરી recursively દૂર કરવી
rm {-rf} /tmp/oldfiles

\# ખસેડવું અને નામ બદલવું
mv /home/user/doc.txt /backup/document.txt

\# ફાઈલને 50{-line chunks માં વિભાજિત કરવી}
split {-l} 50 largefile.txt chunk\_

\# ફાઈલો વચ્ચે તફાવત શોધવો
diff {-u} original.txt modified.txt

\# બહુવિધ ફાઈલોમાં પેટર્ન શોધવો
grep {-r} "TODO" /project/src/
\end{verbatim}

\textbf{સામાન્ય વિકલ્પો}:

\begin{itemize}
\tightlist
\item
  \texttt{rm\ -i}: Interactive mode
\item
  \texttt{mv\ -i}: Overwrite પહેલાં પૂછવું
\item
  \texttt{grep\ -i}: Case insensitive શોધ
\end{itemize}

\end{solutionbox}
\begin{mnemonicbox}
``RMSDG'' - Remove, Move, Split, Diff, Grep

\end{mnemonicbox}
\begin{center}\rule{0.5\linewidth}{0.5pt}\end{center}

\subsection*{પ્રશ્ન 5(ક) OR [7
ગુણ]}\label{uxaaauxab0uxab6uxaa8-5uxa95-or-7-uxa97uxaa3}

\textbf{શેલ સ્ક્રિપ્ટ લખો:પાંચ નંબરો આપી અને પાંચ સંખ્યાઓની સરેરાશ શોધો.}

\begin{solutionbox}

\textbf{પાંચ સંખ્યાઓની સરેરાશ માટે શેલ સ્ક્રિપ્ટ}:

\begin{verbatim}
\#!/bin/bash
\# પાંચ સંખ્યાઓની સરેરાશ ગણતરી માટે સ્ક્રિપ્ટ

echo "=== સરેરાશ કેલ્ક્યુલેટર ==="
echo "પાંચ સંખ્યાઓ દાખલ કરો:"

\# પાંચ સંખ્યાઓ વાંચવી
read {-p} "સંખ્યા 1 દાખલ કરો: " num1
read {-p} "સંખ્યા 2 દાખલ કરો: " num2
read {-p} "સંખ્યા 3 દાખલ કરો: " num3
read {-p} "સંખ્યા 4 દાખલ કરો: " num4
read {-p} "સંખ્યા 5 દાખલ કરો: " num5

\# બાદબાકી ગણતરી
sum=$((num1 + num2 + num3 + num4 + num5))

\# સરેરાશ ગણતરી
average=$((sum / 5))

\# પરિણામો દર્શાવવા
echo "================================"
echo "દાખલ કરેલી સંખ્યાઓ: $num1, $num2, $num3, $num4, $num5"
echo "બાદબાકી: $sum"
echo "સરેરાશ: $average"
echo "================================"

\# દશાંશ ચોકસાઈ સાથે વિસ્તૃત વર્ઝન
sum\_float=$(echo "$num1 + $num2 + $num3 + $num4 + $num5" | bc)
avg\_float=$(echo "scale=2; $sum\_float / 5" | bc)
echo "ચોક્કસ સરેરાશ: $avg\_float"
\end{verbatim}

\textbf{Arrays ઉપયોગ કરીને વૈકલ્પિક પદ્ધતિ}:

\begin{verbatim}
\#!/bin/bash
\# Array approach ઉપયોગ કરીને

declare {-a} numbers
sum=0

echo "5 સંખ્યાઓ દાખલ કરો:"
for i in \{0..4\}; do
    read {-p} "સંખ્યા $((i+1)): " numbers[i]
    sum=$((sum + numbers[i]))
done

average=$((sum / 5))

echo "સંખ્યાઓ: $\{numbers[@]\}"
echo "બાદબાકી: $sum"
echo "સરેરાશ: $average"
\end{verbatim}

\textbf{સ્ક્રિપ્ટ લક્ષણો}:

{\def\LTcaptype{none} % do not increment counter
\begin{longtable}[]{@{}ll@{}}
\toprule\noalign{}
લક્ષણ & વર્ણન \\
\midrule\noalign{}
\endhead
\bottomrule\noalign{}
\endlastfoot
\textbf{Input Validation} & Numeric input ચેક કરવો \\
\textbf{યુઝર-ફ્રેન્ડલી આઉટપુટ} & સ્પષ્ટ ફોર્મેટિંગ \\
\textbf{Array ઉપયોગ} & બહુવિધ વેલ્યુઝ સ્ટોર કરવી \\
\textbf{અંકગણિત ઓપરેશન્સ} & બાદબાકી અને ભાગાકાર \\
\end{longtable}
}

\textbf{Bash માં ગાણિતિક ઓપરેશન્સ}:

\begin{itemize}
\tightlist
\item
  \texttt{\$((expression))}: Integer arithmetic
\item
  \texttt{bc}: Floating point માટે calculator
\item
  \texttt{expr}: Expression evaluation
\end{itemize}

\end{solutionbox}
\begin{mnemonicbox}
``RSAR'' - Read (ઇનપુટ), Sum (ઉમેરવું), Average
(ભાગાકાર), Result (પરિણામ)

\end{mnemonicbox}

\end{document}
