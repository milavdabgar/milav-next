\documentclass[10pt,a4paper]{article}

% content/resources/templates/preamble.tex
\usepackage[margin=0.6in]{geometry}
\author{Milav Dabgar}
\usepackage{amsmath,amssymb,amsthm}
\usepackage{booktabs}
\usepackage{multirow}
\usepackage{xcolor}
\usepackage{tcolorbox}
\tcbuselibrary{breakable,skins}
\usepackage[colorlinks=true,linkcolor=blue]{hyperref}
\usepackage{titlesec}
\usepackage{enumitem}
\usepackage{tikz}
\usepackage{pgfplots}
\usepackage{circuitikz}
\usepackage[version=4]{mhchem}
\usepackage{longtable}
\usepackage{array}
\usepackage{float}
\usepackage{caption}
\usepackage{listings}

\lstset{
  basicstyle=\small\ttfamily,
  breaklines=true,
  breakatwhitespace=false,
  postbreak=\mbox{\textcolor{red}{$\hookrightarrow$}\space},
  float=false,
  numbers=left,
  numberstyle=\tiny\color{gray},
  numbersep=10pt,
  xleftmargin=2em,
  keywordstyle=\color{blue},
  commentstyle=\color{green!60!black},
  stringstyle=\color{purple},
  backgroundcolor=\color{gray!5},
  showstringspaces=false,
  tabsize=2,
  captionpos=b,
  keepspaces=true,
  columns=flexible
}

\pgfplotsset{compat=1.18}
\usetikzlibrary{shapes,arrows,positioning,calc,patterns,decorations.pathmorphing,decorations.markings,arrows.meta}

% Color scheme
\definecolor{headcolor}{RGB}{0,102,204}
\definecolor{keycolor}{RGB}{220,20,60}
\definecolor{solutioncolor}{RGB}{34,139,34}
\definecolor{mnemoniccolor}{RGB}{148,0,211}
\definecolor{codecolor}{RGB}{0,0,100}

% Spacing
\setlength{\parskip}{3pt}
\setlist[itemize]{nosep}
\setlist[enumerate]{nosep}

% Title formatting
\titleformat{\section}{\Large\bfseries\color{headcolor}}{\thesection}{1em}{}
\titleformat{\subsection}{\large\bfseries\color{headcolor}}{\thesubsection}{1em}{}

% Pandoc tightlist compatibility
\providecommand{\tightlist}{%
  \setlength{\itemsep}{0pt}\setlength{\parskip}{0pt}}

% Pandoc longtable compatibility
\newcounter{none}
\def\thenone{}


% content/resources/templates/gujarati-boxes.tex
\usepackage{fontspec}
\usepackage{polyglossia}

% Set Gujarati as main language (document is primarily in Gujarati)
% Note: gloss-gujarati.ldf doesn't exist in polyglossia, but it will use hyphenation patterns
\setdefaultlanguage{gujarati}
\setotherlanguage{english}

% Configure Gujarati font properly
% Use Language=Default to prevent polyglossia from trying to add language-specific features
% that don't exist for Gujarati, which causes "empty feature" warnings
\newfontfamily\gujaratifont[Script=Gujarati,AutoFakeBold=2.5,AutoFakeSlant=0.3]{Noto Sans Gujarati}
\setmainfont[Script=Gujarati,AutoFakeBold=2.5,AutoFakeSlant=0.3]{Noto Sans Gujarati}
% Use Noto Sans Gujarati for monospace to support Gujarati in text
\setmonofont[Scale=0.9]{Noto Sans Gujarati}

% Configure English to use the same font
\newfontfamily\englishfont[Script=Gujarati,AutoFakeBold=2.5,AutoFakeSlant=0.3]{Noto Sans Gujarati}

% Translations for polyglossia
\gappto\captionsgujarati{
  \renewcommand{\tablename}{કોષ્ટક}
  \renewcommand{\figurename}{આકૃતિ}
}

% Helper for TikZ nodes to ensure Gujarati font
\newcommand{\gu}[1]{{\gujaratifont #1}}

% Custom environments
\newtcolorbox{solutionbox}{
    breakable,
    enhanced,
    colback=solutioncolor!5!white,
    colframe=solutioncolor!75!black,
    fonttitle=\bfseries,
    title=જવાબ
}

\newtcolorbox{solutionboxnobreak}{
 colback=solutioncolor!5!white,
 colframe=solutioncolor!75!black,
 fonttitle=\bfseries,
 title=જવાબ
}

\newtcolorbox{keyformula}{
 breakable,
 enhanced,
 colback=keycolor!5!white,
 colframe=keycolor!75!black,
 fonttitle=\bfseries,
 title=રાસાયણિક સમીકરણ/સૂત્ર
}

\newtcolorbox{mnemonicbox}{
 breakable,
 enhanced,
 colback=mnemoniccolor!5!white,
 colframe=mnemoniccolor!75!black,
 fonttitle=\bfseries,
 title=મેમરી ટ્રીક
}


\begin{document}

\begin{center}
{\Huge\bfseries\color{headcolor} Subject Name (Gujarati)}\\[5pt]
{\LARGE 4331602 -- Winter 2023}\\[3pt]
{\large Semester 1 Study Material}\\[3pt]
{\normalsize\textit{Detailed Solutions and Explanations}}
\end{center}

\vspace{10pt}

\subsection*{પ્રશ્ન 1(a) [3
માર્ક્સ]}\label{q1a}

\textbf{Linux ના આર્કિટેક્ચર દોરો અને સંક્ષિપ્તમાં વિવિધ સ્તરો સમજાવો.}

\begin{solutionbox}

\textbf{આકૃતિ:}

\begin{center}
\textbf{Mermaid Diagram (Code)}
\begin{verbatim}
{Shaded}
{Highlighting}[]
graph LR
    A[User Applications] {-{-}{} B[System Call Interface]}
    B {-{-}{} C[Kernel]}
    C {-{-}{} D[Device Drivers]}
    D {-{-}{} E[Hardware]}
    
    C {-{-}{} F[Process Management]}
    C {-{-}{} G[Memory Management]}
    C {-{-}{} H[File System]}
    C {-{-}{} I[Network Management]}
{Highlighting}
{Shaded}
\end{verbatim}
\end{center}

\begin{itemize}
\tightlist
\item
  \textbf{User Space}: વપરાશકર્તા applications અને system utilities સમાવે છે
\item
  \textbf{System Call Interface}: user programs અને kernel વચ્ચે interface
  પ્રદાન કરે છે
\item
  \textbf{Kernel Space}: process, memory, file management સાથે મૂળ
  operating system
\end{itemize}

\end{solutionbox}
\begin{mnemonicbox}
``Users System Kernel Drives Hardware''

\end{mnemonicbox}
\begin{center}\rule{0.5\linewidth}{0.5pt}\end{center}

\subsection*{પ્રશ્ન 1(b) [4
માર્ક્સ]}\label{q1b}

\textbf{રેસની સ્થિતિ શું છે? યોગ્ય ઉદાહરણ સાથે સમજાવો.}

\begin{solutionbox}

{\def\LTcaptype{none} % do not increment counter
\begin{longtable}[]{@{}ll@{}}
\toprule\noalign{}
\textbf{પાસું} & \textbf{વિવરણ} \\
\midrule\noalign{}
\endhead
\bottomrule\noalign{}
\endlastfoot
\textbf{વ્યાખ્યા} & અનેક processes એકસાથે shared resource ને access કરે છે \\
\textbf{સમસ્યા} & timing dependency ને કારણે અનિશ્ચિત પરિણામો \\
\textbf{ઉદાહરણ} & બે transactions દ્વારા બેંક account balance ને update
કરવું \\
\end{longtable}
}

\textbf{ઉદાહરણ પ્રક્રિયા:}

\begin{itemize}
\tightlist
\item
  \textbf{Process A}: balance = 1000 વાંચે છે, 100 ઉમેરે છે
\item
  \textbf{Process B}: balance = 1000 વાંચે છે, 50 બાદ કરે છે\\
\item
  \textbf{પરિણામ}: અંતિમ balance 1050, 950, અથવા 1100 હોઈ શકે યોગ્ય 1050 ને
  બદલે
\end{itemize}

\end{solutionbox}
\begin{mnemonicbox}
``Race Results Random Resources''

\end{mnemonicbox}
\begin{center}\rule{0.5\linewidth}{0.5pt}\end{center}

\subsection*{પ્રશ્ન 1(c) [7
માર્ક્સ]}\label{q1c}

\textbf{વિવિધ પ્રકારની ઓપરેટિંગ સિસ્ટમોની યાદી બનાવો. મલ્ટિપ્રોગ્રામિંગ ઓપરેટિંગ
સિસ્ટમના કાર્યને યોગ્ય ઉદાહરણ સાથે સમજાવો.}

\begin{solutionbox}


{\def\LTcaptype{none} % do not increment counter
\vspace{-5pt}
\captionof{table}{ઓપરેટિંગ સિસ્ટમના પ્રકારો}
\vspace{-10pt}
\begin{longtable}[]{@{}lll@{}}
\toprule\noalign{}
\textbf{પ્રકાર} & \textbf{લક્ષણો} & \textbf{ઉદાહરણ} \\
\midrule\noalign{}
\endhead
\bottomrule\noalign{}
\endlastfoot
\textbf{Batch} & Jobs બેચમાં process થાય છે & IBM mainframes \\
\textbf{Time-sharing} & અનેક વપરાશકર્તાઓ એકસાથે & UNIX \\
\textbf{Real-time} & તાત્કાલિક response જરૂરી & Air traffic control \\
\textbf{Distributed} & અનેક connected computers & Google cluster \\
\textbf{Multiprogramming} & memory માં અનેક programs & Windows, Linux \\
\end{longtable}
}

\textbf{Multiprogramming કાર્યપદ્ધતિ:}

\begin{itemize}
\tightlist
\item
  \textbf{Memory Management}: અનેક programs એકસાથે load થાય છે
\item
  \textbf{CPU Scheduling}: I/O occurrence દરમિયાન programs વચ્ચે switch કરે
  છે
\item
  \textbf{Resource Sharing}: CPU અને memory નો efficient ઉપયોગ
\item
  \textbf{ઉદાહરણ}: Word processor, music player, અને browser એકસાથે ચાલે છે
\end{itemize}

\end{solutionbox}
\begin{mnemonicbox}
``Multiple Programs Maximize Performance''

\end{mnemonicbox}
\begin{center}\rule{0.5\linewidth}{0.5pt}\end{center}

\subsection*{પ્રશ્ન 1(c OR) [7
માર્ક્સ]}\label{uxaaauxab0uxab6uxaa8-1c-or-7-uxaaeuxab0uxa95uxab8}

\textbf{વિવિધ પ્રકારની ઓપરેટિંગ સિસ્ટમોની યાદી બનાવો. બેચ ઓપરેટિંગ સિસ્ટમ્સ
વિગતવાર સમજાવો.}

\begin{solutionbox}

\textbf{ઓપરેટિંગ સિસ્ટમના પ્રકારો:} ઉપરનું કોષ્ટક સમાન.

\textbf{Batch Operating System વિગતો:}

\begin{itemize}
\tightlist
\item
  \textbf{Job Collection}: Jobs offline collect થાય અને batches માં group
  થાય છે
\item
  \textbf{Sequential Processing}: Jobs એક પછી એક execute થાય વપરાશકર્તા
  interaction વગર
\item
  \textbf{No Direct Interaction}: વપરાશકર્તા job submit કરે અને પછીથી
  output collect કરે
\item
  \textbf{Efficiency}: સમાન પ્રકારના jobs માટે high throughput
\item
  \textbf{ગેરફાયદા}: Real-time processing નથી, લાંબો turnaround time
\end{itemize}

\end{solutionbox}
\begin{mnemonicbox}
``Batch Brings Better Business''

\end{mnemonicbox}
\begin{center}\rule{0.5\linewidth}{0.5pt}\end{center}

\subsection*{પ્રશ્ન 2(a) [3
માર્ક્સ]}\label{q2a}

\textbf{પ્રક્રિયા જીવન ચક્ર દોરો અને સમજાવો.}

\begin{solutionbox}

\textbf{આકૃતિ:}

\begin{verbatim}
stateDiagram{-v2}
  direction LR
    [*] {-{-} New}
    New {-{-} Ready : Admitted}
    Ready {-{-} Running : Scheduler\_dispatch}
    Running {-{-} Ready : Interrupt}
    Running {-{-} Waiting : I/O\_request}
    Waiting {-{-} Ready : I/O\_completion}
    Running {-{-} Terminated : Exit}
    Terminated {-{-} [*]}
\end{verbatim}

\begin{itemize}
\tightlist
\item
  \textbf{New}: પ્રક્રિયા બનાવવામાં આવે છે
\item
  \textbf{Ready}: પ્રક્રિયા CPU assignment માટે રાહ જોતી છે
\item
  \textbf{Running}: પ્રક્રિયા હાલમાં execute થઈ રહી છે
\item
  \textbf{Waiting}: પ્રક્રિયા I/O operation માટે રાહ જોતી છે
\item
  \textbf{Terminated}: પ્રક્રિયાએ execution પૂર્ણ કર્યું છે
\end{itemize}

\end{solutionbox}
\begin{mnemonicbox}
``New Ready Running Waiting Terminated''

\end{mnemonicbox}
\begin{center}\rule{0.5\linewidth}{0.5pt}\end{center}

\subsection*{પ્રશ્ન 2(b) [4
માર્ક્સ]}\label{q2b}

\textbf{ડેડલોકને વ્યાખ્યાયિત કરો અને ડેડલોક થવા માટે જરૂરી શરતોની ચર્ચા કરો.}

\begin{solutionbox}

\textbf{વ્યાખ્યા}: ડેડલોક ત્યારે થાય છે જ્યારે processes અન્ય processes દ્વારા
held resources માટે અનિશ્ચિત સમય સુધી રાહ જોતી રહે છે.


{\def\LTcaptype{none} % do not increment counter
\vspace{-5pt}
\captionof{table}{ડેડલોક શરતો}
\vspace{-10pt}
\begin{longtable}[]{@{}
  >{\raggedright\arraybackslash}p{(\linewidth - 2\tabcolsep) * \real{0.4500}}
  >{\raggedright\arraybackslash}p{(\linewidth - 2\tabcolsep) * \real{0.5500}}@{}}
\toprule\noalign{}
\begin{minipage}[b]{\linewidth}\raggedright
\textbf{શરત}
\end{minipage} & \begin{minipage}[b]{\linewidth}\raggedright
\textbf{વિવરણ}
\end{minipage} \\
\midrule\noalign{}
\endhead
\bottomrule\noalign{}
\endlastfoot
\textbf{Mutual Exclusion} & Resources share કરી શકાતા નથી \\
\textbf{Hold and Wait} & Process resource hold કરીને બીજા માટે રાહ જુએ છે \\
\textbf{No Preemption} & Resources જબરદસ્તીથી લઈ શકાતા નથી \\
\textbf{Circular Wait} & Processes resource dependencies નું circular
chain બનાવે છે \\
\end{longtable}
}

\end{solutionbox}
\begin{mnemonicbox}
``My Hold Never Circles''

\end{mnemonicbox}
\begin{center}\rule{0.5\linewidth}{0.5pt}\end{center}

\subsection*{પ્રશ્ન 2(c) [7
માર્ક્સ]}\label{q2c}

\textbf{રાઉન્ડ રોબિન અલ્ગોરિધમનું વર્ણન કરો. આપેલ ડેટા માટે ગેન્ટ ચાર્ટ સાથે સરેરાશ
રાહ જોવાનો સમય અને સરેરાશ ટર્ન-અરાઉન્ડ સમયની ગણતરી કરો. સંદર્ભ સ્વિચ = 01 ms અને
ક્વોન્ટમ સમય = 05 ms ધ્યાનમાં લો.}

\begin{solutionbox}

\textbf{Round Robin અલ્ગોરિધમ:}

\begin{itemize}
\tightlist
\item
  \textbf{Time Quantum}: દરેક process માટે fixed time slice
\item
  \textbf{Preemptive}: Quantum expire થયા પછી process preempt થાય છે
\item
  \textbf{Fair Scheduling}: સમાન CPU time વિતરણ
\end{itemize}

\textbf{આપેલ ડેટા:}

\begin{itemize}
\tightlist
\item
  Context Switch = 1 ms, Quantum = 5 ms
\end{itemize}

\textbf{ગેન્ટ ચાર્ટ:}

\begin{verbatim}
|P1|CS|P2|CS|P3|CS|P4|CS|P1|CS|P3|CS|P1|CS|P3|CS|
0  5  6 10 11 16 17 22 23 28 29 34 35 40 41 46 47
\end{verbatim}

\textbf{ગણતરી કોષ્ટક:}

{\def\LTcaptype{none} % do not increment counter
\begin{longtable}[]{@{}
  >{\raggedright\arraybackslash}p{(\linewidth - 10\tabcolsep) * \real{0.1585}}
  >{\raggedright\arraybackslash}p{(\linewidth - 10\tabcolsep) * \real{0.1585}}
  >{\raggedright\arraybackslash}p{(\linewidth - 10\tabcolsep) * \real{0.1341}}
  >{\raggedright\arraybackslash}p{(\linewidth - 10\tabcolsep) * \real{0.1951}}
  >{\raggedright\arraybackslash}p{(\linewidth - 10\tabcolsep) * \real{0.1951}}
  >{\raggedright\arraybackslash}p{(\linewidth - 10\tabcolsep) * \real{0.1585}}@{}}
\toprule\noalign{}
\begin{minipage}[b]{\linewidth}\raggedright
\textbf{Process}
\end{minipage} & \begin{minipage}[b]{\linewidth}\raggedright
\textbf{Arrival}
\end{minipage} & \begin{minipage}[b]{\linewidth}\raggedright
\textbf{Burst}
\end{minipage} & \begin{minipage}[b]{\linewidth}\raggedright
\textbf{Completion}
\end{minipage} & \begin{minipage}[b]{\linewidth}\raggedright
\textbf{Turnaround}
\end{minipage} & \begin{minipage}[b]{\linewidth}\raggedright
\textbf{Waiting}
\end{minipage} \\
\midrule\noalign{}
\endhead
\bottomrule\noalign{}
\endlastfoot
P1 & 0 & 12 & 40 & 40 & 28 \\
P2 & 3 & 4 & 10 & 7 & 3 \\
P3 & 2 & 15 & 46 & 44 & 29 \\
P4 & 5 & 5 & 22 & 17 & 12 \\
\end{longtable}
}

\begin{itemize}
\tightlist
\item
  \textbf{સરેરાશ Waiting Time}: (28+3+29+12)/4 = 18 ms
\item
  \textbf{સરેરાશ Turnaround Time}: (40+7+44+17)/4 = 27 ms
\end{itemize}

\end{solutionbox}
\begin{mnemonicbox}
``Round Robin Rotates Regularly''

\end{mnemonicbox}
\begin{center}\rule{0.5\linewidth}{0.5pt}\end{center}

\subsection*{પ્રશ્ન 2(a OR) [3
માર્ક્સ]}\label{uxaaauxab0uxab6uxaa8-2a-or-3-uxaaeuxab0uxa95uxab8}

\textbf{તફાવત: CPU બાઉન્ડ પ્રક્રિયા v/s I/O બાઉન્ડ પ્રક્રિયા.}

\begin{solutionbox}


{\def\LTcaptype{none} % do not increment counter
\vspace{-5pt}
\captionof{table}{CPU vs I/O બાઉન્ડ પ્રક્રિયાઓ}
\vspace{-10pt}
\begin{longtable}[]{@{}lll@{}}
\toprule\noalign{}
\textbf{પાસું} & \textbf{CPU બાઉન્ડ} & \textbf{I/O બાઉન્ડ} \\
\midrule\noalign{}
\endhead
\bottomrule\noalign{}
\endlastfoot
\textbf{CPU વપરાશ} & ઉચ્ચ CPU utilization & નીચો CPU utilization \\
\textbf{I/O Operations} & ન્યૂનતમ I/O & વારંવાર I/O \\
\textbf{ઉદાહરણો} & ગાણિતિક ગણતરીઓ & File operations \\
\textbf{Scheduling} & લાંબા time quantum ની જરૂર & ટૂંકા quantum થી ફાયદો \\
\textbf{Performance} & CPU speed થી મર્યાદિત & I/O speed થી મર્યાદિત \\
\end{longtable}
}

\end{solutionbox}
\begin{mnemonicbox}
``CPU Computes, I/O Interacts''

\end{mnemonicbox}
\begin{center}\rule{0.5\linewidth}{0.5pt}\end{center}

\subsection*{પ્રશ્ન 2(b OR) [4
માર્ક્સ]}\label{uxaaauxab0uxab6uxaa8-2b-or-4-uxaaeuxab0uxa95uxab8}

\textbf{ક્રિટિકલ સેક્શનને વ્યાખ્યાયિત કરો અને ક્રિટિકલ સેક્શન સોલ્યુશનની સામાન્ય
રચનાની ચર્ચા કરો.}

\begin{solutionbox}

\textbf{વ્યાખ્યા}: ક્રિટિકલ સેક્શન એ code segment છે જ્યાં shared resources ને
access કરવામાં આવે છે અને atomically execute થવું જ જોઈએ.


{\def\LTcaptype{none} % do not increment counter
\vspace{-5pt}
\captionof{table}{ક્રિટિકલ સેક્શન સ્ટ્રક્ચર}
\vspace{-10pt}
\begin{longtable}[]{@{}ll@{}}
\toprule\noalign{}
\textbf{વિભાગ} & \textbf{હેતુ} \\
\midrule\noalign{}
\endhead
\bottomrule\noalign{}
\endlastfoot
\textbf{Entry Section} & ક્રિટિકલ સેક્શનમાં પ્રવેશ માટે permission માંગે છે \\
\textbf{Critical Section} & Shared resources ને access કરતો કોડ \\
\textbf{Exit Section} & Permission રિલીઝ કરે છે \\
\textbf{Remainder Section} & Shared resources access કરતો નથી તે બીજો
કોડ \\
\end{longtable}
}

\textbf{Solution Requirements:}

\begin{itemize}
\tightlist
\item
  \textbf{Mutual Exclusion}: ક્રિટિકલ સેક્શનમાં ફક્ત એક જ process
\item
  \textbf{Progress}: આગળની process ની selection અનિશ્ચિત સમય સુધી postpone
  થઈ શકતી નથી
\item
  \textbf{Bounded Waiting}: Waiting time પર મર્યાદા
\end{itemize}

\end{solutionbox}
\begin{mnemonicbox}
``Enter Critical Exit Remainder''

\end{mnemonicbox}
\begin{center}\rule{0.5\linewidth}{0.5pt}\end{center}

\subsection*{પ્રશ્ન 2(c OR) [7
માર્ક્સ]}\label{uxaaauxab0uxab6uxaa8-2c-or-7-uxaaeuxab0uxa95uxab8}

\textbf{SJF અલ્ગોરિધમનું વર્ણન કરો. કોષ્ટકમાં આપેલ ડેટા માટે ગેન્ટ ચાર્ટ સાથે સરેરાશ
રાહ જોવાનો સમય અને સરેરાશ ટર્ન-અરાઉન્ડ સમયની ગણતરી કરો.}

\begin{solutionbox}

\textbf{SJF અલ્ગોરિધમ:}

\begin{itemize}
\tightlist
\item
  \textbf{Shortest Job First}: સૌથી નાના burst time વાળી process પહેલાં
  schedule થાય
\item
  \textbf{Non-preemptive}: Process completion સુધી ચાલે છે
\item
  \textbf{Optimal}: સરેરાશ waiting time ને minimize કરે છે
\end{itemize}

\textbf{Execution Order}: P2(4), P4(5), P1(8), P3(9)

\textbf{ગેન્ટ ચાર્ટ:}

\begin{verbatim}
|  P1  |  P2  |  P4  |     P3     |
0      8     12     17          26
\end{verbatim}

\textbf{ગણતરી કોષ્ટક:}

{\def\LTcaptype{none} % do not increment counter
\begin{longtable}[]{@{}
  >{\raggedright\arraybackslash}p{(\linewidth - 12\tabcolsep) * \real{0.1398}}
  >{\raggedright\arraybackslash}p{(\linewidth - 12\tabcolsep) * \real{0.1398}}
  >{\raggedright\arraybackslash}p{(\linewidth - 12\tabcolsep) * \real{0.1183}}
  >{\raggedright\arraybackslash}p{(\linewidth - 12\tabcolsep) * \real{0.1183}}
  >{\raggedright\arraybackslash}p{(\linewidth - 12\tabcolsep) * \real{0.1720}}
  >{\raggedright\arraybackslash}p{(\linewidth - 12\tabcolsep) * \real{0.1720}}
  >{\raggedright\arraybackslash}p{(\linewidth - 12\tabcolsep) * \real{0.1398}}@{}}
\toprule\noalign{}
\begin{minipage}[b]{\linewidth}\raggedright
\textbf{Process}
\end{minipage} & \begin{minipage}[b]{\linewidth}\raggedright
\textbf{Arrival}
\end{minipage} & \begin{minipage}[b]{\linewidth}\raggedright
\textbf{Burst}
\end{minipage} & \begin{minipage}[b]{\linewidth}\raggedright
\textbf{Start}
\end{minipage} & \begin{minipage}[b]{\linewidth}\raggedright
\textbf{Completion}
\end{minipage} & \begin{minipage}[b]{\linewidth}\raggedright
\textbf{Turnaround}
\end{minipage} & \begin{minipage}[b]{\linewidth}\raggedright
\textbf{Waiting}
\end{minipage} \\
\midrule\noalign{}
\endhead
\bottomrule\noalign{}
\endlastfoot
P1 & 0 & 8 & 0 & 8 & 8 & 0 \\
P2 & 3 & 4 & 8 & 12 & 9 & 5 \\
P3 & 5 & 9 & 17 & 26 & 21 & 12 \\
P4 & 6 & 5 & 12 & 17 & 11 & 6 \\
\end{longtable}
}

\begin{itemize}
\tightlist
\item
  \textbf{સરેરાશ Waiting Time}: (0+5+12+6)/4 = 5.75 ms
\item
  \textbf{સરેરાશ Turnaround Time}: (8+9+21+11)/4 = 12.25 ms
\end{itemize}

\end{solutionbox}
\begin{mnemonicbox}
``Shortest Jobs Start Soon''

\end{mnemonicbox}
\begin{center}\rule{0.5\linewidth}{0.5pt}\end{center}

\subsection*{પ્રશ્ન 3(a) [3
માર્ક્સ]}\label{q3a}

\textbf{બે-સ્તરની ડિરેક્ટરી રચના સમજાવો.}

\begin{solutionbox}

\textbf{આકૃતિ:}

\begin{verbatim}
    Master File Directory (MFD)
           |
    +{-{-}{-}{-}{-}{-}+{-}{-}{-}{-}{-}{-}+}
    |             |
   User1         User2
 Directory      Directory
    |             |
  File1         File3
  File2         File4
\end{verbatim}

\begin{itemize}
\tightlist
\item
  \textbf{Master File Directory}: દરેક વપરાશકર્તા માટે entries સમાવે છે
\item
  \textbf{User File Directory}: દરેક વપરાશકર્તાની files માટે અલગ directory
\item
  \textbf{Path Structure}: /user/filename
\item
  \textbf{ફાયદા}: Naming conflicts ને ઉકેલે છે, user isolation પ્રદાન કરે છે
\end{itemize}

\end{solutionbox}
\begin{mnemonicbox}
``Two Tiers Tackle Troubles''

\end{mnemonicbox}
\begin{center}\rule{0.5\linewidth}{0.5pt}\end{center}

\subsection*{પ્રશ્ન 3(b) [4
માર્ક્સ]}\label{q3b}

\textbf{વિવિધ ફાઇલ કામગીરી સમજાવો.}

\begin{solutionbox}


{\def\LTcaptype{none} % do not increment counter
\vspace{-5pt}
\captionof{table}{ફાઇલ ઓપરેશન્સ}
\vspace{-10pt}
\begin{longtable}[]{@{}lll@{}}
\toprule\noalign{}
\textbf{ઓપરેશન} & \textbf{હેતુ} & \textbf{ઉદાહરણ} \\
\midrule\noalign{}
\endhead
\bottomrule\noalign{}
\endlastfoot
\textbf{Create} & નવી file બનાવવી & touch file.txt \\
\textbf{Open} & ઓપરેશન્સ માટે file ને access કરવી & fopen() \\
\textbf{Read} & File માંથી data retrieve કરવો & fread() \\
\textbf{Write} & File માં data store કરવો & fwrite() \\
\textbf{Close} & File access ને terminate કરવી & fclose() \\
\textbf{Delete} & File ને remove કરવી & rm file.txt \\
\end{longtable}
}

\end{solutionbox}
\begin{mnemonicbox}
``Create Open Read Write Close Delete''

\end{mnemonicbox}
\begin{center}\rule{0.5\linewidth}{0.5pt}\end{center}

\subsection*{પ્રશ્ન 3(c) [7
માર્ક્સ]}\label{q3c}

\textbf{વિવિધ ફાઈલ ફાળવણી પદ્ધતિઓની યાદી બનાવો અને જરૂરી રેખાકૃતિ સાથે સંલગ્ન
ફાળવણી સમજાવો.}

\begin{solutionbox}

\textbf{ફાઇલ ફાળવણી પદ્ધતિઓ:}

\begin{itemize}
\tightlist
\item
  \textbf{Contiguous Allocation}
\item
  \textbf{Linked Allocation}
\item
  \textbf{Indexed Allocation}
\end{itemize}

\textbf{સંલગ્ન ફાળવણી:}

\textbf{આકૃતિ:}

\begin{verbatim}
File A: |Block1|Block2|Block3|
File B: |Block4|Block5|
File C: |Block6|Block7|Block8|Block9|
\end{verbatim}


{\def\LTcaptype{none} % do not increment counter
\vspace{-5pt}
\captionof{table}{સંલગ્ન ફાળવણી}
\vspace{-10pt}
\begin{longtable}[]{@{}ll@{}}
\toprule\noalign{}
\textbf{પાસું} & \textbf{વિવરણ} \\
\midrule\noalign{}
\endhead
\bottomrule\noalign{}
\endlastfoot
\textbf{Storage} & Files consecutive blocks માં store થાય છે \\
\textbf{Access} & કોઈપણ block ને direct access \\
\textbf{ફાયદા} & ઝડપી access, સરળ implementation \\
\textbf{ગેરફાયદા} & External fragmentation, expansion મુશ્કેલ \\
\end{longtable}
}

\textbf{Directory Entry}: (Start block, Length)

\end{solutionbox}
\begin{mnemonicbox}
``Contiguous Creates Continuous Clusters''

\end{mnemonicbox}
\begin{center}\rule{0.5\linewidth}{0.5pt}\end{center}

\subsection*{પ્રશ્ન 3(a OR) [3
માર્ક્સ]}\label{uxaaauxab0uxab6uxaa8-3a-or-3-uxaaeuxab0uxa95uxab8}

\textbf{ફાઇલ સ્ટ્રક્ચરના પ્રકારોનું વર્ણન કરો.}

\begin{solutionbox}


{\def\LTcaptype{none} % do not increment counter
\vspace{-5pt}
\captionof{table}{ફાઇલ સ્ટ્રક્ચર પ્રકારો}
\vspace{-10pt}
\begin{longtable}[]{@{}lll@{}}
\toprule\noalign{}
\textbf{પ્રકાર} & \textbf{સંગઠન} & \textbf{Access} \\
\midrule\noalign{}
\endhead
\bottomrule\noalign{}
\endlastfoot
\textbf{Sequential} & Records ક્રમમાં & Sequential જ \\
\textbf{Direct/Random} & Records key દ્વારા & Direct access \\
\textbf{Indexed} & Index records ને point કરે છે & Key-based access \\
\textbf{Hierarchical} & Tree structure & Path-based \\
\end{longtable}
}

\end{solutionbox}
\begin{mnemonicbox}
``Sequential Direct Indexed Hierarchical''

\end{mnemonicbox}
\begin{center}\rule{0.5\linewidth}{0.5pt}\end{center}

\subsection*{પ્રશ્ન 3(b OR) [4
માર્ક્સ]}\label{uxaaauxab0uxab6uxaa8-3b-or-4-uxaaeuxab0uxa95uxab8}

\textbf{વિવિધ ફાઇલ લક્ષણો સમજાવો.}

\begin{solutionbox}


{\def\LTcaptype{none} % do not increment counter
\vspace{-5pt}
\captionof{table}{ફાઇલ લક્ષણો}
\vspace{-10pt}
\begin{longtable}[]{@{}lll@{}}
\toprule\noalign{}
\textbf{લક્ષણ} & \textbf{વિવરણ} & \textbf{ઉદાહરણ} \\
\midrule\noalign{}
\endhead
\bottomrule\noalign{}
\endlastfoot
\textbf{Name} & ફાઇલ identifier & document.txt \\
\textbf{Type} & ફાઇલ format & .txt, .exe \\
\textbf{Size} & ફાઇલ length bytes માં & 1024 bytes \\
\textbf{Location} & Physical storage address & Block 150 \\
\textbf{Permissions} & Access rights & rwx-rwx-rwx \\
\textbf{Timestamps} & Creation, modification dates & 2023-01-16 \\
\end{longtable}
}

\end{solutionbox}
\begin{mnemonicbox}
``Name Type Size Location Permissions Time''

\end{mnemonicbox}
\begin{center}\rule{0.5\linewidth}{0.5pt}\end{center}

\subsection*{પ્રશ્ન 3(c OR) [7
માર્ક્સ]}\label{uxaaauxab0uxab6uxaa8-3c-or-7-uxaaeuxab0uxa95uxab8}

\textbf{વિવિધ ફાઈલ ફાળવણી પદ્ધતિઓની યાદી બનાવો અને જરૂરી રેખાકૃતિ સાથે લિંક કરેલ
ફાળવણી સમજાવો.}

\begin{solutionbox}

\textbf{ફાઇલ ફાળવણી પદ્ધતિઓ:} અગાઉના જવાબ સમાન.

\textbf{લિંક્ડ ફાળવણી:}

\textbf{આકૃતિ:}

\begin{verbatim}
File A: Block1  Block5  Block9  NULL
File B: Block2  Block7  NULL  
File C: Block3  Block4  Block8  NULL
\end{verbatim}


{\def\LTcaptype{none} % do not increment counter
\vspace{-5pt}
\captionof{table}{લિંક્ડ ફાળવણી}
\vspace{-10pt}
\begin{longtable}[]{@{}ll@{}}
\toprule\noalign{}
\textbf{પાસું} & \textbf{વિવરણ} \\
\midrule\noalign{}
\endhead
\bottomrule\noalign{}
\endlastfoot
\textbf{Storage} & Files linked blocks માં store થાય છે \\
\textbf{Pointers} & દરેક block આગળના block નું pointer સમાવે છે \\
\textbf{ફાયદા} & External fragmentation નથી, dynamic size \\
\textbf{ગેરફાયદા} & Sequential access જ, pointer overhead \\
\end{longtable}
}

\textbf{Directory Entry}: (Start block pointer)

\end{solutionbox}
\begin{mnemonicbox}
``Links Lead Logical Locations''

\end{mnemonicbox}
\begin{center}\rule{0.5\linewidth}{0.5pt}\end{center}

\subsection*{પ્રશ્ન 4(a) [3
માર્ક્સ]}\label{q4a}

\textbf{પ્રોગ્રામ ધમકીઓ વ્યાખ્યાયિત કરો અને તેના પ્રકારો સમજાવો.}

\begin{solutionbox}

\textbf{વ્યાખ્યા}: પ્રોગ્રામ ધમકીઓ એ malicious programs છે જે system security
અને integrity ને સાપે છે.


{\def\LTcaptype{none} % do not increment counter
\vspace{-5pt}
\captionof{table}{પ્રોગ્રામ ધમકીના પ્રકારો}
\vspace{-10pt}
\begin{longtable}[]{@{}ll@{}}
\toprule\noalign{}
\textbf{પ્રકાર} & \textbf{વિવરણ} \\
\midrule\noalign{}
\endhead
\bottomrule\noalign{}
\endlastfoot
\textbf{Trojan Horse} & Legitimate program માં છુપાયેલો malicious code \\
\textbf{Virus} & અન્ય programs ને infect કરતો self-replicating code \\
\textbf{Worm} & Networks વચ્ચે replicate થતો standalone program \\
\textbf{Logic Bomb} & Specific conditions દ્વારા trigger થતો code \\
\end{longtable}
}

\end{solutionbox}
\begin{mnemonicbox}
``Trojans Viruses Worms Logic-bombs''

\end{mnemonicbox}
\begin{center}\rule{0.5\linewidth}{0.5pt}\end{center}

\subsection*{પ્રશ્ન 4(b) [4
માર્ક્સ]}\label{q4b}

\textbf{સિસ્ટમ ઓથેન્ટિકેશન સમજાવો.}

\begin{solutionbox}

\textbf{વ્યાખ્યા}: System access આપતાં પહેલાં વપરાશકર્તાની identity ને verify
કરવાની પ્રક્રિયા.


{\def\LTcaptype{none} % do not increment counter
\vspace{-5pt}
\captionof{table}{ઓથેન્ટિકેશન પદ્ધતિઓ}
\vspace{-10pt}
\begin{longtable}[]{@{}lll@{}}
\toprule\noalign{}
\textbf{પદ્ધતિ} & \textbf{વિવરણ} & \textbf{ઉદાહરણ} \\
\midrule\noalign{}
\endhead
\bottomrule\noalign{}
\endlastfoot
\textbf{Password} & Secret text string & username/password \\
\textbf{Biometric} & Physical characteristics & Fingerprint, retina \\
\textbf{Token} & Physical device & Smart card, USB key \\
\textbf{Multi-factor} & પદ્ધતિઓનું combination & Password + OTP \\
\end{longtable}
}

\textbf{ઓથેન્ટિકેશન પ્રક્રિયા:}

\begin{itemize}
\tightlist
\item
  \textbf{Identification}: વપરાશકર્તા identity claim કરે છે
\item
  \textbf{Verification}: System claim ને validate કરે છે
\item
  \textbf{Authorization}: Access rights આપવામાં આવે છે
\end{itemize}

\end{solutionbox}
\begin{mnemonicbox}
``Passwords Biometrics Tokens Multi-factor''

\end{mnemonicbox}
\begin{center}\rule{0.5\linewidth}{0.5pt}\end{center}

\subsection*{પ્રશ્ન 4(c) [7
માર્ક્સ]}\label{q4c}

\textbf{એક્સેસ કંટ્રોલ લિસ્ટને વિગતવાર સમજાવો.}

\begin{solutionbox}

\textbf{વ્યાખ્યા}: ACL દરેક user/group માટે system resources પર permissions
specify કરે છે.


{\def\LTcaptype{none} % do not increment counter
\vspace{-5pt}
\captionof{table}{ACL કમ્પોનન્ટ્સ}
\vspace{-10pt}
\begin{longtable}[]{@{}lll@{}}
\toprule\noalign{}
\textbf{કમ્પોનન્ટ} & \textbf{હેતુ} & \textbf{ઉદાહરણ} \\
\midrule\noalign{}
\endhead
\bottomrule\noalign{}
\endlastfoot
\textbf{Subject} & User અથવા group & john, admin\_group \\
\textbf{Object} & Resource & file.txt, directory \\
\textbf{Permission} & Allowed operations & read, write, execute \\
\textbf{Action} & Allow અથવા deny & permit, deny \\
\end{longtable}
}

\textbf{ACL સ્ટ્રક્ચર:}

\begin{verbatim}
User: john    File: /etc/passwd    Permission: read    Action: allow
Group: users  File: /tmp/*        Permission: write   Action: allow
User: guest   File: /etc/*        Permission: write   Action: deny
\end{verbatim}

\textbf{ફાયદા:}

\begin{itemize}
\tightlist
\item
  \textbf{Granular Control}: Fine-grained permissions
\item
  \textbf{Flexibility}: Per-resource access control
\item
  \textbf{Scalability}: જટિલ organizations ને handle કરે છે
\end{itemize}

\end{solutionbox}
\begin{mnemonicbox}
``Access Controls Limit Users''

\end{mnemonicbox}
\begin{center}\rule{0.5\linewidth}{0.5pt}\end{center}

\subsection*{પ્રશ્ન 4(a OR) [3
માર્ક્સ]}\label{uxaaauxab0uxab6uxaa8-4a-or-3-uxaaeuxab0uxa95uxab8}

\textbf{સિસ્ટમ ધમકીઓ વ્યાખ્યાયિત કરો અને તેના પ્રકારો સમજાવો.}

\begin{solutionbox}

\textbf{વ્યાખ્યા}: સિસ્ટમ ધમકીઓ operating system components અને system
integrity ને target કરે છે.


{\def\LTcaptype{none} % do not increment counter
\vspace{-5pt}
\captionof{table}{સિસ્ટમ ધમકીના પ્રકારો}
\vspace{-10pt}
\begin{longtable}[]{@{}ll@{}}
\toprule\noalign{}
\textbf{પ્રકાર} & \textbf{વિવરણ} \\
\midrule\noalign{}
\endhead
\bottomrule\noalign{}
\endlastfoot
\textbf{Denial of Service} & System resources ને overwhelm કરવા \\
\textbf{Privilege Escalation} & Unauthorized higher privileges મેળવવા \\
\textbf{Buffer Overflow} & Memory management flaws ને exploit કરવા \\
\textbf{Rootkit} & Detection થી malicious activities ને છુપાવવા \\
\end{longtable}
}

\end{solutionbox}
\begin{mnemonicbox}
``Denial Privilege Buffer Rootkit''

\end{mnemonicbox}
\begin{center}\rule{0.5\linewidth}{0.5pt}\end{center}

\subsection*{પ્રશ્ન 4(b OR) [4
માર્ક્સ]}\label{uxaaauxab0uxab6uxaa8-4b-or-4-uxaaeuxab0uxa95uxab8}

\textbf{OS માં રક્ષણની જરૂરિયાતો અને લક્ષ્યોની ચર્ચા કરો.}

\begin{solutionbox}


{\def\LTcaptype{none} % do not increment counter
\vspace{-5pt}
\captionof{table}{રક્ષણની જરૂરિયાતો અને લક્ષ્યો}
\vspace{-10pt}
\begin{longtable}[]{@{}lll@{}}
\toprule\noalign{}
\textbf{જરૂરિયાત} & \textbf{લક્ષ્ય} & \textbf{Implementation} \\
\midrule\noalign{}
\endhead
\bottomrule\noalign{}
\endlastfoot
\textbf{Confidentiality} & Unauthorized access અટકાવવા & Access
controls \\
\textbf{Integrity} & Data accuracy જાળવવા & Checksums, validation \\
\textbf{Availability} & Resource access ensure કરવા & Redundancy,
backup \\
\textbf{Authentication} & User identity verify કરવા & Login
mechanisms \\
\end{longtable}
}

\textbf{રક્ષણ પદ્ધતિઓ:}

\begin{itemize}
\tightlist
\item
  \textbf{Access Control}: Resource access ને limit કરવા
\item
  \textbf{Capability Lists}: User permissions define કરવા
\item
  \textbf{Security Domains}: Processes ને isolate કરવા
\end{itemize}

\end{solutionbox}
\begin{mnemonicbox}
``Confidentiality Integrity Availability
Authentication''

\end{mnemonicbox}
\begin{center}\rule{0.5\linewidth}{0.5pt}\end{center}

\subsection*{પ્રશ્ન 4(c OR) [7
માર્ક્સ]}\label{uxaaauxab0uxab6uxaa8-4c-or-7-uxaaeuxab0uxa95uxab8}

\textbf{વિવિધ ઓપરેટિંગ સિસ્ટમ સુરક્ષા નીતિઓ અને પ્રક્રિયાઓની ચર્ચા કરો.}

\begin{solutionbox}


{\def\LTcaptype{none} % do not increment counter
\vspace{-5pt}
\captionof{table}{સુરક્ષા નીતિઓ અને પ્રક્રિયાઓ}
\vspace{-10pt}
\begin{longtable}[]{@{}
  >{\raggedright\arraybackslash}p{(\linewidth - 4\tabcolsep) * \real{0.4146}}
  >{\raggedright\arraybackslash}p{(\linewidth - 4\tabcolsep) * \real{0.2683}}
  >{\raggedright\arraybackslash}p{(\linewidth - 4\tabcolsep) * \real{0.3171}}@{}}
\toprule\noalign{}
\begin{minipage}[b]{\linewidth}\raggedright
\textbf{નીતિ પ્રકાર}
\end{minipage} & \begin{minipage}[b]{\linewidth}\raggedright
\textbf{વિવરણ}
\end{minipage} & \begin{minipage}[b]{\linewidth}\raggedright
\textbf{પ્રક્રિયા}
\end{minipage} \\
\midrule\noalign{}
\endhead
\bottomrule\noalign{}
\endlastfoot
\textbf{Access Control} & User permissions define કરવા & Regular audit,
role-based access \\
\textbf{Password Policy} & Password requirements & Complexity rules,
expiration \\
\textbf{Backup Policy} & Data protection strategy & Regular backups,
testing \\
\textbf{Incident Response} & Security breach handling & Detection,
containment, recovery \\
\end{longtable}
}

\textbf{સુરક્ષા પ્રક્રિયાઓ:}

\begin{itemize}
\tightlist
\item
  \textbf{Regular Updates}: Patch management
\item
  \textbf{Monitoring}: Log analysis, intrusion detection\\
\item
  \textbf{Training}: User security awareness
\item
  \textbf{Audit}: Compliance checking
\end{itemize}

\end{solutionbox}
\begin{mnemonicbox}
``Access Password Backup Incident''

\end{mnemonicbox}
\begin{center}\rule{0.5\linewidth}{0.5pt}\end{center}

\subsection*{પ્રશ્ન 5(a) [3
માર્ક્સ]}\label{q5a}

\textbf{નીચેના આદેશો સમજાવો: (i) pwd (ii) cd (iii) comm}

\begin{solutionbox}


{\def\LTcaptype{none} % do not increment counter
\vspace{-5pt}
\captionof{table}{Linux Commands}
\vspace{-10pt}
\begin{longtable}[]{@{}lll@{}}
\toprule\noalign{}
\textbf{Command} & \textbf{હેતુ} & \textbf{ઉદાહરણ} \\
\midrule\noalign{}
\endhead
\bottomrule\noalign{}
\endlastfoot
\textbf{pwd} & Present working directory print કરવા & pwd \rightarrow
/home/user \\
\textbf{cd} & Directory change કરવા & cd /tmp \\
\textbf{comm} & Sorted files ને compare કરવા & comm file1.txt
file2.txt \\
\end{longtable}
}

\begin{itemize}
\tightlist
\item
  \textbf{pwd}: હાલની directory નો path દર્શાવે છે
\item
  \textbf{cd}: Directories વચ્ચે navigate કરવા
\item
  \textbf{comm}: Files વચ્ચે common અને unique lines દર્શાવે છે
\end{itemize}

\end{solutionbox}
\begin{mnemonicbox}
``Print Working Directory, Change Directory, Compare
Common''

\end{mnemonicbox}
\begin{center}\rule{0.5\linewidth}{0.5pt}\end{center}

\subsection*{પ્રશ્ન 5(b) [4
માર્ક્સ]}\label{q5b}

\textbf{ત્રીજી ફાઇલમાં બે ફાઇલોના સમાવિષ્ટોને જોડવા માટે શેલ સ્ક્રિપ્ટ લખો.}

\begin{solutionbox}

\textbf{શેલ સ્ક્રિપ્ટ:}

\begin{verbatim}
\#!/bin/bash
\# બે files ને ત્રીજી file માં concatenate કરવા માટે script

echo "પ્રથમ file નું નામ દાખલ કરો:"
read file1
echo "બીજી file નું નામ દાખલ કરો:" 
read file2
echo "Output file નું નામ દાખલ કરો:"
read file3

\# Input files exist છે કે કેમ ચેક કરો
if [ {-f} "$file1" ] \&\& [ {-f} "$file2" ]; then
    cat "$file1" "$file2" {} "$file3"
    echo "Files સફળતાપૂર્વક $file3 માં concatenate થઈ"
else
    echo "Error: Input files મળી નથી"
fi
\end{verbatim}

\end{solutionbox}
\begin{mnemonicbox}
``Cat Combines Content Correctly''

\end{mnemonicbox}
\begin{center}\rule{0.5\linewidth}{0.5pt}\end{center}

\subsection*{પ્રશ્ન 5(c) [7
માર્ક્સ]}\label{q5c}

\textbf{આપેલ 5 અંકની સંખ્યામાં તમામ વ્યક્તિગત અંકોનો સરવાળો શોધવા માટે શેલ સ્ક્રિપ્ટ
લખો.}

\begin{solutionbox}

\textbf{શેલ સ્ક્રિપ્ટ:}

\begin{verbatim}
\#!/bin/bash
\# 5{-અંકની સંખ્યામાં digits નો sum શોધવા માટે script}

echo "5{-અંકની સંખ્યા દાખલ કરો:"}
read number

\# Input validate કરો
if [ $\{\#number\} {-ne} 5 ]; then
    echo "Error: કૃપા કરીને બરાબર 5 digits દાખલ કરો"
    exit 1
fi

sum=0
temp=$number

\# દરેક digit extract કરો અને sum કરો
while [ $temp {-gt} 0 ]; do
    digit=$(($temp \% 10))
    sum=$(($sum + $digit))
    temp=$(($temp / 10))
done

echo "$number માં digits નો sum છે: $sum"
\end{verbatim}

\textbf{અલ્ગોરિધમ:}

\begin{itemize}
\tightlist
\item
  \textbf{Input Validation}: 5-digit number માટે ચેક કરો
\item
  \textbf{Digit Extraction}: Modulo operation વાપરો
\item
  \textbf{Sum Calculation}: દરેક digit ઉમેરો
\item
  \textbf{Display Result}: અંતિમ sum બતાવો
\end{itemize}

\end{solutionbox}
\begin{mnemonicbox}
``Sum Separates Single Symbols''

\end{mnemonicbox}
\begin{center}\rule{0.5\linewidth}{0.5pt}\end{center}

\subsection*{પ્રશ્ન 5(a OR) [3
માર્ક્સ]}\label{uxaaauxab0uxab6uxaa8-5a-or-3-uxaaeuxab0uxa95uxab8}

\textbf{નીચેના આદેશો સમજાવો: (i) man (ii) mkdir (iii) grep}

\begin{solutionbox}


{\def\LTcaptype{none} % do not increment counter
\vspace{-5pt}
\captionof{table}{Linux Commands}
\vspace{-10pt}
\begin{longtable}[]{@{}lll@{}}
\toprule\noalign{}
\textbf{Command} & \textbf{હેતુ} & \textbf{ઉદાહરણ} \\
\midrule\noalign{}
\endhead
\bottomrule\noalign{}
\endlastfoot
\textbf{man} & Manual pages દર્શાવવા & man ls \\
\textbf{mkdir} & Directories બનાવવા & mkdir newdir \\
\textbf{grep} & Text patterns શોધવા & grep ``hello'' file.txt \\
\end{longtable}
}

\begin{itemize}
\tightlist
\item
  \textbf{man}: Commands માટે documentation પ્રદાન કરે છે
\item
  \textbf{mkdir}: Specified names સાથે નવી directories બનાવે છે
\item
  \textbf{grep}: Regular expressions વાપરીને files માં patterns શોધે છે
\end{itemize}

\end{solutionbox}
\begin{mnemonicbox}
``Manual Make Directories, Grep Examines Patterns''

\end{mnemonicbox}
\begin{center}\rule{0.5\linewidth}{0.5pt}\end{center}

\subsection*{પ્રશ્ન 5(b OR) [4
માર્ક્સ]}\label{uxaaauxab0uxab6uxaa8-5b-or-4-uxaaeuxab0uxa95uxab8}

\textbf{ફિબોનાચી શ્રેણી જનરેટ કરવા અને પ્રદર્શિત કરવા માટે શેલ સ્ક્રિપ્ટ લખો.}

\begin{solutionbox}

\textbf{શેલ સ્ક્રિપ્ટ:}

\begin{verbatim}
\#!/bin/bash
\# ફિબોનાચી શ્રેણી generate કરવા માટે script

echo "Terms ની સંખ્યા દાખલ કરો:"
read n

\# Input validate કરો
if [ $n {-le} 0 ]; then
    echo "Error: કૃપા કરીને positive number દાખલ કરો"
    exit 1
fi

\# પ્રથમ બે terms initialize કરો
a=0
b=1

echo "ફિબોનાચી શ્રેણી:"
echo {-n} "$a "

if [ $n {-gt} 1 ]; then
    echo {-n} "$b "
fi

\# બાકીના terms generate કરો
for ((i=3; i{=}n; i++)); do
    c=$(($a + $b))
    echo {-n} "$c "
    a=$b
    b=$c
done
echo
\end{verbatim}

\end{solutionbox}
\begin{mnemonicbox}
``Fibonacci Follows Forward Formula''

\end{mnemonicbox}
\begin{center}\rule{0.5\linewidth}{0.5pt}\end{center}

\subsection*{પ્રશ્ન 5(c OR) [7
માર્ક્સ]}\label{uxaaauxab0uxab6uxaa8-5c-or-7-uxaaeuxab0uxa95uxab8}

\textbf{આપેલ string palindrome છે કે કેમ તે નિર્ધારિત કરવા માટે શેલ સ્ક્રિપ્ટ લખો.}

\begin{solutionbox}

\textbf{શેલ સ્ક્રિપ્ટ:}

\begin{verbatim}
\#!/bin/bash
\# String palindrome છે કે કેમ ચેક કરવા માટે script

echo "String દાખલ કરો:"
read string

\# Lowercase માં convert કરો અને spaces દૂર કરો
clean\_string=$(echo "$string" | tr {[:upper:]} {[:lower:]} | tr {-d} { })

\# String length મેળવો
length=$\{\#clean\_string\}

\# Flag initialize કરો
is\_palindrome=true

\# Palindrome ચેક કરો
for ((i=0; i{}length/2; i++)); do
    if [ "$\{clean\_string:$i:1\}" != "$\{clean\_string:$((length{-}1{-}i)):1\}" ]; then
        is\_palindrome=false
        break
    fi
done

\# પરિણામ દર્શાવો
if [ "$is\_palindrome" = true ]; then
    echo "{}$string{ palindrome છે"}
else
    echo "{}$string{ palindrome નથી"}
fi
\end{verbatim}

\textbf{અલ્ગોરિધમ:}

\begin{itemize}
\tightlist
\item
  \textbf{String Cleaning}: Lowercase માં convert કરો, spaces દૂર કરો
\item
  \textbf{Character Comparison}: બન્ને છેડાથી characters ને compare કરો
\item
  \textbf{Palindrome Check}: બધી comparisons match થાય છે કે કેમ verify કરો
\end{itemize}

\end{solutionbox}
\begin{mnemonicbox}
``Palindromes Proceed Perfectly Parallel''

\end{mnemonicbox}

\end{document}
