\documentclass[10pt,a4paper]{article}

% content/resources/templates/preamble.tex
\usepackage[margin=0.6in]{geometry}
\author{Milav Dabgar}
\usepackage{amsmath,amssymb,amsthm}
\usepackage{booktabs}
\usepackage{multirow}
\usepackage{xcolor}
\usepackage{tcolorbox}
\tcbuselibrary{breakable,skins}
\usepackage[colorlinks=true,linkcolor=blue]{hyperref}
\usepackage{titlesec}
\usepackage{enumitem}
\usepackage{tikz}
\usepackage{pgfplots}
\usepackage{circuitikz}
\usepackage[version=4]{mhchem}
\usepackage{longtable}
\usepackage{array}
\usepackage{float}
\usepackage{caption}
\usepackage{listings}

\lstset{
  basicstyle=\small\ttfamily,
  breaklines=true,
  breakatwhitespace=false,
  postbreak=\mbox{\textcolor{red}{$\hookrightarrow$}\space},
  float=false,
  numbers=left,
  numberstyle=\tiny\color{gray},
  numbersep=10pt,
  xleftmargin=2em,
  keywordstyle=\color{blue},
  commentstyle=\color{green!60!black},
  stringstyle=\color{purple},
  backgroundcolor=\color{gray!5},
  showstringspaces=false,
  tabsize=2,
  captionpos=b,
  keepspaces=true,
  columns=flexible
}

\pgfplotsset{compat=1.18}
\usetikzlibrary{shapes,arrows,positioning,calc,patterns,decorations.pathmorphing,decorations.markings,arrows.meta}

% Color scheme
\definecolor{headcolor}{RGB}{0,102,204}
\definecolor{keycolor}{RGB}{220,20,60}
\definecolor{solutioncolor}{RGB}{34,139,34}
\definecolor{mnemoniccolor}{RGB}{148,0,211}
\definecolor{codecolor}{RGB}{0,0,100}

% Spacing
\setlength{\parskip}{3pt}
\setlist[itemize]{nosep}
\setlist[enumerate]{nosep}

% Title formatting
\titleformat{\section}{\Large\bfseries\color{headcolor}}{\thesection}{1em}{}
\titleformat{\subsection}{\large\bfseries\color{headcolor}}{\thesubsection}{1em}{}

% Pandoc tightlist compatibility
\providecommand{\tightlist}{%
  \setlength{\itemsep}{0pt}\setlength{\parskip}{0pt}}

% Pandoc longtable compatibility
\newcounter{none}
\def\thenone{}


% content/resources/templates/english-boxes.tex
% This file is currently empty - it exists to maintain consistency with the import structure.
% Add custom environments here if needed in the future.


\begin{document}

\begin{center}
{\Huge\bfseries\color{headcolor} Subject Name Solutions}\\[5pt]
{\LARGE 4331602 -- Summer 2024}\\[3pt]
{\large Semester 1 Study Material}\\[3pt]
{\normalsize\textit{Detailed Solutions and Explanations}}
\end{center}

\vspace{10pt}

\subsection*{Question 1(a) [3 marks]}\label{q1a}

\textbf{Define Operating System and give its goal.}

\begin{solutionbox}

\textbf{Operating System Definition}: A program that acts as an
interface between computer hardware and users, managing system resources
and controlling program execution.

\textbf{Goals of Operating System}:

{\def\LTcaptype{none} % do not increment counter
\begin{longtable}[]{@{}
  >{\raggedright\arraybackslash}p{(\linewidth - 2\tabcolsep) * \real{0.3158}}
  >{\raggedright\arraybackslash}p{(\linewidth - 2\tabcolsep) * \real{0.6842}}@{}}
\toprule\noalign{}
\begin{minipage}[b]{\linewidth}\raggedright
Goal
\end{minipage} & \begin{minipage}[b]{\linewidth}\raggedright
Description
\end{minipage} \\
\midrule\noalign{}
\endhead
\bottomrule\noalign{}
\endlastfoot
\textbf{Resource Management} & Efficiently allocate CPU, memory, I/O
devices \\
\textbf{User Convenience} & Provide easy-to-use interface \\
\textbf{System Protection} & Secure system from unauthorized access \\
\end{longtable}
}

\end{solutionbox}
\begin{mnemonicbox}
``RUS'' - Resource management, User convenience,
System protection

\end{mnemonicbox}
\begin{center}\rule{0.5\linewidth}{0.5pt}\end{center}

\subsection*{Question 1(b) [4 marks]}\label{q1b}

\textbf{Give name Components of Computer System \& Explain need of
Operating system.}

\begin{solutionbox}

\textbf{Computer System Components}:

\begin{center}
\textbf{Mermaid Diagram (Code)}
\begin{verbatim}
{Shaded}
{Highlighting}[]
graph TD
    A[Computer System] {-{-}{} B[Hardware]}
    A {-{-}{} C[Operating System]}
    A {-{-}{} D[Application Programs]}
    A {-{-}{} E[Users]}
    
    B {-{-}{} F[CPU]}
    B {-{-}{} G[Memory]}
    B {-{-}{} H[I/O Devices]}
{Highlighting}
{Shaded}
\end{verbatim}
\end{center}

\textbf{Need of Operating System}:

\begin{itemize}
\tightlist
\item
  \textbf{Resource Manager}: Controls hardware allocation
\item
  \textbf{Interface Provider}: Easy communication between user and
  hardware
\item
  \textbf{Security}: Protects system from threats
\item
  \textbf{Error Handling}: Manages system errors efficiently
\end{itemize}

\end{solutionbox}
\begin{mnemonicbox}
``RISE'' - Resource management, Interface, Security,
Error handling

\end{mnemonicbox}
\begin{center}\rule{0.5\linewidth}{0.5pt}\end{center}

\subsection*{Question 1(c) [7 marks]}\label{q1c}

\textbf{Explain below types of Operating system.}

\begin{solutionbox}

\textbf{I. Batch Operating System}

{\def\LTcaptype{none} % do not increment counter
\begin{longtable}[]{@{}ll@{}}
\toprule\noalign{}
Feature & Description \\
\midrule\noalign{}
\endhead
\bottomrule\noalign{}
\endlastfoot
\textbf{Processing} & Jobs processed in batches without user
interaction \\
\textbf{Efficiency} & High throughput, low user interaction \\
\textbf{Example} & IBM mainframes \\
\end{longtable}
}

\textbf{II. Multiprogramming Operating System}

{\def\LTcaptype{none} % do not increment counter
\begin{longtable}[]{@{}ll@{}}
\toprule\noalign{}
Feature & Description \\
\midrule\noalign{}
\endhead
\bottomrule\noalign{}
\endlastfoot
\textbf{Concept} & Multiple programs in memory simultaneously \\
\textbf{CPU Usage} & Better CPU utilization \\
\textbf{Advantage} & Reduced idle time \\
\end{longtable}
}

\textbf{III. Time Sharing Operating System}

{\def\LTcaptype{none} % do not increment counter
\begin{longtable}[]{@{}ll@{}}
\toprule\noalign{}
Feature & Description \\
\midrule\noalign{}
\endhead
\bottomrule\noalign{}
\endlastfoot
\textbf{Time Slices} & CPU time divided among users \\
\textbf{Response} & Quick response time \\
\textbf{Example} & Unix, Linux \\
\end{longtable}
}

\end{solutionbox}
\begin{mnemonicbox}
``BMT'' - Batch (no interaction), Multiprogramming
(many programs), Time-sharing (time slices)

\end{mnemonicbox}
\begin{center}\rule{0.5\linewidth}{0.5pt}\end{center}

\subsection*{Question 1(c) OR [7
marks]}\label{q1c}

\textbf{Explain Linux Architecture \& characteristics with its
components.}

\begin{solutionbox}

\textbf{Linux Architecture}:

\begin{center}
\textbf{Mermaid Diagram (Code)}
\begin{verbatim}
{Shaded}
{Highlighting}[]
graph TD
    A[User Applications] {-{-}{} B[System Libraries]}
    B {-{-}{} C[System Call Interface]}
    C {-{-}{} D[Linux Kernel]}
    D {-{-}{} E[Hardware]}
    
    D {-{-}{} F[Process Management]}
    D {-{-}{} G[Memory Management]}
    D {-{-}{} H[File System]}
    D {-{-}{} I[Device Drivers]}
{Highlighting}
{Shaded}
\end{verbatim}
\end{center}

\textbf{Linux Characteristics}:

{\def\LTcaptype{none} % do not increment counter
\begin{longtable}[]{@{}ll@{}}
\toprule\noalign{}
Characteristic & Description \\
\midrule\noalign{}
\endhead
\bottomrule\noalign{}
\endlastfoot
\textbf{Open Source} & Free and modifiable \\
\textbf{Multiuser} & Multiple users simultaneously \\
\textbf{Multitasking} & Multiple processes concurrently \\
\textbf{Portable} & Runs on various hardware \\
\end{longtable}
}

\textbf{Components}:

\begin{itemize}
\tightlist
\item
  \textbf{Kernel}: Core of operating system
\item
  \textbf{Shell}: Command interpreter
\item
  \textbf{File System}: Organizes data storage
\end{itemize}

\end{solutionbox}
\begin{mnemonicbox}
``COMP'' - Core (kernel), Open source, Multiuser,
Portable

\end{mnemonicbox}
\begin{center}\rule{0.5\linewidth}{0.5pt}\end{center}

\subsection*{Question 2(a) [3 marks]}\label{q2a}

\textbf{Describe Process Control Block. And define (1) PID (2) stack
pointer (3) program counter}

\begin{solutionbox}

\textbf{Process Control Block (PCB)}: Data structure containing process
information for OS management.

\textbf{Definitions}:

{\def\LTcaptype{none} % do not increment counter
\begin{longtable}[]{@{}ll@{}}
\toprule\noalign{}
Term & Definition \\
\midrule\noalign{}
\endhead
\bottomrule\noalign{}
\endlastfoot
\textbf{PID} & Process Identifier - unique number for each process \\
\textbf{Stack Pointer} & Points to top of process stack \\
\textbf{Program Counter} & Contains address of next instruction \\
\end{longtable}
}

\end{solutionbox}
\begin{mnemonicbox}
``PSP'' - PID (identifier), Stack pointer (top),
Program counter (next)

\end{mnemonicbox}
\begin{center}\rule{0.5\linewidth}{0.5pt}\end{center}

\subsection*{Question 2(b) [4 marks]}\label{q2b}

\textbf{Describe the Process Model and Process states}

\begin{solutionbox}

\textbf{Process Model}: Conceptual representation of how processes are
managed by OS.

\textbf{Process States}:

\begin{verbatim}
stateDiagram{-v2}
  direction LR
    [*] {-{-} New}
    New {-{-} Ready}
    Ready {-{-} Running}
    Running {-{-} Waiting}
    Running {-{-} Ready}
    Waiting {-{-} Ready}
    Running {-{-} Terminated}
    Terminated {-{-} [*]}
\end{verbatim}

{\def\LTcaptype{none} % do not increment counter
\begin{longtable}[]{@{}ll@{}}
\toprule\noalign{}
State & Description \\
\midrule\noalign{}
\endhead
\bottomrule\noalign{}
\endlastfoot
\textbf{New} & Process being created \\
\textbf{Ready} & Waiting for CPU \\
\textbf{Running} & Executing instructions \\
\textbf{Waiting} & Waiting for I/O \\
\textbf{Terminated} & Process finished \\
\end{longtable}
}

\end{solutionbox}
\begin{mnemonicbox}
``NRRWT'' - New, Ready, Running, Waiting, Terminated

\end{mnemonicbox}
\begin{center}\rule{0.5\linewidth}{0.5pt}\end{center}

\subsection*{Question 2(c) [7 marks]}\label{q2c}

\textbf{Demonstrate Scheduling Algorithm:(I) First Come First Serve,
(II) Shortest Job First}

\begin{solutionbox}

\textbf{I. First Come First Serve (FCFS)}

{\def\LTcaptype{none} % do not increment counter
\begin{longtable}[]{@{}
  >{\raggedright\arraybackslash}p{(\linewidth - 8\tabcolsep) * \real{0.1304}}
  >{\raggedright\arraybackslash}p{(\linewidth - 8\tabcolsep) * \real{0.2029}}
  >{\raggedright\arraybackslash}p{(\linewidth - 8\tabcolsep) * \real{0.1739}}
  >{\raggedright\arraybackslash}p{(\linewidth - 8\tabcolsep) * \real{0.2464}}
  >{\raggedright\arraybackslash}p{(\linewidth - 8\tabcolsep) * \real{0.2464}}@{}}
\toprule\noalign{}
\begin{minipage}[b]{\linewidth}\raggedright
Process
\end{minipage} & \begin{minipage}[b]{\linewidth}\raggedright
Arrival Time
\end{minipage} & \begin{minipage}[b]{\linewidth}\raggedright
Burst Time
\end{minipage} & \begin{minipage}[b]{\linewidth}\raggedright
Completion Time
\end{minipage} & \begin{minipage}[b]{\linewidth}\raggedright
Turnaround Time
\end{minipage} \\
\midrule\noalign{}
\endhead
\bottomrule\noalign{}
\endlastfoot
P1 & 0 & 4 & 4 & 4 \\
P2 & 1 & 3 & 7 & 6 \\
P3 & 2 & 2 & 9 & 7 \\
\end{longtable}
}

\textbf{Average Turnaround Time} = (4+6+7)/3 = 5.67

\textbf{II. Shortest Job First (SJF)}

{\def\LTcaptype{none} % do not increment counter
\begin{longtable}[]{@{}
  >{\raggedright\arraybackslash}p{(\linewidth - 8\tabcolsep) * \real{0.1304}}
  >{\raggedright\arraybackslash}p{(\linewidth - 8\tabcolsep) * \real{0.2029}}
  >{\raggedright\arraybackslash}p{(\linewidth - 8\tabcolsep) * \real{0.1739}}
  >{\raggedright\arraybackslash}p{(\linewidth - 8\tabcolsep) * \real{0.2464}}
  >{\raggedright\arraybackslash}p{(\linewidth - 8\tabcolsep) * \real{0.2464}}@{}}
\toprule\noalign{}
\begin{minipage}[b]{\linewidth}\raggedright
Process
\end{minipage} & \begin{minipage}[b]{\linewidth}\raggedright
Arrival Time
\end{minipage} & \begin{minipage}[b]{\linewidth}\raggedright
Burst Time
\end{minipage} & \begin{minipage}[b]{\linewidth}\raggedright
Completion Time
\end{minipage} & \begin{minipage}[b]{\linewidth}\raggedright
Turnaround Time
\end{minipage} \\
\midrule\noalign{}
\endhead
\bottomrule\noalign{}
\endlastfoot
P3 & 2 & 2 & 4 & 2 \\
P2 & 1 & 3 & 7 & 6 \\
P1 & 0 & 4 & 11 & 11 \\
\end{longtable}
}

\textbf{Average Turnaround Time} = (2+6+11)/3 = 6.33

\end{solutionbox}
\begin{mnemonicbox}
``FS'' - FCFS (First order), SJF (Shortest first)

\end{mnemonicbox}
\begin{center}\rule{0.5\linewidth}{0.5pt}\end{center}

\subsection*{Question 2(a) OR [3
marks]}\label{q2a}

\textbf{Define Race condition, Mutual Exclusion}

\begin{solutionbox}

{\def\LTcaptype{none} % do not increment counter
\begin{longtable}[]{@{}
  >{\raggedright\arraybackslash}p{(\linewidth - 2\tabcolsep) * \real{0.3333}}
  >{\raggedright\arraybackslash}p{(\linewidth - 2\tabcolsep) * \real{0.6667}}@{}}
\toprule\noalign{}
\begin{minipage}[b]{\linewidth}\raggedright
Term
\end{minipage} & \begin{minipage}[b]{\linewidth}\raggedright
Definition
\end{minipage} \\
\midrule\noalign{}
\endhead
\bottomrule\noalign{}
\endlastfoot
\textbf{Race Condition} & Multiple processes access shared data
simultaneously causing inconsistent results \\
\textbf{Mutual Exclusion} & Only one process can access critical section
at a time \\
\end{longtable}
}

\textbf{Example}: Two processes updating same bank account balance.

\end{solutionbox}
\begin{mnemonicbox}
``RM'' - Race (simultaneous access), Mutual (one at a
time)

\end{mnemonicbox}
\begin{center}\rule{0.5\linewidth}{0.5pt}\end{center}

\subsection*{Question 2(b) OR [4
marks]}\label{q2b}

\textbf{Define all Throughput, Turnaround Time, Waiting Time, Response
Time}

\begin{solutionbox}

{\def\LTcaptype{none} % do not increment counter
\begin{longtable}[]{@{}ll@{}}
\toprule\noalign{}
Term & Definition \\
\midrule\noalign{}
\endhead
\bottomrule\noalign{}
\endlastfoot
\textbf{Throughput} & Number of processes completed per unit time \\
\textbf{Turnaround Time} & Total time from submission to completion \\
\textbf{Waiting Time} & Time spent waiting in ready queue \\
\textbf{Response Time} & Time from submission to first response \\
\end{longtable}
}

\textbf{Formula Table}:

{\def\LTcaptype{none} % do not increment counter
\begin{longtable}[]{@{}ll@{}}
\toprule\noalign{}
Metric & Formula \\
\midrule\noalign{}
\endhead
\bottomrule\noalign{}
\endlastfoot
Turnaround Time & Completion Time - Arrival Time \\
Waiting Time & Turnaround Time - Burst Time \\
Response Time & First CPU Time - Arrival Time \\
\end{longtable}
}

\end{solutionbox}
\begin{mnemonicbox}
``TTWR'' - Throughput, Turnaround, Waiting, Response

\end{mnemonicbox}
\begin{center}\rule{0.5\linewidth}{0.5pt}\end{center}

\subsection*{Question 2(c) OR [7
marks]}\label{q2c}

\textbf{Explain Round Robin Algorithm with example.}

\begin{solutionbox}

\textbf{Round Robin}: Each process gets equal CPU time slice (quantum).

\textbf{Example} (Time Quantum = 2):

{\def\LTcaptype{none} % do not increment counter
\begin{longtable}[]{@{}ll@{}}
\toprule\noalign{}
Process & Burst Time \\
\midrule\noalign{}
\endhead
\bottomrule\noalign{}
\endlastfoot
P1 & 5 \\
P2 & 3 \\
P3 & 4 \\
\end{longtable}
}

\textbf{Execution Timeline}:

\begin{verbatim}
0{-{-}{-}{-}2{-}{-}{-}{-}4{-}{-}{-}{-}6{-}{-}{-}{-}8{-}{-}{-}{-}10{-}{-}{-}12}
 P1   P2   P3   P1   P3   P1
\end{verbatim}

{\def\LTcaptype{none} % do not increment counter
\begin{longtable}[]{@{}lll@{}}
\toprule\noalign{}
Process & Completion Time & Turnaround Time \\
\midrule\noalign{}
\endhead
\bottomrule\noalign{}
\endlastfoot
P1 & 12 & 12 \\
P2 & 6 & 6 \\
P3 & 10 & 10 \\
\end{longtable}
}

\textbf{Average Turnaround Time} = (12+6+10)/3 = 9.33

\textbf{Advantages}:

\begin{itemize}
\tightlist
\item
  \textbf{Fair}: Equal time to all processes
\item
  \textbf{Responsive}: Good for interactive systems
\end{itemize}

\end{solutionbox}
\begin{mnemonicbox}
``RR-FE'' - Round Robin gives Fair and Equal time

\end{mnemonicbox}
\begin{center}\rule{0.5\linewidth}{0.5pt}\end{center}

\subsection*{Question 3(a) [3 marks]}\label{q3a}

\textbf{Give File Access Methods type}

\begin{solutionbox}

{\def\LTcaptype{none} % do not increment counter
\begin{longtable}[]{@{}ll@{}}
\toprule\noalign{}
Access Method & Description \\
\midrule\noalign{}
\endhead
\bottomrule\noalign{}
\endlastfoot
\textbf{Sequential} & Read/write in order from beginning \\
\textbf{Direct} & Access any record directly \\
\textbf{Indexed} & Use index to locate records \\
\end{longtable}
}

\end{solutionbox}
\begin{mnemonicbox}
``SDI'' - Sequential (order), Direct (any), Indexed
(index)

\end{mnemonicbox}
\begin{center}\rule{0.5\linewidth}{0.5pt}\end{center}

\subsection*{Question 3(b) [4 marks]}\label{q3b}

\textbf{Give Deadlock characteristics and Describe: Deadlock Prevention,
Deadlock Avoidance}

\begin{solutionbox}

\textbf{Deadlock Characteristics}:

{\def\LTcaptype{none} % do not increment counter
\begin{longtable}[]{@{}ll@{}}
\toprule\noalign{}
Condition & Description \\
\midrule\noalign{}
\endhead
\bottomrule\noalign{}
\endlastfoot
\textbf{Mutual Exclusion} & Resources cannot be shared \\
\textbf{Hold and Wait} & Process holds resource while waiting \\
\textbf{No Preemption} & Resources cannot be forcibly taken \\
\textbf{Circular Wait} & Circular chain of waiting processes \\
\end{longtable}
}

\textbf{Deadlock Prevention}: Remove any one of four conditions.

\textbf{Deadlock Avoidance}: Use algorithms like Banker's algorithm to
avoid unsafe states.

\end{solutionbox}
\begin{mnemonicbox}
``MHNC'' - Mutual exclusion, Hold and wait, No
preemption, Circular wait

\end{mnemonicbox}
\begin{center}\rule{0.5\linewidth}{0.5pt}\end{center}

\subsection*{Question 3(c) [7 marks]}\label{q3c}

\textbf{Explain the File Allocation Methods Contiguous, linked, indexed}

\begin{solutionbox}

\textbf{File Allocation Methods}:

{\def\LTcaptype{none} % do not increment counter
\begin{longtable}[]{@{}
  >{\raggedright\arraybackslash}p{(\linewidth - 6\tabcolsep) * \real{0.1667}}
  >{\raggedright\arraybackslash}p{(\linewidth - 6\tabcolsep) * \real{0.2708}}
  >{\raggedright\arraybackslash}p{(\linewidth - 6\tabcolsep) * \real{0.2500}}
  >{\raggedright\arraybackslash}p{(\linewidth - 6\tabcolsep) * \real{0.3125}}@{}}
\toprule\noalign{}
\begin{minipage}[b]{\linewidth}\raggedright
Method
\end{minipage} & \begin{minipage}[b]{\linewidth}\raggedright
Description
\end{minipage} & \begin{minipage}[b]{\linewidth}\raggedright
Advantages
\end{minipage} & \begin{minipage}[b]{\linewidth}\raggedright
Disadvantages
\end{minipage} \\
\midrule\noalign{}
\endhead
\bottomrule\noalign{}
\endlastfoot
\textbf{Contiguous} & Sequential blocks & Fast access & External
fragmentation \\
\textbf{Linked} & Scattered blocks with pointers & No fragmentation &
Slow random access \\
\textbf{Indexed} & Index block contains addresses & Fast random access &
Extra overhead \\
\end{longtable}
}

\textbf{Contiguous Allocation}:

\begin{verbatim}
File A: [1][2][3][4][5]
\end{verbatim}

\textbf{Linked Allocation}:

\begin{verbatim}
File A: [1][7][3][9]
\end{verbatim}

\textbf{Indexed Allocation}:

\begin{verbatim}
Index Block: [1,3,7,9,12]
File blocks: [1][3][7][9][12]
\end{verbatim}

\end{solutionbox}
\begin{mnemonicbox}
``CLI'' - Contiguous (together), Linked (pointers),
Indexed (index block)

\end{mnemonicbox}
\begin{center}\rule{0.5\linewidth}{0.5pt}\end{center}

\subsection*{Question 3(a) OR [3
marks]}\label{q3a}

\textbf{Give knowledge Linux File System Structure}

\begin{solutionbox}

\textbf{Linux File System Hierarchy}:

\begin{verbatim}
/
├── bin/     (System binaries)
├── etc/     (Configuration files)
├── home/    (User directories)
├── var/     (Variable data)
├── usr/     (User programs)
└── tmp/     (Temporary files)
\end{verbatim}

{\def\LTcaptype{none} % do not increment counter
\begin{longtable}[]{@{}ll@{}}
\toprule\noalign{}
Directory & Purpose \\
\midrule\noalign{}
\endhead
\bottomrule\noalign{}
\endlastfoot
\textbf{/bin} & Essential system binaries \\
\textbf{/etc} & System configuration files \\
\textbf{/home} & User home directories \\
\end{longtable}
}

\end{solutionbox}
\begin{mnemonicbox}
``BEH'' - Bin (binaries), Etc (config), Home (users)

\end{mnemonicbox}
\begin{center}\rule{0.5\linewidth}{0.5pt}\end{center}

\subsection*{Question 3(b) OR [4
marks]}\label{q3b}

\textbf{Explain Critical Section and Semaphore with example.}

\begin{solutionbox}

\textbf{Critical Section}: Code segment accessing shared resources.

\textbf{Semaphore}: Synchronization tool using counter variable.

\textbf{Example}:

\begin{verbatim}
\# Binary Semaphore
wait(S):
  while S {}= 0 do nothing
  S = S {-} 1

signal(S):
  S = S + 1
\end{verbatim}

\textbf{Critical Section Structure}:

{\def\LTcaptype{none} % do not increment counter
\begin{longtable}[]{@{}ll@{}}
\toprule\noalign{}
Section & Description \\
\midrule\noalign{}
\endhead
\bottomrule\noalign{}
\endlastfoot
\textbf{Entry} & Request permission \\
\textbf{Critical} & Access shared resource \\
\textbf{Exit} & Release permission \\
\textbf{Remainder} & Other code \\
\end{longtable}
}

\end{solutionbox}
\begin{mnemonicbox}
``ECER'' - Entry, Critical, Exit, Remainder

\end{mnemonicbox}
\begin{center}\rule{0.5\linewidth}{0.5pt}\end{center}

\subsection*{Question 3(c) OR [7
marks]}\label{q3c}

\textbf{Define and explain Deadlock Avoidance, Deadlock Detection and
Recovery}

\begin{solutionbox}

\textbf{Deadlock Avoidance}:

\begin{itemize}
\tightlist
\item
  Use \textbf{Banker's Algorithm}
\item
  Check if resource allocation leads to safe state
\end{itemize}

\textbf{Deadlock Detection}:

\begin{itemize}
\tightlist
\item
  Periodically check for deadlock using \textbf{Wait-for Graph}
\end{itemize}

\textbf{Deadlock Recovery Methods}:

{\def\LTcaptype{none} % do not increment counter
\begin{longtable}[]{@{}ll@{}}
\toprule\noalign{}
Method & Description \\
\midrule\noalign{}
\endhead
\bottomrule\noalign{}
\endlastfoot
\textbf{Process Termination} & Kill deadlocked processes \\
\textbf{Resource Preemption} & Take resources from processes \\
\textbf{Rollback} & Return to previous safe state \\
\end{longtable}
}

\textbf{Banker's Algorithm Steps}:

\begin{enumerate}
\tightlist
\item
  Check if request \leq available resources
\item
  Simulate allocation
\item
  Check if safe state exists
\end{enumerate}

\textbf{Wait-for Graph}:

\begin{center}
\textbf{Mermaid Diagram (Code)}
\begin{verbatim}
{Shaded}
{Highlighting}[]
graph LR
    P1 {-{-}{} P2}
    P2 {-{-}{} P3}
    P3 {-{-}{} P1}
{Highlighting}
{Shaded}
\end{verbatim}
\end{center}

\end{solutionbox}
\begin{mnemonicbox}
``ADR-BWT'' - Avoidance (Banker's), Detection
(Wait-for), Recovery (Terminate)

\end{mnemonicbox}
\begin{center}\rule{0.5\linewidth}{0.5pt}\end{center}

\subsection*{Question 4(a) [3 marks]}\label{q4a}

\textbf{Why Need of file Protection explain?}

\begin{solutionbox}

\textbf{Need for File Protection}:

{\def\LTcaptype{none} % do not increment counter
\begin{longtable}[]{@{}ll@{}}
\toprule\noalign{}
Reason & Description \\
\midrule\noalign{}
\endhead
\bottomrule\noalign{}
\endlastfoot
\textbf{Privacy} & Protect personal data \\
\textbf{Security} & Prevent unauthorized access \\
\textbf{Integrity} & Maintain data consistency \\
\end{longtable}
}

\textbf{Protection Mechanisms}:

\begin{itemize}
\tightlist
\item
  \textbf{Access Control Lists (ACL)}
\item
  \textbf{File Permissions} (Read, Write, Execute)
\item
  \textbf{User Authentication}
\end{itemize}

\end{solutionbox}
\begin{mnemonicbox}
``PSI'' - Privacy, Security, Integrity

\end{mnemonicbox}
\begin{center}\rule{0.5\linewidth}{0.5pt}\end{center}

\subsection*{Question 4(b) [4 marks]}\label{q4b}

\textbf{Illustrate Program threats, System threats}

\begin{solutionbox}

\textbf{Program Threats}:

{\def\LTcaptype{none} % do not increment counter
\begin{longtable}[]{@{}ll@{}}
\toprule\noalign{}
Threat & Description \\
\midrule\noalign{}
\endhead
\bottomrule\noalign{}
\endlastfoot
\textbf{Virus} & Self-replicating malicious code \\
\textbf{Worm} & Network-spreading malware \\
\textbf{Trojan Horse} & Disguised malicious program \\
\end{longtable}
}

\textbf{System Threats}:

{\def\LTcaptype{none} % do not increment counter
\begin{longtable}[]{@{}ll@{}}
\toprule\noalign{}
Threat & Description \\
\midrule\noalign{}
\endhead
\bottomrule\noalign{}
\endlastfoot
\textbf{Denial of Service} & Overwhelm system resources \\
\textbf{Port Scanning} & Find vulnerable services \\
\textbf{Man-in-Middle} & Intercept communications \\
\end{longtable}
}

\textbf{Protection Methods}:

\begin{itemize}
\tightlist
\item
  \textbf{Antivirus Software}
\item
  \textbf{Firewalls}
\item
  \textbf{Regular Updates}
\end{itemize}

\end{solutionbox}
\begin{mnemonicbox}
``VWT-DPM'' - Virus, Worm, Trojan; DoS, Port scan,
Man-in-middle

\end{mnemonicbox}
\begin{center}\rule{0.5\linewidth}{0.5pt}\end{center}

\subsection*{Question 4(c) [7 marks]}\label{q4c}

\textbf{Briefly detailing Operating System security policies and
procedures}

\begin{solutionbox}

\textbf{Security Policies}:

{\def\LTcaptype{none} % do not increment counter
\begin{longtable}[]{@{}ll@{}}
\toprule\noalign{}
Policy Type & Description \\
\midrule\noalign{}
\endhead
\bottomrule\noalign{}
\endlastfoot
\textbf{Access Control} & Who can access what resources \\
\textbf{Authentication} & Verify user identity \\
\textbf{Authorization} & Determine user permissions \\
\textbf{Audit} & Monitor and log activities \\
\end{longtable}
}

\textbf{Security Procedures}:

\begin{verbatim}
flowchart LR
    A[User Login] {-{-} B[Authentication]}
    B {-{-} C[Authorization Check]}
    C {-{-} D[Resource Access]}
    D {-{-} E[Activity Logging]}
    E {-{-} F[Audit Review]}
\end{verbatim}

\textbf{Implementation Steps}:

\begin{enumerate}
\tightlist
\item
  \textbf{User Registration} and credential setup
\item
  \textbf{Multi-factor Authentication}
\item
  \textbf{Role-based Access Control}
\item
  \textbf{Regular Security Audits}
\end{enumerate}

\textbf{Common Security Measures}:

\begin{itemize}
\tightlist
\item
  \textbf{Password Policies}
\item
  \textbf{Encryption}
\item
  \textbf{Backup Procedures}
\item
  \textbf{Incident Response Plans}
\end{itemize}

\end{solutionbox}
\begin{mnemonicbox}
``AAAA'' - Access control, Authentication,
Authorization, Audit

\end{mnemonicbox}
\begin{center}\rule{0.5\linewidth}{0.5pt}\end{center}

\subsection*{Question 4(a) OR [3
marks]}\label{q4a}

\textbf{Give idea Authentication and Authorization.}

\begin{solutionbox}

{\def\LTcaptype{none} % do not increment counter
\begin{longtable}[]{@{}lll@{}}
\toprule\noalign{}
Term & Definition & Example \\
\midrule\noalign{}
\endhead
\bottomrule\noalign{}
\endlastfoot
\textbf{Authentication} & Verify user identity & Username/password \\
\textbf{Authorization} & Determine access rights & File permissions \\
\end{longtable}
}

\textbf{Authentication Methods}:

\begin{itemize}
\tightlist
\item
  \textbf{Password-based}
\item
  \textbf{Biometric}
\item
  \textbf{Token-based}
\end{itemize}

\end{solutionbox}
\begin{mnemonicbox}
``AA'' - Authentication (who you are), Authorization
(what you can do)

\end{mnemonicbox}
\begin{center}\rule{0.5\linewidth}{0.5pt}\end{center}

\subsection*{Question 4(b) OR [4
marks]}\label{q4b}

\textbf{Explain Operating System security policies and procedures}

\begin{solutionbox}

\textbf{Security Policies Framework}:

{\def\LTcaptype{none} % do not increment counter
\begin{longtable}[]{@{}ll@{}}
\toprule\noalign{}
Component & Purpose \\
\midrule\noalign{}
\endhead
\bottomrule\noalign{}
\endlastfoot
\textbf{User Management} & Control user accounts \\
\textbf{Data Protection} & Secure sensitive information \\
\textbf{Network Security} & Protect communications \\
\textbf{System Monitoring} & Detect threats \\
\end{longtable}
}

\textbf{Implementation Procedures}:

\begin{enumerate}
\tightlist
\item
  \textbf{Risk Assessment}
\item
  \textbf{Policy Development}
\item
  \textbf{Training Programs}
\item
  \textbf{Regular Reviews}
\end{enumerate}

\end{solutionbox}
\begin{mnemonicbox}
``UDNS'' - User management, Data protection, Network
security, System monitoring

\end{mnemonicbox}
\begin{center}\rule{0.5\linewidth}{0.5pt}\end{center}

\subsection*{Question 4(c) OR [7
marks]}\label{q4c}

\textbf{Detailing the Security measures in Operating System}

\begin{solutionbox}

\textbf{Comprehensive Security Measures}:

{\def\LTcaptype{none} % do not increment counter
\begin{longtable}[]{@{}ll@{}}
\toprule\noalign{}
Layer & Security Measures \\
\midrule\noalign{}
\endhead
\bottomrule\noalign{}
\endlastfoot
\textbf{Physical} & Server room access, biometric locks \\
\textbf{Network} & Firewalls, VPN, intrusion detection \\
\textbf{System} & Antivirus, patches, access controls \\
\textbf{Application} & Input validation, secure coding \\
\textbf{Data} & Encryption, backup, integrity checks \\
\end{longtable}
}

\textbf{Access Control Matrix}:

{\def\LTcaptype{none} % do not increment counter
\begin{longtable}[]{@{}llll@{}}
\toprule\noalign{}
User/Role & File A & File B & Printer \\
\midrule\noalign{}
\endhead
\bottomrule\noalign{}
\endlastfoot
Admin & RWX & RWX & RWX \\
User1 & RW- & R-- & -W- \\
Guest & R-- & --- & --- \\
\end{longtable}
}

\textbf{Security Implementation Timeline}:

\begin{verbatim}
gantt
    title Security Implementation
    dateFormat  YYYY{-MM{-}DD}
    section Phase 1
    Risk Assessment    :2024{-01{-}01, 30d}
    Policy Development :2024{-01{-}15, 45d}
    section Phase 2
    System Hardening   :2024{-02{-}01, 60d}
    Training Program   :2024{-02{-}15, 30d}
\end{verbatim}

\textbf{Monitoring Tools}:

\begin{itemize}
\tightlist
\item
  \textbf{Log Analysis}
\item
  \textbf{Intrusion Detection Systems}
\item
  \textbf{Vulnerability Scanners}
\end{itemize}

\end{solutionbox}
\begin{mnemonicbox}
``PNSAD'' - Physical, Network, System, Application,
Data security

\end{mnemonicbox}
\begin{center}\rule{0.5\linewidth}{0.5pt}\end{center}

\subsection*{Question 5(a) [3 marks]}\label{q5a}

\textbf{Give five Basic commands: calendar, date}

\begin{solutionbox}

\textbf{Basic Linux Commands}:

{\def\LTcaptype{none} % do not increment counter
\begin{longtable}[]{@{}lll@{}}
\toprule\noalign{}
Command & Function & Example \\
\midrule\noalign{}
\endhead
\bottomrule\noalign{}
\endlastfoot
\texttt{cal} & Display calendar & \texttt{cal\ 2024} \\
\texttt{date} & Show current date/time & \texttt{date\ +\%d/\%m/\%Y} \\
\texttt{who} & Show logged users & \texttt{who} \\
\texttt{pwd} & Print working directory & \texttt{pwd} \\
\texttt{clear} & Clear screen & \texttt{clear} \\
\end{longtable}
}

\textbf{Command Examples}:

\begin{verbatim}
\# Display calendar for specific month
cal 6 2024

\# Format date output
date "+\%A, \%B \%d, \%Y"
\end{verbatim}

\end{solutionbox}
\begin{mnemonicbox}
``CDWPC'' - Cal, Date, Who, Pwd, Clear

\end{mnemonicbox}
\begin{center}\rule{0.5\linewidth}{0.5pt}\end{center}

\subsection*{Question 5(b) [4 marks]}\label{q5b}

\textbf{Explain Linux File and Directory Commands: ls, cat, mkdir,
rmdir, pwd.}

\begin{solutionbox}

\textbf{File and Directory Commands}:

{\def\LTcaptype{none} % do not increment counter
\begin{longtable}[]{@{}llll@{}}
\toprule\noalign{}
Command & Function & Syntax & Example \\
\midrule\noalign{}
\endhead
\bottomrule\noalign{}
\endlastfoot
\texttt{ls} & List directory contents &
\texttt{ls\ [options]\ [path]} & \texttt{ls\ -la} \\
\texttt{cat} & Display file content & \texttt{cat\ filename} &
\texttt{cat\ file.txt} \\
\texttt{mkdir} & Create directory & \texttt{mkdir\ dirname} &
\texttt{mkdir\ newdir} \\
\texttt{rmdir} & Remove empty directory & \texttt{rmdir\ dirname} &
\texttt{rmdir\ olddir} \\
\texttt{pwd} & Print working directory & \texttt{pwd} & \texttt{pwd} \\
\end{longtable}
}

\textbf{Usage Examples}:

\begin{verbatim}
\# List files with details
ls {-l} /home/user

\# Create multiple directories
mkdir {-p} dir1/dir2/dir3

\# Display file with line numbers
cat {-n} document.txt
\end{verbatim}

\textbf{Common Options}:

\begin{itemize}
\tightlist
\item
  \texttt{ls\ -l}: Long format
\item
  \texttt{ls\ -a}: Show hidden files
\item
  \texttt{mkdir\ -p}: Create parent directories
\end{itemize}

\end{solutionbox}
\begin{mnemonicbox}
``LCMRP'' - List, Cat, Mkdir, Rmdir, Pwd

\end{mnemonicbox}
\begin{center}\rule{0.5\linewidth}{0.5pt}\end{center}

\subsection*{Question 5(c) [7 marks]}\label{q5c}

\textbf{Understand and apply control statements Write a shell script to
perform given operations: Write a shell script to find maximum number
among three numbers.}

\begin{solutionbox}

\textbf{Shell Script for Maximum of Three Numbers}:

\begin{verbatim}
\#!/bin/bash
\# Script to find maximum of three numbers

echo "Enter three numbers:"
read {-p} "First number: " num1
read {-p} "Second number: " num2
read {-p} "Third number: " num3

\# Method 1: Using if{-elif{-}else}
if [ $num1 {-ge} $num2 ] \&\& [ $num1 {-ge} $num3 ]; then
    max=$num1
elif [ $num2 {-ge} $num1 ] \&\& [ $num2 {-ge} $num3 ]; then
    max=$num2
else
    max=$num3
fi

echo "Maximum number is: $max"

\# Method 2: Using nested if
if [ $num1 {-gt} $num2 ]; then
    if [ $num1 {-gt} $num3 ]; then
        echo "Maximum: $num1"
    else
        echo "Maximum: $num3"
    fi
else
    if [ $num2 {-gt} $num3 ]; then
        echo "Maximum: $num2"
    else
        echo "Maximum: $num3"
    fi
fi
\end{verbatim}

\textbf{Control Statements Used}:

{\def\LTcaptype{none} % do not increment counter
\begin{longtable}[]{@{}ll@{}}
\toprule\noalign{}
Statement & Purpose \\
\midrule\noalign{}
\endhead
\bottomrule\noalign{}
\endlastfoot
\texttt{if-elif-else} & Multiple condition checking \\
\texttt{read} & User input \\
\texttt{echo} & Output display \\
Comparison operators & \texttt{-ge}, \texttt{-gt}, \texttt{-lt} \\
\end{longtable}
}

\textbf{Comparison Operators}:

\begin{itemize}
\tightlist
\item
  \texttt{-eq}: Equal to
\item
  \texttt{-ne}: Not equal to
\item
  \texttt{-gt}: Greater than
\item
  \texttt{-ge}: Greater than or equal to
\item
  \texttt{-lt}: Less than
\item
  \texttt{-le}: Less than or equal to
\end{itemize}

\end{solutionbox}
\begin{mnemonicbox}
``IER'' - If (condition), Echo (output), Read (input)

\end{mnemonicbox}
\begin{center}\rule{0.5\linewidth}{0.5pt}\end{center}

\subsection*{Question 5(a) OR [3
marks]}\label{q5a}

\textbf{What is Linux Process commands: top, ps, kill}

\begin{solutionbox}

\textbf{Linux Process Commands}:

{\def\LTcaptype{none} % do not increment counter
\begin{longtable}[]{@{}lll@{}}
\toprule\noalign{}
Command & Function & Usage \\
\midrule\noalign{}
\endhead
\bottomrule\noalign{}
\endlastfoot
\texttt{top} & Display running processes & \texttt{top} \\
\texttt{ps} & Show process status & \texttt{ps\ aux} \\
\texttt{kill} & Terminate process & \texttt{kill\ PID} \\
\end{longtable}
}

\textbf{Command Details}:

\textbf{top command}:

\begin{itemize}
\tightlist
\item
  Shows real-time process information
\item
  CPU and memory usage
\item
  Load average
\end{itemize}

\textbf{ps command options}:

\begin{itemize}
\tightlist
\item
  \texttt{ps\ aux}: All processes with details
\item
  \texttt{ps\ -ef}: Full format listing
\end{itemize}

\textbf{kill command}:

\begin{itemize}
\tightlist
\item
  \texttt{kill\ -9\ PID}: Force kill process
\item
  \texttt{killall\ process\_name}: Kill by name
\end{itemize}

\end{solutionbox}
\begin{mnemonicbox}
``TPK'' - Top (real-time), Ps (status), Kill
(terminate)

\end{mnemonicbox}
\begin{center}\rule{0.5\linewidth}{0.5pt}\end{center}

\subsection*{Question 5(b) OR [4
marks]}\label{q5b}

\textbf{Linux File and Directory Commands: rm, mv,split,diff, grep}

\begin{solutionbox}

\textbf{Advanced File Commands}:

{\def\LTcaptype{none} % do not increment counter
\begin{longtable}[]{@{}
  >{\raggedright\arraybackslash}p{(\linewidth - 6\tabcolsep) * \real{0.2500}}
  >{\raggedright\arraybackslash}p{(\linewidth - 6\tabcolsep) * \real{0.2778}}
  >{\raggedright\arraybackslash}p{(\linewidth - 6\tabcolsep) * \real{0.2222}}
  >{\raggedright\arraybackslash}p{(\linewidth - 6\tabcolsep) * \real{0.2500}}@{}}
\toprule\noalign{}
\begin{minipage}[b]{\linewidth}\raggedright
Command
\end{minipage} & \begin{minipage}[b]{\linewidth}\raggedright
Function
\end{minipage} & \begin{minipage}[b]{\linewidth}\raggedright
Syntax
\end{minipage} & \begin{minipage}[b]{\linewidth}\raggedright
Example
\end{minipage} \\
\midrule\noalign{}
\endhead
\bottomrule\noalign{}
\endlastfoot
\texttt{rm} & Remove files/directories &
\texttt{rm\ [options]\ file} & \texttt{rm\ -rf\ folder} \\
\texttt{mv} & Move/rename files & \texttt{mv\ source\ dest} &
\texttt{mv\ old.txt\ new.txt} \\
\texttt{split} & Split large files & \texttt{split\ -l\ lines\ file} &
\texttt{split\ -l\ 100\ data.txt} \\
\texttt{diff} & Compare files & \texttt{diff\ file1\ file2} &
\texttt{diff\ old.txt\ new.txt} \\
\texttt{grep} & Search text patterns & \texttt{grep\ pattern\ file} &
\texttt{grep\ "error"\ log.txt} \\
\end{longtable}
}

\textbf{Usage Examples}:

\begin{verbatim}
\# Remove directory recursively
rm {-rf} /tmp/oldfiles

\# Move and rename
mv /home/user/doc.txt /backup/document.txt

\# Split file into 50{-line chunks}
split {-l} 50 largefile.txt chunk\_

\# Find differences between files
diff {-u} original.txt modified.txt

\# Search for pattern in multiple files
grep {-r} "TODO" /project/src/
\end{verbatim}

\textbf{Common Options}:

\begin{itemize}
\tightlist
\item
  \texttt{rm\ -i}: Interactive mode
\item
  \texttt{mv\ -i}: Prompt before overwrite
\item
  \texttt{grep\ -i}: Case insensitive search
\end{itemize}

\end{solutionbox}
\begin{mnemonicbox}
``RMSDG'' - Remove, Move, Split, Diff, Grep

\end{mnemonicbox}
\begin{center}\rule{0.5\linewidth}{0.5pt}\end{center}

\subsection*{Question 5(c) OR [7
marks]}\label{q5c}

\textbf{Write a shell script to read five numbers from user and find
average of five numbers.}

\begin{solutionbox}

\textbf{Shell Script for Average of Five Numbers}:

\begin{verbatim}
\#!/bin/bash
\# Script to calculate average of five numbers

echo "=== Average Calculator ==="
echo "Enter five numbers:"

\# Read five numbers
read {-p} "Enter number 1: " num1
read {-p} "Enter number 2: " num2
read {-p} "Enter number 3: " num3
read {-p} "Enter number 4: " num4
read {-p} "Enter number 5: " num5

\# Calculate sum
sum=$((num1 + num2 + num3 + num4 + num5))

\# Calculate average
average=$((sum / 5))

\# Display results
echo "================================"
echo "Numbers entered: $num1, $num2, $num3, $num4, $num5"
echo "Sum: $sum"
echo "Average: $average"
echo "================================"

\# Enhanced version with decimal precision
sum\_float=$(echo "$num1 + $num2 + $num3 + $num4 + $num5" | bc)
avg\_float=$(echo "scale=2; $sum\_float / 5" | bc)
echo "Precise Average: $avg\_float"
\end{verbatim}

\textbf{Alternative Method using Arrays}:

\begin{verbatim}
\#!/bin/bash
\# Using array approach

declare {-a} numbers
sum=0

echo "Enter 5 numbers:"
for i in \{0..4\}; do
    read {-p} "Number $((i+1)): " numbers[i]
    sum=$((sum + numbers[i]))
done

average=$((sum / 5))

echo "Numbers: $\{numbers[@]\}"
echo "Sum: $sum"
echo "Average: $average"
\end{verbatim}

\textbf{Script Features}:

{\def\LTcaptype{none} % do not increment counter
\begin{longtable}[]{@{}ll@{}}
\toprule\noalign{}
Feature & Description \\
\midrule\noalign{}
\endhead
\bottomrule\noalign{}
\endlastfoot
\textbf{Input Validation} & Check for numeric input \\
\textbf{User-friendly Output} & Clear formatting \\
\textbf{Array Usage} & Store multiple values \\
\textbf{Arithmetic Operations} & Sum and division \\
\end{longtable}
}

\textbf{Mathematical Operations in Bash}:

\begin{itemize}
\tightlist
\item
  \texttt{\$((expression))}: Integer arithmetic
\item
  \texttt{bc}: Calculator for floating point
\item
  \texttt{expr}: Expression evaluation
\end{itemize}

\end{solutionbox}
\begin{mnemonicbox}
``RSAR'' - Read (input), Sum (add), Average (divide),
Result (output)

\end{mnemonicbox}

\end{document}
