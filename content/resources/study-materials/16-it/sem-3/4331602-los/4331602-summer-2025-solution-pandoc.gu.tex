\documentclass[10pt,a4paper]{article}

% content/resources/templates/preamble.tex
\usepackage[margin=0.6in]{geometry}
\author{Milav Dabgar}
\usepackage{amsmath,amssymb,amsthm}
\usepackage{booktabs}
\usepackage{multirow}
\usepackage{xcolor}
\usepackage{tcolorbox}
\tcbuselibrary{breakable,skins}
\usepackage[colorlinks=true,linkcolor=blue]{hyperref}
\usepackage{titlesec}
\usepackage{enumitem}
\usepackage{tikz}
\usepackage{pgfplots}
\usepackage{circuitikz}
\usepackage[version=4]{mhchem}
\usepackage{longtable}
\usepackage{array}
\usepackage{float}
\usepackage{caption}
\usepackage{listings}

\lstset{
  basicstyle=\small\ttfamily,
  breaklines=true,
  breakatwhitespace=false,
  postbreak=\mbox{\textcolor{red}{$\hookrightarrow$}\space},
  float=false,
  numbers=left,
  numberstyle=\tiny\color{gray},
  numbersep=10pt,
  xleftmargin=2em,
  keywordstyle=\color{blue},
  commentstyle=\color{green!60!black},
  stringstyle=\color{purple},
  backgroundcolor=\color{gray!5},
  showstringspaces=false,
  tabsize=2,
  captionpos=b,
  keepspaces=true,
  columns=flexible
}

\pgfplotsset{compat=1.18}
\usetikzlibrary{shapes,arrows,positioning,calc,patterns,decorations.pathmorphing,decorations.markings,arrows.meta}

% Color scheme
\definecolor{headcolor}{RGB}{0,102,204}
\definecolor{keycolor}{RGB}{220,20,60}
\definecolor{solutioncolor}{RGB}{34,139,34}
\definecolor{mnemoniccolor}{RGB}{148,0,211}
\definecolor{codecolor}{RGB}{0,0,100}

% Spacing
\setlength{\parskip}{3pt}
\setlist[itemize]{nosep}
\setlist[enumerate]{nosep}

% Title formatting
\titleformat{\section}{\Large\bfseries\color{headcolor}}{\thesection}{1em}{}
\titleformat{\subsection}{\large\bfseries\color{headcolor}}{\thesubsection}{1em}{}

% Pandoc tightlist compatibility
\providecommand{\tightlist}{%
  \setlength{\itemsep}{0pt}\setlength{\parskip}{0pt}}

% Pandoc longtable compatibility
\newcounter{none}
\def\thenone{}


% content/resources/templates/gujarati-boxes.tex
\usepackage{fontspec}
\usepackage{polyglossia}

% Set Gujarati as main language (document is primarily in Gujarati)
% Note: gloss-gujarati.ldf doesn't exist in polyglossia, but it will use hyphenation patterns
\setdefaultlanguage{gujarati}
\setotherlanguage{english}

% Configure Gujarati font properly
% Use Language=Default to prevent polyglossia from trying to add language-specific features
% that don't exist for Gujarati, which causes "empty feature" warnings
\newfontfamily\gujaratifont[Script=Gujarati,AutoFakeBold=2.5,AutoFakeSlant=0.3]{Noto Sans Gujarati}
\setmainfont[Script=Gujarati,AutoFakeBold=2.5,AutoFakeSlant=0.3]{Noto Sans Gujarati}
% Use Noto Sans Gujarati for monospace to support Gujarati in text
\setmonofont[Scale=0.9]{Noto Sans Gujarati}

% Configure English to use the same font
\newfontfamily\englishfont[Script=Gujarati,AutoFakeBold=2.5,AutoFakeSlant=0.3]{Noto Sans Gujarati}

% Translations for polyglossia
\gappto\captionsgujarati{
  \renewcommand{\tablename}{કોષ્ટક}
  \renewcommand{\figurename}{આકૃતિ}
}

% Helper for TikZ nodes to ensure Gujarati font
\newcommand{\gu}[1]{{\gujaratifont #1}}

% Custom environments
\newtcolorbox{solutionbox}{
    breakable,
    enhanced,
    colback=solutioncolor!5!white,
    colframe=solutioncolor!75!black,
    fonttitle=\bfseries,
    title=જવાબ
}

\newtcolorbox{solutionboxnobreak}{
 colback=solutioncolor!5!white,
 colframe=solutioncolor!75!black,
 fonttitle=\bfseries,
 title=જવાબ
}

\newtcolorbox{keyformula}{
 breakable,
 enhanced,
 colback=keycolor!5!white,
 colframe=keycolor!75!black,
 fonttitle=\bfseries,
 title=રાસાયણિક સમીકરણ/સૂત્ર
}

\newtcolorbox{mnemonicbox}{
 breakable,
 enhanced,
 colback=mnemoniccolor!5!white,
 colframe=mnemoniccolor!75!black,
 fonttitle=\bfseries,
 title=મેમરી ટ્રીક
}


\begin{document}

\begin{center}
{\Huge\bfseries\color{headcolor} Subject Name (Gujarati)}\\[5pt]
{\LARGE 4331602 -- Summer 2025}\\[3pt]
{\large Semester 1 Study Material}\\[3pt]
{\normalsize\textit{Detailed Solutions and Explanations}}
\end{center}

\vspace{10pt}

\subsection*{પ્રશ્ન 1(a) [3
marks]}\label{q1a}

\textbf{ઓપરેટિંગ સિસ્ટમ વ્યાખ્યાયિત કરો અને OS ની જરૂરિયાત સમજાવો.}

\begin{solutionbox}

\textbf{ઓપરેટિંગ સિસ્ટમ} એ સિસ્ટમ સોફ્ટવેર છે જે કમ્પ્યુટર હાર્ડવેર અને એપ્લિકેશન સોફ્ટવેર
વચ્ચે મધ્યસ્થી તરીકે કામ કરે છે. તે હાર્ડવેર રિસોર્સનું સંચાલન કરે છે અને યુઝર પ્રોગ્રામ્સને
સેવાઓ પ્રદાન કરે છે.

\textbf{ઓપરેટિંગ સિસ્ટમની જરૂરિયાત:}

\begin{itemize}
\tightlist
\item
  \textbf{રિસોર્સ મેનેજમેન્ટ}: CPU, મેમરી, સ્ટોરેજ અને I/O ડિવાઇસનું કાર્યક્ષમ સંચાલન
\item
  \textbf{યુઝર ઇન્ટરફેસ}: યુઝર ઇન્ટરેક્શન માટે કમાન્ડ-લાઇન અને ગ્રાફિકલ ઇન્ટરફેસ
  પ્રદાન કરે છે
\item
  \textbf{પ્રોગ્રામ એક્ઝિક્યુશન}: યુઝર પ્રોગ્રામ્સને સુરક્ષિત રીતે લોડ અને એક્ઝિક્યુટ કરે
  છે
\end{itemize}

\end{solutionbox}
\begin{mnemonicbox}
``RUP - રિસોર્સ, યુઝર, પ્રોગ્રામ મેનેજમેન્ટ''

\end{mnemonicbox}
\subsection*{પ્રશ્ન 1(b) [4
marks]}\label{q1b}

\textbf{પ્રક્રિયા નિયંત્રણ બ્લોક (PCB) પર એક ટૂંકી નોંધ લખો.}

\begin{solutionbox}

પ્રોસેસ કન્ટ્રોલ બ્લોક (PCB) એ ડેટા સ્ટ્રક્ચર છે જે ઓપરેટિંગ સિસ્ટમ દ્વારા દરેક ચાલતી
પ્રક્રિયા માટે જાળવવામાં આવે છે.

{\def\LTcaptype{none} % do not increment counter
\begin{longtable}[]{@{}ll@{}}
\toprule\noalign{}
PCB ઘટક & વર્ણન \\
\midrule\noalign{}
\endhead
\bottomrule\noalign{}
\endlastfoot
પ્રોસેસ ID & પ્રક્રિયા માટે અનન્ય ઓળખકર્તા \\
પ્રોસેસ સ્ટેટ & વર્તમાન સ્થિતિ (તૈયાર, ચાલુ, રાહ જોવી) \\
પ્રોગ્રામ કાઉન્ટર & એક્ઝિક્યુટ કરવાની આગળની instruction નું સરનામું \\
CPU રજિસ્ટર્સ & પ્રક્રિયા suspend થાય ત્યારે CPU રજિસ્ટર્સની કિંમતો \\
મેમરી મેનેજમેન્ટ & બેઝ અને લિમિટ રજિસ્ટર્સ, પેજ ટેબલ્સ \\
I/O સ્ટેટસ & ખુલ્લી ફાઇલો અને I/O ડિવાઇસની યાદી \\
\end{longtable}
}

\textbf{મુખ્ય કાર્યો:}

\begin{itemize}
\tightlist
\item
  \textbf{પ્રક્રિયા ઓળખ}: અનન્ય પ્રોસેસ ID અને પેરેન્ટ પ્રોસેસ ID સ્ટોર કરે છે
\item
  \textbf{સ્ટેટ ઇન્ફર્મેશન}: વર્તમાન એક્ઝિક્યુશન સ્ટેટ અને કન્ટેક્સ્ટ જાળવે છે
\item
  \textbf{રિસોર્સ એલોકેશન}: ફાળવેલ રિસોર્સ અને મેમરી ઉપયોગનું ટ્રેકિંગ કરે છે
\end{itemize}

\end{solutionbox}
\begin{mnemonicbox}
``PIS - Process ID, Information, State tracking''

\end{mnemonicbox}
\subsection*{પ્રશ્ન 1(c) [7
marks]}\label{q1c}

\textbf{વિવિધ પ્રકારની ઓપરેટિંગ સિસ્ટમોની યાદી બનાવો. બેચ ઓપરેટિંગ સિસ્ટમના
કાર્યને યોગ્ય ઉદાહરણ સાથે સમજાવો.}

\begin{solutionbox}

\textbf{ઓપરેટિંગ સિસ્ટમના પ્રકારો:}

{\def\LTcaptype{none} % do not increment counter
\begin{longtable}[]{@{}ll@{}}
\toprule\noalign{}
પ્રકાર & વર્ણન \\
\midrule\noalign{}
\endhead
\bottomrule\noalign{}
\endlastfoot
બેચ OS & સમાન જોબ્સને જૂથમાં મૂકીને એકસાથે એક્ઝિક્યુટ કરે છે \\
ટાઇમ-શેરિંગ OS & બહુવિધ વપરાશકર્તાઓ સિસ્ટમને એકસાથે શેર કરે છે \\
રીયલ-ટાઇમ OS & નિશ્ચિત રિસ્પોન્સ ટાઇમની ગેરંટી આપે છે \\
ડિસ્ટ્રિબ્યુટેડ OS & બહુવિધ કનેક્ટેડ કમ્પ્યુટર્સનું સંચાલન કરે છે \\
નેટવર્ક OS & નેટવર્ક સેવાઓ અને રિસોર્સ શેરિંગ પ્રદાન કરે છે \\
મોબાઇલ OS & મોબાઇલ ડિવાઇસ માટે ડિઝાઇન કરેલ \\
\end{longtable}
}

\textbf{બેચ ઓપરેટિંગ સિસ્ટમનું કાર્ય:}

\begin{verbatim}
+{-{-}{-}{-}{-}{-}{-}{-}{-}{-}+    +{-}{-}{-}{-}{-}{-}{-}{-}{-}{-}+    +{-}{-}{-}{-}{-}{-}{-}{-}{-}{-}+    +{-}{-}{-}{-}{-}{-}{-}{-}{-}{-}+}
|  Job 1   |    |  Job 2   |    |  Job 3   |    |  Job 4   |
| COBOL    | {- | FORTRAN  | {-} |   C++    | {-} |  JAVA    |}
| Programs |    | Programs |    | Programs |    | Programs |
+{-{-}{-}{-}{-}{-}{-}{-}{-}{-}+    +{-}{-}{-}{-}{-}{-}{-}{-}{-}{-}+    +{-}{-}{-}{-}{-}{-}{-}{-}{-}{-}+    +{-}{-}{-}{-}{-}{-}{-}{-}{-}{-}+}
     |               |               |               |
     v               v               v               v
+{-{-}{-}{-}{-}{-}{-}{-}{-}{-}{-}{-}{-}{-}{-}{-}{-}{-}{-}{-}{-}{-}{-}{-}{-}{-}{-}{-}{-}{-}{-}{-}{-}{-}{-}{-}{-}{-}{-}{-}{-}{-}{-}{-}{-}{-}{-}{-}{-}{-}{-}{-}{-}{-}{-}{-}+}
|              Batch Queue Manager                       |
+{-{-}{-}{-}{-}{-}{-}{-}{-}{-}{-}{-}{-}{-}{-}{-}{-}{-}{-}{-}{-}{-}{-}{-}{-}{-}{-}{-}{-}{-}{-}{-}{-}{-}{-}{-}{-}{-}{-}{-}{-}{-}{-}{-}{-}{-}{-}{-}{-}{-}{-}{-}{-}{-}{-}{-}+}
     |
     v
+{-{-}{-}{-}{-}{-}{-}{-}{-}{-}+    +{-}{-}{-}{-}{-}{-}{-}{-}{-}{-}+    +{-}{-}{-}{-}{-}{-}{-}{-}{-}{-}+}
|   CPU    |    |  Memory  |    |   I/O    |
| Executor | {- | Manager  | {-} | Handler  |}
+{-{-}{-}{-}{-}{-}{-}{-}{-}{-}+    +{-}{-}{-}{-}{-}{-}{-}{-}{-}{-}+    +{-}{-}{-}{-}{-}{-}{-}{-}{-}{-}+}
\end{verbatim}

\textbf{ઉદાહરણ}: બેંક ટ્રાન્ઝેક્શન પ્રોસેસિંગ જ્યાં દિવસભરના બધા ટ્રાન્ઝેક્શન્સ એકત્રિત
કરીને રાત્રે કાર્યક્ષમતા માટે એકસાથે પ્રોસેસ કરવામાં આવે છે.

\textbf{મુખ્ય લક્ષણો:}

\begin{itemize}
\tightlist
\item
  \textbf{જોબ ગ્રુપિંગ}: કાર્યક્ષમતા માટે સમાન જોબ્સ એકસાથે એક્ઝિક્યુટ કરવામાં આવે છે
\item
  \textbf{કોઈ યુઝર ઇન્ટરેક્શન નહીં}: એકવાર સબમિટ કર્યા પછી જોબ્સ યુઝર દખલ વિના
  ચાલે છે
\item
  \textbf{ઉચ્ચ થ્રુપુટ}: સિસ્ટમ ઉપયોગને મહત્તમ બનાવે છે
\end{itemize}

\end{solutionbox}
\begin{mnemonicbox}
``JNH - Jobs grouped, No interaction, High
throughput''

\end{mnemonicbox}
\subsection*{પ્રશ્ન 1(c) OR [7
marks]}\label{q1c}

\textbf{વિવિધ પ્રકારની ઓપરેટિંગ સિસ્ટમોની યાદી બનાવો. રીયલ ટાઇમ ઓપરેટિંગ
સિસ્ટમ્સ વિગતવાર સમજાવો.}

\begin{solutionbox}

\textbf{ઓપરેટિંગ સિસ્ટમના પ્રકારો:} (ઉપરની જેમ સમાન ટેબલ)

\textbf{રીયલ-ટાઇમ ઓપરેટિંગ સિસ્ટમ (RTOS):}

રીયલ-ટાઇમ OS એ નિર્દિષ્ટ સમય મર્યાદામાં ગેરંટીડ રિસ્પોન્સ પ્રદાન કરે છે જે મહત્વપૂર્ણ
એપ્લિકેશન્સ માટે જરૂરી છે.

\textbf{RTOS ના પ્રકારો:}

{\def\LTcaptype{none} % do not increment counter
\begin{longtable}[]{@{}lll@{}}
\toprule\noalign{}
પ્રકાર & ડેડલાઇન & ઉદાહરણ \\
\midrule\noalign{}
\endhead
\bottomrule\noalign{}
\endlastfoot
હાર્ડ રીયલ-ટાઇમ & ડેડલાઇન પૂરી કરવી જ જોઈએ & એર ટ્રાફિક કંટ્રોલ, પેસમેકર \\
સોફ્ટ રીયલ-ટાઇમ & થોડો વિલંબ સહન કરી શકે & વિડિયો સ્ટ્રીમિંગ, ઓનલાઇન ગેમિંગ \\
ફર્મ રીયલ-ટાઇમ & કભીકભાર ડેડલાઇન મિસ સ્વીકાર્ય & લાઇવ ઓડિયો પ્રોસેસિંગ \\
\end{longtable}
}

\textbf{લક્ષણો:}

\begin{itemize}
\tightlist
\item
  \textbf{નિર્ધારિત}: બધા ઓપરેશન માટે અનુમાનિત રિસ્પોન્સ ટાઇમ
\item
  \textbf{પ્રાયોરિટી-આધારિત શેડ્યુલિંગ}: ઉચ્ચ પ્રાયોરિટી ટાસ્કને તાત્કાલિક ધ્યાન
\item
  \textbf{ન્યૂનતમ ઇન્ટરપ્ટ લેટન્સી}: ઝડપી કન્ટેક્સ્ટ સ્વિચિંગ ક્ષમતાઓ
\item
  \textbf{મેમરી મેનેજમેન્ટ}: વિલંબ વિના રીયલ-ટાઇમ મેમરી એલોકેશન
\end{itemize}

\textbf{એપ્લિકેશન્સ:}

\begin{itemize}
\tightlist
\item
  મેડિકલ ડિવાઇસ, ઓટોમોટિવ સિસ્ટમ્સ, ઇન્ડસ્ટ્રિયલ ઓટોમેશન, એરોસ્પેસ કંટ્રોલ સિસ્ટમ્સ
\end{itemize}

\end{solutionbox}
\begin{mnemonicbox}
``DPMA - Deterministic, Priority-based, Minimal
latency, Applications critical''

\end{mnemonicbox}
\subsection*{પ્રશ્ન 2(a) [3
marks]}\label{q2a}

\textbf{પ્રોગ્રામ અને પ્રક્રિયા વચ્ચે તફાવત કરો.}

\begin{solutionbox}

{\def\LTcaptype{none} % do not increment counter
\begin{longtable}[]{@{}lll@{}}
\toprule\noalign{}
પાસું & પ્રોગ્રામ & પ્રક્રિયા \\
\midrule\noalign{}
\endhead
\bottomrule\noalign{}
\endlastfoot
વ્યાખ્યા & ડિસ્ક પર સંગ્રહિત સ્ટેટિક કોડ & એક્ઝિક્યુશનમાં પ્રોગ્રામ \\
સ્થિતિ & પેસિવ એન્ટિટી & એક્ટિવ એન્ટિટી \\
મેમરી & કોઈ મેમરી એલોકેશન નહીં & એલોકેટેડ મેમરી સ્પેસ \\
જીવનકાળ & ડિલીટ થાય ત્યાં સુધી કાયમી & એક્ઝિક્યુશન દરમિયાન અસ્થાયી \\
રિસોર્સ & કોઈ રિસોર્સ વપરાશ નહીં & CPU, મેમરી, I/O વપરાશ કરે છે \\
\end{longtable}
}

\textbf{મુખ્ય તફાવતો:}

\begin{itemize}
\tightlist
\item
  \textbf{સ્ટેટિક vs ડાયનેમિક}: પ્રોગ્રામ સ્ટેટિક ફાઇલ છે, પ્રક્રિયા ડાયનેમિક
  એક્ઝિક્યુશન છે
\item
  \textbf{રિસોર્સ ઉપયોગ}: પ્રક્રિયા સિસ્ટમ રિસોર્સનો વપરાશ કરે છે, પ્રોગ્રામ નહીં
\item
  \textbf{બહુવિધ ઇન્સ્ટન્સ}: એક પ્રોગ્રામ બહુવિધ પ્રક્રિયાઓ બનાવી શકે છે
\end{itemize}

\end{solutionbox}
\begin{mnemonicbox}
``SDR - Static vs Dynamic, Resource usage, Multiple
instances''

\end{mnemonicbox}
\subsection*{પ્રશ્ન 2(b) [4
marks]}\label{q2b}

\textbf{પ્રક્રિયા સ્થિતિ રેખાકૃતિની મદદથી પ્રક્રિયાની વિવિધ અવસ્થાઓ સમજાવો.}

\begin{solutionbox}

\begin{verbatim}
stateDiagram{-v2}
  direction LR
    [*] {-{-} New: પ્રોગ્રામ લોડ}
    New {-{-} Ready: તૈયાર ક્યુમાં દાખલ}
    Ready {-{-} Running: CPU ફાળવણી}
    Running {-{-} Ready: ટાઇમ ક્વાન્ટમ સમાપ્ત}
    Running {-{-} Waiting: I/O રિક્વેસ્ટ}
    Waiting {-{-} Ready: I/O પૂર્ણ}
    Running {-{-} Terminated: પ્રક્રિયા પૂર્ણ}
    Terminated {-{-} [*]: રિસોર્સ ડીએલોકેટ}
\end{verbatim}

\textbf{પ્રક્રિયા સ્થિતિઓ:}

{\def\LTcaptype{none} % do not increment counter
\begin{longtable}[]{@{}ll@{}}
\toprule\noalign{}
સ્થિતિ & વર્ણન \\
\midrule\noalign{}
\endhead
\bottomrule\noalign{}
\endlastfoot
New & પ્રક્રિયા બનાવવામાં આવી રહી છે \\
Ready & CPU એસાઇનમેન્ટની રાહ જોઈ રહી છે \\
Running & હાલમાં CPU પર એક્ઝિક્યુટ થઈ રહી છે \\
Waiting & I/O અથવા ઇવેન્ટ માટે બ્લોક થયેલ \\
Terminated & પ્રક્રિયાનું એક્ઝિક્યુશન પૂર્ણ થયું \\
\end{longtable}
}

\textbf{સ્થિતિ પરિવર્તનો:}

\begin{itemize}
\tightlist
\item
  \textbf{Ready થી Running}: પ્રોસેસ શેડ્યુલર CPU ફાળવે છે
\item
  \textbf{Running થી Ready}: ટાઇમ સ્લાઇસ સમાપ્ત અથવા ઉચ્ચ પ્રાયોરિટી પ્રોસેસ
  આવે છે
\item
  \textbf{Running થી Waiting}: પ્રક્રિયા I/O ઓપરેશન માંગે છે
\item
  \textbf{Waiting થી Ready}: I/O ઓપરેશન પૂર્ણ થાય છે
\end{itemize}

\end{solutionbox}
\begin{mnemonicbox}
``NRWRT - New, Ready, Waiting, Running, Terminated
states''

\end{mnemonicbox}
\subsection*{પ્રશ્ન 2(c) [7
marks]}\label{q2c}

\textbf{રાઉન્ડ રોબિન અલ્ગોરિધમનું વર્ણન કરો. આપેલ ડેટા માટે ગેન્ટ ચાર્ટ સાથે સરેરાશ
રાહ જોવાનો સમય અને સરેરાશ ટર્ન-અરાઉન્ડ સમયની ગણતરી કરો. કન્ટેક્સ્ટ સ્વિચ = 01 ms
અને ક્વાન્ટમ ટાઇમ = 04 ms ધ્યાનમાં લો.}

\begin{solutionbox}

\textbf{રાઉન્ડ રોબિન અલ્ગોરિધમ:} રાઉન્ડ રોબિન એ પ્રીએમ્પ્ટિવ શેડ્યુલિંગ અલ્ગોરિધમ છે
જ્યાં દરેક પ્રક્રિયાને ગોળાકાર રીતે સમાન CPU સમય (ક્વાન્ટમ) મળે છે.

\textbf{આપેલ ડેટા:}

\begin{itemize}
\tightlist
\item
  ક્વાન્ટમ ટાઇમ = 4 ms
\item
  કન્ટેક્સ્ટ સ્વિચ = 1 ms
\end{itemize}

{\def\LTcaptype{none} % do not increment counter
\begin{longtable}[]{@{}lll@{}}
\toprule\noalign{}
પ્રક્રિયા & આગમન સમય & બર્સ્ટ ટાઇમ \\
\midrule\noalign{}
\endhead
\bottomrule\noalign{}
\endlastfoot
P1 & 0 & 8 \\
P2 & 3 & 3 \\
P3 & 1 & 10 \\
P4 & 4 & 5 \\
\end{longtable}
}

\textbf{ગેન્ટ ચાર્ટ:}

\begin{verbatim}
0   4  5   8  9  13 14  18 19  22 23  26 27  29
|P1 |CS|P3|CS|P1|CS|P2|CS|P3|CS|P4|CS|P3|CS|P4|
\end{verbatim}

\textbf{ગણતરીઓ:}

{\def\LTcaptype{none} % do not increment counter
\begin{longtable}[]{@{}llll@{}}
\toprule\noalign{}
પ્રક્રિયા & કમ્પ્લીશન ટાઇમ & ટર્નઅરાઉન્ડ ટાઇમ & વેઇટિંગ ટાઇમ \\
\midrule\noalign{}
\endhead
\bottomrule\noalign{}
\endlastfoot
P1 & 13 & 13 & 5 \\
P2 & 18 & 15 & 12 \\
P3 & 26 & 25 & 15 \\
P4 & 29 & 25 & 20 \\
\end{longtable}
}

\textbf{સરેરાશ વેઇટિંગ ટાઇમ = (5 + 12 + 15 + 20) / 4 = 13 ms} \textbf{સરેરાશ
ટર્નઅરાઉન્ડ ટાઇમ = (13 + 15 + 25 + 25) / 4 = 19.5 ms}

\textbf{મુખ્ય લક્ષણો:}

\begin{itemize}
\tightlist
\item
  \textbf{ન્યાયી શેડ્યુલિંગ}: દરેક પ્રક્રિયાને સમાન CPU સમય મળે છે
\item
  \textbf{પ્રીએમ્પ્ટિવ}: ક્વાન્ટમ સમાપ્ત થયા પછી રનિંગ પ્રોસેસ ઇન્ટરપ્ટ થાય છે
\item
  \textbf{કન્ટેક્સ્ટ સ્વિચિંગ}: ગણતરીમાં ઓવરહેડ સામેલ છે
\end{itemize}

\end{solutionbox}
\begin{mnemonicbox}
``FPC - Fair, Preemptive, Context switching
overhead''

\end{mnemonicbox}
\subsection*{પ્રશ્ન 2(a) OR [3
marks]}\label{q2a}

\textbf{તફાવત કરો: CPU બાઉન્ડ પ્રક્રિયા v/s I/O બાઉન્ડ પ્રક્રિયા.}

\begin{solutionbox}

{\def\LTcaptype{none} % do not increment counter
\begin{longtable}[]{@{}lll@{}}
\toprule\noalign{}
પાસું & CPU બાઉન્ડ પ્રક્રિયા & I/O બાઉન્ડ પ્રક્રિયા \\
\midrule\noalign{}
\endhead
\bottomrule\noalign{}
\endlastfoot
પ્રાથમિક પ્રવૃત્તિ & સઘન ગણતરીઓ & વારંવાર I/O ઓપરેશન્સ \\
CPU ઉપયોગ & ઉચ્ચ CPU ઉપયોગ & નીચો CPU ઉપયોગ \\
બર્સ્ટ ટાઇમ & લાંબા CPU બર્સ્ટ્સ & ટૂંકા CPU બર્સ્ટ્સ \\
વેઇટિંગ ટાઇમ & ઓછી I/O રાહ & વધુ I/O રાહ \\
ઉદાહરણો & ગાણિતિક ગણતરીઓ, ઇમેજ પ્રોસેસિંગ & ફાઇલ ઓપરેશન્સ, ડેટાબેઝ ક્વેરીઝ \\
\end{longtable}
}

\textbf{મુખ્ય તફાવતો:}

\begin{itemize}
\tightlist
\item
  \textbf{રિસોર્સ વપરાશ}: CPU-બાઉન્ડ વધુ પ્રોસેસર વાપરે છે, I/O-બાઉન્ડ વધુ
  ઇનપુટ/આઉટપુટ વાપરે છે
\item
  \textbf{પર્ફોર્મન્સ ઇમ્પેક્ટ}: CPU-બાઉન્ડ પ્રોસેસર સ્પીડથી પ્રભાવિત, I/O-બાઉન્ડ
  સ્ટોરેજ સ્પીડથી પ્રભાવિત
\item
  \textbf{શેડ્યુલિંગ પ્રાયોરિટી}: વિવિધ અલ્ગોરિધમ્સ દરેક પ્રકારને અલગ રીતે પસંદ કરે છે
\end{itemize}

\end{solutionbox}
\begin{mnemonicbox}
``CIR - CPU intensive, I/O intensive, Resource usage
differs''

\end{mnemonicbox}
\subsection*{પ્રશ્ન 2(b) OR [4
marks]}\label{q2b}

\textbf{ડેડલોક શું છે? ડેડલોક થવા માટે જરૂરી શરતો સમજાવો.}

\begin{solutionbox}

\textbf{ડેડલોક} એ એવી પરિસ્થિતિ છે જ્યાં બે અથવા વધુ પ્રક્રિયાઓ કાયમી રૂપે બ્લોક થાય
છે, દરેક અન્ય દ્વારા રાખવામાં આવેલા રિસોર્સની રાહ જોતી હોય છે.

\textbf{જરૂરી શરતો (કોફમેન કંડિશન્સ):}

{\def\LTcaptype{none} % do not increment counter
\begin{longtable}[]{@{}ll@{}}
\toprule\noalign{}
શરત & વર્ણન \\
\midrule\noalign{}
\endhead
\bottomrule\noalign{}
\endlastfoot
મ્યુચ્યુઅલ એક્સક્લુઝન & રિસોર્સ એકસાથે શેર કરી શકાતા નથી \\
હોલ્ડ એન્ડ વેઇટ & પ્રક્રિયા રિસોર્સ પકડીને અન્યની રાહ જુએ છે \\
નો પ્રીએમ્પ્શન & રિસોર્સ બળજબરીથી પ્રક્રિયામાંથી લઈ શકાતા નથી \\
સર્ક્યુલર વેઇટ & રિસોર્સની રાહ જોતી પ્રક્રિયાઓની વર્તુળાકાર સાંકળ \\
\end{longtable}
}

\textbf{ઉદાહરણ પરિસ્થિતિ:}

\begin{verbatim}
Process A {-{-}{-}{-}holds{-}{-}{-}{-} Resource 1}
    |                        \^{}
    |                        |
    v                        |
waits for              Process B
Resource 2 {{-}{-}{-}holds{-}{-}{-}{-}{-}{-}{-}{-}{-}{-}|}
\end{verbatim}

\textbf{ડેડલોક અટકાવવાની પદ્ધતિઓ:}

\begin{itemize}
\tightlist
\item
  \textbf{મ્યુચ્યુઅલ એક્સક્લુઝન દૂર કરો}: શક્ય હોય ત્યારે રિસોર્સને શેરેબલ બનાવો
\item
  \textbf{હોલ્ડ એન્ડ વેઇટ અટકાવો}: બધા રિસોર્સ એકસાથે જ માંગો
\item
  \textbf{પ્રીએમ્પ્શનને મંજૂરી આપો}: જરૂર પડે ત્યારે બળજબરીથી રિસોર્સ લો
\item
  \textbf{સર્ક્યુલર વેઇટ અટકાવો}: રિસોર્સને ક્રમ આપો અને તે ક્રમમાં જ માંગો
\end{itemize}

\end{solutionbox}
\begin{mnemonicbox}
``MHNC - Mutual exclusion, Hold-wait, No preemption,
Circular wait''

\end{mnemonicbox}
\subsection*{પ્રશ્ન 2(c) OR [7
marks]}\label{q2c}

\textbf{FCFS અલ્ગોરિધમનું વર્ણન કરો. આપેલ ડેટા માટે ગેન્ટ ચાર્ટ સાથે સરેરાશ વેઇટિંગ
ટાઇમ અને સરેરાશ ટર્ન-અરાઉન્ડ ટાઇમની ગણતરી કરો.}

\begin{solutionbox}

\textbf{ફર્સ્ટ કમ ફર્સ્ટ સર્વ (FCFS) અલ્ગોરિધમ:} FCFS એ નોન-પ્રીએમ્પ્ટિવ શેડ્યુલિંગ
અલ્ગોરિધમ છે જ્યાં પ્રક્રિયાઓ આગમન ક્રમમાં એક્ઝિક્યુટ થાય છે.

\textbf{આપેલ ડેટા:}

{\def\LTcaptype{none} % do not increment counter
\begin{longtable}[]{@{}lll@{}}
\toprule\noalign{}
પ્રક્રિયા & આગમન સમય & બર્સ્ટ ટાઇમ \\
\midrule\noalign{}
\endhead
\bottomrule\noalign{}
\endlastfoot
P1 & 0 & 7 \\
P2 & 3 & 6 \\
P3 & 5 & 9 \\
P4 & 6 & 4 \\
\end{longtable}
}

\textbf{ગેન્ટ ચાર્ટ:}

\begin{verbatim}
0        7        13        22        26
|   P1   |   P2   |    P3   |   P4   |
\end{verbatim}

\textbf{ગણતરીઓ:}

{\def\LTcaptype{none} % do not increment counter
\begin{longtable}[]{@{}
  >{\raggedright\arraybackslash}p{(\linewidth - 8\tabcolsep) * \real{0.1370}}
  >{\raggedright\arraybackslash}p{(\linewidth - 8\tabcolsep) * \real{0.1781}}
  >{\raggedright\arraybackslash}p{(\linewidth - 8\tabcolsep) * \real{0.2329}}
  >{\raggedright\arraybackslash}p{(\linewidth - 8\tabcolsep) * \real{0.2466}}
  >{\raggedright\arraybackslash}p{(\linewidth - 8\tabcolsep) * \real{0.2055}}@{}}
\toprule\noalign{}
\begin{minipage}[b]{\linewidth}\raggedright
પ્રક્રિયા
\end{minipage} & \begin{minipage}[b]{\linewidth}\raggedright
સ્ટાર્ટ ટાઇમ
\end{minipage} & \begin{minipage}[b]{\linewidth}\raggedright
કમ્પ્લીશન ટાઇમ
\end{minipage} & \begin{minipage}[b]{\linewidth}\raggedright
ટર્નઅરાઉન્ડ ટાઇમ
\end{minipage} & \begin{minipage}[b]{\linewidth}\raggedright
વેઇટિંગ ટાઇમ
\end{minipage} \\
\midrule\noalign{}
\endhead
\bottomrule\noalign{}
\endlastfoot
P1 & 0 & 7 & 7 & 0 \\
P2 & 7 & 13 & 10 & 4 \\
P3 & 13 & 22 & 17 & 8 \\
P4 & 22 & 26 & 20 & 16 \\
\end{longtable}
}

\textbf{સરેરાશ વેઇટિંગ ટાઇમ = (0 + 4 + 8 + 16) / 4 = 7 ms} \textbf{સરેરાશ
ટર્નઅરાઉન્ડ ટાઇમ = (7 + 10 + 17 + 20) / 4 = 13.5 ms}

\textbf{લક્ષણો:}

\begin{itemize}
\tightlist
\item
  \textbf{સરળ અમલીકરણ}: સમજવામાં અને અમલ કરવામાં સરળ
\item
  \textbf{નોન-પ્રીએમ્પ્ટિવ}: એકવાર શરૂ થયા પછી, પ્રક્રિયા પૂર્ણ થવા સુધી ચાલે છે
\item
  \textbf{કોન્વોય ઇફેક્ટ}: ટૂંકી પ્રક્રિયાઓ લાંબી પ્રક્રિયાઓની રાહ જુએ છે
\end{itemize}

\end{solutionbox}
\begin{mnemonicbox}
``SNC - Simple, Non-preemptive, Convoy effect
possible''

\end{mnemonicbox}
\subsection*{પ્રશ્ન 3(a) [3
marks]}\label{q3a}

\textbf{સિંગલ-લેવલ ડિરેક્ટરી માળખું સમજાવો.}

\begin{solutionbox}

સિંગલ-લેવલ ડિરેક્ટરી સ્ટ્રક્ચર એ સૌથી સરળ ફાઇલ ઓર્ગનાઇઝેશન છે જ્યાં બધી ફાઇલો એક જ
ડિરેક્ટરીમાં સ્ટોર કરવામાં આવે છે.

\begin{verbatim}
        રૂટ ડિરેક્ટરી
    +{-{-}{-}{-}{-}{-}{-}{-}{-}{-}{-}{-}{-}{-}{-}{-}{-}{-}{-}+}
    | file1.txt         |
    | program.exe       |
    | data.dat          |
    | image.jpg         |
    | document.pdf      |
    +{-{-}{-}{-}{-}{-}{-}{-}{-}{-}{-}{-}{-}{-}{-}{-}{-}{-}{-}+}
\end{verbatim}

\textbf{લક્ષણો:}

\begin{itemize}
\tightlist
\item
  \textbf{સરળ માળખું}: બધી ફાઇલો એક જ સ્થાને
\item
  \textbf{અનન્ય નામો}: દરેક ફાઇલનું અનન્ય નામ હોવું જોઈએ
\item
  \textbf{કોઈ ઓર્ગનાઇઝેશન નહીં}: કોઈ ગ્રુપિંગ અથવા કેટેગરાઇઝેશન શક્ય નથી
\end{itemize}

\textbf{મર્યાદાઓ:}

\begin{itemize}
\tightlist
\item
  બહુવિધ યુઝર્સ સમાન નામે ફાઇલો બનાવે ત્યારે નામની અથડામણ
\item
  મોટી સંખ્યામાં ફાઇલોને ઓર્ગનાઇઝ કરવું મુશ્કેલ
\item
  યુઝર્સ વચ્ચે કોઈ પ્રાઇવસી અથવા એક્સેસ કંટ્રોલ નથી
\end{itemize}

\end{solutionbox}
\begin{mnemonicbox}
``SUN - Simple, Unique names, No organization''

\end{mnemonicbox}
\subsection*{પ્રશ્ન 3(b) [4
marks]}\label{q3b}

\textbf{વિવિધ ફાઇલ લક્ષણો સમજાવો.}

\begin{solutionbox}

ફાઇલ એટ્રિબ્યુટ્સ એ મેટાડેટા છે જે ફાઇલ સિસ્ટમમાં સ્ટોર કરેલી ફાઇલો વિશે માહિતી પ્રદાન
કરે છે.

{\def\LTcaptype{none} % do not increment counter
\begin{longtable}[]{@{}ll@{}}
\toprule\noalign{}
એટ્રિબ્યુટ & વર્ણન \\
\midrule\noalign{}
\endhead
\bottomrule\noalign{}
\endlastfoot
નામ & માનવ-વાંચી શકાય તેવું ફાઇલ ઓળખકર્તા \\
પ્રકાર & ફાઇલ ફોર્મેટ (એક્ઝિક્યુટેબલ, ટેક્સ્ટ, ઇમેજ) \\
કદ & વર્તમાન ફાઇલ કદ બાઇટ્સમાં \\
સ્થાન & સ્ટોરેજ ડિવાઇસ પર ભૌતિક સરનામું \\
પ્રોટેક્શન & એક્સેસ પરમિશન્સ (રીડ, રાઇટ, એક્ઝિક્યુટ) \\
ટાઇમ સ્ટેમ્પ્સ & બનાવટ, સુધારા, એક્સેસ સમય \\
માલિક & ફાઇલ બનાવનાર યુઝર \\
\end{longtable}
}

\textbf{સામાન્ય ફાઇલ એટ્રિબ્યુટ્સ:}

\begin{itemize}
\tightlist
\item
  \textbf{ઓળખકર્તા}: ફાઇલ સિસ્ટમ રેફરન્સ માટે અનન્ય નંબર
\item
  \textbf{પ્રકાર માહિતી}: MIME પ્રકાર અથવા ફાઇલ એક્સ્ટેન્શન
\item
  \textbf{કદ અને ફાળવણી}: વર્તમાન કદ અને ફાળવેલ જગ્યા
\item
  \textbf{એક્સેસ કંટ્રોલ}: યુઝર પરમિશન્સ અને ગ્રુપ એક્સેસ રાઇટ્સ
\end{itemize}

\textbf{સ્ટોરેજ સ્થાન:} ફાઇલ એટ્રિબ્યુટ્સ સામાન્ય રીતે ડિરેક્ટરી એન્ટ્રીઝ અથવા ફાઇલ
એલોકેશન ટેબલ્સમાં સ્ટોર કરવામાં આવે છે.

\end{solutionbox}
\begin{mnemonicbox}
``NTSLPTO - Name, Type, Size, Location, Protection,
Time, Owner''

\end{mnemonicbox}
\subsection*{પ્રશ્ન 3(c) [7
marks]}\label{q3c}

\textbf{વિવિધ ફાઇલ ફાળવણી પદ્ધતિઓની યાદી બનાવો અને જરૂરી રેખાકૃતિ સાથે
કન્ટીગ્યુઅસ ફાળવણી સમજાવો.}

\begin{solutionbox}

\textbf{ફાઇલ ફાળવણી પદ્ધતિઓ:}

{\def\LTcaptype{none} % do not increment counter
\begin{longtable}[]{@{}ll@{}}
\toprule\noalign{}
પદ્ધતિ & વર્ણન \\
\midrule\noalign{}
\endhead
\bottomrule\noalign{}
\endlastfoot
કન્ટીગ્યુઅસ & ફાઇલો સતત બ્લોક્સમાં સ્ટોર કરવામાં આવે છે \\
લિંક્ડ & ફાઇલો બ્લોક્સની લિંક્ડ લિસ્ટ વાપરીને સ્ટોર કરવામાં આવે છે \\
ઇન્ડેક્સ્ડ & ડેટા બ્લોક્સ તરફ પોઇન્ટ કરવા માટે ઇન્ડેક્સ બ્લોકનો ઉપયોગ કરે છે \\
\end{longtable}
}

\textbf{કન્ટીગ્યુઅસ ફાળવણી:}

કન્ટીગ્યુઅસ ફાળવણીમાં, દરેક ફાઇલ ડિસ્ક પર સતત બ્લોક્સનો સેટ વ્યાપે છે.

\begin{verbatim}
ડિસ્ક બ્લોક્સ:
+{-{-}{-}+{-}{-}{-}+{-}{-}{-}+{-}{-}{-}+{-}{-}{-}+{-}{-}{-}+{-}{-}{-}+{-}{-}{-}+{-}{-}{-}+{-}{-}{-}+}
| 0 | 1 | 2 | 3 | 4 | 5 | 6 | 7 | 8 | 9 |
+{-{-}{-}+{-}{-}{-}+{-}{-}{-}+{-}{-}{-}+{-}{-}{-}+{-}{-}{-}+{-}{-}{-}+{-}{-}{-}+{-}{-}{-}+{-}{-}{-}+}
|   |ફાઇલ A|   |  ફાઇલ B  |   |ફાઇલ C|
|   |  2{-3  |   |   5{-}7     |   |  9    |}
+{-{-}{-}+{-}{-}{-}+{-}{-}{-}+{-}{-}{-}+{-}{-}{-}+{-}{-}{-}+{-}{-}{-}+{-}{-}{-}+{-}{-}{-}+{-}{-}{-}+}

ડિરેક્ટરી ટેબલ:
+{-{-}{-}{-}{-}{-}{-}{-}{-}{-}+{-}{-}{-}{-}{-}{-}{-}+{-}{-}{-}{-}{-}{-}{-}{-}+}
| ફાઇલનામ  | શરૂઆત | લંબાઈ  |
+{-{-}{-}{-}{-}{-}{-}{-}{-}{-}+{-}{-}{-}{-}{-}{-}{-}+{-}{-}{-}{-}{-}{-}{-}{-}+}
| ફાઇલ A   |   2   |   2    |
| ફાઇલ B   |   5   |   3    |
| ફાઇલ C   |   9   |   1    |
+{-{-}{-}{-}{-}{-}{-}{-}{-}{-}+{-}{-}{-}{-}{-}{-}{-}+{-}{-}{-}{-}{-}{-}{-}{-}+}
\end{verbatim}

\textbf{ફાયદા:}

\begin{itemize}
\tightlist
\item
  \textbf{ઝડપી એક્સેસ}: બ્લોક સરનામાંની પ્રત્યક્ષ ગણતરી
\item
  \textbf{ન્યૂનતમ સીક ટાઇમ}: સતત બ્લોક્સ હેડ મૂવમેન્ટ ઘટાડે છે
\item
  \textbf{સરળ અમલીકરણ}: અમલ કરવામાં અને મેનેજ કરવામાં સરળ
\end{itemize}

\textbf{નુકસાનો:}

\begin{itemize}
\tightlist
\item
  \textbf{એક્સટર્નલ ફ્રેગમેન્ટેશન}: ફાઇલો વચ્ચે વણવપરાશી જગ્યાઓ
\item
  \textbf{ફાઇલ કદની મર્યાદા}: ફાઇલો વિસ્તારવી મુશ્કેલ
\item
  \textbf{કોમ્પેક્શનની જરૂર}: સમયાંતરે પુનઃઆયોજનની જરૂર
\end{itemize}

\end{solutionbox}
\begin{mnemonicbox}
``FMS vs EFC - Fast access, Minimal seek, Simple vs
External fragmentation, File size limits, Compaction needed''

\end{mnemonicbox}
\subsection*{પ્રશ્ન 3(a) OR [3
marks]}\label{q3a}

\textbf{લિનક્સ ફાઇલ સિસ્ટમના વિવિધ પ્રકારો ટૂંકમાં સમજાવો.}

\begin{solutionbox}

{\def\LTcaptype{none} % do not increment counter
\begin{longtable}[]{@{}ll@{}}
\toprule\noalign{}
ફાઇલ સિસ્ટમ & વર્ણન \\
\midrule\noalign{}
\endhead
\bottomrule\noalign{}
\endlastfoot
ext2 & બીજું એક્સ્ટેન્ડેડ ફાઇલસિસ્ટમ, કોઈ જર્નલિંગ નથી \\
ext3 & ત્રીજું એક્સ્ટેન્ડેડ ફાઇલસિસ્ટમ જર્નલિંગ સાથે \\
ext4 & ચોથું એક્સ્ટેન્ડેડ ફાઇલસિસ્ટમ, સુધારેલ પર્ફોર્મન્સ \\
XFS & ઉચ્ચ-પર્ફોર્મન્સ 64-બિટ જર્નલિંગ ફાઇલસિસ્ટમ \\
Btrfs & B-ટ્રી ફાઇલસિસ્ટમ એડવાન્સ્ડ ફીચર્સ સાથે \\
ZFS & કોપી-ઓન-રાઇટ ફાઇલસિસ્ટમ ડેટા ઇન્ટેગ્રિટી સાથે \\
\end{longtable}
}

\textbf{મુખ્ય લક્ષણો:}

\begin{itemize}
\tightlist
\item
  \textbf{જર્નલિંગ}: ext3, ext4, XFS ક્રેશ રિકવરી પ્રદાન કરે છે
\item
  \textbf{પર્ફોર્મન્સ}: ext4, XFS મોટી ફાઇલો માટે ઓપ્ટિમાઇઝ્ડ છે
\item
  \textbf{એડવાન્સ્ડ ફીચર્સ}: Btrfs, ZFS સ્નેપશોટ્સ અને કમ્પ્રેશન ઓફર કરે છે
\end{itemize}

\textbf{પસંદગીના માપદંડો:} પર્ફોર્મન્સ, વિશ્વસનીયતા અને ફીચર આવશ્યકતાઓના આધારે
વિવિધ ફાઇલસિસ્ટમ્સ વિવિધ ઉપયોગ કેસો માટે યોગ્ય છે.

\end{solutionbox}
\begin{mnemonicbox}
``EEXBZ - ext2/3/4, XFS, Btrfs, ZFS વિકલ્પો''

\end{mnemonicbox}
\subsection*{પ્રશ્ન 3(b) OR [4
marks]}\label{q3b}

\textbf{વિવિધ ફાઇલ ઓપરેશન્સ સમજાવો.}

\begin{solutionbox}

{\def\LTcaptype{none} % do not increment counter
\begin{longtable}[]{@{}ll@{}}
\toprule\noalign{}
ઓપરેશન & વર્ણન \\
\midrule\noalign{}
\endhead
\bottomrule\noalign{}
\endlastfoot
બનાવો & નિર્દિષ્ટ નામ અને એટ્રિબ્યુટ્સ સાથે નવી ફાઇલ બનાવો \\
ખોલો & રીડિંગ/રાઇટિંગ ઓપરેશન્સ માટે ફાઇલ તૈયાર કરો \\
વાંચો & વર્તમાન સ્થિતિ પરથી ફાઇલમાંથી ડેટા મેળવો \\
લખો & વર્તમાન સ્થિતિ પર ફાઇલમાં ડેટા સ્ટોર કરો \\
સીક & ફાઇલ પોઇન્ટરને વિશિષ્ટ સ્થિતિ પર ખસેડો \\
બંધ કરો & ફાઇલ રિસોર્સ રિલીઝ કરો અને મેટાડેટા અપડેટ કરો \\
ડિલીટ કરો & ફાઇલ દૂર કરો અને સ્ટોરેજ સ્પેસ ડીએલોકેટ કરો \\
\end{longtable}
}

\textbf{ફાઇલ ઓપરેશન સિક્વન્સ:}

\begin{center}
\textbf{Mermaid Diagram (Code)}
\begin{verbatim}
{Shaded}
{Highlighting}[]
graph LR
    A[ફાઇલ બનાવો] {-{-}{} B[ફાઇલ ખોલો]}
    B {-{-}{} C[વાંચો/લખો]}
    C {-{-}{} D[જરૂર પડે તો સીક]}
    D {-{-}{} C}
    C {-{-}{} E[ફાઇલ બંધ કરો]}
    E {-{-}{} F[જરૂર પડે તો ડિલીટ કરો]}
{Highlighting}
{Shaded}
\end{verbatim}
\end{center}

\textbf{મહત્વપૂર્ણ વિચારણાઓ:}

\begin{itemize}
\tightlist
\item
  \textbf{એરર હેન્ડલિંગ}: દરેક ઓપરેશન નિષ્ફળ થઈ શકે છે અને ચેક કરવું જોઈએ
\item
  \textbf{પરમિશન્સ}: યુઝર પાસે યોગ્ય એક્સેસ રાઇટ્સ હોવા જોઈએ
\item
  \textbf{સમાન સમયે એક્સેસ}: બહુવિધ પ્રક્રિયાઓ એક જ ફાઇલને એક્સેસ કરી શકે છે
\end{itemize}

\end{solutionbox}
\begin{mnemonicbox}
``CORWSCD - Create, Open, Read, Write, Seek, Close,
Delete''

\end{mnemonicbox}
\subsection*{પ્રશ્ન 3(c) OR [7
marks]}\label{q3c}

\textbf{વિવિધ ફાઇલ ફાળવણી પદ્ધતિઓની યાદી બનાવો અને જરૂરી રેખાકૃતિ સાથે ઇન્ડેક્સ્ડ
ફાળવણી સમજાવો.}

\begin{solutionbox}

\textbf{ફાઇલ ફાળવણી પદ્ધતિઓ:}

{\def\LTcaptype{none} % do not increment counter
\begin{longtable}[]{@{}ll@{}}
\toprule\noalign{}
ઓપરેશન & વર્ણન \\
\midrule\noalign{}
\endhead
\bottomrule\noalign{}
\endlastfoot
બનાવો & નિર્દિષ્ટ નામ અને એટ્રિબ્યુટ્સ સાથે નવી ફાઇલ બનાવો \\
ખોલો & રીડિંગ/રાઇટિંગ ઓપરેશન્સ માટે ફાઇલ તૈયાર કરો \\
વાંચો & વર્તમાન સ્થિતિ પરથી ફાઇલમાંથી ડેટા મેળવો \\
લખો & વર્તમાન સ્થિતિ પર ફાઇલમાં ડેટા સ્ટોર કરો \\
સીક & ફાઇલ પોઇન્ટરને વિશિષ્ટ સ્થિતિ પર ખસેડો \\
બંધ કરો & ફાઇલ રિસોર્સ રિલીઝ કરો અને મેટાડેટા અપડેટ કરો \\
ડિલીટ કરો & ફાઇલ દૂર કરો અને સ્ટોરેજ સ્પેસ ડીએલોકેટ કરો \\
\end{longtable}
}

\textbf{ઇન્ડેક્સ્ડ ફાળવણી:}

ઇન્ડેક્સ્ડ ફાળવણીમાં, દરેક ફાઇલ પાસે ડેટા બ્લોક્સના પોઇન્ટર્સ ધરાવતો ઇન્ડેક્સ બ્લોક હોય
છે.

\begin{verbatim}
ફાઇલ A માટે ઇન્ડેક્સ બ્લોક:
+{-{-}{-}+{-}{-}{-}+{-}{-}{-}+{-}{-}{-}+}
| 2 | 5 | 8 | 9 |
+{-{-}{-}+{-}{-}{-}+{-}{-}{-}+{-}{-}{-}+}
  |   |   |   |
  v   v   v   v
ડિસ્ક બ્લોક્સ:
+{-{-}{-}+{-}{-}{-}+{-}{-}{-}+{-}{-}{-}+{-}{-}{-}+{-}{-}{-}+{-}{-}{-}+{-}{-}{-}+{-}{-}{-}+{-}{-}{-}+}
| 0 | 1 | 2 | 3 | 4 | 5 | 6 | 7 | 8 | 9 |
+{-{-}{-}+{-}{-}{-}+{-}{-}{-}+{-}{-}{-}+{-}{-}{-}+{-}{-}{-}+{-}{-}{-}+{-}{-}{-}+{-}{-}{-}+{-}{-}{-}+}
|   |   |ફાઇલA|   |   |ફાઇલA|   |   |ફાઇલA|ફાઇલA|

ડિરેક્ટરી ટેબલ:
+{-{-}{-}{-}{-}{-}{-}{-}{-}{-}+{-}{-}{-}{-}{-}{-}{-}{-}{-}{-}{-}{-}{-}+}
| ફાઇલનામ  | ઇન્ડેક્સ બ્લોક |
+{-{-}{-}{-}{-}{-}{-}{-}{-}{-}+{-}{-}{-}{-}{-}{-}{-}{-}{-}{-}{-}{-}{-}+}
| ફાઇલ A   |      1      |
+{-{-}{-}{-}{-}{-}{-}{-}{-}{-}+{-}{-}{-}{-}{-}{-}{-}{-}{-}{-}{-}{-}{-}+}
\end{verbatim}

\textbf{ઇન્ડેક્સ્ડ ફાળવણીના પ્રકારો:}

\begin{itemize}
\tightlist
\item
  \textbf{સિંગલ-લેવલ}: ફાઇલ દીઠ એક ઇન્ડેક્સ બ્લોક
\item
  \textbf{મલ્ટિ-લેવલ}: ઇન્ડેક્સ બ્લોક્સ અન્ય ઇન્ડેક્સ બ્લોક્સ તરફ પોઇન્ટ કરે છે
\item
  \textbf{કમ્બાઇન્ડ}: ડાઇરેક્ટ અને ઇન્ડાઇરેક્ટ પોઇન્ટર્સનું મિશ્રણ
\end{itemize}

\textbf{ફાયદા:}

\begin{itemize}
\tightlist
\item
  \textbf{કોઈ એક્સટર્નલ ફ્રેગમેન્ટેશન નહીં}: બ્લોક્સ ડિસ્ક પર ગમે ત્યાં હોઈ શકે છે
\item
  \textbf{ડાયનેમિક ફાઇલ કદ}: ફાઇલો વિસ્તારવી સરળ છે
\item
  \textbf{ઝડપી રેન્ડમ એક્સેસ}: કોઈપણ બ્લોકમાં ડાઇરેક્ટ એક્સેસ
\end{itemize}

\textbf{નુકસાનો:}

\begin{itemize}
\tightlist
\item
  \textbf{ઇન્ડેક્સ બ્લોક ઓવરહેડ}: પોઇન્ટર્સ સ્ટોર કરવા માટે વધારાની જગ્યા
\item
  \textbf{બહુવિધ ડિસ્ક એક્સેસ}: બે એક્સેસની જરૂર (ઇન્ડેક્સ + ડેટા)
\item
  \textbf{નાની ફાઇલ અકાર્યક્ષમતા}: નાની ફાઇલો માટે ઓવરહેડ વધુ
\end{itemize}

\end{solutionbox}
\begin{mnemonicbox}
``NDF vs IMI - No fragmentation, Dynamic size, Fast
access vs Index overhead, Multiple access, Inefficient for small files''

\end{mnemonicbox}
\subsection*{પ્રશ્ન 4(a) [3
marks]}\label{q4a}

\textbf{સિસ્ટમ ધમકીઓ વ્યાખ્યાયિત કરો અને તેના પ્રકારો સમજાવો.}

\begin{solutionbox}

\textbf{સિસ્ટમ ધમકીઓ} એ કમ્પ્યુટર સિસ્ટમના કામકાજને ખલેલ પહોંચાડવા, નુકસાન
પહોંચાડવા, માહિતી ચોરવા અથવા અનધિકૃત પ્રવેશ મેળવવાના દુર્ભાવનાપૂર્ણ પ્રયાસો છે.

{\def\LTcaptype{none} % do not increment counter
\begin{longtable}[]{@{}ll@{}}
\toprule\noalign{}
ધમકીનો પ્રકાર & વર્ણન \\
\midrule\noalign{}
\endhead
\bottomrule\noalign{}
\endlastfoot
વર્મ્સ & નેટવર્ક પર ફેલાતા સ્વ-પ્રતિકૃત પ્રોગ્રામ્સ \\
વાયરસ & અન્ય પ્રોગ્રામ્સ સાથે જોડાતા દુર્ભાવનાપૂર્ણ કોડ \\
ટ્રોજન હોર્સ & છુપાયેલા દુર્ભાવનાપૂર્ણ કાર્યો સાથે કાયદેસર દેખાતા પ્રોગ્રામ્સ \\
ડિનાયલ ઓફ સર્વિસ & સિસ્ટમ રિસોર્સને ભરાઈ જવાની હુમલાઓ \\
પોર્ટ સ્કેનિંગ & નેટવર્ક સેવાઓની અનધિકૃત તપાસ \\
\end{longtable}
}

\textbf{સિસ્ટમ ધમકીઓના વર્ગો:}

\begin{itemize}
\tightlist
\item
  \textbf{નેટવર્ક-આધારિત}: નેટવર્ક કનેક્શન્સ અને પ્રોટોકોલ્સ દ્વારા હુમલાઓ
\item
  \textbf{હોસ્ટ-આધારિત}: વિશિષ્ટ કમ્પ્યુટર સિસ્ટમ્સને લક્ષ્ય બનાવતા હુમલાઓ
\item
  \textbf{ભૌતિક}: સિસ્ટમને સમાધાન કરવા માટે પ્રત્યક્ષ ભૌતિક પ્રવેશ
\end{itemize}

\textbf{પ્રભાવ:} સિસ્ટમ ધમકીઓ ડેટા ખોવાઈ જવા, સિસ્ટમ ડાઉનટાઇમ, ગોપનીયતા ભંગ
અને આર્થિક નુકસાન તરફ દોરી શકે છે.

\end{solutionbox}
\begin{mnemonicbox}
``WVTDP - Worms, Viruses, Trojans, DoS, Port
scanning''

\end{mnemonicbox}
\subsection*{પ્રશ્ન 4(b) [4
marks]}\label{q4b}

\textbf{તફાવત કરો: યુઝર ઓથેન્ટિકેશન v/s યુઝર ઓથોરાઇઝેશન.}

\begin{solutionbox}

{\def\LTcaptype{none} % do not increment counter
\begin{longtable}[]{@{}lll@{}}
\toprule\noalign{}
પાસું & યુઝર ઓથેન્ટિકેશન & યુઝર ઓથોરાઇઝેશન \\
\midrule\noalign{}
\endhead
\bottomrule\noalign{}
\endlastfoot
હેતુ & યુઝરની ઓળખ ચકાસવી & યુઝર પરમિશન્સ નક્કી કરવી \\
ક્યારે & સિસ્ટમ એક્સેસ પહેલાં & ઓથેન્ટિકેશન પછી \\
પદ્ધતિઓ & પાસવર્ડ્સ, બાયોમેટ્રિક્સ, ટોકન્સ & એક્સેસ કંટ્રોલ લિસ્ટ્સ, રોલ્સ \\
પ્રશ્ન & ``તમે કોણ છો?'' & ``તમે શું કરી શકો?'' \\
પ્રક્રિયા & લોગિન સમયે એકવાર & સેશન દરમિયાન સતત \\
\end{longtable}
}

\textbf{ઓથેન્ટિકેશનની પદ્ધતિઓ:}

\begin{itemize}
\tightlist
\item
  \textbf{તમે જે જાણો છો}: પાસવર્ડ્સ, PINs
\item
  \textbf{તમે જે છો}: ફિંગરપ્રિન્ટ્સ, રેટિના સ્કેન્સ
\item
  \textbf{તમારી પાસે જે છે}: સ્માર્ટ કાર્ડ્સ, ટોકન્સ
\end{itemize}

\textbf{ઓથોરાઇઝેશન મોડલ્સ:}

\begin{itemize}
\tightlist
\item
  \textbf{રોલ-આધારિત એક્સેસ કંટ્રોલ (RBAC)}: યુઝર રોલ્સના આધારે પરમિશન્સ
\item
  \textbf{ડિસ્ક્રેશનરી એક્સેસ કંટ્રોલ (DAC)}: માલિક એક્સેસ કંટ્રોલ કરે છે
\item
  \textbf{મેન્ડેટરી એક્સેસ કંટ્રોલ (MAC)}: સિસ્ટમ-લાગુ કરેલા સિક્યોરિટી લેવલ્સ
\end{itemize}

\textbf{સંબંધ:} ઓથોરાઇઝેશન પહેલાં ઓથેન્ટિકેશન થવી જોઈએ. વ્યાપક સિક્યોરિટી માટે બંને
જરૂરી છે.

\end{solutionbox}
\begin{mnemonicbox}
``WHO vs WHAT - ઓથેન્ટિકેશન પૂછે છે કોણ, ઓથોરાઇઝેશન નક્કી કરે
છે શું''

\end{mnemonicbox}
\subsection*{પ્રશ્ન 4(c) [7
marks]}\label{q4c}

\textbf{વિવિધ ઓપરેટિંગ સિસ્ટમ સિક્યોરિટી નીતિઓ અને પ્રક્રિયાઓની ચર્ચા કરો.}

\begin{solutionbox}

\textbf{સિક્યોરિટી નીતિઓ:}

{\def\LTcaptype{none} % do not increment counter
\begin{longtable}[]{@{}ll@{}}
\toprule\noalign{}
નીતિનો પ્રકાર & વર્ણન \\
\midrule\noalign{}
\endhead
\bottomrule\noalign{}
\endlastfoot
એક્સેસ કંટ્રોલ & કોણ કયા રિસોર્સને એક્સેસ કરી શકે તે વ્યાખ્યાયિત કરે છે \\
પાસવર્ડ નીતિ & પાસવર્ડ બનાવટ અને સંચાલનના નિયમો \\
ઓડિટ નીતિ & સિસ્ટમ પ્રવૃત્તિઓનું લોગિંગ અને મોનિટરિંગ \\
અપડેટ નીતિ & નિયમિત સિક્યોરિટી પેચ અને અપડેટ્સ \\
ડેટા ક્લાસિફિકેશન & સેન્સિટિવિટી લેવલ્સ પ્રમાણે ડેટાનું વર્ગીકરણ \\
\end{longtable}
}

\textbf{સિક્યોરિટી પ્રક્રિયાઓ:}

\textbf{1. યુઝર એકાઉન્ટ મેનેજમેન્ટ:}

\begin{itemize}
\tightlist
\item
  યુઝર એકાઉન્ટ્સ અને પરમિશન્સની નિયમિત સમીક્ષા
\item
  નિવૃત્ત કર્મચારીઓ માટે એક્સેસનું તાત્કાલિક રદ્દીકરણ
\item
  લીસ્ટ પ્રિવિલેજ સિદ્ધાંતનો અમલ
\end{itemize}

\textbf{2. સિસ્ટમ મોનિટરિંગ:}

\begin{center}
\textbf{Mermaid Diagram (Code)}
\begin{verbatim}
{Shaded}
{Highlighting}[]
graph LR
    A[લોગ કલેક્શન] {-{-}{} B[એનાલિસિસ એન્જિન]}
    B {-{-}{} C[ધમકી શોધ]}
    C {-{-}{} D[એલર્ટ જનરેશન]}
    D {-{-}{} E[રિસ્પોન્સ એક્શન]}
{Highlighting}
{Shaded}
\end{verbatim}
\end{center}

\textbf{3. ઇન્સિડન્ટ રિસ્પોન્સ:}

\begin{itemize}
\tightlist
\item
  \textbf{શોધ}: સિક્યોરિટી ઇન્સિડન્ટ્સને ઝડપથી ઓળખવી
\item
  \textbf{નિયંત્રણ}: નુકસાન મર્યાદિત કરવું અને ફેલાવો રોકવો
\item
  \textbf{પુનઃપ્રાપ્તિ}: સામાન્ય કામકાજ સુરક્ષિત રીતે પુનઃસ્થાપિત કરવું
\end{itemize}

\textbf{4. બેકઅપ અને રિકવરી:}

\begin{itemize}
\tightlist
\item
  પરીક્ષિત રિસ્ટોર પ્રક્રિયાઓ સાથે નિયમિત ડેટા બેકઅપ્સ
\item
  ડિઝાસ્ટર રિકવરી પ્લાનિંગ અને ટેસ્ટિંગ
\item
  બિઝનેસ કન્ટિન્યુઇટી પગલાં
\end{itemize}

\textbf{અમલીકરણ ફ્રેમવર્ક:}

\begin{itemize}
\tightlist
\item
  \textbf{રિસ્ક એસેસમેન્ટ}: નબળાઈઓ અને ધમકીઓની ઓળખ
\item
  \textbf{નીતિ વિકાસ}: વ્યાપક સિક્યોરિટી ગાઇડલાઇન્સ બનાવવી
\item
  \textbf{ટ્રેનિંગ}: યુઝર્સને સિક્યોરિટી પ્રથાઓ વિશે શિક્ષિત કરવા
\item
  \textbf{કમ્પ્લાયન્સ}: નિયમોનું પાલન સુનિશ્ચિત કરવું
\end{itemize}

\end{solutionbox}
\begin{mnemonicbox}
``AAPUD નીતિઓ + UMSIR પ્રક્રિયાઓ - Access, Audit,
Password, Update, Data classification + User management, Monitoring,
System response, Incident handling, Recovery''

\end{mnemonicbox}
\subsection*{પ્રશ્ન 4(a) OR [3
marks]}\label{q4a}

\textbf{પ્રોગ્રામ ધમકીઓ વ્યાખ્યાયિત કરો અને તેના પ્રકારો સમજાવો.}

\begin{solutionbox}

\textbf{પ્રોગ્રામ ધમકીઓ} એ કમ્પ્યુટર પ્રોગ્રામ્સ અને ડેટાને ખલેલ પહોંચાડવા, નુકસાન
પહોંચાડવા અથવા અનધિકૃત પ્રવેશ મેળવવા માટે ડિઝાઇન કરેલ દુર્ભાવનાપૂર્ણ સોફ્ટવેર છે.

{\def\LTcaptype{none} % do not increment counter
\begin{longtable}[]{@{}ll@{}}
\toprule\noalign{}
ધમકીનો પ્રકાર & વર્ણન \\
\midrule\noalign{}
\endhead
\bottomrule\noalign{}
\endlastfoot
મેલવેર & વાયરસ, વર્મ્સ સહિત દુર્ભાવનાપૂર્ણ સોફ્ટવેર \\
સ્પાયવેર & યુઝર પ્રવૃત્તિઓનું ગુપ્ત રીતે મોનિટરિંગ કરતા પ્રોગ્રામ્સ \\
એડવેર & અનિચ્છિત એડવર્ટાઇઝિંગ સોફ્ટવેર \\
રેન્સમવેર & ડેટા એન્ક્રિપ્ટ કરીને પેમેન્ટ માંગે છે \\
રૂટકિટ્સ & શોધથી દુર્ભાવનાપૂર્ણ પ્રવૃત્તિઓ છુપાવે છે \\
\end{longtable}
}

\textbf{પ્રોગ્રામ ધમકીના વર્ગો:}

\begin{itemize}
\tightlist
\item
  \textbf{એક્ઝિક્યુટેબલ ધમકીઓ}: સ્વતંત્ર દુર્ભાવનાપૂર્ણ પ્રોગ્રામ્સ
\item
  \textbf{પેરાસાઇટિક ધમકીઓ}: કાયદેસર પ્રોગ્રામ્સ સાથે જોડાય છે
\item
  \textbf{નેટવર્ક ધમકીઓ}: નેટવર્ક કનેક્શન્સ દ્વારા ફેલાય છે
\end{itemize}

\textbf{સામાન્ય હુમલાના માર્ગો:}

\begin{itemize}
\tightlist
\item
  ઇમેઇલ એટેચમેન્ટ્સ અને ડાઉનલોડ્સ
\item
  ઇન્ફેક્ટેડ રિમૂવેબલ મીડિયા
\item
  નેટવર્ક નબળાઈઓ અને એક્સપ્લોઇટ્સ
\end{itemize}

\end{solutionbox}
\begin{mnemonicbox}
``MSARR - Malware, Spyware, Adware, Ransomware,
Rootkits''

\end{mnemonicbox}
\subsection*{પ્રશ્ન 4(b) OR [4
marks]}\label{q4b}

\textbf{પ્રોટેક્શન ડોમેનને યોગ્ય ઉદાહરણ સાથે સમજાવો.}

\begin{solutionbox}

\textbf{પ્રોટેક્શન ડોમેન} એ ઓબ્જેક્ટ્સ અને એક્સેસ રાઇટ્સનો સેટ છે જે વ્યાખ્યાયિત કરે છે કે
પ્રક્રિયા કયા રિસોર્સને એક્સેસ કરી શકે છે અને કયા ઓપરેશન્સ કરી શકે છે.

{\def\LTcaptype{none} % do not increment counter
\begin{longtable}[]{@{}ll@{}}
\toprule\noalign{}
ઘટક & વર્ણન \\
\midrule\noalign{}
\endhead
\bottomrule\noalign{}
\endlastfoot
ઓબ્જેક્ટ્સ & ફાઇલો, મેમરી, ડિવાઇસ જેવા રિસોર્સ \\
એક્સેસ રાઇટ્સ & રીડ, રાઇટ, એક્ઝિક્યુટ જેવી પરમિશન્સ \\
સબ્જેક્ટ્સ & એક્સેસ માંગતી પ્રક્રિયાઓ અથવા યુઝર્સ \\
\end{longtable}
}

\textbf{ડોમેન સ્ટ્રક્ચર:}

\begin{verbatim}
Protection Domain A
+{-{-}{-}{-}{-}{-}{-}{-}{-}{-}{-}{-}{-}{-}{-}{-}{-}{-}+}
| Objects:         |
| {- File1 (R,W)    |}
| {- Printer (W)    |}
| {- Memory (R,W,X) |}
+{-{-}{-}{-}{-}{-}{-}{-}{-}{-}{-}{-}{-}{-}{-}{-}{-}{-}+}

Protection Domain B
+{-{-}{-}{-}{-}{-}{-}{-}{-}{-}{-}{-}{-}{-}{-}{-}{-}{-}+}
| Objects:         |
| {- File2 (R)      |}
| {- Network (R,W)  |}
| {- Database (R)   |}
+{-{-}{-}{-}{-}{-}{-}{-}{-}{-}{-}{-}{-}{-}{-}{-}{-}{-}+}
\end{verbatim}

\textbf{ઉદાહરણ - યુનિવર્સિટી સિસ્ટમ:}

\begin{itemize}
\tightlist
\item
  \textbf{સ્ટુડન્ટ ડોમેન}: કોર્સ મટીરિયલ્સને રીડ એક્સેસ, એસાઇનમેન્ટ્સને રાઇટ એક્સેસ
\item
  \textbf{ફેકલ્ટી ડોમેન}: ગ્રેડ ડેટાબેઝને રીડ/રાઇટ એક્સેસ, સ્ટુડન્ટ રેકોર્ડ્સને રીડ એક્સેસ
\item
  \textbf{એડમિન ડોમેન}: સિસ્ટમ કન્ફિગરેશન, યુઝર મેનેજમેન્ટને ફુલ એક્સેસ
\end{itemize}

\textbf{ડોમેન સ્વિચિંગ:} પ્રક્રિયાઓ નીચેના આધારે ડોમેન્સ વચ્ચે સ્વિચ કરી શકે છે:

\begin{itemize}
\tightlist
\item
  યુઝર ઓથેન્ટિકેશન અને ઓથોરાઇઝેશન
\item
  પ્રોગ્રામ એક્ઝિક્યુશન કન્ટેક્સ્ટ
\item
  સિક્યોરિટી લેવલ આવશ્યકતાઓ
\end{itemize}

\textbf{ફાયદા:}

\begin{itemize}
\tightlist
\item
  \textbf{આઇસોલેશન}: ડોમેન્સ વચ્ચે અનધિકૃત એક્સેસ અટકાવે છે
\item
  \textbf{લવચીકતા}: નિયંત્રિત રિસોર્સ શેરિંગની મંજૂરી આપે છે
\item
  \textbf{સિક્યોરિટી}: લીસ્ટ પ્રિવિલેજ સિદ્ધાંતનો અમલ કરે છે
\end{itemize}

\end{solutionbox}
\begin{mnemonicbox}
``OAS - Objects, Access rights, Subjects define
domains''

\end{mnemonicbox}
\subsection*{પ્રશ્ન 4(c) OR [7
marks]}\label{q4c}

\textbf{એક્સેસ કંટ્રોલ લિસ્ટ વિગતવાર સમજાવો.}

\begin{solutionbox}

\textbf{એક્સેસ કંટ્રોલ લિસ્ટ (ACL)} એ સિક્યોરિટી મેકેનિઝમ છે જે નિર્દિષ્ટ કરે છે કે કયા
યુઝર્સ અથવા પ્રક્રિયાઓને ઓબ્જેક્ટ્સની એક્સેસ આપવામાં આવે છે અને કયા ઓપરેશન્સની મંજૂરી છે.

\textbf{ACL સ્ટ્રક્ચર:}

{\def\LTcaptype{none} % do not increment counter
\begin{longtable}[]{@{}ll@{}}
\toprule\noalign{}
ઘટક & વર્ણન \\
\midrule\noalign{}
\endhead
\bottomrule\noalign{}
\endlastfoot
સબ્જેક્ટ & એક્સેસ માંગતા યુઝર, ગ્રુપ અથવા પ્રક્રિયા \\
ઓબ્જેક્ટ & સુરક્ષિત કરવામાં આવતા રિસોર્સ (ફાઇલ, ડિવાઇસ, વગેરે) \\
એક્સેસ રાઇટ્સ & આપવામાં આવેલી વિશિષ્ટ પરમિશન્સ \\
\end{longtable}
}

\textbf{ACL અમલીકરણ:}

\begin{verbatim}
File: /home/project/report.txt
+{-{-}{-}{-}{-}{-}{-}{-}{-}{-}{-}{-}{-}{-}{-}{-}{-}{-}{-}{-}{-}{-}{-}{-}{-}{-}{-}{-}{-}{-}{-}{-}+}
| User     | Permissions         |
|{-{-}{-}{-}{-}{-}{-}{-}{-}{-}|{-}{-}{-}{-}{-}{-}{-}{-}{-}{-}{-}{-}{-}{-}{-}{-}{-}{-}{-}{-}{-}|}
| alice    | read, write         |
| bob      | read                |
| admin    | read, write, delete |
| group:dev| read, write         |
+{-{-}{-}{-}{-}{-}{-}{-}{-}{-}{-}{-}{-}{-}{-}{-}{-}{-}{-}{-}{-}{-}{-}{-}{-}{-}{-}{-}{-}{-}{-}{-}+}
\end{verbatim}

\textbf{ACL ના પ્રકારો:}

\begin{itemize}
\tightlist
\item
  \textbf{ડિસ્ક્રેશનરી ACL (DACL)}: માલિક એક્સેસ પરમિશન્સ કંટ્રોલ કરે છે
\item
  \textbf{સિસ્ટમ ACL (SACL)}: સિસ્ટમ ઓડિટિંગ અને લોગિંગ કંટ્રોલ કરે છે
\item
  \textbf{ડિફોલ્ટ ACL}: નવા ઓબ્જેક્ટ્સ માટે વારસામાં મળતી પરમિશન્સ
\end{itemize}

\textbf{ACL vs કેપેબિલિટી લિસ્ટ્સ:}

{\def\LTcaptype{none} % do not increment counter
\begin{longtable}[]{@{}lll@{}}
\toprule\noalign{}
પાસું & ACL & કેપેબિલિટી લિસ્ટ \\
\midrule\noalign{}
\endhead
\bottomrule\noalign{}
\endlastfoot
ઓર્ગનાઇઝેશન & ઓબ્જેક્ટ દીઠ & સબ્જેક્ટ દીઠ \\
સ્ટોરેજ & ઓબ્જેક્ટ સાથે & સબ્જેક્ટ સાથે \\
ચેકિંગ & લિસ્ટ સ્કેન કરો & કેપેબિલિટી પ્રેઝન્ટ કરો \\
રિવોકેશન & સરળ & મુશ્કેલ \\
\end{longtable}
}

\textbf{ફાયદા:}

\begin{itemize}
\tightlist
\item
  \textbf{ગ્રેન્યુલર કંટ્રોલ}: ફાઇન-ગ્રેઇન્ડ પરમિશન મેનેજમેન્ટ
\item
  \textbf{સેન્ટ્રલાઇઝ્ડ મેનેજમેન્ટ}: ઓબ્જેક્ટ પરમિશન્સ સુધારવામાં સરળ
\item
  \textbf{ઓડિટ ટ્રેઇલ}: કોની પાસે એક્સેસ છે તેનો સ્પષ્ટ રેકોર્ડ
\end{itemize}

\textbf{નુકસાનો:}

\begin{itemize}
\tightlist
\item
  \textbf{પર્ફોર્મન્સ ઓવરહેડ}: દરેક એક્સેસ માટે ACL ચેક કરવું જોઈએ
\item
  \textbf{સ્ટોરેજ આવશ્યકતાઓ}: પરમિશન લિસ્ટ્સ માટે જગ્યાની જરૂર
\item
  \textbf{જટિલતા}: ઘણા યુઝર્સ/ઓબ્જેક્ટ્સ માટે મેનેજ કરવું મુશ્કેલ
\end{itemize}

\textbf{વાસ્તવિક દુનિયાનું ઉદાહરણ:} લિનક્સ ફાઇલ પરમિશન્સ માલિક, ગ્રુપ અને અન્યો
માટે રીડ, રાઇટ, એક્ઝિક્યુટ રાઇટ્સ સાથે સરળીકૃત ACL નો ઉપયોગ કરે છે.

\end{solutionbox}
\begin{mnemonicbox}
``SOA structure + GDSC advantages - Subject, Object,
Access rights + Granular, Distributed, Centralized, Audit capabilities''

\end{mnemonicbox}
\subsection*{પ્રશ્ન 5(a) [3
marks]}\label{q5a}

\textbf{નીચેના આદેશો સમજાવો: (i) man (ii) cd (iii) ls}

\begin{solutionbox}

{\def\LTcaptype{none} % do not increment counter
\begin{longtable}[]{@{}lll@{}}
\toprule\noalign{}
આદેશ & હેતુ & સિન્ટેક્સ \\
\midrule\noalign{}
\endhead
\bottomrule\noalign{}
\endlastfoot
man & આદેશો માટે મેન્યુઅલ પેજ દર્શાવે છે & man [આદેશ] \\
cd & વર્તમાન ડિરેક્ટરી બદલે છે & cd [ડિરેક્ટરી] \\
ls & ડિરેક્ટરી સામગ્રીની યાદી દર્શાવે છે & ls [વિકલ્પો] [ડિરેક્ટરી] \\
\end{longtable}
}

\textbf{આદેશની વિગતો:}

\textbf{1. man (મેન્યુઅલ) આદેશ:}

\begin{itemize}
\tightlist
\item
  \textbf{કાર્ય}: લિનક્સ આદેશો માટે વિગતવાર દસ્તાવેજીકરણ દર્શાવે છે
\item
  \textbf{ઉદાહરણ}: \texttt{man\ ls} ls આદેશ માટે મેન્યુઅલ પેજ દર્શાવે છે
\item
  \textbf{વિભાગો}: આદેશો, સિસ્ટમ કોલ્સ, લાઇબ્રેરી ફંક્શન્સ, વગેરે
\end{itemize}

\textbf{2. cd (ચેન્જ ડિરેક્ટરી) આદેશ:}

\begin{itemize}
\tightlist
\item
  \textbf{કાર્ય}: ફાઇલસિસ્ટમમાં ડિરેક્ટરીઓ વચ્ચે નેવિગેટ કરે છે
\item
  \textbf{ઉદાહરણો}: \texttt{cd\ /home}, \texttt{cd\ ..} (પેરેન્ટ),
  \texttt{cd\ \textasciitilde{}} (હોમ)
\item
  \textbf{વિશેષ}: આર્ગ્યુમેન્ટ વિના \texttt{cd} હોમ ડિરેક્ટરીમાં જાય છે
\end{itemize}

\textbf{3. ls (લિસ્ટ) આદેશ:}

\begin{itemize}
\tightlist
\item
  \textbf{કાર્ય}: વર્તમાન અથવા નિર્દિષ્ટ સ્થાનમાં ફાઇલો અને ડિરેક્ટરીઓ દર્શાવે છે
\item
  \textbf{વિકલ્પો}: \texttt{-l} (લાંબો ફોર્મેટ), \texttt{-a} (છુપાયેલ ફાઇલો),
  \texttt{-h} (માનવ-વાંચી શકાય તેવું)
\item
  \textbf{ઉદાહરણ}: \texttt{ls\ -la} છુપાયેલ ફાઇલો સહિત વિગતવાર લિસ્ટિંગ
  દર્શાવે છે
\end{itemize}

\end{solutionbox}
\begin{mnemonicbox}
``MCD - Manual pages, Change directory, Directory
listing''

\end{mnemonicbox}
\subsection*{પ્રશ્ન 5(b) [4
marks]}\label{q5b}

\textbf{ત્રણ સંખ્યાઓ વચ્ચે મહત્તમ સંખ્યા શોધવા માટે શેલ સ્ક્રિપ્ટ લખો.}

\begin{solutionbox}

\begin{verbatim}
\#!/bin/bash
\# ત્રણ સંખ્યાઓ વચ્ચે મહત્તમ શોધવા માટે સ્ક્રિપ્ટ

echo "ત્રણ સંખ્યાઓ દાખલ કરો:"
read {-p} "પ્રથમ સંખ્યા: " num1
read {-p} "બીજી સંખ્યા: " num2  
read {-p} "ત્રીજી સંખ્યા: " num3

\# નેસ્ટેડ if{-else વાપરીને મહત્તમ શોધો}
if [ $num1 {-gt} $num2 ]; then
    if [ $num1 {-gt} $num3 ]; then
        max=$num1
    else
        max=$num3
    fi
else
    if [ $num2 {-gt} $num3 ]; then
        max=$num2
    else
        max=$num3
    fi
fi

echo "મહત્તમ સંખ્યા છે: $max"
\end{verbatim}

\textbf{મુખ્ય લક્ષણો:}

\begin{itemize}
\tightlist
\item
  \textbf{ઇનપુટ વેલિડેશન}: યુઝર પાસેથી ત્રણ સંખ્યાઓ વાંચે છે
\item
  \textbf{સરખાવણી લોજિક}: મહત્તમ શોધવા માટે નેસ્ટેડ if-else વાપરે છે
\item
  \textbf{આઉટપુટ ડિસ્પ્લે}: સ્પષ્ટ સંદેશ સાથે પરિણામ દર્શાવે છે
\end{itemize}

\textbf{વૈકલ્પિક અભિગમ:}

\begin{verbatim}
max=$(echo "$num1 $num2 $num3" | tr { } {n} | sort {-nr} | head {-1})
\end{verbatim}

\end{solutionbox}
\begin{mnemonicbox}
``ICD - Input, Compare, Display result''

\end{mnemonicbox}
\subsection*{પ્રશ્ન 5(c) [7
marks]}\label{q5c}

\textbf{આપેલ 5 અંકની સંખ્યામાં તમામ વ્યક્તિગત અંકોનો સરવાળો શોધવા માટે શેલ સ્ક્રિપ્ટ
લખો.}

\begin{solutionbox}

\begin{verbatim}
\#!/bin/bash
\# 5 અંકની સંખ્યાના અંકોનો સરવાળો શોધવા માટે સ્ક્રિપ્ટ

echo "5 અંકની સંખ્યા દાખલ કરો:"
read number

\# ઇનપુટ ચકાસો
if [ $\{\#number\} {-ne} 5 ] || ! [[ $number ={} \^{}[0{-}9]+$ ]]; then
    echo "ભૂલ: કૃપા કરીને બરાબર 5 અંકો દાખલ કરો"
    exit 1
fi

sum=0
temp=$number

\# દરેક અંક કાઢો અને સરવાળો કરો
while [ $temp {-gt} 0 ]; do
    digit=$((temp \% 10))    \# છેલ્લો અંક મેળવો
    sum=$((sum + digit))    \# સરવાળામાં ઉમેરો
    temp=$((temp / 10))     \# છેલ્લો અંક દૂર કરો
done

echo "સંખ્યા: $number"
echo "અંકોનો સરવાળો: $sum"

\# વિવરણ દર્શાવો
echo "વિવરણ:"
original=$number
echo {-n} "અંકો: "
for ((i=0; i{}5; i++)); do
    digit=$((original \% 10))
    if [ $i {-eq} 4 ]; then
        echo {-n} "$digit"
    else
        echo {-n} "$digit + "
    fi
    original=$((original / 10))
done | tac
echo " = $sum"
\end{verbatim}

\textbf{અલ્ગોરિધમ સ્ટેપ્સ:}

\begin{itemize}
\tightlist
\item
  \textbf{ઇનપુટ વેલિડેશન}: બરાબર 5 અંકો માટે ચકાસો
\item
  \textbf{અંક એક્સ્ટ્રેક્શન}: મોડ્યુલો અને ડિવિઝન ઓપરેશન્સનો ઉપયોગ
\item
  \textbf{સરવાળાની ગણતરી}: દરેક એક્સ્ટ્રેક્ટ કરેલા અંકને ઉમેરો
\item
  \textbf{પરિણામ દર્શાવો}: વિવરણ અને આખરો સરવાળો દર્શાવો
\end{itemize}

\textbf{ઉદાહરણ આઉટપુટ:}

\begin{verbatim}
5 અંકની સંખ્યા દાખલ કરો: 12345
સંખ્યા: 12345
અંકોનો સરવાળો: 15
વિવરણ: 1 + 2 + 3 + 4 + 5 = 15
\end{verbatim}

\end{solutionbox}
\begin{mnemonicbox}
``VEDS - Validate, Extract, Display, Sum digits''

\end{mnemonicbox}
\subsection*{પ્રશ્ન 5(a) OR [3
marks]}\label{q5a}

\textbf{નીચેના આદેશો સમજાવો: (i) date (ii) top (iii) cmp}

\begin{solutionbox}

{\def\LTcaptype{none} % do not increment counter
\begin{longtable}[]{@{}lll@{}}
\toprule\noalign{}
આદેશ & હેતુ & સિન્ટેક્સ \\
\midrule\noalign{}
\endhead
\bottomrule\noalign{}
\endlastfoot
date & સિસ્ટમ તારીખ/સમય દર્શાવે અથવા સેટ કરે છે & date [વિકલ્પો]
[ફોર્મેટ] \\
top & ચાલતી પ્રક્રિયાઓ ડાયનેમિક રીતે દર્શાવે છે & top [વિકલ્પો] \\
cmp & બે ફાઇલોની બાઇટ બાઇ સરખામણી કરે છે & cmp [વિકલ્પો] file1 file2 \\
\end{longtable}
}

\textbf{આદેશની વિગતો:}

\textbf{1. date આદેશ:}

\begin{itemize}
\tightlist
\item
  \textbf{કાર્ય}: વર્તમાન સિસ્ટમ તારીખ અને સમય દર્શાવે છે
\item
  \textbf{ઉદાહરણો}: \texttt{date}, \texttt{date\ +\%Y-\%m-\%d},
  \texttt{date\ +\%H:\%M:\%S}
\item
  \textbf{ફોર્મેટિંગ}: + સિમ્બોલ્સ વાપરીને કસ્ટમ આઉટપુટ ફોર્મેટ્સ
\end{itemize}

\textbf{2. top આદેશ:}

\begin{itemize}
\tightlist
\item
  \textbf{કાર્ય}: સિસ્ટમ પ્રક્રિયાઓ અને રિસોર્સ ઉપયોગનું રીયલ-ટાઇમ ડિસ્પ્લે
\item
  \textbf{ઇન્ટરેક્ટિવ}: બહાર નીકળવા માટે `q' દબાવો, પ્રક્રિયા મારવા માટે `k'
\item
  \textbf{માહિતી}: CPU ઉપયોગ, મેમરી ઉપયોગ, પ્રક્રિયા યાદી
\end{itemize}

\textbf{3. cmp આદેશ:}

\begin{itemize}
\tightlist
\item
  \textbf{કાર્ય}: બે ફાઇલોની સરખામણી કરે છે અને તફાવતોની જાણ કરે છે
\item
  \textbf{આઉટપુટ}: પ્રથમ અલગ બાઇટ પોઝિશન દર્શાવે છે
\item
  \textbf{વિકલ્પો}: \texttt{-s} (મૌન), \texttt{-l} (વર્બોઝ લિસ્ટિંગ)
\end{itemize}

\end{solutionbox}
\begin{mnemonicbox}
``DTC - Date/time, Task monitor, Compare files''

\end{mnemonicbox}
\subsection*{પ્રશ્ન 5(b) OR [4
marks]}\label{q5b}

\textbf{લિનક્સના ઇન્સ્ટોલેશન સ્ટેપ્સ સમજાવો.}

\begin{solutionbox}

{\def\LTcaptype{none} % do not increment counter
\begin{longtable}[]{@{}ll@{}}
\toprule\noalign{}
સ્ટેપ & વર્ણન \\
\midrule\noalign{}
\endhead
\bottomrule\noalign{}
\endlastfoot
1. ISO ડાઉનલોડ કરો & લિનક્સ ડિસ્ટ્રિબ્યુશન ઇમેજ ફાઇલ મેળવો \\
2. બૂટેબલ મીડિયા બનાવો & ISO ને DVD અથવા USB ડ્રાઇવમાં બર્ન કરો \\
3. મીડિયાથી બૂટ કરો & ઇન્સ્ટોલેશન મીડિયાથી કમ્પ્યુટર શરૂ કરો \\
4. ઇન્સ્ટોલેશન પ્રકાર પસંદ કરો & OS સાથે ઇન્સ્ટોલ અથવા રિપ્લેસ પસંદ કરો \\
5. પાર્ટિશન સેટઅપ & ડિસ્ક પાર્ટિશન્સ કન્ફિગર કરો \\
6. યુઝર કન્ફિગરેશન & યુઝર એકાઉન્ટ અને પાસવર્ડ બનાવો \\
7. પેકેજ પસંદગી & ઇન્સ્ટોલ કરવા માટે સોફ્ટવેર પેકેજ પસંદ કરો \\
8. ઇન્સ્ટોલેશન પ્રક્રિયા & ફાઇલો કોપી કરો અને સિસ્ટમ કન્ફિગર કરો \\
9. સિસ્ટમ રીબૂટ કરો & નવા લિનક્સ ઇન્સ્ટોલેશનમાં પુનઃશરૂ કરો \\
\end{longtable}
}

\textbf{પ્રી-ઇન્સ્ટોલેશન આવશ્યકતાઓ:}

\begin{itemize}
\tightlist
\item
  \textbf{હાર્ડવેર સુસંગતતા}: સિસ્ટમ આવશ્યકતાઓ ચકાસો
\item
  \textbf{ડેટા બેકઅપ}: ઇન્સ્ટોલેશન પહેલાં મહત્વપૂર્ણ ફાઇલો સુરક્ષિત કરો
\item
  \textbf{ઇન્ટરનેટ કનેક્શન}: અપડેટ્સ અને વધારાના પેકેજ માટે
\end{itemize}

\textbf{ઇન્સ્ટોલેશન પ્રક્રિયા ફ્લો:}

\begin{center}
\textbf{Mermaid Diagram (Code)}
\begin{verbatim}
{Shaded}
{Highlighting}[]
graph LR
    A[ISO ડાઉનલોડ] {-{-}{} B[બૂટેબલ મીડિયા બનાવો]}
    B {-{-}{} C[મીડિયાથી બૂટ]}
    C {-{-}{} D[ભાષા/કીબોર્ડ સેટઅપ]}
    D {-{-}{} E[ડિસ્ક પાર્ટિશનિંગ]}
    E {-{-}{} F[યુઝર એકાઉન્ટ સેટઅપ]}
    F {-{-}{} G[પેકેજ પસંદગી]}
    G {-{-}{} H[સિસ્ટમ ઇન્સ્ટોલ]}
    H {-{-}{} I[બૂટલોડર કન્ફિગર]}
    I {-{-}{} J[લિનક્સમાં રીબૂટ]}
{Highlighting}
{Shaded}
\end{verbatim}
\end{center}

\textbf{પોસ્ટ-ઇન્સ્ટોલેશન કાર્યો:}

\begin{itemize}
\tightlist
\item
  \textbf{સિસ્ટમ અપડેટ્સ}: નવીનતમ સિક્યોરિટી પેચ ઇન્સ્ટોલ કરો
\item
  \textbf{ડ્રાઇવર ઇન્સ્ટોલેશન}: હાર્ડવેર ડ્રાઇવર્સ કન્ફિગર કરો
\item
  \textbf{સોફ્ટવેર ઇન્સ્ટોલેશન}: આવશ્યક એપ્લિકેશન્સ ઉમેરો
\end{itemize}

\textbf{સામાન્ય પાર્ટિશન સ્કીમ:}

\begin{itemize}
\tightlist
\item
  \texttt{/} (રૂટ): સિસ્ટમ ફાઇલો માટે ન્યૂનતમ 20GB
\item
  \texttt{/home}: યુઝર ડેટા સ્ટોરેજ
\item
  \texttt{swap}: વર્ચ્યુઅલ મેમરી માટે 1-2x RAM કદ
\end{itemize}

\end{solutionbox}
\begin{mnemonicbox}
``DCBCPUPI - Download, Create media, Boot, Choose
type, Partition, User setup, Package selection, Install''

\end{mnemonicbox}
\subsection*{પ્રશ્ન 5(c) OR [7
marks]}\label{q5c}

\textbf{N સંખ્યાઓનો સરવાળો અને સરેરાશ શોધવા માટે શેલ સ્ક્રિપ્ટ લખો.}

\begin{solutionbox}

\begin{verbatim}
\#!/bin/bash
\# N સંખ્યાઓનો સરવાળો અને સરેરાશ શોધવા માટે સ્ક્રિપ્ટ

echo "તમે કેટલી સંખ્યાઓ દાખલ કરવા માંગો છો?"
read n

\# ઇનપુટ ચકાસો
if ! [[ $n ={} \^{}[0{-}9]+$ ]] || [ $n {-le} 0 ]; then
    echo "ભૂલ: કૃપા કરીને સકારાત્મક પૂર્ણાંક દાખલ કરો"
    exit 1
fi

sum=0
echo "$n સંખ્યાઓ દાખલ કરો:"

\# N સંખ્યાઓ વાંચો અને સરવાળો કરો
for ((i=1; i{=}n; i++)); do
    echo {-n} "સંખ્યા $i દાખલ કરો: "
    read number
    
    \# દરેક સંખ્યા ચકાસો
    if ! [[ $number ={} \^{}{-}?[0{-}9]+([.][0{-}9]+)?$ ]]; then
        echo "ભૂલ: અયોગ્ય સંખ્યા ફોર્મેટ"
        exit 1
    fi
    
    sum=$(echo "$sum + $number" | bc {-l})
done

\# સરેરાશ કાઢો
average=$(echo "scale=2; $sum / $n" | bc {-l})

\# પરિણામો દર્શાવો
echo ""
echo "પરિણામો:"
echo "========="
echo "સંખ્યાઓની ગણતરી: $n"
echo "સરવાળો: $sum"
echo "સરેરાશ: $average"

\# વધારાના આંકડા
echo ""
echo "સારાંશ:"
echo "કુલ પ્રોસેસ કરેલી સંખ્યાઓ: $n"
echo "બધી સંખ્યાઓનો સરવાળો: $sum"
echo "સરેરાશ કિંમત: $average"
\end{verbatim}

\textbf{અલ્ગોરિધમ લક્ષણો:}

\begin{itemize}
\tightlist
\item
  \textbf{ઇનપુટ વેલિડેશન}: સકારાત્મક ગણતરી અને વેલિડ સંખ્યાઓ માટે ચકાસે છે
\item
  \textbf{લવચીક ઇનપુટ}: પૂર્ણાંક અને દશાંશ સંખ્યાઓ સ્વીકારે છે
\item
  \textbf{પ્રિસિઝન હેન્ડલિંગ}: ચોક્કસ અંકગણિત માટે bc કેલ્ક્યુલેટરનો ઉપયોગ
\item
  \textbf{એરર હેન્ડલિંગ}: દરેક ઇનપુટ ચકાસે છે અને એરર મેસેજ પ્રદાન કરે છે
\end{itemize}

\textbf{ઉદાહરણ એક્ઝિક્યુશન:}

\begin{verbatim}
તમે કેટલી સંખ્યાઓ દાખલ કરવા માંગો છો? 5
સંખ્યા 1 દાખલ કરો: 10
સંખ્યા 2 દાખલ કરો: 20
સંખ્યા 3 દાખલ કરો: 30
સંખ્યા 4 દાખલ કરો: 40
સંખ્યા 5 દાખલ કરો: 50

પરિણામો:
=========
સંખ્યાઓની ગણતરી: 5
સરવાળો: 150
સરેરાશ: 30.00
\end{verbatim}

\textbf{વૈકલ્પિક સરળ વર્ઝન:}

\begin{verbatim}
\#!/bin/bash
read {-p} "ગણતરી દાખલ કરો: " n
sum=0
for ((i=1; i{=}n; i++)); do
    read {-p} "સંખ્યા $i: " num
    sum=$((sum + num))
done
echo "સરવાળો: $sum"
echo "સરેરાશ: $((sum / n))"
\end{verbatim}

\textbf{મુખ્ય પ્રોગ્રામિંગ કન્સેપ્ટ્સ:}

\begin{itemize}
\tightlist
\item
  \textbf{લૂપ કંટ્રોલ}: N વખત પુનરાવર્તન માટે for લૂપ
\item
  \textbf{અંકગણિત ઓપરેશન્સ}: ઉમેરો અને ભાગાકાર
\item
  \textbf{ઇનપુટ/આઉટપુટ}: યુઝર ઇનપુટ વાંચવું અને પરિણામો દર્શાવવા
\item
  \textbf{ડેટા વેલિડેશન}: ઇનપુટની શુદ્ધતા સુનિશ્ચિત કરવી
\end{itemize}

\end{solutionbox}
\begin{mnemonicbox}
``VLAD - Validate input, Loop for numbers,
Arithmetic calculation, Display results''

\end{mnemonicbox}

\end{document}
