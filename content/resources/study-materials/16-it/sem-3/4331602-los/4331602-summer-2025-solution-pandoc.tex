\documentclass[10pt,a4paper]{article}

% content/resources/templates/preamble.tex
\usepackage[margin=0.6in]{geometry}
\author{Milav Dabgar}
\usepackage{amsmath,amssymb,amsthm}
\usepackage{booktabs}
\usepackage{multirow}
\usepackage{xcolor}
\usepackage{tcolorbox}
\tcbuselibrary{breakable,skins}
\usepackage[colorlinks=true,linkcolor=blue]{hyperref}
\usepackage{titlesec}
\usepackage{enumitem}
\usepackage{tikz}
\usepackage{pgfplots}
\usepackage{circuitikz}
\usepackage[version=4]{mhchem}
\usepackage{longtable}
\usepackage{array}
\usepackage{float}
\usepackage{caption}
\usepackage{listings}

\lstset{
  basicstyle=\small\ttfamily,
  breaklines=true,
  breakatwhitespace=false,
  postbreak=\mbox{\textcolor{red}{$\hookrightarrow$}\space},
  float=false,
  numbers=left,
  numberstyle=\tiny\color{gray},
  numbersep=10pt,
  xleftmargin=2em,
  keywordstyle=\color{blue},
  commentstyle=\color{green!60!black},
  stringstyle=\color{purple},
  backgroundcolor=\color{gray!5},
  showstringspaces=false,
  tabsize=2,
  captionpos=b,
  keepspaces=true,
  columns=flexible
}

\pgfplotsset{compat=1.18}
\usetikzlibrary{shapes,arrows,positioning,calc,patterns,decorations.pathmorphing,decorations.markings,arrows.meta}

% Color scheme
\definecolor{headcolor}{RGB}{0,102,204}
\definecolor{keycolor}{RGB}{220,20,60}
\definecolor{solutioncolor}{RGB}{34,139,34}
\definecolor{mnemoniccolor}{RGB}{148,0,211}
\definecolor{codecolor}{RGB}{0,0,100}

% Spacing
\setlength{\parskip}{3pt}
\setlist[itemize]{nosep}
\setlist[enumerate]{nosep}

% Title formatting
\titleformat{\section}{\Large\bfseries\color{headcolor}}{\thesection}{1em}{}
\titleformat{\subsection}{\large\bfseries\color{headcolor}}{\thesubsection}{1em}{}

% Pandoc tightlist compatibility
\providecommand{\tightlist}{%
  \setlength{\itemsep}{0pt}\setlength{\parskip}{0pt}}

% Pandoc longtable compatibility
\newcounter{none}
\def\thenone{}


% content/resources/templates/english-boxes.tex
% This file is currently empty - it exists to maintain consistency with the import structure.
% Add custom environments here if needed in the future.


\begin{document}

\begin{center}
{\Huge\bfseries\color{headcolor} Subject Name Solutions}\\[5pt]
{\LARGE 4331602 -- Summer 2025}\\[3pt]
{\large Semester 1 Study Material}\\[3pt]
{\normalsize\textit{Detailed Solutions and Explanations}}
\end{center}

\vspace{10pt}

\subsection*{Question 1(a) [3 marks]}\label{q1a}

\textbf{Define Operating System and explain the need of OS.}

\begin{solutionbox}

\textbf{Operating System} is a system software that acts as an
intermediary between computer hardware and application software. It
manages hardware resources and provides services to user programs.

\textbf{Need of Operating System:}

\begin{itemize}
\tightlist
\item
  \textbf{Resource Management}: Manages CPU, memory, storage, and I/O
  devices efficiently
\item
  \textbf{User Interface}: Provides command-line and graphical
  interfaces for user interaction
\item
  \textbf{Program Execution}: Loads and executes user programs safely
\end{itemize}

\end{solutionbox}
\begin{mnemonicbox}
``RUP - Resource, User, Program management''

\end{mnemonicbox}
\subsection*{Question 1(b) [4 marks]}\label{q1b}

\textbf{Write a short note on Process Control Block (PCB).}

\begin{solutionbox}

Process Control Block (PCB) is a data structure maintained by the
operating system for each running process.

{\def\LTcaptype{none} % do not increment counter
\begin{longtable}[]{@{}ll@{}}
\toprule\noalign{}
PCB Component & Description \\
\midrule\noalign{}
\endhead
\bottomrule\noalign{}
\endlastfoot
Process ID & Unique identifier for the process \\
Process State & Current state (ready, running, waiting) \\
Program Counter & Address of next instruction to execute \\
CPU Registers & Values of CPU registers when process is suspended \\
Memory Management & Base and limit registers, page tables \\
I/O Status & List of open files and I/O devices \\
\end{longtable}
}

\textbf{Key Functions:}

\begin{itemize}
\tightlist
\item
  \textbf{Process Identification}: Stores unique process ID and parent
  process ID
\item
  \textbf{State Information}: Maintains current execution state and
  context
\item
  \textbf{Resource Allocation}: Tracks allocated resources and memory
  usage
\end{itemize}

\end{solutionbox}
\begin{mnemonicbox}
``PIS - Process ID, Information, State tracking''

\end{mnemonicbox}
\subsection*{Question 1(c) [7 marks]}\label{q1c}

\textbf{List different types of Operating systems. Explain the working
of batch operating systems with a suitable example.}

\begin{solutionbox}

\textbf{Types of Operating Systems:}

{\def\LTcaptype{none} % do not increment counter
\begin{longtable}[]{@{}ll@{}}
\toprule\noalign{}
Type & Description \\
\midrule\noalign{}
\endhead
\bottomrule\noalign{}
\endlastfoot
Batch OS & Groups similar jobs and executes them together \\
Time-sharing OS & Multiple users share system simultaneously \\
Real-time OS & Provides guaranteed response time \\
Distributed OS & Manages multiple interconnected computers \\
Network OS & Provides network services and resource sharing \\
Mobile OS & Designed for mobile devices \\
\end{longtable}
}

\textbf{Batch Operating System Working:}

\begin{verbatim}
+{-{-}{-}{-}{-}{-}{-}{-}{-}{-}+    +{-}{-}{-}{-}{-}{-}{-}{-}{-}{-}+    +{-}{-}{-}{-}{-}{-}{-}{-}{-}{-}+    +{-}{-}{-}{-}{-}{-}{-}{-}{-}{-}+}
|  Job 1   |    |  Job 2   |    |  Job 3   |    |  Job 4   |
| COBOL    | {- | FORTRAN  | {-} |   C++    | {-} |  JAVA    |}
| Programs |    | Programs |    | Programs |    | Programs |
+{-{-}{-}{-}{-}{-}{-}{-}{-}{-}+    +{-}{-}{-}{-}{-}{-}{-}{-}{-}{-}+    +{-}{-}{-}{-}{-}{-}{-}{-}{-}{-}+    +{-}{-}{-}{-}{-}{-}{-}{-}{-}{-}+}
     |               |               |               |
     v               v               v               v
+{-{-}{-}{-}{-}{-}{-}{-}{-}{-}{-}{-}{-}{-}{-}{-}{-}{-}{-}{-}{-}{-}{-}{-}{-}{-}{-}{-}{-}{-}{-}{-}{-}{-}{-}{-}{-}{-}{-}{-}{-}{-}{-}{-}{-}{-}{-}{-}{-}{-}{-}{-}{-}{-}{-}{-}+}
|              Batch Queue Manager                       |
+{-{-}{-}{-}{-}{-}{-}{-}{-}{-}{-}{-}{-}{-}{-}{-}{-}{-}{-}{-}{-}{-}{-}{-}{-}{-}{-}{-}{-}{-}{-}{-}{-}{-}{-}{-}{-}{-}{-}{-}{-}{-}{-}{-}{-}{-}{-}{-}{-}{-}{-}{-}{-}{-}{-}{-}+}
     |
     v
+{-{-}{-}{-}{-}{-}{-}{-}{-}{-}+    +{-}{-}{-}{-}{-}{-}{-}{-}{-}{-}+    +{-}{-}{-}{-}{-}{-}{-}{-}{-}{-}+}
|   CPU    |    |  Memory  |    |   I/O    |
| Executor | {- | Manager  | {-} | Handler  |}
+{-{-}{-}{-}{-}{-}{-}{-}{-}{-}+    +{-}{-}{-}{-}{-}{-}{-}{-}{-}{-}+    +{-}{-}{-}{-}{-}{-}{-}{-}{-}{-}+}
\end{verbatim}

\textbf{Example}: Bank transaction processing where all day's
transactions are collected and processed together at night for
efficiency.

\textbf{Key Features:}

\begin{itemize}
\tightlist
\item
  \textbf{Job Grouping}: Similar jobs executed together for efficiency
\item
  \textbf{No User Interaction}: Jobs run without user intervention once
  submitted
\item
  \textbf{High Throughput}: Maximizes system utilization
\end{itemize}

\end{solutionbox}
\begin{mnemonicbox}
``JNH - Jobs grouped, No interaction, High
throughput''

\end{mnemonicbox}
\subsection*{Question 1(c) OR [7
marks]}\label{q1c}

\textbf{List different types of Operating systems. Explain the real time
operating systems in detail.}

\begin{solutionbox}

\textbf{Types of Operating Systems:} (Same table as above)

\textbf{Real-Time Operating System (RTOS):}

Real-time OS provides guaranteed response within specified time
constraints for critical applications.

\textbf{Types of RTOS:}

{\def\LTcaptype{none} % do not increment counter
\begin{longtable}[]{@{}
  >{\raggedright\arraybackslash}p{(\linewidth - 4\tabcolsep) * \real{0.2400}}
  >{\raggedright\arraybackslash}p{(\linewidth - 4\tabcolsep) * \real{0.4000}}
  >{\raggedright\arraybackslash}p{(\linewidth - 4\tabcolsep) * \real{0.3600}}@{}}
\toprule\noalign{}
\begin{minipage}[b]{\linewidth}\raggedright
Type
\end{minipage} & \begin{minipage}[b]{\linewidth}\raggedright
Deadline
\end{minipage} & \begin{minipage}[b]{\linewidth}\raggedright
Example
\end{minipage} \\
\midrule\noalign{}
\endhead
\bottomrule\noalign{}
\endlastfoot
Hard Real-time & Must meet deadline & Air traffic control, pacemaker \\
Soft Real-time & Can tolerate some delay & Video streaming, online
gaming \\
Firm Real-time & Occasional deadline miss acceptable & Live audio
processing \\
\end{longtable}
}

\textbf{Characteristics:}

\begin{itemize}
\tightlist
\item
  \textbf{Deterministic}: Predictable response time for all operations
\item
  \textbf{Priority-based Scheduling}: High-priority tasks get immediate
  attention
\item
  \textbf{Minimal Interrupt Latency}: Fast context switching
  capabilities
\item
  \textbf{Memory Management}: Real-time memory allocation without delays
\end{itemize}

\textbf{Applications:}

\begin{itemize}
\tightlist
\item
  Medical devices, automotive systems, industrial automation, aerospace
  control systems
\end{itemize}

\end{solutionbox}
\begin{mnemonicbox}
``DPMA - Deterministic, Priority-based, Minimal
latency, Applications critical''

\end{mnemonicbox}
\subsection*{Question 2(a) [3 marks]}\label{q2a}

\textbf{Differentiate between program and process.}

\begin{solutionbox}

{\def\LTcaptype{none} % do not increment counter
\begin{longtable}[]{@{}lll@{}}
\toprule\noalign{}
Aspect & Program & Process \\
\midrule\noalign{}
\endhead
\bottomrule\noalign{}
\endlastfoot
Definition & Static code stored on disk & Program in execution \\
State & Passive entity & Active entity \\
Memory & No memory allocation & Allocated memory space \\
Lifetime & Permanent until deleted & Temporary during execution \\
Resources & No resource consumption & Consumes CPU, memory, I/O \\
\end{longtable}
}

\textbf{Key Differences:}

\begin{itemize}
\tightlist
\item
  \textbf{Static vs Dynamic}: Program is static file, process is dynamic
  execution
\item
  \textbf{Resource Usage}: Process consumes system resources, program
  doesn't
\item
  \textbf{Multiple Instances}: One program can create multiple processes
\end{itemize}

\end{solutionbox}
\begin{mnemonicbox}
``SDR - Static vs Dynamic, Resource usage, Multiple
instances''

\end{mnemonicbox}
\subsection*{Question 2(b) [4 marks]}\label{q2b}

\textbf{Explain the different states of a process with the help of a
process state diagram.}

\begin{solutionbox}

\begin{verbatim}
stateDiagram{-v2}
    direction LR
    [*] {-{-} New: Program loaded}
    New {-{-} Ready: Admitted to ready queue}
    Ready {-{-} Running: CPU allocated}
    Running {-{-} Ready: Time quantum expires}
    Running {-{-} Waiting: I/O request}
    Waiting {-{-} Ready: I/O completed}
    Running {-{-} Terminated: Process completes}
    Terminated {-{-} [*]: Resources deallocated}
\end{verbatim}

\textbf{Process States:}

{\def\LTcaptype{none} % do not increment counter
\begin{longtable}[]{@{}ll@{}}
\toprule\noalign{}
State & Description \\
\midrule\noalign{}
\endhead
\bottomrule\noalign{}
\endlastfoot
New & Process being created \\
Ready & Waiting for CPU assignment \\
Running & Currently executing on CPU \\
Waiting & Blocked for I/O or event \\
Terminated & Process execution completed \\
\end{longtable}
}

\textbf{State Transitions:}

\begin{itemize}
\tightlist
\item
  \textbf{Ready to Running}: Process scheduler assigns CPU
\item
  \textbf{Running to Ready}: Time slice expires or higher priority
  process arrives
\item
  \textbf{Running to Waiting}: Process requests I/O operation
\item
  \textbf{Waiting to Ready}: I/O operation completes
\end{itemize}

\end{solutionbox}
\begin{mnemonicbox}
``NRWRT - New, Ready, Waiting, Running, Terminated
states''

\end{mnemonicbox}
\subsection*{Question 2(c) [7 marks]}\label{q2c}

\textbf{Describe the Round Robin algorithm. Calculate the average
waiting time \& average turn-around time along with Gantt chart for the
given data. Consider context switch = 01 ms and quantum time = 04 ms.}

\begin{solutionbox}

\textbf{Round Robin Algorithm:} Round Robin is a preemptive scheduling
algorithm where each process gets equal CPU time (quantum) in circular
manner.

\textbf{Given Data:}

\begin{itemize}
\tightlist
\item
  Quantum Time = 4 ms
\item
  Context Switch = 1 ms
\end{itemize}

{\def\LTcaptype{none} % do not increment counter
\begin{longtable}[]{@{}lll@{}}
\toprule\noalign{}
Process & Arrival Time & Burst Time \\
\midrule\noalign{}
\endhead
\bottomrule\noalign{}
\endlastfoot
P1 & 0 & 8 \\
P2 & 3 & 3 \\
P3 & 1 & 10 \\
P4 & 4 & 5 \\
\end{longtable}
}

\textbf{Gantt Chart:}

\begin{verbatim}
0   4  5   8  9  13 14  18 19  22 23  26 27  29
|P1 |CS|P3|CS|P1|CS|P2|CS|P3|CS|P4|CS|P3|CS|P4|
\end{verbatim}

\textbf{Calculations:}

{\def\LTcaptype{none} % do not increment counter
\begin{longtable}[]{@{}llll@{}}
\toprule\noalign{}
Process & Completion Time & Turnaround Time & Waiting Time \\
\midrule\noalign{}
\endhead
\bottomrule\noalign{}
\endlastfoot
P1 & 13 & 13 & 5 \\
P2 & 18 & 15 & 12 \\
P3 & 26 & 25 & 15 \\
P4 & 29 & 25 & 20 \\
\end{longtable}
}

\textbf{Average Waiting Time = (5 + 12 + 15 + 20) / 4 = 13 ms}
\textbf{Average Turnaround Time = (13 + 15 + 25 + 25) / 4 = 19.5 ms}

\textbf{Key Features:}

\begin{itemize}
\tightlist
\item
  \textbf{Fair Scheduling}: Each process gets equal CPU time
\item
  \textbf{Preemptive}: Running process is interrupted after quantum
  expires
\item
  \textbf{Context Switching}: Overhead included in calculations
\end{itemize}

\end{solutionbox}
\begin{mnemonicbox}
``FPC - Fair, Preemptive, Context switching
overhead''

\end{mnemonicbox}
\subsection*{Question 2(a) OR [3
marks]}\label{q2a}

\textbf{Differentiate: CPU bound process v/s I/O bound process.}

\begin{solutionbox}

{\def\LTcaptype{none} % do not increment counter
\begin{longtable}[]{@{}
  >{\raggedright\arraybackslash}p{(\linewidth - 4\tabcolsep) * \real{0.1739}}
  >{\raggedright\arraybackslash}p{(\linewidth - 4\tabcolsep) * \real{0.4130}}
  >{\raggedright\arraybackslash}p{(\linewidth - 4\tabcolsep) * \real{0.4130}}@{}}
\toprule\noalign{}
\begin{minipage}[b]{\linewidth}\raggedright
Aspect
\end{minipage} & \begin{minipage}[b]{\linewidth}\raggedright
CPU Bound Process
\end{minipage} & \begin{minipage}[b]{\linewidth}\raggedright
I/O Bound Process
\end{minipage} \\
\midrule\noalign{}
\endhead
\bottomrule\noalign{}
\endlastfoot
Primary Activity & Intensive calculations & Frequent I/O operations \\
CPU Usage & High CPU utilization & Low CPU utilization \\
Burst Time & Long CPU bursts & Short CPU bursts \\
Waiting Time & Less I/O waiting & More I/O waiting \\
Examples & Mathematical calculations, image processing & File
operations, database queries \\
\end{longtable}
}

\textbf{Key Differences:}

\begin{itemize}
\tightlist
\item
  \textbf{Resource Consumption}: CPU-bound uses more processor,
  I/O-bound uses more input/output
\item
  \textbf{Performance Impact}: CPU-bound affected by processor speed,
  I/O-bound by storage speed
\item
  \textbf{Scheduling Priority}: Different algorithms favor each type
  differently
\end{itemize}

\end{solutionbox}
\begin{mnemonicbox}
``CIR - CPU intensive, I/O intensive, Resource usage
differs''

\end{mnemonicbox}
\subsection*{Question 2(b) OR [4
marks]}\label{q2b}

\textbf{What is a deadlock? Explain the necessary conditions for a
deadlock to occur.}

\begin{solutionbox}

\textbf{Deadlock} is a situation where two or more processes are
permanently blocked, each waiting for resources held by others.

\textbf{Necessary Conditions (Coffman Conditions):}

{\def\LTcaptype{none} % do not increment counter
\begin{longtable}[]{@{}ll@{}}
\toprule\noalign{}
Condition & Description \\
\midrule\noalign{}
\endhead
\bottomrule\noalign{}
\endlastfoot
Mutual Exclusion & Resources cannot be shared simultaneously \\
Hold and Wait & Process holds resources while waiting for others \\
No Preemption & Resources cannot be forcibly taken from processes \\
Circular Wait & Circular chain of processes waiting for resources \\
\end{longtable}
}

\textbf{Example Scenario:}

\begin{verbatim}
Process A {-{-}{-}{-}holds{-}{-}{-}{-} Resource 1}
    |                        \^{}
    |                        |
    v                        |
waits for              Process B
Resource 2 {{-}{-}{-}holds{-}{-}{-}{-}{-}{-}{-}{-}{-}{-}|}
\end{verbatim}

\textbf{Deadlock Prevention:}

\begin{itemize}
\tightlist
\item
  \textbf{Eliminate Mutual Exclusion}: Make resources shareable when
  possible
\item
  \textbf{Prevent Hold and Wait}: Require all resources at once
\item
  \textbf{Allow Preemption}: Forcibly take resources when needed
\item
  \textbf{Prevent Circular Wait}: Order resources and request in
  sequence
\end{itemize}

\end{solutionbox}
\begin{mnemonicbox}
``MHNC - Mutual exclusion, Hold-wait, No preemption,
Circular wait''

\end{mnemonicbox}
\subsection*{Question 2(c) OR [7
marks]}\label{q2c}

\textbf{Describe the FCFS algorithm. Calculate the average waiting time
and average turn-around time along with Gantt chart for the given data.}

\begin{solutionbox}

\textbf{First Come First Serve (FCFS) Algorithm:} FCFS is a
non-preemptive scheduling algorithm where processes are executed in
arrival order.

\textbf{Given Data:}

{\def\LTcaptype{none} % do not increment counter
\begin{longtable}[]{@{}lll@{}}
\toprule\noalign{}
Process & Arrival Time & Burst Time \\
\midrule\noalign{}
\endhead
\bottomrule\noalign{}
\endlastfoot
P1 & 0 & 7 \\
P2 & 3 & 6 \\
P3 & 5 & 9 \\
P4 & 6 & 4 \\
\end{longtable}
}

\textbf{Gantt Chart:}

\begin{verbatim}
0        7        13        22        26
|   P1   |   P2   |    P3   |   P4   |
\end{verbatim}

\textbf{Calculations:}

{\def\LTcaptype{none} % do not increment counter
\begin{longtable}[]{@{}
  >{\raggedright\arraybackslash}p{(\linewidth - 8\tabcolsep) * \real{0.1304}}
  >{\raggedright\arraybackslash}p{(\linewidth - 8\tabcolsep) * \real{0.1739}}
  >{\raggedright\arraybackslash}p{(\linewidth - 8\tabcolsep) * \real{0.2464}}
  >{\raggedright\arraybackslash}p{(\linewidth - 8\tabcolsep) * \real{0.2464}}
  >{\raggedright\arraybackslash}p{(\linewidth - 8\tabcolsep) * \real{0.2029}}@{}}
\toprule\noalign{}
\begin{minipage}[b]{\linewidth}\raggedright
Process
\end{minipage} & \begin{minipage}[b]{\linewidth}\raggedright
Start Time
\end{minipage} & \begin{minipage}[b]{\linewidth}\raggedright
Completion Time
\end{minipage} & \begin{minipage}[b]{\linewidth}\raggedright
Turnaround Time
\end{minipage} & \begin{minipage}[b]{\linewidth}\raggedright
Waiting Time
\end{minipage} \\
\midrule\noalign{}
\endhead
\bottomrule\noalign{}
\endlastfoot
P1 & 0 & 7 & 7 & 0 \\
P2 & 7 & 13 & 10 & 4 \\
P3 & 13 & 22 & 17 & 8 \\
P4 & 22 & 26 & 20 & 16 \\
\end{longtable}
}

\textbf{Average Waiting Time = (0 + 4 + 8 + 16) / 4 = 7 ms}
\textbf{Average Turnaround Time = (7 + 10 + 17 + 20) / 4 = 13.5 ms}

\textbf{Characteristics:}

\begin{itemize}
\tightlist
\item
  \textbf{Simple Implementation}: Easy to understand and implement
\item
  \textbf{Non-preemptive}: Once started, process runs to completion
\item
  \textbf{Convoy Effect}: Short processes wait for long processes
\end{itemize}

\end{solutionbox}
\begin{mnemonicbox}
``SNC - Simple, Non-preemptive, Convoy effect
possible''

\end{mnemonicbox}
\subsection*{Question 3(a) [3 marks]}\label{q3a}

\textbf{Explain single-level directory structure.}

\begin{solutionbox}

Single-level directory structure is the simplest file organization where
all files are stored in one directory.

\begin{verbatim}
        Root Directory
    +{-{-}{-}{-}{-}{-}{-}{-}{-}{-}{-}{-}{-}{-}{-}{-}{-}{-}{-}+}
    | file1.txt         |
    | program.exe       |
    | data.dat          |
    | image.jpg         |
    | document.pdf      |
    +{-{-}{-}{-}{-}{-}{-}{-}{-}{-}{-}{-}{-}{-}{-}{-}{-}{-}{-}+}
\end{verbatim}

\textbf{Characteristics:}

\begin{itemize}
\tightlist
\item
  \textbf{Simple Structure}: All files in one location
\item
  \textbf{Unique Names}: Each file must have unique name
\item
  \textbf{No Organization}: No grouping or categorization possible
\end{itemize}

\textbf{Limitations:}

\begin{itemize}
\tightlist
\item
  Name collision when multiple users create files with same names
\item
  Difficult to organize large number of files
\item
  No privacy or access control between users
\end{itemize}

\end{solutionbox}
\begin{mnemonicbox}
``SUN - Simple, Unique names, No organization''

\end{mnemonicbox}
\subsection*{Question 3(b) [4 marks]}\label{q3b}

\textbf{Explain the different file attributes.}

\begin{solutionbox}

File attributes are metadata that provide information about files stored
in the file system.

{\def\LTcaptype{none} % do not increment counter
\begin{longtable}[]{@{}ll@{}}
\toprule\noalign{}
Attribute & Description \\
\midrule\noalign{}
\endhead
\bottomrule\noalign{}
\endlastfoot
Name & Human-readable file identifier \\
Type & File format (executable, text, image) \\
Size & Current file size in bytes \\
Location & Physical address on storage device \\
Protection & Access permissions (read, write, execute) \\
Time stamps & Creation, modification, access times \\
Owner & User who created the file \\
\end{longtable}
}

\textbf{Common File Attributes:}

\begin{itemize}
\tightlist
\item
  \textbf{Identifier}: Unique number for file system reference
\item
  \textbf{Type Information}: MIME type or file extension
\item
  \textbf{Size and Allocation}: Current size and allocated space
\item
  \textbf{Access Control}: User permissions and group access rights
\end{itemize}

\textbf{Storage Location:} File attributes are typically stored in
directory entries or file allocation tables.

\end{solutionbox}
\begin{mnemonicbox}
``NTSLPTO - Name, Type, Size, Location, Protection,
Time, Owner''

\end{mnemonicbox}
\subsection*{Question 3(c) [7 marks]}\label{q3c}

\textbf{List the different file allocation methods and explain
contiguous allocation with necessary diagram.}

\begin{solutionbox}

\textbf{File Allocation Methods:}

{\def\LTcaptype{none} % do not increment counter
\begin{longtable}[]{@{}ll@{}}
\toprule\noalign{}
Method & Description \\
\midrule\noalign{}
\endhead
\bottomrule\noalign{}
\endlastfoot
Contiguous & Files stored in consecutive blocks \\
Linked & Files stored using linked list of blocks \\
Indexed & Uses index block to point to data blocks \\
\end{longtable}
}

\textbf{Contiguous Allocation:}

In contiguous allocation, each file occupies a set of contiguous blocks
on the disk.

\begin{verbatim}
Disk Blocks:
+{-{-}{-}+{-}{-}{-}+{-}{-}{-}+{-}{-}{-}+{-}{-}{-}+{-}{-}{-}+{-}{-}{-}+{-}{-}{-}+{-}{-}{-}+{-}{-}{-}+}
| 0 | 1 | 2 | 3 | 4 | 5 | 6 | 7 | 8 | 9 |
+{-{-}{-}+{-}{-}{-}+{-}{-}{-}+{-}{-}{-}+{-}{-}{-}+{-}{-}{-}+{-}{-}{-}+{-}{-}{-}+{-}{-}{-}+{-}{-}{-}+}
|   |File A |   |  File B   |   |File C |
|   |  2{-3  |   |   5{-}7     |   |  9    |}
+{-{-}{-}+{-}{-}{-}+{-}{-}{-}+{-}{-}{-}+{-}{-}{-}+{-}{-}{-}+{-}{-}{-}+{-}{-}{-}+{-}{-}{-}+{-}{-}{-}+}

Directory Table:
+{-{-}{-}{-}{-}{-}{-}{-}{-}{-}+{-}{-}{-}{-}{-}{-}{-}+{-}{-}{-}{-}{-}{-}{-}{-}+}
| Filename | Start | Length |
+{-{-}{-}{-}{-}{-}{-}{-}{-}{-}+{-}{-}{-}{-}{-}{-}{-}+{-}{-}{-}{-}{-}{-}{-}{-}+}
| File A   |   2   |   2    |
| File B   |   5   |   3    |
| File C   |   9   |   1    |
+{-{-}{-}{-}{-}{-}{-}{-}{-}{-}+{-}{-}{-}{-}{-}{-}{-}+{-}{-}{-}{-}{-}{-}{-}{-}+}
\end{verbatim}

\textbf{Advantages:}

\begin{itemize}
\tightlist
\item
  \textbf{Fast Access}: Direct calculation of block addresses
\item
  \textbf{Minimal Seek Time}: Consecutive blocks reduce head movement
\item
  \textbf{Simple Implementation}: Easy to implement and manage
\end{itemize}

\textbf{Disadvantages:}

\begin{itemize}
\tightlist
\item
  \textbf{External Fragmentation}: Unused spaces between files
\item
  \textbf{File Size Limitation}: Difficult to extend files
\item
  \textbf{Compaction Needed}: Periodic reorganization required
\end{itemize}

\end{solutionbox}
\begin{mnemonicbox}
``FMS vs EFC - Fast access, Minimal seek, Simple vs
External fragmentation, File size limits, Compaction needed''

\end{mnemonicbox}
\subsection*{Question 3(a) OR [3
marks]}\label{q3a}

\textbf{Explain the different types of Linux file systems in brief.}

\begin{solutionbox}

{\def\LTcaptype{none} % do not increment counter
\begin{longtable}[]{@{}ll@{}}
\toprule\noalign{}
File System & Description \\
\midrule\noalign{}
\endhead
\bottomrule\noalign{}
\endlastfoot
ext2 & Second extended filesystem, no journaling \\
ext3 & Third extended filesystem with journaling \\
ext4 & Fourth extended filesystem, improved performance \\
XFS & High-performance 64-bit journaling filesystem \\
Btrfs & B-tree filesystem with advanced features \\
ZFS & Copy-on-write filesystem with data integrity \\
\end{longtable}
}

\textbf{Key Features:}

\begin{itemize}
\tightlist
\item
  \textbf{Journaling}: ext3, ext4, XFS provide crash recovery
\item
  \textbf{Performance}: ext4, XFS optimized for large files
\item
  \textbf{Advanced Features}: Btrfs, ZFS offer snapshots and compression
\end{itemize}

\textbf{Selection Criteria:} Different filesystems suit different use
cases based on performance, reliability, and feature requirements.

\end{solutionbox}
\begin{mnemonicbox}
``EEXBZ - ext2/3/4, XFS, Btrfs, ZFS options''

\end{mnemonicbox}
\subsection*{Question 3(b) OR [4
marks]}\label{q3b}

\textbf{Explain the different file operations.}

\begin{solutionbox}

{\def\LTcaptype{none} % do not increment counter
\begin{longtable}[]{@{}ll@{}}
\toprule\noalign{}
Operation & Description \\
\midrule\noalign{}
\endhead
\bottomrule\noalign{}
\endlastfoot
Create & Make new file with specified name and attributes \\
Open & Prepare file for reading/writing operations \\
Read & Retrieve data from file at current position \\
Write & Store data to file at current position \\
Seek & Move file pointer to specific position \\
Close & Release file resources and update metadata \\
Delete & Remove file and deallocate storage space \\
\end{longtable}
}

\textbf{File Operation Sequence:}

\begin{center}
\textbf{Mermaid Diagram (Code)}
\begin{verbatim}
{Shaded}
{Highlighting}[]
graph LR
    A[Create File] {-{-}{} B[Open File]}
    B {-{-}{} C[Read/Write]}
    C {-{-}{} D[Seek if needed]}
    D {-{-}{} C}
    C {-{-}{} E[Close File]}
    E {-{-}{} F[Delete if needed]}
{Highlighting}
{Shaded}
\end{verbatim}
\end{center}

\textbf{Important Considerations:}

\begin{itemize}
\tightlist
\item
  \textbf{Error Handling}: Each operation can fail and must be checked
\item
  \textbf{Permissions}: User must have appropriate access rights
\item
  \textbf{Concurrent Access}: Multiple processes may access same file
\end{itemize}

\end{solutionbox}
\begin{mnemonicbox}
``CORWSCD - Create, Open, Read, Write, Seek, Close,
Delete''

\end{mnemonicbox}
\subsection*{Question 3(c) OR [7
marks]}\label{q3c}

\textbf{List the different file allocation methods and explain indexed
allocation with necessary diagram.}

\begin{solutionbox}

\textbf{File Allocation Methods:}

{\def\LTcaptype{none} % do not increment counter
\begin{longtable}[]{@{}ll@{}}
\toprule\noalign{}
Operation & Description \\
\midrule\noalign{}
\endhead
\bottomrule\noalign{}
\endlastfoot
Create & Make new file with specified name and attributes \\
Open & Prepare file for reading/writing operations \\
Read & Retrieve data from file at current position \\
Write & Store data to file at current position \\
Seek & Move file pointer to specific position \\
Close & Release file resources and update metadata \\
Delete & Remove file and deallocate storage space \\
\end{longtable}
}

\textbf{Indexed Allocation:}

In indexed allocation, each file has an index block containing pointers
to data blocks.

\begin{verbatim}
Index Block for File A:
+{-{-}{-}+{-}{-}{-}+{-}{-}{-}+{-}{-}{-}+}
| 2 | 5 | 8 | 9 |
+{-{-}{-}+{-}{-}{-}+{-}{-}{-}+{-}{-}{-}+}
  |   |   |   |
  v   v   v   v
Disk Blocks:
+{-{-}{-}+{-}{-}{-}+{-}{-}{-}+{-}{-}{-}+{-}{-}{-}+{-}{-}{-}+{-}{-}{-}+{-}{-}{-}+{-}{-}{-}+{-}{-}{-}+}
| 0 | 1 | 2 | 3 | 4 | 5 | 6 | 7 | 8 | 9 |
+{-{-}{-}+{-}{-}{-}+{-}{-}{-}+{-}{-}{-}+{-}{-}{-}+{-}{-}{-}+{-}{-}{-}+{-}{-}{-}+{-}{-}{-}+{-}{-}{-}+}
|   |   |FileA|   |   |FileA|   |   |FileA|FileA|

Directory Table:
+{-{-}{-}{-}{-}{-}{-}{-}{-}{-}+{-}{-}{-}{-}{-}{-}{-}{-}{-}{-}{-}{-}{-}+}
| Filename | Index Block |
+{-{-}{-}{-}{-}{-}{-}{-}{-}{-}+{-}{-}{-}{-}{-}{-}{-}{-}{-}{-}{-}{-}{-}+}
| File A   |      1      |
+{-{-}{-}{-}{-}{-}{-}{-}{-}{-}+{-}{-}{-}{-}{-}{-}{-}{-}{-}{-}{-}{-}{-}+}
\end{verbatim}

\textbf{Types of Indexed Allocation:}

\begin{itemize}
\tightlist
\item
  \textbf{Single-level}: One index block per file
\item
  \textbf{Multi-level}: Index blocks point to other index blocks
\item
  \textbf{Combined}: Mix of direct and indirect pointers
\end{itemize}

\textbf{Advantages:}

\begin{itemize}
\tightlist
\item
  \textbf{No External Fragmentation}: Blocks can be anywhere on disk
\item
  \textbf{Dynamic File Size}: Easy to extend files
\item
  \textbf{Fast Random Access}: Direct access to any block
\end{itemize}

\textbf{Disadvantages:}

\begin{itemize}
\tightlist
\item
  \textbf{Index Block Overhead}: Extra space for storing pointers
\item
  \textbf{Multiple Disk Access}: Two accesses needed (index + data)
\item
  \textbf{Small File Inefficiency}: Overhead high for small files
\end{itemize}

\end{solutionbox}
\begin{mnemonicbox}
``NDF vs IMI - No fragmentation, Dynamic size, Fast
access vs Index overhead, Multiple access, Inefficient for small files''

\end{mnemonicbox}
\subsection*{Question 4(a) [3 marks]}\label{q4a}

\textbf{Define System threats and explain its types.}

\begin{solutionbox}

\textbf{System Threats} are malicious attempts to disrupt or damage
computer system operations, steal information, or gain unauthorized
access.

{\def\LTcaptype{none} % do not increment counter
\begin{longtable}[]{@{}
  >{\raggedright\arraybackslash}p{(\linewidth - 2\tabcolsep) * \real{0.5000}}
  >{\raggedright\arraybackslash}p{(\linewidth - 2\tabcolsep) * \real{0.5000}}@{}}
\toprule\noalign{}
\begin{minipage}[b]{\linewidth}\raggedright
Threat Type
\end{minipage} & \begin{minipage}[b]{\linewidth}\raggedright
Description
\end{minipage} \\
\midrule\noalign{}
\endhead
\bottomrule\noalign{}
\endlastfoot
Worms & Self-replicating programs that spread across networks \\
Viruses & Malicious code that attaches to other programs \\
Trojan Horses & Legitimate-looking programs with hidden malicious
functions \\
Denial of Service & Attacks that overwhelm system resources \\
Port Scanning & Unauthorized probing of network services \\
\end{longtable}
}

\textbf{Categories of System Threats:}

\begin{itemize}
\tightlist
\item
  \textbf{Network-based}: Attacks through network connections and
  protocols
\item
  \textbf{Host-based}: Attacks targeting specific computer systems
\item
  \textbf{Physical}: Direct physical access to compromise systems
\end{itemize}

\textbf{Impact:} System threats can lead to data loss, system downtime,
privacy breaches, and financial damage.

\end{solutionbox}
\begin{mnemonicbox}
``WVTDP - Worms, Viruses, Trojans, DoS, Port
scanning''

\end{mnemonicbox}
\subsection*{Question 4(b) [4 marks]}\label{q4b}

\textbf{Differentiate: User Authentication v/s User Authorization.}

\begin{solutionbox}

{\def\LTcaptype{none} % do not increment counter
\begin{longtable}[]{@{}
  >{\raggedright\arraybackslash}p{(\linewidth - 4\tabcolsep) * \real{0.1667}}
  >{\raggedright\arraybackslash}p{(\linewidth - 4\tabcolsep) * \real{0.4167}}
  >{\raggedright\arraybackslash}p{(\linewidth - 4\tabcolsep) * \real{0.4167}}@{}}
\toprule\noalign{}
\begin{minipage}[b]{\linewidth}\raggedright
Aspect
\end{minipage} & \begin{minipage}[b]{\linewidth}\raggedright
User Authentication
\end{minipage} & \begin{minipage}[b]{\linewidth}\raggedright
User Authorization
\end{minipage} \\
\midrule\noalign{}
\endhead
\bottomrule\noalign{}
\endlastfoot
Purpose & Verify user identity & Determine user permissions \\
When & Before system access & After authentication \\
Methods & Passwords, biometrics, tokens & Access control lists, roles \\
Question & ``Who are you?'' & ``What can you do?'' \\
Process & One-time at login & Continuous during session \\
\end{longtable}
}

\textbf{Authentication Methods:}

\begin{itemize}
\tightlist
\item
  \textbf{Something you know}: Passwords, PINs
\item
  \textbf{Something you are}: Fingerprints, retina scans
\item
  \textbf{Something you have}: Smart cards, tokens
\end{itemize}

\textbf{Authorization Models:}

\begin{itemize}
\tightlist
\item
  \textbf{Role-based Access Control (RBAC)}: Permissions based on user
  roles
\item
  \textbf{Discretionary Access Control (DAC)}: Owner controls access
\item
  \textbf{Mandatory Access Control (MAC)}: System-enforced security
  levels
\end{itemize}

\textbf{Relationship:} Authentication must occur before authorization.
Both are essential for comprehensive security.

\end{solutionbox}
\begin{mnemonicbox}
``WHO vs WHAT - Authentication asks WHO,
Authorization determines WHAT''

\end{mnemonicbox}
\subsection*{Question 4(c) [7 marks]}\label{q4c}

\textbf{Discuss various operating system security policies and
procedures.}

\begin{solutionbox}

\textbf{Security Policies:}

{\def\LTcaptype{none} % do not increment counter
\begin{longtable}[]{@{}ll@{}}
\toprule\noalign{}
Policy Type & Description \\
\midrule\noalign{}
\endhead
\bottomrule\noalign{}
\endlastfoot
Access Control & Defines who can access what resources \\
Password Policy & Rules for password creation and management \\
Audit Policy & Logging and monitoring of system activities \\
Update Policy & Regular security patches and updates \\
Data Classification & Categorizing data by sensitivity levels \\
\end{longtable}
}

\textbf{Security Procedures:}

\textbf{1. User Account Management:}

\begin{itemize}
\tightlist
\item
  Regular review of user accounts and permissions
\item
  Immediate revocation of access for terminated employees
\item
  Principle of least privilege implementation
\end{itemize}

\textbf{2. System Monitoring:}

\begin{center}
\textbf{Mermaid Diagram (Code)}
\begin{verbatim}
{Shaded}
{Highlighting}[]
graph LR
    A[Log Collection] {-{-}{} B[Analysis Engine]}
    B {-{-}{} C[Threat Detection]}
    C {-{-}{} D[Alert Generation]}
    D {-{-}{} E[Response Action]}
{Highlighting}
{Shaded}
\end{verbatim}
\end{center}

\textbf{3. Incident Response:}

\begin{itemize}
\tightlist
\item
  \textbf{Detection}: Identify security incidents quickly
\item
  \textbf{Containment}: Limit damage and prevent spread
\item
  \textbf{Recovery}: Restore normal operations safely
\end{itemize}

\textbf{4. Backup and Recovery:}

\begin{itemize}
\tightlist
\item
  Regular data backups with tested restore procedures
\item
  Disaster recovery planning and testing
\item
  Business continuity measures
\end{itemize}

\textbf{Implementation Framework:}

\begin{itemize}
\tightlist
\item
  \textbf{Risk Assessment}: Identify vulnerabilities and threats
\item
  \textbf{Policy Development}: Create comprehensive security guidelines
\item
  \textbf{Training}: Educate users on security practices
\item
  \textbf{Compliance}: Ensure adherence to regulations
\end{itemize}

\end{solutionbox}
\begin{mnemonicbox}
``AAPUD policies + UMSIR procedures - Access, Audit,
Password, Update, Data classification + User management, Monitoring,
System response, Incident handling, Recovery''

\end{mnemonicbox}
\subsection*{Question 4(a) OR [3
marks]}\label{q4a}

\textbf{Define Program threats and explain its types.}

\begin{solutionbox}

\textbf{Program Threats} are malicious software designed to disrupt,
damage, or gain unauthorized access to computer programs and data.

{\def\LTcaptype{none} % do not increment counter
\begin{longtable}[]{@{}ll@{}}
\toprule\noalign{}
Threat Type & Description \\
\midrule\noalign{}
\endhead
\bottomrule\noalign{}
\endlastfoot
Malware & Malicious software including viruses, worms \\
Spyware & Programs that secretly monitor user activities \\
Adware & Unwanted advertising software \\
Ransomware & Encrypts data and demands payment \\
Rootkits & Hide malicious activities from detection \\
\end{longtable}
}

\textbf{Program Threat Categories:}

\begin{itemize}
\tightlist
\item
  \textbf{Executable Threats}: Standalone malicious programs
\item
  \textbf{Parasitic Threats}: Attach to legitimate programs
\item
  \textbf{Network Threats}: Spread through network connections
\end{itemize}

\textbf{Common Attack Vectors:}

\begin{itemize}
\tightlist
\item
  Email attachments and downloads
\item
  Infected removable media
\item
  Network vulnerabilities and exploits
\end{itemize}

\end{solutionbox}
\begin{mnemonicbox}
``MSARR - Malware, Spyware, Adware, Ransomware,
Rootkits''

\end{mnemonicbox}
\subsection*{Question 4(b) OR [4
marks]}\label{q4b}

\textbf{Explain the protection domain with a suitable example.}

\begin{solutionbox}

\textbf{Protection Domain} is a set of objects and access rights that
define what resources a process can access and what operations it can
perform.

{\def\LTcaptype{none} % do not increment counter
\begin{longtable}[]{@{}ll@{}}
\toprule\noalign{}
Component & Description \\
\midrule\noalign{}
\endhead
\bottomrule\noalign{}
\endlastfoot
Objects & Resources like files, memory, devices \\
Access Rights & Permissions like read, write, execute \\
Subjects & Processes or users requesting access \\
\end{longtable}
}

\textbf{Domain Structure:}

\begin{verbatim}
Protection Domain A
+{-{-}{-}{-}{-}{-}{-}{-}{-}{-}{-}{-}{-}{-}{-}{-}{-}{-}+}
| Objects:         |
| {- File1 (R,W)    |}
| {- Printer (W)    |}
| {- Memory (R,W,X) |}
+{-{-}{-}{-}{-}{-}{-}{-}{-}{-}{-}{-}{-}{-}{-}{-}{-}{-}+}

Protection Domain B
+{-{-}{-}{-}{-}{-}{-}{-}{-}{-}{-}{-}{-}{-}{-}{-}{-}{-}+}
| Objects:         |
| {- File2 (R)      |}
| {- Network (R,W)  |}
| {- Database (R)   |}
+{-{-}{-}{-}{-}{-}{-}{-}{-}{-}{-}{-}{-}{-}{-}{-}{-}{-}+}
\end{verbatim}

\textbf{Example - University System:}

\begin{itemize}
\tightlist
\item
  \textbf{Student Domain}: Read access to course materials, write access
  to assignments
\item
  \textbf{Faculty Domain}: Read/write access to grade databases, read
  access to student records
\item
  \textbf{Admin Domain}: Full access to system configuration, user
  management
\end{itemize}

\textbf{Domain Switching:} Processes can switch between domains based
on:

\begin{itemize}
\tightlist
\item
  User authentication and authorization
\item
  Program execution context
\item
  Security level requirements
\end{itemize}

\textbf{Benefits:}

\begin{itemize}
\tightlist
\item
  \textbf{Isolation}: Prevents unauthorized access between domains
\item
  \textbf{Flexibility}: Allows controlled resource sharing
\item
  \textbf{Security}: Implements principle of least privilege
\end{itemize}

\end{solutionbox}
\begin{mnemonicbox}
``OAS - Objects, Access rights, Subjects define
domains''

\end{mnemonicbox}
\subsection*{Question 4(c) OR [7
marks]}\label{q4c}

\textbf{Explain Access Control List in detail.}

\begin{solutionbox}

\textbf{Access Control List (ACL)} is a security mechanism that
specifies which users or processes are granted access to objects and
what operations are allowed.

\textbf{ACL Structure:}

{\def\LTcaptype{none} % do not increment counter
\begin{longtable}[]{@{}ll@{}}
\toprule\noalign{}
Component & Description \\
\midrule\noalign{}
\endhead
\bottomrule\noalign{}
\endlastfoot
Subject & User, group, or process requesting access \\
Object & Resource being protected (file, device, etc.) \\
Access Rights & Specific permissions granted \\
\end{longtable}
}

\textbf{ACL Implementation:}

\begin{verbatim}
File: /home/project/report.txt
+{-{-}{-}{-}{-}{-}{-}{-}{-}{-}{-}{-}{-}{-}{-}{-}{-}{-}{-}{-}{-}{-}{-}{-}{-}{-}{-}{-}{-}{-}{-}{-}+}
| User     | Permissions         |
|{-{-}{-}{-}{-}{-}{-}{-}{-}{-}|{-}{-}{-}{-}{-}{-}{-}{-}{-}{-}{-}{-}{-}{-}{-}{-}{-}{-}{-}{-}{-}|}
| alice    | read, write         |
| bob      | read                |
| admin    | read, write, delete |
| group:dev| read, write         |
+{-{-}{-}{-}{-}{-}{-}{-}{-}{-}{-}{-}{-}{-}{-}{-}{-}{-}{-}{-}{-}{-}{-}{-}{-}{-}{-}{-}{-}{-}{-}{-}+}
\end{verbatim}

\textbf{Types of ACL:}

\begin{itemize}
\tightlist
\item
  \textbf{Discretionary ACL (DACL)}: Owner controls access permissions
\item
  \textbf{System ACL (SACL)}: System controls auditing and logging
\item
  \textbf{Default ACL}: Inherited permissions for new objects
\end{itemize}

\textbf{ACL vs Capability Lists:}

{\def\LTcaptype{none} % do not increment counter
\begin{longtable}[]{@{}lll@{}}
\toprule\noalign{}
Aspect & ACL & Capability List \\
\midrule\noalign{}
\endhead
\bottomrule\noalign{}
\endlastfoot
Organization & Per object & Per subject \\
Storage & With object & With subject \\
Checking & Scan list & Present capability \\
Revocation & Easy & Difficult \\
\end{longtable}
}

\textbf{Advantages:}

\begin{itemize}
\tightlist
\item
  \textbf{Granular Control}: Fine-grained permission management
\item
  \textbf{Centralized Management}: Easy to modify object permissions
\item
  \textbf{Audit Trail}: Clear record of who has access
\end{itemize}

\textbf{Disadvantages:}

\begin{itemize}
\tightlist
\item
  \textbf{Performance Overhead}: Must check ACL for each access
\item
  \textbf{Storage Requirements}: Space needed for permission lists
\item
  \textbf{Complexity}: Difficult to manage for many users/objects
\end{itemize}

\textbf{Real-world Example:} Linux file permissions use simplified ACL
with owner, group, and others having read, write, execute rights.

\end{solutionbox}
\begin{mnemonicbox}
``SOA structure + GDSC advantages - Subject, Object,
Access rights + Granular, Distributed, Centralized, Audit capabilities''

\end{mnemonicbox}
\subsection*{Question 5(a) [3 marks]}\label{q5a}

\textbf{Explain the following commands: (i) man (ii) cd (iii) ls}

\begin{solutionbox}

{\def\LTcaptype{none} % do not increment counter
\begin{longtable}[]{@{}lll@{}}
\toprule\noalign{}
Command & Purpose & Syntax \\
\midrule\noalign{}
\endhead
\bottomrule\noalign{}
\endlastfoot
man & Display manual pages for commands & man [command] \\
cd & Change current directory & cd [directory] \\
ls & List directory contents & ls [options] [directory] \\
\end{longtable}
}

\textbf{Command Details:}

\textbf{1. man (manual) command:}

\begin{itemize}
\tightlist
\item
  \textbf{Function}: Shows detailed documentation for Linux commands
\item
  \textbf{Example}: \texttt{man\ ls} shows manual page for ls command
\item
  \textbf{Sections}: Commands, system calls, library functions, etc.
\end{itemize}

\textbf{2. cd (change directory) command:}

\begin{itemize}
\tightlist
\item
  \textbf{Function}: Navigate between directories in filesystem
\item
  \textbf{Examples}: \texttt{cd\ /home}, \texttt{cd\ ..} (parent),
  \texttt{cd\ \textasciitilde{}} (home)
\item
  \textbf{Special}: \texttt{cd} without arguments goes to home directory
\end{itemize}

\textbf{3. ls (list) command:}

\begin{itemize}
\tightlist
\item
  \textbf{Function}: Display files and directories in current or
  specified location
\item
  \textbf{Options}: \texttt{-l} (long format), \texttt{-a} (hidden
  files), \texttt{-h} (human readable)
\item
  \textbf{Example}: \texttt{ls\ -la} shows detailed listing including
  hidden files
\end{itemize}

\end{solutionbox}
\begin{mnemonicbox}
``MCD - Manual pages, Change directory, Directory
listing''

\end{mnemonicbox}
\subsection*{Question 5(b) [4 marks]}\label{q5b}

\textbf{Write a shell script to find maximum number among three
numbers.}

\begin{solutionbox}

\begin{verbatim}
\#!/bin/bash
\# Script to find maximum among three numbers

echo "Enter three numbers:"
read {-p} "First number: " num1
read {-p} "Second number: " num2  
read {-p} "Third number: " num3

\# Find maximum using nested if{-else}
if [ $num1 {-gt} $num2 ]; then
    if [ $num1 {-gt} $num3 ]; then
        max=$num1
    else
        max=$num3
    fi
else
    if [ $num2 {-gt} $num3 ]; then
        max=$num2
    else
        max=$num3
    fi
fi

echo "Maximum number is: $max"
\end{verbatim}

\textbf{Key Features:}

\begin{itemize}
\tightlist
\item
  \textbf{Input Validation}: Reads three numbers from user
\item
  \textbf{Comparison Logic}: Uses nested if-else for finding maximum
\item
  \textbf{Output Display}: Shows result with clear message
\end{itemize}

\textbf{Alternative Approach:}

\begin{verbatim}
max=$(echo "$num1 $num2 $num3" | tr { } {n} | sort {-nr} | head {-1})
\end{verbatim}

\end{solutionbox}
\begin{mnemonicbox}
``ICD - Input, Compare, Display result''

\end{mnemonicbox}
\subsection*{Question 5(c) [7 marks]}\label{q5c}

\textbf{Write a shell script to find the sum of all the individual
digits in a given 5-digit number.}

\begin{solutionbox}

\begin{verbatim}
\#!/bin/bash
\# Script to find sum of digits in a 5{-digit number}

echo "Enter a 5{-digit number:"}
read number

\# Validate input
if [ $\{\#number\} {-ne} 5 ] || ! [[ $number ={} \^{}[0{-}9]+$ ]]; then
    echo "Error: Please enter exactly 5 digits"
    exit 1
fi

sum=0
temp=$number

\# Extract and sum each digit
while [ $temp {-gt} 0 ]; do
    digit=$((temp \% 10))    \# Get last digit
    sum=$((sum + digit))    \# Add to sum
    temp=$((temp / 10))     \# Remove last digit
done

echo "Number: $number"
echo "Sum of digits: $sum"

\# Display breakdown
echo "Breakdown:"

\# Display individual digits
original=$number
echo {-n} "Digits: "
for ((i=0; i{}5; i++)); do
    digit=$((original \% 10))
    if [ $i {-eq} 4 ]; then
        echo {-n} "$digit"
    else
        echo {-n} "$digit + "
    fi
    original=$((original / 10))
done | tac
echo " = $sum"
\end{verbatim}

\textbf{Algorithm Steps:}

\begin{itemize}
\tightlist
\item
  \textbf{Input Validation}: Check for exactly 5 digits
\item
  \textbf{Digit Extraction}: Use modulo and division operations
\item
  \textbf{Sum Calculation}: Add each extracted digit
\item
  \textbf{Display Results}: Show breakdown and final sum
\end{itemize}

\textbf{Example Output:}

\begin{verbatim}
Enter a 5-digit number: 12345
Number: 12345
Sum of digits: 15
Breakdown: 1 + 2 + 3 + 4 + 5 = 15
\end{verbatim}

\end{solutionbox}
\begin{mnemonicbox}
``VEDS - Validate, Extract, Display, Sum digits''

\end{mnemonicbox}
\subsection*{Question 5(a) OR [3
marks]}\label{q5a}

\textbf{Explain the following commands: (i) date (ii) top (iii) cmp}

\begin{solutionbox}

{\def\LTcaptype{none} % do not increment counter
\begin{longtable}[]{@{}lll@{}}
\toprule\noalign{}
Command & Purpose & Syntax \\
\midrule\noalign{}
\endhead
\bottomrule\noalign{}
\endlastfoot
date & Display or set system date/time & date [options]
[format] \\
top & Display running processes dynamically & top [options] \\
cmp & Compare two files byte by byte & cmp [options] file1 file2 \\
\end{longtable}
}

\textbf{Command Details:}

\textbf{1. date command:}

\begin{itemize}
\tightlist
\item
  \textbf{Function}: Shows current system date and time
\item
  \textbf{Examples}: \texttt{date}, \texttt{date\ +\%Y-\%m-\%d},
  \texttt{date\ +\%H:\%M:\%S}
\item
  \textbf{Formatting}: Custom output formats using + symbols
\end{itemize}

\textbf{2. top command:}

\begin{itemize}
\tightlist
\item
  \textbf{Function}: Real-time display of system processes and resource
  usage
\item
  \textbf{Interactive}: Press `q' to quit, `k' to kill process
\item
  \textbf{Information}: CPU usage, memory usage, process list
\end{itemize}

\textbf{3. cmp command:}

\begin{itemize}
\tightlist
\item
  \textbf{Function}: Compare two files and report differences
\item
  \textbf{Output}: Shows first differing byte position
\item
  \textbf{Options}: \texttt{-s} (silent), \texttt{-l} (verbose listing)
\end{itemize}

\end{solutionbox}
\begin{mnemonicbox}
``DTC - Date/time, Task monitor, Compare files''

\end{mnemonicbox}
\subsection*{Question 5(b) OR [4
marks]}\label{q5b}

\textbf{Explain the installation steps of Linux.}

\begin{solutionbox}

{\def\LTcaptype{none} % do not increment counter
\begin{longtable}[]{@{}ll@{}}
\toprule\noalign{}
Step & Description \\
\midrule\noalign{}
\endhead
\bottomrule\noalign{}
\endlastfoot
1. Download ISO & Get Linux distribution image file \\
2. Create Bootable Media & Burn ISO to DVD or USB drive \\
3. Boot from Media & Start computer from installation media \\
4. Choose Installation Type & Select install alongside or replace OS \\
5. Partition Setup & Configure disk partitions \\
6. User Configuration & Create user account and passwords \\
7. Package Selection & Choose software packages to install \\
8. Installation Process & Copy files and configure system \\
9. Reboot System & Restart into new Linux installation \\
\end{longtable}
}

\textbf{Pre-installation Requirements:}

\begin{itemize}
\tightlist
\item
  \textbf{Hardware Compatibility}: Check system requirements
\item
  \textbf{Backup Data}: Secure important files before installation
\item
  \textbf{Internet Connection}: For updates and additional packages
\end{itemize}

\textbf{Installation Process Flow:}

\begin{center}
\textbf{Mermaid Diagram (Code)}
\begin{verbatim}
{Shaded}
{Highlighting}[]
graph LR
    A[Download ISO] {-{-}{} B[Create Bootable Media]}
    B {-{-}{} C[Boot from Media]}
    C {-{-}{} D[Language/Keyboard Setup]}
    D {-{-}{} E[Disk Partitioning]}
    E {-{-}{} F[User Account Setup]}
    F {-{-}{} G[Package Selection]}
    G {-{-}{} H[Install System]}
    H {-{-}{} I[Configure Bootloader]}
    I {-{-}{} J[Reboot to Linux]}
{Highlighting}
{Shaded}
\end{verbatim}
\end{center}

\textbf{Post-installation Tasks:}

\begin{itemize}
\tightlist
\item
  \textbf{System Updates}: Install latest security patches
\item
  \textbf{Driver Installation}: Configure hardware drivers
\item
  \textbf{Software Installation}: Add required applications
\end{itemize}

\textbf{Common Partition Scheme:}

\begin{itemize}
\tightlist
\item
  \texttt{/} (root): 20GB minimum for system files
\item
  \texttt{/home}: User data storage
\item
  \texttt{swap}: 1-2x RAM size for virtual memory
\end{itemize}

\end{solutionbox}
\begin{mnemonicbox}
``DCBCPUPI - Download, Create media, Boot, Choose
type, Partition, User setup, Package selection, Install''

\end{mnemonicbox}
\subsection*{Question 5(c) OR [7
marks]}\label{q5c}

\textbf{Write a shell script to find sum and average of N numbers.}

\begin{solutionbox}

\begin{verbatim}
\#!/bin/bash
\# Script to find sum and average of N numbers

echo "How many numbers do you want to enter?"
read n

\# Validate input
if ! [[ $n ={} \^{}[0{-}9]+$ ]] || [ $n {-le} 0 ]; then
    echo "Error: Please enter a positive integer"
    exit 1
fi

sum=0
echo "Enter $n numbers:"

\# Read N numbers and calculate sum
for ((i=1; i{=}n; i++)); do
    echo {-n} "Enter number $i: "
    read number
    
    \# Validate each number
    if ! [[ $number ={} \^{}{-}?[0{-}9]+([.][0{-}9]+)?$ ]]; then
        echo "Error: Invalid number format"
        exit 1
    fi
    
    sum=$(echo "$sum + $number" | bc {-l})
done

\# Calculate average
average=$(echo "scale=2; $sum / $n" | bc {-l})

\# Display results
echo ""
echo "Results:"
echo "========="
echo "Count of numbers: $n"
echo "Sum: $sum"
echo "Average: $average"

\# Additional statistics
echo ""
echo "Summary:"
echo "Total numbers processed: $n"
echo "Sum of all numbers: $sum"
echo "Average value: $average"
\end{verbatim}

\textbf{Algorithm Features:}

\begin{itemize}
\tightlist
\item
  \textbf{Input Validation}: Checks for positive count and valid numbers
\item
  \textbf{Flexible Input}: Accepts integers and decimal numbers
\item
  \textbf{Precision Handling}: Uses bc calculator for accurate
  arithmetic
\item
  \textbf{Error Handling}: Validates each input and provides error
  messages
\end{itemize}

\textbf{Example Execution:}

\begin{verbatim}
How many numbers do you want to enter? 5
Enter number 1: 10
Enter number 2: 20
Enter number 3: 30
Enter number 4: 40
Enter number 5: 50

Results:
=========
Count of numbers: 5
Sum: 150
Average: 30.00
\end{verbatim}

\textbf{Alternative Simple Version:}

\begin{verbatim}
\#!/bin/bash
read {-p} "Enter count: " n
sum=0
for ((i=1; i{=}n; i++)); do
    read {-p} "Number $i: " num
    sum=$((sum + num))
done
echo "Sum: $sum"
echo "Average: $((sum / n))"
\end{verbatim}

\textbf{Key Programming Concepts:}

\begin{itemize}
\tightlist
\item
  \textbf{Loop Control}: For loop for iterating N times
\item
  \textbf{Arithmetic Operations}: Addition and division
\item
  \textbf{Input/Output}: Reading user input and displaying results
\item
  \textbf{Data Validation}: Ensuring input correctness
\end{itemize}

\end{solutionbox}
\begin{mnemonicbox}
``VLAD - Validate input, Loop for numbers, Arithmetic
calculation, Display results''

\end{mnemonicbox}

\end{document}
