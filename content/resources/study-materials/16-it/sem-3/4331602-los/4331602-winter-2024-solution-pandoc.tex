\documentclass[10pt,a4paper]{article}

% content/resources/templates/preamble.tex
\usepackage[margin=0.6in]{geometry}
\author{Milav Dabgar}
\usepackage{amsmath,amssymb,amsthm}
\usepackage{booktabs}
\usepackage{multirow}
\usepackage{xcolor}
\usepackage{tcolorbox}
\tcbuselibrary{breakable,skins}
\usepackage[colorlinks=true,linkcolor=blue]{hyperref}
\usepackage{titlesec}
\usepackage{enumitem}
\usepackage{tikz}
\usepackage{pgfplots}
\usepackage{circuitikz}
\usepackage[version=4]{mhchem}
\usepackage{longtable}
\usepackage{array}
\usepackage{float}
\usepackage{caption}
\usepackage{listings}

\lstset{
  basicstyle=\small\ttfamily,
  breaklines=true,
  breakatwhitespace=false,
  postbreak=\mbox{\textcolor{red}{$\hookrightarrow$}\space},
  float=false,
  numbers=left,
  numberstyle=\tiny\color{gray},
  numbersep=10pt,
  xleftmargin=2em,
  keywordstyle=\color{blue},
  commentstyle=\color{green!60!black},
  stringstyle=\color{purple},
  backgroundcolor=\color{gray!5},
  showstringspaces=false,
  tabsize=2,
  captionpos=b,
  keepspaces=true,
  columns=flexible
}

\pgfplotsset{compat=1.18}
\usetikzlibrary{shapes,arrows,positioning,calc,patterns,decorations.pathmorphing,decorations.markings,arrows.meta}

% Color scheme
\definecolor{headcolor}{RGB}{0,102,204}
\definecolor{keycolor}{RGB}{220,20,60}
\definecolor{solutioncolor}{RGB}{34,139,34}
\definecolor{mnemoniccolor}{RGB}{148,0,211}
\definecolor{codecolor}{RGB}{0,0,100}

% Spacing
\setlength{\parskip}{3pt}
\setlist[itemize]{nosep}
\setlist[enumerate]{nosep}

% Title formatting
\titleformat{\section}{\Large\bfseries\color{headcolor}}{\thesection}{1em}{}
\titleformat{\subsection}{\large\bfseries\color{headcolor}}{\thesubsection}{1em}{}

% Pandoc tightlist compatibility
\providecommand{\tightlist}{%
  \setlength{\itemsep}{0pt}\setlength{\parskip}{0pt}}

% Pandoc longtable compatibility
\newcounter{none}
\def\thenone{}


% content/resources/templates/english-boxes.tex
% This file is currently empty - it exists to maintain consistency with the import structure.
% Add custom environments here if needed in the future.


\begin{document}

\begin{center}
{\Huge\bfseries\color{headcolor} Subject Name Solutions}\\[5pt]
{\LARGE 4331602 -- Winter 2024}\\[3pt]
{\large Semester 1 Study Material}\\[3pt]
{\normalsize\textit{Detailed Solutions and Explanations}}
\end{center}

\vspace{10pt}

\subsection*{Question 1(a) [3 marks]}\label{q1a}

\textbf{Explain Multiprogramming Operating System and give its
advantages.}

\begin{solutionbox}

\textbf{Multiprogramming Operating System} allows multiple programs to
reside in memory simultaneously and execute concurrently by sharing CPU
time efficiently.


{\def\LTcaptype{none} % do not increment counter
\vspace{-5pt}
\captionof{table}{Multiprogramming System Features}
\vspace{-10pt}
\begin{longtable}[]{@{}ll@{}}
\toprule\noalign{}
Feature & Description \\
\midrule\noalign{}
\endhead
\bottomrule\noalign{}
\endlastfoot
\textbf{Memory Management} & Multiple programs loaded in memory \\
\textbf{CPU Scheduling} & CPU switches between programs \\
\textbf{Resource Sharing} & Efficient utilization of system resources \\
\end{longtable}
}

\begin{itemize}
\tightlist
\item
  \textbf{Increased CPU utilization}: CPU remains busy switching between
  programs
\item
  \textbf{Better throughput}: More programs completed per unit time
\item
  \textbf{Reduced response time}: Programs execute faster due to
  parallel processing
\end{itemize}

\end{solutionbox}
\begin{mnemonicbox}
``MCP'' - Memory sharing, CPU utilization, Parallel
execution

\end{mnemonicbox}
\subsection*{Question 1(b) [4 marks]}\label{q1b}

\textbf{Explain Characteristics of Linux operating system.}

\begin{solutionbox}


{\def\LTcaptype{none} % do not increment counter
\vspace{-5pt}
\captionof{table}{Linux Operating System Characteristics}
\vspace{-10pt}
\begin{longtable}[]{@{}ll@{}}
\toprule\noalign{}
Characteristic & Description \\
\midrule\noalign{}
\endhead
\bottomrule\noalign{}
\endlastfoot
\textbf{Open Source} & Source code freely available and modifiable \\
\textbf{Multi-user} & Multiple users can access system simultaneously \\
\textbf{Multi-tasking} & Multiple processes run concurrently \\
\textbf{Portable} & Runs on various hardware platforms \\
\textbf{Security} & Strong permission system and access controls \\
\textbf{Stability} & Robust and reliable system performance \\
\end{longtable}
}

\begin{itemize}
\tightlist
\item
  \textbf{Case sensitive}: Distinguishes between uppercase and lowercase
\item
  \textbf{Command line interface}: Powerful shell for system operations
\item
  \textbf{File system hierarchy}: Organized directory structure starting
  from root (/)
\end{itemize}

\end{solutionbox}
\begin{mnemonicbox}
``LAMPS'' - Linux is Accessible, Multi-user,
Portable, Secure

\end{mnemonicbox}
\subsection*{Question 1(c) [7 marks]}\label{q1c}

\textbf{Explain FCFS scheduling algorithm with its advantages and
disadvantages. Calculate Average waiting time and average turnaround
time for FCFS algorithm with gantt chart for following data.}

\begin{solutionbox}

\textbf{First Come First Serve (FCFS)} is a non-preemptive scheduling
algorithm where processes are executed in order of their arrival.


{\def\LTcaptype{none} % do not increment counter
\vspace{-5pt}
\captionof{table}{FCFS Algorithm Analysis}
\vspace{-10pt}
\begin{longtable}[]{@{}ll@{}}
\toprule\noalign{}
Aspect & Description \\
\midrule\noalign{}
\endhead
\bottomrule\noalign{}
\endlastfoot
\textbf{Policy} & First arrived process gets CPU first \\
\textbf{Type} & Non-preemptive \\
\textbf{Implementation} & Simple queue (FIFO) \\
\end{longtable}
}

\textbf{Advantages:}

\begin{itemize}
\tightlist
\item
  \textbf{Simple implementation}: Easy to understand and code
\item
  \textbf{Fair scheduling}: No starvation occurs
\end{itemize}

\textbf{Disadvantages:}

\begin{itemize}
\tightlist
\item
  \textbf{Convoy effect}: Short processes wait for long processes
\item
  \textbf{Poor average waiting time}: Not optimal for system performance
\end{itemize}

\textbf{Gantt Chart Calculation:}

\begin{verbatim}
P0    |    P1  | P2 |   P3    |
0     5     8   10   17
\end{verbatim}


{\def\LTcaptype{none} % do not increment counter
\vspace{-5pt}
\captionof{table}{Process Execution Analysis}
\vspace{-10pt}
\begin{longtable}[]{@{}lllllll@{}}
\toprule\noalign{}
Process & Arrival & Burst & Start & Finish & Waiting & Turnaround \\
\midrule\noalign{}
\endhead
\bottomrule\noalign{}
\endlastfoot
P0 & 0 & 5 & 0 & 5 & 0 & 5 \\
P1 & 3 & 3 & 5 & 8 & 2 & 5 \\
P2 & 5 & 2 & 8 & 10 & 3 & 5 \\
P3 & 6 & 7 & 10 & 17 & 4 & 11 \\
\end{longtable}
}

\textbf{Average Waiting Time} = (0+2+3+4)/4 = \textbf{2.25 ms}
\textbf{Average Turnaround Time} = (5+5+5+11)/4 = \textbf{6.5 ms}

\end{solutionbox}
\begin{mnemonicbox}
``FCFS-SiNo'' - First Come First Serve is Simple but
Not optimal

\end{mnemonicbox}
\begin{center}\rule{0.5\linewidth}{0.5pt}\end{center}

\subsection*{Question 1(c) OR [7
marks]}\label{q1c}

\textbf{Explain Round Robin algorithm with its advantages and
disadvantages. Calculate Average waiting time and average turnaround
time for Round Robin algorithm with gantt chart for following data.
(Time Quantum = 2 ms)}

\begin{solutionbox}

\textbf{Round Robin} is a preemptive scheduling algorithm where each
process gets equal CPU time slice (quantum).


{\def\LTcaptype{none} % do not increment counter
\vspace{-5pt}
\captionof{table}{Round Robin Features}
\vspace{-10pt}
\begin{longtable}[]{@{}ll@{}}
\toprule\noalign{}
Feature & Description \\
\midrule\noalign{}
\endhead
\bottomrule\noalign{}
\endlastfoot
\textbf{Time Quantum} & Fixed time slice for each process \\
\textbf{Preemption} & Process interrupted after quantum expires \\
\textbf{Queue Type} & Circular ready queue \\
\end{longtable}
}

\textbf{Advantages:}

\begin{itemize}
\tightlist
\item
  \textbf{Fair allocation}: Each process gets equal CPU time
\item
  \textbf{No starvation}: All processes eventually get CPU
\end{itemize}

\textbf{Disadvantages:}

\begin{itemize}
\tightlist
\item
  \textbf{Context switching overhead}: Frequent process switching
\item
  \textbf{Performance depends on quantum}: Too small or large affects
  efficiency
\end{itemize}

\textbf{Gantt Chart (Quantum = 2ms):}

\begin{verbatim}
P0|P1|P2|P3|P0|P1|P2|P1|P0|P1|
0 2 4 6 7 9 11 12 13 14 16
\end{verbatim}


{\def\LTcaptype{none} % do not increment counter
\vspace{-5pt}
\captionof{table}{Round Robin Execution}
\vspace{-10pt}
\begin{longtable}[]{@{}llllll@{}}
\toprule\noalign{}
Process & Arrival & Burst & Completion & Waiting & Turnaround \\
\midrule\noalign{}
\endhead
\bottomrule\noalign{}
\endlastfoot
P0 & 0 & 4 & 14 & 10 & 14 \\
P1 & 1 & 5 & 16 & 10 & 15 \\
P2 & 2 & 3 & 12 & 7 & 10 \\
P3 & 3 & 1 & 7 & 3 & 4 \\
\end{longtable}
}

\textbf{Average Waiting Time} = (10+10+7+3)/4 = \textbf{7.5 ms}
\textbf{Average Turnaround Time} = (14+15+10+4)/4 = \textbf{10.75 ms}

\end{solutionbox}
\begin{mnemonicbox}
``RR-TEQ'' - Round Robin uses Time Equal Quantum

\end{mnemonicbox}
\begin{center}\rule{0.5\linewidth}{0.5pt}\end{center}

\subsection*{Question 2(a) [3 marks]}\label{q2a}

\textbf{Explain Real Time Operation System.}

\begin{solutionbox}

\textbf{Real Time Operating System (RTOS)} processes data and responds
to events within strict time constraints.


{\def\LTcaptype{none} % do not increment counter
\vspace{-5pt}
\captionof{table}{RTOS Types}
\vspace{-10pt}
\begin{longtable}[]{@{}lll@{}}
\toprule\noalign{}
Type & Response Time & Example \\
\midrule\noalign{}
\endhead
\bottomrule\noalign{}
\endlastfoot
\textbf{Hard Real-time} & Guaranteed deadline & Missile guidance \\
\textbf{Soft Real-time} & Flexible deadline & Video streaming \\
\end{longtable}
}

\begin{itemize}
\tightlist
\item
  \textbf{Deterministic behavior}: Predictable response times
\item
  \textbf{Priority-based scheduling}: Critical tasks get higher priority
\item
  \textbf{Minimal latency}: Fast interrupt handling and context
  switching
\end{itemize}

\end{solutionbox}
\begin{mnemonicbox}
``RTD'' - Real Time is Deterministic

\end{mnemonicbox}
\subsection*{Question 2(b) [4 marks]}\label{q2b}

\textbf{Explain Process Life Cycle with diagram.}

\begin{solutionbox}

\textbf{Process Life Cycle} shows different states a process goes
through during execution.

\textbf{Diagram: Process State Transition}

\begin{verbatim}
stateDiagram{-v2}
  direction LR
    [*] {-{-} New : Create Process}
    New {-{-} Ready : Admitted}
    Ready {-{-} Running : Scheduler Dispatch}
    Running {-{-} Waiting : I/O Request}
    Running {-{-} Ready : Time Quantum Expired}
    Running {-{-} Terminated : Exit}
    Waiting {-{-} Ready : I/O Complete}
    Terminated {-{-} [*] : Process Cleanup}
\end{verbatim}


{\def\LTcaptype{none} % do not increment counter
\vspace{-5pt}
\captionof{table}{Process States}
\vspace{-10pt}
\begin{longtable}[]{@{}ll@{}}
\toprule\noalign{}
State & Description \\
\midrule\noalign{}
\endhead
\bottomrule\noalign{}
\endlastfoot
\textbf{New} & Process being created \\
\textbf{Ready} & Waiting for CPU assignment \\
\textbf{Running} & Instructions being executed \\
\textbf{Waiting} & Waiting for I/O completion \\
\textbf{Terminated} & Process finished execution \\
\end{longtable}
}

\end{solutionbox}
\begin{mnemonicbox}
``NRRWT'' - New Ready Running Waiting Terminated

\end{mnemonicbox}
\subsection*{Question 2(c) [7 marks]}\label{q2c}

\textbf{Explain Various file and directory related commands in Linux.}

\begin{solutionbox}


{\def\LTcaptype{none} % do not increment counter
\vspace{-5pt}
\captionof{table}{File Commands}
\vspace{-10pt}
\begin{longtable}[]{@{}lll@{}}
\toprule\noalign{}
Command & Function & Example \\
\midrule\noalign{}
\endhead
\bottomrule\noalign{}
\endlastfoot
\textbf{ls} & List directory contents & \texttt{ls\ -la} \\
\textbf{cat} & Display file content & \texttt{cat\ file.txt} \\
\textbf{cp} & Copy files & \texttt{cp\ source\ dest} \\
\textbf{mv} & Move/rename files & \texttt{mv\ old\ new} \\
\textbf{rm} & Remove files & \texttt{rm\ file.txt} \\
\end{longtable}
}


{\def\LTcaptype{none} % do not increment counter
\vspace{-5pt}
\captionof{table}{Directory Commands}
\vspace{-10pt}
\begin{longtable}[]{@{}lll@{}}
\toprule\noalign{}
Command & Function & Example \\
\midrule\noalign{}
\endhead
\bottomrule\noalign{}
\endlastfoot
\textbf{mkdir} & Create directory & \texttt{mkdir\ mydir} \\
\textbf{rmdir} & Remove empty directory & \texttt{rmdir\ mydir} \\
\textbf{cd} & Change directory & \texttt{cd\ /home} \\
\textbf{pwd} & Print working directory & \texttt{pwd} \\
\end{longtable}
}

\begin{itemize}
\tightlist
\item
  \textbf{File permissions}: Use \texttt{chmod} to modify access rights
\item
  \textbf{File ownership}: Use \texttt{chown} to change file owner
\item
  \textbf{File information}: Use \texttt{stat} for detailed file
  information
\end{itemize}

\end{solutionbox}
\begin{mnemonicbox}
``LCCMR-MRCP'' - List, Cat, Copy, Move, Remove for
files; Make, Remove, Change, Print for directories

\end{mnemonicbox}
\begin{center}\rule{0.5\linewidth}{0.5pt}\end{center}

\subsection*{Question 2(a) OR [3
marks]}\label{q2a}

\textbf{Describe operating system services in detail.}

\begin{solutionbox}

\textbf{Operating System Services} provide interface between user
applications and hardware resources.


{\def\LTcaptype{none} % do not increment counter
\vspace{-5pt}
\captionof{table}{OS Services Categories}
\vspace{-10pt}
\begin{longtable}[]{@{}ll@{}}
\toprule\noalign{}
Category & Services \\
\midrule\noalign{}
\endhead
\bottomrule\noalign{}
\endlastfoot
\textbf{User Interface} & GUI, Command Line, Batch \\
\textbf{Program Execution} & Loading, Running, Terminating \\
\textbf{I/O Operations} & File operations, Device communication \\
\textbf{File System} & Creation, Deletion, Manipulation \\
\textbf{Communication} & Process communication, Network \\
\textbf{Error Detection} & Hardware/Software error handling \\
\end{longtable}
}

\begin{itemize}
\tightlist
\item
  \textbf{Resource allocation}: CPU, memory, and device management
\item
  \textbf{Accounting}: Track resource usage and performance
\item
  \textbf{Protection and security}: Access control and authentication
\end{itemize}

\end{solutionbox}
\begin{mnemonicbox}
``UPIFCE'' - User interface, Program execution, I/O,
File system, Communication, Error detection

\end{mnemonicbox}
\subsection*{Question 2(b) OR [4
marks]}\label{q2b}

\textbf{Explain Process Control Block.}

\begin{solutionbox}

\textbf{Process Control Block (PCB)} is a data structure containing all
information about a process.


{\def\LTcaptype{none} % do not increment counter
\vspace{-5pt}
\captionof{table}{PCB Components}
\vspace{-10pt}
\begin{longtable}[]{@{}ll@{}}
\toprule\noalign{}
Component & Information Stored \\
\midrule\noalign{}
\endhead
\bottomrule\noalign{}
\endlastfoot
\textbf{Process ID} & Unique process identifier \\
\textbf{Process State} & Current state (ready, running, waiting) \\
\textbf{CPU Registers} & Program counter, stack pointer, registers \\
\textbf{Memory Management} & Base/limit registers, page tables \\
\textbf{I/O Status} & Open files, allocated devices \\
\textbf{Accounting} & CPU usage, time limits \\
\end{longtable}
}

\textbf{Diagram: PCB Structure}

\begin{verbatim}
+{-{-}{-}{-}{-}{-}{-}{-}{-}{-}{-}{-}{-}{-}{-}{-}{-}{-}+}
| Process ID       |
+{-{-}{-}{-}{-}{-}{-}{-}{-}{-}{-}{-}{-}{-}{-}{-}{-}{-}+}
| Process State    |
+{-{-}{-}{-}{-}{-}{-}{-}{-}{-}{-}{-}{-}{-}{-}{-}{-}{-}+}
| Program Counter  |
+{-{-}{-}{-}{-}{-}{-}{-}{-}{-}{-}{-}{-}{-}{-}{-}{-}{-}+}
| CPU Registers    |
+{-{-}{-}{-}{-}{-}{-}{-}{-}{-}{-}{-}{-}{-}{-}{-}{-}{-}+}
| Memory Limits    |
+{-{-}{-}{-}{-}{-}{-}{-}{-}{-}{-}{-}{-}{-}{-}{-}{-}{-}+}
| Open File List   |
+{-{-}{-}{-}{-}{-}{-}{-}{-}{-}{-}{-}{-}{-}{-}{-}{-}{-}+}
| Accounting Info  |
+{-{-}{-}{-}{-}{-}{-}{-}{-}{-}{-}{-}{-}{-}{-}{-}{-}{-}+}
\end{verbatim}

\end{solutionbox}
\begin{mnemonicbox}
``PPCMIA'' - Process ID, Process state, Program
Counter, CPU registers, Memory, I/O, Accounting

\end{mnemonicbox}
\subsection*{Question 2(c) OR [7
marks]}\label{q2c}

\textbf{Explain installation steps of Linux.}

\begin{solutionbox}

\textbf{Linux Installation} involves preparing system and installing
operating system from bootable media.


{\def\LTcaptype{none} % do not increment counter
\vspace{-5pt}
\captionof{table}{Installation Steps}
\vspace{-10pt}
\begin{longtable}[]{@{}
  >{\raggedright\arraybackslash}p{(\linewidth - 2\tabcolsep) * \real{0.3158}}
  >{\raggedright\arraybackslash}p{(\linewidth - 2\tabcolsep) * \real{0.6842}}@{}}
\toprule\noalign{}
\begin{minipage}[b]{\linewidth}\raggedright
Step
\end{minipage} & \begin{minipage}[b]{\linewidth}\raggedright
Description
\end{minipage} \\
\midrule\noalign{}
\endhead
\bottomrule\noalign{}
\endlastfoot
\textbf{1. Download ISO} & Get Linux distribution image file \\
\textbf{2. Create Bootable Media} & Use USB/DVD to create installation
media \\
\textbf{3. Boot from Media} & Change BIOS/UEFI boot order \\
\textbf{4. Select Language} & Choose installation language \\
\textbf{5. Partition Disk} & Create root, swap, home partitions \\
\textbf{6. Configure Network} & Set IP, DNS, hostname \\
\textbf{7. Create User Account} & Set username, password \\
\textbf{8. Install Bootloader} & Configure GRUB for booting \\
\textbf{9. Complete Installation} & Remove media and reboot \\
\end{longtable}
}

\textbf{Partitioning Scheme:}

\begin{itemize}
\tightlist
\item
  \textbf{Root (/)}: 20GB minimum for system files
\item
  \textbf{Swap}: 2x RAM size for virtual memory
\item
  \textbf{Home (/home)}: Remaining space for user data
\end{itemize}

\textbf{Post-installation:}

\begin{itemize}
\tightlist
\item
  \textbf{Update system}:
  \texttt{sudo\ apt\ update\ \&\&\ sudo\ apt\ upgrade}
\item
  \textbf{Install drivers}: Graphics, network, audio drivers
\item
  \textbf{Configure security}: Firewall, user permissions
\end{itemize}

\end{solutionbox}
\begin{mnemonicbox}
``DCBSLNCIU'' - Download, Create media, Boot, Select
language, Layout disk, Network, Create user, Install bootloader, Update
system

\end{mnemonicbox}
\begin{center}\rule{0.5\linewidth}{0.5pt}\end{center}

\subsection*{Question 3(a) [3 marks]}\label{q3a}

\textbf{Define: Process, Program, Swapping}

\begin{solutionbox}


{\def\LTcaptype{none} % do not increment counter
\vspace{-5pt}
\captionof{table}{Basic Definitions}
\vspace{-10pt}
\begin{longtable}[]{@{}ll@{}}
\toprule\noalign{}
Term & Definition \\
\midrule\noalign{}
\endhead
\bottomrule\noalign{}
\endlastfoot
\textbf{Process} & Program in execution with allocated resources \\
\textbf{Program} & Set of instructions stored on disk \\
\textbf{Swapping} & Moving processes between memory and disk \\
\end{longtable}
}

\begin{itemize}
\tightlist
\item
  \textbf{Process}: Active entity with process ID, memory space, and
  execution state
\item
  \textbf{Program}: Passive entity, executable file stored in secondary
  storage
\item
  \textbf{Swapping}: Memory management technique to handle more
  processes than physical memory
\end{itemize}

\end{solutionbox}
\begin{mnemonicbox}
``PAP-MDS'' - Process is Active Program; Program is
instructions; Swapping is Memory-Disk transfer

\end{mnemonicbox}
\subsection*{Question 3(b) [4 marks]}\label{q3b}

\textbf{List out various file operations and describe each of them.}

\begin{solutionbox}


{\def\LTcaptype{none} % do not increment counter
\vspace{-5pt}
\captionof{table}{File Operations}
\vspace{-10pt}
\begin{longtable}[]{@{}lll@{}}
\toprule\noalign{}
Operation & Description & System Call \\
\midrule\noalign{}
\endhead
\bottomrule\noalign{}
\endlastfoot
\textbf{Create} & Make new file with specified name &
\texttt{creat()} \\
\textbf{Open} & Prepare file for reading/writing & \texttt{open()} \\
\textbf{Read} & Retrieve data from file & \texttt{read()} \\
\textbf{Write} & Store data to file & \texttt{write()} \\
\textbf{Close} & Finish file access, release resources &
\texttt{close()} \\
\textbf{Delete} & Remove file from file system & \texttt{unlink()} \\
\textbf{Seek} & Move file pointer to specific position &
\texttt{lseek()} \\
\end{longtable}
}

\begin{itemize}
\tightlist
\item
  \textbf{File attributes}: Access permissions, timestamps, size
  information
\item
  \textbf{File locking}: Prevent concurrent access conflicts
\item
  \textbf{Buffer management}: Optimize I/O performance through caching
\end{itemize}

\end{solutionbox}
\begin{mnemonicbox}
``CORWCDS'' - Create, Open, Read, Write, Close,
Delete, Seek

\end{mnemonicbox}
\subsection*{Question 3(c) [7 marks]}\label{q3c}

\textbf{Write a shell script to generate and print Fibonacci series.}

\begin{solutionbox}

\textbf{Fibonacci Series} generates numbers where each number is sum of
two preceding numbers.

\textbf{Shell Script:}

\begin{verbatim}
\#!/bin/bash
\# Fibonacci series generator

echo "Enter number of terms:"
read n

a=0
b=1

echo "Fibonacci Series:"
echo {-n} "$a $b "

for((i=2; i{}n; i++))
do
    c=$((a + b))
    echo {-n} "$c "
    a=$b
    b=$c
done
echo
\end{verbatim}


{\def\LTcaptype{none} % do not increment counter
\vspace{-5pt}
\captionof{table}{Script Components}
\vspace{-10pt}
\begin{longtable}[]{@{}ll@{}}
\toprule\noalign{}
Component & Purpose \\
\midrule\noalign{}
\endhead
\bottomrule\noalign{}
\endlastfoot
\textbf{\#!/bin/bash} & Shebang line specifying interpreter \\
\textbf{read n} & Accept user input for number of terms \\
\textbf{for loop} & Iterate to generate sequence \\
\textbf{Arithmetic} & Calculate next number in series \\
\end{longtable}
}

\textbf{Output Example:}

\begin{verbatim}
Enter number of terms: 8
Fibonacci Series: 0 1 1 2 3 5 8 13
\end{verbatim}

\end{solutionbox}
\begin{mnemonicbox}
``FLAB'' - Fibonacci uses Loop with Addition of Both
previous numbers

\end{mnemonicbox}
\begin{center}\rule{0.5\linewidth}{0.5pt}\end{center}

\subsection*{Question 3(a) OR [3
marks]}\label{q3a}

\textbf{List out types of scheduler and explain any one of them.}

\begin{solutionbox}


{\def\LTcaptype{none} % do not increment counter
\vspace{-5pt}
\captionof{table}{Types of Schedulers}
\vspace{-10pt}
\begin{longtable}[]{@{}ll@{}}
\toprule\noalign{}
Scheduler Type & Function \\
\midrule\noalign{}
\endhead
\bottomrule\noalign{}
\endlastfoot
\textbf{Long-term} & Selects processes from job pool to ready queue \\
\textbf{Short-term} & Selects process from ready queue for CPU \\
\textbf{Medium-term} & Handles swapping between memory and disk \\
\end{longtable}
}

\textbf{Short-term Scheduler (CPU Scheduler):}

\begin{itemize}
\tightlist
\item
  \textbf{Frequency}: Executes very frequently (milliseconds)
\item
  \textbf{Function}: Decides which process gets CPU next
\item
  \textbf{Algorithms}: FCFS, SJF, Round Robin, Priority
\item
  \textbf{Goal}: Maximize CPU utilization and throughput
\end{itemize}

\end{solutionbox}
\begin{mnemonicbox}
``LSM-JRC'' - Long-term (Job), Short-term (Ready),
Medium-term (swap Control)

\end{mnemonicbox}
\subsection*{Question 3(b) OR [4
marks]}\label{q3b}

\textbf{List out various file attributes and describe each of them.}

\begin{solutionbox}


{\def\LTcaptype{none} % do not increment counter
\vspace{-5pt}
\captionof{table}{File Attributes}
\vspace{-10pt}
\begin{longtable}[]{@{}ll@{}}
\toprule\noalign{}
Attribute & Description \\
\midrule\noalign{}
\endhead
\bottomrule\noalign{}
\endlastfoot
\textbf{Name} & Human-readable file identifier \\
\textbf{Type} & File format (text, binary, executable) \\
\textbf{Size} & Current file size in bytes \\
\textbf{Location} & Physical address on storage device \\
\textbf{Protection} & Access permissions (read, write, execute) \\
\textbf{Time stamps} & Creation, modification, access times \\
\textbf{Owner} & User who created the file \\
\end{longtable}
}

\textbf{Permission Structure:}

\begin{itemize}
\tightlist
\item
  \textbf{User (u)}: Owner permissions
\item
  \textbf{Group (g)}: Group member permissions\\
\item
  \textbf{Other (o)}: All other users permissions
\end{itemize}

\textbf{Example:} \texttt{-rwxr-xr-\/-}

\begin{itemize}
\tightlist
\item
  File type: regular file (-)
\item
  Owner: read, write, execute (rwx)
\item
  Group: read, execute (r-x)
\item
  Others: read only (r--)
\end{itemize}

\end{solutionbox}
\begin{mnemonicbox}
``NTSLPTO'' - Name, Type, Size, Location, Protection,
Time, Owner

\end{mnemonicbox}
\subsection*{Question 3(c) OR [7
marks]}\label{q3c}

\textbf{Write a shell script to sum of 1 to 10 using while loop.}

\begin{solutionbox}

\textbf{While Loop} continues execution as long as specified condition
remains true.

\textbf{Shell Script:}

\begin{verbatim}
\#!/bin/bash
\# Sum of numbers 1 to 10 using while loop

echo "Calculating sum of 1 to 10:"

i=1
sum=0

while [ $i {-le} 10 ]
do
    sum=$((sum + i))
    echo "Adding $i, current sum: $sum"
    i=$((i + 1))
done

echo "Final sum of 1 to 10 is: $sum"
\end{verbatim}


{\def\LTcaptype{none} % do not increment counter
\vspace{-5pt}
\captionof{table}{Script Logic}
\vspace{-10pt}
\begin{longtable}[]{@{}
  >{\raggedright\arraybackslash}p{(\linewidth - 2\tabcolsep) * \real{0.5500}}
  >{\raggedright\arraybackslash}p{(\linewidth - 2\tabcolsep) * \real{0.4500}}@{}}
\toprule\noalign{}
\begin{minipage}[b]{\linewidth}\raggedright
Component
\end{minipage} & \begin{minipage}[b]{\linewidth}\raggedright
Purpose
\end{minipage} \\
\midrule\noalign{}
\endhead
\bottomrule\noalign{}
\endlastfoot
\textbf{i=1} & Initialize counter variable \\
\textbf{sum=0} & Initialize accumulator \\
\textbf{while [ \(i -le 10 ]** | Continue while i \leq 10 |
| **sum=\)((sum + i))} & Add current number to sum \\
\textbf{i=\$((i + 1))} & Increment counter \\
\end{longtable}
}

\textbf{Output:}

\begin{verbatim}
Calculating sum of 1 to 10:
Adding 1, current sum: 1
Adding 2, current sum: 3
...
Final sum of 1 to 10 is: 55
\end{verbatim}

\end{solutionbox}
\begin{mnemonicbox}
``WICS'' - While loop needs Initialize, Condition,
Sum calculation

\end{mnemonicbox}
\begin{center}\rule{0.5\linewidth}{0.5pt}\end{center}

\subsection*{Question 4(a) [3 marks]}\label{q4a}

\textbf{List out and explain condition for Deadlock to occur.}

\begin{solutionbox}

\textbf{Deadlock} occurs when processes wait indefinitely for resources
held by each other.


{\def\LTcaptype{none} % do not increment counter
\vspace{-5pt}
\captionof{table}{Deadlock Conditions (Coffman Conditions)}
\vspace{-10pt}
\begin{longtable}[]{@{}
  >{\raggedright\arraybackslash}p{(\linewidth - 2\tabcolsep) * \real{0.4583}}
  >{\raggedright\arraybackslash}p{(\linewidth - 2\tabcolsep) * \real{0.5417}}@{}}
\toprule\noalign{}
\begin{minipage}[b]{\linewidth}\raggedright
Condition
\end{minipage} & \begin{minipage}[b]{\linewidth}\raggedright
Description
\end{minipage} \\
\midrule\noalign{}
\endhead
\bottomrule\noalign{}
\endlastfoot
\textbf{Mutual Exclusion} & Only one process can use resource at a
time \\
\textbf{Hold and Wait} & Process holds resources while waiting for
others \\
\textbf{No Preemption} & Resources cannot be forcibly taken away \\
\textbf{Circular Wait} & Circular chain of processes waiting for
resources \\
\end{longtable}
}

\textbf{All four conditions must be true simultaneously for deadlock to
occur.}

\textbf{Example Scenario:}

\begin{itemize}
\tightlist
\item
  Process P1 holds Resource A, needs Resource B
\item
  Process P2 holds Resource B, needs Resource A
\item
  Both processes wait indefinitely
\end{itemize}

\end{solutionbox}
\begin{mnemonicbox}
``MHNC'' - Mutual exclusion, Hold and wait, No
preemption, Circular wait

\end{mnemonicbox}
\subsection*{Question 4(b) [4 marks]}\label{q4b}

\textbf{List out File access methods. Explain any one.}

\begin{solutionbox}


{\def\LTcaptype{none} % do not increment counter
\vspace{-5pt}
\captionof{table}{File Access Methods}
\vspace{-10pt}
\begin{longtable}[]{@{}ll@{}}
\toprule\noalign{}
Method & Description \\
\midrule\noalign{}
\endhead
\bottomrule\noalign{}
\endlastfoot
\textbf{Sequential Access} & Read file from beginning to end \\
\textbf{Direct Access} & Jump to any record directly \\
\textbf{Index Sequential} & Combination of sequential and indexed
access \\
\end{longtable}
}

\textbf{Sequential Access Method:}

\begin{itemize}
\tightlist
\item
  \textbf{Process}: Read records one by one in order
\item
  \textbf{Advantages}: Simple implementation, efficient for batch
  processing
\item
  \textbf{Disadvantages}: Slow for specific record access
\item
  \textbf{Use cases}: Log files, data backup, streaming
\end{itemize}

\textbf{Operations:}

\begin{verbatim}
read_next() - Read next record
write_next() - Write next record  
reset() - Return to beginning
\end{verbatim}

\end{solutionbox}
\begin{mnemonicbox}
``SDI'' - Sequential (start to end), Direct (jump
anywhere), Index (combined approach)

\end{mnemonicbox}
\subsection*{Question 4(c) [7 marks]}\label{q4c}

\textbf{Describe Security measures in operating system.}

\begin{solutionbox}

\textbf{Operating System Security} protects system resources from
unauthorized access and threats.


{\def\LTcaptype{none} % do not increment counter
\vspace{-5pt}
\captionof{table}{Security Mechanisms}
\vspace{-10pt}
\begin{longtable}[]{@{}ll@{}}
\toprule\noalign{}
Mechanism & Description \\
\midrule\noalign{}
\endhead
\bottomrule\noalign{}
\endlastfoot
\textbf{Authentication} & Verify user identity (passwords,
biometrics) \\
\textbf{Authorization} & Control resource access permissions \\
\textbf{Access Control Lists} & Define who can access specific
resources \\
\textbf{Encryption} & Protect data confidentiality \\
\textbf{Audit Logs} & Track system activities and access \\
\textbf{Firewalls} & Control network traffic \\
\end{longtable}
}

\textbf{Security Levels:}

\begin{itemize}
\tightlist
\item
  \textbf{Physical security}: Protect hardware and facilities
\item
  \textbf{User authentication}: Login credentials and biometrics
\item
  \textbf{File permissions}: Read, write, execute controls
\item
  \textbf{Network security}: Secure communication protocols
\end{itemize}

\textbf{Threats Protection:}

\begin{itemize}
\tightlist
\item
  \textbf{Malware}: Antivirus software and sandboxing
\item
  \textbf{Unauthorized access}: Strong passwords and multi-factor
  authentication
\item
  \textbf{Data breaches}: Encryption and backup strategies
\end{itemize}

\end{solutionbox}
\begin{mnemonicbox}
``AAAEAF'' - Authentication, Authorization, Access
control, Encryption, Audit, Firewall

\end{mnemonicbox}
\begin{center}\rule{0.5\linewidth}{0.5pt}\end{center}

\subsection*{Question 4(a) OR [3
marks]}\label{q4a}

\textbf{List out ways to deal with deadlock. Explain deadlock detection
and recovery.}

\begin{solutionbox}


{\def\LTcaptype{none} % do not increment counter
\vspace{-5pt}
\captionof{table}{Deadlock Handling Methods}
\vspace{-10pt}
\begin{longtable}[]{@{}ll@{}}
\toprule\noalign{}
Method & Approach \\
\midrule\noalign{}
\endhead
\bottomrule\noalign{}
\endlastfoot
\textbf{Prevention} & Ensure at least one Coffman condition cannot
hold \\
\textbf{Avoidance} & Dynamically examine resource allocation state \\
\textbf{Detection \& Recovery} & Allow deadlock, then detect and
recover \\
\textbf{Ignore} & Assume deadlock never occurs (Ostrich algorithm) \\
\end{longtable}
}

\textbf{Deadlock Detection:}

\begin{itemize}
\tightlist
\item
  \textbf{Wait-for graph}: Maintain graph of process dependencies
\item
  \textbf{Detection algorithm}: Periodically check for cycles in graph
\item
  \textbf{Resource allocation graph}: Track resource ownership and
  requests
\end{itemize}

\textbf{Deadlock Recovery:}

\begin{itemize}
\tightlist
\item
  \textbf{Process termination}: Kill one or more deadlocked processes
\item
  \textbf{Resource preemption}: Take resources from processes
\item
  \textbf{Rollback}: Return processes to safe state using checkpoints
\end{itemize}

\end{solutionbox}
\begin{mnemonicbox}
``PADI'' - Prevention, Avoidance, Detection, Ignore

\end{mnemonicbox}
\subsection*{Question 4(b) OR [4
marks]}\label{q4b}

\textbf{List out File allocation methods. Explain any one.}

\begin{solutionbox}


{\def\LTcaptype{none} % do not increment counter
\vspace{-5pt}
\captionof{table}{File Allocation Methods}
\vspace{-10pt}
\begin{longtable}[]{@{}ll@{}}
\toprule\noalign{}
Method & Description \\
\midrule\noalign{}
\endhead
\bottomrule\noalign{}
\endlastfoot
\textbf{Contiguous} & Allocate consecutive disk blocks \\
\textbf{Linked} & Use pointers to link scattered blocks \\
\textbf{Indexed} & Use index block to store block addresses \\
\end{longtable}
}

\textbf{Contiguous Allocation:}

\begin{itemize}
\tightlist
\item
  \textbf{Structure}: File occupies consecutive blocks on disk
\item
  \textbf{Advantages}: Fast access, simple implementation, good for
  sequential access
\item
  \textbf{Disadvantages}: External fragmentation, difficult to grow
  files
\item
  \textbf{Directory entry}: Contains starting address and length
\end{itemize}

\textbf{Example:} File ``test.txt'' starts at block 100, length 5 blocks
Occupies blocks: 100, 101, 102, 103, 104

\end{solutionbox}
\begin{mnemonicbox}
``CLI'' - Contiguous (consecutive), Linked
(pointers), Indexed (table)

\end{mnemonicbox}
\subsection*{Question 4(c) OR [7
marks]}\label{q4c}

\textbf{Describe program threats and system threats.}

\begin{solutionbox}

\textbf{Program Threats} are malicious software that can harm system or
data.


{\def\LTcaptype{none} % do not increment counter
\vspace{-5pt}
\captionof{table}{Program Threats}
\vspace{-10pt}
\begin{longtable}[]{@{}
  >{\raggedright\arraybackslash}p{(\linewidth - 2\tabcolsep) * \real{0.5000}}
  >{\raggedright\arraybackslash}p{(\linewidth - 2\tabcolsep) * \real{0.5000}}@{}}
\toprule\noalign{}
\begin{minipage}[b]{\linewidth}\raggedright
Threat Type
\end{minipage} & \begin{minipage}[b]{\linewidth}\raggedright
Description
\end{minipage} \\
\midrule\noalign{}
\endhead
\bottomrule\noalign{}
\endlastfoot
\textbf{Virus} & Self-replicating code that infects other programs \\
\textbf{Worm} & Standalone malware that spreads across networks \\
\textbf{Trojan Horse} & Malicious code disguised as legitimate
software \\
\textbf{Logic Bomb} & Code that triggers malicious action on specific
event \\
\textbf{Backdoor} & Hidden access point bypassing normal
authentication \\
\end{longtable}
}

\textbf{System Threats} target operating system and system resources.


{\def\LTcaptype{none} % do not increment counter
\vspace{-5pt}
\captionof{table}{System Threats}
\vspace{-10pt}
\begin{longtable}[]{@{}
  >{\raggedright\arraybackslash}p{(\linewidth - 2\tabcolsep) * \real{0.5000}}
  >{\raggedright\arraybackslash}p{(\linewidth - 2\tabcolsep) * \real{0.5000}}@{}}
\toprule\noalign{}
\begin{minipage}[b]{\linewidth}\raggedright
Threat Type
\end{minipage} & \begin{minipage}[b]{\linewidth}\raggedright
Description
\end{minipage} \\
\midrule\noalign{}
\endhead
\bottomrule\noalign{}
\endlastfoot
\textbf{Buffer Overflow} & Overflow input buffers to execute malicious
code \\
\textbf{Denial of Service} & Overwhelm system resources to make service
unavailable \\
\textbf{Privilege Escalation} & Gain higher access privileges than
authorized \\
\textbf{Man-in-the-Middle} & Intercept communication between two
parties \\
\end{longtable}
}

\textbf{Protection Strategies:}

\begin{itemize}
\tightlist
\item
  \textbf{Antivirus software}: Detect and remove malicious programs
\item
  \textbf{Regular updates}: Patch security vulnerabilities
\item
  \textbf{Access controls}: Limit user privileges and resource access
\item
  \textbf{Network monitoring}: Detect suspicious activities
\end{itemize}

\end{solutionbox}
\begin{mnemonicbox}
``VWTLB-BPDM'' - Virus, Worm, Trojan, Logic bomb,
Backdoor; Buffer overflow, Privilege escalation, DoS, Man-in-middle

\end{mnemonicbox}
\begin{center}\rule{0.5\linewidth}{0.5pt}\end{center}

\subsection*{Question 5(a) [3 marks]}\label{q5a}

\textbf{Explain Inter Process Communication.}

\begin{solutionbox}

\textbf{Inter Process Communication (IPC)} enables processes to exchange
data and synchronize activities.


{\def\LTcaptype{none} % do not increment counter
\vspace{-5pt}
\captionof{table}{IPC Mechanisms}
\vspace{-10pt}
\begin{longtable}[]{@{}ll@{}}
\toprule\noalign{}
Mechanism & Description \\
\midrule\noalign{}
\endhead
\bottomrule\noalign{}
\endlastfoot
\textbf{Pipes} & Unidirectional communication channel \\
\textbf{Message Queues} & Structured message passing \\
\textbf{Shared Memory} & Common memory area for multiple processes \\
\textbf{Semaphores} & Synchronization using counters \\
\textbf{Signals} & Software interrupts for notification \\
\end{longtable}
}

\begin{itemize}
\tightlist
\item
  \textbf{Synchronous communication}: Sender waits for receiver
  acknowledgment
\item
  \textbf{Asynchronous communication}: Sender continues without waiting
\item
  \textbf{Buffering}: Messages stored temporarily if receiver not ready
\end{itemize}

\end{solutionbox}
\begin{mnemonicbox}
``PMSSS'' - Pipes, Message queues, Shared memory,
Semaphores, Signals

\end{mnemonicbox}
\subsection*{Question 5(b) [4 marks]}\label{q5b}

\textbf{Explain File structure used by Linux.}

\begin{solutionbox}

\textbf{Linux File System} follows hierarchical directory structure
starting from root directory.

\textbf{Diagram: Linux File System Hierarchy}

\begin{verbatim}
         /
        /|{}
       / | {}
    bin  etc  home
    |    |    |
   ls   passwd user1
   cat  hosts   |
   cp          Documents
              Pictures
\end{verbatim}


{\def\LTcaptype{none} % do not increment counter
\vspace{-5pt}
\captionof{table}{Important Directories}
\vspace{-10pt}
\begin{longtable}[]{@{}ll@{}}
\toprule\noalign{}
Directory & Purpose \\
\midrule\noalign{}
\endhead
\bottomrule\noalign{}
\endlastfoot
\textbf{/} & Root directory, top of hierarchy \\
\textbf{/bin} & Essential user commands \\
\textbf{/etc} & System configuration files \\
\textbf{/home} & User home directories \\
\textbf{/var} & Variable data (logs, mail) \\
\textbf{/usr} & User programs and utilities \\
\textbf{/tmp} & Temporary files \\
\end{longtable}
}

\begin{itemize}
\tightlist
\item
  \textbf{Case sensitive}: Distinguishes between File.txt and file.txt
\item
  \textbf{No drive letters}: Everything under single root directory
\item
  \textbf{Mount points}: External devices appear as subdirectories
\end{itemize}

\end{solutionbox}
\begin{mnemonicbox}
``BEHVUT'' - Bin, Etc, Home, Var, Usr, Tmp

\end{mnemonicbox}
\subsection*{Question 5(c) [7 marks]}\label{q5c}

\textbf{Explain operating system security policies and procedures.}

\begin{solutionbox}

\textbf{Security Policies} define rules and guidelines for protecting
system resources and data.


{\def\LTcaptype{none} % do not increment counter
\vspace{-5pt}
\captionof{table}{Security Policy Components}
\vspace{-10pt}
\begin{longtable}[]{@{}ll@{}}
\toprule\noalign{}
Component & Description \\
\midrule\noalign{}
\endhead
\bottomrule\noalign{}
\endlastfoot
\textbf{Access Control Policy} & Who can access what resources \\
\textbf{Password Policy} & Requirements for strong passwords \\
\textbf{Audit Policy} & What activities to monitor and log \\
\textbf{Backup Policy} & Data backup and recovery procedures \\
\textbf{Incident Response} & Steps to handle security breaches \\
\end{longtable}
}

\textbf{Security Procedures:}

\textbf{Authentication Procedures:}

\begin{itemize}
\tightlist
\item
  \textbf{Multi-factor authentication}: Password + token/biometric
\item
  \textbf{Password complexity}: Minimum length, special characters
\item
  \textbf{Account lockout}: Temporary disable after failed attempts
\end{itemize}

\textbf{Authorization Procedures:}

\begin{itemize}
\tightlist
\item
  \textbf{Principle of least privilege}: Minimum necessary access
\item
  \textbf{Role-based access}: Assign permissions based on job function
\item
  \textbf{Regular review}: Periodic audit of user permissions
\end{itemize}

\textbf{Monitoring Procedures:}

\begin{itemize}
\tightlist
\item
  \textbf{Log analysis}: Review system and security logs
\item
  \textbf{Intrusion detection}: Monitor for unauthorized access
\item
  \textbf{Vulnerability scanning}: Identify security weaknesses
\end{itemize}

\end{solutionbox}
\begin{mnemonicbox}
``APABI'' - Access control, Password, Audit, Backup,
Incident response

\end{mnemonicbox}
\begin{center}\rule{0.5\linewidth}{0.5pt}\end{center}

\subsection*{Question 5(a) OR [3
marks]}\label{q5a}

\textbf{Explain Critical section.}

\begin{solutionbox}

\textbf{Critical Section} is code segment where process accesses shared
resources that must not be accessed concurrently.


{\def\LTcaptype{none} % do not increment counter
\vspace{-5pt}
\captionof{table}{Critical Section Properties}
\vspace{-10pt}
\begin{longtable}[]{@{}
  >{\raggedright\arraybackslash}p{(\linewidth - 2\tabcolsep) * \real{0.4348}}
  >{\raggedright\arraybackslash}p{(\linewidth - 2\tabcolsep) * \real{0.5652}}@{}}
\toprule\noalign{}
\begin{minipage}[b]{\linewidth}\raggedright
Property
\end{minipage} & \begin{minipage}[b]{\linewidth}\raggedright
Description
\end{minipage} \\
\midrule\noalign{}
\endhead
\bottomrule\noalign{}
\endlastfoot
\textbf{Mutual Exclusion} & Only one process in critical section at a
time \\
\textbf{Progress} & Selection of next process cannot be postponed
indefinitely \\
\textbf{Bounded Waiting} & Limit on number of times other processes
enter critical section \\
\end{longtable}
}

\textbf{Critical Section Structure:}

\begin{verbatim}
do {
    entry_section();     // Request permission
    critical_section();  // Access shared resource
    exit_section();      // Release permission
    remainder_section(); // Other work
} while(true);
\end{verbatim}

\textbf{Solutions:}

\begin{itemize}
\tightlist
\item
  \textbf{Peterson's algorithm}: Software solution for two processes
\item
  \textbf{Semaphores}: Hardware-supported synchronization
\item
  \textbf{Mutex locks}: Binary semaphore for mutual exclusion
\end{itemize}

\end{solutionbox}
\begin{mnemonicbox}
``MPB'' - Mutual exclusion, Progress, Bounded waiting

\end{mnemonicbox}
\subsection*{Question 5(b) OR [4
marks]}\label{q5b}

\textbf{Explain types of Linux file system.}

\begin{solutionbox}

\textbf{Linux File Systems} organize and manage data storage on disk
devices.


{\def\LTcaptype{none} % do not increment counter
\vspace{-5pt}
\captionof{table}{Linux File System Types}
\vspace{-10pt}
\begin{longtable}[]{@{}ll@{}}
\toprule\noalign{}
File System & Description \\
\midrule\noalign{}
\endhead
\bottomrule\noalign{}
\endlastfoot
\textbf{ext4} & Fourth extended file system, most common \\
\textbf{XFS} & High-performance journaling file system \\
\textbf{Btrfs} & B-tree file system with advanced features \\
\textbf{ZFS} & Zettabyte file system with built-in RAID \\
\textbf{NTFS} & Windows file system support \\
\textbf{FAT32} & Simple file system for compatibility \\
\end{longtable}
}

\textbf{ext4 Features:}

\begin{itemize}
\tightlist
\item
  \textbf{Journaling}: Faster recovery after system crash
\item
  \textbf{Large file support}: Files up to 16TB
\item
  \textbf{Backwards compatibility}: Can mount ext2/ext3 partitions
\item
  \textbf{Extents}: Improve performance for large files
\end{itemize}

\textbf{File System Selection Factors:}

\begin{itemize}
\tightlist
\item
  \textbf{Performance requirements}: Speed vs reliability
\item
  \textbf{File size limits}: Maximum file and partition sizes
\item
  \textbf{Compatibility needs}: Cross-platform support
\end{itemize}

\end{solutionbox}
\begin{mnemonicbox}
``EXBZNF'' - Ext4, XFS, Btrfs, ZFS, NTFS, FAT32

\end{mnemonicbox}
\subsection*{Question 5(c) OR [7
marks]}\label{q5c}

\textbf{Explain need of protection mechanism and various protection
domain.}

\begin{solutionbox}

\textbf{Protection Mechanism} prevents processes from interfering with
each other and system resources.

\textbf{Need for Protection:}

\begin{itemize}
\tightlist
\item
  \textbf{Resource sharing}: Multiple users/processes access same
  resources
\item
  \textbf{Error containment}: Prevent bugs from affecting entire system
\item
  \textbf{Security enforcement}: Implement access control policies
\item
  \textbf{System stability}: Protect critical system components
\end{itemize}


{\def\LTcaptype{none} % do not increment counter
\vspace{-5pt}
\captionof{table}{Protection Domains}
\vspace{-10pt}
\begin{longtable}[]{@{}ll@{}}
\toprule\noalign{}
Domain Type & Description \\
\midrule\noalign{}
\endhead
\bottomrule\noalign{}
\endlastfoot
\textbf{User Domain} & Limited access rights for user processes \\
\textbf{Kernel Domain} & Full access to system resources \\
\textbf{System Domain} & Intermediate privileges for system services \\
\end{longtable}
}

\textbf{Protection Mechanisms:}

\textbf{Hardware Protection:}

\begin{itemize}
\tightlist
\item
  \textbf{Memory protection}: Base and limit registers
\item
  \textbf{CPU protection}: Timer interrupts prevent infinite loops
\item
  \textbf{I/O protection}: Privileged instructions for device access
\end{itemize}

\textbf{Software Protection:}

\begin{itemize}
\tightlist
\item
  \textbf{Access control lists}: Define resource permissions
\item
  \textbf{Capability lists}: Token-based access control
\item
  \textbf{Domain switching}: Change protection levels safely
\end{itemize}


{\def\LTcaptype{none} % do not increment counter
\vspace{-5pt}
\captionof{table}{Access Rights}
\vspace{-10pt}
\begin{longtable}[]{@{}ll@{}}
\toprule\noalign{}
Right & Description \\
\midrule\noalign{}
\endhead
\bottomrule\noalign{}
\endlastfoot
\textbf{Read} & View content of resource \\
\textbf{Write} & Modify resource content \\
\textbf{Execute} & Run program or enter directory \\
\textbf{Append} & Add data without modifying existing \\
\textbf{Delete} & Remove resource from system \\
\end{longtable}
}

\end{solutionbox}
\begin{mnemonicbox}
``RECES-UKS'' - Resource sharing, Error containment,
Security; User domain, Kernel domain, System domain

\end{mnemonicbox}

\end{document}
