\documentclass{article}

% content/resources/templates/preamble.tex
\usepackage[margin=0.6in]{geometry}
\author{Milav Dabgar}
\usepackage{amsmath,amssymb,amsthm}
\usepackage{booktabs}
\usepackage{multirow}
\usepackage{xcolor}
\usepackage{tcolorbox}
\tcbuselibrary{breakable,skins}
\usepackage[colorlinks=true,linkcolor=blue]{hyperref}
\usepackage{titlesec}
\usepackage{enumitem}
\usepackage{tikz}
\usepackage{pgfplots}
\usepackage{circuitikz}
\usepackage[version=4]{mhchem}
\usepackage{longtable}
\usepackage{array}
\usepackage{float}
\usepackage{caption}
\usepackage{listings}

\lstset{
  basicstyle=\small\ttfamily,
  breaklines=true,
  breakatwhitespace=false,
  postbreak=\mbox{\textcolor{red}{$\hookrightarrow$}\space},
  float=false,
  numbers=left,
  numberstyle=\tiny\color{gray},
  numbersep=10pt,
  xleftmargin=2em,
  keywordstyle=\color{blue},
  commentstyle=\color{green!60!black},
  stringstyle=\color{purple},
  backgroundcolor=\color{gray!5},
  showstringspaces=false,
  tabsize=2,
  captionpos=b,
  keepspaces=true,
  columns=flexible
}

\pgfplotsset{compat=1.18}
\usetikzlibrary{shapes,arrows,positioning,calc,patterns,decorations.pathmorphing,decorations.markings,arrows.meta}

% Color scheme
\definecolor{headcolor}{RGB}{0,102,204}
\definecolor{keycolor}{RGB}{220,20,60}
\definecolor{solutioncolor}{RGB}{34,139,34}
\definecolor{mnemoniccolor}{RGB}{148,0,211}
\definecolor{codecolor}{RGB}{0,0,100}

% Spacing
\setlength{\parskip}{3pt}
\setlist[itemize]{nosep}
\setlist[enumerate]{nosep}

% Title formatting
\titleformat{\section}{\Large\bfseries\color{headcolor}}{\thesection}{1em}{}
\titleformat{\subsection}{\large\bfseries\color{headcolor}}{\thesubsection}{1em}{}

% Pandoc tightlist compatibility
\providecommand{\tightlist}{%
  \setlength{\itemsep}{0pt}\setlength{\parskip}{0pt}}

% Pandoc longtable compatibility
\newcounter{none}
\def\thenone{}


% content/resources/templates/gujarati-boxes.tex
\usepackage{fontspec}
\usepackage{polyglossia}

% Set Gujarati as main language (document is primarily in Gujarati)
% Note: gloss-gujarati.ldf doesn't exist in polyglossia, but it will use hyphenation patterns
\setdefaultlanguage{gujarati}
\setotherlanguage{english}

% Configure Gujarati font properly
% Use Language=Default to prevent polyglossia from trying to add language-specific features
% that don't exist for Gujarati, which causes "empty feature" warnings
\newfontfamily\gujaratifont[Script=Gujarati,AutoFakeBold=2.5,AutoFakeSlant=0.3]{Noto Sans Gujarati}
\setmainfont[Script=Gujarati,AutoFakeBold=2.5,AutoFakeSlant=0.3]{Noto Sans Gujarati}
% Use Noto Sans Gujarati for monospace to support Gujarati in text
\setmonofont[Scale=0.9]{Noto Sans Gujarati}

% Configure English to use the same font
\newfontfamily\englishfont[Script=Gujarati,AutoFakeBold=2.5,AutoFakeSlant=0.3]{Noto Sans Gujarati}

% Translations for polyglossia
\gappto\captionsgujarati{
  \renewcommand{\tablename}{કોષ્ટક}
  \renewcommand{\figurename}{આકૃતિ}
}

% Helper for TikZ nodes to ensure Gujarati font
\newcommand{\gu}[1]{{\gujaratifont #1}}

% Custom environments
\newtcolorbox{solutionbox}{
    breakable,
    enhanced,
    colback=solutioncolor!5!white,
    colframe=solutioncolor!75!black,
    fonttitle=\bfseries,
    title=જવાબ
}

\newtcolorbox{solutionboxnobreak}{
 colback=solutioncolor!5!white,
 colframe=solutioncolor!75!black,
 fonttitle=\bfseries,
 title=જવાબ
}

\newtcolorbox{keyformula}{
 breakable,
 enhanced,
 colback=keycolor!5!white,
 colframe=keycolor!75!black,
 fonttitle=\bfseries,
 title=રાસાયણિક સમીકરણ/સૂત્ર
}

\newtcolorbox{mnemonicbox}{
 breakable,
 enhanced,
 colback=mnemoniccolor!5!white,
 colframe=mnemoniccolor!75!black,
 fonttitle=\bfseries,
 title=મેમરી ટ્રીક
}


% Custom commands for GTU solutions
% This file defines semantic commands for consistent formatting

% Question command with automatic formatting
\newcommand{\question}[2]{%
  \section*{Question #1}%
  \textbf{#2}%
}

% OR question variant
\newcommand{\questionor}[2]{%
  \section*{Question #1 OR}%
  \textbf{#2}%
}

% Proper table environment with caption
\newenvironment{answertable}[1]{%
  \begin{table}[htbp]
  \centering
  \caption{#1}
}{%
  \end{table}
}

% Proper figure environment for diagrams
\newenvironment{answerdiagram}[1]{%
  \begin{figure}[htbp]
  \centering
  \caption{#1}
}{%
  \end{figure}
}

% Semantic markup for key terms
\newcommand{\keyword}[1]{\textbf{#1}}
\newcommand{\code}[1]{\texttt{#1}}
\newcommand{\classname}[1]{\texttt{#1}}
\newcommand{\methodname}[1]{\texttt{#1}}

% Proper quotation marks
\newcommand{\mnemonic}[1]{``#1''}


\title{લિનક્સ ઓપરેટિંગ સિસ્ટમ (4331602) - ગ્રીષ્મ 2025 સોલ્યુશન}
\date{13 મે, 2025}

\begin{document}
\maketitle

\questionmarks{1(a)}{3}{ઓપરેટિંગ સિસ્ટમ વ્યાખ્યાયિત કરો અને OS ની જરૂરિયાત સમજાવો.}

\begin{solutionbox}
\textbf{જવાબ}:

\textbf{ઓપરેટિંગ સિસ્ટમ} એ સિસ્ટમ સોફ્ટવેર છે જે કમ્પ્યુટર હાર્ડવેર અને એપ્લિકેશન સોફ્ટવેર વચ્ચે મધ્યસ્થી તરીકે કામ કરે છે. તે હાર્ડવેર રિસોર્સનું સંચાલન કરે છે અને યુઝર પ્રોગ્રામ્સને સેવાઓ પ્રદાન કરે છે.

\textbf{ઓપરેટિંગ સિસ્ટમની જરૂરિયાત:}
\begin{itemize}
    \item \keyword{રિસોર્સ મેનેજમેન્ટ}: CPU, મેમરી, સ્ટોરેજ અને I/O ડિવાઇસનું કાર્યક્ષમ સંચાલન
    \item \keyword{યુઝર ઇન્ટરફેસ}: યુઝર ઇન્ટરેક્શન માટે કમાન્ડ-લાઇન અને ગ્રાફિકલ ઇન્ટરફેસ પ્રદાન કરે છે
    \item \keyword{પ્રોગ્રામ એક્ઝિક્યુશન}: યુઝર પ્રોગ્રામ્સને સુરક્ષિત રીતે લોડ અને એક્ઝિક્યુટ કરે છે
\end{itemize}
\end{solutionbox}

\begin{mnemonicbox}
\mnemonic{RUP - રિસોર્સ, યુઝર, પ્રોગ્રામ મેનેજમેન્ટ}
\end{mnemonicbox}

\questionmarks{1(b)}{4}{પ્રક્રિયા નિયંત્રણ બ્લોક (PCB) પર એક ટૂંકી નોંધ લખો.}

\begin{solutionbox}
\textbf{જવાબ}:

પ્રોસેસ કન્ટ્રોલ બ્લોક (PCB) એ ડેટા સ્ટ્રક્ચર છે જે ઓપરેટિંગ સિસ્ટમ દ્વારા દરેક ચાલતી પ્રક્રિયા માટે જાળવવામાં આવે છે.

\begin{center}
\captionof{table}{PCB ઘટકો}
\begin{tabulary}{\linewidth}{|L|L|}
\hline
\textbf{PCB ઘટક} & \textbf{વર્ણન} \\ \hline
પ્રોસેસ ID & પ્રક્રિયા માટે અનન્ય ઓળખકર્તા \\ \hline
પ્રોસેસ સ્ટેટ & વર્તમાન સ્થિતિ (તૈયાર, ચાલુ, રાહ જોવી) \\ \hline
પ્રોગ્રામ કાઉન્ટર & એક્ઝિક્યુટ કરવાની આગળની instruction નું સરનામું \\ \hline
CPU રજિસ્ટર્સ & પ્રક્રિયા suspend થાય ત્યારે CPU રજિસ્ટર્સની કિંમતો \\ \hline
મેમરી મેનેજમેન્ટ & બેઝ અને લિમિટ રજિસ્ટર્સ, પેજ ટેબલ્સ \\ \hline
I/O સ્ટેટસ & ખુલ્લી ફાઇલો અને I/O ડિવાઇસની યાદી \\ \hline
\end{tabulary}
\end{center}

\textbf{મુખ્ય કાર્યો:}
\begin{itemize}
    \item \keyword{પ્રક્રિયા ઓળખ}: અનન્ય પ્રોસેસ ID અને પેરેન્ટ પ્રોસેસ ID સ્ટોર કરે છે
    \item \keyword{સ્ટેટ ઇન્ફર્મેશન}: વર્તમાન એક્ઝિક્યુશન સ્ટેટ અને કન્ટેક્સ્ટ જાળવે છે
    \item \keyword{રિસોર્સ એલોકેશન}: ફાળવેલ રિસોર્સ અને મેમરી ઉપયોગનું ટ્રેકિંગ કરે છે
\end{itemize}
\end{solutionbox}

\begin{mnemonicbox}
\mnemonic{PIS - Process ID, Information, State tracking}
\end{mnemonicbox}

\questionmarks{1(c)}{7}{વિવિધ પ્રકારની ઓપરેટિંગ સિસ્ટમોની યાદી બનાવો. બેચ ઓપરેટિંગ સિસ્ટમના કાર્યને યોગ્ય ઉદાહરણ સાથે સમજાવો.}

\begin{solutionbox}
\textbf{જવાબ}:

\textbf{ઓપરેટિંગ સિસ્ટમના પ્રકારો:}
\begin{center}
\captionof{table}{ઓપરેટિંગ સિસ્ટમના પ્રકારો}
\begin{tabulary}{\linewidth}{|L|L|}
\hline
\textbf{પ્રકાર} & \textbf{વર્ણન} \\ \hline
બેચ OS & સમાન જોબ્સને જૂથમાં મૂકીને એકસાથે એક્ઝિક્યુટ કરે છે \\ \hline
ટાઇમ-શેરિંગ OS & બહુવિધ વપરાશકર્તાઓ સિસ્ટમને એકસાથે શેર કરે છે \\ \hline
રીયલ-ટાઇમ OS & નિશ્ચિત રિસ્પોન્સ ટાઇમની ગેરંટી આપે છે \\ \hline
ડિસ્ટ્રિબ્યુટેડ OS & બહુવિધ કનેક્ટેડ કમ્પ્યુટર્સનું સંચાલન કરે છે \\ \hline
નેટવર્ક OS & નેટવર્ક સેવાઓ અને રિસોર્સ શેરિંગ પ્રદાન કરે છે \\ \hline
મોબાઇલ OS & મોબાઇલ ડિવાઇસ માટે ડિઝાઇન કરેલ \\ \hline
\end{tabulary}
\end{center}

\textbf{બેચ ઓપરેટિંગ સિસ્ટમનું કાર્ય:}

\begin{center}
\begin{tikzpicture}[node distance=1.5cm, auto]
    \node [gtu block] (j1) {Job 1 \\ COBOL};
    \node [gtu block, right=0.5cm of j1] (j2) {Job 2 \\ FORTRAN};
    \node [gtu block, right=0.5cm of j2] (j3) {Job 3 \\ C++};
    \node [gtu block, right=0.5cm of j3] (j4) {Job 4 \\ JAVA};
    
    \node [gtu block, below=1.5cm of j2, xshift=1cm, minimum width=6cm] (manager) {Batch Queue Manager};
    
    \node [gtu block, below=1.5cm of manager, minimum width=2cm] (memory) {Memory \\ Manager};
    \node [gtu block, left=0.5cm of memory, minimum width=2cm] (cpu) {CPU \\ Executor};
    \node [gtu block, right=0.5cm of memory, minimum width=2cm] (io) {I/O \\ Handler};
    
    \draw [gtu arrow] (j1) -- (manager.north -| j1);
    \draw [gtu arrow] (j2) -- (manager.north -| j2);
    \draw [gtu arrow] (j3) -- (manager.north -| j3);
    \draw [gtu arrow] (j4) -- (manager.north -| j4);
    
    \draw [gtu arrow] (manager) -- (memory);
    \draw [gtu arrow] (memory) -- (cpu);
    \draw [gtu arrow] (memory) -- (io);
\end{tikzpicture}
\captionof{figure}{બેચ ઓપરેટિંગ સિસ્ટમ}
\end{center}

\textbf{ઉદાહરણ}: બેંક ટ્રાન્ઝેક્શન પ્રોસેસિંગ જ્યાં દિવસભરના બધા ટ્રાન્ઝેક્શન્સ એકત્રિત કરીને રાત્રે કાર્યક્ષમતા માટે એકસાથે પ્રોસેસ કરવામાં આવે છે.

\textbf{મુખ્ય લક્ષણો:}
\begin{itemize}
    \item \keyword{જોબ ગ્રુપિંગ}: કાર્યક્ષમતા માટે સમાન જોબ્સ એકસાથે એક્ઝિક્યુટ કરવામાં આવે છે
    \item \keyword{કોઈ યુઝર ઇન્ટરેક્શન નહીં}: એકવાર સબમિટ કર્યા પછી જોબ્સ યુઝર દખલ વિના ચાલે છે
    \item \keyword{ઉચ્ચ થ્રુપુટ}: સિસ્ટમ ઉપયોગને મહત્તમ બનાવે છે
\end{itemize}
\end{solutionbox}

\begin{mnemonicbox}
\mnemonic{JNH - Jobs grouped, No interaction, High throughput}
\end{mnemonicbox}

\questionmarks{1(c) OR}{7}{વિવિધ પ્રકારની ઓપરેટિંગ સિસ્ટમોની યાદી બનાવો. રીયલ ટાઇમ ઓપરેટિંગ સિસ્ટમ્સ વિગતવાર સમજાવો.}

\begin{solutionbox}
\textbf{જવાબ}:

\textbf{ઓપરેટિંગ સિસ્ટમના પ્રકારો:}
(ઉપરની જેમ સમાન ટેબલ)

\textbf{રીયલ-ટાઇમ ઓપરેટિંગ સિસ્ટમ (RTOS):}
રીયલ-ટાઇમ OS એ નિર્દિષ્ટ સમય મર્યાદામાં ગેરંટીડ રિસ્પોન્સ પ્રદાન કરે છે.

\textbf{RTOS ના પ્રકારો:}
\begin{center}
\captionof{table}{RTOS ના પ્રકારો}
\begin{tabulary}{\linewidth}{|L|L|L|}
\hline
\textbf{પ્રકાર} & \textbf{ડેડલાઇન} & \textbf{ઉદાહરણ} \\ \hline
હાર્ડ રીયલ-ટાઇમ & ડેડલાઇન પૂરી કરવી જ જોઈએ & એર ટ્રાફિક કંટ્રોલ, પેસમેકર \\ \hline
સોફ્ટ રીયલ-ટાઇમ & થોડો વિલંબ સહન કરી શકે & વિડિયો સ્ટ્રીમિંગ, ઓનલાઇન ગેમિંગ \\ \hline
ફર્મ રીયલ-ટાઇમ & કભીકભાર ડેડલાઇન મિસ સ્વીકાર્ય & લાઇવ ઓડિયો પ્રોસેસિંગ \\ \hline
\end{tabulary}
\end{center}

\textbf{લક્ષણો:}
\begin{itemize}
    \item \keyword{નિર્ધારિત}: બધા ઓપરેશન માટે અનુમાનિત રિસ્પોન્સ ટાઇમ
    \item \keyword{પ્રાયોરિટી-આધારિત શેડ્યુલિંગ}: ઉચ્ચ પ્રાયોરિટી ટાસ્કને તાત્કાલિક ધ્યાન
    \item \keyword{ન્યૂનતમ ઇન્ટરપ્ટ લેટન્સી}: ઝડપી કન્ટેક્સ્ટ સ્વિચિંગ ક્ષમતાઓ
    \item \keyword{મેમરી મેનેજમેન્ટ}: વિલંબ વિના રીયલ-ટાઇમ મેમરી એલોકેશન
\end{itemize}

\textbf{એપ્લિકેશન્સ:}
\begin{itemize}
    \item મેડિકલ ડિવાઇસ, ઓટોમોટિવ સિસ્ટમ્સ, ઇન્ડસ્ટ્રિયલ ઓટોમેશન
\end{itemize}
\end{solutionbox}

\begin{mnemonicbox}
\mnemonic{DPMA - Deterministic, Priority-based, Minimal latency, Applications critical}
\end{mnemonicbox}

\questionmarks{2(a)}{3}{પ્રોગ્રામ અને પ્રક્રિયા વચ્ચે તફાવત કરો.}

\begin{solutionbox}
\textbf{જવાબ}:

\begin{center}
\captionof{table}{પ્રોગ્રામ vs પ્રક્રિયા}
\begin{tabulary}{\linewidth}{|L|L|L|}
\hline
\textbf{પાસું} & \textbf{પ્રોગ્રામ} & \textbf{પ્રક્રિયા} \\ \hline
વ્યાખ્યા & ડિસ્ક પર સંગ્રહિત સ્ટેટિક કોડ & એક્ઝિક્યુશનમાં પ્રોગ્રામ \\ \hline
સ્થિતિ & પેસિવ એન્ટિટી & એક્ટિવ એન્ટિટી \\ \hline
મેમરી & કોઈ મેમરી એલોકેશન નહીં & એલોકેટેડ મેમરી સ્પેસ \\ \hline
જીવનકાળ & ડિલીટ થાય ત્યાં સુધી કાયમી & એક્ઝિક્યુશન દરમિયાન અસ્થાયી \\ \hline
રિસોર્સ & કોઈ રિસોર્સ વપરાશ નહીં & CPU, મેમરી, I/O વપરાશ કરે છે \\ \hline
\end{tabulary}
\end{center}

\textbf{મુખ્ય તફાવતો:}
\begin{itemize}
    \item \keyword{સ્ટેટિક vs ડાયનેમિક}: પ્રોગ્રામ સ્ટેટિક ફાઇલ છે, પ્રક્રિયા ડાયનેમિક એક્ઝિક્યુશન છે
    \item \keyword{રિસોર્સ ઉપયોગ}: પ્રક્રિયા સિસ્ટમ રિસોર્સનો વપરાશ કરે છે, પ્રોગ્રામ નહીં
    \item \keyword{બહુવિધ ઇન્સ્ટન્સ}: એક પ્રોગ્રામ બહુવિધ પ્રક્રિયાઓ બનાવી શકે છે
\end{itemize}
\end{solutionbox}

\begin{mnemonicbox}
\mnemonic{SDR - Static vs Dynamic, Resource usage, Multiple instances}
\end{mnemonicbox}

\questionmarks{2(b)}{4}{પ્રક્રિયા સ્થિતિ રેખાકૃતિની મદદથી પ્રક્રિયાની વિવિધ અવસ્થાઓ સમજાવો.}

\begin{solutionbox}
\textbf{જવાબ}:

\begin{center}
\begin{tikzpicture}[node distance=2cm, auto]
    \node [gtu state] (new) {New};
    \node [gtu state, right=of new] (ready) {Ready};
    \node [gtu state, right=of ready] (running) {Running};
    \node [gtu state, right=of running] (term) {Terminated};
    \node [gtu state, below=of ready] (wait) {Waiting};

    \path [gtu arrow] (new) -- node {એડમિટેડ} (ready);
    \path [gtu arrow] (ready) -- node {શેડ્યૂલર ડિસ્પેચ} (running);
    \path [gtu arrow] (running) -- node [above] {એક્ઝિટ} (term);
    \path [gtu arrow] (running) edge [bend right] node [above] {ઇન્ટરપ્ટ} (ready);
    \path [gtu arrow] (running) -- node {I/O વેટ} (wait);
    \path [gtu arrow] (wait) -- node {I/O પૂર્ણ} (ready);
\end{tikzpicture}
\captionof{figure}{પ્રક્રિયા સ્થિતિ રેખાકૃતિ}
\end{center}

\textbf{પ્રક્રિયા સ્થિતિઓ:}
\begin{center}
\captionof{table}{પ્રક્રિયા સ્થિતિઓ}
\begin{tabulary}{\linewidth}{|L|L|}
\hline
\textbf{સ્થિતિ} & \textbf{વર્ણન} \\ \hline
New & પ્રક્રિયા બનાવવામાં આવી રહી છે \\ \hline
Ready & CPU એસાઇનમેન્ટની રાહ જોઈ રહી છે \\ \hline
Running & હાલમાં CPU પર એક્ઝિક્યુટ થઈ રહી છે \\ \hline
Waiting & I/O અથવા ઇવેન્ટ માટે બ્લોક થયેલ \\ \hline
Terminated & પ્રક્રિયાનું એક્ઝિક્યુશન પૂર્ણ થયું \\ \hline
\end{tabulary}
\end{center}

\textbf{સ્થિતિ પરિવર્તનો:}
\begin{itemize}
    \item \keyword{Ready થી Running}: પ્રોસેસ શેડ્યુલર CPU ફાળવે છે
    \item \keyword{Running થી Ready}: ટાઇમ સ્લાઇસ સમાપ્ત થાય છે
    \item \keyword{Running થી Waiting}: પ્રક્રિયા I/O ઓપરેશન માંગે છે
    \item \keyword{Waiting થી Ready}: I/O ઓપરેશન પૂર્ણ થાય છે
\end{itemize}
\end{solutionbox}

\begin{mnemonicbox}
\mnemonic{NRWRT - New, Ready, Waiting, Running, Terminated states}
\end{mnemonicbox}

\questionmarks{2(c)}{7}{રાઉન્ડ રોબિન અલ્ગોરિધમનું વર્ણન કરો અને ગણતરી કરો.}

\begin{solutionbox}
\textbf{જવાબ}:

\textbf{રાઉન્ડ રોબિન અલ્ગોરિધમ:}
રાઉન્ડ રોબિન એ પ્રીએમ્પ્ટિવ શેડ્યુલિંગ અલ્ગોરિધમ છે જ્યાં દરેક પ્રક્રિયાને સમાન CPU સમય (ક્વાન્ટમ) મળે છે.

\textbf{આપેલ ડેટા:}
\begin{itemize}
    \item ક્વાન્ટમ ટાઇમ = 4 ms
    \item કન્ટેક્સ્ટ સ્વિચ = 1 ms
\end{itemize}

\textbf{ગેન્ટ ચાર્ટ:}
\begin{center}
\begin{tikzpicture}[x=0.4cm, y=1cm]
    \draw (0,0) rectangle (4,1) node[midway] {P1};
    \draw (4,0) rectangle (5,1) node[midway, font=\tiny] {CS};
    \draw (5,0) rectangle (9,1) node[midway] {P3};
    \draw (9,0) rectangle (10,1) node[midway, font=\tiny] {CS};
    \draw (10,0) rectangle (14,1) node[midway] {P1};
    \draw (14,0) rectangle (15,1) node[midway, font=\tiny] {CS};
    \draw (15,0) rectangle (18,1) node[midway] {P2};
    \draw (18,0) rectangle (19,1) node[midway, font=\tiny] {CS};
    \draw (19,0) rectangle (23,1) node[midway] {P3};
    \draw (23,0) rectangle (24,1) node[midway, font=\tiny] {CS};
    \draw (24,0) rectangle (28,1) node[midway] {P4};
    \draw (28,0) rectangle (29,1) node[midway, font=\tiny] {CS};
    \draw (29,0) rectangle (31,1) node[midway] {P3};
    \draw (31,0) rectangle (32,1) node[midway, font=\tiny] {CS};
    \draw (32,0) rectangle (33,1) node[midway] {P4};
    
    \node[below] at (0,0) {0};
    \node[below] at (4,0) {4};
    \node[below] at (5,0) {5};
    \node[below] at (9,0) {9};
    \node[below] at (10,0) {10};
    \node[below] at (14,0) {14};
    \node[below] at (15,0) {15};
    \node[below] at (18,0) {18};
    \node[below] at (19,0) {19};
    \node[below] at (23,0) {23};
    \node[below] at (24,0) {24};
    \node[below] at (28,0) {28};
    \node[below] at (29,0) {29};
    \node[below] at (31,0) {31};
    \node[below] at (32,0) {32};
    \node[below] at (33,0) {33};
\end{tikzpicture}
\captionof{figure}{ગેન્ટ ચાર્ટ (રાઉન્ડ રોબિન)}
\end{center}

\textbf{ગણતરીઓ:}
\begin{center}
\captionof{table}{ગણતરીઓ}
\begin{tabulary}{\linewidth}{|L|L|L|L|}
\hline
\textbf{પ્રક્રિયા} & \textbf{કમ્પ્લીશન} & \textbf{ટર્નઅરાઉન્ડ} & \textbf{વેઇટિંગ} \\ \hline
P1 & 14 & 14 - 0 = 14 & 14 - 8 = 6 \\ \hline
P2 & 18 & 18 - 3 = 15 & 15 - 3 = 12 \\ \hline
P3 & 31 & 31 - 1 = 30 & 30 - 10 = 20 \\ \hline
P4 & 33 & 33 - 4 = 29 & 29 - 5 = 24 \\ \hline
\end{tabulary}
\end{center}

\textbf{સરેરાશ વેઇટિંગ ટાઇમ} = $15.5$ ms \\
\textbf{સરેરાશ ટર્નઅરાઉન્ડ ટાઇમ} = $22$ ms
\end{solutionbox}

\begin{mnemonicbox}
\mnemonic{FPC - Fair, Preemptive, Context switching overhead}
\end{mnemonicbox}

\questionmarks{2(a) OR}{3}{તફાવત કરો: CPU બાઉન્ડ પ્રક્રિયા v/s I/O બાઉન્ડ પ્રક્રિયા.}

\begin{solutionbox}
\textbf{જવાબ}:

\begin{center}
\captionof{table}{CPU બાઉન્ડ vs I/O બાઉન્ડ}
\begin{tabulary}{\linewidth}{|L|L|L|}
\hline
\textbf{પાસું} & \textbf{CPU બાઉન્ડ} & \textbf{I/O બાઉન્ડ} \\ \hline
પ્રાથમિક પ્રવૃત્તિ & સઘન ગણતરીઓ & વારંવાર I/O ઓપરેશન્સ \\ \hline
CPU ઉપયોગ & ઉચ્ચ CPU ઉપયોગ & નીચો CPU ઉપયોગ \\ \hline
બર્સ્ટ ટાઇમ & લાંબા CPU બર્સ્ટ્સ & ટૂંકા CPU બર્સ્ટ્સ \\ \hline
વેઇટિંગ ટાઇમ & ઓછી I/O રાહ & વધુ I/O રાહ \\ \hline
ઉદાહરણો & ગાણિતિક ગણતરીઓ & ફાઇલ ઓપરેશન્સ \\ \hline
\end{tabulary}
\end{center}
\end{solutionbox}

\begin{mnemonicbox}
\mnemonic{CIR - CPU intensive, I/O intensive, Resource usage differs}
\end{mnemonicbox}

\questionmarks{2(b) OR}{4}{ડેડલોક શું છે? ડેડલોક થવા માટે જરૂરી શરતો સમજાવો.}

\begin{solutionbox}
\textbf{જવાબ}:

\textbf{ડેડલોક} એ એવી પરિસ્થિતિ છે જ્યાં બે અથવા વધુ પ્રક્રિયાઓ કાયમી રૂપે બ્લોક થાય છે.

\textbf{જરૂરી શરતો (કોફમેન કંડિશન્સ):}
\begin{itemize}
    \item \keyword{મ્યુચ્યુઅલ એક્સક્લુઝન}: રિસોર્સ એકસાથે શેર કરી શકાતા નથી
    \item \keyword{હોલ્ડ એન્ડ વેઇટ}: પ્રક્રિયા રિસોર્સ પકડીને અન્યની રાહ જુએ છે
    \item \keyword{નો પ્રીએમ્પ્શન}: રિસોર્સ બળજબરીથી લઈ શકાતા નથી
    \item \keyword{સર્ક્યુલર વેઇટ}: પ્રક્રિયાઓની વર્તુળાકાર સાંકળ
\end{itemize}

\textbf{ઉદાહરણ:}
\begin{center}
\begin{tikzpicture}[node distance=2cm, auto]
    \node [gtu state] (pa) {Process A};
    \node [gtu state, right=3cm of pa] (pb) {Process B};
    \node [gtu block, below=1.5cm of pa, minimum height=1cm] (r2) {Resource 2};
    \node [gtu block, above=1.5cm of pb, minimum height=1cm] (r1) {Resource 1};

    \draw [gtu arrow] (pa) -- node[left] {Holds} (r1);
    \draw [gtu arrow] (r1) -- node[above] {Waits} (pb);
    \draw [gtu arrow] (pb) -- node[right] {Holds} (r2);
    \draw [gtu arrow] (r2) -- node[below] {Waits} (pa);
\end{tikzpicture}
\captionof{figure}{ડેડલોક}
\end{center}
\end{solutionbox}

\begin{mnemonicbox}
\mnemonic{MHNC - Mutual exclusion, Hold-wait, No preemption, Circular wait}
\end{mnemonicbox}

\questionmarks{2(c) OR}{7}{FCFS અલ્ગોરિધમનું વર્ણન કરો અને ગણતરી કરો.}

\begin{solutionbox}
\textbf{જવાબ}:

\textbf{FCFS અલ્ગોરિધમ:}
FCFS એ નોન-પ્રીએમ્પ્ટિવ શેડ્યુલિંગ અલ્ગોરિધમ છે.

\textbf{ગેન્ટ ચાર્ટ:}
\begin{center}
\begin{tikzpicture}[x=0.4cm, y=1cm]
    \draw (0,0) rectangle (7,1) node[midway] {P1};
    \draw (7,0) rectangle (13,1) node[midway] {P2};
    \draw (13,0) rectangle (22,1) node[midway] {P3};
    \draw (22,0) rectangle (26,1) node[midway] {P4};
    
    \node[below] at (0,0) {0};
    \node[below] at (7,0) {7};
    \node[below] at (13,0) {13};
    \node[below] at (22,0) {22};
    \node[below] at (26,0) {26};
\end{tikzpicture}
\captionof{figure}{ગેન્ટ ચાર્ટ (FCFS)}
\end{center}

\textbf{સરેરાશ વેઇટિંગ ટાઇમ} = $7$ ms \\
\textbf{સરેરાશ ટર્નઅરાઉન્ડ ટાઇમ} = $13.5$ ms
\end{solutionbox}

\begin{mnemonicbox}
\mnemonic{SNC - Simple, Non-preemptive, Convoy effect possible}
\end{mnemonicbox}

\questionmarks{3(a)}{3}{સિંગલ-લેવલ ડિરેક્ટરી માળખું સમજાવો.}

\begin{solutionbox}
\textbf{જવાબ}:

સિંગલ-લેવલ ડિરેક્ટરી સ્ટ્રક્ચર સૌથી સરળ છે જ્યાં બધી ફાઇલો એક જ ડિરેક્ટરીમાં હોય છે.

\begin{center}
\begin{tikzpicture}[node distance=1.5cm]
    \node [gtu block, minimum width=8cm] (root) {રૂટ ડિરેક્ટરી};
    
    \node [gtu block, below=1cm of root, xshift=-3cm] (f1) {file1.txt};
    \node [gtu block, right=0.3cm of f1] (f2) {prog.exe};
    \node [gtu block, right=0.3cm of f2] (f3) {data.dat};
    \node [gtu block, right=0.3cm of f3] (f4) {img.jpg};
    
    \draw [gtu arrow] (root) -- (f1);
    \draw [gtu arrow] (root) -- (f2);
    \draw [gtu arrow] (root) -- (f3);
    \draw [gtu arrow] (root) -- (f4);
\end{tikzpicture}
\captionof{figure}{સિંગલ-લેવલ ડિરેક્ટરી}
\end{center}
\end{solutionbox}

\begin{mnemonicbox}
\mnemonic{SUN - Simple, Unique names, No organization}
\end{mnemonicbox}

\questionmarks{3(b)}{4}{વિવિધ ફાઇલ લક્ષણો સમજાવો.}

\begin{solutionbox}
\textbf{જવાબ}:

\begin{center}
\captionof{table}{ફાઇલ એટ્રિબ્યુટ્સ}
\begin{tabulary}{\linewidth}{|L|L|}
\hline
\textbf{એટ્રિબ્યુટ} & \textbf{વર્ણન} \\ \hline
નામ & માનવ-વાંચી શકાય તેવું ફાઇલ ઓળખકર્તા \\ \hline
પ્રકાર & ફાઇલ ફોર્મેટ (એક્ઝિક્યુટેબલ, ટેક્સ્ટ) \\ \hline
કદ & ફાઇલ કદ બાઇટ્સમાં \\ \hline
સ્થાન & સ્ટોરેજ ડિવાઇસ પર સરનામું \\ \hline
પ્રોટેક્શન & એક્સેસ પરમિશન્સ \\ \hline
માલિક & ફાઇલ બનાવનાર યુઝર \\ \hline
\end{tabulary}
\end{center}
\end{solutionbox}

\begin{mnemonicbox}
\mnemonic{NTSLPTO - Name, Type, Size, Location, Protection, Time, Owner}
\end{mnemonicbox}

\questionmarks{3(c)}{7}{કન્ટીગ્યુઅસ ફાળવણી સમજાવો.}

\begin{solutionbox}
\textbf{જવાબ}:

કન્ટીગ્યુઅસ ફાળવણીમાં, દરેક ફાઇલ ડિસ્ક પર સતત બ્લોક્સનો સેટ વ્યાપે છે.

\begin{center}
\begin{tikzpicture}[node distance=0cm, outer sep=0pt]
    \node [rectangle, draw, minimum width=1cm, minimum height=1cm] (b0) {0};
    \node [rectangle, draw, minimum width=1cm, minimum height=1cm, right=of b0] (b1) {1};
    \node [rectangle, draw, minimum width=1cm, minimum height=1cm, right=of b1, fill=blue!10] (b2) {2};
    \node [rectangle, draw, minimum width=1cm, minimum height=1cm, right=of b2, fill=blue!10] (b3) {3};
    \node [rectangle, draw, minimum width=1cm, minimum height=1cm, right=of b3] (b4) {4};
    \node [rectangle, draw, minimum width=1cm, minimum height=1cm, right=of b4, fill=green!10] (b5) {5};
    \node [rectangle, draw, minimum width=1cm, minimum height=1cm, right=of b5, fill=green!10] (b6) {6};
    \node [rectangle, draw, minimum width=1cm, minimum height=1cm, right=of b6, fill=green!10] (b7) {7};
    \node [rectangle, draw, minimum width=1cm, minimum height=1cm, right=of b7] (b8) {8};
    \node [rectangle, draw, minimum width=1cm, minimum height=1cm, right=of b8, fill=red!10] (b9) {9};
    
    \node [below=0.2cm of b2] {ફાઇલ A};
    \node [below=0.2cm of b6] {ફાઇલ B};
    \node [below=0.2cm of b9] {C};
\end{tikzpicture}
\captionof{figure}{કન્ટીગ્યુઅસ ફાળવણી}
\end{center}
\end{solutionbox}

\begin{mnemonicbox}
\mnemonic{FMS vs EFC - Fast access, Minimal seek, Simple vs External fragmentation}
\end{mnemonicbox}

\questionmarks{3(a) OR}{3}{લિનક્સ ફાઇલ સિસ્ટમના વિવિધ પ્રકારો ટૂંકમાં સમજાવો.}

\begin{solutionbox}
\textbf{જવાબ}:

\begin{center}
\captionof{table}{લિનક્સ ફાઇલ સિસ્ટમ}
\begin{tabulary}{\linewidth}{|L|L|}
\hline
\textbf{ફાઇલ સિસ્ટમ} & \textbf{વર્ણન} \\ \hline
ext2 & જર્નલિંગ નથી \\ \hline
ext3 & જર્નલિંગ સાથે \\ \hline
ext4 & સુધારેલ પર્ફોર્મન્સ \\ \hline
XFS & ઉચ્ચ-પર્ફોર્મન્સ \\ \hline
Btrfs & B-ટ્રી ફાઇલસિસ્ટમ \\ \hline
ZFS & કોપી-ઓન-રાઇટ \\ \hline
\end{tabulary}
\end{center}
\end{solutionbox}

\begin{mnemonicbox}
\mnemonic{EEXBZ - ext2/3/4, XFS, Btrfs, ZFS options}
\end{mnemonicbox}

\questionmarks{3(b) OR}{4}{વિવિધ ફાઇલ ઓપરેશન્સ સમજાવો.}

\begin{solutionbox}
\textbf{જવાબ}:

\begin{center}
\captionof{table}{ફાઇલ ઓપરેશન્સ}
\begin{tabulary}{\linewidth}{|L|L|}
\hline
\textbf{ઓપરેશન} & \textbf{વર્ણન} \\ \hline
બનાવો & નવી ફાઇલ બનાવો \\ \hline
ખોલો & ફાઇલ તૈયાર કરો \\ \hline
વાંચો & ડેટા મેળવો \\ \hline
લખો & ડેટા સ્ટોર કરો \\ \hline
સીક & પોઝિશન બદલો \\ \hline
બંધ કરો & રિસોર્સ રિલીઝ કરો \\ \hline
ડિલીટ કરો & ફાઇલ દૂર કરો \\ \hline
\end{tabulary}
\end{center}

\textbf{ફાઇલ ઓપરેશન સિક્વન્સ:}
\begin{center}
\begin{tikzpicture}[node distance=1.5cm, auto]
    \node [gtu state] (create) {ફાઇલ બનાવો};
    \node [gtu state, right=of create] (open) {ફાઇલ ખોલો};
    \node [gtu state, right=of open] (rw) {વાંચો/લખો};
    \node [gtu state, below=of rw] (seek) {સીક};
    \node [gtu state, right=of rw] (close) {ફાઇલ બંધ કરો};
    \node [gtu state, right=of close] (del) {ડિલીટ કરો};

    \path [gtu arrow] (create) -- (open);
    \path [gtu arrow] (open) -- (rw);
    \path [gtu arrow] (rw) -- (close);
    \path [gtu arrow] (close) -- (del);
    \path [gtu arrow] (rw) edge [bend right] (seek);
    \path [gtu arrow] (seek) edge [bend right] (rw);
\end{tikzpicture}
\captionof{figure}{ફાઇલ ઓપરેશન સિક્વન્સ}
\end{center}
\end{solutionbox}

\begin{mnemonicbox}
\mnemonic{CORWSCD - Create, Open, Read, Write, Seek, Close, Delete}
\end{mnemonicbox}

\questionmarks{3(c) OR}{7}{ઇન્ડેક્સ્ડ ફાળવણી સમજાવો.}

\begin{solutionbox}
\textbf{જવાબ}:

ઇન્ડેક્સ્ડ ફાળવણીમાં, દરેક ફાઇલ પાસે ડેટા બ્લોક્સના પોઇન્ટર્સ ધરાવતો ઇન્ડેક્સ બ્લોક હોય છે.

\begin{center}
\begin{tikzpicture}[node distance=1.5cm]
    \node [rectangle, draw, fill=yellow!10, inner sep=0pt] (index) {\begin{tabular}{c}\textbf{ઇન્ડેક્સ બ્લોક}\\ \hline 2\\5\\8\\9\end{tabular}};
    \node [gtu block, right=of index, yshift=1.5cm] (b2) {બ્લોક 2};
    \node [gtu block, right=of index, yshift=0.5cm] (b5) {બ્લોક 5};
    \node [gtu block, right=of index, yshift=-0.5cm] (b8) {બ્લોક 8};
    \node [gtu block, right=of index, yshift=-1.5cm] (b9) {બ્લોક 9};
    
    \draw [->] (index.east) -- (b2.west);
    \draw [->] (index.east) -- (b5.west);
    \draw [->] (index.east) -- (b8.west);
    \draw [->] (index.east) -- (b9.west);
\end{tikzpicture}
\captionof{figure}{ઇન્ડેક્સ્ડ ફાળવણી}
\end{center}
\end{solutionbox}

\begin{mnemonicbox}
\mnemonic{NDF vs IMI - No fragmentation, Dynamic size, Fast access}
\end{mnemonicbox}

\questionmarks{4(a)}{3}{સિસ્ટમ ધમકીઓ વ્યાખ્યાયિત કરો અને તેના પ્રકારો સમજાવો.}

\begin{solutionbox}
\textbf{જવાબ}:

\begin{center}
\captionof{table}{સિસ્ટમ ધમકીઓ}
\begin{tabulary}{\linewidth}{|L|L|}
\hline
\textbf{પ્રકાર} & \textbf{વર્ણન} \\ \hline
વર્મ્સ & નેટવર્ક પર ફેલાતા પ્રોગ્રામ્સ \\ \hline
વાયરસ & અન્ય પ્રોગ્રામ્સ સાથે જોડાતા કોડ \\ \hline
ટ્રોજન હોર્સ & છુપાયેલા દુર્ભાવનાપૂર્ણ કાર્યો \\ \hline
DoS & સિસ્ટમ રિસોર્સને ભરાઈ જવાની હુમલાઓ \\ \hline
પોર્ટ સ્કેનિંગ & નેટવર્ક સેવાઓની તપાસ \\ \hline
\end{tabulary}
\end{center}
\end{solutionbox}

\begin{mnemonicbox}
\mnemonic{WVTDP - Worms, Viruses, Trojans, DoS, Port scanning}
\end{mnemonicbox}

\questionmarks{4(b)}{4}{તફાવત કરો: યુઝર ઓથેન્ટિકેશન v/s યુઝર ઓથોરાઇઝેશન.}

\begin{solutionbox}
\textbf{જવાબ}:

\begin{center}
\captionof{table}{ઓથેન્ટિકેશન vs ઓથોરાઇઝેશન}
\begin{tabulary}{\linewidth}{|L|L|L|}
\hline
\textbf{પાસું} & \textbf{ઓથેન્ટિકેશન} & \textbf{ઓથોરાઇઝેશન} \\ \hline
હેતુ & યુઝરની ઓળખ ચકાસવી & યુઝર પરમિશન્સ નક્કી કરવી \\ \hline
ક્યારે & સિસ્ટમ એક્સેસ પહેલાં & ઓથેન્ટિકેશન પછી \\ \hline
પદ્ધતિઓ & પાસવર્ડ્સ, બાયોમેટ્રિક્સ & એક્સેસ કંટ્રોલ લિસ્ટ્સ \\ \hline
પ્રશ્ન & "તમે કોણ છો?" & "તમે શું કરી શકો?" \\ \hline
\end{tabulary}
\end{center}
\end{solutionbox}

\begin{mnemonicbox}
\mnemonic{WHO vs WHAT - ઓથેન્ટિકેશન પૂછે છે કોણ, ઓથોરાઇઝેશન નક્કી કરે છે શું}
\end{mnemonicbox}

\questionmarks{4(c)}{7}{ઓપરેટિંગ સિસ્ટમ સિક્યોરિટી નીતિઓ અને પ્રક્રિયાઓની ચર્ચા કરો.}

\begin{solutionbox}
\textbf{જવાબ}:

\textbf{સિક્યોરિટી નીતિઓ:}
\begin{itemize}
    \item \keyword{એક્સેસ કંટ્રોલ}: કોણ કયા રિસોર્સને એક્સેસ કરી શકે
    \item \keyword{પાસવર્ડ નીતિ}: પાસવર્ડ બનાવટના મુયમો
    \item \keyword{ઓડિટ નીતિ}: સિસ્ટમ પ્રવૃત્તિઓનું લોગિંગ
\end{itemize}

\textbf{સિસ્ટમ મોનિટરિંગ:}
\begin{center}
\begin{tikzpicture}[node distance=1.5cm, auto]
    \node [gtu state] (log) {લોગ કલેક્શન};
    \node [gtu state, right=of log] (ana) {એનાલિસિસ};
    \node [gtu state, right=of ana] (det) {ધમકી શોધ};
    \node [gtu state, below=of det] (alert) {એલર્ટ};
    \node [gtu state, left=of alert] (resp) {રિસ્પોન્સ};

    \path [gtu arrow] (log) -- (ana);
    \path [gtu arrow] (ana) -- (det);
    \path [gtu arrow] (det) -- (alert);
    \path [gtu arrow] (alert) -- (resp);
\end{tikzpicture}
\captionof{figure}{સિક્યોરિટી મોનિટરિંગ}
\end{center}
\end{solutionbox}

\begin{mnemonicbox}
\mnemonic{AAPUD + UMSIR - Policies + Procedures}
\end{mnemonicbox}

\questionmarks{4(a) OR}{3}{પ્રોગ્રામ ધમકીઓ વ્યાખ્યાયિત કરો અને તેના પ્રકારો સમજાવો.}

\begin{solutionbox}
\textbf{જવાબ}:

\begin{center}
\captionof{table}{પ્રોગ્રામ ધમકીઓ}
\begin{tabulary}{\linewidth}{|L|L|}
\hline
\textbf{પ્રકાર} & \textbf{વર્ણન} \\ \hline
મેલવેર & વાયરસ, વર્મ્સ \\ \hline
સ્પાયવેર & યુઝર પ્રવૃત્તિઓનું મોનિટરિંગ \\ \hline
એડવેર & અનિચ્છિત એડવર્ટાઇઝિંગ \\ \hline
રેન્સમવેર & ડેટા એન્ક્રિપ્ટ કરે છે \\ \hline
રૂટકિટ્સ & દુર્ભાવનાપૂર્ણ પ્રવૃત્તિઓ છુપાવે છે \\ \hline
\end{tabulary}
\end{center}
\end{solutionbox}

\begin{mnemonicbox}
\mnemonic{MSARR - Malware, Spyware, Adware, Ransomware, Rootkits}
\end{mnemonicbox}

\questionmarks{4(b) OR}{4}{પ્રોટેક્શન ડોમેનને યોગ્ય ઉદાહરણ સાથે સમજાવો.}

\begin{solutionbox}
\textbf{જવાબ}:

\textbf{પ્રોટેક્શન ડોમેન} એ ઓબ્જેક્ટ્સ અને એક્સેસ રાઇટ્સનો સેટ છે.

\textbf{ઉદાહરણ:}
\begin{center}
\begin{tikzpicture}[node distance=1cm]
    \node [rectangle, draw, rounded corners, minimum width=4cm, minimum height=2.5cm, align=left] (da) {\textbf{Domain A}\\ Objects:\\ - File1 (R,W)\\ - Printer (W)\\ - Memory (R,W,X)};
    
    \node [rectangle, draw, rounded corners, minimum width=4cm, minimum height=2.5cm, align=left, right=1cm of da] (db) {\textbf{Domain B}\\ Objects:\\ - File2 (R)\\ - Network (R,W)\\ - Database (R)};
\end{tikzpicture}
\captionof{figure}{પ્રોટેક્શન ડોમેન્સ}
\end{center}
\end{solutionbox}

\begin{mnemonicbox}
\mnemonic{OAS - Objects, Access rights, Subjects define domains}
\end{mnemonicbox}

\questionmarks{4(c) OR}{7}{એક્સેસ કંટ્રોલ લિસ્ટ વિગતવાર સમજાવો.}

\begin{solutionbox}
\textbf{જવાબ}:

\textbf{એક્સેસ કંટ્રોલ લિસ્ટ (ACL)} એ સિક્યોરિટી મેકેનિઝમ છે.

\textbf{ACL અમલીકરણ:}
\begin{center}
\begin{tabular}{|l|l|}
\hline
\multicolumn{2}{|c|}{\textbf{File: /home/project/report.txt}} \\ \hline
\textbf{User} & \textbf{Permissions} \\ \hline
alice & read, write \\ \hline
bob & read \\ \hline
admin & read, write, delete \\ \hline
group:dev & read, write \\ \hline
\end{tabular}
\captionof{table}{ACL ઉદાહરણ}
\end{center}
\end{solutionbox}

\begin{mnemonicbox}
\mnemonic{SOA + GDSC}
\end{mnemonicbox}

\questionmarks{5(a)}{3}{નીચેના આદેશો સમજાવો: (i) man (ii) cd (iii) ls}

\begin{solutionbox}
\textbf{જવાબ}:

\begin{center}
\captionof{table}{આદેશો}
\begin{tabulary}{\linewidth}{|L|L|L|}
\hline
\textbf{આદેશ} & \textbf{હેતુ} & \textbf{સિન્ટેક્સ} \\ \hline
man & મેન્યુઅલ પેજ & \code{man [cmd]} \\ \hline
cd & ડિરેક્ટરી બદલો & \code{cd [dir]} \\ \hline
ls & લિસ્ટ કન્ટેન્ટ & \code{ls [opts] [dir]} \\ \hline
\end{tabulary}
\end{center}
\end{solutionbox}

\begin{mnemonicbox}
\mnemonic{MCD - Manual pages, Change directory, Directory listing}
\end{mnemonicbox}

\questionmarks{5(b)}{4}{ત્રણ સંખ્યાઓ વચ્ચે મહત્તમ સંખ્યા શોધવા માટે શેલ સ્ક્રિપ્ટ લખો.}

\begin{solutionbox}
\begin{lstlisting}[language=bash, caption={મહત્તમ સંખ્યા}]
#!/bin/bash
# ત્રણ સંખ્યાઓ વચ્ચે મહત્તમ શોધવા માટે સ્ક્રિપ્ટ

echo "Enter three numbers:"
read -p "First number: " num1
read -p "Second number: " num2  
read -p "Third number: " num3

if [ $num1 -gt $num2 ]; then
    if [ $num1 -gt $num3 ]; then
        max=$num1
    else
        max=$num3
    fi
else
    if [ $num2 -gt $num3 ]; then
        max=$num2
    else
        max=$num3
    fi
fi

echo "Maximum number is: $max"
\end{lstlisting}
\end{solutionbox}

\begin{mnemonicbox}
\mnemonic{ICD - Input, Compare, Display result}
\end{mnemonicbox}

\questionmarks{5(c)}{7}{આપેલ 5 અંકની સંખ્યામાં તમામ અંકોનો સરવાળો શોધવા માટે શેલ સ્ક્રિપ્ટ લખો.}

\begin{solutionbox}
\begin{lstlisting}[language=bash, caption={અંકોનો સરવાળો}]
#!/bin/bash
# 5 અંકની સંખ્યાના અંકોનો સરવાળો શોધવા માટે સ્ક્રિપ્ટ

echo "Enter a 5-digit number:"
read number

# Validate input
if [ ${#number} -ne 5 ] || ! [[ $number =~ ^[0-9]+$ ]]; then
    echo "Error: Please enter exactly 5 digits"
    exit 1
fi

sum=0
temp=$number

# Extract and sum each digit
while [ $temp -gt 0 ]; do
    digit=$((temp % 10))    # Get last digit
    sum=$((sum + digit))    # Add to sum
    temp=$((temp / 10))     # Remove last digit
done

echo "Number: $number"
echo "Sum of digits: $sum"
\end{lstlisting}
\end{solutionbox}

\begin{mnemonicbox}
\mnemonic{VEDS - Validate, Extract, Display, Sum digits}
\end{mnemonicbox}

\questionmarks{5(a) OR}{3}{નીચેના આદેશો સમજાવો: (i) date (ii) top (iii) cmp}

\begin{solutionbox}
\textbf{જવાબ}:

\begin{center}
\captionof{table}{વધુ આદેશો}
\begin{tabulary}{\linewidth}{|L|L|L|}
\hline
\textbf{Cmd} & \textbf{હેતુ} & \textbf{ઉદાહરણ} \\ \hline
date & તારીખ દર્શાવો & \code{date +\%F} \\ \hline
top & પ્રોસેસ વ્યુ & \code{top} \\ \hline
cmp & ફાઇલ સરખામણી & \code{cmp f1 f2} \\ \hline
\end{tabulary}
\end{center}
\end{solutionbox}

\begin{mnemonicbox}
\mnemonic{DTC - Date/time, Task monitor, Compare files}
\end{mnemonicbox}

\questionmarks{5(b) OR}{4}{લિનક્સના ઇન્સ્ટોલેશન સ્ટેપ્સ સમજાવો.}

\begin{solutionbox}
\textbf{જવાબ}:

\textbf{ઇન્સ્ટોલેશન પ્રક્રિયા:}

\begin{center}
\begin{tikzpicture}[node distance=0.8cm, auto]
    \node [gtu state] (iso) {ISO ડાઉનલોડ};
    \node [gtu state, right=of iso] (media) {મીડિયા બનાવો};
    \node [gtu state, right=of media] (boot) {બૂટ};
    \node [gtu state, below=of iso] (part) {પાર્ટિશન};
    \node [gtu state, right=of part] (user) {યુઝર સેટઅપ};
    \node [gtu state, right=of user] (install) {ઇન્સ્ટોલ};
    
    \path [gtu arrow] (iso) -- (media);
    \path [gtu arrow] (media) -- (boot);
    \path [gtu arrow] (boot) -- (part);
    \path [gtu arrow] (part) -- (user);
    \path [gtu arrow] (user) -- (install);
\end{tikzpicture}
\captionof{figure}{ઇન્સ્ટોલેશન ફ્લો}
\end{center}

\textbf{મુખ્ય સ્ટેપ્સ:}
\begin{enumerate}
    \item \keyword{પાર્ટિશનિંગ}: ડિસ્ક સ્પેસ કન્ફિગર કરો
    \item \keyword{કન્ફિગરેશન}: ટાઇમઝોન, કીબોર્ડ, યુઝર સેટ કરો
    \item \keyword{ઇન્સ્ટોલેશન}: ફાઇલો કોપી કરો
\end{enumerate}
\end{solutionbox}

\begin{mnemonicbox}
\mnemonic{DCBCPUPI}
\end{mnemonicbox}

\questionmarks{5(c) OR}{7}{N સંખ્યાઓનો સરવાળો અને સરેરાશ શોધવા માટે શેલ સ્ક્રિપ્ટ લખો.}

\begin{solutionbox}
\begin{lstlisting}[language=bash, caption={સરવાળો અને સરેરાશ}]
#!/bin/bash
# Script to find sum and average of N numbers

echo "How many numbers do you want to enter?"
read n

# Validate input
if ! [[ $n =~ ^[0-9]+$ ]] || [ $n -le 0 ]; then
    echo "Error: Please enter a positive integer"
    exit 1
fi

sum=0
echo "Enter $n numbers:"

# Read N numbers
for ((i=1; i<=n; i++)); do
    echo -n "Enter number $i: "
    read number
    # Simple accumulation
    sum=$(echo "$sum + $number" | bc -l)
done

# Calculate average
average=$(echo "scale=2; $sum / $n" | bc -l)

echo "Sum: $sum"
echo "Average: $average"
\end{lstlisting}
\end{solutionbox}

\begin{mnemonicbox}
\mnemonic{VLAD - Validate, Loop, Arithmetic, Display}
\end{mnemonicbox}

\end{document}
