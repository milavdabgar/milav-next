\documentclass{article}

% content/resources/templates/preamble.tex
\usepackage[margin=0.6in]{geometry}
\author{Milav Dabgar}
\usepackage{amsmath,amssymb,amsthm}
\usepackage{booktabs}
\usepackage{multirow}
\usepackage{xcolor}
\usepackage{tcolorbox}
\tcbuselibrary{breakable,skins}
\usepackage[colorlinks=true,linkcolor=blue]{hyperref}
\usepackage{titlesec}
\usepackage{enumitem}
\usepackage{tikz}
\usepackage{pgfplots}
\usepackage{circuitikz}
\usepackage[version=4]{mhchem}
\usepackage{longtable}
\usepackage{array}
\usepackage{float}
\usepackage{caption}
\usepackage{listings}

\lstset{
  basicstyle=\small\ttfamily,
  breaklines=true,
  breakatwhitespace=false,
  postbreak=\mbox{\textcolor{red}{$\hookrightarrow$}\space},
  float=false,
  numbers=left,
  numberstyle=\tiny\color{gray},
  numbersep=10pt,
  xleftmargin=2em,
  keywordstyle=\color{blue},
  commentstyle=\color{green!60!black},
  stringstyle=\color{purple},
  backgroundcolor=\color{gray!5},
  showstringspaces=false,
  tabsize=2,
  captionpos=b,
  keepspaces=true,
  columns=flexible
}

\pgfplotsset{compat=1.18}
\usetikzlibrary{shapes,arrows,positioning,calc,patterns,decorations.pathmorphing,decorations.markings,arrows.meta}

% Color scheme
\definecolor{headcolor}{RGB}{0,102,204}
\definecolor{keycolor}{RGB}{220,20,60}
\definecolor{solutioncolor}{RGB}{34,139,34}
\definecolor{mnemoniccolor}{RGB}{148,0,211}
\definecolor{codecolor}{RGB}{0,0,100}

% Spacing
\setlength{\parskip}{3pt}
\setlist[itemize]{nosep}
\setlist[enumerate]{nosep}

% Title formatting
\titleformat{\section}{\Large\bfseries\color{headcolor}}{\thesection}{1em}{}
\titleformat{\subsection}{\large\bfseries\color{headcolor}}{\thesubsection}{1em}{}

% Pandoc tightlist compatibility
\providecommand{\tightlist}{%
  \setlength{\itemsep}{0pt}\setlength{\parskip}{0pt}}

% Pandoc longtable compatibility
\newcounter{none}
\def\thenone{}


% content/resources/templates/gujarati-boxes.tex
\usepackage{fontspec}
\usepackage{polyglossia}

% Set Gujarati as main language (document is primarily in Gujarati)
% Note: gloss-gujarati.ldf doesn't exist in polyglossia, but it will use hyphenation patterns
\setdefaultlanguage{gujarati}
\setotherlanguage{english}

% Configure Gujarati font properly
% Use Language=Default to prevent polyglossia from trying to add language-specific features
% that don't exist for Gujarati, which causes "empty feature" warnings
\newfontfamily\gujaratifont[Script=Gujarati,AutoFakeBold=2.5,AutoFakeSlant=0.3]{Noto Sans Gujarati}
\setmainfont[Script=Gujarati,AutoFakeBold=2.5,AutoFakeSlant=0.3]{Noto Sans Gujarati}
% Use Noto Sans Gujarati for monospace to support Gujarati in text
\setmonofont[Scale=0.9]{Noto Sans Gujarati}

% Configure English to use the same font
\newfontfamily\englishfont[Script=Gujarati,AutoFakeBold=2.5,AutoFakeSlant=0.3]{Noto Sans Gujarati}

% Translations for polyglossia
\gappto\captionsgujarati{
  \renewcommand{\tablename}{કોષ્ટક}
  \renewcommand{\figurename}{આકૃતિ}
}

% Helper for TikZ nodes to ensure Gujarati font
\newcommand{\gu}[1]{{\gujaratifont #1}}

% Custom environments
\newtcolorbox{solutionbox}{
    breakable,
    enhanced,
    colback=solutioncolor!5!white,
    colframe=solutioncolor!75!black,
    fonttitle=\bfseries,
    title=જવાબ
}

\newtcolorbox{solutionboxnobreak}{
 colback=solutioncolor!5!white,
 colframe=solutioncolor!75!black,
 fonttitle=\bfseries,
 title=જવાબ
}

\newtcolorbox{keyformula}{
 breakable,
 enhanced,
 colback=keycolor!5!white,
 colframe=keycolor!75!black,
 fonttitle=\bfseries,
 title=રાસાયણિક સમીકરણ/સૂત્ર
}

\newtcolorbox{mnemonicbox}{
 breakable,
 enhanced,
 colback=mnemoniccolor!5!white,
 colframe=mnemoniccolor!75!black,
 fonttitle=\bfseries,
 title=મેમરી ટ્રીક
}


% Custom commands for GTU solutions
% This file defines semantic commands for consistent formatting

% Question command with automatic formatting
\newcommand{\question}[2]{%
  \section*{Question #1}%
  \textbf{#2}%
}

% OR question variant
\newcommand{\questionor}[2]{%
  \section*{Question #1 OR}%
  \textbf{#2}%
}

% Proper table environment with caption
\newenvironment{answertable}[1]{%
  \begin{table}[htbp]
  \centering
  \caption{#1}
}{%
  \end{table}
}

% Proper figure environment for diagrams
\newenvironment{answerdiagram}[1]{%
  \begin{figure}[htbp]
  \centering
  \caption{#1}
}{%
  \end{figure}
}

% Semantic markup for key terms
\newcommand{\keyword}[1]{\textbf{#1}}
\newcommand{\code}[1]{\texttt{#1}}
\newcommand{\classname}[1]{\texttt{#1}}
\newcommand{\methodname}[1]{\texttt{#1}}

% Proper quotation marks
\newcommand{\mnemonic}[1]{``#1''}

\usetikzlibrary{fit}

\title{Linux Operating System (4331602) - Winter 2023 Solution}
\date{January 16, 2024}

\begin{document}
\maketitle

\questionmarks{1(a)}{3}{Linux ના આર્કિટેક્ચર દોરો અને સંક્ષિપ્તમાં વિવિધ સ્તરો સમજાવો.}

\begin{solutionbox}
\textbf{જવાબ}:

\textbf{Linux Architecture:}

\begin{center}
\begin{tikzpicture}[node distance=1.5cm]
    \node [gtu block, minimum width=8cm] (user) {User Applications};
    \node [gtu block, minimum width=8cm, below=0.5cm of user] (shell) {System Call Interface / Shell};
    \node [gtu block, minimum width=8cm, minimum height=2cm, below=0.5cm of shell] (kernel) {Kernel \\ (Process, Memory, File, Network Management)};
    \node [gtu block, minimum width=8cm, below=0.5cm of kernel] (drivers) {Device Drivers};
    \node [gtu block, minimum width=8cm, below=0.5cm of drivers] (hw) {Hardware};

    \draw [gtu arrow] (user) -- (shell);
    \draw [gtu arrow] (shell) -- (kernel);
    \draw [gtu arrow] (kernel) -- (drivers);
    \draw [gtu arrow] (drivers) -- (hw);
\end{tikzpicture}
\captionof{figure}{Linux Architecture}
\end{center}

\textbf{Layers:}
\begin{itemize}
    \item \keyword{User Space}: વપરાશકર્તા applications અને system utilities સમાવે છે.
    \item \keyword{System Call Interface}: user programs અને kernel વચ્ચે interface પ્રદાન કરે છે.
    \item \keyword{Kernel Space}: મૂળ operating system જે process, memory, વગેરેનું management કરે છે.
    \item \keyword{Hardware}: Computer system ના physical components.
\end{itemize}
\end{solutionbox}

\begin{mnemonicbox}
\mnemonic{USKDH - Users System Kernel Drives Hardware}
\end{mnemonicbox}

\questionmarks{1(b)}{4}{રેસની સ્થિતિ શું છે? યોગ્ય ઉદાહરણ સાથે સમજાવો.}

\begin{solutionbox}
\textbf{જવાબ}:

\begin{center}
\captionof{table}{Race Condition}
\begin{tabulary}{\linewidth}{|L|L|}
\hline
\textbf{પાસું} & \textbf{વિવરણ} \\ \hline
વ્યાખ્યા & અનેક processes એકસાથે shared resource ને access કરે છે. \\ \hline
સમસ્યા & timing dependency ને કારણે અનિશ્ચિત પરિણામો. \\ \hline
ઉદાહરણ & બે transactions દ્વારા બેંક account balance ને update કરવું. \\ \hline
\end{tabulary}
\end{center}

\textbf{Example Scenario:}
\begin{enumerate}
    \item \keyword{Process A}: balance = 1000 વાંચે છે, 100 ઉમેરે છે (1100 લખવા તૈયાર).
    \item \keyword{Process B}: balance = 1000 વાંચે છે, 50 બાદ કરે છે (950 લખવા તૈયાર).
    \item \keyword{પરિણામ}: અંતિમ balance 1050 ને બદલે 1100 અથવા 950 હોઈ શકે.
\end{enumerate}
\end{solutionbox}

\begin{mnemonicbox}
\mnemonic{RRRR - Race Results Random Resources}
\end{mnemonicbox}

\questionmarks{1(c)}{7}{વિવિધ પ્રકારની ઓપરેટિંગ સિસ્ટમોની યાદી બનાવો. મલ્ટિપ્રોગ્રામિંગ ઓપરેટિંગ સિસ્ટમના કાર્યને યોગ્ય ઉદાહરણ સાથે સમજાવો.}

\begin{solutionbox}
\textbf{જવાબ}:

\textbf{Types of Operating Systems:}
\begin{center}
\captionof{table}{Types of Operating Systems}
\begin{tabulary}{\linewidth}{|L|L|}
\hline
\textbf{Type} & \textbf{Characteristics} \\ \hline
Batch & Jobs user interaction વિના groups માં process થાય છે. \\ \hline
Time-sharing & અનેક વપરાશકર્તાઓ એકસાથે system share કરે છે. \\ \hline
Real-time & Operations માટે સખત time constraints હોય છે. \\ \hline
Distributed & Networked processors વચ્ચે computations વહેંચાયેલ હોય છે. \\ \hline
Multiprogramming & CPU utilization માટે memory માં અનેક programs રાખવામાં આવે છે. \\ \hline
\end{tabulary}
\end{center}

\textbf{Multiprogramming Working:}
\begin{itemize}
    \item \keyword{Memory Management}: Main memory માં એકસાથે અનેક jobs load થાય છે.
    \item \keyword{CPU Utilization}: જ્યારે એક job I/O માટે રાહ જુએ, ત્યારે CPU બીજી job પર switch કરે છે.
    \item \keyword{Context Switching}: OS હાલની job નું state save કરે છે અને બીજી job load કરે છે.
\end{itemize}

\textbf{ઉદાહરણ}: એક user વેબ browser, music player, અને word processor એકસાથે ચલાવે છે. જ્યારે browser network data ની રાહ જુએ, ત્યારે CPU music player ચલાવે છે.
\end{solutionbox}

\begin{mnemonicbox}
\mnemonic{MPMP - Multiple Programs Maximize Performance}
\end{mnemonicbox}

\questionmarks{1(c) OR}{7}{વિવિધ પ્રકારની ઓપરેટિંગ સિસ્ટમોની યાદી બનાવો. બેચ ઓપરેટિંગ સિસ્ટમ્સ વિગતવાર સમજાવો.}

\begin{solutionbox}
\textbf{જવાબ}:

\textbf{Batch Operating System:}
\begin{itemize}
    \item \keyword{Job Collection}: Jobs (program + data) offline collect થાય છે.
    \item \keyword{Batching}: Operator સમાન jobs ને batches માં group કરે છે.
    \item \keyword{Sequential Execution}: Batch ની દરેક job એક પછી એક execute થાય છે.
    \item \keyword{No Interaction}: Execution દરમ્યાન user interact કરી શકતો નથી.
\end{itemize}
\end{solutionbox}

\begin{mnemonicbox}
\mnemonic{BBBB - Batch Brings Better Business}
\end{mnemonicbox}

\questionmarks{2(a)}{3}{પ્રક્રિયા જીવન ચક્ર દોરો અને સમજાવો.}

\begin{solutionbox}
\textbf{જવાબ}:

\begin{center}
\begin{tikzpicture}[node distance=2cm, auto]
    \node [gtu state] (new) {New};
    \node [gtu state, right=of new] (ready) {Ready};
    \node [gtu state, right=of ready] (running) {Running};
    \node [gtu state, right=of running] (term) {Terminated};
    \node [gtu state, below=of ready] (wait) {Waiting};

    \path [gtu arrow] (new) -- node {Admit} (ready);
    \path [gtu arrow] (ready) -- node {Dispatch} (running);
    \path [gtu arrow] (running) -- node {Exit} (term);
    \path [gtu arrow] (running) edge [bend right] node [above] {Interrupt} (ready);
    \path [gtu arrow] (running) -- node {I/O Wait} (wait);
    \path [gtu arrow] (wait) -- node {I/O Complete} (ready);
\end{tikzpicture}
\captionof{figure}{Process State Diagram}
\end{center}

\textbf{States:}
\begin{itemize}
    \item \keyword{New}: Process બની રહી છે.
    \item \keyword{Ready}: Process processor મળવાની રાહમાં છે.
    \item \keyword{Running}: Instructions execute થઈ રહી છે.
    \item \keyword{Waiting}: Process કોઈ event (I/O) ની રાહમાં છે.
    \item \keyword{Terminated}: Process execution પૂરું કર્યું છે.
\end{itemize}
\end{solutionbox}

\begin{mnemonicbox}
\mnemonic{NRWRT - New Ready Waiting Running Terminated}
\end{mnemonicbox}

\questionmarks{2(b)}{4}{ડેડલોકને વ્યાખ્યાયિત કરો અને ડેડલોક થવા માટે જરૂરી શરતોની ચર્ચા કરો.}

\begin{solutionbox}
\textbf{જવાબ}:

\textbf{Deadlock}: એવી પરિસ્થિતિ જ્યાં processes blocked હોય છે કારણ કે દરેક process resource hold કરે છે અને બીજા resource ની રાહ જુએ છે.

\textbf{શરતો (Coffman Conditions):}
\begin{enumerate}
    \item \keyword{Mutual Exclusion}: Resource share કરી શકાતો નથી.
    \item \keyword{Hold and Wait}: Process resource hold કરી બીજાની રાહ જુએ છે.
    \item \keyword{No Preemption}: Resource જબરદસ્તીથી લઈ શકાતો નથી.
    \item \keyword{Circular Wait}: Processes circular chain માં એકબીજાની રાહ જુએ છે.
\end{enumerate}
\end{solutionbox}

\begin{mnemonicbox}
\mnemonic{MHNC - My Hold Never Circles}
\end{mnemonicbox}

\questionmarks{2(c)}{7}{રાઉન્ડ રોબિન અલ્ગોરિધમનું વર્ણન કરો. આપેલ ડેટા માટે ગેન્ટ ચાર્ટ સાથે સરેરાશ રાહ જોવાનો સમય અને સરેરાશ ટર્ન-અરાઉન્ડ સમયની ગણતરી કરો. સંદર્ભ સ્વિચ = 01 ms અને ક્વોન્ટમ સમય = 05 ms ધ્યાનમાં લો.}

\begin{solutionbox}
\textbf{જવાબ}:

\textbf{Round Robin}: Preemptive scheduling જેમાં દરેક process ને fixed time slice (quantum) મળે છે.

\textbf{Given Data:} Context Switch = 1ms, Quantum = 5ms.

\textbf{Gantt Chart:}
\begin{center}
\begin{tikzpicture}[x=0.3cm, y=1cm]
    \draw (0,0) rectangle (5,1) node[midway] {P1};
    \draw (5,0) rectangle (6,1) node[midway, font=\tiny] {CS};
    \draw (6,0) rectangle (10,1) node[midway] {P2};
    \draw (10,0) rectangle (11,1) node[midway, font=\tiny] {CS};
    \draw (11,0) rectangle (16,1) node[midway] {P3};
    \draw (16,0) rectangle (17,1) node[midway, font=\tiny] {CS};
    \draw (17,0) rectangle (22,1) node[midway] {P4};
    \draw (22,0) rectangle (23,1) node[midway, font=\tiny] {CS};
    \draw (23,0) rectangle (28,1) node[midway] {P1};
    \draw (28,0) rectangle (29,1) node[midway, font=\tiny] {CS};
    \draw (29,0) rectangle (34,1) node[midway] {P3};
    \draw (34,0) rectangle (35,1) node[midway, font=\tiny] {CS};
    \draw (35,0) rectangle (37,1) node[midway] {P1}; \node[below] at (37,0) {37};
    \draw (37,0) rectangle (38,1) node[midway, font=\tiny] {CS};
    \draw (38,0) rectangle (43,1) node[midway] {P3}; \node[below] at (43,0) {43};
\end{tikzpicture}
\captionof{figure}{Gantt Chart (RR)}
\end{center}

\textbf{ગણતરી:}
\begin{itemize}
    \item \textbf{Average Waiting Time} = 16.5 ms
    \item \textbf{Average Turnaround Time} = 25.5 ms
\end{itemize}
\end{solutionbox}

\begin{mnemonicbox}
\mnemonic{RRRR - Round Robin Rotates Regularly}
\end{mnemonicbox}

\questionmarks{2(a) OR}{3}{તફાવત: CPU બાઉન્ડ પ્રક્રિયા v/s I/O બાઉન્ડ પ્રક્રિયા.}

\begin{solutionbox}
\textbf{જવાબ}:

\begin{center}
\captionof{table}{CPU vs I/O Bound}
\begin{tabulary}{\linewidth}{|L|L|L|}
\hline
\textbf{પાસું} & \textbf{CPU Bound} & \textbf{I/O Bound} \\ \hline
Activity & High CPU computations & Frequent I/O operations \\ \hline
Burst Time & Long CPU bursts & Short CPU bursts \\ \hline
Wait States & ઓછી વાર & I/O માટે વારંવાર રાહ જુએ છે \\ \hline
Examples & Scientific calculation & File copy, Data processing \\ \hline
\end{tabulary}
\end{center}
\end{solutionbox}

\begin{mnemonicbox}
\mnemonic{CIC - CPU Computes I/O Interacts}
\end{mnemonicbox}

\questionmarks{2(b) OR}{4}{ક્રિટિકલ સેક્શનને વ્યાખ્યાયિત કરો અને ક્રિટિકલ સેક્શન સોલ્યુશનની સામાન્ય રચનાની ચર્ચા કરો.}

\begin{solutionbox}
\textbf{જવાબ}:

\textbf{Critical Section (CS)}: Code segment જ્યાં shared resources access થાય છે.

\textbf{General Structure:}
\begin{lstlisting}[basicstyle=\ttfamily]
do {
    entry section   // Request permission
       critical section
    exit section    // Release permission
       remainder section
} while (true);
\end{lstlisting}

\textbf{જરૂરિયાતો:}
\begin{itemize}
    \item \keyword{Mutual Exclusion}: CS માં ફક્ત એક જ process.
    \item \keyword{Progress}: જો CS ખાલી હોય, તો next process selection અટકવી ન જોઈએ.
    \item \keyword{Bounded Waiting}: Entry માટે રાહ જોવાનો સમય મર્યાદિત હોવો જોઈએ.
\end{itemize}
\end{solutionbox}

\begin{mnemonicbox}
\mnemonic{ECER - Entry Critical Exit Remainder}
\end{mnemonicbox}

\questionmarks{2(c) OR}{7}{SJF અલ્ગોરિધમનું વર્ણન કરો. કોષ્ટકમાં આપેલ ડેટા માટે ગેન્ટ ચાર્ટ સાથે સરેરાશ રાહ જોવાનો સમય અને સરેરાશ ટર્ન-અરાઉન્ડ સમયની ગણતરી કરો.}

\begin{solutionbox}
\textbf{જવાબ}:

\textbf{Shortest Job First (SJF)}: Non-preemptive algorithm જ્યાં સૌથી નાના burst time વાળી process પહેલા schedule થાય છે.

\textbf{Gantt Chart:}
\begin{center}
\begin{tikzpicture}[x=0.4cm, y=1cm]
    \draw (0,0) rectangle (8,1) node[midway] {P1};
    \draw (8,0) rectangle (12,1) node[midway] {P2};
    \draw (12,0) rectangle (17,1) node[midway] {P4};
    \draw (17,0) rectangle (26,1) node[midway] {P3};
    
    \node[below] at (0,0) {0};
    \node[below] at (8,0) {8};
    \node[below] at (12,0) {12};
    \node[below] at (17,0) {17};
    \node[below] at (26,0) {26};
\end{tikzpicture}
\captionof{figure}{Gantt Chart (SJF)}
\end{center}

\textbf{ગણતરી:}
\begin{itemize}
    \item \textbf{Avg Wait} = 5.75 ms
    \item \textbf{Avg TAT} = 12.25 ms
\end{itemize}
\end{solutionbox}

\begin{mnemonicbox}
\mnemonic{SJSS - Shortest Jobs Start Soon}
\end{mnemonicbox}

\questionmarks{3(a)}{3}{બે-સ્તરની ડિરેક્ટરી રચના સમજાવો.}

\begin{solutionbox}
\textbf{જવાબ}:

\begin{center}
\begin{tikzpicture}[node distance=1.5cm]
    \node [gtu block, minimum width=4cm] (mfd) {Master File Directory};
    \node [gtu block, below left=1cm of mfd] (u1) {User 1 Dir};
    \node [gtu block, below right=1cm of mfd] (u2) {User 2 Dir};
    
    \node [gtu block, below=0.5cm of u1, minimum width=1.5cm] (f1) {File X};
    \node [gtu block, below=0.5cm of u2, minimum width=1.5cm] (f2) {File X};
    
    \draw [gtu arrow] (mfd) -- (u1);
    \draw [gtu arrow] (mfd) -- (u2);
    \draw [gtu arrow] (u1) -- (f1);
    \draw [gtu arrow] (u2) -- (f2);
\end{tikzpicture}
\captionof{figure}{Two-level Directory}
\end{center}

\textbf{વિશેષતા:}
\begin{itemize}
    \item દરેક user માટે અલગ directory (UFD).
    \item Name collision problem ઉકેલે છે.
    \item Users વચ્ચે isolation પૂરું પાડે છે.
\end{itemize}
\end{solutionbox}

\begin{mnemonicbox}
\mnemonic{TTTT - Two Tiers Tackle Troubles}
\end{mnemonicbox}

\questionmarks{3(b)}{4}{વિવિધ ફાઇલ કામગીરી સમજાવો.}

\begin{solutionbox}
\textbf{જવાબ}:

\begin{center}
\captionof{table}{File Operations}
\begin{tabulary}{\linewidth}{|L|L|}
\hline
\textbf{Operation} & \textbf{Description} \\ \hline
Create & Space allocate કરે અને directory entry બનાવે. \\ \hline
Open & Metadata memory માં load કરે access માટે. \\ \hline
Read & Current position થી data વાંચે. \\ \hline
Write & Current position પર data લખે. \\ \hline
Delete & Space release કરે અને entry remove કરે. \\ \hline
Close & Internal resources free કરે. \\ \hline
\end{tabulary}
\end{center}
\end{solutionbox}

\begin{mnemonicbox}
\mnemonic{CORWCD - Create Open Read Write Close Delete}
\end{mnemonicbox}

\questionmarks{3(c)}{7}{વિવિધ ફાઈલ ફાળવણી પદ્ધતિઓની યાદી બનાવો અને જરૂરી રેખાકૃતિ સાથે સંલગ્ન ફાળવણી સમજાવો.}

\begin{solutionbox}
\textbf{જવાબ}:

\textbf{Methods}: Contiguous, Linked, Indexed.

\textbf{Contiguous Allocation:}
File disk પર સતત (consecutive) blocks રોકે છે.

\begin{center}
\begin{tikzpicture}[node distance=0cm]
    \foreach \x/\label in {0/0, 1/1, 2/2, 3/3, 4/4, 5/5}
        \node [rectangle, draw, minimum size=1cm] (b\x) at (\x*1.1, 0) {\label};
        
    \node [fit=(b1)(b2)(b3), fill=blue!20, opacity=0.5] {};
    \node [below=0.2cm of b2] {File A (Start:1, Len:3)};
\end{tikzpicture}
\captionof{figure}{Contiguous Allocation}
\end{center}

\textbf{ફાયદા}: સરળ (start, length), ઝડપી access.
\textbf{ગેરફાયદા}: External fragmentation, file expand કરવામાં મુશ્કેલી.
\end{solutionbox}

\begin{mnemonicbox}
\mnemonic{CCCC - Contiguous Creates Continuous Clusters}
\end{mnemonicbox}

\questionmarks{3(a) OR}{3}{ફાઇલ સ્ટ્રક્ચરના પ્રકારોનું વર્ણન કરો.}

\begin{solutionbox}
\textbf{જવાબ}:

\begin{itemize}
    \item \keyword{Sequential}: Records ક્રમમાં store થાય છે. સરળ પણ search ધીમું.
    \item \keyword{Direct/Random}: Key દ્વારા direct access. ઝડપી access.
    \item \keyword{Indexed}: અલગ index file data records ને point કરે છે.
\end{itemize}
\end{solutionbox}

\begin{mnemonicbox}
\mnemonic{SDI - Sequential Direct Indexed}
\end{mnemonicbox}

\questionmarks{3(b) OR}{4}{વિવિધ ફાઇલ લક્ષણો સમજાવો.}

\begin{solutionbox}
\textbf{જવાબ}:

\begin{center}
\captionof{table}{File Attributes}
\begin{tabulary}{\linewidth}{|L|L|}
\hline
\textbf{Attribute} & \textbf{Description} \\ \hline
Name & Human-readable identifier \\ \hline
Type & File format (.txt, .exe) \\ \hline
Size & File size \\ \hline
Location & Device પર file location નો pointer \\ \hline
Protection & Access control info (R/W/X) \\ \hline
Time/Date & Creation, modification info \\ \hline
\end{tabulary}
\end{center}
\end{solutionbox}

\begin{mnemonicbox}
\mnemonic{NTSLPT - Name Type Size Location Permissions Time}
\end{mnemonicbox}

\questionmarks{3(c) OR}{7}{વિવિધ ફાઈલ ફાળવણી પદ્ધતિઓની યાદી બનાવો અને જરૂરી રેખાકૃતિ સાથે લિંક કરેલ ફાળવણી સમજાવો.}

\begin{solutionbox}
\textbf{જવાબ}:

\textbf{Linked Allocation:}
Files non-contiguous blocks માં store થાય છે. દરેક block next block નું pointer ધરાવે છે.

\begin{center}
\begin{tikzpicture}[node distance=1.5cm, auto]
    \node [gtu block] (b1) {Block 5 \\ Next: 8};
    \node [gtu block, right=of b1] (b2) {Block 8 \\ Next: 2};
    \node [gtu block, right=of b2] (b3) {Block 2 \\ Next: -1};
    
    \draw [gtu arrow] (b1) -- (b2);
    \draw [gtu arrow] (b2) -- (b3);
\end{tikzpicture}
\captionof{figure}{Linked Allocation}
\end{center}

\textbf{ફાયદા}: External fragmentation નથી, file સરળતાથી વધારી શકાય.
\textbf{ગેરફાયદા}: Random access ધીમું છે, pointer overhead.
\end{solutionbox}

\begin{mnemonicbox}
\mnemonic{LLLL - Links Lead Logical Locations}
\end{mnemonicbox}

\questionmarks{4(a)}{3}{પ્રોગ્રામ ધમકીઓ વ્યાખ્યાયિત કરો અને તેના પ્રકારો સમજાવો.}

\begin{solutionbox}
\textbf{જવાબ}:

\textbf{Program Threats}: Malicious code જે program માં embed હોય છે.

\begin{itemize}
    \item \keyword{Trojan Horse}: ઉપયોગી દેખાય છે પણ નુકસાન કરે છે.
    \item \keyword{Trap Door}: Designer દ્વારા છોડવામાં આવેલ secret entry point.
    \item \keyword{Logic Bomb}: Code જે ચોક્કસ શરતો હેઠળ execute થાય (explode) છે.
    \item \keyword{Virus}: Code જે અન્ય programs માં પોતાને embed કરે છે.
\end{itemize}
\end{solutionbox}

\begin{mnemonicbox}
\mnemonic{TTLV - Trojan Trap Logic Virus}
\end{mnemonicbox}

\questionmarks{4(b)}{4}{સિસ્ટમ ઓથેન્ટિકેશન સમજાવો.}

\begin{solutionbox}
\textbf{જવાબ}:

\textbf{Authentication}: User identity ની verification.

\textbf{પદ્ધતિઓ:}
\begin{enumerate}
    \item \keyword{Passwords}: Secret string.
    \item \keyword{Biometrics}: Fingerprint, retina scan.
    \item \keyword{Smart Cards}: Chip વાળા physical token.
    \item \keyword{Two-Factor}: બે પદ્ધતિઓનું combination (e.g., Password + OTP).
\end{enumerate}
\end{solutionbox}

\begin{mnemonicbox}
\mnemonic{PBST - Passwords Biometrics Smartcards Two-factor}
\end{mnemonicbox}

\questionmarks{4(c)}{7}{એક્સેસ કંટ્રોલ લિસ્ટને વિગતવાર સમજાવો.}

\begin{solutionbox}
\textbf{જવાબ}:

\textbf{Access Control List (ACL)}: દરેક object સાથે જોડાયેલ list જે define કરે છે કે કયા users તેને access કરી શકે.

\textbf{Structure:}
File X: $(User A, Read), (User B, Read/Write)$

\textbf{ફાયદા}:
\begin{itemize}
    \item Individual objects પર ચોક્કસ control.
    \item Specific users માટે permission revoke કરવી સરળ.
\end{itemize}

\textbf{ગેરફાયદા}:
\begin{itemize}
    \item ACLs search કરવું ધીમું હોઈ શકે.
    \item ઘણી files માટે ACLs manage કરવું જટિલ છે.
\end{itemize}
\end{solutionbox}

\begin{mnemonicbox}
\mnemonic{ACLU - Access Controls Limit Users}
\end{mnemonicbox}

\questionmarks{4(a) OR}{3}{સિસ્ટમ ધમકીઓ વ્યાખ્યાયિત કરો અને તેના પ્રકારો સમજાવો.}

\begin{solutionbox}
\textbf{જવાબ}:

\textbf{System Threats}: OS અથવા environment ને target કરે છે.
\begin{itemize}
    \item \keyword{Worm}: Independent program જે network પર ફેલાય છે અને resources વાપરે છે.
    \item \keyword{Port Scanning}: Vulnerabilities શોધવા open ports detect કરવા.
    \item \keyword{Denial of Service}: System ને overwhelm કરી legitimate use અટકાવવું.
\end{itemize}
\end{solutionbox}

\begin{mnemonicbox}
\mnemonic{WPD - Worm Port DoS}
\end{mnemonicbox}

\questionmarks{4(c) OR}{7}{વિવિધ ઓપરેટિંગ સિસ્ટમ સુરક્ષા નીતિઓ અને પ્રક્રિયાઓની ચર્ચા કરો.}

\begin{solutionbox}
\textbf{જવાબ}:

\textbf{Policies:}
\begin{itemize}
    \item \keyword{User Policy}: Strong passwords.
    \item \keyword{Access Policy}: Least privilege principle.
    \item \keyword{Data Policy}: Sensitive data નું encryption.
\end{itemize}

\textbf{Procedures (પ્રક્રિયાઓ):}
\begin{itemize}
    \item \keyword{Auditing}: Logs monitor કરવા.
    \item \keyword{Backups}: Regular data backup.
    \item \keyword{Updates}: OS patching.
\end{itemize}
\end{solutionbox}

\begin{mnemonicbox}
\mnemonic{APPI - Access Password Policy Incident}
\end{mnemonicbox}

\questionmarks{5(a)}{3}{નીચેના આદેશો સમજાવો: (i) pwd (ii) cd (iii) comm}

\begin{solutionbox}
\textbf{જવાબ}:

\begin{center}
\captionof{table}{Commands}
\begin{tabulary}{\linewidth}{|L|L|}
\hline
\textbf{Command} & \textbf{Purpose} \\ \hline
\code{pwd} & Print Working Directory. \\ \hline
\code{cd} & Change Directory. Folder બદલવા માટે. \\ \hline
\code{comm} & બે sorted files ને line-by-line compare કરે. \\ \hline
\end{tabulary}
\end{center}
\end{solutionbox}

\begin{mnemonicbox}
\mnemonic{PCC - Pwd Cd Comm}
\end{mnemonicbox}

\questionmarks{5(b)}{4}{ત્રીજી ફાઇલમાં બે ફાઇલોના સમાવિષ્ટોને જોડવા માટે શેલ સ્ક્રિપ્ટ લખો.}

\begin{solutionbox}
\begin{lstlisting}[language=bash, caption={Concatenate Files}]
#!/bin/bash
# Script to concatenate two files

echo "Enter first filename:"
read f1
echo "Enter second filename:"
read f2
echo "Enter output filename:"
read f3

if [ -f "$f1" ] && [ -f "$f2" ]; then
    cat "$f1" "$f2" > "$f3"
    echo "Files merged into $f3"
else
    echo "Files not found"
fi
\end{lstlisting}
\end{solutionbox}

\begin{mnemonicbox}
\mnemonic{CCCC - Cat Combines Content Correctly}
\end{mnemonicbox}

\questionmarks{5(c)}{7}{આપેલ 5 અંકની સંખ્યામાં તમામ વ્યક્તિગત અંકોનો સરવાળો શોધવા માટે શેલ સ્ક્રિપ્ટ લખો.}

\begin{solutionbox}
\begin{lstlisting}[language=bash, caption={Sum of Digits}]
#!/bin/bash
# Sum of 5 digits

echo "Enter 5 digit number:"
read n

if [ ${#n} -ne 5 ]; then
    echo "Please enter 5 digits"
    exit 1
fi

sum=0
while [ $n -gt 0 ]
do
    rem=$((n % 10))
    sum=$((sum + rem))
    n=$((n / 10))
done

echo "Sum of digits: $sum"
\end{lstlisting}
\end{solutionbox}

\begin{mnemonicbox}
\mnemonic{SSSS - Sum Separates Single Symbols}
\end{mnemonicbox}

\questionmarks{5(a) OR}{3}{નીચેના આદેશો સમજાવો: (i) man (ii) mkdir (iii) grep}

\begin{solutionbox}
\textbf{જવાબ}:

\begin{center}
\captionof{table}{More Commands}
\begin{tabulary}{\linewidth}{|L|L|}
\hline
\textbf{Cmd} & \textbf{Purpose} \\ \hline
\code{man} & Manual. Help display કરે છે. \\ \hline
\code{mkdir} & Make Directory. નવું folder બનાવે છે. \\ \hline
\code{grep} & Global Regular Expression Print. Files માં text search કરે છે. \\ \hline
\end{tabulary}
\end{center}
\end{solutionbox}

\begin{mnemonicbox}
\mnemonic{MMG - Manual Make Grep}
\end{mnemonicbox}

\questionmarks{5(b) OR}{4}{ફિબોનાચી શ્રેણી જનરેટ કરવા અને પ્રદર્શિત કરવા માટે શેલ સ્ક્રિપ્ટ લખો.}

\begin{solutionbox}
\begin{lstlisting}[language=bash, caption={Fibonacci Series}]
#!/bin/bash
# Fibonacci Series

echo "Enter N:"
read n
a=0
b=1

echo -n "$a $b "

for (( i=0; i<n-2; i++ ))
do
    c=$((a + b))
    echo -n "$c "
    a=$b
    b=$c
done
echo ""
\end{lstlisting}
\end{solutionbox}

\begin{mnemonicbox}
\mnemonic{FFFF - Fibonacci Follows Forward Formula}
\end{mnemonicbox}

\questionmarks{5(c) OR}{7}{આપેલ string palindrome છે કે કેમ તે નિર્ધારિત કરવા માટે શેલ સ્ક્રિપ્ટ લખો.}

\begin{solutionbox}
\begin{lstlisting}[language=bash, caption={Palindrome Check}]
#!/bin/bash
# Palindrome Check

echo "Enter string:"
read str
len=${#str}
rev=""

for (( i=$len-1; i>=0; i-- ))
do
    rev="$rev${str:$i:1}"
done

if [ "$str" == "$rev" ]; then
    echo "Palindrome"
else
    echo "Not Palindrome"
fi
\end{lstlisting}
\end{solutionbox}

\begin{mnemonicbox}
\mnemonic{PPPP - Palindromes Proceed Perfectly Parallel}
\end{mnemonicbox}

\end{document}
