\documentclass{article}

% content/resources/templates/preamble.tex
\usepackage[margin=0.6in]{geometry}
\author{Milav Dabgar}
\usepackage{amsmath,amssymb,amsthm}
\usepackage{booktabs}
\usepackage{multirow}
\usepackage{xcolor}
\usepackage{tcolorbox}
\tcbuselibrary{breakable,skins}
\usepackage[colorlinks=true,linkcolor=blue]{hyperref}
\usepackage{titlesec}
\usepackage{enumitem}
\usepackage{tikz}
\usepackage{pgfplots}
\usepackage{circuitikz}
\usepackage[version=4]{mhchem}
\usepackage{longtable}
\usepackage{array}
\usepackage{float}
\usepackage{caption}
\usepackage{listings}

\lstset{
  basicstyle=\small\ttfamily,
  breaklines=true,
  breakatwhitespace=false,
  postbreak=\mbox{\textcolor{red}{$\hookrightarrow$}\space},
  float=false,
  numbers=left,
  numberstyle=\tiny\color{gray},
  numbersep=10pt,
  xleftmargin=2em,
  keywordstyle=\color{blue},
  commentstyle=\color{green!60!black},
  stringstyle=\color{purple},
  backgroundcolor=\color{gray!5},
  showstringspaces=false,
  tabsize=2,
  captionpos=b,
  keepspaces=true,
  columns=flexible
}

\pgfplotsset{compat=1.18}
\usetikzlibrary{shapes,arrows,positioning,calc,patterns,decorations.pathmorphing,decorations.markings,arrows.meta}

% Color scheme
\definecolor{headcolor}{RGB}{0,102,204}
\definecolor{keycolor}{RGB}{220,20,60}
\definecolor{solutioncolor}{RGB}{34,139,34}
\definecolor{mnemoniccolor}{RGB}{148,0,211}
\definecolor{codecolor}{RGB}{0,0,100}

% Spacing
\setlength{\parskip}{3pt}
\setlist[itemize]{nosep}
\setlist[enumerate]{nosep}

% Title formatting
\titleformat{\section}{\Large\bfseries\color{headcolor}}{\thesection}{1em}{}
\titleformat{\subsection}{\large\bfseries\color{headcolor}}{\thesubsection}{1em}{}

% Pandoc tightlist compatibility
\providecommand{\tightlist}{%
  \setlength{\itemsep}{0pt}\setlength{\parskip}{0pt}}

% Pandoc longtable compatibility
\newcounter{none}
\def\thenone{}


% content/resources/templates/gujarati-boxes.tex
\usepackage{fontspec}
\usepackage{polyglossia}

% Set Gujarati as main language (document is primarily in Gujarati)
% Note: gloss-gujarati.ldf doesn't exist in polyglossia, but it will use hyphenation patterns
\setdefaultlanguage{gujarati}
\setotherlanguage{english}

% Configure Gujarati font properly
% Use Language=Default to prevent polyglossia from trying to add language-specific features
% that don't exist for Gujarati, which causes "empty feature" warnings
\newfontfamily\gujaratifont[Script=Gujarati,AutoFakeBold=2.5,AutoFakeSlant=0.3]{Noto Sans Gujarati}
\setmainfont[Script=Gujarati,AutoFakeBold=2.5,AutoFakeSlant=0.3]{Noto Sans Gujarati}
% Use Noto Sans Gujarati for monospace to support Gujarati in text
\setmonofont[Scale=0.9]{Noto Sans Gujarati}

% Configure English to use the same font
\newfontfamily\englishfont[Script=Gujarati,AutoFakeBold=2.5,AutoFakeSlant=0.3]{Noto Sans Gujarati}

% Translations for polyglossia
\gappto\captionsgujarati{
  \renewcommand{\tablename}{કોષ્ટક}
  \renewcommand{\figurename}{આકૃતિ}
}

% Helper for TikZ nodes to ensure Gujarati font
\newcommand{\gu}[1]{{\gujaratifont #1}}

% Custom environments
\newtcolorbox{solutionbox}{
    breakable,
    enhanced,
    colback=solutioncolor!5!white,
    colframe=solutioncolor!75!black,
    fonttitle=\bfseries,
    title=જવાબ
}

\newtcolorbox{solutionboxnobreak}{
 colback=solutioncolor!5!white,
 colframe=solutioncolor!75!black,
 fonttitle=\bfseries,
 title=જવાબ
}

\newtcolorbox{keyformula}{
 breakable,
 enhanced,
 colback=keycolor!5!white,
 colframe=keycolor!75!black,
 fonttitle=\bfseries,
 title=રાસાયણિક સમીકરણ/સૂત્ર
}

\newtcolorbox{mnemonicbox}{
 breakable,
 enhanced,
 colback=mnemoniccolor!5!white,
 colframe=mnemoniccolor!75!black,
 fonttitle=\bfseries,
 title=મેમરી ટ્રીક
}


% Custom commands for GTU solutions
% This file defines semantic commands for consistent formatting

% Question command with automatic formatting
\newcommand{\question}[2]{%
  \section*{Question #1}%
  \textbf{#2}%
}

% OR question variant
\newcommand{\questionor}[2]{%
  \section*{Question #1 OR}%
  \textbf{#2}%
}

% Proper table environment with caption
\newenvironment{answertable}[1]{%
  \begin{table}[htbp]
  \centering
  \caption{#1}
}{%
  \end{table}
}

% Proper figure environment for diagrams
\newenvironment{answerdiagram}[1]{%
  \begin{figure}[htbp]
  \centering
  \caption{#1}
}{%
  \end{figure}
}

% Semantic markup for key terms
\newcommand{\keyword}[1]{\textbf{#1}}
\newcommand{\code}[1]{\texttt{#1}}
\newcommand{\classname}[1]{\texttt{#1}}
\newcommand{\methodname}[1]{\texttt{#1}}

% Proper quotation marks
\newcommand{\mnemonic}[1]{``#1''}


\title{Linux Operating System (4331602) - Summer 2024 Solution (Gujarati)}
\date{June 10, 2024}

\begin{document}
\maketitle

\questionmarks{1(a)}{3}{ઓપરેટિંગ સિસ્ટમની વ્યાખ્યા આપો અને તેના હેતુઓ જણાવો.}

\begin{solutionbox}
\textbf{ઓપરેટિંગ સિસ્ટમ વ્યાખ્યા}: એક પ્રોગ્રામ જે કમ્પ્યુટર હાર્ડવેર અને વપરાશકર્તાઓ વચ્ચે ઇન્ટરફેસ તરીકે કાર્ય કરે છે, સિસ્ટમ રિસોર્સિસનું સંચાલન કરે છે અને પ્રોગ્રામ એક્ઝિક્યુશનને નિયંત્રિત કરે છે.

\textbf{ઓપરેટિંગ સિસ્ટમના હેતુઓ}:

\begin{center}
\captionof{table}{OS હેતુઓ}
\begin{tabulary}{\linewidth}{|L|L|}
\hline
\textbf{હેતુ} & \textbf{વર્ણન} \\ \hline
\keyword{Ressource Management} & CPU, મેમરી, I/O ડિવાઇસીસનું કાર્યક્ષમ સંચાલન \\ \hline
\keyword{User Convenience} & ઉપયોગમાં સરળ ઇન્ટરફેસ પ્રદાન કરવું \\ \hline
\keyword{System Protection} & અનધિકૃત એક્સેસથી સિસ્ટમને સુરક્ષિત કરવી \\ \hline
\end{tabulary}
\end{center}
\end{solutionbox}

\begin{mnemonicbox}
\mnemonic{RUS: Resource management, User convenience, System protection}
\end{mnemonicbox}

\questionmarks{1(b)}{4}{કમ્પ્યુટર સિસ્ટમના ઘટકો જણાવો અને ઓપરેટિંગ સિસ્ટમની જરૂરિયાત સમજાવો.}

\begin{solutionbox}
\textbf{કમ્પ્યુટર સિસ્ટમના ઘટકો}:

\begin{center}
\begin{tikzpicture}[node distance=1.5cm, auto]
    \node [gtu block] (users) {Users};
    \node [gtu block, below=of users] (apps) {Application Programs};
    \node [gtu block, below=of apps] (os) {Operating System};
    \node [gtu block, below=of os] (hardware) {Hardware (CPU, Memory, I/O)};
    
    \path [gtu arrow] (users) -- (apps);
    \path [gtu arrow] (apps) -- (os);
    \path [gtu arrow] (os) -- (hardware);
    \path [gtu arrow] (hardware) -- (os);
    \path [gtu arrow] (os) -- (apps);
    \path [gtu arrow] (apps) -- (users);
\end{tikzpicture}
\captionof{figure}{કમ્પ્યુટર સિસ્ટમ હાયરાર્કી}
\end{center}

\textbf{ઓપરેટિંગ સિસ્ટમની જરૂરિયાત}:

\begin{itemize}
    \item \keyword{Resource Manager}: હાર્ડવેર એલોકેશનને કંટ્રોલ કરે છે
    \item \keyword{Interface Provider}: યુઝર અને હાર્ડવેર વચ્ચે સરળ કોમ્યુનિકેશન
    \item \keyword{Security}: જોખમોથી સિસ્ટમનું રક્ષણ
    \item \keyword{Error Handling}: સિસ્ટમ એરર્સનું કાર્યક્ષમ સંચાલન
\end{itemize}
\end{solutionbox}

\begin{mnemonicbox}
\mnemonic{RISE: Resource management, Interface, Security, Error handling}
\end{mnemonicbox}

\questionmarks{1(c)}{7}{નીચે આપેલ ઓપરેટિંગ સિસ્ટમના પ્રકારો સમજાવો.}

\begin{solutionbox}
\textbf{I. Batch Operating System}

\begin{center}
\captionof{table}{Batch OS}
\begin{tabulary}{\linewidth}{|L|L|}
\hline
\textbf{ફીચર} & \textbf{વર્ણન} \\ \hline
\keyword{Processing} & જોબ્સને બેચમાં પ્રોસેસ કરવામાં આવે છે \\ \hline
\keyword{Efficiency} & હાઈ થ્રુપુટ, ઓછું યુઝર ઇન્ટરેક્શન \\ \hline
\keyword{Example} & IBM મેઇનફ્રેમ્સ \\ \hline
\end{tabulary}
\end{center}

\textbf{II. Multiprogramming Operating System}

\begin{center}
\captionof{table}{Multiprogramming OS}
\begin{tabulary}{\linewidth}{|L|L|}
\hline
\textbf{ફીચર} & \textbf{વર્ણન} \\ \hline
\keyword{Concept} & મેમરીમાં એકસાથે મલ્ટિપલ પ્રોગ્રામ્સ \\ \hline
\keyword{CPU Usage} & વધુ સારું CPU યુટિલાઇઝેશન \\ \hline
\keyword{Advantage} & આઈડલ ટાઇમમાં ઘટાડો \\ \hline
\end{tabulary}
\end{center}

\textbf{III. Time Sharing Operating System}

\begin{center}
\captionof{table}{Time Sharing OS}
\begin{tabulary}{\linewidth}{|L|L|}
\hline
\textbf{ફીચર} & \textbf{વર્ણન} \\ \hline
\keyword{Time Slices} & યુઝર્સ વચ્ચે CPU ટાઇમ વહેંચાયેલ \\ \hline
\keyword{Response} & ઝડપી રિસ્પોન્સ ટાઇમ \\ \hline
\keyword{Example} & Unix, Linux \\ \hline
\end{tabulary}
\end{center}
\end{solutionbox}

\begin{mnemonicbox}
\mnemonic{BMT: Batch, Multiprogramming, Time-sharing}
\end{mnemonicbox}

\questionmarks{1(c) OR}{7}{Linux આર્કિટેક્ચર અને તેની લાક્ષણિકતાઓ ઘટકો સાથે સમજાવો.}

\begin{solutionbox}
\textbf{Linux આર્કિટેક્ચર}:

\begin{center}
\begin{tikzpicture}[node distance=1.2cm]
    \node [rectangle, draw, fill=blue!10, text width=8cm, text centered, minimum height=1cm] (users) {User Applications / Compilers};
    \node [rectangle, draw, fill=green!10, text width=8cm, text centered, minimum height=1cm, below=0.5cm of users] (shell) {Shell / System Libraries};
    \node [rectangle, draw, fill=red!10, text width=8cm, text centered, minimum height=1.5cm, below=0.5cm of shell] (kernel) {Linux Kernel\\(File System, Memory Mgmt, Process Mgmt, Device Drivers)};
    \node [rectangle, draw, fill=gray!20, text width=8cm, text centered, minimum height=1cm, below=0.5cm of kernel] (hardware) {Hardware (CPU, RAM, I/O Device)};
    
    \path [gtu arrow] (users) -- (shell);
    \path [gtu arrow] (shell) -- (kernel);
    \path [gtu arrow] (kernel) -- (hardware);
\end{tikzpicture}
\captionof{figure}{Linux Architecture}
\end{center}

\textbf{Linux લાક્ષણિકતાઓ}:

\begin{center}
\captionof{table}{લાક્ષણિકતાઓ}
\begin{tabulary}{\linewidth}{|L|L|}
\hline
\textbf{લાક્ષણિકતા} & \textbf{વર્ણન} \\ \hline
\keyword{Open Source} & ફ્રી અને મોડિફાય કરી શકાય તેવું \\ \hline
\keyword{Multiuser} & એકસાથે મલ્ટિપલ યુઝર્સ \\ \hline
\keyword{Multitasking} & એકસાથે મલ્ટિપલ પ્રોસેસીસ \\ \hline
\keyword{Portable} & વિવિધ હાર્ડવેર પર ચાલે છે \\ \hline
\end{tabulary}
\end{center}

\textbf{ઘટકો}:
\begin{itemize}
    \item \keyword{Kernel}: ઓપરેટિંગ સિસ્ટમનું કોર (હાર્દ)
    \item \keyword{Shell}: કમાન્ડ ઇન્ટરપ્રીટર
    \item \keyword{File System}: ડેટા સ્ટોરેજ ઓર્ગેનાઇઝ કરે છે
\end{itemize}
\end{solutionbox}

\begin{mnemonicbox}
\mnemonic{COMP: Core, Open source, Multiuser, Portable}
\end{mnemonicbox}

\questionmarks{2(a)}{3}{Process Control Block વર્ણવો. અને વ્યાખ્યાયિત કરો (1) PID (2) stack pointer (3) program counter}

\begin{solutionbox}
\textbf{Process Control Block (PCB)}: OS મેનેજમેન્ટ માટે પ્રોસેસ માહિતી ધરાવતું ડેટા સ્ટ્રક્ચર.

\textbf{વ્યાખ્યાઓ}:

\begin{center}
\captionof{table}{PCB વ્યાખ્યાઓ}
\begin{tabulary}{\linewidth}{|L|L|}
\hline
\textbf{શબ્દ} & \textbf{વ્યાખ્યા} \\ \hline
\keyword{PID} & Process Identifier - દરેક પ્રોસેસ માટે યુનિક નંબર \\ \hline
\keyword{Stack Pointer} & પ્રોસેસ સ્ટેકના ટોપને પોઇન્ટ કરે છે \\ \hline
\keyword{Program Counter} & આગામી ઇન્સ્ટ્રક્શનનું એડ્રેસ ધરાવે છે \\ \hline
\end{tabulary}
\end{center}
\end{solutionbox}

\begin{mnemonicbox}
\mnemonic{PSP: PID, Stack pointer, Program counter}
\end{mnemonicbox}

\questionmarks{2(b)}{4}{Process Model અને Process states સમજાવો.}

\begin{solutionbox}
\textbf{Process Model}: OS દ્વારા પ્રોસેસીસ કેવી રીતે મેનેજ થાય છે તેનું કન્સેપ્ચ્યુઅલ રિપ્રેઝન્ટેશન.

\textbf{Process States}:

\begin{center}
\begin{tikzpicture}[node distance=2cm, auto]
    \node [gtu state] (new) {New};
    \node [gtu state, right=of new] (ready) {Ready};
    \node [gtu state, right=of ready] (running) {Running};
    \node [gtu state, right=of running] (terminated) {Terminated};
    \node [gtu state, below=of ready] (waiting) {Waiting};
    
    \path [gtu arrow] (new) -- (ready);
    \path [gtu arrow] (ready) -- node[yshift=0.2cm] {Dispatch} (running);
    \path [gtu arrow] (running) -- node[above] {Exit} (terminated);
    \path [gtu arrow] (running) -- node[right] {I/O Wait} (waiting);
    \path [gtu arrow] (waiting) -- node[left] {I/O Complete} (ready);
    \path [gtu arrow] (running) edge[bend right=45] node[above] {Interrupt} (ready);
\end{tikzpicture}
\captionof{figure}{Process State Diagram}
\end{center}

\begin{itemize}
    \item \keyword{New}: પ્રોસેસ ક્રિએટ થઈ રહી છે
    \item \keyword{Ready}: CPU માટે રાહ જોઈ રહી છે
    \item \keyword{Running}: ઇન્સ્ટ્રક્શન્સ એક્ઝિક્યુટ થઈ રહી છે
    \item \keyword{Waiting}: I/O માટે રાહ જોઈ રહી છે
    \item \keyword{Terminated}: પ્રોસેસ પૂર્ણ થઈ
\end{itemize}
\end{solutionbox}

\begin{mnemonicbox}
\mnemonic{NRRWT: New, Ready, Running, Waiting, Terminated}
\end{mnemonicbox}

\questionmarks{2(c)}{7}{Scheduling Algorithm ઉદાહરણ સાથે સમજાવો: (I) First Come First Serve, (II) Shortest Job First}

\begin{solutionbox}
\textbf{I. First Come First Serve (FCFS)}

\begin{center}
\captionof{table}{FCFS Scheduling}
\begin{tabulary}{\linewidth}{|C|C|C|C|C|}
\hline
\textbf{Process} & \textbf{Arrival} & \textbf{Burst} & \textbf{Completion} & \textbf{Turnaround} \\ \hline
P1 & 0 & 4 & 4 & 4 \\ \hline
P2 & 1 & 3 & 7 & 6 \\ \hline
P3 & 2 & 2 & 9 & 7 \\ \hline
\end{tabulary}
\end{center}

\textbf{Average Turnaround Time} = (4+6+7)/3 = 5.67

\textbf{II. Shortest Job First (SJF)}

\begin{center}
\captionof{table}{SJF Scheduling}
\begin{tabulary}{\linewidth}{|C|C|C|C|C|}
\hline
\textbf{Process} & \textbf{Arrival} & \textbf{Burst} & \textbf{Completion} & \textbf{Turnaround} \\ \hline
P3 & 2 & 2 & 4 & 2 \\ \hline
P2 & 1 & 3 & 7 & 6 \\ \hline
P1 & 0 & 4 & 11 & 11 \\ \hline
\end{tabulary}
\end{center}

\textbf{Average Turnaround Time} = (2+6+11)/3 = 6.33
\end{solutionbox}

\begin{mnemonicbox}
\mnemonic{FS: FCFS (First order), SJF (Shortest first)}
\end{mnemonicbox}

\questionmarks{2(a) OR}{3}{Race condition, Mutual Exclusion વ્યાખ્યાયિત કરો.}

\begin{solutionbox}
\begin{center}
\captionof{table}{Race vs Mutual Exclusion}
\begin{tabulary}{\linewidth}{|L|L|}
\hline
\textbf{શબ્દ} & \textbf{વ્યાખ્યા} \\ \hline
\keyword{Race Condition} & જ્યારે મલ્ટિપલ પ્રોસેસીસ એકસાથે ડેટા એક્સેસ કરે અને પરિણામ અનિશ્ચિત હોય \\ \hline
\keyword{Mutual Exclusion} & એક સમયે માત્ર એક જ પ્રોસેસ ક્રિટિકલ સેક્શન એક્સેસ કરી શકે \\ \hline
\end{tabulary}
\end{center}
\end{solutionbox}

\begin{mnemonicbox}
\mnemonic{RM: Race (simultaneous access), Mutual (one at a time)}
\end{mnemonicbox}

\questionmarks{2(b) OR}{4}{Throughput, Turnaround Time, Waiting Time, Response Time વ્યાખ્યાયિત કરો.}

\begin{solutionbox}
\begin{center}
\captionof{table}{સ્કેડ્યુલિંગ મેટ્રિક્સ}
\begin{tabulary}{\linewidth}{|L|L|}
\hline
\textbf{શબ્દ} & \textbf{વ્યાખ્યા} \\ \hline
\keyword{Throughput} & પ્રતિ એકમ સમયમાં પૂર્ણ થતી પ્રોસેસીસની સંખ્યા \\ \hline
\keyword{Turnaround Time} & સબમિશનથી લઈને પૂર્ણ થવા સુધીનો કુલ સમય \\ \hline
\keyword{Waiting Time} & રેડી કતારમાં વિતાવેલો સમય \\ \hline
\keyword{Response Time} & સબમિશનથી પ્રથમ પ્રતિસાદ સુધીનો સમય \\ \hline
\end{tabulary}
\end{center}
\end{solutionbox}

\begin{mnemonicbox}
\mnemonic{TTWR: Throughput, Turnaround, Waiting, Response}
\end{mnemonicbox}

\questionmarks{2(c) OR}{7}{Round Robin Algorithm ઉદાહરણ સાથે સમજાવો.}

\begin{solutionbox}
\textbf{Round Robin}: દરેક પ્રોસેસને સમાન CPU ટાઇમ સ્લાઇસ (ક્વોન્ટમ) મળે છે.

\textbf{ઉદાહરણ} (Time Quantum = 2):

\begin{center}
\captionof{table}{RR ઉદાહરણ}
\begin{tabulary}{\linewidth}{|C|C|}
\hline
\textbf{Process} & \textbf{Burst Time} \\ \hline
P1 & 5 \\ \hline
P2 & 3 \\ \hline
P3 & 4 \\ \hline
\end{tabulary}
\end{center}

\textbf{Execution Timeline}:

\begin{center}
\begin{tikzpicture}[x=0.8cm, y=0.8cm]
    \draw (0,0) rectangle (2,1) node[midway] {P1};
    \draw (2,0) rectangle (4,1) node[midway] {P2};
    \draw (4,0) rectangle (6,1) node[midway] {P3};
    \draw (6,0) rectangle (8,1) node[midway] {P1};
    \draw (8,0) rectangle (10,1) node[midway] {P3};
    \draw (10,0) rectangle (12,1) node[midway] {P1};
    
    \foreach \x in {0,2,4,6,8,10,12}
        \node at (\x, -0.5) {\x};
        
    \node [anchor=west] at (13,0.5) {Time};
\end{tikzpicture}
\captionof{figure}{RR Execution Timeline}
\end{center}

\textbf{ફાયદા}:
\begin{itemize}
    \item \keyword{Fair}: બધી પ્રોસેસીસને સમાન સમય
    \item \keyword{Responsive}: ઇન્ટરેક્ટિવ સિસ્ટમ્સ માટે સારું
\end{itemize}
\end{solutionbox}

\begin{mnemonicbox}
\mnemonic{RR-FE: Round Robin gives Fair and Equal time}
\end{mnemonicbox}

\questionmarks{3(a)}{3}{File Access Methods ના પ્રકાર આપો.}

\begin{solutionbox}
\begin{center}
\captionof{table}{File Access Methods}
\begin{tabulary}{\linewidth}{|L|L|}
\hline
\textbf{પદ્ધતિ} & \textbf{વર્ણન} \\ \hline
\keyword{Sequential} & શરૂઆતથી ક્રમમાં વાંચવું/લખવું \\ \hline
\keyword{Direct} & કોઈપણ રેકોર્ડને સીધો એક્સેસ કરવો \\ \hline
\keyword{Indexed} & રેકોર્ડ્સ શોધવા માટે ઇન્ડેક્સનો ઉપયોગ કરવો \\ \hline
\end{tabulary}
\end{center}
\end{solutionbox}

\begin{mnemonicbox}
\mnemonic{SDI: Sequential, Direct, Indexed}
\end{mnemonicbox}

\questionmarks{3(b)}{4}{Deadlock ની લાક્ષણિકતાઓ આપો અને વર્ણવો: Deadlock Prevention, Deadlock Avoidance}

\begin{solutionbox}
\textbf{Deadlock લાક્ષણિકતાઓ}:

\begin{center}
\captionof{table}{Deadlock શરતો}
\begin{tabulary}{\linewidth}{|L|L|}
\hline
\textbf{શરત} & \textbf{વર્ણન} \\ \hline
\keyword{Mutual Exclusion} & રિસોર્સિસ શેર કરી શકાતા નથી \\ \hline
\keyword{Hold and Wait} & પ્રોસેસ રિસોર્સ હોલ્ડ કરે છે અને બીજાની રાહ જુએ છે \\ \hline
\keyword{No Preemption} & રિસોર્સિસ બળજબરીથી લઈ શકાતા નથી \\ \hline
\keyword{Circular Wait} & વેઈટિંગ પ્રોસેસીસની ગોળાકાર ચેઇન \\ \hline
\end{tabulary}
\end{center}

\textbf{Deadlock Prevention}: ચારમાંથી કોઈપણ એક શરત દૂર કરવી.

\textbf{Deadlock Avoidance}: Banker's algorithm જેવા અલ્ગોરિધમ્સનો ઉપયોગ કરવો.
\end{solutionbox}

\begin{mnemonicbox}
\mnemonic{MHNC: Mutual exclusion, Hold and wait, No preemption, Circular wait}
\end{mnemonicbox}

\questionmarks{3(c)}{7}{File Allocation Methods સમજાવો: Contiguous, linked, indexed}

\begin{solutionbox}
\textbf{File Allocation Methods}:

\begin{center}
\captionof{table}{Allocation Methods}
\begin{tabulary}{\linewidth}{|L|L|L|}
\hline
\textbf{પદ્ધતિ} & \textbf{વર્ણન} & \textbf{ફાયદા/ગેરફાયદા} \\ \hline
\keyword{Contiguous} & ક્રમિક બ્લોક્સ & ઝડપી એક્સેસ \\ \hline
\keyword{Linked} & પોઇન્ટર્સ સાથે છૂટાછવાયા બ્લોક્સ & કોઈ ફ્રેગમેન્ટેશન નહીં \\ \hline
\keyword{Indexed} & ઇન્ડેક્સ બ્લોક એડ્રેસ ધરાવે છે & ઝડપી રેન્ડમ એક્સેસ \\ \hline
\end{tabulary}
\end{center}

\textbf{I. Contiguous Allocation}:

\begin{center}
\begin{tikzpicture}[node distance=0cm, outer sep=0pt]
    \node [rectangle, draw, minimum width=1cm, minimum height=1cm] (b1) {1};
    \node [rectangle, draw, minimum width=1cm, minimum height=1cm, right=of b1] (b2) {2};
    \node [rectangle, draw, minimum width=1cm, minimum height=1cm, right=of b2] (b3) {3};
    \node [rectangle, draw, minimum width=1cm, minimum height=1cm, right=of b3] (b4) {4};
    \node [rectangle, draw, minimum width=1cm, minimum height=1cm, right=of b4] (b5) {5};
    \node [above=0.2cm of b3] {File A (Start:1, Length:5)};
\end{tikzpicture}
\captionof{figure}{Contiguous Allocation}
\end{center}

\textbf{II. Linked Allocation}:

\begin{center}
\begin{tikzpicture}[node distance=1cm, auto]
    \node [gtu state] (b1) {1};
    \node [gtu state, right=of b1] (b7) {7};
    \node [gtu state, right=of b7] (b3) {3};
    \node [gtu state, right=of b3] (b9) {9};
    \node [gtu state, right=of b9] (null) {NULL};
    
    \path [gtu arrow] (b1) -- (b7);
    \path [gtu arrow] (b7) -- (b3);
    \path [gtu arrow] (b3) -- (b9);
    \path [gtu arrow] (b9) -- (null);
\end{tikzpicture}
\captionof{figure}{Linked Allocation}
\end{center}

\textbf{III. Indexed Allocation}:

\begin{center}
\begin{tikzpicture}[node distance=1.5cm]
    \node [rectangle, draw, fill=yellow!10, inner sep=0pt] (index) {\begin{tabular}{c}Index Block\\ \hline 1\\3\\7\\9\\12\end{tabular}};
    \node [gtu block, right=of index, yshift=1.5cm] (b1) {Block 1};
    \node [gtu block, right=of index, yshift=0.7cm] (b3) {Block 3};
    \node [gtu block, right=of index, yshift=-0.1cm] (b7) {Block 7};
    \node [gtu block, right=of index, yshift=-1.5cm] (b9) {Block 9};
    
    \draw [->] (index.east) -- (b1.west);
    \draw [->] (index.east) -- (b3.west);
    \draw [->] (index.east) -- (b7.west);
    \draw [->] (index.east) -- (b9.west);
\end{tikzpicture}
\captionof{figure}{Indexed Allocation}
\end{center}
\end{solutionbox}

\begin{mnemonicbox}
\mnemonic{CLI: Contiguous, Linked, Indexed}
\end{mnemonicbox}

\questionmarks{3(a) OR}{3}{Linux File System Structure વિશે જણાવો.}

\begin{solutionbox}
\textbf{Linux File System Hierarchy}:

\begin{center}
\begin{tikzpicture}[level distance=1.5cm, level 1/.style={sibling distance=2.5cm}, level 2/.style={sibling distance=1.5cm}]
    \node {/ (Root)}
        child {node {bin}
            child {node {ls}}
            child {node {cp}}
        }
        child {node {etc}
            child {node {passwd}}
        }
        child {node {home}
            child {node {user1}}
            child {node {user2}}
        }
        child {node {tmp}}
        child {node {usr}};
\end{tikzpicture}
\captionof{figure}{File System Tree}
\end{center}

\begin{center}
\captionof{table}{મહત્વની ડિરેક્ટરીઓ}
\begin{tabulary}{\linewidth}{|L|L|}
\hline
\textbf{Directory} & \textbf{હેતુ} \\ \hline
\keyword{/bin} & આવશ્યક સિસ્ટમ બાઈનરીઝ \\ \hline
\keyword{/etc} & સિસ્ટમ કોન્ફિગરેશન ફાઈલો \\ \hline
\keyword{/home} & યુઝર હોમ ડિરેક્ટરીઓ \\ \hline
\end{tabulary}
\end{center}
\end{solutionbox}

\begin{mnemonicbox}
\mnemonic{BEH: Bin, Etc, Home}
\end{mnemonicbox}

\questionmarks{3(b) OR}{4}{Critical Section અને Semaphore ઉદાહરણ સાથે સમજાવો.}

\begin{solutionbox}
\textbf{Critical Section}: કોડ સેગમેન્ટ જે શેર કરેલા રિસોર્સિસનો ઉપયોગ કરે છે.

\textbf{સેક્શનનું માળખું}:
\begin{itemize}
    \item \keyword{Entry}: પરવાનગી વિનંતી
    \item \keyword{Critical}: રિસોર્સ એક્સેસ
    \item \keyword{Exit}: પરવાનગી મુક્ત કરવી
    \item \keyword{Remainder}: અન્ય કોડ
\end{itemize}

\textbf{Semaphore}: સિંક્રોનાઇઝેશન ટૂલ જે કાઉન્ટર વેરિએબલનો ઉપયોગ કરે છે.
\end{solutionbox}

\begin{mnemonicbox}
\mnemonic{ECER: Entry, Critical, Exit, Remainder}
\end{mnemonicbox}

\questionmarks{3(c) OR}{7}{Deadlock Avoidance, Detection અને Recovery વ્યાખ્યાયિત કરો અને સમજાવો.}

\begin{solutionbox}
\textbf{Deadlock Avoidance}:
\begin{itemize}
    \item \keyword{Banker's Algorithm} નો ઉપયોગ કરે છે
    \item રિસોર્સ એલોકેશન સેફ સ્ટેટ તરફ દોરી જાય છે કે નહીં તે તપાસે છે
\end{itemize}

\textbf{Deadlock Detection}:
\begin{itemize}
    \item \keyword{Wait-for Graph} નો ઉપયોગ કરીને ડેડલોક માટે સમયાંતરે તપાસ કરે છે
\end{itemize}

\textbf{Deadlock Recovery પદ્ધતિઓ}:
\begin{itemize}
    \item \keyword{Process Termination}: ડેડલોક થયેલ પ્રોસેસીસને બંધ કરવી
    \item \keyword{Resource Preemption}: પ્રોસેસીસ પાસેથી રિસોર્સિસ પાછા લેવા
    \item \keyword{Rollback}: પાછલા સેફ સ્ટેટ પર પાછા જવું
\end{itemize}
\end{solutionbox}

\begin{mnemonicbox}
\mnemonic{ADR: Avoidance, Detection, Recovery}
\end{mnemonicbox}

\questionmarks{4(a)}{3}{File Protection ની જરૂરિયાત સમજાવો?}

\begin{solutionbox}
\textbf{File Protection ની જરૂરિયાત}:

\begin{center}
\captionof{table}{Protection જરૂરિયાતો}
\begin{tabulary}{\linewidth}{|L|L|}
\hline
\textbf{કારણ} & \textbf{વર્ણન} \\ \hline
\keyword{Privacy} & વ્યક્તિગત ડેટા સુરક્ષિત કરવા \\ \hline
\keyword{Security} & અનધિકૃત એક્સેસ અટકાવવા \\ \hline
\keyword{Integrity} & ડેટાની સાતત્યતા જાળવવા \\ \hline
\end{tabulary}
\end{center}

\textbf{Protection મિકેનિઝમ્સ}:
\begin{itemize}
    \item Access Control Lists (ACL)
    \item File Permissions (Read, Write, Execute)
    \item User Authentication
\end{itemize}
\end{solutionbox}

\begin{mnemonicbox}
\mnemonic{PSI: Privacy, Security, Integrity}
\end{mnemonicbox}

\questionmarks{4(b)}{4}{Program threats અને System threats સમજાવો.}

\begin{solutionbox}
\textbf{Program Threats}:
\begin{itemize}
    \item \keyword{Virus}: સેલ્ફ-રેપ્લિકેટિંગ મેલિશિયસ કોડ
    \item \keyword{Worm}: નેટવર્કમાં ફેલાતું માલવેર
    \item \keyword{Trojan Horse}: છૂપાયેલ મેલિશિયસ પ્રોગ્રામ
\end{itemize}

\textbf{System Threats}:
\begin{itemize}
    \item \keyword{Denial of Service}: સિસ્ટમ રિસોર્સિસને ઓવરવેલ્મ કરવું
    \item \keyword{Port Scanning}: નબળી સર્વિસીસ શોધવી
    \item \keyword{Man-in-Middle}: કોમ્યુનિકેશન્સ ઇન્ટરસેપ્ટ કરવું
\end{itemize}
\end{solutionbox}

\begin{mnemonicbox}
\mnemonic{VWT-DPM: Virus, Worm, Trojan; DoS, Port scan, Man-in-middle}
\end{mnemonicbox}

\questionmarks{4(c)}{7}{Operating System security policies અને procedures વિશે ટૂંકમાં જણાવો.}

\begin{solutionbox}
\textbf{Security Policies}:

\begin{center}
\captionof{table}{Security Policies}
\begin{tabulary}{\linewidth}{|L|L|}
\hline
\textbf{પોલિસી} & \textbf{વર્ણન} \\ \hline
\keyword{Access Control} & કોણ કયા રિસોર્સિસ એક્સેસ કરી શકે \\ \hline
\keyword{Authentication} & યુઝર ઓળખની ખરાઈ \\ \hline
\keyword{Authorization} & યુઝર પરવાનગીઓ નક્કી કરવી \\ \hline
\keyword{Audit} & પ્રવૃત્તિઓ મોનિટર અને લોગ કરવી \\ \hline
\end{tabulary}
\end{center}

\textbf{Security Procedures Flow}:

\begin{center}
\begin{tikzpicture}[node distance=1.5cm, auto]
    \node [gtu state] (login) {User Login};
    \node [gtu decision, right=of login] (auth) {Auth?};
    \node [gtu state, right=of auth] (access) {Access};
    \node [gtu state, below=of access] (log) {Log Activity};
    \node [gtu state, below=of login] (deny) {Deny};
    
    \path [gtu arrow] (login) -- (auth);
    \path [gtu arrow] (auth) -- node[above] {Yes} (access);
    \path [gtu arrow] (auth) -- node[left] {No} (deny);
    \path [gtu arrow] (access) -- (log);
\end{tikzpicture}
\captionof{figure}{Security Flow}
\end{center}

\textbf{ઈમ્પ્લીમેન્ટેશન સ્ટેપ્સ}:
\begin{enumerate}
    \item User Registration અને ક્રેડેન્શિયલ સેટઅપ
    \item Multi-factor Authentication
    \item Role-based Access Control
    \item નિયમિત Security Audits
\end{enumerate}
\end{solutionbox}

\begin{mnemonicbox}
\mnemonic{AAAA: Access control, Authentication, Authorization, Audit}
\end{mnemonicbox}

\questionmarks{4(a) OR}{3}{Authentication અને Authorization વિશે ખ્યાલ આપો.}

\begin{solutionbox}
\begin{center}
\captionof{table}{Auth vs Authz}
\begin{tabulary}{\linewidth}{|L|L|L|}
\hline
\textbf{શબ્દ} & \textbf{વ્યાખ્યા} & \textbf{ઉદાહરણ} \\ \hline
\keyword{Authentication} & યુઝર ઓળખની ખરાઈ & Username/password \\ \hline
\keyword{Authorization} & એક્સેસ અધિકારો નક્કી કરવા & File permissions \\ \hline
\end{tabulary}
\end{center}
\end{solutionbox}

\begin{mnemonicbox}
\mnemonic{AA: Authentication (who), Authorization (what)}
\end{mnemonicbox}

\questionmarks{4(b) OR}{4}{Operating System security policies અને procedures સમજાવો.}

\begin{solutionbox}
\textbf{Security Policies Framework}:

\begin{center}
\captionof{table}{Security Framework}
\begin{tabulary}{\linewidth}{|L|L|}
\hline
\textbf{ઘટક} & \textbf{હેતુ} \\ \hline
\keyword{User Management} & યુઝર એકાઉન્ટ્સ નિયંત્રિત કરવા \\ \hline
\keyword{Data Protection} & સંવેદનશીલ માહિતી સુરક્ષિત કરવી \\ \hline
\keyword{Network Security} & કોમ્યુનિકેશન્સ સુરક્ષિત કરવા \\ \hline
\keyword{System Monitoring} & જોખમો શોધવા \\ \hline
\end{tabulary}
\end{center}
\end{solutionbox}

\begin{mnemonicbox}
\mnemonic{UDNS: User, Data, Network, System}
\end{mnemonicbox}

\questionmarks{4(c) OR}{7}{Operating System માં Security measures વિશે વિગતવાર જણાવો.}

\begin{solutionbox}
\textbf{Comprehensive Security Measures}:
\begin{itemize}
    \item \keyword{Physical}: સર્વર રૂમ એક્સેસ, બાયોમેટ્રિક લોક્સ
    \item \keyword{Network}: Firewalls, VPN
    \item \keyword{System}: Antivirus, patches
    \item \keyword{Application}: Secure coding
    \item \keyword{Data}: Encryption, backups
\end{itemize}

\textbf{Access Control Matrix Example}:
\begin{center}
\captionof{table}{Access Matrix}
\begin{tabulary}{\linewidth}{|L|C|C|}
\hline
\textbf{User} & \textbf{File A} & \textbf{File B} \\ \hline
Admin & RWX & RWX \\ \hline
User1 & RW- & R-- \\ \hline
Guest & R-- & --- \\ \hline
\end{tabulary}
\end{center}

\textbf{Security Implementation Timeline}:

\begin{center}
\begin{tikzpicture}
    \draw [->] (0,0) -- (10,0) node[right] {Time};
    
    \foreach \x in {0,2.5,5,7.5}
        \draw (\x, 0.1) -- (\x, -0.1);
        
    \node [anchor=north] at (0,-0.2) {Day 0};
    \node [anchor=south] at (0,0.2) {Assess};
    
    \node [anchor=north] at (2.5,-0.2) {Day 30};
    \node [anchor=south] at (2.5,0.2) {Plan};
    
    \node [anchor=north] at (5,-0.2) {Day 45};
    \node [anchor=south] at (5,0.2) {Harden};
    
    \node [anchor=north] at (7.5,-0.2) {Day 60};
    \node [anchor=south] at (7.5,0.2) {Train};
\end{tikzpicture}
\captionof{figure}{Implementation Timeline}
\end{center}
\end{solutionbox}

\begin{mnemonicbox}
\mnemonic{PNSAD: Physical, Network, System, Application, Data}
\end{mnemonicbox}

\questionmarks{5(a)}{3}{પાંચ Basic commands જણાવો: calendar, date}

\begin{solutionbox}
\textbf{Basic Linux Commands}:

\begin{center}
\captionof{table}{Basic Commands}
\begin{tabulary}{\linewidth}{|L|L|L|}
\hline
\textbf{Command} & \textbf{કાર્ય} & \textbf{ઉદાહરણ} \\ \hline
\code{cal} & કેલેન્ડર દર્શાવે છે & \code{cal 2024} \\ \hline
\code{date} & વર્તમાન તારીખ/સમય બતાવે છે & \code{date +\%d/\%m/\%Y} \\ \hline
\code{who} & લોગ્ડ યુઝર્સ બતાવે છે & \code{who} \\ \hline
\code{pwd} & વર્કિંગ ડિરેક્ટરી પ્રિન્ટ કરે છે & \code{pwd} \\ \hline
\code{clear} & સ્ક્રીન સાફ કરે છે & \code{clear} \\ \hline
\end{tabulary}
\end{center}
\end{solutionbox}

\begin{mnemonicbox}
\mnemonic{CDWPC: Cal, Date, Who, Pwd, Clear}
\end{mnemonicbox}

\questionmarks{5(b)}{4}{Linux File અને Directory Commands સમજાવો: ls, cat, mkdir, rmdir, pwd.}

\begin{solutionbox}
\textbf{File અને Directory Commands}:

\begin{center}
\captionof{table}{File Commands}
\begin{tabulary}{\linewidth}{|L|L|L|}
\hline
\textbf{Command} & \textbf{કાર્ય} & \textbf{ઉદાહરણ} \\ \hline
\code{ls} & ડિરેક્ટરી કન્ટેન્ટ લિસ્ટ કરે છે & \code{ls -la} \\ \hline
\code{cat} & ફાઈલ કન્ટેન્ટ દર્શાવે છે & \code{cat f.txt} \\ \hline
\code{mkdir} & નવી ડિરેક્ટરી બનાવે છે & \code{mkdir new} \\ \hline
\code{rmdir} & ખાલી ડિરેક્ટરી દૂર કરે છે & \code{rmdir old} \\ \hline
\end{tabulary}
\end{center}
\end{solutionbox}

\begin{mnemonicbox}
\mnemonic{LCMRP: List, Cat, Mkdir, Rmdir, Pwd}
\end{mnemonicbox}

\questionmarks{5(c)}{7}{કંટ્રોલ સ્ટેટમેન્ટ્સ સમજો અને લાગુ કરો: ત્રણ નંબરોમાંથી મહત્તમ શોધવા માટે શેલ સ્ક્રીપ્ટ લખો.}

\begin{solutionbox}
\begin{lstlisting}[language=bash,caption={Maximum of 3 Numbers}]
#!/bin/bash
# Script to find maximum of three numbers

echo "Enter three numbers:"
read -p "First number: " num1
read -p "Second number: " num2
read -p "Third number: " num3

# Method 1: Using if-elif-else
if [ $num1 -ge $num2 ] && [ $num1 -ge $num3 ]; then
    max=$num1
elif [ $num2 -ge $num1 ] && [ $num2 -ge $num3 ]; then
    max=$num2
else
    max=$num3
fi

echo "Maximum number is: $max"
\end{lstlisting}

\textbf{Comparison Operators}:
\begin{itemize}
    \item \code{-gt}: Greater than
    \item \code{-ge}: Greater than or equal to
    \item \code{-eq}: Equal to
\end{itemize}
\end{solutionbox}

\begin{mnemonicbox}
\mnemonic{IER: If (condition), Echo (output), Read (input)}
\end{mnemonicbox}

\questionmarks{5(a) OR}{3}{Linux Process commands શું છે: top, ps, kill}

\begin{solutionbox}
\textbf{Linux Process Commands}:

\begin{center}
\captionof{table}{Process Commands}
\begin{tabulary}{\linewidth}{|L|L|L|}
\hline
\textbf{Command} & \textbf{કાર્ય} & \textbf{ઉપયોગ} \\ \hline
\code{top} & રનિંગ પ્રોસેસીસ દર્શાવે છે & \code{top} \\ \hline
\code{ps} & પ્રોસેસ સ્ટેટસ બતાવે છે & \code{ps aux} \\ \hline
\code{kill} & પ્રોસેસ ટર્મિનેટ કરે છે & \code{kill PID} \\ \hline
\end{tabulary}
\end{center}
\end{solutionbox}

\begin{mnemonicbox}
\mnemonic{TPK: Top, Ps, Kill}
\end{mnemonicbox}

\questionmarks{5(b) OR}{4}{Linux File અને Directory Commands: rm, mv, split, diff, grep}

\begin{solutionbox}
\textbf{Advanced File Commands}:

\begin{center}
\captionof{table}{Advanced Commands}
\begin{tabulary}{\linewidth}{|L|L|L|}
\hline
\textbf{Cmd} & \textbf{કાર્ય} & \textbf{ઉદાહરણ} \\ \hline
\code{rm} & ફાઈલો દૂર કરે છે & \code{rm -rf f} \\ \hline
\code{mv} & મુવ/રીનેમ કરે છે & \code{mv a b} \\ \hline
\code{split} & ફાઈલો સ્પ્લિટ કરે છે & \code{split -l 50} \\ \hline
\code{diff} & ફાઈલો સરખાવે છે & \code{diff a b} \\ \hline
\code{grep} & ટેક્સ્ટ સર્ચ કરે છે & \code{grep "err" f} \\ \hline
\end{tabulary}
\end{center}
\end{solutionbox}

\begin{mnemonicbox}
\mnemonic{RMSDG: Remove, Move, Split, Diff, Grep}
\end{mnemonicbox}

\questionmarks{5(c) OR}{7}{યુઝર પાસેથી પાંચ નંબરો વાંચવા અને પાંચ નંબરોની સરેરાશ શોધવા માટે શેલ સ્ક્રીપ્ટ લખો.}

\begin{solutionbox}
\begin{lstlisting}[language=bash,caption={Average of 5 Numbers}]
#!/bin/bash
# Script to calculate average of five numbers

echo "=== Average Calculator ==="
sum=0

echo "Enter 5 numbers:"
for i in {1..5}; do
    read -p "Enter number $i: " num
    sum=$((sum + num))
done

# Calculate average
average=$((sum / 5))

echo "--------------------------------"
echo "Sum: $sum"
echo "Average: $average"
echo "--------------------------------"
\end{lstlisting}
\end{solutionbox}

\begin{mnemonicbox}
\mnemonic{RSAR: Read, Sum, Average, Result}
\end{mnemonicbox}

\end{document}
