\documentclass[10pt,a4paper]{article}

% content/resources/templates/preamble.tex
\usepackage[margin=0.6in]{geometry}
\author{Milav Dabgar}
\usepackage{amsmath,amssymb,amsthm}
\usepackage{booktabs}
\usepackage{multirow}
\usepackage{xcolor}
\usepackage{tcolorbox}
\tcbuselibrary{breakable,skins}
\usepackage[colorlinks=true,linkcolor=blue]{hyperref}
\usepackage{titlesec}
\usepackage{enumitem}
\usepackage{tikz}
\usepackage{pgfplots}
\usepackage{circuitikz}
\usepackage[version=4]{mhchem}
\usepackage{longtable}
\usepackage{array}
\usepackage{float}
\usepackage{caption}
\usepackage{listings}

\lstset{
  basicstyle=\small\ttfamily,
  breaklines=true,
  breakatwhitespace=false,
  postbreak=\mbox{\textcolor{red}{$\hookrightarrow$}\space},
  float=false,
  numbers=left,
  numberstyle=\tiny\color{gray},
  numbersep=10pt,
  xleftmargin=2em,
  keywordstyle=\color{blue},
  commentstyle=\color{green!60!black},
  stringstyle=\color{purple},
  backgroundcolor=\color{gray!5},
  showstringspaces=false,
  tabsize=2,
  captionpos=b,
  keepspaces=true,
  columns=flexible
}

\pgfplotsset{compat=1.18}
\usetikzlibrary{shapes,arrows,positioning,calc,patterns,decorations.pathmorphing,decorations.markings,arrows.meta}

% Color scheme
\definecolor{headcolor}{RGB}{0,102,204}
\definecolor{keycolor}{RGB}{220,20,60}
\definecolor{solutioncolor}{RGB}{34,139,34}
\definecolor{mnemoniccolor}{RGB}{148,0,211}
\definecolor{codecolor}{RGB}{0,0,100}

% Spacing
\setlength{\parskip}{3pt}
\setlist[itemize]{nosep}
\setlist[enumerate]{nosep}

% Title formatting
\titleformat{\section}{\Large\bfseries\color{headcolor}}{\thesection}{1em}{}
\titleformat{\subsection}{\large\bfseries\color{headcolor}}{\thesubsection}{1em}{}

% Pandoc tightlist compatibility
\providecommand{\tightlist}{%
  \setlength{\itemsep}{0pt}\setlength{\parskip}{0pt}}

% Pandoc longtable compatibility
\newcounter{none}
\def\thenone{}


% content/resources/templates/english-boxes.tex
% This file is currently empty - it exists to maintain consistency with the import structure.
% Add custom environments here if needed in the future.


\begin{document}

\begin{center}
{\Huge\bfseries\color{headcolor} Subject Name Solutions}\\[5pt]
{\LARGE 4331602 -- Winter 2023}\\[3pt]
{\large Semester 1 Study Material}\\[3pt]
{\normalsize\textit{Detailed Solutions and Explanations}}
\end{center}

\vspace{10pt}

\subsection*{Question 1(a) [3 marks]}\label{q1a}

\textbf{Draw the architecture of Linux and explain various layers in
brief.}

\begin{solutionbox}

\textbf{Diagram:}

\begin{center}
\textbf{Mermaid Diagram (Code)}
\begin{verbatim}
{Shaded}
{Highlighting}[]
graph LR
    A[User Applications] {-{-}{} B[System Call Interface]}
    B {-{-}{} C[Kernel]}
    C {-{-}{} D[Device Drivers]}
    D {-{-}{} E[Hardware]}
    
    C {-{-}{} F[Process Management]}
    C {-{-}{} G[Memory Management]}
    C {-{-}{} H[File System]}
    C {-{-}{} I[Network Management]}
{Highlighting}
{Shaded}
\end{verbatim}
\end{center}

\begin{itemize}
\tightlist
\item
  \textbf{User Space}: Contains user applications and system utilities
\item
  \textbf{System Call Interface}: Provides interface between user
  programs and kernel
\item
  \textbf{Kernel Space}: Core operating system with process, memory,
  file management
\end{itemize}

\end{solutionbox}
\begin{mnemonicbox}
``Users System Kernel Drives Hardware''

\end{mnemonicbox}
\begin{center}\rule{0.5\linewidth}{0.5pt}\end{center}

\subsection*{Question 1(b) [4 marks]}\label{q1b}

\textbf{What is a race condition? Explain with a suitable example.}

\begin{solutionbox}

{\def\LTcaptype{none} % do not increment counter
\begin{longtable}[]{@{}
  >{\raggedright\arraybackslash}p{(\linewidth - 2\tabcolsep) * \real{0.4286}}
  >{\raggedright\arraybackslash}p{(\linewidth - 2\tabcolsep) * \real{0.5714}}@{}}
\toprule\noalign{}
\begin{minipage}[b]{\linewidth}\raggedright
\textbf{Aspect}
\end{minipage} & \begin{minipage}[b]{\linewidth}\raggedright
\textbf{Description}
\end{minipage} \\
\midrule\noalign{}
\endhead
\bottomrule\noalign{}
\endlastfoot
\textbf{Definition} & Multiple processes accessing shared resource
simultaneously \\
\textbf{Problem} & Unpredictable results due to timing dependency \\
\textbf{Example} & Bank account balance update by two transactions \\
\end{longtable}
}

\textbf{Example Process:}

\begin{itemize}
\tightlist
\item
  \textbf{Process A}: Reads balance = 1000, adds 100
\item
  \textbf{Process B}: Reads balance = 1000, subtracts 50\\
\item
  \textbf{Result}: Final balance could be 1050, 950, or 1100 instead of
  correct 1050
\end{itemize}

\end{solutionbox}
\begin{mnemonicbox}
``Race Results Random Resources''

\end{mnemonicbox}
\begin{center}\rule{0.5\linewidth}{0.5pt}\end{center}

\subsection*{Question 1(c) [7 marks]}\label{q1c}

\textbf{List different types of Operating systems. Explain the working
of multiprogramming operating systems with a suitable example.}

\begin{solutionbox}


{\def\LTcaptype{none} % do not increment counter
\vspace{-5pt}
\captionof{table}{Types of Operating Systems}
\vspace{-10pt}
\begin{longtable}[]{@{}lll@{}}
\toprule\noalign{}
\textbf{Type} & \textbf{Characteristics} & \textbf{Example} \\
\midrule\noalign{}
\endhead
\bottomrule\noalign{}
\endlastfoot
\textbf{Batch} & Jobs processed in batches & IBM mainframes \\
\textbf{Time-sharing} & Multiple users simultaneously & UNIX \\
\textbf{Real-time} & Immediate response required & Air traffic
control \\
\textbf{Distributed} & Multiple connected computers & Google cluster \\
\textbf{Multiprogramming} & Multiple programs in memory & Windows,
Linux \\
\end{longtable}
}

\textbf{Multiprogramming Working:}

\begin{itemize}
\tightlist
\item
  \textbf{Memory Management}: Multiple programs loaded simultaneously
\item
  \textbf{CPU Scheduling}: Switches between programs when I/O occurs
\item
  \textbf{Resource Sharing}: Efficient utilization of CPU and memory
\item
  \textbf{Example}: Word processor, music player, and browser running
  together
\end{itemize}

\end{solutionbox}
\begin{mnemonicbox}
``Multiple Programs Maximize Performance''

\end{mnemonicbox}
\begin{center}\rule{0.5\linewidth}{0.5pt}\end{center}

\subsection*{Question 1(c OR) [7
marks]}\label{question-1c-or-7-marks}

\textbf{List different types of Operating systems. Explain the Batch
operating systems in detail.}

\begin{solutionbox}

\textbf{Types of Operating Systems:} Same table as above.

\textbf{Batch Operating System Details:}

\begin{itemize}
\tightlist
\item
  \textbf{Job Collection}: Jobs collected offline and grouped into
  batches
\item
  \textbf{Sequential Processing}: Jobs executed one after another
  without user interaction
\item
  \textbf{No Direct Interaction}: User submits job and collects output
  later
\item
  \textbf{Efficiency}: High throughput for similar type jobs
\item
  \textbf{Disadvantages}: No real-time processing, long turnaround time
\end{itemize}

\end{solutionbox}
\begin{mnemonicbox}
``Batch Brings Better Business''

\end{mnemonicbox}
\begin{center}\rule{0.5\linewidth}{0.5pt}\end{center}

\subsection*{Question 2(a) [3 marks]}\label{q2a}

\textbf{Draw and explain the Process life cycle.}

\begin{solutionbox}

\textbf{Diagram:}

\begin{verbatim}
stateDiagram{-v2}
  direction LR
    [*] {-{-} New}
    New {-{-} Ready : Admitted}
    Ready {-{-} Running : Scheduler\_dispatch}
    Running {-{-} Ready : Interrupt}
    Running {-{-} Waiting : I/O\_request}
    Waiting {-{-} Ready : I/O\_completion}
    Running {-{-} Terminated : Exit}
    Terminated {-{-} [*]}
\end{verbatim}

\begin{itemize}
\tightlist
\item
  \textbf{New}: Process being created
\item
  \textbf{Ready}: Process waiting for CPU assignment
\item
  \textbf{Running}: Process currently executing
\item
  \textbf{Waiting}: Process waiting for I/O operation
\item
  \textbf{Terminated}: Process has finished execution
\end{itemize}

\end{solutionbox}
\begin{mnemonicbox}
``New Ready Running Waiting Terminated''

\end{mnemonicbox}
\begin{center}\rule{0.5\linewidth}{0.5pt}\end{center}

\subsection*{Question 2(b) [4 marks]}\label{q2b}

\textbf{Define deadlock and discuss necessary conditions for a deadlock
to occur.}

\begin{solutionbox}

\textbf{Definition}: Deadlock occurs when processes wait indefinitely
for resources held by other processes.


{\def\LTcaptype{none} % do not increment counter
\vspace{-5pt}
\captionof{table}{Deadlock Conditions}
\vspace{-10pt}
\begin{longtable}[]{@{}
  >{\raggedright\arraybackslash}p{(\linewidth - 2\tabcolsep) * \real{0.4839}}
  >{\raggedright\arraybackslash}p{(\linewidth - 2\tabcolsep) * \real{0.5161}}@{}}
\toprule\noalign{}
\begin{minipage}[b]{\linewidth}\raggedright
\textbf{Condition}
\end{minipage} & \begin{minipage}[b]{\linewidth}\raggedright
\textbf{Description}
\end{minipage} \\
\midrule\noalign{}
\endhead
\bottomrule\noalign{}
\endlastfoot
\textbf{Mutual Exclusion} & Resources cannot be shared \\
\textbf{Hold and Wait} & Process holds resource while waiting for
another \\
\textbf{No Preemption} & Resources cannot be forcibly taken \\
\textbf{Circular Wait} & Processes form circular chain of resource
dependencies \\
\end{longtable}
}

\end{solutionbox}
\begin{mnemonicbox}
``My Hold Never Circles''

\end{mnemonicbox}
\begin{center}\rule{0.5\linewidth}{0.5pt}\end{center}

\subsection*{Question 2(c) [7 marks]}\label{q2c}

\textbf{Describe the Round Robin algorithm. Calculate the average
waiting time \& average turn-around time along with Gantt chart for the
given data. Consider context switch = 01 ms and quantum time = 05 ms.}

\begin{solutionbox}

\textbf{Round Robin Algorithm:}

\begin{itemize}
\tightlist
\item
  \textbf{Time Quantum}: Fixed time slice for each process
\item
  \textbf{Preemptive}: Process preempted after quantum expires
\item
  \textbf{Fair Scheduling}: Equal CPU time distribution
\end{itemize}

\textbf{Given Data:}

\begin{itemize}
\tightlist
\item
  Context Switch = 1 ms, Quantum = 5 ms
\end{itemize}

\textbf{Gantt Chart:}

\begin{verbatim}
|P1|CS|P2|CS|P3|CS|P4|CS|P1|CS|P3|CS|P1|CS|P3|CS|
0  5  6 10 11 16 17 22 23 28 29 34 35 40 41 46 47
\end{verbatim}

\textbf{Calculations Table:}

{\def\LTcaptype{none} % do not increment counter
\begin{longtable}[]{@{}
  >{\raggedright\arraybackslash}p{(\linewidth - 10\tabcolsep) * \real{0.1585}}
  >{\raggedright\arraybackslash}p{(\linewidth - 10\tabcolsep) * \real{0.1585}}
  >{\raggedright\arraybackslash}p{(\linewidth - 10\tabcolsep) * \real{0.1341}}
  >{\raggedright\arraybackslash}p{(\linewidth - 10\tabcolsep) * \real{0.1951}}
  >{\raggedright\arraybackslash}p{(\linewidth - 10\tabcolsep) * \real{0.1951}}
  >{\raggedright\arraybackslash}p{(\linewidth - 10\tabcolsep) * \real{0.1585}}@{}}
\toprule\noalign{}
\begin{minipage}[b]{\linewidth}\raggedright
\textbf{Process}
\end{minipage} & \begin{minipage}[b]{\linewidth}\raggedright
\textbf{Arrival}
\end{minipage} & \begin{minipage}[b]{\linewidth}\raggedright
\textbf{Burst}
\end{minipage} & \begin{minipage}[b]{\linewidth}\raggedright
\textbf{Completion}
\end{minipage} & \begin{minipage}[b]{\linewidth}\raggedright
\textbf{Turnaround}
\end{minipage} & \begin{minipage}[b]{\linewidth}\raggedright
\textbf{Waiting}
\end{minipage} \\
\midrule\noalign{}
\endhead
\bottomrule\noalign{}
\endlastfoot
P1 & 0 & 12 & 40 & 40 & 28 \\
P2 & 3 & 4 & 10 & 7 & 3 \\
P3 & 2 & 15 & 46 & 44 & 29 \\
P4 & 5 & 5 & 22 & 17 & 12 \\
\end{longtable}
}

\begin{itemize}
\tightlist
\item
  \textbf{Average Waiting Time}: (28+3+29+12)/4 = 18 ms
\item
  \textbf{Average Turnaround Time}: (40+7+44+17)/4 = 27 ms
\end{itemize}

\end{solutionbox}
\begin{mnemonicbox}
``Round Robin Rotates Regularly''

\end{mnemonicbox}
\begin{center}\rule{0.5\linewidth}{0.5pt}\end{center}

\subsection*{Question 2(a OR) [3
marks]}\label{question-2a-or-3-marks}

\textbf{Differentiate: CPU bound process v/s I/O bound process.}

\begin{solutionbox}


{\def\LTcaptype{none} % do not increment counter
\vspace{-5pt}
\captionof{table}{CPU vs I/O Bound Processes}
\vspace{-10pt}
\begin{longtable}[]{@{}
  >{\raggedright\arraybackslash}p{(\linewidth - 4\tabcolsep) * \real{0.2857}}
  >{\raggedright\arraybackslash}p{(\linewidth - 4\tabcolsep) * \real{0.3571}}
  >{\raggedright\arraybackslash}p{(\linewidth - 4\tabcolsep) * \real{0.3571}}@{}}
\toprule\noalign{}
\begin{minipage}[b]{\linewidth}\raggedright
\textbf{Aspect}
\end{minipage} & \begin{minipage}[b]{\linewidth}\raggedright
\textbf{CPU Bound}
\end{minipage} & \begin{minipage}[b]{\linewidth}\raggedright
\textbf{I/O Bound}
\end{minipage} \\
\midrule\noalign{}
\endhead
\bottomrule\noalign{}
\endlastfoot
\textbf{CPU Usage} & High CPU utilization & Low CPU utilization \\
\textbf{I/O Operations} & Minimal I/O & Frequent I/O \\
\textbf{Examples} & Mathematical calculations & File operations \\
\textbf{Scheduling} & Needs longer time quantum & Benefits from shorter
quantum \\
\textbf{Performance} & Limited by CPU speed & Limited by I/O speed \\
\end{longtable}
}

\end{solutionbox}
\begin{mnemonicbox}
``CPU Computes, I/O Interacts''

\end{mnemonicbox}
\begin{center}\rule{0.5\linewidth}{0.5pt}\end{center}

\subsection*{Question 2(b OR) [4
marks]}\label{question-2b-or-4-marks}

\textbf{Define Critical Section and discuss the general structure of a
critical section solution.}

\begin{solutionbox}

\textbf{Definition}: Critical section is code segment where shared
resources are accessed and must be executed atomically.


{\def\LTcaptype{none} % do not increment counter
\vspace{-5pt}
\captionof{table}{Critical Section Structure}
\vspace{-10pt}
\begin{longtable}[]{@{}ll@{}}
\toprule\noalign{}
\textbf{Section} & \textbf{Purpose} \\
\midrule\noalign{}
\endhead
\bottomrule\noalign{}
\endlastfoot
\textbf{Entry Section} & Request permission to enter critical section \\
\textbf{Critical Section} & Code accessing shared resources \\
\textbf{Exit Section} & Release permission \\
\textbf{Remainder Section} & Other code not accessing shared
resources \\
\end{longtable}
}

\textbf{Solution Requirements:}

\begin{itemize}
\tightlist
\item
  \textbf{Mutual Exclusion}: Only one process in critical section
\item
  \textbf{Progress}: Selection of next process cannot be postponed
  indefinitely
\item
  \textbf{Bounded Waiting}: Limit on waiting time
\end{itemize}

\end{solutionbox}
\begin{mnemonicbox}
``Enter Critical Exit Remainder''

\end{mnemonicbox}
\begin{center}\rule{0.5\linewidth}{0.5pt}\end{center}

\subsection*{Question 2(c OR) [7
marks]}\label{question-2c-or-7-marks}

\textbf{Describe the SJF algorithm. Calculate the average waiting time
and average turn-around time along with Gantt chart for the given data.}

\begin{solutionbox}

\textbf{SJF Algorithm:}

\begin{itemize}
\tightlist
\item
  \textbf{Shortest Job First}: Process with smallest burst time
  scheduled first
\item
  \textbf{Non-preemptive}: Process runs to completion
\item
  \textbf{Optimal}: Minimizes average waiting time
\end{itemize}

\textbf{Execution Order}: P2(4), P4(5), P1(8), P3(9)

\textbf{Gantt Chart:}

\begin{verbatim}
|  P1  |  P2  |  P4  |     P3     |
0      8     12     17          26
\end{verbatim}

\textbf{Calculations Table:}

{\def\LTcaptype{none} % do not increment counter
\begin{longtable}[]{@{}
  >{\raggedright\arraybackslash}p{(\linewidth - 12\tabcolsep) * \real{0.1398}}
  >{\raggedright\arraybackslash}p{(\linewidth - 12\tabcolsep) * \real{0.1398}}
  >{\raggedright\arraybackslash}p{(\linewidth - 12\tabcolsep) * \real{0.1183}}
  >{\raggedright\arraybackslash}p{(\linewidth - 12\tabcolsep) * \real{0.1183}}
  >{\raggedright\arraybackslash}p{(\linewidth - 12\tabcolsep) * \real{0.1720}}
  >{\raggedright\arraybackslash}p{(\linewidth - 12\tabcolsep) * \real{0.1720}}
  >{\raggedright\arraybackslash}p{(\linewidth - 12\tabcolsep) * \real{0.1398}}@{}}
\toprule\noalign{}
\begin{minipage}[b]{\linewidth}\raggedright
\textbf{Process}
\end{minipage} & \begin{minipage}[b]{\linewidth}\raggedright
\textbf{Arrival}
\end{minipage} & \begin{minipage}[b]{\linewidth}\raggedright
\textbf{Burst}
\end{minipage} & \begin{minipage}[b]{\linewidth}\raggedright
\textbf{Start}
\end{minipage} & \begin{minipage}[b]{\linewidth}\raggedright
\textbf{Completion}
\end{minipage} & \begin{minipage}[b]{\linewidth}\raggedright
\textbf{Turnaround}
\end{minipage} & \begin{minipage}[b]{\linewidth}\raggedright
\textbf{Waiting}
\end{minipage} \\
\midrule\noalign{}
\endhead
\bottomrule\noalign{}
\endlastfoot
P1 & 0 & 8 & 0 & 8 & 8 & 0 \\
P2 & 3 & 4 & 8 & 12 & 9 & 5 \\
P3 & 5 & 9 & 17 & 26 & 21 & 12 \\
P4 & 6 & 5 & 12 & 17 & 11 & 6 \\
\end{longtable}
}

\begin{itemize}
\tightlist
\item
  \textbf{Average Waiting Time}: (0+5+12+6)/4 = 5.75 ms
\item
  \textbf{Average Turnaround Time}: (8+9+21+11)/4 = 12.25 ms
\end{itemize}

\end{solutionbox}
\begin{mnemonicbox}
``Shortest Jobs Start Soon''

\end{mnemonicbox}
\begin{center}\rule{0.5\linewidth}{0.5pt}\end{center}

\subsection*{Question 3(a) [3 marks]}\label{q3a}

\textbf{Explain two-level directory structure.}

\begin{solutionbox}

\textbf{Diagram:}

\begin{verbatim}
    Master File Directory (MFD)
           |
    +{-{-}{-}{-}{-}{-}+{-}{-}{-}{-}{-}{-}+}
    |             |
   User1         User2
 Directory      Directory
    |             |
  File1         File3
  File2         File4
\end{verbatim}

\begin{itemize}
\tightlist
\item
  \textbf{Master File Directory}: Contains entries for each user
\item
  \textbf{User File Directory}: Separate directory for each user's files
\item
  \textbf{Path Structure}: /user/filename
\item
  \textbf{Advantages}: Solves naming conflicts, provides user isolation
\end{itemize}

\end{solutionbox}
\begin{mnemonicbox}
``Two Tiers Tackle Troubles''

\end{mnemonicbox}
\begin{center}\rule{0.5\linewidth}{0.5pt}\end{center}

\subsection*{Question 3(b) [4 marks]}\label{q3b}

\textbf{Explain the different file operations.}

\begin{solutionbox}


{\def\LTcaptype{none} % do not increment counter
\vspace{-5pt}
\captionof{table}{File Operations}
\vspace{-10pt}
\begin{longtable}[]{@{}lll@{}}
\toprule\noalign{}
\textbf{Operation} & \textbf{Purpose} & \textbf{Example} \\
\midrule\noalign{}
\endhead
\bottomrule\noalign{}
\endlastfoot
\textbf{Create} & Make new file & touch file.txt \\
\textbf{Open} & Access file for operations & fopen() \\
\textbf{Read} & Retrieve data from file & fread() \\
\textbf{Write} & Store data to file & fwrite() \\
\textbf{Close} & Terminate file access & fclose() \\
\textbf{Delete} & Remove file & rm file.txt \\
\end{longtable}
}

\end{solutionbox}
\begin{mnemonicbox}
``Create Open Read Write Close Delete''

\end{mnemonicbox}
\begin{center}\rule{0.5\linewidth}{0.5pt}\end{center}

\subsection*{Question 3(c) [7 marks]}\label{q3c}

\textbf{List the different file allocation methods and explain
contiguous allocation with necessary diagram.}

\begin{solutionbox}

\textbf{File Allocation Methods:}

\begin{itemize}
\tightlist
\item
  \textbf{Contiguous Allocation}
\item
  \textbf{Linked Allocation}
\item
  \textbf{Indexed Allocation}
\end{itemize}

\textbf{Contiguous Allocation:}

\textbf{Diagram:}

\begin{verbatim}
File A: |Block1|Block2|Block3|
File B: |Block4|Block5|
File C: |Block6|Block7|Block8|Block9|
\end{verbatim}


{\def\LTcaptype{none} % do not increment counter
\vspace{-5pt}
\captionof{table}{Contiguous Allocation}
\vspace{-10pt}
\begin{longtable}[]{@{}ll@{}}
\toprule\noalign{}
\textbf{Aspect} & \textbf{Description} \\
\midrule\noalign{}
\endhead
\bottomrule\noalign{}
\endlastfoot
\textbf{Storage} & Files stored in consecutive blocks \\
\textbf{Access} & Direct access to any block \\
\textbf{Advantages} & Fast access, simple implementation \\
\textbf{Disadvantages} & External fragmentation, difficult expansion \\
\end{longtable}
}

\textbf{Directory Entry}: (Start block, Length)

\end{solutionbox}
\begin{mnemonicbox}
``Contiguous Creates Continuous Clusters''

\end{mnemonicbox}
\begin{center}\rule{0.5\linewidth}{0.5pt}\end{center}

\subsection*{Question 3(a OR) [3
marks]}\label{question-3a-or-3-marks}

\textbf{Describe the types of file structures.}

\begin{solutionbox}


{\def\LTcaptype{none} % do not increment counter
\vspace{-5pt}
\captionof{table}{File Structure Types}
\vspace{-10pt}
\begin{longtable}[]{@{}lll@{}}
\toprule\noalign{}
\textbf{Type} & \textbf{Organization} & \textbf{Access} \\
\midrule\noalign{}
\endhead
\bottomrule\noalign{}
\endlastfoot
\textbf{Sequential} & Records in order & Sequential only \\
\textbf{Direct/Random} & Records by key & Direct access \\
\textbf{Indexed} & Index points to records & Key-based access \\
\textbf{Hierarchical} & Tree structure & Path-based \\
\end{longtable}
}

\end{solutionbox}
\begin{mnemonicbox}
``Sequential Direct Indexed Hierarchical''

\end{mnemonicbox}
\begin{center}\rule{0.5\linewidth}{0.5pt}\end{center}

\subsection*{Question 3(b OR) [4
marks]}\label{question-3b-or-4-marks}

\textbf{Explain the different file attributes.}

\begin{solutionbox}


{\def\LTcaptype{none} % do not increment counter
\vspace{-5pt}
\captionof{table}{File Attributes}
\vspace{-10pt}
\begin{longtable}[]{@{}lll@{}}
\toprule\noalign{}
\textbf{Attribute} & \textbf{Description} & \textbf{Example} \\
\midrule\noalign{}
\endhead
\bottomrule\noalign{}
\endlastfoot
\textbf{Name} & File identifier & document.txt \\
\textbf{Type} & File format & .txt, .exe \\
\textbf{Size} & File length in bytes & 1024 bytes \\
\textbf{Location} & Physical storage address & Block 150 \\
\textbf{Permissions} & Access rights & rwx-rwx-rwx \\
\textbf{Timestamps} & Creation, modification dates & 2023-01-16 \\
\end{longtable}
}

\end{solutionbox}
\begin{mnemonicbox}
``Name Type Size Location Permissions Time''

\end{mnemonicbox}
\begin{center}\rule{0.5\linewidth}{0.5pt}\end{center}

\subsection*{Question 3(c OR) [7
marks]}\label{question-3c-or-7-marks}

\textbf{List the different file allocation methods and explain linked
allocation with necessary diagram.}

\begin{solutionbox}

\textbf{File Allocation Methods:} Same as previous answer.

\textbf{Linked Allocation:}

\textbf{Diagram:}

\begin{verbatim}
File A: Block1  Block5  Block9  NULL
File B: Block2  Block7  NULL  
File C: Block3  Block4  Block8  NULL
\end{verbatim}


{\def\LTcaptype{none} % do not increment counter
\vspace{-5pt}
\captionof{table}{Linked Allocation}
\vspace{-10pt}
\begin{longtable}[]{@{}ll@{}}
\toprule\noalign{}
\textbf{Aspect} & \textbf{Description} \\
\midrule\noalign{}
\endhead
\bottomrule\noalign{}
\endlastfoot
\textbf{Storage} & Files stored in linked blocks \\
\textbf{Pointers} & Each block contains pointer to next \\
\textbf{Advantages} & No external fragmentation, dynamic size \\
\textbf{Disadvantages} & Sequential access only, pointer overhead \\
\end{longtable}
}

\textbf{Directory Entry}: (Start block pointer)

\end{solutionbox}
\begin{mnemonicbox}
``Links Lead Logical Locations''

\end{mnemonicbox}
\begin{center}\rule{0.5\linewidth}{0.5pt}\end{center}

\subsection*{Question 4(a) [3 marks]}\label{q4a}

\textbf{Define Program threats and explain its types.}

\begin{solutionbox}

\textbf{Definition}: Program threats are malicious programs that
compromise system security and integrity.


{\def\LTcaptype{none} % do not increment counter
\vspace{-5pt}
\captionof{table}{Program Threat Types}
\vspace{-10pt}
\begin{longtable}[]{@{}ll@{}}
\toprule\noalign{}
\textbf{Type} & \textbf{Description} \\
\midrule\noalign{}
\endhead
\bottomrule\noalign{}
\endlastfoot
\textbf{Trojan Horse} & Hidden malicious code in legitimate program \\
\textbf{Virus} & Self-replicating code that infects other programs \\
\textbf{Worm} & Standalone program that replicates across networks \\
\textbf{Logic Bomb} & Code triggered by specific conditions \\
\end{longtable}
}

\end{solutionbox}
\begin{mnemonicbox}
``Trojans Viruses Worms Logic-bombs''

\end{mnemonicbox}
\begin{center}\rule{0.5\linewidth}{0.5pt}\end{center}

\subsection*{Question 4(b) [4 marks]}\label{q4b}

\textbf{Explain System Authentication.}

\begin{solutionbox}

\textbf{Definition}: Process of verifying user identity before granting
system access.


{\def\LTcaptype{none} % do not increment counter
\vspace{-5pt}
\captionof{table}{Authentication Methods}
\vspace{-10pt}
\begin{longtable}[]{@{}lll@{}}
\toprule\noalign{}
\textbf{Method} & \textbf{Description} & \textbf{Example} \\
\midrule\noalign{}
\endhead
\bottomrule\noalign{}
\endlastfoot
\textbf{Password} & Secret text string & username/password \\
\textbf{Biometric} & Physical characteristics & Fingerprint, retina \\
\textbf{Token} & Physical device & Smart card, USB key \\
\textbf{Multi-factor} & Combination of methods & Password + OTP \\
\end{longtable}
}

\textbf{Authentication Process:}

\begin{itemize}
\tightlist
\item
  \textbf{Identification}: User claims identity
\item
  \textbf{Verification}: System validates claim
\item
  \textbf{Authorization}: Access rights granted
\end{itemize}

\end{solutionbox}
\begin{mnemonicbox}
``Passwords Biometrics Tokens Multi-factor''

\end{mnemonicbox}
\begin{center}\rule{0.5\linewidth}{0.5pt}\end{center}

\subsection*{Question 4(c) [7 marks]}\label{q4c}

\textbf{Explain Access Control List in detail.}

\begin{solutionbox}

\textbf{Definition}: ACL specifies permissions for each user/group on
system resources.


{\def\LTcaptype{none} % do not increment counter
\vspace{-5pt}
\captionof{table}{ACL Components}
\vspace{-10pt}
\begin{longtable}[]{@{}lll@{}}
\toprule\noalign{}
\textbf{Component} & \textbf{Purpose} & \textbf{Example} \\
\midrule\noalign{}
\endhead
\bottomrule\noalign{}
\endlastfoot
\textbf{Subject} & User or group & john, admin\_group \\
\textbf{Object} & Resource & file.txt, directory \\
\textbf{Permission} & Allowed operations & read, write, execute \\
\textbf{Action} & Allow or deny & permit, deny \\
\end{longtable}
}

\textbf{ACL Structure:}

\begin{verbatim}
User: john    File: /etc/passwd    Permission: read    Action: allow
Group: users  File: /tmp/*        Permission: write   Action: allow
User: guest   File: /etc/*        Permission: write   Action: deny
\end{verbatim}

\textbf{Advantages:}

\begin{itemize}
\tightlist
\item
  \textbf{Granular Control}: Fine-grained permissions
\item
  \textbf{Flexibility}: Per-resource access control
\item
  \textbf{Scalability}: Handles complex organizations
\end{itemize}

\end{solutionbox}
\begin{mnemonicbox}
``Access Controls Limit Users''

\end{mnemonicbox}
\begin{center}\rule{0.5\linewidth}{0.5pt}\end{center}

\subsection*{Question 4(a OR) [3
marks]}\label{question-4a-or-3-marks}

\textbf{Define System threats and explain its types.}

\begin{solutionbox}

\textbf{Definition}: System threats target operating system components
and system integrity.


{\def\LTcaptype{none} % do not increment counter
\vspace{-5pt}
\captionof{table}{System Threat Types}
\vspace{-10pt}
\begin{longtable}[]{@{}ll@{}}
\toprule\noalign{}
\textbf{Type} & \textbf{Description} \\
\midrule\noalign{}
\endhead
\bottomrule\noalign{}
\endlastfoot
\textbf{Denial of Service} & Overwhelm system resources \\
\textbf{Privilege Escalation} & Gain unauthorized higher privileges \\
\textbf{Buffer Overflow} & Exploit memory management flaws \\
\textbf{Rootkit} & Hide malicious activities from detection \\
\end{longtable}
}

\end{solutionbox}
\begin{mnemonicbox}
``Denial Privilege Buffer Rootkit''

\end{mnemonicbox}
\begin{center}\rule{0.5\linewidth}{0.5pt}\end{center}

\subsection*{Question 4(b OR) [4
marks]}\label{question-4b-or-4-marks}

\textbf{Discuss the needs and goals of protection in OS.}

\begin{solutionbox}


{\def\LTcaptype{none} % do not increment counter
\vspace{-5pt}
\captionof{table}{Protection Needs and Goals}
\vspace{-10pt}
\begin{longtable}[]{@{}lll@{}}
\toprule\noalign{}
\textbf{Need} & \textbf{Goal} & \textbf{Implementation} \\
\midrule\noalign{}
\endhead
\bottomrule\noalign{}
\endlastfoot
\textbf{Confidentiality} & Prevent unauthorized access & Access
controls \\
\textbf{Integrity} & Maintain data accuracy & Checksums, validation \\
\textbf{Availability} & Ensure resource access & Redundancy, backup \\
\textbf{Authentication} & Verify user identity & Login mechanisms \\
\end{longtable}
}

\textbf{Protection Mechanisms:}

\begin{itemize}
\tightlist
\item
  \textbf{Access Control}: Limit resource access
\item
  \textbf{Capability Lists}: Define user permissions
\item
  \textbf{Security Domains}: Isolate processes
\end{itemize}

\end{solutionbox}
\begin{mnemonicbox}
``Confidentiality Integrity Availability
Authentication''

\end{mnemonicbox}
\begin{center}\rule{0.5\linewidth}{0.5pt}\end{center}

\subsection*{Question 4(c OR) [7
marks]}\label{question-4c-or-7-marks}

\textbf{Discuss various operating system security policies and
procedures.}

\begin{solutionbox}


{\def\LTcaptype{none} % do not increment counter
\vspace{-5pt}
\captionof{table}{Security Policies and Procedures}
\vspace{-10pt}
\begin{longtable}[]{@{}
  >{\raggedright\arraybackslash}p{(\linewidth - 4\tabcolsep) * \real{0.3542}}
  >{\raggedright\arraybackslash}p{(\linewidth - 4\tabcolsep) * \real{0.3333}}
  >{\raggedright\arraybackslash}p{(\linewidth - 4\tabcolsep) * \real{0.3125}}@{}}
\toprule\noalign{}
\begin{minipage}[b]{\linewidth}\raggedright
\textbf{Policy Type}
\end{minipage} & \begin{minipage}[b]{\linewidth}\raggedright
\textbf{Description}
\end{minipage} & \begin{minipage}[b]{\linewidth}\raggedright
\textbf{Procedure}
\end{minipage} \\
\midrule\noalign{}
\endhead
\bottomrule\noalign{}
\endlastfoot
\textbf{Access Control} & Define user permissions & Regular audit,
role-based access \\
\textbf{Password Policy} & Password requirements & Complexity rules,
expiration \\
\textbf{Backup Policy} & Data protection strategy & Regular backups,
testing \\
\textbf{Incident Response} & Security breach handling & Detection,
containment, recovery \\
\end{longtable}
}

\textbf{Security Procedures:}

\begin{itemize}
\tightlist
\item
  \textbf{Regular Updates}: Patch management
\item
  \textbf{Monitoring}: Log analysis, intrusion detection\\
\item
  \textbf{Training}: User security awareness
\item
  \textbf{Audit}: Compliance checking
\end{itemize}

\end{solutionbox}
\begin{mnemonicbox}
``Access Password Backup Incident''

\end{mnemonicbox}
\begin{center}\rule{0.5\linewidth}{0.5pt}\end{center}

\subsection*{Question 5(a) [3 marks]}\label{q5a}

\textbf{Explain the following commands: (i) pwd (ii) cd (iii) comm}

\begin{solutionbox}


{\def\LTcaptype{none} % do not increment counter
\vspace{-5pt}
\captionof{table}{Linux Commands}
\vspace{-10pt}
\begin{longtable}[]{@{}lll@{}}
\toprule\noalign{}
\textbf{Command} & \textbf{Purpose} & \textbf{Example} \\
\midrule\noalign{}
\endhead
\bottomrule\noalign{}
\endlastfoot
\textbf{pwd} & Print working directory & pwd \rightarrow /home/user \\
\textbf{cd} & Change directory & cd /tmp \\
\textbf{comm} & Compare sorted files & comm file1.txt file2.txt \\
\end{longtable}
}

\begin{itemize}
\tightlist
\item
  \textbf{pwd}: Shows current directory path
\item
  \textbf{cd}: Navigate between directories
\item
  \textbf{comm}: Displays common and unique lines between files
\end{itemize}

\end{solutionbox}
\begin{mnemonicbox}
``Print Working Directory, Change Directory, Compare
Common''

\end{mnemonicbox}
\begin{center}\rule{0.5\linewidth}{0.5pt}\end{center}

\subsection*{Question 5(b) [4 marks]}\label{q5b}

\textbf{Write a shell script to concatenate the contents of two files in
a third file.}

\begin{solutionbox}

\textbf{Shell Script:}

\begin{verbatim}
\#!/bin/bash
\# Script to concatenate two files into third file

echo "Enter first file name:"
read file1
echo "Enter second file name:" 
read file2
echo "Enter output file name:"
read file3

\# Check if input files exist
if [ {-f} "$file1" ] \&\& [ {-f} "$file2" ]; then
    cat "$file1" "$file2" {} "$file3"
    echo "Files concatenated successfully into $file3"
else
    echo "Error: Input files not found"
fi
\end{verbatim}

\end{solutionbox}
\begin{mnemonicbox}
``Cat Combines Content Correctly''

\end{mnemonicbox}
\begin{center}\rule{0.5\linewidth}{0.5pt}\end{center}

\subsection*{Question 5(c) [7 marks]}\label{q5c}

\textbf{Write a shell script to find the sum of all the individual
digits in a given 5 digit number.}

\begin{solutionbox}

\textbf{Shell Script:}

\begin{verbatim}
\#!/bin/bash
\# Script to find sum of digits in 5{-digit number}

echo "Enter a 5{-digit number:"}
read number

\# Validate input
if [ $\{\#number\} {-ne} 5 ]; then
    echo "Error: Please enter exactly 5 digits"
    exit 1
fi

sum=0
temp=$number

\# Extract and sum each digit
while [ $temp {-gt} 0 ]; do
    digit=$(($temp \% 10))
    sum=$(($sum + $digit))
    temp=$(($temp / 10))
done

echo "Sum of digits in $number is: $sum"
\end{verbatim}

\textbf{Algorithm:}

\begin{itemize}
\tightlist
\item
  \textbf{Input Validation}: Check for 5-digit number
\item
  \textbf{Digit Extraction}: Use modulo operation
\item
  \textbf{Sum Calculation}: Add each digit
\item
  \textbf{Display Result}: Show final sum
\end{itemize}

\end{solutionbox}
\begin{mnemonicbox}
``Sum Separates Single Symbols''

\end{mnemonicbox}
\begin{center}\rule{0.5\linewidth}{0.5pt}\end{center}

\subsection*{Question 5(a OR) [3
marks]}\label{question-5a-or-3-marks}

\textbf{Explain the following commands: (i) man (ii) mkdir (iii) grep}

\begin{solutionbox}


{\def\LTcaptype{none} % do not increment counter
\vspace{-5pt}
\captionof{table}{Linux Commands}
\vspace{-10pt}
\begin{longtable}[]{@{}lll@{}}
\toprule\noalign{}
\textbf{Command} & \textbf{Purpose} & \textbf{Example} \\
\midrule\noalign{}
\endhead
\bottomrule\noalign{}
\endlastfoot
\textbf{man} & Display manual pages & man ls \\
\textbf{mkdir} & Create directories & mkdir newdir \\
\textbf{grep} & Search text patterns & grep ``hello'' file.txt \\
\end{longtable}
}

\begin{itemize}
\tightlist
\item
  \textbf{man}: Provides documentation for commands
\item
  \textbf{mkdir}: Creates new directories with specified names
\item
  \textbf{grep}: Searches for patterns in files using regular
  expressions
\end{itemize}

\end{solutionbox}
\begin{mnemonicbox}
``Manual Make Directories, Grep Examines Patterns''

\end{mnemonicbox}
\begin{center}\rule{0.5\linewidth}{0.5pt}\end{center}

\subsection*{Question 5(b OR) [4
marks]}\label{question-5b-or-4-marks}

\textbf{Write a shell script to generate and display Fibonacci series.}

\begin{solutionbox}

\textbf{Shell Script:}

\begin{verbatim}
\#!/bin/bash
\# Script to generate Fibonacci series

echo "Enter number of terms:"
read n

\# Validate input
if [ $n {-le} 0 ]; then
    echo "Error: Please enter positive number"
    exit 1
fi

\# Initialize first two terms
a=0
b=1

echo "Fibonacci Series:"
echo {-n} "$a "

if [ $n {-gt} 1 ]; then
    echo {-n} "$b "
fi

\# Generate remaining terms
for ((i=3; i{=}n; i++)); do
    c=$(($a + $b))
    echo {-n} "$c "
    a=$b
    b=$c
done
echo
\end{verbatim}

\end{solutionbox}
\begin{mnemonicbox}
``Fibonacci Follows Forward Formula''

\end{mnemonicbox}
\begin{center}\rule{0.5\linewidth}{0.5pt}\end{center}

\subsection*{Question 5(c OR) [7
marks]}\label{question-5c-or-7-marks}

\textbf{Write a shell script to determine whether a given string is
palindrome.}

\begin{solutionbox}

\textbf{Shell Script:}

\begin{verbatim}
\#!/bin/bash
\# Script to check if string is palindrome

echo "Enter a string:"
read string

\# Convert to lowercase and remove spaces
clean\_string=$(echo "$string" | tr {[:upper:]} {[:lower:]} | tr {-d} { })

\# Get string length
length=$\{\#clean\_string\}

\# Initialize flag
is\_palindrome=true

\# Check palindrome
for ((i=0; i{}length/2; i++)); do
    if [ "$\{clean\_string:$i:1\}" != "$\{clean\_string:$((length{-}1{-}i)):1\}" ]; then
        is\_palindrome=false
        break
    fi
done

\# Display result
if [ "$is\_palindrome" = true ]; then
    echo "{}$string{ is a palindrome"}
else
    echo "{}$string{ is not a palindrome"}
fi
\end{verbatim}

\textbf{Algorithm:}

\begin{itemize}
\tightlist
\item
  \textbf{String Cleaning}: Convert to lowercase, remove spaces
\item
  \textbf{Character Comparison}: Compare characters from both ends
\item
  \textbf{Palindrome Check}: Verify if all comparisons match
\end{itemize}

\end{solutionbox}
\begin{mnemonicbox}
``Palindromes Proceed Perfectly Parallel''

\end{mnemonicbox}

\end{document}
