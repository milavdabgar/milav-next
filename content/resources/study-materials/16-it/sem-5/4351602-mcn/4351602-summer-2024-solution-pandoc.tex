\documentclass[10pt,a4paper]{article}

% content/resources/templates/preamble.tex
\usepackage[margin=0.6in]{geometry}
\author{Milav Dabgar}
\usepackage{amsmath,amssymb,amsthm}
\usepackage{booktabs}
\usepackage{multirow}
\usepackage{xcolor}
\usepackage{tcolorbox}
\tcbuselibrary{breakable,skins}
\usepackage[colorlinks=true,linkcolor=blue]{hyperref}
\usepackage{titlesec}
\usepackage{enumitem}
\usepackage{tikz}
\usepackage{pgfplots}
\usepackage{circuitikz}
\usepackage[version=4]{mhchem}
\usepackage{longtable}
\usepackage{array}
\usepackage{float}
\usepackage{caption}
\usepackage{listings}

\lstset{
  basicstyle=\small\ttfamily,
  breaklines=true,
  breakatwhitespace=false,
  postbreak=\mbox{\textcolor{red}{$\hookrightarrow$}\space},
  float=false,
  numbers=left,
  numberstyle=\tiny\color{gray},
  numbersep=10pt,
  xleftmargin=2em,
  keywordstyle=\color{blue},
  commentstyle=\color{green!60!black},
  stringstyle=\color{purple},
  backgroundcolor=\color{gray!5},
  showstringspaces=false,
  tabsize=2,
  captionpos=b,
  keepspaces=true,
  columns=flexible
}

\pgfplotsset{compat=1.18}
\usetikzlibrary{shapes,arrows,positioning,calc,patterns,decorations.pathmorphing,decorations.markings,arrows.meta}

% Color scheme
\definecolor{headcolor}{RGB}{0,102,204}
\definecolor{keycolor}{RGB}{220,20,60}
\definecolor{solutioncolor}{RGB}{34,139,34}
\definecolor{mnemoniccolor}{RGB}{148,0,211}
\definecolor{codecolor}{RGB}{0,0,100}

% Spacing
\setlength{\parskip}{3pt}
\setlist[itemize]{nosep}
\setlist[enumerate]{nosep}

% Title formatting
\titleformat{\section}{\Large\bfseries\color{headcolor}}{\thesection}{1em}{}
\titleformat{\subsection}{\large\bfseries\color{headcolor}}{\thesubsection}{1em}{}

% Pandoc tightlist compatibility
\providecommand{\tightlist}{%
  \setlength{\itemsep}{0pt}\setlength{\parskip}{0pt}}

% Pandoc longtable compatibility
\newcounter{none}
\def\thenone{}


% content/resources/templates/english-boxes.tex
% This file is currently empty - it exists to maintain consistency with the import structure.
% Add custom environments here if needed in the future.


\begin{document}

\begin{center}
{\Huge\bfseries\color{headcolor} Subject Name Solutions}\\[5pt]
{\LARGE 4351602 -- Summer 2024}\\[3pt]
{\large Semester 1 Study Material}\\[3pt]
{\normalsize\textit{Detailed Solutions and Explanations}}
\end{center}

\vspace{10pt}

\subsection*{Question 1(a) [3 marks]}\label{q1a}

\textbf{Define Peer to Peer network}

\begin{solutionbox}
A Peer-to-Peer (P2P) network is a distributed network
architecture where each node (peer) acts as both client and server,
sharing resources directly without centralized control.


{\def\LTcaptype{none} % do not increment counter
\begin{longtable}[]{@{}ll@{}}
\toprule\noalign{}
Aspect & Description \\
\midrule\noalign{}
\endhead
\bottomrule\noalign{}
\endlastfoot
\textbf{Structure} & Decentralized network \\
\textbf{Role} & Each peer is client and server \\
\textbf{Control} & No central authority \\
\textbf{Examples} & BitTorrent, Skype \\
\end{longtable}
}

\end{solutionbox}
\begin{mnemonicbox}
``Peers Share Equally''

\end{mnemonicbox}
\begin{center}\rule{0.5\linewidth}{0.5pt}\end{center}

\subsection*{Question 1(b) [4 marks]}\label{q1b}

\textbf{Compare SMTP, POP and IMAP}

\begin{solutionbox}
Email protocols serve different purposes in email
communication system.


{\def\LTcaptype{none} % do not increment counter
\begin{longtable}[]{@{}llll@{}}
\toprule\noalign{}
Feature & SMTP & POP3 & IMAP \\
\midrule\noalign{}
\endhead
\bottomrule\noalign{}
\endlastfoot
\textbf{Purpose} & Send emails & Download emails & Access emails \\
\textbf{Port} & 25, 587 & 110, 995 & 143, 993 \\
\textbf{Storage} & Server forwards & Local storage & Server storage \\
\textbf{Access} & One-way sending & Single device & Multiple devices \\
\end{longtable}
}

\end{solutionbox}
\begin{mnemonicbox}
``Send-Pop-Internet Mail Access''

\end{mnemonicbox}
\begin{center}\rule{0.5\linewidth}{0.5pt}\end{center}

\subsection*{Question 1(c) [7 marks]}\label{q1c}

\textbf{Illustrate OSI model with responsibilities of each layer}

\begin{solutionbox}
The OSI (Open Systems Interconnection) model has seven
layers, each with specific responsibilities for network communication.

\textbf{Diagram:}

\begin{center}
\textbf{Mermaid Diagram (Code)}
\begin{verbatim}
{Shaded}
{Highlighting}[]
graph LR
    A[Application Layer 7] {-{-}{} B[Presentation Layer 6]}
    B {-{-}{} C[Session Layer 5]}
    C {-{-}{} D[Transport Layer 4]}
    D {-{-}{} E[Network Layer 3]}
    E {-{-}{} F[Data Link Layer 2]}
    F {-{-}{} G[Physical Layer 1]}
{Highlighting}
{Shaded}
\end{verbatim}
\end{center}


{\def\LTcaptype{none} % do not increment counter
\begin{longtable}[]{@{}lll@{}}
\toprule\noalign{}
Layer & Name & Responsibilities \\
\midrule\noalign{}
\endhead
\bottomrule\noalign{}
\endlastfoot
\textbf{7} & Application & User interface, network services \\
\textbf{6} & Presentation & Data encryption, compression \\
\textbf{5} & Session & Session management, dialogue control \\
\textbf{4} & Transport & End-to-end delivery, error control \\
\textbf{3} & Network & Routing, logical addressing \\
\textbf{2} & Data Link & Frame formatting, error detection \\
\textbf{1} & Physical & Bit transmission, hardware \\
\end{longtable}
}

\textbf{Key Points:}

\begin{itemize}
\tightlist
\item
  \textbf{Application Layer}: Provides network services to applications
\item
  \textbf{Transport Layer}: Ensures reliable data delivery
\item
  \textbf{Network Layer}: Handles routing between networks
\end{itemize}

\end{solutionbox}
\begin{mnemonicbox}
``All People Seem To Need Data Processing''

\end{mnemonicbox}
\begin{center}\rule{0.5\linewidth}{0.5pt}\end{center}

\subsection*{Question 1(c OR) [7
marks]}\label{question-1c-or-7-marks}

\textbf{Compare the TCP/IP model with OSI model}

\begin{solutionbox}
TCP/IP and OSI models are network architecture
frameworks with different layer structures.

\textbf{Diagram:}

\begin{center}
\textbf{Mermaid Diagram (Code)}
\begin{verbatim}
{Shaded}
{Highlighting}[]
graph LR
    subgraph "OSI Model"
        O1[Application]
        O2[Presentation]
        O3[Session]
        O4[Transport]
        O5[Network]
        O6[Data Link]
        O7[Physical]
    end
    
    subgraph "TCP/IP Model"
        T1[Application]
        T2[Transport]
        T3[Internet]
        T4[Network Access]
    end
{Highlighting}
{Shaded}
\end{verbatim}
\end{center}


{\def\LTcaptype{none} % do not increment counter
\begin{longtable}[]{@{}lll@{}}
\toprule\noalign{}
Aspect & OSI Model & TCP/IP Model \\
\midrule\noalign{}
\endhead
\bottomrule\noalign{}
\endlastfoot
\textbf{Layers} & 7 layers & 4 layers \\
\textbf{Development} & Theoretical & Practical \\
\textbf{Usage} & Reference model & Internet standard \\
\textbf{Complexity} & More detailed & Simplified \\
\end{longtable}
}

\textbf{Key Points:}

\begin{itemize}
\tightlist
\item
  \textbf{OSI}: Theoretical framework with detailed separation
\item
  \textbf{TCP/IP}: Practical implementation for internet
\item
  \textbf{Mapping}: Top 3 OSI layers = Application layer in TCP/IP
\end{itemize}

\end{solutionbox}
\begin{mnemonicbox}
``OSI Seven, TCP Four''

\end{mnemonicbox}
\begin{center}\rule{0.5\linewidth}{0.5pt}\end{center}

\subsection*{Question 2(a) [3 marks]}\label{q2a}

\textbf{Explain Network Address Translation (NAT)}

\begin{solutionbox}
NAT translates private IP addresses to public IP
addresses, enabling multiple devices to share a single public IP.

\textbf{Diagram:}

\begin{verbatim}
Private Network    NAT Router    Internet
192.168.1.10  {-{-}  203.0.113.1  {-}{-}  Server}
192.168.1.20  {-{-}  203.0.113.1  {-}{-}  Server}
192.168.1.30  {-{-}  203.0.113.1  {-}{-}  Server}
\end{verbatim}

\textbf{Key Points:}

\begin{itemize}
\tightlist
\item
  \textbf{Purpose}: IP address translation between networks
\item
  \textbf{Benefit}: Conserves public IP addresses
\item
  \textbf{Security}: Hides internal network structure
\end{itemize}

\end{solutionbox}
\begin{mnemonicbox}
``Network Address Translation''

\end{mnemonicbox}
\begin{center}\rule{0.5\linewidth}{0.5pt}\end{center}

\subsection*{Question 2(b) [4 marks]}\label{q2b}

\textbf{Define Subnetting and Supernetting}

\begin{solutionbox}
Subnetting and Supernetting are IP addressing
techniques for efficient network management.


{\def\LTcaptype{none} % do not increment counter
\begin{longtable}[]{@{}
  >{\raggedright\arraybackslash}p{(\linewidth - 4\tabcolsep) * \real{0.3438}}
  >{\raggedright\arraybackslash}p{(\linewidth - 4\tabcolsep) * \real{0.3750}}
  >{\raggedright\arraybackslash}p{(\linewidth - 4\tabcolsep) * \real{0.2812}}@{}}
\toprule\noalign{}
\begin{minipage}[b]{\linewidth}\raggedright
Technique
\end{minipage} & \begin{minipage}[b]{\linewidth}\raggedright
Definition
\end{minipage} & \begin{minipage}[b]{\linewidth}\raggedright
Purpose
\end{minipage} \\
\midrule\noalign{}
\endhead
\bottomrule\noalign{}
\endlastfoot
\textbf{Subnetting} & Dividing network into smaller subnets & Better
organization \\
\textbf{Supernetting} & Combining multiple networks & Route
aggregation \\
\end{longtable}
}

\textbf{Key Points:}

\begin{itemize}
\tightlist
\item
  \textbf{Subnetting}: Increases network bits, reduces host bits
\item
  \textbf{Supernetting}: Decreases network bits, increases routing
  efficiency
\item
  \textbf{CIDR}: Classless Inter-Domain Routing enables both
\end{itemize}

\end{solutionbox}
\begin{mnemonicbox}
``Sub-divides, Super-combines''

\end{mnemonicbox}
\begin{center}\rule{0.5\linewidth}{0.5pt}\end{center}

\subsection*{Question 2(c) [7 marks]}\label{q2c}

\textbf{Demonstrate Classful and Classless notation addressing scheme of
IPv4}

\begin{solutionbox}
IPv4 addressing uses classful and classless schemes for
network identification.

\textbf{Table - Classful Addressing:}

{\def\LTcaptype{none} % do not increment counter
\begin{longtable}[]{@{}lllll@{}}
\toprule\noalign{}
Class & Range & Default Mask & Networks & Hosts \\
\midrule\noalign{}
\endhead
\bottomrule\noalign{}
\endlastfoot
\textbf{A} & 1-126 & /8 (255.0.0.0) & 126 & 16M \\
\textbf{B} & 128-191 & /16 (255.255.0.0) & 16K & 65K \\
\textbf{C} & 192-223 & /24 (255.255.255.0) & 2M & 254 \\
\end{longtable}
}

\textbf{Classless (CIDR) Examples:}

\begin{itemize}
\tightlist
\item
  \textbf{192.168.1.0/25}: 128 hosts
\item
  \textbf{10.0.0.0/16}: 65,536 hosts
\item
  \textbf{172.16.0.0/20}: 4,096 hosts
\end{itemize}

\textbf{Key Points:}

\begin{itemize}
\tightlist
\item
  \textbf{Classful}: Fixed network/host boundaries
\item
  \textbf{Classless}: Variable Length Subnet Mask (VLSM)
\item
  \textbf{CIDR}: More efficient address allocation
\end{itemize}

\end{solutionbox}
\begin{mnemonicbox}
``Class-Fixed, CIDR-Flexible''

\end{mnemonicbox}
\begin{center}\rule{0.5\linewidth}{0.5pt}\end{center}

\subsection*{Question 2(a OR) [3
marks]}\label{question-2a-or-3-marks}

\textbf{Discuss goals of mobile IP}

\begin{solutionbox}
Mobile IP enables seamless connectivity for mobile
devices across different networks.

\textbf{Key Points:}

\begin{itemize}
\tightlist
\item
  \textbf{Transparency}: Applications unaware of mobility
\item
  \textbf{Compatibility}: Works with existing protocols
\item
  \textbf{Efficiency}: Minimal routing overhead
\end{itemize}

\end{solutionbox}
\begin{mnemonicbox}
``Transparent Compatible Efficient''

\end{mnemonicbox}
\begin{center}\rule{0.5\linewidth}{0.5pt}\end{center}

\subsection*{Question 2(b OR) [4
marks]}\label{question-2b-or-4-marks}

\textbf{Define ARP and RARP}

\begin{solutionbox}
ARP and RARP are address resolution protocols for
mapping between different address types.


{\def\LTcaptype{none} % do not increment counter
\begin{longtable}[]{@{}
  >{\raggedright\arraybackslash}p{(\linewidth - 6\tabcolsep) * \real{0.2439}}
  >{\raggedright\arraybackslash}p{(\linewidth - 6\tabcolsep) * \real{0.2683}}
  >{\raggedright\arraybackslash}p{(\linewidth - 6\tabcolsep) * \real{0.2195}}
  >{\raggedright\arraybackslash}p{(\linewidth - 6\tabcolsep) * \real{0.2683}}@{}}
\toprule\noalign{}
\begin{minipage}[b]{\linewidth}\raggedright
Protocol
\end{minipage} & \begin{minipage}[b]{\linewidth}\raggedright
Full Name
\end{minipage} & \begin{minipage}[b]{\linewidth}\raggedright
Purpose
\end{minipage} & \begin{minipage}[b]{\linewidth}\raggedright
Direction
\end{minipage} \\
\midrule\noalign{}
\endhead
\bottomrule\noalign{}
\endlastfoot
\textbf{ARP} & Address Resolution Protocol & IP to MAC mapping & Logical
to Physical \\
\textbf{RARP} & Reverse ARP & MAC to IP mapping & Physical to Logical \\
\end{longtable}
}

\end{solutionbox}
\begin{mnemonicbox}
``ARP-asks, RARP-reverses''

\end{mnemonicbox}
\begin{center}\rule{0.5\linewidth}{0.5pt}\end{center}

\subsection*{Question 2(c OR) [7
marks]}\label{question-2c-or-7-marks}

\textbf{Demonstrate Stop and Wait, Stop and Wait ARQ data link layer
protocols}

\begin{solutionbox}
These protocols ensure reliable data transmission at
the data link layer.

\textbf{Diagram - Stop and Wait:}

\begin{verbatim}
sequenceDiagram
    participant S as Sender
    participant R as Receiver
    S{-R: Frame 0}
    R{-S: ACK 0}
    S{-R: Frame 1}
    R{-S: ACK 1}
\end{verbatim}


{\def\LTcaptype{none} % do not increment counter
\begin{longtable}[]{@{}llll@{}}
\toprule\noalign{}
Protocol & Error Detection & Efficiency & Complexity \\
\midrule\noalign{}
\endhead
\bottomrule\noalign{}
\endlastfoot
\textbf{Stop and Wait} & Basic & Low & Simple \\
\textbf{Stop and Wait ARQ} & Advanced & Medium & Moderate \\
\end{longtable}
}

\textbf{Key Points:}

\begin{itemize}
\tightlist
\item
  \textbf{Stop and Wait}: Send frame, wait for acknowledgment
\item
  \textbf{ARQ}: Automatic Repeat reQuest on errors
\item
  \textbf{Timeout}: Resend if no acknowledgment received
\end{itemize}

\end{solutionbox}
\begin{mnemonicbox}
``Stop-Wait-Acknowledge''

\end{mnemonicbox}
\begin{center}\rule{0.5\linewidth}{0.5pt}\end{center}

\subsection*{Question 3(a) [3 marks]}\label{q3a}

\textbf{Demonstrate Wireless networks}

\begin{solutionbox}
Wireless networks use radio waves for communication
without physical connections.

\textbf{Key Points:}

\begin{itemize}
\tightlist
\item
  \textbf{Technology}: Radio frequency transmission
\item
  \textbf{Types}: WiFi, Bluetooth, Cellular
\item
  \textbf{Benefits}: Mobility, easy installation
\end{itemize}

\end{solutionbox}
\begin{mnemonicbox}
``Wireless-Radio-Mobile''

\end{mnemonicbox}
\begin{center}\rule{0.5\linewidth}{0.5pt}\end{center}

\subsection*{Question 3(b) [4 marks]}\label{q3b}

\textbf{Define Communication Middleware in mobile computing}

\begin{solutionbox}
Communication middleware provides abstraction layer for
mobile application communication.


{\def\LTcaptype{none} % do not increment counter
\begin{longtable}[]{@{}ll@{}}
\toprule\noalign{}
Aspect & Description \\
\midrule\noalign{}
\endhead
\bottomrule\noalign{}
\endlastfoot
\textbf{Purpose} & Simplify communication \\
\textbf{Location} & Between app and network \\
\textbf{Features} & Protocol handling, data conversion \\
\textbf{Examples} & CORBA, RMI \\
\end{longtable}
}

\end{solutionbox}
\begin{mnemonicbox}
``Middle-Communication-Layer''

\end{mnemonicbox}
\begin{center}\rule{0.5\linewidth}{0.5pt}\end{center}

\subsection*{Question 3(c) [7 marks]}\label{q3c}

\textbf{Discuss the architecture of Mobile Computing}

\begin{solutionbox}
Mobile computing architecture consists of multiple
interconnected components supporting mobile applications.

\textbf{Diagram:}

\begin{center}
\textbf{Mermaid Diagram (Code)}
\begin{verbatim}
{Shaded}
{Highlighting}[]
graph LR
    A[Mobile Device] {-{-}{} B[Wireless Network]}
    B {-{-}{} C[Base Station]}
    C {-{-}{} D[Mobile Support Station]}
    D {-{-}{} E[Fixed Network]}
    E {-{-}{} F[Database/Server]}
{Highlighting}
{Shaded}
\end{verbatim}
\end{center}


{\def\LTcaptype{none} % do not increment counter
\begin{longtable}[]{@{}ll@{}}
\toprule\noalign{}
Component & Function \\
\midrule\noalign{}
\endhead
\bottomrule\noalign{}
\endlastfoot
\textbf{Mobile Device} & User interface, local processing \\
\textbf{Wireless Network} & Radio communication \\
\textbf{Base Station} & Network access point \\
\textbf{MSS} & Mobility management \\
\textbf{Fixed Network} & Backbone infrastructure \\
\end{longtable}
}

\textbf{Key Points:}

\begin{itemize}
\tightlist
\item
  \textbf{Three-tier}: Mobile device, wireless network, fixed network
\item
  \textbf{Mobility Support}: Handoff management
\item
  \textbf{Data Management}: Caching and synchronization
\end{itemize}

\end{solutionbox}
\begin{mnemonicbox}
``Mobile-Wireless-Fixed''

\end{mnemonicbox}
\begin{center}\rule{0.5\linewidth}{0.5pt}\end{center}

\subsection*{Question 3(a OR) [3
marks]}\label{question-3a-or-3-marks}

\textbf{Demonstrate ad-hoc networks}

\begin{solutionbox}
Ad-hoc networks are self-organizing wireless networks
without fixed infrastructure.

\textbf{Key Points:}

\begin{itemize}
\tightlist
\item
  \textbf{Structure}: Peer-to-peer topology
\item
  \textbf{Routing}: Dynamic route discovery
\item
  \textbf{Applications}: Emergency, military
\end{itemize}

\end{solutionbox}
\begin{mnemonicbox}
``Ad-hoc-Self-Organizing''

\end{mnemonicbox}
\begin{center}\rule{0.5\linewidth}{0.5pt}\end{center}

\subsection*{Question 3(b OR) [4
marks]}\label{question-3b-or-4-marks}

\textbf{Define Transaction Processing Middleware in mobile computing}

\begin{solutionbox}
Transaction processing middleware ensures ACID
properties in mobile database transactions.


{\def\LTcaptype{none} % do not increment counter
\begin{longtable}[]{@{}ll@{}}
\toprule\noalign{}
Property & Description \\
\midrule\noalign{}
\endhead
\bottomrule\noalign{}
\endlastfoot
\textbf{Atomicity} & All or nothing execution \\
\textbf{Consistency} & Database integrity maintained \\
\textbf{Isolation} & Concurrent transaction separation \\
\textbf{Durability} & Permanent transaction effects \\
\end{longtable}
}

\end{solutionbox}
\begin{mnemonicbox}
``ACID-Properties''

\end{mnemonicbox}
\begin{center}\rule{0.5\linewidth}{0.5pt}\end{center}

\subsection*{Question 3(c OR) [7
marks]}\label{question-3c-or-7-marks}

\textbf{Discuss the applications and services of mobile computing}

\begin{solutionbox}
Mobile computing enables diverse applications across
multiple domains.


{\def\LTcaptype{none} % do not increment counter
\begin{longtable}[]{@{}lll@{}}
\toprule\noalign{}
Domain & Applications & Services \\
\midrule\noalign{}
\endhead
\bottomrule\noalign{}
\endlastfoot
\textbf{Business} & CRM, ERP & Data synchronization \\
\textbf{Healthcare} & Patient monitoring & Remote diagnosis \\
\textbf{Education} & E-learning & Content delivery \\
\textbf{Entertainment} & Gaming, streaming & Media services \\
\textbf{Navigation} & GPS, maps & Location services \\
\end{longtable}
}

\textbf{Key Points:}

\begin{itemize}
\tightlist
\item
  \textbf{Location-based}: GPS navigation, geo-fencing
\item
  \textbf{Communication}: Email, messaging, video calls
\item
  \textbf{Commerce}: Mobile banking, shopping
\end{itemize}

\end{solutionbox}
\begin{mnemonicbox}
``Business-Health-Education-Entertainment''

\end{mnemonicbox}
\begin{center}\rule{0.5\linewidth}{0.5pt}\end{center}

\subsection*{Question 4(a) [3 marks]}\label{q4a}

\textbf{Describe Indirect TCP in mobile computing}

\begin{solutionbox}
Indirect TCP splits TCP connection to handle mobile
host mobility efficiently.

\textbf{Diagram:}

\begin{verbatim}
Fixed Host {-{-} Base Station {-}{-} Mobile Host}
    TCP1          TCP2
\end{verbatim}

\textbf{Key Points:}

\begin{itemize}
\tightlist
\item
  \textbf{Split Connection}: Two separate TCP connections
\item
  \textbf{Base Station}: Acts as proxy
\item
  \textbf{Advantage}: Faster handoff
\end{itemize}

\end{solutionbox}
\begin{mnemonicbox}
``Indirect-Split-Proxy''

\end{mnemonicbox}
\begin{center}\rule{0.5\linewidth}{0.5pt}\end{center}

\subsection*{Question 4(b) [4 marks]}\label{q4b}

\textbf{Explain the steps of the packet delivery in Mobile IP}

\begin{solutionbox}
Mobile IP packet delivery involves registration,
tunneling, and delivery steps.

\textbf{Steps:}

\begin{enumerate}
\tightlist
\item
  \textbf{Registration}: Mobile node registers with home agent
\item
  \textbf{Tunneling}: Home agent creates tunnel to foreign agent\\
\item
  \textbf{Encapsulation}: Original packet wrapped in new header
\item
  \textbf{Delivery}: Foreign agent delivers to mobile node
\end{enumerate}

\end{solutionbox}
\begin{mnemonicbox}
``Register-Tunnel-Encapsulate-Deliver''

\end{mnemonicbox}
\begin{center}\rule{0.5\linewidth}{0.5pt}\end{center}

\subsection*{Question 4(c) [7 marks]}\label{q4c}

\textbf{Write following three processes of mobile IP: (1) Registration
(2) Tunneling (3) Encapsulation}

\begin{solutionbox}

\textbf{1. Registration Process:}

\begin{itemize}
\tightlist
\item
  Mobile node discovers foreign agent
\item
  Registers care-of address with home agent
\item
  Authentication and binding update
\end{itemize}

\textbf{2. Tunneling Process:}

\begin{itemize}
\tightlist
\item
  Home agent creates virtual tunnel
\item
  Packets forwarded through tunnel
\item
  Maintains end-to-end connectivity
\end{itemize}

\textbf{3. Encapsulation Process:}

\begin{itemize}
\tightlist
\item
  Original packet becomes payload
\item
  New IP header added with care-of address
\item
  Packet delivered to foreign network
\end{itemize}

\textbf{Diagram:}

\begin{center}
\textbf{Mermaid Diagram (Code)}
\begin{verbatim}
{Shaded}
{Highlighting}[]
graph LR
    A[Original Packet] {-{-}{} B[Encapsulation]}
    B {-{-}{} C[Tunneled Packet]}
    C {-{-}{} D[Delivery]}
{Highlighting}
{Shaded}
\end{verbatim}
\end{center}

\textbf{Key Points:}

\begin{itemize}
\tightlist
\item
  \textbf{Registration}: Location update mechanism
\item
  \textbf{Tunneling}: Virtual connection establishment\\
\item
  \textbf{Encapsulation}: Packet wrapping technique
\end{itemize}

\end{solutionbox}
\begin{mnemonicbox}
``Register-Tunnel-Encapsulate''

\end{mnemonicbox}
\begin{center}\rule{0.5\linewidth}{0.5pt}\end{center}

\subsection*{Question 4(a OR) [3
marks]}\label{question-4a-or-3-marks}

\textbf{Describe Snooping TCP in mobile computing}

\begin{solutionbox}
Snooping TCP improves performance by caching and
monitoring TCP segments at base station.

\textbf{Key Points:}

\begin{itemize}
\tightlist
\item
  \textbf{Local Retransmission}: Base station handles losses
\item
  \textbf{Buffer Management}: Caches unacknowledged segments
\item
  \textbf{Transparency}: End-to-end TCP maintained
\end{itemize}

\end{solutionbox}
\begin{mnemonicbox}
``Snoop-Cache-Retransmit''

\end{mnemonicbox}
\begin{center}\rule{0.5\linewidth}{0.5pt}\end{center}

\subsection*{Question 4(b OR) [4
marks]}\label{question-4b-or-4-marks}

\textbf{Explain the Handover Management in mobile IP}

\begin{solutionbox}
Handover management maintains connectivity when mobile
node changes networks.


{\def\LTcaptype{none} % do not increment counter
\begin{longtable}[]{@{}ll@{}}
\toprule\noalign{}
Phase & Process \\
\midrule\noalign{}
\endhead
\bottomrule\noalign{}
\endlastfoot
\textbf{Discovery} & Find new foreign agent \\
\textbf{Registration} & Update care-of address \\
\textbf{Data Forwarding} & Redirect packets \\
\textbf{Cleanup} & Release old resources \\
\end{longtable}
}

\end{solutionbox}
\begin{mnemonicbox}
``Discover-Register-Forward-Cleanup''

\end{mnemonicbox}
\begin{center}\rule{0.5\linewidth}{0.5pt}\end{center}

\subsection*{Question 4(c OR) [7
marks]}\label{question-4c-or-7-marks}

\textbf{Write the goals and the requirements for the Mobile IP}

\begin{solutionbox}

\textbf{Goals:}

\begin{itemize}
\tightlist
\item
  \textbf{Transparency}: Seamless mobility for applications
\item
  \textbf{Compatibility}: Work with existing internet protocols\\
\item
  \textbf{Scalability}: Support large number of mobile nodes
\item
  \textbf{Security}: Authenticate mobile nodes and protect data
\end{itemize}

\textbf{Requirements:}

\begin{itemize}
\tightlist
\item
  \textbf{Home Agent}: Maintains mobile node location
\item
  \textbf{Foreign Agent}: Provides local services
\item
  \textbf{Care-of Address}: Temporary address in foreign network
\item
  \textbf{Tunneling}: Packet forwarding mechanism
\end{itemize}


{\def\LTcaptype{none} % do not increment counter
\begin{longtable}[]{@{}lll@{}}
\toprule\noalign{}
Aspect & Goals & Requirements \\
\midrule\noalign{}
\endhead
\bottomrule\noalign{}
\endlastfoot
\textbf{Mobility} & Seamless movement & Care-of address \\
\textbf{Connectivity} & Maintain sessions & Tunneling \\
\textbf{Performance} & Minimal overhead & Efficient routing \\
\textbf{Security} & Authentication & Secure protocols \\
\end{longtable}
}

\end{solutionbox}
\begin{mnemonicbox}
``Transparent-Compatible-Scalable-Secure''

\end{mnemonicbox}
\begin{center}\rule{0.5\linewidth}{0.5pt}\end{center}

\subsection*{Question 5(a) [3 marks]}\label{q5a}

\textbf{Write the features of 6G in mobile networks}

\begin{solutionbox}
6G represents the next generation of mobile networks
with advanced capabilities.

\textbf{Key Points:}

\begin{itemize}
\tightlist
\item
  \textbf{Speed}: 1 Tbps theoretical speed
\item
  \textbf{Latency}: Sub-millisecond latency
\item
  \textbf{AI Integration}: Native artificial intelligence
\end{itemize}

\end{solutionbox}
\begin{mnemonicbox}
``Tera-Speed-AI-Integration''

\end{mnemonicbox}
\begin{center}\rule{0.5\linewidth}{0.5pt}\end{center}

\subsection*{Question 5(b) [4 marks]}\label{q5b}

\textbf{Describe Dynamic Host Configuration Protocol (DHCP)}

\begin{solutionbox}
DHCP automatically assigns IP addresses and network
configuration to devices.


{\def\LTcaptype{none} % do not increment counter
\begin{longtable}[]{@{}ll@{}}
\toprule\noalign{}
Process & Description \\
\midrule\noalign{}
\endhead
\bottomrule\noalign{}
\endlastfoot
\textbf{Discover} & Client broadcasts request \\
\textbf{Offer} & Server offers IP address \\
\textbf{Request} & Client requests specific IP \\
\textbf{Acknowledge} & Server confirms assignment \\
\end{longtable}
}

\end{solutionbox}
\begin{mnemonicbox}
``Discover-Offer-Request-Acknowledge''

\end{mnemonicbox}
\begin{center}\rule{0.5\linewidth}{0.5pt}\end{center}

\subsection*{Question 5(c) [7 marks]}\label{q5c}

\textbf{Describe the architecture of Wireless Personal Area Network
(WLAN)}

\begin{solutionbox}
WLAN architecture provides wireless connectivity within
local area using IEEE 802.11 standards.

\textbf{Diagram:}

\begin{verbatim}
graph TB
    A[Access Point] {-{-} B[Distribution System]}
    A {-{-} C[Station 1]}
    A {-{-} D[Station 2] }
    A {-{-} E[Station 3]}
    B {-{-} F[Internet/WAN]}
\end{verbatim}


{\def\LTcaptype{none} % do not increment counter
\begin{longtable}[]{@{}ll@{}}
\toprule\noalign{}
Component & Function \\
\midrule\noalign{}
\endhead
\bottomrule\noalign{}
\endlastfoot
\textbf{Access Point} & Central wireless hub \\
\textbf{Station} & Wireless client device \\
\textbf{Distribution System} & Backbone network \\
\textbf{BSS} & Basic Service Set \\
\textbf{ESS} & Extended Service Set \\
\end{longtable}
}

\textbf{Key Points:}

\begin{itemize}
\tightlist
\item
  \textbf{Infrastructure Mode}: Uses access points
\item
  \textbf{Ad-hoc Mode}: Direct device communication
\item
  \textbf{Standards}: 802.11a/b/g/n/ac/ax protocols
\end{itemize}

\end{solutionbox}
\begin{mnemonicbox}
``Access-Station-Distribution''

\end{mnemonicbox}
\begin{center}\rule{0.5\linewidth}{0.5pt}\end{center}

\subsection*{Question 5(a OR) [3
marks]}\label{question-5a-or-3-marks}

\textbf{Write the features of 5G in mobile networks}

\begin{solutionbox}
5G provides enhanced mobile broadband with ultra-low
latency.

\textbf{Key Points:}

\begin{itemize}
\tightlist
\item
  \textbf{Speed}: Up to 10 Gbps download
\item
  \textbf{Latency}: 1ms ultra-low latency
\item
  \textbf{Density}: 1 million devices per km^{2}
\end{itemize}

\end{solutionbox}
\begin{mnemonicbox}
``10G-1ms-1Million''

\end{mnemonicbox}
\begin{center}\rule{0.5\linewidth}{0.5pt}\end{center}

\subsection*{Question 5(b OR) [4
marks]}\label{question-5b-or-4-marks}

\textbf{Explain WWW and HTTP}

\begin{solutionbox}
World Wide Web uses HTTP protocol for web page
communication.


{\def\LTcaptype{none} % do not increment counter
\begin{longtable}[]{@{}lll@{}}
\toprule\noalign{}
Aspect & WWW & HTTP \\
\midrule\noalign{}
\endhead
\bottomrule\noalign{}
\endlastfoot
\textbf{Purpose} & Information sharing & Communication protocol \\
\textbf{Components} & Web pages, browsers & Request/response \\
\textbf{Format} & HTML documents & Text-based protocol \\
\textbf{Port} & Various & 80, 443 \\
\end{longtable}
}

\end{solutionbox}
\begin{mnemonicbox}
``Web-Hypertext-Transfer''

\end{mnemonicbox}
\begin{center}\rule{0.5\linewidth}{0.5pt}\end{center}

\subsection*{Question 5(c OR) [7
marks]}\label{question-5c-or-7-marks}

\textbf{Describe the architecture of Bluetooth}

\begin{solutionbox}
Bluetooth architecture provides short-range wireless
communication using protocol stack.

\textbf{Diagram:}

\begin{center}
\textbf{Mermaid Diagram (Code)}
\begin{verbatim}
{Shaded}
{Highlighting}[]
graph LR
    A[Application Layer] {-{-}{} B[OBEX/SDP]}
    B {-{-}{} C[L2CAP]}
    C {-{-}{} D[HCI]}
    D {-{-}{} E[Link Manager]}
    E {-{-}{} F[Baseband]}
    F {-{-}{} G[Radio Layer]}
{Highlighting}
{Shaded}
\end{verbatim}
\end{center}


{\def\LTcaptype{none} % do not increment counter
\begin{longtable}[]{@{}ll@{}}
\toprule\noalign{}
Layer & Function \\
\midrule\noalign{}
\endhead
\bottomrule\noalign{}
\endlastfoot
\textbf{Radio} & Physical transmission \\
\textbf{Baseband} & Timing and frequency hopping \\
\textbf{Link Manager} & Connection management \\
\textbf{HCI} & Host Controller Interface \\
\textbf{L2CAP} & Logical Link Control \\
\textbf{Applications} & User services \\
\end{longtable}
}

\textbf{Key Points:}

\begin{itemize}
\tightlist
\item
  \textbf{Piconet}: Master-slave network topology
\item
  \textbf{Frequency Hopping}: 79 frequency channels
\item
  \textbf{Power Classes}: Different transmission ranges
\end{itemize}

\end{solutionbox}
\begin{mnemonicbox}
``Radio-Baseband-Link-Host-Logic''

\end{mnemonicbox}

\end{document}
