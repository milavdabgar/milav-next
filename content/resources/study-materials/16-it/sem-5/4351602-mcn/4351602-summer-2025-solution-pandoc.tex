\documentclass[10pt,a4paper]{article}

% content/resources/templates/preamble.tex
\usepackage[margin=0.6in]{geometry}
\author{Milav Dabgar}
\usepackage{amsmath,amssymb,amsthm}
\usepackage{booktabs}
\usepackage{multirow}
\usepackage{xcolor}
\usepackage{tcolorbox}
\tcbuselibrary{breakable,skins}
\usepackage[colorlinks=true,linkcolor=blue]{hyperref}
\usepackage{titlesec}
\usepackage{enumitem}
\usepackage{tikz}
\usepackage{pgfplots}
\usepackage{circuitikz}
\usepackage[version=4]{mhchem}
\usepackage{longtable}
\usepackage{array}
\usepackage{float}
\usepackage{caption}
\usepackage{listings}

\lstset{
  basicstyle=\small\ttfamily,
  breaklines=true,
  breakatwhitespace=false,
  postbreak=\mbox{\textcolor{red}{$\hookrightarrow$}\space},
  float=false,
  numbers=left,
  numberstyle=\tiny\color{gray},
  numbersep=10pt,
  xleftmargin=2em,
  keywordstyle=\color{blue},
  commentstyle=\color{green!60!black},
  stringstyle=\color{purple},
  backgroundcolor=\color{gray!5},
  showstringspaces=false,
  tabsize=2,
  captionpos=b,
  keepspaces=true,
  columns=flexible
}

\pgfplotsset{compat=1.18}
\usetikzlibrary{shapes,arrows,positioning,calc,patterns,decorations.pathmorphing,decorations.markings,arrows.meta}

% Color scheme
\definecolor{headcolor}{RGB}{0,102,204}
\definecolor{keycolor}{RGB}{220,20,60}
\definecolor{solutioncolor}{RGB}{34,139,34}
\definecolor{mnemoniccolor}{RGB}{148,0,211}
\definecolor{codecolor}{RGB}{0,0,100}

% Spacing
\setlength{\parskip}{3pt}
\setlist[itemize]{nosep}
\setlist[enumerate]{nosep}

% Title formatting
\titleformat{\section}{\Large\bfseries\color{headcolor}}{\thesection}{1em}{}
\titleformat{\subsection}{\large\bfseries\color{headcolor}}{\thesubsection}{1em}{}

% Pandoc tightlist compatibility
\providecommand{\tightlist}{%
  \setlength{\itemsep}{0pt}\setlength{\parskip}{0pt}}

% Pandoc longtable compatibility
\newcounter{none}
\def\thenone{}


% content/resources/templates/english-boxes.tex
% This file is currently empty - it exists to maintain consistency with the import structure.
% Add custom environments here if needed in the future.


\begin{document}

\begin{center}
{\Huge\bfseries\color{headcolor} Subject Name Solutions}\\[5pt]
{\LARGE 4351602 -- Summer 2025}\\[3pt]
{\large Semester 1 Study Material}\\[3pt]
{\normalsize\textit{Detailed Solutions and Explanations}}
\end{center}

\vspace{10pt}

\subsection*{Question 1(a) [3 marks]}\label{q1a}

\textbf{Explain working of POP protocol.}

\begin{solutionbox}

POP (Post Office Protocol) is an email retrieval protocol that downloads
emails from server to client device.

\textbf{Working Process:}

{\def\LTcaptype{none} % do not increment counter
\begin{longtable}[]{@{}lll@{}}
\toprule\noalign{}
Step & Action & Description \\
\midrule\noalign{}
\endhead
\bottomrule\noalign{}
\endlastfoot
1 & Connection & Client connects to POP server on port 110 \\
2 & Authentication & User provides username and password \\
3 & Download & Emails downloaded to local device \\
4 & Deletion & Emails deleted from server after download \\
\end{longtable}
}

\begin{itemize}
\tightlist
\item
  \textbf{Download-based}: Emails stored locally on client device
\item
  \textbf{Offline access}: Can read emails without internet connection
\item
  \textbf{Single device}: Best suited for single device access
\end{itemize}

\end{solutionbox}
\begin{mnemonicbox}
``POP Downloads Once Permanently''

\end{mnemonicbox}
\begin{center}\rule{0.5\linewidth}{0.5pt}\end{center}

\subsection*{Question 1(b) [4 marks]}\label{q1b}

\textbf{Compare OSI model with TCP/IP model.}

\begin{solutionbox}

Comparison between OSI and TCP/IP networking models:

{\def\LTcaptype{none} % do not increment counter
\begin{longtable}[]{@{}lll@{}}
\toprule\noalign{}
Aspect & OSI Model & TCP/IP Model \\
\midrule\noalign{}
\endhead
\bottomrule\noalign{}
\endlastfoot
\textbf{Layers} & 7 layers & 4 layers \\
\textbf{Approach} & Theoretical model & Practical implementation \\
\textbf{Development} & ISO standard & DARPA project \\
\textbf{Complexity} & More complex & Simpler structure \\
\end{longtable}
}

\textbf{Key Differences:}

\begin{itemize}
\tightlist
\item
  \textbf{Layer count}: OSI has 7 layers vs TCP/IP's 4 layers
\item
  \textbf{Real-world usage}: TCP/IP widely implemented, OSI mostly
  theoretical
\item
  \textbf{Protocol independence}: OSI is protocol-independent, TCP/IP is
  protocol-specific
\item
  \textbf{Header overhead}: OSI has more overhead due to additional
  layers
\end{itemize}

\end{solutionbox}
\begin{mnemonicbox}
``OSI Seven Theoretical, TCP Four Practical''

\end{mnemonicbox}
\begin{center}\rule{0.5\linewidth}{0.5pt}\end{center}

\subsection*{Question 1(c) [7 marks]}\label{q1c}

\textbf{Explain protocols working at each layer in TCP/IP models.}

\begin{solutionbox}

TCP/IP model consists of 4 layers with specific protocols at each layer:

\begin{center}
\textbf{Mermaid Diagram (Code)}
\begin{verbatim}
{Shaded}
{Highlighting}[]
graph LR
    A[Application Layer] {-{-}{} B[Transport Layer]}
    B {-{-}{} C[Internet Layer]}
    C {-{-}{} D[Network Access Layer]}
    
    A1[HTTP, HTTPS, FTP, SMTP, POP, IMAP, DNS] {-{-}{} A}
    B1[TCP, UDP] {-{-}{} B}
    C1[IP, ICMP, ARP, RARP] {-{-}{} C}
    D1[Ethernet, WiFi, PPP] {-{-}{} D}
{Highlighting}
{Shaded}
\end{verbatim}
\end{center}

\textbf{Layer-wise Protocol Functions:}

{\def\LTcaptype{none} % do not increment counter
\begin{longtable}[]{@{}lll@{}}
\toprule\noalign{}
Layer & Protocols & Function \\
\midrule\noalign{}
\endhead
\bottomrule\noalign{}
\endlastfoot
\textbf{Application} & HTTP, FTP, SMTP, DNS & User interface and
services \\
\textbf{Transport} & TCP, UDP & End-to-end communication \\
\textbf{Internet} & IP, ICMP, ARP & Routing and addressing \\
\textbf{Network Access} & Ethernet, WiFi & Physical transmission \\
\end{longtable}
}

\textbf{Protocol Details:}

\begin{itemize}
\tightlist
\item
  \textbf{HTTP/HTTPS}: Web communication and secure web communication
\item
  \textbf{TCP}: Reliable, connection-oriented data transfer
\item
  \textbf{UDP}: Fast, connectionless data transfer
\item
  \textbf{IP}: Packet routing and addressing
\item
  \textbf{ARP}: Maps IP addresses to MAC addresses
\end{itemize}

\end{solutionbox}
\begin{mnemonicbox}
``Applications Transport Internet Networks Always''

\end{mnemonicbox}
\begin{center}\rule{0.5\linewidth}{0.5pt}\end{center}

\subsection*{Question 1(c OR) [7
marks]}\label{question-1c-or-7-marks}

\textbf{Briefly explain OSI model with all its layers and functionality
of each layer}

\begin{solutionbox}

OSI (Open Systems Interconnection) model has 7 layers for network
communication:

\begin{center}
\textbf{Mermaid Diagram (Code)}
\begin{verbatim}
{Shaded}
{Highlighting}[]
graph LR
    A[Application Layer] {-{-}{} B[Presentation Layer]}
    B {-{-}{} C[Session Layer]}
    C {-{-}{} D[Transport Layer]}
    D {-{-}{} E[Network Layer]}
    E {-{-}{} F[Data Link Layer]}
    F {-{-}{} G[Physical Layer]}
{Highlighting}
{Shaded}
\end{verbatim}
\end{center}

\textbf{Layer Functions:}

{\def\LTcaptype{none} % do not increment counter
\begin{longtable}[]{@{}llll@{}}
\toprule\noalign{}
Layer & Name & Function & Protocols \\
\midrule\noalign{}
\endhead
\bottomrule\noalign{}
\endlastfoot
\textbf{7} & Application & User interface & HTTP, FTP, SMTP \\
\textbf{6} & Presentation & Data formatting, encryption & SSL, JPEG,
MPEG \\
\textbf{5} & Session & Session management & NetBIOS, RPC \\
\textbf{4} & Transport & End-to-end delivery & TCP, UDP \\
\textbf{3} & Network & Routing & IP, ICMP \\
\textbf{2} & Data Link & Frame transmission & Ethernet, PPP \\
\textbf{1} & Physical & Bit transmission & Cables, Radio waves \\
\end{longtable}
}

\textbf{Key Features:}

\begin{itemize}
\tightlist
\item
  \textbf{Modular design}: Each layer has specific responsibilities
\item
  \textbf{Protocol independence}: Layers can use different protocols
\item
  \textbf{Standardization}: Universal networking reference model
\end{itemize}

\end{solutionbox}
\begin{mnemonicbox}
``All People Seem To Need Data Processing''

\end{mnemonicbox}
\begin{center}\rule{0.5\linewidth}{0.5pt}\end{center}

\subsection*{Question 2(a) [3 marks]}\label{q2a}

\textbf{Give the difference between ARP and RARP protocols.}

\begin{solutionbox}

ARP and RARP are address resolution protocols with opposite functions:

{\def\LTcaptype{none} % do not increment counter
\begin{longtable}[]{@{}
  >{\raggedright\arraybackslash}p{(\linewidth - 4\tabcolsep) * \real{0.4211}}
  >{\raggedright\arraybackslash}p{(\linewidth - 4\tabcolsep) * \real{0.2632}}
  >{\raggedright\arraybackslash}p{(\linewidth - 4\tabcolsep) * \real{0.3158}}@{}}
\toprule\noalign{}
\begin{minipage}[b]{\linewidth}\raggedright
Aspect
\end{minipage} & \begin{minipage}[b]{\linewidth}\raggedright
ARP
\end{minipage} & \begin{minipage}[b]{\linewidth}\raggedright
RARP
\end{minipage} \\
\midrule\noalign{}
\endhead
\bottomrule\noalign{}
\endlastfoot
\textbf{Full Form} & Address Resolution Protocol & Reverse Address
Resolution Protocol \\
\textbf{Purpose} & IP to MAC address mapping & MAC to IP address
mapping \\
\textbf{Direction} & Logical to Physical & Physical to Logical \\
\textbf{Usage} & Normal network communication & Diskless workstations \\
\end{longtable}
}

\textbf{Working Process:}

\begin{itemize}
\tightlist
\item
  \textbf{ARP}: ``I know IP address, need MAC address''
\item
  \textbf{RARP}: ``I know MAC address, need IP address''
\item
  \textbf{Cache}: Both maintain address tables for efficiency
\end{itemize}

\end{solutionbox}
\begin{mnemonicbox}
``ARP Asks Physical, RARP Requests IP''

\end{mnemonicbox}
\begin{center}\rule{0.5\linewidth}{0.5pt}\end{center}

\subsection*{Question 2(b) [4 marks]}\label{q2b}

\textbf{Explain working of IMAP protocol.}

\begin{solutionbox}

IMAP (Internet Message Access Protocol) manages emails on server for
multiple device access.

\textbf{Working Process:}

{\def\LTcaptype{none} % do not increment counter
\begin{longtable}[]{@{}lll@{}}
\toprule\noalign{}
Step & Action & Description \\
\midrule\noalign{}
\endhead
\bottomrule\noalign{}
\endlastfoot
1 & Connection & Client connects to IMAP server (port 143/993) \\
2 & Authentication & Login with credentials \\
3 & Folder Access & Browse email folders on server \\
4 & Synchronization & Changes sync across all devices \\
\end{longtable}
}

\textbf{Key Features:}

\begin{itemize}
\tightlist
\item
  \textbf{Server-based}: Emails remain on server
\item
  \textbf{Multi-device}: Access from multiple devices
\item
  \textbf{Synchronization}: Changes reflected everywhere
\item
  \textbf{Selective download}: Download only needed emails
\end{itemize}

\textbf{Advantages:}

\begin{itemize}
\tightlist
\item
  \textbf{Storage efficiency}: Server manages storage
\item
  \textbf{Accessibility}: Access from anywhere
\item
  \textbf{Backup}: Server provides automatic backup
\end{itemize}

\end{solutionbox}
\begin{mnemonicbox}
``IMAP Internet Messages Always Present''

\end{mnemonicbox}
\begin{center}\rule{0.5\linewidth}{0.5pt}\end{center}

\subsection*{Question 2(c) [7 marks]}\label{q2c}

\textbf{Explain Three-tier architecture of mobile computing with
appropriate diagram.}

\begin{solutionbox}

Three-tier architecture separates mobile computing into distinct layers:

\begin{center}
\textbf{Mermaid Diagram (Code)}
\begin{verbatim}
{Shaded}
{Highlighting}[]
graph LR
    A[Presentation Tier{br/{}Mobile Devices] {-}{-}{} B[Application Tier{}br/{}Application Server]}
    B {-{-}{} C[Data Tier{}br/{}Database Server]}
    
    A1[Smartphones{br/{}Tablets{}br/{}Laptops] {-}{-}{} A}
    B1[Business Logic{br/{}Processing{}br/{}API Services] {-}{-}{} B}
    C1[Database{br/{}File Systems{}br/{}Data Storage] {-}{-}{} C}
{Highlighting}
{Shaded}
\end{verbatim}
\end{center}

\textbf{Tier Details:}

{\def\LTcaptype{none} % do not increment counter
\begin{longtable}[]{@{}
  >{\raggedright\arraybackslash}p{(\linewidth - 4\tabcolsep) * \real{0.1667}}
  >{\raggedright\arraybackslash}p{(\linewidth - 4\tabcolsep) * \real{0.3333}}
  >{\raggedright\arraybackslash}p{(\linewidth - 4\tabcolsep) * \real{0.5000}}@{}}
\toprule\noalign{}
\begin{minipage}[b]{\linewidth}\raggedright
Tier
\end{minipage} & \begin{minipage}[b]{\linewidth}\raggedright
Components
\end{minipage} & \begin{minipage}[b]{\linewidth}\raggedright
Responsibilities
\end{minipage} \\
\midrule\noalign{}
\endhead
\bottomrule\noalign{}
\endlastfoot
\textbf{Presentation} & Mobile devices, UI & User interface and
interaction \\
\textbf{Application} & App servers, middleware & Business logic and
processing \\
\textbf{Data} & Databases, storage & Data management and storage \\
\end{longtable}
}

\textbf{Architecture Benefits:}

\begin{itemize}
\tightlist
\item
  \textbf{Scalability}: Each tier can scale independently
\item
  \textbf{Maintainability}: Separate concerns for easier updates
\item
  \textbf{Security}: Data protection through tier separation
\item
  \textbf{Performance}: Distributed processing reduces load
\end{itemize}

\textbf{Communication Flow:}

\begin{itemize}
\tightlist
\item
  \textbf{User request}: Presentation \rightarrow Application \rightarrow Data
\item
  \textbf{Response}: Data \rightarrow Application \rightarrow Presentation
\item
  \textbf{Processing}: Application tier handles business logic
\end{itemize}

\end{solutionbox}
\begin{mnemonicbox}
``Presentation Applies Data Processing''

\end{mnemonicbox}
\begin{center}\rule{0.5\linewidth}{0.5pt}\end{center}

\subsection*{Question 2(a OR) [3
marks]}\label{question-2a-or-3-marks}

\textbf{Explain the limitation of Stop-and-wait data link layer
protocol.}

\begin{solutionbox}

Stop-and-wait protocol has several performance limitations:

\textbf{Major Limitations:}

{\def\LTcaptype{none} % do not increment counter
\begin{longtable}[]{@{}
  >{\raggedright\arraybackslash}p{(\linewidth - 4\tabcolsep) * \real{0.3636}}
  >{\raggedright\arraybackslash}p{(\linewidth - 4\tabcolsep) * \real{0.3939}}
  >{\raggedright\arraybackslash}p{(\linewidth - 4\tabcolsep) * \real{0.2424}}@{}}
\toprule\noalign{}
\begin{minipage}[b]{\linewidth}\raggedright
Limitation
\end{minipage} & \begin{minipage}[b]{\linewidth}\raggedright
Description
\end{minipage} & \begin{minipage}[b]{\linewidth}\raggedright
Impact
\end{minipage} \\
\midrule\noalign{}
\endhead
\bottomrule\noalign{}
\endlastfoot
\textbf{Low Efficiency} & Waits for ACK before next frame & Poor
bandwidth utilization \\
\textbf{High Delay} & Round-trip delay for each frame & Slow data
transmission \\
\textbf{Error Sensitivity} & Single error stops transmission & Reduced
reliability \\
\end{longtable}
}

\textbf{Performance Issues:}

\begin{itemize}
\tightlist
\item
  \textbf{Bandwidth waste}: Link remains idle during wait time
\item
  \textbf{Timeout problems}: Lost ACK causes unnecessary retransmission
\item
  \textbf{Sequential processing}: Cannot send multiple frames
  simultaneously
\end{itemize}

\end{solutionbox}
\begin{mnemonicbox}
``Stop Waits, Bandwidth Wastes''

\end{mnemonicbox}
\begin{center}\rule{0.5\linewidth}{0.5pt}\end{center}

\subsection*{Question 2(b OR) [4
marks]}\label{question-2b-or-4-marks}

\textbf{Explain Advantages of IPV6 over the older IPV4 addressing
scheme.}

\begin{solutionbox}

IPv6 provides significant improvements over IPv4:

\textbf{Key Advantages:}

{\def\LTcaptype{none} % do not increment counter
\begin{longtable}[]{@{}lll@{}}
\toprule\noalign{}
Feature & IPv4 & IPv6 \\
\midrule\noalign{}
\endhead
\bottomrule\noalign{}
\endlastfoot
\textbf{Address Space} & 32-bit (4.3 billion) & 128-bit (340
undecillion) \\
\textbf{Header} & Variable length & Fixed 40 bytes \\
\textbf{Security} & Optional IPSec & Built-in IPSec \\
\textbf{Configuration} & Manual/DHCP & Auto-configuration \\
\end{longtable}
}

\textbf{Major Benefits:}

\begin{itemize}
\tightlist
\item
  \textbf{Unlimited addresses}: Solves address exhaustion problem
\item
  \textbf{Better performance}: Simplified header processing
\item
  \textbf{Enhanced security}: Mandatory encryption support
\item
  \textbf{Mobility support}: Better mobile device connectivity
\end{itemize}

\textbf{Additional Features:}

\begin{itemize}
\tightlist
\item
  \textbf{Quality of Service}: Built-in QoS support
\item
  \textbf{Multicast}: Improved multicast capabilities
\item
  \textbf{No fragmentation}: Routers don't fragment packets
\end{itemize}

\end{solutionbox}
\begin{mnemonicbox}
``IPv6 Improves Performance, Security, Addresses''

\end{mnemonicbox}
\begin{center}\rule{0.5\linewidth}{0.5pt}\end{center}

\subsection*{Question 2(c OR) [7
marks]}\label{question-2c-or-7-marks}

\textbf{Enlist types of networks available in mobile computing. Explain
one of them in detail.}

\begin{solutionbox}

\textbf{Types of Mobile Networks:}

{\def\LTcaptype{none} % do not increment counter
\begin{longtable}[]{@{}llll@{}}
\toprule\noalign{}
Generation & Technology & Speed & Features \\
\midrule\noalign{}
\endhead
\bottomrule\noalign{}
\endlastfoot
\textbf{2G} & GSM, CDMA & 64 Kbps & Voice + SMS \\
\textbf{3G} & UMTS, CDMA2000 & 2 Mbps & Data services \\
\textbf{4G} & LTE, WiMAX & 100 Mbps & High-speed internet \\
\textbf{5G} & New Radio (NR) & 10 Gbps & Ultra-low latency \\
\end{longtable}
}

\textbf{Detailed: 4G LTE Network}

\begin{center}
\textbf{Mermaid Diagram (Code)}
\begin{verbatim}
{Shaded}
{Highlighting}[]
graph LR
    A[Mobile Device] {-{-}{} B[eNodeB{}br/{}Base Station]}
    B {-{-}{} C[Mobility Management Entity{}br/{}MME]}
    B {-{-}{} D[Serving Gateway{}br/{}S{-}GW]}
    D {-{-}{} E[Packet Data Network Gateway{}br/{}P{-}GW]}
    E {-{-}{} F[Internet/External Networks]}
    C {-{-}{} G[Home Subscriber Server{}br/{}HSS]}
{Highlighting}
{Shaded}
\end{verbatim}
\end{center}

\textbf{4G LTE Features:}

\begin{itemize}
\tightlist
\item
  \textbf{High Speed}: Up to 100 Mbps download, 50 Mbps upload
\item
  \textbf{Low Latency}: Less than 10ms for real-time applications
\item
  \textbf{All-IP Network}: Packet-switched architecture
\item
  \textbf{Advanced Antenna}: MIMO technology for better coverage
\end{itemize}

\textbf{Architecture Components:}

\begin{itemize}
\tightlist
\item
  \textbf{eNodeB}: Enhanced base station with advanced features
\item
  \textbf{MME}: Manages mobility and authentication
\item
  \textbf{Gateways}: Handle data routing and external connectivity
\end{itemize}

\textbf{Applications}: Video streaming, online gaming, IoT connectivity

\end{solutionbox}
\begin{mnemonicbox}
``4G LTE: Long Term Evolution''

\end{mnemonicbox}
\begin{center}\rule{0.5\linewidth}{0.5pt}\end{center}

\subsection*{Question 3(a) [3 marks]}\label{q3a}

\textbf{Explain types of Routing.}

\begin{solutionbox}

Routing determines path for data packets across networks:

\textbf{Types of Routing:}

{\def\LTcaptype{none} % do not increment counter
\begin{longtable}[]{@{}
  >{\raggedright\arraybackslash}p{(\linewidth - 4\tabcolsep) * \real{0.2143}}
  >{\raggedright\arraybackslash}p{(\linewidth - 4\tabcolsep) * \real{0.4643}}
  >{\raggedright\arraybackslash}p{(\linewidth - 4\tabcolsep) * \real{0.3214}}@{}}
\toprule\noalign{}
\begin{minipage}[b]{\linewidth}\raggedright
Type
\end{minipage} & \begin{minipage}[b]{\linewidth}\raggedright
Description
\end{minipage} & \begin{minipage}[b]{\linewidth}\raggedright
Example
\end{minipage} \\
\midrule\noalign{}
\endhead
\bottomrule\noalign{}
\endlastfoot
\textbf{Static} & Manual route configuration & Administrative setup \\
\textbf{Dynamic} & Automatic route discovery & RIP, OSPF protocols \\
\textbf{Default} & Fallback route for unknown destinations & Gateway of
last resort \\
\end{longtable}
}

\textbf{Routing Categories:}

\begin{itemize}
\tightlist
\item
  \textbf{Distance Vector}: Uses hop count (RIP)
\item
  \textbf{Link State}: Uses network topology (OSPF)
\item
  \textbf{Hybrid}: Combines both approaches (EIGRP)
\end{itemize}

\textbf{Selection Criteria:}

\begin{itemize}
\tightlist
\item
  \textbf{Shortest path}: Minimum hops or distance
\item
  \textbf{Load balancing}: Distribute traffic evenly
\item
  \textbf{Fault tolerance}: Alternative routes for failures
\end{itemize}

\end{solutionbox}
\begin{mnemonicbox}
``Static Dynamic Default Routes''

\end{mnemonicbox}
\begin{center}\rule{0.5\linewidth}{0.5pt}\end{center}

\subsection*{Question 3(b) [4 marks]}\label{q3b}

\textbf{What is Subnetting and supernetting?}

\begin{solutionbox}

Subnetting and supernetting manage IP address allocation efficiently:

\textbf{Comparison:}

{\def\LTcaptype{none} % do not increment counter
\begin{longtable}[]{@{}lll@{}}
\toprule\noalign{}
Aspect & Subnetting & Supernetting \\
\midrule\noalign{}
\endhead
\bottomrule\noalign{}
\endlastfoot
\textbf{Purpose} & Divide large network & Combine small networks \\
\textbf{Direction} & Top-down approach & Bottom-up approach \\
\textbf{Mask} & Longer subnet mask & Shorter subnet mask \\
\textbf{Result} & Multiple smaller subnets & Single larger network \\
\end{longtable}
}

\textbf{Subnetting Process:}

\begin{itemize}
\tightlist
\item
  \textbf{Borrowing bits}: Take bits from host portion
\item
  \textbf{Create subnets}: Multiple network segments
\item
  \textbf{Reduce broadcast}: Smaller broadcast domains
\end{itemize}

\textbf{Supernetting Process:}

\begin{itemize}
\tightlist
\item
  \textbf{Combine networks}: Merge adjacent networks
\item
  \textbf{Route aggregation}: Single routing entry
\item
  \textbf{Reduce routing table}: Fewer routing entries
\end{itemize}

\textbf{Benefits:}

\begin{itemize}
\tightlist
\item
  \textbf{Subnetting}: Better network management, security
\item
  \textbf{Supernetting}: Simplified routing, reduced overhead
\end{itemize}

\end{solutionbox}
\begin{mnemonicbox}
``Subnetting Splits, Supernetting Sums''

\end{mnemonicbox}
\begin{center}\rule{0.5\linewidth}{0.5pt}\end{center}

\subsection*{Question 3(c) [7 marks]}\label{q3c}

\textbf{Explain IPV6 Addressing. Why need of IPV6 migration?}

\begin{solutionbox}

IPv6 addressing uses 128-bit addresses to solve IPv4 limitations:

\textbf{IPv6 Address Structure:}

\begin{verbatim}
+{-{-}{-}+{-}{-}{-}+{-}{-}{-}+{-}{-}{-}+{-}{-}{-}+{-}{-}{-}+{-}{-}{-}+{-}{-}{-}+{-}{-}{-}+{-}{-}{-}+{-}{-}{-}+{-}{-}{-}+{-}{-}{-}+{-}{-}{-}+{-}{-}{-}+{-}{-}{-}+}
| Global Routing Prefix |Subnet |      Interface Identifier     |
|      (48 bits)        |(16)   |         (64 bits)             |
+{-{-}{-}+{-}{-}{-}+{-}{-}{-}+{-}{-}{-}+{-}{-}{-}+{-}{-}{-}+{-}{-}{-}+{-}{-}{-}+{-}{-}{-}+{-}{-}{-}+{-}{-}{-}+{-}{-}{-}+{-}{-}{-}+{-}{-}{-}+{-}{-}{-}+{-}{-}{-}+}
\end{verbatim}

\textbf{Address Format:}

{\def\LTcaptype{none} % do not increment counter
\begin{longtable}[]{@{}lll@{}}
\toprule\noalign{}
Component & Size & Purpose \\
\midrule\noalign{}
\endhead
\bottomrule\noalign{}
\endlastfoot
\textbf{Global Prefix} & 48 bits & ISP allocation \\
\textbf{Subnet ID} & 16 bits & Organization subnets \\
\textbf{Interface ID} & 64 bits & Device identification \\
\end{longtable}
}

\textbf{Address Types:}

\begin{itemize}
\tightlist
\item
  \textbf{Unicast}: One-to-one communication
\item
  \textbf{Multicast}: One-to-many communication
\item
  \textbf{Anycast}: One-to-nearest communication
\end{itemize}

\textbf{Need for IPv6 Migration:}

\textbf{Critical Issues:}

{\def\LTcaptype{none} % do not increment counter
\begin{longtable}[]{@{}
  >{\raggedright\arraybackslash}p{(\linewidth - 4\tabcolsep) * \real{0.3000}}
  >{\raggedright\arraybackslash}p{(\linewidth - 4\tabcolsep) * \real{0.2000}}
  >{\raggedright\arraybackslash}p{(\linewidth - 4\tabcolsep) * \real{0.5000}}@{}}
\toprule\noalign{}
\begin{minipage}[b]{\linewidth}\raggedright
Problem
\end{minipage} & \begin{minipage}[b]{\linewidth}\raggedright
IPv4
\end{minipage} & \begin{minipage}[b]{\linewidth}\raggedright
IPv6 Solution
\end{minipage} \\
\midrule\noalign{}
\endhead
\bottomrule\noalign{}
\endlastfoot
\textbf{Address Exhaustion} & 4.3 billion addresses & 340 undecillion
addresses \\
\textbf{NAT Complexity} & Required for connectivity & End-to-end
connectivity \\
\textbf{Security} & Add-on feature & Built-in IPSec \\
\textbf{Mobile Support} & Limited & Native mobility \\
\end{longtable}
}

\textbf{Migration Benefits:}

\begin{itemize}
\tightlist
\item
  \textbf{Unlimited growth}: Supports IoT expansion
\item
  \textbf{Simplified configuration}: Auto-configuration features
\item
  \textbf{Better performance}: Optimized header structure
\item
  \textbf{Enhanced security}: Mandatory encryption
\end{itemize}

\textbf{Migration Challenges:}

\begin{itemize}
\tightlist
\item
  \textbf{Dual-stack}: Running both IPv4 and IPv6
\item
  \textbf{Translation}: IPv4-IPv6 interoperability
\item
  \textbf{Training}: Staff education requirements
\end{itemize}

\end{solutionbox}
\begin{mnemonicbox}
``IPv6 Infinite Possibilities, Enhanced Security''

\end{mnemonicbox}
\begin{center}\rule{0.5\linewidth}{0.5pt}\end{center}

\subsection*{Question 3(a OR) [3
marks]}\label{question-3a-or-3-marks}

\textbf{Determine valid IPv4 address from below. If it is a valid IPv4
address then find its class, Network ID and Host ID. If it's an invalid
IPv4 address, then give a reason.}

\textbf{a. 192.108.102.101} \textbf{b. 80.54.256.14}

\begin{solutionbox}

\textbf{Analysis:}

{\def\LTcaptype{none} % do not increment counter
\begin{longtable}[]{@{}
  >{\raggedright\arraybackslash}p{(\linewidth - 10\tabcolsep) * \real{0.1607}}
  >{\raggedright\arraybackslash}p{(\linewidth - 10\tabcolsep) * \real{0.1786}}
  >{\raggedright\arraybackslash}p{(\linewidth - 10\tabcolsep) * \real{0.1250}}
  >{\raggedright\arraybackslash}p{(\linewidth - 10\tabcolsep) * \real{0.2143}}
  >{\raggedright\arraybackslash}p{(\linewidth - 10\tabcolsep) * \real{0.1607}}
  >{\raggedright\arraybackslash}p{(\linewidth - 10\tabcolsep) * \real{0.1607}}@{}}
\toprule\noalign{}
\begin{minipage}[b]{\linewidth}\raggedright
Address
\end{minipage} & \begin{minipage}[b]{\linewidth}\raggedright
Validity
\end{minipage} & \begin{minipage}[b]{\linewidth}\raggedright
Class
\end{minipage} & \begin{minipage}[b]{\linewidth}\raggedright
Network ID
\end{minipage} & \begin{minipage}[b]{\linewidth}\raggedright
Host ID
\end{minipage} & \begin{minipage}[b]{\linewidth}\raggedright
Reason
\end{minipage} \\
\midrule\noalign{}
\endhead
\bottomrule\noalign{}
\endlastfoot
\textbf{192.108.102.101} & Valid & Class C & 192.108.102.0 & 0.0.0.101 &
All octets \leq 255 \\
\textbf{80.54.256.14} & Invalid & - & - & - & Third octet = 256
\textgreater{} 255 \\
\end{longtable}
}

\textbf{Address a: 192.108.102.101}

\begin{itemize}
\tightlist
\item
  \textbf{Valid}: All octets within range (0-255)
\item
  \textbf{Class C}: First octet 192 (192-223 range)
\item
  \textbf{Default mask}: 255.255.255.0 (/24)
\end{itemize}

\textbf{Address b: 80.54.256.14}

\begin{itemize}
\tightlist
\item
  \textbf{Invalid}: Third octet is 256
\item
  \textbf{Rule violation}: Each octet must be 0-255
\item
  \textbf{Correction}: Replace 256 with valid value (0-255)
\end{itemize}

\end{solutionbox}
\begin{mnemonicbox}
``Each Octet Maximum 255''

\end{mnemonicbox}
\begin{center}\rule{0.5\linewidth}{0.5pt}\end{center}

\subsection*{Question 3(b OR) [4
marks]}\label{question-3b-or-4-marks}

\textbf{Write Short note on Network Address Translation.}

\begin{solutionbox}

NAT translates private IP addresses to public IP addresses for internet
access:

\textbf{NAT Process:}

{\def\LTcaptype{none} % do not increment counter
\begin{longtable}[]{@{}
  >{\raggedright\arraybackslash}p{(\linewidth - 4\tabcolsep) * \real{0.2000}}
  >{\raggedright\arraybackslash}p{(\linewidth - 4\tabcolsep) * \real{0.3667}}
  >{\raggedright\arraybackslash}p{(\linewidth - 4\tabcolsep) * \real{0.4333}}@{}}
\toprule\noalign{}
\begin{minipage}[b]{\linewidth}\raggedright
Step
\end{minipage} & \begin{minipage}[b]{\linewidth}\raggedright
Direction
\end{minipage} & \begin{minipage}[b]{\linewidth}\raggedright
Translation
\end{minipage} \\
\midrule\noalign{}
\endhead
\bottomrule\noalign{}
\endlastfoot
\textbf{Outbound} & Private \rightarrow Public & Internal IP mapped to public
IP \\
\textbf{Inbound} & Public \rightarrow Private & Public IP mapped back to internal
IP \\
\end{longtable}
}

\textbf{NAT Types:}

\begin{verbatim}
NAT Types
├── Static NAT (1:1 mapping)
├── Dynamic NAT (Pool mapping)
└── PAT/NAPT (Port translation)
\end{verbatim}

\textbf{Benefits:}

\begin{itemize}
\tightlist
\item
  \textbf{IP conservation}: Multiple devices share one public IP
\item
  \textbf{Security}: Hides internal network structure
\item
  \textbf{Cost reduction}: Fewer public IP addresses needed
\item
  \textbf{Flexibility}: Easy internal network changes
\end{itemize}

\textbf{Limitations:}

\begin{itemize}
\tightlist
\item
  \textbf{End-to-end connectivity}: Breaks direct communication
\item
  \textbf{Protocol issues}: Some protocols don't work through NAT
\item
  \textbf{Performance}: Additional processing overhead
\end{itemize}

\end{solutionbox}
\begin{mnemonicbox}
``NAT Networks Address Translation''

\end{mnemonicbox}
\begin{center}\rule{0.5\linewidth}{0.5pt}\end{center}

\subsection*{Question 3(c OR) [7
marks]}\label{question-3c-or-7-marks}

\textbf{Explain IPV4 Datagram Header in detail.}

\begin{solutionbox}

IPv4 header contains essential information for packet routing:

\begin{verbatim}
 0                   1                   2                   3
 0 1 2 3 4 5 6 7 8 9 0 1 2 3 4 5 6 7 8 9 0 1 2 3 4 5 6 7 8 9 0 1
+{-+{-}+{-}+{-}+{-}+{-}+{-}+{-}+{-}+{-}+{-}+{-}+{-}+{-}+{-}+{-}+{-}+{-}+{-}+{-}+{-}+{-}+{-}+{-}+{-}+{-}+{-}+{-}+{-}+{-}+{-}+{-}+}
|Version|  IHL  |Type of Service|          Total Length         |
+{-+{-}+{-}+{-}+{-}+{-}+{-}+{-}+{-}+{-}+{-}+{-}+{-}+{-}+{-}+{-}+{-}+{-}+{-}+{-}+{-}+{-}+{-}+{-}+{-}+{-}+{-}+{-}+{-}+{-}+{-}+{-}+}
|         Identification        |Flags|      Fragment Offset    |
+{-+{-}+{-}+{-}+{-}+{-}+{-}+{-}+{-}+{-}+{-}+{-}+{-}+{-}+{-}+{-}+{-}+{-}+{-}+{-}+{-}+{-}+{-}+{-}+{-}+{-}+{-}+{-}+{-}+{-}+{-}+{-}+}
|  Time to Live |    Protocol   |         Header Checksum       |
+{-+{-}+{-}+{-}+{-}+{-}+{-}+{-}+{-}+{-}+{-}+{-}+{-}+{-}+{-}+{-}+{-}+{-}+{-}+{-}+{-}+{-}+{-}+{-}+{-}+{-}+{-}+{-}+{-}+{-}+{-}+{-}+}
|                       Source Address                          |
+{-+{-}+{-}+{-}+{-}+{-}+{-}+{-}+{-}+{-}+{-}+{-}+{-}+{-}+{-}+{-}+{-}+{-}+{-}+{-}+{-}+{-}+{-}+{-}+{-}+{-}+{-}+{-}+{-}+{-}+{-}+{-}+}
|                    Destination Address                        |
+{-+{-}+{-}+{-}+{-}+{-}+{-}+{-}+{-}+{-}+{-}+{-}+{-}+{-}+{-}+{-}+{-}+{-}+{-}+{-}+{-}+{-}+{-}+{-}+{-}+{-}+{-}+{-}+{-}+{-}+{-}+{-}+}
|                    Options                    |    Padding    |
+{-+{-}+{-}+{-}+{-}+{-}+{-}+{-}+{-}+{-}+{-}+{-}+{-}+{-}+{-}+{-}+{-}+{-}+{-}+{-}+{-}+{-}+{-}+{-}+{-}+{-}+{-}+{-}+{-}+{-}+{-}+{-}+}
\end{verbatim}

\textbf{Header Fields:}

{\def\LTcaptype{none} % do not increment counter
\begin{longtable}[]{@{}lll@{}}
\toprule\noalign{}
Field & Size & Purpose \\
\midrule\noalign{}
\endhead
\bottomrule\noalign{}
\endlastfoot
\textbf{Version} & 4 bits & IP version (4 for IPv4) \\
\textbf{IHL} & 4 bits & Header length in 32-bit words \\
\textbf{Type of Service} & 8 bits & Quality of service \\
\textbf{Total Length} & 16 bits & Total packet size \\
\textbf{Identification} & 16 bits & Fragment identification \\
\textbf{Flags} & 3 bits & Fragmentation control \\
\textbf{Fragment Offset} & 13 bits & Fragment position \\
\textbf{TTL} & 8 bits & Maximum hops before discard \\
\textbf{Protocol} & 8 bits & Next layer protocol \\
\textbf{Checksum} & 16 bits & Header error detection \\
\textbf{Source Address} & 32 bits & Sender IP address \\
\textbf{Destination} & 32 bits & Receiver IP address \\
\end{longtable}
}

\textbf{Key Functions:}

\begin{itemize}
\tightlist
\item
  \textbf{Routing}: Source and destination addresses
\item
  \textbf{Fragmentation}: Handle large packets
\item
  \textbf{Error detection}: Header checksum
\item
  \textbf{Quality control}: Type of service field
\end{itemize}

\textbf{Important Values:}

\begin{itemize}
\tightlist
\item
  \textbf{Protocol}: TCP=6, UDP=17, ICMP=1
\item
  \textbf{Flags}: Don't Fragment, More Fragments
\item
  \textbf{TTL}: Prevents infinite loops
\end{itemize}

\end{solutionbox}
\begin{mnemonicbox}
``Version IHL Service Length Identify Fragment TTL
Protocol Check Source Destination''

\end{mnemonicbox}
\begin{center}\rule{0.5\linewidth}{0.5pt}\end{center}

\subsection*{Question 4(a) [3 marks]}\label{q4a}

\textbf{Explain working of Indirect TCP.}

\begin{solutionbox}

Indirect TCP splits TCP connection to handle mobile network challenges:

\textbf{Architecture:}

{\def\LTcaptype{none} % do not increment counter
\begin{longtable}[]{@{}lll@{}}
\toprule\noalign{}
Component & Role & Location \\
\midrule\noalign{}
\endhead
\bottomrule\noalign{}
\endlastfoot
\textbf{Mobile Host} & TCP client & Mobile network \\
\textbf{Base Station} & TCP proxy & Fixed network \\
\textbf{Fixed Host} & TCP server & Wired network \\
\end{longtable}
}

\textbf{Connection Split:}

\begin{itemize}
\tightlist
\item
  \textbf{Connection 1}: Mobile Host \leftrightarrow Base Station
\item
  \textbf{Connection 2}: Base Station \leftrightarrow Fixed Host
\item
  \textbf{Proxy function}: Base station acts as TCP proxy
\end{itemize}

\textbf{Working Process:}

\begin{itemize}
\tightlist
\item
  \textbf{Data flow}: Mobile \rightarrow Base Station \rightarrow Fixed Host
\item
  \textbf{ACK handling}: Base station manages acknowledgments
\item
  \textbf{Handover}: Connection maintained during movement
\end{itemize}

\textbf{Advantages:}

\begin{itemize}
\tightlist
\item
  \textbf{Wireless optimization}: Handles wireless link issues
\item
  \textbf{Mobility support}: Seamless handover capability
\item
  \textbf{Error recovery}: Better handling of wireless errors
\end{itemize}

\end{solutionbox}
\begin{mnemonicbox}
``Indirect TCP Through Proxy''

\end{mnemonicbox}
\begin{center}\rule{0.5\linewidth}{0.5pt}\end{center}

\subsection*{Question 4(b) [4 marks]}\label{q4b}

\textbf{Write Short note on Stop and Wait ARQ Protocol.}

\begin{solutionbox}

Stop and Wait ARQ ensures reliable data transmission with error
detection and correction:

\textbf{Protocol Operation:}

{\def\LTcaptype{none} % do not increment counter
\begin{longtable}[]{@{}lll@{}}
\toprule\noalign{}
Step & Action & Purpose \\
\midrule\noalign{}
\endhead
\bottomrule\noalign{}
\endlastfoot
\textbf{Send} & Transmit frame with sequence number & Data delivery \\
\textbf{Wait} & Wait for acknowledgment & Confirm receipt \\
\textbf{Timeout} & Retransmit if no ACK & Handle lost frames \\
\textbf{ACK} & Send acknowledgment for received frame & Confirm
delivery \\
\end{longtable}
}

\textbf{Error Handling:}

\begin{verbatim}
Sender                    Receiver
  |                         |
  |{-{-}{-}{-} Frame 0 {-}{-}{-}{-}{-}{-}{-}{-}{-}{-}|}
  |                         |{-{-}{-}{-} ACK 0}
  |{{-}{-}{-}{-} ACK 0 {-}{-}{-}{-}{-}{-}{-}{-}{-}{-}{-}{-}|}
  |                         |
  |{-{-}{-}{-} Frame 1 {-}{-}{-}{-}{-}{-}{-}{-}{-}{-}| (Lost)}
  |                         |
  |{-{-} Timeout, Retransmit {-}{-}|}
  |{-{-}{-}{-} Frame 1 {-}{-}{-}{-}{-}{-}{-}{-}{-}{-}|}
  |                         |{-{-}{-}{-} ACK 1}
  |{{-}{-}{-}{-} ACK 1 {-}{-}{-}{-}{-}{-}{-}{-}{-}{-}{-}{-}|}
\end{verbatim}

\textbf{Features:}

\begin{itemize}
\tightlist
\item
  \textbf{Sequence numbers}: 0 and 1 alternation
\item
  \textbf{Timeout mechanism}: Handles lost frames/ACKs
\item
  \textbf{Duplicate detection}: Prevents duplicate acceptance
\item
  \textbf{Flow control}: Receiver controls transmission rate
\end{itemize}

\textbf{Limitations:}

\begin{itemize}
\tightlist
\item
  \textbf{Low efficiency}: Only one frame in transit
\item
  \textbf{Bandwidth waste}: Idle time during waiting
\end{itemize}

\end{solutionbox}
\begin{mnemonicbox}
``Stop Send, Wait ACK, Repeat''

\end{mnemonicbox}
\begin{center}\rule{0.5\linewidth}{0.5pt}\end{center}

\subsection*{Question 4(c) [7 marks]}\label{q4c}

\textbf{Explain Communication Middleware in detail.}

\begin{solutionbox}

Communication middleware provides abstraction layer between applications
and network services:

\begin{center}
\textbf{Mermaid Diagram (Code)}
\begin{verbatim}
{Shaded}
{Highlighting}[]
graph LR
    A[Mobile Applications] {-{-}{} B[Communication Middleware]}
    B {-{-}{} C[Network Services]}
    
    B1[Message Passing{br/{}RPC{}br/{}Event Handling] {-}{-}{} B}
    C1[TCP/IP{br/{}Wireless Protocols{}br/{}Network APIs] {-}{-}{} C}
{Highlighting}
{Shaded}
\end{verbatim}
\end{center}

\textbf{Middleware Types:}

{\def\LTcaptype{none} % do not increment counter
\begin{longtable}[]{@{}lll@{}}
\toprule\noalign{}
Type & Function & Example \\
\midrule\noalign{}
\endhead
\bottomrule\noalign{}
\endlastfoot
\textbf{Message-Oriented} & Asynchronous messaging & Message queues \\
\textbf{RPC-based} & Remote procedure calls & CORBA, RMI \\
\textbf{Event-driven} & Event notifications & Publish-subscribe \\
\textbf{Stream-oriented} & Continuous data flow & Multimedia streams \\
\end{longtable}
}

\textbf{Core Services:}

\textbf{Communication Services:}

\begin{itemize}
\tightlist
\item
  \textbf{Message routing}: Efficient message delivery
\item
  \textbf{Protocol conversion}: Different protocol handling
\item
  \textbf{Buffering}: Temporary message storage
\item
  \textbf{Synchronization}: Coordinated communication
\end{itemize}

\textbf{Reliability Services:}

\begin{itemize}
\tightlist
\item
  \textbf{Error detection}: Message integrity checking
\item
  \textbf{Retransmission}: Failed message recovery
\item
  \textbf{Duplicate elimination}: Prevent message duplication
\item
  \textbf{Ordering}: Maintain message sequence
\end{itemize}

\textbf{Mobile-Specific Features:}

\begin{itemize}
\tightlist
\item
  \textbf{Location transparency}: Hide mobility from applications
\item
  \textbf{Disconnection handling}: Manage network interruptions
\item
  \textbf{Bandwidth adaptation}: Adjust to network conditions
\item
  \textbf{Power management}: Optimize battery usage
\end{itemize}

\textbf{Architecture Benefits:}

\begin{itemize}
\tightlist
\item
  \textbf{Abstraction}: Hide network complexity
\item
  \textbf{Portability}: Application independence from network
\item
  \textbf{Scalability}: Support growing number of devices
\item
  \textbf{Interoperability}: Different system communication
\end{itemize}

\textbf{Examples:}

\begin{itemize}
\tightlist
\item
  \textbf{CORBA}: Distributed object communication
\item
  \textbf{Message Queues}: Asynchronous messaging
\item
  \textbf{Web Services}: HTTP-based communication
\end{itemize}

\end{solutionbox}
\begin{mnemonicbox}
``Middleware Manages Mobile Communication''

\end{mnemonicbox}
\begin{center}\rule{0.5\linewidth}{0.5pt}\end{center}

\subsection*{Question 4(a OR) [3
marks]}\label{question-4a-or-3-marks}

\textbf{Explain Handover management in mobile IP.}

\begin{solutionbox}

Handover management maintains connectivity when mobile device moves
between networks:

\textbf{Handover Process:}

{\def\LTcaptype{none} % do not increment counter
\begin{longtable}[]{@{}lll@{}}
\toprule\noalign{}
Phase & Action & Purpose \\
\midrule\noalign{}
\endhead
\bottomrule\noalign{}
\endlastfoot
\textbf{Detection} & Monitor signal strength & Identify need for
handover \\
\textbf{Decision} & Select target network & Choose best network \\
\textbf{Execution} & Switch to new network & Complete handover \\
\end{longtable}
}

\textbf{Types of Handover:}

\begin{itemize}
\tightlist
\item
  \textbf{Horizontal}: Same technology networks
\item
  \textbf{Vertical}: Different technology networks
\item
  \textbf{Hard}: Break-before-make
\item
  \textbf{Soft}: Make-before-break
\end{itemize}

\textbf{Management Components:}

\begin{itemize}
\tightlist
\item
  \textbf{Signal monitoring}: Continuous signal assessment
\item
  \textbf{Network discovery}: Available network identification
\item
  \textbf{Decision algorithm}: Optimal network selection
\end{itemize}

\textbf{Performance Metrics:}

\begin{itemize}
\tightlist
\item
  \textbf{Handover delay}: Time to complete switch
\item
  \textbf{Packet loss}: Data lost during handover
\item
  \textbf{Signaling overhead}: Control message cost
\end{itemize}

\end{solutionbox}
\begin{mnemonicbox}
``Handover Helps Maintain Mobility''

\end{mnemonicbox}
\begin{center}\rule{0.5\linewidth}{0.5pt}\end{center}

\subsection*{Question 4(b OR) [4
marks]}\label{question-4b-or-4-marks}

\textbf{Explain key functions of Communication Gateways.}

\begin{solutionbox}

Communication gateways enable interoperability between different network
systems:

\textbf{Key Functions:}

{\def\LTcaptype{none} % do not increment counter
\begin{longtable}[]{@{}
  >{\raggedright\arraybackslash}p{(\linewidth - 4\tabcolsep) * \real{0.3125}}
  >{\raggedright\arraybackslash}p{(\linewidth - 4\tabcolsep) * \real{0.4062}}
  >{\raggedright\arraybackslash}p{(\linewidth - 4\tabcolsep) * \real{0.2812}}@{}}
\toprule\noalign{}
\begin{minipage}[b]{\linewidth}\raggedright
Function
\end{minipage} & \begin{minipage}[b]{\linewidth}\raggedright
Description
\end{minipage} & \begin{minipage}[b]{\linewidth}\raggedright
Benefit
\end{minipage} \\
\midrule\noalign{}
\endhead
\bottomrule\noalign{}
\endlastfoot
\textbf{Protocol Translation} & Convert between protocols &
Interoperability \\
\textbf{Data Format Conversion} & Transform data formats &
Compatibility \\
\textbf{Security Enforcement} & Apply security policies & Protection \\
\textbf{Load Balancing} & Distribute traffic & Performance \\
\end{longtable}
}

\textbf{Gateway Services:}

\textbf{Protocol Services:}

\begin{itemize}
\tightlist
\item
  \textbf{Multi-protocol support}: Handle various protocols
\item
  \textbf{Translation efficiency}: Fast protocol conversion
\item
  \textbf{Standards compliance}: Follow protocol specifications
\end{itemize}

\textbf{Security Services:}

\begin{itemize}
\tightlist
\item
  \textbf{Authentication}: Verify user identity
\item
  \textbf{Authorization}: Control access permissions
\item
  \textbf{Encryption}: Protect data transmission
\item
  \textbf{Firewall}: Filter malicious traffic
\end{itemize}

\textbf{Performance Services:}

\begin{itemize}
\tightlist
\item
  \textbf{Caching}: Store frequently accessed data
\item
  \textbf{Compression}: Reduce data size
\item
  \textbf{Traffic shaping}: Manage bandwidth usage
\item
  \textbf{Quality of Service}: Prioritize critical traffic
\end{itemize}

\textbf{Management Features:}

\begin{itemize}
\tightlist
\item
  \textbf{Monitoring}: Track gateway performance
\item
  \textbf{Configuration}: Flexible setup options
\item
  \textbf{Logging}: Record activity and errors
\end{itemize}

\end{solutionbox}
\begin{mnemonicbox}
``Gateways Grant Protocol Interoperability''

\end{mnemonicbox}
\begin{center}\rule{0.5\linewidth}{0.5pt}\end{center}

\subsection*{Question 4(c OR) [7
marks]}\label{question-4c-or-7-marks}

\textbf{Explain Process of mobile IP.}

\begin{solutionbox}

Mobile IP enables device mobility while maintaining IP connectivity:

\begin{verbatim}
sequenceDiagram
    participant MN as Mobile Node
    participant HA as Home Agent
    participant FA as Foreign Agent
    participant CN as Correspondent Node
    
    MN{-FA: Agent Solicitation}
    FA{-MN: Agent Advertisement}
    MN{-HA: Registration Request}
    HA{-MN: Registration Reply}
    CN{-HA: Data Packet (Home Address)}
    HA{-FA: Tunneled Packet}
    FA{-MN: Data Packet}
\end{verbatim}

\textbf{Mobile IP Components:}

{\def\LTcaptype{none} % do not increment counter
\begin{longtable}[]{@{}lll@{}}
\toprule\noalign{}
Component & Role & Function \\
\midrule\noalign{}
\endhead
\bottomrule\noalign{}
\endlastfoot
\textbf{Mobile Node} & Moving device & Maintains connectivity \\
\textbf{Home Agent} & Home network router & Forwards packets \\
\textbf{Foreign Agent} & Visited network router & Local delivery \\
\textbf{Care-of Address} & Temporary address & Current location \\
\end{longtable}
}

\textbf{Registration Process:}

\textbf{Phase 1: Agent Discovery}

\begin{itemize}
\tightlist
\item
  \textbf{Advertisement}: Agents broadcast availability
\item
  \textbf{Solicitation}: Mobile node requests agent info
\item
  \textbf{Selection}: Choose appropriate foreign agent
\end{itemize}

\textbf{Phase 2: Registration}

\begin{itemize}
\tightlist
\item
  \textbf{Request}: Mobile node registers with home agent
\item
  \textbf{Authentication}: Verify mobile node identity
\item
  \textbf{Binding}: Create care-of address binding
\item
  \textbf{Confirmation}: Registration acknowledgment
\end{itemize}

\textbf{Phase 3: Packet Delivery}

\begin{itemize}
\tightlist
\item
  \textbf{Interception}: Home agent intercepts packets
\item
  \textbf{Tunneling}: Encapsulate and forward packets
\item
  \textbf{Decapsulation}: Foreign agent extracts packets
\item
  \textbf{Local delivery}: Forward to mobile node
\end{itemize}

\textbf{Tunneling Mechanism:}

\begin{verbatim}
Original Packet: [IP Header|Data]
                 Dest: Home Address

Tunneled Packet: [New IP Header|Original Packet]
                 Dest: Care{-of Address}
\end{verbatim}

\textbf{Key Features:}

\begin{itemize}
\tightlist
\item
  \textbf{Transparency}: Applications unaware of mobility
\item
  \textbf{Triangle routing}: Indirect packet delivery
\item
  \textbf{Location privacy}: Hide actual location
\item
  \textbf{Seamless handover}: Maintain connections
\end{itemize}

\textbf{Challenges:}

\begin{itemize}
\tightlist
\item
  \textbf{Triangle routing}: Inefficient packet path
\item
  \textbf{Ingress filtering}: Firewall compatibility
\item
  \textbf{Security}: Authentication and encryption
\end{itemize}

\end{solutionbox}
\begin{mnemonicbox}
``Mobile IP: Discover Register Tunnel Deliver''

\end{mnemonicbox}
\begin{center}\rule{0.5\linewidth}{0.5pt}\end{center}

\subsection*{Question 5(a) [3 marks]}\label{q5a}

\textbf{List advantages of WPANs.}

\begin{solutionbox}

WPAN (Wireless Personal Area Network) provides short-range connectivity
benefits:

\textbf{Key Advantages:}

{\def\LTcaptype{none} % do not increment counter
\begin{longtable}[]{@{}lll@{}}
\toprule\noalign{}
Advantage & Description & Benefit \\
\midrule\noalign{}
\endhead
\bottomrule\noalign{}
\endlastfoot
\textbf{Low Power} & Minimal battery consumption & Extended device
life \\
\textbf{Low Cost} & Inexpensive implementation & Affordable
deployment \\
\textbf{Easy Setup} & Simple configuration & User-friendly \\
\end{longtable}
}

\textbf{Technical Benefits:}

\begin{itemize}
\tightlist
\item
  \textbf{Short range}: 10-30 feet coverage reduces interference
\item
  \textbf{Ad-hoc networking}: No infrastructure required
\item
  \textbf{Device mobility}: Move freely within range
\item
  \textbf{Automatic discovery}: Devices find each other automatically
\end{itemize}

\textbf{Application Advantages:}

\begin{itemize}
\tightlist
\item
  \textbf{Personal devices}: Connect phones, tablets, headphones
\item
  \textbf{IoT integration}: Smart home device connectivity
\item
  \textbf{File sharing}: Quick data transfer between devices
\item
  \textbf{Peripheral connection}: Wireless keyboards, mice
\end{itemize}

\textbf{Security Benefits:}

\begin{itemize}
\tightlist
\item
  \textbf{Limited range}: Reduced eavesdropping risk
\item
  \textbf{Encryption}: Built-in security protocols
\item
  \textbf{Pairing}: Authenticated device connections
\end{itemize}

\end{solutionbox}
\begin{mnemonicbox}
``WPANs: Wireless Personal Area Networks''

\end{mnemonicbox}
\begin{center}\rule{0.5\linewidth}{0.5pt}\end{center}

\subsection*{Question 5(b) [4 marks]}\label{q5b}

\textbf{Explain steps of packet delivery in mobile IP.}

\begin{solutionbox}

Mobile IP packet delivery involves multiple steps to reach mobile
devices:

\textbf{Packet Delivery Steps:}

{\def\LTcaptype{none} % do not increment counter
\begin{longtable}[]{@{}
  >{\raggedright\arraybackslash}p{(\linewidth - 4\tabcolsep) * \real{0.2400}}
  >{\raggedright\arraybackslash}p{(\linewidth - 4\tabcolsep) * \real{0.3600}}
  >{\raggedright\arraybackslash}p{(\linewidth - 4\tabcolsep) * \real{0.4000}}@{}}
\toprule\noalign{}
\begin{minipage}[b]{\linewidth}\raggedright
Step
\end{minipage} & \begin{minipage}[b]{\linewidth}\raggedright
Process
\end{minipage} & \begin{minipage}[b]{\linewidth}\raggedright
Location
\end{minipage} \\
\midrule\noalign{}
\endhead
\bottomrule\noalign{}
\endlastfoot
\textbf{1. Transmission} & Send packet to home address & Correspondent
Node \\
\textbf{2. Interception} & Capture packet for mobile node & Home
Agent \\
\textbf{3. Tunneling} & Encapsulate and forward & Home to Foreign
Agent \\
\textbf{4. Delivery} & Extract and deliver packet & Foreign Agent to
Mobile \\
\end{longtable}
}

\textbf{Detailed Process:}

\begin{verbatim}
CN {-{-}{-}{-}{-} HA {-}{-}{-}{-}{-} FA {-}{-}{-}{-}{-} MN}
   (1)     (2,3)    (4)
   
Step 1: Normal IP routing to home network
Step 2: Home Agent intercepts packet
Step 3: Tunnel packet to care{-of address  }
Step 4: Foreign Agent delivers to mobile node
\end{verbatim}

\textbf{Tunneling Mechanism:}

\begin{itemize}
\tightlist
\item
  \textbf{Encapsulation}: Add new IP header with care-of address
\item
  \textbf{Forwarding}: Route through internet to foreign network
\item
  \textbf{Decapsulation}: Remove tunnel header at foreign agent
\item
  \textbf{Local delivery}: Standard delivery to mobile node
\end{itemize}

\end{solutionbox}
\begin{mnemonicbox}
``Correspondent Home Foreign Mobile''

\end{mnemonicbox}
\begin{center}\rule{0.5\linewidth}{0.5pt}\end{center}

\subsection*{Question 5(c) [7 marks]}\label{q5c}

\textbf{Briefly Explain architecture of WLAN with diagram.}

\begin{solutionbox}

WLAN (Wireless Local Area Network) architecture provides wireless
connectivity within local area:

\begin{center}
\textbf{Mermaid Diagram (Code)}
\begin{verbatim}
{Shaded}
{Highlighting}[]
graph LR
    A[Distribution System{br/{}Wired Backbone] {-}{-}{} B[Access Point 1]}
    A {-{-}{} C[Access Point 2]}
    A {-{-}{} D[Access Point 3]}
    
    B {-{-}{} E[BSS 1{}br/{}Basic Service Set]}
    C {-{-}{} F[BSS 2{}br/{}Basic Service Set]}
    D {-{-}{} G[BSS 3{}br/{}Basic Service Set]}
    
    E {-{-}{} H[Wireless Stations]}
    F {-{-}{} I[Wireless Stations]}
    G {-{-}{} J[Wireless Stations]}
    
    K[ESS {- Extended Service Set] {-}{-}{} A}
{Highlighting}
{Shaded}
\end{verbatim}
\end{center}

\textbf{WLAN Components:}

{\def\LTcaptype{none} % do not increment counter
\begin{longtable}[]{@{}lll@{}}
\toprule\noalign{}
Component & Function & Coverage \\
\midrule\noalign{}
\endhead
\bottomrule\noalign{}
\endlastfoot
\textbf{Station (STA)} & Wireless device & Individual device \\
\textbf{Access Point (AP)} & Wireless hub & Basic Service Set \\
\textbf{Basic Service Set (BSS)} & Single AP coverage & Local area \\
\textbf{Extended Service Set (ESS)} & Multiple BSS & Large area \\
\end{longtable}
}

\textbf{Architecture Types:}

\textbf{Ad-hoc Mode:}

\begin{itemize}
\tightlist
\item
  \textbf{Independent BSS}: No access point required
\item
  \textbf{Peer-to-peer}: Direct station communication
\item
  \textbf{Limited range}: Single hop communication
\item
  \textbf{Temporary networks}: Conference, meeting rooms
\end{itemize}

\textbf{Infrastructure Mode:}

\begin{itemize}
\tightlist
\item
  \textbf{Access Point}: Central coordination
\item
  \textbf{Distribution System}: Connect multiple APs
\item
  \textbf{Roaming support}: Move between BSS areas
\item
  \textbf{Internet connectivity}: Gateway to external networks
\end{itemize}

\textbf{Key Features:}

\begin{itemize}
\tightlist
\item
  \textbf{Mobility}: Move within coverage area
\item
  \textbf{Scalability}: Add more access points
\item
  \textbf{Interoperability}: IEEE 802.11 standards
\item
  \textbf{Security}: WPA/WPA2 encryption
\end{itemize}

\textbf{Services Provided:}

\begin{itemize}
\tightlist
\item
  \textbf{Association}: Connect to access point
\item
  \textbf{Authentication}: Verify user credentials
\item
  \textbf{Data delivery}: Reliable frame transmission
\item
  \textbf{Power management}: Battery optimization
\end{itemize}

\textbf{Standards:}

\begin{itemize}
\tightlist
\item
  \textbf{802.11a}: 5 GHz, 54 Mbps
\item
  \textbf{802.11b}: 2.4 GHz, 11 Mbps
\item
  \textbf{802.11g}: 2.4 GHz, 54 Mbps
\item
  \textbf{802.11n}: MIMO, 600 Mbps
\item
  \textbf{802.11ac}: 5 GHz, 1 Gbps+
\end{itemize}

\end{solutionbox}
\begin{mnemonicbox}
``WLAN: Wireless Local Area Network''

\end{mnemonicbox}
\begin{center}\rule{0.5\linewidth}{0.5pt}\end{center}

\subsection*{Question 5(a OR) [3
marks]}\label{question-5a-or-3-marks}

\textbf{Explain 5G mobile network features in detail.}

\begin{solutionbox}

5G provides revolutionary mobile network capabilities:

\textbf{Key Features:}

{\def\LTcaptype{none} % do not increment counter
\begin{longtable}[]{@{}lll@{}}
\toprule\noalign{}
Feature & Specification & Benefit \\
\midrule\noalign{}
\endhead
\bottomrule\noalign{}
\endlastfoot
\textbf{Speed} & Up to 10 Gbps & Ultra-fast downloads \\
\textbf{Latency} & Less than 1ms & Real-time applications \\
\textbf{Density} & 1M devices/km^{2} & Massive IoT support \\
\end{longtable}
}

\textbf{Technical Capabilities:}

\begin{itemize}
\tightlist
\item
  \textbf{Enhanced Mobile Broadband}: High-speed internet access
\item
  \textbf{Ultra-Reliable Low Latency}: Critical applications
\item
  \textbf{Massive Machine Communication}: IoT device connectivity
\end{itemize}

\textbf{Advanced Technologies:}

\begin{itemize}
\tightlist
\item
  \textbf{Millimeter waves}: Higher frequency bands
\item
  \textbf{MIMO}: Multiple antenna systems
\item
  \textbf{Network slicing}: Virtual network partitions
\item
  \textbf{Edge computing}: Distributed processing
\end{itemize}

\textbf{Applications:}

\begin{itemize}
\tightlist
\item
  \textbf{Autonomous vehicles}: Real-time control
\item
  \textbf{Smart cities}: Connected infrastructure
\item
  \textbf{Industrial IoT}: Factory automation
\end{itemize}

\end{solutionbox}
\begin{mnemonicbox}
``5G: Fifth Generation Great Speed''

\end{mnemonicbox}
\begin{center}\rule{0.5\linewidth}{0.5pt}\end{center}

\subsection*{Question 5(b OR) [4
marks]}\label{question-5b-or-4-marks}

\textbf{Explain how DHCP works in a mobile network context.}

\begin{solutionbox}

DHCP (Dynamic Host Configuration Protocol) automatically assigns IP
addresses in mobile networks:

\textbf{DHCP Process in Mobile Networks:}

{\def\LTcaptype{none} % do not increment counter
\begin{longtable}[]{@{}llll@{}}
\toprule\noalign{}
Step & Message & Purpose & Direction \\
\midrule\noalign{}
\endhead
\bottomrule\noalign{}
\endlastfoot
\textbf{1} & DHCP Discover & Find DHCP server & Client \rightarrow Broadcast \\
\textbf{2} & DHCP Offer & Offer IP address & Server \rightarrow Client \\
\textbf{3} & DHCP Request & Request specific IP & Client \rightarrow Server \\
\textbf{4} & DHCP ACK & Confirm assignment & Server \rightarrow Client \\
\end{longtable}
}

\textbf{Mobile Network Challenges:}

\begin{verbatim}
Mobile DHCP Process:

Device moves: Network A  Network B

Network A          Network B
DHCP Server        DHCP Server
    |                  |
    |{-{-} Release       |}
    |                  |
    |              Discover |
    |               Offer {-{-}|}
    |              Request |
    |               ACK {-{-}{-}{-}|}
\end{verbatim}

\textbf{Mobile-Specific Features:}

\begin{itemize}
\tightlist
\item
  \textbf{Fast handover}: Quick IP assignment during movement
\item
  \textbf{Lease renewal}: Extend IP address validity
\item
  \textbf{Conflict resolution}: Handle duplicate addresses
\item
  \textbf{Location update}: Notify network of device location
\end{itemize}

\textbf{Configuration Information:}

\begin{itemize}
\tightlist
\item
  \textbf{IP address}: Unique network identifier
\item
  \textbf{Subnet mask}: Network boundary definition
\item
  \textbf{Default gateway}: Router for external communication
\item
  \textbf{DNS servers}: Domain name resolution
\end{itemize}

\textbf{Advantages in Mobile Context:}

\begin{itemize}
\tightlist
\item
  \textbf{Automatic configuration}: No manual setup required
\item
  \textbf{Address conservation}: Reuse addresses efficiently
\item
  \textbf{Mobility support}: Seamless network transitions
\end{itemize}

\end{solutionbox}
\begin{mnemonicbox}
``DHCP: Discover Offer Request ACK''

\end{mnemonicbox}
\begin{center}\rule{0.5\linewidth}{0.5pt}\end{center}

\subsection*{Question 5(c OR) [7
marks]}\label{question-5c-or-7-marks}

\textbf{Explain Bluetooth technology with a neat figure of its protocol
stack.}

\begin{solutionbox}

Bluetooth provides short-range wireless communication for personal
devices:

\begin{center}
\textbf{Mermaid Diagram (Code)}
\begin{verbatim}
{Shaded}
{Highlighting}[]
graph LR
    A[Applications] {-{-}{} B[Application Layer]}
    B {-{-}{} C[L2CAP{}br/{}Logical Link Control]}
    C {-{-}{} D[HCI{}br/{}Host Controller Interface]}
    D {-{-}{} E[Link Manager Protocol{}br/{}LMP]}
    E {-{-}{} F[Baseband Layer]}
    F {-{-}{} G[Radio Layer]}
    
    H[RFCOMM{br/{}Serial Port] {-}{-}{} C}
    I[SDP{br/{}Service Discovery] {-}{-}{} C}
    J[OBEX{br/{}Object Exchange] {-}{-}{} B}
{Highlighting}
{Shaded}
\end{verbatim}
\end{center}

\textbf{Protocol Stack Layers:}

{\def\LTcaptype{none} % do not increment counter
\begin{longtable}[]{@{}lll@{}}
\toprule\noalign{}
Layer & Function & Purpose \\
\midrule\noalign{}
\endhead
\bottomrule\noalign{}
\endlastfoot
\textbf{Radio} & Physical transmission & 2.4 GHz ISM band \\
\textbf{Baseband} & Media access control & Time division duplex \\
\textbf{LMP} & Link management & Connection establishment \\
\textbf{HCI} & Host-controller interface & Hardware abstraction \\
\textbf{L2CAP} & Logical link control & Packet segmentation \\
\textbf{Applications} & User services & File transfer, audio \\
\end{longtable}
}

\textbf{Technical Specifications:}

\textbf{Physical Layer:}

\begin{itemize}
\tightlist
\item
  \textbf{Frequency}: 2.4 GHz ISM band
\item
  \textbf{Hopping}: 79 frequency channels
\item
  \textbf{Modulation}: Frequency shift keying
\item
  \textbf{Power classes}: 1mW to 100mW
\end{itemize}

\textbf{Network Topology:}

\begin{verbatim}
Bluetooth Piconet:

    [Slave 1]
         |
[Slave 2]{-{-}{-}[Master]{-}{-}{-}[Slave 3]}
         |
    [Slave 4]

Max 8 devices per piconet
Master controls communication
\end{verbatim}

\textbf{Connection Types:}

\begin{itemize}
\tightlist
\item
  \textbf{SCO}: Synchronous Connection-Oriented (voice)
\item
  \textbf{ACL}: Asynchronous Connection-Less (data)
\item
  \textbf{eSCO}: Enhanced SCO (improved voice)
\end{itemize}

\textbf{Security Features:}

\begin{itemize}
\tightlist
\item
  \textbf{Authentication}: Device identity verification
\item
  \textbf{Authorization}: Service access control
\item
  \textbf{Encryption}: Data protection (E0 algorithm)
\item
  \textbf{Key management}: Security key exchange
\end{itemize}

\textbf{Bluetooth Versions:}

{\def\LTcaptype{none} % do not increment counter
\begin{longtable}[]{@{}llll@{}}
\toprule\noalign{}
Version & Speed & Range & Features \\
\midrule\noalign{}
\endhead
\bottomrule\noalign{}
\endlastfoot
\textbf{1.x} & 1 Mbps & 10m & Basic connectivity \\
\textbf{2.x} & 3 Mbps & 10m & Enhanced data rate \\
\textbf{3.x} & 24 Mbps & 10m & High-speed option \\
\textbf{4.x} & 1 Mbps & 50m & Low energy (BLE) \\
\textbf{5.x} & 2 Mbps & 240m & Improved range/speed \\
\end{longtable}
}

\textbf{Applications:}

\begin{itemize}
\tightlist
\item
  \textbf{Audio streaming}: Headphones, speakers
\item
  \textbf{File transfer}: Documents, photos
\item
  \textbf{Input devices}: Keyboards, mice
\item
  \textbf{Health monitoring}: Fitness trackers
\end{itemize}

\textbf{Advantages:}

\begin{itemize}
\tightlist
\item
  \textbf{Low power}: Battery-friendly operation
\item
  \textbf{Easy pairing}: Simple device connection
\item
  \textbf{Interoperability}: Universal standard
\item
  \textbf{Cost-effective}: Inexpensive implementation
\end{itemize}

\end{solutionbox}
\begin{mnemonicbox}
``Bluetooth: Radio Baseband LMP HCI L2CAP
Applications''

\end{mnemonicbox}

\end{document}
