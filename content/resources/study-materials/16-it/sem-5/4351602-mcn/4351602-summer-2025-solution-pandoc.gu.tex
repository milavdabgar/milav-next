\documentclass[10pt,a4paper]{article}

% content/resources/templates/preamble.tex
\usepackage[margin=0.6in]{geometry}
\author{Milav Dabgar}
\usepackage{amsmath,amssymb,amsthm}
\usepackage{booktabs}
\usepackage{multirow}
\usepackage{xcolor}
\usepackage{tcolorbox}
\tcbuselibrary{breakable,skins}
\usepackage[colorlinks=true,linkcolor=blue]{hyperref}
\usepackage{titlesec}
\usepackage{enumitem}
\usepackage{tikz}
\usepackage{pgfplots}
\usepackage{circuitikz}
\usepackage[version=4]{mhchem}
\usepackage{longtable}
\usepackage{array}
\usepackage{float}
\usepackage{caption}
\usepackage{listings}

\lstset{
  basicstyle=\small\ttfamily,
  breaklines=true,
  breakatwhitespace=false,
  postbreak=\mbox{\textcolor{red}{$\hookrightarrow$}\space},
  float=false,
  numbers=left,
  numberstyle=\tiny\color{gray},
  numbersep=10pt,
  xleftmargin=2em,
  keywordstyle=\color{blue},
  commentstyle=\color{green!60!black},
  stringstyle=\color{purple},
  backgroundcolor=\color{gray!5},
  showstringspaces=false,
  tabsize=2,
  captionpos=b,
  keepspaces=true,
  columns=flexible
}

\pgfplotsset{compat=1.18}
\usetikzlibrary{shapes,arrows,positioning,calc,patterns,decorations.pathmorphing,decorations.markings,arrows.meta}

% Color scheme
\definecolor{headcolor}{RGB}{0,102,204}
\definecolor{keycolor}{RGB}{220,20,60}
\definecolor{solutioncolor}{RGB}{34,139,34}
\definecolor{mnemoniccolor}{RGB}{148,0,211}
\definecolor{codecolor}{RGB}{0,0,100}

% Spacing
\setlength{\parskip}{3pt}
\setlist[itemize]{nosep}
\setlist[enumerate]{nosep}

% Title formatting
\titleformat{\section}{\Large\bfseries\color{headcolor}}{\thesection}{1em}{}
\titleformat{\subsection}{\large\bfseries\color{headcolor}}{\thesubsection}{1em}{}

% Pandoc tightlist compatibility
\providecommand{\tightlist}{%
  \setlength{\itemsep}{0pt}\setlength{\parskip}{0pt}}

% Pandoc longtable compatibility
\newcounter{none}
\def\thenone{}


% content/resources/templates/gujarati-boxes.tex
\usepackage{fontspec}
\usepackage{polyglossia}

% Set Gujarati as main language (document is primarily in Gujarati)
% Note: gloss-gujarati.ldf doesn't exist in polyglossia, but it will use hyphenation patterns
\setdefaultlanguage{gujarati}
\setotherlanguage{english}

% Configure Gujarati font properly
% Use Language=Default to prevent polyglossia from trying to add language-specific features
% that don't exist for Gujarati, which causes "empty feature" warnings
\newfontfamily\gujaratifont[Script=Gujarati,AutoFakeBold=2.5,AutoFakeSlant=0.3]{Noto Sans Gujarati}
\setmainfont[Script=Gujarati,AutoFakeBold=2.5,AutoFakeSlant=0.3]{Noto Sans Gujarati}
% Use Noto Sans Gujarati for monospace to support Gujarati in text
\setmonofont[Scale=0.9]{Noto Sans Gujarati}

% Configure English to use the same font
\newfontfamily\englishfont[Script=Gujarati,AutoFakeBold=2.5,AutoFakeSlant=0.3]{Noto Sans Gujarati}

% Translations for polyglossia
\gappto\captionsgujarati{
  \renewcommand{\tablename}{કોષ્ટક}
  \renewcommand{\figurename}{આકૃતિ}
}

% Helper for TikZ nodes to ensure Gujarati font
\newcommand{\gu}[1]{{\gujaratifont #1}}

% Custom environments
\newtcolorbox{solutionbox}{
    breakable,
    enhanced,
    colback=solutioncolor!5!white,
    colframe=solutioncolor!75!black,
    fonttitle=\bfseries,
    title=જવાબ
}

\newtcolorbox{solutionboxnobreak}{
 colback=solutioncolor!5!white,
 colframe=solutioncolor!75!black,
 fonttitle=\bfseries,
 title=જવાબ
}

\newtcolorbox{keyformula}{
 breakable,
 enhanced,
 colback=keycolor!5!white,
 colframe=keycolor!75!black,
 fonttitle=\bfseries,
 title=રાસાયણિક સમીકરણ/સૂત્ર
}

\newtcolorbox{mnemonicbox}{
 breakable,
 enhanced,
 colback=mnemoniccolor!5!white,
 colframe=mnemoniccolor!75!black,
 fonttitle=\bfseries,
 title=મેમરી ટ્રીક
}


\begin{document}

\begin{center}
{\Huge\bfseries\color{headcolor} Subject Name (Gujarati)}\\[5pt]
{\LARGE 4351602 -- Summer 2025}\\[3pt]
{\large Semester 1 Study Material}\\[3pt]
{\normalsize\textit{Detailed Solutions and Explanations}}
\end{center}

\vspace{10pt}

\subsection*{પ્રશ્ન 1(અ) [3
ગુણ]}\label{uxaaauxab0uxab6uxaa8-1uxa85-3-uxa97uxaa3}

\textbf{POP પ્રોટોકોલની કામગીરી સમજાવો.}

\begin{solutionbox}

POP (Post Office Protocol) એ ઈમેલ પુનઃપ્રાપ્તિ પ્રોટોકોલ છે જે સર્વરથી ક્લાયન્ટ
ડિવાઇસ પર ઈમેલ્સ ડાઉનલોડ કરે છે.

\textbf{કામગીરીની પ્રક્રિયા:}

{\def\LTcaptype{none} % do not increment counter
\begin{longtable}[]{@{}lll@{}}
\toprule\noalign{}
પગલું & ક્રિયા & વર્ણન \\
\midrule\noalign{}
\endhead
\bottomrule\noalign{}
\endlastfoot
1 & કનેક્શન & ક્લાયન્ટ POP સર્વર સાથે પોર્ટ 110 પર જોડાય છે \\
2 & ઓથેન્ટિકેશન & વપરાશકર્તા યુઝરનેમ અને પાસવર્ડ આપે છે \\
3 & ડાઉનલોડ & ઈમેલ્સ લોકલ ડિવાઇસ પર ડાઉનલોડ થાય છે \\
4 & ડિલીશન & ડાઉનલોડ પછી સર્વરથી ઈમેલ્સ ડિલીટ થાય છે \\
\end{longtable}
}

\begin{itemize}
\tightlist
\item
  \textbf{ડાઉનલોડ-આધારિત}: ઈમેલ્સ ક્લાયન્ટ ડિવાઇસ પર સ્થાનિક રીતે સંગ્રહિત થાય છે
\item
  \textbf{ઓફલાઇન એક્સેસ}: ઈન્ટરનેટ કનેક્શન વગર ઈમેલ્સ વાંચી શકાય છે
\item
  \textbf{સિંગલ ડિવાઇસ}: એક જ ડિવાઇસ એક્સેસ માટે શ્રેષ્ઠ
\end{itemize}

\end{solutionbox}
\begin{mnemonicbox}
``POP ડાઉનલોડ કરે અને કાયમ માટે''

\end{mnemonicbox}
\begin{center}\rule{0.5\linewidth}{0.5pt}\end{center}

\subsection*{પ્રશ્ન 1(બ) [4
ગુણ]}\label{uxaaauxab0uxab6uxaa8-1uxaac-4-uxa97uxaa3}

\textbf{OSI મોડલની TCP/IP મોડલ સાથે સરખામણી કરો.}

\begin{solutionbox}

OSI અને TCP/IP નેટવર્કિંગ મોડલ્સ વચ્ચેની સરખામણી:

{\def\LTcaptype{none} % do not increment counter
\begin{longtable}[]{@{}lll@{}}
\toprule\noalign{}
પાસું & OSI મોડલ & TCP/IP મોડલ \\
\midrule\noalign{}
\endhead
\bottomrule\noalign{}
\endlastfoot
\textbf{લેયર્સ} & 7 લેયર્સ & 4 લેયર્સ \\
\textbf{અભિગમ} & સૈદ્ધાંતિક મોડલ & વ્યવહારિક અમલીકરણ \\
\textbf{વિકાસ} & ISO સ્ટાન્ડર્ડ & DARPA પ્રોજેક્ટ \\
\textbf{જટિલતા} & વધુ જટિલ & સરળ બંધારણ \\
\end{longtable}
}

\textbf{મુખ્ય તફાવતો:}

\begin{itemize}
\tightlist
\item
  \textbf{લેયર કાઉન્ટ}: OSI માં 7 લેયર્સ છે જ્યારે TCP/IP માં 4 લેયર્સ છે
\item
  \textbf{વાસ્તવિક વપરાશ}: TCP/IP વ્યાપકપણે અમલમાં છે, OSI મોટે ભાગે સૈદ્ધાંતિક
\item
  \textbf{પ્રોટોકોલ સ્વતંત્રતા}: OSI પ્રોટોકોલ-સ્વતંત્ર છે, TCP/IP
  પ્રોટોકોલ-વિશિષ્ટ છે
\item
  \textbf{હેડર ઓવરહેડ}: વધારાની લેયર્સને કારણે OSI માં વધુ ઓવરહેડ છે
\end{itemize}

\end{solutionbox}
\begin{mnemonicbox}
``OSI સાત સૈદ્ધાંતિક, TCP ચાર વ્યવહારિક''

\end{mnemonicbox}
\begin{center}\rule{0.5\linewidth}{0.5pt}\end{center}

\subsection*{પ્રશ્ન 1(ક) [7
ગુણ]}\label{uxaaauxab0uxab6uxaa8-1uxa95-7-uxa97uxaa3}

\textbf{TCP/IP મોડલના દરેક લેયરના પ્રોટોકોલ્સની કામગીરી સમજાવો.}

\begin{solutionbox}

TCP/IP મોડલમાં 4 લેયર્સ છે જેમાં દરેક લેયર પર વિશિષ્ટ પ્રોટોકોલ્સ છે:

\begin{center}
\textbf{Mermaid Diagram (Code)}
\begin{verbatim}
{Shaded}
{Highlighting}[]
graph LR
    A[Application Layer] {-{-}{} B[Transport Layer]}
    B {-{-}{} C[Internet Layer]}
    C {-{-}{} D[Network Access Layer]}
    
    A1[HTTP, HTTPS, FTP, SMTP, POP, IMAP, DNS] {-{-}{} A}
    B1[TCP, UDP] {-{-}{} B}
    C1[IP, ICMP, ARP, RARP] {-{-}{} C}
    D1[Ethernet, WiFi, PPP] {-{-}{} D}
{Highlighting}
{Shaded}
\end{verbatim}
\end{center}

\textbf{લેયર મુજબ પ્રોટોકોલ કાર્યો:}

{\def\LTcaptype{none} % do not increment counter
\begin{longtable}[]{@{}lll@{}}
\toprule\noalign{}
લેયર & પ્રોટોકોલ્સ & કાર્ય \\
\midrule\noalign{}
\endhead
\bottomrule\noalign{}
\endlastfoot
\textbf{Application} & HTTP, FTP, SMTP, DNS & વપરાશકર્તા ઈન્ટરફેસ અને
સેવાઓ \\
\textbf{Transport} & TCP, UDP & અંત-થી-અંત સંદેશાવ્યવહાર \\
\textbf{Internet} & IP, ICMP, ARP & રાઉટિંગ અને એડ્રેસિંગ \\
\textbf{Network Access} & Ethernet, WiFi & ભૌતિક ટ્રાન્સમિશન \\
\end{longtable}
}

\textbf{પ્રોટોકોલ વિગતો:}

\begin{itemize}
\tightlist
\item
  \textbf{HTTP/HTTPS}: વેબ કમ્યુનિકેશન અને સુરક્ષિત વેબ કમ્યુનિકેશન
\item
  \textbf{TCP}: વિશ્વસનીય, કનેક્શન-ઓરિએન્ટેડ ડેટા ટ્રાન્સફર
\item
  \textbf{UDP}: ઝડપી, કનેક્શન-રહિત ડેટા ટ્રાન્સફર
\item
  \textbf{IP}: પેકેટ રાઉટિંગ અને એડ્રેસિંગ
\item
  \textbf{ARP}: IP એડ્રેસને MAC એડ્રેસ સાથે મેપ કરે છે
\end{itemize}

\end{solutionbox}
\begin{mnemonicbox}
``એપ્લિકેશન ટ્રાન્સપોર્ટ ઈન્ટરનેટ નેટવર્ક હંમેશા''

\end{mnemonicbox}
\begin{center}\rule{0.5\linewidth}{0.5pt}\end{center}

\subsection*{પ્રશ્ન 1(ક અથવા) [7
ગુણ]}\label{uxaaauxab0uxab6uxaa8-1uxa95-uxa85uxaa5uxab5-7-uxa97uxaa3}

\textbf{OSI મોડલ તેની દરેક લેયર અને દરેક લેયરની કામગીરી સાથે સંક્ષિપ્તમાં સમજાવો.}

\begin{solutionbox}

OSI (Open Systems Interconnection) મોડલમાં નેટવર્ક કમ્યુનિકેશન માટે 7 લેયર્સ છે:

\begin{center}
\textbf{Mermaid Diagram (Code)}
\begin{verbatim}
{Shaded}
{Highlighting}[]
graph LR
    A[Application Layer] {-{-}{} B[Presentation Layer]}
    B {-{-}{} C[Session Layer]}
    C {-{-}{} D[Transport Layer]}
    D {-{-}{} E[Network Layer]}
    E {-{-}{} F[Data Link Layer]}
    F {-{-}{} G[Physical Layer]}
{Highlighting}
{Shaded}
\end{verbatim}
\end{center}

\textbf{લેયર કાર્યો:}

{\def\LTcaptype{none} % do not increment counter
\begin{longtable}[]{@{}llll@{}}
\toprule\noalign{}
લેયર & નામ & કાર્ય & પ્રોટોકોલ્સ \\
\midrule\noalign{}
\endhead
\bottomrule\noalign{}
\endlastfoot
\textbf{7} & Application & વપરાશકર્તા ઈન્ટરફેસ & HTTP, FTP, SMTP \\
\textbf{6} & Presentation & ડેટા ફોર્મેટિંગ, એન્ક્રિપ્શન & SSL, JPEG, MPEG \\
\textbf{5} & Session & સેશન મેનેજમેન્ટ & NetBIOS, RPC \\
\textbf{4} & Transport & અંત-થી-અંત ડિલિવરી & TCP, UDP \\
\textbf{3} & Network & રાઉટિંગ & IP, ICMP \\
\textbf{2} & Data Link & ફ્રેમ ટ્રાન્સમિશન & Ethernet, PPP \\
\textbf{1} & Physical & બિટ ટ્રાન્સમિશન & કેબલ્સ, રેડિયો તરંગો \\
\end{longtable}
}

\textbf{મુખ્ય લક્ષણો:}

\begin{itemize}
\tightlist
\item
  \textbf{મોડ્યુલર ડિઝાઇન}: દરેક લેયરની વિશિષ્ટ જવાબદારીઓ છે
\item
  \textbf{પ્રોટોકોલ સ્વતંત્રતા}: લેયર્સ વિવિધ પ્રોટોકોલ્સ વાપરી શકે છે
\item
  \textbf{માનકીકરણ}: સાર્વત્રિક નેટવર્કિંગ સંદર્ભ મોડલ
\end{itemize}

\end{solutionbox}
\begin{mnemonicbox}
``બધા લોકો સેશન ટ્રાન્સપોર્ટ નેટવર્ક ડેટા પ્રોસેસિંગ કરે''

\end{mnemonicbox}
\begin{center}\rule{0.5\linewidth}{0.5pt}\end{center}

\subsection*{પ્રશ્ન 2(અ) [3
ગુણ]}\label{uxaaauxab0uxab6uxaa8-2uxa85-3-uxa97uxaa3}

\textbf{ARP અને RARP પ્રોટોકોલ્સ વચ્ચેનો તફાવત લખો.}

\begin{solutionbox}

ARP અને RARP વિપરીત કાર્યો સાથે એડ્રેસ રિઝોલ્યુશન પ્રોટોકોલ્સ છે:

{\def\LTcaptype{none} % do not increment counter
\begin{longtable}[]{@{}
  >{\raggedright\arraybackslash}p{(\linewidth - 4\tabcolsep) * \real{0.4211}}
  >{\raggedright\arraybackslash}p{(\linewidth - 4\tabcolsep) * \real{0.2632}}
  >{\raggedright\arraybackslash}p{(\linewidth - 4\tabcolsep) * \real{0.3158}}@{}}
\toprule\noalign{}
\begin{minipage}[b]{\linewidth}\raggedright
પાસું
\end{minipage} & \begin{minipage}[b]{\linewidth}\raggedright
ARP
\end{minipage} & \begin{minipage}[b]{\linewidth}\raggedright
RARP
\end{minipage} \\
\midrule\noalign{}
\endhead
\bottomrule\noalign{}
\endlastfoot
\textbf{પૂરું નામ} & Address Resolution Protocol & Reverse Address
Resolution Protocol \\
\textbf{હેતુ} & IP થી MAC એડ્રેસ મેપિંગ & MAC થી IP એડ્રેસ મેપિંગ \\
\textbf{દિશા} & લોજિકલ થી ફિઝિકલ & ફિઝિકલ થી લોજિકલ \\
\textbf{વપરાશ} & સામાન્ય નેટવર્ક કમ્યુનિકેશન & ડિસ્ક-રહિત વર્કસ્ટેશન્સ \\
\end{longtable}
}

\textbf{કામગીરીની પ્રક્રિયા:}

\begin{itemize}
\tightlist
\item
  \textbf{ARP}: ``મને IP એડ્રેસ ખબર છે, MAC એડ્રેસની જરૂર છે''
\item
  \textbf{RARP}: ``મને MAC એડ્રેસ ખબર છે, IP એડ્રેસની જરૂર છે''
\item
  \textbf{કેશ}: બંને કાર્યક્ષમતા માટે એડ્રેસ ટેબલ મેઇન્ટેઇન કરે છે
\end{itemize}

\end{solutionbox}
\begin{mnemonicbox}
``ARP પૂછે ફિઝિકલ, RARP રિક્વેસ્ટ કરે IP''

\end{mnemonicbox}
\begin{center}\rule{0.5\linewidth}{0.5pt}\end{center}

\subsection*{પ્રશ્ન 2(બ) [4
ગુણ]}\label{uxaaauxab0uxab6uxaa8-2uxaac-4-uxa97uxaa3}

\textbf{IMAP પ્રોટોકોલની કામગીરી સમજાવો.}

\begin{solutionbox}

IMAP (Internet Message Access Protocol) મલ્ટિપલ ડિવાઇસ એક્સેસ માટે સર્વર પર
ઈમેલ્સનું મેનેજમેન્ટ કરે છે.

\textbf{કામગીરીની પ્રક્રિયા:}

{\def\LTcaptype{none} % do not increment counter
\begin{longtable}[]{@{}lll@{}}
\toprule\noalign{}
પગલું & ક્રિયા & વર્ણન \\
\midrule\noalign{}
\endhead
\bottomrule\noalign{}
\endlastfoot
1 & કનેક્શન & ક્લાયન્ટ IMAP સર્વર સાથે જોડાય છે (પોર્ટ 143/993) \\
2 & ઓથેન્ટિકેશન & ક્રેડેન્શિયલ્સ સાથે લોગિન \\
3 & ફોલ્ડર એક્સેસ & સર્વર પર ઈમેલ ફોલ્ડર્સ બ્રાઉઝ કરો \\
4 & સિંક્રોનાઇઝેશન & બધા ડિવાઇસેસ પર બદલાવો સિંક થાય છે \\
\end{longtable}
}

\textbf{મુખ્ય લક્ષણો:}

\begin{itemize}
\tightlist
\item
  \textbf{સર્વર-આધારિત}: ઈમેલ્સ સર્વર પર રહે છે
\item
  \textbf{મલ્ટિ-ડિવાઇસ}: અનેક ડિવાઇસેસથી એક્સેસ
\item
  \textbf{સિંક્રોનાઇઝેશન}: બદલાવો બધે પ્રતિબિંબિત થાય છે
\item
  \textbf{સિલેક્ટિવ ડાઉનલોડ}: માત્ર જરૂરી ઈમેલ્સ ડાઉનલોડ કરો
\end{itemize}

\textbf{ફાયદાઓ:}

\begin{itemize}
\tightlist
\item
  \textbf{સ્ટોરેજ કાર્યક્ષમતા}: સર્વર સ્ટોરેજનું મેનેજમેન્ટ કરે છે
\item
  \textbf{એક્સેસિબિલિટી}: ગમે ત્યાંથી એક્સેસ કરો
\item
  \textbf{બેકઅપ}: સર્વર આપોઆપ બેકઅપ પ્રદાન કરે છે
\end{itemize}

\end{solutionbox}
\begin{mnemonicbox}
``IMAP ઈન્ટરનેટ મેસેજેસ હંમેશા હાજર''

\end{mnemonicbox}
\begin{center}\rule{0.5\linewidth}{0.5pt}\end{center}

\subsection*{પ્રશ્ન 2(ક) [7
ગુણ]}\label{uxaaauxab0uxab6uxaa8-2uxa95-7-uxa97uxaa3}

\textbf{Mobile computing નું Three-tier આર્કિટેક્ચર યોગ્ય ડાયગ્રામ સાથે સમજાવો.}

\begin{solutionbox}

Three-tier આર્કિટેક્ચર મોબાઇલ કમ્પ્યુટિંગને અલગ લેયર્સમાં વિભાજિત કરે છે:

\begin{center}
\textbf{Mermaid Diagram (Code)}
\begin{verbatim}
{Shaded}
{Highlighting}[]
graph LR
    A[Presentation Tier{br/{}મોબાઇલ ડિવાઇસેસ] {-}{-}{} B[Application Tier{}br/{}એપ્લિકેશન સર્વર]}
    B {-{-}{} C[Data Tier{}br/{}ડેટાબેસ સર્વર]}
    
    A1[સ્માર્ટફોન્સ{br/{}ટેબલેટ્સ{}br/{}લેપટોપ્સ] {-}{-}{} A}
    B1[બિઝનેસ લોજિક{br/{}પ્રોસેસિંગ{}br/{}API સેવાઓ] {-}{-}{} B}
    C1[ડેટાબેસ{br/{}ફાઇલ સિસ્ટમ્સ{}br/{}ડેટા સ્ટોરેજ] {-}{-}{} C}
{Highlighting}
{Shaded}
\end{verbatim}
\end{center}

\textbf{ટાયર વિગતો:}

{\def\LTcaptype{none} % do not increment counter
\begin{longtable}[]{@{}lll@{}}
\toprule\noalign{}
ટાયર & ઘટકો & જવાબદારીઓ \\
\midrule\noalign{}
\endhead
\bottomrule\noalign{}
\endlastfoot
\textbf{Presentation} & મોબાઇલ ડિવાઇસેસ, UI & વપરાશકર્તા ઈન્ટરફેસ અને
ઇન્ટરેક્શન \\
\textbf{Application} & એપ્લિકેશન સર્વર્સ, મિડલવેર & બિઝનેસ લોજિક અને પ્રોસેસિંગ \\
\textbf{Data} & ડેટાબેસેસ, સ્ટોરેજ & ડેટા મેનેજમેન્ટ અને સ્ટોરેજ \\
\end{longtable}
}

\textbf{આર્કિટેક્ચરના ફાયદાઓ:}

\begin{itemize}
\tightlist
\item
  \textbf{સ્કેલેબિલિટી}: દરેક ટાયર સ્વતંત્ર રીતે સ્કેલ કરી શકાય છે
\item
  \textbf{મેઇન્ટેનેબિલિટી}: સરળ અપડેટ્સ માટે અલગ કાયદાઓ
\item
  \textbf{સિક્યોરિટી}: ટાયર સેપરેશન દ્વારા ડેટા પ્રોટેક્શન
\item
  \textbf{પરફોર્મન્સ}: વિતરિત પ્રોસેસિંગ લોડ ઘટાડે છે
\end{itemize}

\textbf{કમ્યુનિકેશન ફ્લો:}

\begin{itemize}
\tightlist
\item
  \textbf{વપરાશકર્તા રિક્વેસ્ટ}: Presentation \rightarrow Application \rightarrow Data
\item
  \textbf{રેસ્પોન્સ}: Data \rightarrow Application \rightarrow Presentation
\item
  \textbf{પ્રોસેસિંગ}: એપ્લિકેશન ટાયર બિઝનેસ લોજિક હેન્ડલ કરે છે
\end{itemize}

\end{solutionbox}
\begin{mnemonicbox}
``પ્રેઝન્ટેશન એપ્લાય કરે ડેટા પ્રોસેસિંગ''

\end{mnemonicbox}
\begin{center}\rule{0.5\linewidth}{0.5pt}\end{center}

\subsection*{પ્રશ્ન 2(અ અથવા) [3
ગુણ]}\label{uxaaauxab0uxab6uxaa8-2uxa85-uxa85uxaa5uxab5-3-uxa97uxaa3}

\textbf{Stop-and-wait data link લેયર પ્રોટોકોલની મર્યાદાઓ સમજાવો.}

\begin{solutionbox}

Stop-and-wait પ્રોટોકોલમાં કેટલીક પરફોર્મન્સ મર્યાદાઓ છે:

\textbf{મુખ્ય મર્યાદાઓ:}

{\def\LTcaptype{none} % do not increment counter
\begin{longtable}[]{@{}
  >{\raggedright\arraybackslash}p{(\linewidth - 4\tabcolsep) * \real{0.3636}}
  >{\raggedright\arraybackslash}p{(\linewidth - 4\tabcolsep) * \real{0.3939}}
  >{\raggedright\arraybackslash}p{(\linewidth - 4\tabcolsep) * \real{0.2424}}@{}}
\toprule\noalign{}
\begin{minipage}[b]{\linewidth}\raggedright
મર્યાદા
\end{minipage} & \begin{minipage}[b]{\linewidth}\raggedright
વર્ણન
\end{minipage} & \begin{minipage}[b]{\linewidth}\raggedright
પ્રભાવ
\end{minipage} \\
\midrule\noalign{}
\endhead
\bottomrule\noalign{}
\endlastfoot
\textbf{નીચી કાર્યક્ષમતા} & આગલા ફ્રેમ પહેલાં ACK ની રાહ જુએ છે & ખરાબ બેન્ડવિડ્થ
ઉપયોગ \\
\textbf{વધુ વિલંબ} & દરેક ફ્રેમ માટે રાઉન્ડ-ટ્રિપ વિલંબ & ધીમું ડેટા ટ્રાન્સમિશન \\
\textbf{એરર સેન્સિટિવિટી} & એક જ એરર ટ્રાન્સમિશન અટકાવે છે & ઘટતી
વિશ્વસનીયતા \\
\end{longtable}
}

\textbf{પરફોર્મન્સ સમસ્યાઓ:}

\begin{itemize}
\tightlist
\item
  \textbf{બેન્ડવિડ્થ વેસ્ટ}: રાહ જોવાના સમય દરમિયાન લિંક નિષ્ક્રિય રહે છે
\item
  \textbf{ટાઇમઆઉટ પ્રોબ્લેમ્સ}: ખોવાયેલ ACK બિનજરૂરી પુન:ટ્રાન્સમિશન લાવે છે
\item
  \textbf{સિક્વેન્શિયલ પ્રોસેસિંગ}: એકસાથે મલ્ટિપલ ફ્રેમ્સ મોકલી શકાતા નથી
\end{itemize}

\end{solutionbox}
\begin{mnemonicbox}
``સ્ટોપ રાહ જુએ, બેન્ડવિડ્થ વેસ્ટ કરે''

\end{mnemonicbox}
\begin{center}\rule{0.5\linewidth}{0.5pt}\end{center}

\subsection*{પ્રશ્ન 2(બ અથવા) [4
ગુણ]}\label{uxaaauxab0uxab6uxaa8-2uxaac-uxa85uxaa5uxab5-4-uxa97uxaa3}

\textbf{જૂની IPV4 એડ્રેસિંગ સ્કીમ પર IPV6 ના ફાયદાઓ સમજાવો.}

\begin{solutionbox}

IPv6 એ IPv4 પર નોંધપાત્ર સુધારાઓ પ્રદાન કરે છે:

\textbf{મુખ્ય ફાયદાઓ:}

{\def\LTcaptype{none} % do not increment counter
\begin{longtable}[]{@{}lll@{}}
\toprule\noalign{}
લક્ષણ & IPv4 & IPv6 \\
\midrule\noalign{}
\endhead
\bottomrule\noalign{}
\endlastfoot
\textbf{એડ્રેસ સ્પેસ} & 32-bit (4.3 બિલિયન) & 128-bit (340 અનડેસિલિયન) \\
\textbf{હેડર} & વેરિયેબલ લેન્થ & ફિક્સ્ડ 40 બાઇટ્સ \\
\textbf{સિક્યોરિટી} & વૈકલ્પિક IPSec & બિલ્ટ-ઇન IPSec \\
\textbf{કોન્ફિગરેશન} & મેન્યુઅલ/DHCP & ઓટો-કોન્ફિગરેશન \\
\end{longtable}
}

\textbf{મુખ્ય ફાયદાઓ:}

\begin{itemize}
\tightlist
\item
  \textbf{અનલિમિટેડ એડ્રેસેસ}: એડ્રેસ એક્ઝોસ્ચન પ્રોબ્લેમ ઉકેલે છે
\item
  \textbf{બેહતર પરફોર્મન્સ}: સરળ હેડર પ્રોસેસિંગ
\item
  \textbf{એન્હાન્સ્ડ સિક્યોરિટી}: ફરજિયાત એન્ક્રિપ્શન સપોર્ટ
\item
  \textbf{મોબિલિટી સપોર્ટ}: બેહતર મોબાઇલ ડિવાઇસ કનેક્ટિવિટી
\end{itemize}

\textbf{વધારાની લક્ષણો:}

\begin{itemize}
\tightlist
\item
  \textbf{ક્વોલિટી ઓફ સર્વિસ}: બિલ્ટ-ઇન QoS સપોર્ટ
\item
  \textbf{મલ્ટિકાસ્ટ}: સુધારેલ મલ્ટિકાસ્ટ ક્ષમતાઓ
\item
  \textbf{નો ફ્રેગમેન્ટેશન}: રાઉટર્સ પેકેટ્સને ફ્રેગમેન્ટ કરતા નથી
\end{itemize}

\end{solutionbox}
\begin{mnemonicbox}
``IPv6 સુધારે પરફોર્મન્સ, સિક્યોરિટી, એડ્રેસેસ''

\end{mnemonicbox}
\begin{center}\rule{0.5\linewidth}{0.5pt}\end{center}

\subsection*{પ્રશ્ન 2(ક અથવા) [7
ગુણ]}\label{uxaaauxab0uxab6uxaa8-2uxa95-uxa85uxaa5uxab5-7-uxa97uxaa3}

\textbf{Mobile computing માં ઉપલબ્ધ નેટવર્કના નામ આપો. તેમાંથી કોઈપણ એકને
વિસ્તારથી સમજાવો.}

\begin{solutionbox}

\textbf{મોબાઇલ નેટવર્કના પ્રકારો:}

{\def\LTcaptype{none} % do not increment counter
\begin{longtable}[]{@{}llll@{}}
\toprule\noalign{}
જનરેશન & ટેક્નોલોજી & સ્પીડ & લક્ષણો \\
\midrule\noalign{}
\endhead
\bottomrule\noalign{}
\endlastfoot
\textbf{2G} & GSM, CDMA & 64 Kbps & વૉઇસ + SMS \\
\textbf{3G} & UMTS, CDMA2000 & 2 Mbps & ડેટા સેવાઓ \\
\textbf{4G} & LTE, WiMAX & 100 Mbps & હાઇ-સ્પીડ ઇન્ટરનેટ \\
\textbf{5G} & New Radio (NR) & 10 Gbps & અલ્ટ્રા-લો લેટન્સી \\
\end{longtable}
}

\textbf{વિગતવાર: 4G LTE નેટવર્ક}

\begin{center}
\textbf{Mermaid Diagram (Code)}
\begin{verbatim}
{Shaded}
{Highlighting}[]
graph LR
    A[મોબાઇલ ડિવાઇસ] {-{-}{} B[eNodeB{}br/{}બેસ સ્ટેશન]}
    B {-{-}{} C[Mobility Management Entity{}br/{}MME]}
    B {-{-}{} D[Serving Gateway{}br/{}S{-}GW]}
    D {-{-}{} E[Packet Data Network Gateway{}br/{}P{-}GW]}
    E {-{-}{} F[ઇન્ટરનેટ/એક્સટર્નલ નેટવર્ક્સ]}
    C {-{-}{} G[Home Subscriber Server{}br/{}HSS]}
{Highlighting}
{Shaded}
\end{verbatim}
\end{center}

\textbf{4G LTE લક્ષણો:}

\begin{itemize}
\tightlist
\item
  \textbf{હાઇ સ્પીડ}: 100 Mbps ડાઉનલોડ, 50 Mbps અપલોડ સુધી
\item
  \textbf{લો લેટન્સી}: રિયલ-ટાઇમ એપ્લિકેશન્સ માટે 10ms કરતાં ઓછું
\item
  \textbf{ઓલ-IP નેટવર્ક}: પેકેટ-સ્વિચ્ડ આર્કિટેક્ચર
\item
  \textbf{એડવાન્સ્ડ એન્ટેના}: બેહતર કવરેજ માટે MIMO ટેક્નોલોજી
\end{itemize}

\textbf{આર્કિટેક્ચર ઘટકો:}

\begin{itemize}
\tightlist
\item
  \textbf{eNodeB}: એડવાન્સ્ડ લક્ષણો સાથે એન્હાન્સ્ડ બેસ સ્ટેશન
\item
  \textbf{MME}: મોબિલિટી અને ઓથેન્ટિકેશન મેનેજ કરે છે
\item
  \textbf{ગેટવેઝ}: ડેટા રાઉટિંગ અને એક્સટર્નલ કનેક્ટિવિટી હેન્ડલ કરે છે
\end{itemize}

\textbf{એપ્લિકેશન્સ}: વિડિયો સ્ટ્રીમિંગ, ઓનલાઇન ગેમિંગ, IoT કનેક્ટિવિટી

\end{solutionbox}
\begin{mnemonicbox}
``4G LTE: લોંગ ટર્મ એવોલ્યુશન''

\end{mnemonicbox}
\begin{center}\rule{0.5\linewidth}{0.5pt}\end{center}

\subsection*{પ્રશ્ન 3(અ) [3
ગુણ]}\label{uxaaauxab0uxab6uxaa8-3uxa85-3-uxa97uxaa3}

\textbf{Routing ના પ્રકાર સમજાવો.}

\begin{solutionbox}

રાઉટિંગ નેટવર્ક્સ પર ડેટા પેકેટ્સ માટે પાથ નિર્ધારિત કરે છે:

\textbf{રાઉટિંગના પ્રકારો:}

{\def\LTcaptype{none} % do not increment counter
\begin{longtable}[]{@{}lll@{}}
\toprule\noalign{}
પ્રકાર & વર્ણન & ઉદાહરણ \\
\midrule\noalign{}
\endhead
\bottomrule\noalign{}
\endlastfoot
\textbf{Static} & મેન્યુઅલ રાઉટ કોન્ફિગરેશન & એડમિનિસ્ટ્રેટિવ સેટઅપ \\
\textbf{Dynamic} & ઓટોમેટિક રાઉટ ડિસ્કવરી & RIP, OSPF પ્રોટોકોલ્સ \\
\textbf{Default} & અજાણ્યા ડેસ્ટિનેશન્સ માટે ફોલબેક રાઉટ & ગેટવે ઓફ લાસ્ટ
રિસોર્ટ \\
\end{longtable}
}

\textbf{રાઉટિંગ કેટેગરીઝ:}

\begin{itemize}
\tightlist
\item
  \textbf{Distance Vector}: હોપ કાઉન્ટ વાપરે છે (RIP)
\item
  \textbf{Link State}: નેટવર્ક ટોપોલોજી વાપરે છે (OSPF)
\item
  \textbf{Hybrid}: બંને અભિગમો જોડે છે (EIGRP)
\end{itemize}

\textbf{સિલેક્શન ક્રાઇટેરિયા:}

\begin{itemize}
\tightlist
\item
  \textbf{Shortest path}: મિનિમમ હોપ્સ અથવા ડિસ્ટન્સ
\item
  \textbf{Load balancing}: ટ્રાફિક સમાન રીતે વિતરિત કરો
\item
  \textbf{Fault tolerance}: નિષ્ફળતાઓ માટે વૈકલ્પિક રાઉટ્સ
\end{itemize}

\end{solutionbox}
\begin{mnemonicbox}
``સ્ટેટિક ડાયનેમિક ડિફોલ્ટ રાઉટ્સ''

\end{mnemonicbox}
\begin{center}\rule{0.5\linewidth}{0.5pt}\end{center}

\subsection*{પ્રશ્ન 3(બ) [4
ગુણ]}\label{uxaaauxab0uxab6uxaa8-3uxaac-4-uxa97uxaa3}

\textbf{Subnetting અને supernetting શું છે?}

\begin{solutionbox}

સબનેટિંગ અને સુપરનેટિંગ IP એડ્રેસ એલોકેશનને કાર્યક્ષમ રીતે મેનેજ કરે છે:

\textbf{સરખામણી:}

{\def\LTcaptype{none} % do not increment counter
\begin{longtable}[]{@{}lll@{}}
\toprule\noalign{}
પાસું & સબનેટિંગ & સુપરનેટિંગ \\
\midrule\noalign{}
\endhead
\bottomrule\noalign{}
\endlastfoot
\textbf{હેતુ} & મોટા નેટવર્કને વિભાજિત કરો & નાના નેટવર્ક્સને જોડો \\
\textbf{દિશા} & ટોપ-ડાઉન અભિગમ & બોટમ-અપ અભિગમ \\
\textbf{માસ્ક} & લાંબો સબનેટ માસ્ક & ટૂંકો સબનેટ માસ્ક \\
\textbf{પરિણામ} & અનેક નાના સબનેટ્સ & એક જ મોટું નેટવર્ક \\
\end{longtable}
}

\textbf{સબનેટિંગ પ્રક્રિયા:}

\begin{itemize}
\tightlist
\item
  \textbf{બિટ્સ ઉધાર લેવા}: હોસ્ટ ભાગમાંથી બિટ્સ લો
\item
  \textbf{સબનેટ્સ બનાવો}: અનેક નેટવર્ક સેગમેન્ટ્સ
\item
  \textbf{બ્રોડકાસ્ટ ઘટાડો}: નાના બ્રોડકાસ્ટ ડોમેન્સ
\end{itemize}

\textbf{સુપરનેટિંગ પ્રક્રિયા:}

\begin{itemize}
\tightlist
\item
  \textbf{નેટવર્ક્સ જોડો}: આડીને આવેલા નેટવર્ક્સને મર્જ કરો
\item
  \textbf{રાઉટ એગ્રિગેશન}: સિંગલ રાઉટિંગ એન્ટ્રી
\item
  \textbf{રાઉટિંગ ટેબલ ઘટાડો}: ઓછી રાઉટિંગ એન્ટ્રીઝ
\end{itemize}

\textbf{ફાયદાઓ:}

\begin{itemize}
\tightlist
\item
  \textbf{સબનેટિંગ}: બેહતર નેટવર્ક મેનેજમેન્ટ, સિક્યોરિટી
\item
  \textbf{સુપરનેટિંગ}: સરળ રાઉટિંગ, ઘટેલ ઓવરહેડ
\end{itemize}

\end{solutionbox}
\begin{mnemonicbox}
``સબનેટિંગ સ્પ્લિટ્સ, સુપરનેટિંગ સમ્સ''

\end{mnemonicbox}
\begin{center}\rule{0.5\linewidth}{0.5pt}\end{center}

\subsection*{પ્રશ્ન 3(ક) [7
ગુણ]}\label{uxaaauxab0uxab6uxaa8-3uxa95-7-uxa97uxaa3}

\textbf{IPV6 એડ્રેસિંગ સમજાવો. IPV6 સ્થળાંતરની જરૂરિયાત કેમ છે?}

\begin{solutionbox}

IPv6 એડ્રેસિંગ IPv4 મર્યાદાઓ ઉકેલવા માટે 128-bit એડ્રેસેસ વાપરે છે:

\textbf{IPv6 એડ્રેસ સ્ટ્રક્ચર:}

\begin{verbatim}
+{-{-}{-}+{-}{-}{-}+{-}{-}{-}+{-}{-}{-}+{-}{-}{-}+{-}{-}{-}+{-}{-}{-}+{-}{-}{-}+{-}{-}{-}+{-}{-}{-}+{-}{-}{-}+{-}{-}{-}+{-}{-}{-}+{-}{-}{-}+{-}{-}{-}+{-}{-}{-}+}
| Global Routing Prefix |Subnet |      Interface Identifier     |
|      (48 bits)        |(16)   |         (64 bits)             |
+{-{-}{-}+{-}{-}{-}+{-}{-}{-}+{-}{-}{-}+{-}{-}{-}+{-}{-}{-}+{-}{-}{-}+{-}{-}{-}+{-}{-}{-}+{-}{-}{-}+{-}{-}{-}+{-}{-}{-}+{-}{-}{-}+{-}{-}{-}+{-}{-}{-}+{-}{-}{-}+}
\end{verbatim}

\textbf{એડ્રેસ ફોર્મેટ:}

{\def\LTcaptype{none} % do not increment counter
\begin{longtable}[]{@{}lll@{}}
\toprule\noalign{}
ઘટક & સાઇઝ & હેતુ \\
\midrule\noalign{}
\endhead
\bottomrule\noalign{}
\endlastfoot
\textbf{Global Prefix} & 48 bits & ISP એલોકેશન \\
\textbf{Subnet ID} & 16 bits & સંસ્થાના સબનેટ્સ \\
\textbf{Interface ID} & 64 bits & ડિવાઇસ આઇડેન્ટિફિકેશન \\
\end{longtable}
}

\textbf{એડ્રેસના પ્રકારો:}

\begin{itemize}
\tightlist
\item
  \textbf{Unicast}: એક-થી-એક કમ્યુનિકેશન
\item
  \textbf{Multicast}: એક-થી-અનેક કમ્યુનિકેશન
\item
  \textbf{Anycast}: એક-થી-નજીકના કમ્યુનિકેશન
\end{itemize}

\textbf{IPv6 સ્થળાંતરની જરૂરિયાત:}

\textbf{ગંભીર સમસ્યાઓ:}

{\def\LTcaptype{none} % do not increment counter
\begin{longtable}[]{@{}lll@{}}
\toprule\noalign{}
સમસ્યા & IPv4 & IPv6 સોલ્યુશન \\
\midrule\noalign{}
\endhead
\bottomrule\noalign{}
\endlastfoot
\textbf{એડ્રેસ એક્ઝોસ્ચન} & 4.3 બિલિયન એડ્રેસેસ & 340 અનડેસિલિયન એડ્રેસેસ \\
\textbf{NAT જટિલતા} & કનેક્ટિવિટી માટે જરૂરી & એન્ડ-ટુ-એન્ડ કનેક્ટિવિટી \\
\textbf{સિક્યોરિટી} & એડ-ઓન લક્ષણ & બિલ્ટ-ઇન IPSec \\
\textbf{મોબાઇલ સપોર્ટ} & મર્યાદિત & નેટિવ મોબિલિટી \\
\end{longtable}
}

\textbf{સ્થળાંતરના ફાયદાઓ:}

\begin{itemize}
\tightlist
\item
  \textbf{અનલિમિટેડ ગ્રોથ}: IoT વિસ્તરણને સપોર્ટ કરે છે
\item
  \textbf{સરળ કોન્ફિગરેશન}: ઓટો-કોન્ફિગરેશન લક્ષણો
\item
  \textbf{બેહતર પરફોર્મન્સ}: ઓપ્ટિમાઇઝ્ડ હેડર સ્ટ્રક્ચર
\item
  \textbf{એન્હાન્સ્ડ સિક્યોરિટી}: ફરજિયાત એન્ક્રિપ્શન
\end{itemize}

\textbf{સ્થળાંતરના પડકારો:}

\begin{itemize}
\tightlist
\item
  \textbf{ડ્યુઅલ-સ્ટેક}: IPv4 અને IPv6 બંને ચલાવવું
\item
  \textbf{ટ્રાન્સલેશન}: IPv4-IPv6 ઇન્ટરઓપેરેબિલિટી
\item
  \textbf{ટ્રેનિંગ}: સ્ટાફ શિક્ષણની જરૂરિયાતો
\end{itemize}

\end{solutionbox}
\begin{mnemonicbox}
``IPv6 અનંત શક્યતાઓ, એન્હાન્સ્ડ સિક્યોરિટી''

\end{mnemonicbox}
\begin{center}\rule{0.5\linewidth}{0.5pt}\end{center}

\subsection*{પ્રશ્ન 3(અ અથવા) [3
ગુણ]}\label{uxaaauxab0uxab6uxaa8-3uxa85-uxa85uxaa5uxab5-3-uxa97uxaa3}

\textbf{નીચેનામાંથી માન્ય IPv4 એડ્રેસ શોધો. જો તે માન્ય IPv4 એડ્રેસ હોય તો તેના
class, Network ID અને Host ID શોધો. જો તે માન્ય IPv4 એડ્રેસ ન હોય તો તેનું કારણ
આપો.}

\textbf{અ. 192.108.102.101} \textbf{બ. 80.54.256.14}

\begin{solutionbox}

\textbf{વિશ્લેષણ:}

{\def\LTcaptype{none} % do not increment counter
\begin{longtable}[]{@{}
  >{\raggedright\arraybackslash}p{(\linewidth - 10\tabcolsep) * \real{0.1607}}
  >{\raggedright\arraybackslash}p{(\linewidth - 10\tabcolsep) * \real{0.1786}}
  >{\raggedright\arraybackslash}p{(\linewidth - 10\tabcolsep) * \real{0.1250}}
  >{\raggedright\arraybackslash}p{(\linewidth - 10\tabcolsep) * \real{0.2143}}
  >{\raggedright\arraybackslash}p{(\linewidth - 10\tabcolsep) * \real{0.1607}}
  >{\raggedright\arraybackslash}p{(\linewidth - 10\tabcolsep) * \real{0.1607}}@{}}
\toprule\noalign{}
\begin{minipage}[b]{\linewidth}\raggedright
એડ્રેસ
\end{minipage} & \begin{minipage}[b]{\linewidth}\raggedright
માન્યતા
\end{minipage} & \begin{minipage}[b]{\linewidth}\raggedright
ક્લાસ
\end{minipage} & \begin{minipage}[b]{\linewidth}\raggedright
Network ID
\end{minipage} & \begin{minipage}[b]{\linewidth}\raggedright
Host ID
\end{minipage} & \begin{minipage}[b]{\linewidth}\raggedright
કારણ
\end{minipage} \\
\midrule\noalign{}
\endhead
\bottomrule\noalign{}
\endlastfoot
\textbf{192.108.102.101} & માન્ય & ક્લાસ C & 192.108.102.0 & 0.0.0.101 &
બધા ઓક્ટેટ્સ \leq 255 \\
\textbf{80.54.256.14} & અમાન્ય & - & - & - & ત્રીજો ઓક્ટેટ = 256
\textgreater{} 255 \\
\end{longtable}
}

\textbf{એડ્રેસ અ: 192.108.102.101}

\begin{itemize}
\tightlist
\item
  \textbf{માન્ય}: બધા ઓક્ટેટ્સ રેન્જમાં છે (0-255)
\item
  \textbf{ક્લાસ C}: પ્રથમ ઓક્ટેટ 192 (192-223 રેન્જ)
\item
  \textbf{ડિફોલ્ટ માસ્ક}: 255.255.255.0 (/24)
\end{itemize}

\textbf{એડ્રેસ બ: 80.54.256.14}

\begin{itemize}
\tightlist
\item
  \textbf{અમાન્ય}: ત્રીજો ઓક્ટેટ 256 છે
\item
  \textbf{નિયમ ઉલ્લંઘન}: દરેક ઓક્ટેટ 0-255 હોવો જોઈએ
\item
  \textbf{સુધારો}: 256 ને માન્ય વેલ્યુ (0-255) સાથે બદલો
\end{itemize}

\end{solutionbox}
\begin{mnemonicbox}
``દરેક ઓક્ટેટ મહત્તમ 255''

\end{mnemonicbox}
\begin{center}\rule{0.5\linewidth}{0.5pt}\end{center}

\subsection*{પ્રશ્ન 3(બ અથવા) [4
ગુણ]}\label{uxaaauxab0uxab6uxaa8-3uxaac-uxa85uxaa5uxab5-4-uxa97uxaa3}

\textbf{Network Address Translation પર ટૂંક નોંધ લખો.}

\begin{solutionbox}

NAT ઇન્ટરનેટ એક્સેસ માટે પ્રાઇવેટ IP એડ્રેસેસને પબ્લિક IP એડ્રેસેસમાં ટ્રાન્સલેટ કરે છે:

\textbf{NAT પ્રક્રિયા:}

{\def\LTcaptype{none} % do not increment counter
\begin{longtable}[]{@{}lll@{}}
\toprule\noalign{}
પગલું & દિશા & ટ્રાન્સલેશન \\
\midrule\noalign{}
\endhead
\bottomrule\noalign{}
\endlastfoot
\textbf{આઉટબાઉન્ડ} & પ્રાઇવેટ \rightarrow પબ્લિક & ઇન્ટર્નલ IP પબ્લિક IP સાથે મેપ થાય છે \\
\textbf{ઇનબાઉન્ડ} & પબ્લિક \rightarrow પ્રાઇવેટ & પબ્લિક IP ઇન્ટર્નલ IP સાથે મેપ થાય છે \\
\end{longtable}
}

\textbf{NAT પ્રકારો:}

\begin{verbatim}
NAT Types
├── Static NAT (1:1 mapping)
├── Dynamic NAT (Pool mapping)
└── PAT/NAPT (Port translation)
\end{verbatim}

\textbf{ફાયદાઓ:}

\begin{itemize}
\tightlist
\item
  \textbf{IP બચત}: અનેક ડિવાઇસેસ એક પબ્લિક IP શેર કરે છે
\item
  \textbf{સિક્યોરિટી}: ઇન્ટર્નલ નેટવર્ક સ્ટ્રક્ચર છુપાવે છે
\item
  \textbf{કોસ્ટ રિડક્શન}: ઓછા પબ્લિક IP એડ્રેસેસની જરૂર
\item
  \textbf{લવચીકતા}: સરળ ઇન્ટર્નલ નેટવર્ક બદલાવ
\end{itemize}

\textbf{મર્યાદાઓ:}

\begin{itemize}
\tightlist
\item
  \textbf{એન્ડ-ટુ-એન્ડ કનેક્ટિવિટી}: ડાયરેક્ટ કમ્યુનિકેશન તોડે છે
\item
  \textbf{પ્રોટોકોલ સમસ્યાઓ}: કેટલાક પ્રોટોકોલ્સ NAT મારફતે કામ કરતા નથી
\item
  \textbf{પરફોર્મન્સ}: અતિરિક્ત પ્રોસેસિંગ ઓવરહેડ
\end{itemize}

\end{solutionbox}
\begin{mnemonicbox}
``NAT નેટવર્ક્સ એડ્રેસ ટ્રાન્સલેશન''

\end{mnemonicbox}
\begin{center}\rule{0.5\linewidth}{0.5pt}\end{center}

\subsection*{પ્રશ્ન 3(ક અથવા) [7
ગુણ]}\label{uxaaauxab0uxab6uxaa8-3uxa95-uxa85uxaa5uxab5-7-uxa97uxaa3}

\textbf{IPV4 Datagram Header વિસ્તારથી સમજાવો.}

\begin{solutionbox}

IPv4 હેડરમાં પેકેટ રાઉટિંગ માટે જરૂરી માહિતી છે:

\begin{verbatim}
 0                   1                   2                   3
 0 1 2 3 4 5 6 7 8 9 0 1 2 3 4 5 6 7 8 9 0 1 2 3 4 5 6 7 8 9 0 1
+{-+{-}+{-}+{-}+{-}+{-}+{-}+{-}+{-}+{-}+{-}+{-}+{-}+{-}+{-}+{-}+{-}+{-}+{-}+{-}+{-}+{-}+{-}+{-}+{-}+{-}+{-}+{-}+{-}+{-}+{-}+{-}+}
|Version|  IHL  |Type of Service|          Total Length         |
+{-+{-}+{-}+{-}+{-}+{-}+{-}+{-}+{-}+{-}+{-}+{-}+{-}+{-}+{-}+{-}+{-}+{-}+{-}+{-}+{-}+{-}+{-}+{-}+{-}+{-}+{-}+{-}+{-}+{-}+{-}+{-}+}
|         Identification        |Flags|      Fragment Offset    |
+{-+{-}+{-}+{-}+{-}+{-}+{-}+{-}+{-}+{-}+{-}+{-}+{-}+{-}+{-}+{-}+{-}+{-}+{-}+{-}+{-}+{-}+{-}+{-}+{-}+{-}+{-}+{-}+{-}+{-}+{-}+{-}+}
|  Time to Live |    Protocol   |         Header Checksum       |
+{-+{-}+{-}+{-}+{-}+{-}+{-}+{-}+{-}+{-}+{-}+{-}+{-}+{-}+{-}+{-}+{-}+{-}+{-}+{-}+{-}+{-}+{-}+{-}+{-}+{-}+{-}+{-}+{-}+{-}+{-}+{-}+}
|                       Source Address                          |
+{-+{-}+{-}+{-}+{-}+{-}+{-}+{-}+{-}+{-}+{-}+{-}+{-}+{-}+{-}+{-}+{-}+{-}+{-}+{-}+{-}+{-}+{-}+{-}+{-}+{-}+{-}+{-}+{-}+{-}+{-}+{-}+}
|                    Destination Address                        |
+{-+{-}+{-}+{-}+{-}+{-}+{-}+{-}+{-}+{-}+{-}+{-}+{-}+{-}+{-}+{-}+{-}+{-}+{-}+{-}+{-}+{-}+{-}+{-}+{-}+{-}+{-}+{-}+{-}+{-}+{-}+{-}+}
|                    Options                    |    Padding    |
+{-+{-}+{-}+{-}+{-}+{-}+{-}+{-}+{-}+{-}+{-}+{-}+{-}+{-}+{-}+{-}+{-}+{-}+{-}+{-}+{-}+{-}+{-}+{-}+{-}+{-}+{-}+{-}+{-}+{-}+{-}+{-}+}
\end{verbatim}

\textbf{હેડર ફીલ્ડ્સ:}

{\def\LTcaptype{none} % do not increment counter
\begin{longtable}[]{@{}lll@{}}
\toprule\noalign{}
ફીલ્ડ & સાઇઝ & હેતુ \\
\midrule\noalign{}
\endhead
\bottomrule\noalign{}
\endlastfoot
\textbf{Version} & 4 bits & IP વર્ઝન (IPv4 માટે 4) \\
\textbf{IHL} & 4 bits & 32-bit શબ્દોમાં હેડર લેન્થ \\
\textbf{Type of Service} & 8 bits & સેવાની ગુણવત્તા \\
\textbf{Total Length} & 16 bits & કુલ પેકેટ સાઇઝ \\
\textbf{Identification} & 16 bits & ફ્રેગમેન્ટ આઇડેન્ટિફિકેશન \\
\textbf{Flags} & 3 bits & ફ્રેગમેન્ટેશન કંટ્રોલ \\
\textbf{Fragment Offset} & 13 bits & ફ્રેગમેન્ટ પોઝિશન \\
\textbf{TTL} & 8 bits & ડિસ્કાર્ડ પહેલાં મહત્તમ હોપ્સ \\
\textbf{Protocol} & 8 bits & આગલી લેયર પ્રોટોકોલ \\
\textbf{Checksum} & 16 bits & હેડર એરર ડિટેક્શન \\
\textbf{Source Address} & 32 bits & મોકલનારનું IP એડ્રેસ \\
\textbf{Destination} & 32 bits & મેળવનારનું IP એડ્રેસ \\
\end{longtable}
}

\textbf{મુખ્ય કાર્યો:}

\begin{itemize}
\tightlist
\item
  \textbf{રાઉટિંગ}: સોર્સ અને ડેસ્ટિનેશન એડ્રેસેસ
\item
  \textbf{ફ્રેગમેન્ટેશન}: મોટા પેકેટ્સ હેન્ડલ કરવા
\item
  \textbf{એરર ડિટેક્શન}: હેડર ચેકસમ
\item
  \textbf{ક્વોલિટી કંટ્રોલ}: ટાઇપ ઓફ સર્વિસ ફીલ્ડ
\end{itemize}

\textbf{મહત્વપૂર્ણ વેલ્યુઝ:}

\begin{itemize}
\tightlist
\item
  \textbf{Protocol}: TCP=6, UDP=17, ICMP=1
\item
  \textbf{Flags}: Don't Fragment, More Fragments
\item
  \textbf{TTL}: અનંત લૂપ્સ અટકાવે છે
\end{itemize}

\end{solutionbox}
\begin{mnemonicbox}
``વર્ઝન IHL સર્વિસ લેન્થ આઇડેન્ટિફાઇ ફ્રેગમેન્ટ TTL પ્રોટોકોલ
ચેક સોર્સ ડેસ્ટિનેશન''

\end{mnemonicbox}
\begin{center}\rule{0.5\linewidth}{0.5pt}\end{center}

\subsection*{પ્રશ્ન 4(અ) [3
ગુણ]}\label{uxaaauxab0uxab6uxaa8-4uxa85-3-uxa97uxaa3}

\textbf{Indirect TCP ની કામગીરી સમજાવો.}

\begin{solutionbox}

Indirect TCP મોબાઇલ નેટવર્ક પડકારોને હેન્ડલ કરવા માટે TCP કનેક્શનને વિભાજિત કરે છે:

\textbf{આર્કિટેક્ચર:}

{\def\LTcaptype{none} % do not increment counter
\begin{longtable}[]{@{}lll@{}}
\toprule\noalign{}
ઘટક & ભૂમિકા & સ્થાન \\
\midrule\noalign{}
\endhead
\bottomrule\noalign{}
\endlastfoot
\textbf{Mobile Host} & TCP ક્લાયન્ટ & મોબાઇલ નેટવર્ક \\
\textbf{Base Station} & TCP પ્રોક્સી & ફિક્સ્ડ નેટવર્ક \\
\textbf{Fixed Host} & TCP સર્વર & વાયર્ડ નેટવર્ક \\
\end{longtable}
}

\textbf{કનેક્શન સ્પ્લિટ:}

\begin{itemize}
\tightlist
\item
  \textbf{કનેક્શન 1}: Mobile Host \leftrightarrow Base Station
\item
  \textbf{કનેક્શન 2}: Base Station \leftrightarrow Fixed Host
\item
  \textbf{પ્રોક્સી ફંક્શન}: બેસ સ્ટેશન TCP પ્રોક્સી તરીકે કામ કરે છે
\end{itemize}

\textbf{કામગીરીની પ્રક્રિયા:}

\begin{itemize}
\tightlist
\item
  \textbf{ડેટા ફ્લો}: Mobile \rightarrow Base Station \rightarrow Fixed Host
\item
  \textbf{ACK હેન્ડલિંગ}: બેસ સ્ટેશન એકનોલેજમેન્ટ્સ મેનેજ કરે છે
\item
  \textbf{હેન્ડઓવર}: હલનચલન દરમિયાન કનેક્શન જાળવાય છે
\end{itemize}

\textbf{ફાયદાઓ:}

\begin{itemize}
\tightlist
\item
  \textbf{વાયરલેસ ઓપ્ટિમાઇઝેશન}: વાયરલેસ લિંક સમસ્યાઓ હેન્ડલ કરે છે
\item
  \textbf{મોબિલિટી સપોર્ટ}: સીમલેસ હેન્ડઓવર ક્ષમતા
\item
  \textbf{એરર રિકવરી}: વાયરલેસ એરર્સનું બેહતર હેન્ડલિંગ
\end{itemize}

\end{solutionbox}
\begin{mnemonicbox}
``Indirect TCP પ્રોક્સી મારફતે''

\end{mnemonicbox}
\begin{center}\rule{0.5\linewidth}{0.5pt}\end{center}

\subsection*{પ્રશ્ન 4(બ) [4
ગુણ]}\label{uxaaauxab0uxab6uxaa8-4uxaac-4-uxa97uxaa3}

\textbf{Stop and Wait ARQ પ્રોટોકોલ પર ટૂંક નોંધ લખો.}

\begin{solutionbox}

Stop and Wait ARQ એરર ડિટેક્શન અને કરેક્શન સાથે વિશ્વસનીય ડેટા ટ્રાન્સમિશન સુનિશ્ચિત
કરે છે:

\textbf{પ્રોટોકોલ ઓપરેશન:}

{\def\LTcaptype{none} % do not increment counter
\begin{longtable}[]{@{}lll@{}}
\toprule\noalign{}
પગલું & ક્રિયા & હેતુ \\
\midrule\noalign{}
\endhead
\bottomrule\noalign{}
\endlastfoot
\textbf{Send} & સિક્વન્સ નંબર સાથે ફ્રેમ ટ્રાન્સમિટ કરો & ડેટા ડિલિવરી \\
\textbf{Wait} & એકનોલેજમેન્ટની રાહ જુઓ & રસીદની પુષ્ટિ \\
\textbf{Timeout} & કોઈ ACK ન મળે તો પુન:ટ્રાન્સમિટ & ખોવાયેલા ફ્રેમ્સ હેન્ડલ
કરો \\
\textbf{ACK} & પ્રાપ્ત ફ્રેમ માટે એકનોલેજમેન્ટ મોકલો & ડિલિવરીની પુષ્ટિ \\
\end{longtable}
}

\textbf{એરર હેન્ડલિંગ:}

\begin{verbatim}
Sender                    Receiver
  |                         |
  |{-{-}{-}{-} Frame 0 {-}{-}{-}{-}{-}{-}{-}{-}{-}{-}|}
  |                         |{-{-}{-}{-} ACK 0}
  |{{-}{-}{-}{-} ACK 0 {-}{-}{-}{-}{-}{-}{-}{-}{-}{-}{-}{-}|}
  |                         |
  |{-{-}{-}{-} Frame 1 {-}{-}{-}{-}{-}{-}{-}{-}{-}{-}| (Lost)}
  |                         |
  |{-{-} Timeout, Retransmit {-}{-}|}
  |{-{-}{-}{-} Frame 1 {-}{-}{-}{-}{-}{-}{-}{-}{-}{-}|}
  |                         |{-{-}{-}{-} ACK 1}
  |{{-}{-}{-}{-} ACK 1 {-}{-}{-}{-}{-}{-}{-}{-}{-}{-}{-}{-}|}
\end{verbatim}

\textbf{લક્ષણો:}

\begin{itemize}
\tightlist
\item
  \textbf{સિક્વન્સ નંબર્સ}: 0 અને 1 નું અલ્ટરનેશન
\item
  \textbf{ટાઇમઆઉટ મેકેનિઝમ}: ખોવાયેલા ફ્રેમ્સ/ACKs હેન્ડલ કરે છે
\item
  \textbf{ડુપ્લિકેટ ડિટેક્શન}: ડુપ્લિકેટ સ્વીકારવું અટકાવે છે
\item
  \textbf{ફ્લો કંટ્રોલ}: રિસીવર ટ્રાન્સમિશન રેટ કંટ્રોલ કરે છે
\end{itemize}

\textbf{મર્યાદાઓ:}

\begin{itemize}
\tightlist
\item
  \textbf{નીચી કાર્યક્ષમતા}: ટ્રાન્ઝિટમાં માત્ર એક જ ફ્રેમ
\item
  \textbf{બેન્ડવિડ્થ વેસ્ટ}: રાહ જોવા દરમિયાન નિષ્ક્રિય સમય
\end{itemize}

\end{solutionbox}
\begin{mnemonicbox}
``સ્ટોપ સેન્ડ, વેઇટ ACK, રિપીટ''

\end{mnemonicbox}
\begin{center}\rule{0.5\linewidth}{0.5pt}\end{center}

\subsection*{પ્રશ્ન 4(ક) [7
ગુણ]}\label{uxaaauxab0uxab6uxaa8-4uxa95-7-uxa97uxaa3}

\textbf{Communication Middleware વિસ્તારથી સમજાવો.}

\begin{solutionbox}

Communication middleware એપ્લિકેશન્સ અને નેટવર્ક સેવાઓ વચ્ચે એબ્સ્ટ્રેક્શન લેયર પ્રદાન
કરે છે:

\begin{center}
\textbf{Mermaid Diagram (Code)}
\begin{verbatim}
{Shaded}
{Highlighting}[]
graph LR
    A[મોબાઇલ એપ્લિકેશન્સ] {-{-}{} B[Communication Middleware]}
    B {-{-}{} C[નેટવર્ક સેવાઓ]}
    
    B1[Message Passing{br/{}RPC{}br/{}Event Handling] {-}{-}{} B}
    C1[TCP/IP{br/{}Wireless Protocols{}br/{}Network APIs] {-}{-}{} C}
{Highlighting}
{Shaded}
\end{verbatim}
\end{center}

\textbf{મિડલવેર પ્રકારો:}

{\def\LTcaptype{none} % do not increment counter
\begin{longtable}[]{@{}lll@{}}
\toprule\noalign{}
પ્રકાર & કાર્ય & ઉદાહરણ \\
\midrule\noalign{}
\endhead
\bottomrule\noalign{}
\endlastfoot
\textbf{Message-Oriented} & એસિંક્રોનસ મેસેજિંગ & Message queues \\
\textbf{RPC-based} & રિમોટ પ્રોસીજર કોલ્સ & CORBA, RMI \\
\textbf{Event-driven} & ઇવેન્ટ નોટિફિકેશન્સ & Publish-subscribe \\
\textbf{Stream-oriented} & સતત ડેટા ફ્લો & મલ્ટીમીડિયા સ્ટ્રીમ્સ \\
\end{longtable}
}

\textbf{કોર સેવાઓ:}

\textbf{કમ્યુનિકેશન સેવાઓ:}

\begin{itemize}
\tightlist
\item
  \textbf{Message routing}: કાર્યક્ષમ મેસેજ ડિલિવરી
\item
  \textbf{Protocol conversion}: વિવિધ પ્રોટોકોલ હેન્ડલિંગ
\item
  \textbf{Buffering}: અસ્થાયી મેસેજ સ્ટોરેજ
\item
  \textbf{Synchronization}: સંકલિત સંદેશાવ્યવહાર
\end{itemize}

\textbf{વિશ્વસનીયતા સેવાઓ:}

\begin{itemize}
\tightlist
\item
  \textbf{Error detection}: મેસેજ અખંડતા ચકાસણી
\item
  \textbf{Retransmission}: નિષ્ફળ મેસેજ રિકવરી
\item
  \textbf{Duplicate elimination}: મેસેજ ડુપ્લિકેશન અટકાવો
\item
  \textbf{Ordering}: મેસેજ સિક્વન્સ જાળવો
\end{itemize}

\textbf{મોબાઇલ-સ્પેસિફિક લક્ષણો:}

\begin{itemize}
\tightlist
\item
  \textbf{Location transparency}: એપ્લિકેશન્સથી મોબિલિટી છુપાવો
\item
  \textbf{Disconnection handling}: નેટવર્ક વિક્ષેપો મેનેજ કરો
\item
  \textbf{Bandwidth adaptation}: નેટવર્ક પરિસ્થિતિઓ પ્રમાણે એડજસ્ટ કરો
\item
  \textbf{Power management}: બેટરી વપરાશ ઓપ્ટિમાઇઝ કરો
\end{itemize}

\textbf{આર્કિટેક્ચરના ફાયદાઓ:}

\begin{itemize}
\tightlist
\item
  \textbf{Abstraction}: નેટવર્ક જટિલતા છુપાવો
\item
  \textbf{Portability}: નેટવર્કથી એપ્લિકેશન સ્વતંત્રતા
\item
  \textbf{Scalability}: વધતા ડિવાઇસેસને સપોર્ટ કરો
\item
  \textbf{Interoperability}: વિવિધ સિસ્ટમ કમ્યુનિકેશન
\end{itemize}

\textbf{ઉદાહરણો:}

\begin{itemize}
\tightlist
\item
  \textbf{CORBA}: વિતરિત ઓબ્જેક્ટ કમ્યુનિકેશન
\item
  \textbf{Message Queues}: એસિંક્રોનસ મેસેજિંગ
\item
  \textbf{Web Services}: HTTP-આધારિત કમ્યુનિકેશન
\end{itemize}

\end{solutionbox}
\begin{mnemonicbox}
``મિડલવેર મેનેજે મોબાઇલ કમ્યુનિકેશન''

\end{mnemonicbox}
\begin{center}\rule{0.5\linewidth}{0.5pt}\end{center}

\subsection*{પ્રશ્ન 4(અ અથવા) [3
ગુણ]}\label{uxaaauxab0uxab6uxaa8-4uxa85-uxa85uxaa5uxab5-3-uxa97uxaa3}

\textbf{Mobile IP માં Handover management સમજાવો.}

\begin{solutionbox}

Handover management મોબાઇલ ડિવાઇસ નેટવર્ક્સ વચ્ચે ફરે ત્યારે કનેક્ટિવિટી જાળવે છે:

\textbf{હેન્ડઓવર પ્રક્રિયા:}

{\def\LTcaptype{none} % do not increment counter
\begin{longtable}[]{@{}lll@{}}
\toprule\noalign{}
તબક્કો & ક્રિયા & હેતુ \\
\midrule\noalign{}
\endhead
\bottomrule\noalign{}
\endlastfoot
\textbf{Detection} & સિગ્નલ સ્ટ્રેન્થ મોનિટર કરો & હેન્ડઓવરની જરૂરિયાત ઓળખો \\
\textbf{Decision} & ટાર્ગેટ નેટવર્ક પસંદ કરો & શ્રેષ્ઠ નેટવર્ક પસંદ કરો \\
\textbf{Execution} & નવા નેટવર્ક પર સ્વિચ કરો & હેન્ડઓવર પૂર્ણ કરો \\
\end{longtable}
}

\textbf{હેન્ડઓવરના પ્રકારો:}

\begin{itemize}
\tightlist
\item
  \textbf{Horizontal}: સમાન ટેક્નોલોજી નેટવર્ક્સ
\item
  \textbf{Vertical}: વિવિધ ટેક્નોલોજી નેટવર્ક્સ
\item
  \textbf{Hard}: Break-before-make
\item
  \textbf{Soft}: Make-before-break
\end{itemize}

\textbf{મેનેજમેન્ટ ઘટકો:}

\begin{itemize}
\tightlist
\item
  \textbf{Signal monitoring}: સતત સિગ્નલ મૂલ્યાંકન
\item
  \textbf{Network discovery}: ઉપલબ્ધ નેટવર્ક ઓળખ
\item
  \textbf{Decision algorithm}: શ્રેષ્ઠ નેટવર્ક પસંદગી
\end{itemize}

\textbf{પરફોર્મન્સ મેટ્રિક્સ:}

\begin{itemize}
\tightlist
\item
  \textbf{Handover delay}: સ્વિચ પૂર્ણ કરવાનો સમય
\item
  \textbf{Packet loss}: હેન્ડઓવર દરમિયાન ખોવાયેલ ડેટા
\item
  \textbf{Signaling overhead}: કંટ્રોલ મેસેજ કોસ્ટ
\end{itemize}

\end{solutionbox}
\begin{mnemonicbox}
``હેન્ડઓવર હેલ્પ મેઇન્ટેઇન મોબિલિટી''

\end{mnemonicbox}
\begin{center}\rule{0.5\linewidth}{0.5pt}\end{center}

\subsection*{પ્રશ્ન 4(બ અથવા) [4
ગુણ]}\label{uxaaauxab0uxab6uxaa8-4uxaac-uxa85uxaa5uxab5-4-uxa97uxaa3}

\textbf{Communication Gateways ના મુખ્ય કાર્યો સમજાવો.}

\begin{solutionbox}

Communication gateways વિવિધ નેટવર્ક સિસ્ટમ્સ વચ્ચે ઇન્ટરઓપેરેબિલિટી સક્ષમ કરે છે:

\textbf{મુખ્ય કાર્યો:}

{\def\LTcaptype{none} % do not increment counter
\begin{longtable}[]{@{}lll@{}}
\toprule\noalign{}
કાર્ય & વર્ણન & ફાયદો \\
\midrule\noalign{}
\endhead
\bottomrule\noalign{}
\endlastfoot
\textbf{Protocol Translation} & પ્રોટોકોલ્સ વચ્ચે રૂપાંતર & ઇન્ટરઓપેરેબિલિટી \\
\textbf{Data Format Conversion} & ડેટા ફોર્મેટ્સ રૂપાંતરિત કરો & સુસંગતતા \\
\textbf{Security Enforcement} & સિક્યોરિટી પોલિસીઓ લાગુ કરો & સુરક્ષા \\
\textbf{Load Balancing} & ટ્રાફિક વિતરિત કરો & પરફોર્મન્સ \\
\end{longtable}
}

\textbf{ગેટવે સેવાઓ:}

\textbf{પ્રોટોકોલ સેવાઓ:}

\begin{itemize}
\tightlist
\item
  \textbf{Multi-protocol support}: વિવિધ પ્રોટોકોલ્સ હેન્ડલ કરે છે
\item
  \textbf{Translation efficiency}: ઝડપી પ્રોટોકોલ રૂપાંતર
\item
  \textbf{Standards compliance}: પ્રોટોકોલ સ્પેસિફિકેશન્સ અનુસરે છે
\end{itemize}

\textbf{સિક્યોરિટી સેવાઓ:}

\begin{itemize}
\tightlist
\item
  \textbf{Authentication}: વપરાશકર્તા ઓળખ ચકાસો
\item
  \textbf{Authorization}: એક્સેસ પરમિશન્સ નિયંત્રિત કરો
\item
  \textbf{Encryption}: ડેટા ટ્રાન્સમિશન સુરક્ષિત કરો
\item
  \textbf{Firewall}: દુષ્ટ ટ્રાફિક ફિલ્ટર કરો
\end{itemize}

\textbf{પરફોર્મન્સ સેવાઓ:}

\begin{itemize}
\tightlist
\item
  \textbf{Caching}: વારંવાર એક્સેસ થતા ડેટાને સ્ટોર કરો
\item
  \textbf{Compression}: ડેટા સાઇઝ ઘટાડો
\item
  \textbf{Traffic shaping}: બેન્ડવિડ્થ વપરાશ મેનેજ કરો
\item
  \textbf{Quality of Service}: જટિલ ટ્રાફિકને પ્રાથમિકતા આપો
\end{itemize}

\textbf{મેનેજમેન્ટ લક્ષણો:}

\begin{itemize}
\tightlist
\item
  \textbf{Monitoring}: ગેટવે પરફોર્મન્સ ટ્રેક કરો
\item
  \textbf{Configuration}: લવચીક સેટઅપ વિકલ્પો
\item
  \textbf{Logging}: પ્રવૃત્તિ અને એરર્સ રેકોર્ડ કરો
\end{itemize}

\end{solutionbox}
\begin{mnemonicbox}
``ગેટવેઝ ગ્રાન્ટ પ્રોટોકોલ ઇન્ટરઓપેરેબિલિટી''

\end{mnemonicbox}
\begin{center}\rule{0.5\linewidth}{0.5pt}\end{center}

\subsection*{પ્રશ્ન 4(ક અથવા) [7
ગુણ]}\label{uxaaauxab0uxab6uxaa8-4uxa95-uxa85uxaa5uxab5-7-uxa97uxaa3}

\textbf{Mobile IP ની સમગ્ર પ્રક્રિયા સમજાવો.}

\begin{solutionbox}

Mobile IP IP કનેક્ટિવિટી જાળવી રાખીને ડિવાઇસ મોબિલિટી સક્ષમ કરે છે:

\begin{verbatim}
sequenceDiagram
    participant MN as Mobile Node
    participant HA as Home Agent
    participant FA as Foreign Agent
    participant CN as Correspondent Node
    
    MN{-FA: Agent Solicitation}
    FA{-MN: Agent Advertisement}
    MN{-HA: Registration Request}
    HA{-MN: Registration Reply}
    CN{-HA: Data Packet (Home Address)}
    HA{-FA: Tunneled Packet}
    FA{-MN: Data Packet}
\end{verbatim}

\textbf{Mobile IP ઘટકો:}

{\def\LTcaptype{none} % do not increment counter
\begin{longtable}[]{@{}lll@{}}
\toprule\noalign{}
ઘટક & ભૂમિકા & કાર્ય \\
\midrule\noalign{}
\endhead
\bottomrule\noalign{}
\endlastfoot
\textbf{Mobile Node} & ચલિત ડિવાઇસ & કનેક્ટિવિટી જાળવે છે \\
\textbf{Home Agent} & હોમ નેટવર્ક રાઉટર & પેકેટ્સ ફોરવર્ડ કરે છે \\
\textbf{Foreign Agent} & મુલાકાતી નેટવર્ક રાઉટર & સ્થાનિક ડિલિવરી \\
\textbf{Care-of Address} & અસ્થાયી એડ્રેસ & વર્તમાન સ્થાન \\
\end{longtable}
}

\textbf{રેજિસ્ટ્રેશન પ્રક્રિયા:}

\textbf{તબક્કો 1: એજન્ટ ડિસ્કવરી}

\begin{itemize}
\tightlist
\item
  \textbf{Advertisement}: એજન્ટ્સ ઉપલબ્ધતા બ્રોડકાસ્ટ કરે છે
\item
  \textbf{Solicitation}: મોબાઇલ નોડ એજન્ટ માહિતી માંગે છે
\item
  \textbf{Selection}: યોગ્ય ફોરેન એજન્ટ પસંદ કરો
\end{itemize}

\textbf{તબક્કો 2: રેજિસ્ટ્રેશન}

\begin{itemize}
\tightlist
\item
  \textbf{Request}: મોબાઇલ નોડ હોમ એજન્ટ સાથે રેજિસ્ટર થાય છે
\item
  \textbf{Authentication}: મોબાઇલ નોડ ઓળખ ચકાસો
\item
  \textbf{Binding}: care-of address બાઇન્ડિંગ બનાવો
\item
  \textbf{Confirmation}: રેજિસ્ટ્રેશન પુષ્ટિ
\end{itemize}

\textbf{તબક્કો 3: પેકેટ ડિલિવરી}

\begin{itemize}
\tightlist
\item
  \textbf{Interception}: હોમ એજન્ટ પેકેટ્સ ઇન્ટરસેપ્ટ કરે છે
\item
  \textbf{Tunneling}: પેકેટ્સ એન્કેપ્સુલેટ અને ફોરવર્ડ કરે છે
\item
  \textbf{Decapsulation}: ફોરેન એજન્ટ પેકેટ્સ એક્સટ્રેક્ટ કરે છે
\item
  \textbf{Local delivery}: મોબાઇલ નોડને ફોરવર્ડ કરો
\end{itemize}

\textbf{ટનલિંગ મેકેનિઝમ:}

\begin{verbatim}
Original Packet: [IP Header|Data]
                 Dest: Home Address

Tunneled Packet: [New IP Header|Original Packet]
                 Dest: Care{-of Address}
\end{verbatim}

\textbf{મુખ્ય લક્ષણો:}

\begin{itemize}
\tightlist
\item
  \textbf{Transparency}: એપ્લિકેશન્સ મોબિલિટીથી અજાણ
\item
  \textbf{Triangle routing}: પરોક્ષ પેકેટ ડિલિવરી
\item
  \textbf{Location privacy}: વાસ્તવિક સ્થાન છુપાવો
\item
  \textbf{Seamless handover}: કનેક્શન્સ જાળવો
\end{itemize}

\textbf{પડકારો:}

\begin{itemize}
\tightlist
\item
  \textbf{Triangle routing}: બિનકાર્યક્ષમ પેકેટ પાથ
\item
  \textbf{Ingress filtering}: ફાયરવોલ સુસંગતતા
\item
  \textbf{Security}: ઓથેન્ટિકેશન અને એન્ક્રિપ્શન
\end{itemize}

\end{solutionbox}
\begin{mnemonicbox}
``Mobile IP: ડિસ્કવર રેજિસ્ટર ટનલ ડિલિવર''

\end{mnemonicbox}
\begin{center}\rule{0.5\linewidth}{0.5pt}\end{center}

\subsection*{પ્રશ્ન 5(અ) [3
ગુણ]}\label{uxaaauxab0uxab6uxaa8-5uxa85-3-uxa97uxaa3}

\textbf{WPANs ના ફાયદાઓની યાદી બનાવો.}

\begin{solutionbox}

WPAN (Wireless Personal Area Network) ટૂંકા-અંતરની કનેક્ટિવિટીના ફાયદાઓ પ્રદાન
કરે છે:

\textbf{મુખ્ય ફાયદાઓ:}

{\def\LTcaptype{none} % do not increment counter
\begin{longtable}[]{@{}lll@{}}
\toprule\noalign{}
ફાયદો & વર્ણન & ફાયદો \\
\midrule\noalign{}
\endhead
\bottomrule\noalign{}
\endlastfoot
\textbf{લો પાવર} & ન્યૂનતમ બેટરી વપરાશ & લંબાવેલ ડિવાઇસ જીવન \\
\textbf{લો કોસ્ટ} & સસ્તું અમલીકરણ & કિફાયતી ડિપ્લોયમેન્ટ \\
\textbf{ઈઝી સેટઅપ} & સરળ કોન્ફિગરેશન & વપરાશકર્તા-મૈત્રીપૂર્ણ \\
\end{longtable}
}

\textbf{ટેકનિકલ ફાયદાઓ:}

\begin{itemize}
\tightlist
\item
  \textbf{શોર્ટ રેન્જ}: 10-30 ફૂટ કવરેજ હસ્તક્ષેપ ઘટાડે છે
\item
  \textbf{Ad-hoc networking}: ઈન્ફ્રાસ્ટ્રક્ચરની જરૂર નથી
\item
  \textbf{ડિવાઇસ મોબિલિટી}: રેન્જમાં મુક્તપણે ફરો
\item
  \textbf{ઓટોમેટિક ડિસ્કવરી}: ડિવાઇસેસ એકબીજાને આપોઆપ શોધે છે
\end{itemize}

\textbf{એપ્લિકેશન ફાયદાઓ:}

\begin{itemize}
\tightlist
\item
  \textbf{પર્સનલ ડિવાઇસેસ}: ફોન્સ, ટેબલેટ્સ, હેડફોન્સ કનેક્ટ કરો
\item
  \textbf{IoT ઇન્ટિગ્રેશન}: સ્માર્ટ હોમ ડિવાઇસ કનેક્ટિવિટી
\item
  \textbf{ફાઇલ શેરિંગ}: ડિવાઇસેસ વચ્ચે ઝડપી ડેટા ટ્રાન્સફર
\item
  \textbf{પેરિફેરલ કનેક્શન}: વાયરલેસ કીબોર્ડ્સ, માઉસ
\end{itemize}

\textbf{સિક્યોરિટી ફાયદાઓ:}

\begin{itemize}
\tightlist
\item
  \textbf{લિમિટેડ રેન્જ}: ઈવસ્ડ્રોપિંગ રિસ્ક ઘટાડવું
\item
  \textbf{એન્ક્રિપ્શન}: બિલ્ટ-ઇન સિક્યોરિટી પ્રોટોકોલ્સ
\item
  \textbf{પેરિંગ}: ઓથેન્ટિકેટેડ ડિવાઇસ કનેક્શન્સ
\end{itemize}

\end{solutionbox}
\begin{mnemonicbox}
``WPANs: વાયરલેસ પર્સનલ એરિયા નેટવર્ક્સ''

\end{mnemonicbox}
\begin{center}\rule{0.5\linewidth}{0.5pt}\end{center}

\subsection*{પ્રશ્ન 5(બ) [4
ગુણ]}\label{uxaaauxab0uxab6uxaa8-5uxaac-4-uxa97uxaa3}

\textbf{Mobile IP માં packet delivery ના steps સમજાવો.}

\begin{solutionbox}

Mobile IP માં પેકેટ ડિલિવરીમાં મોબાઇલ ડિવાઇસેસ સુધી પહોંચવા માટે અનેક પગલાં સામેલ
છે:

\textbf{પેકેટ ડિલિવરીના પગલાં:}

{\def\LTcaptype{none} % do not increment counter
\begin{longtable}[]{@{}lll@{}}
\toprule\noalign{}
પગલું & પ્રક્રિયા & સ્થાન \\
\midrule\noalign{}
\endhead
\bottomrule\noalign{}
\endlastfoot
\textbf{1. ટ્રાન્સમિશન} & હોમ એડ્રેસ પર પેકેટ મોકલો & Correspondent Node \\
\textbf{2. ઇન્ટરસેપ્શન} & મોબાઇલ નોડ માટે પેકેટ કેપ્ચર કરો & Home Agent \\
\textbf{3. ટનલિંગ} & એન્કેપ્સુલેટ અને ફોરવર્ડ કરો & Home to Foreign Agent \\
\textbf{4. ડિલિવરી} & પેકેટ એક્સટ્રેક્ટ અને ડિલિવર કરો & Foreign Agent to
Mobile \\
\end{longtable}
}

\textbf{વિગતવાર પ્રક્રિયા:}

\begin{verbatim}
CN {-{-}{-}{-}{-} HA {-}{-}{-}{-}{-} FA {-}{-}{-}{-}{-} MN}
   (1)     (2,3)    (4)
   
Step 1: હોમ નેટવર્ક પર સામાન્ય IP રાઉટિંગ
Step 2: Home Agent પેકેટ ઇન્ટરસેપ્ટ કરે છે
Step 3: care{-of address પર પેકેટ ટનલ કરે છે}
Step 4: Foreign Agent મોબાઇલ નોડને ડિલિવર કરે છે
\end{verbatim}

\textbf{ટનલિંગ મેકેનિઝમ:}

\begin{itemize}
\tightlist
\item
  \textbf{એન્કેપ્સુલેશન}: care-of address સાથે નવો IP હેડર ઉમેરો
\item
  \textbf{ફોરવર્ડિંગ}: ઇન્ટરનેટ મારફતે ફોરેન નેટવર્ક પર રાઉટ કરો
\item
  \textbf{ડીકેપ્સુલેશન}: ફોરેન એજન્ટ પર ટનલ હેડર હટાવો
\item
  \textbf{લોકલ ડિલિવરી}: મોબાઇલ નોડને સ્ટાન્ડર્ડ ડિલિવરી
\end{itemize}

\end{solutionbox}
\begin{mnemonicbox}
``કોરેસ્પોન્ડન્ટ હોમ ફોરેન મોબાઇલ''

\end{mnemonicbox}
\begin{center}\rule{0.5\linewidth}{0.5pt}\end{center}

\subsection*{પ્રશ્ન 5(ક) [7
ગુણ]}\label{uxaaauxab0uxab6uxaa8-5uxa95-7-uxa97uxaa3}

\textbf{WLAN નું આર્કિટેક્ચર આકૃતિ સાથે સમજાવો.}

\begin{solutionbox}

WLAN (Wireless Local Area Network) આર્કિટેક્ચર સ્થાનિક વિસ્તારમાં વાયરલેસ
કનેક્ટિવિટી પ્રદાન કરે છે:

\begin{center}
\textbf{Mermaid Diagram (Code)}
\begin{verbatim}
{Shaded}
{Highlighting}[]
graph LR
    A[Distribution System{br/{}વાયર્ડ બેકબોન] {-}{-}{} B[Access Point 1]}
    A {-{-}{} C[Access Point 2]}
    A {-{-}{} D[Access Point 3]}
    
    B {-{-}{} E[BSS 1{}br/{}Basic Service Set]}
    C {-{-}{} F[BSS 2{}br/{}Basic Service Set]}
    D {-{-}{} G[BSS 3{}br/{}Basic Service Set]}
    
    E {-{-}{} H[વાયરલેસ સ્ટેશન્સ]}
    F {-{-}{} I[વાયરલેસ સ્ટેશન્સ]}
    G {-{-}{} J[વાયરલેસ સ્ટેશન્સ]}
    
    K[ESS {- Extended Service Set] {-}{-}{} A}
{Highlighting}
{Shaded}
\end{verbatim}
\end{center}

\textbf{WLAN ઘટકો:}

{\def\LTcaptype{none} % do not increment counter
\begin{longtable}[]{@{}lll@{}}
\toprule\noalign{}
ઘટક & કાર્ય & કવરેજ \\
\midrule\noalign{}
\endhead
\bottomrule\noalign{}
\endlastfoot
\textbf{Station (STA)} & વાયરલેસ ડિવાઇસ & વ્યક્તિગત ડિવાઇસ \\
\textbf{Access Point (AP)} & વાયરલેસ હબ & Basic Service Set \\
\textbf{Basic Service Set (BSS)} & સિંગલ AP કવરેજ & સ્થાનિક વિસ્તાર \\
\textbf{Extended Service Set (ESS)} & મલ્ટિપલ BSS & મોટો વિસ્તાર \\
\end{longtable}
}

\textbf{આર્કિટેક્ચરના પ્રકારો:}

\textbf{Ad-hoc મોડ:}

\begin{itemize}
\tightlist
\item
  \textbf{Independent BSS}: એક્સેસ પોઈન્ટની જરૂર નથી
\item
  \textbf{Peer-to-peer}: ડાયરેક્ટ સ્ટેશન કમ્યુનિકેશન
\item
  \textbf{લિમિટેડ રેન્જ}: સિંગલ હોપ કમ્યુનિકેશન
\item
  \textbf{ટેમ્પરરી નેટવર્ક્સ}: કોન્ફરન્સ, મીટિંગ રૂમ્સ
\end{itemize}

\textbf{Infrastructure મોડ:}

\begin{itemize}
\tightlist
\item
  \textbf{Access Point}: કેન્દ્રીય સંકલન
\item
  \textbf{Distribution System}: અનેક APs કનેક્ટ કરે છે
\item
  \textbf{રોમિંગ સપોર્ટ}: BSS વિસ્તારો વચ્ચે ફરવું
\item
  \textbf{ઇન્ટરનેટ કનેક્ટિવિટી}: બાહ્ય નેટવર્ક્સ માટે ગેટવે
\end{itemize}

\textbf{મુખ્ય લક્ષણો:}

\begin{itemize}
\tightlist
\item
  \textbf{મોબિલિટી}: કવરેજ વિસ્તારમાં ફરવું
\item
  \textbf{સ્કેલેબિલિટી}: વધુ એક્સેસ પોઈન્ટ્સ ઉમેરો
\item
  \textbf{ઇન્ટરઓપેરેબિલિટી}: IEEE 802.11 સ્ટાન્ડર્ડ્સ
\item
  \textbf{સિક્યોરિટી}: WPA/WPA2 એન્ક્રિપ્શન
\end{itemize}

\textbf{પ્રદાન કરવામાં આવતી સેવાઓ:}

\begin{itemize}
\tightlist
\item
  \textbf{Association}: એક્સેસ પોઈન્ટ સાથે કનેક્ટ થવું
\item
  \textbf{Authentication}: વપરાશકર્તા ક્રેડેન્શિયલ્સ ચકાસવા
\item
  \textbf{Data delivery}: વિશ્વસનીય ફ્રેમ ટ્રાન્સમિશન
\item
  \textbf{Power management}: બેટરી ઓપ્ટિમાઇઝેશન
\end{itemize}

\textbf{સ્ટાન્ડર્ડ્સ:}

\begin{itemize}
\tightlist
\item
  \textbf{802.11a}: 5 GHz, 54 Mbps
\item
  \textbf{802.11b}: 2.4 GHz, 11 Mbps
\item
  \textbf{802.11g}: 2.4 GHz, 54 Mbps
\item
  \textbf{802.11n}: MIMO, 600 Mbps
\item
  \textbf{802.11ac}: 5 GHz, 1 Gbps+
\end{itemize}

\end{solutionbox}
\begin{mnemonicbox}
``WLAN: વાયરલેસ લોકલ એરિયા નેટવર્ક''

\end{mnemonicbox}
\begin{center}\rule{0.5\linewidth}{0.5pt}\end{center}

\subsection*{પ્રશ્ન 5(અ અથવા) [3
ગુણ]}\label{uxaaauxab0uxab6uxaa8-5uxa85-uxa85uxaa5uxab5-3-uxa97uxaa3}

\textbf{5G mobile network ની વિશેષતાઓ લખો.}

\begin{solutionbox}

5G ક્રાંતિકારી મોબાઇલ નેટવર્ક ક્ષમતાઓ પ્રદાન કરે છે:

\textbf{મુખ્ય લક્ષણો:}

{\def\LTcaptype{none} % do not increment counter
\begin{longtable}[]{@{}lll@{}}
\toprule\noalign{}
લક્ષણ & સ્પેસિફિકેશન & ફાયદો \\
\midrule\noalign{}
\endhead
\bottomrule\noalign{}
\endlastfoot
\textbf{સ્પીડ} & 10 Gbps સુધી & અલ્ટ્રા-ફાસ્ટ ડાઉનલોડ્સ \\
\textbf{લેટન્સી} & 1ms કરતાં ઓછું & રિયલ-ટાઇમ એપ્લિકેશન્સ \\
\textbf{ડેન્સિટી} & 1M devices/km^{2} & મેસિવ IoT સપોર્ટ \\
\end{longtable}
}

\textbf{ટેકનિકલ ક્ષમતાઓ:}

\begin{itemize}
\tightlist
\item
  \textbf{Enhanced Mobile Broadband}: હાઇ-સ્પીડ ઇન્ટરનેટ એક્સેસ
\item
  \textbf{Ultra-Reliable Low Latency}: જટિલ એપ્લિકેશન્સ
\item
  \textbf{Massive Machine Communication}: IoT ડિવાઇસ કનેક્ટિવિટી
\end{itemize}

\textbf{એડવાન્સ્ડ ટેક્નોલોજીઝ:}

\begin{itemize}
\tightlist
\item
  \textbf{Millimeter waves}: ઉચ્ચ ફ્રીક્વન્સી બેન્ડ્સ
\item
  \textbf{MIMO}: મલ્ટિપલ એન્ટેના સિસ્ટમ્સ
\item
  \textbf{Network slicing}: વર્ચ્યુઅલ નેટવર્ક પાર્ટીશન્સ
\item
  \textbf{Edge computing}: વિતરિત પ્રોસેસિંગ
\end{itemize}

\textbf{એપ્લિકેશન્સ:}

\begin{itemize}
\tightlist
\item
  \textbf{ઓટોનોમસ વાહનો}: રિયલ-ટાઇમ કંટ્રોલ
\item
  \textbf{સ્માર્ટ સિટીઝ}: કનેક્ટેડ ઇન્ફ્રાસ્ટ્રક્ચર
\item
  \textbf{ઇન્ડસ્ટ્રિયલ IoT}: ફેક્ટરી ઓટોમેશન
\end{itemize}

\end{solutionbox}
\begin{mnemonicbox}
``5G: ફિફ્થ જનરેશન ગ્રેટ સ્પીડ''

\end{mnemonicbox}
\begin{center}\rule{0.5\linewidth}{0.5pt}\end{center}

\subsection*{પ્રશ્ન 5(બ અથવા) [4
ગુણ]}\label{uxaaauxab0uxab6uxaa8-5uxaac-uxa85uxaa5uxab5-4-uxa97uxaa3}

\textbf{Mobile network ના સંદર્ભમાં DHCP કેવી રીતે કામ કરે છે તે સમજાવો.}

\begin{solutionbox}

DHCP (Dynamic Host Configuration Protocol) મોબાઇલ નેટવર્ક્સમાં આપોઆપ IP
એડ્રેસેસ સોંપે છે:

\textbf{મોબાઇલ નેટવર્ક્સમાં DHCP પ્રક્રિયા:}

{\def\LTcaptype{none} % do not increment counter
\begin{longtable}[]{@{}llll@{}}
\toprule\noalign{}
પગલું & મેસેજ & હેતુ & દિશા \\
\midrule\noalign{}
\endhead
\bottomrule\noalign{}
\endlastfoot
\textbf{1} & DHCP Discover & DHCP સર્વર શોધો & Client \rightarrow Broadcast \\
\textbf{2} & DHCP Offer & IP એડ્રેસ ઓફર કરો & Server \rightarrow Client \\
\textbf{3} & DHCP Request & વિશિષ્ટ IP રિક્વેસ્ટ કરો & Client \rightarrow Server \\
\textbf{4} & DHCP ACK & એસાઇનમેન્ટ કન્ફર્મ કરો & Server \rightarrow Client \\
\end{longtable}
}

\textbf{મોબાઇલ નેટવર્ક પડકારો:}

\begin{verbatim}
Mobile DHCP Process:

ડિવાઇસ હલે છે: Network A  Network B

Network A          Network B
DHCP Server        DHCP Server
    |                  |
    |{-{-} Release       |}
    |                  |
    |              Discover |
    |               Offer {-{-}|}
    |              Request |
    |               ACK {-{-}{-}{-}|}
\end{verbatim}

\textbf{મોબાઇલ-સ્પેસિફિક લક્ષણો:}

\begin{itemize}
\tightlist
\item
  \textbf{ફાસ્ટ હેન્ડઓવર}: હલનચલન દરમિયાન ઝડપી IP એસાઇનમેન્ટ
\item
  \textbf{લીઝ રિન્યૂઅલ}: IP એડ્રેસ માન્યતા લંબાવવી
\item
  \textbf{કોન્ફ્લિક્ટ રિઝોલ્યુશન}: ડુપ્લિકેટ એડ્રેસેસ હેન્ડલ કરવા
\item
  \textbf{લોકેશન અપડેટ}: ડિવાઇસ લોકેશનની નેટવર્કને જાણ કરવી
\end{itemize}

\textbf{કોન્ફિગરેશન માહિતી:}

\begin{itemize}
\tightlist
\item
  \textbf{IP એડ્રેસ}: અનન્ય નેટવર્ક આઇડેન્ટિફાયર
\item
  \textbf{સબનેટ માસ્ક}: નેટવર્ક બાઉન્ડરી ડેફિનિશન
\item
  \textbf{ડિફોલ્ટ ગેટવે}: બાહ્ય કમ્યુનિકેશન માટે રાઉટર
\item
  \textbf{DNS સર્વર્સ}: ડોમેન નેમ રિઝોલ્યુશન
\end{itemize}

\textbf{મોબાઇલ કોન્ટેક્સ્ટમાં ફાયદાઓ:}

\begin{itemize}
\tightlist
\item
  \textbf{ઓટોમેટિક કોન્ફિગરેશન}: મેન્યુઅલ સેટઅપની જરૂર નથી
\item
  \textbf{એડ્રેસ કન્ઝર્વેશન}: એડ્રેસેસનો કાર્યક્ષમ પુનઃઉપયોગ
\item
  \textbf{મોબિલિટી સપોર્ટ}: સીમલેસ નેટવર્ક ટ્રાન્ઝિશન
\end{itemize}

\end{solutionbox}
\begin{mnemonicbox}
``DHCP: ડિસ્કવર ઓફર રિક્વેસ્ટ ACK''

\end{mnemonicbox}
\begin{center}\rule{0.5\linewidth}{0.5pt}\end{center}

\subsection*{પ્રશ્ન 5(ક અથવા) [7
ગુણ]}\label{uxaaauxab0uxab6uxaa8-5uxa95-uxa85uxaa5uxab5-7-uxa97uxaa3}

\textbf{Bluetooth technology તેના protocol stack ની સ્વચ્છ આકૃતિ સાથે
સમજાવો.}

\begin{solutionbox}

Bluetooth પર્સનલ ડિવાઇસેસ માટે ટૂંકા-અંતરની વાયરલેસ કમ્યુનિકેશન પ્રદાન કરે છે:

\begin{center}
\textbf{Mermaid Diagram (Code)}
\begin{verbatim}
{Shaded}
{Highlighting}[]
graph LR
    A[Applications] {-{-}{} B[Application Layer]}
    B {-{-}{} C[L2CAP{}br/{}Logical Link Control]}
    C {-{-}{} D[HCI{}br/{}Host Controller Interface]}
    D {-{-}{} E[Link Manager Protocol{}br/{}LMP]}
    E {-{-}{} F[Baseband Layer]}
    F {-{-}{} G[Radio Layer]}
    
    H[RFCOMM{br/{}Serial Port] {-}{-}{} C}
    I[SDP{br/{}Service Discovery] {-}{-}{} C}
    J[OBEX{br/{}Object Exchange] {-}{-}{} B}
{Highlighting}
{Shaded}
\end{verbatim}
\end{center}

\textbf{પ્રોટોકોલ સ્ટેક લેયર્સ:}

{\def\LTcaptype{none} % do not increment counter
\begin{longtable}[]{@{}lll@{}}
\toprule\noalign{}
લેયર & કાર્ય & હેતુ \\
\midrule\noalign{}
\endhead
\bottomrule\noalign{}
\endlastfoot
\textbf{Radio} & ભૌતિક ટ્રાન્સમિશન & 2.4 GHz ISM બેન્ડ \\
\textbf{Baseband} & મીડિયા એક્સેસ કંટ્રોલ & ટાઇમ ડિવિઝન ડુપ્લેક્સ \\
\textbf{LMP} & લિંક મેનેજમેન્ટ & કનેક્શન સ્થાપના \\
\textbf{HCI} & હોસ્ટ-કંટ્રોલર ઇન્ટરફેસ & હાર્ડવેર એબ્સ્ટ્રેક્શન \\
\textbf{L2CAP} & લોજિકલ લિંક કંટ્રોલ & પેકેટ સેગમેન્ટેશન \\
\textbf{Applications} & વપરાશકર્તા સેવાઓ & ફાઇલ ટ્રાન્સફર, ઓડિયો \\
\end{longtable}
}

\textbf{તકનીકી સ્પેસિફિકેશન્સ:}

\textbf{ફિઝિકલ લેયર:}

\begin{itemize}
\tightlist
\item
  \textbf{ફ્રીક્વન્સી}: 2.4 GHz ISM બેન્ડ
\item
  \textbf{હોપિંગ}: 79 ફ્રીક્વન્સી ચેનલ્સ
\item
  \textbf{મોડ્યુલેશન}: ફ્રીક્વન્સી શિફ્ટ કીઇંગ
\item
  \textbf{પાવર ક્લાસેસ}: 1mW થી 100mW
\end{itemize}

\textbf{નેટવર્ક ટોપોલોજી:}

\begin{verbatim}
Bluetooth Piconet:

    [Slave 1]
         |
[Slave 2]{-{-}{-}[Master]{-}{-}{-}[Slave 3]}
         |
    [Slave 4]

પિકોનેટ દીઠ મહત્તમ 8 ડિવાઇસેસ
માસ્ટર કમ્યુનિકેશન કંટ્રોલ કરે છે
\end{verbatim}

\textbf{કનેક્શનના પ્રકારો:}

\begin{itemize}
\tightlist
\item
  \textbf{SCO}: સિંક્રોનસ કનેક્શન-ઓરિએન્ટેડ (વૉઇસ)
\item
  \textbf{ACL}: એસિંક્રોનસ કનેક્શન-લેસ (ડેટા)
\item
  \textbf{eSCO}: એન્હાન્સ્ડ SCO (સુધારેલ વૉઇસ)
\end{itemize}

\textbf{સિક્યોરિટી લક્ષણો:}

\begin{itemize}
\tightlist
\item
  \textbf{Authentication}: ડિવાઇસ ઓળખ ચકાસણી
\item
  \textbf{Authorization}: સર્વિસ એક્સેસ કંટ્રોલ
\item
  \textbf{Encryption}: ડેટા પ્રોટેક્શન (E0 આલ્ગોરિધમ)
\item
  \textbf{Key management}: સિક્યોરિટી કી એક્સચેન્જ
\end{itemize}

\textbf{Bluetooth વર્ઝન્સ:}

{\def\LTcaptype{none} % do not increment counter
\begin{longtable}[]{@{}llll@{}}
\toprule\noalign{}
વર્ઝન & સ્પીડ & રેન્જ & લક્ષણો \\
\midrule\noalign{}
\endhead
\bottomrule\noalign{}
\endlastfoot
\textbf{1.x} & 1 Mbps & 10m & બેસિક કનેક્ટિવિટી \\
\textbf{2.x} & 3 Mbps & 10m & એન્હાન્સ્ડ ડેટા રેટ \\
\textbf{3.x} & 24 Mbps & 10m & હાઇ-સ્પીડ ઓપ્શન \\
\textbf{4.x} & 1 Mbps & 50m & લો એનર્જી (BLE) \\
\textbf{5.x} & 2 Mbps & 240m & સુધારેલ રેન્જ/સ્પીડ \\
\end{longtable}
}

\textbf{એપ્લિકેશન્સ:}

\begin{itemize}
\tightlist
\item
  \textbf{ઓડિયો સ્ટ્રીમિંગ}: હેડફોન્સ, સ્પીકર્સ
\item
  \textbf{ફાઇલ ટ્રાન્સફર}: ડોક્યુમેન્ટ્સ, ફોટોઝ
\item
  \textbf{ઇનપુટ ડિવાઇસેસ}: કીબોર્ડ્સ, માઉસ
\item
  \textbf{હેલ્થ મોનિટરિંગ}: ફિટનેસ ટ્રેકર્સ
\end{itemize}

\textbf{ફાયદાઓ:}

\begin{itemize}
\tightlist
\item
  \textbf{લો પાવર}: બેટરી-ફ્રેન્ડલી ઓપરેશન
\item
  \textbf{ઈઝી પેરિંગ}: સરળ ડિવાઇસ કનેક્શન
\item
  \textbf{ઇન્ટરઓપેરેબિલિટી}: યુનિવર્સલ સ્ટાન્ડર્ડ
\item
  \textbf{કોસ્ટ-ઇફેક્ટિવ}: સસ્તું અમલીકરણ
\end{itemize}

\end{solutionbox}
\begin{mnemonicbox}
``Bluetooth: રેડિયો બેસબેન્ડ LMP HCI L2CAP એપ્લિકેશન્સ''

\end{mnemonicbox}

\end{document}
