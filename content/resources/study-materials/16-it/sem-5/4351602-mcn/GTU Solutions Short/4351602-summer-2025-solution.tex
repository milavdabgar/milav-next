\documentclass{article}
% Adjust the relative path to point to the latex-templates directory

% content/resources/templates/preamble.tex
\usepackage[margin=0.6in]{geometry}
\author{Milav Dabgar}
\usepackage{amsmath,amssymb,amsthm}
\usepackage{booktabs}
\usepackage{multirow}
\usepackage{xcolor}
\usepackage{tcolorbox}
\tcbuselibrary{breakable,skins}
\usepackage[colorlinks=true,linkcolor=blue]{hyperref}
\usepackage{titlesec}
\usepackage{enumitem}
\usepackage{tikz}
\usepackage{pgfplots}
\usepackage{circuitikz}
\usepackage[version=4]{mhchem}
\usepackage{longtable}
\usepackage{array}
\usepackage{float}
\usepackage{caption}
\usepackage{listings}

\lstset{
  basicstyle=\small\ttfamily,
  breaklines=true,
  breakatwhitespace=false,
  postbreak=\mbox{\textcolor{red}{$\hookrightarrow$}\space},
  float=false,
  numbers=left,
  numberstyle=\tiny\color{gray},
  numbersep=10pt,
  xleftmargin=2em,
  keywordstyle=\color{blue},
  commentstyle=\color{green!60!black},
  stringstyle=\color{purple},
  backgroundcolor=\color{gray!5},
  showstringspaces=false,
  tabsize=2,
  captionpos=b,
  keepspaces=true,
  columns=flexible
}

\pgfplotsset{compat=1.18}
\usetikzlibrary{shapes,arrows,positioning,calc,patterns,decorations.pathmorphing,decorations.markings,arrows.meta}

% Color scheme
\definecolor{headcolor}{RGB}{0,102,204}
\definecolor{keycolor}{RGB}{220,20,60}
\definecolor{solutioncolor}{RGB}{34,139,34}
\definecolor{mnemoniccolor}{RGB}{148,0,211}
\definecolor{codecolor}{RGB}{0,0,100}

% Spacing
\setlength{\parskip}{3pt}
\setlist[itemize]{nosep}
\setlist[enumerate]{nosep}

% Title formatting
\titleformat{\section}{\Large\bfseries\color{headcolor}}{\thesection}{1em}{}
\titleformat{\subsection}{\large\bfseries\color{headcolor}}{\thesubsection}{1em}{}

% Pandoc tightlist compatibility
\providecommand{\tightlist}{%
  \setlength{\itemsep}{0pt}\setlength{\parskip}{0pt}}

% Pandoc longtable compatibility
\newcounter{none}
\def\thenone{}


% content/resources/templates/english-boxes.tex

% Custom environments
\newtcolorbox{solutionbox}{
 breakable,
 enhanced,
 colback=solutioncolor!5!white,
 colframe=solutioncolor!75!black,
 fonttitle=\bfseries,
 title=Solution
}

\newtcolorbox{solutionboxnobreak}{
 colback=solutioncolor!5!white,
 colframe=solutioncolor!75!black,
 fonttitle=\bfseries,
 title=Solution
}

\newtcolorbox{keyformula}{
 breakable,
 enhanced,
 colback=keycolor!5!white,
 colframe=keycolor!75!black,
 fonttitle=\bfseries,
 title=Key Formula
}

\newtcolorbox{mnemonicboxenv}{
 breakable,
 enhanced,
 colback=mnemoniccolor!5!white,
 colframe=mnemoniccolor!75!black,
 fonttitle=\bfseries,
 title=Mnemonic
}

\newcommand{\mnemonicbox}[1]{%
  \begin{mnemonicboxenv}
    #1
  \end{mnemonicboxenv}
}


% Custom commands for GTU solutions
% This file defines semantic commands for consistent formatting

% Question command with automatic formatting
\newcommand{\question}[2]{%
  \section*{Question #1}%
  \textbf{#2}%
}

% OR question variant
\newcommand{\questionor}[2]{%
  \section*{Question #1 OR}%
  \textbf{#2}%
}

% Proper table environment with caption
\newenvironment{answertable}[1]{%
  \begin{table}[htbp]
  \centering
  \caption{#1}
}{%
  \end{table}
}

% Proper figure environment for diagrams
\newenvironment{answerdiagram}[1]{%
  \begin{figure}[htbp]
  \centering
  \caption{#1}
}{%
  \end{figure}
}

% Semantic markup for key terms
\newcommand{\keyword}[1]{\textbf{#1}}
\newcommand{\code}[1]{\texttt{#1}}
\newcommand{\classname}[1]{\texttt{#1}}
\newcommand{\methodname}[1]{\texttt{#1}}

% Proper quotation marks
\newcommand{\mnemonic}[1]{``#1''}

\usetikzlibrary{fit, positioning, arrows.meta, shapes.geometric, calc}

\title{Mobile Computing and Networks (4351602) - Summer 2025 Solution}
\date{May 14, 2025}

\begin{document}
\maketitle

\questionmarks{1(a)}{3}{Explain working of POP protocol}

\begin{solutionbox}
POP (Post Office Protocol) is an email retrieval protocol that downloads emails from server to client device.

\textbf{Working Process:}
\begin{center}
\captionof{table}{POP Protocol Steps}
\begin{tabulary}{\linewidth}{|C|L|L|}
\hline
\textbf{Step} & \textbf{Action} & \textbf{Description} \\ \hline
1 & Connection & Client connects to POP server on port 110 \\ \hline
2 & Authentication & User provides username and password \\ \hline
3 & Download & Emails downloaded to local device \\ \hline
4 & Deletion & Emails deleted from server after download \\ \hline
\end{tabulary}
\end{center}

\textbf{Key Points:}
\begin{itemize}
    \item \keyword{Download-based}: Emails stored locally on client device
    \item \keyword{Offline access}: Can read emails without internet connection
    \item \keyword{Single device}: Best suited for single device access
\end{itemize}
\end{solutionbox}

\begin{mnemonicbox}
\mnemonic{POP Downloads Once Permanently}
\end{mnemonicbox}

\questionmarks{1(b)}{4}{Compare OSI model with TCP/IP model}

\begin{solutionbox}
Comparison between OSI and TCP/IP networking models:

\begin{center}
\captionof{table}{OSI vs TCP/IP Model}
\begin{tabulary}{\linewidth}{|L|L|L|}
\hline
\textbf{Aspect} & \textbf{OSI Model} & \textbf{TCP/IP Model} \\ \hline
\textbf{Layers} & 7 layers & 4 layers \\ \hline
\textbf{Approach} & Theoretical model & Practical implementation \\ \hline
\textbf{Development} & ISO standard & DARPA project \\ \hline
\textbf{Complexity} & More complex & Simpler structure \\ \hline
\end{tabulary}
\end{center}

\textbf{Key Differences:}
\begin{itemize}
    \item \keyword{Layer count}: OSI has 7 layers vs TCP/IP's 4 layers
    \item \keyword{Real-world usage}: TCP/IP widely implemented, OSI mostly theoretical
    \item \keyword{Protocol independence}: OSI is protocol-independent, TCP/IP is protocol-specific
    \item \keyword{Header overhead}: OSI has more overhead due to additional layers
\end{itemize}
\end{solutionbox}

\begin{mnemonicbox}
\mnemonic{OSI Seven Theoretical, TCP Four Practical}
\end{mnemonicbox}

\questionmarks{1(c)}{7}{Explain protocols working at each layer in TCP/IP models}

\begin{solutionbox}
TCP/IP model consists of 4 layers with specific protocols at each layer:

\begin{center}
\begin{tikzpicture}[node distance=0.8cm, auto]
    \node [gtu block, minimum width=6cm] (app) {Application Layer\\ \footnotesize (HTTP, FTP, SMTP, DNS)};
    \node [gtu block, minimum width=6cm, below=of app] (trans) {Transport Layer\\ \footnotesize (TCP, UDP)};
    \node [gtu block, minimum width=6cm, below=of trans] (internet) {Internet Layer\\ \footnotesize (IP, ICMP, ARP)};
    \node [gtu block, minimum width=6cm, below=of internet] (net) {Network Access Layer\\ \footnotesize (Ethernet, WiFi)};
    
    \draw [gtu arrow] (app) -- (trans);
    \draw [gtu arrow] (trans) -- (internet);
    \draw [gtu arrow] (internet) -- (net);
\end{tikzpicture}
\captionof{figure}{TCP/IP Protocols}
\end{center}

\textbf{Layer-wise Protocol Functions:}
\begin{center}
\captionof{table}{TCP/IP Layer Protocols}
\begin{tabulary}{\linewidth}{|L|L|L|}
\hline
\textbf{Layer} & \textbf{Protocols} & \textbf{Function} \\ \hline
\textbf{Application} & HTTP, FTP, SMTP, DNS & User interface and services \\ \hline
\textbf{Transport} & TCP, UDP & End-to-end communication \\ \hline
\textbf{Internet} & IP, ICMP, ARP & Routing and addressing \\ \hline
\textbf{Network Access} & Ethernet, WiFi & Physical transmission \\ \hline
\end{tabulary}
\end{center}

\textbf{Protocol Details:}
\begin{itemize}
    \item \keyword{HTTP/HTTPS}: Web communication and secure web communication
    \item \keyword{TCP}: Reliable, connection-oriented data transfer
    \item \keyword{UDP}: Fast, connectionless data transfer
    \item \keyword{IP}: Packet routing and addressing
    \item \keyword{ARP}: Maps IP addresses to MAC addresses
\end{itemize}
\end{solutionbox}

\begin{mnemonicbox}
\mnemonic{Applications Transport Internet Networks Always}
\end{mnemonicbox}

\questionmarks{1(c OR)}{7}{Briefly explain OSI model with all its layers and functionality of each layer}

\begin{solutionbox}
OSI (Open Systems Interconnection) model has 7 layers for network communication:

\begin{center}
\begin{tikzpicture}[node distance=0.6cm, auto]
    \node [gtu block, minimum width=6cm] (app) {7. Application Layer};
    \node [gtu block, minimum width=6cm, below=of app] (pres) {6. Presentation Layer};
    \node [gtu block, minimum width=6cm, below=of pres] (sess) {5. Session Layer};
    \node [gtu block, minimum width=6cm, below=of sess] (trans) {4. Transport Layer};
    \node [gtu block, minimum width=6cm, below=of trans] (net) {3. Network Layer};
    \node [gtu block, minimum width=6cm, below=of net] (data) {2. Data Link Layer};
    \node [gtu block, minimum width=6cm, below=of data] (phys) {1. Physical Layer};

    \draw [gtu arrow] (app) -- (pres);
    \draw [gtu arrow] (pres) -- (sess);
    \draw [gtu arrow] (sess) -- (trans);
    \draw [gtu arrow] (trans) -- (net);
    \draw [gtu arrow] (net) -- (data);
    \draw [gtu arrow] (data) -- (phys);
\end{tikzpicture}
\captionof{figure}{OSI Model Layers}
\end{center}

\textbf{Layer Functions:}
\begin{center}
\captionof{table}{OSI Layers}
\begin{tabulary}{\linewidth}{|C|L|L|L|}
\hline
\textbf{Layer} & \textbf{Name} & \textbf{Function} & \textbf{Protocols} \\ \hline
\textbf{7} & Application & User interface & HTTP, FTP \\ \hline
\textbf{6} & Presentation & Data formatting, encryption & SSL, JPEG \\ \hline
\textbf{5} & Session & Session management & NetBIOS, RPC \\ \hline
\textbf{4} & Transport & End-to-end delivery & TCP, UDP \\ \hline
\textbf{3} & Network & Routing & IP, ICMP \\ \hline
\textbf{2} & Data Link & Frame transmission & Ethernet, PPP \\ \hline
\textbf{1} & Physical & Bit transmission & Cables, Radio \\ \hline
\end{tabulary}
\end{center}

\textbf{Key Features:}
\begin{itemize}
    \item \keyword{Modular design}: Each layer has specific responsibilities
    \item \keyword{Protocol independence}: Layers can use different protocols
    \item \keyword{Standardization}: Universal networking reference model
\end{itemize}
\end{solutionbox}

\begin{mnemonicbox}
\mnemonic{All People Seem To Need Data Processing}
\end{mnemonicbox}

\questionmarks{2(a)}{3}{Give the difference between ARP and RARP protocols}

\begin{solutionbox}
ARP and RARP are address resolution protocols with opposite functions:

\begin{center}
\captionof{table}{ARP vs RARP}
\begin{tabulary}{\linewidth}{|L|L|L|}
\hline
\textbf{Aspect} & \textbf{ARP} & \textbf{RARP} \\ \hline
\textbf{Full Form} & Address Resolution Protocol & Reverse Address Resolution Protocol \\ \hline
\textbf{Purpose} & IP to MAC address mapping & MAC to IP address mapping \\ \hline
\textbf{Direction} & Logical to Physical & Physical to Logical \\ \hline
\textbf{Usage} & Normal network communication & Diskless workstations \\ \hline
\end{tabulary}
\end{center}

\textbf{Working Process:}
\begin{itemize}
    \item \keyword{ARP}: "I know IP address, need MAC address"
    \item \keyword{RARP}: "I know MAC address, need IP address"
    \item \keyword{Cache}: Both maintain address tables for efficiency
\end{itemize}
\end{solutionbox}

\begin{mnemonicbox}
\mnemonic{ARP Asks Physical, RARP Requests IP}
\end{mnemonicbox}

\questionmarks{2(b)}{4}{Explain working of IMAP protocol}

\begin{solutionbox}
IMAP (Internet Message Access Protocol) manages emails on server for multiple device access.

\textbf{Working Process:}
\begin{center}
\captionof{table}{IMAP Process}
\begin{tabulary}{\linewidth}{|C|L|L|}
\hline
\textbf{Step} & \textbf{Action} & \textbf{Description} \\ \hline
1 & Connection & Client connects to IMAP server (port 143/993) \\ \hline
2 & Authentication & Login with credentials \\ \hline
3 & Folder Access & Browse email folders on server \\ \hline
4 & Synchronization & Changes sync across all devices \\ \hline
\end{tabulary}
\end{center}

\textbf{Key Features:}
\begin{itemize}
    \item \keyword{Server-based}: Emails remain on server
    \item \keyword{Multi-device}: Access from multiple devices
    \item \keyword{Synchronization}: Changes reflected everywhere
    \item \keyword{Selective download}: Download only needed emails
\end{itemize}

\textbf{Advantages:}
\begin{itemize}
    \item \keyword{Storage efficiency}: Server manages storage
    \item \keyword{Accessibility}: Access from anywhere
    \item \keyword{Backup}: Server provides automatic backup
\end{itemize}
\end{solutionbox}

\begin{mnemonicbox}
\mnemonic{IMAP Internet Messages Always Present}
\end{mnemonicbox}

\questionmarks{2(c)}{7}{Explain Three-tier architecture of mobile computing with appropriate diagram}

\begin{solutionbox}
Three-tier architecture separates mobile computing into distinct layers:

\begin{center}
\begin{tikzpicture}[node distance=1cm, auto]
    \node [gtu block, align=center] (pres) {Presentation Tier\\ \footnotesize (Mobile Devices)};
    \node [gtu block, right=of pres, align=center] (app) {Application Tier\\ \footnotesize (App Server)};
    \node [gtu block, right=of app, align=center] (data) {Data Tier\\ \footnotesize (Database Server)};

    \draw [gtu arrow] (pres) -- (app);
    \draw [gtu arrow] (app) -- (data);
    \draw [gtu arrow] (data) -- (app);
    \draw [gtu arrow] (app) -- (pres);
    
    \node [below=0.2cm of pres, align=center, font=\scriptsize] {Smartphones, Tablets};
    \node [below=0.2cm of app, align=center, font=\scriptsize] {Business Logic, APIs};
    \node [below=0.2cm of data, align=center, font=\scriptsize] {DB, File Systems};
\end{tikzpicture}
\captionof{figure}{Three-Tier Mobile Architecture}
\end{center}

\textbf{Tier Details:}
\begin{center}
\captionof{table}{Architecture Tiers}
\begin{tabulary}{\linewidth}{|L|L|L|}
\hline
\textbf{Tier} & \textbf{Components} & \textbf{Responsibilities} \\ \hline
\textbf{Presentation} & Mobile devices, UI & User interface and interaction \\ \hline
\textbf{Application} & App servers, middleware & Business logic and processing \\ \hline
\textbf{Data} & Databases, storage & Data management and storage \\ \hline
\end{tabulary}
\end{center}

\textbf{Architecture Benefits:}
\begin{itemize}
    \item \keyword{Scalability}: Each tier can scale independently
    \item \keyword{Maintainability}: Separate concerns for easier updates
    \item \keyword{Security}: Data protection through tier separation
    \item \keyword{Performance}: Distributed processing reduces load
\end{itemize}
\end{solutionbox}

\begin{mnemonicbox}
\mnemonic{Presentation Applies Data Processing}
\end{mnemonicbox}

\questionmarks{2(a OR)}{3}{Explain the limitation of Stop-and-wait data link layer protocol}

\begin{solutionbox}
Stop-and-wait protocol has several performance limitations:

\textbf{Major Limitations:}
\begin{center}
\captionof{table}{Stop-and-Wait Limitations}
\begin{tabulary}{\linewidth}{|L|L|L|}
\hline
\textbf{Limitation} & \textbf{Description} & \textbf{Impact} \\ \hline
\textbf{Low Efficiency} & Waits for ACK before next frame & Poor bandwidth utilization \\ \hline
\textbf{High Delay} & Round-trip delay for each frame & Slow data transmission \\ \hline
\textbf{Error Sensitivity} & Single error stops transmission & Reduced reliability \\ \hline
\end{tabulary}
\end{center}

\textbf{Performance Issues:}
\begin{itemize}
    \item \keyword{Bandwidth waste}: Link remains idle during wait time
    \item \keyword{Timeout problems}: Lost ACK causes unnecessary retransmission
    \item \keyword{Sequential processing}: Cannot send multiple frames simultaneously
\end{itemize}
\end{solutionbox}

\begin{mnemonicbox}
\mnemonic{Stop Waits, Bandwidth Wastes}
\end{mnemonicbox}

\questionmarks{2(b OR)}{4}{Explain Advantages of IPV6 over the older IPV4 addressing scheme}

\begin{solutionbox}
IPv6 provides significant improvements over IPv4:

\textbf{Key Advantages:}
\begin{center}
\captionof{table}{IPv4 vs IPv6}
\begin{tabulary}{\linewidth}{|L|L|L|}
\hline
\textbf{Feature} & \textbf{IPv4} & \textbf{IPv6} \\ \hline
\textbf{Address Space} & 32-bit (4.3 billion) & 128-bit (Undecillion) \\ \hline
\textbf{Header} & Variable length & Fixed 40 bytes \\ \hline
\textbf{Security} & Optional IPSec & Built-in IPSec \\ \hline
\textbf{Configuration} & Manual/DHCP & Auto-configuration \\ \hline
\end{tabulary}
\end{center}

\textbf{Major Benefits:}
\begin{itemize}
    \item \keyword{Unlimited addresses}: Solves address exhaustion problem
    \item \keyword{Better performance}: Simplified header processing
    \item \keyword{Enhanced security}: Mandatory encryption support
    \item \keyword{Mobility support}: Better mobile device connectivity
\end{itemize}
\end{solutionbox}

\begin{mnemonicbox}
\mnemonic{IPv6 Improves Performance, Security, Addresses}
\end{mnemonicbox}

\questionmarks{2(c OR)}{7}{Enlist types of networks available in mobile computing. Explain one of them in detail}

\begin{solutionbox}
\textbf{Types of Mobile Networks:}
\begin{center}
\captionof{table}{Mobile Network Generations}
\begin{tabulary}{\linewidth}{|L|L|L|L|}
\hline
\textbf{Generation} & \textbf{Technology} & \textbf{Speed} & \textbf{Features} \\ \hline
\textbf{2G} & GSM, CDMA & 64 Kbps & Voice + SMS \\ \hline
\textbf{3G} & UMTS, CDMA2000 & 2 Mbps & Data services \\ \hline
\textbf{4G} & LTE, WiMAX & 100 Mbps & High-speed internet \\ \hline
\textbf{5G} & New Radio (NR) & 10 Gbps & Ultra-low latency \\ \hline
\end{tabulary}
\end{center}

\textbf{Detailed: 4G LTE Network}
\begin{center}
\begin{tikzpicture}[node distance=1cm, auto]
    \node [gtu state] (ue) {Mobile Device};
    \node [gtu block, right=of ue] (enodeb) {eNodeB};
    \node [gtu block, right=of enodeb] (sgw) {S-GW};
    \node [gtu block, right=of sgw] (pgw) {P-GW};
    \node [gtu cloud, right=of pgw] (internet) {Internet};
    \node [gtu block, above=of sgw] (mme) {MME};
    \node [gtu block, above=of mme] (hss) {HSS};

    \draw [gtu arrow] (ue) -- (enodeb);
    \draw [gtu arrow] (enodeb) -- (sgw);
    \draw [gtu arrow] (sgw) -- (pgw);
    \draw [gtu arrow] (pgw) -- (internet);
    \draw [gtu arrow] (enodeb) -- (mme);
    \draw [gtu arrow] (mme) -- (sgw);
    \draw [gtu arrow] (mme) -- (hss);
\end{tikzpicture}
\captionof{figure}{4G LTE Architecture}
\end{center}

\textbf{4G LTE Features:}
\begin{itemize}
    \item \keyword{High Speed}: Up to 100 Mbps download
    \item \keyword{Low Latency}: Less than 10ms for real-time applications
    \item \keyword{All-IP Network}: Packet-switched architecture
    \item \keyword{Advanced Antenna}: MIMO technology for better coverage
\end{itemize}
\end{solutionbox}

\begin{mnemonicbox}
\mnemonic{4G LTE: Long Term Evolution}
\end{mnemonicbox}

\questionmarks{3(a)}{3}{Explain types of Routing}

\begin{solutionbox}
Routing determines path for data packets across networks:

\textbf{Types of Routing:}
\begin{center}
\captionof{table}{Routing Types}
\begin{tabulary}{\linewidth}{|L|L|L|}
\hline
\textbf{Type} & \textbf{Description} & \textbf{Example} \\ \hline
\textbf{Static} & Manual route configuration & Administrative setup \\ \hline
\textbf{Dynamic} & Automatic route discovery & RIP, OSPF protocols \\ \hline
\textbf{Default} & Fallback route & Gateway of last resort \\ \hline
\end{tabulary}
\end{center}

\textbf{Routing Categories:}
\begin{itemize}
    \item \keyword{Distance Vector}: Uses hop count (RIP)
    \item \keyword{Link State}: Uses network topology (OSPF)
    \item \keyword{Hybrid}: Combines both approaches (EIGRP)
\end{itemize}
\end{solutionbox}

\begin{mnemonicbox}
\mnemonic{Static Dynamic Default Routes}
\end{mnemonicbox}

\questionmarks{3(b)}{4}{What is Subnetting and supernetting?}

\begin{solutionbox}
Subnetting and supernetting manage IP address allocation efficiently:

\textbf{Comparison:}
\begin{center}
\captionof{table}{Subnetting vs Supernetting}
\begin{tabulary}{\linewidth}{|L|L|L|}
\hline
\textbf{Aspect} & \textbf{Subnetting} & \textbf{Supernetting} \\ \hline
\textbf{Purpose} & Divide large network & Combine small networks \\ \hline
\textbf{Direction} & Top-down approach & Bottom-up approach \\ \hline
\textbf{Result} & Multiple smaller subnets & Single larger network \\ \hline
\end{tabulary}
\end{center}

\textbf{Benefits:}
\begin{itemize}
    \item \keyword{Subnetting}: Better network management, security, reduced broadcast domain
    \item \keyword{Supernetting}: Simplified routing, route aggregation, reduced routing table size
\end{itemize}
\end{solutionbox}

\begin{mnemonicbox}
\mnemonic{Subnetting Splits, Supernetting Sums}
\end{mnemonicbox}

\questionmarks{3(c)}{7}{Explain IPV6 Addressing. Why need of IPV6 migration?}

\begin{solutionbox}
IPv6 addressing uses 128-bit addresses to solve IPv4 limitations:

\textbf{IPv6 Address Structure:}
\begin{center}
\begin{tikzpicture}[node distance=0cm, auto]
    \node [gtu block, minimum width=6cm] (prefix) {Global Routing Prefix (48 bits)};
    \node [gtu block, minimum width=2cm, right=0cm of prefix] (subnet) {Subnet (16)};
    \node [gtu block, minimum width=6cm, right=0cm of subnet] (interface) {Interface Identifier (64 bits)};
\end{tikzpicture}
\captionof{figure}{IPv6 Address Format}
\end{center}

\textbf{Need for IPv6 Migration:}
\begin{center}
\captionof{table}{Migration Drivers}
\begin{tabulary}{\linewidth}{|L|L|L|}
\hline
\textbf{Problem (IPv4)} & \textbf{IPv6 Solution} \\ \hline
Address Exhaustion & 340 undecillion addresses \\ \hline
NAT Complexity & End-to-end connectivity \\ \hline
Security Add-on & Built-in IPSec \\ \hline
Limited Mobile Support & Native mobility \\ \hline
\end{tabulary}
\end{center}

\textbf{Migration Benefits:}
\begin{itemize}
    \item \keyword{Unlimited growth}: Supports IoT expansion
    \item \keyword{Simplified configuration}: Auto-configuration features
    \item \keyword{Better performance}: Optimized header structure
    \item \keyword{Enhanced security}: Mandatory encryption
\end{itemize}
\end{solutionbox}

\begin{mnemonicbox}
\mnemonic{IPv6 Infinite Possibilities, Enhanced Security}
\end{mnemonicbox}

\questionmarks{3(a OR)}{3}{Determine valid IPv4 address from below}

\begin{solutionbox}
\textbf{Analysis:}

\begin{center}
\captionof{table}{IP Address Validation}
\begin{tabulary}{\linewidth}{|L|L|L|L|}
\hline
\textbf{Address} & \textbf{Validity} & \textbf{Class/Reason} & \textbf{Details} \\ \hline
\textbf{192.108.102.101} & Valid & Class C & Network: 192.108.102.0 \\ \hline
\textbf{80.54.256.14} & Invalid & Octet > 255 & Third octet (256) invalid \\ \hline
\end{tabulary}
\end{center}

\textbf{Results:}
\begin{itemize}
    \item \textbf{192.108.102.101}: Valid Class C address.
    \item \textbf{80.54.256.14}: Invalid because 256 exceeds the maximum octet value of 255.
\end{itemize}
\end{solutionbox}

\begin{mnemonicbox}
\mnemonic{Each Octet Maximum 255}
\end{mnemonicbox}

\questionmarks{3(b OR)}{4}{Write Short note on Network Address Translation}

\begin{solutionbox}
NAT translates private IP addresses to public IP addresses for internet access.

\textbf{NAT Types:}
\begin{itemize}
    \item \textbf{Static NAT}: One-to-one mapping (1 private = 1 public)
    \item \textbf{Dynamic NAT}: Pool mapping (First come, first served)
    \item \textbf{PAT/NAPT}: Port translation (Many private = 1 public)
\end{itemize}

\textbf{Benefits:}
\begin{itemize}
    \item \keyword{IP conservation}: Multiple devices share one public IP
    \item \keyword{Security}: Hides internal network structure
    \item \keyword{Flexibility}: Easy internal network changes
\end{itemize}

\textbf{Limitations:}
\begin{itemize}
    \item Breaks end-to-end connectivity model
    \item Adds packet processing overhead
\end{itemize}
\end{solutionbox}

\begin{mnemonicbox}
\mnemonic{NAT Networks Address Translation}
\end{mnemonicbox}

\questionmarks{3(c OR)}{7}{Explain IPV4 Datagram Header in detail}

\begin{solutionbox}
IPv4 header contains essential information for packet routing:

\begin{center}
\begin{tikzpicture}[node distance=0cm, auto]
    \node [gtu block, minimum width=1.5cm] (ver) {Ver};
    \node [gtu block, minimum width=1.5cm, right=0cm of ver] (ihl) {IHL};
    \node [gtu block, minimum width=2cm, right=0cm of ihl] (tos) {Service};
    \node [gtu block, minimum width=5cm, right=0cm of tos] (len) {Total Length};
    
    \node [gtu block, minimum width=5cm, below=0cm of ver.south west, anchor=north west] (id) {Identification};
    \node [gtu block, minimum width=1.5cm, right=0cm of id] (flags) {Flags};
    \node [gtu block, minimum width=3.5cm, right=0cm of flags] (offset) {Fragment Offset};
    
    \node [gtu block, minimum width=2cm, below=0cm of id.south west, anchor=north west] (ttl) {TTL};
    \node [gtu block, minimum width=2cm, right=0cm of ttl] (proto) {Protocol};
    \node [gtu block, minimum width=6cm, right=0cm of proto] (check) {Header Checksum};
    
    \node [gtu block, minimum width=10cm, below=0cm of ttl.south west, anchor=north west] (src) {Source IP Address};
    \node [gtu block, minimum width=10cm, below=0cm of src.south west, anchor=north west] (dst) {Destination IP Address};
\end{tikzpicture}
\captionof{figure}{IPv4 Header Format}
\end{center}

\textbf{Key Fields:}
\begin{center}
\captionof{table}{Header Fields}
\begin{tabulary}{\linewidth}{|L|L|}
\hline
\textbf{Field} & \textbf{Purpose} \\ \hline
\textbf{Version} & IP version (4) \\ \hline
\textbf{IHL} & Header length \\ \hline
\textbf{TTL} & Time To Live (hops) \\ \hline
\textbf{Protocol} & Next layer protocol (TCP/UDP) \\ \hline
\textbf{Source/Dest IP} & Routing addresses \\ \hline
\end{tabulary}
\end{center}

\textbf{Key Functions:}
\begin{itemize}
    \item \keyword{Routing}: Source and destination addresses
    \item \keyword{Fragmentation}: Identification, flags, offset
    \item \keyword{Loop Prevention}: TTL field decrements at each router
\end{itemize}
\end{solutionbox}

\begin{mnemonicbox}
\mnemonic{Header Has Routing Info}
\end{mnemonicbox}

\questionmarks{4(a)}{3}{Explain working of Indirect TCP}

\begin{solutionbox}
Indirect TCP splits TCP connection to handle mobile network challenges:

\begin{center}
\begin{tikzpicture}[node distance=2.5cm, auto]
    \node [gtu block] (fh) {Fixed Host};
    \node [gtu block, right=of fh] (bs) {Base Station};
    \node [gtu block, right=of bs] (mh) {Mobile Host};
    
    \path [gtu arrow] (fh) -- node [above] {TCP Conn 1} (bs);
    \path [gtu arrow] (bs) -- node [above] {TCP Conn 2} (mh);
    
    \node [below=0.2cm of bs, font=\small\itshape] {Acts as Proxy};
\end{tikzpicture}
\captionof{figure}{Indirect TCP}
\end{center}

\textbf{Working Process:}
\begin{itemize}
    \item \keyword{Split Connection}: Connection 1 (Wired) + Connection 2 (Wireless)
    \item \keyword{Proxy}: Base station acts as proxy, buffering packets
    \item \keyword{Handoff}: Base station migrates state during movement
\end{itemize}

\textbf{Advantages:}
\begin{itemize}
    \item Isolates wireless link errors from fixed network
    \item Optimized flow control for each link
\end{itemize}
\end{solutionbox}

\begin{mnemonicbox}
\mnemonic{Indirect TCP Through Proxy}
\end{mnemonicbox}

\questionmarks{4(b)}{4}{Write Short note on Stop and Wait ARQ Protocol}

\begin{solutionbox}
Stop and Wait ARQ ensures reliable data transmission with error detection.

\textbf{Protocol Operation:}
\begin{enumerate}
    \item \keyword{Send}: Transmit frame with sequence number
    \item \keyword{Wait}: Wait for ACK
    \item \keyword{Timeout}: Retransmit if no ACK received
    \item \keyword{ACK}: Receiver confirms delivery
\end{enumerate}

\textbf{Features:}
\begin{itemize}
    \item \keyword{Simplicity}: Easy to implement
    \item \keyword{Reliability}: Guarantees delivery via retransmission
    \item \keyword{Inefficiency}: Channel idle while waiting for ACK
\end{itemize}
\end{solutionbox}

\begin{mnemonicbox}
\mnemonic{Stop Send, Wait ACK, Repeat}
\end{mnemonicbox}

\questionmarks{4(c)}{7}{Explain Communication Middleware in detail}

\begin{solutionbox}
Communication middleware provides abstraction layer between applications and network.

\begin{center}
\begin{tikzpicture}[node distance=1cm, auto]
    \node [gtu block] (app) {Mobile Applications};
    \node [gtu block, below=of app] (middle) {Communication Middleware};
    \node [gtu block, below=of middle] (net) {Network Services};
    
    \node [right=0.5cm of middle, align=left, font=\footnotesize] {
        - Message Routing\\
        - Protocol Conversion\\
        - Buffering\\
        - Synchronization
    };
    
    \draw [gtu arrow] (app) -- (middle);
    \draw [gtu arrow] (middle) -- (net);
\end{tikzpicture}
\captionof{figure}{Middleware Architecture}
\end{center}

\textbf{Middleware Types:}
\begin{center}
\captionof{table}{Middleware Types}
\begin{tabulary}{\linewidth}{|L|L|}
\hline
\textbf{Type} & \textbf{Function} \\ \hline
\textbf{Message-Oriented} & Asynchronous messaging (Queues) \\ \hline
\textbf{RPC-based} & Remote procedure calls (RMI) \\ \hline
\textbf{Event-driven} & Publish-subscribe notifications \\ \hline
\end{tabulary}
\end{center}

\textbf{Mobile-Specific Features:}
\begin{itemize}
    \item \keyword{Location transparency}: Hides mobility details
    \item \keyword{Disconnection handling}: Manages intermittent connectivity
    \item \keyword{Bandwidth adaptation}: Adjusts to varying network quality
\end{itemize}
\end{solutionbox}

\begin{mnemonicbox}
\mnemonic{Middleware Manages Mobile Communication}
\end{mnemonicbox}

\questionmarks{4(a OR)}{3}{Explain Handover management in mobile IP}

\begin{solutionbox}
Handover management maintains connectivity when mobile device moves between networks.

\textbf{Handover Process:}
\begin{enumerate}
    \item \keyword{Detection}: Monitor signal strength
    \item \keyword{Decision}: Select best available network
    \item \keyword{Execution}: Switch to new network
\end{enumerate}

\textbf{Types:}
\begin{itemize}
    \item \keyword{Horizontal}: Same technology (e.g., cell to cell)
    \item \keyword{Vertical}: Different technology (e.g., WiFi to 4G)
    \item \keyword{Hard}: Break-before-make
    \item \keyword{Soft}: Make-before-break
\end{itemize}
\end{solutionbox}

\begin{mnemonicbox}
\mnemonic{Handover Helps Maintain Mobility}
\end{mnemonicbox}

\questionmarks{4(b OR)}{4}{Explain key functions of Communication Gateways}

\begin{solutionbox}
Communication gateways enable interoperability between different systems.

\textbf{Key Functions:}
\begin{center}
\captionof{table}{Gateway Functions}
\begin{tabulary}{\linewidth}{|L|L|}
\hline
\textbf{Function} & \textbf{Benefit} \\ \hline
\textbf{Protocol Translation} & Interoperability between protocols \\ \hline
\textbf{Data Conversion} & Format compatibility \\ \hline
\textbf{Security} & Firewall, authentication \\ \hline
\textbf{Load Balancing} & Performance optimization \\ \hline
\end{tabulary}
\end{center}

\textbf{Services:}
\begin{itemize}
    \item \keyword{Caching}: Store frequently accessed data
    \item \keyword{Compression}: Reduce data size for transmission
    \item \keyword{Traffic Shaping}: Manage bandwidth usage
\end{itemize}
\end{solutionbox}

\begin{mnemonicbox}
\mnemonic{Gateways Grant Protocol Interoperability}
\end{mnemonicbox}

\questionmarks{4(c OR)}{7}{Explain Process of mobile IP}

\begin{solutionbox}
Mobile IP enables global connectivity for moving devices.

\begin{center}
\begin{tikzpicture}[node distance=2cm, auto]
    \node [gtu state] (mn) {Mobile Node};
    \node [gtu block, right=of mn] (fa) {Foreign Agent};
    \node [gtu block, above=of fa] (ha) {Home Agent};
    \node [gtu state, left=of ha] (cn) {Correspondent};

    \draw [gtu arrow] (mn) -- node [below] {1. Discovery} (fa);
    \draw [gtu arrow] (mn) -- node [sloped, above] {2. Register} (ha);
    \draw [gtu arrow] (cn) -- node [above] {3. Data} (ha);
    \draw [gtu arrow, dashed] (ha) -- node [right] {4. Tunnel} (fa);
    \draw [gtu arrow] (fa) -- node [below] {5. Deliver} (mn);
\end{tikzpicture}
\captionof{figure}{Mobile IP Process}
\end{center}

\textbf{Key Phases:}
\begin{enumerate}
    \item \keyword{Agent Discovery}: MN finds Foreign Agent
    \item \keyword{Registration}: MN registers Care-of Address with Home Agent
    \item \keyword{Tunneling}: HA intercepts packets and tunnels to FA
    \item \keyword{Delivery}: FA decapsulates and delivers to MN
\end{enumerate}

\textbf{Components:}
\begin{itemize}
    \item \keyword{Home Agent (HA)}: Router on home network
    \item \keyword{Foreign Agent (FA)}: Router on visited network
    \item \keyword{Care-of Address (CoA)}: Temporary address
\end{itemize}
\end{solutionbox}

\begin{mnemonicbox}
\mnemonic{Mobile IP: Discover Register Tunnel Deliver}
\end{mnemonicbox}

\questionmarks{5(a)}{3}{List advantages of WPANs}

\begin{solutionbox}
WPAN (Wireless Personal Area Network) provides short-range connectivity (e.g., Bluetooth, Zigbee).

\textbf{Advantages:}
\begin{itemize}
    \item \keyword{Low Power}: Extended battery life for devices
    \item \keyword{Low Cost}: Inexpensive implementation
    \item \keyword{Easy Setup}: Automatic discovery and pairing
    \item \keyword{Ad-hoc}: No infrastructure required
\end{itemize}

\textbf{Applications:}
\begin{itemize}
    \item Connecting peripherals (keyboard, mouse)
    \item IoT and smart home integration
    \item Wearable devices (fitness trackers)
\end{itemize}
\end{solutionbox}

\begin{mnemonicbox}
\mnemonic{WPANs: Wireless Personal Area Networks}
\end{mnemonicbox}

\questionmarks{5(b)}{4}{Explain steps of packet delivery in mobile IP}

\begin{solutionbox}
\textbf{Packet Delivery Steps:}

\begin{center}
\captionof{table}{Packet Delivery Flow}
\begin{tabulary}{\linewidth}{|C|L|L|}
\hline
\textbf{Step} & \textbf{location} & \textbf{Action} \\ \hline
1 & Correspondent & Send packet to Home Address \\ \hline
2 & Home Agent & Intercept packet \\ \hline
3 & Tunneling & Encapsulate to Care-of Address \\ \hline
4 & Foreign Agent & Decapsulate packet \\ \hline
5 & Mobile Node & Receive packet \\ \hline
\end{tabulary}
\end{center}

\textbf{Tunneling Mechanism:}
\begin{itemize}
    \item \keyword{Encapsulation}: Original IP packet is wrapped in a new IP packet
    \item \keyword{Outer Header}: Source=HA, Dest=CoA
    \item \keyword{Inner Header}: Source=CN, Dest=Home Address
\end{itemize}
\end{solutionbox}

\begin{mnemonicbox}
\mnemonic{Correspondent Home Foreign Mobile}
\end{mnemonicbox}

\questionmarks{5(c)}{7}{Briefly Explain architecture of WLAN with diagram}

\begin{solutionbox}
WLAN (Wireless Local Area Network) provides local wireless access.

\begin{center}
\begin{tikzpicture}[node distance=1.5cm, auto]
    \node [gtu block, minimum width=4cm] (ds) {Distribution System (DS)};
    \node [gtu block, below left=1cm of ds] (ap1) {Access Point 1};
    \node [gtu block, below right=1cm of ds] (ap2) {Access Point 2};
    
    \node [gtu state, below=0.5cm of ap1] (sta1) {STA 1};
    \node [gtu state, left=0.5cm of sta1] (sta2) {STA 2};
    
    \node [gtu state, below=0.5cm of ap2] (sta3) {STA 3};
    
    \draw [gtu arrow] (ap1) -- (ds);
    \draw [gtu arrow] (ap2) -- (ds);
    \draw [dashed] (sta1) -- (ap1);
    \draw [dashed] (sta2) -- (ap1);
    \draw [dashed] (sta3) -- (ap2);
    
    \node [draw, dashed, fit=(ap1) (sta1) (sta2), label=left:BSS 1] {};
    \node [draw, dashed, fit=(ap2) (sta3), label=right:BSS 2] {};
    \node [draw, fit=(ds) (ap1) (ap2) (sta3), label=above:ESS] {};
\end{tikzpicture}
\captionof{figure}{WLAN Infrastructure Mode}
\end{center}

\textbf{Components:}
\begin{center}
\captionof{table}{WLAN Components}
\begin{tabulary}{\linewidth}{|L|L|}
\hline
\textbf{Component} & \textbf{Function} \\ \hline
\textbf{Station (STA)} & Wireless client device \\ \hline
\textbf{Access Point (AP)} & Wireless base station \\ \hline
\textbf{BSS} & Basic Service Set (AP + Stations) \\ \hline
\textbf{DS} & Wired backbone connecting APs \\ \hline
\textbf{ESS} & Extended Service Set (Multiple BSS) \\ \hline
\end{tabulary}
\end{center}

\textbf{Modes:}
\begin{itemize}
    \item \keyword{Infrastructure}: Uses APs (Home/Office WiFi)
    \item \keyword{Ad-hoc}: Direct device-to-device (IBSS)
\end{itemize}
\end{solutionbox}

\begin{mnemonicbox}
\mnemonic{WLAN: Wireless Local Area Network}
\end{mnemonicbox}

\questionmarks{5(a OR)}{3}{Explain 5G mobile network features in detail}

\begin{solutionbox}
5G is the fifth generation of mobile network technology.

\textbf{Key Features:}
\begin{itemize}
    \item \keyword{Speed}: Up to 10 Gbps (100x faster than 4G)
    \item \keyword{Latency}: < 1ms (Ultra-low latency for realtime control)
    \item \keyword{Density}: Support for 1 million devices/km\textsuperscript{2} (IoT)
\end{itemize}

\textbf{Technologies:}
\begin{itemize}
    \item \keyword{Millimeter Wave}: High frequency for high speed
    \item \keyword{Massive MIMO}: Many antennas for capacity
    \item \keyword{Network Slicing}: Virtual networks for specific needs
\end{itemize}
\end{solutionbox}

\begin{mnemonicbox}
\mnemonic{5G: Fifth Generation Great Speed}
\end{mnemonicbox}

\questionmarks{5(b OR)}{4}{Explain how DHCP works in a mobile network context}

\begin{solutionbox}
DHCP assigns IP addresses. In mobile networks, it must handle movement.

\textbf{DHCP DORA Process:}
\begin{center}
\captionof{table}{DHCP Process}
\begin{tabulary}{\linewidth}{|L|L|}
\hline
\textbf{Message} & \textbf{Description} \\ \hline
\textbf{Discover} & Client looks for server \\ \hline
\textbf{Offer} & Server offers IP \\ \hline
\textbf{Request} & Client requests IP \\ \hline
\textbf{ACK} & Server confirms \\ \hline
\end{tabulary}
\end{center}

\textbf{Mobile Challenges:}
\begin{itemize}
    \item \keyword{Fast Handover}: Need rapid IP assignment when moving
    \item \keyword{Lease Renewal}: Frequent renewal or long leases needed
    \item \keyword{Mobility}: COA assignment in Mobile IP often uses DHCP
\end{itemize}
\end{solutionbox}

\begin{mnemonicbox}
\mnemonic{DHCP: Discover Offer Request ACK}
\end{mnemonicbox}

\questionmarks{5(c OR)}{7}{Explain Bluetooth technology with a neat figure of its protocol stack}

\begin{solutionbox}
Bluetooth is a short-range wireless standard for P2P communication.

\begin{center}
\begin{tikzpicture}[node distance=0cm, outer sep=0pt]
    \node [gtu block, minimum width=6cm] (app) {Applications};
    \node [gtu block, minimum width=6cm, below=0.1cm of app] (l2cap) {L2CAP (Logical Link Control)};
    \node [gtu block, minimum width=6cm, below=0.1cm of l2cap] (hci) {Host Controller Interface (HCI)};
    \node [gtu block, minimum width=6cm, below=0.1cm of hci] (lmp) {Link Manager (LMP)};
    \node [gtu block, minimum width=6cm, below=0.1cm of lmp] (base) {Baseband};
    \node [gtu block, minimum width=6cm, below=0.1cm of base] (radio) {Radio (2.4 GHz)};
    
    \draw [->] (app.south) -- (l2cap.north);
\end{tikzpicture}
\captionof{figure}{Bluetooth Stack}
\end{center}

\textbf{Layer Functions:}
\begin{center}
\captionof{table}{Bluetooth Layers}
\begin{tabulary}{\linewidth}{|L|L|}
\hline
\textbf{Layer} & \textbf{Function} \\ \hline
\textbf{Radio} & Physical transmission (FHSS) \\ \hline
\textbf{Baseband} & Timing, framing, error control \\ \hline
\textbf{LMP} & Connection setup, security, authentication \\ \hline
\textbf{L2CAP} & Multiplexing, segmentation, reassembly \\ \hline
\textbf{Applications} & Profiles (Audio, File Transfer) \\ \hline
\end{tabulary}
\end{center}

\textbf{Features:}
\begin{itemize}
    \item \keyword{Piconet}: Master + up to 7 Slaves
    \item \keyword{Scatternet}: Interconnected Piconets
    \item \keyword{Low Cost/Power}: Designed for portable devices
\end{itemize}
\end{solutionbox}

\begin{mnemonicbox}
\mnemonic{Bluetooth: Radio Baseband LMP HCI L2CAP Applications}
\end{mnemonicbox}

\end{document}
