\documentclass{article}
% Adjust the relative path to point to the latex-templates directory

% content/resources/templates/preamble.tex
\usepackage[margin=0.6in]{geometry}
\author{Milav Dabgar}
\usepackage{amsmath,amssymb,amsthm}
\usepackage{booktabs}
\usepackage{multirow}
\usepackage{xcolor}
\usepackage{tcolorbox}
\tcbuselibrary{breakable,skins}
\usepackage[colorlinks=true,linkcolor=blue]{hyperref}
\usepackage{titlesec}
\usepackage{enumitem}
\usepackage{tikz}
\usepackage{pgfplots}
\usepackage{circuitikz}
\usepackage[version=4]{mhchem}
\usepackage{longtable}
\usepackage{array}
\usepackage{float}
\usepackage{caption}
\usepackage{listings}

\lstset{
  basicstyle=\small\ttfamily,
  breaklines=true,
  breakatwhitespace=false,
  postbreak=\mbox{\textcolor{red}{$\hookrightarrow$}\space},
  float=false,
  numbers=left,
  numberstyle=\tiny\color{gray},
  numbersep=10pt,
  xleftmargin=2em,
  keywordstyle=\color{blue},
  commentstyle=\color{green!60!black},
  stringstyle=\color{purple},
  backgroundcolor=\color{gray!5},
  showstringspaces=false,
  tabsize=2,
  captionpos=b,
  keepspaces=true,
  columns=flexible
}

\pgfplotsset{compat=1.18}
\usetikzlibrary{shapes,arrows,positioning,calc,patterns,decorations.pathmorphing,decorations.markings,arrows.meta}

% Color scheme
\definecolor{headcolor}{RGB}{0,102,204}
\definecolor{keycolor}{RGB}{220,20,60}
\definecolor{solutioncolor}{RGB}{34,139,34}
\definecolor{mnemoniccolor}{RGB}{148,0,211}
\definecolor{codecolor}{RGB}{0,0,100}

% Spacing
\setlength{\parskip}{3pt}
\setlist[itemize]{nosep}
\setlist[enumerate]{nosep}

% Title formatting
\titleformat{\section}{\Large\bfseries\color{headcolor}}{\thesection}{1em}{}
\titleformat{\subsection}{\large\bfseries\color{headcolor}}{\thesubsection}{1em}{}

% Pandoc tightlist compatibility
\providecommand{\tightlist}{%
  \setlength{\itemsep}{0pt}\setlength{\parskip}{0pt}}

% Pandoc longtable compatibility
\newcounter{none}
\def\thenone{}


% content/resources/templates/gujarati-boxes.tex
\usepackage{fontspec}
\usepackage{polyglossia}

% Set Gujarati as main language (document is primarily in Gujarati)
% Note: gloss-gujarati.ldf doesn't exist in polyglossia, but it will use hyphenation patterns
\setdefaultlanguage{gujarati}
\setotherlanguage{english}

% Configure Gujarati font properly
% Use Language=Default to prevent polyglossia from trying to add language-specific features
% that don't exist for Gujarati, which causes "empty feature" warnings
\newfontfamily\gujaratifont[Script=Gujarati,AutoFakeBold=2.5,AutoFakeSlant=0.3]{Noto Sans Gujarati}
\setmainfont[Script=Gujarati,AutoFakeBold=2.5,AutoFakeSlant=0.3]{Noto Sans Gujarati}
% Use Noto Sans Gujarati for monospace to support Gujarati in text
\setmonofont[Scale=0.9]{Noto Sans Gujarati}

% Configure English to use the same font
\newfontfamily\englishfont[Script=Gujarati,AutoFakeBold=2.5,AutoFakeSlant=0.3]{Noto Sans Gujarati}

% Translations for polyglossia
\gappto\captionsgujarati{
  \renewcommand{\tablename}{કોષ્ટક}
  \renewcommand{\figurename}{આકૃતિ}
}

% Helper for TikZ nodes to ensure Gujarati font
\newcommand{\gu}[1]{{\gujaratifont #1}}

% Custom environments
\newtcolorbox{solutionbox}{
    breakable,
    enhanced,
    colback=solutioncolor!5!white,
    colframe=solutioncolor!75!black,
    fonttitle=\bfseries,
    title=જવાબ
}

\newtcolorbox{solutionboxnobreak}{
 colback=solutioncolor!5!white,
 colframe=solutioncolor!75!black,
 fonttitle=\bfseries,
 title=જવાબ
}

\newtcolorbox{keyformula}{
 breakable,
 enhanced,
 colback=keycolor!5!white,
 colframe=keycolor!75!black,
 fonttitle=\bfseries,
 title=રાસાયણિક સમીકરણ/સૂત્ર
}

\newtcolorbox{mnemonicbox}{
 breakable,
 enhanced,
 colback=mnemoniccolor!5!white,
 colframe=mnemoniccolor!75!black,
 fonttitle=\bfseries,
 title=મેમરી ટ્રીક
}


% Custom commands for GTU solutions
% This file defines semantic commands for consistent formatting

% Question command with automatic formatting
\newcommand{\question}[2]{%
  \section*{Question #1}%
  \textbf{#2}%
}

% OR question variant
\newcommand{\questionor}[2]{%
  \section*{Question #1 OR}%
  \textbf{#2}%
}

% Proper table environment with caption
\newenvironment{answertable}[1]{%
  \begin{table}[htbp]
  \centering
  \caption{#1}
}{%
  \end{table}
}

% Proper figure environment for diagrams
\newenvironment{answerdiagram}[1]{%
  \begin{figure}[htbp]
  \centering
  \caption{#1}
}{%
  \end{figure}
}

% Semantic markup for key terms
\newcommand{\keyword}[1]{\textbf{#1}}
\newcommand{\code}[1]{\texttt{#1}}
\newcommand{\classname}[1]{\texttt{#1}}
\newcommand{\methodname}[1]{\texttt{#1}}

% Proper quotation marks
\newcommand{\mnemonic}[1]{``#1''}


\title{Mobile Computing and Networks (4351602) - Winter 2024 Solution}
\date{November 25, 2024}

\begin{document}
\maketitle

\questionmarks{1(અ)}{3}{congestion control ના પ્રકારો જણાવો અને કોઈપણ એક સમજાવો}

\begin{solutionbox}
\textbf{Congestion Control ના પ્રકારો:}
\begin{center}
\captionof{table}{Congestion Control Types}
\begin{tabulary}{\linewidth}{|L|L|}
\hline
\textbf{પ્રકાર} & \textbf{વર્ણન} \\ \hline
\textbf{Open-Loop} & congestion થાય તે પહેલાં અટકાવે \\ \hline
\textbf{Closed-Loop} & congestion detect થયા પછી વ્યવસ્થાપન \\ \hline
\end{tabulary}
\end{center}

\textbf{Open-Loop Congestion Control સમજાવટ:}
\begin{itemize}
    \item \keyword{અટકાવવાનો અભિગમ}: congestion થાય તે પહેલાં action લે
    \item \keyword{Traffic shaping}: sender પર data rate control કરે
    \item \keyword{Admission control}: વધુ traffic દરમિયાન નવા connections limit કરે
    \item \keyword{Load shedding}: buffer full થાય ત્યારે packets drop કરે
\end{itemize}
\end{solutionbox}

\begin{mnemonicbox}
\mnemonic{Open Prevents Traffic Admission Load}
\end{mnemonicbox}

\questionmarks{1(બ)}{4}{Address Resolution Protocol વિસ્તારપૂર્વક સમજાવો}

\begin{solutionbox}
\textbf{ARP (Address Resolution Protocol)} local networks માં IP addresses ને MAC addresses સાથે map કરે છે.

\textbf{કાર્ય પ્રક્રિયા:}
\begin{itemize}
    \item \keyword{ARP Request}: "કોની પાસે IP X છે?" broadcast message
    \item \keyword{ARP Reply}: target device પોતાનું MAC address આપે
    \item \keyword{ARP Cache}: ભવિષ્ય માટે IP-MAC mappings store કરે
    \item \keyword{Dynamic mapping}: entries automatically update કરે
\end{itemize}

\textbf{ARP Message Types:}
\begin{center}
\captionof{table}{ARP Message Types}
\begin{tabulary}{\linewidth}{|L|L|L|}
\hline
\textbf{Type} & \textbf{Purpose} & \textbf{Broadcast} \\ \hline
ARP Request & MAC address શોધવા માટે & Yes \\ \hline
ARP Reply & MAC address આપવા માટે & No \\ \hline
\end{tabulary}
\end{center}
\end{solutionbox}

\begin{mnemonicbox}
\mnemonic{ARP Requests Broadcast, Replies Cache Dynamic}
\end{mnemonicbox}

\questionmarks{1(ક)}{7}{TCP/IP મોડેલના દરેક layers ને તેમની કાર્યક્ષમતા સાથે સમજાવો}

\begin{solutionbox}
\textbf{TCP/IP Model} internet communication માટે four-layer network protocol stack છે.

\begin{center}
\begin{tikzpicture}[node distance=0.8cm, auto]
    \node [gtu block] (app) {Application Layer};
    \node [gtu block, below=of app] (trans) {Transport Layer};
    \node [gtu block, below=of trans] (net) {Internet Layer};
    \node [gtu block, below=of net] (link) {Network Access Layer};

    \draw [gtu arrow] (app) -- (trans);
    \draw [gtu arrow] (trans) -- (net);
    \draw [gtu arrow] (net) -- (link);
\end{tikzpicture}
\captionof{figure}{TCP/IP Model}
\end{center}

\textbf{Layer Functions:}
\begin{center}
\captionof{table}{Layer Functions}
\begin{tabulary}{\linewidth}{|L|L|L|}
\hline
\textbf{Layer} & \textbf{Function} & \textbf{Protocols} \\ \hline
\textbf{Application} & User interface, network services & HTTP, FTP, SMTP \\ \hline
\textbf{Transport} & End-to-end communication & TCP, UDP \\ \hline
\textbf{Internet} & Routing, addressing & IP, ICMP \\ \hline
\textbf{Network Access} & Physical transmission & Ethernet, WiFi \\ \hline
\end{tabulary}
\end{center}

\begin{itemize}
    \item \keyword{Application Layer}: applications ને network services provide કરે
    \item \keyword{Transport Layer}: error control સાથે reliable data delivery ensure કરે
    \item \keyword{Internet Layer}: IP addressing વાપરીને networks વચ્ચે packets route કરે
    \item \keyword{Network Access Layer}: physical data transmission handle કરે
\end{itemize}
\end{solutionbox}

\begin{mnemonicbox}
\mnemonic{All Transport Internet Network}
\end{mnemonicbox}

\questionmarks{1(ક OR)}{7}{OSI model ને તેના દરેક લેયરની કાર્યક્ષમતા સાથે સમજાવો}

\begin{solutionbox}
\textbf{OSI Model} network communication માટે seven-layer reference model છે.

\begin{center}
\begin{tikzpicture}[node distance=0.6cm, auto]
    \node [gtu block] (app) {7. Application Layer};
    \node [gtu block, below=of app] (pres) {6. Presentation Layer};
    \node [gtu block, below=of pres] (sess) {5. Session Layer};
    \node [gtu block, below=of sess] (trans) {4. Transport Layer};
    \node [gtu block, below=of trans] (net) {3. Network Layer};
    \node [gtu block, below=of net] (data) {2. Data Link Layer};
    \node [gtu block, below=of data] (phys) {1. Physical Layer};

    \draw [gtu arrow] (app) -- (pres);
    \draw [gtu arrow] (pres) -- (sess);
    \draw [gtu arrow] (sess) -- (trans);
    \draw [gtu arrow] (trans) -- (net);
    \draw [gtu arrow] (net) -- (data);
    \draw [gtu arrow] (data) -- (phys);
\end{tikzpicture}
\captionof{figure}{OSI Model}
\end{center}

\textbf{Layer Functionalities:}
\begin{center}
\captionof{table}{Layer Functionalities}
\begin{tabulary}{\linewidth}{|L|L|L|}
\hline
\textbf{Layer} & \textbf{Function} & \textbf{Examples} \\ \hline
\textbf{Physical (1)} & Bit transmission & Cables, signals \\ \hline
\textbf{Data Link (2)} & Frame delivery & Ethernet, switches \\ \hline
\textbf{Network (3)} & Routing packets & IP, routers \\ \hline
\textbf{Transport (4)} & End-to-end delivery & TCP, UDP \\ \hline
\textbf{Session (5)} & Dialog management & NetBIOS \\ \hline
\textbf{Presentation (6)} & Data formatting & SSL, compression \\ \hline
\textbf{Application (7)} & User interface & HTTP, email \\ \hline
\end{tabulary}
\end{center}
\end{solutionbox}

\begin{mnemonicbox}
\mnemonic{Physical Data Network Transport Session Presentation Application}
\end{mnemonicbox}

\questionmarks{2(અ)}{3}{Subnetting ને ટૂંકમાં સમજાવો}

\begin{solutionbox}
\textbf{Subnetting} મોટા network ને better management માટે નાના sub-networks માં વહેંચે છે.

\textbf{મુખ્ય સિદ્ધાંતો:}
\begin{itemize}
    \item \keyword{Subnet mask}: network અને host portions define કરે
    \item \keyword{Network efficiency}: broadcast traffic ઘટાડે
    \item \keyword{Address conservation}: વધુ સારો IP utilization
    \item \keyword{Security}: network segments ને isolate કરે
\end{itemize}

\textbf{Example:}
Network: 192.168.1.0/24 $\rightarrow$ Subnets: 192.168.1.0/26, 192.168.1.64/26
\end{solutionbox}

\begin{mnemonicbox}
\mnemonic{Subnet Network Efficiency Address Security}
\end{mnemonicbox}

\questionmarks{2(બ)}{4}{ડેટા લીક લેયરના Stop and wait ARQ પ્રોટોકોલને ઉદાહરણ આપી સમજાવો}

\begin{solutionbox}
\textbf{Stop and Wait ARQ} reliable data transmission ensure કરવા માટેનો flow control protocol છે.

\textbf{કાર્ય પ્રક્રિયા:}
\begin{itemize}
    \item \keyword{Send frame}: Transmitter એક frame મોકલે
    \item \keyword{Wait for ACK}: Sender acknowledgment માટે રાહ જુએ
    \item \keyword{Timeout}: ACK ન મળે તો retransmit કરે
    \item \keyword{Next frame}: ACK મળ્યા પછી next frame મોકલે
\end{itemize}

\begin{center}
\begin{tikzpicture}[node distance=2.5cm, auto]
    \node [gtu state] (s) {Sender};
    \node [gtu state, right=of s] (r) {Receiver};

    \draw [gtu arrow] (s) -- node [above] {Frame 1} (r);
    \draw [gtu arrow] (r) to[bend left=30] node [below] {ACK} (s);
    \draw [gtu arrow] (s) to[bend right=30] node [above] {Frame 2} (r);
\end{tikzpicture}
\captionof{figure}{Stop and Wait ARQ}
\end{center}

\textbf{Example:} File transfer માં દરેક packet confirmation માટે રાહ જુએ before sending next.
\end{solutionbox}

\begin{mnemonicbox}
\mnemonic{Send Wait Timeout Next}
\end{mnemonicbox}

\questionmarks{2(ક)}{7}{IPv4 datagram Header ની આકૃતિ દોરો અને સમજાવો}

\begin{solutionbox}
\textbf{IPv4 Header} packet routing અને delivery માટે control information ધરાવે છે.

\begin{center}
\begin{tikzpicture}[node distance=0cm, outer sep=0pt]
    \node [gtu block, minimum width=1cm] (ver) {Ver};
    \node [gtu block, minimum width=1cm, right=0cm of ver] (ihl) {IHL};
    \node [gtu block, minimum width=2cm, right=0cm of ihl] (tos) {Service};
    \node [gtu block, minimum width=4cm, right=0cm of tos] (len) {Total Length};
    
    \node [gtu block, minimum width=4cm, below=0cm of ver.south west, anchor=north west] (id) {Identification};
    \node [gtu block, minimum width=1cm, right=0cm of id] (flags) {Flg};
    \node [gtu block, minimum width=3cm, right=0cm of flags] (off) {Frag Offset};
    
    \node [gtu block, minimum width=2cm, below=0cm of id.south west, anchor=north west] (ttl) {TTL};
    \node [gtu block, minimum width=2cm, right=0cm of ttl] (proto) {Proto};
    \node [gtu block, minimum width=4cm, right=0cm of proto] (check) {Header Checksum};
    
    \node [gtu block, minimum width=8cm, below=0cm of ttl.south west, anchor=north west] (src) {Source Address};
    \node [gtu block, minimum width=8cm, below=0cm of src.south west, anchor=north west] (dst) {Destination Address};
\end{tikzpicture}
\captionof{figure}{IPv4 Header}
\end{center}

\textbf{Field સમજાવટ:}
\begin{center}
\captionof{table}{Field Explanations}
\begin{tabulary}{\linewidth}{|L|L|L|}
\hline
\textbf{Field} & \textbf{Size} & \textbf{Function} \\ \hline
\textbf{Version} & 4 bits & IP version (IPv4 માટે 4) \\ \hline
\textbf{IHL} & 4 bits & Header length \\ \hline
\textbf{Type of Service} & 8 bits & Quality of service \\ \hline
\textbf{Total Length} & 16 bits & Packet size \\ \hline
\textbf{TTL} & 8 bits & Hop limit \\ \hline
\textbf{Protocol} & 8 bits & Next layer protocol \\ \hline
\textbf{Source/Dest Address} & 32 bits દરેક | IP addresses \\ \hline
\end{tabulary}
\end{center}
\end{solutionbox}

\begin{mnemonicbox}
\mnemonic{Version IHL Service Total TTL Protocol Source Destination}
\end{mnemonicbox}

\questionmarks{2(અ OR)}{3}{HTTPS શું છે? HTTPSની મહત્વની વિશેષતાઓની યાદી લખો}

\begin{solutionbox}
\textbf{HTTPS (HTTP Secure)} secure web communication માટે SSL/TLS વાપરીને encrypted HTTP છે.

\textbf{મુખ્ય વિશેષતાઓ:}
\begin{itemize}
    \item \keyword{Encryption}: Data transit માં encrypted રહે
    \item \keyword{Authentication}: Server identity verify કરે
    \item \keyword{Data integrity}: Data tampering અટકાવે
    \item \keyword{Trust}: SSL certificates validation provide કરે
\end{itemize}

\textbf{Security Benefits:}
\begin{itemize}
    \item Sensitive information protect કરે
    \item Man-in-the-middle attacks prevent કરે
    \item Search engine ranking boost આપે
\end{itemize}
\end{solutionbox}

\begin{mnemonicbox}
\mnemonic{HTTPS Encrypts Authentication Data Trust}
\end{mnemonicbox}

\questionmarks{2(બ OR)}{4}{કોઈપણ બેના જવાબ આપો:}

\begin{solutionbox}
\textbf{1) Class B અને C દ્વારા કેટલા bits HOST ID નો ઉપયોગ થાય છે?}
\begin{itemize}
    \item \textbf{Class B}: HOST ID માટે 16 bits (65,534 hosts)
    \item \textbf{Class C}: HOST ID માટે 8 bits (254 hosts)
\end{itemize}

\textbf{2) Class A અને D ની IP રેન્જ કેટલી છે?}
\begin{itemize}
    \item \textbf{Class A}: 1.0.0.0 to 126.255.255.255
    \item \textbf{Class D}: 224.0.0.0 to 239.255.255.255 (Multicast)
\end{itemize}

\begin{center}
\captionof{table}{IP Classes}
\begin{tabulary}{\linewidth}{|C|L|L|}
\hline
\textbf{Class} & \textbf{Range} & \textbf{Host Bits} \\ \hline
B & 128.0.0.0 - 191.255.255.255 & 16 bits \\ \hline
C & 192.0.0.0 - 223.255.255.255 & 8 bits \\ \hline
A & 1.0.0.0 - 126.255.255.255 & 24 bits \\ \hline
D & 224.0.0.0 - 239.255.255.255 & Multicast \\ \hline
\end{tabulary}
\end{center}
\end{solutionbox}

\begin{mnemonicbox}
\mnemonic{B=16, C=8, A=1-126, D=224-239}
\end{mnemonicbox}

\questionmarks{2(ક OR)}{7}{Classful IPv4 એડ્રેસ સ્કીમ સમજાવો}

\begin{solutionbox}
\textbf{Classful IPv4 Addressing} first octets આધારે IP address space ને પાંચ classes માં વહેંચે છે.

\textbf{Address Classes:}
\begin{center}
\captionof{table}{Address Classes}
\begin{tabulary}{\linewidth}{|L|L|L|L|L|}
\hline
\textbf{Class} & \textbf{Range} & \textbf{Network Bits} & \textbf{Host Bits} & \textbf{Usage} \\ \hline
\textbf{A} & 1-126 & 8 & 24 & Large networks \\ \hline
\textbf{B} & 128-191 & 16 & 16 & Medium networks \\ \hline
\textbf{C} & 192-223 & 24 & 8 & Small networks \\ \hline
\textbf{D} & 224-239 & - & - & Multicast \\ \hline
\textbf{E} & 240-255 & - & - & Experimental \\ \hline
\end{tabulary}
\end{center}

\begin{center}
\begin{tikzpicture}
    \pie[text=legend, radius=2]{50/Class A, 25/Class B, 12.5/Class C, 6.25/Class D, 6.25/Class E}
\end{tikzpicture}
\captionof{figure}{IPv4 Address Classes}
\end{center}

\textbf{લાક્ષણિકતાઓ:}
\begin{itemize}
    \item \keyword{Class A}: network દીઠ 16.7 million hosts
    \item \keyword{Class B}: network દીઠ 65,534 hosts
    \item \keyword{Class C}: network દીઠ 254 hosts
    \item \keyword{મર્યાદાઓ}: Address wastage, inflexible allocation
\end{itemize}
\end{solutionbox}

\begin{mnemonicbox}
\mnemonic{A-Large, B-Medium, C-Small, D-Multicast, E-Experimental}
\end{mnemonicbox}

\questionmarks{3(અ)}{3}{મોબાઇલ કમ્પ્યુટિંગનો ઉપયોગ કરતી એપ્લિકેશનોના પ્રકારોની યાદી બનાવો}

\begin{solutionbox}
\textbf{Mobile Computing Applications:}
\begin{center}
\captionof{table}{Applications}
\begin{tabulary}{\linewidth}{|L|L|}
\hline
\textbf{પ્રકાર} & \textbf{Examples} \\ \hline
\textbf{Communication} & WhatsApp, Email, Video calls \\ \hline
\textbf{Navigation} & GPS, Google Maps \\ \hline
\textbf{E-commerce} & Shopping apps, Mobile banking \\ \hline
\textbf{Entertainment} & Games, Streaming, Social media \\ \hline
\textbf{Business} & CRM, Sales tracking \\ \hline
\textbf{Healthcare} & Health monitoring, Telemedicine \\ \hline
\end{tabulary}
\end{center}

\begin{itemize}
    \item \keyword{Location-based services}: GPS navigation, location sharing
    \item \keyword{Mobile payments}: Digital wallets, UPI transactions
    \item \keyword{Social networking}: Facebook, Instagram, Twitter
\end{itemize}
\end{solutionbox}

\begin{mnemonicbox}
\mnemonic{Communication Navigation E-commerce Entertainment Business Healthcare}
\end{mnemonicbox}

\questionmarks{3(બ)}{4}{Gateways નો ઉપયોગ સમજાવો અને Gateways ના પ્રકારોની યાદી આપો}

\begin{solutionbox}
\textbf{Gateway} અલગ અલગ protocols અને architectures વાળા networks ને connect કરે છે.

\textbf{Gateways ના ઉપયોગ:}
\begin{itemize}
    \item \keyword{Protocol conversion}: વિવિધ protocols વચ્ચે translate કરે
    \item \keyword{Network bridging}: અસમાન networks ને connect કરે
    \item \keyword{Security}: Firewall અને access control
    \item \keyword{Data filtering}: Traffic flow manage કરે
\end{itemize}

\textbf{Gateways ના પ્રકારો:}
\begin{center}
\captionof{table}{Types of Gateways}
\begin{tabulary}{\linewidth}{|L|L|}
\hline
\textbf{Type} & \textbf{Function} \\ \hline
\textbf{Network Gateway} & Networks વચ્ચે route કરે \\ \hline
\textbf{Internet Gateway} & Internet સાથે connect કરે \\ \hline
\textbf{Protocol Gateway} & Protocol translation \\ \hline
\textbf{Application Gateway} & Application-level filtering \\ \hline
\end{tabulary}
\end{center}
\end{solutionbox}

\begin{mnemonicbox}
\mnemonic{Gateways Convert Bridge Secure Filter}
\end{mnemonicbox}

\questionmarks{3(ક)}{7}{Mobile Computing નું આર્કિટેક્ચર દોરો અને સમજાવો}

\begin{solutionbox}
\textbf{Mobile Computing Architecture} એકસાથે કામ કરતા ત્રણ મુખ્ય components ધરાવે છે.

\begin{center}
\begin{tikzpicture}[node distance=1.5cm, auto]
    \node [gtu state, align=center] (mobile) {Mobile Device\\(Hardware, OS, Data)};
    \node [gtu block, right=of mobile, align=center] (network) {Communication Network\\(Wireless, Base Stations)};
    \node [gtu block, right=of network, align=center] (fixed) {Fixed Infrastructure\\(Servers, Internet)};

    \draw [gtu arrow] (mobile) -- (network);
    \draw [gtu arrow] (network) -- (mobile);
    \draw [gtu arrow] (network) -- (fixed);
    \draw [gtu arrow] (fixed) -- (network);
\end{tikzpicture}
\captionof{figure}{Mobile Computing Architecture}
\end{center}

\textbf{Architecture Components:}
\begin{center}
\captionof{table}{Components}
\begin{tabulary}{\linewidth}{|L|L|L|}
\hline
\textbf{Component} & \textbf{Elements} & \textbf{Function} \\ \hline
\textbf{Mobile Unit} & Devices, OS, Apps | User interface, processing \\ \hline
\textbf{Communication Network} & Wireless links, protocols & Data transmission \\ \hline
\textbf{Fixed Infrastructure} & Servers, databases & Backend services \\ \hline
\end{tabulary}
\end{center}

\textbf{મુખ્ય લાક્ષણિકતાઓ:}
\begin{itemize}
    \item \keyword{Mobility}: Users connectivity maintain કરીને move કરી શકે
    \item \keyword{Wireless communication}: Data transmission માટે radio waves
    \item \keyword{Distributed computing}: Multiple devices પર processing
    \item \keyword{Location independence}: કોઈપણ જગ્યાએથી services access
\end{itemize}

\textbf{પડકારો:}
\begin{itemize}
    \item \keyword{Limited bandwidth}: Wireless networks માં capacity constraints
    \item \keyword{Battery life}: Mobile devices માં power limitations
    \item \keyword{Security}: Wireless transmission attacks માટે vulnerable
\end{itemize}
\end{solutionbox}

\begin{mnemonicbox}
\mnemonic{Mobile Communication Fixed - Mobility Wireless Distributed Location}
\end{mnemonicbox}

\questionmarks{3(અ OR)}{3}{મોબાઇલ કમ્પ્યુટિંગમાં સુરક્ષા ધોરણોની યાદી બનાવો}

\begin{solutionbox}
\textbf{Mobile Computing Security Standards:}

\begin{center}
\captionof{table}{Security Standards}
\begin{tabulary}{\linewidth}{|L|L|}
\hline
\textbf{Standard} & \textbf{Purpose} \\ \hline
\textbf{WPA3} & WiFi security protocol \\ \hline
\textbf{SSL/TLS} & Secure data transmission \\ \hline
\textbf{IPSec} & IP layer security \\ \hline
\textbf{EAP} & Authentication framework \\ \hline
\textbf{802.11i} & Wireless LAN security \\ \hline
\textbf{FIPS 140-2} & Cryptographic module standards \\ \hline
\end{tabulary}
\end{center}

\begin{itemize}
    \item \keyword{Authentication protocols}: User identity verify કરે
    \item \keyword{Encryption standards}: Data confidentiality protect કરે
    \item \keyword{Access control}: Resource permissions manage કરે
\end{itemize}
\end{solutionbox}

\begin{mnemonicbox}
\mnemonic{WPA SSL IPSec EAP 802.11i FIPS}
\end{mnemonicbox}

\questionmarks{3(બ OR)}{4}{કોમ્યુનિકેશન Gateway ના મુખ્ય કાર્યો સમજાવો}

\begin{solutionbox}
\textbf{Communication Gateway} વિવિધ network systems વચ્ચે data exchange manage કરે છે.

\textbf{મુખ્ય કાર્યો:}
\begin{center}
\captionof{table}{Functions}
\begin{tabulary}{\linewidth}{|L|L|}
\hline
\textbf{Function} & \textbf{Description} \\ \hline
\textbf{Protocol Translation} & Protocols વચ્ચે convert કરે \\ \hline
\textbf{Data Format Conversion} & Data formats change કરે \\ \hline
\textbf{Routing} & Messages ને destinations પર direct કરે \\ \hline
\textbf{Security} & Access control અને filtering \\ \hline
\end{tabulary}
\end{center}

\textbf{વિગતવાર કાર્યો:}
\begin{itemize}
    \item \keyword{Message routing}: Data માટે optimal path determine કરે
    \item \keyword{Error handling}: Transmission errors અને recovery manage કરે
    \item \keyword{Traffic management}: Data flow અને congestion control કરે
    \item \keyword{Authentication}: Sender અને receiver identity verify કરે
\end{itemize}

\textbf{ફાયદાઓ:}
\begin{itemize}
    \item વિવિધ systems વચ્ચે interoperability enable કરે
    \item Network management centralize કરે
    \item Security checkpoint provide કરે
\end{itemize}
\end{solutionbox}

\begin{mnemonicbox}
\mnemonic{Protocol Data Routing Security - Message Error Traffic Authentication}
\end{mnemonicbox}

\questionmarks{3(ક OR)}{7}{મિડલવેરનો ઉપયોગ અને મિડલવેરના લિસ્ટ પ્રકારો સમજાવો}

\begin{solutionbox}
\textbf{Middleware} distributed computing માટે applications અને operating system વચ્ચે software layer provide કરે છે.

\textbf{Middleware ના ઉપયોગ:}
\begin{itemize}
    \item \keyword{Connectivity}: Distributed applications ને link કરે
    \item \keyword{Interoperability}: વિવિધ systems ને એકસાથે કામ કરવા enable કરે
    \item \keyword{Abstraction}: Underlying systems ની complexity hide કરે
    \item \keyword{Scalability}: System growth અને expansion support કરે
\end{itemize}

\begin{center}
\begin{tikzpicture}[node distance=1cm, auto]
    \node [gtu block, minimum width=4cm] (apps) {Applications};
    \node [gtu block, minimum width=4cm, below=of apps] (mid) {Middleware Layer};
    \node [gtu block, below left=of mid] (os) {OS};
    \node [gtu block, below=of mid] (net) {Network};
    \node [gtu block, below right=of mid] (db) {Database};

    \draw [gtu arrow] (apps) -- (mid);
    \draw [gtu arrow] (mid) -- (os);
    \draw [gtu arrow] (mid) -- (net);
    \draw [gtu arrow] (mid) -- (db);
\end{tikzpicture}
\captionof{figure}{Middleware Layer}
\end{center}

\textbf{Middleware ના પ્રકારો:}
\begin{center}
\captionof{table}{Types of Middleware}
\begin{tabulary}{\linewidth}{|L|L|L|}
\hline
\textbf{Type} & \textbf{Function} & \textbf{Examples} \\ \hline
\textbf{Message-Oriented} & Asynchronous communication & IBM MQ, RabbitMQ \\ \hline
\textbf{Remote Procedure Call} & Synchronous communication & gRPC, XML-RPC \\ \hline
\textbf{Object Request Broker} & Object communication & CORBA \\ \hline
\textbf{Database Middleware} & Database connectivity & ODBC, JDBC \\ \hline
\textbf{Transaction Processing} & Transaction management & Tuxedo \\ \hline
\textbf{Web Middleware} & Web services & Apache, IIS \\ \hline
\end{tabulary}
\end{center}

\textbf{ફાયદાઓ:}
\begin{itemize}
    \item \keyword{Reduced complexity}: Application development simplify કરે
    \item \keyword{Reusability}: Multiple applications માટે common services
    \item \keyword{Maintainability}: Services ના centralized management
    \item \keyword{Platform independence}: વિવિધ systems પર કામ કરે
\end{itemize}
\end{solutionbox}

\begin{mnemonicbox}
\mnemonic{Message RPC Object Database Transaction Web}
\end{mnemonicbox}

\questionmarks{4(અ)}{3}{મોબાઇલ IP ના કાર્યકારી તબક્કાઓ સમજાવો}

\begin{solutionbox}
\textbf{Mobile IP Working Phases} networks પર seamless mobility enable કરે છે.

\textbf{ત્રણ મુખ્ય તબક્કાઓ:}
\begin{center}
\captionof{table}{Phases}
\begin{tabulary}{\linewidth}{|L|L|}
\hline
\textbf{Phase} & \textbf{Function} \\ \hline
\textbf{Agent Discovery} & Home/foreign agents શોધવા \\ \hline
\textbf{Registration} & Foreign agent સાથે register \\ \hline
\textbf{Tunneling} & Mobile node પર packets forward \\ \hline
\end{tabulary}
\end{center}

\textbf{Phase વિગતો:}
\begin{itemize}
    \item \keyword{Agent Discovery}: Mobile node advertisements દ્વારા available agents detect કરે
    \item \keyword{Registration}: Mobile node current location home agent સાથે register કરે
    \item \keyword{Tunneling}: Home agent packets encapsulate કરીને foreign agent પર forward કરે
\end{itemize}
\end{solutionbox}

\begin{mnemonicbox}
\mnemonic{Agent Registration Tunneling}
\end{mnemonicbox}

\questionmarks{4(બ)}{4}{Mobile IP માટે હેન્ડઓવર મેનેજમેન્ટ સમજાવો}

\begin{solutionbox}
\textbf{Handover Management} mobile node networks વચ્ચે move કરે ત્યારે connectivity maintain કરે છે.

\textbf{Handover Process:}
\begin{itemize}
    \item \keyword{Movement detection}: Network attachment માં ફેરફાર identify કરે
    \item \keyword{New agent discovery}: નવા foreign agent શોધે
    \item \keyword{Registration update}: Home agent સાથે location update કરે
    \item \keyword{Data forwarding}: Traffic ને નવા location પર redirect કરે
\end{itemize}

\textbf{Handover ના પ્રકારો:}
\begin{center}
\captionof{table}{Types of Handover}
\begin{tabulary}{\linewidth}{|L|L|}
\hline
\textbf{Type} & \textbf{Description} \\ \hline
\textbf{Hard Handover} & Break-before-make \\ \hline
\textbf{Soft Handover} & Make-before-break \\ \hline
\textbf{Horizontal} & Same technology \\ \hline
\textbf{Vertical} & Different technology \\ \hline
\end{tabulary}
\end{center}

\textbf{પડકારો:}
\begin{itemize}
    \item \keyword{Packet loss}: Handover transition દરમિયાન
    \item \keyword{Delay}: Registration અને tunneling setup time
    \item \keyword{Resource management}: Network resources નો efficient use
\end{itemize}
\end{solutionbox}

\begin{mnemonicbox}
\mnemonic{Movement Discovery Registration Forwarding}
\end{mnemonicbox}

\questionmarks{4(ક)}{7}{Mobile IP માં Registration અને Tunneling સમજાવો}

\begin{solutionbox}
\textbf{Registration અને Tunneling} Mobile IP functionality enable કરવાના core mechanisms છે.

\textbf{Registration Process:}
\begin{center}
\begin{tikzpicture}[node distance=2.5cm, auto]
    \node [gtu state] (mn) {Mobile Node};
    \node [gtu block, right=of mn] (fa) {Foreign Agent};
    \node [gtu block, right=of fa] (ha) {Home Agent};

    \draw [gtu arrow] (mn) -- node [above, font=\small] {Request} (fa);
    \draw [gtu arrow] (fa) -- node [above, font=\small] {Forward} (ha);
    \draw [gtu arrow] (ha) to[bend left=30] node [below, font=\small] {Reply} (fa);
    \draw [gtu arrow] (fa) to[bend left=30] node [below, font=\small] {Forward} (mn);
\end{tikzpicture}
\captionof{figure}{Registration Process}
\end{center}

\textbf{Registration Steps:}
\begin{itemize}
    \item \keyword{Request}: Mobile node foreign agent ને registration request મોકલે
    \item \keyword{Forward}: Foreign agent request ને home agent પર forward કરે
    \item \keyword{Authentication}: Home agent mobile node identity verify કરે
    \item \keyword{Reply}: Home agent registration confirm કરતો reply મોકલે
\end{itemize}

\textbf{Tunneling Mechanism:}
\begin{itemize}
    \item \keyword{Encapsulation}: Original packet ને wrap કરે
    \item \keyword{Tunnel Endpoint}: Home અને foreign agents
    \item \keyword{Decapsulation}: Destination પર packet unwrap કરે
    \item \keyword{Routing}: Tunnel દ્વારા traffic direct કરે
\end{itemize}
\end{solutionbox}

\begin{mnemonicbox}
\mnemonic{Registration Request Forward Authentication - Tunneling Encapsulation Transmission Decapsulation}
\end{mnemonicbox}

\questionmarks{4(અ OR)}{3}{Snooping TCP સમજાવો}

\begin{solutionbox}
\textbf{Snooping TCP} wireless networks પર wireless link errors handle કરીને TCP performance improve કરે છે.

\textbf{કાર્ય પ્રક્રિયા:}
\begin{itemize}
    \item \keyword{Base station monitoring}: TCP packets observe કરે
    \item \keyword{Local retransmission}: Wireless link errors locally handle કરે
    \item \keyword{Cache management}: Transmitted packets ની copies store કરે
    \item \keyword{Error recovery}: Sender involve કર્યા વિના lost packets retransmit કરે
\end{itemize}

\textbf{મુખ્ય લાક્ષણિકતાઓ:}
\begin{center}
\captionof{table}{Features}
\begin{tabulary}{\linewidth}{|L|L|}
\hline
\textbf{Feature} & \textbf{Benefit} \\ \hline
\textbf{Transparent} & TCP endpoints માં કોઈ changes નથી \\ \hline
\textbf{Local recovery} & Faster error correction \\ \hline
\textbf{Reduced timeouts} & Unnecessary retransmissions prevent કરે \\ \hline
\end{tabulary}
\end{center}
\end{solutionbox}

\begin{mnemonicbox}
\mnemonic{Snooping Monitors Local Cache Recovery}
\end{mnemonicbox}

\questionmarks{4(બ OR)}{4}{Mobile IP મા પેકેટ ડિલિવરી સમજાવો}

\begin{solutionbox}
\textbf{Mobile IP માં Packet Delivery} location ને ધ્યાન આપ્યા વિના mobile nodes પર data પહોંચાડે છે.

\begin{center}
\begin{tikzpicture}[node distance=2cm, auto]
    \node [gtu state] (cn) {CN};
    \node [gtu block, right=of cn] (ha) {Home Agent};
    \node [gtu block, right=of ha] (fa) {Foreign Agent};
    \node [gtu state, below=of fa] (mn) {MN};

    \draw [gtu arrow] (cn) -- node [above] {1. Data} (ha);
    \draw [gtu arrow] (ha) -- node [above] {2. Tunnel} (fa);
    \draw [gtu arrow] (fa) -- node [right] {3. Deliver} (mn);
    \draw [gtu arrow] (mn) to[bend left=45] node [below left] {4. Direct Reply} (cn);
\end{tikzpicture}
\captionof{figure}{Packet Delivery}
\end{center}

\textbf{Delivery Scenarios:}
\begin{center}
\captionof{table}{Delivery Scenarios}
\begin{tabulary}{\linewidth}{|L|L|L|}
\hline
\textbf{Scenario} & \textbf{Path} & \textbf{Method} \\ \hline
\textbf{At Home} & Direct | Normal IP routing \\ \hline
\textbf{Away} & Via HA/FA | Tunneling \\ \hline
\textbf{Roaming} & Triangle routing | Indirect path \\ \hline
\end{tabulary}
\end{center}

\textbf{Packet Flow Steps:}
\begin{itemize}
    \item \keyword{Address resolution}: Mobile node location determine કરે
    \item \keyword{Route selection}: Direct અથવા tunneled delivery choose કરે
    \item \keyword{Encapsulation}: Tunneling જરૂરી હોય તો packet wrap કરે
    \item \keyword{Forwarding}: Appropriate destination પર send કરે
    \item \keyword{Decapsulation}: Foreign agent પર packet unwrap કરે
\end{itemize}
\end{solutionbox}

\begin{mnemonicbox}
\mnemonic{Address Route Encapsulation Forward Decapsulation Delivery}
\end{mnemonicbox}

\questionmarks{4(ક OR)}{7}{DHCP કેવી રીતે કાર્ય કરે છે એ આકૃતિ દોરી સમજાવો}

\begin{solutionbox}
\textbf{DHCP (Dynamic Host Configuration Protocol)} devices ને automatically IP addresses અને network configuration assign કરે છે.

\textbf{DHCP Working Process:}
\begin{center}
\begin{tikzpicture}[node distance=3cm, auto]
    \node [gtu state] (client) {Client};
    \node [gtu block, right=of client] (server) {DHCP Server};

    \draw [gtu arrow] (client) -- node [above] {1. DISCOVER} (server);
    \draw [gtu arrow] (server) to[bend left=20] node [below] {2. OFFER} (client);
    \draw [gtu arrow] (client) to[bend left=20] node [above] {3. REQUEST} (server);
    \draw [gtu arrow] (server) to[bend left=40] node [below] {4. ACK} (client);
\end{tikzpicture}
\captionof{figure}{DHCP Process}
\end{center}

\textbf{ચાર-પગલાની પ્રક્રિયા:}
\begin{center}
\captionof{table}{Process Steps}
\begin{tabulary}{\linewidth}{|C|L|L|}
\hline
\textbf{Step} & \textbf{Message} & \textbf{Function} \\ \hline
\textbf{1} & DISCOVER & Client IP માટે broadcast request કરે \\ \hline
\textbf{2} & OFFER & Server available IP address offer કરે \\ \hline
\textbf{3} & REQUEST & Client specific IP address request કરે \\ \hline
\textbf{4} & ACK & Server IP assignment confirm કરે \\ \hline
\end{tabulary}
\end{center}

\textbf{Configuration Information Provided:}
\begin{itemize}
    \item \keyword{IP Address}: Unique network identifier
    \item \keyword{Subnet Mask}: Network boundary definition
    \item \keyword{Default Gateway}: Other networks નો route
    \item \keyword{DNS Servers}: Domain name resolution
    \item \keyword{Lease Time}: IP assignment નો duration
\end{itemize}

\textbf{ફાયદાઓ:}
\begin{itemize}
    \item \keyword{Automatic configuration}: Manual IP assignment ની જરૂર નથી
    \item \keyword{Centralized management}: Network configuration માટે single point
    \item \keyword{Efficient utilization}: Dynamic allocation waste prevent કરે
\end{itemize}
\end{solutionbox}

\begin{mnemonicbox}
\mnemonic{Discover Offer Request ACK - Server Client Relay Pool}
\end{mnemonicbox}

\questionmarks{5(અ)}{3}{WLAN ના પ્રકાર જણાવો અને કોઈપણ એક સમજાવો}

\begin{solutionbox}
\textbf{WLAN પ્રકારો:}
\begin{center}
\captionof{table}{WLAN Types}
\begin{tabulary}{\linewidth}{|L|L|L|}
\hline
\textbf{Type} & \textbf{Standard} & \textbf{Frequency} \\ \hline
\textbf{Infrastructure} & 802.11 & 2.4/5 GHz \\ \hline
\textbf{Ad-hoc} & IBSS & 2.4/5 GHz \\ \hline
\textbf{Mesh} & 802.11s & Multiple \\ \hline
\end{tabulary}
\end{center}

\textbf{Infrastructure WLAN સમજાવટ:}
\begin{itemize}
    \item \keyword{Access Point (AP)}: બધા communications માટે central coordinator
    \item \keyword{BSS (Basic Service Set)}: Single AP નો network coverage area
    \item \keyword{ESS (Extended Service Set)}: Multiple interconnected BSSs
    \item \keyword{Distribution System}: Multiple APs ને connect કરતું backbone
\end{itemize}

\textbf{લાક્ષણિકતાઓ:}
\begin{itemize}
    \item બધા communication access point દ્વારા જાય છે
    \item Centralized network management
    \item વધુ સારું security અને performance control
\end{itemize}
\end{solutionbox}

\begin{mnemonicbox}
\mnemonic{Infrastructure Ad-hoc Mesh - AP BSS ESS Distribution}
\end{mnemonicbox}

\questionmarks{5(બ)}{4}{નીચેના પ્રશ્નોના જવાબ આપો:}

\begin{solutionbox}
\textbf{1) Ad hoc Network ના ઉપયોગોની યાદી આપો:}
\begin{center}
\captionof{table}{Uses}
\begin{tabulary}{\linewidth}{|L|L|}
\hline
\textbf{Use Case} & \textbf{Application} \\ \hline
\textbf{Emergency} & Disaster recovery, rescue operations \\ \hline
\textbf{Military} & Battlefield communications \\ \hline
\textbf{Conferences} & Temporary meeting networks \\ \hline
\textbf{Home} & Device-to-device communication \\ \hline
\textbf{Vehicular} & Car-to-car networks \\ \hline
\end{tabulary}
\end{center}

\textbf{2) મોબાઇલ કમ્પ્યુટિંગની Entities અને Terminology ની યાદી લખો:}
\begin{itemize}
    \item \textbf{Entities}: Mobile Node (MN), Home Agent (HA), Foreign Agent (FA), Correspondent Node (CN)
    \item \textbf{Terminology}: Handover, Roaming, Care-of Address
\end{itemize}
\end{solutionbox}

\begin{mnemonicbox}
\mnemonic{Emergency Military Conference Home Vehicular - MN HA FA CN}
\end{mnemonicbox}

\questionmarks{5(ક)}{7}{સ્વચ્છ આકૃતિ સાથે WLAN ના આર્કિટેક્ચરને સમજાવો}

\begin{solutionbox}
\textbf{WLAN Architecture} access points દ્વારા communicate કરતા wireless stations ધરાવે છે.

\begin{center}
\begin{tikzpicture}[node distance=1cm, auto]
    \node [gtu block] (ds) {Distribution System (DS)};
    \node [gtu block, below left=of ds] (ap1) {AP 1};
    \node [gtu block, below right=of ds] (ap2) {AP 2};
    
    \node [gtu state, below=of ap1] (sta1) {Device 1};
    \node [gtu state, below=of ap2] (sta2) {Device 2};

    \draw [gtu arrow] (ap1) -- (ds);
    \draw [gtu arrow] (ap2) -- (ds);
    \draw [dashed] (sta1) -- (ap1);
    \draw [dashed] (sta2) -- (ap2);
    
    \node [draw, dashed, fit=(ap1) (sta1), label=left:BSS 1] {};
    \node [draw, dashed, fit=(ap2) (sta2), label=right:BSS 2] {};
    \node [draw, fit=(ds) (ap1) (ap2), label=above:ESS] {};
\end{tikzpicture}
\captionof{figure}{WLAN Architecture}
\end{center}

\textbf{Architecture Components:}
\begin{center}
\captionof{table}{Components}
\begin{tabulary}{\linewidth}{|L|L|L|}
\hline
\textbf{Component} & \textbf{Function} & \textbf{Coverage} \\ \hline
\textbf{STA (Station)} & Wireless device | Point \\ \hline
\textbf{AP (Access Point)} & Network coordinator | BSS area \\ \hline
\textbf{BSS (Basic Service Set)} & Single AP coverage | ~100m radius \\ \hline
\textbf{ESS (Extended Service Set)} & Multiple connected BSS | Large area \\ \hline
\textbf{DS (Distribution System)} & AP interconnection | Building/campus \\ \hline
\end{tabulary}
\end{center}

\textbf{WLAN Architecture ના પ્રકારો:}
\begin{itemize}
    \item \keyword{Infrastructure Mode}: Centralized, Managed, Scalable
    \item \keyword{Ad-hoc Mode (IBSS)}: Peer-to-peer, Decentralized, Temporary
\end{itemize}
\end{solutionbox}

\begin{mnemonicbox}
\mnemonic{STA AP BSS ESS DS - Infrastructure Ad-hoc}
\end{mnemonicbox}

\questionmarks{5(અ OR)}{3}{5G ની લાક્ષણિકતાઓ લખો}

\begin{solutionbox}
\textbf{5G મુખ્ય લાક્ષણિકતાઓ:}
\begin{center}
\captionof{table}{5G Features}
\begin{tabulary}{\linewidth}{|L|L|}
\hline
\textbf{Feature} & \textbf{Specification} \\ \hline
\textbf{Speed} & Up to 10 Gbps સુધી \\ \hline
\textbf{Latency} & < 1 millisecond \\ \hline
\textbf{Connectivity} & 1 million devices/km\textsuperscript{2} \\ \hline
\textbf{Reliability} & 99.999\% availability \\ \hline
\textbf{Bandwidth} & 100x વધારો \\ \hline
\textbf{Energy} & 90\% ઘટાડો \\ \hline
\end{tabulary}
\end{center}

\textbf{Advanced Capabilities:}
\begin{itemize}
    \item \keyword{Enhanced Mobile Broadband (eMBB)}: Ultra-fast data speeds
    \item \keyword{Ultra-Reliable Low Latency (URLLC)}: Mission-critical applications
    \item \keyword{Massive Machine Type Communication (mMTC)}: IoT connectivity
\end{itemize}
\end{solutionbox}

\begin{mnemonicbox}
\mnemonic{Speed Latency Connectivity Reliability Bandwidth Energy}
\end{mnemonicbox}

\questionmarks{5(બ OR)}{4}{નીચેના પ્રશ્નોના જવાબ આપો:}

\begin{solutionbox}
\textbf{1) communication middleware ની પ્રકારોની યાદી લખો:}
\begin{itemize}
    \item \keyword{Message-Oriented}: Asynchronous messaging
    \item \keyword{RPC-based}: Remote procedure calls
    \item \keyword{Object-Oriented}: Distributed objects
    \item \keyword{Service-Oriented}: Web services
    \item \keyword{Database}: Data access layer
\end{itemize}

\textbf{2) Mobile IP ના સંદર્ભમાં "Home Agent" ની વ્યાખ્યા આપો:}
\textbf{Home Agent (HA)} mobile node ના home network પરનો router છે.
\textbf{Functions:}
\begin{itemize}
    \item \keyword{Registration maintain કરે}: Mobile node નું current location track કરે
    \item \keyword{Packets tunnel કરે}: Mobile node ના foreign location પર data forward કરે
    \item \keyword{Address management}: Mobile node નું permanent IP address manage કરે
    \item \keyword{Authentication}: Registration દરમિયાન mobile node identity verify કરે
\end{itemize}
\end{solutionbox}

\begin{mnemonicbox}
\mnemonic{Message RPC Object Service Database - HA Maintains Tunnels Address Authentication}
\end{mnemonicbox}

\questionmarks{5(ક OR)}{7}{Bluetooth protocol stack આકૃતિ સાથે સમજાવો}

\begin{solutionbox}
\textbf{Bluetooth Protocol Stack} short-range wireless communication માટે layered architecture provide કરે છે.

\begin{center}
\begin{tikzpicture}[node distance=0.1cm, auto]
    \node [gtu block, minimum width=5cm] (app) {Applications};
    \node [gtu block, minimum width=5cm, below=of app] (mid) {OBEX / SDP / TCS};
    \node [gtu block, minimum width=5cm, below=of mid] (trans) {RFCOMM};
    \node [gtu block, minimum width=5cm, below=of trans] (net) {L2CAP};
    \node [gtu block, minimum width=5cm, below=of net] (hci) {Host Controller Interface (HCI)};
    \node [gtu block, minimum width=5cm, below=of hci] (lmp) {Link Manager (LMP)};
    \node [gtu block, minimum width=5cm, below=of lmp] (base) {Baseband};
    \node [gtu block, minimum width=5cm, below=of base] (radio) {Radio Layer};
\end{tikzpicture}
\captionof{figure}{Bluetooth Stack}
\end{center}

\textbf{Protocol Stack Layers:}
\begin{center}
\captionof{table}{Layers}
\begin{tabulary}{\linewidth}{|L|L|L|}
\hline
\textbf{Layer} & \textbf{Function} & \textbf{Protocols} \\ \hline
\textbf{Application} & User applications & Audio, File transfer \\ \hline
\textbf{Middleware} & Services & OBEX, SDP, TCS \\ \hline
\textbf{Transport} & Data delivery & RFCOMM \\ \hline
\textbf{Network} & Packet management & L2CAP \\ \hline
\textbf{Interface} & Host-Controller & HCI \\ \hline
\textbf{Management} & Link control & LMP \\ \hline
\textbf{Data Link} & Channel access & Baseband \\ \hline
\textbf{Physical} & Radio transmission & 2.4 GHz ISM \\ \hline
\end{tabulary}
\end{center}

\textbf{મુખ્ય લાક્ષણિકતાઓ:}
\begin{itemize}
    \item Frequency Hopping, Piconet, Scatternet, Power Classes
\end{itemize}
\end{solutionbox}

\begin{mnemonicbox}
\mnemonic{Application Middleware Transport Network Interface Management DataLink Physical}
\end{mnemonicbox}

\end{document}
