\documentclass{article}
% Adjust the relative path to point to the latex-templates directory

% content/resources/templates/preamble.tex
\usepackage[margin=0.6in]{geometry}
\author{Milav Dabgar}
\usepackage{amsmath,amssymb,amsthm}
\usepackage{booktabs}
\usepackage{multirow}
\usepackage{xcolor}
\usepackage{tcolorbox}
\tcbuselibrary{breakable,skins}
\usepackage[colorlinks=true,linkcolor=blue]{hyperref}
\usepackage{titlesec}
\usepackage{enumitem}
\usepackage{tikz}
\usepackage{pgfplots}
\usepackage{circuitikz}
\usepackage[version=4]{mhchem}
\usepackage{longtable}
\usepackage{array}
\usepackage{float}
\usepackage{caption}
\usepackage{listings}

\lstset{
  basicstyle=\small\ttfamily,
  breaklines=true,
  breakatwhitespace=false,
  postbreak=\mbox{\textcolor{red}{$\hookrightarrow$}\space},
  float=false,
  numbers=left,
  numberstyle=\tiny\color{gray},
  numbersep=10pt,
  xleftmargin=2em,
  keywordstyle=\color{blue},
  commentstyle=\color{green!60!black},
  stringstyle=\color{purple},
  backgroundcolor=\color{gray!5},
  showstringspaces=false,
  tabsize=2,
  captionpos=b,
  keepspaces=true,
  columns=flexible
}

\pgfplotsset{compat=1.18}
\usetikzlibrary{shapes,arrows,positioning,calc,patterns,decorations.pathmorphing,decorations.markings,arrows.meta}

% Color scheme
\definecolor{headcolor}{RGB}{0,102,204}
\definecolor{keycolor}{RGB}{220,20,60}
\definecolor{solutioncolor}{RGB}{34,139,34}
\definecolor{mnemoniccolor}{RGB}{148,0,211}
\definecolor{codecolor}{RGB}{0,0,100}

% Spacing
\setlength{\parskip}{3pt}
\setlist[itemize]{nosep}
\setlist[enumerate]{nosep}

% Title formatting
\titleformat{\section}{\Large\bfseries\color{headcolor}}{\thesection}{1em}{}
\titleformat{\subsection}{\large\bfseries\color{headcolor}}{\thesubsection}{1em}{}

% Pandoc tightlist compatibility
\providecommand{\tightlist}{%
  \setlength{\itemsep}{0pt}\setlength{\parskip}{0pt}}

% Pandoc longtable compatibility
\newcounter{none}
\def\thenone{}


% content/resources/templates/gujarati-boxes.tex
\usepackage{fontspec}
\usepackage{polyglossia}

% Set Gujarati as main language (document is primarily in Gujarati)
% Note: gloss-gujarati.ldf doesn't exist in polyglossia, but it will use hyphenation patterns
\setdefaultlanguage{gujarati}
\setotherlanguage{english}

% Configure Gujarati font properly
% Use Language=Default to prevent polyglossia from trying to add language-specific features
% that don't exist for Gujarati, which causes "empty feature" warnings
\newfontfamily\gujaratifont[Script=Gujarati,AutoFakeBold=2.5,AutoFakeSlant=0.3]{Noto Sans Gujarati}
\setmainfont[Script=Gujarati,AutoFakeBold=2.5,AutoFakeSlant=0.3]{Noto Sans Gujarati}
% Use Noto Sans Gujarati for monospace to support Gujarati in text
\setmonofont[Scale=0.9]{Noto Sans Gujarati}

% Configure English to use the same font
\newfontfamily\englishfont[Script=Gujarati,AutoFakeBold=2.5,AutoFakeSlant=0.3]{Noto Sans Gujarati}

% Translations for polyglossia
\gappto\captionsgujarati{
  \renewcommand{\tablename}{કોષ્ટક}
  \renewcommand{\figurename}{આકૃતિ}
}

% Helper for TikZ nodes to ensure Gujarati font
\newcommand{\gu}[1]{{\gujaratifont #1}}

% Custom environments
\newtcolorbox{solutionbox}{
    breakable,
    enhanced,
    colback=solutioncolor!5!white,
    colframe=solutioncolor!75!black,
    fonttitle=\bfseries,
    title=જવાબ
}

\newtcolorbox{solutionboxnobreak}{
 colback=solutioncolor!5!white,
 colframe=solutioncolor!75!black,
 fonttitle=\bfseries,
 title=જવાબ
}

\newtcolorbox{keyformula}{
 breakable,
 enhanced,
 colback=keycolor!5!white,
 colframe=keycolor!75!black,
 fonttitle=\bfseries,
 title=રાસાયણિક સમીકરણ/સૂત્ર
}

\newtcolorbox{mnemonicbox}{
 breakable,
 enhanced,
 colback=mnemoniccolor!5!white,
 colframe=mnemoniccolor!75!black,
 fonttitle=\bfseries,
 title=મેમરી ટ્રીક
}


% Custom commands for GTU solutions
% This file defines semantic commands for consistent formatting

% Question command with automatic formatting
\newcommand{\question}[2]{%
  \section*{Question #1}%
  \textbf{#2}%
}

% OR question variant
\newcommand{\questionor}[2]{%
  \section*{Question #1 OR}%
  \textbf{#2}%
}

% Proper table environment with caption
\newenvironment{answertable}[1]{%
  \begin{table}[htbp]
  \centering
  \caption{#1}
}{%
  \end{table}
}

% Proper figure environment for diagrams
\newenvironment{answerdiagram}[1]{%
  \begin{figure}[htbp]
  \centering
  \caption{#1}
}{%
  \end{figure}
}

% Semantic markup for key terms
\newcommand{\keyword}[1]{\textbf{#1}}
\newcommand{\code}[1]{\texttt{#1}}
\newcommand{\classname}[1]{\texttt{#1}}
\newcommand{\methodname}[1]{\texttt{#1}}

% Proper quotation marks
\newcommand{\mnemonic}[1]{``#1''}

\usetikzlibrary{fit}

\title{મોબાઈલ કમ્પ્યુટિંગ અને નેટવર્ક્સ (4351602) - સમર 2024 સોલ્યુશન}
\date{મે 18, 2024}

\begin{document}
\maketitle

\questionmarks{1(અ)}{3}{વ્યાખ્યાયિત કરો : Peer to Peer network}

\begin{solutionbox}
Peer-to-Peer (P2P) નેટવર્ક એ વિતરિત નેટવર્ક આર્કિટેક્ચર છે જ્યાં દરેક નોડ (peer) ક્લાયન્ટ અને સર્વર બંને તરીકે કામ કરે છે અને કેન્દ્રીય નિયંત્રણ વિના સીધા સંસાધનો શેર કરે છે.

\begin{center}
\captionof{table}{Peer to Peer Network Features}
\begin{tabulary}{\linewidth}{|L|L|}
\hline
\textbf{પાસાં} & \textbf{વર્ણન} \\ \hline
\textbf{સ્ટ્રક્ચર} & વિકેન્દ્રીકૃત નેટવર્ક \\ \hline
\textbf{રોલ} & દરેક peer ક્લાયન્ટ અને સર્વર \\ \hline
\textbf{કંટ્રોલ} & કોઈ કેન્દ્રીય સત્તા નથી \\ \hline
\textbf{ઉદાહરણ} & BitTorrent, Skype \\ \hline
\end{tabulary}
\end{center}
\end{solutionbox}

\begin{mnemonicbox}
\mnemonic{Peers Share Equally}
\end{mnemonicbox}

\questionmarks{1(બ)}{4}{તુલના કરો : SMTP, POP અને IMAP}

\begin{solutionbox}
ઈમેઈલ પ્રોટોકોલ્સ ઈમેઈલ કમ્યુનિકેશન સિસ્ટમમાં અલગ અલગ હેતુઓ પૂરા કરે છે.

\begin{center}
\captionof{table}{Comparison of Email Protocols}
\begin{tabulary}{\linewidth}{|L|L|L|L|}
\hline
\textbf{ફીચર} & \textbf{SMTP} & \textbf{POP3} & \textbf{IMAP} \\ \hline
\textbf{હેતુ} & ઈમેઈલ મોકલવા & ઈમેઈલ ડાઉનલોડ કરવા & ઈમેઈલ એક્સેસ કરવા \\ \hline
\textbf{પોર્ટ} & 25, 587 & 110, 995 & 143, 993 \\ \hline
\textbf{સ્ટોરેજ} & સર્વર ફોરવર્ડ કરે & લોકલ સ્ટોરેજ & સર્વર સ્ટોરેજ \\ \hline
\textbf{એક્સેસ} & એક દિશામાં મોકલવું & સિંગલ ડિવાઇસ & મલ્ટિપલ ડિવાઇસ \\ \hline
\end{tabulary}
\end{center}
\end{solutionbox}

\begin{mnemonicbox}
\mnemonic{Send-Pop-Internet Mail Access}
\end{mnemonicbox}

\questionmarks{1(ક)}{7}{દરેક સ્તરની જવાબદારી સાથે OSI model સમજાવો}

\begin{solutionbox}
OSI (Open Systems Interconnection) મોડેલમાં સાત સ્તરો છે, દરેકની નેટવર્ક કમ્યુનિકેશન માટે ચોક્કસ જવાબદારીઓ છે.

\begin{center}
\begin{tikzpicture}[node distance=0.8cm, auto]
    \node [gtu block, minimum width=6cm] (app) {7. Application Layer};
    \node [gtu block, minimum width=6cm, below=of app] (pres) {6. Presentation Layer};
    \node [gtu block, minimum width=6cm, below=of pres] (sess) {5. Session Layer};
    \node [gtu block, minimum width=6cm, below=of sess] (trans) {4. Transport Layer};
    \node [gtu block, minimum width=6cm, below=of trans] (net) {3. Network Layer};
    \node [gtu block, minimum width=6cm, below=of net] (data) {2. Data Link Layer};
    \node [gtu block, minimum width=6cm, below=of data] (phys) {1. Physical Layer};
    
    \draw [gtu arrow] (app) -- (pres);
    \draw [gtu arrow] (pres) -- (sess);
    \draw [gtu arrow] (sess) -- (trans);
    \draw [gtu arrow] (trans) -- (net);
    \draw [gtu arrow] (net) -- (data);
    \draw [gtu arrow] (data) -- (phys);
\end{tikzpicture}
\captionof{figure}{OSI Model Layers}
\end{center}

\begin{center}
\captionof{table}{OSI Layers and Responsibilities}
\begin{tabulary}{\linewidth}{|C|L|L|}
\hline
\textbf{સ્તર} & \textbf{નામ} & \textbf{જવાબદારીઓ} \\ \hline
\textbf{7} & Application & યુઝર ઇન્ટરફેસ, નેટવર્ક સેવાઓ \\ \hline
\textbf{6} & Presentation & ડેટા એન્ક્રિપ્શન, કમ્પ્રેશન \\ \hline
\textbf{5} & Session & સેશન મેનેજમેન્ટ, ડાયલોગ કંટ્રોલ \\ \hline
\textbf{4} & Transport & End-to-end ડિલિવરી, એરર કંટ્રોલ \\ \hline
\textbf{3} & Network & રૂટિંગ, લોજિકલ એડ્રેસિંગ \\ \hline
\textbf{2} & Data Link & ફ્રેમ ફોર્મેટિંગ, એરર ડિટેક્શન \\ \hline
\textbf{1} & Physical & બિટ ટ્રાન્સમિશન, હાર્ડવેર \\ \hline
\end{tabulary}
\end{center}

\textbf{મુખ્ય મુદ્દાઓ:}
\begin{itemize}
    \item \keyword{Application Layer}: એપ્લિકેશનોને નેટવર્ક સેવાઓ પ્રદાન કરે
    \item \keyword{Transport Layer}: વિશ્વસનીય ડેટા ડિલિવરી સુનિશ્ચિત કરે
    \item \keyword{Network Layer}: નેટવર્ક્સ વચ્ચે રૂટિંગ હેન્ડલ કરે
\end{itemize}
\end{solutionbox}

\begin{mnemonicbox}
\mnemonic{All People Seem To Need Data Processing}
\end{mnemonicbox}

\questionmarks{1(ક OR)}{7}{TCP/IP model ની OSI model સાથે તુલના કરો}

\begin{solutionbox}
TCP/IP અને OSI મોડેલ્સ અલગ અલગ લેયર સ્ટ્રક્ચર સાથે નેટવર્ક આર્કિટેક્ચર ફ્રેમવર્ક છે.

\begin{center}
\begin{tikzpicture}[node distance=0.5cm]
    % OSI Stack
    \node [font=\bfseries] (osi_label) {OSI Model};
    \node [gtu block, minimum width=3.5cm, below=0.2cm of osi_label] (o7) {Application};
    \node [gtu block, minimum width=3.5cm, below=0cm of o7] (o6) {Presentation};
    \node [gtu block, minimum width=3.5cm, below=0cm of o6] (o5) {Session};
    \node [gtu block, minimum width=3.5cm, below=0cm of o5] (o4) {Transport};
    \node [gtu block, minimum width=3.5cm, below=0cm of o4] (o3) {Network};
    \node [gtu block, minimum width=3.5cm, below=0cm of o3] (o2) {Data Link};
    \node [gtu block, minimum width=3.5cm, below=0cm of o2] (o1) {Physical};

    % TCP/IP Stack
    \node [font=\bfseries, right=5cm of osi_label] (tcp_label) {TCP/IP Model};
    \node [gtu block, minimum width=3.5cm, minimum height=3cm, below=0.2cm of tcp_label] (t4) {Application};
    \node [gtu block, minimum width=3.5cm, below=0.2cm of t4, yshift=-0.5cm] (t3) {Transport}; 
    \node [gtu block, minimum width=3.5cm, below=0cm of t3] (t2) {Internet};
    \node [gtu block, minimum width=3.5cm, minimum height=1.8cm, below=0cm of t2] (t1) {Network Access};

    % Mapping lines
    \draw [dashed, ->] (o7.east) -- (t4.west);
    \draw [dashed, ->] (o6.east) -- (t4.west);
    \draw [dashed, ->] (o5.east) -- (t4.west);
    \draw [dashed, ->] (o4.east) -- (t3.west);
    \draw [dashed, ->] (o3.east) -- (t2.west);
    \draw [dashed, ->] (o2.east) -- (t1.west);
    \draw [dashed, ->] (o1.east) -- (t1.west);
\end{tikzpicture}
\captionof{figure}{OSI vs TCP/IP Model Comparison}
\end{center}

\begin{center}
\captionof{table}{OSI vs TCP/IP Model}
\begin{tabulary}{\linewidth}{|L|L|L|}
\hline
\textbf{પાસાં} & \textbf{OSI Model} & \textbf{TCP/IP Model} \\ \hline
\textbf{લેયર્સ} & 7 લેયર્સ & 4 લેયર્સ \\ \hline
\textbf{ડેવલપમેન્ટ} & થિયોરેટિકલ & પ્રેક્ટિકલ \\ \hline
\textbf{ઉપયોગ} & રેફરન્સ મોડેલ & ઇન્ટરનેટ સ્ટાન્ડર્ડ \\ \hline
\textbf{જટિલતા} & વધુ વિગતવાર & સરળીકૃત \\ \hline
\end{tabulary}
\end{center}

\textbf{મુખ્ય મુદ્દાઓ:}
\begin{itemize}
    \item \keyword{OSI}: વિગતવાર અલગીકરણ સાથે થિયોરેટિકલ ફ્રેમવર્ક
    \item \keyword{TCP/IP}: ઇન્ટરનેટ માટે પ્રેક્ટિકલ ઇમ્પ્લિમેન્ટેશન
    \item \keyword{મેપિંગ}: OSI ના ટોપ 3 લેયર્સ = TCP/IP માં Application layer
\end{itemize}
\end{solutionbox}

\begin{mnemonicbox}
\mnemonic{OSI Seven, TCP Four}
\end{mnemonicbox}

\questionmarks{2(અ)}{3}{સમજાવો : Network Address Translation (NAT)}

\begin{solutionbox}
NAT પ્રાઇવેટ IP એડ્રેસને પબ્લિક IP એડ્રેસમાં ટ્રાન્સલેટ કરે છે, જે મલ્ટિપલ ડિવાઇસને સિંગલ પબ્લિક IP શેર કરવા સક્ષમ બનાવે છે.

\begin{center}
\begin{tikzpicture}[auto, node distance=2cm]
    \node [gtu block] (priv1) {192.168.1.10};
    \node [gtu block, below=0.5cm of priv1] (priv2) {192.168.1.20};
    \node [gtu block, below=0.5cm of priv2] (priv3) {192.168.1.30};
    
    \node [gtu state, right=3cm of priv2, align=center] (router) {NAT Router\\(203.0.113.1)};
    \node [gtu block, right=3cm of router] (internet) {Internet Server};
    
    \draw [gtu arrow] (priv1) -- (router);
    \draw [gtu arrow] (priv2) -- (router);
    \draw [gtu arrow] (priv3) -- (router);
    \draw [gtu arrow] (router) -- (internet);
\end{tikzpicture}
\captionof{figure}{Network Address Translation}
\end{center}

\textbf{મુખ્ય મુદ્દાઓ:}
\begin{itemize}
    \item \keyword{હેતુ}: નેટવર્ક્સ વચ્ચે IP એડ્રેસ ટ્રાન્સલેશન
    \item \keyword{ફાયદો}: પબ્લિક IP એડ્રેસની બચત
    \item \keyword{સિક્યોરિટી}: આંતરિક નેટવર્ક સ્ટ્રક્ચર છુપાવે છે
\end{itemize}
\end{solutionbox}

\begin{mnemonicbox}
\mnemonic{Network Address Translation}
\end{mnemonicbox}

\questionmarks{2(બ)}{4}{વ્યાખ્યાયિત કરો : Subnetting and Supernetting}

\begin{solutionbox}
Subnetting અને Supernetting કાર્યક્ષમ નેટવર્ક મેનેજમેન્ટ માટે IP એડ્રેસિંગ તકનીકો છે.

\begin{center}
\captionof{table}{Subnetting vs Supernetting}
\begin{tabulary}{\linewidth}{|L|L|L|}
\hline
\textbf{તકનીક} & \textbf{વ્યાખ્યા} & \textbf{હેતુ} \\ \hline
\textbf{Subnetting} & નેટવર્કને નાના સબનેટ્સમાં વિભાજન & બહેતર સંગઠન \\ \hline
\textbf{Supernetting} & મલ્ટિપલ નેટવર્ક્સનું સંયોજન & રૂટ એગ્રિગેશન \\ \hline
\end{tabulary}
\end{center}

\textbf{મુખ્ય મુદ્દાઓ:}
\begin{itemize}
    \item \keyword{Subnetting}: નેટવર્ક બિટ્સ વધારે, હોસ્ટ બિટ્સ ઓછા કરે
    \item \keyword{Supernetting}: નેટવર્ક બિટ્સ ઓછા કરે, રૂટિંગ કાર્યક્ષમતા વધારે
    \item \keyword{CIDR}: Classless Inter-Domain Routing બંનેને સક્ષમ બનાવે
\end{itemize}
\end{solutionbox}

\begin{mnemonicbox}
\mnemonic{Sub-divides, Super-combines}
\end{mnemonicbox}

\questionmarks{2(ક)}{7}{સમજાવો : IPv4 ની Classful અને Classless notation addressing scheme}

\begin{solutionbox}
IPv4 એડ્રેસિંગ નેટવર્ક ઓળખ માટે classful અને classless સ્કીમનો ઉપયોગ કરે છે.

\begin{center}
\captionof{table}{Classful Addressing}
\begin{tabulary}{\linewidth}{|C|C|L|L|L|}
\hline
\textbf{Class} & \textbf{Range} & \textbf{Default Mask} & \textbf{Networks} & \textbf{Hosts} \\ \hline
\textbf{A} & 1-126 & /8 (255.0.0.0) & 126 & 16M \\ \hline
\textbf{B} & 128-191 & /16 (255.255.0.0) & 16K & 65K \\ \hline
\textbf{C} & 192-223 & /24 (255.255.255.0) & 2M & 254 \\ \hline
\end{tabulary}
\end{center}

\textbf{Classless (CIDR) ઉદાહરણો:}
\begin{itemize}
    \item \textbf{192.168.1.0/25}: 128 hosts
    \item \textbf{10.0.0.0/16}: 65,536 hosts
    \item \textbf{172.16.0.0/20}: 4,096 hosts
\end{itemize}

\textbf{મુખ્ય મુદ્દાઓ:}
\begin{itemize}
    \item \keyword{Classful}: ફિક્સ્ડ નેટવર્ક/હોસ્ટ બાઉન્ડરીઝ
    \item \keyword{Classless}: Variable Length Subnet Mask (VLSM)
    \item \keyword{CIDR}: વધુ કાર્યક્ષમ એડ્રેસ એલોકેશન
\end{itemize}
\end{solutionbox}

\begin{mnemonicbox}
\mnemonic{Class-Fixed, CIDR-Flexible}
\end{mnemonicbox}

\questionmarks{2(અ OR)}{3}{મોબાઇલ IP ના ધ્યેયોની ચર્ચા કરો}

\begin{solutionbox}
મોબાઇલ IP મોબાઇલ ડિવાઇસ માટે વિવિધ નેટવર્ક્સમાં સીમલેસ કનેક્ટિવિટી સક્ષમ કરે છે.

\textbf{મુખ્ય મુદ્દાઓ:}
\begin{itemize}
    \item \keyword{પારદર્શિતા}: એપ્લિકેશનોને મોબિલિટીની જાણ નથી
    \item \keyword{સુસંગતતા}: હાલના પ્રોટોકોલ્સ સાથે કામ કરે
    \item \keyword{કાર્યક્ષમતા}: ન્યૂનતમ રૂટિંગ ઓવરહેડ
\end{itemize}
\end{solutionbox}

\begin{mnemonicbox}
\mnemonic{Transparent Compatible Efficient}
\end{mnemonicbox}

\questionmarks{2(બ OR)}{4}{વ્યાખ્યાયિત કરો : ARP and RARP}

\begin{solutionbox}
ARP અને RARP વિવિધ એડ્રેસ પ્રકારો વચ્ચે મેપિંગ માટે એડ્રેસ રિઝોલ્યુશન પ્રોટોકોલ્સ છે.

\begin{center}
\captionof{table}{ARP vs RARP}
\begin{tabulary}{\linewidth}{|L|L|L|L|}
\hline
\textbf{પ્રોટોકોલ} & \textbf{પૂરું નામ} & \textbf{હેતુ} & \textbf{દિશા} \\ \hline
\textbf{ARP} & Address Resolution Protocol & IP to MAC મેપિંગ & લોજિકલ થી ફિઝિકલ \\ \hline
\textbf{RARP} & Reverse ARP & MAC to IP મેપિંગ & ફિઝિકલ થી લોજિકલ \\ \hline
\end{tabulary}
\end{center}
\end{solutionbox}

\begin{mnemonicbox}
\mnemonic{ARP-asks, RARP-reverses}
\end{mnemonicbox}

\questionmarks{2(ક OR)}{7}{સમજાવો : Stop and Wait, Stop and Wait ARQ data link layer protocols}

\begin{solutionbox}
આ પ્રોટોકોલ્સ ડેટા લિંક લેયર પર વિશ્વસનીય ડેટા ટ્રાન્સમિશન સુનિશ્ચિત કરે છે.

\begin{center}
\begin{tikzpicture}[node distance=2.5cm, auto]
    \node [gtu state] (s) {Sender};
    \node [gtu state, right=4cm of s] (r) {Receiver};
    
    \draw [thick] (s) -- ++(0, -5);
    \draw [thick] (r) -- ++(0, -5);
    
    \draw [->, thick] ($(s)+(0,-1)$) -- node [above, sloped] {Frame 0} ($(r)+(0,-1.5)$);
    \draw [->, dashed] ($(r)+(0,-2)$) -- node [above, sloped] {ACK 0} ($(s)+(0,-2.5)$);
    
    \draw [->, thick] ($(s)+(0,-3)$) -- node [above, sloped] {Frame 1} ($(r)+(0,-3.5)$);
    \draw [->, dashed] ($(r)+(0,-4)$) -- node [above, sloped] {ACK 1} ($(s)+(0,-4.5)$);
\end{tikzpicture}
\captionof{figure}{Stop and Wait Protocol Interaction}
\end{center}

\begin{center}
\captionof{table}{Protocol Comparison}
\begin{tabulary}{\linewidth}{|L|L|L|L|}
\hline
\textbf{પ્રોટોકોલ} & \textbf{એરર ડિટેક્શન} & \textbf{કાર્યક્ષમતા} & \textbf{જટિલતા} \\ \hline
\textbf{Stop and Wait} & બેસિક & ઓછી & સરળ \\ \hline
\textbf{Stop and Wait ARQ} & એડવાન્સ્ડ & મધ્યમ & મોડરેટ \\ \hline
\end{tabulary}
\end{center}

\textbf{મુખ્ય મુદ્દાઓ:}
\begin{itemize}
    \item \keyword{Stop and Wait}: ફ્રેમ મોકલો, acknowledgment ની રાહ જુઓ
    \item \keyword{ARQ}: એરર પર Automatic Repeat reQuest
    \item \keyword{Timeout}: કોઈ acknowledgment ન મળે તો ફરીથી મોકલો
\end{itemize}
\end{solutionbox}

\begin{mnemonicbox}
\mnemonic{Stop-Wait-Acknowledge}
\end{mnemonicbox}

\questionmarks{3(અ)}{3}{Wireless networks સમજાવો}

\begin{solutionbox}
વાયરલેસ નેટવર્ક્સ ફિઝિકલ કનેક્શન વિના કમ્યુનિકેશન માટે રેડિયો તરંગોનો ઉપયોગ કરે છે.

\textbf{મુખ્ય મુદ્દાઓ:}
\begin{itemize}
    \item \keyword{ટેકનોલોજી}: રેડિયો ફ્રીક્વન્સી ટ્રાન્સમિશન
    \item \keyword{પ્રકારો}: WiFi, Bluetooth, સેલ્યુલર
    \item \keyword{ફાયદાઓ}: મોબિલિટી, સરળ ઇન્સ્ટોલેશન
\end{itemize}
\end{solutionbox}

\begin{mnemonicbox}
\mnemonic{Wireless-Radio-Mobile}
\end{mnemonicbox}

\questionmarks{3(બ)}{4}{વ્યાખ્યાયિત કરો : Communication Middleware in mobile computing}

\begin{solutionbox}
કમ્યુનિકેશન મિડલવેર મોબાઇલ એપ્લિકેશન કમ્યુનિકેશન માટે અમૂર્તીકરણ લેયર પ્રદાન કરે છે.

\begin{center}
\captionof{table}{Communication Middleware}
\begin{tabulary}{\linewidth}{|L|L|}
\hline
\textbf{પાસાં} & \textbf{વર્ણન} \\ \hline
\textbf{હેતુ} & કમ્યુનિકેશન સરળ બનાવવું \\ \hline
\textbf{સ્થાન} & એપ અને નેટવર્ક વચ્ચે \\ \hline
\textbf{ફીચર્સ} & પ્રોટોકોલ હેન્ડલિંગ, ડેટા કન્વર્ઝન \\ \hline
\textbf{ઉદાહરણો} & CORBA, RMI \\ \hline
\end{tabulary}
\end{center}
\end{solutionbox}

\begin{mnemonicbox}
\mnemonic{Middle-Communication-Layer}
\end{mnemonicbox}

\questionmarks{3(ક)}{7}{મોબાઈલ કમ્પ્યુટિંગના આર્કિટેક્ચરની ચર્ચા કરો}

\begin{solutionbox}
મોબાઇલ કમ્પ્યુટિંગ આર્કિટેક્ચર મોબાઇલ એપ્લિકેશનોને સપોર્ટ કરતા મલ્ટિપલ પરસ્પર જોડાયેલા ઘટકોનો સમાવેશ કરે છે.

\begin{center}
\begin{tikzpicture}[node distance=0.5cm, auto]
    \node [gtu block, align=center] (mobile) {Mobile\\Device};
    \node [gtu block, right=of mobile] (wireless) {Wireless\\Network};
    \node [gtu block, right=of wireless] (bs) {Base\\Station};
    \node [gtu block, right=of bs] (mss) {MSS};
    \node [gtu block, right=of mss, align=center] (fixed) {Fixed\\Network};
    \node [gtu block, right=of fixed] (db) {DB};
    
    \draw [gtu arrow] (mobile) -- (wireless);
    \draw [gtu arrow] (wireless) -- (bs);
    \draw [gtu arrow] (bs) -- (mss);
    \draw [gtu arrow] (mss) -- (fixed);
    \draw [gtu arrow] (fixed) -- (db);
\end{tikzpicture}
\captionof{figure}{Mobile Computing Architecture}
\end{center}

\begin{center}
\captionof{table}{Architecture Components}
\begin{tabulary}{\linewidth}{|L|L|}
\hline
\textbf{ઘટક} & \textbf{કાર્ય} \\ \hline
\textbf{Mobile Device} & યુઝર ઇન્ટરફેસ, લોકલ પ્રોસેસિંગ \\ \hline
\textbf{Wireless Network} & રેડિયો કમ્યુનિકેશન \\ \hline
\textbf{Base Station} & નેટવર્ક એક્સેસ પોઇન્ટ \\ \hline
\textbf{MSS} & મોબિલિટી મેનેજમેન્ટ \\ \hline
\textbf{Fixed Network} & બેકબોન ઇન્ફ્રાસ્ટ્રક્ચર \\ \hline
\end{tabulary}
\end{center}

\textbf{મુખ્ય મુદ્દાઓ:}
\begin{itemize}
    \item \keyword{ત્રણ-સ્તરીય}: મોબાઇલ ડિવાઇસ, વાયરલેસ નેટવર્ક, ફિક્સ્ડ નેટવર્ક
    \item \keyword{મોબિલિટી સપોર્ટ}: હેન્ડઓફ મેનેજમેન્ટ
    \item \keyword{ડેટા મેનેજમેન્ટ}: કેશિંગ અને સિંક્રોનાઇઝેશન
\end{itemize}
\end{solutionbox}

\begin{mnemonicbox}
\mnemonic{Mobile-Wireless-Fixed}
\end{mnemonicbox}

\questionmarks{3(અ OR)}{3}{ad-hoc networks સમજાવો}

\begin{solutionbox}
Ad-hoc નેટવર્ક્સ ફિક્સ્ડ ઇન્ફ્રાસ્ટ્રક્ચર વિના સેલ્ફ-ઓર્ગેનાઇઝિંગ વાયરલેસ નેટવર્ક્સ છે.

\textbf{મુખ્ય મુદ્દાઓ:}
\begin{itemize}
    \item \keyword{સ્ટ્રક્ચર}: Peer-to-peer ટોપોલોજી
    \item \keyword{રૂટિંગ}: ડાયનેમિક રૂટ ડિસ્કવરી
    \item \keyword{એપ્લિકેશનો}: ઇમર્જન્સી, મિલિટરી
\end{itemize}
\end{solutionbox}

\begin{mnemonicbox}
\mnemonic{Ad-hoc-Self-Organizing}
\end{mnemonicbox}

\questionmarks{3(બ OR)}{4}{વ્યાખ્યાયિત કરો : Transaction Processing Middleware in mobile computing}

\begin{solutionbox}
ટ્રાન્ઝેક્શન પ્રોસેસિંગ મિડલવેર મોબાઇલ ડેટાબેસ ટ્રાન્ઝેક્શનોમાં ACID પ્રાપર્ટીઓ સુનિશ્ચિત કરે છે.

\begin{center}
\captionof{table}{ACID Properties}
\begin{tabulary}{\linewidth}{|L|L|}
\hline
\textbf{પ્રાપર્ટી} & \textbf{વર્ણન} \\ \hline
\textbf{Atomicity} & સર્વ અથવા કંઈ નહીં એક્ઝિક્યુશન \\ \hline
\textbf{Consistency} & ડેટાબેસ અખંડિતતા જાળવાય \\ \hline
\textbf{Isolation} & સમાંતર ટ્રાન્ઝેક્શન અલગીકરણ \\ \hline
\textbf{Durability} & કાયમી ટ્રાન્ઝેક્શન અસરો \\ \hline
\end{tabulary}
\end{center}
\end{solutionbox}

\begin{mnemonicbox}
\mnemonic{ACID-Properties}
\end{mnemonicbox}

\questionmarks{3(ક OR)}{7}{મોબાઇલ કમ્પ્યુટિંગની એપ્લિકેશન અને સેવાઓની ચર્ચા કરો}

\begin{solutionbox}
મોબાઇલ કમ્પ્યુટિંગ મલ્ટિપલ ડોમેન્સમાં વિવિધ એપ્લિકેશનોને સક્ષમ બનાવે છે.

\begin{center}
\captionof{table}{Applications and Services}
\begin{tabulary}{\linewidth}{|L|L|L|}
\hline
\textbf{ડોમેન} & \textbf{એપ્લિકેશનો} & \textbf{સેવાઓ} \\ \hline
\textbf{બિઝનેસ} & CRM, ERP & ડેટા સિંક્રોનાઇઝેશન \\ \hline
\textbf{હેલ્થકેર} & પેશન્ટ મોનિટરિંગ & રિમોટ ડાયગ્નોસિસ \\ \hline
\textbf{એજ્યુકેશન} & E-learning & કન્ટેન્ટ ડિલિવરી \\ \hline
\textbf{એન્ટરટેઈનમેન્ટ} & ગેમિંગ, સ્ટ્રીમિંગ & મીડિયા સેવાઓ \\ \hline
\textbf{નેવિગેશન} & GPS, મેપ્સ & લોકેશન સેવાઓ \\ \hline
\end{tabulary}
\end{center}

\textbf{મુખ્ય મુદ્દાઓ:}
\begin{itemize}
    \item \keyword{લોકેશન-આધારિત}: GPS નેવિગેશન, જિયો-ફેન્સિંગ
    \item \keyword{કમ્યુનિકેશન}: ઇમેઇલ, મેસેજિંગ, વિડિયો કોલ્સ
    \item \keyword{કોમર્સ}: મોબાઇલ બેંકિંગ, શોપિંગ
\end{itemize}
\end{solutionbox}

\begin{mnemonicbox}
\mnemonic{Business-Health-Education-Entertainment}
\end{mnemonicbox}

\questionmarks{4(અ)}{3}{વર્ણન કરો : Indirect TCP in mobile computing}

\begin{solutionbox}
Indirect TCP મોબાઇલ હોસ્ટ મોબિલિટી કાર્યક્ષમ રીતે હેન્ડલ કરવા માટે TCP કનેક્શન સ્પ્લિટ કરે છે.

\begin{center}
\begin{tikzpicture}[auto, node distance=2.5cm]
    \node [gtu block] (fh) {Fixed Host};
    \node [gtu block, right=of fh] (bs) {Base Station};
    \node [gtu block, right=of bs] (mh) {Mobile Host};
    
    \path [gtu arrow] (fh) -- node [above] {TCP Connection 1} (bs);
    \path [gtu arrow] (bs) -- node [above] {TCP Connection 2} (mh);
    
    \node [below=0.2cm of bs, font=\small\itshape] {Acts as Proxy};
\end{tikzpicture}
\captionof{figure}{Indirect TCP Split Connection}
\end{center}

\textbf{મુખ્ય મુદ્દાઓ:}
\begin{itemize}
    \item \keyword{સ્પ્લિટ કનેક્શન}: બે અલગ TCP કનેક્શનો
    \item \keyword{બેસ સ્ટેશન}: પ્રોક્સી તરીકે કામ કરે
    \item \keyword{ફાયદો}: ઝડપી હેન્ડઓફ
\end{itemize}
\end{solutionbox}

\begin{mnemonicbox}
\mnemonic{Indirect-Split-Proxy}
\end{mnemonicbox}

\questionmarks{4(બ)}{4}{મોબાઈલ આઈપીમાં પેકેટ ડિલિવરીના સ્ટેપ્સ સમજાવો}

\begin{solutionbox}
મોબાઇલ IP પેકેટ ડિલિવરીમાં રજિસ્ટ્રેશન, ટનલિંગ અને ડિલિવરી સ્ટેપ્સ સામેલ છે.

\textbf{સ્ટેપ્સ:}
\begin{enumerate}
    \item \keyword{રજિસ્ટ્રેશન}: મોબાઇલ નોડ હોમ એજન્ટ સાથે રજિસ્ટર કરે
    \item \keyword{ટનલિંગ}: હોમ એજન્ટ ફોરેન એજન્ટ માટે ટનલ બનાવે
    \item \keyword{એન્કેપ્સુલેશન}: મૂળ પેકેટ નવા હેડરમાં લપેટાય
    \item \keyword{ડિલિવરી}: ફોરેન એજન્ટ મોબાઇલ નોડને ડિલિવર કરે
\end{enumerate}
\end{solutionbox}

\begin{mnemonicbox}
\mnemonic{Register-Tunnel-Encapsulate-Deliver}
\end{mnemonicbox}

\questionmarks{4(ક)}{7}{મોબાઇલ આઈપી ની નીચેની ત્રણ પ્રક્રિયાઓ લખો: (1) Registration (2) Tunneling (3) Encapsulation}

\begin{solutionbox}
\textbf{1. Registration પ્રક્રિયા:}
\begin{itemize}
    \item મોબાઇલ નોડ ફોરેન એજન્ટ શોધે
    \item હોમ એજન્ટ સાથે care-of address રજિસ્ટર કરે
    \item ઓથેન્ટિકેશન અને બાઇન્ડિંગ અપડેટ
\end{itemize}

\textbf{2. Tunneling પ્રક્રિયા:}
\begin{itemize}
    \item હોમ એજન્ટ વર્ચ્યુઅલ ટનલ બનાવે
    \item ટનલ દ્વારા પેકેટ્સ ફોરવર્ડ કરાય
    \item End-to-end કનેક્ટિવિટી જાળવે
\end{itemize}

\textbf{3. Encapsulation પ્રક્રિયા:}
\begin{itemize}
    \item મૂળ પેકેટ પેલોડ બને
    \item Care-of address સાથે નવો IP હેડર ઉમેરાય
    \item પેકેટ ફોરેન નેટવર્કમાં ડિલિવર થાય
\end{itemize}

\begin{center}
\begin{tikzpicture}[node distance=1.5cm, auto]
    \node [gtu block] (orig) {Original Packet};
    \node [gtu process, right=of orig] (encap) {Encapsulation};
    \node [gtu block, right=of encap] (tunnel) {Tunneled Packet};
    \node [gtu process, right=of tunnel] (del) {Delivery};
    
    \draw [gtu arrow] (orig) -- (encap);
    \draw [gtu arrow] (encap) -- (tunnel);
    \draw [gtu arrow] (tunnel) -- (del);
\end{tikzpicture}
\captionof{figure}{Mobile IP Encapsulation Flow}
\end{center}

\textbf{મુખ્ય મુદ્દાઓ:}
\begin{itemize}
    \item \keyword{Registration}: લોકેશન અપડેટ મેકેનિઝમ
    \item \keyword{Tunneling}: વર્ચ્યુઅલ કનેક્શન સ્થાપના
    \item \keyword{Encapsulation}: પેકેટ રેપિંગ તકનીક
\end{itemize}
\end{solutionbox}

\begin{mnemonicbox}
\mnemonic{Register-Tunnel-Encapsulate}
\end{mnemonicbox}

\questionmarks{4(અ OR)}{3}{વર્ણન કરો : Snooping TCP in mobile computing}

\begin{solutionbox}
Snooping TCP બેસ સ્ટેશન પર TCP સેગમેન્ટ્સ કેશ અને મોનિટર કરીને પર્ફોર્મન્સ સુધારે છે.

\textbf{મુખ્ય મુદ્દાઓ:}
\begin{itemize}
    \item \keyword{લોકલ રિટ્રાન્સમિશન}: બેસ સ્ટેશન લોસેસ હેન્ડલ કરે
    \item \keyword{બફર મેનેજમેન્ટ}: અનએકનોલેજ્ડ સેગમેન્ટ્સ કેશ કરે
    \item \keyword{પારદર્શિતા}: End-to-end TCP જાળવાય
\end{itemize}
\end{solutionbox}

\begin{mnemonicbox}
\mnemonic{Snoop-Cache-Retransmit}
\end{mnemonicbox}

\questionmarks{4(બ OR)}{4}{મોબાઈલ આઈપીમાં હેન્ડઓવર મેનેજમેન્ટ સમજાવો}

\begin{solutionbox}
હેન્ડઓવર મેનેજમેન્ટ જ્યારે મોબાઇલ નોડ નેટવર્ક બદલે છે ત્યારે કનેક્ટિવિટી જાળવે છે.

\begin{center}
\captionof{table}{Handover Phases}
\begin{tabulary}{\linewidth}{|L|L|}
\hline
\textbf{તબક્કો} & \textbf{પ્રક્રિયા} \\ \hline
\textbf{ડિસ્કવરી} & નવો ફોરેન એજન્ટ શોધો \\ \hline
\textbf{રજિસ્ટ્રેશન} & Care-of address અપડેટ કરો \\ \hline
\textbf{ડેટા ફોરવર્ડિંગ} & પેકેટ્સ રીડાયરેક્ટ કરો \\ \hline
\textbf{ક્લીનઅપ} & જૂના રિસોર્સ રિલીઝ કરો \\ \hline
\end{tabulary}
\end{center}
\end{solutionbox}

\begin{mnemonicbox}
\mnemonic{Discover-Register-Forward-Cleanup}
\end{mnemonicbox}

\questionmarks{4(ક OR)}{7}{મોબાઇલ આઈપી માટે લક્ષ્યો અને જરૂરિયાતો લખો}

\begin{solutionbox}
\textbf{લક્ષ્યો:}
\begin{itemize}
    \item \keyword{પારદર્શિતા}: એપ્લિકેશનો માટે સીમલેસ મોબિલિટી
    \item \keyword{સુસંગતતા}: હાલના ઇન્ટરનેટ પ્રોટોકોલ્સ સાથે કામ
    \item \keyword{સ્કેલેબિલિટી}: મોટી સંખ્યામાં મોબાઇલ નોડ્સ સપોર્ટ
    \item \keyword{સિક્યોરિટી}: મોબાઇલ નોડ્સ ઓથેન્ટિકેટ અને ડેટા પ્રોટેક્ટ
\end{itemize}

\textbf{જરૂરિયાતો:}
\begin{itemize}
    \item \keyword{હોમ એજન્ટ}: મોબાઇલ નોડ લોકેશન જાળવે
    \item \keyword{ફોરેન એજન્ટ}: લોકલ સેવાઓ પ્રદાન કરે
    \item \keyword{Care-of Address}: ફોરેન નેટવર્કમાં ટેમ્પરરી એડ્રેસ
    \item \keyword{ટનલિંગ}: પેકેટ ફોરવર્ડિંગ મેકેનિઝમ
\end{itemize}

\begin{center}
\captionof{table}{Mobile IP Goals vs Requirements}
\begin{tabulary}{\linewidth}{|L|L|L|}
\hline
\textbf{પાસાં} & \textbf{લક્ષ્યો} & \textbf{જરૂરિયાતો} \\ \hline
\textbf{મોબિલિટી} & સીમલેસ મૂવમેન્ટ & Care-of address \\ \hline
\textbf{કનેક્ટિવિટી} & સેશન જાળવો & ટનલિંગ \\ \hline
\textbf{પર્ફોર્મન્સ} & ન્યૂનતમ ઓવરહેડ & કાર્યક્ષમ રૂટિંગ \\ \hline
\textbf{સિક્યોરિટી} & ઓથેન્ટિકેશન & સિક્યોર પ્રોટોકોલ્સ \\ \hline
\end{tabulary}
\end{center}
\end{solutionbox}

\begin{mnemonicbox}
\mnemonic{Transparent-Compatible-Scalable-Secure}
\end{mnemonicbox}

\questionmarks{5(અ)}{3}{મોબાઇલ નેટવર્કમાં 6G ની વિશેષતાઓ લખો}

\begin{solutionbox}
6G એડવાન્સ્ડ ક્ષમતાઓ સાથે મોબાઇલ નેટવર્ક્સની આવતી પેઢીનું પ્રતિનિધિત્વ કરે છે.

\textbf{મુખ્ય મુદ્દાઓ:}
\begin{itemize}
    \item \keyword{સ્પીડ}: 1 Tbps થિયોરેટિકલ સ્પીડ
    \item \keyword{લેટેન્સી}: સબ-મિલિસેકન્ડ લેટેન્સી
    \item \keyword{AI ઇન્ટિગ્રેશન}: નેટિવ આર્ટિફિશિયલ ઇન્ટેલિજન્સ
\end{itemize}
\end{solutionbox}

\begin{mnemonicbox}
\mnemonic{Tera-Speed-AI-Integration}
\end{mnemonicbox}

\questionmarks{5(બ)}{4}{વર્ણન કરો : Dynamic Host Configuration Protocol (DHCP)}

\begin{solutionbox}
DHCP ડિવાઇસને IP એડ્રેસ અને નેટવર્ક કન્ફિગરેશન આપોઆપ એસાઇન કરે છે.

\begin{center}
\captionof{table}{DHCP Process (DORA)}
\begin{tabulary}{\linewidth}{|L|L|}
\hline
\textbf{પ્રક્રિયા} & \textbf{વર્ણન} \\ \hline
\textbf{Discover} & ક્લાયન્ટ બ્રોડકાસ્ટ રિક્વેસ્ટ \\ \hline
\textbf{Offer} & સર્વર IP એડ્રેસ ઓફર કરે \\ \hline
\textbf{Request} & ક્લાયન્ટ ચોક્કસ IP રિક્વેસ્ટ કરે \\ \hline
\textbf{Acknowledge} & સર્વર એસાઇનમેન્ટ કન્ફર્મ કરે \\ \hline
\end{tabulary}
\end{center}
\end{solutionbox}

\begin{mnemonicbox}
\mnemonic{Discover-Offer-Request-Acknowledge}
\end{mnemonicbox}

\questionmarks{5(ક)}{7}{વર્ણન કરો : architecture of Wireless Personal Area Network (WLAN)}

\begin{solutionbox}
WLAN આર્કિટેક્ચર IEEE 802.11 સ્ટાન્ડર્ડ્સનો ઉપયોગ કરીને લોકલ એરિયાની અંદર વાયરલેસ કનેક્ટિવિટી પ્રદાન કરે છે.

\begin{center}
\begin{tikzpicture}[node distance=1.5cm, auto]
    \node [gtu block] (ds) {Distribution System (DS)};
    \node [gtu block, below=of ds] (ap) {Access Point (AP)};
    
    \node [gtu state, below left=1.5cm and 1cm of ap] (s1) {Station 1};
    \node [gtu state, below=1.5cm of ap] (s2) {Station 2};
    \node [gtu state, below right=1.5cm and 1cm of ap] (s3) {Station 3};
    
    \draw [gtu arrow] (ap) -- (ds);
    \draw [dashed] (s1) -- (ap);
    \draw [dashed] (s2) -- (ap);
    \draw [dashed] (s3) -- (ap);
    
    \node [draw, dashed, fit=(ap) (s1) (s3), label=right:BSS] {};
\end{tikzpicture}
\captionof{figure}{WLAN Architecture (Infrastructure Mode)}
\end{center}

\begin{center}
\captionof{table}{WLAN Components}
\begin{tabulary}{\linewidth}{|L|L|}
\hline
\textbf{ઘટક} & \textbf{કાર્ય} \\ \hline
\textbf{Access Point} & કેન્દ્રીય વાયરલેસ હબ \\ \hline
\textbf{Station} & વાયરલેસ ક્લાયન્ટ ડિવાઇસ \\ \hline
\textbf{Distribution System} & બેકબોન નેટવર્ક \\ \hline
\textbf{BSS} & બેસિક સર્વિસ સેટ \\ \hline
\textbf{ESS} & એક્સટેન્ડેડ સર્વિસ સેટ \\ \hline
\end{tabulary}
\end{center}

\textbf{મુખ્ય મુદ્દાઓ:}
\begin{itemize}
    \item \keyword{ઇન્ફ્રાસ્ટ્રક્ચર મોડ}: એક્સેસ પોઇન્ટ્સનો ઉપયોગ
    \item \keyword{Ad-hoc મોડ}: સીધા ડિવાઇસ કમ્યુનિકેશન
    \item \keyword{સ્ટાન્ડર્ડ્સ}: 802.11a/b/g/n/ac/ax પ્રોટોકોલ્સ
\end{itemize}
\end{solutionbox}

\begin{mnemonicbox}
\mnemonic{Access-Station-Distribution}
\end{mnemonicbox}

\questionmarks{5(અ OR)}{3}{મોબાઇલ નેટવર્કમાં 5G ની વિશેષતાઓ લખો}

\begin{solutionbox}
5G અલ્ટ્રા-લો લેટેન્સી સાથે એન્હાન્સ્ડ મોબાઇલ બ્રોડબેન્ડ પ્રદાન કરે છે.

\textbf{મુખ્ય મુદ્દાઓ:}
\begin{itemize}
    \item \keyword{સ્પીડ}: 10 Gbps સુધી ડાઉનલોડ
    \item \keyword{લેટેન્સી}: 1ms અલ્ટ્રા-લો લેટેન્સી
    \item \keyword{ડેન્સિટી}: પ્રતિ km\textsuperscript{2} 1 મિલિયન ડિવાઇસ
\end{itemize}
\end{solutionbox}

\begin{mnemonicbox}
\mnemonic{10G-1ms-1Million}
\end{mnemonicbox}

\questionmarks{5(બ OR)}{4}{WWW અને HTTP સમજાવો}

\begin{solutionbox}
વર્લ્ડ વાઇડ વેબ વેબ પેજ કમ્યુનિકેશન માટે HTTP પ્રોટોકોલનો ઉપયોગ કરે છે.

\begin{center}
\captionof{table}{WWW vs HTTP}
\begin{tabulary}{\linewidth}{|L|L|L|}
\hline
\textbf{પાસાં} & \textbf{WWW} & \textbf{HTTP} \\ \hline
\textbf{હેતુ} & માહિતી શેરિંગ & કમ્યુનિકેશન પ્રોટોકોલ \\ \hline
\textbf{ઘટકો} & વેબ પેજીસ, બ્રાઉઝર્સ & Request/response \\ \hline
\textbf{ફોર્મેટ} & HTML ડોક્યુમેન્ટ્સ & ટેક્સ્ટ-આધારિત પ્રોટોકોલ \\ \hline
\textbf{પોર્ટ} & વિવિધ & 80, 443 \\ \hline
\end{tabulary}
\end{center}
\end{solutionbox}

\begin{mnemonicbox}
\mnemonic{Web-Hypertext-Transfer}
\end{mnemonicbox}

\questionmarks{5(ક OR)}{7}{બ્લૂટૂથના આર્કિટેક્ચરનું વર્ણન કરો}

\begin{solutionbox}
બ્લૂટૂથ આર્કિટેક્ચર પ્રોટોકોલ સ્ટેકનો ઉપયોગ કરીને શોર્ટ-રેન્જ વાયરલેસ કમ્યુનિકેશન પ્રદાન કરે છે.

\begin{center}
\begin{tikzpicture}[node distance=0cm, outer sep=0pt]
    \node [gtu block, minimum width=6cm] (app) {Application Layer};
    \node [gtu block, minimum width=6cm, below=0.1cm of app] (mid) {OBEX / SDP};
    \node [gtu block, minimum width=6cm, below=0.1cm of mid] (l2cap) {L2CAP};
    \node [gtu block, minimum width=6cm, below=0.1cm of l2cap] (hci) {Host Controller Interface (HCI)};
    \node [gtu block, minimum width=6cm, below=0.1cm of hci] (lm) {Link Manager};
    \node [gtu block, minimum width=6cm, below=0.1cm of lm] (base) {Baseband};
    \node [gtu block, minimum width=6cm, below=0.1cm of base] (radio) {Radio Layer};
    
    \draw [->, thick] (app.south) -- (mid.north);
    \draw [->, thick] (mid.south) -- (l2cap.north);
\end{tikzpicture}
\captionof{figure}{Bluetooth Protocol Stack}
\end{center}

\begin{center}
\captionof{table}{Bluetooth Layers}
\begin{tabulary}{\linewidth}{|L|L|}
\hline
\textbf{લેયર} & \textbf{કાર્ય} \\ \hline
\textbf{Radio} & ફિઝિકલ ટ્રાન્સમિશન \\ \hline
\textbf{Baseband} & ટાઇમિંગ અને ફ્રીક્વન્સી હોપિંગ \\ \hline
\textbf{Link Manager} & કનેક્શન મેનેજમેન્ટ \\ \hline
\textbf{HCI} & હોસ્ટ કંટ્રોલર ઇન્ટરફેસ \\ \hline
\textbf{L2CAP} & લોજિકલ લિંક કંટ્રોલ \\ \hline
\textbf{Applications} & યુઝર સેવાઓ \\ \hline
\end{tabulary}
\end{center}

\textbf{મુખ્ય મુદ્દાઓ:}
\begin{itemize}
    \item \keyword{Piconet}: માસ્ટર-સ્લેવ નેટવર્ક ટોપોલોજી
    \item \keyword{Frequency Hopping}: 79 ફ્રીક્વન્સી ચેનલ્સ
    \item \keyword{Power Classes}: વિવિધ ટ્રાન્સમિશન રેન્જીસ
\end{itemize}
\end{solutionbox}

\begin{mnemonicbox}
\mnemonic{Radio-Baseband-Link-Host-Logic}
\end{mnemonicbox}

\end{document}
