\documentclass{article}
% Adjust the relative path to point to the latex-templates directory

% content/resources/templates/preamble.tex
\usepackage[margin=0.6in]{geometry}
\author{Milav Dabgar}
\usepackage{amsmath,amssymb,amsthm}
\usepackage{booktabs}
\usepackage{multirow}
\usepackage{xcolor}
\usepackage{tcolorbox}
\tcbuselibrary{breakable,skins}
\usepackage[colorlinks=true,linkcolor=blue]{hyperref}
\usepackage{titlesec}
\usepackage{enumitem}
\usepackage{tikz}
\usepackage{pgfplots}
\usepackage{circuitikz}
\usepackage[version=4]{mhchem}
\usepackage{longtable}
\usepackage{array}
\usepackage{float}
\usepackage{caption}
\usepackage{listings}

\lstset{
  basicstyle=\small\ttfamily,
  breaklines=true,
  breakatwhitespace=false,
  postbreak=\mbox{\textcolor{red}{$\hookrightarrow$}\space},
  float=false,
  numbers=left,
  numberstyle=\tiny\color{gray},
  numbersep=10pt,
  xleftmargin=2em,
  keywordstyle=\color{blue},
  commentstyle=\color{green!60!black},
  stringstyle=\color{purple},
  backgroundcolor=\color{gray!5},
  showstringspaces=false,
  tabsize=2,
  captionpos=b,
  keepspaces=true,
  columns=flexible
}

\pgfplotsset{compat=1.18}
\usetikzlibrary{shapes,arrows,positioning,calc,patterns,decorations.pathmorphing,decorations.markings,arrows.meta}

% Color scheme
\definecolor{headcolor}{RGB}{0,102,204}
\definecolor{keycolor}{RGB}{220,20,60}
\definecolor{solutioncolor}{RGB}{34,139,34}
\definecolor{mnemoniccolor}{RGB}{148,0,211}
\definecolor{codecolor}{RGB}{0,0,100}

% Spacing
\setlength{\parskip}{3pt}
\setlist[itemize]{nosep}
\setlist[enumerate]{nosep}

% Title formatting
\titleformat{\section}{\Large\bfseries\color{headcolor}}{\thesection}{1em}{}
\titleformat{\subsection}{\large\bfseries\color{headcolor}}{\thesubsection}{1em}{}

% Pandoc tightlist compatibility
\providecommand{\tightlist}{%
  \setlength{\itemsep}{0pt}\setlength{\parskip}{0pt}}

% Pandoc longtable compatibility
\newcounter{none}
\def\thenone{}


% content/resources/templates/english-boxes.tex

% Custom environments
\newtcolorbox{solutionbox}{
 breakable,
 enhanced,
 colback=solutioncolor!5!white,
 colframe=solutioncolor!75!black,
 fonttitle=\bfseries,
 title=Solution
}

\newtcolorbox{solutionboxnobreak}{
 colback=solutioncolor!5!white,
 colframe=solutioncolor!75!black,
 fonttitle=\bfseries,
 title=Solution
}

\newtcolorbox{keyformula}{
 breakable,
 enhanced,
 colback=keycolor!5!white,
 colframe=keycolor!75!black,
 fonttitle=\bfseries,
 title=Key Formula
}

\newtcolorbox{mnemonicboxenv}{
 breakable,
 enhanced,
 colback=mnemoniccolor!5!white,
 colframe=mnemoniccolor!75!black,
 fonttitle=\bfseries,
 title=Mnemonic
}

\newcommand{\mnemonicbox}[1]{%
  \begin{mnemonicboxenv}
    #1
  \end{mnemonicboxenv}
}


% Custom commands for GTU solutions
% This file defines semantic commands for consistent formatting

% Question command with automatic formatting
\newcommand{\question}[2]{%
  \section*{Question #1}%
  \textbf{#2}%
}

% OR question variant
\newcommand{\questionor}[2]{%
  \section*{Question #1 OR}%
  \textbf{#2}%
}

% Proper table environment with caption
\newenvironment{answertable}[1]{%
  \begin{table}[htbp]
  \centering
  \caption{#1}
}{%
  \end{table}
}

% Proper figure environment for diagrams
\newenvironment{answerdiagram}[1]{%
  \begin{figure}[htbp]
  \centering
  \caption{#1}
}{%
  \end{figure}
}

% Semantic markup for key terms
\newcommand{\keyword}[1]{\textbf{#1}}
\newcommand{\code}[1]{\texttt{#1}}
\newcommand{\classname}[1]{\texttt{#1}}
\newcommand{\methodname}[1]{\texttt{#1}}

% Proper quotation marks
\newcommand{\mnemonic}[1]{``#1''}

\usetikzlibrary{fit, positioning, arrows.meta, shapes.geometric, calc}

\title{Mobile Computing and Networks (4351602) - Winter 2023 Solution}
\date{December 06, 2023}

\begin{document}
\maketitle

\questionmarks{1(a)}{3}{Differentiate between client server and peer to peer network.}

\begin{solutionbox}
\textbf{Comparison:}
\begin{center}
\captionof{table}{Client-Server vs P2P}
\begin{tabulary}{\linewidth}{|L|L|L|}
\hline
\textbf{Parameter} & \textbf{Client-Server Network} & \textbf{Peer-to-Peer Network} \\ \hline
\textbf{Architecture} & Centralized with dedicated server & Decentralized, all nodes equal \\ \hline
\textbf{Cost} & Higher due to server hardware & Lower, uses existing computers \\ \hline
\textbf{Security} & High, centralized control & Lower, distributed control \\ \hline
\textbf{Scalability} & Limited by server capacity & Better, resources increase with nodes \\ \hline
\end{tabulary}
\end{center}
\end{solutionbox}

\begin{mnemonicbox}
\mnemonic{CSS-P: Client-Server = Centralized Security, P2P = Peer Power}
\end{mnemonicbox}

\questionmarks{1(b)}{4}{Explain ARP Protocol with its working.}

\begin{solutionbox}
\textbf{ARP (Address Resolution Protocol)} maps IP addresses to MAC addresses in local networks.

\textbf{Working Process:}
\begin{itemize}
    \item \keyword{Broadcast Request}: Host broadcasts ARP request with target IP
    \item \keyword{Cache Check}: Receiving hosts check if IP matches theirs
    \item \keyword{Reply Generation}: Target host sends ARP reply with MAC address
    \item \keyword{Cache Update}: Requesting host updates ARP table
\end{itemize}

\textbf{ARP Table Example:}
\begin{center}
\begin{tabular}{|l|l|l|}
\hline
\textbf{IP Address} & \textbf{MAC Address} & \textbf{TTL} \\ \hline
192.168.1.1 & 00:1A:2B:3C:4D:5E & 300s \\ \hline
\end{tabular}
\end{center}
\end{solutionbox}

\begin{mnemonicbox}
\mnemonic{BCRU: Broadcast, Cache, Reply, Update}
\end{mnemonicbox}

\questionmarks{1(c)}{7}{Explain OSI model with diagram.}

\begin{solutionbox}
The \textbf{OSI (Open Systems Interconnection)} model has 7 layers for network communication.

\begin{center}
\begin{tikzpicture}[node distance=0.6cm, auto]
    \node [gtu block, minimum width=6cm] (app) {7. Application Layer};
    \node [gtu block, minimum width=6cm, below=of app] (pres) {6. Presentation Layer};
    \node [gtu block, minimum width=6cm, below=of pres] (sess) {5. Session Layer};
    \node [gtu block, minimum width=6cm, below=of sess] (trans) {4. Transport Layer};
    \node [gtu block, minimum width=6cm, below=of trans] (net) {3. Network Layer};
    \node [gtu block, minimum width=6cm, below=of net] (data) {2. Data Link Layer};
    \node [gtu block, minimum width=6cm, below=of data] (phys) {1. Physical Layer};

    \draw [gtu arrow] (app) -- (pres);
    \draw [gtu arrow] (pres) -- (sess);
    \draw [gtu arrow] (sess) -- (trans);
    \draw [gtu arrow] (trans) -- (net);
    \draw [gtu arrow] (net) -- (data);
    \draw [gtu arrow] (data) -- (phys);
\end{tikzpicture}
\captionof{figure}{OSI Model Layers}
\end{center}

\textbf{Layer Functions:}
\begin{itemize}
    \item \keyword{Physical}: Bit transmission over physical medium
    \item \keyword{Data Link}: Frame transmission, error detection
    \item \keyword{Network}: Routing, IP addressing
    \item \keyword{Transport}: End-to-end delivery, TCP/UDP
    \item \keyword{Session}: Connection management
    \item \keyword{Presentation}: Data encryption, compression
    \item \keyword{Application}: User interfaces, email, web
\end{itemize}
\end{solutionbox}

\begin{mnemonicbox}
\mnemonic{All People Seem To Need Data Processing}
\end{mnemonicbox}

\questionmarks{1(c OR)}{7}{What is Congestion? Explain Congestion Control.}

\begin{solutionbox}
\textbf{Congestion} occurs when network traffic exceeds available bandwidth, causing packet delays and losses.

\textbf{Types of Congestion Control:}
\begin{center}
\captionof{table}{Congestion Control Types}
\begin{tabulary}{\linewidth}{|L|L|L|}
\hline
\textbf{Type} & \textbf{Method} & \textbf{Description} \\ \hline
\textbf{Open-Loop} & Prevention & Traffic shaping before congestion \\ \hline
\textbf{Closed-Loop} & Reaction & Feedback-based adjustment \\ \hline
\end{tabulary}
\end{center}

\textbf{Congestion Control Techniques:}
\begin{itemize}
    \item \keyword{Traffic Shaping}: Regulate data transmission rate
    \item \keyword{Admission Control}: Limit new connections during congestion
    \item \keyword{Load Shedding}: Drop packets when buffers full
    \item \keyword{Backpressure}: Send congestion signals upstream
\end{itemize}
\end{solutionbox}

\begin{mnemonicbox}
\mnemonic{TALB: Traffic, Admission, Load, Backpressure}
\end{mnemonicbox}

\questionmarks{2(a)}{3}{What is Ad-hoc Network? Explain it.}

\begin{solutionbox}
\textbf{Ad-hoc Network} is a wireless network without fixed infrastructure where nodes communicate directly.

\textbf{Characteristics:}
\begin{itemize}
    \item \keyword{Self-organizing}: Automatic network formation
    \item \keyword{Dynamic topology}: Nodes can join/leave freely
    \item \keyword{Multi-hop routing}: Messages relay through intermediate nodes
    \item \keyword{Distributed control}: No central authority
\end{itemize}

\textbf{Applications:}
\begin{itemize}
    \item Emergency response, military operations, sensor networks
\end{itemize}
\end{solutionbox}

\begin{mnemonicbox}
\mnemonic{SDMD: Self-organizing, Dynamic, Multi-hop, Distributed}
\end{mnemonicbox}

\questionmarks{2(b)}{4}{Explain Handover Management in Mobile IP.}

\begin{solutionbox}
\textbf{Handover} is the process of maintaining connectivity when a mobile node moves between networks.

\textbf{Handover Process:}
\begin{center}
\begin{tikzpicture}[node distance=2cm, auto]
    \node [gtu state] (mn) {Mobile Node};
    \node [gtu block, right=of mn] (fa2) {Foreign Agent 2};
    \node [gtu block, right=of fa2] (ha) {Home Agent};
    
    \draw [gtu arrow] (mn) -- node [above] {1. Discovery} (fa2);
    \draw [gtu arrow] (fa2) -- node [below] {2. Adv} (mn);
    \draw [gtu arrow] (mn) to[bend left=30] node [above] {3. Register} (ha);
    \draw [gtu arrow] (ha) to[bend left=30] node [below] {4. Reply} (mn);
\end{tikzpicture}
\captionof{figure}{Mobile IP Handover}
\end{center}

\textbf{Types:}
\begin{itemize}
    \item \keyword{Hard Handover}: Break-before-make connection
    \item \keyword{Soft Handover}: Make-before-break connection
\end{itemize}
\end{solutionbox}

\begin{mnemonicbox}
\mnemonic{DARU: Discovery, Advertisement, Registration, Update}
\end{mnemonicbox}

\questionmarks{2(c)}{7}{Explain Three tier architecture of mobile computing with diagram.}

\begin{solutionbox}
\textbf{Three-tier architecture} separates mobile applications into presentation, application logic, and data layers.

\begin{center}
\begin{tikzpicture}[node distance=1cm, auto]
    \node [gtu block, align=center] (pres) {Tier 1: Presentation\\ \footnotesize (Mobile Device, UI)};
    \node [gtu block, below=of pres, align=center] (app) {Tier 2: Application\\ \footnotesize (Business Logic, Middleware)};
    \node [gtu block, below=of app, align=center] (data) {Tier 3: Data\\ \footnotesize (Database, Storage)};

    \draw [gtu arrow] (pres) -- (app);
    \draw [gtu arrow] (app) -- (data);
    \draw [gtu arrow] (data) -- (app);
    \draw [gtu arrow] (app) -- (pres);
\end{tikzpicture}
\captionof{figure}{Three-Tier Mobile Architecture}
\end{center}

\textbf{Layer Functions:}
\begin{itemize}
    \item \keyword{Presentation}: User interface, mobile apps
    \item \keyword{Application}: Business logic, middleware services
    \item \keyword{Data}: Database management, storage systems
\end{itemize}

\textbf{Benefits:}
\begin{itemize}
    \item \keyword{Scalability}: Independent layer scaling
    \item \keyword{Maintainability}: Separate concerns
    \item \keyword{Flexibility}: Technology independence
\end{itemize}
\end{solutionbox}

\begin{mnemonicbox}
\mnemonic{PAD: Presentation, Application, Data}
\end{mnemonicbox}

\questionmarks{2(a OR)}{3}{Explain Need of Wireless Network.}

\begin{solutionbox}
\textbf{Wireless Networks} provide connectivity without physical cables.

\textbf{Needs:}
\begin{itemize}
    \item \keyword{Mobility}: Users can move freely while connected
    \item \keyword{Flexibility}: Easy network expansion and reconfiguration
    \item \keyword{Cost-effective}: Reduced cabling infrastructure costs
    \item \keyword{Accessibility}: Internet access in remote areas
\end{itemize}

\textbf{Applications:}
\begin{itemize}
    \item Mobile communications, WiFi hotspots, IoT devices
\end{itemize}
\end{solutionbox}

\begin{mnemonicbox}
\mnemonic{MFCA: Mobility, Flexibility, Cost, Accessibility}
\end{mnemonicbox}

\questionmarks{2(b OR)}{4}{Explain Registration, tunneling and encapsulation in mobile ip.}

\begin{solutionbox}
\textbf{Mobile IP Components:}

\begin{center}
\captionof{table}{Mobile IP Concepts}
\begin{tabulary}{\linewidth}{|L|L|L|}
\hline
\textbf{Process} & \textbf{Description} & \textbf{Purpose} \\ \hline
\textbf{Registration} & Mobile node registers with home agent & Location update \\ \hline
\textbf{Tunneling} & Creates virtual path between agents & Route packets \\ \hline
\textbf{Encapsulation} & Wraps original packet in new header & Address translation \\ \hline
\end{tabulary}
\end{center}

\textbf{Process Flow:}
\begin{center}
Original Packet $\rightarrow$ Encapsulation $\rightarrow$ Tunnel $\rightarrow$ Decapsulation $\rightarrow$ Destination
\end{center}

\textbf{Registration Steps:}
\begin{itemize}
    \item Mobile node discovers foreign agent
    \item Sends registration request to home agent
    \item Home agent updates location binding
\end{itemize}
\end{solutionbox}

\begin{mnemonicbox}
\mnemonic{RTE: Registration, Tunneling, Encapsulation}
\end{mnemonicbox}

\questionmarks{2(c OR)}{7}{What is Middleware? Write down examples of middleware and explain any one of them in detail.}

\begin{solutionbox}
\textbf{Middleware} is software that connects different applications and services in distributed systems.

\textbf{Examples of Middleware:}
\begin{itemize}
    \item Message-Oriented Middleware (MOM)
    \item Remote Procedure Call (RPC)
    \item Object Request Broker (ORB)
    \item Database Middleware
    \item Web Services
\end{itemize}

\textbf{Message-Oriented Middleware (MOM) - Detailed:}

\textbf{Architecture:}
\begin{center}
\begin{tikzpicture}[node distance=0.5cm, auto]
    \node [gtu state] (sender) {Sender};
    \node [gtu block, right=of sender] (q1) {Queue};
    \node [gtu block, right=of q1] (mom) {MOM Layer};
    \node [gtu block, right=of mom] (q2) {Queue};
    \node [gtu state, right=of q2] (recv) {Receiver};

    \draw [gtu arrow] (sender) -- (q1);
    \draw [gtu arrow] (q1) -- (mom);
    \draw [gtu arrow] (mom) -- (q2);
    \draw [gtu arrow] (q2) -- (recv);
\end{tikzpicture}
\captionof{figure}{MOM Architecture}
\end{center}

\textbf{Features:}
\begin{itemize}
    \item \keyword{Asynchronous Communication}: Non-blocking message exchange
    \item \keyword{Reliability}: Message persistence and delivery guarantees
    \item \keyword{Scalability}: Handle multiple concurrent connections
    \item \keyword{Platform Independence}: Cross-platform communication
\end{itemize}

\textbf{Benefits:}
\begin{itemize}
    \item Loose coupling between applications
    \item Improved system reliability
    \item Better fault tolerance
\end{itemize}
\end{solutionbox}

\begin{mnemonicbox}
\mnemonic{ARSP: Asynchronous, Reliable, Scalable, Platform-independent}
\end{mnemonicbox}

\questionmarks{3(a)}{3}{Give Full form for 'www'. Explain it.}

\begin{solutionbox}
\textbf{WWW = World Wide Web}

\textbf{Explanation:}
\begin{itemize}
    \item \keyword{Global Information System}: Interconnected web of documents
    \item \keyword{HTTP Protocol}: Uses HyperText Transfer Protocol
    \item \keyword{URL Addressing}: Unique resource locators
    \item \keyword{Hyperlinks}: Navigate between web pages
\end{itemize}

\textbf{Components:}
\begin{itemize}
    \item Web servers, browsers, HTML documents, URLs
\end{itemize}
\end{solutionbox}

\begin{mnemonicbox}
\mnemonic{GHUH: Global, HTTP, URL, Hyperlinks}
\end{mnemonicbox}

\questionmarks{3(b)}{4}{Explain applications of Mobile Computing.}

\begin{solutionbox}
\textbf{Mobile Computing Applications:}

\begin{center}
\captionof{table}{Applications}
\begin{tabulary}{\linewidth}{|L|L|L|}
\hline
\textbf{Category} & \textbf{Applications} & \textbf{Benefits} \\ \hline
\textbf{Business} & Email, CRM, Sales & Productivity, Real-time access \\ \hline
\textbf{Healthcare} & Patient monitoring, Telemedicine & Remote care, Emergency response \\ \hline
\textbf{Education} & E-learning, Digital libraries & Flexible learning, Resource access \\ \hline
\textbf{Entertainment} & Gaming, Streaming, Social media & On-demand content, Connectivity \\ \hline
\end{tabulary}
\end{center}

\textbf{Key Features:}
\begin{itemize}
    \item \keyword{Location-based services}: GPS navigation, local search
    \item \keyword{Mobile payments}: Digital wallets, contactless transactions
    \item \keyword{IoT integration}: Smart home, wearable devices
\end{itemize}
\end{solutionbox}

\begin{mnemonicbox}
\mnemonic{BHEE: Business, Healthcare, Education, Entertainment}
\end{mnemonicbox}

\questionmarks{3(c)}{7}{Explain working of DHCP with the help of diagram and explain its advantages.}

\begin{solutionbox}
\textbf{DHCP (Dynamic Host Configuration Protocol)} automatically assigns IP addresses to network devices.

\textbf{DHCP Process (DORA):}
\begin{center}
\begin{tikzpicture}[node distance=3cm, auto]
    \node [gtu state] (client) {Client};
    \node [gtu state, right=of client] (server) {DHCP Server};

    \draw [gtu arrow] (client) -- node [above] {1. DISCOVER} (server);
    \draw [gtu arrow] (server) to[bend left=20] node [below] {2. OFFER} (client);
    \draw [gtu arrow] (client) to[bend left=20] node [above] {3. REQUEST} (server);
    \draw [gtu arrow] (server) to[bend left=40] node [below] {4. ACK} (client);
\end{tikzpicture}
\captionof{figure}{DHCP DORA Process}
\end{center}

\textbf{Configuration Information Provided:}
\begin{itemize}
    \item IP address and subnet mask
    \item Default gateway address
    \item DNS server addresses
    \item Lease duration
\end{itemize}

\textbf{Advantages:}
\begin{itemize}
    \item \keyword{Automatic Configuration}: No manual IP assignment
    \item \keyword{Centralized Management}: Single point of control
    \item \keyword{Efficient IP Usage}: Dynamic allocation prevents waste
    \item \keyword{Reduced Errors}: Eliminates manual configuration mistakes
    \item \keyword{Easy Maintenance}: Simple network changes
\end{itemize}
\end{solutionbox}

\begin{mnemonicbox}
\mnemonic{DORA: Discover, Offer, Request, Acknowledge}
\end{mnemonicbox}

\questionmarks{3(a OR)}{3}{Write down: Importance of HTTPS.}

\begin{solutionbox}
\textbf{HTTPS (HyperText Transfer Protocol Secure)} provides secure web communication.

\textbf{Importance:}
\begin{itemize}
    \item \keyword{Data Encryption}: Protects data in transit using SSL/TLS
    \item \keyword{Authentication}: Verifies server identity with certificates
    \item \keyword{Data Integrity}: Prevents data tampering during transmission
    \item \keyword{Trust Building}: Increases user confidence in websites
\end{itemize}

\textbf{Security Benefits:}
\begin{itemize}
    \item Protection against eavesdropping and man-in-the-middle attacks
\end{itemize}
\end{solutionbox}

\begin{mnemonicbox}
\mnemonic{EADT: Encryption, Authentication, Integrity, Trust}
\end{mnemonicbox}

\questionmarks{3(b OR)}{4}{What is Bearer Network? Explain in Detail.}

\begin{solutionbox}
\textbf{Bearer Network} is the underlying network infrastructure that carries data traffic between endpoints.

\textbf{Types of Bearer Networks:}
\begin{center}
\captionof{table}{Bearer Networks}
\begin{tabulary}{\linewidth}{|L|L|L|}
\hline
\textbf{Type} & \textbf{Technology} & \textbf{Characteristics} \\ \hline
\textbf{Circuit-Switched} & Traditional telephony & Dedicated path, Guaranteed bandwidth \\ \hline
\textbf{Packet-Switched} & Internet, IP networks & Shared resources, Variable bandwidth \\ \hline
\textbf{Wireless} & Cellular, WiFi & Mobile connectivity, Air interface \\ \hline
\end{tabulary}
\end{center}

\textbf{Functions:}
\begin{itemize}
    \item \keyword{Data Transport}: Carry user data and signaling
    \item \keyword{Quality of Service}: Manage bandwidth and latency
    \item \keyword{Routing}: Direct traffic between networks
    \item \keyword{Network Management}: Monitor and control traffic
\end{itemize}
\end{solutionbox}

\begin{mnemonicbox}
\mnemonic{DQRN: Data transport, QoS, Routing, Network management}
\end{mnemonicbox}

\questionmarks{3(c OR)}{7}{List out types of TCP and explain any one in detail.}

\begin{solutionbox}
\textbf{Types of TCP:}
\begin{itemize}
    \item Standard TCP (TCP Tahoe)
    \item TCP Reno
    \item TCP New Reno
    \item TCP Vegas
    \item TCP SACK (Selective Acknowledgment)
    \item TCP Cubic
\end{itemize}

\textbf{TCP Reno - Detailed Explanation:}

\textbf{Features:}
\begin{itemize}
    \item \keyword{Fast Retransmit}: Retransmit lost packets quickly
    \item \keyword{Fast Recovery}: Avoid slow start after fast retransmit
    \item \keyword{Congestion Avoidance}: Linear increase in congestion window
    \item \keyword{Duplicate ACK Detection}: Identify packet loss
\end{itemize}

\textbf{Congestion Control Algorithm:}
\begin{center}
\begin{tikzpicture}[node distance=1.5cm, auto]
    \node [gtu state] (slow) {Slow Start};
    \node [gtu decision, below=of slow] (dupacks) {3 Dup ACKs?};
    \node [gtu block, right=of dupacks] (fastre) {Fast Retransmit};
    \node [gtu block, right=of fastre] (fastrec) {Fast Recovery};
    \node [gtu block, below=of fastrec] (cong) {Congestion Avoidance};
    \node [gtu decision, left=of cong] (timeout) {Timeout?};

    \draw [gtu arrow] (slow) -- (dupacks);
    \draw [gtu arrow] (dupacks) -- node [above] {Yes} (fastre);
    \draw [gtu arrow] (fastre) -- (fastrec);
    \draw [gtu arrow] (fastrec) -- (cong);
    \draw [gtu arrow] (dupacks) -- node [left] {No} (timeout);
    \draw [gtu arrow] (timeout) -- node [left] {Yes} (slow);
    \draw [gtu arrow] (timeout) -- node [above] {No} (cong);
\end{tikzpicture}
\captionof{figure}{TCP Reno phases}
\end{center}

\textbf{Advantages:}
\begin{itemize}
    \item \keyword{Better Performance}: Faster recovery from packet loss
    \item \keyword{Efficiency}: Maintains higher throughput
    \item \keyword{Fairness}: Equitable bandwidth sharing
\end{itemize}
\end{solutionbox}

\begin{mnemonicbox}
\mnemonic{FFCE: Fast retransmit, Fast recovery, Congestion avoidance, Efficiency}
\end{mnemonicbox}

\questionmarks{4(a)}{3}{Define WLAN. List out types of WLAN.}

\begin{solutionbox}
\textbf{WLAN (Wireless Local Area Network)} provides wireless connectivity within a limited area.

\textbf{Types of WLAN:}
\begin{itemize}
    \item \keyword{Infrastructure Mode}: Uses access points for connectivity
    \item \keyword{Ad-hoc Mode}: Direct device-to-device communication
    \item \keyword{Mesh Networks}: Multi-hop wireless connectivity
    \item \keyword{Hybrid Networks}: Combination of infrastructure and ad-hoc
\end{itemize}

\textbf{Standards:}
\begin{itemize}
    \item IEEE 802.11a/b/g/n/ac/ax (WiFi 6)
\end{itemize}
\end{solutionbox}

\begin{mnemonicbox}
\mnemonic{IAMH: Infrastructure, Ad-hoc, Mesh, Hybrid}
\end{mnemonicbox}

\questionmarks{4(b)}{4}{What is Routing? Explain types of Routing.}

\begin{solutionbox}
\textbf{Routing} is the process of selecting paths for data packets across networks.

\textbf{Types of Routing:}
\begin{center}
\captionof{table}{Routing Types}
\begin{tabulary}{\linewidth}{|L|L|L|}
\hline
\textbf{Type} & \textbf{Method} & \textbf{Characteristics} \\ \hline
\textbf{Static Routing} & Manual configuration & Fixed paths, No automatic updates \\ \hline
\textbf{Dynamic Routing} & Automatic updates & Adaptive paths, Real-time changes \\ \hline
\textbf{Default Routing} & Catch-all route & Used when no specific route exists \\ \hline
\textbf{Distance Vector} & Hop count based & RIP protocol, Simple implementation \\ \hline
\textbf{Link State} & Network topology & OSPF protocol, Faster convergence \\ \hline
\end{tabulary}
\end{center}

\textbf{Dynamic Routing Advantages:}
\begin{itemize}
    \item \textbf{Automatic adaptation} to network changes
    \item \textbf{Load balancing} across multiple paths
    \item \textbf{Fault tolerance} with alternate routes
\end{itemize}
\end{solutionbox}

\begin{mnemonicbox}
\mnemonic{SDDL: Static, Dynamic, Default, Link-state}
\end{mnemonicbox}

\questionmarks{4(c)}{7}{Explain architecture of WLAN.}

\begin{solutionbox}
\textbf{WLAN Architecture Components:}

\begin{center}
\begin{tikzpicture}[node distance=1cm, auto]
    \node [gtu block] (ds) {Distribution System (DS)};
    \node [gtu block, below left=of ds] (ap1) {Access Point 1};
    \node [gtu block, below right=of ds] (ap2) {Access Point 2};
    
    \node [gtu state, below=of ap1] (sta1) {Station 1};
    \node [gtu state, below=of ap2] (sta2) {Station 2};

    \draw [gtu arrow] (ap1) -- (ds);
    \draw [gtu arrow] (ap2) -- (ds);
    \draw [dashed] (sta1) -- (ap1);
    \draw [dashed] (sta2) -- (ap2);
    
    \node [draw, dashed, fit=(ap1) (sta1), label=left:BSS 1] {};
    \node [draw, dashed, fit=(ap2) (sta2), label=right:BSS 2] {};
    \node [draw, fit=(ds) (ap1) (ap2), label=above:ESS] {};
\end{tikzpicture}
\captionof{figure}{WLAN Architecture}
\end{center}

\textbf{Architecture Elements:}
\begin{itemize}
    \item \keyword{Station (STA)}: Wireless client devices
    \item \keyword{Access Point (AP)}: Central wireless hub
    \item \keyword{Basic Service Set (BSS)}: Single AP coverage area
    \item \keyword{Extended Service Set (ESS)}: Multiple interconnected APs
    \item \keyword{Distribution System (DS)}: Backend network connecting APs
\end{itemize}

\textbf{WLAN Topologies:}
\begin{itemize}
    \item \keyword{Infrastructure Mode}: Centralized through AP
    \item \keyword{Ad-hoc Mode}: Direct peer-to-peer communication
    \item \keyword{Mesh Topology}: Multi-hop wireless connections
\end{itemize}
\end{solutionbox}

\begin{mnemonicbox}
\mnemonic{SABED: Station, Access Point, BSS, ESS, Distribution System}
\end{mnemonicbox}

\questionmarks{4(a OR)}{3}{Define WPAN. List out applications of WPAN.}

\begin{solutionbox}
\textbf{WPAN (Wireless Personal Area Network)} connects devices within personal space (typically 10 meters).

\textbf{Applications of WPAN:}
\begin{itemize}
    \item \keyword{Device Synchronization}: Phone to computer data transfer
    \item \keyword{Audio Streaming}: Wireless headphones, speakers
    \item \keyword{Input Devices}: Wireless keyboard, mouse
    \item \keyword{Healthcare}: Medical sensors, fitness trackers
    \item \keyword{Smart Home}: IoT device control
\end{itemize}

\textbf{Technologies:}
\begin{itemize}
    \item Bluetooth, Zigbee, NFC, infrared
\end{itemize}
\end{solutionbox}

\begin{mnemonicbox}
\mnemonic{DSAHS: Device sync, Streaming, Audio, Healthcare, Smart home}
\end{mnemonicbox}

\questionmarks{4(b OR)}{4}{Explain working of IMAP Protocol.}

\begin{solutionbox}
\textbf{IMAP (Internet Message Access Protocol)} manages email on mail servers.

\textbf{IMAP Working Process:}
\begin{center}
\captionof{table}{IMAP Process}
\begin{tabulary}{\linewidth}{|L|L|L|}
\hline
\textbf{Step} & \textbf{Action} & \textbf{Description} \\ \hline
\textbf{Connection} & Client connects to server & Establish TCP connection on port 143/993 \\ \hline
\textbf{Authentication} & Login credentials & Username/password verification \\ \hline
\textbf{Mailbox Selection} & Choose folder & Select INBOX or other folders \\ \hline
\textbf{Message Operations} & Read/Delete/Flag & Manipulate messages on server \\ \hline
\end{tabulary}
\end{center}

\textbf{IMAP vs POP3:}
\begin{itemize}
    \item \keyword{Server Storage}: Messages remain on server
    \item \keyword{Multi-device Access}: Sync across devices
    \item \keyword{Folder Management}: Server-side folder structure
    \item \keyword{Partial Download}: Headers first, body on demand
\end{itemize}
\end{solutionbox}

\begin{mnemonicbox}
\mnemonic{CAMS: Connection, Authentication, Mailbox, Storage}
\end{mnemonicbox}

\questionmarks{4(c OR)}{7}{Explain Bluetooth technology with a figure of its protocol stack.}

\begin{solutionbox}
\textbf{Bluetooth} is a short-range wireless communication technology for personal area networks.

\textbf{Bluetooth Protocol Stack:}
\begin{center}
\begin{tikzpicture}[node distance=0cm, outer sep=0pt]
    \node [gtu block, minimum width=6cm] (app) {Applications};
    \node [gtu block, minimum width=6cm, below=0.1cm of app] (obex) {OBEX / SDP};
    \node [gtu block, minimum width=6cm, below=0.1cm of obex] (l2cap) {RFCOMM / L2CAP};
    \node [gtu block, minimum width=6cm, below=0.1cm of l2cap] (hci) {Host Controller Interface (HCI)};
    \node [gtu block, minimum width=6cm, below=0.1cm of hci] (lmp) {Link Manager Protocol (LMP)};
    \node [gtu block, minimum width=6cm, below=0.1cm of lmp] (base) {Baseband / Link Controller};
    \node [gtu block, minimum width=6cm, below=0.1cm of base] (radio) {Radio Layer (2.4 GHz)};
\end{tikzpicture}
\captionof{figure}{Bluetooth Protocol Stack}
\end{center}

\textbf{Layer Functions:}
\begin{itemize}
    \item \keyword{Radio Layer}: 2.4 GHz ISM band, frequency hopping
    \item \keyword{Baseband}: Timing, access control, packet formats
    \item \keyword{LMP}: Link establishment, security, power management
    \item \keyword{L2CAP}: Packet segmentation, protocol multiplexing
    \item \keyword{RFCOMM}: Serial port emulation over wireless
    \item \keyword{SDP}: Service discovery protocol
    \item \keyword{Applications}: File transfer, audio streaming, HID
\end{itemize}
\end{solutionbox}

\begin{mnemonicbox}
\mnemonic{RBLSRA: Radio, Baseband, LMP, SDP, RFCOMM, Applications}
\end{mnemonicbox}

\questionmarks{5(a)}{3}{What is 4G? List out Features of 4G.}

\begin{solutionbox}
\textbf{4G (Fourth Generation)} is a mobile communication standard providing high-speed wireless internet.

\textbf{Features of 4G:}
\begin{itemize}
    \item \keyword{High Data Speed}: Up to 100 Mbps mobile, 1 Gbps stationary
    \item \keyword{All-IP Network}: Packet-switched architecture
    \item \keyword{Low Latency}: Reduced delay for real-time applications
    \item \keyword{Quality of Service}: Guaranteed service levels
    \item \keyword{Global Roaming}: Worldwide compatibility
\end{itemize}

\textbf{Technologies:}
\begin{itemize}
    \item LTE (Long Term Evolution), WiMAX
\end{itemize}
\end{solutionbox}

\begin{mnemonicbox}
\mnemonic{HALQG: High-speed, All-IP, Low latency, QoS, Global roaming}
\end{mnemonicbox}

\questionmarks{5(b)}{4}{Explain Centralized Computing.}

\begin{solutionbox}
\textbf{Centralized Computing} processes all data and applications on a central server.

\textbf{Architecture:}
\begin{center}
\begin{tikzpicture}[node distance=1.5cm, auto]
    \node [gtu block] (server) {Central Server};
    \node [gtu state, below left=of server] (t1) {Terminal 1};
    \node [gtu state, below=of server] (t2) {Terminal 2};
    \node [gtu state, below right=of server] (t3) {Terminal 3};

    \draw [gtu arrow] (t1) -- (server);
    \draw [gtu arrow] (t2) -- (server);
    \draw [gtu arrow] (t3) -- (server);
    
    \node [right=0.2cm of server, align=left, font=\scriptsize] {Processing, Storage, Apps};
\end{tikzpicture}
\captionof{figure}{Centralized Computing}
\end{center}

\textbf{Characteristics:}
\begin{itemize}
    \item \keyword{Single Point of Control}: All processing at central location
    \item \keyword{Thin Clients}: Minimal local processing capability
    \item \keyword{Shared Resources}: CPU, memory, storage centrally managed
    \item \keyword{Network Dependent}: Requires reliable network connectivity
\end{itemize}

\textbf{Advantages:}
\begin{itemize}
    \item \keyword{Security}: Centralized data protection
    \item \keyword{Management}: Easier system administration
    \item \keyword{Cost}: Lower client-side hardware costs
\end{itemize}
\end{solutionbox}

\begin{mnemonicbox}
\mnemonic{SSNG: Single control, Shared resources, Network dependent, Greater security}
\end{mnemonicbox}

\questionmarks{5(c)}{7}{What is ipv4 addressing scheme? Explain with a neat and clean diagram with its working.}

\begin{solutionbox}
\textbf{IPv4 (Internet Protocol version 4)} uses 32-bit addresses for network identification.

\textbf{IPv4 Address Structure:}
\begin{center}
\begin{tikzpicture}[node distance=0cm, outer sep=0pt]
    \node [gtu block, minimum width=6cm] (net) {Network Address};
    \node [gtu block, minimum width=6cm, right=0cm of net] (host) {Host Address};
    \node [fit=(net)(host), label=above:32 Bits Total] {};
\end{tikzpicture}
\captionof{figure}{IPv4 Structure}
\end{center}

\textbf{IPv4 Address Classes:}
\begin{center}
\captionof{table}{Address Classes}
\begin{tabulary}{\linewidth}{|C|L|C|C|L|}
\hline
\textbf{Class} & \textbf{Range} & \textbf{Net Bits} & \textbf{Host Bits} & \textbf{Subnet Mask} \\ \hline
\textbf{A} & 1-126 & 8 & 24 & 255.0.0.0 \\ \hline
\textbf{B} & 128-191 & 16 & 16 & 255.255.0.0 \\ \hline
\textbf{C} & 192-223 & 24 & 8 & 255.255.255.0 \\ \hline
\textbf{D} & 224-239 & Multicast & - & - \\ \hline
\textbf{E} & 240-255 & Exp. & - & - \\ \hline
\end{tabulary}
\end{center}

\textbf{IPv4 Packet Header:}
\begin{center}
\begin{tikzpicture}[node distance=0cm, auto]
    \node [gtu block, minimum width=1cm] (ver) {Ver};
    \node [gtu block, minimum width=1cm, right=0cm of ver] (ihl) {IHL};
    \node [gtu block, minimum width=2cm, right=0cm of ihl] (tos) {Service};
    \node [gtu block, minimum width=4cm, right=0cm of tos] (len) {Total Length};
    
    \node [gtu block, minimum width=4cm, below=0cm of ver.south west, anchor=north west] (id) {Identification};
    \node [gtu block, minimum width=1cm, right=0cm of id] (flags) {Flg};
    \node [gtu block, minimum width=3cm, right=0cm of flags] (off) {Frag Offset};
    
    \node [gtu block, minimum width=2cm, below=0cm of id.south west, anchor=north west] (ttl) {TTL};
    \node [gtu block, minimum width=2cm, right=0cm of ttl] (proto) {Proto};
    \node [gtu block, minimum width=4cm, right=0cm of proto] (check) {Checksum};
    
    \node [gtu block, minimum width=8cm, below=0cm of ttl.south west, anchor=north west] (src) {Source IP};
    \node [gtu block, minimum width=8cm, below=0cm of src.south west, anchor=north west] (dst) {Destination IP};
\end{tikzpicture}
\captionof{figure}{IPv4 Header}
\end{center}
\end{solutionbox}

\begin{mnemonicbox}
\mnemonic{Class A-E, Header: Version IHL TOS Length ID Flags TTL Protocol Checksum Source Dest}
\end{mnemonicbox}

\questionmarks{5(a OR)}{3}{What is 5G? List out Features of 5G.}

\begin{solutionbox}
\textbf{5G (Fifth Generation)} is the latest mobile communication standard with enhanced capabilities.

\textbf{Features of 5G:}
\begin{itemize}
    \item \keyword{Ultra-High Speed}: Up to 10 Gbps data rates
    \item \keyword{Ultra-Low Latency}: Less than 1ms response time
    \item \keyword{Massive Connectivity}: 1 million devices per km\textsuperscript{2}
    \item \keyword{Network Slicing}: Virtual dedicated networks
    \item \keyword{Enhanced Mobile Broadband}: Improved user experience
\end{itemize}

\textbf{Key Technologies:}
\begin{itemize}
    \item Millimeter wave, Massive MIMO, Beamforming
\end{itemize}
\end{solutionbox}

\begin{mnemonicbox}
\mnemonic{UUMNE: Ultra-speed, Ultra-low latency, Massive connectivity, Network slicing, Enhanced broadband}
\end{mnemonicbox}

\questionmarks{5(b OR)}{4}{Explain Distributed Computing}

\begin{solutionbox}
\textbf{Distributed Computing} spreads processing across multiple interconnected computers.

\textbf{Architecture:}
\begin{center}
\begin{tikzpicture}[node distance=1.5cm, auto]
    \node [gtu block, ellipse, minimum height=1cm] (net) {Network};
    \node [gtu state, above=of net] (n1) {Node 1};
    \node [gtu state, left=of net] (n2) {Node 2};
    \node [gtu state, right=of net] (n3) {Node 3};
    \node [gtu state, below=of net] (n4) {Node 4};
    
    \draw [dashed] (n1) -- (net);
    \draw [dashed] (n2) -- (net);
    \draw [dashed] (n3) -- (net);
    \draw [dashed] (n4) -- (net);
\end{tikzpicture}
\captionof{figure}{Distributed System}
\end{center}

\textbf{Characteristics:}
\begin{itemize}
    \item \keyword{Resource Sharing}: Distributed processing and storage
    \item \keyword{Scalability}: Add more nodes to increase capacity
    \item \keyword{Fault Tolerance}: System continues if some nodes fail
    \item \keyword{Location Transparency}: Users unaware of resource locations
\end{itemize}

\textbf{Advantages:}
\begin{itemize}
    \item \keyword{Reliability}: No single point of failure
    \item \keyword{Performance}: Parallel processing capabilities
    \item \keyword{Cost-effectiveness}: Use commodity hardware
\end{itemize}
\end{solutionbox}

\begin{mnemonicbox}
\mnemonic{RSFL: Resource sharing, Scalability, Fault tolerance, Location transparency}
\end{mnemonicbox}

\questionmarks{5(c OR)}{7}{Explain Data Link Layer Protocol.}

\begin{solutionbox}
\textbf{Data Link Layer} provides reliable data transfer between adjacent network nodes.

\textbf{Functions:}
\begin{itemize}
    \item \keyword{Framing}: Organize bits into frames
    \item \keyword{Error Detection}: Identify transmission errors
    \item \keyword{Error Correction}: Fix detected errors
    \item \keyword{Flow Control}: Manage data transmission rate
    \item \keyword{Access Control}: Coordinate shared media access
\end{itemize}

\textbf{Frame Structure:}
\begin{center}
\begin{tikzpicture}[node distance=0cm, auto]
    \node [gtu block, minimum width=2cm] (start) {Start};
    \node [gtu block, minimum width=2cm, right=0cm of start] (addr) {Address};
    \node [gtu block, minimum width=2cm, right=0cm of addr] (ctrl) {Control};
    \node [gtu block, minimum width=3cm, right=0cm of ctrl] (data) {Data};
    \node [gtu block, minimum width=2cm, right=0cm of data] (fcs) {FCS (CRC)};
    \node [gtu block, minimum width=2cm, right=0cm of fcs] (stop) {Stop};
\end{tikzpicture}
\captionof{figure}{Frame Structure}
\end{center}

\textbf{Error Detection Methods:}
\begin{itemize}
    \item Parity Check
    \item Checksum
    \item CRC (Cyclic Redundancy Check)
\end{itemize}

\textbf{Flow Control Protocols:}
\begin{itemize}
    \item Stop-and-Wait
    \item Sliding Window
    \item Go-Back-N ARQ
    \item Selective Repeat ARQ
\end{itemize}

\textbf{Working Process:}
\begin{center}
\begin{tikzpicture}[node distance=3cm, auto]
    \node [gtu state] (sender) {Sender};
    \node [gtu state, right=of sender] (receiver) {Receiver};

    \draw [gtu arrow] (sender) to[bend left=15] node [above] {Data Frame} (receiver);
    \draw [gtu arrow] (receiver) to[bend left=15] node [below] {ACK} (sender);
    
    \path (sender) -- node [below=1cm] {Reliable Transfer} (receiver);
\end{tikzpicture}
\captionof{figure}{Data Link Protocol}
\end{center}
\end{solutionbox}

\begin{mnemonicbox}
\mnemonic{FECFA: Framing, Error detection, Correction, Flow control, Access control}
\end{mnemonicbox}

\end{document}
