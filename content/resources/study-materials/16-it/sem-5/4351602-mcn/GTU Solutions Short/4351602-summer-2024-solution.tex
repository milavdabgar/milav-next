\documentclass{article}
% Adjust the relative path to point to the latex-templates directory

% content/resources/templates/preamble.tex
\usepackage[margin=0.6in]{geometry}
\author{Milav Dabgar}
\usepackage{amsmath,amssymb,amsthm}
\usepackage{booktabs}
\usepackage{multirow}
\usepackage{xcolor}
\usepackage{tcolorbox}
\tcbuselibrary{breakable,skins}
\usepackage[colorlinks=true,linkcolor=blue]{hyperref}
\usepackage{titlesec}
\usepackage{enumitem}
\usepackage{tikz}
\usepackage{pgfplots}
\usepackage{circuitikz}
\usepackage[version=4]{mhchem}
\usepackage{longtable}
\usepackage{array}
\usepackage{float}
\usepackage{caption}
\usepackage{listings}

\lstset{
  basicstyle=\small\ttfamily,
  breaklines=true,
  breakatwhitespace=false,
  postbreak=\mbox{\textcolor{red}{$\hookrightarrow$}\space},
  float=false,
  numbers=left,
  numberstyle=\tiny\color{gray},
  numbersep=10pt,
  xleftmargin=2em,
  keywordstyle=\color{blue},
  commentstyle=\color{green!60!black},
  stringstyle=\color{purple},
  backgroundcolor=\color{gray!5},
  showstringspaces=false,
  tabsize=2,
  captionpos=b,
  keepspaces=true,
  columns=flexible
}

\pgfplotsset{compat=1.18}
\usetikzlibrary{shapes,arrows,positioning,calc,patterns,decorations.pathmorphing,decorations.markings,arrows.meta}

% Color scheme
\definecolor{headcolor}{RGB}{0,102,204}
\definecolor{keycolor}{RGB}{220,20,60}
\definecolor{solutioncolor}{RGB}{34,139,34}
\definecolor{mnemoniccolor}{RGB}{148,0,211}
\definecolor{codecolor}{RGB}{0,0,100}

% Spacing
\setlength{\parskip}{3pt}
\setlist[itemize]{nosep}
\setlist[enumerate]{nosep}

% Title formatting
\titleformat{\section}{\Large\bfseries\color{headcolor}}{\thesection}{1em}{}
\titleformat{\subsection}{\large\bfseries\color{headcolor}}{\thesubsection}{1em}{}

% Pandoc tightlist compatibility
\providecommand{\tightlist}{%
  \setlength{\itemsep}{0pt}\setlength{\parskip}{0pt}}

% Pandoc longtable compatibility
\newcounter{none}
\def\thenone{}


% content/resources/templates/english-boxes.tex
% This file is currently empty - it exists to maintain consistency with the import structure.
% Add custom environments here if needed in the future.


% Custom commands for GTU solutions
% This file defines semantic commands for consistent formatting

% Question command with automatic formatting
\newcommand{\question}[2]{%
  \section*{Question #1}%
  \textbf{#2}%
}

% OR question variant
\newcommand{\questionor}[2]{%
  \section*{Question #1 OR}%
  \textbf{#2}%
}

% Proper table environment with caption
\newenvironment{answertable}[1]{%
  \begin{table}[htbp]
  \centering
  \caption{#1}
}{%
  \end{table}
}

% Proper figure environment for diagrams
\newenvironment{answerdiagram}[1]{%
  \begin{figure}[htbp]
  \centering
  \caption{#1}
}{%
  \end{figure}
}

% Semantic markup for key terms
\newcommand{\keyword}[1]{\textbf{#1}}
\newcommand{\code}[1]{\texttt{#1}}
\newcommand{\classname}[1]{\texttt{#1}}
\newcommand{\methodname}[1]{\texttt{#1}}

% Proper quotation marks
\newcommand{\mnemonic}[1]{``#1''}

\usetikzlibrary{fit}

\title{Mobile Computing and Networks (4351602) - Summer 2024 Solution}
\date{May 18, 2024}

\begin{document}
\maketitle

\questionmarks{1(a)}{3}{Define Peer to Peer network}

\begin{solutionbox}
A Peer-to-Peer (P2P) network is a distributed network architecture where each node (peer) acts as both client and server, sharing resources directly without centralized control.

\begin{center}
\captionof{table}{Peer to Peer Network Features}
\begin{tabulary}{\linewidth}{|L|L|}
\hline
\textbf{Aspect} & \textbf{Description} \\ \hline
\textbf{Structure} & Decentralized network \\ \hline
\textbf{Role} & Each peer is client and server \\ \hline
\textbf{Control} & No central authority \\ \hline
\textbf{Examples} & BitTorrent, Skype \\ \hline
\end{tabulary}
\end{center}
\end{solutionbox}

\begin{mnemonicbox}
\mnemonic{Peers Share Equally}
\end{mnemonicbox}

\questionmarks{1(b)}{4}{Compare SMTP, POP and IMAP}

\begin{solutionbox}
Email protocols serve different purposes in email communication system.

\begin{center}
\captionof{table}{Comparison of Email Protocols}
\begin{tabulary}{\linewidth}{|L|L|L|L|}
\hline
\textbf{Feature} & \textbf{SMTP} & \textbf{POP3} & \textbf{IMAP} \\ \hline
\textbf{Purpose} & Send emails & Download emails & Access emails \\ \hline
\textbf{Port} & 25, 587 & 110, 995 & 143, 993 \\ \hline
\textbf{Storage} & Server forwards & Local storage & Server storage \\ \hline
\textbf{Access} & One-way sending & Single device & Multiple devices \\ \hline
\end{tabulary}
\end{center}
\end{solutionbox}

\begin{mnemonicbox}
\mnemonic{Send-Pop-Internet Mail Access}
\end{mnemonicbox}

\questionmarks{1(c)}{7}{Illustrate OSI model with responsibilities of each layer}

\begin{solutionbox}
The OSI (Open Systems Interconnection) model has seven layers, each with specific responsibilities for network communication.

\begin{center}
\begin{tikzpicture}[node distance=0.8cm, auto]
    \node [gtu block, minimum width=6cm] (app) {7. Application Layer};
    \node [gtu block, minimum width=6cm, below=of app] (pres) {6. Presentation Layer};
    \node [gtu block, minimum width=6cm, below=of pres] (sess) {5. Session Layer};
    \node [gtu block, minimum width=6cm, below=of sess] (trans) {4. Transport Layer};
    \node [gtu block, minimum width=6cm, below=of trans] (net) {3. Network Layer};
    \node [gtu block, minimum width=6cm, below=of net] (data) {2. Data Link Layer};
    \node [gtu block, minimum width=6cm, below=of data] (phys) {1. Physical Layer};
    
    \draw [gtu arrow] (app) -- (pres);
    \draw [gtu arrow] (pres) -- (sess);
    \draw [gtu arrow] (sess) -- (trans);
    \draw [gtu arrow] (trans) -- (net);
    \draw [gtu arrow] (net) -- (data);
    \draw [gtu arrow] (data) -- (phys);
\end{tikzpicture}
\captionof{figure}{OSI Model Layers}
\end{center}

\begin{center}
\captionof{table}{OSI Layers and Responsibilities}
\begin{tabulary}{\linewidth}{|C|L|L|}
\hline
\textbf{Layer} & \textbf{Name} & \textbf{Responsibilities} \\ \hline
\textbf{7} & Application & User interface, network services \\ \hline
\textbf{6} & Presentation & Data encryption, compression \\ \hline
\textbf{5} & Session & Session management, dialogue control \\ \hline
\textbf{4} & Transport & End-to-end delivery, error control \\ \hline
\textbf{3} & Network & Routing, logical addressing \\ \hline
\textbf{2} & Data Link & Frame formatting, error detection \\ \hline
\textbf{1} & Physical & Bit transmission, hardware \\ \hline
\end{tabulary}
\end{center}

\textbf{Key Points:}
\begin{itemize}
    \item \keyword{Application Layer}: Provides network services to applications
    \item \keyword{Transport Layer}: Ensures reliable data delivery
    \item \keyword{Network Layer}: Handles routing between networks
\end{itemize}
\end{solutionbox}

\begin{mnemonicbox}
\mnemonic{All People Seem To Need Data Processing}
\end{mnemonicbox}

\questionmarks{1(c OR)}{7}{Compare the TCP/IP model with OSI model}

\begin{solutionbox}
TCP/IP and OSI models are network architecture frameworks with different layer structures.

\begin{center}
\begin{tikzpicture}[node distance=0.5cm]
    % OSI Stack
    \node [font=\bfseries] (osi_label) {OSI Model};
    \node [gtu block, minimum width=3.5cm, below=0.2cm of osi_label] (o7) {Application};
    \node [gtu block, minimum width=3.5cm, below=0cm of o7] (o6) {Presentation};
    \node [gtu block, minimum width=3.5cm, below=0cm of o6] (o5) {Session};
    \node [gtu block, minimum width=3.5cm, below=0cm of o5] (o4) {Transport};
    \node [gtu block, minimum width=3.5cm, below=0cm of o4] (o3) {Network};
    \node [gtu block, minimum width=3.5cm, below=0cm of o3] (o2) {Data Link};
    \node [gtu block, minimum width=3.5cm, below=0cm of o2] (o1) {Physical};

    % TCP/IP Stack
    \node [font=\bfseries, right=5cm of osi_label] (tcp_label) {TCP/IP Model};
    \node [gtu block, minimum width=3.5cm, minimum height=3cm, below=0.2cm of tcp_label] (t4) {Application};
    \node [gtu block, minimum width=3.5cm, below=0.2cm of t4, yshift=-0.5cm] (t3) {Transport}; 
    \node [gtu block, minimum width=3.5cm, below=0cm of t3] (t2) {Internet};
    \node [gtu block, minimum width=3.5cm, minimum height=1.8cm, below=0cm of t2] (t1) {Network Access};

    % Mapping lines
    \draw [dashed, ->] (o7.east) -- (t4.west);
    \draw [dashed, ->] (o6.east) -- (t4.west);
    \draw [dashed, ->] (o5.east) -- (t4.west);
    \draw [dashed, ->] (o4.east) -- (t3.west);
    \draw [dashed, ->] (o3.east) -- (t2.west);
    \draw [dashed, ->] (o2.east) -- (t1.west);
    \draw [dashed, ->] (o1.east) -- (t1.west);
\end{tikzpicture}
\captionof{figure}{OSI vs TCP/IP Model Comparison}
\end{center}

\begin{center}
\captionof{table}{OSI vs TCP/IP Model}
\begin{tabulary}{\linewidth}{|L|L|L|}
\hline
\textbf{Aspect} & \textbf{OSI Model} & \textbf{TCP/IP Model} \\ \hline
\textbf{Layers} & 7 layers & 4 layers \\ \hline
\textbf{Development} & Theoretical & Practical \\ \hline
\textbf{Usage} & Reference model & Internet standard \\ \hline
\textbf{Complexity} & More detailed & Simplified \\ \hline
\end{tabulary}
\end{center}

\textbf{Key Points:}
\begin{itemize}
    \item \keyword{OSI}: Theoretical framework with detailed separation
    \item \keyword{TCP/IP}: Practical implementation for internet
    \item \keyword{Mapping}: Top 3 OSI layers = Application layer in TCP/IP
\end{itemize}
\end{solutionbox}

\begin{mnemonicbox}
\mnemonic{OSI Seven, TCP Four}
\end{mnemonicbox}

\questionmarks{2(a)}{3}{Explain Network Address Translation (NAT)}

\begin{solutionbox}
NAT translates private IP addresses to public IP addresses, enabling multiple devices to share a single public IP.

\begin{center}
\begin{tikzpicture}[auto, node distance=2cm]
    \node [gtu block] (priv1) {192.168.1.10};
    \node [gtu block, below=0.5cm of priv1] (priv2) {192.168.1.20};
    \node [gtu block, below=0.5cm of priv2] (priv3) {192.168.1.30};
    
    \node [gtu state, right=3cm of priv2, align=center] (router) {NAT Router\\(203.0.113.1)};
    \node [gtu block, right=3cm of router] (internet) {Internet Server};
    
    \draw [gtu arrow] (priv1) -- (router);
    \draw [gtu arrow] (priv2) -- (router);
    \draw [gtu arrow] (priv3) -- (router);
    \draw [gtu arrow] (router) -- (internet);
\end{tikzpicture}
\captionof{figure}{Network Address Translation}
\end{center}

\textbf{Key Points:}
\begin{itemize}
    \item \keyword{Purpose}: IP address translation between networks
    \item \keyword{Benefit}: Conserves public IP addresses
    \item \keyword{Security}: Hides internal network structure
\end{itemize}
\end{solutionbox}

\begin{mnemonicbox}
\mnemonic{Network Address Translation}
\end{mnemonicbox}

\questionmarks{2(b)}{4}{Define Subnetting and Supernetting}

\begin{solutionbox}
Subnetting and Supernetting are IP addressing techniques for efficient network management.

\begin{center}
\captionof{table}{Subnetting vs Supernetting}
\begin{tabulary}{\linewidth}{|L|L|L|}
\hline
\textbf{Technique} & \textbf{Definition} & \textbf{Purpose} \\ \hline
\textbf{Subnetting} & Dividing network into smaller subnets & Better organization \\ \hline
\textbf{Supernetting} & Combining multiple networks & Route aggregation \\ \hline
\end{tabulary}
\end{center}

\textbf{Key Points:}
\begin{itemize}
    \item \keyword{Subnetting}: Increases network bits, reduces host bits
    \item \keyword{Supernetting}: Decreases network bits, increases routing efficiency
    \item \keyword{CIDR}: Classless Inter-Domain Routing enables both
\end{itemize}
\end{solutionbox}

\begin{mnemonicbox}
\mnemonic{Sub-divides, Super-combines}
\end{mnemonicbox}

\questionmarks{2(c)}{7}{Demonstrate Classful and Classless notation addressing scheme of IPv4}

\begin{solutionbox}
IPv4 addressing uses classful and classless schemes for network identification.

\begin{center}
\captionof{table}{Classful Addressing}
\begin{tabulary}{\linewidth}{|C|C|L|L|L|}
\hline
\textbf{Class} & \textbf{Range} & \textbf{Default Mask} & \textbf{Networks} & \textbf{Hosts} \\ \hline
\textbf{A} & 1-126 & /8 (255.0.0.0) & 126 & 16M \\ \hline
\textbf{B} & 128-191 & /16 (255.255.0.0) & 16K & 65K \\ \hline
\textbf{C} & 192-223 & /24 (255.255.255.0) & 2M & 254 \\ \hline
\end{tabulary}
\end{center}

\textbf{Classless (CIDR) Examples:}
\begin{itemize}
    \item \textbf{192.168.1.0/25}: 128 hosts
    \item \textbf{10.0.0.0/16}: 65,536 hosts
    \item \textbf{172.16.0.0/20}: 4,096 hosts
\end{itemize}

\textbf{Key Points:}
\begin{itemize}
    \item \keyword{Classful}: Fixed network/host boundaries
    \item \keyword{Classless}: Variable Length Subnet Mask (VLSM)
    \item \keyword{CIDR}: More efficient address allocation
\end{itemize}
\end{solutionbox}

\begin{mnemonicbox}
\mnemonic{Class-Fixed, CIDR-Flexible}
\end{mnemonicbox}

\questionmarks{2(a OR)}{3}{Discuss goals of mobile IP}

\begin{solutionbox}
Mobile IP enables seamless connectivity for mobile devices across different networks.

\textbf{Key Points:}
\begin{itemize}
    \item \keyword{Transparency}: Applications unaware of mobility
    \item \keyword{Compatibility}: Works with existing protocols
    \item \keyword{Efficiency}: Minimal routing overhead
\end{itemize}
\end{solutionbox}

\begin{mnemonicbox}
\mnemonic{Transparent Compatible Efficient}
\end{mnemonicbox}

\questionmarks{2(b OR)}{4}{Define ARP and RARP}

\begin{solutionbox}
ARP and RARP are address resolution protocols for mapping between different address types.

\begin{center}
\captionof{table}{ARP vs RARP}
\begin{tabulary}{\linewidth}{|L|L|L|L|}
\hline
\textbf{Protocol} & \textbf{Full Name} & \textbf{Purpose} & \textbf{Direction} \\ \hline
\textbf{ARP} & Address Resolution Protocol & IP to MAC mapping & Logical to Physical \\ \hline
\textbf{RARP} & Reverse ARP & MAC to IP mapping & Physical to Logical \\ \hline
\end{tabulary}
\end{center}
\end{solutionbox}

\begin{mnemonicbox}
\mnemonic{ARP-asks, RARP-reverses}
\end{mnemonicbox}

\questionmarks{2(c OR)}{7}{Demonstrate Stop and Wait, Stop and Wait ARQ data link layer protocols}

\begin{solutionbox}
These protocols ensure reliable data transmission at the data link layer.

\begin{center}
\begin{tikzpicture}[node distance=2.5cm, auto]
    \node [gtu state] (s) {Sender};
    \node [gtu state, right=4cm of s] (r) {Receiver};
    
    \draw [thick] (s) -- ++(0, -5);
    \draw [thick] (r) -- ++(0, -5);
    
    \draw [->, thick] ($(s)+(0,-1)$) -- node [above, sloped] {Frame 0} ($(r)+(0,-1.5)$);
    \draw [->, dashed] ($(r)+(0,-2)$) -- node [above, sloped] {ACK 0} ($(s)+(0,-2.5)$);
    
    \draw [->, thick] ($(s)+(0,-3)$) -- node [above, sloped] {Frame 1} ($(r)+(0,-3.5)$);
    \draw [->, dashed] ($(r)+(0,-4)$) -- node [above, sloped] {ACK 1} ($(s)+(0,-4.5)$);
\end{tikzpicture}
\captionof{figure}{Stop and Wait Protocol Interaction}
\end{center}

\begin{center}
\captionof{table}{Protocol Comparison}
\begin{tabulary}{\linewidth}{|L|L|L|L|}
\hline
\textbf{Protocol} & \textbf{Error Detection} & \textbf{Efficiency} & \textbf{Complexity} \\ \hline
\textbf{Stop and Wait} & Basic & Low & Simple \\ \hline
\textbf{Stop and Wait ARQ} & Advanced & Medium & Moderate \\ \hline
\end{tabulary}
\end{center}

\textbf{Key Points:}
\begin{itemize}
    \item \keyword{Stop and Wait}: Send frame, wait for acknowledgment
    \item \keyword{ARQ}: Automatic Repeat reQuest on errors
    \item \keyword{Timeout}: Resend if no acknowledgment received
\end{itemize}
\end{solutionbox}

\begin{mnemonicbox}
\mnemonic{Stop-Wait-Acknowledge}
\end{mnemonicbox}

\questionmarks{3(a)}{3}{Explain Wireless networks}

\begin{solutionbox}
Wireless networks use radio waves for communication without physical connections.

\textbf{Key Points:}
\begin{itemize}
    \item \keyword{Technology}: Radio frequency transmission
    \item \keyword{Types}: WiFi, Bluetooth, Cellular
    \item \keyword{Benefits}: Mobility, easy installation
\end{itemize}
\end{solutionbox}

\begin{mnemonicbox}
\mnemonic{Wireless-Radio-Mobile}
\end{mnemonicbox}

\questionmarks{3(b)}{4}{Define Communication Middleware in mobile computing}

\begin{solutionbox}
Communication middleware provides abstraction layer for mobile application communication.

\begin{center}
\captionof{table}{Communication Middleware}
\begin{tabulary}{\linewidth}{|L|L|}
\hline
\textbf{Aspect} & \textbf{Description} \\ \hline
\textbf{Purpose} & Simplify communication \\ \hline
\textbf{Location} & Between app and network \\ \hline
\textbf{Features} & Protocol handling, data conversion \\ \hline
\textbf{Examples} & CORBA, RMI \\ \hline
\end{tabulary}
\end{center}
\end{solutionbox}

\begin{mnemonicbox}
\mnemonic{Middle-Communication-Layer}
\end{mnemonicbox}

\questionmarks{3(c)}{7}{Discuss the architecture of Mobile Computing}

\begin{solutionbox}
Mobile computing architecture consists of multiple interconnected components supporting mobile applications.

\begin{center}
\begin{tikzpicture}[node distance=0.5cm, auto]
    \node [gtu block, align=center] (mobile) {Mobile\\Device};
    \node [gtu block, right=of mobile] (wireless) {Wireless\\Network};
    \node [gtu block, right=of wireless] (bs) {Base\\Station};
    \node [gtu block, right=of bs] (mss) {MSS};
    \node [gtu block, right=of mss, align=center] (fixed) {Fixed\\Network};
    \node [gtu block, right=of fixed] (db) {DB};
    
    \draw [gtu arrow] (mobile) -- (wireless);
    \draw [gtu arrow] (wireless) -- (bs);
    \draw [gtu arrow] (bs) -- (mss);
    \draw [gtu arrow] (mss) -- (fixed);
    \draw [gtu arrow] (fixed) -- (db);
\end{tikzpicture}
\captionof{figure}{Mobile Computing Architecture}
\end{center}

\begin{center}
\captionof{table}{Architecture Components}
\begin{tabulary}{\linewidth}{|L|L|}
\hline
\textbf{Component} & \textbf{Function} \\ \hline
\textbf{Mobile Device} & User interface, local processing \\ \hline
\textbf{Wireless Network} & Radio communication \\ \hline
\textbf{Base Station} & Network access point \\ \hline
\textbf{MSS} & Mobility management \\ \hline
\textbf{Fixed Network} & Backbone infrastructure \\ \hline
\end{tabulary}
\end{center}

\textbf{Key Points:}
\begin{itemize}
    \item \keyword{Three-tier}: Mobile device, wireless network, fixed network
    \item \keyword{Mobility Support}: Handoff management
    \item \keyword{Data Management}: Caching and synchronization
\end{itemize}
\end{solutionbox}

\begin{mnemonicbox}
\mnemonic{Mobile-Wireless-Fixed}
\end{mnemonicbox}

\questionmarks{3(a OR)}{3}{Demonstrate ad-hoc networks}

\begin{solutionbox}
Ad-hoc networks are self-organizing wireless networks without fixed infrastructure.

\textbf{Key Points:}
\begin{itemize}
    \item \keyword{Structure}: Peer-to-peer topology
    \item \keyword{Routing}: Dynamic route discovery
    \item \keyword{Applications}: Emergency, military
\end{itemize}
\end{solutionbox}

\begin{mnemonicbox}
\mnemonic{Ad-hoc-Self-Organizing}
\end{mnemonicbox}

\questionmarks{3(b OR)}{4}{Define Transaction Processing Middleware in mobile computing}

\begin{solutionbox}
Transaction processing middleware ensures ACID properties in mobile database transactions.

\begin{center}
\captionof{table}{ACID Properties}
\begin{tabulary}{\linewidth}{|L|L|}
\hline
\textbf{Property} & \textbf{Description} \\ \hline
\textbf{Atomicity} & All or nothing execution \\ \hline
\textbf{Consistency} & Database integrity maintained \\ \hline
\textbf{Isolation} & Concurrent transaction separation \\ \hline
\textbf{Durability} & Permanent transaction effects \\ \hline
\end{tabulary}
\end{center}
\end{solutionbox}

\begin{mnemonicbox}
\mnemonic{ACID-Properties}
\end{mnemonicbox}

\questionmarks{3(c OR)}{7}{Discuss the applications and services of mobile computing}

\begin{solutionbox}
Mobile computing enables diverse applications across multiple domains.

\begin{center}
\captionof{table}{Applications and Services}
\begin{tabulary}{\linewidth}{|L|L|L|}
\hline
\textbf{Domain} & \textbf{Applications} & \textbf{Services} \\ \hline
\textbf{Business} & CRM, ERP & Data synchronization \\ \hline
\textbf{Healthcare} & Patient monitoring & Remote diagnosis \\ \hline
\textbf{Education} & E-learning & Content delivery \\ \hline
\textbf{Entertainment} & Gaming, streaming & Media services \\ \hline
\textbf{Navigation} & GPS, maps & Location services \\ \hline
\end{tabulary}
\end{center}

\textbf{Key Points:}
\begin{itemize}
    \item \keyword{Location-based}: GPS navigation, geo-fencing
    \item \keyword{Communication}: Email, messaging, video calls
    \item \keyword{Commerce}: Mobile banking, shopping
\end{itemize}
\end{solutionbox}

\begin{mnemonicbox}
\mnemonic{Business-Health-Education-Entertainment}
\end{mnemonicbox}

\questionmarks{4(a)}{3}{Describe Indirect TCP in mobile computing}

\begin{solutionbox}
Indirect TCP splits TCP connection to handle mobile host mobility efficiently.

\begin{center}
\begin{tikzpicture}[auto, node distance=2.5cm]
    \node [gtu block] (fh) {Fixed Host};
    \node [gtu block, right=of fh] (bs) {Base Station};
    \node [gtu block, right=of bs] (mh) {Mobile Host};
    
    \path [gtu arrow] (fh) -- node [above] {TCP Connection 1} (bs);
    \path [gtu arrow] (bs) -- node [above] {TCP Connection 2} (mh);
    
    \node [below=0.2cm of bs, font=\small\itshape] {Acts as Proxy};
\end{tikzpicture}
\captionof{figure}{Indirect TCP Split Connection}
\end{center}

\textbf{Key Points:}
\begin{itemize}
    \item \keyword{Split Connection}: Two separate TCP connections
    \item \keyword{Base Station}: Acts as proxy
    \item \keyword{Advantage}: Faster handoff
\end{itemize}
\end{solutionbox}

\begin{mnemonicbox}
\mnemonic{Indirect-Split-Proxy}
\end{mnemonicbox}

\questionmarks{4(b)}{4}{Explain the steps of the packet delivery in Mobile IP}

\begin{solutionbox}
Mobile IP packet delivery involves registration, tunneling, and delivery steps.

\textbf{Steps:}
\begin{enumerate}
    \item \keyword{Registration}: Mobile node registers with home agent
    \item \keyword{Tunneling}: Home agent creates tunnel to foreign agent
    \item \keyword{Encapsulation}: Original packet wrapped in new header
    \item \keyword{Delivery}: Foreign agent delivers to mobile node
\end{enumerate}
\end{solutionbox}

\begin{mnemonicbox}
\mnemonic{Register-Tunnel-Encapsulate-Deliver}
\end{mnemonicbox}

\questionmarks{4(c)}{7}{Write following three processes of mobile IP: (1) Registration (2) Tunneling (3) Encapsulation}

\begin{solutionbox}
\textbf{1. Registration Process:}
\begin{itemize}
    \item Mobile node discovers foreign agent
    \item Registers care-of address with home agent
    \item Authentication and binding update
\end{itemize}

\textbf{2. Tunneling Process:}
\begin{itemize}
    \item Home agent creates virtual tunnel
    \item Packets forwarded through tunnel
    \item Maintains end-to-end connectivity
\end{itemize}

\textbf{3. Encapsulation Process:}
\begin{itemize}
    \item Original packet becomes payload
    \item New IP header added with care-of address
    \item Packet delivered to foreign network
\end{itemize}

\begin{center}
\begin{tikzpicture}[node distance=1.5cm, auto]
    \node [gtu block] (orig) {Original Packet};
    \node [gtu process, right=of orig] (encap) {Encapsulation};
    \node [gtu block, right=of encap] (tunnel) {Tunneled Packet};
    \node [gtu process, right=of tunnel] (del) {Delivery};
    
    \draw [gtu arrow] (orig) -- (encap);
    \draw [gtu arrow] (encap) -- (tunnel);
    \draw [gtu arrow] (tunnel) -- (del);
\end{tikzpicture}
\captionof{figure}{Mobile IP Encapsulation Flow}
\end{center}

\textbf{Key Points:}
\begin{itemize}
    \item \keyword{Registration}: Location update mechanism
    \item \keyword{Tunneling}: Virtual connection establishment
    \item \keyword{Encapsulation}: Packet wrapping technique
\end{itemize}
\end{solutionbox}

\begin{mnemonicbox}
\mnemonic{Register-Tunnel-Encapsulate}
\end{mnemonicbox}

\questionmarks{4(a OR)}{3}{Describe Snooping TCP in mobile computing}

\begin{solutionbox}
Snooping TCP improves performance by caching and monitoring TCP segments at base station.

\textbf{Key Points:}
\begin{itemize}
    \item \keyword{Local Retransmission}: Base station handles losses
    \item \keyword{Buffer Management}: Caches unacknowledged segments
    \item \keyword{Transparency}: End-to-end TCP maintained
\end{itemize}
\end{solutionbox}

\begin{mnemonicbox}
\mnemonic{Snoop-Cache-Retransmit}
\end{mnemonicbox}

\questionmarks{4(b OR)}{4}{Explain the Handover Management in mobile IP}

\begin{solutionbox}
Handover management maintains connectivity when mobile node changes networks.

\begin{center}
\captionof{table}{Handover Phases}
\begin{tabulary}{\linewidth}{|L|L|}
\hline
\textbf{Phase} & \textbf{Process} \\ \hline
\textbf{Discovery} & Find new foreign agent \\ \hline
\textbf{Registration} & Update care-of address \\ \hline
\textbf{Data Forwarding} & Redirect packets \\ \hline
\textbf{Cleanup} & Release old resources \\ \hline
\end{tabulary}
\end{center}
\end{solutionbox}

\begin{mnemonicbox}
\mnemonic{Discover-Register-Forward-Cleanup}
\end{mnemonicbox}

\questionmarks{4(c OR)}{7}{Write the goals and the requirements for the Mobile IP}

\begin{solutionbox}
\textbf{Goals:}
\begin{itemize}
    \item \keyword{Transparency}: Seamless mobility for applications
    \item \keyword{Compatibility}: Work with existing internet protocols
    \item \keyword{Scalability}: Support large number of mobile nodes
    \item \keyword{Security}: Authenticate mobile nodes and protect data
\end{itemize}

\textbf{Requirements:}
\begin{itemize}
    \item \keyword{Home Agent}: Maintains mobile node location
    \item \keyword{Foreign Agent}: Provides local services
    \item \keyword{Care-of Address}: Temporary address in foreign network
    \item \keyword{Tunneling}: Packet forwarding mechanism
\end{itemize}

\begin{center}
\captionof{table}{Mobile IP Goals vs Requirements}
\begin{tabulary}{\linewidth}{|L|L|L|}
\hline
\textbf{Aspect} & \textbf{Goals} & \textbf{Requirements} \\ \hline
\textbf{Mobility} & Seamless movement & Care-of address \\ \hline
\textbf{Connectivity} & Maintain sessions & Tunneling \\ \hline
\textbf{Performance} & Minimal overhead & Efficient routing \\ \hline
\textbf{Security} & Authentication & Secure protocols \\ \hline
\end{tabulary}
\end{center}
\end{solutionbox}

\begin{mnemonicbox}
\mnemonic{Transparent-Compatible-Scalable-Secure}
\end{mnemonicbox}

\questionmarks{5(a)}{3}{Write the features of 6G in mobile networks}

\begin{solutionbox}
6G represents the next generation of mobile networks with advanced capabilities.

\textbf{Key Points:}
\begin{itemize}
    \item \keyword{Speed}: 1 Tbps theoretical speed
    \item \keyword{Latency}: Sub-millisecond latency
    \item \keyword{AI Integration}: Native artificial intelligence
\end{itemize}
\end{solutionbox}

\begin{mnemonicbox}
\mnemonic{Tera-Speed-AI-Integration}
\end{mnemonicbox}

\questionmarks{5(b)}{4}{Describe Dynamic Host Configuration Protocol (DHCP)}

\begin{solutionbox}
DHCP automatically assigns IP addresses and network configuration to devices.

\begin{center}
\captionof{table}{DHCP Process (DORA)}
\begin{tabulary}{\linewidth}{|L|L|}
\hline
\textbf{Process} & \textbf{Description} \\ \hline
\textbf{Discover} & Client broadcasts request \\ \hline
\textbf{Offer} & Server offers IP address \\ \hline
\textbf{Request} & Client requests specific IP \\ \hline
\textbf{Acknowledge} & Server confirms assignment \\ \hline
\end{tabulary}
\end{center}
\end{solutionbox}

\begin{mnemonicbox}
\mnemonic{Discover-Offer-Request-Acknowledge}
\end{mnemonicbox}

\questionmarks{5(c)}{7}{Describe the architecture of Wireless Personal Area Network (WLAN)}

\begin{solutionbox}
WLAN architecture provides wireless connectivity within local area using IEEE 802.11 standards.

\begin{center}
\begin{tikzpicture}[node distance=1.5cm, auto]
    \node [gtu block] (ds) {Distribution System (DS)};
    \node [gtu block, below=of ds] (ap) {Access Point (AP)};
    
    \node [gtu state, below left=1.5cm and 1cm of ap] (s1) {Station 1};
    \node [gtu state, below=1.5cm of ap] (s2) {Station 2};
    \node [gtu state, below right=1.5cm and 1cm of ap] (s3) {Station 3};
    
    \draw [gtu arrow] (ap) -- (ds);
    \draw [dashed] (s1) -- (ap);
    \draw [dashed] (s2) -- (ap);
    \draw [dashed] (s3) -- (ap);
    
    \node [draw, dashed, fit=(ap) (s1) (s3), label=right:BSS] {};
\end{tikzpicture}
\captionof{figure}{WLAN Architecture (Infrastructure Mode)}
\end{center}

\begin{center}
\captionof{table}{WLAN Components}
\begin{tabulary}{\linewidth}{|L|L|}
\hline
\textbf{Component} & \textbf{Function} \\ \hline
\textbf{Access Point} & Central wireless hub \\ \hline
\textbf{Station} & Wireless client device \\ \hline
\textbf{Distribution System} & Backbone network \\ \hline
\textbf{BSS} & Basic Service Set \\ \hline
\textbf{ESS} & Extended Service Set \\ \hline
\end{tabulary}
\end{center}

\textbf{Key Points:}
\begin{itemize}
    \item \keyword{Infrastructure Mode}: Uses access points
    \item \keyword{Ad-hoc Mode}: Direct device communication
    \item \keyword{Standards}: 802.11a/b/g/n/ac/ax protocols
\end{itemize}
\end{solutionbox}

\begin{mnemonicbox}
\mnemonic{Access-Station-Distribution}
\end{mnemonicbox}

\questionmarks{5(a OR)}{3}{Write the features of 5G in mobile networks}

\begin{solutionbox}
5G provides enhanced mobile broadband with ultra-low latency.

\textbf{Key Points:}
\begin{itemize}
    \item \keyword{Speed}: Up to 10 Gbps download
    \item \keyword{Latency}: 1ms ultra-low latency
    \item \keyword{Density}: 1 million devices per km\textsuperscript{2}
\end{itemize}
\end{solutionbox}

\begin{mnemonicbox}
\mnemonic{10G-1ms-1Million}
\end{mnemonicbox}

\questionmarks{5(b OR)}{4}{Explain WWW and HTTP}

\begin{solutionbox}
World Wide Web uses HTTP protocol for web page communication.

\begin{center}
\captionof{table}{WWW vs HTTP}
\begin{tabulary}{\linewidth}{|L|L|L|}
\hline
\textbf{Aspect} & \textbf{WWW} & \textbf{HTTP} \\ \hline
\textbf{Purpose} & Information sharing & Communication protocol \\ \hline
\textbf{Components} & Web pages, browsers & Request/response \\ \hline
\textbf{Format} & HTML documents & Text-based protocol \\ \hline
\textbf{Port} & Various & 80, 443 \\ \hline
\end{tabulary}
\end{center}
\end{solutionbox}

\begin{mnemonicbox}
\mnemonic{Web-Hypertext-Transfer}
\end{mnemonicbox}

\questionmarks{5(c OR)}{7}{Describe the architecture of Bluetooth}

\begin{solutionbox}
Bluetooth architecture provides short-range wireless communication using protocol stack.

\begin{center}
\begin{tikzpicture}[node distance=0cm, outer sep=0pt]
    \node [gtu block, minimum width=6cm] (app) {Application Layer};
    \node [gtu block, minimum width=6cm, below=0.1cm of app] (mid) {OBEX / SDP};
    \node [gtu block, minimum width=6cm, below=0.1cm of mid] (l2cap) {L2CAP};
    \node [gtu block, minimum width=6cm, below=0.1cm of l2cap] (hci) {Host Controller Interface (HCI)};
    \node [gtu block, minimum width=6cm, below=0.1cm of hci] (lm) {Link Manager};
    \node [gtu block, minimum width=6cm, below=0.1cm of lm] (base) {Baseband};
    \node [gtu block, minimum width=6cm, below=0.1cm of base] (radio) {Radio Layer};
    
    \draw [->, thick] (app.south) -- (mid.north);
    \draw [->, thick] (mid.south) -- (l2cap.north);
\end{tikzpicture}
\captionof{figure}{Bluetooth Protocol Stack}
\end{center}

\begin{center}
\captionof{table}{Bluetooth Layers}
\begin{tabulary}{\linewidth}{|L|L|}
\hline
\textbf{Layer} & \textbf{Function} \\ \hline
\textbf{Radio} & Physical transmission \\ \hline
\textbf{Baseband} & Timing and frequency hopping \\ \hline
\textbf{Link Manager} & Connection management \\ \hline
\textbf{HCI} & Host Controller Interface \\ \hline
\textbf{L2CAP} & Logical Link Control \\ \hline
\textbf{Applications} & User services \\ \hline
\end{tabulary}
\end{center}

\textbf{Key Points:}
\begin{itemize}
    \item \keyword{Piconet}: Master-slave network topology
    \item \keyword{Frequency Hopping}: 79 frequency channels
    \item \keyword{Power Classes}: Different transmission ranges
\end{itemize}
\end{solutionbox}

\begin{mnemonicbox}
\mnemonic{Radio-Baseband-Link-Host-Logic}
\end{mnemonicbox}

\end{document}
