\documentclass[10pt,a4paper]{article}

% content/resources/templates/preamble.tex
\usepackage[margin=0.6in]{geometry}
\author{Milav Dabgar}
\usepackage{amsmath,amssymb,amsthm}
\usepackage{booktabs}
\usepackage{multirow}
\usepackage{xcolor}
\usepackage{tcolorbox}
\tcbuselibrary{breakable,skins}
\usepackage[colorlinks=true,linkcolor=blue]{hyperref}
\usepackage{titlesec}
\usepackage{enumitem}
\usepackage{tikz}
\usepackage{pgfplots}
\usepackage{circuitikz}
\usepackage[version=4]{mhchem}
\usepackage{longtable}
\usepackage{array}
\usepackage{float}
\usepackage{caption}
\usepackage{listings}

\lstset{
  basicstyle=\small\ttfamily,
  breaklines=true,
  breakatwhitespace=false,
  postbreak=\mbox{\textcolor{red}{$\hookrightarrow$}\space},
  float=false,
  numbers=left,
  numberstyle=\tiny\color{gray},
  numbersep=10pt,
  xleftmargin=2em,
  keywordstyle=\color{blue},
  commentstyle=\color{green!60!black},
  stringstyle=\color{purple},
  backgroundcolor=\color{gray!5},
  showstringspaces=false,
  tabsize=2,
  captionpos=b,
  keepspaces=true,
  columns=flexible
}

\pgfplotsset{compat=1.18}
\usetikzlibrary{shapes,arrows,positioning,calc,patterns,decorations.pathmorphing,decorations.markings,arrows.meta}

% Color scheme
\definecolor{headcolor}{RGB}{0,102,204}
\definecolor{keycolor}{RGB}{220,20,60}
\definecolor{solutioncolor}{RGB}{34,139,34}
\definecolor{mnemoniccolor}{RGB}{148,0,211}
\definecolor{codecolor}{RGB}{0,0,100}

% Spacing
\setlength{\parskip}{3pt}
\setlist[itemize]{nosep}
\setlist[enumerate]{nosep}

% Title formatting
\titleformat{\section}{\Large\bfseries\color{headcolor}}{\thesection}{1em}{}
\titleformat{\subsection}{\large\bfseries\color{headcolor}}{\thesubsection}{1em}{}

% Pandoc tightlist compatibility
\providecommand{\tightlist}{%
  \setlength{\itemsep}{0pt}\setlength{\parskip}{0pt}}

% Pandoc longtable compatibility
\newcounter{none}
\def\thenone{}


% content/resources/templates/english-boxes.tex
% This file is currently empty - it exists to maintain consistency with the import structure.
% Add custom environments here if needed in the future.


\begin{document}

\begin{center}
{\Huge\bfseries\color{headcolor} Subject Name Solutions}\\[5pt]
{\LARGE 4351602 -- Winter 2023}\\[3pt]
{\large Semester 1 Study Material}\\[3pt]
{\normalsize\textit{Detailed Solutions and Explanations}}
\end{center}

\vspace{10pt}

\subsection*{Question 1(a) [03 marks]}\label{q1a}

\textbf{Differentiate between client server and peer to peer network.}

\begin{solutionbox}

{\def\LTcaptype{none} % do not increment counter
\begin{longtable}[]{@{}
  >{\raggedright\arraybackslash}p{(\linewidth - 4\tabcolsep) * \real{0.2037}}
  >{\raggedright\arraybackslash}p{(\linewidth - 4\tabcolsep) * \real{0.4074}}
  >{\raggedright\arraybackslash}p{(\linewidth - 4\tabcolsep) * \real{0.3889}}@{}}
\toprule\noalign{}
\begin{minipage}[b]{\linewidth}\raggedright
Parameter
\end{minipage} & \begin{minipage}[b]{\linewidth}\raggedright
Client-Server Network
\end{minipage} & \begin{minipage}[b]{\linewidth}\raggedright
Peer-to-Peer Network
\end{minipage} \\
\midrule\noalign{}
\endhead
\bottomrule\noalign{}
\endlastfoot
\textbf{Architecture} & Centralized with dedicated server &
Decentralized, all nodes equal \\
\textbf{Cost} & Higher due to server hardware & Lower, uses existing
computers \\
\textbf{Security} & High, centralized control & Lower, distributed
control \\
\textbf{Scalability} & Limited by server capacity & Better, resources
increase with nodes \\
\end{longtable}
}

\end{solutionbox}
\begin{mnemonicbox}
``CSS-P: Client-Server = Centralized Security, P2P =
Peer Power''

\end{mnemonicbox}
\subsection*{Question 1(b) [04 marks]}\label{q1b}

\textbf{Explain ARP Protocol with its working.}

\begin{solutionbox}

\textbf{ARP (Address Resolution Protocol)} maps IP addresses to MAC
addresses in local networks.

\textbf{Working Process:}

\begin{itemize}
\tightlist
\item
  \textbf{Broadcast Request}: Host broadcasts ARP request with target IP
\item
  \textbf{Cache Check}: Receiving hosts check if IP matches theirs
\item
  \textbf{Reply Generation}: Target host sends ARP reply with MAC
  address
\item
  \textbf{Cache Update}: Requesting host updates ARP table
\end{itemize}

\textbf{ARP Table Example:}

\begin{verbatim}
IP Address      MAC Address         TTL
192.168.1.1     00:1A:2B:3C:4D:5E   300s
\end{verbatim}

\end{solutionbox}
\begin{mnemonicbox}
``BCRU: Broadcast, Cache, Reply, Update''

\end{mnemonicbox}
\subsection*{Question 1(c) [07 marks]}\label{q1c}

\textbf{Explain OSI model with diagram.}

\begin{solutionbox}

The \textbf{OSI (Open Systems Interconnection)} model has 7 layers for
network communication.

\begin{center}
\textbf{Mermaid Diagram (Code)}
\begin{verbatim}
{Shaded}
{Highlighting}[]
graph LR
    A[Application Layer {- 7] {-}{-}{} B[Presentation Layer {-} 6]}
    B {-{-}{} C[Session Layer {-} 5]}
    C {-{-}{} D[Transport Layer {-} 4]}
    D {-{-}{} E[Network Layer {-} 3]}
    E {-{-}{} F[Data Link Layer {-} 2]}
    F {-{-}{} G[Physical Layer {-} 1]}
{Highlighting}
{Shaded}
\end{verbatim}
\end{center}

\textbf{Layer Functions:}

\begin{itemize}
\tightlist
\item
  \textbf{Physical}: Bit transmission over physical medium
\item
  \textbf{Data Link}: Frame transmission, error detection
\item
  \textbf{Network}: Routing, IP addressing
\item
  \textbf{Transport}: End-to-end delivery, TCP/UDP
\item
  \textbf{Session}: Connection management
\item
  \textbf{Presentation}: Data encryption, compression
\item
  \textbf{Application}: User interfaces, email, web
\end{itemize}

\end{solutionbox}
\begin{mnemonicbox}
``All People Seem To Need Data Processing''

\end{mnemonicbox}
\subsection*{Question 1(c OR) [07
marks]}\label{question-1c-or-07-marks}

\textbf{What is Congestion? Explain Congestion Control.}

\begin{solutionbox}

\textbf{Congestion} occurs when network traffic exceeds available
bandwidth, causing packet delays and losses.

\textbf{Types of Congestion Control:}

{\def\LTcaptype{none} % do not increment counter
\begin{longtable}[]{@{}lll@{}}
\toprule\noalign{}
Type & Method & Description \\
\midrule\noalign{}
\endhead
\bottomrule\noalign{}
\endlastfoot
\textbf{Open-Loop} & Prevention & Traffic shaping before congestion \\
\textbf{Closed-Loop} & Reaction & Feedback-based adjustment \\
\end{longtable}
}

\textbf{Congestion Control Techniques:}

\begin{itemize}
\tightlist
\item
  \textbf{Traffic Shaping}: Regulate data transmission rate
\item
  \textbf{Admission Control}: Limit new connections during congestion
\item
  \textbf{Load Shedding}: Drop packets when buffers full
\item
  \textbf{Backpressure}: Send congestion signals upstream
\end{itemize}

\end{solutionbox}
\begin{mnemonicbox}
``TALB: Traffic, Admission, Load, Backpressure''

\end{mnemonicbox}
\subsection*{Question 2(a) [03 marks]}\label{q2a}

\textbf{What is Ad-hoc Network? Explain it.}

\begin{solutionbox}

\textbf{Ad-hoc Network} is a wireless network without fixed
infrastructure where nodes communicate directly.

\textbf{Characteristics:}

\begin{itemize}
\tightlist
\item
  \textbf{Self-organizing}: Automatic network formation
\item
  \textbf{Dynamic topology}: Nodes can join/leave freely\\
\item
  \textbf{Multi-hop routing}: Messages relay through intermediate nodes
\item
  \textbf{Distributed control}: No central authority
\end{itemize}

\textbf{Applications:}

\begin{itemize}
\tightlist
\item
  Emergency response, military operations, sensor networks
\end{itemize}

\end{solutionbox}
\begin{mnemonicbox}
``SDMD: Self-organizing, Dynamic, Multi-hop,
Distributed''

\end{mnemonicbox}
\subsection*{Question 2(b) [04 marks]}\label{q2b}

\textbf{Explain Handover Management in Mobile IP.}

\begin{solutionbox}

\textbf{Handover} is the process of maintaining connectivity when a
mobile node moves between networks.

\textbf{Handover Process:}

\begin{verbatim}
sequenceDiagram
    participant MN as Mobile Node
    participant FA1 as Foreign Agent 1
    participant FA2 as Foreign Agent 2
    participant HA as Home Agent
    
    MN{-FA2: Agent Discovery}
    FA2{-MN: Advertisement}
    MN{-HA: Registration Request}
    HA{-MN: Registration Reply}
    HA{-FA1: Update Tunnel}
\end{verbatim}

\textbf{Types:}

\begin{itemize}
\tightlist
\item
  \textbf{Hard Handover}: Break-before-make connection
\item
  \textbf{Soft Handover}: Make-before-break connection
\end{itemize}

\end{solutionbox}
\begin{mnemonicbox}
``DARU: Discovery, Advertisement, Registration,
Update''

\end{mnemonicbox}
\subsection*{Question 2(c) [07 marks]}\label{q2c}

\textbf{Explain Three tier architecture of mobile computing with
diagram.}

\begin{solutionbox}

\textbf{Three-tier architecture} separates mobile applications into
presentation, application logic, and data layers.

\begin{verbatim}
graph TB
    subgraph "Tier 1: Presentation Layer"
        A[Mobile Device]
        B[User Interface]
        C[Input/Output]
    end
    
    subgraph "Tier 2: Application Layer"
        D[Business Logic]
        E[Processing Rules]
        F[Middleware]
    end
    
    subgraph "Tier 3: Data Layer"
        G[Database Server]
        H[Data Storage]
        I[Data Management]
    end
    
    A {-{-} D}
    D {-{-} G}
\end{verbatim}

\textbf{Layer Functions:}

\begin{itemize}
\tightlist
\item
  \textbf{Presentation}: User interface, mobile apps
\item
  \textbf{Application}: Business logic, middleware services
\item
  \textbf{Data}: Database management, storage systems
\end{itemize}

\textbf{Benefits:}

\begin{itemize}
\tightlist
\item
  \textbf{Scalability}: Independent layer scaling
\item
  \textbf{Maintainability}: Separate concerns
\item
  \textbf{Flexibility}: Technology independence
\end{itemize}

\end{solutionbox}
\begin{mnemonicbox}
``PAD: Presentation, Application, Data''

\end{mnemonicbox}
\subsection*{Question 2(a OR) [03
marks]}\label{question-2a-or-03-marks}

\textbf{Explain Need of Wireless Network.}

\begin{solutionbox}

\textbf{Wireless Networks} provide connectivity without physical cables.

\textbf{Needs:}

\begin{itemize}
\tightlist
\item
  \textbf{Mobility}: Users can move freely while connected
\item
  \textbf{Flexibility}: Easy network expansion and reconfiguration
\item
  \textbf{Cost-effective}: Reduced cabling infrastructure costs
\item
  \textbf{Accessibility}: Internet access in remote areas
\end{itemize}

\textbf{Applications:}

\begin{itemize}
\tightlist
\item
  Mobile communications, WiFi hotspots, IoT devices
\end{itemize}

\end{solutionbox}
\begin{mnemonicbox}
``MFCA: Mobility, Flexibility, Cost, Accessibility''

\end{mnemonicbox}
\subsection*{Question 2(b OR) [04
marks]}\label{question-2b-or-04-marks}

\textbf{Explain Registration, tunneling and encapsulation in mobile ip.}

\begin{solutionbox}

\textbf{Mobile IP Components:}

{\def\LTcaptype{none} % do not increment counter
\begin{longtable}[]{@{}
  >{\raggedright\arraybackslash}p{(\linewidth - 4\tabcolsep) * \real{0.2903}}
  >{\raggedright\arraybackslash}p{(\linewidth - 4\tabcolsep) * \real{0.4194}}
  >{\raggedright\arraybackslash}p{(\linewidth - 4\tabcolsep) * \real{0.2903}}@{}}
\toprule\noalign{}
\begin{minipage}[b]{\linewidth}\raggedright
Process
\end{minipage} & \begin{minipage}[b]{\linewidth}\raggedright
Description
\end{minipage} & \begin{minipage}[b]{\linewidth}\raggedright
Purpose
\end{minipage} \\
\midrule\noalign{}
\endhead
\bottomrule\noalign{}
\endlastfoot
\textbf{Registration} & Mobile node registers with home agent & Location
update \\
\textbf{Tunneling} & Creates virtual path between agents & Route
packets \\
\textbf{Encapsulation} & Wraps original packet in new header & Address
translation \\
\end{longtable}
}

\textbf{Process Flow:}

\begin{verbatim}
Original Packet \rightarrow Encapsulation \rightarrow Tunnel \rightarrow Decapsulation \rightarrow Destination
\end{verbatim}

\textbf{Registration Steps:}

\begin{itemize}
\tightlist
\item
  Mobile node discovers foreign agent
\item
  Sends registration request to home agent
\item
  Home agent updates location binding
\end{itemize}

\end{solutionbox}
\begin{mnemonicbox}
``RTE: Registration, Tunneling, Encapsulation''

\end{mnemonicbox}
\subsection*{Question 2(c OR) [07
marks]}\label{question-2c-or-07-marks}

\textbf{What is Middleware? Write down examples of middleware and
explain any one of them in detail.}

\begin{solutionbox}

\textbf{Middleware} is software that connects different applications and
services in distributed systems.

\textbf{Examples of Middleware:}

\begin{itemize}
\tightlist
\item
  \textbf{Message-Oriented Middleware (MOM)}
\item
  \textbf{Remote Procedure Call (RPC)}
\item
  \textbf{Object Request Broker (ORB)}
\item
  \textbf{Database Middleware}
\item
  \textbf{Web Services}
\end{itemize}

\textbf{Message-Oriented Middleware (MOM) - Detailed:}

\textbf{Architecture:}

\begin{center}
\textbf{Mermaid Diagram (Code)}
\begin{verbatim}
{Shaded}
{Highlighting}[]
graph LR
    A[Sender Application] {-{-}{} B[Message Queue]}
    B {-{-}{} C[MOM Layer]}
    C {-{-}{} D[Message Queue]}
    D {-{-}{} E[Receiver Application]}
{Highlighting}
{Shaded}
\end{verbatim}
\end{center}

\textbf{Features:}

\begin{itemize}
\tightlist
\item
  \textbf{Asynchronous Communication}: Non-blocking message exchange
\item
  \textbf{Reliability}: Message persistence and delivery guarantees
\item
  \textbf{Scalability}: Handle multiple concurrent connections
\item
  \textbf{Platform Independence}: Cross-platform communication
\end{itemize}

\textbf{Benefits:}

\begin{itemize}
\tightlist
\item
  Loose coupling between applications
\item
  Improved system reliability
\item
  Better fault tolerance
\end{itemize}

\end{solutionbox}
\begin{mnemonicbox}
``ARSP: Asynchronous, Reliable, Scalable,
Platform-independent''

\end{mnemonicbox}
\subsection*{Question 3(a) [03 marks]}\label{q3a}

\textbf{Give Full form for `www'. Explain it.}

\begin{solutionbox}

\textbf{WWW = World Wide Web}

\textbf{Explanation:}

\begin{itemize}
\tightlist
\item
  \textbf{Global Information System}: Interconnected web of documents
\item
  \textbf{HTTP Protocol}: Uses HyperText Transfer Protocol
\item
  \textbf{URL Addressing}: Unique resource locators
\item
  \textbf{Hyperlinks}: Navigate between web pages
\end{itemize}

\textbf{Components:}

\begin{itemize}
\tightlist
\item
  Web servers, browsers, HTML documents, URLs
\end{itemize}

\end{solutionbox}
\begin{mnemonicbox}
``GHUH: Global, HTTP, URL, Hyperlinks''

\end{mnemonicbox}
\subsection*{Question 3(b) [04 marks]}\label{q3b}

\textbf{Explain applications of Mobile Computing.}

\begin{solutionbox}

\textbf{Mobile Computing Applications:}

{\def\LTcaptype{none} % do not increment counter
\begin{longtable}[]{@{}
  >{\raggedright\arraybackslash}p{(\linewidth - 4\tabcolsep) * \real{0.3030}}
  >{\raggedright\arraybackslash}p{(\linewidth - 4\tabcolsep) * \real{0.3939}}
  >{\raggedright\arraybackslash}p{(\linewidth - 4\tabcolsep) * \real{0.3030}}@{}}
\toprule\noalign{}
\begin{minipage}[b]{\linewidth}\raggedright
Category
\end{minipage} & \begin{minipage}[b]{\linewidth}\raggedright
Applications
\end{minipage} & \begin{minipage}[b]{\linewidth}\raggedright
Benefits
\end{minipage} \\
\midrule\noalign{}
\endhead
\bottomrule\noalign{}
\endlastfoot
\textbf{Business} & Email, CRM, Sales & Productivity, Real-time
access \\
\textbf{Healthcare} & Patient monitoring, Telemedicine & Remote care,
Emergency response \\
\textbf{Education} & E-learning, Digital libraries & Flexible learning,
Resource access \\
\textbf{Entertainment} & Gaming, Streaming, Social media & On-demand
content, Connectivity \\
\end{longtable}
}

\textbf{Key Features:}

\begin{itemize}
\tightlist
\item
  \textbf{Location-based services}: GPS navigation, local search
\item
  \textbf{Mobile payments}: Digital wallets, contactless transactions
\item
  \textbf{IoT integration}: Smart home, wearable devices
\end{itemize}

\end{solutionbox}
\begin{mnemonicbox}
``BHEE: Business, Healthcare, Education,
Entertainment''

\end{mnemonicbox}
\subsection*{Question 3(c) [07 marks]}\label{q3c}

\textbf{Explain working of DHCP with the help of diagram and explain its
advantages.}

\begin{solutionbox}

\textbf{DHCP (Dynamic Host Configuration Protocol)} automatically
assigns IP addresses to network devices.

\textbf{DHCP Process (DORA):}

\begin{verbatim}
sequenceDiagram
    participant C as Client
    participant S as DHCP Server
    
    C{-S: 1. DHCP Discover (Broadcast)}
    S{-C: 2. DHCP Offer (IP + Config)}
    C{-S: 3. DHCP Request (Accept Offer)}
    S{-C: 4. DHCP Acknowledge (Confirm)}
\end{verbatim}

\textbf{Configuration Information Provided:}

\begin{itemize}
\tightlist
\item
  IP address and subnet mask
\item
  Default gateway address
\item
  DNS server addresses
\item
  Lease duration
\end{itemize}

\textbf{Advantages:}

\begin{itemize}
\tightlist
\item
  \textbf{Automatic Configuration}: No manual IP assignment
\item
  \textbf{Centralized Management}: Single point of control
\item
  \textbf{Efficient IP Usage}: Dynamic allocation prevents waste
\item
  \textbf{Reduced Errors}: Eliminates manual configuration mistakes
\item
  \textbf{Easy Maintenance}: Simple network changes
\end{itemize}

\textbf{DHCP Message Types:}

\begin{itemize}
\tightlist
\item
  DISCOVER, OFFER, REQUEST, ACK, NAK, RELEASE, RENEW
\end{itemize}

\end{solutionbox}
\begin{mnemonicbox}
``DORA: Discover, Offer, Request, Acknowledge''

\end{mnemonicbox}
\subsection*{Question 3(a OR) [03
marks]}\label{question-3a-or-03-marks}

\textbf{Write down: Importance of HTTPS.}

\begin{solutionbox}

\textbf{HTTPS (HyperText Transfer Protocol Secure)} provides secure web
communication.

\textbf{Importance:}

\begin{itemize}
\tightlist
\item
  \textbf{Data Encryption}: Protects data in transit using SSL/TLS
\item
  \textbf{Authentication}: Verifies server identity with certificates
\item
  \textbf{Data Integrity}: Prevents data tampering during transmission
\item
  \textbf{Trust Building}: Increases user confidence in websites
\end{itemize}

\textbf{Security Benefits:}

\begin{itemize}
\tightlist
\item
  Protection against eavesdropping and man-in-the-middle attacks
\end{itemize}

\end{solutionbox}
\begin{mnemonicbox}
``EADT: Encryption, Authentication, Integrity,
Trust''

\end{mnemonicbox}
\subsection*{Question 3(b OR) [04
marks]}\label{question-3b-or-04-marks}

\textbf{What is Bearer Network? Explain in Detail.}

\begin{solutionbox}

\textbf{Bearer Network} is the underlying network infrastructure that
carries data traffic between endpoints.

\textbf{Types of Bearer Networks:}

{\def\LTcaptype{none} % do not increment counter
\begin{longtable}[]{@{}
  >{\raggedright\arraybackslash}p{(\linewidth - 4\tabcolsep) * \real{0.1765}}
  >{\raggedright\arraybackslash}p{(\linewidth - 4\tabcolsep) * \real{0.3529}}
  >{\raggedright\arraybackslash}p{(\linewidth - 4\tabcolsep) * \real{0.4706}}@{}}
\toprule\noalign{}
\begin{minipage}[b]{\linewidth}\raggedright
Type
\end{minipage} & \begin{minipage}[b]{\linewidth}\raggedright
Technology
\end{minipage} & \begin{minipage}[b]{\linewidth}\raggedright
Characteristics
\end{minipage} \\
\midrule\noalign{}
\endhead
\bottomrule\noalign{}
\endlastfoot
\textbf{Circuit-Switched} & Traditional telephony & Dedicated path,
Guaranteed bandwidth \\
\textbf{Packet-Switched} & Internet, IP networks & Shared resources,
Variable bandwidth \\
\textbf{Wireless} & Cellular, WiFi & Mobile connectivity, Air
interface \\
\end{longtable}
}

\textbf{Functions:}

\begin{itemize}
\tightlist
\item
  \textbf{Data Transport}: Carry user data and signaling
\item
  \textbf{Quality of Service}: Manage bandwidth and latency
\item
  \textbf{Routing}: Direct traffic between networks
\item
  \textbf{Network Management}: Monitor and control traffic
\end{itemize}

\textbf{Examples:}

\begin{itemize}
\tightlist
\item
  PSTN, Internet backbone, 4G/5G cellular networks
\end{itemize}

\end{solutionbox}
\begin{mnemonicbox}
``DQRN: Data transport, QoS, Routing, Network
management''

\end{mnemonicbox}
\subsection*{Question 3(c OR) [07
marks]}\label{question-3c-or-07-marks}

\textbf{List out types of TCP and explain any one in detail.}

\begin{solutionbox}

\textbf{Types of TCP:}

\begin{itemize}
\tightlist
\item
  \textbf{Standard TCP (TCP Tahoe)}
\item
  \textbf{TCP Reno}
\item
  \textbf{TCP New Reno}\\
\item
  \textbf{TCP Vegas}
\item
  \textbf{TCP SACK (Selective Acknowledgment)}
\item
  \textbf{TCP Cubic}
\end{itemize}

\textbf{TCP Reno - Detailed Explanation:}

\textbf{Features:}

\begin{itemize}
\tightlist
\item
  \textbf{Fast Retransmit}: Retransmit lost packets quickly
\item
  \textbf{Fast Recovery}: Avoid slow start after fast retransmit
\item
  \textbf{Congestion Avoidance}: Linear increase in congestion window
\item
  \textbf{Duplicate ACK Detection}: Identify packet loss
\end{itemize}

\textbf{Congestion Control Algorithm:}

\begin{center}
\textbf{Mermaid Diagram (Code)}
\begin{verbatim}
{Shaded}
{Highlighting}[]
graph LR
    A[Slow Start] {-{-}{} B\{3 Duplicate ACKs?\}}
    B {-{-}{}|Yes| C[Fast Retransmit]}
    C {-{-}{} D[Fast Recovery]}
    D {-{-}{} E[Congestion Avoidance]}
    B {-{-}{}|No| F[Timeout?]}
    F {-{-}{}|Yes| A}
    F {-{-}{}|No| E}
{Highlighting}
{Shaded}
\end{verbatim}
\end{center}

\textbf{Advantages:}

\begin{itemize}
\tightlist
\item
  \textbf{Better Performance}: Faster recovery from packet loss
\item
  \textbf{Efficiency}: Maintains higher throughput
\item
  \textbf{Fairness}: Equitable bandwidth sharing
\end{itemize}

\textbf{Window Management:}

\begin{itemize}
\tightlist
\item
  Exponential growth in slow start
\item
  Linear growth in congestion avoidance
\end{itemize}

\end{solutionbox}
\begin{mnemonicbox}
``FFCE: Fast retransmit, Fast recovery, Congestion
avoidance, Efficiency''

\end{mnemonicbox}
\subsection*{Question 4(a) [03 marks]}\label{q4a}

\textbf{Define WLAN. List out types of WLAN.}

\begin{solutionbox}

\textbf{WLAN (Wireless Local Area Network)} provides wireless
connectivity within a limited area.

\textbf{Types of WLAN:}

\begin{itemize}
\tightlist
\item
  \textbf{Infrastructure Mode}: Uses access points for connectivity
\item
  \textbf{Ad-hoc Mode}: Direct device-to-device communication
\item
  \textbf{Mesh Networks}: Multi-hop wireless connectivity
\item
  \textbf{Hybrid Networks}: Combination of infrastructure and ad-hoc
\end{itemize}

\textbf{Standards:}

\begin{itemize}
\tightlist
\item
  IEEE 802.11a/b/g/n/ac/ax (WiFi 6)
\end{itemize}

\end{solutionbox}
\begin{mnemonicbox}
``IAMH: Infrastructure, Ad-hoc, Mesh, Hybrid''

\end{mnemonicbox}
\subsection*{Question 4(b) [04 marks]}\label{q4b}

\textbf{What is Routing? Explain types of Routing.}

\begin{solutionbox}

\textbf{Routing} is the process of selecting paths for data packets
across networks.

\textbf{Types of Routing:}

{\def\LTcaptype{none} % do not increment counter
\begin{longtable}[]{@{}
  >{\raggedright\arraybackslash}p{(\linewidth - 4\tabcolsep) * \real{0.2000}}
  >{\raggedright\arraybackslash}p{(\linewidth - 4\tabcolsep) * \real{0.2667}}
  >{\raggedright\arraybackslash}p{(\linewidth - 4\tabcolsep) * \real{0.5333}}@{}}
\toprule\noalign{}
\begin{minipage}[b]{\linewidth}\raggedright
Type
\end{minipage} & \begin{minipage}[b]{\linewidth}\raggedright
Method
\end{minipage} & \begin{minipage}[b]{\linewidth}\raggedright
Characteristics
\end{minipage} \\
\midrule\noalign{}
\endhead
\bottomrule\noalign{}
\endlastfoot
\textbf{Static Routing} & Manual configuration & Fixed paths, No
automatic updates \\
\textbf{Dynamic Routing} & Automatic updates & Adaptive paths, Real-time
changes \\
\textbf{Default Routing} & Catch-all route & Used when no specific route
exists \\
\textbf{Distance Vector} & Hop count based & RIP protocol, Simple
implementation \\
\textbf{Link State} & Network topology & OSPF protocol, Faster
convergence \\
\end{longtable}
}

\textbf{Dynamic Routing Advantages:}

\begin{itemize}
\tightlist
\item
  \textbf{Automatic adaptation} to network changes
\item
  \textbf{Load balancing} across multiple paths
\item
  \textbf{Fault tolerance} with alternate routes
\end{itemize}

\end{solutionbox}
\begin{mnemonicbox}
``SDDL: Static, Dynamic, Default, Link-state''

\end{mnemonicbox}
\subsection*{Question 4(c) [07 marks]}\label{q4c}

\textbf{Explain architecture of WLAN.}

\begin{solutionbox}

\textbf{WLAN Architecture Components:}

\begin{verbatim}
graph TB
    subgraph "Basic Service Set (BSS)"
        A[Access Point] 
        B[Station 1]
        C[Station 2]
        D[Station 3]
    end
    
    subgraph "Extended Service Set (ESS)"
        E[AP1] {-{-} F[Distribution System]}
        G[AP2] {-{-} F}
        H[AP3] {-{-} F}
    end
    
    A {-{-} B}
    A {-{-} C  }
    A {-{-} D}
    F {-{-} I[Wired Network/Internet]}
\end{verbatim}

\textbf{Architecture Elements:}

\begin{itemize}
\tightlist
\item
  \textbf{Station (STA)}: Wireless client devices
\item
  \textbf{Access Point (AP)}: Central wireless hub
\item
  \textbf{Basic Service Set (BSS)}: Single AP coverage area
\item
  \textbf{Extended Service Set (ESS)}: Multiple interconnected APs
\item
  \textbf{Distribution System (DS)}: Backend network connecting APs
\end{itemize}

\textbf{WLAN Topologies:}

\begin{itemize}
\tightlist
\item
  \textbf{Infrastructure Mode}: Centralized through AP
\item
  \textbf{Ad-hoc Mode}: Direct peer-to-peer communication
\item
  \textbf{Mesh Topology}: Multi-hop wireless connections
\end{itemize}

\textbf{Services Provided:}

\begin{itemize}
\tightlist
\item
  \textbf{Association}: Device connection to AP
\item
  \textbf{Authentication}: Security verification
\item
  \textbf{Data Delivery}: Packet transmission
\item
  \textbf{Roaming}: Seamless movement between APs
\end{itemize}

\textbf{Frequency Bands:}

\begin{itemize}
\tightlist
\item
  2.4 GHz (802.11b/g/n)
\item
  5 GHz (802.11a/n/ac/ax)
\end{itemize}

\end{solutionbox}
\begin{mnemonicbox}
``SABED: Station, Access Point, BSS, ESS,
Distribution System''

\end{mnemonicbox}
\subsection*{Question 4(a OR) [03
marks]}\label{question-4a-or-03-marks}

\textbf{Define WPAN. List out applications of WPAN.}

\begin{solutionbox}

\textbf{WPAN (Wireless Personal Area Network)} connects devices within
personal space (typically 10 meters).

\textbf{Applications of WPAN:}

\begin{itemize}
\tightlist
\item
  \textbf{Device Synchronization}: Phone to computer data transfer
\item
  \textbf{Audio Streaming}: Wireless headphones, speakers
\item
  \textbf{Input Devices}: Wireless keyboard, mouse
\item
  \textbf{Healthcare}: Medical sensors, fitness trackers
\item
  \textbf{Smart Home}: IoT device control
\end{itemize}

\textbf{Technologies:}

\begin{itemize}
\tightlist
\item
  Bluetooth, Zigbee, NFC, infrared
\end{itemize}

\end{solutionbox}
\begin{mnemonicbox}
``DSAHS: Device sync, Streaming, Audio, Healthcare,
Smart home''

\end{mnemonicbox}
\subsection*{Question 4(b OR) [04
marks]}\label{question-4b-or-04-marks}

\textbf{Explain working of IMAP Protocol.}

\begin{solutionbox}

\textbf{IMAP (Internet Message Access Protocol)} manages email on mail
servers.

\textbf{IMAP Working Process:}

{\def\LTcaptype{none} % do not increment counter
\begin{longtable}[]{@{}
  >{\raggedright\arraybackslash}p{(\linewidth - 4\tabcolsep) * \real{0.2222}}
  >{\raggedright\arraybackslash}p{(\linewidth - 4\tabcolsep) * \real{0.2963}}
  >{\raggedright\arraybackslash}p{(\linewidth - 4\tabcolsep) * \real{0.4815}}@{}}
\toprule\noalign{}
\begin{minipage}[b]{\linewidth}\raggedright
Step
\end{minipage} & \begin{minipage}[b]{\linewidth}\raggedright
Action
\end{minipage} & \begin{minipage}[b]{\linewidth}\raggedright
Description
\end{minipage} \\
\midrule\noalign{}
\endhead
\bottomrule\noalign{}
\endlastfoot
\textbf{Connection} & Client connects to server & Establish TCP
connection on port 143/993 \\
\textbf{Authentication} & Login credentials & Username/password
verification \\
\textbf{Mailbox Selection} & Choose folder & Select INBOX or other
folders \\
\textbf{Message Operations} & Read/Delete/Flag & Manipulate messages on
server \\
\end{longtable}
}

\textbf{IMAP vs POP3:}

\begin{itemize}
\tightlist
\item
  \textbf{Server Storage}: Messages remain on server
\item
  \textbf{Multi-device Access}: Sync across devices
\item
  \textbf{Folder Management}: Server-side folder structure
\item
  \textbf{Partial Download}: Headers first, body on demand
\end{itemize}

\textbf{IMAP Commands:}

\begin{verbatim}
LOGIN user password
SELECT INBOX
FETCH 1 BODY[]
STORE 1 +FLAGS (\Deleted)
\end{verbatim}

\end{solutionbox}
\begin{mnemonicbox}
``CAMS: Connection, Authentication, Mailbox,
Storage''

\end{mnemonicbox}
\subsection*{Question 4(c OR) [07
marks]}\label{question-4c-or-07-marks}

\textbf{Explain Bluetooth technology with a figure of its protocol
stack.}

\begin{solutionbox}

\textbf{Bluetooth} is a short-range wireless communication technology
for personal area networks.

\textbf{Bluetooth Protocol Stack:}

\begin{center}
\textbf{Mermaid Diagram (Code)}
\begin{verbatim}
{Shaded}
{Highlighting}[]
graph LR
    A[Applications] {-{-}{} B[OBEX/SDP]}
    B {-{-}{} C[RFCOMM/L2CAP]}
    C {-{-}{} D[HCI {-} Host Controller Interface]}
    D {-{-}{} E[LMP {-} Link Manager Protocol]}
    E {-{-}{} F[Baseband/LC {-} Link Controller]}
    F {-{-}{} G[Radio Layer]}
{Highlighting}
{Shaded}
\end{verbatim}
\end{center}

\textbf{Layer Functions:}

\begin{itemize}
\tightlist
\item
  \textbf{Radio Layer}: 2.4 GHz ISM band, frequency hopping
\item
  \textbf{Baseband}: Timing, access control, packet formats
\item
  \textbf{LMP}: Link establishment, security, power management
\item
  \textbf{L2CAP}: Packet segmentation, protocol multiplexing
\item
  \textbf{RFCOMM}: Serial port emulation over wireless
\item
  \textbf{SDP}: Service discovery protocol
\item
  \textbf{Applications}: File transfer, audio streaming, HID
\end{itemize}

\textbf{Bluetooth Characteristics:}

\begin{itemize}
\tightlist
\item
  \textbf{Range}: 10 meters (Class 2 devices)
\item
  \textbf{Data Rate}: 1-3 Mbps (depending on version)
\item
  \textbf{Topology}: Star network (piconet)
\item
  \textbf{Security}: Authentication, authorization, encryption
\end{itemize}

\textbf{Bluetooth Versions:}

\begin{itemize}
\tightlist
\item
  Classic Bluetooth (BR/EDR)
\item
  Bluetooth Low Energy (BLE/LE)
\item
  Bluetooth 5.0+ (Enhanced range/speed)
\end{itemize}

\textbf{Applications:}

\begin{itemize}
\tightlist
\item
  Audio devices, keyboards, file transfer, IoT sensors
\end{itemize}

\end{solutionbox}
\begin{mnemonicbox}
``RBLSRA: Radio, Baseband, LMP, SDP, RFCOMM,
Applications''

\end{mnemonicbox}
\subsection*{Question 5(a) [03 marks]}\label{q5a}

\textbf{What is 4G? List out Features of 4G.}

\begin{solutionbox}

\textbf{4G (Fourth Generation)} is a mobile communication standard
providing high-speed wireless internet.

\textbf{Features of 4G:}

\begin{itemize}
\tightlist
\item
  \textbf{High Data Speed}: Up to 100 Mbps mobile, 1 Gbps stationary
\item
  \textbf{All-IP Network}: Packet-switched architecture
\item
  \textbf{Low Latency}: Reduced delay for real-time applications
\item
  \textbf{Quality of Service}: Guaranteed service levels
\item
  \textbf{Global Roaming}: Worldwide compatibility
\end{itemize}

\textbf{Technologies:}

\begin{itemize}
\tightlist
\item
  LTE (Long Term Evolution), WiMAX
\end{itemize}

\end{solutionbox}
\begin{mnemonicbox}
``HALQG: High-speed, All-IP, Low latency, QoS, Global
roaming''

\end{mnemonicbox}
\subsection*{Question 5(b) [04 marks]}\label{q5b}

\textbf{Explain Centralized Computing.}

\begin{solutionbox}

\textbf{Centralized Computing} processes all data and applications on a
central server.

\textbf{Architecture:}

\begin{verbatim}
graph TB
    A[Central Server] {-{-} B[Terminal 1]}
    A {-{-} C[Terminal 2] }
    A {-{-} D[Terminal 3]}
    A {-{-} E[Terminal 4]}
    
    F[Processing Power]
    G[Storage]
    H[Applications]
    
    F {-{-} A}
    G {-{-} A}
    H {-{-} A}
\end{verbatim}

\textbf{Characteristics:}

\begin{itemize}
\tightlist
\item
  \textbf{Single Point of Control}: All processing at central location
\item
  \textbf{Thin Clients}: Minimal local processing capability
\item
  \textbf{Shared Resources}: CPU, memory, storage centrally managed
\item
  \textbf{Network Dependent}: Requires reliable network connectivity
\end{itemize}

\textbf{Advantages:}

\begin{itemize}
\tightlist
\item
  \textbf{Security}: Centralized data protection
\item
  \textbf{Management}: Easier system administration
\item
  \textbf{Cost}: Lower client-side hardware costs
\end{itemize}

\textbf{Disadvantages:}

\begin{itemize}
\tightlist
\item
  \textbf{Single Point of Failure}: Server downtime affects all users
\item
  \textbf{Network Bottleneck}: Heavy reliance on network performance
\end{itemize}

\end{solutionbox}
\begin{mnemonicbox}
``SSNG: Single control, Shared resources, Network
dependent, Greater security''

\end{mnemonicbox}
\subsection*{Question 5(c) [07 marks]}\label{q5c}

\textbf{What is ipv4 addressing scheme? Explain with a neat and clean
diagram with its working.}

\begin{solutionbox}

\textbf{IPv4 (Internet Protocol version 4)} uses 32-bit addresses for
network identification.

\textbf{IPv4 Address Structure:}

\begin{verbatim}
 0                   1                   2                   3
 0 1 2 3 4 5 6 7 8 9 0 1 2 3 4 5 6 7 8 9 0 1 2 3 4 5 6 7 8 9 0 1
+{-+{-}+{-}+{-}+{-}+{-}+{-}+{-}+{-}+{-}+{-}+{-}+{-}+{-}+{-}+{-}+{-}+{-}+{-}+{-}+{-}+{-}+{-}+{-}+{-}+{-}+{-}+{-}+{-}+{-}+{-}+{-}+}
|                        Network Address                        |
+{-+{-}+{-}+{-}+{-}+{-}+{-}+{-}+{-}+{-}+{-}+{-}+{-}+{-}+{-}+{-}+{-}+{-}+{-}+{-}+{-}+{-}+{-}+{-}+{-}+{-}+{-}+{-}+{-}+{-}+{-}+{-}+}
|                         Host Address                          |
+{-+{-}+{-}+{-}+{-}+{-}+{-}+{-}+{-}+{-}+{-}+{-}+{-}+{-}+{-}+{-}+{-}+{-}+{-}+{-}+{-}+{-}+{-}+{-}+{-}+{-}+{-}+{-}+{-}+{-}+{-}+{-}+}
\end{verbatim}

\textbf{IPv4 Address Classes:}

{\def\LTcaptype{none} % do not increment counter
\begin{longtable}[]{@{}lllll@{}}
\toprule\noalign{}
Class & Range & Network Bits & Host Bits & Default Subnet Mask \\
\midrule\noalign{}
\endhead
\bottomrule\noalign{}
\endlastfoot
\textbf{A} & 1-126 & 8 & 24 & 255.0.0.0 \\
\textbf{B} & 128-191 & 16 & 16 & 255.255.0.0 \\
\textbf{C} & 192-223 & 24 & 8 & 255.255.255.0 \\
\textbf{D} & 224-239 & Multicast & - & - \\
\textbf{E} & 240-255 & Experimental & - & - \\
\end{longtable}
}

\textbf{IPv4 Packet Header:}

\begin{verbatim}
 0                   1                   2                   3
 0 1 2 3 4 5 6 7 8 9 0 1 2 3 4 5 6 7 8 9 0 1 2 3 4 5 6 7 8 9 0 1
+{-+{-}+{-}+{-}+{-}+{-}+{-}+{-}+{-}+{-}+{-}+{-}+{-}+{-}+{-}+{-}+{-}+{-}+{-}+{-}+{-}+{-}+{-}+{-}+{-}+{-}+{-}+{-}+{-}+{-}+{-}+{-}+}
|Version|  IHL  |Type of Service|          Total Length         |
+{-+{-}+{-}+{-}+{-}+{-}+{-}+{-}+{-}+{-}+{-}+{-}+{-}+{-}+{-}+{-}+{-}+{-}+{-}+{-}+{-}+{-}+{-}+{-}+{-}+{-}+{-}+{-}+{-}+{-}+{-}+{-}+}
|         Identification        |Flags|      Fragment Offset    |
+{-+{-}+{-}+{-}+{-}+{-}+{-}+{-}+{-}+{-}+{-}+{-}+{-}+{-}+{-}+{-}+{-}+{-}+{-}+{-}+{-}+{-}+{-}+{-}+{-}+{-}+{-}+{-}+{-}+{-}+{-}+{-}+}
|  Time to Live |    Protocol   |         Header Checksum       |
+{-+{-}+{-}+{-}+{-}+{-}+{-}+{-}+{-}+{-}+{-}+{-}+{-}+{-}+{-}+{-}+{-}+{-}+{-}+{-}+{-}+{-}+{-}+{-}+{-}+{-}+{-}+{-}+{-}+{-}+{-}+{-}+}
|                       Source Address                          |
+{-+{-}+{-}+{-}+{-}+{-}+{-}+{-}+{-}+{-}+{-}+{-}+{-}+{-}+{-}+{-}+{-}+{-}+{-}+{-}+{-}+{-}+{-}+{-}+{-}+{-}+{-}+{-}+{-}+{-}+{-}+{-}+}
|                    Destination Address                        |
+{-+{-}+{-}+{-}+{-}+{-}+{-}+{-}+{-}+{-}+{-}+{-}+{-}+{-}+{-}+{-}+{-}+{-}+{-}+{-}+{-}+{-}+{-}+{-}+{-}+{-}+{-}+{-}+{-}+{-}+{-}+{-}+}
|                    Options                    |    Padding    |
+{-+{-}+{-}+{-}+{-}+{-}+{-}+{-}+{-}+{-}+{-}+{-}+{-}+{-}+{-}+{-}+{-}+{-}+{-}+{-}+{-}+{-}+{-}+{-}+{-}+{-}+{-}+{-}+{-}+{-}+{-}+{-}+}
\end{verbatim}

\textbf{Working Process:}

\begin{itemize}
\tightlist
\item
  \textbf{Address Assignment}: Network administrator assigns IP
  addresses
\item
  \textbf{Routing Decision}: Router examines destination IP
\item
  \textbf{Subnet Determination}: Apply subnet mask to find network
\item
  \textbf{Packet Forwarding}: Route to appropriate network interface
\end{itemize}

\textbf{Special Addresses:}

\begin{itemize}
\tightlist
\item
  \textbf{Loopback}: 127.0.0.1 (localhost)
\item
  \textbf{Private}: 10.x.x.x, 172.16-31.x.x, 192.168.x.x
\item
  \textbf{Broadcast}: 255.255.255.255
\end{itemize}

\textbf{Limitations:}

\begin{itemize}
\tightlist
\item
  \textbf{Address Exhaustion}: Only 4.3 billion addresses
\item
  \textbf{Inefficient Allocation}: Class-based wastage
\end{itemize}

\end{solutionbox}
\begin{mnemonicbox}
``ABCDE: Address classes A, B, C, D multicast, E
experimental''

\end{mnemonicbox}
\subsection*{Question 5(a OR) [03
marks]}\label{question-5a-or-03-marks}

\textbf{What is 5G? List out Features of 5G.}

\begin{solutionbox}

\textbf{5G (Fifth Generation)} is the latest mobile communication
standard with enhanced capabilities.

\textbf{Features of 5G:}

\begin{itemize}
\tightlist
\item
  \textbf{Ultra-High Speed}: Up to 10 Gbps data rates
\item
  \textbf{Ultra-Low Latency}: Less than 1ms response time
\item
  \textbf{Massive Connectivity}: 1 million devices per km^{2}
\item
  \textbf{Network Slicing}: Virtual dedicated networks
\item
  \textbf{Enhanced Mobile Broadband}: Improved user experience
\end{itemize}

\textbf{Key Technologies:}

\begin{itemize}
\tightlist
\item
  Millimeter wave, Massive MIMO, Beamforming
\end{itemize}

\end{solutionbox}
\begin{mnemonicbox}
``UUMNE: Ultra-speed, Ultra-low latency, Massive
connectivity, Network slicing, Enhanced broadband''

\end{mnemonicbox}
\subsection*{Question 5(b OR) [04
marks]}\label{question-5b-or-04-marks}

\textbf{Explain Distributed Computing}

\begin{solutionbox}

\textbf{Distributed Computing} spreads processing across multiple
interconnected computers.

\textbf{Architecture:}

\begin{center}
\textbf{Mermaid Diagram (Code)}
\begin{verbatim}
{Shaded}
{Highlighting}[]
graph LR
    subgraph "Distributed System"
        A[Node 1] {{-}{-}{} B[Node 2]}
        B {{-}{-}{} C[Node 3]}
        C {{-}{-}{} D[Node 4]}
        A {{-}{-}{} D}
    end
    
    E[Network] {-{-}{} A}
    E {-{-}{} B}
    E {-{-}{} C}
    E {-{-}{} D}
{Highlighting}
{Shaded}
\end{verbatim}
\end{center}

\textbf{Characteristics:}

\begin{itemize}
\tightlist
\item
  \textbf{Resource Sharing}: Distributed processing and storage
\item
  \textbf{Scalability}: Add more nodes to increase capacity
\item
  \textbf{Fault Tolerance}: System continues if some nodes fail
\item
  \textbf{Location Transparency}: Users unaware of resource locations
\end{itemize}

\textbf{Advantages:}

\begin{itemize}
\tightlist
\item
  \textbf{Reliability}: No single point of failure
\item
  \textbf{Performance}: Parallel processing capabilities
\item
  \textbf{Cost-effectiveness}: Use commodity hardware
\end{itemize}

\textbf{Examples:}

\begin{itemize}
\tightlist
\item
  Cloud computing, peer-to-peer networks, grid computing
\end{itemize}

\end{solutionbox}
\begin{mnemonicbox}
``RSFL: Resource sharing, Scalability, Fault
tolerance, Location transparency''

\end{mnemonicbox}
\subsection*{Question 5(c OR) [07
marks]}\label{question-5c-or-07-marks}

\textbf{Explain Data Link Layer Protocol.}

\begin{solutionbox}

\textbf{Data Link Layer} provides reliable data transfer between
adjacent network nodes.

\textbf{Functions:}

\begin{itemize}
\tightlist
\item
  \textbf{Framing}: Organize bits into frames
\item
  \textbf{Error Detection}: Identify transmission errors
\item
  \textbf{Error Correction}: Fix detected errors
\item
  \textbf{Flow Control}: Manage data transmission rate
\item
  \textbf{Access Control}: Coordinate shared media access
\end{itemize}

\textbf{Frame Structure:}

\begin{verbatim}
+{-{-}{-}{-}{-}{-}{-}{-}{-}{-}+{-}{-}{-}{-}{-}{-}{-}{-}{-}{-}+{-}{-}{-}{-}{-}{-}{-}{-}{-}{-}+{-}{-}{-}{-}{-}{-}{-}{-}{-}{-}+{-}{-}{-}{-}{-}{-}{-}{-}{-}{-}+}
| Start    | Address  | Control  | Data     | FCS      |
| Delimiter| Field    | Field    | Field    | (CRC)    |
+{-{-}{-}{-}{-}{-}{-}{-}{-}{-}+{-}{-}{-}{-}{-}{-}{-}{-}{-}{-}+{-}{-}{-}{-}{-}{-}{-}{-}{-}{-}+{-}{-}{-}{-}{-}{-}{-}{-}{-}{-}+{-}{-}{-}{-}{-}{-}{-}{-}{-}{-}+}
\end{verbatim}

\textbf{Error Detection Methods:}

{\def\LTcaptype{none} % do not increment counter
\begin{longtable}[]{@{}lll@{}}
\toprule\noalign{}
Method & Description & Capability \\
\midrule\noalign{}
\endhead
\bottomrule\noalign{}
\endlastfoot
\textbf{Parity Check} & Single bit addition & Detect single-bit
errors \\
\textbf{Checksum} & Arithmetic sum & Detect multiple errors \\
\textbf{CRC} & Polynomial division & Detect burst errors \\
\end{longtable}
}

\textbf{Flow Control Protocols:}

\begin{itemize}
\tightlist
\item
  \textbf{Stop-and-Wait}: Send one frame, wait for ACK
\item
  \textbf{Sliding Window}: Multiple frames in transit
\item
  \textbf{Stop-and-Wait ARQ}: Add error recovery
\item
  \textbf{Go-Back-N ARQ}: Retransmit from error point
\item
  \textbf{Selective Repeat}: Retransmit only error frames
\end{itemize}

\textbf{Access Control Methods:}

\begin{itemize}
\tightlist
\item
  \textbf{CSMA/CD}: Carrier Sense Multiple Access with Collision
  Detection
\item
  \textbf{CSMA/CA}: Collision Avoidance
\item
  \textbf{Token Passing}: Controlled access using token
\end{itemize}

\textbf{Protocol Examples:}

\begin{itemize}
\tightlist
\item
  Ethernet, PPP, HDLC, LLC
\end{itemize}

\textbf{Working Process:}

\begin{verbatim}
sequenceDiagram
    participant S as Sender
    participant R as Receiver
    
    S{-R: Data Frame}
    R{-S: ACK Frame}
    S{-R: Next Data Frame}
    Note over R: Error Detected
    R{-S: NAK Frame}
    S{-R: Retransmit Frame}
\end{verbatim}

\end{solutionbox}
\begin{mnemonicbox}
``FECFA: Framing, Error detection, Correction, Flow
control, Access control''

\end{mnemonicbox}

\end{document}
