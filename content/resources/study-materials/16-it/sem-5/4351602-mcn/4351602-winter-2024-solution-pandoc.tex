\documentclass[10pt,a4paper]{article}

% content/resources/templates/preamble.tex
\usepackage[margin=0.6in]{geometry}
\author{Milav Dabgar}
\usepackage{amsmath,amssymb,amsthm}
\usepackage{booktabs}
\usepackage{multirow}
\usepackage{xcolor}
\usepackage{tcolorbox}
\tcbuselibrary{breakable,skins}
\usepackage[colorlinks=true,linkcolor=blue]{hyperref}
\usepackage{titlesec}
\usepackage{enumitem}
\usepackage{tikz}
\usepackage{pgfplots}
\usepackage{circuitikz}
\usepackage[version=4]{mhchem}
\usepackage{longtable}
\usepackage{array}
\usepackage{float}
\usepackage{caption}
\usepackage{listings}

\lstset{
  basicstyle=\small\ttfamily,
  breaklines=true,
  breakatwhitespace=false,
  postbreak=\mbox{\textcolor{red}{$\hookrightarrow$}\space},
  float=false,
  numbers=left,
  numberstyle=\tiny\color{gray},
  numbersep=10pt,
  xleftmargin=2em,
  keywordstyle=\color{blue},
  commentstyle=\color{green!60!black},
  stringstyle=\color{purple},
  backgroundcolor=\color{gray!5},
  showstringspaces=false,
  tabsize=2,
  captionpos=b,
  keepspaces=true,
  columns=flexible
}

\pgfplotsset{compat=1.18}
\usetikzlibrary{shapes,arrows,positioning,calc,patterns,decorations.pathmorphing,decorations.markings,arrows.meta}

% Color scheme
\definecolor{headcolor}{RGB}{0,102,204}
\definecolor{keycolor}{RGB}{220,20,60}
\definecolor{solutioncolor}{RGB}{34,139,34}
\definecolor{mnemoniccolor}{RGB}{148,0,211}
\definecolor{codecolor}{RGB}{0,0,100}

% Spacing
\setlength{\parskip}{3pt}
\setlist[itemize]{nosep}
\setlist[enumerate]{nosep}

% Title formatting
\titleformat{\section}{\Large\bfseries\color{headcolor}}{\thesection}{1em}{}
\titleformat{\subsection}{\large\bfseries\color{headcolor}}{\thesubsection}{1em}{}

% Pandoc tightlist compatibility
\providecommand{\tightlist}{%
  \setlength{\itemsep}{0pt}\setlength{\parskip}{0pt}}

% Pandoc longtable compatibility
\newcounter{none}
\def\thenone{}


% content/resources/templates/english-boxes.tex
% This file is currently empty - it exists to maintain consistency with the import structure.
% Add custom environments here if needed in the future.


\begin{document}

\begin{center}
{\Huge\bfseries\color{headcolor} Subject Name Solutions}\\[5pt]
{\LARGE 4351602 -- Winter 2024}\\[3pt]
{\large Semester 1 Study Material}\\[3pt]
{\normalsize\textit{Detailed Solutions and Explanations}}
\end{center}

\vspace{10pt}

\subsection*{Question 1(a) [3 marks]}\label{q1a}

\textbf{List out types of congestion control and explain any one}

\begin{solutionbox}

{\def\LTcaptype{none} % do not increment counter
\begin{longtable}[]{@{}ll@{}}
\toprule\noalign{}
Type & Description \\
\midrule\noalign{}
\endhead
\bottomrule\noalign{}
\endlastfoot
\textbf{Open-Loop} & Prevents congestion before it occurs \\
\textbf{Closed-Loop} & Manages congestion after detection \\
\end{longtable}
}

\textbf{Open-Loop Congestion Control Explanation:}

\begin{itemize}
\tightlist
\item
  \textbf{Prevention approach}: Takes action before congestion occurs
\item
  \textbf{Traffic shaping}: Controls data rate at sender
\item
  \textbf{Admission control}: Limits new connections during high traffic
\item
  \textbf{Load shedding}: Drops packets when buffer full
\end{itemize}

\end{solutionbox}
\begin{mnemonicbox}
``Open Prevents Traffic Admission Load''

\end{mnemonicbox}
\begin{center}\rule{0.5\linewidth}{0.5pt}\end{center}

\subsection*{Question 1(b) [4 marks]}\label{q1b}

\textbf{Explain Address Resolution Protocol briefly}

\begin{solutionbox}

\textbf{ARP (Address Resolution Protocol)} maps IP addresses to MAC
addresses in local networks.

\textbf{Working Process:}

\begin{itemize}
\tightlist
\item
  \textbf{ARP Request}: Broadcast message asking ``Who has IP X?''
\item
  \textbf{ARP Reply}: Target device responds with its MAC address
\item
  \textbf{ARP Cache}: Stores IP-MAC mappings for future use
\item
  \textbf{Dynamic mapping}: Updates entries automatically
\end{itemize}


{\def\LTcaptype{none} % do not increment counter
\vspace{-5pt}
\captionof{table}{ARP Message Types}
\vspace{-10pt}
\begin{longtable}[]{@{}lll@{}}
\toprule\noalign{}
Type & Purpose & Broadcast \\
\midrule\noalign{}
\endhead
\bottomrule\noalign{}
\endlastfoot
ARP Request & Find MAC address & Yes \\
ARP Reply & Provide MAC address & No \\
\end{longtable}
}

\end{solutionbox}
\begin{mnemonicbox}
``ARP Requests Broadcast, Replies Cache Dynamic''

\end{mnemonicbox}
\begin{center}\rule{0.5\linewidth}{0.5pt}\end{center}

\subsection*{Question 1(c) [7 marks]}\label{q1c}

\textbf{Explain TCP/IP model with all layers and functionalities of each
layer}

\begin{solutionbox}

\textbf{TCP/IP Model} is a four-layer network protocol stack for
internet communication.

\begin{center}
\textbf{Mermaid Diagram (Code)}
\begin{verbatim}
{Shaded}
{Highlighting}[]
graph LR
    A[Application Layer] {-{-}{} B[Transport Layer]}
    B {-{-}{} C[Internet Layer] }
    C {-{-}{} D[Network Access Layer]}
{Highlighting}
{Shaded}
\end{verbatim}
\end{center}

\textbf{Layer Functions:}

{\def\LTcaptype{none} % do not increment counter
\begin{longtable}[]{@{}lll@{}}
\toprule\noalign{}
Layer & Function & Protocols \\
\midrule\noalign{}
\endhead
\bottomrule\noalign{}
\endlastfoot
\textbf{Application} & User interface, network services & HTTP, FTP,
SMTP \\
\textbf{Transport} & End-to-end communication & TCP, UDP \\
\textbf{Internet} & Routing, addressing & IP, ICMP \\
\textbf{Network Access} & Physical transmission & Ethernet, WiFi \\
\end{longtable}
}

\begin{itemize}
\tightlist
\item
  \textbf{Application Layer}: Provides network services to applications
\item
  \textbf{Transport Layer}: Ensures reliable data delivery with error
  control
\item
  \textbf{Internet Layer}: Routes packets across networks using IP
  addressing\\
\item
  \textbf{Network Access Layer}: Handles physical data transmission
\end{itemize}

\end{solutionbox}
\begin{mnemonicbox}
``All Transport Internet Network''

\end{mnemonicbox}
\begin{center}\rule{0.5\linewidth}{0.5pt}\end{center}

\subsection*{Question 1(c OR) [7
marks]}\label{question-1c-or-7-marks}

\textbf{Explain OSI model with each layer functionality}

\begin{solutionbox}

\textbf{OSI Model} is a seven-layer reference model for network
communication.

\begin{center}
\textbf{Mermaid Diagram (Code)}
\begin{verbatim}
{Shaded}
{Highlighting}[]
graph LR
    A[Application Layer 7] {-{-}{} B[Presentation Layer 6]}
    B {-{-}{} C[Session Layer 5]}
    C {-{-}{} D[Transport Layer 4]}
    D {-{-}{} E[Network Layer 3]}
    E {-{-}{} F[Data Link Layer 2]}
    F {-{-}{} G[Physical Layer 1]}
{Highlighting}
{Shaded}
\end{verbatim}
\end{center}

\textbf{Layer Functionalities:}

{\def\LTcaptype{none} % do not increment counter
\begin{longtable}[]{@{}lll@{}}
\toprule\noalign{}
Layer & Function & Examples \\
\midrule\noalign{}
\endhead
\bottomrule\noalign{}
\endlastfoot
\textbf{Physical (1)} & Bit transmission & Cables, signals \\
\textbf{Data Link (2)} & Frame delivery & Ethernet, switches \\
\textbf{Network (3)} & Routing packets & IP, routers \\
\textbf{Transport (4)} & End-to-end delivery & TCP, UDP \\
\textbf{Session (5)} & Dialog management & NetBIOS \\
\textbf{Presentation (6)} & Data formatting & SSL, compression \\
\textbf{Application (7)} & User interface & HTTP, email \\
\end{longtable}
}

\end{solutionbox}
\begin{mnemonicbox}
``Physical Data Network Transport Session
Presentation Application''

\end{mnemonicbox}
\begin{center}\rule{0.5\linewidth}{0.5pt}\end{center}

\subsection*{Question 2(a) [3 marks]}\label{q2a}

\textbf{Explain subnetting in short}

\begin{solutionbox}

\textbf{Subnetting} divides a large network into smaller sub-networks
for better management.

\textbf{Key Concepts:}

\begin{itemize}
\tightlist
\item
  \textbf{Subnet mask}: Defines network and host portions
\item
  \textbf{Network efficiency}: Reduces broadcast traffic
\item
  \textbf{Address conservation}: Better IP utilization
\item
  \textbf{Security}: Isolates network segments
\end{itemize}

\textbf{Example:} Network: 192.168.1.0/24 \rightarrow Subnets: 192.168.1.0/26,
192.168.1.64/26

\end{solutionbox}
\begin{mnemonicbox}
``Subnet Network Efficiency Address Security''

\end{mnemonicbox}
\begin{center}\rule{0.5\linewidth}{0.5pt}\end{center}

\subsection*{Question 2(b) [4 marks]}\label{q2b}

\textbf{Explain stop and wait ARQ protocol of data link layer with
example}

\begin{solutionbox}

\textbf{Stop and Wait ARQ} is a flow control protocol ensuring reliable
data transmission.

\textbf{Working Process:}

\begin{itemize}
\tightlist
\item
  \textbf{Send frame}: Transmitter sends one frame
\item
  \textbf{Wait for ACK}: Sender waits for acknowledgment
\item
  \textbf{Timeout}: Retransmits if no ACK received
\item
  \textbf{Next frame}: Sends next frame after ACK
\end{itemize}

\begin{verbatim}
Sender          Receiver
  |     Frame 1   |
  |{-{-}{-}{-}{-}{-}{-}{-}{-}{-}{-}{-}{-}{-}|}
  |               |
  |      ACK      |
  |{{-}{-}{-}{-}{-}{-}{-}{-}{-}{-}{-}{-}{-}{-}|}
  |     Frame 2   |
  |{-{-}{-}{-}{-}{-}{-}{-}{-}{-}{-}{-}{-}{-}|}
\end{verbatim}

\textbf{Example:} File transfer where each packet waits for confirmation
before sending next.

\end{solutionbox}
\begin{mnemonicbox}
``Send Wait Timeout Next''

\end{mnemonicbox}
\begin{center}\rule{0.5\linewidth}{0.5pt}\end{center}

\subsection*{Question 2(c) [7 marks]}\label{q2c}

\textbf{Draw diagram of IPv4 datagram Header and explain it}

\begin{solutionbox}

\textbf{IPv4 Header} contains control information for packet routing and
delivery.

\begin{verbatim}
 0                   1                   2                   3
 0 1 2 3 4 5 6 7 8 9 0 1 2 3 4 5 6 7 8 9 0 1 2 3 4 5 6 7 8 9 0 1
+{-+{-}+{-}+{-}+{-}+{-}+{-}+{-}+{-}+{-}+{-}+{-}+{-}+{-}+{-}+{-}+{-}+{-}+{-}+{-}+{-}+{-}+{-}+{-}+{-}+{-}+{-}+{-}+{-}+{-}+{-}+{-}+}
|Version|  IHL  |Type of Service|          Total Length         |
+{-+{-}+{-}+{-}+{-}+{-}+{-}+{-}+{-}+{-}+{-}+{-}+{-}+{-}+{-}+{-}+{-}+{-}+{-}+{-}+{-}+{-}+{-}+{-}+{-}+{-}+{-}+{-}+{-}+{-}+{-}+{-}+}
|         Identification        |Flags|      Fragment Offset    |
+{-+{-}+{-}+{-}+{-}+{-}+{-}+{-}+{-}+{-}+{-}+{-}+{-}+{-}+{-}+{-}+{-}+{-}+{-}+{-}+{-}+{-}+{-}+{-}+{-}+{-}+{-}+{-}+{-}+{-}+{-}+{-}+}
|  Time to Live |    Protocol   |         Header Checksum       |
+{-+{-}+{-}+{-}+{-}+{-}+{-}+{-}+{-}+{-}+{-}+{-}+{-}+{-}+{-}+{-}+{-}+{-}+{-}+{-}+{-}+{-}+{-}+{-}+{-}+{-}+{-}+{-}+{-}+{-}+{-}+{-}+}
|                       Source Address                          |
+{-+{-}+{-}+{-}+{-}+{-}+{-}+{-}+{-}+{-}+{-}+{-}+{-}+{-}+{-}+{-}+{-}+{-}+{-}+{-}+{-}+{-}+{-}+{-}+{-}+{-}+{-}+{-}+{-}+{-}+{-}+{-}+}
|                    Destination Address                        |
+{-+{-}+{-}+{-}+{-}+{-}+{-}+{-}+{-}+{-}+{-}+{-}+{-}+{-}+{-}+{-}+{-}+{-}+{-}+{-}+{-}+{-}+{-}+{-}+{-}+{-}+{-}+{-}+{-}+{-}+{-}+{-}+}
\end{verbatim}

\textbf{Field Explanations:}

{\def\LTcaptype{none} % do not increment counter
\begin{longtable}[]{@{}lll@{}}
\toprule\noalign{}
Field & Size & Function \\
\midrule\noalign{}
\endhead
\bottomrule\noalign{}
\endlastfoot
\textbf{Version} & 4 bits & IP version (4 for IPv4) \\
\textbf{IHL} & 4 bits & Header length \\
\textbf{Type of Service} & 8 bits & Quality of service \\
\textbf{Total Length} & 16 bits & Packet size \\
\textbf{TTL} & 8 bits & Hop limit \\
\textbf{Protocol} & 8 bits & Next layer protocol \\
\textbf{Source/Dest Address} & 32 bits each & IP addresses \\
\end{longtable}
}

\end{solutionbox}
\begin{mnemonicbox}
``Version IHL Service Total TTL Protocol Source
Destination''

\end{mnemonicbox}
\begin{center}\rule{0.5\linewidth}{0.5pt}\end{center}

\subsection*{Question 2(a OR) [3
marks]}\label{question-2a-or-3-marks}

\textbf{What is HTTPS? List important key features of HTTPS}

\begin{solutionbox}

\textbf{HTTPS (HTTP Secure)} is encrypted HTTP using SSL/TLS for secure
web communication.

\textbf{Key Features:}

\begin{itemize}
\tightlist
\item
  \textbf{Encryption}: Data encrypted in transit
\item
  \textbf{Authentication}: Verifies server identity
\item
  \textbf{Data integrity}: Prevents data tampering
\item
  \textbf{Trust}: SSL certificates provide validation
\end{itemize}

\textbf{Security Benefits:}

\begin{itemize}
\tightlist
\item
  Protects sensitive information
\item
  Prevents man-in-the-middle attacks
\item
  Search engine ranking boost
\end{itemize}

\end{solutionbox}
\begin{mnemonicbox}
``HTTPS Encrypts Authentication Data Trust''

\end{mnemonicbox}
\begin{center}\rule{0.5\linewidth}{0.5pt}\end{center}

\subsection*{Question 2(b OR) [4
marks]}\label{question-2b-or-4-marks}

\textbf{Give Answer of any two:}

\begin{solutionbox}

\textbf{1) How many bits HOST ID use by class B and C?}

\begin{itemize}
\tightlist
\item
  \textbf{Class B}: 16 bits for Host ID (65,534 hosts)
\item
  \textbf{Class C}: 8 bits for Host ID (254 hosts)
\end{itemize}

\textbf{2) What is IP range for Class A and D?}

\begin{itemize}
\tightlist
\item
  \textbf{Class A}: 1.0.0.0 to 126.255.255.255
\item
  \textbf{Class D}: 224.0.0.0 to 239.255.255.255 (Multicast)
\end{itemize}

{\def\LTcaptype{none} % do not increment counter
\begin{longtable}[]{@{}lll@{}}
\toprule\noalign{}
Class & Range & Host Bits \\
\midrule\noalign{}
\endhead
\bottomrule\noalign{}
\endlastfoot
B & 128.0.0.0 - 191.255.255.255 & 16 bits \\
C & 192.0.0.0 - 223.255.255.255 & 8 bits \\
A & 1.0.0.0 - 126.255.255.255 & 24 bits \\
D & 224.0.0.0 - 239.255.255.255 & Multicast \\
\end{longtable}
}

\end{solutionbox}
\begin{mnemonicbox}
``B=16,

C=8,

A=1-126,

D=224-239''


\end{mnemonicbox}
\begin{center}\rule{0.5\linewidth}{0.5pt}\end{center}

\subsection*{Question 2(c OR) [7
marks]}\label{question-2c-or-7-marks}

\textbf{Explain classful IPv4 addresses scheme}

\begin{solutionbox}

\textbf{Classful IPv4 Addressing} divides IP address space into five
classes based on first octets.

\textbf{Address Classes:}

{\def\LTcaptype{none} % do not increment counter
\begin{longtable}[]{@{}lllll@{}}
\toprule\noalign{}
Class & Range & Network Bits & Host Bits & Usage \\
\midrule\noalign{}
\endhead
\bottomrule\noalign{}
\endlastfoot
\textbf{A} & 1-126 & 8 & 24 & Large networks \\
\textbf{B} & 128-191 & 16 & 16 & Medium networks \\
\textbf{C} & 192-223 & 24 & 8 & Small networks \\
\textbf{D} & 224-239 & - & - & Multicast \\
\textbf{E} & 240-255 & - & - & Experimental \\
\end{longtable}
}

\begin{verbatim}
pie title IPv4 Address Classes
    "Class A (50\%)" : 50
    "Class B (25\%)" : 25
    "Class C (12.5\%)" : 12.5
    "Class D (6.25\%)" : 6.25
    "Class E (6.25\%)" : 6.25
\end{verbatim}

\textbf{Characteristics:}

\begin{itemize}
\tightlist
\item
  \textbf{Class A}: 16.7 million hosts per network
\item
  \textbf{Class B}: 65,534 hosts per network\\
\item
  \textbf{Class C}: 254 hosts per network
\item
  \textbf{Limitations}: Address wastage, inflexible allocation
\end{itemize}

\end{solutionbox}
\begin{mnemonicbox}
``A-Large, B-Medium, C-Small, D-Multicast,
E-Experimental''

\end{mnemonicbox}
\begin{center}\rule{0.5\linewidth}{0.5pt}\end{center}

\subsection*{Question 3(a) [3 marks]}\label{q3a}

\textbf{List out types of applications uses mobile computing}

\begin{solutionbox}

\textbf{Mobile Computing Applications:}

{\def\LTcaptype{none} % do not increment counter
\begin{longtable}[]{@{}ll@{}}
\toprule\noalign{}
Type & Examples \\
\midrule\noalign{}
\endhead
\bottomrule\noalign{}
\endlastfoot
\textbf{Communication} & WhatsApp, Email, Video calls \\
\textbf{Navigation} & GPS, Google Maps \\
\textbf{E-commerce} & Shopping apps, Mobile banking \\
\textbf{Entertainment} & Games, Streaming, Social media \\
\textbf{Business} & CRM, Sales tracking \\
\textbf{Healthcare} & Health monitoring, Telemedicine \\
\end{longtable}
}

\begin{itemize}
\tightlist
\item
  \textbf{Location-based services}: GPS navigation, location sharing
\item
  \textbf{Mobile payments}: Digital wallets, UPI transactions
\item
  \textbf{Social networking}: Facebook, Instagram, Twitter
\end{itemize}

\end{solutionbox}
\begin{mnemonicbox}
``Communication Navigation E-commerce Entertainment
Business Healthcare''

\end{mnemonicbox}
\begin{center}\rule{0.5\linewidth}{0.5pt}\end{center}

\subsection*{Question 3(b) [4 marks]}\label{q3b}

\textbf{Explain use of Gateways and list types of Gateways}

\begin{solutionbox}

\textbf{Gateway} connects networks with different protocols and
architectures.

\textbf{Uses of Gateways:}

\begin{itemize}
\tightlist
\item
  \textbf{Protocol conversion}: Translates between different protocols
\item
  \textbf{Network bridging}: Connects dissimilar networks
\item
  \textbf{Security}: Firewall and access control
\item
  \textbf{Data filtering}: Manages traffic flow
\end{itemize}

\textbf{Types of Gateways:}

{\def\LTcaptype{none} % do not increment counter
\begin{longtable}[]{@{}ll@{}}
\toprule\noalign{}
Type & Function \\
\midrule\noalign{}
\endhead
\bottomrule\noalign{}
\endlastfoot
\textbf{Network Gateway} & Routes between networks \\
\textbf{Internet Gateway} & Connects to internet \\
\textbf{Protocol Gateway} & Protocol translation \\
\textbf{Application Gateway} & Application-level filtering \\
\end{longtable}
}

\end{solutionbox}
\begin{mnemonicbox}
``Gateways Convert Bridge Secure Filter''

\end{mnemonicbox}
\begin{center}\rule{0.5\linewidth}{0.5pt}\end{center}

\subsection*{Question 3(c) [7 marks]}\label{q3c}

\textbf{Draw and explain architecture of mobile computing}

\begin{solutionbox}

\textbf{Mobile Computing Architecture} consists of three main components
working together.

\begin{center}
\textbf{Mermaid Diagram (Code)}
\begin{verbatim}
{Shaded}
{Highlighting}[]
graph TD
    A[Mobile Device] {{-}{-}{} B[Communication Network]}
    B {{-}{-}{} C[Fixed Infrastructure]}
    
    A1[Hardware] {-{-}{} A}
    A2[OS \& Apps] {-{-}{} A}
    A3[Data] {-{-}{} A}
    
    B1[Wireless Network] {-{-}{} B}
    B2[Protocols] {-{-}{} B}
    B3[Base Stations] {-{-}{} B}
    
    C1[Servers] {-{-}{} C}
    C2[Databases] {-{-}{} C}
    C3[Internet] {-{-}{} C}
{Highlighting}
{Shaded}
\end{verbatim}
\end{center}

\textbf{Architecture Components:}

{\def\LTcaptype{none} % do not increment counter
\begin{longtable}[]{@{}
  >{\raggedright\arraybackslash}p{(\linewidth - 4\tabcolsep) * \real{0.3548}}
  >{\raggedright\arraybackslash}p{(\linewidth - 4\tabcolsep) * \real{0.3226}}
  >{\raggedright\arraybackslash}p{(\linewidth - 4\tabcolsep) * \real{0.3226}}@{}}
\toprule\noalign{}
\begin{minipage}[b]{\linewidth}\raggedright
Component
\end{minipage} & \begin{minipage}[b]{\linewidth}\raggedright
Elements
\end{minipage} & \begin{minipage}[b]{\linewidth}\raggedright
Function
\end{minipage} \\
\midrule\noalign{}
\endhead
\bottomrule\noalign{}
\endlastfoot
\textbf{Mobile Unit} & Devices, OS, Apps & User interface, processing \\
\textbf{Communication Network} & Wireless links, protocols & Data
transmission \\
\textbf{Fixed Infrastructure} & Servers, databases & Backend services \\
\end{longtable}
}

\textbf{Key Features:}

\begin{itemize}
\tightlist
\item
  \textbf{Mobility}: Users can move while maintaining connectivity
\item
  \textbf{Wireless communication}: Radio waves for data transmission
\item
  \textbf{Distributed computing}: Processing across multiple devices
\item
  \textbf{Location independence}: Access services from anywhere
\end{itemize}

\textbf{Challenges:}

\begin{itemize}
\tightlist
\item
  \textbf{Limited bandwidth}: Wireless networks have capacity
  constraints
\item
  \textbf{Battery life}: Mobile devices have power limitations
\item
  \textbf{Security}: Wireless transmission vulnerable to attacks
\end{itemize}

\end{solutionbox}
\begin{mnemonicbox}
``Mobile Communication Fixed - Mobility Wireless
Distributed Location''

\end{mnemonicbox}
\begin{center}\rule{0.5\linewidth}{0.5pt}\end{center}

\subsection*{Question 3(a OR) [3
marks]}\label{question-3a-or-3-marks}

\textbf{List security standards in mobile computing}

\begin{solutionbox}

\textbf{Mobile Computing Security Standards:}

{\def\LTcaptype{none} % do not increment counter
\begin{longtable}[]{@{}ll@{}}
\toprule\noalign{}
Standard & Purpose \\
\midrule\noalign{}
\endhead
\bottomrule\noalign{}
\endlastfoot
\textbf{WPA3} & WiFi security protocol \\
\textbf{SSL/TLS} & Secure data transmission \\
\textbf{IPSec} & IP layer security \\
\textbf{EAP} & Authentication framework \\
\textbf{802.11i} & Wireless LAN security \\
\textbf{FIPS 140-2} & Cryptographic module standards \\
\end{longtable}
}

\begin{itemize}
\tightlist
\item
  \textbf{Authentication protocols}: Verify user identity
\item
  \textbf{Encryption standards}: Protect data confidentiality
\item
  \textbf{Access control}: Manage resource permissions
\end{itemize}

\end{solutionbox}
\begin{mnemonicbox}
``WPA SSL IPSec EAP 802.11i FIPS''

\end{mnemonicbox}
\begin{center}\rule{0.5\linewidth}{0.5pt}\end{center}

\subsection*{Question 3(b OR) [4
marks]}\label{question-3b-or-4-marks}

\textbf{Explain key functions of communication gateway}

\begin{solutionbox}

\textbf{Communication Gateway} manages data exchange between different
network systems.

\textbf{Key Functions:}

{\def\LTcaptype{none} % do not increment counter
\begin{longtable}[]{@{}ll@{}}
\toprule\noalign{}
Function & Description \\
\midrule\noalign{}
\endhead
\bottomrule\noalign{}
\endlastfoot
\textbf{Protocol Translation} & Converts between protocols \\
\textbf{Data Format Conversion} & Changes data formats \\
\textbf{Routing} & Directs messages to destinations \\
\textbf{Security} & Access control and filtering \\
\end{longtable}
}

\textbf{Detailed Functions:}

\begin{itemize}
\tightlist
\item
  \textbf{Message routing}: Determines optimal path for data
\item
  \textbf{Error handling}: Manages transmission errors and recovery
\item
  \textbf{Traffic management}: Controls data flow and congestion
\item
  \textbf{Authentication}: Verifies sender and receiver identity
\end{itemize}

\textbf{Benefits:}

\begin{itemize}
\tightlist
\item
  Enables interoperability between different systems
\item
  Centralizes network management
\item
  Provides security checkpoint
\end{itemize}

\end{solutionbox}
\begin{mnemonicbox}
``Protocol Data Routing Security - Message Error
Traffic Authentication''

\end{mnemonicbox}
\begin{center}\rule{0.5\linewidth}{0.5pt}\end{center}

\subsection*{Question 3(c OR) [7
marks]}\label{question-3c-or-7-marks}

\textbf{Explain use of middleware and list types of middleware}

\begin{solutionbox}

\textbf{Middleware} provides software layer between applications and
operating system for distributed computing.

\textbf{Uses of Middleware:}

\begin{itemize}
\tightlist
\item
  \textbf{Connectivity}: Links distributed applications
\item
  \textbf{Interoperability}: Enables different systems to work together
\item
  \textbf{Abstraction}: Hides complexity of underlying systems
\item
  \textbf{Scalability}: Supports system growth and expansion
\end{itemize}

\begin{center}
\textbf{Mermaid Diagram (Code)}
\begin{verbatim}
{Shaded}
{Highlighting}[]
graph LR
    A[Applications] {-{-}{} B[Middleware Layer]}
    B {-{-}{} C[Operating System]}
    B {-{-}{} D[Network Services]}
    B {-{-}{} E[Database Services]}
{Highlighting}
{Shaded}
\end{verbatim}
\end{center}

\textbf{Types of Middleware:}

{\def\LTcaptype{none} % do not increment counter
\begin{longtable}[]{@{}
  >{\raggedright\arraybackslash}p{(\linewidth - 4\tabcolsep) * \real{0.2308}}
  >{\raggedright\arraybackslash}p{(\linewidth - 4\tabcolsep) * \real{0.3846}}
  >{\raggedright\arraybackslash}p{(\linewidth - 4\tabcolsep) * \real{0.3846}}@{}}
\toprule\noalign{}
\begin{minipage}[b]{\linewidth}\raggedright
Type
\end{minipage} & \begin{minipage}[b]{\linewidth}\raggedright
Function
\end{minipage} & \begin{minipage}[b]{\linewidth}\raggedright
Examples
\end{minipage} \\
\midrule\noalign{}
\endhead
\bottomrule\noalign{}
\endlastfoot
\textbf{Message-Oriented} & Asynchronous communication & IBM MQ,
RabbitMQ \\
\textbf{Remote Procedure Call} & Synchronous communication & gRPC,
XML-RPC \\
\textbf{Object Request Broker} & Object communication & CORBA \\
\textbf{Database Middleware} & Database connectivity & ODBC, JDBC \\
\textbf{Transaction Processing} & Transaction management & Tuxedo \\
\textbf{Web Middleware} & Web services & Apache, IIS \\
\end{longtable}
}

\textbf{Benefits:}

\begin{itemize}
\tightlist
\item
  \textbf{Reduced complexity}: Simplifies application development
\item
  \textbf{Reusability}: Common services for multiple applications
\item
  \textbf{Maintainability}: Centralized management of services
\item
  \textbf{Platform independence}: Works across different systems
\end{itemize}

\end{solutionbox}
\begin{mnemonicbox}
``Message RPC Object Database Transaction Web''

\end{mnemonicbox}
\begin{center}\rule{0.5\linewidth}{0.5pt}\end{center}

\subsection*{Question 4(a) [3 marks]}\label{q4a}

\textbf{Explain working phases of Mobile IP}

\begin{solutionbox}

\textbf{Mobile IP Working Phases} enable seamless mobility for mobile
devices across networks.

\textbf{Three Main Phases:}

{\def\LTcaptype{none} % do not increment counter
\begin{longtable}[]{@{}ll@{}}
\toprule\noalign{}
Phase & Function \\
\midrule\noalign{}
\endhead
\bottomrule\noalign{}
\endlastfoot
\textbf{Agent Discovery} & Find home/foreign agents \\
\textbf{Registration} & Register with foreign agent \\
\textbf{Tunneling} & Forward packets to mobile node \\
\end{longtable}
}

\textbf{Phase Details:}

\begin{itemize}
\tightlist
\item
  \textbf{Agent Discovery}: Mobile node detects available agents through
  advertisements
\item
  \textbf{Registration}: Mobile node registers current location with
  home agent
\item
  \textbf{Tunneling}: Home agent encapsulates and forwards packets to
  foreign agent
\end{itemize}

\end{solutionbox}
\begin{mnemonicbox}
``Agent Registration Tunneling''

\end{mnemonicbox}
\begin{center}\rule{0.5\linewidth}{0.5pt}\end{center}

\subsection*{Question 4(b) [4 marks]}\label{q4b}

\textbf{Explain Handover management in Mobile IP}

\begin{solutionbox}

\textbf{Handover Management} maintains connectivity when mobile node
moves between networks.

\textbf{Handover Process:}

\begin{itemize}
\tightlist
\item
  \textbf{Movement detection}: Identifies change in network attachment
\item
  \textbf{New agent discovery}: Finds new foreign agent
\item
  \textbf{Registration update}: Updates location with home agent
\item
  \textbf{Data forwarding}: Redirects traffic to new location
\end{itemize}

\textbf{Types of Handover:}

{\def\LTcaptype{none} % do not increment counter
\begin{longtable}[]{@{}ll@{}}
\toprule\noalign{}
Type & Description \\
\midrule\noalign{}
\endhead
\bottomrule\noalign{}
\endlastfoot
\textbf{Hard Handover} & Break-before-make \\
\textbf{Soft Handover} & Make-before-break \\
\textbf{Horizontal} & Same technology \\
\textbf{Vertical} & Different technology \\
\end{longtable}
}

\textbf{Challenges:}

\begin{itemize}
\tightlist
\item
  \textbf{Packet loss}: During handover transition
\item
  \textbf{Delay}: Registration and tunneling setup time
\item
  \textbf{Resource management}: Efficient use of network resources
\end{itemize}

\end{solutionbox}
\begin{mnemonicbox}
``Movement Discovery Registration Forwarding''

\end{mnemonicbox}
\begin{center}\rule{0.5\linewidth}{0.5pt}\end{center}

\subsection*{Question 4(c) [7 marks]}\label{q4c}

\textbf{Explain Registration and Tunneling in Mobile IP}

\begin{solutionbox}

\textbf{Registration and Tunneling} are core mechanisms enabling Mobile
IP functionality.

\textbf{Registration Process:}

\begin{verbatim}
sequenceDiagram
    participant MN as Mobile Node
    participant FA as Foreign Agent
    participant HA as Home Agent
    
    MN{-FA: Registration Request}
    FA{-HA: Forward Request}
    HA{-FA: Registration Reply}
    FA{-MN: Forward Reply}
\end{verbatim}

\textbf{Registration Steps:}

\begin{itemize}
\tightlist
\item
  \textbf{Request}: Mobile node sends registration request to foreign
  agent
\item
  \textbf{Forward}: Foreign agent forwards request to home agent
\item
  \textbf{Authentication}: Home agent verifies mobile node identity
\item
  \textbf{Reply}: Home agent sends registration reply confirming
  registration
\end{itemize}

\textbf{Tunneling Mechanism:}

{\def\LTcaptype{none} % do not increment counter
\begin{longtable}[]{@{}ll@{}}
\toprule\noalign{}
Component & Function \\
\midrule\noalign{}
\endhead
\bottomrule\noalign{}
\endlastfoot
\textbf{Encapsulation} & Wraps original packet \\
\textbf{Tunnel Endpoint} & Home and foreign agents \\
\textbf{Decapsulation} & Unwraps packet at destination \\
\textbf{Routing} & Directs traffic through tunnel \\
\end{longtable}
}

\textbf{Tunneling Process:}

\begin{itemize}
\tightlist
\item
  \textbf{Packet arrival}: Data arrives at home agent for mobile node
\item
  \textbf{Encapsulation}: Home agent wraps packet with foreign agent
  address
\item
  \textbf{Tunnel transmission}: Packet travels through tunnel to foreign
  agent
\item
  \textbf{Decapsulation}: Foreign agent unwraps and delivers to mobile
  node
\end{itemize}

\textbf{Benefits:}

\begin{itemize}
\tightlist
\item
  \textbf{Transparency}: Applications unaware of mobility
\item
  \textbf{Connectivity}: Maintains communication during movement
\item
  \textbf{Scalability}: Supports multiple mobile nodes
\end{itemize}

\textbf{Security Considerations:}

\begin{itemize}
\tightlist
\item
  \textbf{Authentication}: Prevents unauthorized registration
\item
  \textbf{Encryption}: Protects data in tunnels
\end{itemize}

\end{solutionbox}
\begin{mnemonicbox}
``Registration Request Forward Authentication -
Tunneling Encapsulation Transmission Decapsulation''

\end{mnemonicbox}
\begin{center}\rule{0.5\linewidth}{0.5pt}\end{center}

\subsection*{Question 4(a OR) [3
marks]}\label{question-4a-or-3-marks}

\textbf{Explain snooping TCP}

\begin{solutionbox}

\textbf{Snooping TCP} improves TCP performance over wireless networks by
handling wireless link errors.

\textbf{Working Mechanism:}

\begin{itemize}
\tightlist
\item
  \textbf{Base station monitoring}: Observes TCP packets
\item
  \textbf{Local retransmission}: Handles wireless link errors locally
\item
  \textbf{Cache management}: Stores copies of transmitted packets
\item
  \textbf{Error recovery}: Retransmits lost packets without involving
  sender
\end{itemize}

\textbf{Key Features:}

{\def\LTcaptype{none} % do not increment counter
\begin{longtable}[]{@{}ll@{}}
\toprule\noalign{}
Feature & Benefit \\
\midrule\noalign{}
\endhead
\bottomrule\noalign{}
\endlastfoot
\textbf{Transparent} & No changes to TCP endpoints \\
\textbf{Local recovery} & Faster error correction \\
\textbf{Reduced timeouts} & Prevents unnecessary retransmissions \\
\end{longtable}
}

\end{solutionbox}
\begin{mnemonicbox}
``Snooping Monitors Local Cache Recovery''

\end{mnemonicbox}
\begin{center}\rule{0.5\linewidth}{0.5pt}\end{center}

\subsection*{Question 4(b OR) [4
marks]}\label{question-4b-or-4-marks}

\textbf{Explain Packet delivery in Mobile IP}

\begin{solutionbox}

\textbf{Packet Delivery in Mobile IP} ensures data reaches mobile nodes
regardless of location.

\textbf{Delivery Process:}

\begin{center}
\textbf{Mermaid Diagram (Code)}
\begin{verbatim}
{Shaded}
{Highlighting}[]
graph LR
    A[Correspondent Node] {-{-}{} B[Home Network]}
    B {-{-}{} C\{Mobile Node Location?\}}
    C {-{-}{}|Home| D[Direct Delivery]}
    C {-{-}{}|Away| E[Home Agent]}
    E {-{-}{} F[Tunnel to Foreign Agent]}
    F {-{-}{} G[Mobile Node]}
{Highlighting}
{Shaded}
\end{verbatim}
\end{center}

\textbf{Delivery Scenarios:}

{\def\LTcaptype{none} % do not increment counter
\begin{longtable}[]{@{}lll@{}}
\toprule\noalign{}
Scenario & Path & Method \\
\midrule\noalign{}
\endhead
\bottomrule\noalign{}
\endlastfoot
\textbf{At Home} & Direct & Normal IP routing \\
\textbf{Away} & Via HA/FA & Tunneling \\
\textbf{Roaming} & Triangle routing & Indirect path \\
\end{longtable}
}

\textbf{Packet Flow Steps:}

\begin{itemize}
\tightlist
\item
  \textbf{Address resolution}: Determine mobile node location
\item
  \textbf{Route selection}: Choose direct or tunneled delivery
\item
  \textbf{Encapsulation}: Wrap packet if tunneling required
\item
  \textbf{Forwarding}: Send to appropriate destination
\item
  \textbf{Decapsulation}: Unwrap packet at foreign agent
\item
  \textbf{Final delivery}: Deliver to mobile node
\end{itemize}

\textbf{Optimization Techniques:}

\begin{itemize}
\tightlist
\item
  \textbf{Route optimization}: Direct communication when possible
\item
  \textbf{Binding cache}: Store location information
\item
  \textbf{Smooth handover}: Minimize packet loss during movement
\end{itemize}

\end{solutionbox}
\begin{mnemonicbox}
``Address Route Encapsulation Forward Decapsulation
Delivery''

\end{mnemonicbox}
\begin{center}\rule{0.5\linewidth}{0.5pt}\end{center}

\subsection*{Question 4(c OR) [7
marks]}\label{question-4c-or-7-marks}

\textbf{Describe how DHCP working with diagram}

\begin{solutionbox}

\textbf{DHCP (Dynamic Host Configuration Protocol)} automatically
assigns IP addresses and network configuration to devices.

\textbf{DHCP Working Process:}

\begin{verbatim}
sequenceDiagram
    participant C as Client
    participant S as DHCP Server
    
    C{-S: 1. DHCP Discover (Broadcast)}
    S{-C: 2. DHCP Offer}
    C{-S: 3. DHCP Request}
    S{-C: 4. DHCP ACK}
    
    Note over C,S: Lease Time
    
    C{-S: 5. DHCP Renewal Request}
    S{-C: 6. DHCP ACK}
\end{verbatim}

\textbf{Four-Step Process:}

{\def\LTcaptype{none} % do not increment counter
\begin{longtable}[]{@{}lll@{}}
\toprule\noalign{}
Step & Message & Function \\
\midrule\noalign{}
\endhead
\bottomrule\noalign{}
\endlastfoot
\textbf{1} & DISCOVER & Client broadcasts request for IP \\
\textbf{2} & OFFER & Server offers available IP address \\
\textbf{3} & REQUEST & Client requests specific IP address \\
\textbf{4} & ACK & Server confirms IP assignment \\
\end{longtable}
}

\textbf{DHCP Components:}

\begin{itemize}
\tightlist
\item
  \textbf{DHCP Server}: Manages IP address pool and assignments
\item
  \textbf{DHCP Client}: Requests and uses assigned configuration
\item
  \textbf{DHCP Relay}: Forwards DHCP messages across subnets
\item
  \textbf{Address Pool}: Range of available IP addresses
\end{itemize}

\textbf{Configuration Information Provided:}

\begin{itemize}
\tightlist
\item
  \textbf{IP Address}: Unique network identifier
\item
  \textbf{Subnet Mask}: Network boundary definition
\item
  \textbf{Default Gateway}: Route to other networks
\item
  \textbf{DNS Servers}: Domain name resolution
\item
  \textbf{Lease Time}: Duration of IP assignment
\end{itemize}

\textbf{Benefits:}

\begin{itemize}
\tightlist
\item
  \textbf{Automatic configuration}: No manual IP assignment needed
\item
  \textbf{Centralized management}: Single point for network
  configuration
\item
  \textbf{Efficient utilization}: Dynamic allocation prevents waste
\item
  \textbf{Reduced errors}: Eliminates manual configuration mistakes
\end{itemize}

\textbf{DHCP Message Types:}

\begin{itemize}
\tightlist
\item
  \textbf{DISCOVER}: Locate available DHCP servers
\item
  \textbf{OFFER}: Response with configuration offer
\item
  \textbf{REQUEST}: Accept specific server offer
\item
  \textbf{ACK}: Confirm configuration assignment
\item
  \textbf{NAK}: Reject configuration request
\item
  \textbf{RELEASE}: Return IP address to pool
\item
  \textbf{RENEW}: Extend current lease
\end{itemize}

\end{solutionbox}
\begin{mnemonicbox}
``Discover Offer Request ACK - Server Client Relay
Pool''

\end{mnemonicbox}
\begin{center}\rule{0.5\linewidth}{0.5pt}\end{center}

\subsection*{Question 5(a) [3 marks]}\label{q5a}

\textbf{Give types of WLAN and explain any one}

\begin{solutionbox}

\textbf{WLAN Types:}

{\def\LTcaptype{none} % do not increment counter
\begin{longtable}[]{@{}lll@{}}
\toprule\noalign{}
Type & Standard & Frequency \\
\midrule\noalign{}
\endhead
\bottomrule\noalign{}
\endlastfoot
\textbf{Infrastructure} & 802.11 & 2.4/5 GHz \\
\textbf{Ad-hoc} & IBSS & 2.4/5 GHz \\
\textbf{Mesh} & 802.11s & Multiple \\
\end{longtable}
}

\textbf{Infrastructure WLAN Explanation:}

\begin{itemize}
\tightlist
\item
  \textbf{Access Point (AP)}: Central coordinator for all communications
\item
  \textbf{BSS (Basic Service Set)}: Network coverage area of single AP
\item
  \textbf{ESS (Extended Service Set)}: Multiple interconnected BSSs
\item
  \textbf{Distribution System}: Backbone connecting multiple APs
\end{itemize}

\textbf{Characteristics:}

\begin{itemize}
\tightlist
\item
  All communication goes through access point
\item
  Centralized network management
\item
  Better security and performance control
\end{itemize}

\end{solutionbox}
\begin{mnemonicbox}
``Infrastructure Ad-hoc Mesh - AP BSS ESS
Distribution''

\end{mnemonicbox}
\begin{center}\rule{0.5\linewidth}{0.5pt}\end{center}

\subsection*{Question 5(b) [4 marks]}\label{q5b}

\begin{solutionbox}

\end{solutionbox}
\begin{solutionbox}

\textbf{1) List Uses of Ad hoc Network:}

{\def\LTcaptype{none} % do not increment counter
\begin{longtable}[]{@{}ll@{}}
\toprule\noalign{}
Use Case & Application \\
\midrule\noalign{}
\endhead
\bottomrule\noalign{}
\endlastfoot
\textbf{Emergency} & Disaster recovery, rescue operations \\
\textbf{Military} & Battlefield communications \\
\textbf{Conferences} & Temporary meeting networks \\
\textbf{Home} & Device-to-device communication \\
\textbf{Vehicular} & Car-to-car networks \\
\end{longtable}
}

\textbf{2) Enlist entities and terminology of mobile computing:}

\textbf{Entities:}

\begin{itemize}
\tightlist
\item
  \textbf{Mobile Node (MN)}: Moving device
\item
  \textbf{Home Agent (HA)}: Permanent network representative\\
\item
  \textbf{Foreign Agent (FA)}: Temporary network coordinator
\item
  \textbf{Correspondent Node (CN)}: Communication partner
\end{itemize}

\textbf{Terminology:}

\begin{itemize}
\tightlist
\item
  \textbf{Handover}: Network switching process
\item
  \textbf{Roaming}: Moving between networks
\item
  \textbf{Care-of Address}: Temporary IP address
\end{itemize}

\end{solutionbox}
\begin{mnemonicbox}
``Emergency Military Conference Home Vehicular - MN
HA FA CN''

\end{mnemonicbox}
\begin{center}\rule{0.5\linewidth}{0.5pt}\end{center}

\subsection*{Question 5(c) [7 marks]}\label{q5c}

\textbf{Explain architecture of WLAN with neat diagram}

\begin{solutionbox}

\textbf{WLAN Architecture} consists of wireless stations communicating
through access points.

\begin{center}
\textbf{Mermaid Diagram (Code)}
\begin{verbatim}
{Shaded}
{Highlighting}[]
graph TD
    subgraph "BSS 1"
        A[Laptop] {-{-}{} AP1[Access Point 1]}
        B[Phone] {-{-}{} AP1}
        C[Tablet] {-{-}{} AP1}
    end
    
    subgraph "BSS 2"
        D[Desktop] {-{-}{} AP2[Access Point 2]}
        E[Printer] {-{-}{} AP2}
    end
    
    AP1 {-{-}{} DS[Distribution System]}
    AP2 {-{-}{} DS}
    DS {-{-}{} F[Wired Network/Internet]}
    
    subgraph "Ad{-hoc Network"}
        G[Device A] {{-}{-}{} H[Device B]}
        H {{-}{-}{} I[Device C]}
    end
{Highlighting}
{Shaded}
\end{verbatim}
\end{center}

\textbf{Architecture Components:}

{\def\LTcaptype{none} % do not increment counter
\begin{longtable}[]{@{}lll@{}}
\toprule\noalign{}
Component & Function & Coverage \\
\midrule\noalign{}
\endhead
\bottomrule\noalign{}
\endlastfoot
\textbf{STA (Station)} & Wireless device & Point \\
\textbf{AP (Access Point)} & Network coordinator & BSS area \\
\textbf{BSS (Basic Service Set)} & Single AP coverage &
\textasciitilde100m radius \\
\textbf{ESS (Extended Service Set)} & Multiple connected BSS & Large
area \\
\textbf{DS (Distribution System)} & AP interconnection &
Building/campus \\
\end{longtable}
}

\textbf{Types of WLAN Architecture:}

\textbf{1. Infrastructure Mode:}

\begin{itemize}
\tightlist
\item
  \textbf{Centralized}: All traffic through access points
\item
  \textbf{Managed}: Network administration and security
\item
  \textbf{Scalable}: Easy to expand coverage area
\end{itemize}

\textbf{2. Ad-hoc Mode (IBSS):}

\begin{itemize}
\tightlist
\item
  \textbf{Peer-to-peer}: Direct device communication
\item
  \textbf{Decentralized}: No central coordinator
\item
  \textbf{Temporary}: Quick setup for specific needs
\end{itemize}

\textbf{Key Features:}

\begin{itemize}
\tightlist
\item
  \textbf{Mobility}: Users can move within coverage area
\item
  \textbf{Wireless medium}: Radio waves for communication
\item
  \textbf{Shared bandwidth}: Multiple users share channel capacity
\item
  \textbf{Security}: WPA/WPA2/WPA3 protocols for protection
\end{itemize}

\textbf{Standards and Frequencies:}

\begin{itemize}
\tightlist
\item
  \textbf{802.11a}: 5 GHz, up to 54 Mbps
\item
  \textbf{802.11b}: 2.4 GHz, up to 11 Mbps\\
\item
  \textbf{802.11g}: 2.4 GHz, up to 54 Mbps
\item
  \textbf{802.11n}: 2.4/5 GHz, up to 600 Mbps
\item
  \textbf{802.11ac}: 5 GHz, up to 6.93 Gbps
\end{itemize}

\end{solutionbox}
\begin{mnemonicbox}
``STA AP BSS ESS DS - Infrastructure Ad-hoc''

\end{mnemonicbox}
\begin{center}\rule{0.5\linewidth}{0.5pt}\end{center}

\subsection*{Question 5(a OR) [3
marks]}\label{question-5a-or-3-marks}

\textbf{Write features of 5G}

\begin{solutionbox}

\textbf{5G Key Features:}

{\def\LTcaptype{none} % do not increment counter
\begin{longtable}[]{@{}ll@{}}
\toprule\noalign{}
Feature & Specification \\
\midrule\noalign{}
\endhead
\bottomrule\noalign{}
\endlastfoot
\textbf{Speed} & Up to 10 Gbps \\
\textbf{Latency} & \textless{} 1 millisecond \\
\textbf{Connectivity} & 1 million devices/km^{2} \\
\textbf{Reliability} & 99.999\% availability \\
\textbf{Bandwidth} & 100x increase \\
\textbf{Energy} & 90\% reduction \\
\end{longtable}
}

\textbf{Advanced Capabilities:}

\begin{itemize}
\tightlist
\item
  \textbf{Enhanced Mobile Broadband (eMBB)}: Ultra-fast data speeds
\item
  \textbf{Ultra-Reliable Low Latency (URLLC)}: Mission-critical
  applications
\item
  \textbf{Massive Machine Type Communication (mMTC)}: IoT connectivity
\end{itemize}

\end{solutionbox}
\begin{mnemonicbox}
``Speed Latency Connectivity Reliability Bandwidth
Energy''

\end{mnemonicbox}
\begin{center}\rule{0.5\linewidth}{0.5pt}\end{center}

\subsection*{Question 5(b OR) [4
marks]}\label{question-5b-or-4-marks}

\begin{solutionbox}

\end{solutionbox}
\begin{solutionbox}

\textbf{1) List Type of communication middleware:}

{\def\LTcaptype{none} % do not increment counter
\begin{longtable}[]{@{}ll@{}}
\toprule\noalign{}
Type & Function \\
\midrule\noalign{}
\endhead
\bottomrule\noalign{}
\endlastfoot
\textbf{Message-Oriented} & Asynchronous messaging \\
\textbf{RPC-based} & Remote procedure calls \\
\textbf{Object-Oriented} & Distributed objects \\
\textbf{Service-Oriented} & Web services \\
\textbf{Database} & Data access layer \\
\end{longtable}
}

\textbf{2) Define the term ``Home Agent'' in the context of Mobile IP:}

\textbf{Home Agent (HA)} is a router on mobile node's home network that:

\begin{itemize}
\tightlist
\item
  \textbf{Maintains registration}: Tracks mobile node's current location
\item
  \textbf{Tunnels packets}: Forwards data to mobile node's foreign
  location\\
\item
  \textbf{Address management}: Manages mobile node's permanent IP
  address
\item
  \textbf{Authentication}: Verifies mobile node identity during
  registration
\end{itemize}

\textbf{Functions:}

\begin{itemize}
\tightlist
\item
  Acts as proxy for mobile node when away from home
\item
  Intercepts packets destined for mobile node
\item
  Creates tunnels to foreign agents
\end{itemize}

\end{solutionbox}
\begin{mnemonicbox}
``Message RPC Object Service Database - HA Maintains
Tunnels Address Authentication''

\end{mnemonicbox}
\begin{center}\rule{0.5\linewidth}{0.5pt}\end{center}

\subsection*{Question 5(c OR) [7
marks]}\label{question-5c-or-7-marks}

\textbf{Explain Bluetooth protocol stack with diagram}

\begin{solutionbox}

\textbf{Bluetooth Protocol Stack} provides layered architecture for
short-range wireless communication.

\begin{center}
\textbf{Mermaid Diagram (Code)}
\begin{verbatim}
{Shaded}
{Highlighting}[]
graph LR
    A[Applications] {-{-}{} B[Application Layer]}
    B {-{-}{} C[OBEX/SDP/TCS]}
    C {-{-}{} D[RFCOMM]}
    D {-{-}{} E[L2CAP]}
    E {-{-}{} F[HCI {-} Host Controller Interface]}
    F {-{-}{} G[Link Manager Protocol {-} LMP]}
    G {-{-}{} H[Baseband]}
    H {-{-}{} I[Radio Layer]}
{Highlighting}
{Shaded}
\end{verbatim}
\end{center}

\textbf{Protocol Stack Layers:}

{\def\LTcaptype{none} % do not increment counter
\begin{longtable}[]{@{}lll@{}}
\toprule\noalign{}
Layer & Function & Protocols \\
\midrule\noalign{}
\endhead
\bottomrule\noalign{}
\endlastfoot
\textbf{Application} & User applications & Audio, File transfer \\
\textbf{Middleware} & Services & OBEX, SDP, TCS \\
\textbf{Transport} & Data delivery & RFCOMM \\
\textbf{Network} & Packet management & L2CAP \\
\textbf{Interface} & Host-Controller & HCI \\
\textbf{Management} & Link control & LMP \\
\textbf{Data Link} & Channel access & Baseband \\
\textbf{Physical} & Radio transmission & 2.4 GHz ISM \\
\end{longtable}
}

\textbf{Layer Details:}

\textbf{Upper Layers:}

\begin{itemize}
\tightlist
\item
  \textbf{OBEX}: Object Exchange Protocol for file transfers
\item
  \textbf{SDP}: Service Discovery Protocol finds available services
\item
  \textbf{TCS}: Telephony Control Specification for voice calls
\item
  \textbf{RFCOMM}: Serial port emulation over Bluetooth
\end{itemize}

\textbf{Lower Layers:}

\begin{itemize}
\tightlist
\item
  \textbf{L2CAP}: Logical Link Control manages multiple connections
\item
  \textbf{HCI}: Host Controller Interface standardizes communication
\item
  \textbf{LMP}: Link Manager Protocol handles connection setup
\item
  \textbf{Baseband}: Manages time slots and frequency hopping
\end{itemize}

\textbf{Key Features:}

\begin{itemize}
\tightlist
\item
  \textbf{Frequency Hopping}: 1600 hops/second across 79 channels
\item
  \textbf{Piconet}: Network of up to 8 devices
\item
  \textbf{Scatternet}: Multiple overlapping piconets
\item
  \textbf{Power Classes}: Class 1 (100m), Class 2 (10m), Class 3 (1m)
\end{itemize}

\textbf{Advantages:}

\begin{itemize}
\tightlist
\item
  \textbf{Low power consumption}: Suitable for battery devices
\item
  \textbf{Automatic pairing}: Easy device connection
\item
  \textbf{Interference resistance}: Frequency hopping spread spectrum
\item
  \textbf{Cost effective}: Low implementation cost
\end{itemize}

\textbf{Applications:}

\begin{itemize}
\tightlist
\item
  \textbf{Audio streaming}: Headphones, speakers
\item
  \textbf{Data transfer}: File sharing between devices
\item
  \textbf{Input devices}: Keyboards, mice
\item
  \textbf{IoT devices}: Sensors, smart home devices
\end{itemize}

\end{solutionbox}
\begin{mnemonicbox}
``Application Middleware Transport Network Interface
Management DataLink Physical''

\end{mnemonicbox}

\end{document}
