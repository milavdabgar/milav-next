\documentclass[10pt,a4paper]{article}

% content/resources/templates/preamble.tex
\usepackage[margin=0.6in]{geometry}
\author{Milav Dabgar}
\usepackage{amsmath,amssymb,amsthm}
\usepackage{booktabs}
\usepackage{multirow}
\usepackage{xcolor}
\usepackage{tcolorbox}
\tcbuselibrary{breakable,skins}
\usepackage[colorlinks=true,linkcolor=blue]{hyperref}
\usepackage{titlesec}
\usepackage{enumitem}
\usepackage{tikz}
\usepackage{pgfplots}
\usepackage{circuitikz}
\usepackage[version=4]{mhchem}
\usepackage{longtable}
\usepackage{array}
\usepackage{float}
\usepackage{caption}
\usepackage{listings}

\lstset{
  basicstyle=\small\ttfamily,
  breaklines=true,
  breakatwhitespace=false,
  postbreak=\mbox{\textcolor{red}{$\hookrightarrow$}\space},
  float=false,
  numbers=left,
  numberstyle=\tiny\color{gray},
  numbersep=10pt,
  xleftmargin=2em,
  keywordstyle=\color{blue},
  commentstyle=\color{green!60!black},
  stringstyle=\color{purple},
  backgroundcolor=\color{gray!5},
  showstringspaces=false,
  tabsize=2,
  captionpos=b,
  keepspaces=true,
  columns=flexible
}

\pgfplotsset{compat=1.18}
\usetikzlibrary{shapes,arrows,positioning,calc,patterns,decorations.pathmorphing,decorations.markings,arrows.meta}

% Color scheme
\definecolor{headcolor}{RGB}{0,102,204}
\definecolor{keycolor}{RGB}{220,20,60}
\definecolor{solutioncolor}{RGB}{34,139,34}
\definecolor{mnemoniccolor}{RGB}{148,0,211}
\definecolor{codecolor}{RGB}{0,0,100}

% Spacing
\setlength{\parskip}{3pt}
\setlist[itemize]{nosep}
\setlist[enumerate]{nosep}

% Title formatting
\titleformat{\section}{\Large\bfseries\color{headcolor}}{\thesection}{1em}{}
\titleformat{\subsection}{\large\bfseries\color{headcolor}}{\thesubsection}{1em}{}

% Pandoc tightlist compatibility
\providecommand{\tightlist}{%
  \setlength{\itemsep}{0pt}\setlength{\parskip}{0pt}}

% Pandoc longtable compatibility
\newcounter{none}
\def\thenone{}


% content/resources/templates/gujarati-boxes.tex
\usepackage{fontspec}
\usepackage{polyglossia}

% Set Gujarati as main language (document is primarily in Gujarati)
% Note: gloss-gujarati.ldf doesn't exist in polyglossia, but it will use hyphenation patterns
\setdefaultlanguage{gujarati}
\setotherlanguage{english}

% Configure Gujarati font properly
% Use Language=Default to prevent polyglossia from trying to add language-specific features
% that don't exist for Gujarati, which causes "empty feature" warnings
\newfontfamily\gujaratifont[Script=Gujarati,AutoFakeBold=2.5,AutoFakeSlant=0.3]{Noto Sans Gujarati}
\setmainfont[Script=Gujarati,AutoFakeBold=2.5,AutoFakeSlant=0.3]{Noto Sans Gujarati}
% Use Noto Sans Gujarati for monospace to support Gujarati in text
\setmonofont[Scale=0.9]{Noto Sans Gujarati}

% Configure English to use the same font
\newfontfamily\englishfont[Script=Gujarati,AutoFakeBold=2.5,AutoFakeSlant=0.3]{Noto Sans Gujarati}

% Translations for polyglossia
\gappto\captionsgujarati{
  \renewcommand{\tablename}{કોષ્ટક}
  \renewcommand{\figurename}{આકૃતિ}
}

% Helper for TikZ nodes to ensure Gujarati font
\newcommand{\gu}[1]{{\gujaratifont #1}}

% Custom environments
\newtcolorbox{solutionbox}{
    breakable,
    enhanced,
    colback=solutioncolor!5!white,
    colframe=solutioncolor!75!black,
    fonttitle=\bfseries,
    title=જવાબ
}

\newtcolorbox{solutionboxnobreak}{
 colback=solutioncolor!5!white,
 colframe=solutioncolor!75!black,
 fonttitle=\bfseries,
 title=જવાબ
}

\newtcolorbox{keyformula}{
 breakable,
 enhanced,
 colback=keycolor!5!white,
 colframe=keycolor!75!black,
 fonttitle=\bfseries,
 title=રાસાયણિક સમીકરણ/સૂત્ર
}

\newtcolorbox{mnemonicbox}{
 breakable,
 enhanced,
 colback=mnemoniccolor!5!white,
 colframe=mnemoniccolor!75!black,
 fonttitle=\bfseries,
 title=મેમરી ટ્રીક
}


\begin{document}

\begin{center}
{\Huge\bfseries\color{headcolor} Subject Name (Gujarati)}\\[5pt]
{\LARGE 4351602 -- Winter 2023}\\[3pt]
{\large Semester 1 Study Material}\\[3pt]
{\normalsize\textit{Detailed Solutions and Explanations}}
\end{center}

\vspace{10pt}

\subsection*{પ્રશ્ન 1(અ) [03
ગુણ]}\label{uxaaauxab0uxab6uxaa8-1uxa85-03-uxa97uxaa3}

\textbf{કલાઇન્ટ સર્વર અને પીઅર ટૂ પીઅર નેટવર્કનો તફાવત લખો.}

\begin{solutionbox}

{\def\LTcaptype{none} % do not increment counter
\begin{longtable}[]{@{}lll@{}}
\toprule\noalign{}
પેરામીટર & Client-Server Network & Peer-to-Peer Network \\
\midrule\noalign{}
\endhead
\bottomrule\noalign{}
\endlastfoot
\textbf{આર્કિટેક્ચર} & કેન્દ્રિય સર્વર સાથે & વિકેન્દ્રિત, બધા નોડ્સ સમાન \\
\textbf{ખર્ચ} & સર્વર હાર્ડવેરને કારણે વધુ & ઓછો, હાલના કમ્પ્યુટર્સનો ઉપયોગ \\
\textbf{સિક્યોરિટી} & વધુ, કેન્દ્રિય નિયંત્રણ & ઓછી, વિતરિત નિયંત્રણ \\
\textbf{સ્કેલેબિલિટી} & સર્વરની ક્ષમતાથી મર્યાદિત & વધુ સારી, નોડ્સ સાથે સંસાધનો
વધે \\
\end{longtable}
}

\end{solutionbox}
\begin{mnemonicbox}
``CSS-P: Client-Server = કેન્દ્રિય સિક્યોરિટી, P2P =
પીઅર પાવર''

\end{mnemonicbox}
\subsection*{પ્રશ્ન 1(બ) [04
ગુણ]}\label{uxaaauxab0uxab6uxaa8-1uxaac-04-uxa97uxaa3}

\textbf{ARP પ્રોટોકોલ તેની વર્કિંગ સાથે સમજાવો.}

\begin{solutionbox}

\textbf{ARP (Address Resolution Protocol)} લોકલ નેટવર્કમાં IP એડ્રેસને MAC
એડ્રેસ સાથે જોડે છે.

\textbf{વર્કિંગ પ્રોસેસ:}

\begin{itemize}
\tightlist
\item
  \textbf{બ્રોડકાસ્ટ રિક્વેસ્ટ}: હોસ્ટ ટાર્ગેટ IP સાથે ARP રિક્વેસ્ટ બ્રોડકાસ્ટ કરે
\item
  \textbf{કેશ ચેક}: રિસીવિંગ હોસ્ટ્સ તપાસે કે IP મેચ થાય છે કે નહીં
\item
  \textbf{રિપ્લાય જનરેશન}: ટાર્ગેટ હોસ્ટ MAC એડ્રેસ સાથે ARP રિપ્લાય મોકલે
\item
  \textbf{કેશ અપડેટ}: રિક્વેસ્ટિંગ હોસ્ટ ARP ટેબલ અપડેટ કરે
\end{itemize}

\textbf{ARP ટેબલ ઉદાહરણ:}

\begin{verbatim}
IP Address      MAC Address         TTL
192.168.1.1     00:1A:2B:3C:4D:5E   300s
\end{verbatim}

\end{solutionbox}
\begin{mnemonicbox}
``BCRU: બ્રોડકાસ્ટ, કેશ, રિપ્લાય, અપડેટ''

\end{mnemonicbox}
\subsection*{પ્રશ્ન 1(ક) [07
ગુણ]}\label{uxaaauxab0uxab6uxaa8-1uxa95-07-uxa97uxaa3}

\textbf{OSI મોડેલ આકૃતિ સાથે સમજાવો.}

\begin{solutionbox}

\textbf{OSI (Open Systems Interconnection)} મોડેલમાં નેટવર્ક કમ્યુનિકેશન માટે 7
લેયર્સ છે.

\begin{center}
\textbf{Mermaid Diagram (Code)}
\begin{verbatim}
{Shaded}
{Highlighting}[]
graph LR
    A[Application Layer {- 7] {-}{-}{} B[Presentation Layer {-} 6]}
    B {-{-}{} C[Session Layer {-} 5]}
    C {-{-}{} D[Transport Layer {-} 4]}
    D {-{-}{} E[Network Layer {-} 3]}
    E {-{-}{} F[Data Link Layer {-} 2]}
    F {-{-}{} G[Physical Layer {-} 1]}
{Highlighting}
{Shaded}
\end{verbatim}
\end{center}

\textbf{લેયર ફંક્શન્સ:}

\begin{itemize}
\tightlist
\item
  \textbf{Physical}: ફિઝિકલ મીડિયમ પર બિટ ટ્રાન્સમિશન
\item
  \textbf{Data Link}: ફ્રેમ ટ્રાન્સમિશન, એરર ડિટેક્શન
\item
  \textbf{Network}: રાઉટિંગ, IP એડ્રેસિંગ
\item
  \textbf{Transport}: એન્ડ-ટુ-એન્ડ ડિલિવરી, TCP/UDP
\item
  \textbf{Session}: કનેક્શન મેનેજમેન્ટ
\item
  \textbf{Presentation}: ડેટા એન્ક્રિપ્શન, કોમ્પ્રેશન
\item
  \textbf{Application}: યુઝર ઇન્ટરફેસ, ઇમેઇલ, વેબ
\end{itemize}

\end{solutionbox}
\begin{mnemonicbox}
``All People Seem To Need Data Processing''

\end{mnemonicbox}
\subsection*{પ્રશ્ન 1(ક OR) [07
ગુણ]}\label{uxaaauxab0uxab6uxaa8-1uxa95-or-07-uxa97uxaa3}

\textbf{કન્જેશન શું છે? કન્જેશન કંટ્રોલ સમજાવો.}

\begin{solutionbox}

\textbf{કન્જેશન} ત્યારે થાય છે જ્યારે નેટવર્ક ટ્રાફિક ઉપલબ્ધ બેન્ડવિડ્થ કરતાં વધી જાય,
જેથી પેકેટ ડિલે અને લોસ થાય.

\textbf{કન્જેશન કંટ્રોલના પ્રકારો:}

{\def\LTcaptype{none} % do not increment counter
\begin{longtable}[]{@{}lll@{}}
\toprule\noalign{}
પ્રકાર & મેથડ & વર્ણન \\
\midrule\noalign{}
\endhead
\bottomrule\noalign{}
\endlastfoot
\textbf{Open-Loop} & પ્રિવેન્શન & કન્જેશન પહેલાં ટ્રાફિક શેપિંગ \\
\textbf{Closed-Loop} & રિએક્શન & ફીડબેક આધારિત એડજસ્ટમેન્ટ \\
\end{longtable}
}

\textbf{કન્જેશન કંટ્રોલ ટેકનિક્સ:}

\begin{itemize}
\tightlist
\item
  \textbf{ટ્રાફિક શેપિંગ}: ડેટા ટ્રાન્સમિશન રેટ નિયંત્રિત કરો
\item
  \textbf{એડમિશન કંટ્રોલ}: કન્જેશન દરમિયાન નવા કનેક્શન્સ મર્યાદિત કરો
\item
  \textbf{લોડ શેડિંગ}: બફર્સ ભરાઈ જાય ત્યારે પેકેટ્સ ડ્રોપ કરો
\item
  \textbf{બેકપ્રેશર}: અપસ્ટ્રીમ કન્જેશન સિગ્નલ્સ મોકલો
\end{itemize}

\end{solutionbox}
\begin{mnemonicbox}
``TALB: ટ્રાફિક, એડમિશન, લોડ, બેકપ્રેશર''

\end{mnemonicbox}
\subsection*{પ્રશ્ન 2(અ) [03
ગુણ]}\label{uxaaauxab0uxab6uxaa8-2uxa85-03-uxa97uxaa3}

\textbf{એડહોક નેટવર્ક શું છે? તે સમજાવો.}

\begin{solutionbox}

\textbf{એડહોક નેટવર્ક} એક વાયરલેસ નેટવર્ક છે જેમાં કોઈ નિશ્ચિત ઇન્ફ્રાસ્ટ્રક્ચર વગર
નોડ્સ સીધો કમ્યુનિકેટ કરે છે.

\textbf{લક્ષણો:}

\begin{itemize}
\tightlist
\item
  \textbf{સ્વ-આયોજિત}: ઓટોમેટિક નેટવર્ક ફોર્મેશન
\item
  \textbf{ડાયનેમિક ટોપોલોજી}: નોડ્સ મુક્તપણે જોડાઈ/છૂટી શકે
\item
  \textbf{મલ્ટિ-હોપ રાઉટિંગ}: મેસેજ્સ મધ્યવર્તી નોડ્સ દ્વારા રિલે થાય
\item
  \textbf{વિતરિત નિયંત્રણ}: કોઈ કેન્દ્રિય સત્તા નહીં
\end{itemize}

\textbf{એપ્લિકેશન્સ:}

\begin{itemize}
\tightlist
\item
  ઇમર્જન્સી રિસ્પોન્સ, મિલિટરી ઓપરેશન્સ, સેન્સર નેટવર્ક્સ
\end{itemize}

\end{solutionbox}
\begin{mnemonicbox}
``SDMD: સ્વ-આયોજિત, ડાયનેમિક, મલ્ટિ-હોપ, વિતરિત''

\end{mnemonicbox}
\subsection*{પ્રશ્ન 2(બ) [04
ગુણ]}\label{uxaaauxab0uxab6uxaa8-2uxaac-04-uxa97uxaa3}

\textbf{મોબાઈલ IP માં હેન્ડઓવર મેનેજમેન્ટ સમજાવો.}

\begin{solutionbox}

\textbf{હેન્ડઓવર} એ પ્રક્રિયા છે જ્યારે મોબાઈલ નોડ નેટવર્ક્સ વચ્ચે ખસે ત્યારે કનેક્ટિવિટી
જાળવી રાખવાની.

\textbf{હેન્ડઓવર પ્રક્રિયા:}

\begin{verbatim}
sequenceDiagram
    participant MN as Mobile Node
    participant FA1 as Foreign Agent 1
    participant FA2 as Foreign Agent 2
    participant HA as Home Agent
    
    MN{-FA2: Agent Discovery}
    FA2{-MN: Advertisement}
    MN{-HA: Registration Request}
    HA{-MN: Registration Reply}
    HA{-FA1: Update Tunnel}
\end{verbatim}

\textbf{પ્રકારો:}

\begin{itemize}
\tightlist
\item
  \textbf{હાર્ડ હેન્ડઓવર}: બ્રેક-બિફોર-મેક કનેક્શન
\item
  \textbf{સોફ્ટ હેન્ડઓવર}: મેક-બિફોર-બ્રેક કનેક્શન
\end{itemize}

\end{solutionbox}
\begin{mnemonicbox}
``DARU: ડિસ્કવરી, એડવર્ટાઇઝમેન્ટ, રજિસ્ટ્રેશન, અપડેટ''

\end{mnemonicbox}
\subsection*{પ્રશ્ન 2(ક) [07
ગુણ]}\label{uxaaauxab0uxab6uxaa8-2uxa95-07-uxa97uxaa3}

\textbf{મોબાઈલ કમ્પ્યુટિંગનું થ્રી ટાયર આર્કિટેક્ચર આકૃતિ સાથે સમજાવો.}

\begin{solutionbox}

\textbf{થ્રી-ટાયર આર્કિટેક્ચર} મોબાઈલ એપ્લિકેશન્સને પ્રેઝન્ટેશન, એપ્લિકેશન લોજિક અને
ડેટા લેયર્સમાં વિભાજિત કરે છે.

\begin{verbatim}
graph TB
    subgraph "Tier 1: Presentation Layer"
        A[Mobile Device]
        B[User Interface]
        C[Input/Output]
    end
    
    subgraph "Tier 2: Application Layer"
        D[Business Logic]
        E[Processing Rules]
        F[Middleware]
    end
    
    subgraph "Tier 3: Data Layer"
        G[Database Server]
        H[Data Storage]
        I[Data Management]
    end
    
    A {-{-} D}
    D {-{-} G}
\end{verbatim}

\textbf{લેયર ફંક્શન્સ:}

\begin{itemize}
\tightlist
\item
  \textbf{પ્રેઝન્ટેશન}: યુઝર ઇન્ટરફેસ, મોબાઈલ એપ્સ
\item
  \textbf{એપ્લિકેશન}: બિઝનેસ લોજિક, મિડલવેર સર્વિસેસ
\item
  \textbf{ડેટા}: ડેટાબેસ મેનેજમેન્ટ, સ્ટોરેજ સિસ્ટમ્સ
\end{itemize}

\textbf{ફાયદા:}

\begin{itemize}
\tightlist
\item
  \textbf{સ્કેલેબિલિટી}: સ્વતંત્ર લેયર સ્કેલિંગ
\item
  \textbf{મેન્ટેનેબિલિટી}: અલગ ચિંતાવાળા વિષયો
\item
  \textbf{લવચીકતા}: ટેકનોલોજી સ્વતંત્રતા
\end{itemize}

\end{solutionbox}
\begin{mnemonicbox}
``PAD: પ્રેઝન્ટેશન, એપ્લિકેશન, ડેટા''

\end{mnemonicbox}
\subsection*{પ્રશ્ન 2(અ OR) [03
ગુણ]}\label{uxaaauxab0uxab6uxaa8-2uxa85-or-03-uxa97uxaa3}

\textbf{વાયરલેસ નેટવર્કની જરૂરિયાત સમજાવો.}

\begin{solutionbox}

\textbf{વાયરલેસ નેટવર્ક્સ} ફિઝિકલ કેબલ્સ વગર કનેક્ટિવિટી પ્રદાન કરે છે.

\textbf{જરૂરિયાતો:}

\begin{itemize}
\tightlist
\item
  \textbf{મોબિલિટી}: યુઝર્સ કનેક્ટેડ રહીને મુક્તપણે ફરી શકે
\item
  \textbf{લવચીકતા}: સરળ નેટવર્ક વિસ્તરણ અને પુનઃ રૂપરેખાંકન
\item
  \textbf{ખર્ચ-અસરકારક}: કેબલિંગ ઇન્ફ્રાસ્ટ્રક્ચર ખર્ચ ઘટાડો
\item
  \textbf{પહોંચ}: દૂરના વિસ્તારોમાં ઇન્ટરનેટ એક્સેસ
\end{itemize}

\textbf{એપ્લિકેશન્સ:}

\begin{itemize}
\tightlist
\item
  મોબાઈલ કમ્યુનિકેશન્સ, WiFi હોટસ્પોટ્સ, IoT ડિવાઇસ
\end{itemize}

\end{solutionbox}
\begin{mnemonicbox}
``MFCA: મોબિલિટી, લવચીકતા, ખર્ચ, પહોંચ''

\end{mnemonicbox}
\subsection*{પ્રશ્ન 2(બ OR) [04
ગુણ]}\label{uxaaauxab0uxab6uxaa8-2uxaac-or-04-uxa97uxaa3}

\textbf{મોબાઈલ IP માં રજિસ્ટ્રેશન, ટનલિંગ અને ઇન્કેપ્સુલેશન સમજાવો.}

\begin{solutionbox}

\textbf{મોબાઈલ IP કોમ્પોનન્ટ્સ:}

{\def\LTcaptype{none} % do not increment counter
\begin{longtable}[]{@{}lll@{}}
\toprule\noalign{}
પ્રક્રિયા & વર્ણન & હેતુ \\
\midrule\noalign{}
\endhead
\bottomrule\noalign{}
\endlastfoot
\textbf{રજિસ્ટ્રેશન} & મોબાઈલ નોડ હોમ એજન્ટ સાથે રજિસ્ટર થાય & લોકેશન અપડેટ \\
\textbf{ટનલિંગ} & એજન્ટ્સ વચ્ચે વર્ચ્યુઅલ પાથ બનાવે & પેકેટ્સ રૂટ કરવા \\
\textbf{ઇન્કેપ્સુલેશન} & મૂળ પેકેટને નવા હેડરમાં લપેટે & એડ્રેસ ટ્રાન્સલેશન \\
\end{longtable}
}

\textbf{પ્રક્રિયા ફ્લો:}

\begin{verbatim}
મૂળ પેકેટ \rightarrow ઇન્કેપ્સુલેશન \rightarrow ટનલ \rightarrow ડીકેપ્સુલેશન \rightarrow ડેસ્ટિનેશન
\end{verbatim}

\textbf{રજિસ્ટ્રેશન સ્તરો:}

\begin{itemize}
\tightlist
\item
  મોબાઈલ નોડ ફોરેન એજન્ટ શોધે
\item
  હોમ એજન્ટને રજિસ્ટ્રેશન રિક્વેસ્ટ મોકલે
\item
  હોમ એજન્ટ લોકેશન બાઇન્ડિંગ અપડેટ કરે
\end{itemize}

\end{solutionbox}
\begin{mnemonicbox}
``RTE: રજિસ્ટ્રેશન, ટનલિંગ, ઇન્કેપ્સુલેશન''

\end{mnemonicbox}
\subsection*{પ્રશ્ન 2(ક OR) [07
ગુણ]}\label{uxaaauxab0uxab6uxaa8-2uxa95-or-07-uxa97uxaa3}

\textbf{મિડલવેર શું છે? મિડલવેરના ઉદાહરણો લખો અને તેમાંથી કોઈ પણ એકને વિગતે
સમજાવો.}

\begin{solutionbox}

\textbf{મિડલવેર} એ સોફ્ટવેર છે જે વિતરિત સિસ્ટમ્સમાં વિવિધ એપ્લિકેશન્સ અને સેવાઓને જોડે
છે.

\textbf{મિડલવેરના ઉદાહરણો:}

\begin{itemize}
\tightlist
\item
  \textbf{Message-Oriented Middleware (MOM)}
\item
  \textbf{Remote Procedure Call (RPC)}
\item
  \textbf{Object Request Broker (ORB)}
\item
  \textbf{ડેટાબેસ મિડલવેર}
\item
  \textbf{વેબ સર્વિસ}
\end{itemize}

\textbf{Message-Oriented Middleware (MOM) - વિગતવાર:}

\textbf{આર્કિટેક્ચર:}

\begin{center}
\textbf{Mermaid Diagram (Code)}
\begin{verbatim}
{Shaded}
{Highlighting}[]
graph LR
    A[Sender Application] {-{-}{} B[Message Queue]}
    B {-{-}{} C[MOM Layer]}
    C {-{-}{} D[Message Queue]}
    D {-{-}{} E[Receiver Application]}
{Highlighting}
{Shaded}
\end{verbatim}
\end{center}

\textbf{લક્ષણો:}

\begin{itemize}
\tightlist
\item
  \textbf{અસિંક્રોનસ કમ્યુનિકેશન}: નોન-બ્લોકિંગ મેસેજ એક્સચેન્જ
\item
  \textbf{વિશ્વસનીયતા}: મેસેજ પર્સિસ્ટન્સ અને ડિલિવરી ગેરંટી
\item
  \textbf{સ્કેલેબિલિટી}: મલ્ટિપલ કોન્કરન્ટ કનેક્શન્સ હેન્ડલ કરે
\item
  \textbf{પ્લેટફોર્મ સ્વતંત્રતા}: ક્રોસ-પ્લેટફોર્મ કમ્યુનિકેશન
\end{itemize}

\textbf{ફાયદા:}

\begin{itemize}
\tightlist
\item
  એપ્લિકેશન્સ વચ્ચે લૂઝ કપલિંગ
\item
  સિસ્ટમ વિશ્વસનીયતામાં સુધારો
\item
  વધુ સારી ફોલ્ટ ટોલરન્સ
\end{itemize}

\end{solutionbox}
\begin{mnemonicbox}
``ARSP: અસિંક્રોનસ, વિશ્વસનીય, સ્કેલેબલ, પ્લેટફોર્મ-સ્વતંત્ર''

\end{mnemonicbox}
\subsection*{પ્રશ્ન 3(અ) [03
ગુણ]}\label{uxaaauxab0uxab6uxaa8-3uxa85-03-uxa97uxaa3}

\textbf{`www' નું ફુલ ફોર્મ આપો અને તે સમજાવો.}

\begin{solutionbox}

\textbf{WWW = World Wide Web}

\textbf{સમજાવટ:}

\begin{itemize}
\tightlist
\item
  \textbf{ગ્લોબલ ઇન્ફોર્મેશન સિસ્ટમ}: ડોક્યુમેન્ટ્સનો પરસ્પર જોડાયેલો જાળો
\item
  \textbf{HTTP પ્રોટોકોલ}: HyperText Transfer Protocol નો ઉપયોગ કરે
\item
  \textbf{URL એડ્રેસિંગ}: યુનિક રિસોર્સ લોકેટર્સ
\item
  \textbf{હાયપરલિંક્સ}: વેબ પેજો વચ્ચે નેવિગેટ કરવા
\end{itemize}

\textbf{કોમ્પોનન્ટ્સ:}

\begin{itemize}
\tightlist
\item
  વેબ સર્વર્સ, બ્રાઉઝર્સ, HTML ડોક્યુમેન્ટ્સ, URL
\end{itemize}

\end{solutionbox}
\begin{mnemonicbox}
``GHUH: ગ્લોબલ, HTTP, URL, હાયપરલિંક્સ''

\end{mnemonicbox}
\subsection*{પ્રશ્ન 3(બ) [04
ગુણ]}\label{uxaaauxab0uxab6uxaa8-3uxaac-04-uxa97uxaa3}

\textbf{મોબાઈલ કમ્પ્યુટિંગની ઉપયોગિતા સમજાવો.}

\begin{solutionbox}

\textbf{મોબાઈલ કમ્પ્યુટિંગ એપ્લિકેશન્સ:}

{\def\LTcaptype{none} % do not increment counter
\begin{longtable}[]{@{}
  >{\raggedright\arraybackslash}p{(\linewidth - 4\tabcolsep) * \real{0.3030}}
  >{\raggedright\arraybackslash}p{(\linewidth - 4\tabcolsep) * \real{0.3939}}
  >{\raggedright\arraybackslash}p{(\linewidth - 4\tabcolsep) * \real{0.3030}}@{}}
\toprule\noalign{}
\begin{minipage}[b]{\linewidth}\raggedright
કેટેગરી
\end{minipage} & \begin{minipage}[b]{\linewidth}\raggedright
એપ્લિકેશન્સ
\end{minipage} & \begin{minipage}[b]{\linewidth}\raggedright
ફાયદા
\end{minipage} \\
\midrule\noalign{}
\endhead
\bottomrule\noalign{}
\endlastfoot
\textbf{બિઝનેસ} & ઇમેઇલ, CRM, સેલ્સ & પ્રોડક્ટિવિટી, રિયલ-ટાઇમ એક્સેસ \\
\textbf{હેલ્થકેર} & પેશન્ટ મોનિટરિંગ, ટેલિમેડિસિન & રિમોટ કેર, ઇમર્જન્સી રિસ્પોન્સ \\
\textbf{એજ્યુકેશન} & ઇ-લર્નિંગ, ડિજિટલ લાઇબ્રેરી & લવચીક લર્નિંગ, રિસોર્સ એક્સેસ \\
\textbf{મનોરંજન} & ગેમિંગ, સ્ટ્રીમિંગ, સોશિયલ મીડિયા & ઓન-ડિમાન્ડ કન્ટેન્ટ,
કનેક્ટિવિટી \\
\end{longtable}
}

\textbf{મુખ્ય લક્ષણો:}

\begin{itemize}
\tightlist
\item
  \textbf{લોકેશન-બેઝ્ડ સર્વિસ}: GPS નેવિગેશન, લોકલ સર્ચ
\item
  \textbf{મોબાઈલ પેમેન્ટ્સ}: ડિજિટલ વોલેટ, કોન્ટેક્ટલેસ ટ્રાન્ઝેક્શન્સ
\item
  \textbf{IoT ઇન્ટીગ્રેશન}: સ્માર્ટ હોમ, વેરેબલ ડિવાઇસેસ
\end{itemize}

\end{solutionbox}
\begin{mnemonicbox}
``BHEE: બિઝનેસ, હેલ્થકેર, એજ્યુકેશન, મનોરંજન''

\end{mnemonicbox}
\subsection*{પ્રશ્ન 3(ક) [07
ગુણ]}\label{uxaaauxab0uxab6uxaa8-3uxa95-07-uxa97uxaa3}

\textbf{DHCP નું વર્કિંગ આકૃતિ સાથે સમજાવો અને તેના ફાયદા સમજાવો.}

\begin{solutionbox}

\textbf{DHCP (Dynamic Host Configuration Protocol)} નેટવર્ક ડિવાઇસેસને
ઓટોમેટિક IP એડ્રેસ આપે છે.

\textbf{DHCP પ્રક્રિયા (DORA):}

\begin{verbatim}
sequenceDiagram
    participant C as Client
    participant S as DHCP Server
    
    C{-S: 1. DHCP Discover (Broadcast)}
    S{-C: 2. DHCP Offer (IP + Config)}
    C{-S: 3. DHCP Request (Accept Offer)}
    S{-C: 4. DHCP Acknowledge (Confirm)}
\end{verbatim}

\textbf{પ્રદાન કરેલી કોન્ફિગરેશન માહિતી:}

\begin{itemize}
\tightlist
\item
  IP એડ્રેસ અને સબનેટ માસ્ક
\item
  ડિફોલ્ટ ગેટવે એડ્રેસ
\item
  DNS સર્વર એડ્રેસેસ
\item
  લીઝ અવધિ
\end{itemize}

\textbf{ફાયદા:}

\begin{itemize}
\tightlist
\item
  \textbf{ઓટોમેટિક કોન્ફિગરેશન}: મેન્યુઅલ IP અસાઇનમેન્ટ નહીં
\item
  \textbf{કેન્દ્રિત મેનેજમેન્ટ}: એક જ નિયંત્રણ બિંદુ
\item
  \textbf{કાર્યક્ષમ IP ઉપયોગ}: ડાયનેમિક એલોકેશન બગાડ અટકાવે
\item
  \textbf{ભૂલો ઘટાડો}: મેન્યુઅલ કોન્ફિગરેશન ભૂલો દૂર કરે
\item
  \textbf{સરળ મેન્ટેનન્સ}: સરળ નેટવર્ક ફેરફારો
\end{itemize}

\textbf{DHCP મેસેજ પ્રકારો:}

\begin{itemize}
\tightlist
\item
  DISCOVER, OFFER, REQUEST, ACK, NAK, RELEASE, RENEW
\end{itemize}

\end{solutionbox}
\begin{mnemonicbox}
``DORA: ડિસ્કવર, ઓફર, રિક્વેસ્ટ, એકનોલેજ''

\end{mnemonicbox}
\subsection*{પ્રશ્ન 3(અ OR) [03
ગુણ]}\label{uxaaauxab0uxab6uxaa8-3uxa85-or-03-uxa97uxaa3}

\textbf{HTTPS નું મહત્વ લખો.}

\begin{solutionbox}

\textbf{HTTPS (HyperText Transfer Protocol Secure)} સુરક્ષિત વેબ કમ્યુનિકેશન
પ્રદાન કરે છે.

\textbf{HTTPS નું મહત્વ:}

\begin{itemize}
\tightlist
\item
  \textbf{ડેટા એન્ક્રિપ્શન}: SSL/TLS નો ઉપયોગ કરીને ટ્રાન્ઝિટમાં ડેટાને સુરક્ષિત કરે
\item
  \textbf{ઓથેન્ટિકેશન}: સર્ટિફિકેટ્સ સાથે સર્વર આઇડેન્ટિટી વેરિફાઇ કરે
\item
  \textbf{ડેટા ઇન્ટેગ્રિટી}: ટ્રાન્સમિશન દરમિયાન ડેટા ટેમ્પરિંગ અટકાવે
\item
  \textbf{વિશ્વાસ નિર્માણ}: વેબસાઇટ્સમાં યુઝર કોન્ફિડન્સ વધારે
\end{itemize}

\textbf{સિક્યોરિટી લાભો:}

\begin{itemize}
\tightlist
\item
  ઇવ્સડ્રોપિંગ અને મેન-ઇન-ધ-મિડલ એટેક સામે રક્ષણ
\end{itemize}

\end{solutionbox}
\begin{mnemonicbox}
``EADT: એન્ક્રિપ્શન, ઓથેન્ટિકેશન, ઇન્ટેગ્રિટી, વિશ્વાસ''

\end{mnemonicbox}
\subsection*{પ્રશ્ન 3(બ OR) [04
ગુણ]}\label{uxaaauxab0uxab6uxaa8-3uxaac-or-04-uxa97uxaa3}

\textbf{બેરર નેટવર્ક શું છે? તે વિગતે સમજાવો.}

\begin{solutionbox}

\textbf{બેરર નેટવર્ક} એ અંતર્ગત નેટવર્ક ઇન્ફ્રાસ્ટ્રક્ચર છે જે એન્ડપોઇન્ટ્સ વચ્ચે ડેટા
ટ્રાફિક વહન કરે છે.

\textbf{બેરર નેટવર્ક્સના પ્રકારો:}

{\def\LTcaptype{none} % do not increment counter
\begin{longtable}[]{@{}
  >{\raggedright\arraybackslash}p{(\linewidth - 4\tabcolsep) * \real{0.1765}}
  >{\raggedright\arraybackslash}p{(\linewidth - 4\tabcolsep) * \real{0.3529}}
  >{\raggedright\arraybackslash}p{(\linewidth - 4\tabcolsep) * \real{0.4706}}@{}}
\toprule\noalign{}
\begin{minipage}[b]{\linewidth}\raggedright
પ્રકાર
\end{minipage} & \begin{minipage}[b]{\linewidth}\raggedright
ટેકનોલોજી
\end{minipage} & \begin{minipage}[b]{\linewidth}\raggedright
લક્ષણો
\end{minipage} \\
\midrule\noalign{}
\endhead
\bottomrule\noalign{}
\endlastfoot
\textbf{Circuit-Switched} & પરંપરાગત ટેલિફોની & સમર્પિત પાથ, ગેરંટીડ
બેન્ડવિડ્થ \\
\textbf{Packet-Switched} & ઇન્ટરનેટ, IP networks & શેર્ડ રિસોર્સ, વેરિએબલ
બેન્ડવિડ્થ \\
\textbf{વાયરલેસ} & સેલ્યુલર, WiFi & મોબાઇલ કનેક્ટિવિટી, એર ઇન્ટરફેસ \\
\end{longtable}
}

\textbf{ફંક્શન્સ:}

\begin{itemize}
\tightlist
\item
  \textbf{ડેટા ટ્રાન્સપોર્ટ}: યુઝર ડેટા અને સિગ્નલિંગ વહન કરે
\item
  \textbf{Quality of Service}: બેન્ડવિડ્થ અને લેટન્સી મેનેજ કરે
\item
  \textbf{રાઉટિંગ}: નેટવર્ક્સ વચ્ચે ટ્રાફિક ડાયરેક્ટ કરે
\item
  \textbf{નેટવર્ક મેનેજમેન્ટ}: ટ્રાફિક મોનિટર અને કંટ્રોલ કરે
\end{itemize}

\textbf{ઉદાહરણો:}

\begin{itemize}
\tightlist
\item
  PSTN, ઇન્ટરનેટ બેકબોન, 4G/5G સેલ્યુલર નેટવર્ક્સ
\end{itemize}

\end{solutionbox}
\begin{mnemonicbox}
``DQRN: ડેટા ટ્રાન્સપોર્ટ, QoS, રાઉટિંગ, નેટવર્ક મેનેજમેન્ટ''

\end{mnemonicbox}
\subsection*{પ્રશ્ન 3(ક OR) [07
ગુણ]}\label{uxaaauxab0uxab6uxaa8-3uxa95-or-07-uxa97uxaa3}

\textbf{TCP ના પ્રકાર લિસ્ટ કરો અને તેમાંથી કોઈ પણ એક સમજાવો.}

\begin{solutionbox}

\textbf{TCP ના પ્રકારો:}

\begin{itemize}
\tightlist
\item
  \textbf{સ્ટાન્ડાર્ડ TCP (TCP Tahoe)}
\item
  \textbf{TCP Reno}
\item
  \textbf{TCP New Reno}\\
\item
  \textbf{TCP Vegas}
\item
  \textbf{TCP SACK (Selective Acknowledgment)}
\item
  \textbf{TCP Cubic}
\end{itemize}

\textbf{TCP Reno - વિગતવાર સમજાવટ:}

\textbf{લક્ષણો:}

\begin{itemize}
\tightlist
\item
  \textbf{ફાસ્ટ રિટ્રાન્સમિટ}: ખોવાયેલા પેકેટ્સ ઝડપથી ફરીથી મોકલે
\item
  \textbf{ફાસ્ટ રિકવરી}: ફાસ્ટ રિટ્રાન્સમિટ પછી સ્લો સ્ટાર્ટ ટાળે
\item
  \textbf{કન્જેશન એવોઇડન્સ}: કન્જેશન વિન્ડોમાં લિનિયર વધારો
\item
  \textbf{ડુપ્લિકેટ ACK ડિટેક્શન}: પેકેટ લોસ ઓળખે
\end{itemize}

\textbf{કન્જેશન કંટ્રોલ અલ્ગોરિધમ:}

\begin{center}
\textbf{Mermaid Diagram (Code)}
\begin{verbatim}
{Shaded}
{Highlighting}[]
graph LR
    A[Slow Start] {-{-}{} B\{3 Duplicate ACKs?\}}
    B {-{-}{}|હા| C[Fast Retransmit]}
    C {-{-}{} D[Fast Recovery]}
    D {-{-}{} E[Congestion Avoidance]}
    B {-{-}{}|ના| F[Timeout?]}
    F {-{-}{}|હા| A}
    F {-{-}{}|ના| E}
{Highlighting}
{Shaded}
\end{verbatim}
\end{center}

\textbf{ફાયદા:}

\begin{itemize}
\tightlist
\item
  \textbf{વધુ સારી પર્ફોર્મન્સ}: પેકેટ લોસથી ઝડપી રિકવરી
\item
  \textbf{કાર્યક્ષમતા}: ઉચ્ચ થ્રુપુટ જાળવે
\item
  \textbf{ન્યાયીપણું}: સમાન બેન્ડવિડ્થ વહેંચણી
\end{itemize}

\textbf{વિન્ડો મેનેજમેન્ટ:}

\begin{itemize}
\tightlist
\item
  સ્લો સ્ટાર્ટમાં એક્સપોનેન્શિયલ વૃદ્ધિ
\item
  કન્જેશન એવોઇડન્સમાં લિનિયર વૃદ્ધિ
\end{itemize}

\end{solutionbox}
\begin{mnemonicbox}
``FFCE: ફાસ્ટ રિટ્રાન્સમિટ, ફાસ્ટ રિકવરી, કન્જેશન એવોઇડન્સ,
કાર્યક્ષમતા''

\end{mnemonicbox}
\subsection*{પ્રશ્ન 4(અ) [03
ગુણ]}\label{uxaaauxab0uxab6uxaa8-4uxa85-03-uxa97uxaa3}

\textbf{WLAN વ્યાખ્યાયિત કરો. WLAN ના પ્રકારો લિસ્ટ કરો.}

\begin{solutionbox}

\textbf{WLAN (Wireless Local Area Network)} મર્યાદિત વિસ્તારમાં વાયરલેસ
કનેક્ટિવિટી પ્રદાન કરે છે.

\textbf{WLAN ના પ્રકારો:}

\begin{itemize}
\tightlist
\item
  \textbf{ઇન્ફ્રાસ્ટ્રક્ચર મોડ}: કનેક્ટિવિટી માટે એક્સેસ પોઇન્ટ્સનો ઉપયોગ
\item
  \textbf{એડ-હોક મોડ}: સીધો ડિવાઇસ-ટુ-ડિવાઇસ કમ્યુનિકેશન
\item
  \textbf{મેશ નેટવર્ક્સ}: મલ્ટિ-હોપ વાયરલેસ કનેક્ટિવિટી
\item
  \textbf{હાઇબ્રિડ નેટવર્ક્સ}: ઇન્ફ્રાસ્ટ્રક્ચર અને એડ-હોકનું કોમ્બિનેશન
\end{itemize}

\textbf{સ્ટાન્ડાર્ડ્સ:}

\begin{itemize}
\tightlist
\item
  IEEE 802.11a/b/g/n/ac/ax (WiFi 6)
\end{itemize}

\end{solutionbox}
\begin{mnemonicbox}
``IAMH: ઇન્ફ્રાસ્ટ્રક્ચર, એડ-હોક, મેશ, હાઇબ્રિડ''

\end{mnemonicbox}
\subsection*{પ્રશ્ન 4(બ) [04
ગુણ]}\label{uxaaauxab0uxab6uxaa8-4uxaac-04-uxa97uxaa3}

\textbf{રાઉટિંગ શું છે? રાઉટિંગના પ્રકાર સમજાવો.}

\begin{solutionbox}

\textbf{રાઉટિંગ} એ નેટવર્ક્સ પર ડેટા પેકેટ્સ માટે પાથ સિલેક્ટ કરવાની પ્રક્રિયા છે.

\textbf{રાઉટિંગના પ્રકારો:}

{\def\LTcaptype{none} % do not increment counter
\begin{longtable}[]{@{}lll@{}}
\toprule\noalign{}
પ્રકાર & મેથડ & લક્ષણો \\
\midrule\noalign{}
\endhead
\bottomrule\noalign{}
\endlastfoot
\textbf{સ્ટેટિક રાઉટિંગ} & મેન્યુઅલ કોન્ફિગરેશન & નિયત પાથ, કોઈ ઓટોમેટિક અપડેટ્સ
નહીં \\
\textbf{ડાયનેમિક રાઉટિંગ} & ઓટોમેટિક અપડેટ્સ & અનુકૂલનશીલ પાથ, રિયલ-ટાઇમ
ફેરફારો \\
\textbf{ડિફોલ્ટ રાઉટિંગ} & કેચ-ઓલ રૂટ & જ્યારે કોઈ સ્પેસિફિક રૂટ અસ્તિત્વમાં ન
હોય \\
\textbf{ડિસ્ટન્સ વેક્ટર} & હોપ કાઉન્ટ આધારિત & RIP પ્રોટોકોલ, સરળ અમલીકરણ \\
\textbf{લિંક સ્ટેટ} & નેટવર્ક ટોપોલોજી & OSPF પ્રોટોકોલ, ઝડપી કન્વર્જન્સ \\
\end{longtable}
}

\textbf{ડાયનેમિક રાઉટિંગના ફાયદા:}

\begin{itemize}
\tightlist
\item
  \textbf{ઓટોમેટિક અનુકૂલન} નેટવર્ક ફેરફારો માટે
\item
  \textbf{લોડ બેલેન્સિંગ} મલ્ટિપલ પાથ પર
\item
  \textbf{ફોલ્ટ ટોલરન્સ} વૈકલ્પિક રૂટ્સ સાથે
\end{itemize}

\end{solutionbox}
\begin{mnemonicbox}
``SDDL: સ્ટેટિક, ડાયનેમિક, ડિફોલ્ટ, લિંક-સ્ટેટ''

\end{mnemonicbox}
\subsection*{પ્રશ્ન 4(ક) [07
ગુણ]}\label{uxaaauxab0uxab6uxaa8-4uxa95-07-uxa97uxaa3}

\textbf{WLAN નું આર્કિટેક્ચર સમજાવો.}

\begin{solutionbox}

\textbf{WLAN આર્કિટેક્ચર કોમ્પોનન્ટ્સ:}

\begin{verbatim}
graph TB
    subgraph "Basic Service Set (BSS)"
        A[Access Point] 
        B[Station 1]
        C[Station 2]
        D[Station 3]
    end
    
    subgraph "Extended Service Set (ESS)"
        E[AP1] {-{-} F[Distribution System]}
        G[AP2] {-{-} F}
        H[AP3] {-{-} F}
    end
    
    A {-{-} B}
    A {-{-} C  }
    A {-{-} D}
    F {-{-} I[Wired Network/Internet]}
\end{verbatim}

\textbf{આર્કિટેક્ચર એલિમન્ટ્સ:}

\begin{itemize}
\tightlist
\item
  \textbf{સ્ટેશન (STA)}: વાયરલેસ ક્લાયન્ટ ડિવાઇસેસ
\item
  \textbf{એક્સેસ પોઇન્ટ (AP)}: કેન્દ્રિય વાયરલેસ હબ
\item
  \textbf{બેસિક સર્વિસ સેટ (BSS)}: સિંગલ AP કવરેજ એરિયા
\item
  \textbf{એક્સટેન્ડેડ સર્વિસ સેટ (ESS)}: મલ્ટિપલ ઇન્ટરકનેક્ટેડ AP
\item
  \textbf{ડિસ્ટ્રિબ્યુશન સિસ્ટમ (DS)}: AP ને જોડતું બેકએન્ડ નેટવર્ક
\end{itemize}

\textbf{WLAN ટોપોલોજીઝ:}

\begin{itemize}
\tightlist
\item
  \textbf{ઇન્ફ્રાસ્ટ્રક્ચર મોડ}: AP દ્વારા કેન્દ્રિત
\item
  \textbf{એડ-હોક મોડ}: સીધો પીઅર-ટુ-પીઅર કમ્યુનિકેશન
\item
  \textbf{મેશ ટોપોલોજી}: મલ્ટિ-હોપ વાયરલેસ કનેક્શન્સ
\end{itemize}

\textbf{પ્રદાન કરેલી સેવાઓ:}

\begin{itemize}
\tightlist
\item
  \textbf{એસોસિએશન}: AP સાથે ડિવાઇસ કનેક્શન
\item
  \textbf{ઓથેન્ટિકેશન}: સિક્યોરિટી વેરિફિકેશન
\item
  \textbf{ડેટા ડિલિવરી}: પેકેટ ટ્રાન્સમિશન
\item
  \textbf{રોમિંગ}: AP વચ્ચે સીમલેસ મૂવમેન્ટ
\end{itemize}

\textbf{ફ્રીક્વન્સી બેન્ડ્સ:}

\begin{itemize}
\tightlist
\item
  2.4 GHz (802.11b/g/n)
\item
  5 GHz (802.11a/n/ac/ax)
\end{itemize}

\end{solutionbox}
\begin{mnemonicbox}
``SABED: સ્ટેશન, એક્સેસ પોઇન્ટ, BSS, ESS, ડિસ્ટ્રિબ્યુશન
સિસ્ટમ''

\end{mnemonicbox}
\subsection*{પ્રશ્ન 4(અ OR) [03
ગુણ]}\label{uxaaauxab0uxab6uxaa8-4uxa85-or-03-uxa97uxaa3}

\textbf{WPAN વ્યાખ્યાયિત કરો. WPAN ની ઉપયોગિતા લિસ્ટ કરો.}

\begin{solutionbox}

\textbf{WPAN (Wireless Personal Area Network)} વ્યક્તિગત જગ્યામાં ડિવાઇસેસ
જોડે છે (સામાન્ય રીતે 10 મીટર).

\textbf{WPAN ની ઉપયોગિતા:}

\begin{itemize}
\tightlist
\item
  \textbf{ડિવાઇસ સિંક્રોનાઇઝેશન}: ફોનથી કમ્પ્યુટર ડેટા ટ્રાન્સફર
\item
  \textbf{ઓડિયો સ્ટ્રીમિંગ}: વાયરલેસ હેડફોન્સ, સ્પીકર્સ
\item
  \textbf{ઇનપુટ ડિવાઇસેસ}: વાયરલેસ કીબોર્ડ, માઉસ
\item
  \textbf{હેલ્થકેર}: મેડિકલ સેન્સર્સ, ફિટનેસ ટ્રેકર્સ
\item
  \textbf{સ્માર્ટ હોમ}: IoT ડિવાઇસ કંટ્રોલ
\end{itemize}

\textbf{ટેકનોલોજીઝ:}

\begin{itemize}
\tightlist
\item
  Bluetooth, Zigbee, NFC, infrared
\end{itemize}

\end{solutionbox}
\begin{mnemonicbox}
``DSAHS: ડિવાઇસ સિંક, સ્ટ્રીમિંગ, ઓડિયો, હેલ્થકેર, સ્માર્ટ
હોમ''

\end{mnemonicbox}
\subsection*{પ્રશ્ન 4(બ OR) [04
ગુણ]}\label{uxaaauxab0uxab6uxaa8-4uxaac-or-04-uxa97uxaa3}

\textbf{IMAP પ્રોટોકોલનું વર્કિંગ સમજાવો.}

\begin{solutionbox}

\textbf{IMAP (Internet Message Access Protocol)} મેઇલ સર્વર પર ઇમેઇલ મેનેજ કરે
છે.

\textbf{IMAP વર્કિંગ પ્રોસેસ:}

{\def\LTcaptype{none} % do not increment counter
\begin{longtable}[]{@{}
  >{\raggedright\arraybackslash}p{(\linewidth - 4\tabcolsep) * \real{0.2222}}
  >{\raggedright\arraybackslash}p{(\linewidth - 4\tabcolsep) * \real{0.2963}}
  >{\raggedright\arraybackslash}p{(\linewidth - 4\tabcolsep) * \real{0.4815}}@{}}
\toprule\noalign{}
\begin{minipage}[b]{\linewidth}\raggedright
સ્તર
\end{minipage} & \begin{minipage}[b]{\linewidth}\raggedright
ક્રિયા
\end{minipage} & \begin{minipage}[b]{\linewidth}\raggedright
વર્ણન
\end{minipage} \\
\midrule\noalign{}
\endhead
\bottomrule\noalign{}
\endlastfoot
\textbf{કનેક્શન} & ક્લાયન્ટ સર્વર સાથે કનેક્ટ થાય & પોર્ટ 143/993 પર TCP કનેક્શન
સ્થાપિત કરે \\
\textbf{ઓથેન્ટિકેશન} & લોગિન ક્રેડેન્શિયલ્સ & યુઝરનેમ/પાસવર્ડ વેરિફિકેશન \\
\textbf{મેઇલબોક્સ સિલેક્શન} & ફોલ્ડર પસંદ કરો & INBOX અથવા અન્ય ફોલ્ડર્સ સિલેક્ટ
કરો \\
\textbf{મેસેજ ઓપરેશન્સ} & વાંચો/ડિલીટ/ફ્લેગ & સર્વર પર મેસેજ્સ મેનિપ્યુલેટ કરો \\
\end{longtable}
}

\textbf{IMAP vs POP3:}

\begin{itemize}
\tightlist
\item
  \textbf{સર્વર સ્ટોરેજ}: મેસેજ્સ સર્વર પર રહે છે
\item
  \textbf{મલ્ટિ-ડિવાઇસ એક્સેસ}: ડિવાઇસેસ પર સિંક
\item
  \textbf{ફોલ્ડર મેનેજમેન્ટ}: સર્વર-સાઇડ ફોલ્ડર સ્ટ્રક્ચર
\item
  \textbf{પાર્શિયલ ડાઉનલોડ}: પહેલા હેડર્સ, માંગ પર બોડી
\end{itemize}

\textbf{IMAP કમાન્ડ્સ:}

\begin{verbatim}
LOGIN user password
SELECT INBOX
FETCH 1 BODY[]
STORE 1 +FLAGS (\Deleted)
\end{verbatim}

\end{solutionbox}
\begin{mnemonicbox}
``CAMS: કનેક્શન, ઓથેન્ટિકેશન, મેઇલબોક્સ, સ્ટોરેજ''

\end{mnemonicbox}
\subsection*{પ્રશ્ન 4(ક OR) [07
ગુણ]}\label{uxaaauxab0uxab6uxaa8-4uxa95-or-07-uxa97uxaa3}

\textbf{બ્લૂટૂથ ટેકનોલોજી તેના પ્રોટોકોલ સ્ટેક સાથે સમજાવો.}

\begin{solutionbox}

\textbf{બ્લૂટૂથ} એ પર્સનલ એરિયા નેટવર્ક્સ માટે શોર્ટ-રેન્જ વાયરલેસ કમ્યુનિકેશન ટેકનોલોજી
છે.

\textbf{બ્લૂટૂથ પ્રોટોકોલ સ્ટેક:}

\begin{center}
\textbf{Mermaid Diagram (Code)}
\begin{verbatim}
{Shaded}
{Highlighting}[]
graph LR
    A[Applications] {-{-}{} B[OBEX/SDP]}
    B {-{-}{} C[RFCOMM/L2CAP]}
    C {-{-}{} D[HCI {-} Host Controller Interface]}
    D {-{-}{} E[LMP {-} Link Manager Protocol]}
    E {-{-}{} F[Baseband/LC {-} Link Controller]}
    F {-{-}{} G[Radio Layer]}
{Highlighting}
{Shaded}
\end{verbatim}
\end{center}

\textbf{લેયર ફંક્શન્સ:}

\begin{itemize}
\tightlist
\item
  \textbf{રેડિયો લેયર}: 2.4 GHz ISM બેન્ડ, ફ્રીક્વન્સી હોપિંગ
\item
  \textbf{બેસબેન્ડ}: ટાઇમિંગ, એક્સેસ કંટ્રોલ, પેકેટ ફોર્મેટ્સ
\item
  \textbf{LMP}: લિંક સ્થાપના, સિક્યોરિટી, પાવર મેનેજમેન્ટ
\item
  \textbf{L2CAP}: પેકેટ સેગમેન્ટેશન, પ્રોટોકોલ મલ્ટિપ્લેક્સિંગ
\item
  \textbf{RFCOMM}: વાયરલેસ પર સીરિયલ પોર્ટ એમ્યુલેશન
\item
  \textbf{SDP}: સર્વિસ ડિસ્કવરી પ્રોટોકોલ
\item
  \textbf{એપ્લિકેશન્સ}: ફાઇલ ટ્રાન્સફર, ઓડિયો સ્ટ્રીમિંગ, HID
\end{itemize}

\textbf{બ્લૂટૂથ લક્ષણો:}

\begin{itemize}
\tightlist
\item
  \textbf{રેન્જ}: 10 મીટર (Class 2 ડિવાઇસેસ)
\item
  \textbf{ડેટા રેટ}: 1-3 Mbps (વર્ઝન આધારે)
\item
  \textbf{ટોપોલોજી}: સ્ટાર નેટવર્ક (piconet)
\item
  \textbf{સિક્યોરિટી}: ઓથેન્ટિકેશન, ઓથરાઇઝેશન, એન્ક્રિપ્શન
\end{itemize}

\textbf{બ્લૂટૂથ વર્ઝન્સ:}

\begin{itemize}
\tightlist
\item
  ક્લાસિક બ્લૂટૂથ (BR/EDR)
\item
  બ્લૂટૂથ લો એનર્જી (BLE/LE)
\item
  બ્લૂટૂથ 5.0+ (એન્હાન્સ્ડ રેન્જ/સ્પીડ)
\end{itemize}

\textbf{એપ્લિકેશન્સ:}

\begin{itemize}
\tightlist
\item
  ઓડિયો ડિવાઇસેસ, કીબોર્ડ્સ, ફાઇલ ટ્રાન્સફર, IoT સેન્સર્સ
\end{itemize}

\end{solutionbox}
\begin{mnemonicbox}
``RBLSRA: રેડિયો, બેસબેન્ડ, LMP, SDP, RFCOMM, એપ્લિકેશન્સ''

\end{mnemonicbox}
\subsection*{પ્રશ્ન 5(અ) [03
ગુણ]}\label{uxaaauxab0uxab6uxaa8-5uxa85-03-uxa97uxaa3}

\textbf{4G શું છે? 4G ના ફીચર્સ લિસ્ટ કરો.}

\begin{solutionbox}

\textbf{4G (Fourth Generation)} એ હાઇ-સ્પીડ વાયરલેસ ઇન્ટરનેટ પ્રદાન કરતો
મોબાઇલ કમ્યુનિકેશન સ્ટાન્ડાર્ડ છે.

\textbf{4G ના ફીચર્સ:}

\begin{itemize}
\tightlist
\item
  \textbf{હાઇ ડેટા સ્પીડ}: મોબાઇલ પર 100 Mbps, સ્ટેશનરી પર 1 Gbps સુધી
\item
  \textbf{ઓલ-IP નેટવર્ક}: પેકેટ-સ્વિચ્ડ આર્કિટેક્ચર
\item
  \textbf{લો લેટન્સી}: રિયલ-ટાઇમ એપ્લિકેશન્સ માટે ઓછો વિલંબ
\item
  \textbf{Quality of Service}: ગેરંટીડ સર્વિસ લેવલ્સ
\item
  \textbf{ગ્લોબલ રોમિંગ}: વિશ્વવ્યાપી સુસંગતતા
\end{itemize}

\textbf{ટેકનોલોજીઝ:}

\begin{itemize}
\tightlist
\item
  LTE (Long Term Evolution), WiMAX
\end{itemize}

\end{solutionbox}
\begin{mnemonicbox}
``HALQG: હાઇ-સ્પીડ, ઓલ-IP, લો લેટન્સી, QoS, ગ્લોબલ
રોમિંગ''

\end{mnemonicbox}
\subsection*{પ્રશ્ન 5(બ) [04
ગુણ]}\label{uxaaauxab0uxab6uxaa8-5uxaac-04-uxa97uxaa3}

\textbf{સેન્ટ્રલાઇઝ્ડ કમ્પ્યુટિંગ સમજાવો.}

\begin{solutionbox}

\textbf{સેન્ટ્રલાઇઝ્ડ કમ્પ્યુટિંગ} કેન્દ્રિય સર્વર પર બધા ડેટા અને એપ્લિકેશન્સ પ્રોસેસ કરે
છે.

\textbf{આર્કિટેક્ચર:}

\begin{verbatim}
graph TB
    A[Central Server] {-{-} B[Terminal 1]}
    A {-{-} C[Terminal 2] }
    A {-{-} D[Terminal 3]}
    A {-{-} E[Terminal 4]}
    
    F[Processing Power]
    G[Storage]
    H[Applications]
    
    F {-{-} A}
    G {-{-} A}
    H {-{-} A}
\end{verbatim}

\textbf{લક્ષણો:}

\begin{itemize}
\tightlist
\item
  \textbf{સિંગલ પોઇન્ટ ઓફ કંટ્રોલ}: કેન્દ્રિય સ્થાને બધી પ્રોસેસિંગ
\item
  \textbf{થિન ક્લાયન્ટ્સ}: ન્યૂનતમ લોકલ પ્રોસેસિંગ ક્ષમતા
\item
  \textbf{શેર્ડ રિસોર્સ}: CPU, મેમરી, સ્ટોરેજ કેન્દ્રિય રીતે મેનેજ
\item
  \textbf{નેટવર્ક ડિપેન્ડન્ટ}: વિશ્વસનીય નેટવર્ક કનેક્ટિવિટી જરૂરી
\end{itemize}

\textbf{ફાયદા:}

\begin{itemize}
\tightlist
\item
  \textbf{સિક્યોરિટી}: કેન્દ્રિત ડેટા પ્રોટેક્શન
\item
  \textbf{મેનેજમેન્ટ}: સરળ સિસ્ટમ એડમિનિસ્ટ્રેશન
\item
  \textbf{ખર્ચ}: ક્લાયન્ટ-સાઇડ હાર્ડવેર ખર્ચ ઓછો
\end{itemize}

\textbf{નુકસાનો:}

\begin{itemize}
\tightlist
\item
  \textbf{સિંગલ પોઇન્ટ ઓફ ફેઇલ્યર}: સર્વર ડાઉનટાઇમ બધા યુઝર્સને અસર કરે
\item
  \textbf{નેટવર્ક બોટલનેક}: નેટવર્ક પર્ફોર્મન્સ પર ભારે નિર્ભરતા
\end{itemize}

\end{solutionbox}
\begin{mnemonicbox}
``SSNG: સિંગલ કંટ્રોલ, શેર્ડ રિસોર્સ, નેટવર્ક ડિપેન્ડન્ટ, વધુ
સિક્યોરિટી''

\end{mnemonicbox}
\subsection*{પ્રશ્ન 5(ક) [07
ગુણ]}\label{uxaaauxab0uxab6uxaa8-5uxa95-07-uxa97uxaa3}

\textbf{IPv4 શું છે? IPv4 નું વર્કિંગ ડાયાગ્રામ સાથે સમજાવો.}

\begin{solutionbox}

\textbf{IPv4 (Internet Protocol version 4)} નેટવર્ક ઓળખ માટે 32-બિટ એડ્રેસનો
ઉપયોગ કરે છે.

\textbf{IPv4 એડ્રેસ સ્ટ્રક્ચર:}

\begin{verbatim}
 0                   1                   2                   3
 0 1 2 3 4 5 6 7 8 9 0 1 2 3 4 5 6 7 8 9 0 1 2 3 4 5 6 7 8 9 0 1
+{-+{-}+{-}+{-}+{-}+{-}+{-}+{-}+{-}+{-}+{-}+{-}+{-}+{-}+{-}+{-}+{-}+{-}+{-}+{-}+{-}+{-}+{-}+{-}+{-}+{-}+{-}+{-}+{-}+{-}+{-}+{-}+}
|                        Network Address                        |
+{-+{-}+{-}+{-}+{-}+{-}+{-}+{-}+{-}+{-}+{-}+{-}+{-}+{-}+{-}+{-}+{-}+{-}+{-}+{-}+{-}+{-}+{-}+{-}+{-}+{-}+{-}+{-}+{-}+{-}+{-}+{-}+}
|                         Host Address                          |
+{-+{-}+{-}+{-}+{-}+{-}+{-}+{-}+{-}+{-}+{-}+{-}+{-}+{-}+{-}+{-}+{-}+{-}+{-}+{-}+{-}+{-}+{-}+{-}+{-}+{-}+{-}+{-}+{-}+{-}+{-}+{-}+}
\end{verbatim}

\textbf{IPv4 એડ્રેસ ક્લાસેસ:}

{\def\LTcaptype{none} % do not increment counter
\begin{longtable}[]{@{}lllll@{}}
\toprule\noalign{}
ક્લાસ & રેન્જ & નેટવર્ક બિટ્સ & હોસ્ટ બિટ્સ & ડિફોલ્ટ સબનેટ માસ્ક \\
\midrule\noalign{}
\endhead
\bottomrule\noalign{}
\endlastfoot
\textbf{A} & 1-126 & 8 & 24 & 255.0.0.0 \\
\textbf{B} & 128-191 & 16 & 16 & 255.255.0.0 \\
\textbf{C} & 192-223 & 24 & 8 & 255.255.255.0 \\
\textbf{D} & 224-239 & મલ્ટિકાસ્ટ & - & - \\
\textbf{E} & 240-255 & પ્રયોગાત્મક & - & - \\
\end{longtable}
}

\textbf{IPv4 પેકેટ હેડર:}

\begin{verbatim}
 0                   1                   2                   3
 0 1 2 3 4 5 6 7 8 9 0 1 2 3 4 5 6 7 8 9 0 1 2 3 4 5 6 7 8 9 0 1
+{-+{-}+{-}+{-}+{-}+{-}+{-}+{-}+{-}+{-}+{-}+{-}+{-}+{-}+{-}+{-}+{-}+{-}+{-}+{-}+{-}+{-}+{-}+{-}+{-}+{-}+{-}+{-}+{-}+{-}+{-}+{-}+}
|Version|  IHL  |Type of Service|          Total Length         |
+{-+{-}+{-}+{-}+{-}+{-}+{-}+{-}+{-}+{-}+{-}+{-}+{-}+{-}+{-}+{-}+{-}+{-}+{-}+{-}+{-}+{-}+{-}+{-}+{-}+{-}+{-}+{-}+{-}+{-}+{-}+{-}+}
|         Identification        |Flags|      Fragment Offset    |
+{-+{-}+{-}+{-}+{-}+{-}+{-}+{-}+{-}+{-}+{-}+{-}+{-}+{-}+{-}+{-}+{-}+{-}+{-}+{-}+{-}+{-}+{-}+{-}+{-}+{-}+{-}+{-}+{-}+{-}+{-}+{-}+}
|  Time to Live |    Protocol   |         Header Checksum       |
+{-+{-}+{-}+{-}+{-}+{-}+{-}+{-}+{-}+{-}+{-}+{-}+{-}+{-}+{-}+{-}+{-}+{-}+{-}+{-}+{-}+{-}+{-}+{-}+{-}+{-}+{-}+{-}+{-}+{-}+{-}+{-}+}
|                       Source Address                          |
+{-+{-}+{-}+{-}+{-}+{-}+{-}+{-}+{-}+{-}+{-}+{-}+{-}+{-}+{-}+{-}+{-}+{-}+{-}+{-}+{-}+{-}+{-}+{-}+{-}+{-}+{-}+{-}+{-}+{-}+{-}+{-}+}
|                    Destination Address                        |
+{-+{-}+{-}+{-}+{-}+{-}+{-}+{-}+{-}+{-}+{-}+{-}+{-}+{-}+{-}+{-}+{-}+{-}+{-}+{-}+{-}+{-}+{-}+{-}+{-}+{-}+{-}+{-}+{-}+{-}+{-}+{-}+}
|                    Options                    |    Padding    |
+{-+{-}+{-}+{-}+{-}+{-}+{-}+{-}+{-}+{-}+{-}+{-}+{-}+{-}+{-}+{-}+{-}+{-}+{-}+{-}+{-}+{-}+{-}+{-}+{-}+{-}+{-}+{-}+{-}+{-}+{-}+{-}+}
\end{verbatim}

\textbf{વર્કિંગ પ્રક્રિયા:}

\begin{itemize}
\tightlist
\item
  \textbf{એડ્રેસ અસાઇનમેન્ટ}: નેટવર્ક એડમિનિસ્ટ્રેટર IP એડ્રેસ આપે
\item
  \textbf{રાઉટિંગ ડિસિઝન}: રાઉટર ડેસ્ટિનેશન IP તપાસે
\item
  \textbf{સબનેટ ડિટર્મિનેશન}: નેટવર્ક શોધવા સબનેટ માસ્ક લાગુ કરે
\item
  \textbf{પેકેટ ફોરવાર્ડિંગ}: યોગ્ય નેટવર્ક ઇન્ટરફેસ પર રૂટ કરે
\end{itemize}

\textbf{સ્પેશિયલ એડ્રેસેસ:}

\begin{itemize}
\tightlist
\item
  \textbf{લૂપબેક}: 127.0.0.1 (localhost)
\item
  \textbf{પ્રાઇવેટ}: 10.x.x.x, 172.16-31.x.x, 192.168.x.x
\item
  \textbf{બ્રોડકાસ્ટ}: 255.255.255.255
\end{itemize}

\textbf{મર્યાદાઓ:}

\begin{itemize}
\tightlist
\item
  \textbf{એડ્રેસ એક્ઝોશન}: માત્ર 4.3 બિલિયન એડ્રેસ
\item
  \textbf{બિનકાર્યક્ષમ ફાળવણી}: ક્લાસ-આધારિત બગાડ
\end{itemize}

\end{solutionbox}
\begin{mnemonicbox}
``ABCDE: એડ્રેસ ક્લાસ A, B, C, D મલ્ટિકાસ્ટ, E
પ્રયોગાત્મક''

\end{mnemonicbox}
\subsection*{પ્રશ્ન 5(અ OR) [03
ગુણ]}\label{uxaaauxab0uxab6uxaa8-5uxa85-or-03-uxa97uxaa3}

\textbf{5G શું છે? 5G ના ફીચર્સ લિસ્ટ કરો.}

\begin{solutionbox}

\textbf{5G (Fifth Generation)} એ વધારેલી ક્ષમતાઓ સાથે નવીનતમ મોબાઇલ
કમ્યુનિકેશન સ્ટાન્ડાર્ડ છે.

\textbf{5G ના ફીચર્સ:}

\begin{itemize}
\tightlist
\item
  \textbf{અલ્ટ્રા-હાઇ સ્પીડ}: 10 Gbps સુધીના ડેટા રેટ્સ
\item
  \textbf{અલ્ટ્રા-લો લેટન્સી}: 1ms કરતાં ઓછો રિસ્પોન્સ ટાઇમ
\item
  \textbf{મેસિવ કનેક્ટિવિટી}: પ્રતિ km^{2} 1 મિલિયન ડિવાઇસેસ
\item
  \textbf{નેટવર્ક સ્લાઇસિંગ}: વર્ચ્યુઅલ ડેડિકેટેડ નેટવર્ક્સ
\item
  \textbf{એન્હાન્સ્ડ મોબાઇલ બ્રોડબેન્ડ}: સુધારેલ યુઝર એક્સપિરિયન્સ
\end{itemize}

\textbf{મુખ્ય ટેકનોલોજીઝ:}

\begin{itemize}
\tightlist
\item
  મિલિમીટર વેવ, મેસિવ MIMO, બીમફોર્મિંગ
\end{itemize}

\end{solutionbox}
\begin{mnemonicbox}
``UUMNE: અલ્ટ્રા-સ્પીડ, અલ્ટ્રા-લો લેટન્સી, મેસિવ કનેક્ટિવિટી,
નેટવર્ક સ્લાઇસિંગ, એન્હાન્સ્ડ બ્રોડબેન્ડ''

\end{mnemonicbox}
\subsection*{પ્રશ્ન 5(બ OR) [04
ગુણ]}\label{uxaaauxab0uxab6uxaa8-5uxaac-or-04-uxa97uxaa3}

\textbf{ડિસ્ટ્રિબ્યુટેડ કમ્પ્યુટિંગ સમજાવો.}

\begin{solutionbox}

\textbf{ડિસ્ટ્રિબ્યુટેડ કમ્પ્યુટિંગ} મલ્ટિપલ ઇન્ટરકનેક્ટેડ કમ્પ્યુટર્સ પર પ્રોસેસિંગ વિતરિત
કરે છે.

\textbf{આર્કિટેક્ચર:}

\begin{center}
\textbf{Mermaid Diagram (Code)}
\begin{verbatim}
{Shaded}
{Highlighting}[]
graph LR
    subgraph "Distributed System"
        A[Node 1] {{-}{-}{} B[Node 2]}
        B {{-}{-}{} C[Node 3]}
        C {{-}{-}{} D[Node 4]}
        A {{-}{-}{} D}
    end
    
    E[Network] {-{-}{} A}
    E {-{-}{} B}
    E {-{-}{} C}
    E {-{-}{} D}
{Highlighting}
{Shaded}
\end{verbatim}
\end{center}

\textbf{લક્ષણો:}

\begin{itemize}
\tightlist
\item
  \textbf{રિસોર્સ શેરિંગ}: વિતરિત પ્રોસેસિંગ અને સ્ટોરેજ
\item
  \textbf{સ્કેલેબિલિટી}: ક્ષમતા વધારવા વધુ નોડ્સ ઉમેરો
\item
  \textbf{ફોલ્ટ ટોલરન્સ}: કેટલાક નોડ્સ ફેઇલ થાય તો સિસ્ટમ ચાલુ રહે
\item
  \textbf{લોકેશન ટ્રાન્સપેરન્સી}: યુઝર્સને રિસોર્સ લોકેશનની જાણ નથી
\end{itemize}

\textbf{ફાયદા:}

\begin{itemize}
\tightlist
\item
  \textbf{વિશ્વસનીયતા}: કોઈ સિંગલ પોઇન્ટ ઓફ ફેઇલ્યર નથી
\item
  \textbf{પર્ફોર્મન્સ}: પેરેલલ પ્રોસેસિંગ ક્ષમતાઓ
\item
  \textbf{ખર્ચ-અસરકારકતા}: કોમોડિટી હાર્ડવેરનો ઉપયોગ
\end{itemize}

\textbf{ઉદાહરણો:}

\begin{itemize}
\tightlist
\item
  ક્લાઉડ કમ્પ્યુટિંગ, પીઅર-ટુ-પીઅર નેટવર્ક્સ, ગ્રિડ કમ્પ્યુટિંગ
\end{itemize}

\end{solutionbox}
\begin{mnemonicbox}
``RSFL: રિસોર્સ શેરિંગ, સ્કેલેબિલિટી, ફોલ્ટ ટોલરન્સ, લોકેશન
ટ્રાન્સપેરન્સી''

\end{mnemonicbox}
\subsection*{પ્રશ્ન 5(ક OR) [07
ગુણ]}\label{uxaaauxab0uxab6uxaa8-5uxa95-or-07-uxa97uxaa3}

\textbf{ડેટા લિંક લેયર પ્રોટોકોલ સમજાવો.}

\begin{solutionbox}

\textbf{ડેટા લિંક લેયર} અડીને આવેલા નેટવર્ક નોડ્સ વચ્ચે વિશ્વસનીય ડેટા ટ્રાન્સફર પ્રદાન
કરે છે.

\textbf{ફંક્શન્સ:}

\begin{itemize}
\tightlist
\item
  \textbf{ફ્રેમિંગ}: બિટ્સને ફ્રેમ્સમાં ગોઠવો
\item
  \textbf{એરર ડિટેક્શન}: ટ્રાન્સમિશન એરર્સ ઓળખો
\item
  \textbf{એરર કરેક્શન}: શોધાયેલી એરર્સ સુધારો
\item
  \textbf{ફ્લો કંટ્રોલ}: ડેટા ટ્રાન્સમિશન રેટ મેનેજ કરો
\item
  \textbf{એક્સેસ કંટ્રોલ}: શેર્ડ મીડિયા એક્સેસ કોઓર્ડિનેટ કરો
\end{itemize}

\textbf{ફ્રેમ સ્ટ્રક્ચર:}

\begin{verbatim}
+{-{-}{-}{-}{-}{-}{-}{-}{-}{-}+{-}{-}{-}{-}{-}{-}{-}{-}{-}{-}+{-}{-}{-}{-}{-}{-}{-}{-}{-}{-}+{-}{-}{-}{-}{-}{-}{-}{-}{-}{-}+{-}{-}{-}{-}{-}{-}{-}{-}{-}{-}+}
| Start    | Address  | Control  | Data     | FCS      |
| Delimiter| Field    | Field    | Field    | (CRC)    |
+{-{-}{-}{-}{-}{-}{-}{-}{-}{-}+{-}{-}{-}{-}{-}{-}{-}{-}{-}{-}+{-}{-}{-}{-}{-}{-}{-}{-}{-}{-}+{-}{-}{-}{-}{-}{-}{-}{-}{-}{-}+{-}{-}{-}{-}{-}{-}{-}{-}{-}{-}+}
\end{verbatim}

\textbf{એરર ડિટેક્શન મેથડ્સ:}

{\def\LTcaptype{none} % do not increment counter
\begin{longtable}[]{@{}lll@{}}
\toprule\noalign{}
મેથડ & વર્ણન & ક્ષમતા \\
\midrule\noalign{}
\endhead
\bottomrule\noalign{}
\endlastfoot
\textbf{પેરિટી ચેક} & સિંગલ બિટ ઉમેરો & સિંગલ-બિટ એરર્સ શોધો \\
\textbf{ચેકસમ} & અંકગણિત સરવાળો & મલ્ટિપલ એરર્સ શોધો \\
\textbf{CRC} & પોલિનોમિયલ ડિવિઝન & બર્સ્ટ એરર્સ શોધો \\
\end{longtable}
}

\textbf{ફ્લો કંટ્રોલ પ્રોટોકોલ્સ:}

\begin{itemize}
\tightlist
\item
  \textbf{સ્ટોપ-એન્ડ-વેઇટ}: એક ફ્રેમ મોકલો, ACK ની રાહ જુઓ
\item
  \textbf{સ્લાઇડિંગ વિન્ડો}: ટ્રાન્ઝિટમાં મલ્ટિપલ ફ્રેમ્સ
\item
  \textbf{સ્ટોપ-એન્ડ-વેઇટ ARQ}: એરર રિકવરી ઉમેરો
\item
  \textbf{ગો-બેક-N ARQ}: એરર પોઇન્ટથી રિટ્રાન્સમિટ
\item
  \textbf{સિલેક્ટિવ રિપીટ}: માત્ર એરર ફ્રેમ્સ રિટ્રાન્સમિટ
\end{itemize}

\textbf{એક્સેસ કંટ્રોલ મેથડ્સ:}

\begin{itemize}
\tightlist
\item
  \textbf{CSMA/CD}: કેરિયર સેન્સ મલ્ટિપલ એક્સેસ વિથ કોલિઝન ડિટેક્શન
\item
  \textbf{CSMA/CA}: કોલિઝન એવોઇડન્સ
\item
  \textbf{ટોકન પાસિંગ}: ટોકનનો ઉપયોગ કરીને નિયંત્રિત એક્સેસ
\end{itemize}

\textbf{પ્રોટોકોલ ઉદાહરણો:}

\begin{itemize}
\tightlist
\item
  Ethernet, PPP, HDLC, LLC
\end{itemize}

\textbf{વર્કિંગ પ્રક્રિયા:}

\begin{verbatim}
sequenceDiagram
    participant S as Sender
    participant R as Receiver
    
    S{-R: Data Frame}
    R{-S: ACK Frame}
    S{-R: Next Data Frame}
    Note over R: Error Detected
    R{-S: NAK Frame}
    S{-R: Retransmit Frame}
\end{verbatim}

\end{solutionbox}
\begin{mnemonicbox}
``FECFA: ફ્રેમિંગ, એરર ડિટેક્શન, કરેક્શન, ફ્લો કંટ્રોલ, એક્સેસ
કંટ્રોલ''

\end{mnemonicbox}

\end{document}
