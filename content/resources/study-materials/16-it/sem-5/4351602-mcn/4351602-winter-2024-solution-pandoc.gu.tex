\documentclass[10pt,a4paper]{article}

% content/resources/templates/preamble.tex
\usepackage[margin=0.6in]{geometry}
\author{Milav Dabgar}
\usepackage{amsmath,amssymb,amsthm}
\usepackage{booktabs}
\usepackage{multirow}
\usepackage{xcolor}
\usepackage{tcolorbox}
\tcbuselibrary{breakable,skins}
\usepackage[colorlinks=true,linkcolor=blue]{hyperref}
\usepackage{titlesec}
\usepackage{enumitem}
\usepackage{tikz}
\usepackage{pgfplots}
\usepackage{circuitikz}
\usepackage[version=4]{mhchem}
\usepackage{longtable}
\usepackage{array}
\usepackage{float}
\usepackage{caption}
\usepackage{listings}

\lstset{
  basicstyle=\small\ttfamily,
  breaklines=true,
  breakatwhitespace=false,
  postbreak=\mbox{\textcolor{red}{$\hookrightarrow$}\space},
  float=false,
  numbers=left,
  numberstyle=\tiny\color{gray},
  numbersep=10pt,
  xleftmargin=2em,
  keywordstyle=\color{blue},
  commentstyle=\color{green!60!black},
  stringstyle=\color{purple},
  backgroundcolor=\color{gray!5},
  showstringspaces=false,
  tabsize=2,
  captionpos=b,
  keepspaces=true,
  columns=flexible
}

\pgfplotsset{compat=1.18}
\usetikzlibrary{shapes,arrows,positioning,calc,patterns,decorations.pathmorphing,decorations.markings,arrows.meta}

% Color scheme
\definecolor{headcolor}{RGB}{0,102,204}
\definecolor{keycolor}{RGB}{220,20,60}
\definecolor{solutioncolor}{RGB}{34,139,34}
\definecolor{mnemoniccolor}{RGB}{148,0,211}
\definecolor{codecolor}{RGB}{0,0,100}

% Spacing
\setlength{\parskip}{3pt}
\setlist[itemize]{nosep}
\setlist[enumerate]{nosep}

% Title formatting
\titleformat{\section}{\Large\bfseries\color{headcolor}}{\thesection}{1em}{}
\titleformat{\subsection}{\large\bfseries\color{headcolor}}{\thesubsection}{1em}{}

% Pandoc tightlist compatibility
\providecommand{\tightlist}{%
  \setlength{\itemsep}{0pt}\setlength{\parskip}{0pt}}

% Pandoc longtable compatibility
\newcounter{none}
\def\thenone{}


% content/resources/templates/gujarati-boxes.tex
\usepackage{fontspec}
\usepackage{polyglossia}

% Set Gujarati as main language (document is primarily in Gujarati)
% Note: gloss-gujarati.ldf doesn't exist in polyglossia, but it will use hyphenation patterns
\setdefaultlanguage{gujarati}
\setotherlanguage{english}

% Configure Gujarati font properly
% Use Language=Default to prevent polyglossia from trying to add language-specific features
% that don't exist for Gujarati, which causes "empty feature" warnings
\newfontfamily\gujaratifont[Script=Gujarati,AutoFakeBold=2.5,AutoFakeSlant=0.3]{Noto Sans Gujarati}
\setmainfont[Script=Gujarati,AutoFakeBold=2.5,AutoFakeSlant=0.3]{Noto Sans Gujarati}
% Use Noto Sans Gujarati for monospace to support Gujarati in text
\setmonofont[Scale=0.9]{Noto Sans Gujarati}

% Configure English to use the same font
\newfontfamily\englishfont[Script=Gujarati,AutoFakeBold=2.5,AutoFakeSlant=0.3]{Noto Sans Gujarati}

% Translations for polyglossia
\gappto\captionsgujarati{
  \renewcommand{\tablename}{કોષ્ટક}
  \renewcommand{\figurename}{આકૃતિ}
}

% Helper for TikZ nodes to ensure Gujarati font
\newcommand{\gu}[1]{{\gujaratifont #1}}

% Custom environments
\newtcolorbox{solutionbox}{
    breakable,
    enhanced,
    colback=solutioncolor!5!white,
    colframe=solutioncolor!75!black,
    fonttitle=\bfseries,
    title=જવાબ
}

\newtcolorbox{solutionboxnobreak}{
 colback=solutioncolor!5!white,
 colframe=solutioncolor!75!black,
 fonttitle=\bfseries,
 title=જવાબ
}

\newtcolorbox{keyformula}{
 breakable,
 enhanced,
 colback=keycolor!5!white,
 colframe=keycolor!75!black,
 fonttitle=\bfseries,
 title=રાસાયણિક સમીકરણ/સૂત્ર
}

\newtcolorbox{mnemonicbox}{
 breakable,
 enhanced,
 colback=mnemoniccolor!5!white,
 colframe=mnemoniccolor!75!black,
 fonttitle=\bfseries,
 title=મેમરી ટ્રીક
}


\begin{document}

\begin{center}
{\Huge\bfseries\color{headcolor} Subject Name (Gujarati)}\\[5pt]
{\LARGE 4351602 -- Winter 2024}\\[3pt]
{\large Semester 1 Study Material}\\[3pt]
{\normalsize\textit{Detailed Solutions and Explanations}}
\end{center}

\vspace{10pt}

\subsection*{પ્રશ્ન 1(અ) [3
ગુણ]}\label{uxaaauxab0uxab6uxaa8-1uxa85-3-uxa97uxaa3}

\textbf{congestion control ના પ્રકારો જણાવો અને કોઈપણ એક સમજાવો}

\begin{solutionbox}

{\def\LTcaptype{none} % do not increment counter
\begin{longtable}[]{@{}ll@{}}
\toprule\noalign{}
પ્રકાર & વર્ણન \\
\midrule\noalign{}
\endhead
\bottomrule\noalign{}
\endlastfoot
\textbf{Open-Loop} & congestion થાય તે પહેલાં અટકાવે \\
\textbf{Closed-Loop} & congestion detect થયા પછી વ્યવસ્થાપન \\
\end{longtable}
}

\textbf{Open-Loop Congestion Control સમજાવટ:}

\begin{itemize}
\tightlist
\item
  \textbf{અટકાવવાનો અભિગમ}: congestion થાય તે પહેલાં action લે
\item
  \textbf{Traffic shaping}: sender પર data rate control કરે
\item
  \textbf{Admission control}: વધુ traffic દરમિયાન નવા connections limit
  કરે
\item
  \textbf{Load shedding}: buffer full થાય ત્યારે packets drop કરે
\end{itemize}

\end{solutionbox}
\begin{mnemonicbox}
``Open Prevents Traffic Admission Load''

\end{mnemonicbox}
\begin{center}\rule{0.5\linewidth}{0.5pt}\end{center}

\subsection*{પ્રશ્ન 1(બ) [4
ગુણ]}\label{uxaaauxab0uxab6uxaa8-1uxaac-4-uxa97uxaa3}

\textbf{Address Resolution Protocol વિસ્તારપૂર્વક સમજાવો}

\begin{solutionbox}

\textbf{ARP (Address Resolution Protocol)} local networks માં IP
addresses ને MAC addresses સાથે map કરે છે.

\textbf{કાર્ય પ્રક્રિયા:}

\begin{itemize}
\tightlist
\item
  \textbf{ARP Request}: ``કોની પાસે IP X છે?'' broadcast message
\item
  \textbf{ARP Reply}: target device પોતાનું MAC address આપે
\item
  \textbf{ARP Cache}: ભવિષ્ય માટે IP-MAC mappings store કરે
\item
  \textbf{Dynamic mapping}: entries automatically update કરે
\end{itemize}


{\def\LTcaptype{none} % do not increment counter
\vspace{-5pt}
\captionof{table}{ARP Message Types}
\vspace{-10pt}
\begin{longtable}[]{@{}lll@{}}
\toprule\noalign{}
Type & Purpose & Broadcast \\
\midrule\noalign{}
\endhead
\bottomrule\noalign{}
\endlastfoot
ARP Request & MAC address શોધવા માટે & Yes \\
ARP Reply & MAC address આપવા માટે & No \\
\end{longtable}
}

\end{solutionbox}
\begin{mnemonicbox}
``ARP Requests Broadcast, Replies Cache Dynamic''

\end{mnemonicbox}
\begin{center}\rule{0.5\linewidth}{0.5pt}\end{center}

\subsection*{પ્રશ્ન 1(ક) [7
ગુણ]}\label{uxaaauxab0uxab6uxaa8-1uxa95-7-uxa97uxaa3}

\textbf{TCP/IP મોડેલના દરેક layers ને તેમની કાર્યક્ષમતા સાથે સમજાવો}

\begin{solutionbox}

\textbf{TCP/IP Model} internet communication માટે four-layer network
protocol stack છે.

\begin{center}
\textbf{Mermaid Diagram (Code)}
\begin{verbatim}
{Shaded}
{Highlighting}[]
graph LR
    A[Application Layer] {-{-}{} B[Transport Layer]}
    B {-{-}{} C[Internet Layer] }
    C {-{-}{} D[Network Access Layer]}
{Highlighting}
{Shaded}
\end{verbatim}
\end{center}

\textbf{Layer Functions:}

{\def\LTcaptype{none} % do not increment counter
\begin{longtable}[]{@{}lll@{}}
\toprule\noalign{}
Layer & Function & Protocols \\
\midrule\noalign{}
\endhead
\bottomrule\noalign{}
\endlastfoot
\textbf{Application} & User interface, network services & HTTP, FTP,
SMTP \\
\textbf{Transport} & End-to-end communication & TCP, UDP \\
\textbf{Internet} & Routing, addressing & IP, ICMP \\
\textbf{Network Access} & Physical transmission & Ethernet, WiFi \\
\end{longtable}
}

\begin{itemize}
\tightlist
\item
  \textbf{Application Layer}: applications ને network services provide કરે
\item
  \textbf{Transport Layer}: error control સાથે reliable data delivery
  ensure કરે
\item
  \textbf{Internet Layer}: IP addressing વાપરીને networks વચ્ચે packets
  route કરે
\item
  \textbf{Network Access Layer}: physical data transmission handle કરે
\end{itemize}

\end{solutionbox}
\begin{mnemonicbox}
``All Transport Internet Network''

\end{mnemonicbox}
\begin{center}\rule{0.5\linewidth}{0.5pt}\end{center}

\subsection*{પ્રશ્ન 1(ક OR) [7
ગુણ]}\label{uxaaauxab0uxab6uxaa8-1uxa95-or-7-uxa97uxaa3}

\textbf{OSI model ને તેના દરેક લેયરની કાર્યક્ષમતા સાથે સમજાવો}

\begin{solutionbox}

\textbf{OSI Model} network communication માટે seven-layer reference model
છે.

\begin{center}
\textbf{Mermaid Diagram (Code)}
\begin{verbatim}
{Shaded}
{Highlighting}[]
graph LR
    A[Application Layer 7] {-{-}{} B[Presentation Layer 6]}
    B {-{-}{} C[Session Layer 5]}
    C {-{-}{} D[Transport Layer 4]}
    D {-{-}{} E[Network Layer 3]}
    E {-{-}{} F[Data Link Layer 2]}
    F {-{-}{} G[Physical Layer 1]}
{Highlighting}
{Shaded}
\end{verbatim}
\end{center}

\textbf{Layer Functionalities:}

{\def\LTcaptype{none} % do not increment counter
\begin{longtable}[]{@{}lll@{}}
\toprule\noalign{}
Layer & Function & Examples \\
\midrule\noalign{}
\endhead
\bottomrule\noalign{}
\endlastfoot
\textbf{Physical (1)} & Bit transmission & Cables, signals \\
\textbf{Data Link (2)} & Frame delivery & Ethernet, switches \\
\textbf{Network (3)} & Routing packets & IP, routers \\
\textbf{Transport (4)} & End-to-end delivery & TCP, UDP \\
\textbf{Session (5)} & Dialog management & NetBIOS \\
\textbf{Presentation (6)} & Data formatting & SSL, compression \\
\textbf{Application (7)} & User interface & HTTP, email \\
\end{longtable}
}

\end{solutionbox}
\begin{mnemonicbox}
``Physical Data Network Transport Session
Presentation Application''

\end{mnemonicbox}
\begin{center}\rule{0.5\linewidth}{0.5pt}\end{center}

\subsection*{પ્રશ્ન 2(અ) [3
ગુણ]}\label{uxaaauxab0uxab6uxaa8-2uxa85-3-uxa97uxaa3}

\textbf{Subnetting ને ટૂંકમાં સમજાવો}

\begin{solutionbox}

\textbf{Subnetting} મોટા network ને better management માટે નાના
sub-networks માં વહેંચે છે.

\textbf{મુખ્ય સિદ્ધાંતો:}

\begin{itemize}
\tightlist
\item
  \textbf{Subnet mask}: network અને host portions define કરે
\item
  \textbf{Network efficiency}: broadcast traffic ઘટાડે
\item
  \textbf{Address conservation}: વધુ સારો IP utilization
\item
  \textbf{Security}: network segments ને isolate કરે
\end{itemize}

\textbf{Example:} Network: 192.168.1.0/24 \rightarrow Subnets: 192.168.1.0/26,
192.168.1.64/26

\end{solutionbox}
\begin{mnemonicbox}
``Subnet Network Efficiency Address Security''

\end{mnemonicbox}
\begin{center}\rule{0.5\linewidth}{0.5pt}\end{center}

\subsection*{પ્રશ્ન 2(બ) [4
ગુણ]}\label{uxaaauxab0uxab6uxaa8-2uxaac-4-uxa97uxaa3}

\textbf{ડેટા લીક લેયરના Stop and wait ARQ પ્રોટોકોલને ઉદાહરણ આપી સમજાવો}

\begin{solutionbox}

\textbf{Stop and Wait ARQ} reliable data transmission ensure કરવા માટેનો
flow control protocol છે.

\textbf{કાર્ય પ્રક્રિયા:}

\begin{itemize}
\tightlist
\item
  \textbf{Send frame}: Transmitter એક frame મોકલે
\item
  \textbf{Wait for ACK}: Sender acknowledgment માટે રાહ જુએ
\item
  \textbf{Timeout}: ACK ન મળે તો retransmit કરે
\item
  \textbf{Next frame}: ACK મળ્યા પછી next frame મોકલે
\end{itemize}

\begin{verbatim}
Sender          Receiver
  |     Frame 1   |
  |{-{-}{-}{-}{-}{-}{-}{-}{-}{-}{-}{-}{-}{-}|}
  |               |
  |      ACK      |
  |{{-}{-}{-}{-}{-}{-}{-}{-}{-}{-}{-}{-}{-}{-}|}
  |     Frame 2   |
  |{-{-}{-}{-}{-}{-}{-}{-}{-}{-}{-}{-}{-}{-}|}
\end{verbatim}

\textbf{Example:} File transfer માં દરેક packet confirmation માટે રાહ જુએ
before sending next.

\end{solutionbox}
\begin{mnemonicbox}
``Send Wait Timeout Next''

\end{mnemonicbox}
\begin{center}\rule{0.5\linewidth}{0.5pt}\end{center}

\subsection*{પ્રશ્ન 2(ક) [7
ગુણ]}\label{uxaaauxab0uxab6uxaa8-2uxa95-7-uxa97uxaa3}

\textbf{IPv4 datagram Header ની આકૃતિ દોરો અને સમજાવો}

\begin{solutionbox}

\textbf{IPv4 Header} packet routing અને delivery માટે control information
ધરાવે છે.

\begin{verbatim}
 0                   1                   2                   3
 0 1 2 3 4 5 6 7 8 9 0 1 2 3 4 5 6 7 8 9 0 1 2 3 4 5 6 7 8 9 0 1
+{-+{-}+{-}+{-}+{-}+{-}+{-}+{-}+{-}+{-}+{-}+{-}+{-}+{-}+{-}+{-}+{-}+{-}+{-}+{-}+{-}+{-}+{-}+{-}+{-}+{-}+{-}+{-}+{-}+{-}+{-}+{-}+}
|Version|  IHL  |Type of Service|          Total Length         |
+{-+{-}+{-}+{-}+{-}+{-}+{-}+{-}+{-}+{-}+{-}+{-}+{-}+{-}+{-}+{-}+{-}+{-}+{-}+{-}+{-}+{-}+{-}+{-}+{-}+{-}+{-}+{-}+{-}+{-}+{-}+{-}+}
|         Identification        |Flags|      Fragment Offset    |
+{-+{-}+{-}+{-}+{-}+{-}+{-}+{-}+{-}+{-}+{-}+{-}+{-}+{-}+{-}+{-}+{-}+{-}+{-}+{-}+{-}+{-}+{-}+{-}+{-}+{-}+{-}+{-}+{-}+{-}+{-}+{-}+}
|  Time to Live |    Protocol   |         Header Checksum       |
+{-+{-}+{-}+{-}+{-}+{-}+{-}+{-}+{-}+{-}+{-}+{-}+{-}+{-}+{-}+{-}+{-}+{-}+{-}+{-}+{-}+{-}+{-}+{-}+{-}+{-}+{-}+{-}+{-}+{-}+{-}+{-}+}
|                       Source Address                          |
+{-+{-}+{-}+{-}+{-}+{-}+{-}+{-}+{-}+{-}+{-}+{-}+{-}+{-}+{-}+{-}+{-}+{-}+{-}+{-}+{-}+{-}+{-}+{-}+{-}+{-}+{-}+{-}+{-}+{-}+{-}+{-}+}
|                    Destination Address                        |
+{-+{-}+{-}+{-}+{-}+{-}+{-}+{-}+{-}+{-}+{-}+{-}+{-}+{-}+{-}+{-}+{-}+{-}+{-}+{-}+{-}+{-}+{-}+{-}+{-}+{-}+{-}+{-}+{-}+{-}+{-}+{-}+}
\end{verbatim}

\textbf{Field સમજાવટ:}

{\def\LTcaptype{none} % do not increment counter
\begin{longtable}[]{@{}lll@{}}
\toprule\noalign{}
Field & Size & Function \\
\midrule\noalign{}
\endhead
\bottomrule\noalign{}
\endlastfoot
\textbf{Version} & 4 bits & IP version (IPv4 માટે 4) \\
\textbf{IHL} & 4 bits & Header length \\
\textbf{Type of Service} & 8 bits & Quality of service \\
\textbf{Total Length} & 16 bits & Packet size \\
\textbf{TTL} & 8 bits & Hop limit \\
\textbf{Protocol} & 8 bits & Next layer protocol \\
\textbf{Source/Dest Address} & 32 bits દરેક & IP addresses \\
\end{longtable}
}

\end{solutionbox}
\begin{mnemonicbox}
``Version IHL Service Total TTL Protocol Source
Destination''

\end{mnemonicbox}
\begin{center}\rule{0.5\linewidth}{0.5pt}\end{center}

\subsection*{પ્રશ્ન 2(અ OR) [3
ગુણ]}\label{uxaaauxab0uxab6uxaa8-2uxa85-or-3-uxa97uxaa3}

\textbf{HTTPS શું છે? HTTPSની મહત્વની વિશેષતાઓની યાદી લખો}

\begin{solutionbox}

\textbf{HTTPS (HTTP Secure)} secure web communication માટે SSL/TLS વાપરીને
encrypted HTTP છે.

\textbf{મુખ્ય વિશેષતાઓ:}

\begin{itemize}
\tightlist
\item
  \textbf{Encryption}: Data transit માં encrypted રહે
\item
  \textbf{Authentication}: Server identity verify કરે
\item
  \textbf{Data integrity}: Data tampering અટકાવે
\item
  \textbf{Trust}: SSL certificates validation provide કરે
\end{itemize}

\textbf{Security Benefits:}

\begin{itemize}
\tightlist
\item
  Sensitive information protect કરે
\item
  Man-in-the-middle attacks prevent કરે
\item
  Search engine ranking boost આપે
\end{itemize}

\end{solutionbox}
\begin{mnemonicbox}
``HTTPS Encrypts Authentication Data Trust''

\end{mnemonicbox}
\begin{center}\rule{0.5\linewidth}{0.5pt}\end{center}

\subsection*{પ્રશ્ન 2(બ OR) [4
ગુણ]}\label{uxaaauxab0uxab6uxaa8-2uxaac-or-4-uxa97uxaa3}

\textbf{કોઈપણ બેના જવાબ આપો:}

\begin{solutionbox}

\textbf{1) Class B અને C દ્વારા કેટલા bits HOST ID નો ઉપયોગ થાય છે?}

\begin{itemize}
\tightlist
\item
  \textbf{Class B}: HOST ID માટે 16 bits (65,534 hosts)
\item
  \textbf{Class C}: HOST ID માટે 8 bits (254 hosts)
\end{itemize}

\textbf{2) Class A અને D ની IP રેન્જ કેટલી છે?}

\begin{itemize}
\tightlist
\item
  \textbf{Class A}: 1.0.0.0 to 126.255.255.255
\item
  \textbf{Class D}: 224.0.0.0 to 239.255.255.255 (Multicast)
\end{itemize}

{\def\LTcaptype{none} % do not increment counter
\begin{longtable}[]{@{}lll@{}}
\toprule\noalign{}
Class & Range & Host Bits \\
\midrule\noalign{}
\endhead
\bottomrule\noalign{}
\endlastfoot
B & 128.0.0.0 - 191.255.255.255 & 16 bits \\
C & 192.0.0.0 - 223.255.255.255 & 8 bits \\
A & 1.0.0.0 - 126.255.255.255 & 24 bits \\
D & 224.0.0.0 - 239.255.255.255 & Multicast \\
\end{longtable}
}

\end{solutionbox}
\begin{mnemonicbox}
``B=16,

C=8,

A=1-126,

D=224-239''


\end{mnemonicbox}
\begin{center}\rule{0.5\linewidth}{0.5pt}\end{center}

\subsection*{પ્રશ્ન 2(ક OR) [7
ગુણ]}\label{uxaaauxab0uxab6uxaa8-2uxa95-or-7-uxa97uxaa3}

\textbf{Classful IPv4 એડ્રેસ સ્કીમ સમજાવો}

\begin{solutionbox}

\textbf{Classful IPv4 Addressing} first octets આધારે IP address space ને
પાંચ classes માં વહેંચે છે.

\textbf{Address Classes:}

{\def\LTcaptype{none} % do not increment counter
\begin{longtable}[]{@{}lllll@{}}
\toprule\noalign{}
Class & Range & Network Bits & Host Bits & Usage \\
\midrule\noalign{}
\endhead
\bottomrule\noalign{}
\endlastfoot
\textbf{A} & 1-126 & 8 & 24 & Large networks \\
\textbf{B} & 128-191 & 16 & 16 & Medium networks \\
\textbf{C} & 192-223 & 24 & 8 & Small networks \\
\textbf{D} & 224-239 & - & - & Multicast \\
\textbf{E} & 240-255 & - & - & Experimental \\
\end{longtable}
}

\begin{verbatim}
pie title IPv4 Address Classes
    "Class A (50\%)" : 50
    "Class B (25\%)" : 25
    "Class C (12.5\%)" : 12.5
    "Class D (6.25\%)" : 6.25
    "Class E (6.25\%)" : 6.25
\end{verbatim}

\textbf{લાક્ષણિકતાઓ:}

\begin{itemize}
\tightlist
\item
  \textbf{Class A}: network દીઠ 16.7 million hosts
\item
  \textbf{Class B}: network દીઠ 65,534 hosts
\item
  \textbf{Class C}: network દીઠ 254 hosts
\item
  \textbf{મર્યાદાઓ}: Address wastage, inflexible allocation
\end{itemize}

\end{solutionbox}
\begin{mnemonicbox}
``A-Large, B-Medium, C-Small, D-Multicast,
E-Experimental''

\end{mnemonicbox}
\begin{center}\rule{0.5\linewidth}{0.5pt}\end{center}

\subsection*{પ્રશ્ન 3(અ) [3
ગુણ]}\label{uxaaauxab0uxab6uxaa8-3uxa85-3-uxa97uxaa3}

\textbf{મોબાઇલ કમ્પ્યુટિંગનો ઉપયોગ કરતી એપ્લિકેશનોના પ્રકારોની યાદી બનાવો}

\begin{solutionbox}

\textbf{Mobile Computing Applications:}

{\def\LTcaptype{none} % do not increment counter
\begin{longtable}[]{@{}ll@{}}
\toprule\noalign{}
પ્રકાર & Examples \\
\midrule\noalign{}
\endhead
\bottomrule\noalign{}
\endlastfoot
\textbf{Communication} & WhatsApp, Email, Video calls \\
\textbf{Navigation} & GPS, Google Maps \\
\textbf{E-commerce} & Shopping apps, Mobile banking \\
\textbf{Entertainment} & Games, Streaming, Social media \\
\textbf{Business} & CRM, Sales tracking \\
\textbf{Healthcare} & Health monitoring, Telemedicine \\
\end{longtable}
}

\begin{itemize}
\tightlist
\item
  \textbf{Location-based services}: GPS navigation, location sharing
\item
  \textbf{Mobile payments}: Digital wallets, UPI transactions
\item
  \textbf{Social networking}: Facebook, Instagram, Twitter
\end{itemize}

\end{solutionbox}
\begin{mnemonicbox}
``Communication Navigation E-commerce Entertainment
Business Healthcare''

\end{mnemonicbox}
\begin{center}\rule{0.5\linewidth}{0.5pt}\end{center}

\subsection*{પ્રશ્ન 3(બ) [4
ગુણ]}\label{uxaaauxab0uxab6uxaa8-3uxaac-4-uxa97uxaa3}

\textbf{Gateways નો ઉપયોગ સમજાવો અને Gateways ના પ્રકારોની યાદી આપો}

\begin{solutionbox}

\textbf{Gateway} અલગ અલગ protocols અને architectures વાળા networks ને
connect કરે છે.

\textbf{Gateways ના ઉપયોગ:}

\begin{itemize}
\tightlist
\item
  \textbf{Protocol conversion}: વિવિધ protocols વચ્ચે translate કરે
\item
  \textbf{Network bridging}: અસમાન networks ને connect કરે
\item
  \textbf{Security}: Firewall અને access control
\item
  \textbf{Data filtering}: Traffic flow manage કરે
\end{itemize}

\textbf{Gateways ના પ્રકારો:}

{\def\LTcaptype{none} % do not increment counter
\begin{longtable}[]{@{}ll@{}}
\toprule\noalign{}
Type & Function \\
\midrule\noalign{}
\endhead
\bottomrule\noalign{}
\endlastfoot
\textbf{Network Gateway} & Networks વચ્ચે route કરે \\
\textbf{Internet Gateway} & Internet સાથે connect કરે \\
\textbf{Protocol Gateway} & Protocol translation \\
\textbf{Application Gateway} & Application-level filtering \\
\end{longtable}
}

\end{solutionbox}
\begin{mnemonicbox}
``Gateways Convert Bridge Secure Filter''

\end{mnemonicbox}
\begin{center}\rule{0.5\linewidth}{0.5pt}\end{center}

\subsection*{પ્રશ્ન 3(ક) [7
ગુણ]}\label{uxaaauxab0uxab6uxaa8-3uxa95-7-uxa97uxaa3}

\textbf{Mobile Computing નું આર્કિટેક્ચર દોરો અને સમજાવો}

\begin{solutionbox}

\textbf{Mobile Computing Architecture} એકસાથે કામ કરતા ત્રણ મુખ્ય components
ધરાવે છે.

\begin{center}
\textbf{Mermaid Diagram (Code)}
\begin{verbatim}
{Shaded}
{Highlighting}[]
graph TD
    A[Mobile Device] {{-}{-}{} B[Communication Network]}
    B {{-}{-}{} C[Fixed Infrastructure]}
    
    A1[Hardware] {-{-}{} A}
    A2[OS \& Apps] {-{-}{} A}
    A3[Data] {-{-}{} A}
    
    B1[Wireless Network] {-{-}{} B}
    B2[Protocols] {-{-}{} B}
    B3[Base Stations] {-{-}{} B}
    
    C1[Servers] {-{-}{} C}
    C2[Databases] {-{-}{} C}
    C3[Internet] {-{-}{} C}
{Highlighting}
{Shaded}
\end{verbatim}
\end{center}

\textbf{Architecture Components:}

{\def\LTcaptype{none} % do not increment counter
\begin{longtable}[]{@{}
  >{\raggedright\arraybackslash}p{(\linewidth - 4\tabcolsep) * \real{0.3548}}
  >{\raggedright\arraybackslash}p{(\linewidth - 4\tabcolsep) * \real{0.3226}}
  >{\raggedright\arraybackslash}p{(\linewidth - 4\tabcolsep) * \real{0.3226}}@{}}
\toprule\noalign{}
\begin{minipage}[b]{\linewidth}\raggedright
Component
\end{minipage} & \begin{minipage}[b]{\linewidth}\raggedright
Elements
\end{minipage} & \begin{minipage}[b]{\linewidth}\raggedright
Function
\end{minipage} \\
\midrule\noalign{}
\endhead
\bottomrule\noalign{}
\endlastfoot
\textbf{Mobile Unit} & Devices, OS, Apps & User interface, processing \\
\textbf{Communication Network} & Wireless links, protocols & Data
transmission \\
\textbf{Fixed Infrastructure} & Servers, databases & Backend services \\
\end{longtable}
}

\textbf{મુખ્ય લાક્ષણિકતાઓ:}

\begin{itemize}
\tightlist
\item
  \textbf{Mobility}: Users connectivity maintain કરીને move કરી શકે
\item
  \textbf{Wireless communication}: Data transmission માટે radio waves
\item
  \textbf{Distributed computing}: Multiple devices પર processing
\item
  \textbf{Location independence}: કોઈપણ જગ્યાએથી services access
\end{itemize}

\textbf{પડકારો:}

\begin{itemize}
\tightlist
\item
  \textbf{Limited bandwidth}: Wireless networks માં capacity constraints
\item
  \textbf{Battery life}: Mobile devices માં power limitations
\item
  \textbf{Security}: Wireless transmission attacks માટે vulnerable
\end{itemize}

\end{solutionbox}
\begin{mnemonicbox}
``Mobile Communication Fixed - Mobility Wireless
Distributed Location''

\end{mnemonicbox}
\begin{center}\rule{0.5\linewidth}{0.5pt}\end{center}

\subsection*{પ્રશ્ન 3(અ OR) [3
ગુણ]}\label{uxaaauxab0uxab6uxaa8-3uxa85-or-3-uxa97uxaa3}

\textbf{મોબાઇલ કમ્પ્યુટિંગમાં સુરક્ષા ધોરણોની યાદી બનાવો}

\begin{solutionbox}

\textbf{Mobile Computing Security Standards:}

{\def\LTcaptype{none} % do not increment counter
\begin{longtable}[]{@{}ll@{}}
\toprule\noalign{}
Standard & Purpose \\
\midrule\noalign{}
\endhead
\bottomrule\noalign{}
\endlastfoot
\textbf{WPA3} & WiFi security protocol \\
\textbf{SSL/TLS} & Secure data transmission \\
\textbf{IPSec} & IP layer security \\
\textbf{EAP} & Authentication framework \\
\textbf{802.11i} & Wireless LAN security \\
\textbf{FIPS 140-2} & Cryptographic module standards \\
\end{longtable}
}

\begin{itemize}
\tightlist
\item
  \textbf{Authentication protocols}: User identity verify કરે
\item
  \textbf{Encryption standards}: Data confidentiality protect કરે
\item
  \textbf{Access control}: Resource permissions manage કરે
\end{itemize}

\end{solutionbox}
\begin{mnemonicbox}
``WPA SSL IPSec EAP 802.11i FIPS''

\end{mnemonicbox}
\begin{center}\rule{0.5\linewidth}{0.5pt}\end{center}

\subsection*{પ્રશ્ન 3(બ OR) [4
ગુણ]}\label{uxaaauxab0uxab6uxaa8-3uxaac-or-4-uxa97uxaa3}

\textbf{કોમ્યુનિકેશન Gateway ના મુખ્ય કાર્યો સમજાવો}

\begin{solutionbox}

\textbf{Communication Gateway} વિવિધ network systems વચ્ચે data exchange
manage કરે છે.

\textbf{મુખ્ય કાર્યો:}

{\def\LTcaptype{none} % do not increment counter
\begin{longtable}[]{@{}ll@{}}
\toprule\noalign{}
Function & Description \\
\midrule\noalign{}
\endhead
\bottomrule\noalign{}
\endlastfoot
\textbf{Protocol Translation} & Protocols વચ્ચે convert કરે \\
\textbf{Data Format Conversion} & Data formats change કરે \\
\textbf{Routing} & Messages ને destinations પર direct કરે \\
\textbf{Security} & Access control અને filtering \\
\end{longtable}
}

\textbf{વિગતવાર કાર્યો:}

\begin{itemize}
\tightlist
\item
  \textbf{Message routing}: Data માટે optimal path determine કરે
\item
  \textbf{Error handling}: Transmission errors અને recovery manage કરે
\item
  \textbf{Traffic management}: Data flow અને congestion control કરે
\item
  \textbf{Authentication}: Sender અને receiver identity verify કરે
\end{itemize}

\textbf{ફાયદાઓ:}

\begin{itemize}
\tightlist
\item
  વિવિધ systems વચ્ચે interoperability enable કરે
\item
  Network management centralize કરે
\item
  Security checkpoint provide કરે
\end{itemize}

\end{solutionbox}
\begin{mnemonicbox}
``Protocol Data Routing Security - Message Error
Traffic Authentication''

\end{mnemonicbox}
\begin{center}\rule{0.5\linewidth}{0.5pt}\end{center}

\subsection*{પ્રશ્ન 3(ક OR) [7
ગુણ]}\label{uxaaauxab0uxab6uxaa8-3uxa95-or-7-uxa97uxaa3}

\textbf{મિડલવેરનો ઉપયોગ અને મિડલવેરના લિસ્ટ પ્રકારો સમજાવો}

\begin{solutionbox}

\textbf{Middleware} distributed computing માટે applications અને operating
system વચ્ચે software layer provide કરે છે.

\textbf{Middleware ના ઉપયોગ:}

\begin{itemize}
\tightlist
\item
  \textbf{Connectivity}: Distributed applications ને link કરે
\item
  \textbf{Interoperability}: વિવિધ systems ને એકસાથે કામ કરવા enable કરે
\item
  \textbf{Abstraction}: Underlying systems ની complexity hide કરે
\item
  \textbf{Scalability}: System growth અને expansion support કરે
\end{itemize}

\begin{center}
\textbf{Mermaid Diagram (Code)}
\begin{verbatim}
{Shaded}
{Highlighting}[]
graph LR
    A[Applications] {-{-}{} B[Middleware Layer]}
    B {-{-}{} C[Operating System]}
    B {-{-}{} D[Network Services]}
    B {-{-}{} E[Database Services]}
{Highlighting}
{Shaded}
\end{verbatim}
\end{center}

\textbf{Middleware ના પ્રકારો:}

{\def\LTcaptype{none} % do not increment counter
\begin{longtable}[]{@{}
  >{\raggedright\arraybackslash}p{(\linewidth - 4\tabcolsep) * \real{0.2308}}
  >{\raggedright\arraybackslash}p{(\linewidth - 4\tabcolsep) * \real{0.3846}}
  >{\raggedright\arraybackslash}p{(\linewidth - 4\tabcolsep) * \real{0.3846}}@{}}
\toprule\noalign{}
\begin{minipage}[b]{\linewidth}\raggedright
Type
\end{minipage} & \begin{minipage}[b]{\linewidth}\raggedright
Function
\end{minipage} & \begin{minipage}[b]{\linewidth}\raggedright
Examples
\end{minipage} \\
\midrule\noalign{}
\endhead
\bottomrule\noalign{}
\endlastfoot
\textbf{Message-Oriented} & Asynchronous communication & IBM MQ,
RabbitMQ \\
\textbf{Remote Procedure Call} & Synchronous communication & gRPC,
XML-RPC \\
\textbf{Object Request Broker} & Object communication & CORBA \\
\textbf{Database Middleware} & Database connectivity & ODBC, JDBC \\
\textbf{Transaction Processing} & Transaction management & Tuxedo \\
\textbf{Web Middleware} & Web services & Apache, IIS \\
\end{longtable}
}

\textbf{ફાયદાઓ:}

\begin{itemize}
\tightlist
\item
  \textbf{Reduced complexity}: Application development simplify કરે
\item
  \textbf{Reusability}: Multiple applications માટે common services
\item
  \textbf{Maintainability}: Services ના centralized management
\item
  \textbf{Platform independence}: વિવિધ systems પર કામ કરે
\end{itemize}

\textbf{વિગતવાર સમજાવટ:}

\textbf{Message-Oriented Middleware:}

\begin{itemize}
\tightlist
\item
  Asynchronous communication enable કરે
\item
  Message queues દ્વારા data exchange
\item
  Reliability અને fault tolerance provide કરે
\end{itemize}

\textbf{RPC Middleware:}

\begin{itemize}
\tightlist
\item
  Remote functions ને local calls જેવા લાગે
\item
  Synchronous communication support કરે
\item
  Network transparency provide કરે
\end{itemize}

\textbf{Database Middleware:}

\begin{itemize}
\tightlist
\item
  Multiple databases સાથે connectivity
\item
  Data access layer abstraction
\item
  Query optimization અને caching
\end{itemize}

\textbf{Transaction Processing Middleware:}

\begin{itemize}
\tightlist
\item
  ACID properties ensure કરે
\item
  Distributed transactions manage કરે
\item
  Concurrency control provide કરે
\end{itemize}

\textbf{Web Middleware:}

\begin{itemize}
\tightlist
\item
  HTTP requests handle કરે
\item
  Load balancing અને caching
\item
  Security features provide કરે
\end{itemize}

\textbf{Challenges:}

\begin{itemize}
\tightlist
\item
  \textbf{Performance overhead}: Additional layer adds latency
\item
  \textbf{Complexity}: System architecture વધુ complex બને
\item
  \textbf{Vendor dependency}: Specific middleware vendors પર dependency
\end{itemize}

\textbf{Applications:}

\begin{itemize}
\tightlist
\item
  \textbf{Enterprise systems}: Large-scale business applications
\item
  \textbf{E-commerce}: Online shopping platforms
\item
  \textbf{Banking systems}: Financial transaction processing
\item
  \textbf{Telecommunication}: Network service management
\end{itemize}

\end{solutionbox}
\begin{mnemonicbox}
``Message RPC Object Database Transaction Web -
Connectivity Interoperability Abstraction Scalability''

\end{mnemonicbox}
\subsection*{પ્રશ્ન 4(અ) [3
ગુણ]}\label{uxaaauxab0uxab6uxaa8-4uxa85-3-uxa97uxaa3}

\textbf{મોબાઇલ IP ના કાર્યકારી તબક્કાઓ સમજાવો}

\begin{solutionbox}

\textbf{Mobile IP Working Phases} networks પર seamless mobility enable
કરે છે.

\textbf{ત્રણ મુખ્ય તબક્કાઓ:}

{\def\LTcaptype{none} % do not increment counter
\begin{longtable}[]{@{}ll@{}}
\toprule\noalign{}
Phase & Function \\
\midrule\noalign{}
\endhead
\bottomrule\noalign{}
\endlastfoot
\textbf{Agent Discovery} & Home/foreign agents શોધવા \\
\textbf{Registration} & Foreign agent સાથે register \\
\textbf{Tunneling} & Mobile node પર packets forward \\
\end{longtable}
}

\textbf{Phase વિગતો:}

\begin{itemize}
\tightlist
\item
  \textbf{Agent Discovery}: Mobile node advertisements દ્વારા available
  agents detect કરે
\item
  \textbf{Registration}: Mobile node current location home agent સાથે
  register કરે
\item
  \textbf{Tunneling}: Home agent packets encapsulate કરીને foreign agent
  પર forward કરે
\end{itemize}

\end{solutionbox}
\begin{mnemonicbox}
``Agent Registration Tunneling''

\end{mnemonicbox}
\begin{center}\rule{0.5\linewidth}{0.5pt}\end{center}

\subsection*{પ્રશ્ન 4(બ) [4
ગુણ]}\label{uxaaauxab0uxab6uxaa8-4uxaac-4-uxa97uxaa3}

\textbf{Mobile IP માટે હેન્ડઓવર મેનેજમેન્ટ સમજાવો}

\begin{solutionbox}

\textbf{Handover Management} mobile node networks વચ્ચે move કરે ત્યારે
connectivity maintain કરે છે.

\textbf{Handover Process:}

\begin{itemize}
\tightlist
\item
  \textbf{Movement detection}: Network attachment માં ફેરફાર identify કરે
\item
  \textbf{New agent discovery}: નવા foreign agent શોધે
\item
  \textbf{Registration update}: Home agent સાથે location update કરે
\item
  \textbf{Data forwarding}: Traffic ને નવા location પર redirect કરે
\end{itemize}

\textbf{Handover ના પ્રકારો:}

{\def\LTcaptype{none} % do not increment counter
\begin{longtable}[]{@{}ll@{}}
\toprule\noalign{}
Type & Description \\
\midrule\noalign{}
\endhead
\bottomrule\noalign{}
\endlastfoot
\textbf{Hard Handover} & Break-before-make \\
\textbf{Soft Handover} & Make-before-break \\
\textbf{Horizontal} & Same technology \\
\textbf{Vertical} & Different technology \\
\end{longtable}
}

\textbf{પડકારો:}

\begin{itemize}
\tightlist
\item
  \textbf{Packet loss}: Handover transition દરમિયાન
\item
  \textbf{Delay}: Registration અને tunneling setup time
\item
  \textbf{Resource management}: Network resources નો efficient use
\end{itemize}

\end{solutionbox}
\begin{mnemonicbox}
``Movement Discovery Registration Forwarding''

\end{mnemonicbox}
\begin{center}\rule{0.5\linewidth}{0.5pt}\end{center}

\subsection*{પ્રશ્ન 4(ક) [7
ગુણ]}\label{uxaaauxab0uxab6uxaa8-4uxa95-7-uxa97uxaa3}

\textbf{Mobile IP માં Registration અને Tunneling સમજાવો}

\begin{solutionbox}

\textbf{Registration અને Tunneling} Mobile IP functionality enable કરવાના
core mechanisms છે.

\textbf{Registration Process:}

\begin{verbatim}
sequenceDiagram
    participant MN as Mobile Node
    participant FA as Foreign Agent
    participant HA as Home Agent
    
    MN{-FA: Registration Request}
    FA{-HA: Forward Request}
    HA{-FA: Registration Reply}
    FA{-MN: Forward Reply}
\end{verbatim}

\textbf{Registration Steps:}

\begin{itemize}
\tightlist
\item
  \textbf{Request}: Mobile node foreign agent ને registration request
  મોકલે
\item
  \textbf{Forward}: Foreign agent request ને home agent પર forward કરે
\item
  \textbf{Authentication}: Home agent mobile node identity verify કરે
\item
  \textbf{Reply}: Home agent registration confirm કરતો reply મોકલે
\end{itemize}

\textbf{Tunneling Mechanism:}

{\def\LTcaptype{none} % do not increment counter
\begin{longtable}[]{@{}ll@{}}
\toprule\noalign{}
Component & Function \\
\midrule\noalign{}
\endhead
\bottomrule\noalign{}
\endlastfoot
\textbf{Encapsulation} & Original packet ને wrap કરે \\
\textbf{Tunnel Endpoint} & Home અને foreign agents \\
\textbf{Decapsulation} & Destination પર packet unwrap કરે \\
\textbf{Routing} & Tunnel દ્વારા traffic direct કરે \\
\end{longtable}
}

\textbf{Tunneling Process:}

\begin{itemize}
\tightlist
\item
  \textbf{Packet arrival}: Mobile node માટે data home agent પર આવે
\item
  \textbf{Encapsulation}: Home agent packet ને foreign agent address સાથે
  wrap કરે
\item
  \textbf{Tunnel transmission}: Packet tunnel દ્વારા foreign agent પર જાય
\item
  \textbf{Decapsulation}: Foreign agent unwrap કરીને mobile node ને
  deliver કરે
\end{itemize}

\textbf{ફાયદાઓ:}

\begin{itemize}
\tightlist
\item
  \textbf{Transparency}: Applications ને mobility ની જાણ નથી
\item
  \textbf{Connectivity}: Movement દરમિયાન communication maintain કરે
\item
  \textbf{Scalability}: Multiple mobile nodes support કરે
\end{itemize}

\end{solutionbox}
\begin{mnemonicbox}
``Registration Request Forward Authentication -
Tunneling Encapsulation Transmission Decapsulation''

\end{mnemonicbox}
\begin{center}\rule{0.5\linewidth}{0.5pt}\end{center}

\subsection*{પ્રશ્ન 4(અ OR) [3
ગુણ]}\label{uxaaauxab0uxab6uxaa8-4uxa85-or-3-uxa97uxaa3}

\textbf{Snooping TCP સમજાવો}

\begin{solutionbox}

\textbf{Snooping TCP} wireless networks પર wireless link errors handle
કરીને TCP performance improve કરે છે.

\textbf{કાર્ય પ્રક્રિયા:}

\begin{itemize}
\tightlist
\item
  \textbf{Base station monitoring}: TCP packets observe કરે
\item
  \textbf{Local retransmission}: Wireless link errors locally handle કરે
\item
  \textbf{Cache management}: Transmitted packets ની copies store કરે
\item
  \textbf{Error recovery}: Sender involve કર્યા વિના lost packets
  retransmit કરે
\end{itemize}

\textbf{મુખ્ય લાક્ષણિકતાઓ:}

{\def\LTcaptype{none} % do not increment counter
\begin{longtable}[]{@{}ll@{}}
\toprule\noalign{}
Feature & Benefit \\
\midrule\noalign{}
\endhead
\bottomrule\noalign{}
\endlastfoot
\textbf{Transparent} & TCP endpoints માં કોઈ changes નથી \\
\textbf{Local recovery} & Faster error correction \\
\textbf{Reduced timeouts} & Unnecessary retransmissions prevent કરે \\
\end{longtable}
}

\end{solutionbox}
\begin{mnemonicbox}
``Snooping Monitors Local Cache Recovery''

\end{mnemonicbox}
\begin{center}\rule{0.5\linewidth}{0.5pt}\end{center}

\subsection*{પ્રશ્ન 4(બ OR) [4
ગુણ]}\label{uxaaauxab0uxab6uxaa8-4uxaac-or-4-uxa97uxaa3}

\textbf{Mobile IP મા પેકેટ ડિલિવરી સમજાવો}

\begin{solutionbox}

\textbf{Mobile IP માં Packet Delivery} location ને ધ્યાન આપ્યા વિના mobile
nodes પર data પહોંચાડે છે.

\textbf{Delivery Process:}

\begin{center}
\textbf{Mermaid Diagram (Code)}
\begin{verbatim}
{Shaded}
{Highlighting}[]
graph LR
    A[Correspondent Node] {-{-}{} B[Home Network]}
    B {-{-}{} C\{Mobile Node Location?\}}
    C {-{-}{}|Home| D[Direct Delivery]}
    C {-{-}{}|Away| E[Home Agent]}
    E {-{-}{} F[Tunnel to Foreign Agent]}
    F {-{-}{} G[Mobile Node]}
{Highlighting}
{Shaded}
\end{verbatim}
\end{center}

\textbf{Delivery Scenarios:}

{\def\LTcaptype{none} % do not increment counter
\begin{longtable}[]{@{}lll@{}}
\toprule\noalign{}
Scenario & Path & Method \\
\midrule\noalign{}
\endhead
\bottomrule\noalign{}
\endlastfoot
\textbf{At Home} & Direct & Normal IP routing \\
\textbf{Away} & Via HA/FA & Tunneling \\
\textbf{Roaming} & Triangle routing & Indirect path \\
\end{longtable}
}

\textbf{Packet Flow Steps:}

\begin{itemize}
\tightlist
\item
  \textbf{Address resolution}: Mobile node location determine કરે
\item
  \textbf{Route selection}: Direct અથવા tunneled delivery choose કરે
\item
  \textbf{Encapsulation}: Tunneling જરૂરી હોય તો packet wrap કરે
\item
  \textbf{Forwarding}: Appropriate destination પર send કરે
\item
  \textbf{Decapsulation}: Foreign agent પર packet unwrap કરે
\item
  \textbf{Final delivery}: Mobile node ને deliver કરે
\end{itemize}

\end{solutionbox}
\begin{mnemonicbox}
``Address Route Encapsulation Forward Decapsulation
Delivery''

\end{mnemonicbox}
\begin{center}\rule{0.5\linewidth}{0.5pt}\end{center}

\subsection*{પ્રશ્ન 4(ક OR) [7
ગુણ]}\label{uxaaauxab0uxab6uxaa8-4uxa95-or-7-uxa97uxaa3}

\textbf{DHCP કેવી રીતે કાર્ય કરે છે એ આકૃતિ દોરી સમજાવો}

\begin{solutionbox}

\textbf{DHCP (Dynamic Host Configuration Protocol)} devices ને
automatically IP addresses અને network configuration assign કરે છે.

\textbf{DHCP Working Process:}

\begin{verbatim}
sequenceDiagram
    participant C as Client
    participant S as DHCP Server
    
    C{-S: 1. DHCP Discover (Broadcast)}
    S{-C: 2. DHCP Offer}
    C{-S: 3. DHCP Request}
    S{-C: 4. DHCP ACK}
    
    Note over C,S: Lease Time
    
    C{-S: 5. DHCP Renewal Request}
    S{-C: 6. DHCP ACK}
\end{verbatim}

\textbf{ચાર-પગલાની પ્રક્રિયા:}

{\def\LTcaptype{none} % do not increment counter
\begin{longtable}[]{@{}lll@{}}
\toprule\noalign{}
Step & Message & Function \\
\midrule\noalign{}
\endhead
\bottomrule\noalign{}
\endlastfoot
\textbf{1} & DISCOVER & Client IP માટે broadcast request કરે \\
\textbf{2} & OFFER & Server available IP address offer કરે \\
\textbf{3} & REQUEST & Client specific IP address request કરે \\
\textbf{4} & ACK & Server IP assignment confirm કરે \\
\end{longtable}
}

\textbf{DHCP Components:}

\begin{itemize}
\tightlist
\item
  \textbf{DHCP Server}: IP address pool અને assignments manage કરે
\item
  \textbf{DHCP Client}: Assigned configuration request કરે અને વાપરે
\item
  \textbf{DHCP Relay}: Subnets પર DHCP messages forward કરે
\item
  \textbf{Address Pool}: Available IP addresses નો range
\end{itemize}

\textbf{Configuration Information Provided:}

\begin{itemize}
\tightlist
\item
  \textbf{IP Address}: Unique network identifier
\item
  \textbf{Subnet Mask}: Network boundary definition
\item
  \textbf{Default Gateway}: Other networks નો route
\item
  \textbf{DNS Servers}: Domain name resolution
\item
  \textbf{Lease Time}: IP assignment નો duration
\end{itemize}

\textbf{ફાયદાઓ:}

\begin{itemize}
\tightlist
\item
  \textbf{Automatic configuration}: Manual IP assignment ની જરૂર નથી
\item
  \textbf{Centralized management}: Network configuration માટે single
  point
\item
  \textbf{Efficient utilization}: Dynamic allocation waste prevent કરે
\item
  \textbf{Reduced errors}: Manual configuration mistakes eliminate કરે
\end{itemize}

\end{solutionbox}
\begin{mnemonicbox}
``Discover Offer Request ACK - Server Client Relay
Pool''

\end{mnemonicbox}
\begin{center}\rule{0.5\linewidth}{0.5pt}\end{center}

\subsection*{પ્રશ્ન 5(અ) [3
ગુણ]}\label{uxaaauxab0uxab6uxaa8-5uxa85-3-uxa97uxaa3}

\textbf{WLAN ના પ્રકાર જણાવો અને કોઈપણ એક સમજાવો}

\begin{solutionbox}

\textbf{WLAN પ્રકારો:}

{\def\LTcaptype{none} % do not increment counter
\begin{longtable}[]{@{}lll@{}}
\toprule\noalign{}
Type & Standard & Frequency \\
\midrule\noalign{}
\endhead
\bottomrule\noalign{}
\endlastfoot
\textbf{Infrastructure} & 802.11 & 2.4/5 GHz \\
\textbf{Ad-hoc} & IBSS & 2.4/5 GHz \\
\textbf{Mesh} & 802.11s & Multiple \\
\end{longtable}
}

\textbf{Infrastructure WLAN સમજાવટ:}

\begin{itemize}
\tightlist
\item
  \textbf{Access Point (AP)}: બધા communications માટે central coordinator
\item
  \textbf{BSS (Basic Service Set)}: Single AP નો network coverage area
\item
  \textbf{ESS (Extended Service Set)}: Multiple interconnected BSSs
\item
  \textbf{Distribution System}: Multiple APs ને connect કરતું backbone
\end{itemize}

\textbf{લાક્ષણિકતાઓ:}

\begin{itemize}
\tightlist
\item
  બધા communication access point દ્વારા જાય છે
\item
  Centralized network management
\item
  વધુ સારું security અને performance control
\end{itemize}

\end{solutionbox}
\begin{mnemonicbox}
``Infrastructure Ad-hoc Mesh - AP BSS ESS
Distribution''

\end{mnemonicbox}
\begin{center}\rule{0.5\linewidth}{0.5pt}\end{center}

\subsection*{પ્રશ્ન 5(બ) [4
ગુણ]}\label{uxaaauxab0uxab6uxaa8-5uxaac-4-uxa97uxaa3}

\textbf{નીચેના પ્રશ્નોના જવાબ આપો:}

\begin{solutionbox}

\textbf{1) Ad hoc Network ના ઉપયોગોની યાદી આપો:}

{\def\LTcaptype{none} % do not increment counter
\begin{longtable}[]{@{}ll@{}}
\toprule\noalign{}
Use Case & Application \\
\midrule\noalign{}
\endhead
\bottomrule\noalign{}
\endlastfoot
\textbf{Emergency} & Disaster recovery, rescue operations \\
\textbf{Military} & Battlefield communications \\
\textbf{Conferences} & Temporary meeting networks \\
\textbf{Home} & Device-to-device communication \\
\textbf{Vehicular} & Car-to-car networks \\
\end{longtable}
}

\textbf{2) મોબાઇલ કમ્પ્યુટિંગની Entities અને Terminology ની યાદી લખો:}

\textbf{Entities:}

\begin{itemize}
\tightlist
\item
  \textbf{Mobile Node (MN)}: Moving device
\item
  \textbf{Home Agent (HA)}: Permanent network representative
\item
  \textbf{Foreign Agent (FA)}: Temporary network coordinator
\item
  \textbf{Correspondent Node (CN)}: Communication partner
\end{itemize}

\textbf{Terminology:}

\begin{itemize}
\tightlist
\item
  \textbf{Handover}: Network switching process
\item
  \textbf{Roaming}: Moving between networks
\item
  \textbf{Care-of Address}: Temporary IP address
\end{itemize}

\end{solutionbox}
\begin{mnemonicbox}
``Emergency Military Conference Home Vehicular - MN
HA FA CN''

\end{mnemonicbox}
\begin{center}\rule{0.5\linewidth}{0.5pt}\end{center}

\subsection*{પ્રશ્ન 5(ક) [7
ગુણ]}\label{uxaaauxab0uxab6uxaa8-5uxa95-7-uxa97uxaa3}

\textbf{સ્વચ્છ આકૃતિ સાથે WLAN ના આર્કિટેક્ચરને સમજાવો}

\begin{solutionbox}

\textbf{WLAN Architecture} access points દ્વારા communicate કરતા wireless
stations ધરાવે છે.

\begin{center}
\textbf{Mermaid Diagram (Code)}
\begin{verbatim}
{Shaded}
{Highlighting}[]
graph TD
    subgraph "BSS 1"
        A[Laptop] {-{-}{} AP1[Access Point 1]}
        B[Phone] {-{-}{} AP1}
        C[Tablet] {-{-}{} AP1}
    end
    
    subgraph "BSS 2"
        D[Desktop] {-{-}{} AP2[Access Point 2]}
        E[Printer] {-{-}{} AP2}
    end
    
    AP1 {-{-}{} DS[Distribution System]}
    AP2 {-{-}{} DS}
    DS {-{-}{} F[Wired Network/Internet]}
    
    subgraph "Ad{-hoc Network"}
        G[Device A] {{-}{-}{} H[Device B]}
        H {{-}{-}{} I[Device C]}
    end
{Highlighting}
{Shaded}
\end{verbatim}
\end{center}

\textbf{Architecture Components:}

{\def\LTcaptype{none} % do not increment counter
\begin{longtable}[]{@{}lll@{}}
\toprule\noalign{}
Component & Function & Coverage \\
\midrule\noalign{}
\endhead
\bottomrule\noalign{}
\endlastfoot
\textbf{STA (Station)} & Wireless device & Point \\
\textbf{AP (Access Point)} & Network coordinator & BSS area \\
\textbf{BSS (Basic Service Set)} & Single AP coverage &
\textasciitilde100m radius \\
\textbf{ESS (Extended Service Set)} & Multiple connected BSS & Large
area \\
\textbf{DS (Distribution System)} & AP interconnection &
Building/campus \\
\end{longtable}
}

\textbf{WLAN Architecture ના પ્રકારો:}

\textbf{1. Infrastructure Mode:}

\begin{itemize}
\tightlist
\item
  \textbf{Centralized}: બધા traffic access points દ્વારા
\item
  \textbf{Managed}: Network administration અને security
\item
  \textbf{Scalable}: Coverage area expand કરવામાં easy
\end{itemize}

\textbf{2. Ad-hoc Mode (IBSS):}

\begin{itemize}
\tightlist
\item
  \textbf{Peer-to-peer}: Direct device communication
\item
  \textbf{Decentralized}: કોઈ central coordinator નથી
\item
  \textbf{Temporary}: Specific needs માટે quick setup
\end{itemize}

\textbf{મુખ્ય લાક્ષણિકતાઓ:}

\begin{itemize}
\tightlist
\item
  \textbf{Mobility}: Users coverage area માં move કરી શકે
\item
  \textbf{Wireless medium}: Communication માટે radio waves
\item
  \textbf{Shared bandwidth}: Multiple users channel capacity share કરે
\item
  \textbf{Security}: Protection માટે WPA/WPA2/WPA3 protocols
\end{itemize}

\textbf{Standards અને Frequencies:}

\begin{itemize}
\tightlist
\item
  \textbf{802.11a}: 5 GHz, up to 54 Mbps
\item
  \textbf{802.11b}: 2.4 GHz, up to 11 Mbps
\item
  \textbf{802.11g}: 2.4 GHz, up to 54 Mbps
\item
  \textbf{802.11n}: 2.4/5 GHz, up to 600 Mbps
\item
  \textbf{802.11ac}: 5 GHz, up to 6.93 Gbps
\end{itemize}

\end{solutionbox}
\begin{mnemonicbox}
``STA AP BSS ESS DS - Infrastructure Ad-hoc''

\end{mnemonicbox}
\begin{center}\rule{0.5\linewidth}{0.5pt}\end{center}

\subsection*{પ્રશ્ન 5(અ OR) [3
ગુણ]}\label{uxaaauxab0uxab6uxaa8-5uxa85-or-3-uxa97uxaa3}

\textbf{5G ની લાક્ષણિકતાઓ લખો}

\begin{solutionbox}

\textbf{5G મુખ્ય લાક્ષણિકતાઓ:}

{\def\LTcaptype{none} % do not increment counter
\begin{longtable}[]{@{}ll@{}}
\toprule\noalign{}
Feature & Specification \\
\midrule\noalign{}
\endhead
\bottomrule\noalign{}
\endlastfoot
\textbf{Speed} & Up to 10 Gbps સુધી \\
\textbf{Latency} & \textless{} 1 millisecond \\
\textbf{Connectivity} & 1 million devices/km^{2} \\
\textbf{Reliability} & 99.999\% availability \\
\textbf{Bandwidth} & 100x વધારો \\
\textbf{Energy} & 90\% ઘટાડો \\
\end{longtable}
}

\textbf{Advanced Capabilities:}

\begin{itemize}
\tightlist
\item
  \textbf{Enhanced Mobile Broadband (eMBB)}: Ultra-fast data speeds
\item
  \textbf{Ultra-Reliable Low Latency (URLLC)}: Mission-critical
  applications
\item
  \textbf{Massive Machine Type Communication (mMTC)}: IoT connectivity
\end{itemize}

\end{solutionbox}
\begin{mnemonicbox}
``Speed Latency Connectivity Reliability Bandwidth
Energy''

\end{mnemonicbox}
\begin{center}\rule{0.5\linewidth}{0.5pt}\end{center}

\subsection*{પ્રશ્ન 5(બ OR) [4
ગુણ]}\label{uxaaauxab0uxab6uxaa8-5uxaac-or-4-uxa97uxaa3}

\textbf{નીચેના પ્રશ્નોના જવાબ આપો:}

\begin{solutionbox}

\textbf{1) communication middleware ની પ્રકારોની યાદી લખો:}

{\def\LTcaptype{none} % do not increment counter
\begin{longtable}[]{@{}ll@{}}
\toprule\noalign{}
Type & Function \\
\midrule\noalign{}
\endhead
\bottomrule\noalign{}
\endlastfoot
\textbf{Message-Oriented} & Asynchronous messaging \\
\textbf{RPC-based} & Remote procedure calls \\
\textbf{Object-Oriented} & Distributed objects \\
\textbf{Service-Oriented} & Web services \\
\textbf{Database} & Data access layer \\
\end{longtable}
}

\textbf{2) Mobile IP ના સંદર્ભમાં ``Home Agent'' ની વ્યાખ્યા આપો:}

\textbf{Home Agent (HA)} mobile node ના home network પરનો router છે જે:

\begin{itemize}
\tightlist
\item
  \textbf{Registration maintain કરે}: Mobile node નું current location
  track કરે
\item
  \textbf{Packets tunnel કરે}: Mobile node ના foreign location પર data
  forward કરે
\item
  \textbf{Address management}: Mobile node નું permanent IP address manage
  કરે
\item
  \textbf{Authentication}: Registration દરમિયાન mobile node identity
  verify કરે
\end{itemize}

\textbf{Functions:}

\begin{itemize}
\tightlist
\item
  Mobile node home થી દૂર હોય ત્યારે proxy તરીકે કામ કરે
\item
  Mobile node માટે destined packets intercept કરે
\item
  Foreign agents સાથે tunnels create કરે
\end{itemize}

\end{solutionbox}
\begin{mnemonicbox}
``Message RPC Object Service Database - HA Maintains
Tunnels Address Authentication''

\end{mnemonicbox}
\begin{center}\rule{0.5\linewidth}{0.5pt}\end{center}

\subsection*{પ્રશ્ન 5(ક OR) [7
ગુણ]}\label{uxaaauxab0uxab6uxaa8-5uxa95-or-7-uxa97uxaa3}

\textbf{Bluetooth protocol stack આકૃતિ સાથે સમજાવો}

\begin{solutionbox}

\textbf{Bluetooth Protocol Stack} short-range wireless communication માટે
layered architecture provide કરે છે.

\begin{center}
\textbf{Mermaid Diagram (Code)}
\begin{verbatim}
{Shaded}
{Highlighting}[]
graph LR
    A[Applications] {-{-}{} B[Application Layer]}
    B {-{-}{} C[OBEX/SDP/TCS]}
    C {-{-}{} D[RFCOMM]}
    D {-{-}{} E[L2CAP]}
    E {-{-}{} F[HCI {-} Host Controller Interface]}
    F {-{-}{} G[Link Manager Protocol {-} LMP]}
    G {-{-}{} H[Baseband]}
    H {-{-}{} I[Radio Layer]}
{Highlighting}
{Shaded}
\end{verbatim}
\end{center}

\textbf{Protocol Stack Layers:}

{\def\LTcaptype{none} % do not increment counter
\begin{longtable}[]{@{}lll@{}}
\toprule\noalign{}
Layer & Function & Protocols \\
\midrule\noalign{}
\endhead
\bottomrule\noalign{}
\endlastfoot
\textbf{Application} & User applications & Audio, File transfer \\
\textbf{Middleware} & Services & OBEX, SDP, TCS \\
\textbf{Transport} & Data delivery & RFCOMM \\
\textbf{Network} & Packet management & L2CAP \\
\textbf{Interface} & Host-Controller & HCI \\
\textbf{Management} & Link control & LMP \\
\textbf{Data Link} & Channel access & Baseband \\
\textbf{Physical} & Radio transmission & 2.4 GHz ISM \\
\end{longtable}
}

\textbf{Layer વિગતો:}

\textbf{Upper Layers:}

\begin{itemize}
\tightlist
\item
  \textbf{OBEX}: File transfers માટે Object Exchange Protocol
\item
  \textbf{SDP}: Available services શોધવા માટે Service Discovery Protocol
\item
  \textbf{TCS}: Voice calls માટે Telephony Control Specification
\item
  \textbf{RFCOMM}: Bluetooth પર serial port emulation
\end{itemize}

\textbf{Lower Layers:}

\begin{itemize}
\tightlist
\item
  \textbf{L2CAP}: Multiple connections manage કરે છે Logical Link Control
\item
  \textbf{HCI}: Communication standardize કરે છે Host Controller Interface
\item
  \textbf{LMP}: Connection setup handle કરે છે Link Manager Protocol
\item
  \textbf{Baseband}: Time slots અને frequency hopping manage કરે
\end{itemize}

\textbf{મુખ્ય લાક્ષણિકતાઓ:}

\begin{itemize}
\tightlist
\item
  \textbf{Frequency Hopping}: 79 channels પર 1600 hops/second
\item
  \textbf{Piconet}: 8 devices સુધીનું network
\item
  \textbf{Scatternet}: Multiple overlapping piconets
\item
  \textbf{Power Classes}: Class 1 (100m), Class 2 (10m), Class 3 (1m)
\end{itemize}

\textbf{ફાયદાઓ:}

\begin{itemize}
\tightlist
\item
  \textbf{Low power consumption}: Battery devices માટે suitable
\item
  \textbf{Automatic pairing}: Easy device connection
\item
  \textbf{Interference resistance}: Frequency hopping spread spectrum
\item
  \textbf{Cost effective}: Low implementation cost
\end{itemize}

\textbf{Applications:}

\begin{itemize}
\tightlist
\item
  \textbf{Audio streaming}: Headphones, speakers
\item
  \textbf{Data transfer}: Devices વચ્ચે file sharing
\item
  \textbf{Input devices}: Keyboards, mice
\item
  \textbf{IoT devices}: Sensors, smart home devices
\end{itemize}

\end{solutionbox}
\begin{mnemonicbox}
``Application Middleware Transport Network Interface
Management DataLink Physical''

\end{mnemonicbox}

\end{document}
