\documentclass[10pt,a4paper]{article}

% content/resources/templates/preamble.tex
\usepackage[margin=0.6in]{geometry}
\author{Milav Dabgar}
\usepackage{amsmath,amssymb,amsthm}
\usepackage{booktabs}
\usepackage{multirow}
\usepackage{xcolor}
\usepackage{tcolorbox}
\tcbuselibrary{breakable,skins}
\usepackage[colorlinks=true,linkcolor=blue]{hyperref}
\usepackage{titlesec}
\usepackage{enumitem}
\usepackage{tikz}
\usepackage{pgfplots}
\usepackage{circuitikz}
\usepackage[version=4]{mhchem}
\usepackage{longtable}
\usepackage{array}
\usepackage{float}
\usepackage{caption}
\usepackage{listings}

\lstset{
  basicstyle=\small\ttfamily,
  breaklines=true,
  breakatwhitespace=false,
  postbreak=\mbox{\textcolor{red}{$\hookrightarrow$}\space},
  float=false,
  numbers=left,
  numberstyle=\tiny\color{gray},
  numbersep=10pt,
  xleftmargin=2em,
  keywordstyle=\color{blue},
  commentstyle=\color{green!60!black},
  stringstyle=\color{purple},
  backgroundcolor=\color{gray!5},
  showstringspaces=false,
  tabsize=2,
  captionpos=b,
  keepspaces=true,
  columns=flexible
}

\pgfplotsset{compat=1.18}
\usetikzlibrary{shapes,arrows,positioning,calc,patterns,decorations.pathmorphing,decorations.markings,arrows.meta}

% Color scheme
\definecolor{headcolor}{RGB}{0,102,204}
\definecolor{keycolor}{RGB}{220,20,60}
\definecolor{solutioncolor}{RGB}{34,139,34}
\definecolor{mnemoniccolor}{RGB}{148,0,211}
\definecolor{codecolor}{RGB}{0,0,100}

% Spacing
\setlength{\parskip}{3pt}
\setlist[itemize]{nosep}
\setlist[enumerate]{nosep}

% Title formatting
\titleformat{\section}{\Large\bfseries\color{headcolor}}{\thesection}{1em}{}
\titleformat{\subsection}{\large\bfseries\color{headcolor}}{\thesubsection}{1em}{}

% Pandoc tightlist compatibility
\providecommand{\tightlist}{%
  \setlength{\itemsep}{0pt}\setlength{\parskip}{0pt}}

% Pandoc longtable compatibility
\newcounter{none}
\def\thenone{}


% content/resources/templates/gujarati-boxes.tex
\usepackage{fontspec}
\usepackage{polyglossia}

% Set Gujarati as main language (document is primarily in Gujarati)
% Note: gloss-gujarati.ldf doesn't exist in polyglossia, but it will use hyphenation patterns
\setdefaultlanguage{gujarati}
\setotherlanguage{english}

% Configure Gujarati font properly
% Use Language=Default to prevent polyglossia from trying to add language-specific features
% that don't exist for Gujarati, which causes "empty feature" warnings
\newfontfamily\gujaratifont[Script=Gujarati,AutoFakeBold=2.5,AutoFakeSlant=0.3]{Noto Sans Gujarati}
\setmainfont[Script=Gujarati,AutoFakeBold=2.5,AutoFakeSlant=0.3]{Noto Sans Gujarati}
% Use Noto Sans Gujarati for monospace to support Gujarati in text
\setmonofont[Scale=0.9]{Noto Sans Gujarati}

% Configure English to use the same font
\newfontfamily\englishfont[Script=Gujarati,AutoFakeBold=2.5,AutoFakeSlant=0.3]{Noto Sans Gujarati}

% Translations for polyglossia
\gappto\captionsgujarati{
  \renewcommand{\tablename}{કોષ્ટક}
  \renewcommand{\figurename}{આકૃતિ}
}

% Helper for TikZ nodes to ensure Gujarati font
\newcommand{\gu}[1]{{\gujaratifont #1}}

% Custom environments
\newtcolorbox{solutionbox}{
    breakable,
    enhanced,
    colback=solutioncolor!5!white,
    colframe=solutioncolor!75!black,
    fonttitle=\bfseries,
    title=જવાબ
}

\newtcolorbox{solutionboxnobreak}{
 colback=solutioncolor!5!white,
 colframe=solutioncolor!75!black,
 fonttitle=\bfseries,
 title=જવાબ
}

\newtcolorbox{keyformula}{
 breakable,
 enhanced,
 colback=keycolor!5!white,
 colframe=keycolor!75!black,
 fonttitle=\bfseries,
 title=રાસાયણિક સમીકરણ/સૂત્ર
}

\newtcolorbox{mnemonicbox}{
 breakable,
 enhanced,
 colback=mnemoniccolor!5!white,
 colframe=mnemoniccolor!75!black,
 fonttitle=\bfseries,
 title=મેમરી ટ્રીક
}


\begin{document}

\begin{center}
{\Huge\bfseries\color{headcolor} Subject Name (Gujarati)}\\[5pt]
{\LARGE 4351602 -- Summer 2024}\\[3pt]
{\large Semester 1 Study Material}\\[3pt]
{\normalsize\textit{Detailed Solutions and Explanations}}
\end{center}

\vspace{10pt}

\subsection*{પ્રશ્ન 1(અ) [3
ગુણ]}\label{uxaaauxab0uxab6uxaa8-1uxa85-3-uxa97uxaa3}

\textbf{વ્યાખ્યાયિત કરો : Peer to Peer network}

\begin{solutionbox}
Peer-to-Peer (P2P) નેટવર્ક એ વિતરિત નેટવર્ક આર્કિટેક્ચર છે જ્યાં
દરેક નોડ (peer) ક્લાયન્ટ અને સર્વર બંને તરીકે કામ કરે છે અને કેન્દ્રીય નિયંત્રણ વિના
સીધા સંસાધનો શેર કરે છે.

\textbf{ટેબલ:}

{\def\LTcaptype{none} % do not increment counter
\begin{longtable}[]{@{}ll@{}}
\toprule\noalign{}
પાસાં & વર્ણન \\
\midrule\noalign{}
\endhead
\bottomrule\noalign{}
\endlastfoot
\textbf{સ્ટ્રક્ચર} & વિકેન્દ્રીકૃત નેટવર્ક \\
\textbf{રોલ} & દરેક peer ક્લાયન્ટ અને સર્વર \\
\textbf{કંટ્રોલ} & કોઈ કેન્દ્રીય સત્તા નથી \\
\textbf{ઉદાહરણ} & BitTorrent, Skype \\
\end{longtable}
}

\end{solutionbox}
\begin{mnemonicbox}
``Peers Share Equally''

\end{mnemonicbox}
\begin{center}\rule{0.5\linewidth}{0.5pt}\end{center}

\subsection*{પ્રશ્ન 1(બ) [4
ગુણ]}\label{uxaaauxab0uxab6uxaa8-1uxaac-4-uxa97uxaa3}

\textbf{તુલના કરો : SMTP, POP અને IMAP}

\begin{solutionbox}
ઈમેઈલ પ્રોટોકોલ્સ ઈમેઈલ કમ્યુનિકેશન સિસ્ટમમાં અલગ અલગ હેતુઓ પૂરા કરે
છે.

\textbf{ટેબલ:}

{\def\LTcaptype{none} % do not increment counter
\begin{longtable}[]{@{}llll@{}}
\toprule\noalign{}
ફીચર & SMTP & POP3 & IMAP \\
\midrule\noalign{}
\endhead
\bottomrule\noalign{}
\endlastfoot
\textbf{હેતુ} & ઈમેઈલ મોકલવા & ઈમેઈલ ડાઉનલોડ કરવા & ઈમેઈલ એક્સેસ કરવા \\
\textbf{પોર્ટ} & 25, 587 & 110, 995 & 143, 993 \\
\textbf{સ્ટોરેજ} & સર્વર ફોરવર્ડ કરે & લોકલ સ્ટોરેજ & સર્વર સ્ટોરેજ \\
\textbf{એક્સેસ} & એક દિશામાં મોકલવું & સિંગલ ડિવાઇસ & મલ્ટિપલ ડિવાઇસ \\
\end{longtable}
}

\end{solutionbox}
\begin{mnemonicbox}
``Send-Pop-Internet Mail Access''

\end{mnemonicbox}
\begin{center}\rule{0.5\linewidth}{0.5pt}\end{center}

\subsection*{પ્રશ્ન 1(ક) [7
ગુણ]}\label{uxaaauxab0uxab6uxaa8-1uxa95-7-uxa97uxaa3}

\textbf{દરેક સ્તરની જવાબદારી સાથે OSI model સમજાવો}

\begin{solutionbox}
OSI (Open Systems Interconnection) મોડેલમાં સાત સ્તરો છે,
દરેકની નેટવર્ક કમ્યુનિકેશન માટે ચોક્કસ જવાબદારીઓ છે.

\textbf{ડાયાગ્રામ:}

\begin{center}
\textbf{Mermaid Diagram (Code)}
\begin{verbatim}
{Shaded}
{Highlighting}[]
graph LR
    A[Application Layer 7] {-{-}{} B[Presentation Layer 6]}
    B {-{-}{} C[Session Layer 5]}
    C {-{-}{} D[Transport Layer 4]}
    D {-{-}{} E[Network Layer 3]}
    E {-{-}{} F[Data Link Layer 2]}
    F {-{-}{} G[Physical Layer 1]}
{Highlighting}
{Shaded}
\end{verbatim}
\end{center}

\textbf{ટેબલ:}

{\def\LTcaptype{none} % do not increment counter
\begin{longtable}[]{@{}lll@{}}
\toprule\noalign{}
સ્તર & નામ & જવાબદારીઓ \\
\midrule\noalign{}
\endhead
\bottomrule\noalign{}
\endlastfoot
\textbf{7} & Application & યુઝર ઇન્ટરફેસ, નેટવર્ક સેવાઓ \\
\textbf{6} & Presentation & ડેટા એન્ક્રિપ્શન, કમ્પ્રેશન \\
\textbf{5} & Session & સેશન મેનેજમેન્ટ, ડાયલોગ કંટ્રોલ \\
\textbf{4} & Transport & End-to-end ડિલિવરી, એરર કંટ્રોલ \\
\textbf{3} & Network & રૂટિંગ, લોજિકલ એડ્રેસિંગ \\
\textbf{2} & Data Link & ફ્રેમ ફોર્મેટિંગ, એરર ડિટેક્શન \\
\textbf{1} & Physical & બિટ ટ્રાન્સમિશન, હાર્ડવેર \\
\end{longtable}
}

\textbf{મુખ્ય મુદ્દાઓ:}

\begin{itemize}
\tightlist
\item
  \textbf{Application Layer}: એપ્લિકેશનોને નેટવર્ક સેવાઓ પ્રદાન કરે
\item
  \textbf{Transport Layer}: વિશ્વસનીય ડેટા ડિલિવરી સુનિશ્ચિત કરે
\item
  \textbf{Network Layer}: નેટવર્ક્સ વચ્ચે રૂટિંગ હેન્ડલ કરે
\end{itemize}

\end{solutionbox}
\begin{mnemonicbox}
``All People Seem To Need Data Processing''

\end{mnemonicbox}
\begin{center}\rule{0.5\linewidth}{0.5pt}\end{center}

\subsection*{પ્રશ્ન 1(ક OR) [7
ગુણ]}\label{uxaaauxab0uxab6uxaa8-1uxa95-or-7-uxa97uxaa3}

\textbf{TCP/IP model ની OSI model સાથે તુલના કરો}

\begin{solutionbox}
TCP/IP અને OSI મોડેલ્સ અલગ અલગ લેયર સ્ટ્રક્ચર સાથે નેટવર્ક આર્કિટેક્ચર
ફ્રેમવર્ક છે.

\textbf{ડાયાગ્રામ:}

\begin{center}
\textbf{Mermaid Diagram (Code)}
\begin{verbatim}
{Shaded}
{Highlighting}[]
graph LR
    subgraph "OSI Model"
        O1[Application]
        O2[Presentation]
        O3[Session]
        O4[Transport]
        O5[Network]
        O6[Data Link]
        O7[Physical]
    end
    
    subgraph "TCP/IP Model"
        T1[Application]
        T2[Transport]
        T3[Internet]
        T4[Network Access]
    end
{Highlighting}
{Shaded}
\end{verbatim}
\end{center}

\textbf{ટેબલ:}

{\def\LTcaptype{none} % do not increment counter
\begin{longtable}[]{@{}lll@{}}
\toprule\noalign{}
પાસાં & OSI Model & TCP/IP Model \\
\midrule\noalign{}
\endhead
\bottomrule\noalign{}
\endlastfoot
\textbf{લેયર્સ} & 7 લેયર્સ & 4 લેયર્સ \\
\textbf{ડેવલપમેન્ટ} & થિયોરેટિકલ & પ્રેક્ટિકલ \\
\textbf{ઉપયોગ} & રેફરન્સ મોડેલ & ઇન્ટરનેટ સ્ટાન્ડર્ડ \\
\textbf{જટિલતા} & વધુ વિગતવાર & સરળીકૃત \\
\end{longtable}
}

\textbf{મુખ્ય મુદ્દાઓ:}

\begin{itemize}
\tightlist
\item
  \textbf{OSI}: વિગતવાર અલગીકરણ સાથે થિયોરેટિકલ ફ્રેમવર્ક
\item
  \textbf{TCP/IP}: ઇન્ટરનેટ માટે પ્રેક્ટિકલ ઇમ્પ્લિમેન્ટેશન
\item
  \textbf{મેપિંગ}: OSI ના ટોપ 3 લેયર્સ = TCP/IP માં Application layer
\end{itemize}

\end{solutionbox}
\begin{mnemonicbox}
``OSI Seven, TCP Four''

\end{mnemonicbox}
\begin{center}\rule{0.5\linewidth}{0.5pt}\end{center}

\subsection*{પ્રશ્ન 2(અ) [3
ગુણ]}\label{uxaaauxab0uxab6uxaa8-2uxa85-3-uxa97uxaa3}

\textbf{સમજાવો : Network Address Translation (NAT)}

\begin{solutionbox}
NAT પ્રાઇવેટ IP એડ્રેસને પબ્લિક IP એડ્રેસમાં ટ્રાન્સલેટ કરે છે, જે
મલ્ટિપલ ડિવાઇસને સિંગલ પબ્લિક IP શેર કરવા સક્ષમ બનાવે છે.

\textbf{ડાયાગ્રામ:}

\begin{verbatim}
Private Network    NAT Router    Internet
192.168.1.10  {-{-}  203.0.113.1  {-}{-}  Server}
192.168.1.20  {-{-}  203.0.113.1  {-}{-}  Server}
192.168.1.30  {-{-}  203.0.113.1  {-}{-}  Server}
\end{verbatim}

\textbf{મુખ્ય મુદ્દાઓ:}

\begin{itemize}
\tightlist
\item
  \textbf{હેતુ}: નેટવર્ક્સ વચ્ચે IP એડ્રેસ ટ્રાન્સલેશન
\item
  \textbf{ફાયદો}: પબ્લિક IP એડ્રેસની બચત
\item
  \textbf{સિક્યોરિટી}: આંતરિક નેટવર્ક સ્ટ્રક્ચર છુપાવે છે
\end{itemize}

\end{solutionbox}
\begin{mnemonicbox}
``Network Address Translation''

\end{mnemonicbox}
\begin{center}\rule{0.5\linewidth}{0.5pt}\end{center}

\subsection*{પ્રશ્ન 2(બ) [4
ગુણ]}\label{uxaaauxab0uxab6uxaa8-2uxaac-4-uxa97uxaa3}

\textbf{વ્યાખ્યાયિત કરો : Subnetting and Supernetting}

\begin{solutionbox}
Subnetting અને Supernetting કાર્યક્ષમ નેટવર્ક મેનેજમેન્ટ માટે IP
એડ્રેસિંગ તકનીકો છે.

\textbf{ટેબલ:}

{\def\LTcaptype{none} % do not increment counter
\begin{longtable}[]{@{}lll@{}}
\toprule\noalign{}
તકનીક & વ્યાખ્યા & હેતુ \\
\midrule\noalign{}
\endhead
\bottomrule\noalign{}
\endlastfoot
\textbf{Subnetting} & નેટવર્કને નાના સબનેટ્સમાં વિભાજન & બહેતર સંગઠન \\
\textbf{Supernetting} & મલ્ટિપલ નેટવર્ક્સનું સંયોજન & રૂટ એગ્રિગેશન \\
\end{longtable}
}

\textbf{મુખ્ય મુદ્દાઓ:}

\begin{itemize}
\tightlist
\item
  \textbf{Subnetting}: નેટવર્ક બિટ્સ વધારે, હોસ્ટ બિટ્સ ઓછા કરે
\item
  \textbf{Supernetting}: નેટવર્ક બિટ્સ ઓછા કરે, રૂટિંગ કાર્યક્ષમતા વધારે
\item
  \textbf{CIDR}: Classless Inter-Domain Routing બંનેને સક્ષમ બનાવે
\end{itemize}

\end{solutionbox}
\begin{mnemonicbox}
``Sub-divides, Super-combines''

\end{mnemonicbox}
\begin{center}\rule{0.5\linewidth}{0.5pt}\end{center}

\subsection*{પ્રશ્ન 2(ક) [7
ગુણ]}\label{uxaaauxab0uxab6uxaa8-2uxa95-7-uxa97uxaa3}

\textbf{સમજાવો : IPv4 ની Classful અને Classless notation addressing
scheme}

\begin{solutionbox}
IPv4 એડ્રેસિંગ નેટવર્ક ઓળખ માટે classful અને classless સ્કીમનો
ઉપયોગ કરે છે.

\textbf{ટેબલ - Classful Addressing:}

{\def\LTcaptype{none} % do not increment counter
\begin{longtable}[]{@{}lllll@{}}
\toprule\noalign{}
Class & Range & Default Mask & Networks & Hosts \\
\midrule\noalign{}
\endhead
\bottomrule\noalign{}
\endlastfoot
\textbf{A} & 1-126 & /8 (255.0.0.0) & 126 & 16M \\
\textbf{B} & 128-191 & /16 (255.255.0.0) & 16K & 65K \\
\textbf{C} & 192-223 & /24 (255.255.255.0) & 2M & 254 \\
\end{longtable}
}

\textbf{Classless (CIDR) ઉદાહરણો:}

\begin{itemize}
\tightlist
\item
  \textbf{192.168.1.0/25}: 128 hosts
\item
  \textbf{10.0.0.0/16}: 65,536 hosts
\item
  \textbf{172.16.0.0/20}: 4,096 hosts
\end{itemize}

\textbf{મુખ્ય મુદ્દાઓ:}

\begin{itemize}
\tightlist
\item
  \textbf{Classful}: ફિક્સ્ડ નેટવર્ક/હોસ્ટ બાઉન્ડરીઝ
\item
  \textbf{Classless}: Variable Length Subnet Mask (VLSM)
\item
  \textbf{CIDR}: વધુ કાર્યક્ષમ એડ્રેસ એલોકેશન
\end{itemize}

\end{solutionbox}
\begin{mnemonicbox}
``Class-Fixed, CIDR-Flexible''

\end{mnemonicbox}
\begin{center}\rule{0.5\linewidth}{0.5pt}\end{center}

\subsection*{પ્રશ્ન 2(અ OR) [3
ગુણ]}\label{uxaaauxab0uxab6uxaa8-2uxa85-or-3-uxa97uxaa3}

\textbf{મોબાઇલ IP ના ધ્યેયોની ચર્ચા કરો}

\begin{solutionbox}
મોબાઇલ IP મોબાઇલ ડિવાઇસ માટે વિવિધ નેટવર્ક્સમાં સીમલેસ
કનેક્ટિવિટી સક્ષમ કરે છે.

\textbf{મુખ્ય મુદ્દાઓ:}

\begin{itemize}
\tightlist
\item
  \textbf{પારદર્શિતા}: એપ્લિકેશનોને મોબિલિટીની જાણ નથી
\item
  \textbf{સુસંગતતા}: હાલના પ્રોટોકોલ્સ સાથે કામ કરે
\item
  \textbf{કાર્યક્ષમતા}: ન્યૂનતમ રૂટિંગ ઓવરહેડ
\end{itemize}

\end{solutionbox}
\begin{mnemonicbox}
``Transparent Compatible Efficient''

\end{mnemonicbox}
\begin{center}\rule{0.5\linewidth}{0.5pt}\end{center}

\subsection*{પ્રશ્ન 2(બ OR) [4
ગુણ]}\label{uxaaauxab0uxab6uxaa8-2uxaac-or-4-uxa97uxaa3}

\textbf{વ્યાખ્યાયિત કરો : ARP and RARP}

\begin{solutionbox}
ARP અને RARP વિવિધ એડ્રેસ પ્રકારો વચ્ચે મેપિંગ માટે એડ્રેસ રિઝોલ્યુશન
પ્રોટોકોલ્સ છે.

\textbf{ટેબલ:}

{\def\LTcaptype{none} % do not increment counter
\begin{longtable}[]{@{}
  >{\raggedright\arraybackslash}p{(\linewidth - 6\tabcolsep) * \real{0.2439}}
  >{\raggedright\arraybackslash}p{(\linewidth - 6\tabcolsep) * \real{0.2683}}
  >{\raggedright\arraybackslash}p{(\linewidth - 6\tabcolsep) * \real{0.2195}}
  >{\raggedright\arraybackslash}p{(\linewidth - 6\tabcolsep) * \real{0.2683}}@{}}
\toprule\noalign{}
\begin{minipage}[b]{\linewidth}\raggedright
પ્રોટોકોલ
\end{minipage} & \begin{minipage}[b]{\linewidth}\raggedright
પૂરું નામ
\end{minipage} & \begin{minipage}[b]{\linewidth}\raggedright
હેતુ
\end{minipage} & \begin{minipage}[b]{\linewidth}\raggedright
દિશા
\end{minipage} \\
\midrule\noalign{}
\endhead
\bottomrule\noalign{}
\endlastfoot
\textbf{ARP} & Address Resolution Protocol & IP to MAC મેપિંગ & લોજિકલ થી
ફિઝિકલ \\
\textbf{RARP} & Reverse ARP & MAC to IP મેપિંગ & ફિઝિકલ થી લોજિકલ \\
\end{longtable}
}

\end{solutionbox}
\begin{mnemonicbox}
``ARP-asks, RARP-reverses''

\end{mnemonicbox}
\begin{center}\rule{0.5\linewidth}{0.5pt}\end{center}

\subsection*{પ્રશ્ન 2(ક OR) [7
ગુણ]}\label{uxaaauxab0uxab6uxaa8-2uxa95-or-7-uxa97uxaa3}

\textbf{સમજાવો : Stop and Wait, Stop and Wait ARQ data link layer
protocols}

\begin{solutionbox}
આ પ્રોટોકોલ્સ ડેટા લિંક લેયર પર વિશ્વસનીય ડેટા ટ્રાન્સમિશન સુનિશ્ચિત
કરે છે.

\textbf{ડાયાગ્રામ - Stop and Wait:}

\begin{verbatim}
sequenceDiagram
    participant S as Sender
    participant R as Receiver
    S{-R: Frame 0}
    R{-S: ACK 0}
    S{-R: Frame 1}
    R{-S: ACK 1}
\end{verbatim}

\textbf{ટેબલ:}

{\def\LTcaptype{none} % do not increment counter
\begin{longtable}[]{@{}llll@{}}
\toprule\noalign{}
પ્રોટોકોલ & એરર ડિટેક્શન & કાર્યક્ષમતા & જટિલતા \\
\midrule\noalign{}
\endhead
\bottomrule\noalign{}
\endlastfoot
\textbf{Stop and Wait} & બેસિક & ઓછી & સરળ \\
\textbf{Stop and Wait ARQ} & એડવાન્સ્ડ & મધ્યમ & મોડરેટ \\
\end{longtable}
}

\textbf{મુખ્ય મુદ્દાઓ:}

\begin{itemize}
\tightlist
\item
  \textbf{Stop and Wait}: ફ્રેમ મોકલો, acknowledgment ની રાહ જુઓ
\item
  \textbf{ARQ}: એરર પર Automatic Repeat reQuest
\item
  \textbf{Timeout}: કોઈ acknowledgment ન મળે તો ફરીથી મોકલો
\end{itemize}

\end{solutionbox}
\begin{mnemonicbox}
``Stop-Wait-Acknowledge''

\end{mnemonicbox}
\begin{center}\rule{0.5\linewidth}{0.5pt}\end{center}

\subsection*{પ્રશ્ન 3(અ) [3
ગુણ]}\label{uxaaauxab0uxab6uxaa8-3uxa85-3-uxa97uxaa3}

\textbf{Wireless networks સમજાવો}

\begin{solutionbox}
વાયરલેસ નેટવર્ક્સ ફિઝિકલ કનેક્શન વિના કમ્યુનિકેશન માટે રેડિયો તરંગોનો
ઉપયોગ કરે છે.

\textbf{મુખ્ય મુદ્દાઓ:}

\begin{itemize}
\tightlist
\item
  \textbf{ટેકનોલોજી}: રેડિયો ફ્રીક્વન્સી ટ્રાન્સમિશન
\item
  \textbf{પ્રકારો}: WiFi, Bluetooth, સેલ્યુલર
\item
  \textbf{ફાયદાઓ}: મોબિલિટી, સરળ ઇન્સ્ટોલેશન
\end{itemize}

\end{solutionbox}
\begin{mnemonicbox}
``Wireless-Radio-Mobile''

\end{mnemonicbox}
\begin{center}\rule{0.5\linewidth}{0.5pt}\end{center}

\subsection*{પ્રશ્ન 3(બ) [4
ગુણ]}\label{uxaaauxab0uxab6uxaa8-3uxaac-4-uxa97uxaa3}

\textbf{વ્યાખ્યાયિત કરો : Communication Middleware in mobile computing}

\begin{solutionbox}
કમ્યુનિકેશન મિડલવેર મોબાઇલ એપ્લિકેશન કમ્યુનિકેશન માટે અમૂર્તીકરણ લેયર
પ્રદાન કરે છે.

\textbf{ટેબલ:}

{\def\LTcaptype{none} % do not increment counter
\begin{longtable}[]{@{}ll@{}}
\toprule\noalign{}
પાસાં & વર્ણન \\
\midrule\noalign{}
\endhead
\bottomrule\noalign{}
\endlastfoot
\textbf{હેતુ} & કમ્યુનિકેશન સરળ બનાવવું \\
\textbf{સ્થાન} & એપ અને નેટવર્ક વચ્ચે \\
\textbf{ફીચર્સ} & પ્રોટોકોલ હેન્ડલિંગ, ડેટા કન્વર્ઝન \\
\textbf{ઉદાહરણો} & CORBA, RMI \\
\end{longtable}
}

\end{solutionbox}
\begin{mnemonicbox}
``Middle-Communication-Layer''

\end{mnemonicbox}
\begin{center}\rule{0.5\linewidth}{0.5pt}\end{center}

\subsection*{પ્રશ્ન 3(ક) [7
ગુણ]}\label{uxaaauxab0uxab6uxaa8-3uxa95-7-uxa97uxaa3}

\textbf{મોબાઈલ કમ્પ્યુટિંગના આર્કિટેક્ચરની ચર્ચા કરો}

\begin{solutionbox}
મોબાઇલ કમ્પ્યુટિંગ આર્કિટેક્ચર મોબાઇલ એપ્લિકેશનોને સપોર્ટ કરતા
મલ્ટિપલ પરસ્પર જોડાયેલા ઘટકોનો સમાવેશ કરે છે.

\textbf{ડાયાગ્રામ:}

\begin{center}
\textbf{Mermaid Diagram (Code)}
\begin{verbatim}
{Shaded}
{Highlighting}[]
graph LR
    A[Mobile Device] {-{-}{} B[Wireless Network]}
    B {-{-}{} C[Base Station]}
    C {-{-}{} D[Mobile Support Station]}
    D {-{-}{} E[Fixed Network]}
    E {-{-}{} F[Database/Server]}
{Highlighting}
{Shaded}
\end{verbatim}
\end{center}

\textbf{ટેબલ:}

{\def\LTcaptype{none} % do not increment counter
\begin{longtable}[]{@{}ll@{}}
\toprule\noalign{}
ઘટક & કાર્ય \\
\midrule\noalign{}
\endhead
\bottomrule\noalign{}
\endlastfoot
\textbf{Mobile Device} & યુઝર ઇન્ટરફેસ, લોકલ પ્રોસેસિંગ \\
\textbf{Wireless Network} & રેડિયો કમ્યુનિકેશન \\
\textbf{Base Station} & નેટવર્ક એક્સેસ પોઇન્ટ \\
\textbf{MSS} & મોબિલિટી મેનેજમેન્ટ \\
\textbf{Fixed Network} & બેકબોન ઇન્ફ્રાસ્ટ્રક્ચર \\
\end{longtable}
}

\textbf{મુખ્ય મુદ્દાઓ:}

\begin{itemize}
\tightlist
\item
  \textbf{ત્રણ-સ્તરીય}: મોબાઇલ ડિવાઇસ, વાયરલેસ નેટવર્ક, ફિક્સ્ડ નેટવર્ક
\item
  \textbf{મોબિલિટી સપોર્ટ}: હેન્ડઓફ મેનેજમેન્ટ
\item
  \textbf{ડેટા મેનેજમેન્ટ}: કેશિંગ અને સિંક્રોનાઇઝેશન
\end{itemize}

\end{solutionbox}
\begin{mnemonicbox}
``Mobile-Wireless-Fixed''

\end{mnemonicbox}
\begin{center}\rule{0.5\linewidth}{0.5pt}\end{center}

\subsection*{પ્રશ્ન 3(અ OR) [3
ગુણ]}\label{uxaaauxab0uxab6uxaa8-3uxa85-or-3-uxa97uxaa3}

\textbf{ad-hoc networks સમજાવો}

\begin{solutionbox}
Ad-hoc નેટવર્ક્સ ફિક્સ્ડ ઇન્ફ્રાસ્ટ્રક્ચર વિના સેલ્ફ-ઓર્ગેનાઇઝિંગ વાયરલેસ
નેટવર્ક્સ છે.

\textbf{મુખ્ય મુદ્દાઓ:}

\begin{itemize}
\tightlist
\item
  \textbf{સ્ટ્રક્ચર}: Peer-to-peer ટોપોલોજી
\item
  \textbf{રૂટિંગ}: ડાયનેમિક રૂટ ડિસ્કવરી
\item
  \textbf{એપ્લિકેશનો}: ઇમર્જન્સી, મિલિટરી
\end{itemize}

\end{solutionbox}
\begin{mnemonicbox}
``Ad-hoc-Self-Organizing''

\end{mnemonicbox}
\begin{center}\rule{0.5\linewidth}{0.5pt}\end{center}

\subsection*{પ્રશ્ન 3(બ OR) [4
ગુણ]}\label{uxaaauxab0uxab6uxaa8-3uxaac-or-4-uxa97uxaa3}

\textbf{વ્યાખ્યાયિત કરો : Transaction Processing Middleware in mobile
computing}

\begin{solutionbox}
ટ્રાન્ઝેક્શન પ્રોસેસિંગ મિડલવેર મોબાઇલ ડેટાબેસ ટ્રાન્ઝેક્શનોમાં ACID
પ્રાપર્ટીઓ સુનિશ્ચિત કરે છે.

\textbf{ટેબલ:}

{\def\LTcaptype{none} % do not increment counter
\begin{longtable}[]{@{}ll@{}}
\toprule\noalign{}
પ્રાપર્ટી & વર્ણન \\
\midrule\noalign{}
\endhead
\bottomrule\noalign{}
\endlastfoot
\textbf{Atomicity} & સર્વ અથવા કંઈ નહીં એક્ઝિક્યુશન \\
\textbf{Consistency} & ડેટાબેસ અખંડિતતા જાળવાય \\
\textbf{Isolation} & સમાંતર ટ્રાન્ઝેક્શન અલગીકરણ \\
\textbf{Durability} & કાયમી ટ્રાન્ઝેક્શન અસરો \\
\end{longtable}
}

\end{solutionbox}
\begin{mnemonicbox}
``ACID-Properties''

\end{mnemonicbox}
\begin{center}\rule{0.5\linewidth}{0.5pt}\end{center}

\subsection*{પ્રશ્ન 3(ક OR) [7
ગુણ]}\label{uxaaauxab0uxab6uxaa8-3uxa95-or-7-uxa97uxaa3}

\textbf{મોબાઇલ કમ્પ્યુટિંગની એપ્લિકેશન અને સેવાઓની ચર્ચા કરો}

\begin{solutionbox}
મોબાઇલ કમ્પ્યુટિંગ મલ્ટિપલ ડોમેન્સમાં વિવિધ એપ્લિકેશનોને સક્ષમ બનાવે
છે.

\textbf{ટેબલ:}

{\def\LTcaptype{none} % do not increment counter
\begin{longtable}[]{@{}lll@{}}
\toprule\noalign{}
ડોમેન & એપ્લિકેશનો & સેવાઓ \\
\midrule\noalign{}
\endhead
\bottomrule\noalign{}
\endlastfoot
\textbf{બિઝનેસ} & CRM, ERP & ડેટા સિંક્રોનાઇઝેશન \\
\textbf{હેલ્થકેર} & પેશન્ટ મોનિટરિંગ & રિમોટ ડાયગ્નોસિસ \\
\textbf{એજ્યુકેશન} & E-learning & કન્ટેન્ટ ડિલિવરી \\
\textbf{એન્ટરટેઈનમેન્ટ} & ગેમિંગ, સ્ટ્રીમિંગ & મીડિયા સેવાઓ \\
\textbf{નેવિગેશન} & GPS, મેપ્સ & લોકેશન સેવાઓ \\
\end{longtable}
}

\textbf{મુખ્ય મુદ્દાઓ:}

\begin{itemize}
\tightlist
\item
  \textbf{લોકેશન-આધારિત}: GPS નેવિગેશન, જિયો-ફેન્સિંગ
\item
  \textbf{કમ્યુનિકેશન}: ઇમેઇલ, મેસેજિંગ, વિડિયો કોલ્સ
\item
  \textbf{કોમર્સ}: મોબાઇલ બેંકિંગ, શોપિંગ
\end{itemize}

\end{solutionbox}
\begin{mnemonicbox}
``Business-Health-Education-Entertainment''

\end{mnemonicbox}
\begin{center}\rule{0.5\linewidth}{0.5pt}\end{center}

\subsection*{પ્રશ્ન 4(અ) [3
ગુણ]}\label{uxaaauxab0uxab6uxaa8-4uxa85-3-uxa97uxaa3}

\textbf{વર્ણન કરો : Indirect TCP in mobile computing}

\begin{solutionbox}
Indirect TCP મોબાઇલ હોસ્ટ મોબિલિટી કાર્યક્ષમ રીતે હેન્ડલ કરવા
માટે TCP કનેક્શન સ્પ્લિટ કરે છે.

\textbf{ડાયાગ્રામ:}

\begin{verbatim}
Fixed Host {-{-} Base Station {-}{-} Mobile Host}
    TCP1          TCP2
\end{verbatim}

\textbf{મુખ્ય મુદ્દાઓ:}

\begin{itemize}
\tightlist
\item
  \textbf{સ્પ્લિટ કનેક્શન}: બે અલગ TCP કનેક્શનો
\item
  \textbf{બેસ સ્ટેશન}: પ્રોક્સી તરીકે કામ કરે
\item
  \textbf{ફાયદો}: ઝડપી હેન્ડઓફ
\end{itemize}

\end{solutionbox}
\begin{mnemonicbox}
``Indirect-Split-Proxy''

\end{mnemonicbox}
\begin{center}\rule{0.5\linewidth}{0.5pt}\end{center}

\subsection*{પ્રશ્ન 4(બ) [4
ગુણ]}\label{uxaaauxab0uxab6uxaa8-4uxaac-4-uxa97uxaa3}

\textbf{મોબાઈલ આઈપીમાં પેકેટ ડિલિવરીના સ્ટેપ્સ સમજાવો}

\begin{solutionbox}
મોબાઇલ IP પેકેટ ડિલિવરીમાં રજિસ્ટ્રેશન, ટનલિંગ અને ડિલિવરી સ્ટેપ્સ
સામેલ છે.

\textbf{સ્ટેપ્સ:}

\begin{enumerate}
\tightlist
\item
  \textbf{રજિસ્ટ્રેશન}: મોબાઇલ નોડ હોમ એજન્ટ સાથે રજિસ્ટર કરે
\item
  \textbf{ટનલિંગ}: હોમ એજન્ટ ફોરેન એજન્ટ માટે ટનલ બનાવે
\item
  \textbf{એન્કેપ્સુલેશન}: મૂળ પેકેટ નવા હેડરમાં લપેટાય
\item
  \textbf{ડિલિવરી}: ફોરેન એજન્ટ મોબાઇલ નોડને ડિલિવર કરે
\end{enumerate}

\end{solutionbox}
\begin{mnemonicbox}
``Register-Tunnel-Encapsulate-Deliver''

\end{mnemonicbox}
\begin{center}\rule{0.5\linewidth}{0.5pt}\end{center}

\subsection*{પ્રશ્ન 4(ક) [7
ગુણ]}\label{uxaaauxab0uxab6uxaa8-4uxa95-7-uxa97uxaa3}

\textbf{મોબાઇલ આઈપી ની નીચેની ત્રણ પ્રક્રિયાઓ લખો: (1) Registration (2)
Tunneling (3) Encapsulation}

\begin{solutionbox}

\textbf{1. Registration પ્રક્રિયા:}

\begin{itemize}
\tightlist
\item
  મોબાઇલ નોડ ફોરેન એજન્ટ શોધે
\item
  હોમ એજન્ટ સાથે care-of address રજિસ્ટર કરે
\item
  ઓથેન્ટિકેશન અને બાઇન્ડિંગ અપડેટ
\end{itemize}

\textbf{2. Tunneling પ્રક્રિયા:}

\begin{itemize}
\tightlist
\item
  હોમ એજન્ટ વર્ચ્યુઅલ ટનલ બનાવે
\item
  ટનલ દ્વારા પેકેટ્સ ફોરવર્ડ કરાય
\item
  End-to-end કનેક્ટિવિટી જાળવે
\end{itemize}

\textbf{3. Encapsulation પ્રક્રિયા:}

\begin{itemize}
\tightlist
\item
  મૂળ પેકેટ પેલોડ બને
\item
  Care-of address સાથે નવો IP હેડર ઉમેરાય
\item
  પેકેટ ફોરેન નેટવર્કમાં ડિલિવર થાય
\end{itemize}

\textbf{ડાયાગ્રામ:}

\begin{center}
\textbf{Mermaid Diagram (Code)}
\begin{verbatim}
{Shaded}
{Highlighting}[]
graph LR
    A[Original Packet] {-{-}{} B[Encapsulation]}
    B {-{-}{} C[Tunneled Packet]}
    C {-{-}{} D[Delivery]}
{Highlighting}
{Shaded}
\end{verbatim}
\end{center}

\textbf{મુખ્ય મુદ્દાઓ:}

\begin{itemize}
\tightlist
\item
  \textbf{Registration}: લોકેશન અપડેટ મેકેનિઝમ
\item
  \textbf{Tunneling}: વર્ચ્યુઅલ કનેક્શન સ્થાપના
\item
  \textbf{Encapsulation}: પેકેટ રેપિંગ તકનીક
\end{itemize}

\end{solutionbox}
\begin{mnemonicbox}
``Register-Tunnel-Encapsulate''

\end{mnemonicbox}
\begin{center}\rule{0.5\linewidth}{0.5pt}\end{center}

\subsection*{પ્રશ્ન 4(અ OR) [3
ગુણ]}\label{uxaaauxab0uxab6uxaa8-4uxa85-or-3-uxa97uxaa3}

\textbf{વર્ણન કરો : Snooping TCP in mobile computing}

\begin{solutionbox}
Snooping TCP બેસ સ્ટેશન પર TCP સેગમેન્ટ્સ કેશ અને મોનિટર કરીને
પર્ફોર્મન્સ સુધારે છે.

\textbf{મુખ્ય મુદ્દાઓ:}

\begin{itemize}
\tightlist
\item
  \textbf{લોકલ રિટ્રાન્સમિશન}: બેસ સ્ટેશન લોસેસ હેન્ડલ કરે
\item
  \textbf{બફર મેનેજમેન્ટ}: અનએકનોલેજ્ડ સેગમેન્ટ્સ કેશ કરે
\item
  \textbf{પારદર્શિતા}: End-to-end TCP જાળવાય
\end{itemize}

\end{solutionbox}
\begin{mnemonicbox}
``Snoop-Cache-Retransmit''

\end{mnemonicbox}
\begin{center}\rule{0.5\linewidth}{0.5pt}\end{center}

\subsection*{પ્રશ્ન 4(બ OR) [4
ગુણ]}\label{uxaaauxab0uxab6uxaa8-4uxaac-or-4-uxa97uxaa3}

\textbf{મોબાઈલ આઈપીમાં હેન્ડઓવર મેનેજમેન્ટ સમજાવો}

\begin{solutionbox}
હેન્ડઓવર મેનેજમેન્ટ જ્યારે મોબાઇલ નોડ નેટવર્ક બદલે છે ત્યારે કનેક્ટિવિટી
જાળવે છે.

\textbf{ટેબલ:}

{\def\LTcaptype{none} % do not increment counter
\begin{longtable}[]{@{}ll@{}}
\toprule\noalign{}
તબક્કો & પ્રક્રિયા \\
\midrule\noalign{}
\endhead
\bottomrule\noalign{}
\endlastfoot
\textbf{ડિસ્કવરી} & નવો ફોરેન એજન્ટ શોધો \\
\textbf{રજિસ્ટ્રેશન} & Care-of address અપડેટ કરો \\
\textbf{ડેટા ફોરવર્ડિંગ} & પેકેટ્સ રીડાયરેક્ટ કરો \\
\textbf{ક્લીનઅપ} & જૂના રિસોર્સ રિલીઝ કરો \\
\end{longtable}
}

\end{solutionbox}
\begin{mnemonicbox}
``Discover-Register-Forward-Cleanup''

\end{mnemonicbox}
\begin{center}\rule{0.5\linewidth}{0.5pt}\end{center}

\subsection*{પ્રશ્ન 4(ક OR) [7
ગુણ]}\label{uxaaauxab0uxab6uxaa8-4uxa95-or-7-uxa97uxaa3}

\textbf{મોબાઇલ આઈપી માટે લક્ષ્યો અને જરૂરિયાતો લખો}

\begin{solutionbox}

\textbf{લક્ષ્યો:}

\begin{itemize}
\tightlist
\item
  \textbf{પારદર્શિતા}: એપ્લિકેશનો માટે સીમલેસ મોબિલિટી
\item
  \textbf{સુસંગતતા}: હાલના ઇન્ટરનેટ પ્રોટોકોલ્સ સાથે કામ
\item
  \textbf{સ્કેલેબિલિટી}: મોટી સંખ્યામાં મોબાઇલ નોડ્સ સપોર્ટ
\item
  \textbf{સિક્યોરિટી}: મોબાઇલ નોડ્સ ઓથેન્ટિકેટ અને ડેટા પ્રોટેક્ટ
\end{itemize}

\textbf{જરૂરિયાતો:}

\begin{itemize}
\tightlist
\item
  \textbf{હોમ એજન્ટ}: મોબાઇલ નોડ લોકેશન જાળવે
\item
  \textbf{ફોરેન એજન્ટ}: લોકલ સેવાઓ પ્રદાન કરે
\item
  \textbf{Care-of Address}: ફોરેન નેટવર્કમાં ટેમ્પરરી એડ્રેસ
\item
  \textbf{ટનલિંગ}: પેકેટ ફોરવર્ડિંગ મેકેનિઝમ
\end{itemize}

\textbf{ટેબલ:}

{\def\LTcaptype{none} % do not increment counter
\begin{longtable}[]{@{}lll@{}}
\toprule\noalign{}
પાસાં & લક્ષ્યો & જરૂરિયાતો \\
\midrule\noalign{}
\endhead
\bottomrule\noalign{}
\endlastfoot
\textbf{મોબિલિટી} & સીમલેસ મૂવમેન્ટ & Care-of address \\
\textbf{કનેક્ટિવિટી} & સેશન જાળવો & ટનલિંગ \\
\textbf{પર્ફોર્મન્સ} & ન્યૂનતમ ઓવરહેડ & કાર્યક્ષમ રૂટિંગ \\
\textbf{સિક્યોરિટી} & ઓથેન્ટિકેશન & સિક્યોર પ્રોટોકોલ્સ \\
\end{longtable}
}

\end{solutionbox}
\begin{mnemonicbox}
``Transparent-Compatible-Scalable-Secure''

\end{mnemonicbox}
\begin{center}\rule{0.5\linewidth}{0.5pt}\end{center}

\subsection*{પ્રશ્ન 5(અ) [3
ગુણ]}\label{uxaaauxab0uxab6uxaa8-5uxa85-3-uxa97uxaa3}

\textbf{મોબાઇલ નેટવર્કમાં 6G ની વિશેષતાઓ લખો}

\begin{solutionbox}
6G એડવાન્સ્ડ ક્ષમતાઓ સાથે મોબાઇલ નેટવર્ક્સની આવતી પેઢીનું
પ્રતિનિધિત્વ કરે છે.

\textbf{મુખ્ય મુદ્દાઓ:}

\begin{itemize}
\tightlist
\item
  \textbf{સ્પીડ}: 1 Tbps થિયોરેટિકલ સ્પીડ
\item
  \textbf{લેટેન્સી}: સબ-મિલિસેકન્ડ લેટેન્સી
\item
  \textbf{AI ઇન્ટિગ્રેશન}: નેટિવ આર્ટિફિશિયલ ઇન્ટેલિજન્સ
\end{itemize}

\end{solutionbox}
\begin{mnemonicbox}
``Tera-Speed-AI-Integration''

\end{mnemonicbox}
\begin{center}\rule{0.5\linewidth}{0.5pt}\end{center}

\subsection*{પ્રશ્ન 5(બ) [4
ગુણ]}\label{uxaaauxab0uxab6uxaa8-5uxaac-4-uxa97uxaa3}

\textbf{વર્ણન કરો : Dynamic Host Configuration Protocol (DHCP)}

\begin{solutionbox}
DHCP ડિવાઇસને IP એડ્રેસ અને નેટવર્ક કન્ફિગરેશન આપોઆપ એસાઇન કરે છે.

\textbf{ટેબલ:}

{\def\LTcaptype{none} % do not increment counter
\begin{longtable}[]{@{}ll@{}}
\toprule\noalign{}
પ્રક્રિયા & વર્ણન \\
\midrule\noalign{}
\endhead
\bottomrule\noalign{}
\endlastfoot
\textbf{Discover} & ક્લાયન્ટ બ્રોડકાસ્ટ રિક્વેસ્ટ \\
\textbf{Offer} & સર્વર IP એડ્રેસ ઓફર કરે \\
\textbf{Request} & ક્લાયન્ટ ચોક્કસ IP રિક્વેસ્ટ કરે \\
\textbf{Acknowledge} & સર્વર એસાઇનમેન્ટ કન્ફર્મ કરે \\
\end{longtable}
}

\end{solutionbox}
\begin{mnemonicbox}
``Discover-Offer-Request-Acknowledge''

\end{mnemonicbox}
\begin{center}\rule{0.5\linewidth}{0.5pt}\end{center}

\subsection*{પ્રશ્ન 5(ક) [7
ગુણ]}\label{uxaaauxab0uxab6uxaa8-5uxa95-7-uxa97uxaa3}

\textbf{વર્ણન કરો : architecture of Wireless Personal Area Network
(WLAN)}

\begin{solutionbox}
WLAN આર્કિટેક્ચર IEEE 802.11 સ્ટાન્ડર્ડ્સનો ઉપયોગ કરીને લોકલ
એરિયાની અંદર વાયરલેસ કનેક્ટિવિટી પ્રદાન કરે છે.

\textbf{ડાયાગ્રામ:}

\begin{verbatim}
graph TB
    A[Access Point] {-{-} B[Distribution System]}
    A {-{-} C[Station 1]}
    A {-{-} D[Station 2] }
    A {-{-} E[Station 3]}
    B {-{-} F[Internet/WAN]}
\end{verbatim}

\textbf{ટેબલ:}

{\def\LTcaptype{none} % do not increment counter
\begin{longtable}[]{@{}ll@{}}
\toprule\noalign{}
ઘટક & કાર્ય \\
\midrule\noalign{}
\endhead
\bottomrule\noalign{}
\endlastfoot
\textbf{Access Point} & કેન્દ્રીય વાયરલેસ હબ \\
\textbf{Station} & વાયરલેસ ક્લાયન્ટ ડિવાઇસ \\
\textbf{Distribution System} & બેકબોન નેટવર્ક \\
\textbf{BSS} & બેસિક સર્વિસ સેટ \\
\textbf{ESS} & એક્સટેન્ડેડ સર્વિસ સેટ \\
\end{longtable}
}

\textbf{મુખ્ય મુદ્દાઓ:}

\begin{itemize}
\tightlist
\item
  \textbf{ઇન્ફ્રાસ્ટ્રક્ચર મોડ}: એક્સેસ પોઇન્ટ્સનો ઉપયોગ
\item
  \textbf{Ad-hoc મોડ}: સીધા ડિવાઇસ કમ્યુનિકેશન
\item
  \textbf{સ્ટાન્ડર્ડ્સ}: 802.11a/b/g/n/ac/ax પ્રોટોકોલ્સ
\end{itemize}

\end{solutionbox}
\begin{mnemonicbox}
``Access-Station-Distribution''

\end{mnemonicbox}
\begin{center}\rule{0.5\linewidth}{0.5pt}\end{center}

\subsection*{પ્રશ્ન 5(અ OR) [3
ગુણ]}\label{uxaaauxab0uxab6uxaa8-5uxa85-or-3-uxa97uxaa3}

\textbf{મોબાઇલ નેટવર્કમાં 5G ની વિશેષતાઓ લખો}

\begin{solutionbox}
5G અલ્ટ્રા-લો લેટેન્સી સાથે એન્હાન્સ્ડ મોબાઇલ બ્રોડબેન્ડ પ્રદાન કરે છે.

\textbf{મુખ્ય મુદ્દાઓ:}

\begin{itemize}
\tightlist
\item
  \textbf{સ્પીડ}: 10 Gbps સુધી ડાઉનલોડ
\item
  \textbf{લેટેન્સી}: 1ms અલ્ટ્રા-લો લેટેન્સી
\item
  \textbf{ડેન્સિટી}: પ્રતિ km^{2} 1 મિલિયન ડિવાઇસ
\end{itemize}

\end{solutionbox}
\begin{mnemonicbox}
``10G-1ms-1Million''

\end{mnemonicbox}
\begin{center}\rule{0.5\linewidth}{0.5pt}\end{center}

\subsection*{પ્રશ્ન 5(બ OR) [4
ગુણ]}\label{uxaaauxab0uxab6uxaa8-5uxaac-or-4-uxa97uxaa3}

\textbf{WWW અને HTTP સમજાવો}

\begin{solutionbox}
વર્લ્ડ વાઇડ વેબ વેબ પેજ કમ્યુનિકેશન માટે HTTP પ્રોટોકોલનો ઉપયોગ કરે
છે.

\textbf{ટેબલ:}

{\def\LTcaptype{none} % do not increment counter
\begin{longtable}[]{@{}lll@{}}
\toprule\noalign{}
પાસાં & WWW & HTTP \\
\midrule\noalign{}
\endhead
\bottomrule\noalign{}
\endlastfoot
\textbf{હેતુ} & માહિતી શેરિંગ & કમ્યુનિકેશન પ્રોટોકોલ \\
\textbf{ઘટકો} & વેબ પેજીસ, બ્રાઉઝર્સ & Request/response \\
\textbf{ફોર્મેટ} & HTML ડોક્યુમેન્ટ્સ & ટેક્સ્ટ-આધારિત પ્રોટોકોલ \\
\textbf{પોર્ટ} & વિવિધ & 80, 443 \\
\end{longtable}
}

\end{solutionbox}
\begin{mnemonicbox}
``Web-Hypertext-Transfer''

\end{mnemonicbox}
\begin{center}\rule{0.5\linewidth}{0.5pt}\end{center}

\subsection*{પ્રશ્ન 5(ક OR) [7
ગુણ]}\label{uxaaauxab0uxab6uxaa8-5uxa95-or-7-uxa97uxaa3}

\textbf{બ્લૂટૂથના આર્કિટેક્ચરનું વર્ણન કરો}

\begin{solutionbox}
બ્લૂટૂથ આર્કિટેક્ચર પ્રોટોકોલ સ્ટેકનો ઉપયોગ કરીને શોર્ટ-રેન્જ વાયરલેસ
કમ્યુનિકેશન પ્રદાન કરે છે.

\textbf{ડાયાગ્રામ:}

\begin{center}
\textbf{Mermaid Diagram (Code)}
\begin{verbatim}
{Shaded}
{Highlighting}[]
graph LR
    A[Application Layer] {-{-}{} B[OBEX/SDP]}
    B {-{-}{} C[L2CAP]}
    C {-{-}{} D[HCI]}
    D {-{-}{} E[Link Manager]}
    E {-{-}{} F[Baseband]}
    F {-{-}{} G[Radio Layer]}
{Highlighting}
{Shaded}
\end{verbatim}
\end{center}

\textbf{ટેબલ:}

{\def\LTcaptype{none} % do not increment counter
\begin{longtable}[]{@{}ll@{}}
\toprule\noalign{}
લેયર & કાર્ય \\
\midrule\noalign{}
\endhead
\bottomrule\noalign{}
\endlastfoot
\textbf{Radio} & ફિઝિકલ ટ્રાન્સમિશન \\
\textbf{Baseband} & ટાઇમિંગ અને ફ્રીક્વન્સી હોપિંગ \\
\textbf{Link Manager} & કનેક્શન મેનેજમેન્ટ \\
\textbf{HCI} & હોસ્ટ કંટ્રોલર ઇન્ટરફેસ \\
\textbf{L2CAP} & લોજિકલ લિંક કંટ્રોલ \\
\textbf{Applications} & યુઝર સેવાઓ \\
\end{longtable}
}

\textbf{મુખ્ય મુદ્દાઓ:}

\begin{itemize}
\tightlist
\item
  \textbf{Piconet}: માસ્ટર-સ્લેવ નેટવર્ક ટોપોલોજી
\item
  \textbf{Frequency Hopping}: 79 ફ્રીક્વન્સી ચેનલ્સ
\item
  \textbf{Power Classes}: વિવિધ ટ્રાન્સમિશન રેન્જીસ
\end{itemize}

\end{solutionbox}
\begin{mnemonicbox}
``Radio-Baseband-Link-Host-Logic''

\end{mnemonicbox}

\end{document}
