\documentclass{article}
% Adjust the relative path to point to the latex-templates directory

% content/resources/templates/preamble.tex
\usepackage[margin=0.6in]{geometry}
\author{Milav Dabgar}
\usepackage{amsmath,amssymb,amsthm}
\usepackage{booktabs}
\usepackage{multirow}
\usepackage{xcolor}
\usepackage{tcolorbox}
\tcbuselibrary{breakable,skins}
\usepackage[colorlinks=true,linkcolor=blue]{hyperref}
\usepackage{titlesec}
\usepackage{enumitem}
\usepackage{tikz}
\usepackage{pgfplots}
\usepackage{circuitikz}
\usepackage[version=4]{mhchem}
\usepackage{longtable}
\usepackage{array}
\usepackage{float}
\usepackage{caption}
\usepackage{listings}

\lstset{
  basicstyle=\small\ttfamily,
  breaklines=true,
  breakatwhitespace=false,
  postbreak=\mbox{\textcolor{red}{$\hookrightarrow$}\space},
  float=false,
  numbers=left,
  numberstyle=\tiny\color{gray},
  numbersep=10pt,
  xleftmargin=2em,
  keywordstyle=\color{blue},
  commentstyle=\color{green!60!black},
  stringstyle=\color{purple},
  backgroundcolor=\color{gray!5},
  showstringspaces=false,
  tabsize=2,
  captionpos=b,
  keepspaces=true,
  columns=flexible
}

\pgfplotsset{compat=1.18}
\usetikzlibrary{shapes,arrows,positioning,calc,patterns,decorations.pathmorphing,decorations.markings,arrows.meta}

% Color scheme
\definecolor{headcolor}{RGB}{0,102,204}
\definecolor{keycolor}{RGB}{220,20,60}
\definecolor{solutioncolor}{RGB}{34,139,34}
\definecolor{mnemoniccolor}{RGB}{148,0,211}
\definecolor{codecolor}{RGB}{0,0,100}

% Spacing
\setlength{\parskip}{3pt}
\setlist[itemize]{nosep}
\setlist[enumerate]{nosep}

% Title formatting
\titleformat{\section}{\Large\bfseries\color{headcolor}}{\thesection}{1em}{}
\titleformat{\subsection}{\large\bfseries\color{headcolor}}{\thesubsection}{1em}{}

% Pandoc tightlist compatibility
\providecommand{\tightlist}{%
  \setlength{\itemsep}{0pt}\setlength{\parskip}{0pt}}

% Pandoc longtable compatibility
\newcounter{none}
\def\thenone{}


% content/resources/templates/english-boxes.tex
% This file is currently empty - it exists to maintain consistency with the import structure.
% Add custom environments here if needed in the future.


% Custom commands for GTU solutions
% This file defines semantic commands for consistent formatting

% Question command with automatic formatting
\newcommand{\question}[2]{%
  \section*{Question #1}%
  \textbf{#2}%
}

% OR question variant
\newcommand{\questionor}[2]{%
  \section*{Question #1 OR}%
  \textbf{#2}%
}

% Proper table environment with caption
\newenvironment{answertable}[1]{%
  \begin{table}[htbp]
  \centering
  \caption{#1}
}{%
  \end{table}
}

% Proper figure environment for diagrams
\newenvironment{answerdiagram}[1]{%
  \begin{figure}[htbp]
  \centering
  \caption{#1}
}{%
  \end{figure}
}

% Semantic markup for key terms
\newcommand{\keyword}[1]{\textbf{#1}}
\newcommand{\code}[1]{\texttt{#1}}
\newcommand{\classname}[1]{\texttt{#1}}
\newcommand{\methodname}[1]{\texttt{#1}}

% Proper quotation marks
\newcommand{\mnemonic}[1]{``#1''}


\title{Mobile Computing and Networks (4351602) - Winter 2024 Solution}
\date{November 25, 2024}

\begin{document}
\maketitle

\questionmarks{1(a)}{3}{List out types of congestion control and explain any one}

\begin{solutionbox}
\textbf{Types of Congestion Control:}
\begin{center}
\captionof{table}{Congestion Control Types}
\begin{tabulary}{\linewidth}{|L|L|}
\hline
\textbf{Type} & \textbf{Description} \\ \hline
\textbf{Open-Loop} & Prevents congestion before it occurs \\ \hline
\textbf{Closed-Loop} & Manages congestion after detection \\ \hline
\end{tabulary}
\end{center}

\textbf{Open-Loop Congestion Control Explanation:}
\begin{itemize}
    \item \keyword{Prevention approach}: Takes action before congestion occurs
    \item \keyword{Traffic shaping}: Controls data rate at sender
    \item \keyword{Admission control}: Limits new connections during high traffic
    \item \keyword{Load shedding}: Drops packets when buffer full
\end{itemize}
\end{solutionbox}

\begin{mnemonicbox}
\mnemonic{Open Prevents Traffic Admission Load}
\end{mnemonicbox}

\questionmarks{1(b)}{4}{Explain Address Resolution Protocol briefly}

\begin{solutionbox}
\textbf{ARP (Address Resolution Protocol)} maps IP addresses to MAC addresses in local networks.

\textbf{Working Process:}
\begin{itemize}
    \item \keyword{ARP Request}: Broadcast message asking "Who has IP X?"
    \item \keyword{ARP Reply}: Target device responds with its MAC address
    \item \keyword{ARP Cache}: Stores IP-MAC mappings for future use
    \item \keyword{Dynamic mapping}: Updates entries automatically
\end{itemize}

\textbf{ARP Message Types:}
\begin{center}
\captionof{table}{ARP Message Types}
\begin{tabulary}{\linewidth}{|L|L|L|}
\hline
\textbf{Type} & \textbf{Purpose} & \textbf{Broadcast} \\ \hline
ARP Request & Find MAC address & Yes \\ \hline
ARP Reply & Provide MAC address & No \\ \hline
\end{tabulary}
\end{center}
\end{solutionbox}

\begin{mnemonicbox}
\mnemonic{ARP Requests Broadcast, Replies Cache Dynamic}
\end{mnemonicbox}

\questionmarks{1(c)}{7}{Explain TCP/IP model with all layers and functionalities of each layer}

\begin{solutionbox}
\textbf{TCP/IP Model} is a four-layer network protocol stack for internet communication.

\begin{center}
\begin{tikzpicture}[node distance=0.8cm, auto]
    \node [gtu block] (app) {Application Layer};
    \node [gtu block, below=of app] (trans) {Transport Layer};
    \node [gtu block, below=of trans] (net) {Internet Layer};
    \node [gtu block, below=of net] (link) {Network Access Layer};

    \draw [gtu arrow] (app) -- (trans);
    \draw [gtu arrow] (trans) -- (net);
    \draw [gtu arrow] (net) -- (link);
\end{tikzpicture}
\captionof{figure}{TCP/IP Model}
\end{center}

\textbf{Layer Functions:}
\begin{center}
\captionof{table}{Layer Functions}
\begin{tabulary}{\linewidth}{|L|L|L|}
\hline
\textbf{Layer} & \textbf{Function} & \textbf{Protocols} \\ \hline
\textbf{Application} & User interface, network services & HTTP, FTP, SMTP \\ \hline
\textbf{Transport} & End-to-end communication & TCP, UDP \\ \hline
\textbf{Internet} & Routing, addressing & IP, ICMP \\ \hline
\textbf{Network Access} & Physical transmission & Ethernet, WiFi \\ \hline
\end{tabulary}
\end{center}

\begin{itemize}
    \item \keyword{Application Layer}: Provides network services to applications
    \item \keyword{Transport Layer}: Ensures reliable data delivery with error control
    \item \keyword{Internet Layer}: Routes packets across networks using IP addressing
    \item \keyword{Network Access Layer}: Handles physical data transmission
\end{itemize}
\end{solutionbox}

\begin{mnemonicbox}
\mnemonic{All Transport Internet Network}
\end{mnemonicbox}

\questionmarks{1(c OR)}{7}{Explain OSI model with each layer functionality}

\begin{solutionbox}
\textbf{OSI Model} is a seven-layer reference model for network communication.

\begin{center}
\begin{tikzpicture}[node distance=0.6cm, auto]
    \node [gtu block] (app) {7. Application Layer};
    \node [gtu block, below=of app] (pres) {6. Presentation Layer};
    \node [gtu block, below=of pres] (sess) {5. Session Layer};
    \node [gtu block, below=of sess] (trans) {4. Transport Layer};
    \node [gtu block, below=of trans] (net) {3. Network Layer};
    \node [gtu block, below=of net] (data) {2. Data Link Layer};
    \node [gtu block, below=of data] (phys) {1. Physical Layer};

    \draw [gtu arrow] (app) -- (pres);
    \draw [gtu arrow] (pres) -- (sess);
    \draw [gtu arrow] (sess) -- (trans);
    \draw [gtu arrow] (trans) -- (net);
    \draw [gtu arrow] (net) -- (data);
    \draw [gtu arrow] (data) -- (phys);
\end{tikzpicture}
\captionof{figure}{OSI Model}
\end{center}

\textbf{Layer Functionalities:}
\begin{center}
\captionof{table}{Layer Functionalities}
\begin{tabulary}{\linewidth}{|L|L|L|}
\hline
\textbf{Layer} & \textbf{Function} & \textbf{Examples} \\ \hline
\textbf{Physical (1)} & Bit transmission & Cables, signals \\ \hline
\textbf{Data Link (2)} & Frame delivery & Ethernet, switches \\ \hline
\textbf{Network (3)} & Routing packets & IP, routers \\ \hline
\textbf{Transport (4)} & End-to-end delivery & TCP, UDP \\ \hline
\textbf{Session (5)} & Dialog management & NetBIOS \\ \hline
\textbf{Presentation (6)} & Data formatting & SSL, compression \\ \hline
\textbf{Application (7)} & User interface & HTTP, email \\ \hline
\end{tabulary}
\end{center}
\end{solutionbox}

\begin{mnemonicbox}
\mnemonic{Physical Data Network Transport Session Presentation Application}
\end{mnemonicbox}

\questionmarks{2(a)}{3}{Explain subnetting in short}

\begin{solutionbox}
\textbf{Subnetting} divides a large network into smaller sub-networks for better management.

\textbf{Key Concepts:}
\begin{itemize}
    \item \keyword{Subnet mask}: Defines network and host portions
    \item \keyword{Network efficiency}: Reduces broadcast traffic
    \item \keyword{Address conservation}: Better IP utilization
    \item \keyword{Security}: Isolates network segments
\end{itemize}

\textbf{Example:}
Network: 192.168.1.0/24 $\rightarrow$ Subnets: 192.168.1.0/26, 192.168.1.64/26
\end{solutionbox}

\begin{mnemonicbox}
\mnemonic{Subnet Network Efficiency Address Security}
\end{mnemonicbox}

\questionmarks{2(b)}{4}{Explain stop and wait ARQ protocol of data link layer with example}

\begin{solutionbox}
\textbf{Stop and Wait ARQ} is a flow control protocol ensuring reliable data transmission.

\textbf{Working Process:}
\begin{itemize}
    \item \keyword{Send frame}: Transmitter sends one frame
    \item \keyword{Wait for ACK}: Sender waits for acknowledgment
    \item \keyword{Timeout}: Retransmits if no ACK received
    \item \keyword{Next frame}: Sends next frame after ACK
\end{itemize}

\begin{center}
\begin{tikzpicture}[node distance=2.5cm, auto]
    \node [gtu state] (s) {Sender};
    \node [gtu state, right=of s] (r) {Receiver};

    \draw [gtu arrow] (s) -- node [above] {Frame 1} (r);
    \draw [gtu arrow] (r) to[bend left=30] node [below] {ACK} (s);
    \draw [gtu arrow] (s) to[bend right=30] node [above] {Frame 2} (r);
\end{tikzpicture}
\captionof{figure}{Stop and Wait ARQ}
\end{center}

\textbf{Example:} File transfer where each packet waits for confirmation before sending next.
\end{solutionbox}

\begin{mnemonicbox}
\mnemonic{Send Wait Timeout Next}
\end{mnemonicbox}

\questionmarks{2(c)}{7}{Draw diagram of IPv4 datagram Header and explain it}

\begin{solutionbox}
\textbf{IPv4 Header} contains control information for packet routing and delivery.

\begin{center}
\begin{tikzpicture}[node distance=0cm, outer sep=0pt]
    \node [gtu block, minimum width=1cm] (ver) {Ver};
    \node [gtu block, minimum width=1cm, right=0cm of ver] (ihl) {IHL};
    \node [gtu block, minimum width=2cm, right=0cm of ihl] (tos) {Service};
    \node [gtu block, minimum width=4cm, right=0cm of tos] (len) {Total Length};
    
    \node [gtu block, minimum width=4cm, below=0cm of ver.south west, anchor=north west] (id) {Identification};
    \node [gtu block, minimum width=1cm, right=0cm of id] (flags) {Flg};
    \node [gtu block, minimum width=3cm, right=0cm of flags] (off) {Frag Offset};
    
    \node [gtu block, minimum width=2cm, below=0cm of id.south west, anchor=north west] (ttl) {TTL};
    \node [gtu block, minimum width=2cm, right=0cm of ttl] (proto) {Proto};
    \node [gtu block, minimum width=4cm, right=0cm of proto] (check) {Header Checksum};
    
    \node [gtu block, minimum width=8cm, below=0cm of ttl.south west, anchor=north west] (src) {Source Address};
    \node [gtu block, minimum width=8cm, below=0cm of src.south west, anchor=north west] (dst) {Destination Address};
\end{tikzpicture}
\captionof{figure}{IPv4 Header}
\end{center}

\textbf{Field Explanations:}
\begin{center}
\captionof{table}{Field Explanations}
\begin{tabulary}{\linewidth}{|L|L|L|}
\hline
\textbf{Field} & \textbf{Size} & \textbf{Function} \\ \hline
\textbf{Version} & 4 bits & IP version (4 for IPv4) \\ \hline
\textbf{IHL} & 4 bits & Header length \\ \hline
\textbf{Type of Service} & 8 bits & Quality of service \\ \hline
\textbf{Total Length} & 16 bits & Packet size \\ \hline
\textbf{TTL} & 8 bits & Hop limit \\ \hline
\textbf{Protocol} & 8 bits & Next layer protocol \\ \hline
\textbf{Source/Dest Address} & 32 bits each & IP addresses \\ \hline
\end{tabulary}
\end{center}
\end{solutionbox}

\begin{mnemonicbox}
\mnemonic{Version IHL Service Total TTL Protocol Source Destination}
\end{mnemonicbox}

\questionmarks{2(a OR)}{3}{What is HTTPS? List important key features of HTTPS}

\begin{solutionbox}
\textbf{HTTPS (HTTP Secure)} is encrypted HTTP using SSL/TLS for secure web communication.

\textbf{Key Features:}
\begin{itemize}
    \item \keyword{Encryption}: Data encrypted in transit
    \item \keyword{Authentication}: Verifies server identity
    \item \keyword{Data integrity}: Prevents data tampering
    \item \keyword{Trust}: SSL certificates provide validation
\end{itemize}

\textbf{Security Benefits:}
\begin{itemize}
    \item Protects sensitive information
    \item Prevents man-in-the-middle attacks
    \item Search engine ranking boost
\end{itemize}
\end{solutionbox}

\begin{mnemonicbox}
\mnemonic{HTTPS Encrypts Authentication Data Trust}
\end{mnemonicbox}

\questionmarks{2(b OR)}{4}{Give Answer of any two:}

\begin{solutionbox}
\textbf{1) How many bits HOST ID use by class B and C?}
\begin{itemize}
    \item \textbf{Class B}: 16 bits for Host ID (65,534 hosts)
    \item \textbf{Class C}: 8 bits for Host ID (254 hosts)
\end{itemize}

\textbf{2) What is IP range for Class A and D?}
\begin{itemize}
    \item \textbf{Class A}: 1.0.0.0 to 126.255.255.255
    \item \textbf{Class D}: 224.0.0.0 to 239.255.255.255 (Multicast)
\end{itemize}

\begin{center}
\captionof{table}{IP Classes}
\begin{tabulary}{\linewidth}{|C|L|L|}
\hline
\textbf{Class} & \textbf{Range} & \textbf{Host Bits} \\ \hline
B & 128.0.0.0 - 191.255.255.255 & 16 bits \\ \hline
C & 192.0.0.0 - 223.255.255.255 & 8 bits \\ \hline
A & 1.0.0.0 - 126.255.255.255 & 24 bits \\ \hline
D & 224.0.0.0 - 239.255.255.255 & Multicast \\ \hline
\end{tabulary}
\end{center}
\end{solutionbox}

\begin{mnemonicbox}
\mnemonic{B=16, C=8, A=1-126, D=224-239}
\end{mnemonicbox}

\questionmarks{2(c OR)}{7}{Explain classful IPv4 addresses scheme}

\begin{solutionbox}
\textbf{Classful IPv4 Addressing} divides IP address space into five classes based on first octets.

\textbf{Address Classes:}
\begin{center}
\captionof{table}{Address Classes}
\begin{tabulary}{\linewidth}{|L|L|L|L|L|}
\hline
\textbf{Class} & \textbf{Range} & \textbf{Network Bits} & \textbf{Host Bits} & \textbf{Usage} \\ \hline
\textbf{A} & 1-126 & 8 & 24 & Large networks \\ \hline
\textbf{B} & 128-191 & 16 & 16 & Medium networks \\ \hline
\textbf{C} & 192-223 & 24 & 8 & Small networks \\ \hline
\textbf{D} & 224-239 & - & - & Multicast \\ \hline
\textbf{E} & 240-255 & - & - & Experimental \\ \hline
\end{tabulary}
\end{center}

\begin{center}
\begin{tikzpicture}
    \pie[text=legend, radius=2]{50/Class A, 25/Class B, 12.5/Class C, 6.25/Class D, 6.25/Class E}
\end{tikzpicture}
\captionof{figure}{IPv4 Address Classes}
\end{center}

\textbf{Characteristics:}
\begin{itemize}
    \item \keyword{Class A}: 16.7 million hosts per network
    \item \keyword{Class B}: 65,534 hosts per network
    \item \keyword{Class C}: 254 hosts per network
    \item \keyword{Limitations}: Address wastage, inflexible allocation
\end{itemize}
\end{solutionbox}

\begin{mnemonicbox}
\mnemonic{A-Large, B-Medium, C-Small, D-Multicast, E-Experimental}
\end{mnemonicbox}

\questionmarks{3(a)}{3}{List out types of applications uses mobile computing}

\begin{solutionbox}
\textbf{Mobile Computing Applications:}
\begin{center}
\captionof{table}{Applications}
\begin{tabulary}{\linewidth}{|L|L|}
\hline
\textbf{Type} & \textbf{Examples} \\ \hline
\textbf{Communication} & WhatsApp, Email, Video calls \\ \hline
\textbf{Navigation} & GPS, Google Maps \\ \hline
\textbf{E-commerce} & Shopping apps, Mobile banking \\ \hline
\textbf{Entertainment} & Games, Streaming, Social media \\ \hline
\textbf{Business} & CRM, Sales tracking \\ \hline
\textbf{Healthcare} & Health monitoring, Telemedicine \\ \hline
\end{tabulary}
\end{center}

\begin{itemize}
    \item \keyword{Location-based services}: GPS navigation, location sharing
    \item \keyword{Mobile payments}: Digital wallets, UPI transactions
    \item \keyword{Social networking}: Facebook, Instagram, Twitter
\end{itemize}
\end{solutionbox}

\begin{mnemonicbox}
\mnemonic{Communication Navigation E-commerce Entertainment Business Healthcare}
\end{mnemonicbox}

\questionmarks{3(b)}{4}{Explain use of Gateways and list types of Gateways}

\begin{solutionbox}
\textbf{Gateway} connects networks with different protocols and architectures.

\textbf{Uses of Gateways:}
\begin{itemize}
    \item \keyword{Protocol conversion}: Translates between different protocols
    \item \keyword{Network bridging}: Connects dissimilar networks
    \item \keyword{Security}: Firewall and access control
    \item \keyword{Data filtering}: Manages traffic flow
\end{itemize}

\textbf{Types of Gateways:}
\begin{center}
\captionof{table}{Types of Gateways}
\begin{tabulary}{\linewidth}{|L|L|}
\hline
\textbf{Type} & \textbf{Function} \\ \hline
\textbf{Network Gateway} & Routes between networks \\ \hline
\textbf{Internet Gateway} & Connects to internet \\ \hline
\textbf{Protocol Gateway} & Protocol translation \\ \hline
\textbf{Application Gateway} & Application-level filtering \\ \hline
\end{tabulary}
\end{center}
\end{solutionbox}

\begin{mnemonicbox}
\mnemonic{Gateways Convert Bridge Secure Filter}
\end{mnemonicbox}

\questionmarks{3(c)}{7}{Draw and explain architecture of mobile computing}

\begin{solutionbox}
\textbf{Mobile Computing Architecture} consists of three main components working together.

\begin{center}
\begin{tikzpicture}[node distance=1.5cm, auto]
    \node [gtu state, align=center] (mobile) {Mobile Device\\(Hardware, OS, Data)};
    \node [gtu block, right=of mobile, align=center] (network) {Communication Network\\(Wireless, Base Stations)};
    \node [gtu block, right=of network, align=center] (fixed) {Fixed Infrastructure\\(Servers, Internet)};

    \draw [gtu arrow] (mobile) -- (network);
    \draw [gtu arrow] (network) -- (mobile);
    \draw [gtu arrow] (network) -- (fixed);
    \draw [gtu arrow] (fixed) -- (network);
\end{tikzpicture}
\captionof{figure}{Mobile Computing Architecture}
\end{center}

\textbf{Architecture Components:}
\begin{center}
\captionof{table}{Components}
\begin{tabulary}{\linewidth}{|L|L|L|}
\hline
\textbf{Component} & \textbf{Elements} & \textbf{Function} \\ \hline
\textbf{Mobile Unit} & Devices, OS, Apps & User interface, processing \\ \hline
\textbf{Communication Network} & Wireless links, protocols & Data transmission \\ \hline
\textbf{Fixed Infrastructure} & Servers, databases & Backend services \\ \hline
\end{tabulary}
\end{center}

\textbf{Key Features:}
\begin{itemize}
    \item \keyword{Mobility}: Users can move while maintaining connectivity
    \item \keyword{Wireless communication}: Radio waves for data transmission
    \item \keyword{Distributed computing}: Processing across multiple devices
    \item \keyword{Location independence}: Access services from anywhere
\end{itemize}

\textbf{Challenges:}
\begin{itemize}
    \item \keyword{Limited bandwidth}: Wireless networks have capacity constraints
    \item \keyword{Battery life}: Mobile devices have power limitations
    \item \keyword{Security}: Wireless transmission vulnerable to attacks
\end{itemize}
\end{solutionbox}

\begin{mnemonicbox}
\mnemonic{Mobile Communication Fixed - Mobility Wireless Distributed Location}
\end{mnemonicbox}

\questionmarks{3(a OR)}{3}{List security standards in mobile computing}

\begin{solutionbox}
\textbf{Mobile Computing Security Standards:}

\begin{center}
\captionof{table}{Security Standards}
\begin{tabulary}{\linewidth}{|L|L|}
\hline
\textbf{Standard} & \textbf{Purpose} \\ \hline
\textbf{WPA3} & WiFi security protocol \\ \hline
\textbf{SSL/TLS} & Secure data transmission \\ \hline
\textbf{IPSec} & IP layer security \\ \hline
\textbf{EAP} & Authentication framework \\ \hline
\textbf{802.11i} & Wireless LAN security \\ \hline
\textbf{FIPS 140-2} & Cryptographic module standards \\ \hline
\end{tabulary}
\end{center}

\begin{itemize}
    \item \keyword{Authentication protocols}: Verify user identity
    \item \keyword{Encryption standards}: Protect data confidentiality
    \item \keyword{Access control}: Manage resource permissions
\end{itemize}
\end{solutionbox}

\begin{mnemonicbox}
\mnemonic{WPA SSL IPSec EAP 802.11i FIPS}
\end{mnemonicbox}

\questionmarks{3(b OR)}{4}{Explain key functions of communication gateway}

\begin{solutionbox}
\textbf{Communication Gateway} manages data exchange between different network systems.

\textbf{Key Functions:}
\begin{center}
\captionof{table}{Functions}
\begin{tabulary}{\linewidth}{|L|L|}
\hline
\textbf{Function} & \textbf{Description} \\ \hline
\textbf{Protocol Translation} & Converts between protocols \\ \hline
\textbf{Data Format Conversion} & Changes data formats \\ \hline
\textbf{Routing} & Directs messages to destinations \\ \hline
\textbf{Security} & Access control and filtering \\ \hline
\end{tabulary}
\end{center}

\textbf{Detailed Functions:}
\begin{itemize}
    \item \keyword{Message routing}: Determines optimal path for data
    \item \keyword{Error handling}: Manages transmission errors and recovery
    \item \keyword{Traffic management}: Controls data flow and congestion
    \item \keyword{Authentication}: Verifies sender and receiver identity
\end{itemize}

\textbf{Benefits:}
\begin{itemize}
    \item Enables interoperability between different systems
    \item Centralizes network management
    \item Provides security checkpoint
\end{itemize}
\end{solutionbox}

\begin{mnemonicbox}
\mnemonic{Protocol Data Routing Security - Message Error Traffic Authentication}
\end{mnemonicbox}

\questionmarks{3(c OR)}{7}{Explain use of middleware and list types of middleware}

\begin{solutionbox}
\textbf{Middleware} provides software layer between applications and operating system for distributed computing.

\textbf{Uses of Middleware:}
\begin{itemize}
    \item \keyword{Connectivity}: Links distributed applications
    \item \keyword{Interoperability}: Enables different systems to work together
    \item \keyword{Abstraction}: Hides complexity of underlying systems
    \item \keyword{Scalability}: Supports system growth and expansion
\end{itemize}

\begin{center}
\begin{tikzpicture}[node distance=1cm, auto]
    \node [gtu block, minimum width=4cm] (apps) {Applications};
    \node [gtu block, minimum width=4cm, below=of apps] (mid) {Middleware Layer};
    \node [gtu block, below left=of mid] (os) {OS};
    \node [gtu block, below=of mid] (net) {Network};
    \node [gtu block, below right=of mid] (db) {Database};

    \draw [gtu arrow] (apps) -- (mid);
    \draw [gtu arrow] (mid) -- (os);
    \draw [gtu arrow] (mid) -- (net);
    \draw [gtu arrow] (mid) -- (db);
\end{tikzpicture}
\captionof{figure}{Middleware Layer}
\end{center}

\textbf{Types of Middleware:}
\begin{center}
\captionof{table}{Types of Middleware}
\begin{tabulary}{\linewidth}{|L|L|L|}
\hline
\textbf{Type} & \textbf{Function} & \textbf{Examples} \\ \hline
\textbf{Message-Oriented} & Asynchronous communication & IBM MQ, RabbitMQ \\ \hline
\textbf{Remote Procedure Call} & Synchronous communication & gRPC, XML-RPC \\ \hline
\textbf{Object Request Broker} & Object communication & CORBA \\ \hline
\textbf{Database Middleware} & Database connectivity & ODBC, JDBC \\ \hline
\textbf{Transaction Processing} & Transaction management & Tuxedo \\ \hline
\textbf{Web Middleware} & Web services & Apache, IIS \\ \hline
\end{tabulary}
\end{center}

\textbf{Benefits:}
\begin{itemize}
    \item \keyword{Reduced complexity}: Simplifies application development
    \item \keyword{Reusability}: Common services for multiple applications
    \item \keyword{Maintainability}: Centralized management of services
    \item \keyword{Platform independence}: Works across different systems
\end{itemize}
\end{solutionbox}

\begin{mnemonicbox}
\mnemonic{Message RPC Object Database Transaction Web}
\end{mnemonicbox}

\questionmarks{4(a)}{3}{Explain working phases of Mobile IP}

\begin{solutionbox}
\textbf{Mobile IP Working Phases} enable seamless mobility for mobile devices across networks.

\textbf{Three Main Phases:}
\begin{center}
\captionof{table}{Phases}
\begin{tabulary}{\linewidth}{|L|L|}
\hline
\textbf{Phase} & \textbf{Function} \\ \hline
\textbf{Agent Discovery} & Find home/foreign agents \\ \hline
\textbf{Registration} & Register with foreign agent \\ \hline
\textbf{Tunneling} & Forward packets to mobile node \\ \hline
\end{tabulary}
\end{center}

\textbf{Phase Details:}
\begin{itemize}
    \item \keyword{Agent Discovery}: Mobile node detects available agents through advertisements
    \item \keyword{Registration}: Mobile node registers current location with home agent
    \item \keyword{Tunneling}: Home agent encapsulates and forwards packets to foreign agent
\end{itemize}
\end{solutionbox}

\begin{mnemonicbox}
\mnemonic{Agent Registration Tunneling}
\end{mnemonicbox}

\questionmarks{4(b)}{4}{Explain Handover management in Mobile IP}

\begin{solutionbox}
\textbf{Handover Management} maintains connectivity when mobile node moves between networks.

\textbf{Handover Process:}
\begin{itemize}
    \item \keyword{Movement detection}: Identifies change in network attachment
    \item \keyword{New agent discovery}: Finds new foreign agent
    \item \keyword{Registration update}: Updates location with home agent
    \item \keyword{Data forwarding}: Redirects traffic to new location
\end{itemize}

\textbf{Types of Handover:}
\begin{center}
\captionof{table}{Types of Handover}
\begin{tabulary}{\linewidth}{|L|L|}
\hline
\textbf{Type} & \textbf{Description} \\ \hline
\textbf{Hard Handover} & Break-before-make \\ \hline
\textbf{Soft Handover} & Make-before-break \\ \hline
\textbf{Horizontal} & Same technology \\ \hline
\textbf{Vertical} & Different technology \\ \hline
\end{tabulary}
\end{center}

\textbf{Challenges:}
\begin{itemize}
    \item \keyword{Packet loss}: During handover transition
    \item \keyword{Delay}: Registration and tunneling setup time
    \item \keyword{Resource management}: Efficient use of network resources
\end{itemize}
\end{solutionbox}

\begin{mnemonicbox}
\mnemonic{Movement Discovery Registration Forwarding}
\end{mnemonicbox}

\questionmarks{4(c)}{7}{Explain Registration and Tunneling in Mobile IP}

\begin{solutionbox}
\textbf{Registration and Tunneling} are core mechanisms enabling Mobile IP functionality.

\textbf{Registration Process:}
\begin{center}
\begin{tikzpicture}[node distance=2.5cm, auto]
    \node [gtu state] (mn) {Mobile Node};
    \node [gtu block, right=of mn] (fa) {Foreign Agent};
    \node [gtu block, right=of fa] (ha) {Home Agent};

    \draw [gtu arrow] (mn) -- node [above, font=\small] {Request} (fa);
    \draw [gtu arrow] (fa) -- node [above, font=\small] {Forward} (ha);
    \draw [gtu arrow] (ha) to[bend left=30] node [below, font=\small] {Reply} (fa);
    \draw [gtu arrow] (fa) to[bend left=30] node [below, font=\small] {Forward} (mn);
\end{tikzpicture}
\captionof{figure}{Registration Process}
\end{center}

\textbf{Registration Steps:}
\begin{itemize}
    \item \keyword{Request}: Mobile node sends registration request to foreign agent
    \item \keyword{Forward}: Foreign agent forwards request to home agent
    \item \keyword{Authentication}: Home agent verifies mobile node identity
    \item \keyword{Reply}: Home agent sends registration reply confirming registration
\end{itemize}

\textbf{Tunneling Mechanism:}
\begin{itemize}
    \item \keyword{Encapsulation}: Wraps original packet
    \item \keyword{Tunnel Endpoint}: Home and foreign agents
    \item \keyword{Decapsulation}: Unwraps packet at destination
    \item \keyword{Routing}: Directs traffic through tunnel
\end{itemize}
\end{solutionbox}

\begin{mnemonicbox}
\mnemonic{Registration Request Forward Authentication - Tunneling Encapsulation Transmission Decapsulation}
\end{mnemonicbox}

\questionmarks{4(a OR)}{3}{Explain snooping TCP}

\begin{solutionbox}
\textbf{Snooping TCP} improves TCP performance over wireless networks by handling wireless link errors.

\textbf{Working Mechanism:}
\begin{itemize}
    \item \keyword{Base station monitoring}: Observes TCP packets
    \item \keyword{Local retransmission}: Handles wireless link errors locally
    \item \keyword{Cache management}: Stores copies of transmitted packets
    \item \keyword{Error recovery}: Retransmits lost packets without involving sender
\end{itemize}

\textbf{Key Features:}
\begin{center}
\captionof{table}{Features}
\begin{tabulary}{\linewidth}{|L|L|}
\hline
\textbf{Feature} & \textbf{Benefit} \\ \hline
\textbf{Transparent} & No changes to TCP endpoints \\ \hline
\textbf{Local recovery} & Faster error correction \\ \hline
\textbf{Reduced timeouts} & Prevents unnecessary retransmissions \\ \hline
\end{tabulary}
\end{center}
\end{solutionbox}

\begin{mnemonicbox}
\mnemonic{Snooping Monitors Local Cache Recovery}
\end{mnemonicbox}

\questionmarks{4(b OR)}{4}{Explain Packet delivery in Mobile IP}

\begin{solutionbox}
\textbf{Packet Delivery in Mobile IP} ensures data reaches mobile nodes regardless of location.

\begin{center}
\begin{tikzpicture}[node distance=2cm, auto]
    \node [gtu state] (cn) {CN};
    \node [gtu block, right=of cn] (ha) {Home Agent};
    \node [gtu block, right=of ha] (fa) {Foreign Agent};
    \node [gtu state, below=of fa] (mn) {MN};

    \draw [gtu arrow] (cn) -- node [above] {1. Data} (ha);
    \draw [gtu arrow] (ha) -- node [above] {2. Tunnel} (fa);
    \draw [gtu arrow] (fa) -- node [right] {3. Deliver} (mn);
    \draw [gtu arrow] (mn) to[bend left=45] node [below left] {4. Direct Reply} (cn);
\end{tikzpicture}
\captionof{figure}{Packet Delivery}
\end{center}

\textbf{Delivery Scenarios:}
\begin{center}
\captionof{table}{Delivery Scenarios}
\begin{tabulary}{\linewidth}{|L|L|L|}
\hline
\textbf{Scenario} & \textbf{Path} & \textbf{Method} \\ \hline
\textbf{At Home} & Direct & Normal IP routing \\ \hline
\textbf{Away} & Via HA/FA & Tunneling \\ \hline
\textbf{Roaming} & Triangle routing & Indirect path \\ \hline
\end{tabulary}
\end{center}

\textbf{Packet Flow Steps:}
\begin{itemize}
    \item \keyword{Address resolution}: Determine mobile node location
    \item \keyword{Route selection}: Choose direct or tunneled delivery
    \item \keyword{Encapsulation}: Wrap packet if tunneling required
    \item \keyword{Forwarding}: Send to appropriate destination
    \item \keyword{Decapsulation}: Unwrap packet at foreign agent
\end{itemize}
\end{solutionbox}

\begin{mnemonicbox}
\mnemonic{Address Route Encapsulation Forward Decapsulation Delivery}
\end{mnemonicbox}

\questionmarks{4(c OR)}{7}{Describe how DHCP working with diagram}

\begin{solutionbox}
\textbf{DHCP (Dynamic Host Configuration Protocol)} automatically assigns IP addresses and network configuration to devices.

\textbf{DHCP Working Process:}
\begin{center}
\begin{tikzpicture}[node distance=3cm, auto]
    \node [gtu state] (client) {Client};
    \node [gtu block, right=of client] (server) {DHCP Server};

    \draw [gtu arrow] (client) -- node [above] {1. DISCOVER} (server);
    \draw [gtu arrow] (server) to[bend left=20] node [below] {2. OFFER} (client);
    \draw [gtu arrow] (client) to[bend left=20] node [above] {3. REQUEST} (server);
    \draw [gtu arrow] (server) to[bend left=40] node [below] {4. ACK} (client);
\end{tikzpicture}
\captionof{figure}{DHCP Process}
\end{center}

\textbf{Four-Step Process:}
\begin{center}
\captionof{table}{Process Steps}
\begin{tabulary}{\linewidth}{|C|L|L|}
\hline
\textbf{Step} & \textbf{Message} & \textbf{Function} \\ \hline
\textbf{1} & DISCOVER & Client broadcasts request for IP \\ \hline
\textbf{2} & OFFER & Server offers available IP address \\ \hline
\textbf{3} & REQUEST & Client requests specific IP address \\ \hline
\textbf{4} & ACK & Server confirms IP assignment \\ \hline
\end{tabulary}
\end{center}

\textbf{Configuration Information Provided:}
\begin{itemize}
    \item \keyword{IP Address}: Unique network identifier
    \item \keyword{Subnet Mask}: Network boundary definition
    \item \keyword{Default Gateway}: Route to other networks
    \item \keyword{DNS Servers}: Domain name resolution
    \item \keyword{Lease Time}: Duration of IP assignment
\end{itemize}

\textbf{Benefits:}
\begin{itemize}
    \item \keyword{Automatic configuration}: No manual IP assignment needed
    \item \keyword{Centralized management}: Single point for network configuration
    \item \keyword{Efficient utilization}: Dynamic allocation prevents waste
\end{itemize}
\end{solutionbox}

\begin{mnemonicbox}
\mnemonic{Discover Offer Request ACK - Server Client Relay Pool}
\end{mnemonicbox}

\questionmarks{5(a)}{3}{Give types of WLAN and explain any one}

\begin{solutionbox}
\textbf{WLAN Types:}
\begin{center}
\captionof{table}{WLAN Types}
\begin{tabulary}{\linewidth}{|L|L|L|}
\hline
\textbf{Type} & \textbf{Standard} & \textbf{Frequency} \\ \hline
\textbf{Infrastructure} & 802.11 & 2.4/5 GHz \\ \hline
\textbf{Ad-hoc} & IBSS & 2.4/5 GHz \\ \hline
\textbf{Mesh} & 802.11s & Multiple \\ \hline
\end{tabulary}
\end{center}

\textbf{Infrastructure WLAN Explanation:}
\begin{itemize}
    \item \keyword{Access Point (AP)}: Central coordinator for all communications
    \item \keyword{BSS (Basic Service Set)}: Network coverage area of single AP
    \item \keyword{ESS (Extended Service Set)}: Multiple interconnected BSSs
    \item \keyword{Distribution System}: Backbone connecting multiple APs
\end{itemize}

\textbf{Characteristics:}
\begin{itemize}
    \item All communication goes through access point
    \item Centralized network management
    \item Better security and performance control
\end{itemize}
\end{solutionbox}

\begin{mnemonicbox}
\mnemonic{Infrastructure Ad-hoc Mesh - AP BSS ESS Distribution}
\end{mnemonicbox}

\questionmarks{5(b)}{4}{Answer the following questions:}

\begin{solutionbox}
\textbf{1) List Uses of Ad hoc Network:}
\begin{center}
\captionof{table}{Uses}
\begin{tabulary}{\linewidth}{|L|L|}
\hline
\textbf{Use Case} & \textbf{Application} \\ \hline
\textbf{Emergency} & Disaster recovery, rescue operations \\ \hline
\textbf{Military} & Battlefield communications \\ \hline
\textbf{Conferences} & Temporary meeting networks \\ \hline
\textbf{Home} & Device-to-device communication \\ \hline
\textbf{Vehicular} & Car-to-car networks \\ \hline
\end{tabulary}
\end{center}

\textbf{2) Enlist entities and terminology of mobile computing:}
\begin{itemize}
    \item \textbf{Entities}: Mobile Node (MN), Home Agent (HA), Foreign Agent (FA), Correspondent Node (CN)
    \item \textbf{Terminology}: Handover, Roaming, Care-of Address
\end{itemize}
\end{solutionbox}

\begin{mnemonicbox}
\mnemonic{Emergency Military Conference Home Vehicular - MN HA FA CN}
\end{mnemonicbox}

\questionmarks{5(c)}{7}{Explain architecture of WLAN with neat diagram}

\begin{solutionbox}
\textbf{WLAN Architecture} consists of wireless stations communicating through access points.

\begin{center}
\begin{tikzpicture}[node distance=1cm, auto]
    \node [gtu block] (ds) {Distribution System (DS)};
    \node [gtu block, below left=of ds] (ap1) {AP 1};
    \node [gtu block, below right=of ds] (ap2) {AP 2};
    
    \node [gtu state, below=of ap1] (sta1) {Device 1};
    \node [gtu state, below=of ap2] (sta2) {Device 2};

    \draw [gtu arrow] (ap1) -- (ds);
    \draw [gtu arrow] (ap2) -- (ds);
    \draw [dashed] (sta1) -- (ap1);
    \draw [dashed] (sta2) -- (ap2);
    
    \node [draw, dashed, fit=(ap1) (sta1), label=left:BSS 1] {};
    \node [draw, dashed, fit=(ap2) (sta2), label=right:BSS 2] {};
    \node [draw, fit=(ds) (ap1) (ap2), label=above:ESS] {};
\end{tikzpicture}
\captionof{figure}{WLAN Architecture}
\end{center}

\textbf{Architecture Components:}
\begin{center}
\captionof{table}{Components}
\begin{tabulary}{\linewidth}{|L|L|L|}
\hline
\textbf{Component} & \textbf{Function} & \textbf{Coverage} \\ \hline
\textbf{STA (Station)} & Wireless device & Point \\ \hline
\textbf{AP (Access Point)} & Network coordinator & BSS area \\ \hline
\textbf{BSS (Basic Service Set)} & Single AP coverage & ~100m radius \\ \hline
\textbf{ESS (Extended Service Set)} & Multiple connected BSS & Large area \\ \hline
\textbf{DS (Distribution System)} & AP interconnection & Building/campus \\ \hline
\end{tabulary}
\end{center}

\textbf{Types of WLAN Architecture:}
\begin{itemize}
    \item \keyword{Infrastructure Mode}: Centralized, Managed, Scalable
    \item \keyword{Ad-hoc Mode (IBSS)}: Peer-to-peer, Decentralized, Temporary
\end{itemize}
\end{solutionbox}

\begin{mnemonicbox}
\mnemonic{STA AP BSS ESS DS - Infrastructure Ad-hoc}
\end{mnemonicbox}

\questionmarks{5(a OR)}{3}{Write features of 5G}

\begin{solutionbox}
\textbf{5G Key Features:}
\begin{center}
\captionof{table}{5G Features}
\begin{tabulary}{\linewidth}{|L|L|}
\hline
\textbf{Feature} & \textbf{Specification} \\ \hline
\textbf{Speed} & Up to 10 Gbps \\ \hline
\textbf{Latency} & < 1 millisecond \\ \hline
\textbf{Connectivity} & 1 million devices/km\textsuperscript{2} \\ \hline
\textbf{Reliability} & 99.999\% availability \\ \hline
\textbf{Bandwidth} & 100x increase \\ \hline
\textbf{Energy} & 90\% reduction \\ \hline
\end{tabulary}
\end{center}

\textbf{Advanced Capabilities:}
\begin{itemize}
    \item \keyword{Enhanced Mobile Broadband (eMBB)}: Ultra-fast data speeds
    \item \keyword{Ultra-Reliable Low Latency (URLLC)}: Mission-critical applications
    \item \keyword{Massive Machine Type Communication (mMTC)}: IoT connectivity
\end{itemize}
\end{solutionbox}

\begin{mnemonicbox}
\mnemonic{Speed Latency Connectivity Reliability Bandwidth Energy}
\end{mnemonicbox}

\questionmarks{5(b OR)}{4}{Answer the following questions:}

\begin{solutionbox}
\textbf{1) List Type of communication middleware:}
\begin{itemize}
    \item \keyword{Message-Oriented}: Asynchronous messaging
    \item \keyword{RPC-based}: Remote procedure calls
    \item \keyword{Object-Oriented}: Distributed objects
    \item \keyword{Service-Oriented}: Web services
    \item \keyword{Database}: Data access layer
\end{itemize}

\textbf{2) Define the term "Home Agent" in the context of Mobile IP:}
\textbf{Home Agent (HA)} is a router on mobile node's home network.
\textbf{Functions:}
\begin{itemize}
    \item \keyword{Maintains registration}: Tracks mobile node's current location
    \item \keyword{Tunnels packets}: Forwards data to mobile node's foreign location
    \item \keyword{Address management}: Manages mobile node's permanent IP address
    \item \keyword{Authentication}: Verifies mobile node identity during registration
\end{itemize}
\end{solutionbox}

\begin{mnemonicbox}
\mnemonic{Message RPC Object Service Database - HA Maintains Tunnels Address Authentication}
\end{mnemonicbox}

\questionmarks{5(c OR)}{7}{Explain Bluetooth protocol stack with diagram}

\begin{solutionbox}
\textbf{Bluetooth Protocol Stack} provides layered architecture for short-range wireless communication.

\begin{center}
\begin{tikzpicture}[node distance=0.1cm, auto]
    \node [gtu block, minimum width=5cm] (app) {Applications};
    \node [gtu block, minimum width=5cm, below=of app] (mid) {OBEX / SDP / TCS};
    \node [gtu block, minimum width=5cm, below=of mid] (trans) {RFCOMM};
    \node [gtu block, minimum width=5cm, below=of trans] (net) {L2CAP};
    \node [gtu block, minimum width=5cm, below=of net] (hci) {Host Controller Interface (HCI)};
    \node [gtu block, minimum width=5cm, below=of hci] (lmp) {Link Manager (LMP)};
    \node [gtu block, minimum width=5cm, below=of lmp] (base) {Baseband};
    \node [gtu block, minimum width=5cm, below=of base] (radio) {Radio Layer};
\end{tikzpicture}
\captionof{figure}{Bluetooth Stack}
\end{center}

\textbf{Protocol Stack Layers:}
\begin{center}
\captionof{table}{Layers}
\begin{tabulary}{\linewidth}{|L|L|L|}
\hline
\textbf{Layer} & \textbf{Function} & \textbf{Protocols} \\ \hline
\textbf{Application} & User applications & Audio, File transfer \\ \hline
\textbf{Middleware} & Services & OBEX, SDP, TCS \\ \hline
\textbf{Transport} & Data delivery & RFCOMM \\ \hline
\textbf{Network} & Packet management & L2CAP \\ \hline
\textbf{Interface} & Host-Controller & HCI \\ \hline
\textbf{Management} & Link control & LMP \\ \hline
\textbf{Data Link} & Channel access & Baseband \\ \hline
\textbf{Physical} & Radio transmission & 2.4 GHz ISM \\ \hline
\end{tabulary}
\end{center}

\textbf{Key Features:}
\begin{itemize}
    \item Frequency Hopping, Piconet, Scatternet, Power Classes
\end{itemize}
\end{solutionbox}

\begin{mnemonicbox}
\mnemonic{Application Middleware Transport Network Interface Management DataLink Physical}
\end{mnemonicbox}

\end{document}
