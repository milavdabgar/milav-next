\documentclass[10pt,a4paper]{article}

% content/resources/templates/preamble.tex
\usepackage[margin=0.6in]{geometry}
\author{Milav Dabgar}
\usepackage{amsmath,amssymb,amsthm}
\usepackage{booktabs}
\usepackage{multirow}
\usepackage{xcolor}
\usepackage{tcolorbox}
\tcbuselibrary{breakable,skins}
\usepackage[colorlinks=true,linkcolor=blue]{hyperref}
\usepackage{titlesec}
\usepackage{enumitem}
\usepackage{tikz}
\usepackage{pgfplots}
\usepackage{circuitikz}
\usepackage[version=4]{mhchem}
\usepackage{longtable}
\usepackage{array}
\usepackage{float}
\usepackage{caption}
\usepackage{listings}

\lstset{
  basicstyle=\small\ttfamily,
  breaklines=true,
  breakatwhitespace=false,
  postbreak=\mbox{\textcolor{red}{$\hookrightarrow$}\space},
  float=false,
  numbers=left,
  numberstyle=\tiny\color{gray},
  numbersep=10pt,
  xleftmargin=2em,
  keywordstyle=\color{blue},
  commentstyle=\color{green!60!black},
  stringstyle=\color{purple},
  backgroundcolor=\color{gray!5},
  showstringspaces=false,
  tabsize=2,
  captionpos=b,
  keepspaces=true,
  columns=flexible
}

\pgfplotsset{compat=1.18}
\usetikzlibrary{shapes,arrows,positioning,calc,patterns,decorations.pathmorphing,decorations.markings,arrows.meta}

% Color scheme
\definecolor{headcolor}{RGB}{0,102,204}
\definecolor{keycolor}{RGB}{220,20,60}
\definecolor{solutioncolor}{RGB}{34,139,34}
\definecolor{mnemoniccolor}{RGB}{148,0,211}
\definecolor{codecolor}{RGB}{0,0,100}

% Spacing
\setlength{\parskip}{3pt}
\setlist[itemize]{nosep}
\setlist[enumerate]{nosep}

% Title formatting
\titleformat{\section}{\Large\bfseries\color{headcolor}}{\thesection}{1em}{}
\titleformat{\subsection}{\large\bfseries\color{headcolor}}{\thesubsection}{1em}{}

% Pandoc tightlist compatibility
\providecommand{\tightlist}{%
  \setlength{\itemsep}{0pt}\setlength{\parskip}{0pt}}

% Pandoc longtable compatibility
\newcounter{none}
\def\thenone{}


% content/resources/templates/gujarati-boxes.tex
\usepackage{fontspec}
\usepackage{polyglossia}

% Set Gujarati as main language (document is primarily in Gujarati)
% Note: gloss-gujarati.ldf doesn't exist in polyglossia, but it will use hyphenation patterns
\setdefaultlanguage{gujarati}
\setotherlanguage{english}

% Configure Gujarati font properly
% Use Language=Default to prevent polyglossia from trying to add language-specific features
% that don't exist for Gujarati, which causes "empty feature" warnings
\newfontfamily\gujaratifont[Script=Gujarati,AutoFakeBold=2.5,AutoFakeSlant=0.3]{Noto Sans Gujarati}
\setmainfont[Script=Gujarati,AutoFakeBold=2.5,AutoFakeSlant=0.3]{Noto Sans Gujarati}
% Use Noto Sans Gujarati for monospace to support Gujarati in text
\setmonofont[Scale=0.9]{Noto Sans Gujarati}

% Configure English to use the same font
\newfontfamily\englishfont[Script=Gujarati,AutoFakeBold=2.5,AutoFakeSlant=0.3]{Noto Sans Gujarati}

% Translations for polyglossia
\gappto\captionsgujarati{
  \renewcommand{\tablename}{કોષ્ટક}
  \renewcommand{\figurename}{આકૃતિ}
}

% Helper for TikZ nodes to ensure Gujarati font
\newcommand{\gu}[1]{{\gujaratifont #1}}

% Custom environments
\newtcolorbox{solutionbox}{
    breakable,
    enhanced,
    colback=solutioncolor!5!white,
    colframe=solutioncolor!75!black,
    fonttitle=\bfseries,
    title=જવાબ
}

\newtcolorbox{solutionboxnobreak}{
 colback=solutioncolor!5!white,
 colframe=solutioncolor!75!black,
 fonttitle=\bfseries,
 title=જવાબ
}

\newtcolorbox{keyformula}{
 breakable,
 enhanced,
 colback=keycolor!5!white,
 colframe=keycolor!75!black,
 fonttitle=\bfseries,
 title=રાસાયણિક સમીકરણ/સૂત્ર
}

\newtcolorbox{mnemonicbox}{
 breakable,
 enhanced,
 colback=mnemoniccolor!5!white,
 colframe=mnemoniccolor!75!black,
 fonttitle=\bfseries,
 title=મેમરી ટ્રીક
}


\begin{document}

\begin{center}
{\Huge\bfseries\color{headcolor} Subject Name (Gujarati)}\\[5pt]
{\LARGE 4351601 -- Winter 2023}\\[3pt]
{\large Semester 1 Study Material}\\[3pt]
{\normalsize\textit{Detailed Solutions and Explanations}}
\end{center}

\vspace{10pt}

\subsection*{પ્રશ્ન 1(અ) [3
ગુણ]}\label{uxaaauxab0uxab6uxaa8-1uxa85-3-uxa97uxaa3}

\textbf{નીચેની terms ની વ્યાખ્યા આપો: (1) Artificial Intelligence (2)
Expert System.}

\begin{solutionbox}

{\def\LTcaptype{none} % do not increment counter
\begin{longtable}[]{@{}
  >{\raggedright\arraybackslash}p{(\linewidth - 2\tabcolsep) * \real{0.4000}}
  >{\raggedright\arraybackslash}p{(\linewidth - 2\tabcolsep) * \real{0.6000}}@{}}
\toprule\noalign{}
\begin{minipage}[b]{\linewidth}\raggedright
Term
\end{minipage} & \begin{minipage}[b]{\linewidth}\raggedright
વ્યાખ્યા
\end{minipage} \\
\midrule\noalign{}
\endhead
\bottomrule\noalign{}
\endlastfoot
\textbf{Artificial Intelligence} & AI એ computer science ની એક શાખા છે જે
એવા machines બનાવે છે જે સામાન્ય રીતે માનવ બુદ્ધિની જરૂર પડતા કાર્યો કરી શકે છે, જેમ
કે learning, reasoning અને problem-solving. \\
\textbf{Expert System} & Expert system એ એક computer program છે જે
knowledge અને inference rules નો ઉપયોગ કરીને એવી problems solve કરે છે જેમાં
સામાન્ય રીતે ચોક્કસ ક્ષેત્રમાં માનવ expertise ની જરૂર પડે છે. \\
\end{longtable}
}

\begin{itemize}
\tightlist
\item
  \textbf{AI ની વિશેષતાઓ}: Learning, reasoning, perception
\item
  \textbf{Expert system ના ભાગો}: Knowledge base, inference engine
\end{itemize}

\end{solutionbox}
\begin{mnemonicbox}
``AI શીખે છે, Expert સલાહ આપે છે''

\end{mnemonicbox}
\subsection*{પ્રશ્ન 1(બ) [4
ગુણ]}\label{uxaaauxab0uxab6uxaa8-1uxaac-4-uxa97uxaa3}

\textbf{Biological Neural Network અને Artificial Neural Network ની
સરખામણી કરો.}

\begin{solutionbox}

{\def\LTcaptype{none} % do not increment counter
\begin{longtable}[]{@{}
  >{\raggedright\arraybackslash}p{(\linewidth - 4\tabcolsep) * \real{0.1000}}
  >{\raggedright\arraybackslash}p{(\linewidth - 4\tabcolsep) * \real{0.4500}}
  >{\raggedright\arraybackslash}p{(\linewidth - 4\tabcolsep) * \real{0.4500}}@{}}
\toprule\noalign{}
\begin{minipage}[b]{\linewidth}\raggedright
પાસું
\end{minipage} & \begin{minipage}[b]{\linewidth}\raggedright
Biological Neural Network
\end{minipage} & \begin{minipage}[b]{\linewidth}\raggedright
Artificial Neural Network
\end{minipage} \\
\midrule\noalign{}
\endhead
\bottomrule\noalign{}
\endlastfoot
\textbf{Processing} & Parallel processing & Sequential/parallel
processing \\
\textbf{ઝડપ} & ધીમી (milliseconds) & ઝડપી (nanoseconds) \\
\textbf{શીખવું} & સતત શીખવું & Batch/online learning \\
\textbf{Storage} & વિતરિત storage & કેન્દ્રિય storage \\
\end{longtable}
}

\begin{itemize}
\tightlist
\item
  \textbf{Biological}: જટિલ, fault-tolerant, સ્વ-સુધારણા કરે છે
\item
  \textbf{Artificial}: સરળ, ચોક્કસ, programmable
\end{itemize}

\end{solutionbox}
\begin{mnemonicbox}
``Bio જટિલ છે, AI સરળ છે''

\end{mnemonicbox}
\subsection*{પ્રશ્ન 1(ક) [7
ગુણ]}\label{uxaaauxab0uxab6uxaa8-1uxa95-7-uxa97uxaa3}

\textbf{AI ના પ્રકારો તેની applications સાથે સમજાવો.}

\begin{solutionbox}

{\def\LTcaptype{none} % do not increment counter
\begin{longtable}[]{@{}
  >{\raggedright\arraybackslash}p{(\linewidth - 4\tabcolsep) * \real{0.3824}}
  >{\raggedright\arraybackslash}p{(\linewidth - 4\tabcolsep) * \real{0.2353}}
  >{\raggedright\arraybackslash}p{(\linewidth - 4\tabcolsep) * \real{0.3824}}@{}}
\toprule\noalign{}
\begin{minipage}[b]{\linewidth}\raggedright
AI નો પ્રકાર
\end{minipage} & \begin{minipage}[b]{\linewidth}\raggedright
વર્ણન
\end{minipage} & \begin{minipage}[b]{\linewidth}\raggedright
Applications
\end{minipage} \\
\midrule\noalign{}
\endhead
\bottomrule\noalign{}
\endlastfoot
\textbf{Narrow AI} & ચોક્કસ કાર્યો માટે design કરેલ AI & Voice assistants,
recommendation systems \\
\textbf{General AI} & માનવ સ્તરની intelligence વાળી AI & હજુ સુધી પ્રાપ્ત
નથી \\
\textbf{Super AI} & માનવ intelligence કરતાં વધારે AI & સૈદ્ધાંતિક વિભાવના \\
\end{longtable}
}

\begin{center}
\textbf{Mermaid Diagram (Code)}
\begin{verbatim}
{Shaded}
{Highlighting}[]
graph TD
    A[AI ના પ્રકારો] {-{-}{} B[Narrow AI]}
    A {-{-}{} C[General AI]}
    A {-{-}{} D[Super AI]}
    B {-{-}{} E[Siri, Alexa]}
    B {-{-}{} F[Netflix Recommendations]}
    C {-{-}{} G[માનવ સ્તરના કાર્યો]}
    D {-{-}{} H[માનવ Intelligence કરતાં વધારે]}
{Highlighting}
{Shaded}
\end{verbatim}
\end{center}

\begin{itemize}
\tightlist
\item
  \textbf{હાલનું focus}: Narrow AI આજના applications પર પ્રભુત્વ ધરાવે છે
\item
  \textbf{ભવિષ્યનું લક્ષ્ય}: General AI ને સુરક્ષિત રીતે પ્રાપ્ત કરવું
\end{itemize}

\end{solutionbox}
\begin{mnemonicbox}
``હવે Narrow, લક્ષ્ય General, Super ડરામણી''

\end{mnemonicbox}
\subsection*{પ્રશ્ન 1(ક) અથવા [7
ગુણ]}\label{uxaaauxab0uxab6uxaa8-1uxa95-uxa85uxaa5uxab5-7-uxa97uxaa3}

\textbf{AI ethics અને limitations સમજાવો.}

\begin{solutionbox}

{\def\LTcaptype{none} % do not increment counter
\begin{longtable}[]{@{}ll@{}}
\toprule\noalign{}
Ethics નું પાસું & વર્ણન \\
\midrule\noalign{}
\endhead
\bottomrule\noalign{}
\endlastfoot
\textbf{Privacy} & વ્યક્તિગત data અને user information ની સુરક્ષા \\
\textbf{Bias} & વિવિધ જૂથોમાં નિષ્પક્ષતા સુનિશ્ચિત કરવી \\
\textbf{Transparency} & AI નિર્ણયોને સમજાવી શકાય તેવા બનાવવા \\
\textbf{Accountability} & AI actions માટે જવાબદારી નક્કી કરવી \\
\end{longtable}
}

\textbf{મર્યાદાઓ:}

\begin{itemize}
\tightlist
\item
  \textbf{Data dependency}: મોટા, ગુણવત્તાવાળા datasets ની જરૂર
\item
  \textbf{Computational power}: નોંધપાત્ર processing resources ની જરૂર
\item
  \textbf{Creativity નો અભાવ}: ખરેખર મૌલિક concepts બનાવી શકતી નથી
\end{itemize}

\end{solutionbox}
\begin{mnemonicbox}
``Privacy, Bias, Transparency, Accountability''

\end{mnemonicbox}
\subsection*{પ્રશ્ન 2(અ) [3
ગુણ]}\label{uxaaauxab0uxab6uxaa8-2uxa85-3-uxa97uxaa3}

\textbf{નીચેની terms ની વ્યાખ્યા આપો: (1) Well posed Learning Problem (2)
Machine Learning.}

\begin{solutionbox}

{\def\LTcaptype{none} % do not increment counter
\begin{longtable}[]{@{}
  >{\raggedright\arraybackslash}p{(\linewidth - 2\tabcolsep) * \real{0.4000}}
  >{\raggedright\arraybackslash}p{(\linewidth - 2\tabcolsep) * \real{0.6000}}@{}}
\toprule\noalign{}
\begin{minipage}[b]{\linewidth}\raggedright
Term
\end{minipage} & \begin{minipage}[b]{\linewidth}\raggedright
વ્યાખ્યા
\end{minipage} \\
\midrule\noalign{}
\endhead
\bottomrule\noalign{}
\endlastfoot
\textbf{Well posed Learning Problem} & એક learning problem જેમાં સ્પષ્ટ રીતે
વ્યાખ્યાયિત task (T), performance measure (P), અને experience (E) હોય જ્યાં
experience સાથે performance સુધરે છે. \\
\textbf{Machine Learning} & AI નો એક ભાગ જે computers ને experience થી
આપોઆપ શીખવા અને સુધારવા માટે સક્ષમ બનાવે છે, સ્પષ્ટ રીતે program કર્યા વગર. \\
\end{longtable}
}

\begin{itemize}
\tightlist
\item
  \textbf{Well posed formula}: T + P + E = Learning
\item
  \textbf{ML નો ફાયદો}: Data થી આપોઆપ સુધારો
\end{itemize}

\end{solutionbox}
\begin{mnemonicbox}
``Task, Performance, Experience''

\end{mnemonicbox}
\subsection*{પ્રશ્ન 2(બ) [4
ગુણ]}\label{uxaaauxab0uxab6uxaa8-2uxaac-4-uxa97uxaa3}

\textbf{Reinforcement Learning તેમાં ઉપયોગ થતાં terms સાથે સમજાવો.}

\begin{solutionbox}

{\def\LTcaptype{none} % do not increment counter
\begin{longtable}[]{@{}ll@{}}
\toprule\noalign{}
Term & વર્ણન \\
\midrule\noalign{}
\endhead
\bottomrule\noalign{}
\endlastfoot
\textbf{Agent} & શીખનાર અથવા નિર્ણય લેનાર \\
\textbf{Environment} & જે દુનિયામાં agent કામ કરે છે \\
\textbf{Action} & દરેક state માં agent શું કરી શકે છે \\
\textbf{State} & Agent ની હાલની સ્થિતિ \\
\textbf{Reward} & Environment તરફથી feedback \\
\end{longtable}
}

\begin{center}
\textbf{Mermaid Diagram (Code)}
\begin{verbatim}
{Shaded}
{Highlighting}[]
graph LR
    A[Agent] {-{-}{} B[Action]}
    B {-{-}{} C[Environment]}
    C {-{-}{} D[State]}
    C {-{-}{} E[Reward]}
    D {-{-}{} A}
    E {-{-}{} A}
{Highlighting}
{Shaded}
\end{verbatim}
\end{center}

\begin{itemize}
\tightlist
\item
  \textbf{શીખવાની પ્રક્રિયા}: Trial and error approach
\item
  \textbf{લક્ષ્ય}: કુલ reward વધારવું
\end{itemize}

\end{solutionbox}
\begin{mnemonicbox}
``Agent કરે છે, Environment State અને Reward આપે છે''

\end{mnemonicbox}
\subsection*{પ્રશ્ન 2(ક) [7
ગુણ]}\label{uxaaauxab0uxab6uxaa8-2uxa95-7-uxa97uxaa3}

\textbf{Supervised, Unsupervised અને Reinforcement Learning ની સરખામણી
કરો.}

\begin{solutionbox}

{\def\LTcaptype{none} % do not increment counter
\begin{longtable}[]{@{}llll@{}}
\toprule\noalign{}
પાસું & Supervised & Unsupervised & Reinforcement \\
\midrule\noalign{}
\endhead
\bottomrule\noalign{}
\endlastfoot
\textbf{Data} & Labeled data & Unlabeled data & Interactive data \\
\textbf{લક્ષ્ય} & Output predict કરવું & Patterns શોધવા & Reward વધારવું \\
\textbf{Feedback} & તુરંત & કોઈ નહીં & વિલંબિત \\
\textbf{ઉદાહરણો} & Classification & Clustering & Game playing \\
\end{longtable}
}

\begin{itemize}
\tightlist
\item
  \textbf{Supervised}: શિક્ષક-માર્ગદર્શિત learning
\item
  \textbf{Unsupervised}: સ્વ-શોધ learning
\item
  \textbf{Reinforcement}: Trial-and-error learning
\end{itemize}

\end{solutionbox}
\begin{mnemonicbox}
``Supervised પાસે શિક્ષક, Unsupervised શોધે છે,
Reinforcement પ્રયત્ન કરે છે''

\end{mnemonicbox}
\subsection*{પ્રશ્ન 2(અ) અથવા [3
ગુણ]}\label{uxaaauxab0uxab6uxaa8-2uxa85-uxa85uxaa5uxab5-3-uxa97uxaa3}

\textbf{Reinforcement Learning ના key features લખો.}

\begin{solutionbox}

{\def\LTcaptype{none} % do not increment counter
\begin{longtable}[]{@{}ll@{}}
\toprule\noalign{}
Feature & વર્ણન \\
\midrule\noalign{}
\endhead
\bottomrule\noalign{}
\endlastfoot
\textbf{Trial and Error} & પ્રયોગ દ્વારા શીખવું \\
\textbf{Delayed Reward} & Actions પછી feedback મળે છે \\
\textbf{Sequential Decision} & Actions ભવિષ્યના states ને અસર કરે છે \\
\end{longtable}
}

\begin{itemize}
\tightlist
\item
  \textbf{કોઈ supervisor નથી}: Agent સ્વતંત્ર રીતે શીખે છે
\item
  \textbf{Exploration vs Exploitation}: નવા actions અજમાવવા અને જાણીતા
  સારા actions વાપરવા વચ્ચે સંતુલન
\end{itemize}

\end{solutionbox}
\begin{mnemonicbox}
``પ્રયત્ન, વિલંબ, ક્રમ''

\end{mnemonicbox}
\subsection*{પ્રશ્ન 2(બ) અથવા [4
ગુણ]}\label{uxaaauxab0uxab6uxaa8-2uxaac-uxa85uxaa5uxab5-4-uxa97uxaa3}

\textbf{Reinforcement Learning ના પ્રકારો સમજાવો.}

\begin{solutionbox}

{\def\LTcaptype{none} % do not increment counter
\begin{longtable}[]{@{}ll@{}}
\toprule\noalign{}
પ્રકાર & વર્ણન \\
\midrule\noalign{}
\endhead
\bottomrule\noalign{}
\endlastfoot
\textbf{Positive RL} & વર્તણૂક વધારવા માટે positive stimulus ઉમેરવું \\
\textbf{Negative RL} & વર્તણૂક વધારવા માટે negative stimulus દૂર કરવું \\
\end{longtable}
}

\textbf{Learning આધારિત:}

\begin{itemize}
\tightlist
\item
  \textbf{Model-based}: Agent environment model શીખે છે
\item
  \textbf{Model-free}: Agent સીધો experience થી શીખે છે
\end{itemize}

\end{solutionbox}
\begin{mnemonicbox}
``Positive ઉમેરે, Negative દૂર કરે''

\end{mnemonicbox}
\subsection*{પ્રશ્ન 2(ક) અથવા [7
ગુણ]}\label{uxaaauxab0uxab6uxaa8-2uxa95-uxa85uxaa5uxab5-7-uxa97uxaa3}

\textbf{Reinforcement Learning implement કરવા માટેના approaches સમજાવો.}

\begin{solutionbox}

{\def\LTcaptype{none} % do not increment counter
\begin{longtable}[]{@{}lll@{}}
\toprule\noalign{}
Approach & વર્ણન & ઉદાહરણ \\
\midrule\noalign{}
\endhead
\bottomrule\noalign{}
\endlastfoot
\textbf{Value-based} & States/actions ના value શીખવા & Q-Learning \\
\textbf{Policy-based} & Policy સીધી શીખવી & Policy Gradient \\
\textbf{Model-based} & Environment model શીખવું & Dynamic Programming \\
\end{longtable}
}

\begin{center}
\textbf{Mermaid Diagram (Code)}
\begin{verbatim}
{Shaded}
{Highlighting}[]
graph TD
    A[RL Approaches] {-{-}{} B[Value{-}based]}
    A {-{-}{} C[Policy{-}based]}
    A {-{-}{} D[Model{-}based]}
    B {-{-}{} E[Q{-}Learning]}
    C {-{-}{} F[Policy Gradient]}
    D {-{-}{} G[Dynamic Programming]}
{Highlighting}
{Shaded}
\end{verbatim}
\end{center}

\begin{itemize}
\tightlist
\item
  \textbf{Value-based}: Value functions estimate કરે છે
\item
  \textbf{Policy-based}: Policy parameters optimize કરે છે
\item
  \textbf{Model-based}: Environment model વાપરે છે
\end{itemize}

\end{solutionbox}
\begin{mnemonicbox}
``Value, Policy, Model''

\end{mnemonicbox}
\subsection*{પ્રશ્ન 3(અ) [3
ગુણ]}\label{uxaaauxab0uxab6uxaa8-3uxa85-3-uxa97uxaa3}

\textbf{Activation functions ReLU અને sigmoid વર્ણવો.}

\begin{solutionbox}

{\def\LTcaptype{none} % do not increment counter
\begin{longtable}[]{@{}lll@{}}
\toprule\noalign{}
Function & Formula & Range \\
\midrule\noalign{}
\endhead
\bottomrule\noalign{}
\endlastfoot
\textbf{ReLU} & f(x) = max(0, x) & [0, \infty) \\
\textbf{Sigmoid} & f(x) = 1/(1 + e\^{}(-x)) & (0, 1) \\
\end{longtable}
}

\begin{itemize}
\tightlist
\item
  \textbf{ReLU નો ફાયદો}: Vanishing gradient problem નથી
\item
  \textbf{Sigmoid નો ફાયદો}: Smooth gradient, probabilistic output
\end{itemize}

\end{solutionbox}
\begin{mnemonicbox}
``ReLU સુધારે છે, Sigmoid દબાવે છે''

\end{mnemonicbox}
\subsection*{પ્રશ્ન 3(બ) [4
ગુણ]}\label{uxaaauxab0uxab6uxaa8-3uxaac-4-uxa97uxaa3}

\textbf{Multi-layer feed forward ANN સમજાવો.}

\begin{solutionbox}

{\def\LTcaptype{none} % do not increment counter
\begin{longtable}[]{@{}ll@{}}
\toprule\noalign{}
Component & વર્ણન \\
\midrule\noalign{}
\endhead
\bottomrule\noalign{}
\endlastfoot
\textbf{Input Layer} & Input data receive કરે છે \\
\textbf{Hidden Layers} & Information process કરે છે (multiple layers) \\
\textbf{Output Layer} & Final result બનાવે છે \\
\textbf{Connections} & ફક્ત forward direction માં \\
\end{longtable}
}

\begin{itemize}
\tightlist
\item
  \textbf{Information flow}: Input થી output સુધી એક દિશામાં
\item
  \textbf{કોઈ cycles નથી}: કોઈ feedback connections નથી
\end{itemize}

\end{solutionbox}
\begin{mnemonicbox}
``Input \rightarrow Hidden \rightarrow Output (ફક્ત આગળ)''

\end{mnemonicbox}
\subsection*{પ્રશ્ન 3(ક) [7
ગુણ]}\label{uxaaauxab0uxab6uxaa8-3uxa95-7-uxa97uxaa3}

\textbf{ANN નું structure દોરો અને તેના દરેક components ની functionality
સમજાવો.}

\begin{solutionbox}

\begin{center}
\textbf{Mermaid Diagram (Code)}
\begin{verbatim}
{Shaded}
{Highlighting}[]
graph LR
    A[Input Layer] {-{-}{} B[Hidden Layer 1]}
    B {-{-}{} C[Hidden Layer 2]}
    C {-{-}{} D[Output Layer]}
    
    subgraph "Components"
        E[Neurons]
        F[Weights]
        G[Bias]
        H[Activation Function]
    end
{Highlighting}
{Shaded}
\end{verbatim}
\end{center}

{\def\LTcaptype{none} % do not increment counter
\begin{longtable}[]{@{}
  >{\raggedright\arraybackslash}p{(\linewidth - 2\tabcolsep) * \real{0.4231}}
  >{\raggedright\arraybackslash}p{(\linewidth - 2\tabcolsep) * \real{0.5769}}@{}}
\toprule\noalign{}
\begin{minipage}[b]{\linewidth}\raggedright
Component
\end{minipage} & \begin{minipage}[b]{\linewidth}\raggedright
Functionality
\end{minipage} \\
\midrule\noalign{}
\endhead
\bottomrule\noalign{}
\endlastfoot
\textbf{Neurons} & Processing units જે inputs receive કરે છે અને outputs
બનાવે છે \\
\textbf{Weights} & Neurons વચ્ચેની connection strengths \\
\textbf{Bias} & Activation function ને shift કરવા માટે વધારાનું parameter \\
\textbf{Activation Function} & Network માં non-linearity લાવે છે \\
\end{longtable}
}

\begin{itemize}
\tightlist
\item
  \textbf{Input layer}: Input data receive કરે છે અને વિતરિત કરે છે
\item
  \textbf{Hidden layers}: Features અને patterns extract કરે છે
\item
  \textbf{Output layer}: Final classification અથવા prediction બનાવે છે
\item
  \textbf{Connections}: Neurons વચ્ચેની weighted links
\end{itemize}

\end{solutionbox}
\begin{mnemonicbox}
``Neurons સાથે Weights, Bias, અને Activation''

\end{mnemonicbox}
\subsection*{પ્રશ્ન 3(અ) અથવા [3
ગુણ]}\label{uxaaauxab0uxab6uxaa8-3uxa85-uxa85uxaa5uxab5-3-uxa97uxaa3}

\textbf{Backpropagation પર ટૂંક નોંધ લખો.}

\begin{solutionbox}

{\def\LTcaptype{none} % do not increment counter
\begin{longtable}[]{@{}ll@{}}
\toprule\noalign{}
પાસું & વર્ણન \\
\midrule\noalign{}
\endhead
\bottomrule\noalign{}
\endlastfoot
\textbf{હેતુ} & Neural networks માટે training algorithm \\
\textbf{પદ્ધતિ} & Chain rule સાથે gradient descent \\
\textbf{દિશા} & પાછળની તરફ error propagation \\
\end{longtable}
}

\begin{itemize}
\tightlist
\item
  \textbf{પ્રક્રિયા}: Network દ્વારા પાછળની તરફ error gradients calculate
  કરવા
\item
  \textbf{Update}: Error ઘટાડવા માટે weights adjust કરવા
\end{itemize}

\end{solutionbox}
\begin{mnemonicbox}
``પાછળની તરફ Error Propagation''

\end{mnemonicbox}
\subsection*{પ્રશ્ન 3(બ) અથવા [4
ગુણ]}\label{uxaaauxab0uxab6uxaa8-3uxaac-uxa85uxaa5uxab5-4-uxa97uxaa3}

\textbf{Single-layer feed forward network સમજાવો.}

\begin{solutionbox}

{\def\LTcaptype{none} % do not increment counter
\begin{longtable}[]{@{}ll@{}}
\toprule\noalign{}
Feature & વર્ણન \\
\midrule\noalign{}
\endhead
\bottomrule\noalign{}
\endlastfoot
\textbf{Structure} & Input layer સીધી output layer સાથે connected \\
\textbf{Layers} & ફક્ત input અને output layers \\
\textbf{મર્યાદાઓ} & ફક્ત linearly separable problems solve કરી શકે \\
\textbf{ઉદાહરણ} & Perceptron \\
\end{longtable}
}

\begin{itemize}
\tightlist
\item
  \textbf{ક્ષમતા}: Linear decision boundaries સુધી મર્યાદિત
\item
  \textbf{Applications}: સરળ classification tasks
\end{itemize}

\end{solutionbox}
\begin{mnemonicbox}
``Single Layer, Linear મર્યાદાઓ''

\end{mnemonicbox}
\subsection*{પ્રશ્ન 3(ક) અથવા [7
ગુણ]}\label{uxaaauxab0uxab6uxaa8-3uxa95-uxa85uxaa5uxab5-7-uxa97uxaa3}

\textbf{Recurrent neural network નું architecture દોરો અને સમજાવો.}

\begin{solutionbox}

\begin{center}
\textbf{Mermaid Diagram (Code)}
\begin{verbatim}
{Shaded}
{Highlighting}[]
graph LR
    A[Input] {-{-}{} B[Hidden State]}
    B {-{-}{} C[Output]}
    B {-{-}{} B[Self{-}loop]}
    D[Previous State] {-{-}{} B}
{Highlighting}
{Shaded}
\end{verbatim}
\end{center}

{\def\LTcaptype{none} % do not increment counter
\begin{longtable}[]{@{}ll@{}}
\toprule\noalign{}
Component & Function \\
\midrule\noalign{}
\endhead
\bottomrule\noalign{}
\endlastfoot
\textbf{Hidden State} & પાછલા inputs ની memory રાખે છે \\
\textbf{Recurrent Connection} & Hidden state થી તે જ તરફ feedback \\
\textbf{Sequence Processing} & Sequential data handle કરે છે \\
\end{longtable}
}

\begin{itemize}
\tightlist
\item
  \textbf{Memory}: પાછલા time steps ની information રાખે છે
\item
  \textbf{Applications}: Language modeling, speech recognition
\item
  \textbf{ફાયદો}: Variable-length sequences process કરી શકે છે
\end{itemize}

\end{solutionbox}
\begin{mnemonicbox}
``Recurrent યાદ રાખે છે, પાછળ Loop કરે છે''

\end{mnemonicbox}
\subsection*{પ્રશ્ન 4(અ) [3
ગુણ]}\label{uxaaauxab0uxab6uxaa8-4uxa85-3-uxa97uxaa3}

\textbf{NLP ની વ્યાખ્યા આપો અને તેના advantages લખો.}

\begin{solutionbox}

{\def\LTcaptype{none} % do not increment counter
\begin{longtable}[]{@{}
  >{\raggedright\arraybackslash}p{(\linewidth - 2\tabcolsep) * \real{0.4000}}
  >{\raggedright\arraybackslash}p{(\linewidth - 2\tabcolsep) * \real{0.6000}}@{}}
\toprule\noalign{}
\begin{minipage}[b]{\linewidth}\raggedright
Term
\end{minipage} & \begin{minipage}[b]{\linewidth}\raggedright
વ્યાખ્યા
\end{minipage} \\
\midrule\noalign{}
\endhead
\bottomrule\noalign{}
\endlastfoot
\textbf{NLP} & Natural Language Processing - computers ને માનવ ભાષા
સમજવા, interpret કરવા અને generate કરવા માટે સક્ષમ બનાવે છે \\
\end{longtable}
}

\textbf{Advantages:}

\begin{itemize}
\tightlist
\item
  \textbf{Human-computer interaction}: કુદરતી communication
\item
  \textbf{Automation}: આપોઆપ text processing અને analysis
\item
  \textbf{Accessibility}: વિકલાંગ વપરાશકર્તાઓ માટે voice interfaces
\end{itemize}

\end{solutionbox}
\begin{mnemonicbox}
``કુદરતી ભાષા, કુદરતી Interaction''

\end{mnemonicbox}
\subsection*{પ્રશ્ન 4(બ) [4
ગુણ]}\label{uxaaauxab0uxab6uxaa8-4uxaac-4-uxa97uxaa3}

\textbf{NLU અને NLG ની સરખામણી કરો.}

\begin{solutionbox}

{\def\LTcaptype{none} % do not increment counter
\begin{longtable}[]{@{}lll@{}}
\toprule\noalign{}
પાસું & NLU (Understanding) & NLG (Generation) \\
\midrule\noalign{}
\endhead
\bottomrule\noalign{}
\endlastfoot
\textbf{હેતુ} & માનવ ભાષા interpret કરવી & માનવ ભાષા generate કરવી \\
\textbf{Input} & Text/Speech & Structured data \\
\textbf{Output} & Structured data & Text/Speech \\
\textbf{ઉદાહરણો} & Sentiment analysis & Text summarization \\
\end{longtable}
}

\begin{itemize}
\tightlist
\item
  \textbf{NLU}: Unstructured text ને structured data માં convert કરે છે
\item
  \textbf{NLG}: Structured data ને natural text માં convert કરે છે
\end{itemize}

\end{solutionbox}
\begin{mnemonicbox}
``NLU સમજે છે, NLG બનાવે છે''

\end{mnemonicbox}
\subsection*{પ્રશ્ન 4(ક) [7
ગુણ]}\label{uxaaauxab0uxab6uxaa8-4uxa95-7-uxa97uxaa3}

\textbf{Word tokenization અને frequency distribution of words યોગ્ય ઉદાહરણ
સાથે સમજાવો.}

\begin{solutionbox}

{\def\LTcaptype{none} % do not increment counter
\begin{longtable}[]{@{}
  >{\raggedright\arraybackslash}p{(\linewidth - 4\tabcolsep) * \real{0.3704}}
  >{\raggedright\arraybackslash}p{(\linewidth - 4\tabcolsep) * \real{0.2963}}
  >{\raggedright\arraybackslash}p{(\linewidth - 4\tabcolsep) * \real{0.3333}}@{}}
\toprule\noalign{}
\begin{minipage}[b]{\linewidth}\raggedright
પ્રક્રિયા
\end{minipage} & \begin{minipage}[b]{\linewidth}\raggedright
વર્ણન
\end{minipage} & \begin{minipage}[b]{\linewidth}\raggedright
ઉદાહરણ
\end{minipage} \\
\midrule\noalign{}
\endhead
\bottomrule\noalign{}
\endlastfoot
\textbf{Tokenization} & Text ને individual words/tokens માં તોડવું & ``Hello
world'' \rightarrow [``Hello'', ``world''] \\
\textbf{Frequency Distribution} & દરેક token ની occurrence count કરવી &
\{``Hello'': 1, ``world'': 1\} \\
\end{longtable}
}

\textbf{ઉદાહરણ:}

\begin{verbatim}
Text: "The cat sat on the mat"
Tokens: ["The", "cat", "sat", "on", "the", "mat"]
Frequency: {"The": 1, "cat": 1, "sat": 1, "on": 1, "the": 1, "mat": 1}
\end{verbatim}

\begin{itemize}
\tightlist
\item
  \textbf{Case sensitivity}: ``The'' અને ``the'' અલગ અલગ count થાય છે
\item
  \textbf{Applications}: Text analysis, search engines
\item
  \textbf{Preprocessing}: NLP tasks માટે આવશ્યક step
\end{itemize}

\end{solutionbox}
\begin{mnemonicbox}
``Tokenize પછી Count''

\end{mnemonicbox}
\subsection*{પ્રશ્ન 4(અ) અથવા [3
ગુણ]}\label{uxaaauxab0uxab6uxaa8-4uxa85-uxa85uxaa5uxab5-3-uxa97uxaa3}

\textbf{NLP ના disadvantages ની યાદી આપો.}

\begin{solutionbox}

{\def\LTcaptype{none} % do not increment counter
\begin{longtable}[]{@{}ll@{}}
\toprule\noalign{}
Disadvantage & વર્ણન \\
\midrule\noalign{}
\endhead
\bottomrule\noalign{}
\endlastfoot
\textbf{Ambiguity} & Words/sentences ના multiple meanings \\
\textbf{Context dependency} & Context સાથે meaning બદલાય છે \\
\textbf{Language complexity} & Grammar rules અને exceptions \\
\end{longtable}
}

\begin{itemize}
\tightlist
\item
  \textbf{સાંસ્કૃતિક variations}: અલગ ભાષાઓ, dialects
\item
  \textbf{Computational cost}: Resource-intensive processing
\end{itemize}

\end{solutionbox}
\begin{mnemonicbox}
``અસ્પષ્ટ, Contextual, જટિલ''

\end{mnemonicbox}
\subsection*{પ્રશ્ન 4(બ) અથવા [4
ગુણ]}\label{uxaaauxab0uxab6uxaa8-4uxaac-uxa85uxaa5uxab5-4-uxa97uxaa3}

\textbf{NLP માં ambiguities ના પ્રકારો સમજાવો.}

\begin{solutionbox}

{\def\LTcaptype{none} % do not increment counter
\begin{longtable}[]{@{}
  >{\raggedright\arraybackslash}p{(\linewidth - 4\tabcolsep) * \real{0.2609}}
  >{\raggedright\arraybackslash}p{(\linewidth - 4\tabcolsep) * \real{0.3478}}
  >{\raggedright\arraybackslash}p{(\linewidth - 4\tabcolsep) * \real{0.3913}}@{}}
\toprule\noalign{}
\begin{minipage}[b]{\linewidth}\raggedright
પ્રકાર
\end{minipage} & \begin{minipage}[b]{\linewidth}\raggedright
વર્ણન
\end{minipage} & \begin{minipage}[b]{\linewidth}\raggedright
ઉદાહરણ
\end{minipage} \\
\midrule\noalign{}
\endhead
\bottomrule\noalign{}
\endlastfoot
\textbf{Lexical} & Word ના multiple meanings & ``Bank''
(financial/river) \\
\textbf{Syntactic} & Multiple parse trees possible & ``I saw a man with
a telescope'' \\
\textbf{Semantic} & Multiple interpretations & ``Flying planes can be
dangerous'' \\
\end{longtable}
}

\begin{itemize}
\tightlist
\item
  \textbf{Resolution}: Context analysis, statistical models
\item
  \textbf{Challenge}: NLP systems માં મુખ્ય અવરોધ
\end{itemize}

\end{solutionbox}
\begin{mnemonicbox}
``Lexical words, Syntactic structure, Semantic
meaning''

\end{mnemonicbox}
\subsection*{પ્રશ્ન 4(ક) અથવા [7
ગુણ]}\label{uxaaauxab0uxab6uxaa8-4uxa95-uxa85uxaa5uxab5-7-uxa97uxaa3}

\textbf{Stemming words અને parts of speech(POS) tagging યોગ્ય ઉદાહરણ સાથે
સમજાવો.}

\begin{solutionbox}

{\def\LTcaptype{none} % do not increment counter
\begin{longtable}[]{@{}
  >{\raggedright\arraybackslash}p{(\linewidth - 4\tabcolsep) * \real{0.3704}}
  >{\raggedright\arraybackslash}p{(\linewidth - 4\tabcolsep) * \real{0.2963}}
  >{\raggedright\arraybackslash}p{(\linewidth - 4\tabcolsep) * \real{0.3333}}@{}}
\toprule\noalign{}
\begin{minipage}[b]{\linewidth}\raggedright
પ્રક્રિયા
\end{minipage} & \begin{minipage}[b]{\linewidth}\raggedright
વર્ણન
\end{minipage} & \begin{minipage}[b]{\linewidth}\raggedright
ઉદાહરણ
\end{minipage} \\
\midrule\noalign{}
\endhead
\bottomrule\noalign{}
\endlastfoot
\textbf{Stemming} & Words ને root/stem form માં ઘટાડવા & ``running'' \rightarrow
``run'', ``flies'' \rightarrow ``fli'' \\
\textbf{POS Tagging} & Grammatical categories assign કરવા & ``The/DT
cat/NN runs/VB fast/RB'' \\
\end{longtable}
}

\textbf{Stemming ઉદાહરણ:}

\begin{verbatim}
Original: ["running", "runs", "runner"]
Stemmed: ["run", "run", "runner"]
\end{verbatim}

\textbf{POS Tagging ઉદાહરણ:}

\begin{verbatim}
Sentence: "The quick brown fox jumps"
Tagged: "The/DT quick/JJ brown/JJ fox/NN jumps/VB"
\end{verbatim}

\begin{itemize}
\tightlist
\item
  \textbf{Stemming નો હેતુ}: Vocabulary size ઘટાડવું, સંબંધિત words ને group
  કરવા
\item
  \textbf{POS નો હેતુ}: Grammatical structure સમજવું
\item
  \textbf{Applications}: Information retrieval, grammar checking
\end{itemize}

\end{solutionbox}
\begin{mnemonicbox}
``Root સુધી Stem, Grammar પ્રમાણે Tag''

\end{mnemonicbox}
\subsection*{પ્રશ્ન 5(અ) [3
ગુણ]}\label{uxaaauxab0uxab6uxaa8-5uxa85-3-uxa97uxaa3}

\textbf{Word embedding વ્યાખ્યા આપો અને word embedding ની various
techniques ની યાદી આપો.}

\begin{solutionbox}

{\def\LTcaptype{none} % do not increment counter
\begin{longtable}[]{@{}
  >{\raggedright\arraybackslash}p{(\linewidth - 2\tabcolsep) * \real{0.4000}}
  >{\raggedright\arraybackslash}p{(\linewidth - 2\tabcolsep) * \real{0.6000}}@{}}
\toprule\noalign{}
\begin{minipage}[b]{\linewidth}\raggedright
Term
\end{minipage} & \begin{minipage}[b]{\linewidth}\raggedright
વ્યાખ્યા
\end{minipage} \\
\midrule\noalign{}
\endhead
\bottomrule\noalign{}
\endlastfoot
\textbf{Word Embedding} & Words ના dense vector representations જે
semantic relationships capture કરે છે \\
\end{longtable}
}

\textbf{Techniques:}

\begin{itemize}
\tightlist
\item
  \textbf{TF-IDF}: Term Frequency-Inverse Document Frequency
\item
  \textbf{Bag of Words (BoW)}: સરળ word occurrence counting
\item
  \textbf{Word2Vec}: Neural network-based embeddings
\end{itemize}

\end{solutionbox}
\begin{mnemonicbox}
``TF-IDF counts, BoW bags, Word2Vec vectorizes''

\end{mnemonicbox}
\subsection*{પ્રશ્ન 5(બ) [4
ગુણ]}\label{uxaaauxab0uxab6uxaa8-5uxaac-4-uxa97uxaa3}

\textbf{TF-IDF and BoW માટે Challenges સમજાવો.}

\begin{solutionbox}

{\def\LTcaptype{none} % do not increment counter
\begin{longtable}[]{@{}
  >{\raggedright\arraybackslash}p{(\linewidth - 2\tabcolsep) * \real{0.4000}}
  >{\raggedright\arraybackslash}p{(\linewidth - 2\tabcolsep) * \real{0.6000}}@{}}
\toprule\noalign{}
\begin{minipage}[b]{\linewidth}\raggedright
પદ્ધતિ
\end{minipage} & \begin{minipage}[b]{\linewidth}\raggedright
Challenges
\end{minipage} \\
\midrule\noalign{}
\endhead
\bottomrule\noalign{}
\endlastfoot
\textbf{TF-IDF} & Sparse vectors, કોઈ semantic similarity નથી, high
dimensionality \\
\textbf{BoW} & Order ignore થાય છે, context ખોવાય છે, sparse
representation \\
\end{longtable}
}

\textbf{સામાન્ય સમસ્યાઓ:}

\begin{itemize}
\tightlist
\item
  \textbf{Sparsity}: મોટાભાગના vector elements zero છે
\item
  \textbf{કોઈ semantics નથી}: સમાન words ના અલગ vectors
\item
  \textbf{High dimensions}: Memory અને computation intensive
\end{itemize}

\end{solutionbox}
\begin{mnemonicbox}
``Sparse, કોઈ Semantics નથી, High Dimensions''

\end{mnemonicbox}
\subsection*{પ્રશ્ન 5(ક) [7
ગુણ]}\label{uxaaauxab0uxab6uxaa8-5uxa95-7-uxa97uxaa3}

\textbf{NLP ની ઉપયોગીતાઓ યોગ્ય ઉદાહરણ સાથે સમજાવો.}

\begin{solutionbox}

{\def\LTcaptype{none} % do not increment counter
\begin{longtable}[]{@{}
  >{\raggedright\arraybackslash}p{(\linewidth - 4\tabcolsep) * \real{0.4333}}
  >{\raggedright\arraybackslash}p{(\linewidth - 4\tabcolsep) * \real{0.2667}}
  >{\raggedright\arraybackslash}p{(\linewidth - 4\tabcolsep) * \real{0.3000}}@{}}
\toprule\noalign{}
\begin{minipage}[b]{\linewidth}\raggedright
Application
\end{minipage} & \begin{minipage}[b]{\linewidth}\raggedright
વર્ણન
\end{minipage} & \begin{minipage}[b]{\linewidth}\raggedright
ઉદાહરણ
\end{minipage} \\
\midrule\noalign{}
\endhead
\bottomrule\noalign{}
\endlastfoot
\textbf{Machine Translation} & ભાષાઓ વચ્ચે translate કરવું & Google
Translate \\
\textbf{Sentiment Analysis} & Emotional tone નક્કી કરવું & Product review
analysis \\
\textbf{Question Answering} & Text માંથી પ્રશ્નોના જવાબ આપવા & Chatbots,
virtual assistants \\
\textbf{Spam Detection} & અનિચ્છિત emails identify કરવા & Email
filters \\
\textbf{Spelling Correction} & Spelling errors ઠીક કરવા & Text editors
માં auto-correct \\
\end{longtable}
}

\begin{center}
\textbf{Mermaid Diagram (Code)}
\begin{verbatim}
{Shaded}
{Highlighting}[]
graph TD
    A[NLP Applications] {-{-}{} B[Machine Translation]}
    A {-{-}{} C[Sentiment Analysis]}
    A {-{-}{} D[Question Answering]}
    A {-{-}{} E[Spam Detection]}
    A {-{-}{} F[Spelling Correction]}
{Highlighting}
{Shaded}
\end{verbatim}
\end{center}

\begin{itemize}
\tightlist
\item
  \textbf{Real-world impact}: Human-computer interaction સુધારે છે
\item
  \textbf{Business value}: Text processing tasks automate કરે છે
\item
  \textbf{વધતું ક્ષેત્ર}: નવા applications સતત આવતા રહે છે
\end{itemize}

\end{solutionbox}
\begin{mnemonicbox}
``Translate, Sentiment, Question, Spam, Spell''

\end{mnemonicbox}
\subsection*{પ્રશ્ન 5(અ) અથવા [3
ગુણ]}\label{uxaaauxab0uxab6uxaa8-5uxa85-uxa85uxaa5uxab5-3-uxa97uxaa3}

\textbf{Glove(Global Vector for word representation) ને વર્ણવો.}

\begin{solutionbox}

{\def\LTcaptype{none} % do not increment counter
\begin{longtable}[]{@{}
  >{\raggedright\arraybackslash}p{(\linewidth - 2\tabcolsep) * \real{0.4286}}
  >{\raggedright\arraybackslash}p{(\linewidth - 2\tabcolsep) * \real{0.5714}}@{}}
\toprule\noalign{}
\begin{minipage}[b]{\linewidth}\raggedright
પાસું
\end{minipage} & \begin{minipage}[b]{\linewidth}\raggedright
વર્ણન
\end{minipage} \\
\midrule\noalign{}
\endhead
\bottomrule\noalign{}
\endlastfoot
\textbf{હેતુ} & Global corpus statistics વાપરીને word vectors બનાવવા \\
\textbf{પદ્ધતિ} & Global matrix factorization અને local context combine કરે
છે \\
\textbf{ફાયદો} & Global અને local બંને statistical information capture કરે
છે \\
\end{longtable}
}

\begin{itemize}
\tightlist
\item
  \textbf{Global statistics}: Word co-occurrence information વાપરે છે
\item
  \textbf{Pre-trained}: સામાન્ય ઉપયોગ માટે trained vectors ઉપલબ્ધ છે
\end{itemize}

\end{solutionbox}
\begin{mnemonicbox}
``Global Vectors, Local Context''

\end{mnemonicbox}
\subsection*{પ્રશ્ન 5(બ) અથવા [4
ગુણ]}\label{uxaaauxab0uxab6uxaa8-5uxaac-uxa85uxaa5uxab5-4-uxa97uxaa3}

\textbf{Inverse Document Frequency (IDF) સમજાવો.}

\begin{solutionbox}

{\def\LTcaptype{none} % do not increment counter
\begin{longtable}[]{@{}lll@{}}
\toprule\noalign{}
Component & Formula & હેતુ \\
\midrule\noalign{}
\endhead
\bottomrule\noalign{}
\endlastfoot
\textbf{IDF} & log(N/df) & Documents માં word importance measure કરવું \\
\textbf{N} & Total documents & Corpus size \\
\textbf{df} & Document frequency & Term containing documents \\
\end{longtable}
}

\begin{itemize}
\tightlist
\item
  \textbf{High IDF}: દુર્લભ words (વધુ informative)
\item
  \textbf{Low IDF}: સામાન્ય words (ઓછા informative)
\item
  \textbf{Application}: TF-IDF weighting scheme નો ભાગ
\end{itemize}

\end{solutionbox}
\begin{mnemonicbox}
``Inverse Document, દુર્લભ મહત્વપૂર્ણ છે''

\end{mnemonicbox}
\subsection*{પ્રશ્ન 5(ક) અથવા [7
ગુણ]}\label{uxaaauxab0uxab6uxaa8-5uxa95-uxa85uxaa5uxab5-7-uxa97uxaa3}

\textbf{Document માટે TF(Term Frequency) ગણવાનું યોગ્ય ઉદાહરણ સાથે સમજાવો.}

\begin{solutionbox}

{\def\LTcaptype{none} % do not increment counter
\begin{longtable}[]{@{}
  >{\raggedright\arraybackslash}p{(\linewidth - 4\tabcolsep) * \real{0.3200}}
  >{\raggedright\arraybackslash}p{(\linewidth - 4\tabcolsep) * \real{0.3600}}
  >{\raggedright\arraybackslash}p{(\linewidth - 4\tabcolsep) * \real{0.3200}}@{}}
\toprule\noalign{}
\begin{minipage}[b]{\linewidth}\raggedright
પદ્ધતિ
\end{minipage} & \begin{minipage}[b]{\linewidth}\raggedright
Formula
\end{minipage} & \begin{minipage}[b]{\linewidth}\raggedright
વર્ણન
\end{minipage} \\
\midrule\noalign{}
\endhead
\bottomrule\noalign{}
\endlastfoot
\textbf{Raw TF} & f(t,d) & Document માં term ની સરળ count \\
\textbf{Normalized TF} & f(t,d)/max(f(w,d)) & Maximum frequency દ્વારા
normalized \\
\textbf{Log TF} & 1 + log(f(t,d)) & Logarithmic scaling \\
\end{longtable}
}

\textbf{ઉદાહરણ Document:} ``The cat sat on the mat. The mat was soft.''

{\def\LTcaptype{none} % do not increment counter
\begin{longtable}[]{@{}lllll@{}}
\toprule\noalign{}
Term & Count & Raw TF & Normalized TF & Log TF \\
\midrule\noalign{}
\endhead
\bottomrule\noalign{}
\endlastfoot
``the'' & 3 & 3 & 1.0 & 1.48 \\
``cat'' & 1 & 1 & 0.33 & 1.0 \\
``mat'' & 2 & 2 & 0.67 & 1.30 \\
\end{longtable}
}

\textbf{ગણતરીના પગલાં:}

\begin{enumerate}
\tightlist
\item
  દરેક term ની occurrence count કરો
\item
  પસંદ કરેલું TF formula લાગુ કરો
\item
  TF-IDF calculation માં વાપરો
\end{enumerate}

\begin{itemize}
\tightlist
\item
  \textbf{Raw TF}: સીધી counting, સરળ પરંતુ મર્યાદિત
\item
  \textbf{Normalized TF}: Document length ના લીધે bias ઘટાડે છે
\item
  \textbf{Log TF}: Frequency differences ને સમાન કરે છે
\end{itemize}

\end{solutionbox}
\begin{mnemonicbox}
``Count, Normalize, Log''

\end{mnemonicbox}

\end{document}
