\documentclass[10pt,a4paper]{article}

% content/resources/templates/preamble.tex
\usepackage[margin=0.6in]{geometry}
\author{Milav Dabgar}
\usepackage{amsmath,amssymb,amsthm}
\usepackage{booktabs}
\usepackage{multirow}
\usepackage{xcolor}
\usepackage{tcolorbox}
\tcbuselibrary{breakable,skins}
\usepackage[colorlinks=true,linkcolor=blue]{hyperref}
\usepackage{titlesec}
\usepackage{enumitem}
\usepackage{tikz}
\usepackage{pgfplots}
\usepackage{circuitikz}
\usepackage[version=4]{mhchem}
\usepackage{longtable}
\usepackage{array}
\usepackage{float}
\usepackage{caption}
\usepackage{listings}

\lstset{
  basicstyle=\small\ttfamily,
  breaklines=true,
  breakatwhitespace=false,
  postbreak=\mbox{\textcolor{red}{$\hookrightarrow$}\space},
  float=false,
  numbers=left,
  numberstyle=\tiny\color{gray},
  numbersep=10pt,
  xleftmargin=2em,
  keywordstyle=\color{blue},
  commentstyle=\color{green!60!black},
  stringstyle=\color{purple},
  backgroundcolor=\color{gray!5},
  showstringspaces=false,
  tabsize=2,
  captionpos=b,
  keepspaces=true,
  columns=flexible
}

\pgfplotsset{compat=1.18}
\usetikzlibrary{shapes,arrows,positioning,calc,patterns,decorations.pathmorphing,decorations.markings,arrows.meta}

% Color scheme
\definecolor{headcolor}{RGB}{0,102,204}
\definecolor{keycolor}{RGB}{220,20,60}
\definecolor{solutioncolor}{RGB}{34,139,34}
\definecolor{mnemoniccolor}{RGB}{148,0,211}
\definecolor{codecolor}{RGB}{0,0,100}

% Spacing
\setlength{\parskip}{3pt}
\setlist[itemize]{nosep}
\setlist[enumerate]{nosep}

% Title formatting
\titleformat{\section}{\Large\bfseries\color{headcolor}}{\thesection}{1em}{}
\titleformat{\subsection}{\large\bfseries\color{headcolor}}{\thesubsection}{1em}{}

% Pandoc tightlist compatibility
\providecommand{\tightlist}{%
  \setlength{\itemsep}{0pt}\setlength{\parskip}{0pt}}

% Pandoc longtable compatibility
\newcounter{none}
\def\thenone{}


% content/resources/templates/gujarati-boxes.tex
\usepackage{fontspec}
\usepackage{polyglossia}

% Set Gujarati as main language (document is primarily in Gujarati)
% Note: gloss-gujarati.ldf doesn't exist in polyglossia, but it will use hyphenation patterns
\setdefaultlanguage{gujarati}
\setotherlanguage{english}

% Configure Gujarati font properly
% Use Language=Default to prevent polyglossia from trying to add language-specific features
% that don't exist for Gujarati, which causes "empty feature" warnings
\newfontfamily\gujaratifont[Script=Gujarati,AutoFakeBold=2.5,AutoFakeSlant=0.3]{Noto Sans Gujarati}
\setmainfont[Script=Gujarati,AutoFakeBold=2.5,AutoFakeSlant=0.3]{Noto Sans Gujarati}
% Use Noto Sans Gujarati for monospace to support Gujarati in text
\setmonofont[Scale=0.9]{Noto Sans Gujarati}

% Configure English to use the same font
\newfontfamily\englishfont[Script=Gujarati,AutoFakeBold=2.5,AutoFakeSlant=0.3]{Noto Sans Gujarati}

% Translations for polyglossia
\gappto\captionsgujarati{
  \renewcommand{\tablename}{કોષ્ટક}
  \renewcommand{\figurename}{આકૃતિ}
}

% Helper for TikZ nodes to ensure Gujarati font
\newcommand{\gu}[1]{{\gujaratifont #1}}

% Custom environments
\newtcolorbox{solutionbox}{
    breakable,
    enhanced,
    colback=solutioncolor!5!white,
    colframe=solutioncolor!75!black,
    fonttitle=\bfseries,
    title=જવાબ
}

\newtcolorbox{solutionboxnobreak}{
 colback=solutioncolor!5!white,
 colframe=solutioncolor!75!black,
 fonttitle=\bfseries,
 title=જવાબ
}

\newtcolorbox{keyformula}{
 breakable,
 enhanced,
 colback=keycolor!5!white,
 colframe=keycolor!75!black,
 fonttitle=\bfseries,
 title=રાસાયણિક સમીકરણ/સૂત્ર
}

\newtcolorbox{mnemonicbox}{
 breakable,
 enhanced,
 colback=mnemoniccolor!5!white,
 colframe=mnemoniccolor!75!black,
 fonttitle=\bfseries,
 title=મેમરી ટ્રીક
}


\begin{document}

\begin{center}
{\Huge\bfseries\color{headcolor} Subject Name (Gujarati)}\\[5pt]
{\LARGE 4351601 -- Winter 2024}\\[3pt]
{\large Semester 1 Study Material}\\[3pt]
{\normalsize\textit{Detailed Solutions and Explanations}}
\end{center}

\vspace{10pt}

\subsection*{પ્રશ્ન 1(અ) [3
ગુણ]}\label{uxaaauxab0uxab6uxaa8-1uxa85-3-uxa97uxaa3}

\textbf{નીચેના પदોને વ્યાખ્યાયિત કરો: 1) Fuzzy Logic. 2) Expert System.}

\begin{solutionbox}

{\def\LTcaptype{none} % do not increment counter
\begin{longtable}[]{@{}
  >{\raggedright\arraybackslash}p{(\linewidth - 2\tabcolsep) * \real{0.4167}}
  >{\raggedright\arraybackslash}p{(\linewidth - 2\tabcolsep) * \real{0.5833}}@{}}
\toprule\noalign{}
\begin{minipage}[b]{\linewidth}\raggedright
પદ
\end{minipage} & \begin{minipage}[b]{\linewidth}\raggedright
વ્યાખ્યા
\end{minipage} \\
\midrule\noalign{}
\endhead
\bottomrule\noalign{}
\endlastfoot
\textbf{Fuzzy Logic} & અસ્પષ્ટ તર્કશાસ્ત્ર જે 0 અને 1 વચ્ચે સત્યતાની ડિગ્રી સાથે
અંદાજિત તર્ક કરે છે \\
\textbf{Expert System} & કુશળ માનવીના નિર્ણયોની નકલ કરતી AI પ્રોગ્રામ જે
knowledge base અને inference engine વાપરે છે \\
\end{longtable}
}

\begin{itemize}
\tightlist
\item
  \textbf{મુખ્ય લક્ષણો}: બંને અનિશ્ચિતતા અને અધૂરી માહિતી સંભાળે છે
\item
  \textbf{ઉપયોગો}: મેડિકલ નિદાન, ઔદ્યોગિક નિયંત્રણ સિસ્ટમ
\end{itemize}

\end{solutionbox}
\begin{mnemonicbox}
``અસ્પષ્ટ કુશળતા અનિશ્ચિત નિર્ણયો લે છે''

\end{mnemonicbox}
\begin{center}\rule{0.5\linewidth}{0.5pt}\end{center}

\subsection*{પ્રશ્ન 1(બ) [4
ગુણ]}\label{uxaaauxab0uxab6uxaa8-1uxaac-4-uxa97uxaa3}

\textbf{નીચેના પદોને વ્યાખ્યાયિત કરો: 1) Machine Learning. 2) Reinforcement
Learning.}

\begin{solutionbox}

{\def\LTcaptype{none} % do not increment counter
\begin{longtable}[]{@{}
  >{\raggedright\arraybackslash}p{(\linewidth - 4\tabcolsep) * \real{0.1786}}
  >{\raggedright\arraybackslash}p{(\linewidth - 4\tabcolsep) * \real{0.2500}}
  >{\raggedright\arraybackslash}p{(\linewidth - 4\tabcolsep) * \real{0.5714}}@{}}
\toprule\noalign{}
\begin{minipage}[b]{\linewidth}\raggedright
પદ
\end{minipage} & \begin{minipage}[b]{\linewidth}\raggedright
વ્યાખ્યા
\end{minipage} & \begin{minipage}[b]{\linewidth}\raggedright
મુખ્ય લાક્ષણિકતા
\end{minipage} \\
\midrule\noalign{}
\endhead
\bottomrule\noalign{}
\endlastfoot
\textbf{Machine Learning} & અલ્ગોરિધમ જે અનુભવ દ્વારા આપોઆપ પ્રદર્શન સુધારે છે
સ્પષ્ટ programming વિના & ડેટા પેટર્નમાંથી શીખવું \\
\textbf{Reinforcement Learning} & Agent પુરસ્કાર/દંડ વાપરીને પર્યાવરણ સાથે
trial-and-error દ્વારા શ્રેષ્ઠ ક્રિયાઓ શીખે છે & feedback દ્વારા શીખવું \\
\end{longtable}
}

\textbf{આકૃતિ:}

\begin{center}
\textbf{Mermaid Diagram (Code)}
\begin{verbatim}
{Shaded}
{Highlighting}[]
graph LR
    A[ડેટા] {-{-}{} B[ML Algorithm] {-}{-}{} C[Model] {-}{-}{} D[આગાહીઓ]}
    E[પર્યાવરણ] {-{-}{} F[RL Agent] {-}{-}{} G[ક્રિયાઓ] {-}{-}{} E}
    E {-{-}{} H[પુરસ્કારો] {-}{-}{} F}
{Highlighting}
{Shaded}
\end{verbatim}
\end{center}

\end{solutionbox}
\begin{mnemonicbox}
``ML ડેટામાંથી શીખે, RL પુરસ્કારોમાંથી શીખે''

\end{mnemonicbox}
\begin{center}\rule{0.5\linewidth}{0.5pt}\end{center}

\subsection*{પ્રશ્ન 1(ક) [7
ગુણ]}\label{uxaaauxab0uxab6uxaa8-1uxa95-7-uxa97uxaa3}

\textbf{Artificial Intelligence ના પ્રકારો વિશે વિગતવાર સમજૂતી યોગ્ય રેખાકૃતિ
સાથે આપો.}

\begin{solutionbox}

\textbf{AI પ્રકારોનું ટેબલ:}

{\def\LTcaptype{none} % do not increment counter
\begin{longtable}[]{@{}
  >{\raggedright\arraybackslash}p{(\linewidth - 6\tabcolsep) * \real{0.2143}}
  >{\raggedright\arraybackslash}p{(\linewidth - 6\tabcolsep) * \real{0.2143}}
  >{\raggedright\arraybackslash}p{(\linewidth - 6\tabcolsep) * \real{0.2500}}
  >{\raggedright\arraybackslash}p{(\linewidth - 6\tabcolsep) * \real{0.3214}}@{}}
\toprule\noalign{}
\begin{minipage}[b]{\linewidth}\raggedright
પ્રકાર
\end{minipage} & \begin{minipage}[b]{\linewidth}\raggedright
વર્ણન
\end{minipage} & \begin{minipage}[b]{\linewidth}\raggedright
ક્ષમતા
\end{minipage} & \begin{minipage}[b]{\linewidth}\raggedright
ઉદાહરણો
\end{minipage} \\
\midrule\noalign{}
\endhead
\bottomrule\noalign{}
\endlastfoot
\textbf{Narrow AI} & વિશિષ્ટ કાર્યો માટે રચાયેલ & મર્યાદિત ડોમેન કુશળતા & Siri,
Chess programs \\
\textbf{General AI} & બધા ક્ષેત્રોમાં માનવ સ્તરની બુદ્ધિ & બહુ-ડોમેન તર્ક & હાલમાં
સૈદ્ધાંતિક \\
\textbf{Super AI} & માનવ બુદ્ધિ કરતાં વધુ & માનવ ક્ષમતાથી આગળ & ભવિષ્યની
કલ્પના \\
\end{longtable}
}

\textbf{આકૃતિ:}

\begin{center}
\textbf{Mermaid Diagram (Code)}
\begin{verbatim}
{Shaded}
{Highlighting}[]
graph TD
    A[Artificial Intelligence] {-{-}{} B[Narrow AI{}br/{}Weak AI]}
    A {-{-}{} C[General AI{}br/{}Strong AI]}
    A {-{-}{} D[Super AI]}
    
    B {-{-}{} E[કાર્ય{-}વિશિષ્ટ{}br/{}વર્તમાન વાસ્તવિકતા]}
    C {-{-}{} F[માનવ{-}સ્તર{}br/{}ભવિષ્યનું લક્ષ્ય]}
    D {-{-}{} G[માનવથી આગળ{}br/{}સૈદ્ધાંતિક]}
{Highlighting}
{Shaded}
\end{verbatim}
\end{center}

\begin{itemize}
\tightlist
\item
  \textbf{વર્તમાન સ્થિતિ}: આપણે Narrow AI યુગમાં છીએ
\item
  \textbf{વિકાસ માર્ગ}: Narrow \rightarrow General \rightarrow Super AI
\item
  \textbf{સમયમર્યાદા}: General AI 20-30 વર્ષમાં અપેક્ષિત
\end{itemize}

\end{solutionbox}
\begin{mnemonicbox}
``સાંકડું હવે, સામાન્ય લક્ષ્ય, સુપર શીઘ્ર''

\end{mnemonicbox}
\begin{center}\rule{0.5\linewidth}{0.5pt}\end{center}

\subsection*{પ્રશ્ન 1(ક) OR [7
ગુણ]}\label{uxaaauxab0uxab6uxaa8-1uxa95-or-7-uxa97uxaa3}

\textbf{AI system design કરતા સમયે ethics સાથે સંબંધિત વિવિધ પાસાની સમજૂતી
આપો. ઉપરાંત, AI System ની મર્યાદાઓની પણ વિગતવાર સમજૂતી આપો.}

\begin{solutionbox}

\textbf{AI નીતિશાસ્ત્ર ટેબલ:}

{\def\LTcaptype{none} % do not increment counter
\begin{longtable}[]{@{}lll@{}}
\toprule\noalign{}
નૈતિક પાસું & વર્ણન & અમલીકરણ \\
\midrule\noalign{}
\endhead
\bottomrule\noalign{}
\endlastfoot
\textbf{Fairness} & પક્ષપાત અને ભેદભાવ ટાળવું & વિવિધ training data \\
\textbf{Transparency} & સમજાવી શકાય તેવા AI નિર્ણયો & સ્પષ્ટ algorithms \\
\textbf{Privacy} & યુઝર ડેટાનું રક્ષણ & ડેટા encryption \\
\textbf{Accountability} & AI ક્રિયાઓ માટે જવાબદારી & માનવ દેખરેખ \\
\end{longtable}
}

\textbf{AI મર્યાદાઓ:}

\begin{itemize}
\tightlist
\item
  \textbf{ડેટા પરાધીનતા}: મોટા, ગુણવત્તાયુક્ત datasets જોઈએ
\item
  \textbf{સામાન્ય બુદ્ધિનો અભાવ}: માનવોની જેમ સંદર્ભ સમજી શકતું નથી
\item
  \textbf{નાજુકતા}: અનપેક્ષિત પરિસ્થિતિઓમાં નિષ્ફળ જાય છે
\item
  \textbf{Black Box સમસ્યા}: નિર્ણયો સમજાવવા મુશ્કેલ
\end{itemize}

\end{solutionbox}
\begin{mnemonicbox}
``ન્યાયી, પારદર્શક, ખાનગી, જવાબદાર AI ને ડેટા, સામાન્ય
બુદ્ધિ, નાજુકતા, કાળા બોક્સની સમસ્યાઓ છે''

\end{mnemonicbox}
\begin{center}\rule{0.5\linewidth}{0.5pt}\end{center}

\subsection*{પ્રશ્ન 2(અ) [3
ગુણ]}\label{uxaaauxab0uxab6uxaa8-2uxa85-3-uxa97uxaa3}

\textbf{Reinforcement learning ની લાક્ષણિકતાની યાદી આપો.}

\begin{solutionbox}

{\def\LTcaptype{none} % do not increment counter
\begin{longtable}[]{@{}
  >{\raggedright\arraybackslash}p{(\linewidth - 2\tabcolsep) * \real{0.6316}}
  >{\raggedright\arraybackslash}p{(\linewidth - 2\tabcolsep) * \real{0.3684}}@{}}
\toprule\noalign{}
\begin{minipage}[b]{\linewidth}\raggedright
લાક્ષણિકતા
\end{minipage} & \begin{minipage}[b]{\linewidth}\raggedright
વર્ણન
\end{minipage} \\
\midrule\noalign{}
\endhead
\bottomrule\noalign{}
\endlastfoot
\textbf{Trial-and-Error} & Agent પ્રયોગ દ્વારા શીખે છે \\
\textbf{Reward-Based} & પુરસ્કાર/દંડ દ્વારા feedback \\
\textbf{Sequential Decision Making} & ક્રિયાઓ ભવિષ્યની અવસ્થાઓને અસર કરે છે \\
\textbf{Exploration vs Exploitation} & નવી ક્રિયાઓ અજમાવવા અને જાણીતી સારી
ક્રિયાઓ વાપરવા વચ્ચેનું સંતુલન \\
\end{longtable}
}

\end{solutionbox}
\begin{mnemonicbox}
``પ્રયોગ પુરસ્કાર ક્રમિક શોધ''

\end{mnemonicbox}
\begin{center}\rule{0.5\linewidth}{0.5pt}\end{center}

\subsection*{પ્રશ્ન 2(બ) [4
ગુણ]}\label{uxaaauxab0uxab6uxaa8-2uxaac-4-uxa97uxaa3}

\textbf{Positive reinforcement અને Negative reinforcement સમજાવો.}

\begin{solutionbox}

\textbf{તુલનાત્મક ટેબલ:}

{\def\LTcaptype{none} % do not increment counter
\begin{longtable}[]{@{}
  >{\raggedright\arraybackslash}p{(\linewidth - 6\tabcolsep) * \real{0.2308}}
  >{\raggedright\arraybackslash}p{(\linewidth - 6\tabcolsep) * \real{0.2692}}
  >{\raggedright\arraybackslash}p{(\linewidth - 6\tabcolsep) * \real{0.1923}}
  >{\raggedright\arraybackslash}p{(\linewidth - 6\tabcolsep) * \real{0.3077}}@{}}
\toprule\noalign{}
\begin{minipage}[b]{\linewidth}\raggedright
પ્રકાર
\end{minipage} & \begin{minipage}[b]{\linewidth}\raggedright
વ્યાખ્યા
\end{minipage} & \begin{minipage}[b]{\linewidth}\raggedright
અસર
\end{minipage} & \begin{minipage}[b]{\linewidth}\raggedright
ઉદાહરણ
\end{minipage} \\
\midrule\noalign{}
\endhead
\bottomrule\noalign{}
\endlastfoot
\textbf{Positive Reinforcement} & વર્તન વધારવા માટે આનંદદાયક stimulus ઉમેરવું
& ઇચ્છિત વર્તન વધારે છે & સારા પ્રદર્શન માટે ઇનામ આપવું \\
\textbf{Negative Reinforcement} & વર્તન વધારવા માટે અપ્રિય stimulus દૂર કરવું
& ઇચ્છિત વર્તન વધારે છે & કાર્ય પૂર્ણ થયા પછી alarm બંધ કરવું \\
\end{longtable}
}

\textbf{મુખ્ય તફાવત}: બંને વર્તન વધારે છે, પરંતુ positive પુરસ્કાર ઉમેરે છે જ્યારે
negative સજા દૂર કરે છે.

\end{solutionbox}
\begin{mnemonicbox}
``હકારાત્મક આનંદ ઉમેરે, નકારાત્મક દુખ દૂર કરે''

\end{mnemonicbox}
\begin{center}\rule{0.5\linewidth}{0.5pt}\end{center}

\subsection*{પ્રશ્ન 2(ક) [7
ગુણ]}\label{uxaaauxab0uxab6uxaa8-2uxa95-7-uxa97uxaa3}

\textbf{Supervised learning વિશે વિગતવાર સમજાવો.}

\begin{solutionbox}

\textbf{વ્યાખ્યા}: શીખવાની પદ્ધતિ જે labeled training data માંથી શીખીને નવા
ડેટા પર આગાહીઓ કરે છે.

\textbf{પ્રક્રિયા ટેબલ:}

{\def\LTcaptype{none} % do not increment counter
\begin{longtable}[]{@{}
  >{\raggedright\arraybackslash}p{(\linewidth - 4\tabcolsep) * \real{0.2727}}
  >{\raggedright\arraybackslash}p{(\linewidth - 4\tabcolsep) * \real{0.3182}}
  >{\raggedright\arraybackslash}p{(\linewidth - 4\tabcolsep) * \real{0.4091}}@{}}
\toprule\noalign{}
\begin{minipage}[b]{\linewidth}\raggedright
પગલું
\end{minipage} & \begin{minipage}[b]{\linewidth}\raggedright
વર્ણન
\end{minipage} & \begin{minipage}[b]{\linewidth}\raggedright
ઉદાહરણ
\end{minipage} \\
\midrule\noalign{}
\endhead
\bottomrule\noalign{}
\endlastfoot
\textbf{Training} & Input-output જોડીઓથી algorithm શીખે છે & Email \rightarrow
Spam/Not Spam \\
\textbf{Validation} & અદ્રશ્ય ડેટા પર model ચકાસવું & accuracy તપાસવી \\
\textbf{Prediction} & નવા inputs માટે outputs બનાવવું & નવા emails ને
classify કરવું \\
\end{longtable}
}

\textbf{પ્રકારો:}

\begin{itemize}
\tightlist
\item
  \textbf{Classification}: કેટેગરીઓની આગાહી (spam detection)
\item
  \textbf{Regression}: સતત મૂલ્યોની આગાહી (ઘરના ભાવ)
\end{itemize}

\textbf{આકૃતિ:}

\begin{center}
\textbf{Mermaid Diagram (Code)}
\begin{verbatim}
{Shaded}
{Highlighting}[]
graph LR
    A[Training Data{br/{}X,Y જોડીઓ] {-}{-}{} B[Learning Algorithm] {-}{-}{} C[Model]}
    D[નવો Input X] {-{-}{} C {-}{-}{} E[આગાહી Y]}
{Highlighting}
{Shaded}
\end{verbatim}
\end{center}

\end{solutionbox}
\begin{mnemonicbox}
``દેખરેખ = શિક્ષક સાચા જવાબો આપે છે''

\end{mnemonicbox}
\begin{center}\rule{0.5\linewidth}{0.5pt}\end{center}

\subsection*{પ્રશ્ન 2(અ) OR [3
ગુણ]}\label{uxaaauxab0uxab6uxaa8-2uxa85-or-3-uxa97uxaa3}

\textbf{Human learning માં સામેલ key components ની યાદી આપો.}

\begin{solutionbox}

{\def\LTcaptype{none} % do not increment counter
\begin{longtable}[]{@{}ll@{}}
\toprule\noalign{}
ઘટક & કાર્ય \\
\midrule\noalign{}
\endhead
\bottomrule\noalign{}
\endlastfoot
\textbf{Observation} & પર્યાવરણમાંથી માહિતી એકત્રિત કરવી \\
\textbf{Memory} & અનુભવો સંગ્રહિત અને પુનઃપ્રાપ્ત કરવા \\
\textbf{Practice} & કુશળતા સુધારવા માટે પુનરાવર્તન \\
\textbf{Feedback} & પ્રદર્શન વિશેની માહિતી \\
\end{longtable}
}

\end{solutionbox}
\begin{mnemonicbox}
``નિરીક્ષણ, યાદદાશ્ત, પ્રેક્ટિસ, પ્રતિસાદ''

\end{mnemonicbox}
\begin{center}\rule{0.5\linewidth}{0.5pt}\end{center}

\subsection*{પ્રશ્ન 2(બ) OR [4
ગુણ]}\label{uxaaauxab0uxab6uxaa8-2uxaac-or-4-uxa97uxaa3}

\textbf{Well-posed learning problem વિશે વિગતવાર સમજાવો.}

\begin{solutionbox}

\textbf{વ્યાખ્યા}: સ્પષ્ટ રીતે વ્યાખ્યાયિત કાર્ય, પ્રદર્શન માપદંડ અને અનુભવ સાથેની
શીખવાની સમસ્યા.

\textbf{ઘટકો ટેબલ:}

{\def\LTcaptype{none} % do not increment counter
\begin{longtable}[]{@{}lll@{}}
\toprule\noalign{}
ઘટક & વર્ણન & ઉદાહરણ \\
\midrule\noalign{}
\endhead
\bottomrule\noalign{}
\endlastfoot
\textbf{Task (T)} & સિસ્ટમે શું શીખવું જોઈએ & શતરંજ રમવું \\
\textbf{Performance (P)} & સફળતા કેવી રીતે માપવી & જીતવાની ટકાવારી \\
\textbf{Experience (E)} & Training data અથવા પ્રેક્ટિસ & અગાઉના રમતો \\
\end{longtable}
}

\textbf{સૂત્ર}: શીખવું = E દ્વારા T પર P સુધારવું

\textbf{માપદંડો}: સમસ્યા માપી શકાય તેવી, હાંસલ કરી શકાય તેવી અને ઉપલબ્ધ ડેટા હોવું
જોઈએ.

\end{solutionbox}
\begin{mnemonicbox}
``કાર્ય પ્રદર્શન અનુભવ = શીખવા માટે TPE''

\end{mnemonicbox}
\begin{center}\rule{0.5\linewidth}{0.5pt}\end{center}

\subsection*{પ્રશ્ન 2(ક) OR [7
ગુણ]}\label{uxaaauxab0uxab6uxaa8-2uxa95-or-7-uxa97uxaa3}

\textbf{Unsupervised learning વિશે વિગતવાર સમજાવો.}

\begin{solutionbox}

\textbf{વ્યાખ્યા}: Labeled ઉદાહરણો અથવા target outputs વિના ડેટામાંથી પેટર્ન
શીખવું.

\textbf{પ્રકારો ટેબલ:}

{\def\LTcaptype{none} % do not increment counter
\begin{longtable}[]{@{}llll@{}}
\toprule\noalign{}
પ્રકાર & ઉદ્દેશ્ય & Algorithm & ઉદાહરણ \\
\midrule\noalign{}
\endhead
\bottomrule\noalign{}
\endlastfoot
\textbf{Clustering} & સમાન ડેટાને જૂથમાં રાખવું & K-means & ગ્રાહક વિભાજન \\
\textbf{Association} & સંબંધો શોધવા & Apriori & બજાર બાસ્કેટ વિશ્લેષણ \\
\textbf{Dimensionality Reduction} & લક્ષણો ઘટાડવા & PCA & ડેટા દૃશ્યીકરણ \\
\end{longtable}
}

\textbf{આકૃતિ:}

\begin{center}
\textbf{Mermaid Diagram (Code)}
\begin{verbatim}
{Shaded}
{Highlighting}[]
graph TD
    A[Unlabeled Data] {-{-}{} B[Unsupervised Algorithm]}
    B {-{-}{} C[Clustering]}
    B {-{-}{} D[Association Rules]}
    B {-{-}{} E[Dimensionality Reduction]}
{Highlighting}
{Shaded}
\end{verbatim}
\end{center}

\begin{itemize}
\tightlist
\item
  \textbf{કોઈ શિક્ષક નથી}: Algorithm સ્વતંત્ર રીતે છુપાયેલા patterns શોધે છે
\item
  \textbf{શોધખોળ}: ડેટામાં અજાણ્યા માળખાઓ શોધે છે
\end{itemize}

\end{solutionbox}
\begin{mnemonicbox}
``બિનદેખરેખ = કોઈ શિક્ષક નથી, જાતે patterns શોધો''

\end{mnemonicbox}
\begin{center}\rule{0.5\linewidth}{0.5pt}\end{center}

\subsection*{પ્રશ્ન 3(અ) [3
ગુણ]}\label{uxaaauxab0uxab6uxaa8-3uxa85-3-uxa97uxaa3}

\textbf{SIGMOID function સમજાવો. ઉપરાંત, તેનો graph દોરો અને SIGMOID
function નું ઉદાહરણ આપો.}

\begin{solutionbox}

\textbf{વ્યાખ્યા}: Activation function જે કોઈપણ વાસ્તવિક સંખ્યાને 0 અને 1 વચ્ચેના
મૂલ્યમાં map કરે છે.

\textbf{સૂત્ર}: σ(x) = 1/(1 + e\^{}(-x))

\textbf{Graph (ASCII):}

\begin{verbatim}
    1 |     .\_{-}
      |   .{-}
    0.5|.{-}
      |{}
    0 +{-{-}{-}{-}{-}{-}{-}{-}{-}{-}}
     {-5  0   5}
\end{verbatim}

\textbf{ઉદાહરણ}: x = 0 માટે, σ(0) = 1/(1 + e\^{}0) = 1/2 = 0.5

\textbf{ગુણધર્મો}: S-આકારનો વળાંક, સરળ gradient, binary classification માં
વપરાય છે

\end{solutionbox}
\begin{mnemonicbox}
``Sigmoid મૂલ્યોને 0 અને 1 વચ્ચે દબાવે છે''

\end{mnemonicbox}
\begin{center}\rule{0.5\linewidth}{0.5pt}\end{center}

\subsection*{પ્રશ્ન 3(બ) [4
ગુણ]}\label{uxaaauxab0uxab6uxaa8-3uxaac-4-uxa97uxaa3}

\textbf{નીચેના પદને વ્યાખ્યાયિત કરો: 1) Activation function. 2) Artificial
neural network.}

\begin{solutionbox}

{\def\LTcaptype{none} % do not increment counter
\begin{longtable}[]{@{}
  >{\raggedright\arraybackslash}p{(\linewidth - 4\tabcolsep) * \real{0.2500}}
  >{\raggedright\arraybackslash}p{(\linewidth - 4\tabcolsep) * \real{0.3500}}
  >{\raggedright\arraybackslash}p{(\linewidth - 4\tabcolsep) * \real{0.4000}}@{}}
\toprule\noalign{}
\begin{minipage}[b]{\linewidth}\raggedright
પદ
\end{minipage} & \begin{minipage}[b]{\linewidth}\raggedright
વ્યાખ્યા
\end{minipage} & \begin{minipage}[b]{\linewidth}\raggedright
ઉદ્દેશ્ય
\end{minipage} \\
\midrule\noalign{}
\endhead
\bottomrule\noalign{}
\endlastfoot
\textbf{Activation Function} & ગાણિતિક function જે weighted inputs આધારે
neuron output નક્કી કરે છે & Neural networks માં non-linearity લાવે છે \\
\textbf{Artificial Neural Network} & Biological neural networks થી પ્રેરિત
computing system જેમાં interconnected nodes હોય છે & Pattern recognition અને
machine learning \\
\end{longtable}
}

\textbf{મુખ્ય લક્ષણો:}

\begin{itemize}
\tightlist
\item
  \textbf{Non-linear processing} જટિલ pattern learning સક્ષમ બનાવે છે
\item
  \textbf{Layered architecture} માહિતીને hierarchical રીતે process કરે છે
\end{itemize}

\end{solutionbox}
\begin{mnemonicbox}
``Activation કૃત્રિમ રીતે મગજના neurons ની નકલ કરે છે''

\end{mnemonicbox}
\begin{center}\rule{0.5\linewidth}{0.5pt}\end{center}

\subsection*{પ્રશ્ન 3(ક) [7
ગુણ]}\label{uxaaauxab0uxab6uxaa8-3uxa95-7-uxa97uxaa3}

\textbf{Recurrent network ના architecture ને આકૃતિ સાથે વિગતવાર સમજાવો.}

\begin{solutionbox}

\textbf{વ્યાખ્યા}: Neural network જેમાં connections loops બનાવે છે, જે માહિતીને
સ્થાયી રાખવાની મંજૂરી આપે છે.

\textbf{Architecture આકૃતિ:}

\begin{center}
\textbf{Mermaid Diagram (Code)}
\begin{verbatim}
{Shaded}
{Highlighting}[]
graph LR
    A[Input x\_t] {-{-}{} B[Hidden State h\_t]}
    B {-{-}{} C[Output y\_t]}
    B {-{-}{} D[Hidden State h\_t+1]}
    E[Previous State h\_t{-1] {-}{-}{} B}
    
    subgraph "Time Steps"
    direction LR
    F[t{-1] {-}{-}{} G[t] {-}{-}{} H[t+1]}
    end
{Highlighting}
{Shaded}
\end{verbatim}
\end{center}

\textbf{ઘટકો ટેબલ:}

{\def\LTcaptype{none} % do not increment counter
\begin{longtable}[]{@{}
  >{\raggedright\arraybackslash}p{(\linewidth - 4\tabcolsep) * \real{0.3125}}
  >{\raggedright\arraybackslash}p{(\linewidth - 4\tabcolsep) * \real{0.3125}}
  >{\raggedright\arraybackslash}p{(\linewidth - 4\tabcolsep) * \real{0.3750}}@{}}
\toprule\noalign{}
\begin{minipage}[b]{\linewidth}\raggedright
ઘટક
\end{minipage} & \begin{minipage}[b]{\linewidth}\raggedright
કાર્ય
\end{minipage} & \begin{minipage}[b]{\linewidth}\raggedright
સૂત્ર
\end{minipage} \\
\midrule\noalign{}
\endhead
\bottomrule\noalign{}
\endlastfoot
\textbf{Hidden State} & અગાઉના inputs ની યાદદાશ્ત & h\_t = f(W\_h
\emph{h\_t-1 + W\_x} x\_t) \\
\textbf{Input Layer} & વર્તમાન time step input & x\_t \\
\textbf{Output Layer} & સમય t પર આગાહી & y\_t = W\_y * h\_t \\
\end{longtable}
}

\textbf{ઉપયોગો}: વાણી ઓળખ, ભાષા અનુવાદ, time series આગાહી

\textbf{ફાયદો}: ભૂતકાળની માહિતીની યાદદાશ્ત સાથે sequential data handle કરે છે

\end{solutionbox}
\begin{mnemonicbox}
``પુનરાવર્તિત = પાછલી અવસ્થાઓ યાદ રાખે છે''

\end{mnemonicbox}
\begin{center}\rule{0.5\linewidth}{0.5pt}\end{center}

\subsection*{પ્રશ્ન 3(અ) OR [3
ગુણ]}\label{uxaaauxab0uxab6uxaa8-3uxa85-or-3-uxa97uxaa3}

\textbf{TANH function સમજાવો. ઉપરાંત, તેનો graph દોરો અને TANH function નું
ઉદાહરણ આપો.}

\begin{solutionbox}

\textbf{વ્યાખ્યા}: Hyperbolic tangent activation function જે મૂલ્યોને -1 અને 1
વચ્ચે map કરે છે.

\textbf{સૂત્ર}: tanh(x) = (e\^{}x - e\textsuperscript{(-x))/(e}x +
e\^{}(-x))

\textbf{Graph (ASCII):}

\begin{verbatim}
    1 |      .\_{-}
      |    .{-}
    0 +.{-{-}{-}{-}{-}{-}{-}{-}}
      |.{-}
   {-1 |}
      +{-{-}{-}{-}{-}{-}{-}{-}{-}{-}}
     {-3  0   3}
\end{verbatim}

\textbf{ઉદાહરણ}: x = 0 માટે, tanh(0) = (1-1)/(1+1) = 0

\textbf{ગુણધર્મો}: શૂન્ય-કેન્દ્રિત, S-આકારનું, sigmoid કરતાં મજબૂત gradients

\end{solutionbox}
\begin{mnemonicbox}
``TANH = બે-તરફી sigmoid (-1 થી +1)''

\end{mnemonicbox}
\begin{center}\rule{0.5\linewidth}{0.5pt}\end{center}

\subsection*{પ્રશ્ન 3(બ) OR [4
ગુણ]}\label{uxaaauxab0uxab6uxaa8-3uxaac-or-4-uxa97uxaa3}

\textbf{નીચેના પદને વ્યાખ્યાયિત કરો: 1) Biological neural network. 2) Loss
calculation.}

\begin{solutionbox}

{\def\LTcaptype{none} % do not increment counter
\begin{longtable}[]{@{}
  >{\raggedright\arraybackslash}p{(\linewidth - 4\tabcolsep) * \real{0.2000}}
  >{\raggedright\arraybackslash}p{(\linewidth - 4\tabcolsep) * \real{0.2800}}
  >{\raggedright\arraybackslash}p{(\linewidth - 4\tabcolsep) * \real{0.5200}}@{}}
\toprule\noalign{}
\begin{minipage}[b]{\linewidth}\raggedright
પદ
\end{minipage} & \begin{minipage}[b]{\linewidth}\raggedright
વ્યાખ્યા
\end{minipage} & \begin{minipage}[b]{\linewidth}\raggedright
મુખ્ય પાસાઓ
\end{minipage} \\
\midrule\noalign{}
\endhead
\bottomrule\noalign{}
\endlastfoot
\textbf{Biological Neural Network} & જીવંત જીવોમાં interconnected neurons
નું નેટવર્ક જે માહિતી process કરે છે & Dendrites, cell body, axon, synapses \\
\textbf{Loss calculation} & આગાહી કરેલા અને વાસ્તવિક outputs વચ્ચેના તફાવતનું
ગાણિતિક માપ & Backpropagation દ્વારા શીખવાને માર્ગદર્શન આપે છે \\
\end{longtable}
}

\textbf{જૈવિક માળખું}: Neurons \rightarrow Synapses \rightarrow Neural Networks \rightarrow મગજ
\textbf{Loss પ્રકારો}: Mean Squared Error, Cross-entropy, Absolute Error

\end{solutionbox}
\begin{mnemonicbox}
``જીવવિજ્ઞાન AI ને પ્રેરણા આપે છે, Loss શીખવાની પ્રગતિ માપે
છે''

\end{mnemonicbox}
\begin{center}\rule{0.5\linewidth}{0.5pt}\end{center}

\subsection*{પ્રશ્ન 3(ક) OR [7
ગુણ]}\label{uxaaauxab0uxab6uxaa8-3uxa95-or-7-uxa97uxaa3}

\textbf{Multi-layer feed-forward network ના architecture ને આકૃતિ સાથે
વિગતવાર વર્ણવો.}

\begin{solutionbox}

\textbf{વ્યાખ્યા}: બહુવિધ layers સાથેનું neural network જ્યાં માહિતી input થી
output તરફ આગળ વહે છે.

\textbf{Architecture આકૃતિ:}

\begin{center}
\textbf{Mermaid Diagram (Code)}
\begin{verbatim}
{Shaded}
{Highlighting}[]
graph LR
    subgraph "Input Layer"
    A1[x1] 
    A2[x2]
    A3[x3]
    end
    
    subgraph "Hidden Layer 1"
    B1[h1]
    B2[h2]
    B3[h3]
    end
    
    subgraph "Hidden Layer 2"
    C1[h4]
    C2[h5]
    end
    
    subgraph "Output Layer"
    D1[y1]
    D2[y2]
    end
    
    A1 {-{-}{} B1}
    A1 {-{-}{} B2}
    A2 {-{-}{} B1}
    A2 {-{-}{} B3}
    A3 {-{-}{} B2}
    A3 {-{-}{} B3}
    
    B1 {-{-}{} C1}
    B2 {-{-}{} C1}
    B2 {-{-}{} C2}
    B3 {-{-}{} C2}
    
    C1 {-{-}{} D1}
    C1 {-{-}{} D2}
    C2 {-{-}{} D1}
    C2 {-{-}{} D2}
{Highlighting}
{Shaded}
\end{verbatim}
\end{center}

\textbf{Layer કાર્યો ટેબલ:}

{\def\LTcaptype{none} % do not increment counter
\begin{longtable}[]{@{}lll@{}}
\toprule\noalign{}
Layer & કાર્ય & Processing \\
\midrule\noalign{}
\endhead
\bottomrule\noalign{}
\endlastfoot
\textbf{Input} & ડેટા પ્રાપ્ત કરે છે & કોઈ processing નથી, ફક્ત વિતરણ \\
\textbf{Hidden} & Feature extraction & Weighted sum + activation
function \\
\textbf{Output} & અંતિમ આગાહી & Classification અથવા regression output \\
\end{longtable}
}

\textbf{માહિતી પ્રવાહ}: Input \rightarrow Hidden Layer(s) \rightarrow Output (એકદિશીય)
\textbf{શીખવું}: Backpropagation error આધારે weights adjust કરે છે

\end{solutionbox}
\begin{mnemonicbox}
``બહુ-સ્તર = જટિલ શીખવા માટે બહુવિધ hidden layers''

\end{mnemonicbox}
\begin{center}\rule{0.5\linewidth}{0.5pt}\end{center}

\subsection*{પ્રશ્ન 4(અ) [3
ગુણ]}\label{uxaaauxab0uxab6uxaa8-4uxa85-3-uxa97uxaa3}

\textbf{NLP ના ફાયદાઓની યાદી વિગતવાર આપો.}

\begin{solutionbox}

{\def\LTcaptype{none} % do not increment counter
\begin{longtable}[]{@{}
  >{\raggedright\arraybackslash}p{(\linewidth - 2\tabcolsep) * \real{0.5000}}
  >{\raggedright\arraybackslash}p{(\linewidth - 2\tabcolsep) * \real{0.5000}}@{}}
\toprule\noalign{}
\begin{minipage}[b]{\linewidth}\raggedright
ફાયદો
\end{minipage} & \begin{minipage}[b]{\linewidth}\raggedright
વર્ણન
\end{minipage} \\
\midrule\noalign{}
\endhead
\bottomrule\noalign{}
\endlastfoot
\textbf{Automation} & માનવી પ્રયાસ જોઈતા text processing કાર્યોને આપોઆપ કરે
છે \\
\textbf{Language Understanding} & બહુવિધ ભાષાઓ અને બોલીઓ અસરકારક રીતે
process કરે છે \\
\textbf{24/7 Availability} & માનવી હસ્તક્ષેપ વિના સતત કામ કરે છે \\
\textbf{Scalability} & મોટા પ્રમાણમાં text data કાર્યક્ષમ રીતે handle કરે છે \\
\end{longtable}
}

\textbf{ઉપયોગો}: Chatbots, અનુવાદ, sentiment analysis, document
processing

\end{solutionbox}
\begin{mnemonicbox}
``NLP = ભાષા સમજણને 24/7 પાયે આપોઆપ કરે છે''

\end{mnemonicbox}
\begin{center}\rule{0.5\linewidth}{0.5pt}\end{center}

\subsection*{પ્રશ્ન 4(બ) [4
ગુણ]}\label{uxaaauxab0uxab6uxaa8-4uxaac-4-uxa97uxaa3}

\textbf{Natural Language Generation વિગતવાર સમજાવો.}

\begin{solutionbox}

\textbf{વ્યાખ્યા}: AI પ્રક્રિયા જે structured data ને કુદરતી માનવી ભાષાના text
માં convert કરે છે.

\textbf{પ્રક્રિયા ટેબલ:}

{\def\LTcaptype{none} % do not increment counter
\begin{longtable}[]{@{}lll@{}}
\toprule\noalign{}
પગલું & વર્ણન & કાર્ય \\
\midrule\noalign{}
\endhead
\bottomrule\noalign{}
\endlastfoot
\textbf{Content Planning} & કઈ માહિતી સામેલ કરવી તે નક્કી કરવું & ડેટા
પસંદગી \\
\textbf{Sentence Planning} & વાક્યો અને ફકરાઓની રચના કરવી & ટેક્સ્ટ વ્યવસ્થા \\
\textbf{Surface Realization} & વ્યાકરણ સાથે વાસ્તવિક ટેક્સ્ટ બનાવવું & અંતિમ
આઉટપુટ \\
\end{longtable}
}

\textbf{ઉપયોગો}: રિપોર્ટ જનરેશન, chatbots, આપોઆપ પત્રકારત્વ, વ્યક્તિગત સામગ્રી

\textbf{ઉદાહરણ}: વેચાણ ડેટા \rightarrow ``ઇલેક્ટ્રોનિક્સમાં મજબૂત પ્રદર્શનને કારણે આ ક્વાર્ટરમાં
વેચાણ 15\% વધ્યું.''

\end{solutionbox}
\begin{mnemonicbox}
``NLG = સંખ્યાઓને કથામાં ફેરવે છે''

\end{mnemonicbox}
\begin{center}\rule{0.5\linewidth}{0.5pt}\end{center}

\subsection*{પ્રશ્ન 4(ક) [7
ગુણ]}\label{uxaaauxab0uxab6uxaa8-4uxa95-7-uxa97uxaa3}

\textbf{NLP માં રહેલી અસ્પષ્ટતા સમજાવો. ઉપરાંત, દરેક અસ્પષ્ટતાનું ઉદાહરણ આપો.}

\begin{solutionbox}

\textbf{અસ્પષ્ટતા પ્રકારો ટેબલ:}

{\def\LTcaptype{none} % do not increment counter
\begin{longtable}[]{@{}
  >{\raggedright\arraybackslash}p{(\linewidth - 6\tabcolsep) * \real{0.2143}}
  >{\raggedright\arraybackslash}p{(\linewidth - 6\tabcolsep) * \real{0.2500}}
  >{\raggedright\arraybackslash}p{(\linewidth - 6\tabcolsep) * \real{0.3214}}
  >{\raggedright\arraybackslash}p{(\linewidth - 6\tabcolsep) * \real{0.2143}}@{}}
\toprule\noalign{}
\begin{minipage}[b]{\linewidth}\raggedright
પ્રકાર
\end{minipage} & \begin{minipage}[b]{\linewidth}\raggedright
વર્ણન
\end{minipage} & \begin{minipage}[b]{\linewidth}\raggedright
ઉદાહરણ
\end{minipage} & \begin{minipage}[b]{\linewidth}\raggedright
ઉકેલ
\end{minipage} \\
\midrule\noalign{}
\endhead
\bottomrule\noalign{}
\endlastfoot
\textbf{Lexical} & શબ્દના અનેક અર્થો હોય છે & ``Bank'' (નદી/નાણાકીય) & સંદર્ભ
વિશ્લેષણ \\
\textbf{Syntactic} & વાક્ય રચના અસ્પષ્ટ હોય છે & ``મેં telescope સાથે માણસને
જોયો'' & Parse trees \\
\textbf{Semantic} & અર્થ અસ્પષ્ટ હોય છે & ``રંગહીન લીલા વિચારો'' & Semantic
rules \\
\textbf{Pragmatic} & સંદર્ભ-આધારિત અર્થ & ``શું તમે મીઠું આપી શકો છો?''
(વિનંતી/પ્રશ્ન) & પરિસ્થિતિનો સંદર્ભ \\
\end{longtable}
}

\textbf{આકૃતિ:}

\begin{center}
\textbf{Mermaid Diagram (Code)}
\begin{verbatim}
{Shaded}
{Highlighting}[]
graph TD
    A[NLP અસ્પષ્ટતાઓ] {-{-}{} B[Lexical{}br/{}શબ્દ સ્તર]}
    A {-{-}{} C[Syntactic{}br/{}વ્યાકરણ સ્તર]}
    A {-{-}{} D[Semantic{}br/{}અર્થ સ્તર]}
    A {-{-}{} E[Pragmatic{}br/{}સંદર્ભ સ્તર]}
{Highlighting}
{Shaded}
\end{verbatim}
\end{center}

\textbf{ઉકેલ વ્યૂહરચના}: સંદર્ભ વિશ્લેષણ, આંકડાકીય મોડેલ્સ, knowledge bases

\end{solutionbox}
\begin{mnemonicbox}
``શાબ્દિક વ્યાકરણિક અર્થપૂર્ણ વ્યાવહારિક = SVAV અસ્પષ્ટતાઓ''

\end{mnemonicbox}
\begin{center}\rule{0.5\linewidth}{0.5pt}\end{center}

\subsection*{પ્રશ્ન 4(અ) OR [3
ગુણ]}\label{uxaaauxab0uxab6uxaa8-4uxa85-or-3-uxa97uxaa3}

\textbf{NLP ના ગેરફાયદાઓની યાદી વિગતવાર આપો.}

\begin{solutionbox}

{\def\LTcaptype{none} % do not increment counter
\begin{longtable}[]{@{}ll@{}}
\toprule\noalign{}
ગેરફાયદો & વર્ણન \\
\midrule\noalign{}
\endhead
\bottomrule\noalign{}
\endlastfoot
\textbf{Context Limitations} & વ્યંગ, હાસ્ય, સાંસ્કૃતિક સંદર્ભો સાથે મુશ્કેલી \\
\textbf{Language Complexity} & વાક્યપ્રયોગો, સ્લેંગ, પ્રાદેશિક બોલીઓ સાથે
મુશ્કેલી \\
\textbf{Data Requirements} & મોટા પ્રમાણમાં training data જરૂરી \\
\textbf{Computational Cost} & નોંધપાત્ર processing power અને memory જરૂરી \\
\end{longtable}
}

\textbf{પડકારો}: અસ્પષ્ટતા, બહુભાષીય સપોર્ટ, real-time processing

\end{solutionbox}
\begin{mnemonicbox}
``NLP પડકારો = સંદર્ભ, ભાષા, ડેટા, ગણતરી''

\end{mnemonicbox}
\begin{center}\rule{0.5\linewidth}{0.5pt}\end{center}

\subsection*{પ્રશ્ન 4(બ) OR [4
ગુણ]}\label{uxaaauxab0uxab6uxaa8-4uxaac-or-4-uxa97uxaa3}

\textbf{Natural Language Understanding વિગતવાર સમજાવો.}

\begin{solutionbox}

\textbf{વ્યાખ્યા}: માનવી ભાષાના અર્થ અને હેતુને સમજવા અને અર્થઘટન કરવાની AI
ક્ષમતા.

\textbf{ઘટકો ટેબલ:}

{\def\LTcaptype{none} % do not increment counter
\begin{longtable}[]{@{}
  >{\raggedright\arraybackslash}p{(\linewidth - 4\tabcolsep) * \real{0.2632}}
  >{\raggedright\arraybackslash}p{(\linewidth - 4\tabcolsep) * \real{0.2632}}
  >{\raggedright\arraybackslash}p{(\linewidth - 4\tabcolsep) * \real{0.4737}}@{}}
\toprule\noalign{}
\begin{minipage}[b]{\linewidth}\raggedright
ઘટક
\end{minipage} & \begin{minipage}[b]{\linewidth}\raggedright
કાર્ય
\end{minipage} & \begin{minipage}[b]{\linewidth}\raggedright
ઉદાહરણ
\end{minipage} \\
\midrule\noalign{}
\endhead
\bottomrule\noalign{}
\endlastfoot
\textbf{Tokenization} & ટેક્સ્ટને શબ્દો/વાક્યાંશોમાં વિભાજીત કરવું & ``Hello
world'' \rightarrow [``Hello'', ``world''] \\
\textbf{Parsing} & વ્યાકરણિક માળખાનું વિશ્લેષણ & કર્તા, ક્રિયા, કર્મ ઓળખવું \\
\textbf{Semantic Analysis} & અર્થ કાઢવો & શબ્દો વચ્ચેના સંબંધો સમજવા \\
\textbf{Intent Recognition} & યુઝરનો હેતુ ઓળખવો & ``ફ્લાઇટ બુક કરો'' \rightarrow ટ્રાવેલ
બુકિંગ intent \\
\end{longtable}
}

\textbf{પ્રક્રિયા પ્રવાહ}: Text Input \rightarrow Tokenization \rightarrow Parsing \rightarrow Semantic
Analysis \rightarrow Intent Understanding

\textbf{ઉપયોગો}: Virtual assistants, chatbots, voice commands

\end{solutionbox}
\begin{mnemonicbox}
``NLU = કુદરતી રીતે ભાષા સમજે છે''

\end{mnemonicbox}
\begin{center}\rule{0.5\linewidth}{0.5pt}\end{center}

\subsection*{પ્રશ્ન 4(ક) OR [7
ગુણ]}\label{uxaaauxab0uxab6uxaa8-4uxa95-or-7-uxa97uxaa3}

\textbf{Stemming અને Lemmatization વિગતવાર સમજાવો. ઉપરાંત દરેકના બે ઉદાહરણ
આપો.}

\begin{solutionbox}

\textbf{વ્યાખ્યાઓ:}

{\def\LTcaptype{none} % do not increment counter
\begin{longtable}[]{@{}
  >{\raggedright\arraybackslash}p{(\linewidth - 6\tabcolsep) * \real{0.2727}}
  >{\raggedright\arraybackslash}p{(\linewidth - 6\tabcolsep) * \real{0.2121}}
  >{\raggedright\arraybackslash}p{(\linewidth - 6\tabcolsep) * \real{0.2424}}
  >{\raggedright\arraybackslash}p{(\linewidth - 6\tabcolsep) * \real{0.2727}}@{}}
\toprule\noalign{}
\begin{minipage}[b]{\linewidth}\raggedright
પ્રક્રિયા
\end{minipage} & \begin{minipage}[b]{\linewidth}\raggedright
વર્ણન
\end{minipage} & \begin{minipage}[b]{\linewidth}\raggedright
પદ્ધતિ
\end{minipage} & \begin{minipage}[b]{\linewidth}\raggedright
આઉટપુટ
\end{minipage} \\
\midrule\noalign{}
\endhead
\bottomrule\noalign{}
\endlastfoot
\textbf{Stemming} & Suffixes દૂર કરીને શબ્દોને મૂળ સ્વરૂપમાં ઘટાડવું & Rule-based
કાપવું & Word stem \\
\textbf{Lemmatization} & શબ્દોને શબ્દકોશના આધાર સ્વરૂપમાં ઘટાડવું & Dictionary
lookup & માન્ય શબ્દ \\
\end{longtable}
}

\textbf{Stemming ઉદાહરણો:}

\begin{enumerate}
\tightlist
\item
  ``running'', ``runs'', ``ran'' \rightarrow ``run''
\item
  ``fishing'', ``fished'', ``fisher'' \rightarrow ``fish''
\end{enumerate}

\textbf{Lemmatization ઉદાહરણો:}

\begin{enumerate}
\tightlist
\item
  ``better'' \rightarrow ``good'' (comparative to base)
\item
  ``children'' \rightarrow ``child'' (બહુવચનથી એકવચન)
\end{enumerate}

\textbf{તુલના ટેબલ:}

{\def\LTcaptype{none} % do not increment counter
\begin{longtable}[]{@{}lll@{}}
\toprule\noalign{}
પાસું & Stemming & Lemmatization \\
\midrule\noalign{}
\endhead
\bottomrule\noalign{}
\endlastfoot
\textbf{ઝડપ} & વધુ ઝડપી & ધીમું \\
\textbf{સચોટતા} & ઓછી & વધુ \\
\textbf{આઉટપુટ} & કદાચ માન્ય શબ્દ ન હોય & હંમેશા માન્ય શબ્દ \\
\end{longtable}
}

\end{solutionbox}
\begin{mnemonicbox}
``Stemming = ઝડપ, Lemmatization = ભાષાની સચોટતા''

\end{mnemonicbox}
\begin{center}\rule{0.5\linewidth}{0.5pt}\end{center}

\subsection*{પ્રશ્ન 5(અ) [3
ગુણ]}\label{uxaaauxab0uxab6uxaa8-5uxa85-3-uxa97uxaa3}

\textbf{વ્યાખ્યા આપો: 1) Word embeddings. 2) Machine Translation.}

\begin{solutionbox}

{\def\LTcaptype{none} % do not increment counter
\begin{longtable}[]{@{}
  >{\raggedright\arraybackslash}p{(\linewidth - 4\tabcolsep) * \real{0.2500}}
  >{\raggedright\arraybackslash}p{(\linewidth - 4\tabcolsep) * \real{0.3500}}
  >{\raggedright\arraybackslash}p{(\linewidth - 4\tabcolsep) * \real{0.4000}}@{}}
\toprule\noalign{}
\begin{minipage}[b]{\linewidth}\raggedright
પદ
\end{minipage} & \begin{minipage}[b]{\linewidth}\raggedright
વ્યાખ્યા
\end{minipage} & \begin{minipage}[b]{\linewidth}\raggedright
ઉદ્દેશ્ય
\end{minipage} \\
\midrule\noalign{}
\endhead
\bottomrule\noalign{}
\endlastfoot
\textbf{Word Embeddings} & શબ્દોના ઘન વેક્ટર પ્રતિનિધિત્વ જે semantic સંબંધો
capture કરે છે & ટેક્સ્ટને ML માટે સંખ્યાત્મક સ્વરૂપમાં convert કરવું \\
\textbf{Machine Translation} & એક ભાષામાંથી બીજી ભાષામાં ટેક્સ્ટનું આપોઆપ
રૂપાંતરણ & ભાષાઓ વચ્ચે સંવાદ સક્ષમ બનાવવું \\
\end{longtable}
}

\textbf{મુખ્ય લક્ષણો}:

\begin{itemize}
\tightlist
\item
  \textbf{Word embeddings} વેક્ટર સ્પેસમાં શબ્દ સંબંધો જાળવે છે
\item
  \textbf{Machine translation} ભાષાઓ વચ્ચે અર્થ જાળવે છે
\end{itemize}

\end{solutionbox}
\begin{mnemonicbox}
``શબ્દો વેક્ટર્સ બને છે, ભાષાઓ અનુવાદ બને છે''

\end{mnemonicbox}
\begin{center}\rule{0.5\linewidth}{0.5pt}\end{center}

\subsection*{પ્રશ્ન 5(બ) [4
ગુણ]}\label{uxaaauxab0uxab6uxaa8-5uxaac-4-uxa97uxaa3}

\textbf{Word2Vec વિશે વિગતવાર સમજાવો.}

\begin{solutionbox}

\textbf{વ્યાખ્યા}: Neural network તકનીક જે મોટા text corpus માંથી શબ્દ સંબંધો
શીખીને word embeddings બનાવે છે.

\textbf{Architecture પ્રકારો:}

{\def\LTcaptype{none} % do not increment counter
\begin{longtable}[]{@{}
  >{\raggedright\arraybackslash}p{(\linewidth - 4\tabcolsep) * \real{0.2857}}
  >{\raggedright\arraybackslash}p{(\linewidth - 4\tabcolsep) * \real{0.3333}}
  >{\raggedright\arraybackslash}p{(\linewidth - 4\tabcolsep) * \real{0.3810}}@{}}
\toprule\noalign{}
\begin{minipage}[b]{\linewidth}\raggedright
મોડેલ
\end{minipage} & \begin{minipage}[b]{\linewidth}\raggedright
વર્ણન
\end{minipage} & \begin{minipage}[b]{\linewidth}\raggedright
આગાહી
\end{minipage} \\
\midrule\noalign{}
\endhead
\bottomrule\noalign{}
\endlastfoot
\textbf{CBOW (Continuous Bag of Words)} & સંદર્ભમાંથી target શબ્દની આગાહી કરે
છે & સંદર્ભ \rightarrow લક્ષ્ય \\
\textbf{Skip-gram} & Target શબ્દમાંથી સંદર્ભ શબ્દોની આગાહી કરે છે & લક્ષ્ય \rightarrow
સંદર્ભ \\
\end{longtable}
}

\textbf{પ્રક્રિયા}:

\begin{enumerate}
\tightlist
\item
  \textbf{Training}: Neural network શબ્દ સંબંધો શીખે છે
\item
  \textbf{Vector Creation}: દરેક શબ્દને અનન્ય વેક્ટર પ્રતિનિધિત્વ મળે છે
\item
  \textbf{Similarity}: સમાન શબ્દોના સમાન વેક્ટર્સ હોય છે
\end{enumerate}

\textbf{ઉદાહરણ}: vector(``king'') - vector(``man'') + vector(``woman'')
\approx vector(``queen'')

\end{solutionbox}
\begin{mnemonicbox}
``Word2Vec = સંદર્ભ દ્વારા શબ્દોથી વેક્ટર્સ''

\end{mnemonicbox}
\begin{center}\rule{0.5\linewidth}{0.5pt}\end{center}

\subsection*{પ્રશ્ન 5(ક) [7
ગુણ]}\label{uxaaauxab0uxab6uxaa8-5uxa95-7-uxa97uxaa3}

\textbf{ઉત્પાદનના ઉત્પાદકે ગ્રાહક પાસેથી feedback એકત્રિત કર્યો છે અને હવે તેના પર
sentiment analysis કરવા ઈચ્છે છે. તેના માટે કયા પગલાઓ અનુસરવા જોઈએ? વિગતવાર
સમજાવો.}

\begin{solutionbox}

\textbf{Sentiment Analysis Pipeline:}

{\def\LTcaptype{none} % do not increment counter
\begin{longtable}[]{@{}
  >{\raggedright\arraybackslash}p{(\linewidth - 4\tabcolsep) * \real{0.2143}}
  >{\raggedright\arraybackslash}p{(\linewidth - 4\tabcolsep) * \real{0.2500}}
  >{\raggedright\arraybackslash}p{(\linewidth - 4\tabcolsep) * \real{0.5357}}@{}}
\toprule\noalign{}
\begin{minipage}[b]{\linewidth}\raggedright
પગલું
\end{minipage} & \begin{minipage}[b]{\linewidth}\raggedright
વર્ણન
\end{minipage} & \begin{minipage}[b]{\linewidth}\raggedright
Tools/Methods
\end{minipage} \\
\midrule\noalign{}
\endhead
\bottomrule\noalign{}
\endlastfoot
\textbf{Data Collection} & ગ્રાહક feedback એકત્રિત કરવું & સર્વે, સમીક્ષાઓ,
સોશિયલ મીડિયા \\
\textbf{Data Preprocessing} & ટેક્સ્ટ સાફ અને તૈયાર કરવું & Noise દૂર કરવું,
tokenization \\
\textbf{Feature Extraction} & ટેક્સ્ટને સંખ્યાત્મક સ્વરૂપમાં બદલવું & TF-IDF, Word
embeddings \\
\textbf{Model Training} & Sentiment classifier તાલીમ આપવી & Naive Bayes,
SVM, Neural networks \\
\textbf{Prediction} & Sentiment વર્ગીકરણ કરવું & હકારાત્મક/નકારાત્મક/તટસ્થ \\
\textbf{Analysis} & પરિણામોનું અર્થઘટન & Insights અને રિપોર્ટ્સ બનાવવા \\
\end{longtable}
}

\textbf{અમલીકરણ પ્રવાહ:}

\begin{center}
\textbf{Mermaid Diagram (Code)}
\begin{verbatim}
{Shaded}
{Highlighting}[]
graph LR
    A[ગ્રાહક Feedback] {-{-}{} B[Text Preprocessing]}
    B {-{-}{} C[Feature Extraction]}
    C {-{-}{} D[Sentiment Model]}
    D {-{-}{} E[Classification]}
    E {-{-}{} F[Business Insights]}
{Highlighting}
{Shaded}
\end{verbatim}
\end{center}

\textbf{Preprocessing પગલાં:}

\begin{itemize}
\tightlist
\item
  \textbf{વિશેષ અક્ષરો} અને URLs દૂર કરવા
\item
  \textbf{Lowercase માં convert} કરવા સુસંગતતા માટે
\item
  \textbf{Stop words દૂર કરવા} (the, and, or)
\item
  \textbf{Negations handle કરવા} (not good \rightarrow negative sentiment)
\end{itemize}

\textbf{Model મૂલ્યાંકન}: accuracy, precision, recall, F1-score જેવા metrics
વાપરવા

\textbf{વ્યાપારિક મૂલ્ય}: ગ્રાહક સંતુષ્ટિ સમજવી, ઉત્પાદનો સુધારવા, સમસ્યાઓ ઓળખવી

\end{solutionbox}
\begin{mnemonicbox}
``એકત્રિત કરો, સાફ કરો, કાઢો, તાલીમ આપો, આગાહી કરો,
વિશ્લેષણ કરો = ESTAVA''

\end{mnemonicbox}
\begin{center}\rule{0.5\linewidth}{0.5pt}\end{center}

\subsection*{પ્રશ્ન 5(અ) OR [3
ગુણ]}\label{uxaaauxab0uxab6uxaa8-5uxa85-or-3-uxa97uxaa3}

\textbf{GloVe ના ફાયદાઓ NLP ના સંદર્ભમાં સમજાવો.}

\begin{solutionbox}

{\def\LTcaptype{none} % do not increment counter
\begin{longtable}[]{@{}
  >{\raggedright\arraybackslash}p{(\linewidth - 2\tabcolsep) * \real{0.5000}}
  >{\raggedright\arraybackslash}p{(\linewidth - 2\tabcolsep) * \real{0.5000}}@{}}
\toprule\noalign{}
\begin{minipage}[b]{\linewidth}\raggedright
ફાયદો
\end{minipage} & \begin{minipage}[b]{\linewidth}\raggedright
વર્ણન
\end{minipage} \\
\midrule\noalign{}
\endhead
\bottomrule\noalign{}
\endlastfoot
\textbf{Global Context} & સ્થાનિક સંદર્ભ જ નહીં પરંતુ સમગ્ર corpus આંકડા ધ્યાનમાં
રાખે છે \\
\textbf{Linear Relationships} & વેક્ટર અંકગણિત દ્વારા semantic સંબંધો capture
કરે છે \\
\textbf{Efficiency} & મોટા datasets પર Word2Vec કરતાં ઝડપી training \\
\textbf{Stability} & બહુવિધ training runs માં સુસંગત પરિણામો \\
\end{longtable}
}

\textbf{મુખ્ય લાભો}: Word analogy કાર્યોમાં સારું પ્રદર્શન, સ્થાનિક અને વૈશ્વિક બંને
આંકડા capture કરે છે

\end{solutionbox}
\begin{mnemonicbox}
``GloVe = વૈશ્વિક વેક્ટર શ્રેષ્ઠતા''

\end{mnemonicbox}
\begin{center}\rule{0.5\linewidth}{0.5pt}\end{center}

\subsection*{પ્રશ્ન 5(બ) OR [4
ગુણ]}\label{uxaaauxab0uxab6uxaa8-5uxaac-or-4-uxa97uxaa3}

\textbf{TFIDF અને BoW સાથેના પડકારો વિશે સમજાવો.}

\begin{solutionbox}

\textbf{પડકારો ટેબલ:}

{\def\LTcaptype{none} % do not increment counter
\begin{longtable}[]{@{}
  >{\raggedright\arraybackslash}p{(\linewidth - 4\tabcolsep) * \real{0.3636}}
  >{\raggedright\arraybackslash}p{(\linewidth - 4\tabcolsep) * \real{0.4091}}
  >{\raggedright\arraybackslash}p{(\linewidth - 4\tabcolsep) * \real{0.2273}}@{}}
\toprule\noalign{}
\begin{minipage}[b]{\linewidth}\raggedright
પદ્ધતિ
\end{minipage} & \begin{minipage}[b]{\linewidth}\raggedright
પડકારો
\end{minipage} & \begin{minipage}[b]{\linewidth}\raggedright
અસર
\end{minipage} \\
\midrule\noalign{}
\endhead
\bottomrule\noalign{}
\endlastfoot
\textbf{TF-IDF} & 1. શબ્દ ક્રમ અવગણે છે2. Sparse vectors3. Semantic
similarity નથી & મર્યાદિત સંદર્ભ સમજ \\
\textbf{BoW} & 1. Sequence માહિતી ગુમાવે છે2. ઉચ્ચ dimensionality3. શબ્દ સંબંધો
નથી & નબળું semantic પ્રતિનિધિત્વ \\
\end{longtable}
}

\textbf{સામાન્ય સમસ્યાઓ:}

\begin{itemize}
\tightlist
\item
  \textbf{Vocabulary size}: ખૂબ મોટા, sparse matrices બનાવે છે
\item
  \textbf{Out-of-vocabulary}: નવા શબ્દો handle કરી શકતું નથી
\item
  \textbf{Semantic gap}: ``સારું'' અને ``ઉત્તમ'' ને અલગ ગણે છે
\end{itemize}

\textbf{ઉકેલ}: Word embeddings (Word2Vec, GloVe) વાપરો dense, semantic
રજૂઆત માટે

\end{solutionbox}
\begin{mnemonicbox}
``TF-IDF અને BoW = Sparse, કોઈ ક્રમ નથી, કોઈ semantics
નથી''

\end{mnemonicbox}
\begin{center}\rule{0.5\linewidth}{0.5pt}\end{center}

\subsection*{પ્રશ્ન 5(ક) OR [7
ગુણ]}\label{uxaaauxab0uxab6uxaa8-5uxa95-or-7-uxa97uxaa3}

\textbf{E-mail સેવા પ્રદાતા SPAM detection તકનીક લાગુ કરવા ઈચ્છે છે. SPAM
E-mail શોધવા માટે કયા પગલાઓ અનુસરવા જોઈએ? વિગતવાર સમજાવો.}

\begin{solutionbox}

\textbf{SPAM Detection Pipeline:}

{\def\LTcaptype{none} % do not increment counter
\begin{longtable}[]{@{}
  >{\raggedright\arraybackslash}p{(\linewidth - 4\tabcolsep) * \real{0.2727}}
  >{\raggedright\arraybackslash}p{(\linewidth - 4\tabcolsep) * \real{0.3182}}
  >{\raggedright\arraybackslash}p{(\linewidth - 4\tabcolsep) * \real{0.4091}}@{}}
\toprule\noalign{}
\begin{minipage}[b]{\linewidth}\raggedright
પગલું
\end{minipage} & \begin{minipage}[b]{\linewidth}\raggedright
વર્ણન
\end{minipage} & \begin{minipage}[b]{\linewidth}\raggedright
તકનીકો
\end{minipage} \\
\midrule\noalign{}
\endhead
\bottomrule\noalign{}
\endlastfoot
\textbf{Data Collection} & Labeled spam/ham emails એકત્રિત કરવા & Email
datasets, યુઝર રિપોર્ટ્સ \\
\textbf{Feature Engineering} & સંબંધિત features કાઢવા & Subject વિશ્લેષણ,
sender patterns \\
\textbf{Text Preprocessing} & Email content સાફ કરવું & HTML દૂર કરવું, text
normalize કરવું \\
\textbf{Feature Extraction} & સંખ્યાત્મક સ્વરૂપમાં convert કરવું & TF-IDF,
N-grams, metadata \\
\textbf{Model Training} & Classifier તાલીમ આપવી & Naive Bayes, SVM,
Random Forest \\
\textbf{Validation} & Model પ્રદર્શન ચકાસવું & Cross-validation, test set \\
\textbf{Deployment} & Email system સાથે એકીકરણ & Real-time
classification \\
\end{longtable}
}

\textbf{Feature પ્રકારો:}

{\def\LTcaptype{none} % do not increment counter
\begin{longtable}[]{@{}lll@{}}
\toprule\noalign{}
Feature Category & ઉદાહરણો & ઉદ્દેશ્ય \\
\midrule\noalign{}
\endhead
\bottomrule\noalign{}
\endlastfoot
\textbf{Content-based} & Keywords, phrases, HTML tags & Email body
વિશ્લેષણ \\
\textbf{Header-based} & Sender, subject, timestamps & Metadata તપાસવું \\
\textbf{Behavioral} & Sending patterns, frequency & શંકાશીલ વર્તન ઓળખવું \\
\end{longtable}
}

\textbf{અમલીકરણ આકૃતિ:}

\begin{center}
\textbf{Mermaid Diagram (Code)}
\begin{verbatim}
{Shaded}
{Highlighting}[]
graph LR
    A[આવતો Email] {-{-}{} B[Feature Extraction]}
    B {-{-}{} C\{SPAM Classifier\}}
    C {-{-}{}|Spam| D[Spam Folder]}
    C {-{-}{}|Ham| E[Inbox]}
    F[યુઝર Feedback] {-{-}{} G[Model Update]}
{Highlighting}
{Shaded}
\end{verbatim}
\end{center}

\textbf{Model મૂલ્યાંકન Metrics:}

\begin{itemize}
\tightlist
\item
  \textbf{Precision}: False positives ટાળવા (કાયદેસર emails spam તરીકે
  mark ન થાય)
\item
  \textbf{Recall}: વાસ્તવિક spam emails પકડવા
\item
  \textbf{F1-Score}: Precision અને recall વચ્ચે સંતુલન
\end{itemize}

\textbf{સતત શીખવું}: નવા spam patterns અને યુઝર feedback સાથે model update
કરવું

\end{solutionbox}
\begin{mnemonicbox}
``એકત્રિત કરો, ઇજનેર કરો, પ્રોસેસ કરો, કાઢો, તાલીમ આપો,
માન્ય કરો, જમાવો = EIPKTMJ''

\end{mnemonicbox}

\end{document}
