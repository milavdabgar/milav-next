\documentclass[10pt,a4paper]{article}

% content/resources/templates/preamble.tex
\usepackage[margin=0.6in]{geometry}
\author{Milav Dabgar}
\usepackage{amsmath,amssymb,amsthm}
\usepackage{booktabs}
\usepackage{multirow}
\usepackage{xcolor}
\usepackage{tcolorbox}
\tcbuselibrary{breakable,skins}
\usepackage[colorlinks=true,linkcolor=blue]{hyperref}
\usepackage{titlesec}
\usepackage{enumitem}
\usepackage{tikz}
\usepackage{pgfplots}
\usepackage{circuitikz}
\usepackage[version=4]{mhchem}
\usepackage{longtable}
\usepackage{array}
\usepackage{float}
\usepackage{caption}
\usepackage{listings}

\lstset{
  basicstyle=\small\ttfamily,
  breaklines=true,
  breakatwhitespace=false,
  postbreak=\mbox{\textcolor{red}{$\hookrightarrow$}\space},
  float=false,
  numbers=left,
  numberstyle=\tiny\color{gray},
  numbersep=10pt,
  xleftmargin=2em,
  keywordstyle=\color{blue},
  commentstyle=\color{green!60!black},
  stringstyle=\color{purple},
  backgroundcolor=\color{gray!5},
  showstringspaces=false,
  tabsize=2,
  captionpos=b,
  keepspaces=true,
  columns=flexible
}

\pgfplotsset{compat=1.18}
\usetikzlibrary{shapes,arrows,positioning,calc,patterns,decorations.pathmorphing,decorations.markings,arrows.meta}

% Color scheme
\definecolor{headcolor}{RGB}{0,102,204}
\definecolor{keycolor}{RGB}{220,20,60}
\definecolor{solutioncolor}{RGB}{34,139,34}
\definecolor{mnemoniccolor}{RGB}{148,0,211}
\definecolor{codecolor}{RGB}{0,0,100}

% Spacing
\setlength{\parskip}{3pt}
\setlist[itemize]{nosep}
\setlist[enumerate]{nosep}

% Title formatting
\titleformat{\section}{\Large\bfseries\color{headcolor}}{\thesection}{1em}{}
\titleformat{\subsection}{\large\bfseries\color{headcolor}}{\thesubsection}{1em}{}

% Pandoc tightlist compatibility
\providecommand{\tightlist}{%
  \setlength{\itemsep}{0pt}\setlength{\parskip}{0pt}}

% Pandoc longtable compatibility
\newcounter{none}
\def\thenone{}


% content/resources/templates/gujarati-boxes.tex
\usepackage{fontspec}
\usepackage{polyglossia}

% Set Gujarati as main language (document is primarily in Gujarati)
% Note: gloss-gujarati.ldf doesn't exist in polyglossia, but it will use hyphenation patterns
\setdefaultlanguage{gujarati}
\setotherlanguage{english}

% Configure Gujarati font properly
% Use Language=Default to prevent polyglossia from trying to add language-specific features
% that don't exist for Gujarati, which causes "empty feature" warnings
\newfontfamily\gujaratifont[Script=Gujarati,AutoFakeBold=2.5,AutoFakeSlant=0.3]{Noto Sans Gujarati}
\setmainfont[Script=Gujarati,AutoFakeBold=2.5,AutoFakeSlant=0.3]{Noto Sans Gujarati}
% Use Noto Sans Gujarati for monospace to support Gujarati in text
\setmonofont[Scale=0.9]{Noto Sans Gujarati}

% Configure English to use the same font
\newfontfamily\englishfont[Script=Gujarati,AutoFakeBold=2.5,AutoFakeSlant=0.3]{Noto Sans Gujarati}

% Translations for polyglossia
\gappto\captionsgujarati{
  \renewcommand{\tablename}{કોષ્ટક}
  \renewcommand{\figurename}{આકૃતિ}
}

% Helper for TikZ nodes to ensure Gujarati font
\newcommand{\gu}[1]{{\gujaratifont #1}}

% Custom environments
\newtcolorbox{solutionbox}{
    breakable,
    enhanced,
    colback=solutioncolor!5!white,
    colframe=solutioncolor!75!black,
    fonttitle=\bfseries,
    title=જવાબ
}

\newtcolorbox{solutionboxnobreak}{
 colback=solutioncolor!5!white,
 colframe=solutioncolor!75!black,
 fonttitle=\bfseries,
 title=જવાબ
}

\newtcolorbox{keyformula}{
 breakable,
 enhanced,
 colback=keycolor!5!white,
 colframe=keycolor!75!black,
 fonttitle=\bfseries,
 title=રાસાયણિક સમીકરણ/સૂત્ર
}

\newtcolorbox{mnemonicbox}{
 breakable,
 enhanced,
 colback=mnemoniccolor!5!white,
 colframe=mnemoniccolor!75!black,
 fonttitle=\bfseries,
 title=મેમરી ટ્રીક
}


\begin{document}

\begin{center}
{\Huge\bfseries\color{headcolor} Subject Name (Gujarati)}\\[5pt]
{\LARGE 4351603 -- Winter 2024}\\[3pt]
{\large Semester 1 Study Material}\\[3pt]
{\normalsize\textit{Detailed Solutions and Explanations}}
\end{center}

\vspace{10pt}

\subsection*{પ્રશ્ન 1(અ) [3
ગુણ]}\label{uxaaauxab0uxab6uxaa8-1uxa85-3-uxa97uxaa3}

\textbf{તેના ઉપયોગ સાથે JFC નું વર્ણન કરો.}

\begin{solutionbox}

JFC (Java Foundation Classes) એ જાવામાં ડેસ્કટોપ એપ્લિકેશન બનાવવા માટેનું વ્યાપક
GUI ફ્રેમવર્ક છે.

{\def\LTcaptype{none} % do not increment counter
\begin{longtable}[]{@{}ll@{}}
\toprule\noalign{}
કમ્પોનન્ટ & વર્ણન \\
\midrule\noalign{}
\endhead
\bottomrule\noalign{}
\endlastfoot
\textbf{Swing} & હળવા વજનના GUI કમ્પોનન્ટ \\
\textbf{AWT} & મૂળભૂત વિન્ડોઇંગ ટૂલકિટ \\
\textbf{Java 2D} & એડવાન્સ ગ્રાફિક્સ અને ઇમેજિંગ \\
\textbf{Accessibility} & સહાયક ટેકનોલોજી માટે સપોર્ટ \\
\end{longtable}
}

\begin{itemize}
\tightlist
\item
  \textbf{મુખ્ય ઉપયોગ}: સમૃદ્ધ ડેસ્કટોપ એપ્લિકેશન બનાવવું
\item
  \textbf{મુખ્ય ફાયદો}: પ્લેટફોર્મ સ્વતંત્રતા અને સુસંગત દેખાવ
\end{itemize}

\end{solutionbox}
\begin{mnemonicbox}
``JFC = Java's Fantastic Components''

\end{mnemonicbox}
\subsection*{પ્રશ્ન 1(બ) [4
ગુણ]}\label{uxaaauxab0uxab6uxaa8-1uxaac-4-uxa97uxaa3}

\textbf{AWT અને સ્વિંગ વચ્ચેનો તફાવત સમજાવો.}

\begin{solutionbox}

{\def\LTcaptype{none} % do not increment counter
\begin{longtable}[]{@{}lll@{}}
\toprule\noalign{}
લક્ષણ & AWT & Swing \\
\midrule\noalign{}
\endhead
\bottomrule\noalign{}
\endlastfoot
\textbf{કમ્પોનન્ટ} & હેવીવેઇટ (native) & લાઇટવેઇટ (શુદ્ધ જાવા) \\
\textbf{પ્લેટફોર્મ} & પ્લેટફોર્મ આધારિત & પ્લેટફોર્મ સ્વતંત્ર \\
\textbf{દેખાવ} & Native OS લુક & બદલી શકાય તેવું લુક \& ફીલ \\
\textbf{પ્રદર્શન} & વધુ ઝડપી & થોડું ધીમું \\
\end{longtable}
}

\begin{itemize}
\tightlist
\item
  \textbf{AWT મર્યાદા}: મર્યાદિત કમ્પોનન્ટ, પ્લેટફોર્મ-વિશિષ્ટ દેખાવ
\item
  \textbf{Swing ફાયદો}: સમૃદ્ધ કમ્પોનન્ટ સેટ, કસ્ટમાઇઝેબલ UI
\end{itemize}

\end{solutionbox}
\begin{mnemonicbox}
``AWT = Always Weighs Too-much, Swing = Simply Works
In New Generation''

\end{mnemonicbox}
\subsection*{પ્રશ્ન 1(ક) [7
ગુણ]}\label{uxaaauxab0uxab6uxaa8-1uxa95-7-uxa97uxaa3}

\textbf{વિવિધ ઇવેન્ટ લિસ્નર ની યાદી બનાવો. કોઈપણ એક સમજાવો.}

\begin{solutionbox}

\textbf{ઇવેન્ટ લિસ્નર યાદી:}

{\def\LTcaptype{none} % do not increment counter
\begin{longtable}[]{@{}ll@{}}
\toprule\noalign{}
લિસ્નર & હેતુ \\
\midrule\noalign{}
\endhead
\bottomrule\noalign{}
\endlastfoot
\textbf{ActionListener} & બટન ક્લિક, મેનુ પસંદગી \\
\textbf{MouseListener} & માઉસ ઇવેન્ટ (ક્લિક, પ્રેસ, રિલીઝ) \\
\textbf{KeyListener} & કીબોર્ડ ઇનપુટ ઇવેન્ટ \\
\textbf{WindowListener} & વિન્ડો સ્ટેટ ફેરફાર \\
\textbf{FocusListener} & કમ્પોનન્ટ ફોકસ ઇવેન્ટ \\
\textbf{ItemListener} & ચેકબોક્સ/રેડિયો બટન ફેરફાર \\
\end{longtable}
}

\textbf{ActionListener સમજાવટ:}

\begin{itemize}
\tightlist
\item
  \textbf{ઇન્ટરફેસ મેથડ}: \texttt{actionPerformed(ActionEvent\ e)}
\item
  \textbf{ઉપયોગ}: બટન ક્લિક અને મેનુ ક્રિયાઓ હેન્ડલ કરે
\item
  \textbf{અમલીકરણ}: અનામિક ક્લાસ અથવા lambda expression
\end{itemize}

\begin{verbatim}
button.addActionListener(e {-} \{
    System.out.println("Button clicked!");
\);}
\end{verbatim}

\end{solutionbox}
\begin{mnemonicbox}
``AMKWFI Listeners = Action Mouse Key Window Focus
Item''

\end{mnemonicbox}
\subsection*{પ્રશ્ન 1(ક OR) [7
ગુણ]}\label{uxaaauxab0uxab6uxaa8-1uxa95-or-7-uxa97uxaa3}

\textbf{વિવિધ લેઆઉટ મેનેજરોની યાદી બનાવો. કોઈપણ એક સમજાવો.}

\begin{solutionbox}

\textbf{લેઆઉટ મેનેજર યાદી:}

{\def\LTcaptype{none} % do not increment counter
\begin{longtable}[]{@{}ll@{}}
\toprule\noalign{}
લેઆઉટ મેનેજર & હેતુ \\
\midrule\noalign{}
\endhead
\bottomrule\noalign{}
\endlastfoot
\textbf{FlowLayout} & ક્રમિક કમ્પોનન્ટ પ્લેસમેન્ટ \\
\textbf{BorderLayout} & પાંચ પ્રદેશો (ઉત્તર, દક્ષિણ, પૂર્વ, પશ્ચિમ, કેન્દ્ર) \\
\textbf{GridLayout} & ગ્રિડ-આધારિત ગોઠવણી \\
\textbf{CardLayout} & કમ્પોનન્ટનો સ્ટેક \\
\textbf{BoxLayout} & એક પંક્તિ અથવા સ્તંભ \\
\textbf{GridBagLayout} & કન્સ્ટ્રેઇન્ટ સાથે જટિલ ગ્રિડ \\
\end{longtable}
}

\textbf{BorderLayout સમજાવટ:}

\begin{itemize}
\tightlist
\item
  \textbf{ડિફોલ્ટ લેઆઉટ}: JFrame અને JDialog માટે
\item
  \textbf{પાંચ પ્રદેશો}: ઉત્તર, દક્ષિણ, પૂર્વ, પશ્ચિમ, કેન્દ્ર
\item
  \textbf{રીસાઈઝિંગ}: કેન્દ્ર વિસ્તરે છે, અન્ય પ્રાથમિક કદ રાખે છે
\end{itemize}

\begin{center}
\textbf{Mermaid Diagram (Code)}
\begin{verbatim}
{Shaded}
{Highlighting}[]
graph LR
    A[ઉત્તર] 
    B[પશ્ચિમ] 
    C[કેન્દ્ર] 
    D[પૂર્વ]
    E[દક્ષિણ]
    
    A {-{-}{-} B}
    A {-{-}{-} C}
    A {-{-}{-} D}
    B {-{-}{-} C}
    C {-{-}{-} D}
    B {-{-}{-} E}
    C {-{-}{-} E}
    D {-{-}{-} E}
{Highlighting}
{Shaded}
\end{verbatim}
\end{center}

\end{solutionbox}
\begin{mnemonicbox}
``FBGCBG Layouts = Flow Border Grid Card Box
GridBag''

\end{mnemonicbox}
\subsection*{પ્રશ્ન 2(અ) [3
ગુણ]}\label{uxaaauxab0uxab6uxaa8-2uxa85-3-uxa97uxaa3}

\textbf{ડેટાબેઝને કનેક્ટ કરવાના પગલાંની યાદી બનાવો અને સમજાવો.}

\begin{solutionbox}

\textbf{ડેટાબેઝ કનેક્શન પગલાં:}

{\def\LTcaptype{none} % do not increment counter
\begin{longtable}[]{@{}ll@{}}
\toprule\noalign{}
પગલું & ક્રિયા \\
\midrule\noalign{}
\endhead
\bottomrule\noalign{}
\endlastfoot
\textbf{1. ડ્રાઇવર લોડ} & \texttt{Class.forName("driver.class")} \\
\textbf{2. કનેક્શન બનાવો} & \texttt{DriverManager.getConnection()} \\
\textbf{3. સ્ટેટમેન્ટ બનાવો} & \texttt{connection.createStatement()} \\
\textbf{4. ક્વેરી એક્ઝિક્યુટ કરો} & \texttt{statement.executeQuery()} \\
\textbf{5. પરિણામ પર પ્રોસેસ કરો} & \texttt{resultSet.next()} \\
\textbf{6. રિસોર્સ બંધ કરો} & બધા કનેક્શન બંધ કરો \\
\end{longtable}
}

\end{solutionbox}
\begin{mnemonicbox}
``LCD EPR = Load Create Driver, Execute Process
Results''

\end{mnemonicbox}
\subsection*{પ્રશ્ન 2(બ) [4
ગુણ]}\label{uxaaauxab0uxab6uxaa8-2uxaac-4-uxa97uxaa3}

\textbf{3-tier આર્કિટેક્ચર ડાયાગ્રામ સાથે સમજાવો.}

\begin{solutionbox}

3-tier આર્કિટેક્ચર એપ્લિકેશનને બહેતર જાળવણી માટે ત્રણ લોજિકલ લેયરમાં વિભાજિત કરે છે.

\begin{center}
\textbf{Mermaid Diagram (Code)}
\begin{verbatim}
{Shaded}
{Highlighting}[]
graph LR
    A[પ્રેઝન્ટેશન ટાયર{br/{}વેબ બ્રાઉઝર/UI] }
    B[એપ્લિકેશન ટાયર{br/{}બિઝનેસ લોજિક/સર્વલેટ] }
    C[ડેટા ટાયર{br/{}ડેટાબેઝ સર્વર]}
    
    A {{-}{-}{} B}
    B {{-}{-}{} C}
{Highlighting}
{Shaded}
\end{verbatim}
\end{center}

{\def\LTcaptype{none} % do not increment counter
\begin{longtable}[]{@{}ll@{}}
\toprule\noalign{}
ટાયર & જવાબદારી \\
\midrule\noalign{}
\endhead
\bottomrule\noalign{}
\endlastfoot
\textbf{પ્રેઝન્ટેશન} & યુઝર ઇન્ટરફેસ અને યુઝર ઇન્ટરેક્શન \\
\textbf{એપ્લિકેશન} & બિઝનેસ લોજિક અને પ્રોસેસિંગ \\
\textbf{ડેટા} & ડેટા સ્ટોરેજ અને મેનેજમેન્ટ \\
\end{longtable}
}

\begin{itemize}
\tightlist
\item
  \textbf{ફાયદો}: બહેતર સ્કેલેબિલિટી અને જાળવણી
\item
  \textbf{ઉદાહરણ}: વેબ બ્રાઉઝર \rightarrow વેબ સર્વર \rightarrow ડેટાબેઝ
\end{itemize}

\end{solutionbox}
\begin{mnemonicbox}
``PAD = Presentation Application Data''

\end{mnemonicbox}
\subsection*{પ્રશ્ન 2(ક) [7
ગુણ]}\label{uxaaauxab0uxab6uxaa8-2uxa95-7-uxa97uxaa3}

\textbf{ઇન્ટરફેસ અને વર્ગો સાથે JDBC API નું વર્ણન કરો.}

\begin{solutionbox}

\textbf{JDBC API કમ્પોનન્ટ્સ:}

{\def\LTcaptype{none} % do not increment counter
\begin{longtable}[]{@{}lll@{}}
\toprule\noalign{}
પ્રકાર & કમ્પોનન્ટ & હેતુ \\
\midrule\noalign{}
\endhead
\bottomrule\noalign{}
\endlastfoot
\textbf{ઇન્ટરફેસ} & Connection & ડેટાબેઝ કનેક્શન \\
\textbf{ઇન્ટરફેસ} & Statement & SQL એક્ઝિક્યુશન \\
\textbf{ઇન્ટરફેસ} & ResultSet & ક્વેરી પરિણામો \\
\textbf{ઇન્ટરફેસ} & PreparedStatement & પ્રીકમ્પાઇલ્ડ SQL \\
\textbf{ક્લાસ} & DriverManager & ડ્રાઇવર મેનેજમેન્ટ \\
\textbf{ક્લાસ} & SQLException & એરર હેન્ડલિંગ \\
\end{longtable}
}

\textbf{JDBC આર્કિટેક્ચર:}

\begin{center}
\textbf{Mermaid Diagram (Code)}
\begin{verbatim}
{Shaded}
{Highlighting}[]
graph LR
    A[Java Application] {-{-}{} B[JDBC API]}
    B {-{-}{} C[JDBC Driver Manager]}
    C {-{-}{} D[JDBC Driver]}
    D {-{-}{} E[Database]}
{Highlighting}
{Shaded}
\end{verbatim}
\end{center}

\begin{itemize}
\tightlist
\item
  \textbf{મુખ્ય ઇન્ટરફેસ}: Connection, Statement, ResultSet,
  PreparedStatement
\item
  \textbf{મુખ્ય ક્લાસ}: કનેક્શન મેનેજમેન્ટ માટે DriverManager
\item
  \textbf{એક્સેપ્શન હેન્ડલિંગ}: ડેટાબેઝ એરર માટે SQLException
\end{itemize}

\end{solutionbox}
\begin{mnemonicbox}
``CSRP Classes = Connection Statement ResultSet
PreparedStatement''

\end{mnemonicbox}
\subsection*{પ્રશ્ન 2(અ OR) [3
ગુણ]}\label{uxaaauxab0uxab6uxaa8-2uxa85-or-3-uxa97uxaa3}

\textbf{JDBC ના ફાયદા અને ગેરફાયદાની યાદી બનાવો.}

\begin{solutionbox}

\textbf{JDBC ફાયદા વિ ગેરફાયદા:}

{\def\LTcaptype{none} % do not increment counter
\begin{longtable}[]{@{}ll@{}}
\toprule\noalign{}
ફાયદા & ગેરફાયદા \\
\midrule\noalign{}
\endhead
\bottomrule\noalign{}
\endlastfoot
\textbf{પ્લેટફોર્મ સ્વતંત્ર} & \textbf{પર્ફોર્મન્સ ઓવરહેડ} \\
\textbf{સ્ટાન્ડર્ડ API} & \textbf{જટિલ કન્ફિગરેશન} \\
\textbf{બહુવિધ ડેટાબેઝ સપોર્ટ} & \textbf{મર્યાદિત ORM ફીચર્સ} \\
\end{longtable}
}

\begin{itemize}
\tightlist
\item
  \textbf{લાભો}: એકવાર લખો, કોઈપણ ડેટાબેઝ સાથે ગમે ત્યાં ચલાવો
\item
  \textbf{ખામીઓ}: મેન્યુઅલ SQL અને કનેક્શન મેનેજમેન્ટની જરૂરિયાત
\end{itemize}

\end{solutionbox}
\begin{mnemonicbox}
``PSM vs PCL = Platform Standard Multiple vs
Performance Complex Limited''

\end{mnemonicbox}
\subsection*{પ્રશ્ન 2(બ OR) [4
ગુણ]}\label{uxaaauxab0uxab6uxaa8-2uxaac-or-4-uxa97uxaa3}

\textbf{2-tier આર્કિટેક્ચર ડાયાગ્રામ સાથે સમજાવો.}

\begin{solutionbox}

2-tier આર્કિટેક્ચર ક્લાયન્ટને ડેટાબેઝ સર્વર સાથે સીધું જોડે છે.

\begin{center}
\textbf{Mermaid Diagram (Code)}
\begin{verbatim}
{Shaded}
{Highlighting}[]
graph LR
    A[ક્લાયન્ટ ટાયર{br/{}એપ્લિકેશન/UI] {}{-}{-}{} B[ડેટા ટાયર{}br/{}ડેટાબેઝ સર્વર]}
{Highlighting}
{Shaded}
\end{verbatim}
\end{center}

{\def\LTcaptype{none} % do not increment counter
\begin{longtable}[]{@{}ll@{}}
\toprule\noalign{}
ટાયર & જવાબદારી \\
\midrule\noalign{}
\endhead
\bottomrule\noalign{}
\endlastfoot
\textbf{ક્લાયન્ટ} & યુઝર ઇન્ટરફેસ અને બિઝનેસ લોજિક \\
\textbf{સર્વર} & ડેટા સ્ટોરેજ અને મેનેજમેન્ટ \\
\end{longtable}
}

\begin{itemize}
\tightlist
\item
  \textbf{ફાયદો}: સરળ આર્કિટેક્ચર, સીધો કમ્યુનિકેશન
\item
  \textbf{ગેરફાયદો}: મર્યાદિત સ્કેલેબિલિટી, ટાઈટ કપલિંગ
\item
  \textbf{ઉદાહરણ}: ડેસ્કટોપ એપ્લિકેશન સીધું ડેટાબેઝ સાથે જોડાય
\end{itemize}

\end{solutionbox}
\begin{mnemonicbox}
``CD = Client Data (direct connection)''

\end{mnemonicbox}
\subsection*{પ્રશ્ન 2(ક OR) [7
ગુણ]}\label{uxaaauxab0uxab6uxaa8-2uxa95-or-7-uxa97uxaa3}

\textbf{JDBC ડ્રાઇવર પ્રકારોની યાદી બનાવો અને TYPE-4 સમજાવો.}

\begin{solutionbox}

\textbf{JDBC ડ્રાઇવર પ્રકારો:}

{\def\LTcaptype{none} % do not increment counter
\begin{longtable}[]{@{}lll@{}}
\toprule\noalign{}
પ્રકાર & નામ & વર્ણન \\
\midrule\noalign{}
\endhead
\bottomrule\noalign{}
\endlastfoot
\textbf{Type-1} & JDBC-ODBC Bridge & ODBC ડ્રાઇવર વાપરે \\
\textbf{Type-2} & Native-API Driver & આંશિક જાવા, આંશિક native \\
\textbf{Type-3} & Network Protocol Driver & શુદ્ધ જાવા, middleware \\
\textbf{Type-4} & Native Protocol Driver & શુદ્ધ જાવા, સીધું \\
\end{longtable}
}

\textbf{TYPE-4 ડ્રાઇવર સમજાવટ:}

\begin{itemize}
\tightlist
\item
  \textbf{શુદ્ધ જાવા}: સંપૂર્ણપણે જાવામાં લખાયેલું
\item
  \textbf{સીધો કમ્યુનિકેશન}: ડેટાબેઝ સાથે સીધો વાતચીત
\item
  \textbf{પ્લેટફોર્મ સ્વતંત્ર}: native લાઇબ્રેરીની જરૂર નથી
\item
  \textbf{શ્રેષ્ઠ પ્રદર્શન}: બધા પ્રકારોમાં સૌથી ઝડપી
\item
  \textbf{ઉદાહરણો}: MySQL Connector/J, PostgreSQL JDBC
\end{itemize}

\begin{center}
\textbf{Mermaid Diagram (Code)}
\begin{verbatim}
{Shaded}
{Highlighting}[]
graph LR
    A[Java Application] {-{-}{} B[Type{-}4 JDBC Driver{}br/{}Pure Java] {-}{-}{} C[Database Server]}
{Highlighting}
{Shaded}
\end{verbatim}
\end{center}

\end{solutionbox}
\begin{mnemonicbox}
``ONNN Drivers = ODBC Native Network Native-pure''

\end{mnemonicbox}
\subsection*{પ્રશ્ન 3(અ) [3
ગુણ]}\label{uxaaauxab0uxab6uxaa8-3uxa85-3-uxa97uxaa3}

\textbf{સર્વલેટની એપ્લિકેશન સમજાવો.}

\begin{solutionbox}

\textbf{સર્વલેટ એપ્લિકેશન:}

{\def\LTcaptype{none} % do not increment counter
\begin{longtable}[]{@{}ll@{}}
\toprule\noalign{}
એપ્લિકેશન & ઉપયોગ \\
\midrule\noalign{}
\endhead
\bottomrule\noalign{}
\endlastfoot
\textbf{વેબ ફોર્મ} & HTML ફોર્મ ડેટા પ્રોસેસ કરવું \\
\textbf{ડેટાબેઝ ઓપરેશન} & ડેટાબેઝ કનેક્ટ અને મેનિપ્યુલેટ કરવું \\
\textbf{સેશન મેનેજમેન્ટ} & યુઝર સેશન ટ્રેક કરવું \\
\textbf{ફાઈલ અપલોડ} & ફાઈલ અપલોડ હેન્ડલ કરવું \\
\end{longtable}
}

\begin{itemize}
\tightlist
\item
  \textbf{મુખ્ય ઉપયોગ}: વેબ એપ્લિકેશન માટે સર્વર-સાઇડ જાવા પ્રોગ્રામ
\item
  \textbf{સામાન્ય કાર્યો}: રિક્વેસ્ટ પ્રોસેસિંગ, રિસ્પોન્સ જનરેશન
\end{itemize}

\end{solutionbox}
\begin{mnemonicbox}
``WDSF = Web Database Session File''

\end{mnemonicbox}
\subsection*{પ્રશ્ન 3(બ) [4
ગુણ]}\label{uxaaauxab0uxab6uxaa8-3uxaac-4-uxa97uxaa3}

\textbf{એપ્લેટ અને સર્વલેટ વચ્ચેનો તફાવત સમજાવો.}

\begin{solutionbox}

{\def\LTcaptype{none} % do not increment counter
\begin{longtable}[]{@{}lll@{}}
\toprule\noalign{}
લક્ષણ & એપ્લેટ & સર્વલેટ \\
\midrule\noalign{}
\endhead
\bottomrule\noalign{}
\endlastfoot
\textbf{એક્ઝિક્યુશન} & ક્લાયન્ટ-સાઇડ (બ્રાઉઝર) & સર્વર-સાઇડ (વેબ સર્વર) \\
\textbf{હેતુ} & યુઝર ઇન્ટરફેસ & રિક્વેસ્ટ પ્રોસેસિંગ \\
\textbf{સિક્યોરિટી} & પ્રતિબંધિત (sandbox) & સર્વરની સંપૂર્ણ પહોંચ \\
\textbf{પ્રદર્શન} & ક્લાયન્ટ દ્વારા મર્યાદિત & સર્વર રિસોર્સ \\
\end{longtable}
}

\begin{itemize}
\tightlist
\item
  \textbf{એપ્લેટ}: વેબ બ્રાઉઝરમાં ચાલે, મર્યાદિત ક્ષમતાઓ
\item
  \textbf{સર્વલેટ}: વેબ સર્વર પર ચાલે, સંપૂર્ણ જાવા ક્ષમતાઓ
\end{itemize}

\end{solutionbox}
\begin{mnemonicbox}
``Client vs Server = એપ્લેટ vs સર્વલેટ''

\end{mnemonicbox}
\subsection*{પ્રશ્ન 3(ક) [7
ગુણ]}\label{uxaaauxab0uxab6uxaa8-3uxa95-7-uxa97uxaa3}

\textbf{સર્વલેટ ની લાઈફ સાઇકલ વિગતવાર સમજાવો.}

\begin{solutionbox}

\textbf{સર્વલેટ લાઇફ સાઇકલ:}

\begin{center}
\textbf{Mermaid Diagram (Code)}
\begin{verbatim}
{Shaded}
{Highlighting}[]
graph LR
    A[સર્વલેટ ક્લાસ લોડ] {-{-}{} B[init કોલ]}
    B {-{-}{} C[service રિક્વેસ્ટ હેન્ડલ કરે]}
    C {-{-}{} C}
    C {-{-}{} D[destroy કોલ]}
    D {-{-}{} E[સર્વલેટ અનલોડ]}
{Highlighting}
{Shaded}
\end{verbatim}
\end{center}

{\def\LTcaptype{none} % do not increment counter
\begin{longtable}[]{@{}lll@{}}
\toprule\noalign{}
તબક્કો & મેથડ & વર્ણન \\
\midrule\noalign{}
\endhead
\bottomrule\noalign{}
\endlastfoot
\textbf{લોડિંગ} & ક્લાસ લોડિંગ & વેબ કન્ટેનર સર્વલેટ ક્લાસ લોડ કરે \\
\textbf{ઇનિશિયલાઇઝેશન} & \texttt{init()} & એકવાર કોલ થાય, રિસોર્સ સેટઅપ \\
\textbf{સર્વિસ} & \texttt{service()} & દરેક રિક્વેસ્ટ હેન્ડલ કરે
(doGet/doPost) \\
\textbf{ડિસ્ટ્રક્શન} & \texttt{destroy()} & અનલોડ કરતા પહેલા સફાઈ \\
\end{longtable}
}

\begin{itemize}
\tightlist
\item
  \textbf{થ્રેડ સેફ્ટી}: બહુવિધ રિક્વેસ્ટ એકસાથે હેન્ડલ થાય
\item
  \textbf{સિંગલ ઇન્સ્ટન્સ}: એક સર્વલેટ ઇન્સ્ટન્સ બધી રિક્વેસ્ટ હેન્ડલ કરે
\item
  \textbf{કન્ટેનર મેનેજ્ડ}: વેબ કન્ટેનર લાઇફસાઇકલ મેનેજ કરે
\end{itemize}

\end{solutionbox}
\begin{mnemonicbox}
``LISD = Load Init Service Destroy''

\end{mnemonicbox}
\subsection*{પ્રશ્ન 3(અ OR) [3
ગુણ]}\label{uxaaauxab0uxab6uxaa8-3uxa85-or-3-uxa97uxaa3}

\textbf{સર્વલેટ માં web.xml ફાઇલ સમજાવો.}

\begin{solutionbox}

\textbf{web.xml હેતુ:}

{\def\LTcaptype{none} % do not increment counter
\begin{longtable}[]{@{}ll@{}}
\toprule\noalign{}
એલિમેન્ટ & વર્ણન \\
\midrule\noalign{}
\endhead
\bottomrule\noalign{}
\endlastfoot
\textbf{ડિપ્લોયમેન્ટ ડિસ્ક્રિપ્ટર} & વેબ એપ્લિકેશન માટે કન્ફિગરેશન ફાઇલ \\
\textbf{સર્વલેટ મેપિંગ} & URL પેટર્ન સર્વલેટ સાથે મેપ કરે \\
\textbf{ઇનિશિયલાઇઝેશન} & સર્વલેટ પેરામીટર અને લોડ ઓર્ડર \\
\end{longtable}
}

\begin{itemize}
\tightlist
\item
  \textbf{સ્થાન}: WEB-INF ડિરેક્ટરી
\item
  \textbf{ફોર્મેટ}: XML કન્ફિગરેશન ફાઇલ
\end{itemize}

\end{solutionbox}
\begin{mnemonicbox}
``DMI = Deployment Mapping Initialization''

\end{mnemonicbox}
\subsection*{પ્રશ્ન 3(બ OR) [4
ગુણ]}\label{uxaaauxab0uxab6uxaa8-3uxaac-or-4-uxa97uxaa3}

\textbf{સર્વલેટની વિશેષતાની યાદી બનાવો અને સમજાવો.}

\begin{solutionbox}

\textbf{સર્વલેટ વિશેષતાઓ:}

{\def\LTcaptype{none} % do not increment counter
\begin{longtable}[]{@{}ll@{}}
\toprule\noalign{}
વિશેષતા & વર્ણન \\
\midrule\noalign{}
\endhead
\bottomrule\noalign{}
\endlastfoot
\textbf{પ્લેટફોર્મ સ્વતંત્ર} & એકવાર લખો, ગમે ત્યાં ચલાવો \\
\textbf{સર્વર-સાઇડ} & વેબ સર્વર પર એક્ઝિક્યુટ થાય \\
\textbf{પ્રોટોકોલ સ્વતંત્ર} & HTTP, FTP વગેરે સપોર્ટ કરે \\
\textbf{પર્સિસ્ટન્ટ} & રિક્વેસ્ટ વચ્ચે મેમરીમાં રહે \\
\textbf{સિક્યોર} & બિલ્ટ-ઇન સિક્યોરિટી ફીચર્સ \\
\end{longtable}
}

\begin{itemize}
\tightlist
\item
  \textbf{પ્રદર્શન}: CGI સ્ક્રિપ્ટ કરતાં બહેતર
\item
  \textbf{સ્કેલેબિલિટી}: બહુવિધ રિક્વેસ્ટ કાર્યક્ષમતાથી સંભાળે
\end{itemize}

\end{solutionbox}
\begin{mnemonicbox}
``PSPPS = Platform Server Protocol Persistent
Secure''

\end{mnemonicbox}
\subsection*{પ્રશ્ન 3(ક OR) [7
ગુણ]}\label{uxaaauxab0uxab6uxaa8-3uxa95-or-7-uxa97uxaa3}

\textbf{સર્વલેટમાં સેશન ટ્રેકિંગ સમજાવો.}

\begin{solutionbox}

\textbf{સેશન ટ્રેકિંગ મેથડ:}

{\def\LTcaptype{none} % do not increment counter
\begin{longtable}[]{@{}ll@{}}
\toprule\noalign{}
મેથડ & વર્ણન \\
\midrule\noalign{}
\endhead
\bottomrule\noalign{}
\endlastfoot
\textbf{કુકીઝ} & બ્રાઉઝરમાં સ્ટોર થતો નાનો ડેટા \\
\textbf{URL રીરાઇટિંગ} & URL માં સેશન ID \\
\textbf{હિડન ફોર્મ ફીલ્ડ} & ફોર્મમાં સેશન ડેટા \\
\textbf{HttpSession} & સર્વર-સાઇડ સેશન ઓબ્જેક્ટ \\
\end{longtable}
}

\textbf{HttpSession અમલીકરણ:}

\begin{verbatim}
HttpSession session = request.getSession();
session.setAttribute("user", username);
String user = (String) session.getAttribute("user");
\end{verbatim}

\begin{center}
\textbf{Mermaid Diagram (Code)}
\begin{verbatim}
{Shaded}
{Highlighting}[]
graph LR
    A[ક્લાયન્ટ રિક્વેસ્ટ] {-{-}{} B[સર્વર સેશન ID ચેક કરે]}
    B {-{-}{} C[સેશન અસ્તિત્વમાં છે?]}
    C {-{-}{}|હા| D[વર્તમાન સેશન વાપરો]}
    C {-{-}{}|ના| E[નવું સેશન બનાવો]}
    D {-{-}{} F[રિક્વેસ્ટ પ્રોસેસ કરો]}
    E {-{-}{} F}
{Highlighting}
{Shaded}
\end{verbatim}
\end{center}

\begin{itemize}
\tightlist
\item
  \textbf{હેતુ}: HTTP રિક્વેસ્ટ વચ્ચે સ્ટેટ જાળવવું
\item
  \textbf{HttpSession}: સૌથી સામાન્ય રીતે વપરાતો મેથડ
\end{itemize}

\end{solutionbox}
\begin{mnemonicbox}
``CUHH = Cookies URL Hidden HttpSession''

\end{mnemonicbox}
\subsection*{પ્રશ્ન 4(અ) [3
ગુણ]}\label{uxaaauxab0uxab6uxaa8-4uxa85-3-uxa97uxaa3}

\textbf{JSP નું આર્કિટેક્ચર ડાયાગ્રામ સાથે સમજાવો.}

\begin{solutionbox}

\textbf{JSP આર્કિટેક્ચર:}

\begin{center}
\textbf{Mermaid Diagram (Code)}
\begin{verbatim}
{Shaded}
{Highlighting}[]
graph LR
    A[JSP પેજ] {-{-}{} B[JSP એન્જિન/કન્ટેનર]}
    B {-{-}{} C[સર્વલેટ કોડ જનરેટ]}
    C {-{-}{} D[બાઇટકોડમાં કમ્પાઇલ]}
    D {-{-}{} E[સર્વલેટ એક્ઝિક્યુટ]}
    E {-{-}{} F[HTML રિસ્પોન્સ]}
{Highlighting}
{Shaded}
\end{verbatim}
\end{center}

{\def\LTcaptype{none} % do not increment counter
\begin{longtable}[]{@{}ll@{}}
\toprule\noalign{}
કમ્પોનન્ટ & ભૂમિકા \\
\midrule\noalign{}
\endhead
\bottomrule\noalign{}
\endlastfoot
\textbf{JSP એન્જિન} & JSP ને સર્વલેટમાં રૂપાંતરિત કરે \\
\textbf{વેબ કન્ટેનર} & JSP લાઇફસાઇકલ મેનેજ કરે \\
\textbf{જનરેટેડ સર્વલેટ} & વાસ્તવિક એક્ઝિક્યુશન યુનિટ \\
\end{longtable}
}

\end{solutionbox}
\begin{mnemonicbox}
``JSP = Java Server Pages (Page to Servlet)''

\end{mnemonicbox}
\subsection*{પ્રશ્ન 4(બ) [4
ગુણ]}\label{uxaaauxab0uxab6uxaa8-4uxaac-4-uxa97uxaa3}

\textbf{ઉદાહરણ સાથે JSP scripting elements સમજાવો.}

\begin{solutionbox}

\textbf{JSP સ્ક્રિપ્ટિંગ એલિમેન્ટ્સ:}

{\def\LTcaptype{none} % do not increment counter
\begin{longtable}[]{@{}lll@{}}
\toprule\noalign{}
એલિમેન્ટ & સિન્ટેક્સ & હેતુ \\
\midrule\noalign{}
\endhead
\bottomrule\noalign{}
\endlastfoot
\textbf{સ્ક્રિપ્ટલેટ} & \texttt{\textless{}\%\ code\ \%\textgreater{}} &
જાવા કોડ બ્લોક \\
\textbf{એક્સ્પ્રેશન} & \texttt{\textless{}\%=\ expression\ \%\textgreater{}}
& આઉટપુટ વેલ્યુ \\
\textbf{ડિક્લેરેશન} &
\texttt{\textless{}\%!\ declaration\ \%\textgreater{}} & વેરિયેબલ/મેથડ \\
\end{longtable}
}

\textbf{ઉદાહરણો:}

\begin{verbatim}
{\%!} int count = 0; \%{}               {!{-}{-} ડિક્લેરેશન {-}{-}}
{\%} count++; \%{}                      {!{-}{-} સ્ક્રિપ્ટલેટ {-}{-}}
{\%=} "Count: " + count \%{}            {!{-}{-} એક્સ્પ્રેશન {-}{-}}
\end{verbatim}

\end{solutionbox}
\begin{mnemonicbox}
``SED = Scriptlet Expression Declaration''

\end{mnemonicbox}
\subsection*{પ્રશ્ન 4(ક) [7
ગુણ]}\label{uxaaauxab0uxab6uxaa8-4uxa95-7-uxa97uxaa3}

\textbf{JSP જીવન ચક્ર સમજાવો.}

\begin{solutionbox}

\textbf{JSP લાઇફ સાઇકલ તબક્કા:}

\begin{center}
\textbf{Mermaid Diagram (Code)}
\begin{verbatim}
{Shaded}
{Highlighting}[]
graph LR
    A[JSP પેજ બનાવ્યું] {-{-}{} B[સર્વલેટમાં ટ્રાન્સલેશન]}
    B {-{-}{} C[સર્વલેટ કમ્પાઇલેશન]}
    C {-{-}{} D[ક્લાસ લોડિંગ]}
    D {-{-}{} E[ઇન્સ્ટેન્શિએશન]}
    E {-{-}{} F[jspInit કોલ]}
    F {-{-}{} G[\_jspService રિક્વેસ્ટ હેન્ડલ કરે]}
    G {-{-}{} G}
    G {-{-}{} H[jspDestroy કોલ]}
{Highlighting}
{Shaded}
\end{verbatim}
\end{center}

{\def\LTcaptype{none} % do not increment counter
\begin{longtable}[]{@{}ll@{}}
\toprule\noalign{}
તબક્કો & વર્ણન \\
\midrule\noalign{}
\endhead
\bottomrule\noalign{}
\endlastfoot
\textbf{ટ્રાન્સલેશન} & JSP સર્વલેટ સોર્સમાં કન્વર્ટ \\
\textbf{કમ્પાઇલેશન} & સર્વલેટ સોર્સ બાઇટકોડમાં કમ્પાઇલ \\
\textbf{લોડિંગ} & સર્વલેટ ક્લાસ JVM દ્વારા લોડ \\
\textbf{ઇન્સ્ટેન્શિએશન} & સર્વલેટ ઓબ્જેક્ટ બનાવ્યું \\
\textbf{ઇનિશિયલાઇઝેશન} & \texttt{jspInit()} મેથડ કોલ \\
\textbf{રિક્વેસ્ટ પ્રોસેસિંગ} & \texttt{\_jspService()} રિક્વેસ્ટ હેન્ડલ કરે \\
\textbf{ડિસ્ટ્રક્શન} & \texttt{jspDestroy()} સફાઈ મેથડ \\
\end{longtable}
}

\begin{itemize}
\tightlist
\item
  \textbf{કન્ટેનર મેનેજ્ડ}: વેબ કન્ટેનર સંપૂર્ણ લાઇફસાઇકલ હેન્ડલ કરે
\item
  \textbf{ઓટોમેટિક}: ટ્રાન્સલેશન અને કમ્પાઇલેશન આપોઆપ થાય
\end{itemize}

\end{solutionbox}
\begin{mnemonicbox}
``TCLIIRD = Translation Compilation Loading
Instantiation Init Request Destroy''

\end{mnemonicbox}
\subsection*{પ્રશ્ન 4(અ OR) [3
ગુણ]}\label{uxaaauxab0uxab6uxaa8-4uxa85-or-3-uxa97uxaa3}

\textbf{JSP અને સર્વલેટ વચ્ચેનો તફાવત સમજાવો.}

\begin{solutionbox}

{\def\LTcaptype{none} % do not increment counter
\begin{longtable}[]{@{}lll@{}}
\toprule\noalign{}
લક્ષણ & JSP & સર્વલેટ \\
\midrule\noalign{}
\endhead
\bottomrule\noalign{}
\endlastfoot
\textbf{કોડ સ્ટાઇલ} & HTML સાથે જાવા & શુદ્ધ જાવા કોડ \\
\textbf{ડેવલપમેન્ટ} & UI માટે સરળ & લોજિક માટે વધુ સારું \\
\textbf{કમ્પાઇલેશન} & ઓટોમેટિક & મેન્યુઅલ \\
\textbf{મોડિફિકેશન} & પુનઃકમ્પાઇલેશનની જરૂર નથી & પુનઃકમ્પાઇલેશન જરૂરી \\
\end{longtable}
}

\end{solutionbox}
\begin{mnemonicbox}
``HTML vs Java = JSP vs સર્વલેટ''

\end{mnemonicbox}
\subsection*{પ્રશ્ન 4(બ OR) [4
ગુણ]}\label{uxaaauxab0uxab6uxaa8-4uxaac-or-4-uxa97uxaa3}

\textbf{JSP ના ફાયદાની યાદી બનાવો અને સમજાવો.}

\begin{solutionbox}

\textbf{JSP ફાયદા:}

{\def\LTcaptype{none} % do not increment counter
\begin{longtable}[]{@{}ll@{}}
\toprule\noalign{}
ફાયદો & વર્ણન \\
\midrule\noalign{}
\endhead
\bottomrule\noalign{}
\endlastfoot
\textbf{સરળ ડેવલપમેન્ટ} & HTML જેવું સિન્ટેક્સ જાવા સાથે \\
\textbf{ઓટોમેટિક કમ્પાઇલેશન} & મેન્યુઅલ કમ્પાઇલેશનની જરૂર નથી \\
\textbf{પ્લેટફોર્મ સ્વતંત્ર} & કોઈપણ જાવા-સક્ષમ સર્વર પર ચાલે \\
\textbf{ચિંતાઓનું વિભાજન} & ડિઝાઇન લોજિકથી અલગ \\
\textbf{પુનઃઉપયોગી કમ્પોનન્ટ} & ટેગ લાઇબ્રેરી અને બીન્સ \\
\end{longtable}
}

\begin{itemize}
\tightlist
\item
  \textbf{ડેવલપર ફ્રેન્ડલી}: વેબ ડિઝાઇનર JSP સાથે સરળતાથી કામ કરી શકે
\item
  \textbf{જાળવણી}: સર્વલેટ કરતાં મોડિફાઇ કરવું સરળ
\end{itemize}

\end{solutionbox}
\begin{mnemonicbox}
``EAPSR = Easy Automatic Platform Separation
Reusable''

\end{mnemonicbox}
\subsection*{પ્રશ્ન 4(ક OR) [7
ગુણ]}\label{uxaaauxab0uxab6uxaa8-4uxa95-or-7-uxa97uxaa3}

\textbf{કુકી શું છે? JSP પૃષ્ઠનો ઉપયોગ કરીને કુકી કેવી રીતે વાંચવી અને કાઢી નાખવી તે
સમજાવો.}

\begin{solutionbox}

\textbf{કુકી ઓવરવ્યૂ:} કુકી એ ક્લાયન્ટના બ્રાઉઝર પર સ્ટોર થતો નાનો ડેટા છે જે સ્ટેટ
જાળવવા માટે વપરાય છે.

\textbf{કુકી ઓપરેશન:}

{\def\LTcaptype{none} % do not increment counter
\begin{longtable}[]{@{}ll@{}}
\toprule\noalign{}
ઓપરેશન & JSP કોડ \\
\midrule\noalign{}
\endhead
\bottomrule\noalign{}
\endlastfoot
\textbf{બનાવવું} &
\texttt{Cookie\ cookie\ =\ new\ Cookie("name",\ "value");} \\
\textbf{ઉમેરવું} & \texttt{response.addCookie(cookie);} \\
\textbf{વાંચવું} &
\texttt{Cookie[]\ cookies\ =\ request.getCookies();} \\
\textbf{કાઢવું} & \texttt{cookie.setMaxAge(0);} \\
\end{longtable}
}

\textbf{કુકી વાંચવાનું ઉદાહરણ:}

\begin{verbatim}
{\%}
Cookie[] cookies = request.getCookies();
if (cookies != null) \{
    for (Cookie cookie : cookies) \{
        if ("username".equals(cookie.getName())) \{
            out.println("યુઝર: " + cookie.getValue());
        \}
    \}
\}
\%{}
\end{verbatim}

\textbf{કુકી કાઢવાનું ઉદાહરણ:}

\begin{verbatim}
{\%}
Cookie cookie = new Cookie("username", "");
cookie.setMaxAge(0);
response.addCookie(cookie);
\%{}
\end{verbatim}

\end{solutionbox}
\begin{mnemonicbox}
``CARD = Create Add Read Delete''

\end{mnemonicbox}
\subsection*{પ્રશ્ન 5(અ) [3
ગુણ]}\label{uxaaauxab0uxab6uxaa8-5uxa85-3-uxa97uxaa3}

\textbf{MVC આર્કિટેક્ચરનું મહત્વ સમજાવો.}

\begin{solutionbox}

\textbf{MVC મહત્વ:}

{\def\LTcaptype{none} % do not increment counter
\begin{longtable}[]{@{}ll@{}}
\toprule\noalign{}
લાભ & વર્ણન \\
\midrule\noalign{}
\endhead
\bottomrule\noalign{}
\endlastfoot
\textbf{ચિંતાઓનું વિભાજન} & લોજિક, પ્રેઝન્ટેશન, ડેટા અલગ \\
\textbf{જાળવણીયોગ્યતા} & વ્યક્તિગત કમ્પોનન્ટ સરળતાથી મોડિફાઇ કરી શકાય \\
\textbf{ટેસ્ટેબિલિટી} & કમ્પોનન્ટ સ્વતંત્ર રીતે ટેસ્ટ કરી શકાય \\
\end{longtable}
}

\begin{itemize}
\tightlist
\item
  \textbf{કોડ ઓર્ગેનાઇઝેશન}: વધુ સારી સ્ટ્રક્ચર અને ઓર્ગેનાઇઝેશન
\item
  \textbf{ટીમ ડેવલપમેન્ટ}: બહુવિધ ડેવલપર એકસાથે કામ કરી શકે
\end{itemize}

\end{solutionbox}
\begin{mnemonicbox}
``SMT = Separation Maintainability Testability''

\end{mnemonicbox}
\subsection*{પ્રશ્ન 5(બ) [4
ગુણ]}\label{uxaaauxab0uxab6uxaa8-5uxaac-4-uxa97uxaa3}

\textbf{સંક્ષિપ્તમાં આસ્પેક્ટ ઓરિએન્ટેડ પ્રોગ્રામિંગ અને ડિપેન્ડન્સી ઇન્જેક્શન સમજાવો.}

\begin{solutionbox}

\textbf{આસ્પેક્ટ ઓરિએન્ટેડ પ્રોગ્રામિંગ (AOP):}

{\def\LTcaptype{none} % do not increment counter
\begin{longtable}[]{@{}ll@{}}
\toprule\noalign{}
કન્સેપ્ટ & વર્ણન \\
\midrule\noalign{}
\endhead
\bottomrule\noalign{}
\endlastfoot
\textbf{ક્રોસ-કટિંગ કન્સર્ન} & લોગિંગ, સિક્યોરિટી, ટ્રાન્ઝેક્શન \\
\textbf{આસ્પેક્ટ} & ક્રોસ-કટિંગ ફંક્શનાલિટીના મોડ્યુલર યુનિટ \\
\textbf{જોઇન પોઇન્ટ} & જ્યાં આસ્પેક્ટ લાગુ કરવાય \\
\end{longtable}
}

\textbf{ડિપેન્ડન્સી ઇન્જેક્શન (DI):}

{\def\LTcaptype{none} % do not increment counter
\begin{longtable}[]{@{}ll@{}}
\toprule\noalign{}
કન્સેપ્ટ & વર્ણન \\
\midrule\noalign{}
\endhead
\bottomrule\noalign{}
\endlastfoot
\textbf{ઇન્વર્સન ઓફ કન્ટ્રોલ} & ડિપેન્ડન્સી બાહ્યથી આપવામાં આવે \\
\textbf{લૂઝ કપલિંગ} & ઓબ્જેક્ટ ડિપેન્ડન્સી બનાવતા નથી \\
\textbf{કન્ફિગરેશન} & ડિપેન્ડન્સી બાહ્યથી કન્ફિગર કરાય \\
\end{longtable}
}

\end{solutionbox}
\begin{mnemonicbox}
``AOP = Aspects Over Points, DI = Dependencies
Injected''

\end{mnemonicbox}
\subsection*{પ્રશ્ન 5(ક) [7
ગુણ]}\label{uxaaauxab0uxab6uxaa8-5uxa95-7-uxa97uxaa3}

\textbf{MVC આર્કિટેક્ચર સમજાવો.}

\begin{solutionbox}

\textbf{MVC કમ્પોનન્ટ્સ:}

\begin{center}
\textbf{Mermaid Diagram (Code)}
\begin{verbatim}
{Shaded}
{Highlighting}[]
graph LR
    A[View{br/{}પ્રેઝન્ટેશન લેયર] }
    B[Controller{br/{}કન્ટ્રોલ લેયર] }
    C[Model{br/{}બિઝનેસ લોજિક]}
    
    A {-{-}{} B}
    B {-{-}{} C}
    C {-{-}{} B}
    B {-{-}{} A}
{Highlighting}
{Shaded}
\end{verbatim}
\end{center}

{\def\LTcaptype{none} % do not increment counter
\begin{longtable}[]{@{}ll@{}}
\toprule\noalign{}
કમ્પોનન્ટ & જવાબદારી \\
\midrule\noalign{}
\endhead
\bottomrule\noalign{}
\endlastfoot
\textbf{મોડેલ} & બિઝનેસ લોજિક અને ડેટા મેનેજમેન્ટ \\
\textbf{વ્યૂ} & યુઝર ઇન્ટરફેસ અને પ્રેઝન્ટેશન \\
\textbf{કન્ટ્રોલર} & રિક્વેસ્ટ હેન્ડલિંગ અને ફ્લો કન્ટ્રોલ \\
\end{longtable}
}

\textbf{MVC ફ્લો:}

\begin{enumerate}
\tightlist
\item
  \textbf{યુઝર રિક્વેસ્ટ} \rightarrow કન્ટ્રોલર રિક્વેસ્ટ પ્રાપ્ત કરે
\item
  \textbf{કન્ટ્રોલર} \rightarrow રિક્વેસ્ટ પ્રોસેસ કરે, મોડેલ કોલ કરે
\item
  \textbf{મોડેલ} \rightarrow બિઝનેસ લોજિક પર્ફોર્મ કરે, ડેટા રિટર્ન કરે
\item
  \textbf{કન્ટ્રોલર} \rightarrow યોગ્ય વ્યૂ સિલેક્ટ કરે
\item
  \textbf{વ્યૂ} \rightarrow યુઝરને રિસ્પોન્સ રેન્ડર કરે
\end{enumerate}

\textbf{ફાયદા:}

\begin{itemize}
\tightlist
\item
  \textbf{જાળવણીયોગ્યતા}: જવાબદારીઓનું સ્પષ્ટ વિભાજન
\item
  \textbf{પુનઃઉપયોગિતા}: કમ્પોનન્ટ પુનઃવાપરી શકાય
\item
  \textbf{ટેસ્ટેબિલિટી}: દરેક લેયર સ્વતંત્ર રીતે ટેસ્ટ કરી શકાય
\end{itemize}

\end{solutionbox}
\begin{mnemonicbox}
``MVC = Model View Controller (Business UI
Control)''

\end{mnemonicbox}
\subsection*{પ્રશ્ન 5(અ OR) [3
ગુણ]}\label{uxaaauxab0uxab6uxaa8-5uxa85-or-3-uxa97uxaa3}

\textbf{MVC આર્કિટેક્ચરના ફાયદા સમજાવો.}

\begin{solutionbox}

\textbf{MVC ફાયદા:}

{\def\LTcaptype{none} % do not increment counter
\begin{longtable}[]{@{}ll@{}}
\toprule\noalign{}
ફાયદો & વર્ણન \\
\midrule\noalign{}
\endhead
\bottomrule\noalign{}
\endlastfoot
\textbf{કોડ પુનઃઉપયોગિતા} & કમ્પોનન્ટ વિવિધ એપ્લિકેશનમાં પુનઃવાપરી શકાય \\
\textbf{સમાંતર ડેવલપમેન્ટ} & બહુવિધ ડેવલપર વિવિધ લેયર પર કામ કરી શકે \\
\textbf{સરળ ટેસ્ટિંગ} & દરેક કમ્પોનન્ટ સ્વતંત્ર રીતે ટેસ્ટ કરાય \\
\textbf{જાળવણી} & એક લેયરમાં ફેરફાર અન્યને અસર કરતા નથી \\
\end{longtable}
}

\end{solutionbox}
\begin{mnemonicbox}
``CPEM = Code Parallel Easy Maintenance''

\end{mnemonicbox}
\subsection*{પ્રશ્ન 5(બ OR) [4
ગુણ]}\label{uxaaauxab0uxab6uxaa8-5uxaac-or-4-uxa97uxaa3}

\textbf{સ્પ્રિંગ અને સ્પ્રિંગ બૂટ વચ્ચેનો તફાવત સમજાવો.}

\begin{solutionbox}

{\def\LTcaptype{none} % do not increment counter
\begin{longtable}[]{@{}lll@{}}
\toprule\noalign{}
લક્ષણ & સ્પ્રિંગ & સ્પ્રિંગ બૂટ \\
\midrule\noalign{}
\endhead
\bottomrule\noalign{}
\endlastfoot
\textbf{કન્ફિગરેશન} & મેન્યુઅલ XML/Java કન્ફિગ & ઓટો-કન્ફિગરેશન \\
\textbf{સેટઅપ ટાઇમ} & વધુ સેટઅપ જરૂરી & ન્યૂનતમ સેટઅપ \\
\textbf{એમ્બેડેડ સર્વર} & બાહ્ય સર્વરની જરૂર & બિલ્ટ-ઇન સર્વર \\
\textbf{ડિપેન્ડન્સી} & મેન્યુઅલ ડિપેન્ડન્સી મેનેજમેન્ટ & સ્ટાર્ટર ડિપેન્ડન્સી \\
\end{longtable}
}

\begin{itemize}
\tightlist
\item
  \textbf{સ્પ્રિંગ}: કન્ફિગરેશન જરૂરી વ્યાપક ફ્રેમવર્ક
\item
  \textbf{સ્પ્રિંગ બૂટ}: કન્ફિગરેશન ઉપર કન્વેન્શન અપ્રોચ
\end{itemize}

\end{solutionbox}
\begin{mnemonicbox}
``Manual vs Auto = સ્પ્રિંગ vs સ્પ્રિંગ બૂટ''

\end{mnemonicbox}
\subsection*{પ્રશ્ન 5(ક OR) [7
ગુણ]}\label{uxaaauxab0uxab6uxaa8-5uxa95-or-7-uxa97uxaa3}

\textbf{સ્પ્રિંગ ફ્રેમવર્કનું આર્કિટેક્ચર સમજાવો.}

\begin{solutionbox}

\textbf{સ્પ્રિંગ ફ્રેમવર્ક આર્કિટેક્ચર:}

\begin{center}
\textbf{Mermaid Diagram (Code)}
\begin{verbatim}
{Shaded}
{Highlighting}[]
graph TD
    A[કોર કન્ટેનર{br/{}IoC \& DI] }
    B[ડેટા એક્સેસ{br/{}JDBC, ORM] }
    C[વેબ લેયર{br/{}MVC, WebFlux]}
    D[AOP{br/{}આસ્પેક્ટ ઓરિએન્ટેડ]}
    E[ટેસ્ટ{br/{}ટેસ્ટિંગ સપોર્ટ]}
    
    A {-{-}{} B}
    A {-{-}{} C}
    A {-{-}{} D}
    A {-{-}{} E}
{Highlighting}
{Shaded}
\end{verbatim}
\end{center}

\textbf{સ્પ્રિંગ મોડ્યુલ:}

{\def\LTcaptype{none} % do not increment counter
\begin{longtable}[]{@{}ll@{}}
\toprule\noalign{}
મોડ્યુલ & હેતુ \\
\midrule\noalign{}
\endhead
\bottomrule\noalign{}
\endlastfoot
\textbf{કોર કન્ટેનર} & IoC કન્ટેનર, ડિપેન્ડન્સી ઇન્જેક્શન \\
\textbf{ડેટા એક્સેસ} & JDBC, ORM, ટ્રાન્ઝેક્શન મેનેજમેન્ટ \\
\textbf{વેબ} & વેબ MVC, REST સર્વિસ \\
\textbf{AOP} & આસ્પેક્ટ-ઓરિએન્ટેડ પ્રોગ્રામિંગ \\
\textbf{સિક્યોરિટી} & ઓથેન્ટિકેશન અને ઓથોરાઇઝેશન \\
\textbf{ટેસ્ટ} & ટેસ્ટિંગ સપોર્ટ અને મોક ઓબ્જેક્ટ \\
\end{longtable}
}

\textbf{મુખ્ય લક્ષણો:}

\begin{itemize}
\tightlist
\item
  \textbf{IoC કન્ટેનર}: ઓબ્જેક્ટ બનાવટ અને ડિપેન્ડન્સી મેનેજ કરે
\item
  \textbf{AOP સપોર્ટ}: ક્રોસ-કટિંગ કન્સર્ન હેન્ડલિંગ
\item
  \textbf{ટ્રાન્ઝેક્શન મેનેજમેન્ટ}: ડિક્લેરેટિવ ટ્રાન્ઝેક્શન સપોર્ટ
\item
  \textbf{MVC ફ્રેમવર્ક}: વેબ એપ્લિકેશન ડેવલપમેન્ટ
\end{itemize}

\end{solutionbox}
\begin{mnemonicbox}
``CDWAST = Core Data Web AOP Security Test''

\end{mnemonicbox}

\end{document}
