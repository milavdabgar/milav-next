\documentclass[10pt,a4paper]{article}

% content/resources/templates/preamble.tex
\usepackage[margin=0.6in]{geometry}
\author{Milav Dabgar}
\usepackage{amsmath,amssymb,amsthm}
\usepackage{booktabs}
\usepackage{multirow}
\usepackage{xcolor}
\usepackage{tcolorbox}
\tcbuselibrary{breakable,skins}
\usepackage[colorlinks=true,linkcolor=blue]{hyperref}
\usepackage{titlesec}
\usepackage{enumitem}
\usepackage{tikz}
\usepackage{pgfplots}
\usepackage{circuitikz}
\usepackage[version=4]{mhchem}
\usepackage{longtable}
\usepackage{array}
\usepackage{float}
\usepackage{caption}
\usepackage{listings}

\lstset{
  basicstyle=\small\ttfamily,
  breaklines=true,
  breakatwhitespace=false,
  postbreak=\mbox{\textcolor{red}{$\hookrightarrow$}\space},
  float=false,
  numbers=left,
  numberstyle=\tiny\color{gray},
  numbersep=10pt,
  xleftmargin=2em,
  keywordstyle=\color{blue},
  commentstyle=\color{green!60!black},
  stringstyle=\color{purple},
  backgroundcolor=\color{gray!5},
  showstringspaces=false,
  tabsize=2,
  captionpos=b,
  keepspaces=true,
  columns=flexible
}

\pgfplotsset{compat=1.18}
\usetikzlibrary{shapes,arrows,positioning,calc,patterns,decorations.pathmorphing,decorations.markings,arrows.meta}

% Color scheme
\definecolor{headcolor}{RGB}{0,102,204}
\definecolor{keycolor}{RGB}{220,20,60}
\definecolor{solutioncolor}{RGB}{34,139,34}
\definecolor{mnemoniccolor}{RGB}{148,0,211}
\definecolor{codecolor}{RGB}{0,0,100}

% Spacing
\setlength{\parskip}{3pt}
\setlist[itemize]{nosep}
\setlist[enumerate]{nosep}

% Title formatting
\titleformat{\section}{\Large\bfseries\color{headcolor}}{\thesection}{1em}{}
\titleformat{\subsection}{\large\bfseries\color{headcolor}}{\thesubsection}{1em}{}

% Pandoc tightlist compatibility
\providecommand{\tightlist}{%
  \setlength{\itemsep}{0pt}\setlength{\parskip}{0pt}}

% Pandoc longtable compatibility
\newcounter{none}
\def\thenone{}


% content/resources/templates/english-boxes.tex
% This file is currently empty - it exists to maintain consistency with the import structure.
% Add custom environments here if needed in the future.


\begin{document}

\begin{center}
{\Huge\bfseries\color{headcolor} Subject Name Solutions}\\[5pt]
{\LARGE 4351603 -- Summer 2024}\\[3pt]
{\large Semester 1 Study Material}\\[3pt]
{\normalsize\textit{Detailed Solutions and Explanations}}
\end{center}

\vspace{10pt}

\subsection*{Question 1(a) [3 marks]}\label{q1a}

\textbf{Explain the difference between AWT and Swing.}

\begin{solutionbox}

{\def\LTcaptype{none} % do not increment counter
\begin{longtable}[]{@{}lll@{}}
\toprule\noalign{}
Feature & AWT & Swing \\
\midrule\noalign{}
\endhead
\bottomrule\noalign{}
\endlastfoot
\textbf{Platform} & Platform dependent & Platform independent \\
\textbf{Components} & Heavy weight & Light weight \\
\textbf{Look \& Feel} & Native OS look & Pluggable look \& feel \\
\textbf{Performance} & Faster & Slower than AWT \\
\end{longtable}
}

\textbf{Key Points:}

\begin{itemize}
\tightlist
\item
  \textbf{Heavy vs Light}: AWT uses native OS components, Swing uses
  pure Java
\item
  \textbf{Appearance}: AWT follows OS style, Swing offers consistent
  look across platforms
\item
  \textbf{Features}: Swing provides more advanced components like
  JTable, JTree
\end{itemize}

\end{solutionbox}
\begin{mnemonicbox}
``Swing Provides Lightweight Components''

\end{mnemonicbox}
\begin{center}\rule{0.5\linewidth}{0.5pt}\end{center}

\subsection*{Question 1(b) [4 marks]}\label{q1b}

\textbf{Explain Mouse Motion Listener with example.}

\begin{solutionbox}

MouseMotionListener interface handles mouse movement events in Java
Swing applications.


{\def\LTcaptype{none} % do not increment counter
\vspace{-5pt}
\captionof{table}{Mouse Motion Events}
\vspace{-10pt}
\begin{longtable}[]{@{}ll@{}}
\toprule\noalign{}
Method & Purpose \\
\midrule\noalign{}
\endhead
\bottomrule\noalign{}
\endlastfoot
\textbf{mouseDragged()} & Called when mouse is dragged \\
\textbf{mouseMoved()} & Called when mouse is moved \\
\end{longtable}
}

\textbf{Code Example:}

\begin{verbatim}
import javax.swing.*;
import java.awt.event.*;

class MouseMotionExample extends JFrame implements MouseMotionListener \{
    JLabel label;
    
    MouseMotionExample() \{
        label = new JLabel("Move mouse here");
        add(label);
        addMouseMotionListener(this);
        setSize(400, 300);
        setVisible(true);
    \}
    
    public void mouseMoved(MouseEvent e) \{
        label.setText("Mouse at: " + e.getX() + ", " + e.getY());
    \}
    
    public void mouseDragged(MouseEvent e) \{
        label.setText("Dragging at: " + e.getX() + ", " + e.getY());
    \}
\}
\end{verbatim}

\end{solutionbox}
\begin{mnemonicbox}
``Mouse Motion Makes Dynamic''

\end{mnemonicbox}
\begin{center}\rule{0.5\linewidth}{0.5pt}\end{center}

\subsection*{Question 1(c) [7 marks]}\label{q1c}

\textbf{Develop a program to create checkboxes for different courses
belonging to a university such that the course selected would be
displayed.}

\begin{solutionbox}

\begin{verbatim}
import javax.swing.*;
import java.awt.*;
import java.awt.event.*;

public class CourseSelection extends JFrame implements ItemListener \{
    JCheckBox java, python, cpp, web;
    JTextArea display;
    
    public CourseSelection() \{
        setTitle("University Course Selection");
        setLayout(new FlowLayout());
        
        // Create checkboxes
        java = new JCheckBox("Java Programming");
        python = new JCheckBox("Python Programming");
        cpp = new JCheckBox("C++ Programming");
        web = new JCheckBox("Web Development");
        
        // Add listeners
        java.addItemListener(this);
        python.addItemListener(this);
        cpp.addItemListener(this);
        web.addItemListener(this);
        
        // Display area
        display = new JTextArea(10, 30);
        display.setEditable(false);
        
        // Add components
        add(new JLabel("Select Courses:"));
        add(java); add(python); add(cpp); add(web);
        add(new JScrollPane(display));
        
        setSize(400, 300);
        setDefaultCloseOperation(JFrame.EXIT\_ON\_CLOSE);
        setVisible(true);
    \}
    
    public void itemStateChanged(ItemEvent e) \{
        String courses = "Selected Courses:{n}";
        if(java.isSelected()) courses += "{- Java Programming}{n}";
        if(python.isSelected()) courses += "{- Python Programming}{n}";
        if(cpp.isSelected()) courses += "{- C++ Programming}{n}";
        if(web.isSelected()) courses += "{- Web Development}{n}";
        display.setText(courses);
    \}
    
    public static void main(String[] args) \{
        new CourseSelection();
    \}
\}
\end{verbatim}

\textbf{Key Features:}

\begin{itemize}
\tightlist
\item
  \textbf{ItemListener}: Detects checkbox state changes
\item
  \textbf{Dynamic Display}: Updates selected courses in real-time
\item
  \textbf{Multiple Selection}: Allows selecting multiple courses
\end{itemize}

\end{solutionbox}
\begin{mnemonicbox}
``Check Items Listen Dynamically''

\end{mnemonicbox}
\begin{center}\rule{0.5\linewidth}{0.5pt}\end{center}

\subsection*{Question 1(c) OR [7
marks]}\label{q1c}

\textbf{Develop a program to Implement Traffic signal (Red, Green and
Yellow) by using Swing components (Using JFrame, JRadioButton,
ItemListener etc.)}

\begin{solutionbox}

\begin{verbatim}
import javax.swing.*;
import java.awt.*;
import java.awt.event.*;

public class TrafficSignal extends JFrame implements ItemListener \{
    JRadioButton red, green, yellow;
    ButtonGroup group;
    JPanel signalPanel;
    
    public TrafficSignal() \{
        setTitle("Traffic Signal Simulator");
        setLayout(new BorderLayout());
        
        // Create radio buttons
        red = new JRadioButton("Red");
        green = new JRadioButton("Green"); 
        yellow = new JRadioButton("Yellow");
        
        // Group radio buttons
        group = new ButtonGroup();
        group.add(red); group.add(green); group.add(yellow);
        
        // Add listeners
        red.addItemListener(this);
        green.addItemListener(this);
        yellow.addItemListener(this);
        
        // Signal display panel
        signalPanel = new JPanel() \{
            public void paintComponent(Graphics g) \{
                super.paintComponent(g);
                g.setColor(Color.BLACK);
                g.fillRect(50, 50, 100, 200);
                
                // Draw circles
                g.setColor(red.isSelected() ? Color.RED : Color.GRAY);
                g.fillOval(65, 65, 70, 70);
                
                g.setColor(yellow.isSelected() ? Color.YELLOW : Color.GRAY);
                g.fillOval(65, 105, 70, 70);
                
                g.setColor(green.isSelected() ? Color.GREEN : Color.GRAY);
                g.fillOval(65, 145, 70, 70);
            \}
        \;}
        
        JPanel controlPanel = new JPanel();
        controlPanel.add(red); controlPanel.add(yellow); controlPanel.add(green);
        
        add(controlPanel, BorderLayout.SOUTH);
        add(signalPanel, BorderLayout.CENTER);
        
        setSize(300, 400);
        setDefaultCloseOperation(JFrame.EXIT\_ON\_CLOSE);
        setVisible(true);
    \}
    
    public void itemStateChanged(ItemEvent e) \{
        signalPanel.repaint();
    \}
    
    public static void main(String[] args) \{
        new TrafficSignal();
    \}
\}
\end{verbatim}

\textbf{Diagram:}

\begin{verbatim}
+{-{-}{-}{-}{-}{-}{-}{-}{-}{-}{-}{-}{-}{-}{-}{-}+}
|  Traffic Box   |
|   ┌─────────┐  |
|   │   RED   │  |
|   ├─────────┤  |
|   │ YELLOW  │  |
|   ├─────────┤  |
|   │  GREEN  │  |
|   └─────────┘  |
+{-{-}{-}{-}{-}{-}{-}{-}{-}{-}{-}{-}{-}{-}{-}{-}+}
  [R] [Y] [G]
\end{verbatim}

\end{solutionbox}
\begin{mnemonicbox}
``Radio Buttons Paint Graphics''

\end{mnemonicbox}
\begin{center}\rule{0.5\linewidth}{0.5pt}\end{center}

\subsection*{Question 2(a) [3 marks]}\label{q2a}

\textbf{Explain JDBC type-4 driver.}

\begin{solutionbox}

\textbf{JDBC Type-4 Driver (Native Protocol Driver)}

{\def\LTcaptype{none} % do not increment counter
\begin{longtable}[]{@{}ll@{}}
\toprule\noalign{}
Feature & Description \\
\midrule\noalign{}
\endhead
\bottomrule\noalign{}
\endlastfoot
\textbf{Type} & Pure Java driver \\
\textbf{Communication} & Direct database protocol \\
\textbf{Platform} & Platform independent \\
\textbf{Performance} & Highest performance \\
\end{longtable}
}

\textbf{Key Points:}

\begin{itemize}
\tightlist
\item
  \textbf{Pure Java}: No native code required
\item
  \textbf{Direct Connection}: Communicates directly with database
\item
  \textbf{Network Protocol}: Uses database's native network protocol
\item
  \textbf{Best Performance}: Fastest among all driver types
\end{itemize}

\end{solutionbox}
\begin{mnemonicbox}
``Pure Java Direct Protocol''

\end{mnemonicbox}
\begin{center}\rule{0.5\linewidth}{0.5pt}\end{center}

\subsection*{Question 2(b) [4 marks]}\label{q2b}

\textbf{Explain Commonly used Methods of Component class.}

\begin{solutionbox}


{\def\LTcaptype{none} % do not increment counter
\vspace{-5pt}
\captionof{table}{Component Class Methods}
\vspace{-10pt}
\begin{longtable}[]{@{}ll@{}}
\toprule\noalign{}
Method & Purpose \\
\midrule\noalign{}
\endhead
\bottomrule\noalign{}
\endlastfoot
\textbf{add()} & Adds component to container \\
\textbf{setSize()} & Sets component dimensions \\
\textbf{setLayout()} & Sets layout manager \\
\textbf{setVisible()} & Makes component visible/invisible \\
\textbf{setBounds()} & Sets position and size \\
\textbf{getSize()} & Returns component size \\
\end{longtable}
}

\textbf{Key Features:}

\begin{itemize}
\tightlist
\item
  \textbf{Layout Management}: Controls component arrangement
\item
  \textbf{Visibility Control}: Shows/hides components
\item
  \textbf{Size Management}: Controls component dimensions
\item
  \textbf{Container Operations}: Manages child components
\end{itemize}

\end{solutionbox}
\begin{mnemonicbox}
``Add Set Get Visibility''

\end{mnemonicbox}
\begin{center}\rule{0.5\linewidth}{0.5pt}\end{center}

\subsection*{Question 2(c) [7 marks]}\label{q2c}

\textbf{Develop a program using JDBC to display student's record (Enroll
No, Name, Address, Mobile No and Email-ID) from table `StuRec'.}

\begin{solutionbox}

\begin{verbatim}
import java.sql.*;
import javax.swing.*;
import javax.swing.table.DefaultTableModel;

public class StudentRecordDisplay extends JFrame \{
    JTable table;
    DefaultTableModel model;
    
    public StudentRecordDisplay() \{
        setTitle("Student Records");
        
        // Create table model
        String[] columns = \{"Enroll No", "Name", "Address", "Mobile", "Email"\;}
        model = new DefaultTableModel(columns, 0);
        table = new JTable(model);
        
        // Load data
        loadStudentData();
        
        add(new JScrollPane(table));
        setSize(600, 400);
        setDefaultCloseOperation(JFrame.EXIT\_ON\_CLOSE);
        setVisible(true);
    \}
    
    private void loadStudentData() \{
        try \{
            // Database connection
            Class.forName("com.mysql.cj.jdbc.Driver");
            Connection con = DriverManager.getConnection(
                "jdbc:mysql://localhost:3306/university", "root", "password");
            
            // Execute query
            Statement stmt = con.createStatement();
            ResultSet rs = stmt.executeQuery("SELECT * FROM StuRec");
            
            // Add data to table
            while(rs.next()) \{
                String[] row = \{
                    rs.getString("enrollno"),
                    rs.getString("name"),
                    rs.getString("address"),
                    rs.getString("mobile"),
                    rs.getString("email")
                \;}
                model.addRow(row);
            \}
            
            con.close();
        \} catch(Exception e) \{
            JOptionPane.showMessageDialog(this, "Error: " + e.getMessage());
        \}
    \}
    
    public static void main(String[] args) \{
        new StudentRecordDisplay();
    \}
\}
\end{verbatim}

\textbf{Database Table Structure:}

\begin{verbatim}
CREATE TABLE StuRec (
    enrollno VARCHAR(20) PRIMARY KEY,
    name VARCHAR(50),
    address VARCHAR(100),
    mobile VARCHAR(15),
    email VARCHAR(50)
);
\end{verbatim}

\end{solutionbox}
\begin{mnemonicbox}
``Connect Query Display Records''

\end{mnemonicbox}
\begin{center}\rule{0.5\linewidth}{0.5pt}\end{center}

\subsection*{Question 2(a) OR [3
marks]}\label{q2a}

\textbf{Write down the advantages and disadvantages of JDBC.}

\begin{solutionbox}


{\def\LTcaptype{none} % do not increment counter
\vspace{-5pt}
\captionof{table}{JDBC Advantages and Disadvantages}
\vspace{-10pt}
\begin{longtable}[]{@{}ll@{}}
\toprule\noalign{}
Advantages & Disadvantages \\
\midrule\noalign{}
\endhead
\bottomrule\noalign{}
\endlastfoot
\textbf{Platform Independent} & \textbf{Performance Overhead} \\
\textbf{Database Independent} & \textbf{Complex for beginners} \\
\textbf{Standard API} & \textbf{SQL dependency} \\
\textbf{Supports transactions} & \textbf{Manual resource management} \\
\end{longtable}
}

\textbf{Key Points:}

\begin{itemize}
\tightlist
\item
  \textbf{Portability}: Works across different platforms and databases
\item
  \textbf{Standardization}: Uniform API for database operations
\item
  \textbf{Performance}: Additional layer causes overhead
\item
  \textbf{Complexity}: Requires proper resource management
\end{itemize}

\end{solutionbox}
\begin{mnemonicbox}
``Platform Independent Standard Complex''

\end{mnemonicbox}
\begin{center}\rule{0.5\linewidth}{0.5pt}\end{center}

\subsection*{Question 2(b) OR [4
marks]}\label{q2b}

\textbf{Explain Border Layout.}

\begin{solutionbox}

BorderLayout divides container into five regions: North, South, East,
West, and Center.

\textbf{Diagram:}

\begin{verbatim}
+{-{-}{-}{-}{-}{-}{-}{-}{-}{-}{-}{-}{-}{-}{-}{-}{-}{-}+}
|      NORTH       |
+{-{-}{-}{-}{-}+{-}{-}{-}{-}{-}{-}+{-}{-}{-}{-}{-}+}
|WEST |CENTER| EAST|
+{-{-}{-}{-}{-}+{-}{-}{-}{-}{-}{-}+{-}{-}{-}{-}{-}+}
|      SOUTH       |
+{-{-}{-}{-}{-}{-}{-}{-}{-}{-}{-}{-}{-}{-}{-}{-}{-}{-}+}
\end{verbatim}


{\def\LTcaptype{none} % do not increment counter
\vspace{-5pt}
\captionof{table}{Border Layout Regions}
\vspace{-10pt}
\begin{longtable}[]{@{}lll@{}}
\toprule\noalign{}
Region & Position & Behavior \\
\midrule\noalign{}
\endhead
\bottomrule\noalign{}
\endlastfoot
\textbf{NORTH} & Top & Preferred height, full width \\
\textbf{SOUTH} & Bottom & Preferred height, full width \\
\textbf{EAST} & Right & Preferred width, full height \\
\textbf{WEST} & Left & Preferred width, full height \\
\textbf{CENTER} & Middle & Takes remaining space \\
\end{longtable}
}

\textbf{Code Example:}

\begin{verbatim}
setLayout(new BorderLayout());
add(new JButton("North"), BorderLayout.NORTH);
add(new JButton("Center"), BorderLayout.CENTER);
\end{verbatim}

\end{solutionbox}
\begin{mnemonicbox}
``North South East West Center''

\end{mnemonicbox}
\begin{center}\rule{0.5\linewidth}{0.5pt}\end{center}

\subsection*{Question 2(c) OR [7
marks]}\label{q2c}

\textbf{Develop an application to store, update, fetch and delete data
of Employee (NAME, AGE, SALARY and DEPARTMENT) using Hibernate CRUD
operations.}

\begin{solutionbox}

\textbf{Employee Entity Class:}

\begin{verbatim}
import javax.persistence.*;

@Entity
@Table(name = "employees")
public class Employee \{
    @Id
    @GeneratedValue(strategy = GenerationType.IDENTITY)
    private int id;
    
    private String name;
    private int age;
    private double salary;
    private String department;
    
    // Constructors, getters, setters
    public Employee() \{\}
    
    public Employee(String name, int age, double salary, String dept) \{
        this.name = name;
        this.age = age;
        this.salary = salary;
        this.department = dept;
    \}
    
    // Getters and Setters
    public int getId() \{ return id; \}
    public void setId(int id) \{ this.id = id; \}
    
    public String getName() \{ return name; \}
    public void setName(String name) \{ this.name = name; \}
    
    // ... other getters/setters
\}
\end{verbatim}

\textbf{CRUD Operations Class:}

\begin{verbatim}
import org.hibernate.*;
import org.hibernate.cfg.Configuration;

public class EmployeeCRUD \{
    private SessionFactory factory;
    
    public EmployeeCRUD() \{
        factory = new Configuration()
                    .configure("hibernate.cfg.xml")
                    .addAnnotatedClass(Employee.class)
                    .buildSessionFactory();
    \}
    
    // CREATE
    public void saveEmployee(Employee emp) \{
        Session session = factory.openSession();
        Transaction tx = session.beginTransaction();
        session.save(emp);
        tx.commit();
        session.close();
    \}
    
    // READ
    public Employee getEmployee(int id) \{
        Session session = factory.openSession();
        Employee emp = session.get(Employee.class, id);
        session.close();
        return emp;
    \}
    
    // UPDATE
    public void updateEmployee(Employee emp) \{
        Session session = factory.openSession();
        Transaction tx = session.beginTransaction();
        session.update(emp);
        tx.commit();
        session.close();
    \}
    
    // DELETE
    public void deleteEmployee(int id) \{
        Session session = factory.openSession();
        Transaction tx = session.beginTransaction();
        Employee emp = session.get(Employee.class, id);
        session.delete(emp);
        tx.commit();
        session.close();
    \}
\}
\end{verbatim}

\end{solutionbox}
\begin{mnemonicbox}
``Save Get Update Delete Hibernate''

\end{mnemonicbox}
\begin{center}\rule{0.5\linewidth}{0.5pt}\end{center}

\subsection*{Question 3(a) [3 marks]}\label{q3a}

\textbf{Explain Deployment Descriptor.}

\begin{solutionbox}

Deployment Descriptor (web.xml) is configuration file for web
applications containing servlet mappings, initialization parameters, and
security settings.


{\def\LTcaptype{none} % do not increment counter
\vspace{-5pt}
\captionof{table}{Deployment Descriptor Elements}
\vspace{-10pt}
\begin{longtable}[]{@{}ll@{}}
\toprule\noalign{}
Element & Purpose \\
\midrule\noalign{}
\endhead
\bottomrule\noalign{}
\endlastfoot
\textbf{\textless servlet\textgreater{}} & Defines servlet
configuration \\
\textbf{\textless servlet-mapping\textgreater{}} & Maps servlet to URL
pattern \\
\textbf{\textless init-param\textgreater{}} & Sets initialization
parameters \\
\textbf{\textless welcome-file-list\textgreater{}} & Default files to
serve \\
\end{longtable}
}

\textbf{Key Features:}

\begin{itemize}
\tightlist
\item
  \textbf{Configuration}: Central configuration for web app
\item
  \textbf{Servlet Mapping}: URL to servlet mapping
\item
  \textbf{Parameters}: Initialization and context parameters
\item
  \textbf{Security}: Authentication and authorization settings
\end{itemize}

\end{solutionbox}
\begin{mnemonicbox}
``Web XML Configuration Mapping''

\end{mnemonicbox}
\begin{center}\rule{0.5\linewidth}{0.5pt}\end{center}

\subsection*{Question 3(b) [4 marks]}\label{q3b}

\textbf{Explain the difference between get and post method in servlet.}

\begin{solutionbox}


{\def\LTcaptype{none} % do not increment counter
\vspace{-5pt}
\captionof{table}{GET vs POST Methods}
\vspace{-10pt}
\begin{longtable}[]{@{}lll@{}}
\toprule\noalign{}
Feature & GET & POST \\
\midrule\noalign{}
\endhead
\bottomrule\noalign{}
\endlastfoot
\textbf{Data Location} & URL query string & Request body \\
\textbf{Data Size} & Limited (2048 chars) & Unlimited \\
\textbf{Security} & Less secure (visible) & More secure \\
\textbf{Caching} & Can be cached & Not cached \\
\textbf{Bookmarking} & Can bookmark & Cannot bookmark \\
\textbf{Purpose} & Retrieve data & Submit/modify data \\
\end{longtable}
}

\textbf{Key Points:}

\begin{itemize}
\tightlist
\item
  \textbf{Visibility}: GET data visible in URL, POST hidden
\item
  \textbf{Capacity}: POST can handle large data
\item
  \textbf{Security}: POST better for sensitive data
\item
  \textbf{Usage}: GET for fetching, POST for form submission
\end{itemize}

\end{solutionbox}
\begin{mnemonicbox}
``GET Visible Limited, POST Hidden Unlimited''

\end{mnemonicbox}
\begin{center}\rule{0.5\linewidth}{0.5pt}\end{center}

\subsection*{Question 3(c) [7 marks]}\label{q3c}

\textbf{Develop a simple servlet program which maintains a counter for
the number of times it has been accessed since its loading; initialize
the counter using deployment descriptor.}

\begin{solutionbox}

\textbf{Servlet Code:}

\begin{verbatim}
import java.io.*;
import javax.servlet.*;
import javax.servlet.http.*;

public class CounterServlet extends HttpServlet \{
    private int counter;
    
    public void init() throws ServletException \{
        String initialValue = getInitParameter("initialCount");
        counter = Integer.parseInt(initialValue);
    \}
    
    protected void doGet(HttpServletRequest request, 
                        HttpServletResponse response) 
                        throws ServletException, IOException \{
        
        response.setContentType("text/html");
        PrintWriter out = response.getWriter();
        
        synchronized(this) \{
            counter++;
        \}
        
        out.println("{htmlbody"});
        out.println("{h2Page Access Counter/h2"});
        out.println("{pThis page has been accessed "} + counter + " times{/p"});
        out.println("{pa href=CounterServletRefresh/a/p"});
        out.println("{/body/html"});
        
        out.close();
    \}
\}
\end{verbatim}

\textbf{web.xml Configuration:}

\begin{verbatim}
{?xml} version="1.0" encoding="UTF{-8"}?{}
{}web{-app}{}
    {}servlet{}
        {}servlet{-name}{CounterServlet/}servlet{-name}{}
        {}servlet{-class}{CounterServlet/}servlet{-class}{}
        {}init{-param}{}
            {}param{-name}{initialCount/}param{-name}{}
            {}param{-value}{0/}param{-value}{}
        {/}init{-param}{}
        {}load{-on{-}startup}{1/}load{-on{-}startup}{}
    {/}servlet{}
    
    {}servlet{-mapping}{}
        {}servlet{-name}{CounterServlet/}servlet{-name}{}
        {}url{-pattern}{/counter/}url{-pattern}{}
    {/}servlet{-mapping}{}
{/}web{-app}{}
\end{verbatim}

\textbf{Key Features:}

\begin{itemize}
\tightlist
\item
  \textbf{Thread Safety}: Synchronized counter increment
\item
  \textbf{Initialization}: Counter initialized from web.xml
\item
  \textbf{Persistent}: Counter maintained across requests
\item
  \textbf{Configuration}: Deployment descriptor setup
\end{itemize}

\end{solutionbox}
\begin{mnemonicbox}
``Initialize Synchronize Count Display''

\end{mnemonicbox}
\begin{center}\rule{0.5\linewidth}{0.5pt}\end{center}

\subsection*{Question 3(a) OR [3
marks]}\label{q3a}

\textbf{Explain the life cycle of a servlet.}

\begin{solutionbox}

\textbf{Servlet Life Cycle Diagram:}

\begin{verbatim}
stateDiagram{-v2}
  direction LR
    [*] {-{-} Loading}
    Loading {-{-} init()}
    init() {-{-} service()}
    service() {-{-} service() : Multiple requests}
    service() {-{-} destroy()}
    destroy() {-{-} [*]}
\end{verbatim}


{\def\LTcaptype{none} % do not increment counter
\vspace{-5pt}
\captionof{table}{Servlet Life Cycle Methods}
\vspace{-10pt}
\begin{longtable}[]{@{}lll@{}}
\toprule\noalign{}
Method & Purpose & Called \\
\midrule\noalign{}
\endhead
\bottomrule\noalign{}
\endlastfoot
\textbf{init()} & Initialize servlet & Once at startup \\
\textbf{service()} & Handle requests & For each request \\
\textbf{destroy()} & Cleanup resources & Once at shutdown \\
\end{longtable}
}

\textbf{Key Points:}

\begin{itemize}
\tightlist
\item
  \textbf{Initialization}: Called once when servlet loads
\item
  \textbf{Service}: Handles all client requests
\item
  \textbf{Cleanup}: Called before servlet unloads
\item
  \textbf{Container Managed}: Web container controls lifecycle
\end{itemize}

\end{solutionbox}
\begin{mnemonicbox}
``Initialize Service Destroy''

\end{mnemonicbox}
\begin{center}\rule{0.5\linewidth}{0.5pt}\end{center}

\subsection*{Question 3(b) OR [4
marks]}\label{q3b}

\textbf{Explain Servlet Config class with suitable example.}

\begin{solutionbox}

ServletConfig provides servlet-specific configuration information and
initialization parameters.


{\def\LTcaptype{none} % do not increment counter
\vspace{-5pt}
\captionof{table}{ServletConfig Methods}
\vspace{-10pt}
\begin{longtable}[]{@{}ll@{}}
\toprule\noalign{}
Method & Purpose \\
\midrule\noalign{}
\endhead
\bottomrule\noalign{}
\endlastfoot
\textbf{getInitParameter()} & Gets init parameter value \\
\textbf{getInitParameterNames()} & Gets all parameter names \\
\textbf{getServletContext()} & Gets servlet context \\
\textbf{getServletName()} & Gets servlet name \\
\end{longtable}
}

\textbf{Example:}

\begin{verbatim}
public class ConfigServlet extends HttpServlet \{
    String databaseURL, username;
    
    public void init() throws ServletException \{
        ServletConfig config = getServletConfig();
        databaseURL = config.getInitParameter("dbURL");
        username = config.getInitParameter("dbUser");
    \}
    
    protected void doGet(HttpServletRequest request, 
                        HttpServletResponse response) 
                        throws ServletException, IOException \{
        
        PrintWriter out = response.getWriter();
        out.println("Database URL: " + databaseURL);
        out.println("Username: " + username);
    \}
\}
\end{verbatim}

\textbf{web.xml:}

\begin{verbatim}
{}servlet{}
    {}servlet{-name}{ConfigServlet/}servlet{-name}{}
    {}servlet{-class}{ConfigServlet/}servlet{-class}{}
    {}init{-param}{}
        {}param{-name}{dbURL/}param{-name}{}
        {}param{-value}{jdbc:mysql://localhost:3306/test/}param{-value}{}
    {/}init{-param}{}
    {}init{-param}{}
        {}param{-name}{dbUser/}param{-name}{}
        {}param{-value}{root/}param{-value}{}
    {/}init{-param}{}
{/}servlet{}
\end{verbatim}

\end{solutionbox}
\begin{mnemonicbox}
``Config Gets Parameters Context''

\end{mnemonicbox}
\begin{center}\rule{0.5\linewidth}{0.5pt}\end{center}

\subsection*{Question 3(c) OR [7
marks]}\label{q3c}

\textbf{Develop a simple program, when user select the subject code,
name of the subject will be displayed using servlet and mysql database.}

\begin{solutionbox}

\textbf{HTML Form (index.html):}

\begin{verbatim}
{!DOCTYPE} html{}
{}html{}
{}head{}
    {}title{}Subject Selection{/}title{}
{/}head{}
{}body{}
    {}h2{}Select Subject Code{/}h2{}
    {}form action="SubjectServlet" method="get"{}
        {}select name="subjectCode"{}
            {}option value=""{}Select Subject{/}option{}
            {}option value="4351603"{}4351603{/}option{}
            {}option value="4351604"{}4351604{/}option{}
            {}option value="4351605"{}4351605{/}option{}
        {/}select{}
        {}input type="submit" value="Get Subject Name"{}
    {/}form{}
{/}body{}
{/}html{}
\end{verbatim}

\textbf{Servlet Code:}

\begin{verbatim}
import java.io.*;
import java.sql.*;
import javax.servlet.*;
import javax.servlet.http.*;

public class SubjectServlet extends HttpServlet \{
    
    protected void doGet(HttpServletRequest request, 
                        HttpServletResponse response) 
                        throws ServletException, IOException \{
        
        response.setContentType("text/html");
        PrintWriter out = response.getWriter();
        
        String subjectCode = request.getParameter("subjectCode");
        String subjectName = "";
        
        if(subjectCode != null \&\& !subjectCode.equals("")) \{
            try \{
                Class.forName("com.mysql.cj.jdbc.Driver");
                Connection con = DriverManager.getConnection(
                    "jdbc:mysql://localhost:3306/university", "root", "password");
                
                PreparedStatement ps = con.prepareStatement(
                    "SELECT subject\_name FROM subjects WHERE subject\_code = ?");
                ps.setString(1, subjectCode);
                
                ResultSet rs = ps.executeQuery();
                if(rs.next()) \{
                    subjectName = rs.getString("subject\_name");
                \}
                
                con.close();
            \} catch(Exception e) \{
                subjectName = "Error: " + e.getMessage();
            \}
        \}
        
        out.println("{htmlbody"});
        out.println("{h2Subject Information/h2"});
        if(!subjectName.equals("")) \{
            out.println("{pSubject Code: "} + subjectCode + "{/p"});
            out.println("{pSubject Name: "} + subjectName + "{/p"});
        \} else \{
            out.println("{pPlease select a subject code/p"});
        \}
        out.println("{pa href=index.htmlBack/a/p"});
        out.println("{/body/html"});
    \}
\}
\end{verbatim}

\textbf{Database Table:}

\begin{verbatim}
CREATE TABLE subjects (
    subject\_code VARCHAR(10) PRIMARY KEY,
    subject\_name VARCHAR(100)
);

INSERT INTO subjects VALUES 
({4351603}, {Advanced Java Programming}),
({4351604}, {Web Technology}),
({4351605}, {Database Management System});
\end{verbatim}

\end{solutionbox}
\begin{mnemonicbox}
``Select Query Display Subject''

\end{mnemonicbox}
\begin{center}\rule{0.5\linewidth}{0.5pt}\end{center}

\subsection*{Question 4(a) [3 marks]}\label{q4a}

\textbf{Explain JSP life cycle.}

\begin{solutionbox}

\textbf{JSP Life Cycle Diagram:}

\begin{verbatim}
stateDiagram{-v2}
  direction LR
    [*] {-{-} Translation}
    Translation {-{-} Compilation}
    Compilation {-{-} Loading}
    Loading {-{-} jspInit()}
    jspInit() {-{-} \_jspService()}
    \_jspService() {-{-} \_jspService() : Multiple requests}
    \_jspService() {-{-} jspDestroy()}
    jspDestroy() {-{-} [*]}
\end{verbatim}


{\def\LTcaptype{none} % do not increment counter
\vspace{-5pt}
\captionof{table}{JSP Life Cycle Phases}
\vspace{-10pt}
\begin{longtable}[]{@{}ll@{}}
\toprule\noalign{}
Phase & Description \\
\midrule\noalign{}
\endhead
\bottomrule\noalign{}
\endlastfoot
\textbf{Translation} & JSP to Servlet conversion \\
\textbf{Compilation} & Servlet to bytecode \\
\textbf{Loading} & Load servlet class \\
\textbf{Initialization} & jspInit() called \\
\textbf{Request Processing} & \_jspService() handles requests \\
\textbf{Destruction} & jspDestroy() cleanup \\
\end{longtable}
}

\end{solutionbox}
\begin{mnemonicbox}
``Translate Compile Load Initialize Service Destroy''

\end{mnemonicbox}
\begin{center}\rule{0.5\linewidth}{0.5pt}\end{center}

\subsection*{Question 4(b) [4 marks]}\label{q4b}

\textbf{Compare JSP and Servlet.}

\begin{solutionbox}


{\def\LTcaptype{none} % do not increment counter
\vspace{-5pt}
\captionof{table}{JSP vs Servlet Comparison}
\vspace{-10pt}
\begin{longtable}[]{@{}
  >{\raggedright\arraybackslash}p{(\linewidth - 4\tabcolsep) * \real{0.3913}}
  >{\raggedright\arraybackslash}p{(\linewidth - 4\tabcolsep) * \real{0.2174}}
  >{\raggedright\arraybackslash}p{(\linewidth - 4\tabcolsep) * \real{0.3913}}@{}}
\toprule\noalign{}
\begin{minipage}[b]{\linewidth}\raggedright
Feature
\end{minipage} & \begin{minipage}[b]{\linewidth}\raggedright
JSP
\end{minipage} & \begin{minipage}[b]{\linewidth}\raggedright
Servlet
\end{minipage} \\
\midrule\noalign{}
\endhead
\bottomrule\noalign{}
\endlastfoot
\textbf{Code Type} & HTML with Java code & Pure Java code \\
\textbf{Development} & Easier for web designers & Better for Java
developers \\
\textbf{Compilation} & Automatic & Manual \\
\textbf{Modification} & No restart needed & Restart required \\
\textbf{Performance} & Slower first request & Faster \\
\textbf{Maintenance} & Easier & More complex \\
\end{longtable}
}

\textbf{Key Points:}

\begin{itemize}
\tightlist
\item
  \textbf{Ease of Use}: JSP easier for presentation layer
\item
  \textbf{Performance}: Servlet better for business logic
\item
  \textbf{Flexibility}: JSP better for dynamic content
\item
  \textbf{Control}: Servlet provides more control
\end{itemize}

\end{solutionbox}
\begin{mnemonicbox}
``JSP Easy HTML, Servlet Pure Java''

\end{mnemonicbox}
\begin{center}\rule{0.5\linewidth}{0.5pt}\end{center}

\subsection*{Question 4(c) [7 marks]}\label{q4c}

\textbf{Develop a JSP web application to display student monthly
attendance in each subject of current semester via enrolment number.}

\begin{solutionbox}

\textbf{Input Form (attendance.html):}

\begin{verbatim}
{!DOCTYPE} html{}
{}html{}
{}head{}
    {}title{}Student Attendance{/}title{}
{/}head{}
{}body{}
    {}h2{}Check Student Attendance{/}h2{}
    {}form action="attendanceCheck.jsp" method="post"{}
        {}table{}
            {}tr{}
                {}td{}Enrollment Number:{/}td{}
                {}td{}input type="text" name="enrollNo" required{/}td{}
            {/}tr{}
            {}tr{}
                {}td{}Month:{/}td{}
                {}td{}
                    {}select name="month" required{}
                        {}option value=""{}Select Month{/}option{}
                        {}option value="January"{}January{/}option{}
                        {}option value="February"{}February{/}option{}
                        {}option value="March"{}March{/}option{}
                    {/}select{}
                {/}td{}
            {/}tr{}
            {}tr{}
                {}td colspan="2"{}
                    {}input type="submit" value="Check Attendance"{}
                {/}td{}
            {/}tr{}
        {/}table{}
    {/}form{}
{/}body{}
{/}html{}
\end{verbatim}

\textbf{JSP Page (attendanceCheck.jsp):}

\begin{verbatim}
{\%@ page} import="java.sql.*" \%{}
{\%@ page} contentType="text/html;charset=UTF{-8"} \%{}

{html}
{head}
    {titleAttendance Report/title}
    {style}
        table \{ border{-collapse}: collapse; width: 100\%; \}
        th, td \{ border: 1px solid black; padding: 8px; text{-align}: center; \}
        th \{ background{-color}: \#f2f2f2; \}
    {/style}
{/head}
{body}
    {h2Monthly Attendance Report/h2}
    
    {\%}
        String enrollNo = request.getParameter("enrollNo");
        String month = request.getParameter("month");
        
        if(enrollNo != null \&\& month != null) \{
            try \{
                Class.forName("com.mysql.cj.jdbc.Driver");
                Connection con = DriverManager.getConnection(
                    "jdbc:mysql://localhost:3306/university", "root", "password");
                
                // Get student info
                PreparedStatement ps1 = con.prepareStatement(
                    "SELECT name FROM students WHERE enroll\_no = ?");
                ps1.setString(1, enrollNo);
                ResultSet rs1 = ps1.executeQuery();
                
                String studentName = "";
                if(rs1.next()) \{
                    studentName = rs1.getString("name");
                \}
                
                out.println("{pstrongStudent:/strong "} + studentName + 
                           " (" + enrollNo + "){/p"});
                out.println("{pstrongMonth:/strong "} + month + "{/p"});
                
                // Get attendance data
                PreparedStatement ps2 = con.prepareStatement(
                    "SELECT s.subject\_name, a.total\_classes, a.attended\_classes, " +
                    "ROUND((a.attended\_classes/a.total\_classes)*100, 2) as percentage " +
                    "FROM attendance a JOIN subjects s ON a.subject\_code = s.subject\_code " +
                    "WHERE a.enroll\_no = ? AND a.month = ?");
                ps2.setString(1, enrollNo);
                ps2.setString(2, month);
                ResultSet rs2 = ps2.executeQuery();
                
                out.println("{table"});
                out.println("{trthSubject/ththTotal Classes/th"} +
                           "{thAttended/ththPercentage/ththStatus/th/tr"});
                
                while(rs2.next()) \{
                    String subjectName = rs2.getString("subject\_name");
                    int totalClasses = rs2.getInt("total\_classes");
                    int attendedClasses = rs2.getInt("attended\_classes");
                    double percentage = rs2.getDouble("percentage");
                    String status = percentage {=} 75 ? "Good" : "Poor";
                    String rowColor = percentage {=} 75 ? "lightgreen" : "lightcoral";
                    
                    out.println("{tr style=background{-}color:"} + rowColor + "{"});
                    out.println("{td"} + subjectName + "{/td"});
                    out.println("{td"} + totalClasses + "{/td"});
                    out.println("{td"} + attendedClasses + "{/td"});
                    out.println("{td"} + percentage + "\%{/td"});
                    out.println("{td"} + status + "{/td"});
                    out.println("{/tr"});
                \}
                
                out.println("{/table"});
                con.close();
                
            \} catch(Exception e) \{
                out.println("{p style=color:redError: "} + e.getMessage() + "{/p"});
            \}
        \}
    \%{}
    
    {br} /{}
    {a} href="attendance.html"{Check Another Student/a}
{/body}
{/html}
\end{verbatim}

\textbf{Database Tables:}

\begin{verbatim}
CREATE TABLE students (
    enroll\_no VARCHAR(20) PRIMARY KEY,
    name VARCHAR(50)
);

CREATE TABLE subjects (
    subject\_code VARCHAR(10) PRIMARY KEY,
    subject\_name VARCHAR(100)
);

CREATE TABLE attendance (
    id INT AUTO\_INCREMENT PRIMARY KEY,
    enroll\_no VARCHAR(20),
    subject\_code VARCHAR(10),
    month VARCHAR(15),
    total\_classes INT,
    attended\_classes INT,
    FOREIGN KEY (enroll\_no) REFERENCES students(enroll\_no),
    FOREIGN KEY (subject\_code) REFERENCES subjects(subject\_code)
);
\end{verbatim}

\end{solutionbox}
\begin{mnemonicbox}
``JSP Database Query Display Table''

\end{mnemonicbox}
\begin{center}\rule{0.5\linewidth}{0.5pt}\end{center}

\subsection*{Question 4(a) OR [3
marks]}\label{q4a}

\textbf{Explain implicit objects in JSP.}

\begin{solutionbox}


{\def\LTcaptype{none} % do not increment counter
\vspace{-5pt}
\captionof{table}{JSP Implicit Objects}
\vspace{-10pt}
\begin{longtable}[]{@{}lll@{}}
\toprule\noalign{}
Object & Type & Purpose \\
\midrule\noalign{}
\endhead
\bottomrule\noalign{}
\endlastfoot
\textbf{request} & HttpServletRequest & Gets request data \\
\textbf{response} & HttpServletResponse & Sends response \\
\textbf{out} & JspWriter & Output to client \\
\textbf{session} & HttpSession & Session management \\
\textbf{application} & ServletContext & Application scope \\
\textbf{config} & ServletConfig & Servlet configuration \\
\textbf{pageContext} & PageContext & Page scope access \\
\textbf{page} & Object & Current servlet instance \\
\textbf{exception} & Throwable & Error page exception \\
\end{longtable}
}

\textbf{Key Features:}

\begin{itemize}
\tightlist
\item
  \textbf{Automatic}: Available without declaration
\item
  \textbf{Scope Access}: Different scope levels
\item
  \textbf{Request Handling}: Input/output operations
\item
  \textbf{Session Management}: User session tracking
\end{itemize}

\end{solutionbox}
\begin{mnemonicbox}
``Request Response Out Session Application''

\end{mnemonicbox}
\begin{center}\rule{0.5\linewidth}{0.5pt}\end{center}

\subsection*{Question 4(b) OR [4
marks]}\label{q4b}

\textbf{Explain why JSP is preferred over servlet.}

\begin{solutionbox}


{\def\LTcaptype{none} % do not increment counter
\vspace{-5pt}
\captionof{table}{JSP Advantages over Servlet}
\vspace{-10pt}
\begin{longtable}[]{@{}ll@{}}
\toprule\noalign{}
Aspect & JSP Advantage \\
\midrule\noalign{}
\endhead
\bottomrule\noalign{}
\endlastfoot
\textbf{Development} & Easier HTML integration \\
\textbf{Maintenance} & Separates presentation from logic \\
\textbf{Compilation} & Automatic compilation \\
\textbf{Modification} & No server restart needed \\
\textbf{Design} & Web designer friendly \\
\textbf{Code Reuse} & Tag libraries and custom tags \\
\end{longtable}
}

\textbf{Key Points:}

\begin{itemize}
\tightlist
\item
  \textbf{Separation of Concerns}: Clear separation of presentation and
  business logic
\item
  \textbf{Rapid Development}: Faster development cycle
\item
  \textbf{Designer Friendly}: Web designers can work with HTML-like
  syntax
\item
  \textbf{Automatic Features}: Container handles compilation and
  lifecycle
\end{itemize}

\end{solutionbox}
\begin{mnemonicbox}
``Easy HTML Automatic Designer Friendly''

\end{mnemonicbox}
\begin{center}\rule{0.5\linewidth}{0.5pt}\end{center}

\subsection*{Question 4(c) OR [7
marks]}\label{q4c}

\textbf{Develop a JSP program to display the grade of a student by
accepting the marks of five subjects.}

\begin{solutionbox}

\textbf{Input Form (gradeInput.html):}

\begin{verbatim}
{!DOCTYPE} html{}
{}html{}
{}head{}
    {}title{}Student Grade Calculator{/}title{}
    {}style{}
        table \{ margin: auto; border{-collapse}: collapse; \}
        td \{ padding: 10px; \}
        input[type="number"] \{ width: 100px; \}
        input[type="submit"] \{ padding: 10px 20px; \}
    {/}style{}
{/}head{}
{}body{}
    {}h2 style="text{-align: center;"}{}Student Grade Calculator{/}h2{}
    {}form action="gradeCalculator.jsp" method="post"{}
        {}table border="1"{}
            {}tr{}
                {}td{}Student Name:{/}td{}
                {}td{}input type="text" name="studentName" required{/}td{}
            {/}tr{}
            {}tr{}
                {}td{}Subject 1 Marks:{/}td{}
                {}td{}input type="number" name="marks1" min="0" max="100" required{/}td{}
            {/}tr{}
            {}tr{}
                {}td{}Subject 2 Marks:{/}td{}
                {}td{}input type="number" name="marks2" min="0" max="100" required{/}td{}
            {/}tr{}
            {}tr{}
                {}td{}Subject 3 Marks:{/}td{}
                {}td{}input type="number" name="marks3" min="0" max="100" required{/}td{}
            {/}tr{}
            {}tr{}
                {}td{}Subject 4 Marks:{/}td{}
                {}td{}input type="number" name="marks4" min="0" max="100" required{/}td{}
            {/}tr{}
            {}tr{}
                {}td{}Subject 5 Marks:{/}td{}
                {}td{}input type="number" name="marks5" min="0" max="100" required{/}td{}
            {/}tr{}
            {}tr{}
                {}td colspan="2" style="text{-align: center;"}{}
                    {}input type="submit" value="Calculate Grade"{}
                {/}td{}
            {/}tr{}
        {/}table{}
    {/}form{}
{/}body{}
{/}html{}
\end{verbatim}

\textbf{JSP Grade Calculator (gradeCalculator.jsp):}

\begin{verbatim}
{\%@ page} contentType="text/html;charset=UTF{-8"} \%{}

{html}
{head}
    {titleGrade Result/title}
    {style}
        .result{-table} \{ 
            margin: auto; 
            border{-collapse}: collapse; 
            margin{-top}: 20px;
        \}
        .result{-table} th, .result{-table} td \{ 
            border: 1px solid black; 
            padding: 10px; 
            text{-align}: center; 
        \}
        .result{-table} th \{ background{-color}: \#f2f2f2; \}
        .grade{-A} \{ background{-color}: \#90EE90; \}
        .grade{-B} \{ background{-color}: \#87CEEB; \}
        .grade{-C} \{ background{-color}: \#F0E68C; \}
        .grade{-D} \{ background{-color}: \#FFA07A; \}
        .grade{-F} \{ background{-color}: \#FFB6C1; \}
    {/style}
{/head}
{body}
    {h2} style="text{-align: center;"}{Grade Report/h2}
    
    {\%}
        String studentName = request.getParameter("studentName");
        
        // Get marks
        int marks1 = Integer.parseInt(request.getParameter("marks1"));
        int marks2 = Integer.parseInt(request.getParameter("marks2"));
        int marks3 = Integer.parseInt(request.getParameter("marks3"));
        int marks4 = Integer.parseInt(request.getParameter("marks4"));
        int marks5 = Integer.parseInt(request.getParameter("marks5"));
        
        // Calculate total and percentage
        int totalMarks = marks1 + marks2 + marks3 + marks4 + marks5;
        double percentage = totalMarks / 5.0;
        
        // Determine grade
        String grade;
        String gradeClass;
        if(percentage {=} 90) \{
            grade = "A+";
            gradeClass = "grade{-A"};
        \} else if(percentage {=} 80) \{
            grade = "A";
            gradeClass = "grade{-A"};
        \} else if(percentage {=} 70) \{
            grade = "B";
            gradeClass = "grade{-B"};
        \} else if(percentage {=} 60) \{
            grade = "C";
            gradeClass = "grade{-C"};
        \} else if(percentage {=} 50) \{
            grade = "D";
            gradeClass = "grade{-D"};
        \} else \{
            grade = "F";
            gradeClass = "grade{-F"};
        \}
        
        // Determine result
        String result = percentage {=} 50 ? "PASS" : "FAIL";
    \%{}
    
    {table} class="result{-table"}{}
        {tr}
            {th} colspan="2"{Student Information/th}
        {/tr}
        {tr}
            {tdstrongName:/strong/td}
            {td}{\%=} studentName \%{}{/td}
        {/tr}
        {tr}
            {th} colspan="2"{Subject{-}wise Marks/th}
        {/tr}
        {tr}
            {tdSubject 1/td}
            {td}{\%=} marks1 \%{}{/td}
        {/tr}
        {tr}
            {tdSubject 2/td}
            {td}{\%=} marks2 \%{}{/td}
        {/tr}
        {tr}
            {tdSubject 3/td}
            {td}{\%=} marks3 \%{}{/td}
        {/tr}
        {tr}
            {tdSubject 4/td}
            {td}{\%=} marks4 \%{}{/td}
        {/tr}
        {tr}
            {tdSubject 5/td}
            {td}{\%=} marks5 \%{}{/td}
        {/tr}
        {tr}
            {th} colspan="2"{Result Summary/th}
        {/tr}
        {tr}
            {tdstrongTotal Marks:/strong/td}
            {td}{\%=} totalMarks \%{} / 500{/td}
        {/tr}
        {tr}
            {tdstrongPercentage:/strong/td}
            {td}{\%=} String.format("\%.2f", percentage) \%{}\%{/td}
        {/tr}
        {tr} class="{\%=} gradeClass \%{}"{}
            {tdstrongGrade:/strong/td}
            {td}{\%=} grade \%{}{/td}
        {/tr}
        {tr}
            {tdstrongResult:/strong/td}
            {td}{\%=} result \%{}{/td}
        {/tr}
    {/table}
    
    {div} style="text{-align: center; margin{-}top: 20px;"}{}
        {a} href="gradeInput.html"{Calculate Another Grade/a}
    {/div}
{/body}
{/html}
\end{verbatim}

\textbf{Grade Scale Table:}

{\def\LTcaptype{none} % do not increment counter
\begin{longtable}[]{@{}lll@{}}
\toprule\noalign{}
Percentage & Grade & Description \\
\midrule\noalign{}
\endhead
\bottomrule\noalign{}
\endlastfoot
\textbf{90-100} & A+ & Excellent \\
\textbf{80-89} & A & Very Good \\
\textbf{70-79} & B & Good \\
\textbf{60-69} & C & Average \\
\textbf{50-59} & D & Below Average \\
\textbf{0-49} & F & Fail \\
\end{longtable}
}

\end{solutionbox}
\begin{mnemonicbox}
``Calculate Total Percentage Grade Result''

\end{mnemonicbox}
\begin{center}\rule{0.5\linewidth}{0.5pt}\end{center}

\subsection*{Question 5(a) [3 marks]}\label{q5a}

\textbf{Explain Aspect-oriented programming (AOP).}

\begin{solutionbox}

AOP is programming paradigm that separates cross-cutting concerns from
business logic using aspects.


{\def\LTcaptype{none} % do not increment counter
\vspace{-5pt}
\captionof{table}{AOP Core Concepts}
\vspace{-10pt}
\begin{longtable}[]{@{}ll@{}}
\toprule\noalign{}
Concept & Description \\
\midrule\noalign{}
\endhead
\bottomrule\noalign{}
\endlastfoot
\textbf{Aspect} & Module encapsulating cross-cutting concern \\
\textbf{Join Point} & Point in program execution \\
\textbf{Pointcut} & Set of join points \\
\textbf{Advice} & Action taken at join point \\
\textbf{Weaving} & Process of applying aspects \\
\end{longtable}
}

\textbf{Key Benefits:}

\begin{itemize}
\tightlist
\item
  \textbf{Separation}: Separates business logic from system services
\item
  \textbf{Modularity}: Improves code modularity
\item
  \textbf{Reusability}: Cross-cutting concerns are reusable
\item
  \textbf{Maintenance}: Easier to maintain and modify
\end{itemize}

\end{solutionbox}
\begin{mnemonicbox}
``Aspect Join Pointcut Advice Weaving''

\end{mnemonicbox}
\begin{center}\rule{0.5\linewidth}{0.5pt}\end{center}

\subsection*{Question 5(b) [4 marks]}\label{q5b}

\textbf{List various features of Servlet.}

\begin{solutionbox}


{\def\LTcaptype{none} % do not increment counter
\vspace{-5pt}
\captionof{table}{Servlet Features}
\vspace{-10pt}
\begin{longtable}[]{@{}ll@{}}
\toprule\noalign{}
Feature & Description \\
\midrule\noalign{}
\endhead
\bottomrule\noalign{}
\endlastfoot
\textbf{Platform Independent} & Runs on any server supporting Java \\
\textbf{Server Independent} & Works with different web servers \\
\textbf{Protocol Independent} & Supports HTTP, HTTPS, FTP \\
\textbf{Persistent} & Remains in memory between requests \\
\textbf{Robust} & Strong memory management \\
\textbf{Secure} & Built-in security features \\
\textbf{Portable} & Write once, run anywhere \\
\textbf{Powerful} & Full Java API access \\
\end{longtable}
}

\textbf{Key Points:}

\begin{itemize}
\tightlist
\item
  \textbf{Performance}: Better performance than CGI
\item
  \textbf{Memory Management}: Efficient memory usage
\item
  \textbf{Multithreading}: Handles multiple requests simultaneously
\item
  \textbf{Extensible}: Can be extended for specific protocols
\end{itemize}

\end{solutionbox}
\begin{mnemonicbox}
``Platform Server Protocol Persistent Robust''

\end{mnemonicbox}
\begin{center}\rule{0.5\linewidth}{0.5pt}\end{center}

\subsection*{Question 5(c) [7 marks]}\label{q5c}

\textbf{Explain Model layer, View layer and Controller layer in
details.}

\begin{solutionbox}

\textbf{MVC Architecture Diagram:}

\begin{verbatim}
graph TB
    U[User] {-{-} C[Controller]}
    C {-{-} M[Model]}
    M {-{-} C}
    C {-{-} V[View]}
    V {-{-} U}
    
    subgraph "MVC Layers"
        M
        V
        C
    end
\end{verbatim}


{\def\LTcaptype{none} % do not increment counter
\vspace{-5pt}
\captionof{table}{MVC Layer Details}
\vspace{-10pt}
\begin{longtable}[]{@{}
  >{\raggedright\arraybackslash}p{(\linewidth - 6\tabcolsep) * \real{0.1628}}
  >{\raggedright\arraybackslash}p{(\linewidth - 6\tabcolsep) * \real{0.3488}}
  >{\raggedright\arraybackslash}p{(\linewidth - 6\tabcolsep) * \real{0.2791}}
  >{\raggedright\arraybackslash}p{(\linewidth - 6\tabcolsep) * \real{0.2093}}@{}}
\toprule\noalign{}
\begin{minipage}[b]{\linewidth}\raggedright
Layer
\end{minipage} & \begin{minipage}[b]{\linewidth}\raggedright
Responsibility
\end{minipage} & \begin{minipage}[b]{\linewidth}\raggedright
Components
\end{minipage} & \begin{minipage}[b]{\linewidth}\raggedright
Purpose
\end{minipage} \\
\midrule\noalign{}
\endhead
\bottomrule\noalign{}
\endlastfoot
\textbf{Model} & Data and business logic & Entities, DAOs, Services &
Data management \\
\textbf{View} & Presentation layer & JSP, HTML, CSS & User interface \\
\textbf{Controller} & Request handling & Servlets, Actions & Flow
control \\
\end{longtable}
}

\textbf{Model Layer Details:}

\begin{itemize}
\tightlist
\item
  \textbf{Data Access}: Database operations and data persistence
\item
  \textbf{Business Logic}: Core application logic and rules
\item
  \textbf{Validation}: Data validation and integrity checks
\item
  \textbf{Entity Classes}: Java beans representing data structures
\end{itemize}

\textbf{Example Model:}

\begin{verbatim}
public class Student \{
    private String enrollNo;
    private String name;
    private double marks;
    
    // Business logic
    public String calculateGrade() \{
        if(marks {=} 90) return "A";
        else if(marks {=} 80) return "B";
        else if(marks {=} 70) return "C";
        else return "D";
    \}
\}
\end{verbatim}

\textbf{View Layer Details:}

\begin{itemize}
\tightlist
\item
  \textbf{Presentation}: User interface rendering
\item
  \textbf{Display Logic}: How data is presented to user
\item
  \textbf{User Interaction}: Forms, buttons, navigation
\item
  \textbf{Responsive Design}: Adapts to different devices
\end{itemize}

\textbf{Controller Layer Details:}

\begin{itemize}
\tightlist
\item
  \textbf{Request Handling}: Processes user requests
\item
  \textbf{Flow Control}: Determines next view to display
\item
  \textbf{Model Coordination}: Calls appropriate model methods
\item
  \textbf{Response Generation}: Prepares response for user
\end{itemize}

\textbf{Example Controller:}

\begin{verbatim}
@WebServlet("/student")
public class StudentController extends HttpServlet \{
    protected void doGet(HttpServletRequest request, 
                        HttpServletResponse response) \{
        String action = request.getParameter("action");
        
        if("view".equals(action)) \{
            // Get data from model
            Student student = studentService.getStudent(enrollNo);
            // Set in request scope
            request.setAttribute("student", student);
            // Forward to view
            RequestDispatcher rd = request.getRequestDispatcher("student.jsp");
            rd.forward(request, response);
        \}
    \}
\}
\end{verbatim}

\textbf{Benefits of MVC:}

\begin{itemize}
\tightlist
\item
  \textbf{Separation of Concerns}: Clear responsibility division
\item
  \textbf{Maintainability}: Easier to maintain and modify
\item
  \textbf{Testability}: Each layer can be tested independently
\item
  \textbf{Scalability}: Supports large application development
\item
  \textbf{Team Development}: Multiple developers can work simultaneously
\end{itemize}

\end{solutionbox}
\begin{mnemonicbox}
``Model Data View Present Controller Handle''

\end{mnemonicbox}
\begin{center}\rule{0.5\linewidth}{0.5pt}\end{center}

\subsection*{Question 5(a) OR [3
marks]}\label{q5a}

\textbf{Explain Features in Spring Boot.}

\begin{solutionbox}


{\def\LTcaptype{none} % do not increment counter
\vspace{-5pt}
\captionof{table}{Spring Boot Features}
\vspace{-10pt}
\begin{longtable}[]{@{}
  >{\raggedright\arraybackslash}p{(\linewidth - 2\tabcolsep) * \real{0.4091}}
  >{\raggedright\arraybackslash}p{(\linewidth - 2\tabcolsep) * \real{0.5909}}@{}}
\toprule\noalign{}
\begin{minipage}[b]{\linewidth}\raggedright
Feature
\end{minipage} & \begin{minipage}[b]{\linewidth}\raggedright
Description
\end{minipage} \\
\midrule\noalign{}
\endhead
\bottomrule\noalign{}
\endlastfoot
\textbf{Auto Configuration} & Automatic configuration based on
dependencies \\
\textbf{Starter Dependencies} & Curated set of dependencies \\
\textbf{Embedded Servers} & Built-in Tomcat, Jetty servers \\
\textbf{Production Ready} & Health checks, metrics, monitoring \\
\textbf{No XML Configuration} & Annotation-based configuration \\
\textbf{Developer Tools} & Hot reloading, automatic restart \\
\end{longtable}
}

\textbf{Key Benefits:}

\begin{itemize}
\tightlist
\item
  \textbf{Rapid Development}: Quick project setup and development
\item
  \textbf{Convention over Configuration}: Sensible defaults
\item
  \textbf{Microservices Ready}: Easy microservices development
\item
  \textbf{Cloud Native}: Ready for cloud deployment
\end{itemize}

\end{solutionbox}
\begin{mnemonicbox}
``Auto Starter Embedded Production Annotation
Developer''

\end{mnemonicbox}
\begin{center}\rule{0.5\linewidth}{0.5pt}\end{center}

\subsection*{Question 5(b) OR [4
marks]}\label{q5b}

\textbf{Write Short note on JSP scripting elements.}

\begin{solutionbox}


{\def\LTcaptype{none} % do not increment counter
\vspace{-5pt}
\captionof{table}{JSP Scripting Elements}
\vspace{-10pt}
\begin{longtable}[]{@{}
  >{\raggedright\arraybackslash}p{(\linewidth - 6\tabcolsep) * \real{0.2571}}
  >{\raggedright\arraybackslash}p{(\linewidth - 6\tabcolsep) * \real{0.2286}}
  >{\raggedright\arraybackslash}p{(\linewidth - 6\tabcolsep) * \real{0.2571}}
  >{\raggedright\arraybackslash}p{(\linewidth - 6\tabcolsep) * \real{0.2571}}@{}}
\toprule\noalign{}
\begin{minipage}[b]{\linewidth}\raggedright
Element
\end{minipage} & \begin{minipage}[b]{\linewidth}\raggedright
Syntax
\end{minipage} & \begin{minipage}[b]{\linewidth}\raggedright
Purpose
\end{minipage} & \begin{minipage}[b]{\linewidth}\raggedright
Example
\end{minipage} \\
\midrule\noalign{}
\endhead
\bottomrule\noalign{}
\endlastfoot
\textbf{Scriptlet} & \texttt{\textless{}\%\ \%\textgreater{}} & Java
code execution &
\texttt{\textless{}\%\ int\ x\ =\ 10;\ \%\textgreater{}} \\
\textbf{Expression} & \texttt{\textless{}\%=\ \%\textgreater{}} & Output
value & \texttt{\textless{}\%=\ x\ +\ 5\ \%\textgreater{}} \\
\textbf{Declaration} & \texttt{\textless{}\%!\ \%\textgreater{}} &
Variable/method declaration &
\texttt{\textless{}\%!\ int\ count\ =\ 0;\ \%\textgreater{}} \\
\textbf{Directive} & \texttt{\textless{}\%@\ \%\textgreater{}} & Page
configuration &
\texttt{\textless{}\%@\ page\ import="java.util.*"\ \%\textgreater{}} \\
\textbf{Comment} & \texttt{\textless{}\%-\/-\ -\/-\%\textgreater{}} &
JSP comments &
\texttt{\textless{}\%-\/-\ This\ is\ comment\ -\/-\%\textgreater{}} \\
\end{longtable}
}

\textbf{Examples:}

\begin{verbatim}
{\%{-}{-} JSP Comment {-}{-}\%}
{\%@ page} contentType="text/html" \%{}

{\%!} 
    // Declaration {- instance variable}
    private int counter = 0;
    
    // Declaration {- method}
    public String getMessage() \{
        return "Hello JSP!";
    \}
\%{}

{html}
{body}
    {\%} 
        // Scriptlet {- Java code}
        String name = "Student";
        counter++;
    \%{}
    
    {h1}{\%=} getMessage() \%{}{/h1}
    {pWelcome }{\%=} name \%{}!{/p}
    {pPage visited }{\%=} counter \%{} times{/p}
{/body}
{/html}
\end{verbatim}

\textbf{Key Points:}

\begin{itemize}
\tightlist
\item
  \textbf{Scriptlet}: Contains Java statements
\item
  \textbf{Expression}: Evaluates and outputs result
\item
  \textbf{Declaration}: Creates instance variables/methods
\item
  \textbf{Directive}: Provides page-level information
\end{itemize}

\end{solutionbox}
\begin{mnemonicbox}
``Script Express Declare Direct Comment''

\end{mnemonicbox}
\begin{center}\rule{0.5\linewidth}{0.5pt}\end{center}

\subsection*{Question 5(c) OR [7
marks]}\label{q5c}

\textbf{Explain Dependency injection (DI) and Plain Old Java Object
(POJO) in details.}

\begin{solutionbox}

\textbf{Dependency Injection (DI):}

Dependency Injection is design pattern where objects receive their
dependencies from external source rather than creating them internally.


{\def\LTcaptype{none} % do not increment counter
\vspace{-5pt}
\captionof{table}{DI Types}
\vspace{-10pt}
\begin{longtable}[]{@{}
  >{\raggedright\arraybackslash}p{(\linewidth - 4\tabcolsep) * \real{0.2143}}
  >{\raggedright\arraybackslash}p{(\linewidth - 4\tabcolsep) * \real{0.4643}}
  >{\raggedright\arraybackslash}p{(\linewidth - 4\tabcolsep) * \real{0.3214}}@{}}
\toprule\noalign{}
\begin{minipage}[b]{\linewidth}\raggedright
Type
\end{minipage} & \begin{minipage}[b]{\linewidth}\raggedright
Description
\end{minipage} & \begin{minipage}[b]{\linewidth}\raggedright
Example
\end{minipage} \\
\midrule\noalign{}
\endhead
\bottomrule\noalign{}
\endlastfoot
\textbf{Constructor Injection} & Dependencies via constructor &
\texttt{public\ Service(Repository\ repo)} \\
\textbf{Setter Injection} & Dependencies via setter methods &
\texttt{setRepository(Repository\ repo)} \\
\textbf{Field Injection} & Direct field injection &
\texttt{@Autowired\ Repository\ repo} \\
\end{longtable}
}

\textbf{DI Example:}

\begin{verbatim}
// Without DI {- Tight coupling}
public class StudentService \{
    private StudentRepository repo = new StudentRepository(); // Hard dependency
    
    public Student getStudent(String id) \{
        return repo.findById(id);
    \}
\}

// With DI {- Loose coupling}
public class StudentService \{
    private StudentRepository repo;
    
    // Constructor injection
    public StudentService(StudentRepository repo) \{
        this.repo = repo;
    \}
    
    public Student getStudent(String id) \{
        return repo.findById(id);
    \}
\}
\end{verbatim}

\textbf{Spring DI Configuration:}

\begin{verbatim}
@Service
public class StudentService \{
    @Autowired
    private StudentRepository repository;
    
    public List{}Student{} getAllStudents() \{
        return repository.findAll();
    \}
\}

@Repository
public class StudentRepository \{
    public List{}Student{} findAll() \{
        // Database operations
        return studentList;
    \}
\}
\end{verbatim}

\textbf{Plain Old Java Object (POJO):}

POJO is simple Java object that doesn't inherit from any specific
framework classes or implement specific interfaces.

\textbf{POJO Characteristics:}

\begin{itemize}
\tightlist
\item
  \textbf{No inheritance}: Doesn't extend framework classes
\item
  \textbf{No interfaces}: Doesn't implement framework interfaces
\item
  \textbf{No annotations}: Can work without framework annotations
\item
  \textbf{Simple}: Contains only business logic and data
\end{itemize}

\textbf{POJO Example:}

\begin{verbatim}
// This is a POJO
public class Student \{
    private String enrollNo;
    private String name;
    private int age;
    private String course;
    
    // Default constructor
    public Student() \{\}
    
    // Parameterized constructor
    public Student(String enrollNo, String name, int age, String course) \{
        this.enrollNo = enrollNo;
        this.name = name;
        this.age = age;
        this.course = course;
    \}
    
    // Getters and Setters
    public String getEnrollNo() \{
        return enrollNo;
    \}
    
    public void setEnrollNo(String enrollNo) \{
        this.enrollNo = enrollNo;
    \}
    
    public String getName() \{
        return name;
    \}
    
    public void setName(String name) \{
        this.name = name;
    \}
    
    // Business methods
    public boolean isEligibleForExam() \{
        return age {=} 18;
    \}
    
    public String getStudentInfo() \{
        return "Student: " + name + " (" + enrollNo + "), Course: " + course;
    \}
\}
\end{verbatim}

\textbf{Benefits of DI:}

\begin{itemize}
\tightlist
\item
  \textbf{Loose Coupling}: Reduces dependencies between classes
\item
  \textbf{Testability}: Easy to inject mock objects for testing
\item
  \textbf{Flexibility}: Easy to change implementations
\item
  \textbf{Maintainability}: Easier to maintain and extend code
\end{itemize}

\textbf{Benefits of POJO:}

\begin{itemize}
\tightlist
\item
  \textbf{Simplicity}: Easy to understand and maintain
\item
  \textbf{Testability}: Simple to unit test
\item
  \textbf{Portability}: Can be used across different frameworks
\item
  \textbf{Lightweight}: No framework overhead
\end{itemize}

\textbf{DI and POJO Together:}

\begin{verbatim}
// POJO Entity
public class Student \{
    private String name;
    private String email;
    // constructors, getters, setters
\}

// Service with DI
@Service
public class StudentService \{
    @Autowired
    private StudentRepository repository;
    
    public Student createStudent(String name, String email) \{
        Student student = new Student(); // POJO creation
        student.setName(name);
        student.setEmail(email);
        return repository.save(student);
    \}
\}
\end{verbatim}

\end{solutionbox}
\begin{mnemonicbox}
``DI Injects Dependencies, POJO Plain Objects''

\end{mnemonicbox}

\end{document}
