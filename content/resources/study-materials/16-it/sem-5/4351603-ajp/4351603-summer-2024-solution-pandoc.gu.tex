\documentclass[10pt,a4paper]{article}

% content/resources/templates/preamble.tex
\usepackage[margin=0.6in]{geometry}
\author{Milav Dabgar}
\usepackage{amsmath,amssymb,amsthm}
\usepackage{booktabs}
\usepackage{multirow}
\usepackage{xcolor}
\usepackage{tcolorbox}
\tcbuselibrary{breakable,skins}
\usepackage[colorlinks=true,linkcolor=blue]{hyperref}
\usepackage{titlesec}
\usepackage{enumitem}
\usepackage{tikz}
\usepackage{pgfplots}
\usepackage{circuitikz}
\usepackage[version=4]{mhchem}
\usepackage{longtable}
\usepackage{array}
\usepackage{float}
\usepackage{caption}
\usepackage{listings}

\lstset{
  basicstyle=\small\ttfamily,
  breaklines=true,
  breakatwhitespace=false,
  postbreak=\mbox{\textcolor{red}{$\hookrightarrow$}\space},
  float=false,
  numbers=left,
  numberstyle=\tiny\color{gray},
  numbersep=10pt,
  xleftmargin=2em,
  keywordstyle=\color{blue},
  commentstyle=\color{green!60!black},
  stringstyle=\color{purple},
  backgroundcolor=\color{gray!5},
  showstringspaces=false,
  tabsize=2,
  captionpos=b,
  keepspaces=true,
  columns=flexible
}

\pgfplotsset{compat=1.18}
\usetikzlibrary{shapes,arrows,positioning,calc,patterns,decorations.pathmorphing,decorations.markings,arrows.meta}

% Color scheme
\definecolor{headcolor}{RGB}{0,102,204}
\definecolor{keycolor}{RGB}{220,20,60}
\definecolor{solutioncolor}{RGB}{34,139,34}
\definecolor{mnemoniccolor}{RGB}{148,0,211}
\definecolor{codecolor}{RGB}{0,0,100}

% Spacing
\setlength{\parskip}{3pt}
\setlist[itemize]{nosep}
\setlist[enumerate]{nosep}

% Title formatting
\titleformat{\section}{\Large\bfseries\color{headcolor}}{\thesection}{1em}{}
\titleformat{\subsection}{\large\bfseries\color{headcolor}}{\thesubsection}{1em}{}

% Pandoc tightlist compatibility
\providecommand{\tightlist}{%
  \setlength{\itemsep}{0pt}\setlength{\parskip}{0pt}}

% Pandoc longtable compatibility
\newcounter{none}
\def\thenone{}


% content/resources/templates/gujarati-boxes.tex
\usepackage{fontspec}
\usepackage{polyglossia}

% Set Gujarati as main language (document is primarily in Gujarati)
% Note: gloss-gujarati.ldf doesn't exist in polyglossia, but it will use hyphenation patterns
\setdefaultlanguage{gujarati}
\setotherlanguage{english}

% Configure Gujarati font properly
% Use Language=Default to prevent polyglossia from trying to add language-specific features
% that don't exist for Gujarati, which causes "empty feature" warnings
\newfontfamily\gujaratifont[Script=Gujarati,AutoFakeBold=2.5,AutoFakeSlant=0.3]{Noto Sans Gujarati}
\setmainfont[Script=Gujarati,AutoFakeBold=2.5,AutoFakeSlant=0.3]{Noto Sans Gujarati}
% Use Noto Sans Gujarati for monospace to support Gujarati in text
\setmonofont[Scale=0.9]{Noto Sans Gujarati}

% Configure English to use the same font
\newfontfamily\englishfont[Script=Gujarati,AutoFakeBold=2.5,AutoFakeSlant=0.3]{Noto Sans Gujarati}

% Translations for polyglossia
\gappto\captionsgujarati{
  \renewcommand{\tablename}{કોષ્ટક}
  \renewcommand{\figurename}{આકૃતિ}
}

% Helper for TikZ nodes to ensure Gujarati font
\newcommand{\gu}[1]{{\gujaratifont #1}}

% Custom environments
\newtcolorbox{solutionbox}{
    breakable,
    enhanced,
    colback=solutioncolor!5!white,
    colframe=solutioncolor!75!black,
    fonttitle=\bfseries,
    title=જવાબ
}

\newtcolorbox{solutionboxnobreak}{
 colback=solutioncolor!5!white,
 colframe=solutioncolor!75!black,
 fonttitle=\bfseries,
 title=જવાબ
}

\newtcolorbox{keyformula}{
 breakable,
 enhanced,
 colback=keycolor!5!white,
 colframe=keycolor!75!black,
 fonttitle=\bfseries,
 title=રાસાયણિક સમીકરણ/સૂત્ર
}

\newtcolorbox{mnemonicbox}{
 breakable,
 enhanced,
 colback=mnemoniccolor!5!white,
 colframe=mnemoniccolor!75!black,
 fonttitle=\bfseries,
 title=મેમરી ટ્રીક
}


\begin{document}

\begin{center}
{\Huge\bfseries\color{headcolor} Subject Name (Gujarati)}\\[5pt]
{\LARGE 4351603 -- Summer 2024}\\[3pt]
{\large Semester 1 Study Material}\\[3pt]
{\normalsize\textit{Detailed Solutions and Explanations}}
\end{center}

\vspace{10pt}

\subsection*{પ્રશ્ન 1(અ) [3
ગુણ]}\label{uxaaauxab0uxab6uxaa8-1uxa85-3-uxa97uxaa3}

\textbf{AWT અને Swing વચ્ચેનો તફાવત સમજાવો.}

\begin{solutionbox}

{\def\LTcaptype{none} % do not increment counter
\begin{longtable}[]{@{}lll@{}}
\toprule\noalign{}
લક્ષણ & AWT & Swing \\
\midrule\noalign{}
\endhead
\bottomrule\noalign{}
\endlastfoot
\textbf{Platform} & Platform dependent & Platform independent \\
\textbf{Components} & Heavy weight & Light weight \\
\textbf{Look \& Feel} & Native OS look & Pluggable look \& feel \\
\textbf{Performance} & ઝડપી & AWT કરતાં ધીમું \\
\end{longtable}
}

\textbf{મુખ્ય મુદ્દાઓ:}

\begin{itemize}
\tightlist
\item
  \textbf{Heavy vs Light}: AWT native OS components વાપરે છે, Swing pure
  Java વાપરે છે
\item
  \textbf{દેખાવ}: AWT OS style અનુસરે છે, Swing બધા platforms પર સમાન look
  આપે છે
\item
  \textbf{સુવિધાઓ}: Swing વધુ advanced components જેમ કે JTable, JTree પ્રદાન
  કરે છે
\end{itemize}

\end{solutionbox}
\begin{mnemonicbox}
``Swing Provides Lightweight Components''

\end{mnemonicbox}
\begin{center}\rule{0.5\linewidth}{0.5pt}\end{center}

\subsection*{પ્રશ્ન 1(બ) [4
ગુણ]}\label{uxaaauxab0uxab6uxaa8-1uxaac-4-uxa97uxaa3}

\textbf{Mouse Motion Listener ને ઉદાહરણ સાથે સમજાવો.}

\begin{solutionbox}

MouseMotionListener interface Java Swing applications માં mouse movement
events ને handle કરે છે.


{\def\LTcaptype{none} % do not increment counter
\vspace{-5pt}
\captionof{table}{Mouse Motion Events}
\vspace{-10pt}
\begin{longtable}[]{@{}ll@{}}
\toprule\noalign{}
Method & હેતુ \\
\midrule\noalign{}
\endhead
\bottomrule\noalign{}
\endlastfoot
\textbf{mouseDragged()} & જ્યારે mouse drag થાય ત્યારે call થાય \\
\textbf{mouseMoved()} & જ્યારે mouse ખસે ત્યારે call થાય \\
\end{longtable}
}

\textbf{કોડ ઉદાહરણ:}

\begin{verbatim}
import javax.swing.*;
import java.awt.event.*;

class MouseMotionExample extends JFrame implements MouseMotionListener \{
    JLabel label;
    
    MouseMotionExample() \{
        label = new JLabel("અહીં mouse ખસાડો");
        add(label);
        addMouseMotionListener(this);
        setSize(400, 300);
        setVisible(true);
    \}
    
    public void mouseMoved(MouseEvent e) \{
        label.setText("Mouse આ સ્થાને: " + e.getX() + ", " + e.getY());
    \}
    
    public void mouseDragged(MouseEvent e) \{
        label.setText("Dragging આ સ્થાને: " + e.getX() + ", " + e.getY());
    \}
\}
\end{verbatim}

\end{solutionbox}
\begin{mnemonicbox}
``Mouse Motion Makes Dynamic''

\end{mnemonicbox}
\begin{center}\rule{0.5\linewidth}{0.5pt}\end{center}

\subsection*{પ્રશ્ન 1(ક) [7
ગુણ]}\label{uxaaauxab0uxab6uxaa8-1uxa95-7-uxa97uxaa3}

\textbf{યુનિવર્સિટી સાથે જોડાયેલા વિવિધ અભ્યાસક્રમો માટે checkboxes બનાવવા માટે
એક પ્રોગ્રામ ડેવલપ કરો જેથી પસંદ કરેલ કોર્સ પ્રદર્શિત થાય.}

\begin{solutionbox}

\begin{verbatim}
import javax.swing.*;
import java.awt.*;
import java.awt.event.*;

public class CourseSelection extends JFrame implements ItemListener \{
    JCheckBox java, python, cpp, web;
    JTextArea display;
    
    public CourseSelection() \{
        setTitle("યુનિવર્સિટી કોર્સ પસંદગી");
        setLayout(new FlowLayout());
        
        // checkboxes બનાવો
        java = new JCheckBox("Java Programming");
        python = new JCheckBox("Python Programming");
        cpp = new JCheckBox("C++ Programming");
        web = new JCheckBox("Web Development");
        
        // listeners ઉમેરો
        java.addItemListener(this);
        python.addItemListener(this);
        cpp.addItemListener(this);
        web.addItemListener(this);
        
        // Display area
        display = new JTextArea(10, 30);
        display.setEditable(false);
        
        // components ઉમેરો
        add(new JLabel("કોર્સ પસંદ કરો:"));
        add(java); add(python); add(cpp); add(web);
        add(new JScrollPane(display));
        
        setSize(400, 300);
        setDefaultCloseOperation(JFrame.EXIT\_ON\_CLOSE);
        setVisible(true);
    \}
    
    public void itemStateChanged(ItemEvent e) \{
        String courses = "પસંદ કરેલ કોર્સ:{n}";
        if(java.isSelected()) courses += "{- Java Programming}{n}";
        if(python.isSelected()) courses += "{- Python Programming}{n}";
        if(cpp.isSelected()) courses += "{- C++ Programming}{n}";
        if(web.isSelected()) courses += "{- Web Development}{n}";
        display.setText(courses);
    \}
    
    public static void main(String[] args) \{
        new CourseSelection();
    \}
\}
\end{verbatim}

\textbf{મુખ્ય લક્ષણો:}

\begin{itemize}
\tightlist
\item
  \textbf{ItemListener}: checkbox state changes ને detect કરે છે
\item
  \textbf{Dynamic Display}: real-time માં પસંદ કરેલા કોર્સ update કરે છે
\item
  \textbf{Multiple Selection}: એકથી વધુ કોર્સ પસંદ કરવાની મંજૂરી આપે છે
\end{itemize}

\end{solutionbox}
\begin{mnemonicbox}
``Check Items Listen Dynamically''

\end{mnemonicbox}
\begin{center}\rule{0.5\linewidth}{0.5pt}\end{center}

\subsection*{પ્રશ્ન 1(ક) અથવા [7
ગુણ]}\label{uxaaauxab0uxab6uxaa8-1uxa95-uxa85uxaa5uxab5-7-uxa97uxaa3}

\textbf{Swing components નો ઉપયોગ કરીને (JFrame, JRadioButton,
ItemListener વગેરેનો ઉપયોગ કરીને) Traffic signal (લાલ, લીલો અને પીળો)
implement કરવા માટે એક પ્રોગ્રામ વિકસાવો.}

\begin{solutionbox}

\begin{verbatim}
import javax.swing.*;
import java.awt.*;
import java.awt.event.*;

public class TrafficSignal extends JFrame implements ItemListener \{
    JRadioButton red, green, yellow;
    ButtonGroup group;
    JPanel signalPanel;
    
    public TrafficSignal() \{
        setTitle("Traffic Signal સિમ્યુલેટર");
        setLayout(new BorderLayout());
        
        // radio buttons બનાવો
        red = new JRadioButton("લાલ");
        green = new JRadioButton("લીલો"); 
        yellow = new JRadioButton("પીળો");
        
        // radio buttons ને group કરો
        group = new ButtonGroup();
        group.add(red); group.add(green); group.add(yellow);
        
        // listeners ઉમેરો
        red.addItemListener(this);
        green.addItemListener(this);
        yellow.addItemListener(this);
        
        // Signal display panel
        signalPanel = new JPanel() \{
            public void paintComponent(Graphics g) \{
                super.paintComponent(g);
                g.setColor(Color.BLACK);
                g.fillRect(50, 50, 100, 200);
                
                // વર્તુળો દોરો
                g.setColor(red.isSelected() ? Color.RED : Color.GRAY);
                g.fillOval(65, 65, 70, 70);
                
                g.setColor(yellow.isSelected() ? Color.YELLOW : Color.GRAY);
                g.fillOval(65, 105, 70, 70);
                
                g.setColor(green.isSelected() ? Color.GREEN : Color.GRAY);
                g.fillOval(65, 145, 70, 70);
            \}
        \;}
        
        JPanel controlPanel = new JPanel();
        controlPanel.add(red); controlPanel.add(yellow); controlPanel.add(green);
        
        add(controlPanel, BorderLayout.SOUTH);
        add(signalPanel, BorderLayout.CENTER);
        
        setSize(300, 400);
        setDefaultCloseOperation(JFrame.EXIT\_ON\_CLOSE);
        setVisible(true);
    \}
    
    public void itemStateChanged(ItemEvent e) \{
        signalPanel.repaint();
    \}
    
    public static void main(String[] args) \{
        new TrafficSignal();
    \}
\}
\end{verbatim}

\textbf{આકૃતિ:}

\begin{verbatim}
+{-{-}{-}{-}{-}{-}{-}{-}{-}{-}{-}{-}{-}{-}{-}{-}+}
|  Traffic Box   |
|   ┌─────────┐  |
|   │   RED   │  |
|   ├─────────┤  |
|   │ YELLOW  │  |
|   ├─────────┤  |
|   │  GREEN  │  |
|   └─────────┘  |
+{-{-}{-}{-}{-}{-}{-}{-}{-}{-}{-}{-}{-}{-}{-}{-}+}
  [R] [Y] [G]
\end{verbatim}

\end{solutionbox}
\begin{mnemonicbox}
``Radio Buttons Paint Graphics''

\end{mnemonicbox}
\begin{center}\rule{0.5\linewidth}{0.5pt}\end{center}

\subsection*{પ્રશ્ન 2(અ) [3
ગુણ]}\label{uxaaauxab0uxab6uxaa8-2uxa85-3-uxa97uxaa3}

\textbf{JDBC Type-4 driver સમજાવો.}

\begin{solutionbox}

\textbf{JDBC Type-4 Driver (Native Protocol Driver)}

{\def\LTcaptype{none} % do not increment counter
\begin{longtable}[]{@{}ll@{}}
\toprule\noalign{}
લક્ષણ & વર્ણન \\
\midrule\noalign{}
\endhead
\bottomrule\noalign{}
\endlastfoot
\textbf{પ્રકાર} & Pure Java driver \\
\textbf{Communication} & Direct database protocol \\
\textbf{Platform} & Platform independent \\
\textbf{Performance} & સર્વોચ્ચ પ્રદર્શન \\
\end{longtable}
}

\textbf{મુખ્ય મુદ્દાઓ:}

\begin{itemize}
\tightlist
\item
  \textbf{Pure Java}: કોઈ native code ની જરૂર નથી
\item
  \textbf{Direct Connection}: ડેટાબેઝ સાથે સીધો સંપર્ક કરે છે
\item
  \textbf{Network Protocol}: ડેટાબેઝના native network protocol નો ઉપયોગ કરે
  છે
\item
  \textbf{શ્રેષ્ઠ પ્રદર્શન}: બધા driver types માં સૌથી ઝડપી
\end{itemize}

\end{solutionbox}
\begin{mnemonicbox}
``Pure Java Direct Protocol''

\end{mnemonicbox}
\begin{center}\rule{0.5\linewidth}{0.5pt}\end{center}

\subsection*{પ્રશ્ન 2(બ) [4
ગુણ]}\label{uxaaauxab0uxab6uxaa8-2uxaac-4-uxa97uxaa3}

\textbf{Component class ની સામાન્ય રીતે વપરાતી methods સમજાવો.}

\begin{solutionbox}


{\def\LTcaptype{none} % do not increment counter
\vspace{-5pt}
\captionof{table}{Component Class Methods}
\vspace{-10pt}
\begin{longtable}[]{@{}ll@{}}
\toprule\noalign{}
Method & હેતુ \\
\midrule\noalign{}
\endhead
\bottomrule\noalign{}
\endlastfoot
\textbf{add()} & container માં component ઉમેરે છે \\
\textbf{setSize()} & component ના dimensions સેટ કરે છે \\
\textbf{setLayout()} & layout manager સેટ કરે છે \\
\textbf{setVisible()} & component ને દૃશ્યમાન/અદૃશ્ય બનાવે છે \\
\textbf{setBounds()} & position અને size સેટ કરે છે \\
\textbf{getSize()} & component નું size return કરે છે \\
\end{longtable}
}

\textbf{મુખ્ય લક્ષણો:}

\begin{itemize}
\tightlist
\item
  \textbf{Layout Management}: component arrangement ને control કરે છે
\item
  \textbf{Visibility Control}: components ને દેખાડે/છુપાવે છે
\item
  \textbf{Size Management}: component dimensions ને control કરે છે
\item
  \textbf{Container Operations}: child components ને manage કરે છે
\end{itemize}

\end{solutionbox}
\begin{mnemonicbox}
``Add Set Get Visibility''

\end{mnemonicbox}
\begin{center}\rule{0.5\linewidth}{0.5pt}\end{center}

\subsection*{પ્રશ્ન 2(ક) [7
ગુણ]}\label{uxaaauxab0uxab6uxaa8-2uxa95-7-uxa97uxaa3}

\textbf{ટેબલ `StuRec' માંથી વિદ્યાર્થીના રેકોર્ડ (Enroll No, Name, Address,
Mobile No અને Email-ID) દર્શાવવા માટે JDBC નો ઉપયોગ કરીને પ્રોગ્રામ વિકસાવો.}

\begin{solutionbox}

\begin{verbatim}
import java.sql.*;
import javax.swing.*;
import javax.swing.table.DefaultTableModel;

public class StudentRecordDisplay extends JFrame \{
    JTable table;
    DefaultTableModel model;
    
    public StudentRecordDisplay() \{
        setTitle("વિદ્યાર્થી રેકોર્ડ્સ");
        
        // table model બનાવો
        String[] columns = \{"Enroll No", "Name", "Address", "Mobile", "Email"\;}
        model = new DefaultTableModel(columns, 0);
        table = new JTable(model);
        
        // ડેટા લોડ કરો
        loadStudentData();
        
        add(new JScrollPane(table));
        setSize(600, 400);
        setDefaultCloseOperation(JFrame.EXIT\_ON\_CLOSE);
        setVisible(true);
    \}
    
    private void loadStudentData() \{
        try \{
            // ડેટાબેઝ કનેક્શન
            Class.forName("com.mysql.cj.jdbc.Driver");
            Connection con = DriverManager.getConnection(
                "jdbc:mysql://localhost:3306/university", "root", "password");
            
            // query execute કરો
            Statement stmt = con.createStatement();
            ResultSet rs = stmt.executeQuery("SELECT * FROM StuRec");
            
            // table માં ડેટા ઉમેરો
            while(rs.next()) \{
                String[] row = \{
                    rs.getString("enrollno"),
                    rs.getString("name"),
                    rs.getString("address"),
                    rs.getString("mobile"),
                    rs.getString("email")
                \;}
                model.addRow(row);
            \}
            
            con.close();
        \} catch(Exception e) \{
            JOptionPane.showMessageDialog(this, "Error: " + e.getMessage());
        \}
    \}
    
    public static void main(String[] args) \{
        new StudentRecordDisplay();
    \}
\}
\end{verbatim}

\textbf{ડેટાબેઝ ટેબલ માળખું:}

\begin{verbatim}
CREATE TABLE StuRec (
    enrollno VARCHAR(20) PRIMARY KEY,
    name VARCHAR(50),
    address VARCHAR(100),
    mobile VARCHAR(15),
    email VARCHAR(50)
);
\end{verbatim}

\end{solutionbox}
\begin{mnemonicbox}
``Connect Query Display Records''

\end{mnemonicbox}
\begin{center}\rule{0.5\linewidth}{0.5pt}\end{center}

\subsection*{પ્રશ્ન 2(અ) અથવા [3
ગુણ]}\label{uxaaauxab0uxab6uxaa8-2uxa85-uxa85uxaa5uxab5-3-uxa97uxaa3}

\textbf{JDBC ના ફાયદા અને ગેરફાયદા લખો.}

\begin{solutionbox}


{\def\LTcaptype{none} % do not increment counter
\vspace{-5pt}
\captionof{table}{JDBC ફાયદા અને ગેરફાયદા}
\vspace{-10pt}
\begin{longtable}[]{@{}ll@{}}
\toprule\noalign{}
ફાયદા & ગેરફાયદા \\
\midrule\noalign{}
\endhead
\bottomrule\noalign{}
\endlastfoot
\textbf{Platform Independent} & \textbf{Performance Overhead} \\
\textbf{Database Independent} & \textbf{શરૂઆતી લોકો માટે જટિલ} \\
\textbf{Standard API} & \textbf{SQL dependency} \\
\textbf{Transactions ને support કરે} & \textbf{Manual resource
management} \\
\end{longtable}
}

\textbf{મુખ્ય મુદ્દાઓ:}

\begin{itemize}
\tightlist
\item
  \textbf{પોર્ટેબિલિટી}: વિવિધ platforms અને databases પર કામ કરે છે
\item
  \textbf{સ્ટાન્ડર્ડાઇઝેશન}: database operations માટે uniform API
\item
  \textbf{પ્રદર્શન}: વધારાનું layer performance માં overhead લાવે છે
\item
  \textbf{જટિલતા}: યોગ્ય resource management જરૂરી
\end{itemize}

\end{solutionbox}
\begin{mnemonicbox}
``Platform Independent Standard Complex''

\end{mnemonicbox}
\begin{center}\rule{0.5\linewidth}{0.5pt}\end{center}

\subsection*{પ્રશ્ન 2(બ) અથવા [4
ગુણ]}\label{uxaaauxab0uxab6uxaa8-2uxaac-uxa85uxaa5uxab5-4-uxa97uxaa3}

\textbf{Border Layout સમજાવો.}

\begin{solutionbox}

BorderLayout container ને પાંચ વિસ્તારોમાં વહેંચે છે: North, South, East, West,
અને Center.

\textbf{આકૃતિ:}

\begin{verbatim}
+{-{-}{-}{-}{-}{-}{-}{-}{-}{-}{-}{-}{-}{-}{-}{-}{-}{-}+}
|      NORTH       |
+{-{-}{-}{-}{-}+{-}{-}{-}{-}{-}{-}+{-}{-}{-}{-}{-}+}
|WEST |CENTER| EAST|
+{-{-}{-}{-}{-}+{-}{-}{-}{-}{-}{-}+{-}{-}{-}{-}{-}+}
|      SOUTH       |
+{-{-}{-}{-}{-}{-}{-}{-}{-}{-}{-}{-}{-}{-}{-}{-}{-}{-}+}
\end{verbatim}


{\def\LTcaptype{none} % do not increment counter
\vspace{-5pt}
\captionof{table}{Border Layout વિસ્તારો}
\vspace{-10pt}
\begin{longtable}[]{@{}lll@{}}
\toprule\noalign{}
વિસ્તાર & સ્થાન & વર્તન \\
\midrule\noalign{}
\endhead
\bottomrule\noalign{}
\endlastfoot
\textbf{NORTH} & ઉપર & Preferred height, full width \\
\textbf{SOUTH} & નીચે & Preferred height, full width \\
\textbf{EAST} & જમણે & Preferred width, full height \\
\textbf{WEST} & ડાબે & Preferred width, full height \\
\textbf{CENTER} & વચ્ચે & બાકીની જગ્યા લે છે \\
\end{longtable}
}

\textbf{કોડ ઉદાહરણ:}

\begin{verbatim}
setLayout(new BorderLayout());
add(new JButton("ઉત્તર"), BorderLayout.NORTH);
add(new JButton("મધ્ય"), BorderLayout.CENTER);
\end{verbatim}

\end{solutionbox}
\begin{mnemonicbox}
``North South East West Center''

\end{mnemonicbox}
\begin{center}\rule{0.5\linewidth}{0.5pt}\end{center}

\subsection*{પ્રશ્ન 2(ક) અથવા [7
ગુણ]}\label{uxaaauxab0uxab6uxaa8-2uxa95-uxa85uxaa5uxab5-7-uxa97uxaa3}

\textbf{Hibernate CRUD operations નો ઉપયોગ કરીને Employee (NAME, AGE,
SALARY અને DEPARTMENT) નો ડેટા store, update, fetch અને delete માટે એપ્લિકેશન
ડેવલપ કરો.}

\begin{solutionbox}

\textbf{Employee Entity Class:}

\begin{verbatim}
import javax.persistence.*;

@Entity
@Table(name = "employees")
public class Employee \{
    @Id
    @GeneratedValue(strategy = GenerationType.IDENTITY)
    private int id;
    
    private String name;
    private int age;
    private double salary;
    private String department;
    
    // Constructors, getters, setters
    public Employee() \{\}
    
    public Employee(String name, int age, double salary, String dept) \{
        this.name = name;
        this.age = age;
        this.salary = salary;
        this.department = dept;
    \}
    
    // Getters અને Setters
    public int getId() \{ return id; \}
    public void setId(int id) \{ this.id = id; \}
    
    public String getName() \{ return name; \}
    public void setName(String name) \{ this.name = name; \}
    
    // ... અન્ય getters/setters
\}
\end{verbatim}

\textbf{CRUD Operations Class:}

\begin{verbatim}
import org.hibernate.*;
import org.hibernate.cfg.Configuration;

public class EmployeeCRUD \{
    private SessionFactory factory;
    
    public EmployeeCRUD() \{
        factory = new Configuration()
                    .configure("hibernate.cfg.xml")
                    .addAnnotatedClass(Employee.class)
                    .buildSessionFactory();
    \}
    
    // CREATE
    public void saveEmployee(Employee emp) \{
        Session session = factory.openSession();
        Transaction tx = session.beginTransaction();
        session.save(emp);
        tx.commit();
        session.close();
    \}
    
    // READ
    public Employee getEmployee(int id) \{
        Session session = factory.openSession();
        Employee emp = session.get(Employee.class, id);
        session.close();
        return emp;
    \}
    
    // UPDATE
    public void updateEmployee(Employee emp) \{
        Session session = factory.openSession();
        Transaction tx = session.beginTransaction();
        session.update(emp);
        tx.commit();
        session.close();
    \}
    
    // DELETE
    public void deleteEmployee(int id) \{
        Session session = factory.openSession();
        Transaction tx = session.beginTransaction();
        Employee emp = session.get(Employee.class, id);
        session.delete(emp);
        tx.commit();
        session.close();
    \}
\}
\end{verbatim}

\end{solutionbox}
\begin{mnemonicbox}
``Save Get Update Delete Hibernate''

\end{mnemonicbox}
\begin{center}\rule{0.5\linewidth}{0.5pt}\end{center}

\subsection*{પ્રશ્ન 3(અ) [3
ગુણ]}\label{uxaaauxab0uxab6uxaa8-3uxa85-3-uxa97uxaa3}

\textbf{Deployment Descriptor સમજાવો.}

\begin{solutionbox}

Deployment Descriptor (web.xml) web applications માટે configuration file
છે જેમાં servlet mappings, initialization parameters, અને security settings
હોય છે.


{\def\LTcaptype{none} % do not increment counter
\vspace{-5pt}
\captionof{table}{Deployment Descriptor Elements}
\vspace{-10pt}
\begin{longtable}[]{@{}ll@{}}
\toprule\noalign{}
Element & હેતુ \\
\midrule\noalign{}
\endhead
\bottomrule\noalign{}
\endlastfoot
\textbf{\textless servlet\textgreater{}} & servlet configuration define
કરે છે \\
\textbf{\textless servlet-mapping\textgreater{}} & servlet ને URL pattern
સાથે map કરે છે \\
\textbf{\textless init-param\textgreater{}} & initialization parameters
સેટ કરે છે \\
\textbf{\textless welcome-file-list\textgreater{}} & default files serve
કરવા માટે \\
\end{longtable}
}

\textbf{મુખ્ય લક્ષણો:}

\begin{itemize}
\tightlist
\item
  \textbf{Configuration}: web app માટે કેન્દ્રીય configuration
\item
  \textbf{Servlet Mapping}: URL to servlet mapping
\item
  \textbf{Parameters}: initialization અને context parameters
\item
  \textbf{Security}: authentication અને authorization settings
\end{itemize}

\end{solutionbox}
\begin{mnemonicbox}
``Web XML Configuration Mapping''

\end{mnemonicbox}
\begin{center}\rule{0.5\linewidth}{0.5pt}\end{center}

\subsection*{પ્રશ્ન 3(બ) [4
ગુણ]}\label{uxaaauxab0uxab6uxaa8-3uxaac-4-uxa97uxaa3}

\textbf{servlet માં get અને post method વચ્ચેનો તફાવત સમજાવો.}

\begin{solutionbox}


{\def\LTcaptype{none} % do not increment counter
\vspace{-5pt}
\captionof{table}{GET vs POST Methods}
\vspace{-10pt}
\begin{longtable}[]{@{}lll@{}}
\toprule\noalign{}
લક્ષણ & GET & POST \\
\midrule\noalign{}
\endhead
\bottomrule\noalign{}
\endlastfoot
\textbf{Data Location} & URL query string & Request body \\
\textbf{Data Size} & મર્યાદિત (2048 chars) & અમર્યાદિત \\
\textbf{Security} & ઓછું સુરક્ષિત (દૃશ્યમાન) & વધુ સુરક્ષિત \\
\textbf{Caching} & Cache થઈ શકે છે & Cache થતું નથી \\
\textbf{Bookmarking} & Bookmark કરી શકાય & Bookmark કરી શકાતું નથી \\
\textbf{હેતુ} & ડેટા retrieve કરવા & ડેટા submit/modify કરવા \\
\end{longtable}
}

\textbf{મુખ્ય મુદ્દાઓ:}

\begin{itemize}
\tightlist
\item
  \textbf{દૃશ્યતા}: GET ડેટા URL માં દેખાય છે, POST છુપાયેલું હોય છે
\item
  \textbf{ક્ષમતા}: POST મોટો ડેટા handle કરી શકે છે
\item
  \textbf{સુરક્ષા}: POST sensitive ડેટા માટે વધુ સારી
\item
  \textbf{ઉપયોગ}: GET fetching માટે, POST form submission માટે
\end{itemize}

\end{solutionbox}
\begin{mnemonicbox}
``GET Visible Limited, POST Hidden Unlimited''

\end{mnemonicbox}
\begin{center}\rule{0.5\linewidth}{0.5pt}\end{center}

\subsection*{પ્રશ્ન 3(ક) [7
ગુણ]}\label{uxaaauxab0uxab6uxaa8-3uxa95-7-uxa97uxaa3}

\textbf{એક સરળ servlet પ્રોગ્રામ વિકસાવો જે તેના લોડિંગ પછી કેટલી વખત તેને
access કરવામાં આવ્યું છે તેમાટે counter જાળવી રાખે છે; deployment descriptor નો
ઉપયોગ કરીને counter ને પ્રારંભ કરો.}

\begin{solutionbox}

\textbf{Servlet કોડ:}

\begin{verbatim}
import java.io.*;
import javax.servlet.*;
import javax.servlet.http.*;

public class CounterServlet extends HttpServlet \{
    private int counter;
    
    public void init() throws ServletException \{
        String initialValue = getInitParameter("initialCount");
        counter = Integer.parseInt(initialValue);
    \}
    
    protected void doGet(HttpServletRequest request, 
                        HttpServletResponse response) 
                        throws ServletException, IOException \{
        
        response.setContentType("text/html");
        PrintWriter out = response.getWriter();
        
        synchronized(this) \{
            counter++;
        \}
        
        out.println("{htmlbody"});
        out.println("{h2પેજ Access કાઉન્ટર/h2"});
        out.println("{pઆ પેજ "} + counter + " વખત access કરવામાં આવ્યું છે{/p"});
        out.println("{pa href=CounterServletRefresh/a/p"});
        out.println("{/body/html"});
        
        out.close();
    \}
\}
\end{verbatim}

\textbf{web.xml Configuration:}

\begin{verbatim}
{?xml} version="1.0" encoding="UTF{-8"}?{}
{}web{-app}{}
    {}servlet{}
        {}servlet{-name}{CounterServlet/}servlet{-name}{}
        {}servlet{-class}{CounterServlet/}servlet{-class}{}
        {}init{-param}{}
            {}param{-name}{initialCount/}param{-name}{}
            {}param{-value}{0/}param{-value}{}
        {/}init{-param}{}
        {}load{-on{-}startup}{1/}load{-on{-}startup}{}
    {/}servlet{}
    
    {}servlet{-mapping}{}
        {}servlet{-name}{CounterServlet/}servlet{-name}{}
        {}url{-pattern}{/counter/}url{-pattern}{}
    {/}servlet{-mapping}{}
{/}web{-app}{}
\end{verbatim}

\textbf{મુખ્ય લક્ષણો:}

\begin{itemize}
\tightlist
\item
  \textbf{Thread Safety}: synchronized counter increment
\item
  \textbf{Initialization}: web.xml માંથી counter initialized
\item
  \textbf{Persistent}: requests ની વચ્ચે counter maintained
\item
  \textbf{Configuration}: deployment descriptor setup
\end{itemize}

\end{solutionbox}
\begin{mnemonicbox}
``Initialize Synchronize Count Display''

\end{mnemonicbox}
\begin{center}\rule{0.5\linewidth}{0.5pt}\end{center}

\subsection*{પ્રશ્ન 3(અ) અથવા [3
ગુણ]}\label{uxaaauxab0uxab6uxaa8-3uxa85-uxa85uxaa5uxab5-3-uxa97uxaa3}

\textbf{servlet ના life cycle સમજાવો.}

\begin{solutionbox}

\textbf{Servlet Life Cycle આકૃતિ:}

\begin{verbatim}
stateDiagram{-v2}
  direction LR
    [*] {-{-} Loading}
    Loading {-{-} init()}
    init() {-{-} service()}
    service() {-{-} service() : Multiple requests}
    service() {-{-} destroy()}
    destroy() {-{-} [*]}
\end{verbatim}


{\def\LTcaptype{none} % do not increment counter
\vspace{-5pt}
\captionof{table}{Servlet Life Cycle Methods}
\vspace{-10pt}
\begin{longtable}[]{@{}lll@{}}
\toprule\noalign{}
Method & હેતુ & Called \\
\midrule\noalign{}
\endhead
\bottomrule\noalign{}
\endlastfoot
\textbf{init()} & servlet initialize કરે છે & startup પર એક વખત \\
\textbf{service()} & requests handle કરે છે & દરેક request માટે \\
\textbf{destroy()} & resources cleanup કરે છે & shutdown પર એક વખત \\
\end{longtable}
}

\textbf{મુખ્ય મુદ્દાઓ:}

\begin{itemize}
\tightlist
\item
  \textbf{Initialization}: servlet load થાય ત્યારે એક વખત call થાય છે
\item
  \textbf{Service}: બધી client requests handle કરે છે
\item
  \textbf{Cleanup}: servlet unload થાય તે પહેલાં call થાય છે
\item
  \textbf{Container Managed}: web container lifecycle ને control કરે છે
\end{itemize}

\end{solutionbox}
\begin{mnemonicbox}
``Initialize Service Destroy''

\end{mnemonicbox}
\begin{center}\rule{0.5\linewidth}{0.5pt}\end{center}

\subsection*{પ્રશ્ન 3(બ) અથવા [4
ગુણ]}\label{uxaaauxab0uxab6uxaa8-3uxaac-uxa85uxaa5uxab5-4-uxa97uxaa3}

\textbf{Servlet Config class ને યોગ્ય ઉદાહરણ સાથે સમજાવો.}

\begin{solutionbox}

ServletConfig servlet-specific configuration information અને
initialization parameters પ્રદાન કરે છે.


{\def\LTcaptype{none} % do not increment counter
\vspace{-5pt}
\captionof{table}{ServletConfig Methods}
\vspace{-10pt}
\begin{longtable}[]{@{}ll@{}}
\toprule\noalign{}
Method & હેતુ \\
\midrule\noalign{}
\endhead
\bottomrule\noalign{}
\endlastfoot
\textbf{getInitParameter()} & init parameter value મેળવે છે \\
\textbf{getInitParameterNames()} & બધા parameter names મેળવે છે \\
\textbf{getServletContext()} & servlet context મેળવે છે \\
\textbf{getServletName()} & servlet name મેળવે છે \\
\end{longtable}
}

\textbf{ઉદાહરણ:}

\begin{verbatim}
public class ConfigServlet extends HttpServlet \{
    String databaseURL, username;
    
    public void init() throws ServletException \{
        ServletConfig config = getServletConfig();
        databaseURL = config.getInitParameter("dbURL");
        username = config.getInitParameter("dbUser");
    \}
    
    protected void doGet(HttpServletRequest request, 
                        HttpServletResponse response) 
                        throws ServletException, IOException \{
        
        PrintWriter out = response.getWriter();
        out.println("Database URL: " + databaseURL);
        out.println("Username: " + username);
    \}
\}
\end{verbatim}

\textbf{web.xml:}

\begin{verbatim}
{}servlet{}
    {}servlet{-name}{ConfigServlet/}servlet{-name}{}
    {}servlet{-class}{ConfigServlet/}servlet{-class}{}
    {}init{-param}{}
        {}param{-name}{dbURL/}param{-name}{}
        {}param{-value}{jdbc:mysql://localhost:3306/test/}param{-value}{}
    {/}init{-param}{}
    {}init{-param}{}
        {}param{-name}{dbUser/}param{-name}{}
        {}param{-value}{root/}param{-value}{}
    {/}init{-param}{}
{/}servlet{}
\end{verbatim}

\end{solutionbox}
\begin{mnemonicbox}
``Config Gets Parameters Context''

\end{mnemonicbox}
\begin{center}\rule{0.5\linewidth}{0.5pt}\end{center}

\subsection*{પ્રશ્ન 3(ક) અથવા [7
ગુણ]}\label{uxaaauxab0uxab6uxaa8-3uxa95-uxa85uxaa5uxab5-7-uxa97uxaa3}

\textbf{એક સરળ પ્રોગ્રામ ડેવલપ કરો, જ્યારે વપરાશકર્તા subject code પસંદ કરશે,
ત્યારે subject નું નામ servlet અને MySQL database નો ઉપયોગ કરીને પ્રદર્શિત થશે.}

\begin{solutionbox}

\textbf{HTML Form (index.html):}

\begin{verbatim}
{!DOCTYPE} html{}
{}html{}
{}head{}
    {}title{}વિષય પસંદગી{/}title{}
{/}head{}
{}body{}
    {}h2{}વિષય કોડ પસંદ કરો{/}h2{}
    {}form action="SubjectServlet" method="get"{}
        {}select name="subjectCode"{}
            {}option value=""{}વિષય પસંદ કરો{/}option{}
            {}option value="4351603"{}4351603{/}option{}
            {}option value="4351604"{}4351604{/}option{}
            {}option value="4351605"{}4351605{/}option{}
        {/}select{}
        {}input type="submit" value="વિષયનું નામ મેળવો"{}
    {/}form{}
{/}body{}
{/}html{}
\end{verbatim}

\textbf{Servlet કોડ:}

\begin{verbatim}
import java.io.*;
import java.sql.*;
import javax.servlet.*;
import javax.servlet.http.*;

public class SubjectServlet extends HttpServlet \{
    
    protected void doGet(HttpServletRequest request, 
                        HttpServletResponse response) 
                        throws ServletException, IOException \{
        
        response.setContentType("text/html;charset=UTF{-8"});
        PrintWriter out = response.getWriter();
        
        String subjectCode = request.getParameter("subjectCode");
        String subjectName = "";
        
        if(subjectCode != null \&\& !subjectCode.equals("")) \{
            try \{
                Class.forName("com.mysql.cj.jdbc.Driver");
                Connection con = DriverManager.getConnection(
                    "jdbc:mysql://localhost:3306/university", "root", "password");
                
                PreparedStatement ps = con.prepareStatement(
                    "SELECT subject\_name FROM subjects WHERE subject\_code = ?");
                ps.setString(1, subjectCode);
                
                ResultSet rs = ps.executeQuery();
                if(rs.next()) \{
                    subjectName = rs.getString("subject\_name");
                \}
                
                con.close();
            \} catch(Exception e) \{
                subjectName = "Error: " + e.getMessage();
            \}
        \}
        
        out.println("{htmlbody"});
        out.println("{h2વિષયની માહિતી/h2"});
        if(!subjectName.equals("")) \{
            out.println("{pવિષય કોડ: "} + subjectCode + "{/p"});
            out.println("{pવિષયનું નામ: "} + subjectName + "{/p"});
        \} else \{
            out.println("{pકૃપા કરીને વિષય કોડ પસંદ કરો/p"});
        \}
        out.println("{pa href=index.htmlપાછા જાઓ/a/p"});
        out.println("{/body/html"});
    \}
\}
\end{verbatim}

\textbf{ડેટાબેઝ ટેબલ:}

\begin{verbatim}
CREATE TABLE subjects (
    subject\_code VARCHAR(10) PRIMARY KEY,
    subject\_name VARCHAR(100)
);

INSERT INTO subjects VALUES 
({4351603}, {Advanced Java Programming}),
({4351604}, {Web Technology}),
({4351605}, {Database Management System});
\end{verbatim}

\end{solutionbox}
\begin{mnemonicbox}
``Select Query Display Subject''

\end{mnemonicbox}
\begin{center}\rule{0.5\linewidth}{0.5pt}\end{center}

\subsection*{પ્રશ્ન 4(અ) [3
ગુણ]}\label{uxaaauxab0uxab6uxaa8-4uxa85-3-uxa97uxaa3}

\textbf{JSP life cycle સમજાવો.}

\begin{solutionbox}

\textbf{JSP Life Cycle આકૃતિ:}

\begin{verbatim}
stateDiagram{-v2}
  direction LR
    [*] {-{-} Translation}
    Translation {-{-} Compilation}
    Compilation {-{-} Loading}
    Loading {-{-} jspInit()}
    jspInit() {-{-} \_jspService()}
    \_jspService() {-{-} \_jspService() : Multiple requests}
    \_jspService() {-{-} jspDestroy()}
    jspDestroy() {-{-} [*]}
\end{verbatim}


{\def\LTcaptype{none} % do not increment counter
\vspace{-5pt}
\captionof{table}{JSP Life Cycle તબક્કાઓ}
\vspace{-10pt}
\begin{longtable}[]{@{}ll@{}}
\toprule\noalign{}
તબક્કો & વર્ણન \\
\midrule\noalign{}
\endhead
\bottomrule\noalign{}
\endlastfoot
\textbf{Translation} & JSP to Servlet conversion \\
\textbf{Compilation} & Servlet to bytecode \\
\textbf{Loading} & servlet class ને load કરે છે \\
\textbf{Initialization} & jspInit() call થાય છે \\
\textbf{Request Processing} & \_jspService() requests handle કરે છે \\
\textbf{Destruction} & jspDestroy() cleanup \\
\end{longtable}
}

\end{solutionbox}
\begin{mnemonicbox}
``Translate Compile Load Initialize Service
Destroy''

\end{mnemonicbox}
\begin{center}\rule{0.5\linewidth}{0.5pt}\end{center}

\subsection*{પ્રશ્ન 4(બ) [4
ગુણ]}\label{uxaaauxab0uxab6uxaa8-4uxaac-4-uxa97uxaa3}

\textbf{JSP અને Servlet ની સરખામણી કરો.}

\begin{solutionbox}


{\def\LTcaptype{none} % do not increment counter
\vspace{-5pt}
\captionof{table}{JSP vs Servlet સરખામણી}
\vspace{-10pt}
\begin{longtable}[]{@{}lll@{}}
\toprule\noalign{}
લક્ષણ & JSP & Servlet \\
\midrule\noalign{}
\endhead
\bottomrule\noalign{}
\endlastfoot
\textbf{કોડ પ્રકાર} & HTML with Java code & Pure Java code \\
\textbf{ડેવલપમેન્ટ} & web designers માટે સરળ & Java developers માટે વધુ સારું \\
\textbf{કમ્પાઈલેશન} & આપોઆપ & મેન્યુઅલ \\
\textbf{ફેરફાર} & restart ની જરૂર નથી & restart જરૂરી \\
\textbf{પર્ફોર્મન્સ} & પહેલી request ધીમી & ઝડપી \\
\textbf{જાળવણી} & સરળ & જટિલ \\
\end{longtable}
}

\textbf{મુખ્ય મુદ્દાઓ:}

\begin{itemize}
\tightlist
\item
  \textbf{ઉપયોગમાં સરળતા}: JSP presentation layer માટે સરળ
\item
  \textbf{પર્ફોર્મન્સ}: Servlet business logic માટે વધુ સારું
\item
  \textbf{લવચીકતા}: JSP dynamic content માટે વધુ સારું
\item
  \textbf{નિયંત્રણ}: Servlet વધુ control પ્રદાન કરે છે
\end{itemize}

\end{solutionbox}
\begin{mnemonicbox}
``JSP Easy HTML, Servlet Pure Java''

\end{mnemonicbox}
\begin{center}\rule{0.5\linewidth}{0.5pt}\end{center}

\subsection*{પ્રશ્ન 4(ક) [7
ગુણ]}\label{uxaaauxab0uxab6uxaa8-4uxa95-7-uxa97uxaa3}

\textbf{Enrollment number દ્વારા વર્તમાન સેમેસ્ટરના દરેક વિષયમાં વિદ્યાર્થીની
માસિક હાજરી દર્શાવવા માટે JSP web application ડેવલપ કરો.}

\begin{solutionbox}

\textbf{Input Form (attendance.html):}

\begin{verbatim}
{!DOCTYPE} html{}
{}html{}
{}head{}
    {}title{}વિદ્યાર્થી હાજરી{/}title{}
{/}head{}
{}body{}
    {}h2{}વિદ્યાર્થી હાજરી તપાસો{/}h2{}
    {}form action="attendanceCheck.jsp" method="post"{}
        {}table{}
            {}tr{}
                {}td{}Enrollment નંબર:{/}td{}
                {}td{}input type="text" name="enrollNo" required{/}td{}
            {/}tr{}
            {}tr{}
                {}td{}મહિનો:{/}td{}
                {}td{}
                    {}select name="month" required{}
                        {}option value=""{}મહિનો પસંદ કરો{/}option{}
                        {}option value="January"{}જાન્યુઆરી{/}option{}
                        {}option value="February"{}ફેબ્રુઆરી{/}option{}
                        {}option value="March"{}માર્ચ{/}option{}
                    {/}select{}
                {/}td{}
            {/}tr{}
            {}tr{}
                {}td colspan="2"{}
                    {}input type="submit" value="હાજરી તપાસો"{}
                {/}td{}
            {/}tr{}
        {/}table{}
    {/}form{}
{/}body{}
{/}html{}
\end{verbatim}

\textbf{JSP Page (attendanceCheck.jsp):}

\begin{verbatim}
{\%@ page} import="java.sql.*" \%{}
{\%@ page} contentType="text/html;charset=UTF{-8"} \%{}

{html}
{head}
    {titleહાજરી રિપોર્ટ/title}
    {style}
        table \{ border{-collapse}: collapse; width: 100\%; \}
        th, td \{ border: 1px solid black; padding: 8px; text{-align}: center; \}
        th \{ background{-color}: \#f2f2f2; \}
    {/style}
{/head}
{body}
    {h2માસિક હાજરી રિપોર્ટ/h2}
    
    {\%}
        String enrollNo = request.getParameter("enrollNo");
        String month = request.getParameter("month");
        
        if(enrollNo != null \&\& month != null) \{
            try \{
                Class.forName("com.mysql.cj.jdbc.Driver");
                Connection con = DriverManager.getConnection(
                    "jdbc:mysql://localhost:3306/university", "root", "password");
                
                // વિદ્યાર્થીની માહિતી મેળવો
                PreparedStatement ps1 = con.prepareStatement(
                    "SELECT name FROM students WHERE enroll\_no = ?");
                ps1.setString(1, enrollNo);
                ResultSet rs1 = ps1.executeQuery();
                
                String studentName = "";
                if(rs1.next()) \{
                    studentName = rs1.getString("name");
                \}
                
                out.println("{pstrongવિદ્યાર્થી:/strong "} + studentName + 
                           " (" + enrollNo + "){/p"});
                out.println("{pstrongમહિનો:/strong "} + month + "{/p"});
                
                // હાજરી ડેટા મેળવો
                PreparedStatement ps2 = con.prepareStatement(
                    "SELECT s.subject\_name, a.total\_classes, a.attended\_classes, " +
                    "ROUND((a.attended\_classes/a.total\_classes)*100, 2) as percentage " +
                    "FROM attendance a JOIN subjects s ON a.subject\_code = s.subject\_code " +
                    "WHERE a.enroll\_no = ? AND a.month = ?");
                ps2.setString(1, enrollNo);
                ps2.setString(2, month);
                ResultSet rs2 = ps2.executeQuery();
                
                out.println("{table"});
                out.println("{trthવિષય/ththકુલ વર્ગો/th"} +
                           "{thહાજર થયેલ/ththટકાવારી/ththસ્થિતિ/th/tr"});
                
                while(rs2.next()) \{
                    String subjectName = rs2.getString("subject\_name");
                    int totalClasses = rs2.getInt("total\_classes");
                    int attendedClasses = rs2.getInt("attended\_classes");
                    double percentage = rs2.getDouble("percentage");
                    String status = percentage {=} 75 ? "સારી" : "નબળી";
                    String rowColor = percentage {=} 75 ? "lightgreen" : "lightcoral";
                    
                    out.println("{tr style=background{-}color:"} + rowColor + "{"});
                    out.println("{td"} + subjectName + "{/td"});
                    out.println("{td"} + totalClasses + "{/td"});
                    out.println("{td"} + attendedClasses + "{/td"});
                    out.println("{td"} + percentage + "\%{/td"});
                    out.println("{td"} + status + "{/td"});
                    out.println("{/tr"});
                \}
                
                out.println("{/table"});
                con.close();
                
            \} catch(Exception e) \{
                out.println("{p style=color:redError: "} + e.getMessage() + "{/p"});
            \}
        \}
    \%{}
    
    {br} /{}
    {a} href="attendance.html"{બીજા વિદ્યાર્થીની તપાસ કરો/a}
{/body}
{/html}
\end{verbatim}

\textbf{ડેટાબેઝ ટેબલ્સ:}

\begin{verbatim}
CREATE TABLE students (
    enroll\_no VARCHAR(20) PRIMARY KEY,
    name VARCHAR(50)
);

CREATE TABLE subjects (
    subject\_code VARCHAR(10) PRIMARY KEY,
    subject\_name VARCHAR(100)
);

CREATE TABLE attendance (
    id INT AUTO\_INCREMENT PRIMARY KEY,
    enroll\_no VARCHAR(20),
    subject\_code VARCHAR(10),
    month VARCHAR(15),
    total\_classes INT,
    attended\_classes INT,
    FOREIGN KEY (enroll\_no) REFERENCES students(enroll\_no),
    FOREIGN KEY (subject\_code) REFERENCES subjects(subject\_code)
);
\end{verbatim}

\end{solutionbox}
\begin{mnemonicbox}
``JSP Database Query Display Table''

\end{mnemonicbox}
\begin{center}\rule{0.5\linewidth}{0.5pt}\end{center}

\subsection*{પ્રશ્ન 4(અ) અથવા [3
ગુણ]}\label{uxaaauxab0uxab6uxaa8-4uxa85-uxa85uxaa5uxab5-3-uxa97uxaa3}

\textbf{JSP માં implicit objects સમજાવો.}

\begin{solutionbox}


{\def\LTcaptype{none} % do not increment counter
\vspace{-5pt}
\captionof{table}{JSP Implicit Objects}
\vspace{-10pt}
\begin{longtable}[]{@{}lll@{}}
\toprule\noalign{}
Object & Type & હેતુ \\
\midrule\noalign{}
\endhead
\bottomrule\noalign{}
\endlastfoot
\textbf{request} & HttpServletRequest & request ડેટા મેળવે છે \\
\textbf{response} & HttpServletResponse & response મોકલે છે \\
\textbf{out} & JspWriter & client ને output \\
\textbf{session} & HttpSession & session management \\
\textbf{application} & ServletContext & application scope \\
\textbf{config} & ServletConfig & servlet configuration \\
\textbf{pageContext} & PageContext & page scope access \\
\textbf{page} & Object & વર્તમાન servlet instance \\
\textbf{exception} & Throwable & error page exception \\
\end{longtable}
}

\textbf{મુખ્ય લક્ષણો:}

\begin{itemize}
\tightlist
\item
  \textbf{આપોઆપ}: declaration વિના ઉપલબ્ધ
\item
  \textbf{Scope Access}: વિવિધ scope levels
\item
  \textbf{Request Handling}: input/output operations
\item
  \textbf{Session Management}: વપરાશકર્તા session tracking
\end{itemize}

\end{solutionbox}
\begin{mnemonicbox}
``Request Response Out Session Application''

\end{mnemonicbox}
\begin{center}\rule{0.5\linewidth}{0.5pt}\end{center}

\subsection*{પ્રશ્ન 4(બ) અથવા [4
ગુણ]}\label{uxaaauxab0uxab6uxaa8-4uxaac-uxa85uxaa5uxab5-4-uxa97uxaa3}

\textbf{servlet કરતાં JSP શા માટે પસંદ કરવામાં આવે છે તે સમજાવો.}

\begin{solutionbox}


{\def\LTcaptype{none} % do not increment counter
\vspace{-5pt}
\captionof{table}{Servlet કરતાં JSP ના ફાયદા}
\vspace{-10pt}
\begin{longtable}[]{@{}ll@{}}
\toprule\noalign{}
પાસું & JSP ફાયદો \\
\midrule\noalign{}
\endhead
\bottomrule\noalign{}
\endlastfoot
\textbf{ડેવલપમેન્ટ} & HTML integration સરળ \\
\textbf{જાળવણી} & presentation ને logic થી અલગ કરે \\
\textbf{કમ્પાઈલેશન} & આપોઆપ compilation \\
\textbf{ફેરફાર} & server restart ની જરૂર નથી \\
\textbf{ડિઝાઈન} & web designer friendly \\
\textbf{કોડ પુનઃઉપયોગ} & tag libraries અને custom tags \\
\end{longtable}
}

\textbf{મુખ્ય મુદ્દાઓ:}

\begin{itemize}
\tightlist
\item
  \textbf{Separation of Concerns}: presentation અને business logic નું સ્પષ્ટ
  વિભાજન
\item
  \textbf{ઝડપી ડેવલપમેન્ટ}: ઝડપી development cycle
\item
  \textbf{Designer Friendly}: web designers HTML-જેવા syntax સાથે કામ કરી
  શકે
\item
  \textbf{આપોઆપ સુવિધાઓ}: container compilation અને lifecycle handle કરે
\end{itemize}

\end{solutionbox}
\begin{mnemonicbox}
``Easy HTML Automatic Designer Friendly''

\end{mnemonicbox}
\begin{center}\rule{0.5\linewidth}{0.5pt}\end{center}

\subsection*{પ્રશ્ન 4(ક) અથવા [7
ગુણ]}\label{uxaaauxab0uxab6uxaa8-4uxa95-uxa85uxaa5uxab5-7-uxa97uxaa3}

\textbf{પાંચ વિષયોના ગુણ સ્વીકારીને વિદ્યાર્થીના ગ્રેડ દર્શાવવા માટે JSP પ્રોગ્રામ
વિકસાવો.}

\begin{solutionbox}

\textbf{Input Form (gradeInput.html):}

\begin{verbatim}
{!DOCTYPE} html{}
{}html{}
{}head{}
    {}title{}વિદ્યાર્થી ગ્રેડ કેલ્ક્યુલેટર{/}title{}
    {}style{}
        table \{ margin: auto; border{-collapse}: collapse; \}
        td \{ padding: 10px; \}
        input[type="number"] \{ width: 100px; \}
        input[type="submit"] \{ padding: 10px 20px; \}
    {/}style{}
{/}head{}
{}body{}
    {}h2 style="text{-align: center;"}{}વિદ્યાર્થી ગ્રેડ કેલ્ક્યુલેટર{/}h2{}
    {}form action="gradeCalculator.jsp" method="post"{}
        {}table border="1"{}
            {}tr{}
                {}td{}વિદ્યાર્થીનું નામ:{/}td{}
                {}td{}input type="text" name="studentName" required{/}td{}
            {/}tr{}
            {}tr{}
                {}td{}વિષય 1 ગુણ:{/}td{}
                {}td{}input type="number" name="marks1" min="0" max="100" required{/}td{}
            {/}tr{}
            {}tr{}
                {}td{}વિષય 2 ગુણ:{/}td{}
                {}td{}input type="number" name="marks2" min="0" max="100" required{/}td{}
            {/}tr{}
            {}tr{}
                {}td{}વિષય 3 ગુણ:{/}td{}
                {}td{}input type="number" name="marks3" min="0" max="100" required{/}td{}
            {/}tr{}
            {}tr{}
                {}td{}વિષય 4 ગુણ:{/}td{}
                {}td{}input type="number" name="marks4" min="0" max="100" required{/}td{}
            {/}tr{}
            {}tr{}
                {}td{}વિષય 5 ગુણ:{/}td{}
                {}td{}input type="number" name="marks5" min="0" max="100" required{/}td{}
            {/}tr{}
            {}tr{}
                {}td colspan="2" style="text{-align: center;"}{}
                    {}input type="submit" value="ગ્રેડ કેલ્ક્યુલેટ કરો"{}
                {/}td{}
            {/}tr{}
        {/}table{}
    {/}form{}
{/}body{}
{/}html{}
\end{verbatim}

\textbf{JSP Grade Calculator (gradeCalculator.jsp):}

\begin{verbatim}
{\%@ page} contentType="text/html;charset=UTF{-8"} \%{}

{html}
{head}
    {titleગ્રેડ પરિણામ/title}
    {style}
        .result{-table} \{ 
            margin: auto; 
            border{-collapse}: collapse; 
            margin{-top}: 20px;
        \}
        .result{-table} th, .result{-table} td \{ 
            border: 1px solid black; 
            padding: 10px; 
            text{-align}: center; 
        \}
        .result{-table} th \{ background{-color}: \#f2f2f2; \}
        .grade{-A} \{ background{-color}: \#90EE90; \}
        .grade{-B} \{ background{-color}: \#87CEEB; \}
        .grade{-C} \{ background{-color}: \#F0E68C; \}
        .grade{-D} \{ background{-color}: \#FFA07A; \}
        .grade{-F} \{ background{-color}: \#FFB6C1; \}
    {/style}
{/head}
{body}
    {h2} style="text{-align: center;"}{ગ્રેડ રિપોર્ટ/h2}
    
    {\%}
        String studentName = request.getParameter("studentName");
        
        // ગુણ મેળવો
        int marks1 = Integer.parseInt(request.getParameter("marks1"));
        int marks2 = Integer.parseInt(request.getParameter("marks2"));
        int marks3 = Integer.parseInt(request.getParameter("marks3"));
        int marks4 = Integer.parseInt(request.getParameter("marks4"));
        int marks5 = Integer.parseInt(request.getParameter("marks5"));
        
        // કુલ અને ટકાવારી કેલ્ક્યુલેટ કરો
        int totalMarks = marks1 + marks2 + marks3 + marks4 + marks5;
        double percentage = totalMarks / 5.0;
        
        // ગ્રેડ નક્કી કરો
        String grade;
        String gradeClass;
        if(percentage {=} 90) \{
            grade = "A+";
            gradeClass = "grade{-A"};
        \} else if(percentage {=} 80) \{
            grade = "A";
            gradeClass = "grade{-A"};
        \} else if(percentage {=} 70) \{
            grade = "B";
            gradeClass = "grade{-B"};
        \} else if(percentage {=} 60) \{
            grade = "C";
            gradeClass = "grade{-C"};
        \} else if(percentage {=} 50) \{
            grade = "D";
            gradeClass = "grade{-D"};
        \} else \{
            grade = "F";
            gradeClass = "grade{-F"};
        \}
        
        // પરિણામ નક્કી કરો
        String result = percentage {=} 50 ? "પાસ" : "ફેલ";
    \%{}
    
    {table} class="result{-table"}{}
        {tr}
            {th} colspan="2"{વિદ્યાર્થીની માહિતી/th}
        {/tr}
        {tr}
            {tdstrongનામ:/strong/td}
            {td}{\%=} studentName \%{}{/td}
        {/tr}
        {tr}
            {th} colspan="2"{વિષય પ્રમાણે ગુણ/th}
        {/tr}
        {tr}
            {tdવિષય 1/td}
            {td}{\%=} marks1 \%{}{/td}
        {/tr}
        {tr}
            {tdવિષય 2/td}
            {td}{\%=} marks2 \%{}{/td}
        {/tr}
        {tr}
            {tdવિષય 3/td}
            {td}{\%=} marks3 \%{}{/td}
        {/tr}
        {tr}
            {tdવિષય 4/td}
            {td}{\%=} marks4 \%{}{/td}
        {/tr}
        {tr}
            {tdવિષય 5/td}
            {td}{\%=} marks5 \%{}{/td}
        {/tr}
        {tr}
            {th} colspan="2"{પરિણામ સારાંશ/th}
        {/tr}
        {tr}
            {tdstrongકુલ ગુણ:/strong/td}
            {td}{\%=} totalMarks \%{} / 500{/td}
        {/tr}
        {tr}
            {tdstrongટકાવારી:/strong/td}
            {td}{\%=} String.format("\%.2f", percentage) \%{}\%{/td}
        {/tr}
        {tr} class="{\%=} gradeClass \%{}"{}
            {tdstrongગ્રેડ:/strong/td}
            {td}{\%=} grade \%{}{/td}
        {/tr}
        {tr}
            {tdstrongપરિણામ:/strong/td}
            {td}{\%=} result \%{}{/td}
        {/tr}
    {/table}
    
    {div} style="text{-align: center; margin{-}top: 20px;"}{}
        {a} href="gradeInput.html"{બીજા ગ્રેડની ગણતરી કરો/a}
    {/div}
{/body}
{/html}
\end{verbatim}

\textbf{ગ્રેડ સ્કેલ કોષ્ટક:}

{\def\LTcaptype{none} % do not increment counter
\begin{longtable}[]{@{}lll@{}}
\toprule\noalign{}
ટકાવારી & ગ્રેડ & વર્ણન \\
\midrule\noalign{}
\endhead
\bottomrule\noalign{}
\endlastfoot
\textbf{90-100} & A+ & ઉત્કૃષ્ટ \\
\textbf{80-89} & A & ખૂબ સારું \\
\textbf{70-79} & B & સારું \\
\textbf{60-69} & C & સરેરાશ \\
\textbf{50-59} & D & સરેરાશથી નીચે \\
\textbf{0-49} & F & ફેલ \\
\end{longtable}
}

\end{solutionbox}
\begin{mnemonicbox}
``Calculate Total Percentage Grade Result''

\end{mnemonicbox}
\begin{center}\rule{0.5\linewidth}{0.5pt}\end{center}

\subsection*{પ્રશ્ન 5(અ) [3
ગુણ]}\label{uxaaauxab0uxab6uxaa8-5uxa85-3-uxa97uxaa3}

\textbf{Aspect-oriented programming (AOP) સમજાવો.}

\begin{solutionbox}

AOP એ programming paradigm છે જે cross-cutting concerns ને business logic
થી aspects નો ઉપયોગ કરીને અલગ કરે છે.


{\def\LTcaptype{none} % do not increment counter
\vspace{-5pt}
\captionof{table}{AOP મુખ્ય ખ્યાલો}
\vspace{-10pt}
\begin{longtable}[]{@{}ll@{}}
\toprule\noalign{}
ખ્યાલ & વર્ણન \\
\midrule\noalign{}
\endhead
\bottomrule\noalign{}
\endlastfoot
\textbf{Aspect} & cross-cutting concern ને encapsulate કરતું module \\
\textbf{Join Point} & program execution માં બિંદુ \\
\textbf{Pointcut} & join points નો સમૂહ \\
\textbf{Advice} & join point પર લેવાતી action \\
\textbf{Weaving} & aspects apply કરવાની પ્રક્રિયા \\
\end{longtable}
}

\textbf{મુખ્ય લાભો:}

\begin{itemize}
\tightlist
\item
  \textbf{વિભાજન}: business logic ને system services થી અલગ કરે છે
\item
  \textbf{મોડ્યુલારિટી}: કોડ modularity સુધારે છે
\item
  \textbf{પુનઃઉપયોગ}: cross-cutting concerns reusable છે
\item
  \textbf{જાળવણી}: maintain અને modify કરવું સરળ
\end{itemize}

\end{solutionbox}
\begin{mnemonicbox}
``Aspect Join Pointcut Advice Weaving''

\end{mnemonicbox}
\begin{center}\rule{0.5\linewidth}{0.5pt}\end{center}

\subsection*{પ્રશ્ન 5(બ) [4
ગુણ]}\label{uxaaauxab0uxab6uxaa8-5uxaac-4-uxa97uxaa3}

\textbf{Servlet ની વિવિધ વિશેષતાઓની યાદી બનાવો.}

\begin{solutionbox}


{\def\LTcaptype{none} % do not increment counter
\vspace{-5pt}
\captionof{table}{Servlet વિશેષતાઓ}
\vspace{-10pt}
\begin{longtable}[]{@{}ll@{}}
\toprule\noalign{}
વિશેષતા & વર્ણન \\
\midrule\noalign{}
\endhead
\bottomrule\noalign{}
\endlastfoot
\textbf{Platform Independent} & Java સપોર્ટ કરતા કોઈપણ server પર ચાલે છે \\
\textbf{Server Independent} & વિવિધ web servers સાથે કામ કરે છે \\
\textbf{Protocol Independent} & HTTP, HTTPS, FTP સપોર્ટ કરે છે \\
\textbf{Persistent} & requests ની વચ્ચે memory માં રહે છે \\
\textbf{Robust} & મજબૂત memory management \\
\textbf{Secure} & Built-in security features \\
\textbf{Portable} & એક વખત લખો, ગમે ત્યાં ચલાવો \\
\textbf{Powerful} & સંપૂર્ણ Java API access \\
\end{longtable}
}

\textbf{મુખ્ય મુદ્દાઓ:}

\begin{itemize}
\tightlist
\item
  \textbf{પર્ફોર્મન્સ}: CGI કરતાં વધુ સારું પર્ફોર્મન્સ
\item
  \textbf{Memory Management}: કાર્યક્ષમ memory ઉપયોગ
\item
  \textbf{Multithreading}: એકસાથે અનેક requests handle કરે છે
\item
  \textbf{Extensible}: ચોક્કસ protocols માટે extend કરી શકાય છે
\end{itemize}

\end{solutionbox}
\begin{mnemonicbox}
``Platform Server Protocol Persistent Robust''

\end{mnemonicbox}
\begin{center}\rule{0.5\linewidth}{0.5pt}\end{center}

\subsection*{પ્રશ્ન 5(ક) [7
ગુણ]}\label{uxaaauxab0uxab6uxaa8-5uxa95-7-uxa97uxaa3}

\textbf{Model layer, View layer અને Controller layer ને વિગતોમાં સમજાવો.}

\begin{solutionbox}

\textbf{MVC આર્કિટેક્ચર આકૃતિ:}

\begin{verbatim}
graph TB
    U[User] {-{-} C[Controller]}
    C {-{-} M[Model]}
    M {-{-} C}
    C {-{-} V[View]}
    V {-{-} U}
    
    subgraph "MVC Layers"
        M
        V
        C
    end
\end{verbatim}


{\def\LTcaptype{none} % do not increment counter
\vspace{-5pt}
\captionof{table}{MVC Layer વિગતો}
\vspace{-10pt}
\begin{longtable}[]{@{}
  >{\raggedright\arraybackslash}p{(\linewidth - 6\tabcolsep) * \real{0.2000}}
  >{\raggedright\arraybackslash}p{(\linewidth - 6\tabcolsep) * \real{0.2857}}
  >{\raggedright\arraybackslash}p{(\linewidth - 6\tabcolsep) * \real{0.3429}}
  >{\raggedright\arraybackslash}p{(\linewidth - 6\tabcolsep) * \real{0.1714}}@{}}
\toprule\noalign{}
\begin{minipage}[b]{\linewidth}\raggedright
Layer
\end{minipage} & \begin{minipage}[b]{\linewidth}\raggedright
જવાબદારી
\end{minipage} & \begin{minipage}[b]{\linewidth}\raggedright
Components
\end{minipage} & \begin{minipage}[b]{\linewidth}\raggedright
હેતુ
\end{minipage} \\
\midrule\noalign{}
\endhead
\bottomrule\noalign{}
\endlastfoot
\textbf{Model} & ડેટા અને business logic & Entities, DAOs, Services & ડેટા
management \\
\textbf{View} & Presentation layer & JSP, HTML, CSS & વપરાશકર્તા
interface \\
\textbf{Controller} & Request handling & Servlets, Actions & Flow
control \\
\end{longtable}
}

\textbf{Model Layer વિગતો:}

\begin{itemize}
\tightlist
\item
  \textbf{ડેટા Access}: ડેટાબેઝ operations અને data persistence
\item
  \textbf{Business Logic}: મુખ્ય application logic અને rules
\item
  \textbf{Validation}: ડેટા validation અને integrity checks
\item
  \textbf{Entity Classes}: ડેટા structures ને represent કરતા Java beans
\end{itemize}

\textbf{ઉદાહરણ Model:}

\begin{verbatim}
public class Student \{
    private String enrollNo;
    private String name;
    private double marks;
    
    // Business logic
    public String calculateGrade() \{
        if(marks {=} 90) return "A";
        else if(marks {=} 80) return "B";
        else if(marks {=} 70) return "C";
        else return "D";
    \}
\}
\end{verbatim}

\textbf{View Layer વિગતો:}

\begin{itemize}
\tightlist
\item
  \textbf{Presentation}: વપરાશકર્તા interface rendering
\item
  \textbf{Display Logic}: વપરાશકર્તાને ડેટા કેવી રીતે present કરવો
\item
  \textbf{User Interaction}: forms, buttons, navigation
\item
  \textbf{Responsive Design}: વિવિધ devices માટે adapt થાય છે
\end{itemize}

\textbf{Controller Layer વિગતો:}

\begin{itemize}
\tightlist
\item
  \textbf{Request Handling}: વપરાશકર્તાની requests process કરે છે
\item
  \textbf{Flow Control}: આગળનું કયું view display કરવું તે નક્કી કરે છે
\item
  \textbf{Model Coordination}: યોગ્ય model methods ને call કરે છે
\item
  \textbf{Response Generation}: વપરાશકર્તા માટે response તૈયાર કરે છે
\end{itemize}

\textbf{ઉદાહરણ Controller:}

\begin{verbatim}
@WebServlet("/student")
public class StudentController extends HttpServlet \{
    protected void doGet(HttpServletRequest request, 
                        HttpServletResponse response) \{
        String action = request.getParameter("action");
        
        if("view".equals(action)) \{
            // model માંથી ડેટા મેળવો
            Student student = studentService.getStudent(enrollNo);
            // request scope માં set કરો
            request.setAttribute("student", student);
            // view માં forward કરો
            RequestDispatcher rd = request.getRequestDispatcher("student.jsp");
            rd.forward(request, response);
        \}
    \}
\}
\end{verbatim}

\textbf{MVC ના ફાયદા:}

\begin{itemize}
\tightlist
\item
  \textbf{Separation of Concerns}: જવાબદારીનું સ્પષ્ટ વિભાજન
\item
  \textbf{Maintainability}: maintain અને modify કરવું સરળ
\item
  \textbf{Testability}: દરેક layer ને અલગ થી test કરી શકાય
\item
  \textbf{Scalability}: મોટા application development ને સપોર્ટ કરે છે
\item
  \textbf{Team Development}: અનેક developers એકસાથે કામ કરી શકે છે
\end{itemize}

\end{solutionbox}
\begin{mnemonicbox}
``Model Data View Present Controller Handle''

\end{mnemonicbox}
\begin{center}\rule{0.5\linewidth}{0.5pt}\end{center}

\subsection*{પ્રશ્ન 5(અ) અથવા [3
ગુણ]}\label{uxaaauxab0uxab6uxaa8-5uxa85-uxa85uxaa5uxab5-3-uxa97uxaa3}

\textbf{Spring Boot ની વિશેષતાઓ સમજાવો.}

\begin{solutionbox}


{\def\LTcaptype{none} % do not increment counter
\vspace{-5pt}
\captionof{table}{Spring Boot વિશેષતાઓ}
\vspace{-10pt}
\begin{longtable}[]{@{}ll@{}}
\toprule\noalign{}
વિશેષતા & વર્ણન \\
\midrule\noalign{}
\endhead
\bottomrule\noalign{}
\endlastfoot
\textbf{Auto Configuration} & dependencies આધારે આપોઆપ configuration \\
\textbf{Starter Dependencies} & curated dependencies નો સેટ \\
\textbf{Embedded Servers} & built-in Tomcat, Jetty servers \\
\textbf{Production Ready} & health checks, metrics, monitoring \\
\textbf{No XML Configuration} & annotation-based configuration \\
\textbf{Developer Tools} & hot reloading, automatic restart \\
\end{longtable}
}

\textbf{મુખ્ય લાભો:}

\begin{itemize}
\tightlist
\item
  \textbf{ઝડપી ડેવલપમેન્ટ}: ઝડપી project setup અને development
\item
  \textbf{Convention over Configuration}: sensible defaults
\item
  \textbf{Microservices Ready}: સરળ microservices development
\item
  \textbf{Cloud Native}: cloud deployment માટે તૈયાર
\end{itemize}

\end{solutionbox}
\begin{mnemonicbox}
``Auto Starter Embedded Production Annotation
Developer''

\end{mnemonicbox}
\begin{center}\rule{0.5\linewidth}{0.5pt}\end{center}

\subsection*{પ્રશ્ન 5(બ) અથવા [4
ગુણ]}\label{uxaaauxab0uxab6uxaa8-5uxaac-uxa85uxaa5uxab5-4-uxa97uxaa3}

\textbf{JSP scripting elements પર ટૂંકી નોંધ લખો.}

\begin{solutionbox}


{\def\LTcaptype{none} % do not increment counter
\vspace{-5pt}
\captionof{table}{JSP Scripting Elements}
\vspace{-10pt}
\begin{longtable}[]{@{}
  >{\raggedright\arraybackslash}p{(\linewidth - 6\tabcolsep) * \real{0.2812}}
  >{\raggedright\arraybackslash}p{(\linewidth - 6\tabcolsep) * \real{0.2500}}
  >{\raggedright\arraybackslash}p{(\linewidth - 6\tabcolsep) * \real{0.1875}}
  >{\raggedright\arraybackslash}p{(\linewidth - 6\tabcolsep) * \real{0.2812}}@{}}
\toprule\noalign{}
\begin{minipage}[b]{\linewidth}\raggedright
Element
\end{minipage} & \begin{minipage}[b]{\linewidth}\raggedright
Syntax
\end{minipage} & \begin{minipage}[b]{\linewidth}\raggedright
હેતુ
\end{minipage} & \begin{minipage}[b]{\linewidth}\raggedright
ઉદાહરણ
\end{minipage} \\
\midrule\noalign{}
\endhead
\bottomrule\noalign{}
\endlastfoot
\textbf{Scriptlet} & \texttt{\textless{}\%\ \%\textgreater{}} & Java
code execution &
\texttt{\textless{}\%\ int\ x\ =\ 10;\ \%\textgreater{}} \\
\textbf{Expression} & \texttt{\textless{}\%=\ \%\textgreater{}} & Output
value & \texttt{\textless{}\%=\ x\ +\ 5\ \%\textgreater{}} \\
\textbf{Declaration} & \texttt{\textless{}\%!\ \%\textgreater{}} &
Variable/method declaration &
\texttt{\textless{}\%!\ int\ count\ =\ 0;\ \%\textgreater{}} \\
\textbf{Directive} & \texttt{\textless{}\%@\ \%\textgreater{}} & Page
configuration &
\texttt{\textless{}\%@\ page\ import="java.util.*"\ \%\textgreater{}} \\
\textbf{Comment} & \texttt{\textless{}\%-\/-\ -\/-\%\textgreater{}} &
JSP comments &
\texttt{\textless{}\%-\/-\ આ\ comment\ છે\ -\/-\%\textgreater{}} \\
\end{longtable}
}

\textbf{ઉદાહરણો:}

\begin{verbatim}
{\%{-}{-} JSP Comment {-}{-}\%}
{\%@ page} contentType="text/html" \%{}

{\%!} 
    // Declaration {- instance variable}
    private int counter = 0;
    
    // Declaration {- method}
    public String getMessage() \{
        return "Hello JSP!";
    \}
\%{}

{html}
{body}
    {\%} 
        // Scriptlet {- Java code}
        String name = "Student";
        counter++;
    \%{}
    
    {h1}{\%=} getMessage() \%{}{/h1}
    {pસ્વાગત }{\%=} name \%{}!{/p}
    {pપેજ }{\%=} counter \%{} વખત visit કર્યું{/p}
{/body}
{/html}
\end{verbatim}

\textbf{મુખ્ય મુદ્દાઓ:}

\begin{itemize}
\tightlist
\item
  \textbf{Scriptlet}: Java statements ધરાવે છે
\item
  \textbf{Expression}: result evaluate કરે અને output આપે છે
\item
  \textbf{Declaration}: instance variables/methods બનાવે છે
\item
  \textbf{Directive}: page-level માહિતી પ્રદાન કરે છે
\end{itemize}

\end{solutionbox}
\begin{mnemonicbox}
``Script Express Declare Direct Comment''

\end{mnemonicbox}
\begin{center}\rule{0.5\linewidth}{0.5pt}\end{center}

\subsection*{પ્રશ્ન 5(ક) અથવા [7
ગુણ]}\label{uxaaauxab0uxab6uxaa8-5uxa95-uxa85uxaa5uxab5-7-uxa97uxaa3}

\textbf{Dependency injection (DI) અને Plain Old Java Object (POJO) ને
વિગતોમાં સમજાવો.}

\begin{solutionbox}

\textbf{Dependency Injection (DI):}

Dependency Injection એ design pattern છે જ્યાં objects તેમની dependencies
external source માંથી receive કરે છે internal creation કરવાને બદલે.


{\def\LTcaptype{none} % do not increment counter
\vspace{-5pt}
\captionof{table}{DI પ્રકારો}
\vspace{-10pt}
\begin{longtable}[]{@{}
  >{\raggedright\arraybackslash}p{(\linewidth - 4\tabcolsep) * \real{0.2609}}
  >{\raggedright\arraybackslash}p{(\linewidth - 4\tabcolsep) * \real{0.3478}}
  >{\raggedright\arraybackslash}p{(\linewidth - 4\tabcolsep) * \real{0.3913}}@{}}
\toprule\noalign{}
\begin{minipage}[b]{\linewidth}\raggedright
પ્રકાર
\end{minipage} & \begin{minipage}[b]{\linewidth}\raggedright
વર્ણન
\end{minipage} & \begin{minipage}[b]{\linewidth}\raggedright
ઉદાહરણ
\end{minipage} \\
\midrule\noalign{}
\endhead
\bottomrule\noalign{}
\endlastfoot
\textbf{Constructor Injection} & constructor દ્વારા dependencies &
\texttt{public\ Service(Repository\ repo)} \\
\textbf{Setter Injection} & setter methods દ્વારા dependencies &
\texttt{setRepository(Repository\ repo)} \\
\textbf{Field Injection} & સીધું field injection &
\texttt{@Autowired\ Repository\ repo} \\
\end{longtable}
}

\textbf{DI ઉદાહરણ:}

\begin{verbatim}
// DI વિના {- Tight coupling}
public class StudentService \{
    private StudentRepository repo = new StudentRepository(); // Hard dependency
    
    public Student getStudent(String id) \{
        return repo.findById(id);
    \}
\}

// DI સાથે {- Loose coupling}
public class StudentService \{
    private StudentRepository repo;
    
    // Constructor injection
    public StudentService(StudentRepository repo) \{
        this.repo = repo;
    \}
    
    public Student getStudent(String id) \{
        return repo.findById(id);
    \}
\}
\end{verbatim}

\textbf{Spring DI Configuration:}

\begin{verbatim}
@Service
public class StudentService \{
    @Autowired
    private StudentRepository repository;
    
    public List{}Student{} getAllStudents() \{
        return repository.findAll();
    \}
\}

@Repository
public class StudentRepository \{
    public List{}Student{} findAll() \{
        // Database operations
        return studentList;
    \}
\}
\end{verbatim}

\textbf{Plain Old Java Object (POJO):}

POJO એ સરળ Java object છે જે કોઈ ચોક્કસ framework classes માંથી inherit કરતું
નથી અથવા ચોક્કસ interfaces implement કરતું નથી.

\textbf{POJO લાક્ષણિકતાઓ:}

\begin{itemize}
\tightlist
\item
  \textbf{કોઈ inheritance નથી}: framework classes માંથી extend કરતું નથી
\item
  \textbf{કોઈ interfaces નથી}: framework interfaces implement કરતું નથી
\item
  \textbf{કોઈ annotations નથી}: framework annotations વિના કામ કરી શકે છે
\item
  \textbf{સરળ}: માત્ર business logic અને ડેટા ધરાવે છે
\end{itemize}

\textbf{POJO ઉદાહરણ:}

\begin{verbatim}
// આ એક POJO છે
public class Student \{
    private String enrollNo;
    private String name;
    private int age;
    private String course;
    
    // Default constructor
    public Student() \{\}
    
    // Parameterized constructor
    public Student(String enrollNo, String name, int age, String course) \{
        this.enrollNo = enrollNo;
        this.name = name;
        this.age = age;
        this.course = course;
    \}
    
    // Getters અને Setters
    public String getEnrollNo() \{
        return enrollNo;
    \}
    
    public void setEnrollNo(String enrollNo) \{
        this.enrollNo = enrollNo;
    \}
    
    public String getName() \{
        return name;
    \}
    
    public void setName(String name) \{
        this.name = name;
    \}
    
    // Business methods
    public boolean isEligibleForExam() \{
        return age {=} 18;
    \}
    
    public String getStudentInfo() \{
        return "Student: " + name + " (" + enrollNo + "), Course: " + course;
    \}
\}
\end{verbatim}

\textbf{DI ના ફાયદા:}

\begin{itemize}
\tightlist
\item
  \textbf{Loose Coupling}: classes વચ્ચે dependencies ઘટાડે છે
\item
  \textbf{Testability}: testing માટે mock objects inject કરવું સરળ
\item
  \textbf{Flexibility}: implementations બદલવું સરળ
\item
  \textbf{Maintainability}: કોડ maintain અને extend કરવું સરળ
\end{itemize}

\textbf{POJO ના ફાયદા:}

\begin{itemize}
\tightlist
\item
  \textbf{સરળતા}: સમજવું અને maintain કરવું સરળ
\item
  \textbf{Testability}: unit test કરવું સરળ
\item
  \textbf{Portability}: વિવિધ frameworks માં ઉપયોગ કરી શકાય
\item
  \textbf{Lightweight}: કોઈ framework overhead નથી
\end{itemize}

\textbf{DI અને POJO એકસાથે:}

\begin{verbatim}
// POJO Entity
public class Student \{
    private String name;
    private String email;
    // constructors, getters, setters
\}

// Service with DI
@Service
public class StudentService \{
    @Autowired
    private StudentRepository repository;
    
    public Student createStudent(String name, String email) \{
        Student student = new Student(); // POJO creation
        student.setName(name);
        student.setEmail(email);
        return repository.save(student);
    \}
\}
\end{verbatim}

\end{solutionbox}
\begin{mnemonicbox}
``DI Injects Dependencies, POJO Plain Objects''

\end{mnemonicbox}

\end{document}
