\documentclass{article}

% content/resources/templates/preamble.tex
\usepackage[margin=0.6in]{geometry}
\author{Milav Dabgar}
\usepackage{amsmath,amssymb,amsthm}
\usepackage{booktabs}
\usepackage{multirow}
\usepackage{xcolor}
\usepackage{tcolorbox}
\tcbuselibrary{breakable,skins}
\usepackage[colorlinks=true,linkcolor=blue]{hyperref}
\usepackage{titlesec}
\usepackage{enumitem}
\usepackage{tikz}
\usepackage{pgfplots}
\usepackage{circuitikz}
\usepackage[version=4]{mhchem}
\usepackage{longtable}
\usepackage{array}
\usepackage{float}
\usepackage{caption}
\usepackage{listings}

\lstset{
  basicstyle=\small\ttfamily,
  breaklines=true,
  breakatwhitespace=false,
  postbreak=\mbox{\textcolor{red}{$\hookrightarrow$}\space},
  float=false,
  numbers=left,
  numberstyle=\tiny\color{gray},
  numbersep=10pt,
  xleftmargin=2em,
  keywordstyle=\color{blue},
  commentstyle=\color{green!60!black},
  stringstyle=\color{purple},
  backgroundcolor=\color{gray!5},
  showstringspaces=false,
  tabsize=2,
  captionpos=b,
  keepspaces=true,
  columns=flexible
}

\pgfplotsset{compat=1.18}
\usetikzlibrary{shapes,arrows,positioning,calc,patterns,decorations.pathmorphing,decorations.markings,arrows.meta}

% Color scheme
\definecolor{headcolor}{RGB}{0,102,204}
\definecolor{keycolor}{RGB}{220,20,60}
\definecolor{solutioncolor}{RGB}{34,139,34}
\definecolor{mnemoniccolor}{RGB}{148,0,211}
\definecolor{codecolor}{RGB}{0,0,100}

% Spacing
\setlength{\parskip}{3pt}
\setlist[itemize]{nosep}
\setlist[enumerate]{nosep}

% Title formatting
\titleformat{\section}{\Large\bfseries\color{headcolor}}{\thesection}{1em}{}
\titleformat{\subsection}{\large\bfseries\color{headcolor}}{\thesubsection}{1em}{}

% Pandoc tightlist compatibility
\providecommand{\tightlist}{%
  \setlength{\itemsep}{0pt}\setlength{\parskip}{0pt}}

% Pandoc longtable compatibility
\newcounter{none}
\def\thenone{}


% content/resources/templates/english-boxes.tex

% Custom environments
\newtcolorbox{solutionbox}{
 breakable,
 enhanced,
 colback=solutioncolor!5!white,
 colframe=solutioncolor!75!black,
 fonttitle=\bfseries,
 title=Solution
}

\newtcolorbox{solutionboxnobreak}{
 colback=solutioncolor!5!white,
 colframe=solutioncolor!75!black,
 fonttitle=\bfseries,
 title=Solution
}

\newtcolorbox{keyformula}{
 breakable,
 enhanced,
 colback=keycolor!5!white,
 colframe=keycolor!75!black,
 fonttitle=\bfseries,
 title=Key Formula
}

\newtcolorbox{mnemonicboxenv}{
 breakable,
 enhanced,
 colback=mnemoniccolor!5!white,
 colframe=mnemoniccolor!75!black,
 fonttitle=\bfseries,
 title=Mnemonic
}

\newcommand{\mnemonicbox}[1]{%
  \begin{mnemonicboxenv}
    #1
  \end{mnemonicboxenv}
}


% Custom commands for GTU solutions
% This file defines semantic commands for consistent formatting

% Question command with automatic formatting
\newcommand{\question}[2]{%
  \section*{Question #1}%
  \textbf{#2}%
}

% OR question variant
\newcommand{\questionor}[2]{%
  \section*{Question #1 OR}%
  \textbf{#2}%
}

% Proper table environment with caption
\newenvironment{answertable}[1]{%
  \begin{table}[htbp]
  \centering
  \caption{#1}
}{%
  \end{table}
}

% Proper figure environment for diagrams
\newenvironment{answerdiagram}[1]{%
  \begin{figure}[htbp]
  \centering
  \caption{#1}
}{%
  \end{figure}
}

% Semantic markup for key terms
\newcommand{\keyword}[1]{\textbf{#1}}
\newcommand{\code}[1]{\texttt{#1}}
\newcommand{\classname}[1]{\texttt{#1}}
\newcommand{\methodname}[1]{\texttt{#1}}

% Proper quotation marks
\newcommand{\mnemonic}[1]{``#1''}


\title{Advanced Java Programming (4351603) - Summer 2025 Solution}
\date{May 16, 2025}

\begin{document}
\maketitle

\questionmarks{1(a)}{3}{Write a difference between AWT and Swing.}

\begin{solutionbox}
\begin{center}
\captionof{table}{AWT vs Swing Comparison}
\begin{tabulary}{\linewidth}{|L|L|L|}
\hline
\textbf{Feature} & \textbf{AWT} & \textbf{Swing} \\ \hline
\textbf{Platform} & Platform dependent & Platform independent \\ \hline
\textbf{Components} & Heavy weight & Light weight \\ \hline
\textbf{Look and Feel} & Native OS look & Pluggable look and feel \\ \hline
\textbf{Performance} & Faster & Slower than AWT \\ \hline
\end{tabulary}
\end{center}

\begin{itemize}
    \item \textbf{AWT}: Uses native OS components.
    \item \textbf{Swing}: Uses Java's own components.
    \item \textbf{Pluggability}: Swing supports customizable UI.
\end{itemize}
\end{solutionbox}

\begin{mnemonicbox}
\mnemonic{Swing is Smart - Platform Independent and Pluggable}
\end{mnemonicbox}

\questionmarks{1(b)}{4}{List out various Layout Managers. Explain Flow Layout manager with example.}

\begin{solutionbox}
\textbf{Layout Managers List:}
\begin{itemize}
    \item \keyword{FlowLayout}: Left to right arrangement.
    \item \keyword{BorderLayout}: North, South, East, West, Center.
    \item \keyword{GridLayout}: Equal sized grid cells.
    \item \keyword{CardLayout}: Stack of components.
    \item \keyword{BoxLayout}: Single row or column.
\end{itemize}

\textbf{FlowLayout Example:}
\begin{lstlisting}[language=Java]
import javax.swing.*;
import java.awt.*;

public class FlowExample extends JFrame {
    public FlowExample() {
        setLayout(new FlowLayout());
        add(new JButton("Button 1"));
        add(new JButton("Button 2"));
        add(new JButton("Button 3"));
        setSize(300, 100);
        setVisible(true);
    }
}
\end{lstlisting}
\end{solutionbox}

\begin{mnemonicbox}
\mnemonic{Flow Like Water - Left to Right}
\end{mnemonicbox}

\questionmarks{1(c)}{7}{Create a Swing program using checkbox that allows users to select multiple items from a list of options. Display the selected items.}

\begin{solutionbox}
\begin{lstlisting}[language=Java]
import javax.swing.*;
import java.awt.*;
import java.awt.event.*;

public class CheckboxExample extends JFrame implements ItemListener {
    JCheckBox java, python, cpp;
    JTextArea display;
    
    public CheckboxExample() {
        setLayout(new FlowLayout());
        
        java = new JCheckBox("Java");
        python = new JCheckBox("Python");
        cpp = new JCheckBox("C++");
        
        java.addItemListener(this);
        python.addItemListener(this);
        cpp.addItemListener(this);
        
        display = new JTextArea(5, 20);
        
        add(java);
        add(python);
        add(cpp);
        add(new JScrollPane(display));
        
        setSize(300, 200);
        setDefaultCloseOperation(JFrame.EXIT_ON_CLOSE);
        setVisible(true);
    }
    
    public void itemStateChanged(ItemEvent e) {
        String result = "Selected: ";
        if(java.isSelected()) result += "Java ";
        if(python.isSelected()) result += "Python ";
        if(cpp.isSelected()) result += "C++ ";
        display.setText(result);
    }
    
    public static void main(String[] args) {
        new CheckboxExample();
    }
}
\end{lstlisting}

\textbf{Key Features:}
\begin{itemize}
    \item \keyword{Multiple Selection}: Users can select multiple checkboxes.
    \item \keyword{Real-time Display}: Shows selected items immediately.
    \item \keyword{ItemListener}: Handles checkbox state changes.
\end{itemize}
\end{solutionbox}

\begin{mnemonicbox}
\mnemonic{Check Multiple, Display All}
\end{mnemonicbox}

\questionmarks{1(c OR)}{7}{Develop a Java program using various swing components.}

\begin{solutionbox}
\begin{lstlisting}[language=Java]
import javax.swing.*;
import java.awt.*;
import java.awt.event.*;

public class SwingComponents extends JFrame implements ActionListener {
    JTextField nameField;
    JComboBox<String> cityCombo;
    JRadioButton male, female;
    JButton submit;
    JTextArea display;
    
    public SwingComponents() {
        setLayout(new FlowLayout());
        
        add(new JLabel("Name:"));
        nameField = new JTextField(15);
        add(nameField);
        
        add(new JLabel("City:"));
        cityCombo = new JComboBox<>(new String[]{"Mumbai", "Delhi", "Bangalore"});
        add(cityCombo);
        
        ButtonGroup gender = new ButtonGroup();
        male = new JRadioButton("Male");
        female = new JRadioButton("Female");
        gender.add(male);
        gender.add(female);
        add(male);
        add(female);
        
        submit = new JButton("Submit");
        submit.addActionListener(this);
        add(submit);
        
        display = new JTextArea(5, 25);
        add(new JScrollPane(display));
        
        setSize(400, 300);
        setDefaultCloseOperation(JFrame.EXIT_ON_CLOSE);
        setVisible(true);
    }
    
    public void actionPerformed(ActionEvent e) {
        String name = nameField.getText();
        String city = (String)cityCombo.getSelectedItem();
        String gender = male.isSelected() ? "Male" : "Female";
        
        display.setText("Name: " + name + "\nCity: " + city + "\nGender: " + gender);
    }
    
    public static void main(String[] args) {
        new SwingComponents();
    }
}
\end{lstlisting}

\textbf{Components Used:}
\begin{itemize}
    \item \keyword{JTextField}: Text input.
    \item \keyword{JComboBox}: Dropdown selection.
    \item \keyword{JRadioButton}: Single selection.
    \item \keyword{JButton}: Action trigger.
\end{itemize}
\end{solutionbox}

\begin{mnemonicbox}
\mnemonic{Text, Combo, Radio, Button - Complete Form}
\end{mnemonicbox}

\questionmarks{2(a)}{3}{Explain Swing controls with example.}

\begin{solutionbox}
\begin{center}
\captionof{table}{Common Swing Controls}
\begin{tabulary}{\linewidth}{|L|L|L|}
\hline
\textbf{Control} & \textbf{Purpose} & \textbf{Example} \\ \hline
\textbf{JButton} & Click actions & \code{new JButton("Click Me")} \\ \hline
\textbf{JTextField} & Text input & \code{new JTextField(10)} \\ \hline
\textbf{JLabel} & Display text & \code{new JLabel("Hello")} \\ \hline
\textbf{JCheckBox} & Multiple selection & \code{new JCheckBox("Option")} \\ \hline
\end{tabulary}
\end{center}

\textbf{Basic Example:}
\begin{lstlisting}[language=Java]
JFrame frame = new JFrame();
JButton btn = new JButton("Submit");
frame.add(btn);
frame.setSize(200, 100);
frame.setVisible(true);
\end{lstlisting}
\end{solutionbox}

\begin{mnemonicbox}
\mnemonic{Button, Text, Label, Check - Basic Four}
\end{mnemonicbox}

\questionmarks{2(b)}{4}{List JDBC drivers and explain any two.}

\begin{solutionbox}
\textbf{JDBC Drivers List:}
\begin{enumerate}
    \item \textbf{Type 1}: JDBC-ODBC Bridge
    \item \textbf{Type 2}: Native API Driver
    \item \textbf{Type 3}: Network Protocol Driver
    \item \textbf{Type 4}: Thin Driver
\end{enumerate}

\textbf{Detailed Explanation:}

\textbf{Type 1 - JDBC-ODBC Bridge:}
\begin{itemize}
    \item \keyword{Purpose}: Converts JDBC calls to ODBC calls.
    \item \keyword{Advantage}: Works with any ODBC database.
    \item \keyword{Disadvantage}: Platform dependent, slower performance.
\end{itemize}

\textbf{Type 4 - Thin Driver:}
\begin{itemize}
    \item \keyword{Purpose}: Pure Java driver, direct database communication.
    \item \keyword{Advantage}: Platform independent, best performance.
    \item \keyword{Disadvantage}: Database specific.
\end{itemize}
\end{solutionbox}

\begin{mnemonicbox}
\mnemonic{Bridge-Native-Network-Thin: 1-2-3-4}
\end{mnemonicbox}

\questionmarks{2(c)}{7}{Explain Object Relational Mapping (ORM) with its advantages and tools.}

\begin{solutionbox}
\textbf{Object Relational Mapping (ORM):}
ORM is a technique that maps object-oriented programming concepts to relational database structures.

\begin{center}
\begin{tikzpicture}[node distance=2cm, auto]
    \node [gtu block] (Obj) {Java Object};
    \node [gtu block, right=of Obj] (ORM) {ORM Framework};
    \node [gtu block, right=of ORM] (DB) {Database Table};
    
    \path [gtu arrow, <->] (Obj) -- (ORM);
    \path [gtu arrow, <->] (ORM) -- (DB);
\end{tikzpicture}
\captionof{figure}{ORM Concept}
\end{center}

\begin{center}
\captionof{table}{ORM Advantages}
\begin{tabulary}{\linewidth}{|L|L|}
\hline
\textbf{Advantage} & \textbf{Description} \\ \hline
\textbf{Productivity} & Reduces coding time \\ \hline
\textbf{Maintainability} & Easy to modify and update \\ \hline
\textbf{Database Independence} & Switch databases easily \\ \hline
\textbf{Object-Oriented} & Works with OOP concepts \\ \hline
\end{tabulary}
\end{center}

\textbf{Popular ORM Tools:}
\begin{itemize}
    \item \keyword{Hibernate}: Most popular Java ORM.
    \item \keyword{JPA}: Java Persistence API standard.
    \item \keyword{MyBatis}: SQL mapping framework.
    \item \keyword{EclipseLink}: Reference implementation.
\end{itemize}

\textbf{Working Model:}
\begin{itemize}
    \item \textbf{Objects} $\rightarrow$ \textbf{ORM} $\rightarrow$ \textbf{Tables}
    \item Automatic SQL generation
    \item Type-safe queries
\end{itemize}
\end{solutionbox}

\begin{mnemonicbox}
\mnemonic{Objects Relate Magically}
\end{mnemonicbox}

\questionmarks{2(a OR)}{3}{Describe MOUSEEVENT and MOUSELISTENER interface with example.}

\begin{solutionbox}
\textbf{MouseEvent:}
Generated when mouse actions occur on components.

\textbf{MouseListener Interface Methods:}
\begin{itemize}
    \item \keyword{mouseClicked()}: Mouse button clicked.
    \item \keyword{mousePressed()}: Mouse button pressed.
    \item \keyword{mouseReleased()}: Mouse button released.
    \item \keyword{mouseEntered()}: Mouse enters component.
    \item \keyword{mouseExited()}: Mouse exits component.
\end{itemize}

\textbf{Example:}
\begin{lstlisting}[language=Java]
public class MouseExample extends JFrame implements MouseListener {
    JLabel label;
    
    public MouseExample() {
        label = new JLabel("Click me!");
        label.addMouseListener(this);
        add(label);
        setSize(200, 100);
        setVisible(true);
    }
    
    public void mouseClicked(MouseEvent e) {
        label.setText("Clicked!");
    }
    
    // Other methods...
}
\end{lstlisting}
\end{solutionbox}

\begin{mnemonicbox}
\mnemonic{Click-Press-Release-Enter-Exit}
\end{mnemonicbox}

\questionmarks{2(b OR)}{4}{List and explain the components of the JDBC API.}

\begin{solutionbox}
\begin{center}
\captionof{table}{JDBC API Components}
\begin{tabulary}{\linewidth}{|L|L|L|}
\hline
\textbf{Component} & \textbf{Purpose} & \textbf{Key Classes} \\ \hline
\textbf{DriverManager} & Manages drivers & \code{DriverManager.getConnection()} \\ \hline
\textbf{Connection} & Database connection & \code{Connection conn} \\ \hline
\textbf{Statement} & SQL execution & \code{Statement stmt} \\ \hline
\textbf{ResultSet} & Query results & \code{ResultSet rs} \\ \hline
\end{tabulary}
\end{center}

\textbf{Component Details:}
\begin{itemize}
    \item \keyword{DriverManager}: Establishes connection with database.
    \item \keyword{Connection}: Represents database session.
    \item \keyword{Statement}: Executes SQL queries.
    \item \keyword{ResultSet}: Holds query results.
\end{itemize}

\textbf{Basic Usage:}
\begin{lstlisting}[language=Java]
Connection conn = DriverManager.getConnection(url, user, pass);
Statement stmt = conn.createStatement();
ResultSet rs = stmt.executeQuery("SELECT * FROM users");
\end{lstlisting}
\end{solutionbox}

\begin{mnemonicbox}
\mnemonic{Driver Connects, Statement Executes, ResultSet Returns}
\end{mnemonicbox}

\questionmarks{2(c OR)}{7}{Draw and explain the architecture of Hibernate.}

\begin{solutionbox}
\begin{center}
\begin{tikzpicture}[node distance=1.5cm, auto]
    \node [gtu block] (App) {Java Application};
    \node [gtu block, below=of App] (API) {Hibernate API};
    \node [gtu block, below left=of API] (Config) {Configuration};
    \node [gtu block, below right=of API] (Factory) {SessionFactory};
    
    \node [gtu block, below=of Factory] (Session) {Session};
    \node [gtu block, below left=of Session] (Trans) {Transaction};
    \node [gtu block, below=of Session] (Query) {Query};
    \node [gtu block, below right=of Session] (Crit) {Criteria};
    
    \node [gtu block, below=of Query] (DB) {Database};
    
    \node [left=of Config] (CfgFile) {hibernate.cfg.xml};
    \node [right=of Factory] (MapFile) {Mapping Files};
    
    \path [gtu arrow] (App) -- (API);
    \path [gtu arrow] (API) -- (Config);
    \path [gtu arrow] (API) -- (Factory);
    \path [gtu arrow] (Factory) -- (Session);
    \path [gtu arrow] (Session) -- (Trans);
    \path [gtu arrow] (Session) -- (Query);
    \path [gtu arrow] (Session) -- (Crit);
    \path [gtu arrow] (CfgFile) -- (Config);
    \path [gtu arrow] (MapFile) -- (Factory);
    \path [gtu arrow] (Session) -- (DB);
\end{tikzpicture}
\captionof{figure}{Hibernate Architecture}
\end{center}

\textbf{Table: Hibernate Architecture Components}
\begin{center}
\begin{tabulary}{\linewidth}{|L|L|}
\hline
\textbf{Component} & \textbf{Function} \\ \hline
\textbf{Configuration} & Reads config files \\ \hline
\textbf{SessionFactory} & Creates Session objects \\ \hline
\textbf{Session} & Interface to database \\ \hline
\textbf{Transaction} & Manages transactions \\ \hline
\textbf{Query} & HQL/SQL queries \\ \hline
\end{tabulary}
\end{center}

\textbf{Layer Description:}
\begin{itemize}
    \item \keyword{Application Layer}: Java objects and business logic.
    \item \keyword{Hibernate Layer}: ORM mapping and session management.
    \item \keyword{Database Layer}: Actual data storage.
\end{itemize}

\textbf{Key Features:}
\begin{itemize}
    \item \keyword{Automatic table creation}: Based on entity classes.
    \item \keyword{HQL support}: Object-oriented query language.
    \item \keyword{Caching}: First and second level caching.
\end{itemize}
\end{solutionbox}

\begin{mnemonicbox}
\mnemonic{Config-Factory-Session-Transaction: CFST}
\end{mnemonicbox}

\questionmarks{3(a)}{3}{Describe various features of Servlet.}

\begin{solutionbox}
\begin{center}
\captionof{table}{Servlet Features}
\begin{tabulary}{\linewidth}{|L|L|}
\hline
\textbf{Feature} & \textbf{Description} \\ \hline
\textbf{Platform Independent} & Runs on any OS with JVM \\ \hline
\textbf{Performance} & Better than CGI \\ \hline
\textbf{Robust} & JVM managed memory \\ \hline
\textbf{Secure} & Java security features \\ \hline
\end{tabulary}
\end{center}

\textbf{Key Features:}
\begin{itemize}
    \item \keyword{Server-side processing}: Handles client requests.
    \item \keyword{Protocol independent}: HTTP, FTP, SMTP support.
    \item \keyword{Extensible}: Can be extended easily.
    \item \keyword{Portable}: Write once, run anywhere.
\end{itemize}
\end{solutionbox}

\begin{mnemonicbox}
\mnemonic{Platform Performance Robust Secure}
\end{mnemonicbox}

\questionmarks{3(b)}{4}{Explain Servlet life cycle.}

\begin{solutionbox}
\begin{center}
\begin{tikzpicture}[node distance=1.5cm, auto]
    \node [gtu state] (Load) {Loading};
    \node [gtu state, right=of Load] (Inst) {Instantiation};
    \node [gtu state, right=of Inst] (Init) {init()};
    \node [gtu state, below=of Init] (Serv) {service()};
    \node [gtu state, left=of Serv] (Dest) {destroy()};
    \node [gtu state, left=of Dest] (Unload) {Unloaded};
    
    \path [gtu arrow] (Load) -- (Inst);
    \path [gtu arrow] (Inst) -- (Init);
    \path [gtu arrow] (Init) -- (Serv);
    \path [gtu arrow] (Serv) edge [loop below] node {Requests} (Serv);
    \path [gtu arrow] (Serv) -- (Dest);
    \path [gtu arrow] (Dest) -- (Unload);
\end{tikzpicture}
\captionof{figure}{Servlet Life Cycle}
\end{center}

\textbf{Table: Servlet Life Cycle Stages}
\begin{center}
\begin{tabulary}{\linewidth}{|L|L|L|}
\hline
\textbf{Stage} & \textbf{Method} & \textbf{Purpose} \\ \hline
\textbf{Loading} & Class loading & JVM loads servlet class \\ \hline
\textbf{Instantiation} & Constructor & Creates servlet object \\ \hline
\textbf{Initialization} & \code{init()} & One-time setup \\ \hline
\textbf{Request Processing} & \code{service()} & Handles requests \\ \hline
\textbf{Destruction} & \code{destroy()} & Cleanup resources \\ \hline
\end{tabulary}
\end{center}

\textbf{Method Details:}
\begin{itemize}
    \item \keyword{init()}: Called once when servlet loads.
    \item \keyword{service()}: Called for each request.
    \item \keyword{destroy()}: Called when servlet unloads.
\end{itemize}
\end{solutionbox}

\begin{mnemonicbox}
\mnemonic{Load-Create-Init-Service-Destroy}
\end{mnemonicbox}

\questionmarks{3(c)}{7}{Explain the session tracking in Servlet with example.}

\begin{solutionbox}
\textbf{Session Tracking Methods:}

\begin{center}
\captionof{table}{Session Tracking Techniques}
\begin{tabulary}{\linewidth}{|L|L|L|}
\hline
\textbf{Method} & \textbf{Description} & \textbf{Pros/Cons} \\ \hline
\textbf{Cookies} & Client-side storage & Easy/Privacy issues \\ \hline
\textbf{URL Rewriting} & Append session ID & Universal/Ugly URLs \\ \hline
\textbf{Hidden Fields} & Form-based tracking & Simple/Form dependent \\ \hline
\textbf{HttpSession} & Server-side object & Secure/Memory usage \\ \hline
\end{tabulary}
\end{center}

\textbf{HttpSession Example:}
\begin{lstlisting}[language=Java]
protected void doGet(HttpServletRequest request, 
                    HttpServletResponse response) {
    HttpSession session = request.getSession();
    
    // Store data
    session.setAttribute("username", "john");
    
    // Retrieve data
    String user = (String) session.getAttribute("username");
    
    // Session info
    String sessionId = session.getId();
    boolean isNew = session.isNew();
    
    PrintWriter out = response.getWriter();
    out.println("User: " + user);
    out.println("Session ID: " + sessionId);
}
\end{lstlisting}

\textbf{Session Management:}
\begin{itemize}
    \item \keyword{Creation}: \code{request.getSession()}
    \item \keyword{Storage}: \code{session.setAttribute()}
    \item \keyword{Retrieval}: \code{session.getAttribute()}
    \item \keyword{Invalidation}: \code{session.invalidate()}
\end{itemize}
\end{solutionbox}

\begin{mnemonicbox}
\mnemonic{Cookies-URLs-Hidden-HttpSession: CUHS}
\end{mnemonicbox}

\questionmarks{3(a OR)}{3}{Explain methods of Servlet life cycle.}

\begin{solutionbox}
\textbf{Life Cycle Methods:}

\begin{center}
\captionof{table}{Servlet Life Cycle Methods}
\begin{tabulary}{\linewidth}{|L|L|L|}
\hline
\textbf{Method} & \textbf{Called When} & \textbf{Parameters} \\ \hline
\textbf{init()} & Servlet initialization & \code{ServletConfig config} \\ \hline
\textbf{service()} & Each request & \code{ServletRequest req, ServletResponse res} \\ \hline
\textbf{destroy()} & Servlet cleanup & None \\ \hline
\end{tabulary}
\end{center}

\textbf{Method Details:}
\begin{itemize}
    \item \keyword{init(ServletConfig config)}: Initialization code, database connections.
    \item \keyword{service(req, res)}: Request handling, business logic.
    \item \keyword{destroy()}: Cleanup code, close resources.
\end{itemize}

\textbf{Example:}
\begin{lstlisting}[language=Java]
public void init(ServletConfig config) {
    // Initialize database connection
}

public void service(ServletRequest req, ServletResponse res) {
    // Handle request
}

public void destroy() {
    // Close connections
}
\end{lstlisting}
\end{solutionbox}

\begin{mnemonicbox}
\mnemonic{Init-Service-Destroy: ISD}
\end{mnemonicbox}

\questionmarks{3(b OR)}{4}{Describe HTTPSERVLET class with example.}

\begin{solutionbox}
\textbf{HttpServlet Class:}
Abstract class extending GenericServlet, specifically for HTTP protocol.

\textbf{HTTP Methods:}

\begin{center}
\captionof{table}{HttpServlet Methods}
\begin{tabulary}{\linewidth}{|L|L|L|}
\hline
\textbf{Method} & \textbf{HTTP Verb} & \textbf{Purpose} \\ \hline
\textbf{doGet()} & GET & Retrieve data \\ \hline
\textbf{doPost()} & POST & Submit data \\ \hline
\textbf{doPut()} & PUT & Update data \\ \hline
\textbf{doDelete()} & DELETE & Remove data \\ \hline
\end{tabulary}
\end{center}

\textbf{Example:}
\begin{lstlisting}[language=Java]
public class MyServlet extends HttpServlet {
    protected void doGet(HttpServletRequest request,
                        HttpServletResponse response) {
        response.setContentType("text/html");
        PrintWriter out = response.getWriter();
        out.println("<h1>GET Request</h1>");
    }
    
    protected void doPost(HttpServletRequest request,
                         HttpServletResponse response) {
        String name = request.getParameter("name");
        response.getWriter().println("Hello " + name);
    }
}
\end{lstlisting}

\textbf{Key Features:}
\begin{itemize}
    \item \keyword{HTTP-specific}: Designed for web applications.
    \item \keyword{Method handling}: Separate methods for different HTTP verbs.
    \item \keyword{Request/Response}: HttpServletRequest and HttpServletResponse.
\end{itemize}
\end{solutionbox}

\begin{mnemonicbox}
\mnemonic{Get-Post-Put-Delete: GPPD}
\end{mnemonicbox}

\questionmarks{3(c OR)}{7}{Differentiate GET and POST methods and write a java code to develop Servlet using POST method.}

\begin{solutionbox}
\begin{center}
\captionof{table}{GET vs POST Comparison}
\begin{tabulary}{\linewidth}{|L|L|L|}
\hline
\textbf{Feature} & \textbf{GET} & \textbf{POST} \\ \hline
\textbf{Data Location} & URL parameters & Request body \\ \hline
\textbf{Data Limit} & Limited ($\sim$2KB) & Unlimited \\ \hline
\textbf{Security} & Less secure & More secure \\ \hline
\textbf{Caching} & Cacheable & Not cacheable \\ \hline
\textbf{Bookmarking} & Possible & Not possible \\ \hline
\end{tabulary}
\end{center}

\textbf{POST Method Servlet Example:}
\begin{lstlisting}[language=Java]
import javax.servlet.*;
import javax.servlet.http.*;
import java.io.*;

public class LoginServlet extends HttpServlet {
    protected void doPost(HttpServletRequest request,
                         HttpServletResponse response) 
                         throws ServletException, IOException {
        
        response.setContentType("text/html");
        PrintWriter out = response.getWriter();
        
        // Get form data
        String username = request.getParameter("username");
        String password = request.getParameter("password");
        
        // Validate credentials
        if("admin".equals(username) && "123".equals(password)) {
            out.println("<h2>Login Successful!</h2>");
            out.println("<p>Welcome " + username + "</p>");
        } else {
            out.println("<h2>Login Failed!</h2>");
            out.println("<p>Invalid credentials</p>");
        }
        
        out.close();
    }
}
\end{lstlisting}

\textbf{HTML Form:}
\begin{lstlisting}[language=HTML]
<form method="post" action="LoginServlet">
    Username: <input type="text" name="username"><br />
    Password: <input type="password" name="password"><br />
    <input type="submit" value="Login">
</form>
\end{lstlisting}

\textbf{Key Differences:}
\begin{itemize}
    \item \keyword{GET}: Data in URL, visible, limited size.
    \item \keyword{POST}: Data in body, hidden, unlimited size.
\end{itemize}
\end{solutionbox}

\begin{mnemonicbox}
\mnemonic{GET Grabs, POST Protects}
\end{mnemonicbox}

\questionmarks{4(a)}{3}{List JSP Implicit Objects and explain any two.}

\begin{solutionbox}
\textbf{JSP Implicit Objects List:}
\begin{enumerate}
    \item \textbf{request} (HttpServletRequest)
    \item \textbf{response} (HttpServletResponse)
    \item \textbf{session} (HttpSession)
    \item \textbf{application} (ServletContext)
    \item \textbf{out} (JspWriter)
    \item \textbf{page} (Object)
    \item \textbf{pageContext} (PageContext)
    \item \textbf{config} (ServletConfig)
    \item \textbf{exception} (Throwable)
\end{enumerate}

\textbf{Detailed Explanation:}

\textbf{request Object:}
\begin{itemize}
    \item \keyword{Type}: HttpServletRequest
    \item \keyword{Purpose}: Access request data and parameters.
    \item \keyword{Example}: \code{String name = request.getParameter("name");}
\end{itemize}

\textbf{session Object:}
\begin{itemize}
    \item \keyword{Type}: HttpSession
    \item \keyword{Purpose}: Store user-specific data across requests.
    \item \keyword{Example}: \code{session.setAttribute("user", username);}
\end{itemize}
\end{solutionbox}

\begin{mnemonicbox}
\mnemonic{Request Response Session Application Out}
\end{mnemonicbox}

\questionmarks{4(b)}{4}{Explain features of JSP.}

\begin{solutionbox}
\begin{center}
\captionof{table}{JSP Features}
\begin{tabulary}{\linewidth}{|L|L|L|}
\hline
\textbf{Feature} & \textbf{Description} & \textbf{Benefit} \\ \hline
\textbf{Easy Development} & HTML + Java & Faster coding \\ \hline
\textbf{Platform Independent} & Write once, run anywhere & Portability \\ \hline
\textbf{Component-based} & Reusable components & Maintainability \\ \hline
\textbf{Secure} & Java security model & Safe execution \\ \hline
\end{tabulary}
\end{center}

\textbf{Key Features:}
\begin{itemize}
    \item \keyword{Separation of Concerns}: Design and logic separated.
    \item \keyword{Extensible}: Custom tags and libraries.
    \item \keyword{Compiled}: Translated to servlets for performance.
    \item \keyword{Expression Language}: Simplified syntax.
\end{itemize}

\textbf{JSP Elements:}
\begin{itemize}
    \item \keyword{Directives}: \code{<\%@ \%>}
    \item \keyword{Declarations}: \code{<\%! \%>}
    \item \keyword{Expressions}: \code{<\%= \%>}
    \item \keyword{Scriptlets}: \code{<\% \%>}
\end{itemize}
\end{solutionbox}

\begin{mnemonicbox}
\mnemonic{Easy Platform Component Secure}
\end{mnemonicbox}

\questionmarks{4(c)}{7}{Describe how to call JSP from servlet with example.}

\begin{solutionbox}
\textbf{Methods to Call JSP from Servlet:}

\begin{center}
\captionof{table}{JSP Calling Methods}
\begin{tabulary}{\linewidth}{|L|L|L|}
\hline
\textbf{Method} & \textbf{Interface} & \textbf{Purpose} \\ \hline
\textbf{Forward} & RequestDispatcher & Transfer control \\ \hline
\textbf{Include} & RequestDispatcher & Include content \\ \hline
\textbf{Redirect} & HttpServletResponse & New request \\ \hline
\end{tabulary}
\end{center}

\textbf{Forward Example:}

\textbf{Servlet Code:}
\begin{lstlisting}[language=Java]
public class DataServlet extends HttpServlet {
    protected void doGet(HttpServletRequest request,
                        HttpServletResponse response) 
                        throws ServletException, IOException {
        
        // Process data
        String username = "John Doe";
        int age = 25;
        
        // Set attributes
        request.setAttribute("username", username);
        request.setAttribute("age", age);
        
        // Forward to JSP
        RequestDispatcher dispatcher = 
            request.getRequestDispatcher("display.jsp");
        dispatcher.forward(request, response);
    }
}
\end{lstlisting}

\textbf{JSP Code (display.jsp):}
\begin{lstlisting}[language=Java]
<%@ page language="java" contentType="text/html" %>
<html>
<head><title>User Info</title></head>
<body>
    <h2>User Information</h2>
    <p>Name: <%= request.getAttribute("username") %></p>
    <p>Age: <%= request.getAttribute("age") %></p>
</body>
</html>
\end{lstlisting}

\textbf{Steps:}
\begin{enumerate}
    \item \textbf{Process data} in servlet.
    \item \textbf{Set attributes} in request.
    \item \textbf{Get RequestDispatcher} with JSP path.
    \item \textbf{Forward} to JSP.
\end{enumerate}
\end{solutionbox}

\begin{mnemonicbox}
\mnemonic{Process-Set-Get-Forward: PSGF}
\end{mnemonicbox}

\questionmarks{4(a OR)}{3}{List and explain JSP scripting elements.}

\begin{solutionbox}
\begin{center}
\captionof{table}{JSP Scripting Elements}
\begin{tabulary}{\linewidth}{|L|L|L|L|}
\hline
\textbf{Element} & \textbf{Syntax} & \textbf{Purpose} & \textbf{Example} \\ \hline
\textbf{Directive} & \code{<\%@ \%>} & Page settings & \code{<\%@ page import... \%>} \\ \hline
\textbf{Declaration} & \code{<\%! \%>} & Define methods/vars & \code{<\%! int count = 0; \%>} \\ \hline
\textbf{Expression} & \code{<\%= \%>} & Output values & \code{<\%= new Date() \%>} \\ \hline
\textbf{Scriptlet} & \code{<\% \%>} & Java code & \code{<\% for(int i=0... \%>} \\ \hline
\end{tabulary}
\end{center}

\textbf{Detailed Explanation:}
\begin{itemize}
    \item \keyword{Directives}: Page directive, Include directive, Taglib directive.
    \item \keyword{Declarations}: Define instance variables and methods. Become part of servlet class.
\end{itemize}
\end{solutionbox}

\begin{mnemonicbox}
\mnemonic{Direct Declare Express Script}
\end{mnemonicbox}

\questionmarks{4(b OR)}{4}{Explain JSP life cycle.}

\begin{solutionbox}
\begin{center}
\begin{tikzpicture}[node distance=1.5cm, auto]
    \node [gtu state] (Req) {Request};
    \node [gtu state, right=of Req] (Trans) {Translation};
    \node [gtu state, right=of Trans] (Comp) {Compilation};
    \node [gtu state, below=of Comp] (Load) {Loading};
    \node [gtu state, left=of Load] (Init) {jspInit()};
    \node [gtu state, left=of Init] (Serv) {\_jspService()};
    \node [gtu state, below=of Serv] (Dest) {jspDestroy()};
    
    \path [gtu arrow] (Req) -- (Trans);
    \path [gtu arrow] (Trans) -- (Comp);
    \path [gtu arrow] (Comp) -- (Load);
    \path [gtu arrow] (Load) -- (Init);
    \path [gtu arrow] (Init) -- (Serv);
    \path [gtu arrow] (Serv) edge [loop below] node {Requests} (Serv);
    \path [gtu arrow] (Serv) -- (Dest);
\end{tikzpicture}
\captionof{figure}{JSP Life Cycle}
\end{center}

\begin{center}
\captionof{table}{JSP Life Cycle Phases}
\begin{tabulary}{\linewidth}{|L|L|L|}
\hline
\textbf{Phase} & \textbf{Method} & \textbf{Purpose} \\ \hline
\textbf{Translation} & - & JSP to Java servlet \\ \hline
\textbf{Compilation} & - & Java to bytecode \\ \hline
\textbf{Initialization} & \code{jspInit()} & Setup resources \\ \hline
\textbf{Request Processing} & \code{\_jspService()} & Handle requests \\ \hline
\textbf{Destruction} & \code{jspDestroy()} & Cleanup \\ \hline
\end{tabulary}
\end{center}

\textbf{Key Points:}
\begin{itemize}
    \item \keyword{Translation}: JSP engine converts JSP to servlet.
    \item \keyword{Compilation}: Java compiler creates .class file.
    \item \keyword{Execution}: Servlet container executes compiled servlet.
\end{itemize}
\end{solutionbox}

\begin{mnemonicbox}
\mnemonic{Translate-Compile-Init-Service-Destroy}
\end{mnemonicbox}

\questionmarks{4(c OR)}{7}{Define cookie. Explain working of cookie with example.}

\begin{solutionbox}
\textbf{Cookie Definition:}
A cookie is a small piece of data stored on the client's computer by the web browser while browsing a website.

\textbf{Cookie Working Process:}
\begin{itemize}
    \item Client sends HTTP Request.
    \item Server sends HTTP Response + \code{Set-Cookie}.
    \item Client sends HTTP Request + \code{Cookie}.
    \item Server sends HTTP Response (uses cookie data).
\end{itemize}

\begin{center}
\captionof{table}{Cookie Attributes}
\begin{tabulary}{\linewidth}{|L|L|L|}
\hline
\textbf{Attribute} & \textbf{Purpose} & \textbf{Example} \\ \hline
\textbf{Name} & Cookie identifier & \code{username} \\ \hline
\textbf{Value} & Cookie data & \code{john123} \\ \hline
\textbf{Domain} & Valid domain & \code{.example.com} \\ \hline
\textbf{Path} & Valid path & \code{/shop/} \\ \hline
\textbf{Max-Age} & Expiry time & \code{3600} seconds \\ \hline
\end{tabulary}
\end{center}

\textbf{Cookie Example:}

\textbf{Creating Cookie (Servlet):}
\begin{lstlisting}[language=Java]
public class SetCookieServlet extends HttpServlet {
    protected void doGet(HttpServletRequest request,
                        HttpServletResponse response) {
        
        // Create cookie
        Cookie userCookie = new Cookie("username", "john123");
        userCookie.setMaxAge(60 * 60 * 24); // 1 day
        userCookie.setPath("/");
        
        // Add to response
        response.addCookie(userCookie);
        
        response.getWriter().println("Cookie set successfully!");
    }
}
\end{lstlisting}

\textbf{Reading Cookie (Servlet):}
\begin{lstlisting}[language=Java]
public class GetCookieServlet extends HttpServlet {
    protected void doGet(HttpServletRequest request,
                        HttpServletResponse response) {
        
        Cookie[] cookies = request.getCookies();
        String username = null;
        
        if(cookies != null) {
            for(Cookie cookie : cookies) {
                if("username".equals(cookie.getName())) {
                    username = cookie.getValue();
                    break;
                }
            }
        }
        
        response.getWriter().println("Welcome back, " + username);
    }
}
\end{lstlisting}

\textbf{Cookie Benefits:}
\begin{itemize}
    \item \keyword{User personalization}: Remember preferences.
    \item \keyword{Session tracking}: Maintain state.
    \item \keyword{Analytics}: Track user behavior.
\end{itemize}
\end{solutionbox}

\begin{mnemonicbox}
\mnemonic{Create-Set-Add-Read: CSAR}
\end{mnemonicbox}

\questionmarks{5(a)}{3}{Write difference between JSP and Servlet.}

\begin{solutionbox}
\begin{center}
\captionof{table}{JSP vs Servlet Comparison}
\begin{tabulary}{\linewidth}{|L|L|L|}
\hline
\textbf{Feature} & \textbf{JSP} & \textbf{Servlet} \\ \hline
\textbf{Development} & HTML + Java & Pure Java \\ \hline
\textbf{Compilation} & Automatic & Manual \\ \hline
\textbf{Maintenance} & Easier & More complex \\ \hline
\textbf{Performance} & Slower (first request) & Faster \\ \hline
\textbf{Purpose} & Presentation layer & Business logic \\ \hline
\end{tabulary}
\end{center}

\textbf{Key Differences:}
\begin{itemize}
    \item \keyword{JSP}: Better for presentation, easier for web designers.
    \item \keyword{Servlet}: Better for business logic, more control.
    \item \keyword{Coding}: JSP mixes HTML and Java, Servlet is pure Java.
\end{itemize}
\end{solutionbox}

\begin{mnemonicbox}
\mnemonic{JSP for Presentation, Servlet for Logic}
\end{mnemonicbox}

\questionmarks{5(b)}{4}{Define Spring Boot and explain its advantages.}

\begin{solutionbox}
\textbf{Spring Boot Definition:}
Spring Boot is a framework that simplifies the development of Spring-based applications by providing auto-configuration and embedded servers.

\begin{center}
\captionof{table}{Spring Boot Advantages}
\begin{tabulary}{\linewidth}{|L|L|}
\hline
\textbf{Advantage} & \textbf{Description} \\ \hline
\textbf{Auto Configuration} & Automatically configures Spring applications \\ \hline
\textbf{Embedded Servers} & Built-in Tomcat, Jetty support \\ \hline
\textbf{Starter Dependencies} & Pre-configured dependency sets \\ \hline
\textbf{Production Ready} & Health checks, metrics, monitoring \\ \hline
\end{tabulary}
\end{center}

\textbf{Key Features:}
\begin{itemize}
    \item \keyword{Rapid Development}: Minimal configuration required.
    \item \keyword{Microservices}: Perfect for microservice architecture.
    \item \keyword{No XML}: Convention over configuration.
    \item \keyword{Cloud Ready}: Easy deployment to cloud platforms.
\end{itemize}

\textbf{Example:}
\begin{lstlisting}[language=Java]
@SpringBootApplication
public class MyApplication {
    public static void main(String[] args) {
        SpringApplication.run(MyApplication.class, args);
    }
}
\end{lstlisting}
\end{solutionbox}

\begin{mnemonicbox}
\mnemonic{Auto Embedded Starter Production}
\end{mnemonicbox}

\questionmarks{5(c)}{7}{Explain the architecture of Spring framework.}

\begin{solutionbox}
\begin{center}
\begin{tikzpicture}[node distance=1.5cm, auto]
    \node [gtu block] (App) {Spring Framework Architecture};
    \node [gtu block, below left=of App] (Core) {Core Container};
    \node [gtu block, below=of App] (Data) {Data Access};
    \node [gtu block, below right=of App] (Web) {Web MVC};
    \node [gtu block, below=of Core] (AOP) {AOP};
    \node [gtu block, below=of Web] (Test) {Test};
    
    \path [gtu arrow] (App) -- (Core);
    \path [gtu arrow] (App) -- (Data);
    \path [gtu arrow] (App) -- (Web);
    \path [gtu arrow] (App) -- (AOP);
    \path [gtu arrow] (App) -- (Test);
\end{tikzpicture}
\captionof{figure}{Spring Framework Architecture}
\end{center}

\textbf{Table: Spring Framework Modules}
\begin{center}
\begin{tabulary}{\linewidth}{|L|L|L|}
\hline
\textbf{Module} & \textbf{Components} & \textbf{Purpose} \\ \hline
\textbf{Core Container} & Core, Beans, Context & IoC and DI \\ \hline
\textbf{Data Access} & JDBC, ORM, JMS & Database operations \\ \hline
\textbf{Web MVC} & Web, Servlet, MVC & Web applications \\ \hline
\textbf{AOP} & Aspects, Weaving & Cross-cutting concerns \\ \hline
\end{tabulary}
\end{center}

\textbf{Core Concepts:}
\begin{itemize}
    \item \keyword{IoC (Inversion of Control)}: Framework controls object creation.
    \item \keyword{DI (Dependency Injection)}: Dependencies injected automatically.
    \item \keyword{AOP}: Modular cross-cutting concerns.
    \item \keyword{MVC}: Model-View-Controller pattern.
\end{itemize}
\end{solutionbox}

\begin{mnemonicbox}
\mnemonic{Core Data Web AOP Test}
\end{mnemonicbox}

\questionmarks{5(a OR)}{3}{Write advantages of JSP over Servlet.}

\begin{solutionbox}
\begin{center}
\captionof{table}{JSP Advantages over Servlet}
\begin{tabulary}{\linewidth}{|L|L|L|}
\hline
\textbf{Advantage} & \textbf{JSP} & \textbf{Servlet Limitation} \\ \hline
\textbf{Easy Development} & HTML + Java tags & Complex HTML in Java \\ \hline
\textbf{Automatic Compilation} & Auto-compiled & Manual compilation \\ \hline
\textbf{Designer Friendly} & Web designers can work & Java knowledge required \\ \hline
\textbf{Maintenance} & Easier to modify & Code changes need recompilation \\ \hline
\end{tabulary}
\end{center}

\textbf{Key Advantages:}
\begin{itemize}
    \item \keyword{Separation of Design and Logic}: HTML and Java separated.
    \item \keyword{Rapid Development}: Faster prototyping and development.
    \item \keyword{Less Code}: No need for out.println() statements.
\end{itemize}
\end{solutionbox}

\begin{mnemonicbox}
\mnemonic{Easy Auto Designer Maintenance}
\end{mnemonicbox}

\questionmarks{5(b OR)}{4}{Explain the advantages of Spring Boot.}

\begin{solutionbox}
\begin{center}
\captionof{table}{Spring Boot Advantages}
\begin{tabulary}{\linewidth}{|L|L|L|}
\hline
\textbf{Advantage} & \textbf{Description} & \textbf{Benefit} \\ \hline
\textbf{Auto Configuration} & Automatic setup based on classpath & Reduced configuration \\ \hline
\textbf{Embedded Server} & Built-in Tomcat/Jetty & No external deployment \\ \hline
\textbf{Starter POMs} & Pre-configured dependencies & Simplified dependency management \\ \hline
\textbf{Actuator} & Production monitoring & Health checks and metrics \\ \hline
\end{tabulary}
\end{center}

\textbf{Detailed Advantages:}
\begin{enumerate}
    \item \textbf{Auto Configuration}: Automatically configures Spring application based on dependencies.
    \item \textbf{Embedded Servers}: No need for external application servers. Easy to run.
    \item \textbf{Starter Dependencies}: Pre-configured dependency sets.
    \item \textbf{Production Features}: Health endpoints, metrics collection.
\end{enumerate}
\end{solutionbox}

\begin{mnemonicbox}
\mnemonic{Auto Embedded Starter Production}
\end{mnemonicbox}

\questionmarks{5(c OR)}{7}{Explain MVC architecture.}

\begin{solutionbox}
\textbf{MVC (Model-View-Controller) Architecture:}

\begin{center}
\begin{tikzpicture}[node distance=2cm, auto]
    \node [gtu block] (View) {View};
    \node [gtu block, right=of View] (Cont) {Controller};
    \node [gtu block, right=of Cont] (Model) {Model};
    \node [gtu block, below=of View] (User) {User};
    
    \path [gtu arrow] (User) -- (View);
    \path [gtu arrow] (View) -- (Cont);
    \path [gtu arrow] (Cont) -- (Model);
    \path [gtu arrow] (Model) -- (Cont);
    \path [gtu arrow] (Cont) -- (View);
    \path [gtu arrow] (View) -- (User);
\end{tikzpicture}
\captionof{figure}{MVC Architecture}
\end{center}

\begin{center}
\captionof{table}{MVC Components}
\begin{tabulary}{\linewidth}{|L|L|L|}
\hline
\textbf{Component} & \textbf{Responsibility} & \textbf{Example} \\ \hline
\textbf{Model} & Data and business logic & Entity classes, DAOs \\ \hline
\textbf{View} & User interface & JSP, HTML, Templates \\ \hline
\textbf{Controller} & Request handling & Servlets, Spring Controllers \\ \hline
\end{tabulary}
\end{center}

\textbf{MVC Flow:}
\begin{enumerate}
    \item User sends input to View.
    \item View sends request to Controller.
    \item Controller processes data with Model.
    \item Model returns data to Controller.
    \item Controller selects View.
    \item View sends response to User.
\end{enumerate}

\textbf{Spring MVC Example:}

\textbf{Controller:}
\begin{lstlisting}[language=Java]
@Controller
public class StudentController {
    @Autowired
    private StudentService studentService;
    
    @GetMapping("/students")
    public ModelAndView getStudents() {
        List<Student> students = studentService.getAllStudents();
        ModelAndView mv = new ModelAndView("students");
        mv.addObject("studentList", students);
        return mv;
    }
}
\end{lstlisting}

\textbf{MVC Advantages:}
\begin{itemize}
    \item \keyword{Separation of Concerns}: Clear separation of responsibilities.
    \item \keyword{Maintainability}: Easy to maintain and modify.
    \item \keyword{Testability}: Each component can be tested independently.
\end{itemize}
\end{solutionbox}

\begin{mnemonicbox}
\mnemonic{Model manages data, View shows data, Controller controls flow}
\end{mnemonicbox}

\end{document}
