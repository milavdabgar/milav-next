\documentclass{article}

% content/resources/templates/preamble.tex
\usepackage[margin=0.6in]{geometry}
\author{Milav Dabgar}
\usepackage{amsmath,amssymb,amsthm}
\usepackage{booktabs}
\usepackage{multirow}
\usepackage{xcolor}
\usepackage{tcolorbox}
\tcbuselibrary{breakable,skins}
\usepackage[colorlinks=true,linkcolor=blue]{hyperref}
\usepackage{titlesec}
\usepackage{enumitem}
\usepackage{tikz}
\usepackage{pgfplots}
\usepackage{circuitikz}
\usepackage[version=4]{mhchem}
\usepackage{longtable}
\usepackage{array}
\usepackage{float}
\usepackage{caption}
\usepackage{listings}

\lstset{
  basicstyle=\small\ttfamily,
  breaklines=true,
  breakatwhitespace=false,
  postbreak=\mbox{\textcolor{red}{$\hookrightarrow$}\space},
  float=false,
  numbers=left,
  numberstyle=\tiny\color{gray},
  numbersep=10pt,
  xleftmargin=2em,
  keywordstyle=\color{blue},
  commentstyle=\color{green!60!black},
  stringstyle=\color{purple},
  backgroundcolor=\color{gray!5},
  showstringspaces=false,
  tabsize=2,
  captionpos=b,
  keepspaces=true,
  columns=flexible
}

\pgfplotsset{compat=1.18}
\usetikzlibrary{shapes,arrows,positioning,calc,patterns,decorations.pathmorphing,decorations.markings,arrows.meta}

% Color scheme
\definecolor{headcolor}{RGB}{0,102,204}
\definecolor{keycolor}{RGB}{220,20,60}
\definecolor{solutioncolor}{RGB}{34,139,34}
\definecolor{mnemoniccolor}{RGB}{148,0,211}
\definecolor{codecolor}{RGB}{0,0,100}

% Spacing
\setlength{\parskip}{3pt}
\setlist[itemize]{nosep}
\setlist[enumerate]{nosep}

% Title formatting
\titleformat{\section}{\Large\bfseries\color{headcolor}}{\thesection}{1em}{}
\titleformat{\subsection}{\large\bfseries\color{headcolor}}{\thesubsection}{1em}{}

% Pandoc tightlist compatibility
\providecommand{\tightlist}{%
  \setlength{\itemsep}{0pt}\setlength{\parskip}{0pt}}

% Pandoc longtable compatibility
\newcounter{none}
\def\thenone{}


% content/resources/templates/gujarati-boxes.tex
\usepackage{fontspec}
\usepackage{polyglossia}

% Set Gujarati as main language (document is primarily in Gujarati)
% Note: gloss-gujarati.ldf doesn't exist in polyglossia, but it will use hyphenation patterns
\setdefaultlanguage{gujarati}
\setotherlanguage{english}

% Configure Gujarati font properly
% Use Language=Default to prevent polyglossia from trying to add language-specific features
% that don't exist for Gujarati, which causes "empty feature" warnings
\newfontfamily\gujaratifont[Script=Gujarati,AutoFakeBold=2.5,AutoFakeSlant=0.3]{Noto Sans Gujarati}
\setmainfont[Script=Gujarati,AutoFakeBold=2.5,AutoFakeSlant=0.3]{Noto Sans Gujarati}
% Use Noto Sans Gujarati for monospace to support Gujarati in text
\setmonofont[Scale=0.9]{Noto Sans Gujarati}

% Configure English to use the same font
\newfontfamily\englishfont[Script=Gujarati,AutoFakeBold=2.5,AutoFakeSlant=0.3]{Noto Sans Gujarati}

% Translations for polyglossia
\gappto\captionsgujarati{
  \renewcommand{\tablename}{કોષ્ટક}
  \renewcommand{\figurename}{આકૃતિ}
}

% Helper for TikZ nodes to ensure Gujarati font
\newcommand{\gu}[1]{{\gujaratifont #1}}

% Custom environments
\newtcolorbox{solutionbox}{
    breakable,
    enhanced,
    colback=solutioncolor!5!white,
    colframe=solutioncolor!75!black,
    fonttitle=\bfseries,
    title=જવાબ
}

\newtcolorbox{solutionboxnobreak}{
 colback=solutioncolor!5!white,
 colframe=solutioncolor!75!black,
 fonttitle=\bfseries,
 title=જવાબ
}

\newtcolorbox{keyformula}{
 breakable,
 enhanced,
 colback=keycolor!5!white,
 colframe=keycolor!75!black,
 fonttitle=\bfseries,
 title=રાસાયણિક સમીકરણ/સૂત્ર
}

\newtcolorbox{mnemonicbox}{
 breakable,
 enhanced,
 colback=mnemoniccolor!5!white,
 colframe=mnemoniccolor!75!black,
 fonttitle=\bfseries,
 title=મેમરી ટ્રીક
}


% Custom commands for GTU solutions
% This file defines semantic commands for consistent formatting

% Question command with automatic formatting
\newcommand{\question}[2]{%
  \section*{Question #1}%
  \textbf{#2}%
}

% OR question variant
\newcommand{\questionor}[2]{%
  \section*{Question #1 OR}%
  \textbf{#2}%
}

% Proper table environment with caption
\newenvironment{answertable}[1]{%
  \begin{table}[htbp]
  \centering
  \caption{#1}
}{%
  \end{table}
}

% Proper figure environment for diagrams
\newenvironment{answerdiagram}[1]{%
  \begin{figure}[htbp]
  \centering
  \caption{#1}
}{%
  \end{figure}
}

% Semantic markup for key terms
\newcommand{\keyword}[1]{\textbf{#1}}
\newcommand{\code}[1]{\texttt{#1}}
\newcommand{\classname}[1]{\texttt{#1}}
\newcommand{\methodname}[1]{\texttt{#1}}

% Proper quotation marks
\newcommand{\mnemonic}[1]{``#1''}


\title{Advanced Java Programming (4351603) - Summer 2025 Solution}
\date{May 16, 2025}

\begin{document}
\maketitle

\questionmarks{1(a)}{3}{AWT અને Swing વચ્ચેનો તફાવત લખો.}

\begin{solutionbox}
\begin{center}
\captionof{table}{AWT વિ. Swing તુલના}
\begin{tabulary}{\linewidth}{|L|L|L|}
\hline
\textbf{વિશેષતા} & \textbf{AWT} & \textbf{Swing} \\ \hline
\textbf{પ્લેટફોર્મ} & પ્લેટફોર્મ આધારિત & પ્લેટફોર્મ સ્વતંત્ર \\ \hline
\textbf{કોમ્પોનેન્ટ્સ} & હેવી વેઇટ & લાઇટ વેઇટ \\ \hline
\textbf{લુક એન્ડ ફીલ} & નેટિવ OS લુક & પ્લગેબલ લુક એન્ડ ફીલ \\ \hline
\textbf{પ્રદર્શન} & ઝડપી & AWT કરતાં ધીમું \\ \hline
\end{tabulary}
\end{center}

\begin{itemize}
    \item \textbf{AWT}: નેટિવ OS કોમ્પોનેન્ટ્સ વાપરે છે.
    \item \textbf{Swing}: જાવાના પોતાના કોમ્પોનેન્ટ્સ વાપરે છે.
    \item \textbf{પ્લગેબિલિટી}: Swing કસ્ટમાઇઝેબલ UI સપોર્ટ કરે છે.
\end{itemize}
\end{solutionbox}

\begin{mnemonicbox}
\mnemonic{Swing is Smart - Platform Independent and Pluggable}
\end{mnemonicbox}

\questionmarks{1(b)}{4}{વિવિધ Layout Manager ની યાદી બનાવો. Flow Layout manager ને ઉદાહરણ સાથે સમજાવો.}

\begin{solutionbox}
\textbf{Layout Manager ની યાદી:}
\begin{itemize}
    \item \keyword{FlowLayout}: ડાબેથી જમણે ગોઠવણી.
    \item \keyword{BorderLayout}: ઉત્તર, દક્ષિણ, પૂર્વ, પશ્ચિમ, કેન્દ્ર.
    \item \keyword{GridLayout}: સમાન કદના ગ્રિડ સેલ્સ.
    \item \keyword{CardLayout}: કોમ્પોનેન્ટ્સનો સ્ટેક.
    \item \keyword{BoxLayout}: એક પંક્તિ અથવા કોલમ.
\end{itemize}

\textbf{FlowLayout ઉદાહરણ:}
\begin{lstlisting}[language=Java]
import javax.swing.*;
import java.awt.*;

public class FlowExample extends JFrame {
    public FlowExample() {
        setLayout(new FlowLayout());
        add(new JButton("બટન 1"));
        add(new JButton("બટન 2"));
        add(new JButton("બટન 3"));
        setSize(300, 100);
        setVisible(true);
    }
}
\end{lstlisting}
\end{solutionbox}

\begin{mnemonicbox}
\mnemonic{પાણીની જેમ વહે - ડાબેથી જમણે}
\end{mnemonicbox}

\questionmarks{1(c)}{7}{ચેકબૉક્સનો ઉપયોગ કરીને Swing program લખો જે વપરાશકર્તાઓને વિકલ્પોની સૂચિમાંથી બહુવિધ વિકલ્પો પસંદ કરવાની મંજૂરી આપે અને પસંદ કરેલ વિકલ્પો દર્શાવે.}

\begin{solutionbox}
\begin{lstlisting}[language=Java]
import javax.swing.*;
import java.awt.*;
import java.awt.event.*;

public class CheckboxExample extends JFrame implements ItemListener {
    JCheckBox java, python, cpp;
    JTextArea display;
    
    public CheckboxExample() {
        setLayout(new FlowLayout());
        
        java = new JCheckBox("Java");
        python = new JCheckBox("Python");
        cpp = new JCheckBox("C++");
        
        java.addItemListener(this);
        python.addItemListener(this);
        cpp.addItemListener(this);
        
        display = new JTextArea(5, 20);
        
        add(java);
        add(python);
        add(cpp);
        add(new JScrollPane(display));
        
        setSize(300, 200);
        setDefaultCloseOperation(JFrame.EXIT_ON_CLOSE);
        setVisible(true);
    }
    
    public void itemStateChanged(ItemEvent e) {
        String result = "પસંદ કરેલ: ";
        if(java.isSelected()) result += "Java ";
        if(python.isSelected()) result += "Python ";
        if(cpp.isSelected()) result += "C++ ";
        display.setText(result);
    }
    
    public static void main(String[] args) {
        new CheckboxExample();
    }
}
\end{lstlisting}

\textbf{મુખ્ય વિશેષતાઓ:}
\begin{itemize}
    \item \keyword{બહુવિધ પસંદગી}: વપરાશકર્તાઓ બહુવિધ ચેકબૉક્સ પસંદ કરી શકે.
    \item \keyword{રીઅલ-ટાઇમ ડિસ્પ્લે}: પસંદ કરેલ આઇટમ્સ તરત જ દર્શાવે.
    \item \keyword{ItemListener}: ચેકબૉક્સ સ્ટેટ બદલાવ હેન્ડલ કરે.
\end{itemize}
\end{solutionbox}

\begin{mnemonicbox}
\mnemonic{બહુવિધ ચેક કરો, બધું દર્શાવો}
\end{mnemonicbox}

\questionmarks{1(c OR)}{7}{વિવિધ swing components ની મદદથી Java program લખો.}

\begin{solutionbox}
\begin{lstlisting}[language=Java]
import javax.swing.*;
import java.awt.*;
import java.awt.event.*;

public class SwingComponents extends JFrame implements ActionListener {
    JTextField nameField;
    JComboBox<String> cityCombo;
    JRadioButton male, female;
    JButton submit;
    JTextArea display;
    
    public SwingComponents() {
        setLayout(new FlowLayout());
        
        add(new JLabel("નામ:"));
        nameField = new JTextField(15);
        add(nameField);
        
        add(new JLabel("શહેર:"));
        cityCombo = new JComboBox<>(new String[]{"મુંબઈ", "દિલ્હી", "બેંગ્લોર"});
        add(cityCombo);
        
        ButtonGroup gender = new ButtonGroup();
        male = new JRadioButton("પુરુષ");
        female = new JRadioButton("સ્ત્રી");
        gender.add(male);
        gender.add(female);
        add(male);
        add(female);
        
        submit = new JButton("સબમિટ");
        submit.addActionListener(this);
        add(submit);
        
        display = new JTextArea(5, 25);
        add(new JScrollPane(display));
        
        setSize(400, 300);
        setDefaultCloseOperation(JFrame.EXIT_ON_CLOSE);
        setVisible(true);
    }
    
    public void actionPerformed(ActionEvent e) {
        String name = nameField.getText();
        String city = (String)cityCombo.getSelectedItem();
        String gender = male.isSelected() ? "પુરુષ" : "સ્ત્રી";
        
        display.setText("નામ: " + name + "\nશહેર: " + city + "\nલિંગ: " + gender);
    }
    
    public static void main(String[] args) {
        new SwingComponents();
    }
}
\end{lstlisting}

\textbf{વપરાયેલ કોમ્પોનેન્ટ્સ:}
\begin{itemize}
    \item \keyword{JTextField}: ટેક્સ્ટ ઇનપુટ.
    \item \keyword{JComboBox}: ડ્રોપડાઉન પસંદગી.
    \item \keyword{JRadioButton}: એક પસંદગી.
    \item \keyword{JButton}: એક્શન ટ્રિગર.
\end{itemize}
\end{solutionbox}

\begin{mnemonicbox}
\mnemonic{ટેક્સ્ટ, કોમ્બો, રેડિયો, બટન - સંપૂર્ણ ફોર્મ}
\end{mnemonicbox}

\questionmarks{2(a)}{3}{ઉદાહરણ સાથે Swing controls સમજાવો.}

\begin{solutionbox}
\begin{center}
\captionof{table}{સામાન્ય Swing Controls}
\begin{tabulary}{\linewidth}{|L|L|L|}
\hline
\textbf{Control} & \textbf{હેતુ} & \textbf{ઉદાહરણ} \\ \hline
\textbf{JButton} & ક્લિક એક્શન્સ & \code{new JButton("મને ક્લિક કરો")} \\ \hline
\textbf{JTextField} & ટેક્સ્ટ ઇનપુટ & \code{new JTextField(10)} \\ \hline
\textbf{JLabel} & ટેક્સ્ટ દર્શાવવું & \code{new JLabel("નમસ્કાર")} \\ \hline
\textbf{JCheckBox} & બહુવિધ પસંદગી & \code{new JCheckBox("વિકલ્પ")} \\ \hline
\end{tabulary}
\end{center}

\textbf{મૂળભૂત ઉદાહરણ:}
\begin{lstlisting}[language=Java]
JFrame frame = new JFrame();
JButton btn = new JButton("સબમિટ");
frame.add(btn);
frame.setSize(200, 100);
frame.setVisible(true);
\end{lstlisting}
\end{solutionbox}

\begin{mnemonicbox}
\mnemonic{બટન, ટેક્સ્ટ, લેબલ, ચેક - મૂળભૂત ચાર}
\end{mnemonicbox}

\questionmarks{2(b)}{4}{JDBC drivers ની યાદી બનાવો અને કોઈપણ બે સમજાવો.}

\begin{solutionbox}
\textbf{JDBC Drivers ની યાદી:}
\begin{enumerate}
    \item \textbf{Type 1}: JDBC-ODBC Bridge
    \item \textbf{Type 2}: Native API Driver
    \item \textbf{Type 3}: Network Protocol Driver
    \item \textbf{Type 4}: Thin Driver
\end{enumerate}

\textbf{વિગતવાર સમજૂતી:}

\textbf{Type 1 - JDBC-ODBC Bridge:}
\begin{itemize}
    \item \keyword{હેતુ}: JDBC કોલ્સને ODBC કોલ્સમાં રૂપાંતરિત કરે.
    \item \keyword{ફાયદો}: કોઈપણ ODBC ડેટાબેસ સાથે કામ કરે.
    \item \keyword{નુકસાન}: પ્લેટફોર્મ આધારિત, ધીમી કાર્યક્ષમતા.
\end{itemize}

\textbf{Type 4 - Thin Driver:}
\begin{itemize}
    \item \keyword{હેતુ}: શુદ્ધ Java driver, સીધો ડેટાબેસ કમ્યુનિકેશન.
    \item \keyword{ફાયદો}: પ્લેટફોર્મ સ્વતંત્ર, શ્રેષ્ઠ કાર્યક્ષમતા.
    \item \keyword{નુકસાન}: ડેટાબેસ વિશિષ્ટ.
\end{itemize}
\end{solutionbox}

\begin{mnemonicbox}
\mnemonic{Bridge-Native-Network-Thin: 1-2-3-4}
\end{mnemonicbox}

\questionmarks{2(c)}{7}{Object Relational Mapping (ORM) સમજાવો તથા તેના ફાયદા અને tools સમજાવો.}

\begin{solutionbox}
\textbf{Object Relational Mapping (ORM):}
ORM એ તકનીક છે જે ઑબ્જેક્ટ-ઓરિએન્ટેડ પ્રોગ્રામિંગ કન્સેપ્ટ્સને રિલેશનલ ડેટાબેસ સ્ટ્રક્ચર સાથે મેપ કરે છે.

\begin{center}
\begin{tikzpicture}[node distance=2cm, auto]
    \node [gtu block] (Obj) {Java Object};
    \node [gtu block, right=of Obj] (ORM) {ORM Framework};
    \node [gtu block, right=of ORM] (DB) {Database Table};
    
    \path [gtu arrow, <->] (Obj) -- (ORM);
    \path [gtu arrow, <->] (ORM) -- (DB);
\end{tikzpicture}
\captionof{figure}{ORM કન્સેપ્ટ}
\end{center}

\begin{center}
\captionof{table}{ORM ફાયદા}
\begin{tabulary}{\linewidth}{|L|L|}
\hline
\textbf{ફાયદો} & \textbf{વર્ણન} \\ \hline
\textbf{ઉત્પાદકતા} & કોડિંગ સમય ઘટાડે \\ \hline
\textbf{જાળવણી} & સુધારા અને અપડેટ કરવા સરળ \\ \hline
\textbf{ડેટાબેસ સ્વતંત્રતા} & ડેટાબેસ સરળતાથી બદલી શકાય \\ \hline
\textbf{ઑબ્જેક્ટ-ઓરિએન્ટેડ} & OOP કન્સેપ્ટ્સ સાથે કામ કરે \\ \hline
\end{tabulary}
\end{center}

\textbf{લોકપ્રિય ORM Tools:}
\begin{itemize}
    \item \keyword{Hibernate}: સૌથી લોકપ્રિય Java ORM.
    \item \keyword{JPA}: Java Persistence API સ્ટાન્ડર્ડ.
    \item \keyword{MyBatis}: SQL મેપિંગ ફ્રેમવર્ક.
    \item \keyword{EclipseLink}: રેફરન્સ ઇમ્પ્લિમેન્ટેશન.
\end{itemize}

\textbf{વર્કિંગ મોડલ:}
\begin{itemize}
    \item \textbf{Objects} $\rightarrow$ \textbf{ORM} $\rightarrow$ \textbf{Tables}
    \item ઑટોમેટિક SQL જનરેશન
    \item Type-safe queries
\end{itemize}
\end{solutionbox}

\begin{mnemonicbox}
\mnemonic{Objects Relate Magically}
\end{mnemonicbox}

\questionmarks{2(a OR)}{3}{ઉદાહરણ સાથે MOUSEEVENT અને MOUSELISTENER interface સમજાવો.}

\begin{solutionbox}
\textbf{MouseEvent:}
જ્યારે કોમ્પોનેન્ટ્સ પર માઉસ એક્શન્સ થાય ત્યારે જનરેટ થાય છે.

\textbf{MouseListener Interface Methods:}
\begin{itemize}
    \item \keyword{mouseClicked()}: માઉસ બટન ક્લિક.
    \item \keyword{mousePressed()}: માઉસ બટન દબાવ્યું.
    \item \keyword{mouseReleased()}: માઉસ બટન છોડ્યું.
    \item \keyword{mouseEntered()}: માઉસ કોમ્પોનેન્ટમાં પ્રવેશ.
    \item \keyword{mouseExited()}: માઉસ કોમ્પોનેન્ટમાંથી બહાર.
\end{itemize}

\textbf{ઉદાહરણ:}
\begin{lstlisting}[language=Java]
public class MouseExample extends JFrame implements MouseListener {
    JLabel label;
    
    public MouseExample() {
        label = new JLabel("મને ક્લિક કરો!");
        label.addMouseListener(this);
        add(label);
        setSize(200, 100);
        setVisible(true);
    }
    
    public void mouseClicked(MouseEvent e) {
        label.setText("ક્લિક થયું!");
    }
    
    // અન્ય methods...
}
\end{lstlisting}
\end{solutionbox}

\begin{mnemonicbox}
\mnemonic{Click-Press-Release-Enter-Exit}
\end{mnemonicbox}

\questionmarks{2(b OR)}{4}{JDBC API ના components ની સૂચિ બનાવો અને સમજાવો.}

\begin{solutionbox}
\begin{center}
\captionof{table}{JDBC API Components}
\begin{tabulary}{\linewidth}{|L|L|L|}
\hline
\textbf{Component} & \textbf{હેતુ} & \textbf{મુખ્ય Classes} \\ \hline
\textbf{DriverManager} & ડ્રાઇવર્સ મેનેજ કરે & \code{DriverManager.getConnection()} \\ \hline
\textbf{Connection} & ડેટાબેસ કનેક્શન & \code{Connection conn} \\ \hline
\textbf{Statement} & SQL એક્ઝિક્યુશન & \code{Statement stmt} \\ \hline
\textbf{ResultSet} & ક્વેરી પરિણામો & \code{ResultSet rs} \\ \hline
\end{tabulary}
\end{center}

\textbf{Component વિગતો:}
\begin{itemize}
    \item \keyword{DriverManager}: ડેટાબેસ સાથે કનેક્શન સ્થાપિત કરે.
    \item \keyword{Connection}: ડેટાબેસ સેશન રજૂ કરે.
    \item \keyword{Statement}: SQL ક્વેરીઓ એક્ઝિક્યુટ કરે.
    \item \keyword{ResultSet}: ક્વેરી પરિણામો સાચવે.
\end{itemize}

\textbf{મૂળભૂત ઉપયોગ:}
\begin{lstlisting}[language=Java]
Connection conn = DriverManager.getConnection(url, user, pass);
Statement stmt = conn.createStatement();
ResultSet rs = stmt.executeQuery("SELECT * FROM users");
\end{lstlisting}
\end{solutionbox}

\begin{mnemonicbox}
\mnemonic{ડ્રાઇવર કનેક્ટ કરે, સ્ટેટમેન્ટ એક્ઝિક્યુટ કરે, ResultSet રિટર્ન કરે}
\end{mnemonicbox}

\questionmarks{2(c OR)}{7}{હાઇબર્નેટનું આર્કિટેક્ચર દોરો અને સમજાવો.}

\begin{solutionbox}
\begin{center}
\begin{tikzpicture}[node distance=1.5cm, auto]
    \node [gtu block] (App) {Java Application};
    \node [gtu block, below=of App] (API) {Hibernate API};
    \node [gtu block, below left=of API] (Config) {Configuration};
    \node [gtu block, below right=of API] (Factory) {SessionFactory};
    
    \node [gtu block, below=of Factory] (Session) {Session};
    \node [gtu block, below left=of Session] (Trans) {Transaction};
    \node [gtu block, below=of Session] (Query) {Query};
    \node [gtu block, below right=of Session] (Crit) {Criteria};
    
    \node [gtu block, below=of Query] (DB) {Database};
    
    \node [left=of Config] (CfgFile) {hibernate.cfg.xml};
    \node [right=of Factory] (MapFile) {Mapping Files};
    
    \path [gtu arrow] (App) -- (API);
    \path [gtu arrow] (API) -- (Config);
    \path [gtu arrow] (API) -- (Factory);
    \path [gtu arrow] (Factory) -- (Session);
    \path [gtu arrow] (Session) -- (Trans);
    \path [gtu arrow] (Session) -- (Query);
    \path [gtu arrow] (Session) -- (Crit);
    \path [gtu arrow] (CfgFile) -- (Config);
    \path [gtu arrow] (MapFile) -- (Factory);
    \path [gtu arrow] (Session) -- (DB);
\end{tikzpicture}
\captionof{figure}{Hibernate આર્કિટેક્ચર}
\end{center}

\textbf{ટેબલ: Hibernate આર્કિટેક્ચર કોમ્પોનેન્ટ્સ}
\begin{center}
\begin{tabulary}{\linewidth}{|L|L|}
\hline
\textbf{Component} & \textbf{કાર્ય} \\ \hline
\textbf{Configuration} & કોન્ફિગ ફાઇલો વાંચે \\ \hline
\textbf{SessionFactory} & Session ઑબ્જેક્ટ્સ બનાવે \\ \hline
\textbf{Session} & ડેટાબેસ ઇન્ટરફેસ \\ \hline
\textbf{Transaction} & ટ્રાન્ઝેક્શન મેનેજ કરે \\ \hline
\textbf{Query} & HQL/SQL ક્વેરીઓ \\ \hline
\end{tabulary}
\end{center}

\textbf{લેયર વર્ણન:}
\begin{itemize}
    \item \keyword{Application Layer}: Java ઑબ્જેક્ટ્સ અને બિઝનેસ લોજિક.
    \item \keyword{Hibernate Layer}: ORM મેપિંગ અને સેશન મેનેજમેન્ટ.
    \item \keyword{Database Layer}: વાસ્તવિક ડેટા સ્ટોરેજ.
\end{itemize}

\textbf{મુખ્ય વિશેષતાઓ:}
\begin{itemize}
    \item \keyword{ઑટોમેટિક ટેબલ ક્રિએશન}: એન્ટિટી ક્લાસીસ આધારે.
    \item \keyword{HQL સપોર્ટ}: ઑબ્જેક્ટ-ઓરિએન્ટેડ ક્વેરી લેંગ્વેજ.
    \item \keyword{કેશિંગ}: પ્રથમ અને દ્વિતીય સ્તરની કેશિંગ.
\end{itemize}
\end{solutionbox}

\begin{mnemonicbox}
\mnemonic{Config-Factory-Session-Transaction: CFST}
\end{mnemonicbox}

\questionmarks{3(a)}{3}{Servlet ની વિવિધ વિશેષતાઓ સમજાવો.}

\begin{solutionbox}
\begin{center}
\captionof{table}{Servlet વિશેષતાઓ}
\begin{tabulary}{\linewidth}{|L|L|}
\hline
\textbf{વિશેષતા} & \textbf{વર્ણન} \\ \hline
\textbf{પ્લેટફોર્મ સ્વતંત્ર} & JVM સાથે કોઈપણ OS પર ચાલે \\ \hline
\textbf{કાર્યક્ષમતા} & CGI કરતાં વધુ સારી \\ \hline
\textbf{મજબૂત} & JVM મેનેજ્ડ મેમરી \\ \hline
\textbf{સુરક્ષિત} & Java સિક્યોરિટી ફીચર્સ \\ \hline
\end{tabulary}
\end{center}

\textbf{મુખ્ય વિશેષતાઓ:}
\begin{itemize}
    \item \keyword{સર્વર-સાઇડ પ્રોસેસિંગ}: ક્લાયન્ટ રિક્વેસ્ટ્સ હેન્ડલ કરે.
    \item \keyword{પ્રોટોકોલ સ્વતંત્ર}: HTTP, FTP, SMTP સપોર્ટ.
    \item \keyword{વિસ્તરણીય}: સરળતાથી વિસ્તૃત કરી શકાય.
    \item \keyword{પોર્ટેબલ}: એકવાર લખો, ગમે ત્યાં ચલાવો.
\end{itemize}
\end{solutionbox}

\begin{mnemonicbox}
\mnemonic{Platform Performance Robust Secure}
\end{mnemonicbox}

\questionmarks{3(b)}{4}{Servlet life cycle સમજાવો.}

\begin{solutionbox}
\begin{center}
\begin{tikzpicture}[node distance=1.5cm, auto]
    \node [gtu state] (Load) {Loading};
    \node [gtu state, right=of Load] (Inst) {Instantiation};
    \node [gtu state, right=of Inst] (Init) {init()};
    \node [gtu state, below=of Init] (Serv) {service()};
    \node [gtu state, left=of Serv] (Dest) {destroy()};
    \node [gtu state, left=of Dest] (Unload) {Unloaded};
    
    \path [gtu arrow] (Load) -- (Inst);
    \path [gtu arrow] (Inst) -- (Init);
    \path [gtu arrow] (Init) -- (Serv);
    \path [gtu arrow] (Serv) edge [loop below] node {Requests} (Serv);
    \path [gtu arrow] (Serv) -- (Dest);
    \path [gtu arrow] (Dest) -- (Unload);
\end{tikzpicture}
\captionof{figure}{Servlet Life Cycle}
\end{center}

\textbf{ટેબલ: Servlet લાઇફ સાઇકલ તબક્કાઓ}
\begin{center}
\begin{tabulary}{\linewidth}{|L|L|L|}
\hline
\textbf{તબક્કો} & \textbf{Method} & \textbf{હેતુ} \\ \hline
\textbf{લોડિંગ} & ક્લાસ લોડિંગ & JVM servlet ક્લાસ લોડ કરે \\ \hline
\textbf{ઇન્સ્ટન્શિએશન} & Constructor & servlet ઑબ્જેક્ટ બનાવે \\ \hline
\textbf{ઇનિશિયલાઇઝેશન} & \code{init()} & એકવારની સેટઅપ \\ \hline
\textbf{રિક્વેસ્ટ પ્રોસેસિંગ} & \code{service()} & રિક્વેસ્ટ્સ હેન્ડલ કરે \\ \hline
\textbf{વિનાશ} & \code{destroy()} & રિસોર્સ સાફ કરે \\ \hline
\end{tabulary}
\end{center}

\textbf{Method વિગતો:}
\begin{itemize}
    \item \keyword{init()}: servlet લોડ થાય ત્યારે એકવાર કોલ થાય.
    \item \keyword{service()}: દરેક રિક્વેસ્ટ માટે કોલ થાય.
    \item \keyword{destroy()}: servlet અનલોડ થાય ત્યારે કોલ થાય.
\end{itemize}
\end{solutionbox}

\begin{mnemonicbox}
\mnemonic{Load-Create-Init-Service-Destroy}
\end{mnemonicbox}

\questionmarks{3(c)}{7}{Servlet માં session tracking ઉદાહરણ સાથે સમજાવો.}

\begin{solutionbox}
\textbf{Session Tracking પદ્ધતિઓ:}

\begin{center}
\captionof{table}{Session Tracking તકનીકો}
\begin{tabulary}{\linewidth}{|L|L|L|}
\hline
\textbf{પદ્ધતિ} & \textbf{વર્ણન} & \textbf{ફાયદા/નુકસાન} \\ \hline
\textbf{Cookies} & ક્લાયન્ટ-સાઇડ સ્ટોરેજ & સરળ/ગોપનીયતાના મુદ્દા \\ \hline
\textbf{URL Rewriting} & session ID ઉમેરવું & સાર્વત્રિક/કદરૂપ URLs \\ \hline
\textbf{Hidden Fields} & ફોર્મ-આધારિત ટ્રેકિંગ & સરળ/ફોર્મ પર આધારિત \\ \hline
\textbf{HttpSession} & સર્વર-સાઇડ ઑબ્જેક્ટ & સુરક્ષિત/મેમરી ઉપયોગ \\ \hline
\end{tabulary}
\end{center}

\textbf{HttpSession ઉદાહરણ:}
\begin{lstlisting}[language=Java]
protected void doGet(HttpServletRequest request, 
                    HttpServletResponse response) {
    HttpSession session = request.getSession();
    
    // ડેટા સ્ટોર કરવું
    session.setAttribute("username", "john");
    
    // ડેટા મેળવવું
    String user = (String) session.getAttribute("username");
    
    // Session ની માહિતી
    String sessionId = session.getId();
    boolean isNew = session.isNew();
    
    PrintWriter out = response.getWriter();
    out.println("વપરાશકર્તા: " + user);
    out.println("Session ID: " + sessionId);
}
\end{lstlisting}

\textbf{Session મેનેજમેન્ટ:}
\begin{itemize}
    \item \keyword{બનાવટ}: \code{request.getSession()}
    \item \keyword{સ્ટોરેજ}: \code{session.setAttribute()}
    \item \keyword{પુનઃપ્રાપ્તિ}: \code{session.getAttribute()}
    \item \keyword{રદ કરવું}: \code{session.invalidate()}
\end{itemize}
\end{solutionbox}

\begin{mnemonicbox}
\mnemonic{Cookies-URLs-Hidden-HttpSession: CUHS}
\end{mnemonicbox}

\questionmarks{3(a OR)}{3}{Servlet life cycle ની methods સમજાવો.}

\begin{solutionbox}
\textbf{લાઇફ સાઇકલ Methods:}

\begin{center}
\captionof{table}{Servlet લાઇફ સાઇકલ Methods}
\begin{tabulary}{\linewidth}{|L|L|L|}
\hline
\textbf{Method} & \textbf{ક્યારે કોલ થાય} & \textbf{Parameters} \\ \hline
\textbf{init()} & Servlet ઇનિશિયલાઇઝેશન & \code{ServletConfig config} \\ \hline
\textbf{service()} & દરેક રિક્વેસ્ટ & \code{ServletRequest req, ServletResponse res} \\ \hline
\textbf{destroy()} & Servlet cleanup & કોઈ નહીં \\ \hline
\end{tabulary}
\end{center}

\textbf{Method વિગતો:}
\begin{itemize}
    \item \keyword{init(ServletConfig config)}: ઇનિશિયલાઇઝેશન કોડ, ડેટાબેસ કનેક્શન્સ.
    \item \keyword{service(req, res)}: રિક્વેસ્ટ હેન્ડલિંગ, બિઝનેસ લોજિક.
    \item \keyword{destroy()}: cleanup કોડ, કનેક્શન્સ બંધ કરવા.
\end{itemize}

\textbf{ઉદાહરણ:}
\begin{lstlisting}[language=Java]
public void init(ServletConfig config) {
    // ડેટાબેસ કનેક્શન ઇનિશિયલાઇઝ કરો
}

public void service(ServletRequest req, ServletResponse res) {
    // રિક્વેસ્ટ હેન્ડલ કરો
}

public void destroy() {
    // કનેક્શન્સ બંધ કરો
}
\end{lstlisting}
\end{solutionbox}

\begin{mnemonicbox}
\mnemonic{Init-Service-Destroy: ISD}
\end{mnemonicbox}

\questionmarks{3(b OR)}{4}{ઉદાહરણ સાથે HTTPSERVLET class સમજાવો.}

\begin{solutionbox}
\textbf{HttpServlet Class:}
HTTP પ્રોટોકોલ માટે ખાસ કરીને GenericServlet ને વિસ્તૃત કરતો abstract class.

\textbf{HTTP Methods:}

\begin{center}
\captionof{table}{HttpServlet Methods}
\begin{tabulary}{\linewidth}{|L|L|L|}
\hline
\textbf{Method} & \textbf{HTTP Verb} & \textbf{હેતુ} \\ \hline
\textbf{doGet()} & GET & ડેટા પુનઃપ્રાપ્ત કરવું \\ \hline
\textbf{doPost()} & POST & ડેટા સબમિટ કરવું \\ \hline
\textbf{doPut()} & PUT & ડેટા અપડેટ કરવું \\ \hline
\textbf{doDelete()} & DELETE & ડેટા દૂર કરવું \\ \hline
\end{tabulary}
\end{center}

\textbf{ઉદાહરણ:}
\begin{lstlisting}[language=Java]
public class MyServlet extends HttpServlet {
    protected void doGet(HttpServletRequest request,
                        HttpServletResponse response) {
        response.setContentType("text/html");
        PrintWriter out = response.getWriter();
        out.println("<h1>GET Request</h1>");
    }
    
    protected void doPost(HttpServletRequest request,
                         HttpServletResponse response) {
        String name = request.getParameter("name");
        response.getWriter().println("નમસ્કાર " + name);
    }
}
\end{lstlisting}

\textbf{મુખ્ય વિશેષતાઓ:}
\begin{itemize}
    \item \keyword{HTTP-વિશિષ્ટ}: વેબ એપ્લિકેશન માટે ડિઝાઇન કરેલ.
    \item \keyword{Method handling}: વિવિધ HTTP verbs માટે અલગ methods.
    \item \keyword{Request/Response}: HttpServletRequest અને HttpServletResponse.
\end{itemize}
\end{solutionbox}

\begin{mnemonicbox}
\mnemonic{Get-Post-Put-Delete: GPPD}
\end{mnemonicbox}

\questionmarks{3(c OR)}{7}{GET અને POST method નો તફાવત લખો અને POST method નો ઉપયોગ કરીને Servlet બનવા માટેનો java code લખો.}

\begin{solutionbox}
\begin{center}
\captionof{table}{GET વિ. POST તુલના}
\begin{tabulary}{\linewidth}{|L|L|L|}
\hline
\textbf{વિશેષતા} & \textbf{GET} & \textbf{POST} \\ \hline
\textbf{ડેટા સ્થાન} & URL parameters & Request body \\ \hline
\textbf{ડેટા મર્યાદા} & મર્યાદિત ($\sim$2KB) & અમર્યાદિત \\ \hline
\textbf{સુરક્ષા} & ઓછી સુરક્ષિત & વધુ સુરક્ષિત \\ \hline
\textbf{કેશિંગ} & કેશેબલ & કેશેબલ નથી \\ \hline
\textbf{બુકમાર્કિંગ} & શક્ય & શક્ય નથી \\ \hline
\end{tabulary}
\end{center}

\textbf{POST Method Servlet ઉદાહરણ:}
\begin{lstlisting}[language=Java]
import javax.servlet.*;
import javax.servlet.http.*;
import java.io.*;

public class LoginServlet extends HttpServlet {
    protected void doPost(HttpServletRequest request,
                         HttpServletResponse response) 
                         throws ServletException, IOException {
        
        response.setContentType("text/html");
        PrintWriter out = response.getWriter();
        
        // ફોર્મ ડેટા મેળવવું
        String username = request.getParameter("username");
        String password = request.getParameter("password");
        
        // ઓળખપત્રો ચકાસવા
        if("admin".equals(username) && "123".equals(password)) {
            out.println("<h2>લોગિન સફળ!</h2>");
            out.println("<p>આવકાર છે " + username + "</p>");
        } else {
            out.println("<h2>લોગિન નિષ્ફળ!</h2>");
            out.println("<p>અયોગ્ય ઓળખપત્રો</p>");
        }
        
        out.close();
    }
}
\end{lstlisting}

\textbf{HTML ફોર્મ:}
\begin{lstlisting}[language=HTML]
<form method="post" action="LoginServlet">
    વપરાશકર્તા નામ: <input type="text" name="username"><br />
    પાસવર્ડ: <input type="password" name="password"><br />
    <input type="submit" value="લોગિન">
</form>
\end{lstlisting}

\textbf{મુખ્ય તફાવતો:}
\begin{itemize}
    \item \keyword{GET}: ડેટા URL માં, દેખાય છે, મર્યાદિત કદ.
    \item \keyword{POST}: ડેટા body માં, છુપાયેલ, અમર્યાદિત કદ.
\end{itemize}
\end{solutionbox}

\begin{mnemonicbox}
\mnemonic{GET મેળવે, POST સુરક્ષિત કરે}
\end{mnemonicbox}

\questionmarks{4(a)}{3}{JSP Implicit Objects ની યાદી બનાવો અને કોઈપણ બે સમજાવો.}

\begin{solutionbox}
\textbf{JSP Implicit Objects યાદી:}
\begin{enumerate}
    \item \textbf{request} (HttpServletRequest)
    \item \textbf{response} (HttpServletResponse)
    \item \textbf{session} (HttpSession)
    \item \textbf{application} (ServletContext)
    \item \textbf{out} (JspWriter)
    \item \textbf{page} (Object)
    \item \textbf{pageContext} (PageContext)
    \item \textbf{config} (ServletConfig)
    \item \textbf{exception} (Throwable)
\end{enumerate}

\textbf{વિગતવાર સમજૂતી:}

\textbf{request Object:}
\begin{itemize}
    \item \keyword{પ્રકાર}: HttpServletRequest
    \item \keyword{હેતુ}: રિક્વેસ્ટ ડેટા અને parameters ને એક્સેસ કરવું.
    \item \keyword{ઉદાહરણ}: \code{String name = request.getParameter("name");}
\end{itemize}

\textbf{session Object:}
\begin{itemize}
    \item \keyword{પ્રકાર}: HttpSession
    \item \keyword{હેતુ}: રિક્વેસ્ટ્સ પર વપરાશકર્તા-વિશિષ્ટ ડેટા સ્ટોર કરવું.
    \item \keyword{ઉદાહરણ}: \code{session.setAttribute("user", username);}
\end{itemize}
\end{solutionbox}

\begin{mnemonicbox}
\mnemonic{Request Response Session Application Out}
\end{mnemonicbox}

\questionmarks{4(b)}{4}{JSP ની વિવિધ વિશેષતાઓ સમજાવો.}

\begin{solutionbox}
\begin{center}
\captionof{table}{JSP વિશેષતાઓ}
\begin{tabulary}{\linewidth}{|L|L|L|}
\hline
\textbf{વિશેષતા} & \textbf{વર્ણન} & \textbf{ફાયદો} \\ \hline
\textbf{સહેલું ડેવલપમેન્ટ} & HTML + Java & ઝડપી કોડિંગ \\ \hline
\textbf{પ્લેટફોર્મ સ્વતંત્ર} & એકવાર લખો, ગમે ત્યાં ચલાવો & પોર્ટેબિલિટી \\ \hline
\textbf{કોમ્પોનેન્ટ-આધારિત} & પુનઃઉપયોગ્ય કોમ્પોનેન્ટ્સ & જાળવણીયોગ્યતા \\ \hline
\textbf{સુરક્ષિત} & Java સિક્યોરિટી મોડલ & સુરક્ષિત એક્ઝિક્યુશન \\ \hline
\end{tabulary}
\end{center}

\textbf{મુખ્ય વિશેષતાઓ:}
\begin{itemize}
    \item \keyword{Separation of Concerns}: ડિઝાઇન અને લોજિક અલગ.
    \item \keyword{વિસ્તરણીય}: કસ્ટમ ટેગ્સ અને લાઇબ્રેરીઓ.
    \item \keyword{કમ્પાઇલ્ડ}: કાર્યક્ષમતા માટે servlets માં ટ્રાન્સલેટ.
    \item \keyword{Expression Language}: સરળ સિન્ટેક્સ.
\end{itemize}

\textbf{JSP એલિમેન્ટ્સ:}
\begin{itemize}
    \item \keyword{Directives}: \code{<\%@ \%>}
    \item \keyword{Declarations}: \code{<\%! \%>}
    \item \keyword{Expressions}: \code{<\%= \%>}
    \item \keyword{Scriptlets}: \code{<\% \%>}
\end{itemize}
\end{solutionbox}

\begin{mnemonicbox}
\mnemonic{Easy Platform Component Secure}
\end{mnemonicbox}

\questionmarks{4(c)}{7}{Servlet માંથી JSP કઇ રીતે કોલ થશે તે ઉદાહરણ સાથે સમજાવો.}

\begin{solutionbox}
\textbf{Servlet માંથી JSP કોલ કરવાની પદ્ધતિઓ:}

\begin{center}
\captionof{table}{JSP કોલિંગ પદ્ધતિઓ}
\begin{tabulary}{\linewidth}{|L|L|L|}
\hline
\textbf{પદ્ધતિ} & \textbf{Interface} & \textbf{હેતુ} \\ \hline
\textbf{Forward} & RequestDispatcher & કંટ્રોલ ટ્રાન્સફર કરવું \\ \hline
\textbf{Include} & RequestDispatcher & કન્ટેન્ટ ઇન્ક્લુડ કરવું \\ \hline
\textbf{Redirect} & HttpServletResponse & નવી રિક્વેસ્ટ \\ \hline
\end{tabulary}
\end{center}

\textbf{Forward ઉદાહરણ:}

\textbf{Servlet કોડ:}
\begin{lstlisting}[language=Java]
public class DataServlet extends HttpServlet {
    protected void doGet(HttpServletRequest request,
                        HttpServletResponse response) 
                        throws ServletException, IOException {
        
        // ડેટા પ્રોસેસ કરવું
        String username = "જોન ડો";
        int age = 25;
        
        // Attributes સેટ કરવા
        request.setAttribute("username", username);
        request.setAttribute("age", age);
        
        // JSP ને ફોરવર્ડ કરવું
        RequestDispatcher dispatcher = 
            request.getRequestDispatcher("display.jsp");
        dispatcher.forward(request, response);
    }
}
\end{lstlisting}

\textbf{JSP કોડ (display.jsp):}
\begin{lstlisting}[language=Java]
<%@ page language="java" contentType="text/html" %>
<html>
<head><title>વપરાશકર્તા માહિતી</title></head>
<body>
    <h2>વપરાશકર્તા માહિતી</h2>
    <p>નામ: <%= request.getAttribute("username") %></p>
    <p>ઉંમર: <%= request.getAttribute("age") %></p>
</body>
</html>
\end{lstlisting}

\textbf{પગલાઓ:}
\begin{enumerate}
    \item \textbf{ડેટા પ્રોસેસ} servlet માં.
    \item \textbf{Attributes સેટ} કરવા request માં.
    \item \textbf{RequestDispatcher મેળવવું} JSP path સાથે.
    \item \textbf{Forward} JSP ને.
\end{enumerate}
\end{solutionbox}

\begin{mnemonicbox}
\mnemonic{Process-Set-Get-Forward: PSGF}
\end{mnemonicbox}

\questionmarks{4(a OR)}{3}{JSP scripting elements ની યાદી બનાવો અને સમજાવો.}

\begin{solutionbox}
\begin{center}
\captionof{table}{JSP Scripting Elements}
\begin{tabulary}{\linewidth}{|L|L|L|L|}
\hline
\textbf{Element} & \textbf{Syntax} & \textbf{હેતુ} & \textbf{ઉદાહરણ} \\ \hline
\textbf{Directive} & \code{<\%@ \%>} & પેજ સેટિંગ્સ & \code{<\%@ page import... \%>} \\ \hline
\textbf{Declaration} & \code{<\%! \%>} & methods/variables વ્યાખ્યા & \code{<\%! int count = 0; \%>} \\ \hline
\textbf{Expression} & \code{<\%= \%>} & વેલ્યુઝ આઉટપુટ & \code{<\%= new Date() \%>} \\ \hline
\textbf{Scriptlet} & \code{<\% \%>} & Java કોડ & \code{<\% for(int i=0... \%>} \\ \hline
\end{tabulary}
\end{center}

\textbf{વિગતવાર સમજૂતી:}
\begin{itemize}
    \item \keyword{Directives}: Page directive, Include directive, Taglib directive.
    \item \keyword{Declarations}: Instance variables અને methods વ્યાખ્યાયિત કરવા. servlet class ના ભાગ બને છે.
\end{itemize}
\end{solutionbox}

\begin{mnemonicbox}
\mnemonic{Direct Declare Express Script}
\end{mnemonicbox}

\questionmarks{4(b OR)}{4}{JSP life cycle સમજાવો.}

\begin{solutionbox}
\begin{center}
\begin{tikzpicture}[node distance=1.5cm, auto]
    \node [gtu state] (Req) {Request};
    \node [gtu state, right=of Req] (Trans) {Translation};
    \node [gtu state, right=of Trans] (Comp) {Compilation};
    \node [gtu state, below=of Comp] (Load) {Loading};
    \node [gtu state, left=of Load] (Init) {jspInit()};
    \node [gtu state, left=of Init] (Serv) {\_jspService()};
    \node [gtu state, below=of Serv] (Dest) {jspDestroy()};
    
    \path [gtu arrow] (Req) -- (Trans);
    \path [gtu arrow] (Trans) -- (Comp);
    \path [gtu arrow] (Comp) -- (Load);
    \path [gtu arrow] (Load) -- (Init);
    \path [gtu arrow] (Init) -- (Serv);
    \path [gtu arrow] (Serv) edge [loop below] node {Requests} (Serv);
    \path [gtu arrow] (Serv) -- (Dest);
\end{tikzpicture}
\captionof{figure}{JSP Life Cycle}
\end{center}

\begin{center}
\captionof{table}{JSP લાઇફ સાઇકલ તબક્કાઓ}
\begin{tabulary}{\linewidth}{|L|L|L|}
\hline
\textbf{તબક્કો} & \textbf{Method} & \textbf{હેતુ} \\ \hline
\textbf{Translation} & - & JSP થી Java servlet \\ \hline
\textbf{Compilation} & - & Java થી bytecode \\ \hline
\textbf{Initialization} & \code{jspInit()} & રિસોર્સ સેટઅપ \\ \hline
\textbf{Request Processing} & \code{\_jspService()} & રિક્વેસ્ટ હેન્ડલ કરવા \\ \hline
\textbf{Destruction} & \code{jspDestroy()} & Cleanup \\ \hline
\end{tabulary}
\end{center}

\textbf{મુખ્ય મુદ્દાઓ:}
\begin{itemize}
    \item \keyword{Translation}: JSP engine JSP ને servlet માં કન્વર્ટ કરે.
    \item \keyword{Compilation}: Java compiler .class ફાઇલ બનાવે.
    \item \keyword{Execution}: Servlet container કમ્પાઇલ થયેલ servlet એક્ઝિક્યુટ કરે.
\end{itemize}
\end{solutionbox}

\begin{mnemonicbox}
\mnemonic{Translate-Compile-Init-Service-Destroy}
\end{mnemonicbox}

\questionmarks{4(c OR)}{7}{Cookie શું છે? ઉદાહરણ સાથે cookie નું working સમજાવો.}

\begin{solutionbox}
\textbf{Cookie વ્યાખ્યા:}
Cookie એ વેબ બ્રાઉઝર દ્વારા ક્લાયન્ટના કમ્પ્યુટર પર સ્ટોર કરવામાં આવતા ડેટાનો નાનો ભાગ છે.

\textbf{Cookie Working Process:}
\begin{itemize}
    \item Client HTTP Request મોકલે છે.
    \item Server HTTP Response + \code{Set-Cookie} મોકલે છે.
    \item Client HTTP Request + \code{Cookie} મોકલે છે.
    \item Server HTTP Response (cookie ડેટા વાપરે) મોકલે છે.
\end{itemize}

\begin{center}
\captionof{table}{Cookie Attributes}
\begin{tabulary}{\linewidth}{|L|L|L|}
\hline
\textbf{Attribute} & \textbf{હેતુ} & \textbf{ઉદાહરણ} \\ \hline
\textbf{Name} & Cookie identifier & \code{username} \\ \hline
\textbf{Value} & Cookie ડેટા & \code{john123} \\ \hline
\textbf{Domain} & વેલિડ ડોમેન & \code{.example.com} \\ \hline
\textbf{Path} & વેલિડ પાથ & \code{/shop/} \\ \hline
\textbf{Max-Age} & એક્સપાયરી ટાઇમ & \code{3600} સેકંડ \\ \hline
\end{tabulary}
\end{center}

\textbf{Cookie ઉદાહરણ:}

\textbf{Cookie બનાવવું (Servlet):}
\begin{lstlisting}[language=Java]
public class SetCookieServlet extends HttpServlet {
    protected void doGet(HttpServletRequest request,
                        HttpServletResponse response) {
        
        // Cookie બનાવવી
        Cookie userCookie = new Cookie("username", "john123");
        userCookie.setMaxAge(60 * 60 * 24); // 1 દિવસ
        userCookie.setPath("/");
        
        // Response માં ઉમેરવી
        response.addCookie(userCookie);
        
        response.getWriter().println("Cookie સફળતાપૂર્વક સેટ થઈ!");
    }
}
\end{lstlisting}

\textbf{Cookie વાંચવી (Servlet):}
\begin{lstlisting}[language=Java]
public class GetCookieServlet extends HttpServlet {
    protected void doGet(HttpServletRequest request,
                        HttpServletResponse response) {
        
        Cookie[] cookies = request.getCookies();
        String username = null;
        
        if(cookies != null) {
            for(Cookie cookie : cookies) {
                if("username".equals(cookie.getName())) {
                    username = cookie.getValue();
                    break;
                }
            }
        }
        
        response.getWriter().println("પાછા આવવા બદલ આભાર, " + username);
    }
}
\end{lstlisting}

\textbf{Cookie ફાયદા:}
\begin{itemize}
    \item \keyword{વપરાશકર્તા વ્યક્તિકરણ}: પસંદગીઓ યાદ રાખવી.
    \item \keyword{Session tracking}: state જાળવવા.
    \item \keyword{Analytics}: વપરાશકર્તા વર્તન ટ્રેક કરવું.
\end{itemize}
\end{solutionbox}

\begin{mnemonicbox}
\mnemonic{Create-Set-Add-Read: CSAR}
\end{mnemonicbox}

\questionmarks{5(a)}{3}{JSP અને Servlet વચ્ચેનો તફાવત લખો.}

\begin{solutionbox}
\begin{center}
\captionof{table}{JSP વિ. Servlet તુલના}
\begin{tabulary}{\linewidth}{|L|L|L|}
\hline
\textbf{વિશેષતા} & \textbf{JSP} & \textbf{Servlet} \\ \hline
\textbf{ડેવલપમેન્ટ} & HTML + Java & શુદ્ધ Java \\ \hline
\textbf{કમ્પાઇલેશન} & ઑટોમેટિક & મેન્યુઅલ \\ \hline
\textbf{જાળવણી} & સરળ & વધુ જટિલ \\ \hline
\textbf{કાર્યક્ષમતા} & ધીમી (પ્રથમ રિક્વેસ્ટ) & ઝડપી \\ \hline
\textbf{હેતુ} & Presentation layer & Business logic \\ \hline
\end{tabulary}
\end{center}

\textbf{મુખ્ય તફાવતો:}
\begin{itemize}
    \item \keyword{JSP}: presentation માટે સારી, વેબ ડિઝાઇનર્સ માટે સરળ.
    \item \keyword{Servlet}: business logic માટે સારી, વધુ કંટ્રોલ.
    \item \keyword{કોડિંગ}: JSP HTML અને Java મિક્સ કરે, Servlet શુદ્ધ Java.
\end{itemize}
\end{solutionbox}

\begin{mnemonicbox}
\mnemonic{JSP Presentation માટે, Servlet Logic માટે}
\end{mnemonicbox}

\questionmarks{5(b)}{4}{Spring Boot શું છે? અને તેના ફાયદા સમજાવો.}

\begin{solutionbox}
\textbf{Spring Boot વ્યાખ્યા:}
Spring Boot એ ફ્રેમવર્ક છે જે ઑટો-કોન્ફિગરેશન અને embedded servers પ્રદાન કરીને Spring-આધારિત એપ્લિકેશનોનું ડેવલપમેન્ટ સરળ બનાવે છે.

\begin{center}
\captionof{table}{Spring Boot ફાયદા}
\begin{tabulary}{\linewidth}{|L|L|}
\hline
\textbf{ફાયદો} & \textbf{વર્ણન} \\ \hline
\textbf{ઑટો કોન્ફિગરેશન} & Spring એપ્લિકેશનો ઑટોમેટિક કોન્ફિગર કરે \\ \hline
\textbf{Embedded Servers} & બિલ્ટ-ઇન Tomcat, Jetty સપોર્ટ \\ \hline
\textbf{Starter Dependencies} & પ્રી-કોન્ફિગર્ડ dependency સેટ્સ \\ \hline
\textbf{Production Ready} & Health checks, metrics, monitoring \\ \hline
\end{tabulary}
\end{center}

\textbf{મુખ્ય વિશેષતાઓ:}
\begin{itemize}
    \item \keyword{ઝડપી ડેવલપમેન્ટ}: ન્યૂનતમ કોન્ફિગરેશન જરૂરી.
    \item \keyword{Microservices}: microservice આર્કિટેક્ચર માટે પરફેક્ટ.
    \item \keyword{XML નહીં}: Convention over configuration.
    \item \keyword{Cloud Ready}: cloud platforms પર સરળ deployment.
\end{itemize}

\textbf{ઉદાહરણ:}
\begin{lstlisting}[language=Java]
@SpringBootApplication
public class MyApplication {
    public static void main(String[] args) {
        SpringApplication.run(MyApplication.class, args);
    }
}
\end{lstlisting}
\end{solutionbox}

\begin{mnemonicbox}
\mnemonic{Auto Embedded Starter Production}
\end{mnemonicbox}

\questionmarks{5(c)}{7}{Spring framework નું આર્કિટેક્ચર સમજાવો.}

\begin{solutionbox}
\begin{center}
\begin{tikzpicture}[node distance=1.5cm, auto]
    \node [gtu block] (App) {Spring Framework Architecture};
    \node [gtu block, below left=of App] (Core) {Core Container};
    \node [gtu block, below=of App] (Data) {Data Access};
    \node [gtu block, below right=of App] (Web) {Web MVC};
    \node [gtu block, below=of Core] (AOP) {AOP};
    \node [gtu block, below=of Web] (Test) {Test};
    
    \path [gtu arrow] (App) -- (Core);
    \path [gtu arrow] (App) -- (Data);
    \path [gtu arrow] (App) -- (Web);
    \path [gtu arrow] (App) -- (AOP);
    \path [gtu arrow] (App) -- (Test);
\end{tikzpicture}
\captionof{figure}{Spring Framework Architecture}
\end{center}

\textbf{ટેબલ: Spring Framework Modules}
\begin{center}
\begin{tabulary}{\linewidth}{|L|L|L|}
\hline
\textbf{Module} & \textbf{Components} & \textbf{હેતુ} \\ \hline
\textbf{Core Container} & Core, Beans, Context & IoC અને DI \\ \hline
\textbf{Data Access} & JDBC, ORM, JMS & ડેટાબેસ ઓપરેશન્સ \\ \hline
\textbf{Web MVC} & Web, Servlet, MVC & વેબ એપ્લિકેશન્સ \\ \hline
\textbf{AOP} & Aspects, Weaving & Cross-cutting concerns \\ \hline
\end{tabulary}
\end{center}

\textbf{મુખ્ય કન્સેપ્ટ્સ:}
\begin{itemize}
    \item \keyword{IoC (Inversion of Control)}: ફ્રેમવર્ક ઑબ્જેક્ટ બનાવટ કંટ્રોલ કરે.
    \item \keyword{DI (Dependency Injection)}: Dependencies ઑટોમેટિક inject થાય.
    \item \keyword{AOP}: Modular cross-cutting concerns.
    \item \keyword{MVC}: Model-View-Controller પેટર્ન.
\end{itemize}
\end{solutionbox}

\begin{mnemonicbox}
\mnemonic{Core Data Web AOP Test}
\end{mnemonicbox}

\questionmarks{5(a OR)}{3}{Servlet ની સરખામણીમાં JSP ના ફાયદા લખો.}

\begin{solutionbox}
\begin{center}
\captionof{table}{Servlet પર JSP ના ફાયદા}
\begin{tabulary}{\linewidth}{|L|L|L|}
\hline
\textbf{ફાયદો} & \textbf{JSP} & \textbf{Servlet મર્યાદા} \\ \hline
\textbf{સહેલું ડેવલપમેન્ટ} & HTML + Java ટેગ્સ & Java માં જટિલ HTML \\ \hline
\textbf{ઑટોમેટિક કમ્પાઇલેશન} & ઑટો-કમ્પાઇલ્ડ & મેન્યુઅલ કમ્પાઇલેશન \\ \hline
\textbf{ડિઝાઇનર ફ્રેન્ડલી} & વેબ ડિઝાઇનર્સ કામ કરી શકે & Java જ્ઞાન જરૂરી \\ \hline
\textbf{જાળવણી} & સુધારવા સરળ & કોડ બદલાવ માટે recompilation \\ \hline
\end{tabulary}
\end{center}

\textbf{મુખ્ય ફાયદા:}
\begin{itemize}
    \item \keyword{ડિઝાઇન અને લોજિકનું વિભાજન}: HTML અને Java અલગ.
    \item \keyword{ઝડપી ડેવલપમેન્ટ}: ઝડપી prototyping અને development.
    \item \keyword{ઓછો કોડ}: out.println() statements ની જરૂર નહીં.
\end{itemize}
\end{solutionbox}

\begin{mnemonicbox}
\mnemonic{Easy Auto Designer Maintenance}
\end{mnemonicbox}

\questionmarks{5(b OR)}{4}{Spring Boot ના ફાયદા સમજાવો.}

\begin{solutionbox}
\begin{center}
\captionof{table}{Spring Boot ફાયદા}
\begin{tabulary}{\linewidth}{|L|L|L|}
\hline
\textbf{ફાયદો} & \textbf{વર્ણન} & \textbf{લાભ} \\ \hline
\textbf{ઑટો કોન્ફિગરેશન} & Classpath આધારે ઑટોમેટિક સેટઅપ & ઘટેલ કોન્ફિગરેશન \\ \hline
\textbf{Embedded Server} & બિલ્ટ-ઇન Tomcat/Jetty & બાહ્ય deployment નહીં \\ \hline
\textbf{Starter POMs} & પ્રી-કોન્ફિગર્ડ dependencies & સરળ dependency management \\ \hline
\textbf{Actuator} & Production monitoring & Health checks અને metrics \\ \hline
\end{tabulary}
\end{center}

\textbf{વિગતવાર ફાયદા:}
\begin{enumerate}
    \item \textbf{ઑટો કોન્ફિગરેશન}: Dependencies આધારે Spring application ઑટોમેટિક કોન્ફિગર કરે.
    \item \textbf{Embedded Servers}: બાહ્ય application servers ની જરૂર નહીં. ચલાવવા સરળ.
    \item \textbf{Starter Dependencies}: પ્રી-કોન્ફિગર્ડ dependency સેટ્સ.
    \item \textbf{Production ફીચર્સ}: Health endpoints, metrics collection.
\end{enumerate}
\end{solutionbox}

\begin{mnemonicbox}
\mnemonic{Auto Embedded Starter Production}
\end{mnemonicbox}

\questionmarks{5(c OR)}{7}{MVC આર્કિટેક્ચર સમજાવો.}

\begin{solutionbox}
\textbf{MVC (Model-View-Controller) આર્કિટેક્ચર:}

\begin{center}
\begin{tikzpicture}[node distance=2cm, auto]
    \node [gtu block] (View) {View};
    \node [gtu block, right=of View] (Cont) {Controller};
    \node [gtu block, right=of Cont] (Model) {Model};
    \node [gtu block, below=of View] (User) {User};
    
    \path [gtu arrow] (User) -- (View);
    \path [gtu arrow] (View) -- (Cont);
    \path [gtu arrow] (Cont) -- (Model);
    \path [gtu arrow] (Model) -- (Cont);
    \path [gtu arrow] (Cont) -- (View);
    \path [gtu arrow] (View) -- (User);
\end{tikzpicture}
\captionof{figure}{MVC આર્કિટેક્ચર}
\end{center}

\begin{center}
\captionof{table}{MVC કોમ્પોનેન્ટ્સ}
\begin{tabulary}{\linewidth}{|L|L|L|}
\hline
\textbf{Component} & \textbf{જવાબદારી} & \textbf{ઉદાહરણ} \\ \hline
\textbf{Model} & ડેટા અને બિઝનેસ લોજિક & Entity classes, DAOs \\ \hline
\textbf{View} & વપરાશકર્તા ઇન્ટરફેસ & JSP, HTML, Templates \\ \hline
\textbf{Controller} & રિક્વેસ્ટ હેન્ડલિંગ & Servlets, Spring Controllers \\ \hline
\end{tabulary}
\end{center}

\textbf{MVC ફ્લો:}
\begin{enumerate}
    \item વપરાશકર્તા Input View ને મોકલે છે.
    \item View Controller ને રિક્વેસ્ટ મોકલે છે.
    \item Controller Model સાથે ડેટા પ્રોસેસ કરે છે.
    \item Model Controller ને ડેટા રિટર્ન કરે છે.
    \item Controller View પસંદ કરે છે.
    \item View વપરાશકર્તા ને Response આપે છે.
\end{enumerate}

\textbf{Spring MVC ઉદાહરણ:}

\textbf{Controller:}
\begin{lstlisting}[language=Java]
@Controller
public class StudentController {
    @Autowired
    private StudentService studentService;
    
    @GetMapping("/students")
    public ModelAndView getStudents() {
        List<Student> students = studentService.getAllStudents();
        ModelAndView mv = new ModelAndView("students");
        mv.addObject("studentList", students);
        return mv;
    }
}
\end{lstlisting}

\textbf{MVC ફાયદા:}
\begin{itemize}
    \item \keyword{Separation of Concerns}: જવાબદારીઓનું સ્પષ્ટ વિભાજન.
    \item \keyword{જાળવણીયોગ્યતા}: જાળવવા અને સુધારવા સરળ.
    \item \keyword{ટેસ્ટેબિલિટી}: દરેક કોમ્પોનેન્ટ સ્વતંત્ર રીતે ટેસ્ટ કરી શકાય.
\end{itemize}
\end{solutionbox}

\begin{mnemonicbox}
\mnemonic{Model ડેટા મેનેજ કરે, View ડેટા દર્શાવે, Controller ફ્લો કંટ્રોલ કરે}
\end{mnemonicbox}

\end{document}
