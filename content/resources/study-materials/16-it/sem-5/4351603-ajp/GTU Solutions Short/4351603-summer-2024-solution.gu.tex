\documentclass{article}

% content/resources/templates/preamble.tex
\usepackage[margin=0.6in]{geometry}
\author{Milav Dabgar}
\usepackage{amsmath,amssymb,amsthm}
\usepackage{booktabs}
\usepackage{multirow}
\usepackage{xcolor}
\usepackage{tcolorbox}
\tcbuselibrary{breakable,skins}
\usepackage[colorlinks=true,linkcolor=blue]{hyperref}
\usepackage{titlesec}
\usepackage{enumitem}
\usepackage{tikz}
\usepackage{pgfplots}
\usepackage{circuitikz}
\usepackage[version=4]{mhchem}
\usepackage{longtable}
\usepackage{array}
\usepackage{float}
\usepackage{caption}
\usepackage{listings}

\lstset{
  basicstyle=\small\ttfamily,
  breaklines=true,
  breakatwhitespace=false,
  postbreak=\mbox{\textcolor{red}{$\hookrightarrow$}\space},
  float=false,
  numbers=left,
  numberstyle=\tiny\color{gray},
  numbersep=10pt,
  xleftmargin=2em,
  keywordstyle=\color{blue},
  commentstyle=\color{green!60!black},
  stringstyle=\color{purple},
  backgroundcolor=\color{gray!5},
  showstringspaces=false,
  tabsize=2,
  captionpos=b,
  keepspaces=true,
  columns=flexible
}

\pgfplotsset{compat=1.18}
\usetikzlibrary{shapes,arrows,positioning,calc,patterns,decorations.pathmorphing,decorations.markings,arrows.meta}

% Color scheme
\definecolor{headcolor}{RGB}{0,102,204}
\definecolor{keycolor}{RGB}{220,20,60}
\definecolor{solutioncolor}{RGB}{34,139,34}
\definecolor{mnemoniccolor}{RGB}{148,0,211}
\definecolor{codecolor}{RGB}{0,0,100}

% Spacing
\setlength{\parskip}{3pt}
\setlist[itemize]{nosep}
\setlist[enumerate]{nosep}

% Title formatting
\titleformat{\section}{\Large\bfseries\color{headcolor}}{\thesection}{1em}{}
\titleformat{\subsection}{\large\bfseries\color{headcolor}}{\thesubsection}{1em}{}

% Pandoc tightlist compatibility
\providecommand{\tightlist}{%
  \setlength{\itemsep}{0pt}\setlength{\parskip}{0pt}}

% Pandoc longtable compatibility
\newcounter{none}
\def\thenone{}


% content/resources/templates/gujarati-boxes.tex
\usepackage{fontspec}
\usepackage{polyglossia}

% Set Gujarati as main language (document is primarily in Gujarati)
% Note: gloss-gujarati.ldf doesn't exist in polyglossia, but it will use hyphenation patterns
\setdefaultlanguage{gujarati}
\setotherlanguage{english}

% Configure Gujarati font properly
% Use Language=Default to prevent polyglossia from trying to add language-specific features
% that don't exist for Gujarati, which causes "empty feature" warnings
\newfontfamily\gujaratifont[Script=Gujarati,AutoFakeBold=2.5,AutoFakeSlant=0.3]{Noto Sans Gujarati}
\setmainfont[Script=Gujarati,AutoFakeBold=2.5,AutoFakeSlant=0.3]{Noto Sans Gujarati}
% Use Noto Sans Gujarati for monospace to support Gujarati in text
\setmonofont[Scale=0.9]{Noto Sans Gujarati}

% Configure English to use the same font
\newfontfamily\englishfont[Script=Gujarati,AutoFakeBold=2.5,AutoFakeSlant=0.3]{Noto Sans Gujarati}

% Translations for polyglossia
\gappto\captionsgujarati{
  \renewcommand{\tablename}{કોષ્ટક}
  \renewcommand{\figurename}{આકૃતિ}
}

% Helper for TikZ nodes to ensure Gujarati font
\newcommand{\gu}[1]{{\gujaratifont #1}}

% Custom environments
\newtcolorbox{solutionbox}{
    breakable,
    enhanced,
    colback=solutioncolor!5!white,
    colframe=solutioncolor!75!black,
    fonttitle=\bfseries,
    title=જવાબ
}

\newtcolorbox{solutionboxnobreak}{
 colback=solutioncolor!5!white,
 colframe=solutioncolor!75!black,
 fonttitle=\bfseries,
 title=જવાબ
}

\newtcolorbox{keyformula}{
 breakable,
 enhanced,
 colback=keycolor!5!white,
 colframe=keycolor!75!black,
 fonttitle=\bfseries,
 title=રાસાયણિક સમીકરણ/સૂત્ર
}

\newtcolorbox{mnemonicbox}{
 breakable,
 enhanced,
 colback=mnemoniccolor!5!white,
 colframe=mnemoniccolor!75!black,
 fonttitle=\bfseries,
 title=મેમરી ટ્રીક
}


% Custom commands for GTU solutions
% This file defines semantic commands for consistent formatting

% Question command with automatic formatting
\newcommand{\question}[2]{%
  \section*{Question #1}%
  \textbf{#2}%
}

% OR question variant
\newcommand{\questionor}[2]{%
  \section*{Question #1 OR}%
  \textbf{#2}%
}

% Proper table environment with caption
\newenvironment{answertable}[1]{%
  \begin{table}[htbp]
  \centering
  \caption{#1}
}{%
  \end{table}
}

% Proper figure environment for diagrams
\newenvironment{answerdiagram}[1]{%
  \begin{figure}[htbp]
  \centering
  \caption{#1}
}{%
  \end{figure}
}

% Semantic markup for key terms
\newcommand{\keyword}[1]{\textbf{#1}}
\newcommand{\code}[1]{\texttt{#1}}
\newcommand{\classname}[1]{\texttt{#1}}
\newcommand{\methodname}[1]{\texttt{#1}}

% Proper quotation marks
\newcommand{\mnemonic}[1]{``#1''}


\title{એડવાન્સ જાવા પ્રોગ્રામિંગ (4351603) - ઉનાળા 2024 સોલ્યુશન}
\date{મે 21, 2024}

\begin{document}
\maketitle

\questionmarks{1(અ)}{3}{AWT અને Swing વચ્ચેનો તફાવત સમજાવો.}

\begin{solutionbox}
\begin{center}
\captionof{table}{AWT vs Swing}
\begin{tabulary}{\linewidth}{|L|L|L|}
\hline
\textbf{લક્ષણ} & \textbf{AWT} & \textbf{Swing} \\ \hline
\textbf{Platform} & Platform dependent & Platform independent \\ \hline
\textbf{Components} & Heavy weight & Light weight \\ \hline
\textbf{Look \& Feel} & Native OS look & Pluggable look \& feel \\ \hline
\textbf{Performance} & ઝડપી & AWT કરતાં ધીમું \\ \hline
\end{tabulary}
\end{center}

\textbf{મુખ્ય મુદ્દાઓ:}
\begin{itemize}
    \item \keyword{Heavy vs Light}: AWT native OS components વાપરે છે, Swing pure Java વાપરે છે.
    \item \keyword{દેખાવ}: AWT OS style અનુસરે છે, Swing બધા platforms પર સમાન look આપે છે.
    \item \keyword{સુવિધાઓ}: Swing વધુ advanced components જેમ કે JTable, JTree પ્રદાન કરે છે.
\end{itemize}
\end{solutionbox}

\begin{mnemonicbox}
\mnemonic{Swing Provides Lightweight Components}
\end{mnemonicbox}

\questionmarks{1(બ)}{4}{Mouse Motion Listener ને ઉદાહરણ સાથે સમજાવો.}

\begin{solutionbox}
MouseMotionListener interface Java Swing applications માં mouse movement events ને handle કરે છે.

\begin{center}
\captionof{table}{Mouse Motion Events}
\begin{tabulary}{\linewidth}{|L|L|}
\hline
\textbf{Method} & \textbf{હેતુ} \\ \hline
\textbf{mouseDragged()} & જ્યારે mouse drag થાય ત્યારે call થાય \\ \hline
\textbf{mouseMoved()} & જ્યારે mouse ખસે ત્યારે call થાય \\ \hline
\end{tabulary}
\end{center}

\textbf{કોડ ઉદાહરણ:}
\begin{lstlisting}[language=Java]
import javax.swing.*;
import java.awt.event.*;

class MouseMotionExample extends JFrame implements MouseMotionListener {
    JLabel label;
    
    MouseMotionExample() {
        label = new JLabel("અહીં mouse ખસાડો");
        add(label);
        addMouseMotionListener(this);
        setSize(400, 300);
        setVisible(true);
    }
    
    public void mouseMoved(MouseEvent e) {
        label.setText("Mouse આ સ્થાને: " + e.getX() + ", " + e.getY());
    }
    
    public void mouseDragged(MouseEvent e) {
        label.setText("Dragging આ સ્થાને: " + e.getX() + ", " + e.getY());
    }
}
\end{lstlisting}
\end{solutionbox}

\begin{mnemonicbox}
\mnemonic{Mouse Motion Makes Dynamic}
\end{mnemonicbox}

\questionmarks{1(ક)}{7}{યુનિવર્સિટી સાથે જોડાયેલા વિવિધ અભ્યાસક્રમો માટે checkboxes બનાવવા માટે એક પ્રોગ્રામ ડેવલપ કરો જેથી પસંદ કરેલ કોર્સ પ્રદર્શિત થાય.}

\begin{solutionbox}
\begin{lstlisting}[language=Java]
import javax.swing.*;
import java.awt.*;
import java.awt.event.*;

public class CourseSelection extends JFrame implements ItemListener {
    JCheckBox java, python, cpp, web;
    JTextArea display;
    
    public CourseSelection() {
        setTitle("યુનિવર્સિટી કોર્સ પસંદગી");
        setLayout(new FlowLayout());
        
        // checkboxes બનાવો
        java = new JCheckBox("Java Programming");
        python = new JCheckBox("Python Programming");
        cpp = new JCheckBox("C++ Programming");
        web = new JCheckBox("Web Development");
        
        // listeners ઉમેરો
        java.addItemListener(this);
        python.addItemListener(this);
        cpp.addItemListener(this);
        web.addItemListener(this);
        
        // Display area
        display = new JTextArea(10, 30);
        display.setEditable(false);
        
        // components ઉમેરો
        add(new JLabel("કોર્સ પસંદ કરો:"));
        add(java); add(python); add(cpp); add(web);
        add(new JScrollPane(display));
        
        setSize(400, 300);
        setDefaultCloseOperation(JFrame.EXIT_ON_CLOSE);
        setVisible(true);
    }
    
    public void itemStateChanged(ItemEvent e) {
        String courses = "પસંદ કરેલ કોર્સ:\n";
        if(java.isSelected()) courses += "- Java Programming\n";
        if(python.isSelected()) courses += "- Python Programming\n";
        if(cpp.isSelected()) courses += "- C++ Programming\n";
        if(web.isSelected()) courses += "- Web Development\n";
        display.setText(courses);
    }
    
    public static void main(String[] args) {
        new CourseSelection();
    }
}
\end{lstlisting}

\textbf{મુખ્ય લક્ષણો:}
\begin{itemize}
    \item \keyword{ItemListener}: checkbox state changes ને detect કરે છે.
    \item \keyword{Dynamic Display}: real-time માં પસંદ કરેલા કોર્સ update કરે છે.
    \item \keyword{Multiple Selection}: એકથી વધુ કોર્સ પસંદ કરવાની મંજૂરી આપે છે.
\end{itemize}
\end{solutionbox}

\begin{mnemonicbox}
\mnemonic{Check Items Listen Dynamically}
\end{mnemonicbox}

\questionmarks{1(ક OR)}{7}{Swing components નો ઉપયોગ કરીને (JFrame, JRadioButton, ItemListener વગેરેનો ઉપયોગ કરીને) Traffic signal (લાલ, લીલો અને પીળો) implement કરવા માટે એક પ્રોગ્રામ વિકસાવો.}

\begin{solutionbox}
\begin{lstlisting}[language=Java]
import javax.swing.*;
import java.awt.*;
import java.awt.event.*;

public class TrafficSignal extends JFrame implements ItemListener {
    JRadioButton red, green, yellow;
    ButtonGroup group;
    JPanel signalPanel;
    
    public TrafficSignal() {
        setTitle("Traffic Signal સિમ્યુલેટર");
        setLayout(new BorderLayout());
        
        // radio buttons બનાવો
        red = new JRadioButton("લાલ");
        green = new JRadioButton("લીલો"); 
        yellow = new JRadioButton("પીળો");
        
        // radio buttons ને group કરો
        group = new ButtonGroup();
        group.add(red); group.add(green); group.add(yellow);
        
        // listeners ઉમેરો
        red.addItemListener(this);
        green.addItemListener(this);
        yellow.addItemListener(this);
        
        // Signal display panel
        signalPanel = new JPanel() {
            public void paintComponent(Graphics g) {
                super.paintComponent(g);
                g.setColor(Color.BLACK);
                g.fillRect(50, 50, 100, 200);
                
                // વર્તુળો દોરો
                g.setColor(red.isSelected() ? Color.RED : Color.GRAY);
                g.fillOval(65, 65, 70, 70);
                
                g.setColor(yellow.isSelected() ? Color.YELLOW : Color.GRAY);
                g.fillOval(65, 105, 70, 70);
                
                g.setColor(green.isSelected() ? Color.GREEN : Color.GRAY);
                g.fillOval(65, 145, 70, 70);
            }
        };
        
        JPanel controlPanel = new JPanel();
        controlPanel.add(red); controlPanel.add(yellow); controlPanel.add(green);
        
        add(controlPanel, BorderLayout.SOUTH);
        add(signalPanel, BorderLayout.CENTER);
        
        setSize(300, 400);
        setDefaultCloseOperation(JFrame.EXIT_ON_CLOSE);
        setVisible(true);
    }
    
    public void itemStateChanged(ItemEvent e) {
        signalPanel.repaint();
    }
    
    public static void main(String[] args) {
        new TrafficSignal();
    }
}
\end{lstlisting}

\begin{center}
\begin{tikzpicture}
    % Traffic Box
    \draw[thick] (0,0) rectangle (2,4);
    % Circles
    \draw (1,3.2) circle (0.5) node {RED};
    \draw (1,2) circle (0.5) node {YEL};
    \draw (1,0.8) circle (0.5) node {GRN};
\end{tikzpicture}
\captionof{figure}{Traffic Signal Representation}
\end{center}
\end{solutionbox}

\begin{mnemonicbox}
\mnemonic{Radio Buttons Paint Graphics}
\end{mnemonicbox}

\questionmarks{2(અ)}{3}{JDBC Type-4 driver સમજાવો.}

\begin{solutionbox}
\textbf{JDBC Type-4 Driver (Native Protocol Driver)}

\begin{center}
\captionof{table}{JDBC Type-4 Driver}
\begin{tabulary}{\linewidth}{|L|L|}
\hline
\textbf{લક્ષણ} & \textbf{વર્ણન} \\ \hline
\textbf{પ્રકાર} & Pure Java driver \\ \hline
\textbf{Communication} & Direct database protocol \\ \hline
\textbf{Platform} & Platform independent \\ \hline
\textbf{Performance} & સર્વોચ્ચ પ્રદર્શન \\ \hline
\end{tabulary}
\end{center}

\textbf{મુખ્ય મુદ્દાઓ:}
\begin{itemize}
    \item \keyword{Pure Java}: કોઈ native code ની જરૂર નથી.
    \item \keyword{Direct Connection}: ડેટાબેઝ સાથે સીધો સંપર્ક કરે છે.
    \item \keyword{Network Protocol}: ડેટાબેઝના native network protocol નો ઉપયોગ કરે છે.
    \item \keyword{શ્રેષ્ઠ પ્રદર્શન}: બધા driver types માં સૌથી ઝડપી.
\end{itemize}
\end{solutionbox}

\begin{mnemonicbox}
\mnemonic{Pure Java Direct Protocol}
\end{mnemonicbox}

\questionmarks{2(બ)}{4}{Component class ની સામાન્ય રીતે વપરાતી methods સમજાવો.}

\begin{solutionbox}
\begin{center}
\captionof{table}{Component Class Methods}
\begin{tabulary}{\linewidth}{|L|L|}
\hline
\textbf{Method} & \textbf{હેતુ} \\ \hline
\textbf{add()} & container માં component ઉમેરે છે \\ \hline
\textbf{setSize()} & component ના dimensions સેટ કરે છે \\ \hline
\textbf{setLayout()} & layout manager સેટ કરે છે \\ \hline
\textbf{setVisible()} & component ને દૃશ્યમાન/અદૃશ્ય બનાવે છે \\ \hline
\textbf{setBounds()} & position અને size સેટ કરે છે \\ \hline
\textbf{getSize()} & component નું size return કરે છે \\ \hline
\end{tabulary}
\end{center}

\textbf{મુખ્ય લક્ષણો:}
\begin{itemize}
    \item \keyword{Layout Management}: component arrangement ને control કરે છે.
    \item \keyword{Visibility Control}: components ને દેખાડે/છુપાવે છે.
    \item \keyword{Size Management}: component dimensions ને control કરે છે.
    \item \keyword{Container Operations}: child components ને manage કરે છે.
\end{itemize}
\end{solutionbox}

\begin{mnemonicbox}
\mnemonic{Add Set Get Visibility}
\end{mnemonicbox}

\questionmarks{2(ક)}{7}{ટેબલ 'StuRec' માંથી વિદ્યાર્થીના રેકોર્ડ (Enroll No, Name, Address, Mobile No અને Email-ID) દર્શાવવા માટે JDBC નો ઉપયોગ કરીને પ્રોગ્રામ વિકસાવો.}

\begin{solutionbox}
\begin{lstlisting}[language=Java]
import java.sql.*;
import javax.swing.*;
import javax.swing.table.DefaultTableModel;

public class StudentRecordDisplay extends JFrame {
    JTable table;
    DefaultTableModel model;
    
    public StudentRecordDisplay() {
        setTitle("વિદ્યાર્થી રેકોર્ડ્સ");
        
        // table model બનાવો
        String[] columns = {"Enroll No", "Name", "Address", "Mobile", "Email"};
        model = new DefaultTableModel(columns, 0);
        table = new JTable(model);
        
        // ડેટા લોડ કરો
        loadStudentData();
        
        add(new JScrollPane(table));
        setSize(600, 400);
        setDefaultCloseOperation(JFrame.EXIT_ON_CLOSE);
        setVisible(true);
    }
    
    private void loadStudentData() {
        try {
            // ડેટાબેઝ કનેક્શન
            Class.forName("com.mysql.cj.jdbc.Driver");
            Connection con = DriverManager.getConnection(
                "jdbc:mysql://localhost:3306/university", "root", "password");
            
            // query execute કરો
            Statement stmt = con.createStatement();
            ResultSet rs = stmt.executeQuery("SELECT * FROM StuRec");
            
            // table માં ડેટા ઉમેરો
            while(rs.next()) {
                String[] row = {
                    rs.getString("enrollno"),
                    rs.getString("name"),
                    rs.getString("address"),
                    rs.getString("mobile"),
                    rs.getString("email")
                };
                model.addRow(row);
            }
            
            con.close();
        } catch(Exception e) {
            JOptionPane.showMessageDialog(this, "Error: " + e.getMessage());
        }
    }
    
    public static void main(String[] args) {
        new StudentRecordDisplay();
    }
}
\end{lstlisting}

\textbf{ડેટાબેઝ ટેબલ માળખું:}
\begin{lstlisting}[language=SQL]
CREATE TABLE StuRec (
    enrollno VARCHAR(20) PRIMARY KEY,
    name VARCHAR(50),
    address VARCHAR(100),
    mobile VARCHAR(15),
    email VARCHAR(50)
);
\end{lstlisting}
\end{solutionbox}

\begin{mnemonicbox}
\mnemonic{Connect Query Display Records}
\end{mnemonicbox}

\questionmarks{2(અ OR)}{3}{JDBC ના ફાયદા અને ગેરફાયદા લખો.}

\begin{solutionbox}
\begin{center}
\captionof{table}{JDBC ફાયદા અને ગેરફાયદા}
\begin{tabulary}{\linewidth}{|L|L|}
\hline
\textbf{ફાયદા} & \textbf{ગેરફાયદા} \\ \hline
\textbf{Platform Independent} & \textbf{Performance Overhead} \\ \hline
\textbf{Database Independent} & \textbf{શરૂઆતી લોકો માટે જટિલ} \\ \hline
\textbf{Standard API} & \textbf{SQL dependency} \\ \hline
\textbf{Transactions ને support કરે} & \textbf{Manual resource management} \\ \hline
\end{tabulary}
\end{center}

\textbf{મુખ્ય મુદ્દાઓ:}
\begin{itemize}
    \item \keyword{પોર્ટેબિલિટી}: વિવિધ platforms અને databases પર કામ કરે છે.
    \item \keyword{સ્ટાન્ડર્ડાઇઝેશન}: database operations માટે uniform API.
    \item \keyword{પ્રદર્શન}: વધારાનું layer performance માં overhead લાવે છે.
    \item \keyword{જટિલતા}: યોગ્ય resource management જરૂરી.
\end{itemize}
\end{solutionbox}

\begin{mnemonicbox}
\mnemonic{Platform Independent Standard Complex}
\end{mnemonicbox}

\questionmarks{2(બ OR)}{4}{Border Layout સમજાવો.}

\begin{solutionbox}
BorderLayout container ને પાંચ વિસ્તારોમાં વહેંચે છે: North, South, East, West, અને Center.

\begin{center}
\begin{tikzpicture}
    \draw (0,0) rectangle (4,3);
    \draw (0,2.5) -- (4,2.5);
    \draw (0,0.5) -- (4,0.5);
    \draw (1,0.5) -- (1,2.5);
    \draw (3,0.5) -- (3,2.5);
    
    \node at (2,2.75) {NORTH};
    \node at (2,0.25) {SOUTH};
    \node at (0.5,1.5) {WEST};
    \node at (3.5,1.5) {EAST};
    \node at (2,1.5) {CENTER};
\end{tikzpicture}
\captionof{figure}{Border Layout Regions}
\end{center}

\begin{center}
\captionof{table}{Border Layout વિસ્તારો}
\begin{tabulary}{\linewidth}{|L|L|L|}
\hline
\textbf{વિસ્તાર} & \textbf{સ્થાન} & \textbf{વર્તન} \\ \hline
\textbf{NORTH} & ઉપર & Preferred height, full width \\ \hline
\textbf{SOUTH} & નીચે & Preferred height, full width \\ \hline
\textbf{EAST} & જમણે & Preferred width, full height \\ \hline
\textbf{WEST} & ડાબે & Preferred width, full height \\ \hline
\textbf{CENTER} & વચ્ચે & બાકીની જગ્યા લે છે \\ \hline
\end{tabulary}
\end{center}

\textbf{કોડ ઉદાહરણ:}
\begin{lstlisting}[language=Java]
setLayout(new BorderLayout());
add(new JButton("ઉત્તર"), BorderLayout.NORTH);
add(new JButton("મધ્ય"), BorderLayout.CENTER);
\end{lstlisting}
\end{solutionbox}

\begin{mnemonicbox}
\mnemonic{North South East West Center}
\end{mnemonicbox}

\questionmarks{2(ક OR)}{7}{Hibernate CRUD operations નો ઉપયોગ કરીને Employee (NAME, AGE, SALARY અને DEPARTMENT) નો ડેટા store, update, fetch અને delete માટે એપ્લિકેશન ડેવલપ કરો.}

\begin{solutionbox}
\textbf{Employee Entity Class:}
\begin{lstlisting}[language=Java]
import javax.persistence.*;

@Entity
@Table(name = "employees")
public class Employee {
    @Id
    @GeneratedValue(strategy = GenerationType.IDENTITY)
    private int id;
    
    private String name;
    private int age;
    private double salary;
    private String department;
    
    // Constructors, getters, setters
    public Employee() {}
    
    public Employee(String name, int age, double salary, String dept) {
        this.name = name;
        this.age = age;
        this.salary = salary;
        this.department = dept;
    }
    
    // Getters અને Setters
    public int getId() { return id; }
    public void setId(int id) { this.id = id; }
    
    public String getName() { return name; }
    public void setName(String name) { this.name = name; }
    
    // ... અન્ય getters/setters
}
\end{lstlisting}

\textbf{CRUD Operations Class:}
\begin{lstlisting}[language=Java]
import org.hibernate.*;
import org.hibernate.cfg.Configuration;

public class EmployeeCRUD {
    private SessionFactory factory;
    
    public EmployeeCRUD() {
        factory = new Configuration()
                    .configure("hibernate.cfg.xml")
                    .addAnnotatedClass(Employee.class)
                    .buildSessionFactory();
    }
    
    // CREATE
    public void saveEmployee(Employee emp) {
        Session session = factory.openSession();
        Transaction tx = session.beginTransaction();
        session.save(emp);
        tx.commit();
        session.close();
    }
    
    // READ
    public Employee getEmployee(int id) {
        Session session = factory.openSession();
        Employee emp = session.get(Employee.class, id);
        session.close();
        return emp;
    }
    
    // UPDATE
    public void updateEmployee(Employee emp) {
        Session session = factory.openSession();
        Transaction tx = session.beginTransaction();
        session.update(emp);
        tx.commit();
        session.close();
    }
    
    // DELETE
    public void deleteEmployee(int id) {
        Session session = factory.openSession();
        Transaction tx = session.beginTransaction();
        Employee emp = session.get(Employee.class, id);
        session.delete(emp);
        tx.commit();
        session.close();
    }
}
\end{lstlisting}
\end{solutionbox}

\begin{mnemonicbox}
\mnemonic{Save Get Update Delete Hibernate}
\end{mnemonicbox}

\questionmarks{3(અ)}{3}{Deployment Descriptor સમજાવો.}

\begin{solutionbox}
Deployment Descriptor (web.xml) web applications માટે configuration file છે જેમાં servlet mappings, initialization parameters, અને security settings હોય છે.

\begin{center}
\captionof{table}{Deployment Descriptor Elements}
\begin{tabulary}{\linewidth}{|L|L|}
\hline
\textbf{Element} & \textbf{હેતુ} \\ \hline
\textbf{\textless servlet\textgreater} & servlet configuration define કરે છે \\ \hline
\textbf{\textless servlet-mapping\textgreater} & servlet ને URL pattern સાથે map કરે છે \\ \hline
\textbf{\textless init-param\textgreater} & initialization parameters સેટ કરે છે \\ \hline
\textbf{\textless welcome-file-list\textgreater} & default files serve કરવા માટે \\ \hline
\end{tabulary}
\end{center}

\textbf{મુખ્ય લક્ષણો:}
\begin{itemize}
    \item \keyword{Configuration}: web app માટે કેન્દ્રીય configuration.
    \item \keyword{Servlet Mapping}: URL to servlet mapping.
    \item \keyword{Parameters}: initialization અને context parameters.
    \item \keyword{Security}: authentication અને authorization settings.
\end{itemize}
\end{solutionbox}

\begin{mnemonicbox}
\mnemonic{Web XML Configuration Mapping}
\end{mnemonicbox}

\questionmarks{3(બ)}{4}{servlet માં get અને post method વચ્ચેનો તફાવત સમજાવો.}

\begin{solutionbox}
\begin{center}
\captionof{table}{GET vs POST Methods}
\begin{tabulary}{\linewidth}{|L|L|L|}
\hline
\textbf{લક્ષણ} & \textbf{GET} & \textbf{POST} \\ \hline
\textbf{Data Location} & URL query string & Request body \\ \hline
\textbf{Data Size} & મર્યાદિત (2048 chars) & અમર્યાદિત \\ \hline
\textbf{Security} & ઓછું સુરક્ષિત (દૃશ્યમાન) & વધુ સુરક્ષિત \\ \hline
\textbf{Caching} & Cache થઈ શકે છે & Cache થતું નથી \\ \hline
\textbf{Bookmarking} & Bookmark કરી શકાય & Bookmark કરી શકાતું નથી \\ \hline
\textbf{હેતુ} & ડેટા retrieve કરવા & ડેટા submit/modify કરવા \\ \hline
\end{tabulary}
\end{center}

\textbf{મુખ્ય મુદ્દાઓ:}
\begin{itemize}
    \item \keyword{દૃશ્યતા}: GET ડેટા URL માં દેખાય છે, POST છુપાયેલું હોય છે.
    \item \keyword{ક્ષમતા}: POST મોટો ડેટા handle કરી શકે છે.
    \item \keyword{સુરક્ષા}: POST sensitive ડેટા માટે વધુ સારી.
    \item \keyword{ઉપયોગ}: GET fetching માટે, POST form submission માટે.
\end{itemize}
\end{solutionbox}

\begin{mnemonicbox}
\mnemonic{GET Visible Limited, POST Hidden Unlimited}
\end{mnemonicbox}

\questionmarks{3(ક)}{7}{એક સરળ servlet પ્રોગ્રામ વિકસાવો જે તેના લોડિંગ પછી કેટલી વખત તેને access કરવામાં આવ્યું છે તેમાટે counter જાળવી રાખે છે; deployment descriptor નો ઉપયોગ કરીને counter ને પ્રારંભ કરો.}

\begin{solutionbox}
\textbf{Servlet કોડ:}
\begin{lstlisting}[language=Java]
import java.io.*;
import javax.servlet.*;
import javax.servlet.http.*;

public class CounterServlet extends HttpServlet {
    private int counter;
    
    public void init() throws ServletException {
        String initialValue = getInitParameter("initialCount");
        counter = Integer.parseInt(initialValue);
    }
    
    protected void doGet(HttpServletRequest request, 
                        HttpServletResponse response) 
                        throws ServletException, IOException {
        
        response.setContentType("text/html");
        PrintWriter out = response.getWriter();
        
        synchronized(this) {
            counter++;
        }
        
        out.println("<html><body>");
        out.println("<h2>પેજ Access કાઉન્ટર</h2>");
        out.println("<p>આ પેજ " + counter + " વખત access કરવામાં આવ્યું છે</p>");
        out.println("<p><a href='CounterServlet'>Refresh</a></p>");
        out.println("</body></html>");
        
        out.close();
    }
}
\end{lstlisting}

\textbf{web.xml Configuration:}
\begin{lstlisting}[language=XML]
<?xml version="1.0" encoding="UTF-8"?>
<web-app>
    <servlet>
        <servlet-name>CounterServlet</servlet-name>
        <servlet-class>CounterServlet</servlet-class>
        <init-param>
            <param-name>initialCount</param-name>
            <param-value>0</param-value>
        </init-param>
        <load-on-startup>1</load-on-startup>
    </servlet>
    
    <servlet-mapping>
        <servlet-name>CounterServlet</servlet-name>
        <url-pattern>/counter</url-pattern>
    </servlet-mapping>
</web-app>
\end{lstlisting}

\textbf{મુખ્ય લક્ષણો:}
\begin{itemize}
    \item \keyword{Thread Safety}: synchronized counter increment.
    \item \keyword{Initialization}: web.xml માંથી counter initialized.
    \item \keyword{Persistent}: requests ની વચ્ચે counter maintained.
    \item \keyword{Configuration}: deployment descriptor setup.
\end{itemize}
\end{solutionbox}

\begin{mnemonicbox}
\mnemonic{Initialize Synchronize Count Display}
\end{mnemonicbox}

\questionmarks{3(અ OR)}{3}{servlet ના life cycle સમજાવો.}

\begin{solutionbox}
\begin{center}
\begin{tikzpicture}[node distance=1.5cm, auto]
    \node [gtu state] (Load) {Loading};
    \node [gtu state, right=of Load] (Init) {init()};
    \node [gtu state, right=of Init] (Serv) {service()};
    \node [gtu state, right=of Serv] (Dest) {destroy()};
    
    \path [gtu arrow] (Load) -- (Init);
    \path [gtu arrow] (Init) -- (Serv);
    \path [gtu arrow] (Serv) -- (Dest);
    
    \path [gtu arrow] (Serv) edge [loop above] node {Requests} (Serv);
\end{tikzpicture}
\captionof{figure}{Servlet Life Cycle}
\end{center}

\begin{center}
\captionof{table}{Servlet Life Cycle Methods}
\begin{tabulary}{\linewidth}{|L|L|L|}
\hline
\textbf{Method} & \textbf{હેતુ} & \textbf{Called} \\ \hline
\textbf{init()} & servlet initialize કરે છે & startup પર એક વખત \\ \hline
\textbf{service()} & requests handle કરે છે & દરેક request માટે \\ \hline
\textbf{destroy()} & resources cleanup કરે છે & shutdown પર એક વખત \\ \hline
\end{tabulary}
\end{center}

\textbf{મુખ્ય મુદ્દાઓ:}
\begin{itemize}
    \item \keyword{Initialization}: servlet load થાય ત્યારે એક વખત call થાય છે.
    \item \keyword{Service}: બધી client requests handle કરે છે.
    \item \keyword{Cleanup}: servlet unload થાય તે પહેલાં call થાય છે.
    \item \keyword{Container Managed}: web container lifecycle ને control કરે છે.
\end{itemize}
\end{solutionbox}

\begin{mnemonicbox}
\mnemonic{Initialize Service Destroy}
\end{mnemonicbox}

\questionmarks{3(બ OR)}{4}{Servlet Config class ને યોગ્ય ઉદાહરણ સાથે સમજાવો.}

\begin{solutionbox}
ServletConfig servlet-specific configuration information અને initialization parameters પ્રદાન કરે છે.

\begin{center}
\captionof{table}{ServletConfig Methods}
\begin{tabulary}{\linewidth}{|L|L|}
\hline
\textbf{Method} & \textbf{હેતુ} \\ \hline
\textbf{getInitParameter()} & init parameter value મેળવે છે \\ \hline
\textbf{getInitParameterNames()} & બધા parameter names મેળવે છે \\ \hline
\textbf{getServletContext()} & servlet context મેળવે છે \\ \hline
\textbf{getServletName()} & servlet name મેળવે છે \\ \hline
\end{tabulary}
\end{center}

\textbf{ઉદાહરણ:}
\begin{lstlisting}[language=Java]
public class ConfigServlet extends HttpServlet {
    String databaseURL, username;
    
    public void init() throws ServletException {
        ServletConfig config = getServletConfig();
        databaseURL = config.getInitParameter("dbURL");
        username = config.getInitParameter("dbUser");
    }
    
    protected void doGet(HttpServletRequest request, 
                        HttpServletResponse response) 
                        throws ServletException, IOException {
        
        PrintWriter out = response.getWriter();
        out.println("Database URL: " + databaseURL);
        out.println("Username: " + username);
    }
}
\end{lstlisting}

\textbf{web.xml:}
\begin{lstlisting}[language=XML]
<servlet>
    <servlet-name>ConfigServlet</servlet-name>
    <servlet-class>ConfigServlet</servlet-class>
    <init-param>
        <param-name>dbURL</param-name>
        <param-value>jdbc:mysql://localhost:3306/test</param-value>
    </init-param>
    <init-param>
        <param-name>dbUser</param-name>
        <param-value>root</param-value>
    </init-param>
</servlet>
\end{lstlisting}
\end{solutionbox}

\begin{mnemonicbox}
\mnemonic{Config Gets Parameters Context}
\end{mnemonicbox}

\questionmarks{3(ક OR)}{7}{એક સરળ પ્રોગ્રામ ડેવલપ કરો, જ્યારે વપરાશકર્તા subject code પસંદ કરશે, ત્યારે subject નું નામ servlet અને MySQL database નો ઉપયોગ કરીને પ્રદર્શિત થશે.}

\begin{solutionbox}
\textbf{HTML Form (index.html):}
\begin{lstlisting}[language=HTML]
<!DOCTYPE html>
<html>
<head>
    <title>વિષય પસંદગી</title>
</head>
<body>
    <h2>વિષય કોડ પસંદ કરો</h2>
    <form action="SubjectServlet" method="get">
        <select name="subjectCode">
            <option value="">વિષય પસંદ કરો</option>
            <option value="4351603">4351603</option>
            <option value="4351604">4351604</option>
            <option value="4351605">4351605</option>
        </select>
        <input type="submit" value="વિષયનું નામ મેળવો">
    </form>
</body>
</html>
\end{lstlisting}

\textbf{Servlet કોડ:}
\begin{lstlisting}[language=Java]
import java.io.*;
import java.sql.*;
import javax.servlet.*;
import javax.servlet.http.*;

public class SubjectServlet extends HttpServlet {
    
    protected void doGet(HttpServletRequest request, 
                        HttpServletResponse response) 
                        throws ServletException, IOException {
        
        response.setContentType("text/html;charset=UTF-8");
        PrintWriter out = response.getWriter();
        
        String subjectCode = request.getParameter("subjectCode");
        String subjectName = "";
        
        if(subjectCode != null && !subjectCode.equals("")) {
            try {
                Class.forName("com.mysql.cj.jdbc.Driver");
                Connection con = DriverManager.getConnection(
                    "jdbc:mysql://localhost:3306/university", "root", "password");
                
                PreparedStatement ps = con.prepareStatement(
                    "SELECT subject_name FROM subjects WHERE subject_code = ?");
                ps.setString(1, subjectCode);
                
                ResultSet rs = ps.executeQuery();
                if(rs.next()) {
                    subjectName = rs.getString("subject_name");
                }
                
                con.close();
            } catch(Exception e) {
                subjectName = "Error: " + e.getMessage();
            }
        }
        
        out.println("<html><body>");
        out.println("<h2>વિષયની માહિતી</h2>");
        if(!subjectName.equals("")) {
            out.println("<p>વિષય કોડ: " + subjectCode + "</p>");
            out.println("<p>વિષયનું નામ: " + subjectName + "</p>");
        } else {
            out.println("<p>કૃપા કરીને વિષય કોડ પસંદ કરો</p>");
        }
        out.println("<p><a href='index.html'>પાછા જાઓ</a></p>");
        out.println("</body></html>");
    }
}
\end{lstlisting}

\textbf{ડેટાબેઝ ટેબલ:}
\begin{lstlisting}[language=SQL]
CREATE TABLE subjects (
    subject_code VARCHAR(10) PRIMARY KEY,
    subject_name VARCHAR(100)
);

INSERT INTO subjects VALUES 
('4351603', 'Advanced Java Programming'),
('4351604', 'Web Technology'),
('4351605', 'Database Management System');
\end{lstlisting}
\end{solutionbox}

\begin{mnemonicbox}
\mnemonic{Select Query Display Subject}
\end{mnemonicbox}

\questionmarks{4(અ)}{3}{JSP life cycle સમજાવો.}

\begin{solutionbox}
\begin{center}
\begin{tikzpicture}[node distance=1.2cm, auto]
    \node [gtu state] (Trans) {Translation};
    \node [gtu state, right=of Trans] (Comp) {Compilation};
    \node [gtu state, right=of Comp] (Load) {Loading};
    \node [gtu state, below=of Trans] (Init) {jspInit()};
    \node [gtu state, right=of Init] (Serv) {\_jspService()};
    \node [gtu state, right=of Serv] (Dest) {jspDestroy()};
    
    \path [gtu arrow] (Trans) -- (Comp);
    \path [gtu arrow] (Comp) -- (Load);
    \path [gtu arrow] (Load) -- (Init);
    \path [gtu arrow] (Init) -- (Serv);
    \path [gtu arrow] (Serv) -- (Dest);
    \path [gtu arrow] (Serv) edge [loop above] node {Requests} (Serv);
\end{tikzpicture}
\captionof{figure}{JSP Life Cycle}
\end{center}

\begin{center}
\captionof{table}{JSP Life Cycle તબક્કાઓ}
\begin{tabulary}{\linewidth}{|L|L|}
\hline
\textbf{તબક્કો} & \textbf{વર્ણન} \\ \hline
\textbf{Translation} & JSP to Servlet conversion \\ \hline
\textbf{Compilation} & Servlet to bytecode \\ \hline
\textbf{Loading} & servlet class ને load કરે છે \\ \hline
\textbf{Initialization} & jspInit() call થાય છે \\ \hline
\textbf{Request Processing} & \_jspService() requests handle કરે છે \\ \hline
\textbf{Destruction} & jspDestroy() cleanup \\ \hline
\end{tabulary}
\end{center}
\end{solutionbox}

\begin{mnemonicbox}
\mnemonic{Translate Compile Load Initialize Service Destroy}
\end{mnemonicbox}

\questionmarks{4(બ)}{4}{JSP અને Servlet ની સરખામણી કરો.}

\begin{solutionbox}
\begin{center}
\captionof{table}{JSP vs Servlet સરખામણી}
\begin{tabulary}{\linewidth}{|L|L|L|}
\hline
\textbf{લક્ષણ} & \textbf{JSP} & \textbf{Servlet} \\ \hline
\textbf{કોડ પ્રકાર} & HTML with Java code & Pure Java code \\ \hline
\textbf{ડેવલપમેન્ટ} & web designers માટે સરળ & Java developers માટે વધુ સારું \\ \hline
\textbf{કમ્પાઈલેશન} & આપોઆપ & મેન્યુઅલ \\ \hline
\textbf{ફેરફાર} & restart ની જરૂર નથી & restart જરૂરી \\ \hline
\textbf{પર્ફોર્મન્સ} & પહેલી request ધીમી & ઝડપી \\ \hline
\textbf{જાળવણી} & સરળ & જટિલ \\ \hline
\end{tabulary}
\end{center}

\textbf{મુખ્ય મુદ્દાઓ:}
\begin{itemize}
    \item \keyword{ઉપયોગમાં સરળતા}: JSP presentation layer માટે સરળ.
    \item \keyword{પર્ફોર્મન્સ}: Servlet business logic માટે વધુ સારું.
    \item \keyword{લવચીકતા}: JSP dynamic content માટે વધુ સારું.
    \item \keyword{નિયંત્રણ}: Servlet વધુ control પ્રદાન કરે છે.
\end{itemize}
\end{solutionbox}

\begin{mnemonicbox}
\mnemonic{JSP Easy HTML, Servlet Pure Java}
\end{mnemonicbox}

\questionmarks{4(ક)}{7}{Enrollment number દ્વારા વર્તમાન સેમેસ્ટરના દરેક વિષયમાં વિદ્યાર્થીની માસિક હાજરી દર્શાવવા માટે JSP web application ડેવલપ કરો.}

\begin{solutionbox}
\textbf{Input Form (attendance.html):}
\begin{lstlisting}[language=HTML]
<!DOCTYPE html>
<html>
<head>
    <title>વિદ્યાર્થી હાજરી</title>
</head>
<body>
    <h2>વિદ્યાર્થી હાજરી તપાસો</h2>
    <form action="attendanceCheck.jsp" method="post">
        <table>
            <tr>
                <td>Enrollment નંબર:</td>
                <td><input type="text" name="enrollNo" required></td>
            </tr>
            <tr>
                <td>મહિનો:</td>
                <td>
                    <select name="month" required>
                        <option value="">મહિનો પસંદ કરો</option>
                        <option value="January">જાન્યુઆરી</option>
                        <option value="February">ફેબ્રુઆરી</option>
                        <option value="March">માર્ચ</option>
                    </select>
                </td>
            </tr>
            <tr>
                <td colspan="2">
                    <input type="submit" value="હાજરી તપાસો">
                </td>
            </tr>
        </table>
    </form>
</body>
</html>
\end{lstlisting}

\textbf{JSP Page (attendanceCheck.jsp):}
\begin{lstlisting}[language=Java]
<%@ page import="java.sql.*" %>
<%@ page contentType="text/html;charset=UTF-8" %>

<html>
<head>
    <title>હાજરી રિપોર્ટ</title>
    <style>
        table { border-collapse: collapse; width: 100%; }
        th, td { border: 1px solid black; padding: 8px; text-align: center; }
        th { background-color: #f2f2f2; }
    </style>
</head>
<body>
    <h2>માસિક હાજરી રિપોર્ટ</h2>
    
    <%
        String enrollNo = request.getParameter("enrollNo");
        String month = request.getParameter("month");
        
        if(enrollNo != null && month != null) {
            try {
                Class.forName("com.mysql.cj.jdbc.Driver");
                Connection con = DriverManager.getConnection(
                    "jdbc:mysql://localhost:3306/university", "root", "password");
                
                // વિદ્યાર્થીની માહિતી મેળવો
                PreparedStatement ps1 = con.prepareStatement(
                    "SELECT name FROM students WHERE enroll_no = ?");
                ps1.setString(1, enrollNo);
                ResultSet rs1 = ps1.executeQuery();
                
                String studentName = "";
                if(rs1.next()) {
                    studentName = rs1.getString("name");
                }
                
                out.println("<p><strong>વિદ્યાર્થી:</strong> " + studentName + 
                           " (" + enrollNo + ")</p>");
                out.println("<p><strong>મહિનો:</strong> " + month + "</p>");
                
                // હાજરી ડેટા મેળવો
                PreparedStatement ps2 = con.prepareStatement(
                    "SELECT s.subject_name, a.total_classes, a.attended_classes, " +
                    "ROUND((a.attended_classes/a.total_classes)*100, 2) as percentage " +
                    "FROM attendance a JOIN subjects s ON a.subject_code = s.subject_code " +
                    "WHERE a.enroll_no = ? AND a.month = ?");
                ps2.setString(1, enrollNo);
                ps2.setString(2, month);
                ResultSet rs2 = ps2.executeQuery();
                
                out.println("<table>");
                out.println("<tr><th>વિષય</th><th>કુલ વર્ગો</th>" +
                           "<th>હાજર થયેલ</th><th>ટકાવારી</th><th>સ્થિતિ</th></tr>");
                
                while(rs2.next()) {
                    String subjectName = rs2.getString("subject_name");
                    int totalClasses = rs2.getInt("total_classes");
                    int attendedClasses = rs2.getInt("attended_classes");
                    double percentage = rs2.getDouble("percentage");
                    String status = percentage >= 75 ? "સારી" : "નબળી";
                    String rowColor = percentage >= 75 ? "lightgreen" : "lightcoral";
                    
                    out.println("<tr style='background-color:" + rowColor + "'>");
                    out.println("<td>" + subjectName + "</td>");
                    out.println("<td>" + totalClasses + "</td>");
                    out.println("<td>" + attendedClasses + "</td>");
                    out.println("<td>" + percentage + "%</td>");
                    out.println("<td>" + status + "</td>");
                    out.println("</tr>");
                }
                
                out.println("</table>");
                con.close();
                
            } catch(Exception e) {
                out.println("<p style='color:red'>Error: " + e.getMessage() + "</p>");
            }
        }
    %>
    
    <br />
    <a href="attendance.html">બીજા વિદ્યાર્થીની તપાસ કરો</a>
</body>
</html>
\end{lstlisting}
\end{solutionbox}

\begin{mnemonicbox}
\mnemonic{JSP Database Query Display Table}
\end{mnemonicbox}

\questionmarks{4(અ OR)}{3}{JSP માં implicit objects સમજાવો.}

\begin{solutionbox}
\begin{center}
\captionof{table}{JSP Implicit Objects}
\begin{tabulary}{\linewidth}{|L|L|L|}
\hline
\textbf{Object} & \textbf{Type} & \textbf{હેતુ} \\ \hline
\textbf{request} & HttpServletRequest & request ડેટા મેળવે છે \\ \hline
\textbf{response} & HttpServletResponse & response મોકલે છે \\ \hline
\textbf{out} & JspWriter & client ને output \\ \hline
\textbf{session} & HttpSession & session management \\ \hline
\textbf{application} & ServletContext & application scope \\ \hline
\textbf{config} & ServletConfig & servlet configuration \\ \hline
\textbf{pageContext} & PageContext & page scope access \\ \hline
\textbf{page} & Object & વર્તમાન servlet instance \\ \hline
\textbf{exception} & Throwable & error page exception \\ \hline
\end{tabulary}
\end{center}

\textbf{મુખ્ય લક્ષણો:}
\begin{itemize}
    \item \keyword{આપોઆપ}: declaration વિના ઉપલબ્ધ.
    \item \keyword{Scope Access}: વિવિધ scope levels.
    \item \keyword{Request Handling}: input/output operations.
    \item \keyword{Session Management}: વપરાશકર્તા session tracking.
\end{itemize}
\end{solutionbox}

\begin{mnemonicbox}
\mnemonic{Request Response Out Session Application}
\end{mnemonicbox}

\questionmarks{4(બ OR)}{4}{servlet કરતાં JSP શા માટે પસંદ કરવામાં આવે છે તે સમજાવો.}

\begin{solutionbox}
\begin{center}
\captionof{table}{Servlet કરતાં JSP ના ફાયદા}
\begin{tabulary}{\linewidth}{|L|L|}
\hline
\textbf{પાસું} & \textbf{JSP ફાયદો} \\ \hline
\textbf{ડેવલપમેન્ટ} & HTML integration સરળ \\ \hline
\textbf{જાળવણી} & presentation ને logic થી અલગ કરે \\ \hline
\textbf{કમ્પાઈલેશન} & આપોઆપ compilation \\ \hline
\textbf{ફેરફાર} & server restart ની જરૂર નથી \\ \hline
\textbf{ડિઝાઈન} & web designer friendly \\ \hline
\textbf{કોડ પુનઃઉપયોગ} & tag libraries અને custom tags \\ \hline
\end{tabulary}
\end{center}

\textbf{મુખ્ય મુદ્દાઓ:}
\begin{itemize}
    \item \keyword{Separation of Concerns}: presentation અને business logic નું સ્પષ્ટ વિભાજન.
    \item \keyword{ઝડપી ડેવલપમેન્ટ}: ઝડપી development cycle.
    \item \keyword{Designer Friendly}: web designers HTML-જેવા syntax સાથે કામ કરી શકે.
    \item \keyword{આપોઆપ સુવિધાઓ}: container compilation અને lifecycle handle કરે.
\end{itemize}
\end{solutionbox}

\begin{mnemonicbox}
\mnemonic{Easy HTML Automatic Designer Friendly}
\end{mnemonicbox}

\questionmarks{4(ક OR)}{7}{પાંચ વિષયોના ગુણ સ્વીકારીને વિદ્યાર્થીના ગ્રેડ દર્શાવવા માટે JSP પ્રોગ્રામ વિકસાવો.}

\begin{solutionbox}
\textbf{Input Form (gradeInput.html):}
\begin{lstlisting}[language=HTML]
<!DOCTYPE html>
<html>
<head>
    <title>વિદ્યાર્થી ગ્રેડ કેલ્ક્યુલેટર</title>
</head>
<body>
    <h2 style="text-align: center;">વિદ્યાર્થી ગ્રેડ કેલ્ક્યુલેટર</h2>
    <form action="gradeCalculator.jsp" method="post">
        <table border="1">
            <tr>
                <td>વિદ્યાર્થીનું નામ:</td>
                <td><input type="text" name="studentName" required></td>
            </tr>
            <tr>
                <td>વિષય 1 ગુણ:</td>
                <td><input type="number" name="marks1" min="0" max="100" required></td>
            </tr>
            <!-- Repeat for marks2 to marks5 -->
            <tr>
                <td colspan="2" style="text-align: center;">
                    <input type="submit" value="ગ્રેડ કેલ્ક્યુલેટ કરો">
                </td>
            </tr>
        </table>
    </form>
</body>
</html>
\end{lstlisting}

\textbf{JSP Grade Calculator (gradeCalculator.jsp):}
\begin{lstlisting}[language=Java]
<%@ page contentType="text/html;charset=UTF-8" %>
<html>
<head>
    <title>ગ્રેડ પરિણામ</title>
</head>
<body>
    <h2 style="text-align: center;">ગ્રેડ રિપોર્ટ</h2>
    
    <%
        String studentName = request.getParameter("studentName");
        
        // ગુણ મેળવો
        int marks1 = Integer.parseInt(request.getParameter("marks1"));
        int marks2 = Integer.parseInt(request.getParameter("marks2"));
        int marks3 = Integer.parseInt(request.getParameter("marks3"));
        int marks4 = Integer.parseInt(request.getParameter("marks4"));
        int marks5 = Integer.parseInt(request.getParameter("marks5"));
        
        // કુલ અને ટકાવારી કેલ્ક્યુલેટ કરો
        int totalMarks = marks1 + marks2 + marks3 + marks4 + marks5;
        double percentage = totalMarks / 5.0;
        
        // ગ્રેડ નક્કી કરો
        String grade;
        if(percentage >= 90) grade = "A+";
        else if(percentage >= 80) grade = "A";
        else if(percentage >= 70) grade = "B";
        else if(percentage >= 60) grade = "C";
        else if(percentage >= 50) grade = "D";
        else grade = "F";
        
        String result = percentage >= 50 ? "પાસ" : "ફેલ";
    %>
    
    <table>
        <tr><th colspan="2">વિદ્યાર્થીની માહિતી</th></tr>
        <tr><td><strong>નામ:</strong></td><td><%= studentName %></td></tr>
        <tr><th colspan="2">પરિણામ સારાંશ</th></tr>
        <tr><td><strong>કુલ ગુણ:</strong></td><td><%= totalMarks %> / 500</td></tr>
        <tr><td><strong>ટકાવારી:</strong></td><td><%= String.format("%.2f", percentage) %>%</td></tr>
        <tr><td><strong>ગ્રેડ:</strong></td><td><%= grade %></td></tr>
        <tr><td><strong>પરિણામ:</strong></td><td><%= result %></td></tr>
    </table>
</body>
</html>
\end{lstlisting}
\end{solutionbox}

\begin{mnemonicbox}
\mnemonic{Calculate Total Percentage Grade Result}
\end{mnemonicbox}

\questionmarks{5(અ)}{3}{Aspect-oriented programming (AOP) સમજાવો.}

\begin{solutionbox}
AOP એ programming paradigm છે જે cross-cutting concerns ને business logic થી aspects નો ઉપયોગ કરીને અલગ કરે છે.

\begin{center}
\captionof{table}{AOP મુખ્ય ખ્યાલો}
\begin{tabulary}{\linewidth}{|L|L|}
\hline
\textbf{ખ્યાલ} & \textbf{વર્ણન} \\ \hline
\textbf{Aspect} & cross-cutting concern ને encapsulate કરતું module \\ \hline
\textbf{Join Point} & program execution માં બિંદુ \\ \hline
\textbf{Pointcut} & join points નો સમૂહ \\ \hline
\textbf{Advice} & join point પર લેવાતી action \\ \hline
\textbf{Weaving} & aspects apply કરવાની પ્રક્રિયા \\ \hline
\end{tabulary}
\end{center}

\textbf{મુખ્ય લાભો:}
\begin{itemize}
    \item \keyword{વિભાજન}: business logic ને system services થી અલગ કરે છે.
    \item \keyword{મોડ્યુલારિટી}: કોડ modularity સુધારે છે.
    \item \keyword{પુનઃઉપયોગ}: cross-cutting concerns reusable છે.
    \item \keyword{જાળવણી}: maintain અને modify કરવું સરળ.
\end{itemize}
\end{solutionbox}

\begin{mnemonicbox}
\mnemonic{Aspect Join Pointcut Advice Weaving}
\end{mnemonicbox}

\questionmarks{5(બ)}{4}{Servlet ની વિવિધ વિશેષતાઓની યાદી બનાવો.}

\begin{solutionbox}
\begin{center}
\captionof{table}{Servlet વિશેષતાઓ}
\begin{tabulary}{\linewidth}{|L|L|}
\hline
\textbf{વિશેષતા} & \textbf{વર્ણન} \\ \hline
\textbf{Platform Independent} & Java સપોર્ટ કરતા કોઈપણ server પર ચાલે છે \\ \hline
\textbf{Server Independent} & વિવિધ web servers સાથે કામ કરે છે \\ \hline
\textbf{Protocol Independent} & HTTP, HTTPS, FTP સપોર્ટ કરે છે \\ \hline
\textbf{Persistent} & requests ની વચ્ચે memory માં રહે છે \\ \hline
\textbf{Robust} & મજબૂત memory management \\ \hline
\textbf{Secure} & Built-in security features \\ \hline
\textbf{Portable} & એક વખત લખો, ગમે ત્યાં ચલાવો \\ \hline
\textbf{Powerful} & સંપૂર્ણ Java API access \\ \hline
\end{tabulary}
\end{center}

\textbf{મુખ્ય મુદ્દાઓ:}
\begin{itemize}
    \item \keyword{પર્ફોર્મન્સ}: CGI કરતાં વધુ સારું પર્ફોર્મન્સ.
    \item \keyword{Memory Management}: કાર્યક્ષમ memory ઉપયોગ.
    \item \keyword{Multithreading}: એકસાથે અનેક requests handle કરે છે.
    \item \keyword{Extensible}: ચોક્કસ protocols માટે extend કરી શકાય છે.
\end{itemize}
\end{solutionbox}

\begin{mnemonicbox}
\mnemonic{Platform Server Protocol Persistent Robust}
\end{mnemonicbox}

\questionmarks{5(ક)}{7}{Model layer, View layer અને Controller layer ને વિગતોમાં સમજાવો.}

\begin{solutionbox}
\begin{center}
\begin{tikzpicture}[node distance=2cm, auto]
    \node [gtu block] (User) {User};
    \node [gtu block, below right=of User] (View) {View};
    \node [gtu block, right=of View] (Cont) {Controller};
    \node [gtu block, above=of Cont] (Model) {Model};
    
    \path [gtu arrow] (User) -- (Cont);
    \path [gtu arrow] (Cont) -- (Model);
    \path [gtu arrow, <->] (Model) -- (View);
    \path [gtu arrow] (View) -- (User);
\end{tikzpicture}
\captionof{figure}{MVC આર્કિટેક્ચર}
\end{center}

\begin{center}
\captionof{table}{MVC Layer વિગતો}
\begin{tabulary}{\linewidth}{|L|L|L|L|}
\hline
\textbf{Layer} & \textbf{જવાબદારી} & \textbf{Components} & \textbf{હેતુ} \\ \hline
\textbf{Model} & ડેટા અને business logic & Entities, DAOs, Services & ડેટા management \\ \hline
\textbf{View} & Presentation layer & JSP, HTML, CSS & વપરાશકર્તા interface \\ \hline
\textbf{Controller} & Request handling & Servlets, Actions & Flow control \\ \hline
\end{tabulary}
\end{center}

\textbf{Layer વિગતો:}
\begin{itemize}
    \item \textbf{Model}: ડેટાબેઝ operations, business logic, validation, અને entity classes.
    \item \textbf{View}: Presentation, display logic, user interaction, અને responsive design.
    \item \textbf{Controller}: Request handling, flow control, model coordination, અને response generation.
\end{itemize}

\textbf{MVC ના ફાયદા:}
\begin{itemize}
    \item \keyword{Separation of Concerns}: જવાબદારીનું સ્પષ્ટ વિભાજન.
    \item \keyword{Maintainability}: maintain અને modify કરવું સરળ.
    \item \keyword{Testability}: દરેક layer ને અલગ થી test કરી શકાય.
    \item \keyword{Scalability}: મોટા application development ને સપોર્ટ કરે છે.
\end{itemize}
\end{solutionbox}

\begin{mnemonicbox}
\mnemonic{Model Data View Present Controller Handle}
\end{mnemonicbox}

\questionmarks{5(અ OR)}{3}{Spring Boot ની વિશેષતાઓ સમજાવો.}

\begin{solutionbox}
\begin{center}
\captionof{table}{Spring Boot વિશેષતાઓ}
\begin{tabulary}{\linewidth}{|L|L|}
\hline
\textbf{વિશેષતા} & \textbf{વર્ણન} \\ \hline
\textbf{Auto Configuration} & dependencies આધારે આપોઆપ configuration \\ \hline
\textbf{Starter Dependencies} & curated dependencies નો સેટ \\ \hline
\textbf{Embedded Servers} & built-in Tomcat, Jetty servers \\ \hline
\textbf{Production Ready} & health checks, metrics, monitoring \\ \hline
\textbf{No XML Configuration} & annotation-based configuration \\ \hline
\textbf{Developer Tools} & hot reloading, automatic restart \\ \hline
\end{tabulary}
\end{center}

\textbf{મુખ્ય લાભો:}
\begin{itemize}
    \item \keyword{ઝડપી ડેવલપમેન્ટ}: ઝડપી project setup અને development.
    \item \keyword{Convention over Configuration}: sensible defaults.
    \item \keyword{Microservices Ready}: સરળ microservices development.
\end{itemize}
\end{solutionbox}

\begin{mnemonicbox}
\mnemonic{Auto Starter Embedded Production Annotation Developer}
\end{mnemonicbox}

\questionmarks{5(બ OR)}{4}{JSP scripting elements પર ટૂંકી નોંધ લખો.}

\begin{solutionbox}
\begin{center}
\captionof{table}{JSP Scripting Elements}
\begin{tabulary}{\linewidth}{|L|L|L|}
\hline
\textbf{Element} & \textbf{Syntax} & \textbf{હેતુ} \\ \hline
\textbf{Scriptlet} & \code{<\% \%>} & Java code execution \\ \hline
\textbf{Expression} & \code{<\%= \%>} & Output value \\ \hline
\textbf{Declaration} & \code{<\%! \%>} & Variable/method declaration \\ \hline
\textbf{Directive} & \code{<\%@ \%>} & Page configuration \\ \hline
\textbf{Comment} & \code{<\%-- --\%>} & JSP comments \\ \hline
\end{tabulary}
\end{center}

\textbf{ઉદાહરણ:}
\begin{lstlisting}[language=Java]
<%-- JSP Comment --%>
<%@ page contentType="text/html" %>

<%! 
    private int counter = 0;
%>

<html>
<body>
    <% 
        String name = "Student";
        counter++;
    %>
    
    <h1><%= "Welcome " + name %></h1>
    <p>પેજ <%= counter %> વખત visit કર્યું</p>
</body>
</html>
\end{lstlisting}
\end{solutionbox}

\begin{mnemonicbox}
\mnemonic{Script Express Declare Direct Comment}
\end{mnemonicbox}

\questionmarks{5(ક OR)}{7}{Dependency injection (DI) અને Plain Old Java Object (POJO) ને વિગતોમાં સમજાવો.}

\begin{solutionbox}
\textbf{Dependency Injection (DI):}
Dependency Injection એ design pattern છે જ્યાં objects તેમની dependencies external source માંથી receive કરે છે internal creation કરવાને બદલે.

\begin{center}
\captionof{table}{DI પ્રકારો}
\begin{tabulary}{\linewidth}{|L|L|L|}
\hline
\textbf{પ્રકાર} & \textbf{વર્ણન} & \textbf{ઉદાહરણ} \\ \hline
\textbf{Constructor} & constructor દ્વારા dependencies & \code{public Service(Repo r)} \\ \hline
\textbf{Setter} & setter methods દ્વારા dependencies & \code{setRepo(Repo r)} \\ \hline
\textbf{Field} & સીધું field injection & \code{@Autowired Repo r} \\ \hline
\end{tabulary}
\end{center}

\textbf{Plain Old Java Object (POJO):}
POJO એ સરળ Java object છે જે કોઈ ચોક્કસ framework classes માંથી inherit કરતું નથી અથવા ચોક્કસ interfaces implement કરતું નથી.

\textbf{POJO લાક્ષણિકતાઓ:}
\begin{itemize}
    \item \keyword{કોઈ inheritance નથી}: framework classes માંથી extend કરતું નથી.
    \item \keyword{કોઈ interfaces નથી}: framework interfaces implement કરતું નથી.
    \item \keyword{કોઈ annotations નથી}: framework annotations વિના કામ કરી શકે છે.
    \item \keyword{સરળ}: માત્ર business logic અને ડેટા ધરાવે છે.
\end{itemize}

\textbf{ઉદાહરણ:}
\begin{lstlisting}[language=Java]
// POJO Entity
public class Student {
    private String name;
    // constructors, getters, setters
}

// Service with DI
@Service
public class StudentService {
    @Autowired
    private StudentRepository repository;
    
    public void save(Student s) {
        repository.save(s);
    }
}
\end{lstlisting}

\textbf{ફાયદા:}
\begin{itemize}
    \item \keyword{DI}: Loose Coupling, Testability, Flexibility.
    \item \keyword{POJO}: સરળતા, Testability, Portability, Lightweight.
\end{itemize}
\end{solutionbox}

\begin{mnemonicbox}
\mnemonic{DI Injects Dependencies, POJO Plain Objects}
\end{mnemonicbox}

\end{document}
