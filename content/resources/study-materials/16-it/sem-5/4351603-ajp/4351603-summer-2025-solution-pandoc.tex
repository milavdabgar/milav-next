\documentclass[10pt,a4paper]{article}

% content/resources/templates/preamble.tex
\usepackage[margin=0.6in]{geometry}
\author{Milav Dabgar}
\usepackage{amsmath,amssymb,amsthm}
\usepackage{booktabs}
\usepackage{multirow}
\usepackage{xcolor}
\usepackage{tcolorbox}
\tcbuselibrary{breakable,skins}
\usepackage[colorlinks=true,linkcolor=blue]{hyperref}
\usepackage{titlesec}
\usepackage{enumitem}
\usepackage{tikz}
\usepackage{pgfplots}
\usepackage{circuitikz}
\usepackage[version=4]{mhchem}
\usepackage{longtable}
\usepackage{array}
\usepackage{float}
\usepackage{caption}
\usepackage{listings}

\lstset{
  basicstyle=\small\ttfamily,
  breaklines=true,
  breakatwhitespace=false,
  postbreak=\mbox{\textcolor{red}{$\hookrightarrow$}\space},
  float=false,
  numbers=left,
  numberstyle=\tiny\color{gray},
  numbersep=10pt,
  xleftmargin=2em,
  keywordstyle=\color{blue},
  commentstyle=\color{green!60!black},
  stringstyle=\color{purple},
  backgroundcolor=\color{gray!5},
  showstringspaces=false,
  tabsize=2,
  captionpos=b,
  keepspaces=true,
  columns=flexible
}

\pgfplotsset{compat=1.18}
\usetikzlibrary{shapes,arrows,positioning,calc,patterns,decorations.pathmorphing,decorations.markings,arrows.meta}

% Color scheme
\definecolor{headcolor}{RGB}{0,102,204}
\definecolor{keycolor}{RGB}{220,20,60}
\definecolor{solutioncolor}{RGB}{34,139,34}
\definecolor{mnemoniccolor}{RGB}{148,0,211}
\definecolor{codecolor}{RGB}{0,0,100}

% Spacing
\setlength{\parskip}{3pt}
\setlist[itemize]{nosep}
\setlist[enumerate]{nosep}

% Title formatting
\titleformat{\section}{\Large\bfseries\color{headcolor}}{\thesection}{1em}{}
\titleformat{\subsection}{\large\bfseries\color{headcolor}}{\thesubsection}{1em}{}

% Pandoc tightlist compatibility
\providecommand{\tightlist}{%
  \setlength{\itemsep}{0pt}\setlength{\parskip}{0pt}}

% Pandoc longtable compatibility
\newcounter{none}
\def\thenone{}


% content/resources/templates/english-boxes.tex
% This file is currently empty - it exists to maintain consistency with the import structure.
% Add custom environments here if needed in the future.


\begin{document}

\begin{center}
{\Huge\bfseries\color{headcolor} Subject Name Solutions}\\[5pt]
{\LARGE 4351603 -- Summer 2025}\\[3pt]
{\large Semester 1 Study Material}\\[3pt]
{\normalsize\textit{Detailed Solutions and Explanations}}
\end{center}

\vspace{10pt}

\subsection*{Question 1(a) [3 marks]}\label{q1a}

\textbf{Write a difference between AWT and Swing.}

\begin{solutionbox}


{\def\LTcaptype{none} % do not increment counter
\vspace{-5pt}
\captionof{table}{AWT vs Swing Comparison}
\vspace{-10pt}
\begin{longtable}[]{@{}lll@{}}
\toprule\noalign{}
Feature & AWT & Swing \\
\midrule\noalign{}
\endhead
\bottomrule\noalign{}
\endlastfoot
\textbf{Platform} & Platform dependent & Platform independent \\
\textbf{Components} & Heavy weight & Light weight \\
\textbf{Look and Feel} & Native OS look & Pluggable look and feel \\
\textbf{Performance} & Faster & Slower than AWT \\
\end{longtable}
}

\begin{itemize}
\tightlist
\item
  \textbf{AWT}: Uses native OS components
\item
  \textbf{Swing}: Uses Java's own components
\item
  \textbf{Pluggability}: Swing supports customizable UI
\end{itemize}

\end{solutionbox}
\begin{mnemonicbox}
``Swing is Smart - Platform Independent and
Pluggable''

\end{mnemonicbox}
\subsection*{Question 1(b) [4 marks]}\label{q1b}

\textbf{List out various Layout Managers. Explain Flow Layout manager
with example.}

\begin{solutionbox}

\textbf{Layout Managers List:}

\begin{itemize}
\tightlist
\item
  \textbf{FlowLayout}: Left to right arrangement
\item
  \textbf{BorderLayout}: North, South, East, West, Center
\item
  \textbf{GridLayout}: Equal sized grid cells
\item
  \textbf{CardLayout}: Stack of components
\item
  \textbf{BoxLayout}: Single row or column
\end{itemize}

\textbf{FlowLayout Example:}

\begin{verbatim}
import javax.swing.*;
import java.awt.*;

public class FlowExample extends JFrame \{
    public FlowExample() \{
        setLayout(new FlowLayout());
        add(new JButton("Button 1"));
        add(new JButton("Button 2"));
        add(new JButton("Button 3"));
        setSize(300, 100);
        setVisible(true);
    \}
\}
\end{verbatim}

\end{solutionbox}
\begin{mnemonicbox}
``Flow Like Water - Left to Right''

\end{mnemonicbox}
\subsection*{Question 1(c) [7 marks]}\label{q1c}

\textbf{Create a Swing program using checkbox that allows users to
select multiple items from a list of options. Display the selected
items.}

\begin{solutionbox}

\begin{verbatim}
import javax.swing.*;
import java.awt.*;
import java.awt.event.*;

public class CheckboxExample extends JFrame implements ItemListener \{
    JCheckBox java, python, cpp;
    JTextArea display;
    
    public CheckboxExample() \{
        setLayout(new FlowLayout());
        
        java = new JCheckBox("Java");
        python = new JCheckBox("Python");
        cpp = new JCheckBox("C++");
        
        java.addItemListener(this);
        python.addItemListener(this);
        cpp.addItemListener(this);
        
        display = new JTextArea(5, 20);
        
        add(java);
        add(python);
        add(cpp);
        add(new JScrollPane(display));
        
        setSize(300, 200);
        setDefaultCloseOperation(JFrame.EXIT\_ON\_CLOSE);
        setVisible(true);
    \}
    
    public void itemStateChanged(ItemEvent e) \{
        String result = "Selected: ";
        if(java.isSelected()) result += "Java ";
        if(python.isSelected()) result += "Python ";
        if(cpp.isSelected()) result += "C++ ";
        display.setText(result);
    \}
    
    public static void main(String[] args) \{
        new CheckboxExample();
    \}
\}
\end{verbatim}

\textbf{Key Features:}

\begin{itemize}
\tightlist
\item
  \textbf{Multiple Selection}: Users can select multiple checkboxes
\item
  \textbf{Real-time Display}: Shows selected items immediately
\item
  \textbf{ItemListener}: Handles checkbox state changes
\end{itemize}

\end{solutionbox}
\begin{mnemonicbox}
``Check Multiple, Display All''

\end{mnemonicbox}
\subsection*{Question 1(c) OR [7
marks]}\label{q1c}

\textbf{Develop a Java program using various swing components.}

\begin{solutionbox}

\begin{verbatim}
import javax.swing.*;
import java.awt.*;
import java.awt.event.*;

public class SwingComponents extends JFrame implements ActionListener \{
    JTextField nameField;
    JComboBox{}String{} cityCombo;
    JRadioButton male, female;
    JButton submit;
    JTextArea display;
    
    public SwingComponents() \{
        setLayout(new FlowLayout());
        
        add(new JLabel("Name:"));
        nameField = new JTextField(15);
        add(nameField);
        
        add(new JLabel("City:"));
        cityCombo = new JComboBox{(}new String[]\{"Mumbai", "Delhi", "Bangalore"\);}
        add(cityCombo);
        
        ButtonGroup gender = new ButtonGroup();
        male = new JRadioButton("Male");
        female = new JRadioButton("Female");
        gender.add(male);
        gender.add(female);
        add(male);
        add(female);
        
        submit = new JButton("Submit");
        submit.addActionListener(this);
        add(submit);
        
        display = new JTextArea(5, 25);
        add(new JScrollPane(display));
        
        setSize(400, 300);
        setDefaultCloseOperation(JFrame.EXIT\_ON\_CLOSE);
        setVisible(true);
    \}
    
    public void actionPerformed(ActionEvent e) \{
        String name = nameField.getText();
        String city = (String)cityCombo.getSelectedItem();
        String gender = male.isSelected() ? "Male" : "Female";
        
        display.setText("Name: " + name + "{n}City: " + city + "{n}Gender: " + gender);
    \}
    
    public static void main(String[] args) \{
        new SwingComponents();
    \}
\}
\end{verbatim}

\textbf{Components Used:}

\begin{itemize}
\tightlist
\item
  \textbf{JTextField}: Text input
\item
  \textbf{JComboBox}: Dropdown selection
\item
  \textbf{JRadioButton}: Single selection
\item
  \textbf{JButton}: Action trigger
\end{itemize}

\end{solutionbox}
\begin{mnemonicbox}
``Text, Combo, Radio, Button - Complete Form''

\end{mnemonicbox}
\subsection*{Question 2(a) [3 marks]}\label{q2a}

\textbf{Explain Swing controls with example.}

\begin{solutionbox}


{\def\LTcaptype{none} % do not increment counter
\vspace{-5pt}
\captionof{table}{Common Swing Controls}
\vspace{-10pt}
\begin{longtable}[]{@{}lll@{}}
\toprule\noalign{}
Control & Purpose & Example \\
\midrule\noalign{}
\endhead
\bottomrule\noalign{}
\endlastfoot
\textbf{JButton} & Click actions & \texttt{new\ JButton("Click\ Me")} \\
\textbf{JTextField} & Text input & \texttt{new\ JTextField(10)} \\
\textbf{JLabel} & Display text & \texttt{new\ JLabel("Hello")} \\
\textbf{JCheckBox} & Multiple selection &
\texttt{new\ JCheckBox("Option")} \\
\end{longtable}
}

\textbf{Basic Example:}

\begin{verbatim}
JFrame frame = new JFrame();
JButton btn = new JButton("Submit");
frame.add(btn);
frame.setSize(200, 100);
frame.setVisible(true);
\end{verbatim}

\end{solutionbox}
\begin{mnemonicbox}
``Button, Text, Label, Check - Basic Four''

\end{mnemonicbox}
\subsection*{Question 2(b) [4 marks]}\label{q2b}

\textbf{List JDBC drivers and explain any two.}

\begin{solutionbox}

\textbf{JDBC Drivers List:}

\begin{enumerate}
\tightlist
\item
  \textbf{Type 1}: JDBC-ODBC Bridge
\item
  \textbf{Type 2}: Native API Driver
\item
  \textbf{Type 3}: Network Protocol Driver
\item
  \textbf{Type 4}: Thin Driver
\end{enumerate}

\textbf{Detailed Explanation:}

\textbf{Type 1 - JDBC-ODBC Bridge:}

\begin{itemize}
\tightlist
\item
  \textbf{Purpose}: Converts JDBC calls to ODBC calls
\item
  \textbf{Advantage}: Works with any ODBC database
\item
  \textbf{Disadvantage}: Platform dependent, slower performance
\end{itemize}

\textbf{Type 4 - Thin Driver:}

\begin{itemize}
\tightlist
\item
  \textbf{Purpose}: Pure Java driver, direct database communication
\item
  \textbf{Advantage}: Platform independent, best performance
\item
  \textbf{Disadvantage}: Database specific
\end{itemize}

\end{solutionbox}
\begin{mnemonicbox}
``Bridge-Native-Network-Thin: 1-2-3-4''

\end{mnemonicbox}
\subsection*{Question 2(c) [7 marks]}\label{q2c}

\textbf{Explain Object Relational Mapping (ORM) with its advantages and
tools.}

\begin{solutionbox}

\textbf{Object Relational Mapping (ORM):} ORM is a technique that maps
object-oriented programming concepts to relational database structures.

\begin{center}
\textbf{Mermaid Diagram (Code)}
\begin{verbatim}
{Shaded}
{Highlighting}[]
graph LR
    A[Java Object] {-{-}{} B[ORM Framework]}
    B {-{-}{} C[Database Table]}
    C {-{-}{} B}
    B {-{-}{} A}
{Highlighting}
{Shaded}
\end{verbatim}
\end{center}


{\def\LTcaptype{none} % do not increment counter
\vspace{-5pt}
\captionof{table}{ORM Advantages}
\vspace{-10pt}
\begin{longtable}[]{@{}ll@{}}
\toprule\noalign{}
Advantage & Description \\
\midrule\noalign{}
\endhead
\bottomrule\noalign{}
\endlastfoot
\textbf{Productivity} & Reduces coding time \\
\textbf{Maintainability} & Easy to modify and update \\
\textbf{Database Independence} & Switch databases easily \\
\textbf{Object-Oriented} & Works with OOP concepts \\
\end{longtable}
}

\textbf{Popular ORM Tools:}

\begin{itemize}
\tightlist
\item
  \textbf{Hibernate}: Most popular Java ORM
\item
  \textbf{JPA}: Java Persistence API standard
\item
  \textbf{MyBatis}: SQL mapping framework
\item
  \textbf{EclipseLink}: Reference implementation
\end{itemize}

\textbf{Working Model:}

\begin{itemize}
\tightlist
\item
  \textbf{Objects} \rightarrow \textbf{ORM} \rightarrow \textbf{Tables}
\item
  Automatic SQL generation
\item
  Type-safe queries
\end{itemize}

\end{solutionbox}
\begin{mnemonicbox}
``Objects Relate Magically''

\end{mnemonicbox}
\subsection*{Question 2(a) OR [3
marks]}\label{q2a}

\textbf{Describe MOUSEEVENT and MOUSELISTENER interface with example.}

\begin{solutionbox}

\textbf{MouseEvent:} Generated when mouse actions occur on components.

\textbf{MouseListener Interface Methods:}

\begin{itemize}
\tightlist
\item
  \textbf{mouseClicked()}: Mouse button clicked
\item
  \textbf{mousePressed()}: Mouse button pressed
\item
  \textbf{mouseReleased()}: Mouse button released
\item
  \textbf{mouseEntered()}: Mouse enters component
\item
  \textbf{mouseExited()}: Mouse exits component
\end{itemize}

\textbf{Example:}

\begin{verbatim}
public class MouseExample extends JFrame implements MouseListener \{
    JLabel label;
    
    public MouseExample() \{
        label = new JLabel("Click me!");
        label.addMouseListener(this);
        add(label);
        setSize(200, 100);
        setVisible(true);
    \}
    
    public void mouseClicked(MouseEvent e) \{
        label.setText("Clicked!");
    \}
    
    // Other methods...
\}
\end{verbatim}

\end{solutionbox}
\begin{mnemonicbox}
``Click-Press-Release-Enter-Exit''

\end{mnemonicbox}
\subsection*{Question 2(b) OR [4
marks]}\label{q2b}

\textbf{List and explain the components of the JDBC API.}

\begin{solutionbox}


{\def\LTcaptype{none} % do not increment counter
\vspace{-5pt}
\captionof{table}{JDBC API Components}
\vspace{-10pt}
\begin{longtable}[]{@{}
  >{\raggedright\arraybackslash}p{(\linewidth - 4\tabcolsep) * \real{0.3333}}
  >{\raggedright\arraybackslash}p{(\linewidth - 4\tabcolsep) * \real{0.2727}}
  >{\raggedright\arraybackslash}p{(\linewidth - 4\tabcolsep) * \real{0.3939}}@{}}
\toprule\noalign{}
\begin{minipage}[b]{\linewidth}\raggedright
Component
\end{minipage} & \begin{minipage}[b]{\linewidth}\raggedright
Purpose
\end{minipage} & \begin{minipage}[b]{\linewidth}\raggedright
Key Classes
\end{minipage} \\
\midrule\noalign{}
\endhead
\bottomrule\noalign{}
\endlastfoot
\textbf{DriverManager} & Manages drivers &
\texttt{DriverManager.getConnection()} \\
\textbf{Connection} & Database connection & \texttt{Connection\ conn} \\
\textbf{Statement} & SQL execution & \texttt{Statement\ stmt} \\
\textbf{ResultSet} & Query results & \texttt{ResultSet\ rs} \\
\end{longtable}
}

\textbf{Component Details:}

\begin{itemize}
\tightlist
\item
  \textbf{DriverManager}: Establishes connection with database
\item
  \textbf{Connection}: Represents database session
\item
  \textbf{Statement}: Executes SQL queries
\item
  \textbf{ResultSet}: Holds query results
\end{itemize}

\textbf{Basic Usage:}

\begin{verbatim}
Connection conn = DriverManager.getConnection(url, user, pass);
Statement stmt = conn.createStatement();
ResultSet rs = stmt.executeQuery("SELECT * FROM users");
\end{verbatim}

\end{solutionbox}
\begin{mnemonicbox}
``Driver Connects, Statement Executes, ResultSet
Returns''

\end{mnemonicbox}
\subsection*{Question 2(c) OR [7
marks]}\label{q2c}

\textbf{Draw and explain the architecture of Hibernate.}

\begin{solutionbox}

\begin{verbatim}
graph TB
    A[Java Application] {-{-} B[Hibernate API]}
    B {-{-} C[Configuration]}
    B {-{-} D[SessionFactory]}
    D {-{-} E[Session]}
    E {-{-} F[Transaction]}
    E {-{-} G[Query]}
    E {-{-} H[Criteria]}
    I[Mapping Files] {-{-} D}
    J[hibernate.cfg.xml] {-{-} C}
    E {-{-} K[Database]}
\end{verbatim}

\textbf{Architecture Components:}


{\def\LTcaptype{none} % do not increment counter
\vspace{-5pt}
\captionof{table}{Hibernate Architecture}
\vspace{-10pt}
\begin{longtable}[]{@{}ll@{}}
\toprule\noalign{}
Component & Function \\
\midrule\noalign{}
\endhead
\bottomrule\noalign{}
\endlastfoot
\textbf{Configuration} & Reads config files \\
\textbf{SessionFactory} & Creates Session objects \\
\textbf{Session} & Interface to database \\
\textbf{Transaction} & Manages transactions \\
\textbf{Query} & HQL/SQL queries \\
\end{longtable}
}

\textbf{Layer Description:}

\begin{itemize}
\tightlist
\item
  \textbf{Application Layer}: Java objects and business logic
\item
  \textbf{Hibernate Layer}: ORM mapping and session management
\item
  \textbf{Database Layer}: Actual data storage
\end{itemize}

\textbf{Key Features:}

\begin{itemize}
\tightlist
\item
  \textbf{Automatic table creation}: Based on entity classes
\item
  \textbf{HQL support}: Object-oriented query language
\item
  \textbf{Caching}: First and second level caching
\end{itemize}

\end{solutionbox}
\begin{mnemonicbox}
``Config-Factory-Session-Transaction: CFST''

\end{mnemonicbox}
\subsection*{Question 3(a) [3 marks]}\label{q3a}

\textbf{Describe various features of Servlet.}

\begin{solutionbox}


{\def\LTcaptype{none} % do not increment counter
\vspace{-5pt}
\captionof{table}{Servlet Features}
\vspace{-10pt}
\begin{longtable}[]{@{}ll@{}}
\toprule\noalign{}
Feature & Description \\
\midrule\noalign{}
\endhead
\bottomrule\noalign{}
\endlastfoot
\textbf{Platform Independent} & Runs on any OS with JVM \\
\textbf{Performance} & Better than CGI \\
\textbf{Robust} & JVM managed memory \\
\textbf{Secure} & Java security features \\
\end{longtable}
}

\textbf{Key Features:}

\begin{itemize}
\tightlist
\item
  \textbf{Server-side processing}: Handles client requests
\item
  \textbf{Protocol independent}: HTTP, FTP, SMTP support
\item
  \textbf{Extensible}: Can be extended easily
\item
  \textbf{Portable}: Write once, run anywhere
\end{itemize}

\end{solutionbox}
\begin{mnemonicbox}
``Platform Performance Robust Secure''

\end{mnemonicbox}
\subsection*{Question 3(b) [4 marks]}\label{q3b}

\textbf{Explain Servlet life cycle.}

\begin{solutionbox}

\begin{center}
\textbf{Mermaid Diagram (Code)}
\begin{verbatim}
{Shaded}
{Highlighting}[]
graph LR
    A[Servlet Loading] {-{-}{} B[Servlet Instantiation]}
    B {-{-}{} C[init Method]}
    C {-{-}{} D[service Method]}
    D {-{-}{} E\{More Requests?\}}
    E {-{-}{}|Yes| D}
    E {-{-}{}|No| F[destroy Method]}
    F {-{-}{} G[Servlet Unloaded]}
{Highlighting}
{Shaded}
\end{verbatim}
\end{center}

\textbf{Life Cycle Stages:}


{\def\LTcaptype{none} % do not increment counter
\vspace{-5pt}
\captionof{table}{Servlet Life Cycle}
\vspace{-10pt}
\begin{longtable}[]{@{}lll@{}}
\toprule\noalign{}
Stage & Method & Purpose \\
\midrule\noalign{}
\endhead
\bottomrule\noalign{}
\endlastfoot
\textbf{Loading} & Class loading & JVM loads servlet class \\
\textbf{Instantiation} & Constructor & Creates servlet object \\
\textbf{Initialization} & \texttt{init()} & One-time setup \\
\textbf{Request Processing} & \texttt{service()} & Handles requests \\
\textbf{Destruction} & \texttt{destroy()} & Cleanup resources \\
\end{longtable}
}

\textbf{Method Details:}

\begin{itemize}
\tightlist
\item
  \textbf{init()}: Called once when servlet loads
\item
  \textbf{service()}: Called for each request
\item
  \textbf{destroy()}: Called when servlet unloads
\end{itemize}

\end{solutionbox}
\begin{mnemonicbox}
``Load-Create-Init-Service-Destroy''

\end{mnemonicbox}
\subsection*{Question 3(c) [7 marks]}\label{q3c}

\textbf{Explain the session tracking in Servlet with example.}

\begin{solutionbox}

\textbf{Session Tracking Methods:}


{\def\LTcaptype{none} % do not increment counter
\vspace{-5pt}
\captionof{table}{Session Tracking Techniques}
\vspace{-10pt}
\begin{longtable}[]{@{}lll@{}}
\toprule\noalign{}
Method & Description & Pros/Cons \\
\midrule\noalign{}
\endhead
\bottomrule\noalign{}
\endlastfoot
\textbf{Cookies} & Client-side storage & Easy/Privacy issues \\
\textbf{URL Rewriting} & Append session ID & Universal/Ugly URLs \\
\textbf{Hidden Fields} & Form-based tracking & Simple/Form dependent \\
\textbf{HttpSession} & Server-side object & Secure/Memory usage \\
\end{longtable}
}

\textbf{HttpSession Example:}

\begin{verbatim}
protected void doGet(HttpServletRequest request, 
                    HttpServletResponse response) \{
    HttpSession session = request.getSession();
    
    // Store data
    session.setAttribute("username", "john");
    
    // Retrieve data
    String user = (String) session.getAttribute("username");
    
    // Session info
    String sessionId = session.getId();
    boolean isNew = session.isNew();
    
    PrintWriter out = response.getWriter();
    out.println("User: " + user);
    out.println("Session ID: " + sessionId);
\}
\end{verbatim}

\textbf{Session Management:}

\begin{itemize}
\tightlist
\item
  \textbf{Creation}: \texttt{request.getSession()}
\item
  \textbf{Storage}: \texttt{session.setAttribute()}
\item
  \textbf{Retrieval}: \texttt{session.getAttribute()}
\item
  \textbf{Invalidation}: \texttt{session.invalidate()}
\end{itemize}

\end{solutionbox}
\begin{mnemonicbox}
``Cookies-URLs-Hidden-HttpSession: CUHS''

\end{mnemonicbox}
\subsection*{Question 3(a) OR [3
marks]}\label{q3a}

\textbf{Explain methods of Servlet life cycle.}

\begin{solutionbox}

\textbf{Life Cycle Methods:}


{\def\LTcaptype{none} % do not increment counter
\vspace{-5pt}
\captionof{table}{Servlet Life Cycle Methods}
\vspace{-10pt}
\begin{longtable}[]{@{}
  >{\raggedright\arraybackslash}p{(\linewidth - 4\tabcolsep) * \real{0.2424}}
  >{\raggedright\arraybackslash}p{(\linewidth - 4\tabcolsep) * \real{0.3939}}
  >{\raggedright\arraybackslash}p{(\linewidth - 4\tabcolsep) * \real{0.3636}}@{}}
\toprule\noalign{}
\begin{minipage}[b]{\linewidth}\raggedright
Method
\end{minipage} & \begin{minipage}[b]{\linewidth}\raggedright
Called When
\end{minipage} & \begin{minipage}[b]{\linewidth}\raggedright
Parameters
\end{minipage} \\
\midrule\noalign{}
\endhead
\bottomrule\noalign{}
\endlastfoot
\textbf{init()} & Servlet initialization &
\texttt{ServletConfig\ config} \\
\textbf{service()} & Each request &
\texttt{ServletRequest\ req,\ ServletResponse\ res} \\
\textbf{destroy()} & Servlet cleanup & None \\
\end{longtable}
}

\textbf{Method Details:}

\begin{itemize}
\tightlist
\item
  \textbf{init(ServletConfig config)}: Initialization code, database
  connections
\item
  \textbf{service(req, res)}: Request handling, business logic
\item
  \textbf{destroy()}: Cleanup code, close resources
\end{itemize}

\textbf{Example:}

\begin{verbatim}
public void init(ServletConfig config) \{
    // Initialize database connection
\}

public void service(ServletRequest req, ServletResponse res) \{
    // Handle request
\}

public void destroy() \{
    // Close connections
\}
\end{verbatim}

\end{solutionbox}
\begin{mnemonicbox}
``Init-Service-Destroy: ISD''

\end{mnemonicbox}
\subsection*{Question 3(b) OR [4
marks]}\label{q3b}

\textbf{Describe HTTPSERVLET class with example.}

\begin{solutionbox}

\textbf{HttpServlet Class:} Abstract class extending GenericServlet,
specifically for HTTP protocol.

\textbf{HTTP Methods:}


{\def\LTcaptype{none} % do not increment counter
\vspace{-5pt}
\captionof{table}{HttpServlet Methods}
\vspace{-10pt}
\begin{longtable}[]{@{}lll@{}}
\toprule\noalign{}
Method & HTTP Verb & Purpose \\
\midrule\noalign{}
\endhead
\bottomrule\noalign{}
\endlastfoot
\textbf{doGet()} & GET & Retrieve data \\
\textbf{doPost()} & POST & Submit data \\
\textbf{doPut()} & PUT & Update data \\
\textbf{doDelete()} & DELETE & Remove data \\
\end{longtable}
}

\textbf{Example:}

\begin{verbatim}
public class MyServlet extends HttpServlet \{
    protected void doGet(HttpServletRequest request,
                        HttpServletResponse response) \{
        response.setContentType("text/html");
        PrintWriter out = response.getWriter();
        out.println("{h1GET Request/h1"});
    \}
    
    protected void doPost(HttpServletRequest request,
                         HttpServletResponse response) \{
        String name = request.getParameter("name");
        response.getWriter().println("Hello " + name);
    \}
\}
\end{verbatim}

\textbf{Key Features:}

\begin{itemize}
\tightlist
\item
  \textbf{HTTP-specific}: Designed for web applications
\item
  \textbf{Method handling}: Separate methods for different HTTP verbs
\item
  \textbf{Request/Response}: HttpServletRequest and HttpServletResponse
\end{itemize}

\end{solutionbox}
\begin{mnemonicbox}
``Get-Post-Put-Delete: GPPD''

\end{mnemonicbox}
\subsection*{Question 3(c) OR [7
marks]}\label{q3c}

\textbf{Differentiate GET and POST methods and write a java code to
develop Servlet using POST method.}

\begin{solutionbox}


{\def\LTcaptype{none} % do not increment counter
\vspace{-5pt}
\captionof{table}{GET vs POST Comparison}
\vspace{-10pt}
\begin{longtable}[]{@{}lll@{}}
\toprule\noalign{}
Feature & GET & POST \\
\midrule\noalign{}
\endhead
\bottomrule\noalign{}
\endlastfoot
\textbf{Data Location} & URL parameters & Request body \\
\textbf{Data Limit} & Limited (\textasciitilde2KB) & Unlimited \\
\textbf{Security} & Less secure & More secure \\
\textbf{Caching} & Cacheable & Not cacheable \\
\textbf{Bookmarking} & Possible & Not possible \\
\end{longtable}
}

\textbf{POST Method Servlet Example:}

\begin{verbatim}
import javax.servlet.*;
import javax.servlet.http.*;
import java.io.*;

public class LoginServlet extends HttpServlet \{
    protected void doPost(HttpServletRequest request,
                         HttpServletResponse response) 
                         throws ServletException, IOException \{
        
        response.setContentType("text/html");
        PrintWriter out = response.getWriter();
        
        // Get form data
        String username = request.getParameter("username");
        String password = request.getParameter("password");
        
        // Validate credentials
        if("admin".equals(username) \&\& "123".equals(password)) \{
            out.println("{h2Login Successful!/h2"});
            out.println("{pWelcome "} + username + "{/p"});
        \} else \{
            out.println("{h2Login Failed!/h2"});
            out.println("{pInvalid credentials/p"});
        \}
        
        out.close();
    \}
\}
\end{verbatim}

\textbf{HTML Form:}

\begin{verbatim}
{}form method="post" action="LoginServlet"{}
    Username: {}input type="text" name="username"{}br /{}
    Password: {}input type="password" name="password"{}br /{}
    {}input type="submit" value="Login"{}
{/}form{}
\end{verbatim}

\textbf{Key Differences:}

\begin{itemize}
\tightlist
\item
  \textbf{GET}: Data in URL, visible, limited size
\item
  \textbf{POST}: Data in body, hidden, unlimited size
\end{itemize}

\end{solutionbox}
\begin{mnemonicbox}
``GET Grabs, POST Protects''

\end{mnemonicbox}
\subsection*{Question 4(a) [3 marks]}\label{q4a}

\textbf{List JSP Implicit Objects and explain any two.}

\begin{solutionbox}

\textbf{JSP Implicit Objects List:}

\begin{enumerate}
\tightlist
\item
  \textbf{request} (HttpServletRequest)
\item
  \textbf{response} (HttpServletResponse)
\item
  \textbf{session} (HttpSession)
\item
  \textbf{application} (ServletContext)
\item
  \textbf{out} (JspWriter)
\item
  \textbf{page} (Object)
\item
  \textbf{pageContext} (PageContext)
\item
  \textbf{config} (ServletConfig)
\item
  \textbf{exception} (Throwable)
\end{enumerate}

\textbf{Detailed Explanation:}

\textbf{request Object:}

\begin{itemize}
\tightlist
\item
  \textbf{Type}: HttpServletRequest
\item
  \textbf{Purpose}: Access request data and parameters
\item
  \textbf{Example}:
  \texttt{String\ name\ =\ request.getParameter("name");}
\end{itemize}

\textbf{session Object:}

\begin{itemize}
\tightlist
\item
  \textbf{Type}: HttpSession
\item
  \textbf{Purpose}: Store user-specific data across requests
\item
  \textbf{Example}: \texttt{session.setAttribute("user",\ username);}
\end{itemize}

\end{solutionbox}
\begin{mnemonicbox}
``Request Response Session Application Out''

\end{mnemonicbox}
\subsection*{Question 4(b) [4 marks]}\label{q4b}

\textbf{Explain features of JSP.}

\begin{solutionbox}


{\def\LTcaptype{none} % do not increment counter
\vspace{-5pt}
\captionof{table}{JSP Features}
\vspace{-10pt}
\begin{longtable}[]{@{}lll@{}}
\toprule\noalign{}
Feature & Description & Benefit \\
\midrule\noalign{}
\endhead
\bottomrule\noalign{}
\endlastfoot
\textbf{Easy Development} & HTML + Java & Faster coding \\
\textbf{Platform Independent} & Write once, run anywhere &
Portability \\
\textbf{Component-based} & Reusable components & Maintainability \\
\textbf{Secure} & Java security model & Safe execution \\
\end{longtable}
}

\textbf{Key Features:}

\begin{itemize}
\tightlist
\item
  \textbf{Separation of Concerns}: Design and logic separated
\item
  \textbf{Extensible}: Custom tags and libraries
\item
  \textbf{Compiled}: Translated to servlets for performance
\item
  \textbf{Expression Language}: Simplified syntax
\end{itemize}

\textbf{JSP Elements:}

\begin{itemize}
\tightlist
\item
  \textbf{Directives}: \texttt{\textless{}\%@\ \%\textgreater{}}
\item
  \textbf{Declarations}: \texttt{\textless{}\%!\ \%\textgreater{}}
\item
  \textbf{Expressions}: \texttt{\textless{}\%=\ \%\textgreater{}}
\item
  \textbf{Scriptlets}: \texttt{\textless{}\%\ \%\textgreater{}}
\end{itemize}

\end{solutionbox}
\begin{mnemonicbox}
``Easy Platform Component Secure''

\end{mnemonicbox}
\subsection*{Question 4(c) [7 marks]}\label{q4c}

\textbf{Describe how to call JSP from servlet with example.}

\begin{solutionbox}

\textbf{Methods to Call JSP from Servlet:}


{\def\LTcaptype{none} % do not increment counter
\vspace{-5pt}
\captionof{table}{JSP Calling Methods}
\vspace{-10pt}
\begin{longtable}[]{@{}lll@{}}
\toprule\noalign{}
Method & Interface & Purpose \\
\midrule\noalign{}
\endhead
\bottomrule\noalign{}
\endlastfoot
\textbf{Forward} & RequestDispatcher & Transfer control \\
\textbf{Include} & RequestDispatcher & Include content \\
\textbf{Redirect} & HttpServletResponse & New request \\
\end{longtable}
}

\textbf{Forward Example:}

\textbf{Servlet Code:}

\begin{verbatim}
public class DataServlet extends HttpServlet \{
    protected void doGet(HttpServletRequest request,
                        HttpServletResponse response) 
                        throws ServletException, IOException \{
        
        // Process data
        String username = "John Doe";
        int age = 25;
        
        // Set attributes
        request.setAttribute("username", username);
        request.setAttribute("age", age);
        
        // Forward to JSP
        RequestDispatcher dispatcher = 
            request.getRequestDispatcher("display.jsp");
        dispatcher.forward(request, response);
    \}
\}
\end{verbatim}

\textbf{JSP Code (display.jsp):}

\begin{verbatim}
{\%@ page} language="java" contentType="text/html" \%{}
{html}
{headtitleUser Info/title/head}
{body}
    {h2User Information/h2}
    {pName: }{\%=} request.getAttribute("username") \%{}{/p}
    {pAge: }{\%=} request.getAttribute("age") \%{}{/p}
{/body}
{/html}
\end{verbatim}

\textbf{Steps:}

\begin{enumerate}
\tightlist
\item
  \textbf{Process data} in servlet
\item
  \textbf{Set attributes} in request
\item
  \textbf{Get RequestDispatcher} with JSP path
\item
  \textbf{Forward} to JSP
\end{enumerate}

\end{solutionbox}
\begin{mnemonicbox}
``Process-Set-Get-Forward: PSGF''

\end{mnemonicbox}
\subsection*{Question 4(a) OR [3
marks]}\label{q4a}

\textbf{List and explain JSP scripting elements.}

\begin{solutionbox}


{\def\LTcaptype{none} % do not increment counter
\vspace{-5pt}
\captionof{table}{JSP Scripting Elements}
\vspace{-10pt}
\begin{longtable}[]{@{}
  >{\raggedright\arraybackslash}p{(\linewidth - 6\tabcolsep) * \real{0.2571}}
  >{\raggedright\arraybackslash}p{(\linewidth - 6\tabcolsep) * \real{0.2286}}
  >{\raggedright\arraybackslash}p{(\linewidth - 6\tabcolsep) * \real{0.2571}}
  >{\raggedright\arraybackslash}p{(\linewidth - 6\tabcolsep) * \real{0.2571}}@{}}
\toprule\noalign{}
\begin{minipage}[b]{\linewidth}\raggedright
Element
\end{minipage} & \begin{minipage}[b]{\linewidth}\raggedright
Syntax
\end{minipage} & \begin{minipage}[b]{\linewidth}\raggedright
Purpose
\end{minipage} & \begin{minipage}[b]{\linewidth}\raggedright
Example
\end{minipage} \\
\midrule\noalign{}
\endhead
\bottomrule\noalign{}
\endlastfoot
\textbf{Directive} & \texttt{\textless{}\%@\ \%\textgreater{}} & Page
settings &
\texttt{\textless{}\%@\ page\ import="java.util.*"\ \%\textgreater{}} \\
\textbf{Declaration} & \texttt{\textless{}\%!\ \%\textgreater{}} &
Define methods/variables &
\texttt{\textless{}\%!\ int\ count\ =\ 0;\ \%\textgreater{}} \\
\textbf{Expression} & \texttt{\textless{}\%=\ \%\textgreater{}} & Output
values & \texttt{\textless{}\%=\ new\ Date()\ \%\textgreater{}} \\
\textbf{Scriptlet} & \texttt{\textless{}\%\ \%\textgreater{}} & Java
code &
\texttt{\textless{}\%\ for(int\ i=0;\ i\textless{}5;\ i++)\ \{\ \%\textgreater{}} \\
\end{longtable}
}

\textbf{Detailed Explanation:}

\textbf{Directives:}

\begin{itemize}
\tightlist
\item
  \textbf{Page directive}: Import packages, set content type
\item
  \textbf{Include directive}: Include other files
\item
  \textbf{Taglib directive}: Custom tag libraries
\end{itemize}

\textbf{Declarations:}

\begin{itemize}
\tightlist
\item
  Define instance variables and methods
\item
  Become part of servlet class
\end{itemize}

\end{solutionbox}
\begin{mnemonicbox}
``Direct Declare Express Script''

\end{mnemonicbox}
\subsection*{Question 4(b) OR [4
marks]}\label{q4b}

\textbf{Explain JSP life cycle.}

\begin{solutionbox}

\begin{center}
\textbf{Mermaid Diagram (Code)}
\begin{verbatim}
{Shaded}
{Highlighting}[]
graph LR
    A[JSP Page Request] {-{-}{} B[Translation to Servlet]}
    B {-{-}{} C[Compilation to Bytecode]}
    C {-{-}{} D[Servlet Loading]}
    D {-{-}{} E[jspInit Method]}
    E {-{-}{} F[\_jspService Method]}
    F {-{-}{} G\{More Requests?\}}
    G {-{-}{}|Yes| F}
    G {-{-}{}|No| H[jspDestroy Method]}
{Highlighting}
{Shaded}
\end{verbatim}
\end{center}

\textbf{Life Cycle Phases:}


{\def\LTcaptype{none} % do not increment counter
\vspace{-5pt}
\captionof{table}{JSP Life Cycle}
\vspace{-10pt}
\begin{longtable}[]{@{}lll@{}}
\toprule\noalign{}
Phase & Method & Purpose \\
\midrule\noalign{}
\endhead
\bottomrule\noalign{}
\endlastfoot
\textbf{Translation} & - & JSP to Java servlet \\
\textbf{Compilation} & - & Java to bytecode \\
\textbf{Initialization} & \texttt{jspInit()} & Setup resources \\
\textbf{Request Processing} & \texttt{\_jspService()} & Handle
requests \\
\textbf{Destruction} & \texttt{jspDestroy()} & Cleanup \\
\end{longtable}
}

\textbf{Key Points:}

\begin{itemize}
\tightlist
\item
  \textbf{Translation}: JSP engine converts JSP to servlet
\item
  \textbf{Compilation}: Java compiler creates .class file
\item
  \textbf{Execution}: Servlet container executes compiled servlet
\end{itemize}

\end{solutionbox}
\begin{mnemonicbox}
``Translate-Compile-Init-Service-Destroy''

\end{mnemonicbox}
\subsection*{Question 4(c) OR [7
marks]}\label{q4c}

\textbf{Define cookie. Explain working of cookie with example.}

\begin{solutionbox}

\textbf{Cookie Definition:} A cookie is a small piece of data stored on
the client's computer by the web browser while browsing a website.

\textbf{Cookie Working Process:}

\begin{verbatim}
sequenceDiagram
    participant Client
    participant Server
    Client{-Server: HTTP Request}
    Server{-Client: HTTP Response + Set{-}Cookie}
    Client{-Server: HTTP Request + Cookie}
    Server{-Client: HTTP Response (uses cookie data)}
\end{verbatim}


{\def\LTcaptype{none} % do not increment counter
\vspace{-5pt}
\captionof{table}{Cookie Attributes}
\vspace{-10pt}
\begin{longtable}[]{@{}lll@{}}
\toprule\noalign{}
Attribute & Purpose & Example \\
\midrule\noalign{}
\endhead
\bottomrule\noalign{}
\endlastfoot
\textbf{Name} & Cookie identifier & \texttt{username} \\
\textbf{Value} & Cookie data & \texttt{john123} \\
\textbf{Domain} & Valid domain & \texttt{.example.com} \\
\textbf{Path} & Valid path & \texttt{/shop/} \\
\textbf{Max-Age} & Expiry time & \texttt{3600} seconds \\
\end{longtable}
}

\textbf{Cookie Example:}

\textbf{Creating Cookie (Servlet):}

\begin{verbatim}
public class SetCookieServlet extends HttpServlet \{
    protected void doGet(HttpServletRequest request,
                        HttpServletResponse response) \{
        
        // Create cookie
        Cookie userCookie = new Cookie("username", "john123");
        userCookie.setMaxAge(60 * 60 * 24); // 1 day
        userCookie.setPath("/");
        
        // Add to response
        response.addCookie(userCookie);
        
        response.getWriter().println("Cookie set successfully!");
    \}
\}
\end{verbatim}

\textbf{Reading Cookie (Servlet):}

\begin{verbatim}
public class GetCookieServlet extends HttpServlet \{
    protected void doGet(HttpServletRequest request,
                        HttpServletResponse response) \{
        
        Cookie[] cookies = request.getCookies();
        String username = null;
        
        if(cookies != null) \{
            for(Cookie cookie : cookies) \{
                if("username".equals(cookie.getName())) \{
                    username = cookie.getValue();
                    break;
                \}
            \}
        \}
        
        response.getWriter().println("Welcome back, " + username);
    \}
\}
\end{verbatim}

\textbf{Cookie Benefits:}

\begin{itemize}
\tightlist
\item
  \textbf{User personalization}: Remember preferences
\item
  \textbf{Session tracking}: Maintain state
\item
  \textbf{Analytics}: Track user behavior
\end{itemize}

\end{solutionbox}
\begin{mnemonicbox}
``Create-Set-Add-Read: CSAR''

\end{mnemonicbox}
\subsection*{Question 5(a) [3 marks]}\label{q5a}

\textbf{Write difference between JSP and Servlet.}

\begin{solutionbox}


{\def\LTcaptype{none} % do not increment counter
\vspace{-5pt}
\captionof{table}{JSP vs Servlet Comparison}
\vspace{-10pt}
\begin{longtable}[]{@{}lll@{}}
\toprule\noalign{}
Feature & JSP & Servlet \\
\midrule\noalign{}
\endhead
\bottomrule\noalign{}
\endlastfoot
\textbf{Development} & HTML + Java & Pure Java \\
\textbf{Compilation} & Automatic & Manual \\
\textbf{Maintenance} & Easier & More complex \\
\textbf{Performance} & Slower (first request) & Faster \\
\textbf{Purpose} & Presentation layer & Business logic \\
\end{longtable}
}

\textbf{Key Differences:}

\begin{itemize}
\tightlist
\item
  \textbf{JSP}: Better for presentation, easier for web designers
\item
  \textbf{Servlet}: Better for business logic, more control
\item
  \textbf{Coding}: JSP mixes HTML and Java, Servlet is pure Java
\item
  \textbf{Compilation}: JSP auto-compiles, Servlet needs manual
  compilation
\end{itemize}

\end{solutionbox}
\begin{mnemonicbox}
``JSP for Presentation, Servlet for Logic''

\end{mnemonicbox}
\subsection*{Question 5(b) [4 marks]}\label{q5b}

\textbf{Define Spring Boot and explain its advantages.}

\begin{solutionbox}

\textbf{Spring Boot Definition:} Spring Boot is a framework that
simplifies the development of Spring-based applications by providing
auto-configuration and embedded servers.


{\def\LTcaptype{none} % do not increment counter
\vspace{-5pt}
\captionof{table}{Spring Boot Advantages}
\vspace{-10pt}
\begin{longtable}[]{@{}
  >{\raggedright\arraybackslash}p{(\linewidth - 2\tabcolsep) * \real{0.4583}}
  >{\raggedright\arraybackslash}p{(\linewidth - 2\tabcolsep) * \real{0.5417}}@{}}
\toprule\noalign{}
\begin{minipage}[b]{\linewidth}\raggedright
Advantage
\end{minipage} & \begin{minipage}[b]{\linewidth}\raggedright
Description
\end{minipage} \\
\midrule\noalign{}
\endhead
\bottomrule\noalign{}
\endlastfoot
\textbf{Auto Configuration} & Automatically configures Spring
applications \\
\textbf{Embedded Servers} & Built-in Tomcat, Jetty support \\
\textbf{Starter Dependencies} & Pre-configured dependency sets \\
\textbf{Production Ready} & Health checks, metrics, monitoring \\
\end{longtable}
}

\textbf{Key Features:}

\begin{itemize}
\tightlist
\item
  \textbf{Rapid Development}: Minimal configuration required
\item
  \textbf{Microservices}: Perfect for microservice architecture
\item
  \textbf{No XML}: Convention over configuration
\item
  \textbf{Cloud Ready}: Easy deployment to cloud platforms
\end{itemize}

\textbf{Example:}

\begin{verbatim}
@SpringBootApplication
public class MyApplication \{
    public static void main(String[] args) \{
        SpringApplication.run(MyApplication.class, args);
    \}
\}
\end{verbatim}

\end{solutionbox}
\begin{mnemonicbox}
``Auto Embedded Starter Production''

\end{mnemonicbox}
\subsection*{Question 5(c) [7 marks]}\label{q5c}

\textbf{Explain the architecture of Spring framework.}

\begin{solutionbox}

\begin{verbatim}
graph TB
    A[Spring Framework Architecture]
    A {-{-} B[Core Container]}
    A {-{-} C[Data Access/Integration]}
    A {-{-} D[Web MVC]}
    A {-{-} E[AOP]}
    A {-{-} F[Test]}
    
    B {-{-} B1[Core]}
    B {-{-} B2[Beans]}
    B {-{-} B3[Context]}
    B {-{-} B4[Expression]}
    
    C {-{-} C1[JDBC]}
    C {-{-} C2[ORM]}
    C {-{-} C3[JMS]}
    C {-{-} C4[Transaction]}
    
    D {-{-} D1[Web]}
    D {-{-} D2[Servlet]}
    D {-{-} D3[Portlet]}
    D {-{-} D4[Struts]}
\end{verbatim}

\textbf{Architecture Layers:}


{\def\LTcaptype{none} % do not increment counter
\vspace{-5pt}
\captionof{table}{Spring Framework Modules}
\vspace{-10pt}
\begin{longtable}[]{@{}lll@{}}
\toprule\noalign{}
Module & Components & Purpose \\
\midrule\noalign{}
\endhead
\bottomrule\noalign{}
\endlastfoot
\textbf{Core Container} & Core, Beans, Context & IoC and DI \\
\textbf{Data Access} & JDBC, ORM, JMS & Database operations \\
\textbf{Web MVC} & Web, Servlet, MVC & Web applications \\
\textbf{AOP} & Aspects, Weaving & Cross-cutting concerns \\
\end{longtable}
}

\textbf{Core Concepts:}

\begin{itemize}
\tightlist
\item
  \textbf{IoC (Inversion of Control)}: Framework controls object
  creation
\item
  \textbf{DI (Dependency Injection)}: Dependencies injected
  automatically
\item
  \textbf{AOP (Aspect-Oriented Programming)}: Modular cross-cutting
  concerns
\item
  \textbf{MVC}: Model-View-Controller pattern
\end{itemize}

\textbf{Spring Container:}

\begin{itemize}
\tightlist
\item
  \textbf{BeanFactory}: Basic container
\item
  \textbf{ApplicationContext}: Advanced container with additional
  features
\end{itemize}

\textbf{Configuration Methods:}

\begin{itemize}
\tightlist
\item
  \textbf{XML Configuration}: Traditional approach
\item
  \textbf{Annotation Configuration}: Modern approach
\item
  \textbf{Java Configuration}: Type-safe configuration
\end{itemize}

\end{solutionbox}
\begin{mnemonicbox}
``Core Data Web AOP Test''

\end{mnemonicbox}
\subsection*{Question 5(a) OR [3
marks]}\label{q5a}

\textbf{Write advantages of JSP over Servlet.}

\begin{solutionbox}


{\def\LTcaptype{none} % do not increment counter
\vspace{-5pt}
\captionof{table}{JSP Advantages over Servlet}
\vspace{-10pt}
\begin{longtable}[]{@{}
  >{\raggedright\arraybackslash}p{(\linewidth - 4\tabcolsep) * \real{0.3143}}
  >{\raggedright\arraybackslash}p{(\linewidth - 4\tabcolsep) * \real{0.1429}}
  >{\raggedright\arraybackslash}p{(\linewidth - 4\tabcolsep) * \real{0.5429}}@{}}
\toprule\noalign{}
\begin{minipage}[b]{\linewidth}\raggedright
Advantage
\end{minipage} & \begin{minipage}[b]{\linewidth}\raggedright
JSP
\end{minipage} & \begin{minipage}[b]{\linewidth}\raggedright
Servlet Limitation
\end{minipage} \\
\midrule\noalign{}
\endhead
\bottomrule\noalign{}
\endlastfoot
\textbf{Easy Development} & HTML + Java tags & Complex HTML in Java \\
\textbf{Automatic Compilation} & Auto-compiled & Manual compilation \\
\textbf{Designer Friendly} & Web designers can work & Java knowledge
required \\
\textbf{Maintenance} & Easier to modify & Code changes need
recompilation \\
\end{longtable}
}

\textbf{Key Advantages:}

\begin{itemize}
\tightlist
\item
  \textbf{Separation of Design and Logic}: HTML and Java separated
\item
  \textbf{Rapid Development}: Faster prototyping and development
\item
  \textbf{Less Code}: No need for out.println() statements
\item
  \textbf{Built-in Objects}: Implicit objects readily available
\end{itemize}

\textbf{Example Comparison:}

\textbf{JSP Code:}

\begin{verbatim}
{html}
{body}
    {h1Welcome }{\%=} request.getParameter("name") \%{}{/h1}
{/body}
{/html}
\end{verbatim}

\textbf{Servlet Code:}

\begin{verbatim}
out.println("{html"});
out.println("{body"});
out.println("{h1Welcome "} + request.getParameter("name") + "{/h1"});
out.println("{/body"});
out.println("{/html"});
\end{verbatim}

\end{solutionbox}
\begin{mnemonicbox}
``Easy Auto Designer Maintenance''

\end{mnemonicbox}
\subsection*{Question 5(b) OR [4
marks]}\label{q5b}

\textbf{Explain the advantages of Spring Boot.}

\begin{solutionbox}


{\def\LTcaptype{none} % do not increment counter
\vspace{-5pt}
\captionof{table}{Spring Boot Advantages}
\vspace{-10pt}
\begin{longtable}[]{@{}
  >{\raggedright\arraybackslash}p{(\linewidth - 4\tabcolsep) * \real{0.3333}}
  >{\raggedright\arraybackslash}p{(\linewidth - 4\tabcolsep) * \real{0.3939}}
  >{\raggedright\arraybackslash}p{(\linewidth - 4\tabcolsep) * \real{0.2727}}@{}}
\toprule\noalign{}
\begin{minipage}[b]{\linewidth}\raggedright
Advantage
\end{minipage} & \begin{minipage}[b]{\linewidth}\raggedright
Description
\end{minipage} & \begin{minipage}[b]{\linewidth}\raggedright
Benefit
\end{minipage} \\
\midrule\noalign{}
\endhead
\bottomrule\noalign{}
\endlastfoot
\textbf{Auto Configuration} & Automatic setup based on classpath &
Reduced configuration \\
\textbf{Embedded Server} & Built-in Tomcat/Jetty & No external
deployment \\
\textbf{Starter POMs} & Pre-configured dependencies & Simplified
dependency management \\
\textbf{Actuator} & Production monitoring & Health checks and metrics \\
\end{longtable}
}

\textbf{Detailed Advantages:}

\textbf{1. Auto Configuration:}

\begin{itemize}
\tightlist
\item
  Automatically configures Spring application based on dependencies
\item
  Reduces boilerplate configuration code
\item
  Convention over configuration approach
\end{itemize}

\textbf{2. Embedded Servers:}

\begin{itemize}
\tightlist
\item
  No need for external application servers
\item
  Easy to run applications with \texttt{java\ -jar}
\item
  Simplified deployment process
\end{itemize}

\textbf{3. Starter Dependencies:}

\begin{itemize}
\tightlist
\item
  Pre-configured dependency sets
\item
  Version compatibility managed
\item
  Quick project setup
\end{itemize}

\textbf{4. Production Features:}

\begin{itemize}
\tightlist
\item
  Health endpoints
\item
  Metrics collection
\item
  Application monitoring
\end{itemize}

\textbf{Example:}

\begin{verbatim}
@SpringBootApplication
@RestController
public class HelloApp \{
    @GetMapping("/hello")
    public String hello() \{
        return "Hello Spring Boot!";
    \}
    
    public static void main(String[] args) \{
        SpringApplication.run(HelloApp.class, args);
    \}
\}
\end{verbatim}

\end{solutionbox}
\begin{mnemonicbox}
``Auto Embedded Starter Production''

\end{mnemonicbox}
\subsection*{Question 5(c) OR [7
marks]}\label{q5c}

\textbf{Explain MVC architecture.}

\begin{solutionbox}

\textbf{MVC (Model-View-Controller) Architecture:}

\begin{center}
\textbf{Mermaid Diagram (Code)}
\begin{verbatim}
{Shaded}
{Highlighting}[]
graph LR
    A[View] {-{-}{} B[Controller]}
    B {-{-}{} C[Model]}
    C {-{-}{} A}
    D[User] {-{-}{} A}
    A {-{-}{} D}
{Highlighting}
{Shaded}
\end{verbatim}
\end{center}

\textbf{MVC Components:}


{\def\LTcaptype{none} % do not increment counter
\vspace{-5pt}
\captionof{table}{MVC Components}
\vspace{-10pt}
\begin{longtable}[]{@{}lll@{}}
\toprule\noalign{}
Component & Responsibility & Example \\
\midrule\noalign{}
\endhead
\bottomrule\noalign{}
\endlastfoot
\textbf{Model} & Data and business logic & Entity classes, DAOs \\
\textbf{View} & User interface & JSP, HTML, Templates \\
\textbf{Controller} & Request handling & Servlets, Spring Controllers \\
\end{longtable}
}

\textbf{Detailed Explanation:}

\textbf{Model:}

\begin{itemize}
\tightlist
\item
  Represents data and business logic
\item
  Database operations
\item
  Data validation
\item
  Business rules implementation
\end{itemize}

\textbf{View:}

\begin{itemize}
\tightlist
\item
  Presentation layer
\item
  User interface components
\item
  Display data to users
\item
  Collect user input
\end{itemize}

\textbf{Controller:}

\begin{itemize}
\tightlist
\item
  Handles user requests
\item
  Coordinates between Model and View
\item
  Process user input
\item
  Select appropriate View
\end{itemize}

\textbf{MVC Flow:}

\begin{verbatim}
sequenceDiagram
    participant User
    participant View
    participant Controller
    participant Model
    
    User{-View: User Input}
    View{-Controller: Request}
    Controller{-Model: Process Data}
    Model{-Controller: Return Data}
    Controller{-View: Select View}
    View{-User: Response}
\end{verbatim}

\textbf{Spring MVC Example:}

\textbf{Controller:}

\begin{verbatim}
@Controller
public class StudentController \{
    @Autowired
    private StudentService studentService;
    
    @GetMapping("/students")
    public ModelAndView getStudents() \{
        List{}Student{} students = studentService.getAllStudents();
        ModelAndView mv = new ModelAndView("students");
        mv.addObject("studentList", students);
        return mv;
    \}
\}
\end{verbatim}

\textbf{Model:}

\begin{verbatim}
@Entity
public class Student \{
    @Id
    private int id;
    private String name;
    private String email;
    
    // getters and setters
\}
\end{verbatim}

\textbf{View (JSP):}

\begin{verbatim}
{html}
{body}
    {h2Student List/h2}
    {c:forEach} items="$\{studentList\}" var="student"{}
        {p}$\{student.name\} {- }$\{student.email\}{/p}
    {/c:forEach}
{/body}
{/html}
\end{verbatim}

\textbf{MVC Advantages:}

\begin{itemize}
\tightlist
\item
  \textbf{Separation of Concerns}: Clear separation of responsibilities
\item
  \textbf{Maintainability}: Easy to maintain and modify
\item
  \textbf{Reusability}: Components can be reused
\item
  \textbf{Testability}: Each component can be tested independently
\item
  \textbf{Parallel Development}: Different teams can work on different
  components
\end{itemize}

\textbf{MVC in Web Applications:}

\begin{itemize}
\tightlist
\item
  \textbf{Model}: Database entities, business logic
\item
  \textbf{View}: JSP pages, HTML templates
\item
  \textbf{Controller}: Servlets, Spring controllers
\end{itemize}

\textbf{Design Patterns Used:}

\begin{itemize}
\tightlist
\item
  \textbf{Front Controller}: Single entry point for requests
\item
  \textbf{Observer Pattern}: Model notifies View of changes
\item
  \textbf{Strategy Pattern}: Different Views for same Model
\end{itemize}

\end{solutionbox}
\begin{mnemonicbox}
``Model manages data, View shows data, Controller
controls flow''

\end{mnemonicbox}

\end{document}
