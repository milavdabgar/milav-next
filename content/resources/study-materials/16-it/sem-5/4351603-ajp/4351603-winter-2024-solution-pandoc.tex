\documentclass[10pt,a4paper]{article}

% content/resources/templates/preamble.tex
\usepackage[margin=0.6in]{geometry}
\author{Milav Dabgar}
\usepackage{amsmath,amssymb,amsthm}
\usepackage{booktabs}
\usepackage{multirow}
\usepackage{xcolor}
\usepackage{tcolorbox}
\tcbuselibrary{breakable,skins}
\usepackage[colorlinks=true,linkcolor=blue]{hyperref}
\usepackage{titlesec}
\usepackage{enumitem}
\usepackage{tikz}
\usepackage{pgfplots}
\usepackage{circuitikz}
\usepackage[version=4]{mhchem}
\usepackage{longtable}
\usepackage{array}
\usepackage{float}
\usepackage{caption}
\usepackage{listings}

\lstset{
  basicstyle=\small\ttfamily,
  breaklines=true,
  breakatwhitespace=false,
  postbreak=\mbox{\textcolor{red}{$\hookrightarrow$}\space},
  float=false,
  numbers=left,
  numberstyle=\tiny\color{gray},
  numbersep=10pt,
  xleftmargin=2em,
  keywordstyle=\color{blue},
  commentstyle=\color{green!60!black},
  stringstyle=\color{purple},
  backgroundcolor=\color{gray!5},
  showstringspaces=false,
  tabsize=2,
  captionpos=b,
  keepspaces=true,
  columns=flexible
}

\pgfplotsset{compat=1.18}
\usetikzlibrary{shapes,arrows,positioning,calc,patterns,decorations.pathmorphing,decorations.markings,arrows.meta}

% Color scheme
\definecolor{headcolor}{RGB}{0,102,204}
\definecolor{keycolor}{RGB}{220,20,60}
\definecolor{solutioncolor}{RGB}{34,139,34}
\definecolor{mnemoniccolor}{RGB}{148,0,211}
\definecolor{codecolor}{RGB}{0,0,100}

% Spacing
\setlength{\parskip}{3pt}
\setlist[itemize]{nosep}
\setlist[enumerate]{nosep}

% Title formatting
\titleformat{\section}{\Large\bfseries\color{headcolor}}{\thesection}{1em}{}
\titleformat{\subsection}{\large\bfseries\color{headcolor}}{\thesubsection}{1em}{}

% Pandoc tightlist compatibility
\providecommand{\tightlist}{%
  \setlength{\itemsep}{0pt}\setlength{\parskip}{0pt}}

% Pandoc longtable compatibility
\newcounter{none}
\def\thenone{}


% content/resources/templates/english-boxes.tex
% This file is currently empty - it exists to maintain consistency with the import structure.
% Add custom environments here if needed in the future.


\begin{document}

\begin{center}
{\Huge\bfseries\color{headcolor} Subject Name Solutions}\\[5pt]
{\LARGE 4351603 -- Winter 2024}\\[3pt]
{\large Semester 1 Study Material}\\[3pt]
{\normalsize\textit{Detailed Solutions and Explanations}}
\end{center}

\vspace{10pt}

\subsection*{Question 1(a) [3 marks]}\label{q1a}

\textbf{Describe JFC with its usage.}

\begin{solutionbox}

JFC (Java Foundation Classes) is a comprehensive GUI framework for
building desktop applications in Java.

{\def\LTcaptype{none} % do not increment counter
\begin{longtable}[]{@{}ll@{}}
\toprule\noalign{}
Component & Description \\
\midrule\noalign{}
\endhead
\bottomrule\noalign{}
\endlastfoot
\textbf{Swing} & Lightweight GUI components \\
\textbf{AWT} & Basic windowing toolkit \\
\textbf{Java 2D} & Advanced graphics and imaging \\
\textbf{Accessibility} & Support for assistive technologies \\
\end{longtable}
}

\begin{itemize}
\tightlist
\item
  \textbf{Primary Usage}: Creating rich desktop applications
\item
  \textbf{Key Advantage}: Platform independence and consistent look
\end{itemize}

\end{solutionbox}
\begin{mnemonicbox}
``JFC = Java's Fantastic Components''

\end{mnemonicbox}
\subsection*{Question 1(b) [4 marks]}\label{q1b}

\textbf{Explain Difference between AWT and Swing.}

\begin{solutionbox}

{\def\LTcaptype{none} % do not increment counter
\begin{longtable}[]{@{}lll@{}}
\toprule\noalign{}
Feature & AWT & Swing \\
\midrule\noalign{}
\endhead
\bottomrule\noalign{}
\endlastfoot
\textbf{Components} & Heavyweight (native) & Lightweight (pure Java) \\
\textbf{Platform} & Platform dependent & Platform independent \\
\textbf{Look \& Feel} & Native OS look & Pluggable look and feel \\
\textbf{Performance} & Faster & Slightly slower \\
\end{longtable}
}

\begin{itemize}
\tightlist
\item
  \textbf{AWT Limitation}: Limited components, platform-specific
  appearance
\item
  \textbf{Swing Advantage}: Rich component set, customizable UI
\end{itemize}

\end{solutionbox}
\begin{mnemonicbox}
``AWT = Always Weighs Too-much, Swing = Simply Works
In New Generation''

\end{mnemonicbox}
\subsection*{Question 1(c) [7 marks]}\label{q1c}

\textbf{List out various Event Listener. Explain anyone.}

\begin{solutionbox}

\textbf{Event Listeners List:}

{\def\LTcaptype{none} % do not increment counter
\begin{longtable}[]{@{}ll@{}}
\toprule\noalign{}
Listener & Purpose \\
\midrule\noalign{}
\endhead
\bottomrule\noalign{}
\endlastfoot
\textbf{ActionListener} & Button clicks, menu selections \\
\textbf{MouseListener} & Mouse events (click, press, release) \\
\textbf{KeyListener} & Keyboard input events \\
\textbf{WindowListener} & Window state changes \\
\textbf{FocusListener} & Component focus events \\
\textbf{ItemListener} & Checkbox/radio button changes \\
\end{longtable}
}

\textbf{ActionListener Explanation:}

\begin{itemize}
\tightlist
\item
  \textbf{Interface Method}: \texttt{actionPerformed(ActionEvent\ e)}
\item
  \textbf{Usage}: Handles button clicks and menu actions
\item
  \textbf{Implementation}: Anonymous class or lambda expression
\end{itemize}

\begin{verbatim}
button.addActionListener(e {-} \{
    System.out.println("Button clicked!");
\);}
\end{verbatim}

\end{solutionbox}
\begin{mnemonicbox}
``AMKWFI Listeners = Action Mouse Key Window Focus
Item''

\end{mnemonicbox}
\subsection*{Question 1(c OR) [7
marks]}\label{question-1c-or-7-marks}

\textbf{List out various Layout Managers. Explain anyone.}

\begin{solutionbox}

\textbf{Layout Managers List:}

{\def\LTcaptype{none} % do not increment counter
\begin{longtable}[]{@{}ll@{}}
\toprule\noalign{}
Layout Manager & Purpose \\
\midrule\noalign{}
\endhead
\bottomrule\noalign{}
\endlastfoot
\textbf{FlowLayout} & Sequential component placement \\
\textbf{BorderLayout} & Five regions (North, South, East, West,
Center) \\
\textbf{GridLayout} & Grid-based arrangement \\
\textbf{CardLayout} & Stack of components \\
\textbf{BoxLayout} & Single row or column \\
\textbf{GridBagLayout} & Complex grid with constraints \\
\end{longtable}
}

\textbf{BorderLayout Explanation:}

\begin{itemize}
\tightlist
\item
  \textbf{Default Layout}: For JFrame and JDialog
\item
  \textbf{Five Regions}: North, South, East, West, Center
\item
  \textbf{Resizing}: Center expands, others stay preferred size
\end{itemize}

\begin{center}
\textbf{Mermaid Diagram (Code)}
\begin{verbatim}
{Shaded}
{Highlighting}[]
graph LR
    A[North] 
    B[West] 
    C[Center] 
    D[East]
    E[South]
    
    A {-{-}{-} B}
    A {-{-}{-} C}
    A {-{-}{-} D}
    B {-{-}{-} C}
    C {-{-}{-} D}
    B {-{-}{-} E}
    C {-{-}{-} E}
    D {-{-}{-} E}
{Highlighting}
{Shaded}
\end{verbatim}
\end{center}

\end{solutionbox}
\begin{mnemonicbox}
``FBGCBG Layouts = Flow Border Grid Card Box
GridBag''

\end{mnemonicbox}
\subsection*{Question 2(a) [3 marks]}\label{q2a}

\textbf{List out and explain steps to connect database.}

\begin{solutionbox}

\textbf{Database Connection Steps:}

{\def\LTcaptype{none} % do not increment counter
\begin{longtable}[]{@{}ll@{}}
\toprule\noalign{}
Step & Action \\
\midrule\noalign{}
\endhead
\bottomrule\noalign{}
\endlastfoot
\textbf{1. Load Driver} & \texttt{Class.forName("driver.class")} \\
\textbf{2. Create Connection} &
\texttt{DriverManager.getConnection()} \\
\textbf{3. Create Statement} & \texttt{connection.createStatement()} \\
\textbf{4. Execute Query} & \texttt{statement.executeQuery()} \\
\textbf{5. Process Results} & \texttt{resultSet.next()} \\
\textbf{6. Close Resources} & Close all connections \\
\end{longtable}
}

\end{solutionbox}
\begin{mnemonicbox}
``LCD EPR = Load Create Driver, Execute Process
Results''

\end{mnemonicbox}
\subsection*{Question 2(b) [4 marks]}\label{q2b}

\textbf{Explain 3-tier architecture with diagram.}

\begin{solutionbox}

3-tier architecture separates application into three logical layers for
better maintainability.

\begin{center}
\textbf{Mermaid Diagram (Code)}
\begin{verbatim}
{Shaded}
{Highlighting}[]
graph LR
    A[Presentation Tier{br/{}Web Browser/UI] }
    B[Application Tier{br/{}Business Logic/Servlets] }
    C[Data Tier{br/{}Database Server]}
    
    A {{-}{-}{} B}
    B {{-}{-}{} C}
{Highlighting}
{Shaded}
\end{verbatim}
\end{center}

{\def\LTcaptype{none} % do not increment counter
\begin{longtable}[]{@{}ll@{}}
\toprule\noalign{}
Tier & Responsibility \\
\midrule\noalign{}
\endhead
\bottomrule\noalign{}
\endlastfoot
\textbf{Presentation} & User interface and user interaction \\
\textbf{Application} & Business logic and processing \\
\textbf{Data} & Data storage and management \\
\end{longtable}
}

\begin{itemize}
\tightlist
\item
  \textbf{Advantage}: Better scalability and maintainability
\item
  \textbf{Example}: Web browser \rightarrow Web server \rightarrow Database
\end{itemize}

\end{solutionbox}
\begin{mnemonicbox}
``PAD = Presentation Application Data''

\end{mnemonicbox}
\subsection*{Question 2(c) [7 marks]}\label{q2c}

\textbf{Describe JDBC API with interfaces and classes.}

\begin{solutionbox}

\textbf{JDBC API Components:}

{\def\LTcaptype{none} % do not increment counter
\begin{longtable}[]{@{}lll@{}}
\toprule\noalign{}
Type & Component & Purpose \\
\midrule\noalign{}
\endhead
\bottomrule\noalign{}
\endlastfoot
\textbf{Interface} & Connection & Database connection \\
\textbf{Interface} & Statement & SQL execution \\
\textbf{Interface} & ResultSet & Query results \\
\textbf{Interface} & PreparedStatement & Precompiled SQL \\
\textbf{Class} & DriverManager & Driver management \\
\textbf{Class} & SQLException & Error handling \\
\end{longtable}
}

\textbf{JDBC Architecture:}

\begin{center}
\textbf{Mermaid Diagram (Code)}
\begin{verbatim}
{Shaded}
{Highlighting}[]
graph LR
    A[Java Application] {-{-}{} B[JDBC API]}
    B {-{-}{} C[JDBC Driver Manager]}
    C {-{-}{} D[JDBC Driver]}
    D {-{-}{} E[Database]}
{Highlighting}
{Shaded}
\end{verbatim}
\end{center}

\begin{itemize}
\tightlist
\item
  \textbf{Core Interfaces}: Connection, Statement, ResultSet,
  PreparedStatement
\item
  \textbf{Key Classes}: DriverManager for connection management
\item
  \textbf{Exception Handling}: SQLException for database errors
\end{itemize}

\end{solutionbox}
\begin{mnemonicbox}
``CSRP Classes = Connection Statement ResultSet
PreparedStatement''

\end{mnemonicbox}
\subsection*{Question 2(a OR) [3
marks]}\label{question-2a-or-3-marks}

\textbf{List out advantages and disadvantages of JDBC.}

\begin{solutionbox}

\textbf{JDBC Advantages vs Disadvantages:}

{\def\LTcaptype{none} % do not increment counter
\begin{longtable}[]{@{}ll@{}}
\toprule\noalign{}
Advantages & Disadvantages \\
\midrule\noalign{}
\endhead
\bottomrule\noalign{}
\endlastfoot
\textbf{Platform Independent} & \textbf{Performance Overhead} \\
\textbf{Standard API} & \textbf{Complex Configuration} \\
\textbf{Multiple Database Support} & \textbf{Limited ORM Features} \\
\end{longtable}
}

\begin{itemize}
\tightlist
\item
  \textbf{Benefits}: Write once, run anywhere with any database
\item
  \textbf{Drawbacks}: Requires manual SQL and connection management
\end{itemize}

\end{solutionbox}
\begin{mnemonicbox}
``PSM vs PCL = Platform Standard Multiple vs
Performance Complex Limited''

\end{mnemonicbox}
\subsection*{Question 2(b OR) [4
marks]}\label{question-2b-or-4-marks}

\textbf{Explain 2-tier architecture with diagram.}

\begin{solutionbox}

2-tier architecture directly connects client to database server.

\begin{center}
\textbf{Mermaid Diagram (Code)}
\begin{verbatim}
{Shaded}
{Highlighting}[]
graph LR
    A[Client Tier{br/{}Application/UI] {}{-}{-}{} B[Data Tier{}br/{}Database Server]}
{Highlighting}
{Shaded}
\end{verbatim}
\end{center}

{\def\LTcaptype{none} % do not increment counter
\begin{longtable}[]{@{}ll@{}}
\toprule\noalign{}
Tier & Responsibility \\
\midrule\noalign{}
\endhead
\bottomrule\noalign{}
\endlastfoot
\textbf{Client} & User interface and business logic \\
\textbf{Server} & Data storage and management \\
\end{longtable}
}

\begin{itemize}
\tightlist
\item
  \textbf{Advantage}: Simple architecture, direct communication
\item
  \textbf{Disadvantage}: Limited scalability, tight coupling
\item
  \textbf{Example}: Desktop application connecting directly to database
\end{itemize}

\end{solutionbox}
\begin{mnemonicbox}
``CD = Client Data (direct connection)''

\end{mnemonicbox}
\subsection*{Question 2(c OR) [7
marks]}\label{question-2c-or-7-marks}

\textbf{List out JDBC driver types and Explain TYPE-4.}

\begin{solutionbox}

\textbf{JDBC Driver Types:}

{\def\LTcaptype{none} % do not increment counter
\begin{longtable}[]{@{}lll@{}}
\toprule\noalign{}
Type & Name & Description \\
\midrule\noalign{}
\endhead
\bottomrule\noalign{}
\endlastfoot
\textbf{Type-1} & JDBC-ODBC Bridge & Uses ODBC driver \\
\textbf{Type-2} & Native-API Driver & Part Java, part native \\
\textbf{Type-3} & Network Protocol Driver & Pure Java, middleware \\
\textbf{Type-4} & Native Protocol Driver & Pure Java, direct \\
\end{longtable}
}

\textbf{TYPE-4 Driver Explanation:}

\begin{itemize}
\tightlist
\item
  \textbf{Pure Java}: Completely written in Java
\item
  \textbf{Direct Communication}: Directly communicates with database
\item
  \textbf{Platform Independent}: No native libraries required
\item
  \textbf{Best Performance}: Fastest among all types
\item
  \textbf{Examples}: MySQL Connector/J, PostgreSQL JDBC
\end{itemize}

\begin{center}
\textbf{Mermaid Diagram (Code)}
\begin{verbatim}
{Shaded}
{Highlighting}[]
graph LR
    A[Java Application] {-{-}{} B[Type{-}4 JDBC Driver{}br/{}Pure Java] {-}{-}{} C[Database Server]}
{Highlighting}
{Shaded}
\end{verbatim}
\end{center}

\end{solutionbox}
\begin{mnemonicbox}
``ONNN Drivers = ODBC Native Network Native-pure''

\end{mnemonicbox}
\subsection*{Question 3(a) [3 marks]}\label{q3a}

\textbf{Explain Application of servlet.}

\begin{solutionbox}

\textbf{Servlet Applications:}

{\def\LTcaptype{none} % do not increment counter
\begin{longtable}[]{@{}ll@{}}
\toprule\noalign{}
Application & Usage \\
\midrule\noalign{}
\endhead
\bottomrule\noalign{}
\endlastfoot
\textbf{Web Forms} & Process HTML form data \\
\textbf{Database Operations} & Connect and manipulate database \\
\textbf{Session Management} & Track user sessions \\
\textbf{File Upload} & Handle file uploads \\
\end{longtable}
}

\begin{itemize}
\tightlist
\item
  \textbf{Primary Use}: Server-side Java programs for web applications
\item
  \textbf{Common Tasks}: Request processing, response generation
\end{itemize}

\end{solutionbox}
\begin{mnemonicbox}
``WDSF = Web Database Session File''

\end{mnemonicbox}
\subsection*{Question 3(b) [4 marks]}\label{q3b}

\textbf{Explain difference between Applet and Servlet.}

\begin{solutionbox}

{\def\LTcaptype{none} % do not increment counter
\begin{longtable}[]{@{}lll@{}}
\toprule\noalign{}
Feature & Applet & Servlet \\
\midrule\noalign{}
\endhead
\bottomrule\noalign{}
\endlastfoot
\textbf{Execution} & Client-side (browser) & Server-side (web server) \\
\textbf{Purpose} & User interface & Request processing \\
\textbf{Security} & Restricted (sandbox) & Full server access \\
\textbf{Performance} & Limited by client & Server resources \\
\end{longtable}
}

\begin{itemize}
\tightlist
\item
  \textbf{Applet}: Runs in web browser, limited capabilities
\item
  \textbf{Servlet}: Runs on web server, full Java capabilities
\end{itemize}

\end{solutionbox}
\begin{mnemonicbox}
``Client vs Server = Applet vs Servlet''

\end{mnemonicbox}
\subsection*{Question 3(c) [7 marks]}\label{q3c}

\textbf{Explain life cycle of a servlet in detail.}

\begin{solutionbox}

\textbf{Servlet Life Cycle:}

\begin{center}
\textbf{Mermaid Diagram (Code)}
\begin{verbatim}
{Shaded}
{Highlighting}[]
graph LR
    A[Servlet Class Loaded] {-{-}{} B[init called]}
    B {-{-}{} C[service handles requests]}
    C {-{-}{} C}
    C {-{-}{} D[destroy called]}
    D {-{-}{} E[Servlet Unloaded]}
{Highlighting}
{Shaded}
\end{verbatim}
\end{center}

{\def\LTcaptype{none} % do not increment counter
\begin{longtable}[]{@{}lll@{}}
\toprule\noalign{}
Phase & Method & Description \\
\midrule\noalign{}
\endhead
\bottomrule\noalign{}
\endlastfoot
\textbf{Loading} & Class loading & Web container loads servlet class \\
\textbf{Initialization} & \texttt{init()} & Called once, setup
resources \\
\textbf{Service} & \texttt{service()} & Handles each request
(doGet/doPost) \\
\textbf{Destruction} & \texttt{destroy()} & Cleanup before unloading \\
\end{longtable}
}

\begin{itemize}
\tightlist
\item
  \textbf{Thread Safety}: Multiple requests handled concurrently
\item
  \textbf{Single Instance}: One servlet instance handles all requests
\item
  \textbf{Container Managed}: Web container manages lifecycle
\end{itemize}

\end{solutionbox}
\begin{mnemonicbox}
``LISD = Load Init Service Destroy''

\end{mnemonicbox}
\subsection*{Question 3(a OR) [3
marks]}\label{question-3a-or-3-marks}

\textbf{Explain web.xml file in servlet.}

\begin{solutionbox}

\textbf{web.xml Purpose:}

{\def\LTcaptype{none} % do not increment counter
\begin{longtable}[]{@{}ll@{}}
\toprule\noalign{}
Element & Description \\
\midrule\noalign{}
\endhead
\bottomrule\noalign{}
\endlastfoot
\textbf{Deployment Descriptor} & Configuration file for web
application \\
\textbf{Servlet Mapping} & Maps URL patterns to servlets \\
\textbf{Initialization} & Servlet parameters and load order \\
\end{longtable}
}

\begin{itemize}
\tightlist
\item
  \textbf{Location}: WEB-INF directory
\item
  \textbf{Format}: XML configuration file
\end{itemize}

\end{solutionbox}
\begin{mnemonicbox}
``DMI = Deployment Mapping Initialization''

\end{mnemonicbox}
\subsection*{Question 3(b OR) [4
marks]}\label{question-3b-or-4-marks}

\textbf{List out and Explain feature of servlet.}

\begin{solutionbox}

\textbf{Servlet Features:}

{\def\LTcaptype{none} % do not increment counter
\begin{longtable}[]{@{}ll@{}}
\toprule\noalign{}
Feature & Description \\
\midrule\noalign{}
\endhead
\bottomrule\noalign{}
\endlastfoot
\textbf{Platform Independent} & Write once, run anywhere \\
\textbf{Server-side} & Executes on web server \\
\textbf{Protocol Independent} & Supports HTTP, FTP, etc. \\
\textbf{Persistent} & Stays in memory between requests \\
\textbf{Secure} & Built-in security features \\
\end{longtable}
}

\begin{itemize}
\tightlist
\item
  \textbf{Performance}: Better than CGI scripts
\item
  \textbf{Scalability}: Handles multiple requests efficiently
\end{itemize}

\end{solutionbox}
\begin{mnemonicbox}
``PSPPS = Platform Server Protocol Persistent
Secure''

\end{mnemonicbox}
\subsection*{Question 3(c OR) [7
marks]}\label{question-3c-or-7-marks}

\textbf{Explain session tracking in servlet.}

\begin{solutionbox}

\textbf{Session Tracking Methods:}

{\def\LTcaptype{none} % do not increment counter
\begin{longtable}[]{@{}ll@{}}
\toprule\noalign{}
Method & Description \\
\midrule\noalign{}
\endhead
\bottomrule\noalign{}
\endlastfoot
\textbf{Cookies} & Small data stored in browser \\
\textbf{URL Rewriting} & Session ID in URL \\
\textbf{Hidden Form Fields} & Session data in forms \\
\textbf{HttpSession} & Server-side session object \\
\end{longtable}
}

\textbf{HttpSession Implementation:}

\begin{verbatim}
HttpSession session = request.getSession();
session.setAttribute("user", username);
String user = (String) session.getAttribute("user");
\end{verbatim}

\begin{center}
\textbf{Mermaid Diagram (Code)}
\begin{verbatim}
{Shaded}
{Highlighting}[]
graph LR
    A[Client Request] {-{-}{} B[Server checks Session ID]}
    B {-{-}{} C[Session exists?]}
    C {-{-}{}|Yes| D[Use existing session]}
    C {-{-}{}|No| E[Create new session]}
    D {-{-}{} F[Process Request]}
    E {-{-}{} F}
{Highlighting}
{Shaded}
\end{verbatim}
\end{center}

\begin{itemize}
\tightlist
\item
  \textbf{Purpose}: Maintain state across HTTP requests
\item
  \textbf{HttpSession}: Most commonly used method
\end{itemize}

\end{solutionbox}
\begin{mnemonicbox}
``CUHH = Cookies URL Hidden HttpSession''

\end{mnemonicbox}
\subsection*{Question 4(a) [3 marks]}\label{q4a}

\textbf{Explain architecture of JSP with diagram.}

\begin{solutionbox}

\textbf{JSP Architecture:}

\begin{center}
\textbf{Mermaid Diagram (Code)}
\begin{verbatim}
{Shaded}
{Highlighting}[]
graph LR
    A[JSP Page] {-{-}{} B[JSP Engine/Container]}
    B {-{-}{} C[Servlet Code Generated]}
    C {-{-}{} D[Compiled to Bytecode]}
    D {-{-}{} E[Servlet Executed]}
    E {-{-}{} F[HTML Response]}
{Highlighting}
{Shaded}
\end{verbatim}
\end{center}

{\def\LTcaptype{none} % do not increment counter
\begin{longtable}[]{@{}ll@{}}
\toprule\noalign{}
Component & Role \\
\midrule\noalign{}
\endhead
\bottomrule\noalign{}
\endlastfoot
\textbf{JSP Engine} & Translates JSP to servlet \\
\textbf{Web Container} & Manages JSP lifecycle \\
\textbf{Generated Servlet} & Actual execution unit \\
\end{longtable}
}

\end{solutionbox}
\begin{mnemonicbox}
``JSP = Java Server Pages (Page to Servlet)''

\end{mnemonicbox}
\subsection*{Question 4(b) [4 marks]}\label{q4b}

\textbf{Explain JSP scripting elements with example.}

\begin{solutionbox}

\textbf{JSP Scripting Elements:}

{\def\LTcaptype{none} % do not increment counter
\begin{longtable}[]{@{}lll@{}}
\toprule\noalign{}
Element & Syntax & Purpose \\
\midrule\noalign{}
\endhead
\bottomrule\noalign{}
\endlastfoot
\textbf{Scriptlet} & \texttt{\textless{}\%\ code\ \%\textgreater{}} &
Java code block \\
\textbf{Expression} &
\texttt{\textless{}\%=\ expression\ \%\textgreater{}} & Output value \\
\textbf{Declaration} &
\texttt{\textless{}\%!\ declaration\ \%\textgreater{}} &
Variables/methods \\
\end{longtable}
}

\textbf{Examples:}

\begin{verbatim}
{\%!} int count = 0; \%{}               {!{-}{-} Declaration {-}{-}}
{\%} count++; \%{}                      {!{-}{-} Scriptlet {-}{-}}
{\%=} "Count: " + count \%{}            {!{-}{-} Expression {-}{-}}
\end{verbatim}

\end{solutionbox}
\begin{mnemonicbox}
``SED = Scriptlet Expression Declaration''

\end{mnemonicbox}
\subsection*{Question 4(c) [7 marks]}\label{q4c}

\textbf{Explain JSP life cycle.}

\begin{solutionbox}

\textbf{JSP Life Cycle Phases:}

\begin{center}
\textbf{Mermaid Diagram (Code)}
\begin{verbatim}
{Shaded}
{Highlighting}[]
graph LR
    A[JSP Page Created] {-{-}{} B[Translation to Servlet]}
    B {-{-}{} C[Servlet Compilation]}
    C {-{-}{} D[Class Loading]}
    D {-{-}{} E[Instantiation]}
    E {-{-}{} F[jspInit called]}
    F {-{-}{} G[\_jspService handles requests]}
    G {-{-}{} G}
    G {-{-}{} H[jspDestroy called]}
{Highlighting}
{Shaded}
\end{verbatim}
\end{center}

{\def\LTcaptype{none} % do not increment counter
\begin{longtable}[]{@{}ll@{}}
\toprule\noalign{}
Phase & Description \\
\midrule\noalign{}
\endhead
\bottomrule\noalign{}
\endlastfoot
\textbf{Translation} & JSP converted to servlet source \\
\textbf{Compilation} & Servlet source compiled to bytecode \\
\textbf{Loading} & Servlet class loaded by JVM \\
\textbf{Instantiation} & Servlet object created \\
\textbf{Initialization} & \texttt{jspInit()} method called \\
\textbf{Request Processing} & \texttt{\_jspService()} handles
requests \\
\textbf{Destruction} & \texttt{jspDestroy()} cleanup method \\
\end{longtable}
}

\begin{itemize}
\tightlist
\item
  \textbf{Container Managed}: Web container handles entire lifecycle
\item
  \textbf{Automatic}: Translation and compilation happen automatically
\end{itemize}

\end{solutionbox}
\begin{mnemonicbox}
``TCLIIRD = Translation Compilation Loading
Instantiation Init Request Destroy''

\end{mnemonicbox}
\subsection*{Question 4(a OR) [3
marks]}\label{question-4a-or-3-marks}

\textbf{Explain difference between JSP and Servlet.}

\begin{solutionbox}

{\def\LTcaptype{none} % do not increment counter
\begin{longtable}[]{@{}lll@{}}
\toprule\noalign{}
Feature & JSP & Servlet \\
\midrule\noalign{}
\endhead
\bottomrule\noalign{}
\endlastfoot
\textbf{Code Style} & HTML with Java & Pure Java code \\
\textbf{Development} & Easier for UI & Better for logic \\
\textbf{Compilation} & Automatic & Manual \\
\textbf{Modification} & No recompilation needed & Requires
recompilation \\
\end{longtable}
}

\end{solutionbox}
\begin{mnemonicbox}
``HTML vs Java = JSP vs Servlet''

\end{mnemonicbox}
\subsection*{Question 4(b OR) [4
marks]}\label{question-4b-or-4-marks}

\textbf{List out and Explain advantage of JSP.}

\begin{solutionbox}

\textbf{JSP Advantages:}

{\def\LTcaptype{none} % do not increment counter
\begin{longtable}[]{@{}ll@{}}
\toprule\noalign{}
Advantage & Description \\
\midrule\noalign{}
\endhead
\bottomrule\noalign{}
\endlastfoot
\textbf{Easy Development} & HTML-like syntax with Java \\
\textbf{Automatic Compilation} & No manual compilation needed \\
\textbf{Platform Independent} & Runs on any Java-enabled server \\
\textbf{Separation of Concerns} & Design separated from logic \\
\textbf{Reusable Components} & Tag libraries and beans \\
\end{longtable}
}

\begin{itemize}
\tightlist
\item
  \textbf{Developer Friendly}: Web designers can work with JSP easily
\item
  \textbf{Maintenance}: Easier to modify than servlets
\end{itemize}

\end{solutionbox}
\begin{mnemonicbox}
``EAPSR = Easy Automatic Platform Separation
Reusable''

\end{mnemonicbox}
\subsection*{Question 4(c OR) [7
marks]}\label{question-4c-or-7-marks}

\textbf{What is cookie? Explain how to Read and delete cookie using JSP
page.}

\begin{solutionbox}

\textbf{Cookie Overview:} Cookie is a small piece of data stored on
client's browser to maintain state.

\textbf{Cookie Operations:}

{\def\LTcaptype{none} % do not increment counter
\begin{longtable}[]{@{}ll@{}}
\toprule\noalign{}
Operation & JSP Code \\
\midrule\noalign{}
\endhead
\bottomrule\noalign{}
\endlastfoot
\textbf{Create} &
\texttt{Cookie\ cookie\ =\ new\ Cookie("name",\ "value");} \\
\textbf{Add} & \texttt{response.addCookie(cookie);} \\
\textbf{Read} &
\texttt{Cookie[]\ cookies\ =\ request.getCookies();} \\
\textbf{Delete} & \texttt{cookie.setMaxAge(0);} \\
\end{longtable}
}

\textbf{Reading Cookie Example:}

\begin{verbatim}
{\%}
Cookie[] cookies = request.getCookies();
if (cookies != null) \{
    for (Cookie cookie : cookies) \{
        if ("username".equals(cookie.getName())) \{
            out.println("User: " + cookie.getValue());
        \}
    \}
\}
\%{}
\end{verbatim}

\textbf{Deleting Cookie Example:}

\begin{verbatim}
{\%}
Cookie cookie = new Cookie("username", "");
cookie.setMaxAge(0);
response.addCookie(cookie);
\%{}
\end{verbatim}

\end{solutionbox}
\begin{mnemonicbox}
``CARD = Create Add Read Delete''

\end{mnemonicbox}
\subsection*{Question 5(a) [3 marks]}\label{q5a}

\textbf{Explain importance of MVC architecture.}

\begin{solutionbox}

\textbf{MVC Importance:}

{\def\LTcaptype{none} % do not increment counter
\begin{longtable}[]{@{}ll@{}}
\toprule\noalign{}
Benefit & Description \\
\midrule\noalign{}
\endhead
\bottomrule\noalign{}
\endlastfoot
\textbf{Separation of Concerns} & Logic, presentation, data separated \\
\textbf{Maintainability} & Easy to modify individual components \\
\textbf{Testability} & Components can be tested independently \\
\end{longtable}
}

\begin{itemize}
\tightlist
\item
  \textbf{Code Organization}: Better structure and organization
\item
  \textbf{Team Development}: Multiple developers can work simultaneously
\end{itemize}

\end{solutionbox}
\begin{mnemonicbox}
``SMT = Separation Maintainability Testability''

\end{mnemonicbox}
\subsection*{Question 5(b) [4 marks]}\label{q5b}

\textbf{Explain Aspect oriented programming and dependency injection in
brief.}

\begin{solutionbox}

\textbf{Aspect Oriented Programming (AOP):}

{\def\LTcaptype{none} % do not increment counter
\begin{longtable}[]{@{}ll@{}}
\toprule\noalign{}
Concept & Description \\
\midrule\noalign{}
\endhead
\bottomrule\noalign{}
\endlastfoot
\textbf{Cross-cutting Concerns} & Logging, security, transactions \\
\textbf{Aspects} & Modular units of cross-cutting functionality \\
\textbf{Join Points} & Points where aspects are applied \\
\end{longtable}
}

\textbf{Dependency Injection (DI):}

{\def\LTcaptype{none} % do not increment counter
\begin{longtable}[]{@{}ll@{}}
\toprule\noalign{}
Concept & Description \\
\midrule\noalign{}
\endhead
\bottomrule\noalign{}
\endlastfoot
\textbf{Inversion of Control} & Dependencies provided externally \\
\textbf{Loose Coupling} & Objects don't create dependencies \\
\textbf{Configuration} & Dependencies configured externally \\
\end{longtable}
}

\end{solutionbox}
\begin{mnemonicbox}
``AOP = Aspects Over Points, DI = Dependencies
Injected''

\end{mnemonicbox}
\subsection*{Question 5(c) [7 marks]}\label{q5c}

\textbf{Explain MVC architecture.}

\begin{solutionbox}

\textbf{MVC Components:}

\begin{center}
\textbf{Mermaid Diagram (Code)}
\begin{verbatim}
{Shaded}
{Highlighting}[]
graph LR
    A[View{br/{}Presentation Layer] }
    B[Controller{br/{}Control Layer] }
    C[Model{br/{}Business Logic]}
    
    A {-{-}{} B}
    B {-{-}{} C}
    C {-{-}{} B}
    B {-{-}{} A}
{Highlighting}
{Shaded}
\end{verbatim}
\end{center}

{\def\LTcaptype{none} % do not increment counter
\begin{longtable}[]{@{}ll@{}}
\toprule\noalign{}
Component & Responsibility \\
\midrule\noalign{}
\endhead
\bottomrule\noalign{}
\endlastfoot
\textbf{Model} & Business logic and data management \\
\textbf{View} & User interface and presentation \\
\textbf{Controller} & Request handling and flow control \\
\end{longtable}
}

\textbf{MVC Flow:}

\begin{enumerate}
\tightlist
\item
  \textbf{User Request} \rightarrow Controller receives request
\item
  \textbf{Controller} \rightarrow Processes request, calls Model
\item
  \textbf{Model} \rightarrow Performs business logic, returns data
\item
  \textbf{Controller} \rightarrow Selects appropriate View
\item
  \textbf{View} \rightarrow Renders response to user
\end{enumerate}

\textbf{Advantages:}

\begin{itemize}
\tightlist
\item
  \textbf{Maintainability}: Clear separation of responsibilities
\item
  \textbf{Reusability}: Components can be reused
\item
  \textbf{Testability}: Each layer can be tested independently
\end{itemize}

\end{solutionbox}
\begin{mnemonicbox}
``MVC = Model View Controller (Business UI Control)''

\end{mnemonicbox}
\subsection*{Question 5(a OR) [3
marks]}\label{question-5a-or-3-marks}

\textbf{Explain advantages of MVC architecture.}

\begin{solutionbox}

\textbf{MVC Advantages:}

{\def\LTcaptype{none} % do not increment counter
\begin{longtable}[]{@{}
  >{\raggedright\arraybackslash}p{(\linewidth - 2\tabcolsep) * \real{0.4583}}
  >{\raggedright\arraybackslash}p{(\linewidth - 2\tabcolsep) * \real{0.5417}}@{}}
\toprule\noalign{}
\begin{minipage}[b]{\linewidth}\raggedright
Advantage
\end{minipage} & \begin{minipage}[b]{\linewidth}\raggedright
Description
\end{minipage} \\
\midrule\noalign{}
\endhead
\bottomrule\noalign{}
\endlastfoot
\textbf{Code Reusability} & Components can be reused across
applications \\
\textbf{Parallel Development} & Multiple developers work on different
layers \\
\textbf{Easy Testing} & Each component tested independently \\
\textbf{Maintenance} & Changes in one layer don't affect others \\
\end{longtable}
}

\end{solutionbox}
\begin{mnemonicbox}
``CPEM = Code Parallel Easy Maintenance''

\end{mnemonicbox}
\subsection*{Question 5(b OR) [4
marks]}\label{question-5b-or-4-marks}

\textbf{Explain difference between spring and spring boot.}

\begin{solutionbox}

{\def\LTcaptype{none} % do not increment counter
\begin{longtable}[]{@{}
  >{\raggedright\arraybackslash}p{(\linewidth - 4\tabcolsep) * \real{0.3000}}
  >{\raggedright\arraybackslash}p{(\linewidth - 4\tabcolsep) * \real{0.2667}}
  >{\raggedright\arraybackslash}p{(\linewidth - 4\tabcolsep) * \real{0.4333}}@{}}
\toprule\noalign{}
\begin{minipage}[b]{\linewidth}\raggedright
Feature
\end{minipage} & \begin{minipage}[b]{\linewidth}\raggedright
Spring
\end{minipage} & \begin{minipage}[b]{\linewidth}\raggedright
Spring Boot
\end{minipage} \\
\midrule\noalign{}
\endhead
\bottomrule\noalign{}
\endlastfoot
\textbf{Configuration} & Manual XML/Java config & Auto-configuration \\
\textbf{Setup Time} & More setup required & Minimal setup \\
\textbf{Embedded Server} & External server needed & Built-in server \\
\textbf{Dependencies} & Manual dependency management & Starter
dependencies \\
\end{longtable}
}

\begin{itemize}
\tightlist
\item
  \textbf{Spring}: Comprehensive framework requiring configuration
\item
  \textbf{Spring Boot}: Convention over configuration approach
\end{itemize}

\end{solutionbox}
\begin{mnemonicbox}
``Manual vs Auto = Spring vs SpringBoot''

\end{mnemonicbox}
\subsection*{Question 5(c OR) [7
marks]}\label{question-5c-or-7-marks}

\textbf{Explain architecture of Spring framework.}

\begin{solutionbox}

\textbf{Spring Framework Architecture:}

\begin{center}
\textbf{Mermaid Diagram (Code)}
\begin{verbatim}
{Shaded}
{Highlighting}[]
graph TD
    A[Core Container{br/{}IoC \& DI] }
    B[Data Access{br/{}JDBC, ORM] }
    C[Web Layer{br/{}MVC, WebFlux]}
    D[AOP{br/{}Aspect Oriented]}
    E[Test{br/{}Testing Support]}
    
    A {-{-}{} B}
    A {-{-}{} C}
    A {-{-}{} D}
    A {-{-}{} E}
{Highlighting}
{Shaded}
\end{verbatim}
\end{center}

\textbf{Spring Modules:}

{\def\LTcaptype{none} % do not increment counter
\begin{longtable}[]{@{}ll@{}}
\toprule\noalign{}
Module & Purpose \\
\midrule\noalign{}
\endhead
\bottomrule\noalign{}
\endlastfoot
\textbf{Core Container} & IoC container, dependency injection \\
\textbf{Data Access} & JDBC, ORM, transaction management \\
\textbf{Web} & Web MVC, REST services \\
\textbf{AOP} & Aspect-oriented programming \\
\textbf{Security} & Authentication and authorization \\
\textbf{Test} & Testing support and mock objects \\
\end{longtable}
}

\textbf{Key Features:}

\begin{itemize}
\tightlist
\item
  \textbf{IoC Container}: Manages object creation and dependencies
\item
  \textbf{AOP Support}: Cross-cutting concerns handling
\item
  \textbf{Transaction Management}: Declarative transaction support
\item
  \textbf{MVC Framework}: Web application development
\end{itemize}

\end{solutionbox}
\begin{mnemonicbox}
``CDWAST = Core Data Web AOP Security Test''

\end{mnemonicbox}

\end{document}
