\documentclass{article}

% content/resources/templates/preamble.tex
\usepackage[margin=0.6in]{geometry}
\author{Milav Dabgar}
\usepackage{amsmath,amssymb,amsthm}
\usepackage{booktabs}
\usepackage{multirow}
\usepackage{xcolor}
\usepackage{tcolorbox}
\tcbuselibrary{breakable,skins}
\usepackage[colorlinks=true,linkcolor=blue]{hyperref}
\usepackage{titlesec}
\usepackage{enumitem}
\usepackage{tikz}
\usepackage{pgfplots}
\usepackage{circuitikz}
\usepackage[version=4]{mhchem}
\usepackage{longtable}
\usepackage{array}
\usepackage{float}
\usepackage{caption}
\usepackage{listings}

\lstset{
  basicstyle=\small\ttfamily,
  breaklines=true,
  breakatwhitespace=false,
  postbreak=\mbox{\textcolor{red}{$\hookrightarrow$}\space},
  float=false,
  numbers=left,
  numberstyle=\tiny\color{gray},
  numbersep=10pt,
  xleftmargin=2em,
  keywordstyle=\color{blue},
  commentstyle=\color{green!60!black},
  stringstyle=\color{purple},
  backgroundcolor=\color{gray!5},
  showstringspaces=false,
  tabsize=2,
  captionpos=b,
  keepspaces=true,
  columns=flexible
}

\pgfplotsset{compat=1.18}
\usetikzlibrary{shapes,arrows,positioning,calc,patterns,decorations.pathmorphing,decorations.markings,arrows.meta}

% Color scheme
\definecolor{headcolor}{RGB}{0,102,204}
\definecolor{keycolor}{RGB}{220,20,60}
\definecolor{solutioncolor}{RGB}{34,139,34}
\definecolor{mnemoniccolor}{RGB}{148,0,211}
\definecolor{codecolor}{RGB}{0,0,100}

% Spacing
\setlength{\parskip}{3pt}
\setlist[itemize]{nosep}
\setlist[enumerate]{nosep}

% Title formatting
\titleformat{\section}{\Large\bfseries\color{headcolor}}{\thesection}{1em}{}
\titleformat{\subsection}{\large\bfseries\color{headcolor}}{\thesubsection}{1em}{}

% Pandoc tightlist compatibility
\providecommand{\tightlist}{%
  \setlength{\itemsep}{0pt}\setlength{\parskip}{0pt}}

% Pandoc longtable compatibility
\newcounter{none}
\def\thenone{}


% content/resources/templates/gujarati-boxes.tex
\usepackage{fontspec}
\usepackage{polyglossia}

% Set Gujarati as main language (document is primarily in Gujarati)
% Note: gloss-gujarati.ldf doesn't exist in polyglossia, but it will use hyphenation patterns
\setdefaultlanguage{gujarati}
\setotherlanguage{english}

% Configure Gujarati font properly
% Use Language=Default to prevent polyglossia from trying to add language-specific features
% that don't exist for Gujarati, which causes "empty feature" warnings
\newfontfamily\gujaratifont[Script=Gujarati,AutoFakeBold=2.5,AutoFakeSlant=0.3]{Noto Sans Gujarati}
\setmainfont[Script=Gujarati,AutoFakeBold=2.5,AutoFakeSlant=0.3]{Noto Sans Gujarati}
% Use Noto Sans Gujarati for monospace to support Gujarati in text
\setmonofont[Scale=0.9]{Noto Sans Gujarati}

% Configure English to use the same font
\newfontfamily\englishfont[Script=Gujarati,AutoFakeBold=2.5,AutoFakeSlant=0.3]{Noto Sans Gujarati}

% Translations for polyglossia
\gappto\captionsgujarati{
  \renewcommand{\tablename}{કોષ્ટક}
  \renewcommand{\figurename}{આકૃતિ}
}

% Helper for TikZ nodes to ensure Gujarati font
\newcommand{\gu}[1]{{\gujaratifont #1}}

% Custom environments
\newtcolorbox{solutionbox}{
    breakable,
    enhanced,
    colback=solutioncolor!5!white,
    colframe=solutioncolor!75!black,
    fonttitle=\bfseries,
    title=જવાબ
}

\newtcolorbox{solutionboxnobreak}{
 colback=solutioncolor!5!white,
 colframe=solutioncolor!75!black,
 fonttitle=\bfseries,
 title=જવાબ
}

\newtcolorbox{keyformula}{
 breakable,
 enhanced,
 colback=keycolor!5!white,
 colframe=keycolor!75!black,
 fonttitle=\bfseries,
 title=રાસાયણિક સમીકરણ/સૂત્ર
}

\newtcolorbox{mnemonicbox}{
 breakable,
 enhanced,
 colback=mnemoniccolor!5!white,
 colframe=mnemoniccolor!75!black,
 fonttitle=\bfseries,
 title=મેમરી ટ્રીક
}


% Custom commands for GTU solutions
% This file defines semantic commands for consistent formatting

% Question command with automatic formatting
\newcommand{\question}[2]{%
  \section*{Question #1}%
  \textbf{#2}%
}

% OR question variant
\newcommand{\questionor}[2]{%
  \section*{Question #1 OR}%
  \textbf{#2}%
}

% Proper table environment with caption
\newenvironment{answertable}[1]{%
  \begin{table}[htbp]
  \centering
  \caption{#1}
}{%
  \end{table}
}

% Proper figure environment for diagrams
\newenvironment{answerdiagram}[1]{%
  \begin{figure}[htbp]
  \centering
  \caption{#1}
}{%
  \end{figure}
}

% Semantic markup for key terms
\newcommand{\keyword}[1]{\textbf{#1}}
\newcommand{\code}[1]{\texttt{#1}}
\newcommand{\classname}[1]{\texttt{#1}}
\newcommand{\methodname}[1]{\texttt{#1}}

% Proper quotation marks
\newcommand{\mnemonic}[1]{``#1''}


\usetikzlibrary{shapes.geometric, arrows.meta}

\title{Python Programming (4311601)}
\date{Winter 2023}

\begin{document}
\maketitle

\questionmarks{પ્રશ્ન 1(અ)}{03}{ફ્લો ચાર્ટ શું છે? ફ્લો ચાર્ટમાં વપરાતા પ્રતીકોની યાદી બનાવો.}

\begin{solutionbox}
\keyword{ફ્લો ચાર્ટ} એ અલ્ગોરિધમની ગ્રાફિકલ રજૂઆત છે જે પ્રક્રિયાના પગલાંઓ અને નિર્ણય બિંદુઓ દર્શાવે છે.

\begin{answertable}{ફ્લો ચાર્ટ પ્રતીકો}
\begin{tabulary}{\linewidth}{|l|l|L|}
\hline
\textbf{પ્રતીક} & \textbf{નામ} & \textbf{ઉપયોગ} \\
\hline
અંડાકાર & ટર્મિનલ & પ્રારંભ/અંત \\
\hline
લંબચોરસ & પ્રોસેસ & પ્રક્રિયા/ગણતરી \\
\hline
હીરો & નિર્ણય & શરતી નિવેદનો \\
\hline
સમાંતર ચતુષ્કોણ & ઇનપુટ/આઉટપુટ & ડેટા લેવો/આપવો \\
\hline
વૃત્ત & કનેક્ટર & ભાગોને જોડવા \\
\hline
તીર & ફ્લો લાઇન & દિશા \\
\hline
\end{tabulary}
\end{answertable}

\textbf{મુખ્ય બિંદુઓ:}
\begin{itemize}
    \item \textbf{વિઝ્યુઅલ રજૂઆત}: પ્રોગ્રામ લોજિક ગ્રાફિકલી દર્શાવે
    \item \textbf{પગલાં દર પગલાં}: ક્રમિક ઓપરેશનનો ફ્લો
    \item \textbf{નિર્ણય લેવો}: હીરા શરતી શાખાઓ દર્શાવે
\end{itemize}

\begin{mnemonicbox}\mnemonic{ફ્લો ચાર્ટ્સ પ્રોગ્રામ સ્ટેપ્સ વિઝ્યુઅલી દર્શાવે}\end{mnemonicbox}
\end{solutionbox}

\questionmarks{પ્રશ્ન 1(બ)}{04}{for લૂપ માટે ટૂંકી નોંધ લખો.}

\begin{solutionbox}
\keyword{for લૂપ} Python માં સિક્વન્સ (list, tuple, string, range) પર iterate કરવા માટે વપરાય છે.

\begin{answertable}{For લૂપ ટેબલ}
\begin{tabulary}{\linewidth}{|l|l|l|}
\hline
\textbf{ઘટક} & \textbf{સિન્ટેક્સ} & \textbf{ઉદાહરણ} \\
\hline
મૂળભૂત & \code{for variable in sequence:} & \code{for i in range(5):} \\
\hline
રેન્જ & \code{range(start, stop, step)} & \code{range(1, 10, 2)} \\
\hline
યાદી & \code{for item in list:} & \code{for x in [1,2,3]:} \\
\hline
સ્ટ્રિંગ & \code{for char in string:} & \code{for c in "hello":} \\
\hline
\end{tabulary}
\end{answertable}

\textbf{સરળ કોડ ઉદાહરણ:}
\begin{lstlisting}[language=Python]
for i in range(3):
    print(i)
# આઉટપુટ: 0, 1, 2
\end{lstlisting}

\textbf{મુખ્ય લક્ષણો:}
\begin{itemize}
    \item \textbf{ઓટોમેટિક iteration}: મેન્યુઅલ કાઉન્ટરની જરૂર નથી
    \item \textbf{સિક્વન્સ ટ્રાવર્સલ}: કોઈપણ iterable ઓબ્જેક્ટ સાથે કામ કરે
    \item \textbf{રેન્જ ફંક્શન}: નંબર સિક્વન્સ સરળતાથી બનાવે
\end{itemize}

\begin{mnemonicbox}\mnemonic{For લૂપ્સ સિક્વન્સમાં iterate કરે}\end{mnemonicbox}
\end{solutionbox}

\questionmarks{પ્રશ્ન 1(ક)}{07}{ફિબોનાકી શ્રેણીને nમી ટર્મ સુધી દર્શાવવા માટે એક પ્રોગ્રામ લખો જ્યાં યુઝર દ્વારા n આપવામાં આવે છે.}

\begin{solutionbox}
\textbf{ફિબોનાકી શ્રેણી પ્રોગ્રામ:}
\begin{lstlisting}[language=Python]
# યુઝર પાસેથી ટર્મ્સની સંખ્યા લો
n = int(input("ટર્મ્સની સંખ્યા દાખલ કરો: "))

# પ્રથમ બે ટર્મ્સ initialize કરો
a, b = 0, 1

# પ્રથમ ટર્મ દર્શાવો
if n >= 1:
    print(a, end=" ")
    
# બીજી ટર્મ દર્શાવો
if n >= 2:
    print(b, end=" ")

# બાકીની ટર્મ્સ જનરેટ કરો
for i in range(2, n):
    c = a + b
    print(c, end=" ")
    a, b = b, c
\end{lstlisting}

\textbf{અલ્ગોરિધમ ફ્લો:}
\begin{center}
\begin{tikzpicture}[node distance=2cm, auto]
    \node[draw, ellipse, fill=red!10, align=center, minimum height=2em] (start) {શરૂ};
    \node[trapezium, trapezium left angle=70, trapezium right angle=110, draw, fill=orange!10, align=center, minimum height=2em, below=of start] (input) {n ઇનપુટ};
    \node[diamond, draw, fill=green!10, align=center, aspect=2, font=\small, below=of input] (cond1) {n >= 1?};
    \node[trapezium, trapezium left angle=70, trapezium right angle=110, draw, fill=orange!10, align=center, minimum height=2em, right=3cm of cond1] (print0) {0 પ્રિન્ટ કરો};
    \node[diamond, draw, fill=green!10, align=center, aspect=2, font=\small, below=of cond1] (cond2) {n >= 2?};
    \node[trapezium, trapezium left angle=70, trapezium right angle=110, draw, fill=orange!10, align=center, minimum height=2em, right=3cm of cond2] (print1) {1 પ્રિન્ટ કરો};
    \node[rectangle, draw, fill=blue!10, align=center, rounded corners, minimum height=2em, font=\small, below=of cond2] (loop) {લૂપ i=2 થી n-1};
    \node[rectangle, draw, fill=blue!10, align=center, rounded corners, minimum height=2em, font=\small, below=of loop] (calc) {c = a + b\\c પ્રિન્ટ કરો\\a = b, b = c};
    \node[draw, ellipse, fill=red!10, align=center, minimum height=2em, below=of calc] (stop) {અંત};

    \draw[draw, -latex] (start) -- (input);
    \draw[draw, -latex] (input) -- (cond1);
    \draw[draw, -latex] (cond1) -- node[above] {હા} (print0);
    \draw[draw, -latex] (print0) |- (cond2);
    \draw[draw, -latex] (cond1) -- node[left] {ના} (stop);
    
    \draw[draw, -latex] (cond2) -- node[above] {હા} (print1);
    \draw[draw, -latex] (print1) |- (loop);
    \draw[draw, -latex] (cond2) -- node[left] {ના} (stop.west);
    
    \draw[draw, -latex] (loop) -- (calc);
    \draw[draw, -latex] (calc) -- (stop);
\end{tikzpicture}
\end{center}

\textbf{મુખ્ય કોન્સેપ્ટ્સ:}
\begin{itemize}
    \item \textbf{સિક્વેન્શિયલ જનરેશન}: દરેક ટર્મ = પાછલી બે ટર્મનો સરવાળો
    \item \textbf{વેરિયેબલ સ્વેપિંગ}: a, b વેલ્યુઝ અસરકારક રીતે અપડેટ કરો
    \item \textbf{યુઝર ઇનપુટ}: ડાયનેમિક શ્રેણી લેન્થ
\end{itemize}

\begin{mnemonicbox}\mnemonic{ફિબોનાકી: પાછલા બે નંબરો ઉમેરો}\end{mnemonicbox}
\end{solutionbox}

\questionmarks{પ્રશ્ન 1(ક OR)}{07}{1 થી 100 સુધીના ODD નંબરો પ્રિન્ટ કરવા માટે ફ્લો ચાર્ટ દોરો.}

\begin{solutionbox}
\textbf{1 થી 100 ODD નંબરો માટે ફ્લોચાર્ટ:}

\begin{center}
\begin{tikzpicture}[node distance=2cm, auto]
    \node[draw, ellipse, fill=red!10, align=center, minimum height=2em] (start) {શરૂ};
    \node[rectangle, draw, fill=blue!10, align=center, rounded corners, minimum height=2em, font=\small, below=of start] (init) {i = 1};
    \node[diamond, draw, fill=green!10, align=center, aspect=2, font=\small, below=of init] (cond_loop) {i <= 100?};
    \node[diamond, draw, fill=green!10, align=center, aspect=2, font=\small, right=1.5cm of cond_loop] (cond_odd) {i \% 2 != 0?};
    \node[trapezium, trapezium left angle=70, trapezium right angle=110, draw, fill=orange!10, align=center, minimum height=2em, below=of cond_odd] (print) {i પ્રિન્ટ કરો};
    \node[rectangle, draw, fill=blue!10, align=center, rounded corners, minimum height=2em, font=\small, left=1.5cm of cond_loop] (inc) {i = i + 1};
    \node[draw, ellipse, fill=red!10, align=center, minimum height=2em, below=of cond_loop] (stop) {અંત};

    \draw[draw, -latex] (start) -- (init);
    \draw[draw, -latex] (init) -- (cond_loop);
    \draw[draw, -latex] (cond_loop) -- node[above] {હા} (cond_odd);
    \draw[draw, -latex] (cond_loop) -- node[left] {ના} (stop);
    
    \draw[draw, -latex] (cond_odd) -- node[right] {હા} (print);
    \draw[draw, -latex] (cond_odd.north) |- node[above] {ના} (inc.north);
    
    \draw[draw, -latex] (print.south) |- (inc.south);
    \draw[draw, -latex] (inc) -- (cond_loop);
\end{tikzpicture}
\end{center}

\textbf{અનુસંગિક Python કોડ:}
\begin{lstlisting}[language=Python]
for i in range(1, 101):
    if i % 2 != 0:
        print(i, end=" ")
\end{lstlisting}

\textbf{વૈકલ્પિક પદ્ધતિ:}
\begin{lstlisting}[language=Python]
for i in range(1, 101, 2):
    print(i, end=" ")
\end{lstlisting}

\textbf{મુખ્ય તત્વો:}
\begin{itemize}
    \item \textbf{લૂપ કંટ્રોલ}: i 1 થી 100 સુધી
    \item \textbf{વિષમ ચેક}: i \% 2 != 0 શરત
    \item \textbf{સ્ટેપ વધારો}: આગલા નંબર પર જાઓ
\end{itemize}

\begin{mnemonicbox}\mnemonic{વિષમ નંબરો: 2 થી ભાગ્યે 1 બાકી}\end{mnemonicbox}
\end{solutionbox}

\questionmarks{પ્રશ્ન 2(અ)}{03}{નંબર પેલિન્ડ્રોમ છે કે નહીં તે શોધવા માટે પ્રોગ્રામ લખો.}

\begin{solutionbox}
\textbf{પેલિન્ડ્રોમ ચેક પ્રોગ્રામ:}
\begin{lstlisting}[language=Python]
# નંબર ઇનપુટ
num = int(input("નંબર દાખલ કરો: "))
temp = num
reverse = 0

# નંબરને રિવર્સ કરો
while temp > 0:
    reverse = reverse * 10 + temp % 10
    temp = temp // 10

# પેલિન્ડ્રોમ ચેક કરો
if num == reverse:
    print(f"{num} પેલિન્ડ્રોમ છે")
else:
    print(f"{num} પેલિન્ડ્રોમ નથી")
\end{lstlisting}

\textbf{અલ્ગોરિધમ ટેબલ:}
\begin{answertable}{અલ્ગોરિધમ}
\begin{tabulary}{\linewidth}{|l|l|l|}
\hline
\textbf{પગલું} & \textbf{ઓપરેશન} & \textbf{ઉદાહરણ (121)} \\
\hline
1 & છેલ્લો અંક મેળવો & 121 \% 10 = 1 \\
\hline
2 & રિવર્સ બનાવો & 0*10 + 1 = 1 \\
\hline
3 & છેલ્લો અંક દૂર કરો & 121 // 10 = 12 \\
\hline
4 & 0 સુધી પુનરાવર્તન & પ્રક્રિયા ચાલુ રાખો \\
\hline
\end{tabulary}
\end{answertable}

\begin{mnemonicbox}\mnemonic{પેલિન્ડ્રોમ આગળ પાછળ સરખું વાંચાય}\end{mnemonicbox}
\end{solutionbox}

\questionmarks{પ્રશ્ન 2(બ)}{04}{Python પ્રોગ્રામિંગની વિશેષતાઓ સમજાવો.}

\begin{solutionbox}
\textbf{Python વિશેષતાઓનું ટેબલ:}
\begin{answertable}{Python વિશેષતાઓ}
\begin{tabulary}{\linewidth}{|l|l|l|}
\hline
\textbf{વિશેષતા} & \textbf{વર્ણન} & \textbf{ફાયદો} \\
\hline
સરળ સિન્ટેક્સ & સાદો, વાંચી શકાય તેવો કોડ & ઝડપી ડેવલપમેન્ટ \\
\hline
ઇન્ટરપ્રિટેડ & કમ્પાઇલેશનની જરૂર નથી & ઝડપી ટેસ્ટિંગ \\
\hline
ઓબ્જેક્ટ-ઓરિએન્ટેડ & ક્લાસ અને ઓબ્જેક્ટ સપોર્ટ & કોડ રિયુઝેબિલિટી \\
\hline
ઓપન સોર્સ & વાપરવા માટે ફ્રી & લાઇસન્સિંગ કોસ્ટ નથી \\
\hline
ક્રોસ-પ્લેટફોર્મ & મલ્ટિપલ OS પર ચાલે & વ્યાપક કમ્પેટિબિલિટી \\
\hline
મોટી લાઇબ્રેરીઓ & વ્યાપક બિલ્ટ-ઇન મોડ્યુલ્સ & સમૃદ્ધ કાર્યક્ષમતા \\
\hline
\end{tabulary}
\end{answertable}

\textbf{મુખ્ય ફાયદાઓ:}
\begin{itemize}
    \item \textbf{શિખાઉ-મિત્ર}: શીખવામાં અને સમજવામાં સરળ
    \item \textbf{વર્સેટાઇલ}: વેબ ડેવલપમેન્ટ, AI, ડેટા સાયન્સ
    \item \textbf{કોમ્યુનિટી સપોર્ટ}: મોટો ડેવલપર કોમ્યુનિટી
    \item \textbf{ડાયનેમિક ટાઇપિંગ}: વેરિયેબલ ટાઇપ ડિક્લેરેશનની જરૂર નથી
\end{itemize}

\begin{mnemonicbox}\mnemonic{Python: સરળ, શક્તિશાળી, લોકપ્રિય પ્રોગ્રામિંગ}\end{mnemonicbox}
\end{solutionbox}

\questionmarks{પ્રશ્ન 2(ક)}{07}{Python પ્રોગ્રામની બેસિક સ્ટ્રક્ચર સમજાવો.}

\begin{solutionbox}
\textbf{Python પ્રોગ્રામ સ્ટ્રક્ચર:}
\begin{lstlisting}[language=Python]
#!/usr/bin/env python3
# Shebang લાઇન (વૈકલ્પિક)

"""
ડોક્યુમેન્ટેશન સ્ટ્રિંગ (docstring)
પ્રોગ્રામનો હેતુ વર્ણવે છે
"""

# Import સ્ટેટમેન્ટ્સ
import math
from datetime import date

# ગ્લોબલ વેરિયેબલ્સ
PI = 3.14159
count = 0

# ફંક્શન ડેફિનિશન્સ
def calculate_area(radius):
    """વર્તુળનો ક્ષેત્રફળ કેલ્ક્યુલેટ કરે"""
    return PI * radius * radius

# ક્લાસ ડેફિનિશન્સ
class Calculator:
    def __init__(self):
        self.result = 0

# મેઇન પ્રોગ્રામ એક્ઝિક્યુશન
if __name__ == "__main__":
    # પ્રોગ્રામ લોજિક અહીં
    radius = 5
    area = calculate_area(radius)
    print(f"ક્ષેત્રફળ: {area}")
\end{lstlisting}

\textbf{સ્ટ્રક્ચર કમ્પોનન્ટ્સ ટેબલ:}
\begin{answertable}{સ્ટ્રક્ચર}
\begin{tabular}{|l|l|l|}
\hline
\textbf{ઘટક} & \textbf{હેતુ} & \textbf{ઉદાહરણ} \\
\hline
Shebang & સિસ્ટમ ઇન્ટરપ્રિટર & \code{\#!/usr/bin/env python3} \\
\hline
Docstring & પ્રોગ્રામ દસ્તાવેજીકરણ & \code{"""Program description"""} \\
\hline
Imports & બાહ્ય મોડ્યુલ્સ & \code{import math} \\
\hline
Variables & ગ્લોબલ ડેટા સ્ટોરેજ & \code{PI = 3.14159} \\
\hline
Functions & પુનઃઉપયોગી કોડ બ્લોક્સ & \code{def function\_name():} \\
\hline
\end{tabular}
\end{answertable}

\begin{mnemonicbox}\mnemonic{સ્ટ્રક્ચર: ઇમ્પોર્ટ, ડિફાઇન, એક્ઝિક્યુટ}\end{mnemonicbox}
\end{solutionbox}

\questionmarks{પ્રશ્ન 2(અ OR)}{03}{સ્ટ્રિંગને રિવર્સ કરવા માટે પ્રોગ્રામ લખો.}

\begin{solutionbox}
\textbf{સ્ટ્રિંગ રિવર્સલ પ્રોગ્રામ:}
\begin{lstlisting}[language=Python]
# પદ્ધતિ 1: સ્લાઇસિંગ વાપરીને
string = input("સ્ટ્રિંગ દાખલ કરો: ")
reversed_string = string[::-1]
print(f"રિવર્સ: {reversed_string}")

# પદ્ધતિ 2: લૂપ વાપરીને
string = input("સ્ટ્રિંગ દાખલ કરો: ")
reversed_string = ""
for char in string:
    reversed_string = char + reversed_string
print(f"રિવર્સ: {reversed_string}")
\end{lstlisting}

\textbf{રિવર્સલ પદ્ધતિઓનું ટેબલ:}
\begin{answertable}{રિવર્સલ પદ્ધતિઓ}
\begin{tabulary}{\linewidth}{|l|l|l|}
\hline
\textbf{પદ્ધતિ} & \textbf{સિન્ટેક્સ} & \textbf{ઉદાહરણ} \\
\hline
સ્લાઇસિંગ & \code{string[::-1]} & "hello" \rightarrow "olleh" \\
\hline
લૂપ & કેરેક્ટર દ્વારા કેરેક્ટર બનાવો & દરેક char આગળ ઉમેરો \\
\hline
બિલ્ટ-ઇન & \code{"".join(reversed(string))} & રિવર્સ સિક્વન્સ જોડો \\
\hline
\end{tabulary}
\end{answertable}

\begin{mnemonicbox}\mnemonic{રિવર્સ: છેલ્લો કેરેક્ટર પહેલો}\end{mnemonicbox}
\end{solutionbox}

\questionmarks{પ્રશ્ન 2(બ OR)}{04}{લોજિકલ ઓપરેટર્સને ઉદાહરણ સાથે સમજાવો.}

\begin{solutionbox}
\textbf{Python લોજિકલ ઓપરેટર્સ:}
\begin{answertable}{લોજિકલ ઓપરેટર્સ}
\begin{tabulary}{\linewidth}{|l|l|L|l|l|}
\hline
\textbf{ઓપરેટર} & \textbf{સિમ્બોલ} & \textbf{વર્ણન} & \textbf{ઉદાહરણ} & \textbf{પરિણામ} \\
\hline
AND & \code{and} & બંને શરતો સાચી & \code{True and False} & \code{False} \\
\hline
OR & \code{or} & ઓછામાં ઓછી એક શરત સાચી & \code{True or False} & \code{True} \\
\hline
NOT & \code{not} & શરતની વિરુદ્ધ & \code{not True} & \code{False} \\
\hline
\end{tabulary}
\end{answertable}

\textbf{ઉદાહરણ કોડ:}
\begin{lstlisting}[language=Python]
a = 10
b = 5

# AND ઓપરેટર
if a > 5 and b < 10:
    print("બંને શરતો સાચી")

# OR ઓપરેટર  
if a > 15 or b < 10:
    print("ઓછામાં ઓછી એક શરત સાચી")

# NOT ઓપરેટર
if not (a < 5):
    print("a 5 કરતાં નાનું નથી")
\end{lstlisting}

\textbf{ટ્રુથ ટેબલ:}
\begin{answertable}{ટ્રુથ ટેબલ}
\begin{tabulary}{\linewidth}{|c|c|c|c|c|}
\hline
\textbf{A} & \textbf{B} & \textbf{A and B} & \textbf{A or B} & \textbf{not A} \\
\hline
T & T & T & T & F \\
\hline
T & F & F & T & F \\
\hline
F & T & F & T & T \\
\hline
F & F & F & F & T \\
\hline
\end{tabulary}
\end{answertable}

\begin{mnemonicbox}\mnemonic{AND બધાની જરૂર, OR એકની જરૂર, NOT ઉલટાવે}\end{mnemonicbox}
\end{solutionbox}

\questionmarks{પ્રશ્ન 2(ક OR)}{07}{Python માં વિવિધ ડેટા પ્રકારો સમજાવો}

\begin{solutionbox}
\textbf{Python ડેટા ટાઇપ્સ વર્ગીકરણ:}

\begin{center}
\begin{tikzpicture}[grow=down, level 1/.style={sibling distance=3cm}, level 2/.style={sibling distance=1.5cm}]
    \node[rectangle, draw, fill=blue!10, align=center, rounded corners, minimum height=2em, font=\small, fill=purple!10] {Data Types}
    child { node[rectangle, draw, fill=blue!10, align=center, rounded corners, minimum height=2em, font=\small, fill=yellow!10] {Numeric}
        child { node[rectangle, draw, fill=blue!10, align=center, rounded corners, minimum height=2em, font=\small, fill=yellow!10] {int} }
        child { node[rectangle, draw, fill=blue!10, align=center, rounded corners, minimum height=2em, font=\small, fill=yellow!10] {float} }
        child { node[rectangle, draw, fill=blue!10, align=center, rounded corners, minimum height=2em, font=\small, fill=yellow!10] {complex} }
    }
    child { node[rectangle, draw, fill=blue!10, align=center, rounded corners, minimum height=2em, font=\small, fill=yellow!10] {Sequence}
        child { node[rectangle, draw, fill=blue!10, align=center, rounded corners, minimum height=2em, font=\small, fill=yellow!10] {str} }
        child { node[rectangle, draw, fill=blue!10, align=center, rounded corners, minimum height=2em, font=\small, fill=yellow!10] {list} }
        child { node[rectangle, draw, fill=blue!10, align=center, rounded corners, minimum height=2em, font=\small, fill=yellow!10] {tuple} }
    }
    child { node[rectangle, draw, fill=blue!10, align=center, rounded corners, minimum height=2em, font=\small, fill=yellow!10] {Boolean}
        child { node[rectangle, draw, fill=blue!10, align=center, rounded corners, minimum height=2em, font=\small, fill=yellow!10] {bool} }
    };
\end{tikzpicture}
\end{center}

\textbf{ડેટા ટાઇપ્સ ટેબલ:}
\begin{answertable}{ડેટા ટાઇપ્સ}
\begin{tabulary}{\linewidth}{|l|l|l|l|}
\hline
\textbf{ટાઇપ} & \textbf{ઉદાહરણ} & \textbf{વર્ણન} & \textbf{Mutable} \\
\hline
int & \code{42} & પૂર્ણ સંખ્યાઓ | અને ના \\
\hline
float & \code{3.14} & દશાંશ સંખ્યાઓ & ના \\
\hline
str & \code{"hello"} & ટેક્સ્ટ ડેટા & ના \\
\hline
list & \code{[1,2,3]} & ક્રમાંકિત સંગ્રહ & હા \\
\hline
tuple & \code{(1,2,3)} & ક્રમાંકિત અપરિવર્તનીય & ના \\
\hline
dict & \code{\{"a":1\}} & કી-વેલ્યુ જોડીઓ & હા \\
\hline
bool & \code{True/False} & બુલિયન વેલ્યુઝ & ના \\
\hline
set & \code{\{1,2,3\}}, & યુનિક તત્વો & હા \\
\hline
\end{tabulary}
\end{answertable}

\begin{mnemonicbox}\mnemonic{Python ટાઇપ્સ: નંબર્સ, સિક્વન્સીસ, કલેક્શન્સ}\end{mnemonicbox}
\end{solutionbox}

\questionmarks{પ્રશ્ન 3(અ)}{03}{Python માં ફ્લો કંટ્રોલ શું છે? ઉદાહરણ સાથે સમજાવો}

\begin{solutionbox}
\textbf{ફ્લો કંટ્રોલ} શરતી અને લૂપ સ્ટ્રક્ચર્સ વાપરીને પ્રોગ્રામ સ્ટેટમેન્ટ્સનો એક્ઝિક્યુશન ઓર્ડર મેનેજ કરે છે.

\textbf{ફ્લો કંટ્રોલ પ્રકારોનું ટેબલ:}
\begin{answertable}{ફ્લો કંટ્રોલ}
\begin{tabulary}{\linewidth}{|l|l|l|l|}
\hline
\textbf{પ્રકાર} & \textbf{સ્ટેટમેન્ટ} & \textbf{હેતુ} & \textbf{ઉદાહરણ} \\
\hline
સિક્વેન્શિયલ & સામાન્ય એક્ઝિક્યુશન & લાઇન બાય લાઇન & \code{print("Hello")} \\
\hline
સિલેક્શન & if, elif, else & નિર્ણય લેવો & \code{if x > 0:} \\
\hline
Iteration & for, while & પુનરાવર્તન & \code{for i in range(5):} \\
\hline
Jump & break, continue & લૂપ કંટ્રોલ & \code{break} \\
\hline
\end{tabulary}
\end{answertable}

\begin{mnemonicbox}\mnemonic{ફ્લો કંટ્રોલ: નિર્ણય, પુનરાવર્તન, Jump}\end{mnemonicbox}
\end{solutionbox}

\questionmarks{પ્રશ્ન 3(બ)}{04}{નેસ્ટેડ if સ્ટેટમેન્ટ સમજાવવા માટે પ્રોગ્રામ લખો.}

\begin{solutionbox}
\textbf{નેસ્ટેડ If સ્ટેટમેન્ટ પ્રોગ્રામ:}
\begin{lstlisting}[language=Python]
# નેસ્ટેડ if વાપરીને ગ્રેડ કેલ્ક્યુલેશન
marks = int(input("માર્ક્સ દાખલ કરો: "))

if marks >= 0 and marks <= 100:
    if marks >= 90:
        grade = "A+"
    elif marks >= 80:
        if marks >= 85:
            grade = "A"
        else:
            grade = "B+"
    elif marks >= 70:
        grade = "B"
    elif marks >= 60:
        grade = "C"
    else:
        grade = "F"
    print(f"ગ્રેડ: {grade}")
else:
    print("અયોગ્ય માર્ક્સ")
\end{lstlisting}

\textbf{નેસ્ટેડ સ્ટ્રક્ચર ડાયાગ્રામ:}
\begin{center}
\begin{tikzpicture}[node distance=1.5cm, auto]
    \node[trapezium, trapezium left angle=70, trapezium right angle=110, draw, fill=orange!10, align=center, minimum height=2em] (input) {Marks Input};
    \node[diamond, draw, fill=green!10, align=center, aspect=2, font=\small, below=of input] (range) {0<=marks<=100?};
    \node[diamond, draw, fill=green!10, align=center, aspect=2, font=\small, below left=of range] (grade90) {marks>=90?};
    \node[rectangle, draw, fill=blue!10, align=center, rounded corners, minimum height=2em, font=\small, below right=of range] (invalid) {Invalid};
    
    \node[rectangle, draw, fill=blue!10, align=center, rounded corners, minimum height=2em, font=\small, below left=of grade90] (Ap) {A+};
    \node[diamond, draw, fill=green!10, align=center, aspect=2, font=\small, below right=of grade90] (grade80) {marks>=80?};

    \draw[draw, -latex] (input) -- (range);
    \draw[draw, -latex] (range) -- node[left] {True} (grade90);
    \draw[draw, -latex] (range) -- node[right] {False} (invalid);
    \draw[draw, -latex] (grade90) -- node[left] {Yes} (Ap);
    \draw[draw, -latex] (grade90) -- node[right] {No} (grade80);
\end{tikzpicture}
\end{center}

\begin{mnemonicbox}\mnemonic{નેસ્ટેડ If: નિર્ણયોની અંદર નિર્ણયો}\end{mnemonicbox}
\end{solutionbox}

\questionmarks{પ્રશ્ન 3(ક)}{07}{Arguments અને Parameters ના પ્રકારો સમજાવવા માટે એક પ્રોગ્રામ લખો.}

\begin{solutionbox}
\textbf{Arguments અને Parameters ના પ્રકારો:}
\begin{lstlisting}[language=Python]
# 1. પોઝિશનલ Arguments
def greet(name, age):
    print(f"હેલો {name}, તમારી ઉંમર {age} વર્ષ છે")

greet("જોન", 25)  # પોઝિશનલ arguments

# 2. કીવર્ડ Arguments  
greet(age=30, name="એલિસ")  # કીવર્ડ arguments

# 3. ડિફૉલ્ટ Parameters
def introduce(name, city="અજાણ"):
    print(f"{name} {city} માં રહે છે")

introduce("બોબ")  # ડિફૉલ્ટ વેલ્યુ વાપરે
introduce("કેરોલ", "મુંબઈ")  # ડિફૉલ્ટ ઓવરરાઇડ

# 4. વેરિયેબલ-લેન્થ Arguments (*args)
def sum_all(*numbers):
    return sum(numbers)

result = sum_all(1, 2, 3, 4, 5)
print(f"સરવાળો: {result}")

# 5. કીવર્ડ વેરિયેબલ Arguments (**kwargs)
def display_info(**info):
    for key, value in info.items():
        print(f"{key}: {value}")

display_info(name="ડેવિડ", age=28, city="બોસ્ટન")
\end{lstlisting}

\begin{mnemonicbox}\mnemonic{Parameters: પોઝિશન, કીવર્ડ્સ, ડિફૉલ્ટ્સ, વેરિયેબલ્સ}\end{mnemonicbox}
\end{solutionbox}

\questionmarks{પ્રશ્ન 3(અ OR)}{03}{break અને continue statement ને ઉદાહરણ સાથે સમજાવો.}

\begin{solutionbox}
\textbf{Break અને Continue સ્ટેટમેન્ટ્સ:}

\textbf{Break સ્ટેટમેન્ટ:}
\begin{lstlisting}[language=Python]
# Break ઉદાહરણ - લૂપમાંથી બહાર નીકળો
for i in range(10):
    if i == 5:
        break
    print(i)
# આઉટપુટ: 0, 1, 2, 3, 4
\end{lstlisting}

\textbf{Continue સ્ટેટમેન્ટ:}
\begin{lstlisting}[language=Python]
# Continue ઉદાહરણ - iteration છોડો
for i in range(5):
    if i == 2:
        continue
    print(i)
# આઉટપુટ: 0, 1, 3, 4
\end{lstlisting}

\textbf{સરખામણી ટેબલ:}
\begin{answertable}{Break vs Continue}
\begin{tabulary}{\linewidth}{|l|l|l|l|}
\hline
\textbf{સ્ટેટમેન્ટ} & \textbf{હેતુ} & \textbf{ક્રિયા} & \textbf{ઉદાહરણ ઉપયોગ} \\
\hline
break & લૂપમાંથી બહાર નીકળો & સંપૂર્ણ લૂપ સમાપ્ત કરે & શરત પર બહાર નીકળો \\
\hline
continue & iteration છોડો & આગલા iteration પર જાઓ & સ્પેસિફિક વેલ્યુઝ છોડો \\
\hline
\end{tabulary}
\end{answertable}

\begin{mnemonicbox}\mnemonic{Break બહાર નીકળે, Continue છોડે}\end{mnemonicbox}
\end{solutionbox}

\questionmarks{પ્રશ્ન 3(બ OR)}{04}{નીચેની પેટર્ન દર્શાવવા માટે એક પ્રોગ્રામ બનાવો}

\begin{solutionbox}
\textbf{પેટર્ન:}
\begin{verbatim}
1
12
123
1234
12345
\end{verbatim}

\textbf{નંબર પેટર્ન પ્રોગ્રામ:}
\begin{lstlisting}[language=Python]
# પદ્ધતિ 1: નેસ્ટેડ લૂપ્સ વાપરીને
rows = 5
for i in range(1, rows + 1):
    for j in range(1, i + 1):
        print(j, end="")
    print()  # નવી લાઇન
    
# પદ્ધતિ 2: સ્ટ્રિંગ મેનિપ્યુલેશન વાપરીને
for i in range(1, 6):
    line = ""
    for j in range(1, i + 1):
        line += str(j)
    print(line)
\end{lstlisting}

\begin{mnemonicbox}\mnemonic{પેટર્ન: પંક્તિ નંબર કોલમ કાઉન્ટ નક્કી કરે}\end{mnemonicbox}
\end{solutionbox}

\questionmarks{પ્રશ્ન 3(ક OR)}{07}{દરેક માટે કોડ લખીને નીચેના ગાણિતિક કાર્યો સમજાવો: 1. abs() 2. max() 3. pow() 4. sum()}

\begin{solutionbox}
\textbf{Python માં ગાણિતિક ફંક્શન્સ:}
\begin{lstlisting}[language=Python]
# 1. abs() - એબ્સોલ્યુટ વેલ્યુ
numbers = [-5, 3.7, -10.2, 0]
print("abs() ફંક્શન ઉદાહરણો:")
for num in numbers:
    print(f"abs({num}) = {abs(num)}")

# 2. max() - મહત્તમ વેલ્યુ
list1 = [4, 7, 2, 9, 1]
print(f"\nmax() ફંક્શન ઉદાહરણો:")
print(f"max({list1}) = {max(list1)}")
print(f"max(10, 25, 5) = {max(10, 25, 5)}")

# 3. pow() - પાવર ફંક્શન
print(f"\npow() ફંક્શન ઉદાહરણો:")
print(f"pow(2, 3) = {pow(2, 3)}")      # 2^3 = 8
print(f"pow(5, 2) = {pow(5, 2)}")      # 5^2 = 25
\end{lstlisting}

\begin{mnemonicbox}\mnemonic{Math ફંક્શન્સ: એબ્સોલ્યુટ, મહત્તમ, પાવર, સરવાળો}\end{mnemonicbox}
\end{solutionbox}

\questionmarks{પ્રશ્ન 4(અ)}{03}{Variables નો scope સમજાવો.}

\begin{solutionbox}
\textbf{વેરિયેબલ સ્કોપ} એ પ્રોગ્રામમાં તે પ્રદેશનો સંદર્ભ આપે છે જ્યાં વેરિયેબલ એક્સેસ કરી શકાય.

\textbf{સ્કોપ પ્રકારોનું ટેબલ:}
\begin{answertable}{સ્કોપ પ્રકારો}
\begin{tabulary}{\linewidth}{|l|l|l|l|}
\hline
\textbf{સ્કોપ} & \textbf{વર્ણન} & \textbf{જીવનકાળ} & \textbf{એક્સેસ} \\
\hline
લોકલ & ફંક્શનની અંદર & ફંક્શન એક્ઝિક્યુશન & માત્ર ફંક્શન \\
\hline
ગ્લોબલ & ફંક્શનોની બહાર & પ્રોગ્રામ એક્ઝિક્યુશન & આખો પ્રોગ્રામ \\
\hline
બિલ્ટ-ઇન & Python કીવર્ડ્સ & Python સેશન & બધે \\
\hline
\end{tabulary}
\end{answertable}

\begin{mnemonicbox}\mnemonic{સ્કોપ: લોકલ ફંક્શનમાં રહે, ગ્લોબલ બધે રહે}\end{mnemonicbox}
\end{solutionbox}

\questionmarks{પ્રશ્ન 4(બ)}{04}{નેસ્ટેડ LOOP બનાવવા અને નંબરો દર્શાવવા માટે એક પ્રોગ્રામ વિકસાવો.}

\begin{solutionbox}
\textbf{નેસ્ટેડ લૂપ પ્રોગ્રામ:}
\begin{lstlisting}[language=Python]
# ઉદાહરણ 1: નંબર ગ્રીડ
print("નંબર ગ્રીડ પેટર્ન:")
for i in range(1, 4):
    for j in range(1, 5):
        print(f"{i}{j}", end=" ")
    print()  # દરેક પંક્તિ પછી નવી લાઇન
\end{lstlisting}

\textbf{નેસ્ટેડ લૂપ સ્ટ્રક્ચર:}
\begin{center}
\begin{tikzpicture}[node distance=1.5cm, auto]
    \node[rectangle, draw, fill=blue!10, align=center, rounded corners, minimum height=2em, font=\small] (outer) {બાહ્ય લૂપ (i)};
    \node[rectangle, draw, fill=blue!10, align=center, rounded corners, minimum height=2em, font=\small, below=of outer] (inner) {અંદરૂની લૂપ (j)};
    \node[rectangle, draw, fill=blue!10, align=center, rounded corners, minimum height=2em, font=\small, right=of inner] (action) {ક્રિયા};

    \draw[draw, -latex] (outer) -- (inner);
    \draw[draw, -latex] (inner) -- (action);
    \draw[draw, -latex] (action) |- (inner);
    \draw[draw, -latex] (inner) -| (outer);
\end{tikzpicture}
\end{center}

\begin{mnemonicbox}\mnemonic{નેસ્ટેડ લૂપ્સ: બાહ્ય નિયંત્રણ કરે અંદરૂની}\end{mnemonicbox}
\end{solutionbox}

\questionmarks{પ્રશ્ન 4(ક)}{07}{1 થી 50 ની રેન્જમાં ODD અને EVEN નંબરોની યાદી બનાવવા માટે પ્રોગ્રામ લખો.}

\begin{solutionbox}
\textbf{ODD અને EVEN નંબરો પ્રોગ્રામ:}
\begin{lstlisting}[language=Python]
# પદ્ધતિ 1: લૂપ્સ અને શરતો વાપરીને
odd_numbers = []
even_numbers = []

for i in range(1, 51):
    if i % 2 == 0:
        even_numbers.append(i)
    else:
        odd_numbers.append(i)

print("ODD નંબરો (1-50):")
print(odd_numbers)

print("\nEVEN નંબરો (1-50):")
print(even_numbers)
\end{lstlisting}

\begin{mnemonicbox}\mnemonic{Odd/Even: 2 થી ભાગતા બાકી 1/0}\end{mnemonicbox}
\end{solutionbox}

\questionmarks{પ્રશ્ન 4(અ OR)}{03}{ઉદાહરણ સાથે String Slicing સમજાવો.}

\begin{solutionbox}
\textbf{સ્ટ્રિંગ સ્લાઇસિંગ} \code{[start:stop:step]} સિન્ટેક્સ વાપરીને સ્ટ્રિંગના ભાગો કાઢે છે.

\textbf{સ્લાઇસિંગ સિન્ટેક્સ ટેબલ:}
\begin{answertable}{સ્લાઇસિંગ}
\begin{tabulary}{\linewidth}{|l|l|l|l|}
\hline
\textbf{સિન્ટેક્સ} & \textbf{વર્ણન} & \textbf{ઉદાહરણ} & \textbf{પરિણામ} \\
\hline
\code{s[start:stop]} & શરૂઆતથી stop-1 સુધી & \code{"hello"[1:4]} & "ell" \\
\hline
\code{s[start:]} & શરૂઆતથી અંત સુધી & \code{"hello"[2:]} & "llo" \\
\hline
\code{s[:stop]} & શરૂઆતથી stop-1 સુધી & \code{"hello"[:3]} & "hel" \\
\hline
\code{s[::step]} & દરેક step કેરેક્ટર & \code{"hello"[::2]} & "hlo" \\
\hline
\code{s[::-1]} & રિવર્સ સ્ટ્રિંગ & \code{"hello"[::-1]} & "olleh" \\
\hline
\end{tabulary}
\end{answertable}

\begin{mnemonicbox}\mnemonic{સ્લાઇસ: સ્ટાર્ટ, સ્ટોપ, સ્ટેપ}\end{mnemonicbox}
\end{solutionbox}

\questionmarks{પ્રશ્ન 4(બ OR)}{04}{આપેલ નંબરનું ફેક્ટોરિયલ શોધવા માટે યુઝર ડિફાઇન્ડ ફંક્શન વાપરીને પ્રોગ્રામ લખો.}

\begin{solutionbox}
\textbf{ફેક્ટોરિયલ ફંક્શન પ્રોગ્રામ:}
\begin{lstlisting}[language=Python]
def factorial(n):
    """રિકર્ઝન વાપરીને ફેક્ટોરિયલ"""
    if n == 0 or n == 1:
        return 1
    else:
        return n * factorial(n - 1)

# મેઇન પ્રોગ્રામ
number = int(input("નંબર દાખલ કરો: "))
if number < 0:
    print("નેગેટિવ નંબરો માટે ફેક્ટોરિયલ નથી")
else:
    result1 = factorial(number)
    print(f"{number} નું ફેક્ટોરિયલ = {result1}")
\end{lstlisting}

\begin{mnemonicbox}\mnemonic{ફેક્ટોરિયલ: નીચેના તમામ નંબરો ગુણો}\end{mnemonicbox}
\end{solutionbox}

\questionmarks{પ્રશ્ન 4(ક OR)}{07}{આપેલ સ્ટ્રિંગમાં સબ સ્ટ્રિંગ હાજર છે કે નહીં તે તપાસવા માટે યુઝર ડિફાઇન્ડ ફંક્શન લખો.}

\begin{solutionbox}
\textbf{સબસ્ટ્રિંગ ચેક ફંક્શન:}
\begin{lstlisting}[language=Python]
def find_substring(main_string, sub_string):
    """મેઇન સ્ટ્રિંગમાં સબસ્ટ્રિંગ છે કે નહીં તે તપાસો"""
    if sub_string in main_string:
        index = main_string.find(sub_string)
        return True, index
    else:
        return False, -1

# મેઇન પ્રોગ્રામ
text = input("મેઇન સ્ટ્રિંગ દાખલ કરો: ")
search = input("સર્ચ કરવા માટે સબસ્ટ્રિંગ: ")

found, position = find_substring(text, search)
if found:
    print(f"સબસ્ટ્રિંગ '{search}' પોઝિશન {position} પર મળ્યું")
else:
    print(f"સબસ્ટ્રિંગ '{search}' મળ્યું નથી")
\end{lstlisting}

\begin{mnemonicbox}\mnemonic{સબસ્ટ્રિંગ: સર્ચ, ફાઇન્ડ, કાઉન્ટ, પોઝિશન}\end{mnemonicbox}
\end{solutionbox}

\questionmarks{પ્રશ્ન 5(અ)}{03}{List કેવી રીતે બનાવવું અને એક્સેસ કરવું તે ઉદાહરણ સાથે સમજાવો.}

\begin{solutionbox}
\textbf{લિસ્ટ બનાવવું અને એક્સેસ કરવું:}
\begin{lstlisting}[language=Python]
# લિસ્ટ બનાવવું
numbers = [1, 2, 3, 4, 5]

# એક્સેસિંગ
print(f"પ્રથમ તત્વ: {numbers[0]}")      # 1
print(f"છેલ્લું તત્વ: {numbers[-1]}")   # 5
print(f"સ્લાઇસ: {numbers[1:4]}")        # [2, 3, 4]
\end{lstlisting}

\begin{mnemonicbox}\mnemonic{લિટ્સ: ક્રિએટ, ઇન્ડેક્સ, એક્સેસ}\end{mnemonicbox}
\end{solutionbox}

\questionmarks{પ્રશ્ન 5(બ)}{04}{LIST પર કરી શકાતા ઓપરેશન્સની યાદી બનાવો. એક List બનાવી બીજા List માં કોપી કરવા માટે પ્રોગ્રામ લખો.}

\begin{solutionbox}
\textbf{લિસ્ટ ઓપરેશન્સ અને કોપી પ્રોગ્રામ:}
\begin{lstlisting}[language=Python]
# ઓરિજિનલ લિસ્ટ
original = [1, 2, 3, 4, 5]
print(f"ઓરિજિનલ લિસ્ટ: {original}")

# કોપી પદ્ધતિઓ
shallow_copy = original.copy()
slice_copy = original[:]
list_copy = list(original)

# ઓરિજિનલ બદલો
original.append(6)
print(f"એપેન્ડ પછી: {original}")
print(f"શેલો કોપી: {shallow_copy}")
\end{lstlisting}

\textbf{લિસ્ટ ઓપરેશન્સ ટેબલ:}
\begin{answertable}{લિસ્ટ ઓપરેશન્સ}
\begin{tabulary}{\linewidth}{|l|l|l|l|}
\hline
\textbf{ઓપરેશન} & \textbf{પદ્ધતિ} & \textbf{ઉદાહરણ} & \textbf{પરિણામ} \\
\hline
Add & \code{append()} & \code{[1,2].append(3)} & [1,2,3] \\
\hline
Insert & \code{insert()} & \code{[1,3].insert(1,2)} & [1,2,3] \\
\hline
Remove & \code{remove()} & \code{[1,2,3].remove(2)} & [1,3] \\
\hline
Pop & \code{pop()} & \code{[1,2,3].pop()} & [1,2] \\
\hline
\end{tabulary}
\end{answertable}

\begin{mnemonicbox}\mnemonic{લિસ્ટ ઓપરેશન્સ: એડ, ઇન્સર્ટ, રિમૂવ, પોપ, કોપી}\end{mnemonicbox}
\end{solutionbox}

\questionmarks{પ્રશ્ન 5(ક)}{07}{LIST ના વિવિધ Built in methods ની યાદી અને ઉપયોગ આપો}

\begin{solutionbox}
\textbf{બિલ્ટ-ઇન લિસ્ટ મેથડ્સ:}
\begin{lstlisting}[language=Python]
# ડેમો લિસ્ટ
fruits = ['apple', 'banana', 'cherry', 'apple']

# મોડિફિકેશન
fruits.append('date')              # અંતે ઉમેરો
fruits.insert(1, 'avocado')       # ઇન્ડેક્સ પર
fruits.remove('apple')            # પ્રથમ રિમૂવ

# સર્ચ
count = fruits.count('apple')     # ગણતરી
index = fruits.index('banana')    # ઇન્ડેક્સ શોધો

# સોર્ટિંગ
fruits.sort()                     # સોર્ટ
fruits.reverse()                  # રિવર્સ
\end{lstlisting}

\begin{mnemonicbox}\mnemonic{લિસ્ટ મેથડ્સ: એડ, રિમૂવ, સર્ચ, સોર્ટ, કોપી}\end{mnemonicbox}
\end{solutionbox}

\questionmarks{પ્રશ્ન 5(અ OR)}{03}{સ્ટ્રિંગ કેવી રીતે બનાવવી અને ટ્રાવર્સ કરવી તે ઉદાહરણ આપી સમજાવો.}

\begin{solutionbox}
\textbf{સ્ટ્રિંગ ક્રિએશન અને ટ્રાવર્સલ:}
\begin{lstlisting}[language=Python]
# સ્ટ્રિંગ બનાવવી
string1 = "Hello World"        # ડબલ ક્વોટ્સ
string2 = 'Python'             # સિંગલ ક્વોટ્સ

# સ્ટ્રિંગ ટ્રાવર્સલ
text = "Python"

# પદ્ધતિ 1: લૂપ વાપરીને
for char in text:
    print(char, end=" ")
\end{lstlisting}

\begin{mnemonicbox}\mnemonic{સ્ટ્રિંગ્સ: ક્રિએટ, લૂપ, એક્સેસ}\end{mnemonicbox}
\end{solutionbox}

\questionmarks{પ્રશ્ન 5(બ OR)}{04}{String પર કરી શકાતા ઓપરેશન્સની યાદી બનાવો. કોઈ પણ 2 ઓપરેશન્સ માટે કોડ લખો}

\begin{solutionbox}
\textbf{સ્ટ્રિંગ ઓપરેશન્સ:}
\begin{lstlisting}[language=Python]
# ઓપરેશન 1: સ્ટ્રિંગ કન્કેટેનેશન
first = "જય"
last = "હિન્દ"
full = first + " " + last
print(f"કન્કેટેનેશન: {full}")

# ઓપરેશન 2: કેસ કન્વર્ઝન
sentence = "learn python"
title_case = sentence.title()
print(f"ટાઇટલ કેસ: {title_case}")
\end{lstlisting}

\begin{mnemonicbox}\mnemonic{સ્ટ્રિંગ ઓપરેશન્સ: જોઈન, કેસ, સ્પ્લિટ, ફાઇન્ડ}\end{mnemonicbox}
\end{solutionbox}

\questionmarks{પ્રશ્ન 5(ક OR)}{07}{String ના વિવિધ built – in methods ની યાદી અને ઉપયોગ આપો.}

\begin{solutionbox}
\textbf{બિલ્ટ-ઇન સ્ટ્રિંગ મેથડ્સ:}
\begin{lstlisting}[language=Python]
# ડેમો સ્ટ્રિંગ
text = "  Python Programming  "

# કેસ
print(f"upper(): {text.upper()}")
print(f"lower(): {text.lower()}")

# વ્હાઇટસ્પેસ
print(f"strip(): '{text.strip()}'")

# સર્ચ
print(f"find('Python'): {text.find('Python')}")

# સ્પ્લિટ
words = text.split()
print(f"split(): {words}")
\end{lstlisting}

\begin{mnemonicbox}\mnemonic{સ્ટ્રિંગ મેથડ્સ: કેસ, ક્લીન, ચેક, ચેન્જ}\end{mnemonicbox}
\end{solutionbox}

\end{document}
