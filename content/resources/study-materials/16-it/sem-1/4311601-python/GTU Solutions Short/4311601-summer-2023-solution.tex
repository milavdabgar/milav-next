\documentclass{article}

% content/resources/templates/preamble.tex
\usepackage[margin=0.6in]{geometry}
\author{Milav Dabgar}
\usepackage{amsmath,amssymb,amsthm}
\usepackage{booktabs}
\usepackage{multirow}
\usepackage{xcolor}
\usepackage{tcolorbox}
\tcbuselibrary{breakable,skins}
\usepackage[colorlinks=true,linkcolor=blue]{hyperref}
\usepackage{titlesec}
\usepackage{enumitem}
\usepackage{tikz}
\usepackage{pgfplots}
\usepackage{circuitikz}
\usepackage[version=4]{mhchem}
\usepackage{longtable}
\usepackage{array}
\usepackage{float}
\usepackage{caption}
\usepackage{listings}

\lstset{
  basicstyle=\small\ttfamily,
  breaklines=true,
  breakatwhitespace=false,
  postbreak=\mbox{\textcolor{red}{$\hookrightarrow$}\space},
  float=false,
  numbers=left,
  numberstyle=\tiny\color{gray},
  numbersep=10pt,
  xleftmargin=2em,
  keywordstyle=\color{blue},
  commentstyle=\color{green!60!black},
  stringstyle=\color{purple},
  backgroundcolor=\color{gray!5},
  showstringspaces=false,
  tabsize=2,
  captionpos=b,
  keepspaces=true,
  columns=flexible
}

\pgfplotsset{compat=1.18}
\usetikzlibrary{shapes,arrows,positioning,calc,patterns,decorations.pathmorphing,decorations.markings,arrows.meta}

% Color scheme
\definecolor{headcolor}{RGB}{0,102,204}
\definecolor{keycolor}{RGB}{220,20,60}
\definecolor{solutioncolor}{RGB}{34,139,34}
\definecolor{mnemoniccolor}{RGB}{148,0,211}
\definecolor{codecolor}{RGB}{0,0,100}

% Spacing
\setlength{\parskip}{3pt}
\setlist[itemize]{nosep}
\setlist[enumerate]{nosep}

% Title formatting
\titleformat{\section}{\Large\bfseries\color{headcolor}}{\thesection}{1em}{}
\titleformat{\subsection}{\large\bfseries\color{headcolor}}{\thesubsection}{1em}{}

% Pandoc tightlist compatibility
\providecommand{\tightlist}{%
  \setlength{\itemsep}{0pt}\setlength{\parskip}{0pt}}

% Pandoc longtable compatibility
\newcounter{none}
\def\thenone{}


% content/resources/templates/english-boxes.tex

% Custom environments
\newtcolorbox{solutionbox}{
 breakable,
 enhanced,
 colback=solutioncolor!5!white,
 colframe=solutioncolor!75!black,
 fonttitle=\bfseries,
 title=Solution
}

\newtcolorbox{solutionboxnobreak}{
 colback=solutioncolor!5!white,
 colframe=solutioncolor!75!black,
 fonttitle=\bfseries,
 title=Solution
}

\newtcolorbox{keyformula}{
 breakable,
 enhanced,
 colback=keycolor!5!white,
 colframe=keycolor!75!black,
 fonttitle=\bfseries,
 title=Key Formula
}

\newtcolorbox{mnemonicboxenv}{
 breakable,
 enhanced,
 colback=mnemoniccolor!5!white,
 colframe=mnemoniccolor!75!black,
 fonttitle=\bfseries,
 title=Mnemonic
}

\newcommand{\mnemonicbox}[1]{%
  \begin{mnemonicboxenv}
    #1
  \end{mnemonicboxenv}
}


% Custom commands for GTU solutions
% This file defines semantic commands for consistent formatting

% Question command with automatic formatting
\newcommand{\question}[2]{%
  \section*{Question #1}%
  \textbf{#2}%
}

% OR question variant
\newcommand{\questionor}[2]{%
  \section*{Question #1 OR}%
  \textbf{#2}%
}

% Proper table environment with caption
\newenvironment{answertable}[1]{%
  \begin{table}[htbp]
  \centering
  \caption{#1}
}{%
  \end{table}
}

% Proper figure environment for diagrams
\newenvironment{answerdiagram}[1]{%
  \begin{figure}[htbp]
  \centering
  \caption{#1}
}{%
  \end{figure}
}

% Semantic markup for key terms
\newcommand{\keyword}[1]{\textbf{#1}}
\newcommand{\code}[1]{\texttt{#1}}
\newcommand{\classname}[1]{\texttt{#1}}
\newcommand{\methodname}[1]{\texttt{#1}}

% Proper quotation marks
\newcommand{\mnemonic}[1]{``#1''}


\title{Python Programming (4311601) - Summer 2023 Solution}
\date{August 09, 2023}

\begin{document}
\maketitle

\questionmarks{1(a)}{3}{Explain the steps involved in problem-solving.}

\begin{solutionbox}
\begin{center}
\captionof{table}{Problem Solving Steps}
\begin{tabulary}{\linewidth}{|L|L|}
\hline
\textbf{Step} & \textbf{Description} \\ \hline
\textbf{Problem Understanding} & Read and understand the problem clearly \\ \hline
\textbf{Analysis} & Break down the problem into smaller parts \\ \hline
\textbf{Algorithm Design} & Create step-by-step solution approach \\ \hline
\textbf{Implementation} & Code the solution using programming language \\ \hline
\textbf{Testing} & Verify solution with different test cases \\ \hline
\textbf{Documentation} & Document the solution for future reference \\ \hline
\end{tabulary}
\end{center}

\textbf{Key Points:}
\begin{itemize}
    \item \keyword{Problem Definition}: Clearly identify what needs to be solved
    \item \keyword{Input/Output}: Determine required inputs and expected outputs
    \item \keyword{Logic Building}: Create logical flow of solution
\end{itemize}
\end{solutionbox}

\begin{mnemonicbox}
\mnemonic{People Always Design Implementation Tests Daily}
\end{mnemonicbox}

\questionmarks{1(b)}{4}{Write features of Python.}

\begin{solutionbox}
\begin{center}
\captionof{table}{Python Features}
\begin{tabulary}{\linewidth}{|L|L|}
\hline
\textbf{Feature} & \textbf{Description} \\ \hline
\textbf{Simple Syntax} & Easy to read and write code \\ \hline
\textbf{Interpreted} & No compilation needed, runs directly \\ \hline
\textbf{Platform Independent} & Runs on Windows, Mac, Linux \\ \hline
\textbf{Object-Oriented} & Supports classes and objects \\ \hline
\textbf{Large Library} & Extensive built-in modules \\ \hline
\textbf{Dynamic Typing} & No need to declare variable types \\ \hline
\end{tabulary}
\end{center}

\textbf{Key Features:}
\begin{itemize}
    \item \keyword{Free and Open Source}: Available for everyone to use
    \item \keyword{High-level Language}: Close to human language
    \item \keyword{Extensive Support}: Large community and documentation
\end{itemize}
\end{solutionbox}

\begin{mnemonicbox}
\mnemonic{Simple Interpreted Platform-independent Object-oriented Libraries Dynamic}
\end{mnemonicbox}

\questionmarks{1(c)}{7}{Draw a flowchart and write algorithm to calculate the factorial of a given number.}

\begin{solutionbox}
\textbf{Flowchart:}
\begin{center}
\begin{tikzpicture}[node distance=2cm, auto]
    \node [gtu state] (start) {Start};
    \node [gtu block, below of=start] (input) {Input number n};
    \node [gtu decision, below of=input] (dec1) {n < 0?};
    \node [gtu block, right=2cm of dec1] (invalid) {Print "Invalid"};
    \node [gtu block, below of=dec1, yshift=-0.5cm] (init) {fact = 1, i = 1};
    \node [gtu decision, below of=init] (loop) {i <= n?};
    \node [gtu block, below of=loop, yshift=-0.5cm] (calc) {fact = fact * i\\i = i + 1};
    \node [gtu block, right=2cm of loop] (print) {Print fact};
    \node [gtu state, below of=print] (stop) {End};

    \path [gtu arrow] (start) -- (input);
    \path [gtu arrow] (input) -- (dec1);
    \path [gtu arrow] (dec1) -- node {Yes} (invalid);
    \path [gtu arrow] (dec1) -- node {No} (init);
    \path [gtu arrow] (init) -- (loop);
    \path [gtu arrow] (loop) -- node {Yes} (calc);
    \path [gtu arrow] (calc) -- ++(-2,0) |- (loop);
    \path [gtu arrow] (loop) -- node {No} (print);
    \path [gtu arrow] (print) -- (stop);
    \path [gtu arrow] (invalid) |- (stop);
\end{tikzpicture}
\captionof{figure}{Flowchart for Factorial}
\end{center}

\textbf{Algorithm:}
\begin{enumerate}
    \item Start
    \item Input number n
    \item If n < 0, print ``Invalid input'' and go to step 8
    \item Initialize fact = 1, i = 1
    \item While i <= n, do:
    \begin{itemize}
        \item fact = fact * i
        \item i = i + 1
    \end{itemize}
    \item Print fact
    \item End
\end{enumerate}

\textbf{Key Points:}
\begin{itemize}
    \item \keyword{Base Case}: 0! = 1 and 1! = 1
    \item \keyword{Validation}: Check for negative numbers
    \item \keyword{Loop Logic}: Multiply all numbers from 1 to n
\end{itemize}
\end{solutionbox}

\begin{mnemonicbox}
\mnemonic{Input Validate Initialize Loop Print}
\end{mnemonicbox}

\questionmarks{1(c OR)}{7}{Explain relational and assignment operators with example.}

\begin{solutionbox}
\begin{center}
\captionof{table}{Relational Operators}
\begin{tabulary}{\linewidth}{|C|L|L|}
\hline
\textbf{Operator} & \textbf{Description} & \textbf{Example} \\ \hline
\textbf{==} & Equal to & 5 == 5 (True) \\ \hline
\textbf{!=} & Not equal to & 5 != 3 (True) \\ \hline
\textbf{>} & Greater than & 7 > 3 (True) \\ \hline
\textbf{<} & Less than & 2 < 8 (True) \\ \hline
\textbf{>=} & Greater than or equal & 5 >= 5 (True) \\ \hline
\textbf{<=} & Less than or equal & 4 <= 6 (True) \\ \hline
\end{tabulary}
\end{center}

\begin{center}
\captionof{table}{Assignment Operators}
\begin{tabulary}{\linewidth}{|C|L|L|}
\hline
\textbf{Operator} & \textbf{Description} & \textbf{Example} \\ \hline
\textbf{=} & Simple assignment & x = 5 \\ \hline
\textbf{+=} & Add and assign & x += 3 (x = x + 3) \\ \hline
\textbf{-=} & Subtract and assign & x -= 2 (x = x - 2) \\ \hline
\textbf{*=} & Multiply and assign & x *= 4 (x = x * 4) \\ \hline
\textbf{/=} & Divide and assign & x /= 2 (x = x / 2) \\ \hline
\end{tabulary}
\end{center}

\begin{lstlisting}[language=Python,caption={Operators Example}]
# Relational operators
a, b = 10, 5
print(a > b)   # True
print(a == b)  # False

# Assignment operators
x = 10
x += 5  # x becomes 15
x *= 2  # x becomes 30
\end{lstlisting}
\end{solutionbox}

\begin{mnemonicbox}
\mnemonic{Compare Relations, Assign Values}
\end{mnemonicbox}

\questionmarks{2(a)}{3}{Draw various symbols used for flowchart and write purpose of each symbol.}

\begin{solutionbox}
\begin{center}
\captionof{table}{Flowchart Symbols}
\begin{tabulary}{\linewidth}{|C|L|L|}
\hline
\textbf{Symbol} & \textbf{Name} & \textbf{Purpose} \\ \hline
\tikz{\node[gtu state, minimum width=1.5cm, minimum height=0.8cm] {Start};} & Terminal & Start/End of program \\ \hline
\tikz{\node[gtu block, minimum width=1.5cm, minimum height=0.8cm] {Proc};} & Process & Processing operations \\ \hline
\tikz{\node[gtu decision, minimum width=1.5cm, minimum height=0.8cm] {?};} & Decision & Conditional statements \\ \hline
\tikz{\node[draw, trapezium, trapezium left angle=70, trapezium right angle=110, minimum width=1.5cm, minimum height=0.8cm] {I/O};} & Input/Output & Data input/output \\ \hline
\tikz{\node[draw, circle, minimum size=0.8cm] {};} & Connector & Connect different parts \\ \hline
\tikz{\draw[gtu arrow] (0,0) -- (1,0);} & Flow line & Direction of flow \\ \hline
\end{tabulary}
\end{center}

\textbf{Key Points:}
\begin{itemize}
    \item \keyword{Standard Symbols}: Universally recognized shapes
    \item \keyword{Clear Flow}: Arrows show program direction
    \item \keyword{Logical Structure}: Helps visualize program logic
\end{itemize}
\end{solutionbox}

\begin{mnemonicbox}
\mnemonic{Terminals Process Decisions Input Connectors Flow}
\end{mnemonicbox}

\questionmarks{2(b)}{4}{List out characteristics of good algorithm.}

\begin{solutionbox}
\begin{center}
\captionof{table}{Characteristics of Good Algorithm}
\begin{tabulary}{\linewidth}{|L|L|}
\hline
\textbf{Characteristic} & \textbf{Description} \\ \hline
\textbf{Finite} & Must terminate after finite steps \\ \hline
\textbf{Definite} & Each step clearly defined \\ \hline
\textbf{Input} & Zero or more inputs specified \\ \hline
\textbf{Output} & At least one output produced \\ \hline
\textbf{Effective} & Steps must be simple and feasible \\ \hline
\textbf{Unambiguous} & Each step has only one meaning \\ \hline
\end{tabulary}
\end{center}

\textbf{Key Characteristics:}
\begin{itemize}
    \item \keyword{Correctness}: Produces correct results for all valid inputs
    \item \keyword{Efficiency}: Uses minimum time and space resources
    \item \keyword{Clarity}: Easy to understand and implement
\end{itemize}
\end{solutionbox}

\begin{mnemonicbox}
\mnemonic{Finite Definite Input Output Effective Unambiguous}
\end{mnemonicbox}

\questionmarks{2(c)}{7}{Use proper data type to represent the following data values.}

\begin{solutionbox}
\begin{center}
\captionof{table}{Data Type Mapping}
\begin{tabulary}{\linewidth}{|L|L|L|}
\hline
\textbf{Data Value} & \textbf{Data Type} & \textbf{Example} \\ \hline
(1) Number of days in a week & \textbf{int} & \code{days = 7} \\ \hline
(2) Resident of Gujarat or not & \textbf{bool} & \code{is\_resident = True} \\ \hline
(3) Mobile number & \textbf{str} & \code{mobile = "9876543210"} \\ \hline
(4) Bank account balance & \textbf{float} & \code{balance = 15000.50} \\ \hline
(5) Volume of a sphere & \textbf{float} & \code{volume = 523.33} \\ \hline
(6) Perimeter of a square & \textbf{float} & \code{perimeter = 20.0} \\ \hline
(7) Name of the student & \textbf{str} & \code{name = "Rahul"} \\ \hline
\end{tabulary}
\end{center}

\begin{lstlisting}[language=Python,caption={Data Type Examples}]
# Data type examples
days = 7                    # int
is_resident = True          # bool
mobile = "9876543210"       # str
balance = 15000.50          # float
volume = 523.33             # float
perimeter = 20.0            # float
name = "Rahul"              # str
\end{lstlisting}

\textbf{Key Points:}
\begin{itemize}
    \item \keyword{int}: Whole numbers without decimals
    \item \keyword{float}: Numbers with decimal points
    \item \keyword{str}: Text data in quotes
    \item \keyword{bool}: True/False values only
\end{itemize}
\end{solutionbox}

\begin{mnemonicbox}
\mnemonic{Integers Float Strings Booleans}
\end{mnemonicbox}

\questionmarks{2(a OR)}{3}{Find the output of following code.}

\begin{solutionbox}
\begin{lstlisting}[language=Python,caption={Code Snippet}]
num1 = 2+9*((3*12)-8)/10
print(num1)
\end{lstlisting}

\textbf{Step-by-step calculation:}
\begin{itemize}
    \item Step 1: $3 \times 12 = 36$
    \item Step 2: $36 - 8 = 28$
    \item Step 3: $9 \times 28 = 252$
    \item Step 4: $252 / 10 = 25.2$
    \item Step 5: $2 + 25.2 = 27.2$
\end{itemize}

\textbf{Output:} \code{27.2}

\textbf{Key Points:}
\begin{itemize}
    \item \keyword{BODMAS Rule}: Brackets, Orders, Division, Multiplication, Addition, Subtraction
    \item \keyword{Operator Precedence}: Parentheses first, then multiplication/division
    \item \keyword{Result Type}: Float due to division operation
\end{itemize}
\end{solutionbox}

\begin{mnemonicbox}
\mnemonic{Brackets Orders Division Multiplication Addition Subtraction}
\end{mnemonicbox}

\questionmarks{2(b OR)}{4}{List out the various types of operators used in Python.}

\begin{solutionbox}
\begin{center}
\captionof{table}{Python Operators}
\begin{tabulary}{\linewidth}{|L|L|L|}
\hline
\textbf{Type} & \textbf{Operators} & \textbf{Example} \\ \hline
\textbf{Arithmetic} & +, -, *, /, \%, **, // & \code{5 + 3 = 8} \\ \hline
\textbf{Comparison} & ==, !=, >, <, >=, <= & \code{5 > 3 = True} \\ \hline
\textbf{Logical} & and, or, not & \code{True and False = False} \\ \hline
\textbf{Assignment} & =, +=, -=, *=, /= & \code{x += 5} \\ \hline
\textbf{Bitwise} & \&, |, \^, \~, <<, >> & \code{5 \& 3 = 1} \\ \hline
\textbf{Membership} & in, not in & \code{'a' in 'cat' = True} \\ \hline
\textbf{Identity} & is, is not & \code{x is y} \\ \hline
\end{tabulary}
\end{center}

\textbf{Key Points:}
\begin{itemize}
    \item \keyword{Arithmetic}: Mathematical operations
    \item \keyword{Comparison}: Compare values and return boolean
    \item \keyword{Logical}: Combine boolean expressions
\end{itemize}
\end{solutionbox}

\begin{mnemonicbox}
\mnemonic{Arithmetic Comparison Logical Assignment Bitwise Membership Identity}
\end{mnemonicbox}

\questionmarks{2(c OR)}{7}{Write a program to find the sum and average of all the positive numbers entered by the user. As soon as the user enters a negative number, stop taking in any further input from the user and display the sum and average.}

\begin{solutionbox}
\begin{lstlisting}[language=Python,caption={Sum and Average Program}]
# Program to find sum and average of positive numbers
total_sum = 0
count = 0

print("Enter positive numbers (negative to stop):")

while True:
    num = float(input("Enter number: "))
    
    if num < 0:
        break
    
    total_sum += num
    count += 1

if count > 0:
    average = total_sum / count
    print(f"Sum: {total_sum}")
    print(f"Average: {average}")
else:
    print("No positive numbers entered")
\end{lstlisting}

\textbf{Key Points:}
\begin{itemize}
    \item \keyword{Loop Control}: While loop with break statement
    \item \keyword{Input Validation}: Check for negative numbers
    \item \keyword{Division by Zero}: Handle case when no numbers entered
\end{itemize}
\end{solutionbox}

\begin{mnemonicbox}
\mnemonic{Input Loop Check Calculate Display}
\end{mnemonicbox}

\questionmarks{3(a)}{3}{Explain while loop with example.}

\begin{solutionbox}
\textbf{While Loop Structure:}
\begin{lstlisting}[language=Python]
while condition:
    # statements
    # update condition
\end{lstlisting}

\textbf{Example:}
\begin{lstlisting}[language=Python,caption={While Loop Example}]
# Print numbers 1 to 5
i = 1
while i <= 5:
    print(i)
    i += 1
\end{lstlisting}

\textbf{Key Points:}
\begin{itemize}
    \item \keyword{Pre-tested Loop}: Condition checked before execution
    \item \keyword{Infinite Loop Risk}: Condition must eventually become False
    \item \keyword{Loop Variable}: Must be updated inside loop
\end{itemize}
\end{solutionbox}

\begin{mnemonicbox}
\mnemonic{While Condition True Execute}
\end{mnemonicbox}

\questionmarks{3(b)}{4}{Write a program to find the sum of digits of an integer number, input by the user.}

\begin{solutionbox}
\begin{lstlisting}[language=Python,caption={Sum of Digits Program}]
# Program to find sum of digits
num = int(input("Enter a number: "))
original_num = num
digit_sum = 0

while num > 0:
    digit = num % 10
    digit_sum += digit
    num = num // 10

print(f"Sum of digits of {original_num} is {digit_sum}")
\end{lstlisting}

\textbf{Key Points:}
\begin{itemize}
    \item \keyword{Modulo Operation}: Extract last digit using \code{\%10}
    \item \keyword{Integer Division}: Remove last digit using \code{//10}
    \item \keyword{Loop Until Zero}: Continue until no digits remain
\end{itemize}
\end{solutionbox}

\begin{mnemonicbox}
\mnemonic{Extract Add Remove Repeat}
\end{mnemonicbox}

\questionmarks{3(c)}{7}{Write a program to print Armstrong numbers between 100 to 10000 using a user-defined function.}

\begin{solutionbox}
\begin{lstlisting}[language=Python,caption={Armstrong Numbers Program}]
def is_armstrong(num):
    """Check if number is Armstrong number"""
    original = num
    num_digits = len(str(num))
    sum_powers = 0
    
    while num > 0:
        digit = num % 10
        sum_powers += digit ** num_digits
        num //= 10
    
    return sum_powers == original

def print_armstrong_range(start, end):
    """Print Armstrong numbers in given range"""
    print(f"Armstrong numbers between {start} and {end}:")
    
    for num in range(start, end + 1):
        if is_armstrong(num):
            print(num, end=" ")
    print()

# Main program
print_armstrong_range(100, 10000)
\end{lstlisting}

\textbf{Key Points:}
\begin{itemize}
    \item \keyword{Function Definition}: \code{def} keyword to create functions
    \item \keyword{Armstrong Logic}: Sum of digits raised to power of number of digits
    \item \keyword{Range Function}: Generate numbers in specified range
\end{itemize}
\end{solutionbox}

\begin{mnemonicbox}
\mnemonic{Define Check Calculate Compare Print}
\end{mnemonicbox}

\questionmarks{3(a OR)}{3}{Write a Program to print following pattern.}

\begin{solutionbox}
\begin{verbatim}
5 4 3 2 1
4 3 2 1
3 2 1
2 1
1
\end{verbatim}

\begin{lstlisting}[language=Python,caption={Pattern Printing}]
# Pattern printing program
for i in range(5, 0, -1):
    for j in range(i, 0, -1):
        print(j, end=" ")
    print()
\end{lstlisting}

\textbf{Key Points:}
\begin{itemize}
    \item \keyword{Nested Loops}: Outer loop for rows, inner for columns
    \item \keyword{Reverse Range}: \code{range(start, stop, -1)} for decreasing
    \item \keyword{Print Control}: \code{end=" "} for space, \code{print()} for newline
\end{itemize}
\end{solutionbox}

\begin{mnemonicbox}
\mnemonic{Outer Inner Reverse Print}
\end{mnemonicbox}

\questionmarks{3(b OR)}{4}{Explain nested if...else statement.}

\begin{solutionbox}
\textbf{Structure:}
\begin{lstlisting}[language=Python]
if condition1:
    if condition2:
        # statements
    else:
        # statements
else:
    if condition3:
        # statements
    else:
        # statements
\end{lstlisting}

\textbf{Example:}
\begin{lstlisting}[language=Python,caption={Nested If-Else Example}]
marks = 85

if marks >= 50:
    if marks >= 90:
        grade = "A+"
    elif marks >= 80:
        grade = "A"
    else:
        grade = "B"
else:
    grade = "F"

print(f"Grade: {grade}")
\end{lstlisting}

\textbf{Key Points:}
\begin{itemize}
    \item \keyword{Inner Conditions}: if-else inside another if-else
    \item \keyword{Multiple Levels}: Can nest multiple levels deep
    \item \keyword{Logical Flow}: Inner conditions execute only if outer is true
\end{itemize}
\end{solutionbox}

\begin{mnemonicbox}
\mnemonic{Outer Inner Multiple Levels}
\end{mnemonicbox}

\questionmarks{3(c OR)}{7}{Write a program to enter n numbers in list and using statistics module find mean, median and mode.}

\begin{solutionbox}
\begin{lstlisting}[language=Python,caption={Statistics Module Program}]
import statistics

# Input number of elements
n = int(input("Enter number of elements: "))
numbers = []

# Input numbers
for i in range(n):
    num = float(input(f"Enter number {i+1}: "))
    numbers.append(num)

# Calculate statistics
mean_val = statistics.mean(numbers)
median_val = statistics.median(numbers)

try:
    mode_val = statistics.mode(numbers)
except statistics.StatisticsError:
    mode_val = "No unique mode"

# Display results
print(f"Numbers: {numbers}")
print(f"Mean: {mean_val}")
print(f"Median: {median_val}")
print(f"Mode: {mode_val}")
\end{lstlisting}

\textbf{Key Points:}
\begin{itemize}
    \item \keyword{Statistics Module}: Built-in module for statistical functions
    \item \keyword{List Input}: Store numbers in list for processing
    \item \keyword{Exception Handling}: Handle mode calculation errors
\end{itemize}
\end{solutionbox}

\begin{mnemonicbox}
\mnemonic{Import Input Calculate Display}
\end{mnemonicbox}

\questionmarks{4(a)}{3}{Differentiate between a for loop and a while loop in python.}

\begin{solutionbox}
\begin{center}
\captionof{table}{For Loop vs While Loop}
\begin{tabulary}{\linewidth}{|L|L|L|}
\hline
\textbf{Feature} & \textbf{For Loop} & \textbf{While Loop} \\ \hline
\textbf{Purpose} & Known iterations & Unknown iterations \\ \hline
\textbf{Syntax} & \code{for var in sequence} & \code{while condition} \\ \hline
\textbf{Initialization} & Automatic & Manual \\ \hline
\textbf{Update} & Automatic & Manual \\ \hline
\textbf{Use Case} & Iterate over collections & Repeat until condition \\ \hline
\end{tabulary}
\end{center}

\begin{lstlisting}[language=Python,caption={Loop Comparison}]
# For loop
for i in range(5):
    print(i)

# While loop  
i = 0
while i < 5:
    print(i)
    i += 1
\end{lstlisting}
\end{solutionbox}

\begin{mnemonicbox}
\mnemonic{For Known While Unknown}
\end{mnemonicbox}

\questionmarks{4(b)}{4}{Match the following.}

\begin{solutionbox}
\begin{itemize}
    \item \textbf{A. If statement} $\rightarrow$ \textbf{3.} Used to conditionally execute a block of code based on a certain condition
    \item \textbf{B. While loop} $\rightarrow$ \textbf{1.} Executes a block of code repeatedly as long as a certain condition is met
    \item \textbf{C. Break statement} $\rightarrow$ \textbf{5.} Terminates the current loop and moves on to the next iteration
    \item \textbf{D. Continue statement} $\rightarrow$ \textbf{2.} Skips the current iteration and moves on to the next one
\end{itemize}

\textbf{Key Points:}
\begin{itemize}
    \item \keyword{If Statement}: Conditional execution
    \item \keyword{While Loop}: Repeated execution with condition
    \item \keyword{Break}: Exit loop completely
    \item \keyword{Continue}: Skip current iteration only
\end{itemize}
\end{solutionbox}

\begin{mnemonicbox}
\mnemonic{If Conditions While Repeats Break Exits Continue Skips}
\end{mnemonicbox}

\questionmarks{4(c)}{7}{Differentiate between following with the help of an example: a) Argument and Parameter b) Global and Local variable}

\begin{solutionbox}
\textbf{a) Argument vs Parameter:}
\begin{lstlisting}[language=Python,caption={Arguments vs Parameters}]
def greet(name, age):  # name, age are parameters
    print(f"Hello {name}, you are {age} years old")

greet("Raj", 20)  # "Raj", 20 are arguments
\end{lstlisting}

\textbf{b) Global vs Local Variable:}
\begin{lstlisting}[language=Python,caption={Global vs Local Variables}]
x = 10  # Global variable

def my_function():
    y = 5  # Local variable
    global x
    x = 15  # Modifying global variable
    print(f"Local y: {y}")
    print(f"Global x: {x}")

my_function()
print(f"Global x outside: {x}")
\end{lstlisting}

\begin{center}
\captionof{table}{Comparison Overview}
\begin{tabulary}{\linewidth}{|L|L|L|L|}
\hline
\textbf{Type} & \textbf{Scope} & \textbf{Access} & \textbf{Example} \\ \hline
\textbf{Parameter} & Function definition & Receives values & \code{def func(param):} \\ \hline
\textbf{Argument} & Function call & Passes values & \code{func(argument)} \\ \hline
\textbf{Global} & Entire program & Everywhere & \code{x = 10} \\ \hline
\textbf{Local} & Inside function & Function only & \code{y = 5} in function \\ \hline
\end{tabulary}
\end{center}
\end{solutionbox}

\begin{mnemonicbox}
\mnemonic{Parameters Receive Arguments Pass Globals Everywhere Locals Function}
\end{mnemonicbox}

\questionmarks{4(a OR)}{3}{Find the output of following statements.}

\begin{solutionbox}
\begin{lstlisting}[language=Python,caption={Math Functions}]
import math
(i) print(math.ceil(-9.7))   # Output: -9
(ii) print(math.floor(-9.7)) # Output: -10  
(iii) print(math.fabs(-12.3)) # Output: 12.3
\end{lstlisting}

\textbf{Explanation:}
\begin{itemize}
    \item \textbf{ceil(-9.7)}: Ceiling rounds up to nearest integer = -9
    \item \textbf{floor(-9.7)}: Floor rounds down to nearest integer = -10
    \item \textbf{fabs(-12.3)}: Absolute value removes negative sign = 12.3
\end{itemize}

\textbf{Key Points:}
\begin{itemize}
    \item \keyword{Math Module}: Import required for mathematical functions
    \item \keyword{Negative Numbers}: Ceiling and floor work differently with negatives
    \item \keyword{Absolute Value}: Always returns positive value
\end{itemize}
\end{solutionbox}

\begin{mnemonicbox}
\mnemonic{Ceiling Up Floor Down Absolute Positive}
\end{mnemonicbox}

\questionmarks{4(b OR)}{4}{Write advantages of function.}

\begin{solutionbox}
\begin{center}
\captionof{table}{Advantages of Function}
\begin{tabulary}{\linewidth}{|L|L|}
\hline
\textbf{Advantage} & \textbf{Description} \\ \hline
\textbf{Code Reusability} & Write once, use multiple times \\ \hline
\textbf{Modularity} & Break complex problems into smaller parts \\ \hline
\textbf{Easier Debugging} & Locate and fix errors easily \\ \hline
\textbf{Code Organization} & Better structure and readability \\ \hline
\textbf{Maintainability} & Easy to update and modify \\ \hline
\textbf{Reduced Complexity} & Simplify complex operations \\ \hline
\end{tabulary}
\end{center}

\textbf{Key Benefits:}
\begin{itemize}
    \item \keyword{Avoid Repetition}: No need to write same code again
    \item \keyword{Team Collaboration}: Different people can work on different functions
    \item \keyword{Testing}: Each function can be tested independently
\end{itemize}
\end{solutionbox}

\begin{mnemonicbox}
\mnemonic{Reuse Modular Debug Organize Maintain Reduce}
\end{mnemonicbox}

\questionmarks{4(c OR)}{7}{Write a program to find the smallest and largest number in a given list without using in built functions.}

\begin{solutionbox}
\begin{lstlisting}[language=Python,caption={Find Min Max Code}]
# Program to find smallest and largest without built-in functions
def find_min_max(numbers):
    """Find minimum and maximum without built-in functions"""
    if not numbers:
        return None, None
    
    smallest = numbers[0]
    largest = numbers[0]
    
    for num in numbers[1:]:
        if num < smallest:
            smallest = num
        if num > largest:
            largest = num
    
    return smallest, largest

# Input list
n = int(input("Enter number of elements: "))
numbers = []

for i in range(n):
    num = float(input(f"Enter number {i+1}: "))
    numbers.append(num)

# Find min and max
min_num, max_num = find_min_max(numbers)

print(f"List: {numbers}")
print(f"Smallest number: {min_num}")
print(f"Largest number: {max_num}")
\end{lstlisting}

\textbf{Key Points:}
\begin{itemize}
    \item \keyword{Manual Comparison}: Use if conditions instead of min()/max()
    \item \keyword{Initialize Variables}: Start with first element
    \item \keyword{Loop Through}: Compare each element with current min/max
\end{itemize}
\end{solutionbox}

\begin{mnemonicbox}
\mnemonic{Initialize Compare Update Return}
\end{mnemonicbox}

\questionmarks{5(a)}{3}{Differentiate sort() and sorted() methods for list in python.}

\begin{solutionbox}
\begin{center}
\captionof{table}{sort() vs sorted()}
\begin{tabulary}{\linewidth}{|L|L|L|}
\hline
\textbf{Feature} & \textbf{sort()} & \textbf{sorted()} \\ \hline
\textbf{Return Type} & None (modifies original) & New sorted list \\ \hline
\textbf{Original List} & Modified in-place & Unchanged \\ \hline
\textbf{Usage} & \code{list.sort()} & \code{sorted(list)} \\ \hline
\textbf{Memory} & Efficient & Uses extra memory \\ \hline
\end{tabulary}
\end{center}

\begin{lstlisting}[language=Python,caption={Sort Comparison}]
# sort() method
list1 = [3, 1, 4, 2]
list1.sort()
print(list1)  # [1, 2, 3, 4]

# sorted() function
list2 = [3, 1, 4, 2]
new_list = sorted(list2)
print(list2)      # [3, 1, 4, 2] (unchanged)
print(new_list)   # [1, 2, 3, 4]
\end{lstlisting}
\end{solutionbox}

\begin{mnemonicbox}
\mnemonic{Sort Modifies Sorted Creates}
\end{mnemonicbox}

\questionmarks{5(b)}{4}{Explain different way of traversing a string in python with example.}

\begin{solutionbox}
\textbf{String Traversal Methods:}

\textbf{1. Using For Loop:}
\begin{lstlisting}[language=Python]
text = "Python"
for char in text:
    print(char, end=" ")  # P y t h o n
\end{lstlisting}

\textbf{2. Using Index:}
\begin{lstlisting}[language=Python]
text = "Python"
for i in range(len(text)):
    print(text[i], end=" ")  # P y t h o n
\end{lstlisting}

\textbf{3. Using While Loop:}
\begin{lstlisting}[language=Python]
text = "Python"
i = 0
while i < len(text):
    print(text[i], end=" ")
    i += 1
\end{lstlisting}

\textbf{4. Using Enumerate:}
\begin{lstlisting}[language=Python]
text = "Python"
for index, char in enumerate(text):
    print(f"{index}:{char}", end=" ")  # 0:P 1:y 2:t 3:h 4:o 5:n
\end{lstlisting}
\end{solutionbox}

\begin{mnemonicbox}
\mnemonic{For Index While Enumerate}
\end{mnemonicbox}

\questionmarks{5(c)}{7}{Write output of following scripts.}

\begin{solutionbox}
\begin{lstlisting}[language=Python,caption={String Scripts Output}]
(1) s = "Hello, World!"
    print(s[0:5])              # Output: Hello

(2) lst = [1, 2, 3, 4, 5]
    print(lst[2:4])            # Output: [3, 4]

(3) s = "python"
    print(len(s))              # Output: 6

(4) lst = [5, 2, 3, 1, 8]
    lst.sort()                 # lst becomes [1, 2, 3, 5, 8]

(5) s1 = "hello"
    s2 = "world"
    print(s1 + s2)             # Output: helloworld

(6) lst = [1, 2, 3, 4, 5]
    print(sum(lst))            # Output: 15

(7) s = "python"
    print(s[::-1])             # Output: nohtyp
\end{lstlisting}

\textbf{Key Points:}
\begin{itemize}
    \item \keyword{Slicing}: \code{[start:end]} extracts substring/sublist
    \item \keyword{String Length}: \code{len()} returns character count
    \item \keyword{List Sorting}: \code{sort()} modifies list in-place
    \item \keyword{String Concatenation}: + operator joins strings
    \item \keyword{Sum Function}: Adds all list elements
    \item \keyword{Reverse Slicing}: \code{[::-1]} reverses sequence
\end{itemize}
\end{solutionbox}

\begin{mnemonicbox}
\mnemonic{Slice Length Sort Concatenate Sum Reverse}
\end{mnemonicbox}

\questionmarks{5(a OR)}{3}{Explain type conversion in python.}

\begin{solutionbox}
\begin{center}
\captionof{table}{Type Conversion}
\begin{tabulary}{\linewidth}{|L|L|L|}
\hline
\textbf{Type} & \textbf{Function} & \textbf{Example} \\ \hline
\textbf{int()} & Convert to integer & \code{int("5") -> 5} \\ \hline
\textbf{float()} & Convert to float & \code{float("3.14") -> 3.14} \\ \hline
\textbf{str()} & Convert to string & \code{str(25) -> "25"} \\ \hline
\textbf{bool()} & Convert to boolean & \code{bool(1) -> True} \\ \hline
\textbf{list()} & Convert to list & \code{list("abc") -> ['a','b','c']} \\ \hline
\end{tabulary}
\end{center}

\begin{lstlisting}[language=Python,caption={Type Conversion Examples}]
# Implicit conversion
x = 5 + 3.2  # int + float = float (8.2)

# Explicit conversion
num_str = "123"
num_int = int(num_str)  # "123" -> 123
\end{lstlisting}

\textbf{Key Points:}
\begin{itemize}
    \item \keyword{Implicit}: Python automatically converts
    \item \keyword{Explicit}: Programmer manually converts using functions
    \item \keyword{Type Safety}: Some conversions may raise errors
\end{itemize}
\end{solutionbox}

\begin{mnemonicbox}
\mnemonic{Implicit Automatic Explicit Manual}
\end{mnemonicbox}

\questionmarks{5(b OR)}{4}{Explain concatenation and repetition operation on string with example.}

\begin{solutionbox}
\textbf{String Operations:}

\textbf{1. Concatenation (+):}
\begin{lstlisting}[language=Python]
str1 = "Hello"
str2 = "World"
result = str1 + " " + str2
print(result)  # Hello World

# Multiple concatenation
name = "Python"
version = "3.9"
info = "Language: " + name + " Version: " + version
print(info)  # Language: Python Version: 3.9
\end{lstlisting}

\textbf{2. Repetition (*):}
\begin{lstlisting}[language=Python]
text = "Hi! "
repeated = text * 3
print(repeated)  # Hi! Hi! Hi! 

# Pattern creation
pattern = "-" * 10
print(pattern)  # ----------
\end{lstlisting}

\textbf{Key Points:}
\begin{itemize}
    \item \keyword{Concatenation}: Joins strings together using +
    \item \keyword{Repetition}: Repeats string n times using *
    \item \keyword{Immutable}: Original strings remain unchanged
\end{itemize}
\end{solutionbox}

\begin{mnemonicbox}
\mnemonic{Plus Joins Star Repeats}
\end{mnemonicbox}

\questionmarks{5(c OR)}{7}{Write a program to count and display the number of vowels, consonants, uppercase, lowercase characters in a string.}

\begin{solutionbox}
\begin{lstlisting}[language=Python,caption={String Analysis Program}]
def analyze_string(text):
    """Analyze string for different character types"""
    vowels = "aeiouAEIOU"
    
    vowel_count = 0
    consonant_count = 0
    uppercase_count = 0
    lowercase_count = 0
    
    for char in text:
        if char.isalpha():  # Check if character is alphabet
            if char in vowels:
                vowel_count += 1
            else:
                consonant_count += 1
            
            if char.isupper():
                uppercase_count += 1
            elif char.islower():
                lowercase_count += 1
    
    return vowel_count, consonant_count, uppercase_count, lowercase_count

# Input string
text = input("Enter a string: ")

# Analyze string
vowels, consonants, uppercase, lowercase = analyze_string(text)

# Display results
print(f"String: '{text}'")
print(f"Vowels: {vowels}")
print(f"Consonants: {consonants}")
print(f"Uppercase: {uppercase}")
print(f"Lowercase: {lowercase}")
\end{lstlisting}

\textbf{Key Points:}
\begin{itemize}
    \item \keyword{Character Classification}: Use \code{isalpha()}, \code{isupper()}, \code{islower()}
    \item \keyword{Vowel Check}: Compare with vowel string
    \item \keyword{Loop Processing}: Check each character individually
\end{itemize}
\end{solutionbox}

\begin{mnemonicbox}
\mnemonic{Check Classify Count Display}
\end{mnemonicbox}

\end{document}
