\documentclass{article}
% Adjust the relative path to point to the latex-templates directory

% content/resources/templates/preamble.tex
\usepackage[margin=0.6in]{geometry}
\author{Milav Dabgar}
\usepackage{amsmath,amssymb,amsthm}
\usepackage{booktabs}
\usepackage{multirow}
\usepackage{xcolor}
\usepackage{tcolorbox}
\tcbuselibrary{breakable,skins}
\usepackage[colorlinks=true,linkcolor=blue]{hyperref}
\usepackage{titlesec}
\usepackage{enumitem}
\usepackage{tikz}
\usepackage{pgfplots}
\usepackage{circuitikz}
\usepackage[version=4]{mhchem}
\usepackage{longtable}
\usepackage{array}
\usepackage{float}
\usepackage{caption}
\usepackage{listings}

\lstset{
  basicstyle=\small\ttfamily,
  breaklines=true,
  breakatwhitespace=false,
  postbreak=\mbox{\textcolor{red}{$\hookrightarrow$}\space},
  float=false,
  numbers=left,
  numberstyle=\tiny\color{gray},
  numbersep=10pt,
  xleftmargin=2em,
  keywordstyle=\color{blue},
  commentstyle=\color{green!60!black},
  stringstyle=\color{purple},
  backgroundcolor=\color{gray!5},
  showstringspaces=false,
  tabsize=2,
  captionpos=b,
  keepspaces=true,
  columns=flexible
}

\pgfplotsset{compat=1.18}
\usetikzlibrary{shapes,arrows,positioning,calc,patterns,decorations.pathmorphing,decorations.markings,arrows.meta}

% Color scheme
\definecolor{headcolor}{RGB}{0,102,204}
\definecolor{keycolor}{RGB}{220,20,60}
\definecolor{solutioncolor}{RGB}{34,139,34}
\definecolor{mnemoniccolor}{RGB}{148,0,211}
\definecolor{codecolor}{RGB}{0,0,100}

% Spacing
\setlength{\parskip}{3pt}
\setlist[itemize]{nosep}
\setlist[enumerate]{nosep}

% Title formatting
\titleformat{\section}{\Large\bfseries\color{headcolor}}{\thesection}{1em}{}
\titleformat{\subsection}{\large\bfseries\color{headcolor}}{\thesubsection}{1em}{}

% Pandoc tightlist compatibility
\providecommand{\tightlist}{%
  \setlength{\itemsep}{0pt}\setlength{\parskip}{0pt}}

% Pandoc longtable compatibility
\newcounter{none}
\def\thenone{}


% content/resources/templates/english-boxes.tex

% Custom environments
\newtcolorbox{solutionbox}{
 breakable,
 enhanced,
 colback=solutioncolor!5!white,
 colframe=solutioncolor!75!black,
 fonttitle=\bfseries,
 title=Solution
}

\newtcolorbox{solutionboxnobreak}{
 colback=solutioncolor!5!white,
 colframe=solutioncolor!75!black,
 fonttitle=\bfseries,
 title=Solution
}

\newtcolorbox{keyformula}{
 breakable,
 enhanced,
 colback=keycolor!5!white,
 colframe=keycolor!75!black,
 fonttitle=\bfseries,
 title=Key Formula
}

\newtcolorbox{mnemonicboxenv}{
 breakable,
 enhanced,
 colback=mnemoniccolor!5!white,
 colframe=mnemoniccolor!75!black,
 fonttitle=\bfseries,
 title=Mnemonic
}

\newcommand{\mnemonicbox}[1]{%
  \begin{mnemonicboxenv}
    #1
  \end{mnemonicboxenv}
}


% Custom commands for GTU solutions
% This file defines semantic commands for consistent formatting

% Question command with automatic formatting
\newcommand{\question}[2]{%
  \section*{Question #1}%
  \textbf{#2}%
}

% OR question variant
\newcommand{\questionor}[2]{%
  \section*{Question #1 OR}%
  \textbf{#2}%
}

% Proper table environment with caption
\newenvironment{answertable}[1]{%
  \begin{table}[htbp]
  \centering
  \caption{#1}
}{%
  \end{table}
}

% Proper figure environment for diagrams
\newenvironment{answerdiagram}[1]{%
  \begin{figure}[htbp]
  \centering
  \caption{#1}
}{%
  \end{figure}
}

% Semantic markup for key terms
\newcommand{\keyword}[1]{\textbf{#1}}
\newcommand{\code}[1]{\texttt{#1}}
\newcommand{\classname}[1]{\texttt{#1}}
\newcommand{\methodname}[1]{\texttt{#1}}

% Proper quotation marks
\newcommand{\mnemonic}[1]{``#1''}


\title{Python Programming (4311601) - Summer 2024 Solution}
\date{June 18, 2024}

\begin{document}
\maketitle

\questionmarks{1(a)}{3}{Define problem solving and list out the steps of problem solving.}

\begin{solutionbox}
\textbf{Problem solving} is a systematic approach to identify, analyze, and resolve challenges or issues using logical thinking and structured methods.

\textbf{Steps of Problem Solving:}

\begin{center}
\captionof{table}{Problem Solving Steps}
\begin{tabulary}{\linewidth}{|L|L|}
\hline
\textbf{Step} & \textbf{Description} \\ \hline
1. \textbf{Problem Identification} & Clearly understand and define the problem \\ \hline
2. \textbf{Problem Analysis} & Break down the problem into smaller parts \\ \hline
3. \textbf{Solution Design} & Develop possible solutions or algorithms \\ \hline
4. \textbf{Implementation} & Execute the chosen solution \\ \hline
5. \textbf{Testing \& Validation} & Verify the solution works correctly \\ \hline
6. \textbf{Documentation} & Record the solution for future reference \\ \hline
\end{tabulary}
\end{center}
\end{solutionbox}

\begin{mnemonicbox}
\mnemonic{I Always Design Implementation Tests Daily}
\end{mnemonicbox}

\questionmarks{1(b)}{4}{Define variable and mention rule for choosing names of variable.}

\begin{solutionbox}
\textbf{Variable}: A named storage location in memory that holds data values which can be changed during program execution.

\textbf{Variable Naming Rules:}

\begin{center}
\captionof{table}{Variable Naming Rules}
\begin{tabulary}{\linewidth}{|L|L|}
\hline
\textbf{Rule} & \textbf{Description} \\ \hline
\textbf{Start Character} & Must begin with letter (a-z, A-Z) or underscore (\_) \\ \hline
\textbf{Allowed Characters} & Can contain letters, digits (0-9), and underscores \\ \hline
\textbf{Case Sensitive} & \code{myVar} and \code{MyVar} are different variables \\ \hline
\textbf{No Keywords} & Cannot use Python reserved words (if, for, while) \\ \hline
\textbf{No Spaces} & Use underscore instead of spaces \\ \hline
\textbf{Descriptive Names} & Choose meaningful names (age, not x) \\ \hline
\end{tabulary}
\end{center}
\end{solutionbox}

\begin{mnemonicbox}
\mnemonic{Start Alphabetically, Continue Carefully, Never Keywords}
\end{mnemonicbox}

\questionmarks{1(c)}{7}{Design a flowchart to find maximum number out of three given numbers.}

\begin{solutionbox}
A flowchart shows the logical flow to find the maximum of three numbers using comparison operations.

\textbf{Flowchart:}

\begin{center}
\begin{tikzpicture}[node distance=2cm, auto]
    \node [gtu state] (start) {Start};
    \node [gtu block, below=1cm of start] (input) {Input: num1, num2, num3};
    \node [gtu decision, below=1cm of input] (dec1) {num1 $>$ num2?};
    
    \node [gtu decision, below left=1.5cm and 1cm of dec1] (dec2) {num1 $>$ num3?};
    \node [gtu decision, below right=1.5cm and 1cm of dec1] (dec3) {num2 $>$ num3?};
    
    \node [gtu block, below left=1.5cm and 0.5cm of dec2] (res1) {max = num1};
    \node [gtu block, below right=1.5cm and 0.5cm of dec2] (res2) {max = num3};
    
    \node [gtu block, below left=1.5cm and 0.5cm of dec3] (res3) {max = num2};
    \node [gtu block, below right=1.5cm and 0.5cm of dec3] (res4) {max = num3};
    
    \node [gtu block, below=4cm of dec1] (output) {Output: max};
    \node [gtu state, below=1cm of output] (end) {End};

    \path [gtu arrow] (start) -- (input);
    \path [gtu arrow] (input) -- (dec1);
    
    \path [gtu arrow] (dec1) -| node [near start] {Yes} (dec2);
    \path [gtu arrow] (dec1) -| node [near start] {No} (dec3);
    
    \path [gtu arrow] (dec2) -| node [near start] {Yes} (res1);
    \path [gtu arrow] (dec2) -| node [near start] {No} (res2);
    
    \path [gtu arrow] (dec3) -| node [near start] {Yes} (res3);
    \path [gtu arrow] (dec3) -| node [near start] {No} (res4);
    
    \path [gtu arrow] (res1) -- (output);
    \path [gtu arrow] (res2) -- (output);
    \path [gtu arrow] (res3) -- (output);
    \path [gtu arrow] (res4) -- (output);
    \path [gtu arrow] (output) -- (end);
\end{tikzpicture}
\captionof{figure}{Flowchart for Maximum of Three Numbers}
\end{center}

\textbf{Key Points:}
\begin{itemize}
    \item \keyword{Input}: Three numbers (num1, num2, num3)
    \item \keyword{Process}: Compare numbers using nested conditions
    \item \keyword{Output}: Maximum value among the three
\end{itemize}
\end{solutionbox}

\begin{mnemonicbox}
\mnemonic{Compare First Two, Then With Third}
\end{mnemonicbox}

\questionmarks{1(c OR)}{7}{Construct an algorithm which checks entered number is positive and greater than 5 or not.}

\begin{solutionbox}
An algorithm to verify if a number is both positive and greater than 5.

\textbf{Algorithm:}

\begin{lstlisting}
Algorithm: CheckPositiveGreaterThan5
Step 1: START
Step 2: INPUT number
Step 3: IF number > 0 AND number > 5 THEN
           PRINT "Number is positive and greater than 5"
        ELSE
           PRINT "Number does not meet criteria"
        END IF
Step 4: END
\end{lstlisting}

\textbf{Flowchart:}

\begin{center}
\begin{tikzpicture}[node distance=2cm, auto]
    \node [gtu state] (start) {Start};
    \node [gtu block, below=1cm of start] (input) {Input: number};
    \node [gtu decision, below=1cm of input, aspect=2.5] (dec) {number $>$ 0 AND number $>$ 5?};
    \node [gtu block, below left=1.5cm and 0.5cm of dec] (yes) {Print: Matches criteria};
    \node [gtu block, below right=1.5cm and 0.5cm of dec] (no) {Print: No match};
    \node [gtu state, below=3.5cm of dec] (end) {End};

    \path [gtu arrow] (start) -- (input);
    \path [gtu arrow] (input) -- (dec);
    \path [gtu arrow] (dec) -| node [near start] {Yes} (yes);
    \path [gtu arrow] (dec) -| node [near start] {No} (no);
    \path [gtu arrow] (yes) |- (end);
    \path [gtu arrow] (no) |- (end);
\end{tikzpicture}
\captionof{figure}{Positive and Greater than 5 Flowchart}
\end{center}

\textbf{Key Conditions:}
\begin{itemize}
    \item \keyword{Positive}: number $>$ 0
    \item \keyword{Greater than 5}: number $>$ 5
    \item \keyword{Combined}: Both conditions must be true
\end{itemize}
\end{solutionbox}

\begin{mnemonicbox}
\mnemonic{Positive Plus Five}
\end{mnemonicbox}

\questionmarks{2(a)}{3}{Write a short note on arithmetic operators.}

\begin{solutionbox}
Arithmetic operators perform mathematical calculations on numeric values in Python programming.

\begin{center}
\captionof{table}{Arithmetic Operators}
\begin{tabulary}{\linewidth}{|C|L|L|C|}
\hline
\textbf{Op} & \textbf{Name} & \textbf{Example} & \textbf{Result} \\ \hline
\code{+} & Addition & \code{5 + 3} & 8 \\ \hline
\code{-} & Subtraction & \code{5 - 3} & 2 \\ \hline
\code{*} & Multiplication & \code{5 * 3} & 15 \\ \hline
\code{/} & Division & \code{5 / 3} & 1.67 \\ \hline
\code{//} & Floor Division & \code{5 // 3} & 1 \\ \hline
\code{\%} & Modulus & \code{5 \% 3} & 2 \\ \hline
\code{**} & Exponentiation & \code{5 ** 3} & 125 \\ \hline
\end{tabulary}
\end{center}
\end{solutionbox}

\begin{mnemonicbox}
\mnemonic{Add Subtract Multiply Divide Floor Mod Power}
\end{mnemonicbox}

\questionmarks{2(b)}{4}{Explain the need for continue and break statements.}

\begin{solutionbox}
Continue and break statements control loop execution flow for efficient programming.

\textbf{Statement Comparison:}

\begin{center}
\captionof{table}{break vs continue}
\begin{tabulary}{\linewidth}{|L|L|L|}
\hline
\textbf{Statement} & \textbf{Purpose} & \textbf{Action} \\ \hline
\keyword{break} & Exit loop completely & Terminates entire loop \\ \hline
\keyword{continue} & Skip current iteration & Jumps to next iteration \\ \hline
\end{tabulary}
\end{center}

\textbf{Usage Examples:}
\begin{itemize}
    \item \keyword{break}: Exit when condition met (finding specific value)
    \item \keyword{continue}: Skip invalid data (negative numbers in positive list)
\end{itemize}

\textbf{Benefits:}
\begin{itemize}
    \item \keyword{Efficiency}: Avoid unnecessary iterations
    \item \keyword{Control}: Better program flow management
    \item \keyword{Clarity}: Cleaner code logic
\end{itemize}
\end{solutionbox}

\begin{mnemonicbox}
\mnemonic{Break Exits, Continue Skips}
\end{mnemonicbox}

\questionmarks{2(c)}{7}{Create a program to check whether entered number is even or odd.}

\begin{solutionbox}
A Python program using modulus operator to determine if a number is even or odd.

\textbf{Python Code:}
\begin{lstlisting}[language=Python]
# Program to check even or odd
number = int(input("Enter a number: "))

if number % 2 == 0:
    print(f"{number} is Even")
else:
    print(f"{number} is Odd")
\end{lstlisting}

\textbf{Logic Explanation:}

\begin{center}
\captionof{table}{Even vs Odd Logic}
\begin{tabulary}{\linewidth}{|L|L|L|}
\hline
\textbf{Condition} & \textbf{Result} & \textbf{Explanation} \\ \hline
\code{number \% 2 == 0} & Even & Divisible by 2, no remainder \\ \hline
\code{number \% 2 == 1} & Odd & Not divisible by 2, remainder 1 \\ \hline
\end{tabulary}
\end{center}

\textbf{Sample Output:}
\begin{itemize}
    \item Input: 8 $\rightarrow$ Output: "8 is Even"
    \item Input: 7 $\rightarrow$ Output: "7 is Odd"
\end{itemize}
\end{solutionbox}

\begin{mnemonicbox}
\mnemonic{Modulus Zero Even, One Odd}
\end{mnemonicbox}

\questionmarks{2(a OR)}{3}{Summarize the comparison operators of python.}

\begin{solutionbox}
Comparison operators compare values and return boolean results (True/False).

\begin{center}
\captionof{table}{Comparison Operators}
\begin{tabulary}{\linewidth}{|C|L|L|C|}
\hline
\textbf{Op} & \textbf{Name} & \textbf{Example} & \textbf{Result} \\ \hline
\code{==} & Equal to & \code{5 == 5} & True \\ \hline
\code{!=} & Not equal to & \code{5 != 3} & True \\ \hline
\code{>} & Greater than & \code{5 > 3} & True \\ \hline
\code{<} & Less than & \code{5 < 3} & False \\ \hline
\code{>=} & Greater/Equal & \code{5 >= 5} & True \\ \hline
\code{<=} & Less/Equal & \code{5 <= 3} & False \\ \hline
\end{tabulary}
\end{center}

\textbf{Return Type:} All operators return boolean values (True/False)
\end{solutionbox}

\begin{mnemonicbox}
\mnemonic{Equal Not Greater Less Greater-Equal Less-Equal}
\end{mnemonicbox}

\questionmarks{2(b OR)}{4}{Write short note on while loop.}

\begin{solutionbox}
While loop repeatedly executes code block as long as condition remains true.

\textbf{While Loop Structure:}

\begin{center}
\captionof{table}{While Loop Components}
\begin{tabulary}{\linewidth}{|L|L|}
\hline
\textbf{Component} & \textbf{Description} \\ \hline
\textbf{Initialization} & Set initial value before loop \\ \hline
\textbf{Condition} & Boolean expression to test \\ \hline
\textbf{Body} & Code to execute repeatedly \\ \hline
\textbf{Update} & Modify variable to avoid infinite loop \\ \hline
\end{tabulary}
\end{center}

\textbf{Syntax:}
\begin{lstlisting}[language=Python]
while condition:
    # loop body
    # update statement
\end{lstlisting}

\textbf{Characteristics:}
\begin{itemize}
    \item \keyword{Pre-tested}: Condition checked before execution
    \item \keyword{Variable iterations}: Unknown number of repetitions
    \item \keyword{Control}: Condition determines continuation
\end{itemize}
\end{solutionbox}

\begin{mnemonicbox}
\mnemonic{While Condition True, Execute Loop}
\end{mnemonicbox}

\questionmarks{2(c OR)}{7}{Create a program to read three numbers from the user and find the average of the numbers.}

\begin{solutionbox}
A Python program to calculate average of three user-input numbers.

\textbf{Python Code:}
\begin{lstlisting}[language=Python]
# Program to find average of three numbers
num1 = float(input("Enter first number: "))
num2 = float(input("Enter second number: "))
num3 = float(input("Enter third number: "))

average = (num1 + num2 + num3) / 3

print(f"Average of {num1}, {num2}, {num3} is: {average:.2f}")
\end{lstlisting}

\textbf{Calculation Process:}
\begin{itemize}
    \item \keyword{Input}: Read three numbers
    \item \keyword{Sum}: Add all three numbers
    \item \keyword{Divide}: Sum / 3
    \item \keyword{Output}: Display formatted result
\end{itemize}

\textbf{Sample Execution:}
\begin{itemize}
    \item Input: 10, 20, 30
    \item Sum: 60
    \item Average: 20.00
\end{itemize}
\end{solutionbox}

\begin{mnemonicbox}
\mnemonic{Sum Three Divide Display}
\end{mnemonicbox}

\questionmarks{3(a)}{3}{Define control structures, List out control structures available in python.}

\begin{solutionbox}
Control structures determine the execution flow and order of statements in a program.

\textbf{Python Control Structures:}

\begin{center}
\captionof{table}{Control Structures}
\begin{tabulary}{\linewidth}{|L|L|L|}
\hline
\textbf{Type} & \textbf{Structures} & \textbf{Purpose} \\ \hline
\textbf{Sequential} & Normal flow & Execute statements in order \\ \hline
\textbf{Selection} & if, if-else, elif & Choose between alternatives \\ \hline
\textbf{Iteration} & for, while & Repeat code blocks \\ \hline
\textbf{Jump} & break, continue, pass & Alter normal flow \\ \hline
\end{tabulary}
\end{center}

\textbf{Categories:}
\begin{itemize}
    \item \keyword{Conditional}: Decision making (if statements)
    \item \keyword{Looping}: Repetition (for/while loops)
    \item \keyword{Branching}: Flow control (break/continue)
\end{itemize}
\end{solutionbox}

\begin{mnemonicbox}
\mnemonic{Sequence Select Iterate Jump}
\end{mnemonicbox}

\questionmarks{3(b)}{4}{Explain how to define and call user defined function by giving example.}

\begin{solutionbox}
User-defined functions are custom blocks of reusable code that perform specific tasks.

\textbf{Function Structure:}

\begin{center}
\captionof{table}{Function Components}
\begin{tabulary}{\linewidth}{|L|L|L|}
\hline
\textbf{Component} & \textbf{Syntax} & \textbf{Purpose} \\ \hline
\textbf{Definition} & \code{def name():} & Create function \\ \hline
\textbf{Parameters} & \code{def f(p1, p2):} & Accept inputs \\ \hline
\textbf{Body} & Indented block & Function logic \\ \hline
\textbf{Return} & \code{return val} & Send result back \\ \hline
\textbf{Call} & \code{name()} & Execute function \\ \hline
\end{tabulary}
\end{center}

\textbf{Example Code:}
\begin{lstlisting}[language=Python]
# Function definition
def greet_user(name):
    message = f"Hello, {name}!"
    return message

# Function call
result = greet_user("Python")
print(result)  # Output: Hello, Python!
\end{lstlisting}
\end{solutionbox}

\begin{mnemonicbox}
\mnemonic{Define Parameters Body Return Call}
\end{mnemonicbox}

\questionmarks{3(c)}{7}{Create a program to display the following patterns using loop concept}

\begin{solutionbox}
A Python program using nested loops to create number patterns.

\textbf{Python Code:}
\begin{lstlisting}[language=Python]
# Pattern printing program
for i in range(1, 6):
    for j in range(1, i + 1):
        print(i, end="")
    print()  # New line after each row
\end{lstlisting}

\textbf{Pattern Logic:}

\begin{center}
\captionof{table}{Pattern Logic}
\begin{tabulary}{\linewidth}{|C|C|L|}
\hline
\textbf{Row} & \textbf{Iterations} & \textbf{Output} \\ \hline
1 & 1 time & 1 \\ \hline
2 & 2 times & 22 \\ \hline
3 & 3 times & 333 \\ \hline
4 & 4 times & 4444 \\ \hline
5 & 5 times & 55555 \\ \hline
\end{tabulary}
\end{center}

\textbf{Loop Structure:}
\begin{itemize}
    \item \keyword{Outer loop}: Controls rows (1 to 5)
    \item \keyword{Inner loop}: Prints current row number
    \item \keyword{Pattern}: Row number repeated row times
\end{itemize}
\end{solutionbox}

\begin{mnemonicbox}
\mnemonic{Outer Rows Inner Repeats}
\end{mnemonicbox}

\questionmarks{3(a OR)}{3}{Explain nested loop using suitable example.}

\begin{solutionbox}
Nested loop is a loop inside another loop where inner loop completes all iterations for each outer loop iteration.

\textbf{Nested Loop Structure:}
\begin{itemize}
    \item \keyword{Outer Loop}: Controls main iterations
    \item \keyword{Inner Loop}: Executes completely for each outer iteration
    \item \keyword{Execution}: Inner loop runs $n \times m$ times total
\end{itemize}

\textbf{Example Code:}
\begin{lstlisting}[language=Python]
# Nested loop example - Multiplication table
for i in range(1, 4):      # Outer loop
    for j in range(1, 4):  # Inner loop
        print(f"{i}x{j}={i*j}", end=" ")
    print()  # New line
\end{lstlisting}

\textbf{Output Pattern:}
\begin{lstlisting}
1x1=1 1x2=2 1x3=3
2x1=2 2x2=4 2x3=6
3x1=3 3x2=6 3x3=9
\end{lstlisting}
\end{solutionbox}

\begin{mnemonicbox}
\mnemonic{Loop Inside Loop}
\end{mnemonicbox}

\questionmarks{3(b OR)}{4}{Write short note on local and global scope of variables}

\begin{solutionbox}
Variable scope determines where variables can be accessed in a program.

\textbf{Scope Comparison:}

\begin{center}
\captionof{table}{Local vs Global Scope}
\begin{tabulary}{\linewidth}{|L|L|L|L|}
\hline
\textbf{Type} & \textbf{Definition} & \textbf{Access} & \textbf{Lifetime} \\ \hline
\textbf{Local} & Inside function & Function only & Function exec \\ \hline
\textbf{Global} & Outside func & Entire program & Program exec \\ \hline
\end{tabulary}
\end{center}

\textbf{Example Code:}
\begin{lstlisting}[language=Python]
global_var = "I am global"  # Global scope

def my_function():
    local_var = "I am local"    # Local scope
    global global_var
    print(global_var)   # Accessible
    print(local_var)    # Accessible

print(global_var)   # Accessible
# print(local_var)  # Error - not accessible
\end{lstlisting}
\end{solutionbox}

\begin{mnemonicbox}
\mnemonic{Local Limited, Global General}
\end{mnemonicbox}

\questionmarks{3(c OR)}{7}{Develop a user-defined function to find the factorial of a given number.}

\begin{solutionbox}
A recursive function to calculate factorial of a positive integer.

\textbf{Python Code:}
\begin{lstlisting}[language=Python]
def factorial(n):
    """Calculate factorial of n"""
    if n == 0 or n == 1:
        return 1
    else:
        return n * factorial(n - 1)

# Test the function
number = int(input("Enter a number: "))
if number < 0:
    print("Factorial not defined for negative numbers")
else:
    result = factorial(number)
    print(f"Factorial of {number} is {result}")
\end{lstlisting}

\textbf{Factorial Logic:}

\begin{center}
\captionof{table}{Factorial Calculation}
\begin{tabulary}{\linewidth}{|C|L|C|}
\hline
\textbf{Input} & \textbf{Calculation} & \textbf{Result} \\ \hline
0 & Base case & 1 \\ \hline
1 & Base case & 1 \\ \hline
5 & $5 \times 4 \times 3 \times 2 \times 1$ & 120 \\ \hline
\end{tabulary}
\end{center}

\textbf{Function Features:}
\begin{itemize}
    \item \keyword{Recursive}: Function calls itself
    \item \keyword{Base case}: Stops recursion at n=0 or n=1
    \item \keyword{Validation}: Handles negative inputs
\end{itemize}
\end{solutionbox}

\begin{mnemonicbox}
\mnemonic{Multiply All Previous Numbers}
\end{mnemonicbox}

\questionmarks{4(a)}{3}{Explain math module with various functions}

\begin{solutionbox}
Math module provides mathematical functions and constants for numerical computations.

\textbf{Math Module Functions:}

\begin{center}
\captionof{table}{Math Module Functions}
\begin{tabulary}{\linewidth}{|L|L|L|}
\hline
\textbf{Function} & \textbf{Purpose} & \textbf{Example} \\ \hline
\code{math.sqrt()} & Square root & \code{math.sqrt(16) = 4.0} \\ \hline
\code{math.pow()} & Power calculation & \code{math.pow(2, 3) = 8.0} \\ \hline
\code{math.ceil()} & Round up & \code{math.ceil(4.3) = 5} \\ \hline
\code{math.floor()} & Round down & \code{math.floor(4.7) = 4} \\ \hline
\code{math.factorial()} & Factorial & \code{math.factorial(5) = 120} \\ \hline
\end{tabulary}
\end{center}

\textbf{Usage:}
\begin{lstlisting}[language=Python]
import math
result = math.sqrt(25)  # Returns 5.0
\end{lstlisting}
\end{solutionbox}

\begin{mnemonicbox}
\mnemonic{Square Power Ceiling Floor Factorial}
\end{mnemonicbox}

\questionmarks{4(b)}{4}{Discuss the following list functions: i. len() ii. sum() iii. sort() iv. index()}

\begin{solutionbox}
Essential list functions for data manipulation and analysis.

\textbf{List Functions Comparison:}

\begin{center}
\captionof{table}{List Functions}
\begin{tabulary}{\linewidth}{|L|L|L|L|}
\hline
\textbf{Function} & \textbf{Purpose} & \textbf{Return Type} & \textbf{Example} \\ \hline
\code{len()} & Count elements & Integer & \code{len([1,2,3]) = 3} \\ \hline
\code{sum()} & Add all numbers & Number & \code{sum([1,2,3]) = 6} \\ \hline
\code{sort()} & Arrange in order & None (modifies list) & \code{list.sort()} \\ \hline
\code{index()} & Find element position & Integer & \code{[1,2,3].index(2) = 1} \\ \hline
\end{tabulary}
\end{center}

\textbf{Usage Notes:}
\begin{itemize}
    \item \code{len()}: Works with any sequence
    \item \code{sum()}: Only numeric lists
    \item \code{sort()}: Modifies original list
    \item \code{index()}: Returns first occurrence
\end{itemize}
\end{solutionbox}

\begin{mnemonicbox}
\mnemonic{Length Sum Sort Index}
\end{mnemonicbox}

\questionmarks{4(c)}{7}{Create a user-defined function to print the Fibonacci series of 0 to N numbers. (Where N is an integer number and passed as an argument)}

\begin{solutionbox}
A function to generate and display Fibonacci sequence up to N terms.

\textbf{Python Code:}
\begin{lstlisting}[language=Python]
def fibonacci_series(n):
    """Print Fibonacci series of n terms"""
    if n <= 0:
        print("Please enter a positive number")
        return
    
    # First two terms
    a, b = 0, 1
    
    if n == 1:
        print(f"Fibonacci series: {a}")
        return
    
    print(f"Fibonacci series: {a}, {b}", end="")
    
    # Generate remaining terms
    for i in range(2, n):
        c = a + b
        print(f", {c}", end="")
        a, b = b, c
    print()  # New line

# Test function
num = int(input("Enter number of terms: "))
fibonacci_series(num)
\end{lstlisting}

\textbf{Fibonacci Logic:}

\begin{center}
\captionof{table}{Fibonacci Sequence}
\begin{tabulary}{\linewidth}{|C|C|L|}
\hline
\textbf{Term} & \textbf{Value} & \textbf{Calculation} \\ \hline
1st & 0 & Given \\ \hline
2nd & 1 & Given \\ \hline
3rd & 1 & 0 + 1 \\ \hline
4th & 2 & 1 + 1 \\ \hline
5th & 3 & 1 + 2 \\ \hline
\end{tabulary}
\end{center}
\end{solutionbox}

\begin{mnemonicbox}
\mnemonic{Add Previous Two Numbers}
\end{mnemonicbox}

\questionmarks{4(a OR)}{3}{Explain random module with various functions}

\begin{solutionbox}
Random module generates random numbers and makes random selections for various applications.

\textbf{Random Module Functions:}

\begin{center}
\captionof{table}{Random Module Functions}
\begin{tabulary}{\linewidth}{|L|L|L|}
\hline
\textbf{Function} & \textbf{Purpose} & \textbf{Example} \\ \hline
\code{random()} & Float 0.0 to 1.0 & \code{random.random()} \\ \hline
\code{randint()} & Integer in range & \code{random.randint(1, 10)} \\ \hline
\code{choice()} & Random list element & \code{random.choice([1,2,3])} \\ \hline
\code{shuffle()} & Mix list order & \code{random.shuffle(list)} \\ \hline
\code{uniform()} & Float in range & \code{random.uniform(1.0, 5.0)} \\ \hline
\end{tabulary}
\end{center}

\textbf{Usage:}
\begin{lstlisting}[language=Python]
import random
number = random.randint(1, 100)
\end{lstlisting}

\textbf{Applications}: Games, simulations, testing, cryptography
\end{solutionbox}

\begin{mnemonicbox}
\mnemonic{Random Range Choice Shuffle Uniform}
\end{mnemonicbox}

\questionmarks{4(b OR)}{4}{Build a python code to check whether given element is member of list or not.}

\begin{solutionbox}
A Python program to verify if an element exists in a list using membership operator.

\textbf{Python Code:}
\begin{lstlisting}[language=Python]
# Check element membership in list
def check_membership():
    # Sample list
    numbers = [10, 20, 30, 40, 50]
    
    # Get element to search
    element = int(input("Enter element to search: "))
    
    # Check membership
    if element in numbers:
        print(f"{element} is present in the list")
        print(f"Position: {numbers.index(element)}")
    else:
        print(f"{element} is not present in the list")

# Call function
check_membership()
\end{lstlisting}

\textbf{Membership Methods:}

\begin{center}
\captionof{table}{Membership Operators}
\begin{tabulary}{\linewidth}{|L|L|L|}
\hline
\textbf{Method} & \textbf{Syntax} & \textbf{Returns} \\ \hline
\textbf{in operator} & \code{element in list} & Boolean \\ \hline
\textbf{not in operator} & \code{element not in list} & Boolean \\ \hline
\textbf{count() method} & \code{list.count(element)} & Integer \\ \hline
\end{tabulary}
\end{center}
\end{solutionbox}

\begin{mnemonicbox}
\mnemonic{In List True False}
\end{mnemonicbox}

\questionmarks{4(c OR)}{7}{Develop a user defined function that reverses the entered string words}

\begin{solutionbox}
A function to reverse each word in a string while maintaining word positions.

\textbf{Python Code:}
\begin{lstlisting}[language=Python]
def reverse_string_words(text):
    """Reverse each word in the string"""
    # Split string into words
    words = text.split()
    
    # Reverse each word
    reversed_words = []
    for word in words:
        reversed_word = word[::-1]  # Slice notation for reversal
        reversed_words.append(reversed_word)
    
    # Join words back
    result = " ".join(reversed_words)
    return result

# Test function
input_string = input("Enter a string: ")
output = reverse_string_words(input_string)
print(f"Input: \"{input_string}\"")
print(f"Output: \"{output}\"")

# Example with given input
test_input = "Hello IT"
test_output = reverse_string_words(test_input)
print(f"Input: \"{test_input}\"")
print(f"Output: \"{test_output}\"")  # Output: "olleH TI"
\end{lstlisting}

\textbf{Process Steps:}

\begin{center}
\captionof{table}{Reversal Process}
\begin{tabulary}{\linewidth}{|C|L|L|}
\hline
\textbf{Step} & \textbf{Operation} & \textbf{Example} \\ \hline
1 & Split into words & ["Hello", "IT"] \\ \hline
2 & Reverse each word & ["olleH", "TI"] \\ \hline
3 & Join with spaces & "olleH TI" \\ \hline
\end{tabulary}
\end{center}
\end{solutionbox}

\begin{mnemonicbox}
\mnemonic{Split Reverse Join}
\end{mnemonicbox}

\questionmarks{5(a)}{3}{Explain given string methods: i. count() ii. strip() iii. replace()}

\begin{solutionbox}
Essential string methods for text processing and manipulation.

\textbf{String Methods Comparison:}

\begin{center}
\captionof{table}{String Methods}
\begin{tabulary}{\linewidth}{|L|L|L|L|}
\hline
\textbf{Method} & \textbf{Purpose} & \textbf{Syntax} & \textbf{Example} \\ \hline
\code{count()} & Count occurrences & \code{str.count(sub)} & \code{"hello".count("l") = 2} \\ \hline
\code{strip()} & Remove whitespace & \code{str.strip()} & \code{" t ".strip() = "t"} \\ \hline
\code{replace()} & Replace substring & \code{str.replace(o, n)} & \code{"hi".replace("i", "ello")} \\ \hline
\end{tabulary}
\end{center}

\textbf{Return Values:}
\begin{itemize}
    \item \keyword{count()}: Integer (number of occurrences)
    \item \keyword{strip()}: New string (whitespace removed)
    \item \keyword{replace()}: New string (replacements made)
\end{itemize}
\end{solutionbox}

\begin{mnemonicbox}
\mnemonic{Count Strip Replace}
\end{mnemonicbox}

\questionmarks{5(b)}{4}{Explain how to traverse a string by giving example.}

\begin{solutionbox}
String traversal means accessing each character in a string sequentially.

\textbf{Traversal Methods:}

\begin{center}
\captionof{table}{String Traversal}
\begin{tabulary}{\linewidth}{|L|L|L|}
\hline
\textbf{Method} & \textbf{Syntax} & \textbf{Use Case} \\ \hline
\textbf{Index-based} & \code{for i in range(len(str))} & Need position \\ \hline
\textbf{Direct iteration} & \code{for char in string} & Just characters \\ \hline
\textbf{Enumerate} & \code{for i, c in enumerate(str)} & Both \\ \hline
\end{tabulary}
\end{center}

\textbf{Example Code:}
\begin{lstlisting}[language=Python]
text = "Python"

# Method 1: Direct iteration
for char in text:
    print(char, end=" ")  # P y t h o n

# Method 2: Index-based
for i in range(len(text)):
    print(f"{i}: {text[i]}")

# Method 3: Enumerate
for index, character in enumerate(text):
    print(f"Position {index}: {character}")
\end{lstlisting}
\end{solutionbox}

\begin{mnemonicbox}
\mnemonic{Direct Index Enumerate}
\end{mnemonicbox}

\questionmarks{5(c)}{7}{Develop programs to perform the following list operations:}

\begin{solutionbox}
Two programs for essential list operations and analysis.

\textbf{Program 1: Check Element Existence}
\begin{lstlisting}[language=Python]
def check_element_exists(lst, element):
    """Check if element exists in list"""
    if element in lst:
        return True, lst.index(element)
    else:
        return False, -1

# Test program 1
numbers = [10, 25, 30, 45, 50]
search_item = int(input("Enter element to search: "))
exists, position = check_element_exists(numbers, search_item)

if exists:
    print(f"{search_item} found at position {position}")
else:
    print(f"{search_item} not found in list")
\end{lstlisting}

\textbf{Program 2: Find Smallest and Largest}
\begin{lstlisting}[language=Python]
def find_min_max(lst):
    """Find smallest and largest elements"""
    if not lst:  # Empty list check
        return None, None
    
    smallest = min(lst)
    largest = max(lst)
    return smallest, largest

# Test program 2
numbers = [15, 8, 23, 4, 16, 42]
min_val, max_val = find_min_max(numbers)
print(f"List: {numbers}")
print(f"Smallest: {min_val}")
print(f"Largest: {max_val}")
\end{lstlisting}

\textbf{Key Operations:}
\begin{itemize}
    \item \keyword{Membership}: Using 'in' operator
    \item \keyword{Min/Max}: Built-in functions
    \item \keyword{Validation}: Empty list handling
\end{itemize}
\end{solutionbox}

\begin{mnemonicbox}
\mnemonic{Search Find Compare}
\end{mnemonicbox}

\questionmarks{5(a OR)}{3}{Explain slicing of list with example.}

\begin{solutionbox}
List slicing extracts specific portions of a list using index ranges.

\textbf{Slicing Syntax:}

\begin{center}
\captionof{table}{Slicing Syntax}
\begin{tabulary}{\linewidth}{|L|L|L|}
\hline
\textbf{Format} & \textbf{Description} & \textbf{Example} \\ \hline
\code{list[start:end]} & Elements from start to end-1 & \code{[1,2,3,4][1:3] = [2,3]} \\ \hline
\code{list[:end]} & From beginning to end-1 & \code{[1,2,3,4][:2] = [1,2]} \\ \hline
\code{list[start:]} & From start to end & \code{[1,2,3,4][2:] = [3,4]} \\ \hline
\code{list[::step]} & Every step element & \code{[1,2,3,4][::2] = [1,3]} \\ \hline
\end{tabulary}
\end{center}

\textbf{Example:}
\begin{lstlisting}[language=Python]
numbers = [0, 1, 2, 3, 4, 5]
print(numbers[1:4])   # [1, 2, 3]
print(numbers[:3])    # [0, 1, 2]
print(numbers[3:])    # [3, 4, 5]
print(numbers[::2])   # [0, 2, 4]
\end{lstlisting}
\end{solutionbox}

\begin{mnemonicbox}
\mnemonic{Start End Step}
\end{mnemonicbox}

\questionmarks{5(b OR)}{4}{Explain how to traverse a list by giving example.}

\begin{solutionbox}
List traversal involves accessing each element in a list systematically.

\textbf{Traversal Techniques:}

\begin{center}
\captionof{table}{List Traversal}
\begin{tabulary}{\linewidth}{|L|L|L|}
\hline
\textbf{Method} & \textbf{Syntax} & \textbf{Output Type} \\ \hline
\textbf{Value iteration} & \code{for item in list} & Elements only \\ \hline
\textbf{Index iteration} & \code{for i in range(len(list))} & Index access \\ \hline
\textbf{Enumerate} & \code{for i, v in enumerate(list)} & Index and value \\ \hline
\end{tabulary}
\end{center}

\textbf{Example Code:}
\begin{lstlisting}[language=Python]
fruits = ["apple", "banana", "orange"]

# Method 1: Direct value access
print("Values only:")
for fruit in fruits:
    print(fruit)

# Method 2: Index-based access
print("\nWith indices:")
for i in range(len(fruits)):
    print(f"Index {i}: {fruits[i]}")

# Method 3: Enumerate
print("\nUsing enumerate:")
for index, fruit in enumerate(fruits):
    print(f"{index} -> {fruit}")
\end{lstlisting}

\textbf{Use Cases:}
\begin{itemize}
    \item \keyword{Value only}: Simple processing
    \item \keyword{Index access}: Position-dependent operations
    \item \keyword{Enumerate}: Both index and value needed
\end{itemize}
\end{solutionbox}

\begin{mnemonicbox}
\mnemonic{Value Index Both}
\end{mnemonicbox}

\questionmarks{5(c OR)}{7}{Develop python code to create list of prime and non-prime numbers in range 1 to 50.}

\begin{solutionbox}
A Python program to categorize numbers into prime and non-prime lists.

\textbf{Python Code:}
\begin{lstlisting}[language=Python]
def is_prime(n):
    """Check if a number is prime"""
    if n < 2:
        return False
    for i in range(2, int(n**0.5) + 1):
        if n % i == 0:
            return False
    return True

def categorize_numbers(start, end):
    """Create lists of prime and non-prime numbers"""
    prime_numbers = []
    non_prime_numbers = []
    
    for num in range(start, end + 1):
        if is_prime(num):
            prime_numbers.append(num)
        else:
            non_prime_numbers.append(num)
    
    return prime_numbers, non_prime_numbers

# Generate lists for 1 to 50
primes, non_primes = categorize_numbers(1, 50)

print("Prime numbers (1-50):")
print(primes)
print(f"\nTotal prime numbers: {len(primes)}")

print("\nNon-prime numbers (1-50):")
print(non_primes)
print(f"\nTotal non-prime numbers: {len(non_primes)}")
\end{lstlisting}

\textbf{Prime Logic:}

\begin{center}
\captionof{table}{Prime vs Non-Prime}
\begin{tabulary}{\linewidth}{|L|L|L|}
\hline
\textbf{Number Type} & \textbf{Condition} & \textbf{Examples} \\ \hline
\textbf{Prime} & Only divisible by 1 and itself & 2, 3, 5, 7, 11 \\ \hline
\textbf{Non-Prime} & Has other divisors & 1, 4, 6, 8, 9 \\ \hline
\end{tabulary}
\end{center}

\textbf{Algorithm Steps:}
\begin{itemize}
    \item \keyword{Check divisibility} from 2 to $\sqrt{n}$
    \item \keyword{Categorize} based on prime test
    \item \keyword{Store} in appropriate lists
\end{itemize}
\end{solutionbox}

\begin{mnemonicbox}
\mnemonic{Check Divide Categorize Store}
\end{mnemonicbox}

\end{document}

