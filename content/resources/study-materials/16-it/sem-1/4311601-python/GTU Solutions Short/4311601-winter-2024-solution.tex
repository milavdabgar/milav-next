\documentclass{article}

% content/resources/templates/preamble.tex
\usepackage[margin=0.6in]{geometry}
\author{Milav Dabgar}
\usepackage{amsmath,amssymb,amsthm}
\usepackage{booktabs}
\usepackage{multirow}
\usepackage{xcolor}
\usepackage{tcolorbox}
\tcbuselibrary{breakable,skins}
\usepackage[colorlinks=true,linkcolor=blue]{hyperref}
\usepackage{titlesec}
\usepackage{enumitem}
\usepackage{tikz}
\usepackage{pgfplots}
\usepackage{circuitikz}
\usepackage[version=4]{mhchem}
\usepackage{longtable}
\usepackage{array}
\usepackage{float}
\usepackage{caption}
\usepackage{listings}

\lstset{
  basicstyle=\small\ttfamily,
  breaklines=true,
  breakatwhitespace=false,
  postbreak=\mbox{\textcolor{red}{$\hookrightarrow$}\space},
  float=false,
  numbers=left,
  numberstyle=\tiny\color{gray},
  numbersep=10pt,
  xleftmargin=2em,
  keywordstyle=\color{blue},
  commentstyle=\color{green!60!black},
  stringstyle=\color{purple},
  backgroundcolor=\color{gray!5},
  showstringspaces=false,
  tabsize=2,
  captionpos=b,
  keepspaces=true,
  columns=flexible
}

\pgfplotsset{compat=1.18}
\usetikzlibrary{shapes,arrows,positioning,calc,patterns,decorations.pathmorphing,decorations.markings,arrows.meta}

% Color scheme
\definecolor{headcolor}{RGB}{0,102,204}
\definecolor{keycolor}{RGB}{220,20,60}
\definecolor{solutioncolor}{RGB}{34,139,34}
\definecolor{mnemoniccolor}{RGB}{148,0,211}
\definecolor{codecolor}{RGB}{0,0,100}

% Spacing
\setlength{\parskip}{3pt}
\setlist[itemize]{nosep}
\setlist[enumerate]{nosep}

% Title formatting
\titleformat{\section}{\Large\bfseries\color{headcolor}}{\thesection}{1em}{}
\titleformat{\subsection}{\large\bfseries\color{headcolor}}{\thesubsection}{1em}{}

% Pandoc tightlist compatibility
\providecommand{\tightlist}{%
  \setlength{\itemsep}{0pt}\setlength{\parskip}{0pt}}

% Pandoc longtable compatibility
\newcounter{none}
\def\thenone{}


% content/resources/templates/english-boxes.tex

% Custom environments
\newtcolorbox{solutionbox}{
 breakable,
 enhanced,
 colback=solutioncolor!5!white,
 colframe=solutioncolor!75!black,
 fonttitle=\bfseries,
 title=Solution
}

\newtcolorbox{solutionboxnobreak}{
 colback=solutioncolor!5!white,
 colframe=solutioncolor!75!black,
 fonttitle=\bfseries,
 title=Solution
}

\newtcolorbox{keyformula}{
 breakable,
 enhanced,
 colback=keycolor!5!white,
 colframe=keycolor!75!black,
 fonttitle=\bfseries,
 title=Key Formula
}

\newtcolorbox{mnemonicboxenv}{
 breakable,
 enhanced,
 colback=mnemoniccolor!5!white,
 colframe=mnemoniccolor!75!black,
 fonttitle=\bfseries,
 title=Mnemonic
}

\newcommand{\mnemonicbox}[1]{%
  \begin{mnemonicboxenv}
    #1
  \end{mnemonicboxenv}
}


% Custom commands for GTU solutions
% This file defines semantic commands for consistent formatting

% Question command with automatic formatting
\newcommand{\question}[2]{%
  \section*{Question #1}%
  \textbf{#2}%
}

% OR question variant
\newcommand{\questionor}[2]{%
  \section*{Question #1 OR}%
  \textbf{#2}%
}

% Proper table environment with caption
\newenvironment{answertable}[1]{%
  \begin{table}[htbp]
  \centering
  \caption{#1}
}{%
  \end{table}
}

% Proper figure environment for diagrams
\newenvironment{answerdiagram}[1]{%
  \begin{figure}[htbp]
  \centering
  \caption{#1}
}{%
  \end{figure}
}

% Semantic markup for key terms
\newcommand{\keyword}[1]{\textbf{#1}}
\newcommand{\code}[1]{\texttt{#1}}
\newcommand{\classname}[1]{\texttt{#1}}
\newcommand{\methodname}[1]{\texttt{#1}}

% Proper quotation marks
\newcommand{\mnemonic}[1]{``#1''}


\title{Python Programming (4311601) - Winter 2024 Solution}
\date{January 13, 2025}

\begin{document}
\maketitle

\questionmarks{Question 1(a)}{03}{Define Problem Solving, Algorithm and Pseudo Code.}

\begin{solutionbox}
\textbf{Definitions:}
\begin{answertable}{Core Concepts}
\begin{tabular}{|l|l|}
\hline
\textbf{Term} & \textbf{Definition} \\
\hline
\keyword{Problem Solving} & Systematic process of finding solutions to complex issues using logical thinking \\
\hline
\keyword{Algorithm} & Step-by-step procedure to solve a problem with finite operations \\
\hline
\keyword{Pseudo Code} & Informal description of program logic using plain English-like syntax \\
\hline
\end{tabular}
\end{answertable}

\textbf{Key Points:}
\begin{itemize}
    \item \textbf{Problem Solving}: Breaking down complex problems into manageable steps
    \item \textbf{Algorithm}: Must be finite, definite, effective, and produce correct output
    \item \textbf{Pseudo Code}: Bridge between human language and programming code
\end{itemize}

\begin{mnemonicbox}
\mnemonic{PAP - Problem, Algorithm, Pseudo}
\end{mnemonicbox}
\end{solutionbox}

\questionmarks{Question 1(b)}{04}{Explain various Flowchart Symbols. Design a Flowchart to find maximum number out of two given numbers}

\begin{solutionbox}
\textbf{Flowchart Symbols:}
\begin{answertable}{Symbols}
\begin{tabular}{|l|c|l|}
\hline
\textbf{Symbol} & \textbf{Shape} & \textbf{Purpose} \\
\hline
\keyword{Oval} & ⬭ & Start/End \\
\hline
\keyword{Rectangle} & ▭ & Process/Action \\
\hline
\keyword{Diamond} & ◊ & Decision \\
\hline
\keyword{Parallelogram} & ▱ & Input/Output \\
\hline
\end{tabular}
\end{answertable}

\textbf{Flowchart for Maximum of Two Numbers:}
\begin{center}
\begin{tikzpicture}[gtu flow]
    \node[gtustart] (start) {Start};
    \node[gtuio, below=of start] (input) {Input A, B};
    \node[gtudecision, below=of input] (dec) {A > B?};
    \node[gtuprocess, below left=of dec, xshift=-1cm] (maxA) {Max = A};
    \node[gtuprocess, below right=of dec, xshift=1cm] (maxB) {Max = B};
    \node[gtuio, below=of dec, yshift=-3cm] (disp) {Display Max};
    \node[gtustop, below=of disp] (end) {End};

    \path [gtuarrow] (start) -- (input);
    \path [gtuarrow] (input) -- (dec);
    \path [gtuarrow] (dec) -| node[above left] {Yes} (maxA);
    \path [gtuarrow] (dec) -| node[above right] {No} (maxB);
    \path [gtuarrow] (maxA) |- (disp);
    \path [gtuarrow] (maxB) |- (disp);
    \path [gtuarrow] (disp) -- (end);
\end{tikzpicture}
\end{center}

\textbf{Explanation:}
\begin{itemize}
    \item \textbf{Start/End}: Entry and exit points
    \item \textbf{Input/Output}: Data flow operations
    \item \textbf{Decision}: Conditional branching
    \item \textbf{Process}: Computational steps
\end{itemize}

\begin{mnemonicbox}
\mnemonic{SIPO - Start, Input, Process, Output}
\end{mnemonicbox}
\end{solutionbox}

\questionmarks{Question 1(c)}{07}{List out various arithmetic operators of python. Write Python Code that performs various arithmetic operations.}

\begin{solutionbox}
\textbf{Arithmetic Operators:}
\begin{answertable}{Arithmetic Operators}
\begin{tabular}{|l|c|l|l|}
\hline
\textbf{Operator} & \textbf{Symbol} & \textbf{Example} & \textbf{Result} \\
\hline
Addition & + & \code{5 + 3} & 8 \\
\hline
Subtraction & - & \code{5 - 3} & 2 \\
\hline
Multiplication & * & \code{5 * 3} & 15 \\
\hline
Division & / & \code{5 / 3} & 1.667 \\
\hline
Floor Division & // & \code{5 // 3} & 1 \\
\hline
Modulus & \% & \code{5 \% 3} & 2 \\
\hline
Exponentiation & ** & \code{5 ** 3} & 125 \\
\hline
\end{tabular}
\end{answertable}

\textbf{Code:}
\begin{lstlisting}[language=Python]
a = 10
b = 3
print(f"Addition: {a + b}")
print(f"Subtraction: {a - b}")
print(f"Multiplication: {a * b}")
print(f"Division: {a / b}")
print(f"Floor Division: {a // b}")
print(f"Modulus: {a % b}")
print(f"Power: {a ** b}")
\end{lstlisting}

\begin{mnemonicbox}
\mnemonic{Add-Sub-Mul-Div-Floor-Mod-Pow}
\end{mnemonicbox}
\end{solutionbox}

\questionmarks{Question 1(c OR)}{07}{List out various comparison operators of python. Write Python Code which performs various comparison operations.}

\begin{solutionbox}
\textbf{Comparison Operators:}
\begin{answertable}{Comparison Operators}
\begin{tabular}{|l|c|l|l|}
\hline
\textbf{Operator} & \textbf{Symbol} & \textbf{Purpose} & \textbf{Example} \\
\hline
Equal & == & Check equality & \code{5 == 3} \rightarrow False \\
\hline
Not Equal & != & Check inequality & \code{5 != 3} \rightarrow True \\
\hline
Greater Than & > & Check greater & \code{5 > 3} \rightarrow True \\
\hline
Less Than & < & Check smaller & \code{5 < 3} \rightarrow False \\
\hline
Greater Equal & >= & Check greater/equal & \code{5 >= 3} \rightarrow True \\
\hline
Less Equal & <= & Check smaller/equal & \code{5 <= 3} \rightarrow False \\
\hline
\end{tabular}
\end{answertable}

\textbf{Code:}
\begin{lstlisting}[language=Python]
x = 8
y = 5
print(f"Equal: {x == y}")
print(f"Not Equal: {x != y}")
print(f"Greater: {x > y}")
print(f"Less: {x < y}")
print(f"Greater Equal: {x >= y}")
print(f"Less Equal: {x <= y}")
\end{lstlisting}

\begin{mnemonicbox}
\mnemonic{Equal-Not-Greater-Less-GreaterEqual-LessEqual}
\end{mnemonicbox}
\end{solutionbox}

\questionmarks{Question 2(a)}{03}{Write short note on membership operators.}

\begin{solutionbox}
\textbf{Membership Operators:}
\begin{answertable}{Membership Operators}
\begin{tabular}{|l|l|l|}
\hline
\textbf{Operator} & \textbf{Purpose} & \textbf{Example} \\
\hline
\code{in} & Check if element exists & \code{'a' in 'apple'} \rightarrow True \\
\hline
\code{not in} & Check if element doesn't exist & \code{'z' not in 'apple'} \rightarrow True \\
\hline
\end{tabular}
\end{answertable}

\textbf{Key Points:}
\begin{itemize}
    \item \textbf{in operator}: Returns True if element found in sequence
    \item \textbf{not in operator}: Returns True if element not found in sequence
    \item \textbf{Usage}: Lists, strings, tuples, dictionaries
\end{itemize}

\begin{mnemonicbox}
\mnemonic{In-Not-In for membership testing}
\end{mnemonicbox}
\end{solutionbox}

\questionmarks{Question 2(b)}{04}{Define Python. Write down various applications of Python Programming.}

\begin{solutionbox}
\textbf{Python Definition}: High-level, interpreted programming language known for simplicity and readability.

\textbf{Applications:}
\begin{answertable}{Applications}
\begin{tabular}{|l|l|}
\hline
\textbf{Application Area} & \textbf{Examples} \\
\hline
\keyword{Web Development} & Django, Flask frameworks \\
\hline
\keyword{Data Science} & NumPy, Pandas, Matplotlib \\
\hline
\keyword{AI/ML} & TensorFlow, Scikit-learn \\
\hline
\keyword{Desktop Apps} & Tkinter, PyQt \\
\hline
\keyword{Game Development} & Pygame library \\
\hline
\end{tabular}
\end{answertable}

\textbf{Features:}
\begin{itemize}
    \item \textbf{Interpreted}: No compilation needed
    \item \textbf{Cross-platform}: Runs on multiple OS
    \item \textbf{Large libraries}: Extensive standard library
\end{itemize}

\begin{mnemonicbox}
\mnemonic{Web-Data-AI-Desktop-Games}
\end{mnemonicbox}
\end{solutionbox}

\questionmarks{Question 2(c)}{07}{Write python program which calculates electricity bill using following details.}

\begin{solutionbox}
\textbf{Table of Rates:}
\begin{answertable}{Electricity Rates}
\begin{tabular}{|l|l|}
\hline
\textbf{Unit Range} & \textbf{Rate per Unit} \\
\hline
$\le$ 100 & Rs 5.00 \\
\hline
101-200 & Rs 7.50 \\
\hline
201-300 & Rs 10.00 \\
\hline
$\ge$ 301 & Rs 15.00 \\
\hline
\end{tabular}
\end{answertable}

\textbf{Code:}
\begin{lstlisting}[language=Python]
units = int(input("Enter consumed units: "))

if units <= 100:
    bill = units * 5.00
elif units <= 200:
    bill = units * 7.50
elif units <= 300:
    bill = units * 10.00
else:
    bill = units * 15.00

print(f"Total Bill: Rs {bill}")
\end{lstlisting}

\textbf{Explanation:}
\begin{itemize}
    \item \textbf{Conditional logic}: if-elif-else structure
    \item \textbf{Rate calculation}: Based on unit slabs
    \item \textbf{User input}: Interactive billing system
\end{itemize}

\begin{mnemonicbox}
\mnemonic{Input-Check-Calculate-Display}
\end{mnemonicbox}
\end{solutionbox}

\questionmarks{Question 2(a OR)}{03}{Write short note on identity operators.}

\begin{solutionbox}
\textbf{Identity Operators:}
\begin{answertable}{Identity Operators}
\begin{tabular}{|l|l|l|}
\hline
\textbf{Operator} & \textbf{Purpose} & \textbf{Example} \\
\hline
\code{is} & Check same object & \code{a is b} \\
\hline
\code{is not} & Check different object & \code{a is not b} \\
\hline
\end{tabular}
\end{answertable}

\textbf{Key Points:}
\begin{itemize}
    \item \textbf{is operator}: Compares object identity, not values
    \item \textbf{is not operator}: Checks if objects are different
    \item \textbf{Memory comparison}: Checks same memory location
\end{itemize}

\begin{mnemonicbox}
\mnemonic{Is-IsNot for object identity}
\end{mnemonicbox}
\end{solutionbox}

\questionmarks{Question 2(b OR)}{04}{What is indentation in Python? Explain various features of Python.}

\begin{solutionbox}
\textbf{Indentation}: Whitespace at line beginning to define code blocks.

\textbf{Features:}
\begin{answertable}{Python Features}
\begin{tabular}{|l|l|}
\hline
\textbf{Feature} & \textbf{Description} \\
\hline
\keyword{Simple Syntax} & Easy to read and write \\
\hline
\keyword{Interpreted} & No compilation step \\
\hline
\keyword{Object-Oriented} & Supports OOP concepts \\
\hline
\keyword{Cross-Platform} & Runs on multiple OS \\
\hline
\keyword{Large Library} & Extensive standard library \\
\hline
\end{tabular}
\end{answertable}

\textbf{Importance of Indentation:}
\begin{itemize}
    \item \textbf{Indentation}: Replaces curly braces {}
    \item \textbf{Consistent}: Usually 4 spaces per level
    \item \textbf{Mandatory}: Creates code structure
\end{itemize}

\begin{mnemonicbox}
\mnemonic{Simple-Interpreted-Object-Cross-Large}
\end{mnemonicbox}
\end{solutionbox}

\questionmarks{Question 2(c OR)}{07}{Write a python program that calculates Student's class/grade using following details.}

\begin{solutionbox}
\textbf{Grading Table:}
\begin{answertable}{Grading Scheme}
\begin{tabular}{|l|l|}
\hline
\textbf{Percentage} & \textbf{Grade} \\
\hline
$\ge$ 70 & Distinction \\
\hline
60-69 & First Class \\
\hline
50-59 & Second Class \\
\hline
35-49 & Pass Class \\
\hline
< 35 & Fail \\
\hline
\end{tabular}
\end{answertable}

\textbf{Code:}
\begin{lstlisting}[language=Python]
percentage = float(input("Enter percentage: "))

if percentage >= 70:
    grade = "Distinction"
elif percentage >= 60:
    grade = "First Class"
elif percentage >= 50:
    grade = "Second Class"
elif percentage >= 35:
    grade = "Pass Class"
else:
    grade = "Fail"

print(f"Grade: {grade}")
\end{lstlisting}

\textbf{Explanation:}
\begin{itemize}
    \item \textbf{Multiple conditions}: Nested if-elif structure
    \item \textbf{Grade assignment}: Based on percentage ranges
    \item \textbf{Float input}: Handles decimal percentages
\end{itemize}

\begin{mnemonicbox}
\mnemonic{Distinction-First-Second-Pass-Fail}
\end{mnemonicbox}
\end{solutionbox}

\questionmarks{Question 3(a)}{03}{What is Selection Control Statement? List it out.}

\begin{solutionbox}
\textbf{Selection Control Statements:}
\begin{answertable}{Selection Statements}
\begin{tabular}{|l|l|}
\hline
\textbf{Statement Type} & \textbf{Purpose} \\
\hline
\keyword{if} & Single condition check \\
\hline
\keyword{if-else} & Two-way branching \\
\hline
\keyword{if-elif-else} & Multi-way branching \\
\hline
\keyword{nested if} & Conditions within conditions \\
\hline
\end{tabular}
\end{answertable}

\textbf{Key Concepts:}
\begin{itemize}
    \item \textbf{Selection statements}: Control program flow based on conditions
    \item \textbf{Boolean evaluation}: Uses True/False logic
    \item \textbf{Branching}: Different paths of execution
\end{itemize}

\begin{mnemonicbox}
\mnemonic{If-IfElse-IfElif-Nested}
\end{mnemonicbox}
\end{solutionbox}

\questionmarks{Question 3(b)}{04}{Write short note on nested loops.}

\begin{solutionbox}
\textbf{Nested Loops:}
\begin{answertable}{Loop Structure}
\begin{tabular}{|l|l|}
\hline
\textbf{Loop Type} & \textbf{Structure} \\
\hline
\keyword{Outer Loop} & Controls iterations \\
\hline
\keyword{Inner Loop} & Executes completely for each outer iteration \\
\hline
\textbf{Total Iterations} & Outer $\times$ Inner \\
\hline
\end{tabular}
\end{answertable}

\textbf{Key Points:}
\begin{itemize}
    \item \textbf{Nested structure}: Loop inside another loop
    \item \textbf{Complete execution}: Inner loop finishes before outer continues
    \item \textbf{Pattern creation}: Useful for 2D structures
\end{itemize}

\textbf{Code Example:}
\begin{lstlisting}[language=Python]
for i in range(3):
    for j in range(2):
        print(f"i={i}, j={j}")
\end{lstlisting}

\begin{mnemonicbox}
\mnemonic{Outer-Inner-Complete-Pattern}
\end{mnemonicbox}
\end{solutionbox}

\questionmarks{Question 3(c)}{07}{Write a user-define function that displays all numbers, which are divisible by 4 from 1 to 100.}

\begin{solutionbox}
\textbf{Code:}
\begin{lstlisting}[language=Python]
def display_divisible_by_4():
    print("Numbers divisible by 4 from 1 to 100:")
    for num in range(1, 101):
        if num % 4 == 0:
            print(num, end=" ")
    print()

# Function call
display_divisible_by_4()
\end{lstlisting}

\textbf{Alternative with return:}
\begin{lstlisting}[language=Python]
def get_divisible_by_4():
    return [num for num in range(1, 101) if num % 4 == 0]

result = get_divisible_by_4()
print(result)
\end{lstlisting}

\textbf{Key Concepts:}
\begin{itemize}
    \item \textbf{Function definition}: def keyword usage
    \item \textbf{Range function}: 1 to 100 iteration
    \item \textbf{Modulus check}: num \% 4 == 0 condition
    \item \textbf{List comprehension}: Alternative approach
\end{itemize}

\begin{mnemonicbox}
\mnemonic{Define-Range-Check-Display}
\end{mnemonicbox}
\end{solutionbox}

\questionmarks{Question 3(a OR)}{03}{What is Repetition Control Statement? List it out.}

\begin{solutionbox}
\textbf{Repetition Control Statements:}
\begin{answertable}{Loops}
\begin{tabular}{|l|l|}
\hline
\textbf{Statement Type} & \textbf{Purpose} \\
\hline
\keyword{for loop} & Known number of iterations \\
\hline
\keyword{while loop} & Condition-based repetition \\
\hline
\keyword{nested loop} & Loop within loop \\
\hline
\end{tabular}
\end{answertable}

\textbf{Key Concepts:}
\begin{itemize}
    \item \textbf{Repetition statements}: Execute code blocks repeatedly
    \item \textbf{Iteration control}: Different methods of looping
    \item \textbf{Loop variables}: Track iteration progress
\end{itemize}

\begin{mnemonicbox}
\mnemonic{For-While-Nested}
\end{mnemonicbox}
\end{solutionbox}

\questionmarks{Question 3(b OR)}{04}{Differentiate break and continue statements.}

\begin{solutionbox}
\textbf{Difference:}
\begin{answertable}{Break vs Continue}
\begin{tabular}{|l|l|l|}
\hline
\textbf{Aspect} & \textbf{break} & \textbf{continue} \\
\hline
\textbf{Purpose} & Exit loop completely & Skip current iteration \\
\hline
\textbf{Execution} & Jumps out of loop & Jumps to next iteration \\
\hline
\textbf{Usage} & Terminate loop early & Skip specific conditions \\
\hline
\textbf{Effect} & Loop ends & Loop continues \\
\hline
\end{tabular}
\end{answertable}

\textbf{Code Example:}
\begin{lstlisting}[language=Python]
# break example
for i in range(5):
    if i == 3:
        break
    print(i)  # Output: 0, 1, 2

# continue example
for i in range(5):
    if i == 2:
        continue
    print(i)  # Output: 0, 1, 3, 4
\end{lstlisting}

\begin{mnemonicbox}
\mnemonic{Break-Exit, Continue-Skip}
\end{mnemonicbox}
\end{solutionbox}

\questionmarks{Question 3(c OR)}{07}{Write a user-define function which displays all even numbers from 1 to 100.}

\begin{solutionbox}
\textbf{Code:}
\begin{lstlisting}[language=Python]
def display_even_numbers():
    print("Even numbers from 1 to 100:")
    for num in range(2, 101, 2):
        print(num, end=" ")
    print()

# Alternative method
def display_even_alt():
    even_nums = []
    for num in range(1, 101):
        if num % 2 == 0:
            even_nums.append(num)
    print(even_nums)

# Function call
display_even_numbers()
\end{lstlisting}

\textbf{Explanation:}
\begin{itemize}
    \item \textbf{Efficient range}: \code{range(2, 101, 2)} for even numbers
    \item \textbf{Modulus method}: Alternative checking with \% 2 == 0
    \item \textbf{Function design}: Reusable code block
\end{itemize}

\begin{mnemonicbox}
\mnemonic{Range-Step-Even-Display}
\end{mnemonicbox}
\end{solutionbox}

\questionmarks{Question 4(a)}{03}{Define Function. List out various types of Functions available in Python.}

\begin{solutionbox}
\textbf{Function}: Reusable block of code that performs specific task.

\textbf{Function Types:}
\begin{answertable}{Types}
\begin{tabular}{|l|l|}
\hline
\textbf{Function Type} & \textbf{Description} \\
\hline
\keyword{Built-in} & Pre-defined functions (print, len) \\
\hline
\keyword{User-defined} & Created by programmer \\
\hline
\keyword{Lambda} & Anonymous single-line functions \\
\hline
\keyword{Recursive} & Functions calling themselves \\
\hline
\end{tabular}
\end{answertable}

\textbf{Benefits:}
\begin{itemize}
    \item \textbf{Code reusability}: Write once, use many times
    \item \textbf{Modularity}: Breaking complex problems into smaller parts
    \item \textbf{Parameters}: Input values to functions
\end{itemize}

\begin{mnemonicbox}
\mnemonic{Built-User-Lambda-Recursive}
\end{mnemonicbox}
\end{solutionbox}

\questionmarks{Question 4(b)}{04}{Write short note on Scope of a variable.}

\begin{solutionbox}
\textbf{Variable Scope:}
\begin{answertable}{Scope Types}
\begin{tabular}{|l|l|l|}
\hline
\textbf{Scope Type} & \textbf{Description} & \textbf{Example} \\
\hline
\keyword{Local} & Inside function only & Function variables \\
\hline
\keyword{Global} & Throughout program & Module-level variables \\
\hline
\keyword{Built-in} & Python keywords & print, len, type \\
\hline
\end{tabular}
\end{answertable}

\textbf{Code Example:}
\begin{lstlisting}[language=Python]
x = 10  # Global variable

def my_function():
    y = 20  # Local variable
    print(x)  # Access global
    print(y)  # Access local

my_function()
# print(y)  # Error: y not accessible
\end{lstlisting}

\textbf{Key Concepts:}
\begin{itemize}
    \item \textbf{Variable accessibility}: Where variables can be used
    \item \textbf{LEGB rule}: Local, Enclosing, Global, Built-in
\end{itemize}

\begin{mnemonicbox}
\mnemonic{Local-Global-Builtin}
\end{mnemonicbox}
\end{solutionbox}

\questionmarks{Question 4(c)}{07}{Write Python code which asks user for Main string and Substring and checks membership of a Substring in the Main String.}

\begin{solutionbox}
\textbf{Code:}
\begin{lstlisting}[language=Python]
def check_substring():
    main_string = input("Enter main string: ")
    substring = input("Enter substring: ")

    if substring in main_string:
        print(f"'{substring}' found in '{main_string}'")
        print(f"Position: {main_string.find(substring)}")
    else:
        print(f"'{substring}' not found in '{main_string}'")

# Enhanced version with case handling
def check_substring_enhanced():
    main_string = input("Enter main string: ")
    substring = input("Enter substring: ")

    if substring.lower() in main_string.lower():
        print("Substring found (case-insensitive)")
    else:
        print("Substring not found")

check_substring()
\end{lstlisting}

\textbf{Explanation:}
\begin{itemize}
    \item \textbf{User interaction}: \code{input()} for string collection
    \item \textbf{Membership testing}: \code{in} operator usage
    \item \textbf{Case sensitivity}: Optional case handling
\end{itemize}

\begin{mnemonicbox}
\mnemonic{Input-Check-Report-Position}
\end{mnemonicbox}
\end{solutionbox}

\questionmarks{Question 4(a OR)}{03}{What is Local variable and Global variable?}

\begin{solutionbox}
\textbf{Comparison:}
\begin{answertable}{Local vs Global}
\begin{tabular}{|l|l|l|l|}
\hline
\textbf{Variable Type} & \textbf{Scope} & \textbf{Lifetime} & \textbf{Access} \\
\hline
\keyword{Local} & Function only & Function execution & Limited \\
\hline
\keyword{Global} & Entire program & Program execution & Widespread \\
\hline
\end{tabular}
\end{answertable}

\textbf{Example:}
\begin{lstlisting}[language=Python]
global_var = 100  # Global

def function():
    local_var = 50  # Local
    print(global_var)  # Accessible
    print(local_var)   # Accessible

print(global_var)  # Accessible
# print(local_var)  # Error
\end{lstlisting}

\begin{mnemonicbox}
\mnemonic{Local-Limited, Global-Everywhere}
\end{mnemonicbox}
\end{solutionbox}

\questionmarks{Question 4(b OR)}{04}{Explain any four built-in functions of Python.}

\begin{solutionbox}
\textbf{Built-in Functions:}
\begin{answertable}{Functions}
\begin{tabular}{|l|l|l|}
\hline
\textbf{Function} & \textbf{Purpose} & \textbf{Example} \\
\hline
\code{len()} & Returns length & \code{len("hello")} \rightarrow 5 \\
\hline
\code{type()} & Returns data type & \code{type(10)} \rightarrow <class 'int'> \\
\hline
\code{input()} & Gets user input & \code{name = input("Name: ")} \\
\hline
\code{print()} & Displays output & \code{print("Hello")} \\
\hline
\end{tabular}
\end{answertable}

\textbf{Additional Examples:}
\begin{lstlisting}[language=Python]
# len() function
print(len([1, 2, 3, 4]))  # Output: 4

# type() function
print(type(3.14))  # Output: <class 'float'>

# input() function
age = input("Enter age: ")

# print() function
print("Your age is:", age)
\end{lstlisting}

\begin{mnemonicbox}
\mnemonic{Length-Type-Input-Print}
\end{mnemonicbox}
\end{solutionbox}

\questionmarks{Question 4(c OR)}{07}{Write Python code which locates a substring in a given string.}

\begin{solutionbox}
\textbf{Code:}
\begin{lstlisting}[language=Python]
def locate_substring():
    main_string = input("Enter main string: ")
    substring = input("Enter substring to find: ")

    # Method 1: Using find()
    position = main_string.find(substring)
    if position != -1:
        print(f"Found at index: {position}")
    else:
        print("Substring not found")

    # Method 2: Using index() with exception handling
    try:
        position = main_string.index(substring)
        print(f"Located at index: {position}")
    except ValueError:
        print("Substring not found")

    # Method 3: Find all occurrences
    positions = []
    start = 0
    while True:
        pos = main_string.find(substring, start)
        if pos == -1:
            break
        positions.append(pos)
        start = pos + 1

    if positions:
        print(f"All positions: {positions}")

locate_substring()
\end{lstlisting}

\textbf{Key Methods:}
\begin{itemize}
    \item \textbf{find() method}: Returns index or -1
    \item \textbf{index() method}: Returns index or raises exception
    \item \textbf{Multiple occurrences}: Loop to find all positions
\end{itemize}

\begin{mnemonicbox}
\mnemonic{Find-Index-Exception-Multiple}
\end{mnemonicbox}
\end{solutionbox}

\questionmarks{Question 5(a)}{03}{Define String. List out various string operations.}

\begin{solutionbox}
\textbf{String}: Sequence of characters enclosed in quotes.

\textbf{Operations:}
\begin{answertable}{String Operations}
\begin{tabular}{|l|l|l|}
\hline
\textbf{Operation} & \textbf{Method} & \textbf{Example} \\
\hline
Concatenation & + & "Hello" + "World" \\
\hline
Repetition & * & "Hi" * 3 \\
\hline
Slicing & [start:end] & "Hello"[1:4] \\
\hline
Length & \code{len()} & \code{len("Hello")} \\
\hline
Case & \code{upper()}, \code{lower()} & "hello".upper() \\
\hline
\end{tabular}
\end{answertable}

\textbf{Characteristics:}
\begin{itemize}
    \item \textbf{Immutable}: Strings cannot be changed after creation
    \item \textbf{Indexing}: Access individual characters
    \item \textbf{Methods}: Built-in functions for manipulation
\end{itemize}

\begin{mnemonicbox}
\mnemonic{Concat-Repeat-Slice-Length-Case}
\end{mnemonicbox}
\end{solutionbox}

\questionmarks{Question 5(b)}{04}{How can we identify whether an element is a member of a list or not? Explain with a suitable example.}

\begin{solutionbox}
\textbf{Methods:}
\begin{answertable}{Membership Check}
\begin{tabular}{|l|l|l|}
\hline
\textbf{Method} & \textbf{Operator} & \textbf{Returns} \\
\hline
\code{in} & element in list & True/False \\
\hline
\code{not in} & element not in list & True/False \\
\hline
\code{count()} & list.count(element) & Number of occurrences \\
\hline
\end{tabular}
\end{answertable}

\textbf{Example:}
\begin{lstlisting}[language=Python]
fruits = ["apple", "banana", "orange", "mango"]

# Using 'in' operator
if "apple" in fruits:
    print("Apple is available")

# Using 'not in' operator
if "grapes" not in fruits:
    print("Grapes not available")

# Using count() method
count = fruits.count("apple")
if count > 0:
    print(f"Apple found {count} times")
\end{lstlisting}

\begin{mnemonicbox}
\mnemonic{In-NotIn-Count for membership}
\end{mnemonicbox}
\end{solutionbox}

\questionmarks{Question 5(c)}{07}{Write Python code that replaces a substring with another substring of a given string. Consider the given string as 'Welcome to GTU' and replace the substring 'GTU' with 'Gujarat Technological University'.}

\begin{solutionbox}
\textbf{Code:}
\begin{lstlisting}[language=Python]
def replace_substring():
    # Given string
    original = "Welcome to GTU"
    old_substring = "GTU"
    new_substring = "Gujarat Technological University"

    # Method 1: Using replace()
    result1 = original.replace(old_substring, new_substring)
    print(f"Original: {original}")
    print(f"Modified: {result1}")

    # Method 2: Manual replacement
    if old_substring in original:
        index = original.find(old_substring)
        result2 = original[:index] + new_substring + original[index + len(old_substring):]
        print(f"Manual method: {result2}")

    # Method 3: Replace all occurrences
    test_string = "GTU offers GTU degree from GTU"
    result3 = test_string.replace("GTU", "Gujarat Technological University")
    print(f"Multiple replacements: {result3}")

replace_substring()
\end{lstlisting}

\textbf{Output:}
\begin{verbatim}
Original: Welcome to GTU
Modified: Welcome to Gujarat Technological University
\end{verbatim}

\textbf{Key Points:}
\begin{itemize}
    \item \textbf{replace() method}: Built-in string function
    \item \textbf{Slicing method}: Manual string manipulation
    \item \textbf{All occurrences}: Replaces every instance
\end{itemize}

\begin{mnemonicbox}
\mnemonic{Find-Replace-Slice-All}
\end{mnemonicbox}
\end{solutionbox}

\questionmarks{Question 5(a OR)}{03}{Define List. List out various list operations.}

\begin{solutionbox}
\textbf{List}: Ordered collection of items that can be modified.

\textbf{Operations:}
\begin{answertable}{List Operations}
\begin{tabular}{|l|l|l|}
\hline
\textbf{Operation} & \textbf{Method} & \textbf{Example} \\
\hline
Add & \code{append()}, \code{insert()} & \code{list.append(item)} \\
\hline
Remove & \code{remove()}, \code{pop()} & \code{list.remove(item)} \\
\hline
Access & \code{[index]} & \code{list[0]} \\
\hline
Slice & \code{[start:end]} & \code{list[1:3]} \\
\hline
Sort & \code{sort()} & \code{list.sort()} \\
\hline
\end{tabular}
\end{answertable}

\textbf{Features:}
\begin{itemize}
    \item \textbf{Mutable}: Lists can be changed after creation
    \item \textbf{Indexed}: Elements accessed by position
    \item \textbf{Dynamic}: Size can grow or shrink
\end{itemize}

\begin{mnemonicbox}
\mnemonic{Add-Remove-Access-Slice-Sort}
\end{mnemonicbox}
\end{solutionbox}

\questionmarks{Question 5(b OR)}{04}{Write short note on String Slicing. Explain with suitable example.}

\begin{solutionbox}
\textbf{String Slicing}: Extracting parts of string using \code{[start:end:step]}.

\textbf{Syntax:}
\begin{answertable}{Slicing Syntax}
\begin{tabular}{|l|l|l|}
\hline
\textbf{Syntax} & \textbf{Description} & \textbf{Example} \\
\hline
\code{[start:]} & From start to end & "Hello"[1:] \rightarrow "ello" \\
\hline
\code{[:end]} & From beginning to end & "Hello"[:3] \rightarrow "Hel" \\
\hline
\code{[start:end]} & Specific range & "Hello"[1:4] \rightarrow "ell" \\
\hline
\code{[::-1]} & Reverse string & "Hello"[::-1] \rightarrow "olleH" \\
\hline
\end{tabular}
\end{answertable}

\textbf{Example:}
\begin{lstlisting}[language=Python]
text = "Python Programming"

print(text[0:6])    # "Python"
print(text[7:])     # "Programming"
print(text[:6])     # "Python"
print(text[::2])    # "Pto rgamn"
print(text[::-1])   # "gnimmargorP nohtyP"
\end{lstlisting}

\begin{mnemonicbox}
\mnemonic{Start-End-Step-Reverse}
\end{mnemonicbox}
\end{solutionbox}

\end{document}
