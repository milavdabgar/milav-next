\documentclass{article}

% content/resources/templates/preamble.tex
\usepackage[margin=0.6in]{geometry}
\author{Milav Dabgar}
\usepackage{amsmath,amssymb,amsthm}
\usepackage{booktabs}
\usepackage{multirow}
\usepackage{xcolor}
\usepackage{tcolorbox}
\tcbuselibrary{breakable,skins}
\usepackage[colorlinks=true,linkcolor=blue]{hyperref}
\usepackage{titlesec}
\usepackage{enumitem}
\usepackage{tikz}
\usepackage{pgfplots}
\usepackage{circuitikz}
\usepackage[version=4]{mhchem}
\usepackage{longtable}
\usepackage{array}
\usepackage{float}
\usepackage{caption}
\usepackage{listings}

\lstset{
  basicstyle=\small\ttfamily,
  breaklines=true,
  breakatwhitespace=false,
  postbreak=\mbox{\textcolor{red}{$\hookrightarrow$}\space},
  float=false,
  numbers=left,
  numberstyle=\tiny\color{gray},
  numbersep=10pt,
  xleftmargin=2em,
  keywordstyle=\color{blue},
  commentstyle=\color{green!60!black},
  stringstyle=\color{purple},
  backgroundcolor=\color{gray!5},
  showstringspaces=false,
  tabsize=2,
  captionpos=b,
  keepspaces=true,
  columns=flexible
}

\pgfplotsset{compat=1.18}
\usetikzlibrary{shapes,arrows,positioning,calc,patterns,decorations.pathmorphing,decorations.markings,arrows.meta}

% Color scheme
\definecolor{headcolor}{RGB}{0,102,204}
\definecolor{keycolor}{RGB}{220,20,60}
\definecolor{solutioncolor}{RGB}{34,139,34}
\definecolor{mnemoniccolor}{RGB}{148,0,211}
\definecolor{codecolor}{RGB}{0,0,100}

% Spacing
\setlength{\parskip}{3pt}
\setlist[itemize]{nosep}
\setlist[enumerate]{nosep}

% Title formatting
\titleformat{\section}{\Large\bfseries\color{headcolor}}{\thesection}{1em}{}
\titleformat{\subsection}{\large\bfseries\color{headcolor}}{\thesubsection}{1em}{}

% Pandoc tightlist compatibility
\providecommand{\tightlist}{%
  \setlength{\itemsep}{0pt}\setlength{\parskip}{0pt}}

% Pandoc longtable compatibility
\newcounter{none}
\def\thenone{}


% content/resources/templates/gujarati-boxes.tex
\usepackage{fontspec}
\usepackage{polyglossia}

% Set Gujarati as main language (document is primarily in Gujarati)
% Note: gloss-gujarati.ldf doesn't exist in polyglossia, but it will use hyphenation patterns
\setdefaultlanguage{gujarati}
\setotherlanguage{english}

% Configure Gujarati font properly
% Use Language=Default to prevent polyglossia from trying to add language-specific features
% that don't exist for Gujarati, which causes "empty feature" warnings
\newfontfamily\gujaratifont[Script=Gujarati,AutoFakeBold=2.5,AutoFakeSlant=0.3]{Noto Sans Gujarati}
\setmainfont[Script=Gujarati,AutoFakeBold=2.5,AutoFakeSlant=0.3]{Noto Sans Gujarati}
% Use Noto Sans Gujarati for monospace to support Gujarati in text
\setmonofont[Scale=0.9]{Noto Sans Gujarati}

% Configure English to use the same font
\newfontfamily\englishfont[Script=Gujarati,AutoFakeBold=2.5,AutoFakeSlant=0.3]{Noto Sans Gujarati}

% Translations for polyglossia
\gappto\captionsgujarati{
  \renewcommand{\tablename}{કોષ્ટક}
  \renewcommand{\figurename}{આકૃતિ}
}

% Helper for TikZ nodes to ensure Gujarati font
\newcommand{\gu}[1]{{\gujaratifont #1}}

% Custom environments
\newtcolorbox{solutionbox}{
    breakable,
    enhanced,
    colback=solutioncolor!5!white,
    colframe=solutioncolor!75!black,
    fonttitle=\bfseries,
    title=જવાબ
}

\newtcolorbox{solutionboxnobreak}{
 colback=solutioncolor!5!white,
 colframe=solutioncolor!75!black,
 fonttitle=\bfseries,
 title=જવાબ
}

\newtcolorbox{keyformula}{
 breakable,
 enhanced,
 colback=keycolor!5!white,
 colframe=keycolor!75!black,
 fonttitle=\bfseries,
 title=રાસાયણિક સમીકરણ/સૂત્ર
}

\newtcolorbox{mnemonicbox}{
 breakable,
 enhanced,
 colback=mnemoniccolor!5!white,
 colframe=mnemoniccolor!75!black,
 fonttitle=\bfseries,
 title=મેમરી ટ્રીક
}


% Custom commands for GTU solutions
% This file defines semantic commands for consistent formatting

% Question command with automatic formatting
\newcommand{\question}[2]{%
  \section*{Question #1}%
  \textbf{#2}%
}

% OR question variant
\newcommand{\questionor}[2]{%
  \section*{Question #1 OR}%
  \textbf{#2}%
}

% Proper table environment with caption
\newenvironment{answertable}[1]{%
  \begin{table}[htbp]
  \centering
  \caption{#1}
}{%
  \end{table}
}

% Proper figure environment for diagrams
\newenvironment{answerdiagram}[1]{%
  \begin{figure}[htbp]
  \centering
  \caption{#1}
}{%
  \end{figure}
}

% Semantic markup for key terms
\newcommand{\keyword}[1]{\textbf{#1}}
\newcommand{\code}[1]{\texttt{#1}}
\newcommand{\classname}[1]{\texttt{#1}}
\newcommand{\methodname}[1]{\texttt{#1}}

% Proper quotation marks
\newcommand{\mnemonic}[1]{``#1''}


\title{Python Programming (4311601) - Winter 2024 Solution}
\date{January 13, 2025}

\begin{document}
\maketitle

\questionmarks{પ્રશ્ન 1(અ)}{03}{પ્રોબ્લમ સોલવિંગ, અલ્ગોરિધમ અને સ્યુડો કોડ વ્યાખ્યાયિત કરો.}

\begin{solutionbox}
\textbf{વ્યાખ્યાઓ:}
\begin{answertable}{Core Concepts}
\begin{tabular}{|l|l|}
\hline
\textbf{શબ્દ} & \textbf{વ્યાખ્યા} \\
\hline
\keyword{પ્રોબ્લમ સોલવિંગ} & તર્કસંગત વિચારસરણી વાપરીને જટિલ સમસ્યાઓનાં ઉકેલ શોધવાની પદ્ધતિ \\
\hline
\keyword{અલ્ગોરિધમ} & મર્યાદિત ઓપરેશન સાથે સમસ્યા ઉકેલવાની પગલું-દર-પગલું પ્રક્રિયા \\
\hline
\keyword{સ્યુડો કોડ} & સામાન્ય અંગ્રેજી જેવા syntax નો ઉપયોગ કરીને program logic નું અનૌપચારિક વર્ણન \\
\hline
\end{tabular}
\end{answertable}

\textbf{મુખ્ય મુદ્દાઓ:}
\begin{itemize}
    \item \textbf{પ્રોબ્લમ સોલવિંગ}: જટિલ સમસ્યાઓને વ્યવસ્થિત પગલાઓમાં વહેંચવું
    \item \textbf{અલ્ગોરિધમ}: મર્યાદિત, નિશ્ચિત, અસરકારક અને યોગ્ય આઉટપુટ આપતું હોવું જોઈએ
    \item \textbf{સ્યુડો કોડ}: માનવ ભાષા અને programming કોડ વચ્ચેનો સેતુ
\end{itemize}

\begin{mnemonicbox}
\mnemonic{PAP - Problem, Algorithm, Pseudo}
\end{mnemonicbox}
\end{solutionbox}

\questionmarks{પ્રશ્ન 1(બ)}{04}{ફ્લોચાર્ટના જુદા જુદા સિમ્બોલ સમજાવો. બે નંબર માંથી મહત્તમ નંબર શોધતો ફ્લોચાર્ટ ડિઝાઇન કરો.}

\begin{solutionbox}
\textbf{ફ્લોચાર્ટ સિમ્બોલ:}
\begin{answertable}{Symbols}
\begin{tabular}{|l|c|l|}
\hline
\textbf{સિમ્બોલ} & \textbf{આકાર} & \textbf{હેતુ} \\
\hline
\keyword{અંડાકાર} & ⬭ & શરૂઆત/અંત \\
\hline
\keyword{લંબચોરસ} & ▭ & પ્રક્રિયા/ક્રિયા \\
\hline
\keyword{હીરો} & ◊ & નિર્ણય \\
\hline
\keyword{સમાંતર ચતુષ્કોણ} & ▱ & ઇનપુટ/આઉટપુટ \\
\hline
\end{tabular}
\end{answertable}

\textbf{બે નંબરના મહત્તમ માટે ફ્લોચાર્ટ:}
\begin{center}
\begin{tikzpicture}[gtu flow]
    \node[gtustart] (start) {શરૂઆત};
    \node[gtuio, below=of start] (input) {ઇનપુટ A, B};
    \node[gtudecision, below=of input] (dec) {A > B?};
    \node[gtuprocess, below left=of dec, xshift=-1cm] (maxA) {Max = A};
    \node[gtuprocess, below right=of dec, xshift=1cm] (maxB) {Max = B};
    \node[gtuio, below=of dec, yshift=-3cm] (disp) {Max દર્શાવો};
    \node[gtustop, below=of disp] (end) {અંત};

    \path [gtuarrow] (start) -- (input);
    \path [gtuarrow] (input) -- (dec);
    \path [gtuarrow] (dec) -| node[above left] {હા} (maxA);
    \path [gtuarrow] (dec) -| node[above right] {ના} (maxB);
    \path [gtuarrow] (maxA) |- (disp);
    \path [gtuarrow] (maxB) |- (disp);
    \path [gtuarrow] (disp) -- (end);
\end{tikzpicture}
\end{center}

\textbf{સમજૂતી:}
\begin{itemize}
    \item \textbf{શરૂઆત/અંત}: પ્રવેશ અને બહાર નીકળવાના બિંદુઓ
    \item \textbf{ઇનપુટ/આઉટપુટ}: ડેટા ફ્લો ઓપરેશન્સ
    \item \textbf{નિર્ણય}: શરતી branching
    \item \textbf{પ્રક્રિયા}: ગણતરીના પગલાં
\end{itemize}

\begin{mnemonicbox}
\mnemonic{SIPO - Start, Input, Process, Output}
\end{mnemonicbox}
\end{solutionbox}

\questionmarks{પ્રશ્ન 1(ક)}{07}{પાયથોનના વિવિધ એરિથમેટિક ઓપરેટરોની યાદી બનાવો. વિવિધ એરિથમેટિક ઓપરેશન્સ માટેનો Python કોડ લખો.}

\begin{solutionbox}
\textbf{એરિથમેટિક ઓપરેટરો:}
\begin{answertable}{Arithmetic Operators}
\begin{tabular}{|l|c|l|l|}
\hline
\textbf{ઓપરેટર} & \textbf{સિમ્બોલ} & \textbf{ઉદાહરણ} & \textbf{પરિણામ} \\
\hline
ઉમેરો & + & \code{5 + 3} & 8 \\
\hline
બાદબાકી & - & \code{5 - 3} & 2 \\
\hline
ગુણાકાર & * & \code{5 * 3} & 15 \\
\hline
ભાગાકાર & / & \code{5 / 3} & 1.667 \\
\hline
ફ્લોર ડિવિઝન & // & \code{5 // 3} & 1 \\
\hline
મોડ્યુલસ & \% & \code{5 \% 3} & 2 \\
\hline
ઘાત & ** & \code{5 ** 3} & 125 \\
\hline
\end{tabular}
\end{answertable}

\textbf{કોડ:}
\begin{lstlisting}[language=Python]
a = 10
b = 3
print(f"Addition: {a + b}")
print(f"Subtraction: {a - b}")
print(f"Multiplication: {a * b}")
print(f"Division: {a / b}")
print(f"Floor Division: {a // b}")
print(f"Modulus: {a % b}")
print(f"Power: {a ** b}")
\end{lstlisting}

\begin{mnemonicbox}
\mnemonic{Add-Sub-Mul-Div-Floor-Mod-Pow}
\end{mnemonicbox}
\end{solutionbox}

\questionmarks{પ્રશ્ન 1(ક OR)}{07}{પાયથોનના વિવિધ કંપેરિઝન ઓપરેટરોની યાદી બનાવો. વિવિધ કંપેરિઝન ઓપરેશન્સ માટેનો Python કોડ લખો.}

\begin{solutionbox}
\textbf{કંપેરિઝન ઓપરેટરો:}
\begin{answertable}{Comparison Operators}
\begin{tabular}{|l|c|l|l|}
\hline
\textbf{ઓપરેટર} & \textbf{સિમ્બોલ} & \textbf{હેતુ} & \textbf{ઉદાહરણ} \\
\hline
સમાન & == & સમાનતા ચકાસો & \code{5 == 3} \rightarrow False \\
\hline
અસમાન & != & અસમાનતા ચકાસો & \code{5 != 3} \rightarrow True \\
\hline
મોટું & > & મોટું ચકાસો & \code{5 > 3} \rightarrow True \\
\hline
નાનું & < & નાનું ચકાસો & \code{5 < 3} \rightarrow False \\
\hline
મોટું સમાન & >= & મોટું/સમાન ચકાસો & \code{5 >= 3} \rightarrow True \\
\hline
નાનું સમાન & <= & નાનું/સમાન ચકાસો & \code{5 <= 3} \rightarrow False \\
\hline
\end{tabular}
\end{answertable}

\textbf{કોડ:}
\begin{lstlisting}[language=Python]
x = 8
y = 5
print(f"Equal: {x == y}")
print(f"Not Equal: {x != y}")
print(f"Greater: {x > y}")
print(f"Less: {x < y}")
print(f"Greater Equal: {x >= y}")
print(f"Less Equal: {x <= y}")
\end{lstlisting}

\begin{mnemonicbox}
\mnemonic{Equal-Not-Greater-Less-GreaterEqual-LessEqual}
\end{mnemonicbox}
\end{solutionbox}

\questionmarks{પ્રશ્ન 2(અ)}{03}{મેમ્બરશિપ ઓપરેટર્સ ઉપર ટૂંક નોંધ લખો.}

\begin{solutionbox}
\textbf{મેમ્બરશિપ ઓપરેટર્સ:}
\begin{answertable}{Membership Operators}
\begin{tabular}{|l|l|l|}
\hline
\textbf{ઓપરેટર} & \textbf{હેતુ} & \textbf{ઉદાહરણ} \\
\hline
\code{in} & એલિમેન્ટ અસ્તિત્વ ચકાસો & \code{'a' in 'apple'} \rightarrow True \\
\hline
\code{not in} & એલિમેન્ટ અનસ્તિત્વ ચકાસો & \code{'z' not in 'apple'} \rightarrow True \\
\hline
\end{tabular}
\end{answertable}

\textbf{મુખ્ય મુદ્દાઓ:}
\begin{itemize}
    \item \textbf{in ઓપરેટર}: જો એલિમેન્ટ sequence માં મળે તો True આપે
    \item \textbf{not in ઓપરેટર}: જો એલિમેન્ટ sequence માં ન મળે તો True આપે
    \item \textbf{ઉપયોગ}: Lists, strings, tuples, dictionaries માં
\end{itemize}

\begin{mnemonicbox}
\mnemonic{In-Not-In for membership testing}
\end{mnemonicbox}
\end{solutionbox}

\questionmarks{પ્રશ્ન 2(બ)}{04}{પાયથોન વ્યાખ્યાયિત કરો. પાયથોન પ્રોગ્રામિંગની વિવિધ એપ્લિકેશનો લખો.}

\begin{solutionbox}
\textbf{પાયથોન વ્યાખ્યા}: સરળતા અને વાંચનીયતા માટે જાણીતી high-level, interpreted programming language.

\textbf{એપ્લિકેશનો:}
\begin{answertable}{Applications}
\begin{tabular}{|l|l|}
\hline
\textbf{એપ્લિકેશન ક્ષેત્ર} & \textbf{ઉદાહરણો} \\
\hline
\keyword{વેબ ડેવલપમેન્ટ} & Django, Flask frameworks \\
\hline
\keyword{ડેટા સાયન્સ} & NumPy, Pandas, Matplotlib \\
\hline
\keyword{AI/ML} & TensorFlow, Scikit-learn \\
\hline
\keyword{ડેસ્કટોપ એપ્સ} & Tkinter, PyQt \\
\hline
\keyword{ગેમ ડેવલપમેન્ટ} & Pygame library \\
\hline
\end{tabular}
\end{answertable}

\textbf{વિશેષતાઓ:}
\begin{itemize}
    \item \textbf{Interpreted}: compilation ની જરૂર નથી
    \item \textbf{Cross-platform}: બહુવિધ OS પર ચાલે છે
    \item \textbf{વિશાળ libraries}: વ્યાપક standard library
\end{itemize}

\begin{mnemonicbox}
\mnemonic{Web-Data-AI-Desktop-Games}
\end{mnemonicbox}
\end{solutionbox}

\questionmarks{પ્રશ્ન 2(ક)}{07}{પાયથોન પ્રોગ્રામ લખો જે નીચેની વિગતોનો ઉપયોગ કરીને વીજળી બિલની ગણતરી કરે છે.}

\begin{solutionbox}
\textbf{દરોનું ટેબલ:}
\begin{answertable}{Electricity Rates}
\begin{tabular}{|l|l|}
\hline
\textbf{યુનિટ રેન્જ} & \textbf{દર પ્રતિ યુનિટ} \\
\hline
$\le$ 100 & રૂ 5.00 \\
\hline
101-200 & રૂ 7.50 \\
\hline
201-300 & રૂ 10.00 \\
\hline
$\ge$ 301 & રૂ 15.00 \\
\hline
\end{tabular}
\end{answertable}

\textbf{કોડ:}
\begin{lstlisting}[language=Python]
units = int(input("Enter consumed units: "))

if units <= 100:
    bill = units * 5.00
elif units <= 200:
    bill = units * 7.50
elif units <= 300:
    bill = units * 10.00
else:
    bill = units * 15.00

print(f"Total Bill: Rs {bill}")
\end{lstlisting}

\textbf{સમજૂતી:}
\begin{itemize}
    \item \textbf{શરતી તર્ક}: if-elif-else structure
    \item \textbf{દર ગણતરી}: યુનિટ slabs આધારિત
    \item \textbf{યુઝર ઇનપુટ}: interactive billing system
\end{itemize}

\begin{mnemonicbox}
\mnemonic{Input-Check-Calculate-Display}
\end{mnemonicbox}
\end{solutionbox}

\questionmarks{પ્રશ્ન 2(અ OR)}{03}{આઇડેન્ટિટી ઓપરેટર્સ ઉપર ટૂંક નોંધ લખો.}

\begin{solutionbox}
\textbf{આઇડેન્ટિટી ઓપરેટર્સ:}
\begin{answertable}{Identity Operators}
\begin{tabular}{|l|l|l|}
\hline
\textbf{ઓપરેટર} & \textbf{હેતુ} & \textbf{ઉદાહરણ} \\
\hline
\code{is} & સમાન ઓબ્જેક્ટ ચકાસો & \code{a is b} \\
\hline
\code{is not} & જુદા ઓબ્જેક્ટ ચકાસો & \code{a is not b} \\
\hline
\end{tabular}
\end{answertable}

\textbf{મુખ્ય મુદ્દાઓ:}
\begin{itemize}
    \item \textbf{is ઓપરેટર}: ઓબ્જેક્ટ identity સરખાવે, values નહીં
    \item \textbf{is not ઓપરેટર}: ઓબ્જેક્ટ્સ જુદા છે કે નહીં ચકાસે
    \item \textbf{મેમરી સરખામણી}: સમાન મેમરી સ્થાન ચકાસે
\end{itemize}

\begin{mnemonicbox}
\mnemonic{Is-IsNot for object identity}
\end{mnemonicbox}
\end{solutionbox}

\questionmarks{પ્રશ્ન 2(બ OR)}{04}{પાયથોનમાં ઇન્ડેન્ટેશન શું છે? પાયથોનની વિવિધ વિશેષતાઓ સમજાવો.}

\begin{solutionbox}
\textbf{ઇન્ડેન્ટેશન}: કોડ બ્લોક્સ વ્યાખ્યાયિત કરવા માટે લાઇનની શરૂઆતમાં whitespace.

\textbf{વિશેષતાઓ:}
\begin{answertable}{Python Features}
\begin{tabular}{|l|l|}
\hline
\textbf{વિશેષતા} & \textbf{વર્ણન} \\
\hline
\keyword{સરળ Syntax} & વાંચવા અને લખવામાં સરળ \\
\hline
\keyword{Interpreted} & compilation step નથી \\
\hline
\keyword{Object-Oriented} & OOP concepts સપોર્ટ કરે \\
\hline
\keyword{Cross-Platform} & બહુવિધ OS પર ચાલે \\
\hline
\keyword{વિશાળ Library} & વ્યાપક standard library \\
\hline
\end{tabular}
\end{answertable}

\textbf{મહત્વ:}
\begin{itemize}
    \item \textbf{ઇન્ડેન્ટેશન}: curly braces {} ને બદલે છે
    \item \textbf{સુસંગત}: સામાન્ય રીતે પ્રતિ level 4 spaces
    \item \textbf{ફરજિયાત}: કોડ માળખું બનાવે છે
\end{itemize}

\begin{mnemonicbox}
\mnemonic{Simple-Interpreted-Object-Cross-Large}
\end{mnemonicbox}
\end{solutionbox}

\questionmarks{પ્રશ્ન 2(ક OR)}{07}{પાયથોન પ્રોગ્રામ લખો જે નીચેની વિગતોનો ઉપયોગ કરીને વિદ્યાર્થીના વર્ગ/ગ્રેડની ગણતરી કરતો પાયથોન પ્રોગ્રામ લખો.}

\begin{solutionbox}
\textbf{ગ્રેડિંગ ટેબલ:}
\begin{answertable}{Grading Scheme}
\begin{tabular}{|l|l|}
\hline
\textbf{ટકાવારી} & \textbf{ગ્રેડ} \\
\hline
$\ge$ 70 & ડિસ્ટિંક્શન \\
\hline
60-69 & ફર્સ્ટ ક્લાસ \\
\hline
50-59 & સેકન્ડ ક્લાસ \\
\hline
35-49 & પાસ ક્લાસ \\
\hline
< 35 & નિષ્ફળ \\
\hline
\end{tabular}
\end{answertable}

\textbf{કોડ:}
\begin{lstlisting}[language=Python]
percentage = float(input("Enter percentage: "))

if percentage >= 70:
    grade = "Distinction"
elif percentage >= 60:
    grade = "First Class"
elif percentage >= 50:
    grade = "Second Class"
elif percentage >= 35:
    grade = "Pass Class"
else:
    grade = "Fail"

print(f"Grade: {grade}")
\end{lstlisting}

\textbf{સમજૂતી:}
\begin{itemize}
    \item \textbf{બહુવિધ શરતો}: Nested if-elif structure
    \item \textbf{ગ્રેડ નિર્ધારણ}: ટકાવારી ranges આધારિત
    \item \textbf{Float ઇનપુટ}: દશાંશ ટકાવારી handle કરે
\end{itemize}

\begin{mnemonicbox}
\mnemonic{Distinction-First-Second-Pass-Fail}
\end{mnemonicbox}
\end{solutionbox}

\questionmarks{પ્રશ્ન 3(અ)}{03}{સિલેક્શન કંટ્રોલ સ્ટેટમેન્ટ શું છે? તેની યાદી બનાવો.}

\begin{solutionbox}
\textbf{સિલેક્શન કંટ્રોલ સ્ટેટમેન્ટ:}
\begin{answertable}{Selection Statements}
\begin{tabular}{|l|l|}
\hline
\textbf{સ્ટેટમેન્ટ પ્રકાર} & \textbf{હેતુ} \\
\hline
\keyword{if} & એક શરત ચકાસણી \\
\hline
\keyword{if-else} & બે-માર્ગી branching \\
\hline
\keyword{if-elif-else} & બહુ-માર્ગી branching \\
\hline
\keyword{nested if} & શરતોની અંદર શરતો \\
\hline
\end{tabular}
\end{answertable}

\textbf{મુખ્ય ખ્યાલો:}
\begin{itemize}
    \item \textbf{Selection statements}: શરતો આધારે program flow control કરે
    \item \textbf{Boolean evaluation}: True/False logic વાપરે
    \item \textbf{Branching}: execution ના જુદા રસ્તાઓ
\end{itemize}

\begin{mnemonicbox}
\mnemonic{If-IfElse-IfElif-Nested}
\end{mnemonicbox}
\end{solutionbox}

\questionmarks{પ્રશ્ન 3(બ)}{04}{નેસ્ટેડ લૂપ ઉપર ટૂંક નોંધ લખો.}

\begin{solutionbox}
\textbf{નેસ્ટેડ લૂપ:}
\begin{answertable}{Loop Structure}
\begin{tabular}{|l|l|}
\hline
\textbf{લૂપ પ્રકાર} & \textbf{માળખું} \\
\hline
\keyword{બાહ્ય લૂપ} & iterations control કરે \\
\hline
\keyword{આંતરિક લૂપ} & દરેક બાહ્ય iteration માટે સંપૂર્ણ execute થાય \\
\hline
\textbf{કુલ Iterations} & બાહ્ય $\times$ આંતરિક \\
\hline
\end{tabular}
\end{answertable}

\textbf{મુખ્ય મુદ્દાઓ:}
\begin{itemize}
    \item \textbf{Nested માળખું}: બીજા લૂપની અંદર લૂપ
    \item \textbf{સંપૂર્ણ execution}: આંતરિક લૂપ પૂરું થાય પછી બાહ્ય આગળ વધે
    \item \textbf{Pattern creation}: 2D structures માટે ઉપયોગી
\end{itemize}

\textbf{કોડ ઉદાહરણ:}
\begin{lstlisting}[language=Python]
for i in range(3):
    for j in range(2):
        print(f"i={i}, j={j}")
\end{lstlisting}

\begin{mnemonicbox}
\mnemonic{Outer-Inner-Complete-Pattern}
\end{mnemonicbox}
\end{solutionbox}

\questionmarks{પ્રશ્ન 3(ક)}{07}{યુઝર ડિફાઇન ફંક્શન લખો જે 1 થી 100 સુધીની બધી સંખ્યાઓ દર્શાવે, જે 4 થી વિભાજ્ય છે.}

\begin{solutionbox}
\textbf{કોડ:}
\begin{lstlisting}[language=Python]
def display_divisible_by_4():
    print("Numbers divisible by 4 from 1 to 100:")
    for num in range(1, 101):
        if num % 4 == 0:
            print(num, end=" ")
    print()

# Function call
display_divisible_by_4()
\end{lstlisting}

\textbf{Return સાથે વિકલ્પ:}
\begin{lstlisting}[language=Python]
def get_divisible_by_4():
    return [num for num in range(1, 101) if num % 4 == 0]

result = get_divisible_by_4()
print(result)
\end{lstlisting}

\textbf{મુખ્ય ખ્યાલો:}
\begin{itemize}
    \item \textbf{ફંક્શન વ્યાખ્યા}: def keyword નો ઉપયોગ
    \item \textbf{Range ફંક્શન}: 1 થી 100 iteration
    \item \textbf{Modulus ચકાસણી}: num \% 4 == 0 શરત
    \item \textbf{List comprehension}: વૈકલ્પિક અભિગમ
\end{itemize}

\begin{mnemonicbox}
\mnemonic{Define-Range-Check-Display}
\end{mnemonicbox}
\end{solutionbox}

\questionmarks{પ્રશ્ન 3(અ OR)}{03}{રિપીટેશન કંટ્રોલ સ્ટેટમેન્ટ શું છે? તેની યાદી બનાવો.}

\begin{solutionbox}
\textbf{રિપીટેશન કંટ્રોલ સ્ટેટમેન્ટ:}
\begin{answertable}{Loops}
\begin{tabular}{|l|l|}
\hline
\textbf{સ્ટેટમેન્ટ પ્રકાર} & \textbf{હેતુ} \\
\hline
\keyword{for loop} & જાણીતી સંખ્યાના iterations \\
\hline
\keyword{while loop} & શરત આધારિત repetition \\
\hline
\keyword{nested loop} & લૂપની અંદર લૂપ \\
\hline
\end{tabular}
\end{answertable}

\textbf{મુખ્ય ખ્યાલો:}
\begin{itemize}
    \item \textbf{Repetition statements}: કોડ બ્લોક્સ વારંવાર execute કરે
    \item \textbf{Iteration control}: looping ની જુદી પદ્ધતિઓ
    \item \textbf{Loop variables}: iteration progress track કરે
\end{itemize}

\begin{mnemonicbox}
\mnemonic{For-While-Nested}
\end{mnemonicbox}
\end{solutionbox}

\questionmarks{પ્રશ્ન 3(બ OR)}{04}{break અને continue સ્ટેટમેન્ટ વચ્ચેનો તફાવત આપો.}

\begin{solutionbox}
\textbf{તફાવત:}
\begin{answertable}{Break vs Continue}
\begin{tabular}{|l|l|l|}
\hline
\textbf{પાસું} & \textbf{break} & \textbf{continue} \\
\hline
\textbf{હેતુ} & લૂપ સંપૂર્ણ બહાર નીકળો & વર્તમાન iteration છોડો \\
\hline
\textbf{Execution} & લૂપમાંથી બહાર jump કરે & આગલા iteration પર jump કરે \\
\hline
\textbf{ઉપયોગ} & લૂપ જલ્દી સમાપ્ત કરો & ખાસ શરતો છોડો \\
\hline
\textbf{અસર} & લૂપ સમાપ્ત થાય & લૂપ ચાલુ રહે \\
\hline
\end{tabular}
\end{answertable}

\textbf{કોડ ઉદાહરણ:}
\begin{lstlisting}[language=Python]
# break example
for i in range(5):
    if i == 3:
        break
    print(i)  # Output: 0, 1, 2

# continue example
for i in range(5):
    if i == 2:
        continue
    print(i)  # Output: 0, 1, 3, 4
\end{lstlisting}

\begin{mnemonicbox}
\mnemonic{Break-Exit, Continue-Skip}
\end{mnemonicbox}
\end{solutionbox}

\questionmarks{પ્રશ્ન 3(ક OR)}{07}{યુઝર ડિફાઇન ફંક્શન લખો જે 1 થી 100 સુધીની બધી બેકી સંખ્યાઓ દર્શાવે.}

\begin{solutionbox}
\textbf{કોડ:}
\begin{lstlisting}[language=Python]
def display_even_numbers():
    print("Even numbers from 1 to 100:")
    for num in range(2, 101, 2):
        print(num, end=" ")
    print()

# Alternative method
def display_even_alt():
    even_nums = []
    for num in range(1, 101):
        if num % 2 == 0:
            even_nums.append(num)
    print(even_nums)

# Function call
display_even_numbers()
\end{lstlisting}

\textbf{સમજૂતી:}
\begin{itemize}
    \item \textbf{કાર્યક્ષમ range}: બેકી સંખ્યાઓ માટે \code{range(2, 101, 2)}
    \item \textbf{Modulus પદ્ધતિ}: \% 2 == 0 સાથે વૈકલ્પિક ચકાસણી
    \item \textbf{ફંક્શન ડિઝાઇન}: પુનઃઉપયોગી કોડ બ્લોક
\end{itemize}

\begin{mnemonicbox}
\mnemonic{Range-Step-Even-Display}
\end{mnemonicbox}
\end{solutionbox}

\questionmarks{પ્રશ્ન 4(અ)}{03}{ફંક્શન વ્યાખ્યાયિત કરો. પાયથોનમાં ઉપલબ્ધ વિવિધ પ્રકારના ફંક્શનની યાદી આપો.}

\begin{solutionbox}
\textbf{ફંક્શન}: ખાસ કાર્ય કરતો પુનઃઉપયોગી કોડ બ્લોક.

\textbf{ફંક્શન પ્રકારો:}
\begin{answertable}{Types}
\begin{tabular}{|l|l|}
\hline
\textbf{ફંક્શન પ્રકાર} & \textbf{વર્ણન} \\
\hline
\keyword{Built-in} & પૂર્વ-નિર્ધારિત ફંક્શન્સ (print, len) \\
\hline
\keyword{User-defined} & પ્રોગ્રામર દ્વારા બનાવાયેલ \\
\hline
\keyword{Lambda} & અનામ એક-લાઇન ફંક્શન્સ \\
\hline
\keyword{Recursive} & પોતાને call કરતા ફંક્શન્સ \\
\hline
\end{tabular}
\end{answertable}

\textbf{ફાયદા:}
\begin{itemize}
    \item \textbf{કોડ પુનઃઉપયોગ}: એકવાર લખો, ઘણીવાર વાપરો
    \item \textbf{મોડ્યુલારિટી}: જટિલ સમસ્યાઓને નાના ભાગોમાં વહેંચવી
    \item \textbf{Parameters}: ફંક્શન્સ માટે ઇનપુટ values
\end{itemize}

\begin{mnemonicbox}
\mnemonic{Built-User-Lambda-Recursive}
\end{mnemonicbox}
\end{solutionbox}

\questionmarks{પ્રશ્ન 4(બ)}{04}{વેરિએબલના સ્કોપ ઉપર ટૂંક નોંધ લખો.}

\begin{solutionbox}
\textbf{વેરિએબલ સ્કોપ:}
\begin{answertable}{Scope Types}
\begin{tabular}{|l|l|l|}
\hline
\textbf{સ્કોપ પ્રકાર} & \textbf{વર્ણન} & \textbf{ઉદાહરણ} \\
\hline
\keyword{Local} & ફંક્શનની અંદર જ & ફંક્શન variables \\
\hline
\keyword{Global} & સમગ્ર પ્રોગ્રામમાં & Module-level variables \\
\hline
\keyword{Built-in} & Python keywords & print, len, type \\
\hline
\end{tabular}
\end{answertable}

\textbf{કોડ ઉદાહરણ:}
\begin{lstlisting}[language=Python]
x = 10  # Global variable

def my_function():
    y = 20  # Local variable
    print(x)  # Access global
    print(y)  # Access local

my_function()
# print(y)  # Error: y not accessible
\end{lstlisting}

\textbf{મુખ્ય ખ્યાલો:}
\begin{itemize}
    \item \textbf{Variable accessibility}: variables ક્યાં વાપરી શકાય
    \item \textbf{LEGB rule}: Local, Enclosing, Global, Built-in
\end{itemize}

\begin{mnemonicbox}
\mnemonic{Local-Global-Builtin}
\end{mnemonicbox}
\end{solutionbox}

\questionmarks{પ્રશ્ન 4(ક)}{07}{Python કોડ લખો જે ઉપભોક્તાને મુખ્ય સ્ટ્રિંગ અને સબસ્ટ્રિંગ માટે પૂછે છે અને મુખ્ય સ્ટ્રિંગમાં સબસ્ટ્રિંગની મેમ્બરશિપ તપાસે છે.}

\begin{solutionbox}
\textbf{કોડ:}
\begin{lstlisting}[language=Python]
def check_substring():
    main_string = input("Enter main string: ")
    substring = input("Enter substring: ")

    if substring in main_string:
        print(f"'{substring}' found in '{main_string}'")
        print(f"Position: {main_string.find(substring)}")
    else:
        print(f"'{substring}' not found in '{main_string}'")

# Enhanced version with case handling
def check_substring_enhanced():
    main_string = input("Enter main string: ")
    substring = input("Enter substring: ")

    if substring.lower() in main_string.lower():
        print("Substring found (case-insensitive)")
    else:
        print("Substring not found")

check_substring()
\end{lstlisting}

\textbf{સમજૂતી:}
\begin{itemize}
    \item \textbf{યુઝર ઇન્ટરેક્શન}: string collection માટે \code{input()}
    \item \textbf{Membership testing}: \code{in} operator નો ઉપયોગ
    \item \textbf{Case sensitivity}: વૈકલ્પિક case handling
\end{itemize}

\begin{mnemonicbox}
\mnemonic{Input-Check-Report-Position}
\end{mnemonicbox}
\end{solutionbox}

\questionmarks{પ્રશ્ન 4(અ OR)}{03}{લોકલ વેરિએબલ અને ગ્લોબલ વેરિએબલ શું છે?}

\begin{solutionbox}
\textbf{સરખામણી:}
\begin{answertable}{Local vs Global}
\begin{tabular}{|l|l|l|l|}
\hline
\textbf{વેરિએબલ પ્રકાર} & \textbf{સ્કોપ} & \textbf{આયુષ્ય} & \textbf{પ્રવેશ} \\
\hline
\keyword{Local} & ફક્ત ફંક્શનમાં & ફંક્શન execution & મર્યાદિત \\
\hline
\keyword{Global} & સમગ્ર પ્રોગ્રામ & પ્રોગ્રામ execution & વ્યાપક \\
\hline
\end{tabular}
\end{answertable}

\textbf{ઉદાહરણ:}
\begin{lstlisting}[language=Python]
global_var = 100  # Global

def function():
    local_var = 50  # Local
    print(global_var)  # Accessible
    print(local_var)   # Accessible

print(global_var)  # Accessible
# print(local_var)  # Error
\end{lstlisting}

\begin{mnemonicbox}
\mnemonic{Local-Limited, Global-Everywhere}
\end{mnemonicbox}
\end{solutionbox}

\questionmarks{પ્રશ્ન 4(બ OR)}{04}{પાયથોનના કોઈપણ ચાર બિલ્ટ-ઇન ફંક્શન સમજાવો.}

\begin{solutionbox}
\textbf{બિલ્ટ-ઇન ફંક્શન:}
\begin{answertable}{Functions}
\begin{tabular}{|l|l|l|}
\hline
\textbf{ફંક્શન} & \textbf{હેતુ} & \textbf{ઉદાહરણ} \\
\hline
\code{len()} & લંબાઈ આપે & \code{len("hello")} \rightarrow 5 \\
\hline
\code{type()} & ડેટા ટાઇપ આપે & \code{type(10)} \rightarrow <class 'int'> \\
\hline
\code{input()} & યુઝર ઇનપુટ લે & \code{name = input("Name: ")} \\
\hline
\code{print()} & આઉટપુટ દર્શાવે & \code{print("Hello")} \\
\hline
\end{tabular}
\end{answertable}

\textbf{વધારાના ઉદાહરણો:}
\begin{lstlisting}[language=Python]
# len() function
print(len([1, 2, 3, 4]))  # Output: 4

# type() function
print(type(3.14))  # Output: <class 'float'>

# input() function
age = input("Enter age: ")

# print() function
print("Your age is:", age)
\end{lstlisting}

\begin{mnemonicbox}
\mnemonic{Length-Type-Input-Print}
\end{mnemonicbox}
\end{solutionbox}

\questionmarks{પ્રશ્ન 4(ક OR)}{07}{Python કોડ લખો જે આપેલ સ્ટ્રિંગમાં સબસ્ટ્રિંગને શોધે છે.}

\begin{solutionbox}
\textbf{કોડ:}
\begin{lstlisting}[language=Python]
def locate_substring():
    main_string = input("Enter main string: ")
    substring = input("Enter substring to find: ")

    # Method 1: Using find()
    position = main_string.find(substring)
    if position != -1:
        print(f"Found at index: {position}")
    else:
        print("Substring not found")

    # Method 2: Using index() with exception handling
    try:
        position = main_string.index(substring)
        print(f"Located at index: {position}")
    except ValueError:
        print("Substring not found")

    # Method 3: Find all occurrences
    positions = []
    start = 0
    while True:
        pos = main_string.find(substring, start)
        if pos == -1:
            break
        positions.append(pos)
        start = pos + 1

    if positions:
        print(f"All positions: {positions}")

locate_substring()
\end{lstlisting}

\textbf{મુખ્ય મેથડ:}
\begin{itemize}
    \item \textbf{find() method}: index આપે અથવા -1
    \item \textbf{index() method}: index આપે અથવા exception raise કરે
    \item \textbf{બહુવિધ occurrences}: બધી સ્થિતિઓ શોધવા માટે લૂપ
\end{itemize}

\begin{mnemonicbox}
\mnemonic{Find-Index-Exception-Multiple}
\end{mnemonicbox}
\end{solutionbox}

\questionmarks{પ્રશ્ન 5(અ)}{03}{સ્ટ્રિંગ વ્યાખ્યાયિત કરો. વિવિધ સ્ટ્રિંગ ઓપરેશન્સની યાદી બનાવો.}

\begin{solutionbox}
\textbf{સ્ટ્રિંગ}: quotes માં બંધ characters ની sequence.

\textbf{ઓપરેશન્સ:}
\begin{answertable}{String Operations}
\begin{tabular}{|l|l|l|}
\hline
\textbf{ઓપરેશન} & \textbf{મેથડ} & \textbf{ઉદાહરણ} \\
\hline
સંયોજન & + & "Hello" + "World" \\
\hline
પુનરાવર્તન & * & "Hi" * 3 \\
\hline
સ્લાઇસિંગ & [start:end] & "Hello"[1:4] \\
\hline
લંબાઈ & \code{len()} & \code{len("Hello")} \\
\hline
કેસ & \code{upper()}, \code{lower()} & "hello".upper() \\
\hline
\end{tabular}
\end{answertable}

\textbf{વિશેષતાઓ:}
\begin{itemize}
    \item \textbf{Immutable}: સ્ટ્રિંગ બનાવ્યા પછી બદલી શકાતી નથી
    \item \textbf{Indexing}: વ્યક્તિગત characters access કરવું
    \item \textbf{Methods}: manipulation માટે built-in functions
\end{itemize}

\begin{mnemonicbox}
\mnemonic{Concat-Repeat-Slice-Length-Case}
\end{mnemonicbox}
\end{solutionbox}

\questionmarks{પ્રશ્ન 5(બ)}{04}{આપણે કેવી રીતે ઓળખી શકીએ કે એલિમેન્ટ એ લિસ્ટનો સભ્ય છે કે નહીં? યોગ્ય ઉદાહરણ સાથે સમજાવો.}

\begin{solutionbox}
\textbf{પદ્ધતિઓ:}
\begin{answertable}{Membership Check}
\begin{tabular}{|l|l|l|}
\hline
\textbf{પદ્ધતિ} & \textbf{ઓપરેટર} & \textbf{પરિણામ} \\
\hline
\code{in} & element in list & True/False \\
\hline
\code{not in} & element not in list & True/False \\
\hline
\code{count()} & list.count(element) & occurrences ની સંખ્યા \\
\hline
\end{tabular}
\end{answertable}

\textbf{ઉદાહરણ:}
\begin{lstlisting}[language=Python]
fruits = ["apple", "banana", "orange", "mango"]

# Using 'in' operator
if "apple" in fruits:
    print("Apple is available")

# Using 'not in' operator
if "grapes" not in fruits:
    print("Grapes not available")

# Using count() method
count = fruits.count("apple")
if count > 0:
    print(f"Apple found {count} times")
\end{lstlisting}

\begin{mnemonicbox}
\mnemonic{In-NotIn-Count for membership}
\end{mnemonicbox}
\end{solutionbox}

\questionmarks{પ્રશ્ન 5(ક)}{07}{Python કોડ લખો જે આપેલ સ્ટ્રિંગના બીજા સબસ્ટ્રિંગ સાથે સબસ્ટ્રિંગને બદલે છે. આપેલ સ્ટ્રિંગ 'Welcome to GTU' તરીકે ધ્યાનમાં લો અને સબસ્ટ્રિંગ 'GTU' ને 'Gujarat Technological University' સાથે બદલો.}

\begin{solutionbox}
\textbf{કોડ:}
\begin{lstlisting}[language=Python]
def replace_substring():
    # Given string
    original = "Welcome to GTU"
    old_substring = "GTU"
    new_substring = "Gujarat Technological University"

    # Method 1: Using replace()
    result1 = original.replace(old_substring, new_substring)
    print(f"Original: {original}")
    print(f"Modified: {result1}")

    # Method 2: Manual replacement
    if old_substring in original:
        index = original.find(old_substring)
        result2 = original[:index] + new_substring + original[index + len(old_substring):]
        print(f"Manual method: {result2}")

    # Method 3: Replace all occurrences
    test_string = "GTU offers GTU degree from GTU"
    result3 = test_string.replace("GTU", "Gujarat Technological University")
    print(f"Multiple replacements: {result3}")

replace_substring()
\end{lstlisting}

\textbf{આઉટપુટ:}
\begin{verbatim}
Original: Welcome to GTU
Modified: Welcome to Gujarat Technological University
\end{verbatim}

\textbf{મુખ્ય મુદ્દાઓ:}
\begin{itemize}
    \item \textbf{replace() method}: built-in string function
    \item \textbf{Slicing method}: મેન્યુઅલ string manipulation
    \item \textbf{બધી occurrences}: દરેક instance બદલે છે
\end{itemize}

\begin{mnemonicbox}
\mnemonic{Find-Replace-Slice-All}
\end{mnemonicbox}
\end{solutionbox}

\questionmarks{પ્રશ્ન 5(અ OR)}{03}{લિસ્ટ વ્યાખ્યાયિત કરો. વિવિધ લિસ્ટ ઓપરેશન્સની યાદી બનાવો.}

\begin{solutionbox}
\textbf{લિસ્ટ}: ક્રમબદ્ધ items નો collection જે modify કરી શકાય છે.

\textbf{ઓપરેશન્સ:}
\begin{answertable}{List Operations}
\begin{tabular}{|l|l|l|}
\hline
\textbf{ઓપરેશન} & \textbf{મેથડ} & \textbf{ઉદાહરણ} \\
\hline
ઉમેરો & \code{append()}, \code{insert()} & \code{list.append(item)} \\
\hline
દૂર કરો & \code{remove()}, \code{pop()} & \code{list.remove(item)} \\
\hline
પ્રવેશ & \code{[index]} & \code{list[0]} \\
\hline
સ્લાઇસ & \code{[start:end]} & \code{list[1:3]} \\
\hline
સોર્ટ & \code{sort()} & \code{list.sort()} \\
\hline
\end{tabular}
\end{answertable}

\textbf{વિશેષતાઓ:}
\begin{itemize}
    \item \textbf{Mutable}: લિસ્ટ બનાવ્યા પછી બદલી શકાય છે
    \item \textbf{Indexed}: સ્થિતિ દ્વારા elements access કરાય છે
    \item \textbf{Dynamic}: કદ વધી અથવા ઘટી શકે છે
\end{itemize}

\begin{mnemonicbox}
\mnemonic{Add-Remove-Access-Slice-Sort}
\end{mnemonicbox}
\end{solutionbox}

\questionmarks{પ્રશ્ન 5(બ OR)}{04}{સ્ટ્રિંગ સ્લાઇસિંગ ઉપર ટૂંક નોંધ લખો. યોગ્ય ઉદાહરણ સાથે સમજાવો.}

\begin{solutionbox}
\textbf{સ્ટ્રિંગ સ્લાઇસિંગ}: \code{[start:end:step]} વાપરીને string ના ભાગો extract કરવું.

\textbf{Syntax:}
\begin{answertable}{Slicing Syntax}
\begin{tabular}{|l|l|l|}
\hline
\textbf{Syntax} & \textbf{વર્ણન} & \textbf{ઉદાહરણ} \\
\hline
\code{[start:]} & start થી અંત સુધી & "Hello"[1:] \rightarrow "ello" \\
\hline
\code{[:end]} & શરૂઆત થી end સુધી & "Hello"[:3] \rightarrow "Hel" \\
\hline
\code{[start:end]} & ખાસ રેન્જ & "Hello"[1:4] \rightarrow "ell" \\
\hline
\code{[::-1]} & રિવર્સ સ્ટ્રિંગ & "Hello"[::-1] \rightarrow "olleH" \\
\hline
\end{tabular}
\end{answertable}

\textbf{ઉદાહરણ:}
\begin{lstlisting}[language=Python]
text = "Python Programming"

print(text[0:6])    # "Python"
print(text[7:])     # "Programming"
print(text[:6])     # "Python"
print(text[::2])    # "Pto rgamn"
print(text[::-1])   # "gnimmargorP nohtyP"
\end{lstlisting}

\begin{mnemonicbox}
\mnemonic{Start-End-Step-Reverse}
\end{mnemonicbox}
\end{solutionbox}

\end{document}
