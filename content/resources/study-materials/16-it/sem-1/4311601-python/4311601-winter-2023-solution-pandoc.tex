\documentclass[10pt,a4paper]{article}

% content/resources/templates/preamble.tex
\usepackage[margin=0.6in]{geometry}
\author{Milav Dabgar}
\usepackage{amsmath,amssymb,amsthm}
\usepackage{booktabs}
\usepackage{multirow}
\usepackage{xcolor}
\usepackage{tcolorbox}
\tcbuselibrary{breakable,skins}
\usepackage[colorlinks=true,linkcolor=blue]{hyperref}
\usepackage{titlesec}
\usepackage{enumitem}
\usepackage{tikz}
\usepackage{pgfplots}
\usepackage{circuitikz}
\usepackage[version=4]{mhchem}
\usepackage{longtable}
\usepackage{array}
\usepackage{float}
\usepackage{caption}
\usepackage{listings}

\lstset{
  basicstyle=\small\ttfamily,
  breaklines=true,
  breakatwhitespace=false,
  postbreak=\mbox{\textcolor{red}{$\hookrightarrow$}\space},
  float=false,
  numbers=left,
  numberstyle=\tiny\color{gray},
  numbersep=10pt,
  xleftmargin=2em,
  keywordstyle=\color{blue},
  commentstyle=\color{green!60!black},
  stringstyle=\color{purple},
  backgroundcolor=\color{gray!5},
  showstringspaces=false,
  tabsize=2,
  captionpos=b,
  keepspaces=true,
  columns=flexible
}

\pgfplotsset{compat=1.18}
\usetikzlibrary{shapes,arrows,positioning,calc,patterns,decorations.pathmorphing,decorations.markings,arrows.meta}

% Color scheme
\definecolor{headcolor}{RGB}{0,102,204}
\definecolor{keycolor}{RGB}{220,20,60}
\definecolor{solutioncolor}{RGB}{34,139,34}
\definecolor{mnemoniccolor}{RGB}{148,0,211}
\definecolor{codecolor}{RGB}{0,0,100}

% Spacing
\setlength{\parskip}{3pt}
\setlist[itemize]{nosep}
\setlist[enumerate]{nosep}

% Title formatting
\titleformat{\section}{\Large\bfseries\color{headcolor}}{\thesection}{1em}{}
\titleformat{\subsection}{\large\bfseries\color{headcolor}}{\thesubsection}{1em}{}

% Pandoc tightlist compatibility
\providecommand{\tightlist}{%
  \setlength{\itemsep}{0pt}\setlength{\parskip}{0pt}}

% Pandoc longtable compatibility
\newcounter{none}
\def\thenone{}


% content/resources/templates/english-boxes.tex
% This file is currently empty - it exists to maintain consistency with the import structure.
% Add custom environments here if needed in the future.


\begin{document}

\begin{center}
{\Huge\bfseries\color{headcolor} Subject Name Solutions}\\[5pt]
{\LARGE 4311601 -- Winter 2023}\\[3pt]
{\large Semester 1 Study Material}\\[3pt]
{\normalsize\textit{Detailed Solutions and Explanations}}
\end{center}

\vspace{10pt}

\subsection*{Question 1(a) [3 marks]}\label{q1a}

\textbf{What is Flow chart? List out symbols used in Flow chart.}

\begin{solutionbox}

A \textbf{flowchart} is a graphical representation of an algorithm that
shows the sequence of steps and decision points in a process using
standardized symbols.

\textbf{Flowchart Symbols Table:}

{\def\LTcaptype{none} % do not increment counter
\begin{longtable}[]{@{}lll@{}}
\toprule\noalign{}
Symbol & Name & Purpose \\
\midrule\noalign{}
\endhead
\bottomrule\noalign{}
\endlastfoot
Oval & Terminal & Start/End of program \\
Rectangle & Process & Processing/Calculation steps \\
Diamond & Decision & Conditional statements \\
Parallelogram & Input/Output & Data input or output \\
Circle & Connector & Connect flowchart parts \\
Arrow & Flow line & Direction of flow \\
\end{longtable}
}

\textbf{Key Points:}

\begin{itemize}
\tightlist
\item
  \textbf{Visual representation}: Shows program logic graphically
\item
  \textbf{Step-by-step}: Displays sequential flow of operations
\item
  \textbf{Decision making}: Diamond symbols show conditional branches
\end{itemize}

\end{solutionbox}
\begin{mnemonicbox}
``Flow Charts Show Program Steps Visually''

\end{mnemonicbox}
\subsection*{Question 1(b) [4 marks]}\label{q1b}

\textbf{Write a short note on for loop.}

\begin{solutionbox}

The \textbf{for loop} is used to iterate over a sequence (list, tuple,
string, range) in Python.

\textbf{For Loop Table:}

{\def\LTcaptype{none} % do not increment counter
\begin{longtable}[]{@{}lll@{}}
\toprule\noalign{}
Component & Syntax & Example \\
\midrule\noalign{}
\endhead
\bottomrule\noalign{}
\endlastfoot
Basic & \texttt{for\ variable\ in\ sequence:} &
\texttt{for\ i\ in\ range(5):} \\
Range & \texttt{range(start,\ stop,\ step)} &
\texttt{range(1,\ 10,\ 2)} \\
List & \texttt{for\ item\ in\ list:} &
\texttt{for\ x\ in\ [1,2,3]:} \\
String & \texttt{for\ char\ in\ string:} &
\texttt{for\ c\ in\ "hello":} \\
\end{longtable}
}

\textbf{Simple Code Example:}

\begin{verbatim}
for i in range(3):
    print(i)
\# Output: 0, 1, 2
\end{verbatim}

\textbf{Key Features:}

\begin{itemize}
\tightlist
\item
  \textbf{Automatic iteration}: No manual counter needed
\item
  \textbf{Sequence traversal}: Works with any iterable object
\item
  \textbf{Range function}: Creates number sequences easily
\end{itemize}

\end{solutionbox}
\begin{mnemonicbox}
``For Loops Iterate Through Sequences''

\end{mnemonicbox}
\subsection*{Question 1(c) [7 marks]}\label{q1c}

\textbf{Write a program to display Fibonacci series up to nth term where
n is provided by the user.}

\begin{solutionbox}

\textbf{Fibonacci Series Program:}

\begin{verbatim}
\# Get number of terms from user
n = int(input("Enter number of terms: "))

\# Initialize first two terms
a, b = 0, 1

\# Display first term
if n {=} 1:
    print(a, end=" ")
    
\# Display second term
if n {=} 2:
    print(b, end=" ")

\# Generate remaining terms
for i in range(2, n):
    c = a + b
    print(c, end=" ")
    a, b = b, c
\end{verbatim}

\textbf{Algorithm Flow:}

\begin{verbatim}
flowchart LR
    A[Start] {-{-} B[Input n]}
    B {-{-} C\{n = 1?\}}
    C {-{-}|Yes| D[Print 0]}
    C {-{-}|No| H[End]}
    D {-{-} E\{n = 2?\}}
    E {-{-}|Yes| F[Print 1]}
    E {-{-}|No| H}
    F {-{-} G[Loop i=2 to n{-}1]}
    G {-{-} I[c = a + b]}
    I {-{-} J[Print c]}
J {-{-} K[a = b,

b = c]}

    K {-{-} L\{i  n{-}1?\}}
    L {-{-}|Yes| G}
    L {-{-}|No| H[End]}
\end{verbatim}

\textbf{Key Concepts:}

\begin{itemize}
\tightlist
\item
  \textbf{Sequential generation}: Each term = sum of previous two
\item
  \textbf{Variable swapping}: Update a, b values efficiently
\item
  \textbf{User input}: Dynamic series length
\end{itemize}

\end{solutionbox}
\begin{mnemonicbox}
``Fibonacci: Add Previous Two Numbers''

\end{mnemonicbox}
\subsection*{Question 1(c OR) [7
marks]}\label{question-1c-or-7-marks}

\textbf{Draw a flow chart to print ODD numbers from 1 to 100.}

\begin{solutionbox}

\textbf{Flowchart for ODD Numbers 1 to 100:}

\begin{verbatim}
flowchart LR
    A[Start] {-{-} B[i = 1]}
    B {-{-} C\{i = 100?\}}
    C {-{-}|Yes| D\{i \% 2 != 0?\}}
    D {-{-}|Yes| E[Print i]}
    D {-{-}|No| F[i = i + 1]}
    E {-{-} F}
    F {-{-} C}
    C {-{-}|No| G[End]}
\end{verbatim}

\textbf{Corresponding Python Code:}

\begin{verbatim}
for i in range(1, 101):
    if i \% 2 != 0:
        print(i, end=" ")
\end{verbatim}

\textbf{Alternative Method:}

\begin{verbatim}
for i in range(1, 101, 2):
    print(i, end=" ")
\end{verbatim}

\textbf{Key Elements:}

\begin{itemize}
\tightlist
\item
  \textbf{Loop control}: i from 1 to 100
\item
  \textbf{Odd check}: i \% 2 != 0 condition
\item
  \textbf{Step increment}: Move to next number
\end{itemize}

\end{solutionbox}
\begin{mnemonicbox}
``Odd Numbers: Remainder 1 When Divided by 2''

\end{mnemonicbox}
\subsection*{Question 2(a) [3 marks]}\label{q2a}

\textbf{Write a Program to find whether a number is Palindrome or not.}

\begin{solutionbox}

\textbf{Palindrome Check Program:}

\begin{verbatim}
\# Input number
num = int(input("Enter a number: "))
temp = num
reverse = 0

\# Reverse the number
while temp {} 0:
    reverse = reverse * 10 + temp \% 10
    temp = temp // 10

\# Check palindrome
if num == reverse:
    print(f"\{num\} is palindrome")
else:
    print(f"\{num\} is not palindrome")
\end{verbatim}

\textbf{Algorithm Table:}

{\def\LTcaptype{none} % do not increment counter
\begin{longtable}[]{@{}lll@{}}
\toprule\noalign{}
Step & Operation & Example (121) \\
\midrule\noalign{}
\endhead
\bottomrule\noalign{}
\endlastfoot
1 & Get last digit & 121 \% 10 = 1 \\
2 & Build reverse & 0*10 + 1 = 1 \\
3 & Remove last digit & 121 // 10 = 12 \\
4 & Repeat until 0 & Continue process \\
\end{longtable}
}

\textbf{Key Points:}

\begin{itemize}
\tightlist
\item
  \textbf{Digit extraction}: Use modulo (\%) operator
\item
  \textbf{Reverse building}: Multiply by 10 and add digit
\item
  \textbf{Comparison}: Original equals reversed
\end{itemize}

\end{solutionbox}
\begin{mnemonicbox}
``Palindrome Reads Same Forward Backward''

\end{mnemonicbox}
\subsection*{Question 2(b) [4 marks]}\label{q2b}

\textbf{Explain features of Python Programming.}

\begin{solutionbox}

\textbf{Python Features Table:}

{\def\LTcaptype{none} % do not increment counter
\begin{longtable}[]{@{}lll@{}}
\toprule\noalign{}
Feature & Description & Benefit \\
\midrule\noalign{}
\endhead
\bottomrule\noalign{}
\endlastfoot
Easy Syntax & Simple, readable code & Faster development \\
Interpreted & No compilation needed & Quick testing \\
Object-Oriented & Classes and objects support & Code reusability \\
Open Source & Free to use & No licensing cost \\
Cross-Platform & Runs on multiple OS & Wide compatibility \\
Large Libraries & Extensive built-in modules & Rich functionality \\
\end{longtable}
}

\textbf{Key Advantages:}

\begin{itemize}
\tightlist
\item
  \textbf{Beginner-friendly}: Easy to learn and understand
\item
  \textbf{Versatile}: Web development, AI, data science
\item
  \textbf{Community support}: Large developer community
\item
  \textbf{Dynamic typing}: No variable type declaration needed
\end{itemize}

\end{solutionbox}
\begin{mnemonicbox}
``Python: Easy, Powerful, Popular Programming''

\end{mnemonicbox}
\subsection*{Question 2(c) [7 marks]}\label{q2c}

\textbf{Explain basic structure of Python Program.}

\begin{solutionbox}

\textbf{Python Program Structure:}

\begin{verbatim}
\#!/usr/bin/env python3
\# Shebang line (optional)

"""
Documentation string (docstring)
Describes program purpose
"""

\# Import statements
import math
from datetime import date

\# Global variables
PI = 3.14159
count = 0

\# Function definitions
def calculate\_area(radius):
    """Calculate circle area"""
    return PI * radius * radius

\# Class definitions
class Calculator:
    def \_\_init\_\_(self):
        self.result = 0

\# Main program execution
if \_\_name\_\_ == "\_\_main\_\_":
    \# Program logic here
    radius = 5
    area = calculate\_area(radius)
    print(f"Area: \{area\}")
\end{verbatim}

\textbf{Structure Components Table:}

{\def\LTcaptype{none} % do not increment counter
\begin{longtable}[]{@{}lll@{}}
\toprule\noalign{}
Component & Purpose & Example \\
\midrule\noalign{}
\endhead
\bottomrule\noalign{}
\endlastfoot
Shebang & System interpreter & \texttt{\#!/usr/bin/env\ python3} \\
Docstring & Program documentation &
\texttt{"""Program\ description"""} \\
Imports & External modules & \texttt{import\ math} \\
Variables & Global data storage & \texttt{PI\ =\ 3.14159} \\
Functions & Reusable code blocks & \texttt{def\ function\_name():} \\
Classes & Object templates & \texttt{class\ ClassName:} \\
Main block & Program execution &
\texttt{if\ \_\_name\_\_\ ==\ "\_\_main\_\_":} \\
\end{longtable}
}

\textbf{Key Principles:}

\begin{itemize}
\tightlist
\item
  \textbf{Indentation}: Defines code blocks (4 spaces recommended)
\item
  \textbf{Comments}: Use \# for single line, ``\,``\,'' ``\,``\,'' for
  multi-line
\item
  \textbf{Modularity}: Organize code in functions and classes
\end{itemize}

\end{solutionbox}
\begin{mnemonicbox}
``Structure: Import, Define, Execute''

\end{mnemonicbox}
\subsection*{Question 2(a OR) [3
marks]}\label{question-2a-or-3-marks}

\textbf{Write a Program to reverse a string.}

\begin{solutionbox}

\textbf{String Reversal Program:}

\begin{verbatim}
\# Method 1: Using slicing
string = input("Enter a string: ")
reversed\_string = string[::{-}1]
print(f"Reversed: \{reversed\_string\}")

\# Method 2: Using loop
string = input("Enter a string: ")
reversed\_string = ""
for char in string:
    reversed\_string = char + reversed\_string
print(f"Reversed: \{reversed\_string\}")
\end{verbatim}

\textbf{Reversal Methods Table:}

{\def\LTcaptype{none} % do not increment counter
\begin{longtable}[]{@{}lll@{}}
\toprule\noalign{}
Method & Syntax & Example \\
\midrule\noalign{}
\endhead
\bottomrule\noalign{}
\endlastfoot
Slicing & \texttt{string[::-1]} & ``hello'' \rightarrow ``olleh'' \\
Loop & Build character by character & Add each char to front \\
Built-in & \texttt{"".join(reversed(string))} & Join reversed
sequence \\
\end{longtable}
}

\textbf{Key Concepts:}

\begin{itemize}
\tightlist
\item
  \textbf{Slicing}: Most efficient method
\item
  \textbf{Concatenation}: Build string character by character
\item
  \textbf{Indexing}: Access string positions
\end{itemize}

\end{solutionbox}
\begin{mnemonicbox}
``Reverse: Last Character First''

\end{mnemonicbox}
\subsection*{Question 2(b OR) [4
marks]}\label{question-2b-or-4-marks}

\textbf{Explain Logical Operators with example.}

\begin{solutionbox}

\textbf{Python Logical Operators:}

{\def\LTcaptype{none} % do not increment counter
\begin{longtable}[]{@{}lllll@{}}
\toprule\noalign{}
Operator & Symbol & Description & Example & Result \\
\midrule\noalign{}
\endhead
\bottomrule\noalign{}
\endlastfoot
AND & \texttt{and} & Both conditions true & \texttt{True\ and\ False} &
\texttt{False} \\
OR & \texttt{or} & At least one condition true &
\texttt{True\ or\ False} & \texttt{True} \\
NOT & \texttt{not} & Opposite of condition & \texttt{not\ True} &
\texttt{False} \\
\end{longtable}
}

\textbf{Example Code:}

\begin{verbatim}
a = 10
b = 5

\# AND operator
if a {} 5 and b {} 10:
    print("Both conditions true")

\# OR operator  
if a {} 15 or b {} 10:
    print("At least one condition true")

\# NOT operator
if not (a {} 5):
    print("a is not less than 5")
\end{verbatim}

\textbf{Truth Table:}

{\def\LTcaptype{none} % do not increment counter
\begin{longtable}[]{@{}lllll@{}}
\toprule\noalign{}
A & B & A and B & A or B & not A \\
\midrule\noalign{}
\endhead
\bottomrule\noalign{}
\endlastfoot
T & T & T & T & F \\
T & F & F & T & F \\
F & T & F & T & T \\
F & F & F & F & T \\
\end{longtable}
}

\textbf{Key Uses:}

\begin{itemize}
\tightlist
\item
  \textbf{Complex conditions}: Combine multiple checks
\item
  \textbf{Decision making}: Control program flow
\item
  \textbf{Boolean logic}: True/False operations
\end{itemize}

\end{solutionbox}
\begin{mnemonicbox}
``AND needs All, OR needs One, NOT reverses''

\end{mnemonicbox}
\subsection*{Question 2(c OR) [7
marks]}\label{question-2c-or-7-marks}

\textbf{Explain different Data Types in Python Programming language}

\begin{solutionbox}

\textbf{Python Data Types Classification:}

\begin{center}
\textbf{Mermaid Diagram (Code)}
\begin{verbatim}
{Shaded}
{Highlighting}[]
graph TD
    A[Python Data Types] {-{-}{} B[Numeric]}
    A {-{-}{} C[Sequence]}
    A {-{-}{} D[Boolean]}
    A {-{-}{} E[Set]}
    A {-{-}{} F[Dictionary]}
    B {-{-}{} G[int]}
    B {-{-}{} H[float]}
    B {-{-}{} I[complex]}
    C {-{-}{} J[str]}
    C {-{-}{} K[list]}
    C {-{-}{} L[tuple]}
{Highlighting}
{Shaded}
\end{verbatim}
\end{center}

\textbf{Data Types Table:}

{\def\LTcaptype{none} % do not increment counter
\begin{longtable}[]{@{}llll@{}}
\toprule\noalign{}
Type & Example & Description & Mutable \\
\midrule\noalign{}
\endhead
\bottomrule\noalign{}
\endlastfoot
int & \texttt{42} & Whole numbers & No \\
float & \texttt{3.14} & Decimal numbers & No \\
str & \texttt{"hello"} & Text data & No \\
list & \texttt{[1,2,3]} & Ordered collection & Yes \\
tuple & \texttt{(1,2,3)} & Ordered immutable & No \\
dict & \texttt{\{"a":1\}} & Key-value pairs & Yes \\
bool & \texttt{True/False} & Boolean values & No \\
set & \texttt{\{1,2,3\}} & Unique elements & Yes \\
\end{longtable}
}

\textbf{Example Code:}

\begin{verbatim}
\# Numeric types
age = 25          \# int
price = 99.99     \# float
complex\_num = 3+4j \# complex

\# Sequence types
name = "Python"         \# string
numbers = [1,2,3,4]     \# list
coordinates = (10,20)   \# tuple

\# Other types
is\_active = True        \# boolean
unique\_items = \{1,2,3\  }\# set
student = \{"name":"John", "age":20\  }\# dict
\end{verbatim}

\textbf{Key Features:}

\begin{itemize}
\tightlist
\item
  \textbf{Dynamic typing}: No need to declare variable types
\item
  \textbf{Type conversion}: Convert between compatible types
\item
  \textbf{Built-in functions}: \texttt{type()}, \texttt{isinstance()}
  for checking types
\end{itemize}

\end{solutionbox}
\begin{mnemonicbox}
``Python Types: Numbers, Sequences, Collections''

\end{mnemonicbox}
\subsection*{Question 3(a) [3 marks]}\label{q3a}

\textbf{What is flow control in Python? Explain with example}

\begin{solutionbox}

\textbf{Flow control} manages the execution order of program statements
using conditional and loop structures.

\textbf{Flow Control Types Table:}

{\def\LTcaptype{none} % do not increment counter
\begin{longtable}[]{@{}llll@{}}
\toprule\noalign{}
Type & Statement & Purpose & Example \\
\midrule\noalign{}
\endhead
\bottomrule\noalign{}
\endlastfoot
Sequential & Normal execution & Line by line &
\texttt{print("Hello")} \\
Selection & if, elif, else & Decision making &
\texttt{if\ x\ \textgreater{}\ 0:} \\
Iteration & for, while & Repetition & \texttt{for\ i\ in\ range(5):} \\
Jump & break, continue & Loop control & \texttt{break} \\
\end{longtable}
}

\textbf{Example Code:}

\begin{verbatim}
\# Selection example
age = 18
if age {=} 18:
    print("Adult")
else:
    print("Minor")

\# Iteration example
for i in range(3):
    print(f"Count: \{i\}")
\end{verbatim}

\textbf{Key Concepts:}

\begin{itemize}
\tightlist
\item
  \textbf{Conditional execution}: Code runs based on conditions
\item
  \textbf{Loop structures}: Repeat code blocks
\item
  \textbf{Program flow}: Control execution path
\end{itemize}

\end{solutionbox}
\begin{mnemonicbox}
``Flow Control: Decide, Repeat, Jump''

\end{mnemonicbox}
\subsection*{Question 3(b) [4 marks]}\label{q3b}

\textbf{Write a program to explain nested if statement.}

\begin{solutionbox}

\textbf{Nested If Statement Program:}

\begin{verbatim}
\# Grade calculation using nested if
marks = int(input("Enter marks: "))

if marks {=} 0 and marks {=} 100:
    if marks {=} 90:
        grade = "A+"
    elif marks {=} 80:
        if marks {=} 85:
            grade = "A"
        else:
            grade = "B+"
    elif marks {=} 70:
        grade = "B"
    elif marks {=} 60:
        grade = "C"
    else:
        grade = "F"
    print(f"Grade: \{grade\}")
else:
    print("Invalid marks")
\end{verbatim}

\textbf{Nested Structure Diagram:}

\begin{verbatim}
                    marks input
                         |
                   ┌─────▼─────┐
                   │0{=marks=100│}
                   └─────┬─────┘
              True       │       False
           ┌─────────────┼─────────────┐
           │             │             │
      ┌────▼────┐        │        ┌────▼────┐
      │marks{=90│        │        │ Invalid │}
      └────┬────┘        │        └─────────┘
           │             │
         A+│           ┌─▼─┐
           │           │...│
           │           └───┘
\end{verbatim}

\textbf{Key Features:}

\begin{itemize}
\tightlist
\item
  \textbf{Multiple levels}: if inside if statements
\item
  \textbf{Complex conditions}: Handle multiple criteria
\item
  \textbf{Logical structure}: Organize decision trees
\end{itemize}

\end{solutionbox}
\begin{mnemonicbox}
``Nested If: Decisions Within Decisions''

\end{mnemonicbox}
\subsection*{Question 3(c) [7 marks]}\label{q3c}

\textbf{Write a program to Explain types of Arguments and Parameters.}

\begin{solutionbox}

\textbf{Types of Arguments and Parameters:}

\begin{verbatim}
\# 1. Positional Arguments
def greet(name, age):
    print(f"Hello \{name\}, you are \{age\} years old")

greet("John", 25)  \# Positional arguments

\# 2. Keyword Arguments  
greet(age=30, name="Alice")  \# Keyword arguments

\# 3. Default Parameters
def introduce(name, city="Unknown"):
    print(f"\{name\} lives in \{city\}")

introduce("Bob")  \# Uses default value
introduce("Carol", "NYC")  \# Override default

\# 4. Variable{-length Arguments (*args)}
def sum\_all(*numbers):
    return sum(numbers)

result = sum\_all(1, 2, 3, 4, 5)
print(f"Sum: \{result\}")

\# 5. Keyword Variable Arguments (**kwargs)
def display\_info(**info):
    for key, value in info.items():
        print(f"\{key\}: \{value\}")

display\_info(name="David", age=28, city="Boston")
\end{verbatim}

\textbf{Parameters Types Table:}

{\def\LTcaptype{none} % do not increment counter
\begin{longtable}[]{@{}
  >{\raggedright\arraybackslash}p{(\linewidth - 6\tabcolsep) * \real{0.1667}}
  >{\raggedright\arraybackslash}p{(\linewidth - 6\tabcolsep) * \real{0.2222}}
  >{\raggedright\arraybackslash}p{(\linewidth - 6\tabcolsep) * \real{0.2500}}
  >{\raggedright\arraybackslash}p{(\linewidth - 6\tabcolsep) * \real{0.3611}}@{}}
\toprule\noalign{}
\begin{minipage}[b]{\linewidth}\raggedright
Type
\end{minipage} & \begin{minipage}[b]{\linewidth}\raggedright
Syntax
\end{minipage} & \begin{minipage}[b]{\linewidth}\raggedright
Example
\end{minipage} & \begin{minipage}[b]{\linewidth}\raggedright
Description
\end{minipage} \\
\midrule\noalign{}
\endhead
\bottomrule\noalign{}
\endlastfoot
Positional & \texttt{def\ func(a,\ b):} & \texttt{func(1,\ 2)} & Order
matters \\
Keyword & \texttt{def\ func(a,\ b):} & \texttt{func(b=2,\

a=1)} & Name

specified \\
Default & \texttt{def\ func(a,\ b=10):} & \texttt{func(5)} & Default
value \\
*args & \texttt{def\ func(*args):} & \texttt{func(1,2,3)} & Variable
positional \\
**kwargs & \texttt{def\ func(**kwargs):} & \texttt{func(a=1,\

b=2)} &

Variable keyword \\
\end{longtable}
}

\textbf{Function Call Examples:}

\begin{verbatim}
def example(pos1, pos2, default=100, *args, **kwargs):
    print(f"pos1: \{pos1\}")
    print(f"pos2: \{pos2\}")  
    print(f"default: \{default\}")
    print(f"args: \{args\}")
    print(f"kwargs: \{kwargs\}")

example(1, 2, 3, 4, 5, name="test", value=42)
\end{verbatim}

\textbf{Key Concepts:}

\begin{itemize}
\tightlist
\item
  \textbf{Flexibility}: Different ways to pass data
\item
  \textbf{Order importance}: Positional vs keyword
\item
  \textbf{Variable arguments}: Handle unknown number of inputs
\end{itemize}

\end{solutionbox}
\begin{mnemonicbox}
``Parameters: Position, Keywords, Defaults,
Variables''

\end{mnemonicbox}
\subsection*{Question 3(a OR) [3
marks]}\label{question-3a-or-3-marks}

\textbf{Explain break and continue statement with example.}

\begin{solutionbox}

\textbf{Break and Continue Statements:}

\textbf{Break Statement:}

\begin{verbatim}
\# Break example {- exit loop}
for i in range(10):
if

i == 5:

        break
    print(i)
\# Output: 0, 1, 2, 3, 4
\end{verbatim}

\textbf{Continue Statement:}

\begin{verbatim}
\# Continue example {- skip iteration}
for i in range(5):
if

i == 2:

        continue
    print(i)
\# Output: 0, 1, 3, 4
\end{verbatim}

\textbf{Comparison Table:}

{\def\LTcaptype{none} % do not increment counter
\begin{longtable}[]{@{}
  >{\raggedright\arraybackslash}p{(\linewidth - 6\tabcolsep) * \real{0.2683}}
  >{\raggedright\arraybackslash}p{(\linewidth - 6\tabcolsep) * \real{0.2195}}
  >{\raggedright\arraybackslash}p{(\linewidth - 6\tabcolsep) * \real{0.1951}}
  >{\raggedright\arraybackslash}p{(\linewidth - 6\tabcolsep) * \real{0.3171}}@{}}
\toprule\noalign{}
\begin{minipage}[b]{\linewidth}\raggedright
Statement
\end{minipage} & \begin{minipage}[b]{\linewidth}\raggedright
Purpose
\end{minipage} & \begin{minipage}[b]{\linewidth}\raggedright
Action
\end{minipage} & \begin{minipage}[b]{\linewidth}\raggedright
Example Use
\end{minipage} \\
\midrule\noalign{}
\endhead
\bottomrule\noalign{}
\endlastfoot
break & Exit loop & Terminates entire loop & Exit on condition \\
continue & Skip iteration & Jump to next iteration & Skip specific
values \\
\end{longtable}
}

\textbf{Key Differences:}

\begin{itemize}
\tightlist
\item
  \textbf{Break}: Completely exits loop
\item
  \textbf{Continue}: Skips current iteration only
\item
  \textbf{Flow control}: Manage loop execution
\end{itemize}

\end{solutionbox}
\begin{mnemonicbox}
``Break Exits, Continue Skips''

\end{mnemonicbox}
\subsection*{Question 3(b OR) [4
marks]}\label{question-3b-or-4-marks}

\textbf{Create a program to display the following pattern}

\begin{verbatim}
1
12
123
1234
12345
\end{verbatim}

\begin{solutionbox}

\textbf{Number Pattern Program:}

\begin{verbatim}
\# Method 1: Using nested loops
rows = 5
for i in range(1, rows + 1):
    for j in range(1, i + 1):
        print(j, end="")
    print()  \# New line

\# Method 2: Using string manipulation
for i in range(1, 6):
    line = ""
    for j in range(1, i + 1):
        line += str(j)
    print(line)

\# Method 3: Using join
for i in range(1, 6):
    numbers = [str(j) for j in range(1, i + 1)]
    print("".join(numbers))
\end{verbatim}

\textbf{Pattern Logic Table:}

{\def\LTcaptype{none} % do not increment counter
\begin{longtable}[]{@{}llll@{}}
\toprule\noalign{}
Row & Numbers & Range & Output \\
\midrule\noalign{}
\endhead
\bottomrule\noalign{}
\endlastfoot
1 & 1 & 1 to 1 & 1 \\
2 & 1,2 & 1 to 2 & 12 \\
3 & 1,2,3 & 1 to 3 & 123 \\
4 & 1,2,3,4 & 1 to 4 & 1234 \\
5 & 1,2,3,4,5 & 1 to 5 & 12345 \\
\end{longtable}
}

\textbf{Key Concepts:}

\begin{itemize}
\tightlist
\item
  \textbf{Nested loops}: Outer for rows, inner for numbers
\item
  \textbf{Range function}: Generate number sequences
\item
  \textbf{Print control}: Use end=``\,'' to avoid newlines
\end{itemize}

\end{solutionbox}
\begin{mnemonicbox}
``Pattern: Row Number Determines Column Count''

\end{mnemonicbox}
\subsection*{Question 3(c OR) [7
marks]}\label{question-3c-or-7-marks}

\textbf{Explain the following mathematical functions by writing a code
for each: 1. abs() 2. max() 3. pow() 4. sum()}

\begin{solutionbox}

\textbf{Mathematical Functions in Python:}

\begin{verbatim}
\# 1. abs() {- Absolute value}
numbers = [{-}5, 3.7, {-}10.2, 0]
print("abs() function examples:")
for num in numbers:
    print(f"abs(\{num\}) = \{abs(num)\}")

\# 2. max() {- Maximum value}
list1 = [4, 7, 2, 9, 1]
print(f"{n}max() function examples:")
print(f"max(\{list1\}) = \{max(list1)\}")
print(f"max(10, 25, 5) = \{max(10, 25, 5)\}")
print(f"max({hello) = }\{max({hello})\}")  \# Alphabetically

\# 3. pow() {- Power function}
print(f"{n}pow() function examples:")
print(f"pow(2, 3) = \{pow(2, 3)\}")      \# 2\^{3 = 8}
print(f"pow(5, 2) = \{pow(5, 2)\}")      \# 5\^{2 = 25}
print(f"pow(8, 1/3) = \{pow(8, 1/3)\}")  \# Cube root of 8

\# 4. sum() {- Sum of sequence}
numbers = [1, 2, 3, 4, 5]
print(f"{n}sum() function examples:")
print(f"sum(\{numbers\}) = \{sum(numbers)\}")
print(f"sum(\{numbers\}, 10) = \{sum(numbers, 10)\}")  \# With start value
\end{verbatim}

\textbf{Functions Summary Table:}

{\def\LTcaptype{none} % do not increment counter
\begin{longtable}[]{@{}lllll@{}}
\toprule\noalign{}
Function & Syntax & Purpose & Example & Result \\
\midrule\noalign{}
\endhead
\bottomrule\noalign{}
\endlastfoot
abs() & \texttt{abs(x)} & Absolute value & \texttt{abs(-5)} & 5 \\
max() & \texttt{max(iterable)} & Maximum value &
\texttt{max([1,5,3])} & 5 \\
pow() & \texttt{pow(x,\ y)} & x raised to power y & \texttt{pow(2,\ 3)}
& 8 \\
sum() & \texttt{sum(iterable)} & Sum of values &
\texttt{sum([1,2,3])} & 6 \\
\end{longtable}
}

\textbf{Detailed Examples:}

\begin{verbatim}
\# Real{-world usage}
distances = [{-}10, 15, {-}8, 12]  \# Signed distances
actual\_distances = [abs(d) for d in distances]
print(f"Actual distances: \{actual\_distances\}")

scores = [85, 92, 78, 96, 88]
highest\_score = max(scores)
total\_score = sum(scores)
print(f"Highest: \{highest\_score\}, Total: \{total\_score\}")

\# Calculate compound interest
principal = 1000
rate = 0.05
time = 3
amount = principal * pow(1 + rate, time)
print(f"Compound Interest Amount: \{amount\}")
\end{verbatim}

\textbf{Key Applications:}

\begin{itemize}
\tightlist
\item
  \textbf{abs()}: Distance calculations, error handling
\item
  \textbf{max()}: Finding extremes, competition results
\item
  \textbf{pow()}: Scientific calculations, compound interest
\item
  \textbf{sum()}: Total calculations, statistics
\end{itemize}

\end{solutionbox}
\begin{mnemonicbox}
``Math Functions: Absolute, Maximum, Power, Sum''

\end{mnemonicbox}
\subsection*{Question 4(a) [3 marks]}\label{q4a}

\textbf{Explain scope of variables.}

\begin{solutionbox}

\textbf{Variable Scope} refers to the region where a variable can be
accessed in a program.

\textbf{Scope Types Table:}

{\def\LTcaptype{none} % do not increment counter
\begin{longtable}[]{@{}llll@{}}
\toprule\noalign{}
Scope & Description & Lifetime & Access \\
\midrule\noalign{}
\endhead
\bottomrule\noalign{}
\endlastfoot
Local & Inside function & Function execution & Function only \\
Global & Outside functions & Program execution & Entire program \\
Built-in & Python keywords & Python session & Everywhere \\
\end{longtable}
}

\textbf{Example Code:}

\begin{verbatim}
x = 10  \# Global variable

def my\_function():
    y = 20  \# Local variable
    print(f"Local y: \{y\}")
    print(f"Global x: \{x\}")

my\_function()
print(f"Global x: \{x\}")
\# print(y)  \# Error: y not accessible here
\end{verbatim}

\textbf{Key Rules:}

\begin{itemize}
\tightlist
\item
  \textbf{Local variables}: Created inside functions
\item
  \textbf{Global variables}: Accessible throughout program
\item
  \textbf{LEGB rule}: Local \rightarrow Enclosing \rightarrow Global \rightarrow Built-in
\end{itemize}

\end{solutionbox}
\begin{mnemonicbox}
``Scope: Local Lives in Functions, Global Lives
Everywhere''

\end{mnemonicbox}
\subsection*{Question 4(b) [4 marks]}\label{q4b}

\textbf{Develop a program to create nested LOOP and display numbers.}

\begin{solutionbox}

\textbf{Nested Loop Program:}

\begin{verbatim}
\# Example 1: Number grid
print("Number Grid Pattern:")
for i in range(1, 4):
    for j in range(1, 5):
        print(f"\{i\\{}j\}", end=" ")
    print()  \# New line after each row

\# Example 2: Multiplication table
print("{n}Multiplication Table:")
for i in range(1, 4):
    for j in range(1, 6):
        result = i * j
        print(f"\{result:3\}", end=" ")
    print()

\# Example 3: Number pyramid
print("{n}Number Pyramid:")
for i in range(1, 5):
    for j in range(1, i + 1):
        print(j, end=" ")
    print()
\end{verbatim}

\textbf{Nested Loop Structure:}

\begin{verbatim}
    Outer Loop (i)
         │
    ┌────▼────┐
    │  i = 1  │
    └────┬────┘
         │
    Inner Loop (j)
    ┌────▼────┐
    │j=1,2,3,4│
    └────┬────┘
         │
    ┌────▼────┐
    │  i = 2  │
    └─────────┘
\end{verbatim}

\textbf{Key Concepts:}

\begin{itemize}
\tightlist
\item
  \textbf{Outer loop}: Controls rows/major iterations
\item
  \textbf{Inner loop}: Controls columns/minor iterations
\item
  \textbf{Execution flow}: Inner completes before outer increments
\end{itemize}

\end{solutionbox}
\begin{mnemonicbox}
``Nested Loops: Outer Controls Inner''

\end{mnemonicbox}
\subsection*{Question 4(c) [7 marks]}\label{q4c}

\textbf{Write a program to create a list of ODD and EVEN numbers in
range of 1 to 50.}

\begin{solutionbox}

\textbf{ODD and EVEN Numbers Program:}

\begin{verbatim}
\# Method 1: Using loops and conditions
odd\_numbers = []
even\_numbers = []

for i in range(1, 51):
    if i \% 2 == 0:
        even\_numbers.append(i)
    else:
        odd\_numbers.append(i)

print("ODD Numbers (1{-50):"})
print(odd\_numbers)
print(f"Count: \{len(odd\_numbers)\}")

print("{n}EVEN Numbers (1{-50):"})
print(even\_numbers)
print(f"Count: \{len(even\_numbers)\}")

\# Method 2: Using list comprehension
odd\_list = [i for i in range(1, 51) if i \% 2 != 0]
even\_list = [i for i in range(1, 51) if i \% 2 == 0]

print(f"{n}Odd (List Comprehension): \{odd\_list[:10]\}...")  \# First 10
print(f"Even (List Comprehension): \{even\_list[:10]\}...")  \# First 10

\# Method 3: Using range with step
odd\_range = list(range(1, 51, 2))   \# Start 1, step 2
even\_range = list(range(2, 51, 2))  \# Start 2, step 2

print(f"{n}Odd (Range method): \{odd\_range[:10]\}...")
print(f"Even (Range method): \{even\_range[:10]\}...")
\end{verbatim}

\textbf{Number Classification Table:}

{\def\LTcaptype{none} % do not increment counter
\begin{longtable}[]{@{}llll@{}}
\toprule\noalign{}
Type & Condition & Range 1-10 & Count (1-50) \\
\midrule\noalign{}
\endhead
\bottomrule\noalign{}
\endlastfoot
ODD & \texttt{n\ \%\ 2\ !=\ 0} & 1,3,5,7,9 & 25 \\
EVEN & \texttt{n\ \%\ 2\ ==\ 0} & 2,4,6,8,10 & 25 \\
\end{longtable}
}

\textbf{Statistical Analysis:}

\begin{verbatim}
\# Analysis of generated lists
print(f"{n}Statistical Analysis:")
print(f"Total numbers: \{len(odd\_numbers) + len(even\_numbers)\}")
print(f"Odd percentage: \{len(odd\_numbers)/50*100\}\%")
print(f"Even percentage: \{len(even\_numbers)/50*100\}\%")
print(f"Largest odd: \{max(odd\_numbers)\}")
print(f"Largest even: \{max(even\_numbers)\}")
\end{verbatim}

\textbf{Key Techniques:}

\begin{itemize}
\tightlist
\item
  \textbf{Modulo operator}: \texttt{\%} for remainder check
\item
  \textbf{List comprehension}: Concise list creation
\item
  \textbf{Range function}: Generate sequences efficiently
\end{itemize}

\end{solutionbox}
\begin{mnemonicbox}
``Odd/Even: Remainder 1/0 When Divided by 2''

\end{mnemonicbox}
\subsection*{Question 4(a OR) [3
marks]}\label{question-4a-or-3-marks}

\textbf{Explain String Slicing with example.}

\begin{solutionbox}

\textbf{String Slicing} extracts parts of a string using
\texttt{[start:stop:step]} syntax.

\textbf{Slicing Syntax Table:}

{\def\LTcaptype{none} % do not increment counter
\begin{longtable}[]{@{}llll@{}}
\toprule\noalign{}
Syntax & Description & Example & Result \\
\midrule\noalign{}
\endhead
\bottomrule\noalign{}
\endlastfoot
\texttt{s[start:stop]} & From start to stop-1 &
\texttt{"hello"[1:4]} & ``ell'' \\
\texttt{s[start:]} & From start to end & \texttt{"hello"[2:]} &
``llo'' \\
\texttt{s[:stop]} & From beginning to stop-1 &
\texttt{"hello"[:3]} & ``hel'' \\
\texttt{s[::step]} & Every step character &
\texttt{"hello"[::2]} & ``hlo'' \\
\texttt{s[::-1]} & Reverse string & \texttt{"hello"[::-1]} &
``olleh'' \\
\end{longtable}
}

\textbf{Example Code:}

\begin{verbatim}
text = "Python Programming"

\# Basic slicing
print(f"First 6 chars: \{text[:6]\}")      \# "Python"
print(f"Last 11 chars: \{text[7:]\}")      \# "Programming"
print(f"Middle part: \{text[2:8]\}")       \# "thon P"

\# Step slicing
print(f"Every 2nd char: \{text[::2]\}")    \# "Pto rgamn"

\# Negative indexing:**
print(f"Last character: \{text[{-}1]\}")     \# "g"
print(f"Reverse: \{text[::{-}1]\}")          \# "gnimmargorP nohtyP"
\end{verbatim}

\textbf{Key Features:}

\begin{itemize}
\tightlist
\item
  \textbf{Zero-based indexing}: Start from 0
\item
  \textbf{Negative indexing}: Count from end (-1)
\item
  \textbf{Immutable}: Original string unchanged
\end{itemize}

\end{solutionbox}
\begin{mnemonicbox}
``Slice: Start, Stop, Step''

\end{mnemonicbox}
\subsection*{Question 4(b OR) [4
marks]}\label{question-4b-or-4-marks}

\textbf{Write a program using user defined function to find the
factorial of a given number.}

\begin{solutionbox}

\textbf{Factorial Function Program:}

\begin{verbatim}
def factorial(n):
    """Calculate factorial using recursion"""
if

n == 0 or

n == 1:

        return 1
    else:
        return n * factorial(n {-} 1)

def factorial\_iterative(n):
    """Calculate factorial using loop"""
    result = 1
    for i in range(1, n + 1):
        result *= i
    return result

\# Main program
number = int(input("Enter a number: "))
if number {} 0:
    print("Factorial not defined for negative numbers")
else:
    result1 = factorial(number)
    result2 = factorial\_iterative(number)
    print(f"Factorial of \{number\} = \{result1\}")
\end{verbatim}

\textbf{Factorial Table:}

{\def\LTcaptype{none} % do not increment counter
\begin{longtable}[]{@{}lll@{}}
\toprule\noalign{}
n & Factorial & Calculation \\
\midrule\noalign{}
\endhead
\bottomrule\noalign{}
\endlastfoot
0 & 1 & Base case \\
1 & 1 & Base case \\
3 & 6 & 3 \times 2 \times 1 \\
5 & 120 & 5 \times 4 \times 3 \times 2 \times 1 \\
\end{longtable}
}

\textbf{Key Concepts:}

\begin{itemize}
\tightlist
\item
  \textbf{Recursion}: Function calls itself
\item
  \textbf{Base case}: Stops recursive calls
\item
  \textbf{User-defined}: Custom function creation
\end{itemize}

\end{solutionbox}
\begin{mnemonicbox}
``Factorial: Multiply All Numbers Below''

\end{mnemonicbox}
\subsection*{Question 4(c OR) [7
marks]}\label{question-4c-or-7-marks}

\textbf{Write a user defined function to check whether a sub string is
present in a given string.}

\begin{solutionbox}

\textbf{Substring Check Function:}

\begin{verbatim}
def find\_substring(main\_string, sub\_string):
    """Check if substring exists in main string"""
    if sub\_string in main\_string:
        index = main\_string.find(sub\_string)
        return True, index
    else:
        return False, {-}1

def count\_substring(main\_string, sub\_string):
    """Count occurrences of substring"""
    return main\_string.count(sub\_string)

def find\_all\_positions(main\_string, sub\_string):
    """Find all positions of substring"""
    positions = []
    start = 0
    while True:
        pos = main\_string.find(sub\_string, start)
        if pos == {-}1:
            break
        positions.append(pos)
        start = pos + 1
    return positions

\# Main program
text = input("Enter main string: ")
search = input("Enter substring to search: ")

found, position = find\_substring(text, search)
if found:
    print(f"Substring {}\{search\}{ found at position }\{position\}")
    count = count\_substring(text, search)
    all\_pos = find\_all\_positions(text, search)
    print(f"Total occurrences: \{count\}")
    print(f"All positions: \{all\_pos\}")
else:
    print(f"Substring {}\{search\}{ not found"})
\end{verbatim}

\textbf{String Methods Table:}

{\def\LTcaptype{none} % do not increment counter
\begin{longtable}[]{@{}
  >{\raggedright\arraybackslash}p{(\linewidth - 6\tabcolsep) * \real{0.2353}}
  >{\raggedright\arraybackslash}p{(\linewidth - 6\tabcolsep) * \real{0.2647}}
  >{\raggedright\arraybackslash}p{(\linewidth - 6\tabcolsep) * \real{0.2647}}
  >{\raggedright\arraybackslash}p{(\linewidth - 6\tabcolsep) * \real{0.2353}}@{}}
\toprule\noalign{}
\begin{minipage}[b]{\linewidth}\raggedright
Method
\end{minipage} & \begin{minipage}[b]{\linewidth}\raggedright
Purpose
\end{minipage} & \begin{minipage}[b]{\linewidth}\raggedright
Example
\end{minipage} & \begin{minipage}[b]{\linewidth}\raggedright
Result
\end{minipage} \\
\midrule\noalign{}
\endhead
\bottomrule\noalign{}
\endlastfoot
\texttt{find()} & Find first position & \texttt{"hello".find("ll")} &
2 \\
\texttt{count()} & Count occurrences & \texttt{"hello".count("l")} &
2 \\
\texttt{in} & Check existence & \texttt{"ll"\ in\ "hello"} & True \\
\texttt{index()} & Find position (error if not found) &
\texttt{"hello".index("e")} & 1 \\
\end{longtable}
}

\textbf{Key Features:}

\begin{itemize}
\tightlist
\item
  \textbf{Multiple methods}: Different ways to search
\item
  \textbf{Position tracking}: Return index of found substring
\item
  \textbf{Error handling}: Check before processing
\end{itemize}

\end{solutionbox}
\begin{mnemonicbox}
``Substring: Search, Find, Count, Position''

\end{mnemonicbox}
\subsection*{Question 5(a) [3 marks]}\label{q5a}

\textbf{Explain how to create and access a List with example.}

\begin{solutionbox}

\textbf{List Creation and Access:}

\begin{verbatim}
\# Creating lists
empty\_list = []
numbers = [1, 2, 3, 4, 5]
mixed = [1, "hello", 3.14, True]
nested = [[1, 2], [3, 4], [5, 6]]

\# Accessing elements
print(f"First element: \{numbers[0]\}")      \# 1
print(f"Last element: \{numbers[{-}1]\}")      \# 5
print(f"Slice: \{numbers[1:4]\}")           \# [2, 3, 4]
\end{verbatim}

\textbf{List Access Methods:}

{\def\LTcaptype{none} % do not increment counter
\begin{longtable}[]{@{}llll@{}}
\toprule\noalign{}
Method & Syntax & Example & Result \\
\midrule\noalign{}
\endhead
\bottomrule\noalign{}
\endlastfoot
Index & \texttt{list[i]} & \texttt{[1,2,3][1]} & 2 \\
Negative & \texttt{list[-i]} & \texttt{[1,2,3][-1]} & 3 \\
Slice & \texttt{list[start:stop]} & \texttt{[1,2,3,4][1:3]}
& [2,3] \\
\end{longtable}
}

\textbf{Key Features:}

\begin{itemize}
\tightlist
\item
  \textbf{Ordered collection}: Elements have positions
\item
  \textbf{Mutable}: Can be modified after creation
\item
  \textbf{Mixed types}: Different data types allowed
\end{itemize}

\end{solutionbox}
\begin{mnemonicbox}
``Lists: Create, Index, Access''

\end{mnemonicbox}
\subsection*{Question 5(b) [4 marks]}\label{q5b}

\textbf{List out the operations that can be performed on a LIST. Write a
program to create and copy one List into another List.}

\begin{solutionbox}

\textbf{List Operations and Copy Program:}

\begin{verbatim}
\# Original list
original = [1, 2, 3, 4, 5]
print(f"Original list: \{original\}")

\# Copying methods
shallow\_copy = original.copy()
slice\_copy = original[:]
list\_copy = list(original)

\# Modify original
original.append(6)
print(f"After append: \{original\}")
print(f"Shallow copy: \{shallow\_copy\}")

\# List operations demonstration
numbers = [10, 20, 30]
numbers.append(40)          \# Add to end
numbers.insert(1, 15)       \# Insert at position
numbers.remove(20)          \# Remove specific value
popped = numbers.pop()      \# Remove and return last
\end{verbatim}

\textbf{List Operations Table:}

{\def\LTcaptype{none} % do not increment counter
\begin{longtable}[]{@{}llll@{}}
\toprule\noalign{}
Operation & Method & Example & Result \\
\midrule\noalign{}
\endhead
\bottomrule\noalign{}
\endlastfoot
Add & \texttt{append()} & \texttt{[1,2].append(3)} & [1,2,3] \\
Insert & \texttt{insert()} & \texttt{[1,3].insert(1,2)} &
[1,2,3] \\
Remove & \texttt{remove()} & \texttt{[1,2,3].remove(2)} &
[1,3] \\
Pop & \texttt{pop()} & \texttt{[1,2,3].pop()} & [1,2] \\
\end{longtable}
}

\textbf{Key Concepts:}

\begin{itemize}
\tightlist
\item
  \textbf{Shallow copy}: Independent list with same elements
\item
  \textbf{Deep copy}: Needed for nested structures
\item
  \textbf{Multiple methods}: Different copying techniques
\end{itemize}

\end{solutionbox}
\begin{mnemonicbox}
``List Operations: Add, Insert, Remove, Pop, Copy''

\end{mnemonicbox}
\subsection*{Question 5(c) [7 marks]}\label{q5c}

\textbf{List and give use of various Built in methods of LIST}

\begin{solutionbox}

\textbf{Built-in List Methods:}

\begin{verbatim}
\# Sample list for demonstrations
fruits = [{apple}, {banana}, {cherry}, {apple}]
numbers = [3, 1, 4, 1, 5, 9, 2]

\# Modification methods
fruits.append({date})              \# Add to end
fruits.insert(1, {avocado})       \# Insert at index
fruits.remove({apple})            \# Remove first occurrence
last\_fruit = fruits.pop()         \# Remove and return last
fruits.clear()                    \# Remove all elements

\# Search and count methods
fruits = [{apple}, {banana}, {apple}, {cherry}]
count = fruits.count({apple})     \# Count occurrences
index = fruits.index({banana})    \# Find first index

\# Sorting and reversing
numbers.sort()                    \# Sort in place
numbers.reverse()                 \# Reverse in place
sorted\_copy = sorted(fruits)      \# Return sorted copy

\# Extension
more\_fruits = [{grape}, {orange}]
fruits.extend(more\_fruits)        \# Add multiple items
\end{verbatim}

\textbf{List Methods Summary:}

{\def\LTcaptype{none} % do not increment counter
\begin{longtable}[]{@{}lllll@{}}
\toprule\noalign{}
Category & Method & Purpose & Returns & Modifies Original \\
\midrule\noalign{}
\endhead
\bottomrule\noalign{}
\endlastfoot
Add & \texttt{append(x)} & Add item to end & None & Yes \\
Add & \texttt{insert(i,x)} & Insert at position & None & Yes \\
Add & \texttt{extend(list)} & Add multiple items & None & Yes \\
Remove & \texttt{remove(x)} & Remove first x & None & Yes \\
Remove & \texttt{pop(i)} & Remove at index & Removed item & Yes \\
Remove & \texttt{clear()} & Remove all & None & Yes \\
Search & \texttt{index(x)} & Find position & Index & No \\
Search & \texttt{count(x)} & Count occurrences & Count & No \\
Sort & \texttt{sort()} & Sort in place & None & Yes \\
Sort & \texttt{reverse()} & Reverse order & None & Yes \\
Copy & \texttt{copy()} & Shallow copy & New list & No \\
\end{longtable}
}

\textbf{Practical Examples:}

\begin{verbatim}
\# Shopping cart example
cart = []
cart.append({milk})
cart.extend([{bread}, {eggs}, {butter}])
print(f"Items in cart: \{len(cart)\}")

if {milk} in cart:
    cart.remove({milk})
    print("Milk removed from cart")

cart.sort()
print(f"Sorted cart: \{cart\}")
\end{verbatim}

\textbf{Key Applications:}

\begin{itemize}
\tightlist
\item
  \textbf{Data management}: Add, remove, organize items
\item
  \textbf{Search operations}: Find and count elements
\item
  \textbf{Sorting}: Organize data in order
\end{itemize}

\end{solutionbox}
\begin{mnemonicbox}
``List Methods: Add, Remove, Search, Sort, Copy''

\end{mnemonicbox}
\subsection*{Question 5(a OR) [3
marks]}\label{question-5a-or-3-marks}

\textbf{Explain how to create and traverse a string by giving an
example.}

\begin{solutionbox}

\textbf{String Creation and Traversal:}

\begin{verbatim}
\# String creation methods
string1 = "Hello World"        \# Double quotes
string2 = {Python Programming} \# Single quotes
string3 = """Multi{-line}
string example"""              \# Triple quotes

\# String traversal methods
text = "Python"

\# Method 1: Using for loop
for char in text:
    print(char, end=" ")
print()

\# Method 2: Using index
for i in range(len(text)):
    print(f"\{text[i]\} at index \{i\}")

\# Method 3: Using enumerate
for index, char in enumerate(text):
    print(f"Index \{index\}: \{char\}")
\end{verbatim}

\textbf{Traversal Methods Table:}

{\def\LTcaptype{none} % do not increment counter
\begin{longtable}[]{@{}
  >{\raggedright\arraybackslash}p{(\linewidth - 4\tabcolsep) * \real{0.3077}}
  >{\raggedright\arraybackslash}p{(\linewidth - 4\tabcolsep) * \real{0.3077}}
  >{\raggedright\arraybackslash}p{(\linewidth - 4\tabcolsep) * \real{0.3846}}@{}}
\toprule\noalign{}
\begin{minipage}[b]{\linewidth}\raggedright
Method
\end{minipage} & \begin{minipage}[b]{\linewidth}\raggedright
Syntax
\end{minipage} & \begin{minipage}[b]{\linewidth}\raggedright
Use Case
\end{minipage} \\
\midrule\noalign{}
\endhead
\bottomrule\noalign{}
\endlastfoot
Direct & \texttt{for\ char\ in\ string:} & Simple character access \\
Index & \texttt{for\ i\ in\ range(len(s)):} & Need position info \\
Enumerate & \texttt{for\ i,\ char\ in\ enumerate(s):} & Both index and
character \\
\end{longtable}
}

\textbf{Key Concepts:}

\begin{itemize}
\tightlist
\item
  \textbf{Immutable}: Strings cannot be changed
\item
  \textbf{Iterable}: Can loop through characters
\item
  \textbf{Indexing}: Access individual characters
\end{itemize}

\end{solutionbox}
\begin{mnemonicbox}
``Strings: Create, Loop, Access''

\end{mnemonicbox}
\subsection*{Question 5(b OR) [4
marks]}\label{question-5b-or-4-marks}

\textbf{List out the operations that can be performed on a String. Write
a code for any 2 operations}

\begin{solutionbox}

\textbf{String Operations:}

\begin{verbatim}
\# String operations examples
text = "Python Programming"

\# Operation 1: String concatenation and formatting
first\_name = "John"
last\_name = "Doe"
full\_name = first\_name + " " + last\_name
formatted = f"Hello, \{full\_name\}!"
print(f"Concatenation: \{full\_name\}")
print(f"Formatting: \{formatted\}")

\# Operation 2: String case conversion and splitting
sentence = "learn python programming easily"
title\_case = sentence.title()
upper\_case = sentence.upper()
words = sentence.split()
print(f"Title case: \{title\_case\}")
print(f"Upper case: \{upper\_case\}")
print(f"Split words: \{words\}")
\end{verbatim}

\textbf{String Operations Table:}

{\def\LTcaptype{none} % do not increment counter
\begin{longtable}[]{@{}llll@{}}
\toprule\noalign{}
Category & Operation & Example & Result \\
\midrule\noalign{}
\endhead
\bottomrule\noalign{}
\endlastfoot
Join & Concatenation & \texttt{"Hello"\ +\ "\ World"} & ``Hello
World'' \\
Case & \texttt{upper()} & \texttt{"hello".upper()} & ``HELLO'' \\
Case & \texttt{lower()} & \texttt{"HELLO".lower()} & ``hello'' \\
Case & \texttt{title()} & \texttt{"hello\ world".title()} & ``Hello
World'' \\
Split & \texttt{split()} & \texttt{"a,b,c".split(",")} &
[`a',`b',`c'] \\
Replace & \texttt{replace()} & \texttt{"hello".replace("l","x")} &
``hexxo'' \\
Strip & \texttt{strip()} & \texttt{"\ hello\ ".strip()} & ``hello'' \\
Find & \texttt{find()} & \texttt{"hello".find("e")} & 1 \\
\end{longtable}
}

\textbf{Key Features:}

\begin{itemize}
\tightlist
\item
  \textbf{Immutable}: Operations return new strings
\item
  \textbf{Method chaining}: Combine multiple operations
\item
  \textbf{Flexible}: Many built-in operations available
\end{itemize}

\end{solutionbox}
\begin{mnemonicbox}
``String Operations: Join, Case, Split, Find''

\end{mnemonicbox}
\subsection*{Question 5(c OR) [7
marks]}\label{question-5c-or-7-marks}

\textbf{List and give use of various built -- in methods of String.}

\begin{solutionbox}

\textbf{Built-in String Methods:}

\begin{verbatim}
\# Sample string for demonstration
text = "  Python Programming Language  "
sample = "Hello World Programming"

\# Case conversion methods
print(f"Original: {}\{text\}{"})
print(f"upper(): \{text.upper()\}")
print(f"lower(): \{text.lower()\}")
print(f"title(): \{text.title()\}")
print(f"capitalize(): \{text.capitalize()\}")
print(f"swapcase(): {Hello.swapcase()}\}")

\# Whitespace methods
print(f"strip(): {}\{text.strip()\}{"})
print(f"lstrip(): {}\{text.lstrip()\}{"})
print(f"rstrip(): {}\{text.rstrip()\}{"})

\# Search and check methods
print(f"find({Python): }\{text.find({Python})\}")
print(f"count({o): }\{sample.count({o})\}")
print(f"startswith({  Py): }\{text.startswith({  Py})\}")
print(f"endswith({ge  ): }\{text.endswith({ge  })\}")

\# Character type checking
test\_string = "Python123"
print(f"isalpha(): \{{Python}.isalpha()\}")
print(f"isdigit(): \{{123}.isdigit()\}")
print(f"isalnum(): \{test\_string.isalnum()\}")

\# Split and join methods
words = sample.split()
joined = "{-"}.join(words)
print(f"split(): \{words\}")
print(f"join(): \{joined\}")

\# Replace method
replaced = sample.replace("World", "Universe")
print(f"replace(): \{replaced\}")
\end{verbatim}

\textbf{String Methods Classification:}

{\def\LTcaptype{none} % do not increment counter
\begin{longtable}[]{@{}
  >{\raggedright\arraybackslash}p{(\linewidth - 6\tabcolsep) * \real{0.2703}}
  >{\raggedright\arraybackslash}p{(\linewidth - 6\tabcolsep) * \real{0.2432}}
  >{\raggedright\arraybackslash}p{(\linewidth - 6\tabcolsep) * \real{0.2432}}
  >{\raggedright\arraybackslash}p{(\linewidth - 6\tabcolsep) * \real{0.2432}}@{}}
\toprule\noalign{}
\begin{minipage}[b]{\linewidth}\raggedright
Category
\end{minipage} & \begin{minipage}[b]{\linewidth}\raggedright
Methods
\end{minipage} & \begin{minipage}[b]{\linewidth}\raggedright
Purpose
\end{minipage} & \begin{minipage}[b]{\linewidth}\raggedright
Example
\end{minipage} \\
\midrule\noalign{}
\endhead
\bottomrule\noalign{}
\endlastfoot
Case & \texttt{upper(),\ lower(),\ title(),\ capitalize()} & Change case
& \texttt{"hello".upper()} \rightarrow ``HELLO'' \\
Whitespace & \texttt{strip(),\ lstrip(),\ rstrip()} & Remove spaces &
\texttt{"\ hi\ ".strip()} \rightarrow ``hi'' \\
Search & \texttt{find(),\ index(),\ count()} & Find substrings &
\texttt{"hello".find("e")} \rightarrow 1 \\
Check & \texttt{startswith(),\ endswith()} & Test string ends &
\texttt{"hello".startswith("h")} \rightarrow True \\
Type Check & \texttt{isalpha(),\ isdigit(),\ isalnum()} & Character
types & \texttt{"123".isdigit()} \rightarrow True \\
Split/Join & \texttt{split(),\ join()} & Break/combine &
\texttt{"a-b".split("-")} \rightarrow [`a',`b'] \\
Replace & \texttt{replace()} & Substitute text &
\texttt{"hi".replace("i","o")} \rightarrow ``ho'' \\
\end{longtable}
}

\textbf{Real-world Examples:}

\begin{verbatim}
\# Email validation example
email = "  USER@EXAMPLE.COM  "
clean\_email = email.strip().lower()
is\_valid = "@" in clean\_email and "." in clean\_email
print(f"Clean email: \{clean\_email\}")
print(f"Valid format: \{is\_valid\}")

\# Text processing example
user\_input = "python programming"
formatted\_title = user\_input.title()
word\_count = len(user\_input.split())
print(f"Formatted: \{formatted\_title\}")
print(f"Word count: \{word\_count\}")
\end{verbatim}

\textbf{Key Applications:}

\begin{itemize}
\tightlist
\item
  \textbf{Data cleaning}: Remove unwanted spaces, fix case
\item
  \textbf{Text processing}: Search, replace, split content
\item
  \textbf{Validation}: Check string format and content
\item
  \textbf{Formatting}: Prepare text for display
\end{itemize}

\end{solutionbox}
\begin{mnemonicbox}
``String Methods: Case, Clean, Check, Change''

\end{mnemonicbox}

\end{document}
