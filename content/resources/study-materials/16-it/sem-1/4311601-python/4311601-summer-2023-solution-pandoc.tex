\documentclass[10pt,a4paper]{article}

% content/resources/templates/preamble.tex
\usepackage[margin=0.6in]{geometry}
\author{Milav Dabgar}
\usepackage{amsmath,amssymb,amsthm}
\usepackage{booktabs}
\usepackage{multirow}
\usepackage{xcolor}
\usepackage{tcolorbox}
\tcbuselibrary{breakable,skins}
\usepackage[colorlinks=true,linkcolor=blue]{hyperref}
\usepackage{titlesec}
\usepackage{enumitem}
\usepackage{tikz}
\usepackage{pgfplots}
\usepackage{circuitikz}
\usepackage[version=4]{mhchem}
\usepackage{longtable}
\usepackage{array}
\usepackage{float}
\usepackage{caption}
\usepackage{listings}

\lstset{
  basicstyle=\small\ttfamily,
  breaklines=true,
  breakatwhitespace=false,
  postbreak=\mbox{\textcolor{red}{$\hookrightarrow$}\space},
  float=false,
  numbers=left,
  numberstyle=\tiny\color{gray},
  numbersep=10pt,
  xleftmargin=2em,
  keywordstyle=\color{blue},
  commentstyle=\color{green!60!black},
  stringstyle=\color{purple},
  backgroundcolor=\color{gray!5},
  showstringspaces=false,
  tabsize=2,
  captionpos=b,
  keepspaces=true,
  columns=flexible
}

\pgfplotsset{compat=1.18}
\usetikzlibrary{shapes,arrows,positioning,calc,patterns,decorations.pathmorphing,decorations.markings,arrows.meta}

% Color scheme
\definecolor{headcolor}{RGB}{0,102,204}
\definecolor{keycolor}{RGB}{220,20,60}
\definecolor{solutioncolor}{RGB}{34,139,34}
\definecolor{mnemoniccolor}{RGB}{148,0,211}
\definecolor{codecolor}{RGB}{0,0,100}

% Spacing
\setlength{\parskip}{3pt}
\setlist[itemize]{nosep}
\setlist[enumerate]{nosep}

% Title formatting
\titleformat{\section}{\Large\bfseries\color{headcolor}}{\thesection}{1em}{}
\titleformat{\subsection}{\large\bfseries\color{headcolor}}{\thesubsection}{1em}{}

% Pandoc tightlist compatibility
\providecommand{\tightlist}{%
  \setlength{\itemsep}{0pt}\setlength{\parskip}{0pt}}

% Pandoc longtable compatibility
\newcounter{none}
\def\thenone{}


% content/resources/templates/english-boxes.tex
% This file is currently empty - it exists to maintain consistency with the import structure.
% Add custom environments here if needed in the future.


\begin{document}

\begin{center}
{\Huge\bfseries\color{headcolor} Subject Name Solutions}\\[5pt]
{\LARGE 4311601 -- Summer 2023}\\[3pt]
{\large Semester 1 Study Material}\\[3pt]
{\normalsize\textit{Detailed Solutions and Explanations}}
\end{center}

\vspace{10pt}

\subsection*{Question 1(a) [3 marks]}\label{q1a}

\textbf{Explain the steps involved in problem-solving.}

\begin{solutionbox}


{\def\LTcaptype{none} % do not increment counter
\begin{longtable}[]{@{}ll@{}}
\toprule\noalign{}
Step & Description \\
\midrule\noalign{}
\endhead
\bottomrule\noalign{}
\endlastfoot
\textbf{Problem Understanding} & Read and understand the problem
clearly \\
\textbf{Analysis} & Break down the problem into smaller parts \\
\textbf{Algorithm Design} & Create step-by-step solution approach \\
\textbf{Implementation} & Code the solution using programming
language \\
\textbf{Testing} & Verify solution with different test cases \\
\textbf{Documentation} & Document the solution for future reference \\
\end{longtable}
}

\textbf{Key Points:}

\begin{itemize}
\tightlist
\item
  \textbf{Problem Definition}: Clearly identify what needs to be solved
\item
  \textbf{Input/Output}: Determine required inputs and expected outputs
\item
  \textbf{Logic Building}: Create logical flow of solution
\end{itemize}

\end{solutionbox}
\begin{mnemonicbox}
``People Always Design Implementation Tests Daily''

\end{mnemonicbox}
\subsection*{Question 1(b) [4 marks]}\label{q1b}

\textbf{Write features of Python.}

\begin{solutionbox}


{\def\LTcaptype{none} % do not increment counter
\begin{longtable}[]{@{}ll@{}}
\toprule\noalign{}
Feature & Description \\
\midrule\noalign{}
\endhead
\bottomrule\noalign{}
\endlastfoot
\textbf{Simple Syntax} & Easy to read and write code \\
\textbf{Interpreted} & No compilation needed, runs directly \\
\textbf{Platform Independent} & Runs on Windows, Mac, Linux \\
\textbf{Object-Oriented} & Supports classes and objects \\
\textbf{Large Library} & Extensive built-in modules \\
\textbf{Dynamic Typing} & No need to declare variable types \\
\end{longtable}
}

\textbf{Key Features:}

\begin{itemize}
\tightlist
\item
  \textbf{Free and Open Source}: Available for everyone to use
\item
  \textbf{High-level Language}: Close to human language
\item
  \textbf{Extensive Support}: Large community and documentation
\end{itemize}

\end{solutionbox}
\begin{mnemonicbox}
``Simple Interpreted Platform-independent
Object-oriented Libraries Dynamic''

\end{mnemonicbox}
\subsection*{Question 1(c) [7 marks]}\label{q1c}

\textbf{Draw a flowchart and write algorithm to calculate the factorial
of a given number.}

\begin{solutionbox}

\textbf{Flowchart:}

\begin{verbatim}
flowchart LR
    A[Start] {-{-} B[Input number n]}
    B {-{-} C\{n  0?\}}
    C {-{-}|Yes| D[Print {-} Invalid input]}
C {-{-}|No| E[Initialize fact = 1,

i = 1]}

    E {-{-} F\{i = n?\}}
    F {-{-}|Yes| G[fact = fact * i]}
    G {-{-} H[i = i + 1]}
    H {-{-} F}
    F {-{-}|No| I[Print fact]}
    I {-{-} J[End]}
    D {-{-} J}
\end{verbatim}

\textbf{Algorithm:}

\begin{enumerate}
\tightlist
\item
  Start
\item
  Input number n
\item
  If n \textless{} 0, print ``Invalid input'' and go to step 8
\item
Initialize fact = 1,

i = 1

\item
  While i \textless= n, do:

  \begin{itemize}
  \tightlist
  \item
    fact = fact * i
  \item
    i = i + 1
  \end{itemize}
\item
  Print fact
\item
  End
\end{enumerate}

\textbf{Key Points:}

\begin{itemize}
\tightlist
\item
  \textbf{Base Case}: 0! = 1 and 1! = 1
\item
  \textbf{Validation}: Check for negative numbers
\item
  \textbf{Loop Logic}: Multiply all numbers from 1 to n
\end{itemize}

\end{solutionbox}
\begin{mnemonicbox}
``Input Validate Initialize Loop Print''

\end{mnemonicbox}
\subsection*{Question 1(c OR) [7
marks]}\label{question-1c-or-7-marks}

\textbf{Explain relational and assignment operators with example.}

\begin{solutionbox}

\textbf{Relational Operators Table:}

{\def\LTcaptype{none} % do not increment counter
\begin{longtable}[]{@{}lll@{}}
\toprule\noalign{}
Operator & Description & Example \\
\midrule\noalign{}
\endhead
\bottomrule\noalign{}
\endlastfoot
\textbf{==} & Equal to & 5 == 5 (True) \\
\textbf{!=} & Not equal to & 5 != 3 (True) \\
\textbf{\textgreater{}} & Greater than & 7 \textgreater{} 3 (True) \\
\textbf{\textless{}} & Less than & 2 \textless{} 8 (True) \\
\textbf{\textgreater=} & Greater than or equal & 5 \textgreater= 5
(True) \\
\textbf{\textless=} & Less than or equal & 4 \textless= 6 (True) \\
\end{longtable}
}

\textbf{Assignment Operators Table:}

{\def\LTcaptype{none} % do not increment counter
\begin{longtable}[]{@{}lll@{}}
\toprule\noalign{}
Operator & Description & Example \\
\midrule\noalign{}
\endhead
\bottomrule\noalign{}
\endlastfoot
\textbf{=} & Simple assignment & x = 5 \\
\textbf{+=} & Add and assign & x += 3 (x = x + 3) \\
\textbf{-=} & Subtract and assign & x -= 2 (x = x - 2) \\
\textbf{*=} & Multiply and assign & x *= 4 (x = x * 4) \\
\textbf{/=} & Divide and assign & x /= 2 (x = x / 2) \\
\end{longtable}
}

\textbf{Code Example:}

\begin{verbatim}
\# Relational operators
a, b = 10, 5
print(a {} b)   \# True
print(a == b)  \# False

\# Assignment operators
x = 10
x += 5  \# x becomes 15
x *= 2  \# x becomes 30
\end{verbatim}

\end{solutionbox}
\begin{mnemonicbox}
``Compare Relations, Assign Values''

\end{mnemonicbox}
\subsection*{Question 2(a) [3 marks]}\label{q2a}

\textbf{Draw various symbols used for flowchart and write purpose of
each symbol.}

\begin{solutionbox}

\textbf{Flowchart Symbols Table:}

{\def\LTcaptype{none} % do not increment counter
\begin{longtable}[]{@{}lll@{}}
\toprule\noalign{}
Symbol & Name & Purpose \\
\midrule\noalign{}
\endhead
\bottomrule\noalign{}
\endlastfoot
\textbf{Oval} & Terminal & Start/End of program \\
\textbf{Rectangle} & Process & Processing operations \\
\textbf{Diamond} & Decision & Conditional statements \\
\textbf{Parallelogram} & Input/Output & Data input/output \\
\textbf{Circle} & Connector & Connect different parts \\
\textbf{Arrow} & Flow line & Direction of flow \\
\end{longtable}
}

\textbf{ASCII Diagram:}

\begin{verbatim}
   ( Start/End )     [ Process ]     { Decision }
        
   / Input/Output {     O Connector     {-}{-}{-} Flow}
\end{verbatim}

\textbf{Key Points:}

\begin{itemize}
\tightlist
\item
  \textbf{Standard Symbols}: Universally recognized shapes
\item
  \textbf{Clear Flow}: Arrows show program direction
\item
  \textbf{Logical Structure}: Helps visualize program logic
\end{itemize}

\end{solutionbox}
\begin{mnemonicbox}
``Terminals Process Decisions Input Connectors Flow''

\end{mnemonicbox}
\subsection*{Question 2(b) [4 marks]}\label{q2b}

\textbf{List out characteristics of good algorithm.}

\begin{solutionbox}


{\def\LTcaptype{none} % do not increment counter
\begin{longtable}[]{@{}ll@{}}
\toprule\noalign{}
Characteristic & Description \\
\midrule\noalign{}
\endhead
\bottomrule\noalign{}
\endlastfoot
\textbf{Finite} & Must terminate after finite steps \\
\textbf{Definite} & Each step clearly defined \\
\textbf{Input} & Zero or more inputs specified \\
\textbf{Output} & At least one output produced \\
\textbf{Effective} & Steps must be simple and feasible \\
\textbf{Unambiguous} & Each step has only one meaning \\
\end{longtable}
}

\textbf{Key Characteristics:}

\begin{itemize}
\tightlist
\item
  \textbf{Correctness}: Produces correct results for all valid inputs
\item
  \textbf{Efficiency}: Uses minimum time and space resources
\item
  \textbf{Clarity}: Easy to understand and implement
\end{itemize}

\end{solutionbox}
\begin{mnemonicbox}
``Finite Definite Input Output Effective
Unambiguous''

\end{mnemonicbox}
\subsection*{Question 2(c) [7 marks]}\label{q2c}

\textbf{Use proper data type to represent the following data values.}

\begin{solutionbox}

\textbf{Data Type Mapping Table:}

{\def\LTcaptype{none} % do not increment counter
\begin{longtable}[]{@{}lll@{}}
\toprule\noalign{}
Data Value & Data Type & Example \\
\midrule\noalign{}
\endhead
\bottomrule\noalign{}
\endlastfoot
\textbf{(1) Number of days in a week} & \textbf{int} &
\texttt{days\ =\ 7} \\
\textbf{(2) Resident of Gujarat or not} & \textbf{bool} &
\texttt{is\_resident\ =\ True} \\
\textbf{(3) Mobile number} & \textbf{str} &
\texttt{mobile\ =\ "9876543210"} \\
\textbf{(4) Bank account balance} & \textbf{float} &
\texttt{balance\ =\ 15000.50} \\
\textbf{(5) Volume of a sphere} & \textbf{float} &
\texttt{volume\ =\ 523.33} \\
\textbf{(6) Perimeter of a square} & \textbf{float} &
\texttt{perimeter\ =\ 20.0} \\
\textbf{(7) Name of the student} & \textbf{str} &
\texttt{name\ =\ "Rahul"} \\
\end{longtable}
}

\textbf{Code Example:}

\begin{verbatim}
\# Data type examples
days = 7                    \# int
is\_resident = True          \# bool
mobile = "9876543210"       \# str
balance = 15000.50          \# float
volume = 523.33            \# float
perimeter = 20.0           \# float
name = "Rahul"             \# str
\end{verbatim}

\textbf{Key Points:}

\begin{itemize}
\tightlist
\item
  \textbf{int}: Whole numbers without decimals
\item
  \textbf{float}: Numbers with decimal points
\item
  \textbf{str}: Text data in quotes
\item
  \textbf{bool}: True/False values only
\end{itemize}

\end{solutionbox}
\begin{mnemonicbox}
``Integers Float Strings Booleans''

\end{mnemonicbox}
\subsection*{Question 2(a OR) [3
marks]}\label{question-2a-or-3-marks}

\textbf{Find the output of following code.}

\begin{verbatim}
num1 = 2+9*((3*12){-}8)/10
print(num1)
\end{verbatim}

\begin{solutionbox}

\textbf{Step-by-step calculation:}

\begin{verbatim}
num1 = 2+9*((3*12){-}8)/10
\# Step 1: 3*12 = 36
\# Step 2: 36{-8 = 28}
\# Step 3: 9*28 = 252
\# Step 4: 252/10 = 25.2
\# Step 5: 2+25.2 = 27.2
\end{verbatim}

\textbf{Output:} \texttt{27.2}

\textbf{Key Points:}

\begin{itemize}
\tightlist
\item
  \textbf{BODMAS Rule}: Brackets, Orders, Division, Multiplication,
  Addition, Subtraction
\item
  \textbf{Operator Precedence}: Parentheses first, then
  multiplication/division
\item
  \textbf{Result Type}: Float due to division operation
\end{itemize}

\end{solutionbox}
\begin{mnemonicbox}
``Brackets Orders Division Multiplication Addition
Subtraction''

\end{mnemonicbox}
\subsection*{Question 2(b OR) [4
marks]}\label{question-2b-or-4-marks}

\textbf{List out the various types of operators used in Python.}

\begin{solutionbox}

\textbf{Python Operators Table:}

{\def\LTcaptype{none} % do not increment counter
\begin{longtable}[]{@{}lll@{}}
\toprule\noalign{}
Type & Operators & Example \\
\midrule\noalign{}
\endhead
\bottomrule\noalign{}
\endlastfoot
\textbf{Arithmetic} & +, -, *, /, \%, **, // & \texttt{5\ +\ 3\ =\ 8} \\
\textbf{Comparison} & ==, !=, \textgreater, \textless, \textgreater=,
\textless= & \texttt{5\ \textgreater{}\ 3\ =\ True} \\
\textbf{Logical} & and, or, not & \texttt{True\ and\ False\ =\ False} \\
\textbf{Assignment} & =, +=, -=, *=, /= & \texttt{x\ +=\ 5} \\
\textbf{Bitwise} & \&, \textbar, \^{}, \textasciitilde,
\textless\textless, \textgreater\textgreater{} &
\texttt{5\ \&\ 3\ =\ 1} \\
\textbf{Membership} & in, not in &
\texttt{\textquotesingle{}a\textquotesingle{}\ in\ \textquotesingle{}cat\textquotesingle{}\ =\ True} \\
\textbf{Identity} & is, is not & \texttt{x\ is\ y} \\
\end{longtable}
}

\textbf{Key Points:}

\begin{itemize}
\tightlist
\item
  \textbf{Arithmetic}: Mathematical operations
\item
  \textbf{Comparison}: Compare values and return boolean
\item
  \textbf{Logical}: Combine boolean expressions
\end{itemize}

\end{solutionbox}
\begin{mnemonicbox}
``Arithmetic Comparison Logical Assignment Bitwise
Membership Identity''

\end{mnemonicbox}
\subsection*{Question 2(c OR) [7
marks]}\label{question-2c-or-7-marks}

\textbf{Write a program to find the sum and average of all the positive
numbers entered by the user. As soon as the user enters a negative
number, stop taking in any further input from the user and display the
sum and average.}

\begin{solutionbox}

\textbf{Code:}

\begin{verbatim}
\# Program to find sum and average of positive numbers
total\_sum = 0
count = 0

print("Enter positive numbers (negative to stop):")

while True:
    num = float(input("Enter number: "))
    
    if num {} 0:
        break
    
    total\_sum += num
    count += 1

if count {} 0:
    average = total\_sum / count
    print(f"Sum: \{total\_sum\}")
    print(f"Average: \{average\}")
else:
    print("No positive numbers entered")
\end{verbatim}

\textbf{Key Points:}

\begin{itemize}
\tightlist
\item
  \textbf{Loop Control}: While loop with break statement
\item
  \textbf{Input Validation}: Check for negative numbers
\item
  \textbf{Division by Zero}: Handle case when no numbers entered
\end{itemize}

\end{solutionbox}
\begin{mnemonicbox}
``Input Loop Check Calculate Display''

\end{mnemonicbox}
\subsection*{Question 3(a) [3 marks]}\label{q3a}

\textbf{Explain while loop with example.}

\begin{solutionbox}

\textbf{While Loop Structure:}

\begin{verbatim}
while condition:
    \# statements
    \# update condition
\end{verbatim}

\textbf{Example:}

\begin{verbatim}
\# Print numbers 1 to 5
i = 1
while i {=} 5:
    print(i)
    i += 1
\end{verbatim}

\textbf{Key Points:}

\begin{itemize}
\tightlist
\item
  \textbf{Pre-tested Loop}: Condition checked before execution
\item
  \textbf{Infinite Loop Risk}: Condition must eventually become False
\item
  \textbf{Loop Variable}: Must be updated inside loop
\end{itemize}

\end{solutionbox}
\begin{mnemonicbox}
``While Condition True Execute''

\end{mnemonicbox}
\subsection*{Question 3(b) [4 marks]}\label{q3b}

\textbf{Write a program to find the sum of digits of an integer number,
input by the user.}

\begin{solutionbox}

\textbf{Code:}

\begin{verbatim}
\# Program to find sum of digits
num = int(input("Enter a number: "))
original\_num = num
digit\_sum = 0

while num {} 0:
    digit = num \% 10
    digit\_sum += digit
    num = num // 10

print(f"Sum of digits of \{original\_num\} is \{digit\_sum\}")
\end{verbatim}

\textbf{Key Points:}

\begin{itemize}
\tightlist
\item
  \textbf{Modulo Operation}: Extract last digit using \%10
\item
  \textbf{Integer Division}: Remove last digit using //10
\item
  \textbf{Loop Until Zero}: Continue until no digits remain
\end{itemize}

\end{solutionbox}
\begin{mnemonicbox}
``Extract Add Remove Repeat''

\end{mnemonicbox}
\subsection*{Question 3(c) [7 marks]}\label{q3c}

\textbf{Write a program to print Armstrong numbers between 100 to 10000
using a user-defined function.}

\begin{solutionbox}

\textbf{Code:}

\begin{verbatim}
def is\_armstrong(num):
    """Check if number is Armstrong number"""
    original = num
    num\_digits = len(str(num))
    sum\_powers = 0
    
    while num {} 0:
        digit = num \% 10
        sum\_powers += digit ** num\_digits
        num //= 10
    
    return sum\_powers == original

def print\_armstrong\_range(start, end):
    """Print Armstrong numbers in given range"""
    print(f"Armstrong numbers between \{start\} and \{end\}:")
    
    for num in range(start, end + 1):
        if is\_armstrong(num):
            print(num, end=" ")
    print()

\# Main program
print\_armstrong\_range(100, 10000)
\end{verbatim}

\textbf{Key Points:}

\begin{itemize}
\tightlist
\item
  \textbf{Function Definition}: def keyword to create functions
\item
  \textbf{Armstrong Logic}: Sum of digits raised to power of number of
  digits
\item
  \textbf{Range Function}: Generate numbers in specified range
\end{itemize}

\end{solutionbox}
\begin{mnemonicbox}
``Define Check Calculate Compare Print''

\end{mnemonicbox}
\subsection*{Question 3(a OR) [3
marks]}\label{question-3a-or-3-marks}

\textbf{Write a Program to print following pattern.}

\begin{verbatim}
5 4 3 2 1
4 3 2 1
3 2 1
2 1
1
\end{verbatim}

\begin{solutionbox}

\textbf{Code:}

\begin{verbatim}
\# Pattern printing program
for i in range(5, 0, {-}1):
    for j in range(i, 0, {-}1):
        print(j, end=" ")
    print()
\end{verbatim}

\textbf{Key Points:}

\begin{itemize}
\tightlist
\item
  \textbf{Nested Loops}: Outer loop for rows, inner for columns
\item
  \textbf{Reverse Range}: range(start, stop, -1) for decreasing
\item
  \textbf{Print Control}: end='' '' for space, print() for newline
\end{itemize}

\end{solutionbox}
\begin{mnemonicbox}
``Outer Inner Reverse Print''

\end{mnemonicbox}
\subsection*{Question 3(b OR) [4
marks]}\label{question-3b-or-4-marks}

\textbf{Explain nested if\ldots else statement.}

\begin{solutionbox}

\textbf{Structure:}

\begin{verbatim}
if condition1:
    if condition2:
        \# statements
    else:
        \# statements
else:
    if condition3:
        \# statements
    else:
        \# statements
\end{verbatim}

\textbf{Example:}

\begin{verbatim}
marks = 85

if marks {=} 50:
    if marks {=} 90:
        grade = "A+"
    elif marks {=} 80:
        grade = "A"
    else:
        grade = "B"
else:
    grade = "F"

print(f"Grade: \{grade\}")
\end{verbatim}

\textbf{Key Points:}

\begin{itemize}
\tightlist
\item
  \textbf{Inner Conditions}: if-else inside another if-else
\item
  \textbf{Multiple Levels}: Can nest multiple levels deep
\item
  \textbf{Logical Flow}: Inner conditions execute only if outer is true
\end{itemize}

\end{solutionbox}
\begin{mnemonicbox}
``Outer Inner Multiple Levels''

\end{mnemonicbox}
\subsection*{Question 3(c OR) [7
marks]}\label{question-3c-or-7-marks}

\textbf{Write a program to enter n numbers in list and using statistics
module find mean, median and mode.}

\begin{solutionbox}

\textbf{Code:}

\begin{verbatim}
import statistics

\# Input number of elements
n = int(input("Enter number of elements: "))
numbers = []

\# Input numbers
for i in range(n):
    num = float(input(f"Enter number \{i+1\}: "))
    numbers.append(num)

\# Calculate statistics
mean\_val = statistics.mean(numbers)
median\_val = statistics.median(numbers)

try:
    mode\_val = statistics.mode(numbers)
except statistics.StatisticsError:
    mode\_val = "No unique mode"

\# Display results
print(f"Numbers: \{numbers\}")
print(f"Mean: \{mean\_val\}")
print(f"Median: \{median\_val\}")
print(f"Mode: \{mode\_val\}")
\end{verbatim}

\textbf{Key Points:}

\begin{itemize}
\tightlist
\item
  \textbf{Statistics Module}: Built-in module for statistical functions
\item
  \textbf{List Input}: Store numbers in list for processing
\item
  \textbf{Exception Handling}: Handle mode calculation errors
\end{itemize}

\end{solutionbox}
\begin{mnemonicbox}
``Import Input Calculate Display''

\end{mnemonicbox}
\subsection*{Question 4(a) [3 marks]}\label{q4a}

\textbf{Differentiate between a for loop and a while loop in python.}

\begin{solutionbox}

\textbf{Comparison Table:}

{\def\LTcaptype{none} % do not increment counter
\begin{longtable}[]{@{}lll@{}}
\toprule\noalign{}
Feature & For Loop & While Loop \\
\midrule\noalign{}
\endhead
\bottomrule\noalign{}
\endlastfoot
\textbf{Purpose} & Known iterations & Unknown iterations \\
\textbf{Syntax} & for var in sequence & while condition \\
\textbf{Initialization} & Automatic & Manual \\
\textbf{Update} & Automatic & Manual \\
\textbf{Use Case} & Iterate over collections & Repeat until condition \\
\end{longtable}
}

\textbf{Examples:}

\begin{verbatim}
\# For loop
for i in range(5):
    print(i)

\# While loop  
i = 0
while i {} 5:
    print(i)
    i += 1
\end{verbatim}

\end{solutionbox}
\begin{mnemonicbox}
``For Known While Unknown''

\end{mnemonicbox}
\subsection*{Question 4(b) [4 marks]}\label{q4b}

\textbf{Match the following.}

\begin{solutionbox}

\textbf{Correct Matching:}

\begin{itemize}
\tightlist
\item
  \textbf{A. If statement \rightarrow 3.} Used to conditionally execute a block of
  code based on a certain condition
\item
  \textbf{B. While loop \rightarrow 1.} Executes a block of code repeatedly as
  long as a certain condition is met\\
\item
  \textbf{C. Break statement \rightarrow 5.} Terminates the current loop and moves
  on to the next iteration
\item
  \textbf{D. Continue statement \rightarrow 2.} Skips the current iteration and
  moves on to the next one
\end{itemize}

\textbf{Key Points:}

\begin{itemize}
\tightlist
\item
  \textbf{If Statement}: Conditional execution
\item
  \textbf{While Loop}: Repeated execution with condition
\item
  \textbf{Break}: Exit loop completely
\item
  \textbf{Continue}: Skip current iteration only
\end{itemize}

\end{solutionbox}
\begin{mnemonicbox}
``If Conditions While Repeats Break Exits Continue
Skips''

\end{mnemonicbox}
\subsection*{Question 4(c) [7 marks]}\label{q4c}

\textbf{Differentiate between following with the help of an example:}
\textbf{a) Argument and Parameter} \textbf{b) Global and Local variable}

\begin{solutionbox}

\textbf{a) Argument vs Parameter:}

\begin{verbatim}
def greet(name, age):  \# name, age are parameters
    print(f"Hello \{name\}, you are \{age\} years old")

greet("Raj", 20)  \# "Raj", 20 are arguments
\end{verbatim}

\textbf{b) Global vs Local Variable:}

\begin{verbatim}
x = 10  \# Global variable

def my\_function():
    y = 5  \# Local variable
    global x
    x = 15  \# Modifying global variable
    print(f"Local y: \{y\}")
    print(f"Global x: \{x\}")

my\_function()
print(f"Global x outside: \{x\}")
\end{verbatim}

\textbf{Comparison Table:}

{\def\LTcaptype{none} % do not increment counter
\begin{longtable}[]{@{}
  >{\raggedright\arraybackslash}p{(\linewidth - 6\tabcolsep) * \real{0.2000}}
  >{\raggedright\arraybackslash}p{(\linewidth - 6\tabcolsep) * \real{0.2333}}
  >{\raggedright\arraybackslash}p{(\linewidth - 6\tabcolsep) * \real{0.2667}}
  >{\raggedright\arraybackslash}p{(\linewidth - 6\tabcolsep) * \real{0.3000}}@{}}
\toprule\noalign{}
\begin{minipage}[b]{\linewidth}\raggedright
Type
\end{minipage} & \begin{minipage}[b]{\linewidth}\raggedright
Scope
\end{minipage} & \begin{minipage}[b]{\linewidth}\raggedright
Access
\end{minipage} & \begin{minipage}[b]{\linewidth}\raggedright
Example
\end{minipage} \\
\midrule\noalign{}
\endhead
\bottomrule\noalign{}
\endlastfoot
\textbf{Parameter} & Function definition & Receives values &
\texttt{def\ func(param):} \\
\textbf{Argument} & Function call & Passes values &
\texttt{func(argument)} \\
\textbf{Global} & Entire program & Everywhere & \texttt{x\ =\ 10} \\
\textbf{Local} & Inside function & Function only & \texttt{y\ =\ 5} in
function \\
\end{longtable}
}

\end{solutionbox}
\begin{mnemonicbox}
``Parameters Receive Arguments Pass Globals
Everywhere Locals Function''

\end{mnemonicbox}
\subsection*{Question 4(a OR) [3
marks]}\label{question-4a-or-3-marks}

\textbf{Find the output of following statements.}

\begin{solutionbox}

\textbf{Code Analysis:}

\begin{verbatim}
import math
(i) print(math.ceil({-}9.7))   \# Output: {-9}
(ii) print(math.floor({-}9.7)) \# Output: {-10  }
(iii) print(math.fabs({-}12.3)) \# Output: 12.3
\end{verbatim}

\textbf{Explanation:}

\begin{itemize}
\tightlist
\item
  \textbf{ceil(-9.7)}: Ceiling rounds up to nearest integer = -9
\item
  \textbf{floor(-9.7)}: Floor rounds down to nearest integer = -10
\item
  \textbf{fabs(-12.3)}: Absolute value removes negative sign = 12.3
\end{itemize}

\textbf{Key Points:}

\begin{itemize}
\tightlist
\item
  \textbf{Math Module}: Import required for mathematical functions
\item
  \textbf{Negative Numbers}: Ceiling and floor work differently with
  negatives
\item
  \textbf{Absolute Value}: Always returns positive value
\end{itemize}

\end{solutionbox}
\begin{mnemonicbox}
``Ceiling Up Floor Down Absolute Positive''

\end{mnemonicbox}
\subsection*{Question 4(b OR) [4
marks]}\label{question-4b-or-4-marks}

\textbf{Write advantages of function.}

\begin{solutionbox}

\textbf{Advantages Table:}

{\def\LTcaptype{none} % do not increment counter
\begin{longtable}[]{@{}ll@{}}
\toprule\noalign{}
Advantage & Description \\
\midrule\noalign{}
\endhead
\bottomrule\noalign{}
\endlastfoot
\textbf{Code Reusability} & Write once, use multiple times \\
\textbf{Modularity} & Break complex problems into smaller parts \\
\textbf{Easier Debugging} & Locate and fix errors easily \\
\textbf{Code Organization} & Better structure and readability \\
\textbf{Maintainability} & Easy to update and modify \\
\textbf{Reduced Complexity} & Simplify complex operations \\
\end{longtable}
}

\textbf{Key Benefits:}

\begin{itemize}
\tightlist
\item
  \textbf{Avoid Repetition}: No need to write same code again
\item
  \textbf{Team Collaboration}: Different people can work on different
  functions
\item
  \textbf{Testing}: Each function can be tested independently
\end{itemize}

\end{solutionbox}
\begin{mnemonicbox}
``Reuse Modular Debug Organize Maintain Reduce''

\end{mnemonicbox}
\subsection*{Question 4(c OR) [7
marks]}\label{question-4c-or-7-marks}

\textbf{Write a program to find the smallest and largest number in a
given list without using in built functions.}

\begin{solutionbox}

\textbf{Code:}

\begin{verbatim}
\# Program to find smallest and largest without built{-in functions}
def find\_min\_max(numbers):
    """Find minimum and maximum without built{-in functions"""}
    if not numbers:
        return None, None
    
    smallest = numbers[0]
    largest = numbers[0]
    
    for num in numbers[1:]:
        if num {} smallest:
            smallest = num
        if num {} largest:
            largest = num
    
    return smallest, largest

\# Input list
n = int(input("Enter number of elements: "))
numbers = []

for i in range(n):
    num = float(input(f"Enter number \{i+1\}: "))
    numbers.append(num)

\# Find min and max
min\_num, max\_num = find\_min\_max(numbers)

print(f"List: \{numbers\}")
print(f"Smallest number: \{min\_num\}")
print(f"Largest number: \{max\_num\}")
\end{verbatim}

\textbf{Key Points:}

\begin{itemize}
\tightlist
\item
  \textbf{Manual Comparison}: Use if conditions instead of min()/max()
\item
  \textbf{Initialize Variables}: Start with first element
\item
  \textbf{Loop Through}: Compare each element with current min/max
\end{itemize}

\end{solutionbox}
\begin{mnemonicbox}
``Initialize Compare Update Return''

\end{mnemonicbox}
\subsection*{Question 5(a) [3 marks]}\label{q5a}

\textbf{Differentiate sort() and sorted() methods for list in python.}

\begin{solutionbox}

\textbf{Comparison Table:}

{\def\LTcaptype{none} % do not increment counter
\begin{longtable}[]{@{}lll@{}}
\toprule\noalign{}
Feature & sort() & sorted() \\
\midrule\noalign{}
\endhead
\bottomrule\noalign{}
\endlastfoot
\textbf{Return Type} & None (modifies original) & New sorted list \\
\textbf{Original List} & Modified in-place & Unchanged \\
\textbf{Usage} & list.sort() & sorted(list) \\
\textbf{Memory} & Efficient & Uses extra memory \\
\end{longtable}
}

\textbf{Examples:}

\begin{verbatim}
\# sort() method
list1 = [3, 1, 4, 2]
list1.sort()
print(list1)  \# [1, 2, 3, 4]

\# sorted() function
list2 = [3, 1, 4, 2]
new\_list = sorted(list2)
print(list2)      \# [3, 1, 4, 2] (unchanged)
print(new\_list)   \# [1, 2, 3, 4]
\end{verbatim}

\end{solutionbox}
\begin{mnemonicbox}
``Sort Modifies Sorted Creates''

\end{mnemonicbox}
\subsection*{Question 5(b) [4 marks]}\label{q5b}

\textbf{Explain different way of traversing a string in python with
example.}

\begin{solutionbox}

\textbf{String Traversal Methods:}

\textbf{1. Using For Loop:}

\begin{verbatim}
text = "Python"
for char in text:
    print(char, end=" ")  \# P y t h o n
\end{verbatim}

\textbf{2. Using Index:}

\begin{verbatim}
text = "Python"
for i in range(len(text)):
    print(text[i], end=" ")  \# P y t h o n
\end{verbatim}

\textbf{3. Using While Loop:}

\begin{verbatim}
text = "Python"
i = 0
while i {} len(text):
    print(text[i], end=" ")
    i += 1
\end{verbatim}

\textbf{4. Using Enumerate:}

\begin{verbatim}
text = "Python"
for index, char in enumerate(text):
    print(f"\{index\}:\{char\}", end=" ")  \# 0:P 1:y 2:t 3:h 4:o 5:n
\end{verbatim}

\end{solutionbox}
\begin{mnemonicbox}
``For Index While Enumerate''

\end{mnemonicbox}
\subsection*{Question 5(c) [7 marks]}\label{q5c}

\textbf{Write output of following scripts.}

\begin{solutionbox}

\textbf{Output Results:}

\begin{verbatim}
(1) s = "Hello, World!"
    print(s[0:5])              \# Output: Hello

(2) lst = [1, 2, 3, 4, 5]
    print(lst[2:4])            \# Output: [3, 4]

(3) s = "python"
    print(len(s))              \# Output: 6

(4) lst = [5, 2, 3, 1, 8]
    lst.sort()                 \# lst becomes [1, 2, 3, 5, 8]

(5) s1 = "hello"
    s2 = "world"
    print(s1 + s2)             \# Output: helloworld

(6) lst = [1, 2, 3, 4, 5]
    print(sum(lst))            \# Output: 15

(7) s = "python"
    print(s[::{-}1])             \# Output: nohtyp
\end{verbatim}

\textbf{Key Points:}

\begin{itemize}
\tightlist
\item
  \textbf{Slicing}: [start:end] extracts substring/sublist
\item
  \textbf{String Length}: len() returns character count
\item
  \textbf{List Sorting}: sort() modifies list in-place
\item
  \textbf{String Concatenation}: + operator joins strings
\item
  \textbf{Sum Function}: Adds all list elements
\item
  \textbf{Reverse Slicing}: [::-1] reverses sequence
\end{itemize}

\end{solutionbox}
\begin{mnemonicbox}
``Slice Length Sort Concatenate Sum Reverse''

\end{mnemonicbox}
\subsection*{Question 5(a OR) [3
marks]}\label{question-5a-or-3-marks}

\textbf{Explain type conversion in python.}

\begin{solutionbox}

\textbf{Type Conversion Table:}

{\def\LTcaptype{none} % do not increment counter
\begin{longtable}[]{@{}lll@{}}
\toprule\noalign{}
Type & Function & Example \\
\midrule\noalign{}
\endhead
\bottomrule\noalign{}
\endlastfoot
\textbf{int()} & Convert to integer & \texttt{int("5")} \rightarrow 5 \\
\textbf{float()} & Convert to float & \texttt{float("3.14")} \rightarrow 3.14 \\
\textbf{str()} & Convert to string & \texttt{str(25)} \rightarrow ``25'' \\
\textbf{bool()} & Convert to boolean & \texttt{bool(1)} \rightarrow True \\
\textbf{list()} & Convert to list & \texttt{list("abc")} \rightarrow
[`a',`b',`c'] \\
\end{longtable}
}

\textbf{Examples:}

\begin{verbatim}
\# Implicit conversion
x = 5 + 3.2  \# int + float = float (8.2)

\# Explicit conversion
num\_str = "123"
num\_int = int(num\_str)  \# "123"  123
\end{verbatim}

\textbf{Key Points:}

\begin{itemize}
\tightlist
\item
  \textbf{Implicit}: Python automatically converts
\item
  \textbf{Explicit}: Programmer manually converts using functions
\item
  \textbf{Type Safety}: Some conversions may raise errors
\end{itemize}

\end{solutionbox}
\begin{mnemonicbox}
``Implicit Automatic Explicit Manual''

\end{mnemonicbox}
\subsection*{Question 5(b OR) [4
marks]}\label{question-5b-or-4-marks}

\textbf{Explain concatenation and repetition operation on string with
example.}

\begin{solutionbox}

\textbf{String Operations:}

\textbf{1. Concatenation (+):}

\begin{verbatim}
str1 = "Hello"
str2 = "World"
result = str1 + " " + str2
print(result)  \# Hello World

\# Multiple concatenation
name = "Python"
version = "3.9"
info = "Language: " + name + " Version: " + version
print(info)  \# Language: Python Version: 3.9
\end{verbatim}

**2. Repetition (*):**

\begin{verbatim}
text = "Hi! "
repeated = text * 3
print(repeated)  \# Hi! Hi! Hi! 

\# Pattern creation
pattern = "{-"} * 10
print(pattern)  \# {-{-}{-}{-}{-}{-}{-}{-}{-}{-}}
\end{verbatim}

\textbf{Key Points:}

\begin{itemize}
\tightlist
\item
  \textbf{Concatenation}: Joins strings together using +
\item
  \textbf{Repetition}: Repeats string n times using *
\item
  \textbf{Immutable}: Original strings remain unchanged
\end{itemize}

\end{solutionbox}
\begin{mnemonicbox}
``Plus Joins Star Repeats''

\end{mnemonicbox}
\subsection*{Question 5(c OR) [7
marks]}\label{question-5c-or-7-marks}

\textbf{Write a program to count and display the number of vowels,
consonants, uppercase, lowercase characters in a string.}

\begin{solutionbox}

\textbf{Code:}

\begin{verbatim}
def analyze\_string(text):
    """Analyze string for different character types"""
    vowels = "aeiouAEIOU"
    
    vowel\_count = 0
    consonant\_count = 0
    uppercase\_count = 0
    lowercase\_count = 0
    
    for char in text:
        if char.isalpha():  \# Check if character is alphabet
            if char in vowels:
                vowel\_count += 1
            else:
                consonant\_count += 1
            
            if char.isupper():
                uppercase\_count += 1
            elif char.islower():
                lowercase\_count += 1
    
    return vowel\_count, consonant\_count, uppercase\_count, lowercase\_count

\# Input string
text = input("Enter a string: ")

\# Analyze string
vowels, consonants, uppercase, lowercase = analyze\_string(text)

\# Display results
print(f"String: {}\{text\}{"})
print(f"Vowels: \{vowels\}")
print(f"Consonants: \{consonants\}")
print(f"Uppercase: \{uppercase\}")
print(f"Lowercase: \{lowercase\}")
\end{verbatim}

\textbf{Key Points:}

\begin{itemize}
\tightlist
\item
  \textbf{Character Classification}: Use isalpha(), isupper(), islower()
\item
  \textbf{Vowel Check}: Compare with vowel string
\item
  \textbf{Loop Processing}: Check each character individually
\end{itemize}

\end{solutionbox}
\begin{mnemonicbox}
``Check Classify Count Display''

\end{mnemonicbox}

\end{document}
