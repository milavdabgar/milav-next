\documentclass[10pt,a4paper]{article}

% content/resources/templates/preamble.tex
\usepackage[margin=0.6in]{geometry}
\author{Milav Dabgar}
\usepackage{amsmath,amssymb,amsthm}
\usepackage{booktabs}
\usepackage{multirow}
\usepackage{xcolor}
\usepackage{tcolorbox}
\tcbuselibrary{breakable,skins}
\usepackage[colorlinks=true,linkcolor=blue]{hyperref}
\usepackage{titlesec}
\usepackage{enumitem}
\usepackage{tikz}
\usepackage{pgfplots}
\usepackage{circuitikz}
\usepackage[version=4]{mhchem}
\usepackage{longtable}
\usepackage{array}
\usepackage{float}
\usepackage{caption}
\usepackage{listings}

\lstset{
  basicstyle=\small\ttfamily,
  breaklines=true,
  breakatwhitespace=false,
  postbreak=\mbox{\textcolor{red}{$\hookrightarrow$}\space},
  float=false,
  numbers=left,
  numberstyle=\tiny\color{gray},
  numbersep=10pt,
  xleftmargin=2em,
  keywordstyle=\color{blue},
  commentstyle=\color{green!60!black},
  stringstyle=\color{purple},
  backgroundcolor=\color{gray!5},
  showstringspaces=false,
  tabsize=2,
  captionpos=b,
  keepspaces=true,
  columns=flexible
}

\pgfplotsset{compat=1.18}
\usetikzlibrary{shapes,arrows,positioning,calc,patterns,decorations.pathmorphing,decorations.markings,arrows.meta}

% Color scheme
\definecolor{headcolor}{RGB}{0,102,204}
\definecolor{keycolor}{RGB}{220,20,60}
\definecolor{solutioncolor}{RGB}{34,139,34}
\definecolor{mnemoniccolor}{RGB}{148,0,211}
\definecolor{codecolor}{RGB}{0,0,100}

% Spacing
\setlength{\parskip}{3pt}
\setlist[itemize]{nosep}
\setlist[enumerate]{nosep}

% Title formatting
\titleformat{\section}{\Large\bfseries\color{headcolor}}{\thesection}{1em}{}
\titleformat{\subsection}{\large\bfseries\color{headcolor}}{\thesubsection}{1em}{}

% Pandoc tightlist compatibility
\providecommand{\tightlist}{%
  \setlength{\itemsep}{0pt}\setlength{\parskip}{0pt}}

% Pandoc longtable compatibility
\newcounter{none}
\def\thenone{}


% content/resources/templates/gujarati-boxes.tex
\usepackage{fontspec}
\usepackage{polyglossia}

% Set Gujarati as main language (document is primarily in Gujarati)
% Note: gloss-gujarati.ldf doesn't exist in polyglossia, but it will use hyphenation patterns
\setdefaultlanguage{gujarati}
\setotherlanguage{english}

% Configure Gujarati font properly
% Use Language=Default to prevent polyglossia from trying to add language-specific features
% that don't exist for Gujarati, which causes "empty feature" warnings
\newfontfamily\gujaratifont[Script=Gujarati,AutoFakeBold=2.5,AutoFakeSlant=0.3]{Noto Sans Gujarati}
\setmainfont[Script=Gujarati,AutoFakeBold=2.5,AutoFakeSlant=0.3]{Noto Sans Gujarati}
% Use Noto Sans Gujarati for monospace to support Gujarati in text
\setmonofont[Scale=0.9]{Noto Sans Gujarati}

% Configure English to use the same font
\newfontfamily\englishfont[Script=Gujarati,AutoFakeBold=2.5,AutoFakeSlant=0.3]{Noto Sans Gujarati}

% Translations for polyglossia
\gappto\captionsgujarati{
  \renewcommand{\tablename}{કોષ્ટક}
  \renewcommand{\figurename}{આકૃતિ}
}

% Helper for TikZ nodes to ensure Gujarati font
\newcommand{\gu}[1]{{\gujaratifont #1}}

% Custom environments
\newtcolorbox{solutionbox}{
    breakable,
    enhanced,
    colback=solutioncolor!5!white,
    colframe=solutioncolor!75!black,
    fonttitle=\bfseries,
    title=જવાબ
}

\newtcolorbox{solutionboxnobreak}{
 colback=solutioncolor!5!white,
 colframe=solutioncolor!75!black,
 fonttitle=\bfseries,
 title=જવાબ
}

\newtcolorbox{keyformula}{
 breakable,
 enhanced,
 colback=keycolor!5!white,
 colframe=keycolor!75!black,
 fonttitle=\bfseries,
 title=રાસાયણિક સમીકરણ/સૂત્ર
}

\newtcolorbox{mnemonicbox}{
 breakable,
 enhanced,
 colback=mnemoniccolor!5!white,
 colframe=mnemoniccolor!75!black,
 fonttitle=\bfseries,
 title=મેમરી ટ્રીક
}


\begin{document}

\begin{center}
{\Huge\bfseries\color{headcolor} Subject Name (Gujarati)}\\[5pt]
{\LARGE 4311601 -- Winter 2024}\\[3pt]
{\large Semester 1 Study Material}\\[3pt]
{\normalsize\textit{Detailed Solutions and Explanations}}
\end{center}

\vspace{10pt}

\subsection*{પ્રશ્ન 1(અ) [3
ગુણ]}\label{uxaaauxab0uxab6uxaa8-1uxa85-3-uxa97uxaa3}

\textbf{પ્રોબ્લમ સોલવિંગ, અલ્ગોરિધમ અને સ્યુડો કોડ વ્યાખ્યાયિત કરો.}

\begin{solutionbox}

{\def\LTcaptype{none} % do not increment counter
\begin{longtable}[]{@{}
  >{\raggedright\arraybackslash}p{(\linewidth - 2\tabcolsep) * \real{0.3333}}
  >{\raggedright\arraybackslash}p{(\linewidth - 2\tabcolsep) * \real{0.6667}}@{}}
\toprule\noalign{}
\begin{minipage}[b]{\linewidth}\raggedright
શબ્દ
\end{minipage} & \begin{minipage}[b]{\linewidth}\raggedright
વ્યાખ્યા
\end{minipage} \\
\midrule\noalign{}
\endhead
\bottomrule\noalign{}
\endlastfoot
\textbf{પ્રોબ્લમ સોલવિંગ} & તર્કસંગત વિચારસરણી વાપરીને જટિલ સમસ્યાઓનાં ઉકેલ
શોધવાની પદ્ધતિ \\
\textbf{અલ્ગોરિધમ} & મર્યાદિત ઓપરેશન સાથે સમસ્યા ઉકેલવાની પગલું-દર-પગલું
પ્રક્રિયા \\
\textbf{સ્યુડો કોડ} & સામાન્ય અંગ્રેજી જેવા syntax નો ઉપયોગ કરીને program logic
નું અનૌપચારિક વર્ણન \\
\end{longtable}
}

\begin{itemize}
\tightlist
\item
  \textbf{પ્રોબ્લમ સોલવિંગ}: જટિલ સમસ્યાઓને વ્યવસ્થિત પગલાઓમાં વહેંચવું
\item
  \textbf{અલ્ગોરિધમ}: મર્યાદિત, નિશ્ચિત, અસરકારક અને યોગ્ય આઉટપુટ આપતું હોવું જોઈએ
\item
  \textbf{સ્યુડો કોડ}: માનવ ભાષા અને programming કોડ વચ્ચેનો સેતુ
\end{itemize}

\end{solutionbox}
\begin{mnemonicbox}
``PAP - Problem, Algorithm, Pseudo''

\end{mnemonicbox}
\subsection*{પ્રશ્ન 1(બ) [4
ગુણ]}\label{uxaaauxab0uxab6uxaa8-1uxaac-4-uxa97uxaa3}

\textbf{ફ્લોચાર્ટના જુદા જુદા સિમ્બોલ સમજાવો. બે નંબર માંથી મહત્તમ નંબર શોધતો
ફ્લોચાર્ટ ડિઝાઇન કરો.}

\begin{solutionbox}

{\def\LTcaptype{none} % do not increment counter
\begin{longtable}[]{@{}lll@{}}
\toprule\noalign{}
સિમ્બોલ & આકાર & હેતુ \\
\midrule\noalign{}
\endhead
\bottomrule\noalign{}
\endlastfoot
\textbf{અંડાકાર} & ⬭ & શરૂઆત/અંત \\
\textbf{લંબચોરસ} & ▭ & પ્રક્રિયા/ક્રિયા \\
\textbf{હીરો} & ◊ & નિર્ણય \\
\textbf{સમાંતર ચતુષ્કોણ} & ▱ & ઇનપુટ/આઉટપુટ \\
\end{longtable}
}

\textbf{બે નંબરના મહત્તમ માટે ફ્લોચાર્ટ:}

\begin{verbatim}
flowchart LR
    A([શરૂઆત]) {-{-} B[/A, B ઇનપુટ કરો/]}
    B {-{-} C\{A  B?\}}
    C {-{-}|હા| D[Max = A]}
    C {-{-}|ના| E[Max = B]}
    D {-{-} F[/Max દર્શાવો/]}
    E {-{-} F}
    F {-{-} G([અંત])}
\end{verbatim}

\begin{itemize}
\tightlist
\item
  \textbf{શરૂઆત/અંત}: પ્રવેશ અને બહાર નીકળવાના બિંદુઓ
\item
  \textbf{ઇનપુટ/આઉટપુટ}: ડેટા ફ્લો ઓપરેશન્સ
\item
  \textbf{નિર્ણય}: શરતી branching
\item
  \textbf{પ્રક્રિયા}: ગણતરીના પગલાં
\end{itemize}

\end{solutionbox}
\begin{mnemonicbox}
``SIPO - Start, Input, Process, Output''

\end{mnemonicbox}
\subsection*{પ્રશ્ન 1(ક) [7
ગુણ]}\label{uxaaauxab0uxab6uxaa8-1uxa95-7-uxa97uxaa3}

\textbf{પાયથોનના વિવિધ એરિથમેટિક ઓપરેટરોની યાદી બનાવો. વિવિધ એરિથમેટિક
ઓપરેશન્સ માટેનો Python કોડ લખો.}

\begin{solutionbox}

{\def\LTcaptype{none} % do not increment counter
\begin{longtable}[]{@{}llll@{}}
\toprule\noalign{}
ઓપરેટર & સિમ્બોલ & ઉદાહરણ & પરિણામ \\
\midrule\noalign{}
\endhead
\bottomrule\noalign{}
\endlastfoot
\textbf{ઉમેરો} & + & 5 + 3 & 8 \\
\textbf{બાદબાકી} & - & 5 - 3 & 2 \\
\textbf{ગુણાકાર} & * & 5 * 3 & 15 \\
\textbf{ભાગાકાર} & / & 5 / 3 & 1.667 \\
\textbf{ફ્લોર ડિવિઝન} & // & 5 // 3 & 1 \\
\textbf{મોડ્યુલસ} & \% & 5 \% 3 & 2 \\
\textbf{ઘાત} & ** & 5 ** 3 & 125 \\
\end{longtable}
}

\textbf{કોડ:}

\begin{verbatim}
a = 10
b = 3
print(f"ઉમેરો: \{a + b\}")
print(f"બાદબાકી: \{a {-} b\}")
print(f"ગુણાકાર: \{a * b\}")
print(f"ભાગાકાર: \{a / b\}")
print(f"ફ્લોર ડિવિઝન: \{a // b\}")
print(f"મોડ્યુલસ: \{a \% b\}")
print(f"ઘાત: \{a ** b\}")
\end{verbatim}

\end{solutionbox}
\begin{mnemonicbox}
``Add-Sub-Mul-Div-Floor-Mod-Pow''

\end{mnemonicbox}
\subsection*{પ્રશ્ન 1(ક OR) [7
ગુણ]}\label{uxaaauxab0uxab6uxaa8-1uxa95-or-7-uxa97uxaa3}

\textbf{પાયથોનના વિવિધ કંપેરિઝન ઓપરેટરોની યાદી બનાવો. વિવિધ કંપેરિઝન ઓપરેશન્સ
માટેનો Python કોડ લખો.}

\begin{solutionbox}

{\def\LTcaptype{none} % do not increment counter
\begin{longtable}[]{@{}llll@{}}
\toprule\noalign{}
ઓપરેટર & સિમ્બોલ & હેતુ & ઉદાહરણ \\
\midrule\noalign{}
\endhead
\bottomrule\noalign{}
\endlastfoot
\textbf{સમાન} & == & સમાનતા ચકાસો & 5 == 3 \rightarrow False \\
\textbf{અસમાન} & != & અસમાનતા ચકાસો & 5 != 3 \rightarrow True \\
\textbf{મોટું} & \textgreater{} & મોટું ચકાસો & 5 \textgreater{} 3 \rightarrow True \\
\textbf{નાનું} & \textless{} & નાનું ચકાસો & 5 \textless{} 3 \rightarrow False \\
\textbf{મોટું સમાન} & \textgreater= & મોટું/સમાન ચકાસો & 5 \textgreater= 3 \rightarrow
True \\
\textbf{નાનું સમાન} & \textless= & નાનું/સમાન ચકાસો & 5 \textless= 3 \rightarrow
False \\
\end{longtable}
}

\textbf{કોડ:}

\begin{verbatim}
x = 8
y = 5
print(f"સમાન: \{x == y\}")
print(f"અસમાન: \{x != y\}")
print(f"મોટું: \{x {} y\}")
print(f"નાનું: \{x {} y\}")
print(f"મોટું સમાન: \{x {=} y\}")
print(f"નાનું સમાન: \{x {=} y\}")
\end{verbatim}

\end{solutionbox}
\begin{mnemonicbox}
``Equal-Not-Greater-Less-GreaterEqual-LessEqual''

\end{mnemonicbox}
\subsection*{પ્રશ્ન 2(અ) [3
ગુણ]}\label{uxaaauxab0uxab6uxaa8-2uxa85-3-uxa97uxaa3}

\textbf{મેમ્બરશિપ ઓપરેટર્સ ઉપર ટૂંક નોંધ લખો.}

\begin{solutionbox}

{\def\LTcaptype{none} % do not increment counter
\begin{longtable}[]{@{}lll@{}}
\toprule\noalign{}
ઓપરેટર & હેતુ & ઉદાહરણ \\
\midrule\noalign{}
\endhead
\bottomrule\noalign{}
\endlastfoot
\textbf{in} & એલિમેન્ટ અસ્તિત્વ ચકાસો & `a' in `apple' \rightarrow True \\
\textbf{not in} & એલિમેન્ટ અનસ્તિત્વ ચકાસો & `z' not in `apple' \rightarrow True \\
\end{longtable}
}

\begin{itemize}
\tightlist
\item
  \textbf{in ઓપરેટર}: જો એલિમેન્ટ sequence માં મળે તો True આપે
\item
  \textbf{not in ઓપરેટર}: જો એલિમેન્ટ sequence માં ન મળે તો True આપે
\item
  \textbf{ઉપયોગ}: Lists, strings, tuples, dictionaries માં
\end{itemize}

\end{solutionbox}
\begin{mnemonicbox}
``In-Not-In for membership testing''

\end{mnemonicbox}
\subsection*{પ્રશ્ન 2(બ) [4
ગુણ]}\label{uxaaauxab0uxab6uxaa8-2uxaac-4-uxa97uxaa3}

\textbf{પાયથોન વ્યાખ્યાયિત કરો. પાયથોન પ્રોગ્રામિંગની વિવિધ એપ્લિકેશનો લખો.}

\begin{solutionbox}

\textbf{પાયથોન વ્યાખ્યા}: સરળતા અને વાંચનીયતા માટે જાણીતી high-level,
interpreted programming language.

{\def\LTcaptype{none} % do not increment counter
\begin{longtable}[]{@{}ll@{}}
\toprule\noalign{}
એપ્લિકેશન ક્ષેત્ર & ઉદાહરણો \\
\midrule\noalign{}
\endhead
\bottomrule\noalign{}
\endlastfoot
\textbf{વેબ ડેવલપમેન્ટ} & Django, Flask frameworks \\
\textbf{ડેટા સાયન્સ} & NumPy, Pandas, Matplotlib \\
\textbf{AI/ML} & TensorFlow, Scikit-learn \\
\textbf{ડેસ્કટોપ એપ્સ} & Tkinter, PyQt \\
\textbf{ગેમ ડેવલપમેન્ટ} & Pygame library \\
\end{longtable}
}

\begin{itemize}
\tightlist
\item
  \textbf{Interpreted}: compilation ની જરૂર નથી
\item
  \textbf{Cross-platform}: બહુવિધ OS પર ચાલે છે
\item
  \textbf{વિશાળ libraries}: વ્યાપક standard library
\end{itemize}

\end{solutionbox}
\begin{mnemonicbox}
``Web-Data-AI-Desktop-Games''

\end{mnemonicbox}
\subsection*{પ્રશ્ન 2(ક) [7
ગુણ]}\label{uxaaauxab0uxab6uxaa8-2uxa95-7-uxa97uxaa3}

\textbf{પાયથોન પ્રોગ્રામ લખો જે નીચેની વિગતોનો ઉપયોગ કરીને વીજળી બિલની ગણતરી
કરે છે.}

\begin{solutionbox}

\textbf{દરોનું ટેબલ:}

{\def\LTcaptype{none} % do not increment counter
\begin{longtable}[]{@{}ll@{}}
\toprule\noalign{}
યુનિટ રેન્જ & દર પ્રતિ યુનિટ \\
\midrule\noalign{}
\endhead
\bottomrule\noalign{}
\endlastfoot
\leq 100 & રૂ 5.00 \\
101-200 & રૂ 7.50 \\
201-300 & રૂ 10.00 \\
\geq 301 & રૂ 15.00 \\
\end{longtable}
}

\textbf{કોડ:}

\begin{verbatim}
units = int(input("વપરાયેલ યુનિટ્સ દાખલ કરો: "))

if units {=} 100:
    bill = units * 5.00
elif units {=} 200:
    bill = units * 7.50
elif units {=} 300:
    bill = units * 10.00
else:
    bill = units * 15.00

print(f"કુલ બિલ: રૂ \{bill\}")
\end{verbatim}

\begin{itemize}
\tightlist
\item
  \textbf{શરતી તર્ક}: if-elif-else structure
\item
  \textbf{દર ગણતરી}: યુનિટ slabs આધારિત
\item
  \textbf{યુઝર ઇનપુટ}: interactive billing system
\end{itemize}

\end{solutionbox}
\begin{mnemonicbox}
``Input-Check-Calculate-Display''

\end{mnemonicbox}
\subsection*{પ્રશ્ન 2(અ OR) [3
ગુણ]}\label{uxaaauxab0uxab6uxaa8-2uxa85-or-3-uxa97uxaa3}

\textbf{આઇડેન્ટિટી ઓપરેટર્સ ઉપર ટૂંક નોંધ લખો.}

\begin{solutionbox}

{\def\LTcaptype{none} % do not increment counter
\begin{longtable}[]{@{}lll@{}}
\toprule\noalign{}
ઓપરેટર & હેતુ & ઉદાહરણ \\
\midrule\noalign{}
\endhead
\bottomrule\noalign{}
\endlastfoot
\textbf{is} & સમાન ઓબ્જેક્ટ ચકાસો & a is b \\
\textbf{is not} & જુદા ઓબ્જેક્ટ ચકાસો & a is not b \\
\end{longtable}
}

\begin{itemize}
\tightlist
\item
  \textbf{is ઓપરેટર}: ઓબ્જેક્ટ identity સરખાવે, values નહીં
\item
  \textbf{is not ઓપરેટર}: ઓબ્જેક્ટ્સ જુદા છે કે નહીં ચકાસે
\item
  \textbf{મેમરી સરખામણી}: સમાન મેમરી સ્થાન ચકાસે
\end{itemize}

\end{solutionbox}
\begin{mnemonicbox}
``Is-IsNot for object identity''

\end{mnemonicbox}
\subsection*{પ્રશ્ન 2(બ OR) [4
ગુણ]}\label{uxaaauxab0uxab6uxaa8-2uxaac-or-4-uxa97uxaa3}

\textbf{પાયથોનમાં ઇન્ડેન્ટેશન શું છે? પાયથોનની વિવિધ વિશેષતાઓ સમજાવો.}

\begin{solutionbox}

\textbf{ઇન્ડેન્ટેશન}: કોડ બ્લોક્સ વ્યાખ્યાયિત કરવા માટે લાઇનની શરૂઆતમાં whitespace.

{\def\LTcaptype{none} % do not increment counter
\begin{longtable}[]{@{}ll@{}}
\toprule\noalign{}
વિશેષતા & વર્ણન \\
\midrule\noalign{}
\endhead
\bottomrule\noalign{}
\endlastfoot
\textbf{સરળ Syntax} & વાંચવા અને લખવામાં સરળ \\
\textbf{Interpreted} & compilation step નથી \\
\textbf{Object-Oriented} & OOP concepts સપોર્ટ કરે \\
\textbf{Cross-Platform} & બહુવિધ OS પર ચાલે \\
\textbf{વિશાળ Library} & વ્યાપક standard library \\
\end{longtable}
}

\begin{itemize}
\tightlist
\item
  \textbf{ઇન્ડેન્ટેશન}: curly braces \{\} ને બદલે છે
\item
  \textbf{સુસંગત}: સામાન્ય રીતે પ્રતિ level 4 spaces
\item
  \textbf{ફરજિયાત}: કોડ માળખું બનાવે છે
\end{itemize}

\end{solutionbox}
\begin{mnemonicbox}
``Simple-Interpreted-Object-Cross-Large''

\end{mnemonicbox}
\subsection*{પ્રશ્ન 2(ક OR) [7
ગુણ]}\label{uxaaauxab0uxab6uxaa8-2uxa95-or-7-uxa97uxaa3}

\textbf{પાયથોન પ્રોગ્રામ લખો જે નીચેની વિગતોનો ઉપયોગ કરીને વિદ્યાર્થીના
વર્ગ/ગ્રેડની ગણતરી કરતો પાયથોન પ્રોગ્રામ લખો.}

\begin{solutionbox}

\textbf{ગ્રેડિંગ ટેબલ:}

{\def\LTcaptype{none} % do not increment counter
\begin{longtable}[]{@{}ll@{}}
\toprule\noalign{}
ટકાવારી & ગ્રેડ \\
\midrule\noalign{}
\endhead
\bottomrule\noalign{}
\endlastfoot
\geq 70 & ડિસ્ટિંક્શન \\
60-69 & ફર્સ્ટ ક્લાસ \\
50-59 & સેકન્ડ ક્લાસ \\
35-49 & પાસ ક્લાસ \\
\textless{} 35 & નિષ્ફળ \\
\end{longtable}
}

\textbf{કોડ:}

\begin{verbatim}
percentage = float(input("ટકાવારી દાખલ કરો: "))

if percentage {=} 70:
    grade = "ડિસ્ટિંક્શન"
elif percentage {=} 60:
    grade = "ફર્સ્ટ ક્લાસ"
elif percentage {=} 50:
    grade = "સેકન્ડ ક્લાસ"
elif percentage {=} 35:
    grade = "પાસ ક્લાસ"
else:
    grade = "નિષ્ફળ"

print(f"ગ્રેડ: \{grade\}")
\end{verbatim}

\begin{itemize}
\tightlist
\item
  \textbf{બહુવિધ શરતો}: Nested if-elif structure
\item
  \textbf{ગ્રેડ નિર્ધારણ}: ટકાવારી ranges આધારિત
\item
  \textbf{Float ઇનપુટ}: દશાંશ ટકાવારી handle કરે
\end{itemize}

\end{solutionbox}
\begin{mnemonicbox}
``Distinction-First-Second-Pass-Fail''

\end{mnemonicbox}
\subsection*{પ્રશ્ન 3(અ) [3
ગુણ]}\label{uxaaauxab0uxab6uxaa8-3uxa85-3-uxa97uxaa3}

\textbf{સિલેક્શન કંટ્રોલ સ્ટેટમેન્ટ શું છે? તેની યાદી બનાવો.}

\begin{solutionbox}

{\def\LTcaptype{none} % do not increment counter
\begin{longtable}[]{@{}ll@{}}
\toprule\noalign{}
સ્ટેટમેન્ટ પ્રકાર & હેતુ \\
\midrule\noalign{}
\endhead
\bottomrule\noalign{}
\endlastfoot
\textbf{if} & એક શરત ચકાસણી \\
\textbf{if-else} & બે-માર્ગી branching \\
\textbf{if-elif-else} & બહુ-માર્ગી branching \\
\textbf{nested if} & શરતોની અંદર શરતો \\
\end{longtable}
}

\begin{itemize}
\tightlist
\item
  \textbf{Selection statements}: શરતો આધારે program flow control કરે
\item
  \textbf{Boolean evaluation}: True/False logic વાપરે
\item
  \textbf{Branching}: execution ના જુદા રસ્તાઓ
\end{itemize}

\end{solutionbox}
\begin{mnemonicbox}
``If-IfElse-IfElif-Nested''

\end{mnemonicbox}
\subsection*{પ્રશ્ન 3(બ) [4
ગુણ]}\label{uxaaauxab0uxab6uxaa8-3uxaac-4-uxa97uxaa3}

\textbf{નેસ્ટેડ લૂપ ઉપર ટૂંક નોંધ લખો.}

\begin{solutionbox}

{\def\LTcaptype{none} % do not increment counter
\begin{longtable}[]{@{}ll@{}}
\toprule\noalign{}
લૂપ પ્રકાર & માળખું \\
\midrule\noalign{}
\endhead
\bottomrule\noalign{}
\endlastfoot
\textbf{બાહ્ય લૂપ} & iterations control કરે \\
\textbf{આંતરિક લૂપ} & દરેક બાહ્ય iteration માટે સંપૂર્ણ execute થાય \\
\textbf{કુલ Iterations} & બાહ્ય \times આંતરિક \\
\end{longtable}
}

\begin{itemize}
\tightlist
\item
  \textbf{Nested માળખું}: બીજા લૂપની અંદર લૂપ
\item
  \textbf{સંપૂર્ણ execution}: આંતરિક લૂપ પૂરું થાય પછી બાહ્ય આગળ વધે
\item
  \textbf{Pattern creation}: 2D structures માટે ઉપયોગી
\end{itemize}

\textbf{કોડ ઉદાહરણ:}

\begin{verbatim}
for i in range(3):
    for j in range(2):
print(f"i=\{i\},

j=\{j\}")

\end{verbatim}

\end{solutionbox}
\begin{mnemonicbox}
``Outer-Inner-Complete-Pattern''

\end{mnemonicbox}
\subsection*{પ્રશ્ન 3(ક) [7
ગુણ]}\label{uxaaauxab0uxab6uxaa8-3uxa95-7-uxa97uxaa3}

\textbf{યુઝર ડિફાઇન ફંક્શન લખો જે 1 થી 100 સુધીની બધી સંખ્યાઓ દર્શાવે, જે 4 થી
વિભાજ્ય છે.}

\begin{solutionbox}

\textbf{કોડ:}

\begin{verbatim}
def display\_divisible\_by\_4():
    print("1 થી 100 સુધીની 4 થી વિભાજ્ય સંખ્યાઓ:")
    for num in range(1, 101):
        if num \% 4 == 0:
            print(num, end=" ")
    print()

\# ફંક્શન કૉલ
display\_divisible\_by\_4()
\end{verbatim}

\textbf{Return સાથે વિકલ્પ:}

\begin{verbatim}
def get\_divisible\_by\_4():
    return [num for num in range(1, 101) if num \% 4 == 0]

result = get\_divisible\_by\_4()
print(result)
\end{verbatim}

\begin{itemize}
\tightlist
\item
  \textbf{ફંક્શન વ્યાખ્યા}: def keyword નો ઉપયોગ
\item
  \textbf{Range ફંક્શન}: 1 થી 100 iteration
\item
  \textbf{Modulus ચકાસણી}: num \% 4 == 0 શરત
\item
  \textbf{List comprehension}: વૈકલ્પિક અભિગમ
\end{itemize}

\end{solutionbox}
\begin{mnemonicbox}
``Define-Range-Check-Display''

\end{mnemonicbox}
\subsection*{પ્રશ્ન 3(અ OR) [3
ગુણ]}\label{uxaaauxab0uxab6uxaa8-3uxa85-or-3-uxa97uxaa3}

\textbf{રિપીટેશન કંટ્રોલ સ્ટેટમેન્ટ શું છે? તેની યાદી બનાવો.}

\begin{solutionbox}

{\def\LTcaptype{none} % do not increment counter
\begin{longtable}[]{@{}ll@{}}
\toprule\noalign{}
સ્ટેટમેન્ટ પ્રકાર & હેતુ \\
\midrule\noalign{}
\endhead
\bottomrule\noalign{}
\endlastfoot
\textbf{for loop} & જાણીતી સંખ્યાના iterations \\
\textbf{while loop} & શરત આધારિત repetition \\
\textbf{nested loop} & લૂપની અંદર લૂપ \\
\end{longtable}
}

\begin{itemize}
\tightlist
\item
  \textbf{Repetition statements}: કોડ બ્લોક્સ વારંવાર execute કરે
\item
  \textbf{Iteration control}: looping ની જુદી પદ્ધતિઓ
\item
  \textbf{Loop variables}: iteration progress track કરે
\end{itemize}

\end{solutionbox}
\begin{mnemonicbox}
``For-While-Nested''

\end{mnemonicbox}
\subsection*{પ્રશ્ન 3(બ OR) [4
ગુણ]}\label{uxaaauxab0uxab6uxaa8-3uxaac-or-4-uxa97uxaa3}

\textbf{break અને continue સ્ટેટમેન્ટ વચ્ચેનો તફાવત આપો.}

\begin{solutionbox}

{\def\LTcaptype{none} % do not increment counter
\begin{longtable}[]{@{}lll@{}}
\toprule\noalign{}
પાસું & break & continue \\
\midrule\noalign{}
\endhead
\bottomrule\noalign{}
\endlastfoot
\textbf{હેતુ} & લૂપ સંપૂર્ણ બહાર નીકળો & વર્તમાન iteration છોડો \\
\textbf{Execution} & લૂપમાંથી બહાર jump કરે & આગલા iteration પર jump કરે \\
\textbf{ઉપયોગ} & લૂપ જલ્દી સમાપ્ત કરો & ખાસ શરતો છોડો \\
\textbf{અસર} & લૂપ સમાપ્ત થાય & લૂપ ચાલુ રહે \\
\end{longtable}
}

\textbf{કોડ ઉદાહરણ:}

\begin{verbatim}
\# break ઉદાહરણ
for i in range(5):
if

i == 3:

        break
    print(i)  \# આઉટપુટ: 0, 1, 2

\# continue ઉદાહરણ  
for i in range(5):
if

i == 2:

        continue
    print(i)  \# આઉટપુટ: 0, 1, 3, 4
\end{verbatim}

\end{solutionbox}
\begin{mnemonicbox}
``Break-Exit, Continue-Skip''

\end{mnemonicbox}
\subsection*{પ્રશ્ન 3(ક OR) [7
ગુણ]}\label{uxaaauxab0uxab6uxaa8-3uxa95-or-7-uxa97uxaa3}

\textbf{યુઝર ડિફાઇન ફંક્શન લખો જે 1 થી 100 સુધીની બધી બેકી સંખ્યાઓ દર્શાવે.}

\begin{solutionbox}

\textbf{કોડ:}

\begin{verbatim}
def display\_even\_numbers():
    print("1 થી 100 સુધીની બેકી સંખ્યાઓ:")
    for num in range(2, 101, 2):
        print(num, end=" ")
    print()

\# વૈકલ્પિક પદ્ધતિ
def display\_even\_alt():
    even\_nums = []
    for num in range(1, 101):
        if num \% 2 == 0:
            even\_nums.append(num)
    print(even\_nums)

\# ફંક્શન કૉલ
display\_even\_numbers()
\end{verbatim}

\begin{itemize}
\tightlist
\item
  \textbf{કાર્યક્ષમ range}: બેકી સંખ્યાઓ માટે range(2, 101, 2)
\item
  \textbf{Modulus પદ્ધતિ}: \% 2 == 0 સાથે વૈકલ્પિક ચકાસણી
\item
  \textbf{ફંક્શન ડિઝાઇન}: પુનઃઉપયોગી કોડ બ્લોક
\end{itemize}

\end{solutionbox}
\begin{mnemonicbox}
``Range-Step-Even-Display''

\end{mnemonicbox}
\subsection*{પ્રશ્ન 4(અ) [3
ગુણ]}\label{uxaaauxab0uxab6uxaa8-4uxa85-3-uxa97uxaa3}

\textbf{ફંક્શન વ્યાખ્યાયિત કરો. પાયથોનમાં ઉપલબ્ધ વિવિધ પ્રકારના ફંક્શનની યાદી
આપો.}

\begin{solutionbox}

\textbf{ફંક્શન}: ખાસ કાર્ય કરતો પુનઃઉપયોગી કોડ બ્લોક.

{\def\LTcaptype{none} % do not increment counter
\begin{longtable}[]{@{}ll@{}}
\toprule\noalign{}
ફંક્શન પ્રકાર & વર્ણન \\
\midrule\noalign{}
\endhead
\bottomrule\noalign{}
\endlastfoot
\textbf{Built-in} & પૂર્વ-નિર્ધારિત ફંક્શન્સ (print, len) \\
\textbf{User-defined} & પ્રોગ્રામર દ્વારા બનાવાયેલ \\
\textbf{Lambda} & અનામ એક-લાઇન ફંક્શન્સ \\
\textbf{Recursive} & પોતાને call કરતા ફંક્શન્સ \\
\end{longtable}
}

\begin{itemize}
\tightlist
\item
  \textbf{કોડ પુનઃઉપયોગ}: એકવાર લખો, ઘણીવાર વાપરો
\item
  \textbf{મોડ્યુલારિટી}: જટિલ સમસ્યાઓને નાના ભાગોમાં વહેંચવી
\item
  \textbf{Parameters}: ફંક્શન્સ માટે ઇનપુટ values
\end{itemize}

\end{solutionbox}
\begin{mnemonicbox}
``Built-User-Lambda-Recursive''

\end{mnemonicbox}
\subsection*{પ્રશ્ન 4(બ) [4
ગુણ]}\label{uxaaauxab0uxab6uxaa8-4uxaac-4-uxa97uxaa3}

\textbf{વેરિએબલના સ્કોપ ઉપર ટૂંક નોંધ લખો.}

\begin{solutionbox}

{\def\LTcaptype{none} % do not increment counter
\begin{longtable}[]{@{}lll@{}}
\toprule\noalign{}
સ્કોપ પ્રકાર & વર્ણન & ઉદાહરણ \\
\midrule\noalign{}
\endhead
\bottomrule\noalign{}
\endlastfoot
\textbf{Local} & ફંક્શનની અંદર જ & ફંક્શન variables \\
\textbf{Global} & સમગ્ર પ્રોગ્રામમાં & Module-level variables \\
\textbf{Built-in} & Python keywords & print, len, type \\
\end{longtable}
}

\textbf{કોડ ઉદાહરણ:}

\begin{verbatim}
x = 10  \# Global variable

def my\_function():
    y = 20  \# Local variable
    print(x)  \# Global access
    print(y)  \# Local access

my\_function()
\# print(y)  \# Error: y accessible નથી
\end{verbatim}

\begin{itemize}
\tightlist
\item
  \textbf{Variable accessibility}: variables ક્યાં વાપરી શકાય
\item
  \textbf{LEGB rule}: Local, Enclosing, Global, Built-in
\end{itemize}

\end{solutionbox}
\begin{mnemonicbox}
``Local-Global-Builtin''

\end{mnemonicbox}
\subsection*{પ્રશ્ન 4(ક) [7
ગુણ]}\label{uxaaauxab0uxab6uxaa8-4uxa95-7-uxa97uxaa3}

\textbf{Python કોડ લખો જે ઉપભોક્તાને મુખ્ય સ્ટ્રિંગ અને સબસ્ટ્રિંગ માટે પૂછે છે અને મુખ્ય
સ્ટ્રિંગમાં સબસ્ટ્રિંગની મેમ્બરશિપ તપાસે છે.}

\begin{solutionbox}

\textbf{કોડ:}

\begin{verbatim}
def check\_substring():
    main\_string = input("મુખ્ય સ્ટ્રિંગ દાખલ કરો: ")
    substring = input("સબસ્ટ્રિંગ દાખલ કરો: ")
    
    if substring in main\_string:
        print(f"{}\{substring\}{ મળ્યું }\{main\_string\}{ માં"})
        print(f"સ્થિતિ: \{main\_string.find(substring)\}")
    else:
        print(f"{}\{substring\}{ મળ્યું નથી }\{main\_string\}{ માં"})

\# વિસ્તૃત વર્ઝન case handling સાથે
def check\_substring\_enhanced():
    main\_string = input("મુખ્ય સ્ટ્રિંગ દાખલ કરો: ")
    substring = input("સબસ્ટ્રિંગ દાખલ કરો: ")
    
    if substring.lower() in main\_string.lower():
        print("સબસ્ટ્રિંગ મળ્યું (case{-insensitive)"})
    else:
        print("સબસ્ટ્રિંગ મળ્યું નથી")

check\_substring()
\end{verbatim}

\begin{itemize}
\tightlist
\item
  \textbf{યુઝર ઇન્ટરેક્શન}: string collection માટે input()
\item
  \textbf{Membership testing}: `in' operator નો ઉપયોગ
\item
  \textbf{Case sensitivity}: વૈકલ્પિક case handling
\end{itemize}

\end{solutionbox}
\begin{mnemonicbox}
``Input-Check-Report-Position''

\end{mnemonicbox}
\subsection*{પ્રશ્ન 4(અ OR) [3
ગુણ]}\label{uxaaauxab0uxab6uxaa8-4uxa85-or-3-uxa97uxaa3}

\textbf{લોકલ વેરિએબલ અને ગ્લોબલ વેરિએબલ શું છે?}

\begin{solutionbox}

{\def\LTcaptype{none} % do not increment counter
\begin{longtable}[]{@{}llll@{}}
\toprule\noalign{}
વેરિએબલ પ્રકાર & સ્કોપ & આયુષ્ય & પ્રવેશ \\
\midrule\noalign{}
\endhead
\bottomrule\noalign{}
\endlastfoot
\textbf{Local} & ફક્ત ફંક્શનમાં & ફંક્શન execution & મર્યાદિત \\
\textbf{Global} & સમગ્ર પ્રોગ્રામ & પ્રોગ્રામ execution & વ્યાપક \\
\end{longtable}
}

\textbf{ઉદાહરણ:}

\begin{verbatim}
global\_var = 100  \# Global

def function():
    local\_var = 50  \# Local
    print(global\_var)  \# ✓ સુલભ
    print(local\_var)   \# ✓ સુલભ

print(global\_var)  \# ✓ સુલભ
\# print(local\_var)  \# ✗ Error
\end{verbatim}

\begin{itemize}
\tightlist
\item
  \textbf{Local variables}: ફંક્શન્સની અંદર બનાવાયેલ
\item
  \textbf{Global variables}: ફંક્શન્સની બહાર બનાવાયેલ
\end{itemize}

\end{solutionbox}
\begin{mnemonicbox}
``Local-Limited, Global-Everywhere''

\end{mnemonicbox}
\subsection*{પ્રશ્ન 4(બ OR) [4
ગુણ]}\label{uxaaauxab0uxab6uxaa8-4uxaac-or-4-uxa97uxaa3}

\textbf{પાયથોનના કોઈપણ ચાર બિલ્ટ-ઇન ફંક્શન સમજાવો.}

\begin{solutionbox}

{\def\LTcaptype{none} % do not increment counter
\begin{longtable}[]{@{}lll@{}}
\toprule\noalign{}
ફંક્શન & હેતુ & ઉદાહરણ \\
\midrule\noalign{}
\endhead
\bottomrule\noalign{}
\endlastfoot
\textbf{len()} & લંબાઈ આપે & len(``hello'') \rightarrow 5 \\
\textbf{type()} & ડેટા ટાઇપ આપે & type(10) \rightarrow \textless class
`int'\textgreater{} \\
\textbf{input()} & યુઝર ઇનપુટ લે & name = input(``નામ:'') \\
\textbf{print()} & આઉટપુટ દર્શાવે & print(``હેલો'') \\
\end{longtable}
}

\textbf{વધારાના ઉદાહરણો:}

\begin{verbatim}
\# len() ફંક્શન
print(len([1, 2, 3, 4]))  \# આઉટપુટ: 4

\# type() ફંક્શન  
print(type(3.14))  \# આઉટપુટ: {class float}

\# input() ફંક્શન
age = input("ઉંમર દાખલ કરો: ")

\# print() ફંક્શન
print("તમારી ઉંમર છે:", age)
\end{verbatim}

\end{solutionbox}
\begin{mnemonicbox}
``Length-Type-Input-Print''

\end{mnemonicbox}
\subsection*{પ્રશ્ન 4(ક OR) [7
ગુણ]}\label{uxaaauxab0uxab6uxaa8-4uxa95-or-7-uxa97uxaa3}

\textbf{Python કોડ લખો જે આપેલ સ્ટ્રિંગમાં સબસ્ટ્રિંગને શોધે છે.}

\begin{solutionbox}

\textbf{કોડ:}

\begin{verbatim}
def locate\_substring():
    main\_string = input("મુખ્ય સ્ટ્રિંગ દાખલ કરો: ")
    substring = input("શોધવા માટે સબસ્ટ્રિંગ દાખલ કરો: ")
    
    \# પદ્ધતિ 1: find() વાપરીને
    position = main\_string.find(substring)
    if position != {-}1:
        print(f"index પર મળ્યું: \{position\}")
    else:
        print("સબસ્ટ્રિંગ મળ્યું નથી")
    
    \# પદ્ધતિ 2: index() exception handling સાથે
    try:
        position = main\_string.index(substring)
        print(f"index પર સ્થિત: \{position\}")
    except ValueError:
        print("સબસ્ટ્રિંગ મળ્યું નથી")
    
    \# પદ્ધતિ 3: બધી occurrences શોધો
    positions = []
    start = 0
    while True:
        pos = main\_string.find(substring, start)
        if pos == {-}1:
            break
        positions.append(pos)
        start = pos + 1
    
    if positions:
        print(f"બધી સ્થિતિઓ: \{positions\}")

locate\_substring()
\end{verbatim}

\begin{itemize}
\tightlist
\item
  \textbf{find() method}: index આપે અથવા -1
\item
  \textbf{index() method}: index આપે અથવા exception raise કરે
\item
  \textbf{બહુવિધ occurrences}: બધી સ્થિતિઓ શોધવા માટે લૂપ
\end{itemize}

\end{solutionbox}
\begin{mnemonicbox}
``Find-Index-Exception-Multiple''

\end{mnemonicbox}
\subsection*{પ્રશ્ન 5(અ) [3
ગુણ]}\label{uxaaauxab0uxab6uxaa8-5uxa85-3-uxa97uxaa3}

\textbf{સ્ટ્રિંગ વ્યાખ્યાયિત કરો. વિવિધ સ્ટ્રિંગ ઓપરેશન્સની યાદી બનાવો.}

\begin{solutionbox}

\textbf{સ્ટ્રિંગ}: quotes માં બંધ characters ની sequence.

{\def\LTcaptype{none} % do not increment counter
\begin{longtable}[]{@{}lll@{}}
\toprule\noalign{}
ઓપરેશન & મેથડ & ઉદાહરણ \\
\midrule\noalign{}
\endhead
\bottomrule\noalign{}
\endlastfoot
\textbf{સંયોજન} & + & ``Hello'' + ``World'' \\
\textbf{પુનરાવર્તન} & * & ``Hi'' * 3 \\
\textbf{સ્લાઇસિંગ} & [start:end] & ``Hello''[1:4] \\
\textbf{લંબાઈ} & len() & len(``Hello'') \\
\textbf{કેસ} & upper(), lower() & ``hello''.upper() \\
\end{longtable}
}

\begin{itemize}
\tightlist
\item
  \textbf{Immutable}: સ્ટ્રિંગ બનાવ્યા પછી બદલી શકાતી નથી
\item
  \textbf{Indexing}: વ્યક્તિગત characters access કરવું
\item
  \textbf{Methods}: manipulation માટે built-in functions
\end{itemize}

\end{solutionbox}
\begin{mnemonicbox}
``Concat-Repeat-Slice-Length-Case''

\end{mnemonicbox}
\subsection*{પ્રશ્ન 5(બ) [4
ગુણ]}\label{uxaaauxab0uxab6uxaa8-5uxaac-4-uxa97uxaa3}

\textbf{આપણે કેવી રીતે ઓળખી શકીએ કે એલિમેન્ટ એ લિસ્ટનો સભ્ય છે કે નહીં? યોગ્ય ઉદાહરણ
સાથે સમજાવો.}

\begin{solutionbox}

{\def\LTcaptype{none} % do not increment counter
\begin{longtable}[]{@{}lll@{}}
\toprule\noalign{}
પદ્ધતિ & ઓપરેટર & પરિણામ \\
\midrule\noalign{}
\endhead
\bottomrule\noalign{}
\endlastfoot
\textbf{in} & element in list & True/False \\
\textbf{not in} & element not in list & True/False \\
\textbf{count()} & list.count(element) & occurrences ની સંખ્યા \\
\end{longtable}
}

\textbf{ઉદાહરણ:}

\begin{verbatim}
fruits = ["apple", "banana", "orange", "mango"]

\# {in ઓપરેટર વાપરીને}
if "apple" in fruits:
    print("Apple ઉપલબ્ધ છે")

\# {not in ઓપરેટર વાપરીને  }
if "grapes" not in fruits:
    print("Grapes ઉપલબ્ધ નથી")

\# count() method વાપરીને
count = fruits.count("apple")
if count {} 0:
    print(f"Apple \{count\} વખત મળ્યું")
\end{verbatim}

\begin{itemize}
\tightlist
\item
  \textbf{Boolean પરિણામ}: મળે તો True, નહીં તો False
\item
  \textbf{Case sensitive}: ``Apple'' \neq ``apple''
\item
  \textbf{કાર્યક્ષમતા}: `in' ઓપરેટર સૌથી સામાન્ય
\end{itemize}

\end{solutionbox}
\begin{mnemonicbox}
``In-NotIn-Count for membership''

\end{mnemonicbox}
\subsection*{પ્રશ્ન 5(ક) [7
ગુણ]}\label{uxaaauxab0uxab6uxaa8-5uxa95-7-uxa97uxaa3}

\textbf{Python કોડ લખો જે આપેલ સ્ટ્રિંગના બીજા સબસ્ટ્રિંગ સાથે સબસ્ટ્રિંગને બદલે છે.
આપેલ સ્ટ્રિંગ `Welcome to GTU' તરીકે ધ્યાનમાં લો અને સબસ્ટ્રિંગ `GTU' ને `Gujarat
Technological University' સાથે બદલો.}

\begin{solutionbox}

\textbf{કોડ:}

\begin{verbatim}
def replace\_substring():
    \# આપેલ સ્ટ્રિંગ
    original = "Welcome to GTU"
    old\_substring = "GTU"
    new\_substring = "Gujarat Technological University"
    
    \# પદ્ધતિ 1: replace() વાપરીને
    result1 = original.replace(old\_substring, new\_substring)
    print(f"મૂળ: \{original\}")
    print(f"બદલાયેલ: \{result1\}")
    
    \# પદ્ધતિ 2: મેન્યુઅલ રિપ્લેસમેન્ટ
    if old\_substring in original:
        index = original.find(old\_substring)
        result2 = original[:index] + new\_substring + original[index + len(old\_substring):]
        print(f"મેન્યુઅલ પદ્ધતિ: \{result2\}")
    
    \# પદ્ધતિ 3: બધી occurrences બદલો
    test\_string = "GTU offers GTU degree from GTU"
    result3 = test\_string.replace("GTU", "Gujarat Technological University")
    print(f"બહુવિધ બદલાવ: \{result3\}")

replace\_substring()
\end{verbatim}

\textbf{આઉટપુટ:}

\begin{verbatim}
મૂળ: Welcome to GTU
બદલાયેલ: Welcome to Gujarat Technological University
\end{verbatim}

\begin{itemize}
\tightlist
\item
  \textbf{replace() method}: built-in string function
\item
  \textbf{Slicing method}: મેન્યુઅલ string manipulation
\item
  \textbf{બધી occurrences}: દરેક instance બદલે છે
\end{itemize}

\end{solutionbox}
\begin{mnemonicbox}
``Find-Replace-Slice-All''

\end{mnemonicbox}
\subsection*{પ્રશ્ન 5(અ OR) [3
ગુણ]}\label{uxaaauxab0uxab6uxaa8-5uxa85-or-3-uxa97uxaa3}

\textbf{લિસ્ટ વ્યાખ્યાયિત કરો. વિવિધ લિસ્ટ ઓપરેશન્સની યાદી બનાવો.}

\begin{solutionbox}

\textbf{લિસ્ટ}: ક્રમબદ્ધ items નો collection જે modify કરી શકાય છે.

{\def\LTcaptype{none} % do not increment counter
\begin{longtable}[]{@{}lll@{}}
\toprule\noalign{}
ઓપરેશન & મેથડ & ઉદાહરણ \\
\midrule\noalign{}
\endhead
\bottomrule\noalign{}
\endlastfoot
\textbf{ઉમેરો} & append(), insert() & list.append(item) \\
\textbf{દૂર કરો} & remove(), pop() & list.remove(item) \\
\textbf{પ્રવેશ} & [index] & list[0] \\
\textbf{સ્લાઇસ} & [start:end] & list[1:3] \\
\textbf{સોર્ટ} & sort() & list.sort() \\
\end{longtable}
}

\begin{itemize}
\tightlist
\item
  \textbf{Mutable}: લિસ્ટ બનાવ્યા પછી બદલી શકાય છે
\item
  \textbf{Indexed}: સ્થિતિ દ્વારા elements access કરાય છે
\item
  \textbf{Dynamic}: કદ વધી અથવા ઘટી શકે છે
\end{itemize}

\end{solutionbox}
\begin{mnemonicbox}
``Add-Remove-Access-Slice-Sort''

\end{mnemonicbox}
\subsection*{પ્રશ્ન 5(બ OR) [4
ગુણ]}\label{uxaaauxab0uxab6uxaa8-5uxaac-or-4-uxa97uxaa3}

\textbf{સ્ટ્રિંગ સ્લાઇસિંગ ઉપર ટૂંક નોંધ લખો. યોગ્ય ઉદાહરણ સાથે સમજાવો.}

\begin{solutionbox}

\textbf{સ્ટ્રિંગ સ્લાઇસિંગ}: [start:end:step] વાપરીને string ના ભાગો
extract કરવું.

{\def\LTcaptype{none} % do not increment counter
\begin{longtable}[]{@{}lll@{}}
\toprule\noalign{}
Syntax & વર્ણન & ઉદાહરણ \\
\midrule\noalign{}
\endhead
\bottomrule\noalign{}
\endlastfoot
\textbf{[start:]} & start થી અંત સુધી & ``Hello''[1:] \rightarrow
``ello'' \\
\textbf{[:end]} & શરૂઆત થી end સુધી & ``Hello''[:3] \rightarrow ``Hel'' \\
\textbf{[start:end]} & ખાસ રેન્જ & ``Hello''[1:4] \rightarrow ``ell'' \\
\textbf{[::-1]} & રિવર્સ સ્ટ્રિંગ & ``Hello''[::-1] \rightarrow ``olleH'' \\
\end{longtable}
}

\textbf{ઉદાહરણ:}

\begin{verbatim}
text = "Python Programming"

print(text[0:6])    \# "Python"
print(text[7:])     \# "Programming"  
print(text[:6])     \# "Python"
print(text[::2])    \# "Pto rgamn"
print(text[::{-}1])   \# "gnimmargorP nohtyP"
\end{verbatim}

\begin{itemize}
\tightlist
\item
  \textbf{Negative indexing}: છેલ્લા character માટે -1
\item
  \textbf{Step parameter}: increment control કરે છે
\end{itemize}

\end{solutionbox}
\begin{mnemonicbox}
``Start-End-Step for slicing''

\end{mnemonicbox}
\subsection*{પ્રશ્ન 5(ક OR) [7
ગુણ]}\label{uxaaauxab0uxab6uxaa8-5uxa95-or-7-uxa97uxaa3}

\textbf{Python કોડ લખો જે લિસ્ટમાં સ્પેસિફાઇડ એલિમેન્ટ કેટલી વખત દેખાય છે તેની
ગણતરી કરે છે.}

\begin{solutionbox}

\textbf{કોડ:}

\begin{verbatim}
def count\_element\_occurrences():
    \# નમૂના લિસ્ટ બનાવો
    numbers = [1, 2, 3, 2, 4, 2, 5, 2, 6]
    element = int(input("ગણવા માટે એલિમેન્ટ દાખલ કરો: "))
    
    \# પદ્ધતિ 1: count() method વાપરીને
    count1 = numbers.count(element)
    print(f"count() વાપરીને: \{element\} એ \{count1\} વખત દેખાય છે")
    
    \# પદ્ધતિ 2: મેન્યુઅલ ગણતરી
    count2 = 0
    for num in numbers:
        if num == element:
            count2 += 1
    print(f"મેન્યુઅલ ગણતરી: \{element\} એ \{count2\} વખત દેખાય છે")
    
    \# પદ્ધતિ 3: List comprehension
count3 = len([x for x in numbers if

x == element])

    print(f"List comprehension: \{element\} એ \{count3\} વખત દેખાય છે")
    
    \# પદ્ધતિ 4: કોઈપણ પ્રકારની લિસ્ટ માટે
    mixed\_list = [1, "hello", 3.14, "hello", True, "hello"]
    element\_str = input("મિશ્ર લિસ્ટમાં શોધવા માટે એલિમેન્ટ દાખલ કરો: ")
    count4 = mixed\_list.count(element\_str)
    print(f"મિશ્ર લિસ્ટમાં: {}\{element\_str\}{ એ }\{count4\} વખત દેખાય છે")

count\_element\_occurrences()
\end{verbatim}

\begin{itemize}
\tightlist
\item
  \textbf{count() method}: built-in list function
\item
  \textbf{મેન્યુઅલ iteration}: ગણતરી માટે loops વાપરવું
\item
  \textbf{List comprehension}: ગણતરીની Pythonic રીત
\item
  \textbf{Type flexibility}: કોઈપણ ડેટા ટાઇપ સાથે કામ કરે
\end{itemize}

\end{solutionbox}
\begin{mnemonicbox}
``Count-Manual-Comprehension-Flexible''

\end{mnemonicbox}

\end{document}
