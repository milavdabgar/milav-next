\documentclass[10pt,a4paper]{article}

% content/resources/templates/preamble.tex
\usepackage[margin=0.6in]{geometry}
\author{Milav Dabgar}
\usepackage{amsmath,amssymb,amsthm}
\usepackage{booktabs}
\usepackage{multirow}
\usepackage{xcolor}
\usepackage{tcolorbox}
\tcbuselibrary{breakable,skins}
\usepackage[colorlinks=true,linkcolor=blue]{hyperref}
\usepackage{titlesec}
\usepackage{enumitem}
\usepackage{tikz}
\usepackage{pgfplots}
\usepackage{circuitikz}
\usepackage[version=4]{mhchem}
\usepackage{longtable}
\usepackage{array}
\usepackage{float}
\usepackage{caption}
\usepackage{listings}

\lstset{
  basicstyle=\small\ttfamily,
  breaklines=true,
  breakatwhitespace=false,
  postbreak=\mbox{\textcolor{red}{$\hookrightarrow$}\space},
  float=false,
  numbers=left,
  numberstyle=\tiny\color{gray},
  numbersep=10pt,
  xleftmargin=2em,
  keywordstyle=\color{blue},
  commentstyle=\color{green!60!black},
  stringstyle=\color{purple},
  backgroundcolor=\color{gray!5},
  showstringspaces=false,
  tabsize=2,
  captionpos=b,
  keepspaces=true,
  columns=flexible
}

\pgfplotsset{compat=1.18}
\usetikzlibrary{shapes,arrows,positioning,calc,patterns,decorations.pathmorphing,decorations.markings,arrows.meta}

% Color scheme
\definecolor{headcolor}{RGB}{0,102,204}
\definecolor{keycolor}{RGB}{220,20,60}
\definecolor{solutioncolor}{RGB}{34,139,34}
\definecolor{mnemoniccolor}{RGB}{148,0,211}
\definecolor{codecolor}{RGB}{0,0,100}

% Spacing
\setlength{\parskip}{3pt}
\setlist[itemize]{nosep}
\setlist[enumerate]{nosep}

% Title formatting
\titleformat{\section}{\Large\bfseries\color{headcolor}}{\thesection}{1em}{}
\titleformat{\subsection}{\large\bfseries\color{headcolor}}{\thesubsection}{1em}{}

% Pandoc tightlist compatibility
\providecommand{\tightlist}{%
  \setlength{\itemsep}{0pt}\setlength{\parskip}{0pt}}

% Pandoc longtable compatibility
\newcounter{none}
\def\thenone{}


% content/resources/templates/gujarati-boxes.tex
\usepackage{fontspec}
\usepackage{polyglossia}

% Set Gujarati as main language (document is primarily in Gujarati)
% Note: gloss-gujarati.ldf doesn't exist in polyglossia, but it will use hyphenation patterns
\setdefaultlanguage{gujarati}
\setotherlanguage{english}

% Configure Gujarati font properly
% Use Language=Default to prevent polyglossia from trying to add language-specific features
% that don't exist for Gujarati, which causes "empty feature" warnings
\newfontfamily\gujaratifont[Script=Gujarati,AutoFakeBold=2.5,AutoFakeSlant=0.3]{Noto Sans Gujarati}
\setmainfont[Script=Gujarati,AutoFakeBold=2.5,AutoFakeSlant=0.3]{Noto Sans Gujarati}
% Use Noto Sans Gujarati for monospace to support Gujarati in text
\setmonofont[Scale=0.9]{Noto Sans Gujarati}

% Configure English to use the same font
\newfontfamily\englishfont[Script=Gujarati,AutoFakeBold=2.5,AutoFakeSlant=0.3]{Noto Sans Gujarati}

% Translations for polyglossia
\gappto\captionsgujarati{
  \renewcommand{\tablename}{કોષ્ટક}
  \renewcommand{\figurename}{આકૃતિ}
}

% Helper for TikZ nodes to ensure Gujarati font
\newcommand{\gu}[1]{{\gujaratifont #1}}

% Custom environments
\newtcolorbox{solutionbox}{
    breakable,
    enhanced,
    colback=solutioncolor!5!white,
    colframe=solutioncolor!75!black,
    fonttitle=\bfseries,
    title=જવાબ
}

\newtcolorbox{solutionboxnobreak}{
 colback=solutioncolor!5!white,
 colframe=solutioncolor!75!black,
 fonttitle=\bfseries,
 title=જવાબ
}

\newtcolorbox{keyformula}{
 breakable,
 enhanced,
 colback=keycolor!5!white,
 colframe=keycolor!75!black,
 fonttitle=\bfseries,
 title=રાસાયણિક સમીકરણ/સૂત્ર
}

\newtcolorbox{mnemonicbox}{
 breakable,
 enhanced,
 colback=mnemoniccolor!5!white,
 colframe=mnemoniccolor!75!black,
 fonttitle=\bfseries,
 title=મેમરી ટ્રીક
}


\begin{document}

\begin{center}
{\Huge\bfseries\color{headcolor} Subject Name (Gujarati)}\\[5pt]
{\LARGE 4311601 -- Summer 2024}\\[3pt]
{\large Semester 1 Study Material}\\[3pt]
{\normalsize\textit{Detailed Solutions and Explanations}}
\end{center}

\vspace{10pt}

\subsection*{પ્રશ્ન 1(અ) [3
ગુણ]}\label{uxaaauxab0uxab6uxaa8-1uxa85-3-uxa97uxaa3}

\textbf{સમસ્યાનું નિરાકરણ વ્યાખ્યાયિત કરો અને સમસ્યા હલ કરવાના પગલાંની સૂચિ
બનાવો.}

\begin{solutionbox}
સમસ્યાનું નિરાકરણ એ એક વ્યવસ્થિત પદ્ધતિ છે જે તર્કસંગત વિચારસરણી અને
સંરચિત પદ્ધતિઓનો ઉપયોગ કરીને સમસ્યાઓને ઓળખવા, તેનું વિશ્લેષણ કરવા અને હલ કરવા માટે
વપરાય છે.

\textbf{સમસ્યા નિરાકરણના પગલાં:}

{\def\LTcaptype{none} % do not increment counter
\begin{longtable}[]{@{}ll@{}}
\toprule\noalign{}
પગલું & વર્ણન \\
\midrule\noalign{}
\endhead
\bottomrule\noalign{}
\endlastfoot
1. \textbf{સમસ્યાની ઓળખ} & સમસ્યાને સ્પષ્ટપણે સમજવી અને વ્યાખ્યાયિત કરવી \\
2. \textbf{સમસ્યાનું વિશ્લેષણ} & સમસ્યાને નાના ભાગોમાં વિભાજિત કરવી \\
3. \textbf{સોલ્યુશન ડિઝાઇન} & સંભવિત ઉકેલો અથવા એલ્ગોરિધમ વિકસાવવા \\
4. \textbf{અમલીકરણ} & પસંદ કરેલા ઉકેલને અમલમાં મૂકવો \\
5. \textbf{ટેસ્ટિંગ અને વેલિડેશન} & ઉકેલ યોગ્ય રીતે કામ કરે છે તેની ખાતરી કરવી \\
6. \textbf{ડોક્યુમેન્ટેશન} & ભાવિ સંદર્ભ માટે ઉકેલને રેકોર્ડ કરવો \\
\end{longtable}
}

\end{solutionbox}
\begin{mnemonicbox}
``હું હંમેશા ડિઝાઇન અમલીકરણ ટેસ્ટ દૈનિક''

\end{mnemonicbox}
\begin{center}\rule{0.5\linewidth}{0.5pt}\end{center}

\subsection*{પ્રશ્ન 1(બ) [4
ગુણ]}\label{uxaaauxab0uxab6uxaa8-1uxaac-4-uxa97uxaa3}

\textbf{વેરિએબલ વ્યાખ્યાયિત કરો અને વેરિએબલના નામ પસંદ કરવા માટેના નિયમનો ઉલ્લેખ
કરો.}

\begin{solutionbox}
વેરિએબલ એ મેમરીમાં એક નામાંકિત સ્ટોરેજ સ્થાન છે જે ડેટા વેલ્યુઝ ધરાવે છે
અને પ્રોગ્રામ એક્ઝિક્યુશન દરમિયાન બદલાઈ શકે છે.

\textbf{વેરિએબલ નામકરણ નિયમો:}

{\def\LTcaptype{none} % do not increment counter
\begin{longtable}[]{@{}ll@{}}
\toprule\noalign{}
નિયમ & વર્ણન \\
\midrule\noalign{}
\endhead
\bottomrule\noalign{}
\endlastfoot
\textbf{શરૂઆતી અક્ષર} & અક્ષર (a-z, A-Z) અથવા અન્ડરસ્કોર (\_) થી શરૂ થવું
જોઈએ \\
\textbf{મંજૂર અક્ષરો} & અક્ષરો, અંકો (0-9), અને અન્ડરસ્કોર હોઈ શકે \\
\textbf{કેસ સેંસિટિવ} & myVar અને MyVar જુદા વેરિએબલ છે \\
\textbf{કોઈ કીવર્ડ્સ નહીં} & Python ના રિઝર્વ્ડ શબ્દો વાપરી શકાતા નથી \\
\textbf{કોઈ સ્પેસ નહીં} & સ્પેસની જગ્યાએ અન્ડરસ્કોર વાપરો \\
\textbf{વર્ણનાત્મક નામ} & અર્થપૂર્ણ નામ પસંદ કરો (age, x નહીં) \\
\end{longtable}
}

\end{solutionbox}
\begin{mnemonicbox}
``અક્ષરથી શરૂઆત, સાવધાનીથી ચાલુ, ક્યારેય કીવર્ડ્સ નહીં''

\end{mnemonicbox}
\begin{center}\rule{0.5\linewidth}{0.5pt}\end{center}

\subsection*{પ્રશ્ન 1(ક) [7
ગુણ]}\label{uxaaauxab0uxab6uxaa8-1uxa95-7-uxa97uxaa3}

\textbf{આપેલ ત્રણ નંબરોમાંથી મહત્તમ સંખ્યા શોધવા માટે ફ્લોચાર્ટ ડિઝાઇન કરો.}

\begin{solutionbox}
ફ્લોચાર્ટ કમ્પેરિઝન ઓપરેશન્સ વાપરીને ત્રણ નંબરોમાંથી મહત્તમ શોધવાના
તાર્કિક પ્રવાહને દર્શાવે છે.

\textbf{ફ્લોચાર્ટ:}

\begin{verbatim}
flowchart LR
    A[Start] {-{-} B[Input: num1, num2, num3]}
    B {-{-} C\{num1  num2?\}}
    C {-{-}|Yes| D\{num1  num3?\}}
    C {-{-}|No| E\{num2  num3?\}}
    D {-{-}|Yes| F[max = num1]}
    D {-{-}|No| G[max = num3]}
    E {-{-}|Yes| H[max = num2]}
    E {-{-}|No| I[max = num3]}
    F {-{-} J[Output: max]}
    G {-{-} J}
    H {-{-} J}
    I {-{-} J}
    J {-{-} K[End]}
\end{verbatim}

\textbf{મુખ્ય મુદ્દાઓ:}

\begin{itemize}
\tightlist
\item
  \textbf{ઇનપુટ}: ત્રણ નંબરો (num1, num2, num3)
\item
  \textbf{પ્રોસેસ}: નેસ્ટેડ કંડિશન્સ વાપરીને નંબરોની તુલના
\item
  \textbf{આઉટપુટ}: ત્રણેય વચ્ચે મહત્તમ મૂલ્ય
\end{itemize}

\end{solutionbox}
\begin{mnemonicbox}
``પહેલા બેની તુલના, પછી ત્રીજા સાથે''

\end{mnemonicbox}
\begin{center}\rule{0.5\linewidth}{0.5pt}\end{center}

\subsection*{પ્રશ્ન 1(ક અથવા) [7
ગુણ]}\label{uxaaauxab0uxab6uxaa8-1uxa95-uxa85uxaa5uxab5-7-uxa97uxaa3}

\textbf{દાખલ કરેલ નંબર પોઝિટિવ છે અને 5 કરતા વધારે છે કે નહીં તે તપાસવા એક
એલ્ગોરિધમ બનાવો.}

\begin{solutionbox}
એક નંબર પોઝિટિવ અને 5 કરતા વધારે છે કે કેમ તે ચકાસવા માટેનું
એલ્ગોરિધમ.

\textbf{એલ્ગોરિધમ:}

\begin{verbatim}
Algorithm: CheckPositiveGreaterThan5
Step 1: START
Step 2: INPUT number
Step 3: IF number > 0 AND number > 5 THEN
           PRINT "Number is positive and greater than 5"
        ELSE
           PRINT "Number does not meet criteria"
        END IF
Step 4: END
\end{verbatim}

\textbf{ફ્લોચાર્ટ:}

\begin{verbatim}
flowchart LR
    A[Start] {-{-} B[Input: number]}
    B {-{-} C\{number  0 AND number  5?\}}
    C {-{-}|Yes| D[Print: Number is positive and greater than 5]}
    C {-{-}|No| E[Print: Number does not meet criteria]}
    D {-{-} F[End]}
    E {-{-} F}
\end{verbatim}

\textbf{મુખ્ય શરતો:}

\begin{itemize}
\tightlist
\item
  \textbf{પોઝિટિવ}: number \textgreater{} 0
\item
  \textbf{5 કરતા વધારે}: number \textgreater{} 5
\item
  \textbf{સંયુક્ત}: બંને શરતો સાચી હોવી જોઈએ
\end{itemize}

\end{solutionbox}
\begin{mnemonicbox}
``પોઝિટિવ પ્લસ પાંચ''

\end{mnemonicbox}
\begin{center}\rule{0.5\linewidth}{0.5pt}\end{center}

\subsection*{પ્રશ્ન 2(અ) [3
ગુણ]}\label{uxaaauxab0uxab6uxaa8-2uxa85-3-uxa97uxaa3}

\textbf{એરિથમેટિક ઓપરેટરો પર શોર્ટ નોટ લખો.}

\begin{solutionbox}
એરિથમેટિક ઓપરેટરો Python પ્રોગ્રામિંગમાં ન્યુમેરિક વેલ્યુઝ પર ગાણિતિક
ગણતરીઓ કરે છે.

\textbf{એરિથમેટિક ઓપરેટરો ટેબલ:}

{\def\LTcaptype{none} % do not increment counter
\begin{longtable}[]{@{}llll@{}}
\toprule\noalign{}
ઓપરેટર & નામ & ઉદાહરણ & પરિણામ \\
\midrule\noalign{}
\endhead
\bottomrule\noalign{}
\endlastfoot
+ & ઉમેરાણ & 5 + 3 & 8 \\
- & બાદબાકી & 5 - 3 & 2 \\
* & ગુણાકાર & 5 * 3 & 15 \\
/ & ભાગાકાર & 5 / 3 & 1.67 \\
// & ફ્લોર ડિવિઝન & 5 // 3 & 1 \\
\% & મોડ્યુલસ & 5 \% 3 & 2 \\
** & ઘાત & 5 ** 3 & 125 \\
\end{longtable}
}

\end{solutionbox}
\begin{mnemonicbox}
``ઉમેરો બાદ કરો ગુણો ભાગો ફ્લોર મોડ પાવર''

\end{mnemonicbox}
\begin{center}\rule{0.5\linewidth}{0.5pt}\end{center}

\subsection*{પ્રશ્ન 2(બ) [4
ગુણ]}\label{uxaaauxab0uxab6uxaa8-2uxaac-4-uxa97uxaa3}

\textbf{કંટિન્યુ અને બ્રેક સ્ટેટમેંટની જરૂરિયાત સમજાવો.}

\begin{solutionbox}
કંટિન્યુ અને બ્રેક સ્ટેટમેંટ્સ કાર્યક્ષમ પ્રોગ્રામિંગ માટે લૂપ એક્ઝિક્યુશન
ફ્લોને નિયંત્રિત કરે છે.

\textbf{સ્ટેટમેંટ કમ્પેરિઝન:}

{\def\LTcaptype{none} % do not increment counter
\begin{longtable}[]{@{}lll@{}}
\toprule\noalign{}
સ્ટેટમેંટ & હેતુ & ક્રિયા \\
\midrule\noalign{}
\endhead
\bottomrule\noalign{}
\endlastfoot
\textbf{break} & લૂપમાંથી સંપૂર્ણ બહાર નીકળવું & સંપૂર્ણ લૂપને સમાપ્ત કરે છે \\
\textbf{continue} & વર્તમાન આવૃત્તિ છોડવી & આગલી આવૃત્તિ પર જાય છે \\
\end{longtable}
}

\textbf{વપરાશના ઉદાહરણો:}

\begin{itemize}
\tightlist
\item
  \textbf{break}: શરત પૂરી થાય ત્યારે બહાર નીકળવું (ચોક્કસ મૂલ્ય શોધવું)
\item
  \textbf{continue}: અયોગ્ય ડેટા છોડવો (પોઝિટિવ લિસ્ટમાં નેગેટિવ નંબરો)
\end{itemize}

\textbf{ફાયદાઓ:}

\begin{itemize}
\tightlist
\item
  \textbf{કાર્યક્ષમતા}: બિનજરૂરી આવૃત્તિઓ ટાળવી
\item
  \textbf{નિયંત્રણ}: પ્રોગ્રામ ફ્લોનું વધુ સારું મેનેજમેંટ
\item
  \textbf{સ્પષ્ટતા}: વધુ સ્વચ્છ કોડ લોજિક
\end{itemize}

\end{solutionbox}
\begin{mnemonicbox}
``બ્રેક બહાર નીકળે, કંટિન્યુ છોડે''

\end{mnemonicbox}
\begin{center}\rule{0.5\linewidth}{0.5pt}\end{center}

\subsection*{પ્રશ્ન 2(ક) [7
ગુણ]}\label{uxaaauxab0uxab6uxaa8-2uxa95-7-uxa97uxaa3}

\textbf{દાખલ કરેલ સંખ્યા સમ છે કે વિષમ છે તે તપાસવા માટે એક પ્રોગ્રામ બનાવો.}

\begin{solutionbox}
મોડ્યુલસ ઓપરેટર વાપરીને નંબર સમ કે વિષમ છે તે નિર્ધારિત કરવા માટેનો
Python પ્રોગ્રામ.

\textbf{Python કોડ:}

\begin{verbatim}
\# સમ કે વિષમ તપાસવા માટેનો પ્રોગ્રામ
number = int(input("એક નંબર દાખલ કરો: "))

if number \% 2 == 0:
    print(f"\{number\} સમ છે")
else:
    print(f"\{number\} વિષમ છે")
\end{verbatim}

\textbf{લોજિક સમજૂતી:}

{\def\LTcaptype{none} % do not increment counter
\begin{longtable}[]{@{}lll@{}}
\toprule\noalign{}
શરત & પરિણામ & સમજૂતી \\
\midrule\noalign{}
\endhead
\bottomrule\noalign{}
\endlastfoot
number \% 2 == 0 & સમ & 2 વડે વિભાજ્ય, કોઈ બાકી નહીં \\
number \% 2 == 1 & વિષમ & 2 વડે વિભાજ્ય નહીં, બાકી 1 \\
\end{longtable}
}

\textbf{સેમ્પલ આઉટપુટ:}

\begin{itemize}
\tightlist
\item
  ઇનપુટ: 8 \rightarrow આઉટપુટ: ``8 સમ છે''
\item
  ઇનપુટ: 7 \rightarrow આઉટપુટ: ``7 વિષમ છે''
\end{itemize}

\end{solutionbox}
\begin{mnemonicbox}
``મોડ્યુલસ શૂન્ય સમ, એક વિષમ''

\end{mnemonicbox}
\begin{center}\rule{0.5\linewidth}{0.5pt}\end{center}

\subsection*{પ્રશ્ન 2(અ અથવા) [3
ગુણ]}\label{uxaaauxab0uxab6uxaa8-2uxa85-uxa85uxaa5uxab5-3-uxa97uxaa3}

\textbf{Python ના કમ્પેરિઝન ઓપરેટરોનો સારાંશ આપો.}

\begin{solutionbox}
કમ્પેરિઝન ઓપરેટરો વેલ્યુઝની તુલના કરે છે અને બુલિયન પરિણામો
(True/False) આપે છે.

\textbf{કમ્પેરિઝન ઓપરેટરો ટેબલ:}

{\def\LTcaptype{none} % do not increment counter
\begin{longtable}[]{@{}llll@{}}
\toprule\noalign{}
ઓપરેટર & નામ & ઉદાહરણ & પરિણામ \\
\midrule\noalign{}
\endhead
\bottomrule\noalign{}
\endlastfoot
== & બરાબર & 5 == 5 & True \\
!= & બરાબર નથી & 5 != 3 & True \\
\textgreater{} & મોટું & 5 \textgreater{} 3 & True \\
\textless{} & નાનું & 5 \textless{} 3 & False \\
\textgreater= & મોટું અથવા બરાબર & 5 \textgreater= 5 & True \\
\textless= & નાનું અથવા બરાબર & 5 \textless= 3 & False \\
\end{longtable}
}

\textbf{રિટર્ન ટાઇપ:} બધા ઓપરેટરો બુલિયન વેલ્યુઝ (True/False) આપે છે

\end{solutionbox}
\begin{mnemonicbox}
``બરાબર નહીં મોટું નાનું મોટું-બરાબર નાનું-બરાબર''

\end{mnemonicbox}
\begin{center}\rule{0.5\linewidth}{0.5pt}\end{center}

\subsection*{પ્રશ્ન 2(બ અથવા) [4
ગુણ]}\label{uxaaauxab0uxab6uxaa8-2uxaac-uxa85uxaa5uxab5-4-uxa97uxaa3}

\textbf{While લૂપ પર ટૂંકી નોંધ લખો.}

\begin{solutionbox}
While લૂપ જ્યાં સુધી શરત સાચી રહે છે ત્યાં સુધી કોડ બ્લોકને વારંવાર
એક્ઝિક્યુટ કરે છે.

\textbf{While લૂપ સ્ટ્રક્ચર:}

{\def\LTcaptype{none} % do not increment counter
\begin{longtable}[]{@{}ll@{}}
\toprule\noalign{}
ઘટક & વર્ણન \\
\midrule\noalign{}
\endhead
\bottomrule\noalign{}
\endlastfoot
\textbf{પ્રારંભિકરણ} & લૂપ પહેલાં પ્રારંભિક મૂલ્ય સેટ કરવું \\
\textbf{શરત} & તપાસવા માટેનું બુલિયન એક્સપ્રેશન \\
\textbf{બોડી} & વારંવાર એક્ઝિક્યુટ કરવાનો કોડ \\
\textbf{અપડેટ} & અનંત લૂપ ટાળવા માટે વેરિએબલ બદલવો \\
\end{longtable}
}

\textbf{સિન્ટેક્સ:}

\begin{verbatim}
while condition:
    \# loop body
    \# update statement
\end{verbatim}

\textbf{લક્ષણો:}

\begin{itemize}
\tightlist
\item
  \textbf{પ્રી-ટેસ્ટેડ}: એક્ઝિક્યુશન પહેલાં શરત તપાસાય છે
\item
  \textbf{વેરિએબલ આવૃત્તિઓ}: અજાણી સંખ્યામાં પુનરાવર્તન
\item
  \textbf{નિયંત્રણ}: શરત ચાલુ રાખવું નક્કી કરે છે
\end{itemize}

\end{solutionbox}
\begin{mnemonicbox}
``જ્યારે શરત સાચી, લૂપ ચલાવો''

\end{mnemonicbox}
\begin{center}\rule{0.5\linewidth}{0.5pt}\end{center}

\subsection*{પ્રશ્ન 2(ક અથવા) [7
ગુણ]}\label{uxaaauxab0uxab6uxaa8-2uxa95-uxa85uxaa5uxab5-7-uxa97uxaa3}

\textbf{યુઝર પાસેથી ત્રણ નંબરો વાંચવા અને તે નંબરોની સરેરાશ શોધવા માટે એક પ્રોગ્રામ
બનાવો.}

\begin{solutionbox}
યુઝર-ઇનપુટ ત્રણ નંબરોની સરેરાશ ગણવા માટેનો Python પ્રોગ્રામ.

\textbf{Python કોડ:}

\begin{verbatim}
\# ત્રણ નંબરોની સરેરાશ શોધવા માટેનો પ્રોગ્રામ
num1 = float(input("પહેલો નંબર દાખલ કરો: "))
num2 = float(input("બીજો નંબર દાખલ કરો: "))
num3 = float(input("ત્રીજો નંબર દાખલ કરો: "))

average = (num1 + num2 + num3) / 3

print(f"\{num1\}, \{num2\}, \{num3\} ની સરેરાશ: \{average:.2f\} છે")
\end{verbatim}

\textbf{ગણતરી પ્રક્રિયા:}

{\def\LTcaptype{none} % do not increment counter
\begin{longtable}[]{@{}ll@{}}
\toprule\noalign{}
પગલું & ઓપરેશન \\
\midrule\noalign{}
\endhead
\bottomrule\noalign{}
\endlastfoot
\textbf{ઇનપુટ} & ત્રણ નંબરો વાંચો \\
\textbf{સરવાળો} & ત્રણેય નંબરો ઉમેરો \\
\textbf{ભાગાકાર} & સરવાળો \div 3 \\
\textbf{આઉટપુટ} & ફોર્મેટ કરેલ પરિણામ દર્શાવો \\
\end{longtable}
}

\textbf{સેમ્પલ એક્ઝિક્યુશન:}

\begin{itemize}
\tightlist
\item
  ઇનપુટ: 10, 20, 30
\item
  સરવાળો: 60
\item
  સરેરાશ: 20.00
\end{itemize}

\end{solutionbox}
\begin{mnemonicbox}
``ત્રણ ઉમેરો ભાગો દર્શાવો''

\end{mnemonicbox}
\begin{center}\rule{0.5\linewidth}{0.5pt}\end{center}

\subsection*{પ્રશ્ન 3(અ) [3
ગુણ]}\label{uxaaauxab0uxab6uxaa8-3uxa85-3-uxa97uxaa3}

\textbf{કંટ્રોલ સ્ટ્રક્ચર્સ વ્યાખ્યાયિત કરો, પાયથોનમાં ઉપલબ્ધ કંટ્રોલ સ્ટ્રક્ચર્સની સૂચિ
બનાવો.}

\begin{solutionbox}
કંટ્રોલ સ્ટ્રક્ચર્સ પ્રોગ્રામમાં એક્ઝિક્યુશન ફ્લો અને સ્ટેટમેંટ્સનો ક્રમ
નિર્ધારિત કરે છે.

\textbf{Python કંટ્રોલ સ્ટ્રક્ચર્સ:}

{\def\LTcaptype{none} % do not increment counter
\begin{longtable}[]{@{}lll@{}}
\toprule\noalign{}
પ્રકાર & સ્ટ્રક્ચર્સ & હેતુ \\
\midrule\noalign{}
\endhead
\bottomrule\noalign{}
\endlastfoot
\textbf{સિક્વેન્શિયલ} & સામાન્ય ફ્લો & સ્ટેટમેંટ્સ ક્રમમાં એક્ઝિક્યુટ કરવા \\
\textbf{સિલેક્શન} & if, if-else, elif & વિકલ્પો વચ્ચે પસંદગી \\
\textbf{આઇટરેશન} & for, while & કોડ બ્લોક્સનું પુનરાવર્તન \\
\textbf{જમ્પ} & break, continue, pass & સામાન્ય ફ્લો બદલવો \\
\end{longtable}
}

\textbf{કેટેગરીઝ:}

\begin{itemize}
\tightlist
\item
  \textbf{કંડિશનલ}: નિર્ણય લેવો (if સ્ટેટમેંટ્સ)
\item
  \textbf{લૂપિંગ}: પુનરાવર્તન (for/while લૂપ્સ)
\item
  \textbf{બ્રાન્ચિંગ}: ફ્લો કંટ્રોલ (break/continue)
\end{itemize}

\end{solutionbox}
\begin{mnemonicbox}
``સિક્વેન્સ સિલેક્ટ આઇટરેટ જમ્પ''

\end{mnemonicbox}
\begin{center}\rule{0.5\linewidth}{0.5pt}\end{center}

\subsection*{પ્રશ્ન 3(બ) [4
ગુણ]}\label{uxaaauxab0uxab6uxaa8-3uxaac-4-uxa97uxaa3}

\textbf{યુઝર ડિફાઇન્ડ ફંકશન વ્યાખ્યાયિત કરો અને કેવી રીતે યુઝર ડિફાઇન્ડ ફંકશન કૉલ
કરવું તે ઉદાહરણ આપીને સમજાવો.}

\begin{solutionbox}
યુઝર-ડિફાઇન્ડ ફંકશન્સ ચોક્કસ કાર્યો કરતા પુનઃ ઉપયોગી કોડના કસ્ટમ
બ્લોક્સ છે.

\textbf{ફંકશન સ્ટ્રક્ચર:}

{\def\LTcaptype{none} % do not increment counter
\begin{longtable}[]{@{}lll@{}}
\toprule\noalign{}
ઘટક & સિન્ટેક્સ & હેતુ \\
\midrule\noalign{}
\endhead
\bottomrule\noalign{}
\endlastfoot
\textbf{ડેફિનિશન} & def function\_name(): & ફંકશન બનાવવું \\
\textbf{પેરામીટર્સ} & def func(param1, param2): & ઇનપુટ્સ સ્વીકારવા \\
\textbf{બોડી} & ઇન્ડેન્ટેડ કોડ બ્લોક & ફંકશન લોજિક \\
\textbf{રિટર્ન} & return value & પરિણામ પાછું મોકલવું \\
\textbf{કૉલ} & function\_name() & ફંકશન એક્ઝિક્યુટ કરવું \\
\end{longtable}
}

\textbf{ઉદાહરણ કોડ:}

\begin{verbatim}
\# ફંકશન ડેફિનિશન
def greet\_user(name):
    message = f"નમસ્તે, \{name\}!"
    return message

\# ફંકશન કૉલ
result = greet\_user("Python")
print(result)  \# આઉટપુટ: નમસ્તે, Python!
\end{verbatim}

\end{solutionbox}
\begin{mnemonicbox}
``ડિફાઇન પેરામીટર્સ બોડી રિટર્ન કૉલ''

\end{mnemonicbox}
\begin{center}\rule{0.5\linewidth}{0.5pt}\end{center}

\subsection*{પ્રશ્ન 3(ક) [7
ગુણ]}\label{uxaaauxab0uxab6uxaa8-3uxa95-7-uxa97uxaa3}

\textbf{લૂપ કોન્સેપ્ટનો ઉપયોગ કરીને નીચેની પેટર્ન દર્શાવવા માટે એક પ્રોગ્રામ બનાવો}

\begin{solutionbox}
નેસ્ટેડ લૂપ્સ વાપરીને નંબર પેટર્ન બનાવવા માટેનો Python પ્રોગ્રામ.

\textbf{Python કોડ:}

\begin{verbatim}
\# પેટર્ન પ્રિન્ટિંગ પ્રોગ્રામ
for i in range(1, 6):
    for j in range(1, i + 1):
        print(i, end="")
    print()  \# દરેક પંક્તિ પછી નવી લાઇન
\end{verbatim}

\textbf{પેટર્ન લોજિક:}

{\def\LTcaptype{none} % do not increment counter
\begin{longtable}[]{@{}lll@{}}
\toprule\noalign{}
પંક્તિ & આવૃત્તિઓ & આઉટપુટ \\
\midrule\noalign{}
\endhead
\bottomrule\noalign{}
\endlastfoot
1 & 1 વખત & 1 \\
2 & 2 વખત & 22 \\
3 & 3 વખત & 333 \\
4 & 4 વખત & 4444 \\
5 & 5 વખત & 55555 \\
\end{longtable}
}

\textbf{લૂપ સ્ટ્રક્ચર:}

\begin{itemize}
\tightlist
\item
  \textbf{બાહ્ય લૂપ}: પંક્તિઓને નિયંત્રિત કરે છે (1 થી 5)
\item
  \textbf{આંતરિક લૂપ}: વર્તમાન પંક્તિ નંબર પ્રિન્ટ કરે છે
\item
  \textbf{પેટર્ન}: પંક્તિ નંબર પંક્તિ વખત પુનરાવર્તિત
\end{itemize}

\end{solutionbox}
\begin{mnemonicbox}
``બાહ્ય પંક્તિઓ આંતરિક પુનરાવર્તન''

\end{mnemonicbox}
\begin{center}\rule{0.5\linewidth}{0.5pt}\end{center}

\subsection*{પ્રશ્ન 3(અ અથવા) [3
ગુણ]}\label{uxaaauxab0uxab6uxaa8-3uxa85-uxa85uxaa5uxab5-3-uxa97uxaa3}

\textbf{યોગ્ય ઉદાહરણનો ઉપયોગ કરીને નેસ્ટેડ લૂપ સમજાવો.}

\begin{solutionbox}
નેસ્ટેડ લૂપ એ બીજા લૂપની અંદર આવેલ લૂપ છે જ્યાં દરેક બાહ્ય લૂપ આવૃત્તિ
માટે આંતરિક લૂપ તેની બધી આવૃત્તિઓ પૂર્ણ કરે છે.

\textbf{નેસ્ટેડ લૂપ સ્ટ્રક્ચર:}

{\def\LTcaptype{none} % do not increment counter
\begin{longtable}[]{@{}ll@{}}
\toprule\noalign{}
ઘટક & વર્ણન \\
\midrule\noalign{}
\endhead
\bottomrule\noalign{}
\endlastfoot
\textbf{બાહ્ય લૂપ} & મુખ્ય આવૃત્તિઓને નિયંત્રિત કરે છે \\
\textbf{આંતરિક લૂપ} & દરેક બાહ્ય આવૃત્તિ માટે સંપૂર્ણ એક્ઝિક્યુટ થાય છે \\
\textbf{એક્ઝિક્યુશન} & આંતરિક લૂપ કુલ n\timesm વખત ચાલે છે \\
\end{longtable}
}

\textbf{ઉદાહરણ કોડ:}

\begin{verbatim}
\# નેસ્ટેડ લૂપ ઉદાહરણ {- ગુણાકાર કોષ્ટક}
for i in range(1, 4):      \# બાહ્ય લૂપ
    for j in range(1, 4):  \# આંતરિક લૂપ
        print(f"\{i\}\{j\}=\{i*j\}", end=" ")
    print()  \# નવી લાઇન
\end{verbatim}

\textbf{આઉટપુટ પેટર્ન:}

\begin{verbatim}
1\times1=1 1\times2=2 1\times3=3
2\times1=2 2\times2=4 2\times3=6
3\times1=3 3\times2=6 3\times3=9
\end{verbatim}

\end{solutionbox}
\begin{mnemonicbox}
``લૂપ અંદર લૂપ''

\end{mnemonicbox}
\begin{center}\rule{0.5\linewidth}{0.5pt}\end{center}

\subsection*{પ્રશ્ન 3(બ અથવા) [4
ગુણ]}\label{uxaaauxab0uxab6uxaa8-3uxaac-uxa85uxaa5uxab5-4-uxa97uxaa3}

\textbf{વેરિએબલના લોકલ અને ગ્લોબલ સ્કોપ પર શોર્ટ નોંધ લખો}

\begin{solutionbox}
વેરિએબલ સ્કોપ નિર્ધારિત કરે છે કે પ્રોગ્રામમાં વેરિએબલ્સ ક્યાં એક્સેસ
કરી શકાય છે.

\textbf{સ્કોપ કમ્પેરિઝન:}

{\def\LTcaptype{none} % do not increment counter
\begin{longtable}[]{@{}llll@{}}
\toprule\noalign{}
સ્કોપ પ્રકાર & વ્યાખ્યા & એક્સેસ & જીવનકાળ \\
\midrule\noalign{}
\endhead
\bottomrule\noalign{}
\endlastfoot
\textbf{લોકલ} & ફંકશનની અંદર & ફક્ત ફંકશન & ફંકશન એક્ઝિક્યુશન \\
\textbf{ગ્લોબલ} & ફંકશન્સની બહાર & સંપૂર્ણ પ્રોગ્રામ & પ્રોગ્રામ એક્ઝિક્યુશન \\
\end{longtable}
}

\textbf{ઉદાહરણ કોડ:}

\begin{verbatim}
global\_var = "હું ગ્લોબલ છું"  \# ગ્લોબલ સ્કોપ

def my\_function():
    local\_var = "હું લોકલ છું"    \# લોકલ સ્કોપ
    global global\_var
    print(global\_var)   \# એક્સેસિબલ
    print(local\_var)    \# એક્સેસિબલ

print(global\_var)   \# એક્સેસિબલ
\# print(local\_var)  \# એરર {- એક્સેસિબલ નથી}
\end{verbatim}

\textbf{મુખ્ય મુદ્દાઓ:}

\begin{itemize}
\tightlist
\item
  \textbf{લોકલ}: ફંકશન-સ્પેસિફિક વેરિએબલ્સ
\item
  \textbf{ગ્લોબલ}: પ્રોગ્રામ-વ્યાપી વેરિએબલ્સ
\item
  \textbf{એક્સેસ}: ફંકશન્સમાં લોકલ ગ્લોબલને ઓવરરાઇડ કરે છે
\end{itemize}

\end{solutionbox}
\begin{mnemonicbox}
``લોકલ મર્યાદિત, ગ્લોબલ સામાન્ય''

\end{mnemonicbox}
\begin{center}\rule{0.5\linewidth}{0.5pt}\end{center}

\subsection*{પ્રશ્ન 3(ક અથવા) [7
ગુણ]}\label{uxaaauxab0uxab6uxaa8-3uxa95-uxa85uxaa5uxab5-7-uxa97uxaa3}

\textbf{આપેલ સંખ્યાના ફેક્ટોરિયલ શોધવા માટે યુઝર ડિફાઇન્ડ ફંકશન વિકસાવો.}

\begin{solutionbox}
પોઝિટિવ પૂર્ણાંકના ફેક્ટોરિયલની ગણતરી કરવા માટેનું રિકર્સિવ ફંકશન.

\textbf{Python કોડ:}

\begin{verbatim}
def factorial(n):
    """n નું ફેક્ટોરિયલ ગણવું"""
if

n == 0 or

n == 1:

        return 1
    else:
        return n * factorial(n {-} 1)

\# ફંકશનને ટેસ્ટ કરવું
number = int(input("એક નંબર દાખલ કરો: "))
if number {} 0:
    print("નેગેટિવ નંબરો માટે ફેક્ટોરિયલ વ્યાખ્યાયિત નથી")
else:
    result = factorial(number)
    print(f"\{number\} નું ફેક્ટોરિયલ \{result\} છે")
\end{verbatim}

\textbf{ફેક્ટોરિયલ લોજિક:}

{\def\LTcaptype{none} % do not increment counter
\begin{longtable}[]{@{}lll@{}}
\toprule\noalign{}
ઇનપુટ & ગણતરી & પરિણામ \\
\midrule\noalign{}
\endhead
\bottomrule\noalign{}
\endlastfoot
0 & બેઝ કેસ & 1 \\
1 & બેઝ કેસ & 1 \\
5 & 5 \times 4 \times 3 \times 2 \times 1 & 120 \\
\end{longtable}
}

\textbf{ફંકશન લક્ષણો:}

\begin{itemize}
\tightlist
\item
  \textbf{રિકર્સિવ}: ફંકશન પોતાને કૉલ કરે છે
\item
  \textbf{બેઝ કેસ}: n=0 અથવા n=1 પર રિકર્શન રોકે છે
\item
  \textbf{વેલિડેશન}: નેગેટિવ ઇનપુટ્સને હેન્ડલ કરે છે
\end{itemize}

\end{solutionbox}
\begin{mnemonicbox}
``બધા પાછલા નંબરોનો ગુણાકાર''

\end{mnemonicbox}
\begin{center}\rule{0.5\linewidth}{0.5pt}\end{center}

\subsection*{પ્રશ્ન 4(અ) [3
ગુણ]}\label{uxaaauxab0uxab6uxaa8-4uxa85-3-uxa97uxaa3}

\textbf{મેથ મોડ્યુલ વિવિધ ફંકશન સાથે સમજાવો}

\begin{solutionbox}
મેથ મોડ્યુલ ન્યુમેરિકલ કોમ્પ્યુટેશન્સ માટે ગાણિતિક ફંકશન્સ અને કોન્સ્ટન્ટ્સ
પ્રદાન કરે છે.

\textbf{મેથ મોડ્યુલ ફંકશન્સ:}

{\def\LTcaptype{none} % do not increment counter
\begin{longtable}[]{@{}lll@{}}
\toprule\noalign{}
ફંકશન & હેતુ & ઉદાહરણ \\
\midrule\noalign{}
\endhead
\bottomrule\noalign{}
\endlastfoot
\textbf{math.sqrt()} & વર્ગમૂળ & math.sqrt(16) = 4.0 \\
\textbf{math.pow()} & ઘાત ગણતરી & math.pow(2, 3) = 8.0 \\
\textbf{math.ceil()} & ઉપર રાઉન્ડ & math.ceil(4.3) = 5 \\
\textbf{math.floor()} & નીચે રાઉન્ડ & math.floor(4.7) = 4 \\
\textbf{math.factorial()} & ફેક્ટોરિયલ & math.factorial(5) = 120 \\
\end{longtable}
}

\textbf{વપરાશ:}

\begin{verbatim}
import math
result = math.sqrt(25)  \# 5.0 રિટર્ન કરે છે
\end{verbatim}

\end{solutionbox}
\begin{mnemonicbox}
``વર્ગ ઘાત સીલિંગ ફ્લોર ફેક્ટોરિયલ''

\end{mnemonicbox}
\begin{center}\rule{0.5\linewidth}{0.5pt}\end{center}

\subsection*{પ્રશ્ન 4(બ) [4
ગુણ]}\label{uxaaauxab0uxab6uxaa8-4uxaac-4-uxa97uxaa3}

\textbf{નીચેના લિસ્ટના ફંકશનની ચર્ચા કરો: i. len() ii. sum() iii. sort() iv.
index()}

\begin{solutionbox}
ડેટા મેનિપ્યુલેશન અને વિશ્લેષણ માટેના આવશ્યક લિસ્ટ ફંકશન્સ.

\textbf{લિસ્ટ ફંકશન્સ કમ્પેરિઝન:}

{\def\LTcaptype{none} % do not increment counter
\begin{longtable}[]{@{}llll@{}}
\toprule\noalign{}
ફંકશન & હેતુ & રિટર્ન ટાઇપ & ઉદાહરણ \\
\midrule\noalign{}
\endhead
\bottomrule\noalign{}
\endlastfoot
\textbf{len()} & એલિમેન્ટ્સ ગણવા & Integer & len([1,2,3]) = 3 \\
\textbf{sum()} & બધા નંબરોનો સરવાળો & Number & sum([1,2,3]) = 6 \\
\textbf{sort()} & ક્રમમાં ગોઠવવું & None (લિસ્ટ બદલે છે) & list.sort() \\
\textbf{index()} & એલિમેન્ટની સ્થિતિ શોધવી & Integer & [1,2,3].index(2)
= 1 \\
\end{longtable}
}

\textbf{વપરાશની નોંધો:}

\begin{itemize}
\tightlist
\item
  \textbf{len()}: કોઈપણ સિક્વેન્સ સાથે કામ કરે છે
\item
  \textbf{sum()}: ફક્ત ન્યુમેરિક લિસ્ટ્સ
\item
  \textbf{sort()}: મૂળ લિસ્ટને બદલે છે
\item
  \textbf{index()}: પ્રથમ ઓકરન્સ રિટર્ન કરે છે
\end{itemize}

\end{solutionbox}
\begin{mnemonicbox}
``લેન્થ સમ સોર્ટ ઇન્ડેક્સ''

\end{mnemonicbox}
\begin{center}\rule{0.5\linewidth}{0.5pt}\end{center}

\subsection*{પ્રશ્ન 4(ક) [7
ગુણ]}\label{uxaaauxab0uxab6uxaa8-4uxa95-7-uxa97uxaa3}

\textbf{0 થી N નંબરોની ફિબોનાકી શ્રેણીને છાપવા માટે યુઝર ડિફાઇન્ડ ફંકશન બનાવો.
(જ્યાં N પૂર્ણાંક સંખ્યા છે અને આર્ગ્યુમેન્ટ તરીકે પસાર થાય છે)}

\begin{solutionbox}
N ટર્મ્સ સુધી ફિબોનાકી સિક્વેન્સ જનરેટ અને ડિસ્પ્લે કરવા માટેનું ફંકશન.

\textbf{Python કોડ:}

\begin{verbatim}
def fibonacci\_series(n):
    """n ટર્મ્સની ફિબોનાકી શ્રેણી પ્રિન્ટ કરવું"""
    if n {=} 0:
        print("કૃપા કરીને પોઝિટિવ નંબર દાખલ કરો")
        return
    
    \# પ્રથમ બે ટર્મ્સ
    a, b = 0, 1
    
if

n == 1:

        print(f"ફિબોનાકી શ્રેણી: \{a\}")
        return
    
    print(f"ફિબોનાકી શ્રેણી: \{a\}, \{b\}", end="")
    
    \# બાકીના ટર્મ્સ જનરેટ કરવા
    for i in range(2, n):
        c = a + b
        print(f", \{c\}", end="")
        a, b = b, c
    print()  \# નવી લાઇન

\# ફંકશનને ટેસ્ટ કરવું
num = int(input("ટર્મ્સની સંખ્યા દાખલ કરો: "))
fibonacci\_series(num)
\end{verbatim}

\textbf{ફિબોનાકી લોજિક:}

{\def\LTcaptype{none} % do not increment counter
\begin{longtable}[]{@{}lll@{}}
\toprule\noalign{}
ટર્મ & મૂલ્ય & ગણતરી \\
\midrule\noalign{}
\endhead
\bottomrule\noalign{}
\endlastfoot
1મી & 0 & આપેલ \\
2જી & 1 & આપેલ \\
3જી & 1 & 0 + 1 \\
4થી & 2 & 1 + 1 \\
5મી & 3 & 1 + 2 \\
\end{longtable}
}

\end{solutionbox}
\begin{mnemonicbox}
``પાછલા બે નંબરોનો ઉમેરો''

\end{mnemonicbox}
\begin{center}\rule{0.5\linewidth}{0.5pt}\end{center}

\subsection*{પ્રશ્ન 4(અ અથવા) [3
ગુણ]}\label{uxaaauxab0uxab6uxaa8-4uxa85-uxa85uxaa5uxab5-3-uxa97uxaa3}

\textbf{રેન્ડમ મોડ્યુલ વિવિધ ફંકશન સાથે સમજાવો}

\begin{solutionbox}
રેન્ડમ મોડ્યુલ વિવિધ એપ્લિકેશન્સ માટે રેન્ડમ નંબરો જનરેટ કરે છે અને રેન્ડમ
સિલેક્શન્સ કરે છે.

\textbf{રેન્ડમ મોડ્યુલ ફંકશન્સ:}

{\def\LTcaptype{none} % do not increment counter
\begin{longtable}[]{@{}lll@{}}
\toprule\noalign{}
ફંકશન & હેતુ & ઉદાહરણ \\
\midrule\noalign{}
\endhead
\bottomrule\noalign{}
\endlastfoot
\textbf{random()} & 0.0 થી 1.0 ફ્લોટ & random.random() \\
\textbf{randint()} & રેન્જમાં ઇન્ટિજર & random.randint(1, 10) \\
\textbf{choice()} & રેન્ડમ લિસ્ટ એલિમેન્ટ & random.choice([1,2,3]) \\
\textbf{shuffle()} & લિસ્ટનો ક્રમ ભેળસેળ કરવો & random.shuffle(list) \\
\textbf{uniform()} & રેન્જમાં ફ્લોટ & random.uniform(1.0, 5.0) \\
\end{longtable}
}

\textbf{વપરાશ:}

\begin{verbatim}
import random
number = random.randint(1, 100)
\end{verbatim}

\textbf{એપ્લિકેશન્સ:} ગેમ્સ, સિમ્યુલેશન્સ, ટેસ્ટિંગ, ક્રિપ્ટોગ્રાફી

\end{solutionbox}
\begin{mnemonicbox}
``રેન્ડમ રેન્જ ચોઇસ શફલ યુનિફોર્મ''

\end{mnemonicbox}
\begin{center}\rule{0.5\linewidth}{0.5pt}\end{center}

\subsection*{પ્રશ્ન 4(બ અથવા) [4
ગુણ]}\label{uxaaauxab0uxab6uxaa8-4uxaac-uxa85uxaa5uxab5-4-uxa97uxaa3}

\textbf{આપેલ એલિમેન્ટ લિસ્ટનું સભ્ય છે કે નહીં તે તપાસવા માટે પાયથોન કોડ બનાવો}

\begin{solutionbox}
મેમ્બરશિપ ઓપરેટર વાપરીને લિસ્ટમાં એલિમેન્ટ અસ્તિત્વમાં છે કે કેમ તે
ચકાસવા માટેનો Python પ્રોગ્રામ.

\textbf{Python કોડ:}

\begin{verbatim}
\# લિસ્ટમાં એલિમેન્ટ મેમ્બરશિપ તપાસવું
def check\_membership():
    \# સેમ્પલ લિસ્ટ
    numbers = [10, 20, 30, 40, 50]
    
    \# શોધવા માટેનું એલિમેન્ટ મેળવવું
    element = int(input("શોધવા માટે એલિમેન્ટ દાખલ કરો: "))
    
    \# મેમ્બરશિપ તપાસવી
    if element in numbers:
        print(f"\{element\} લિસ્ટમાં હાજર છે")
        print(f"સ્થિતિ: \{numbers.index(element)\}")
    else:
        print(f"\{element\} લિસ્ટમાં હાજર નથી")

\# ફંકશન કૉલ કરવું
check\_membership()
\end{verbatim}

\textbf{મેમ્બરશિપ મેથડ્સ:}

{\def\LTcaptype{none} % do not increment counter
\begin{longtable}[]{@{}lll@{}}
\toprule\noalign{}
મેથડ & સિન્ટેક્સ & રિટર્ન કરે છે \\
\midrule\noalign{}
\endhead
\bottomrule\noalign{}
\endlastfoot
\textbf{in ઓપરેટર} & element in list & Boolean \\
\textbf{not in ઓપરેટર} & element not in list & Boolean \\
\textbf{count() મેથડ} & list.count(element) & Integer \\
\end{longtable}
}

\end{solutionbox}
\begin{mnemonicbox}
``લિસ્ટમાં ટ્રુ ફોલ્સ''

\end{mnemonicbox}
\begin{center}\rule{0.5\linewidth}{0.5pt}\end{center}

\subsection*{પ્રશ્ન 4(ક અથવા) [7
ગુણ]}\label{uxaaauxab0uxab6uxaa8-4uxa95-uxa85uxaa5uxab5-7-uxa97uxaa3}

\textbf{દાખલ કરેલ શબ્દમાળા શબ્દોને ઉલટાવે તે માટે યુઝર ડિફાઇન્ડ ફંકશન વિકસાવો}

\begin{solutionbox}
શબ્દની સ્થિતિ જાળવીને સ્ટ્રિંગમાં દરેક શબ્દને ઉલટાવવા માટેનું ફંકશન.

\textbf{Python કોડ:}

\begin{verbatim}
def reverse\_string\_words(text):
    """સ્ટ્રિંગમાં દરેક શબ્દને ઉલટાવવું"""
    \# સ્ટ્રિંગને શબ્દોમાં વિભાજિત કરવી
    words = text.split()
    
    \# દરેક શબ્દને ઉલટાવવું
    reversed\_words = []
    for word in words:
        reversed\_word = word[::{-}1]  \# ઉલટાવવા માટે સ્લાઇસ નોટેશન
        reversed\_words.append(reversed\_word)
    
    \# શબ્દોને પાછા જોડવા
    result = " ".join(reversed\_words)
    return result

\# ફંકશનને ટેસ્ટ કરવું
input\_string = input("એક સ્ટ્રિંગ દાખલ કરો: ")
output = reverse\_string\_words(input\_string)
print(f"ઇનપુટ: {"}\{input\_string\}{"}")
print(f"આઉટપુટ: {"}\{output\}{"}")

\# આપેલ ઇનપુટ સાથે ઉદાહરણ
test\_input = "Hello IT"
test\_output = reverse\_string\_words(test\_input)
print(f"ઇનપુટ: {"}\{test\_input\}{"}")
print(f"આઉટપુટ: {"}\{test\_output\}{"}")  \# આઉટપુટ: "olleH TI"
\end{verbatim}

\textbf{પ્રોસેસ સ્ટેપ્સ:}

{\def\LTcaptype{none} % do not increment counter
\begin{longtable}[]{@{}lll@{}}
\toprule\noalign{}
સ્ટેપ & ઓપરેશન & ઉદાહરણ \\
\midrule\noalign{}
\endhead
\bottomrule\noalign{}
\endlastfoot
1 & શબ્દોમાં વિભાજિત કરવું & [``Hello'', ``IT''] \\
2 & દરેક શબ્દ ઉલટાવવો & [``olleH'', ``TI''] \\
3 & સ્પેસ સાથે જોડવું & ``olleH TI'' \\
\end{longtable}
}

\end{solutionbox}
\begin{mnemonicbox}
``વિભાજિત ઉલટાવો જોડો''

\end{mnemonicbox}
\begin{center}\rule{0.5\linewidth}{0.5pt}\end{center}

\subsection*{પ્રશ્ન 5(અ) [3
ગુણ]}\label{uxaaauxab0uxab6uxaa8-5uxa85-3-uxa97uxaa3}

\textbf{આપેલ સ્ટ્રિંગની પદ્ધતિઓ સમજાવો: i. count() ii. strip() iii.
replace()}

\begin{solutionbox}
ટેક્સ્ટ પ્રોસેસિંગ અને મેનિપ્યુલેશન માટેના આવશ્યક સ્ટ્રિંગ મેથડ્સ.

\textbf{સ્ટ્રિંગ મેથડ્સ કમ્પેરિઝન:}

{\def\LTcaptype{none} % do not increment counter
\begin{longtable}[]{@{}
  >{\raggedright\arraybackslash}p{(\linewidth - 6\tabcolsep) * \real{0.2000}}
  >{\raggedright\arraybackslash}p{(\linewidth - 6\tabcolsep) * \real{0.2000}}
  >{\raggedright\arraybackslash}p{(\linewidth - 6\tabcolsep) * \real{0.3000}}
  >{\raggedright\arraybackslash}p{(\linewidth - 6\tabcolsep) * \real{0.3000}}@{}}
\toprule\noalign{}
\begin{minipage}[b]{\linewidth}\raggedright
મેથડ
\end{minipage} & \begin{minipage}[b]{\linewidth}\raggedright
હેતુ
\end{minipage} & \begin{minipage}[b]{\linewidth}\raggedright
સિન્ટેક્સ
\end{minipage} & \begin{minipage}[b]{\linewidth}\raggedright
ઉદાહરણ
\end{minipage} \\
\midrule\noalign{}
\endhead
\bottomrule\noalign{}
\endlastfoot
\textbf{count()} & ઓકરન્સ ગણવા & str.count(substring) &
``hello''.count(``l'') = 2 \\
\textbf{strip()} & વ્હાઇટસ્પેસ હટાવવો & str.strip() & '' text ``.strip()
=''text'' \\
\textbf{replace()} & સબસ્ટ્રિંગ બદલવો & str.replace(old, new) &
``hi''.replace(``i'', ``ello'') = ``hello'' \\
\end{longtable}
}

\textbf{રિટર્ન વેલ્યુઝ:}

\begin{itemize}
\tightlist
\item
  \textbf{count()}: ઇન્ટિજર (ઓકરન્સની સંખ્યા)
\item
  \textbf{strip()}: નવી સ્ટ્રિંગ (વ્હાઇટસ્પેસ હટાવેલ)
\item
  \textbf{replace()}: નવી સ્ટ્રિંગ (બદલાવ કરેલ)
\end{itemize}

\end{solutionbox}
\begin{mnemonicbox}
``ગણો સ્ટ્રિપ બદલો''

\end{mnemonicbox}
\begin{center}\rule{0.5\linewidth}{0.5pt}\end{center}

\subsection*{પ્રશ્ન 5(બ) [4
ગુણ]}\label{uxaaauxab0uxab6uxaa8-5uxaac-4-uxa97uxaa3}

\textbf{સ્ટ્રિંગમાં કેવી રીતે ટ્રાવર્સલ કરવું તે ઉદાહરણ આપીને સમજાવો.}

\begin{solutionbox}
સ્ટ્રિંગ ટ્રાવર્સલ માનેે સ્ટ્રિંગમાં દરેક કેરેક્ટરને ક્રમિક રીતે એક્સેસ કરવું.

\textbf{ટ્રાવર્સલ મેથડ્સ:}

{\def\LTcaptype{none} % do not increment counter
\begin{longtable}[]{@{}lll@{}}
\toprule\noalign{}
મેથડ & સિન્ટેક્સ & ઉપયોગ \\
\midrule\noalign{}
\endhead
\bottomrule\noalign{}
\endlastfoot
\textbf{ઇન્ડેક્સ-બેઝ્ડ} & for i in range(len(str)) & સ્થિતિ જરૂરી \\
\textbf{ડાયરેક્ટ આઇટરેશન} & for char in string & ફક્ત કેરેક્ટર્સ \\
\textbf{એન્યુમરેટ} & for i, char in enumerate(str) & ઇન્ડેક્સ અને કેરેક્ટર બંને \\
\end{longtable}
}

\textbf{ઉદાહરણ કોડ:}

\begin{verbatim}
text = "Python"

\# મેથડ 1: ડાયરેક્ટ આઇટરેશન
for char in text:
    print(char, end=" ")  \# P y t h o n

\# મેથડ 2: ઇન્ડેક્સ{-બેઝ્ડ}
for i in range(len(text)):
    print(f"\{i\}: \{text[i]\}")

\# મેથડ 3: એન્યુમરેટ
for index, character in enumerate(text):
    print(f"સ્થિતિ \{index\}: \{character\}")
\end{verbatim}

\end{solutionbox}
\begin{mnemonicbox}
``ડાયરેક્ટ ઇન્ડેક્સ એન્યુમરેટ''

\end{mnemonicbox}
\begin{center}\rule{0.5\linewidth}{0.5pt}\end{center}

\subsection*{પ્રશ્ન 5(ક) [7
ગુણ]}\label{uxaaauxab0uxab6uxaa8-5uxa95-7-uxa97uxaa3}

\textbf{નીચેની આપેલ લિસ્ટના ઓપરેશન માટેના પ્રોગ્રામ વિકસાવો:}

\begin{solutionbox}
આવશ્યક લિસ્ટ ઓપરેશન્સ અને વિશ્લેષણ માટેના બે પ્રોગ્રામ્સ.

\textbf{પ્રોગ્રામ 1: એલિમેન્ટ અસ્તિત્વ તપાસવું}

\begin{verbatim}
def check\_element\_exists(lst, element):
    """લિસ્ટમાં એલિમેન્ટ અસ્તિત્વમાં છે કે કેમ તપાસવું"""
    if element in lst:
        return True, lst.index(element)
    else:
        return False, {-}1

\# પ્રોગ્રામ 1 ટેસ્ટ કરવું
numbers = [10, 25, 30, 45, 50]
search\_item = int(input("શોધવા માટે એલિમેન્ટ દાખલ કરો: "))
exists, position = check\_element\_exists(numbers, search\_item)

if exists:
    print(f"\{search\_item\} સ્થિતિ \{position\} પર મળ્યું")
else:
    print(f"\{search\_item\} લિસ્ટમાં નથી મળ્યું")
\end{verbatim}

\textbf{પ્રોગ્રામ 2: સૌથી નાનું અને મોટું શોધવું}

\begin{verbatim}
def find\_min\_max(lst):
    """સૌથી નાના અને મોટા એલિમેન્ટ્સ શોધવા"""
    if not lst:  \# ખાલી લિસ્ટ તપાસવી
        return None, None
    
    smallest = min(lst)
    largest = max(lst)
    return smallest, largest

\# પ્રોગ્રામ 2 ટેસ્ટ કરવું
numbers = [15, 8, 23, 4, 16, 42]
min\_val, max\_val = find\_min\_max(numbers)
print(f"લિસ્ટ: \{numbers\}")
print(f"સૌથી નાનું: \{min\_val\}")
print(f"સૌથી મોટું: \{max\_val\}")
\end{verbatim}

\textbf{મુખ્ય ઓપરેશન્સ:}

\begin{itemize}
\tightlist
\item
  \textbf{મેમ્બરશિપ}: `in' ઓપરેટર વાપરવો
\item
  \textbf{Min/Max}: બિલ્ટ-ઇન ફંકશન્સ
\item
  \textbf{વેલિડેશન}: ખાલી લિસ્ટ હેન્ડલિંગ
\end{itemize}

\end{solutionbox}
\begin{mnemonicbox}
``શોધો મેળવો તુલના કરો''

\end{mnemonicbox}
\begin{center}\rule{0.5\linewidth}{0.5pt}\end{center}

\subsection*{પ્રશ્ન 5(અ અથવા) [3
ગુણ]}\label{uxaaauxab0uxab6uxaa8-5uxa85-uxa85uxaa5uxab5-3-uxa97uxaa3}

\textbf{લિસ્ટનું સ્લાઇસિંગ ઉદાહરણ સાથે સમજાવો.}

\begin{solutionbox}
લિસ્ટ સ્લાઇસિંગ ઇન્ડેક્સ રેન્જ વાપરીને લિસ્ટના ચોક્કસ ભાગો કાઢે છે.

\textbf{સ્લાઇસિંગ સિન્ટેક્સ:}

{\def\LTcaptype{none} % do not increment counter
\begin{longtable}[]{@{}
  >{\raggedright\arraybackslash}p{(\linewidth - 4\tabcolsep) * \real{0.3333}}
  >{\raggedright\arraybackslash}p{(\linewidth - 4\tabcolsep) * \real{0.2917}}
  >{\raggedright\arraybackslash}p{(\linewidth - 4\tabcolsep) * \real{0.3750}}@{}}
\toprule\noalign{}
\begin{minipage}[b]{\linewidth}\raggedright
ફોર્મેટ
\end{minipage} & \begin{minipage}[b]{\linewidth}\raggedright
વર્ણન
\end{minipage} & \begin{minipage}[b]{\linewidth}\raggedright
ઉદાહરણ
\end{minipage} \\
\midrule\noalign{}
\endhead
\bottomrule\noalign{}
\endlastfoot
\textbf{list[start:end]} & start થી end-1 સુધીના એલિમેન્ટ્સ &
[1,2,3,4][1:3] = [2,3] \\
\textbf{list[:end]} & શરૂઆતથી end-1 સુધી & [1,2,3,4][:2] =
[1,2] \\
\textbf{list[start:]} & start થી અંત સુધી & [1,2,3,4][2:] =
[3,4] \\
\textbf{list[::step]} & દરેક step એલિમેન્ટ & [1,2,3,4][::2] =
[1,3] \\
\end{longtable}
}

\textbf{ઉદાહરણ:}

\begin{verbatim}
numbers = [0, 1, 2, 3, 4, 5]
print(numbers[1:4])   \# [1, 2, 3]
print(numbers[:3])    \# [0, 1, 2]
print(numbers[3:])    \# [3, 4, 5]
print(numbers[::2])   \# [0, 2, 4]
\end{verbatim}

\end{solutionbox}
\begin{mnemonicbox}
``શરૂઆત અંત સ્ટેપ''

\end{mnemonicbox}
\begin{center}\rule{0.5\linewidth}{0.5pt}\end{center}

\subsection*{પ્રશ્ન 5(બ અથવા) [4
ગુણ]}\label{uxaaauxab0uxab6uxaa8-5uxaac-uxa85uxaa5uxab5-4-uxa97uxaa3}

\textbf{લિસ્ટમાં કેવી રીતે ટ્રાવર્સલ કરવું તે ઉદાહરણ આપીને સમજાવો.}

\begin{solutionbox}
લિસ્ટ ટ્રાવર્સલમાં લિસ્ટમાં દરેક એલિમેન્ટને વ્યવસ્થિત રીતે એક્સેસ કરવાનો
સમાવેશ થાય છે.

\textbf{ટ્રાવર્સલ ટેકનિક્સ:}

{\def\LTcaptype{none} % do not increment counter
\begin{longtable}[]{@{}lll@{}}
\toprule\noalign{}
મેથડ & સિન્ટેક્સ & આઉટપુટ ટાઇપ \\
\midrule\noalign{}
\endhead
\bottomrule\noalign{}
\endlastfoot
\textbf{વેલ્યુ આઇટરેશન} & for item in list & ફક્ત એલિમેન્ટ્સ \\
\textbf{ઇન્ડેક્સ આઇટરેશન} & for i in range(len(list)) & ઇન્ડેક્સ એક્સેસ \\
\textbf{એન્યુમરેટ} & for i, item in enumerate(list) & ઇન્ડેક્સ અને વેલ્યુ \\
\end{longtable}
}

\textbf{ઉદાહરણ કોડ:}

\begin{verbatim}
fruits = ["સફરજન", "કેળું", "નારંગી"]

\# મેથડ 1: ડાયરેક્ટ વેલ્યુ એક્સેસ
print("ફક્ત વેલ્યુઝ:")
for fruit in fruits:
    print(fruit)

\# મેથડ 2: ઇન્ડેક્સ{-બેઝ્ડ એક્સેસ}
print("{n}ઇન્ડેક્સ સાથે:")
for i in range(len(fruits)):
    print(f"ઇન્ડેક્સ \{i\}: \{fruits[i]\}")

\# મેથડ 3: એન્યુમરેટ
print("{n}એન્યુમરેટ વાપરીને:")
for index, fruit in enumerate(fruits):
    print(f"\{index\} {- }\{fruit\}")
\end{verbatim}

\textbf{ઉપયોગના કેસ:}

\begin{itemize}
\tightlist
\item
  \textbf{ફક્ત વેલ્યુ}: સાદી પ્રોસેસિંગ
\item
  \textbf{ઇન્ડેક્સ એક્સેસ}: પોઝિશન-આધારિત ઓપરેશન્સ
\item
  \textbf{એન્યુમરેટ}: ઇન્ડેક્સ અને વેલ્યુ બંને જરૂરી
\end{itemize}

\end{solutionbox}
\begin{mnemonicbox}
``વેલ્યુ ઇન્ડેક્સ બંને''

\end{mnemonicbox}
\begin{center}\rule{0.5\linewidth}{0.5pt}\end{center}

\subsection*{પ્રશ્ન 5(ક અથવા) [7
ગુણ]}\label{uxaaauxab0uxab6uxaa8-5uxa95-uxa85uxaa5uxab5-7-uxa97uxaa3}

\textbf{1 થી 50 ની શ્રેણીમાં પ્રાઇમ અને નોન-પ્રાઇમ નંબરોનું લિસ્ટ બનાવવા માટે
પાયથોન કોડ વિકસાવો.}

\begin{solutionbox}
નંબરોને પ્રાઇમ અને નોન-પ્રાઇમ લિસ્ટ્સમાં વર્ગીકૃત કરવા માટેનો Python
પ્રોગ્રામ.

\textbf{Python કોડ:}

\begin{verbatim}
def is\_prime(n):
    """નંબર પ્રાઇમ છે કે કેમ તે તપાસવું"""
    if n {} 2:
        return False
    for i in range(2, int(n**0.5) + 1):
if n \%

i == 0:

            return False
    return True

def categorize\_numbers(start, end):
    """પ્રાઇમ અને નોન{-પ્રાઇમ નંબરોની લિસ્ટ બનાવવી"""}
    prime\_numbers = []
    non\_prime\_numbers = []
    
    for num in range(start, end + 1):
        if is\_prime(num):
            prime\_numbers.append(num)
        else:
            non\_prime\_numbers.append(num)
    
    return prime\_numbers, non\_prime\_numbers

\# 1 થી 50 માટે લિસ્ટ્સ જનરેટ કરવી
primes, non\_primes = categorize\_numbers(1, 50)

print("પ્રાઇમ નંબરો (1{-50):"})
print(primes)
print(f"{n}કુલ પ્રાઇમ નંબરો: \{len(primes)\}")

print("{n}નોન{-પ્રાઇમ નંબરો (1{-}50):"})
print(non\_primes)
print(f"{n}કુલ નોન{-પ્રાઇમ નંબરો: }\{len(non\_primes)\}")
\end{verbatim}

\textbf{પ્રાઇમ લોજિક:}

{\def\LTcaptype{none} % do not increment counter
\begin{longtable}[]{@{}lll@{}}
\toprule\noalign{}
નંબર પ્રકાર & શરત & ઉદાહરણો \\
\midrule\noalign{}
\endhead
\bottomrule\noalign{}
\endlastfoot
\textbf{પ્રાઇમ} & ફક્ત 1 અને પોતાના વડે જ ભાગાય & 2, 3, 5, 7, 11 \\
\textbf{નોન-પ્રાઇમ} & અન્ય ભાજકો છે & 1, 4, 6, 8, 9 \\
\end{longtable}
}

\textbf{એલ્ગોરિધમ સ્ટેપ્સ:}

\begin{itemize}
\tightlist
\item
  \textbf{ભાજ્યતા તપાસવી} 2 થી \sqrtn સુધી
\item
  \textbf{વર્ગીકરણ} પ્રાઇમ ટેસ્ટના આધારે
\item
  \textbf{સ્ટોર} યોગ્ય લિસ્ટ્સમાં
\end{itemize}

\end{solutionbox}
\begin{mnemonicbox}
``તપાસો ભાગો વર્ગીકૃત કરો સ્ટોર કરો''

\end{mnemonicbox}

\end{document}
