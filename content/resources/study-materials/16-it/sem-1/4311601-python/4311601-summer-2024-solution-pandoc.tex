\documentclass[10pt,a4paper]{article}

% content/resources/templates/preamble.tex
\usepackage[margin=0.6in]{geometry}
\author{Milav Dabgar}
\usepackage{amsmath,amssymb,amsthm}
\usepackage{booktabs}
\usepackage{multirow}
\usepackage{xcolor}
\usepackage{tcolorbox}
\tcbuselibrary{breakable,skins}
\usepackage[colorlinks=true,linkcolor=blue]{hyperref}
\usepackage{titlesec}
\usepackage{enumitem}
\usepackage{tikz}
\usepackage{pgfplots}
\usepackage{circuitikz}
\usepackage[version=4]{mhchem}
\usepackage{longtable}
\usepackage{array}
\usepackage{float}
\usepackage{caption}
\usepackage{listings}

\lstset{
  basicstyle=\small\ttfamily,
  breaklines=true,
  breakatwhitespace=false,
  postbreak=\mbox{\textcolor{red}{$\hookrightarrow$}\space},
  float=false,
  numbers=left,
  numberstyle=\tiny\color{gray},
  numbersep=10pt,
  xleftmargin=2em,
  keywordstyle=\color{blue},
  commentstyle=\color{green!60!black},
  stringstyle=\color{purple},
  backgroundcolor=\color{gray!5},
  showstringspaces=false,
  tabsize=2,
  captionpos=b,
  keepspaces=true,
  columns=flexible
}

\pgfplotsset{compat=1.18}
\usetikzlibrary{shapes,arrows,positioning,calc,patterns,decorations.pathmorphing,decorations.markings,arrows.meta}

% Color scheme
\definecolor{headcolor}{RGB}{0,102,204}
\definecolor{keycolor}{RGB}{220,20,60}
\definecolor{solutioncolor}{RGB}{34,139,34}
\definecolor{mnemoniccolor}{RGB}{148,0,211}
\definecolor{codecolor}{RGB}{0,0,100}

% Spacing
\setlength{\parskip}{3pt}
\setlist[itemize]{nosep}
\setlist[enumerate]{nosep}

% Title formatting
\titleformat{\section}{\Large\bfseries\color{headcolor}}{\thesection}{1em}{}
\titleformat{\subsection}{\large\bfseries\color{headcolor}}{\thesubsection}{1em}{}

% Pandoc tightlist compatibility
\providecommand{\tightlist}{%
  \setlength{\itemsep}{0pt}\setlength{\parskip}{0pt}}

% Pandoc longtable compatibility
\newcounter{none}
\def\thenone{}


% content/resources/templates/english-boxes.tex
% This file is currently empty - it exists to maintain consistency with the import structure.
% Add custom environments here if needed in the future.


\begin{document}

\begin{center}
{\Huge\bfseries\color{headcolor} Subject Name Solutions}\\[5pt]
{\LARGE 4311601 -- Summer 2024}\\[3pt]
{\large Semester 1 Study Material}\\[3pt]
{\normalsize\textit{Detailed Solutions and Explanations}}
\end{center}

\vspace{10pt}

\subsection*{Question 1(a) [3 marks]}\label{q1a}

\textbf{Define problem solving and list out the steps of problem
solving.}

\begin{solutionbox}
Problem solving is a systematic approach to identify,
analyze, and resolve challenges or issues using logical thinking and
structured methods.

\textbf{Steps of Problem Solving:}

{\def\LTcaptype{none} % do not increment counter
\begin{longtable}[]{@{}
  >{\raggedright\arraybackslash}p{(\linewidth - 2\tabcolsep) * \real{0.3158}}
  >{\raggedright\arraybackslash}p{(\linewidth - 2\tabcolsep) * \real{0.6842}}@{}}
\toprule\noalign{}
\begin{minipage}[b]{\linewidth}\raggedright
Step
\end{minipage} & \begin{minipage}[b]{\linewidth}\raggedright
Description
\end{minipage} \\
\midrule\noalign{}
\endhead
\bottomrule\noalign{}
\endlastfoot
1. \textbf{Problem Identification} & Clearly understand and define the
problem \\
2. \textbf{Problem Analysis} & Break down the problem into smaller
parts \\
3. \textbf{Solution Design} & Develop possible solutions or
algorithms \\
4. \textbf{Implementation} & Execute the chosen solution \\
5. \textbf{Testing \& Validation} & Verify the solution works
correctly \\
6. \textbf{Documentation} & Record the solution for future reference \\
\end{longtable}
}

\end{solutionbox}
\begin{mnemonicbox}
``I Always Design Implementation Tests Daily''

\end{mnemonicbox}
\begin{center}\rule{0.5\linewidth}{0.5pt}\end{center}

\subsection*{Question 1(b) [4 marks]}\label{q1b}

\textbf{Define variable and mention rule for choosing names of
variable.}

\begin{solutionbox}
A variable is a named storage location in memory that
holds data values which can be changed during program execution.

\textbf{Variable Naming Rules:}

{\def\LTcaptype{none} % do not increment counter
\begin{longtable}[]{@{}
  >{\raggedright\arraybackslash}p{(\linewidth - 2\tabcolsep) * \real{0.3158}}
  >{\raggedright\arraybackslash}p{(\linewidth - 2\tabcolsep) * \real{0.6842}}@{}}
\toprule\noalign{}
\begin{minipage}[b]{\linewidth}\raggedright
Rule
\end{minipage} & \begin{minipage}[b]{\linewidth}\raggedright
Description
\end{minipage} \\
\midrule\noalign{}
\endhead
\bottomrule\noalign{}
\endlastfoot
\textbf{Start Character} & Must begin with letter (a-z, A-Z) or
underscore (\_) \\
\textbf{Allowed Characters} & Can contain letters, digits (0-9), and
underscores \\
\textbf{Case Sensitive} & myVar and MyVar are different variables \\
\textbf{No Keywords} & Cannot use Python reserved words (if, for,
while) \\
\textbf{No Spaces} & Use underscore instead of spaces \\
\textbf{Descriptive Names} & Choose meaningful names (age, not x) \\
\end{longtable}
}

\end{solutionbox}
\begin{mnemonicbox}
``Start Alphabetically, Continue Carefully, Never
Keywords''

\end{mnemonicbox}
\begin{center}\rule{0.5\linewidth}{0.5pt}\end{center}

\subsection*{Question 1(c) [7 marks]}\label{q1c}

\textbf{Design a flowchart to find maximum number out of three given
numbers.}

\begin{solutionbox}
A flowchart shows the logical flow to find the maximum
of three numbers using comparison operations.

\textbf{Flowchart:}

\begin{verbatim}
flowchart LR
    A[Start] {-{-} B[Input: num1, num2, num3]}
    B {-{-} C\{num1  num2?\}}
    C {-{-}|Yes| D\{num1  num3?\}}
    C {-{-}|No| E\{num2  num3?\}}
    D {-{-}|Yes| F[max = num1]}
    D {-{-}|No| G[max = num3]}
    E {-{-}|Yes| H[max = num2]}
    E {-{-}|No| I[max = num3]}
    F {-{-} J[Output: max]}
    G {-{-} J}
    H {-{-} J}
    I {-{-} J}
    J {-{-} K[End]}
\end{verbatim}

\textbf{Key Points:}

\begin{itemize}
\tightlist
\item
  \textbf{Input}: Three numbers (num1, num2, num3)
\item
  \textbf{Process}: Compare numbers using nested conditions
\item
  \textbf{Output}: Maximum value among the three
\end{itemize}

\end{solutionbox}
\begin{mnemonicbox}
``Compare First Two, Then With Third''

\end{mnemonicbox}
\begin{center}\rule{0.5\linewidth}{0.5pt}\end{center}

\subsection*{Question 1(c OR) [7
marks]}\label{question-1c-or-7-marks}

\textbf{Construct an algorithm which checks entered number is positive
and greater than 5 or not.}

\begin{solutionbox}
An algorithm to verify if a number is both positive and
greater than 5.

\textbf{Algorithm:}

\begin{verbatim}
Algorithm: CheckPositiveGreaterThan5
Step 1: START
Step 2: INPUT number
Step 3: IF number > 0 AND number > 5 THEN
           PRINT "Number is positive and greater than 5"
        ELSE
           PRINT "Number does not meet criteria"
        END IF
Step 4: END
\end{verbatim}

\textbf{Flowchart:}

\begin{verbatim}
flowchart LR
    A[Start] {-{-} B[Input: number]}
    B {-{-} C\{number  0 AND number  5?\}}
    C {-{-}|Yes| D[Print: Number is positive and greater than 5]}
    C {-{-}|No| E[Print: Number does not meet criteria]}
    D {-{-} F[End]}
    E {-{-} F}
\end{verbatim}

\textbf{Key Conditions:}

\begin{itemize}
\tightlist
\item
  \textbf{Positive}: number \textgreater{} 0
\item
  \textbf{Greater than 5}: number \textgreater{} 5
\item
  \textbf{Combined}: Both conditions must be true
\end{itemize}

\end{solutionbox}
\begin{mnemonicbox}
``Positive Plus Five''

\end{mnemonicbox}
\begin{center}\rule{0.5\linewidth}{0.5pt}\end{center}

\subsection*{Question 2(a) [3 marks]}\label{q2a}

\textbf{Write a short note on arithmetic operators.}

\begin{solutionbox}
Arithmetic operators perform mathematical calculations
on numeric values in Python programming.

\textbf{Arithmetic Operators Table:}

{\def\LTcaptype{none} % do not increment counter
\begin{longtable}[]{@{}llll@{}}
\toprule\noalign{}
Operator & Name & Example & Result \\
\midrule\noalign{}
\endhead
\bottomrule\noalign{}
\endlastfoot
+ & Addition & 5 + 3 & 8 \\
- & Subtraction & 5 - 3 & 2 \\
* & Multiplication & 5 * 3 & 15 \\
/ & Division & 5 / 3 & 1.67 \\
// & Floor Division & 5 // 3 & 1 \\
\% & Modulus & 5 \% 3 & 2 \\
** & Exponentiation & 5 ** 3 & 125 \\
\end{longtable}
}

\end{solutionbox}
\begin{mnemonicbox}
``Add Subtract Multiply Divide Floor Mod Power''

\end{mnemonicbox}
\begin{center}\rule{0.5\linewidth}{0.5pt}\end{center}

\subsection*{Question 2(b) [4 marks]}\label{q2b}

\textbf{Explain the need for continue and break statements.}

\begin{solutionbox}
Continue and break statements control loop execution
flow for efficient programming.

\textbf{Statement Comparison:}

{\def\LTcaptype{none} % do not increment counter
\begin{longtable}[]{@{}lll@{}}
\toprule\noalign{}
Statement & Purpose & Action \\
\midrule\noalign{}
\endhead
\bottomrule\noalign{}
\endlastfoot
\textbf{break} & Exit loop completely & Terminates entire loop \\
\textbf{continue} & Skip current iteration & Jumps to next iteration \\
\end{longtable}
}

\textbf{Usage Examples:}

\begin{itemize}
\tightlist
\item
  \textbf{break}: Exit when condition met (finding specific value)
\item
  \textbf{continue}: Skip invalid data (negative numbers in positive
  list)
\end{itemize}

\textbf{Benefits:}

\begin{itemize}
\tightlist
\item
  \textbf{Efficiency}: Avoid unnecessary iterations
\item
  \textbf{Control}: Better program flow management
\item
  \textbf{Clarity}: Cleaner code logic
\end{itemize}

\end{solutionbox}
\begin{mnemonicbox}
``Break Exits, Continue Skips''

\end{mnemonicbox}
\begin{center}\rule{0.5\linewidth}{0.5pt}\end{center}

\subsection*{Question 2(c) [7 marks]}\label{q2c}

\textbf{Create a program to check whether entered number is even or
odd.}

\begin{solutionbox}
A Python program using modulus operator to determine if
a number is even or odd.

\textbf{Python Code:}

\begin{verbatim}
\# Program to check even or odd
number = int(input("Enter a number: "))

if number \% 2 == 0:
    print(f"\{number\} is Even")
else:
    print(f"\{number\} is Odd")
\end{verbatim}

\textbf{Logic Explanation:}

{\def\LTcaptype{none} % do not increment counter
\begin{longtable}[]{@{}lll@{}}
\toprule\noalign{}
Condition & Result & Explanation \\
\midrule\noalign{}
\endhead
\bottomrule\noalign{}
\endlastfoot
number \% 2 == 0 & Even & Divisible by 2, no remainder \\
number \% 2 == 1 & Odd & Not divisible by 2, remainder 1 \\
\end{longtable}
}

\textbf{Sample Output:}

\begin{itemize}
\tightlist
\item
  Input: 8 \rightarrow Output: ``8 is Even''
\item
  Input: 7 \rightarrow Output: ``7 is Odd''
\end{itemize}

\end{solutionbox}
\begin{mnemonicbox}
``Modulus Zero Even, One Odd''

\end{mnemonicbox}
\begin{center}\rule{0.5\linewidth}{0.5pt}\end{center}

\subsection*{Question 2(a OR) [3
marks]}\label{question-2a-or-3-marks}

\textbf{Summarize the comparison operators of python.}

\begin{solutionbox}
Comparison operators compare values and return boolean
results (True/False).

\textbf{Comparison Operators Table:}

{\def\LTcaptype{none} % do not increment counter
\begin{longtable}[]{@{}llll@{}}
\toprule\noalign{}
Operator & Name & Example & Result \\
\midrule\noalign{}
\endhead
\bottomrule\noalign{}
\endlastfoot
== & Equal to & 5 == 5 & True \\
!= & Not equal to & 5 != 3 & True \\
\textgreater{} & Greater than & 5 \textgreater{} 3 & True \\
\textless{} & Less than & 5 \textless{} 3 & False \\
\textgreater= & Greater than or equal & 5 \textgreater= 5 & True \\
\textless= & Less than or equal & 5 \textless= 3 & False \\
\end{longtable}
}

\textbf{Return Type:} All operators return boolean values (True/False)

\end{solutionbox}
\begin{mnemonicbox}
``Equal Not Greater Less Greater-Equal Less-Equal''

\end{mnemonicbox}
\begin{center}\rule{0.5\linewidth}{0.5pt}\end{center}

\subsection*{Question 2(b OR) [4
marks]}\label{question-2b-or-4-marks}

\textbf{Write short note on while loop.}

\begin{solutionbox}
While loop repeatedly executes code block as long as
condition remains true.

\textbf{While Loop Structure:}

{\def\LTcaptype{none} % do not increment counter
\begin{longtable}[]{@{}ll@{}}
\toprule\noalign{}
Component & Description \\
\midrule\noalign{}
\endhead
\bottomrule\noalign{}
\endlastfoot
\textbf{Initialization} & Set initial value before loop \\
\textbf{Condition} & Boolean expression to test \\
\textbf{Body} & Code to execute repeatedly \\
\textbf{Update} & Modify variable to avoid infinite loop \\
\end{longtable}
}

\textbf{Syntax:}

\begin{verbatim}
while condition:
    \# loop body
    \# update statement
\end{verbatim}

\textbf{Characteristics:}

\begin{itemize}
\tightlist
\item
  \textbf{Pre-tested}: Condition checked before execution
\item
  \textbf{Variable iterations}: Unknown number of repetitions
\item
  \textbf{Control}: Condition determines continuation
\end{itemize}

\end{solutionbox}
\begin{mnemonicbox}
``While Condition True, Execute Loop''

\end{mnemonicbox}
\begin{center}\rule{0.5\linewidth}{0.5pt}\end{center}

\subsection*{Question 2(c OR) [7
marks]}\label{question-2c-or-7-marks}

\textbf{Create a program to read three numbers from the user and find
the average of the numbers.}

\begin{solutionbox}
A Python program to calculate average of three
user-input numbers.

\textbf{Python Code:}

\begin{verbatim}
\# Program to find average of three numbers
num1 = float(input("Enter first number: "))
num2 = float(input("Enter second number: "))
num3 = float(input("Enter third number: "))

average = (num1 + num2 + num3) / 3

print(f"Average of \{num1\}, \{num2\}, \{num3\} is: \{average:.2f\}")
\end{verbatim}

\textbf{Calculation Process:}

{\def\LTcaptype{none} % do not increment counter
\begin{longtable}[]{@{}ll@{}}
\toprule\noalign{}
Step & Operation \\
\midrule\noalign{}
\endhead
\bottomrule\noalign{}
\endlastfoot
\textbf{Input} & Read three numbers \\
\textbf{Sum} & Add all three numbers \\
\textbf{Divide} & Sum \div 3 \\
\textbf{Output} & Display formatted result \\
\end{longtable}
}

\textbf{Sample Execution:}

\begin{itemize}
\tightlist
\item
  Input: 10, 20, 30
\item
  Sum: 60
\item
  Average: 20.00
\end{itemize}

\end{solutionbox}
\begin{mnemonicbox}
``Sum Three Divide Display''

\end{mnemonicbox}
\begin{center}\rule{0.5\linewidth}{0.5pt}\end{center}

\subsection*{Question 3(a) [3 marks]}\label{q3a}

\textbf{Define control structures, List out control structures available
in python.}

\begin{solutionbox}
Control structures determine the execution flow and
order of statements in a program.

\textbf{Python Control Structures:}

{\def\LTcaptype{none} % do not increment counter
\begin{longtable}[]{@{}lll@{}}
\toprule\noalign{}
Type & Structures & Purpose \\
\midrule\noalign{}
\endhead
\bottomrule\noalign{}
\endlastfoot
\textbf{Sequential} & Normal flow & Execute statements in order \\
\textbf{Selection} & if, if-else, elif & Choose between alternatives \\
\textbf{Iteration} & for, while & Repeat code blocks \\
\textbf{Jump} & break, continue, pass & Alter normal flow \\
\end{longtable}
}

\textbf{Categories:}

\begin{itemize}
\tightlist
\item
  \textbf{Conditional}: Decision making (if statements)
\item
  \textbf{Looping}: Repetition (for/while loops)
\item
  \textbf{Branching}: Flow control (break/continue)
\end{itemize}

\end{solutionbox}
\begin{mnemonicbox}
``Sequence Select Iterate Jump''

\end{mnemonicbox}
\begin{center}\rule{0.5\linewidth}{0.5pt}\end{center}

\subsection*{Question 3(b) [4 marks]}\label{q3b}

\textbf{Explain how to define and call user defined function by giving
example.}

\begin{solutionbox}
User-defined functions are custom blocks of reusable
code that perform specific tasks.

\textbf{Function Structure:}

{\def\LTcaptype{none} % do not increment counter
\begin{longtable}[]{@{}lll@{}}
\toprule\noalign{}
Component & Syntax & Purpose \\
\midrule\noalign{}
\endhead
\bottomrule\noalign{}
\endlastfoot
\textbf{Definition} & def function\_name(): & Create function \\
\textbf{Parameters} & def func(param1, param2): & Accept inputs \\
\textbf{Body} & Indented code block & Function logic \\
\textbf{Return} & return value & Send result back \\
\textbf{Call} & function\_name() & Execute function \\
\end{longtable}
}

\textbf{Example Code:}

\begin{verbatim}
\# Function definition
def greet\_user(name):
    message = f"Hello, \{name\}!"
    return message

\# Function call
result = greet\_user("Python")
print(result)  \# Output: Hello, Python!
\end{verbatim}

\end{solutionbox}
\begin{mnemonicbox}
``Define Parameters Body Return Call''

\end{mnemonicbox}
\begin{center}\rule{0.5\linewidth}{0.5pt}\end{center}

\subsection*{Question 3(c) [7 marks]}\label{q3c}

\textbf{Create a program to display the following patterns using loop
concept}

\begin{solutionbox}
A Python program using nested loops to create number
patterns.

\textbf{Python Code:}

\begin{verbatim}
\# Pattern printing program
for i in range(1, 6):
    for j in range(1, i + 1):
        print(i, end="")
    print()  \# New line after each row
\end{verbatim}

\textbf{Pattern Logic:}

{\def\LTcaptype{none} % do not increment counter
\begin{longtable}[]{@{}lll@{}}
\toprule\noalign{}
Row & Iterations & Output \\
\midrule\noalign{}
\endhead
\bottomrule\noalign{}
\endlastfoot
1 & 1 time & 1 \\
2 & 2 times & 22 \\
3 & 3 times & 333 \\
4 & 4 times & 4444 \\
5 & 5 times & 55555 \\
\end{longtable}
}

\textbf{Loop Structure:}

\begin{itemize}
\tightlist
\item
  \textbf{Outer loop}: Controls rows (1 to 5)
\item
  \textbf{Inner loop}: Prints current row number
\item
  \textbf{Pattern}: Row number repeated row times
\end{itemize}

\end{solutionbox}
\begin{mnemonicbox}
``Outer Rows Inner Repeats''

\end{mnemonicbox}
\begin{center}\rule{0.5\linewidth}{0.5pt}\end{center}

\subsection*{Question 3(a OR) [3
marks]}\label{question-3a-or-3-marks}

\textbf{Explain nested loop using suitable example.}

\begin{solutionbox}
Nested loop is a loop inside another loop where inner
loop completes all iterations for each outer loop iteration.

\textbf{Nested Loop Structure:}

{\def\LTcaptype{none} % do not increment counter
\begin{longtable}[]{@{}ll@{}}
\toprule\noalign{}
Component & Description \\
\midrule\noalign{}
\endhead
\bottomrule\noalign{}
\endlastfoot
\textbf{Outer Loop} & Controls main iterations \\
\textbf{Inner Loop} & Executes completely for each outer iteration \\
\textbf{Execution} & Inner loop runs n\timesm times total \\
\end{longtable}
}

\textbf{Example Code:}

\begin{verbatim}
\# Nested loop example {- Multiplication table}
for i in range(1, 4):      \# Outer loop
    for j in range(1, 4):  \# Inner loop
        print(f"\{i\}\{j\}=\{i*j\}", end=" ")
    print()  \# New line
\end{verbatim}

\textbf{Output Pattern:}

\begin{verbatim}
1\times1=1 1\times2=2 1\times3=3
2\times1=2 2\times2=4 2\times3=6
3\times1=3 3\times2=6 3\times3=9
\end{verbatim}

\end{solutionbox}
\begin{mnemonicbox}
``Loop Inside Loop''

\end{mnemonicbox}
\begin{center}\rule{0.5\linewidth}{0.5pt}\end{center}

\subsection*{Question 3(b OR) [4
marks]}\label{question-3b-or-4-marks}

\textbf{Write short note on local and global scope of variables}

\begin{solutionbox}
Variable scope determines where variables can be
accessed in a program.

\textbf{Scope Comparison:}

{\def\LTcaptype{none} % do not increment counter
\begin{longtable}[]{@{}llll@{}}
\toprule\noalign{}
Scope Type & Definition & Access & Lifetime \\
\midrule\noalign{}
\endhead
\bottomrule\noalign{}
\endlastfoot
\textbf{Local} & Inside function & Function only & Function execution \\
\textbf{Global} & Outside functions & Entire program & Program
execution \\
\end{longtable}
}

\textbf{Example Code:}

\begin{verbatim}
global\_var = "I am global"  \# Global scope

def my\_function():
    local\_var = "I am local"    \# Local scope
    global global\_var
    print(global\_var)   \# Accessible
    print(local\_var)    \# Accessible

print(global\_var)   \# Accessible
\# print(local\_var)  \# Error {- not accessible}
\end{verbatim}

\textbf{Key Points:}

\begin{itemize}
\tightlist
\item
  \textbf{Local}: Function-specific variables
\item
  \textbf{Global}: Program-wide variables
\item
  \textbf{Access}: Local overrides global in functions
\end{itemize}

\end{solutionbox}
\begin{mnemonicbox}
``Local Limited, Global General''

\end{mnemonicbox}
\begin{center}\rule{0.5\linewidth}{0.5pt}\end{center}

\subsection*{Question 3(c OR) [7
marks]}\label{question-3c-or-7-marks}

\textbf{Develop a user-defined function to find the factorial of a given
number.}

\begin{solutionbox}
A recursive function to calculate factorial of a
positive integer.

\textbf{Python Code:}

\begin{verbatim}
def factorial(n):
    """Calculate factorial of n"""
if

n == 0 or

n == 1:

        return 1
    else:
        return n * factorial(n {-} 1)

\# Test the function
number = int(input("Enter a number: "))
if number {} 0:
    print("Factorial not defined for negative numbers")
else:
    result = factorial(number)
    print(f"Factorial of \{number\} is \{result\}")
\end{verbatim}

\textbf{Factorial Logic:}

{\def\LTcaptype{none} % do not increment counter
\begin{longtable}[]{@{}lll@{}}
\toprule\noalign{}
Input & Calculation & Result \\
\midrule\noalign{}
\endhead
\bottomrule\noalign{}
\endlastfoot
0 & Base case & 1 \\
1 & Base case & 1 \\
5 & 5 \times 4 \times 3 \times 2 \times 1 & 120 \\
\end{longtable}
}

\textbf{Function Features:}

\begin{itemize}
\tightlist
\item
  \textbf{Recursive}: Function calls itself
\item
  \textbf{Base case}: Stops recursion at n=0 or n=1
\item
  \textbf{Validation}: Handles negative inputs
\end{itemize}

\end{solutionbox}
\begin{mnemonicbox}
``Multiply All Previous Numbers''

\end{mnemonicbox}
\begin{center}\rule{0.5\linewidth}{0.5pt}\end{center}

\subsection*{Question 4(a) [3 marks]}\label{q4a}

\textbf{Explain math module with various functions}

\begin{solutionbox}
Math module provides mathematical functions and
constants for numerical computations.

\textbf{Math Module Functions:}

{\def\LTcaptype{none} % do not increment counter
\begin{longtable}[]{@{}lll@{}}
\toprule\noalign{}
Function & Purpose & Example \\
\midrule\noalign{}
\endhead
\bottomrule\noalign{}
\endlastfoot
\textbf{math.sqrt()} & Square root & math.sqrt(16) = 4.0 \\
\textbf{math.pow()} & Power calculation & math.pow(2, 3) = 8.0 \\
\textbf{math.ceil()} & Round up & math.ceil(4.3) = 5 \\
\textbf{math.floor()} & Round down & math.floor(4.7) = 4 \\
\textbf{math.factorial()} & Factorial & math.factorial(5) = 120 \\
\end{longtable}
}

\textbf{Usage:}

\begin{verbatim}
import math
result = math.sqrt(25)  \# Returns 5.0
\end{verbatim}

\end{solutionbox}
\begin{mnemonicbox}
``Square Power Ceiling Floor Factorial''

\end{mnemonicbox}
\begin{center}\rule{0.5\linewidth}{0.5pt}\end{center}

\subsection*{Question 4(b) [4 marks]}\label{q4b}

\textbf{Discuss the following list functions: i. len() ii. sum() iii.
sort() iv. index()}

\begin{solutionbox}
Essential list functions for data manipulation and
analysis.

\textbf{List Functions Comparison:}

{\def\LTcaptype{none} % do not increment counter
\begin{longtable}[]{@{}llll@{}}
\toprule\noalign{}
Function & Purpose & Return Type & Example \\
\midrule\noalign{}
\endhead
\bottomrule\noalign{}
\endlastfoot
\textbf{len()} & Count elements & Integer & len([1,2,3]) = 3 \\
\textbf{sum()} & Add all numbers & Number & sum([1,2,3]) = 6 \\
\textbf{sort()} & Arrange in order & None (modifies list) &
list.sort() \\
\textbf{index()} & Find element position & Integer &
[1,2,3].index(2) = 1 \\
\end{longtable}
}

\textbf{Usage Notes:}

\begin{itemize}
\tightlist
\item
  \textbf{len()}: Works with any sequence
\item
  \textbf{sum()}: Only numeric lists
\item
  \textbf{sort()}: Modifies original list
\item
  \textbf{index()}: Returns first occurrence
\end{itemize}

\end{solutionbox}
\begin{mnemonicbox}
``Length Sum Sort Index''

\end{mnemonicbox}
\begin{center}\rule{0.5\linewidth}{0.5pt}\end{center}

\subsection*{Question 4(c) [7 marks]}\label{q4c}

\textbf{Create a user-defined function to print the Fibonacci series of
0 to N numbers. (Where N is an integer number and passed as an
argument)}

\begin{solutionbox}
A function to generate and display Fibonacci sequence
up to N terms.

\textbf{Python Code:}

\begin{verbatim}
def fibonacci\_series(n):
    """Print Fibonacci series of n terms"""
    if n {=} 0:
        print("Please enter a positive number")
        return
    
    \# First two terms
    a, b = 0, 1
    
if

n == 1:

        print(f"Fibonacci series: \{a\}")
        return
    
    print(f"Fibonacci series: \{a\}, \{b\}", end="")
    
    \# Generate remaining terms
    for i in range(2, n):
        c = a + b
        print(f", \{c\}", end="")
        a, b = b, c
    print()  \# New line

\# Test function
num = int(input("Enter number of terms: "))
fibonacci\_series(num)
\end{verbatim}

\textbf{Fibonacci Logic:}

{\def\LTcaptype{none} % do not increment counter
\begin{longtable}[]{@{}lll@{}}
\toprule\noalign{}
Term & Value & Calculation \\
\midrule\noalign{}
\endhead
\bottomrule\noalign{}
\endlastfoot
1st & 0 & Given \\
2nd & 1 & Given \\
3rd & 1 & 0 + 1 \\
4th & 2 & 1 + 1 \\
5th & 3 & 1 + 2 \\
\end{longtable}
}

\end{solutionbox}
\begin{mnemonicbox}
``Add Previous Two Numbers''

\end{mnemonicbox}
\begin{center}\rule{0.5\linewidth}{0.5pt}\end{center}

\subsection*{Question 4(a OR) [3
marks]}\label{question-4a-or-3-marks}

\textbf{Explain random module with various functions}

\begin{solutionbox}
Random module generates random numbers and makes random
selections for various applications.

\textbf{Random Module Functions:}

{\def\LTcaptype{none} % do not increment counter
\begin{longtable}[]{@{}lll@{}}
\toprule\noalign{}
Function & Purpose & Example \\
\midrule\noalign{}
\endhead
\bottomrule\noalign{}
\endlastfoot
\textbf{random()} & Float 0.0 to 1.0 & random.random() \\
\textbf{randint()} & Integer in range & random.randint(1, 10) \\
\textbf{choice()} & Random list element & random.choice([1,2,3]) \\
\textbf{shuffle()} & Mix list order & random.shuffle(list) \\
\textbf{uniform()} & Float in range & random.uniform(1.0, 5.0) \\
\end{longtable}
}

\textbf{Usage:}

\begin{verbatim}
import random
number = random.randint(1, 100)
\end{verbatim}

\textbf{Applications:} Games, simulations, testing, cryptography

\end{solutionbox}
\begin{mnemonicbox}
``Random Range Choice Shuffle Uniform''

\end{mnemonicbox}
\begin{center}\rule{0.5\linewidth}{0.5pt}\end{center}

\subsection*{Question 4(b OR) [4
marks]}\label{question-4b-or-4-marks}

\textbf{Build a python code to check whether given element is member of
list or not.}

\begin{solutionbox}
A Python program to verify if an element exists in a
list using membership operator.

\textbf{Python Code:}

\begin{verbatim}
\# Check element membership in list
def check\_membership():
    \# Sample list
    numbers = [10, 20, 30, 40, 50]
    
    \# Get element to search
    element = int(input("Enter element to search: "))
    
    \# Check membership
    if element in numbers:
        print(f"\{element\} is present in the list")
        print(f"Position: \{numbers.index(element)\}")
    else:
        print(f"\{element\} is not present in the list")

\# Call function
check\_membership()
\end{verbatim}

\textbf{Membership Methods:}

{\def\LTcaptype{none} % do not increment counter
\begin{longtable}[]{@{}lll@{}}
\toprule\noalign{}
Method & Syntax & Returns \\
\midrule\noalign{}
\endhead
\bottomrule\noalign{}
\endlastfoot
\textbf{in operator} & element in list & Boolean \\
\textbf{not in operator} & element not in list & Boolean \\
\textbf{count() method} & list.count(element) & Integer \\
\end{longtable}
}

\end{solutionbox}
\begin{mnemonicbox}
``In List True False''

\end{mnemonicbox}
\begin{center}\rule{0.5\linewidth}{0.5pt}\end{center}

\subsection*{Question 4(c OR) [7
marks]}\label{question-4c-or-7-marks}

\textbf{Develop a user defined function that reverses the entered string
words}

\begin{solutionbox}
A function to reverse each word in a string while
maintaining word positions.

\textbf{Python Code:}

\begin{verbatim}
def reverse\_string\_words(text):
    """Reverse each word in the string"""
    \# Split string into words
    words = text.split()
    
    \# Reverse each word
    reversed\_words = []
    for word in words:
        reversed\_word = word[::{-}1]  \# Slice notation for reversal
        reversed\_words.append(reversed\_word)
    
    \# Join words back
    result = " ".join(reversed\_words)
    return result

\# Test function
input\_string = input("Enter a string: ")
output = reverse\_string\_words(input\_string)
print(f"Input: {"}\{input\_string\}{"}")
print(f"Output: {"}\{output\}{"}")

\# Example with given input
test\_input = "Hello IT"
test\_output = reverse\_string\_words(test\_input)
print(f"Input: {"}\{test\_input\}{"}")
print(f"Output: {"}\{test\_output\}{"}")  \# Output: "olleH TI"
\end{verbatim}

\textbf{Process Steps:}

{\def\LTcaptype{none} % do not increment counter
\begin{longtable}[]{@{}lll@{}}
\toprule\noalign{}
Step & Operation & Example \\
\midrule\noalign{}
\endhead
\bottomrule\noalign{}
\endlastfoot
1 & Split into words & [``Hello'', ``IT''] \\
2 & Reverse each word & [``olleH'', ``TI''] \\
3 & Join with spaces & ``olleH TI'' \\
\end{longtable}
}

\end{solutionbox}
\begin{mnemonicbox}
``Split Reverse Join''

\end{mnemonicbox}
\begin{center}\rule{0.5\linewidth}{0.5pt}\end{center}

\subsection*{Question 5(a) [3 marks]}\label{q5a}

\textbf{Explain given string methods: i. count() ii. strip() iii.
replace()}

\begin{solutionbox}
Essential string methods for text processing and
manipulation.

\textbf{String Methods Comparison:}

{\def\LTcaptype{none} % do not increment counter
\begin{longtable}[]{@{}
  >{\raggedright\arraybackslash}p{(\linewidth - 6\tabcolsep) * \real{0.2353}}
  >{\raggedright\arraybackslash}p{(\linewidth - 6\tabcolsep) * \real{0.2647}}
  >{\raggedright\arraybackslash}p{(\linewidth - 6\tabcolsep) * \real{0.2353}}
  >{\raggedright\arraybackslash}p{(\linewidth - 6\tabcolsep) * \real{0.2647}}@{}}
\toprule\noalign{}
\begin{minipage}[b]{\linewidth}\raggedright
Method
\end{minipage} & \begin{minipage}[b]{\linewidth}\raggedright
Purpose
\end{minipage} & \begin{minipage}[b]{\linewidth}\raggedright
Syntax
\end{minipage} & \begin{minipage}[b]{\linewidth}\raggedright
Example
\end{minipage} \\
\midrule\noalign{}
\endhead
\bottomrule\noalign{}
\endlastfoot
\textbf{count()} & Count occurrences & str.count(substring) &
``hello''.count(``l'') = 2 \\
\textbf{strip()} & Remove whitespace & str.strip() & '' text ``.strip()
=''text'' \\
\textbf{replace()} & Replace substring & str.replace(old, new) &
``hi''.replace(``i'', ``ello'') = ``hello'' \\
\end{longtable}
}

\textbf{Return Values:}

\begin{itemize}
\tightlist
\item
  \textbf{count()}: Integer (number of occurrences)
\item
  \textbf{strip()}: New string (whitespace removed)
\item
  \textbf{replace()}: New string (replacements made)
\end{itemize}

\end{solutionbox}
\begin{mnemonicbox}
``Count Strip Replace''

\end{mnemonicbox}
\begin{center}\rule{0.5\linewidth}{0.5pt}\end{center}

\subsection*{Question 5(b) [4 marks]}\label{q5b}

\textbf{Explain how to traverse a string by giving example.}

\begin{solutionbox}
String traversal means accessing each character in a
string sequentially.

\textbf{Traversal Methods:}

{\def\LTcaptype{none} % do not increment counter
\begin{longtable}[]{@{}
  >{\raggedright\arraybackslash}p{(\linewidth - 4\tabcolsep) * \real{0.3077}}
  >{\raggedright\arraybackslash}p{(\linewidth - 4\tabcolsep) * \real{0.3077}}
  >{\raggedright\arraybackslash}p{(\linewidth - 4\tabcolsep) * \real{0.3846}}@{}}
\toprule\noalign{}
\begin{minipage}[b]{\linewidth}\raggedright
Method
\end{minipage} & \begin{minipage}[b]{\linewidth}\raggedright
Syntax
\end{minipage} & \begin{minipage}[b]{\linewidth}\raggedright
Use Case
\end{minipage} \\
\midrule\noalign{}
\endhead
\bottomrule\noalign{}
\endlastfoot
\textbf{Index-based} & for i in range(len(str)) & Need position \\
\textbf{Direct iteration} & for char in string & Just characters \\
\textbf{Enumerate} & for i, char in enumerate(str) & Both index and
character \\
\end{longtable}
}

\textbf{Example Code:}

\begin{verbatim}
text = "Python"

\# Method 1: Direct iteration
for char in text:
    print(char, end=" ")  \# P y t h o n

\# Method 2: Index{-based}
for i in range(len(text)):
    print(f"\{i\}: \{text[i]\}")

\# Method 3: Enumerate
for index, character in enumerate(text):
    print(f"Position \{index\}: \{character\}")
\end{verbatim}

\end{solutionbox}
\begin{mnemonicbox}
``Direct Index Enumerate''

\end{mnemonicbox}
\begin{center}\rule{0.5\linewidth}{0.5pt}\end{center}

\subsection*{Question 5(c) [7 marks]}\label{q5c}

\textbf{Develop programs to perform the following list operations:}

\begin{solutionbox}
Two programs for essential list operations and
analysis.

\textbf{Program 1: Check Element Existence}

\begin{verbatim}
def check\_element\_exists(lst, element):
    """Check if element exists in list"""
    if element in lst:
        return True, lst.index(element)
    else:
        return False, {-}1

\# Test program 1
numbers = [10, 25, 30, 45, 50]
search\_item = int(input("Enter element to search: "))
exists, position = check\_element\_exists(numbers, search\_item)

if exists:
    print(f"\{search\_item\} found at position \{position\}")
else:
    print(f"\{search\_item\} not found in list")
\end{verbatim}

\textbf{Program 2: Find Smallest and Largest}

\begin{verbatim}
def find\_min\_max(lst):
    """Find smallest and largest elements"""
    if not lst:  \# Empty list check
        return None, None
    
    smallest = min(lst)
    largest = max(lst)
    return smallest, largest

\# Test program 2
numbers = [15, 8, 23, 4, 16, 42]
min\_val, max\_val = find\_min\_max(numbers)
print(f"List: \{numbers\}")
print(f"Smallest: \{min\_val\}")
print(f"Largest: \{max\_val\}")
\end{verbatim}

\textbf{Key Operations:}

\begin{itemize}
\tightlist
\item
  \textbf{Membership}: Using `in' operator
\item
  \textbf{Min/Max}: Built-in functions
\item
  \textbf{Validation}: Empty list handling
\end{itemize}

\end{solutionbox}
\begin{mnemonicbox}
``Search Find Compare''

\end{mnemonicbox}
\begin{center}\rule{0.5\linewidth}{0.5pt}\end{center}

\subsection*{Question 5(a OR) [3
marks]}\label{question-5a-or-3-marks}

\textbf{Explain slicing of list with example.}

\begin{solutionbox}
List slicing extracts specific portions of a list using
index ranges.

\textbf{Slicing Syntax:}

{\def\LTcaptype{none} % do not increment counter
\begin{longtable}[]{@{}
  >{\raggedright\arraybackslash}p{(\linewidth - 4\tabcolsep) * \real{0.2667}}
  >{\raggedright\arraybackslash}p{(\linewidth - 4\tabcolsep) * \real{0.4333}}
  >{\raggedright\arraybackslash}p{(\linewidth - 4\tabcolsep) * \real{0.3000}}@{}}
\toprule\noalign{}
\begin{minipage}[b]{\linewidth}\raggedright
Format
\end{minipage} & \begin{minipage}[b]{\linewidth}\raggedright
Description
\end{minipage} & \begin{minipage}[b]{\linewidth}\raggedright
Example
\end{minipage} \\
\midrule\noalign{}
\endhead
\bottomrule\noalign{}
\endlastfoot
\textbf{list[start:end]} & Elements from start to end-1 &
[1,2,3,4][1:3] = [2,3] \\
\textbf{list[:end]} & From beginning to end-1 &
[1,2,3,4][:2] = [1,2] \\
\textbf{list[start:]} & From start to end & [1,2,3,4][2:] =
[3,4] \\
\textbf{list[::step]} & Every step element & [1,2,3,4][::2]
= [1,3] \\
\end{longtable}
}

\textbf{Example:}

\begin{verbatim}
numbers = [0, 1, 2, 3, 4, 5]
print(numbers[1:4])   \# [1, 2, 3]
print(numbers[:3])    \# [0, 1, 2]
print(numbers[3:])    \# [3, 4, 5]
print(numbers[::2])   \# [0, 2, 4]
\end{verbatim}

\end{solutionbox}
\begin{mnemonicbox}
``Start End Step''

\end{mnemonicbox}
\begin{center}\rule{0.5\linewidth}{0.5pt}\end{center}

\subsection*{Question 5(b OR) [4
marks]}\label{question-5b-or-4-marks}

\textbf{Explain how to traverse a list by giving example.}

\begin{solutionbox}
List traversal involves accessing each element in a
list systematically.

\textbf{Traversal Techniques:}

{\def\LTcaptype{none} % do not increment counter
\begin{longtable}[]{@{}lll@{}}
\toprule\noalign{}
Method & Syntax & Output Type \\
\midrule\noalign{}
\endhead
\bottomrule\noalign{}
\endlastfoot
\textbf{Value iteration} & for item in list & Elements only \\
\textbf{Index iteration} & for i in range(len(list)) & Index access \\
\textbf{Enumerate} & for i, item in enumerate(list) & Index and value \\
\end{longtable}
}

\textbf{Example Code:}

\begin{verbatim}
fruits = ["apple", "banana", "orange"]

\# Method 1: Direct value access
print("Values only:")
for fruit in fruits:
    print(fruit)

\# Method 2: Index{-based access}
print("{n}With indices:")
for i in range(len(fruits)):
    print(f"Index \{i\}: \{fruits[i]\}")

\# Method 3: Enumerate
print("{n}Using enumerate:")
for index, fruit in enumerate(fruits):
    print(f"\{index\} {- }\{fruit\}")
\end{verbatim}

\textbf{Use Cases:}

\begin{itemize}
\tightlist
\item
  \textbf{Value only}: Simple processing
\item
  \textbf{Index access}: Position-dependent operations
\item
  \textbf{Enumerate}: Both index and value needed
\end{itemize}

\end{solutionbox}
\begin{mnemonicbox}
``Value Index Both''

\end{mnemonicbox}
\begin{center}\rule{0.5\linewidth}{0.5pt}\end{center}

\subsection*{Question 5(c OR) [7
marks]}\label{question-5c-or-7-marks}

\textbf{Develop python code to create list of prime and non-prime
numbers in range 1 to 50.}

\begin{solutionbox}
A Python program to categorize numbers into prime and
non-prime lists.

\textbf{Python Code:}

\begin{verbatim}
def is\_prime(n):
    """Check if a number is prime"""
    if n {} 2:
        return False
    for i in range(2, int(n**0.5) + 1):
if n \%

i == 0:

            return False
    return True

def categorize\_numbers(start, end):
    """Create lists of prime and non{-prime numbers"""}
    prime\_numbers = []
    non\_prime\_numbers = []
    
    for num in range(start, end + 1):
        if is\_prime(num):
            prime\_numbers.append(num)
        else:
            non\_prime\_numbers.append(num)
    
    return prime\_numbers, non\_prime\_numbers

\# Generate lists for 1 to 50
primes, non\_primes = categorize\_numbers(1, 50)

print("Prime numbers (1{-50):"})
print(primes)
print(f"{n}Total prime numbers: \{len(primes)\}")

print("{n}Non{-prime numbers (1{-}50):"})
print(non\_primes)
print(f"{n}Total non{-prime numbers: }\{len(non\_primes)\}")
\end{verbatim}

\textbf{Prime Logic:}

{\def\LTcaptype{none} % do not increment counter
\begin{longtable}[]{@{}lll@{}}
\toprule\noalign{}
Number Type & Condition & Examples \\
\midrule\noalign{}
\endhead
\bottomrule\noalign{}
\endlastfoot
\textbf{Prime} & Only divisible by 1 and itself & 2, 3, 5, 7, 11 \\
\textbf{Non-Prime} & Has other divisors & 1, 4, 6, 8, 9 \\
\end{longtable}
}

\textbf{Algorithm Steps:}

\begin{itemize}
\tightlist
\item
  \textbf{Check divisibility} from 2 to \sqrtn
\item
  \textbf{Categorize} based on prime test
\item
  \textbf{Store} in appropriate lists
\end{itemize}

\end{solutionbox}
\begin{mnemonicbox}
``Check Divide Categorize Store''

\end{mnemonicbox}

\end{document}
