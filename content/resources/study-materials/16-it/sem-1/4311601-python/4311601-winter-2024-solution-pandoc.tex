\documentclass[10pt,a4paper]{article}

% content/resources/templates/preamble.tex
\usepackage[margin=0.6in]{geometry}
\author{Milav Dabgar}
\usepackage{amsmath,amssymb,amsthm}
\usepackage{booktabs}
\usepackage{multirow}
\usepackage{xcolor}
\usepackage{tcolorbox}
\tcbuselibrary{breakable,skins}
\usepackage[colorlinks=true,linkcolor=blue]{hyperref}
\usepackage{titlesec}
\usepackage{enumitem}
\usepackage{tikz}
\usepackage{pgfplots}
\usepackage{circuitikz}
\usepackage[version=4]{mhchem}
\usepackage{longtable}
\usepackage{array}
\usepackage{float}
\usepackage{caption}
\usepackage{listings}

\lstset{
  basicstyle=\small\ttfamily,
  breaklines=true,
  breakatwhitespace=false,
  postbreak=\mbox{\textcolor{red}{$\hookrightarrow$}\space},
  float=false,
  numbers=left,
  numberstyle=\tiny\color{gray},
  numbersep=10pt,
  xleftmargin=2em,
  keywordstyle=\color{blue},
  commentstyle=\color{green!60!black},
  stringstyle=\color{purple},
  backgroundcolor=\color{gray!5},
  showstringspaces=false,
  tabsize=2,
  captionpos=b,
  keepspaces=true,
  columns=flexible
}

\pgfplotsset{compat=1.18}
\usetikzlibrary{shapes,arrows,positioning,calc,patterns,decorations.pathmorphing,decorations.markings,arrows.meta}

% Color scheme
\definecolor{headcolor}{RGB}{0,102,204}
\definecolor{keycolor}{RGB}{220,20,60}
\definecolor{solutioncolor}{RGB}{34,139,34}
\definecolor{mnemoniccolor}{RGB}{148,0,211}
\definecolor{codecolor}{RGB}{0,0,100}

% Spacing
\setlength{\parskip}{3pt}
\setlist[itemize]{nosep}
\setlist[enumerate]{nosep}

% Title formatting
\titleformat{\section}{\Large\bfseries\color{headcolor}}{\thesection}{1em}{}
\titleformat{\subsection}{\large\bfseries\color{headcolor}}{\thesubsection}{1em}{}

% Pandoc tightlist compatibility
\providecommand{\tightlist}{%
  \setlength{\itemsep}{0pt}\setlength{\parskip}{0pt}}

% Pandoc longtable compatibility
\newcounter{none}
\def\thenone{}


% content/resources/templates/english-boxes.tex
% This file is currently empty - it exists to maintain consistency with the import structure.
% Add custom environments here if needed in the future.


\begin{document}

\begin{center}
{\Huge\bfseries\color{headcolor} Subject Name Solutions}\\[5pt]
{\LARGE 4311601 -- Winter 2024}\\[3pt]
{\large Semester 1 Study Material}\\[3pt]
{\normalsize\textit{Detailed Solutions and Explanations}}
\end{center}

\vspace{10pt}

\subsection*{Question 1(a) [3 marks]}\label{q1a}

\textbf{Define Problem Solving, Algorithm and Pseudo Code.}

\begin{solutionbox}

{\def\LTcaptype{none} % do not increment counter
\begin{longtable}[]{@{}
  >{\raggedright\arraybackslash}p{(\linewidth - 2\tabcolsep) * \real{0.3333}}
  >{\raggedright\arraybackslash}p{(\linewidth - 2\tabcolsep) * \real{0.6667}}@{}}
\toprule\noalign{}
\begin{minipage}[b]{\linewidth}\raggedright
Term
\end{minipage} & \begin{minipage}[b]{\linewidth}\raggedright
Definition
\end{minipage} \\
\midrule\noalign{}
\endhead
\bottomrule\noalign{}
\endlastfoot
\textbf{Problem Solving} & Systematic process of finding solutions to
complex issues using logical thinking \\
\textbf{Algorithm} & Step-by-step procedure to solve a problem with
finite operations \\
\textbf{Pseudo Code} & Informal description of program logic using plain
English-like syntax \\
\end{longtable}
}

\begin{itemize}
\tightlist
\item
  \textbf{Problem Solving}: Breaking down complex problems into
  manageable steps
\item
  \textbf{Algorithm}: Must be finite, definite, effective, and produce
  correct output
\item
  \textbf{Pseudo Code}: Bridge between human language and programming
  code
\end{itemize}

\end{solutionbox}
\begin{mnemonicbox}
``PAP - Problem, Algorithm, Pseudo''

\end{mnemonicbox}
\subsection*{Question 1(b) [4 marks]}\label{q1b}

\textbf{Explain various Flowchart Symbols. Design a Flowchart to find
maximum number out of two given numbers}

\begin{solutionbox}

{\def\LTcaptype{none} % do not increment counter
\begin{longtable}[]{@{}lll@{}}
\toprule\noalign{}
Symbol & Shape & Purpose \\
\midrule\noalign{}
\endhead
\bottomrule\noalign{}
\endlastfoot
\textbf{Oval} & ⬭ & Start/End \\
\textbf{Rectangle} & ▭ & Process/Action \\
\textbf{Diamond} & ◊ & Decision \\
\textbf{Parallelogram} & ▱ & Input/Output \\
\end{longtable}
}

\textbf{Flowchart for Maximum of Two Numbers:}

\begin{verbatim}
flowchart LR
    A([Start]) {-{-} B[/Input A, B/]}
    B {-{-} C\{A  B?\}}
    C {-{-}|Yes| D[Max = A]}
    C {-{-}|No| E[Max = B]}
    D {-{-} F[/Display Max/]}
    E {-{-} F}
    F {-{-} G([End])}
\end{verbatim}

\begin{itemize}
\tightlist
\item
  \textbf{Start/End}: Entry and exit points
\item
  \textbf{Input/Output}: Data flow operations
\item
  \textbf{Decision}: Conditional branching
\item
  \textbf{Process}: Computational steps
\end{itemize}

\end{solutionbox}
\begin{mnemonicbox}
``SIPO - Start, Input, Process, Output''

\end{mnemonicbox}
\subsection*{Question 1(c) [7 marks]}\label{q1c}

\textbf{List out various arithmetic operators of python. Write Python
Code that performs various arithmetic operations.}

\begin{solutionbox}

{\def\LTcaptype{none} % do not increment counter
\begin{longtable}[]{@{}llll@{}}
\toprule\noalign{}
Operator & Symbol & Example & Result \\
\midrule\noalign{}
\endhead
\bottomrule\noalign{}
\endlastfoot
\textbf{Addition} & + & 5 + 3 & 8 \\
\textbf{Subtraction} & - & 5 - 3 & 2 \\
\textbf{Multiplication} & * & 5 * 3 & 15 \\
\textbf{Division} & / & 5 / 3 & 1.667 \\
\textbf{Floor Division} & // & 5 // 3 & 1 \\
\textbf{Modulus} & \% & 5 \% 3 & 2 \\
\textbf{Exponentiation} & ** & 5 ** 3 & 125 \\
\end{longtable}
}

\textbf{Code:}

\begin{verbatim}
a = 10
b = 3
print(f"Addition: \{a + b\}")
print(f"Subtraction: \{a {-} b\}")
print(f"Multiplication: \{a * b\}")
print(f"Division: \{a / b\}")
print(f"Floor Division: \{a // b\}")
print(f"Modulus: \{a \% b\}")
print(f"Power: \{a ** b\}")
\end{verbatim}

\end{solutionbox}
\begin{mnemonicbox}
``Add-Sub-Mul-Div-Floor-Mod-Pow''

\end{mnemonicbox}
\subsection*{Question 1(c OR) [7
marks]}\label{question-1c-or-7-marks}

\textbf{List out various comparison operators of python. Write Python
Code which performs various comparison operations.}

\begin{solutionbox}

{\def\LTcaptype{none} % do not increment counter
\begin{longtable}[]{@{}llll@{}}
\toprule\noalign{}
Operator & Symbol & Purpose & Example \\
\midrule\noalign{}
\endhead
\bottomrule\noalign{}
\endlastfoot
\textbf{Equal} & == & Check equality & 5 == 3 \rightarrow False \\
\textbf{Not Equal} & != & Check inequality & 5 != 3 \rightarrow True \\
\textbf{Greater Than} & \textgreater{} & Check greater & 5
\textgreater{} 3 \rightarrow True \\
\textbf{Less Than} & \textless{} & Check smaller & 5 \textless{} 3 \rightarrow
False \\
\textbf{Greater Equal} & \textgreater= & Check greater/equal & 5
\textgreater= 3 \rightarrow True \\
\textbf{Less Equal} & \textless= & Check smaller/equal & 5 \textless= 3
\rightarrow False \\
\end{longtable}
}

\textbf{Code:}

\begin{verbatim}
x = 8
y = 5
print(f"Equal: \{x == y\}")
print(f"Not Equal: \{x != y\}")
print(f"Greater: \{x {} y\}")
print(f"Less: \{x {} y\}")
print(f"Greater Equal: \{x {=} y\}")
print(f"Less Equal: \{x {=} y\}")
\end{verbatim}

\end{solutionbox}
\begin{mnemonicbox}
``Equal-Not-Greater-Less-GreaterEqual-LessEqual''

\end{mnemonicbox}
\subsection*{Question 2(a) [3 marks]}\label{q2a}

\textbf{Write short note on membership operators.}

\begin{solutionbox}

{\def\LTcaptype{none} % do not increment counter
\begin{longtable}[]{@{}
  >{\raggedright\arraybackslash}p{(\linewidth - 4\tabcolsep) * \real{0.3571}}
  >{\raggedright\arraybackslash}p{(\linewidth - 4\tabcolsep) * \real{0.3214}}
  >{\raggedright\arraybackslash}p{(\linewidth - 4\tabcolsep) * \real{0.3214}}@{}}
\toprule\noalign{}
\begin{minipage}[b]{\linewidth}\raggedright
Operator
\end{minipage} & \begin{minipage}[b]{\linewidth}\raggedright
Purpose
\end{minipage} & \begin{minipage}[b]{\linewidth}\raggedright
Example
\end{minipage} \\
\midrule\noalign{}
\endhead
\bottomrule\noalign{}
\endlastfoot
\textbf{in} & Check if element exists & `a' in `apple' \rightarrow True \\
\textbf{not in} & Check if element doesn't exist & `z' not in `apple' \rightarrow
True \\
\end{longtable}
}

\begin{itemize}
\tightlist
\item
  \textbf{in operator}: Returns True if element found in sequence
\item
  \textbf{not in operator}: Returns True if element not found in
  sequence
\item
  \textbf{Usage}: Lists, strings, tuples, dictionaries
\end{itemize}

\end{solutionbox}
\begin{mnemonicbox}
``In-Not-In for membership testing''

\end{mnemonicbox}
\subsection*{Question 2(b) [4 marks]}\label{q2b}

\textbf{Define Python. Write down various applications of Python
Programming.}

\begin{solutionbox}

\textbf{Python Definition}: High-level, interpreted programming language
known for simplicity and readability.

{\def\LTcaptype{none} % do not increment counter
\begin{longtable}[]{@{}ll@{}}
\toprule\noalign{}
Application Area & Examples \\
\midrule\noalign{}
\endhead
\bottomrule\noalign{}
\endlastfoot
\textbf{Web Development} & Django, Flask frameworks \\
\textbf{Data Science} & NumPy, Pandas, Matplotlib \\
\textbf{AI/ML} & TensorFlow, Scikit-learn \\
\textbf{Desktop Apps} & Tkinter, PyQt \\
\textbf{Game Development} & Pygame library \\
\end{longtable}
}

\begin{itemize}
\tightlist
\item
  \textbf{Interpreted}: No compilation needed
\item
  \textbf{Cross-platform}: Runs on multiple OS
\item
  \textbf{Large libraries}: Extensive standard library
\end{itemize}

\end{solutionbox}
\begin{mnemonicbox}
``Web-Data-AI-Desktop-Games''

\end{mnemonicbox}
\subsection*{Question 2(c) [7 marks]}\label{q2c}

\textbf{Write python program which calculates electricity bill using
following details.}

\begin{solutionbox}

\textbf{Table of Rates:}

{\def\LTcaptype{none} % do not increment counter
\begin{longtable}[]{@{}ll@{}}
\toprule\noalign{}
Unit Range & Rate per Unit \\
\midrule\noalign{}
\endhead
\bottomrule\noalign{}
\endlastfoot
\leq 100 & Rs 5.00 \\
101-200 & Rs 7.50 \\
201-300 & Rs 10.00 \\
\geq 301 & Rs 15.00 \\
\end{longtable}
}

\textbf{Code:}

\begin{verbatim}
units = int(input("Enter consumed units: "))

if units {=} 100:
    bill = units * 5.00
elif units {=} 200:
    bill = units * 7.50
elif units {=} 300:
    bill = units * 10.00
else:
    bill = units * 15.00

print(f"Total Bill: Rs \{bill\}")
\end{verbatim}

\begin{itemize}
\tightlist
\item
  \textbf{Conditional logic}: if-elif-else structure
\item
  \textbf{Rate calculation}: Based on unit slabs
\item
  \textbf{User input}: Interactive billing system
\end{itemize}

\end{solutionbox}
\begin{mnemonicbox}
``Input-Check-Calculate-Display''

\end{mnemonicbox}
\subsection*{Question 2(a OR) [3
marks]}\label{question-2a-or-3-marks}

\textbf{Write short note on identity operators.}

\begin{solutionbox}

{\def\LTcaptype{none} % do not increment counter
\begin{longtable}[]{@{}lll@{}}
\toprule\noalign{}
Operator & Purpose & Example \\
\midrule\noalign{}
\endhead
\bottomrule\noalign{}
\endlastfoot
\textbf{is} & Check same object & a is b \\
\textbf{is not} & Check different object & a is not b \\
\end{longtable}
}

\begin{itemize}
\tightlist
\item
  \textbf{is operator}: Compares object identity, not values
\item
  \textbf{is not operator}: Checks if objects are different
\item
  \textbf{Memory comparison}: Checks same memory location
\end{itemize}

\end{solutionbox}
\begin{mnemonicbox}
``Is-IsNot for object identity''

\end{mnemonicbox}
\subsection*{Question 2(b OR) [4
marks]}\label{question-2b-or-4-marks}

\textbf{What is indentation in Python? Explain various features of
Python.}

\begin{solutionbox}

\textbf{Indentation}: Whitespace at line beginning to define code
blocks.

{\def\LTcaptype{none} % do not increment counter
\begin{longtable}[]{@{}ll@{}}
\toprule\noalign{}
Feature & Description \\
\midrule\noalign{}
\endhead
\bottomrule\noalign{}
\endlastfoot
\textbf{Simple Syntax} & Easy to read and write \\
\textbf{Interpreted} & No compilation step \\
\textbf{Object-Oriented} & Supports OOP concepts \\
\textbf{Cross-Platform} & Runs on multiple OS \\
\textbf{Large Library} & Extensive standard library \\
\end{longtable}
}

\begin{itemize}
\tightlist
\item
  \textbf{Indentation}: Replaces curly braces \{\}
\item
  \textbf{Consistent}: Usually 4 spaces per level
\item
  \textbf{Mandatory}: Creates code structure
\end{itemize}

\end{solutionbox}
\begin{mnemonicbox}
``Simple-Interpreted-Object-Cross-Large''

\end{mnemonicbox}
\subsection*{Question 2(c OR) [7
marks]}\label{question-2c-or-7-marks}

\textbf{Write a python program that calculates Student's class/grade
using following details.}

\begin{solutionbox}

\textbf{Grading Table:}

{\def\LTcaptype{none} % do not increment counter
\begin{longtable}[]{@{}ll@{}}
\toprule\noalign{}
Percentage & Grade \\
\midrule\noalign{}
\endhead
\bottomrule\noalign{}
\endlastfoot
\geq 70 & Distinction \\
60-69 & First Class \\
50-59 & Second Class \\
35-49 & Pass Class \\
\textless{} 35 & Fail \\
\end{longtable}
}

\textbf{Code:}

\begin{verbatim}
percentage = float(input("Enter percentage: "))

if percentage {=} 70:
    grade = "Distinction"
elif percentage {=} 60:
    grade = "First Class"
elif percentage {=} 50:
    grade = "Second Class"
elif percentage {=} 35:
    grade = "Pass Class"
else:
    grade = "Fail"

print(f"Grade: \{grade\}")
\end{verbatim}

\begin{itemize}
\tightlist
\item
  \textbf{Multiple conditions}: Nested if-elif structure
\item
  \textbf{Grade assignment}: Based on percentage ranges
\item
  \textbf{Float input}: Handles decimal percentages
\end{itemize}

\end{solutionbox}
\begin{mnemonicbox}
``Distinction-First-Second-Pass-Fail''

\end{mnemonicbox}
\subsection*{Question 3(a) [3 marks]}\label{q3a}

\textbf{What is Selection Control Statement? List it out.}

\begin{solutionbox}

{\def\LTcaptype{none} % do not increment counter
\begin{longtable}[]{@{}ll@{}}
\toprule\noalign{}
Statement Type & Purpose \\
\midrule\noalign{}
\endhead
\bottomrule\noalign{}
\endlastfoot
\textbf{if} & Single condition check \\
\textbf{if-else} & Two-way branching \\
\textbf{if-elif-else} & Multi-way branching \\
\textbf{nested if} & Conditions within conditions \\
\end{longtable}
}

\begin{itemize}
\tightlist
\item
  \textbf{Selection statements}: Control program flow based on
  conditions
\item
  \textbf{Boolean evaluation}: Uses True/False logic
\item
  \textbf{Branching}: Different paths of execution
\end{itemize}

\end{solutionbox}
\begin{mnemonicbox}
``If-IfElse-IfElif-Nested''

\end{mnemonicbox}
\subsection*{Question 3(b) [4 marks]}\label{q3b}

\textbf{Write short note on nested loops.}

\begin{solutionbox}

{\def\LTcaptype{none} % do not increment counter
\begin{longtable}[]{@{}ll@{}}
\toprule\noalign{}
Loop Type & Structure \\
\midrule\noalign{}
\endhead
\bottomrule\noalign{}
\endlastfoot
\textbf{Outer Loop} & Controls iterations \\
\textbf{Inner Loop} & Executes completely for each outer iteration \\
\textbf{Total Iterations} & Outer \times Inner \\
\end{longtable}
}

\begin{itemize}
\tightlist
\item
  \textbf{Nested structure}: Loop inside another loop
\item
  \textbf{Complete execution}: Inner loop finishes before outer
  continues
\item
  \textbf{Pattern creation}: Useful for 2D structures
\end{itemize}

\textbf{Code Example:}

\begin{verbatim}
for i in range(3):
    for j in range(2):
print(f"i=\{i\},

j=\{j\}")

\end{verbatim}

\end{solutionbox}
\begin{mnemonicbox}
``Outer-Inner-Complete-Pattern''

\end{mnemonicbox}
\subsection*{Question 3(c) [7 marks]}\label{q3c}

\textbf{Write a user-define function that displays all numbers, which
are divisible by 4 from 1 to 100.}

\begin{solutionbox}

\textbf{Code:}

\begin{verbatim}
def display\_divisible\_by\_4():
    print("Numbers divisible by 4 from 1 to 100:")
    for num in range(1, 101):
        if num \% 4 == 0:
            print(num, end=" ")
    print()

\# Function call
display\_divisible\_by\_4()
\end{verbatim}

\textbf{Alternative with return:}

\begin{verbatim}
def get\_divisible\_by\_4():
    return [num for num in range(1, 101) if num \% 4 == 0]

result = get\_divisible\_by\_4()
print(result)
\end{verbatim}

\begin{itemize}
\tightlist
\item
  \textbf{Function definition}: def keyword usage
\item
  \textbf{Range function}: 1 to 100 iteration
\item
  \textbf{Modulus check}: num \% 4 == 0 condition
\item
  \textbf{List comprehension}: Alternative approach
\end{itemize}

\end{solutionbox}
\begin{mnemonicbox}
``Define-Range-Check-Display''

\end{mnemonicbox}
\subsection*{Question 3(a OR) [3
marks]}\label{question-3a-or-3-marks}

\textbf{What is Repetition Control Statement? List it out.}

\begin{solutionbox}

{\def\LTcaptype{none} % do not increment counter
\begin{longtable}[]{@{}ll@{}}
\toprule\noalign{}
Statement Type & Purpose \\
\midrule\noalign{}
\endhead
\bottomrule\noalign{}
\endlastfoot
\textbf{for loop} & Known number of iterations \\
\textbf{while loop} & Condition-based repetition \\
\textbf{nested loop} & Loop within loop \\
\end{longtable}
}

\begin{itemize}
\tightlist
\item
  \textbf{Repetition statements}: Execute code blocks repeatedly
\item
  \textbf{Iteration control}: Different methods of looping
\item
  \textbf{Loop variables}: Track iteration progress
\end{itemize}

\end{solutionbox}
\begin{mnemonicbox}
``For-While-Nested''

\end{mnemonicbox}
\subsection*{Question 3(b OR) [4
marks]}\label{question-3b-or-4-marks}

\textbf{Differentiate break and continue statements.}

\begin{solutionbox}

{\def\LTcaptype{none} % do not increment counter
\begin{longtable}[]{@{}lll@{}}
\toprule\noalign{}
Aspect & break & continue \\
\midrule\noalign{}
\endhead
\bottomrule\noalign{}
\endlastfoot
\textbf{Purpose} & Exit loop completely & Skip current iteration \\
\textbf{Execution} & Jumps out of loop & Jumps to next iteration \\
\textbf{Usage} & Terminate loop early & Skip specific conditions \\
\textbf{Effect} & Loop ends & Loop continues \\
\end{longtable}
}

\textbf{Code Example:}

\begin{verbatim}
\# break example
for i in range(5):
if

i == 3:

        break
    print(i)  \# Output: 0, 1, 2

\# continue example  
for i in range(5):
if

i == 2:

        continue
    print(i)  \# Output: 0, 1, 3, 4
\end{verbatim}

\end{solutionbox}
\begin{mnemonicbox}
``Break-Exit, Continue-Skip''

\end{mnemonicbox}
\subsection*{Question 3(c OR) [7
marks]}\label{question-3c-or-7-marks}

\textbf{Write a user-define function which displays all even numbers
from 1 to 100.}

\begin{solutionbox}

\textbf{Code:}

\begin{verbatim}
def display\_even\_numbers():
    print("Even numbers from 1 to 100:")
    for num in range(2, 101, 2):
        print(num, end=" ")
    print()

\# Alternative method
def display\_even\_alt():
    even\_nums = []
    for num in range(1, 101):
        if num \% 2 == 0:
            even\_nums.append(num)
    print(even\_nums)

\# Function call
display\_even\_numbers()
\end{verbatim}

\begin{itemize}
\tightlist
\item
  \textbf{Efficient range}: range(2, 101, 2) for even numbers
\item
  \textbf{Modulus method}: Alternative checking with \% 2 == 0
\item
  \textbf{Function design}: Reusable code block
\end{itemize}

\end{solutionbox}
\begin{mnemonicbox}
``Range-Step-Even-Display''

\end{mnemonicbox}
\subsection*{Question 4(a) [3 marks]}\label{q4a}

\textbf{Define Function. List out various types of Functions available
in Python.}

\begin{solutionbox}

\textbf{Function}: Reusable block of code that performs specific task.

{\def\LTcaptype{none} % do not increment counter
\begin{longtable}[]{@{}ll@{}}
\toprule\noalign{}
Function Type & Description \\
\midrule\noalign{}
\endhead
\bottomrule\noalign{}
\endlastfoot
\textbf{Built-in} & Pre-defined functions (print, len) \\
\textbf{User-defined} & Created by programmer \\
\textbf{Lambda} & Anonymous single-line functions \\
\textbf{Recursive} & Functions calling themselves \\
\end{longtable}
}

\begin{itemize}
\tightlist
\item
  \textbf{Code reusability}: Write once, use many times
\item
  \textbf{Modularity}: Breaking complex problems into smaller parts
\item
  \textbf{Parameters}: Input values to functions
\end{itemize}

\end{solutionbox}
\begin{mnemonicbox}
``Built-User-Lambda-Recursive''

\end{mnemonicbox}
\subsection*{Question 4(b) [4 marks]}\label{q4b}

\textbf{Write short note on Scope of a variable.}

\begin{solutionbox}

{\def\LTcaptype{none} % do not increment counter
\begin{longtable}[]{@{}lll@{}}
\toprule\noalign{}
Scope Type & Description & Example \\
\midrule\noalign{}
\endhead
\bottomrule\noalign{}
\endlastfoot
\textbf{Local} & Inside function only & Function variables \\
\textbf{Global} & Throughout program & Module-level variables \\
\textbf{Built-in} & Python keywords & print, len, type \\
\end{longtable}
}

\textbf{Code Example:}

\begin{verbatim}
x = 10  \# Global variable

def my\_function():
    y = 20  \# Local variable
    print(x)  \# Access global
    print(y)  \# Access local

my\_function()
\# print(y)  \# Error: y not accessible
\end{verbatim}

\begin{itemize}
\tightlist
\item
  \textbf{Variable accessibility}: Where variables can be used
\item
  \textbf{LEGB rule}: Local, Enclosing, Global, Built-in
\end{itemize}

\end{solutionbox}
\begin{mnemonicbox}
``Local-Global-Builtin''

\end{mnemonicbox}
\subsection*{Question 4(c) [7 marks]}\label{q4c}

\textbf{Write Python code which asks user for Main string and Substring
and checks membership of a Substring in the Main String.}

\begin{solutionbox}

\textbf{Code:}

\begin{verbatim}
def check\_substring():
    main\_string = input("Enter main string: ")
    substring = input("Enter substring: ")
    
    if substring in main\_string:
        print(f"{}\{substring\}{ found in }\{main\_string\}{"})
        print(f"Position: \{main\_string.find(substring)\}")
    else:
        print(f"{}\{substring\}{ not found in }\{main\_string\}{"})

\# Enhanced version with case handling
def check\_substring\_enhanced():
    main\_string = input("Enter main string: ")
    substring = input("Enter substring: ")
    
    if substring.lower() in main\_string.lower():
        print("Substring found (case{-insensitive)"})
    else:
        print("Substring not found")

check\_substring()
\end{verbatim}

\begin{itemize}
\tightlist
\item
  \textbf{User interaction}: input() for string collection
\item
  \textbf{Membership testing}: `in' operator usage
\item
  \textbf{Case sensitivity}: Optional case handling
\end{itemize}

\end{solutionbox}
\begin{mnemonicbox}
``Input-Check-Report-Position''

\end{mnemonicbox}
\subsection*{Question 4(a OR) [3
marks]}\label{question-4a-or-3-marks}

\textbf{What is Local variable and Global variable?}

\begin{solutionbox}

{\def\LTcaptype{none} % do not increment counter
\begin{longtable}[]{@{}llll@{}}
\toprule\noalign{}
Variable Type & Scope & Lifetime & Access \\
\midrule\noalign{}
\endhead
\bottomrule\noalign{}
\endlastfoot
\textbf{Local} & Function only & Function execution & Limited \\
\textbf{Global} & Entire program & Program execution & Widespread \\
\end{longtable}
}

\textbf{Example:}

\begin{verbatim}
global\_var = 100  \# Global

def function():
    local\_var = 50  \# Local
    print(global\_var)  \# ✓ Accessible
    print(local\_var)   \# ✓ Accessible

print(global\_var)  \# ✓ Accessible
\# print(local\_var)  \# ✗ Error
\end{verbatim}

\begin{itemize}
\tightlist
\item
  \textbf{Local variables}: Created inside functions
\item
  \textbf{Global variables}: Created outside functions
\end{itemize}

\end{solutionbox}
\begin{mnemonicbox}
``Local-Limited, Global-Everywhere''

\end{mnemonicbox}
\subsection*{Question 4(b OR) [4
marks]}\label{question-4b-or-4-marks}

\textbf{Explain any four built-in functions of Python.}

\begin{solutionbox}

{\def\LTcaptype{none} % do not increment counter
\begin{longtable}[]{@{}lll@{}}
\toprule\noalign{}
Function & Purpose & Example \\
\midrule\noalign{}
\endhead
\bottomrule\noalign{}
\endlastfoot
\textbf{len()} & Returns length & len(``hello'') \rightarrow 5 \\
\textbf{type()} & Returns data type & type(10) \rightarrow \textless class
`int'\textgreater{} \\
\textbf{input()} & Gets user input & name = input(``Name:'') \\
\textbf{print()} & Displays output & print(``Hello'') \\
\end{longtable}
}

\textbf{Additional Examples:}

\begin{verbatim}
\# len() function
print(len([1, 2, 3, 4]))  \# Output: 4

\# type() function  
print(type(3.14))  \# Output: {class float}

\# input() function
age = input("Enter age: ")

\# print() function
print("Your age is:", age)
\end{verbatim}

\end{solutionbox}
\begin{mnemonicbox}
``Length-Type-Input-Print''

\end{mnemonicbox}
\subsection*{Question 4(c OR) [7
marks]}\label{question-4c-or-7-marks}

\textbf{Write Python code which locates a substring in a given string.}

\begin{solutionbox}

\textbf{Code:}

\begin{verbatim}
def locate\_substring():
    main\_string = input("Enter main string: ")
    substring = input("Enter substring to find: ")
    
    \# Method 1: Using find()
    position = main\_string.find(substring)
    if position != {-}1:
        print(f"Found at index: \{position\}")
    else:
        print("Substring not found")
    
    \# Method 2: Using index() with exception handling
    try:
        position = main\_string.index(substring)
        print(f"Located at index: \{position\}")
    except ValueError:
        print("Substring not found")
    
    \# Method 3: Find all occurrences
    positions = []
    start = 0
    while True:
        pos = main\_string.find(substring, start)
        if pos == {-}1:
            break
        positions.append(pos)
        start = pos + 1
    
    if positions:
        print(f"All positions: \{positions\}")

locate\_substring()
\end{verbatim}

\begin{itemize}
\tightlist
\item
  \textbf{find() method}: Returns index or -1
\item
  \textbf{index() method}: Returns index or raises exception
\item
  \textbf{Multiple occurrences}: Loop to find all positions
\end{itemize}

\end{solutionbox}
\begin{mnemonicbox}
``Find-Index-Exception-Multiple''

\end{mnemonicbox}
\subsection*{Question 5(a) [3 marks]}\label{q5a}

\textbf{Define String. List out various string operations.}

\begin{solutionbox}

\textbf{String}: Sequence of characters enclosed in quotes.

{\def\LTcaptype{none} % do not increment counter
\begin{longtable}[]{@{}lll@{}}
\toprule\noalign{}
Operation & Method & Example \\
\midrule\noalign{}
\endhead
\bottomrule\noalign{}
\endlastfoot
\textbf{Concatenation} & + & ``Hello'' + ``World'' \\
\textbf{Repetition} & * & ``Hi'' * 3 \\
\textbf{Slicing} & [start:end] & ``Hello''[1:4] \\
\textbf{Length} & len() & len(``Hello'') \\
\textbf{Case} & upper(), lower() & ``hello''.upper() \\
\end{longtable}
}

\begin{itemize}
\tightlist
\item
  \textbf{Immutable}: Strings cannot be changed after creation
\item
  \textbf{Indexing}: Access individual characters
\item
  \textbf{Methods}: Built-in functions for manipulation
\end{itemize}

\end{solutionbox}
\begin{mnemonicbox}
``Concat-Repeat-Slice-Length-Case''

\end{mnemonicbox}
\subsection*{Question 5(b) [4 marks]}\label{q5b}

\textbf{How can we identify whether an element is a member of a list or
not? Explain with a suitable example.}

\begin{solutionbox}

{\def\LTcaptype{none} % do not increment counter
\begin{longtable}[]{@{}lll@{}}
\toprule\noalign{}
Method & Operator & Returns \\
\midrule\noalign{}
\endhead
\bottomrule\noalign{}
\endlastfoot
\textbf{in} & element in list & True/False \\
\textbf{not in} & element not in list & True/False \\
\textbf{count()} & list.count(element) & Number of occurrences \\
\end{longtable}
}

\textbf{Example:}

\begin{verbatim}
fruits = ["apple", "banana", "orange", "mango"]

\# Using {in operator}
if "apple" in fruits:
    print("Apple is available")

\# Using {not in operator  }
if "grapes" not in fruits:
    print("Grapes not available")

\# Using count() method
count = fruits.count("apple")
if count {} 0:
    print(f"Apple found \{count\} times")
\end{verbatim}

\begin{itemize}
\tightlist
\item
  \textbf{Boolean result}: True if found, False otherwise
\item
  \textbf{Case sensitive}: ``Apple'' \neq ``apple''
\item
  \textbf{Efficiency}: `in' operator is most common
\end{itemize}

\end{solutionbox}
\begin{mnemonicbox}
``In-NotIn-Count for membership''

\end{mnemonicbox}
\subsection*{Question 5(c) [7 marks]}\label{q5c}

\textbf{Write Python code that replaces a substring with another
substring of a given string. Consider the given string as `Welcome to
GTU' and replace the substring `GTU' with `Gujarat Technological
University'.}

\begin{solutionbox}

\textbf{Code:}

\begin{verbatim}
def replace\_substring():
    \# Given string
    original = "Welcome to GTU"
    old\_substring = "GTU"
    new\_substring = "Gujarat Technological University"
    
    \# Method 1: Using replace()
    result1 = original.replace(old\_substring, new\_substring)
    print(f"Original: \{original\}")
    print(f"Modified: \{result1\}")
    
    \# Method 2: Manual replacement
    if old\_substring in original:
        index = original.find(old\_substring)
        result2 = original[:index] + new\_substring + original[index + len(old\_substring):]
        print(f"Manual method: \{result2\}")
    
    \# Method 3: Replace all occurrences
    test\_string = "GTU offers GTU degree from GTU"
    result3 = test\_string.replace("GTU", "Gujarat Technological University")
    print(f"Multiple replacements: \{result3\}")

replace\_substring()
\end{verbatim}

\textbf{Output:}

\begin{verbatim}
Original: Welcome to GTU
Modified: Welcome to Gujarat Technological University
\end{verbatim}

\begin{itemize}
\tightlist
\item
  \textbf{replace() method}: Built-in string function
\item
  \textbf{Slicing method}: Manual string manipulation
\item
  \textbf{All occurrences}: Replaces every instance
\end{itemize}

\end{solutionbox}
\begin{mnemonicbox}
``Find-Replace-Slice-All''

\end{mnemonicbox}
\subsection*{Question 5(a OR) [3
marks]}\label{question-5a-or-3-marks}

\textbf{Define List. List out various list operations.}

\begin{solutionbox}

\textbf{List}: Ordered collection of items that can be modified.

{\def\LTcaptype{none} % do not increment counter
\begin{longtable}[]{@{}lll@{}}
\toprule\noalign{}
Operation & Method & Example \\
\midrule\noalign{}
\endhead
\bottomrule\noalign{}
\endlastfoot
\textbf{Add} & append(), insert() & list.append(item) \\
\textbf{Remove} & remove(), pop() & list.remove(item) \\
\textbf{Access} & [index] & list[0] \\
\textbf{Slice} & [start:end] & list[1:3] \\
\textbf{Sort} & sort() & list.sort() \\
\end{longtable}
}

\begin{itemize}
\tightlist
\item
  \textbf{Mutable}: Lists can be changed after creation
\item
  \textbf{Indexed}: Elements accessed by position
\item
  \textbf{Dynamic}: Size can grow or shrink
\end{itemize}

\end{solutionbox}
\begin{mnemonicbox}
``Add-Remove-Access-Slice-Sort''

\end{mnemonicbox}
\subsection*{Question 5(b OR) [4
marks]}\label{question-5b-or-4-marks}

\textbf{Write short note on String Slicing. Explain with suitable
example.}

\begin{solutionbox}

\textbf{String Slicing}: Extracting parts of string using
[start:end:step].

{\def\LTcaptype{none} % do not increment counter
\begin{longtable}[]{@{}lll@{}}
\toprule\noalign{}
Syntax & Description & Example \\
\midrule\noalign{}
\endhead
\bottomrule\noalign{}
\endlastfoot
\textbf{[start:]} & From start to end & ``Hello''[1:] \rightarrow
``ello'' \\
\textbf{[:end]} & From beginning to end & ``Hello''[:3] \rightarrow
``Hel'' \\
\textbf{[start:end]} & Specific range & ``Hello''[1:4] \rightarrow
``ell'' \\
\textbf{[::-1]} & Reverse string & ``Hello''[::-1] \rightarrow
``olleH'' \\
\end{longtable}
}

\textbf{Example:}

\begin{verbatim}
text = "Python Programming"

print(text[0:6])    \# "Python"
print(text[7:])     \# "Programming"  
print(text[:6])     \# "Python"
print(text[::2])    \# "Pto rgamn"
print(text[::{-}1])   \# "gnimmargorP nohtyP"
\end{verbatim}

\begin{itemize}
\tightlist
\item
  \textbf{Negative indexing}: -1 for last character
\item
  \textbf{Step parameter}: Controls increment
\end{itemize}

\end{solutionbox}
\begin{mnemonicbox}
``Start-End-Step for slicing''

\end{mnemonicbox}
\subsection*{Question 5(c OR) [7
marks]}\label{question-5c-or-7-marks}

\textbf{Write Python code which counts the number of times the specified
element appears in the list.}

\begin{solutionbox}

\textbf{Code:}

\begin{verbatim}
def count\_element\_occurrences():
    \# Create a sample list
    numbers = [1, 2, 3, 2, 4, 2, 5, 2, 6]
    element = int(input("Enter element to count: "))
    
    \# Method 1: Using count() method
    count1 = numbers.count(element)
    print(f"Using count(): \{element\} appears \{count1\} times")
    
    \# Method 2: Manual counting
    count2 = 0
    for num in numbers:
        if num == element:
            count2 += 1
    print(f"Manual count: \{element\} appears \{count2\} times")
    
    \# Method 3: List comprehension
count3 = len([x for x in numbers if

x == element])

    print(f"List comprehension: \{element\} appears \{count3\} times")
    
    \# Method 4: For any type of list
    mixed\_list = [1, "hello", 3.14, "hello", True, "hello"]
    element\_str = input("Enter element to search in mixed list: ")
    count4 = mixed\_list.count(element\_str)
    print(f"In mixed list: {}\{element\_str\}{ appears }\{count4\} times")

count\_element\_occurrences()
\end{verbatim}

\begin{itemize}
\tightlist
\item
  \textbf{count() method}: Built-in list function
\item
  \textbf{Manual iteration}: Using loops for counting
\item
  \textbf{List comprehension}: Pythonic way of counting
\item
  \textbf{Type flexibility}: Works with any data type
\end{itemize}

\end{solutionbox}
\begin{mnemonicbox}
``Count-Manual-Comprehension-Flexible''

\end{mnemonicbox}

\end{document}
