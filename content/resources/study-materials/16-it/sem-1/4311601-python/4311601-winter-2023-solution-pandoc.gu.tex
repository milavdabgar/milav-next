\documentclass[10pt,a4paper]{article}

% content/resources/templates/preamble.tex
\usepackage[margin=0.6in]{geometry}
\author{Milav Dabgar}
\usepackage{amsmath,amssymb,amsthm}
\usepackage{booktabs}
\usepackage{multirow}
\usepackage{xcolor}
\usepackage{tcolorbox}
\tcbuselibrary{breakable,skins}
\usepackage[colorlinks=true,linkcolor=blue]{hyperref}
\usepackage{titlesec}
\usepackage{enumitem}
\usepackage{tikz}
\usepackage{pgfplots}
\usepackage{circuitikz}
\usepackage[version=4]{mhchem}
\usepackage{longtable}
\usepackage{array}
\usepackage{float}
\usepackage{caption}
\usepackage{listings}

\lstset{
  basicstyle=\small\ttfamily,
  breaklines=true,
  breakatwhitespace=false,
  postbreak=\mbox{\textcolor{red}{$\hookrightarrow$}\space},
  float=false,
  numbers=left,
  numberstyle=\tiny\color{gray},
  numbersep=10pt,
  xleftmargin=2em,
  keywordstyle=\color{blue},
  commentstyle=\color{green!60!black},
  stringstyle=\color{purple},
  backgroundcolor=\color{gray!5},
  showstringspaces=false,
  tabsize=2,
  captionpos=b,
  keepspaces=true,
  columns=flexible
}

\pgfplotsset{compat=1.18}
\usetikzlibrary{shapes,arrows,positioning,calc,patterns,decorations.pathmorphing,decorations.markings,arrows.meta}

% Color scheme
\definecolor{headcolor}{RGB}{0,102,204}
\definecolor{keycolor}{RGB}{220,20,60}
\definecolor{solutioncolor}{RGB}{34,139,34}
\definecolor{mnemoniccolor}{RGB}{148,0,211}
\definecolor{codecolor}{RGB}{0,0,100}

% Spacing
\setlength{\parskip}{3pt}
\setlist[itemize]{nosep}
\setlist[enumerate]{nosep}

% Title formatting
\titleformat{\section}{\Large\bfseries\color{headcolor}}{\thesection}{1em}{}
\titleformat{\subsection}{\large\bfseries\color{headcolor}}{\thesubsection}{1em}{}

% Pandoc tightlist compatibility
\providecommand{\tightlist}{%
  \setlength{\itemsep}{0pt}\setlength{\parskip}{0pt}}

% Pandoc longtable compatibility
\newcounter{none}
\def\thenone{}


% content/resources/templates/gujarati-boxes.tex
\usepackage{fontspec}
\usepackage{polyglossia}

% Set Gujarati as main language (document is primarily in Gujarati)
% Note: gloss-gujarati.ldf doesn't exist in polyglossia, but it will use hyphenation patterns
\setdefaultlanguage{gujarati}
\setotherlanguage{english}

% Configure Gujarati font properly
% Use Language=Default to prevent polyglossia from trying to add language-specific features
% that don't exist for Gujarati, which causes "empty feature" warnings
\newfontfamily\gujaratifont[Script=Gujarati,AutoFakeBold=2.5,AutoFakeSlant=0.3]{Noto Sans Gujarati}
\setmainfont[Script=Gujarati,AutoFakeBold=2.5,AutoFakeSlant=0.3]{Noto Sans Gujarati}
% Use Noto Sans Gujarati for monospace to support Gujarati in text
\setmonofont[Scale=0.9]{Noto Sans Gujarati}

% Configure English to use the same font
\newfontfamily\englishfont[Script=Gujarati,AutoFakeBold=2.5,AutoFakeSlant=0.3]{Noto Sans Gujarati}

% Translations for polyglossia
\gappto\captionsgujarati{
  \renewcommand{\tablename}{કોષ્ટક}
  \renewcommand{\figurename}{આકૃતિ}
}

% Helper for TikZ nodes to ensure Gujarati font
\newcommand{\gu}[1]{{\gujaratifont #1}}

% Custom environments
\newtcolorbox{solutionbox}{
    breakable,
    enhanced,
    colback=solutioncolor!5!white,
    colframe=solutioncolor!75!black,
    fonttitle=\bfseries,
    title=જવાબ
}

\newtcolorbox{solutionboxnobreak}{
 colback=solutioncolor!5!white,
 colframe=solutioncolor!75!black,
 fonttitle=\bfseries,
 title=જવાબ
}

\newtcolorbox{keyformula}{
 breakable,
 enhanced,
 colback=keycolor!5!white,
 colframe=keycolor!75!black,
 fonttitle=\bfseries,
 title=રાસાયણિક સમીકરણ/સૂત્ર
}

\newtcolorbox{mnemonicbox}{
 breakable,
 enhanced,
 colback=mnemoniccolor!5!white,
 colframe=mnemoniccolor!75!black,
 fonttitle=\bfseries,
 title=મેમરી ટ્રીક
}


\begin{document}

\begin{center}
{\Huge\bfseries\color{headcolor} Subject Name (Gujarati)}\\[5pt]
{\LARGE 4311601 -- Winter 2023}\\[3pt]
{\large Semester 1 Study Material}\\[3pt]
{\normalsize\textit{Detailed Solutions and Explanations}}
\end{center}

\vspace{10pt}

\subsection*{પ્રશ્ન 1(અ) [3
ગુણ]}\label{uxaaauxab0uxab6uxaa8-1uxa85-3-uxa97uxaa3}

\textbf{ફ્લો ચાર્ટ શું છે? ફ્લો ચાર્ટમાં વપરાતા પ્રતીકોની યાદી બનાવો.}

\begin{solutionbox}

\textbf{ફ્લો ચાર્ટ} એ અલ્ગોરિધમની ગ્રાફિકલ રજૂઆત છે જે પ્રક્રિયાના પગલાંઓ અને નિર્ણય
બિંદુઓ દર્શાવે છે.

\textbf{ફ્લો ચાર્ટ પ્રતીકોનું ટેબલ:}

{\def\LTcaptype{none} % do not increment counter
\begin{longtable}[]{@{}lll@{}}
\toprule\noalign{}
પ્રતીક & નામ & ઉપયોગ \\
\midrule\noalign{}
\endhead
\bottomrule\noalign{}
\endlastfoot
અંડાકાર & ટર્મિનલ & પ્રારંભ/અંત \\
લંબચોરસ & પ્રોસેસ & પ્રક્રિયા/ગણતરી \\
હીરો & નિર્ણય & શરતી નિવેદનો \\
સમાંતર ચતુષ્કોણ & ઇનપુટ/આઉટપુટ & ડેટા લેવો/આપવો \\
વૃત્ત & કનેક્ટર & ભાગોને જોડવા \\
તીર & ફ્લો લાઇન & દિશા \\
\end{longtable}
}

\textbf{મુખ્ય બિંદુઓ:}

\begin{itemize}
\tightlist
\item
  \textbf{વિઝ્યુઅલ રજૂઆત}: પ્રોગ્રામ લોજિક ગ્રાફિકલી દર્શાવે
\item
  \textbf{પગલાં દર પગલાં}: ક્રમિક ઓપરેશનનો ફ્લો
\item
  \textbf{નિર્ણય લેવો}: હીરા શરતી શાખાઓ દર્શાવે
\end{itemize}

\end{solutionbox}
\begin{mnemonicbox}
``ફ્લો ચાર્ટ્સ પ્રોગ્રામ સ્ટેપ્સ વિઝ્યુઅલી દર્શાવે''

\end{mnemonicbox}
\subsection*{પ્રશ્ન 1(બ) [4
ગુણ]}\label{uxaaauxab0uxab6uxaa8-1uxaac-4-uxa97uxaa3}

\textbf{for લૂપ માટે ટૂંકી નોંધ લખો.}

\begin{solutionbox}

\textbf{for લૂપ} Python માં સિક્વન્સ (list, tuple, string, range) પર iterate
કરવા માટે વપરાય છે.

\textbf{For લૂપ ટેબલ:}

{\def\LTcaptype{none} % do not increment counter
\begin{longtable}[]{@{}lll@{}}
\toprule\noalign{}
ઘટક & સિન્ટેક્સ & ઉદાહરણ \\
\midrule\noalign{}
\endhead
\bottomrule\noalign{}
\endlastfoot
મૂળભૂત & \texttt{for\ variable\ in\ sequence:} &
\texttt{for\ i\ in\ range(5):} \\
રેન્જ & \texttt{range(start,\ stop,\ step)} &
\texttt{range(1,\ 10,\ 2)} \\
યાદી & \texttt{for\ item\ in\ list:} &
\texttt{for\ x\ in\ [1,2,3]:} \\
સ્ટ્રિંગ & \texttt{for\ char\ in\ string:} &
\texttt{for\ c\ in\ "hello":} \\
\end{longtable}
}

\textbf{સરળ કોડ ઉદાહરણ:}

\begin{verbatim}
for i in range(3):
    print(i)
\# આઉટપુટ: 0, 1, 2
\end{verbatim}

\textbf{મુખ્ય લક્ષણો:}

\begin{itemize}
\tightlist
\item
  \textbf{ઓટોમેટિક iteration}: મેન્યુઅલ કાઉન્ટરની જરૂર નથી
\item
  \textbf{સિક્વન્સ ટ્રાવર્સલ}: કોઈપણ iterable ઓબ્જેક્ટ સાથે કામ કરે
\item
  \textbf{રેન્જ ફંક્શન}: નંબર સિક્વન્સ સરળતાથી બનાવે
\end{itemize}

\end{solutionbox}
\begin{mnemonicbox}
``For લૂપ્સ સિક્વન્સમાં iterate કરે''

\end{mnemonicbox}
\subsection*{પ્રશ્ન 1(ક) [7
ગુણ]}\label{uxaaauxab0uxab6uxaa8-1uxa95-7-uxa97uxaa3}

\textbf{ફિબોનાકી શ્રેણીને nમી ટર્મ સુધી દર્શાવવા માટે એક પ્રોગ્રામ લખો જ્યાં યુઝર
દ્વારા n આપવામાં આવે છે.}

\begin{solutionbox}

\textbf{ફિબોનાકી શ્રેણી પ્રોગ્રામ:}

\begin{verbatim}
\# યુઝર પાસેથી ટર્મ્સની સંખ્યા લો
n = int(input("ટર્મ્સની સંખ્યા દાખલ કરો: "))

\# પ્રથમ બે ટર્મ્સ initialize કરો
a, b = 0, 1

\# પ્રથમ ટર્મ દર્શાવો
if n {=} 1:
    print(a, end=" ")
    
\# બીજી ટર્મ દર્શાવો
if n {=} 2:
    print(b, end=" ")

\# બાકીની ટર્મ્સ જનરેટ કરો
for i in range(2, n):
    c = a + b
    print(c, end=" ")
    a, b = b, c
\end{verbatim}

\textbf{અલ્ગોરિધમ ફ્લો:}

\begin{verbatim}
flowchart LR
    A[શરૂ] {-{-} B[n ઇનપુટ]}
    B {-{-} C\{n = 1?\}}
    C {-{-}|હા| D[0 પ્રિન્ટ કરો]}
    C {-{-}|ના| H[અંત]}
    D {-{-} E\{n = 2?\}}
    E {-{-}|હા| F[1 પ્રિન્ટ કરો]}
    E {-{-}|ના| H}
    F {-{-} G[લૂપ i=2 થી n{-}1]}
    G {-{-} I[c = a + b]}
    I {-{-} J[c પ્રિન્ટ કરો]}
J {-{-} K[a = b,

b = c]}

    K {-{-} L\{i  n{-}1?\}}
    L {-{-}|હા| G}
    L {-{-}|ના| H[અંત]}
\end{verbatim}

\textbf{મુખ્ય કોન્સેપ્ટ્સ:}

\begin{itemize}
\tightlist
\item
  \textbf{સિક્વેન્શિયલ જનરેશન}: દરેક ટર્મ = પાછલી બે ટર્મનો સરવાળો
\item
  \textbf{વેરિયેબલ સ્વેપિંગ}: a, b વેલ્યુઝ અસરકારક રીતે અપડેટ કરો
\item
  \textbf{યુઝર ઇનપુટ}: ડાયનેમિક શ્રેણી લેન્થ
\end{itemize}

\end{solutionbox}
\begin{mnemonicbox}
``ફિબોનાકી: પાછલા બે નંબરો ઉમેરો''

\end{mnemonicbox}
\subsection*{પ્રશ્ન 1(ક OR) [7
ગુણ]}\label{uxaaauxab0uxab6uxaa8-1uxa95-or-7-uxa97uxaa3}

\textbf{1 થી 100 સુધીના ODD નંબરો પ્રિન્ટ કરવા માટે ફ્લો ચાર્ટ દોરો.}

\begin{solutionbox}

\textbf{1 થી 100 ODD નંબરો માટે ફ્લોચાર્ટ:}

\begin{verbatim}
flowchart LR
    A[શરૂ] {-{-} B[i = 1]}
    B {-{-} C\{i = 100?\}}
    C {-{-}|હા| D\{i \% 2 != 0?\}}
    D {-{-}|હા| E[i પ્રિન્ટ કરો]}
    D {-{-}|ના| F[i = i + 1]}
    E {-{-} F}
    F {-{-} C}
    C {-{-}|ના| G[અંત]}
\end{verbatim}

\textbf{અનુસંગિક Python કોડ:}

\begin{verbatim}
for i in range(1, 101):
    if i \% 2 != 0:
        print(i, end=" ")
\end{verbatim}

\textbf{વૈકલ્પિક પદ્ધતિ:}

\begin{verbatim}
for i in range(1, 101, 2):
    print(i, end=" ")
\end{verbatim}

\textbf{મુખ્ય તત્વો:}

\begin{itemize}
\tightlist
\item
  \textbf{લૂપ કંટ્રોલ}: i 1 થી 100 સુધી
\item
  \textbf{વિષમ ચેક}: i \% 2 != 0 શરત
\item
  \textbf{સ્ટેપ વધારો}: આગલા નંબર પર જાઓ
\end{itemize}

\end{solutionbox}
\begin{mnemonicbox}
``વિષમ નંબરો: 2 થી ભાગ્યે 1 બાકી''

\end{mnemonicbox}
\subsection*{પ્રશ્ન 2(અ) [3
ગુણ]}\label{uxaaauxab0uxab6uxaa8-2uxa85-3-uxa97uxaa3}

\textbf{નંબર પેલિન્ડ્રોમ છે કે નહીં તે શોધવા માટે પ્રોગ્રામ લખો.}

\begin{solutionbox}

\textbf{પેલિન્ડ્રોમ ચેક પ્રોગ્રામ:}

\begin{verbatim}
\# નંબર ઇનપુટ
num = int(input("નંબર દાખલ કરો: "))
temp = num
reverse = 0

\# નંબરને રિવર્સ કરો
while temp {} 0:
    reverse = reverse * 10 + temp \% 10
    temp = temp // 10

\# પેલિન્ડ્રોમ ચેક કરો
if num == reverse:
    print(f"\{num\} પેલિન્ડ્રોમ છે")
else:
    print(f"\{num\} પેલિન્ડ્રોમ નથી")
\end{verbatim}

\textbf{અલ્ગોરિધમ ટેબલ:}

{\def\LTcaptype{none} % do not increment counter
\begin{longtable}[]{@{}lll@{}}
\toprule\noalign{}
પગલું & ઓપરેશન & ઉદાહરણ (121) \\
\midrule\noalign{}
\endhead
\bottomrule\noalign{}
\endlastfoot
1 & છેલ્લો અંક મેળવો & 121 \% 10 = 1 \\
2 & રિવર્સ બનાવો & 0*10 + 1 = 1 \\
3 & છેલ્લો અંક દૂર કરો & 121 // 10 = 12 \\
4 & 0 સુધી પુનરાવર્તન & પ્રક્રિયા ચાલુ રાખો \\
\end{longtable}
}

\textbf{મુખ્ય બિંદુઓ:}

\begin{itemize}
\tightlist
\item
  \textbf{ડિજિટ એક્સ્ટ્રેક્શન}: મોડ્યુલો (\%) ઓપરેટર વાપરો
\item
  \textbf{રિવર્સ બિલ્ડિંગ}: 10 થી ગુણા કરી ડિજિટ ઉમેરો
\item
  \textbf{સરખામણી}: મૂળ બરાબર રિવર્સ
\end{itemize}

\end{solutionbox}
\begin{mnemonicbox}
``પેલિન્ડ્રોમ આગળ પાછળ સરખું વાંચાય''

\end{mnemonicbox}
\subsection*{પ્રશ્ન 2(બ) [4
ગુણ]}\label{uxaaauxab0uxab6uxaa8-2uxaac-4-uxa97uxaa3}

\textbf{Python પ્રોગ્રામિંગની વિશેષતાઓ સમજાવો.}

\begin{solutionbox}

\textbf{Python વિશેષતાઓનું ટેબલ:}

{\def\LTcaptype{none} % do not increment counter
\begin{longtable}[]{@{}lll@{}}
\toprule\noalign{}
વિશેષતા & વર્ણન & ફાયદો \\
\midrule\noalign{}
\endhead
\bottomrule\noalign{}
\endlastfoot
સરળ સિન્ટેક્સ & સાદો, વાંચી શકાય તેવો કોડ & ઝડપી ડેવલપમેન્ટ \\
ઇન્ટરપ્રિટેડ & કમ્પાઇલેશનની જરૂર નથી & ઝડપી ટેસ્ટિંગ \\
ઓબ્જેક્ટ-ઓરિએન્ટેડ & ક્લાસ અને ઓબ્જેક્ટ સપોર્ટ & કોડ રિયુઝેબિલિટી \\
ઓપન સોર્સ & વાપરવા માટે ફ્રી & લાઇસન્સિંગ કોસ્ટ નથી \\
ક્રોસ-પ્લેટફોર્મ & મલ્ટિપલ OS પર ચાલે & વ્યાપક કમ્પેટિબિલિટી \\
મોટી લાઇબ્રેરીઓ & વ્યાપક બિલ્ટ-ઇન મોડ્યુલ્સ & સમૃદ્ધ કાર્યક્ષમતા \\
\end{longtable}
}

\textbf{મુખ્ય ફાયદાઓ:}

\begin{itemize}
\tightlist
\item
  \textbf{શિખાઉ-મિત્ર}: શીખવામાં અને સમજવામાં સરળ
\item
  \textbf{વર્સેટાઇલ}: વેબ ડેવલપમેન્ટ, AI, ડેટા સાયન્સ
\item
  \textbf{કોમ્યુનિટી સપોર્ટ}: મોટો ડેવલપર કોમ્યુનિટી
\item
  \textbf{ડાયનેમિક ટાઇપિંગ}: વેરિયેબલ ટાઇપ ડિક્લેરેશનની જરૂર નથી
\end{itemize}

\end{solutionbox}
\begin{mnemonicbox}
``Python: સરળ, શક્તિશાળી, લોકપ્રિય પ્રોગ્રામિંગ''

\end{mnemonicbox}
\subsection*{પ્રશ્ન 2(ક) [7
ગુણ]}\label{uxaaauxab0uxab6uxaa8-2uxa95-7-uxa97uxaa3}

\textbf{Python પ્રોગ્રામની બેસિક સ્ટ્રક્ચર સમજાવો.}

\begin{solutionbox}

\textbf{Python પ્રોગ્રામ સ્ટ્રક્ચર:}

\begin{verbatim}
\#!/usr/bin/env python3
\# Shebang લાઇન (વૈકલ્પિક)

"""
ડોક્યુમેન્ટેશન સ્ટ્રિંગ (docstring)
પ્રોગ્રામનો હેતુ વર્ણવે છે
"""

\# Import સ્ટેટમેન્ટ્સ
import math
from datetime import date

\# ગ્લોબલ વેરિયેબલ્સ
PI = 3.14159
count = 0

\# ફંક્શન ડેફિનિશન્સ
def calculate\_area(radius):
    """વર્તુળનો ક્ષેત્રફળ કેલ્ક્યુલેટ કરે"""
    return PI * radius * radius

\# ક્લાસ ડેફિનિશન્સ
class Calculator:
    def \_\_init\_\_(self):
        self.result = 0

\# મેઇન પ્રોગ્રામ એક્ઝિક્યુશન
if \_\_name\_\_ == "\_\_main\_\_":
    \# પ્રોગ્રામ લોજિક અહીં
    radius = 5
    area = calculate\_area(radius)
    print(f"ક્ષેત્રફળ: \{area\}")
\end{verbatim}

\textbf{સ્ટ્રક્ચર કમ્પોનન્ટ્સ ટેબલ:}

{\def\LTcaptype{none} % do not increment counter
\begin{longtable}[]{@{}lll@{}}
\toprule\noalign{}
કમ્પોનન્ટ & હેતુ & ઉદાહરણ \\
\midrule\noalign{}
\endhead
\bottomrule\noalign{}
\endlastfoot
Shebang & સિસ્ટમ ઇન્ટરપ્રિટર & \texttt{\#!/usr/bin/env\ python3} \\
Docstring & પ્રોગ્રામ ડોક્યુમેન્ટેશન & \texttt{"""પ્રોગ્રામ\ વર્ણન"""} \\
Imports & બાહ્ય મોડ્યુલ્સ & \texttt{import\ math} \\
વેરિયેબલ્સ & ગ્લોબલ ડેટા સ્ટોરેજ & \texttt{PI\ =\ 3.14159} \\
ફંક્શન્સ & પુનઃવપરાશ કોડ બ્લોક્સ & \texttt{def\ function\_name():} \\
ક્લાસીસ & ઓબ્જેક્ટ ટેમ્પ્લેટ્સ & \texttt{class\ ClassName:} \\
મેઇન બ્લોક & પ્રોગ્રામ એક્ઝિક્યુશન &
\texttt{if\ \_\_name\_\_\ ==\ "\_\_main\_\_":} \\
\end{longtable}
}

\textbf{મુખ્ય સિદ્ધાંતો:}

\begin{itemize}
\tightlist
\item
  \textbf{ઇન્ડેન્ટેશન}: કોડ બ્લોક્સ વ્યાખ્યાયિત કરે (4 સ્પેસીસ આગ્રહણીય)
\item
  \textbf{કોમેન્ટ્સ}: સિંગલ લાઇન માટે \#, મલ્ટિ-લાઇન માટે ``\,``\,'' ``\,``\,''
\item
  \textbf{મોડ્યુલેરિટી}: ફંક્શન અને ક્લાસમાં કોડ ગોઠવો
\end{itemize}

\end{solutionbox}
\begin{mnemonicbox}
``સ્ટ્રક્ચર: ઇમ્પોર્ટ, ડિફાઇન, એક્ઝિક્યુટ''

\end{mnemonicbox}
\subsection*{પ્રશ્ન 2(અ OR) [3
ગુણ]}\label{uxaaauxab0uxab6uxaa8-2uxa85-or-3-uxa97uxaa3}

\textbf{સ્ટ્રિંગને રિવર્સ કરવા માટે પ્રોગ્રામ લખો.}

\begin{solutionbox}

\textbf{સ્ટ્રિંગ રિવર્સલ પ્રોગ્રામ:}

\begin{verbatim}
\# પદ્ધતિ 1: સ્લાઇસિંગ વાપરીને
string = input("સ્ટ્રિંગ દાખલ કરો: ")
reversed\_string = string[::{-}1]
print(f"રિવર્સ: \{reversed\_string\}")

\# પદ્ધતિ 2: લૂપ વાપરીને
string = input("સ્ટ્રિંગ દાખલ કરો: ")
reversed\_string = ""
for char in string:
    reversed\_string = char + reversed\_string
print(f"રિવર્સ: \{reversed\_string\}")
\end{verbatim}

\textbf{રિવર્સલ પદ્ધતિઓનું ટેબલ:}

{\def\LTcaptype{none} % do not increment counter
\begin{longtable}[]{@{}lll@{}}
\toprule\noalign{}
પદ્ધતિ & સિન્ટેક્સ & ઉદાહરણ \\
\midrule\noalign{}
\endhead
\bottomrule\noalign{}
\endlastfoot
સ્લાઇસિંગ & \texttt{string[::-1]} & ``hello'' \rightarrow ``olleh'' \\
લૂપ & કેરેક્ટર દ્વારા કેરેક્ટર બનાવો & દરેક char આગળ ઉમેરો \\
બિલ્ટ-ઇન & \texttt{"".join(reversed(string))} & રિવર્સ સિક્વન્સ જોડો \\
\end{longtable}
}

\textbf{મુખ્ય કોન્સેપ્ટ્સ:}

\begin{itemize}
\tightlist
\item
  \textbf{સ્લાઇસિંગ}: સૌથી અસરકારક પદ્ધતિ
\item
  \textbf{કન્કેટેનેશન}: કેરેક્ટર દ્વારા કેરેક્ટર સ્ટ્રિંગ બનાવો
\item
  \textbf{ઇન્ડેક્સિંગ}: સ્ટ્રિંગ પોઝિશન્સ એક્સેસ કરો
\end{itemize}

\end{solutionbox}
\begin{mnemonicbox}
``રિવર્સ: છેલ્લો કેરેક્ટર પહેલો''

\end{mnemonicbox}
\subsection*{પ્રશ્ન 2(બ OR) [4
ગુણ]}\label{uxaaauxab0uxab6uxaa8-2uxaac-or-4-uxa97uxaa3}

\textbf{લોજિકલ ઓપરેટર્સને ઉદાહરણ સાથે સમજાવો.}

\begin{solutionbox}

\textbf{Python લોજિકલ ઓપરેટર્સ:}

{\def\LTcaptype{none} % do not increment counter
\begin{longtable}[]{@{}lllll@{}}
\toprule\noalign{}
ઓપરેટર & સિમ્બોલ & વર્ણન & ઉદાહરણ & પરિણામ \\
\midrule\noalign{}
\endhead
\bottomrule\noalign{}
\endlastfoot
AND & \texttt{and} & બંને શરતો સાચી & \texttt{True\ and\ False} &
\texttt{False} \\
OR & \texttt{or} & ઓછામાં ઓછી એક શરત સાચી & \texttt{True\ or\ False} &
\texttt{True} \\
NOT & \texttt{not} & શરતની વિરુદ્ધ & \texttt{not\ True} &
\texttt{False} \\
\end{longtable}
}

\textbf{ઉદાહરણ કોડ:}

\begin{verbatim}
a = 10
b = 5

\# AND ઓપરેટર
if a {} 5 and b {} 10:
    print("બંને શરતો સાચી")

\# OR ઓપરેટર  
if a {} 15 or b {} 10:
    print("ઓછામાં ઓછી એક શરત સાચી")

\# NOT ઓપરેટર
if not (a {} 5):
    print("a 5 કરતાં નાનું નથી")
\end{verbatim}

\textbf{ટ્રુથ ટેબલ:}

{\def\LTcaptype{none} % do not increment counter
\begin{longtable}[]{@{}lllll@{}}
\toprule\noalign{}
A & B & A and B & A or B & not A \\
\midrule\noalign{}
\endhead
\bottomrule\noalign{}
\endlastfoot
T & T & T & T & F \\
T & F & F & T & F \\
F & T & F & T & T \\
F & F & F & F & T \\
\end{longtable}
}

\textbf{મુખ્ય ઉપયોગો:}

\begin{itemize}
\tightlist
\item
  \textbf{જટિલ શરતો}: બહુવિધ ચેક્સ કંબાઇન કરો
\item
  \textbf{નિર્ણય લેવો}: પ્રોગ્રામ ફ્લો કંટ્રોલ કરો
\item
  \textbf{બુલિયન લોજિક}: True/False ઓપરેશન્સ
\end{itemize}

\end{solutionbox}
\begin{mnemonicbox}
``AND બધાની જરૂર, OR એકની જરૂર, NOT ઉલટાવે''

\end{mnemonicbox}
\subsection*{પ્રશ્ન 2(ક OR) [7
ગુણ]}\label{uxaaauxab0uxab6uxaa8-2uxa95-or-7-uxa97uxaa3}

\textbf{Python માં વિવિધ ડેટા પ્રકારો સમજાવો}

\begin{solutionbox}

\textbf{Python ડેટા ટાઇપ્સ વર્ગીકરણ:}

\begin{center}
\textbf{Mermaid Diagram (Code)}
\begin{verbatim}
{Shaded}
{Highlighting}[]
graph TD
    A[Python ડેટા ટાઇપ્સ] {-{-}{} B[સંખ્યાત્મક]}
    A {-{-}{} C[સિક્વન્સ]}
    A {-{-}{} D[બુલિયન]}
    A {-{-}{} E[સેટ]}
    A {-{-}{} F[ડિક્શનરી]}
    B {-{-}{} G[int]}
    B {-{-}{} H[float]}
    B {-{-}{} I[complex]}
    C {-{-}{} J[str]}
    C {-{-}{} K[list]}
    C {-{-}{} L[tuple]}
{Highlighting}
{Shaded}
\end{verbatim}
\end{center}

\textbf{ડેટા ટાઇપ્સ ટેબલ:}

{\def\LTcaptype{none} % do not increment counter
\begin{longtable}[]{@{}llll@{}}
\toprule\noalign{}
ટાઇપ & ઉદાહરણ & વર્ણન & Mutable \\
\midrule\noalign{}
\endhead
\bottomrule\noalign{}
\endlastfoot
int & \texttt{42} & પૂર્ણ સંખ્યાઓ & ના \\
float & \texttt{3.14} & દશાંશ સંખ્યાઓ & ના \\
str & \texttt{"hello"} & ટેક્સ્ટ ડેટા & ના \\
list & \texttt{[1,2,3]} & ક્રમાંકિત સંગ્રહ & હા \\
tuple & \texttt{(1,2,3)} & ક્રમાંકિત અપરિવર્તનીય & ના \\
dict & \texttt{\{"a":1\}} & કી-વેલ્યુ જોડીઓ & હા \\
bool & \texttt{True/False} & બુલિયન વેલ્યુઝ & ના \\
set & \texttt{\{1,2,3\}} & યુનિક તત્વો & હા \\
\end{longtable}
}

\textbf{ઉદાહરણ કોડ:}

\begin{verbatim}
\# સંખ્યાત્મક ટાઇપ્સ
age = 25          \# int
price = 99.99     \# float
complex\_num = 3+4j \# complex

\# સિક્વન્સ ટાઇપ્સ
name = "Python"         \# string
numbers = [1,2,3,4]     \# list
coordinates = (10,20)   \# tuple

\# અન્ય ટાઇપ્સ
is\_active = True        \# boolean
unique\_items = \{1,2,3\  }\# set
student = \{"name":"John", "age":20\  }\# dict
\end{verbatim}

\textbf{મુખ્ય લક્ષણો:}

\begin{itemize}
\tightlist
\item
  \textbf{ડાયનેમિક ટાઇપિંગ}: વેરિયેબલ ટાઇપ ડિક્લેર કરવાની જરૂર નથી
\item
  \textbf{ટાઇપ કન્વર્ઝન}: સુસંગત ટાઇપ્સ વચ્ચે કન્વર્ટ કરો
\item
  \textbf{બિલ્ટ-ઇન ફંક્શન્સ}: ચેકિંગ માટે \texttt{type()},
  \texttt{isinstance()}
\end{itemize}

\end{solutionbox}
\begin{mnemonicbox}
``Python ટાઇપ્સ: નંબર્સ, સિક્વન્સીસ, કલેક્શન્સ''

\end{mnemonicbox}
\subsection*{પ્રશ્ન 3(અ) [3
ગુણ]}\label{uxaaauxab0uxab6uxaa8-3uxa85-3-uxa97uxaa3}

\textbf{Python માં ફ્લો કંટ્રોલ શું છે? ઉદાહરણ સાથે સમજાવો}

\begin{solutionbox}

\textbf{ફ્લો કંટ્રોલ} શરતી અને લૂપ સ્ટ્રક્ચર્સ વાપરીને પ્રોગ્રામ સ્ટેટમેન્ટ્સનો એક્ઝિક્યુશન
ઓર્ડર મેનેજ કરે છે.

\textbf{ફ્લો કંટ્રોલ પ્રકારોનું ટેબલ:}

{\def\LTcaptype{none} % do not increment counter
\begin{longtable}[]{@{}llll@{}}
\toprule\noalign{}
પ્રકાર & સ્ટેટમેન્ટ & હેતુ & ઉદાહરણ \\
\midrule\noalign{}
\endhead
\bottomrule\noalign{}
\endlastfoot
સિક્વેન્શિયલ & સામાન્ય એક્ઝિક્યુશન & લાઇન બાય લાઇન & \texttt{print("Hello")} \\
સિલેક્શન & if, elif, else & નિર્ણય લેવો &
\texttt{if\ x\ \textgreater{}\ 0:} \\
Iteration & for, while & પુનરાવર્તન & \texttt{for\ i\ in\ range(5):} \\
Jump & break, continue & લૂપ કંટ્રોલ & \texttt{break} \\
\end{longtable}
}

\textbf{ઉદાહરણ કોડ:}

\begin{verbatim}
\# સિલેક્શન ઉદાહરણ
age = 18
if age {=} 18:
    print("પુખ્ત")
else:
    print("બાલક")

\# Iteration ઉદાહરણ
for i in range(3):
    print(f"ગણતરી: \{i\}")
\end{verbatim}

\textbf{મુખ્ય કોન્સેપ્ટ્સ:}

\begin{itemize}
\tightlist
\item
  \textbf{શરતી એક્ઝિક્યુશન}: શરતોના આધારે કોડ ચાલે
\item
  \textbf{લૂપ સ્ટ્રક્ચર્સ}: કોડ બ્લોક્સ પુનરાવર્તન
\item
  \textbf{પ્રોગ્રામ ફ્લો}: એક્ઝિક્યુશન પાથ કંટ્રોલ
\end{itemize}

\end{solutionbox}
\begin{mnemonicbox}
``ફ્લો કંટ્રોલ: નિર્ણય, પુનરાવર્તન, Jump''

\end{mnemonicbox}
\subsection*{પ્રશ્ન 3(બ) [4
ગુણ]}\label{uxaaauxab0uxab6uxaa8-3uxaac-4-uxa97uxaa3}

\textbf{નેસ્ટેડ if સ્ટેટમેન્ટ સમજાવવા માટે પ્રોગ્રામ લખો.}

\begin{solutionbox}

\textbf{નેસ્ટેડ If સ્ટેટમેન્ટ પ્રોગ્રામ:}

\begin{verbatim}
\# નેસ્ટેડ if વાપરીને ગ્રેડ કેલ્ક્યુલેશન
marks = int(input("માર્ક્સ દાખલ કરો: "))

if marks {=} 0 and marks {=} 100:
    if marks {=} 90:
        grade = "A+"
    elif marks {=} 80:
        if marks {=} 85:
            grade = "A"
        else:
            grade = "B+"
    elif marks {=} 70:
        grade = "B"
    elif marks {=} 60:
        grade = "C"
    else:
        grade = "F"
    print(f"ગ્રેડ: \{grade\}")
else:
    print("અયોગ્ય માર્ક્સ")
\end{verbatim}

\textbf{નેસ્ટેડ સ્ટ્રક્ચર ડાયાગ્રામ:}

\begin{verbatim}
                    marks input
                         |
                   ┌─────▼───────┐
                   │0{=marks=100│}
                   └─────┬───────┘
              True       │       False
           ┌─────────────┼─────────────┐
           │             │             │
      ┌────▼────┐        │        ┌────▼────┐
      │marks{=90│        │        │ Invalid │}
      └────┬────┘        │        └─────────┘
           │             │
         A+│           ┌─▼─┐
           │           │...│
           │           └───┘
\end{verbatim}

\textbf{મુખ્ય લક્ષણો:}

\begin{itemize}
\tightlist
\item
  \textbf{બહુવિધ સ્તરો}: if સ્ટેટમેન્ટ્સ અંદર if સ્ટેટમેન્ટ્સ
\item
  \textbf{જટિલ શરતો}: બહુવિધ માપદંડો હેન્ડલ કરો
\item
  \textbf{લોજિકલ સ્ટ્રક્ચર}: નિર્ણય વૃક્ષો ગોઠવો
\end{itemize}

\end{solutionbox}
\begin{mnemonicbox}
``નેસ્ટેડ If: નિર્ણયોની અંદર નિર્ણયો''

\end{mnemonicbox}
\subsection*{પ્રશ્ન 3(ક) [7
ગુણ]}\label{uxaaauxab0uxab6uxaa8-3uxa95-7-uxa97uxaa3}

\textbf{Arguments અને Parameters ના પ્રકારો સમજાવવા માટે એક પ્રોગ્રામ લખો.}

\begin{solutionbox}

\textbf{Arguments અને Parameters ના પ્રકારો:}

\begin{verbatim}
\# 1. પોઝિશનલ Arguments
def greet(name, age):
    print(f"હેલો \{name\}, તમારી ઉંમર \{age\} વર્ષ છે")

greet("જોન", 25)  \# પોઝિશનલ arguments

\# 2. કીવર્ડ Arguments  
greet(age=30, name="એલિસ")  \# કીવર્ડ arguments

\# 3. ડિફૉલ્ટ Parameters
def introduce(name, city="અજાણ"):
    print(f"\{name\} \{city\} માં રહે છે")

introduce("બોબ")  \# ડિફૉલ્ટ વેલ્યુ વાપરે
introduce("કેરોલ", "મુંબઈ")  \# ડિફૉલ્ટ ઓવરરાઇડ

\# 4. વેરિયેબલ{-લેન્થ Arguments (*args)}
def sum\_all(*numbers):
    return sum(numbers)

result = sum\_all(1, 2, 3, 4, 5)
print(f"સરવાળો: \{result\}")

\# 5. કીવર્ડ વેરિયેબલ Arguments (**kwargs)
def display\_info(**info):
    for key, value in info.items():
        print(f"\{key\}: \{value\}")

display\_info(name="ડેવિડ", age=28, city="બોસ્ટન")
\end{verbatim}

\textbf{Parameters પ્રકારોનું ટેબલ:}

{\def\LTcaptype{none} % do not increment counter
\begin{longtable}[]{@{}llll@{}}
\toprule\noalign{}
પ્રકાર & સિન્ટેક્સ & ઉદાહરણ & વર્ણન \\
\midrule\noalign{}
\endhead
\bottomrule\noalign{}
\endlastfoot
પોઝિશનલ & \texttt{def\ func(a,\ b):} & \texttt{func(1,\ 2)} & ક્રમ
મહત્વનો \\
કીવર્ડ & \texttt{def\ func(a,\ b):} & \texttt{func(b=2,\

a=1)} & નામ

સ્પેસિફાઇડ \\
ડિફૉલ્ટ & \texttt{def\ func(a,\ b=10):} & \texttt{func(5)} & ડિફૉલ્ટ
વેલ્યુ \\
*args & \texttt{def\ func(*args):} & \texttt{func(1,2,3)} & વેરિયેબલ
પોઝિશનલ \\
**kwargs & \texttt{def\ func(**kwargs):} & \texttt{func(a=1,\

b=2)} &

વેરિયેબલ કીવર્ડ \\
\end{longtable}
}

\textbf{મુખ્ય કોન્સેપ્ટ્સ:}

\begin{itemize}
\tightlist
\item
  \textbf{લવચીકતા}: ડેટા પાસ કરવાની વિવિધ રીતો
\item
  \textbf{ક્રમ મહત્વ}: પોઝિશનલ vs કીવર્ડ
\item
  \textbf{વેરિયેબલ arguments}: અજાણી સંખ્યાના ઇનપુટ્સ હેન્ડલ કરો
\end{itemize}

\end{solutionbox}
\begin{mnemonicbox}
``Parameters: પોઝિશન, કીવર્ડ્સ, ડિફૉલ્ટ્સ, વેરિયેબલ્સ''

\end{mnemonicbox}
\subsection*{પ્રશ્ન 3(અ OR) [3
ગુણ]}\label{uxaaauxab0uxab6uxaa8-3uxa85-or-3-uxa97uxaa3}

\textbf{break અને continue statement ને ઉદાહરણ સાથે સમજાવો.}

\begin{solutionbox}

\textbf{Break અને Continue સ્ટેટમેન્ટ્સ:}

\textbf{Break સ્ટેટમેન્ટ:}

\begin{verbatim}
\# Break ઉદાહરણ {- લૂપમાંથી બહાર નીકળો}
for i in range(10):
if

i == 5:

        break
    print(i)
\# આઉટપુટ: 0, 1, 2, 3, 4
\end{verbatim}

\textbf{Continue સ્ટેટમેન્ટ:}

\begin{verbatim}
\# Continue ઉદાહરણ {- iteration છોડો}
for i in range(5):
if

i == 2:

        continue
    print(i)
\# આઉટપુટ: 0, 1, 3, 4
\end{verbatim}

\textbf{સરખામણી ટેબલ:}

{\def\LTcaptype{none} % do not increment counter
\begin{longtable}[]{@{}
  >{\raggedright\arraybackslash}p{(\linewidth - 6\tabcolsep) * \real{0.2683}}
  >{\raggedright\arraybackslash}p{(\linewidth - 6\tabcolsep) * \real{0.2195}}
  >{\raggedright\arraybackslash}p{(\linewidth - 6\tabcolsep) * \real{0.1951}}
  >{\raggedright\arraybackslash}p{(\linewidth - 6\tabcolsep) * \real{0.3171}}@{}}
\toprule\noalign{}
\begin{minipage}[b]{\linewidth}\raggedright
સ્ટેટમેન્ટ
\end{minipage} & \begin{minipage}[b]{\linewidth}\raggedright
હેતુ
\end{minipage} & \begin{minipage}[b]{\linewidth}\raggedright
ક્રિયા
\end{minipage} & \begin{minipage}[b]{\linewidth}\raggedright
ઉદાહરણ ઉપયોગ
\end{minipage} \\
\midrule\noalign{}
\endhead
\bottomrule\noalign{}
\endlastfoot
break & લૂપમાંથી બહાર નીકળો & સંપૂર્ણ લૂપ સમાપ્ત કરે & શરત પર બહાર નીકળો \\
continue & iteration છોડો & આગલા iteration પર જાઓ & સ્પેસિફિક વેલ્યુઝ છોડો \\
\end{longtable}
}

\textbf{મુખ્ય તફાવતો:}

\begin{itemize}
\tightlist
\item
  \textbf{Break}: લૂપમાંથી સંપૂર્ણે બહાર નીકળે
\item
  \textbf{Continue}: માત્ર વર્તમાન iteration છોડે
\item
  \textbf{ફ્લો કંટ્રોલ}: લૂપ એક્ઝિક્યુશન મેનેજ કરે
\end{itemize}

\end{solutionbox}
\begin{mnemonicbox}
``Break બહાર નીકળે, Continue છોડે''

\end{mnemonicbox}
\subsection*{પ્રશ્ન 3(બ OR) [4
ગુણ]}\label{uxaaauxab0uxab6uxaa8-3uxaac-or-4-uxa97uxaa3}

\textbf{નીચેની પેટર્ન દર્શાવવા માટે એક પ્રોગ્રામ બનાવો}

\begin{verbatim}
1
12
123
1234
12345
\end{verbatim}

\begin{solutionbox}

\textbf{નંબર પેટર્ન પ્રોગ્રામ:}

\begin{verbatim}
\# પદ્ધતિ 1: નેસ્ટેડ લૂપ્સ વાપરીને
rows = 5
for i in range(1, rows + 1):
    for j in range(1, i + 1):
        print(j, end="")
    print()  \# નવી લાઇન

\# પદ્ધતિ 2: સ્ટ્રિંગ મેનિપ્યુલેશન વાપરીને
for i in range(1, 6):
    line = ""
    for j in range(1, i + 1):
        line += str(j)
    print(line)

\# પદ્ધતિ 3: join વાપરીને
for i in range(1, 6):
    numbers = [str(j) for j in range(1, i + 1)]
    print("".join(numbers))
\end{verbatim}

\textbf{પેટર્ન લોજિક ટેબલ:}

{\def\LTcaptype{none} % do not increment counter
\begin{longtable}[]{@{}llll@{}}
\toprule\noalign{}
પંક્તિ & નંબર્સ & રેન્જ & આઉટપુટ \\
\midrule\noalign{}
\endhead
\bottomrule\noalign{}
\endlastfoot
1 & 1 & 1 થી 1 & 1 \\
2 & 1,2 & 1 થી 2 & 12 \\
3 & 1,2,3 & 1 થી 3 & 123 \\
4 & 1,2,3,4 & 1 થી 4 & 1234 \\
5 & 1,2,3,4,5 & 1 થી 5 & 12345 \\
\end{longtable}
}

\textbf{મુખ્ય કોન્સેપ્ટ્સ:}

\begin{itemize}
\tightlist
\item
  \textbf{નેસ્ટેડ લૂપ્સ}: બાહ્ય પંક્તિઓ માટે, અંદરૂની નંબર્સ માટે
\item
  \textbf{રેન્જ ફંક્શન}: નંબર સિક્વન્સ જનરેટ કરે
\item
  \textbf{પ્રિન્ટ કંટ્રોલ}: નવી લાઇનો ટાળવા માટે end=``\,'' વાપરો
\end{itemize}

\end{solutionbox}
\begin{mnemonicbox}
``પેટર્ન: પંક્તિ નંબર કોલમ કાઉન્ટ નક્કી કરે''

\end{mnemonicbox}
\subsection*{પ્રશ્ન 3(ક OR) [7
ગુણ]}\label{uxaaauxab0uxab6uxaa8-3uxa95-or-7-uxa97uxaa3}

\textbf{દરેક માટે કોડ લખીને નીચેના ગાણિતિક કાર્યો સમજાવો: 1. abs() 2. max()
3. pow() 4. sum()}

\begin{solutionbox}

\textbf{Python માં ગાણિતિક ફંક્શન્સ:}

\begin{verbatim}
\# 1. abs() {- એબ્સોલ્યુટ વેલ્યુ}
numbers = [{-}5, 3.7, {-}10.2, 0]
print("abs() ફંક્શન ઉદાહરણો:")
for num in numbers:
    print(f"abs(\{num\}) = \{abs(num)\}")

\# 2. max() {- મહત્તમ વેલ્યુ}
list1 = [4, 7, 2, 9, 1]
print(f"{n}max() ફંક્શન ઉદાહરણો:")
print(f"max(\{list1\}) = \{max(list1)\}")
print(f"max(10, 25, 5) = \{max(10, 25, 5)\}")
print(f"max({hello) = }\{max({hello})\}")  \# વર્ણમાળા પ્રમાણે

\# 3. pow() {- પાવર ફંક્શન}
print(f"{n}pow() ફંક્શન ઉદાહરણો:")
print(f"pow(2, 3) = \{pow(2, 3)\}")      \# 2\^{3 = 8}
print(f"pow(5, 2) = \{pow(5, 2)\}")      \# 5\^{2 = 25}
print(f"pow(8, 1/3) = \{pow(8, 1/3)\}")  \# 8 નો ઘન મૂળ

\# 4. sum() {- સરવાળો ફંક્શન}
numbers = [1, 2, 3, 4, 5]
print(f"{n}sum() ફંક્શન ઉદાહરણો:")
print(f"sum(\{numbers\}) = \{sum(numbers)\}")
print(f"sum(\{numbers\}, 10) = \{sum(numbers, 10)\}")  \# શરૂઆતી વેલ્યુ સાથે
\end{verbatim}

\textbf{ફંક્શન્સ સારાંશ ટેબલ:}

{\def\LTcaptype{none} % do not increment counter
\begin{longtable}[]{@{}lllll@{}}
\toprule\noalign{}
ફંક્શન & સિન્ટેક્સ & હેતુ & ઉદાહરણ & પરિણામ \\
\midrule\noalign{}
\endhead
\bottomrule\noalign{}
\endlastfoot
abs() & \texttt{abs(x)} & એબ્સોલ્યુટ વેલ્યુ & \texttt{abs(-5)} & 5 \\
max() & \texttt{max(iterable)} & મહત્તમ વેલ્યુ & \texttt{max([1,5,3])} &
5 \\
pow() & \texttt{pow(x,\ y)} & x ને y ની પાવર & \texttt{pow(2,\ 3)} & 8 \\
sum() & \texttt{sum(iterable)} & વેલ્યુઝનો સરવાળો &
\texttt{sum([1,2,3])} & 6 \\
\end{longtable}
}

\textbf{મુખ્ય ઉપયોગો:}

\begin{itemize}
\tightlist
\item
  \textbf{abs()}: અંતર ગણતરી, એરર હેન્ડલિંગ
\item
  \textbf{max()}: મહત્તમ શોધવું, સ્પર્ધાના પરિણામો
\item
  \textbf{pow()}: વૈજ્ઞાનિક ગણતરી, ચક્રવૃદ્ધિ વ્યાજ
\item
  \textbf{sum()}: કુલ ગણતરી, આંકડાશાસ્ત્ર
\end{itemize}

\end{solutionbox}
\begin{mnemonicbox}
``Math ફંક્શન્સ: એબ્સોલ્યુટ, મહત્તમ, પાવર, સરવાળો''

\end{mnemonicbox}
\subsection*{પ્રશ્ન 4(અ) [3
ગુણ]}\label{uxaaauxab0uxab6uxaa8-4uxa85-3-uxa97uxaa3}

\textbf{Variables નો scope સમજાવો.}

\begin{solutionbox}

\textbf{વેરિયેબલ સ્કોપ} એ પ્રોગ્રામમાં તે પ્રદેશનો સંદર્ભ આપે છે જ્યાં વેરિયેબલ એક્સેસ કરી
શકાય.

\textbf{સ્કોપ પ્રકારોનું ટેબલ:}

{\def\LTcaptype{none} % do not increment counter
\begin{longtable}[]{@{}llll@{}}
\toprule\noalign{}
સ્કોપ & વર્ણન & જીવનકાળ & એક્સેસ \\
\midrule\noalign{}
\endhead
\bottomrule\noalign{}
\endlastfoot
લોકલ & ફંક્શનની અંદર & ફંક્શન એક્ઝિક્યુશન & માત્ર ફંક્શન \\
ગ્લોબલ & ફંક્શનોની બહાર & પ્રોગ્રામ એક્ઝિક્યુશન & આખો પ્રોગ્રામ \\
બિલ્ટ-ઇન & Python કીવર્ડ્સ & Python સેશન & બધે \\
\end{longtable}
}

\textbf{ઉદાહરણ કોડ:}

\begin{verbatim}
x = 10  \# ગ્લોબલ વેરિયેબલ

def my\_function():
    y = 20  \# લોકલ વેરિયેબલ
    print(f"લોકલ y: \{y\}")
    print(f"ગ્લોબલ x: \{x\}")

my\_function()
print(f"ગ્લોબલ x: \{x\}")
\# print(y)  \# એરર: y અહીં એક્સેસિબલ નથી
\end{verbatim}

\textbf{મુખ્ય નિયમો:}

\begin{itemize}
\tightlist
\item
  \textbf{લોકલ વેરિયેબલ્સ}: ફંક્શનોની અંદર બનાવાય
\item
  \textbf{ગ્લોબલ વેરિયેબલ્સ}: સમગ્ર પ્રોગ્રામમાં એક્સેસિબલ
\item
  \textbf{LEGB નિયમ}: Local \rightarrow Enclosing \rightarrow Global \rightarrow Built-in
\end{itemize}

\end{solutionbox}
\begin{mnemonicbox}
``સ્કોપ: લોકલ ફંક્શનમાં રહે, ગ્લોબલ બધે રહે''

\end{mnemonicbox}
\subsection*{પ્રશ્ન 4(બ) [4
ગુણ]}\label{uxaaauxab0uxab6uxaa8-4uxaac-4-uxa97uxaa3}

\textbf{નેસ્ટેડ LOOP અને નંબર્સ ડિસ્પ્લે કરવા માટે પ્રોગ્રામ ડેવલપ કરો.}

\begin{solutionbox}

\textbf{નેસ્ટેડ લૂપ પ્રોગ્રામ:}

\begin{verbatim}
\# ઉદાહરણ 1: નંબર ગ્રિડ
print("નંબર ગ્રિડ પેટર્ન:")
for i in range(1, 4):
    for j in range(1, 5):
        print(f"\{i\\{}j\}", end=" ")
    print()  \# દરેક પંક્તિ પછી નવી લાઇન

\# ઉદાહરણ 2: ગુણાકાર ટેબલ
print("{n}ગુણાકાર ટેબલ:")
for i in range(1, 4):
    for j in range(1, 6):
        result = i * j
        print(f"\{result:3\}", end=" ")
    print()

\# ઉદાહરણ 3: નંબર પિરામિડ
print("{n}નંબર પિરામિડ:")
for i in range(1, 5):
    for j in range(1, i + 1):
        print(j, end=" ")
    print()
\end{verbatim}

\textbf{નેસ્ટેડ લૂપ સ્ટ્રક્ચર:}

\begin{verbatim}
    બાહ્ય લૂપ (i)
         │
    ┌────▼────┐
    │  i = 1  │
    └────┬────┘
         │
    અંદરૂની લૂપ (j)
    ┌────▼────┐
    │j=1,2,3,4│
    └────┬────┘
         │
    ┌────▼────┐
    │  i = 2  │
    └─────────┘
\end{verbatim}

\textbf{મુખ્ય કોન્સેપ્ટ્સ:}

\begin{itemize}
\tightlist
\item
  \textbf{બાહ્ય લૂપ}: પંક્તિઓ/મુખ્ય iterations કંટ્રોલ કરે
\item
  \textbf{અંદરૂની લૂપ}: કોલમ્સ/નાના iterations કંટ્રોલ કરે
\item
  \textbf{એક્ઝિક્યુશન ફ્લો}: અંદરૂનું પૂર્ણ થાય પછી બાહ્ય વધે
\end{itemize}

\end{solutionbox}
\begin{mnemonicbox}
``નેસ્ટેડ લૂપ્સ: બાહ્ય અંદરૂનીને કંટ્રોલ કરે''

\end{mnemonicbox}
\subsection*{પ્રશ્ન 4(ક) [7
ગુણ]}\label{uxaaauxab0uxab6uxaa8-4uxa95-7-uxa97uxaa3}

\textbf{1 થી 50 ની રેન્જમાં ODD અને EVEN નંબરોની LIST બનાવવા માટે પ્રોગ્રામ
લખો.}

\begin{solutionbox}

\textbf{ODD અને EVEN નંબર્સ પ્રોગ્રામ:}

\begin{verbatim}
\# પદ્ધતિ 1: લૂપ્સ અને શરતો વાપરીને
odd\_numbers = []
even\_numbers = []

for i in range(1, 51):
    if i \% 2 == 0:
        even\_numbers.append(i)
    else:
        odd\_numbers.append(i)

print("વિષમ નંબર્સ (1{-50):"})
print(odd\_numbers)
print(f"ગણતરી: \{len(odd\_numbers)\}")

print("{n}સમ નંબર્સ (1{-50):"})
print(even\_numbers)
print(f"ગણતરી: \{len(even\_numbers)\}")

\# પદ્ધતિ 2: લિસ્ટ કમ્પ્રીહેન્શન વાપરીને
odd\_list = [i for i in range(1, 51) if i \% 2 != 0]
even\_list = [i for i in range(1, 51) if i \% 2 == 0]

print(f"{n}વિષમ (લિસ્ટ કમ્પ્રીહેન્શન): \{odd\_list[:10]\}...")  \# પ્રથમ 10
print(f"સમ (લિસ્ટ કમ્પ્રીહેન્શન): \{even\_list[:10]\}...")  \# પ્રથમ 10

\# પદ્ધતિ 3: સ્ટેપ સાથે રેન્જ વાપરીને
odd\_range = list(range(1, 51, 2))   \# શરૂ 1, સ્ટેપ 2
even\_range = list(range(2, 51, 2))  \# શરૂ 2, સ્ટેપ 2

print(f"{n}વિષમ (રેન્જ પદ્ધતિ): \{odd\_range[:10]\}...")
print(f"સમ (રેન્જ પદ્ધતિ): \{even\_range[:10]\}...")
\end{verbatim}

\textbf{નંબર વર્ગીકરણ ટેબલ:}

{\def\LTcaptype{none} % do not increment counter
\begin{longtable}[]{@{}llll@{}}
\toprule\noalign{}
પ્રકાર & શરત & રેન્જ 1-10 & કાઉન્ટ (1-50) \\
\midrule\noalign{}
\endhead
\bottomrule\noalign{}
\endlastfoot
વિષમ & \texttt{n\ \%\ 2\ !=\ 0} & 1,3,5,7,9 & 25 \\
સમ & \texttt{n\ \%\ 2\ ==\ 0} & 2,4,6,8,10 & 25 \\
\end{longtable}
}

\textbf{મુખ્ય તકનીકો:}

\begin{itemize}
\tightlist
\item
  \textbf{મોડ્યુલો ઓપરેટર}: બાકીની ચેક માટે \texttt{\%}
\item
  \textbf{લિસ્ટ કમ્પ્રીહેન્શન}: સંક્ષિપ્ત લિસ્ટ સર્જન
\item
  \textbf{રેન્જ ફંક્શન}: સિક્વન્સ અસરકારક રીતે જનરેટ કરે
\end{itemize}

\end{solutionbox}
\begin{mnemonicbox}
``વિષમ/સમ: 2 થી ભાગ્યે બાકી 1/0''

\end{mnemonicbox}
\subsection*{પ્રશ્ન 4(અ OR) [3
ગુણ]}\label{uxaaauxab0uxab6uxaa8-4uxa85-or-3-uxa97uxaa3}

\textbf{સ્ટ્રિંગ સ્લાઇસિંગને ઉદાહરણ સાથે સમજાવો.}

\begin{solutionbox}

\textbf{સ્ટ્રિંગ સ્લાઇસિંગ} \texttt{[start:stop:step]} સિન્ટેક્સ વાપરીને
સ્ટ્રિંગના ભાગો એક્સ્ટ્રેક્ટ કરે છે.

\textbf{સ્લાઇસિંગ સિન્ટેક્સ ટેબલ:}

{\def\LTcaptype{none} % do not increment counter
\begin{longtable}[]{@{}llll@{}}
\toprule\noalign{}
સિન્ટેક્સ & વર્ણન & ઉદાહરણ & પરિણામ \\
\midrule\noalign{}
\endhead
\bottomrule\noalign{}
\endlastfoot
\texttt{s[start:stop]} & start થી stop-1 સુધી &
\texttt{"hello"[1:4]} & ``ell'' \\
\texttt{s[start:]} & start થી અંત સુધી & \texttt{"hello"[2:]} &
``llo'' \\
\texttt{s[:stop]} & શરૂઆતથી stop-1 સુધી & \texttt{"hello"[:3]} &
``hel'' \\
\texttt{s[::step]} & દરેક step કેરેક્ટર & \texttt{"hello"[::2]} &
``hlo'' \\
\texttt{s[::-1]} & સ્ટ્રિંગ રિવર્સ & \texttt{"hello"[::-1]} &
``olleh'' \\
\end{longtable}
}

\textbf{ઉદાહરણ કોડ:}

\begin{verbatim}
text = "Python Programming"

\# મૂળભૂત સ્લાઇસિંગ
print(f"પ્રથમ 6 અક્ષરો: \{text[:6]\}")      \# "Python"
print(f"છેલ્લા 11 અક્ષરો: \{text[7:]\}")      \# "Programming"
print(f"મધ્ય ભાગ: \{text[2:8]\}")       \# "thon P"

\# સ્ટેપ સ્લાઇસિંગ
print(f"દર 2જો અક્ષર: \{text[::2]\}")    \# "Pto rgamn"

\# નેગેટિવ ઇન્ડેક્સિંગ
print(f"છેલ્લો અક્ષર: \{text[{-}1]\}")     \# "g"
print(f"રિવર્સ: \{text[::{-}1]\}")          \# "gnimmargorP nohtyP"
\end{verbatim}

\textbf{મુખ્ય લક્ષણો:}

\begin{itemize}
\tightlist
\item
  \textbf{ઝીરો-બેસ્ડ ઇન્ડેક્સિંગ}: 0 થી શરૂ
\item
  \textbf{નેગેટિવ ઇન્ડેક્સિંગ}: અંતથી ગણતરી (-1)
\item
  \textbf{અપરિવર્તનીય}: મૂળ સ્ટ્રિંગ અપરિવર્તિત
\end{itemize}

\end{solutionbox}
\begin{mnemonicbox}
``સ્લાઇસ: શરૂ, બંધ, સ્ટેપ''

\end{mnemonicbox}
\subsection*{પ્રશ્ન 4(બ OR) [4
ગુણ]}\label{uxaaauxab0uxab6uxaa8-4uxaac-or-4-uxa97uxaa3}

\textbf{આપેલ સંખ્યાના ફેક્ટોરિયલ શોધવા માટે user defined function નો ઉપયોગ
કરીને પ્રોગ્રામ લખો.}

\begin{solutionbox}

\textbf{ફેક્ટોરિયલ ફંક્શન પ્રોગ્રામ:}

\begin{verbatim}
def factorial(n):
    """રિકર્શન વાપરીને ફેક્ટોરિયલ કેલ્ક્યુલેટ કરે"""
if

n == 0 or

n == 1:

        return 1
    else:
        return n * factorial(n {-} 1)

def factorial\_iterative(n):
    """લૂપ વાપરીને ફેક્ટોરિયલ કેલ્ક્યુલેટ કરે"""
    result = 1
    for i in range(1, n + 1):
        result *= i
    return result

\# મુખ્ય પ્રોગ્રામ
number = int(input("સંખ્યા દાખલ કરો: "))
if number {} 0:
    print("નેગેટિવ સંખ્યાઓ માટે ફેક્ટોરિયલ વ્યાખ્યાયિત નથી")
else:
    result1 = factorial(number)
    result2 = factorial\_iterative(number)
    print(f"\{number\} નું ફેક્ટોરિયલ = \{result1\}")
\end{verbatim}

\textbf{ફેક્ટોરિયલ ટેબલ:}

{\def\LTcaptype{none} % do not increment counter
\begin{longtable}[]{@{}lll@{}}
\toprule\noalign{}
n & ફેક્ટોરિયલ & ગણતરી \\
\midrule\noalign{}
\endhead
\bottomrule\noalign{}
\endlastfoot
0 & 1 & બેઝ કેસ \\
1 & 1 & બેઝ કેસ \\
3 & 6 & 3 \times 2 \times 1 \\
5 & 120 & 5 \times 4 \times 3 \times 2 \times 1 \\
\end{longtable}
}

\textbf{મુખ્ય કોન્સેપ્ટ્સ:}

\begin{itemize}
\tightlist
\item
  \textbf{રિકર્શન}: ફંક્શન પોતાને કોલ કરે
\item
  \textbf{બેઝ કેસ}: રિકર્સિવ કોલ્સ બંધ કરે
\item
  \textbf{યુઝર-ડિફાઇન્ડ}: કસ્ટમ ફંક્શન સર્જન
\end{itemize}

\end{solutionbox}
\begin{mnemonicbox}
``ફેક્ટોરિયલ: નીચેના બધા નંબર્સ ગુણા કરો''

\end{mnemonicbox}
\subsection*{પ્રશ્ન 4(ક OR) [7
ગુણ]}\label{uxaaauxab0uxab6uxaa8-4uxa95-or-7-uxa97uxaa3}

\textbf{આપેલ સ્ટ્રિંગમાં સબ સ્ટ્રિંગ હાજર છે કે કેમ તે તપાસવા માટે user defined
function લખો.}

\begin{solutionbox}

\textbf{સબસ્ટ્રિંગ ચેક ફંક્શન:}

\begin{verbatim}
def find\_substring(main\_string, sub\_string):
    """મુખ્ય સ્ટ્રિંગમાં સબસ્ટ્રિંગ અસ્તિત્વ ચેક કરે"""
    if sub\_string in main\_string:
        index = main\_string.find(sub\_string)
        return True, index
    else:
        return False, {-}1

def count\_substring(main\_string, sub\_string):
    """સબસ્ટ્રિંગની ઘટનાઓ ગણે"""
    return main\_string.count(sub\_string)

def find\_all\_positions(main\_string, sub\_string):
    """સબસ્ટ્રિંગની બધી પોઝિશન્સ શોધે"""
    positions = []
    start = 0
    while True:
        pos = main\_string.find(sub\_string, start)
        if pos == {-}1:
            break
        positions.append(pos)
        start = pos + 1
    return positions

\# મુખ્ય પ્રોગ્રામ
text = input("મુખ્ય સ્ટ્રિંગ દાખલ કરો: ")
search = input("શોધવા માટે સબસ્ટ્રિંગ દાખલ કરો: ")

found, position = find\_substring(text, search)
if found:
    print(f"સબસ્ટ્રિંગ {}\{search\}{ પોઝિશન }\{position\} પર મળ્યું")
    count = count\_substring(text, search)
    all\_pos = find\_all\_positions(text, search)
    print(f"કુલ ઘટનાઓ: \{count\}")
    print(f"બધી પોઝિશન્સ: \{all\_pos\}")
else:
    print(f"સબસ્ટ્રિંગ {}\{search\}{ મળ્યું નથી"})
\end{verbatim}

\textbf{સ્ટ્રિંગ મેથડ્સ ટેબલ:}

{\def\LTcaptype{none} % do not increment counter
\begin{longtable}[]{@{}llll@{}}
\toprule\noalign{}
મેથડ & હેતુ & ઉદાહરણ & પરિણામ \\
\midrule\noalign{}
\endhead
\bottomrule\noalign{}
\endlastfoot
\texttt{find()} & પ્રથમ પોઝિશન શોધે & \texttt{"hello".find("ll")} & 2 \\
\texttt{count()} & ઘટનાઓ ગણે & \texttt{"hello".count("l")} & 2 \\
\texttt{in} & અસ્તિત્વ ચેક કરે & \texttt{"ll"\ in\ "hello"} & True \\
\texttt{index()} & પોઝિશન શોધે (ન મળે તો એરર) &
\texttt{"hello".index("e")} & 1 \\
\end{longtable}
}

\textbf{મુખ્ય લક્ષણો:}

\begin{itemize}
\tightlist
\item
  \textbf{બહુવિધ પદ્ધતિઓ}: શોધવાની વિવિધ રીતો
\item
  \textbf{પોઝિશન ટ્રેકિંગ}: મળેલ સબસ્ટ્રિંગનો ઇન્ડેક્સ પરત કરે
\item
  \textbf{એરર હેન્ડલિંગ}: પ્રોસેસિંગ પહેલાં ચેક કરે
\end{itemize}

\end{solutionbox}
\begin{mnemonicbox}
``સબસ્ટ્રિંગ: શોધ, મેળવ, ગણ, પોઝિશન''

\end{mnemonicbox}
\subsection*{પ્રશ્ન 5(અ) [3
ગુણ]}\label{uxaaauxab0uxab6uxaa8-5uxa85-3-uxa97uxaa3}

\textbf{ઉદાહરણ સાથે List કેવી રીતે બનાવવી અને એક્સેસ કરવી તે સમજાવો.}

\begin{solutionbox}

\textbf{લિસ્ટ સર્જન અને એક્સેસ:}

\begin{verbatim}
\# લિસ્ટ બનાવવી
empty\_list = []
numbers = [1, 2, 3, 4, 5]
mixed = [1, "hello", 3.14, True]
nested = [[1, 2], [3, 4], [5, 6]]

\# તત્વો એક્સેસ કરવા
print(f"પ્રથમ તત્વ: \{numbers[0]\}")      \# 1
print(f"છેલ્લું તત્વ: \{numbers[{-}1]\}")      \# 5
print(f"સ્લાઇસ: \{numbers[1:4]\}")           \# [2, 3, 4]
\end{verbatim}

\textbf{લિસ્ટ એક્સેસ પદ્ધતિઓ:}

{\def\LTcaptype{none} % do not increment counter
\begin{longtable}[]{@{}llll@{}}
\toprule\noalign{}
પદ્ધતિ & સિન્ટેક્સ & ઉદાહરણ & પરિણામ \\
\midrule\noalign{}
\endhead
\bottomrule\noalign{}
\endlastfoot
ઇન્ડેક્સ & \texttt{list[i]} & \texttt{[1,2,3][1]} & 2 \\
નેગેટિવ & \texttt{list[-i]} & \texttt{[1,2,3][-1]} & 3 \\
સ્લાઇસ & \texttt{list[start:stop]} & \texttt{[1,2,3,4][1:3]}
& [2,3] \\
\end{longtable}
}

\textbf{મુખ્ય લક્ષણો:}

\begin{itemize}
\tightlist
\item
  \textbf{ક્રમાંકિત સંગ્રહ}: તત્વોની પોઝિશન્સ છે
\item
  \textbf{પરિવર્તનશીલ}: સર્જન પછી સુધારી શકાય
\item
  \textbf{મિશ્ર પ્રકાર}: વિવિધ ડેટા ટાઇપ્સની મંજૂરી
\end{itemize}

\end{solutionbox}
\begin{mnemonicbox}
``લિસ્ટ્સ: બનાવો, ઇન્ડેક્સ, એક્સેસ''

\end{mnemonicbox}
\subsection*{પ્રશ્ન 5(બ) [4
ગુણ]}\label{uxaaauxab0uxab6uxaa8-5uxaac-4-uxa97uxaa3}

\textbf{LIST પર કરી શકાય તેવી કામગીરીની યાદી બનાવો. એક લિસ્ટને બીજી લિસ્ટમાં
બનાવવા અને કૉપી કરવા માટે પ્રોગ્રામ લખો.}

\begin{solutionbox}

\textbf{લિસ્ટ ઓપરેશન્સ અને કૉપી પ્રોગ્રામ:}

\begin{verbatim}
\# મૂળ લિસ્ટ
original = [1, 2, 3, 4, 5]
print(f"મૂળ લિસ્ટ: \{original\}")

\# કૉપી કરવાની પદ્ધતિઓ
shallow\_copy = original.copy()
slice\_copy = original[:]
list\_copy = list(original)

\# મૂળ લિસ્ટ સુધારો
original.append(6)
print(f"append પછી: \{original\}")
print(f"શેલો કૉપી: \{shallow\_copy\}")

\# લિસ્ટ ઓપરેશન્સ પ્રદર્શન
numbers = [10, 20, 30]
numbers.append(40)          \# અંતે ઉમેરો
numbers.insert(1, 15)       \# પોઝિશન પર ઇન્સર્ટ કરો
numbers.remove(20)          \# સ્પેસિફિક વેલ્યુ દૂર કરો
popped = numbers.pop()      \# છેલ્લું દૂર કરી પરત કરો
\end{verbatim}

\textbf{લિસ્ટ ઓપરેશન્સ ટેબલ:}

{\def\LTcaptype{none} % do not increment counter
\begin{longtable}[]{@{}llll@{}}
\toprule\noalign{}
ઓપરેશન & મેથડ & ઉદાહરણ & પરિણામ \\
\midrule\noalign{}
\endhead
\bottomrule\noalign{}
\endlastfoot
ઉમેરો & \texttt{append()} & \texttt{[1,2].append(3)} & [1,2,3] \\
ઇન્સર્ટ & \texttt{insert()} & \texttt{[1,3].insert(1,2)} &
[1,2,3] \\
દૂર કરો & \texttt{remove()} & \texttt{[1,2,3].remove(2)} &
[1,3] \\
પોપ & \texttt{pop()} & \texttt{[1,2,3].pop()} & [1,2] \\
\end{longtable}
}

\textbf{મુખ્ય કોન્સેપ્ટ્સ:}

\begin{itemize}
\tightlist
\item
  \textbf{શેલો કૉપી}: સમાન તત્વો સાથે સ્વતંત્ર લિસ્ટ
\item
  \textbf{ડીપ કૉપી}: નેસ્ટેડ સ્ટ્રક્ચર માટે જરૂરી
\item
  \textbf{બહુવિધ પદ્ધતિઓ}: કૉપી કરવાની વિવિધ તકનીકો
\end{itemize}

\end{solutionbox}
\begin{mnemonicbox}
``લિસ્ટ ઓપરેશન્સ: ઉમેરો, ઇન્સર્ટ, દૂર કરો, પોપ, કૉપી''

\end{mnemonicbox}
\subsection*{પ્રશ્ન 5(ક) [7
ગુણ]}\label{uxaaauxab0uxab6uxaa8-5uxa95-7-uxa97uxaa3}

\textbf{LIST ની વિવિધ બિલ્ટ-ઇન methods ની સૂચિ બનાવો અને ઉપયોગ દર્શાવો}

\begin{solutionbox}

\textbf{બિલ્ટ-ઇન લિસ્ટ મેથડ્સ:}

\begin{verbatim}
\# નમૂના લિસ્ટ પ્રદર્શન માટે
fruits = [{apple}, {banana}, {cherry}, {apple}]
numbers = [3, 1, 4, 1, 5, 9, 2]

\# સુધારણા મેથડ્સ
fruits.append({date})              \# અંતે ઉમેરો
fruits.insert(1, {avocado})       \# ઇન્ડેક્સ પર ઇન્સર્ટ કરો
fruits.remove({apple})            \# પ્રથમ occurrence દૂર કરો
last\_fruit = fruits.pop()         \# છેલ્લું દૂર કરી પરત કરો
fruits.clear()                    \# બધા તત્વો દૂર કરો

\# શોધ અને ગણતરી મેથડ્સ
fruits = [{apple}, {banana}, {apple}, {cherry}]
count = fruits.count({apple})     \# occurrences ગણો
index = fruits.index({banana})    \# પ્રથમ ઇન્ડેક્સ શોધો

\# સોર્ટિંગ અને રિવર્સિંગ
numbers.sort()                    \# in place સોર્ટ કરો
numbers.reverse()                 \# in place રિવર્સ કરો
sorted\_copy = sorted(fruits)      \# સોર્ટેડ કૉપી પરત કરો

\# એક્સ્ટેન્શન
more\_fruits = [{grape}, {orange}]
fruits.extend(more\_fruits)        \# બહુવિધ આઇટમ્સ ઉમેરો
\end{verbatim}

\textbf{લિસ્ટ મેથડ્સ સારાંશ:}

{\def\LTcaptype{none} % do not increment counter
\begin{longtable}[]{@{}lllll@{}}
\toprule\noalign{}
કેટેગરી & મેથડ & હેતુ & પરત કરે & મૂળ સુધારે \\
\midrule\noalign{}
\endhead
\bottomrule\noalign{}
\endlastfoot
ઉમેરો & \texttt{append(x)} & અંતે આઇટમ ઉમેરો & None & હા \\
ઉમેરો & \texttt{insert(i,x)} & પોઝિશન પર ઇન્સર્ટ કરો & None & હા \\
ઉમેરો & \texttt{extend(list)} & બહુવિધ આઇટમ્સ ઉમેરો & None & હા \\
દૂર કરો & \texttt{remove(x)} & પ્રથમ x દૂર કરો & None & હા \\
દૂર કરો & \texttt{pop(i)} & ઇન્ડેક્સ પર દૂર કરો & દૂર કરેલ આઇટમ & હા \\
દૂર કરો & \texttt{clear()} & બધું દૂર કરો & None & હા \\
શોધ & \texttt{index(x)} & પોઝિશન શોધો & ઇન્ડેક્સ & ના \\
શોધ & \texttt{count(x)} & occurrences ગણો & કાઉન્ટ & ના \\
સોર્ટ & \texttt{sort()} & in place સોર્ટ કરો & None & હા \\
સોર્ટ & \texttt{reverse()} & ક્રમ ઉલટાવો & None & હા \\
કૉપી & \texttt{copy()} & શેલો કૉપી & નવી લિસ્ટ & ના \\
\end{longtable}
}

\textbf{વ્યવહારિક ઉદાહરણો:}

\begin{verbatim}
\# શોપિંગ કાર્ટ ઉદાહરણ
cart = []
cart.append({દૂધ})
cart.extend([{બ્રેડ}, {ઈંડા}, {માખણ}])
print(f"કાર્ટમાં વસ્તુઓ: \{len(cart)\}")

if {દૂધ} in cart:
    cart.remove({દૂધ})
    print("દૂધ કાર્ટમાંથી દૂર કર્યું")

cart.sort()
print(f"સોર્ટેડ કાર્ટ: \{cart\}")
\end{verbatim}

\textbf{મુખ્ય ઉપયોગો:}

\begin{itemize}
\tightlist
\item
  \textbf{ડેટા મેનેજમેન્ટ}: આઇટમ્સ ઉમેરો, દૂર કરો, ગોઠવો
\item
  \textbf{શોધ ઓપરેશન્સ}: તત્વો શોધો અને ગણો
\item
  \textbf{સોર્ટિંગ}: ડેટાને ક્રમમાં ગોઠવો
\end{itemize}

\end{solutionbox}
\begin{mnemonicbox}
``લિસ્ટ મેથડ્સ: ઉમેરો, દૂર કરો, શોધો, સોર્ટ, કૉપી''

\end{mnemonicbox}
\subsection*{પ્રશ્ન 5(અ OR) [3
ગુણ]}\label{uxaaauxab0uxab6uxaa8-5uxa85-or-3-uxa97uxaa3}

\textbf{ઉદાહરણ આપીને string ને કેવી રીતે create અને traverse કરવી તે સમજાવો.}

\begin{solutionbox}

\textbf{સ્ટ્રિંગ સર્જન અને ટ્રાવર્સલ:}

\begin{verbatim}
\# સ્ટ્રિંગ સર્જન પદ્ધતિઓ
string1 = "Hello World"        \# ડબલ કોટ્સ
string2 = {Python Programming} \# સિંગલ કોટ્સ
string3 = """મલ્ટિ{-લાઇન}
સ્ટ્રિંગ ઉદાહરણ"""              \# ટ્રિપલ કોટ્સ

\# સ્ટ્રિંગ ટ્રાવર્સલ પદ્ધતિઓ
text = "Python"

\# પદ્ધતિ 1: for લૂપ વાપરીને
for char in text:
    print(char, end=" ")
print()

\# પદ્ધતિ 2: ઇન્ડેક્સ વાપરીને
for i in range(len(text)):
    print(f"\{text[i]\} ઇન્ડેક્સ \{i\} પર")

\# પદ્ધતિ 3: enumerate વાપરીને
for index, char in enumerate(text):
    print(f"ઇન્ડેક્સ \{index\}: \{char\}")
\end{verbatim}

\textbf{ટ્રાવર્સલ પદ્ધતિઓનું ટેબલ:}

{\def\LTcaptype{none} % do not increment counter
\begin{longtable}[]{@{}lll@{}}
\toprule\noalign{}
પદ્ધતિ & સિન્ટેક્સ & ઉપયોગ કેસ \\
\midrule\noalign{}
\endhead
\bottomrule\noalign{}
\endlastfoot
ડાયરેક્ટ & \texttt{for\ char\ in\ string:} & સાદી કેરેક્ટર એક્સેસ \\
ઇન્ડેક્સ & \texttt{for\ i\ in\ range(len(s)):} & પોઝિશન માહિતી જોઈએ \\
Enumerate & \texttt{for\ i,\ char\ in\ enumerate(s):} & ઇન્ડેક્સ અને કેરેક્ટર
બંને \\
\end{longtable}
}

\textbf{મુખ્ય કોન્સેપ્ટ્સ:}

\begin{itemize}
\tightlist
\item
  \textbf{અપરિવર્તનીય}: સ્ટ્રિંગ્સ બદલી શકાતી નથી
\item
  \textbf{Iterable}: કેરેક્ટર્સમાં લૂપ કરી શકાય
\item
  \textbf{ઇન્ડેક્સિંગ}: વ્યક્તિગત કેરેક્ટર્સ એક્સેસ કરી શકાય
\end{itemize}

\end{solutionbox}
\begin{mnemonicbox}
``સ્ટ્રિંગ્સ: બનાવો, લૂપ, એક્સેસ''

\end{mnemonicbox}
\subsection*{પ્રશ્ન 5(બ OR) [4
ગુણ]}\label{uxaaauxab0uxab6uxaa8-5uxaac-or-4-uxa97uxaa3}

\textbf{સ્ટ્રિંગ પર કરી શકાય તેવી કામગીરીની યાદી બનાવો. કોઈપણ 2 કામગીરી માટે
કોડ લખો}

\begin{solutionbox}

\textbf{સ્ટ્રિંગ ઓપરેશન્સ:}

\begin{verbatim}
\# સ્ટ્રિંગ ઓપરેશન્સ ઉદાહરણો
text = "Python Programming"

\# ઓપરેશન 1: સ્ટ્રિંગ કન્કેટેનેશન અને ફોર્મેટિંગ
first\_name = "જોન"
last\_name = "ડો"
full\_name = first\_name + " " + last\_name
formatted = f"નમસ્તે, \{full\_name\}!"
print(f"કન્કેટેનેશન: \{full\_name\}")
print(f"ફોર્મેટિંગ: \{formatted\}")

\# ઓપરેશન 2: સ્ટ્રિંગ કેસ કન્વર્ઝન અને સ્પ્લિટિંગ
sentence = "python programming સરળતાથી શીખો"
title\_case = sentence.title()
upper\_case = sentence.upper()
words = sentence.split()
print(f"ટાઇટલ કેસ: \{title\_case\}")
print(f"અપર કેસ: \{upper\_case\}")
print(f"સ્પ્લિટ શબ્દો: \{words\}")
\end{verbatim}

\textbf{સ્ટ્રિંગ ઓપરેશન્સ ટેબલ:}

{\def\LTcaptype{none} % do not increment counter
\begin{longtable}[]{@{}llll@{}}
\toprule\noalign{}
કેટેગરી & ઓપરેશન & ઉદાહરણ & પરિણામ \\
\midrule\noalign{}
\endhead
\bottomrule\noalign{}
\endlastfoot
જોડાણ & કન્કેટેનેશન & \texttt{"Hello"\ +\ "\ World"} & ``Hello World'' \\
કેસ & \texttt{upper()} & \texttt{"hello".upper()} & ``HELLO'' \\
કેસ & \texttt{lower()} & \texttt{"HELLO".lower()} & ``hello'' \\
કેસ & \texttt{title()} & \texttt{"hello\ world".title()} & ``Hello
World'' \\
સ્પ્લિટ & \texttt{split()} & \texttt{"a,b,c".split(",")} &
[`a',`b',`c'] \\
રિપ્લેસ & \texttt{replace()} & \texttt{"hello".replace("l","x")} &
``hexxo'' \\
સ્ટ્રિપ & \texttt{strip()} & \texttt{"\ hello\ ".strip()} & ``hello'' \\
શોધ & \texttt{find()} & \texttt{"hello".find("e")} & 1 \\
\end{longtable}
}

\textbf{મુખ્ય લક્ષણો:}

\begin{itemize}
\tightlist
\item
  \textbf{અપરિવર્તનીય}: ઓપરેશન્સ નવી સ્ટ્રિંગ્સ પરત કરે
\item
  \textbf{મેથડ ચેઇનિંગ}: બહુવિધ ઓપરેશન્સ કંબાઇન કરો
\item
  \textbf{લવચીક}: ઘણા બિલ્ટ-ઇન ઓપરેશન્સ ઉપલબ્ધ
\end{itemize}

\end{solutionbox}
\begin{mnemonicbox}
``સ્ટ્રિંગ ઓપરેશન્સ: જોડો, કેસ, સ્પ્લિટ, શોધો''

\end{mnemonicbox}
\subsection*{પ્રશ્ન 5(ક OR) [7
ગુણ]}\label{uxaaauxab0uxab6uxaa8-5uxa95-or-7-uxa97uxaa3}

\textbf{સ્ટ્રિંગની વિવિધ બિલ્ટ-ઇન methods ની સૂચિ બનાવો અને ઉપયોગ દર્શાવો.}

\begin{solutionbox}

\textbf{બિલ્ટ-ઇન સ્ટ્રિંગ મેથડ્સ:}

\begin{verbatim}
\# પ્રદર્શન માટે નમૂના સ્ટ્રિંગ
text = "  Python Programming Language  "
sample = "Hello World Programming"

\# કેસ કન્વર્ઝન મેથડ્સ
print(f"મૂળ: {}\{text\}{"})
print(f"upper(): \{text.upper()\}")
print(f"lower(): \{text.lower()\}")
print(f"title(): \{text.title()\}")
print(f"capitalize(): \{text.capitalize()\}")
print(f"swapcase(): \{{Hello}.swapcase()\}")

\# વ્હાઇટસ્પેસ મેથડ્સ
print(f"strip(): {}\{text.strip()\}{"})
print(f"lstrip(): {}\{text.lstrip()\}{"})
print(f"rstrip(): {}\{text.rstrip()\}{"})

\# શોધ અને ચેક મેથડ્સ
print(f"find({Python): }\{text.find({Python})\}")
print(f"count({o): }\{sample.count({o})\}")
print(f"startswith({  Py): }\{text.startswith({  Py})\}")
print(f"endswith({ge  ): }\{text.endswith({ge  })\}")

\# કેરેક્ટર ટાઇપ ચેકિંગ
test\_string = "Python123"
print(f"isalpha(): \{{Python}.isalpha()\}")
print(f"isdigit(): \{{123}.isdigit()\}")
print(f"isalnum(): \{test\_string.isalnum()\}")

\# સ્પ્લિટ અને જોઇન મેથડ્સ
words = sample.split()
joined = "{-"}.join(words)
print(f"split(): \{words\}")
print(f"join(): \{joined\}")

\# રિપ્લેસ મેથડ
replaced = sample.replace("World", "Universe")
print(f"replace(): \{replaced\}")
\end{verbatim}

\textbf{સ્ટ્રિંગ મેથડ્સ વર્ગીકરણ:}

{\def\LTcaptype{none} % do not increment counter
\begin{longtable}[]{@{}
  >{\raggedright\arraybackslash}p{(\linewidth - 6\tabcolsep) * \real{0.2703}}
  >{\raggedright\arraybackslash}p{(\linewidth - 6\tabcolsep) * \real{0.2432}}
  >{\raggedright\arraybackslash}p{(\linewidth - 6\tabcolsep) * \real{0.2432}}
  >{\raggedright\arraybackslash}p{(\linewidth - 6\tabcolsep) * \real{0.2432}}@{}}
\toprule\noalign{}
\begin{minipage}[b]{\linewidth}\raggedright
કેટેગરી
\end{minipage} & \begin{minipage}[b]{\linewidth}\raggedright
મેથડ્સ
\end{minipage} & \begin{minipage}[b]{\linewidth}\raggedright
હેતુ
\end{minipage} & \begin{minipage}[b]{\linewidth}\raggedright
ઉદાહરણ
\end{minipage} \\
\midrule\noalign{}
\endhead
\bottomrule\noalign{}
\endlastfoot
કેસ & \texttt{upper(),\ lower(),\ title(),\ capitalize()} & કેસ બદલો &
\texttt{"hello".upper()} \rightarrow ``HELLO'' \\
વ્હાઇટસ્પેસ & \texttt{strip(),\ lstrip(),\ rstrip()} & સ્પેસીસ દૂર કરો &
\texttt{"\ hi\ ".strip()} \rightarrow ``hi'' \\
શોધ & \texttt{find(),\ index(),\ count()} & સબસ્ટ્રિંગ્સ શોધો &
\texttt{"hello".find("e")} \rightarrow 1 \\
ચેક & \texttt{startswith(),\ endswith()} & સ્ટ્રિંગ અંત ટેસ્ટ કરો &
\texttt{"hello".startswith("h")} \rightarrow True \\
ટાઇપ ચેક & \texttt{isalpha(),\ isdigit(),\ isalnum()} & કેરેક્ટર પ્રકાર &
\texttt{"123".isdigit()} \rightarrow True \\
સ્પ્લિટ/જોઇન & \texttt{split(),\ join()} & તોડો/જોડો &
\texttt{"a-b".split("-")} \rightarrow [`a',`b'] \\
રિપ્લેસ & \texttt{replace()} & ટેક્સ્ટ બદલો & \texttt{"hi".replace("i","o")}
\rightarrow ``ho'' \\
\end{longtable}
}

\textbf{વાસ્તવિક જીવનના ઉદાહરણો:}

\begin{verbatim}
\# ઈમેઇલ વેલિડેશન ઉદાહરણ
email = "  USER@EXAMPLE.COM  "
clean\_email = email.strip().lower()
is\_valid = "@" in clean\_email and "." in clean\_email
print(f"સાફ ઈમેઇલ: \{clean\_email\}")
print(f"યોગ્ય ફોર્મેટ: \{is\_valid\}")

\# ટેક્સ્ટ પ્રોસેસિંગ ઉદાહરણ
user\_input = "python programming"
formatted\_title = user\_input.title()
word\_count = len(user\_input.split())
print(f"ફોર્મેટેડ: \{formatted\_title\}")
print(f"શબ્દ ગણતરી: \{word\_count\}")
\end{verbatim}

\textbf{મુખ્ય ઉપયોગો:}

\begin{itemize}
\tightlist
\item
  \textbf{ડેટા ક્લીનિંગ}: અનઇચ્છિત સ્પેસીસ દૂર કરો, કેસ ઠીક કરો
\item
  \textbf{ટેક્સ્ટ પ્રોસેસિંગ}: સર્ચ, રિપ્લેસ, સ્પ્લિટ કન્ટેન્ટ
\item
  \textbf{વેલિડેશન}: સ્ટ્રિંગ ફોર્મેટ અને કન્ટેન્ટ ચેક કરો
\item
  \textbf{ફોર્મેટિંગ}: ડિસ્પ્લે માટે ટેક્સ્ટ તૈયાર કરો
\end{itemize}

\end{solutionbox}
\begin{mnemonicbox}
``સ્ટ્રિંગ મેથડ્સ: કેસ, સાફ, ચેક, બદલો''

\end{mnemonicbox}

\end{document}
