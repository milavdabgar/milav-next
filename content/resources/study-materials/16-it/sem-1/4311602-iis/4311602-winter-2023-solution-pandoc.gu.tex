\documentclass[10pt,a4paper]{article}

% content/resources/templates/preamble.tex
\usepackage[margin=0.6in]{geometry}
\author{Milav Dabgar}
\usepackage{amsmath,amssymb,amsthm}
\usepackage{booktabs}
\usepackage{multirow}
\usepackage{xcolor}
\usepackage{tcolorbox}
\tcbuselibrary{breakable,skins}
\usepackage[colorlinks=true,linkcolor=blue]{hyperref}
\usepackage{titlesec}
\usepackage{enumitem}
\usepackage{tikz}
\usepackage{pgfplots}
\usepackage{circuitikz}
\usepackage[version=4]{mhchem}
\usepackage{longtable}
\usepackage{array}
\usepackage{float}
\usepackage{caption}
\usepackage{listings}

\lstset{
  basicstyle=\small\ttfamily,
  breaklines=true,
  breakatwhitespace=false,
  postbreak=\mbox{\textcolor{red}{$\hookrightarrow$}\space},
  float=false,
  numbers=left,
  numberstyle=\tiny\color{gray},
  numbersep=10pt,
  xleftmargin=2em,
  keywordstyle=\color{blue},
  commentstyle=\color{green!60!black},
  stringstyle=\color{purple},
  backgroundcolor=\color{gray!5},
  showstringspaces=false,
  tabsize=2,
  captionpos=b,
  keepspaces=true,
  columns=flexible
}

\pgfplotsset{compat=1.18}
\usetikzlibrary{shapes,arrows,positioning,calc,patterns,decorations.pathmorphing,decorations.markings,arrows.meta}

% Color scheme
\definecolor{headcolor}{RGB}{0,102,204}
\definecolor{keycolor}{RGB}{220,20,60}
\definecolor{solutioncolor}{RGB}{34,139,34}
\definecolor{mnemoniccolor}{RGB}{148,0,211}
\definecolor{codecolor}{RGB}{0,0,100}

% Spacing
\setlength{\parskip}{3pt}
\setlist[itemize]{nosep}
\setlist[enumerate]{nosep}

% Title formatting
\titleformat{\section}{\Large\bfseries\color{headcolor}}{\thesection}{1em}{}
\titleformat{\subsection}{\large\bfseries\color{headcolor}}{\thesubsection}{1em}{}

% Pandoc tightlist compatibility
\providecommand{\tightlist}{%
  \setlength{\itemsep}{0pt}\setlength{\parskip}{0pt}}

% Pandoc longtable compatibility
\newcounter{none}
\def\thenone{}


% content/resources/templates/gujarati-boxes.tex
\usepackage{fontspec}
\usepackage{polyglossia}

% Set Gujarati as main language (document is primarily in Gujarati)
% Note: gloss-gujarati.ldf doesn't exist in polyglossia, but it will use hyphenation patterns
\setdefaultlanguage{gujarati}
\setotherlanguage{english}

% Configure Gujarati font properly
% Use Language=Default to prevent polyglossia from trying to add language-specific features
% that don't exist for Gujarati, which causes "empty feature" warnings
\newfontfamily\gujaratifont[Script=Gujarati,AutoFakeBold=2.5,AutoFakeSlant=0.3]{Noto Sans Gujarati}
\setmainfont[Script=Gujarati,AutoFakeBold=2.5,AutoFakeSlant=0.3]{Noto Sans Gujarati}
% Use Noto Sans Gujarati for monospace to support Gujarati in text
\setmonofont[Scale=0.9]{Noto Sans Gujarati}

% Configure English to use the same font
\newfontfamily\englishfont[Script=Gujarati,AutoFakeBold=2.5,AutoFakeSlant=0.3]{Noto Sans Gujarati}

% Translations for polyglossia
\gappto\captionsgujarati{
  \renewcommand{\tablename}{કોષ્ટક}
  \renewcommand{\figurename}{આકૃતિ}
}

% Helper for TikZ nodes to ensure Gujarati font
\newcommand{\gu}[1]{{\gujaratifont #1}}

% Custom environments
\newtcolorbox{solutionbox}{
    breakable,
    enhanced,
    colback=solutioncolor!5!white,
    colframe=solutioncolor!75!black,
    fonttitle=\bfseries,
    title=જવાબ
}

\newtcolorbox{solutionboxnobreak}{
 colback=solutioncolor!5!white,
 colframe=solutioncolor!75!black,
 fonttitle=\bfseries,
 title=જવાબ
}

\newtcolorbox{keyformula}{
 breakable,
 enhanced,
 colback=keycolor!5!white,
 colframe=keycolor!75!black,
 fonttitle=\bfseries,
 title=રાસાયણિક સમીકરણ/સૂત્ર
}

\newtcolorbox{mnemonicbox}{
 breakable,
 enhanced,
 colback=mnemoniccolor!5!white,
 colframe=mnemoniccolor!75!black,
 fonttitle=\bfseries,
 title=મેમરી ટ્રીક
}


\begin{document}

\begin{center}
{\Huge\bfseries\color{headcolor} Subject Name (Gujarati)}\\[5pt]
{\LARGE 4311602 -- Winter 2023}\\[3pt]
{\large Semester 1 Study Material}\\[3pt]
{\normalsize\textit{Detailed Solutions and Explanations}}
\end{center}

\vspace{10pt}

\subsection*{પ્રશ્ન 1(અ) [3
ગુણ]}\label{uxaaauxab0uxab6uxaa8-1uxa85-3-uxa97uxaa3}

\textbf{Information અને Knowledge વચ્ચેનો તફાવત આપો.}

\begin{solutionbox}

{\def\LTcaptype{none} % do not increment counter
\begin{longtable}[]{@{}lll@{}}
\toprule\noalign{}
\textbf{પાસાં} & \textbf{Information} & \textbf{Knowledge} \\
\midrule\noalign{}
\endhead
\bottomrule\noalign{}
\endlastfoot
\textbf{વ્યાખ્યા} & કાચા તથ્યો અને આંકડાઓ & અનુભવ સાથે પ્રક્રિયા કરેલી માહિતી \\
\textbf{પ્રક્રિયા} & ગોઠવેલો ડેટા & અનુભવ સાથે જોડાયેલી માહિતી \\
\textbf{ઉપયોગ} & સહેલાઈથી શેર કરી શકાય & અર્થઘટન અને સંદર્ભ જરૂરી \\
\end{longtable}
}

\begin{itemize}
\tightlist
\item
  \textbf{Information}: કાચા તથ્યો, ડેટા અને આંકડાઓ કે જેની પ્રક્રિયા કરી શકાય
\item
  \textbf{Knowledge}: અનુભવ અને શિક્ષણ દ્વારા પ્રાપ્ત સમજ
\end{itemize}

\end{solutionbox}
\begin{mnemonicbox}
``Information માહિતી આપે, Knowledge જ્ઞાન આપે''

\end{mnemonicbox}
\subsection*{પ્રશ્ન 1(બ) [4
ગુણ]}\label{uxaaauxab0uxab6uxaa8-1uxaac-4-uxa97uxaa3}

\textbf{OS ના કાર્યો સમજાવો.}

\begin{solutionbox}

\textbf{ઓપરેટિંગ સિસ્ટમના મુખ્ય કાર્યો:}

{\def\LTcaptype{none} % do not increment counter
\begin{longtable}[]{@{}ll@{}}
\toprule\noalign{}
\textbf{કાર્ય} & \textbf{વર્ણન} \\
\midrule\noalign{}
\endhead
\bottomrule\noalign{}
\endlastfoot
\textbf{Process Management} & પ્રોગ્રામ્સના અમલીકરણને નિયંત્રિત કરે \\
\textbf{Memory Management} & મેમરી ફાળવણી અને મુક્તિ \\
\textbf{File Management} & ફાઇલોનું સંગઠન અને વ્યવસ્થાપન \\
\textbf{Device Management} & ઇનપુટ/આઉટપુટ ઉપકરણોનું નિયંત્રણ \\
\end{longtable}
}

\begin{itemize}
\tightlist
\item
  \textbf{Process Control}: ચાલતા પ્રોગ્રામ્સનું શેડ્યુલિંગ અને વ્યવસ્થાપન
\item
  \textbf{Resource Allocation}: સિસ્ટમ સંસાધનોનું કાર્યક્ષમ વિતરણ
\item
  \textbf{User Interface}: યુઝર અને કમ્પ્યુટર વચ્ચે ક્રિયાપ્રતિક્રિયા
\end{itemize}

\end{solutionbox}
\begin{mnemonicbox}
``PMFD - Process, Memory, File, Device''

\end{mnemonicbox}
\subsection*{પ્રશ્ન 1(ક) [7
ગુણ]}\label{uxaaauxab0uxab6uxaa8-1uxa95-7-uxa97uxaa3}

\textbf{યુનિવર્સલ ગેટ વ્યાખ્યાયિત કરો અને NAND યુનિવર્સલ ગેટનો ઉપયોગ કરીને બેસિક ગેટ
બનાવો.}

\begin{solutionbox}

\textbf{યુનિવર્સલ ગેટની વ્યાખ્યા:} એવા લોજિક ગેટ કે જે અન્ય કોઈ ગેટનો ઉપયોગ કર્યા
વિના કોઈપણ Boolean function અમલ કરી શકે.

\textbf{NAND ગેટ Truth Table:}

{\def\LTcaptype{none} % do not increment counter
\begin{longtable}[]{@{}lll@{}}
\toprule\noalign{}
A & B & NAND આઉટપુટ \\
\midrule\noalign{}
\endhead
\bottomrule\noalign{}
\endlastfoot
0 & 0 & 1 \\
0 & 1 & 1 \\
1 & 0 & 1 \\
1 & 1 & 0 \\
\end{longtable}
}

\textbf{NAND વડે બેસિક ગેટ્સ:}

\begin{verbatim}
NOT Gate using NAND:
A {-{-}{-}{-}+}
      |
      NAND {-{-}{-}{-} આઉટપુટ (NOT A)}
      |
A {-{-}{-}{-}+}

AND Gate using NAND:
A {-{-}{-}{-}+}
      |
      NAND {-{-}{-}{-} NAND {-}{-}{-}{-} આઉટપુટ (A AND B)}
      |
B {-{-}{-}{-}+}

OR Gate using NAND:
A {-{-}{-}{-} NAND {-}{-}{-}{-}+}
                |
                NAND {-{-}{-}{-} આઉટપુટ (A OR B)}
                |
B {-{-}{-}{-} NAND {-}{-}{-}{-}+}
\end{verbatim}

\begin{itemize}
\tightlist
\item
  \textbf{NOT}: બંને NAND ઇનપુટમાં એક જ ઇનપુટ આપવું
\item
  \textbf{AND}: NAND પછી NOT (બીજું NAND)
\item
  \textbf{OR}: બંને ઇનપુટ્સને NOT કરો, પછી NAND કરો
\end{itemize}

\end{solutionbox}
\begin{mnemonicbox}
``NAND ને બીજા NAND ની નિશ્ચિત જરૂર''

\end{mnemonicbox}
\subsection*{પ્રશ્ન 1(ક OR) [7
ગુણ]}\label{uxaaauxab0uxab6uxaa8-1uxa95-or-7-uxa97uxaa3}

\textbf{નીચેના રૂપાંતરણ કરો:}

\begin{solutionbox}

\textbf{રૂપાંતરણ ઉકેલો:}

{\def\LTcaptype{none} % do not increment counter
\begin{longtable}[]{@{}llll@{}}
\toprule\noalign{}
\textbf{માંથી} & \textbf{માં} & \textbf{પ્રક્રિયા} & \textbf{પરિણામ} \\
\midrule\noalign{}
\endhead
\bottomrule\noalign{}
\endlastfoot
(1456)_{8} & Base 16 & 8\rightarrow10\rightarrow16 & (32E)_{1}_{6} \\
(1011)_{2} & Base 10 & Binary to Decimal & (11)_{1}_{0} \\
(247.38)_{1}_{0} & Base 8 & Integer અને Fraction અલગ & (367.3)_{8} \\
\end{longtable}
}

\textbf{વિગતવાર ઉકેલ:}

\begin{enumerate}
\tightlist
\item
  \textbf{(1456)_{8} = (32E)_{1}_{6}}

  \begin{itemize}
  \tightlist
  \item
    1\times8^{3} + 4\times8^{2} + 5\times8^{1} + 6\times8^{0} = 512 + 256 + 40 + 6 = (814)_{1}_{0}
  \item
    814 \div 16 = 50 remainder 14(E), 50 \div 16 = 3 remainder 2
  \item
    પરિણામ: (32E)_{1}_{6}
  \end{itemize}
\item
  \textbf{(1011)_{2} = (11)_{1}_{0}}

  \begin{itemize}
  \tightlist
  \item
    1\times2^{3} + 0\times2^{2} + 1\times2^{1} + 1\times2^{0} = 8 + 0 + 2 + 1 = (11)_{1}_{0}
  \end{itemize}
\item
  \textbf{(247.38)_{1}_{0} = (367.3)_{8}}

  \begin{itemize}
  \tightlist
  \item
    પૂર્ણાંક: 247 \div 8 = 30 બાકી 7, 30 \div 8 = 3 બાકી 6, 3 \div 8 = 0 બાકી 3
  \item
    દશાંશ: 0.38 \times 8 = 3.04 (3 લો)
  \item
    પરિણામ: (367.3)_{8}
  \end{itemize}
\end{enumerate}

\end{solutionbox}
\begin{mnemonicbox}
``રૂપાંતરણ સાવચેતીથી, ગણતરી ચકાસીને''

\end{mnemonicbox}
\subsection*{પ્રશ્ન 2(અ) [3
ગુણ]}\label{uxaaauxab0uxab6uxaa8-2uxa85-3-uxa97uxaa3}

\textbf{મેમરીના પ્રકારોની સૂચિ બનાવો.}

\begin{solutionbox}

\textbf{મેમરી વર્ગીકરણ:}

{\def\LTcaptype{none} % do not increment counter
\begin{longtable}[]{@{}lll@{}}
\toprule\noalign{}
\textbf{પ્રકાર} & \textbf{ઉદાહરણ} & \textbf{લાક્ષણિકતાઓ} \\
\midrule\noalign{}
\endhead
\bottomrule\noalign{}
\endlastfoot
\textbf{Primary Memory} & RAM, ROM, Cache & CPU દ્વારા સીધી પહોંચ \\
\textbf{Secondary Memory} & HDD, SSD, CD/DVD & બિન-અસ્થાયી સંગ્રહ \\
\textbf{Cache Memory} & L1, L2, L3 & હાઇ-સ્પીડ બફર મેમરી \\
\end{longtable}
}

\begin{itemize}
\tightlist
\item
  \textbf{Volatile}: પાવર બંધ કરવાથી ડેટા ગુમાવે (RAM)
\item
  \textbf{Non-volatile}: પાવર વિના ડેટા જાળવે (ROM, HDD)
\item
  \textbf{ઍક્સેસ સ્પીડ}: Cache \textgreater{} RAM \textgreater{} Secondary
  Storage
\end{itemize}

\end{solutionbox}
\begin{mnemonicbox}
``Primary પ્રક્રિયા કરે, Secondary સંગ્રહ કરે''

\end{mnemonicbox}
\subsection*{પ્રશ્ન 2(બ) [4
ગુણ]}\label{uxaaauxab0uxab6uxaa8-2uxaac-4-uxa97uxaa3}

\textbf{Kernel Mode અને User Mode વચ્ચે તફાવત આપો.}

\begin{solutionbox}

{\def\LTcaptype{none} % do not increment counter
\begin{longtable}[]{@{}lll@{}}
\toprule\noalign{}
\textbf{પાસાં} & \textbf{Kernel Mode} & \textbf{User Mode} \\
\midrule\noalign{}
\endhead
\bottomrule\noalign{}
\endlastfoot
\textbf{અધિકાર સ્તર} & સંપૂર્ણ સિસ્ટમ ઍક્સેસ & મર્યાદિત ઍક્સેસ \\
\textbf{સૂચનાઓ} & બધી સૂચનાઓની મંજૂરી & મર્યાદિત સૂચના સેટ \\
\textbf{મેમરી ઍક્સેસ} & સંપૂર્ણ મેમરી ઍક્સેસ & મર્યાદિત મેમરી વિસ્તારો \\
\textbf{સિસ્ટમ કૉલ્સ} & સીધી હાર્ડવેર ઍક્સેસ & માત્ર સિસ્ટમ કૉલ્સ દ્વારા \\
\end{longtable}
}

\begin{itemize}
\tightlist
\item
  \textbf{Kernel Mode}: ઓપરેટિંગ સિસ્ટમ સંપૂર્ણ અધિકારો સાથે ચાલે
\item
  \textbf{User Mode}: એપ્લિકેશન્સ મર્યાદિત અધિકારો સાથે ચાલે
\item
  \textbf{સુરક્ષા}: મોડ સ્વિચિંગ અનધિકૃત ઍક્સેસ અટકાવે
\end{itemize}

\end{solutionbox}
\begin{mnemonicbox}
``Kernel નિયંત્રણ કરે, User ઉપયોગ કરે''

\end{mnemonicbox}
\subsection*{પ્રશ્ન 2(ક) [7
ગુણ]}\label{uxaaauxab0uxab6uxaa8-2uxa95-7-uxa97uxaa3}

\textbf{OS ના પ્રકારોની યાદી બનાવો અને કોઈપણ બે OS સમજાવો}

\begin{solutionbox}

\textbf{ઓપરેટિંગ સિસ્ટમના પ્રકારો:}

{\def\LTcaptype{none} % do not increment counter
\begin{longtable}[]{@{}lll@{}}
\toprule\noalign{}
\textbf{પ્રકાર} & \textbf{ઉદાહરણ} & \textbf{લાક્ષણિકતાઓ} \\
\midrule\noalign{}
\endhead
\bottomrule\noalign{}
\endlastfoot
\textbf{Batch OS} & પ્રારંભિક mainframes & યુઝર ક્રિયાપ્રતિક્રિયા નથી \\
\textbf{Time-sharing OS} & UNIX, Linux & એકસાથે બહુવિધ યુઝર્સ \\
\textbf{Real-time OS} & Embedded systems & ગેરંટીડ પ્રતિસાદ સમય \\
\textbf{Distributed OS} & Cloud systems & બહુવિધ જોડાયેલા કમ્પ્યુટર્સ \\
\textbf{Network OS} & Windows Server & નેટવર્ક સંસાધન વ્યવસ્થાપન \\
\textbf{Mobile OS} & Android, iOS & સ્માર્ટફોન/ટેબલેટ સિસ્ટમ્સ \\
\end{longtable}
}

\textbf{વિગતવાર સમજૂતી:}

\textbf{1. Time-sharing OS (Linux):}

\begin{itemize}
\tightlist
\item
  \textbf{Multi-user}: બહુવિધ યુઝર્સ એકસાથે ઍક્સેસ કરી શકે
\item
  \textbf{Multi-tasking}: બહુવિધ પ્રક્રિયાઓ સમાંતર ચલાવે
\item
  \textbf{સંસાધન શેરિંગ}: CPU સમય પ્રક્રિયાઓ વચ્ચે વહેંચાય
\item
  \textbf{ઉદાહરણ}: UNIX, Linux, Windows
\end{itemize}

\textbf{2. Real-time OS:}

\begin{itemize}
\tightlist
\item
  \textbf{નિર્ધારિત}: સમય મર્યાદામાં ગેરંટીડ પ્રતિસાદ
\item
  \textbf{પ્રાથમિકતા આધારિત}: મહત્વપૂર્ણ કાર્યોને ઊંચી પ્રાથમિકતા
\item
  \textbf{ઉપયોગ}: મેડિકલ ઉપકરણો, ઔદ્યોગિક નિયંત્રણ
\item
  \textbf{પ્રકાર}: Hard real-time અને Soft real-time
\end{itemize}

\end{solutionbox}
\begin{mnemonicbox}
``સમય ટિક કરે, Real-time રિએક્ટ કરે''

\end{mnemonicbox}
\subsection*{પ્રશ્ન 2(અ OR) [3
ગુણ]}\label{uxaaauxab0uxab6uxaa8-2uxa85-or-3-uxa97uxaa3}

\textbf{Linux Operating System નું આર્કિટેક્ચર સમજાવો.}

\begin{solutionbox}

\textbf{Linux આર્કિટેક્ચર સ્તરો:}

\begin{center}
\textbf{Mermaid Diagram (Code)}
\begin{verbatim}
{Shaded}
{Highlighting}[]
graph LR
    A[User Applications] {-{-}{} B[System Call Interface]}
    B {-{-}{} C[Kernel Space]}
    C {-{-}{} D[Process Management]}
    C {-{-}{} E[Memory Management]}
    C {-{-}{} F[File System]}
    C {-{-}{} G[Device Drivers]}
    G {-{-}{} H[Hardware Layer]}
{Highlighting}
{Shaded}
\end{verbatim}
\end{center}

\begin{itemize}
\tightlist
\item
  \textbf{User Space}: એપ્લિકેશન્સ અને યુઝર પ્રોગ્રામ્સ
\item
  \textbf{System Calls}: યુઝર અને kernel વચ્ચેનું ઇન્ટરફેસ
\item
  \textbf{Kernel}: મુખ્ય ઓપરેટિંગ સિસ્ટમ કાર્યો
\end{itemize}

\end{solutionbox}
\begin{mnemonicbox}
``યુઝર્સ ઉપયોગ કરે, Kernel નિયંત્રણ કરે''

\end{mnemonicbox}
\subsection*{પ્રશ્ન 2(બ OR) [4
ગુણ]}\label{uxaaauxab0uxab6uxaa8-2uxaac-or-4-uxa97uxaa3}

\textbf{Search Engine ની કામગીરી સમજાવો.}

\begin{solutionbox}

\textbf{Search Engine કામકાજની પ્રક્રિયા:}

{\def\LTcaptype{none} % do not increment counter
\begin{longtable}[]{@{}lll@{}}
\toprule\noalign{}
\textbf{સ્ટેપ} & \textbf{પ્રક્રિયા} & \textbf{કાર્ય} \\
\midrule\noalign{}
\endhead
\bottomrule\noalign{}
\endlastfoot
\textbf{Crawling} & વેબ સ્પાઇડર્સ વેબસાઇટ્સ સ્કેન કરે & વેબ પેજીસ શોધે \\
\textbf{Indexing} & કન્ટેન્ટ વિશ્લેષણ અને સંગ્રહ & શોધી શકાય તેવો ડેટાબેસ બનાવે \\
\textbf{Ranking} & ઍલ્ગોરિધમ લાગુ કરે & સુસંગતતાનો ક્રમ નક્કી કરે \\
\textbf{Retrieval} & પરિણામો પરત કરે & ક્રમબદ્ધ પરિણામો દર્શાવે \\
\end{longtable}
}

\textbf{કામકાજના પગલાં:}

\begin{itemize}
\tightlist
\item
  \textbf{વેબ ક્રોલર્સ}: ઓટોમેટેડ બોટ્સ ઇન્ટરનેટ કન્ટેન્ટ સ્કેન કરે
\item
  \textbf{ઇન્ડેક્સ ડેટાબેસ}: વેબપેજ માહિતી સંગ્રહિત અને ગોઠવે
\item
  \textbf{ક્વેરી પ્રોસેસિંગ}: યુઝર શોધ શબ્દોનું વિશ્લેષણ કરે
\item
  \textbf{પરિણામ રેન્કિંગ}: સુસંગતતા અનુસાર પરિણામોનો ક્રમ કરે
\end{itemize}

\end{solutionbox}
\begin{mnemonicbox}
``ક્રોલ, ઇન્ડેક્સ, રેન્ક, પુનઃપ્રાપ્ત''

\end{mnemonicbox}
\subsection*{પ્રશ્ન 2(ક OR) [7
ગુણ]}\label{uxaaauxab0uxab6uxaa8-2uxa95-or-7-uxa97uxaa3}

\textbf{Open Source Software અને Proprietary Software વચ્ચે તફાવત આપો.}

\begin{solutionbox}

{\def\LTcaptype{none} % do not increment counter
\begin{longtable}[]{@{}
  >{\raggedright\arraybackslash}p{(\linewidth - 4\tabcolsep) * \real{0.1818}}
  >{\raggedright\arraybackslash}p{(\linewidth - 4\tabcolsep) * \real{0.4091}}
  >{\raggedright\arraybackslash}p{(\linewidth - 4\tabcolsep) * \real{0.4091}}@{}}
\toprule\noalign{}
\begin{minipage}[b]{\linewidth}\raggedright
\textbf{પાસાં}
\end{minipage} & \begin{minipage}[b]{\linewidth}\raggedright
\textbf{Open Source Software}
\end{minipage} & \begin{minipage}[b]{\linewidth}\raggedright
\textbf{Proprietary Software}
\end{minipage} \\
\midrule\noalign{}
\endhead
\bottomrule\noalign{}
\endlastfoot
\textbf{સોર્સ કોડ} & મુક્તપણે ઉપલબ્ધ અને સુધારી શકાય & બંધ અને સુરક્ષિત \\
\textbf{કિંમત} & સામાન્યતે મફત & લાઇસન્સ ખરીદવાની જરૂર \\
\textbf{સપોર્ટ} & કમ્યુનિટી આધારિત & વેન્ડર દ્વારા પૂરું પાડવામાં આવે \\
\textbf{કસ્ટમાઇઝેશન} & સંપૂર્ણ કસ્ટમાઇઝ કરી શકાય & મર્યાદિત કસ્ટમાઇઝેશન \\
\textbf{ઉદાહરણ} & Linux, Firefox, LibreOffice & Windows, MS Office,
Photoshop \\
\textbf{સુરક્ષા} & પારદર્શક, કમ્યુનિટી ઓડિટેડ & અસ્પષ્ટતા દ્વારા સુરક્ષા \\
\textbf{અપડેટ્સ} & કમ્યુનિટી સંચાલિત & વેન્ડર નિયંત્રિત \\
\end{longtable}
}

\textbf{મુખ્ય તફાવતો:}

\begin{itemize}
\tightlist
\item
  \textbf{લાઇસન્સિંગ}: Open source પુનઃવિતરણ અને સુધારાની મંજૂરી આપે vs
  proprietary પેઇડ
\item
  \textbf{કિંમત મોડેલ}: Open source સામાન્યતે મફત vs proprietary પેઇડ
\item
  \textbf{ડેવલપમેન્ટ}: કમ્યુનિટી સહયોગ vs કંપની નિયંત્રિત
\item
  \textbf{પારદર્શિતા}: Open source કોડ દૃશ્યમાન vs proprietary છુપાયેલ
\end{itemize}

\textbf{ફાયદા:}

\begin{itemize}
\tightlist
\item
  \textbf{Open Source}: કિફાયતી, કસ્ટમાઇઝ કરી શકાય, સુરક્ષિત
\item
  \textbf{Proprietary}: વ્યાવસાયિક સપોર્ટ, એકીકૃત લક્ષણો, યુઝર-ફ્રેન્ડલી
\end{itemize}

\end{solutionbox}
\begin{mnemonicbox}
``Open ખુલ્લું કરે, Proprietary સુરક્ષિત કરે''

\end{mnemonicbox}
\subsection*{પ્રશ્ન 3(અ) [3
ગુણ]}\label{uxaaauxab0uxab6uxaa8-3uxa85-3-uxa97uxaa3}

\textbf{નીચેનાનું સંપૂર્ણ નામ આપો: OSI, LLC, FTP}

\begin{solutionbox}

\textbf{સંપૂર્ણ રૂપો:}

{\def\LTcaptype{none} % do not increment counter
\begin{longtable}[]{@{}ll@{}}
\toprule\noalign{}
\textbf{સંક્ષેપ} & \textbf{સંપૂર્ણ રૂપ} \\
\midrule\noalign{}
\endhead
\bottomrule\noalign{}
\endlastfoot
\textbf{OSI} & Open Systems Interconnection \\
\textbf{LLC} & Logical Link Control \\
\textbf{FTP} & File Transfer Protocol \\
\end{longtable}
}

\begin{itemize}
\tightlist
\item
  \textbf{OSI}: 7 સ્તરો સાથેનું નેટવર્કિંગ સંદર્ભ મોડેલ
\item
  \textbf{LLC}: OSI મોડેલમાં Data Link Layer નું સબલેયર
\item
  \textbf{FTP}: નેટવર્ક પર ફાઇલો ટ્રાન્સફર કરવા માટેનું પ્રોટોકોલ
\end{itemize}

\end{solutionbox}
\begin{mnemonicbox}
``Open Logic Files''

\end{mnemonicbox}
\subsection*{પ્રશ્ન 3(બ) [4
ગુણ]}\label{uxaaauxab0uxab6uxaa8-3uxaac-4-uxa97uxaa3}

\textbf{Twisted Pair Cable ના ફાયદા અને ગેરફાયદા આપો}

\begin{solutionbox}

\textbf{Twisted Pair Cable વિશ્લેષણ:}

{\def\LTcaptype{none} % do not increment counter
\begin{longtable}[]{@{}ll@{}}
\toprule\noalign{}
\textbf{ફાયદા} & \textbf{ગેરફાયદા} \\
\midrule\noalign{}
\endhead
\bottomrule\noalign{}
\endlastfoot
\textbf{ઓછી કિંમત} & \textbf{મર્યાદિત અંતર} \\
\textbf{સરળ ઇન્સ્ટોલેશન} & \textbf{ઇલેક્ટ્રોમેગ્નેટિક હસ્તક્ષેપ} \\
\textbf{લવચીક} & \textbf{ઓછી બેન્ડવિડ્થ} \\
\textbf{વ્યાપકપણે ઉપલબ્ધ} & \textbf{સુરક્ષા સમસ્યાઓ} \\
\end{longtable}
}

\textbf{ફાયદા:}

\begin{itemize}
\tightlist
\item
  \textbf{કિફાયતી}: સૌથી સસ્તો નેટવર્કિંગ કેબલ વિકલ્પ
\item
  \textbf{સરળ ઇન્સ્ટોલેશન}: ઇન્સ્ટોલ અને જાળવણી સરળ
\item
  \textbf{લવચીકતા}: સહેલાઈથી વાળી અને રૂટ કરી શકાય
\end{itemize}

\textbf{ગેરફાયદા:}

\begin{itemize}
\tightlist
\item
  \textbf{અંતર મર્યાદા}: રિપીટર વિના મહત્તમ 100 મીટર
\item
  \textbf{હસ્તક્ષેપ}: ઇલેક્ટ્રોમેગ્નેટિક હસ્તક્ષેપ માટે સંવેદનશીલ
\item
  \textbf{બેન્ડવિડ્થ}: ફાઇબર કરતાં ઓછા ડેટા ટ્રાન્સમિશન રેટ
\end{itemize}

\end{solutionbox}
\begin{mnemonicbox}
``Twisted સસ્તું પણ મર્યાદિત''

\end{mnemonicbox}
\subsection*{પ્રશ્ન 3(ક) [7
ગુણ]}\label{uxaaauxab0uxab6uxaa8-3uxa95-7-uxa97uxaa3}

\textbf{Modulation શું છે? Analog Modulation સમજાવો.}

\begin{solutionbox}

\textbf{Modulation ની વ્યાખ્યા:} લાંબા અંતર સુધી માહિતી ટ્રાન્સમિટ કરવા માટે
carrier signal ની લાક્ષણિકતાઓ બદલવાની પ્રક્રિયા.

\textbf{Analog Modulation પ્રકારો:}

{\def\LTcaptype{none} % do not increment counter
\begin{longtable}[]{@{}lll@{}}
\toprule\noalign{}
\textbf{પ્રકાર} & \textbf{બદલાતું પરિમાણ} & \textbf{ઉપયોગ} \\
\midrule\noalign{}
\endhead
\bottomrule\noalign{}
\endlastfoot
\textbf{AM} & Amplitude & રેડિયો બ્રોડકાસ્ટિંગ \\
\textbf{FM} & Frequency & FM રેડિયો, TV સાઉન્ડ \\
\textbf{PM} & Phase & ડિજિટલ કમ્યુનિકેશન્સ \\
\end{longtable}
}

\textbf{Amplitude Modulation (AM):}

\begin{center}
\textbf{Mermaid Diagram (Code)}
\begin{verbatim}
{Shaded}
{Highlighting}[]
graph LR
    A[Message Signal] {-{-}{} C[Modulator]}
    B[Carrier Signal] {-{-}{} C}
    C {-{-}{} D[Modulated Signal]}
{Highlighting}
{Shaded}
\end{verbatim}
\end{center}

\textbf{મુખ્ય ખ્યાલો:}

\begin{itemize}
\tightlist
\item
  \textbf{Carrier Wave}: ટ્રાન્સમિશન માટે હાઇ-ફ્રીક્વન્સી સિગ્નલ
\item
  \textbf{Message Signal}: ટ્રાન્સમિટ કરવાની માહિતી
\item
  \textbf{Modulation Index}: લાગુ કરેલ modulation ની માત્રા
\end{itemize}

\textbf{ઉપયોગ:}

\begin{itemize}
\tightlist
\item
  \textbf{AM Radio}: 530-1710 kHz ફ્રીક્વન્સી બેન્ડ
\item
  \textbf{FM Radio}: 88-108 MHz ફ્રીક્વન્સી બેન્ડ
\item
  \textbf{ટેલિવિઝન}: વિવિધ modulation તકનીકો
\end{itemize}

\textbf{ફાયદા:}

\begin{itemize}
\tightlist
\item
  \textbf{લાંબું અંતર}: લાંબા અંતરની કમ્યુનિકેશન શક્ય બનાવે
\item
  \textbf{Noise Immunity}: FM વધુ સારી noise પ્રતિકાર આપે
\end{itemize}

\end{solutionbox}
\begin{mnemonicbox}
``Amplitude બદલાય, Frequency ફ્લક્ચ્યુએટ કરે''

\end{mnemonicbox}
\subsection*{પ્રશ્ન 3(અ OR) [3
ગુણ]}\label{uxaaauxab0uxab6uxaa8-3uxa85-or-3-uxa97uxaa3}

\textbf{Network Topology ની યાદી બનાવો. Bus Topology ના ફાયદા અને ગેરફાયદા
લખો.}

\begin{solutionbox}

\textbf{નેટવર્ક ટોપોલોજીઓ:}

\begin{itemize}
\tightlist
\item
  \textbf{Bus Topology}
\item
  \textbf{Star Topology}
\item
  \textbf{Ring Topology}
\item
  \textbf{Mesh Topology}
\item
  \textbf{Hybrid Topology}
\end{itemize}

\textbf{Bus Topology વિશ્લેષણ:}

{\def\LTcaptype{none} % do not increment counter
\begin{longtable}[]{@{}ll@{}}
\toprule\noalign{}
\textbf{ફાયદા} & \textbf{ગેરફાયદા} \\
\midrule\noalign{}
\endhead
\bottomrule\noalign{}
\endlastfoot
\textbf{સરળ ડિઝાઇન} & \textbf{સિંગલ પોઇન્ટ ઓફ ફેઇલ્યુર} \\
\textbf{કિફાયતી} & \textbf{મર્યાદિત કેબલ લંબાઇ} \\
\textbf{સરળ વિસ્તરણ} & \textbf{પર્ફોર્મન્સ ઘટાડો} \\
\end{longtable}
}

\end{solutionbox}
\begin{mnemonicbox}
``Bus સરળ પણ સિંગલ-ફેઇલ્યુર-પ્રોન''

\end{mnemonicbox}
\subsection*{પ્રશ્ન 3(બ OR) [4
ગુણ]}\label{uxaaauxab0uxab6uxaa8-3uxaac-or-4-uxa97uxaa3}

\textbf{Serial અને Parallel Transmission વચ્ચેનો તફાવત જણાવો.}

\begin{solutionbox}

{\def\LTcaptype{none} % do not increment counter
\begin{longtable}[]{@{}lll@{}}
\toprule\noalign{}
\textbf{પાસાં} & \textbf{Serial Transmission} & \textbf{Parallel
Transmission} \\
\midrule\noalign{}
\endhead
\bottomrule\noalign{}
\endlastfoot
\textbf{ડેટા પાથ} & સિંગલ કમ્યુનિકેશન લાઇન & એકસાથે બહુવિધ લાઇન્સ \\
\textbf{સ્પીડ} & ટૂંકા અંતર માટે ધીમું & ટૂંકા અંતર માટે ઝડપી \\
\textbf{કિંમત} & ઓછી કિંમત & વધારે કિંમત \\
\textbf{અંતર} & લાંબા અંતર માટે યોગ્ય & ટૂંકા અંતર માટે મર્યાદિત \\
\end{longtable}
}

\textbf{લાક્ષણિકતાઓ:}

\begin{itemize}
\tightlist
\item
  \textbf{Serial}: બિટ્સ એક પછી એક ટ્રાન્સમિટ થાય
\item
  \textbf{Parallel}: બહુવિધ બિટ્સ એકસાથે ટ્રાન્સમિટ થાય
\item
  \textbf{ઉપયોગ}: નેટવર્ક માટે Serial, આંતરિક બસ માટે Parallel
\end{itemize}

\end{solutionbox}
\begin{mnemonicbox}
``Serial સિંગલ-ફાઇલ, Parallel પ્રોસેસીસ''

\end{mnemonicbox}
\subsection*{પ્રશ્ન 3(ક OR) [7
ગુણ]}\label{uxaaauxab0uxab6uxaa8-3uxa95-or-7-uxa97uxaa3}

\textbf{Transmission Modes સમજાવો.}

\begin{solutionbox}

\textbf{Transmission Modes વર્ગીકરણ:}

{\def\LTcaptype{none} % do not increment counter
\begin{longtable}[]{@{}llll@{}}
\toprule\noalign{}
\textbf{મોડ} & \textbf{દિશા} & \textbf{ઉદાહરણ} & \textbf{ઉપયોગ} \\
\midrule\noalign{}
\endhead
\bottomrule\noalign{}
\endlastfoot
\textbf{Simplex} & માત્ર એક દિશા & રેડિયો, TV બ્રોડકાસ્ટ & બ્રોડકાસ્ટિંગ \\
\textbf{Half-duplex} & બંને દિશા, એકસાથે નહીં & વોકી-ટૉકી & વારાફરતી
કમ્યુનિકેશન \\
\textbf{Full-duplex} & બંને દિશા એકસાથે & ટેલિફોન & રિયલ-ટાઇમ કમ્યુનિકેશન \\
\end{longtable}
}

\textbf{વિગતવાર સમજૂતી:}

\textbf{1. Simplex Mode:}

\begin{itemize}
\tightlist
\item
  \textbf{એકદિશીય}: ડેટા માત્ર એક દિશામાં વહે
\item
  \textbf{ઉદાહરણ}: ટેલિવિઝન બ્રોડકાસ્ટિંગ, રેડિયો ટ્રાન્સમિશન
\item
  \textbf{ફાયદો}: સરળ અમલીકરણ
\item
  \textbf{ગેરફાયદો}: ફીડબેક શક્ય નથી
\end{itemize}

\textbf{2. Half-duplex Mode:}

\begin{itemize}
\tightlist
\item
  \textbf{દ્વિદિશીય}: બંને દિશામાં ડેટા વહી શકે, પણ એકસાથે નહીં
\item
  \textbf{ઉદાહરણ}: વોકી-ટૉકીઝ, CB રેડિયો
\item
  \textbf{ફાયદો}: સિંગલ ચેનલ સાથે બે-દિશીય કમ્યુનિકેશન
\item
  \textbf{ગેરફાયદો}: એકસાથે મોકલી અને મેળવી શકાતું નથી
\end{itemize}

\textbf{3. Full-duplex Mode:}

\begin{itemize}
\tightlist
\item
  \textbf{એકસાથે દ્વિદિશીય}: બંને દિશામાં એક જ સમયે ડેટા વહે
\item
  \textbf{ઉદાહરણ}: ટેલિફોન વાતચીત, આધુનિક નેટવર્ક્સ
\item
  \textbf{ફાયદો}: કાર્યક્ષમ રિયલ-ટાઇમ કમ્યુનિકેશન
\item
  \textbf{ગેરફાયદો}: વધુ જટિલ અમલીકરણ જરૂરી
\end{itemize}

\end{solutionbox}
\begin{mnemonicbox}
``Simplex સિંગલ, Half-duplex અટકે, Full-duplex વહે''

\end{mnemonicbox}
\subsection*{પ્રશ્ન 4(અ) [3
ગુણ]}\label{uxaaauxab0uxab6uxaa8-4uxa85-3-uxa97uxaa3}

\textbf{Crossover Ethernet Cable દોરો.}

\begin{solutionbox}

\textbf{Crossover Cable વાયરિંગ ડાયાગ્રામ:}

\begin{verbatim}
RJ{-45 Connector A          RJ{-}45 Connector B}
Pin 1: White{-Orange  {-}{-}{-} Pin 3: White{-}Green}
Pin 2: Orange        {{-}{-}{-} Pin 6: Green}
Pin 3: White{-Green   {-}{-}{-} Pin 1: White{-}Orange}
Pin 4: Blue          {{-}{-}{-} Pin 4: Blue}
Pin 5: White{-Blue    {-}{-}{-} Pin 5: White{-}Blue}
Pin 6: Green         {{-}{-}{-} Pin 2: Orange}
Pin 7: White{-Brown   {-}{-}{-} Pin 7: White{-}Brown}
Pin 8: Brown         {{-}{-}{-} Pin 8: Brown}
\end{verbatim}

\textbf{મુખ્ય મુદ્દાઓ:}

\begin{itemize}
\tightlist
\item
  \textbf{હેતુ}: સમાન ઉપકરણો વચ્ચે સીધું કનેક્શન
\item
  \textbf{ક્રોસ્ડ પેર્સ}: ટ્રાન્સમિટ અને રિસીવ પેર્સ અદલાબદલી
\item
  \textbf{ઉપયોગ}: PC થી PC, Switch થી Switch કનેક્શન્સ
\end{itemize}

\end{solutionbox}
\begin{mnemonicbox}
``Cross કમ્પ્યુટર્સને કનેક્ટ કરે''

\end{mnemonicbox}
\subsection*{પ્રશ્ન 4(બ) [4
ગુણ]}\label{uxaaauxab0uxab6uxaa8-4uxaac-4-uxa97uxaa3}

\textbf{IPv4 અને IPv6 વચ્ચેનો તફાવત જણાવો.}

\begin{solutionbox}

{\def\LTcaptype{none} % do not increment counter
\begin{longtable}[]{@{}lll@{}}
\toprule\noalign{}
\textbf{લક્ષણ} & \textbf{IPv4} & \textbf{IPv6} \\
\midrule\noalign{}
\endhead
\bottomrule\noalign{}
\endlastfoot
\textbf{એડ્રેસ સાઇઝ} & 32 બિટ્સ & 128 બિટ્સ \\
\textbf{એડ્રેસ ફોર્મેટ} & ડોટેડ ડેસિમલ & હેક્સાડેસિમલ કોલોન \\
\textbf{એડ્રેસ સ્પેસ} & 4.3 બિલિયન એડ્રેસ & 340 અનડેસિલિયન એડ્રેસ \\
\textbf{હેડર સાઇઝ} & વેરિયેબલ (20-60 બાઇટ્સ) & ફિક્સ્ડ (40 બાઇટ્સ) \\
\end{longtable}
}

\textbf{મુખ્ય તફાવતો:}

\begin{itemize}
\tightlist
\item
  \textbf{IPv4 ઉદાહરણ}: 192.168.1.1
\item
  \textbf{IPv6 ઉદાહરણ}: 2001:0db8:85a3:0000:0000:8a2e:0370:7334
\item
  \textbf{સુરક્ષા}: IPv6 માં બિલ્ટ-ઇન IPSec સપોર્ટ
\item
  \textbf{NAT}: IPv4 ને NAT જરૂરી, IPv6 જરૂરિયાત દૂર કરે
\end{itemize}

\end{solutionbox}
\begin{mnemonicbox}
``IPv4 ચાર-બિલિયન, IPv6 છ-ગણાં-વધારે''

\end{mnemonicbox}
\subsection*{પ્રશ્ન 4(ક) [7
ગુણ]}\label{uxaaauxab0uxab6uxaa8-4uxa95-7-uxa97uxaa3}

\textbf{OSI મોડલની સુઘડ અને સ્વચ્છ આકૃતિ દોરો અને Physical Layer અને Data Link
Layer ની કાર્યક્ષમતા લખો.}

\begin{solutionbox}

\textbf{OSI મોડલ ડાયાગ્રામ:}

\begin{center}
\textbf{Mermaid Diagram (Code)}
\begin{verbatim}
{Shaded}
{Highlighting}[]
graph LR
    A[Application Layer {- 7] {-}{-}{} B[Presentation Layer {-} 6]}
    B {-{-}{} C[Session Layer {-} 5]}
    C {-{-}{} D[Transport Layer {-} 4]}
    D {-{-}{} E[Network Layer {-} 3]}
    E {-{-}{} F[Data Link Layer {-} 2]}
    F {-{-}{} G[Physical Layer {-} 1]}
{Highlighting}
{Shaded}
\end{verbatim}
\end{center}

\textbf{લેયર કાર્યો:}

{\def\LTcaptype{none} % do not increment counter
\begin{longtable}[]{@{}
  >{\raggedright\arraybackslash}p{(\linewidth - 4\tabcolsep) * \real{0.2821}}
  >{\raggedright\arraybackslash}p{(\linewidth - 4\tabcolsep) * \real{0.3590}}
  >{\raggedright\arraybackslash}p{(\linewidth - 4\tabcolsep) * \real{0.3590}}@{}}
\toprule\noalign{}
\begin{minipage}[b]{\linewidth}\raggedright
\textbf{લેયર}
\end{minipage} & \begin{minipage}[b]{\linewidth}\raggedright
\textbf{કાર્ય}
\end{minipage} & \begin{minipage}[b]{\linewidth}\raggedright
\textbf{ઉદાહરણ}
\end{minipage} \\
\midrule\noalign{}
\endhead
\bottomrule\noalign{}
\endlastfoot
\textbf{Physical (Layer 1)} & માધ્યમ પર બિટ ટ્રાન્સમિશન & કેબલ્સ, હબ્સ,
રિપીટર્સ \\
\textbf{Data Link (Layer 2)} & નજીકના નોડ્સ વચ્ચે ફ્રેમ ડિલિવરી & સ્વિચ, MAC
એડ્રેસ \\
\end{longtable}
}

\textbf{Physical Layer કાર્યો:}

\begin{itemize}
\tightlist
\item
  \textbf{બિટ ટ્રાન્સમિશન}: ડેટાને ઇલેક્ટ્રિકલ/ઑપ્ટિકલ સિગ્નલમાં રૂપાંતરિત કરે
\item
  \textbf{માધ્યમ સ્પેસિફિકેશન}: કેબલ પ્રકારો અને કનેક્ટર્સ વ્યાખ્યાયિત કરે
\item
  \textbf{સિગ્નલ એન્કોડિંગ}: બિટ્સ કેવી રીતે રજૂ કરવા નક્કી કરે
\item
  \textbf{ટ્રાન્સમિશન રેટ}: ડેટા સ્પીડ નિયંત્રિત કરે
\end{itemize}

\textbf{Data Link Layer કાર્યો:}

\begin{itemize}
\tightlist
\item
  \textbf{ફ્રેમ ફોર્મેશન}: બિટ્સને ફ્રેમ્સમાં ગોઠવે
\item
  \textbf{એરર ડિટેક્શન}: ટ્રાન્સમિશન એરર્સ ઓળખે
\item
  \textbf{ફ્લો કંટ્રોલ}: ડેટા ટ્રાન્સમિશન રેટ મેનેજ કરે
\item
  \textbf{MAC એડ્રેસિંગ}: લોકલ ડિલિવરી માટે હાર્ડવેર એડ્રેસ ઉપયોગ કરે
\end{itemize}

\end{solutionbox}
\begin{mnemonicbox}
``Physical ધકેલે, Data-Link પહોંચાડે''

\end{mnemonicbox}
\subsection*{પ્રશ્ન 4(અ OR) [3
ગુણ]}\label{uxaaauxab0uxab6uxaa8-4uxa85-or-3-uxa97uxaa3}

\textbf{Time Division Multiplexing સમજાવો.}

\begin{solutionbox}

\textbf{Time Division Multiplexing (TDM):}

\begin{verbatim}
gantt
    title TDM Time Slots
    dateFormat X
    axisFormat \%L
    
    section Channel A
    Data A1 :0, 100
    Data A2 :300, 400
    
    section Channel B
    Data B1 :100, 200
    Data B2 :400, 500
    
    section Channel C
    Data C1 :200, 300
    Data C2 :500, 600
\end{verbatim}

\textbf{TDM લાક્ષણિકતાઓ:}

\begin{itemize}
\tightlist
\item
  \textbf{ટાઇમ સ્લોટ્સ}: દરેક ચેનલને સમર્પિત સમય અવધિ મળે
\item
  \textbf{સિંક્રોનાઇઝેશન}: બધી ચેનલો સિંક્રોનાઇઝ હોવી જોઈએ
\item
  \textbf{બેન્ડવિડ્થ શેરિંગ}: બહુવિધ ચેનલો વચ્ચે સિંગલ હાઇ-સ્પીડ લિંક શેર
\end{itemize}

\end{solutionbox}
\begin{mnemonicbox}
``ટાઇમ વળતા લે''

\end{mnemonicbox}
\subsection*{પ્રશ્ન 4(બ OR) [4
ગુણ]}\label{uxaaauxab0uxab6uxaa8-4uxaac-or-4-uxa97uxaa3}

\textbf{નેટવર્કિંગ ઉપકરણના પ્રકારોની યાદી બનાવો અને કોઈપણ એક સમજાવો.}

\begin{solutionbox}

\textbf{નેટવર્કિંગ ઉપકરણો:}

{\def\LTcaptype{none} % do not increment counter
\begin{longtable}[]{@{}lll@{}}
\toprule\noalign{}
\textbf{ઉપકરણ} & \textbf{લેયર} & \textbf{કાર્ય} \\
\midrule\noalign{}
\endhead
\bottomrule\noalign{}
\endlastfoot
\textbf{Hub} & Physical & સિગ્નલ રિપીટર \\
\textbf{Switch} & Data Link & ફ્રેમ સ્વિચિંગ \\
\textbf{Router} & Network & પેકેટ રાઉટિંગ \\
\textbf{Bridge} & Data Link & નેટવર્ક સેગમેન્ટેશન \\
\end{longtable}
}

\textbf{Switch સમજૂતી:}

\begin{itemize}
\tightlist
\item
  \textbf{કાર્ય}: MAC એડ્રેસ આધારે ફ્રેમ્સ ફોરવર્ડ કરે
\item
  \textbf{લર્નિંગ}: MAC એડ્રેસ ટેબલ ડાયનેમિકલી બનાવે
\item
  \textbf{કોલિઝન ડોમેન}: દરેક પોર્ટ અલગ કોલિઝન ડોમેન બનાવે
\item
  \textbf{ફુલ-ડુપ્લેક્સ}: દરેક પોર્ટ પર એકસાથે મોકલી/મેળવી શકે
\end{itemize}

\textbf{ફાયદા:}

\begin{itemize}
\tightlist
\item
  \textbf{બેન્ડવિડ્થ}: દરેક પોર્ટ માટે સંપૂર્ણ બેન્ડવિડ્થ
\item
  \textbf{સુરક્ષા}: ફ્રેમ્સ માત્ર ઇચ્છિત પ્રાપ્તકર્તાને મોકલાય
\item
  \textbf{કોલિઝન}: કોલિઝન દૂર કરે
\end{itemize}

\end{solutionbox}
\begin{mnemonicbox}
``Switch સ્માર્ટલી મોકલે''

\end{mnemonicbox}
\subsection*{પ્રશ્ન 4(ક OR) [7
ગુણ]}\label{uxaaauxab0uxab6uxaa8-4uxa95-or-7-uxa97uxaa3}

\textbf{Computer Network શું છે? Computer Network ના પ્રકારો સમજાવો.}

\begin{solutionbox}

\textbf{Computer Network વ્યાખ્યા:} આંતરસંબંધિત સ્વતંત્ર કમ્પ્યુટર્સનો સંગ્રહ કે જે
કમ્યુનિકેટ કરી શકે અને સંસાધનો શેર કરી શકે.

\textbf{Computer Networks ના પ્રકારો:}

{\def\LTcaptype{none} % do not increment counter
\begin{longtable}[]{@{}
  >{\raggedright\arraybackslash}p{(\linewidth - 6\tabcolsep) * \real{0.1754}}
  >{\raggedright\arraybackslash}p{(\linewidth - 6\tabcolsep) * \real{0.2456}}
  >{\raggedright\arraybackslash}p{(\linewidth - 6\tabcolsep) * \real{0.2456}}
  >{\raggedright\arraybackslash}p{(\linewidth - 6\tabcolsep) * \real{0.3333}}@{}}
\toprule\noalign{}
\begin{minipage}[b]{\linewidth}\raggedright
\textbf{પ્રકાર}
\end{minipage} & \begin{minipage}[b]{\linewidth}\raggedright
\textbf{કવરેજ}
\end{minipage} & \begin{minipage}[b]{\linewidth}\raggedright
\textbf{ઉદાહરણ}
\end{minipage} & \begin{minipage}[b]{\linewidth}\raggedright
\textbf{લાક્ષણિકતાઓ}
\end{minipage} \\
\midrule\noalign{}
\endhead
\bottomrule\noalign{}
\endlastfoot
\textbf{LAN} & લોકલ એરિયા (બિલ્ડિંગ) & ઑફિસ નેટવર્ક & હાઇ સ્પીડ, લો કોસ્ટ \\
\textbf{MAN} & મેટ્રોપોલિટન એરિયા (શહેર) & શહેરવ્યાપી નેટવર્ક & મીડિયમ સ્પીડ,
મોડરેટ કોસ્ટ \\
\textbf{WAN} & વાઇડ એરિયા (દેશ/વિશ્વ) & ઇન્ટરનેટ & ઓછી સ્પીડ, વધારે કિંમત \\
\end{longtable}
}

\textbf{વિગતવાર સમજૂતી:}

\textbf{1. Local Area Network (LAN):}

\begin{itemize}
\tightlist
\item
  \textbf{કવરેજ}: સિંગલ બિલ્ડિંગ કે કેમ્પસ
\item
  \textbf{સ્પીડ}: હાઇ (100 Mbps થી 10 Gbps)
\item
  \textbf{ટેકનોલોજી}: Ethernet, Wi-Fi
\item
  \textbf{માલિકી}: સિંગલ સંસ્થા
\end{itemize}

\textbf{2. Metropolitan Area Network (MAN):}

\begin{itemize}
\tightlist
\item
  \textbf{કવરેજ}: શહેર કે મેટ્રોપોલિટન એરિયા
\item
  \textbf{સ્પીડ}: મીડિયમ (10-100 Mbps)
\item
  \textbf{ટેકનોલોજી}: ફાઇબર ઑપ્ટિક, માઇક્રોવેવ
\item
  \textbf{ઉદાહરણ}: કેબલ TV નેટવર્ક્સ
\end{itemize}

\textbf{3. Wide Area Network (WAN):}

\begin{itemize}
\tightlist
\item
  \textbf{કવરેજ}: દેશો કે ખંડો
\item
  \textbf{સ્પીડ}: વેરિયેબલ (ટેકનોલોજી પર આધાર)
\item
  \textbf{ટેકનોલોજી}: સેટેલાઇટ, લીઝ્ડ લાઇન્સ
\item
  \textbf{ઉદાહરણ}: ઇન્ટરનેટ, કોર્પોરેટ નેટવર્ક્સ
\end{itemize}

\textbf{નેટવર્ક ફાયદા:}

\begin{itemize}
\tightlist
\item
  \textbf{સંસાધન શેરિંગ}: ફાઇલો, પ્રિન્ટર્સ, એપ્લિકેશન્સ
\item
  \textbf{કમ્યુનિકેશન}: ઇમેઇલ, મેસેજિંગ, વિડિયો કોન્ફરન્સિંગ
\item
  \textbf{કિંમત ઘટાડો}: શેર કરેલ સંસાધનો કિંમત ઘટાડે
\item
  \textbf{ડેટા બેકઅપ}: કેન્દ્રીકૃત બેકઅપ સિસ્ટમ્સ
\end{itemize}

\end{solutionbox}
\begin{mnemonicbox}
``લોકલ પ્રેમ કરે, મેટ્રો મેનેજ કરે, વાઇડ ભટકે''

\end{mnemonicbox}
\subsection*{પ્રશ્ન 5(અ) [3
ગુણ]}\label{uxaaauxab0uxab6uxaa8-5uxa85-3-uxa97uxaa3}

\textbf{Information security ની જરૂરિયાત સમજાવો.}

\begin{solutionbox}

\textbf{માહિતી સુરક્ષાની જરૂરિયાતો:}

{\def\LTcaptype{none} % do not increment counter
\begin{longtable}[]{@{}lll@{}}
\toprule\noalign{}
\textbf{ધમકી} & \textbf{અસર} & \textbf{સુરક્ષા જરૂર} \\
\midrule\noalign{}
\endhead
\bottomrule\noalign{}
\endlastfoot
\textbf{ડેટા ચોરી} & આર્થિક નુકસાન & ગોપનીયતા \\
\textbf{અનધિકૃત પ્રવેશ} & ગોપનીયતા ભંગ & પ્રવેશ નિયંત્રણ \\
\textbf{સિસ્ટમ હુમલા} & સેવા વિક્ષેપ & ઉપલબ્ધતા \\
\end{longtable}
}

\textbf{મુખ્ય આવશ્યકતાઓ:}

\begin{itemize}
\tightlist
\item
  \textbf{ગોપનીયતા}: અનધિકૃત પ્રવેશથી સંવેદનશીલ માહિતીનું રક્ષણ
\item
  \textbf{ડેટા સુરક્ષા}: મૂલ્યવાન ડેટાના નુકસાન કે દૂષિતતા અટકાવવું
\item
  \textbf{બિઝનેસ કન્ટિન્યુઇટી}: સિસ્ટમ્સ ચાલુ રહેવાની ખાતરી
\end{itemize}

\end{solutionbox}
\begin{mnemonicbox}
``સુરક્ષા સંવેદનશીલ સિસ્ટમ્સ બચાવે''

\end{mnemonicbox}
\subsection*{પ્રશ્ન 5(બ) [4
ગુણ]}\label{uxaaauxab0uxab6uxaa8-5uxaac-4-uxa97uxaa3}

\textbf{Fiber Optic Cable ના ફાયદા અને ગેરફાયદા લખો.}

\begin{solutionbox}

{\def\LTcaptype{none} % do not increment counter
\begin{longtable}[]{@{}ll@{}}
\toprule\noalign{}
\textbf{ફાયદા} & \textbf{ગેરફાયદા} \\
\midrule\noalign{}
\endhead
\bottomrule\noalign{}
\endlastfoot
\textbf{વધારે બેન્ડવિડ્થ} & \textbf{વધારે કિંમત} \\
\textbf{EMI થી મુક્તિ} & \textbf{મુશ્કેલ ઇન્સ્ટોલેશન} \\
\textbf{લાંબું અંતર} & \textbf{નાજુક પ્રકૃતિ} \\
\textbf{સુરક્ષિત ટ્રાન્સમિશન} & \textbf{વિશેષ સાધનો} \\
\end{longtable}
}

\textbf{ફાયદા:}

\begin{itemize}
\tightlist
\item
  \textbf{સ્પીડ}: સૌથી વધારે ડેટા ટ્રાન્સમિશન રેટ
\item
  \textbf{અંતર}: સિગ્નલ ડિગ્રેડેશન વિના લાંબા અંતર સુધી જઈ શકે
\item
  \textbf{સુરક્ષા}: ટેપ કરવું મુશ્કેલ, સુરક્ષિત કમ્યુનિકેશન આપે
\end{itemize}

\textbf{ગેરફાયદા:}

\begin{itemize}
\tightlist
\item
  \textbf{કિંમત}: મોંઘા કેબલ અને સાધનો
\item
  \textbf{ઇન્સ્ટોલેશન}: કુશળ ટેકનિશિયન જરૂરી
\item
  \textbf{જાળવણી}: રિપેર અને સ્પ્લાઇસ કરવું મુશ્કેલ
\end{itemize}

\end{solutionbox}
\begin{mnemonicbox}
``ફાઇબર ફાસ્ટ પણ નાજુક''

\end{mnemonicbox}
\subsection*{પ્રશ્ન 5(ક) [7
ગુણ]}\label{uxaaauxab0uxab6uxaa8-5uxa95-7-uxa97uxaa3}

\textbf{Attack ના પ્રકારોની યાદી બનાવો. અને કોઈપણ બે Web આધારિત Attack ને
સમજાવો.}

\begin{solutionbox}

\textbf{હુમલાના પ્રકારો:}

{\def\LTcaptype{none} % do not increment counter
\begin{longtable}[]{@{}lll@{}}
\toprule\noalign{}
\textbf{કેટેગરી} & \textbf{હુમલાના પ્રકારો} & \textbf{લક્ષ્ય} \\
\midrule\noalign{}
\endhead
\bottomrule\noalign{}
\endlastfoot
\textbf{વેબ-આધારિત} & SQL Injection, XSS, CSRF & વેબ એપ્લિકેશન્સ \\
\textbf{નેટવર્ક} & DoS, DDoS, Man-in-Middle & નેટવર્ક ઇન્ફ્રાસ્ટ્રક્ચર \\
\textbf{મેલવેર} & વાઇરસ, ટ્રોજન, રેન્સમવેર & સિસ્ટમ્સ અને ડેટા \\
\textbf{સામાજિક} & ફિશિંગ, સોશિયલ એન્જિનિયરિંગ & માનવ યુઝર્સ \\
\end{longtable}
}

\textbf{વેબ-આધારિત હુમલાઓ સમજાવ્યા:}

\textbf{1. SQL Injection:}

\begin{itemize}
\tightlist
\item
  \textbf{પદ્ધતિ}: વેબ એપ્લિકેશન ઇનપુટ્સમાં દુર્ભાવનાપૂર્ણ SQL કોડ દાખલ કરવો
\item
  \textbf{અસર}: અનધિકૃત ડેટાબેસ ઍક્સેસ, ડેટા ચોરી
\item
  \textbf{ઉદાહરણ}: લોગિન ફોર્મમાં
  \texttt{\textquotesingle{};\ DROP\ TABLE\ users;-\/-} દાખલ કરવું
\item
  \textbf{અટકાવવાનો ઉપાય}: ઇનપુટ વેલિડેશન, પેરામીટરાઇઝ્ડ ક્વેરીઝ
\item
  \textbf{ગંભીરતા}: સંપૂર્ણ ડેટાબેસ કમ્પ્રોમાઇઝ કરી શકે
\end{itemize}

\textbf{2. Cross-Site Scripting (XSS):}

\begin{itemize}
\tightlist
\item
  \textbf{પદ્ધતિ}: વેબ પેજીસમાં દુર્ભાવનાપૂર્ણ સ્ક્રિપ્ટ્સ ઇન્જેક્ટ કરવી
\item
  \textbf{અસર}: સેશન હાઇજેકિંગ, કૂકી ચોરી, પેજ ડિફેસમેન્ટ
\item
  \textbf{પ્રકારો}: Stored XSS, Reflected XSS, DOM-based XSS
\item
  \textbf{અટકાવવાનો ઉપાય}: ઇનપુટ સેનિટાઇઝેશન, આઉટપુટ એન્કોડિંગ
\item
  \textbf{લક્ષ્ય}: કમ્પ્રોમાઇઝ્ડ વેબસાઇટ્સ મુલાકાત લેતા યુઝર્સને અસર કરે
\end{itemize}

\textbf{હુમલાની લાક્ષણિકતાઓ:}

\begin{itemize}
\tightlist
\item
  \textbf{SQL Injection}: વેબ એપ્લિકેશન દ્વારા ડેટાબેસને લક્ષ્ય બનાવે
\item
  \textbf{XSS}: કમ્પ્રોમાઇઝ્ડ વેબ પેજીસ દ્વારા યુઝર્સને લક્ષ્ય બનાવે
\item
  \textbf{સામાન્ય પરિબળ}: બંને અપૂરતા ઇનપુટ વેલિડેશનનો લાભ લે
\end{itemize}

\textbf{અટકાવવાના ઉપાયો:}

\begin{itemize}
\tightlist
\item
  \textbf{ઇનપુટ વેલિડેશન}: બધા યુઝર ઇનપુટ્સ ચકાસો
\item
  \textbf{નિયમિત અપડેટ્સ}: સોફ્ટવેર અને સિસ્ટમ્સ અપડેટ રાખો
\item
  \textbf{સુરક્ષા પ્રશિક્ષણ}: યુઝર્સને હુમલાની પદ્ધતિઓ શીખવો
\end{itemize}

\end{solutionbox}
\begin{mnemonicbox}
``SQL ચોરે, XSS સ્ક્રિપ્ટ્સ એક્સપ્લોઇટ કરે''

\end{mnemonicbox}
\subsection*{પ્રશ્ન 5(અ OR) [3
ગુણ]}\label{uxaaauxab0uxab6uxaa8-5uxa85-or-3-uxa97uxaa3}

\textbf{Confidentiality, Integrity અને Availability સમજાવો.}

\begin{solutionbox}

\textbf{CIA ત્રિકોણના ઘટકો:}

{\def\LTcaptype{none} % do not increment counter
\begin{longtable}[]{@{}
  >{\raggedright\arraybackslash}p{(\linewidth - 4\tabcolsep) * \real{0.3333}}
  >{\raggedright\arraybackslash}p{(\linewidth - 4\tabcolsep) * \real{0.3556}}
  >{\raggedright\arraybackslash}p{(\linewidth - 4\tabcolsep) * \real{0.3111}}@{}}
\toprule\noalign{}
\begin{minipage}[b]{\linewidth}\raggedright
\textbf{ઘટક}
\end{minipage} & \begin{minipage}[b]{\linewidth}\raggedright
\textbf{વ્યાખ્યા}
\end{minipage} & \begin{minipage}[b]{\linewidth}\raggedright
\textbf{ઉદાહરણ}
\end{minipage} \\
\midrule\noalign{}
\endhead
\bottomrule\noalign{}
\endlastfoot
\textbf{Confidentiality} & માત્ર અધિકૃત યુઝર્સ દ્વારા માહિતીની પ્રવેશ &
એન્ક્રિપ્શન, ઍક્સેસ કંટ્રોલ્સ \\
\textbf{Integrity} & ડેટાની સચોટતા અને સંપૂર્ણતા & ચેકસમ્સ, ડિજિટલ સિગ્નેચર્સ \\
\textbf{Availability} & જરૂર પડે ત્યારે સિસ્ટમ્સ ઍક્સેસિબલ & રીડન્ડન્સી, બેકઅપ
સિસ્ટમ્સ \\
\end{longtable}
}

\textbf{મુખ્ય ખ્યાલો:}

\begin{itemize}
\tightlist
\item
  \textbf{Confidentiality}: અનધિકૃત યુઝર્સથી માહિતી ગુપ્ત રાખે
\item
  \textbf{Integrity}: ડેટા અનધિકૃત રીતે સુધારાયો નથી તેની ખાતરી કરે
\item
  \textbf{Availability}: જરૂર પડે ત્યારે સિસ્ટમ્સ ચાલુ હોવાની ગેરંટી આપે
\end{itemize}

\end{solutionbox}
\begin{mnemonicbox}
``CIA સંપૂર્ણપણે માહિતીનું રક્ષણ કરે''

\end{mnemonicbox}
\subsection*{પ્રશ્ન 5(બ OR) [4
ગુણ]}\label{uxaaauxab0uxab6uxaa8-5uxaac-or-4-uxa97uxaa3}

\textbf{નીચેના IP સરનામાઓનો Class શોધો.}

\begin{solutionbox}

\textbf{IP એડ્રેસ Class ઓળખ:}

{\def\LTcaptype{none} % do not increment counter
\begin{longtable}[]{@{}llll@{}}
\toprule\noalign{}
\textbf{IP એડ્રેસ} & \textbf{પ્રથમ ઓક્ટેટ} & \textbf{ક્લાસ} & \textbf{રેન્જ} \\
\midrule\noalign{}
\endhead
\bottomrule\noalign{}
\endlastfoot
\textbf{192.12.44.12} & 192 & Class C & 192-223 \\
\textbf{123.77.42.213} & 123 & Class A & 1-126 \\
\textbf{190.65.22.15} & 190 & Class B & 128-191 \\
\textbf{10.0.0.11} & 10 & Class A (Private) & 1-126 \\
\end{longtable}
}

\textbf{ક્લાસ લાક્ષણિકતાઓ:}

\begin{itemize}
\tightlist
\item
  \textbf{Class A}: 1-126 (પ્રથમ બિટ 0), મોટા નેટવર્ક્સને સપોર્ટ કરે
\item
  \textbf{Class B}: 128-191 (પ્રથમ બે બિટ્સ 10), મધ્યમ નેટવર્ક્સ
\item
  \textbf{Class C}: 192-223 (પ્રથમ ત્રણ બિટ્સ 110), નાના નેટવર્ક્સ
\item
  \textbf{Private IPs}: 10.x.x.x, 172.16-31.x.x, 192.168.x.x
\end{itemize}

\end{solutionbox}
\begin{mnemonicbox}
``A ઓસમ, B બેટર, C કોમ્પેક્ટ''

\end{mnemonicbox}
\subsection*{પ્રશ્ન 5(ક OR) [7
ગુણ]}\label{uxaaauxab0uxab6uxaa8-5uxa95-or-7-uxa97uxaa3}

\textbf{Cryptography સમજાવો.}

\begin{solutionbox}

\textbf{Cryptography વ્યાખ્યા:} માત્ર અધિકૃત પક્ષકારો જ ઍક્સેસ કરી શકે તે રીતે
માહિતીને એન્કોડ કરીને કમ્યુનિકેશન સુરક્ષિત કરવાનું વિજ્ઞાન.

\textbf{Cryptography પ્રકારો:}

{\def\LTcaptype{none} % do not increment counter
\begin{longtable}[]{@{}
  >{\raggedright\arraybackslash}p{(\linewidth - 6\tabcolsep) * \real{0.1754}}
  >{\raggedright\arraybackslash}p{(\linewidth - 6\tabcolsep) * \real{0.2632}}
  >{\raggedright\arraybackslash}p{(\linewidth - 6\tabcolsep) * \real{0.2456}}
  >{\raggedright\arraybackslash}p{(\linewidth - 6\tabcolsep) * \real{0.3158}}@{}}
\toprule\noalign{}
\begin{minipage}[b]{\linewidth}\raggedright
\textbf{પ્રકાર}
\end{minipage} & \begin{minipage}[b]{\linewidth}\raggedright
\textbf{કી ઉપયોગ}
\end{minipage} & \begin{minipage}[b]{\linewidth}\raggedright
\textbf{ઉદાહરણ}
\end{minipage} & \begin{minipage}[b]{\linewidth}\raggedright
\textbf{ઉપયોગ}
\end{minipage} \\
\midrule\noalign{}
\endhead
\bottomrule\noalign{}
\endlastfoot
\textbf{Symmetric} & સિંગલ શેર્ડ કી & DES, AES & ઝડપી બલ્ક એન્ક્રિપ્શન \\
\textbf{Asymmetric} & પબ્લિક-પ્રાઇવેટ કી જોડી & RSA, ECC & ડિજિટલ સિગ્નેચર્સ,
કી એક્સચેન્જ \\
\textbf{Hash Functions} & એક-દિશીય રૂપાંતરણ & MD5, SHA & ડેટા ઇન્ટેગ્રિટી,
પાસવર્ડ્સ \\
\end{longtable}
}

\textbf{મુખ્ય ખ્યાલો:}

\textbf{1. Symmetric Cryptography:}

\begin{itemize}
\tightlist
\item
  \textbf{સિંગલ કી}: એન્ક્રિપ્શન અને ડિક્રિપ્શન માટે સમાન કી
\item
  \textbf{સ્પીડ}: મોટા ડેટા માટે ઝડપી પ્રોસેસિંગ
\item
  \textbf{પડકાર}: સુરક્ષિત કી વિતરણ
\item
  \textbf{ઉદાહરણ}: AES-256, 3DES
\end{itemize}

\textbf{2. Asymmetric Cryptography:}

\begin{itemize}
\tightlist
\item
  \textbf{કી જોડી}: પબ્લિક કી (શેર કરી શકાય) અને પ્રાઇવેટ કી (ગુપ્ત)
\item
  \textbf{ડિજિટલ સિગ્નેચર્સ}: પ્રામાણિકતા અને બિન-ઇનકાર સાબિત કરે
\item
  \textbf{કી એક્સચેન્જ}: સિમેટ્રિક કીઝ શેર કરવાની સુરક્ષિત પદ્ધતિ
\item
  \textbf{ઉદાહરણ}: RSA, Elliptic Curve Cryptography
\end{itemize}

\textbf{3. Hash Functions:}

\begin{itemize}
\tightlist
\item
  \textbf{એક-દિશીય}: હેશ ગણતરી કરવી સરળ, ઉલટાવવી મુશ્કેલ
\item
  \textbf{નિશ્ચિત આઉટપુટ}: હંમેશા સમાન લંબાઇનું આઉટપુટ આપે
\item
  \textbf{કોલિઝન પ્રતિકાર}: અલગ ઇનપુટ્સ અલગ હેશ આપવા જોઈએ
\item
  \textbf{ઉપયોગ}: પાસવર્ડ સંગ્રહ, ડિજિટલ ફોરેન્સિક્સ
\end{itemize}

\textbf{ક્રિપ્ટોગ્રાફિક પ્રક્રિયા:}

\begin{center}
\textbf{Mermaid Diagram (Code)}
\begin{verbatim}
{Shaded}
{Highlighting}[]
graph LR
    A[Plaintext] {-{-}{} B[Encryption Algorithm]}
    C[Key] {-{-}{} B}
    B {-{-}{} D[Ciphertext]}
    D {-{-}{} E[Decryption Algorithm]}
    C {-{-}{} E}
    E {-{-}{} F[Plaintext]}
{Highlighting}
{Shaded}
\end{verbatim}
\end{center}

\textbf{ઉપયોગ:}

\begin{itemize}
\tightlist
\item
  \textbf{સુરક્ષિત કમ્યુનિકેશન}: HTTPS, VPN, ઇમેઇલ એન્ક્રિપ્શન
\item
  \textbf{ડેટા સુરક્ષા}: ફાઇલ એન્ક્રિપ્શન, ડેટાબેસ સિક્યોરિટી
\item
  \textbf{ઓથેન્ટિકેશન}: ડિજિટલ સર્ટિફિકેટ્સ, પાસવર્ડ હેશિંગ
\item
  \textbf{નાણાકીય સિસ્ટમ્સ}: ઑનલાઇન બેન્કિંગ, ક્રિપ્ટોકરન્સી
\end{itemize}

\textbf{આધુનિક પડકારો:}

\begin{itemize}
\tightlist
\item
  \textbf{ક્વાન્ટમ કમ્પ્યુટિંગ}: વર્તમાન એન્ક્રિપ્શન પદ્ધતિઓ માટે ધમકી
\item
  \textbf{કી મેનેજમેન્ટ}: કીઝનો સુરક્ષિત સંગ્રહ અને વિતરણ
\item
  \textbf{પર્ફોર્મન્સ}: સિસ્ટમ પર્ફોર્મન્સ સાથે સુરક્ષાનું સંતુલન
\end{itemize}

\end{solutionbox}
\begin{mnemonicbox}
``Cryptography કોડેડ કમ્યુનિકેશન્સ બનાવે''

\end{mnemonicbox}

\end{document}
