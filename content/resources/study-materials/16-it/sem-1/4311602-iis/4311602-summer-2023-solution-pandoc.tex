\documentclass[10pt,a4paper]{article}

% content/resources/templates/preamble.tex
\usepackage[margin=0.6in]{geometry}
\author{Milav Dabgar}
\usepackage{amsmath,amssymb,amsthm}
\usepackage{booktabs}
\usepackage{multirow}
\usepackage{xcolor}
\usepackage{tcolorbox}
\tcbuselibrary{breakable,skins}
\usepackage[colorlinks=true,linkcolor=blue]{hyperref}
\usepackage{titlesec}
\usepackage{enumitem}
\usepackage{tikz}
\usepackage{pgfplots}
\usepackage{circuitikz}
\usepackage[version=4]{mhchem}
\usepackage{longtable}
\usepackage{array}
\usepackage{float}
\usepackage{caption}
\usepackage{listings}

\lstset{
  basicstyle=\small\ttfamily,
  breaklines=true,
  breakatwhitespace=false,
  postbreak=\mbox{\textcolor{red}{$\hookrightarrow$}\space},
  float=false,
  numbers=left,
  numberstyle=\tiny\color{gray},
  numbersep=10pt,
  xleftmargin=2em,
  keywordstyle=\color{blue},
  commentstyle=\color{green!60!black},
  stringstyle=\color{purple},
  backgroundcolor=\color{gray!5},
  showstringspaces=false,
  tabsize=2,
  captionpos=b,
  keepspaces=true,
  columns=flexible
}

\pgfplotsset{compat=1.18}
\usetikzlibrary{shapes,arrows,positioning,calc,patterns,decorations.pathmorphing,decorations.markings,arrows.meta}

% Color scheme
\definecolor{headcolor}{RGB}{0,102,204}
\definecolor{keycolor}{RGB}{220,20,60}
\definecolor{solutioncolor}{RGB}{34,139,34}
\definecolor{mnemoniccolor}{RGB}{148,0,211}
\definecolor{codecolor}{RGB}{0,0,100}

% Spacing
\setlength{\parskip}{3pt}
\setlist[itemize]{nosep}
\setlist[enumerate]{nosep}

% Title formatting
\titleformat{\section}{\Large\bfseries\color{headcolor}}{\thesection}{1em}{}
\titleformat{\subsection}{\large\bfseries\color{headcolor}}{\thesubsection}{1em}{}

% Pandoc tightlist compatibility
\providecommand{\tightlist}{%
  \setlength{\itemsep}{0pt}\setlength{\parskip}{0pt}}

% Pandoc longtable compatibility
\newcounter{none}
\def\thenone{}


% content/resources/templates/english-boxes.tex
% This file is currently empty - it exists to maintain consistency with the import structure.
% Add custom environments here if needed in the future.


\begin{document}

\begin{center}
{\Huge\bfseries\color{headcolor} Subject Name Solutions}\\[5pt]
{\LARGE 4311602 -- Summer 2023}\\[3pt]
{\large Semester 1 Study Material}\\[3pt]
{\normalsize\textit{Detailed Solutions and Explanations}}
\end{center}

\vspace{10pt}

\subsection*{Question 1(a) [3 marks]}\label{q1a}

\textbf{Discuss the main components of the Computer.}

\begin{solutionbox}


{\def\LTcaptype{none} % do not increment counter
\vspace{-5pt}
\captionof{table}{Main Components of Computer}
\vspace{-10pt}
\begin{longtable}[]{@{}
  >{\raggedright\arraybackslash}p{(\linewidth - 4\tabcolsep) * \real{0.3667}}
  >{\raggedright\arraybackslash}p{(\linewidth - 4\tabcolsep) * \real{0.3333}}
  >{\raggedright\arraybackslash}p{(\linewidth - 4\tabcolsep) * \real{0.3000}}@{}}
\toprule\noalign{}
\begin{minipage}[b]{\linewidth}\raggedright
Component
\end{minipage} & \begin{minipage}[b]{\linewidth}\raggedright
Function
\end{minipage} & \begin{minipage}[b]{\linewidth}\raggedright
Example
\end{minipage} \\
\midrule\noalign{}
\endhead
\bottomrule\noalign{}
\endlastfoot
\textbf{Input Unit} & Receives data and instructions & Keyboard,
Mouse \\
\textbf{CPU} & Processes data and controls operations & Intel i5, AMD
Ryzen \\
\textbf{Memory} & Stores data temporarily/permanently & RAM, Hard
Disk \\
\textbf{Output Unit} & Displays processed results & Monitor, Printer \\
\end{longtable}
}

\textbf{Key Components:}

\begin{itemize}
\tightlist
\item
  \textbf{Hardware}: Physical parts like CPU, RAM, motherboard
\item
  \textbf{Software}: Programs and operating systems
\item
  \textbf{Data}: Information processed by computer
\end{itemize}

\end{solutionbox}
\begin{mnemonicbox}
``I Can Make Output'' (Input-CPU-Memory-Output)

\end{mnemonicbox}
\subsection*{Question 1(b) [4 marks]}\label{q1b}

\textbf{Explain the web browser and its type.}

\begin{solutionbox}

A \textbf{web browser} is software that accesses and displays web pages
from the internet.


{\def\LTcaptype{none} % do not increment counter
\vspace{-5pt}
\captionof{table}{Types of Web Browsers}
\vspace{-10pt}
\begin{longtable}[]{@{}
  >{\raggedright\arraybackslash}p{(\linewidth - 4\tabcolsep) * \real{0.4118}}
  >{\raggedright\arraybackslash}p{(\linewidth - 4\tabcolsep) * \real{0.2941}}
  >{\raggedright\arraybackslash}p{(\linewidth - 4\tabcolsep) * \real{0.2941}}@{}}
\toprule\noalign{}
\begin{minipage}[b]{\linewidth}\raggedright
Browser Type
\end{minipage} & \begin{minipage}[b]{\linewidth}\raggedright
Features
\end{minipage} & \begin{minipage}[b]{\linewidth}\raggedright
Examples
\end{minipage} \\
\midrule\noalign{}
\endhead
\bottomrule\noalign{}
\endlastfoot
\textbf{Graphical} & GUI interface, multimedia support & Chrome,
Firefox \\
\textbf{Text-based} & Command line, fast loading & Lynx, Links \\
\textbf{Mobile} & Touch interface, optimized for phones & Safari Mobile,
Chrome Mobile \\
\end{longtable}
}

\textbf{Features:}

\begin{itemize}
\tightlist
\item
  \textbf{Navigation}: Forward, back, refresh buttons
\item
  \textbf{Bookmarks}: Save favorite websites
\item
  \textbf{Tabs}: Multiple pages in one window
\item
  \textbf{Security}: HTTPS support, popup blockers
\end{itemize}

\end{solutionbox}
\begin{mnemonicbox}
``Browse Safely Online'' (Bookmarks-Security-Online)

\end{mnemonicbox}
\subsection*{Question 1(c) [7 marks]}\label{q1c}

\textbf{Explain LAN, MAN and WAN with example.}

\begin{solutionbox}


{\def\LTcaptype{none} % do not increment counter
\vspace{-5pt}
\captionof{table}{Network Types Comparison}
\vspace{-10pt}
\begin{longtable}[]{@{}
  >{\raggedright\arraybackslash}p{(\linewidth - 8\tabcolsep) * \real{0.2143}}
  >{\raggedright\arraybackslash}p{(\linewidth - 8\tabcolsep) * \real{0.2381}}
  >{\raggedright\arraybackslash}p{(\linewidth - 8\tabcolsep) * \real{0.1905}}
  >{\raggedright\arraybackslash}p{(\linewidth - 8\tabcolsep) * \real{0.2143}}
  >{\raggedright\arraybackslash}p{(\linewidth - 8\tabcolsep) * \real{0.1429}}@{}}
\toprule\noalign{}
\begin{minipage}[b]{\linewidth}\raggedright
Network
\end{minipage} & \begin{minipage}[b]{\linewidth}\raggedright
Coverage
\end{minipage} & \begin{minipage}[b]{\linewidth}\raggedright
Speed
\end{minipage} & \begin{minipage}[b]{\linewidth}\raggedright
Example
\end{minipage} & \begin{minipage}[b]{\linewidth}\raggedright
Cost
\end{minipage} \\
\midrule\noalign{}
\endhead
\bottomrule\noalign{}
\endlastfoot
\textbf{LAN} & Building/Campus & High (100Mbps-1Gbps) & Office network &
Low \\
\textbf{MAN} & City/Metropolitan & Medium (10-100Mbps) & Cable TV
network & Medium \\
\textbf{WAN} & Country/Global & Variable (1-100Mbps) & Internet &
High \\
\end{longtable}
}

\textbf{Detailed Explanation:}

\textbf{LAN (Local Area Network):}

\begin{itemize}
\tightlist
\item
  \textbf{Coverage}: Within building or small area
\item
  \textbf{Technology}: Ethernet, Wi-Fi
\item
  \textbf{Example}: Computer lab, home network
\end{itemize}

\textbf{MAN (Metropolitan Area Network):}

\begin{itemize}
\tightlist
\item
  \textbf{Coverage}: Across city or metropolitan area
\item
  \textbf{Technology}: Fiber optic, microwave
\item
  \textbf{Example}: City-wide cable internet
\end{itemize}

\textbf{WAN (Wide Area Network):}

\begin{itemize}
\tightlist
\item
  \textbf{Coverage}: Multiple cities/countries
\item
  \textbf{Technology}: Satellite, fiber optic
\item
  \textbf{Example}: Internet, bank ATM networks
\end{itemize}

\textbf{Diagram:}

\begin{center}
\textbf{Mermaid Diagram (Code)}
\begin{verbatim}
{Shaded}
{Highlighting}[]
graph LR
    A[LAN {- Building] {-}{-}{} B[MAN {-} City]}
    B {-{-}{} C[WAN {-} Global]}
    A {-{-}{} D[Office Network]}
    B {-{-}{} E[City Cable TV]}
    C {-{-}{} F[Internet]}
{Highlighting}
{Shaded}
\end{verbatim}
\end{center}

\end{solutionbox}
\begin{mnemonicbox}
``Local Metro World'' (LAN-MAN-WAN)

\end{mnemonicbox}
\subsection*{Question 1(c OR) [7
marks]}\label{question-1c-or-7-marks}

\textbf{Difference between DOS and Unix Operating system.}

\begin{solutionbox}


{\def\LTcaptype{none} % do not increment counter
\vspace{-5pt}
\captionof{table}{DOS vs Unix Comparison}
\vspace{-10pt}
\begin{longtable}[]{@{}lll@{}}
\toprule\noalign{}
Feature & DOS & Unix \\
\midrule\noalign{}
\endhead
\bottomrule\noalign{}
\endlastfoot
\textbf{Interface} & Command Line (text-based) & Command Line + GUI \\
\textbf{Multi-user} & Single user & Multi-user support \\
\textbf{Multitasking} & Limited & Full multitasking \\
\textbf{Security} & Basic & Advanced security \\
\textbf{File System} & FAT16/FAT32 & Various (ext3, ext4) \\
\textbf{Cost} & Commercial (Microsoft) & Free/Open source variants \\
\end{longtable}
}

\textbf{Key Differences:}

\textbf{DOS (Disk Operating System):}

\begin{itemize}
\tightlist
\item
  \textbf{Architecture}: 16-bit, single-user
\item
  \textbf{Memory}: Limited to 640KB conventional
\item
  \textbf{Commands}: DIR, COPY, DEL
\item
  \textbf{File naming}: 8.3 format limitation
\end{itemize}

\textbf{Unix:}

\begin{itemize}
\tightlist
\item
  \textbf{Architecture}: 32/64-bit, multi-user
\item
  \textbf{Memory}: Advanced memory management
\item
  \textbf{Commands}: ls, cp, rm, grep
\item
  \textbf{File naming}: Case-sensitive, long names
\end{itemize}

\textbf{Examples:}

\begin{itemize}
\tightlist
\item
  \textbf{DOS}: MS-DOS, PC-DOS
\item
  \textbf{Unix}: Linux, Solaris, AIX
\end{itemize}

\end{solutionbox}
\begin{mnemonicbox}
``DOS Simple, Unix Powerful'' (Single vs Multi-user)

\end{mnemonicbox}
\subsection*{Question 2(a) [3 marks]}\label{q2a}

\textbf{List out features of operating system.}

\begin{solutionbox}


{\def\LTcaptype{none} % do not increment counter
\vspace{-5pt}
\captionof{table}{Operating System Features}
\vspace{-10pt}
\begin{longtable}[]{@{}ll@{}}
\toprule\noalign{}
Feature & Description \\
\midrule\noalign{}
\endhead
\bottomrule\noalign{}
\endlastfoot
\textbf{Process Management} & Controls program execution \\
\textbf{Memory Management} & Allocates RAM efficiently \\
\textbf{File Management} & Organizes data storage \\
\textbf{Device Management} & Controls hardware devices \\
\end{longtable}
}

\textbf{Core Features:}

\begin{itemize}
\tightlist
\item
  \textbf{User Interface}: GUI or command line
\item
  \textbf{Security}: User authentication, access control
\item
  \textbf{Multitasking}: Run multiple programs simultaneously
\item
  \textbf{Resource Allocation}: CPU, memory distribution
\end{itemize}

\end{solutionbox}
\begin{mnemonicbox}
``Please Manage Files Properly''
(Process-Memory-File-Device)

\end{mnemonicbox}
\subsection*{Question 2(b) [4 marks]}\label{q2b}

\textbf{Define half duplex and full duplex transmission modes.}

\begin{solutionbox}


{\def\LTcaptype{none} % do not increment counter
\vspace{-5pt}
\captionof{table}{Transmission Modes Comparison}
\vspace{-10pt}
\begin{longtable}[]{@{}
  >{\raggedright\arraybackslash}p{(\linewidth - 6\tabcolsep) * \real{0.1579}}
  >{\raggedright\arraybackslash}p{(\linewidth - 6\tabcolsep) * \real{0.2895}}
  >{\raggedright\arraybackslash}p{(\linewidth - 6\tabcolsep) * \real{0.2368}}
  >{\raggedright\arraybackslash}p{(\linewidth - 6\tabcolsep) * \real{0.3158}}@{}}
\toprule\noalign{}
\begin{minipage}[b]{\linewidth}\raggedright
Mode
\end{minipage} & \begin{minipage}[b]{\linewidth}\raggedright
Direction
\end{minipage} & \begin{minipage}[b]{\linewidth}\raggedright
Example
\end{minipage} & \begin{minipage}[b]{\linewidth}\raggedright
Efficiency
\end{minipage} \\
\midrule\noalign{}
\endhead
\bottomrule\noalign{}
\endlastfoot
\textbf{Half Duplex} & Bidirectional (one at a time) & Walkie-talkie &
Medium \\
\textbf{Full Duplex} & Bidirectional (simultaneous) & Telephone &
High \\
\end{longtable}
}

\textbf{Definitions:}

\textbf{Half Duplex:}

\begin{itemize}
\tightlist
\item
  \textbf{Communication}: Two-way but not simultaneous
\item
  \textbf{Example}: Radio communication, old Ethernet hubs
\item
  \textbf{Limitation}: Turn-taking required
\end{itemize}

\textbf{Full Duplex:}

\begin{itemize}
\tightlist
\item
  \textbf{Communication}: Two-way simultaneous
\item
  \textbf{Example}: Modern Ethernet, telephone calls
\item
  \textbf{Advantage}: No waiting time
\end{itemize}

\textbf{Diagram:}

\begin{verbatim}
Half Duplex:
A {-{-}{-}{-}{-} B  (A sends)}
A {{-}{-}{-}{-}{-} B  (B sends {-} A waits)}

Full Duplex:
A {{-}{-}{-}{-} B  (Both send/receive simultaneously)}
\end{verbatim}

\end{solutionbox}
\begin{mnemonicbox}
``Half waits, Full flows'' (Half=waiting,
Full=simultaneous)

\end{mnemonicbox}
\subsection*{Question 2(c) [7 marks]}\label{q2c}

\textbf{Difference between open source and proprietary software.}

\begin{solutionbox}


{\def\LTcaptype{none} % do not increment counter
\vspace{-5pt}
\captionof{table}{Open Source vs Proprietary Software}
\vspace{-10pt}
\begin{longtable}[]{@{}lll@{}}
\toprule\noalign{}
Aspect & Open Source & Proprietary \\
\midrule\noalign{}
\endhead
\bottomrule\noalign{}
\endlastfoot
\textbf{Source Code} & Freely available & Hidden/Protected \\
\textbf{Cost} & Usually free & Paid licenses \\
\textbf{Modification} & Allowed & Restricted \\
\textbf{Support} & Community-based & Vendor support \\
\textbf{Security} & Transparent & Security through obscurity \\
\textbf{Examples} & Linux, Firefox, Apache & Windows, MS Office \\
\end{longtable}
}

\textbf{Detailed Comparison:}

\textbf{Open Source Software:}

\begin{itemize}
\tightlist
\item
  \textbf{Definition}: Source code publicly available
\item
  \textbf{Licensing}: GPL, MIT, Apache licenses
\item
  \textbf{Benefits}: Cost-effective, customizable, transparent
\item
  \textbf{Examples}: LibreOffice, GIMP, MySQL
\end{itemize}

\textbf{Proprietary Software:}

\begin{itemize}
\tightlist
\item
  \textbf{Definition}: Owned by individual/company
\item
  \textbf{Licensing}: End User License Agreement (EULA)
\item
  \textbf{Benefits}: Professional support, guaranteed updates
\item
  \textbf{Examples}: Adobe Photoshop, Oracle Database
\end{itemize}

\textbf{Advantages \& Disadvantages:}

\textbf{Open Source Pros:} Free, flexible, community support
\textbf{Open Source Cons:} Limited professional support

\textbf{Proprietary Pros:} Professional support, warranty
\textbf{Proprietary Cons:} Expensive, vendor lock-in

\end{solutionbox}
\begin{mnemonicbox}
``Open = Free to See, Proprietary = Pay to Use''

\end{mnemonicbox}
\subsection*{Question 2(a OR) [3
marks]}\label{question-2a-or-3-marks}

\textbf{Differentiate between RAM and ROM.}

\begin{solutionbox}


{\def\LTcaptype{none} % do not increment counter
\vspace{-5pt}
\captionof{table}{RAM vs ROM Comparison}
\vspace{-10pt}
\begin{longtable}[]{@{}lll@{}}
\toprule\noalign{}
Feature & RAM & ROM \\
\midrule\noalign{}
\endhead
\bottomrule\noalign{}
\endlastfoot
\textbf{Full Form} & Random Access Memory & Read Only Memory \\
\textbf{Volatility} & Volatile (loses data) & Non-volatile (retains
data) \\
\textbf{Access} & Read/Write & Read only \\
\textbf{Speed} & Very fast & Slower than RAM \\
\end{longtable}
}

\textbf{Key Differences:}

\begin{itemize}
\tightlist
\item
  \textbf{Purpose}: RAM for temporary storage, ROM for permanent
\item
  \textbf{Cost}: RAM more expensive per GB
\item
  \textbf{Usage}: RAM for programs, ROM for firmware
\end{itemize}

\end{solutionbox}
\begin{mnemonicbox}
``RAM Runs, ROM Remembers'' (temporary vs permanent)

\end{mnemonicbox}
\subsection*{Question 2(b OR) [4
marks]}\label{question-2b-or-4-marks}

\textbf{Explain AND logic gate with Example.}

\begin{solutionbox}

\textbf{AND Gate Definition:} Output is HIGH only when ALL inputs are
HIGH.

\textbf{Truth Table:}

{\def\LTcaptype{none} % do not increment counter
\begin{longtable}[]{@{}lll@{}}
\toprule\noalign{}
Input A & Input B & Output (A AND B) \\
\midrule\noalign{}
\endhead
\bottomrule\noalign{}
\endlastfoot
0 & 0 & 0 \\
0 & 1 & 0 \\
1 & 0 & 0 \\
1 & 1 & 1 \\
\end{longtable}
}

\textbf{Symbol:}

\begin{verbatim}
    A {-{-}{-}{-}}
           {{-}{-}{-}{-} Output}
    B {-{-}{-}{-}/}
\end{verbatim}

\textbf{Example Applications:}

\begin{itemize}
\tightlist
\item
  \textbf{Security System}: Door opens only with key AND card
\item
  \textbf{Car Starting}: Engine starts with key AND foot on brake
\item
  \textbf{Boolean Expression}: Y = A · B or Y = A \wedge B
\end{itemize}

\textbf{Real-life Example:} Washing machine starts only when door is
closed AND power button is pressed.

\end{solutionbox}
\begin{mnemonicbox}
``ALL inputs True = Output True''

\end{mnemonicbox}
\subsection*{Question 2(c OR) [7
marks]}\label{question-2c-or-7-marks}

\textbf{Explain the Ethernet Cable Color code.}

\begin{solutionbox}

\textbf{Standard: TIA/EIA-568B Color Code}


{\def\LTcaptype{none} % do not increment counter
\vspace{-5pt}
\captionof{table}{Wire Color Sequence}
\vspace{-10pt}
\begin{longtable}[]{@{}lll@{}}
\toprule\noalign{}
Pin & Color & Function \\
\midrule\noalign{}
\endhead
\bottomrule\noalign{}
\endlastfoot
1 & White/Orange & Transmit+ \\
2 & Orange & Transmit- \\
3 & White/Green & Receive+ \\
4 & Blue & Not used \\
5 & White/Blue & Not used \\
6 & Green & Receive- \\
7 & White/Brown & Not used \\
8 & Brown & Not used \\
\end{longtable}
}

\textbf{Cable Types:}

\textbf{Straight-Through Cable (568B both ends):}

\begin{itemize}
\tightlist
\item
  \textbf{Use}: Computer to switch/hub
\item
  \textbf{Color sequence}: Same on both ends
\end{itemize}

\textbf{Cross-Over Cable (568A one end, 568B other):}

\begin{itemize}
\tightlist
\item
  \textbf{Use}: Computer to computer direct
\item
  \textbf{Pins swapped}: 1\leftrightarrow3, 2\leftrightarrow6
\end{itemize}

\textbf{Wiring Diagram:}

\begin{verbatim}
RJ{-45 Connector (568B):}
Pin 1: White/Orange
Pin 2: Orange  
Pin 3: White/Green
Pin 4: Blue
Pin 5: White/Blue
Pin 6: Green
Pin 7: White/Brown
Pin 8: Brown
\end{verbatim}

\textbf{Preparation Steps:}

\begin{enumerate}
\tightlist
\item
  Strip outer jacket (1 inch)
\item
  Arrange wires in color order
\item
  Cut wires evenly
\item
  Insert into RJ-45 connector
\item
  Crimp with crimping tool
\end{enumerate}

\end{solutionbox}
\begin{mnemonicbox}
``White Orange, Orange, White Green, Blue, White
Blue, Green, White Brown, Brown''

\end{mnemonicbox}
\subsection*{Question 3(a) [3 marks]}\label{q3a}

\textbf{Compare wired and Wireless Communication.}

\begin{solutionbox}


{\def\LTcaptype{none} % do not increment counter
\vspace{-5pt}
\captionof{table}{Wired vs Wireless Communication}
\vspace{-10pt}
\begin{longtable}[]{@{}lll@{}}
\toprule\noalign{}
Aspect & Wired & Wireless \\
\midrule\noalign{}
\endhead
\bottomrule\noalign{}
\endlastfoot
\textbf{Medium} & Cables (copper/fiber) & Radio waves/infrared \\
\textbf{Speed} & Higher (up to 100Gbps) & Lower (up to 1Gbps) \\
\textbf{Security} & More secure & Less secure \\
\textbf{Mobility} & Limited & High mobility \\
\textbf{Cost} & Higher installation & Lower installation \\
\textbf{Interference} & Minimal & Signal interference \\
\end{longtable}
}

\textbf{Key Points:}

\begin{itemize}
\tightlist
\item
  \textbf{Wired}: Reliable, fast, secure but limited mobility
\item
  \textbf{Wireless}: Mobile, flexible but security concerns
\end{itemize}

\end{solutionbox}
\begin{mnemonicbox}
``Wires are Fast, Wireless is Free'' (speed vs
mobility)

\end{mnemonicbox}
\subsection*{Question 3(b) [4 marks]}\label{q3b}

\textbf{Discuss the different types of computer systems.}

\begin{solutionbox}


{\def\LTcaptype{none} % do not increment counter
\vspace{-5pt}
\captionof{table}{Computer System Types}
\vspace{-10pt}
\begin{longtable}[]{@{}
  >{\raggedright\arraybackslash}p{(\linewidth - 6\tabcolsep) * \real{0.1538}}
  >{\raggedright\arraybackslash}p{(\linewidth - 6\tabcolsep) * \real{0.1538}}
  >{\raggedright\arraybackslash}p{(\linewidth - 6\tabcolsep) * \real{0.4615}}
  >{\raggedright\arraybackslash}p{(\linewidth - 6\tabcolsep) * \real{0.2308}}@{}}
\toprule\noalign{}
\begin{minipage}[b]{\linewidth}\raggedright
Type
\end{minipage} & \begin{minipage}[b]{\linewidth}\raggedright
Size
\end{minipage} & \begin{minipage}[b]{\linewidth}\raggedright
Processing Power
\end{minipage} & \begin{minipage}[b]{\linewidth}\raggedright
Example
\end{minipage} \\
\midrule\noalign{}
\endhead
\bottomrule\noalign{}
\endlastfoot
\textbf{Supercomputer} & Room-sized & Extremely high & Weather
forecasting \\
\textbf{Mainframe} & Large cabinet & Very high & Bank transactions \\
\textbf{Minicomputer} & Desk-sized & Medium & Small business \\
\textbf{Microcomputer} & Desktop/laptop & Low to medium & Personal
use \\
\end{longtable}
}

\textbf{Classifications:}

\textbf{By Size \& Power:}

\begin{itemize}
\tightlist
\item
  \textbf{Supercomputer}: Scientific calculations, research
\item
  \textbf{Mainframe}: Large organizations, concurrent users
\item
  \textbf{Personal Computer}: Individual users, office work
\item
  \textbf{Embedded Systems}: Specific functions (washing machines)
\end{itemize}

\textbf{By Purpose:}

\begin{itemize}
\tightlist
\item
  \textbf{General Purpose}: Versatile, multiple applications
\item
  \textbf{Special Purpose}: Dedicated tasks (ATM, gaming console)
\end{itemize}

\end{solutionbox}
\begin{mnemonicbox}
``Super Main Mini Micro'' (decreasing size order)

\end{mnemonicbox}
\subsection*{Question 3(c) [7 marks]}\label{q3c}

\textbf{Write short note on TDM, FDM, and OFDM.}

\begin{solutionbox}

\textbf{Multiplexing Techniques for Efficient Communication}


{\def\LTcaptype{none} % do not increment counter
\vspace{-5pt}
\captionof{table}{Multiplexing Comparison}
\vspace{-10pt}
\begin{longtable}[]{@{}
  >{\raggedright\arraybackslash}p{(\linewidth - 6\tabcolsep) * \real{0.2157}}
  >{\raggedright\arraybackslash}p{(\linewidth - 6\tabcolsep) * \real{0.3137}}
  >{\raggedright\arraybackslash}p{(\linewidth - 6\tabcolsep) * \real{0.2549}}
  >{\raggedright\arraybackslash}p{(\linewidth - 6\tabcolsep) * \real{0.2157}}@{}}
\toprule\noalign{}
\begin{minipage}[b]{\linewidth}\raggedright
Technique
\end{minipage} & \begin{minipage}[b]{\linewidth}\raggedright
Division Method
\end{minipage} & \begin{minipage}[b]{\linewidth}\raggedright
Application
\end{minipage} & \begin{minipage}[b]{\linewidth}\raggedright
Advantage
\end{minipage} \\
\midrule\noalign{}
\endhead
\bottomrule\noalign{}
\endlastfoot
\textbf{TDM} & Time slots & Digital telephony & Simple implementation \\
\textbf{FDM} & Frequency bands & Radio/TV broadcasting & Simultaneous
transmission \\
\textbf{OFDM} & Multiple carriers & Wi-Fi, 4G/5G & High data rates \\
\end{longtable}
}

\textbf{Time Division Multiplexing (TDM):}

\begin{itemize}
\tightlist
\item
  \textbf{Principle}: Each user gets fixed time slot
\item
  \textbf{Implementation}: Sequential data transmission
\item
  \textbf{Example}: Digital telephone systems, GSM
\item
  \textbf{Advantage}: Efficient use of bandwidth
\end{itemize}

\textbf{Frequency Division Multiplexing (FDM):}

\begin{itemize}
\tightlist
\item
  \textbf{Principle}: Each user gets unique frequency band
\item
  \textbf{Implementation}: Simultaneous transmission
\item
  \textbf{Example}: FM radio, cable TV
\item
  \textbf{Advantage}: No timing coordination needed
\end{itemize}

\textbf{Orthogonal Frequency Division Multiplexing (OFDM):}

\begin{itemize}
\tightlist
\item
  \textbf{Principle}: Multiple orthogonal subcarriers
\item
  \textbf{Implementation}: Parallel data streams
\item
  \textbf{Example}: Wi-Fi (802.11), LTE, DSL
\item
  \textbf{Advantage}: High spectral efficiency, robust against
  interference
\end{itemize}

\textbf{Diagram:}

\begin{center}
\textbf{Mermaid Diagram (Code)}
\begin{verbatim}
{Shaded}
{Highlighting}[]
graph TD
    A[Data Stream] {-{-}{} B[TDM {-} Time Slots]}
    A {-{-}{} C[FDM {-} Frequency Bands]}
    A {-{-}{} D[OFDM {-} Multiple Carriers]}
    B {-{-}{} E["T1|T2|T3|T4"]}
    C {-{-}{} F[F1 + F2 + F3 + F4]}
    D {-{-}{} G[Orthogonal Subcarriers]}
{Highlighting}
{Shaded}
\end{verbatim}
\end{center}

\textbf{Applications:}

\begin{itemize}
\tightlist
\item
  \textbf{TDM}: ISDN, T1/E1 lines
\item
  \textbf{FDM}: Analog TV, radio
\item
  \textbf{OFDM}: Modern wireless systems
\end{itemize}

\end{solutionbox}
\begin{mnemonicbox}
``Time Frequency Orthogonal'' (TDM-FDM-OFDM)

\end{mnemonicbox}
\subsection*{Question 3(a OR) [3
marks]}\label{question-3a-or-3-marks}

\textbf{Discuss FSK and PSK.}

\begin{solutionbox}

\textbf{Digital Modulation Techniques}


{\def\LTcaptype{none} % do not increment counter
\vspace{-5pt}
\captionof{table}{FSK vs PSK}
\vspace{-10pt}
\begin{longtable}[]{@{}lll@{}}
\toprule\noalign{}
Aspect & FSK & PSK \\
\midrule\noalign{}
\endhead
\bottomrule\noalign{}
\endlastfoot
\textbf{Parameter} & Frequency & Phase \\
\textbf{Complexity} & Simple & Complex \\
\textbf{Noise Immunity} & Good & Excellent \\
\textbf{Bandwidth} & Higher & Lower \\
\end{longtable}
}

\textbf{FSK (Frequency Shift Keying):}

\begin{itemize}
\tightlist
\item
  \textbf{Principle}: Different frequencies for 0 and 1
\item
  \textbf{Implementation}: f1 for `0', f2 for `1'
\item
  \textbf{Example}: Computer modems, RFID
\end{itemize}

\textbf{PSK (Phase Shift Keying):}

\begin{itemize}
\tightlist
\item
  \textbf{Principle}: Phase changes represent data
\item
  \textbf{Implementation}: 0^\circ for `0', 180^\circ for `1'
\item
  \textbf{Example}: Wi-Fi, satellite communication
\end{itemize}

\end{solutionbox}
\begin{mnemonicbox}
``Frequency Shifts, Phase Shifts'' (FSK-PSK)

\end{mnemonicbox}
\subsection*{Question 3(b OR) [4
marks]}\label{question-3b-or-4-marks}

\textbf{Differentiate between Multitasking and Multi programming OS.}

\begin{solutionbox}


{\def\LTcaptype{none} % do not increment counter
\vspace{-5pt}
\captionof{table}{Multitasking vs Multiprogramming}
\vspace{-10pt}
\begin{longtable}[]{@{}lll@{}}
\toprule\noalign{}
Feature & Multitasking & Multiprogramming \\
\midrule\noalign{}
\endhead
\bottomrule\noalign{}
\endlastfoot
\textbf{User Interaction} & Interactive & Batch processing \\
\textbf{Response Time} & Fast & Slower \\
\textbf{CPU Sharing} & Time slicing & Job switching \\
\textbf{Example} & Windows, Linux & Early mainframes \\
\end{longtable}
}

\textbf{Multitasking:}

\begin{itemize}
\tightlist
\item
  \textbf{Definition}: Multiple tasks run seemingly simultaneously
\item
  \textbf{Method}: Time sharing with quick switching
\item
  \textbf{User Experience}: Interactive, responsive
\item
  \textbf{Types}: Preemptive, cooperative
\end{itemize}

\textbf{Multiprogramming:}

\begin{itemize}
\tightlist
\item
  \textbf{Definition}: Multiple programs in memory
\item
  \textbf{Method}: CPU switches when I/O operations occur
\item
  \textbf{User Experience}: Batch job processing
\item
  \textbf{Purpose}: CPU utilization improvement
\end{itemize}

\end{solutionbox}
\begin{mnemonicbox}
``Tasks are Interactive, Programs are Batched''

\end{mnemonicbox}
\subsection*{Question 3(c OR) [7
marks]}\label{question-3c-or-7-marks}

\textbf{Write short note on network topologies.}

\begin{solutionbox}

\textbf{Network Topology Types and Characteristics}


{\def\LTcaptype{none} % do not increment counter
\vspace{-5pt}
\captionof{table}{Topology Comparison}
\vspace{-10pt}
\begin{longtable}[]{@{}
  >{\raggedright\arraybackslash}p{(\linewidth - 8\tabcolsep) * \real{0.1852}}
  >{\raggedright\arraybackslash}p{(\linewidth - 8\tabcolsep) * \real{0.2037}}
  >{\raggedright\arraybackslash}p{(\linewidth - 8\tabcolsep) * \real{0.2222}}
  >{\raggedright\arraybackslash}p{(\linewidth - 8\tabcolsep) * \real{0.2778}}
  >{\raggedright\arraybackslash}p{(\linewidth - 8\tabcolsep) * \real{0.1111}}@{}}
\toprule\noalign{}
\begin{minipage}[b]{\linewidth}\raggedright
Topology
\end{minipage} & \begin{minipage}[b]{\linewidth}\raggedright
Structure
\end{minipage} & \begin{minipage}[b]{\linewidth}\raggedright
Advantages
\end{minipage} & \begin{minipage}[b]{\linewidth}\raggedright
Disadvantages
\end{minipage} & \begin{minipage}[b]{\linewidth}\raggedright
Cost
\end{minipage} \\
\midrule\noalign{}
\endhead
\bottomrule\noalign{}
\endlastfoot
\textbf{Bus} & Linear & Simple, cost-effective & Single point failure &
Low \\
\textbf{Star} & Central hub & Easy troubleshooting & Hub failure affects
all & Medium \\
\textbf{Ring} & Circular & Equal access & Break affects network &
Medium \\
\textbf{Mesh} & Interconnected & High reliability & Complex, expensive &
High \\
\textbf{Hybrid} & Mixed & Flexible & Complex management & Variable \\
\end{longtable}
}

\textbf{Detailed Descriptions:}

\textbf{Bus Topology:}

\begin{itemize}
\tightlist
\item
  \textbf{Structure}: Single backbone cable
\item
  \textbf{Termination}: Required at both ends
\item
  \textbf{Example}: Early Ethernet (10BASE2)
\item
  \textbf{Failure Impact}: Cable break stops entire network
\end{itemize}

\textbf{Star Topology:}

\begin{itemize}
\tightlist
\item
  \textbf{Structure}: Central switch/hub with spokes
\item
  \textbf{Scalability}: Easy to add/remove nodes
\item
  \textbf{Example}: Modern Ethernet networks
\item
  \textbf{Failure Impact}: Only affected node fails
\end{itemize}

\textbf{Ring Topology:}

\begin{itemize}
\tightlist
\item
  \textbf{Structure}: Nodes connected in circle
\item
  \textbf{Data Flow}: Unidirectional token passing
\item
  \textbf{Example}: Token Ring, FDDI
\item
  \textbf{Failure Impact}: Single break stops network
\end{itemize}

\textbf{Mesh Topology:}

\begin{itemize}
\tightlist
\item
  \textbf{Structure}: Every node connected to every other
\item
  \textbf{Types}: Full mesh, partial mesh
\item
  \textbf{Example}: Internet backbone, military networks
\item
  \textbf{Reliability}: Multiple paths available
\end{itemize}

\textbf{Hybrid Topology:}

\begin{itemize}
\tightlist
\item
  \textbf{Structure}: Combination of topologies
\item
  \textbf{Example}: Star-bus, star-ring
\item
  \textbf{Flexibility}: Best features of each type
\end{itemize}

\textbf{Diagram:}

\begin{center}
\textbf{Mermaid Diagram (Code)}
\begin{verbatim}
{Shaded}
{Highlighting}[]
graph TD
    A[Network Topologies] {-{-}{} B[Bus]}
    A {-{-}{} C[Star]}
    A {-{-}{} D[Ring]}
    A {-{-}{} E[Mesh]}
    A {-{-}{} F[Hybrid]}
    
    B {-{-}{} G[Linear Connection]}
    C {-{-}{} H[Central Hub]}
    D {-{-}{} I[Circular Connection]}
    E {-{-}{} J[Full Interconnection]}
    F {-{-}{} K[Mixed Structure]}
{Highlighting}
{Shaded}
\end{verbatim}
\end{center}

\textbf{Selection Criteria:}

\begin{itemize}
\tightlist
\item
  \textbf{Cost}: Bus \textless{} Star \textless{} Ring \textless{} Mesh
\item
  \textbf{Reliability}: Bus \textless{} Ring \textless{} Star
  \textless{} Mesh
\item
  \textbf{Scalability}: Ring \textless{} Bus \textless{} Star
  \textless{} Mesh
\end{itemize}

\end{solutionbox}
\begin{mnemonicbox}
``Bus Star Ring Mesh Hybrid'' (increasing complexity)

\end{mnemonicbox}
\subsection*{Question 4(a) [3 marks]}\label{q4a}

\textbf{Explain Switch.}

\begin{solutionbox}

\textbf{Network Switch Definition and Functions}


{\def\LTcaptype{none} % do not increment counter
\vspace{-5pt}
\captionof{table}{Switch Characteristics}
\vspace{-10pt}
\begin{longtable}[]{@{}ll@{}}
\toprule\noalign{}
Feature & Description \\
\midrule\noalign{}
\endhead
\bottomrule\noalign{}
\endlastfoot
\textbf{Function} & Connects devices in LAN \\
\textbf{Layer} & Data Link Layer (Layer 2) \\
\textbf{Method} & MAC address learning \\
\textbf{Collision} & Eliminates collisions \\
\end{longtable}
}

\textbf{Key Features:}

\begin{itemize}
\tightlist
\item
  \textbf{MAC Address Table}: Learns and stores device addresses
\item
  \textbf{Full Duplex}: Simultaneous send/receive
\item
  \textbf{Dedicated Bandwidth}: Each port gets full bandwidth
\item
  \textbf{VLAN Support}: Virtual network segregation
\end{itemize}

\textbf{Functions:}

\begin{itemize}
\tightlist
\item
  \textbf{Frame Forwarding}: Sends data to specific port
\item
  \textbf{Address Learning}: Builds MAC address table
\item
  \textbf{Loop Prevention}: Spanning Tree Protocol
\end{itemize}

\end{solutionbox}
\begin{mnemonicbox}
``Switch Learns MAC Addresses''

\end{mnemonicbox}
\subsection*{Question 4(b) [4 marks]}\label{q4b}

\textbf{Define Cyberthreat with an example.}

\begin{solutionbox}

\textbf{Cyberthreat Definition:} Malicious attempt to damage, disrupt,
or gain unauthorized access to computer systems.


{\def\LTcaptype{none} % do not increment counter
\vspace{-5pt}
\captionof{table}{Cyberthreat Types}
\vspace{-10pt}
\begin{longtable}[]{@{}
  >{\raggedright\arraybackslash}p{(\linewidth - 6\tabcolsep) * \real{0.1935}}
  >{\raggedright\arraybackslash}p{(\linewidth - 6\tabcolsep) * \real{0.2581}}
  >{\raggedright\arraybackslash}p{(\linewidth - 6\tabcolsep) * \real{0.2903}}
  >{\raggedright\arraybackslash}p{(\linewidth - 6\tabcolsep) * \real{0.2581}}@{}}
\toprule\noalign{}
\begin{minipage}[b]{\linewidth}\raggedright
Type
\end{minipage} & \begin{minipage}[b]{\linewidth}\raggedright
Method
\end{minipage} & \begin{minipage}[b]{\linewidth}\raggedright
Example
\end{minipage} & \begin{minipage}[b]{\linewidth}\raggedright
Impact
\end{minipage} \\
\midrule\noalign{}
\endhead
\bottomrule\noalign{}
\endlastfoot
\textbf{Malware} & Malicious software & Virus, Trojan & Data
corruption \\
\textbf{Phishing} & Fake emails/websites & Fake bank emails & Identity
theft \\
\textbf{Ransomware} & Encrypt files & WannaCry attack & Financial
loss \\
\textbf{DDoS} & Traffic overload & Server flooding & Service
disruption \\
\end{longtable}
}

\textbf{Example - Phishing Attack:}

\begin{itemize}
\tightlist
\item
  \textbf{Method}: Fake email from ``bank''
\item
  \textbf{Request}: Login credentials
\item
  \textbf{Result}: Account compromise
\item
  \textbf{Prevention}: Verify sender authenticity
\end{itemize}

\textbf{Common Indicators:}

\begin{itemize}
\tightlist
\item
  \textbf{Suspicious emails}: Unknown senders, urgent requests
\item
  \textbf{Unusual system behavior}: Slow performance, pop-ups
\item
  \textbf{Unauthorized access}: Changed passwords, new files
\end{itemize}

\end{solutionbox}
\begin{mnemonicbox}
``Cyber Criminals Create Chaos'' (threats cause
damage)

\end{mnemonicbox}
\subsection*{Question 4(c) [7 marks]}\label{q4c}

\textbf{Compare TCP/IP and OSI networking models.}

\begin{solutionbox}


{\def\LTcaptype{none} % do not increment counter
\vspace{-5pt}
\captionof{table}{TCP/IP vs OSI Model Comparison}
\vspace{-10pt}
\begin{longtable}[]{@{}
  >{\raggedright\arraybackslash}p{(\linewidth - 6\tabcolsep) * \real{0.2000}}
  >{\raggedright\arraybackslash}p{(\linewidth - 6\tabcolsep) * \real{0.2545}}
  >{\raggedright\arraybackslash}p{(\linewidth - 6\tabcolsep) * \real{0.2545}}
  >{\raggedright\arraybackslash}p{(\linewidth - 6\tabcolsep) * \real{0.2909}}@{}}
\toprule\noalign{}
\begin{minipage}[b]{\linewidth}\raggedright
OSI Layer
\end{minipage} & \begin{minipage}[b]{\linewidth}\raggedright
OSI Function
\end{minipage} & \begin{minipage}[b]{\linewidth}\raggedright
TCP/IP Layer
\end{minipage} & \begin{minipage}[b]{\linewidth}\raggedright
TCP/IP Function
\end{minipage} \\
\midrule\noalign{}
\endhead
\bottomrule\noalign{}
\endlastfoot
\textbf{Application} & User interface & \textbf{Application} & User
services \\
\textbf{Presentation} & Data formatting & \textbf{Application} &
(Combined) \\
\textbf{Session} & Session management & \textbf{Application} &
(Combined) \\
\textbf{Transport} & Reliable delivery & \textbf{Transport} & End-to-end
delivery \\
\textbf{Network} & Routing & \textbf{Internet} & IP addressing \\
\textbf{Data Link} & Frame handling & \textbf{Network Access} & Physical
transmission \\
\textbf{Physical} & Electrical signals & \textbf{Network Access} &
(Combined) \\
\end{longtable}
}

\textbf{Key Differences:}

\textbf{OSI Model (7 layers):}

\begin{itemize}
\tightlist
\item
  \textbf{Purpose}: Theoretical reference model
\item
  \textbf{Development}: ISO standard
\item
  \textbf{Layers}: Clearly separated functions
\item
  \textbf{Usage}: Educational, troubleshooting
\end{itemize}

\textbf{TCP/IP Model (4 layers):}

\begin{itemize}
\tightlist
\item
  \textbf{Purpose}: Practical implementation
\item
  \textbf{Development}: DARPA/Internet
\item
  \textbf{Layers}: Combined functionality
\item
  \textbf{Usage}: Internet, real networks
\end{itemize}

\textbf{Advantages:}

\textbf{OSI Model:}

\begin{itemize}
\tightlist
\item
  \textbf{Standardization}: Universal reference
\item
  \textbf{Troubleshooting}: Layer-by-layer analysis
\item
  \textbf{Education}: Clear concept separation
\end{itemize}

\textbf{TCP/IP Model:}

\begin{itemize}
\tightlist
\item
  \textbf{Simplicity}: Fewer layers
\item
  \textbf{Practicality}: Internet-proven
\item
  \textbf{Flexibility}: Protocol independence
\end{itemize}

\textbf{Protocols Examples:}

\begin{itemize}
\tightlist
\item
  \textbf{OSI}: Conceptual framework
\item
  \textbf{TCP/IP}: HTTP, FTP, TCP, UDP, IP
\end{itemize}

\textbf{Diagram:}

\begin{center}
\textbf{Mermaid Diagram (Code)}
\begin{verbatim}
{Shaded}
{Highlighting}[]
graph TD
    A[OSI {- 7 Layers] {-}{-}{} B[Application]}
    A {-{-}{} C[Presentation]  }
    A {-{-}{} D[Session]}
    A {-{-}{} E[Transport]}
    A {-{-}{} F[Network]}
    A {-{-}{} G[Data Link]}
    A {-{-}{} H[Physical]}
    
    I[TCP/IP {- 4 Layers] {-}{-}{} J[Application]}
    I {-{-}{} K[Transport]}
    I {-{-}{} L[Internet]}
    I {-{-}{} M[Network Access]}
{Highlighting}
{Shaded}
\end{verbatim}
\end{center}

\end{solutionbox}
\begin{mnemonicbox}
``OSI is Perfect Theory, TCP/IP is Practical
Reality''

\end{mnemonicbox}
\subsection*{Question 4(a OR) [3
marks]}\label{question-4a-or-3-marks}

\textbf{Write main objectives of cyber security.}

\begin{solutionbox}


{\def\LTcaptype{none} % do not increment counter
\vspace{-5pt}
\captionof{table}{Cyber Security Objectives (CIA Triad)}
\vspace{-10pt}
\begin{longtable}[]{@{}
  >{\raggedright\arraybackslash}p{(\linewidth - 4\tabcolsep) * \real{0.3333}}
  >{\raggedright\arraybackslash}p{(\linewidth - 4\tabcolsep) * \real{0.3939}}
  >{\raggedright\arraybackslash}p{(\linewidth - 4\tabcolsep) * \real{0.2727}}@{}}
\toprule\noalign{}
\begin{minipage}[b]{\linewidth}\raggedright
Objective
\end{minipage} & \begin{minipage}[b]{\linewidth}\raggedright
Description
\end{minipage} & \begin{minipage}[b]{\linewidth}\raggedright
Example
\end{minipage} \\
\midrule\noalign{}
\endhead
\bottomrule\noalign{}
\endlastfoot
\textbf{Confidentiality} & Protect data from unauthorized access &
Encryption, passwords \\
\textbf{Integrity} & Ensure data accuracy and completeness & Digital
signatures, checksums \\
\textbf{Availability} & Ensure system accessibility & Backup systems,
redundancy \\
\end{longtable}
}

\textbf{Additional Objectives:}

\begin{itemize}
\tightlist
\item
  \textbf{Authentication}: Verify user identity
\item
  \textbf{Authorization}: Control access rights
\item
  \textbf{Non-repudiation}: Prevent denial of actions
\end{itemize}

\end{solutionbox}
\begin{mnemonicbox}
``CIA protects data''
(Confidentiality-Integrity-Availability)

\end{mnemonicbox}
\subsection*{Question 4(b OR) [4
marks]}\label{question-4b-or-4-marks}

\textbf{List out different types of networking devices used in the
networking.}

\begin{solutionbox}


{\def\LTcaptype{none} % do not increment counter
\vspace{-5pt}
\captionof{table}{Networking Devices}
\vspace{-10pt}
\begin{longtable}[]{@{}
  >{\raggedright\arraybackslash}p{(\linewidth - 6\tabcolsep) * \real{0.2105}}
  >{\raggedright\arraybackslash}p{(\linewidth - 6\tabcolsep) * \real{0.1842}}
  >{\raggedright\arraybackslash}p{(\linewidth - 6\tabcolsep) * \real{0.2632}}
  >{\raggedright\arraybackslash}p{(\linewidth - 6\tabcolsep) * \real{0.3421}}@{}}
\toprule\noalign{}
\begin{minipage}[b]{\linewidth}\raggedright
Device
\end{minipage} & \begin{minipage}[b]{\linewidth}\raggedright
Layer
\end{minipage} & \begin{minipage}[b]{\linewidth}\raggedright
Function
\end{minipage} & \begin{minipage}[b]{\linewidth}\raggedright
Example Use
\end{minipage} \\
\midrule\noalign{}
\endhead
\bottomrule\noalign{}
\endlastfoot
\textbf{Hub} & Physical & Signal repeater & Legacy networks \\
\textbf{Switch} & Data Link & Frame forwarding & LAN connectivity \\
\textbf{Router} & Network & Packet routing & Internet connection \\
\textbf{Bridge} & Data Link & Network segmentation & LAN extension \\
\textbf{Gateway} & All layers & Protocol conversion & Network
interconnection \\
\textbf{Repeater} & Physical & Signal amplification & Cable extension \\
\textbf{Access Point} & Data Link & Wireless connectivity & Wi-Fi
networks \\
\textbf{Firewall} & Network+ & Security filtering & Network
protection \\
\end{longtable}
}

\textbf{Functions:}

\begin{itemize}
\tightlist
\item
  \textbf{Connectivity}: Hub, switch, bridge
\item
  \textbf{Routing}: Router, gateway
\item
  \textbf{Security}: Firewall, proxy
\item
  \textbf{Wireless}: Access point, wireless router
\end{itemize}

\end{solutionbox}
\begin{mnemonicbox}
``Hubs Switch Routes Bridges Gateways''

\end{mnemonicbox}
\subsection*{Question 4(c OR) [7
marks]}\label{question-4c-or-7-marks}

\textbf{Write different types of security attacks.}

\begin{solutionbox}

\textbf{Classification of Security Attacks}


{\def\LTcaptype{none} % do not increment counter
\vspace{-5pt}
\captionof{table}{Attack Types and Characteristics}
\vspace{-10pt}
\begin{longtable}[]{@{}
  >{\raggedright\arraybackslash}p{(\linewidth - 8\tabcolsep) * \real{0.2600}}
  >{\raggedright\arraybackslash}p{(\linewidth - 8\tabcolsep) * \real{0.1600}}
  >{\raggedright\arraybackslash}p{(\linewidth - 8\tabcolsep) * \real{0.1600}}
  >{\raggedright\arraybackslash}p{(\linewidth - 8\tabcolsep) * \real{0.1800}}
  >{\raggedright\arraybackslash}p{(\linewidth - 8\tabcolsep) * \real{0.2400}}@{}}
\toprule\noalign{}
\begin{minipage}[b]{\linewidth}\raggedright
Attack Type
\end{minipage} & \begin{minipage}[b]{\linewidth}\raggedright
Method
\end{minipage} & \begin{minipage}[b]{\linewidth}\raggedright
Target
\end{minipage} & \begin{minipage}[b]{\linewidth}\raggedright
Example
\end{minipage} & \begin{minipage}[b]{\linewidth}\raggedright
Prevention
\end{minipage} \\
\midrule\noalign{}
\endhead
\bottomrule\noalign{}
\endlastfoot
\textbf{Passive} & Eavesdropping & Information & Traffic analysis &
Encryption \\
\textbf{Active} & System modification & Integrity & Data alteration &
Authentication \\
\textbf{Physical} & Hardware access & Equipment & Device theft &
Physical security \\
\textbf{Social Engineering} & Human manipulation & Users & Phishing &
User education \\
\end{longtable}
}

\textbf{Detailed Attack Categories:}

\textbf{1. Network Attacks:}

\begin{itemize}
\tightlist
\item
  \textbf{Man-in-the-Middle}: Intercept communication
\item
  \textbf{DDoS}: Overwhelm server with traffic
\item
  \textbf{Packet Sniffing}: Capture network data
\item
  \textbf{IP Spoofing}: Fake source addresses
\end{itemize}

\textbf{2. Application Attacks:}

\begin{itemize}
\tightlist
\item
  \textbf{SQL Injection}: Database manipulation
\item
  \textbf{Cross-site Scripting (XSS)}: Web vulnerability
\item
  \textbf{Buffer Overflow}: Memory corruption
\item
  \textbf{Zero-day Exploits}: Unknown vulnerabilities
\end{itemize}

\textbf{3. Malware Attacks:}

\begin{itemize}
\tightlist
\item
  \textbf{Virus}: Self-replicating code
\item
  \textbf{Worm}: Network-spreading malware
\item
  \textbf{Trojan}: Disguised malicious software
\item
  \textbf{Ransomware}: Data encryption for payment
\end{itemize}

\textbf{4. Social Engineering:}

\begin{itemize}
\tightlist
\item
  \textbf{Phishing}: Fake emails/websites
\item
  \textbf{Pretexting}: False scenarios
\item
  \textbf{Baiting}: Malicious downloads
\item
  \textbf{Tailgating}: Physical access following
\end{itemize}

\textbf{5. Cryptographic Attacks:}

\begin{itemize}
\tightlist
\item
  \textbf{Brute Force}: Try all combinations
\item
  \textbf{Dictionary Attack}: Common passwords
\item
  \textbf{Rainbow Tables}: Pre-computed hashes
\item
  \textbf{Side-channel}: Information leakage
\end{itemize}

\textbf{Attack Vectors:}

\begin{itemize}
\tightlist
\item
  \textbf{External}: Internet-based attacks
\item
  \textbf{Internal}: Insider threats
\item
  \textbf{Physical}: Direct hardware access
\item
  \textbf{Wireless}: Wi-Fi vulnerabilities
\end{itemize}

\textbf{Prevention Strategies:}

\begin{itemize}
\tightlist
\item
  \textbf{Technical}: Firewalls, antivirus, encryption
\item
  \textbf{Administrative}: Policies, procedures
\item
  \textbf{Physical}: Locks, surveillance
\item
  \textbf{Education}: User awareness training
\end{itemize}

\end{solutionbox}
\begin{mnemonicbox}
``Network Application Malware Social Crypto'' (attack
categories)

\end{mnemonicbox}
\subsection*{Question 5(a) [3 marks]}\label{q5a}

\textbf{Calculate binary of (5AB.4) hexadecimal number.}

\begin{solutionbox}

\textbf{Hexadecimal to Binary Conversion}

\textbf{Method:} Convert each hex digit to 4-bit binary


{\def\LTcaptype{none} % do not increment counter
\vspace{-5pt}
\captionof{table}{Hex to Binary Conversion}
\vspace{-10pt}
\begin{longtable}[]{@{}llll@{}}
\toprule\noalign{}
Hex Digit & Binary & Hex Digit & Binary \\
\midrule\noalign{}
\endhead
\bottomrule\noalign{}
\endlastfoot
5 & 0101 & B & 1011 \\
A & 1010 & 4 & 0100 \\
\end{longtable}
}

\textbf{Step-by-step Conversion:}

\begin{itemize}
\tightlist
\item
  \textbf{5} \rightarrow \textbf{0101}
\item
  \textbf{A} \rightarrow \textbf{1010}
\item
  \textbf{B} \rightarrow \textbf{1011}
\item
  \textbf{.} \rightarrow \textbf{.} (decimal point)
\item
  \textbf{4} \rightarrow \textbf{0100}
\end{itemize}

\textbf{Final Answer:} (5AB.4)_{1}_{6} = (010110101011.0100)_{2}

\textbf{Simplified:} (10110101011.01)_{2}

\end{solutionbox}
\begin{mnemonicbox}
``Each Hex = 4 Bits''

\end{mnemonicbox}
\subsection*{Question 5(b) [4 marks]}\label{q5b}

\textbf{List out the main features of Digi-Locker, e-rupi.}

\begin{solutionbox}


{\def\LTcaptype{none} % do not increment counter
\vspace{-5pt}
\captionof{table}{Digital Platform Features}
\vspace{-10pt}
\begin{longtable}[]{@{}
  >{\raggedright\arraybackslash}p{(\linewidth - 6\tabcolsep) * \real{0.2326}}
  >{\raggedright\arraybackslash}p{(\linewidth - 6\tabcolsep) * \real{0.2093}}
  >{\raggedright\arraybackslash}p{(\linewidth - 6\tabcolsep) * \real{0.3256}}
  >{\raggedright\arraybackslash}p{(\linewidth - 6\tabcolsep) * \real{0.2326}}@{}}
\toprule\noalign{}
\begin{minipage}[b]{\linewidth}\raggedright
Platform
\end{minipage} & \begin{minipage}[b]{\linewidth}\raggedright
Purpose
\end{minipage} & \begin{minipage}[b]{\linewidth}\raggedright
Key Features
\end{minipage} & \begin{minipage}[b]{\linewidth}\raggedright
Benefits
\end{minipage} \\
\midrule\noalign{}
\endhead
\bottomrule\noalign{}
\endlastfoot
\textbf{Digi-Locker} & Document storage & Cloud storage, digital
certificates & Paperless verification \\
\textbf{e-RUPI} & Digital payment & QR/SMS voucher, pre-paid & Targeted
welfare delivery \\
\end{longtable}
}

\textbf{Digi-Locker Features:}

\begin{itemize}
\tightlist
\item
  \textbf{Digital Wallet}: Store documents in cloud
\item
  \textbf{Authentication}: Aadhaar-based verification\\
\item
  \textbf{Integration}: Government department access
\item
  \textbf{Sharing}: Secure document sharing
\end{itemize}

\textbf{e-RUPI Features:}

\begin{itemize}
\tightlist
\item
  \textbf{Prepaid Voucher}: Purpose-specific payments
\item
  \textbf{Contact-less}: QR code/SMS based
\item
  \textbf{Security}: No personal/bank details shared
\item
  \textbf{Usage}: Healthcare, education, welfare schemes
\end{itemize}

\end{solutionbox}
\begin{mnemonicbox}
``Digi Stores, e-RUPI Pays'' (storage vs payment)

\end{mnemonicbox}
\subsection*{Question 5(c) [7 marks]}\label{q5c}

\textbf{Describe different generations of a computer system.}

\begin{solutionbox}

\textbf{Computer Generations Evolution}


{\def\LTcaptype{none} % do not increment counter
\vspace{-5pt}
\captionof{table}{Computer Generations Comparison}
\vspace{-10pt}
\begin{longtable}[]{@{}
  >{\raggedright\arraybackslash}p{(\linewidth - 10\tabcolsep) * \real{0.2182}}
  >{\raggedright\arraybackslash}p{(\linewidth - 10\tabcolsep) * \real{0.1455}}
  >{\raggedright\arraybackslash}p{(\linewidth - 10\tabcolsep) * \real{0.2182}}
  >{\raggedright\arraybackslash}p{(\linewidth - 10\tabcolsep) * \real{0.1091}}
  >{\raggedright\arraybackslash}p{(\linewidth - 10\tabcolsep) * \real{0.1273}}
  >{\raggedright\arraybackslash}p{(\linewidth - 10\tabcolsep) * \real{0.1818}}@{}}
\toprule\noalign{}
\begin{minipage}[b]{\linewidth}\raggedright
Generation
\end{minipage} & \begin{minipage}[b]{\linewidth}\raggedright
Period
\end{minipage} & \begin{minipage}[b]{\linewidth}\raggedright
Technology
\end{minipage} & \begin{minipage}[b]{\linewidth}\raggedright
Size
\end{minipage} & \begin{minipage}[b]{\linewidth}\raggedright
Speed
\end{minipage} & \begin{minipage}[b]{\linewidth}\raggedright
Examples
\end{minipage} \\
\midrule\noalign{}
\endhead
\bottomrule\noalign{}
\endlastfoot
\textbf{First} & 1940-1956 & Vacuum Tubes & Room-sized & Slow & ENIAC,
UNIVAC \\
\textbf{Second} & 1956-1963 & Transistors & Smaller & Faster & IBM 1401,
CDC 1604 \\
\textbf{Third} & 1964-1971 & Integrated Circuits & Desk-sized & Much
faster & IBM 360, PDP-8 \\
\textbf{Fourth} & 1971-1980s & Microprocessors & Personal & Very fast &
Intel 4004, Apple II \\
\textbf{Fifth} & 1980s-Present & AI/Parallel Processing & Portable &
Extremely fast & Modern PCs, smartphones \\
\end{longtable}
}

\textbf{Detailed Description:}

\textbf{First Generation (1940-1956):}

\begin{itemize}
\tightlist
\item
  \textbf{Technology}: Vacuum tubes for logic/memory
\item
  \textbf{Programming}: Machine language, punch cards
\item
  \textbf{Characteristics}: Large, expensive, unreliable
\item
  \textbf{Heat}: Generated enormous heat
\item
  \textbf{Examples}: ENIAC (30 tons), UNIVAC I
\end{itemize}

\textbf{Second Generation (1956-1963):}

\begin{itemize}
\tightlist
\item
  \textbf{Technology}: Transistors replaced vacuum tubes
\item
  \textbf{Programming}: Assembly language, FORTRAN, COBOL
\item
  \textbf{Improvements}: Smaller, faster, more reliable
\item
  \textbf{Memory}: Magnetic core memory
\item
  \textbf{Examples}: IBM 1401, Honeywell 400
\end{itemize}

\textbf{Third Generation (1964-1971):}

\begin{itemize}
\tightlist
\item
  \textbf{Technology}: Integrated Circuits (ICs)
\item
  \textbf{Programming}: High-level languages
\item
  \textbf{Features}: Operating systems, multiprocessing
\item
  \textbf{Size}: Mini-computer emergence
\item
  \textbf{Examples}: IBM System/360, PDP-8
\end{itemize}

\textbf{Fourth Generation (1971-1980s):}

\begin{itemize}
\tightlist
\item
  \textbf{Technology}: Microprocessors (CPU on chip)
\item
  \textbf{Development}: Personal computers born
\item
  \textbf{Features}: GUI, networking capabilities
\item
  \textbf{Storage}: Floppy disks, hard drives
\item
  \textbf{Examples}: Intel 8080, Apple II, IBM PC
\end{itemize}

\textbf{Fifth Generation (1980s-Present):}

\begin{itemize}
\tightlist
\item
  \textbf{Technology}: AI, parallel processing, VLSI
\item
  \textbf{Features}: Internet, multimedia, mobile computing
\item
  \textbf{Characteristics}: User-friendly, portable, powerful
\item
  \textbf{Current}: Smartphones, tablets, cloud computing
\item
  \textbf{Examples}: Modern laptops, smartphones, supercomputers
\end{itemize}

\textbf{Key Innovations by Generation:}

\begin{itemize}
\tightlist
\item
  \textbf{1st}: Electronic computing
\item
  \textbf{2nd}: Stored programs
\item
  \textbf{3rd}: Operating systems
\item
  \textbf{4th}: Personal computing
\item
  \textbf{5th}: Internet and AI
\end{itemize}

\textbf{Diagram:}

\begin{verbatim}
timeline
    title Computer Generations
    1940{-1956 : First Generation}
              : Vacuum Tubes
              : Room{-sized}
    1956{-1963 : Second Generation}
              : Transistors
              : Smaller size
    1964{-1971 : Third Generation}
              : Integrated Circuits
              : Minicomputers
    1971{-1980s : Fourth Generation}
               : Microprocessors
               : Personal Computers
    1980s{-Present : Fifth Generation}
                  : AI \& Internet
                  : Mobile Computing
\end{verbatim}

\end{solutionbox}
\begin{mnemonicbox}
``Vacuum Transistor IC Micro AI'' (technology
progression)

\end{mnemonicbox}
\subsection*{Question 5(a OR) [3
marks]}\label{question-5a-or-3-marks}

\textbf{Write Difference between Data and Information with example.}

\begin{solutionbox}


{\def\LTcaptype{none} % do not increment counter
\vspace{-5pt}
\captionof{table}{Data vs Information}
\vspace{-10pt}
\begin{longtable}[]{@{}lll@{}}
\toprule\noalign{}
Aspect & Data & Information \\
\midrule\noalign{}
\endhead
\bottomrule\noalign{}
\endlastfoot
\textbf{Definition} & Raw facts/figures & Processed data \\
\textbf{Meaning} & No context & Has context \\
\textbf{Example} & 85, 92, 78 & Average score: 85\% \\
\textbf{Purpose} & Input for processing & Output for decision-making \\
\end{longtable}
}

\textbf{Examples:}

\begin{itemize}
\tightlist
\item
  \textbf{Data}: Student marks (85, 92, 78, 88)
\item
  \textbf{Information}: Class average is 85.75\%
\end{itemize}

\textbf{Characteristics:}

\begin{itemize}
\tightlist
\item
  \textbf{Data}: Unorganized, raw, needs processing
\item
  \textbf{Information}: Organized, meaningful, useful for decisions
\end{itemize}

\end{solutionbox}
\begin{mnemonicbox}
``Data is Raw, Information is Refined''

\end{mnemonicbox}
\subsection*{Question 5(b OR) [4
marks]}\label{question-5b-or-4-marks}

\textbf{Compare analog modulation and digital modulation.}

\begin{solutionbox}


{\def\LTcaptype{none} % do not increment counter
\vspace{-5pt}
\captionof{table}{Analog vs Digital Modulation}
\vspace{-10pt}
\begin{longtable}[]{@{}lll@{}}
\toprule\noalign{}
Feature & Analog Modulation & Digital Modulation \\
\midrule\noalign{}
\endhead
\bottomrule\noalign{}
\endlastfoot
\textbf{Signal Type} & Continuous & Discrete (0s and 1s) \\
\textbf{Noise Immunity} & Poor & Excellent \\
\textbf{Bandwidth} & Lower & Higher \\
\textbf{Quality} & Degrades with distance & Maintains quality \\
\textbf{Examples} & AM, FM radio & FSK, PSK, QAM \\
\end{longtable}
}

\textbf{Analog Modulation:}

\begin{itemize}
\tightlist
\item
  \textbf{Types}: AM (Amplitude), FM (Frequency), PM (Phase)
\item
  \textbf{Applications}: Radio broadcasting, analog TV
\item
  \textbf{Advantages}: Simple, lower bandwidth
\item
  \textbf{Disadvantages}: Noise susceptible, quality loss
\end{itemize}

\textbf{Digital Modulation:}

\begin{itemize}
\tightlist
\item
  \textbf{Types}: ASK, FSK, PSK, QAM
\item
  \textbf{Applications}: Wi-Fi, cellular, satellite
\item
  \textbf{Advantages}: Noise resistant, error correction
\item
  \textbf{Disadvantages}: Complex, higher bandwidth
\end{itemize}

\end{solutionbox}
\begin{mnemonicbox}
``Analog is Simple, Digital is Smart''

\end{mnemonicbox}
\subsection*{Question 5(c OR) [7
marks]}\label{question-5c-or-7-marks}

\textbf{Discuss the range of IP addresses in IPv4}

\begin{solutionbox}

\textbf{IPv4 Address Range and Classification}


{\def\LTcaptype{none} % do not increment counter
\vspace{-5pt}
\captionof{table}{IPv4 Address Classes}
\vspace{-10pt}
\begin{longtable}[]{@{}
  >{\raggedright\arraybackslash}p{(\linewidth - 10\tabcolsep) * \real{0.1061}}
  >{\raggedright\arraybackslash}p{(\linewidth - 10\tabcolsep) * \real{0.1061}}
  >{\raggedright\arraybackslash}p{(\linewidth - 10\tabcolsep) * \real{0.2424}}
  >{\raggedright\arraybackslash}p{(\linewidth - 10\tabcolsep) * \real{0.1515}}
  >{\raggedright\arraybackslash}p{(\linewidth - 10\tabcolsep) * \real{0.2879}}
  >{\raggedright\arraybackslash}p{(\linewidth - 10\tabcolsep) * \real{0.1061}}@{}}
\toprule\noalign{}
\begin{minipage}[b]{\linewidth}\raggedright
Class
\end{minipage} & \begin{minipage}[b]{\linewidth}\raggedright
Range
\end{minipage} & \begin{minipage}[b]{\linewidth}\raggedright
Default Subnet
\end{minipage} & \begin{minipage}[b]{\linewidth}\raggedright
Networks
\end{minipage} & \begin{minipage}[b]{\linewidth}\raggedright
Hosts per Network
\end{minipage} & \begin{minipage}[b]{\linewidth}\raggedright
Usage
\end{minipage} \\
\midrule\noalign{}
\endhead
\bottomrule\noalign{}
\endlastfoot
\textbf{A} & 1.0.0.0 - 126.0.0.0 & /8 (255.0.0.0) & 126 & 16,777,214 &
Large organizations \\
\textbf{B} & 128.0.0.0 - 191.255.0.0 & /16 (255.255.0.0) & 16,384 &
65,534 & Medium organizations \\
\textbf{C} & 192.0.0.0 - 223.255.255.0 & /24 (255.255.255.0) & 2,097,152
& 254 & Small organizations \\
\textbf{D} & 224.0.0.0 - 239.255.255.255 & N/A & N/A & N/A &
Multicast \\
\textbf{E} & 240.0.0.0 - 255.255.255.255 & N/A & N/A & N/A &
Reserved/Experimental \\
\end{longtable}
}

\textbf{Special Address Ranges:}

\textbf{Private IP Ranges (RFC 1918):}

\begin{itemize}
\tightlist
\item
  \textbf{Class A}: 10.0.0.0 - 10.255.255.255 (/8)
\item
  \textbf{Class B}: 172.16.0.0 - 172.31.255.255 (/12)\\
\item
  \textbf{Class C}: 192.168.0.0 - 192.168.255.255 (/16)
\end{itemize}

\textbf{Reserved Addresses:}

\begin{itemize}
\tightlist
\item
  \textbf{Loopback}: 127.0.0.0 - 127.255.255.255
\item
  \textbf{Link-local}: 169.254.0.0 - 169.254.255.255
\item
  \textbf{Broadcast}: x.x.x.255 (last address in subnet)
\item
  \textbf{Network}: x.x.x.0 (first address in subnet)
\end{itemize}

\textbf{Address Structure:}

\begin{itemize}
\tightlist
\item
  \textbf{Total IPv4 space}: 4,294,967,296 addresses (2^{3}^{2})
\item
  \textbf{Format}: 32-bit address in dotted decimal
\item
  \textbf{Example}: 192.168.1.100
\end{itemize}

\textbf{Subnet Calculation Example:}

\begin{itemize}
\tightlist
\item
  \textbf{Network}: 192.168.1.0/24
\item
  \textbf{Subnet Mask}: 255.255.255.0
\item
  \textbf{Host Range}: 192.168.1.1 - 192.168.1.254
\item
  \textbf{Broadcast}: 192.168.1.255
\end{itemize}

\textbf{CIDR Notation:}

\begin{itemize}
\tightlist
\item
  \textbf{/8}: 255.0.0.0 (Class A default)
\item
  \textbf{/16}: 255.255.0.0 (Class B default)
\item
  \textbf{/24}: 255.255.255.0 (Class C default)
\item
  \textbf{/30}: 255.255.255.252 (Point-to-point links)
\end{itemize}

\textbf{IPv4 Exhaustion:}

\begin{itemize}
\tightlist
\item
  \textbf{Problem}: Limited address space
\item
  \textbf{Solution}: IPv6 (128-bit addresses)
\item
  \textbf{Temporary fixes}: NAT, CIDR, private addressing
\end{itemize}

\textbf{Diagram:}

\begin{center}
\textbf{Mermaid Diagram (Code)}
\begin{verbatim}
{Shaded}
{Highlighting}[]
graph TD
    A[IPv4 Address Space] {-{-}{} B[Class A: 1{-}126]}
    A {-{-}{} C[Class B: 128{-}191]  }
    A {-{-}{} D[Class C: 192{-}223]}
    A {-{-}{} E[Class D: 224{-}239 Multicast]}
    A {-{-}{} F[Class E: 240{-}255 Reserved]}
    
    B {-{-}{} G[Large Networks]}
    C {-{-}{} H[Medium Networks]}
    D {-{-}{} I[Small Networks]}
{Highlighting}
{Shaded}
\end{verbatim}
\end{center}

\textbf{Applications:}

\begin{itemize}
\tightlist
\item
  \textbf{Public IPs}: Internet routing
\item
  \textbf{Private IPs}: Internal networks
\item
  \textbf{Multicast}: One-to-many communication
\item
  \textbf{Loopback}: Local testing
\end{itemize}

\end{solutionbox}
\begin{mnemonicbox}
``A Big Company Delivered Everything'' (Classes
A-B-C-D-E)

\end{mnemonicbox}

\end{document}
