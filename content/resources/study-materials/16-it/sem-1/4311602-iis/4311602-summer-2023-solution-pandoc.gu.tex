\documentclass[10pt,a4paper]{article}

% content/resources/templates/preamble.tex
\usepackage[margin=0.6in]{geometry}
\author{Milav Dabgar}
\usepackage{amsmath,amssymb,amsthm}
\usepackage{booktabs}
\usepackage{multirow}
\usepackage{xcolor}
\usepackage{tcolorbox}
\tcbuselibrary{breakable,skins}
\usepackage[colorlinks=true,linkcolor=blue]{hyperref}
\usepackage{titlesec}
\usepackage{enumitem}
\usepackage{tikz}
\usepackage{pgfplots}
\usepackage{circuitikz}
\usepackage[version=4]{mhchem}
\usepackage{longtable}
\usepackage{array}
\usepackage{float}
\usepackage{caption}
\usepackage{listings}

\lstset{
  basicstyle=\small\ttfamily,
  breaklines=true,
  breakatwhitespace=false,
  postbreak=\mbox{\textcolor{red}{$\hookrightarrow$}\space},
  float=false,
  numbers=left,
  numberstyle=\tiny\color{gray},
  numbersep=10pt,
  xleftmargin=2em,
  keywordstyle=\color{blue},
  commentstyle=\color{green!60!black},
  stringstyle=\color{purple},
  backgroundcolor=\color{gray!5},
  showstringspaces=false,
  tabsize=2,
  captionpos=b,
  keepspaces=true,
  columns=flexible
}

\pgfplotsset{compat=1.18}
\usetikzlibrary{shapes,arrows,positioning,calc,patterns,decorations.pathmorphing,decorations.markings,arrows.meta}

% Color scheme
\definecolor{headcolor}{RGB}{0,102,204}
\definecolor{keycolor}{RGB}{220,20,60}
\definecolor{solutioncolor}{RGB}{34,139,34}
\definecolor{mnemoniccolor}{RGB}{148,0,211}
\definecolor{codecolor}{RGB}{0,0,100}

% Spacing
\setlength{\parskip}{3pt}
\setlist[itemize]{nosep}
\setlist[enumerate]{nosep}

% Title formatting
\titleformat{\section}{\Large\bfseries\color{headcolor}}{\thesection}{1em}{}
\titleformat{\subsection}{\large\bfseries\color{headcolor}}{\thesubsection}{1em}{}

% Pandoc tightlist compatibility
\providecommand{\tightlist}{%
  \setlength{\itemsep}{0pt}\setlength{\parskip}{0pt}}

% Pandoc longtable compatibility
\newcounter{none}
\def\thenone{}


% content/resources/templates/gujarati-boxes.tex
\usepackage{fontspec}
\usepackage{polyglossia}

% Set Gujarati as main language (document is primarily in Gujarati)
% Note: gloss-gujarati.ldf doesn't exist in polyglossia, but it will use hyphenation patterns
\setdefaultlanguage{gujarati}
\setotherlanguage{english}

% Configure Gujarati font properly
% Use Language=Default to prevent polyglossia from trying to add language-specific features
% that don't exist for Gujarati, which causes "empty feature" warnings
\newfontfamily\gujaratifont[Script=Gujarati,AutoFakeBold=2.5,AutoFakeSlant=0.3]{Noto Sans Gujarati}
\setmainfont[Script=Gujarati,AutoFakeBold=2.5,AutoFakeSlant=0.3]{Noto Sans Gujarati}
% Use Noto Sans Gujarati for monospace to support Gujarati in text
\setmonofont[Scale=0.9]{Noto Sans Gujarati}

% Configure English to use the same font
\newfontfamily\englishfont[Script=Gujarati,AutoFakeBold=2.5,AutoFakeSlant=0.3]{Noto Sans Gujarati}

% Translations for polyglossia
\gappto\captionsgujarati{
  \renewcommand{\tablename}{કોષ્ટક}
  \renewcommand{\figurename}{આકૃતિ}
}

% Helper for TikZ nodes to ensure Gujarati font
\newcommand{\gu}[1]{{\gujaratifont #1}}

% Custom environments
\newtcolorbox{solutionbox}{
    breakable,
    enhanced,
    colback=solutioncolor!5!white,
    colframe=solutioncolor!75!black,
    fonttitle=\bfseries,
    title=જવાબ
}

\newtcolorbox{solutionboxnobreak}{
 colback=solutioncolor!5!white,
 colframe=solutioncolor!75!black,
 fonttitle=\bfseries,
 title=જવાબ
}

\newtcolorbox{keyformula}{
 breakable,
 enhanced,
 colback=keycolor!5!white,
 colframe=keycolor!75!black,
 fonttitle=\bfseries,
 title=રાસાયણિક સમીકરણ/સૂત્ર
}

\newtcolorbox{mnemonicbox}{
 breakable,
 enhanced,
 colback=mnemoniccolor!5!white,
 colframe=mnemoniccolor!75!black,
 fonttitle=\bfseries,
 title=મેમરી ટ્રીક
}


\begin{document}

\begin{center}
{\Huge\bfseries\color{headcolor} Subject Name (Gujarati)}\\[5pt]
{\LARGE 4311602 -- Summer 2023}\\[3pt]
{\large Semester 1 Study Material}\\[3pt]
{\normalsize\textit{Detailed Solutions and Explanations}}
\end{center}

\vspace{10pt}

\subsection*{પ્રશ્ન ૧(આ) [૩
ગુણ]}\label{uxaaauxab0uxab6uxaa8-uxae7uxa86-uxae9-uxa97uxaa3}

\textbf{કમ્પ્યુટરના મુખ્ય ઘટકોની ચચાર્ કરો.}

\begin{solutionbox}


{\def\LTcaptype{none} % do not increment counter
\vspace{-5pt}
\captionof{table}{કમ્પ્યુટરના મુખ્ય ઘટકો}
\vspace{-10pt}
\begin{longtable}[]{@{}lll@{}}
\toprule\noalign{}
ઘટક & કાર્ય & ઉદાહરણ \\
\midrule\noalign{}
\endhead
\bottomrule\noalign{}
\endlastfoot
\textbf{ઇનપુટ યુનિટ} & ડેટા અને સૂચનાઓ પ્રાપ્ત કરે & કીબોર્ડ, માઉસ \\
\textbf{સીપીયુ} & ડેટા પ્રોસેસ કરે અને કંટ્રોલ કરે & Intel i5, AMD Ryzen \\
\textbf{મેમરી} & ડેટા અસ્થાયી/કાયમી સંગ્રહ કરે & RAM, હાર્ડ ડિસ્ક \\
\textbf{આઉટપુટ યુનિટ} & પ્રોસેસ કરેલા પરિણામો દર્શાવે & મોનિટર, પ્રિન્ટર \\
\end{longtable}
}

\textbf{મુખ્ય ઘટકો:}

\begin{itemize}
\tightlist
\item
  \textbf{હાર્ડવેર}: ભૌતિક ભાગો જેવા કે CPU, RAM, મધરબોર્ડ
\item
  \textbf{સોફ્ટવેર}: પ્રોગ્રામ્સ અને ઓપરેટિંગ સિસ્ટમ
\item
  \textbf{ડેટા}: કમ્પ્યુટર દ્વારા પ્રોસેસ થતી માહિતી
\end{itemize}

\end{solutionbox}
\begin{mnemonicbox}
``ઇનપુટ સીપીયુ મેમરી આઉટપુટ''

\end{mnemonicbox}
\subsection*{પ્રશ્ન ૧(બ) [૪
ગુણ]}\label{uxaaauxab0uxab6uxaa8-uxae7uxaac-uxaea-uxa97uxaa3}

\textbf{વેબ બ્રાઉઝર અને તેનો પ્રકાર સમજાવો.}

\begin{solutionbox}

\textbf{વેબ બ્રાઉઝર} એ એવો સોફ્ટવેર છે જે ઇન્ટરનેટથી વેબ પૃષ્ઠોને ઍક્સેસ કરે અને દર્શાવે
છે.


{\def\LTcaptype{none} % do not increment counter
\vspace{-5pt}
\captionof{table}{વેબ બ્રાઉઝરના પ્રકારો}
\vspace{-10pt}
\begin{longtable}[]{@{}
  >{\raggedright\arraybackslash}p{(\linewidth - 4\tabcolsep) * \real{0.4333}}
  >{\raggedright\arraybackslash}p{(\linewidth - 4\tabcolsep) * \real{0.2667}}
  >{\raggedright\arraybackslash}p{(\linewidth - 4\tabcolsep) * \real{0.3000}}@{}}
\toprule\noalign{}
\begin{minipage}[b]{\linewidth}\raggedright
બ્રાઉઝર પ્રકાર
\end{minipage} & \begin{minipage}[b]{\linewidth}\raggedright
વિશેષતાઓ
\end{minipage} & \begin{minipage}[b]{\linewidth}\raggedright
ઉદાહરણો
\end{minipage} \\
\midrule\noalign{}
\endhead
\bottomrule\noalign{}
\endlastfoot
\textbf{ગ્રાફિકલ} & GUI ઇન્ટરફેસ, મલ્ટિમીડિયા સપોર્ટ & Chrome, Firefox \\
\textbf{ટેક્સ્ટ-આધારિત} & કમાન્ડ લાઇન, ઝડપી લોડિંગ & Lynx, Links \\
\textbf{મોબાઇલ} & ટચ ઇન્ટરફેસ, ફોન માટે અનુકૂલિત & Safari Mobile, Chrome
Mobile \\
\end{longtable}
}

\textbf{વિશેષતાઓ:}

\begin{itemize}
\tightlist
\item
  \textbf{નેવિગેશન}: આગળ, પાછળ, રિફ્રેશ બટન્સ
\item
  \textbf{બુકમાર્ક્સ}: પ્રિય વેબસાઇટ્સ સેવ કરો
\item
  \textbf{ટેબ્સ}: એક વિન્ડોમાં બહુવિધ પૃષ્ઠો
\item
  \textbf{સિક્યોરિટી}: HTTPS સપોર્ટ, પોપઅપ બ્લોકર્સ
\end{itemize}

\end{solutionbox}
\begin{mnemonicbox}
``બ્રાઉઝ કરો સલામત રીતે ઓનલાઇન''

\end{mnemonicbox}
\subsection*{પ્રશ્ન ૧(સ) [૭
ગુણ]}\label{uxaaauxab0uxab6uxaa8-uxae7uxab8-uxaed-uxa97uxaa3}

\textbf{LAN, MAN અને WAN ને ઉદાહરણો સાથે સમજાવો.}

\begin{solutionbox}


{\def\LTcaptype{none} % do not increment counter
\vspace{-5pt}
\captionof{table}{નેટવર્ક પ્રકારોની સરખામણી}
\vspace{-10pt}
\begin{longtable}[]{@{}lllll@{}}
\toprule\noalign{}
નેટવર્ક & કવરેજ & સ્પીડ & ઉદાહરણ & ખર્ચ \\
\midrule\noalign{}
\endhead
\bottomrule\noalign{}
\endlastfoot
\textbf{LAN} & બિલ્ડિંગ/કેમ્પસ & ઊંચી (100Mbps-1Gbps) & ઓફિસ નેટવર્ક & ઓછો \\
\textbf{MAN} & શહેર/મેટ્રોપોલિટન & મધ્યમ (10-100Mbps) & કેબલ ટીવી નેટવર્ક &
મધ્યમ \\
\textbf{WAN} & દેશ/વૈશ્વિક & બદલાતી (1-100Mbps) & ઇન્ટરનેટ & વધુ \\
\end{longtable}
}

\textbf{વિસ્તૃત સમજાવટ:}

\textbf{LAN (Local Area Network):}

\begin{itemize}
\tightlist
\item
  \textbf{કવરેજ}: બિલ્ડિંગ કે નાના વિસ્તારમાં
\item
  \textbf{ટેકનોલોજી}: Ethernet, Wi-Fi
\item
  \textbf{ઉદાહરણ}: કમ્પ્યુટર લેબ, ઘરેલું નેટવર્ક
\end{itemize}

\textbf{MAN (Metropolitan Area Network):}

\begin{itemize}
\tightlist
\item
  \textbf{કવરેજ}: શહેર કે મેટ્રોપોલિટન વિસ્તાર
\item
  \textbf{ટેકનોલોજી}: ફાઇબર ઓપ્ટિક, માઇક્રોવેવ
\item
  \textbf{ઉદાહરણ}: શહેરવ્યાપી કેબલ ઇન્ટરનેટ
\end{itemize}

\textbf{WAN (Wide Area Network):}

\begin{itemize}
\tightlist
\item
  \textbf{કવરેજ}: બહુવિધ શહેરો/દેશો
\item
  \textbf{ટેકનોલોજી}: સેટેલાઇટ, ફાઇબર ઓપ્ટિક
\item
  \textbf{ઉદાહરણ}: ઇન્ટરનેટ, બેંક ATM નેટવર્ક
\end{itemize}

\textbf{આકૃતિ:}

\begin{center}
\textbf{Mermaid Diagram (Code)}
\begin{verbatim}
{Shaded}
{Highlighting}[]
graph LR
    A[LAN {- બિલ્ડિંગ] {-}{-}{} B[MAN {-} શહેર]}
    B {-{-}{} C[WAN {-} વૈશ્વિક]}
    A {-{-}{} D[ઓફિસ નેટવર્ક]}
    B {-{-}{} E[શહેરી કેબલ ટીવી]}
    C {-{-}{} F[ઇન્ટરનેટ]}
{Highlighting}
{Shaded}
\end{verbatim}
\end{center}

\end{solutionbox}
\begin{mnemonicbox}
``લોકલ મેટ્રો વર્લ્ડ'' (LAN-MAN-WAN)

\end{mnemonicbox}
\subsection*{પ્રશ્ન ૧(સ અથવા) [૭
ગુણ]}\label{uxaaauxab0uxab6uxaa8-uxae7uxab8-uxa85uxaa5uxab5-uxaed-uxa97uxaa3}

\textbf{ડોસ અને યુનિક્સ ઓપરેટિંગ સિસ્ટમ વચ્ચે તફાવત લખો.}

\begin{solutionbox}


{\def\LTcaptype{none} % do not increment counter
\vspace{-5pt}
\captionof{table}{DOS વિ Unix સરખામણી}
\vspace{-10pt}
\begin{longtable}[]{@{}lll@{}}
\toprule\noalign{}
વિશેષતા & DOS & Unix \\
\midrule\noalign{}
\endhead
\bottomrule\noalign{}
\endlastfoot
\textbf{ઇન્ટરફેસ} & કમાન્ડ લાઇન (ટેક્સ્ટ-આધારિત) & કમાન્ડ લાઇન + GUI \\
\textbf{મલ્ટિ-યુઝર} & સિંગલ યુઝર & મલ્ટિ-યુઝર સપોર્ટ \\
\textbf{મલ્ટિટાસ્કિંગ} & મર્યાદિત & સંપૂર્ણ મલ્ટિટાસ્કિંગ \\
\textbf{સિક્યોરિટી} & મૂળભૂત & અદ્યતન સિક્યોરિટી \\
\textbf{ફાઇલ સિસ્ટમ} & FAT16/FAT32 & વિવિધ (ext3, ext4) \\
\textbf{કિંમત} & કોમર્શિયલ (Microsoft) & ફ્રી/ઓપન સોર્સ વેરિયન્ટ્સ \\
\end{longtable}
}

\textbf{મુખ્ય તફાવતો:}

\textbf{DOS (Disk Operating System):}

\begin{itemize}
\tightlist
\item
  \textbf{આર્કિટેક્ચર}: 16-bit, સિંગલ-યુઝર
\item
  \textbf{મેમરી}: 640KB પરંપરાગત મેમરી મર્યાદા
\item
  \textbf{કમાન્ડ્સ}: DIR, COPY, DEL
\item
  \textbf{ફાઇલ નેમિંગ}: 8.3 ફોર્મેટ મર્યાદા
\end{itemize}

\textbf{Unix:}

\begin{itemize}
\tightlist
\item
  \textbf{આર્કિટેક્ચર}: 32/64-bit, મલ્ટિ-યુઝર
\item
  \textbf{મેમરી}: અદ્યતન મેમરી મેનેજમેન્ટ
\item
  \textbf{કમાન્ડ્સ}: ls, cp, rm, grep
\item
  \textbf{ફાઇલ નેમિંગ}: કેસ-સેન્સિટિવ, લાંબા નામો
\end{itemize}

\textbf{ઉદાહરણો:}

\begin{itemize}
\tightlist
\item
  \textbf{DOS}: MS-DOS, PC-DOS
\item
  \textbf{Unix}: Linux, Solaris, AIX
\end{itemize}

\end{solutionbox}
\begin{mnemonicbox}
``DOS સરળ, Unix શક્તિશાળી'' (સિંગલ વિ મલ્ટિ-યુઝર)

\end{mnemonicbox}
\subsection*{પ્રશ્ન ૨(આ) [૩
ગુણ]}\label{uxaaauxab0uxab6uxaa8-uxae8uxa86-uxae9-uxa97uxaa3}

\textbf{ઓપરેટિંગ સિસ્ટમના લક્ષણોની યાદી આપો.}

\begin{solutionbox}


{\def\LTcaptype{none} % do not increment counter
\vspace{-5pt}
\captionof{table}{ઓપરેટિંગ સિસ્ટમની વિશેષતાઓ}
\vspace{-10pt}
\begin{longtable}[]{@{}ll@{}}
\toprule\noalign{}
વિશેષતા & વર્ણન \\
\midrule\noalign{}
\endhead
\bottomrule\noalign{}
\endlastfoot
\textbf{પ્રોસેસ મેનેજમેન્ટ} & પ્રોગ્રામ એક્ઝિક્યુશન કંટ્રોલ કરે \\
\textbf{મેમરી મેનેજમેન્ટ} & RAM કાર્યક્ષમ રીતે વહેંચે \\
\textbf{ફાઇલ મેનેજમેન્ટ} & ડેટા સ્ટોરેજ વ્યવસ્થિત કરે \\
\textbf{ડિવાઇસ મેનેજમેન્ટ} & હાર્ડવેર ડિવાઇસો કંટ્રોલ કરે \\
\end{longtable}
}

\textbf{મુખ્ય વિશેષતાઓ:}

\begin{itemize}
\tightlist
\item
  \textbf{યુઝર ઇન્ટરફેસ}: GUI અથવા કમાન્ડ લાઇન
\item
  \textbf{સિક્યોરિટી}: યુઝર ઓથેન્ટિકેશન, ઍક્સેસ કંટ્રોલ
\item
  \textbf{મલ્ટિટાસ્કિંગ}: એકસાથે બહુવિધ પ્રોગ્રામ્સ ચલાવો
\item
  \textbf{રિસોર્સ ઍલોકેશન}: CPU, મેમરી વિતરણ
\end{itemize}

\end{solutionbox}
\begin{mnemonicbox}
``પ્રોસેસ મેમરી ફાઇલ ડિવાઇસ''

\end{mnemonicbox}
\subsection*{પ્રશ્ન ૨(બ) [૪
ગુણ]}\label{uxaaauxab0uxab6uxaa8-uxae8uxaac-uxaea-uxa97uxaa3}

\textbf{હાફ ડુપ્લેક્સ અને ફુલ ડુપ્લેક્સ ટ્રાન્સમિશન મોડ્સ વ્યાખ્યાયિત લખો.}

\begin{solutionbox}


{\def\LTcaptype{none} % do not increment counter
\vspace{-5pt}
\captionof{table}{ટ્રાન્સમિશન મોડ્સની સરખામણી}
\vspace{-10pt}
\begin{longtable}[]{@{}llll@{}}
\toprule\noalign{}
મોડ & દિશા & ઉદાહરણ & કાર્યક્ષમતા \\
\midrule\noalign{}
\endhead
\bottomrule\noalign{}
\endlastfoot
\textbf{હાફ ડુપ્લેક્સ} & દ્વિદિશીય (એક સમયે એક) & વોકી-ટોકી & મધ્યમ \\
\textbf{ફુલ ડુપ્લેક્સ} & દ્વિદિશીય (એકસાથે) & ટેલિફોન & ઊંચી \\
\end{longtable}
}

\textbf{વ્યાખ્યાઓ:}

\textbf{હાફ ડુપ્લેક્સ:}

\begin{itemize}
\tightlist
\item
  \textbf{કોમ્યુનિકેશન}: બે-તરફી પણ એકસાથે નહીં
\item
  \textbf{ઉદાહરણ}: રેડિયો કોમ્યુનિકેશન, જૂના Ethernet hubs
\item
  \textbf{મર્યાદા}: વારો લેવાની જરૂર
\end{itemize}

\textbf{ફુલ ડુપ્લેક્સ:}

\begin{itemize}
\tightlist
\item
  \textbf{કોમ્યુનિકેશન}: બે-તરફી એકસાથે
\item
  \textbf{ઉદાહરણ}: આધુનિક Ethernet, ટેલિફોન કૉલ્સ
\item
  \textbf{ફાયદો}: રાહ જોવાનો સમય નથી
\end{itemize}

\textbf{આકૃતિ:}

\begin{verbatim}
હાફ ડુપ્લેક્સ:
A {-{-}{-}{-}{-} B  (A મોકલે)}
A {{-}{-}{-}{-}{-} B  (B મોકલે {-} A રાહ જુએ)}

ફુલ ડુપ્લેક્સ:
A {{-}{-}{-}{-} B  (બંને એકસાથે મોકલે/મેળવે)}
\end{verbatim}

\end{solutionbox}
\begin{mnemonicbox}
``હાફ રાહ જુએ, ફુલ વહે છે'' (હાફ=રાહ, ફુલ=એકસાથે)

\end{mnemonicbox}
\subsection*{પ્રશ્ન ૨(સ) [૭
ગુણ]}\label{uxaaauxab0uxab6uxaa8-uxae8uxab8-uxaed-uxa97uxaa3}

\textbf{ઓપન સોર્સ અને પ્રોપરાઇટરી સોફ્ટવેર વચ્ચેનો તફાવત.}

\begin{solutionbox}


{\def\LTcaptype{none} % do not increment counter
\vspace{-5pt}
\captionof{table}{ઓપન સોર્સ વિ પ્રોપરાઇટરી સોફ્ટવેર}
\vspace{-10pt}
\begin{longtable}[]{@{}lll@{}}
\toprule\noalign{}
પાસા & ઓપન સોર્સ & પ્રોપરાઇટરી \\
\midrule\noalign{}
\endhead
\bottomrule\noalign{}
\endlastfoot
\textbf{સોર્સ કોડ} & ફ્રીમાં ઉપલબ્ધ & છુપાયેલો/સુરક્ષિત \\
\textbf{કિંમત} & સામાન્ય રીતે ફ્રી & પેઇડ લાઇસન્સ \\
\textbf{મોડિફિકેશન} & મંજૂર & પ્રતિબંધિત \\
\textbf{સપોર્ટ} & કોમ્યુનિટી-આધારિત & વેન્ડર સપોર્ટ \\
\textbf{સિક્યોરિટી} & ટ્રાન્સપેરન્ટ & સિક્યોરિટી through obscurity \\
\textbf{ઉદાહરણો} & Linux, Firefox, Apache & Windows, MS Office \\
\end{longtable}
}

\textbf{વિસ્તૃત સરખામણી:}

\textbf{ઓપન સોર્સ સોફ્ટવેર:}

\begin{itemize}
\tightlist
\item
  \textbf{વ્યાખ્યા}: સોર્સ કોડ જાહેરમાં ઉપલબ્ધ
\item
  \textbf{લાઇસન્સિંગ}: GPL, MIT, Apache લાઇસન્સ
\item
  \textbf{ફાયદા}: ખર્ચ-અસરકારક, કસ્ટમાઇઝેબલ, પારદર્શક
\item
  \textbf{ઉદાહરણો}: LibreOffice, GIMP, MySQL
\end{itemize}

\textbf{પ્રોપરાઇટરી સોફ્ટવેર:}

\begin{itemize}
\tightlist
\item
  \textbf{વ્યાખ્યા}: વ્યક્તિ/કંપની દ્વારા માલિકી
\item
  \textbf{લાઇસન્સિંગ}: End User License Agreement (EULA)
\item
  \textbf{ફાયદા}: વ્યાવસાયિક સપોર્ટ, ગેરેંટીશુદા અપડેટ્સ
\item
  \textbf{ઉદાહરણો}: Adobe Photoshop, Oracle Database
\end{itemize}

\textbf{ફાયદા અને નુકસાનો:}

\textbf{ઓપન સોર્સ ફાયદા:} ફ્રી, લવચીક, કોમ્યુનિટી સપોર્ટ \textbf{ઓપન સોર્સ
નુકસાન:} મર્યાદિત વ્યાવસાયિક સપોર્ટ

\textbf{પ્રોપરાઇટરી ફાયદા:} વ્યાવસાયિક સપોર્ટ, વોરન્ટી \textbf{પ્રોપરાઇટરી
નુકસાન:} મોંઘું, વેન્ડર લોક-ઇન

\end{solutionbox}
\begin{mnemonicbox}
``ઓપન = જોવા માટે ફ્રી, પ્રોપરાઇટરી = વાપરવા માટે પૈસા
આપો''

\end{mnemonicbox}
\subsection*{પ્રશ્ન ૨(આ અથવા) [૩
ગુણ]}\label{uxaaauxab0uxab6uxaa8-uxae8uxa86-uxa85uxaa5uxab5-uxae9-uxa97uxaa3}

\textbf{RAM અને ROM વચ્ચે તફાવત લખો.}

\begin{solutionbox}


{\def\LTcaptype{none} % do not increment counter
\vspace{-5pt}
\captionof{table}{RAM વિ ROM સરખામણી}
\vspace{-10pt}
\begin{longtable}[]{@{}lll@{}}
\toprule\noalign{}
વિશેષતા & RAM & ROM \\
\midrule\noalign{}
\endhead
\bottomrule\noalign{}
\endlastfoot
\textbf{પૂર્ણ નામ} & Random Access Memory & Read Only Memory \\
\textbf{વોલેટિલિટી} & વોલેટાઇલ (ડેટા ગુમાવે) & નોન-વોલેટાઇલ (ડેટા જાળવે) \\
\textbf{ઍક્સેસ} & રીડ/રાઇટ & ફક્ત રીડ \\
\textbf{સ્પીડ} & ખૂબ ઝડપી & RAM કરતાં ધીમી \\
\end{longtable}
}

\textbf{મુખ્ય તફાવતો:}

\begin{itemize}
\tightlist
\item
  \textbf{હેતુ}: RAM અસ્થાયી સ્ટોરેજ માટે, ROM કાયમી માટે
\item
  \textbf{કિંમત}: RAM પ્રતિ GB વધુ મોંઘી
\item
  \textbf{વપરાશ}: RAM પ્રોગ્રામ્સ માટે, ROM ફર્મવેર માટે
\end{itemize}

\end{solutionbox}
\begin{mnemonicbox}
``RAM દોડે, ROM યાદ રાખે'' (અસ્થાયી વિ કાયમી)

\end{mnemonicbox}
\subsection*{પ્રશ્ન ૨(બ અથવા) [૪
ગુણ]}\label{uxaaauxab0uxab6uxaa8-uxae8uxaac-uxa85uxaa5uxab5-uxaea-uxa97uxaa3}

\textbf{ઉદાહરણ સાથે AND લોજિક ગેટ સમજાવો.}

\begin{solutionbox}

\textbf{AND ગેટ વ્યાખ્યા:} આઉટપુટ ત્યારે જ HIGH આવે જ્યારે બધા ઇનપુટ્સ HIGH હોય.

\textbf{ટ્રુથ ટેબલ:}

{\def\LTcaptype{none} % do not increment counter
\begin{longtable}[]{@{}lll@{}}
\toprule\noalign{}
ઇનપુટ A & ઇનપુટ B & આઉટપુટ (A AND B) \\
\midrule\noalign{}
\endhead
\bottomrule\noalign{}
\endlastfoot
0 & 0 & 0 \\
0 & 1 & 0 \\
1 & 0 & 0 \\
1 & 1 & 1 \\
\end{longtable}
}

\textbf{સિમ્બોલ:}

\begin{verbatim}
    A {-{-}{-}{-}}
           {{-}{-}{-}{-} આઉટપુટ}
    B {-{-}{-}{-}/}
\end{verbatim}

\textbf{ઉદાહરણ ઍપ્લિકેશન્સ:}

\begin{itemize}
\tightlist
\item
  \textbf{સિક્યોરિટી સિસ્ટમ}: દરવાજો ચાવી AND કાર્ડ બંનેથી ખુલે
\item
  \textbf{કાર સ્ટાર્ટિંગ}: એન્જિન ચાવી AND બ્રેક પર પગ બંનેથી ચાલે
\item
  \textbf{બુલિયન એક્સપ્રેશન}: Y = A · B અથવા Y = A \wedge B
\end{itemize}

\textbf{વાસ્તવિક જીવનનું ઉદાહરણ:} વોશિંગ મશીન ત્યારે જ ચાલે જ્યારે દરવાજો બંધ હોય
AND પાવર બટન દબાયેલ હોય.

\end{solutionbox}
\begin{mnemonicbox}
``બધા ઇનપુટ્સ સાચા = આઉટપુટ સાચો''

\end{mnemonicbox}
\subsection*{પ્રશ્ન ૨(સ અથવા) [૭
ગુણ]}\label{uxaaauxab0uxab6uxaa8-uxae8uxab8-uxa85uxaa5uxab5-uxaed-uxa97uxaa3}

\textbf{ઈથરનેટ કેબલ કલર કોડ સમજાવો.}

\begin{solutionbox}

\textbf{સ્ટાન્ડર્ડ: TIA/EIA-568B કલર કોડ}


{\def\LTcaptype{none} % do not increment counter
\vspace{-5pt}
\captionof{table}{વાયર કલર સિક્વન્સ}
\vspace{-10pt}
\begin{longtable}[]{@{}lll@{}}
\toprule\noalign{}
પિન & રંગ & કાર્ય \\
\midrule\noalign{}
\endhead
\bottomrule\noalign{}
\endlastfoot
1 & વાઇટ/ઓરેન્જ & ટ્રાન્સમિટ+ \\
2 & ઓરેન્જ & ટ્રાન્સમિટ- \\
3 & વાઇટ/ગ્રીન & રિસીવ+ \\
4 & બ્લુ & વાપરતા નથી \\
5 & વાઇટ/બ્લુ & વાપરતા નથી \\
6 & ગ્રીન & રિસીવ- \\
7 & વાઇટ/બ્રાઉન & વાપરતા નથી \\
8 & બ્રાઉન & વાપરતા નથી \\
\end{longtable}
}

\textbf{કેબલના પ્રકારો:}

\textbf{સ્ટ્રેઇટ-થ્રુ કેબલ (568B બંને છેડે):}

\begin{itemize}
\tightlist
\item
  \textbf{વપરાશ}: કમ્પ્યુટર થી સ્વિચ/હબ
\item
  \textbf{કલર સિક્વન્સ}: બંને છેડે સમાન
\end{itemize}

\textbf{ક્રોસ-ઓવર કેબલ (568A એક છેડે, 568B બીજે):}

\begin{itemize}
\tightlist
\item
  \textbf{વપરાશ}: કમ્પ્યુટર થી કમ્પ્યુટર સીધું
\item
  \textbf{પિન્સ સ્વેપ}: 1\leftrightarrow3, 2\leftrightarrow6
\end{itemize}

\textbf{વાયરિંગ આકૃતિ:}

\begin{verbatim}
RJ{-45 કનેક્ટર (568B):}
પિન 1: વાઇટ/ઓરેન્જ
પિન 2: ઓરેન્જ  
પિન 3: વાઇટ/ગ્રીન
પિન 4: બ્લુ
પિન 5: વાઇટ/બ્લુ
પિન 6: ગ્રીન
પિન 7: વાઇટ/બ્રાઉન
પિન 8: બ્રાઉન
\end{verbatim}

\textbf{તૈયારીના પગલાં:}

\begin{enumerate}
\tightlist
\item
  બાહ્ય જેકેટ સ્ટ્રિપ કરો (1 ઇંચ)
\item
  વાયર્સને કલર ક્રમમાં ગોઠવો
\item
  વાયર્સને સરખી કાપો
\item
  RJ-45 કનેક્ટરમાં નાખો
\item
  ક્રિમ્પિંગ ટૂલથી ક્રિમ્પ કરો
\end{enumerate}

\end{solutionbox}
\begin{mnemonicbox}
``વાઇટ ઓરેન્જ, ઓરેન્જ, વાઇટ ગ્રીન, બ્લુ, વાઇટ બ્લુ, ગ્રીન,
વાઇટ બ્રાઉન, બ્રાઉન''

\end{mnemonicbox}
\subsection*{પ્રશ્ન ૩(આ) [૩
ગુણ]}\label{uxaaauxab0uxab6uxaa8-uxae9uxa86-uxae9-uxa97uxaa3}

\textbf{વાયર્ડ અને વાયરલેસ કોમ્યુનિકેશનની સરખામણી લખો.}

\begin{solutionbox}


{\def\LTcaptype{none} % do not increment counter
\vspace{-5pt}
\captionof{table}{વાયર્ડ વિ વાયરલેસ કોમ્યુનિકેશન}
\vspace{-10pt}
\begin{longtable}[]{@{}lll@{}}
\toprule\noalign{}
પાસા & વાયર્ડ & વાયરલેસ \\
\midrule\noalign{}
\endhead
\bottomrule\noalign{}
\endlastfoot
\textbf{માધ્યમ} & કેબલ્સ (કોપર/ફાઇબર) & રેડિયો તરંગો/ઇન્ફ્રારેડ \\
\textbf{સ્પીડ} & વધુ (100Gbps સુધી) & ઓછી (1Gbps સુધી) \\
\textbf{સિક્યોરિટી} & વધુ સુરક્ષિત & ઓછી સુરક્ષિત \\
\textbf{મોબિલિટી} & મર્યાદિત & ઊંચી મોબિલિટી \\
\textbf{કિંમત} & વધુ ઇન્સ્ટોલેશન & ઓછી ઇન્સ્ટોલેશન \\
\textbf{ઇન્ટરફેરન્સ} & ન્યૂનતમ & સિગ્નલ ઇન્ટરફેરન્સ \\
\end{longtable}
}

\textbf{મુખ્ય મુદ્દા:}

\begin{itemize}
\tightlist
\item
  \textbf{વાયર્ડ}: વિશ્વસનીય, ઝડપી, સુરક્ષિત પણ મર્યાદિત મોબિલિટી
\item
  \textbf{વાયરલેસ}: મોબાઇલ, લવચીક પણ સિક્યોરિટીની ચિંતા
\end{itemize}

\end{solutionbox}
\begin{mnemonicbox}
``વાયર્સ ઝડપી, વાયરલેસ મુક્ત'' (સ્પીડ વિ મોબિલિટી)

\end{mnemonicbox}
\subsection*{પ્રશ્ન ૩(બ) [૪
ગુણ]}\label{uxaaauxab0uxab6uxaa8-uxae9uxaac-uxaea-uxa97uxaa3}

\textbf{કમ્પ્યુટર સિસ્ટમના વિવિધ પ્રકારોની ચર્ચા કરો.}

\begin{solutionbox}


{\def\LTcaptype{none} % do not increment counter
\vspace{-5pt}
\captionof{table}{કમ્પ્યુટર સિસ્ટમના પ્રકારો}
\vspace{-10pt}
\begin{longtable}[]{@{}llll@{}}
\toprule\noalign{}
પ્રકાર & સાઇઝ & પ્રોસેસિંગ પાવર & ઉદાહરણ \\
\midrule\noalign{}
\endhead
\bottomrule\noalign{}
\endlastfoot
\textbf{સુપરકમ્પ્યુટર} & રૂમ-સાઇઝ્ડ & અત્યંત ઊંચી & હવામાન આગાહી \\
\textbf{મેઇનફ્રેમ} & મોટી કેબિનેટ & ખૂબ ઊંચી & બેંક ટ્રાન્ઝેક્શન્સ \\
\textbf{મિનિકમ્પ્યુટર} & ડેસ્ક-સાઇઝ્ડ & મધ્યમ & નાના બિઝનેસ \\
\textbf{માઇક્રોકમ્પ્યુટર} & ડેસ્કટોપ/લેપટોપ & ઓછીથી મધ્યમ & વ્યક્તિગત વપરાશ \\
\end{longtable}
}

\textbf{વર્ગીકરણ:}

\textbf{સાઇઝ અને પાવર દ્વારા:}

\begin{itemize}
\tightlist
\item
  \textbf{સુપરકમ્પ્યુટર}: વૈજ્ઞાનિક ગણતરીઓ, સંશોધન
\item
  \textbf{મેઇનફ્રેમ}: મોટી સંસ્થાઓ, એકસાથે વધારે યુઝર્સ
\item
  \textbf{પર્સનલ કમ્પ્યુટર}: વ્યક્તિગત યુઝર્સ, ઓફિસ વર્ક
\item
  \textbf{એમ્બેડેડ સિસ્ટમ્સ}: ચોક્કસ કાર્યો (વોશિંગ મશીન)
\end{itemize}

\textbf{હેતુ દ્વારા:}

\begin{itemize}
\tightlist
\item
  \textbf{જનરલ પર્પઝ}: બહુમુખી, બહુવિધ ઍપ્લિકેશન્સ
\item
  \textbf{સ્પેશિયલ પર્પઝ}: સમર્પિત કાર્યો (ATM, ગેમિંગ કન્સોલ)
\end{itemize}

\end{solutionbox}
\begin{mnemonicbox}
``સુપર મેઇન મિની માઇક્રો'' (ઘટતા સાઇઝનો ક્રમ)

\end{mnemonicbox}
\subsection*{પ્રશ્ન ૩(સ) [૭
ગુણ]}\label{uxaaauxab0uxab6uxaa8-uxae9uxab8-uxaed-uxa97uxaa3}

\textbf{TDM, FDM, OFDM પર ટૂંકી નોંધ લખો.}

\begin{solutionbox}

\textbf{કાર્યક્ષમ કોમ્યુનિકેશન માટે મલ્ટિપ્લેક્સિંગ તકનીકો}


{\def\LTcaptype{none} % do not increment counter
\vspace{-5pt}
\captionof{table}{મલ્ટિપ્લેક્સિંગ સરખામણી}
\vspace{-10pt}
\begin{longtable}[]{@{}llll@{}}
\toprule\noalign{}
તકનીક & વિભાજન પદ્ધતિ & ઍપ્લિકેશન & ફાયદો \\
\midrule\noalign{}
\endhead
\bottomrule\noalign{}
\endlastfoot
\textbf{TDM} & સમય સ્લોટ્સ & ડિજિટલ ટેલિફોની & સરળ અમલીકરણ \\
\textbf{FDM} & ફ્રીક્વન્સી બેન્ડ્સ & રેડિયો/ટીવી બ્રોડકાસ્ટિંગ & એકસાથે
ટ્રાન્સમિશન \\
\textbf{OFDM} & બહુવિધ કેરિયર્સ & Wi-Fi, 4G/5G & ઊંચા ડેટા રેટ્સ \\
\end{longtable}
}

\textbf{ટાઇમ ડિવિઝન મલ્ટિપ્લેક્સિંગ (TDM):}

\begin{itemize}
\tightlist
\item
  \textbf{સિદ્ધાંત}: દરેક યુઝરને નિશ્ચિત સમય સ્લોટ મળે
\item
  \textbf{અમલીકરણ}: અનુક્રમિક ડેટા ટ્રાન્સમિશન
\item
  \textbf{ઉદાહરણ}: ડિજિટલ ટેલિફોન સિસ્ટમ્સ, GSM
\item
  \textbf{ફાયદો}: બેન્ડવિડ્થનો કાર્યક્ષમ ઉપયોગ
\end{itemize}

\textbf{ફ્રીક્વન્સી ડિવિઝન મલ્ટિપ્લેક્સિંગ (FDM):}

\begin{itemize}
\tightlist
\item
  \textbf{સિદ્ધાંત}: દરેક યુઝરને અનન્ય ફ્રીક્વન્સી બેન્ડ મળે
\item
  \textbf{અમલીકરણ}: એકસાથે ટ્રાન્સમિશન
\item
  \textbf{ઉદાહરણ}: FM રેડિયો, કેબલ ટીવી
\item
  \textbf{ફાયદો}: ટાઇમિંગ કોઓર્ડિનેશનની જરૂર નથી
\end{itemize}

\textbf{ઓર્થોગોનલ ફ્રીક્વન્સી ડિવિઝન મલ્ટિપ્લેક્સિંગ (OFDM):}

\begin{itemize}
\tightlist
\item
  \textbf{સિદ્ધાંત}: બહુવિધ ઓર્થોગોનલ સબકેરિયર્સ
\item
  \textbf{અમલીકરણ}: પેરેલલ ડેટા સ્ટ્રીમ્સ
\item
  \textbf{ઉદાહરણ}: Wi-Fi (802.11), LTE, DSL
\item
  \textbf{ફાયદો}: ઊંચી સ્પેક્ટ્રલ કાર્યક્ષમતા, ઇન્ટરફેરન્સ સામે મજબૂત
\end{itemize}

\textbf{આકૃતિ:}

\begin{center}
\textbf{Mermaid Diagram (Code)}
\begin{verbatim}
{Shaded}
{Highlighting}[]
graph TD
    A[ડેટા સ્ટ્રીમ] {-{-}{} B[TDM {-} સમય સ્લોટ્સ]}
    A {-{-}{} C[FDM {-} ફ્રીક્વન્સી બેન્ડ્સ]}
    A {-{-}{} D[OFDM {-} બહુવિધ કેરિયર્સ]}
    B {-{-}{} E["T1|T2|T3|T4"]}
    C {-{-}{} F[F1 + F2 + F3 + F4]}
    D {-{-}{} G[ઓર્થોગોનલ સબકેરિયર્સ]}
{Highlighting}
{Shaded}
\end{verbatim}
\end{center}

\textbf{ઍપ્લિકેશન્સ:}

\begin{itemize}
\tightlist
\item
  \textbf{TDM}: ISDN, T1/E1 લાઇન્સ
\item
  \textbf{FDM}: એનાલોગ ટીવી, રેડિયો
\item
  \textbf{OFDM}: આધુનિક વાયરલેસ સિસ્ટમ્સ
\end{itemize}

\end{solutionbox}
\begin{mnemonicbox}
``સમય ફ્રીક્વન્સી ઓર્થોગોનલ'' (TDM-FDM-OFDM)

\end{mnemonicbox}
\subsection*{પ્રશ્ન ૩(આ અથવા) [૩
ગુણ]}\label{uxaaauxab0uxab6uxaa8-uxae9uxa86-uxa85uxaa5uxab5-uxae9-uxa97uxaa3}

\textbf{FSK અને PSK ની ચર્ચા કરો.}

\begin{solutionbox}

\textbf{ડિજિટલ મોડ્યુલેશન તકનીકો}


{\def\LTcaptype{none} % do not increment counter
\vspace{-5pt}
\captionof{table}{FSK વિ PSK}
\vspace{-10pt}
\begin{longtable}[]{@{}lll@{}}
\toprule\noalign{}
પાસા & FSK & PSK \\
\midrule\noalign{}
\endhead
\bottomrule\noalign{}
\endlastfoot
\textbf{પેરામીટર} & ફ્રીક્વન્સી & ફેઝ \\
\textbf{કોમ્પ્લેક્સિટી} & સરળ & જટિલ \\
\textbf{નોઇઝ ઇમ્યુનિટી} & સારી & ઉત્તમ \\
\textbf{બેન્ડવિડ્થ} & વધુ & ઓછી \\
\end{longtable}
}

\textbf{FSK (Frequency Shift Keying):}

\begin{itemize}
\tightlist
\item
  \textbf{સિદ્ધાંત}: 0 અને 1 માટે અલગ ફ્રીક્વન્સીઝ
\item
  \textbf{અમલીકરણ}: `0' માટે f1, `1' માટે f2
\item
  \textbf{ઉદાહરણ}: કમ્પ્યુટર મોડેમ્સ, RFID
\end{itemize}

\textbf{PSK (Phase Shift Keying):}

\begin{itemize}
\tightlist
\item
  \textbf{સિદ્ધાંત}: ફેઝ ચેન્જેસ ડેટા દર્શાવે
\item
  \textbf{અમલીકરણ}: `0' માટે 0^\circ, `1' માટે 180^\circ
\item
  \textbf{ઉદાહરણ}: Wi-Fi, સેટેલાઇટ કોમ્યુનિકેશન
\end{itemize}

\end{solutionbox}
\begin{mnemonicbox}
``ફ્રીક્વન્સી શિફ્ટ, ફેઝ શિફ્ટ'' (FSK-PSK)

\end{mnemonicbox}
\subsection*{પ્રશ્ન ૩(બ અથવા) [૪
ગુણ]}\label{uxaaauxab0uxab6uxaa8-uxae9uxaac-uxa85uxaa5uxab5-uxaea-uxa97uxaa3}

\textbf{મલ્ટિટાસ્કિંગ અને મલ્ટિપ્રોગ્રામિંગ OS વચ્ચે તફાવત લખો.}

\begin{solutionbox}


{\def\LTcaptype{none} % do not increment counter
\vspace{-5pt}
\captionof{table}{મલ્ટિટાસ્કિંગ વિ મલ્ટિપ્રોગ્રામિંગ}
\vspace{-10pt}
\begin{longtable}[]{@{}lll@{}}
\toprule\noalign{}
વિશેષતા & મલ્ટિટાસ્કિંગ & મલ્ટિપ્રોગ્રામિંગ \\
\midrule\noalign{}
\endhead
\bottomrule\noalign{}
\endlastfoot
\textbf{યુઝર ઇન્ટરેક્શન} & ઇન્ટરેક્ટિવ & બેચ પ્રોસેસિંગ \\
\textbf{રિસ્પોન્સ ટાઇમ} & ઝડપી & ધીમી \\
\textbf{CPU શેરિંગ} & ટાઇમ સ્લાઇસિંગ & જોબ સ્વિચિંગ \\
\textbf{ઉદાહરણ} & Windows, Linux & પ્રારંભિક મેઇનફ્રેમ્સ \\
\end{longtable}
}

\textbf{મલ્ટિટાસ્કિંગ:}

\begin{itemize}
\tightlist
\item
  \textbf{વ્યાખ્યા}: બહુવિધ કાર્યો દેખીતી રીતે એકસાથે ચાલે
\item
  \textbf{પદ્ધતિ}: ઝડપી સ્વિચિંગ સાથે ટાઇમ શેરિંગ
\item
  \textbf{યુઝર અનુભવ}: ઇન્ટરેક્ટિવ, પ્રતિસાદી
\item
  \textbf{પ્રકારો}: પ્રીએમ્પ્ટિવ, કોઓપરેટિવ
\end{itemize}

\textbf{મલ્ટિપ્રોગ્રામિંગ:}

\begin{itemize}
\tightlist
\item
  \textbf{વ્યાખ્યા}: મેમરીમાં બહુવિધ પ્રોગ્રામ્સ
\item
  \textbf{પદ્ધતિ}: I/O ઓપરેશન્સ દરમિયાન CPU સ્વિચ કરે
\item
  \textbf{યુઝર અનુભવ}: બેચ જોબ પ્રોસેસિંગ
\item
  \textbf{હેતુ}: CPU ઉપયોગિતા સુધારો
\end{itemize}

\end{solutionbox}
\begin{mnemonicbox}
``ટાસ્ક્સ ઇન્ટરેક્ટિવ, પ્રોગ્રામ્સ બેચ્ડ''

\end{mnemonicbox}
\subsection*{પ્રશ્ન ૩(સ અથવા) [૭
ગુણ]}\label{uxaaauxab0uxab6uxaa8-uxae9uxab8-uxa85uxaa5uxab5-uxaed-uxa97uxaa3}

\textbf{નેટવર્ક ટોપોલોજી પર ટૂંકી નોંધ લખો.}

\begin{solutionbox}

\textbf{નેટવર્ક ટોપોલોજીના પ્રકારો અને લાક્ષણિકતાઓ}


{\def\LTcaptype{none} % do not increment counter
\vspace{-5pt}
\captionof{table}{ટોપોલોજી સરખામણી}
\vspace{-10pt}
\begin{longtable}[]{@{}lllll@{}}
\toprule\noalign{}
ટોપોલોજી & માળખું & ફાયદા & નુકસાન & કિંમત \\
\midrule\noalign{}
\endhead
\bottomrule\noalign{}
\endlastfoot
\textbf{બસ} & રેખીય & સરળ, કિફાયતી & સિંગલ પોઇન્ટ ફેઇલ્યુર & ઓછી \\
\textbf{સ્ટાર} & સેન્ટ્રલ હબ & ટ્રબલશૂટિંગ સરળ & હબ ફેઇલ થાય તો બધાને અસર &
મધ્યમ \\
\textbf{રિંગ} & વર્તુળાકાર & સમાન ઍક્સેસ & બ્રેક નેટવર્કને અસર કરે & મધ્યમ \\
\textbf{મેશ} & આંતર-જોડાયેલ & ઊંચી વિશ્વસનીયતા & જટિલ, મોંઘું & ઊંચી \\
\textbf{હાઇબ્રિડ} & મિશ્રિત & લવચીક & જટિલ મેનેજમેન્ટ & બદલાતી \\
\end{longtable}
}

\textbf{વિસ્તૃત વર્ણનો:}

\textbf{બસ ટોપોલોજી:}

\begin{itemize}
\tightlist
\item
  \textbf{માળખું}: સિંગલ બેકબોન કેબલ
\item
  \textbf{ટર્મિનેશન}: બંને છેડે જરૂરી
\item
  \textbf{ઉદાહરણ}: પ્રારંભિક Ethernet (10BASE2)
\item
  \textbf{ફેઇલ્યુર ઇમ્પેક્ટ}: કેબલ તૂટે તો આખું નેટવર્ક બંધ
\end{itemize}

\textbf{સ્ટાર ટોપોલોજી:}

\begin{itemize}
\tightlist
\item
  \textbf{માળખું}: સેન્ટ્રલ સ્વિચ/હબ સાથે સ્પોક્સ
\item
  \textbf{સ્કેલેબિલિટી}: નોડ્સ ઉમેરવા/દૂર કરવા સરળ
\item
  \textbf{ઉદાહરણ}: આધુનિક Ethernet નેટવર્ક્સ
\item
  \textbf{ફેઇલ્યુર ઇમ્પેક્ટ}: ફક્ત અસરગ્રસ્ત નોડ ફેઇલ થાય
\end{itemize}

\textbf{રિંગ ટોપોલોજી:}

\begin{itemize}
\tightlist
\item
  \textbf{માળખું}: વર્તુળમાં નોડ્સ જોડાયેલ
\item
  \textbf{ડેટા ફ્લો}: એકદિશીય ટોકન પેસિંગ
\item
  \textbf{ઉદાહરણ}: Token Ring, FDDI
\item
  \textbf{ફેઇલ્યુર ઇમ્પેક્ટ}: સિંગલ બ્રેક નેટવર્ક બંધ કરે
\end{itemize}

\textbf{મેશ ટોપોલોજી:}

\begin{itemize}
\tightlist
\item
  \textbf{માળખું}: દરેક નોડ બીજા બધા સાથે જોડાયેલ
\item
  \textbf{પ્રકારો}: ફુલ મેશ, પાર્શિયલ મેશ
\item
  \textbf{ઉદાહરણ}: ઇન્ટરનેટ બેકબોન, મિલિટરી નેટવર્ક્સ
\item
  \textbf{વિશ્વસનીયતા}: બહુવિધ પાથ ઉપલબ્ધ
\end{itemize}

\textbf{હાઇબ્રિડ ટોપોલોજી:}

\begin{itemize}
\tightlist
\item
  \textbf{માળખું}: ટોપોલોજીઓનું મિશ્રણ
\item
  \textbf{ઉદાહરણ}: સ્ટાર-બસ, સ્ટાર-રિંગ
\item
  \textbf{લવચીકતા}: દરેક પ્રકારની શ્રેષ્ઠ વિશેષતાઓ
\end{itemize}

\textbf{આકૃતિ:}

\begin{center}
\textbf{Mermaid Diagram (Code)}
\begin{verbatim}
{Shaded}
{Highlighting}[]
graph TD
    A[નેટવર્ક ટોપોલોજીઓ] {-{-}{} B[બસ]}
    A {-{-}{} C[સ્ટાર]}
    A {-{-}{} D[રિંગ]}
    A {-{-}{} E[મેશ]}
    A {-{-}{} F[હાઇબ્રિડ]}
    
    B {-{-}{} G[રેખીય જોડાણ]}
    C {-{-}{} H[સેન્ટ્રલ હબ]}
    D {-{-}{} I[વર્તુળાકાર જોડાણ]}
    E {-{-}{} J[સંપૂર્ણ આંતર{-}જોડાણ]}
    F {-{-}{} K[મિશ્રિત માળખું]}
{Highlighting}
{Shaded}
\end{verbatim}
\end{center}

\textbf{પસંદગીના માપદંડો:}

\begin{itemize}
\tightlist
\item
  \textbf{કિંમત}: બસ \textless{} સ્ટાર \textless{} રિંગ \textless{} મેશ
\item
  \textbf{વિશ્વસનીયતા}: બસ \textless{} રિંગ \textless{} સ્ટાર \textless{}
  મેશ
\item
  \textbf{સ્કેલેબિલિટી}: રિંગ \textless{} બસ \textless{} સ્ટાર \textless{} મેશ
\end{itemize}

\end{solutionbox}
\begin{mnemonicbox}
``બસ સ્ટાર રિંગ મેશ હાઇબ્રિડ'' (વધતી જટિલતા)

\end{mnemonicbox}
\subsection*{પ્રશ્ન ૪(આ) [૩
ગુણ]}\label{uxaaauxab0uxab6uxaa8-uxaeauxa86-uxae9-uxa97uxaa3}

\textbf{સ્વિચ સમજાવો.}

\begin{solutionbox}

\textbf{નેટવર્ક સ્વિચ વ્યાખ્યા અને કાર્યો}


{\def\LTcaptype{none} % do not increment counter
\vspace{-5pt}
\captionof{table}{સ્વિચની લાક્ષણિકતાઓ}
\vspace{-10pt}
\begin{longtable}[]{@{}ll@{}}
\toprule\noalign{}
વિશેષતા & વર્ણન \\
\midrule\noalign{}
\endhead
\bottomrule\noalign{}
\endlastfoot
\textbf{કાર્ય} & LAN માં ડિવાઇસો કનેક્ટ કરે \\
\textbf{લેયર} & ડેટા લિંક લેયર (લેયર 2) \\
\textbf{પદ્ધતિ} & MAC એડ્રેસ લર્નિંગ \\
\textbf{કોલિઝન} & કોલિઝન દૂર કરે \\
\end{longtable}
}

\textbf{મુખ્ય વિશેષતાઓ:}

\begin{itemize}
\tightlist
\item
  \textbf{MAC એડ્રેસ ટેબલ}: ડિવાઇસ એડ્રેસ શીખે અને સ્ટોર કરે
\item
  \textbf{ફુલ ડુપ્લેક્સ}: એકસાથે મોકલવું/મેળવવું
\item
  \textbf{ડેડિકેટેડ બેન્ડવિડ્થ}: દરેક પોર્ટને સંપૂર્ણ બેન્ડવિડ્થ મળે
\item
  \textbf{VLAN સપોર્ટ}: વર્ચ્યુઅલ નેટવર્ક સેગ્રિગેશન
\end{itemize}

\textbf{કાર્યો:}

\begin{itemize}
\tightlist
\item
  \textbf{ફ્રેમ ફોરવર્ડિંગ}: ચોક્કસ પોર્ટને ડેટા મોકલે
\item
  \textbf{એડ્રેસ લર્નિંગ}: MAC એડ્રેસ ટેબલ બનાવે
\item
  \textbf{લૂપ પ્રિવેન્શન}: સ્પેનિંગ ટ્રી પ્રોટોકોલ
\end{itemize}

\end{solutionbox}
\begin{mnemonicbox}
``સ્વિચ MAC એડ્રેસ શીખે''

\end{mnemonicbox}
\subsection*{પ્રશ્ન ૪(બ) [૪
ગુણ]}\label{uxaaauxab0uxab6uxaa8-uxaeauxaac-uxaea-uxa97uxaa3}

\textbf{સાયબરથ્રેટને ઉદાહરણ સાથે વ્યાખ્યાયિત કરો.}

\begin{solutionbox}

\textbf{સાયબરથ્રેટ વ્યાખ્યા:} કમ્પ્યુટર સિસ્ટમને નુકસાન, વિક્ષેપ અથવા અનધિકૃત પ્રવેશ
મેળવવાનો દુષ્ટ પ્રયાસ.


{\def\LTcaptype{none} % do not increment counter
\vspace{-5pt}
\captionof{table}{સાયબરથ્રેટના પ્રકારો}
\vspace{-10pt}
\begin{longtable}[]{@{}llll@{}}
\toprule\noalign{}
પ્રકાર & પદ્ધતિ & ઉદાહરણ & અસર \\
\midrule\noalign{}
\endhead
\bottomrule\noalign{}
\endlastfoot
\textbf{મેલવેર} & દુષ્ટ સોફ્ટવેર & વાયરસ, ટ્રોજન & ડેટા કરપ્શન \\
\textbf{ફિશિંગ} & નકલી ઇમેઇલ્સ/વેબસાઇટ્સ & નકલી બેંક ઇમેઇલ્સ & આઇડેન્ટિટી ચોરી \\
\textbf{રેન્સમવેર} & ફાઇલો એન્ક્રિપ્ટ કરે & WannaCry એટેક & આર્થિક નુકસાન \\
\textbf{DDoS} & ટ્રાફિક ઓવરલોડ & સર્વર ફ્લડિંગ & સેવા ડિસરપ્શન \\
\end{longtable}
}

\textbf{ઉદાહરણ - ફિશિંગ એટેક:}

\begin{itemize}
\tightlist
\item
  \textbf{પદ્ધતિ}: ``બેંક'' તરફથી નકલી ઇમેઇલ
\item
  \textbf{વિનંતી}: લૉગિન ક્રેડેન્શિયલ્સ
\item
  \textbf{પરિણામ}: એકાઉન્ટ કોમ્પ્રોમાઇઝ
\item
  \textbf{પ્રિવેન્શન}: મોકલનારની પ્રામાણિકતા ચકાસો
\end{itemize}

\textbf{સામાન્ય સંકેતો:}

\begin{itemize}
\tightlist
\item
  \textbf{શંકાસ્પદ ઇમેઇલ્સ}: અજાણ્યા મોકલનારા, તાત્કાલિક વિનંતીઓ
\item
  \textbf{અસામાન્ય સિસ્ટમ વર્તન}: ધીમી કામગીરી, પોપઅપ્સ
\item
  \textbf{અનધિકૃત પ્રવેશ}: બદલાયેલા પાસવર્ડ્સ, નવી ફાઇલો
\end{itemize}

\end{solutionbox}
\begin{mnemonicbox}
``સાયબર ક્રિમિનલ્સ ચેઓસ ક્રિએટ કરે'' (ખતરાઓ નુકસાન કરે)

\end{mnemonicbox}
\subsection*{પ્રશ્ન ૪(સ) [૭
ગુણ]}\label{uxaaauxab0uxab6uxaa8-uxaeauxab8-uxaed-uxa97uxaa3}

\textbf{TCP/IP અને OSI નેટવર્કિંગ મોડલ્સની સરખામણી કરો.}

\begin{solutionbox}


{\def\LTcaptype{none} % do not increment counter
\vspace{-5pt}
\captionof{table}{TCP/IP વિ OSI મોડલ સરખામણી}
\vspace{-10pt}
\begin{longtable}[]{@{}llll@{}}
\toprule\noalign{}
OSI લેયર & OSI કાર્ય & TCP/IP લેયર & TCP/IP કાર્ય \\
\midrule\noalign{}
\endhead
\bottomrule\noalign{}
\endlastfoot
\textbf{એપ્લિકેશન} & યુઝર ઇન્ટરફેસ & \textbf{એપ્લિકેશન} & યુઝર સેવાઓ \\
\textbf{પ્રેઝન્ટેશન} & ડેટા ફોર્મેટિંગ & \textbf{એપ્લિકેશન} & (સંયુક્ત) \\
\textbf{સેશન} & સેશન મેનેજમેન્ટ & \textbf{એપ્લિકેશન} & (સંયુક્ત) \\
\textbf{ટ્રાન્સપોર્ટ} & વિશ્વસનીય ડિલિવરી & \textbf{ટ્રાન્સપોર્ટ} & એન્ડ-ટુ-એન્ડ
ડિલિવરી \\
\textbf{નેટવર્ક} & રાઉટિંગ & \textbf{ઇન્ટરનેટ} & IP એડ્રેસિંગ \\
\textbf{ડેટા લિંક} & ફ્રેમ હેન્ડલિંગ & \textbf{નેટવર્ક એક્સેસ} & ફિઝિકલ
ટ્રાન્સમિશન \\
\textbf{ફિઝિકલ} & ઇલેક્ટ્રિકલ સિગ્નલ્સ & \textbf{નેટવર્ક એક્સેસ} & (સંયુક્ત) \\
\end{longtable}
}

\textbf{મુખ્ય તફાવતો:}

\textbf{OSI મોડલ (7 લેયર્સ):}

\begin{itemize}
\tightlist
\item
  \textbf{હેતુ}: થિયોરેટિકલ રેફરન્સ મોડલ
\item
  \textbf{ડેવલપમેન્ટ}: ISO સ્ટાન્ડર્ડ
\item
  \textbf{લેયર્સ}: સ્પષ્ટ રીતે અલગ કાર્યો
\item
  \textbf{વપરાશ}: શિક્ષણ, ટ્રબલશૂટિંગ
\end{itemize}

\textbf{TCP/IP મોડલ (4 લેયર્સ):}

\begin{itemize}
\tightlist
\item
  \textbf{હેતુ}: પ્રેક્ટિકલ અમલીકરણ
\item
  \textbf{ડેવલપમેન્ટ}: DARPA/ઇન્ટરનેટ
\item
  \textbf{લેયર્સ}: સંયુક્ત કાર્યક્ષમતા
\item
  \textbf{વપરાશ}: ઇન્ટરનેટ, વાસ્તવિક નેટવર્ક્સ
\end{itemize}

\textbf{ફાયદા:}

\textbf{OSI મોડલ:}

\begin{itemize}
\tightlist
\item
  \textbf{સ્ટાન્ડર્ડાઇઝેશન}: યુનિવર્સલ રેફરન્સ
\item
  \textbf{ટ્રબલશૂટિંગ}: લેયર-બાય-લેયર વિશ્લેષણ
\item
  \textbf{શિક્ષણ}: સ્પષ્ટ કન્સેપ્ટ સેપરેશન
\end{itemize}

\textbf{TCP/IP મોડલ:}

\begin{itemize}
\tightlist
\item
  \textbf{સરળતા}: ઓછી લેયર્સ
\item
  \textbf{પ્રેક્ટિકલિટી}: ઇન્ટરનેટ-પ્રુવન
\item
  \textbf{લવચીકતા}: પ્રોટોકોલ ઇન્ડિપેન્ડન્સ
\end{itemize}

\textbf{પ્રોટોકોલ ઉદાહરણો:}

\begin{itemize}
\tightlist
\item
  \textbf{OSI}: કન્સેપ્ચ્યુઅલ ફ્રેમવર્ક
\item
  \textbf{TCP/IP}: HTTP, FTP, TCP, UDP, IP
\end{itemize}

\textbf{આકૃતિ:}

\begin{center}
\textbf{Mermaid Diagram (Code)}
\begin{verbatim}
{Shaded}
{Highlighting}[]
graph TD
    A[OSI {- 7 લેયર્સ] {-}{-}{} B[એપ્લિકેશન]}
    A {-{-}{} C[પ્રેઝન્ટેશન]  }
    A {-{-}{} D[સેશન]}
    A {-{-}{} E[ટ્રાન્સપોર્ટ]}
    A {-{-}{} F[નેટવર્ક]}
    A {-{-}{} G[ડેટા લિંક]}
    A {-{-}{} H[ફિઝિકલ]}
    
    I[TCP/IP {- 4 લેયર્સ] {-}{-}{} J[એપ્લિકેશન]}
    I {-{-}{} K[ટ્રાન્સપોર્ટ]}
    I {-{-}{} L[ઇન્ટરનેટ]}
    I {-{-}{} M[નેટવર્ક એક્સેસ]}
{Highlighting}
{Shaded}
\end{verbatim}
\end{center}

\end{solutionbox}
\begin{mnemonicbox}
``OSI પરફેક્ટ થિયોરી, TCP/IP પ્રેક્ટિકલ રિયાલિટી''

\end{mnemonicbox}
\subsection*{પ્રશ્ન ૪(આ અથવા) [૩
ગુણ]}\label{uxaaauxab0uxab6uxaa8-uxaeauxa86-uxa85uxaa5uxab5-uxae9-uxa97uxaa3}

\textbf{સાયબર સુરક્ષાના મુખ્ય ઉદ્દેશો લખો.}

\begin{solutionbox}


{\def\LTcaptype{none} % do not increment counter
\vspace{-5pt}
\captionof{table}{સાયબર સિક્યોરિટી ઉદ્દેશ્યો (CIA ટ્રાયડ)}
\vspace{-10pt}
\begin{longtable}[]{@{}
  >{\raggedright\arraybackslash}p{(\linewidth - 4\tabcolsep) * \real{0.3600}}
  >{\raggedright\arraybackslash}p{(\linewidth - 4\tabcolsep) * \real{0.2800}}
  >{\raggedright\arraybackslash}p{(\linewidth - 4\tabcolsep) * \real{0.3600}}@{}}
\toprule\noalign{}
\begin{minipage}[b]{\linewidth}\raggedright
ઉદ્દેશ્ય
\end{minipage} & \begin{minipage}[b]{\linewidth}\raggedright
વર્ણન
\end{minipage} & \begin{minipage}[b]{\linewidth}\raggedright
ઉદાહરણ
\end{minipage} \\
\midrule\noalign{}
\endhead
\bottomrule\noalign{}
\endlastfoot
\textbf{ગુપ્તતા (Confidentiality)} & અનધિકૃત ઍક્સેસથી ડેટા સુરક્ષિત કરો &
એન્ક્રિપ્શન, પાસવર્ડ્સ \\
\textbf{અખંડતા (Integrity)} & ડેટાની ચોકસાઈ અને સંપૂર્ણતા સુનિશ્ચિત કરો & ડિજિટલ
સિગ્નેચર્સ, ચેકસમ્સ \\
\textbf{ઉપલબ્ધતા (Availability)} & સિસ્ટમની પહોંચ સુનિશ્ચિત કરો & બેકઅપ
સિસ્ટમ્સ, રિડન્ડન્સી \\
\end{longtable}
}

\textbf{વધારાના ઉદ્દેશ્યો:}

\begin{itemize}
\tightlist
\item
  \textbf{ઓથેન્ટિકેશન}: યુઝર આઇડેન્ટિટી ચકાસો
\item
  \textbf{ઓથોરાઇઝેશન}: ઍક્સેસ રાઇટ્સ કંટ્રોલ કરો
\item
  \textbf{નોન-રિપ્યુડિએશન}: ક્રિયાઓનો ઇનકાર અટકાવો
\end{itemize}

\end{solutionbox}
\begin{mnemonicbox}
``CIA ડેટાને પ્રોટેક્ટ કરે''
(Confidentiality-Integrity-Availability)

\end{mnemonicbox}
\subsection*{પ્રશ્ન ૪(બ અથવા) [૪
ગુણ]}\label{uxaaauxab0uxab6uxaa8-uxaeauxaac-uxa85uxaa5uxab5-uxaea-uxa97uxaa3}

\textbf{નેટવર્કિંગમાં વપરાતા નવિવિધ પ્રકારના નેટવર્કિંગ ઉપકરણોની યાદી બનાવો.}

\begin{solutionbox}


{\def\LTcaptype{none} % do not increment counter
\vspace{-5pt}
\captionof{table}{નેટવર્કિંગ ઉપકરણો}
\vspace{-10pt}
\begin{longtable}[]{@{}llll@{}}
\toprule\noalign{}
ઉપકરણ & લેયર & કાર્ય & ઉદાહરણ વપરાશ \\
\midrule\noalign{}
\endhead
\bottomrule\noalign{}
\endlastfoot
\textbf{હબ} & ફિઝિકલ & સિગ્નલ રિપીટર & લેગસી નેટવર્ક્સ \\
\textbf{સ્વિચ} & ડેટા લિંક & ફ્રેમ ફોરવર્ડિંગ & LAN કનેક્ટિવિટી \\
\textbf{રાઉટર} & નેટવર્ક & પેકેટ રાઉટિંગ & ઇન્ટરનેટ કનેક્શન \\
\textbf{બ્રિજ} & ડેટા લિંક & નેટવર્ક સેગ્મેન્ટેશન & LAN એક્સટેન્શન \\
\textbf{ગેટવે} & ઓલ લેયર્સ & પ્રોટોકોલ કન્વર્ઝન & નેટવર્ક ઇન્ટરકનેક્શન \\
\textbf{રિપીટર} & ફિઝિકલ & સિગ્નલ એમ્પ્લિફિકેશન & કેબલ એક્સટેન્શન \\
\textbf{એક્સેસ પોઇન્ટ} & ડેટા લિંક & વાયરલેસ કનેક્ટિવિટી & Wi-Fi નેટવર્ક્સ \\
\textbf{ફાયરવોલ} & નેટવર્ક+ & સિક્યોરિટી ફિલ્ટરિંગ & નેટવર્ક પ્રોટેક્શન \\
\end{longtable}
}

\textbf{કાર્યો:}

\begin{itemize}
\tightlist
\item
  \textbf{કનેક્ટિવિટી}: હબ, સ્વિચ, બ્રિજ
\item
  \textbf{રાઉટિંગ}: રાઉટર, ગેટવે
\item
  \textbf{સિક્યોરિટી}: ફાયરવોલ, પ્રોક્સી
\item
  \textbf{વાયરલેસ}: એક્સેસ પોઇન્ટ, વાયરલેસ રાઉટર
\end{itemize}

\end{solutionbox}
\begin{mnemonicbox}
``હબ્સ સ્વિચ રાઉટ બ્રિજ ગેટવે''

\end{mnemonicbox}
\subsection*{પ્રશ્ન ૪(સ અથવા) [૭
ગુણ]}\label{uxaaauxab0uxab6uxaa8-uxaeauxab8-uxa85uxaa5uxab5-uxaed-uxa97uxaa3}

\textbf{વિવિધ પ્રકારના સુરક્ષા હુમલાઓ લખો.}

\begin{solutionbox}

\textbf{સિક્યોરિટી એટેક્સનું વર્ગીકરણ}


{\def\LTcaptype{none} % do not increment counter
\vspace{-5pt}
\captionof{table}{એટેક પ્રકારો અને લાક્ષણિકતાઓ}
\vspace{-10pt}
\begin{longtable}[]{@{}lllll@{}}
\toprule\noalign{}
એટેક પ્રકાર & પદ્ધતિ & લક્ષ્ય & ઉદાહરણ & પ્રિવેન્શન \\
\midrule\noalign{}
\endhead
\bottomrule\noalign{}
\endlastfoot
\textbf{પેસિવ} & છૂપું સાંભળવું & માહિતી & ટ્રાફિક એનાલિસિસ & એન્ક્રિપ્શન \\
\textbf{એક્ટિવ} & સિસ્ટમ મોડિફિકેશન & અખંડતા & ડેટા ઓલ્ટરેશન & ઓથેન્ટિકેશન \\
\textbf{ફિઝિકલ} & હાર્ડવેર ઍક્સેસ & ઉપકરણ & ડિવાઇસ ચોરી & ફિઝિકલ
સિક્યોરિટી \\
\textbf{સોશિયલ એન્જિનિયરિંગ} & મનુષ્ય મેનિપ્યુલેશન & યુઝર્સ & ફિશિંગ & યુઝર
એજ્યુકેશન \\
\end{longtable}
}

\textbf{વિસ્તૃત એટેક કેટેગરીઝ:}

\textbf{1. નેટવર્ક એટેક્સ:}

\begin{itemize}
\tightlist
\item
  \textbf{મેન-ઇન-ધ-મિડલ}: કોમ્યુનિકેશન ઇન્ટરસેપ્ટ કરો
\item
  \textbf{DDoS}: સર્વરને ટ્રાફિકથી ભરાવો
\item
  \textbf{પેકેટ સ્નિફિંગ}: નેટવર્ક ડેટા કેપ્ચર કરો
\item
  \textbf{IP સ્પૂફિંગ}: નકલી સોર્સ એડ્રેસ
\end{itemize}

\textbf{2. એપ્લિકેશન એટેક્સ:}

\begin{itemize}
\tightlist
\item
  \textbf{SQL ઇન્જેક્શન}: ડેટાબેઝ મેનિપ્યુલેશન
\item
  \textbf{ક્રોસ-સાઇટ સ્ક્રિપ્ટિંગ (XSS)}: વેબ વલ્નરેબિલિટી
\item
  \textbf{બફર ઓવરફ્લો}: મેમરી કરપ્શન
\item
  \textbf{ઝીરો-ડે એક્સપ્લોઇટ્સ}: અજાણ્યા વલ્નરેબિલિટીઝ
\end{itemize}

\textbf{3. મેલવેર એટેક્સ:}

\begin{itemize}
\tightlist
\item
  \textbf{વાયરસ}: સેલ્ફ-રેપ્લિકેટિંગ કોડ
\item
  \textbf{વોર્મ}: નેટવર્ક-સ્પ્રેડિંગ મેલવેર
\item
  \textbf{ટ્રોજન}: છદ્મવેશી દુષ્ટ સોફ્ટવેર
\item
  \textbf{રેન્સમવેર}: પેમેન્ટ માટે ડેટા એન્ક્રિપ્શન
\end{itemize}

\textbf{4. સોશિયલ એન્જિનિયરિંગ:}

\begin{itemize}
\tightlist
\item
  \textbf{ફિશિંગ}: નકલી ઇમેઇલ્સ/વેબસાઇટ્સ
\item
  \textbf{પ્રીટેક્સ્ટિંગ}: ખોટા સિનારિયો
\item
  \textbf{બેટિંગ}: દુષ્ટ ડાઉનલોડ્સ
\item
  \textbf{ટેઇલગેટિંગ}: ફિઝિકલ ઍક્સેસ ફોલોઇંગ
\end{itemize}

\textbf{5. ક્રિપ્ટોગ્રાફિક એટેક્સ:}

\begin{itemize}
\tightlist
\item
  \textbf{બ્રુટ ફોર્સ}: બધા કોમ્બિનેશન્સ ટ્રાય કરો
\item
  \textbf{ડિક્શનરી એટેક}: કોમન પાસવર્ડ્સ
\item
  \textbf{રેઇનબો ટેબલ્સ}: પ્રી-કમ્પ્યુટેડ હેશેસ
\item
  \textbf{સાઇડ-ચેનલ}: ઇન્ફોર્મેશન લીકેજ
\end{itemize}

\textbf{એટેક વેક્ટર્સ:}

\begin{itemize}
\tightlist
\item
  \textbf{એક્સટર્નલ}: ઇન્ટરનેટ-આધારિત એટેક્સ
\item
  \textbf{ઇન્ટર્નલ}: ઇનસાઇડર થ્રેટ્સ
\item
  \textbf{ફિઝિકલ}: ડાયરેક્ટ હાર્ડવેર ઍક્સેસ
\item
  \textbf{વાયરલેસ}: Wi-Fi વલ્નરેબિલિટીઝ
\end{itemize}

\textbf{પ્રિવેન્શન સ્ટ્રેટેજીઝ:}

\begin{itemize}
\tightlist
\item
  \textbf{ટેકનિકલ}: ફાયરવોલ્સ, એન્ટિવાયરસ, એન્ક્રિપ્શન
\item
  \textbf{એડમિનિસ્ટ્રેટિવ}: પોલિસીઝ, પ્રોસીજર્સ
\item
  \textbf{ફિઝિકલ}: લોક્સ, સર્વેલન્સ
\item
  \textbf{એજ્યુકેશન}: યુઝર અવેરનેસ ટ્રેનિંગ
\end{itemize}

\end{solutionbox}
\begin{mnemonicbox}
``નેટવર્ક એપ્લિકેશન મેલવેર સોશિયલ ક્રિપ્ટો'' (એટેક કેટેગરીઝ)

\end{mnemonicbox}
\subsection*{પ્રશ્ન ૫(આ) [૩
ગુણ]}\label{uxaaauxab0uxab6uxaa8-uxaebuxa86-uxae9-uxa97uxaa3}

\textbf{(5AB.4) હેક્સાડેસિમલ સંખ્યાની બાઈનરી ગણતરી કરો.}

\begin{solutionbox}

\textbf{હેક્સાડેસિમલ થી બાઈનરી કન્વર્ઝન}

\textbf{પદ્ધતિ:} દરેક હેક્સ ડિજિટને 4-બિટ બાઈનરીમાં કન્વર્ટ કરો


{\def\LTcaptype{none} % do not increment counter
\vspace{-5pt}
\captionof{table}{હેક્સ થી બાઈનરી કન્વર્ઝન}
\vspace{-10pt}
\begin{longtable}[]{@{}llll@{}}
\toprule\noalign{}
હેક્સ ડિજિટ & બાઈનરી & હેક્સ ડિજિટ & બાઈનરી \\
\midrule\noalign{}
\endhead
\bottomrule\noalign{}
\endlastfoot
5 & 0101 & B & 1011 \\
A & 1010 & 4 & 0100 \\
\end{longtable}
}

\textbf{સ્ટેપ-બાય-સ્ટેપ કન્વર્ઝન:}

\begin{itemize}
\tightlist
\item
  \textbf{5} \rightarrow \textbf{0101}
\item
  \textbf{A} \rightarrow \textbf{1010}
\item
  \textbf{B} \rightarrow \textbf{1011}
\item
  \textbf{.} \rightarrow \textbf{.} (દશાંશ બિંદુ)
\item
  \textbf{4} \rightarrow \textbf{0100}
\end{itemize}

\textbf{અંતિમ જવાબ:} (5AB.4)_{1}_{6} = (010110101011.0100)_{2}

\textbf{સરળીકૃત:} (10110101011.01)_{2}

\end{solutionbox}
\begin{mnemonicbox}
``દરેક હેક્સ = 4 બિટ્સ''

\end{mnemonicbox}
\subsection*{પ્રશ્ન ૫(બ) [૪
ગુણ]}\label{uxaaauxab0uxab6uxaa8-uxaebuxaac-uxaea-uxa97uxaa3}

\textbf{Digi-Locker, e-rupi ની મુખ્ય વિશેષતાઓની યાદી બનાવો.}

\begin{solutionbox}


{\def\LTcaptype{none} % do not increment counter
\vspace{-5pt}
\captionof{table}{ડિજિટલ પ્લેટફોર્મ વિશેષતાઓ}
\vspace{-10pt}
\begin{longtable}[]{@{}
  >{\raggedright\arraybackslash}p{(\linewidth - 6\tabcolsep) * \real{0.2778}}
  >{\raggedright\arraybackslash}p{(\linewidth - 6\tabcolsep) * \real{0.1389}}
  >{\raggedright\arraybackslash}p{(\linewidth - 6\tabcolsep) * \real{0.3889}}
  >{\raggedright\arraybackslash}p{(\linewidth - 6\tabcolsep) * \real{0.1944}}@{}}
\toprule\noalign{}
\begin{minipage}[b]{\linewidth}\raggedright
પ્લેટફોર્મ
\end{minipage} & \begin{minipage}[b]{\linewidth}\raggedright
હેતુ
\end{minipage} & \begin{minipage}[b]{\linewidth}\raggedright
મુખ્ય વિશેષતાઓ
\end{minipage} & \begin{minipage}[b]{\linewidth}\raggedright
ફાયદા
\end{minipage} \\
\midrule\noalign{}
\endhead
\bottomrule\noalign{}
\endlastfoot
\textbf{Digi-Locker} & ડોક્યુમેન્ટ સ્ટોરેજ & ક્લાઉડ સ્ટોરેજ, ડિજિટલ સર્ટિફિકેટ્સ &
પેપરલેસ વેરિફિકેશન \\
\textbf{e-RUPI} & ડિજિટલ પેમેન્ટ & QR/SMS વાઉચર, પ્રી-પેઇડ & ટાર્ગેટેડ વેલ્ફેર
ડિલિવરી \\
\end{longtable}
}

\textbf{Digi-Locker વિશેષતાઓ:}

\begin{itemize}
\tightlist
\item
  \textbf{ડિજિટલ વોલેટ}: ક્લાઉડમાં ડોક્યુમેન્ટ્સ સ્ટોર કરો
\item
  \textbf{ઓથેન્ટિકેશન}: આધાર-આધારિત વેરિફિકેશન\\
\item
  \textbf{ઇન્ટિગ્રેશન}: સરકારી વિભાગોનો ઍક્સેસ
\item
  \textbf{શેરિંગ}: સુરક્ષિત ડોક્યુમેન્ટ શેરિંગ
\end{itemize}

\textbf{e-RUPI વિશેષતાઓ:}

\begin{itemize}
\tightlist
\item
  \textbf{પ્રીપેઇડ વાઉચર}: હેતુ-વિશિષ્ટ પેમેન્ટ્સ
\item
  \textbf{કોન્ટેક્ટ-લેસ}: QR કોડ/SMS આધારિત
\item
  \textbf{સિક્યોરિટી}: કોઈ વ્યક્તિગત/બેંક વિગતો શેર નથી
\item
  \textbf{વપરાશ}: હેલ્થકેર, એજ્યુકેશન, વેલ્ફેર સ્કીમ્સ
\end{itemize}

\end{solutionbox}
\begin{mnemonicbox}
``Digi સ્ટોર કરે, e-RUPI પેમેન્ટ કરે'' (સ્ટોરેજ વિ પેમેન્ટ)

\end{mnemonicbox}
\subsection*{પ્રશ્ન ૫(સ) [૭
ગુણ]}\label{uxaaauxab0uxab6uxaa8-uxaebuxab8-uxaed-uxa97uxaa3}

\textbf{કમ્પ્યુટર સિસ્ટમની વિવિધ પેઢીઓનું વર્ણન કરો.}

\begin{solutionbox}

\textbf{કમ્પ્યુટર પેઢીઓનું ઉત્ક્રાંતિ}


{\def\LTcaptype{none} % do not increment counter
\vspace{-5pt}
\captionof{table}{કમ્પ્યુટર પેઢીઓની સરખામણી}
\vspace{-10pt}
\begin{longtable}[]{@{}
  >{\raggedright\arraybackslash}p{(\linewidth - 10\tabcolsep) * \real{0.1250}}
  >{\raggedright\arraybackslash}p{(\linewidth - 10\tabcolsep) * \real{0.1875}}
  >{\raggedright\arraybackslash}p{(\linewidth - 10\tabcolsep) * \real{0.2292}}
  >{\raggedright\arraybackslash}p{(\linewidth - 10\tabcolsep) * \real{0.1250}}
  >{\raggedright\arraybackslash}p{(\linewidth - 10\tabcolsep) * \real{0.1458}}
  >{\raggedright\arraybackslash}p{(\linewidth - 10\tabcolsep) * \real{0.1875}}@{}}
\toprule\noalign{}
\begin{minipage}[b]{\linewidth}\raggedright
પેઢી
\end{minipage} & \begin{minipage}[b]{\linewidth}\raggedright
સમયગાળો
\end{minipage} & \begin{minipage}[b]{\linewidth}\raggedright
ટેકનોલોજી
\end{minipage} & \begin{minipage}[b]{\linewidth}\raggedright
સાઇઝ
\end{minipage} & \begin{minipage}[b]{\linewidth}\raggedright
સ્પીડ
\end{minipage} & \begin{minipage}[b]{\linewidth}\raggedright
ઉદાહરણો
\end{minipage} \\
\midrule\noalign{}
\endhead
\bottomrule\noalign{}
\endlastfoot
\textbf{પ્રથમ} & 1940-1956 & વેક્યુમ ટ્યુબ્સ & રૂમ-સાઇઝ્ડ & ધીમી & ENIAC,
UNIVAC \\
\textbf{બીજી} & 1956-1963 & ટ્રાન્ઝિસ્ટર્સ & નાની & ઝડપી & IBM 1401, CDC
1604 \\
\textbf{ત્રીજી} & 1964-1971 & ઇન્ટિગ્રેટેડ સર્કિટ્સ & ડેસ્ક-સાઇઝ્ડ & વધુ ઝડપી & IBM
360, PDP-8 \\
\textbf{ચોથી} & 1971-1980s & માઇક્રોપ્રોસેસર્સ & પર્સનલ & ખૂબ ઝડપી & Intel
4004, Apple II \\
\textbf{પાંચમી} & 1980s-વર્તમાન & AI/પેરેલલ પ્રોસેસિંગ & પોર્ટેબલ & અત્યંત ઝડપી &
આધુનિક PCs, સ્માર્ટફોન્સ \\
\end{longtable}
}

\textbf{વિસ્તૃત વર્ણન:}

\textbf{પ્રથમ પેઢી (1940-1956):}

\begin{itemize}
\tightlist
\item
  \textbf{ટેકનોલોજી}: લોજિક/મેમરી માટે વેક્યુમ ટ્યુબ્સ
\item
  \textbf{પ્રોગ્રામિંગ}: મશીન લેન્ગ્વેજ, પંચ કાર્ડ્સ
\item
  \textbf{લાક્ષણિકતાઓ}: મોટા, મોંઘા, અવિશ્વસનીય
\item
  \textbf{ગરમી}: ભારે ગરમી ઉત્પન્ન કરતા
\item
  \textbf{ઉદાહરણો}: ENIAC (30 ટન), UNIVAC I
\end{itemize}

\textbf{બીજી પેઢી (1956-1963):}

\begin{itemize}
\tightlist
\item
  \textbf{ટેકનોલોજી}: વેક્યુમ ટ્યુબ્સની જગ્યાએ ટ્રાન્ઝિસ્ટર્સ
\item
  \textbf{પ્રોગ્રામિંગ}: એસેમ્બલી લેન્ગ્વેજ, FORTRAN, COBOL
\item
  \textbf{સુધારા}: નાના, ઝડપી, વધુ વિશ્વસનીય
\item
  \textbf{મેમરી}: મેગ્નેટિક કોર મેમરી
\item
  \textbf{ઉદાહરણો}: IBM 1401, Honeywell 400
\end{itemize}

\textbf{ત્રીજી પેઢી (1964-1971):}

\begin{itemize}
\tightlist
\item
  \textbf{ટેકનોલોજી}: ઇન્ટિગ્રેટેડ સર્કિટ્સ (ICs)
\item
  \textbf{પ્રોગ્રામિંગ}: હાઇ-લેવલ લેન્ગ્વેજેસ
\item
  \textbf{વિશેષતાઓ}: ઓપરેટિંગ સિસ્ટમ્સ, મલ્ટિપ્રોસેસિંગ
\item
  \textbf{સાઇઝ}: મિની-કમ્પ્યુટરનો ઉદભવ
\item
  \textbf{ઉદાહરણો}: IBM System/360, PDP-8
\end{itemize}

\textbf{ચોથી પેઢી (1971-1980s):}

\begin{itemize}
\tightlist
\item
  \textbf{ટેકનોલોજી}: માઇક્રોપ્રોસેસર્સ (ચિપ પર CPU)
\item
  \textbf{ડેવલપમેન્ટ}: પર્સનલ કમ્પ્યુટર્સનો જન્મ
\item
  \textbf{વિશેષતાઓ}: GUI, નેટવર્કિંગ ક્ષમતાઓ
\item
  \textbf{સ્ટોરેજ}: ફ્લોપી ડિસ્ક્સ, હાર્ડ ડ્રાઇવ્સ
\item
  \textbf{ઉદાહરણો}: Intel 8080, Apple II, IBM PC
\end{itemize}

\textbf{પાંચમી પેઢી (1980s-વર્તમાન):}

\begin{itemize}
\tightlist
\item
  \textbf{ટેકનોલોજી}: AI, પેરેલલ પ્રોસેસિંગ, VLSI
\item
  \textbf{વિશેષતાઓ}: ઇન્ટરનેટ, મલ્ટિમીડિયા, મોબાઇલ કમ્પ્યુટિંગ
\item
  \textbf{લાક્ષણિકતાઓ}: યુઝર-ફ્રેન્ડલી, પોર્ટેબલ, શક્તિશાળી
\item
  \textbf{વર્તમાન}: સ્માર્ટફોન્સ, ટેબલેટ્સ, ક્લાઉડ કમ્પ્યુટિંગ
\item
  \textbf{ઉદાહરણો}: આધુનિક લેપટોપ્સ, સ્માર્ટફોન્સ, સુપરકમ્પ્યુટર્સ
\end{itemize}

\textbf{પેઢી દ્વારા મુખ્ય નવીનતાઓ:}

\begin{itemize}
\tightlist
\item
  \textbf{1મી}: ઇલેક્ટ્રોનિક કમ્પ્યુટિંગ
\item
  \textbf{2જી}: સ્ટોર્ડ પ્રોગ્રામ્સ
\item
  \textbf{3જી}: ઓપરેટિંગ સિસ્ટમ્સ
\item
  \textbf{4થી}: પર્સનલ કમ્પ્યુટિંગ
\item
  \textbf{5મી}: ઇન્ટરનેટ અને AI
\end{itemize}

\textbf{આકૃતિ:}

\begin{verbatim}
timeline
    title કમ્પ્યુટર પેઢીઓ
    1940{-1956 : પ્રથમ પેઢી}
              : વેક્યુમ ટ્યુબ્સ
              : રૂમ{-સાઇઝ્ડ}
    1956{-1963 : બીજી પેઢી}
              : ટ્રાન્ઝિસ્ટર્સ
              : નાનું સાઇઝ
    1964{-1971 : ત્રીજી પેઢી}
              : ઇન્ટિગ્રેટેડ સર્કિટ્સ
              : મિનિકમ્પ્યુટર્સ
    1971{-1980s : ચોથી પેઢી}
               : માઇક્રોપ્રોસેસર્સ
               : પર્સનલ કમ્પ્યુટર્સ
    1980s{-વર્તમાન : પાંચમી પેઢી}
                  : AI અને ઇન્ટરનેટ
                  : મોબાઇલ કમ્પ્યુટિંગ
\end{verbatim}

\end{solutionbox}
\begin{mnemonicbox}
``વેક્યુમ ટ્રાન્ઝિસ્ટર IC માઇક્રો AI'' (ટેકનોલોજી પ્રોગ્રેશન)

\end{mnemonicbox}
\subsection*{પ્રશ્ન ૫(આ અથવા) [૩
ગુણ]}\label{uxaaauxab0uxab6uxaa8-uxaebuxa86-uxa85uxaa5uxab5-uxae9-uxa97uxaa3}

\textbf{ઉદાહરણ સાથે ડેટા અને ઇન્ફોર્મેશન વચ્ચેનો તફાવત લખો.}

\begin{solutionbox}


{\def\LTcaptype{none} % do not increment counter
\vspace{-5pt}
\captionof{table}{ડેટા વિ ઇન્ફોર્મેશન}
\vspace{-10pt}
\begin{longtable}[]{@{}lll@{}}
\toprule\noalign{}
પાસા & ડેટા & ઇન્ફોર્મેશન \\
\midrule\noalign{}
\endhead
\bottomrule\noalign{}
\endlastfoot
\textbf{વ્યાખ્યા} & કાચા તથ્યો/આંકડા & પ્રોસેસ કરેલો ડેટા \\
\textbf{અર્થ} & કોઈ સંદર્ભ નથી & સંદર્ભ ધરાવે \\
\textbf{ઉદાહરણ} & 85, 92, 78 & સરેરાશ સ્કોર: 85\% \\
\textbf{હેતુ} & પ્રોસેસિંગ માટે ઇનપુટ & નિર્ણય માટે આઉટપુટ \\
\end{longtable}
}

\textbf{ઉદાહરણો:}

\begin{itemize}
\tightlist
\item
  \textbf{ડેટા}: વિદ્યાર્થીના ગુણ (85, 92, 78, 88)
\item
  \textbf{ઇન્ફોર્મેશન}: વર્ગની સરેરાશ 85.75\% છે
\end{itemize}

\textbf{લાક્ષણિકતાઓ:}

\begin{itemize}
\tightlist
\item
  \textbf{ડેટા}: અવ્યવસ્થિત, કાચો, પ્રોસેસિંગની જરૂર
\item
  \textbf{ઇન્ફોર્મેશન}: વ્યવસ્થિત, અર્થપૂર્ણ, નિર્ણયો માટે ઉપયોગી
\end{itemize}

\end{solutionbox}
\begin{mnemonicbox}
``ડેટા કાચો, ઇન્ફોર્મેશન રિફાઇન્ડ''

\end{mnemonicbox}
\subsection*{પ્રશ્ન ૫(બ અથવા) [૪
ગુણ]}\label{uxaaauxab0uxab6uxaa8-uxaebuxaac-uxa85uxaa5uxab5-uxaea-uxa97uxaa3}

\textbf{એનાલોગ મોડ્યુલેશન અને ડિજિટલ મોડ્યુલેશનની સરખામણી કરો.}

\begin{solutionbox}


{\def\LTcaptype{none} % do not increment counter
\vspace{-5pt}
\captionof{table}{એનાલોગ વિ ડિજિટલ મોડ્યુલેશન}
\vspace{-10pt}
\begin{longtable}[]{@{}lll@{}}
\toprule\noalign{}
વિશેષતા & એનાલોગ મોડ્યુલેશન & ડિજિટલ મોડ્યુલેશન \\
\midrule\noalign{}
\endhead
\bottomrule\noalign{}
\endlastfoot
\textbf{સિગ્નલ પ્રકાર} & કન્ટિન્યુઅસ & ડિસ્ક્રીટ (0s અને 1s) \\
\textbf{નોઇઝ ઇમ્યુનિટી} & નબળી & ઉત્તમ \\
\textbf{બેન્ડવિડ્થ} & ઓછી & વધુ \\
\textbf{ક્વોલિટી} & અંતર સાથે ઘટે & ક્વોલિટી જાળવે \\
\textbf{ઉદાહરણો} & AM, FM રેડિયો & FSK, PSK, QAM \\
\end{longtable}
}

\textbf{એનાલોગ મોડ્યુલેશન:}

\begin{itemize}
\tightlist
\item
  \textbf{પ્રકારો}: AM (એમ્પ્લિટ્યુડ), FM (ફ્રીક્વન્સી), PM (ફેઝ)
\item
  \textbf{ઍપ્લિકેશન્સ}: રેડિયો બ્રોડકાસ્ટિંગ, એનાલોગ ટીવી
\item
  \textbf{ફાયદા}: સરળ, ઓછી બેન્ડવિડ્થ
\item
  \textbf{નુકસાન}: નોઇઝ સંવેદનશીલ, ક્વોલિટી લોસ
\end{itemize}

\textbf{ડિજિટલ મોડ્યુલેશન:}

\begin{itemize}
\tightlist
\item
  \textbf{પ્રકારો}: ASK, FSK, PSK, QAM
\item
  \textbf{ઍપ્લિકેશન્સ}: Wi-Fi, સેલ્યુલર, સેટેલાઇટ
\item
  \textbf{ફાયદા}: નોઇઝ રેઝિસ્ટન્ટ, એરર કરેક્શન
\item
  \textbf{નુકસાન}: જટિલ, વધુ બેન્ડવિડ્થ
\end{itemize}

\end{solutionbox}
\begin{mnemonicbox}
``એનાલોગ સરળ, ડિજિટલ સ્માર્ટ''

\end{mnemonicbox}
\subsection*{પ્રશ્ન ૫(સ અથવા) [૭
ગુણ]}\label{uxaaauxab0uxab6uxaa8-uxaebuxab8-uxa85uxaa5uxab5-uxaed-uxa97uxaa3}

\textbf{IPv4 માં IP સરનામાની શ્રેણીની ચર્ચા કરો.}

\begin{solutionbox}

\textbf{IPv4 એડ્રેસ રેન્જ અને વર્ગીકરણ}


{\def\LTcaptype{none} % do not increment counter
\vspace{-5pt}
\captionof{table}{IPv4 એડ્રેસ ક્લાસેસ}
\vspace{-10pt}
\begin{longtable}[]{@{}
  >{\raggedright\arraybackslash}p{(\linewidth - 10\tabcolsep) * \real{0.1094}}
  >{\raggedright\arraybackslash}p{(\linewidth - 10\tabcolsep) * \real{0.1094}}
  >{\raggedright\arraybackslash}p{(\linewidth - 10\tabcolsep) * \real{0.2344}}
  >{\raggedright\arraybackslash}p{(\linewidth - 10\tabcolsep) * \real{0.1406}}
  >{\raggedright\arraybackslash}p{(\linewidth - 10\tabcolsep) * \real{0.2969}}
  >{\raggedright\arraybackslash}p{(\linewidth - 10\tabcolsep) * \real{0.1094}}@{}}
\toprule\noalign{}
\begin{minipage}[b]{\linewidth}\raggedright
ક્લાસ
\end{minipage} & \begin{minipage}[b]{\linewidth}\raggedright
રેન્જ
\end{minipage} & \begin{minipage}[b]{\linewidth}\raggedright
ડિફોલ્ટ સબનેટ
\end{minipage} & \begin{minipage}[b]{\linewidth}\raggedright
નેટવર્ક્સ
\end{minipage} & \begin{minipage}[b]{\linewidth}\raggedright
પ્રતિ નેટવર્ક હોસ્ટ્સ
\end{minipage} & \begin{minipage}[b]{\linewidth}\raggedright
વપરાશ
\end{minipage} \\
\midrule\noalign{}
\endhead
\bottomrule\noalign{}
\endlastfoot
\textbf{A} & 1.0.0.0 - 126.0.0.0 & /8 (255.0.0.0) & 126 & 16,777,214 &
મોટી સંસ્થાઓ \\
\textbf{B} & 128.0.0.0 - 191.255.0.0 & /16 (255.255.0.0) & 16,384 &
65,534 & મધ્યમ સંસ્થાઓ \\
\textbf{C} & 192.0.0.0 - 223.255.255.0 & /24 (255.255.255.0) & 2,097,152
& 254 & નાની સંસ્થાઓ \\
\textbf{D} & 224.0.0.0 - 239.255.255.255 & N/A & N/A & N/A & મલ્ટિકાસ્ટ \\
\textbf{E} & 240.0.0.0 - 255.255.255.255 & N/A & N/A & N/A &
રિઝર્વ્ડ/એક્સપેરિમેન્ટલ \\
\end{longtable}
}

\textbf{સ્પેશિયલ એડ્રેસ રેન્જ:}

\textbf{પ્રાઇવેટ IP રેન્જ (RFC 1918):}

\begin{itemize}
\tightlist
\item
  \textbf{ક્લાસ A}: 10.0.0.0 - 10.255.255.255 (/8)
\item
  \textbf{ક્લાસ B}: 172.16.0.0 - 172.31.255.255 (/12)\\
\item
  \textbf{ક્લાસ C}: 192.168.0.0 - 192.168.255.255 (/16)
\end{itemize}

\textbf{રિઝર્વ્ડ એડ્રેસેસ:}

\begin{itemize}
\tightlist
\item
  \textbf{લૂપબેક}: 127.0.0.0 - 127.255.255.255
\item
  \textbf{લિંક-લોકલ}: 169.254.0.0 - 169.254.255.255
\item
  \textbf{બ્રોડકાસ્ટ}: x.x.x.255 (સબનેટનું છેલ્લું એડ્રેસ)
\item
  \textbf{નેટવર્ક}: x.x.x.0 (સબનેટનું પ્રથમ એડ્રેસ)
\end{itemize}

\textbf{એડ્રેસ સ્ટ્રક્ચર:}

\begin{itemize}
\tightlist
\item
  \textbf{કુલ IPv4 સ્પેસ}: 4,294,967,296 એડ્રેસેસ (2^{3}^{2})
\item
  \textbf{ફોર્મેટ}: ડોટેડ ડેસિમલમાં 32-બિટ એડ્રેસ
\item
  \textbf{ઉદાહરણ}: 192.168.1.100
\end{itemize}

\textbf{સબનેટ ગણતરીનું ઉદાહરણ:}

\begin{itemize}
\tightlist
\item
  \textbf{નેટવર્ક}: 192.168.1.0/24
\item
  \textbf{સબનેટ માસ્ક}: 255.255.255.0
\item
  \textbf{હોસ્ટ રેન્જ}: 192.168.1.1 - 192.168.1.254
\item
  \textbf{બ્રોડકાસ્ટ}: 192.168.1.255
\end{itemize}

\textbf{CIDR નોટેશન:}

\begin{itemize}
\tightlist
\item
  \textbf{/8}: 255.0.0.0 (ક્લાસ A ડિફોલ્ટ)
\item
  \textbf{/16}: 255.255.0.0 (ક્લાસ B ડિફોલ્ટ)
\item
  \textbf{/24}: 255.255.255.0 (ક્લાસ C ડિફોલ્ટ)
\item
  \textbf{/30}: 255.255.255.252 (પોઇન્ટ-ટુ-પોઇન્ટ લિંક્સ)
\end{itemize}

\textbf{IPv4 એક્ઝોશન:}

\begin{itemize}
\tightlist
\item
  \textbf{સમસ્યા}: મર્યાદિત એડ્રેસ સ્પેસ
\item
  \textbf{સોલ્યુશન}: IPv6 (128-બિટ એડ્રેસેસ)
\item
  \textbf{અસ્થાયી ઉકેલો}: NAT, CIDR, પ્રાઇવેટ એડ્રેસિંગ
\end{itemize}

\textbf{આકૃતિ:}

\begin{center}
\textbf{Mermaid Diagram (Code)}
\begin{verbatim}
{Shaded}
{Highlighting}[]
graph TD
    A[IPv4 એડ્રેસ સ્પેસ] {-{-}{} B[ક્લાસ A: 1{-}126]}
    A {-{-}{} C[ક્લાસ B: 128{-}191]  }
    A {-{-}{} D[ક્લાસ C: 192{-}223]}
    A {-{-}{} E[ક્લાસ D: 224{-}239 મલ્ટિકાસ્ટ]}
    A {-{-}{} F[ક્લાસ E: 240{-}255 રિઝર્વ્ડ]}
    
    B {-{-}{} G[મોટા નેટવર્ક્સ]}
    C {-{-}{} H[મધ્યમ નેટવર્ક્સ]}
    D {-{-}{} I[નાના નેટવર્ક્સ]}
{Highlighting}
{Shaded}
\end{verbatim}
\end{center}

\textbf{ઍપ્લિકેશન્સ:}

\begin{itemize}
\tightlist
\item
  \textbf{પબ્લિક IPs}: ઇન્ટરનેટ રાઉટિંગ
\item
  \textbf{પ્રાઇવેટ IPs}: ઇન્ટર્નલ નેટવર્ક્સ
\item
  \textbf{મલ્ટિકાસ્ટ}: વન-ટુ-મેની કોમ્યુનિકેશન
\item
  \textbf{લૂપબેક}: લોકલ ટેસ્ટિંગ
\end{itemize}

\end{solutionbox}
\begin{mnemonicbox}
``A બિગ કંપની ડિલિવર્ડ એવરીથિંગ'' (ક્લાસેસ A-B-C-D-E)

\end{mnemonicbox}

\end{document}
