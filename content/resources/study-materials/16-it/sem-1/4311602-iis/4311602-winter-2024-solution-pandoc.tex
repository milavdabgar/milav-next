\documentclass[10pt,a4paper]{article}

% content/resources/templates/preamble.tex
\usepackage[margin=0.6in]{geometry}
\author{Milav Dabgar}
\usepackage{amsmath,amssymb,amsthm}
\usepackage{booktabs}
\usepackage{multirow}
\usepackage{xcolor}
\usepackage{tcolorbox}
\tcbuselibrary{breakable,skins}
\usepackage[colorlinks=true,linkcolor=blue]{hyperref}
\usepackage{titlesec}
\usepackage{enumitem}
\usepackage{tikz}
\usepackage{pgfplots}
\usepackage{circuitikz}
\usepackage[version=4]{mhchem}
\usepackage{longtable}
\usepackage{array}
\usepackage{float}
\usepackage{caption}
\usepackage{listings}

\lstset{
  basicstyle=\small\ttfamily,
  breaklines=true,
  breakatwhitespace=false,
  postbreak=\mbox{\textcolor{red}{$\hookrightarrow$}\space},
  float=false,
  numbers=left,
  numberstyle=\tiny\color{gray},
  numbersep=10pt,
  xleftmargin=2em,
  keywordstyle=\color{blue},
  commentstyle=\color{green!60!black},
  stringstyle=\color{purple},
  backgroundcolor=\color{gray!5},
  showstringspaces=false,
  tabsize=2,
  captionpos=b,
  keepspaces=true,
  columns=flexible
}

\pgfplotsset{compat=1.18}
\usetikzlibrary{shapes,arrows,positioning,calc,patterns,decorations.pathmorphing,decorations.markings,arrows.meta}

% Color scheme
\definecolor{headcolor}{RGB}{0,102,204}
\definecolor{keycolor}{RGB}{220,20,60}
\definecolor{solutioncolor}{RGB}{34,139,34}
\definecolor{mnemoniccolor}{RGB}{148,0,211}
\definecolor{codecolor}{RGB}{0,0,100}

% Spacing
\setlength{\parskip}{3pt}
\setlist[itemize]{nosep}
\setlist[enumerate]{nosep}

% Title formatting
\titleformat{\section}{\Large\bfseries\color{headcolor}}{\thesection}{1em}{}
\titleformat{\subsection}{\large\bfseries\color{headcolor}}{\thesubsection}{1em}{}

% Pandoc tightlist compatibility
\providecommand{\tightlist}{%
  \setlength{\itemsep}{0pt}\setlength{\parskip}{0pt}}

% Pandoc longtable compatibility
\newcounter{none}
\def\thenone{}


% content/resources/templates/english-boxes.tex
% This file is currently empty - it exists to maintain consistency with the import structure.
% Add custom environments here if needed in the future.


\begin{document}

\begin{center}
{\Huge\bfseries\color{headcolor} Subject Name Solutions}\\[5pt]
{\LARGE 4311602 -- Winter 2024}\\[3pt]
{\large Semester 1 Study Material}\\[3pt]
{\normalsize\textit{Detailed Solutions and Explanations}}
\end{center}

\vspace{10pt}

\subsection*{Question 1(a) [3 marks]}\label{q1a}

\textbf{Explain NAND logic gate.}

\begin{solutionbox}

NAND gate is a universal logic gate that produces output 0 only when all
inputs are 1.

\textbf{Truth Table:}

{\def\LTcaptype{none} % do not increment counter
\begin{longtable}[]{@{}lll@{}}
\toprule\noalign{}
A & B & Y = A NAND B \\
\midrule\noalign{}
\endhead
\bottomrule\noalign{}
\endlastfoot
0 & 0 & 1 \\
0 & 1 & 1 \\
1 & 0 & 1 \\
1 & 1 & 0 \\
\end{longtable}
}

\textbf{Symbol:}

\begin{verbatim}
    A {-{-}{-}{-}+{-}{-}{-}Do{-}{-}{-} Y}
          |   |
    B {-{-}{-}{-}+   |}
              |
\end{verbatim}

\begin{itemize}
\tightlist
\item
  \textbf{NAND Function}: Output is complement of AND operation
\item
  \textbf{Universal Gate}: Can implement any logic function
\item
  \textbf{Low Power}: Requires fewer transistors in IC design
\end{itemize}

\end{solutionbox}
\begin{mnemonicbox}
``NOT AND = NAND''

\end{mnemonicbox}
\subsection*{Question 1(b) [4 marks]}\label{q1b}

\textbf{Draw AND logic Gate using NOR Gate only.}

\begin{solutionbox}

AND gate can be implemented using NOR gates by applying De Morgan's
theorem.

\textbf{Circuit Diagram:}

\begin{center}
\textbf{Mermaid Diagram (Code)}
\begin{verbatim}
{Shaded}
{Highlighting}[]
graph LR
    A[A] {-{-}{} N1[NOR]}
    A {-{-}{} N1}
    B[B] {-{-}{} N2[NOR]}
    B {-{-}{} N2}
    N1 {-{-}{} N3[NOR]}
    N2 {-{-}{} N3}
    N3 {-{-}{} Y[Y = A.B]}
{Highlighting}
{Shaded}
\end{verbatim}
\end{center}

\textbf{Implementation Steps:}

\begin{itemize}
\tightlist
\item
  \textbf{Step 1}: Create NOT A using NOR gate (A NOR A = A')
\item
  \textbf{Step 2}: Create NOT B using NOR gate (B NOR B = B')
\item
  \textbf{Step 3}: Apply De Morgan's: A.B = (A' + B')'
\item
  \textbf{Final Output}: A AND B
\end{itemize}

\end{solutionbox}
\begin{mnemonicbox}
``Double inversion gives original function''

\end{mnemonicbox}
\subsection*{Question 1(c) [7 marks]}\label{q1c}

\textbf{Explain components of Information System with diagram.}

\begin{solutionbox}

Information System consists of five key components working together to
process data into useful information.

\textbf{System Diagram:}

\begin{verbatim}
graph TB
    subgraph "Information System"
        H[Hardware]
        S[Software]
        D[Data]
        P[Procedures]
        Pe[People]
        
        H {-{-} P}
        S {-{-} P}
        D {-{-} P}
        Pe {-{-} P}
        P {-{-} H}
    end
    
    Input[Input] {-{-} H}
    H {-{-} Output[Output]}
\end{verbatim}

\textbf{Components:}

{\def\LTcaptype{none} % do not increment counter
\begin{longtable}[]{@{}
  >{\raggedright\arraybackslash}p{(\linewidth - 4\tabcolsep) * \real{0.3235}}
  >{\raggedright\arraybackslash}p{(\linewidth - 4\tabcolsep) * \real{0.3824}}
  >{\raggedright\arraybackslash}p{(\linewidth - 4\tabcolsep) * \real{0.2941}}@{}}
\toprule\noalign{}
\begin{minipage}[b]{\linewidth}\raggedright
Component
\end{minipage} & \begin{minipage}[b]{\linewidth}\raggedright
Description
\end{minipage} & \begin{minipage}[b]{\linewidth}\raggedright
Examples
\end{minipage} \\
\midrule\noalign{}
\endhead
\bottomrule\noalign{}
\endlastfoot
\textbf{Hardware} & Physical devices & CPU, Memory, Keyboards \\
\textbf{Software} & Programs and applications & OS, Applications,
Utilities \\
\textbf{Data} & Raw facts and figures & Numbers, Text, Images \\
\textbf{Procedures} & Rules and instructions & User manuals, SOPs \\
\textbf{People} & Users and operators & End users, IT staff \\
\end{longtable}
}

\begin{itemize}
\tightlist
\item
  \textbf{Input Processing}: Data enters through hardware
\item
  \textbf{Storage Management}: Data stored and retrieved efficiently\\
\item
  \textbf{Output Generation}: Information presented to users
\item
  \textbf{Integration}: All components work cohesively
\end{itemize}

\end{solutionbox}
\begin{mnemonicbox}
``Hardware Supports Data Processing People''

\end{mnemonicbox}
\subsection*{Question 1(c OR) [7
marks]}\label{question-1c-or-7-marks}

\textbf{Explain the working of Google Search Engine with example.}

\begin{solutionbox}

Google Search Engine uses complex algorithms to find and rank web pages
based on user queries.

\textbf{Working Process:}

\begin{verbatim}
sequenceDiagram
    participant U as User
    participant G as Google
    participant I as Index
    participant W as Web Pages
    
    U{-G: Enter Search Query}
    G{-I: Query Processing}
    I{-G: Retrieve Relevant Pages}
    G{-G: Rank Pages (PageRank)}
    G{-U: Display Results}
\end{verbatim}

\textbf{Key Components:}

{\def\LTcaptype{none} % do not increment counter
\begin{longtable}[]{@{}lll@{}}
\toprule\noalign{}
Stage & Process & Example \\
\midrule\noalign{}
\endhead
\bottomrule\noalign{}
\endlastfoot
\textbf{Crawling} & Discover web pages & Googlebot visits websites \\
\textbf{Indexing} & Store page content & Keywords stored in database \\
\textbf{Ranking} & Order by relevance & PageRank algorithm \\
\textbf{Serving} & Display results & Search results page \\
\end{longtable}
}

\textbf{Example Search Process:}

\begin{itemize}
\item
  \textbf{Query}: ``Introduction to IT Systems''
\item
  \textbf{Processing}: Parse keywords, check index
\item
  \textbf{Ranking}: Educational sites ranked higher
\item
  \textbf{Results}: GTU syllabus, tutorials, courses
\item
  \textbf{PageRank Algorithm}: Links determine page importance
\item
  \textbf{Machine Learning}: Improves search accuracy over time
\item
  \textbf{Real-time Updates}: Fresh content prioritized
\end{itemize}

\end{solutionbox}
\begin{mnemonicbox}
``Crawl Index Rank Serve''

\end{mnemonicbox}
\subsection*{Question 2(a) [3 marks]}\label{q2a}

\textbf{Convert (16.75)10= ( )8}

\begin{solutionbox}

Converting decimal 16.75 to octal requires separate conversion of
integer and fractional parts.

\textbf{Integer Part Conversion (16):}

{\def\LTcaptype{none} % do not increment counter
\begin{longtable}[]{@{}lll@{}}
\toprule\noalign{}
Division & Quotient & Remainder \\
\midrule\noalign{}
\endhead
\bottomrule\noalign{}
\endlastfoot
16 \div 8 & 2 & 0 \\
2 \div 8 & 0 & 2 \\
\end{longtable}
}

\textbf{Fractional Part Conversion (0.75):}

{\def\LTcaptype{none} % do not increment counter
\begin{longtable}[]{@{}ll@{}}
\toprule\noalign{}
Multiplication & Integer Part \\
\midrule\noalign{}
\endhead
\bottomrule\noalign{}
\endlastfoot
0.75 \times 8 = 6.0 & 6 \\
\end{longtable}
}

\textbf{Final Answer}: (16.75)10 = (20.6)8

\textbf{Verification}: 2\times8^{1} + 0\times8^{0} + 6\times8^{-}^{1} = 16 + 0 + 0.75 = 16.75 ✓

\end{solutionbox}
\begin{mnemonicbox}
``Divide integer, Multiply fraction''

\end{mnemonicbox}
\subsection*{Question 2(b) [4 marks]}\label{q2b}

\textbf{Explain Multiprocessing Operating System.}

\begin{solutionbox}

Multiprocessing OS manages multiple processors working simultaneously to
execute processes.

\textbf{Architecture Diagram:}

\begin{verbatim}
graph TB
    subgraph "Multiprocessing System"
        CPU1[CPU 1]
        CPU2[CPU 2]
        CPU3[CPU 3]
        SM[Shared Memory]
        OS[Operating System]
        
        CPU1 {-{-} SM}
        CPU2 {-{-} SM}
        CPU3 {-{-} SM}
        OS {-{-} CPU1}
        OS {-{-} CPU2}
        OS {-{-} CPU3}
    end
\end{verbatim}

\textbf{Key Features:}

{\def\LTcaptype{none} % do not increment counter
\begin{longtable}[]{@{}
  >{\raggedright\arraybackslash}p{(\linewidth - 4\tabcolsep) * \real{0.2903}}
  >{\raggedright\arraybackslash}p{(\linewidth - 4\tabcolsep) * \real{0.4194}}
  >{\raggedright\arraybackslash}p{(\linewidth - 4\tabcolsep) * \real{0.2903}}@{}}
\toprule\noalign{}
\begin{minipage}[b]{\linewidth}\raggedright
Feature
\end{minipage} & \begin{minipage}[b]{\linewidth}\raggedright
Description
\end{minipage} & \begin{minipage}[b]{\linewidth}\raggedright
Benefit
\end{minipage} \\
\midrule\noalign{}
\endhead
\bottomrule\noalign{}
\endlastfoot
\textbf{Parallel Processing} & Multiple CPUs work together & Faster
execution \\
\textbf{Load Balancing} & Tasks distributed evenly & Optimal resource
usage \\
\textbf{Fault Tolerance} & System continues if one CPU fails & Higher
reliability \\
\textbf{Shared Resources} & Common memory and I/O devices & Cost
effective \\
\end{longtable}
}

\begin{itemize}
\tightlist
\item
  \textbf{Symmetric Multiprocessing}: All processors have equal access
\item
  \textbf{Process Synchronization}: Coordinates between processors
\item
  \textbf{Enhanced Performance}: Linear speedup with processor count
\end{itemize}

\end{solutionbox}
\begin{mnemonicbox}
``Multiple Processors Process Parallel''

\end{mnemonicbox}
\subsection*{Question 2(c) [7 marks]}\label{q2c}

\textbf{Define Operating System. List out and Explain the functions of
Operating System.}

\begin{solutionbox}

\textbf{Definition}: Operating System is system software that manages
computer hardware and provides services to application programs.

\textbf{Core Functions:}

\begin{verbatim}
mindmap
  root((Operating System))
    Process Management
      Process Creation
      Scheduling
      Synchronization
    Memory Management
      Allocation
      Virtual Memory
      Paging
    File Management
      File Operations
      Directory Structure
      Access Control
    I/O Management
      Device Drivers
      Buffering
      Spooling
\end{verbatim}

\textbf{Detailed Functions:}

{\def\LTcaptype{none} % do not increment counter
\begin{longtable}[]{@{}
  >{\raggedright\arraybackslash}p{(\linewidth - 4\tabcolsep) * \real{0.3030}}
  >{\raggedright\arraybackslash}p{(\linewidth - 4\tabcolsep) * \real{0.3939}}
  >{\raggedright\arraybackslash}p{(\linewidth - 4\tabcolsep) * \real{0.3030}}@{}}
\toprule\noalign{}
\begin{minipage}[b]{\linewidth}\raggedright
Function
\end{minipage} & \begin{minipage}[b]{\linewidth}\raggedright
Description
\end{minipage} & \begin{minipage}[b]{\linewidth}\raggedright
Examples
\end{minipage} \\
\midrule\noalign{}
\endhead
\bottomrule\noalign{}
\endlastfoot
\textbf{Process Management} & Controls program execution & Task
scheduling, multitasking \\
\textbf{Memory Management} & Allocates RAM efficiently & Virtual memory,
paging \\
\textbf{File Management} & Organizes data storage & File systems,
directories \\
\textbf{I/O Management} & Controls input/output devices & Printer
spooling, disk access \\
\textbf{Security} & Protects system resources & User authentication,
access control \\
\end{longtable}
}

\begin{itemize}
\tightlist
\item
  \textbf{Resource Allocation}: Distributes CPU time and memory
\item
  \textbf{User Interface}: Provides command line or GUI interaction
\item
  \textbf{Error Handling}: Manages system failures gracefully
\item
  \textbf{System Calls}: Interface between applications and hardware
\end{itemize}

\end{solutionbox}
\begin{mnemonicbox}
``Process Memory Files Input-Output Security''

\end{mnemonicbox}
\subsection*{Question 2(a OR) [3
marks]}\label{question-2a-or-3-marks}

\textbf{Convert (1111111.11)2= ( )10}

\begin{solutionbox}

Converting binary to decimal using positional notation method.

\textbf{Conversion Table:}

{\def\LTcaptype{none} % do not increment counter
\begin{longtable}[]{@{}llll@{}}
\toprule\noalign{}
Position & Bit & Power & Value \\
\midrule\noalign{}
\endhead
\bottomrule\noalign{}
\endlastfoot
6 & 1 & 2^{6} & 64 \\
5 & 1 & 2^{5} & 32 \\
4 & 1 & 2^{4} & 16 \\
3 & 1 & 2^{3} & 8 \\
2 & 1 & 2^{2} & 4 \\
1 & 1 & 2^{1} & 2 \\
0 & 1 & 2^{0} & 1 \\
-1 & 1 & 2^{-}^{1} & 0.5 \\
-2 & 1 & 2^{-}^{2} & 0.25 \\
\end{longtable}
}

\textbf{Calculation}: 64 + 32 + 16 + 8 + 4 + 2 + 1 + 0.5 + 0.25 = 127.75

\textbf{Final Answer}: (1111111.11)2 = (127.75)10

\end{solutionbox}
\begin{mnemonicbox}
``Powers of Two add Together''

\end{mnemonicbox}
\subsection*{Question 2(b OR) [4
marks]}\label{question-2b-or-4-marks}

\textbf{Explain Batch Operating System.}

\begin{solutionbox}

Batch OS processes jobs in groups without user interaction during
execution.

\textbf{Working Model:}

\begin{center}
\textbf{Mermaid Diagram (Code)}
\begin{verbatim}
{Shaded}
{Highlighting}[]
graph LR
    subgraph "Batch Processing"
        J1[Job 1] {-{-}{} Q[Job Queue]}
        J2[Job 2] {-{-}{} Q}
        J3[Job 3] {-{-}{} Q}
        Q {-{-}{} CPU[CPU Processing]}
        CPU {-{-}{} O[Output]}
    end
{Highlighting}
{Shaded}
\end{verbatim}
\end{center}

\textbf{Characteristics:}

{\def\LTcaptype{none} % do not increment counter
\begin{longtable}[]{@{}
  >{\raggedright\arraybackslash}p{(\linewidth - 4\tabcolsep) * \real{0.3000}}
  >{\raggedright\arraybackslash}p{(\linewidth - 4\tabcolsep) * \real{0.4333}}
  >{\raggedright\arraybackslash}p{(\linewidth - 4\tabcolsep) * \real{0.2667}}@{}}
\toprule\noalign{}
\begin{minipage}[b]{\linewidth}\raggedright
Feature
\end{minipage} & \begin{minipage}[b]{\linewidth}\raggedright
Description
\end{minipage} & \begin{minipage}[b]{\linewidth}\raggedright
Impact
\end{minipage} \\
\midrule\noalign{}
\endhead
\bottomrule\noalign{}
\endlastfoot
\textbf{No Interaction} & Jobs run without user input & High
throughput \\
\textbf{Job Queue} & Multiple jobs wait in sequence & Efficient
processing \\
\textbf{Automatic Scheduling} & OS selects next job & Minimal
overhead \\
\textbf{Batch Processing} & Similar jobs grouped together & Resource
optimization \\
\end{longtable}
}

\begin{itemize}
\tightlist
\item
  \textbf{Advantages}: High system utilization, cost effective
\item
  \textbf{Disadvantages}: No real-time interaction, debugging difficulty
\item
  \textbf{Applications}: Payroll processing, data backup systems
\end{itemize}

\end{solutionbox}
\begin{mnemonicbox}
``Batch Jobs Queue Automatically''

\end{mnemonicbox}
\subsection*{Question 2(c OR) [7
marks]}\label{question-2c-or-7-marks}

\textbf{Explain Architecture and modes of Linux System with Diagram.}

\begin{solutionbox}

Linux follows layered architecture with distinct user and kernel modes.

\textbf{System Architecture:}

\begin{verbatim}
graph TB
    subgraph "User Space"
        UA[User Applications]
        SL[System Libraries]
        SC[System Calls]
    end
    
    subgraph "Kernel Space"
        VFS[Virtual File System]
        PM[Process Management]
        MM[Memory Management]
        NM[Network Management]
        DM[Device Management]
    end
    
    HW[Hardware]
    
    UA {-{-} SL}
    SL {-{-} SC}
    SC {-{-} VFS}
    SC {-{-} PM}
    SC {-{-} MM}
    SC {-{-} NM}
    SC {-{-} DM}
    VFS {-{-} HW}
    PM {-{-} HW}
    MM {-{-} HW}
    NM {-{-} HW}
    DM {-{-} HW}
\end{verbatim}

\textbf{Operating Modes:}

{\def\LTcaptype{none} % do not increment counter
\begin{longtable}[]{@{}
  >{\raggedright\arraybackslash}p{(\linewidth - 4\tabcolsep) * \real{0.1818}}
  >{\raggedright\arraybackslash}p{(\linewidth - 4\tabcolsep) * \real{0.3939}}
  >{\raggedright\arraybackslash}p{(\linewidth - 4\tabcolsep) * \real{0.4242}}@{}}
\toprule\noalign{}
\begin{minipage}[b]{\linewidth}\raggedright
Mode
\end{minipage} & \begin{minipage}[b]{\linewidth}\raggedright
Description
\end{minipage} & \begin{minipage}[b]{\linewidth}\raggedright
Access Level
\end{minipage} \\
\midrule\noalign{}
\endhead
\bottomrule\noalign{}
\endlastfoot
\textbf{User Mode} & Applications run here & Limited privileges \\
\textbf{Kernel Mode} & OS core functions & Full hardware access \\
\textbf{System Call Interface} & Communication bridge & Controlled
transition \\
\end{longtable}
}

\textbf{Key Components:}

\begin{itemize}
\item
  \textbf{Shell}: Command interpreter interface
\item
  \textbf{Kernel}: Core system management
\item
  \textbf{File System}: Hierarchical data organization
\item
  \textbf{Device Drivers}: Hardware abstraction layer
\item
  \textbf{Security Model}: Permission-based access control
\item
  \textbf{Modularity}: Loadable kernel modules for flexibility
\item
  \textbf{Portability}: Runs on multiple hardware platforms
\end{itemize}

\end{solutionbox}
\begin{mnemonicbox}
``Users call Kernel for Hardware''

\end{mnemonicbox}
\subsection*{Question 3(a) [3 marks]}\label{q3a}

\textbf{Differentiate between Open-source Software and Proprietary
Software.}

\begin{solutionbox}

\textbf{Comparison Table:}

{\def\LTcaptype{none} % do not increment counter
\begin{longtable}[]{@{}
  >{\raggedright\arraybackslash}p{(\linewidth - 4\tabcolsep) * \real{0.1600}}
  >{\raggedright\arraybackslash}p{(\linewidth - 4\tabcolsep) * \real{0.4200}}
  >{\raggedright\arraybackslash}p{(\linewidth - 4\tabcolsep) * \real{0.4200}}@{}}
\toprule\noalign{}
\begin{minipage}[b]{\linewidth}\raggedright
Aspect
\end{minipage} & \begin{minipage}[b]{\linewidth}\raggedright
Open-source Software
\end{minipage} & \begin{minipage}[b]{\linewidth}\raggedright
Proprietary Software
\end{minipage} \\
\midrule\noalign{}
\endhead
\bottomrule\noalign{}
\endlastfoot
\textbf{Source Code} & Freely available & Closed and protected \\
\textbf{Cost} & Usually free & Commercial license required \\
\textbf{Modification} & Can be modified & Cannot be modified \\
\textbf{Examples} & Linux, Firefox, LibreOffice & Windows, MS Office,
Photoshop \\
\textbf{Support} & Community-based & Vendor-provided \\
\textbf{Licensing} & GPL, MIT, Apache & EULA, Commercial \\
\end{longtable}
}

\textbf{Key Differences:}

\begin{itemize}
\tightlist
\item
  \textbf{Freedom}: Open-source allows complete customization
\item
  \textbf{Security}: Open code enables community security reviews
\item
  \textbf{Vendor Lock-in}: Proprietary creates dependency on vendor
\end{itemize}

\end{solutionbox}
\begin{mnemonicbox}
``Open Shares, Proprietary Protects''

\end{mnemonicbox}
\subsection*{Question 3(b) [4 marks]}\label{q3b}

\textbf{Explain Ethernet Cable.}

\begin{solutionbox}

Ethernet cable is the standard wired networking medium for LAN
connections.

\textbf{Cable Types:}

\begin{center}
\textbf{Mermaid Diagram (Code)}
\begin{verbatim}
{Shaded}
{Highlighting}[]
graph TD
    subgraph "Ethernet Cables"
        UTP[Unshielded Twisted Pair]
        STP[Shielded Twisted Pair]
        Coax[Coaxial Cable]
        Fiber[Fiber Optic]
    end
    
    UTP {-{-}{} Cat5[Cat 5/5e/6/6a]}
    Fiber {-{-}{} SM[Single Mode]}
    Fiber {-{-}{} MM[Multi Mode]}
{Highlighting}
{Shaded}
\end{verbatim}
\end{center}

\textbf{Cable Specifications:}

{\def\LTcaptype{none} % do not increment counter
\begin{longtable}[]{@{}llll@{}}
\toprule\noalign{}
Type & Speed & Distance & Usage \\
\midrule\noalign{}
\endhead
\bottomrule\noalign{}
\endlastfoot
\textbf{Cat 5e} & 1 Gbps & 100m & Basic networking \\
\textbf{Cat 6} & 10 Gbps & 55m & High-speed LAN \\
\textbf{Cat 6a} & 10 Gbps & 100m & Enterprise networks \\
\textbf{Fiber Optic} & 100+ Gbps & 40km+ & Long-distance, high-speed \\
\end{longtable}
}

\begin{itemize}
\tightlist
\item
  \textbf{Connector Type}: RJ-45 for twisted pair cables
\item
  \textbf{Wiring Standards}: T568A and T568B color codes
\item
  \textbf{Applications}: Internet connectivity, file sharing, VoIP
\end{itemize}

\end{solutionbox}
\begin{mnemonicbox}
``Twisted pairs Carry Digital Data''

\end{mnemonicbox}
\subsection*{Question 3(c) [7 marks]}\label{q3c}

\textbf{Explain Time Division Multiplexing with diagram.}

\begin{solutionbox}

TDM allows multiple signals to share single transmission medium by
allocating time slots.

\textbf{TDM Process:}

\begin{verbatim}
gantt
    title Time Division Multiplexing
    dateFormat X
    axisFormat \%s
    
    section Channel A
    Slot A1 :0, 1
    Slot A2 :4, 5
    Slot A3 :8, 9
    
    section Channel B
    Slot B1 :1, 2
    Slot B2 :5, 6
    Slot B3 :9, 10
    
    section Channel C
    Slot C1 :2, 3
    Slot C2 :6, 7
    Slot C3 :10, 11
    
    section Channel D
    Slot D1 :3, 4
    Slot D2 :7, 8
    Slot D3 :11, 12
\end{verbatim}

\textbf{System Components:}

{\def\LTcaptype{none} % do not increment counter
\begin{longtable}[]{@{}
  >{\raggedright\arraybackslash}p{(\linewidth - 4\tabcolsep) * \real{0.3667}}
  >{\raggedright\arraybackslash}p{(\linewidth - 4\tabcolsep) * \real{0.3333}}
  >{\raggedright\arraybackslash}p{(\linewidth - 4\tabcolsep) * \real{0.3000}}@{}}
\toprule\noalign{}
\begin{minipage}[b]{\linewidth}\raggedright
Component
\end{minipage} & \begin{minipage}[b]{\linewidth}\raggedright
Function
\end{minipage} & \begin{minipage}[b]{\linewidth}\raggedright
Purpose
\end{minipage} \\
\midrule\noalign{}
\endhead
\bottomrule\noalign{}
\endlastfoot
\textbf{Multiplexer} & Combines input signals & Single transmission \\
\textbf{Time Slots} & Fixed duration intervals & Fair channel access \\
\textbf{Demultiplexer} & Separates combined signal & Original signal
recovery \\
\textbf{Synchronization} & Maintains timing alignment & Error-free
transmission \\
\end{longtable}
}

\textbf{Types of TDM:}

\begin{itemize}
\item
  \textbf{Synchronous TDM}: Fixed time slots for each channel
\item
  \textbf{Asynchronous TDM}: Dynamic slot allocation based on demand
\item
  \textbf{Statistical TDM}: Optimizes bandwidth utilization
\item
  \textbf{Advantages}: Efficient bandwidth usage, digital compatibility
\item
  \textbf{Applications}: Telephone systems, digital TV broadcasting
\item
  \textbf{Bandwidth Efficiency}: Multiple channels share single link
\end{itemize}

\end{solutionbox}
\begin{mnemonicbox}
``Time Divides Multiple Signals''

\end{mnemonicbox}
\subsection*{Question 3(a OR) [3
marks]}\label{question-3a-or-3-marks}

\textbf{Differentiate between Hard Real Time and Soft Real Time
Operating System.}

\begin{solutionbox}

\textbf{Comparison Table:}

{\def\LTcaptype{none} % do not increment counter
\begin{longtable}[]{@{}
  >{\raggedright\arraybackslash}p{(\linewidth - 4\tabcolsep) * \real{0.2000}}
  >{\raggedright\arraybackslash}p{(\linewidth - 4\tabcolsep) * \real{0.4000}}
  >{\raggedright\arraybackslash}p{(\linewidth - 4\tabcolsep) * \real{0.4000}}@{}}
\toprule\noalign{}
\begin{minipage}[b]{\linewidth}\raggedright
Aspect
\end{minipage} & \begin{minipage}[b]{\linewidth}\raggedright
Hard Real Time
\end{minipage} & \begin{minipage}[b]{\linewidth}\raggedright
Soft Real Time
\end{minipage} \\
\midrule\noalign{}
\endhead
\bottomrule\noalign{}
\endlastfoot
\textbf{Deadline} & Must be met absolutely & Preferred but flexible \\
\textbf{Consequences} & System failure if missed & Performance
degradation \\
\textbf{Examples} & Aircraft control, Pacemaker & Video streaming,
Gaming \\
\textbf{Response Time} & Guaranteed maximum & Best effort basis \\
\textbf{Cost} & High development cost & Moderate cost \\
\textbf{Reliability} & Critical system reliability & User experience
focused \\
\end{longtable}
}

\textbf{Key Characteristics:}

\begin{itemize}
\tightlist
\item
  \textbf{Hard RT}: Zero tolerance for deadline misses
\item
  \textbf{Soft RT}: Occasional delays acceptable
\item
  \textbf{Applications}: Safety-critical vs user-interactive systems
\end{itemize}

\end{solutionbox}
\begin{mnemonicbox}
``Hard requires Precision, Soft allows Flexibility''

\end{mnemonicbox}
\subsection*{Question 3(b OR) [4
marks]}\label{question-3b-or-4-marks}

\textbf{Explain Transmission Modes.}

\begin{solutionbox}

Transmission modes define direction of data flow between communicating
devices.

\textbf{Mode Types:}

\begin{center}
\textbf{Mermaid Diagram (Code)}
\begin{verbatim}
{Shaded}
{Highlighting}[]
graph TD
    subgraph "Transmission Modes"
        S[Simplex]
        HD[Half Duplex]  
        FD[Full Duplex]
    end
    
    S {-{-}{} One[One Direction Only]}
    HD {-{-}{} Alt[Alternate Directions]}
    FD {-{-}{} Both[Both Directions Simultaneously]}
{Highlighting}
{Shaded}
\end{verbatim}
\end{center}

\textbf{Detailed Comparison:}

{\def\LTcaptype{none} % do not increment counter
\begin{longtable}[]{@{}
  >{\raggedright\arraybackslash}p{(\linewidth - 6\tabcolsep) * \real{0.1463}}
  >{\raggedright\arraybackslash}p{(\linewidth - 6\tabcolsep) * \real{0.2683}}
  >{\raggedright\arraybackslash}p{(\linewidth - 6\tabcolsep) * \real{0.2439}}
  >{\raggedright\arraybackslash}p{(\linewidth - 6\tabcolsep) * \real{0.3415}}@{}}
\toprule\noalign{}
\begin{minipage}[b]{\linewidth}\raggedright
Mode
\end{minipage} & \begin{minipage}[b]{\linewidth}\raggedright
Data Flow
\end{minipage} & \begin{minipage}[b]{\linewidth}\raggedright
Examples
\end{minipage} & \begin{minipage}[b]{\linewidth}\raggedright
Applications
\end{minipage} \\
\midrule\noalign{}
\endhead
\bottomrule\noalign{}
\endlastfoot
\textbf{Simplex} & One direction only & Radio, TV broadcast &
Broadcasting systems \\
\textbf{Half Duplex} & Both directions, not simultaneous &
Walkie-talkie, CB radio & Two-way radios \\
\textbf{Full Duplex} & Both directions simultaneously & Telephone,
Ethernet & Modern communication \\
\end{longtable}
}

\begin{itemize}
\tightlist
\item
  \textbf{Bandwidth Efficiency}: Full duplex maximizes channel
  utilization
\item
  \textbf{Cost Factor}: Simplex cheapest, full duplex most expensive
\item
  \textbf{Use Cases}: Choose based on application requirements
\end{itemize}

\end{solutionbox}
\begin{mnemonicbox}
``Simplex Single, Half switches, Full flows Both''

\end{mnemonicbox}
\subsection*{Question 3(c OR) [7
marks]}\label{question-3c-or-7-marks}

\textbf{List out types of Analog Modulation. Explain Amplitude
Modulation with diagram.}

\begin{solutionbox}

\textbf{Types of Analog Modulation:}

\begin{enumerate}
\tightlist
\item
  \textbf{Amplitude Modulation (AM)}
\item
  \textbf{Frequency Modulation (FM)}
\item
  \textbf{Phase Modulation (PM)}
\end{enumerate}

\textbf{Amplitude Modulation Process:}

\begin{verbatim}
graph TB
    subgraph "AM Modulation"
        MS[Message Signal] {-{-} M[Modulator]}
        CS[Carrier Signal] {-{-} M}
        M {-{-} AMS[AM Signal]}
    end
    
    subgraph "Waveforms"
        MW[Message Wave {- Low Frequency]}
        CW[Carrier Wave {- High Frequency]}
        AMW[AM Wave {- Modulated Output]}
    end
\end{verbatim}

\textbf{AM Characteristics:}

{\def\LTcaptype{none} % do not increment counter
\begin{longtable}[]{@{}
  >{\raggedright\arraybackslash}p{(\linewidth - 4\tabcolsep) * \real{0.2750}}
  >{\raggedright\arraybackslash}p{(\linewidth - 4\tabcolsep) * \real{0.3250}}
  >{\raggedright\arraybackslash}p{(\linewidth - 4\tabcolsep) * \real{0.4000}}@{}}
\toprule\noalign{}
\begin{minipage}[b]{\linewidth}\raggedright
Parameter
\end{minipage} & \begin{minipage}[b]{\linewidth}\raggedright
Description
\end{minipage} & \begin{minipage}[b]{\linewidth}\raggedright
Typical Values
\end{minipage} \\
\midrule\noalign{}
\endhead
\bottomrule\noalign{}
\endlastfoot
\textbf{Carrier Frequency} & High frequency base signal & 550-1600 kHz
(AM radio) \\
\textbf{Message Frequency} & Information signal & 20 Hz - 20 kHz
(audio) \\
\textbf{Modulation Index} & Depth of modulation & 0 to 1 (0-100\%) \\
\textbf{Bandwidth} & Frequency spectrum used & 2 \times Message frequency \\
\end{longtable}
}

\textbf{Mathematical Expression:}

\begin{itemize}
\tightlist
\item
  \textbf{AM Signal}: s(t) = Ac[1 + m·cos(ωmt)]cos(ωct)
\item
  \textbf{Where}: Ac = carrier amplitude, m = modulation index
\end{itemize}

\textbf{Applications:}

\begin{itemize}
\item
  \textbf{Broadcasting}: AM radio stations
\item
  \textbf{Aviation}: Air traffic control communication
\item
  \textbf{Citizens Band}: CB radio systems
\item
  \textbf{Advantages}: Simple implementation, low cost receivers
\item
  \textbf{Disadvantages}: Susceptible to noise, power inefficient
\end{itemize}

\end{solutionbox}
\begin{mnemonicbox}
``Amplitude Varies with Message''

\end{mnemonicbox}
\subsection*{Question 4(a) [3 marks]}\label{q4a}

\textbf{Draw Diagram of FSK AND PSK.}

\begin{solutionbox}

\textbf{Frequency Shift Keying (FSK):}

\begin{verbatim}
Binary Data:  1    0    1    1    0
             
FSK Signal:   ╭╲╱╲╱╲╱╲╱╮  ╭╱╲╱╲╱╲╱╮  ╭╲╱╲╱╲╱╲╱╮
             ╱           ╲╱         ╲╱           ╲
            ╱             ╲         ╱             ╲
           ╱               ╲\_\_\_\_\_\_\_╱               ╲
          
          f1 (High Freq)    f2 (Low Freq)    f1 (High Freq)
\end{verbatim}

\textbf{Phase Shift Keying (PSK):}

\begin{verbatim}
Binary Data:  1      0      1      1      0
             
PSK Signal:   ╭─╲ ╱─╮   ╭╲ ╱╮   ╭─╲ ╱─╮   ╭─╲ ╱─╮   ╭╲ ╱╮
             ╱   ╲╱   ╲ ╱  ╲╱  ╲ ╱   ╲╱   ╲ ╱   ╲╱   ╲ ╱  ╲╱  ╲
            ╱         ╲╱        ╲╱         ╲╱         ╲╱        ╲
           
           0^ Phase      180^ Phase    0^ Phase     0^ Phase    180^ Phase
\end{verbatim}

\textbf{Key Differences:}

\begin{itemize}
\tightlist
\item
  \textbf{FSK}: Different frequencies for 1 and 0
\item
  \textbf{PSK}: Different phases for 1 and 0
\end{itemize}

\end{solutionbox}
\begin{mnemonicbox}
``FSK changes Frequency, PSK changes Phase''

\end{mnemonicbox}
\subsection*{Question 4(b) [4 marks]}\label{q4b}

\textbf{If number of links in mesh topology are 45 than find maximum
number of required nodes.}

\begin{solutionbox}

\textbf{Formula for Mesh Topology:} Number of links = n(n-1)/2

Where n = number of nodes

\textbf{Given}: Number of links = 45

\textbf{Calculation:} 45 = n(n-1)/2 90 = n(n-1) n^{2} - n - 90 = 0

\textbf{Solving Quadratic Equation:} Using quadratic formula: n = [-b
\pm \sqrt(b^{2} - 4ac)] / 2a

Where

a=1,

b=-1,

c=-90


n = [1 \pm \sqrt(1 + 360)] / 2

n = [1 \pm \sqrt361] / 2\\

n = [1 \pm 19] / 2

\textbf{Solutions:} n = (1 + 19)/2 = 10 or n = (1 - 19)/2 = -9

\end{solutionbox}
\begin{solutionbox}
Maximum number of nodes = 10

\textbf{Verification}: 10(10-1)/2 = 10\times9/2 = 45 ✓

\end{solutionbox}
\begin{mnemonicbox}
``n nodes need n(n-1)/2 links''

\end{mnemonicbox}
\subsection*{Question 4(c) [7 marks]}\label{q4c}

\textbf{Explain OSI Model with diagram.}

\begin{solutionbox}

OSI (Open Systems Interconnection) model defines seven layers for
network communication.

\textbf{OSI Layer Stack:}

\begin{verbatim}
graph TB
    subgraph "OSI Model"
        L7[Layer 7: Application]
        L6[Layer 6: Presentation] 
        L5[Layer 5: Session]
        L4[Layer 4: Transport]
        L3[Layer 3: Network]
        L2[Layer 2: Data Link]
        L1[Layer 1: Physical]
    end
    
    L7 {-{-} L6}
    L6 {-{-} L5}
    L5 {-{-} L4}
    L4 {-{-} L3}
    L3 {-{-} L2}
    L2 {-{-} L1}
\end{verbatim}

\textbf{Layer Functions:}

{\def\LTcaptype{none} % do not increment counter
\begin{longtable}[]{@{}
  >{\raggedright\arraybackslash}p{(\linewidth - 8\tabcolsep) * \real{0.1628}}
  >{\raggedright\arraybackslash}p{(\linewidth - 8\tabcolsep) * \real{0.1395}}
  >{\raggedright\arraybackslash}p{(\linewidth - 8\tabcolsep) * \real{0.2326}}
  >{\raggedright\arraybackslash}p{(\linewidth - 8\tabcolsep) * \real{0.2558}}
  >{\raggedright\arraybackslash}p{(\linewidth - 8\tabcolsep) * \real{0.2093}}@{}}
\toprule\noalign{}
\begin{minipage}[b]{\linewidth}\raggedright
Layer
\end{minipage} & \begin{minipage}[b]{\linewidth}\raggedright
Name
\end{minipage} & \begin{minipage}[b]{\linewidth}\raggedright
Function
\end{minipage} & \begin{minipage}[b]{\linewidth}\raggedright
Protocols
\end{minipage} & \begin{minipage}[b]{\linewidth}\raggedright
Devices
\end{minipage} \\
\midrule\noalign{}
\endhead
\bottomrule\noalign{}
\endlastfoot
\textbf{7} & Application & User interface & HTTP, FTP, SMTP &
Gateways \\
\textbf{6} & Presentation & Data formatting & SSL, JPEG, MPEG &
Gateways \\
\textbf{5} & Session & Connection management & NetBIOS, RPC &
Gateways \\
\textbf{4} & Transport & End-to-end delivery & TCP, UDP & Gateways \\
\textbf{3} & Network & Routing & IP, ICMP & Routers \\
\textbf{2} & Data Link & Frame transmission & Ethernet, PPP &
Switches \\
\textbf{1} & Physical & Bit transmission & Ethernet cables & Hubs,
Repeaters \\
\end{longtable}
}

\textbf{Data Flow Process:}

\begin{itemize}
\item
  \textbf{Encapsulation}: Data moves down layers, headers added
\item
  \textbf{Transmission}: Physical layer sends bits across medium
\item
  \textbf{Decapsulation}: Receiving end moves up layers, headers removed
\item
  \textbf{Standardization}: Enables interoperability between vendors
\item
  \textbf{Modularity}: Each layer has specific responsibilities
\item
  \textbf{Troubleshooting}: Isolates problems to specific layers
\end{itemize}

\end{solutionbox}
\begin{mnemonicbox}
``All People Seem To Need Data Processing''

\end{mnemonicbox}
\subsection*{Question 4(a OR) [3
marks]}\label{question-4a-or-3-marks}

\textbf{Explain Classful IPv4 addressing scheme with example.}

\begin{solutionbox}

IPv4 classful addressing divides IP space into predefined classes based
on network size.

\textbf{Class Structure:}

{\def\LTcaptype{none} % do not increment counter
\begin{longtable}[]{@{}lllll@{}}
\toprule\noalign{}
Class & Range & Default Mask & Networks & Hosts per Network \\
\midrule\noalign{}
\endhead
\bottomrule\noalign{}
\endlastfoot
\textbf{A} & 1-126 & /8 (255.0.0.0) & 126 & 16,777,214 \\
\textbf{B} & 128-191 & /16 (255.255.0.0) & 16,384 & 65,534 \\
\textbf{C} & 192-223 & /24 (255.255.255.0) & 2,097,152 & 254 \\
\end{longtable}
}

\textbf{Examples:}

\begin{itemize}
\tightlist
\item
  \textbf{Class A}: 10.0.0.1 (Large networks like ISPs)
\item
  \textbf{Class B}: 172.16.0.1 (Medium networks like universities)
\item
  \textbf{Class C}: 192.168.1.1 (Small networks like offices)
\end{itemize}

\textbf{Address Format:}

\begin{itemize}
\tightlist
\item
  \textbf{Class A}: N.H.H.H (N=Network, H=Host)
\item
  \textbf{Class B}: N.N.H.H
\item
  \textbf{Class C}: N.N.N.H
\end{itemize}

\end{solutionbox}
\begin{mnemonicbox}
``A for All (large), B for Business (medium), C for
Company (small)''

\end{mnemonicbox}
\subsection*{Question 4(b OR) [4
marks]}\label{question-4b-or-4-marks}

\textbf{If number of nodes in mesh topology are 11 than find minimum
number of required links.}

\begin{solutionbox}

\textbf{Formula for Mesh Topology:} Number of links = n(n-1)/2

Where n = number of nodes

\textbf{Given}: Number of nodes = 11

\textbf{Calculation:} Number of links = 11(11-1)/2 = 11 \times 10/2 = 110/2 =
55

\end{solutionbox}
\begin{solutionbox}
Minimum number of required links = 55

\textbf{Explanation:}

\begin{itemize}
\tightlist
\item
  In mesh topology, every node connects to every other node
\item
  Each node has (n-1) connections
\item
  Total connections = n(n-1), but each link counted twice
\item
  Therefore, actual links = n(n-1)/2
\end{itemize}

\end{solutionbox}
\begin{mnemonicbox}
``Every node connects to Every other''

\end{mnemonicbox}
\subsection*{Question 4(c OR) [7
marks]}\label{question-4c-or-7-marks}

\textbf{Explain domain name system (DNS) with diagram.}

\begin{solutionbox}

DNS translates human-readable domain names into IP addresses for network
routing.

\textbf{DNS Hierarchy:}

\begin{verbatim}
graph TB
    subgraph "DNS Hierarchy"
        Root["Root Servers (.)"]
        TLD["Top Level Domain (.com, .org, .edu)"]
        SLD["Second Level Domain (google, example)"]
        Sub["Subdomain (www, mail, ftp)"]
    end
    
    Root {-{-} TLD}
    TLD {-{-} SLD}
    SLD {-{-} Sub}
    
    subgraph "DNS Resolution Process"
        Client[Client] {-{-} Local[Local DNS Server]}
        Local {-{-} RootNS[Root Name Server]}
        RootNS {-{-} TLDNS[TLD Name Server]}
        TLDNS {-{-} AuthNS[Authoritative Name Server]}
        AuthNS {-{-} Local}
        Local {-{-} Client}
    end
\end{verbatim}

\textbf{DNS Components:}

{\def\LTcaptype{none} % do not increment counter
\begin{longtable}[]{@{}
  >{\raggedright\arraybackslash}p{(\linewidth - 4\tabcolsep) * \real{0.3548}}
  >{\raggedright\arraybackslash}p{(\linewidth - 4\tabcolsep) * \real{0.3226}}
  >{\raggedright\arraybackslash}p{(\linewidth - 4\tabcolsep) * \real{0.3226}}@{}}
\toprule\noalign{}
\begin{minipage}[b]{\linewidth}\raggedright
Component
\end{minipage} & \begin{minipage}[b]{\linewidth}\raggedright
Function
\end{minipage} & \begin{minipage}[b]{\linewidth}\raggedright
Examples
\end{minipage} \\
\midrule\noalign{}
\endhead
\bottomrule\noalign{}
\endlastfoot
\textbf{Root Servers} & Top-level authority & 13 root servers
worldwide \\
\textbf{TLD Servers} & Manage top-level domains & .com, .org, .edu,
.gov \\
\textbf{Authoritative Servers} & Hold actual DNS records & Company DNS
servers \\
\textbf{Local DNS Servers} & Cache and forward queries & ISP DNS
servers \\
\end{longtable}
}

\textbf{DNS Record Types:}

\begin{itemize}
\tightlist
\item
  \textbf{A Record}: Maps domain to IPv4 address
\item
  \textbf{AAAA Record}: Maps domain to IPv6 address\\
\item
  \textbf{CNAME}: Creates domain aliases
\item
  \textbf{MX Record}: Specifies mail servers
\item
  \textbf{NS Record}: Identifies name servers
\end{itemize}

\textbf{Resolution Process:}

\begin{enumerate}
\tightlist
\item
  \textbf{Client Query}: User enters domain name
\item
  \textbf{Local Cache Check}: Check local DNS cache
\item
  \textbf{Recursive Query}: Local server queries hierarchy
\item
  \textbf{Response Return}: IP address returned to client
\end{enumerate}

\begin{itemize}
\tightlist
\item
  \textbf{Caching}: Improves performance and reduces network traffic
\item
  \textbf{Redundancy}: Multiple servers ensure availability
\item
  \textbf{Load Distribution}: Balances query load across servers
\end{itemize}

\end{solutionbox}
\begin{mnemonicbox}
``Domains Need Systematic name-to-address
translation''

\end{mnemonicbox}
\subsection*{Question 5(a) [3 marks]}\label{q5a}

\textbf{Explain the need of IPv6.}

\begin{solutionbox}

IPv6 was developed to address limitations of IPv4 and support future
internet growth.

\textbf{Key Requirements:}

{\def\LTcaptype{none} % do not increment counter
\begin{longtable}[]{@{}
  >{\raggedright\arraybackslash}p{(\linewidth - 4\tabcolsep) * \real{0.2250}}
  >{\raggedright\arraybackslash}p{(\linewidth - 4\tabcolsep) * \real{0.4000}}
  >{\raggedright\arraybackslash}p{(\linewidth - 4\tabcolsep) * \real{0.3750}}@{}}
\toprule\noalign{}
\begin{minipage}[b]{\linewidth}\raggedright
Problem
\end{minipage} & \begin{minipage}[b]{\linewidth}\raggedright
IPv4 Limitation
\end{minipage} & \begin{minipage}[b]{\linewidth}\raggedright
IPv6 Solution
\end{minipage} \\
\midrule\noalign{}
\endhead
\bottomrule\noalign{}
\endlastfoot
\textbf{Address Space} & 4.3 billion addresses & 340 undecillion
addresses \\
\textbf{NAT Complexity} & Private-public translation & End-to-end
connectivity \\
\textbf{Security} & Optional IPSec & Built-in IPSec support \\
\textbf{Mobile Support} & Limited mobility & Native mobility support \\
\end{longtable}
}

\textbf{Critical Needs:}

\begin{itemize}
\item
  \textbf{IoT Explosion}: Billions of connected devices need unique
  addresses
\item
  \textbf{Mobile Growth}: Smartphones and tablets require internet
  access
\item
  \textbf{Global Connectivity}: Emerging markets joining internet
\item
  \textbf{Address Format}: 128-bit vs 32-bit in IPv4
\item
  \textbf{Simplified Header}: More efficient packet processing
\item
  \textbf{No Fragmentation}: Routers don't fragment packets
\end{itemize}

\end{solutionbox}
\begin{mnemonicbox}
``IPv6 provides Infinite addresses for Internet
growth''

\end{mnemonicbox}
\subsection*{Question 5(b) [4 marks]}\label{q5b}

\textbf{Explain confidentiality using Asymmetric Key encryption.}

\begin{solutionbox}

Asymmetric encryption uses key pairs (public-private) to ensure data
confidentiality.

\textbf{Encryption Process:}

\begin{verbatim}
sequenceDiagram
    participant S as Sender
    participant R as Receiver
    
    Note over R: Generate Key Pair
    R{-S: Public Key}
    Note over S: Encrypt with Public Key
    S{-R: Encrypted Message}
    Note over R: Decrypt with Private Key
    R{-R: Original Message}
\end{verbatim}

\textbf{Key Characteristics:}

{\def\LTcaptype{none} % do not increment counter
\begin{longtable}[]{@{}lll@{}}
\toprule\noalign{}
Aspect & Description & Security Benefit \\
\midrule\noalign{}
\endhead
\bottomrule\noalign{}
\endlastfoot
\textbf{Public Key} & Freely distributed & Anyone can encrypt \\
\textbf{Private Key} & Kept secret & Only owner can decrypt \\
\textbf{Key Pair} & Mathematically related & Secure communication \\
\textbf{Algorithm} & RSA, ECC, DSA & Strong encryption \\
\end{longtable}
}

\textbf{Confidentiality Process:}

\begin{itemize}
\item
  \textbf{Step 1}: Receiver generates public-private key pair
\item
  \textbf{Step 2}: Public key shared with sender
\item
  \textbf{Step 3}: Sender encrypts message with public key
\item
  \textbf{Step 4}: Only receiver's private key can decrypt
\item
  \textbf{No Key Exchange}: Eliminates key distribution problem
\item
  \textbf{Non-repudiation}: Sender cannot deny sending message
\item
  \textbf{Digital Signatures}: Authentication and integrity
\end{itemize}

\end{solutionbox}
\begin{mnemonicbox}
``Public locks, Private unlocks''

\end{mnemonicbox}
\subsection*{Question 5(c) [7 marks]}\label{q5c}

\textbf{Explain man-in-middle attack with example.}

\begin{solutionbox}

Man-in-the-middle attack intercepts communication between two parties
without their knowledge.

\textbf{Attack Process:}

\begin{verbatim}
sequenceDiagram
    participant A as Alice
    participant M as Mallory (Attacker)
    participant B as Bob
    
    A{-M: Message for Bob}
    Note over M: Intercepts \& Reads
    M{-B: Modified/Original Message}
    B{-M: Reply for Alice}
    Note over M: Intercepts \& Reads
    M{-A: Modified/Original Reply}
\end{verbatim}

\textbf{Attack Stages:}

{\def\LTcaptype{none} % do not increment counter
\begin{longtable}[]{@{}lll@{}}
\toprule\noalign{}
Stage & Attacker Action & Victim Impact \\
\midrule\noalign{}
\endhead
\bottomrule\noalign{}
\endlastfoot
\textbf{Interception} & Position between parties & Unknown to victims \\
\textbf{Decryption} & Break/bypass encryption & Access to data \\
\textbf{Modification} & Alter messages & False information \\
\textbf{Re-encryption} & Hide tampering & Maintain illusion \\
\end{longtable}
}

\textbf{Real-world Example:}

\begin{itemize}
\tightlist
\item
  \textbf{Scenario}: Online banking session
\item
  \textbf{Attack}: Attacker on public WiFi intercepts traffic
\item
  \textbf{Method}: Creates fake access point ``Free\_WiFi''
\item
  \textbf{Result}: Steals banking credentials and transfers money
\end{itemize}

\textbf{Common Targets:}

\begin{itemize}
\tightlist
\item
  \textbf{Public WiFi}: Coffee shops, airports, hotels
\item
  \textbf{Email Communication}: Corporate communications
\item
  \textbf{Online Shopping}: Credit card information theft
\item
  \textbf{Social Media}: Personal information harvesting
\end{itemize}

\textbf{Prevention Measures:}

\begin{itemize}
\tightlist
\item
  \textbf{SSL/TLS}: End-to-end encryption protocols
\item
  \textbf{VPN Usage}: Secure tunnel for all traffic
\item
  \textbf{Certificate Verification}: Check website authenticity
\item
  \textbf{Avoid Public WiFi}: Use cellular data for sensitive tasks
\end{itemize}

\end{solutionbox}
\begin{mnemonicbox}
``Mallory Intercepts Messages between Alice and Bob''

\end{mnemonicbox}
\subsection*{Question 5(a OR) [3
marks]}\label{question-5a-or-3-marks}

\textbf{Give the name of OSI model layers with respect to the following
devices.} \textbf{1. Repeater 2. Router 3. Switch}

\begin{solutionbox}

\textbf{Device-Layer Mapping:}

{\def\LTcaptype{none} % do not increment counter
\begin{longtable}[]{@{}llll@{}}
\toprule\noalign{}
Device & OSI Layer & Layer Name & Function \\
\midrule\noalign{}
\endhead
\bottomrule\noalign{}
\endlastfoot
\textbf{Repeater} & Layer 1 & Physical Layer & Signal amplification \\
\textbf{Router} & Layer 3 & Network Layer & IP routing decisions \\
\textbf{Switch} & Layer 2 & Data Link Layer & Frame switching \\
\end{longtable}
}

\textbf{Detailed Functions:}

\begin{itemize}
\tightlist
\item
  \textbf{Repeater}: Regenerates electrical signals to extend network
  distance
\item
  \textbf{Router}: Makes forwarding decisions based on IP addresses
\item
  \textbf{Switch}: Forwards frames based on MAC addresses
\end{itemize}

\end{solutionbox}
\begin{mnemonicbox}
``Repeaters work Physically, Switches link Data,
Routers route Networks''

\end{mnemonicbox}
\subsection*{Question 5(b OR) [4
marks]}\label{question-5b-or-4-marks}

\textbf{Explain confidentiality using Symmetric Key encryption.}

\begin{solutionbox}

Symmetric encryption uses single shared key for both encryption and
decryption.

\textbf{Encryption Process:}

\begin{center}
\textbf{Mermaid Diagram (Code)}
\begin{verbatim}
{Shaded}
{Highlighting}[]
graph LR
    subgraph "Symmetric Encryption"
    direction LR
        PT[Plain Text] {-{-}{} E[Encryption]}
        K[Shared Key] {-{-}{} E}
        E {-{-}{} CT[Cipher Text]}
        CT {-{-}{} D[Decryption]}
        K {-{-}{} D}
        D {-{-}{} PT2[Plain Text]}
    end
{Highlighting}
{Shaded}
\end{verbatim}
\end{center}

\textbf{Key Characteristics:}

{\def\LTcaptype{none} % do not increment counter
\begin{longtable}[]{@{}lll@{}}
\toprule\noalign{}
Feature & Description & Example \\
\midrule\noalign{}
\endhead
\bottomrule\noalign{}
\endlastfoot
\textbf{Single Key} & Same key for encrypt/decrypt & AES-256 key \\
\textbf{Fast Processing} & Efficient algorithms & Real-time
communication \\
\textbf{Key Distribution} & Secure key sharing required & Pre-shared
keys \\
\textbf{Algorithm Types} & Block and stream ciphers & AES, DES, RC4 \\
\end{longtable}
}

\textbf{Confidentiality Mechanism:}

\begin{itemize}
\item
  \textbf{Shared Secret}: Both parties must have same key
\item
  \textbf{Encryption}: Sender encrypts with shared key
\item
  \textbf{Transmission}: Cipher text sent over insecure channel
\item
  \textbf{Decryption}: Receiver decrypts with same key
\item
  \textbf{Advantages}: Fast execution, low computational overhead
\item
  \textbf{Disadvantages}: Key distribution challenge, scalability issues
\item
  \textbf{Applications}: VPN tunnels, file encryption, database security
\end{itemize}

\end{solutionbox}
\begin{mnemonicbox}
``Same key Encrypts and Decrypts''

\end{mnemonicbox}
\subsection*{Question 5(c OR) [7
marks]}\label{question-5c-or-7-marks}

\textbf{Explain denial of service attack with example.}

\begin{solutionbox}

DoS attack makes network resources unavailable to legitimate users by
overwhelming the system.

\textbf{Attack Types:}

\begin{verbatim}
graph TB
    subgraph "DoS Attack Types"
        V[Volume{-based]}
        P[Protocol{-based]}
        A[Application{-based]}
    end
    
    V {-{-} VE[Volumetric Examples]}
    P {-{-} PE[Protocol Examples]}
    A {-{-} AE[Application Examples]}
    
    VE {-{-} UDP[UDP Flood]}
    VE {-{-} ICMP[ICMP Flood]}
    
    PE {-{-} SYN[SYN Flood]}
    PE {-{-} SMURF[Smurf Attack]}
    
    AE {-{-} HTTP[HTTP Flood]}
    AE {-{-} SLOW[Slowloris]}
\end{verbatim}

\textbf{Attack Categories:}

{\def\LTcaptype{none} % do not increment counter
\begin{longtable}[]{@{}
  >{\raggedright\arraybackslash}p{(\linewidth - 6\tabcolsep) * \real{0.2000}}
  >{\raggedright\arraybackslash}p{(\linewidth - 6\tabcolsep) * \real{0.2667}}
  >{\raggedright\arraybackslash}p{(\linewidth - 6\tabcolsep) * \real{0.2667}}
  >{\raggedright\arraybackslash}p{(\linewidth - 6\tabcolsep) * \real{0.2667}}@{}}
\toprule\noalign{}
\begin{minipage}[b]{\linewidth}\raggedright
Type
\end{minipage} & \begin{minipage}[b]{\linewidth}\raggedright
Method
\end{minipage} & \begin{minipage}[b]{\linewidth}\raggedright
Target
\end{minipage} & \begin{minipage}[b]{\linewidth}\raggedright
Impact
\end{minipage} \\
\midrule\noalign{}
\endhead
\bottomrule\noalign{}
\endlastfoot
\textbf{Volume-based} & Flood with traffic & Bandwidth & Network
congestion \\
\textbf{Protocol-based} & Exploit protocol weakness & Server resources &
Service unavailability \\
\textbf{Application-based} & Target application layer & Application
server & Service degradation \\
\end{longtable}
}

\textbf{Real-world Example - DDoS on E-commerce:}

\begin{itemize}
\tightlist
\item
  \textbf{Target}: Online shopping website during sale season
\item
  \textbf{Method}: Botnet of 10,000 infected computers
\item
  \textbf{Attack}: Each bot sends 100 requests per second
\item
  \textbf{Result}: 1 million requests/second overwhelm servers
\item
  \textbf{Impact}: Website crashes, customers cannot purchase, revenue
  loss
\end{itemize}

\textbf{Common DoS Techniques:}

\begin{itemize}
\tightlist
\item
  \textbf{SYN Flood}: Exploits TCP handshake process
\item
  \textbf{UDP Flood}: Sends large number of UDP packets
\item
  \textbf{Ping of Death}: Oversized ping packets crash systems
\item
  \textbf{Slowloris}: Keeps connections open to exhaust server
\end{itemize}

\textbf{Defense Strategies:}

\begin{itemize}
\tightlist
\item
  \textbf{Rate Limiting}: Restrict requests per IP address
\item
  \textbf{Firewall Rules}: Block suspicious traffic patterns
\item
  \textbf{DDoS Protection Services}: CloudFlare, AWS Shield
\item
  \textbf{Load Balancing}: Distribute traffic across servers
\item
  \textbf{Traffic Analysis}: Monitor for abnormal patterns
\end{itemize}

\textbf{Business Impact:}

\begin{itemize}
\tightlist
\item
  \textbf{Revenue Loss}: Customers cannot access services
\item
  \textbf{Reputation Damage}: Users lose trust in reliability
\item
  \textbf{Operational Cost}: Resources spent on mitigation
\item
  \textbf{Legal Issues}: SLA violations, compliance problems
\end{itemize}

\end{solutionbox}
\begin{mnemonicbox}
``Deny service by Overwhelming with requests''

\end{mnemonicbox}

\end{document}
