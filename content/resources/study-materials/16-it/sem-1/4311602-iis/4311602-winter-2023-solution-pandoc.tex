\documentclass[10pt,a4paper]{article}

% content/resources/templates/preamble.tex
\usepackage[margin=0.6in]{geometry}
\author{Milav Dabgar}
\usepackage{amsmath,amssymb,amsthm}
\usepackage{booktabs}
\usepackage{multirow}
\usepackage{xcolor}
\usepackage{tcolorbox}
\tcbuselibrary{breakable,skins}
\usepackage[colorlinks=true,linkcolor=blue]{hyperref}
\usepackage{titlesec}
\usepackage{enumitem}
\usepackage{tikz}
\usepackage{pgfplots}
\usepackage{circuitikz}
\usepackage[version=4]{mhchem}
\usepackage{longtable}
\usepackage{array}
\usepackage{float}
\usepackage{caption}
\usepackage{listings}

\lstset{
  basicstyle=\small\ttfamily,
  breaklines=true,
  breakatwhitespace=false,
  postbreak=\mbox{\textcolor{red}{$\hookrightarrow$}\space},
  float=false,
  numbers=left,
  numberstyle=\tiny\color{gray},
  numbersep=10pt,
  xleftmargin=2em,
  keywordstyle=\color{blue},
  commentstyle=\color{green!60!black},
  stringstyle=\color{purple},
  backgroundcolor=\color{gray!5},
  showstringspaces=false,
  tabsize=2,
  captionpos=b,
  keepspaces=true,
  columns=flexible
}

\pgfplotsset{compat=1.18}
\usetikzlibrary{shapes,arrows,positioning,calc,patterns,decorations.pathmorphing,decorations.markings,arrows.meta}

% Color scheme
\definecolor{headcolor}{RGB}{0,102,204}
\definecolor{keycolor}{RGB}{220,20,60}
\definecolor{solutioncolor}{RGB}{34,139,34}
\definecolor{mnemoniccolor}{RGB}{148,0,211}
\definecolor{codecolor}{RGB}{0,0,100}

% Spacing
\setlength{\parskip}{3pt}
\setlist[itemize]{nosep}
\setlist[enumerate]{nosep}

% Title formatting
\titleformat{\section}{\Large\bfseries\color{headcolor}}{\thesection}{1em}{}
\titleformat{\subsection}{\large\bfseries\color{headcolor}}{\thesubsection}{1em}{}

% Pandoc tightlist compatibility
\providecommand{\tightlist}{%
  \setlength{\itemsep}{0pt}\setlength{\parskip}{0pt}}

% Pandoc longtable compatibility
\newcounter{none}
\def\thenone{}


% content/resources/templates/english-boxes.tex
% This file is currently empty - it exists to maintain consistency with the import structure.
% Add custom environments here if needed in the future.


\begin{document}

\begin{center}
{\Huge\bfseries\color{headcolor} Subject Name Solutions}\\[5pt]
{\LARGE 4311602 -- Winter 2023}\\[3pt]
{\large Semester 1 Study Material}\\[3pt]
{\normalsize\textit{Detailed Solutions and Explanations}}
\end{center}

\vspace{10pt}

\subsection*{Question 1(a) [3 marks]}\label{q1a}

\textbf{Differentiate between Information and Knowledge.}

\begin{solutionbox}

{\def\LTcaptype{none} % do not increment counter
\begin{longtable}[]{@{}
  >{\raggedright\arraybackslash}p{(\linewidth - 4\tabcolsep) * \real{0.2727}}
  >{\raggedright\arraybackslash}p{(\linewidth - 4\tabcolsep) * \real{0.3864}}
  >{\raggedright\arraybackslash}p{(\linewidth - 4\tabcolsep) * \real{0.3409}}@{}}
\toprule\noalign{}
\begin{minipage}[b]{\linewidth}\raggedright
\textbf{Aspect}
\end{minipage} & \begin{minipage}[b]{\linewidth}\raggedright
\textbf{Information}
\end{minipage} & \begin{minipage}[b]{\linewidth}\raggedright
\textbf{Knowledge}
\end{minipage} \\
\midrule\noalign{}
\endhead
\bottomrule\noalign{}
\endlastfoot
\textbf{Definition} & Raw facts and figures & Processed information with
understanding \\
\textbf{Processing} & Data that is organized & Information combined with
experience \\
\textbf{Application} & Can be shared easily & Requires interpretation
and context \\
\end{longtable}
}

\begin{itemize}
\tightlist
\item
  \textbf{Information}: Raw facts, data, and figures that can be
  processed
\item
  \textbf{Knowledge}: Understanding gained through experience and
  learning
\end{itemize}

\end{solutionbox}
\begin{mnemonicbox}
``Information Informs, Knowledge Knows''

\end{mnemonicbox}
\subsection*{Question 1(b) [4 marks]}\label{q1b}

\textbf{Explain Functions of OS.}

\begin{solutionbox}

\textbf{Primary Functions of Operating System:}

{\def\LTcaptype{none} % do not increment counter
\begin{longtable}[]{@{}ll@{}}
\toprule\noalign{}
\textbf{Function} & \textbf{Description} \\
\midrule\noalign{}
\endhead
\bottomrule\noalign{}
\endlastfoot
\textbf{Process Management} & Controls execution of programs \\
\textbf{Memory Management} & Allocates and deallocates memory \\
\textbf{File Management} & Organizes and manages files \\
\textbf{Device Management} & Controls input/output devices \\
\end{longtable}
}

\begin{itemize}
\tightlist
\item
  \textbf{Process Control}: Scheduling and managing running programs
\item
  \textbf{Resource Allocation}: Distributing system resources
  efficiently
\item
  \textbf{User Interface}: Providing interaction between user and
  computer
\end{itemize}

\end{solutionbox}
\begin{mnemonicbox}
``PMFD - Process, Memory, File, Device''

\end{mnemonicbox}
\subsection*{Question 1(c) [7 marks]}\label{q1c}

\textbf{Define Universal gate and Build Basic gate using NAND Universal
gate.}

\begin{solutionbox}

\textbf{Universal Gate Definition:} A logic gate that can implement any
Boolean function without using any other type of gate.

\textbf{NAND Gate Truth Table:}

{\def\LTcaptype{none} % do not increment counter
\begin{longtable}[]{@{}lll@{}}
\toprule\noalign{}
A & B & NAND Output \\
\midrule\noalign{}
\endhead
\bottomrule\noalign{}
\endlastfoot
0 & 0 & 1 \\
0 & 1 & 1 \\
1 & 0 & 1 \\
1 & 1 & 0 \\
\end{longtable}
}

\textbf{Basic Gates using NAND:}

\begin{verbatim}
NOT Gate using NAND:
A {-{-}{-}{-}+}
      |
      NAND {-{-}{-}{-} Output (NOT A)}
      |
A {-{-}{-}{-}+}

AND Gate using NAND:
A {-{-}{-}{-}+}
      |
      NAND {-{-}{-}{-} NAND {-}{-}{-}{-} Output (A AND B)}
      |
B {-{-}{-}{-}+}

OR Gate using NAND:
A {-{-}{-}{-} NAND {-}{-}{-}{-}+}
                |
                NAND {-{-}{-}{-} Output (A OR B)}
                |
B {-{-}{-}{-} NAND {-}{-}{-}{-}+}
\end{verbatim}

\begin{itemize}
\tightlist
\item
  \textbf{NOT}: Single input to both NAND inputs
\item
  \textbf{AND}: NAND followed by NOT (another NAND)
\item
  \textbf{OR}: NOT both inputs, then NAND result
\end{itemize}

\end{solutionbox}
\begin{mnemonicbox}
``NAND Needs Another NAND Definitely''

\end{mnemonicbox}
\subsection*{Question 1(c OR) [7
marks]}\label{question-1c-or-7-marks}

\textbf{Perform Following Conversion:}

\begin{solutionbox}

\textbf{Conversion Solutions:}

{\def\LTcaptype{none} % do not increment counter
\begin{longtable}[]{@{}llll@{}}
\toprule\noalign{}
\textbf{From} & \textbf{To} & \textbf{Process} & \textbf{Result} \\
\midrule\noalign{}
\endhead
\bottomrule\noalign{}
\endlastfoot
(1456)_{8} & Base 16 & 8\rightarrow10\rightarrow16 & (32E)_{1}_{6} \\
(1011)_{2} & Base 10 & Binary to Decimal & (11)_{1}_{0} \\
(247.38)_{1}_{0} & Base 8 & Integer and Fraction separately & (367.3)_{8} \\
\end{longtable}
}

\textbf{Detailed Solutions:}

\begin{enumerate}
\tightlist
\item
  \textbf{(1456)_{8} = (32E)_{1}_{6}}

  \begin{itemize}
  \tightlist
  \item
    1\times8^{3} + 4\times8^{2} + 5\times8^{1} + 6\times8^{0} = 512 + 256 + 40 + 6 = (814)_{1}_{0}
  \item
    814 \div 16 = 50 remainder 14(E), 50 \div 16 = 3 remainder 2
  \item
    Result: (32E)_{1}_{6}
  \end{itemize}
\item
  \textbf{(1011)_{2} = (11)_{1}_{0}}

  \begin{itemize}
  \tightlist
  \item
    1\times2^{3} + 0\times2^{2} + 1\times2^{1} + 1\times2^{0} = 8 + 0 + 2 + 1 = (11)_{1}_{0}
  \end{itemize}
\item
  \textbf{(247.38)_{1}_{0} = (367.3)_{8}}

  \begin{itemize}
  \tightlist
  \item
    Integer: 247 \div 8 = 30 remainder 7, 30 \div 8 = 3 remainder 6, 3 \div 8 = 0
    remainder 3
  \item
    Fraction: 0.38 \times 8 = 3.04 (take 3)
  \item
    Result: (367.3)_{8}
  \end{itemize}
\end{enumerate}

\end{solutionbox}
\begin{mnemonicbox}
``Convert Carefully, Check Calculations''

\end{mnemonicbox}
\subsection*{Question 2(a) [3 marks]}\label{q2a}

\textbf{List out types of Memory.}

\begin{solutionbox}

\textbf{Memory Classification:}

{\def\LTcaptype{none} % do not increment counter
\begin{longtable}[]{@{}lll@{}}
\toprule\noalign{}
\textbf{Type} & \textbf{Examples} & \textbf{Characteristics} \\
\midrule\noalign{}
\endhead
\bottomrule\noalign{}
\endlastfoot
\textbf{Primary Memory} & RAM, ROM, Cache & Directly accessible by
CPU \\
\textbf{Secondary Memory} & HDD, SSD, CD/DVD & Non-volatile storage \\
\textbf{Cache Memory} & L1, L2, L3 & High-speed buffer memory \\
\end{longtable}
}

\begin{itemize}
\tightlist
\item
  \textbf{Volatile}: Loses data when power off (RAM)
\item
  \textbf{Non-volatile}: Retains data without power (ROM, HDD)
\item
  \textbf{Access Speed}: Cache \textgreater{} RAM \textgreater{}
  Secondary Storage
\end{itemize}

\end{solutionbox}
\begin{mnemonicbox}
``Primary Processes, Secondary Stores''

\end{mnemonicbox}
\subsection*{Question 2(b) [4 marks]}\label{q2b}

\textbf{Differentiate Kernel Mode Vs User Mode.}

\begin{solutionbox}

{\def\LTcaptype{none} % do not increment counter
\begin{longtable}[]{@{}
  >{\raggedright\arraybackslash}p{(\linewidth - 4\tabcolsep) * \real{0.2727}}
  >{\raggedright\arraybackslash}p{(\linewidth - 4\tabcolsep) * \real{0.3864}}
  >{\raggedright\arraybackslash}p{(\linewidth - 4\tabcolsep) * \real{0.3409}}@{}}
\toprule\noalign{}
\begin{minipage}[b]{\linewidth}\raggedright
\textbf{Aspect}
\end{minipage} & \begin{minipage}[b]{\linewidth}\raggedright
\textbf{Kernel Mode}
\end{minipage} & \begin{minipage}[b]{\linewidth}\raggedright
\textbf{User Mode}
\end{minipage} \\
\midrule\noalign{}
\endhead
\bottomrule\noalign{}
\endlastfoot
\textbf{Privilege Level} & Full system access & Restricted access \\
\textbf{Instructions} & All instructions allowed & Limited instruction
set \\
\textbf{Memory Access} & Complete memory access & Limited memory
regions \\
\textbf{System Calls} & Direct hardware access & Through system calls
only \\
\end{longtable}
}

\begin{itemize}
\tightlist
\item
  \textbf{Kernel Mode}: Operating system runs with full privileges
\item
  \textbf{User Mode}: Applications run with limited privileges
\item
  \textbf{Security}: Mode switching prevents unauthorized access
\end{itemize}

\end{solutionbox}
\begin{mnemonicbox}
``Kernel Controls, User Consumes''

\end{mnemonicbox}
\subsection*{Question 2(c) [7 marks]}\label{q2c}

\textbf{List out types of OS and Explain any two OS}

\begin{solutionbox}

\textbf{Types of Operating Systems:}

{\def\LTcaptype{none} % do not increment counter
\begin{longtable}[]{@{}lll@{}}
\toprule\noalign{}
\textbf{Type} & \textbf{Examples} & \textbf{Characteristics} \\
\midrule\noalign{}
\endhead
\bottomrule\noalign{}
\endlastfoot
\textbf{Batch OS} & Early mainframes & No user interaction \\
\textbf{Time-sharing OS} & UNIX, Linux & Multiple users
simultaneously \\
\textbf{Real-time OS} & Embedded systems & Guaranteed response time \\
\textbf{Distributed OS} & Cloud systems & Multiple connected
computers \\
\textbf{Network OS} & Windows Server & Network resource management \\
\textbf{Mobile OS} & Android, iOS & Smartphone/tablet systems \\
\end{longtable}
}

\textbf{Detailed Explanation:}

\textbf{1. Time-sharing OS (Linux):}

\begin{itemize}
\tightlist
\item
  \textbf{Multi-user}: Multiple users can access simultaneously
\item
  \textbf{Multi-tasking}: Runs multiple processes concurrently
\item
  \textbf{Resource Sharing}: CPU time divided among processes
\item
  \textbf{Examples}: UNIX, Linux, Windows
\end{itemize}

\textbf{2. Real-time OS:}

\begin{itemize}
\tightlist
\item
  \textbf{Deterministic}: Guaranteed response within time limits
\item
  \textbf{Priority-based}: Critical tasks get higher priority
\item
  \textbf{Applications}: Medical devices, industrial control
\item
  \textbf{Types}: Hard real-time and Soft real-time
\end{itemize}

\end{solutionbox}
\begin{mnemonicbox}
``Time Ticks, Real-time Reacts''

\end{mnemonicbox}
\subsection*{Question 2(a OR) [3
marks]}\label{question-2a-or-3-marks}

\textbf{Explain Architecture of Linux Operating System.}

\begin{solutionbox}

\textbf{Linux Architecture Layers:}

\begin{center}
\textbf{Mermaid Diagram (Code)}
\begin{verbatim}
{Shaded}
{Highlighting}[]
graph LR
    A[User Applications] {-{-}{} B[System Call Interface]}
    B {-{-}{} C[Kernel Space]}
    C {-{-}{} D[Process Management]}
    C {-{-}{} E[Memory Management]}
    C {-{-}{} F[File System]}
    C {-{-}{} G[Device Drivers]}
    G {-{-}{} H[Hardware Layer]}
{Highlighting}
{Shaded}
\end{verbatim}
\end{center}

\begin{itemize}
\tightlist
\item
  \textbf{User Space}: Applications and user programs
\item
  \textbf{System Calls}: Interface between user and kernel
\item
  \textbf{Kernel}: Core operating system functions
\end{itemize}

\end{solutionbox}
\begin{mnemonicbox}
``Users Use, Kernel Controls''

\end{mnemonicbox}
\subsection*{Question 2(b OR) [4
marks]}\label{question-2b-or-4-marks}

\textbf{Explain Working of Search Engine.}

\begin{solutionbox}

\textbf{Search Engine Working Process:}

{\def\LTcaptype{none} % do not increment counter
\begin{longtable}[]{@{}
  >{\raggedright\arraybackslash}p{(\linewidth - 4\tabcolsep) * \real{0.2703}}
  >{\raggedright\arraybackslash}p{(\linewidth - 4\tabcolsep) * \real{0.3514}}
  >{\raggedright\arraybackslash}p{(\linewidth - 4\tabcolsep) * \real{0.3784}}@{}}
\toprule\noalign{}
\begin{minipage}[b]{\linewidth}\raggedright
\textbf{Step}
\end{minipage} & \begin{minipage}[b]{\linewidth}\raggedright
\textbf{Process}
\end{minipage} & \begin{minipage}[b]{\linewidth}\raggedright
\textbf{Function}
\end{minipage} \\
\midrule\noalign{}
\endhead
\bottomrule\noalign{}
\endlastfoot
\textbf{Crawling} & Web spiders scan websites & Discovers web pages \\
\textbf{Indexing} & Analyzes and stores content & Creates searchable
database \\
\textbf{Ranking} & Applies algorithms & Determines relevance order \\
\textbf{Retrieval} & Returns results & Displays ranked results \\
\end{longtable}
}

\textbf{Working Steps:}

\begin{itemize}
\tightlist
\item
  \textbf{Web Crawlers}: Automated bots scan internet content
\item
  \textbf{Index Database}: Stores and organizes webpage information
\item
  \textbf{Query Processing}: Analyzes user search terms
\item
  \textbf{Result Ranking}: Uses algorithms to order results by relevance
\end{itemize}

\end{solutionbox}
\begin{mnemonicbox}
``Crawl, Index, Rank, Retrieve''

\end{mnemonicbox}
\subsection*{Question 2(c OR) [7
marks]}\label{question-2c-or-7-marks}

\textbf{Difference between Open Source Software and Proprietary
Software.}

\begin{solutionbox}

{\def\LTcaptype{none} % do not increment counter
\begin{longtable}[]{@{}
  >{\raggedright\arraybackslash}p{(\linewidth - 4\tabcolsep) * \real{0.1818}}
  >{\raggedright\arraybackslash}p{(\linewidth - 4\tabcolsep) * \real{0.4091}}
  >{\raggedright\arraybackslash}p{(\linewidth - 4\tabcolsep) * \real{0.4091}}@{}}
\toprule\noalign{}
\begin{minipage}[b]{\linewidth}\raggedright
\textbf{Aspect}
\end{minipage} & \begin{minipage}[b]{\linewidth}\raggedright
\textbf{Open Source Software}
\end{minipage} & \begin{minipage}[b]{\linewidth}\raggedright
\textbf{Proprietary Software}
\end{minipage} \\
\midrule\noalign{}
\endhead
\bottomrule\noalign{}
\endlastfoot
\textbf{Source Code} & Freely available and modifiable & Closed and
protected \\
\textbf{Cost} & Usually free & Requires license purchase \\
\textbf{Support} & Community-based & Vendor-provided \\
\textbf{Customization} & Fully customizable & Limited customization \\
\textbf{Examples} & Linux, Firefox, LibreOffice & Windows, MS Office,
Photoshop \\
\textbf{Security} & Transparent, community-audited & Security through
obscurity \\
\textbf{Updates} & Community-driven & Vendor-controlled \\
\end{longtable}
}

\textbf{Key Differences:}

\begin{itemize}
\tightlist
\item
  \textbf{Licensing}: Open source allows redistribution and modification
\item
  \textbf{Cost Model}: Open source typically free vs.~proprietary paid
\item
  \textbf{Development}: Community collaboration vs.~company-controlled
\item
  \textbf{Transparency}: Open source code visible vs.~proprietary hidden
\end{itemize}

\textbf{Advantages:}

\begin{itemize}
\tightlist
\item
  \textbf{Open Source}: Cost-effective, customizable, secure
\item
  \textbf{Proprietary}: Professional support, integrated features,
  user-friendly
\end{itemize}

\end{solutionbox}
\begin{mnemonicbox}
``Open Opens, Proprietary Protects''

\end{mnemonicbox}
\subsection*{Question 3(a) [3 marks]}\label{q3a}

\textbf{Give full form of the following: OSI, LLC, FTP}

\begin{solutionbox}

\textbf{Full Forms:}

{\def\LTcaptype{none} % do not increment counter
\begin{longtable}[]{@{}ll@{}}
\toprule\noalign{}
\textbf{Abbreviation} & \textbf{Full Form} \\
\midrule\noalign{}
\endhead
\bottomrule\noalign{}
\endlastfoot
\textbf{OSI} & Open Systems Interconnection \\
\textbf{LLC} & Logical Link Control \\
\textbf{FTP} & File Transfer Protocol \\
\end{longtable}
}

\begin{itemize}
\tightlist
\item
  \textbf{OSI}: Networking reference model with 7 layers
\item
  \textbf{LLC}: Sublayer of Data Link Layer in OSI model
\item
  \textbf{FTP}: Protocol for transferring files over network
\end{itemize}

\end{solutionbox}
\begin{mnemonicbox}
``Open Logic Files''

\end{mnemonicbox}
\subsection*{Question 3(b) [4 marks]}\label{q3b}

\textbf{Give advantages and disadvantages of Twisted Pair Cable.}

\begin{solutionbox}

\textbf{Twisted Pair Cable Analysis:}

{\def\LTcaptype{none} % do not increment counter
\begin{longtable}[]{@{}ll@{}}
\toprule\noalign{}
\textbf{Advantages} & \textbf{Disadvantages} \\
\midrule\noalign{}
\endhead
\bottomrule\noalign{}
\endlastfoot
\textbf{Low Cost} & \textbf{Limited Distance} \\
\textbf{Easy Installation} & \textbf{Electromagnetic Interference} \\
\textbf{Flexible} & \textbf{Lower Bandwidth} \\
\textbf{Widely Available} & \textbf{Security Issues} \\
\end{longtable}
}

\textbf{Advantages:}

\begin{itemize}
\tightlist
\item
  \textbf{Cost-effective}: Cheapest networking cable option
\item
  \textbf{Easy Installation}: Simple to install and maintain
\item
  \textbf{Flexibility}: Can be bent and routed easily
\end{itemize}

\textbf{Disadvantages:}

\begin{itemize}
\tightlist
\item
  \textbf{Distance Limitation}: Maximum 100 meters without repeater
\item
  \textbf{Interference}: Susceptible to electromagnetic interference
\item
  \textbf{Bandwidth}: Lower data transmission rates compared to fiber
\end{itemize}

\end{solutionbox}
\begin{mnemonicbox}
``Twisted is Cheap but Limited''

\end{mnemonicbox}
\subsection*{Question 3(c) [7 marks]}\label{q3c}

\textbf{What is Modulation? Explain Analog Modulation.}

\begin{solutionbox}

\textbf{Modulation Definition:} Process of varying carrier signal
characteristics to transmit information over long distances.

\textbf{Analog Modulation Types:}

{\def\LTcaptype{none} % do not increment counter
\begin{longtable}[]{@{}lll@{}}
\toprule\noalign{}
\textbf{Type} & \textbf{Parameter Varied} & \textbf{Application} \\
\midrule\noalign{}
\endhead
\bottomrule\noalign{}
\endlastfoot
\textbf{AM} & Amplitude & Radio broadcasting \\
\textbf{FM} & Frequency & FM radio, TV sound \\
\textbf{PM} & Phase & Digital communications \\
\end{longtable}
}

\textbf{Amplitude Modulation (AM):}

\begin{center}
\textbf{Mermaid Diagram (Code)}
\begin{verbatim}
{Shaded}
{Highlighting}[]
graph LR
    A[Message Signal] {-{-}{} C[Modulator]}
    B[Carrier Signal] {-{-}{} C}
    C {-{-}{} D[Modulated Signal]}
{Highlighting}
{Shaded}
\end{verbatim}
\end{center}

\textbf{Key Concepts:}

\begin{itemize}
\tightlist
\item
  \textbf{Carrier Wave}: High-frequency signal for transmission
\item
  \textbf{Message Signal}: Information to be transmitted
\item
  \textbf{Modulation Index}: Degree of modulation applied
\end{itemize}

\textbf{Applications:}

\begin{itemize}
\tightlist
\item
  \textbf{AM Radio}: 530-1710 kHz frequency band
\item
  \textbf{FM Radio}: 88-108 MHz frequency band
\item
  \textbf{Television}: Various modulation techniques
\end{itemize}

\textbf{Advantages:}

\begin{itemize}
\tightlist
\item
  \textbf{Long Distance}: Enables long-range communication
\item
  \textbf{Noise Immunity}: FM provides better noise resistance
\end{itemize}

\end{solutionbox}
\begin{mnemonicbox}
``Amplitude Alters, Frequency Fluctuates''

\end{mnemonicbox}
\subsection*{Question 3(a OR) [3
marks]}\label{question-3a-or-3-marks}

\textbf{List out Network Topologies. Write Advantages and Disadvantages
of Bus Topology.}

\begin{solutionbox}

\textbf{Network Topologies:}

\begin{itemize}
\tightlist
\item
  \textbf{Bus Topology}
\item
  \textbf{Star Topology}
\item
  \textbf{Ring Topology}
\item
  \textbf{Mesh Topology}
\item
  \textbf{Hybrid Topology}
\end{itemize}

\textbf{Bus Topology Analysis:}

{\def\LTcaptype{none} % do not increment counter
\begin{longtable}[]{@{}ll@{}}
\toprule\noalign{}
\textbf{Advantages} & \textbf{Disadvantages} \\
\midrule\noalign{}
\endhead
\bottomrule\noalign{}
\endlastfoot
\textbf{Simple Design} & \textbf{Single Point of Failure} \\
\textbf{Cost-effective} & \textbf{Limited Cable Length} \\
\textbf{Easy to Expand} & \textbf{Performance Degradation} \\
\end{longtable}
}

\end{solutionbox}
\begin{mnemonicbox}
``Bus is Simple but Single-failure-prone''

\end{mnemonicbox}
\subsection*{Question 3(b OR) [4
marks]}\label{question-3b-or-4-marks}

\textbf{Differentiate Serial and Parallel Transmission.}

\begin{solutionbox}

{\def\LTcaptype{none} % do not increment counter
\begin{longtable}[]{@{}
  >{\raggedright\arraybackslash}p{(\linewidth - 4\tabcolsep) * \real{0.1875}}
  >{\raggedright\arraybackslash}p{(\linewidth - 4\tabcolsep) * \real{0.3906}}
  >{\raggedright\arraybackslash}p{(\linewidth - 4\tabcolsep) * \real{0.4219}}@{}}
\toprule\noalign{}
\begin{minipage}[b]{\linewidth}\raggedright
\textbf{Aspect}
\end{minipage} & \begin{minipage}[b]{\linewidth}\raggedright
\textbf{Serial Transmission}
\end{minipage} & \begin{minipage}[b]{\linewidth}\raggedright
\textbf{Parallel Transmission}
\end{minipage} \\
\midrule\noalign{}
\endhead
\bottomrule\noalign{}
\endlastfoot
\textbf{Data Path} & Single communication line & Multiple lines
simultaneously \\
\textbf{Speed} & Slower for short distances & Faster for short
distances \\
\textbf{Cost} & Lower cost & Higher cost \\
\textbf{Distance} & Suitable for long distances & Limited to short
distances \\
\end{longtable}
}

\textbf{Characteristics:}

\begin{itemize}
\tightlist
\item
  \textbf{Serial}: Bits transmitted one after another
\item
  \textbf{Parallel}: Multiple bits transmitted simultaneously
\item
  \textbf{Applications}: Serial for networks, Parallel for internal
  buses
\end{itemize}

\end{solutionbox}
\begin{mnemonicbox}
``Serial Single-file, Parallel Processes''

\end{mnemonicbox}
\subsection*{Question 3(c OR) [7
marks]}\label{question-3c-or-7-marks}

\textbf{Explain Transmission Modes.}

\begin{solutionbox}

\textbf{Transmission Modes Classification:}

{\def\LTcaptype{none} % do not increment counter
\begin{longtable}[]{@{}
  >{\raggedright\arraybackslash}p{(\linewidth - 6\tabcolsep) * \real{0.1754}}
  >{\raggedright\arraybackslash}p{(\linewidth - 6\tabcolsep) * \real{0.2632}}
  >{\raggedright\arraybackslash}p{(\linewidth - 6\tabcolsep) * \real{0.2456}}
  >{\raggedright\arraybackslash}p{(\linewidth - 6\tabcolsep) * \real{0.3158}}@{}}
\toprule\noalign{}
\begin{minipage}[b]{\linewidth}\raggedright
\textbf{Mode}
\end{minipage} & \begin{minipage}[b]{\linewidth}\raggedright
\textbf{Direction}
\end{minipage} & \begin{minipage}[b]{\linewidth}\raggedright
\textbf{Examples}
\end{minipage} & \begin{minipage}[b]{\linewidth}\raggedright
\textbf{Applications}
\end{minipage} \\
\midrule\noalign{}
\endhead
\bottomrule\noalign{}
\endlastfoot
\textbf{Simplex} & One-way only & Radio, TV broadcast & Broadcasting \\
\textbf{Half-duplex} & Two-way, not simultaneous & Walkie-talkie &
Turn-based communication \\
\textbf{Full-duplex} & Two-way simultaneous & Telephone & Real-time
communication \\
\end{longtable}
}

\textbf{Detailed Explanation:}

\textbf{1. Simplex Mode:}

\begin{itemize}
\tightlist
\item
  \textbf{Unidirectional}: Data flows in one direction only
\item
  \textbf{Examples}: Television broadcasting, radio transmission
\item
  \textbf{Advantage}: Simple implementation
\item
  \textbf{Disadvantage}: No feedback possible
\end{itemize}

\textbf{2. Half-duplex Mode:}

\begin{itemize}
\tightlist
\item
  \textbf{Bidirectional}: Data can flow both ways, but not
  simultaneously
\item
  \textbf{Examples}: Walkie-talkies, CB radio
\item
  \textbf{Advantage}: Two-way communication with single channel
\item
  \textbf{Disadvantage}: Cannot send and receive simultaneously
\end{itemize}

\textbf{3. Full-duplex Mode:}

\begin{itemize}
\tightlist
\item
  \textbf{Simultaneous Bidirectional}: Data flows both ways at same time
\item
  \textbf{Examples}: Telephone conversations, modern networks
\item
  \textbf{Advantage}: Efficient real-time communication
\item
  \textbf{Disadvantage}: Requires more complex implementation
\end{itemize}

\end{solutionbox}
\begin{mnemonicbox}
``Simplex Single, Half-duplex Halts, Full-duplex
Flows''

\end{mnemonicbox}
\subsection*{Question 4(a) [3 marks]}\label{q4a}

\textbf{Draw Crossover Ethernet Cable.}

\begin{solutionbox}

\textbf{Crossover Cable Wiring Diagram:}

\begin{verbatim}
RJ{-45 Connector A          RJ{-}45 Connector B}
Pin 1: White{-Orange  {-}{-}{-} Pin 3: White{-}Green}
Pin 2: Orange        {{-}{-}{-} Pin 6: Green}
Pin 3: White{-Green   {-}{-}{-} Pin 1: White{-}Orange}
Pin 4: Blue          {{-}{-}{-} Pin 4: Blue}
Pin 5: White{-Blue    {-}{-}{-} Pin 5: White{-}Blue}
Pin 6: Green         {{-}{-}{-} Pin 2: Orange}
Pin 7: White{-Brown   {-}{-}{-} Pin 7: White{-}Brown}
Pin 8: Brown         {{-}{-}{-} Pin 8: Brown}
\end{verbatim}

\textbf{Key Points:}

\begin{itemize}
\tightlist
\item
  \textbf{Purpose}: Direct connection between similar devices
\item
  \textbf{Crossed Pairs}: Transmit and receive pairs are swapped
\item
  \textbf{Usage}: PC to PC, Switch to Switch connections
\end{itemize}

\end{solutionbox}
\begin{mnemonicbox}
``Cross Connects Computers''

\end{mnemonicbox}
\subsection*{Question 4(b) [4 marks]}\label{q4b}

\textbf{Difference between IPv4 and IPv6.}

\begin{solutionbox}

{\def\LTcaptype{none} % do not increment counter
\begin{longtable}[]{@{}
  >{\raggedright\arraybackslash}p{(\linewidth - 4\tabcolsep) * \real{0.3939}}
  >{\raggedright\arraybackslash}p{(\linewidth - 4\tabcolsep) * \real{0.3030}}
  >{\raggedright\arraybackslash}p{(\linewidth - 4\tabcolsep) * \real{0.3030}}@{}}
\toprule\noalign{}
\begin{minipage}[b]{\linewidth}\raggedright
\textbf{Feature}
\end{minipage} & \begin{minipage}[b]{\linewidth}\raggedright
\textbf{IPv4}
\end{minipage} & \begin{minipage}[b]{\linewidth}\raggedright
\textbf{IPv6}
\end{minipage} \\
\midrule\noalign{}
\endhead
\bottomrule\noalign{}
\endlastfoot
\textbf{Address Size} & 32 bits & 128 bits \\
\textbf{Address Format} & Dotted decimal & Hexadecimal colon \\
\textbf{Address Space} & 4.3 billion addresses & 340 undecillion
addresses \\
\textbf{Header Size} & Variable (20-60 bytes) & Fixed (40 bytes) \\
\end{longtable}
}

\textbf{Key Differences:}

\begin{itemize}
\tightlist
\item
  \textbf{IPv4 Example}: 192.168.1.1
\item
  \textbf{IPv6 Example}: 2001:0db8:85a3:0000:0000:8a2e:0370:7334
\item
  \textbf{Security}: IPv6 has built-in IPSec support
\item
  \textbf{NAT}: IPv4 requires NAT, IPv6 eliminates need
\end{itemize}

\end{solutionbox}
\begin{mnemonicbox}
``IPv4 Four-billion, IPv6 Six-teen-times-more''

\end{mnemonicbox}
\subsection*{Question 4(c) [7 marks]}\label{q4c}

\textbf{Draw neat and clean figure of OSI Model and write down the
functionality of Physical Layer and Data Link Layer.}

\begin{solutionbox}

\textbf{OSI Model Diagram:}

\begin{center}
\textbf{Mermaid Diagram (Code)}
\begin{verbatim}
{Shaded}
{Highlighting}[]
graph LR
    A[Application Layer {- 7] {-}{-}{} B[Presentation Layer {-} 6]}
    B {-{-}{} C[Session Layer {-} 5]}
    C {-{-}{} D[Transport Layer {-} 4]}
    D {-{-}{} E[Network Layer {-} 3]}
    E {-{-}{} F[Data Link Layer {-} 2]}
    F {-{-}{} G[Physical Layer {-} 1]}
{Highlighting}
{Shaded}
\end{verbatim}
\end{center}

\textbf{Layer Functions:}

{\def\LTcaptype{none} % do not increment counter
\begin{longtable}[]{@{}
  >{\raggedright\arraybackslash}p{(\linewidth - 4\tabcolsep) * \real{0.2821}}
  >{\raggedright\arraybackslash}p{(\linewidth - 4\tabcolsep) * \real{0.3590}}
  >{\raggedright\arraybackslash}p{(\linewidth - 4\tabcolsep) * \real{0.3590}}@{}}
\toprule\noalign{}
\begin{minipage}[b]{\linewidth}\raggedright
\textbf{Layer}
\end{minipage} & \begin{minipage}[b]{\linewidth}\raggedright
\textbf{Function}
\end{minipage} & \begin{minipage}[b]{\linewidth}\raggedright
\textbf{Examples}
\end{minipage} \\
\midrule\noalign{}
\endhead
\bottomrule\noalign{}
\endlastfoot
\textbf{Physical (Layer 1)} & Bit transmission over medium & Cables,
hubs, repeaters \\
\textbf{Data Link (Layer 2)} & Frame delivery between adjacent nodes &
Switches, MAC addresses \\
\end{longtable}
}

\textbf{Physical Layer Functions:}

\begin{itemize}
\tightlist
\item
  \textbf{Bit Transmission}: Converts data into electrical/optical
  signals
\item
  \textbf{Medium Specification}: Defines cable types and connectors
\item
  \textbf{Signal Encoding}: Determines how bits are represented
\item
  \textbf{Transmission Rate}: Controls data speed
\end{itemize}

\textbf{Data Link Layer Functions:}

\begin{itemize}
\tightlist
\item
  \textbf{Frame Formation}: Organizes bits into frames
\item
  \textbf{Error Detection}: Identifies transmission errors
\item
  \textbf{Flow Control}: Manages data transmission rate
\item
  \textbf{MAC Addressing}: Uses hardware addresses for local delivery
\end{itemize}

\end{solutionbox}
\begin{mnemonicbox}
``Physical Pushes, Data-Link Delivers''

\end{mnemonicbox}
\subsection*{Question 4(a OR) [3
marks]}\label{question-4a-or-3-marks}

\textbf{Explain Time Division Multiplexing.}

\begin{solutionbox}

\textbf{Time Division Multiplexing (TDM):}

\begin{verbatim}
gantt
    title TDM Time Slots
    dateFormat X
    axisFormat \%L
    
    section Channel A
    Data A1 :0, 100
    Data A2 :300, 400
    
    section Channel B
    Data B1 :100, 200
    Data B2 :400, 500
    
    section Channel C
    Data C1 :200, 300
    Data C2 :500, 600
\end{verbatim}

\textbf{TDM Characteristics:}

\begin{itemize}
\tightlist
\item
  \textbf{Time Slots}: Each channel gets dedicated time period
\item
  \textbf{Synchronization}: All channels must be synchronized
\item
  \textbf{Bandwidth Sharing}: Single high-speed link shared among
  multiple channels
\end{itemize}

\end{solutionbox}
\begin{mnemonicbox}
``Time Takes Turns''

\end{mnemonicbox}
\subsection*{Question 4(b OR) [4
marks]}\label{question-4b-or-4-marks}

\textbf{List out types of Networking Device and Explain any one.}

\begin{solutionbox}

\textbf{Networking Devices:}

{\def\LTcaptype{none} % do not increment counter
\begin{longtable}[]{@{}lll@{}}
\toprule\noalign{}
\textbf{Device} & \textbf{Layer} & \textbf{Function} \\
\midrule\noalign{}
\endhead
\bottomrule\noalign{}
\endlastfoot
\textbf{Hub} & Physical & Signal repeater \\
\textbf{Switch} & Data Link & Frame switching \\
\textbf{Router} & Network & Packet routing \\
\textbf{Bridge} & Data Link & Network segmentation \\
\end{longtable}
}

\textbf{Switch Explanation:}

\begin{itemize}
\tightlist
\item
  \textbf{Function}: Forwards frames based on MAC addresses
\item
  \textbf{Learning}: Builds MAC address table dynamically
\item
  \textbf{Collision Domain}: Each port creates separate collision domain
\item
  \textbf{Full-duplex}: Simultaneous send/receive on each port
\end{itemize}

\textbf{Advantages:}

\begin{itemize}
\tightlist
\item
  \textbf{Bandwidth}: Full bandwidth per port
\item
  \textbf{Security}: Frames sent only to intended recipient
\item
  \textbf{Collision}: Eliminates collisions
\end{itemize}

\end{solutionbox}
\begin{mnemonicbox}
``Switch Smartly Sends''

\end{mnemonicbox}
\subsection*{Question 4(c OR) [7
marks]}\label{question-4c-or-7-marks}

\textbf{What is Computer Network? Explain types of Computer Network.}

\begin{solutionbox}

\textbf{Computer Network Definition:} Interconnected collection of
autonomous computers that can communicate and share resources.

\textbf{Types of Computer Networks:}

{\def\LTcaptype{none} % do not increment counter
\begin{longtable}[]{@{}
  >{\raggedright\arraybackslash}p{(\linewidth - 6\tabcolsep) * \real{0.1754}}
  >{\raggedright\arraybackslash}p{(\linewidth - 6\tabcolsep) * \real{0.2456}}
  >{\raggedright\arraybackslash}p{(\linewidth - 6\tabcolsep) * \real{0.2456}}
  >{\raggedright\arraybackslash}p{(\linewidth - 6\tabcolsep) * \real{0.3333}}@{}}
\toprule\noalign{}
\begin{minipage}[b]{\linewidth}\raggedright
\textbf{Type}
\end{minipage} & \begin{minipage}[b]{\linewidth}\raggedright
\textbf{Coverage}
\end{minipage} & \begin{minipage}[b]{\linewidth}\raggedright
\textbf{Examples}
\end{minipage} & \begin{minipage}[b]{\linewidth}\raggedright
\textbf{Characteristics}
\end{minipage} \\
\midrule\noalign{}
\endhead
\bottomrule\noalign{}
\endlastfoot
\textbf{LAN} & Local area (building) & Office network & High speed, low
cost \\
\textbf{MAN} & Metropolitan area (city) & City-wide network & Medium
speed, moderate cost \\
\textbf{WAN} & Wide area (country/world) & Internet & Lower speed, high
cost \\
\end{longtable}
}

\textbf{Detailed Explanation:}

\textbf{1. Local Area Network (LAN):}

\begin{itemize}
\tightlist
\item
  \textbf{Coverage}: Single building or campus
\item
  \textbf{Speed}: High (100 Mbps to 10 Gbps)
\item
  \textbf{Technology}: Ethernet, Wi-Fi
\item
  \textbf{Ownership}: Single organization
\end{itemize}

\textbf{2. Metropolitan Area Network (MAN):}

\begin{itemize}
\tightlist
\item
  \textbf{Coverage}: City or metropolitan area
\item
  \textbf{Speed}: Medium (10-100 Mbps)
\item
  \textbf{Technology}: Fiber optic, microwave
\item
  \textbf{Examples}: Cable TV networks
\end{itemize}

\textbf{3. Wide Area Network (WAN):}

\begin{itemize}
\tightlist
\item
  \textbf{Coverage}: Countries or continents
\item
  \textbf{Speed}: Variable (depends on technology)
\item
  \textbf{Technology}: Satellite, leased lines
\item
  \textbf{Examples}: Internet, corporate networks
\end{itemize}

\textbf{Network Benefits:}

\begin{itemize}
\tightlist
\item
  \textbf{Resource Sharing}: Files, printers, applications
\item
  \textbf{Communication}: Email, messaging, video conferencing
\item
  \textbf{Cost Reduction}: Shared resources reduce costs
\item
  \textbf{Data Backup}: Centralized backup systems
\end{itemize}

\end{solutionbox}
\begin{mnemonicbox}
``Local Loves, Metro Manages, Wide Wanders''

\end{mnemonicbox}
\subsection*{Question 5(a) [3 marks]}\label{q5a}

\textbf{Explain the need for information security.}

\begin{solutionbox}

\textbf{Information Security Needs:}

{\def\LTcaptype{none} % do not increment counter
\begin{longtable}[]{@{}lll@{}}
\toprule\noalign{}
\textbf{Threat} & \textbf{Impact} & \textbf{Protection Need} \\
\midrule\noalign{}
\endhead
\bottomrule\noalign{}
\endlastfoot
\textbf{Data Theft} & Financial loss & Confidentiality \\
\textbf{Unauthorized Access} & Privacy breach & Access control \\
\textbf{System Attacks} & Service disruption & Availability \\
\end{longtable}
}

\textbf{Key Requirements:}

\begin{itemize}
\tightlist
\item
  \textbf{Confidentiality}: Protecting sensitive information from
  unauthorized access
\item
  \textbf{Data Protection}: Preventing loss or corruption of valuable
  data
\item
  \textbf{Business Continuity}: Ensuring systems remain operational
\end{itemize}

\end{solutionbox}
\begin{mnemonicbox}
``Security Secures Sensitive Systems''

\end{mnemonicbox}
\subsection*{Question 5(b) [4 marks]}\label{q5b}

\textbf{Write advantages and disadvantages of Fiber Optic Cable.}

\begin{solutionbox}

{\def\LTcaptype{none} % do not increment counter
\begin{longtable}[]{@{}ll@{}}
\toprule\noalign{}
\textbf{Advantages} & \textbf{Disadvantages} \\
\midrule\noalign{}
\endhead
\bottomrule\noalign{}
\endlastfoot
\textbf{High Bandwidth} & \textbf{High Cost} \\
\textbf{Immunity to EMI} & \textbf{Difficult Installation} \\
\textbf{Long Distance} & \textbf{Fragile Nature} \\
\textbf{Secure Transmission} & \textbf{Specialized Equipment} \\
\end{longtable}
}

\textbf{Advantages:}

\begin{itemize}
\tightlist
\item
  \textbf{Speed}: Highest data transmission rates
\item
  \textbf{Distance}: Can span long distances without signal degradation
\item
  \textbf{Security}: Difficult to tap, providing secure communication
\end{itemize}

\textbf{Disadvantages:}

\begin{itemize}
\tightlist
\item
  \textbf{Cost}: Expensive cable and equipment
\item
  \textbf{Installation}: Requires skilled technicians
\item
  \textbf{Maintenance}: Difficult to repair and splice
\end{itemize}

\end{solutionbox}
\begin{mnemonicbox}
``Fiber is Fast but Fragile''

\end{mnemonicbox}
\subsection*{Question 5(c) [7 marks]}\label{q5c}

\textbf{List out types of Attack. And Explain any two web based attack.}

\begin{solutionbox}

\textbf{Types of Attacks:}

{\def\LTcaptype{none} % do not increment counter
\begin{longtable}[]{@{}lll@{}}
\toprule\noalign{}
\textbf{Category} & \textbf{Attack Types} & \textbf{Target} \\
\midrule\noalign{}
\endhead
\bottomrule\noalign{}
\endlastfoot
\textbf{Web-based} & SQL Injection, XSS, CSRF & Web applications \\
\textbf{Network} & DoS, DDoS, Man-in-Middle & Network infrastructure \\
\textbf{Malware} & Virus, Trojan, Ransomware & Systems and data \\
\textbf{Social} & Phishing, Social Engineering & Human users \\
\end{longtable}
}

\textbf{Web-based Attacks Explained:}

\textbf{1. SQL Injection:}

\begin{itemize}
\tightlist
\item
  \textbf{Method}: Inserting malicious SQL code into web application
  inputs
\item
  \textbf{Impact}: Unauthorized database access, data theft
\item
  \textbf{Example}: Entering
  \texttt{\textquotesingle{};\ DROP\ TABLE\ users;-\/-} in login form
\item
  \textbf{Prevention}: Input validation, parameterized queries
\item
  \textbf{Severity}: Can compromise entire database
\end{itemize}

\textbf{2. Cross-Site Scripting (XSS):}

\begin{itemize}
\tightlist
\item
  \textbf{Method}: Injecting malicious scripts into web pages
\item
  \textbf{Impact}: Session hijacking, cookie theft, page defacement
\item
  \textbf{Types}: Stored XSS, Reflected XSS, DOM-based XSS
\item
  \textbf{Prevention}: Input sanitization, output encoding
\item
  \textbf{Target}: Affects users visiting compromised websites
\end{itemize}

\textbf{Attack Characteristics:}

\begin{itemize}
\tightlist
\item
  \textbf{SQL Injection}: Targets database through web application
\item
  \textbf{XSS}: Targets users through compromised web pages
\item
  \textbf{Common Factor}: Both exploit insufficient input validation
\end{itemize}

\textbf{Prevention Measures:}

\begin{itemize}
\tightlist
\item
  \textbf{Input Validation}: Check all user inputs
\item
  \textbf{Regular Updates}: Keep software and systems updated
\item
  \textbf{Security Training}: Educate users about attack methods
\end{itemize}

\end{solutionbox}
\begin{mnemonicbox}
``SQL Steals, XSS eXploits Scripts''

\end{mnemonicbox}
\subsection*{Question 5(a OR) [3
marks]}\label{question-5a-or-3-marks}

\textbf{Explain Confidentiality, Integrity and Availability.}

\begin{solutionbox}

\textbf{CIA Triad Components:}

{\def\LTcaptype{none} % do not increment counter
\begin{longtable}[]{@{}
  >{\raggedright\arraybackslash}p{(\linewidth - 4\tabcolsep) * \real{0.3333}}
  >{\raggedright\arraybackslash}p{(\linewidth - 4\tabcolsep) * \real{0.3556}}
  >{\raggedright\arraybackslash}p{(\linewidth - 4\tabcolsep) * \real{0.3111}}@{}}
\toprule\noalign{}
\begin{minipage}[b]{\linewidth}\raggedright
\textbf{Component}
\end{minipage} & \begin{minipage}[b]{\linewidth}\raggedright
\textbf{Definition}
\end{minipage} & \begin{minipage}[b]{\linewidth}\raggedright
\textbf{Examples}
\end{minipage} \\
\midrule\noalign{}
\endhead
\bottomrule\noalign{}
\endlastfoot
\textbf{Confidentiality} & Information access only by authorized users &
Encryption, access controls \\
\textbf{Integrity} & Data accuracy and completeness & Checksums, digital
signatures \\
\textbf{Availability} & Systems accessible when needed & Redundancy,
backup systems \\
\end{longtable}
}

\textbf{Key Concepts:}

\begin{itemize}
\tightlist
\item
  \textbf{Confidentiality}: Keeps information secret from unauthorized
  users
\item
  \textbf{Integrity}: Ensures data hasn't been modified without
  authorization
\item
  \textbf{Availability}: Guarantees systems are operational when
  required
\end{itemize}

\end{solutionbox}
\begin{mnemonicbox}
``CIA Completely Protects Information''

\end{mnemonicbox}
\subsection*{Question 5(b OR) [4
marks]}\label{question-5b-or-4-marks}

\textbf{Find class of following IP addresses.}

\begin{solutionbox}

\textbf{IP Address Class Identification:}

{\def\LTcaptype{none} % do not increment counter
\begin{longtable}[]{@{}llll@{}}
\toprule\noalign{}
\textbf{IP Address} & \textbf{First Octet} & \textbf{Class} &
\textbf{Range} \\
\midrule\noalign{}
\endhead
\bottomrule\noalign{}
\endlastfoot
\textbf{192.12.44.12} & 192 & Class C & 192-223 \\
\textbf{123.77.42.213} & 123 & Class A & 1-126 \\
\textbf{190.65.22.15} & 190 & Class B & 128-191 \\
\textbf{10.0.0.11} & 10 & Class A (Private) & 1-126 \\
\end{longtable}
}

\textbf{Class Characteristics:}

\begin{itemize}
\tightlist
\item
  \textbf{Class A}: 1-126 (first bit 0), supports large networks
\item
  \textbf{Class B}: 128-191 (first two bits 10), medium networks\\
\item
  \textbf{Class C}: 192-223 (first three bits 110), small networks
\item
  \textbf{Private IPs}: 10.x.x.x, 172.16-31.x.x, 192.168.x.x
\end{itemize}

\end{solutionbox}
\begin{mnemonicbox}
``A is Awesome, B is Better, C is Compact''

\end{mnemonicbox}
\subsection*{Question 5(c OR) [7
marks]}\label{question-5c-or-7-marks}

\textbf{Explain Cryptography.}

\begin{solutionbox}

\textbf{Cryptography Definition:} Science of securing communication
through encoding information so only authorized parties can access it.

\textbf{Cryptography Types:}

{\def\LTcaptype{none} % do not increment counter
\begin{longtable}[]{@{}
  >{\raggedright\arraybackslash}p{(\linewidth - 6\tabcolsep) * \real{0.1754}}
  >{\raggedright\arraybackslash}p{(\linewidth - 6\tabcolsep) * \real{0.2632}}
  >{\raggedright\arraybackslash}p{(\linewidth - 6\tabcolsep) * \real{0.2456}}
  >{\raggedright\arraybackslash}p{(\linewidth - 6\tabcolsep) * \real{0.3158}}@{}}
\toprule\noalign{}
\begin{minipage}[b]{\linewidth}\raggedright
\textbf{Type}
\end{minipage} & \begin{minipage}[b]{\linewidth}\raggedright
\textbf{Key Usage}
\end{minipage} & \begin{minipage}[b]{\linewidth}\raggedright
\textbf{Examples}
\end{minipage} & \begin{minipage}[b]{\linewidth}\raggedright
\textbf{Applications}
\end{minipage} \\
\midrule\noalign{}
\endhead
\bottomrule\noalign{}
\endlastfoot
\textbf{Symmetric} & Single shared key & DES, AES & Fast bulk
encryption \\
\textbf{Asymmetric} & Public-private key pair & RSA, ECC & Digital
signatures, key exchange \\
\textbf{Hash Functions} & One-way transformation & MD5, SHA & Data
integrity, passwords \\
\end{longtable}
}

\textbf{Key Concepts:}

\textbf{1. Symmetric Cryptography:}

\begin{itemize}
\tightlist
\item
  \textbf{Single Key}: Same key for encryption and decryption
\item
  \textbf{Speed}: Fast processing for large amounts of data
\item
  \textbf{Challenge}: Secure key distribution
\item
  \textbf{Examples}: AES-256, 3DES
\end{itemize}

\textbf{2. Asymmetric Cryptography:}

\begin{itemize}
\tightlist
\item
  \textbf{Key Pairs}: Public key (shareable) and private key (secret)
\item
  \textbf{Digital Signatures}: Proves authenticity and non-repudiation
\item
  \textbf{Key Exchange}: Secure method to share symmetric keys
\item
  \textbf{Examples}: RSA, Elliptic Curve Cryptography
\end{itemize}

\textbf{3. Hash Functions:}

\begin{itemize}
\tightlist
\item
  \textbf{One-way}: Easy to compute hash, difficult to reverse
\item
  \textbf{Fixed Output}: Always produces same length output
\item
  \textbf{Collision Resistance}: Different inputs should produce
  different hashes
\item
  \textbf{Applications}: Password storage, digital forensics
\end{itemize}

\textbf{Cryptographic Process:}

\begin{center}
\textbf{Mermaid Diagram (Code)}
\begin{verbatim}
{Shaded}
{Highlighting}[]
graph LR
    A[Plaintext] {-{-}{} B[Encryption Algorithm]}
    C[Key] {-{-}{} B}
    B {-{-}{} D[Ciphertext]}
    D {-{-}{} E[Decryption Algorithm]}
    C {-{-}{} E}
    E {-{-}{} F[Plaintext]}
{Highlighting}
{Shaded}
\end{verbatim}
\end{center}

\textbf{Applications:}

\begin{itemize}
\tightlist
\item
  \textbf{Secure Communication}: HTTPS, VPN, email encryption
\item
  \textbf{Data Protection}: File encryption, database security
\item
  \textbf{Authentication}: Digital certificates, password hashing
\item
  \textbf{Financial Systems}: Online banking, cryptocurrency
\end{itemize}

\textbf{Modern Challenges:}

\begin{itemize}
\tightlist
\item
  \textbf{Quantum Computing}: Threat to current encryption methods
\item
  \textbf{Key Management}: Secure storage and distribution of keys
\item
  \textbf{Performance}: Balancing security with system performance
\end{itemize}

\end{solutionbox}
\begin{mnemonicbox}
``Cryptography Creates Coded Communications''

\end{mnemonicbox}

\end{document}
