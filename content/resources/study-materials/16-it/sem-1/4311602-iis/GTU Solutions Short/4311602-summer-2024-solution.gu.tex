\documentclass{article}

% content/resources/templates/preamble.tex
\usepackage[margin=0.6in]{geometry}
\author{Milav Dabgar}
\usepackage{amsmath,amssymb,amsthm}
\usepackage{booktabs}
\usepackage{multirow}
\usepackage{xcolor}
\usepackage{tcolorbox}
\tcbuselibrary{breakable,skins}
\usepackage[colorlinks=true,linkcolor=blue]{hyperref}
\usepackage{titlesec}
\usepackage{enumitem}
\usepackage{tikz}
\usepackage{pgfplots}
\usepackage{circuitikz}
\usepackage[version=4]{mhchem}
\usepackage{longtable}
\usepackage{array}
\usepackage{float}
\usepackage{caption}
\usepackage{listings}

\lstset{
  basicstyle=\small\ttfamily,
  breaklines=true,
  breakatwhitespace=false,
  postbreak=\mbox{\textcolor{red}{$\hookrightarrow$}\space},
  float=false,
  numbers=left,
  numberstyle=\tiny\color{gray},
  numbersep=10pt,
  xleftmargin=2em,
  keywordstyle=\color{blue},
  commentstyle=\color{green!60!black},
  stringstyle=\color{purple},
  backgroundcolor=\color{gray!5},
  showstringspaces=false,
  tabsize=2,
  captionpos=b,
  keepspaces=true,
  columns=flexible
}

\pgfplotsset{compat=1.18}
\usetikzlibrary{shapes,arrows,positioning,calc,patterns,decorations.pathmorphing,decorations.markings,arrows.meta}

% Color scheme
\definecolor{headcolor}{RGB}{0,102,204}
\definecolor{keycolor}{RGB}{220,20,60}
\definecolor{solutioncolor}{RGB}{34,139,34}
\definecolor{mnemoniccolor}{RGB}{148,0,211}
\definecolor{codecolor}{RGB}{0,0,100}

% Spacing
\setlength{\parskip}{3pt}
\setlist[itemize]{nosep}
\setlist[enumerate]{nosep}

% Title formatting
\titleformat{\section}{\Large\bfseries\color{headcolor}}{\thesection}{1em}{}
\titleformat{\subsection}{\large\bfseries\color{headcolor}}{\thesubsection}{1em}{}

% Pandoc tightlist compatibility
\providecommand{\tightlist}{%
  \setlength{\itemsep}{0pt}\setlength{\parskip}{0pt}}

% Pandoc longtable compatibility
\newcounter{none}
\def\thenone{}


% content/resources/templates/gujarati-boxes.tex
\usepackage{fontspec}
\usepackage{polyglossia}

% Set Gujarati as main language (document is primarily in Gujarati)
% Note: gloss-gujarati.ldf doesn't exist in polyglossia, but it will use hyphenation patterns
\setdefaultlanguage{gujarati}
\setotherlanguage{english}

% Configure Gujarati font properly
% Use Language=Default to prevent polyglossia from trying to add language-specific features
% that don't exist for Gujarati, which causes "empty feature" warnings
\newfontfamily\gujaratifont[Script=Gujarati,AutoFakeBold=2.5,AutoFakeSlant=0.3]{Noto Sans Gujarati}
\setmainfont[Script=Gujarati,AutoFakeBold=2.5,AutoFakeSlant=0.3]{Noto Sans Gujarati}
% Use Noto Sans Gujarati for monospace to support Gujarati in text
\setmonofont[Scale=0.9]{Noto Sans Gujarati}

% Configure English to use the same font
\newfontfamily\englishfont[Script=Gujarati,AutoFakeBold=2.5,AutoFakeSlant=0.3]{Noto Sans Gujarati}

% Translations for polyglossia
\gappto\captionsgujarati{
  \renewcommand{\tablename}{કોષ્ટક}
  \renewcommand{\figurename}{આકૃતિ}
}

% Helper for TikZ nodes to ensure Gujarati font
\newcommand{\gu}[1]{{\gujaratifont #1}}

% Custom environments
\newtcolorbox{solutionbox}{
    breakable,
    enhanced,
    colback=solutioncolor!5!white,
    colframe=solutioncolor!75!black,
    fonttitle=\bfseries,
    title=જવાબ
}

\newtcolorbox{solutionboxnobreak}{
 colback=solutioncolor!5!white,
 colframe=solutioncolor!75!black,
 fonttitle=\bfseries,
 title=જવાબ
}

\newtcolorbox{keyformula}{
 breakable,
 enhanced,
 colback=keycolor!5!white,
 colframe=keycolor!75!black,
 fonttitle=\bfseries,
 title=રાસાયણિક સમીકરણ/સૂત્ર
}

\newtcolorbox{mnemonicbox}{
 breakable,
 enhanced,
 colback=mnemoniccolor!5!white,
 colframe=mnemoniccolor!75!black,
 fonttitle=\bfseries,
 title=મેમરી ટ્રીક
}


% Custom commands for GTU solutions
% This file defines semantic commands for consistent formatting

% Question command with automatic formatting
\newcommand{\question}[2]{%
  \section*{Question #1}%
  \textbf{#2}%
}

% OR question variant
\newcommand{\questionor}[2]{%
  \section*{Question #1 OR}%
  \textbf{#2}%
}

% Proper table environment with caption
\newenvironment{answertable}[1]{%
  \begin{table}[htbp]
  \centering
  \caption{#1}
}{%
  \end{table}
}

% Proper figure environment for diagrams
\newenvironment{answerdiagram}[1]{%
  \begin{figure}[htbp]
  \centering
  \caption{#1}
}{%
  \end{figure}
}

% Semantic markup for key terms
\newcommand{\keyword}[1]{\textbf{#1}}
\newcommand{\code}[1]{\texttt{#1}}
\newcommand{\classname}[1]{\texttt{#1}}
\newcommand{\methodname}[1]{\texttt{#1}}

% Proper quotation marks
\newcommand{\mnemonic}[1]{``#1''}


% Gujarati Font Setup (Ensuring consistency)
\usepackage{fontspec}
\setmainfont{Noto Serif Gujarati}
\newfontfamily\gujaratifont{Noto Serif Gujarati}

\title{આઈટી સિસ્ટમ્સનો પરિચય (4311602) - ઉનાળો 2024 સોલ્યુશન}
\date{June 14, 2024}

\begin{document}
\maketitle

\questionmarks{1(અ)}{3}{નીચેની મુદ્દાઓ વ્યાખ્યાયિત કરો: 1. ડેટા 2. માહિતી 3. જ્ઞાન}

\begin{solutionbox}
\textbf{જવાબ:}

\begin{answertable}{ડેટા, માહિતી અને જ્ઞાનની વ્યાખ્યાઓ}
\begin{tabulary}{\linewidth}{|L|L|}
\hline
\textbf{શબ્દ} & \textbf{વ્યાખ્યા} \\
\hline
\keyword{ડેટા} & કાચા તથ્યો અને આંકડાઓ જેમાં અર્થ અથવા સંદર્ભ નથી \\
\hline
\keyword{માહિતી} & પ્રોસેસ કરેલો ડેટા જે અર્થપૂર્ણ અને ઉપયોગી હોય \\
\hline
\keyword{જ્ઞાન} & અનુભવ અને સમજ સાથે જોડાયેલી માહિતી \\
\hline
\end{tabulary}
\end{answertable}

\begin{itemize}
    \item \textbf{ડેટા}: અર્થઘટન વિના મૂળભૂત બિલ્ડિંગ બ્લોક્સ
    \item \textbf{માહિતી}: અર્થપૂર્ણ સંદર્ભ પ્રદાન કરવા માટે પ્રોસેસ કરેલો ડેટા
    \item \textbf{જ્ઞાન}: માનવીય અંતર્દૃષ્ટિ અને વિવેક સાથે વધારેલી માહિતી
\end{itemize}

\begin{mnemonicbox}
\mnemonic{DIK - Data Is Knowledge's foundation}
\end{mnemonicbox}
\end{solutionbox}

\questionmarks{1(બ)}{4}{સંક્ષિપ્તમાં પ્રાથમિક મેમરી સમજાવો.}

\begin{solutionbox}
\textbf{જવાબ:}

\begin{answertable}{પ્રાથમિક મેમરીની લાક્ષણિકતાઓ}
\begin{tabulary}{\linewidth}{|L|L|}
\hline
\textbf{પાસાં} & \textbf{વિવરણ} \\
\hline
\textbf{વ્યાખ્યા} & મુખ્ય મેમરી જે સીપીયુ સાથે સીધું કમ્યુનિકેશન કરે \\
\hline
\textbf{એક્સેસ સ્પીડ} & ખૂબ જ ઝડપી એક્સેસ ટાઇમ \\
\hline
\textbf{વોલેટિલિટી} & વોલેટાઇલ (પાવર બંધ થતાં ડેટા ગુમ થાય) \\
\hline
\textbf{ઉદાહરણો} & RAM, કેશ મેમરી \\
\hline
\end{tabulary}
\end{answertable}

\begin{itemize}
    \item \textbf{RAM (રેન્ડમ એક્સેસ મેમરી)}: વર્તમાન પ્રોગ્રામ્સ માટેની મુખ્ય કાર્યકારી મેમરી
    \item \textbf{કેશ મેમરી}: સીપીયુ અને RAM વચ્ચે અતિ-ઝડપી મેમરી
    \item \textbf{વોલેટાઇલ પ્રકૃતિ}: કમ્પ્યુટર બંધ થતાં ડેટા અદૃશ્ય થઈ જાય
    \item \textbf{સીધું સીપીયુ એક્સેસ}: સીપીયુ સીધું ડેટા વાંચી/લખી શકે
\end{itemize}

\begin{mnemonicbox}
\mnemonic{Primary is Fast but Forgetful}
\end{mnemonicbox}
\end{solutionbox}

\questionmarks{1(ક)}{7}{ઉદાહરણ સાથે રિયલ ટાઇમ OSના પ્રકારો સમજાવો.}

\begin{solutionbox}
\textbf{જવાબ:}

\begin{answertable}{રિયલ-ટાઇમ ઓપરેટિંગ સિસ્ટમના પ્રકારો}
\begin{tabulary}{\linewidth}{|L|L|L|L|}
\hline
\textbf{પ્રકાર} & \textbf{રિસ્પોન્સ ટાઇમ} & \textbf{ઉદાહરણો} & \textbf{ઉપયોગ} \\
\hline
\keyword{હાર્ડ રિયલ-ટાઇમ} & ગેરંટીડ ડેડલાઇન & QNX, VxWorks & મેડિકલ ડિવાઇસ, એરક્રાફ્ટ \\
\hline
\keyword{સોફ્ટ રિયલ-ટાઇમ} & શ્રેષ્ઠ પ્રયાસ ટાઇમિંગ & Windows RT, Linux RT & મલ્ટીમીડિયા, ગેમિંગ \\
\hline
\keyword{ફર્મ રિયલ-ટાઇમ} & ક્યારેક ડેડલાઇન મિસ & Embedded Linux & ઇન્ડસ્ટ્રિયલ કંટ્રોલ \\
\hline
\end{tabulary}
\end{answertable}

\begin{answerdiagram}{રિયલ-ટાઇમ OS પ્રકારો}
\begin{tikzpicture}[gtu tree]
    \node [gtu root] {રિયલ-ટાઇમ OS}
        child { node [gtu child] {હાર્ડ રિયલ-ટાઇમ} 
            child { node [gtu block, fill=orange!10] {ક્રિટિકલ સિસ્ટમ્સ} }
        }
        child { node [gtu child] {સોફ્ટ રિયલ-ટાઇમ} 
            child { node [gtu block, fill=orange!10] {મલ્ટીમીડિયા એપ્સ} }
        }
        child { node [gtu child] {ફર્મ રિયલ-ટાઇમ} 
            child { node [gtu block, fill=orange!10] {ઇન્ડસ્ટ્રિયલ કંટ્રોલ} }
        };
\end{tikzpicture}
\end{answerdiagram}

\begin{itemize}
    \item \textbf{હાર્ડ રિયલ-ટાઇમ}: ડેડલાઇન ચૂકવાથી સિસ્ટમ ફેઇલ થાય
    \item \textbf{સોફ્ટ રિયલ-ટાઇમ}: વિલંબિત રિસ્પોન્સ પરફોર્મન્સ ઘટાડે પરંતુ સિસ્ટમ ચાલુ રહે
    \item \textbf{નિર્ધારિત રિસ્પોન્સ}: અનુમાનિત ટાઇમિંગ વર્તણૂક આવશ્યક
\end{itemize}

\begin{mnemonicbox}
\mnemonic{HSF - Hard, Soft, Firm timing requirements}
\end{mnemonicbox}
\end{solutionbox}

\questionmarks{1(ક OR)}{7}{Linux આર્કિટેક્ચરનું વર્ણન કરો અને Linux ની કામગીરીના મોડની ચર્ચા કરો.}

\begin{solutionbox}
\textbf{જવાબ:}

\begin{answerdiagram}{Linux આર્કિટેક્ચર}
\begin{tikzpicture}[node distance=1.5cm, auto]
    % Concentric circles approach for architecture
    \fill[fill=blue!5] (0,0) circle (3.5cm);
    \fill[fill=green!10] (0,0) circle (2.5cm);
    \fill[fill=red!10] (0,0) circle (1.5cm);
    \fill[fill=gray!20] (0,0) circle (0.8cm);

    \node at (0,0) {\textbf{હાર્ડવેર}};
    \node at (0, 1.1) {Linux કર્નલ};
    \node at (0, 2.0) {સિસ્ટમ લાઇબ્રેરીઓ};
    \node at (0, 3.0) {યુઝર એપ્લિકેશન્સ};
    
    % Legend or annotations
    \node[anchor=west] at (4, 3) {યુઝર સ્પેસ};
    \node[anchor=west] at (4, 1.1) {કર્નલ સ્પેસ};
    \draw[->] (3.8, 3) -- (2.5, 2.5);
    \draw[->] (3.8, 1.1) -- (1.5, 0.5);
\end{tikzpicture}
\end{answerdiagram}

\begin{answertable}{Linux ઓપરેશન મોડ્સ}
\begin{tabulary}{\linewidth}{|L|L|L|L|}
\hline
\textbf{મોડ} & \textbf{વિવરણ} & \textbf{એક્સેસ લેવલ} & \textbf{ઉદાહરણો} \\
\hline
\keyword{યુઝર મોડ} & પ્રતિબંધિત એક્સેસ & મર્યાદિત અધિકારો & એપ્લિકેશન્સ, યુઝર પ્રોગ્રામ્સ \\
\hline
\keyword{કર્નલ મોડ} & સંપૂર્ણ સિસ્ટમ એક્સેસ & સંપૂર્ણ નિયંત્રણ & ડિવાઇસ ડ્રાઇવર્સ, OS ફંક્શન્સ \\
\hline
\end{tabulary}
\end{answertable}

\begin{itemize}
    \item \textbf{લેયર્ડ આર્કિટેક્ચર}: યુઝર અને સિસ્ટમ કમ્પોનન્ટ્સ વચ્ચે સ્પષ્ટ અલગીકરણ
    \item \textbf{મોડ સ્વિચિંગ}: સીપીયુ યુઝર અને કર્નલ મોડ્સ વચ્ચે સ્વિચ કરે
    \item \textbf{સિસ્ટમ કોલ્સ}: યુઝર પ્રોગ્રામ્સ માટે કર્નલ સેવાઓ એક્સેસ કરવાનું ઇન્ટરફેસ
    \item \textbf{સિક્યોરિટી}: યુઝર મોડ સીધું હાર્ડવેર એક્સેસ અટકાવે
\end{itemize}

\begin{mnemonicbox}
\mnemonic{LUSK - Linux Uses Safe Kernel protection}
\end{mnemonicbox}
\end{solutionbox}

\questionmarks{2(અ)}{3}{XOR ગેટ તેના સત્ય કોષ્ટક સાથે વર્ણવો.}

\begin{solutionbox}
\textbf{જવાબ:}

\begin{answerdiagram}{XOR ગેટ સિમ્બોલ}
\begin{tikzpicture}
    \node[xor gate US, draw, logic gate inputs=nn] (xor) {};
    \draw (xor.input 1) -- ++(-0.5,0) node[left] {A};
    \draw (xor.input 2) -- ++(-0.5,0) node[left] {B};
    \draw (xor.output) -- ++(0.5,0) node[right] {આઉટપુટ ($A \oplus B$)};
\end{tikzpicture}
\end{answerdiagram}

\begin{answertable}{સત્ય કોષ્ટક}
\begin{tabulary}{\linewidth}{|C|C|C|}
\hline
\textbf{A} & \textbf{B} & \textbf{આઉટપુટ (A $\oplus$ B)} \\
\hline
0 & 0 & 0 \\
\hline
0 & 1 & 1 \\
\hline
1 & 0 & 1 \\
\hline
1 & 1 & 0 \\
\hline
\end{tabulary}
\end{answertable}

\begin{itemize}
    \item \textbf{એક્સક્લુસિવ OR}: જ્યારે ઇનપુટ્સ અલગ હોય ત્યારે આઉટપુટ 1
    \item \textbf{લોજિક ફંક્શન}: $A \oplus B = A'B + AB'$
    \item \textbf{એપ્લિકેશન્સ}: હાફ એડર, પેરિટી ચેકર, એન્ક્રિપ્શન
\end{itemize}

\begin{mnemonicbox}
\mnemonic{XOR - eXclusive OR gives 1 for different inputs}
\end{mnemonicbox}
\end{solutionbox}

\questionmarks{2(બ)}{4}{નીચેના ઉકેલો. i) (4C6)16 = (\_)2 = (\_)10 ii) (186)10 = (\_)8 = (\_)2}

\begin{solutionbox}
\textbf{જવાબ:}

\begin{answertable}{રૂપાંતરણ કોષ્ટક}
\begin{tabulary}{\linewidth}{|L|L|L|}
\hline
\textbf{રૂપાંતરણ} & \textbf{પગલું} & \textbf{પરિણામ} \\
\hline
\textbf{(4C6)$_{16}$} & હેક્સ ટુ બાઇનરી & \textbf{10011000110$_2$} \\
\hline
 & બાઇનરી ટુ ડેસિમલ & \textbf{1222$_{10}$} \\
\hline
\textbf{(186)$_{10}$} & ડેસિમલ ટુ ઓક્ટલ & \textbf{272$_8$} \\
\hline
 & ડેસિમલ ટુ બાઇનરી & \textbf{10111010$_2$} \\
\hline
\end{tabulary}
\end{answertable}

\textbf{વિગતવાર સોલ્યુશન્સ:}

i) \textbf{(4C6)$_{16}$ = (10011000110)$_2$ = (1222)$_{10}$}
\begin{itemize}
    \item $4 = 0100, C = 1100, 6 = 0110$
    \item સંયુક્ત: $010011000110 = 10011000110_2$
    \item ડેસિમલ: $1 \times 2^{10} + 0 \times 2^9 + 0 \times 2^8 + 1 \times 2^7 + 1 \times 2^6 + 0 \times 2^5 + 0 \times 2^4 + 0 \times 2^3 + 1 \times 2^2 + 1 \times 2^1 + 0 \times 2^0 = 1222_{10}$
\end{itemize}

ii) \textbf{(186)$_{10}$ = (272)$_8$ = (10111010)$_2$}
\begin{itemize}
    \item ઓક્ટલ: $186 \div 8 = 23 \text{ બાકી } 2, 23 \div 8 = 2 \text{ બાકી } 7, 2 \div 8 = 0 \text{ બાકી } 2 \rightarrow 272_8$
    \item બાઇનરી: $186 = 128 + 32 + 16 + 8 + 2 = 10111010_2$
\end{itemize}

\begin{mnemonicbox}
\mnemonic{HDB - Hex, Decimal, Binary conversions}
\end{mnemonicbox}
\end{solutionbox}

\questionmarks{2(ક)}{7}{નીચેના OS ને સમજાવો: i) નેટવર્ક ઓપરેટિંગ સિસ્ટમ ii) મોબાઇલ ઓપરેટિંગ સિસ્ટમ}

\begin{solutionbox}
\textbf{જવાબ:}

\begin{answertable}{ઓપરેટિંગ સિસ્ટમ સરખામણી}
\begin{tabulary}{\linewidth}{|L|L|L|}
\hline
\textbf{લાક્ષણિકતા} & \textbf{નેટવર્ક OS} & \textbf{મોબાઇલ OS} \\
\hline
\textbf{હેતુ} & નેટવર્ક રિસોર્સ મેનેજ કરવું & મોબાઇલ ડિવાઇસ મેનેજમેન્ટ \\
\hline
\textbf{ઉદાહરણો} & Windows Server, Linux Server & Android, iOS, Windows Mobile \\
\hline
\textbf{મુખ્ય ફીચર્સ} & ફાઇલ શેરિંગ, પ્રિન્ટર શેરિંગ & ટચ ઇન્ટરફેસ, બેટરી મેનેજમેન્ટ \\
\hline
\textbf{યુઝર્સ} & મલ્ટિપલ સાથોસાથ યુઝર્સ & સામાન્ય રીતે સિંગલ યુઝર \\
\hline
\end{tabulary}
\end{answertable}

\begin{answerdiagram}{OS પ્રકારો}
\begin{tikzpicture}[gtu tree]
    \node [gtu root] {ઓપરેટિંગ સિસ્ટમ્સ}
        child { node [gtu child] {નેટવર્ક OS} 
            child { node [gtu block, fill=blue!5] {ફાઇલ સર્વર} }
            child { node [gtu block, fill=blue!5] {પ્રિન્ટ સર્વર} }
            child { node [gtu block, fill=blue!5] {એપ્લિકેશન સર્વર} }
        }
        child { node [gtu child] {મોબાઇલ OS} 
            child { node [gtu block, fill=green!5] {ટચ ઇન્ટરફેસ} }
            child { node [gtu block, fill=green!5] {એપ સ્ટોર} }
            child { node [gtu block, fill=green!5] {બેટરી મેનેજમેન્ટ} }
        };
\end{tikzpicture}
\end{answerdiagram}

\textbf{i) નેટવર્ક ઓપરેટિંગ સિસ્ટમ:}
\begin{itemize}
    \item \textbf{મલ્ટિ-યુઝર સપોર્ટ}: મલ્ટિપલ સાથોસાથ યુઝર્સ હેન્ડલ કરે
    \item \textbf{રિસોર્સ શેરિંગ}: ફાઇલો, પ્રિન્ટર્સ, એપ્લિકેશન્સ નેટવર્કમાં શેર કરાય
    \item \textbf{સિક્યોરિટી મેનેજમેન્ટ}: યુઝર ઓથેન્ટિકેશન અને એક્સેસ કંટ્રોલ
\end{itemize}

\textbf{ii) મોબાઇલ ઓપરેટિંગ સિસ્ટમ:}
\begin{itemize}
    \item \textbf{ટચ-ઓપ્ટિમાઇઝ્ડ}: આંગળી-આધારિત ઇન્ટરેક્શન માટે ડિઝાઇન
    \item \textbf{પાવર મેનેજમેન્ટ}: કાર્યક્ષમ બેટરી ઉપયોગ
    \item \textbf{એપ ઇકોસિસ્ટમ}: કેન્દ્રીકૃત એપ વિતરણ અને મેનેજમેન્ટ
\end{itemize}

\begin{mnemonicbox}
\mnemonic{NOS for Networks, MOS for Mobility}
\end{mnemonicbox}
\end{solutionbox}

\questionmarks{2(અ OR)}{3}{ફક્ત NAND ગેટનો ઉપયોગ કરીને OR ગેટ અને NOT ગેટનું લોજિક સર્કિટ દોરો.}

\begin{solutionbox}
\textbf{જવાબ:}

\begin{answerdiagram}{NAND ઉપયોગ કરી OR ગેટ}
\begin{tikzpicture}
    % OR using NAND: (A'B')' = A+B
    \node[nand gate US, draw, logic gate inputs=nn] (n1) at (0,1) {};
    \draw (n1.input 1) -- ++(-0.5,0) node[left] {A}; 
    \draw (n1.input 2) -- ++(-0.5,0) node[left] {A};
    \draw (n1.input 1) -- (n1.input 2);
    
    \node[nand gate US, draw, logic gate inputs=nn] (n2) at (0,-1) {};
    \draw (n2.input 1) -- ++(-0.5,0) node[left] {B};
    \draw (n2.input 2) -- ++(-0.5,0) node[left] {B};
    \draw (n2.input 1) -- (n2.input 2);
    
    \node[nand gate US, draw, logic gate inputs=nn] (n3) at (3,0) {};
    \draw (n1.output) -- (n3.input 1);
    \draw (n2.output) -- (n3.input 2);
    
    \draw (n3.output) -- ++(0.5,0) node[right] {A+B};
\end{tikzpicture}
\end{answerdiagram}

\begin{answerdiagram}{NAND ઉપયોગ કરી NOT ગેટ}
\begin{tikzpicture}
    \node[nand gate US, draw, logic gate inputs=nn] (n1) {};
    \draw (n1.input 1) -- ++(-0.5,0) coordinate (in);
    \draw (n1.input 2) -- ++(-0.5,0);
    \draw (in) -- ++(-0.2,0) node[left] {A};
    \draw (in) ++(0,-0.13) -- (in |- n1.input 2);
    \draw (n1.output) -- ++(0.5,0) node[right] {A'};
\end{tikzpicture}
\end{answerdiagram}

\begin{answertable}{સત્ય વેરિફિકેશન કોષ્ટક}
\begin{tabulary}{\linewidth}{|C|C|C|C|C|}
\hline
\textbf{A} & \textbf{B} & \textbf{A'} & \textbf{B'} & \textbf{(A'$\cdot$B')' = A+B} \\
\hline
0 & 0 & 1 & 1 & 0 \\
\hline
0 & 1 & 1 & 0 & 1 \\
\hline
1 & 0 & 0 & 1 & 1 \\
\hline
1 & 1 & 0 & 0 & 1 \\
\hline
\end{tabulary}
\end{answertable}

\begin{itemize}
    \item \textbf{NAND યુનિવર્સલ}: કોઈ પણ લોજિક ફંક્શન ઇમ્પ્લિમેન્ટ કરી શકે
    \item \textbf{ડી મોર્ગનનો નિયમ}: $(A' \cdot B')' = A+B$
\end{itemize}

\begin{mnemonicbox}
\mnemonic{NAND is Universal - can make all gates}
\end{mnemonicbox}
\end{solutionbox}

\questionmarks{2(બ OR)}{4}{i) બાઇનરી ટુ ડેસિમલ: (i) 11101 (ii) 10011 ii) ડેસિમલ ટુ બાઇનરી: (i) 19 (ii) 64}

\begin{solutionbox}
\textbf{જવાબ:}

\begin{answertable}{રૂપાંતરણ કોષ્ટક}
\begin{tabulary}{\linewidth}{|L|L|L|L|}
\hline
\textbf{પ્રકાર} & \textbf{સંખ્યા} & \textbf{પ્રક્રિયા} & \textbf{પરિણામ} \\
\hline
\textbf{બાઇનરી ટુ ડેસિમલ} & $11101_2$ & $1 \times 2^4+1 \times 2^3+1 \times 2^2+0 \times 2^1+1 \times 2^0$ & \textbf{29$_{10}$} \\
\hline
 & $10011_2$ & $1 \times 2^4+0 \times 2^3+0 \times 2^2+1 \times 2^1+1 \times 2^0$ & \textbf{19$_{10}$} \\
\hline
\textbf{ડેસિમલ ટુ બાઇનરી} & $19_{10}$ & ભાગાકાર પદ્ધતિ & \textbf{10011$_2$} \\
\hline
 & $64_{10}$ & ભાગાકાર પદ્ધતિ & \textbf{1000000$_2$} \\
\hline
\end{tabulary}
\end{answertable}

\textbf{વિગતવાર સોલ્યુશન્સ:}

\textbf{i) બાઇનરી ટુ ડેસિમલ:}
\begin{itemize}
    \item $11101_2 = 16 + 8 + 4 + 0 + 1 = 29_{10}$
    \item $10011_2 = 16 + 0 + 0 + 2 + 1 = 19_{10}$
\end{itemize}

\textbf{ii) ડેસિમલ ટુ બાઇનરી:}
\begin{itemize}
    \item $19 \div 2 = 9 \text{ બાકી } 1, 9 \div 2 = 4 \text{ બાકી } 1... \rightarrow 10011_2$
    \item $64 \div 2 = 32 \text{ બાકી } 0... \rightarrow 1000000_2$
\end{itemize}

\begin{mnemonicbox}
\mnemonic{Powers of 2 for Binary to Decimal}
\end{mnemonicbox}
\end{solutionbox}

\questionmarks{2(ક OR)}{7}{ઓપન સોર્સ સોફ્ટવેર અને પ્રોપ્રાઇટરી સોફ્ટવેર સમજાવો.}

\begin{solutionbox}
\textbf{જવાબ:}

\begin{answertable}{સોફ્ટવેર પ્રકાર સરખામણી}
\begin{tabulary}{\linewidth}{|L|L|L|}
\hline
\textbf{પાસાં} & \textbf{ઓપન-સોર્સ} & \textbf{પ્રોપ્રાઇટરી} \\
\hline
\textbf{સોર્સ કોડ} & મુક્તપણે ઉપલબ્ધ & બંધ/છુપાયેલ \\
\hline
\textbf{કિંમત} & સામાન્ય રીતે મફત & કોમર્શિયલ લાઇસન્સ \\
\hline
\textbf{મોડિફિકેશન} & મંજૂર & પ્રતિબંધિત \\
\hline
\textbf{સપોર્ટ} & કમ્યુનિટી-આધારિત & વેન્ડર સપોર્ટ \\
\hline
\end{tabulary}
\end{answertable}

\begin{answertable}{સોફ્ટવેર ઉદાહરણો}
\begin{tabulary}{\linewidth}{|L|L|}
\hline
\textbf{ઓપન-સોર્સ} & \textbf{પ્રોપ્રાઇટરી} \\
\hline
Linux & Microsoft Windows \\
\hline
LibreOffice & Microsoft Office \\
\hline
Firefox & Internet Explorer \\
\hline
GIMP & Adobe Photoshop \\
\hline
MySQL & Oracle Database \\
\hline
\end{tabulary}
\end{answertable}

\begin{answerdiagram}{સોફ્ટવેર વિતરણ}
\begin{tikzpicture}
    \pie[text=legend]{40/ઓપન-સોર્સ, 60/પ્રોપ્રાઇટરી}
\end{tikzpicture}
\end{answerdiagram}

\begin{itemize}
    \item \textbf{ઓપન-સોર્સ લાક્ષણિકતાઓ}: મોડિફાઇ કરવાની સ્વતંત્રતા, કમ્યુનિટી ડેવલપમેન્ટ, પારદર્શિતા
    \item \textbf{પ્રોપ્રાઇટરી લાક્ષણિકતાઓ}: કોમર્શિયલ મોડેલ, પ્રોફેશનલ સપોર્ટ, ગુણવત્તા ખાતરી
\end{itemize}

\begin{mnemonicbox}
\mnemonic{FOSS is Free, Open, Shared, Supported by community}
\end{mnemonicbox}
\end{solutionbox}

\questionmarks{3(અ)}{3}{વ્યાખ્યાયિત કરો: 1. મોડ્યુલેશન 2. મલ્ટિપ્લેક્સિંગ}

\begin{solutionbox}
\textbf{જવાબ:}

\begin{answertable}{વ્યાખ્યા કોષ્ટક}
\begin{tabulary}{\linewidth}{|L|L|L|}
\hline
\textbf{શબ્દ} & \textbf{વ્યાખ્યા} & \textbf{હેતુ} \\
\hline
\keyword{મોડ્યુલેશન} & કેરિયર સિગ્નલના ગુણધર્મો બદલવાની પ્રક્રિયા & લાંબા અંતરનું ટ્રાન્સમિશન સક્ષમ કરવું \\
\hline
\keyword{મલ્ટિપ્લેક્સિંગ} & ટ્રાન્સમિશન માટે મલ્ટિપલ સિગ્નલો જોડવા & કાર્યક્ષમ ચેનલ ઉપયોગ \\
\hline
\end{tabulary}
\end{answertable}

\begin{itemize}
    \item \textbf{મોડ્યુલેશન}: કેરિયર વેવના એમ્પ્લિટ્યુડ, ફ્રીક્વન્સી અથવા ફેઝ બદલે
    \item \textbf{મલ્ટિપ્લેક્સિંગ}: મલ્ટિપલ યુઝર્સને એક જ કમ્યુનિકેશન મીડિયમ શેર કરવાની મંજૂરી આપે
    \item \textbf{સિગ્નલ પ્રોસેસિંગ}: બંને તકનીકો કમ્યુનિકેશન કાર્યક્ષમતા સુધારે
\end{itemize}

\begin{mnemonicbox}
\mnemonic{MM - Modulation Modifies, Multiplexing Merges}
\end{mnemonicbox}
\end{solutionbox}

\questionmarks{3(બ)}{4}{સ્ટાર ટોપોલોજી સમજાવો.}

\begin{solutionbox}
\textbf{જવાબ:}

\begin{answerdiagram}{સ્ટાર ટોપોલોજી}
\begin{tikzpicture}[node distance=2.5cm]
    \node[gtu block] (hub) {હબ/સ્વિચ};
    \node[gtu child, above=of hub] (pc1) {PC1};
    \node[gtu child, right=of hub] (pc2) {PC2};
    \node[gtu child, below=of hub] (pc3) {PC3};
    \node[gtu child, left=of hub] (pc4) {PC4};
    
    \draw[gtu arrow] (hub) -- (pc1);
    \draw[gtu arrow] (hub) -- (pc2);
    \draw[gtu arrow] (hub) -- (pc3);
    \draw[gtu arrow] (hub) -- (pc4);
\end{tikzpicture}
\end{answerdiagram}

\begin{answertable}{સ્ટાર ટોપોલોજી ફીચર્સ}
\begin{tabulary}{\linewidth}{|L|L|}
\hline
\textbf{ફીચર} & \textbf{વિવરણ} \\
\hline
\textbf{કેન્દ્રીય ડિવાઇસ} & હબ/સ્વિચ બધા નોડ્સને જોડે \\
\hline
\textbf{ફોલ્ટ ટોલરન્સ} & સિંગલ નોડ ફેઇલ્યૂર અન્યને અસર કરતું નથી \\
\hline
\textbf{પર્ફોર્મન્સ} & દરેક કનેક્શન માટે સમર્પિત બેન્ડવિથ \\
\hline
\textbf{સ્કેલેબિલિટી} & નોડ્સ ઉમેરવા/હટાવવા સરળ \\
\hline
\end{tabulary}
\end{answertable}

\begin{itemize}
    \item \textbf{કેન્દ્રીય હબ}: બધું કમ્યુનિકેશન કેન્દ્રીય ડિવાઇસ દ્વારા પસાર થાય
    \item \textbf{સરળ ટ્રબલશૂટિંગ}: સમસ્યાઓ વ્યક્તિગત કનેક્શન્સમાં અલગ
    \item \textbf{વધુ કિંમત}: બસ ટોપોલોજી કરતાં વધુ કેબલ જરૂરી
    \item \textbf{સિંગલ પોઇન્ટ ઓફ ફેઇલ્યૂર}: હબ ફેઇલ થવાથી આખું નેટવર્ક અસર પામે
\end{itemize}

\begin{mnemonicbox}
\mnemonic{STAR - Single point, Troubleshooting easy, All through hub, Reliable}
\end{mnemonicbox}
\end{solutionbox}

\questionmarks{3(ક)}{7}{ટાઇમ ડિવિઝન મલ્ટિપ્લેક્સિંગ (TDM) પર ટૂંકી નોંધ તૈયાર કરો}

\begin{solutionbox}
\textbf{જવાબ:}

\begin{answerdiagram}{ટાઇમ ડિવિઝન મલ્ટિપ્લેક્સિંગ}
\begin{tikzpicture}[x=1.5cm, y=1cm]
    % Time axis
    \draw[->] (0,0) -- (6,0) node[right] {Time};
    
    % Slots
    \draw[fill=red!20] (0.5,0.2) rectangle (1.5,1) node[midway] {User 1};
    \draw[fill=blue!20] (1.5,0.2) rectangle (2.5,1) node[midway] {User 2};
    \draw[fill=green!20] (2.5,0.2) rectangle (3.5,1) node[midway] {User 3};
    \draw[fill=red!20] (3.5,0.2) rectangle (4.5,1) node[midway] {User 1};
    \draw[fill=blue!20] (4.5,0.2) rectangle (5.5,1) node[midway] {User 2};
    
    \node at (1, -0.5) {Frame 1};
    \node at (4, -0.5) {Frame 2};
    
    \draw[dashed] (3.5, -0.2) -- (3.5, 1.2);
\end{tikzpicture}
\end{answerdiagram}

\begin{answertable}{TDM લાક્ષણિકતાઓ}
\begin{tabulary}{\linewidth}{|L|L|}
\hline
\textbf{ફીચર} & \textbf{વિવરણ} \\
\hline
\textbf{સિદ્ધાંત} & વિવિધ યુઝર્સને વિવિધ ટાઇમ સ્લોટ્સ ફાળવાય \\
\hline
\textbf{સિન્ક્રોનાઇઝેશન} & બધા ડિવાઇસ સિન્ક્રોનાઇઝ હોવા જોઈએ \\
\hline
\textbf{કાર્યક્ષમતા} & સ્લોટ્સ ભરાયા હોય ત્યારે સંપૂર્ણ બેન્ડવિથ ઉપયોગ \\
\hline
\textbf{એપ્લિકેશન્સ} & ડિજિટલ ટેલિફોન સિસ્ટમ્સ, T1/E1 લાઇન્સ \\
\hline
\end{tabulary}
\end{answertable}

\textbf{TDM પ્રકારો:}
\begin{itemize}
    \item \textbf{સિન્ક્રોનસ TDM}: ડેટા ઉપલબ્ધતાને ધ્યાનમાં લીધા વિના નિશ્ચિત ટાઇમ સ્લોટ્સ
    \item \textbf{એસિન્ક્રોનસ TDM}: માંગના આધારે ડાયનેમિક સ્લોટ ફાળવણી
    \item \textbf{સ્ટેટિસ્ટિકલ TDM}: આંકડાકીય આધારે સ્લોટ્સ ફાળવાય
\end{itemize}

\textbf{ફાયદાઓ:}
\begin{itemize}
    \item \textbf{ન્યાયી શેરિંગ}: બધા યુઝર્સ માટે સમાન ટાઇમ ફાળવણી
    \item \textbf{કોઈ સિગ્નલ ઇન્ટરફેરન્સ નહીં}: ટાઇમ-આધારિત અલગીકરણ સંઘર્ષ અટકાવે
\end{itemize}

\begin{mnemonicbox}
\mnemonic{TDM - Time Divides Medium fairly}
\end{mnemonicbox}
\end{solutionbox}

\questionmarks{3(અ OR)}{3}{એમ્પ્લિટ્યુડ મોડ્યુલેશન (AM) સમજાવો.}

\begin{solutionbox}
\textbf{જવાબ:}

\begin{answerdiagram}{AM સિગ્નલ}
\begin{tikzpicture}
    % Message Signal
    \draw[blue, thick] plot[domain=0:6, samples=100] (\x, {0.5*sin(2*\x r)}) node[right] {Message};
    % Carrier Signal
    \draw[red, dashed] plot[domain=0:6, samples=200] (\x, {sin(20*\x r) - 2}) node[right] {Carrier};
    % AM Signal
    \draw[purple, thick] plot[domain=0:6, samples=300] (\x, {(1 + 0.5*sin(2*\x r)) * sin(20*\x r) - 5}) node[right] {AM Output};
\end{tikzpicture}
\end{answerdiagram}

\begin{answertable}{AM લાક્ષણિકતાઓ}
\begin{tabulary}{\linewidth}{|L|L|}
\hline
\textbf{પેરામીટર} & \textbf{વિવરણ} \\
\hline
\textbf{વ્યાખ્યા} & મેસેજ સિગ્નલ સાથે કેરિયરનું એમ્પ્લિટ્યુડ બદલાય \\
\hline
\textbf{ફ્રીક્વન્સી રેન્જ} & 535-1605 kHz (AM રેડિયો) \\
\hline
\textbf{બેન્ડવિથ} & મેસેજ સિગ્નલ ફ્રીક્વન્સીથી બમણું \\
\hline
\end{tabulary}
\end{answertable}

\begin{itemize}
    \item \textbf{કેરિયર વેવ}: માહિતી વહન કરતું હાઇ ફ્રીક્વન્સી સિગ્નલ
    \item \textbf{મોડ્યુલેશન ઇન્ડેક્સ}: એમ્પ્લિટ્યુડ વેરિએશનની ઊંડાઈ નક્કી કરે
    \item \textbf{એપ્લિકેશન્સ}: AM રેડિયો બ્રોડકાસ્ટિંગ, એરક્રાફ્ટ કમ્યુનિકેશન
\end{itemize}

\begin{mnemonicbox}
\mnemonic{AM - Amplitude Modifies with message}
\end{mnemonicbox}
\end{solutionbox}

\questionmarks{3(બ OR)}{4}{DNS વર્ણવો.}

\begin{solutionbox}
\textbf{જવાબ:}

\begin{answerdiagram}{DNS હાયરાર્કી}
\begin{tikzpicture}[gtu tree]
    \node [gtu root] {રૂટ (.)}
        child { node [gtu block, fill=yellow!10] {.com} 
            child { node [gtu child] {google.com} 
                child { node [gtu block, fill=orange!10] {mail} }
                child { node [gtu block, fill=orange!10] {www} }
            }
            child { node [gtu child] {microsoft.com} }
        }
        child { node [gtu block, fill=yellow!10] {.org} };
\end{tikzpicture}
\end{answerdiagram}

\begin{answertable}{DNS કમ્પોનન્ટ્સ}
\begin{tabulary}{\linewidth}{|L|L|}
\hline
\textbf{કમ્પોનન્ટ} & \textbf{ફંક્શન} \\
\hline
\textbf{ડોમેઇન નેમ} & માનવ-વાંચી શકાય તેવું વેબ એડ્રેસ \\
\hline
\textbf{IP એડ્રેસ} & સર્વરનું સંખ્યાકીય એડ્રેસ \\
\hline
\textbf{DNS સર્વર} & નામોને IP એડ્રેસમાં ટ્રાન્સલેટ કરે \\
\hline
\textbf{રેકોર્ડ્સ} & વિવિધ પ્રકારો (A, MX, CNAME) \\
\hline
\end{tabulary}
\end{answertable}

\begin{itemize}
    \item \textbf{નેમ રિઝોલ્યુશન}: ડોમેઇન નામોને IP એડ્રેસમાં કન્વર્ટ કરે
    \item \textbf{હાયરાર્કિકલ સ્ટ્રક્ચર}: રૂટ, TLD, સેકન્ડ-લેવલ ડોમેઇન્સ
    \item \textbf{ડિસ્ટ્રિબ્યુટેડ ડેટાબેસ}: કોઈ સિંગલ પોઇન્ટ ઓફ ફેઇલ્યૂર નથી
    \item \textbf{કેશિંગ}: તાજેતરના લુકઅપ્સ સ્ટોર કરીને પર્ફોર્મન્સ સુધારે
\end{itemize}

\begin{mnemonicbox}
\mnemonic{DNS - Domain Name System translates addresses}
\end{mnemonicbox}
\end{solutionbox}

\questionmarks{3(ક OR)}{7}{નીચેનું વર્ણન કરો: 1. સીરિયલ કમ્યુનિકેશન 2. સિન્ક્રોનસ ટ્રાન્સમિશન}

\begin{solutionbox}
\textbf{જવાબ:}

\begin{answerdiagram}{કમ્યુનિકેશન પ્રકારો}
\begin{tikzpicture}[gtu tree]
    \node [gtu root] {ડેટા કમ્યુનિકેશન}
        child { node [gtu child] {સીરિયલ} 
            child { node [gtu block] {સિન્ક્રોનસ} }
            child { node [gtu block] {એસિન્ક્રોનસ} }
        }
        child { node [gtu child] {પેરેલલ} };
\end{tikzpicture}
\end{answerdiagram}

\begin{answertable}{કમ્યુનિકેશન સરખામણી}
\begin{tabulary}{\linewidth}{|L|L|L|L|}
\hline
\textbf{પ્રકાર} & \textbf{વિવરણ} & \textbf{ટાઇમિંગ} & \textbf{ઉદાહરણો} \\
\hline
\keyword{સીરિયલ કમ્યુનિકેશન} & ડેટા બિટ્સ એક પછી એક મોકલાય & ધીમું પરંતુ વિશ્વસનીય & RS-232, USB, ઇથરનેટ \\
\hline
\keyword{સિન્ક્રોનસ ટ્રાન્સમિશન} & ક્લોક સિગ્નલ સેન્ડર/રિસીવર સિન્ક કરે & ચોક્કસ ટાઇમિંગ & HDLC, SDLC \\
\hline
\end{tabulary}
\end{answertable}

\textbf{1. સીરિયલ કમ્યુનિકેશન:}
\begin{itemize}
    \item \textbf{સિંગલ વાયર}: ડેટા સિંગલ ચેનલ પર બિટ બાય બિટ ટ્રાન્સમિટ થાય
    \item \textbf{કોસ્ટ ઇફેક્ટિવ}: પેરેલલ કરતાં ઓછા વાયર જરૂરી
    \item \textbf{લાંબો અંતર}: નોઇઝ અને ઇન્ટરફેરન્સને ઓછું સંવેદનશીલ
    \item \textbf{એરર ડિટેક્શન}: ડેટા ઇન્ટેગ્રિટી માટે બિલ્ટ-ઇન મેકેનિઝમ
\end{itemize}

\textbf{2. સિન્ક્રોનસ ટ્રાન્સમિશન:}
\begin{itemize}
    \item \textbf{ક્લોક સિન્ક્રોનાઇઝેશન}: અલગ ક્લોક સિગ્નલ અથવા એમ્બેડેડ ટાઇમિંગ
    \item \textbf{બ્લોક ટ્રાન્સમિશન}: ડેટા સતત બ્લોક્સમાં મોકલાય
    \item \textbf{વધુ કાર્યક્ષમતા}: સ્ટાર્ટ/સ્ટોપ બિટ્સની જરૂર નથી
    \item \textbf{કોમ્પ્લેક્સ હાર્ડવેર}: સિન્ક્રોનાઇઝ્ડ ક્લોક્સ જરૂરી
\end{itemize}

\begin{mnemonicbox}
\mnemonic{Serial is Sequential, Synchronous is Simultaneous}
\end{mnemonicbox}
\end{solutionbox}


\questionmarks{4(અ)}{3}{મેશ અને બસ ટોપોલોજીમાં તફાવત કરો.}

\begin{solutionbox}
\textbf{જવાબ:}

\begin{answertable}{ટોપોલોજી સરખામણી}
\begin{tabulary}{\linewidth}{|L|L|L|}
\hline
\textbf{ફીચર} & \textbf{મેશ ટોપોલોજી} & \textbf{બસ ટોપોલોજી} \\
\hline
\textbf{કનેક્શન} & દરેક નોડ બીજા દરેક સાથે જોડાયેલ & બધા નોડ્સ સિંગલ કેબલ પર \\
\hline
\textbf{ફોલ્ટ ટોલરન્સ} & ખૂબ વધારે & ઓછું (સિંગલ પોઇન્ટ ઓફ ફેઇલ્યૂર) \\
\hline
\textbf{કિંમત} & ખૂબ મોંઘું & આર્થિક \\
\hline
\textbf{પર્ફોર્મન્સ} & ઉત્તમ & વધુ નોડ્સ સાથે ઘટે \\
\hline
\end{tabulary}
\end{answertable}

\begin{answerdiagram}{મેશ vs બસ ટોપોલોજી}
\begin{tikzpicture}[node distance=1.5cm]
    % Mesh
    \begin{scope}[shift={(0,0)}]
        \node[gtu state] (A) at (0,2) {A};
        \node[gtu state] (B) at (2,2) {B};
        \node[gtu state] (C) at (0,0) {C};
        \node[gtu state] (D) at (2,0) {D};
        \draw (A)--(B) (A)--(C) (A)--(D) (B)--(C) (B)--(D) (C)--(D);
        \node at (1,-1) {મેશ ટોપોલોજી};
    \end{scope}
    
    % Bus
    \begin{scope}[shift={(4,0.5)}]
       \draw[thick] (0,1) -- (5,1) node[right] {ટર્મિનેટર};
       \draw[thick] (0,1) node[left] {ટર્મિનેટર};
       \node[gtu child] (A) at (0.5,0) {A};
       \node[gtu child] (B) at (2,0) {B};
       \node[gtu child] (C) at (3.5,0) {C};
       \draw (A) -- (0.5,1);
       \draw (B) -- (2,1);
       \draw (C) -- (3.5,1);
       \node at (2.5,-1.5) {બસ ટોપોલોજી};
    \end{scope}
\end{tikzpicture}
\end{answerdiagram}

\begin{itemize}
    \item \textbf{મેશ ફાયદાઓ}: રિડન્ડન્ટ પાથ, ઉચ્ચ વિશ્વસનીયતા
    \item \textbf{બસ ફાયદાઓ}: સરળ ઇન્સ્ટોલેશન, કોસ્ટ-ઇફેક્ટિવ
    \item \textbf{કેબલ જરૂરિયાતો}: મેશને n(n-1)/2 કનેક્શન્સ જરૂરી, બસને સિંગલ કેબલ
\end{itemize}

\begin{mnemonicbox}
\mnemonic{Mesh is Many connections, Bus is Basic single line}
\end{mnemonicbox}
\end{solutionbox}

\questionmarks{4(બ)}{4}{FDM અને TDM ની સરખામણી કરો.}

\begin{solutionbox}
\textbf{જવાબ:}

\begin{answertable}{FDM vs TDM સરખામણી}
\begin{tabulary}{\linewidth}{|L|L|L|}
\hline
\textbf{પેરામીટર} & \textbf{FDM} & \textbf{TDM} \\
\hline
\textbf{ફુલ ફોર્મ} & ફ્રીક્વન્સી ડિવિઝન મલ્ટિપ્લેક્સિંગ & ટાઇમ ડિવિઝન મલ્ટિપ્લેક્સિંગ \\
\hline
\textbf{વિભાજન આધાર} & ફ્રીક્વન્સી બેન્ડ્સ & ટાઇમ સ્લોટ્સ \\
\hline
\textbf{સિગ્નલ પ્રકાર} & એનાલોગ & ડિજિટલ \\
\hline
\textbf{ક્રોસટોક} & ચેનલો વચ્ચે શક્ય & કોઈ ક્રોસટોક નથી \\
\hline
\textbf{સિન્ક્રોનાઇઝેશન} & જરૂરી નથી & જરૂરી \\
\hline
\textbf{કાર્યક્ષમતા} & ગાર્ડ બેન્ડ્સને કારણે ઓછી & વધુ કાર્યક્ષમતા \\
\hline
\end{tabulary}
\end{answertable}

\begin{answerdiagram}{મલ્ટિપ્લેક્સિંગ હાયરાર્કી}
\begin{tikzpicture}[gtu tree]
    \node [gtu root] {મલ્ટિપ્લેક્સિંગ}
        child { node [gtu child] {FDM} 
            child { node [gtu block] {રેડિયો} }
            child { node [gtu block] {કેબલ TV} }
        }
        child { node [gtu child] {TDM} 
            child { node [gtu block] {ટેલિફોની} }
            child { node [gtu block] {નેટવર્ક્સ} }
        };
\end{tikzpicture}
\end{answerdiagram}

\textbf{FDM લાક્ષણિકતાઓ:}
\begin{itemize}
    \item \textbf{ફ્રીક્વન્સી સેપેરેશન}: દરેક સિગ્નલને અલગ ફ્રીક્વન્સી બેન્ડ ફાળવાય
    \item \textbf{સાથોસાથ ટ્રાન્સમિશન}: બધા સિગ્નલો એક જ સમયે ટ્રાન્સમિટ થાય
    \item \textbf{ગાર્ડ બેન્ડ્સ}: ચેનલો વચ્ચે ઇન્ટરફેરન્સ અટકાવે
\end{itemize}

\textbf{TDM લાક્ષણિકતાઓ:}
\begin{itemize}
    \item \textbf{ટાઇમ સેપેરેશન}: દરેક સિગ્નલને અલગ ટાઇમ સ્લોટ ફાળવાય
    \item \textbf{ક્રમિક ટ્રાન્સમિશન}: સિગ્નલો એક પછી એક ટ્રાન્સમિટ થાય
    \item \textbf{ચોક્કસ ટાઇમિંગ}: સિન્ક્રોનાઇઝ્ડ ક્લોક્સ જરૂરી
\end{itemize}

\begin{mnemonicbox}
\mnemonic{FDM uses Frequency, TDM uses Time}
\end{mnemonicbox}
\end{solutionbox}

\questionmarks{4(ક)}{7}{OSI રેફરન્સ મોડેલ દોરો અને સમજાવો.}

\begin{solutionbox}
\textbf{જવાબ:}

\begin{answerdiagram}{OSI રેફરન્સ મોડેલ}
\begin{tikzpicture}[node distance=0cm, every node/.style={gtu block, minimum width=6cm, minimum height=0.8cm}]
    \node (app) {7. એપ્લિકેશન લેયર (HTTP, FTP)};
    \node [below=of app] (pres) {6. પ્રેઝન્ટેશન લેયર (એન્ક્રિપ્શન)};
    \node [below=of pres] (sess) {5. સેશન લેયર (RPC)};
    \node [below=of sess] (trans) {4. ટ્રાન્સપોર્ટ લેયર (TCP, UDP)};
    \node [below=of trans] (net) {3. નેટવર્ક લેયર (IP, રાઉટિંગ)};
    \node [below=of net] (link) {2. ડેટા લિંક લેયર (ઇથરનેટ)};
    \node [below=of link] (phy) {1. ફિઝિકલ લેયર (કેબલ્સ, હબ્સ)};
    
    \draw[->] (app.east) -- ++(1,0) |- (phy.east) node[midway, right] {ડેટા એન્કેપ્સુલેશન};
    \draw[->] (phy.west) -- ++(-1,0) |- (app.west) node[midway, left] {ડેટા ડિકેપ્સુલેશન};
\end{tikzpicture}
\end{answerdiagram}

\begin{answertable}{OSI લેયર ફંક્શન્સ}
\begin{tabulary}{\linewidth}{|C|L|L|L|}
\hline
\textbf{લેયર} & \textbf{નામ} & \textbf{ફંક્શન} & \textbf{ઉદાહરણો} \\
\hline
\textbf{7} & એપ્લિકેશન & યુઝર ઇન્ટરફેસ & HTTP, FTP, SMTP \\
\hline
\textbf{6} & પ્રેઝન્ટેશન & ડેટા ફોર્મેટિંગ & એન્ક્રિપ્શન, કમ્પ્રેશન \\
\hline
\textbf{5} & સેશન & સેશન મેનેજમેન્ટ & NetBIOS, RPC \\
\hline
\textbf{4} & ટ્રાન્સપોર્ટ & એન્ડ-ટુ-એન્ડ ડિલિવરી & TCP, UDP \\
\hline
\textbf{3} & નેટવર્ક & રાઉટિંગ & IP, ICMP \\
\hline
\textbf{2} & ડેટા લિંક & ફ્રેમ ડિલિવરી & ઇથરનેટ, PPP \\
\hline
\textbf{1} & ફિઝિકલ & બિટ ટ્રાન્સમિશન & કેબલ્સ, હબ્સ \\
\hline
\end{tabulary}
\end{answertable}

\begin{itemize}
    \item \textbf{લેયર્ડ આર્કિટેક્ચર}: દરેક લેયરની ચોક્કસ જવાબદારીઓ
    \item \textbf{પ્રોટોકોલ ઇન્ડિપેન્ડન્સ}: લેયર્સ સ્વતંત્ર રીતે મોડિફાઇ કરી શકાય
    \item \textbf{સ્ટાન્ડર્ડાઇઝેશન}: નેટવર્ક કમ્યુનિકેશન માટે સામાન્ય ફ્રેમવર્ક
    \item \textbf{એન્કેપ્સુલેશન}: દરેક લેયર પોતાનું હેડર ઉમેરે
\end{itemize}

\begin{mnemonicbox}
\mnemonic{All People Seem To Need Data Processing}
\end{mnemonicbox}
\end{solutionbox}

\questionmarks{4(અ OR)}{3}{સંક્ષિપ્તમાં હબનું વર્ણન કરો.}

\begin{solutionbox}
\textbf{જવાબ:}

\begin{answerdiagram}{નેટવર્ક હબ}
\begin{tikzpicture}
    \node[gtu block, minimum width=2cm, minimum height=1cm, fill=gray!20] (hub) {HUB};
    \node[gtu child, above left=of hub] (pc1) {PC1};
    \node[gtu child, above right=of hub] (pc2) {PC2};
    \node[gtu child, below=of hub] (pc3) {PC3};
    \draw (hub) -- (pc1);
    \draw (hub) -- (pc2);
    \draw (hub) -- (pc3);
\end{tikzpicture}
\end{answerdiagram}

\begin{answertable}{હબ લાક્ષણિકતાઓ}
\begin{tabulary}{\linewidth}{|L|L|}
\hline
\textbf{ફીચર} & \textbf{વિવરણ} \\
\hline
\textbf{ફંક્શન} & ડિવાઇસ માટે કેન્દ્રીય કનેક્શન પોઇન્ટ \\
\hline
\textbf{પ્રકાર} & ફિઝિકલ લેયર ડિવાઇસ (લેયર 1) \\
\hline
\textbf{ડેટા હેન્ડલિંગ} & બધા કનેક્ટેડ ડિવાઇસમાં બ્રોડકાસ્ટ \\
\hline
\textbf{કોલિઝન ડોમેઇન} & બધા પોર્ટ્સ એક જ કોલિઝન ડોમેઇન શેર કરે \\
\hline
\end{tabulary}
\end{answertable}

\begin{itemize}
    \item \textbf{શે ર્ડ બેન્ડવિથ}: બધા કનેક્ટેડ ડિવાઇસ કુલ બેન્ડવિથ શેર કરે
    \item \textbf{હાફ-ડુપ્લેક્સ}: સાથોસાથ મોકલી અને મેળવી શકતું નથી
    \item \textbf{સિક્યોરિટી ઇશ્યૂઝ}: બધા ડિવાઇસ બધો ટ્રાન્સમિટ થયેલો ડેટા મેળવે
    \item \textbf{અપ્રચલિત ટેકનોલોજી}: આધુનિક નેટવર્ક્સમાં સ્વિચ દ્વારા બદલાયું
\end{itemize}

\begin{mnemonicbox}
\mnemonic{Hub is Half-duplex, shares Bandwidth}
\end{mnemonicbox}
\end{solutionbox}

\questionmarks{4(બ OR)}{4}{STP અને UTP ની સરખામણી કરો.}

\begin{solutionbox}
\textbf{જવાબ:}

\begin{answertable}{STP vs UTP કેબલ સરખામણી}
\begin{tabulary}{\linewidth}{|L|L|L|}
\hline
\textbf{ફીચર} & \textbf{STP (શિલ્ડેડ)} & \textbf{UTP (અનશિલ્ડેડ)} \\
\hline
\textbf{શિલ્ડિંગ} & મેટલ ફોઇલ/બ્રેઇડ પ્રોટેક્શન & કોઈ શિલ્ડિંગ નથી \\
\hline
\textbf{કિંમત} & વધુ મોંઘું & ઓછું મોંઘું \\
\hline
\textbf{ઇન્સ્ટોલેશન} & ગ્રાઉન્ડિંગને કારણે જટિલ & સરળ ઇન્સ્ટોલેશન \\
\hline
\textbf{EMI રેઝિસ્ટન્સ} & ઉત્તમ પ્રોટેક્શન & મધ્યમ પ્રોટેક્શન \\
\hline
\textbf{એપ્લિકેશન્સ} & ઇન્ડસ્ટ્રિયલ વાતાવરણ & ઓફિસ વાતાવરણ \\
\hline
\end{tabulary}
\end{answertable}

\begin{answerdiagram}{કેબલ સ્ટ્રક્ચર}
\begin{tikzpicture}
    % UTP
    \node[draw, cylinder, shape border rotate=0, minimum height=2cm, minimum width=1cm, label=below:UTP] (utp) at (0,0) {Wires};
    
    % STP
    \node[draw, cylinder, shape border rotate=0, minimum height=2cm, minimum width=1cm, fill=gray!30, label=below:STP] (stp) at (4,0) {Shield + Wires};
\end{tikzpicture}
\end{answerdiagram}

\begin{itemize}
    \item \textbf{STP ફાયદાઓ}: બેહતર નોઇઝ ઇમ્યુનિટી, હાયર ડેટા રેટ્સ, સિક્યોર ટ્રાન્સમિશન
    \item \textbf{UTP ફાયદાઓ}: કોસ્ટ ઇફેક્ટિવ, ઇઝી ઇન્સ્ટોલેશન, ફ્લેક્સિબિલિટી
\end{itemize}

\begin{mnemonicbox}
\mnemonic{STP is Shielded but Pricey, UTP is Unshielded but Popular}
\end{mnemonicbox}
\end{solutionbox}

\questionmarks{4(ક OR)}{7}{LAN, MAN, WAN મા ભેદ પાડો.}

\begin{solutionbox}
\textbf{જવાબ:}

\begin{answerdiagram}{નેટવર્ક પ્રકારો હાયરાર્કી}
\begin{tikzpicture}[gtu tree]
    \node [gtu root] {કમ્પ્યુટર નેટવર્ક્સ}
        child { node [gtu child] {LAN} 
            child { node [gtu block] {કેમ્પસ} }
        }
        child { node [gtu child] {MAN} 
            child { node [gtu block] {શહેર} }
        }
        child { node [gtu child] {WAN} 
            child { node [gtu block] {વૈશ્વિક} }
        };
\end{tikzpicture}
\end{answerdiagram}

\begin{answertable}{નેટવર્ક પ્રકાર સરખામણી}
\begin{tabulary}{\linewidth}{|L|L|L|L|}
\hline
\textbf{પેરામીટર} & \textbf{LAN} & \textbf{MAN} & \textbf{WAN} \\
\hline
\textbf{કવરેજ} & બિલ્ડિંગ/કેમ્પસ & શહેર/મેટ્રોપોલિટન વિસ્તાર & દેશ/ખંડ \\
\hline
\textbf{સ્પીડ} & ઉચ્ચ (1 Gbps+) & મધ્યમ & ઓછી/પરિવર્તનશીલ \\
\hline
\textbf{કિંમત} & ઓછી & મધ્યમ & વધારે \\
\hline
\textbf{માલિકી} & પ્રાઇવેટ & પ્રાઇવેટ/પબ્લિક & પબ્લિક/લીઝ્ડ \\
\hline
\textbf{ટેકનોલોજી} & ઇથરનેટ, Wi-Fi & ફાઇબર, WiMAX & સ્ટેલાઇટ, લીઝ્ડ લાઇન્સ \\
\hline
\end{tabulary}
\end{answertable}

\begin{itemize}
    \item \textbf{LAN (લોકલ એરિયા નેટવર્ક)}: ઉચ્ચ સ્પીડ, ઓછી કિંમત, પ્રાઇવેટ માલિકી
    \item \textbf{MAN (મેટ્રોપોલિટન એરિયા નેટવર્ક)}: શહેર-વ્યાપી, મધ્યમ સ્પીડ, મિશ્ર માલિકી
    \item \textbf{WAN (વાઇડ એરિયા નેટવર્ક)}: વૈશ્વિક કવરેજ, પબ્લિક ઇન્ફ્રાસ્ટ્રક્ચર, પરિવર્તનશીલ સ્પીડ
\end{itemize}

\begin{mnemonicbox}
\mnemonic{LAN is Local, MAN is Metropolitan, WAN is Wide}
\end{mnemonicbox}
\end{solutionbox}

\questionmarks{5(અ)}{3}{ડિનાયલ ઓફ સર્વિસ અટેક સમજાવો.}

\begin{solutionbox}
\textbf{જવાબ:}

\begin{answerdiagram}{DoS અટેક}
\begin{tikzpicture}
    \node[gtu block, fill=red!10] (attacker) {હુમલાખોર};
    \node[gtu block, right=4cm of attacker] (target) {ટાર્ગેટ સર્વર};
    
    \draw[->, thick, red] (attacker) -- (target) node[midway, above] {Request 1};
    \draw[->, thick, red] (attacker.north) to[bend left=20] node[midway, above] {Request 2} (target.north);
    \draw[->, thick, red] (attacker.south) to[bend right=20] node[midway, below] {Request 3...N} (target.south);
    
    \node[below=0.5cm of target, text width=3cm, align=center] {ઓવરવ્હેલ્મ્ડ};
\end{tikzpicture}
\end{answerdiagram}

\begin{itemize}
    \item \textbf{વ્યાખ્યા}: હુમલો જ્યાં કાયદેસર યુઝર્સ માહિતી સિસ્ટમ્સ એક્સેસ કરી શકતા નથી
    \item \textbf{પદ્ધતિ}: સિસ્ટમ રિસોર્સને ઓવરલોડ કરવા વધુ પડતી રિક્વેસ્ટ્સથી ટાર્ગેટને ફ્લડ કરવું
    \item \textbf{અસર}: સર્વિસ ડાઉનટાઇમ, ફાઇનાન્શિયલ લોસ, રેપ્યુટેશન ડેમેજ
\end{itemize}

\end{solutionbox}


\questionmarks{5(બ)}{4}{i) ડેટા ટ્રાન્સમિશનનું વર્ગીકરણ કરો. ii) બસ ટોપોલોજીમાં ટર્મિનેટરનો ઉપયોગ લખો.}

\begin{solutionbox}
\textbf{જવાબ:}

\textbf{i) ડેટા ટ્રાન્સમિશન વર્ગીકરણ:}

\begin{answerdiagram}{ડેટા ટ્રાન્સમિશન પ્રકારો}
\begin{tikzpicture}[gtu tree]
    \node [gtu root] {ડેટા ટ્રાન્સમિશન}
        child { node [gtu child] {દિશા} 
            child { node [gtu block] {સિમ્પ્લેક્સ} }
            child { node [gtu block] {હાફ-ડુપ્લેક્સ} }
            child { node [gtu block] {ફુલ-ડુપ્લેક્સ} }
        }
        child { node [gtu child] {ટાઇમિંગ} 
            child { node [gtu block] {સિન્ક્રોનસ} }
            child { node [gtu block] {એસિન્ક્રોનસ} }
        }
        child { node [gtu child] {મોડ} 
            child { node [gtu block] {સીરિયલ} }
            child { node [gtu block] {પેરેલલ} }
        };
\end{tikzpicture}
\end{answerdiagram}

\textbf{ii) બસ ટોપોલોજીમાં ટર્મિનેટર:}

\begin{answertable}{ટર્મિનેટર ફંક્શન્સ}
\begin{tabulary}{\linewidth}{|L|L|}
\hline
\textbf{ફંક્શન} & \textbf{વિવરણ} \\
\hline
\textbf{સિગ્નલ એબ્સોર્પ્શન} & સિગ્નલ રિફ્લેક્શન અટકાવે \\
\hline
\textbf{ઇમ્પીડન્સ મેચિંગ} & કેબલ ઇમ્પીડન્સ મેચ કરે \\
\hline
\textbf{નેટવર્ક ઇન્ટેગ્રિટી} & સિગ્નલ ગુણવત્તા જાળવે \\
\hline
\end{tabulary}
\end{answertable}

\begin{itemize}
    \item \textbf{રિફ્લેક્શન પ્રિવેન્શન}: સિગ્નલને વાપસ બાઉન્સ થવાથી રોકે
    \item \textbf{સિગ્નલ ક્વોલિટી}: સ્વચ્છ સિગ્નલ ટ્રાન્સમિશન જાળવે
    \item \textbf{બંને છેડે જરૂરી}: બસ ટોપોલોજીને કેબલના બંને છેડે ટર્મિનેટર જોઈએ
    \item \textbf{રેઝિસ્ટન્સ વેલ્યુ}: ઇથરનેટ નેટવર્ક્સ માટે સામાન્ય રીતે 50 ઓહ્મ
\end{itemize}

\begin{mnemonicbox}
\mnemonic{Terminator Stops signal Travel}
\end{mnemonicbox}
\end{solutionbox}

\questionmarks{5(ક)}{7}{CIA ટ્રાઇડ વર્ણવો.}

\begin{solutionbox}
\textbf{જવાબ:}

\begin{answerdiagram}{CIA ટ્રાઇડ}
\begin{tikzpicture}[gtu tree]
    \node [gtu root] {CIA ટ્રાઇડ}
        child { node [gtu child] {કોન્ફિડેન્શિયાલિટી} 
            child { node [gtu block] {એન્ક્રિપ્શન} }
            child { node [gtu block] {એક્સેસ કંટ્રોલ} }
        }
        child { node [gtu child] {ઇન્ટેગ્રિટી} 
            child { node [gtu block] {હેશ ફંક્શન્સ} }
            child { node [gtu block] {ડિજિટલ સિગ્નેચર્સ} }
        }
        child { node [gtu child] {અવેઇલેબિલિટી} 
            child { node [gtu block] {રિડન્ડન્સી} }
            child { node [gtu block] {બેકઅપ સિસ્ટમ્સ} }
        };
\end{tikzpicture}
\end{answerdiagram}

\begin{answertable}{CIA ટ્રાઇડ કમ્પોનન્ટ્સ}
\begin{tabulary}{\linewidth}{|L|L|L|L|}
\hline
\textbf{કમ્પોનન્ટ} & \textbf{વ્યાખ્યા} & \textbf{ઇમ્પ્લિમેન્ટેશન} & \textbf{જોખમો} \\
\hline
\textbf{કોન્ફિડેન્શિયાલિટી} & માહિતીની ગુપ્તતા & એન્ક્રિપ્શન, એક્સેસ કંટ્રોલ & અનધિકૃત ડિસક્લોઝર \\
\hline
\textbf{ઇન્ટેગ્રિટી} & ડેટાની ચોકસાઈ અને સંપૂર્ણતા & હેશ ફંક્શન્સ, ડિજિટલ સિગ્નેચર્સ & ડેટા મોડિફિકેશન \\
\hline
\textbf{અવેઇલેબિલિટી} & માહિતીની પહોંચ યોગ્યતા & રિડન્ડન્સી, બેકઅપ સિસ્ટમ્સ & સર્વિસ ડિસરપ્શન \\
\hline
\end{tabulary}
\end{answertable}

\textbf{કોન્ફિડેન્શિયાલિટી:}
\begin{itemize}
    \item \textbf{ડેટા પ્રોટેક્શન}: ફક્ત અધિકૃત યુઝર્સ જ માહિતી એક્સેસ કરી શકે
    \item \textbf{પ્રાઇવસી પગલાં}: એન્ક્રિપ્શન, ઓથેન્ટિકેશન, એક્સેસ કંટ્રોલ્સ
    \item \textbf{ઉદાહરણો}: પાસવર્ડ પ્રોટેક્શન, ફાઇલ પરમિશન્સ
\end{itemize}

\textbf{ઇન્ટેગ્રિટી:}
\begin{itemize}
    \item \textbf{ડેટા એક્યુરસી}: ટ્રાન્સમિશન/સ્ટોરેજ દરમિયાન માહિતી બદલાતી નથી
    \item \textbf{વેરિફિકેશન પદ્ધતિઓ}: ચેકસમ્સ, ડિજિટલ સિગ્નેચર્સ, વર્ઝन કંટ્રોલ
    \item \textbf{ઉદાહરણો}: હેશ ફંક્શન્સ, ડેટાબેસ કન્સ્ટ્રેઇન્ટ્સ
\end{itemize}

\textbf{અવેઇલેબિલિટી:}
\begin{itemize}
    \item \textbf{સિસ્ટમ એક્સેસિબિલિટી}: જરૂર પડે ત્યારે માહિતી અને સેવાઓ ઉપલબ્ધ
    \item \textbf{રિલાયબિલિટી પગલાં}: રિડન્ડન્સી, ફોલ્ટ ટોલરન્સ, ડિઝાસ્ટર રિકવરી
    \item \textbf{ઉદાહરણો}: લોડ બેલેન્સિંગ, બેકઅપ સિસ્ટમ્સ, UPS
\end{itemize}

\begin{mnemonicbox}
\mnemonic{CIA protects - Confidentiality, Integrity, Availability}
\end{mnemonicbox}
\end{solutionbox}

\questionmarks{5(અ OR)}{3}{વ્યાખ્યાયિત કરો: 1. ક્રિપ્ટોગ્રાફી 2. ડિક્રિપ્શન}

\begin{solutionbox}
\textbf{જવાબ:}

\begin{answertable}{વ્યાખ્યા કોષ્ટક}
\begin{tabulary}{\linewidth}{|L|L|L|}
\hline
\textbf{શબ્દ} & \textbf{વ્યાખ્યા} & \textbf{હેતુ} \\
\hline
\keyword{ક્રિપ્ટોગ્રાફી} & એન્કોડિંગ દ્વારા માહિતી સુરક્ષિત કરવાનું વિજ્ઞાન & ડેટા કોન્ફિડેન્શિયાલિટી સુરક્ષિત કરવી \\
\hline
\keyword{ડિક્રિપ્શન} & એન્ક્રિપ્ટેડ ડેટાને મૂળ સ્વરૂપમાં પાછું કન્વર્ટ કરવાની પ્રક્રિયા & મૂળ માહિતી પુનઃપ્રાપ્ત કરવી \\
\hline
\end{tabulary}
\end{answertable}

\begin{itemize}
    \item \textbf{ક્રિપ્ટોગ્રાફી}: વાંચી શકાય તેવા ડેટાને વાંચી ન શકાય તેવા ફોર્મેટમાં ટ્રાન્સફોર્મ કરવા માટે ગાણિતિક અલ્ગોરિધમ્સ ઉપયોગ કરે
    \item \textbf{ડિક્રિપ્શન}: કીઝ ઉપયોગ કરીને મૂળ ડેટા પુનઃસ્થાપિત કરવાની વિપરીત પ્રક્રિયા
    \item \textbf{કી-બેસ્ડ સિક્યોરિટી}: બંને પ્રક્રિયાઓ ક્રિપ્ટોગ્રાફિક કીઝ પર આધાર રાખે
\end{itemize}

\begin{mnemonicbox}
\mnemonic{Crypto Conceals, Decryption Discloses}
\end{mnemonicbox}
\end{solutionbox}

\questionmarks{5(બ OR)}{4}{i) ટ્વિસ્ટેડ પેર કેબલ્સમાં વાયરો શા માટે ટ્વિસ્ટેડ રાખવામાં આવે છે તેનું કારણ જણાવો. ii) OSI મોડેલના સ્તરને ઓળખો કે જેના પર નીચેના નેટવર્ક ઉપકરણો સપોર્ટ કરે છે: 1. રાઉટર 2. બ્રિજ}

\begin{solutionbox}
\textbf{જવાબ:}

\textbf{i) ટ્વિસ્ટેડ પેર કેબલ ડિઝાઇન:}

\begin{answertable}{વાયર ટ્વિસ્ટિંગ ફાયદાઓ}
\begin{tabulary}{\linewidth}{|L|L|}
\hline
\textbf{ફાયદો} & \textbf{વિવરણ} \\
\hline
\textbf{નોઇ ઝ રિડક્શન} & ઇલેક્ટ્રોમેગ્નેટિક ઇન્ટરફેરન્સ કેન્સલ કરે \\
\hline
\textbf{ક્રોસટોક પ્રિવેન્શન} & પેર્સ વચ્ચે સિગ્નલ ઇન્ટરફેરન્સ ઘટાડે \\
\hline
\textbf{સિગ્નલ ક્વોલિટી} & બેહતર સિગ્નલ ઇન્ટેગ્રિટી જાળવે \\
\hline
\end{tabulary}
\end{answertable}

\textbf{ii) OSI લેયર આઇડેન્ટિફિકેશન:}

\begin{answertable}{નેટવર્ક ડિવાઇસ અને OSI લેયર્સ}
\begin{tabulary}{\linewidth}{|L|L|L|}
\hline
\textbf{ડિવાઇસ} & \textbf{OSI લેયર} & \textbf{ફંક્શન} \\
\hline
\textbf{રાઉટર} & લેયર 3 (નેટવર્ક) & વિવિધ નેટવર્ક્સ વચ્ચે રાઉટિંગ \\
\hline
\textbf{બ્રિજ} & લેયર 2 (ડેટા લિંક) & નેટવર્ક સેગમેન્ટ્સ કનેક્ટ કરવા \\
\hline
\end{tabulary}
\end{answertable}

\begin{itemize}
    \item \textbf{વાયર ટ્વિસ્ટિંગ}: દરેક ટ્વિસ્ટ બાજુના વાયરમાંથી ઇલેક્ટ્રોમેગ્નેટિક ઇન્ટરફેરન્સ કેન્સલ કરે
    \item \textbf{ઇન્ટરફેરન્સ કેન્સલેશન}: નોઇઝ બંને વાયરને સમાન રીતે પરંતુ વિપરીત દિશામાં અસર કરે
    \item \textbf{રાઉટર ફંક્શન}: IP એડ્રેસના આધારે રાઉટિંગ નિર્ણયો લે
    \item \textbf{બ્રિજ ફંક્શન}: MAC એડ્રેસના આધારે ફ્રેમ્સ ફોરવર્ડ કરે
\end{itemize}

\begin{mnemonicbox}
\mnemonic{Twisted wires Reduce interference, Router at layer 3, Bridge at layer 2}
\end{mnemonicbox}
\end{solutionbox}

\questionmarks{5(ક OR)}{7}{સાયબર એટેકને વ્યાખ્યાયિત કરો અને વિવિધ સાયબર હુમલાઓને સંક્ષિપ્તમાં સમજાવો}

\begin{solutionbox}
\textbf{જવાબ:}

\textbf{સાયબર એટેક વ્યાખ્યા:}
સાયબર એટેક એ કમ્પ્યુટર સિસ્ટમ્સ, નેટવર્ક્સ અથવા ડિજિટલ ડિવાઇસને કમ્પ્રોમાઇઝ કરવાનો ઇરાદાપૂર્વકનો પ્રયાસ છે જેથી ડેટા ચોરી, બદલાવ અથવા નાશ કરી શકાય.

\begin{answerdiagram}{સાયબર એટેક પ્રકારો}
\begin{tikzpicture}[gtu tree]
    \node [gtu root] {સાયબર એટેક્સ}
        child { node [gtu child] {મેલવેર} 
            child { node [gtu block, fill=red!10] {વાયરસ, વોર્મ, ટ્રોજન} }
        }
        child { node [gtu child] {ફિશિંગ} 
            child { node [gtu block, fill=orange!10] {ઇમેઇલ, વેબસાઇટ} }
        }
        child { node [gtu child] {DoS/DDoS} 
            child { node [gtu block, fill=yellow!10] {ટ્રાફિક ફ્લડિંગ} }
        }
        child { node [gtu child] {મેન-ઇન-મિડલ} 
            child { node [gtu block, fill=blue!10] {ઇવ્સડ્રોપિંગ} }
        }
        child { node [gtu child] {SQL ઇન્જેક્શન} 
            child { node [gtu block, fill=green!10] {ડેટાબેસ એટેક} }
        };
\end{tikzpicture}
\end{answerdiagram}

\begin{answertable}{સાયબર એટેક પ્રકારો}
\begin{tabulary}{\linewidth}{|L|L|L|L|}
\hline
\textbf{હુમલાનો પ્રકાર} & \textbf{વિવરણ} & \textbf{અસર} & \textbf{પ્રિવેન્શન} \\
\hline
\textbf{મેલવેર} & દુર્ભાવનાપૂર્ણ સોફ્ટવેર (વાયરસ, વોર્મ, ટ્રોજન) & સિસ્ટમ કરપ્શન, ડેટા ચોરી & એન્ટીવાયરસ, અપડેટ્સ \\
\hline
\textbf{ફિશિંગ} & ક્રેડેન્શિયલ્સ ચોરવા માટે ફ્રોડ ઇમેઇલ્સ/વેબસાઇટ્સ & આઇડેન્ટિટી થેફ્ટ, ફાઇનાન્શિયલ લોસ & યુઝર જાગૃતિ, ઇમેઇલ ફિલ્ટર્સ \\
\hline
\textbf{DoS/DDoS} & ટાર્ગેટને ટ્રાફિક સાથે ઓવરવ્હેલ્મ કરવું & સર્વિસ અનઉપલબ્ધતા & ફાયરવોલ્સ, લોડ બેલેન્સર્સ \\
\hline
\textbf{મેન-ઇન-મિડલ} & પક્ષો વચ્ચે કમ્યુનિકેશન ઇન્ટરસેપ્ટ કરવું & ડેટા ઇવ્સડ્રોપિંગ & એન્ક્રિપ્શન, સિક્યોર પ્રોટોકોલ્સ \\
\hline
\textbf{SQL ઇન્જેક્શન} & ડેટાબેસ ક્વેરીમાં દુર્ભાવનાપૂર્ણ કોડ દાખલ કરવો & ડેટાબેસ કમ્પ્રોમાઇઝ & ઇનપુટ વેલિડેશન, પેરામીટરાઇઝ્ડ ક્વેરીઝ \\
\hline
\end{tabulary}
\end{answertable}

\textbf{મેલવેર એટેક્સ:}
\begin{itemize}
    \item \textbf{વાયરસ}: ફાઇલોમાં જોડાતો સ્વ-પ્રતિકૃતિ કોડ
    \item \textbf{વોર્મ}: નેટવર્ક્સમાં ફેલાતો સ્ટેન્ડઅલોન મેલવેર
    \item \textbf{ટ્રોજન}: કાયદેસર દેખાતો છુપાયેલો મેલવેર
\end{itemize}

\textbf{સોશિયલ એન્જિનીયરિંગ:}
\begin{itemize}
    \item \textbf{ફિશિંગ}: સંવેદનશીલ માહિતી માંગતી નકલી ઇમેઇલ્સ
    \item \textbf{સ્પીયર ફિશિંગ}: ચોક્કસ વ્યક્તિઓ પર ટાર્ગેટેડ હુમલાઓ
    \item \textbf{બેઇટિંગ}: મેલવેર પહોંચાડવા માટે આકર્ષક ઓફર્સનો ઉપયોગ
\end{itemize}

\textbf{નેટવર્ક એટેક્સ:}
\begin{itemize}
    \item \textbf{પેકેટ સ્નિફિંગ}: વિશ્લેષણ માટે નેટવર્ક ટ્રાફિક કેપ્ચર કરવું
    \item \textbf{સેશન હાઇજેકિંગ}: યુઝર સેશન્સ કબજે કરવા
    \item \textbf{પાસવર્ડ એટેક્સ}: બ્રુટ ફોર્સ, ડિક્શનરી એટેક્સ
\end{itemize}

\begin{mnemonicbox}
\mnemonic{MPDMS - Malware, Phishing, DoS, Man-in-middle, SQL injection}
\end{mnemonicbox}
\end{solutionbox}

\end{document}
