\documentclass{article}

% content/resources/templates/preamble.tex
\usepackage[margin=0.6in]{geometry}
\author{Milav Dabgar}
\usepackage{amsmath,amssymb,amsthm}
\usepackage{booktabs}
\usepackage{multirow}
\usepackage{xcolor}
\usepackage{tcolorbox}
\tcbuselibrary{breakable,skins}
\usepackage[colorlinks=true,linkcolor=blue]{hyperref}
\usepackage{titlesec}
\usepackage{enumitem}
\usepackage{tikz}
\usepackage{pgfplots}
\usepackage{circuitikz}
\usepackage[version=4]{mhchem}
\usepackage{longtable}
\usepackage{array}
\usepackage{float}
\usepackage{caption}
\usepackage{listings}

\lstset{
  basicstyle=\small\ttfamily,
  breaklines=true,
  breakatwhitespace=false,
  postbreak=\mbox{\textcolor{red}{$\hookrightarrow$}\space},
  float=false,
  numbers=left,
  numberstyle=\tiny\color{gray},
  numbersep=10pt,
  xleftmargin=2em,
  keywordstyle=\color{blue},
  commentstyle=\color{green!60!black},
  stringstyle=\color{purple},
  backgroundcolor=\color{gray!5},
  showstringspaces=false,
  tabsize=2,
  captionpos=b,
  keepspaces=true,
  columns=flexible
}

\pgfplotsset{compat=1.18}
\usetikzlibrary{shapes,arrows,positioning,calc,patterns,decorations.pathmorphing,decorations.markings,arrows.meta}

% Color scheme
\definecolor{headcolor}{RGB}{0,102,204}
\definecolor{keycolor}{RGB}{220,20,60}
\definecolor{solutioncolor}{RGB}{34,139,34}
\definecolor{mnemoniccolor}{RGB}{148,0,211}
\definecolor{codecolor}{RGB}{0,0,100}

% Spacing
\setlength{\parskip}{3pt}
\setlist[itemize]{nosep}
\setlist[enumerate]{nosep}

% Title formatting
\titleformat{\section}{\Large\bfseries\color{headcolor}}{\thesection}{1em}{}
\titleformat{\subsection}{\large\bfseries\color{headcolor}}{\thesubsection}{1em}{}

% Pandoc tightlist compatibility
\providecommand{\tightlist}{%
  \setlength{\itemsep}{0pt}\setlength{\parskip}{0pt}}

% Pandoc longtable compatibility
\newcounter{none}
\def\thenone{}


% content/resources/templates/english-boxes.tex

% Custom environments
\newtcolorbox{solutionbox}{
 breakable,
 enhanced,
 colback=solutioncolor!5!white,
 colframe=solutioncolor!75!black,
 fonttitle=\bfseries,
 title=Solution
}

\newtcolorbox{solutionboxnobreak}{
 colback=solutioncolor!5!white,
 colframe=solutioncolor!75!black,
 fonttitle=\bfseries,
 title=Solution
}

\newtcolorbox{keyformula}{
 breakable,
 enhanced,
 colback=keycolor!5!white,
 colframe=keycolor!75!black,
 fonttitle=\bfseries,
 title=Key Formula
}

\newtcolorbox{mnemonicboxenv}{
 breakable,
 enhanced,
 colback=mnemoniccolor!5!white,
 colframe=mnemoniccolor!75!black,
 fonttitle=\bfseries,
 title=Mnemonic
}

\newcommand{\mnemonicbox}[1]{%
  \begin{mnemonicboxenv}
    #1
  \end{mnemonicboxenv}
}


% Custom commands for GTU solutions
% This file defines semantic commands for consistent formatting

% Question command with automatic formatting
\newcommand{\question}[2]{%
  \section*{Question #1}%
  \textbf{#2}%
}

% OR question variant
\newcommand{\questionor}[2]{%
  \section*{Question #1 OR}%
  \textbf{#2}%
}

% Proper table environment with caption
\newenvironment{answertable}[1]{%
  \begin{table}[htbp]
  \centering
  \caption{#1}
}{%
  \end{table}
}

% Proper figure environment for diagrams
\newenvironment{answerdiagram}[1]{%
  \begin{figure}[htbp]
  \centering
  \caption{#1}
}{%
  \end{figure}
}

% Semantic markup for key terms
\newcommand{\keyword}[1]{\textbf{#1}}
\newcommand{\code}[1]{\texttt{#1}}
\newcommand{\classname}[1]{\texttt{#1}}
\newcommand{\methodname}[1]{\texttt{#1}}

% Proper quotation marks
\newcommand{\mnemonic}[1]{``#1''}


\title{Introduction to IT Systems (4311602) - Summer 2023 Solution}
\date{August 7, 2023}

\begin{document}
\maketitle

\questionmarks{Question 1(a)}{03}{Discuss the main components of the Computer.}

\begin{solutionbox}
\textbf{Answer:}
\begin{answertable}{Main Components of Computer}
\begin{tabular}{|l|l|l|}
\hline
\textbf{Component} & \textbf{Function} & \textbf{Example} \\
\hline
\keyword{Input Unit} & Receives data and instructions & Keyboard, Mouse \\
\hline
\keyword{CPU} & Processes data and controls operations & Intel i5, AMD Ryzen \\
\hline
\keyword{Memory} & Stores data temporarily/permanently & RAM, Hard Disk \\
\hline
\keyword{Output Unit} & Displays processed results & Monitor, Printer \\
\hline
\end{tabular}
\end{answertable}

\textbf{Key Components:}
\begin{itemize}
    \item \textbf{Hardware}: Physical parts like CPU, RAM, motherboard
    \item \textbf{Software}: Programs and operating systems
    \item \textbf{Data}: Information processed by computer
\end{itemize}

\begin{mnemonicbox}
\mnemonic{I Can Make Output (Input-CPU-Memory-Output)}
\end{mnemonicbox}
\end{solutionbox}

\questionmarks{Question 1(b)}{04}{Explain the web browser and its type.}

\begin{solutionbox}
\textbf{Answer:}
A \textbf{web browser} is software that accesses and displays web pages from the internet.
\begin{answertable}{Types of Web Browsers}
\begin{tabular}{|l|l|l|}
\hline
\textbf{Browser Type} & \textbf{Features} & \textbf{Examples} \\
\hline
\keyword{Graphical} & GUI interface, multimedia support & Chrome, Firefox \\
\hline
\keyword{Text-based} & Command line, fast loading & Lynx, Links \\
\hline
\keyword{Mobile} & Touch interface, optimized for phones & Safari Mobile, Chrome Mobile \\
\hline
\end{tabular}
\end{answertable}

\textbf{Features:}
\begin{itemize}
    \item \textbf{Navigation}: Forward, back, refresh buttons
    \item \textbf{Bookmarks}: Save favorite websites
    \item \textbf{Tabs}: Multiple pages in one window
    \item \textbf{Security}: HTTPS support, popup blockers
\end{itemize}

\begin{mnemonicbox}
\mnemonic{Browse Safely Online (Bookmarks-Security-Online)}
\end{mnemonicbox}
\end{solutionbox}

\questionmarks{Question 1(c)}{07}{Explain LAN, MAN and WAN with example.}

\begin{solutionbox}
\textbf{Answer:}
\begin{answertable}{Network Types Comparison}
\begin{tabular}{|l|l|l|l|l|}
\hline
\textbf{Network} & \textbf{Coverage} & \textbf{Speed} & \textbf{Example} & \textbf{Cost} \\
\hline
\keyword{LAN} & Building/Campus & High (100Mbps-1Gbps) & Office network & Low \\
\hline
\keyword{MAN} & City/Metropolitan & Medium (10-100Mbps) & Cable TV network & Medium \\
\hline
\keyword{WAN} & Country/Global & Variable (1-100Mbps) & Internet & High \\
\hline
\end{tabular}
\end{answertable}

\textbf{Detailed Explanation:}
\begin{itemize}
    \item \textbf{LAN (Local Area Network)}:
    \begin{itemize}
        \item \textbf{Coverage}: Within building or small area
        \item \textbf{Technology}: Ethernet, Wi-Fi
        \item \textbf{Example}: Computer lab, home network
    \end{itemize}
    \item \textbf{MAN (Metropolitan Area Network)}:
    \begin{itemize}
        \item \textbf{Coverage}: Across city or metropolitan area
        \item \textbf{Technology}: Fiber optic, microwave
        \item \textbf{Example}: City-wide cable internet
    \end{itemize}
    \item \textbf{WAN (Wide Area Network)}:
    \begin{itemize}
        \item \textbf{Coverage}: Multiple cities/countries
        \item \textbf{Technology}: Satellite, fiber optic
        \item \textbf{Example}: Internet, bank ATM networks
    \end{itemize}
\end{itemize}

\textbf{Diagram:}
\begin{center}
\begin{tikzpicture}[gtu flow]
    \node[gtu block] (lan) {LAN (Building)};
    \node[gtu block, right=of lan] (man) {MAN (City)};
    \node[gtu block, right=of man] (wan) {WAN (Global)};
    
    \node[gtu process, below=of lan] (ex1) {Office Network};
    \node[gtu process, below=of man] (ex2) {City Cable TV};
    \node[gtu process, below=of wan] (ex3) {Internet};

    \path [gtu arrow] (lan) -- (man);
    \path [gtu arrow] (man) -- (wan);
    \path [gtu arrow] (lan) -- (ex1);
    \path [gtu arrow] (man) -- (ex2);
    \path [gtu arrow] (wan) -- (ex3);
\end{tikzpicture}
\end{center}

\begin{mnemonicbox}
\mnemonic{Local Metro World (LAN-MAN-WAN)}
\end{mnemonicbox}
\end{solutionbox}

\questionmarks{Question 1(c OR)}{07}{Difference between DOS and Unix Operating system.}

\begin{solutionbox}
\textbf{Answer:}
\begin{answertable}{DOS vs Unix Comparison}
\begin{tabular}{|l|l|l|}
\hline
\textbf{Feature} & \textbf{DOS} & \textbf{Unix} \\
\hline
\keyword{Interface} & Command Line (text-based) & Command Line + GUI \\
\hline
\keyword{Multi-user} & Single user & Multi-user support \\
\hline
\keyword{Multitasking} & Limited & Full multitasking \\
\hline
\keyword{Security} & Basic & Advanced security \\
\hline
\keyword{File System} & FAT16/FAT32 & Various (ext3, ext4) \\
\hline
\keyword{Cost} & Commercial (Microsoft) & Free/Open source variants \\
\hline
\end{tabular}
\end{answertable}

\textbf{Key Differences:}
\begin{itemize}
    \item \textbf{DOS (Disk Operating System)}:
    \begin{itemize}
        \item \textbf{Architecture}: 16-bit, single-user
        \item \textbf{Memory}: Limited to 640KB conventional
        \item \textbf{Commands}: DIR, COPY, DEL
        \item \textbf{File naming}: 8.3 format limitation
    \end{itemize}
    \item \textbf{Unix}:
    \begin{itemize}
        \item \textbf{Architecture}: 32/64-bit, multi-user
        \item \textbf{Memory}: Advanced memory management
        \item \textbf{Commands}: ls, cp, rm, grep
        \item \textbf{File naming}: Case-sensitive, long names
    \end{itemize}
\end{itemize}

\textbf{Examples:}
\begin{itemize}
    \item \textbf{DOS}: MS-DOS, PC-DOS
    \item \textbf{Unix}: Linux, Solaris, AIX
\end{itemize}

\begin{mnemonicbox}
\mnemonic{DOS Simple, Unix Powerful (Single vs Multi-user)}
\end{mnemonicbox}
\end{solutionbox}

\questionmarks{Question 2(a)}{03}{List out features of operating system.}

\begin{solutionbox}
\textbf{Answer:}
\begin{answertable}{Operating System Features}
\begin{tabular}{|l|l|}
\hline
\textbf{Feature} & \textbf{Description} \\
\hline
\keyword{Process Management} & Controls program execution \\
\hline
\keyword{Memory Management} & Allocates RAM efficiently \\
\hline
\keyword{File Management} & Organizes data storage \\
\hline
\keyword{Device Management} & Controls hardware devices \\
\hline
\end{tabular}
\end{answertable}

\textbf{Core Features:}
\begin{itemize}
    \item \textbf{User Interface}: GUI or command line
    \item \textbf{Security}: User authentication, access control
    \item \textbf{Multitasking}: Run multiple programs simultaneously
    \item \textbf{Resource Allocation}: CPU, memory distribution
\end{itemize}

\begin{mnemonicbox}
\mnemonic{Please Manage Files Properly (Process-Memory-File-Device)}
\end{mnemonicbox}
\end{solutionbox}

\questionmarks{Question 2(b)}{04}{Define half duplex and full duplex transmission modes.}

\begin{solutionbox}
\textbf{Answer:}
\begin{answertable}{Transmission Modes Comparison}
\begin{tabular}{|l|l|l|l|}
\hline
\textbf{Mode} & \textbf{Direction} & \textbf{Example} & \textbf{Efficiency} \\
\hline
\keyword{Half Duplex} & Bidirectional (one at a time) & Walkie-talkie & Medium \\
\hline
\keyword{Full Duplex} & Bidirectional (simultaneous) & Telephone & High \\
\hline
\end{tabular}
\end{answertable}

\textbf{Definitions:}
\begin{itemize}
    \item \textbf{Half Duplex}:
    \begin{itemize}
        \item \textbf{Communication}: Two-way but not simultaneous
        \item \textbf{Example}: Radio communication, old Ethernet hubs
        \item \textbf{Limitation}: Turn-taking required
    \end{itemize}
    \item \textbf{Full Duplex}:
    \begin{itemize}
        \item \textbf{Communication}: Two-way simultaneous
        \item \textbf{Example}: Modern Ethernet, telephone calls
        \item \textbf{Advantage}: No waiting time
    \end{itemize}
\end{itemize}

\textbf{Diagram:}
\begin{center}
\begin{tikzpicture}[gtu flow]
    \node[gtu block] (a1) {A};
    \node[gtu block, right=of a1, xshift=2cm] (b1) {B};
    \draw[<->, dashed] (a1) -- node[above] {One at a time} (b1);
    \node[below=of a1, xshift=2cm] {Half Duplex};

    \node[gtu block, below=of a1, yshift=-1cm] (a2) {A};
    \node[gtu block, right=of a2, xshift=2cm] (b2) {B};
    \draw[transform canvas={yshift=0.7ex}, ->] (a2) -- (b2);
    \draw[transform canvas={yshift=-0.7ex}, <-] (a2) -- node[below] {Simultaneous} (b2);
    \node[below=of a2, xshift=2cm] {Full Duplex};
\end{tikzpicture}
\end{center}

\begin{mnemonicbox}
\mnemonic{Half waits, Full flows (Half=waiting, Full=simultaneous)}
\end{mnemonicbox}
\end{solutionbox}

\questionmarks{Question 2(c)}{07}{Difference between open source and proprietary software.}

\begin{solutionbox}
\textbf{Answer:}
\begin{answertable}{Open Source vs Proprietary Software}
\begin{tabular}{|l|l|l|}
\hline
\textbf{Aspect} & \textbf{Open Source} & \textbf{Proprietary} \\
\hline
\keyword{Source Code} & Freely available & Hidden/Protected \\
\hline
\keyword{Cost} & Usually free & Paid licenses \\
\hline
\keyword{Modification} & Allowed & Restricted \\
\hline
\keyword{Support} & Community-based & Vendor support \\
\hline
\keyword{Security} & Transparent & Security through obscurity \\
\hline
\keyword{Examples} & Linux, Firefox, Apache & Windows, MS Office \\
\hline
\end{tabular}
\end{answertable}

\textbf{Detailed Comparison:}
\begin{itemize}
    \item \textbf{Open Source Software}:
    \begin{itemize}
        \item \textbf{Definition}: Source code publicly available
        \item \textbf{Licensing}: GPL, MIT, Apache licenses
        \item \textbf{Benefits}: Cost-effective, customizable, transparent
        \item \textbf{Examples}: LibreOffice, GIMP, MySQL
    \end{itemize}
    \item \textbf{Proprietary Software}:
    \begin{itemize}
        \item \textbf{Definition}: Owned by individual/company
        \item \textbf{Licensing}: End User License Agreement (EULA)
        \item \textbf{Benefits}: Professional support, guaranteed updates
        \item \textbf{Examples}: Adobe Photoshop, Oracle Database
    \end{itemize}
\end{itemize}

\textbf{Advantages \& Disadvantages:}
\begin{itemize}
    \item \textbf{Open Source Pros}: Free, flexible, community support
    \item \textbf{Open Source Cons}: Limited professional support
    \item \textbf{Proprietary Pros}: Professional support, warranty
    \item \textbf{Proprietary Cons}: Expensive, vendor lock-in
\end{itemize}

\begin{mnemonicbox}
\mnemonic{Open = Free to See, Proprietary = Pay to Use}
\end{mnemonicbox}
\end{solutionbox}

\questionmarks{Question 2(a OR)}{03}{Differentiate between RAM and ROM.}

\begin{solutionbox}
\textbf{Answer:}
\begin{answertable}{RAM vs ROM Comparison}
\begin{tabular}{|l|l|l|}
\hline
\textbf{Feature} & \textbf{RAM} & \textbf{ROM} \\
\hline
\keyword{Full Form} & Random Access Memory & Read Only Memory \\
\hline
\keyword{Volatility} & Volatile (loses data) & Non-volatile (retains data) \\
\hline
\keyword{Access} & Read/Write & Read only \\
\hline
\keyword{Speed} & Very fast & Slower than RAM \\
\hline
\end{tabular}
\end{answertable}

\textbf{Key Differences:}
\begin{itemize}
    \item \textbf{Purpose}: RAM for temporary storage, ROM for permanent
    \item \textbf{Cost}: RAM more expensive per GB
    \item \textbf{Usage}: RAM for programs, ROM for firmware
\end{itemize}

\begin{mnemonicbox}
\mnemonic{RAM Runs, ROM Remembers (temporary vs permanent)}
\end{mnemonicbox}
\end{solutionbox}

\questionmarks{Question 2(b OR)}{04}{Explain AND logic gate with Example.}

\begin{solutionbox}
\textbf{Answer:}
\textbf{AND Gate Definition:} Output is HIGH only when ALL inputs are HIGH.

\textbf{Truth Table:}
\begin{answertable}{AND Gate Truth Table}
\begin{tabular}{|c|c|c|}
\hline
\textbf{Input A} & \textbf{Input B} & \textbf{Output (A AND B)} \\
\hline
0 & 0 & 0 \\
\hline
0 & 1 & 0 \\
\hline
1 & 0 & 0 \\
\hline
1 & 1 & 1 \\
\hline
\end{tabular}
\end{answertable}

\textbf{Symbol:}
\begin{center}
\begin{tikzpicture}
    % Logic Gate Symbol
    \node (A) at (0, 0.5) {A};
    \node (B) at (0, -0.5) {B};
    \draw (1,0.5) -- (2,0.5);
    \draw (1,-0.5) -- (2,-0.5);
    \draw (2,-1) -- (2,1) -- (3,1) arc (90:-90:1) -- (2,-1);
    \draw (4,0) -- (5,0) node[right] {Output};
\end{tikzpicture}
\end{center}

\textbf{Example Applications:}
\begin{itemize}
    \item \textbf{Security System}: Door opens only with key AND card
    \item \textbf{Car Starting}: Engine starts with key AND foot on brake
    \item \textbf{Boolean Expression}: $Y = A \cdot B$ or $Y = A \land B$
\end{itemize}

\textbf{Real-life Example:} Washing machine starts only when door is closed AND power button is pressed.

\begin{mnemonicbox}
\mnemonic{ALL inputs True = Output True}
\end{mnemonicbox}
\end{solutionbox}

\questionmarks{Question 2(c OR)}{07}{Explain the Ethernet Cable Color code.}

\begin{solutionbox}
\textbf{Answer:}
\textbf{Standard: TIA/EIA-568B Color Code}

\textbf{Table: Wire Color Sequence}
\begin{answertable}{Ethernet Pinout (568B)}
\begin{tabular}{|l|l|l|}
\hline
\textbf{Pin} & \textbf{Color} & \textbf{Function} \\
\hline
1 & White/Orange & Transmit+ \\
\hline
2 & Orange & Transmit- \\
\hline
3 & White/Green & Receive+ \\
\hline
4 & Blue & Not used \\
\hline
5 & White/Blue & Not used \\
\hline
6 & Green & Receive- \\
\hline
7 & White/Brown & Not used \\
\hline
8 & Brown & Not used \\
\hline
\end{tabular}
\end{answertable}

\textbf{Cable Types:}
\begin{itemize}
    \item \textbf{Straight-Through Cable (568B both ends)}:
    \begin{itemize}
        \item \textbf{Use}: Computer to switch/hub
        \item \textbf{Color sequence}: Same on both ends
    \end{itemize}
    \item \textbf{Cross-Over Cable (568A one end, 568B other)}:
    \begin{itemize}
        \item \textbf{Use}: Computer to computer direct
        \item \textbf{Pins swapped}: $1 \leftrightarrow 3$, $2 \leftrightarrow 6$
    \end{itemize}
\end{itemize}

\textbf{Preparation Steps:}
\begin{enumerate}
    \item Strip outer jacket (1 inch)
    \item Arrange wires in color order
    \item Cut wires evenly
    \item Insert into RJ-45 connector
    \item Crimp with crimping tool
\end{enumerate}

\begin{mnemonicbox}
\mnemonic{White Orange, Orange, White Green, Blue, White Blue, Green, White Brown, Brown}
\end{mnemonicbox}
\end{solutionbox}

\questionmarks{Question 3(a)}{03}{Compare wired and Wireless Communication.}

\begin{solutionbox}
\textbf{Answer:}
\begin{answertable}{Wired vs Wireless Communication}
\begin{tabular}{|l|l|l|}
\hline
\textbf{Aspect} & \textbf{Wired} & \textbf{Wireless} \\
\hline
\keyword{Medium} & Cables (copper/fiber) & Radio waves/infrared \\
\hline
\keyword{Speed} & Higher (up to 100Gbps) & Lower (up to 1Gbps) \\
\hline
\keyword{Security} & More secure & Less secure \\
\hline
\keyword{Mobility} & Limited & High mobility \\
\hline
\keyword{Cost} & Higher installation & Lower installation \\
\hline
\keyword{Interference} & Minimal & Signal interference \\
\hline
\end{tabular}
\end{answertable}

\textbf{Key Points:}
\begin{itemize}
    \item \textbf{Wired}: Reliable, fast, secure but limited mobility
    \item \textbf{Wireless}: Mobile, flexible but security concerns
\end{itemize}

\begin{mnemonicbox}
\mnemonic{Wires are Fast, Wireless is Free (speed vs mobility)}
\end{mnemonicbox}
\end{solutionbox}

\questionmarks{Question 3(b)}{04}{Discuss the different types of computer systems.}

\begin{solutionbox}
\textbf{Answer:}
\begin{answertable}{Computer System Types}
\begin{tabular}{|l|l|l|l|}
\hline
\textbf{Type} & \textbf{Size} & \textbf{Processing Power} & \textbf{Example} \\
\hline
\keyword{Supercomputer} & Room-sized & Extremely high & Weather forecasting \\
\hline
\keyword{Mainframe} & Large cabinet & Very high & Bank transactions \\
\hline
\keyword{Minicomputer} & Desk-sized & Medium & Small business \\
\hline
\keyword{Microcomputer} & Desktop/laptop & Low to medium & Personal use \\
\hline
\end{tabular}
\end{answertable}

\textbf{Classifications:}
\begin{itemize}
    \item \textbf{By Size \& Power}:
    \begin{itemize}
        \item \textbf{Supercomputer}: Scientific calculations, research
        \item \textbf{Mainframe}: Large organizations, concurrent users
        \item \textbf{Personal Computer}: Individual users, office work
        \item \textbf{Embedded Systems}: Specific functions (washing machines)
    \end{itemize}
    \item \textbf{By Purpose}:
    \begin{itemize}
        \item \textbf{General Purpose}: Versatile, multiple applications
        \item \textbf{Special Purpose}: Dedicated tasks (ATM, gaming console)
    \end{itemize}
\end{itemize}

\begin{mnemonicbox}
\mnemonic{Super Main Mini Micro (decreasing size order)}
\end{mnemonicbox}
\end{solutionbox}

\questionmarks{Question 3(c)}{07}{Write short note on TDM, FDM, and OFDM.}

\begin{solutionbox}
\textbf{Answer:}
\textbf{Multiplexing Techniques for Efficient Communication}
\begin{answertable}{Multiplexing Comparison}
\begin{tabular}{|l|l|l|l|}
\hline
\textbf{Technique} & \textbf{Division Method} & \textbf{Application} & \textbf{Advantage} \\
\hline
\keyword{TDM} & Time slots & Digital telephony & Simple implementation \\
\hline
\keyword{FDM} & Frequency bands & Radio/TV broadcasting & Simultaneous transmission \\
\hline
\keyword{OFDM} & Multiple carriers & Wi-Fi, 4G/5G & High data rates \\
\hline
\end{tabular}
\end{answertable}

\textbf{Descriptions:}
\begin{itemize}
    \item \textbf{Time Division Multiplexing (TDM)}:
    \begin{itemize}
        \item \textbf{Principle}: Each user gets fixed time slot
        \item \textbf{Example}: Digital telephone systems, GSM
        \item \textbf{Advantage}: Efficient use of bandwidth
    \end{itemize}
    \item \textbf{Frequency Division Multiplexing (FDM)}:
    \begin{itemize}
        \item \textbf{Principle}: Each user gets unique frequency band
        \item \textbf{Example}: FM radio, cable TV
    \end{itemize}
    \item \textbf{Orthogonal Frequency Division Multiplexing (OFDM)}:
    \begin{itemize}
        \item \textbf{Principle}: Multiple orthogonal subcarriers
        \item \textbf{Example}: Wi-Fi (802.11), LTE, DSL
    \end{itemize}
\end{itemize}

\textbf{Diagram:}
\begin{center}
\begin{tikzpicture}[gtu flow]
    \node[gtu block] (data) {Data Stream};
    \node[gtu process, below left=of data] (tdm) {TDM (Time Slots)};
    \node[gtu process, below=of data] (fdm) {FDM (Freq Bands)};
    \node[gtu process, below right=of data] (ofdm) {OFDM (Carriers)};
    
    \draw[gtu arrow] (data) -- (tdm);
    \draw[gtu arrow] (data) -- (fdm);
    \draw[gtu arrow] (data) -- (ofdm);
\end{tikzpicture}
\end{center}

\begin{mnemonicbox}
\mnemonic{Time Frequency Orthogonal (TDM-FDM-OFDM)}
\end{mnemonicbox}
\end{solutionbox}

\questionmarks{Question 3(a OR)}{03}{Discuss FSK and PSK.}

\begin{solutionbox}
\textbf{Answer:}
\begin{answertable}{FSK vs PSK}
\begin{tabular}{|l|l|l|}
\hline
\textbf{Aspect} & \textbf{FSK} & \textbf{PSK} \\
\hline
\keyword{Parameter} & Frequency & Phase \\
\hline
\keyword{Complexity} & Simple & Complex \\
\hline
\keyword{Noise Immunity} & Good & Excellent \\
\hline
\keyword{Bandwidth} & Higher & Lower \\
\hline
\end{tabular}
\end{answertable}

\textbf{Digital Modulation Techniques:}
\begin{itemize}
    \item \textbf{FSK (Frequency Shift Keying)}: Different frequencies for 0 and 1 (f1 for '0', f2 for '1'). Example: Computer modems.
    \item \textbf{PSK (Phase Shift Keying)}: Phase changes represent data ($0^\circ$ for '0', $180^\circ$ for '1'). Example: Wi-Fi.
\end{itemize}

\begin{mnemonicbox}
\mnemonic{Frequency Shifts, Phase Shifts (FSK-PSK)}
\end{mnemonicbox}
\end{solutionbox}

\questionmarks{Question 3(b OR)}{04}{Differentiate between Multitasking and Multi programming OS.}

\begin{solutionbox}
\textbf{Answer:}
\begin{answertable}{Multitasking vs Multiprogramming}
\begin{tabular}{|l|l|l|}
\hline
\textbf{Feature} & \textbf{Multitasking} & \textbf{Multiprogramming} \\
\hline
\keyword{User Interaction} & Interactive & Batch processing \\
\hline
\keyword{Response Time} & Fast & Slower \\
\hline
\keyword{CPU Sharing} & Time slicing & Job switching \\
\hline
\keyword{Example} & Windows, Linux & Early mainframes \\
\hline
\end{tabular}
\end{answertable}

\textbf{Comparison:}
\begin{itemize}
    \item \textbf{Multitasking}: Multiple tasks run seemingly simultaneously; interactive user experience.
    \item \textbf{Multiprogramming}: Multiple programs in memory; switch CPU only on I/O wait; for CPU utilization.
\end{itemize}

\begin{mnemonicbox}
\mnemonic{Tasks are Interactive, Programs are Batched}
\end{mnemonicbox}
\end{solutionbox}

\questionmarks{Question 3(c OR)}{07}{Write short note on network topologies.}

\begin{solutionbox}
\textbf{Answer:}
\begin{answertable}{Topology Comparison}
\begin{tabular}{|l|l|l|l|}
\hline
\textbf{Topology} & \textbf{Structure} & \textbf{Advantages} & \textbf{Disadvantages} \\
\hline
\keyword{Bus} & Linear & Simple, cost-effective & Single point failure \\
\hline
\keyword{Star} & Central hub & Easy troubleshooting & Hub failure affects all \\
\hline
\keyword{Ring} & Circular & Equal access & Break affects network \\
\hline
\keyword{Mesh} & Interconnected & High reliability & Complex, expensive \\
\hline
\keyword{Hybrid} & Mixed & Flexible & Complex management \\
\hline
\end{tabular}
\end{answertable}

\textbf{Diagram:}
\begin{center}
\begin{tikzpicture}[gtu flow]
    \node[gtu block] (root) {Network Topologies};
    \node[gtu process, below left=of root, xshift=-1cm] (bus) {Bus (Linear)};
    \node[gtu process, below left=of root, xshift=2cm] (star) {Star (Hub)};
    \node[gtu process, below=of root] (ring) {Ring (Circle)};
    \node[gtu process, below right=of root, xshift=-2cm] (mesh) {Mesh (All)};
    \node[gtu process, below right=of root, xshift=1cm] (hybrid) {Hybrid (Mix)};

    \draw[gtu arrow] (root) -- (bus);
    \draw[gtu arrow] (root) -- (star);
    \draw[gtu arrow] (root) -- (ring);
    \draw[gtu arrow] (root) -- (mesh);
    \draw[gtu arrow] (root) -- (hybrid);
\end{tikzpicture}
\end{center}

\begin{mnemonicbox}
\mnemonic{Bus Star Ring Mesh Hybrid (increasing complexity)}
\end{mnemonicbox}
\end{solutionbox}

\questionmarks{Question 4(a)}{03}{Explain Switch.}

\begin{solutionbox}
\textbf{Answer:}
\textbf{Network Switch}: Connects devices in a LAN at Data Link Layer (Layer 2).
\begin{answertable}{Switch Characteristics}
\begin{tabular}{|l|l|}
\hline
\textbf{Feature} & \textbf{Description} \\
\hline
\keyword{Function} & Connects devices in LAN \\
\hline
\keyword{Method} & MAC address learning \\
\hline
\keyword{Collision} & Eliminates collisions \\
\hline
\keyword{Bandwidth} & Dedicated per port \\
\hline
\end{tabular}
\end{answertable}

\textbf{Functions}:
\begin{itemize}
    \item \textbf{Frame Forwarding}: Sends data to specific port
    \item \textbf{Address Learning}: Builds MAC address table
    \item \textbf{Loop Prevention}: Spanning Tree Protocol
\end{itemize}

\begin{mnemonicbox}
\mnemonic{Switch Learns MAC Addresses}
\end{mnemonicbox}
\end{solutionbox}

\questionmarks{Question 4(b)}{04}{Define Cyberthreat with an example.}

\begin{solutionbox}
\textbf{Answer:}
\textbf{Cyberthreat}: Malicious attempt to damage, disrupt, or gain unauthorized access to computer systems.
\begin{answertable}{Cyberthreat Types}
\begin{tabular}{|l|l|l|}
\hline
\textbf{Type} & \textbf{Method} & \textbf{Example} \\
\hline
\keyword{Malware} & Malicious software & Virus, Trojan \\
\hline
\keyword{Phishing} & Fake emails/websites & Fake bank emails \\
\hline
\keyword{Ransomware} & Encrypt files & WannaCry attack \\
\hline
\keyword{DDoS} & Traffic overload & Server flooding \\
\hline
\end{tabular}
\end{answertable}

\textbf{Example - Phishing Attack}:
\begin{itemize}
    \item \textbf{Method}: Fake email from "bank" requesting login credentials
    \item \textbf{Result}: Account compromise
    \item \textbf{Prevention}: Verify sender authenticity
\end{itemize}

\begin{mnemonicbox}
\mnemonic{Cyber Criminals Create Chaos (threats cause damage)}
\end{mnemonicbox}
\end{solutionbox}

\questionmarks{Question 4(c)}{07}{Compare TCP/IP and OSI networking models.}

\begin{solutionbox}
\textbf{Answer:}
\begin{answertable}{TCP/IP vs OSI Model Comparison}
\begin{tabular}{|l|l|l|l|}
\hline
\textbf{OSI Layer} & \textbf{OSI Function} & \textbf{TCP/IP Layer} & \textbf{TCP/IP Function} \\
\hline
\keyword{Application} & User interface & \textbf{Application} & User services \\
\hline
\keyword{Presentation} & Data formatting & \textbf{Application} & (Combined) \\
\hline
\keyword{Session} & Session management & \textbf{Application} & (Combined) \\
\hline
\keyword{Transport} & Reliable delivery & \textbf{Transport} & End-to-end delivery \\
\hline
\keyword{Network} & Routing & \textbf{Internet} & IP addressing \\
\hline
\keyword{Data Link} & Frame handling & \textbf{Network Access} & Physical transmission \\
\hline
\keyword{Physical} & Electrical signals & \textbf{Network Access} & (Combined) \\
\hline
\end{tabular}
\end{answertable}

\textbf{Diagram:}
\begin{center}
\begin{tikzpicture}[gtu flow]
    \node[gtu block] (osi) {OSI (7 Layers)};
    \node[gtu block, right=of osi, xshift=3cm] (tcp) {TCP/IP (4 Layers)};
    
    \node[gtu process, below=of osi] (ol1) {App, Pres, Sess};
    \node[gtu process, below=of ol1] (ol2) {Transport};
    \node[gtu process, below=of ol2] (ol3) {Network};
    \node[gtu process, below=of ol3] (ol4) {Data Link, Phys};

    \node[gtu process, below=of tcp] (tl1) {Application};
    \node[gtu process, below=of tl1] (tl2) {Transport};
    \node[gtu process, below=of tl2] (tl3) {Internet};
    \node[gtu process, below=of tl3] (tl4) {Network Access};

    \draw[->, dashed] (ol1) -- (tl1);
    \draw[->, dashed] (ol2) -- (tl2);
    \draw[->, dashed] (ol3) -- (tl3);
    \draw[->, dashed] (ol4) -- (tl4);
\end{tikzpicture}
\end{center}

\begin{mnemonicbox}
\mnemonic{OSI is Perfect Theory, TCP/IP is Practical Reality}
\end{mnemonicbox}
\end{solutionbox}

\questionmarks{Question 4(a OR)}{03}{Write main objectives of cyber security.}

\begin{solutionbox}
\textbf{Answer:}
\begin{answertable}{Cyber Security Objectives (CIA Triad)}
\begin{tabular}{|l|l|l|}
\hline
\textbf{Objective} & \textbf{Description} & \textbf{Example} \\
\hline
\keyword{Confidentiality} & Protect from unauthorized access & Encryption \\
\hline
\keyword{Integrity} & Ensure accuracy/completeness & Checksums \\
\hline
\keyword{Availability} & Ensure system accessibility & Backups \\
\hline
\end{tabular}
\end{answertable}

\textbf{Additional Objectives}: Authentication, Authorization, Non-repudiation.

\begin{mnemonicbox}
\mnemonic{CIA protects data (Confidentiality-Integrity-Availability)}
\end{mnemonicbox}
\end{solutionbox}

\questionmarks{Question 4(b OR)}{04}{List out different types of networking devices used in the networking.}

\begin{solutionbox}
\textbf{Answer:}
\begin{answertable}{Networking Devices}
\begin{tabular}{|l|l|l|}
\hline
\textbf{Device} & \textbf{Layer} & \textbf{Function} \\
\hline
\keyword{Hub} & Physical & Signal repeater \\
\hline
\keyword{Switch} & Data Link & Frame forwarding \\
\hline
\keyword{Router} & Network & Packet routing \\
\hline
\keyword{Bridge} & Data Link & Network segmentation \\
\hline
\keyword{Gateway} & All layers & Protocol conversion \\
\hline
\keyword{Repeater} & Physical & Signal amplification \\
\hline
\keyword{Access Point} & Data Link & Wireless connectivity \\
\hline
\keyword{Firewall} & Network+ & Security filtering \\
\hline
\end{tabular}
\end{answertable}

\begin{mnemonicbox}
\mnemonic{Hubs Switch Routes Bridges Gateways}
\end{mnemonicbox}
\end{solutionbox}

\questionmarks{Question 4(c OR)}{07}{Write different types of security attacks.}

\begin{solutionbox}
\textbf{Answer:}
\begin{answertable}{Attack Types and Characteristics}
\begin{tabular}{|l|l|l|l|}
\hline
\textbf{Type} & \textbf{Method} & \textbf{Target} & \textbf{Prevention} \\
\hline
\keyword{Passive} & Eavesdropping & Information & Encryption \\
\hline
\keyword{Active} & Modification & Integrity & Authentication \\
\hline
\keyword{Physical} & Hardware access & Equipment & Locks \\
\hline
\keyword{Social Eng.} & Manipulation & Users & Education \\
\hline
\end{tabular}
\end{answertable}

\textbf{Categories}:
\begin{itemize}
    \item \textbf{Network Attacks}: Man-in-the-Middle, DDoS, Packet Sniffing
    \item \textbf{Application Attacks}: SQL Injection, XSS
    \item \textbf{Malware}: Virus, Worm, Trojan, Ransomware
    \item \textbf{Social Engineering}: Phishing, Pretexting
    \item \textbf{Cryptographic}: Brute Force, Dictionary Attack
\end{itemize}

\begin{mnemonicbox}
\mnemonic{Network Application Malware Social Crypto (attack categories)}
\end{mnemonicbox}
\end{solutionbox}

\questionmarks{Question 5(a)}{03}{Calculate binary of (5AB.4) hexadecimal number.}

\begin{solutionbox}
\textbf{Answer:}
\textbf{Hexadecimal to Binary Conversion}: Convert each hex digit to 4-bit binary.
\begin{answertable}{Hex to Binary}
\begin{tabular}{|c|c|c|c|}
\hline
\textbf{Hex} & \textbf{Binary} & \textbf{Hex} & \textbf{Binary} \\
\hline
5 & 0101 & B & 1011 \\
\hline
A & 1010 & 4 & 0100 \\
\hline
\end{tabular}
\end{answertable}

\textbf{Steps}:
\begin{itemize}
    \item 5 $\rightarrow$ 0101
    \item A $\rightarrow$ 1010
    \item B $\rightarrow$ 1011
    \item . $\rightarrow$ .
    \item 4 $\rightarrow$ 0100
\end{itemize}

\textbf{Final Answer}: $(5AB.4)_{16} = (10110101011.01)_2$

\begin{mnemonicbox}
\mnemonic{Each Hex = 4 Bits}
\end{mnemonicbox}
\end{solutionbox}

\questionmarks{Question 5(b)}{04}{List out the main features of Digi-Locker, e-rupi.}

\begin{solutionbox}
\textbf{Answer:}
\begin{answertable}{Digital Platform Features}
\begin{tabular}{|l|l|l|}
\hline
\textbf{Platform} & \textbf{Purpose} & \textbf{Key Features} \\
\hline
\keyword{Digi-Locker} & Document storage & Cloud storage, Aadhaar auth, Paperless \\
\hline
\keyword{e-RUPI} & Digital payment & QR/SMS voucher, Contactless, Prepaid \\
\hline
\end{tabular}
\end{answertable}

\textbf{Benefits}:
\begin{itemize}
    \item \textbf{Digi-Locker}: Secure access to genuine documents anytime
    \item \textbf{e-RUPI}: Leak-proof delivery of welfare benefits
\end{itemize}

\begin{mnemonicbox}
\mnemonic{Digi Stores, e-RUPI Pays (storage vs payment)}
\end{mnemonicbox}
\end{solutionbox}

\questionmarks{Question 5(c)}{07}{Describe different generations of a computer system.}

\begin{solutionbox}
\textbf{Answer:}
\begin{answertable}{Computer Generations}
\begin{tabular}{|l|l|l|l|}
\hline
\textbf{Gen} & \textbf{Period} & \textbf{Technology} & \textbf{Example} \\
\hline
\keyword{1st} & 1940-56 & Vacuum Tubes & ENIAC \\
\hline
\keyword{2nd} & 1956-63 & Transistors & IBM 1401 \\
\hline
\keyword{3rd} & 1964-71 & ICs & IBM 360 \\
\hline
\keyword{4th} & 1971-80s & Microprocessors & PC \\
\hline
\keyword{5th} & 1980s+ & AI/Parallel & Smartphone \\
\hline
\end{tabular}
\end{answertable}

\textbf{Diagram:}
\begin{center}
\begin{tikzpicture}[gtu flow]
    \node[gtu block] (g1) {1st: Vacuum Tubes};
    \node[gtu block, right=of g1] (g2) {2nd: Transistors};
    \node[gtu block, right=of g2] (g3) {3rd: ICs};
    \node[gtu block, below=of g1] (g4) {4th: Microprocessors};
    \node[gtu block, right=of g4] (g5) {5th: AI \& Internet};
    
    \draw[gtu arrow] (g1) -- (g2);
    \draw[gtu arrow] (g2) -- (g3);
    \draw[gtu arrow] (g3) -- (g4);
    \draw[gtu arrow] (g4) -- (g5);
\end{tikzpicture}
\end{center}

\begin{mnemonicbox}
\mnemonic{Vacuum Transistor IC Micro AI}
\end{mnemonicbox}
\end{solutionbox}

\questionmarks{Question 5(a OR)}{03}{Write Difference between Data and Information with example.}

\begin{solutionbox}
\textbf{Answer:}
\begin{answertable}{Data vs Information}
\begin{tabular}{|l|l|l|}
\hline
\textbf{Aspect} & \textbf{Data} & \textbf{Information} \\
\hline
\keyword{Definition} & Raw facts/figures & Processed data \\
\hline
\keyword{Meaning} & No context & Has context \\
\hline
\keyword{Example} & 85, 92, 78 & Avg: 85\% \\
\hline
\end{tabular}
\end{answertable}

\begin{mnemonicbox}
\mnemonic{Data is Raw, Information is Refined}
\end{mnemonicbox}
\end{solutionbox}

\questionmarks{Question 5(b OR)}{04}{Compare analog modulation and digital modulation.}

\begin{solutionbox}
\textbf{Answer:}
\begin{answertable}{Analog vs Digital Modulation}
\begin{tabular}{|l|l|l|}
\hline
\textbf{Feature} & \textbf{Analog} & \textbf{Digital} \\
\hline
\keyword{Signal} & Continuous & Discrete (0s, 1s) \\
\hline
\keyword{Noise Immunity} & Poor & Excellent \\
\hline
\keyword{Examples} & AM, FM & FSK, PSK \\
\hline
\end{tabular}
\end{answertable}

\begin{mnemonicbox}
\mnemonic{Analog is Simple, Digital is Smart}
\end{mnemonicbox}
\end{solutionbox}

\questionmarks{Question 5(c OR)}{07}{Discuss the range of IP addresses in IPv4}

\begin{solutionbox}
\textbf{Answer:}
\begin{answertable}{IPv4 Address Classes}
\begin{tabular}{|l|l|l|l|}
\hline
\textbf{Class} & \textbf{Range} & \textbf{Networks} & \textbf{Purpose} \\
\hline
\keyword{A} & 1.0.0.0 - 126.0.0.0 & Large & Govt/Big Corp \\
\hline
\keyword{B} & 128.0.0.0 - 191.255.0.0 & Medium & Universities \\
\hline
\keyword{C} & 192.0.0.0 - 223.255.255.0 & Small & Small Business \\
\hline
\keyword{D} & 224 - 239 & N/A & Multicast \\
\hline
\keyword{E} & 240 - 255 & N/A & Experimental \\
\hline
\end{tabular}
\end{answertable}

\textbf{Diagram:}
\begin{center}
\begin{tikzpicture}[gtu flow]
    \node[gtu block] (ipv4) {IPv4 Address Space};
    \node[gtu process, below=of ipv4] (abc) {Unicast (Human Use)};
    \node[gtu process, left=of abc] (d) {Multicast (Class D)};
    \node[gtu process, right=of abc] (e) {Reserved (Class E)};
    
    \node[gtu decision, below=of abc] (classes) {Class Range};
    \node[gtu process, below left=of classes] (ca) {A: 1-126};
    \node[gtu process, below=of classes] (cb) {B: 128-191};
    \node[gtu process, below right=of classes] (cc) {C: 192-223};
    
    \draw[gtu arrow] (ipv4) -- (abc);
    \draw[gtu arrow] (ipv4) -- (d);
    \draw[gtu arrow] (ipv4) -- (e);
    \draw[gtu arrow] (abc) -- (classes);
    \draw[gtu arrow] (classes) -- (ca);
    \draw[gtu arrow] (classes) -- (cb);
    \draw[gtu arrow] (classes) -- (cc);
\end{tikzpicture}
\end{center}

\begin{mnemonicbox}
\mnemonic{A Big Company Delivered Everything (Classes A-B-C-D-E)}
\end{mnemonicbox}
\end{solutionbox}

\end{document}
