\documentclass{article}

% content/resources/templates/preamble.tex
\usepackage[margin=0.6in]{geometry}
\author{Milav Dabgar}
\usepackage{amsmath,amssymb,amsthm}
\usepackage{booktabs}
\usepackage{multirow}
\usepackage{xcolor}
\usepackage{tcolorbox}
\tcbuselibrary{breakable,skins}
\usepackage[colorlinks=true,linkcolor=blue]{hyperref}
\usepackage{titlesec}
\usepackage{enumitem}
\usepackage{tikz}
\usepackage{pgfplots}
\usepackage{circuitikz}
\usepackage[version=4]{mhchem}
\usepackage{longtable}
\usepackage{array}
\usepackage{float}
\usepackage{caption}
\usepackage{listings}

\lstset{
  basicstyle=\small\ttfamily,
  breaklines=true,
  breakatwhitespace=false,
  postbreak=\mbox{\textcolor{red}{$\hookrightarrow$}\space},
  float=false,
  numbers=left,
  numberstyle=\tiny\color{gray},
  numbersep=10pt,
  xleftmargin=2em,
  keywordstyle=\color{blue},
  commentstyle=\color{green!60!black},
  stringstyle=\color{purple},
  backgroundcolor=\color{gray!5},
  showstringspaces=false,
  tabsize=2,
  captionpos=b,
  keepspaces=true,
  columns=flexible
}

\pgfplotsset{compat=1.18}
\usetikzlibrary{shapes,arrows,positioning,calc,patterns,decorations.pathmorphing,decorations.markings,arrows.meta}

% Color scheme
\definecolor{headcolor}{RGB}{0,102,204}
\definecolor{keycolor}{RGB}{220,20,60}
\definecolor{solutioncolor}{RGB}{34,139,34}
\definecolor{mnemoniccolor}{RGB}{148,0,211}
\definecolor{codecolor}{RGB}{0,0,100}

% Spacing
\setlength{\parskip}{3pt}
\setlist[itemize]{nosep}
\setlist[enumerate]{nosep}

% Title formatting
\titleformat{\section}{\Large\bfseries\color{headcolor}}{\thesection}{1em}{}
\titleformat{\subsection}{\large\bfseries\color{headcolor}}{\thesubsection}{1em}{}

% Pandoc tightlist compatibility
\providecommand{\tightlist}{%
  \setlength{\itemsep}{0pt}\setlength{\parskip}{0pt}}

% Pandoc longtable compatibility
\newcounter{none}
\def\thenone{}


% content/resources/templates/gujarati-boxes.tex
\usepackage{fontspec}
\usepackage{polyglossia}

% Set Gujarati as main language (document is primarily in Gujarati)
% Note: gloss-gujarati.ldf doesn't exist in polyglossia, but it will use hyphenation patterns
\setdefaultlanguage{gujarati}
\setotherlanguage{english}

% Configure Gujarati font properly
% Use Language=Default to prevent polyglossia from trying to add language-specific features
% that don't exist for Gujarati, which causes "empty feature" warnings
\newfontfamily\gujaratifont[Script=Gujarati,AutoFakeBold=2.5,AutoFakeSlant=0.3]{Noto Sans Gujarati}
\setmainfont[Script=Gujarati,AutoFakeBold=2.5,AutoFakeSlant=0.3]{Noto Sans Gujarati}
% Use Noto Sans Gujarati for monospace to support Gujarati in text
\setmonofont[Scale=0.9]{Noto Sans Gujarati}

% Configure English to use the same font
\newfontfamily\englishfont[Script=Gujarati,AutoFakeBold=2.5,AutoFakeSlant=0.3]{Noto Sans Gujarati}

% Translations for polyglossia
\gappto\captionsgujarati{
  \renewcommand{\tablename}{કોષ્ટક}
  \renewcommand{\figurename}{આકૃતિ}
}

% Helper for TikZ nodes to ensure Gujarati font
\newcommand{\gu}[1]{{\gujaratifont #1}}

% Custom environments
\newtcolorbox{solutionbox}{
    breakable,
    enhanced,
    colback=solutioncolor!5!white,
    colframe=solutioncolor!75!black,
    fonttitle=\bfseries,
    title=જવાબ
}

\newtcolorbox{solutionboxnobreak}{
 colback=solutioncolor!5!white,
 colframe=solutioncolor!75!black,
 fonttitle=\bfseries,
 title=જવાબ
}

\newtcolorbox{keyformula}{
 breakable,
 enhanced,
 colback=keycolor!5!white,
 colframe=keycolor!75!black,
 fonttitle=\bfseries,
 title=રાસાયણિક સમીકરણ/સૂત્ર
}

\newtcolorbox{mnemonicbox}{
 breakable,
 enhanced,
 colback=mnemoniccolor!5!white,
 colframe=mnemoniccolor!75!black,
 fonttitle=\bfseries,
 title=મેમરી ટ્રીક
}


% Custom commands for GTU solutions
% This file defines semantic commands for consistent formatting

% Question command with automatic formatting
\newcommand{\question}[2]{%
  \section*{Question #1}%
  \textbf{#2}%
}

% OR question variant
\newcommand{\questionor}[2]{%
  \section*{Question #1 OR}%
  \textbf{#2}%
}

% Proper table environment with caption
\newenvironment{answertable}[1]{%
  \begin{table}[htbp]
  \centering
  \caption{#1}
}{%
  \end{table}
}

% Proper figure environment for diagrams
\newenvironment{answerdiagram}[1]{%
  \begin{figure}[htbp]
  \centering
  \caption{#1}
}{%
  \end{figure}
}

% Semantic markup for key terms
\newcommand{\keyword}[1]{\textbf{#1}}
\newcommand{\code}[1]{\texttt{#1}}
\newcommand{\classname}[1]{\texttt{#1}}
\newcommand{\methodname}[1]{\texttt{#1}}

% Proper quotation marks
\newcommand{\mnemonic}[1]{``#1''}


\title{Introduction to IT Systems (4311602) - Summer 2023 Solution}
\date{August 7, 2023}

\begin{document}
\maketitle

\questionmarks{પ્રશ્ન ૧(આ)}{03}{કમ્પ્યુટરના મુખ્ય ઘટકોની ચચાર્ કરો.}

\begin{solutionbox}
\textbf{જવાબ:}
\begin{answertable}{Main Components of Computer}
\begin{tabular}{|l|l|l|}
\hline
\textbf{ઘટક} & \textbf{કાર્ય} & \textbf{ઉદાહરણ} \\
\hline
\keyword{ઇનપુટ યુનિટ} & ડેટા અને સૂચનાઓ પ્રાપ્ત કરે & કીબોર્ડ, માઉસ \\
\hline
\keyword{સીપીયુ} & ડેટા પ્રોસેસ કરે અને કંટ્રોલ કરે & Intel i5, AMD Ryzen \\
\hline
\keyword{મેમરી} & ડેટા અસ્થાયી/કાયમી સંગ્રહ કરે & RAM, હાર્ડ ડિસ્ક \\
\hline
\keyword{આઉટપુટ યુનિટ} & પ્રોસેસ કરેલા પરિણામો દર્શાવે & મોનિટર, પ્રિન્ટર \\
\hline
\end{tabular}
\end{answertable}

\textbf{મુખ્ય ઘટકો:}
\begin{itemize}
    \item \textbf{હાર્ડવેર}: ભૌતિક ભાગો જેવા કે CPU, RAM, મધરબોર્ડ
    \item \textbf{સોફ્ટવેર}: પ્રોગ્રામ્સ અને ઓપરેટિંગ સિસ્ટમ
    \item \textbf{ડેટા}: કમ્પ્યુટર દ્વારા પ્રોસેસ થતી માહિતી
\end{itemize}

\begin{mnemonicbox}
\mnemonic{ઇનપુટ સીપીયુ મેમરી આઉટપુટ (I Can Make Output)}
\end{mnemonicbox}
\end{solutionbox}

\questionmarks{પ્રશ્ન ૧(બ)}{04}{વેબ બ્રાઉઝર અને તેનો પ્રકાર સમજાવો.}

\begin{solutionbox}
\textbf{જવાબ:}
\textbf{વેબ બ્રાઉઝર} એ એવો સોફ્ટવેર છે જે ઇન્ટરનેટથી વેબ પૃષ્ઠોને ઍક્સેસ કરે અને દર્શાવે છે.
\begin{answertable}{Types of Web Browsers}
\begin{tabular}{|l|l|l|}
\hline
\textbf{બ્રાઉઝર પ્રકાર} & \textbf{વિશેષતાઓ} & \textbf{ઉદાહરણો} \\
\hline
\keyword{ગ્રાફિકલ} & GUI ઇન્ટરફેસ, મલ્ટિમીડિયા સપોર્ટ & Chrome, Firefox \\
\hline
\keyword{ટેક્સ્ટ-આધારિત} & કમાન્ડ લાઇન, ઝડપી લોડિંગ & Lynx, Links \\
\hline
\keyword{મોબાઇલ} & ટચ ઇન્ટરફેસ, ફોન માટે અનુકૂલિત & Safari Mobile, Chrome Mobile \\
\hline
\end{tabular}
\end{answertable}

\textbf{વિશેષતાઓ:}
\begin{itemize}
    \item \textbf{નેવિગેશન}: આગળ, પાછળ, રિફ્રેશ બટન્સ
    \item \textbf{બુકમાર્ક્સ}: પ્રિય વેબસાઇટ્સ સેવ કરો
    \item \textbf{ટેબ્સ}: એક વિન્ડોમાં બહુવિધ પૃષ્ઠો
    \item \textbf{સિક્યોરિટી}: HTTPS સપોર્ટ, પોપઅપ બ્લોકર્સ
\end{itemize}

\begin{mnemonicbox}
\mnemonic{બ્રાઉઝ કરો સલામત રીતે ઓનલાઇન (Bookmarks-Security-Online)}
\end{mnemonicbox}
\end{solutionbox}

\questionmarks{પ્રશ્ન ૧(સ)}{07}{LAN, MAN અને WAN ને ઉદાહરણો સાથે સમજાવો.}

\begin{solutionbox}
\textbf{જવાબ:}
\begin{answertable}{Network Types Comparison}
\begin{tabular}{|l|l|l|l|l|}
\hline
\textbf{નેટવર્ક} & \textbf{કવરેજ} & \textbf{સ્પીડ} & \textbf{ઉદાહરણ} & \textbf{ખર્ચ} \\
\hline
\keyword{LAN} & બિલ્ડિંગ/કેમ્પસ & ઊંચી & ઓફિસ નેટવર્ક & ઓછો \\
\hline
\keyword{MAN} & શહેર & મધ્યમ & કેબલ ટીવી & મધ્યમ \\
\hline
\keyword{WAN} & દેશ/વૈશ્વિક & બદલાતી & ઇન્ટરનેટ & વધુ \\
\hline
\end{tabular}
\end{answertable}

\textbf{વિસ્તૃત સમજાવટ:}
\begin{itemize}
    \item \textbf{LAN (Local Area Network)}: બિલ્ડિંગ કે નાના વિસ્તારમાં (ઉદા. કમ્પ્યુટર લેબ)
    \item \textbf{MAN (Metropolitan Area Network)}: શહેર કે મેટ્રોપોલિટન વિસ્તાર (ઉદા. કેબલ ટીવી)
    \item \textbf{WAN (Wide Area Network)}: બહુવિધ શહેરો/દેશો (ઉદા. ઇન્ટરનેટ)
\end{itemize}

\textbf{આકૃતિ:}
\begin{center}
\begin{tikzpicture}[gtu flow]
    \node[gtu block] (lan) {LAN (બિલ્ડિંગ)};
    \node[gtu block, right=of lan] (man) {MAN (શહેર)};
    \node[gtu block, right=of man] (wan) {WAN (વૈશ્વિક)};
    
    \node[gtu process, below=of lan] (ex1) {ઓફિસ નેટવર્ક};
    \node[gtu process, below=of man] (ex2) {કેબલ ટીવી};
    \node[gtu process, below=of wan] (ex3) {ઇન્ટરનેટ};

    \path [gtu arrow] (lan) -- (man);
    \path [gtu arrow] (man) -- (wan);
    \path [gtu arrow] (lan) -- (ex1);
    \path [gtu arrow] (man) -- (ex2);
    \path [gtu arrow] (wan) -- (ex3);
\end{tikzpicture}
\end{center}

\begin{mnemonicbox}
\mnemonic{લોકલ મેટ્રો વર્લ્ડ (LAN-MAN-WAN)}
\end{mnemonicbox}
\end{solutionbox}

\questionmarks{પ્રશ્ન ૧(સ અથવા)}{07}{ડોસ અને યુનિક્સ ઓપરેટિંગ સિસ્ટમ વચ્ચે તફાવત લખો.}

\begin{solutionbox}
\textbf{જવાબ:}
\begin{answertable}{DOS vs Unix Comparison}
\begin{tabular}{|l|l|l|}
\hline
\textbf{વિશેષતા} & \textbf{DOS} & \textbf{Unix} \\
\hline
\keyword{ઇન્ટરફેસ} & કમાન્ડ લાઇન & કમાન્ડ લાઇન + GUI \\
\hline
\keyword{મલ્ટિ-યુઝર} & સિંગલ યુઝર & મલ્ટિ-યુઝર સપોર્ટ \\
\hline
\keyword{મલ્ટિટાસ્કિંગ} & મર્યાદિત & સંપૂર્ણ મલ્ટિટાસ્કિંગ \\
\hline
\keyword{સિક્યોરિટી} & મૂળભૂત & અદ્યતન સિક્યોરિટી \\
\hline
\keyword{ફાઇલ સિસ્ટમ} & FAT16/FAT32 & વિવિધ (ext3, ext4) \\
\hline
\end{tabular}
\end{answertable}

\begin{mnemonicbox}
\mnemonic{DOS સરળ, Unix શક્તિશાળી (Single vs Multi-user)}
\end{mnemonicbox}
\end{solutionbox}

\questionmarks{પ્રશ્ન ૨(આ)}{03}{ઓપરેટિંગ સિસ્ટમના લક્ષણોની યાદી આપો.}

\begin{solutionbox}
\textbf{જવાબ:}
\begin{answertable}{Operating System Features}
\begin{tabular}{|l|l|}
\hline
\textbf{વિશેષતા} & \textbf{વર્ણન} \\
\hline
\keyword{પ્રોસેસ મેનેજમેન્ટ} & પ્રોગ્રામ એક્ઝિક્યુશન કંટ્રોલ કરે \\
\hline
\keyword{મેમરી મેનેજમેન્ટ} & RAM કાર્યક્ષમ રીતે વહેંચે \\
\hline
\keyword{ફાઇલ મેનેજમેન્ટ} & ડેટા સ્ટોરેજ વ્યવસ્થિત કરે \\
\hline
\keyword{ડિવાઇસ મેનેજમેન્ટ} & હાર્ડવેર ડિવાઇસો કંટ્રોલ કરે \\
\hline
\end{tabular}
\end{answertable}

\begin{mnemonicbox}
\mnemonic{પ્રોસેસ મેમરી ફાઇલ ડિવાઇસ (Please Manage Files Properly)}
\end{mnemonicbox}
\end{solutionbox}

\questionmarks{પ્રશ્ન ૨(બ)}{04}{હાફ ડુપ્લેક્સ અને ફુલ ડુપ્લેક્સ ટ્રાન્સમિશન મોડ્સ વ્યાખ્યાયિત લખો.}

\begin{solutionbox}
\textbf{જવાબ:}
\begin{answertable}{Transmission Modes Comparison}
\begin{tabular}{|l|l|l|l|}
\hline
\textbf{મોડ} & \textbf{દિશા} & \textbf{ઉદાહરણ} & \textbf{કાર્યક્ષમતા} \\
\hline
\keyword{હાફ ડુપ્લેક્સ} & દ્વિદિશીય (એક સમયે એક) & વોકી-ટોકી & મધ્યમ \\
\hline
\keyword{ફુલ ડુપ્લેક્સ} & દ્વિદિશીય (એકસાથે) & ટેલિફોન & ઊંચી \\
\hline
\end{tabular}
\end{answertable}

\textbf{આકૃતિ:}
\begin{center}
\begin{tikzpicture}[gtu flow]
    \node[gtu block] (a1) {A};
    \node[gtu block, right=of a1, xshift=2cm] (b1) {B};
    \draw[<->, dashed] (a1) -- node[above] {એક સમયે એક} (b1);
    \node[below=of a1, xshift=2cm] {હાફ ડુપ્લેક્સ};

    \node[gtu block, below=of a1, yshift=-1cm] (a2) {A};
    \node[gtu block, right=of a2, xshift=2cm] (b2) {B};
    \draw[transform canvas={yshift=0.7ex}, ->] (a2) -- (b2);
    \draw[transform canvas={yshift=-0.7ex}, <-] (a2) -- node[below] {એકસાથે} (b2);
    \node[below=of a2, xshift=2cm] {ફુલ ડુપ્લેક્સ};
\end{tikzpicture}
\end{center}

\begin{mnemonicbox}
\mnemonic{હાફ રાહ જુએ, ફુલ વહે છે (Half waits, Full flows)}
\end{mnemonicbox}
\end{solutionbox}

\questionmarks{પ્રશ્ન ૨(સ)}{07}{ઓપન સોર્સ અને પ્રોપરાઇટરી સોફ્ટવેર વચ્ચેનો તફાવત.}

\begin{solutionbox}
\textbf{જવાબ:}
\begin{answertable}{Open Source vs Proprietary Software}
\begin{tabular}{|l|l|l|}
\hline
\textbf{પાસા} & \textbf{ઓપન સોર્સ} & \textbf{પ્રોપરાઇટરી} \\
\hline
\keyword{સોર્સ કોડ} & ફ્રીમાં ઉપલબ્ધ & છુપાયેલો/સુરક્ષિત \\
\hline
\keyword{કિંમત} & સામાન્ય રીતે ફ્રી & પેઇડ લાઇસન્સ \\
\hline
\keyword{મોડિફિકેશન} & મંજૂર & પ્રતિબંધિત \\
\hline
\keyword{સપોર્ટ} & કોમ્યુનિટી-આધારિત & વેન્ડર સપોર્ટ \\
\hline
\keyword{સિક્યોરિટી} & ટ્રાન્સપેરન્ટ & ઓબ્સ્ક્યુરિટી દ્વારા \\
\hline
\keyword{ઉદાહરણો} & Linux, Firefox & Windows, MS Office \\
\hline
\end{tabular}
\end{answertable}

\begin{mnemonicbox}
\mnemonic{ઓપન = જોવા માટે ફ્રી, પ્રોપરાઇટરી = વાપરવા માટે પૈસા (Open vs Paid)}
\end{mnemonicbox}
\end{solutionbox}

\questionmarks{પ્રશ્ન ૨(આ અથવા)}{03}{RAM અને ROM વચ્ચે તફાવત લખો.}

\begin{solutionbox}
\textbf{જવાબ:}
\begin{answertable}{RAM vs ROM Comparison}
\begin{tabular}{|l|l|l|}
\hline
\textbf{વિશેષતા} & \textbf{RAM} & \textbf{ROM} \\
\hline
\keyword{પૂર્ણ નામ} & Random Access Memory & Read Only Memory \\
\hline
\keyword{વોલેટિલિટી} & વોલેટાઇલ (ડેટા ગુમાવે) & નોન-વોલેટાઇલ (ડેટા જાળવે) \\
\hline
\keyword{ઍક્સેસ} & રીડ/રાઇટ & ફક્ત રીડ \\
\hline
\end{tabular}
\end{answertable}

\begin{mnemonicbox}
\mnemonic{RAM દોડે, ROM યાદ રાખે (Runs vs Remembers)}
\end{mnemonicbox}
\end{solutionbox}

\questionmarks{પ્રશ્ન ૨(બ અથવા)}{04}{ઉદાહરણ સાથે AND લોજિક ગેટ સમજાવો.}

\begin{solutionbox}
\textbf{જવાબ:}
\textbf{AND ગેટ}: આઉટપુટ ત્યારે જ HIGH આવે જ્યારે બધા ઇનપુટ્સ HIGH હોય.

\textbf{ટ્રુથ ટેબલ:}
\begin{answertable}{AND Gate Truth Table}
\begin{tabular}{|c|c|c|}
\hline
\textbf{ઇનપુટ A} & \textbf{ઇનપુટ B} & \textbf{આઉટપુટ (A AND B)} \\
\hline
0 & 0 & 0 \\
\hline
0 & 1 & 0 \\
\hline
1 & 0 & 0 \\
\hline
1 & 1 & 1 \\
\hline
\end{tabular}
\end{answertable}

\textbf{આકૃતિ:}
\begin{center}
\begin{tikzpicture}
    % Logic Gate Symbol
    \node (A) at (0, 0.5) {A};
    \node (B) at (0, -0.5) {B};
    \draw (1,0.5) -- (2,0.5);
    \draw (1,-0.5) -- (2,-0.5);
    \draw (2,-1) -- (2,1) -- (3,1) arc (90:-90:1) -- (2,-1);
    \draw (4,0) -- (5,0) node[right] {આઉટપુટ};
\end{tikzpicture}
\end{center}

\begin{mnemonicbox}
\mnemonic{બધા ઇનપુટ્સ સાચા = આઉટપુટ સાચો (All True = True)}
\end{mnemonicbox}
\end{solutionbox}

\questionmarks{પ્રશ્ન ૨(સ અથવા)}{07}{ઈથરનેટ કેબલ કલર કોડ સમજાવો.}

\begin{solutionbox}
\textbf{જવાબ:}
\textbf{TIA/EIA-568B કલર કોડ}
\begin{answertable}{Ethernet Pinout (568B)}
\begin{tabular}{|l|l|l|}
\hline
\textbf{પિન} & \textbf{રંગ} & \textbf{કાર્ય} \\
\hline
1 & વાઇટ/ઓરેન્જ & ટ્રાન્સમિટ+ \\
\hline
2 & ઓરેન્જ & ટ્રાન્સમિટ- \\
\hline
3 & વાઇટ/ગ્રીન & રિસીવ+ \\
\hline
4 & બ્લુ & વાપરતા નથી \\
\hline
5 & વાઇટ/બ્લુ & વાપરતા નથી \\
\hline
6 & ગ્રીન & રિસીવ- \\
\hline
7 & વાઇટ/બ્રાઉન & વાપરતા નથી \\
\hline
8 & બ્રાઉન & વાપરતા નથી \\
\hline
\end{tabular}
\end{answertable}

\begin{mnemonicbox}
\mnemonic{વાઇટ ઓરેન્જ, ઓરેન્જ, વાઇટ ગ્રીન, બ્લુ, વાઇટ બ્લુ, ગ્રીન, વાઇટ બ્રાઉન, બ્રાઉન}
\end{mnemonicbox}
\end{solutionbox}

\questionmarks{પ્રશ્ન ૩(આ)}{03}{વાયર્ડ અને વાયરલેસ કોમ્યુનિકેશનની સરખામણી લખો.}

\begin{solutionbox}
\textbf{જવાબ:}
\begin{answertable}{Wired vs Wireless Communication}
\begin{tabular}{|l|l|l|}
\hline
\textbf{પાસા} & \textbf{વાયર્ડ} & \textbf{વાયરલેસ} \\
\hline
\keyword{માધ્યમ} & કેબલ્સ (કોપર/ફાઇબર) & રેડિયો તરંગો \\
\hline
\keyword{સ્પીડ} & વધુ (100Gbps સુધી) & ઓછી (1Gbps સુધી) \\
\hline
\keyword{સિક્યોરિટી} & વધુ સુરક્ષિત & ઓછી સુરક્ષિત \\
\hline
\keyword{મોબિલિટી} & મર્યાદિત & ઊંચી મોબિલિટી \\
\hline
\end{tabular}
\end{answertable}

\begin{mnemonicbox}
\mnemonic{વાયર્સ ઝડપી, વાયરલેસ મુક્ત (Wires Fast, Wireless Free)}
\end{mnemonicbox}
\end{solutionbox}

\questionmarks{પ્રશ્ન ૩(બ)}{04}{કમ્પ્યુટર સિસ્ટમના વિવિધ પ્રકારોની ચર્ચા કરો.}

\begin{solutionbox}
\textbf{જવાબ:}
\begin{answertable}{Computer System Types}
\begin{tabular}{|l|l|l|l|}
\hline
\textbf{પ્રકાર} & \textbf{સાઇઝ} & \textbf{પાવર} & \textbf{ઉદાહરણ} \\
\hline
\keyword{સુપરકમ્પ્યુટર} & રૂમ-સાઇઝ્ડ & અત્યંત ઊંચી & હવામાન આગાહી \\
\hline
\keyword{મેઇનફ્રેમ} & મોટી કેબિનેટ & ખૂબ ઊંચી & બેંક \\
\hline
\keyword{મિનિકમ્પ્યુટર} & ડેસ્ક-સાઇઝ્ડ & મધ્યમ & નાના બિઝનેસ \\
\hline
\keyword{માઇક્રોકમ્પ્યુટર} & ડેસ્કટોપ & ઓછી & લેપટોપ \\
\hline
\end{tabular}
\end{answertable}

\begin{mnemonicbox}
\mnemonic{સુપર મેઇન મિની માઇક્રો (Decreasing size)}
\end{mnemonicbox}
\end{solutionbox}

\questionmarks{પ્રશ્ન ૩(સ)}{07}{TDM, FDM, OFDM પર ટૂંકી નોંધ લખો.}

\begin{solutionbox}
\textbf{જવાબ:}
\begin{answertable}{Multiplexing Comparison}
\begin{tabular}{|l|l|l|}
\hline
\textbf{તકનીક} & \textbf{વિભાજન પદ્ધતિ} & \textbf{ઍપ્લિકેશન} \\
\hline
\keyword{TDM} & સમય સ્લોટ્સ & ડિજિટલ ટેલિફોની \\
\hline
\keyword{FDM} & ફ્રીક્વન્સી બેન્ડ્સ & રેડિયો/ટીવી \\
\hline
\keyword{OFDM} & બહુવિધ કેરિયર્સ & Wi-Fi, 4G/5G \\
\hline
\end{tabular}
\end{answertable}

\textbf{આકૃતિ:}
\begin{center}
\begin{tikzpicture}[gtu flow]
    \node[gtu block] (data) {ડેટા સ્ટ્રીમ};
    \node[gtu process, below left=of data] (tdm) {TDM (સમય)};
    \node[gtu process, below=of data] (fdm) {FDM (ફ્રીક્વન્સી)};
    \node[gtu process, below right=of data] (ofdm) {OFDM (કેરિયર્સ)};
    
    \draw[gtu arrow] (data) -- (tdm);
    \draw[gtu arrow] (data) -- (fdm);
    \draw[gtu arrow] (data) -- (ofdm);
\end{tikzpicture}
\end{center}

\begin{mnemonicbox}
\mnemonic{સમય ફ્રીક્વન્સી ઓર્થોગોનલ (Time Freq Orthogonal)}
\end{mnemonicbox}
\end{solutionbox}

\questionmarks{પ્રશ્ન ૩(આ અથવા)}{03}{FSK અને PSK ની ચર્ચા કરો.}

\begin{solutionbox}
\textbf{જવાબ:}
\begin{answertable}{FSK vs PSK}
\begin{tabular}{|l|l|l|}
\hline
\textbf{પાસા} & \textbf{FSK} & \textbf{PSK} \\
\hline
\keyword{પેરામીટર} & ફ્રીક્વન્સી & ફેઝ \\
\hline
\keyword{કોમ્પ્લેક્સિટી} & સરળ & જટિલ \\
\hline
\keyword{બેન્ડવિડ્થ} & વધુ & ઓછી \\
\hline
\end{tabular}
\end{answertable}

\begin{mnemonicbox}
\mnemonic{ફ્રીક્વન્સી શિફ્ટ, ફેઝ શિફ્ટ (FSK-PSK)}
\end{mnemonicbox}
\end{solutionbox}

\questionmarks{પ્રશ્ન ૩(બ અથવા)}{04}{મલ્ટિટાસ્કિંગ અને મલ્ટિપ્રોગ્રામિંગ OS વચ્ચે તફાવત લખો.}

\begin{solutionbox}
\textbf{જવાબ:}
\begin{answertable}{Multitasking vs Multiprogramming}
\begin{tabular}{|l|l|l|}
\hline
\textbf{વિશેષતા} & \textbf{મલ્ટિટાસ્કિંગ} & \textbf{મલ્ટિપ્રોગ્રામિંગ} \\
\hline
\keyword{યુઝર ઇન્ટરેક્શન} & ઇન્ટરેક્ટિવ & બેચ પ્રોસેસિંગ \\
\hline
\keyword{રિસ્પોન્સ ટાઇમ} & ઝડપી & ધીમી \\
\hline
\keyword{CPU શેરિંગ} & ટાઇમ સ્લાઇસિંગ & જોબ સ્વિચિંગ \\
\hline
\end{tabular}
\end{answertable}

\begin{mnemonicbox}
\mnemonic{ટાસ્ક્સ ઇન્ટરેક્ટિવ, પ્રોગ્રામ્સ બેચ્ડ (Interactive vs Batch)}
\end{mnemonicbox}
\end{solutionbox}

\questionmarks{પ્રશ્ન ૩(સ અથવા)}{07}{નેટવર્ક ટોપોલોજી પર ટૂંકી નોંધ લખો.}

\begin{solutionbox}
\textbf{જવાબ:}
\begin{answertable}{Topology Comparison}
\begin{tabular}{|l|l|l|}
\hline
\textbf{ટોપોલોજી} & \textbf{માળખું} & \textbf{ફાયદા} \\
\hline
\keyword{બસ} & રેખીય & સરળ, કિફાયતી \\
\hline
\keyword{સ્ટાર} & સેન્ટ્રલ હબ & ટ્રબલશૂટિંગ સરળ \\
\hline
\keyword{રિંગ} & વર્તુળાકાર & સમાન ઍક્સેસ \\
\hline
\keyword{મેશ} & આંતર-જોડાયેલ & ઊંચી વિશ્વસનીયતા \\
\hline
\keyword{હાઇબ્રિડ} & મિશ્રિત & લવચીક \\
\hline
\end{tabular}
\end{answertable}

\textbf{આકૃતિ:}
\begin{center}
\begin{tikzpicture}[gtu flow]
    \node[gtu block] (root) {નેટવર્ક ટોપોલોજીઓ};
    \node[gtu process, below left=of root, xshift=-1cm] (bus) {બસ (રેખીય)};
    \node[gtu process, below left=of root, xshift=2cm] (star) {સ્ટાર (હબ)};
    \node[gtu process, below=of root] (ring) {રિંગ (વર્તુળ)};
    \node[gtu process, below right=of root, xshift=-2cm] (mesh) {મેશ (બધા)};
    \node[gtu process, below right=of root, xshift=1cm] (hybrid) {હાઇબ્રિડ (મિશ્ર)};

    \draw[gtu arrow] (root) -- (bus);
    \draw[gtu arrow] (root) -- (star);
    \draw[gtu arrow] (root) -- (ring);
    \draw[gtu arrow] (root) -- (mesh);
    \draw[gtu arrow] (root) -- (hybrid);
\end{tikzpicture}
\end{center}

\begin{mnemonicbox}
\mnemonic{bus star ring mesh hybrid}
\end{mnemonicbox}
\end{solutionbox}

\questionmarks{પ્રશ્ન ૪(આ)}{03}{સ્વિચ સમજાવો.}

\begin{solutionbox}
\textbf{જવાબ:}
\textbf{નેટવર્ક સ્વિચ}: LAN માં ડિવાઇસો કનેક્ટ કરે. ડેટા લિંક લેયર (લેયર 2) પર કામ કરે.

\textbf{કાર્યો:}
\begin{itemize}
    \item \textbf{ફ્રેમ ફોરવર્ડિંગ}: ચોક્કસ પોર્ટને ડેટા મોકલે
    \item \textbf{એડ્રેસ લર્નિંગ}: MAC એડ્રેસ ટેબલ બનાવે
    \item \textbf{લૂપ પ્રિવેન્શન}: સ્પેનિંગ ટ્રી પ્રોટોકોલ
\end{itemize}

\begin{mnemonicbox}
\mnemonic{સ્વિચ MAC એડ્રેસ શીખે (Learns MAC)}
\end{mnemonicbox}
\end{solutionbox}

\questionmarks{પ્રશ્ન ૪(બ)}{04}{સાયબરથ્રેટને ઉદાહરણ સાથે વ્યાખ્યાયિત કરો.}

\begin{solutionbox}
\textbf{જવાબ:}
\textbf{સાયબરથ્રેટ}: સિસ્ટમને નુકસાન પહોંચાડવાનો દુષ્ટ પ્રયાસ.
\begin{answertable}{Cyberthreat Types}
\begin{tabular}{|l|l|l|}
\hline
\textbf{પ્રકાર} & \textbf{પદ્ધતિ} & \textbf{ઉદાહરણ} \\
\hline
\keyword{મેલવેર} & દુષ્ટ સોફ્ટવેર & વાયરસ, ટ્રોજન \\
\hline
\keyword{ફિશિંગ} & નકલી ઇમેઇલ્સ & બેંક ફ્રોડ \\
\hline
\keyword{રેન્સમવેર} & ફાઇલો એન્ક્રિપ્ટ કરે & WannaCry \\
\hline
\keyword{DDoS} & ટ્રાફિક ઓવરલોડ & સર્વર ડાઉન \\
\hline
\end{tabular}
\end{answertable}

\begin{mnemonicbox}
\mnemonic{સાયબર ક્રિમિનલ્સ ચેઓસ ક્રિએટ કરે (Create Chaos)}
\end{mnemonicbox}
\end{solutionbox}

\questionmarks{પ્રશ્ન ૪(સ)}{07}{TCP/IP અને OSI નેટવર્કિંગ મોડલ્સની સરખામણી કરો.}

\begin{solutionbox}
\textbf{જવાબ:}
\begin{answertable}{TCP/IP vs OSI Model Comparison}
\begin{tabular}{|l|l|l|l|}
\hline
\textbf{OSI લેયર} & \textbf{OSI કાર્ય} & \textbf{TCP/IP લેયર} & \textbf{TCP/IP કાર્ય} \\
\hline
\keyword{એપ્લિકેશન} & યુઝર ઇન્ટરફેસ & \textbf{એપ્લિકેશન} & યુઝર સેવાઓ \\
\hline
\keyword{પ્રેઝન્ટેશન} & ડેટા ફોર્મેટિંગ & \textbf{એપ્લિકેશન} & (સંયુક્ત) \\
\hline
\keyword{સેશન} & સેશન મેનેજમેન્ટ & \textbf{એપ્લિકેશન} & (સંયુક્ત) \\
\hline
\keyword{ટ્રાન્સપોર્ટ} & વિશ્વસનીય ડિલિવરી & \textbf{ટ્રાન્સપોર્ટ} & એન્ડ-ટુ-એન્ડ \\
\hline
\keyword{નેટવર્ક} & રાઉટિંગ & \textbf{ઇન્ટરનેટ} & IP એડ્રેસિંગ \\
\hline
\keyword{ડેટા લિંક} & ફ્રેમ હેન્ડલિંગ & \textbf{નેટવર્ક એક્સેસ} & ફિઝિકલ \\
\hline
\keyword{ફિઝિકલ} & સિગ્નલ્સ & \textbf{નેટવર્ક એક્સેસ} & (સંયુક્ત) \\
\hline
\end{tabular}
\end{answertable}

\textbf{આકૃતિ:}
\begin{center}
\begin{tikzpicture}[gtu flow]
    \node[gtu block] (osi) {OSI (7 લેયર્સ)};
    \node[gtu block, right=of osi, xshift=3cm] (tcp) {TCP/IP (4 લેયર્સ)};
    
    \node[gtu process, below=of osi] (ol1) {App, Pres, Sess};
    \node[gtu process, below=of ol1] (ol2) {ટ્રાન્સપોર્ટ};
    \node[gtu process, below=of ol2] (ol3) {નેટવર્ક};
    \node[gtu process, below=of ol3] (ol4) {Link, Phys};

    \node[gtu process, below=of tcp] (tl1) {એપ્લિકેશન};
    \node[gtu process, below=of tl1] (tl2) {ટ્રાન્સપોર્ટ};
    \node[gtu process, below=of tl2] (tl3) {ઇન્ટરનેટ};
    \node[gtu process, below=of tl3] (tl4) {નેટવર્ક એક્સેસ};

    \draw[->, dashed] (ol1) -- (tl1);
    \draw[->, dashed] (ol2) -- (tl2);
    \draw[->, dashed] (ol3) -- (tl3);
    \draw[->, dashed] (ol4) -- (tl4);
\end{tikzpicture}
\end{center}

\begin{mnemonicbox}
\mnemonic{OSI પરફેક્ટ થિયોરી, TCP/IP પ્રેક્ટિકલ રિયાલિટી (Theory vs Practical)}
\end{mnemonicbox}
\end{solutionbox}

\questionmarks{પ્રશ્ન ૪(આ અથવા)}{03}{સાયબર સુરક્ષાના મુખ્ય ઉદ્દેશો લખો.}

\begin{solutionbox}
\textbf{જવાબ:}
\begin{answertable}{Cyber Security Objectives}
\begin{tabular}{|l|l|l|}
\hline
\textbf{ઉદ્દેશ્ય} & \textbf{વર્ણન} & \textbf{ઉદાહરણ} \\
\hline
\keyword{ગુપ્તતા} & અનધિકૃત ઍક્સેસ અટકાવો & એન્ક્રિપ્શન \\
\hline
\keyword{અખંડતા} & ડેટાની ચોકસાઈ & ચેકસમ્સ \\
\hline
\keyword{ઉપલબ્ધતા} & સિસ્ટમની પહોંચ & બેકઅપ \\
\hline
\end{tabular}
\end{answertable}

\begin{mnemonicbox}
\mnemonic{CIA ડેટાને પ્રોટેક્ટ કરે (Confidentiality-Integrity-Availability)}
\end{mnemonicbox}
\end{solutionbox}

\questionmarks{પ્રશ્ન ૪(બ અથવા)}{04}{નેટવર્કિંગમાં વપરાતા નવિવિધ પ્રકારના નેટવર્કિંગ ઉપકરણોની યાદી બનાવો.}

\begin{solutionbox}
\textbf{જવાબ:}
\begin{answertable}{Networking Devices}
\begin{tabular}{|l|l|l|}
\hline
\textbf{ઉપકરણ} & \textbf{લેયર} & \textbf{કાર્ય} \\
\hline
\keyword{હબ} & ફિઝિકલ & રિપીટર \\
\hline
\keyword{સ્વિચ} & ડેટા લિંક & ફોરવર્ડિંગ \\
\hline
\keyword{રાઉટર} & નેટવર્ક & રાઉટિંગ \\
\hline
\keyword{બ્રિજ} & ડેટા લિંક & સેગ્મેન્ટેશન \\
\hline
\keyword{ગેટવે} & ઓલ & કન્વર્ઝન \\
\hline
\keyword{ફાયરવોલ} & નેટવર્ક+ & સિક્યોરિટી \\
\hline
\end{tabular}
\end{answertable}

\begin{mnemonicbox}
\mnemonic{Hubs Switch Routes Bridges Gateways}
\end{mnemonicbox}
\end{solutionbox}

\questionmarks{પ્રશ્ન ૪(સ અથવા)}{07}{વિવિધ પ્રકારના સુરક્ષા હુમલાઓ લખો.}

\begin{solutionbox}
\textbf{જવાબ:}
\begin{answertable}{Attack Types}
\begin{tabular}{|l|l|l|}
\hline
\textbf{પ્રકાર} & \textbf{પદ્ધતિ} & \textbf{ઉદાહરણ} \\
\hline
\keyword{પેસિવ} & છૂપું સાંભળવું & સ્નિફિંગ \\
\hline
\keyword{એક્ટિવ} & મોડિફિકેશન & ડેટા ફેરફાર \\
\hline
\keyword{ફિઝિકલ} & હાર્ડવેર ઍક્સેસ & ચોરી \\
\hline
\keyword{સોશિયલ} & મેનિપ્યુલેશન & ફિશિંગ \\
\hline
\end{tabular}
\end{answertable}

\begin{mnemonicbox}
\mnemonic{નેટવર્ક એપ્લિકેશન મેલવેર સોશિયલ ક્રિપ્ટો (Attack Categories)}
\end{mnemonicbox}
\end{solutionbox}

\questionmarks{પ્રશ્ન ૫(આ)}{03}{(5AB.4) હેક્સાડેસિમલ સંખ્યાની બાઈનરી ગણતરી કરો.}

\begin{solutionbox}
\textbf{જવાબ:}
\textbf{હેક્સાડેસિમલ થી બાઈનરી}:

\begin{itemize}
    \item 5 $\rightarrow$ 0101
    \item A $\rightarrow$ 1010
    \item B $\rightarrow$ 1011
    \item . $\rightarrow$ .
    \item 4 $\rightarrow$ 0100
\end{itemize}

\textbf{અંતિમ જવાબ:} $(5AB.4)_{16} = (010110101011.0100)_2$

\begin{mnemonicbox}
\mnemonic{દરેક હેક્સ = 4 બિટ્સ (Each Hex 4 Bits)}
\end{mnemonicbox}
\end{solutionbox}

\questionmarks{પ્રશ્ન ૫(બ)}{04}{Digi-Locker, e-rupi ની મુખ્ય વિશેષતાઓની યાદી બનાવો.}

\begin{solutionbox}
\textbf{જવાબ:}
\begin{answertable}{Digital Platform Features}
\begin{tabular}{|l|l|l|}
\hline
\textbf{પ્લેટફોર્મ} & \textbf{હેતુ} & \textbf{વિશેષતાઓ} \\
\hline
\keyword{Digi-Locker} & ડોક્યુમેન્ટ સ્ટોરેજ & ક્લાઉડ, પેપરલેસ \\
\hline
\keyword{e-RUPI} & ડિજિટલ પેમેન્ટ & QR વાઉચર, પ્રી-પેઇડ \\
\hline
\end{tabular}
\end{answertable}

\begin{mnemonicbox}
\mnemonic{Digi સ્ટોર કરે, e-RUPI પેમેન્ટ કરે (Store vs Pay)}
\end{mnemonicbox}
\end{solutionbox}

\questionmarks{પ્રશ્ન ૫(સ)}{07}{કમ્પ્યુટર સિસ્ટમની વિવિધ પેઢીઓનું વર્ણન કરો.}

\begin{solutionbox}
\textbf{જવાબ:}
\begin{answertable}{Computer Generations}
\begin{tabular}{|l|l|l|l|}
\hline
\textbf{પેઢી} & \textbf{સમય} & \textbf{ટેકનોલોજી} & \textbf{ઉદાહરણ} \\
\hline
\keyword{1લી} & 1940-56 & વેક્યુમ ટ્યુબ્સ & ENIAC \\
\hline
\keyword{2જી} & 1956-63 & ટ્રાન્ઝિસ્ટર્સ & IBM 1401 \\
\hline
\keyword{3જી} & 1964-71 & ICs & IBM 360 \\
\hline
\keyword{4થી} & 1971-80s & માઇક્રોપ્રોિ. & PC \\
\hline
\keyword{5મી} & 1980s+ & AI & સ્માર્ટફોન \\
\hline
\end{tabular}
\end{answertable}

\textbf{આકૃતિ:}
\begin{center}
\begin{tikzpicture}[gtu flow]
    \node[gtu block] (g1) {1લી: વેક્યુમ ટ્યુબ્સ};
    \node[gtu block, right=of g1] (g2) {2જી: ટ્રાન્ઝિસ્ટર્સ};
    \node[gtu block, right=of g2] (g3) {3જી: ICs};
    \node[gtu block, below=of g1] (g4) {4થી: માઇક્રોપ્રોસેસર્સ};
    \node[gtu block, right=of g4] (g5) {5મી: AI અને ઇન્ટરનેટ};
    
    \draw[gtu arrow] (g1) -- (g2);
    \draw[gtu arrow] (g2) -- (g3);
    \draw[gtu arrow] (g3) -- (g4);
    \draw[gtu arrow] (g4) -- (g5);
\end{tikzpicture}
\end{center}

\begin{mnemonicbox}
\mnemonic{વેક્યુમ ટ્રાન્ઝિસ્ટર IC માઇક્રો AI}
\end{mnemonicbox}
\end{solutionbox}

\questionmarks{પ્રશ્ન ૫(આ અથવા)}{03}{ઉદાહરણ સાથે ડેટા અને ઇન્ફોર્મેશન વચ્ચેનો તફાવત લખો.}

\begin{solutionbox}
\textbf{જવાબ:}
\begin{answertable}{Data vs Information}
\begin{tabular}{|l|l|l|}
\hline
\textbf{પાસા} & \textbf{ડેટા} & \textbf{ઇન્ફોર્મેશન} \\
\hline
\keyword{વ્યાખ્યા} & કાચા તથ્યો & પ્રોસેસ કરેલો ડેટા \\
\hline
\keyword{અર્થ} & સંદર્ભ નથી & સંદર્ભ છે \\
\hline
\keyword{ઉદાહરણ} & 85, 92 & 85\% સરેરાશ \\
\hline
\end{tabular}
\end{answertable}

\begin{mnemonicbox}
\mnemonic{ડેટા કાચો, ઇન્ફોર્મેશન રિફાઇન્ડ (Raw vs Refined)}
\end{mnemonicbox}
\end{solutionbox}

\questionmarks{પ્રશ્ન ૫(બ અથવા)}{04}{એનાલોગ મોડ્યુલેશન અને ડિજિટલ મોડ્યુલેશનની સરખામણી કરો.}

\begin{solutionbox}
\textbf{જવાબ:}
\begin{answertable}{Analog vs Digital Modulation}
\begin{tabular}{|l|l|l|}
\hline
\textbf{વિશેષતા} & \textbf{એનાલોગ} & \textbf{ડિજિટલ} \\
\hline
\keyword{સિગ્નલ} & કન્ટિન્યુઅસ & ડિસ્ક્રીટ \\
\hline
\keyword{ક્વોલિટી} & ઓછી & સારી \\
\hline
\keyword{ઉદાહરણ} & AM, FM & Wi-Fi \\
\hline
\end{tabular}
\end{answertable}

\begin{mnemonicbox}
\mnemonic{એનાલોગ સરળ, ડિજિટલ સ્માર્ટ (Simple vs Smart)}
\end{mnemonicbox}
\end{solutionbox}

\questionmarks{પ્રશ્ન ૫(સ અથવા)}{07}{IPv4 માં IP સરનામાની શ્રેણીની ચર્ચા કરો.}

\begin{solutionbox}
\textbf{જવાબ:}
\begin{answertable}{IPv4 Address Classes}
\begin{tabular}{|l|l|l|}
\hline
\textbf{ક્લાસ} & \textbf{રેન્જ} & \textbf{હેતુ} \\
\hline
\keyword{A} & 1.0.0.0 - 126.x.x.x & મોટી સંસ્થાઓ \\
\hline
\keyword{B} & 128.0.0.0 - 191.x.x.x & મધ્યમ સંસ્થાઓ \\
\hline
\keyword{C} & 192.0.0.0 - 223.x.x.x & નાની સંસ્થાઓ \\
\hline
\keyword{D} & 224 - 239 & મલ્ટિકાસ્ટ \\
\hline
\keyword{E} & 240 - 255 & રિઝર્વ્ડ \\
\hline
\end{tabular}
\end{answertable}

\textbf{આકૃતિ:}
\begin{center}
\begin{tikzpicture}[gtu flow]
    \node[gtu block] (ipv4) {IPv4 એડ્રેસ સ્પેસ};
    \node[gtu process, below=of ipv4] (abc) {યુનિકાસ્ટ};
    \node[gtu process, left=of abc] (d) {મલ્ટિકાસ્ટ (D)};
    \node[gtu process, right=of abc] (e) {રિઝર્વ્ડ (E)};
    
    \node[gtu decision, below=of abc] (classes) {ક્લાસ રેન્જ};
    \node[gtu process, below left=of classes] (ca) {A: 1-126};
    \node[gtu process, below=of classes] (cb) {B: 128-191};
    \node[gtu process, below right=of classes] (cc) {C: 192-223};
    
    \draw[gtu arrow] (ipv4) -- (abc);
    \draw[gtu arrow] (ipv4) -- (d);
    \draw[gtu arrow] (ipv4) -- (e);
    \draw[gtu arrow] (abc) -- (classes);
    \draw[gtu arrow] (classes) -- (ca);
    \draw[gtu arrow] (classes) -- (cb);
    \draw[gtu arrow] (classes) -- (cc);
\end{tikzpicture}
\end{center}

\begin{mnemonicbox}
\mnemonic{A Big Company Delivered Everything (Classes A-B-C-D-E)}
\end{mnemonicbox}
\end{solutionbox}

\end{document}
