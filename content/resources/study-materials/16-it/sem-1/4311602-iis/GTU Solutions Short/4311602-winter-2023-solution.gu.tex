\documentclass{article}

% content/resources/templates/preamble.tex
\usepackage[margin=0.6in]{geometry}
\author{Milav Dabgar}
\usepackage{amsmath,amssymb,amsthm}
\usepackage{booktabs}
\usepackage{multirow}
\usepackage{xcolor}
\usepackage{tcolorbox}
\tcbuselibrary{breakable,skins}
\usepackage[colorlinks=true,linkcolor=blue]{hyperref}
\usepackage{titlesec}
\usepackage{enumitem}
\usepackage{tikz}
\usepackage{pgfplots}
\usepackage{circuitikz}
\usepackage[version=4]{mhchem}
\usepackage{longtable}
\usepackage{array}
\usepackage{float}
\usepackage{caption}
\usepackage{listings}

\lstset{
  basicstyle=\small\ttfamily,
  breaklines=true,
  breakatwhitespace=false,
  postbreak=\mbox{\textcolor{red}{$\hookrightarrow$}\space},
  float=false,
  numbers=left,
  numberstyle=\tiny\color{gray},
  numbersep=10pt,
  xleftmargin=2em,
  keywordstyle=\color{blue},
  commentstyle=\color{green!60!black},
  stringstyle=\color{purple},
  backgroundcolor=\color{gray!5},
  showstringspaces=false,
  tabsize=2,
  captionpos=b,
  keepspaces=true,
  columns=flexible
}

\pgfplotsset{compat=1.18}
\usetikzlibrary{shapes,arrows,positioning,calc,patterns,decorations.pathmorphing,decorations.markings,arrows.meta}

% Color scheme
\definecolor{headcolor}{RGB}{0,102,204}
\definecolor{keycolor}{RGB}{220,20,60}
\definecolor{solutioncolor}{RGB}{34,139,34}
\definecolor{mnemoniccolor}{RGB}{148,0,211}
\definecolor{codecolor}{RGB}{0,0,100}

% Spacing
\setlength{\parskip}{3pt}
\setlist[itemize]{nosep}
\setlist[enumerate]{nosep}

% Title formatting
\titleformat{\section}{\Large\bfseries\color{headcolor}}{\thesection}{1em}{}
\titleformat{\subsection}{\large\bfseries\color{headcolor}}{\thesubsection}{1em}{}

% Pandoc tightlist compatibility
\providecommand{\tightlist}{%
  \setlength{\itemsep}{0pt}\setlength{\parskip}{0pt}}

% Pandoc longtable compatibility
\newcounter{none}
\def\thenone{}


% content/resources/templates/gujarati-boxes.tex
\usepackage{fontspec}
\usepackage{polyglossia}

% Set Gujarati as main language (document is primarily in Gujarati)
% Note: gloss-gujarati.ldf doesn't exist in polyglossia, but it will use hyphenation patterns
\setdefaultlanguage{gujarati}
\setotherlanguage{english}

% Configure Gujarati font properly
% Use Language=Default to prevent polyglossia from trying to add language-specific features
% that don't exist for Gujarati, which causes "empty feature" warnings
\newfontfamily\gujaratifont[Script=Gujarati,AutoFakeBold=2.5,AutoFakeSlant=0.3]{Noto Sans Gujarati}
\setmainfont[Script=Gujarati,AutoFakeBold=2.5,AutoFakeSlant=0.3]{Noto Sans Gujarati}
% Use Noto Sans Gujarati for monospace to support Gujarati in text
\setmonofont[Scale=0.9]{Noto Sans Gujarati}

% Configure English to use the same font
\newfontfamily\englishfont[Script=Gujarati,AutoFakeBold=2.5,AutoFakeSlant=0.3]{Noto Sans Gujarati}

% Translations for polyglossia
\gappto\captionsgujarati{
  \renewcommand{\tablename}{કોષ્ટક}
  \renewcommand{\figurename}{આકૃતિ}
}

% Helper for TikZ nodes to ensure Gujarati font
\newcommand{\gu}[1]{{\gujaratifont #1}}

% Custom environments
\newtcolorbox{solutionbox}{
    breakable,
    enhanced,
    colback=solutioncolor!5!white,
    colframe=solutioncolor!75!black,
    fonttitle=\bfseries,
    title=જવાબ
}

\newtcolorbox{solutionboxnobreak}{
 colback=solutioncolor!5!white,
 colframe=solutioncolor!75!black,
 fonttitle=\bfseries,
 title=જવાબ
}

\newtcolorbox{keyformula}{
 breakable,
 enhanced,
 colback=keycolor!5!white,
 colframe=keycolor!75!black,
 fonttitle=\bfseries,
 title=રાસાયણિક સમીકરણ/સૂત્ર
}

\newtcolorbox{mnemonicbox}{
 breakable,
 enhanced,
 colback=mnemoniccolor!5!white,
 colframe=mnemoniccolor!75!black,
 fonttitle=\bfseries,
 title=મેમરી ટ્રીક
}


% Custom commands for GTU solutions
% This file defines semantic commands for consistent formatting

% Question command with automatic formatting
\newcommand{\question}[2]{%
  \section*{Question #1}%
  \textbf{#2}%
}

% OR question variant
\newcommand{\questionor}[2]{%
  \section*{Question #1 OR}%
  \textbf{#2}%
}

% Proper table environment with caption
\newenvironment{answertable}[1]{%
  \begin{table}[htbp]
  \centering
  \caption{#1}
}{%
  \end{table}
}

% Proper figure environment for diagrams
\newenvironment{answerdiagram}[1]{%
  \begin{figure}[htbp]
  \centering
  \caption{#1}
}{%
  \end{figure}
}

% Semantic markup for key terms
\newcommand{\keyword}[1]{\textbf{#1}}
\newcommand{\code}[1]{\texttt{#1}}
\newcommand{\classname}[1]{\texttt{#1}}
\newcommand{\methodname}[1]{\texttt{#1}}

% Proper quotation marks
\newcommand{\mnemonic}[1]{``#1''}


\title{આઈટી સિસ્ટમ્સનો પરિચય (4311602) - શિયાળો 2023 ઉકેલ}
\date{જાન્યુઆરી 18, 2024}

\begin{document}
\maketitle

\questionmarks{1(અ)}{3}{Information અને Knowledge વચ્ચેનો તફાવત આપો.}

\begin{solutionbox}
\textbf{જવાબ:}

\begin{answertable}{Information vs Knowledge}
\begin{tabulary}{\linewidth}{|L|L|L|}
\hline
\textbf{પાસાં} & \textbf{Information} & \textbf{Knowledge} \\
\hline
\textbf{વ્યાખ્યા} & કાચા તથ્યો અને આંકડાઓ & અનુભવ સાથે પ્રક્રિયા કરેલી માહિતી \\
\hline
\textbf{પ્રક્રિયા} & ગોઠવેલો ડેટા & અનુભવ સાથે જોડાયેલી માહિતી \\
\hline
\textbf{ઉપયોગ} & સહેલાઈથી શેર કરી શકાય & અર્થઘટન અને સંદર્ભ જરૂરી \\
\hline
\end{tabulary}
\end{answertable}

\begin{itemize}
    \item \keyword{Information}: કાચા તથ્યો, ડેટા અને આંકડાઓ કે જેની પ્રક્રિયા કરી શકાય
    \item \keyword{Knowledge}: અનુભવ અને શિક્ષણ દ્વારા પ્રાપ્ત સમજ
\end{itemize}

\begin{mnemonicbox}
\mnemonic{Information માહિતી આપે, Knowledge જ્ઞાન આપે}
\end{mnemonicbox}
\end{solutionbox}

\questionmarks{1(બ)}{4}{OS ના કાર્યો સમજાવો.}

\begin{solutionbox}
\textbf{જવાબ:}

\textbf{ઓપરેટિંગ સિસ્ટમના મુખ્ય કાર્યો:}

\begin{answertable}{OS ના કાર્યો}
\begin{tabulary}{\linewidth}{|L|L|}
\hline
\textbf{કાર્ય} & \textbf{વર્ણન} \\
\hline
\textbf{Process Management} & પ્રોગ્રામ્સના અમલીકરણને નિયંત્રિત કરે \\
\hline
\textbf{Memory Management} & મેમરી ફાળવણી અને મુક્તિ \\
\hline
\textbf{File Management} & ફાઇલોનું સંગઠન અને વ્યવસ્થાપન \\
\hline
\textbf{Device Management} & ઇનપુટ/આઉટપુટ ઉપકરણોનું નિયંત્રણ \\
\hline
\end{tabulary}
\end{answertable}

\begin{itemize}
    \item \textbf{Process Control}: ચાલતા પ્રોગ્રામ્સનું શેડ્યુલિંગ અને વ્યવસ્થાપન
    \item \textbf{Resource Allocation}: સિસ્ટમ સંસાધનોનું કાર્યક્ષમ વિતરણ
    \item \textbf{User Interface}: યુઝર અને કમ્પ્યુટર વચ્ચે ક્રિયાપ્રતિક્રિયા
\end{itemize}

\begin{mnemonicbox}
\mnemonic{PMFD - Process, Memory, File, Device}
\end{mnemonicbox}
\end{solutionbox}

\questionmarks{1(ક)}{7}{યુનિવર્સલ ગેટ વ્યાખ્યાયિત કરો અને NAND યુનિવર્સલ ગેટનો ઉપયોગ કરીને બેસિક ગેટ બનાવો.}

\begin{solutionbox}
\textbf{જવાબ:}

\textbf{યુનિવર્સલ ગેટની વ્યાખ્યા:}
એવા લોજિક ગેટ કે જે અન્ય કોઈ ગેટનો ઉપયોગ કર્યા વિના કોઈપણ Boolean function અમલ કરી શકે.

\begin{answertable}{NAND ગેટ Truth Table}
\begin{tabulary}{\linewidth}{|C|C|C|}
\hline
\textbf{A} & \textbf{B} & \textbf{NAND આઉટપુટ} \\
\hline
0 & 0 & 1 \\
\hline
0 & 1 & 1 \\
\hline
1 & 0 & 1 \\
\hline
1 & 1 & 0 \\
\hline
\end{tabulary}
\end{answertable}

\textbf{NAND વડે બેસિક ગેટ્સ:}

\begin{answerdiagram}{NOT Gate using NAND}
\begin{tikzpicture}[circuit logic US]
    \node [nand gate, inputs={nn}] (n1) {};
    \draw (n1.input 1) -- ++(-0.5,0) node[left] {A};
    \draw (n1.input 2) -- ++(-0.5,0) node[left] {A};
    \draw (n1.output) -- ++(0.5,0) node[right] {આઉટપુટ (NOT A)};
    % Connect inputs together
    \draw (n1.input 1) -- ++(-0.2,0) |- (n1.input 2);
\end{tikzpicture}
\end{answerdiagram}

\begin{answerdiagram}{AND Gate using NAND}
\begin{tikzpicture}[circuit logic US]
    \node [nand gate] (n1) at (0,0) {};
    \node [nand gate, inputs={nn}] (n2) at (2.5,0) {}; % Acts as NOT
    
    \draw (n1.input 1) -- ++(-0.5,0) node[left] {A};
    \draw (n1.input 2) -- ++(-0.5,0) node[left] {B};
    
    \draw (n1.output) -- (n2.input 1);
    \draw (n1.output) -- (n2.input 2); % NOT configuration
    \draw (n2.output) -- ++(0.5,0) node[right] {આઉટપુટ (A AND B)};
\end{tikzpicture}
\end{answerdiagram}

\begin{answerdiagram}{OR Gate using NAND}
\begin{tikzpicture}[circuit logic US]
    \node [nand gate, inputs={nn}] (n1) at (0,1) {}; % NOT A
    \node [nand gate, inputs={nn}] (n2) at (0,-1) {}; % NOT B
    \node [nand gate] (n3) at (2.5,0) {};
    
    \draw (n1.input 1) -- ++(-0.5,0) node[left] {A};
    \draw (n1.input 2) -- ++(-0.5,0) node[left] {A};
    \draw (n1.input 1) -- ++(-0.2,0) |- (n1.input 2);
    
    \draw (n2.input 1) -- ++(-0.5,0) node[left] {B};
    \draw (n2.input 2) -- ++(-0.5,0) node[left] {B};
    \draw (n2.input 1) -- ++(-0.2,0) |- (n2.input 2);
    
    \draw (n1.output) -- (n3.input 1);
    \draw (n2.output) -- (n3.input 2);
    \draw (n3.output) -- ++(0.5,0) node[right] {આઉટપુટ (A OR B)};
\end{tikzpicture}
\end{answerdiagram}

\begin{itemize}
    \item \textbf{NOT}: બંને NAND ઇનપુટમાં એક જ ઇનપુટ આપવું
    \item \textbf{AND}: NAND પછી NOT (બીજું NAND)
    \item \textbf{OR}: બંને ઇનપુટ્સને NOT કરો, પછી NAND કરો
\end{itemize}

\begin{mnemonicbox}
\mnemonic{NAND ને બીજા NAND ની નિશ્ચિત જરૂર}
\end{mnemonicbox}
\end{solutionbox}

\questionmarks{1(ક OR)}{7}{નીચેના રૂપાંતરણ કરો:}

\begin{solutionbox}
\textbf{જવાબ:}

\textbf{રૂપાંતરણ ઉકેલો:}

\begin{answertable}{રૂપાંતરણ સારાંશ}
\begin{tabulary}{\linewidth}{|L|L|L|L|}
\hline
\textbf{માંથી} & \textbf{માં} & \textbf{પ્રક્રિયા} & \textbf{પરિણામ} \\
\hline
$(1456)_8$ & Base 16 & $8 \to 10 \to 16$ & $(32E)_{16}$ \\
\hline
$(1011)_2$ & Base 10 & Binary to Decimal & $(11)_{10}$ \\
\hline
$(247.38)_{10}$ & Base 8 & Purnank ane Apurnank alag & $(367.3)_8$ \\
\hline
\end{tabulary}
\end{answertable}

\textbf{વિગતવાર ઉકેલ:}

1) \textbf{$(1456)_8 = (32E)_{16}$}
\begin{itemize}
    \item $1 \times 8^3 + 4 \times 8^2 + 5 \times 8^1 + 6 \times 8^0 = 512 + 256 + 40 + 6 = (814)_{10}$
    \item $814 \div 16 = 50$ remainder $14(E)$, $50 \div 16 = 3$ remainder $2$
    \item પરિણામ: $(32E)_{16}$
\end{itemize}

2) \textbf{$(1011)_2 = (11)_{10}$}
\begin{itemize}
    \item $1 \times 2^3 + 0 \times 2^2 + 1 \times 2^1 + 1 \times 2^0 = 8 + 0 + 2 + 1 = (11)_{10}$
\end{itemize}

3) \textbf{$(247.38)_{10} = (367.3)_8$}
\begin{itemize}
    \item પૂર્ણાંક: $247 \div 8 = 30$ બાકી 7, $30 \div 8 = 3$ બાકી 6, $3 \div 8 = 0$ બાકી 3
    \item દશાંશ: $0.38 \times 8 = 3.04$ (3 લો)
    \item પરિણામ: $(367.3)_8$
\end{itemize}

\begin{mnemonicbox}
\mnemonic{રૂપાંતરણ સાવચેતીથી, ગણતરી ચકાસીને}
\end{mnemonicbox}
\end{solutionbox}

\questionmarks{2(અ)}{3}{મેમરીના પ્રકારોની સૂચિ બનાવો.}

\begin{solutionbox}
\textbf{જવાબ:}

\textbf{મેમરી વર્ગીકરણ:}

\begin{answertable}{મેમરી પ્રકારો}
\begin{tabulary}{\linewidth}{|L|L|L|}
\hline
\textbf{પ્રકાર} & \textbf{ઉદાહરણ} & \textbf{લાક્ષણિકતાઓ} \\
\hline
\textbf{Primary Memory} & RAM, ROM, Cache & CPU દ્વારા સીધી પહોંચ \\
\hline
\textbf{Secondary Memory} & HDD, SSD, CD/DVD & બિન-અસ્થાયી સંગ્રહ \\
\hline
\textbf{Cache Memory} & L1, L2, L3 & હાઇ-સ્પીડ બફર મેમરી \\
\hline
\end{tabulary}
\end{answertable}

\begin{itemize}
    \item \textbf{Volatile}: પાવર બંધ કરવાથી ડેટા ગુમાવે (RAM)
    \item \textbf{Non-volatile}: પાવર વિના ડેટા જાળવે (ROM, HDD)
    \item \textbf{ઍક્સેસ સ્પીડ}: Cache > RAM > Secondary Storage
\end{itemize}

\begin{mnemonicbox}
\mnemonic{Primary પ્રક્રિયા કરે, Secondary સંગ્રહ કરે}
\end{mnemonicbox}
\end{solutionbox}

\questionmarks{2(બ)}{4}{Kernel Mode અને User Mode વચ્ચે તફાવત આપો.}

\begin{solutionbox}
\textbf{જવાબ:}

\begin{answertable}{Kernel vs User Mode}
\begin{tabulary}{\linewidth}{|L|L|L|}
\hline
\textbf{પાસાં} & \textbf{Kernel Mode} & \textbf{User Mode} \\
\hline
\textbf{અધિકાર સ્તર} & સંપૂર્ણ સિસ્ટમ ઍક્સેસ & મર્યાદિત ઍક્સેસ \\
\hline
\textbf{સૂચનાઓ} & બધી સૂચનાઓની મંજૂરી & મર્યાદિત સૂચના સેટ \\
\hline
\textbf{મેમરી ઍક્સેસ} & સંપૂર્ણ મેમરી ઍક્સેસ & મર્યાદિત મેમરી વિસ્તારો \\
\hline
\textbf{સિસ્ટમ કૉલ્સ} & સીધી હાર્ડવેર ઍક્સેસ & માત્ર સિસ્ટમ કૉલ્સ દ્વારા \\
\hline
\end{tabulary}
\end{answertable}

\begin{itemize}
    \item \textbf{Kernel Mode}: ઓપરેટિંગ સિસ્ટમ સંપૂર્ણ અધિકારો સાથે ચાલે
    \item \textbf{User Mode}: એપ્લિકેશન્સ મર્યાદિત અધિકારો સાથે ચાલે
    \item \textbf{સુરક્ષા}: મોડ સ્વિચિંગ અનધિકૃત ઍક્સેસ અટકાવે
\end{itemize}

\begin{mnemonicbox}
\mnemonic{Kernel નિયંત્રણ કરે, User ઉપયોગ કરે}
\end{mnemonicbox}
\end{solutionbox}

\questionmarks{2(ક)}{7}{OS ના પ્રકારોની યાદી બનાવો અને કોઈપણ બે OS સમજાવો}

\begin{solutionbox}
\textbf{જવાબ:}

\textbf{ઓપરેટિંગ સિસ્ટમના પ્રકારો:}

\begin{answertable}{ઓપરેટિંગ સિસ્ટમ પ્રકારો}
\begin{tabulary}{\linewidth}{|L|L|L|}
\hline
\textbf{પ્રકાર} & \textbf{ઉદાહરણ} & \textbf{લાક્ષણિકતાઓ} \\
\hline
\textbf{Batch OS} & પ્રારંભિક mainframes & યુઝર ક્રિયાપ્રતિક્રિયા નથી \\
\hline
\textbf{Time-sharing OS} & UNIX, Linux & એકસાથે બહુવિધ યુઝર્સ \\
\hline
\textbf{Real-time OS} & Embedded systems & ગેરંટીડ પ્રતિસાદ સમય \\
\hline
\textbf{Distributed OS} & Cloud systems & બહુવિધ જોડાયેલા કમ્પ્યુટર્સ \\
\hline
\textbf{Network OS} & Windows Server & નેટવર્ક સંસાધન વ્યવસ્થાપન \\
\hline
\textbf{Mobile OS} & Android, iOS & સ્માર્ટફોન/ટેબલેટ સિસ્ટમ્સ \\
\hline
\end{tabulary}
\end{answertable}

\textbf{વિગતવાર સમજૂતી:}

\textbf{1. Time-sharing OS (Linux):}
\begin{itemize}
    \item \textbf{Multi-user}: બહુવિધ યુઝર્સ એકસાથે ઍક્સેસ કરી શકે
    \item \textbf{Multi-tasking}: બહુવિધ પ્રક્રિયાઓ સમાંતર ચલાવે
    \item \textbf{સંસાધન શેરિંગ}: CPU સમય પ્રક્રિયાઓ વચ્ચે વહેંચાય
    \item \textbf{ઉદાહરણ}: UNIX, Linux, Windows
\end{itemize}

\textbf{2. Real-time OS:}
\begin{itemize}
    \item \textbf{નિર્ધારિત}: સમય મર્યાદામાં ગેરંટીડ પ્રતિસાદ
    \item \textbf{પ્રાથમિકતા આધારિત}: મહત્વપૂર્ણ કાર્યોને ઊંચી પ્રાથમિકતા
    \item \textbf{ઉપયોગ}: મેડિકલ ઉપકરણો, ઔદ્યોગિક નિયંત્રણ
    \item \textbf{પ્રકાર}: Hard real-time અને Soft real-time
\end{itemize}

\begin{mnemonicbox}
\mnemonic{સમય ટિક કરે, Real-time રિએક્ટ કરે}
\end{mnemonicbox}
\end{solutionbox}

\questionmarks{2(અ OR)}{3}{Linux Operating System નું આર્કિટેક્ચર સમજાવો.}

\begin{solutionbox}
\textbf{જવાબ:}

\textbf{Linux આર્કિટેક્ચર સ્તરો:}

\begin{answerdiagram}{Linux Architecture (Concentric View)}
\begin{tikzpicture}
    % Concentric circles
    \draw [fill=blue!10] (0,0) circle (3.5cm);
    \node at (0,3) {\textbf{User Apps}};
    
    \draw [fill=green!10] (0,0) circle (2.5cm);
    \node at (0,2) {\textbf{Sys Call IF}};
    
    \draw [fill=orange!10] (0,0) circle (1.5cm);
    \node at (0,1) {\textbf{Kernel}};
    
    \draw [fill=red!20] (0,0) circle (0.5cm);
    \node at (0,0) {\textbf{HW}};
    
    % Legend or labels
    \node [align=left] at (4,2) {HW: Hardware\\Kernel: Core Functions\\Sys Call IF: Interface\\User Apps: Applications};
\end{tikzpicture}
\end{answerdiagram}

\begin{itemize}
    \item \textbf{User Space}: એપ્લિકેશન્સ અને યુઝર પ્રોગ્રામ્સ
    \item \textbf{System Calls}: યુઝર અને kernel વચ્ચેનું ઇન્ટરફેસ
    \item \textbf{Kernel}: મુખ્ય ઓપરેટિંગ સિસ્ટમ કાર્યો
\end{itemize}

\begin{mnemonicbox}
\mnemonic{યુઝર્સ ઉપયોગ કરે, Kernel નિયંત્રણ કરે}
\end{mnemonicbox}
\end{solutionbox}

\questionmarks{2(બ OR)}{4}{Search Engine ની કામગીરી સમજાવો.}

\begin{solutionbox}
\textbf{જવાબ:}

\textbf{Search Engine કામકાજની પ્રક્રિયા:}

\begin{answertable}{Search Engine સ્ટેપ્સ}
\begin{tabulary}{\linewidth}{|L|L|L|}
\hline
\textbf{સ્ટેપ} & \textbf{પ્રક્રિયા} & \textbf{કાર્ય} \\
\hline
\textbf{Crawling} & વેબ સ્પાઇડર્સ વેબસાઇટ્સ સ્કેન કરે & વેબ પેજીસ શોધે \\
\hline
\textbf{Indexing} & કન્ટેન્ટ વિશ્લેષણ અને સંગ્રહ & શોધી શકાય તેવો ડેટાબેસ બનાવે \\
\hline
\textbf{Ranking} & ઍલ્ગોરિધમ લાગુ કરે & સુસંગતતાનો ક્રમ નક્કી કરે \\
\hline
\textbf{Retrieval} & પરિણામો પરત કરે & ક્રમબદ્ધ પરિણામો દર્શાવે \\
\hline
\end{tabulary}
\end{answertable}

\begin{itemize}
    \item \textbf{વેબ ક્રોલર્સ}: ઓટોમેટેડ બોટ્સ ઇન્ટરનેટ કન્ટેન્ટ સ્કેન કરે
    \item \textbf{ઇન્ડેક્સ ડેટાબેસ}: વેબપેજ માહિતી સંગ્રહિત અને ગોઠવે
    \item \textbf{ક્વેરી પ્રોસેસિંગ}: યુઝર શોધ શબ્દોનું વિશ્લેષણ કરે
    \item \textbf{પરિણામ રેન્કિંગ}: સુસંગતતા અનુસાર પરિણામોનો ક્રમ કરે
\end{itemize}

\begin{mnemonicbox}
\mnemonic{ક્રોલ, ઇન્ડેક્સ, રેન્ક, પુનઃપ્રાપ્ત}
\end{mnemonicbox}
\end{solutionbox}

\questionmarks{2(ક OR)}{7}{Open Source Software અને Proprietary Software વચ્ચે તફાવત આપો.}

\begin{solutionbox}
\textbf{જવાબ:}

\begin{answertable}{Open Source vs Proprietary Software}
\begin{tabulary}{\linewidth}{|L|L|L|}
\hline
\textbf{પાસાં} & \textbf{Open Source Software} & \textbf{Proprietary Software} \\
\hline
\textbf{સોર્સ કોડ} & મુક્તપણે ઉપલબ્ધ અને સુધારી શકાય & બંધ અને સુરક્ષિત \\
\hline
\textbf{કિંમત} & સામાન્યતે મફત & લાઇસન્સ ખરીદવાની જરૂર \\
\hline
\textbf{સપોર્ટ} & કમ્યુનિટી આધારિત & વેન્ડર દ્વારા પૂરું પાડવામાં આવે \\
\hline
\textbf{કસ્ટમાઇઝેશન} & સંપૂર્ણ કસ્ટમાઇઝ કરી શકાય & મર્યાદિત કસ્ટમાઇઝેશન \\
\hline
\textbf{ઉદાહરણ} & Linux, Firefox, LibreOffice & Windows, MS Office, Photoshop \\
\hline
\textbf{સુરક્ષા} & પારદર્શક, કમ્યુનિટી ઓડિટેડ & અસ્પષ્ટતા દ્વારા સુરક્ષા \\
\hline
\textbf{અપડેટ્સ} & કમ્યુનિટી સંચાલિત & વેન્ડર નિયંત્રિત \\
\hline
\end{tabulary}
\end{answertable}

\textbf{મુખ્ય તફાવતો:}
\begin{itemize}
    \item \textbf{લાઇસન્સિંગ}: Open source પુનઃવિતરણ અને સુધારાની મંજૂરી આપે vs proprietary પેઇડ
    \item \textbf{કિંમત મોડેલ}: Open source સામાન્યતે મફત vs proprietary પેઇડ
    \item \textbf{ડેવલપમેન્ટ}: કમ્યુનિટી સહયોગ vs કંપની નિયંત્રિત
    \item \textbf{પારદર્શિતા}: Open source કોડ દૃશ્યમાન vs proprietary છુપાયેલ
\end{itemize}

\textbf{ફાયદા:}
\begin{itemize}
    \item \textbf{Open Source}: કિફાયતી, કસ્ટમાઇઝ કરી શકાય, સુરક્ષિત
    \item \textbf{Proprietary}: વ્યાવસાયિક સપોર્ટ, એકીકૃત લક્ષણો, યુઝર-ફ્રેન્ડલી
\end{itemize}

\begin{mnemonicbox}
\mnemonic{Open ખુલ્લું કરે, Proprietary સુરક્ષિત કરે}
\end{mnemonicbox}
\end{solutionbox}

\questionmarks{3(અ)}{3}{નીચેનાનું સંપૂર્ણ નામ આપો: OSI, LLC, FTP}

\begin{solutionbox}
\textbf{જવાબ:}

\textbf{સંપૂર્ણ રૂપો:}

\begin{answertable}{સંક્ષેપ અને પૂરું નામ}
\begin{tabulary}{\linewidth}{|L|L|}
\hline
\textbf{સંક્ષેપ} & \textbf{સંપૂર્ણ રૂપ} \\
\hline
\textbf{OSI} & Open Systems Interconnection \\
\hline
\textbf{LLC} & Logical Link Control \\
\hline
\textbf{FTP} & File Transfer Protocol \\
\hline
\end{tabulary}
\end{answertable}

\begin{itemize}
    \item \textbf{OSI}: 7 સ્તરો સાથેનું નેટવર્કિંગ સંદર્ભ મોડેલ
    \item \textbf{LLC}: OSI મોડેલમાં Data Link Layer નું સબલેયર
    \item \textbf{FTP}: નેટવર્ક પર ફાઇલો ટ્રાન્સફર કરવા માટેનું પ્રોટોકોલ
\end{itemize}

\begin{mnemonicbox}
\mnemonic{Open Logic Files}
\end{mnemonicbox}
\end{solutionbox}

\questionmarks{3(બ)}{4}{Twisted Pair Cable ના ફાયદા અને ગેરફાયદા આપો}

\begin{solutionbox}
\textbf{જવાબ:}

\textbf{Twisted Pair Cable વિશ્લેષણ:}

\begin{answertable}{Twisted Pair ફાયદા અને ગેરફાયદા}
\begin{tabulary}{\linewidth}{|L|L|}
\hline
\textbf{ફાયદા} & \textbf{ગેરફાયદા} \\
\hline
\textbf{ઓછી કિંમત} & \textbf{મર્યાદિત અંતર} \\
\hline
\textbf{સરળ ઇન્સ્ટોલેશન} & \textbf{ઇલેક્ટ્રોમેગ્નેટિક હસ્તક્ષેપ} \\
\hline
\textbf{લવચીક} & \textbf{ઓછી બેન્ડવિડ્થ} \\
\hline
\textbf{વ્યાપકપણે ઉપલબ્ધ} & \textbf{સુરક્ષા સમસ્યાઓ} \\
\hline
\end{tabulary}
\end{answertable}

\textbf{ફાયદા:}
\begin{itemize}
    \item \textbf{કિફાયતી}: સૌથી સસ્તો નેટવર્કિંગ કેબલ વિકલ્પ
    \item \textbf{સરળ ઇન્સ્ટોલેશન}: ઇન્સ્ટોલ અને જાળવણી સરળ
    \item \textbf{લવચીકતા}: સહેલાઈથી વાળી અને રૂટ કરી શકાય
\end{itemize}

\textbf{ગેરફાયદા:}
\begin{itemize}
    \item \textbf{અંતર મર્યાદા}: રિપીટર વિના મહત્તમ 100 મીટર
    \item \textbf{હસ્તક્ષેપ}: ઇલેક્ટ્રોમેગ્નેટિક હસ્તક્ષેપ માટે સંવેદનશીલ
    \item \textbf{બેન્ડવિડ્થ}: ફાઇબર કરતાં ઓછા ડેટા ટ્રાન્સમિશન રેટ
\end{itemize}

\begin{mnemonicbox}
\mnemonic{Twisted સસ્તું પણ મર્યાદિત}
\end{mnemonicbox}
\end{solutionbox}

\questionmarks{3(ક)}{7}{Modulation શું છે? Analog Modulation સમજાવો.}

\begin{solutionbox}
\textbf{જવાબ:}

\textbf{Modulation ની વ્યાખ્યા:}
લાંબા અંતર સુધી માહિતી ટ્રાન્સમિટ કરવા માટે carrier signal ની લાક્ષણિકતાઓ બદલવાની પ્રક્રિયા.

\textbf{Analog Modulation પ્રકારો:}

\begin{answertable}{Analog Modulation પ્રકારો}
\begin{tabulary}{\linewidth}{|L|L|L|}
\hline
\textbf{પ્રકાર} & \textbf{બદલાતું પરિમાણ} & \textbf{ઉપયોગ} \\
\hline
\textbf{AM} & Amplitude & રેડિયો બ્રોડકાસ્ટિંગ \\
\hline
\textbf{FM} & Frequency & FM રેડિયો, TV સાઉન્ડ \\
\hline
\textbf{PM} & Phase & ડિજિટલ કમ્યુનિકેશન્સ \\
\hline
\end{tabulary}
\end{answertable}

\textbf{Amplitude Modulation (AM):}

\begin{answerdiagram}{Amplitude Modulation}
\begin{tikzpicture}[auto, node distance=2cm]
    \node [gtu block] (modulator) {મોડ્યુલેટર};
    \node [left=of modulator, yshift=1cm] (message) {મેસેજ સિગ્નલ};
    \node [left=of modulator, yshift=-1cm] (carrier) {કેરિયર સિગ્નલ};
    \node [right=of modulator] (output) {મોડ્યુલેટેડ સિગ્નલ};
    
    \draw [gtu arrow] (message) -| (modulator);
    \draw [gtu arrow] (carrier) -| (modulator);
    \draw [gtu arrow] (modulator) -- (output);
\end{tikzpicture}
\end{answerdiagram}

\textbf{મુખ્ય ખ્યાલો:}
\begin{itemize}
    \item \textbf{Carrier Wave}: ટ્રાન્સમિશન માટે હાઇ-ફ્રીક્વન્સી સિગ્નલ
    \item \textbf{Message Signal}: ટ્રાન્સમિટ કરવાની માહિતી
    \item \textbf{Modulation Index}: લાગુ કરેલ modulation ની માત્રા
\end{itemize}

\textbf{ઉપયોગ:}
\begin{itemize}
    \item \textbf{AM Radio}: 530-1710 kHz ફ્રીક્વન્સી બેન્ડ
    \item \textbf{FM Radio}: 88-108 MHz ફ્રીક્વન્સી બેન્ડ
    \item \textbf{ટેલિવિઝન}: વિવિધ modulation તકનીકો
\end{itemize}

\textbf{ફાયદા:}
\begin{itemize}
    \item \textbf{લાંબું અંતર}: લાંબા અંતરની કમ્યુનિકેશન શક્ય બનાવે
    \item \textbf{Noise Immunity}: FM વધુ સારી noise પ્રતિકાર આપે
\end{itemize}

\begin{mnemonicbox}
\mnemonic{Amplitude બદલાય, Frequency ફ્લક્ચ્યુએટ કરે}
\end{mnemonicbox}
\end{solutionbox}

\questionmarks{3(અ OR)}{3}{Network Topology ની યાદી બનાવો. Bus Topology ના ફાયદા અને ગેરફાયદા લખો.}

\begin{solutionbox}
\textbf{જવાબ:}

\textbf{નેટવર્ક ટોપોલોજીઓ:}
\begin{itemize}
    \item \textbf{Bus Topology}
    \item \textbf{Star Topology}
    \item \textbf{Ring Topology}
    \item \textbf{Mesh Topology}
    \item \textbf{Hybrid Topology}
\end{itemize}

\textbf{Bus Topology વિશ્લેષણ:}

\begin{answertable}{Bus Topology ફાયદા અને ગેરફાયદા}
\begin{tabulary}{\linewidth}{|L|L|}
\hline
\textbf{ફાયદા} & \textbf{ગેરફાયદા} \\
\hline
\textbf{સરળ ડિઝાઇન} & \textbf{સિંગલ પોઇન્ટ ઓફ ફેઇલ્યુર} \\
\hline
\textbf{કિફાયતી} & \textbf{મર્યાદિત કેબલ લંબાઇ} \\
\hline
\textbf{સરળ વિસ્તરણ} & \textbf{પર્ફોર્મન્સ ઘટાડો} \\
\hline
\end{tabulary}
\end{answertable}

\begin{mnemonicbox}
\mnemonic{Bus સરળ પણ સિંગલ-ફેઇલ્યુર-પ્રોન}
\end{mnemonicbox}
\end{solutionbox}

\questionmarks{3(બ OR)}{4}{Serial અને Parallel Transmission વચ્ચેનો તફાવત જણાવો.}

\begin{solutionbox}
\textbf{જવાબ:}

\begin{answertable}{Serial vs Parallel Transmission}
\begin{tabulary}{\linewidth}{|L|L|L|}
\hline
\textbf{પાસાં} & \textbf{Serial Transmission} & \textbf{Parallel Transmission} \\
\hline
\textbf{ડેટા પાથ} & સિંગલ કમ્યુનિકેશન લાઇન & એકસાથે બહુવિધ લાઇન્સ \\
\hline
\textbf{સ્પીડ} & ટૂંકા અંતર માટે ધીમું & ટૂંકા અંતર માટે ઝડપી \\
\hline
\textbf{કિંમત} & ઓછી કિંમત & વધારે કિંમત \\
\hline
\textbf{અંતર} & લાંબા અંતર માટે યોગ્ય & ટૂંકા અંતર માટે મર્યાદિત \\
\hline
\end{tabulary}
\end{answertable}

\textbf{લાક્ષણિકતાઓ:}
\begin{itemize}
    \item \textbf{Serial}: બિટ્સ એક પછી એક ટ્રાન્સમિટ થાય
    \item \textbf{Parallel}: બહુવિધ બિટ્સ એકસાથે ટ્રાન્સમિટ થાય
    \item \textbf{ઉપયોગ}: નેટવર્ક માટે Serial, આંતરિક બસ માટે Parallel
\end{itemize}

\begin{mnemonicbox}
\mnemonic{Serial સિંગલ-ફાઇલ, Parallel પ્રોસેસીસ}
\end{mnemonicbox}
\end{solutionbox}

\questionmarks{3(ક OR)}{7}{Transmission Modes સમજાવો.}

\begin{solutionbox}
\textbf{જવાબ:}

\textbf{Transmission Modes વર્ગીકરણ:}

\begin{answertable}{Transmission Modes}
\begin{tabulary}{\linewidth}{|L|L|L|L|}
\hline
\textbf{મોડ} & \textbf{દિશા} & \textbf{ઉદાહરણ} & \textbf{ઉપયોગ} \\
\hline
\textbf{Simplex} & માત્ર એક દિશા & રેડિયો, TV બ્રોડકાસ્ટ & બ્રોડકાસ્ટિંગ \\
\hline
\textbf{Half-duplex} & બંને દિશા, એકસાથે નહીં & વોકી-ટૉકી & વારાફરતી કમ્યુનિકેશન \\
\hline
\textbf{Full-duplex} & બંને દિશા એકસાથે & ટેલિફોન & રિયલ-ટાઇમ કમ્યુનિકેશન \\
\hline
\end{tabulary}
\end{answertable}

\textbf{વિગતવાર સમજૂતી:}

\textbf{1. Simplex Mode:}
\begin{itemize}
    \item \textbf{એકદિશીય}: ડેટા માત્ર એક દિશામાં વહે
    \item \textbf{ઉદાહરણ}: ટેલિવિઝન બ્રોડકાસ્ટિંગ, રેડિયો ટ્રાન્સમિશન
    \item \textbf{ફાયદો}: સરળ અમલીકરણ
    \item \textbf{ગેરફાયદો}: ફીડબેક શક્ય નથી
\end{itemize}

\textbf{2. Half-duplex Mode:}
\begin{itemize}
    \item \textbf{દ્વિદિશીય}: બંને દિશામાં ડેટા વહી શકે, પણ એકસાથે નહીં
    \item \textbf{ઉદાહરણ}: વોકી-ટૉકીઝ, CB રેડિયો
    \item \textbf{ફાયદો}: સિંગલ ચેનલ સાથે બે-દિશીય કમ્યુનિકેશન
    \item \textbf{ગેરફાયદો}: એકસાથે મોકલી અને મેળવી શકાતું નથી
\end{itemize}

\textbf{3. Full-duplex Mode:}
\begin{itemize}
    \item \textbf{એકસાથે દ્વિદિશીય}: બંને દિશામાં એક જ સમયે ડેટા વહે
    \item \textbf{ઉદાહરણ}: ટેલિફોન વાતચીત, આધુનિક નેટવર્ક્સ
    \item \textbf{ફાયદો}: કાર્યક્ષમ રિયલ-ટાઇમ કમ્યુનિકેશન
    \item \textbf{ગેરફાયદો}: વધુ જટિલ અમલીકરણ જરૂરી
\end{itemize}

\begin{mnemonicbox}
\mnemonic{Simplex સિંગલ, Half-duplex અટકે, Full-duplex વહે}
\end{mnemonicbox}
\end{solutionbox}

\questionmarks{4(અ)}{3}{Crossover Ethernet Cable દોરો.}

\begin{solutionbox}
\textbf{જવાબ:}

\textbf{Crossover Cable વાયરિંગ ડાયાગ્રામ:}

\begin{answerdiagram}{Crossover Cable Wiring}
\begin{tikzpicture}
    % RJ-45 Connector A
    \node (A1) at (0,7) {Pin 1 (WO)};
    \node (A2) at (0,6) {Pin 2 (O)};
    \node (A3) at (0,5) {Pin 3 (WG)};
    \node (A4) at (0,4) {Pin 4 (B)};
    \node (A5) at (0,3) {Pin 5 (WB)};
    \node (A6) at (0,2) {Pin 6 (G)};
    \node (A7) at (0,1) {Pin 7 (WBr)};
    \node (A8) at (0,0) {Pin 8 (Br)};
    
    \node at (0,8) {\textbf{Connector A}};

    % RJ-45 Connector B
    \node (B1) at (5,7) {Pin 1 (WO)};
    \node (B2) at (5,6) {Pin 2 (O)};
    \node (B3) at (5,5) {Pin 3 (WG)};
    \node (B4) at (5,4) {Pin 4 (B)};
    \node (B5) at (5,3) {Pin 5 (WB)};
    \node (B6) at (5,2) {Pin 6 (G)};
    \node (B7) at (5,1) {Pin 7 (WBr)};
    \node (B8) at (5,0) {Pin 8 (Br)};
    
    \node at (5,8) {\textbf{Connector B}};
    
    % Connections for Crossover (1-3, 2-6, 3-1, 6-2)
    \draw [thick, blue] (A1.east) -- (B3.west);
    \draw [thick, orange] (A2.east) -- (B6.west);
    \draw [thick, green] (A3.east) -- (B1.west);
    \draw [thick, purple] (A4.east) -- (B4.west);
    \draw [thick, cyan] (A5.east) -- (B5.west);
    \draw [thick, red] (A6.east) -- (B2.west);
    \draw [thick, brown] (A7.east) -- (B7.west);
    \draw [thick, black] (A8.east) -- (B8.west);
\end{tikzpicture}
\end{answerdiagram}

\textbf{મુખ્ય મુદ્દાઓ:}
\begin{itemize}
    \item \textbf{હેતુ}: સમાન ઉપકરણો વચ્ચે સીધું કનેક્શન
    \item \textbf{Crossover}: ટ્રાન્સમિટ અને રિસીવ પેર્સ અદલાબદલી
    \item \textbf{ઉપયોગ}: PC થી PC, Switch થી Switch કનેક્શન્સ
\end{itemize}

\begin{mnemonicbox}
\mnemonic{Cross કમ્પ્યુટર્સને કનેક્ટ કરે}
\end{mnemonicbox}
\end{solutionbox}

\questionmarks{4(બ)}{4}{IPv4 અને IPv6 વચ્ચેનો તફાવત જણાવો.}

\begin{solutionbox}
\textbf{જવાબ:}

\begin{answertable}{IPv4 vs IPv6}
\begin{tabulary}{\linewidth}{|L|L|L|}
\hline
\textbf{લક્ષણ} & \textbf{IPv4} & \textbf{IPv6} \\
\hline
\textbf{એડ્રેસ સાઇઝ} & 32 બિટ્સ & 128 બિટ્સ \\
\hline
\textbf{એડ્રેસ ફોર્મેટ} & ડોટેડ ડેસિમલ & હેક્સાડેસિમલ કોલોન \\
\hline
\textbf{એડ્રેસ સ્પેસ} & 4.3 બિલિયન એડ્રેસ & 340 અનડેસિલિયન એડ્રેસ \\
\hline
\textbf{હેડર સાઇઝ} & વેરિયેબલ (20-60 બાઇટ્સ) & ફિક્સ્ડ (40 બાઇટ્સ) \\
\hline
\end{tabulary}
\end{answertable}

\textbf{મુખ્ય તફાવતો:}
\begin{itemize}
    \item \textbf{IPv4 ઉદાહરણ}: 192.168.1.1
    \item \textbf{IPv6 ઉદાહરણ}: 2001:0db8:85a3:0000:0000:8a2e:0370:7334
    \item \textbf{સુરક્ષા}: IPv6 માં બિલ્ટ-ઇન IPSec સપોર્ટ
    \item \textbf{NAT}: IPv4 ને NAT જરૂરી, IPv6 જરૂરિયાત દૂર કરે
\end{itemize}

\begin{mnemonicbox}
\mnemonic{IPv4 ચાર-બિલિયન, IPv6 છ-ગણાં-વધારે}
\end{mnemonicbox}
\end{solutionbox}

\questionmarks{4(ક)}{7}{OSI મોડલની સુઘડ અને સ્વચ્છ આકૃતિ દોરો અને Physical Layer અને Data Link Layer ની કાર્યક્ષમતા લખો.}

\begin{solutionbox}
\textbf{જવાબ:}

\textbf{OSI મોડલ ડાયાગ્રામ:}

\begin{answerdiagram}{OSI Model}
\begin{tikzpicture}[gtu tree]
    % Simple stack
    \node [gtu block] (app) at (0,6) {Application Layer (7)};
    \node [gtu block] (pres) at (0,5) {Presentation Layer (6)};
    \node [gtu block] (sess) at (0,4) {Session Layer (5)};
    \node [gtu block] (trans) at (0,3) {Transport Layer (4)};
    \node [gtu block] (net) at (0,2) {Network Layer (3)};
    \node [gtu block] (data) at (0,1) {Data Link Layer (2)};
    \node [gtu block] (phys) at (0,0) {Physical Layer (1)};
    
    \draw [->] (app) -- (pres);
    \draw [->] (pres) -- (sess);
    \draw [->] (sess) -- (trans);
    \draw [->] (trans) -- (net);
    \draw [->] (net) -- (data);
    \draw [->] (data) -- (phys);
\end{tikzpicture}
\end{answerdiagram}

\textbf{લેયર કાર્યો:}

\begin{answertable}{ટોચના બે લેયર કાર્યો}
\begin{tabulary}{\linewidth}{|L|L|L|}
\hline
\textbf{લેયર} & \textbf{કાર્ય} & \textbf{ઉદાહરણ} \\
\hline
\textbf{Physical (Layer 1)} & માધ્યમ પર બિટ ટ્રાન્સમિશન & કેબલ્સ, હબ્સ, રિપીટર્સ \\
\hline
\textbf{Data Link (Layer 2)} & નજીકના નોડ્સ વચ્ચે ફ્રેમ ડિલિવરી & સ્વિચ, MAC એડ્રેસ \\
\hline
\end{tabulary}
\end{answertable}

\textbf{Physical Layer કાર્યો:}
\begin{itemize}
    \item \textbf{બિટ ટ્રાન્સમિશન}: ડેટાને ઇલેક્ટ્રિકલ/ઑપ્ટિકલ સિગ્નલમાં રૂપાંતરિત કરે
    \item \textbf{માધ્યમ સ્પેસિફિકેશન}: કેબલ પ્રકારો અને કનેક્ટર્સ વ્યાખ્યાયિત કરે
    \item \textbf{સિગ્નલ એન્કોડિંગ}: બિટ્સ કેવી રીતે રજૂ કરવા નક્કી કરે
    \item \textbf{ટ્રાન્સમિશન રેટ}: ડેટા સ્પીડ નિયંત્રિત કરે
\end{itemize}

\textbf{Data Link Layer કાર્યો:}
\begin{itemize}
    \item \textbf{ફ્રેમ ફોર્મેશન}: બિટ્સને ફ્રેમ્સમાં ગોઠવે
    \item \textbf{એરર ડિટેક્શન}: ટ્રાન્સમિશન એરર્સ ઓળખે
    \item \textbf{ફ્લો કંટ્રોલ}: ડેટા ટ્રાન્સમિશન રેટ મેનેજ કરે
    \item \textbf{MAC એડ્રેસિંગ}: લોકલ ડિલિવરી માટે હાર્ડવેર એડ્રેસ ઉપયોગ કરે
\end{itemize}

\begin{mnemonicbox}
\mnemonic{Physical ધકેલે, Data-Link પહોંચાડે}
\end{mnemonicbox}
\end{solutionbox}

\questionmarks{4(અ OR)}{3}{Time Division Multiplexing સમજાવો.}

\begin{solutionbox}
\textbf{જવાબ:}

\textbf{Time Division Multiplexing (TDM):}

\begin{answerdiagram}{TDM Time Slots}
\begin{tikzpicture}[scale=0.8]
    % Channel A
    \fill [blue!20] (0,2) rectangle (2,3); \node at (1,2.5) {A1};
    \fill [blue!20] (6,2) rectangle (8,3); \node at (7,2.5) {A2};
    \node at (-1, 2.5) {ચેનલ A};

    % Channel B
    \fill [red!20] (2,1) rectangle (4,2); \node at (3,1.5) {B1};
    \fill [red!20] (8,1) rectangle (10,2); \node at (9,1.5) {B2};
    \node at (-1, 1.5) {ચેનલ B};
    
    % Channel C
    \fill [green!20] (4,0) rectangle (6,1); \node at (5,0.5) {C1};
    \fill [green!20] (10,0) rectangle (12,1); \node at (11,0.5) {C2};
    \node at (-1, 0.5) {ચેનલ C};
    
    % Draw timeline/frames
    \draw [->] (0,-1) -- (13,-1) node[right] {સમય};
    
    % Frame 1
    \draw [dashed] (0,-0.5) -- (0,3.5);
    \draw [dashed] (6,-0.5) -- (6,3.5);
    \node at (3,-1.5) {ફ્રેમ 1};
    
    % Frame 2
    \draw [dashed] (12,-0.5) -- (12,3.5);
    \node at (9,-1.5) {ફ્રેમ 2};
    
    % Multiplexed Output Stream
    \fill [blue!20] (0,-3) rectangle (2,-2); \node at (1,-2.5) {A1};
    \fill [red!20] (2,-3) rectangle (4,-2); \node at (3,-2.5) {B1};
    \fill [green!20] (4,-3) rectangle (6,-2); \node at (5,-2.5) {C1};
    \fill [blue!20] (6,-3) rectangle (8,-2); \node at (7,-2.5) {A2};
    \fill [red!20] (8,-3) rectangle (10,-2); \node at (9,-2.5) {B2};
    \fill [green!20] (10,-3) rectangle (12,-2); \node at (11,-2.5) {C2};
    \node at (-1, -2.5) {આઉટપુટ};
    
\end{tikzpicture}
\end{answerdiagram}

\textbf{TDM લાક્ષણિકતાઓ:}
\begin{itemize}
    \item \textbf{ટાઇમ સ્લોટ્સ}: દરેક ચેનલને સમર્પિત સમય અવધિ મળે
    \item \textbf{સિંક્રોનાઇઝેશન}: બધી ચેનલો સિંક્રોનાઇઝ હોવી જોઈએ
    \item \textbf{બેન્ડવિડ્થ શેરિંગ}: બહુવિધ ચેનલો વચ્ચે સિંગલ હાઇ-સ્પીડ લિંક શેર
\end{itemize}

\begin{mnemonicbox}
\mnemonic{ટાઇમ વળતા લે}
\end{mnemonicbox}
\end{solutionbox}

\questionmarks{4(બ OR)}{4}{નેટવર્કિંગ ઉપકરણના પ્રકારોની યાદી બનાવો અને કોઈપણ એક સમજાવો.}

\begin{solutionbox}
\textbf{જવાબ:}

\textbf{નેટવર્કિંગ ઉપકરણો:}

\begin{answertable}{નેટવર્ક ઉપકરણો}
\begin{tabulary}{\linewidth}{|L|L|L|}
\hline
\textbf{ઉપકરણ} & \textbf{લેયર} & \textbf{કાર્ય} \\
\hline
\textbf{Hub} & Physical & સિગ્નલ રિપીટર \\
\hline
\textbf{Switch} & Data Link & ફ્રેમ સ્વિચિંગ \\
\hline
\textbf{Router} & Network & પેકેટ રાઉટિંગ \\
\hline
\textbf{Bridge} & Data Link & નેટવર્ક સેગમેન્ટેશન \\
\hline
\end{tabulary}
\end{answertable}

\textbf{Switch સમજૂતી:}
\begin{itemize}
    \item \textbf{કાર્ય}: MAC એડ્રેસ આધારે ફ્રેમ્સ ફોરવર્ડ કરે
    \item \textbf{લર્નિંગ}: MAC એડ્રેસ ટેબલ ડાયનેમિકલી બનાવે
    \item \textbf{કોલિઝન ડોમેન}: દરેક પોર્ટ અલગ કોલિઝન ડોમેન બનાવે
    \item \textbf{ફુલ-ડુપ્લેક્સ}: દરેક પોર્ટ પર એકસાથે મોકલી/મેળવી શકે
\end{itemize}

\textbf{ફાયદા:}
\begin{itemize}
    \item \textbf{બેન્ડવિડ્થ}: દરેક પોર્ટ માટે સંપૂર્ણ બેન્ડવિડ્થ
    \item \textbf{સુરક્ષા}: ફ્રેમ્સ માત્ર ઇચ્છિત પ્રાપ્તકર્તાને મોકલાય
    \item \textbf{કોલિઝન}: કોલિઝન દૂર કરે
\end{itemize}

\begin{mnemonicbox}
\mnemonic{Switch સ્માર્ટલી મોકલે}
\end{mnemonicbox}
\end{solutionbox}

\questionmarks{4(ક OR)}{7}{Computer Network શું છે? Computer Network ના પ્રકારો સમજાવો.}

\begin{solutionbox}
\textbf{જવાબ:}

\textbf{Computer Network વ્યાખ્યા:}
આંતરસંબંધિત સ્વતંત્ર કમ્પ્યુટર્સનો સંગ્રહ કે જે કમ્યુનિકેટ કરી શકે અને સંસાધનો શેર કરી શકે.

\textbf{Computer Networks ના પ્રકારો:}

\begin{answertable}{નેટવર્ક પ્રકારો}
\begin{tabulary}{\linewidth}{|L|L|L|L|}
\hline
\textbf{પ્રકાર} & \textbf{કવરેજ} & \textbf{ઉદાહરણ} & \textbf{લાક્ષણિકતાઓ} \\
\hline
\textbf{LAN} & લોકલ એરિયા (બિલ્ડિંગ) & ઑફિસ નેટવર્ક & હાઇ સ્પીડ, લો કોસ્ટ \\
\hline
\textbf{MAN} & મેટ્રોપોલિટન એરિયા (શહેર) & શહેરવ્યાપી નેટવર્ક & મીડિયમ સ્પીડ, મોડરેટ કોસ્ટ \\
\hline
\textbf{WAN} & વાઇડ એરિયા (દેશ/વિશ્વ) & ઇન્ટરનેટ & ઓછી સ્પીડ, વધારે કિંમત \\
\hline
\end{tabulary}
\end{answertable}

\textbf{વિગતવાર સમજૂતી:}

\textbf{1. Local Area Network (LAN):}
\begin{itemize}
    \item \textbf{કવરેજ}: સિંગલ બિલ્ડિંગ કે કેમ્પસ
    \item \textbf{સ્પીડ}: હાઇ (100 Mbps થી 10 Gbps)
    \item \textbf{ટેકનોલોજી}: Ethernet, Wi-Fi
    \item \textbf{માલિકી}: સિંગલ સંસ્થા
\end{itemize}

\textbf{2. Metropolitan Area Network (MAN):}
\begin{itemize}
    \item \textbf{કવરેજ}: શહેર કે મેટ્રોપોલિટન એરિયા
    \item \textbf{સ્પીડ}: મીડિયમ (10-100 Mbps)
    \item \textbf{ટેકનોલોજી}: ફાઇબર ઑપ્ટિક, માઇક્રોવેવ
    \item \textbf{ઉદાહરણ}: કેબલ TV નેટવર્ક્સ
\end{itemize}

\textbf{3. Wide Area Network (WAN):}
\begin{itemize}
    \item \textbf{કવરેજ}: દેશો કે ખંડો
    \item \textbf{સ્પીડ}: વેરિયેબલ (ટેકનોલોજી પર આધાર)
    \item \textbf{ટેકનોલોજી}: સેટેલાઇટ, લીઝ્ડ લાઇન્સ
    \item \textbf{ઉદાહરણ}: ઇન્ટરનેટ, કોર્પોરેટ નેટવર્ક્સ
\end{itemize}

\textbf{નેટવર્ક ફાયદા:}
\begin{itemize}
    \item \textbf{સંસાધન શેરિંગ}: ફાઇલો, પ્રિન્ટર્સ, એપ્લિકેશન્સ
    \item \textbf{કમ્યુનિકેશન}: ઇમેઇલ, મેસેજિંગ, વિડિયો કોન્ફરન્સિંગ
    \item \textbf{કિંમત ઘટાડો}: શેર કરેલ સંસાધનો કિંમત ઘટાડે
    \item \textbf{ડેટા બેકઅપ}: કેન્દ્રીકૃત બેકઅપ સિસ્ટમ્સ
\end{itemize}

\begin{mnemonicbox}
\mnemonic{લોકલ પ્રેમ કરે, મેટ્રો મેનેજ કરે, વાઇડ ભટકે}
\end{mnemonicbox}
\end{solutionbox}

\questionmarks{5(અ)}{3}{ઇન્ફર્મેશન સિક્યુરિટીની જરૂરિયાત સમજાવો.}

\begin{solutionbox}
\textbf{જવાબ:}

\textbf{ઇન્ફર્મેશન સિક્યુરિટી જરૂરિયાતો:}

\begin{answertable}{સુરક્ષા જરૂરિયાતો}
\begin{tabulary}{\linewidth}{|L|L|L|}
\hline
\textbf{ખતરો} & \textbf{અસર} & \textbf{સુરક્ષા જરૂરિયાત} \\
\hline
\textbf{ડેટા ચોરી} & આર્થિક નુકસાન & ગોપનીયતા (Confidentiality) \\
\hline
\textbf{અનધિકૃત ઍક્સેસ} & પ્રાઇવસી ભંગ & એક્સેસ કંટ્રોલ \\
\hline
\textbf{સિસ્ટમ હુમલા} & સેવા વિક્ષેપ & ઉપલબ્ધતા (Availability) \\
\hline
\end{tabulary}
\end{answertable}

\textbf{મુખ્ય આવશ્યકતાઓ:}
\begin{itemize}
    \item \textbf{Confidentiality}: સંવેદનશીલ માહિતીને અનધિકૃત ઍક્સેસથી સુરક્ષિત કરવી
    \item \textbf{Data Protection}: મૂલ્યવાન ડેટાના નુકસાન કે ભ્રષ્ટાચારને અટકાવવું
    \item \textbf{Business Continuity}: સિસ્ટમ્સ કાર્યરત રહે તે સુનિશ્ચિત કરવું
\end{itemize}

\begin{mnemonicbox}
\mnemonic{સુરક્ષા સંવેદનશીલ સિસ્ટમ્સ સાચવે}
\end{mnemonicbox}
\end{solutionbox}

\questionmarks{5(બ)}{4}{Fiber Optic Cable ના ફાયદા અને ગેરફાયદા લખો.}

\begin{solutionbox}
\textbf{જવાબ:}

\begin{answertable}{Fiber Optic ફાયદા અને ગેરફાયદા}
\begin{tabulary}{\linewidth}{|L|L|}
\hline
\textbf{ફાયદા} & \textbf{ગેરફાયદા} \\
\hline
\textbf{હાઇ બેન્ડવિડ્થ} & \textbf{વધારે કિંમત} \\
\hline
\textbf{EMI સામે રક્ષણ} & \textbf{મુશ્કેલ ઇન્સ્ટોલેશન} \\
\hline
\textbf{લાંબું અંતર} & \textbf{નાજુક પ્રકૃતિ} \\
\hline
\textbf{સુરક્ષિત ટ્રાન્સમિશન} & \textbf{વિશિષ્ટ સાધનો} \\
\hline
\end{tabulary}
\end{answertable}

\textbf{ફાયદા:}
\begin{itemize}
    \item \textbf{સ્પીડ}: સૌથી વધુ ડેટા ટ્રાન્સમિશન રેટ
    \item \textbf{અંતર}: સિગ્નલ નબળા પડ્યા વિના લાંબા અંતર સુધી
    \item \textbf{સુરક્ષા}: ટેપ કરવું મુશ્કેલ, સુરક્ષિત કમ્યુનિકેશન પૂરું પાડે
\end{itemize}

\textbf{ગેરફાયદા:}
\begin{itemize}
    \item \textbf{કિંમત}: મોંઘા કેબલ અને સાધનો
    \item \textbf{ઇન્સ્ટોલેશન}: કુશળ ટેકનિશિયનની જરૂર
    \item \textbf{જાળવણી}: રિપેર અને સ્પ્લાઇસ કરવું મુશ્કેલ
\end{itemize}

\begin{mnemonicbox}
\mnemonic{Fiber ઝડપી પણ નાજુક}
\end{mnemonicbox}
\end{solutionbox}

\questionmarks{5(ક)}{7}{Attack ના પ્રકારોની યાદી બનાવો. અને કોઈપણ બે web based attack સમજાવો.}

\begin{solutionbox}
\textbf{જવાબ:}

\textbf{હુમલાના પ્રકારો:}

\begin{answertable}{હુમલાની શ્રેણીઓ}
\begin{tabulary}{\linewidth}{|L|L|L|}
\hline
\textbf{શ્રેણી} & \textbf{હુમલાના પ્રકાર} & \textbf{ટાર્ગેટ} \\
\hline
\textbf{Web-based} & SQL Injection, XSS, CSRF & વેબ એપ્લિકેશન્સ \\
\hline
\textbf{Network} & DoS, DDoS, Man-in-Middle & નેટવર્ક ઇન્ફ્રાસ્ટ્રક્ચર \\
\hline
\textbf{Malware} & Virus, Trojan, Ransomware & સિસ્ટમ્સ અને ડેટા \\
\hline
\textbf{Social} & Phishing, Social Engineering & માનવ યુઝર્સ \\
\hline
\end{tabulary}
\end{answertable}

\textbf{Web-based Attacks સમજૂતી:}

\textbf{1. SQL Injection:}
\begin{itemize}
    \item \textbf{પદ્ધતિ}: વેબ એપ્લિકેશન ઇનપુટ્સમાં દૂષિત SQL કોડ દાખલ કરવો
    \item \textbf{અસર}: અનધિકૃત ડેટાબેસ ઍક્સેસ, ડેટા ચોરી
    \item \textbf{ઉદાહરણ}: લોગિન ફોર્મમાં \code{'; DROP TABLE users;--} દાખલ કરવું
    \item \textbf{નિવારણ}: ઇનપુટ વેલિડેશન, પેરામીટરાઇઝ્ડ ક્વેરીઝ
    \item \textbf{ગંભીરતા}: આખા ડેટાબેસ સાથે ચેડા કરી શકે
\end{itemize}

\textbf{2. Cross-Site Scripting (XSS):}
\begin{itemize}
    \item \textbf{પદ્ધતિ}: વેબ પેજીસમાં દૂષિત સ્ક્રિપ્ટ્સ ઇન્જેક્ટ કરવી
    \item \textbf{અસર}: સેશન હાઇજેકિંગ, કૂકી ચોરી, પેજ ડિફેસમેન્ટ
    \item \textbf{પ્રકાર}: Stored XSS, Reflected XSS, DOM-based XSS
    \item \textbf{નિવારણ}: ઇનપુટ સેનિટાઇઝેશન, આઉટપુટ એન્કોડિંગ
    \item \textbf{ટાર્ગેટ}: ચેડા થયેલી વેબસાઇટની મુલાકાત લેતા યુઝર્સ
\end{itemize}

\begin{mnemonicbox}
\mnemonic{SQL ચોરે, XSS સ્ક્રિપ્ટ્સ વાપરે}
\end{mnemonicbox}
\end{solutionbox}

\questionmarks{5(અ OR)}{3}{Confidentiality, Integrity અને Availability સમજાવો.}

\begin{solutionbox}
\textbf{જવાબ:}

\textbf{CIA Triad ઘટકો:}

\begin{answertable}{CIA Triad}
\begin{tabulary}{\linewidth}{|L|L|L|}
\hline
\textbf{ઘટક} & \textbf{વ્યાખ્યા} & \textbf{ઉદાહરણ} \\
\hline
\textbf{Confidentiality} & માહિતી માત્ર અધિકૃત યુઝર્સ જોઈ શકે & એન્ક્રિપ્શન, એક્સેસ કંટ્રોલ \\
\hline
\textbf{Integrity} & ડેટાની ચોકસાઈ અને પૂર્ણતા & ચેકસમ, ડિજિટલ હસ્તાક્ષર \\
\hline
\textbf{Availability} & સિસ્ટમ્સ જ્યારે જરૂર હોય ત્યારે ઉપલબ્ધ & રીડન્ડન્સી, બેકઅપ સિસ્ટમ્સ \\
\hline
\end{tabulary}
\end{answertable}

\textbf{મુખ્ય ખ્યાલો:}
\begin{itemize}
    \item \textbf{Confidentiality}: માહિતીને અનધિકૃત યુઝર્સથી ગુપ્ત રાખે છે
    \item \textbf{Integrity}: ખાતરી કરે છે કે ડેટામાં અનધિકૃત ફેરફાર થયો નથી
    \item \textbf{Availability}: ખાતરી કરે છે કે સિસ્ટમ્સ જરૂર પડે ત્યારે કાર્યરત છે
\end{itemize}

\begin{mnemonicbox}
\mnemonic{CIA માહિતીને સંપૂર્ણ સુરક્ષિત કરે}
\end{mnemonicbox}
\end{solutionbox}

\questionmarks{5(બ OR)}{4}{નીચેના IP addresses નો ક્લાસ શોધો.}

\begin{solutionbox}
\textbf{જવાબ:}

\textbf{IP Address Class ઓળખ:}

\begin{answertable}{IP Class Finder}
\begin{tabulary}{\linewidth}{|L|L|L|L|}
\hline
\textbf{IP Address} & \textbf{First Octet} & \textbf{Class} & \textbf{Range} \\
\hline
\textbf{192.12.44.12} & 192 & Class C & 192-223 \\
\hline
\textbf{123.77.42.213} & 123 & Class A & 1-126 \\
\hline
\textbf{190.65.22.15} & 190 & Class B & 128-191 \\
\hline
\textbf{10.0.0.11} & 10 & Class A (Private) & 1-126 \\
\hline
\end{tabulary}
\end{answertable}

\textbf{ક્લાસ લાક્ષણિકતાઓ:}
\begin{itemize}
    \item \textbf{Class A}: 1-126 (પ્રથમ બીટ 0), મોટા નેટવર્ક્સ
    \item \textbf{Class B}: 128-191 (પ્રથમ બે બીટ 10), મધ્યમ નેટવર્ક્સ
    \item \textbf{Class C}: 192-223 (પ્રથમ ત્રણ બીટ 110), નાના નેટવર્ક્સ
    \item \textbf{Private IPs}: 10.x.x.x, 172.16-31.x.x, 192.168.x.x
\end{itemize}

\begin{mnemonicbox}
\mnemonic{A અદ્ભુત, B બહેતર, C કોમ્પેક્ટ}
\end{mnemonicbox}
\end{solutionbox}

\questionmarks{5(ક OR)}{7}{Cryptography સમજાવો.}

\begin{solutionbox}
\textbf{જવાબ:}

\textbf{Cryptography વ્યાખ્યા:}
માહિતીને એન્કોડ કરીને સુરક્ષિત કરવાની વિજ્ઞાન જેથી માત્ર અધિકૃત પક્ષો જ તેને ઍક્સેસ કરી શકે.

\textbf{Cryptography પ્રકારો:}

\begin{answertable}{Crypto પ્રકારો}
\begin{tabulary}{\linewidth}{|L|L|L|L|}
\hline
\textbf{પ્રકાર} & \textbf{કી ઉપયોગ} & \textbf{ઉદાહરણ} & \textbf{ઉપયોગ} \\
\hline
\textbf{Symmetric} & સિંગલ શેર્ડ કી & DES, AES & ઝડપી બલ્ક એન્ક્રિપ્શન \\
\hline
\textbf{Asymmetric} & પબ્લિક-પ્રાઇવેટ કી જોડી & RSA, ECC & ડિજિટલ સહી, કી એક્સચેન્જ \\
\hline
\textbf{Hash Functions} & વન-વે ટ્રાન્સફોર્મેશન & MD5, SHA & ડેટા અખંડિતતા, પાસવર્ડ્સ \\
\hline
\end{tabulary}
\end{answertable}

\textbf{ક્રિપ્ટોગ્રાફિક પ્રક્રિયા:}

\begin{answerdiagram}{Cryptography Flow}
\begin{tikzpicture}[auto, node distance=2cm]
    \node [gtu block] (plain) {પ્લેઇનટેક્સ્ટ};
    \node [gtu block, right=of plain] (encrypt) {એન્ક્રિપ્શન};
    \node [gtu block, right=of encrypt] (cipher) {સાઈફરટેક્સ્ટ};
    \node [gtu block, right=of cipher] (decrypt) {ડિક્રિપ્શન};
    \node [gtu block, right=of decrypt] (plain2) {પ્લેઇનટેક્સ્ટ};
    
    \node [gtu block, above=of encrypt] (key) {કી};
    
    \draw [gtu arrow] (plain) -- (encrypt);
    \draw [gtu arrow] (encrypt) -- (cipher);
    \draw [gtu arrow] (cipher) -- (decrypt);
    \draw [gtu arrow] (decrypt) -- (plain2);
    
    \draw [gtu arrow] (key) -- (encrypt);
    \draw [gtu arrow] (key) -| (decrypt);
\end{tikzpicture}
\end{answerdiagram}

\textbf{વિગતવાર સમજૂતી:}

\textbf{1. Symmetric Cryptography:}
\begin{itemize}
    \item \textbf{સિંગલ કી}: એન્ક્રિપ્શન અને ડિક્રિપ્શન માટે સમાન કી
    \item \textbf{ઝડપ}: મોટા પ્રમાણના ડેટા માટે ઝડપી પ્રક્રિયા
    \item \textbf{પડકાર}: સુરક્ષિત કી વિતરણ
\end{itemize}

\textbf{2. Asymmetric Cryptography:}
\begin{itemize}
    \item \textbf{કી જોડીઓ}: પબ્લિક કી (શેર કરી શકાય) અને પ્રાઇવેટ કી (ગુપ્ત)
    \item \textbf{ડિજિટલ સહીઓ}: અધિકૃતતા અને અસ્વીકારક્ષમતા સાબિત કરે
\end{itemize}

\begin{mnemonicbox}
\mnemonic{Cryptography કોડેડ કમ્યુનિકેશન બનાવે}
\end{mnemonicbox}
\end{solutionbox}

\end{document}
