\documentclass[10pt,a4paper]{article}

% content/resources/templates/preamble.tex
\usepackage[margin=0.6in]{geometry}
\author{Milav Dabgar}
\usepackage{amsmath,amssymb,amsthm}
\usepackage{booktabs}
\usepackage{multirow}
\usepackage{xcolor}
\usepackage{tcolorbox}
\tcbuselibrary{breakable,skins}
\usepackage[colorlinks=true,linkcolor=blue]{hyperref}
\usepackage{titlesec}
\usepackage{enumitem}
\usepackage{tikz}
\usepackage{pgfplots}
\usepackage{circuitikz}
\usepackage[version=4]{mhchem}
\usepackage{longtable}
\usepackage{array}
\usepackage{float}
\usepackage{caption}
\usepackage{listings}

\lstset{
  basicstyle=\small\ttfamily,
  breaklines=true,
  breakatwhitespace=false,
  postbreak=\mbox{\textcolor{red}{$\hookrightarrow$}\space},
  float=false,
  numbers=left,
  numberstyle=\tiny\color{gray},
  numbersep=10pt,
  xleftmargin=2em,
  keywordstyle=\color{blue},
  commentstyle=\color{green!60!black},
  stringstyle=\color{purple},
  backgroundcolor=\color{gray!5},
  showstringspaces=false,
  tabsize=2,
  captionpos=b,
  keepspaces=true,
  columns=flexible
}

\pgfplotsset{compat=1.18}
\usetikzlibrary{shapes,arrows,positioning,calc,patterns,decorations.pathmorphing,decorations.markings,arrows.meta}

% Color scheme
\definecolor{headcolor}{RGB}{0,102,204}
\definecolor{keycolor}{RGB}{220,20,60}
\definecolor{solutioncolor}{RGB}{34,139,34}
\definecolor{mnemoniccolor}{RGB}{148,0,211}
\definecolor{codecolor}{RGB}{0,0,100}

% Spacing
\setlength{\parskip}{3pt}
\setlist[itemize]{nosep}
\setlist[enumerate]{nosep}

% Title formatting
\titleformat{\section}{\Large\bfseries\color{headcolor}}{\thesection}{1em}{}
\titleformat{\subsection}{\large\bfseries\color{headcolor}}{\thesubsection}{1em}{}

% Pandoc tightlist compatibility
\providecommand{\tightlist}{%
  \setlength{\itemsep}{0pt}\setlength{\parskip}{0pt}}

% Pandoc longtable compatibility
\newcounter{none}
\def\thenone{}


% content/resources/templates/gujarati-boxes.tex
\usepackage{fontspec}
\usepackage{polyglossia}

% Set Gujarati as main language (document is primarily in Gujarati)
% Note: gloss-gujarati.ldf doesn't exist in polyglossia, but it will use hyphenation patterns
\setdefaultlanguage{gujarati}
\setotherlanguage{english}

% Configure Gujarati font properly
% Use Language=Default to prevent polyglossia from trying to add language-specific features
% that don't exist for Gujarati, which causes "empty feature" warnings
\newfontfamily\gujaratifont[Script=Gujarati,AutoFakeBold=2.5,AutoFakeSlant=0.3]{Noto Sans Gujarati}
\setmainfont[Script=Gujarati,AutoFakeBold=2.5,AutoFakeSlant=0.3]{Noto Sans Gujarati}
% Use Noto Sans Gujarati for monospace to support Gujarati in text
\setmonofont[Scale=0.9]{Noto Sans Gujarati}

% Configure English to use the same font
\newfontfamily\englishfont[Script=Gujarati,AutoFakeBold=2.5,AutoFakeSlant=0.3]{Noto Sans Gujarati}

% Translations for polyglossia
\gappto\captionsgujarati{
  \renewcommand{\tablename}{કોષ્ટક}
  \renewcommand{\figurename}{આકૃતિ}
}

% Helper for TikZ nodes to ensure Gujarati font
\newcommand{\gu}[1]{{\gujaratifont #1}}

% Custom environments
\newtcolorbox{solutionbox}{
    breakable,
    enhanced,
    colback=solutioncolor!5!white,
    colframe=solutioncolor!75!black,
    fonttitle=\bfseries,
    title=જવાબ
}

\newtcolorbox{solutionboxnobreak}{
 colback=solutioncolor!5!white,
 colframe=solutioncolor!75!black,
 fonttitle=\bfseries,
 title=જવાબ
}

\newtcolorbox{keyformula}{
 breakable,
 enhanced,
 colback=keycolor!5!white,
 colframe=keycolor!75!black,
 fonttitle=\bfseries,
 title=રાસાયણિક સમીકરણ/સૂત્ર
}

\newtcolorbox{mnemonicbox}{
 breakable,
 enhanced,
 colback=mnemoniccolor!5!white,
 colframe=mnemoniccolor!75!black,
 fonttitle=\bfseries,
 title=મેમરી ટ્રીક
}


\begin{document}

\begin{center}
{\Huge\bfseries\color{headcolor} Subject Name (Gujarati)}\\[5pt]
{\LARGE 4311602 -- Summer 2024}\\[3pt]
{\large Semester 1 Study Material}\\[3pt]
{\normalsize\textit{Detailed Solutions and Explanations}}
\end{center}

\vspace{10pt}

\subsection*{પ્રશ્ન 1(અ) [3
ગુણ]}\label{uxaaauxab0uxab6uxaa8-1uxa85-3-uxa97uxaa3}

\textbf{નીચેની મુદ્દાઓ વ્યાખ્યાયિત કરો:} \textbf{1. ડેટા} \textbf{2. માહિતી}
\textbf{3. જ્ઞાન}

\begin{solutionbox}

\textbf{ડેટા, માહિતી અને જ્ઞાનની વ્યાખ્યાઓ:}

{\def\LTcaptype{none} % do not increment counter
\begin{longtable}[]{@{}ll@{}}
\toprule\noalign{}
શબ્દ & વ્યાખ્યા \\
\midrule\noalign{}
\endhead
\bottomrule\noalign{}
\endlastfoot
\textbf{ડેટા} & કાચા તથ્યો અને આંકડાઓ જેમાં અર્થ અથવા સંદર્ભ નથી \\
\textbf{માહિતી} & પ્રોસેસ કરેલો ડેટા જે અર્થપૂર્ણ અને ઉપયોગી હોય \\
\textbf{જ્ઞાન} & અનુભવ અને સમજ સાથે જોડાયેલી માહિતી \\
\end{longtable}
}

\begin{itemize}
\tightlist
\item
  \textbf{ડેટા}: અર્થઘટન વિના મૂળભૂત બિલ્ડિંગ બ્લોક્સ
\item
  \textbf{માહિતી}: અર્થપૂર્ણ સંદર્ભ પ્રદાન કરવા માટે પ્રોસેસ કરેલો ડેટા
\item
  \textbf{જ્ઞાન}: માનવીય અંતર્દૃષ્ટિ અને વિવેક સાથે વધારેલી માહિતી
\end{itemize}

\end{solutionbox}
\begin{mnemonicbox}
``DIK - ડેટા ઈઝ નોલેજના પાયા''

\end{mnemonicbox}
\subsection*{પ્રશ્ન 1(બ) [4
ગુણ]}\label{uxaaauxab0uxab6uxaa8-1uxaac-4-uxa97uxaa3}

\textbf{સંક્ષિપ્તમાં પ્રાથમિક મેમરી સમજાવો.}

\begin{solutionbox}

\textbf{પ્રાથમિક મેમરીની લાક્ષણિકતાઓ:}

{\def\LTcaptype{none} % do not increment counter
\begin{longtable}[]{@{}ll@{}}
\toprule\noalign{}
પાસાં & વિવરણ \\
\midrule\noalign{}
\endhead
\bottomrule\noalign{}
\endlastfoot
\textbf{વ્યાખ્યા} & મુખ્ય મેમરી જે સીપીયુ સાથે સીધું કમ્યુનિકેશન કરે \\
\textbf{એક્સેસ સ્પીડ} & ખૂબ જ ઝડપી એક્સેસ ટાઇમ \\
\textbf{વોલેટિલિટી} & વોલેટાઇલ (પાવર બંધ થતાં ડેટા ગુમ થાય) \\
\textbf{ઉદાહરણો} & RAM, કેશ મેમરી \\
\end{longtable}
}

\begin{itemize}
\tightlist
\item
  \textbf{RAM (રેન્ડમ એક્સેસ મેમરી)}: વર્તમાન પ્રોગ્રામ્સ માટેની મુખ્ય કાર્યકારી મેમરી
\item
  \textbf{કેશ મેમરી}: સીપીયુ અને RAM વચ્ચે અતિ-ઝડપી મેમરી
\item
  \textbf{વોલેટાઇલ પ્રકૃતિ}: કમ્પ્યુટર બંધ થતાં ડેટા અદૃશ્ય થઈ જાય
\item
  \textbf{સીધું સીપીયુ એક્સેસ}: સીપીયુ સીધું ડેટા વાંચી/લખી શકે
\end{itemize}

\end{solutionbox}
\begin{mnemonicbox}
``પ્રાઇમરી ઈઝ ફાસ્ટ બટ ફોરગેટફુલ''

\end{mnemonicbox}
\subsection*{પ્રશ્ન 1(ક) [7
ગુણ]}\label{uxaaauxab0uxab6uxaa8-1uxa95-7-uxa97uxaa3}

\textbf{ઉદાહરણ સાથે રિયલ ટાઇમ OSના પ્રકારો સમજાવો.}

\begin{solutionbox}

\textbf{રિયલ-ટાઇમ ઓપરેટિંગ સિસ્ટમના પ્રકારો:}

{\def\LTcaptype{none} % do not increment counter
\begin{longtable}[]{@{}
  >{\raggedright\arraybackslash}p{(\linewidth - 6\tabcolsep) * \real{0.1795}}
  >{\raggedright\arraybackslash}p{(\linewidth - 6\tabcolsep) * \real{0.3846}}
  >{\raggedright\arraybackslash}p{(\linewidth - 6\tabcolsep) * \real{0.2821}}
  >{\raggedright\arraybackslash}p{(\linewidth - 6\tabcolsep) * \real{0.1538}}@{}}
\toprule\noalign{}
\begin{minipage}[b]{\linewidth}\raggedright
પ્રકાર
\end{minipage} & \begin{minipage}[b]{\linewidth}\raggedright
રિસ્પોન્સ ટાઇમ
\end{minipage} & \begin{minipage}[b]{\linewidth}\raggedright
ઉદાહરણો
\end{minipage} & \begin{minipage}[b]{\linewidth}\raggedright
ઉપયોગ
\end{minipage} \\
\midrule\noalign{}
\endhead
\bottomrule\noalign{}
\endlastfoot
\textbf{હાર્ડ રિયલ-ટાઇમ} & ગેરંટીડ ડેડલાઇન & QNX, VxWorks & મેડિકલ ડિવાઇસ,
એરક્રાફ્ટ \\
\textbf{સોફ્ટ રિયલ-ટાઇમ} & શ્રેષ્ઠ પ્રયાસ ટાઇમિંગ & Windows RT, Linux RT &
મલ્ટીમીડિયા, ગેમિંગ \\
\textbf{ફર્મ રિયલ-ટાઇમ} & ક્યારેક ડેડલાઇન મિસ & Embedded Linux & ઇન્ડસ્ટ્રિયલ
કંટ્રોલ \\
\end{longtable}
}

\begin{center}
\textbf{Mermaid Diagram (Code)}
\begin{verbatim}
{Shaded}
{Highlighting}[]
graph TD
    A[રિયલ{-ટાઇમ OS] {-}{-}{} B[હાર્ડ રિયલ{-}ટાઇમ]}
    A {-{-}{} C[સોફ્ટ રિયલ{-}ટાઇમ]}
    A {-{-}{} D[ફર્મ રિયલ{-}ટાઇમ]}
    B {-{-}{} E[ક્રિટિકલ સિસ્ટમ્સ]}
    C {-{-}{} F[મલ્ટીમીડિયા એપ્સ]}
    D {-{-}{} G[ઇન્ડસ્ટ્રિયલ કંટ્રોલ]}
{Highlighting}
{Shaded}
\end{verbatim}
\end{center}

\begin{itemize}
\tightlist
\item
  \textbf{હાર્ડ રિયલ-ટાઇમ}: ડેડલાઇન ચૂકવાથી સિસ્ટમ ફેઇલ થાય
\item
  \textbf{સોફ્ટ રિયલ-ટાઇમ}: વિલંબિત રિસ્પોન્સ પરફોર્મન્સ ઘટાડે પરંતુ સિસ્ટમ ચાલુ રહે
\item
  \textbf{નિર્ધારિત રિસ્પોન્સ}: અનુમાનિત ટાઇમિંગ વર્તણૂક આવશ્યક
\end{itemize}

\end{solutionbox}
\begin{mnemonicbox}
``HSF - હાર્ડ, સોફ્ટ, ફર્મ ટાઇમિંગ જરૂરિયાતો''

\end{mnemonicbox}
\subsection*{પ્રશ્ન 1(ક OR) [7
ગુણ]}\label{uxaaauxab0uxab6uxaa8-1uxa95-or-7-uxa97uxaa3}

\textbf{Linux આર્કિટેક્ચરનું વર્ણન કરો અને Linux ની કામગીરીના મોડની ચર્ચા કરો.}

\begin{solutionbox}

\textbf{Linux આર્કિટેક્ચર ડાયાગ્રામ:}

\begin{center}
\textbf{Mermaid Diagram (Code)}
\begin{verbatim}
{Shaded}
{Highlighting}[]
graph LR
    A[યુઝર એપ્લિકેશન્સ] {-{-}{} B[સિસ્ટમ લાઇબ્રેરીઓ]}
    B {-{-}{} C[સિસ્ટમ કોલ ઇન્ટરફેસ]}
    C {-{-}{} D[Linux કર્નલ]}
    D {-{-}{} E[હાર્ડવેર લેયર]}
    
    subgraph "કર્નલ સ્પેસ"
    D
    end
    
    subgraph "યુઝર સ્પેસ"
    A
    B
    C
    end
{Highlighting}
{Shaded}
\end{verbatim}
\end{center}

\textbf{Linux ઓપરેશન મોડ્સ:}

{\def\LTcaptype{none} % do not increment counter
\begin{longtable}[]{@{}
  >{\raggedright\arraybackslash}p{(\linewidth - 6\tabcolsep) * \real{0.1471}}
  >{\raggedright\arraybackslash}p{(\linewidth - 6\tabcolsep) * \real{0.2059}}
  >{\raggedright\arraybackslash}p{(\linewidth - 6\tabcolsep) * \real{0.3824}}
  >{\raggedright\arraybackslash}p{(\linewidth - 6\tabcolsep) * \real{0.2647}}@{}}
\toprule\noalign{}
\begin{minipage}[b]{\linewidth}\raggedright
મોડ
\end{minipage} & \begin{minipage}[b]{\linewidth}\raggedright
વિવરણ
\end{minipage} & \begin{minipage}[b]{\linewidth}\raggedright
એક્સેસ લેવલ
\end{minipage} & \begin{minipage}[b]{\linewidth}\raggedright
ઉદાહરણો
\end{minipage} \\
\midrule\noalign{}
\endhead
\bottomrule\noalign{}
\endlastfoot
\textbf{યુઝર મોડ} & પ્રતિબંધિત એક્સેસ & મર્યાદિત અધિકારો & એપ્લિકેશન્સ, યુઝર
પ્રોગ્રામ્સ \\
\textbf{કર્નલ મોડ} & સંપૂર્ણ સિસ્ટમ એક્સેસ & સંપૂર્ણ નિયંત્રણ & ડિવાઇસ ડ્રાઇવર્સ, OS
ફંક્શન્સ \\
\end{longtable}
}

\begin{itemize}
\tightlist
\item
  \textbf{લેયર્ડ આર્કિટેક્ચર}: યુઝર અને સિસ્ટમ કમ્પોનન્ટ્સ વચ્ચે સ્પષ્ટ અલગીકરણ
\item
  \textbf{મોડ સ્વિચિંગ}: સીપીયુ યુઝર અને કર્નલ મોડ્સ વચ્ચે સ્વિચ કરે
\item
  \textbf{સિસ્ટમ કોલ્સ}: યુઝર પ્રોગ્રામ્સ માટે કર્નલ સેવાઓ એક્સેસ કરવાનું ઇન્ટરફેસ
\item
  \textbf{સિક્યોરિટી}: યુઝર મોડ સીધું હાર્ડવેર એક્સેસ અટકાવે
\end{itemize}

\end{solutionbox}
\begin{mnemonicbox}
``LUSK - Linux Uses Safe Kernel protection''

\end{mnemonicbox}
\subsection*{પ્રશ્ન 2(અ) [3
ગુણ]}\label{uxaaauxab0uxab6uxaa8-2uxa85-3-uxa97uxaa3}

\textbf{XOR ગેટ તેના સત્ય કોષ્ટક સાથે વર્ણવો.}

\begin{solutionbox}

\textbf{XOR ગેટ સિમ્બોલ:}

\begin{verbatim}
    A ──┐
        │ )──── આઉટપુટ
    B ──┘
\end{verbatim}

\textbf{સત્ય કોષ્ટક:}

{\def\LTcaptype{none} % do not increment counter
\begin{longtable}[]{@{}lll@{}}
\toprule\noalign{}
A & B & આઉટપુટ (A \oplus B) \\
\midrule\noalign{}
\endhead
\bottomrule\noalign{}
\endlastfoot
0 & 0 & 0 \\
0 & 1 & 1 \\
1 & 0 & 1 \\
1 & 1 & 0 \\
\end{longtable}
}

\begin{itemize}
\tightlist
\item
  \textbf{એક્સક્લુસિવ OR}: જ્યારે ઇનપુટ્સ અલગ હોય ત્યારે આઉટપુટ 1
\item
  \textbf{લોજિક ફંક્શન}: A \oplus B = A'B + AB'
\item
  \textbf{એપ્લિકેશન્સ}: હાફ એડર, પેરિટી ચેકર, એન્ક્રિપ્શન
\end{itemize}

\end{solutionbox}
\begin{mnemonicbox}
``XOR - eXclusive OR અલગ ઇનપુટ્સ માટે 1 આપે''

\end{mnemonicbox}
\subsection*{પ્રશ્ન 2(બ) [4
ગુણ]}\label{uxaaauxab0uxab6uxaa8-2uxaac-4-uxa97uxaa3}

\textbf{નીચેના ઉકેલો.} \textbf{i) (4C6)_{1}_{6} = (\_\_\_\_\_)_{2} =
(\_\_\_\_\_)_{1}_{0}} \textbf{ii) (186)_{1}_{0} = (\_\_\_\_\_)_{8} = (\_\_\_\_\_)_{2}}

\begin{solutionbox}

\textbf{રૂપાંતરણ કોષ્ટક:}

{\def\LTcaptype{none} % do not increment counter
\begin{longtable}[]{@{}lll@{}}
\toprule\noalign{}
રૂપાંતરણ & પગલું & પરિણામ \\
\midrule\noalign{}
\endhead
\bottomrule\noalign{}
\endlastfoot
\textbf{(4C6)_{1}_{6}} & હેક્સ ટુ બાઇનરી & \textbf{10011000110_{2}} \\
& બાઇનરી ટુ ડેસિમલ & \textbf{1222_{1}_{0}} \\
\textbf{(186)_{1}_{0}} & ડેસિમલ ટુ ઓક્ટલ & \textbf{272_{8}} \\
& ડેસિમલ ટુ બાઇનરી & \textbf{10111010_{2}} \\
\end{longtable}
}

\textbf{વિગતવાર સોલ્યુશન્સ:}

\begin{enumerate}
\tightlist
\item
  \textbf{(4C6)_{1}_{6} = (10011000110)_{2} = (1222)_{1}_{0}}
\end{enumerate}

\begin{itemize}
\tightlist
\item
4 = 0100,

C = 1100, 6 = 0110

\item
  સંયુક્ત: 010011000110 = 10011000110_{2}
\item
  ડેસિમલ: 1\times2^{1}^{0} + 0\times2^{9} + 0\times2^{8} + 1\times2^{7} + 1\times2^{6} + 0\times2^{5} + 0\times2^{4} + 0\times2^{3} + 1\times2^{2} +
  1\times2^{1} + 0\times2^{0} = 1222_{1}_{0}
\end{itemize}

\begin{enumerate}
\tightlist
\item
  \textbf{(186)_{1}_{0} = (272)_{8} = (10111010)_{2}}
\end{enumerate}

\begin{itemize}
\tightlist
\item
  ઓક્ટલ: 186 \div 8 = 23 બાકી 2, 23 \div 8 = 2 બાકી 7, 2 \div 8 = 0 બાકી 2 \rightarrow 272_{8}
\item
  બાઇનરી: 186 = 128 + 32 + 16 + 8 + 2 = 10111010_{2}
\end{itemize}

\end{solutionbox}
\begin{mnemonicbox}
``HDB - હેક્સ, ડેસિમલ, બાઇનરી કન્વર્શન્સ''

\end{mnemonicbox}
\subsection*{પ્રશ્ન 2(ક) [7
ગુણ]}\label{uxaaauxab0uxab6uxaa8-2uxa95-7-uxa97uxaa3}

\textbf{નીચેના OS ને સમજાવો} \textbf{i) નેટવર્ક ઓપરેટિંગ સિસ્ટમ} \textbf{ii)
મોબાઇલ ઓપરેટિંગ સિસ્ટમ}

\begin{solutionbox}

\textbf{ઓપરેટિંગ સિસ્ટમ સરખામણી કોષ્ટક:}

{\def\LTcaptype{none} % do not increment counter
\begin{longtable}[]{@{}
  >{\raggedright\arraybackslash}p{(\linewidth - 4\tabcolsep) * \real{0.3611}}
  >{\raggedright\arraybackslash}p{(\linewidth - 4\tabcolsep) * \real{0.3333}}
  >{\raggedright\arraybackslash}p{(\linewidth - 4\tabcolsep) * \real{0.3056}}@{}}
\toprule\noalign{}
\begin{minipage}[b]{\linewidth}\raggedright
લાક્ષણિકતા
\end{minipage} & \begin{minipage}[b]{\linewidth}\raggedright
નેટવર્ક OS
\end{minipage} & \begin{minipage}[b]{\linewidth}\raggedright
મોબાઇલ OS
\end{minipage} \\
\midrule\noalign{}
\endhead
\bottomrule\noalign{}
\endlastfoot
\textbf{હેતુ} & નેટવર્ક રિસોર્સ મેનેજ કરવું & મોબાઇલ ડિવાઇસ મેનેજમેન્ટ \\
\textbf{ઉદાહરણો} & Windows Server, Linux Server & Android, iOS, Windows
Mobile \\
\textbf{મુખ્ય ફીચર્સ} & ફાઇલ શેરિંગ, પ્રિન્ટર શેરિંગ & ટચ ઇન્ટરફેસ, બેટરી મેનેજમેન્ટ \\
\textbf{યુઝર્સ} & મલ્ટિપલ સાથોસાથ યુઝર્સ & સામાન્ય રીતે સિંગલ યુઝર \\
\end{longtable}
}

\begin{center}
\textbf{Mermaid Diagram (Code)}
\begin{verbatim}
{Shaded}
{Highlighting}[]
graph TD
    A[નેટવર્ક OS] {-{-}{} B[ફાઇલ સર્વર]}
    A {-{-}{} C[પ્રિન્ટ સર્વર]}
    A {-{-}{} D[એપ્લિકેશન સર્વર]}
    
    E[મોબાઇલ OS] {-{-}{} F[ટચ ઇન્ટરફેસ]}
    E {-{-}{} G[એપ સ્ટોર]}
    E {-{-}{} H[બેટરી મેનેજમેન્ટ]}
{Highlighting}
{Shaded}
\end{verbatim}
\end{center}

\textbf{i) નેટવર્ક ઓપરેટિંગ સિસ્ટમ:}

\begin{itemize}
\tightlist
\item
  \textbf{મલ્ટિ-યુઝર સપોર્ટ}: મલ્ટિપલ સાથોસાથ યુઝર્સ હેન્ડલ કરે
\item
  \textbf{રિસોર્સ શેરિંગ}: ફાઇલો, પ્રિન્ટર્સ, એપ્લિકેશન્સ નેટવર્કમાં શેર કરાય
\item
  \textbf{સિક્યોરિટી મેનેજમેન્ટ}: યુઝર ઓથેન્ટિકેશન અને એક્સેસ કંટ્રોલ
\end{itemize}

\textbf{ii) મોબાઇલ ઓપરેટિંગ સિસ્ટમ:}

\begin{itemize}
\tightlist
\item
  \textbf{ટચ-ઓપ્ટિમાઇઝ્ડ}: આંગળી-આધારિત ઇન્ટરેક્શન માટે ડિઝાઇન
\item
  \textbf{પાવર મેનેજમેન્ટ}: કાર્યક્ષમ બેટરી ઉપયોગ
\item
  \textbf{એપ ઇકોસિસ્ટમ}: કેન્દ્રીકૃત એપ વિતરણ અને મેનેજમેન્ટ
\end{itemize}

\end{solutionbox}
\begin{mnemonicbox}
``NOS ફોર નેટવર્ક્સ, MOS ફોર મોબિલિટી''

\end{mnemonicbox}
\subsection*{પ્રશ્ન 2(અ OR) [3
ગુણ]}\label{uxaaauxab0uxab6uxaa8-2uxa85-or-3-uxa97uxaa3}

\textbf{ફક્ત NAND ગેટનો ઉપયોગ કરીને OR ગેટ અને NOT ગેટનું લોજિક સર્કિટ દોરો.}

\begin{solutionbox}

\textbf{NAND ઉપયોગ કરી OR ગેટ:}

\begin{verbatim}
A ──┐  ┌─── NOT A ──┐
    │ )────────────── │
    └─┘              │ )──── A+B
B ──┐  ┌─── NOT B ──┘
    │ )──────────────
    └─┘
\end{verbatim}

\textbf{NAND ઉપયોગ કરી NOT ગેટ:}

\begin{verbatim}
A ──┐
    │ )──── A{}
A ──┘
\end{verbatim}

\textbf{સત્ય વેરિફિકેશન કોષ્ટક:}

{\def\LTcaptype{none} % do not increment counter
\begin{longtable}[]{@{}lllll@{}}
\toprule\noalign{}
A & B & A' & B' & (A'·B')' = A+B \\
\midrule\noalign{}
\endhead
\bottomrule\noalign{}
\endlastfoot
0 & 0 & 1 & 1 & 0 \\
0 & 1 & 1 & 0 & 1 \\
1 & 0 & 0 & 1 & 1 \\
1 & 1 & 0 & 0 & 1 \\
\end{longtable}
}

\begin{itemize}
\tightlist
\item
  \textbf{NAND યુનિવર્સલ}: કોઈ પણ લોજિક ફંક્શન ઇમ્પ્લિમેન્ટ કરી શકે
\item
  \textbf{ડી મોર્ગનનો નિયમ}: (A'·B')' = A+B
\end{itemize}

\end{solutionbox}
\begin{mnemonicbox}
``NAND ઈઝ યુનિવર્સલ - બધા ગેટ્સ બનાવી શકે''

\end{mnemonicbox}
\subsection*{પ્રશ્ન 2(બ OR) [4
ગુણ]}\label{uxaaauxab0uxab6uxaa8-2uxaac-or-4-uxa97uxaa3}

\textbf{i) બાઇનરી સંખ્યાને દશાંશ સંખ્યામાં રૂપાંતરિત કરો: (i) 11101 (ii) 10011}
\textbf{ii) દશાંશ સંખ્યાને બાઇનરી સંખ્યામાં રૂપાંતરિત કરો: (i) 19 (ii) 64}

\begin{solutionbox}

\textbf{રૂપાંતરણ કોષ્ટક:}

{\def\LTcaptype{none} % do not increment counter
\begin{longtable}[]{@{}llll@{}}
\toprule\noalign{}
પ્રકાર & સંખ્યા & પ્રક્રિયા & પરિણામ \\
\midrule\noalign{}
\endhead
\bottomrule\noalign{}
\endlastfoot
\textbf{બાઇનરી ટુ ડેસિમલ} & 11101_{2} & 1\times2^{4}+1\times2^{3}+1\times2^{2}+0\times2^{1}+1\times2^{0} &
\textbf{29_{1}_{0}} \\
& 10011_{2} & 1\times2^{4}+0\times2^{3}+0\times2^{2}+1\times2^{1}+1\times2^{0} & \textbf{19_{1}_{0}} \\
\textbf{ડેસિમલ ટુ બાઇનરી} & 19_{1}_{0} & 2 વડે ભાગાકાર પદ્ધતિ & \textbf{10011_{2}} \\
& 64_{1}_{0} & 2 વડે ભાગાકાર પદ્ધતિ & \textbf{1000000_{2}} \\
\end{longtable}
}

\textbf{વિગતવાર સોલ્યુશન્સ:}

\textbf{i) બાઇનરી ટુ ડેસિમલ:}

\begin{itemize}
\tightlist
\item
  11101_{2} = 16 + 8 + 4 + 0 + 1 = 29_{1}_{0}
\item
  10011_{2} = 16 + 0 + 0 + 2 + 1 = 19_{1}_{0}
\end{itemize}

\textbf{ii) ડેસિમલ ટુ બાઇનરી:}

\begin{itemize}
\tightlist
\item
  19 \div 2 = 9 બાકી 1, 9 \div 2 = 4 બાકી 1, 4 \div 2 = 2 બાકી 0, 2 \div 2 = 1 બાકી
  0, 1 \div 2 = 0 બાકી 1 \rightarrow 10011_{2}
\item
  64 \div 2 = 32 બાકી 0\ldots{} \rightarrow 1000000_{2}
\end{itemize}

\end{solutionbox}
\begin{mnemonicbox}
``બાઇનરી ટુ ડેસિમલ માટે 2 ની શક્તિઓ''

\end{mnemonicbox}
\subsection*{પ્રશ્ન 2(ક OR) [7
ગુણ]}\label{uxaaauxab0uxab6uxaa8-2uxa95-or-7-uxa97uxaa3}

\textbf{ઓપન સોર્સ સોફ્ટવેર અને પ્રોપ્રાઇટરી સોફ્ટવેર સમજાવો. બંને પ્રકારના સોફ્ટવેરના
ઓછામાં ઓછા પાંચ ઉદાહરણો આપો.}

\begin{solutionbox}

\textbf{સોફ્ટવેર પ્રકાર સરખામણી કોષ્ટક:}

{\def\LTcaptype{none} % do not increment counter
\begin{longtable}[]{@{}lll@{}}
\toprule\noalign{}
પાસાં & ઓપન-સોર્સ & પ્રોપ્રાઇટરી \\
\midrule\noalign{}
\endhead
\bottomrule\noalign{}
\endlastfoot
\textbf{સોર્સ કોડ} & મુક્તપણે ઉપલબ્ધ & બંધ/છુપાયેલ \\
\textbf{કિંમત} & સામાન્ય રીતે મફત & કોમર્શિયલ લાઇસન્સ \\
\textbf{મોડિફિકેશન} & મંજૂર & પ્રતિબંધિત \\
\textbf{સપોર્ટ} & કમ્યુનિટી-આધારિત & વેન્ડર સપોર્ટ \\
\end{longtable}
}

\textbf{સોફ્ટવેર ઉદાહરણો:}

{\def\LTcaptype{none} % do not increment counter
\begin{longtable}[]{@{}ll@{}}
\toprule\noalign{}
ઓપન-સોર્સ & પ્રોપ્રાઇટરી \\
\midrule\noalign{}
\endhead
\bottomrule\noalign{}
\endlastfoot
Linux & Microsoft Windows \\
LibreOffice & Microsoft Office \\
Firefox & Internet Explorer \\
GIMP & Adobe Photoshop \\
MySQL & Oracle Database \\
\end{longtable}
}

\begin{verbatim}
pie title સોફ્ટવેર વિતરણ
    "ઓપન{-સોર્સ" : 40}
    "પ્રોપ્રાઇટરી" : 60
\end{verbatim}

\textbf{ઓપન-સોર્સ લાક્ષણિકતાઓ:}

\begin{itemize}
\tightlist
\item
  \textbf{મોડિફાઇ કરવાની સ્વતંત્રતા}: યુઝર્સ સોર્સ કોડ બદલી શકે
\item
  \textbf{કમ્યુનિટી ડેવલપમેન્ટ}: સહયોગી સુધારણા
\item
  \textbf{પારદર્શિતા}: તમામ કોડ દૃશ્યમાન અને ઓડિટ કરી શકાય
\end{itemize}

\textbf{પ્રોપ્રાઇટરી લાક્ષણિકતાઓ:}

\begin{itemize}
\tightlist
\item
  \textbf{કોમર્શિયલ મોડેલ}: લાઇસન્સિંગ દ્વારા આવક
\item
  \textbf{પ્રોફેશનલ સપોર્ટ}: સમર્પિત કસ્ટમર સેવા
\item
  \textbf{ગુણવત્તા ખાતરી}: કઠોર પરીક્ષણ અને માન્યતા
\end{itemize}

\end{solutionbox}
\begin{mnemonicbox}
``FOSS ઈઝ ફ્રી, ઓપન, શેર્ડ, કમ્યુનિટી દ્વારા સપોર્ટેડ''

\end{mnemonicbox}
\subsection*{પ્રશ્ન 3(અ) [3
ગુણ]}\label{uxaaauxab0uxab6uxaa8-3uxa85-3-uxa97uxaa3}

\textbf{વ્યાખ્યાયિત કરો} \textbf{1. મોડ્યુલેશન} \textbf{2. મલ્ટિપ્લેક્સિંગ}

\begin{solutionbox}

\textbf{વ્યાખ્યા કોષ્ટક:}

{\def\LTcaptype{none} % do not increment counter
\begin{longtable}[]{@{}
  >{\raggedright\arraybackslash}p{(\linewidth - 4\tabcolsep) * \real{0.2857}}
  >{\raggedright\arraybackslash}p{(\linewidth - 4\tabcolsep) * \real{0.4286}}
  >{\raggedright\arraybackslash}p{(\linewidth - 4\tabcolsep) * \real{0.2857}}@{}}
\toprule\noalign{}
\begin{minipage}[b]{\linewidth}\raggedright
શબ્દ
\end{minipage} & \begin{minipage}[b]{\linewidth}\raggedright
વ્યાખ્યા
\end{minipage} & \begin{minipage}[b]{\linewidth}\raggedright
હેતુ
\end{minipage} \\
\midrule\noalign{}
\endhead
\bottomrule\noalign{}
\endlastfoot
\textbf{મોડ્યુલેશન} & કેરિયર સિગ્નલના ગુણધર્મો બદલવાની પ્રક્રિયા & લાંબા અંતરનું
ટ્રાન્સમિશન સક્ષમ કરવું \\
\textbf{મલ્ટિપ્લેક્સિંગ} & ટ્રાન્સમિશન માટે મલ્ટિપલ સિગ્નલો જોડવા & કાર્યક્ષમ ચેનલ
ઉપયોગ \\
\end{longtable}
}

\begin{itemize}
\tightlist
\item
  \textbf{મોડ્યુલેશન}: કેરિયર વેવના એમ્પ્લિટ્યુડ, ફ્રીક્વન્સી અથવા ફેઝ બદલે
\item
  \textbf{મલ્ટિપ્લેક્સિંગ}: મલ્ટિપલ યુઝર્સને એક જ કમ્યુનિકેશન મીડિયમ શેર કરવાની મંજૂરી
  આપે
\item
  \textbf{સિગ્નલ પ્રોસેસિંગ}: બંને તકનીકો કમ્યુનિકેશન કાર્યક્ષમતા સુધારે
\end{itemize}

\end{solutionbox}
\begin{mnemonicbox}
``MM - મોડ્યુલેશન મોડિફાઇ કરે, મલ્ટિપ્લેક્સિંગ મર્જ કરે''

\end{mnemonicbox}
\subsection*{પ્રશ્ન 3(બ) [4
ગુણ]}\label{uxaaauxab0uxab6uxaa8-3uxaac-4-uxa97uxaa3}

\textbf{સ્ટાર ટોપોલોજી સમજાવો.}

\begin{solutionbox}

\textbf{સ્ટાર ટોપોલોજી ડાયાગ્રામ:}

\begin{verbatim}
    Computer1
        |
Computer4──Hub──Computer2
        |
    Computer3
\end{verbatim}

\textbf{સ્ટાર ટોપોલોજી ફીચર્સ કોષ્ટક:}

{\def\LTcaptype{none} % do not increment counter
\begin{longtable}[]{@{}ll@{}}
\toprule\noalign{}
ફીચર & વિવરણ \\
\midrule\noalign{}
\endhead
\bottomrule\noalign{}
\endlastfoot
\textbf{કેન્દ્રીય ડિવાઇસ} & હબ/સ્વિચ બધા નોડ્સને જોડે \\
\textbf{ફોલ્ટ ટોલરન્સ} & સિંગલ નોડ ફેઇલ્યૂર અન્યને અસર કરતું નથી \\
\textbf{પર્ફોર્મન્સ} & દરેક કનેક્શન માટે સમર્પિત બેન્ડવિથ \\
\textbf{સ્કેલેબિલિટી} & નોડ્સ ઉમેરવા/હટાવવા સરળ \\
\end{longtable}
}

\begin{itemize}
\tightlist
\item
  \textbf{કેન્દ્રીય હબ}: બધું કમ્યુનિકેશન કેન્દ્રીય ડિવાઇસ દ્વારા પસાર થાય
\item
  \textbf{સરળ ટ્રબલશૂટિંગ}: સમસ્યાઓ વ્યક્તિગત કનેક્શન્સમાં અલગ
\item
  \textbf{વધુ કિંમત}: બસ ટોપોલોજી કરતાં વધુ કેબલ જરૂરી
\item
  \textbf{સિંગલ પોઇન્ટ ઓફ ફેઇલ્યૂર}: હબ ફેઇલ થવાથી આખું નેટવર્ક અસર પામે
\end{itemize}

\end{solutionbox}
\begin{mnemonicbox}
``STAR - સિંગલ પોઇન્ટ, ટ્રબલશૂટિંગ ઇઝી, ઓલ થ્રુ હબ,
રિલાયબલ''

\end{mnemonicbox}
\subsection*{પ્રશ્ન 3(ક) [7
ગુણ]}\label{uxaaauxab0uxab6uxaa8-3uxa95-7-uxa97uxaa3}

\textbf{ટાઇમ ડિવિઝન મલ્ટિપ્લેક્સિંગ (TDM) પર ટૂંકી નોંધ તૈયાર કરો}

\begin{solutionbox}

\textbf{TDM કન્સેપ્ટ ડાયાગ્રામ:}

\begin{verbatim}
gantt
    title ટાઇમ ડિવિઝન મલ્ટિપ્લેક્સિંગ
    dateFormat X
    axisFormat \%s
    
    section ચેનલ A
    સ્લોટ 1    :0, 1
    સ્લોટ 4    :3, 4
    સ્લોટ 7    :6, 7
    
    section ચેનલ B  
    સ્લોટ 2    :1, 2
    સ્લોટ 5    :4, 5
    સ્લોટ 8    :7, 8
    
    section ચેનલ C
    સ્લોટ 3    :2, 3
    સ્લોટ 6    :5, 6
    સ્લોટ 9    :8, 9
\end{verbatim}

\textbf{TDM લાક્ષણિકતાઓ કોષ્ટક:}

{\def\LTcaptype{none} % do not increment counter
\begin{longtable}[]{@{}ll@{}}
\toprule\noalign{}
ફીચર & વિવરણ \\
\midrule\noalign{}
\endhead
\bottomrule\noalign{}
\endlastfoot
\textbf{સિદ્ધાંત} & વિવિધ યુઝર્સને વિવિધ ટાઇમ સ્લોટ્સ ફાળવાય \\
\textbf{સિન્ક્રોનાઇઝેશન} & બધા ડિવાઇસ સિન્ક્રોનાઇઝ હોવા જોઈએ \\
\textbf{કાર્યક્ષમતા} & સ્લોટ્સ ભરાયા હોય ત્યારે સંપૂર્ણ બેન્ડવિથ ઉપયોગ \\
\textbf{એપ્લિકેશન્સ} & ડિજિટલ ટેલિફોન સિસ્ટમ્સ, T1/E1 લાઇન્સ \\
\end{longtable}
}

\textbf{TDM પ્રકારો:}

\begin{itemize}
\tightlist
\item
  \textbf{સિન્ક્રોનસ TDM}: ડેટા ઉપલબ્ધતાને ધ્યાનમાં લીધા વિના નિશ્ચિત ટાઇમ સ્લોટ્સ
\item
  \textbf{એસિન્ક્રોનસ TDM}: માંગના આધારે ડાયનેમિક સ્લોટ ફાળવણી
\item
  \textbf{સ્ટેટિસ્ટિકલ TDM}: આંકડાકીય આધારે સ્લોટ્સ ફાળવાય
\end{itemize}

\textbf{ફાયદાઓ:}

\begin{itemize}
\tightlist
\item
  \textbf{ન્યાયી શેરિંગ}: બધા યુઝર્સ માટે સમાન ટાઇમ ફાળવણી
\item
  \textbf{કોઈ સિગ્નલ ઇન્ટરફેરન્સ નહીં}: ટાઇમ-આધારિત અલગીકરણ સંઘર્ષ અટકાવે
\end{itemize}

\end{solutionbox}
\begin{mnemonicbox}
``TDM - ટાઇમ ડિવાઇડ્સ મીડિયમ ન્યાયથી''

\end{mnemonicbox}
\subsection*{પ્રશ્ન 3(અ OR) [3
ગુણ]}\label{uxaaauxab0uxab6uxaa8-3uxa85-or-3-uxa97uxaa3}

\textbf{એમ્પ્લિટ્યુડ મોડ્યુલેશન (AM) સમજાવો.}

\begin{solutionbox}

\textbf{AM સિગ્નલ ડાયાગ્રામ:}

\begin{verbatim}
Message Signal:    {}
                  
Carrier Signal:    ||||||||||||||||||||

AM Output:         |{|||||||||}
\end{verbatim}

\textbf{AM લાક્ષણિકતાઓ કોષ્ટક:}

{\def\LTcaptype{none} % do not increment counter
\begin{longtable}[]{@{}ll@{}}
\toprule\noalign{}
પેરામીટર & વિવરણ \\
\midrule\noalign{}
\endhead
\bottomrule\noalign{}
\endlastfoot
\textbf{વ્યાખ્યા} & મેસેજ સિગ્નલ સાથે કેરિયરનું એમ્પ્લિટ્યુડ બદલાય \\
\textbf{ફ્રીક્વન્સી રેન્જ} & 535-1605 kHz (AM રેડિયો) \\
\textbf{બેન્ડવિથ} & મેસેજ સિગ્નલ ફ્રીક્વન્સીથી બમણું \\
\end{longtable}
}

\begin{itemize}
\tightlist
\item
  \textbf{કેરિયર વેવ}: માહિતી વહન કરતું હાઇ ફ્રીક્વન્સી સિગ્નલ
\item
  \textbf{મોડ્યુલેશન ઇન્ડેક્સ}: એમ્પ્લિટ્યુડ વેરિએશનની ઊંડાઈ નક્કી કરે
\item
  \textbf{એપ્લિકેશન્સ}: AM રેડિયો બ્રોડકાસ્ટિંગ, એરક્રાફ્ટ કમ્યુનિકેશન
\end{itemize}

\end{solutionbox}
\begin{mnemonicbox}
``AM - એમ્પ્લિટ્યુડ મેસેજ સાથે મોડિફાઇ થાય''

\end{mnemonicbox}
\subsection*{પ્રશ્ન 3(બ OR) [4
ગુણ]}\label{uxaaauxab0uxab6uxaa8-3uxaac-or-4-uxa97uxaa3}

\textbf{DNS વર્ણવો.}

\begin{solutionbox}

\textbf{DNS હાયરાર્કી:}

\begin{center}
\textbf{Mermaid Diagram (Code)}
\begin{verbatim}
{Shaded}
{Highlighting}[]
graph LR
    A[રૂટ .] {-{-}{} B[ટોપ લેવલ .com]}
    A {-{-}{} C[ટોપ લેવલ .org]}
    B {-{-}{} D[google.com]}
    B {-{-}{} E[microsoft.com]}
    D {-{-}{} F[www.google.com]}
    D {-{-}{} G[mail.google.com]}
{Highlighting}
{Shaded}
\end{verbatim}
\end{center}

\textbf{DNS કમ્પોનન્ટ્સ કોષ્ટક:}

{\def\LTcaptype{none} % do not increment counter
\begin{longtable}[]{@{}ll@{}}
\toprule\noalign{}
કમ્પોનન્ટ & ફંક્શન \\
\midrule\noalign{}
\endhead
\bottomrule\noalign{}
\endlastfoot
\textbf{ડોમેઇન નેમ} & માનવ-વાંચી શકાય તેવું વેબ એડ્રેસ \\
\textbf{IP એડ્રેસ} & સર્વરનું સંખ્યાકીય એડ્રેસ \\
\textbf{DNS સર્વર} & નામોને IP એડ્રેસમાં ટ્રાન્સલેટ કરે \\
\textbf{રેકોર્ડ્સ} & વિવિધ પ્રકારો (A, MX, CNAME) \\
\end{longtable}
}

\begin{itemize}
\tightlist
\item
  \textbf{નેમ રિઝોલ્યુશન}: ડોમેઇન નામોને IP એડ્રેસમાં કન્વર્ટ કરે
\item
  \textbf{હાયરાર્કિકલ સ્ટ્રક્ચર}: રૂટ, TLD, સેકન્ડ-લેવલ ડોમેઇન્સ
\item
  \textbf{ડિસ્ટ્રિબ્યુટેડ ડેટાબેસ}: કોઈ સિંગલ પોઇન્ટ ઓફ ફેઇલ્યૂર નથી
\item
  \textbf{કેશિંગ}: તાજેતરના લુકઅપ્સ સ્ટોર કરીને પર્ફોર્મન્સ સુધારે
\end{itemize}

\end{solutionbox}
\begin{mnemonicbox}
``DNS - ડોમેઇન નેમ સિસ્ટમ એડ્રેસ ટ્રાન્સલેટ કરે''

\end{mnemonicbox}
\subsection*{પ્રશ્ન 3(ક OR) [7
ગુણ]}\label{uxaaauxab0uxab6uxaa8-3uxa95-or-7-uxa97uxaa3}

\textbf{નીચેનું વર્ણન કરો.} \textbf{1. સીરિયલ કમ્યુનિકેશન} \textbf{2. સિન્ક્રોનસ
ટ્રાન્સમિશન}

\begin{solutionbox}

\textbf{કમ્યુનિકેશન પ્રકારો ડાયાગ્રામ:}

\begin{center}
\textbf{Mermaid Diagram (Code)}
\begin{verbatim}
{Shaded}
{Highlighting}[]
graph LR
    A[ડેટા કમ્યુનિકેશન] {-{-}{} B[સીરિયલ]}
    A {-{-}{} C[પેરેલલ]}
    B {-{-}{} D[સિન્ક્રોનસ]}
    B {-{-}{} E[એસિન્ક્રોનસ]}
{Highlighting}
{Shaded}
\end{verbatim}
\end{center}

\textbf{કમ્યુનિકેશન સરખામણી કોષ્ટક:}

{\def\LTcaptype{none} % do not increment counter
\begin{longtable}[]{@{}
  >{\raggedright\arraybackslash}p{(\linewidth - 6\tabcolsep) * \real{0.2188}}
  >{\raggedright\arraybackslash}p{(\linewidth - 6\tabcolsep) * \real{0.2188}}
  >{\raggedright\arraybackslash}p{(\linewidth - 6\tabcolsep) * \real{0.2812}}
  >{\raggedright\arraybackslash}p{(\linewidth - 6\tabcolsep) * \real{0.2812}}@{}}
\toprule\noalign{}
\begin{minipage}[b]{\linewidth}\raggedright
પ્રકાર
\end{minipage} & \begin{minipage}[b]{\linewidth}\raggedright
વિવરણ
\end{minipage} & \begin{minipage}[b]{\linewidth}\raggedright
ટાઇમિંગ
\end{minipage} & \begin{minipage}[b]{\linewidth}\raggedright
ઉદાહરણો
\end{minipage} \\
\midrule\noalign{}
\endhead
\bottomrule\noalign{}
\endlastfoot
\textbf{સીરિયલ કમ્યુનિકેશન} & ડેટા બિટ્સ એક પછી એક મોકલાય & ધીમું પરંતુ વિશ્વસનીય
& RS-232, USB, ઇથરનેટ \\
\textbf{સિન્ક્રોનસ ટ્રાન્સમિશન} & ક્લોક સિગ્નલ સેન્ડર/રિસીવર સિન્ક કરે & ચોક્કસ
ટાઇમિંગ & HDLC, SDLC \\
\end{longtable}
}

\textbf{1. સીરિયલ કમ્યુનિકેશન:}

\begin{itemize}
\tightlist
\item
  \textbf{સિંગલ વાયર}: ડેટા સિંગલ ચેનલ પર બિટ બાય બિટ ટ્રાન્સમિટ થાય
\item
  \textbf{કોસ્ટ ઇફેક્ટિવ}: પેરેલલ કરતાં ઓછા વાયર જરૂરી
\item
  \textbf{લાંબો અંતર}: નોઇઝ અને ઇન્ટરફેરન્સને ઓછું સંવેદનશીલ
\item
  \textbf{એરર ડિટેક્શન}: ડેટા ઇન્ટેગ્રિટી માટે બિલ્ટ-ઇન મેકેનિઝમ
\end{itemize}

\textbf{2. સિન્ક્રોનસ ટ્રાન્સમિશન:}

\begin{itemize}
\tightlist
\item
  \textbf{ક્લોક સિન્ક્રોનાઇઝેશન}: અલગ ક્લોક સિગ્નલ અથવા એમ્બેડેડ ટાઇમિંગ
\item
  \textbf{બ્લોક ટ્રાન્સમિશન}: ડેટા સતત બ્લોક્સમાં મોકલાય
\item
  \textbf{વધુ કાર્યક્ષમતા}: સ્ટાર્ટ/સ્ટોપ બિટ્સની જરૂર નથી
\item
  \textbf{કોમ્પ્લેક્સ હાર્ડવેર}: સિન્ક્રોનાઇઝ્ડ ક્લોક્સ જરૂરી
\end{itemize}

\end{solutionbox}
\begin{mnemonicbox}
``સીરિયલ ઈઝ સિક્વેન્શિયલ, સિન્ક્રોનસ ઈઝ સાયમલ્ટેનિયસ''

\end{mnemonicbox}
\subsection*{પ્રશ્ન 4(અ) [3
ગુણ]}\label{uxaaauxab0uxab6uxaa8-4uxa85-3-uxa97uxaa3}

\textbf{મેશ અને બસ ટોપોલોજીમાં તફાવત કરો.}

\begin{solutionbox}

\textbf{ટોપોલોજી સરખામણી કોષ્ટક:}

{\def\LTcaptype{none} % do not increment counter
\begin{longtable}[]{@{}lll@{}}
\toprule\noalign{}
ફીચર & મેશ ટોપોલોજી & બસ ટોપોલોજી \\
\midrule\noalign{}
\endhead
\bottomrule\noalign{}
\endlastfoot
\textbf{કનેક્શન} & દરેક નોડ બીજા દરેક સાથે જોડાયેલ & બધા નોડ્સ સિંગલ કેબલ પર \\
\textbf{ફોલ્ટ ટોલરન્સ} & ખૂબ વધારે & ઓછું (સિંગલ પોઇન્ટ ઓફ ફેઇલ્યૂર) \\
\textbf{કિંમત} & ખૂબ મોંઘું & આર્થિક \\
\textbf{પર્ફોર્મન્સ} & ઉત્તમ & વધુ નોડ્સ સાથે ઘટે \\
\end{longtable}
}

\textbf{મેશ ટોપોલોજી:}

\begin{verbatim}
A ─── B
│ { / │}
│  X  │
│ / { │}
C ─── D
\end{verbatim}

\textbf{બસ ટોપોલોજી:}

\begin{verbatim}
A ── B ── C ── D ── ટર્મિનેટર
\end{verbatim}

\begin{itemize}
\tightlist
\item
  \textbf{મેશ ફાયદાઓ}: રિડન્ડન્ટ પાથ, ઉચ્ચ વિશ્વસનીયતા
\item
  \textbf{બસ ફાયદાઓ}: સરળ ઇન્સ્ટોલેશન, કોસ્ટ-ઇફેક્ટિવ
\item
  \textbf{કેબલ જરૂરિયાતો}: મેશને n(n-1)/2 કનેક્શન્સ જરૂરી, બસને સિંગલ કેબલ
\end{itemize}

\end{solutionbox}
\begin{mnemonicbox}
``મેશ ઈઝ મેની કનેક્શન્સ, બસ ઈઝ બેસિક સિંગલ લાઇન''

\end{mnemonicbox}
\subsection*{પ્રશ્ન 4(બ) [4
ગુણ]}\label{uxaaauxab0uxab6uxaa8-4uxaac-4-uxa97uxaa3}

\textbf{FDM અને TDM ની સરખામણી કરો.}

\begin{solutionbox}

\textbf{FDM vs TDM સરખામણી કોષ્ટક:}

{\def\LTcaptype{none} % do not increment counter
\begin{longtable}[]{@{}lll@{}}
\toprule\noalign{}
પેરામીટર & FDM & TDM \\
\midrule\noalign{}
\endhead
\bottomrule\noalign{}
\endlastfoot
\textbf{ફુલ ફોર્મ} & ફ્રીક્વન્સી ડિવિઝન મલ્ટિપ્લેક્સિંગ & ટાઇમ ડિવિઝન
મલ્ટિપ્લેક્સિંગ \\
\textbf{વિભાજન આધાર} & ફ્રીક્વન્સી બેન્ડ્સ & ટાઇમ સ્લોટ્સ \\
\textbf{સિગ્નલ પ્રકાર} & એનાલોગ & ડિજિટલ \\
\textbf{ક્રોસટોક} & ચેનલો વચ્ચે શક્ય & કોઈ ક્રોસટોક નથી \\
\textbf{સિન્ક્રોનાઇઝેશન} & જરૂરી નથી & જરૂરી \\
\textbf{કાર્યક્ષમતા} & ગાર્ડ બેન્ડ્સને કારણે ઓછી & વધુ કાર્યક્ષમતા \\
\end{longtable}
}

\begin{verbatim}
graph TB
    A[મલ્ટિપ્લેક્સિંગ ટેકનિક્સ] {-{-} B[FDM]}
    A {-{-} C[TDM]}
    B {-{-} D[રેડિયો બ્રોડકાસ્ટિંગ]}
    B {-{-} E[કેબલ ટીવી]}
    C {-{-} F[ડિજિટલ ટેલિફોની]}
    C {-{-} G[કમ્પ્યુટર નેટવર્ક્સ]}
\end{verbatim}

\textbf{FDM લાક્ષણિકતાઓ:}

\begin{itemize}
\tightlist
\item
  \textbf{ફ્રીક્વન્સી સેપેરેશન}: દરેક સિગ્નલને અલગ ફ્રીક્વન્સી બેન્ડ ફાળવાય
\item
  \textbf{સાથોસાથ ટ્રાન્સમિશન}: બધા સિગ્નલો એક જ સમયે ટ્રાન્સમિટ થાય
\item
  \textbf{ગાર્ડ બેન્ડ્સ}: ચેનલો વચ્ચે ઇન્ટરફેરન્સ અટકાવે
\end{itemize}

\textbf{TDM લાક્ષણિકતાઓ:}

\begin{itemize}
\tightlist
\item
  \textbf{ટાઇમ સેપેરેશન}: દરેક સિગ્નલને અલગ ટાઇમ સ્લોટ ફાળવાય
\item
  \textbf{ક્રમિક ટ્રાન્સમિશન}: સિગ્નલો એક પછી એક ટ્રાન્સમિટ થાય
\item
  \textbf{ચોક્કસ ટાઇમિંગ}: સિન્ક્રોનાઇઝ્ડ ક્લોક્સ જરૂરી
\end{itemize}

\end{solutionbox}
\begin{mnemonicbox}
``FDM ફ્રીક્વન્સી ઉપયોગ કરે, TDM ટાઇમ ઉપયોગ કરે''

\end{mnemonicbox}
\subsection*{પ્રશ્ન 4(ક) [7
ગુણ]}\label{uxaaauxab0uxab6uxaa8-4uxa95-7-uxa97uxaa3}

\textbf{OSI રેફરન્સ મોડેલ દોરો અને સમજાવો.}

\begin{solutionbox}

\textbf{OSI મોડેલ ડાયાગ્રામ:}

\begin{center}
\textbf{Mermaid Diagram (Code)}
\begin{verbatim}
{Shaded}
{Highlighting}[]
graph LR
    A[એપ્લિકેશન લેયર {- લેયર 7] {-}{-}{} B[પ્રેઝન્ટેશન લેયર {-} લેયર 6]}
    B {-{-}{} C[સેશન લેયર {-} લેયર 5]}
    C {-{-}{} D[ટ્રાન્સપોર્ટ લેયર {-} લેયર 4]}
    D {-{-}{} E[નેટવર્ક લેયર {-} લેયર 3]}
    E {-{-}{} F[ડેટા લિંક લેયર {-} લેયર 2]}
    F {-{-}{} G[ફિઝિકલ લેયર {-} લેયર 1]}
{Highlighting}
{Shaded}
\end{verbatim}
\end{center}

\textbf{OSI લેયર ફંક્શન્સ કોષ્ટક:}

{\def\LTcaptype{none} % do not increment counter
\begin{longtable}[]{@{}llll@{}}
\toprule\noalign{}
લેયર & નામ & ફંક્શન & ઉદાહરણો \\
\midrule\noalign{}
\endhead
\bottomrule\noalign{}
\endlastfoot
\textbf{7} & એપ્લિકેશન & યુઝર ઇન્ટરફેસ & HTTP, FTP, SMTP \\
\textbf{6} & પ્રેઝન્ટેશન & ડેટા ફોર્મેટિંગ & એન્ક્રિપ્શન, કમ્પ્રેશન \\
\textbf{5} & સેશન & સેશન મેનેજમેન્ટ & NetBIOS, RPC \\
\textbf{4} & ટ્રાન્સપોર્ટ & એન્ડ-ટુ-એન્ડ ડિલિવરી & TCP, UDP \\
\textbf{3} & નેટવર્ક & રાઉટિંગ & IP, ICMP \\
\textbf{2} & ડેટા લિંક & ફ્રેમ ડિલિવરી & ઇથરનેટ, PPP \\
\textbf{1} & ફિઝિકલ & બિટ ટ્રાન્સમિશન & કેબલ્સ, હબ્સ \\
\end{longtable}
}

\textbf{મુખ્ય ફીચર્સ:}

\begin{itemize}
\tightlist
\item
  \textbf{લેયર્ડ આર્કિટેક્ચર}: દરેક લેયરની ચોક્કસ જવાબદારીઓ
\item
  \textbf{પ્રોટોકોલ ઇન્ડિપેન્ડન્સ}: લેયર્સ સ્વતંત્ર રીતે મોડિફાઇ કરી શકાય
\item
  \textbf{સ્ટાન્ડર્ડાઇઝેશન}: નેટવર્ક કમ્યુનિકેશન માટે સામાન્ય ફ્રેમવર્ક
\item
  \textbf{એન્કેપ્સુલેશન}: દરેક લેયર પોતાનું હેડર ઉમેરે
\end{itemize}

\end{solutionbox}
\begin{mnemonicbox}
``All People Seem To Need Data Processing''

\end{mnemonicbox}
\subsection*{પ્રશ્ન 4(અ OR) [3
ગુણ]}\label{uxaaauxab0uxab6uxaa8-4uxa85-or-3-uxa97uxaa3}

\textbf{સંક્ષિપ્તમાં હબનું વર્ણન કરો.}

\begin{solutionbox}

\textbf{હબ ડાયાગ્રામ:}

\begin{verbatim}
    PC1
     |
PC4──HUB──PC2
     |
    PC3
\end{verbatim}

\textbf{હબ લાક્ષણિકતાઓ કોષ્ટક:}

{\def\LTcaptype{none} % do not increment counter
\begin{longtable}[]{@{}ll@{}}
\toprule\noalign{}
ફીચર & વિવરણ \\
\midrule\noalign{}
\endhead
\bottomrule\noalign{}
\endlastfoot
\textbf{ફંક્શન} & ડિવાઇસ માટે કેન્દ્રીય કનેક્શન પોઇન્ટ \\
\textbf{પ્રકાર} & ફિઝિકલ લેયર ડિવાઇસ (લેયર 1) \\
\textbf{ડેટા હેન્ડલિંગ} & બધા કનેક્ટેડ ડિવાઇસમાં બ્રોડકાસ્ટ \\
\textbf{કોલિઝન ડોમેઇન} & બધા પોર્ટ્સ એક જ કોલિઝન ડોમેઇન શેર કરે \\
\end{longtable}
}

\begin{itemize}
\tightlist
\item
  \textbf{શેર્ડ બેન્ડવિથ}: બધા કનેક્ટેડ ડિવાઇસ કુલ બેન્ડવિથ શેર કરે
\item
  \textbf{હાફ-ડુપ્લેક્સ}: સાથોસાથ મોકલી અને મેળવી શકતું નથી
\item
  \textbf{સિક્યોરિટી ઇશ્યૂઝ}: બધા ડિવાઇસ બધો ટ્રાન્સમિટ થયેલો ડેટા મેળવે
\item
  \textbf{અપ્રચલિત ટેકનોલોજી}: આધુનિક નેટવર્ક્સમાં સ્વિચ દ્વારા બદલાયું
\end{itemize}

\end{solutionbox}
\begin{mnemonicbox}
``હબ ઈઝ હાફ-ડુપ્લેક્સ, શેર્સ બેન્ડવિથ''

\end{mnemonicbox}
\subsection*{પ્રશ્ન 4(બ OR) [4
ગુણ]}\label{uxaaauxab0uxab6uxaa8-4uxaac-or-4-uxa97uxaa3}

\textbf{STP અને UTP ની સરખામણી કરો.}

\begin{solutionbox}

\textbf{STP vs UTP કેબલ સરખામણી કોષ્ટક:}

{\def\LTcaptype{none} % do not increment counter
\begin{longtable}[]{@{}lll@{}}
\toprule\noalign{}
ફીચર & STP (શિલ્ડેડ ટ્વિસ્ટેડ પેર) & UTP (અનશિલ્ડેડ ટ્વિસ્ટેડ પેર) \\
\midrule\noalign{}
\endhead
\bottomrule\noalign{}
\endlastfoot
\textbf{શિલ્ડિંગ} & મેટલ ફોઇલ/બ્રેઇડ પ્રોટેક્શન & કોઈ શિલ્ડિંગ નથી \\
\textbf{કિંમત} & વધુ મોંઘું & ઓછું મોંઘું \\
\textbf{ઇન્સ્ટોલેશન} & ગ્રાઉન્ડિંગને કારણે જટિલ & સરળ ઇન્સ્ટોલેશન \\
\textbf{EMI રેઝિસ્ટન્સ} & ઉત્તમ પ્રોટેક્શન & મધ્યમ પ્રોટેક્શન \\
\textbf{એપ્લિકેશન્સ} & ઇન્ડસ્ટ્રિયલ વાતાવરણ & ઓફિસ વાતાવરણ \\
\end{longtable}
}

\textbf{કેબલ સ્ટ્રક્ચર:}

\begin{verbatim}
UTP:    |wire1 wire2|
        |wire3 wire4|

STP:    |Shield|wire1 wire2|Shield|
        |Shield|wire3 wire4|Shield|
\end{verbatim}

\textbf{STP ફાયદાઓ:}

\begin{itemize}
\tightlist
\item
  \textbf{બેહતર નોઇઝ ઇમ્યુનિટી}: શિલ્ડ ઇલેક્ટ્રોમેગ્નેટિક ઇન્ટરફેરન્સ બ્લોક કરે
\item
  \textbf{હાયર ડેટા રેટ્સ}: ઝડપી ટ્રાન્સમિશન સ્પીડ સપોર્ટ કરે
\item
  \textbf{સિક્યોર ટ્રાન્સમિશન}: ઇવ્સડ્રોપિંગ માટે ઓછું સંવેદનશીલ
\end{itemize}

\textbf{UTP ફાયદાઓ:}

\begin{itemize}
\tightlist
\item
  \textbf{કોસ્ટ ઇફેક્ટિવ}: STP કરતાં સસ્તું
\item
  \textbf{ઇઝી ઇન્સ્ટોલેશન}: ગ્રાઉન્ડિંગ જરૂરિયાતો નથી
\item
  \textbf{ફ્લેક્સિબિલિટી}: વધુ લવચીક અને હેન્ડલ કરવામાં સરળ
\end{itemize}

\end{solutionbox}
\begin{mnemonicbox}
``STP ઈઝ શિલ્ડેડ બટ પ્રાઇસી, UTP ઈઝ અનશિલ્ડેડ બટ પોપ્યુલર''

\end{mnemonicbox}
\subsection*{પ્રશ્ન 4(ક OR) [7
ગુણ]}\label{uxaaauxab0uxab6uxaa8-4uxa95-or-7-uxa97uxaa3}

\textbf{LAN, MAN, WAN મા ભેદ પાડો.}

\begin{solutionbox}

\textbf{નેટવર્ક સાઇઝ સરખામણી:}

\begin{verbatim}
graph TB
    A[કમ્પ્યુટર નેટવર્ક્સ] {-{-} B[LAN {-} લોકલ એરિયા નેટવર્ક]}
    A {-{-} C[MAN {-} મેટ્રોપોલિટન એરિયા નેટવર્ક]  }
    A {-{-} D[WAN {-} વાઇડ એરિયા નેટવર્ક]}
    
    B {-{-} E[બિલ્ડિંગ/કેમ્પસ]}
    C {-{-} F[શહેર/મેટ્રોપોલિટન વિસ્તાર]}
    D {-{-} G[દેશ/ખંડ]}
\end{verbatim}

\textbf{નેટવર્ક પ્રકાર સરખામણી કોષ્ટક:}

{\def\LTcaptype{none} % do not increment counter
\begin{longtable}[]{@{}
  >{\raggedright\arraybackslash}p{(\linewidth - 6\tabcolsep) * \real{0.4231}}
  >{\raggedright\arraybackslash}p{(\linewidth - 6\tabcolsep) * \real{0.1923}}
  >{\raggedright\arraybackslash}p{(\linewidth - 6\tabcolsep) * \real{0.1923}}
  >{\raggedright\arraybackslash}p{(\linewidth - 6\tabcolsep) * \real{0.1923}}@{}}
\toprule\noalign{}
\begin{minipage}[b]{\linewidth}\raggedright
પેરામીટર
\end{minipage} & \begin{minipage}[b]{\linewidth}\raggedright
LAN
\end{minipage} & \begin{minipage}[b]{\linewidth}\raggedright
MAN
\end{minipage} & \begin{minipage}[b]{\linewidth}\raggedright
WAN
\end{minipage} \\
\midrule\noalign{}
\endhead
\bottomrule\noalign{}
\endlastfoot
\textbf{કવરેજ} & બિલ્ડિંગ/કેમ્પસ & શહેર/મેટ્રોપોલિટન વિસ્તાર & દેશ/ખંડ \\
\textbf{સ્પીડ} & 10 Mbps - 1 Gbps & 1-100 Mbps & 56 Kbps - 100 Mbps \\
\textbf{કિંમત} & ઓછી & મધ્યમ & વધારે \\
\textbf{માલિકી} & પ્રાઇવેટ & પ્રાઇવેટ/પબ્લિક & પબ્લિક/લીઝ્ડ \\
\textbf{ટેકનોલોજી} & ઇથરનેટ, Wi-Fi & ફાઇબર ઓપ્ટિક, WiMAX & સેટેલાઇટ, લીઝ્ડ
લાઇન્સ \\
\textbf{એરર રેટ} & ખૂબ ઓછો & ઓછો & વધારે \\
\end{longtable}
}

\textbf{વિગતવાર લાક્ષણિકતાઓ:}

\textbf{LAN (લોકલ એરિયા નેટવર્ક):}

\begin{itemize}
\tightlist
\item
  \textbf{હાઇ સ્પીડ}: નાના વિસ્તારમાં ઝડપી ડેટા ટ્રાન્સમિશન
\item
  \textbf{લો કોસ્ટ}: સેટ અપ અને મેન્ટેઇન કરવા માટે સસ્તું
\item
  \textbf{પ્રાઇવેટ ઓનરશિપ}: સામાન્ય રીતે સિંગલ સંસ્થાની માલિકી
\end{itemize}

\textbf{MAN (મેટ્રોપોલિટન એરિયા નેટવર્ક):}

\begin{itemize}
\tightlist
\item
  \textbf{સિટી-વાઇડ કવરેજ}: મેટ્રોપોલિટન વિસ્તારમાં ફેલાયેલું
\item
  \textbf{મીડિયમ સ્પીડ}: મધ્યમ ટ્રાન્સમિશન સ્પીડ
\item
  \textbf{મિક્સ્ડ ઓનરશિપ}: પબ્લિક અથવા પ્રાઇવેટ હોઈ શકે
\end{itemize}

\textbf{WAN (વાઇડ એરિયા નેટવર્ક):}

\begin{itemize}
\tightlist
\item
  \textbf{ગ્લોબલ કવરેજ}: દેશો અને ખંડોમાં ફેલાયેલું
\item
  \textbf{વેરિયેબલ સ્પીડ}: કનેક્શન પ્રકાર પર આધાર રાખે
\item
  \textbf{પબ્લિક ઇન્ફ્રાસ્ટ્રક્ચર}: પબ્લિક ટેલિકમ્યુનિકેશન નેટવર્ક્સ ઉપયોગ કરે
\end{itemize}

\end{solutionbox}
\begin{mnemonicbox}
``LAN ઈઝ લોકલ, MAN ઈઝ મેટ્રોપોલિટન, WAN ઈઝ વાઇડ''

\end{mnemonicbox}
\subsection*{પ્રશ્ન 5(અ) [3
ગુણ]}\label{uxaaauxab0uxab6uxaa8-5uxa85-3-uxa97uxaa3}

\textbf{ડિનાયલ ઓફ સર્વિસ અટેક સમજાવો.}

\begin{solutionbox}

\textbf{DoS અટેક ડાયાગ્રામ:}

\begin{center}
\textbf{Mermaid Diagram (Code)}
\begin{verbatim}
{Shaded}
{Highlighting}[]
graph LR
    A[હુમલાખોર] {-{-}{} B[મલ્ટિપલ રિક્વેસ્ટ્સ]}
    B {-{-}{} C[ટાર્ગેટ સર્વર]}
    C {-{-}{} D[સર્વર ઓવરવ્હેલ્મ્ડ]}
    D {-{-}{} E[સર્વિસ અનઉપલબ્ધ]}
{Highlighting}
{Shaded}
\end{verbatim}
\end{center}

\textbf{DoS અટેક પ્રકારો કોષ્ટક:}

{\def\LTcaptype{none} % do not increment counter
\begin{longtable}[]{@{}ll@{}}
\toprule\noalign{}
પ્રકાર & વિવરણ \\
\midrule\noalign{}
\endhead
\bottomrule\noalign{}
\endlastfoot
\textbf{વોલ્યુમ-બેસ્ડ} & ટ્રાફિક સાથે બેન્ડવિથ ફ્લડ કરે \\
\textbf{પ્રોટોકોલ-બેસ્ડ} & પ્રોટોકોલ નબળાઈઓનો ફાયદો લે \\
\textbf{એપ્લિકેશન-બેસ્ડ} & એપ્લિકેશન રિસોર્સને ટાર્ગેટ કરે \\
\end{longtable}
}

\begin{itemize}
\tightlist
\item
  \textbf{ઉદ્દેશ્ય}: કાયદેસર યુઝર્સ માટે સેવાઓ અનઉપલબ્ધ બનાવવી
\item
  \textbf{પદ્ધતિઓ}: ટ્રાફિક ફ્લડિંગ, રિસોર્સ એક્ઝોશન, નબળાઈઓનો ફાયદો
\item
  \textbf{અસર}: સર્વિસ ડિસરપ્શન, ફાઇનાન્શિયલ લોસ, રેપ્યુટેશન ડેમેજ
\item
  \textbf{પ્રિવેન્શન}: ફાયરવોલ્સ, લોડ બેલેન્સર્સ, ઇન્ટ્રુઝન ડિટેક્શન સિસ્ટમ્સ
\end{itemize}

\end{solutionbox}
\begin{mnemonicbox}
``DoS ડિનાયઝ અધર્સ સર્વિસ''

\end{mnemonicbox}
\subsection*{પ્રશ્ન 5(બ) [4
ગુણ]}\label{uxaaauxab0uxab6uxaa8-5uxaac-4-uxa97uxaa3}

\textbf{i) ડેટા ટ્રાન્સમિશનનું વર્ગીકરણ કરો.} \textbf{ii) બસ ટોપોલોજીમાં
ટર્મિનેટરનો ઉપયોગ લખો.}

\begin{solutionbox}

\textbf{i) ડેટા ટ્રાન્સમિશન વર્ગીકરણ:}

\begin{center}
\textbf{Mermaid Diagram (Code)}
\begin{verbatim}
{Shaded}
{Highlighting}[]
graph TD
    A[ડેટા ટ્રાન્સમિશન] {-{-}{} B[દિશા]}
    A {-{-}{} C[ટાઇમિંગ]}
    A {-{-}{} D[મોડ]}
    
    B {-{-}{} E[સિમ્પ્લેક્સ]}
    B {-{-}{} F[હાફ{-}ડુપ્લેક્સ]}
    B {-{-}{} G[ફુલ{-}ડુપ્લેક્સ]}
    
    C {-{-}{} H[સિન્ક્રોનસ]}
    C {-{-}{} I[એસિન્ક્રોનસ]}
    
    D {-{-}{} J[સીરિયલ]}
    D {-{-}{} K[પેરેલલ]}
{Highlighting}
{Shaded}
\end{verbatim}
\end{center}

\textbf{ii) બસ ટોપોલોજીમાં ટર્મિનેટર:}

\textbf{ટર્મિનેટર ફંક્શન્સ કોષ્ટક:}

{\def\LTcaptype{none} % do not increment counter
\begin{longtable}[]{@{}ll@{}}
\toprule\noalign{}
ફંક્શન & વિવરણ \\
\midrule\noalign{}
\endhead
\bottomrule\noalign{}
\endlastfoot
\textbf{સિગ્નલ એબ્સોર્પ્શન} & સિગ્નલ રિફ્લેક્શન અટકાવે \\
\textbf{ઇમ્પીડન્સ મેચિંગ} & કેબલ ઇમ્પીડન્સ મેચ કરે \\
\textbf{નેટવર્ક ઇન્ટેગ્રિટી} & સિગ્નલ ગુણવત્તા જાળવે \\
\end{longtable}
}

\begin{itemize}
\tightlist
\item
  \textbf{રિફ્લેક્શન પ્રિવેન્શન}: સિગ્નલને વાપસ બાઉન્સ થવાથી રોકે
\item
  \textbf{સિગ્નલ ક્વોલિટી}: સ્વચ્છ સિગ્નલ ટ્રાન્સમિશન જાળવે
\item
  \textbf{બંને છેડે જરૂરી}: બસ ટોપોલોજીને કેબલના બંને છેડે ટર્મિનેટર જોઈએ
\item
  \textbf{રેઝિસ્ટન્સ વેલ્યુ}: ઇથરનેટ નેટવર્ક્સ માટે સામાન્ય રીતે 50 ઓહ્મ
\end{itemize}

\end{solutionbox}
\begin{mnemonicbox}
``ટર્મિનેટર સ્ટોપ્સ સિગ્નલ ટ્રાવેલ''

\end{mnemonicbox}
\subsection*{પ્રશ્ન 5(ક) [7
ગુણ]}\label{uxaaauxab0uxab6uxaa8-5uxa95-7-uxa97uxaa3}

\textbf{CIA ટ્રાઇડ વર્ણવો.}

\begin{solutionbox}

\textbf{CIA ટ્રાઇડ ડાયાગ્રામ:}

\begin{center}
\textbf{Mermaid Diagram (Code)}
\begin{verbatim}
{Shaded}
{Highlighting}[]
graph TD
    A[CIA ટ્રાઇડ] {-{-}{} B[કોન્ફિડેન્શિયાલિટી]}
    A {-{-}{} C[ઇન્ટેગ્રિટી]}
    A {-{-}{} D[અવેઇલેબિલિટી]}
    
    B {-{-}{} E[એન્ક્રિપ્શન]}
    B {-{-}{} F[એક્સેસ કંટ્રોલ]}
    
    C {-{-}{} G[હેશ ફંક્શન્સ]}
    C {-{-}{} H[ડિજિટલ સિગ્નેચર્સ]}
    
    D {-{-}{} I[રિડન્ડન્સી]}
    D {-{-}{} J[બેકઅપ સિસ્ટમ્સ]}
{Highlighting}
{Shaded}
\end{verbatim}
\end{center}

\textbf{CIA ટ્રાઇડ કમ્પોનન્ટ્સ કોષ્ટક:}

{\def\LTcaptype{none} % do not increment counter
\begin{longtable}[]{@{}
  >{\raggedright\arraybackslash}p{(\linewidth - 6\tabcolsep) * \real{0.2558}}
  >{\raggedright\arraybackslash}p{(\linewidth - 6\tabcolsep) * \real{0.2093}}
  >{\raggedright\arraybackslash}p{(\linewidth - 6\tabcolsep) * \real{0.3953}}
  >{\raggedright\arraybackslash}p{(\linewidth - 6\tabcolsep) * \real{0.1395}}@{}}
\toprule\noalign{}
\begin{minipage}[b]{\linewidth}\raggedright
કમ્પોનન્ટ
\end{minipage} & \begin{minipage}[b]{\linewidth}\raggedright
વ્યાખ્યા
\end{minipage} & \begin{minipage}[b]{\linewidth}\raggedright
ઇમ્પ્લિમેન્ટેશન
\end{minipage} & \begin{minipage}[b]{\linewidth}\raggedright
જોખમો
\end{minipage} \\
\midrule\noalign{}
\endhead
\bottomrule\noalign{}
\endlastfoot
\textbf{કોન્ફિડેન્શિયાલિટી} & માહિતીની ગુપ્તતા & એન્ક્રિપ્શન, એક્સેસ કંટ્રોલ &
અનધિકૃત ડિસક્લોઝર \\
\textbf{ઇન્ટેગ્રિટી} & ડેટાની ચોકસાઈ અને સંપૂર્ણતા & હેશ ફંક્શન્સ, ડિજિટલ સિગ્નેચર્સ &
ડેટા મોડિફિકેશન \\
\textbf{અવેઇલેબિલિટી} & માહિતીની પહોંચ યોગ્યતા & રિડન્ડન્સી, બેકઅપ સિસ્ટમ્સ &
સર્વિસ ડિસરપ્શન \\
\end{longtable}
}

\textbf{વિગતવાર સમજૂતી:}

\textbf{કોન્ફિડેન્શિયાલિટી:}

\begin{itemize}
\tightlist
\item
  \textbf{ડેટા પ્રોટેક્શન}: ફક્ત અધિકૃત યુઝર્સ જ માહિતી એક્સેસ કરી શકે
\item
  \textbf{પ્રાઇવસી પગલાં}: એન્ક્રિપ્શન, ઓથેન્ટિકેશન, એક્સેસ કંટ્રોલ્સ
\item
  \textbf{ઉદાહરણો}: પાસવર્ડ પ્રોટેક્શન, ફાઇલ પરમિશન્સ
\end{itemize}

\textbf{ઇન્ટેગ્રિટી:}

\begin{itemize}
\tightlist
\item
  \textbf{ડેટા એક્યુરસી}: ટ્રાન્સમિશન/સ્ટોરેજ દરમિયાન માહિતી બદલાતી નથી
\item
  \textbf{વેરિફિકેશન પદ્ધતિઓ}: ચેકસમ્સ, ડિજિટલ સિગ્નેચર્સ, વર્ઝન કંટ્રોલ
\item
  \textbf{ઉદાહરણો}: હેશ ફંક્શન્સ, ડેટાબેસ કન્સ્ટ્રેઇન્ટ્સ
\end{itemize}

\textbf{અવેઇલેબિલિટી:}

\begin{itemize}
\tightlist
\item
  \textbf{સિસ્ટમ એક્સેસિબિલિટી}: જરૂર પડે ત્યારે માહિતી અને સેવાઓ ઉપલબ્ધ
\item
  \textbf{રિલાયબિલિટી પગલાં}: રિડન્ડન્સી, ફોલ્ટ ટોલરન્સ, ડિઝાસ્ટર રિકવરી
\item
  \textbf{ઉદાહરણો}: લોડ બેલેન્સિંગ, બેકઅપ સિસ્ટમ્સ, UPS
\end{itemize}

\end{solutionbox}
\begin{mnemonicbox}
``CIA પ્રોટેક્ટ્સ - કોન્ફિડેન્શિયાલિટી, ઇન્ટેગ્રિટી,
અવેઇલેબિલિટી''

\end{mnemonicbox}
\subsection*{પ્રશ્ન 5(અ OR) [3
ગુણ]}\label{uxaaauxab0uxab6uxaa8-5uxa85-or-3-uxa97uxaa3}

\textbf{વ્યાખ્યાયિત કરો} \textbf{1. ક્રિપ્ટોગ્રાફી} \textbf{2. ડિક્રિપ્શન}

\begin{solutionbox}

\textbf{વ્યાખ્યા કોષ્ટક:}

{\def\LTcaptype{none} % do not increment counter
\begin{longtable}[]{@{}
  >{\raggedright\arraybackslash}p{(\linewidth - 4\tabcolsep) * \real{0.2857}}
  >{\raggedright\arraybackslash}p{(\linewidth - 4\tabcolsep) * \real{0.4286}}
  >{\raggedright\arraybackslash}p{(\linewidth - 4\tabcolsep) * \real{0.2857}}@{}}
\toprule\noalign{}
\begin{minipage}[b]{\linewidth}\raggedright
શબ્દ
\end{minipage} & \begin{minipage}[b]{\linewidth}\raggedright
વ્યાખ્યા
\end{minipage} & \begin{minipage}[b]{\linewidth}\raggedright
હેતુ
\end{minipage} \\
\midrule\noalign{}
\endhead
\bottomrule\noalign{}
\endlastfoot
\textbf{ક્રિપ્ટોગ્રાફી} & એન્કોડિંગ દ્વારા માહિતી સુરક્ષિત કરવાનું વિજ્ઞાન & ડેટા
કોન્ફિડેન્શિયાલિટી સુરક્ષિત કરવી \\
\textbf{ડિક્રિપ્શન} & એન્ક્રિપ્ટેડ ડેટાને મૂળ સ્વરૂપમાં પાછું કન્વર્ટ કરવાની પ્રક્રિયા &
મૂળ માહિતી પુનઃપ્રાપ્ત કરવી \\
\end{longtable}
}

\begin{itemize}
\tightlist
\item
  \textbf{ક્રિપ્ટોગ્રાફી}: વાંચી શકાય તેવા ડેટાને વાંચી ન શકાય તેવા ફોર્મેટમાં
  ટ્રાન્સફોર્મ કરવા માટે ગાણિતિક અલ્ગોરિધમ્સ ઉપયોગ કરે
\item
  \textbf{ડિક્રિપ્શન}: કીઝ ઉપયોગ કરીને મૂળ ડેટા પુનઃસ્થાપિત કરવાની વિપરીત
  પ્રક્રિયા
\item
  \textbf{કી-બેસ્ડ સિક્યોરિટી}: બંને પ્રક્રિયાઓ ક્રિપ્ટોગ્રાફિક કીઝ પર આધાર રાખે
\end{itemize}

\end{solutionbox}
\begin{mnemonicbox}
``ક્રિપ્ટો કન્સીલ્સ, ડિક્રિપ્શન ડિસ્ક્લોઝ''

\end{mnemonicbox}
\subsection*{પ્રશ્ન 5(બ OR) [4
ગુણ]}\label{uxaaauxab0uxab6uxaa8-5uxaac-or-4-uxa97uxaa3}

\textbf{i) ટ્વિસ્ટેડ પેર કેબલ્સમાં વાયરો શા માટે ટ્વિસ્ટેડ રાખવામાં આવે છે તેનું કારણ
જણાવો.} \textbf{ii) OSI મોડેલના સ્તરને ઓળખો કે જેના પર નીચેના નેટવર્ક ઉપકરણો
સપોર્ટ કરે છે 1. રાઉટર 2. બ્રિજ}

\begin{solutionbox}

\textbf{i) ટ્વિસ્ટેડ પેર કેબલ ડિઝાઇન:}

\begin{verbatim}
Normal Wires:     ||||||||||||
                  ||||||||||||
                  (પેરેલલ ઇન્ટરફેરન્સ)

Twisted Wires:    {//////}
                  /{/////}
                  (કેન્સલેશન ઇફેક્ટ)
\end{verbatim}

\textbf{વાયર ટ્વિસ્ટિંગ ફાયદાઓ કોષ્ટક:}

{\def\LTcaptype{none} % do not increment counter
\begin{longtable}[]{@{}ll@{}}
\toprule\noalign{}
ફાયદો & વિવરણ \\
\midrule\noalign{}
\endhead
\bottomrule\noalign{}
\endlastfoot
\textbf{નોઇઝ રિડક્શન} & ઇલેક્ટ્રોમેગ્નેટિક ઇન્ટરફેરન્સ કેન્સલ કરે \\
\textbf{ક્રોસટોક પ્રિવેન્શન} & પેર્સ વચ્ચે સિગ્નલ ઇન્ટરફેરન્સ ઘટાડે \\
\textbf{સિગ્નલ ક્વોલિટી} & બેહતર સિગ્નલ ઇન્ટેગ્રિટી જાળવે \\
\end{longtable}
}

\textbf{ii) OSI લેયર આઇડેન્ટિફિકેશન:}

\textbf{નેટવર્ક ડિવાઇસ અને OSI લેયર્સ કોષ્ટક:}

{\def\LTcaptype{none} % do not increment counter
\begin{longtable}[]{@{}lll@{}}
\toprule\noalign{}
ડિવાઇસ & OSI લેયર & ફંક્શન \\
\midrule\noalign{}
\endhead
\bottomrule\noalign{}
\endlastfoot
\textbf{રાઉટર} & લેયર 3 (નેટવર્ક) & વિવિધ નેટવર્ક્સ વચ્ચે રાઉટિંગ \\
\textbf{બ્રિજ} & લેયર 2 (ડેટા લિંક) & નેટવર્ક સેગમેન્ટ્સ કનેક્ટ કરવા \\
\end{longtable}
}

\begin{itemize}
\tightlist
\item
  \textbf{વાયર ટ્વિસ્ટિંગ}: દરેક ટ્વિસ્ટ બાજુના વાયરમાંથી ઇલેક્ટ્રોમેગ્નેટિક ઇન્ટરફેરન્સ
  કેન્સલ કરે
\item
  \textbf{ઇન્ટરફેરન્સ કેન્સલેશન}: નોઇઝ બંને વાયરને સમાન રીતે પરંતુ વિપરીત દિશામાં અસર
  કરે
\item
  \textbf{રાઉટર ફંક્શન}: IP એડ્રેસના આધારે રાઉટિંગ નિર્ણયો લે
\item
  \textbf{બ્રિજ ફંક્શન}: MAC એડ્રેસના આધારે ફ્રેમ્સ ફોરવર્ડ કરે
\end{itemize}

\end{solutionbox}
\begin{mnemonicbox}
``ટ્વિસ્ટેડ વાયર્સ રિડ્યુસ ઇન્ટરફેરન્સ, રાઉટર એટ લેયર 3, બ્રિજ
એટ લેયર 2''

\end{mnemonicbox}
\subsection*{પ્રશ્ન 5(ક OR) [7
ગુણ]}\label{uxaaauxab0uxab6uxaa8-5uxa95-or-7-uxa97uxaa3}

\textbf{સાયબર એટેકને વ્યાખ્યાયિત કરો અને વિવિધ સાયબર હુમલાઓને સંક્ષિપ્તમાં સમજાવો}

\begin{solutionbox}

\textbf{સાયબર એટેક વ્યાખ્યા:} સાયબર એટેક એ કમ્પ્યુટર સિસ્ટમ્સ, નેટવર્ક્સ અથવા ડિજિટલ
ડિવાઇસને કમ્પ્રોમાઇઝ કરવાનો ઇરાદાપૂર્વકનો પ્રયાસ છે જેથી ડેટા ચોરી, બદલાવ અથવા
નાશ કરી શકાય.

\textbf{સાયબર હુમલાઓના પ્રકારો:}

\begin{center}
\textbf{Mermaid Diagram (Code)}
\begin{verbatim}
{Shaded}
{Highlighting}[]
graph TD
    A[સાયબર એટેક્સ] {-{-}{} B[મેલવેર]}
    A {-{-}{} C[ફિશિંગ]}
    A {-{-}{} D[DoS/DDoS]}
    A {-{-}{} E[મેન{-}ઇન{-}મિડલ]}
    A {-{-}{} F[SQL ઇન્જેક્શન]}
    
    B {-{-}{} G[વાયરસ, વોર્મ, ટ્રોજન]}
    C {-{-}{} H[ઇમેઇલ, વેબસાઇટ]}
    D {-{-}{} I[ટ્રાફિક ફ્લડિંગ]}
    E {-{-}{} J[ઇવ્સડ્રોપિંગ]}
    F {-{-}{} K[ડેટાબેસ એટેક]}
{Highlighting}
{Shaded}
\end{verbatim}
\end{center}

\textbf{સાયબર એટેક પ્રકારો કોષ્ટક:}

{\def\LTcaptype{none} % do not increment counter
\begin{longtable}[]{@{}
  >{\raggedright\arraybackslash}p{(\linewidth - 6\tabcolsep) * \real{0.3611}}
  >{\raggedright\arraybackslash}p{(\linewidth - 6\tabcolsep) * \real{0.1944}}
  >{\raggedright\arraybackslash}p{(\linewidth - 6\tabcolsep) * \real{0.1389}}
  >{\raggedright\arraybackslash}p{(\linewidth - 6\tabcolsep) * \real{0.3056}}@{}}
\toprule\noalign{}
\begin{minipage}[b]{\linewidth}\raggedright
હુમલાનો પ્રકાર
\end{minipage} & \begin{minipage}[b]{\linewidth}\raggedright
વિવરણ
\end{minipage} & \begin{minipage}[b]{\linewidth}\raggedright
અસર
\end{minipage} & \begin{minipage}[b]{\linewidth}\raggedright
પ્રિવેન્શન
\end{minipage} \\
\midrule\noalign{}
\endhead
\bottomrule\noalign{}
\endlastfoot
\textbf{મેલવેર} & દુર્ભાવનાપૂર્ણ સોફ્ટવેર (વાયરસ, વોર્મ, ટ્રોજન) & સિસ્ટમ કરપ્શન,
ડેટા ચોરી & એન્ટીવાયરસ, અપડેટ્સ \\
\textbf{ફિશિંગ} & ક્રેડેન્શિયલ્સ ચોરવા માટે ફ્રોડ ઇમેઇલ્સ/વેબસાઇટ્સ & આઇડેન્ટિટી થેફ્ટ,
ફાઇનાન્શિયલ લોસ & યુઝર જાગૃતિ, ઇમેઇલ ફિલ્ટર્સ \\
\textbf{DoS/DDoS} & ટાર્ગેટને ટ્રાફિક સાથે ઓવરવ્હેલ્મ કરવું & સર્વિસ અનઉપલબ્ધતા &
ફાયરવોલ્સ, લોડ બેલેન્સર્સ \\
\textbf{મેન-ઇન-મિડલ} & પક્ષો વચ્ચે કમ્યુનિકેશન ઇન્ટરસેપ્ટ કરવું & ડેટા ઇવ્સડ્રોપિંગ &
એન્ક્રિપ્શન, સિક્યોર પ્રોટોકોલ્સ \\
\textbf{SQL ઇન્જેક્શન} & ડેટાબેસ ક્વેરીમાં દુર્ભાવનાપૂર્ણ કોડ દાખલ કરવો & ડેટાબેસ
કમ્પ્રોમાઇઝ & ઇનપુટ વેલિડેશન, પેરામીટરાઇઝ્ડ ક્વેરીઝ \\
\end{longtable}
}

\textbf{વિગતવાર હુમલાઓની સમજૂતી:}

\textbf{મેલવેર એટેક્સ:}

\begin{itemize}
\tightlist
\item
  \textbf{વાયરસ}: ફાઇલોમાં જોડાતો સ્વ-પ્રતિકૃતિ કોડ
\item
  \textbf{વોર્મ}: નેટવર્ક્સમાં ફેલાતો સ્ટેન્ડઅલોન મેલવેર
\item
  \textbf{ટ્રોજન}: કાયદેસર દેખાતો છુપાયેલો મેલવેર
\end{itemize}

\textbf{સોશિયલ એન્જિનીયરિંગ:}

\begin{itemize}
\tightlist
\item
  \textbf{ફિશિંગ}: સંવેદનશીલ માહિતી માંગતી નકલી ઇમેઇલ્સ
\item
  \textbf{સ્પીયર ફિશિંગ}: ચોક્કસ વ્યક્તિઓ પર ટાર્ગેટેડ હુમલાઓ
\item
  \textbf{બેઇટિંગ}: મેલવેર પહોંચાડવા માટે આકર્ષક ઓફર્સનો ઉપયોગ
\end{itemize}

\textbf{નેટવર્ક એટેક્સ:}

\begin{itemize}
\tightlist
\item
  \textbf{પેકેટ સ્નિફિંગ}: વિશ્લેષણ માટે નેટવર્ક ટ્રાફિક કેપ્ચર કરવું
\item
  \textbf{સેશન હાઇજેકિંગ}: યુઝર સેશન્સ કબજે કરવા
\item
  \textbf{પાસવર્ડ એટેક્સ}: બ્રુટ ફોર્સ, ડિક્શનરી એટેક્સ
\end{itemize}

\end{solutionbox}
\begin{mnemonicbox}
``MPDMS - મેલવેર, ફિશિંગ, DoS, મેન-ઇન-મિડલ, SQL ઇન્જેક્શન''

\end{mnemonicbox}

\end{document}
