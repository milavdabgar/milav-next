\documentclass[10pt,a4paper]{article}

% content/resources/templates/preamble.tex
\usepackage[margin=0.6in]{geometry}
\author{Milav Dabgar}
\usepackage{amsmath,amssymb,amsthm}
\usepackage{booktabs}
\usepackage{multirow}
\usepackage{xcolor}
\usepackage{tcolorbox}
\tcbuselibrary{breakable,skins}
\usepackage[colorlinks=true,linkcolor=blue]{hyperref}
\usepackage{titlesec}
\usepackage{enumitem}
\usepackage{tikz}
\usepackage{pgfplots}
\usepackage{circuitikz}
\usepackage[version=4]{mhchem}
\usepackage{longtable}
\usepackage{array}
\usepackage{float}
\usepackage{caption}
\usepackage{listings}

\lstset{
  basicstyle=\small\ttfamily,
  breaklines=true,
  breakatwhitespace=false,
  postbreak=\mbox{\textcolor{red}{$\hookrightarrow$}\space},
  float=false,
  numbers=left,
  numberstyle=\tiny\color{gray},
  numbersep=10pt,
  xleftmargin=2em,
  keywordstyle=\color{blue},
  commentstyle=\color{green!60!black},
  stringstyle=\color{purple},
  backgroundcolor=\color{gray!5},
  showstringspaces=false,
  tabsize=2,
  captionpos=b,
  keepspaces=true,
  columns=flexible
}

\pgfplotsset{compat=1.18}
\usetikzlibrary{shapes,arrows,positioning,calc,patterns,decorations.pathmorphing,decorations.markings,arrows.meta}

% Color scheme
\definecolor{headcolor}{RGB}{0,102,204}
\definecolor{keycolor}{RGB}{220,20,60}
\definecolor{solutioncolor}{RGB}{34,139,34}
\definecolor{mnemoniccolor}{RGB}{148,0,211}
\definecolor{codecolor}{RGB}{0,0,100}

% Spacing
\setlength{\parskip}{3pt}
\setlist[itemize]{nosep}
\setlist[enumerate]{nosep}

% Title formatting
\titleformat{\section}{\Large\bfseries\color{headcolor}}{\thesection}{1em}{}
\titleformat{\subsection}{\large\bfseries\color{headcolor}}{\thesubsection}{1em}{}

% Pandoc tightlist compatibility
\providecommand{\tightlist}{%
  \setlength{\itemsep}{0pt}\setlength{\parskip}{0pt}}

% Pandoc longtable compatibility
\newcounter{none}
\def\thenone{}


% content/resources/templates/gujarati-boxes.tex
\usepackage{fontspec}
\usepackage{polyglossia}

% Set Gujarati as main language (document is primarily in Gujarati)
% Note: gloss-gujarati.ldf doesn't exist in polyglossia, but it will use hyphenation patterns
\setdefaultlanguage{gujarati}
\setotherlanguage{english}

% Configure Gujarati font properly
% Use Language=Default to prevent polyglossia from trying to add language-specific features
% that don't exist for Gujarati, which causes "empty feature" warnings
\newfontfamily\gujaratifont[Script=Gujarati,AutoFakeBold=2.5,AutoFakeSlant=0.3]{Noto Sans Gujarati}
\setmainfont[Script=Gujarati,AutoFakeBold=2.5,AutoFakeSlant=0.3]{Noto Sans Gujarati}
% Use Noto Sans Gujarati for monospace to support Gujarati in text
\setmonofont[Scale=0.9]{Noto Sans Gujarati}

% Configure English to use the same font
\newfontfamily\englishfont[Script=Gujarati,AutoFakeBold=2.5,AutoFakeSlant=0.3]{Noto Sans Gujarati}

% Translations for polyglossia
\gappto\captionsgujarati{
  \renewcommand{\tablename}{કોષ્ટક}
  \renewcommand{\figurename}{આકૃતિ}
}

% Helper for TikZ nodes to ensure Gujarati font
\newcommand{\gu}[1]{{\gujaratifont #1}}

% Custom environments
\newtcolorbox{solutionbox}{
    breakable,
    enhanced,
    colback=solutioncolor!5!white,
    colframe=solutioncolor!75!black,
    fonttitle=\bfseries,
    title=જવાબ
}

\newtcolorbox{solutionboxnobreak}{
 colback=solutioncolor!5!white,
 colframe=solutioncolor!75!black,
 fonttitle=\bfseries,
 title=જવાબ
}

\newtcolorbox{keyformula}{
 breakable,
 enhanced,
 colback=keycolor!5!white,
 colframe=keycolor!75!black,
 fonttitle=\bfseries,
 title=રાસાયણિક સમીકરણ/સૂત્ર
}

\newtcolorbox{mnemonicbox}{
 breakable,
 enhanced,
 colback=mnemoniccolor!5!white,
 colframe=mnemoniccolor!75!black,
 fonttitle=\bfseries,
 title=મેમરી ટ્રીક
}


\begin{document}

\begin{center}
{\Huge\bfseries\color{headcolor} Subject Name (Gujarati)}\\[5pt]
{\LARGE 4311602 -- Winter 2024}\\[3pt]
{\large Semester 1 Study Material}\\[3pt]
{\normalsize\textit{Detailed Solutions and Explanations}}
\end{center}

\vspace{10pt}

\subsection*{પ્રશ્ન 1(a) [3
ગુણ]}\label{q1a}

\textbf{NAND લૉજિક ગેટ સમજાવો.}

\begin{solutionbox}

NAND ગેટ એક યુનિવર્સલ લૉજિક ગેટ છે જે માત્ર ત્યારે જ 0 આઉટપુટ આપે છે જ્યારે બધા ઇનપુટ્સ
1 હોય.

\textbf{ટ્રુથ ટેબલ:}

{\def\LTcaptype{none} % do not increment counter
\begin{longtable}[]{@{}lll@{}}
\toprule\noalign{}
A & B & Y = A NAND B \\
\midrule\noalign{}
\endhead
\bottomrule\noalign{}
\endlastfoot
0 & 0 & 1 \\
0 & 1 & 1 \\
1 & 0 & 1 \\
1 & 1 & 0 \\
\end{longtable}
}

\textbf{સિમ્બોલ:}

\begin{verbatim}
    A {-{-}{-}{-}+{-}{-}{-}Do{-}{-}{-} Y}
          |   |
    B {-{-}{-}{-}+   |}
              |
\end{verbatim}

\begin{itemize}
\tightlist
\item
  \textbf{NAND ફંક્શન}: આઉટપુટ એ AND ઓપરેશનનું કમ્પલિમેન્ટ છે
\item
  \textbf{યુનિવર્સલ ગેટ}: કોઈપણ લૉજિક ફંક્શન બનાવી શકે છે
\item
  \textbf{લો પાવર}: IC ડિઝાઇનમાં ઓછા ટ્રાન્ઝિસ્ટરની જરૂર
\end{itemize}

\end{solutionbox}
\begin{mnemonicbox}
``NOT AND = NAND''

\end{mnemonicbox}
\subsection*{પ્રશ્ન 1(b) [4
ગુણ]}\label{q1b}

\textbf{AND લૉજિક ગેટ ફક્ત NOR ગેટ વાપરીને દોરો.}

\begin{solutionbox}

AND ગેટને NOR ગેટ્સ વાપરીને ડી મોર્ગનના થિયરમ લાગુ કરીને બનાવી શકાય છે.

\textbf{સર્કિટ ડાયાગ્રામ:}

\begin{center}
\textbf{Mermaid Diagram (Code)}
\begin{verbatim}
{Shaded}
{Highlighting}[]
graph LR
    A[A] {-{-}{} N1[NOR]}
    A {-{-}{} N1}
    B[B] {-{-}{} N2[NOR]}
    B {-{-}{} N2}
    N1 {-{-}{} N3[NOR]}
    N2 {-{-}{} N3}
    N3 {-{-}{} Y[Y = A.B]}
{Highlighting}
{Shaded}
\end{verbatim}
\end{center}

\textbf{અમલીકરણના પગલાં:}

\begin{itemize}
\tightlist
\item
  \textbf{પગલું 1}: NOR ગેટ વાપરીને NOT A બનાવો (A NOR A = A')
\item
  \textbf{પગલું 2}: NOR ગેટ વાપરીને NOT B બનાવો (B NOR B = B')
\item
  \textbf{પગલું 3}: ડી મોર્ગન લાગુ કરો: A.B = (A' + B')'
\item
  \textbf{અંતિમ આઉટપુટ}: A AND B
\end{itemize}

\end{solutionbox}
\begin{mnemonicbox}
``ડબલ ઇન્વર્શન ઓરિજિનલ ફંક્શન આપે છે''

\end{mnemonicbox}
\subsection*{પ્રશ્ન 1(c) [7
ગુણ]}\label{q1c}

\textbf{ઇન્ફોર્મેશન સિસ્ટમના ઘટકો આકૃતિ સાથે સમજાવો.}

\begin{solutionbox}

ઇન્ફોર્મેશન સિસ્ટમમાં પાંચ મુખ્ય ઘટકો છે જે ડેટાને ઉપયોગી માહિતીમાં બદલવા માટે સાથે કામ
કરે છે.

\textbf{સિસ્ટમ ડાયાગ્રામ:}

\begin{verbatim}
graph TB
    subgraph "ઇન્ફોર્મેશન સિસ્ટમ"
        H[હાર્ડવેર]
        S[સોફ્ટવેર]
        D[ડેટા]
        P[પ્રોસીજર્સ]
        Pe[લોકો]
        
        H {-{-} P}
        S {-{-} P}
        D {-{-} P}
        Pe {-{-} P}
        P {-{-} H}
    end
    
    Input[ઇનપુટ] {-{-} H}
    H {-{-} Output[આઉટપુટ]}
\end{verbatim}

\textbf{ઘટકો:}

{\def\LTcaptype{none} % do not increment counter
\begin{longtable}[]{@{}lll@{}}
\toprule\noalign{}
ઘટક & વર્ણન & ઉદાહરણો \\
\midrule\noalign{}
\endhead
\bottomrule\noalign{}
\endlastfoot
\textbf{હાર્ડવેર} & ભૌતિક ઉપકરણો & CPU, મેમરી, કીબોર્ડ \\
\textbf{સોફ્ટવેર} & પ્રોગ્રામ્સ અને એપ્લિકેશન્સ & OS, એપ્લિકેશન્સ, યુટિલિટીઝ \\
\textbf{ડેટા} & કાચા તથ્યો અને આંકડાઓ & નંબરો, ટેક્સ્ટ, ઇમેજીસ \\
\textbf{પ્રોસીજર્સ} & નિયમો અને સૂચનાઓ & યુઝર મેન્યુઅલ્સ, SOPs \\
\textbf{લોકો} & વપરાશકર્તાઓ અને ઓપરેટર્સ & એન્ડ યુઝર્સ, IT સ્ટાફ \\
\end{longtable}
}

\begin{itemize}
\tightlist
\item
  \textbf{ઇનપુટ પ્રોસેસિંગ}: ડેટા હાર્ડવેર દ્વારા પ્રવેશે છે
\item
  \textbf{સ્ટોરેજ મેનેજમેન્ટ}: ડેટા કાર્યક્ષમતાથી સ્ટોર અને રિટ્રીવ થાય છે
\item
  \textbf{આઉટપુટ જનરેશન}: માહિતી વપરાશકર્તાઓને પ્રસ્તુત કરવામાં આવે છે
\item
  \textbf{ઇન્ટીગ્રેશન}: બધા ઘટકો સમન્વયથી કામ કરે છે
\end{itemize}

\end{solutionbox}
\begin{mnemonicbox}
``હાર્ડવેર સપોર્ટ્સ ડેટા પ્રોસેસિંગ પીપલ''

\end{mnemonicbox}
\subsection*{પ્રશ્ન 1(c OR) [7
ગુણ]}\label{uxaaauxab0uxab6uxaa8-1c-or-7-uxa97uxaa3}

\textbf{Google Search Engine ની કાર્યપદ્ધતિ ઉદાહરણ સાથે સમજાવો.}

\begin{solutionbox}

Google Search Engine વપરાશકર્તાના ક્વેરીઝના આધારે વેબ પેજીસ શોધવા અને રેન્ક કરવા
માટે જટિલ અલ્ગોરિધમ્સ વાપરે છે.

\textbf{કાર્યપ્રક્રિયા:}

\begin{verbatim}
sequenceDiagram
    participant U as વપરાશકર્તા
    participant G as Google
    participant I as ઇન્ડેક્સ
    participant W as વેબ પેજીસ
    
    U{-G: સર્ચ ક્વેરી દાખલ કરો}
    G{-I: ક્વેરી પ્રોસેસિંગ}
    I{-G: સંબંધિત પેજીસ પુનઃપ્રાપ્ત કરો}
    G{-G: પેજીસ રેન્ક કરો (PageRank)}
    G{-U: પરિણામો દર્શાવો}
\end{verbatim}

\textbf{મુખ્ય ઘટકો:}

{\def\LTcaptype{none} % do not increment counter
\begin{longtable}[]{@{}lll@{}}
\toprule\noalign{}
તબક્કો & પ્રક્રિયા & ઉદાહરણ \\
\midrule\noalign{}
\endhead
\bottomrule\noalign{}
\endlastfoot
\textbf{ક્રોલિંગ} & વેબ પેજીસ શોધો & Googlebot વેબસાઇટ્સની મુલાકાત લે છે \\
\textbf{ઇન્ડેક્સિંગ} & પેજ કન્ટેન્ટ સ્ટોર કરો & કીવર્ડ્સ ડેટાબેઝમાં સ્ટોર થાય છે \\
\textbf{રેન્કિંગ} & પ્રાસંગિકતા પ્રમાણે ક્રમાંકિત કરો & PageRank અલ્ગોરિધમ \\
\textbf{સર્વિંગ} & પરિણામો પ્રદર્શિત કરો & સર્ચ રિઝલ્ટ પેજ \\
\end{longtable}
}

\textbf{ઉદાહરણ સર્ચ પ્રક્રિયા:}

\begin{itemize}
\item
  \textbf{ક્વેરી}: ``Introduction to IT Systems''
\item
  \textbf{પ્રોસેસિંગ}: કીવર્ડ્સ પાર્સ કરો, ઇન્ડેક્સ ચેક કરો
\item
  \textbf{રેન્કિંગ}: શૈક્ષણિક સાઇટ્સને વધુ રેન્ક આપો
\item
  \textbf{પરિણામો}: GTU સિલેબસ, ટ્યુટોરિયલ્સ, કોર્સીસ
\item
  \textbf{PageRank અલ્ગોરિધમ}: લિંક્સ પેજની મહત્વતા નક્કી કરે છે
\item
  \textbf{મશીન લર્નિંગ}: સમય જતાં સર્ચ અચોક્કસતા સુધારે છે
\item
  \textbf{રીઅલ-ટાઇમ અપડેટ્સ}: નવા કન્ટેન્ટને પ્રાથમિકતા
\end{itemize}

\end{solutionbox}
\begin{mnemonicbox}
``ક્રોલ ઇન્ડેક્સ રેન્ક સર્વ''

\end{mnemonicbox}
\subsection*{પ્રશ્ન 2(a) [3
ગુણ]}\label{q2a}

\textbf{રૂપાંતરણ (16.75)10= ( )8}

\begin{solutionbox}

દશાંશ 16.75 ને અષ્ટાંશમાં રૂપાંતરિત કરવા માટે પૂર્ણાંક અને દશાંશ ભાગનું અલગ રૂપાંતરણ જરૂરી
છે.

\textbf{પૂર્ણાંક ભાગનું રૂપાંતરણ (16):}

{\def\LTcaptype{none} % do not increment counter
\begin{longtable}[]{@{}lll@{}}
\toprule\noalign{}
ભાગાકાર & ભાગફળ & શેષ \\
\midrule\noalign{}
\endhead
\bottomrule\noalign{}
\endlastfoot
16 \div 8 & 2 & 0 \\
2 \div 8 & 0 & 2 \\
\end{longtable}
}

\textbf{દશાંશ ભાગનું રૂપાંતરણ (0.75):}

{\def\LTcaptype{none} % do not increment counter
\begin{longtable}[]{@{}ll@{}}
\toprule\noalign{}
ગુણાકાર & પૂર્ણાંક ભાગ \\
\midrule\noalign{}
\endhead
\bottomrule\noalign{}
\endlastfoot
0.75 \times 8 = 6.0 & 6 \\
\end{longtable}
}

\textbf{અંતિમ જવાબ}: (16.75)10 = (20.6)8

\textbf{ચકાસણી}: 2\times8^{1} + 0\times8^{0} + 6\times8^{-}^{1} = 16 + 0 + 0.75 = 16.75 ✓

\end{solutionbox}
\begin{mnemonicbox}
``પૂર્ણાંકનો ભાગાકાર, દશાંશનો ગુણાકાર''

\end{mnemonicbox}
\subsection*{પ્રશ્ન 2(b) [4
ગુણ]}\label{q2b}

\textbf{મલ્ટિપ્રોસેસિંગ ઓપરેટિંગ સિસ્ટમ સમજાવો.}

\begin{solutionbox}

મલ્ટિપ્રોસેસિંગ OS એકસાથે કામ કરતા બહુવિધ પ્રોસેસર્સનું સંચાલન કરીને પ્રોસેસીસ એક્ઝીક્યુટ
કરે છે.

\textbf{આર્કિટેક્ચર ડાયાગ્રામ:}

\begin{verbatim}
graph TB
    subgraph "મલ્ટિપ્રોસેસિંગ સિસ્ટમ"
        CPU1[CPU 1]
        CPU2[CPU 2]
        CPU3[CPU 3]
        SM[શેર્ડ મેમરી]
        OS[ઓપરેટિંગ સિસ્ટમ]
        
        CPU1 {-{-} SM}
        CPU2 {-{-} SM}
        CPU3 {-{-} SM}
        OS {-{-} CPU1}
        OS {-{-} CPU2}
        OS {-{-} CPU3}
    end
\end{verbatim}

\textbf{મુખ્ય લક્ષણો:}

{\def\LTcaptype{none} % do not increment counter
\begin{longtable}[]{@{}lll@{}}
\toprule\noalign{}
લક્ષણ & વર્ણન & ફાયદો \\
\midrule\noalign{}
\endhead
\bottomrule\noalign{}
\endlastfoot
\textbf{પેરેલલ પ્રોસેસિંગ} & બહુવિધ CPUs સાથે કામ કરે છે & ઝડપી એક્ઝીક્યુશન \\
\textbf{લોડ બેલેન્સિંગ} & કાર્યો સમાનરૂપે વિતરિત કરે છે & શ્રેષ્ઠ રિસોર્સ ઉપયોગ \\
\textbf{ફૉલ્ટ ટોલરન્સ} & એક CPU ફેઇલ થાય તો સિસ્ટમ ચાલુ રહે છે & વધુ
વિશ્વસનીયતા \\
\textbf{શેર્ડ રિસોર્સીસ} & સામાન્ય મેમરી અને I/O ઉપકરણો & ખર્ચ અસરકારક \\
\end{longtable}
}

\begin{itemize}
\tightlist
\item
  \textbf{સિમેટ્રિક મલ્ટિપ્રોસેસિંગ}: બધા પ્રોસેસર્સને સમાન એક્સેસ
\item
  \textbf{પ્રોસેસ સિન્ક્રોનાઇઝેશન}: પ્રોસેસર્સ વચ્ચે સમન્વય
\item
  \textbf{વર્ધિત પ્રદર્શન}: પ્રોસેસર કાઉન્ટ સાથે લિનિયર સ્પીડઅપ
\end{itemize}

\end{solutionbox}
\begin{mnemonicbox}
``મલ્ટિપલ પ્રોસેસર્સ પેરેલલ પ્રોસેસ''

\end{mnemonicbox}
\subsection*{પ્રશ્ન 2(c) [7
ગુણ]}\label{q2c}

\textbf{ઓપરેટિંગ સિસ્ટમની વ્યાખ્યા આપો. ઓપરેટિંગ સિસ્ટમના કાર્યોની યાદી બનાવો અને
સમજાવો.}

\begin{solutionbox}

\textbf{વ્યાખ્યા}: ઓપરેટિંગ સિસ્ટમ એ સિસ્ટમ સોફ્ટવેર છે જે કમ્પ્યુટર હાર્ડવેરનું સંચાલન કરે
છે અને એપ્લિકેશન પ્રોગ્રામ્સને સેવાઓ પૂરી પાડે છે.

\textbf{મુખ્ય કાર્યો:}

\begin{verbatim}
mindmap
  root((ઓપરેટિંગ સિસ્ટમ))
    પ્રોસેસ મેનેજમેન્ટ
      પ્રોસેસ ક્રિએશન
      શેડ્યુલિંગ
      સિન્ક્રોનાઇઝેશન
    મેમરી મેનેજમેન્ટ
      એલોકેશન
      વર્ચ્યુઅલ મેમરી
      પેજિંગ
    ફાઇલ મેનેજમેન્ટ
      ફાઇલ ઓપરેશન્સ
      ડિરેક્ટરી સ્ટ્રક્ચર
      એક્સેસ કન્ટ્રોલ
    I/O મેનેજમેન્ટ
      ડિવાઇસ ડ્રાઇવર્સ
      બફરિંગ
      સ્પૂલિંગ
\end{verbatim}

\textbf{વિગતવાર કાર્યો:}

{\def\LTcaptype{none} % do not increment counter
\begin{longtable}[]{@{}
  >{\raggedright\arraybackslash}p{(\linewidth - 4\tabcolsep) * \real{0.2500}}
  >{\raggedright\arraybackslash}p{(\linewidth - 4\tabcolsep) * \real{0.3333}}
  >{\raggedright\arraybackslash}p{(\linewidth - 4\tabcolsep) * \real{0.4167}}@{}}
\toprule\noalign{}
\begin{minipage}[b]{\linewidth}\raggedright
કાર્ય
\end{minipage} & \begin{minipage}[b]{\linewidth}\raggedright
વર્ણન
\end{minipage} & \begin{minipage}[b]{\linewidth}\raggedright
ઉદાહરણો
\end{minipage} \\
\midrule\noalign{}
\endhead
\bottomrule\noalign{}
\endlastfoot
\textbf{પ્રોસેસ મેનેજમેન્ટ} & પ્રોગ્રામ એક્ઝીક્યુશનનું નિયંત્રણ & ટાસ્ક શેડ્યુલિંગ,
મલ્ટિટાસ્કિંગ \\
\textbf{મેમરી મેનેજમેન્ટ} & RAM ને કાર્યક્ષમતાથી ફાળવે છે & વર્ચ્યુઅલ મેમરી, પેજિંગ \\
\textbf{ફાઇલ મેનેજમેન્ટ} & ડેટા સ્ટોરેજનું આયોજન & ફાઇલ સિસ્ટમ્સ, ડિરેક્ટરીઝ \\
\textbf{I/O મેનેજમેન્ટ} & ઇનપુટ/આઉટપુટ ઉપકરણોનું નિયંત્રણ & પ્રિન્ટર સ્પૂલિંગ, ડિસ્ક
એક્સેસ \\
\textbf{સિક્યોરિટી} & સિસ્ટમ રિસોર્સીસનું રક્ષણ & યુઝર ઓથેન્ટિકેશન, એક્સેસ કન્ટ્રોલ \\
\end{longtable}
}

\begin{itemize}
\tightlist
\item
  \textbf{રિસોર્સ એલોકેશન}: CPU ટાઇમ અને મેમરીનું વિતરણ
\item
  \textbf{યુઝર ઇન્ટરફેસ}: કમાન્ડ લાઇન અથવા GUI ઇન્ટરેક્શન પૂરું પાડે છે
\item
  \textbf{એરર હેન્ડલિંગ}: સિસ્ટમ ફેઇલ્યોર્સનું ગ્રેસફુલ મેનેજમેન્ટ
\item
  \textbf{સિસ્ટમ કૉલ્સ}: એપ્લિકેશન્સ અને હાર્ડવેર વચ્ચે ઇન્ટરફેસ
\end{itemize}

\end{solutionbox}
\begin{mnemonicbox}
``પ્રોસેસ મેમરી ફાઇલ્સ ઇનપુટ-આઉટપુટ સિક્યોરિટી''

\end{mnemonicbox}
\subsection*{પ્રશ્ન 2(a OR) [3
ગુણ]}\label{uxaaauxab0uxab6uxaa8-2a-or-3-uxa97uxaa3}

\textbf{રૂપાંતરણ (1111111.11)2 = ( )10}

\begin{solutionbox}

દ્વિસંખ્યાને દશાંશમાં સ્થાનિક સંકેત પદ્ધતિ વાપરીને રૂપાંતરિત કરવું.

\textbf{રૂપાંતરણ ટેબલ:}

{\def\LTcaptype{none} % do not increment counter
\begin{longtable}[]{@{}llll@{}}
\toprule\noalign{}
સ્થાન & બિટ & ઘાત & મૂલ્ય \\
\midrule\noalign{}
\endhead
\bottomrule\noalign{}
\endlastfoot
6 & 1 & 2^{6} & 64 \\
5 & 1 & 2^{5} & 32 \\
4 & 1 & 2^{4} & 16 \\
3 & 1 & 2^{3} & 8 \\
2 & 1 & 2^{2} & 4 \\
1 & 1 & 2^{1} & 2 \\
0 & 1 & 2^{0} & 1 \\
-1 & 1 & 2^{-}^{1} & 0.5 \\
-2 & 1 & 2^{-}^{2} & 0.25 \\
\end{longtable}
}

\textbf{ગણતરી}: 64 + 32 + 16 + 8 + 4 + 2 + 1 + 0.5 + 0.25 = 127.75

\textbf{અંતિમ જવાબ}: (1111111.11)2 = (127.75)10

\end{solutionbox}
\begin{mnemonicbox}
``બેની ઘાતાઓ એકસાથે ઉમેરો''

\end{mnemonicbox}
\subsection*{પ્રશ્ન 2(b OR) [4
ગુણ]}\label{uxaaauxab0uxab6uxaa8-2b-or-4-uxa97uxaa3}

\textbf{બેચ ઓપરેટિંગ સિસ્ટમ સમજાવો.}

\begin{solutionbox}

બેચ OS એક્ઝીક્યુશન દરમિયાન યુઝર ઇન્ટરેક્શન વિના જ જોબ્સને ગ્રૂપમાં પ્રોસેસ કરે છે.

\textbf{વર્કિંગ મોડલ:}

\begin{center}
\textbf{Mermaid Diagram (Code)}
\begin{verbatim}
{Shaded}
{Highlighting}[]
graph LR
    subgraph "બેચ પ્રોસેસિંગ"
        J1[જોબ 1] {-{-}{} Q[જોબ ક્યૂ]}
        J2[જોબ 2] {-{-}{} Q}
        J3[જોબ 3] {-{-}{} Q}
        Q {-{-}{} CPU[CPU પ્રોસેસિંગ]}
        CPU {-{-}{} O[આઉટપુટ]}
    end
{Highlighting}
{Shaded}
\end{verbatim}
\end{center}

\textbf{લક્ષણો:}

{\def\LTcaptype{none} % do not increment counter
\begin{longtable}[]{@{}lll@{}}
\toprule\noalign{}
લક્ષણ & વર્ણન & અસર \\
\midrule\noalign{}
\endhead
\bottomrule\noalign{}
\endlastfoot
\textbf{કોઈ ઇન્ટરેક્શન નહીં} & જોબ્સ યુઝર ઇનપુટ વિના ચાલે છે & ઉચ્ચ થ્રુપુટ \\
\textbf{જોબ ક્યૂ} & બહુવિધ જોબ્સ ક્રમમાં રાહ જુએ છે & કાર્યક્ષમ પ્રોસેસિંગ \\
\textbf{ઓટોમેટિક શેડ્યુલિંગ} & OS આગળનો જોબ પસંદ કરે છે & ન્યૂનતમ ઓવરહેડ \\
\textbf{બેચ પ્રોસેસિંગ} & સમાન જોબ્સ એકસાથે ગ્રૂપ કરવામાં આવે છે & રિસોર્સ
ઓપ્ટિમાઇઝેશન \\
\end{longtable}
}

\begin{itemize}
\tightlist
\item
  \textbf{ફાયદાઓ}: ઉચ્ચ સિસ્ટમ ઉપયોગ, ખર્ચ અસરકારક
\item
  \textbf{નુકસાનો}: કોઈ રીઅલ-ટાઇમ ઇન્ટરેક્શન નહીં, ડીબગિંગ મુશ્કેલી
\item
  \textbf{એપ્લિકેશન્સ}: પેરોલ પ્રોસેસિંગ, ડેટા બેકઅપ સિસ્ટમ્સ
\end{itemize}

\end{solutionbox}
\begin{mnemonicbox}
``બેચ જોબ્સ ક્યૂ ઓટોમેટિકલી''

\end{mnemonicbox}
\subsection*{પ્રશ્ન 2(c OR) [7
ગુણ]}\label{uxaaauxab0uxab6uxaa8-2c-or-7-uxa97uxaa3}

\textbf{લિનક્સ સિસ્ટમનું આર્કિટેક્ચર અને મોડ્સ આકૃતિ સાથે સમજાવો.}

\begin{solutionbox}

લિનક્સ વિશિષ્ટ યુઝર અને કર્નલ મોડ્સ સાથે સ્તરીય આર્કિટેક્ચરને અનુસરે છે.

\textbf{સિસ્ટમ આર્કિટેક્ચર:}

\begin{verbatim}
graph TB
    subgraph "યુઝર સ્પેસ"
        UA[યુઝર એપ્લિકેશન્સ]
        SL[સિસ્ટમ લાઇબ્રેરીઝ]
        SC[સિસ્ટમ કૉલ્સ]
    end
    
    subgraph "કર્નલ સ્પેસ"
        VFS[વર્ચ્યુઅલ ફાઇલ સિસ્ટમ]
        PM[પ્રોસેસ મેનેજમેન્ટ]
        MM[મેમરી મેનેજમેન્ટ]
        NM[નેટવર્ક મેનેજમેન્ટ]
        DM[ડિવાઇસ મેનેજમેન્ટ]
    end
    
    HW[હાર્ડવેર]
    
    UA {-{-} SL}
    SL {-{-} SC}
    SC {-{-} VFS}
    SC {-{-} PM}
    SC {-{-} MM}
    SC {-{-} NM}
    SC {-{-} DM}
    VFS {-{-} HW}
    PM {-{-} HW}
    MM {-{-} HW}
    NM {-{-} HW}
    DM {-{-} HW}
\end{verbatim}

\textbf{ઓપરેટિંગ મોડ્સ:}

{\def\LTcaptype{none} % do not increment counter
\begin{longtable}[]{@{}lll@{}}
\toprule\noalign{}
મોડ & વર્ણન & એક્સેસ લેવલ \\
\midrule\noalign{}
\endhead
\bottomrule\noalign{}
\endlastfoot
\textbf{યુઝર મોડ} & એપ્લિકેશન્સ અહીં ચાલે છે & મર્યાદિત વિશેષાધિકારો \\
\textbf{કર્નલ મોડ} & OS કોર ફંક્શન્સ & સંપૂર્ણ હાર્ડવેર એક્સેસ \\
\textbf{સિસ્ટમ કૉલ ઇન્ટરફેસ} & કમ્યુનિકેશન બ્રિજ & નિયંત્રિત સંક્રમણ \\
\end{longtable}
}

\textbf{મુખ્ય ઘટકો:}

\begin{itemize}
\item
  \textbf{શેલ}: કમાન્ડ ઇન્ટરપ્રીટર ઇન્ટરફેસ
\item
  \textbf{કર્નલ}: કોર સિસ્ટમ મેનેજમેન્ટ
\item
  \textbf{ફાઇલ સિસ્ટમ}: હાયરાર્કિકલ ડેટા ઓર્ગેનાઇઝેશન
\item
  \textbf{ડિવાઇસ ડ્રાઇવર્સ}: હાર્ડવેર એબ્સ્ટ્રેક્શન લેયર
\item
  \textbf{સિક્યોરિટી મોડલ}: પરમિશન-આધારિત એક્સેસ કન્ટ્રોલ
\item
  \textbf{મોડ્યુલેરિટી}: લોડેબલ કર્નલ મોડ્યુલ્સ લવચીકતા માટે
\item
  \textbf{પોર્ટેબિલિટી}: બહુવિધ હાર્ડવેર પ્લેટફોર્મ પર ચાલે છે
\end{itemize}

\end{solutionbox}
\begin{mnemonicbox}
``યુઝર્સ કર્નલને હાર્ડવેર માટે કૉલ કરે છે''

\end{mnemonicbox}
\subsection*{પ્રશ્ન 3(a) [3
ગુણ]}\label{q3a}

\textbf{ઓપન સોર્સ સોફ્ટવેર અને પ્રોપ્રાઇટરી સોફ્ટવેર વચ્ચે ફરક લખો.}

\begin{solutionbox}

\textbf{તુલના ટેબલ:}

{\def\LTcaptype{none} % do not increment counter
\begin{longtable}[]{@{}
  >{\raggedright\arraybackslash}p{(\linewidth - 4\tabcolsep) * \real{0.1304}}
  >{\raggedright\arraybackslash}p{(\linewidth - 4\tabcolsep) * \real{0.4130}}
  >{\raggedright\arraybackslash}p{(\linewidth - 4\tabcolsep) * \real{0.4565}}@{}}
\toprule\noalign{}
\begin{minipage}[b]{\linewidth}\raggedright
પાસું
\end{minipage} & \begin{minipage}[b]{\linewidth}\raggedright
ઓપન સોર્સ સોફ્ટવેર
\end{minipage} & \begin{minipage}[b]{\linewidth}\raggedright
પ્રોપ્રાઇટરી સોફ્ટવેર
\end{minipage} \\
\midrule\noalign{}
\endhead
\bottomrule\noalign{}
\endlastfoot
\textbf{સોર્સ કોડ} & મુક્તપણે ઉપલબ્ધ & બંધ અને સુરક્ષિત \\
\textbf{કિંમત} & સામાન્યપણે મફત & કોમર્શિયલ લાઇસન્સ જરૂરી \\
\textbf{મોડિફિકેશન} & બદલી શકાય છે & બદલી શકાતું નથી \\
\textbf{ઉદાહરણો} & Linux, Firefox, LibreOffice & Windows, MS Office,
Photoshop \\
\textbf{સપોર્ટ} & કમ્યુનિટી-આધારિત & વેન્ડર-પ્રદાન \\
\textbf{લાઇસન્સિંગ} & GPL, MIT, Apache & EULA, કોમર્શિયલ \\
\end{longtable}
}

\textbf{મુખ્ય ફરકો:}

\begin{itemize}
\tightlist
\item
  \textbf{સ્વતંત્રતા}: ઓપન સોર્સ સંપૂર્ણ કસ્ટમાઇઝેશનની મંજૂરી આપે છે
\item
  \textbf{સિક્યોરિટી}: ઓપન કોડ કમ્યુનિટી સિક્યોરિટી રિવ્યુ સક્ષમ કરે છે
\item
  \textbf{વેન્ડર લોક-ઇન}: પ્રોપ્રાઇટરી વેન્ડર પર નિર્ભરતા બનાવે છે
\end{itemize}

\end{solutionbox}
\begin{mnemonicbox}
``ઓપન શેર કરે છે, પ્રોપ્રાઇટરી રક્ષણ કરે છે''

\end{mnemonicbox}
\subsection*{પ્રશ્ન 3(b) [4
ગુણ]}\label{q3b}

\textbf{ઇથરનેટ કેબલ સમજાવો.}

\begin{solutionbox}

ઇથરનેટ કેબલ LAN કનેક્શન્સ માટે સ્ટાન્ડર્ડ વાયર્ડ નેટવર્કિંગ માધ્યમ છે.

\textbf{કેબલ પ્રકારો:}

\begin{center}
\textbf{Mermaid Diagram (Code)}
\begin{verbatim}
{Shaded}
{Highlighting}[]
graph TD
    subgraph "ઇથરનેટ કેબલ્સ"
        UTP[અનશીલ્ડેડ ટ્વિસ્ટેડ પેર]
        STP[શીલ્ડેડ ટ્વિસ્ટેડ પેર]
        Coax[કોએક્સિયલ કેબલ]
        Fiber[ફાઇબર ઓપ્ટિક]
    end
    
    UTP {-{-}{} Cat5[Cat 5/5e/6/6a]}
    Fiber {-{-}{} SM[સિંગલ મોડ]}
    Fiber {-{-}{} MM[મલ્ટિ મોડ]}
{Highlighting}
{Shaded}
\end{verbatim}
\end{center}

\textbf{કેબલ સ્પેસિફિકેશન્સ:}

{\def\LTcaptype{none} % do not increment counter
\begin{longtable}[]{@{}llll@{}}
\toprule\noalign{}
પ્રકાર & સ્પીડ & અંતર & ઉપયોગ \\
\midrule\noalign{}
\endhead
\bottomrule\noalign{}
\endlastfoot
\textbf{Cat 5e} & 1 Gbps & 100m & બેઝિક નેટવર્કિંગ \\
\textbf{Cat 6} & 10 Gbps & 55m & હાઇ-સ્પીડ LAN \\
\textbf{Cat 6a} & 10 Gbps & 100m & એન્ટરપ્રાઇઝ નેટવર્ક્સ \\
\textbf{ફાઇબર ઓપ્ટિક} & 100+ Gbps & 40km+ & લાંબા અંતર, હાઇ-સ્પીડ \\
\end{longtable}
}

\begin{itemize}
\tightlist
\item
  \textbf{કનેક્ટર ટાઇપ}: ટ્વિસ્ટેડ પેર કેબલ્સ માટે RJ-45
\item
  \textbf{વાયરિંગ સ્ટાન્ડર્ડ્સ}: T568A અને T568B કલર કોડ્સ
\item
  \textbf{એપ્લિકેશન્સ}: ઇન્ટરનેટ કનેક્ટિવિટી, ફાઇલ શેરિંગ, VoIP
\end{itemize}

\end{solutionbox}
\begin{mnemonicbox}
``ટ્વિસ્ટેડ પેર્સ ડિજિટલ ડેટા વહન કરે છે''

\end{mnemonicbox}
\subsection*{પ્રશ્ન 3(c) [7
ગુણ]}\label{q3c}

\textbf{ટાઇમ ડિવિઝન મલ્ટિપ્લેક્સિંગ આકૃતિ સાથે સમજાવો.}

\begin{solutionbox}

TDM ટાઇમ સ્લોટ્સ ફાળવીને બહુવિધ સિગ્નલ્સને સિંગલ ટ્રાન્સમિશન માધ્યમ શેર કરવાની મંજૂરી
આપે છે.

\textbf{TDM પ્રક્રિયા:}

\begin{verbatim}
gantt
    title ટાઇમ ડિવિઝન મલ્ટિપ્લેક્સિંગ
    dateFormat X
    axisFormat \%s
    
    section ચેનલ A
    સ્લોટ A1 :0, 1
    સ્લોટ A2 :4, 5
    સ્લોટ A3 :8, 9
    
    section ચેનલ B
    સ્લોટ B1 :1, 2
    સ્લોટ B2 :5, 6
    સ્લોટ B3 :9, 10
    
    section ચેનલ C
    સ્લોટ C1 :2, 3
    સ્લોટ C2 :6, 7
    સ્લોટ C3 :10, 11
    
    section ચેનલ D
    સ્લોટ D1 :3, 4
    સ્લોટ D2 :7, 8
    સ્લોટ D3 :11, 12
\end{verbatim}

\textbf{સિસ્ટમ ઘટકો:}

{\def\LTcaptype{none} % do not increment counter
\begin{longtable}[]{@{}lll@{}}
\toprule\noalign{}
ઘટક & કાર્ય & હેતુ \\
\midrule\noalign{}
\endhead
\bottomrule\noalign{}
\endlastfoot
\textbf{મલ્ટિપ્લેક્સર} & ઇનપુટ સિગ્નલ્સને જોડે છે & સિંગલ ટ્રાન્સમિશન \\
\textbf{ટાઇમ સ્લોટ્સ} & નિશ્ચિત અવધિના અંતરાલો & ન્યાયી ચેનલ એક્સેસ \\
\textbf{ડીમલ્ટિપ્લેક્સર} & કંબાઇન્ડ સિગ્નલને અલગ કરે છે & ઓરિજિનલ સિગ્નલ રિકવરી \\
\textbf{સિંક્રોનાઇઝેશન} & ટાઇમિંગ એલાઇનમેન્ટ જાળવે છે & એરર-ફ્રી ટ્રાન્સમિશન \\
\end{longtable}
}

\textbf{TDM ના પ્રકારો:}

\begin{itemize}
\item
  \textbf{સિંક્રોનસ TDM}: દરેક ચેનલ માટે નિશ્ચિત ટાઇમ સ્લોટ્સ
\item
  \textbf{એસિંક્રોનસ TDM}: માંગના આધારે ડાયનેમિક સ્લોટ એલોકેશન
\item
  \textbf{સ્ટેટિસ્ટિકલ TDM}: બેન્ડવિડ્થ ઉપયોગને ઓપ્ટિમાઇઝ કરે છે
\item
  \textbf{ફાયદાઓ}: કાર્યક્ષમ બેન્ડવિડ્થ ઉપયોગ, ડિજિટલ સુસંગતતા
\item
  \textbf{એપ્લિકેશન્સ}: ટેલિફોન સિસ્ટમ્સ, ડિજિટલ TV બ્રોડકાસ્ટિંગ
\item
  \textbf{બેન્ડવિડ્થ કાર્યક્ષમતા}: બહુવિધ ચેનલ્સ સિંગલ લિંક શેર કરે છે
\end{itemize}

\end{solutionbox}
\begin{mnemonicbox}
``ટાઇમ બહુવિધ સિગ્નલ્સને વિભાજિત કરે છે''

\end{mnemonicbox}
\subsection*{પ્રશ્ન 3(a OR) [3
ગુણ]}\label{uxaaauxab0uxab6uxaa8-3a-or-3-uxa97uxaa3}

\textbf{હાર્ડ રીઅલ ટાઇમ અને સોફ્ટ રીઅલ ટાઇમ ઓપરેટિંગ સિસ્ટમ વચ્ચે ફરક લખો.}

\begin{solutionbox}

\textbf{તુલના ટેબલ:}

{\def\LTcaptype{none} % do not increment counter
\begin{longtable}[]{@{}lll@{}}
\toprule\noalign{}
પાસું & હાર્ડ રીઅલ ટાઇમ & સોફ્ટ રીઅલ ટાઇમ \\
\midrule\noalign{}
\endhead
\bottomrule\noalign{}
\endlastfoot
\textbf{ડેડલાઇન} & સંપૂર્ણપણે પૂરી કરવી જ જોઈએ & પ્રાધાન્ય પરંતુ લવચીક \\
\textbf{પરિણામો} & ચૂકી જવાથી સિસ્ટમ ફેઇલ & પ્રદર્શનમાં ઘટાડો \\
\textbf{ઉદાહરણો} & એરક્રાફ્ટ કન્ટ્રોલ, પેસમેકર & વિડિયો સ્ટ્રીમિંગ, ગેમિંગ \\
\textbf{રિસ્પોન્સ ટાઇમ} & ગેરેન્ટીડ મહત્તમ & બેસ્ટ એફર્ટ આધાર \\
\textbf{કિંમત} & ઉચ્ચ ડેવલપમેન્ટ કોસ્ટ & મધ્યમ કિંમત \\
\textbf{વિશ્વસનીયતા} & ક્રિટિકલ સિસ્ટમ વિશ્વસનીયતા & યુઝર એક્સપિરિયન્સ ફોકસ્ડ \\
\end{longtable}
}

\textbf{મુખ્ય લક્ષણો:}

\begin{itemize}
\tightlist
\item
  \textbf{હાર્ડ RT}: ડેડલાઇન મિસ માટે શૂન્ય ટોલરન્સ
\item
  \textbf{સોફ્ટ RT}: અવારનવાર વિલંબ સ્વીકાર્ય
\item
  \textbf{એપ્લિકેશન્સ}: સેફ્ટી-ક્રિટિકલ વિ યુઝર-ઇન્ટરેક્ટિવ સિસ્ટમ્સ
\end{itemize}

\end{solutionbox}
\begin{mnemonicbox}
``હાર્ડને ચોકસાઈ જોઈએ, સોફ્ટ લવચીકતાની મંજૂરી આપે છે''

\end{mnemonicbox}
\subsection*{પ્રશ્ન 3(b OR) [4
ગુણ]}\label{uxaaauxab0uxab6uxaa8-3b-or-4-uxa97uxaa3}

\textbf{ટ્રાન્સમિશન મોડ્સ સમજાવો.}

\begin{solutionbox}

ટ્રાન્સમિશન મોડ્સ કમ્યુનિકેટિંગ ડિવાઇસીસ વચ્ચે ડેટા ફ્લોની દિશા વ્યાખ્યાયિત કરે છે.

\textbf{મોડ પ્રકારો:}

\begin{center}
\textbf{Mermaid Diagram (Code)}
\begin{verbatim}
{Shaded}
{Highlighting}[]
graph TD
    subgraph "ટ્રાન્સમિશન મોડ્સ"
        S[સિમ્પ્લેક્સ]
        HD[હાફ ડુપ્લેક્સ]  
        FD[ફુલ ડુપ્લેક્સ]
    end
    
    S {-{-}{} One[માત્ર એક દિશા]}
    HD {-{-}{} Alt[વૈકલ્પિક દિશાઓ]}
    FD {-{-}{} Both[બંને દિશાઓ એકસાથે]}
{Highlighting}
{Shaded}
\end{verbatim}
\end{center}

\textbf{વિગતવાર તુલના:}

{\def\LTcaptype{none} % do not increment counter
\begin{longtable}[]{@{}
  >{\raggedright\arraybackslash}p{(\linewidth - 6\tabcolsep) * \real{0.1316}}
  >{\raggedright\arraybackslash}p{(\linewidth - 6\tabcolsep) * \real{0.2632}}
  >{\raggedright\arraybackslash}p{(\linewidth - 6\tabcolsep) * \real{0.2632}}
  >{\raggedright\arraybackslash}p{(\linewidth - 6\tabcolsep) * \real{0.3421}}@{}}
\toprule\noalign{}
\begin{minipage}[b]{\linewidth}\raggedright
મોડ
\end{minipage} & \begin{minipage}[b]{\linewidth}\raggedright
ડેટા ફ્લો
\end{minipage} & \begin{minipage}[b]{\linewidth}\raggedright
ઉદાહરણો
\end{minipage} & \begin{minipage}[b]{\linewidth}\raggedright
એપ્લિકેશન્સ
\end{minipage} \\
\midrule\noalign{}
\endhead
\bottomrule\noalign{}
\endlastfoot
\textbf{સિમ્પ્લેક્સ} & માત્ર એક દિશા & રેડિયો, TV બ્રોડકાસ્ટ & બ્રોડકાસ્ટિંગ
સિસ્ટમ્સ \\
\textbf{હાફ ડુપ્લેક્સ} & બંને દિશા, એકસાથે નહીં & વોકી-ટોકી, CB રેડિયો & બે-માર્ગી
રેડિયો \\
\textbf{ફુલ ડુપ્લેક્સ} & બંને દિશાઓ એકસાથે & ટેલિફોન, ઇથરનેટ & આધુનિક કમ્યુનિકેશન \\
\end{longtable}
}

\begin{itemize}
\tightlist
\item
  \textbf{બેન્ડવિડ્થ કાર્યક્ષમતા}: ફુલ ડુપ્લેક્સ ચેનલ ઉપયોગને મહત્તમ બનાવે છે
\item
  \textbf{કિંમત ફેક્ટર}: સિમ્પ્લેક્સ સૌથી સસ્તું, ફુલ ડુપ્લેક્સ સૌથી મોંઘું
\item
  \textbf{ઉપયોગ કેસીસ}: એપ્લિકેશન આવશ્યકતાઓના આધારે પસંદ કરો
\end{itemize}

\end{solutionbox}
\begin{mnemonicbox}
``સિમ્પ્લેક્સ સિંગલ, હાફ સ્વિચ કરે છે, ફુલ બંને ફ્લો કરે છે''

\end{mnemonicbox}
\subsection*{પ્રશ્ન 3(c OR) [7
ગુણ]}\label{uxaaauxab0uxab6uxaa8-3c-or-7-uxa97uxaa3}

\textbf{એનાલોગ મોડ્યુલેશનના પ્રકારોની યાદી બનાવો. એમ્પ્લીટ્યુડ મોડ્યુલેશન આકૃતિ સાથે
સમજાવો.}

\begin{solutionbox}

\textbf{એનાલોગ મોડ્યુલેશનના પ્રકારો:}

\begin{enumerate}
\tightlist
\item
  \textbf{એમ્પ્લીટ્યુડ મોડ્યુલેશન (AM)}
\item
  \textbf{ફ્રીક્વન્સી મોડ્યુલેશન (FM)}
\item
  \textbf{ફેઝ મોડ્યુલેશન (PM)}
\end{enumerate}

\textbf{એમ્પ્લીટ્યુડ મોડ્યુલેશન પ્રક્રિયા:}

\begin{verbatim}
graph TB
    subgraph "AM મોડ્યુલેશન"
        MS[મેસેજ સિગ્નલ] {-{-} M[મોડ્યુલેટર]}
        CS[કેરિયર સિગ્નલ] {-{-} M}
        M {-{-} AMS[AM સિગ્નલ]}
    end
    
    subgraph "વેવફોર્મ્સ"
        MW[મેસેજ વેવ {- લો ફ્રીક્વન્સી]}
        CW[કેરિયર વેવ {- હાઇ ફ્રીક્વન્સી]}
        AMW[AM વેવ {- મોડ્યુલેટેડ આઉટપુટ]}
    end
\end{verbatim}

\textbf{AM લક્ષણો:}

{\def\LTcaptype{none} % do not increment counter
\begin{longtable}[]{@{}lll@{}}
\toprule\noalign{}
પેરામીટર & વર્ણન & ટિપિકલ વેલ્યુઝ \\
\midrule\noalign{}
\endhead
\bottomrule\noalign{}
\endlastfoot
\textbf{કેરિયર ફ્રીક્વન્સી} & હાઇ ફ્રીક્વન્સી બેઝ સિગ્નલ & 550-1600 kHz (AM
રેડિયો) \\
\textbf{મેસેજ ફ્રીક્વન્સી} & ઇન્ફોર્મેશન સિગ્નલ & 20 Hz - 20 kHz (ઓડિયો) \\
\textbf{મોડ્યુલેશન ઇન્ડેક્સ} & મોડ્યુલેશનની ગહરાઈ & 0 થી 1 (0-100\%) \\
\textbf{બેન્ડવિડ્થ} & વપરાયેલ ફ્રીક્વન્સી સ્પેક્ટ્રમ & 2 \times મેસેજ ફ્રીક્વન્સી \\
\end{longtable}
}

\textbf{ગાણિતિક અભિવ્યક્તિ:}

\begin{itemize}
\tightlist
\item
  \textbf{AM સિગ્નલ}: s(t) = Ac[1 + m·cos(ωmt)]cos(ωct)
\item
  \textbf{જ્યાં}: Ac = કેરિયર એમ્પ્લીટ્યુડ, m = મોડ્યુલેશન ઇન્ડેક્સ
\end{itemize}

\textbf{એપ્લિકેશન્સ:}

\begin{itemize}
\item
  \textbf{બ્રોડકાસ્ટિંગ}: AM રેડિયો સ્ટેશન્સ
\item
  \textbf{એવિએશન}: એર ટ્રાફિક કન્ટ્રોલ કમ્યુનિકેશન
\item
  \textbf{સિટિઝન્સ બેન્ડ}: CB રેડિયો સિસ્ટમ્સ
\item
  \textbf{ફાયદાઓ}: સિમ્પલ ઇમ્પ્લીમેન્ટેશન, લો કોસ્ટ રિસીવર્સ
\item
  \textbf{નુકસાનો}: નોઇઝ માટે સંવેદનશીલ, પાવર ઇન્ફિશિયન્ટ
\end{itemize}

\end{solutionbox}
\begin{mnemonicbox}
``એમ્પ્લીટ્યુડ મેસેજ સાથે બદલાય છે''

\end{mnemonicbox}
\subsection*{પ્રશ્ન 4(a) [3
ગુણ]}\label{q4a}

\textbf{FSK અને PSK ની આકૃતિ દોરો.}

\begin{solutionbox}

\textbf{ફ્રીક્વન્સી શિફ્ટ કીઇંગ (FSK):}

\begin{verbatim}
Binary Data:  1    0    1    1    0
             
FSK Signal:   ╭╲╱╲╱╲╱╲╱╮  ╭╱╲╱╲╱╲╱╮  ╭╲╱╲╱╲╱╲╱╮
             ╱           ╲╱         ╲╱          ╲
            ╱             ╲         ╱            ╲
           ╱               ╲\_\_\_\_\_\_\_╱              ╲
          
          f1 (ઉચ્ચ ફ્રીક્વ)   f2 (નીચી ફ્રીક્વ)   f1 (ઉચ્ચ ફ્રીક્વ)
\end{verbatim}

\textbf{ફેઝ શિફ્ટ કીઇંગ (PSK):}

\begin{verbatim}
Binary Data:  1      0      1      1      0
             
PSK Signal:   ╭─╲ ╱─╮   ╭╲ ╱╮   ╭─╲ ╱─╮   ╭─╲ ╱─╮   ╭╲ ╱╮
             ╱   ╲╱   ╲ ╱  ╲╱  ╲ ╱   ╲╱   ╲ ╱   ╲╱   ╲ ╱  ╲╱  ╲
            ╱         ╲╱        ╲╱         ╲╱         ╲╱        ╲
           
           0^ ફેઝ       180^ ફેઝ    0^ ફેઝ      0^ ફેઝ     180^ ફેઝ
\end{verbatim}

\textbf{મુખ્ય ફરકો:}

\begin{itemize}
\tightlist
\item
  \textbf{FSK}: 1 અને 0 માટે અલગ ફ્રીક્વન્સીઝ
\item
  \textbf{PSK}: 1 અને 0 માટે અલગ ફેઝીસ
\end{itemize}

\end{solutionbox}
\begin{mnemonicbox}
``FSK ફ્રીક્વન્સી બદલે છે, PSK ફેઝ બદલે છે''

\end{mnemonicbox}
\subsection*{પ્રશ્ન 4(b) [4
ગુણ]}\label{q4b}

\textbf{જો મેશ ટોપોલોજીમાં 45 લિંક્સ છે, તો વધુમાં વધુ કેટલા નોડ્સ હોવા જોઈએ તે
શોધો.}

\begin{solutionbox}

\textbf{મેશ ટોપોલોજી માટે ફોર્મ્યુલા:} લિંક્સની સંખ્યા = n(n-1)/2

જ્યાં n = નોડ્સની સંખ્યા

\textbf{આપેલ}: લિંક્સની સંખ્યા = 45

\textbf{ગણતરી:} 45 = n(n-1)/2 90 = n(n-1) n^{2} - n - 90 = 0

\textbf{ક્વાડ્રેટિક સમીકરણ ઉકેલવું:} ક્વાડ્રેટિક ફોર્મ્યુલા વાપરીને: n = [-b \pm \sqrt(b^{2}
- 4ac)] / 2a

જ્યાં

a=1,

b=-1,

c=-90


n = [1 \pm \sqrt(1 + 360)] / 2

n = [1 \pm \sqrt361] / 2\\

n = [1 \pm 19] / 2

\textbf{ઉકેલો:} n = (1 + 19)/2 = 10 અથવા n = (1 - 19)/2 = -9

\end{solutionbox}
\begin{solutionbox}
વધુમાં વધુ નોડ્સની સંખ્યા = 10

\textbf{ચકાસણી}: 10(10-1)/2 = 10\times9/2 = 45 ✓

\end{solutionbox}
\begin{mnemonicbox}
``n નોડ્સને n(n-1)/2 લિંક્સની જરૂર''

\end{mnemonicbox}
\subsection*{પ્રશ્ન 4(c) [7
ગુણ]}\label{q4c}

\textbf{OSI મોડેલ આકૃતિ સાથે સમજાવો.}

\begin{solutionbox}

OSI (ઓપન સિસ્ટમ્સ ઇન્ટરકનેક્શન) મોડેલ નેટવર્ક કમ્યુનિકેશન માટે સાત સ્તરો વ્યાખ્યાયિત કરે
છે.

\textbf{OSI લેયર સ્ટેક:}

\begin{verbatim}
graph TB
    subgraph "OSI મોડેલ"
        L7[લેયર 7: એપ્લિકેશન]
        L6[લેયર 6: પ્રેઝન્ટેશન] 
        L5[લેયર 5: સેશન]
        L4[લેયર 4: ટ્રાન્સપોર્ટ]
        L3[લેયર 3: નેટવર્ક]
        L2[લેયર 2: ડેટા લિંક]
        L1[લેયર 1: ફિઝિકલ]
    end
    
    L7 {-{-} L6}
    L6 {-{-} L5}
    L5 {-{-} L4}
    L4 {-{-} L3}
    L3 {-{-} L2}
    L2 {-{-} L1}
\end{verbatim}

\textbf{લેયર કાર્યો:}

{\def\LTcaptype{none} % do not increment counter
\begin{longtable}[]{@{}lllll@{}}
\toprule\noalign{}
લેયર & નામ & કાર્ય & પ્રોટોકોલ્સ & ડિવાઇસીસ \\
\midrule\noalign{}
\endhead
\bottomrule\noalign{}
\endlastfoot
\textbf{7} & એપ્લિકેશન & યુઝર ઇન્ટરફેસ & HTTP, FTP, SMTP & ગેટવેઝ \\
\textbf{6} & પ્રેઝન્ટેશન & ડેટા ફોર્મેટિંગ & SSL, JPEG, MPEG & ગેટવેઝ \\
\textbf{5} & સેશન & કનેક્શન મેનેજમેન્ટ & NetBIOS, RPC & ગેટવેઝ \\
\textbf{4} & ટ્રાન્સપોર્ટ & એન્ડ-ટુ-એન્ડ ડેલિવરી & TCP, UDP & ગેટવેઝ \\
\textbf{3} & નેટવર્ક & રાઉટિંગ & IP, ICMP & રાઉટર્સ \\
\textbf{2} & ડેટા લિંક & ફ્રેમ ટ્રાન્સમિશન & Ethernet, PPP & સ્વિચીસ \\
\textbf{1} & ફિઝિકલ & બિટ ટ્રાન્સમિશન & Ethernet cables & હબ્સ, રિપીટર્સ \\
\end{longtable}
}

\textbf{ડેટા ફ્લો પ્રોસેસ:}

\begin{itemize}
\item
  \textbf{એન્કેપ્સ્યુલેશન}: ડેટા લેયર્સ નીચે જાય છે, હેડર્સ ઉમેરાય છે
\item
  \textbf{ટ્રાન્સમિશન}: ફિઝિકલ લેયર માધ્યમ પર બિટ્સ મોકલે છે
\item
  \textbf{ડીકેપ્સ્યુલેશન}: રિસીવિંગ એન્ડ લેયર્સ ઉપર જાય છે, હેડર્સ દૂર કરાય છે
\item
  \textbf{સ્ટાન્ડર્ડાઇઝેશન}: વેન્ડર્સ વચ્ચે ઇન્ટરઓપરેબિલિટી સક્ષમ કરે છે
\item
  \textbf{મોડ્યુલેરિટી}: દરેક લેયરની વિશિષ્ટ જવાબદારીઓ
\item
  \textbf{ટ્રબલશૂટિંગ}: ચોક્કસ લેયર્સમાં સમસ્યાઓને અલગ કરે છે
\end{itemize}

\end{solutionbox}
\begin{mnemonicbox}
``બધા લોકો સેશન ટ્રાન્સપોર્ટ નેટવર્ક ડેટા પ્રોસેસિંગ જોઈએ''

\end{mnemonicbox}
\subsection*{પ્રશ્ન 4(a OR) [3
ગુણ]}\label{uxaaauxab0uxab6uxaa8-4a-or-3-uxa97uxaa3}

\textbf{IPv4 ક્લાસફુલ એડ્રેસિંગ સ્કીમ ઉદાહરણ સાથે સમજાવો.}

\begin{solutionbox}

IPv4 ક્લાસફુલ એડ્રેસિંગ નેટવર્ક સાઇઝના આધારે IP સ્પેસને પૂર્વવ્યાખ્યાયિત ક્લાસીસમાં
વિભાજિત કરે છે.

\textbf{ક્લાસ સ્ટ્રક્ચર:}

{\def\LTcaptype{none} % do not increment counter
\begin{longtable}[]{@{}lllll@{}}
\toprule\noalign{}
ક્લાસ & રેન્જ & ડિફોલ્ટ માસ્ક & નેટવર્ક્સ & નેટવર્ક દીઠ હોસ્ટ્સ \\
\midrule\noalign{}
\endhead
\bottomrule\noalign{}
\endlastfoot
\textbf{A} & 1-126 & /8 (255.0.0.0) & 126 & 16,777,214 \\
\textbf{B} & 128-191 & /16 (255.255.0.0) & 16,384 & 65,534 \\
\textbf{C} & 192-223 & /24 (255.255.255.0) & 2,097,152 & 254 \\
\end{longtable}
}

\textbf{ઉદાહરણો:}

\begin{itemize}
\tightlist
\item
  \textbf{ક્લાસ A}: 10.0.0.1 (ISPs જેવા મોટા નેટવર્ક્સ)
\item
  \textbf{ક્લાસ B}: 172.16.0.1 (યુનિવર્સિટીઝ જેવા મધ્યમ નેટવર્ક્સ)
\item
  \textbf{ક્લાસ C}: 192.168.1.1 (ઓફિસીસ જેવા નાના નેટવર્ક્સ)
\end{itemize}

\textbf{એડ્રેસ ફોર્મેટ:}

\begin{itemize}
\tightlist
\item
  \textbf{ક્લાસ A}: N.H.H.H (N=નેટવર્ક, H=હોસ્ટ)
\item
  \textbf{ક્લાસ B}: N.N.H.H
\item
  \textbf{ક્લાસ C}: N.N.N.H
\end{itemize}

\end{solutionbox}
\begin{mnemonicbox}
``A ઓલ (મોટા) માટે, B બિઝનેસ (મધ્યમ) માટે, C કંપની (નાના)
માટે''

\end{mnemonicbox}
\subsection*{પ્રશ્ન 4(b OR) [4
ગુણ]}\label{uxaaauxab0uxab6uxaa8-4b-or-4-uxa97uxaa3}

\textbf{જો મેશ ટોપોલોજીમાં 11 નોડ્સ છે તો ઓછામાં ઓછી કેટલી લિંક્સ હોવી જોઈએ તે
શોધો.}

\begin{solutionbox}

\textbf{મેશ ટોપોલોજી માટે ફોર્મ્યુલા:} લિંક્સની સંખ્યા = n(n-1)/2

જ્યાં n = નોડ્સની સંખ્યા

\textbf{આપેલ}: નોડ્સની સંખ્યા = 11

\textbf{ગણતરી:} લિંક્સની સંખ્યા = 11(11-1)/2 = 11 \times 10/2 = 110/2 = 55

\end{solutionbox}
\begin{solutionbox}
ઓછામાં ઓછી જરૂરી લિંક્સની સંખ્યા = 55

\textbf{સમજૂતી:}

\begin{itemize}
\tightlist
\item
  મેશ ટોપોલોજીમાં, દરેક નોડ બીજા દરેક નોડ સાથે જોડાય છે
\item
  દરેક નોડને (n-1) કનેક્શન્સ છે
\item
  કુલ કનેક્શન્સ = n(n-1), પરંતુ દરેક લિંક બે વાર ગણાય છે
\item
  તેથી, વાસ્તવિક લિંક્સ = n(n-1)/2
\end{itemize}

\end{solutionbox}
\begin{mnemonicbox}
``દરેક નોડ બીજા દરેક સાથે જોડાય છે''

\end{mnemonicbox}
\subsection*{પ્રશ્ન 4(c OR) [7
ગુણ]}\label{uxaaauxab0uxab6uxaa8-4c-or-7-uxa97uxaa3}

\textbf{ડોમેન નેમ સિસ્ટમ (DNS) આકૃતિ સાથે સમજાવો.}

\begin{solutionbox}

DNS માનવ-વાંચી શકાય તેવા ડોમેન નેમ્સને નેટવર્ક રાઉટિંગ માટે IP એડ્રેસીસમાં ટ્રાન્સલેટ કરે
છે.

\textbf{DNS હાયરાર્કી:}

\begin{verbatim}
graph TB
    subgraph "DNS હાયરાર્કી"
        Root["રૂટ સર્વર્સ (.)"]
        TLD["ટોપ લેવલ ડોમેન (.com, .org, .edu)"]
        SLD["સેકન્ડ લેવલ ડોમેન (google, example)"]
        Sub["સબડોમેન (www, mail, ftp)"]
    end
    
    Root {-{-} TLD}
    TLD {-{-} SLD}
    SLD {-{-} Sub}
    
    subgraph "DNS રિઝોલ્યુશન પ્રોસેસ"
        Client[ક્લાયન્ટ] {-{-} Local[લોકલ DNS સર્વર]}
        Local {-{-} RootNS[રૂટ નેમ સર્વર]}
        RootNS {-{-} TLDNS[TLD નેમ સર્વર]}
        TLDNS {-{-} AuthNS[ઓથોરિટેટિવ નેમ સર્વર]}
        AuthNS {-{-} Local}
        Local {-{-} Client}
    end
\end{verbatim}

\textbf{DNS ઘટકો:}

{\def\LTcaptype{none} % do not increment counter
\begin{longtable}[]{@{}lll@{}}
\toprule\noalign{}
ઘટક & કાર્ય & ઉદાહરણો \\
\midrule\noalign{}
\endhead
\bottomrule\noalign{}
\endlastfoot
\textbf{રૂટ સર્વર્સ} & ટોપ-લેવલ ઓથોરિટી & વિશ્વભરમાં 13 રૂટ સર્વર્સ \\
\textbf{TLD સર્વર્સ} & ટોપ-લેવલ ડોમેન્સનું સંચાલન & .com, .org, .edu, .gov \\
\textbf{ઓથોરિટેટિવ સર્વર્સ} & વાસ્તવિક DNS રેકોર્ડ્સ સ્ટોર કરે છે & કંપની DNS
સર્વર્સ \\
\textbf{લોકલ DNS સર્વર્સ} & ક્વેરીઝ કેશ અને ફોરવર્ડ કરે છે & ISP DNS સર્વર્સ \\
\end{longtable}
}

\textbf{DNS રેકોર્ડ પ્રકારો:}

\begin{itemize}
\tightlist
\item
  \textbf{A રેકોર્ડ}: ડોમેનને IPv4 એડ્રેસ સાથે મેપ કરે છે
\item
  \textbf{AAAA રેકોર્ડ}: ડોમેનને IPv6 એડ્રેસ સાથે મેપ કરે છે\\
\item
  \textbf{CNAME}: ડોમેન એલિયાસીસ બનાવે છે
\item
  \textbf{MX રેકોર્ડ}: મેઇલ સર્વર્સ સ્પેસિફાઇ કરે છે
\item
  \textbf{NS રેકોર્ડ}: નેમ સર્વર્સ આઇડેન્ટિફાઇ કરે છે
\end{itemize}

\textbf{રિઝોલ્યુશન પ્રોસેસ:}

\begin{enumerate}
\tightlist
\item
  \textbf{ક્લાયન્ટ ક્વેરી}: યુઝર ડોમેન નેમ એન્ટર કરે છે
\item
  \textbf{લોકલ કેશ ચેક}: લોકલ DNS કેશ ચેક કરે છે
\item
  \textbf{રિકર્સિવ ક્વેરી}: લોકલ સર્વર હાયરાર્કી ક્વેરી કરે છે
\item
  \textbf{રિસ્પોન્સ રિટર્ન}: IP એડ્રેસ ક્લાયન્ટને પરત કરવામાં આવે છે
\end{enumerate}

\begin{itemize}
\tightlist
\item
  \textbf{કેશિંગ}: પ્રદર્શન સુધારે છે અને નેટવર્ક ટ્રાફિક ઘટાડે છે
\item
  \textbf{રીડન્ડન્સી}: બહુવિધ સર્વર્સ ઉપલબ્ધતા સુનિશ્ચિત કરે છે
\item
  \textbf{લોડ ડિસ્ટ્રિબ્યુશન}: સર્વર્સમાં ક્વેરી લોડ સંતુલિત કરે છે
\end{itemize}

\end{solutionbox}
\begin{mnemonicbox}
``ડોમેન્સને સિસ્ટેમેટિક નેમ-ટુ-એડ્રેસ ટ્રાન્સલેશનની જરૂર છે''

\end{mnemonicbox}
\subsection*{પ્રશ્ન 5(a) [3
ગુણ]}\label{q5a}

\textbf{IPv6 ની જરૂરિયાત સમજાવો.}

\begin{solutionbox}

IPv6 ને IPv4 ની મર્યાદાઓને સંબોધવા અને ભવિષ્યની ઇન્ટરનેટ વૃદ્ધિને સપોર્ટ કરવા માટે
વિકસાવવામાં આવ્યું.

\textbf{મુખ્ય આવશ્યકતાઓ:}

{\def\LTcaptype{none} % do not increment counter
\begin{longtable}[]{@{}lll@{}}
\toprule\noalign{}
સમસ્યા & IPv4 મર્યાદા & IPv6 ઉકેલ \\
\midrule\noalign{}
\endhead
\bottomrule\noalign{}
\endlastfoot
\textbf{એડ્રેસ સ્પેસ} & 4.3 બિલિયન એડ્રેસીસ & 340 અંડેસિલિયન એડ્રેસીસ \\
\textbf{NAT જટિલતા} & પ્રાઇવેટ-પબ્લિક ટ્રાન્સલેશન & એન્ડ-ટુ-એન્ડ કનેક્ટિવિટી \\
\textbf{સિક્યોરિટી} & વૈકલ્પિક IPSec & બિલ્ટ-ઇન IPSec સપોર્ટ \\
\textbf{મોબાઇલ સપોર્ટ} & મર્યાદિત મોબિલિટી & નેટિવ મોબિલિટી સપોર્ટ \\
\end{longtable}
}

\textbf{મહત્વપૂર્ણ જરૂરિયાતો:}

\begin{itemize}
\item
  \textbf{IoT વિસ્ફોટ}: અબજો કનેક્ટેડ ડિવાઇસીસને અનન્ય એડ્રેસીસની જરૂર
\item
  \textbf{મોબાઇલ વૃદ્ધિ}: સ્માર્ટફોન્સ અને ટેબ્લેટ્સને ઇન્ટરનેટ એક્સેસ જોઈએ
\item
  \textbf{ગ્લોબલ કનેક્ટિવિટી}: ઉભરતા બજારો ઇન્ટરનેટમાં જોડાય છે
\item
  \textbf{એડ્રેસ ફોર્મેટ}: IPv4 માં 32-બિટ વિ 128-બિટ
\item
  \textbf{સિમ્પ્લિફાઇડ હેડર}: વધુ કાર્યક્ષમ પેકેટ પ્રોસેસિંગ
\item
  \textbf{નો ફ્રેગમેન્ટેશન}: રાઉટર્સ પેકેટ્સને ફ્રેગમેન્ટ કરતા નથી
\end{itemize}

\end{solutionbox}
\begin{mnemonicbox}
``IPv6 ઇન્ટરનેટ વૃદ્ધિ માટે અનંત એડ્રેસીસ પૂરું પાડે છે''

\end{mnemonicbox}
\subsection*{પ્રશ્ન 5(b) [4
ગુણ]}\label{q5b}

\textbf{એસિમેટ્રિક કી એન્ક્રિપ્શનનું ઉપયોગ કરીને કોન્ફિડેન્શિયાલિટી સમજાવો.}

\begin{solutionbox}

એસિમેટ્રિક એન્ક્રિપ્શન ડેટા કોન્ફિડેન્શિયાલિટી સુનિશ્ચિત કરવા માટે કી પેર્સ
(પબ્લિક-પ્રાઇવેટ) વાપરે છે.

\textbf{એન્ક્રિપ્શન પ્રોસેસ:}

\begin{verbatim}
sequenceDiagram
    participant S as મોકલનાર
    participant R as રિસીવર
    
    Note over R: કી પેર જનરેટ કરો
    R{-S: પબ્લિક કી}
    Note over S: પબ્લિક કી સાથે એન્ક્રિપ્ટ કરો
    S{-R: એન્ક્રિપ્ટેડ મેસેજ}
    Note over R: પ્રાઇવેટ કી સાથે ડિક્રિપ્ટ કરો
    R{-R: ઓરિજિનલ મેસેજ}
\end{verbatim}

\textbf{મુખ્ય લક્ષણો:}

{\def\LTcaptype{none} % do not increment counter
\begin{longtable}[]{@{}lll@{}}
\toprule\noalign{}
પાસું & વર્ણન & સિક્યોરિટી બેનિફિટ \\
\midrule\noalign{}
\endhead
\bottomrule\noalign{}
\endlastfoot
\textbf{પબ્લિક કી} & મુક્તપણે વિતરિત & કોઈપણ એન્ક્રિપ્ટ કરી શકે છે \\
\textbf{પ્રાઇવેટ કી} & ગુપ્ત રાખવામાં આવે છે & માત્ર માલિક ડિક્રિપ્ટ કરી શકે છે \\
\textbf{કી પેર} & ગાણિતિક રીતે સંબંધિત & સુરક્ષિત કમ્યુનિકેશન \\
\textbf{અલ્ગોરિધમ} & RSA, ECC, DSA & મજબૂત એન્ક્રિપ્શન \\
\end{longtable}
}

\textbf{કોન્ફિડેન્શિયાલિટી પ્રોસેસ:}

\begin{itemize}
\item
  \textbf{પગલું 1}: રિસીવર પબ્લિક-પ્રાઇવેટ કી પેર જનરેટ કરે છે
\item
  \textbf{પગલું 2}: પબ્લિક કી મોકલનાર સાથે શેર કરવામાં આવે છે
\item
  \textbf{પગલું 3}: મોકલનાર પબ્લિક કી સાથે મેસેજ એન્ક્રિપ્ટ કરે છે
\item
  \textbf{પગલું 4}: માત્ર રિસીવરની પ્રાઇવેટ કી ડિક્રિપ્ટ કરી શકે છે
\item
  \textbf{કોઈ કી એક્સચેન્જ નહીં}: કી ડિસ્ટ્રિબ્યુશન સમસ્યાને દૂર કરે છે
\item
  \textbf{નોન-રિપ્યુડિયેશન}: મોકલનાર મેસેજ મોકલવાનો ઇનકાર કરી શકે નહીં
\item
  \textbf{ડિજિટલ સિગ્નેચર્સ}: ઓથેન્ટિકેશન અને ઇન્ટેગ્રિટી
\end{itemize}

\end{solutionbox}
\begin{mnemonicbox}
``પબ્લિક લોક કરે છે, પ્રાઇવેટ અનલોક કરે છે''

\end{mnemonicbox}
\subsection*{પ્રશ્ન 5(c) [7
ગુણ]}\label{q5c}

\textbf{મેન ઇન મિડલ અટેક ઉદાહરણ સાથે સમજાવો.}

\begin{solutionbox}

મેન-ઇન-ધ-મિડલ અટેક બે પક્ષો વચ્ચેનો સંદેશાવ્યવહાર તેમની જાણ વિના અટકાવે છે.

\textbf{અટેક પ્રોસેસ:}

\begin{verbatim}
sequenceDiagram
    participant A as આલિસ
    participant M as મેલોરી (અટેકર)
    participant B as બોબ
    
    A{-M: બોબ માટે મેસેજ}
    Note over M: અટકાવે છે અને વાંચે છે
    M{-B: બદલેલ/ઓરિજિનલ મેસેજ}
    B{-M: આલિસ માટે જવાબ}
    Note over M: અટકાવે છે અને વાંચે છે
    M{-A: બદલેલ/ઓરિજિનલ જવાબ}
\end{verbatim}

\textbf{અટેક તબક્કાઓ:}

{\def\LTcaptype{none} % do not increment counter
\begin{longtable}[]{@{}lll@{}}
\toprule\noalign{}
તબક્કો & અટેકરની ક્રિયા & પીડિતની અસર \\
\midrule\noalign{}
\endhead
\bottomrule\noalign{}
\endlastfoot
\textbf{ઇન્ટરસેપ્શન} & પક્ષો વચ્ચે સ્થિતિ & પીડિતોને અજાણ \\
\textbf{ડિક્રિપ્શન} & એન્ક્રિપ્શન તોડે/બાયપાસ કરે & ડેટાની એક્સેસ \\
\textbf{મોડિફિકેશન} & મેસેજીસ બદલે & ખોટી માહિતી \\
\textbf{રી-એન્ક્રિપ્શન} & ટેમ્પરિંગ છુપાવે & ભ્રમ જાળવે છે \\
\end{longtable}
}

\textbf{વાસ્તવિક જગતનું ઉદાહરણ:}

\begin{itemize}
\tightlist
\item
  \textbf{સિનેરિયો}: ઓનલાઇન બેંકિંગ સેશન
\item
  \textbf{અટેક}: પબ્લિક WiFi પર અટેકર ટ્રાફિક અટકાવે છે
\item
  \textbf{પદ્ધતિ}: નકલી એક્સેસ પોઇન્ટ ``Free\_WiFi'' બનાવે છે
\item
  \textbf{પરિણામ}: બેંકિંગ ક્રેડેન્શિયલ્સ ચોરે છે અને પૈસા ટ્રાન્સફર કરે છે
\end{itemize}

\textbf{સામાન્ય ટાર્ગેટ્સ:}

\begin{itemize}
\tightlist
\item
  \textbf{પબ્લિક WiFi}: કોફી શોપ્સ, એરપોર્ટ્સ, હોટેલ્સ
\item
  \textbf{ઇમેઇલ કમ્યુનિકેશન}: કોર્પોરેટ કમ્યુનિકેશન્સ
\item
  \textbf{ઓનલાઇન શોપિંગ}: ક્રેડિટ કાર્ડ માહિતી ચોરી
\item
  \textbf{સોશિયલ મીડિયા}: વ્યક્તિગત માહિતી હાર્વેસ્ટિંગ
\end{itemize}

\textbf{બચાવના પગલાં:}

\begin{itemize}
\tightlist
\item
  \textbf{SSL/TLS}: એન્ડ-ટુ-એન્ડ એન્ક્રિપ્શન પ્રોટોકોલ્સ
\item
  \textbf{VPN ઉપયોગ}: બધા ટ્રાફિક માટે સુરક્ષિત ટનલ
\item
  \textbf{સર્ટિફિકેટ વેરિફિકેશન}: વેબસાઇટની અધિકૃતતા ચેક કરો
\item
  \textbf{પબ્લિક WiFi ટાળો}: સંવેદનશીલ કાર્યો માટે સેલ્યુલર ડેટા વાપરો
\end{itemize}

\end{solutionbox}
\begin{mnemonicbox}
``મેલોરી આલિસ અને બોબ વચ્ચે મેસેજીસ અટકાવે છે''

\end{mnemonicbox}
\subsection*{પ્રશ્ન 5(a OR) [3
ગુણ]}\label{uxaaauxab0uxab6uxaa8-5a-or-3-uxa97uxaa3}

\textbf{નીચે દશાર્વેલ ડિવાઇસીસ માટે સંબંધિત OSI મોડેલના લેયર્સના નામ આપો.}
\textbf{1. Repeater 2. Router 3. Switch}

\begin{solutionbox}

\textbf{ડિવાઇસ-લેયર મેપિંગ:}

{\def\LTcaptype{none} % do not increment counter
\begin{longtable}[]{@{}llll@{}}
\toprule\noalign{}
ડિવાઇસ & OSI લેયર & લેયર નામ & કાર્ય \\
\midrule\noalign{}
\endhead
\bottomrule\noalign{}
\endlastfoot
\textbf{Repeater} & લેયર 1 & ફિઝિકલ લેયર & સિગ્નલ એમ્પ્લિફિકેશન \\
\textbf{Router} & લેયર 3 & નેટવર્ક લેયર & IP રાઉટિંગ ડિસિઝન્સ \\
\textbf{Switch} & લેયર 2 & ડેટા લિંક લેયર & ફ્રેમ સ્વિચિંગ \\
\end{longtable}
}

\textbf{વિગતવાર કાર્યો:}

\begin{itemize}
\tightlist
\item
  \textbf{Repeater}: નેટવર્ક ડિસ્ટન્સ વધારવા માટે ઇલેક્ટ્રિકલ સિગ્નલ્સ પુનર્જીવિત કરે
  છે
\item
  \textbf{Router}: IP એડ્રેસીસના આધારે ફોરવર્ડિંગ ડિસિઝન્સ લે છે
\item
  \textbf{Switch}: MAC એડ્રેસીસના આધારે ફ્રેમ્સ ફોરવર્ડ કરે છે
\end{itemize}

\end{solutionbox}
\begin{mnemonicbox}
``રિપીટર્સ ફિઝિકલ કામ કરે છે, સ્વિચીસ ડેટા લિંક કરે છે,
રાઉટર્સ નેટવર્ક રાઉટ કરે છે''

\end{mnemonicbox}
\subsection*{પ્રશ્ન 5(b OR) [4
ગુણ]}\label{uxaaauxab0uxab6uxaa8-5b-or-4-uxa97uxaa3}

\textbf{સિમેટ્રિક કી એન્ક્રિપ્શનનો ઉપયોગ કરીને કોન્ફિડેન્શિયાલિટી સમજાવો.}

\begin{solutionbox}

સિમેટ્રિક એન્ક્રિપ્શન એન્ક્રિપ્શન અને ડિક્રિપ્શન બંને માટે સિંગલ શેર્ડ કી વાપરે છે.

\textbf{એન્ક્રિપ્શન પ્રોસેસ:}

\begin{center}
\textbf{Mermaid Diagram (Code)}
\begin{verbatim}
{Shaded}
{Highlighting}[]
graph LR
    subgraph "સિમેટ્રિક એન્ક્રિપ્શન"
    direction LR
        PT[પ્લેન ટેક્સ્ટ] {-{-}{} E[એન્ક્રિપ્શન]}
        K[શેર્ડ કી] {-{-}{} E}
        E {-{-}{} CT[સાઇફર ટેક્સ્ટ]}
        CT {-{-}{} D[ડિક્રિપ્શન]}
        K {-{-}{} D}
        D {-{-}{} PT2[પ્લેન ટેક્સ્ટ]}
    end
{Highlighting}
{Shaded}
\end{verbatim}
\end{center}

\textbf{મુખ્ય લક્ષણો:}

{\def\LTcaptype{none} % do not increment counter
\begin{longtable}[]{@{}lll@{}}
\toprule\noalign{}
લક્ષણ & વર્ણન & ઉદાહરણ \\
\midrule\noalign{}
\endhead
\bottomrule\noalign{}
\endlastfoot
\textbf{સિંગલ કી} & એન્ક્રિપ્ટ/ડિક્રિપ્ટ માટે સમાન કી & AES-256 કી \\
\textbf{ઝડપી પ્રોસેસિંગ} & કાર્યક્ષમ અલ્ગોરિધમ્સ & રીઅલ-ટાઇમ કમ્યુનિકેશન \\
\textbf{કી ડિસ્ટ્રિબ્યુશન} & સુરક્ષિત કી શેરિંગ જરૂરી & પ્રી-શેર્ડ કીઝ \\
\textbf{અલ્ગોરિધમ પ્રકારો} & બ્લોક અને સ્ટ્રીમ સાઇફર્સ & AES, DES, RC4 \\
\end{longtable}
}

\textbf{કોન્ફિડેન્શિયાલિટી મેકેનિઝમ:}

\begin{itemize}
\item
  \textbf{શેર્ડ સિક્રેટ}: બંને પક્ષો પાસે સમાન કી હોવી જોઈએ
\item
  \textbf{એન્ક્રિપ્શન}: મોકલનાર શેર્ડ કી સાથે એન્ક્રિપ્ટ કરે છે
\item
  \textbf{ટ્રાન્સમિશન}: સાઇફર ટેક્સ્ટ અસુરક્ષિત ચેનલ પર મોકલવામાં આવે છે
\item
  \textbf{ડિક્રિપ્શન}: રિસીવર સમાન કી સાથે ડિક્રિપ્ટ કરે છે
\item
  \textbf{ફાયદાઓ}: ઝડપી એક્ઝીક્યુશન, લો કોમ્પ્યુટેશનલ ઓવરહેડ
\item
  \textbf{નુકસાનો}: કી ડિસ્ટ્રિબ્યુશન ચેલેન્જ, સ્કેલેબિલિટી ઇશ્યુઝ
\item
  \textbf{એપ્લિકેશન્સ}: VPN ટનલ્સ, ફાઇલ એન્ક્રિપ્શન, ડેટાબેઝ સિક્યોરિટી
\end{itemize}

\end{solutionbox}
\begin{mnemonicbox}
``સમાન કી એન્ક્રિપ્ટ અને ડિક્રિપ્ટ કરે છે''

\end{mnemonicbox}
\subsection*{પ્રશ્ન 5(c OR) [7
ગુણ]}\label{uxaaauxab0uxab6uxaa8-5c-or-7-uxa97uxaa3}

\textbf{ડિનાયલ ઓફ સર્વિસ અટેક ઉદાહરણ સાથે સમજાવો.}

\begin{solutionbox}

DoS અટેક સિસ્ટમને ઓવરવેલ્મ કરીને કાયદેસર વપરાશકર્તાઓ માટે નેટવર્ક રિસોર્સીસને અનુપલબ્ધ
બનાવે છે.

\textbf{અટેક પ્રકારો:}

\begin{verbatim}
graph TB
    subgraph "DoS અટેક પ્રકારો"
        V[વોલ્યુમ{-આધારિત]}
        P[પ્રોટોકોલ{-આધારિત]}
        A[એપ્લિકેશન{-આધારિત]}
    end
    
    V {-{-} VE[વોલ્યુમેટ્રિક ઉદાહરણો]}
    P {-{-} PE[પ્રોટોકોલ ઉદાહરણો]}
    A {-{-} AE[એપ્લિકેશન ઉદાહરણો]}
    
    VE {-{-} UDP[UDP ફ્લડ]}
    VE {-{-} ICMP[ICMP ફ્લડ]}
    
    PE {-{-} SYN[SYN ફ્લડ]}
    PE {-{-} SMURF[સ્મર્ફ અટેક]}
    
    AE {-{-} HTTP[HTTP ફ્લડ]}
    AE {-{-} SLOW[સ્લોલોરિસ]}
\end{verbatim}

\textbf{અટેક કેટેગરીઝ:}

{\def\LTcaptype{none} % do not increment counter
\begin{longtable}[]{@{}
  >{\raggedright\arraybackslash}p{(\linewidth - 6\tabcolsep) * \real{0.2222}}
  >{\raggedright\arraybackslash}p{(\linewidth - 6\tabcolsep) * \real{0.2963}}
  >{\raggedright\arraybackslash}p{(\linewidth - 6\tabcolsep) * \real{0.2963}}
  >{\raggedright\arraybackslash}p{(\linewidth - 6\tabcolsep) * \real{0.1852}}@{}}
\toprule\noalign{}
\begin{minipage}[b]{\linewidth}\raggedright
પ્રકાર
\end{minipage} & \begin{minipage}[b]{\linewidth}\raggedright
પદ્ધતિ
\end{minipage} & \begin{minipage}[b]{\linewidth}\raggedright
ટાર્ગેટ
\end{minipage} & \begin{minipage}[b]{\linewidth}\raggedright
અસર
\end{minipage} \\
\midrule\noalign{}
\endhead
\bottomrule\noalign{}
\endlastfoot
\textbf{વોલ્યુમ-આધારિત} & ટ્રાફિક સાથે ફ્લડ & બેન્ડવિડ્થ & નેટવર્ક કંજેશન \\
\textbf{પ્રોટોકોલ-આધારિત} & પ્રોટોકોલ વીકનેસનો ઉપયોગ & સર્વર રિસોર્સીસ &
સર્વિસ અનુપલબ્ધતા \\
\textbf{એપ્લિકેશન-આધારિત} & એપ્લિકેશન લેયર ટાર્ગેટ & એપ્લિકેશન સર્વર & સર્વિસ
ડિગ્રેડેશન \\
\end{longtable}
}

\textbf{વાસ્તવિક જગતનું ઉદાહરણ - ઇ-કોમર્સ પર DDoS:}

\begin{itemize}
\tightlist
\item
  \textbf{ટાર્ગેટ}: સેલ સિઝન દરમિયાન ઓનલાઇન શોપિંગ વેબસાઇટ
\item
  \textbf{પદ્ધતિ}: 10,000 ઇન્ફેક્ટેડ કમ્પ્યુટર્સનું બોટનેટ
\item
  \textbf{અટેક}: દરેક બોટ સેકન્ડ દીઠ 100 રિક્વેસ્ટ્સ મોકલે છે
\item
  \textbf{પરિણામ}: સેકન્ડ દીઠ 1 મિલિયન રિક્વેસ્ટ્સ સર્વર્સને ઓવરવેલ્મ કરે છે
\item
  \textbf{અસર}: વેબસાઇટ ક્રેશ થાય છે, ગ્રાહકો પર્ચેઝ કરી શકતા નથી, આવકની ખોટ
\end{itemize}

\textbf{સામાન્ય DoS તકનીકો:}

\begin{itemize}
\tightlist
\item
  \textbf{SYN ફ્લડ}: TCP હેન્ડશેક પ્રોસેસનો દુરુપયોગ કરે છે
\item
  \textbf{UDP ફ્લડ}: મોટી સંખ્યામાં UDP પેકેટ્સ મોકલે છે
\item
  \textbf{પિંગ ઓફ ડેથ}: ઓવરસાઇઝ્ડ પિંગ પેકેટ્સ સિસ્ટમ્સને ક્રેશ કરે છે
\item
  \textbf{સ્લોલોરિસ}: સર્વર એક્સોસ્ટ કરવા માટે કનેક્શન્સ ઓપન રાખે છે
\end{itemize}

\textbf{ડિફેન્સ સ્ટ્રેટેજીઝ:}

\begin{itemize}
\tightlist
\item
  \textbf{રેટ લિમિટિંગ}: IP એડ્રેસ દીઠ રિક્વેસ્ટ્સ પ્રતિબંધિત કરે છે
\item
  \textbf{ફાયરવોલ રૂલ્સ}: શંકાસ્પદ ટ્રાફિક પેટર્ન્સ બ્લોક કરે છે
\item
  \textbf{DDoS પ્રોટેક્શન સર્વિસીસ}: CloudFlare, AWS Shield
\item
  \textbf{લોડ બેલેન્સિંગ}: સર્વર્સમાં ટ્રાફિક વિતરિત કરે છે
\item
  \textbf{ટ્રાફિક એનાલિસિસ}: અસામાન્ય પેટર્ન્સ માટે મોનિટર કરે છે
\end{itemize}

\textbf{બિઝનેસ અસર:}

\begin{itemize}
\tightlist
\item
  \textbf{આવકની ખોટ}: ગ્રાહકો સર્વિસીસ એક્સેસ કરી શકતા નથી
\item
  \textbf{પ્રતિષ્ઠાને નુકસાન}: વપરાશકર્તાઓ વિશ્વસનીયતામાં વિશ્વાસ ગુમાવે છે
\item
  \textbf{ઓપરેશનલ કોસ્ટ}: મિટિગેશન પર રિસોર્સીસ ખર્ચાય છે
\item
  \textbf{કાનૂની મુદ્દાઓ}: SLA વાયોલેશન્સ, કમ્પ્લાયન્સ પ્રોબ્લેમ્સ
\end{itemize}

\end{solutionbox}
\begin{mnemonicbox}
``રિક્વેસ્ટ્સ સાથે ઓવરવેલ્મિંગ દ્વારા સર્વિસ ડિનાઇ કરો''

\end{mnemonicbox}

\end{document}
