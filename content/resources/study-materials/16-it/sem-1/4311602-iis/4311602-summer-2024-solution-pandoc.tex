\documentclass[10pt,a4paper]{article}

% content/resources/templates/preamble.tex
\usepackage[margin=0.6in]{geometry}
\author{Milav Dabgar}
\usepackage{amsmath,amssymb,amsthm}
\usepackage{booktabs}
\usepackage{multirow}
\usepackage{xcolor}
\usepackage{tcolorbox}
\tcbuselibrary{breakable,skins}
\usepackage[colorlinks=true,linkcolor=blue]{hyperref}
\usepackage{titlesec}
\usepackage{enumitem}
\usepackage{tikz}
\usepackage{pgfplots}
\usepackage{circuitikz}
\usepackage[version=4]{mhchem}
\usepackage{longtable}
\usepackage{array}
\usepackage{float}
\usepackage{caption}
\usepackage{listings}

\lstset{
  basicstyle=\small\ttfamily,
  breaklines=true,
  breakatwhitespace=false,
  postbreak=\mbox{\textcolor{red}{$\hookrightarrow$}\space},
  float=false,
  numbers=left,
  numberstyle=\tiny\color{gray},
  numbersep=10pt,
  xleftmargin=2em,
  keywordstyle=\color{blue},
  commentstyle=\color{green!60!black},
  stringstyle=\color{purple},
  backgroundcolor=\color{gray!5},
  showstringspaces=false,
  tabsize=2,
  captionpos=b,
  keepspaces=true,
  columns=flexible
}

\pgfplotsset{compat=1.18}
\usetikzlibrary{shapes,arrows,positioning,calc,patterns,decorations.pathmorphing,decorations.markings,arrows.meta}

% Color scheme
\definecolor{headcolor}{RGB}{0,102,204}
\definecolor{keycolor}{RGB}{220,20,60}
\definecolor{solutioncolor}{RGB}{34,139,34}
\definecolor{mnemoniccolor}{RGB}{148,0,211}
\definecolor{codecolor}{RGB}{0,0,100}

% Spacing
\setlength{\parskip}{3pt}
\setlist[itemize]{nosep}
\setlist[enumerate]{nosep}

% Title formatting
\titleformat{\section}{\Large\bfseries\color{headcolor}}{\thesection}{1em}{}
\titleformat{\subsection}{\large\bfseries\color{headcolor}}{\thesubsection}{1em}{}

% Pandoc tightlist compatibility
\providecommand{\tightlist}{%
  \setlength{\itemsep}{0pt}\setlength{\parskip}{0pt}}

% Pandoc longtable compatibility
\newcounter{none}
\def\thenone{}


% content/resources/templates/english-boxes.tex
% This file is currently empty - it exists to maintain consistency with the import structure.
% Add custom environments here if needed in the future.


\begin{document}

\begin{center}
{\Huge\bfseries\color{headcolor} Subject Name Solutions}\\[5pt]
{\LARGE 4311602 -- Summer 2024}\\[3pt]
{\large Semester 1 Study Material}\\[3pt]
{\normalsize\textit{Detailed Solutions and Explanations}}
\end{center}

\vspace{10pt}

\subsection*{Question 1(a) [3 marks]}\label{q1a}

\textbf{Define Following Term:} \textbf{1. Data} \textbf{2. Information}
\textbf{3. Knowledge}

\begin{solutionbox}


{\def\LTcaptype{none} % do not increment counter
\vspace{-5pt}
\captionof{table}{Data, Information, and Knowledge Definitions}
\vspace{-10pt}
\begin{longtable}[]{@{}
  >{\raggedright\arraybackslash}p{(\linewidth - 2\tabcolsep) * \real{0.3333}}
  >{\raggedright\arraybackslash}p{(\linewidth - 2\tabcolsep) * \real{0.6667}}@{}}
\toprule\noalign{}
\begin{minipage}[b]{\linewidth}\raggedright
Term
\end{minipage} & \begin{minipage}[b]{\linewidth}\raggedright
Definition
\end{minipage} \\
\midrule\noalign{}
\endhead
\bottomrule\noalign{}
\endlastfoot
\textbf{Data} & Raw facts and figures without meaning or context \\
\textbf{Information} & Processed data that has meaning and is useful \\
\textbf{Knowledge} & Information combined with experience and
understanding \\
\end{longtable}
}

\begin{itemize}
\tightlist
\item
  \textbf{Data}: Basic building blocks without interpretation
\item
  \textbf{Information}: Data processed to provide meaningful context
\item
  \textbf{Knowledge}: Information enhanced with human insight and wisdom
\end{itemize}

\end{solutionbox}
\begin{mnemonicbox}
``DIK - Data Is Knowledge's foundation''

\end{mnemonicbox}
\subsection*{Question 1(b) [4 marks]}\label{q1b}

\textbf{Explain Primary Memory in brief.}

\begin{solutionbox}


{\def\LTcaptype{none} % do not increment counter
\vspace{-5pt}
\captionof{table}{Primary Memory Characteristics}
\vspace{-10pt}
\begin{longtable}[]{@{}ll@{}}
\toprule\noalign{}
Aspect & Description \\
\midrule\noalign{}
\endhead
\bottomrule\noalign{}
\endlastfoot
\textbf{Definition} & Main memory that directly communicates with CPU \\
\textbf{Access Speed} & Very fast access time \\
\textbf{Volatility} & Volatile (loses data when power off) \\
\textbf{Examples} & RAM, Cache memory \\
\end{longtable}
}

\begin{itemize}
\tightlist
\item
  \textbf{RAM (Random Access Memory)}: Main working memory for current
  programs
\item
  \textbf{Cache Memory}: Ultra-fast memory between CPU and RAM
\item
  \textbf{Volatile Nature}: Data disappears when computer shuts down
\item
  \textbf{Direct CPU Access}: CPU can directly read/write data
\end{itemize}

\end{solutionbox}
\begin{mnemonicbox}
``Primary is Fast but Forgetful''

\end{mnemonicbox}
\subsection*{Question 1(c) [7 marks]}\label{q1c}

\textbf{Explain types of real time OS with example.}

\begin{solutionbox}


{\def\LTcaptype{none} % do not increment counter
\vspace{-5pt}
\captionof{table}{Real-Time Operating System Types}
\vspace{-10pt}
\begin{longtable}[]{@{}
  >{\raggedright\arraybackslash}p{(\linewidth - 6\tabcolsep) * \real{0.1429}}
  >{\raggedright\arraybackslash}p{(\linewidth - 6\tabcolsep) * \real{0.3571}}
  >{\raggedright\arraybackslash}p{(\linewidth - 6\tabcolsep) * \real{0.2381}}
  >{\raggedright\arraybackslash}p{(\linewidth - 6\tabcolsep) * \real{0.2619}}@{}}
\toprule\noalign{}
\begin{minipage}[b]{\linewidth}\raggedright
Type
\end{minipage} & \begin{minipage}[b]{\linewidth}\raggedright
Response Time
\end{minipage} & \begin{minipage}[b]{\linewidth}\raggedright
Examples
\end{minipage} & \begin{minipage}[b]{\linewidth}\raggedright
Use Cases
\end{minipage} \\
\midrule\noalign{}
\endhead
\bottomrule\noalign{}
\endlastfoot
\textbf{Hard Real-Time} & Guaranteed deadline & QNX, VxWorks & Medical
devices, Aircraft \\
\textbf{Soft Real-Time} & Best effort timing & Windows RT, Linux RT &
Multimedia, Gaming \\
\textbf{Firm Real-Time} & Occasional deadline miss & Embedded Linux &
Industrial control \\
\end{longtable}
}

\begin{center}
\textbf{Mermaid Diagram (Code)}
\begin{verbatim}
{Shaded}
{Highlighting}[]
graph TD
    A[Real{-Time OS] {-}{-}{} B[Hard Real{-}Time]}
    A {-{-}{} C[Soft Real{-}Time]}
    A {-{-}{} D[Firm Real{-}Time]}
    B {-{-}{} E[Critical Systems]}
    C {-{-}{} F[Multimedia Apps]}
    D {-{-}{} G[Industrial Control]}
{Highlighting}
{Shaded}
\end{verbatim}
\end{center}

\begin{itemize}
\tightlist
\item
  \textbf{Hard Real-Time}: Missing deadline causes system failure
\item
  \textbf{Soft Real-Time}: Delayed response reduces performance but
  system continues
\item
  \textbf{Deterministic Response}: Predictable timing behavior is
  essential
\end{itemize}

\end{solutionbox}
\begin{mnemonicbox}
``HSF - Hard, Soft, Firm timing requirements''

\end{mnemonicbox}
\subsection*{Question 1(c OR) [7
marks]}\label{question-1c-or-7-marks}

\textbf{Describe Linux architecture and discuss the mode of the
operation of Linux}

\begin{solutionbox}

\textbf{Linux Architecture Diagram:}

\begin{center}
\textbf{Mermaid Diagram (Code)}
\begin{verbatim}
{Shaded}
{Highlighting}[]
graph LR
    A[User Applications] {-{-}{} B[System Libraries]}
    B {-{-}{} C[System Call Interface]}
    C {-{-}{} D[Linux Kernel]}
    D {-{-}{} E[Hardware Layer]}
    
    subgraph "Kernel Space"
    D
    end
    
    subgraph "User Space"
    A
    B
    C
    end
{Highlighting}
{Shaded}
\end{verbatim}
\end{center}


{\def\LTcaptype{none} % do not increment counter
\vspace{-5pt}
\captionof{table}{Linux Operation Modes}
\vspace{-10pt}
\begin{longtable}[]{@{}
  >{\raggedright\arraybackslash}p{(\linewidth - 6\tabcolsep) * \real{0.1395}}
  >{\raggedright\arraybackslash}p{(\linewidth - 6\tabcolsep) * \real{0.3023}}
  >{\raggedright\arraybackslash}p{(\linewidth - 6\tabcolsep) * \real{0.3256}}
  >{\raggedright\arraybackslash}p{(\linewidth - 6\tabcolsep) * \real{0.2326}}@{}}
\toprule\noalign{}
\begin{minipage}[b]{\linewidth}\raggedright
Mode
\end{minipage} & \begin{minipage}[b]{\linewidth}\raggedright
Description
\end{minipage} & \begin{minipage}[b]{\linewidth}\raggedright
Access Level
\end{minipage} & \begin{minipage}[b]{\linewidth}\raggedright
Examples
\end{minipage} \\
\midrule\noalign{}
\endhead
\bottomrule\noalign{}
\endlastfoot
\textbf{User Mode} & Restricted access & Limited privileges &
Applications, user programs \\
\textbf{Kernel Mode} & Full system access & Complete control & Device
drivers, OS functions \\
\end{longtable}
}

\begin{itemize}
\tightlist
\item
  \textbf{Layered Architecture}: Clear separation between user and
  system components
\item
  \textbf{Mode Switching}: CPU switches between user and kernel modes
\item
  \textbf{System Calls}: Interface for user programs to access kernel
  services
\item
  \textbf{Security}: User mode prevents direct hardware access
\end{itemize}

\end{solutionbox}
\begin{mnemonicbox}
``LUSK - Linux Uses Safe Kernel protection''

\end{mnemonicbox}
\subsection*{Question 2(a) [3 marks]}\label{q2a}

\textbf{Describe XOR gate with its truth table.}

\begin{solutionbox}

\textbf{XOR Gate Symbol:}

\begin{verbatim}
    A ──┐
        │ )──── Output
    B ──┘
\end{verbatim}

\textbf{Truth Table:}

{\def\LTcaptype{none} % do not increment counter
\begin{longtable}[]{@{}lll@{}}
\toprule\noalign{}
A & B & Output (A \oplus B) \\
\midrule\noalign{}
\endhead
\bottomrule\noalign{}
\endlastfoot
0 & 0 & 0 \\
0 & 1 & 1 \\
1 & 0 & 1 \\
1 & 1 & 0 \\
\end{longtable}
}

\begin{itemize}
\tightlist
\item
  \textbf{Exclusive OR}: Output is 1 when inputs are different
\item
  \textbf{Logic Function}: A \oplus B = A'B + AB'
\item
  \textbf{Applications}: Half adder, parity checker, encryption
\end{itemize}

\end{solutionbox}
\begin{mnemonicbox}
``XOR - eXclusive OR gives 1 for different inputs''

\end{mnemonicbox}
\subsection*{Question 2(b) [4 marks]}\label{q2b}

\textbf{Solve following.} \textbf{i) (4C6)_{1}_{6} = (\_\_\_\_\_)_{2} =
(\_\_\_\_\_)_{1}_{0}} \textbf{ii) (186)_{1}_{0} = (\_\_\_\_\_)_{8} = (\_\_\_\_\_)_{2}}

\begin{solutionbox}

\textbf{Solution Table:}

{\def\LTcaptype{none} % do not increment counter
\begin{longtable}[]{@{}lll@{}}
\toprule\noalign{}
Conversion & Step & Result \\
\midrule\noalign{}
\endhead
\bottomrule\noalign{}
\endlastfoot
\textbf{(4C6)_{1}_{6}} & Hex to Binary & \textbf{10011000110_{2}} \\
& Binary to Decimal & \textbf{1222_{1}_{0}} \\
\textbf{(186)_{1}_{0}} & Decimal to Octal & \textbf{272_{8}} \\
& Decimal to Binary & \textbf{10111010_{2}} \\
\end{longtable}
}

\textbf{Detailed Solutions:}

\begin{enumerate}
\tightlist
\item
  \textbf{(4C6)_{1}_{6} = (10011000110)_{2} = (1222)_{1}_{0}}
\end{enumerate}

\begin{itemize}
\tightlist
\item
4 = 0100,

C = 1100, 6 = 0110

\item
  Combined: 010011000110 = 10011000110_{2}
\item
  Decimal: 1\times2^{1}^{0} + 0\times2^{9} + 0\times2^{8} + 1\times2^{7} + 1\times2^{6} + 0\times2^{5} + 0\times2^{4} + 0\times2^{3} + 1\times2^{2}
  + 1\times2^{1} + 0\times2^{0} = 1222_{1}_{0}
\end{itemize}

\begin{enumerate}
\tightlist
\item
  \textbf{(186)_{1}_{0} = (272)_{8} = (10111010)_{2}}
\end{enumerate}

\begin{itemize}
\tightlist
\item
  Octal: 186 \div 8 = 23 remainder 2, 23 \div 8 = 2 remainder 7, 2 \div 8 = 0
  remainder 2 \rightarrow 272_{8}
\item
  Binary: 186 = 128 + 32 + 16 + 8 + 2 = 10111010_{2}
\end{itemize}

\end{solutionbox}
\begin{mnemonicbox}
``HDB - Hex, Decimal, Binary conversions''

\end{mnemonicbox}
\subsection*{Question 2(c) [7 marks]}\label{q2c}

\textbf{Illustrate following OS} \textbf{i) Network Operating System}
\textbf{ii) Mobile Operating System}

\begin{solutionbox}


{\def\LTcaptype{none} % do not increment counter
\vspace{-5pt}
\captionof{table}{Operating System Comparison}
\vspace{-10pt}
\begin{longtable}[]{@{}
  >{\raggedright\arraybackslash}p{(\linewidth - 4\tabcolsep) * \real{0.2812}}
  >{\raggedright\arraybackslash}p{(\linewidth - 4\tabcolsep) * \real{0.3750}}
  >{\raggedright\arraybackslash}p{(\linewidth - 4\tabcolsep) * \real{0.3438}}@{}}
\toprule\noalign{}
\begin{minipage}[b]{\linewidth}\raggedright
Feature
\end{minipage} & \begin{minipage}[b]{\linewidth}\raggedright
Network OS
\end{minipage} & \begin{minipage}[b]{\linewidth}\raggedright
Mobile OS
\end{minipage} \\
\midrule\noalign{}
\endhead
\bottomrule\noalign{}
\endlastfoot
\textbf{Purpose} & Manage network resources & Mobile device
management \\
\textbf{Examples} & Windows Server, Linux Server & Android, iOS, Windows
Mobile \\
\textbf{Key Features} & File sharing, printer sharing & Touch interface,
battery management \\
\textbf{Users} & Multiple simultaneous users & Single user typically \\
\end{longtable}
}

\begin{center}
\textbf{Mermaid Diagram (Code)}
\begin{verbatim}
{Shaded}
{Highlighting}[]
graph TD
    A[Network OS] {-{-}{} B[File Server]}
    A {-{-}{} C[Print Server]}
    A {-{-}{} D[Application Server]}
    
    E[Mobile OS] {-{-}{} F[Touch Interface]}
    E {-{-}{} G[App Store]}
    E {-{-}{} H[Battery Management]}
{Highlighting}
{Shaded}
\end{verbatim}
\end{center}

\textbf{i) Network Operating System:}

\begin{itemize}
\tightlist
\item
  \textbf{Multi-user Support}: Handles multiple concurrent users
\item
  \textbf{Resource Sharing}: Files, printers, applications shared across
  network
\item
  \textbf{Security Management}: User authentication and access control
\end{itemize}

\textbf{ii) Mobile Operating System:}

\begin{itemize}
\tightlist
\item
  \textbf{Touch-Optimized}: Designed for finger-based interaction
\item
  \textbf{Power Management}: Efficient battery usage
\item
  \textbf{App Ecosystem}: Centralized app distribution and management
\end{itemize}

\end{solutionbox}
\begin{mnemonicbox}
``NOS for Networks, MOS for Mobility''

\end{mnemonicbox}
\subsection*{Question 2(a OR) [3
marks]}\label{question-2a-or-3-marks}

\textbf{Draw Logic circuit of OR gate and NOT gate using only NAND
gate.}

\begin{solutionbox}

\textbf{OR Gate using NAND:}

\begin{verbatim}
A ──┐  ┌─── NOT A ──┐
    │ )────────────── │
    └─┘              │ )──── A+B
B ──┐  ┌─── NOT B ──┘
    │ )──────────────
    └─┘
\end{verbatim}

\textbf{NOT Gate using NAND:}

\begin{verbatim}
A ──┐
    │ )──── A{}
A ──┘
\end{verbatim}

\textbf{Truth Verification Table:}

{\def\LTcaptype{none} % do not increment counter
\begin{longtable}[]{@{}lllll@{}}
\toprule\noalign{}
A & B & A' & B' & (A'·B')' = A+B \\
\midrule\noalign{}
\endhead
\bottomrule\noalign{}
\endlastfoot
0 & 0 & 1 & 1 & 0 \\
0 & 1 & 1 & 0 & 1 \\
1 & 0 & 0 & 1 & 1 \\
1 & 1 & 0 & 0 & 1 \\
\end{longtable}
}

\begin{itemize}
\tightlist
\item
  \textbf{NAND Universal}: Can implement any logic function
\item
  \textbf{De Morgan's Law}: (A'·B')' = A+B
\end{itemize}

\end{solutionbox}
\begin{mnemonicbox}
``NAND is Universal - can make all gates''

\end{mnemonicbox}
\subsection*{Question 2(b OR) [4
marks]}\label{question-2b-or-4-marks}

\textbf{i) Convert Binary number into Decimal number: (i) 11101 (ii)
10011} \textbf{ii) Convert decimal number into binary number: (i) 19
(ii) 64}

\begin{solutionbox}

\textbf{Conversion Table:}

{\def\LTcaptype{none} % do not increment counter
\begin{longtable}[]{@{}llll@{}}
\toprule\noalign{}
Type & Number & Process & Result \\
\midrule\noalign{}
\endhead
\bottomrule\noalign{}
\endlastfoot
\textbf{Binary to Decimal} & 11101_{2} & 1\times2^{4}+1\times2^{3}+1\times2^{2}+0\times2^{1}+1\times2^{0} &
\textbf{29_{1}_{0}} \\
& 10011_{2} & 1\times2^{4}+0\times2^{3}+0\times2^{2}+1\times2^{1}+1\times2^{0} & \textbf{19_{1}_{0}} \\
\textbf{Decimal to Binary} & 19_{1}_{0} & Division by 2 method &
\textbf{10011_{2}} \\
& 64_{1}_{0} & Division by 2 method & \textbf{1000000_{2}} \\
\end{longtable}
}

\textbf{Detailed Solutions:}

\textbf{i) Binary to Decimal:}

\begin{itemize}
\tightlist
\item
  11101_{2} = 16 + 8 + 4 + 0 + 1 = 29_{1}_{0}
\item
  10011_{2} = 16 + 0 + 0 + 2 + 1 = 19_{1}_{0}
\end{itemize}

\textbf{ii) Decimal to Binary:}

\begin{itemize}
\tightlist
\item
  19 \div 2 = 9 remainder 1, 9 \div 2 = 4 remainder 1, 4 \div 2 = 2 remainder 0,
  2 \div 2 = 1 remainder 0, 1 \div 2 = 0 remainder 1 \rightarrow 10011_{2}
\item
  64 \div 2 = 32 remainder 0\ldots{} \rightarrow 1000000_{2}
\end{itemize}

\end{solutionbox}
\begin{mnemonicbox}
``Powers of 2 for Binary to Decimal''

\end{mnemonicbox}
\subsection*{Question 2(c OR) [7
marks]}\label{question-2c-or-7-marks}

\textbf{Explain Open-source software and Proprietary software. Give at
least five examples of both the types of software.}

\begin{solutionbox}


{\def\LTcaptype{none} % do not increment counter
\vspace{-5pt}
\captionof{table}{Software Type Comparison}
\vspace{-10pt}
\begin{longtable}[]{@{}lll@{}}
\toprule\noalign{}
Aspect & Open-Source & Proprietary \\
\midrule\noalign{}
\endhead
\bottomrule\noalign{}
\endlastfoot
\textbf{Source Code} & Freely available & Closed/Hidden \\
\textbf{Cost} & Usually free & Commercial license \\
\textbf{Modification} & Allowed & Restricted \\
\textbf{Support} & Community-based & Vendor support \\
\end{longtable}
}

\textbf{Software Examples:}

{\def\LTcaptype{none} % do not increment counter
\begin{longtable}[]{@{}ll@{}}
\toprule\noalign{}
Open-Source & Proprietary \\
\midrule\noalign{}
\endhead
\bottomrule\noalign{}
\endlastfoot
Linux & Microsoft Windows \\
LibreOffice & Microsoft Office \\
Firefox & Internet Explorer \\
GIMP & Adobe Photoshop \\
MySQL & Oracle Database \\
\end{longtable}
}

\begin{verbatim}
pie title Software Distribution
    "Open{-Source" : 40}
    "Proprietary" : 60
\end{verbatim}

\textbf{Open-Source Characteristics:}

\begin{itemize}
\tightlist
\item
  \textbf{Freedom to Modify}: Users can change source code
\item
  \textbf{Community Development}: Collaborative improvement
\item
  \textbf{Transparency}: All code is visible and auditable
\end{itemize}

\textbf{Proprietary Characteristics:}

\begin{itemize}
\tightlist
\item
  \textbf{Commercial Model}: Revenue through licensing
\item
  \textbf{Professional Support}: Dedicated customer service
\item
  \textbf{Quality Assurance}: Rigorous testing and validation
\end{itemize}

\end{solutionbox}
\begin{mnemonicbox}
``FOSS is Free, Open, Shared, Supported by
community''

\end{mnemonicbox}
\subsection*{Question 3(a) [3 marks]}\label{q3a}

\textbf{Define} \textbf{1. Modulation} \textbf{2. Multiplexing}

\begin{solutionbox}

\textbf{Definition Table:}

{\def\LTcaptype{none} % do not increment counter
\begin{longtable}[]{@{}
  >{\raggedright\arraybackslash}p{(\linewidth - 4\tabcolsep) * \real{0.2222}}
  >{\raggedright\arraybackslash}p{(\linewidth - 4\tabcolsep) * \real{0.4444}}
  >{\raggedright\arraybackslash}p{(\linewidth - 4\tabcolsep) * \real{0.3333}}@{}}
\toprule\noalign{}
\begin{minipage}[b]{\linewidth}\raggedright
Term
\end{minipage} & \begin{minipage}[b]{\linewidth}\raggedright
Definition
\end{minipage} & \begin{minipage}[b]{\linewidth}\raggedright
Purpose
\end{minipage} \\
\midrule\noalign{}
\endhead
\bottomrule\noalign{}
\endlastfoot
\textbf{Modulation} & Process of varying carrier signal properties &
Enable long-distance transmission \\
\textbf{Multiplexing} & Combining multiple signals for transmission &
Efficient channel utilization \\
\end{longtable}
}

\begin{itemize}
\tightlist
\item
  \textbf{Modulation}: Changes amplitude, frequency, or phase of carrier
  wave
\item
  \textbf{Multiplexing}: Allows multiple users to share same
  communication medium
\item
  \textbf{Signal Processing}: Both techniques improve communication
  efficiency
\end{itemize}

\end{solutionbox}
\begin{mnemonicbox}
``MM - Modulation Modifies, Multiplexing Merges''

\end{mnemonicbox}
\subsection*{Question 3(b) [4 marks]}\label{q3b}

\textbf{Explain star topology.}

\begin{solutionbox}

\textbf{Star Topology Diagram:}

\begin{verbatim}
    Computer1
        |
Computer4──Hub──Computer2
        |
    Computer3
\end{verbatim}


{\def\LTcaptype{none} % do not increment counter
\vspace{-5pt}
\captionof{table}{Star Topology Features}
\vspace{-10pt}
\begin{longtable}[]{@{}ll@{}}
\toprule\noalign{}
Feature & Description \\
\midrule\noalign{}
\endhead
\bottomrule\noalign{}
\endlastfoot
\textbf{Central Device} & Hub/Switch connects all nodes \\
\textbf{Fault Tolerance} & Single node failure doesn't affect others \\
\textbf{Performance} & Dedicated bandwidth per connection \\
\textbf{Scalability} & Easy to add/remove nodes \\
\end{longtable}
}

\begin{itemize}
\tightlist
\item
  \textbf{Central Hub}: All communication passes through central device
\item
  \textbf{Easy Troubleshooting}: Problems isolated to individual
  connections
\item
  \textbf{Higher Cost}: Requires more cable than bus topology
\item
  \textbf{Single Point of Failure}: Hub failure affects entire network
\end{itemize}

\end{solutionbox}
\begin{mnemonicbox}
``STAR - Single point, Troubleshooting easy, All
through hub, Reliable''

\end{mnemonicbox}
\subsection*{Question 3(c) [7 marks]}\label{q3c}

\textbf{Prepare a short note on Time Division Multiplexing (TDM)}

\begin{solutionbox}

\textbf{TDM Concept Diagram:}

\begin{verbatim}
gantt
    title Time Division Multiplexing
    dateFormat X
    axisFormat \%s
    
    section Channel A
    Slot 1    :0, 1
    Slot 4    :3, 4
    Slot 7    :6, 7
    
    section Channel B  
    Slot 2    :1, 2
    Slot 5    :4, 5
    Slot 8    :7, 8
    
    section Channel C
    Slot 3    :2, 3
    Slot 6    :5, 6
    Slot 9    :8, 9
\end{verbatim}


{\def\LTcaptype{none} % do not increment counter
\vspace{-5pt}
\captionof{table}{TDM Characteristics}
\vspace{-10pt}
\begin{longtable}[]{@{}ll@{}}
\toprule\noalign{}
Feature & Description \\
\midrule\noalign{}
\endhead
\bottomrule\noalign{}
\endlastfoot
\textbf{Principle} & Different users allocated different time slots \\
\textbf{Synchronization} & All devices must be synchronized \\
\textbf{Efficiency} & Full bandwidth utilization when slots filled \\
\textbf{Applications} & Digital telephone systems, T1/E1 lines \\
\end{longtable}
}

\textbf{TDM Types:}

\begin{itemize}
\tightlist
\item
  \textbf{Synchronous TDM}: Fixed time slots regardless of data
  availability
\item
  \textbf{Asynchronous TDM}: Dynamic slot allocation based on demand
\item
  \textbf{Statistical TDM}: Slots allocated on statistical basis
\end{itemize}

\textbf{Advantages:}

\begin{itemize}
\tightlist
\item
  \textbf{Fair Sharing}: Equal time allocation for all users
\item
  \textbf{No Signal Interference}: Time-based separation prevents
  conflicts
\end{itemize}

\end{solutionbox}
\begin{mnemonicbox}
``TDM - Time Divides Medium fairly''

\end{mnemonicbox}
\subsection*{Question 3(a OR) [3
marks]}\label{question-3a-or-3-marks}

\textbf{Explain Amplitude Modulation (AM).}

\begin{solutionbox}

\textbf{AM Signal Diagram:}

\begin{verbatim}
Message Signal:    {}
                  
Carrier Signal:    ||||||||||||||||||||

AM Output:         |{|||||||||}
\end{verbatim}


{\def\LTcaptype{none} % do not increment counter
\vspace{-5pt}
\captionof{table}{AM Characteristics}
\vspace{-10pt}
\begin{longtable}[]{@{}ll@{}}
\toprule\noalign{}
Parameter & Description \\
\midrule\noalign{}
\endhead
\bottomrule\noalign{}
\endlastfoot
\textbf{Definition} & Amplitude of carrier varies with message signal \\
\textbf{Frequency Range} & 535-1605 kHz (AM radio) \\
\textbf{Bandwidth} & Twice the message signal frequency \\
\end{longtable}
}

\begin{itemize}
\tightlist
\item
  \textbf{Carrier Wave}: High frequency signal that carries information
\item
  \textbf{Modulation Index}: Determines depth of amplitude variation
\item
  \textbf{Applications}: AM radio broadcasting, aircraft communication
\end{itemize}

\end{solutionbox}
\begin{mnemonicbox}
``AM - Amplitude Modifies with message''

\end{mnemonicbox}
\subsection*{Question 3(b OR) [4
marks]}\label{question-3b-or-4-marks}

\textbf{Describe DNS.}

\begin{solutionbox}

\textbf{DNS Hierarchy:}

\begin{center}
\textbf{Mermaid Diagram (Code)}
\begin{verbatim}
{Shaded}
{Highlighting}[]
graph TD
    A[Root .] {-{-}{} B[Top Level .com]}
    A {-{-}{} C[Top Level .org]}
    B {-{-}{} D[google.com]}
    B {-{-}{} E[microsoft.com]}
    D {-{-}{} F[www.google.com]}
    D {-{-}{} G[mail.google.com]}
{Highlighting}
{Shaded}
\end{verbatim}
\end{center}


{\def\LTcaptype{none} % do not increment counter
\vspace{-5pt}
\captionof{table}{DNS Components}
\vspace{-10pt}
\begin{longtable}[]{@{}ll@{}}
\toprule\noalign{}
Component & Function \\
\midrule\noalign{}
\endhead
\bottomrule\noalign{}
\endlastfoot
\textbf{Domain Name} & Human-readable web address \\
\textbf{IP Address} & Numerical address of server \\
\textbf{DNS Server} & Translates names to IP addresses \\
\textbf{Records} & Different types (A, MX, CNAME) \\
\end{longtable}
}

\begin{itemize}
\tightlist
\item
  \textbf{Name Resolution}: Converts domain names to IP addresses
\item
  \textbf{Hierarchical Structure}: Root, TLD, second-level domains
\item
  \textbf{Distributed Database}: No single point of failure
\item
  \textbf{Caching}: Improves performance by storing recent lookups
\end{itemize}

\end{solutionbox}
\begin{mnemonicbox}
``DNS - Domain Name System translates addresses''

\end{mnemonicbox}
\subsection*{Question 3(c OR) [7
marks]}\label{question-3c-or-7-marks}

\textbf{Describe following} \textbf{1. Serial Communication} \textbf{2.
Synchronous Transmission}

\begin{solutionbox}

\textbf{Communication Types Diagram:}

\begin{center}
\textbf{Mermaid Diagram (Code)}
\begin{verbatim}
{Shaded}
{Highlighting}[]
graph LR
    A[Data Communication] {-{-}{} B[Serial]}
    A {-{-}{} C[Parallel]}
    B {-{-}{} D[Synchronous]}
    B {-{-}{} E[Asynchronous]}
{Highlighting}
{Shaded}
\end{verbatim}
\end{center}


{\def\LTcaptype{none} % do not increment counter
\vspace{-5pt}
\captionof{table}{Communication Comparison}
\vspace{-10pt}
\begin{longtable}[]{@{}
  >{\raggedright\arraybackslash}p{(\linewidth - 6\tabcolsep) * \real{0.1622}}
  >{\raggedright\arraybackslash}p{(\linewidth - 6\tabcolsep) * \real{0.3514}}
  >{\raggedright\arraybackslash}p{(\linewidth - 6\tabcolsep) * \real{0.2162}}
  >{\raggedright\arraybackslash}p{(\linewidth - 6\tabcolsep) * \real{0.2703}}@{}}
\toprule\noalign{}
\begin{minipage}[b]{\linewidth}\raggedright
Type
\end{minipage} & \begin{minipage}[b]{\linewidth}\raggedright
Description
\end{minipage} & \begin{minipage}[b]{\linewidth}\raggedright
Timing
\end{minipage} & \begin{minipage}[b]{\linewidth}\raggedright
Examples
\end{minipage} \\
\midrule\noalign{}
\endhead
\bottomrule\noalign{}
\endlastfoot
\textbf{Serial Communication} & Data bits sent one after another &
Slower but reliable & RS-232, USB, Ethernet \\
\textbf{Synchronous Transmission} & Clock signal synchronizes
sender/receiver & Precise timing & HDLC, SDLC \\
\end{longtable}
}

\textbf{1. Serial Communication:}

\begin{itemize}
\tightlist
\item
  \textbf{Single Wire}: Data transmitted bit by bit over single channel
\item
  \textbf{Cost Effective}: Requires fewer wires than parallel
\item
  \textbf{Long Distance}: Less susceptible to noise and interference
\item
  \textbf{Error Detection}: Built-in mechanisms for data integrity
\end{itemize}

\textbf{2. Synchronous Transmission:}

\begin{itemize}
\tightlist
\item
  \textbf{Clock Synchronization}: Separate clock signal or embedded
  timing
\item
  \textbf{Block Transmission}: Data sent in continuous blocks
\item
  \textbf{Higher Efficiency}: No start/stop bits needed
\item
  \textbf{Complex Hardware}: Requires synchronized clocks
\end{itemize}

\end{solutionbox}
\begin{mnemonicbox}
``Serial is Sequential, Synchronous is Simultaneous''

\end{mnemonicbox}
\subsection*{Question 4(a) [3 marks]}\label{q4a}

\textbf{Differentiate Mesh and Bus topology.}

\begin{solutionbox}

\textbf{Topology Comparison Table:}

{\def\LTcaptype{none} % do not increment counter
\begin{longtable}[]{@{}
  >{\raggedright\arraybackslash}p{(\linewidth - 4\tabcolsep) * \real{0.2368}}
  >{\raggedright\arraybackslash}p{(\linewidth - 4\tabcolsep) * \real{0.3947}}
  >{\raggedright\arraybackslash}p{(\linewidth - 4\tabcolsep) * \real{0.3684}}@{}}
\toprule\noalign{}
\begin{minipage}[b]{\linewidth}\raggedright
Feature
\end{minipage} & \begin{minipage}[b]{\linewidth}\raggedright
Mesh Topology
\end{minipage} & \begin{minipage}[b]{\linewidth}\raggedright
Bus Topology
\end{minipage} \\
\midrule\noalign{}
\endhead
\bottomrule\noalign{}
\endlastfoot
\textbf{Connection} & Every node connected to every other & All nodes on
single cable \\
\textbf{Fault Tolerance} & Very high & Low (single point of failure) \\
\textbf{Cost} & Very expensive & Economical \\
\textbf{Performance} & Excellent & Degrades with more nodes \\
\end{longtable}
}

\textbf{Mesh Topology:}

\begin{verbatim}
A ─── B
│ { / │}
│  X  │
│ / { │}
C ─── D
\end{verbatim}

\textbf{Bus Topology:}

\begin{verbatim}
A ── B ── C ── D ── Terminator
\end{verbatim}

\begin{itemize}
\tightlist
\item
  \textbf{Mesh Advantages}: Redundant paths, high reliability
\item
  \textbf{Bus Advantages}: Simple installation, cost-effective
\item
  \textbf{Cable Requirements}: Mesh needs n(n-1)/2 connections, Bus
  needs single cable
\end{itemize}

\end{solutionbox}
\begin{mnemonicbox}
``Mesh is Many connections, Bus is Basic single
line''

\end{mnemonicbox}
\subsection*{Question 4(b) [4 marks]}\label{q4b}

\textbf{Compare FDM and TDM.}

\begin{solutionbox}


{\def\LTcaptype{none} % do not increment counter
\vspace{-5pt}
\captionof{table}{FDM vs TDM Comparison}
\vspace{-10pt}
\begin{longtable}[]{@{}
  >{\raggedright\arraybackslash}p{(\linewidth - 4\tabcolsep) * \real{0.5238}}
  >{\raggedright\arraybackslash}p{(\linewidth - 4\tabcolsep) * \real{0.2381}}
  >{\raggedright\arraybackslash}p{(\linewidth - 4\tabcolsep) * \real{0.2381}}@{}}
\toprule\noalign{}
\begin{minipage}[b]{\linewidth}\raggedright
Parameter
\end{minipage} & \begin{minipage}[b]{\linewidth}\raggedright
FDM
\end{minipage} & \begin{minipage}[b]{\linewidth}\raggedright
TDM
\end{minipage} \\
\midrule\noalign{}
\endhead
\bottomrule\noalign{}
\endlastfoot
\textbf{Full Form} & Frequency Division Multiplexing & Time Division
Multiplexing \\
\textbf{Division Basis} & Frequency bands & Time slots \\
\textbf{Signal Type} & Analog & Digital \\
\textbf{Crosstalk} & Possible between channels & No crosstalk \\
\textbf{Synchronization} & Not required & Required \\
\textbf{Efficiency} & Lower due to guard bands & Higher efficiency \\
\end{longtable}
}

\begin{verbatim}
graph TB
    A[Multiplexing Techniques] {-{-} B[FDM]}
    A {-{-} C[TDM]}
    B {-{-} D[Radio Broadcasting]}
    B {-{-} E[Cable TV]}
    C {-{-} F[Digital Telephony]}
    C {-{-} G[Computer Networks]}
\end{verbatim}

\textbf{FDM Characteristics:}

\begin{itemize}
\tightlist
\item
  \textbf{Frequency Separation}: Each signal allocated different
  frequency band
\item
  \textbf{Simultaneous Transmission}: All signals transmitted at same
  time
\item
  \textbf{Guard Bands}: Prevent interference between channels
\end{itemize}

\textbf{TDM Characteristics:}

\begin{itemize}
\tightlist
\item
  \textbf{Time Separation}: Each signal allocated different time slot
\item
  \textbf{Sequential Transmission}: Signals transmitted one after
  another
\item
  \textbf{Precise Timing}: Requires synchronized clocks
\end{itemize}

\end{solutionbox}
\begin{mnemonicbox}
``FDM uses Frequency, TDM uses Time''

\end{mnemonicbox}
\subsection*{Question 4(c) [7 marks]}\label{q4c}

\textbf{Draw and illustrate OSI reference model.}

\begin{solutionbox}

\textbf{OSI Model Diagram:}

\begin{center}
\textbf{Mermaid Diagram (Code)}
\begin{verbatim}
{Shaded}
{Highlighting}[]
graph LR
    A[Application Layer {- Layer 7] {-}{-}{} B[Presentation Layer {-} Layer 6]}
    B {-{-}{} C[Session Layer {-} Layer 5]}
    C {-{-}{} D[Transport Layer {-} Layer 4]}
    D {-{-}{} E[Network Layer {-} Layer 3]}
    E {-{-}{} F[Data Link Layer {-} Layer 2]}
    F {-{-}{} G[Physical Layer {-} Layer 1]}
{Highlighting}
{Shaded}
\end{verbatim}
\end{center}


{\def\LTcaptype{none} % do not increment counter
\vspace{-5pt}
\captionof{table}{OSI Layer Functions}
\vspace{-10pt}
\begin{longtable}[]{@{}llll@{}}
\toprule\noalign{}
Layer & Name & Function & Examples \\
\midrule\noalign{}
\endhead
\bottomrule\noalign{}
\endlastfoot
\textbf{7} & Application & User interface & HTTP, FTP, SMTP \\
\textbf{6} & Presentation & Data formatting & Encryption, Compression \\
\textbf{5} & Session & Session management & NetBIOS, RPC \\
\textbf{4} & Transport & End-to-end delivery & TCP, UDP \\
\textbf{3} & Network & Routing & IP, ICMP \\
\textbf{2} & Data Link & Frame delivery & Ethernet, PPP \\
\textbf{1} & Physical & Bit transmission & Cables, Hubs \\
\end{longtable}
}

\textbf{Key Features:}

\begin{itemize}
\tightlist
\item
  \textbf{Layered Architecture}: Each layer has specific
  responsibilities
\item
  \textbf{Protocol Independence}: Layers can be modified independently
\item
  \textbf{Standardization}: Common framework for network communication
\item
  \textbf{Encapsulation}: Each layer adds its own header
\end{itemize}

\end{solutionbox}
\begin{mnemonicbox}
``All People Seem To Need Data Processing''

\end{mnemonicbox}
\subsection*{Question 4(a OR) [3
marks]}\label{question-4a-or-3-marks}

\textbf{Describe Hub in brief.}

\begin{solutionbox}

\textbf{Hub Diagram:}

\begin{verbatim}
    PC1
     |
PC4──HUB──PC2
     |
    PC3
\end{verbatim}


{\def\LTcaptype{none} % do not increment counter
\vspace{-5pt}
\captionof{table}{Hub Characteristics}
\vspace{-10pt}
\begin{longtable}[]{@{}ll@{}}
\toprule\noalign{}
Feature & Description \\
\midrule\noalign{}
\endhead
\bottomrule\noalign{}
\endlastfoot
\textbf{Function} & Central connection point for devices \\
\textbf{Type} & Physical layer device (Layer 1) \\
\textbf{Data Handling} & Broadcasts to all connected devices \\
\textbf{Collision Domain} & All ports share same collision domain \\
\end{longtable}
}

\begin{itemize}
\tightlist
\item
  \textbf{Shared Bandwidth}: All connected devices share total bandwidth
\item
  \textbf{Half-Duplex}: Cannot send and receive simultaneously
\item
  \textbf{Security Issues}: All devices receive all transmitted data
\item
  \textbf{Obsolete Technology}: Replaced by switches in modern networks
\end{itemize}

\end{solutionbox}
\begin{mnemonicbox}
``Hub is Half-duplex, shares Bandwidth''

\end{mnemonicbox}
\subsection*{Question 4(b OR) [4
marks]}\label{question-4b-or-4-marks}

\textbf{Compare STP and UTP.}

\begin{solutionbox}


{\def\LTcaptype{none} % do not increment counter
\vspace{-5pt}
\captionof{table}{STP vs UTP Cable Comparison}
\vspace{-10pt}
\begin{longtable}[]{@{}
  >{\raggedright\arraybackslash}p{(\linewidth - 4\tabcolsep) * \real{0.1324}}
  >{\raggedright\arraybackslash}p{(\linewidth - 4\tabcolsep) * \real{0.4118}}
  >{\raggedright\arraybackslash}p{(\linewidth - 4\tabcolsep) * \real{0.4559}}@{}}
\toprule\noalign{}
\begin{minipage}[b]{\linewidth}\raggedright
Feature
\end{minipage} & \begin{minipage}[b]{\linewidth}\raggedright
STP (Shielded Twisted Pair)
\end{minipage} & \begin{minipage}[b]{\linewidth}\raggedright
UTP (Unshielded Twisted Pair)
\end{minipage} \\
\midrule\noalign{}
\endhead
\bottomrule\noalign{}
\endlastfoot
\textbf{Shielding} & Metal foil/braid protection & No shielding \\
\textbf{Cost} & More expensive & Less expensive \\
\textbf{Installation} & Complex due to grounding & Simple
installation \\
\textbf{EMI Resistance} & Excellent protection & Moderate protection \\
\textbf{Applications} & Industrial environments & Office environments \\
\end{longtable}
}

\textbf{Cable Structure:}

\begin{verbatim}
UTP:    |wire1 wire2|
        |wire3 wire4|

STP:    |Shield|wire1 wire2|Shield|
        |Shield|wire3 wire4|Shield|
\end{verbatim}

\textbf{STP Advantages:}

\begin{itemize}
\tightlist
\item
  \textbf{Better Noise Immunity}: Shield blocks electromagnetic
  interference
\item
  \textbf{Higher Data Rates}: Supports faster transmission speeds
\item
  \textbf{Secure Transmission}: Less susceptible to eavesdropping
\end{itemize}

\textbf{UTP Advantages:}

\begin{itemize}
\tightlist
\item
  \textbf{Cost Effective}: Cheaper than STP
\item
  \textbf{Easy Installation}: No grounding requirements
\item
  \textbf{Flexibility}: More flexible and easier to handle
\end{itemize}

\end{solutionbox}
\begin{mnemonicbox}
``STP is Shielded but Pricey, UTP is Unshielded but
Popular''

\end{mnemonicbox}
\subsection*{Question 4(c OR) [7
marks]}\label{question-4c-or-7-marks}

\textbf{Distinguish LAN, MAN, WAN.}

\begin{solutionbox}

\textbf{Network Size Comparison:}

\begin{verbatim}
graph TB
    A[Computer Networks] {-{-} B[LAN {-} Local Area Network]}
    A {-{-} C[MAN {-} Metropolitan Area Network]  }
    A {-{-} D[WAN {-} Wide Area Network]}
    
    B {-{-} E[Building/Campus]}
    C {-{-} F[City/Metropolitan Area]}
    D {-{-} G[Country/Continent]}
\end{verbatim}


{\def\LTcaptype{none} % do not increment counter
\vspace{-5pt}
\captionof{table}{Network Type Comparison}
\vspace{-10pt}
\begin{longtable}[]{@{}
  >{\raggedright\arraybackslash}p{(\linewidth - 6\tabcolsep) * \real{0.4231}}
  >{\raggedright\arraybackslash}p{(\linewidth - 6\tabcolsep) * \real{0.1923}}
  >{\raggedright\arraybackslash}p{(\linewidth - 6\tabcolsep) * \real{0.1923}}
  >{\raggedright\arraybackslash}p{(\linewidth - 6\tabcolsep) * \real{0.1923}}@{}}
\toprule\noalign{}
\begin{minipage}[b]{\linewidth}\raggedright
Parameter
\end{minipage} & \begin{minipage}[b]{\linewidth}\raggedright
LAN
\end{minipage} & \begin{minipage}[b]{\linewidth}\raggedright
MAN
\end{minipage} & \begin{minipage}[b]{\linewidth}\raggedright
WAN
\end{minipage} \\
\midrule\noalign{}
\endhead
\bottomrule\noalign{}
\endlastfoot
\textbf{Coverage} & Building/Campus & City/Metropolitan area &
Country/Continent \\
\textbf{Speed} & 10 Mbps - 1 Gbps & 1-100 Mbps & 56 Kbps - 100 Mbps \\
\textbf{Cost} & Low & Medium & High \\
\textbf{Ownership} & Private & Private/Public & Public/Leased \\
\textbf{Technology} & Ethernet, Wi-Fi & Fiber optic, WiMAX & Satellite,
Leased lines \\
\textbf{Error Rate} & Very low & Low & Higher \\
\end{longtable}
}

\textbf{Detailed Characteristics:}

\textbf{LAN (Local Area Network):}

\begin{itemize}
\tightlist
\item
  \textbf{High Speed}: Fast data transmission within small area
\item
  \textbf{Low Cost}: Inexpensive to set up and maintain
\item
  \textbf{Private Ownership}: Usually owned by single organization
\end{itemize}

\textbf{MAN (Metropolitan Area Network):}

\begin{itemize}
\tightlist
\item
  \textbf{City-wide Coverage}: Spans across metropolitan area
\item
  \textbf{Medium Speed}: Moderate transmission speeds
\item
  \textbf{Mixed Ownership}: Can be public or private
\end{itemize}

\textbf{WAN (Wide Area Network):}

\begin{itemize}
\tightlist
\item
  \textbf{Global Coverage}: Spans countries and continents
\item
  \textbf{Variable Speed}: Depends on connection type
\item
  \textbf{Public Infrastructure}: Uses public telecommunication networks
\end{itemize}

\end{solutionbox}
\begin{mnemonicbox}
``LAN is Local, MAN is Metropolitan, WAN is Wide''

\end{mnemonicbox}
\subsection*{Question 5(a) [3 marks]}\label{q5a}

\textbf{Explain Denial of Service Attack.}

\begin{solutionbox}

\textbf{DoS Attack Diagram:}

\begin{center}
\textbf{Mermaid Diagram (Code)}
\begin{verbatim}
{Shaded}
{Highlighting}[]
graph LR
    A[Attacker] {-{-}{} B[Multiple Requests]}
    B {-{-}{} C[Target Server]}
    C {-{-}{} D[Server Overwhelmed]}
    D {-{-}{} E[Service Unavailable]}
{Highlighting}
{Shaded}
\end{verbatim}
\end{center}


{\def\LTcaptype{none} % do not increment counter
\vspace{-5pt}
\captionof{table}{DoS Attack Types}
\vspace{-10pt}
\begin{longtable}[]{@{}ll@{}}
\toprule\noalign{}
Type & Description \\
\midrule\noalign{}
\endhead
\bottomrule\noalign{}
\endlastfoot
\textbf{Volume-based} & Floods bandwidth with traffic \\
\textbf{Protocol-based} & Exploits protocol weaknesses \\
\textbf{Application-based} & Targets application resources \\
\end{longtable}
}

\begin{itemize}
\tightlist
\item
  \textbf{Objective}: Make services unavailable to legitimate users
\item
  \textbf{Methods}: Traffic flooding, resource exhaustion, exploit
  vulnerabilities
\item
  \textbf{Impact}: Service disruption, financial loss, reputation damage
\item
  \textbf{Prevention}: Firewalls, load balancers, intrusion detection
  systems
\end{itemize}

\end{solutionbox}
\begin{mnemonicbox}
``DoS Denies Others Service''

\end{mnemonicbox}
\subsection*{Question 5(b) [4 marks]}\label{q5b}

\textbf{i) Classify data transmission} \textbf{ii) Write down use of
Terminator in Bus Topology.}

\begin{solutionbox}

\textbf{i) Data Transmission Classification:}

\begin{center}
\textbf{Mermaid Diagram (Code)}
\begin{verbatim}
{Shaded}
{Highlighting}[]
graph TD
    A[Data Transmission] {-{-}{} B[Direction]}
    A {-{-}{} C[Timing]}
    A {-{-}{} D[Mode]}
    
    B {-{-}{} E[Simplex]}
    B {-{-}{} F[Half{-}Duplex]}
    B {-{-}{} G[Full{-}Duplex]}
    
    C {-{-}{} H[Synchronous]}
    C {-{-}{} I[Asynchronous]}
    
    D {-{-}{} J[Serial]}
    D {-{-}{} K[Parallel]}
{Highlighting}
{Shaded}
\end{verbatim}
\end{center}

\textbf{ii) Terminator in Bus Topology:}


{\def\LTcaptype{none} % do not increment counter
\vspace{-5pt}
\captionof{table}{Terminator Functions}
\vspace{-10pt}
\begin{longtable}[]{@{}ll@{}}
\toprule\noalign{}
Function & Description \\
\midrule\noalign{}
\endhead
\bottomrule\noalign{}
\endlastfoot
\textbf{Signal Absorption} & Prevents signal reflection \\
\textbf{Impedance Matching} & Matches cable impedance \\
\textbf{Network Integrity} & Maintains signal quality \\
\end{longtable}
}

\begin{itemize}
\tightlist
\item
  \textbf{Prevention of Reflection}: Stops signals from bouncing back
\item
  \textbf{Signal Quality}: Maintains clean signal transmission
\item
  \textbf{Required at Both Ends}: Bus topology needs terminators at both
  cable ends
\item
  \textbf{Resistance Value}: Usually 50 ohms for Ethernet networks
\end{itemize}

\end{solutionbox}
\begin{mnemonicbox}
``Terminator Stops signal Travel''

\end{mnemonicbox}
\subsection*{Question 5(c) [7 marks]}\label{q5c}

\textbf{Describe CIA triad.}

\begin{solutionbox}

\textbf{CIA Triad Diagram:}

\begin{center}
\textbf{Mermaid Diagram (Code)}
\begin{verbatim}
{Shaded}
{Highlighting}[]
graph TD
    A[CIA Triad] {-{-}{} B[Confidentiality]}
    A {-{-}{} C[Integrity]}
    A {-{-}{} D[Availability]}
    
    B {-{-}{} E[Encryption]}
    B {-{-}{} F[Access Control]}
    
    C {-{-}{} G[Hash Functions]}
    C {-{-}{} H[Digital Signatures]}
    
    D {-{-}{} I[Redundancy]}
    D {-{-}{} J[Backup Systems]}
{Highlighting}
{Shaded}
\end{verbatim}
\end{center}


{\def\LTcaptype{none} % do not increment counter
\vspace{-5pt}
\captionof{table}{CIA Triad Components}
\vspace{-10pt}
\begin{longtable}[]{@{}
  >{\raggedright\arraybackslash}p{(\linewidth - 6\tabcolsep) * \real{0.2292}}
  >{\raggedright\arraybackslash}p{(\linewidth - 6\tabcolsep) * \real{0.2500}}
  >{\raggedright\arraybackslash}p{(\linewidth - 6\tabcolsep) * \real{0.3333}}
  >{\raggedright\arraybackslash}p{(\linewidth - 6\tabcolsep) * \real{0.1875}}@{}}
\toprule\noalign{}
\begin{minipage}[b]{\linewidth}\raggedright
Component
\end{minipage} & \begin{minipage}[b]{\linewidth}\raggedright
Definition
\end{minipage} & \begin{minipage}[b]{\linewidth}\raggedright
Implementation
\end{minipage} & \begin{minipage}[b]{\linewidth}\raggedright
Threats
\end{minipage} \\
\midrule\noalign{}
\endhead
\bottomrule\noalign{}
\endlastfoot
\textbf{Confidentiality} & Information secrecy & Encryption, Access
control & Unauthorized disclosure \\
\textbf{Integrity} & Data accuracy and completeness & Hash functions,
Digital signatures & Data modification \\
\textbf{Availability} & Information accessibility & Redundancy, Backup
systems & Service disruption \\
\end{longtable}
}

\textbf{Detailed Explanation:}

\textbf{Confidentiality:}

\begin{itemize}
\tightlist
\item
  \textbf{Data Protection}: Only authorized users can access information
\item
  \textbf{Privacy Measures}: Encryption, authentication, access controls
\item
  \textbf{Examples}: Password protection, file permissions
\end{itemize}

\textbf{Integrity:}

\begin{itemize}
\tightlist
\item
  \textbf{Data Accuracy}: Information remains unaltered during
  transmission/storage
\item
  \textbf{Verification Methods}: Checksums, digital signatures, version
  control
\item
  \textbf{Examples}: Hash functions, database constraints
\end{itemize}

\textbf{Availability:}

\begin{itemize}
\tightlist
\item
  \textbf{System Accessibility}: Information and services available when
  needed
\item
  \textbf{Reliability Measures}: Redundancy, fault tolerance, disaster
  recovery
\item
  \textbf{Examples}: Load balancing, backup systems, UPS
\end{itemize}

\end{solutionbox}
\begin{mnemonicbox}
``CIA protects - Confidentiality, Integrity,
Availability''

\end{mnemonicbox}
\subsection*{Question 5(a OR) [3
marks]}\label{question-5a-or-3-marks}

\textbf{Define} \textbf{1. Cryptography} \textbf{2. Decryption}

\begin{solutionbox}

\textbf{Definition Table:}

{\def\LTcaptype{none} % do not increment counter
\begin{longtable}[]{@{}
  >{\raggedright\arraybackslash}p{(\linewidth - 4\tabcolsep) * \real{0.2222}}
  >{\raggedright\arraybackslash}p{(\linewidth - 4\tabcolsep) * \real{0.4444}}
  >{\raggedright\arraybackslash}p{(\linewidth - 4\tabcolsep) * \real{0.3333}}@{}}
\toprule\noalign{}
\begin{minipage}[b]{\linewidth}\raggedright
Term
\end{minipage} & \begin{minipage}[b]{\linewidth}\raggedright
Definition
\end{minipage} & \begin{minipage}[b]{\linewidth}\raggedright
Purpose
\end{minipage} \\
\midrule\noalign{}
\endhead
\bottomrule\noalign{}
\endlastfoot
\textbf{Cryptography} & Science of securing information through encoding
& Protect data confidentiality \\
\textbf{Decryption} & Process of converting encrypted data back to
original & Retrieve original information \\
\end{longtable}
}

\begin{itemize}
\tightlist
\item
  \textbf{Cryptography}: Uses mathematical algorithms to transform
  readable data into unreadable format
\item
  \textbf{Decryption}: Reverse process using keys to restore original
  data
\item
  \textbf{Key-based Security}: Both processes rely on cryptographic keys
\end{itemize}

\end{solutionbox}
\begin{mnemonicbox}
``Crypto Conceals, Decryption Discloses''

\end{mnemonicbox}
\subsection*{Question 5(b OR) [4
marks]}\label{question-5b-or-4-marks}

\textbf{i) State the reason why wires are twisted in twisted pair
cables.} \textbf{ii) Identify the layer of OSI model at which the
following network devices support 1. Router 2. Bridge}

\begin{solutionbox}

\textbf{i) Twisted Pair Cable Design:}

\begin{verbatim}
Normal Wires:     ||||||||||||
                  ||||||||||||
                  (Parallel interference)

Twisted Wires:    {//////}
                  /{/////}
                  (Cancellation effect)
\end{verbatim}


{\def\LTcaptype{none} % do not increment counter
\vspace{-5pt}
\captionof{table}{Wire Twisting Benefits}
\vspace{-10pt}
\begin{longtable}[]{@{}ll@{}}
\toprule\noalign{}
Benefit & Description \\
\midrule\noalign{}
\endhead
\bottomrule\noalign{}
\endlastfoot
\textbf{Noise Reduction} & Cancels electromagnetic interference \\
\textbf{Crosstalk Prevention} & Reduces signal interference between
pairs \\
\textbf{Signal Quality} & Maintains better signal integrity \\
\end{longtable}
}

\textbf{ii) OSI Layer Identification:}


{\def\LTcaptype{none} % do not increment counter
\vspace{-5pt}
\captionof{table}{Network Devices and OSI Layers}
\vspace{-10pt}
\begin{longtable}[]{@{}lll@{}}
\toprule\noalign{}
Device & OSI Layer & Function \\
\midrule\noalign{}
\endhead
\bottomrule\noalign{}
\endlastfoot
\textbf{Router} & Layer 3 (Network) & Routing between different
networks \\
\textbf{Bridge} & Layer 2 (Data Link) & Connecting network segments \\
\end{longtable}
}

\begin{itemize}
\tightlist
\item
  \textbf{Wire Twisting}: Each twist cancels out electromagnetic
  interference from adjacent wire
\item
  \textbf{Interference Cancellation}: Noise affects both wires equally
  but in opposite directions
\item
  \textbf{Router Function}: Makes routing decisions based on IP
  addresses
\item
  \textbf{Bridge Function}: Forwards frames based on MAC addresses
\end{itemize}

\end{solutionbox}
\begin{mnemonicbox}
``Twisted wires Reduce interference, Router at layer
3, Bridge at layer 2''

\end{mnemonicbox}
\subsection*{Question 5(c OR) [7
marks]}\label{question-5c-or-7-marks}

\textbf{Define Cyber Attack and Explain various cyber-attacks in brief}

\begin{solutionbox}

\textbf{Cyber Attack Definition:} Cyber attack is a deliberate attempt
to compromise computer systems, networks, or digital devices to steal,
alter, or destroy data.

\textbf{Types of Cyber Attacks:}

\begin{center}
\textbf{Mermaid Diagram (Code)}
\begin{verbatim}
{Shaded}
{Highlighting}[]
graph TD
    A[Cyber Attacks] {-{-}{} B[Malware]}
    A {-{-}{} C[Phishing]}
    A {-{-}{} D[DoS/DDoS]}
    A {-{-}{} E[Man{-}in{-}Middle]}
    A {-{-}{} F[SQL Injection]}
    
    B {-{-}{} G[Virus, Worm, Trojan]}
    C {-{-}{} H[Email, Website]}
    D {-{-}{} I[Traffic Flooding]}
    E {-{-}{} J[Eavesdropping]}
    F {-{-}{} K[Database Attack]}
{Highlighting}
{Shaded}
\end{verbatim}
\end{center}


{\def\LTcaptype{none} % do not increment counter
\vspace{-5pt}
\captionof{table}{Cyber Attack Types}
\vspace{-10pt}
\begin{longtable}[]{@{}
  >{\raggedright\arraybackslash}p{(\linewidth - 6\tabcolsep) * \real{0.2826}}
  >{\raggedright\arraybackslash}p{(\linewidth - 6\tabcolsep) * \real{0.2826}}
  >{\raggedright\arraybackslash}p{(\linewidth - 6\tabcolsep) * \real{0.1739}}
  >{\raggedright\arraybackslash}p{(\linewidth - 6\tabcolsep) * \real{0.2609}}@{}}
\toprule\noalign{}
\begin{minipage}[b]{\linewidth}\raggedright
Attack Type
\end{minipage} & \begin{minipage}[b]{\linewidth}\raggedright
Description
\end{minipage} & \begin{minipage}[b]{\linewidth}\raggedright
Impact
\end{minipage} & \begin{minipage}[b]{\linewidth}\raggedright
Prevention
\end{minipage} \\
\midrule\noalign{}
\endhead
\bottomrule\noalign{}
\endlastfoot
\textbf{Malware} & Malicious software (virus, worm, trojan) & System
corruption, data theft & Antivirus, updates \\
\textbf{Phishing} & Fraudulent emails/websites to steal credentials &
Identity theft, financial loss & User awareness, email filters \\
\textbf{DoS/DDoS} & Overwhelming target with traffic & Service
unavailability & Firewalls, load balancers \\
\textbf{Man-in-Middle} & Intercepting communication between parties &
Data eavesdropping & Encryption, secure protocols \\
\textbf{SQL Injection} & Malicious code inserted into database queries &
Database compromise & Input validation, parameterized queries \\
\end{longtable}
}

\textbf{Detailed Attack Explanations:}

\textbf{Malware Attacks:}

\begin{itemize}
\tightlist
\item
  \textbf{Virus}: Self-replicating code that attaches to files
\item
  \textbf{Worm}: Standalone malware that spreads across networks
\item
  \textbf{Trojan}: Disguised malware that appears legitimate
\end{itemize}

\textbf{Social Engineering:}

\begin{itemize}
\tightlist
\item
  \textbf{Phishing}: Fake emails requesting sensitive information
\item
  \textbf{Spear Phishing}: Targeted attacks on specific individuals
\item
  \textbf{Baiting}: Using attractive offers to deliver malware
\end{itemize}

\textbf{Network Attacks:}

\begin{itemize}
\tightlist
\item
  \textbf{Packet Sniffing}: Capturing network traffic for analysis
\item
  \textbf{Session Hijacking}: Taking over user sessions
\item
  \textbf{Password Attacks}: Brute force, dictionary attacks
\end{itemize}

\end{solutionbox}
\begin{mnemonicbox}
``MPDMS - Malware, Phishing, DoS, Man-in-middle, SQL
injection''

\end{mnemonicbox}

\end{document}
