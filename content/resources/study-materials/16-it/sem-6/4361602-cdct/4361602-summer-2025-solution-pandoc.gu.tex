\documentclass[10pt,a4paper]{article}

% content/resources/templates/preamble.tex
\usepackage[margin=0.6in]{geometry}
\author{Milav Dabgar}
\usepackage{amsmath,amssymb,amsthm}
\usepackage{booktabs}
\usepackage{multirow}
\usepackage{xcolor}
\usepackage{tcolorbox}
\tcbuselibrary{breakable,skins}
\usepackage[colorlinks=true,linkcolor=blue]{hyperref}
\usepackage{titlesec}
\usepackage{enumitem}
\usepackage{tikz}
\usepackage{pgfplots}
\usepackage{circuitikz}
\usepackage[version=4]{mhchem}
\usepackage{longtable}
\usepackage{array}
\usepackage{float}
\usepackage{caption}
\usepackage{listings}

\lstset{
  basicstyle=\small\ttfamily,
  breaklines=true,
  breakatwhitespace=false,
  postbreak=\mbox{\textcolor{red}{$\hookrightarrow$}\space},
  float=false,
  numbers=left,
  numberstyle=\tiny\color{gray},
  numbersep=10pt,
  xleftmargin=2em,
  keywordstyle=\color{blue},
  commentstyle=\color{green!60!black},
  stringstyle=\color{purple},
  backgroundcolor=\color{gray!5},
  showstringspaces=false,
  tabsize=2,
  captionpos=b,
  keepspaces=true,
  columns=flexible
}

\pgfplotsset{compat=1.18}
\usetikzlibrary{shapes,arrows,positioning,calc,patterns,decorations.pathmorphing,decorations.markings,arrows.meta}

% Color scheme
\definecolor{headcolor}{RGB}{0,102,204}
\definecolor{keycolor}{RGB}{220,20,60}
\definecolor{solutioncolor}{RGB}{34,139,34}
\definecolor{mnemoniccolor}{RGB}{148,0,211}
\definecolor{codecolor}{RGB}{0,0,100}

% Spacing
\setlength{\parskip}{3pt}
\setlist[itemize]{nosep}
\setlist[enumerate]{nosep}

% Title formatting
\titleformat{\section}{\Large\bfseries\color{headcolor}}{\thesection}{1em}{}
\titleformat{\subsection}{\large\bfseries\color{headcolor}}{\thesubsection}{1em}{}

% Pandoc tightlist compatibility
\providecommand{\tightlist}{%
  \setlength{\itemsep}{0pt}\setlength{\parskip}{0pt}}

% Pandoc longtable compatibility
\newcounter{none}
\def\thenone{}


% content/resources/templates/gujarati-boxes.tex
\usepackage{fontspec}
\usepackage{polyglossia}

% Set Gujarati as main language (document is primarily in Gujarati)
% Note: gloss-gujarati.ldf doesn't exist in polyglossia, but it will use hyphenation patterns
\setdefaultlanguage{gujarati}
\setotherlanguage{english}

% Configure Gujarati font properly
% Use Language=Default to prevent polyglossia from trying to add language-specific features
% that don't exist for Gujarati, which causes "empty feature" warnings
\newfontfamily\gujaratifont[Script=Gujarati,AutoFakeBold=2.5,AutoFakeSlant=0.3]{Noto Sans Gujarati}
\setmainfont[Script=Gujarati,AutoFakeBold=2.5,AutoFakeSlant=0.3]{Noto Sans Gujarati}
% Use Noto Sans Gujarati for monospace to support Gujarati in text
\setmonofont[Scale=0.9]{Noto Sans Gujarati}

% Configure English to use the same font
\newfontfamily\englishfont[Script=Gujarati,AutoFakeBold=2.5,AutoFakeSlant=0.3]{Noto Sans Gujarati}

% Translations for polyglossia
\gappto\captionsgujarati{
  \renewcommand{\tablename}{કોષ્ટક}
  \renewcommand{\figurename}{આકૃતિ}
}

% Helper for TikZ nodes to ensure Gujarati font
\newcommand{\gu}[1]{{\gujaratifont #1}}

% Custom environments
\newtcolorbox{solutionbox}{
    breakable,
    enhanced,
    colback=solutioncolor!5!white,
    colframe=solutioncolor!75!black,
    fonttitle=\bfseries,
    title=જવાબ
}

\newtcolorbox{solutionboxnobreak}{
 colback=solutioncolor!5!white,
 colframe=solutioncolor!75!black,
 fonttitle=\bfseries,
 title=જવાબ
}

\newtcolorbox{keyformula}{
 breakable,
 enhanced,
 colback=keycolor!5!white,
 colframe=keycolor!75!black,
 fonttitle=\bfseries,
 title=રાસાયણિક સમીકરણ/સૂત્ર
}

\newtcolorbox{mnemonicbox}{
 breakable,
 enhanced,
 colback=mnemoniccolor!5!white,
 colframe=mnemoniccolor!75!black,
 fonttitle=\bfseries,
 title=મેમરી ટ્રીક
}


\begin{document}

\begin{center}
{\Huge\bfseries\color{headcolor} Subject Name (Gujarati)}\\[5pt]
{\LARGE 4361602 -- Summer 2025}\\[3pt]
{\large Semester 1 Study Material}\\[3pt]
{\normalsize\textit{Detailed Solutions and Explanations}}
\end{center}

\vspace{10pt}

\subsection*{પ્રશ્ન 1(અ) [3
ગુણ]}\label{uxaaauxab0uxab6uxaa8-1uxa85-3-uxa97uxaa3}

\textbf{ક્લાઉડ કમ્પ્યુટિંગની વ્યાખ્યા આપો. ક્લાઉડ કમ્પ્યુટિંગના ઉપયોગો સમજાવો.}

\begin{solutionbox}

\textbf{ક્લાઉડ કમ્પ્યુટિંગ} એ ઇન્ટરનેટ (``ક્લાઉડ'') દ્વારા કમ્પ્યુટિંગ સેવાઓ જેવી કે
સર્વર, સ્ટોરેજ, ડેટાબેઝ, નેટવર્કિંગ, સોફ્ટવેર અને વિશ્લેષણની ડિલિવરી છે.

\textbf{ક્લાઉડ કમ્પ્યુટિંગના ઉપયોગો:}

{\def\LTcaptype{none} % do not increment counter
\begin{longtable}[]{@{}ll@{}}
\toprule\noalign{}
ઉપયોગ & વર્ણન \\
\midrule\noalign{}
\endhead
\bottomrule\noalign{}
\endlastfoot
\textbf{ડેટા સ્ટોરેજ} & ફાઇલો અને દસ્તાવેજો ઓનલાઇન સ્ટોર કરવા \\
\textbf{વેબ એપ્લિકેશન} & વેબ બ્રાઉઝર દ્વારા સોફ્ટવેર ચલાવવા \\
\textbf{ઇમેઇલ સેવાઓ} & Gmail, Outlook ક્લાઉડ પર હોસ્ટ કરવા \\
\textbf{બેકઅપ અને રિકવરી} & ઓટોમેટિક ડેટા બેકઅપ અને આપત્તિ પુનઃપ્રાપ્તિ \\
\end{longtable}
}

\end{solutionbox}
\begin{mnemonicbox}
``SWEB'' - Storage, Web apps, Email, Backup

\end{mnemonicbox}
\begin{center}\rule{0.5\linewidth}{0.5pt}\end{center}

\subsection*{પ્રશ્ન 1(બ) [4
ગુણ]}\label{uxaaauxab0uxab6uxaa8-1uxaac-4-uxa97uxaa3}

\textbf{ક્લાઉડ સ્ટોરેજ સોલ્યુશન શું છે? ઓબ્જેક્ટ સ્ટોરેજ વિગતે સમજાવો.}

\begin{solutionbox}

\textbf{ક્લાઉડ સ્ટોરેજ સોલ્યુશન} એ ઓનલાઇન સેવાઓ છે જે ઇન્ટરનેટ-કનેક્ટેડ ઉપકરણો દ્વારા
ડેટા સ્ટોરેજ, મેનેજમેન્ટ અને એક્સેસ પ્રદાન કરે છે.

\textbf{ઓબ્જેક્ટ સ્ટોરેજની વિગતો:}

{\def\LTcaptype{none} % do not increment counter
\begin{longtable}[]{@{}ll@{}}
\toprule\noalign{}
વિશેષતા & વર્ણન \\
\midrule\noalign{}
\endhead
\bottomrule\noalign{}
\endlastfoot
\textbf{સ્ટ્રક્ચર} & બકેટ/કન્ટેનરમાં ઓબ્જેક્ટ તરીકે ડેટા સ્ટોર કરે છે \\
\textbf{મેટાડેટા} & દરેક ઓબ્જેક્ટમાં ડેટા, મેટાડેટા અને યુનિક ID હોય છે \\
\textbf{સ્કેલેબિલિટી} & વર્ચ્યુઅલી અનલિમિટેડ સ્ટોરેજ ક્ષમતા \\
\textbf{એક્સેસ} & પ્રોગ્રામેટિક એક્સેસ માટે RESTful APIs \\
\end{longtable}
}

\textbf{ડાયાગ્રામ:}

\begin{verbatim}
┌─────────────────┐    ┌─────────────────┐    ┌─────────────────┐
│    Object 1     │    │    Object 2     │    │    Object 3     │
│                 │    │                 │    │                 │
│ Data + Metadata │    │ Data + Metadata │    │ Data + Metadata │
│ Unique ID: 001  │    │ Unique ID: 002  │    │ Unique ID: 003  │
└─────────────────┘    └─────────────────┘    └─────────────────┘
        │                       │                       │
        └───────────────────────┼───────────────────────┘
                                │
                    ┌─────────────────┐
                    │     Bucket      │
                    │   (Container)   │
                    └─────────────────┘
\end{verbatim}

\end{solutionbox}
\begin{mnemonicbox}
``SMAR'' - Scalable, Metadata-rich, API-accessible,
Resilient

\end{mnemonicbox}
\begin{center}\rule{0.5\linewidth}{0.5pt}\end{center}

\subsection*{પ્રશ્ન 1(ક) [7
ગુણ]}\label{uxaaauxab0uxab6uxaa8-1uxa95-7-uxa97uxaa3}

\textbf{હાર્ડવેર વર્ચ્યુઅલાઇઝેશન અને સોફ્ટવેર વર્ચ્યુઅલાઇઝેશન વિગતે સમજાવો.}

\begin{solutionbox}

\textbf{હાર્ડવેર વર્ચ્યુઅલાઇઝેશન:}

\begin{itemize}
\tightlist
\item
  \textbf{ભૌતિક સ્તર અમૂર્તીકરણ} જે ભૌતિક હાર્ડવેર ઘટકોના વર્ચ્યુઅલ વર્ઝન બનાવે છે
\item
  \textbf{હાઇપરવાઇઝર} એક જ ભૌતિક સર્વર પર બહુવિધ વર્ચ્યુઅલ મશીનોનું સંચાલન કરે છે
\end{itemize}

\textbf{સોફ્ટવેર વર્ચ્યુઅલાઇઝેશન:}

\begin{itemize}
\tightlist
\item
  \textbf{એપ્લિકેશન સ્તર અમૂર્તીકરણ} જે સોફ્ટવેરને અલગ વાતાવરણમાં ચલાવવાની મંજૂરી આપે
  છે
\item
  \textbf{રનટાઇમ વાતાવરણ} વિવિધ પ્લેટફોર્મ પર સુસંગતતા પ્રદાન કરે છે
\end{itemize}

\textbf{તુલના કોષ્ટક:}

{\def\LTcaptype{none} % do not increment counter
\begin{longtable}[]{@{}lll@{}}
\toprule\noalign{}
પાસું & હાર્ડવેર વર્ચ્યુઅલાઇઝેશન & સોફ્ટવેર વર્ચ્યુઅલાઇઝેશન \\
\midrule\noalign{}
\endhead
\bottomrule\noalign{}
\endlastfoot
\textbf{સ્તર} & હાર્ડવેર/OS સ્તર & એપ્લિકેશન સ્તર \\
\textbf{પ્રદર્શન} & મૂળ જેવું & થોડું ઓવરહેડ \\
\textbf{રિસોર્સ ઉપયોગ} & ઊંચો & મધ્યમ \\
\textbf{આઇસોલેશન} & સંપૂર્ણ & એપ્લિકેશન-વિશિષ્ટ \\
\end{longtable}
}

\textbf{આર્કિટેક્ચર ડાયાગ્રામ:}

\begin{verbatim}
graph TB
    A[Physical Hardware] {-{-} B[Hypervisor]}
    B {-{-} C[VM1 {-} OS + Apps]}
    B {-{-} D[VM2 {-} OS + Apps]}
    B {-{-} E[VM3 {-} OS + Apps]}
    
    F[Host OS] {-{-} G[Software Virtualization Layer]}
    G {-{-} H[App Container 1]}
    G {-{-} I[App Container 2]}
    G {-{-} J[App Container 3]}
\end{verbatim}

\end{solutionbox}
\begin{mnemonicbox}
``HAPI'' - Hardware abstraction, Application
isolation, Performance consideration, Infrastructure management

\end{mnemonicbox}
\begin{center}\rule{0.5\linewidth}{0.5pt}\end{center}

\subsection*{પ્રશ્ન 1(ક) OR [7
ગુણ]}\label{uxaaauxab0uxab6uxaa8-1uxa95-or-7-uxa97uxaa3}

\textbf{ક્લાઉડ વર્ચ્યુઅલાઇઝેશન શું છે? વર્ચ્યુઅલાઇઝેશનની લાક્ષણિકતાઓ સમજાવો.}

\begin{solutionbox}

\textbf{ક્લાઉડ વર્ચ્યુઅલાઇઝેશન} એ ક્લાઉડ વાતાવરણમાં ગતિશીલ રીતે ફાળવી અને સંચાલિત
કરી શકાય તેવા કમ્પ્યુટિંગ રિસોર્સ (સર્વર, સ્ટોરેજ, નેટવર્ક)ના વર્ચ્યુઅલ વર્ઝન બનાવવાની
પ્રક્રિયા છે.

\textbf{વર્ચ્યુઅલાઇઝેશનની લાક્ષણિકતાઓ:}

{\def\LTcaptype{none} % do not increment counter
\begin{longtable}[]{@{}ll@{}}
\toprule\noalign{}
લાક્ષણિકતા & વર્ણન \\
\midrule\noalign{}
\endhead
\bottomrule\noalign{}
\endlastfoot
\textbf{રિસોર્સ પુલિંગ} & બહુવિધ ભૌતિક રિસોર્સને પુલમાં જોડવા \\
\textbf{આઇસોલેશન} & વર્ચ્યુઅલ મશીનો સ્વતંત્ર રીતે કામ કરે છે \\
\textbf{લાસ્ટિસિટી} & માંગ પર આધારિત ગતિશીલ સ્કેલિંગ \\
\textbf{કાર્યક્ષમતા} & બહેતર હાર્ડવેર ઉપયોગ \\
\end{longtable}
}

\textbf{ફાયદાઓ:}

\begin{itemize}
\tightlist
\item
  હાર્ડવેર એકીકરણ દ્વારા \textbf{ખર્ચમાં ઘટાડો}
\item
  રિસોર્સ ફાળવણીમાં \textbf{લવચીકતા}
\item
  વધતી માંગ માટે \textbf{સ્કેલેબિલિટી}
\item
  કેન્દ્રીકરણ દ્વારા સરળીકૃત \textbf{મેનેજમેન્ટ}
\end{itemize}

\textbf{વર્ચ્યુઅલાઇઝેશન સ્ટેક:}

\begin{verbatim}
graph BT
    A[Physical Hardware] {-{-} B[Hypervisor/VMM]}
    B {-{-} C[Virtual Machine 1]}
    B {-{-} D[Virtual Machine 2]}
    B {-{-} E[Virtual Machine 3]}
    C {-{-} F[Guest OS 1]}
    D {-{-} G[Guest OS 2]}
    E {-{-} H[Guest OS 3]}
\end{verbatim}

\end{solutionbox}
\begin{mnemonicbox}
``RIEM'' - Resource pooling, Isolation, Elasticity,
Management

\end{mnemonicbox}
\begin{center}\rule{0.5\linewidth}{0.5pt}\end{center}

\subsection*{પ્રશ્ન 2(અ) [3
ગુણ]}\label{uxaaauxab0uxab6uxaa8-2uxa85-3-uxa97uxaa3}

\textbf{ક્લાઉડ સિક્યુરિટી ચેલેન્જીસ કયાં છે?}

\begin{solutionbox}

\textbf{ક્લાઉડ સિક્યુરિટી ચેલેન્જીસ:}

{\def\LTcaptype{none} % do not increment counter
\begin{longtable}[]{@{}ll@{}}
\toprule\noalign{}
ચેલેન્જ & વર્ણન \\
\midrule\noalign{}
\endhead
\bottomrule\noalign{}
\endlastfoot
\textbf{ડેટા બ્રીચ} & સંવેદનશીલ માહિતીની અનધિકૃત ઍક્સેસ \\
\textbf{ઍક્સેસ મેનેજમેન્ટ} & યુઝર પરમિશન અને ઓથેન્ટિકેશન નિયંત્રણ \\
\textbf{કોમ્પ્લાયન્સ} & નિયમનકારી અને ઉદ્યોગ ધોરણો પૂરા કરવા \\
\textbf{વેન્ડર લોક-ઇન} & ચોક્કસ ક્લાઉડ પ્રોવાઇડર પર નિર્ભરતા \\
\end{longtable}
}

\end{solutionbox}
\begin{mnemonicbox}
``DACV'' - Data breaches, Access control,
Compliance, Vendor dependency

\end{mnemonicbox}
\begin{center}\rule{0.5\linewidth}{0.5pt}\end{center}

\subsection*{પ્રશ્ન 2(બ) [4
ગુણ]}\label{uxaaauxab0uxab6uxaa8-2uxaac-4-uxa97uxaa3}

\textbf{IaaS વિગતે સમજાવો.}

\begin{solutionbox}

\textbf{Infrastructure as a Service (IaaS)} ઇન્ટરનેટ પર વર્ચ્યુઅલાઇઝ્ડ કમ્પ્યુટિંગ
ઇન્ફ્રાસ્ટ્રક્ચર પ્રદાન કરે છે, જેમાં સર્વર, સ્ટોરેજ અને નેટવર્કિંગ શામેલ છે.

\textbf{IaaS ઘટકો:}

{\def\LTcaptype{none} % do not increment counter
\begin{longtable}[]{@{}ll@{}}
\toprule\noalign{}
ઘટક & વર્ણન \\
\midrule\noalign{}
\endhead
\bottomrule\noalign{}
\endlastfoot
\textbf{કમ્પ્યુટ} & વર્ચ્યુઅલ મશીનો અને પ્રોસેસિંગ પાવર \\
\textbf{સ્ટોરેજ} & બ્લોક, ફાઇલ અને ઓબ્જેક્ટ સ્ટોરેજ \\
\textbf{નેટવર્કિંગ} & વર્ચ્યુઅલ નેટવર્ક, લોડ બેલેન્સર, ફાયરવૉલ \\
\textbf{મેનેજમેન્ટ} & મોનિટરિંગ, સિક્યુરિટી અને બેકઅપ ટૂલ્સ \\
\end{longtable}
}

\textbf{IaaS આર્કિટેક્ચર:}

\begin{verbatim}
graph TB
    A[User/Customer] {-{-} B[IaaS Management Portal]}
    B {-{-} C[Compute Resources]}
    B {-{-} D[Storage Resources]}
    B {-{-} E[Network Resources]}
    C {-{-} F[Physical Servers]}
    D {-{-} G[Storage Arrays]}
    E {-{-} H[Network Infrastructure]}
\end{verbatim}

\textbf{ફાયદાઓ:}

\begin{itemize}
\tightlist
\item
  \textbf{પે-પ્રર-યુઝ} પ્રાઇસિંગ મોડલ
\item
  માંગ પર \textbf{સ્કેલેબિલિટી}
\item
  \textbf{ઘટેલા} મૂડી ખર્ચ
\end{itemize}

\end{solutionbox}
\begin{mnemonicbox}
``CSNM'' - Compute, Storage, Network, Management

\end{mnemonicbox}
\begin{center}\rule{0.5\linewidth}{0.5pt}\end{center}

\subsection*{પ્રશ્ન 2(ક) [7
ગુણ]}\label{uxaaauxab0uxab6uxaa8-2uxa95-7-uxa97uxaa3}

\textbf{Identity and access management વિગતે સમજાવો.}

\begin{solutionbox}

\textbf{Identity and Access Management (IAM)} એ ક્લાઉડ વાતાવરણમાં ડિજિટલ
ઓળખ અને રિસોર્સની ઍક્સેસ નિયંત્રિત કરવા માટેનું ફ્રેમવર્ક છે.

\textbf{IAM ઘટકો:}

{\def\LTcaptype{none} % do not increment counter
\begin{longtable}[]{@{}ll@{}}
\toprule\noalign{}
ઘટક & કાર્ય \\
\midrule\noalign{}
\endhead
\bottomrule\noalign{}
\endlastfoot
\textbf{ઓથેન્ટિકેશન} & યુઝર ઓળખ ચકાસવી \\
\textbf{ઓથરાઇઝેશન} & ઍક્સેસ પરમિશન નક્કી કરવી \\
\textbf{યુઝર મેનેજમેન્ટ} & યુઝર એકાઉન્ટ બનાવવા, બદલવા, ડિલીટ કરવા \\
\textbf{રોલ-બેઝ્ડ ઍક્સેસ} & ભૂમિકા પર આધારિત પરમિશન આપવી \\
\end{longtable}
}

\textbf{IAM પ્રોસેસ ફ્લો:}

\begin{center}
\textbf{Mermaid Diagram (Code)}
\begin{verbatim}
{Shaded}
{Highlighting}[]
graph LR
    A[User Request] {-{-}{} B[Authentication]}
    B {-{-}{} C\{Valid Identity?\}}
    C {-{-}{}|Yes| D[Authorization Check]}
    C {-{-}{}|No| E[Access Denied]}
    D {-{-}{} F\{Permission Granted?\}}
    F {-{-}{}|Yes| G[Resource Access]}
    F {-{-}{}|No| H[Access Denied]}
{Highlighting}
{Shaded}
\end{verbatim}
\end{center}

\textbf{મુખ્ય વિશેષતાઓ:}

\begin{itemize}
\tightlist
\item
  સીમલેસ ઍક્સેસ માટે \textbf{Single Sign-On (SSO)}
\item
  વધારેલી સુરક્ષા માટે \textbf{Multi-Factor Authentication (MFA)}
\item
  ઍક્સેસ નિયંત્રણ માટે \textbf{પોલિસી મેનેજમેન્ટ}
\item
  કોમ્પ્લાયન્સ ટ્રેકિંગ માટે \textbf{ઓડિટ લોગિંગ}
\end{itemize}

\textbf{સુરક્ષા ફાયદાઓ:}

\begin{itemize}
\tightlist
\item
  \textbf{કેન્દ્રીકૃત} ઓળખ મેનેજમેન્ટ
\item
  \textbf{ઘટેલા} સુરક્ષા જોખમો
\item
  નિયમોનું \textbf{કોમ્પ્લાયન્સ}
\item
  \textbf{સુધારેલ} યુઝર અનુભવ
\end{itemize}

\end{solutionbox}
\begin{mnemonicbox}
``AURU'' - Authentication, Authorization, User
management, Role-based access

\end{mnemonicbox}
\begin{center}\rule{0.5\linewidth}{0.5pt}\end{center}

\subsection*{પ્રશ્ન 2(અ) OR [3
ગુણ]}\label{uxaaauxab0uxab6uxaa8-2uxa85-or-3-uxa97uxaa3}

\textbf{ક્લાઉડમાં Access control અને authentication ની જરૂરિયાત.}

\begin{solutionbox}

\textbf{Access Control અને Authentication ની જરૂરિયાત:}

{\def\LTcaptype{none} % do not increment counter
\begin{longtable}[]{@{}ll@{}}
\toprule\noalign{}
જરૂરિયાત & કારણ \\
\midrule\noalign{}
\endhead
\bottomrule\noalign{}
\endlastfoot
\textbf{ડેટા પ્રોટેક્શન} & સંવેદનશીલ ડેટાની અનધિકૃત ઍક્સેસ અટકાવવા \\
\textbf{રેગ્યુલેટરી કોમ્પ્લાયન્સ} & કાનૂની અને ઉદ્યોગ આવશ્યકતાઓ પૂરી કરવા \\
\textbf{રિસોર્સ સિક્યુરિટી} & કોણ ક્લાઉડ રિસોર્સ વાપરી શકે તે નિયંત્રિત કરવા \\
\textbf{કોસ્ટ મેનેજમેન્ટ} & અનધિકૃત રિસોર્સ વપરાશ અટકાવવા \\
\end{longtable}
}

\end{solutionbox}
\begin{mnemonicbox}
``DRRC'' - Data protection, Regulatory compliance,
Resource security, Cost management

\end{mnemonicbox}
\begin{center}\rule{0.5\linewidth}{0.5pt}\end{center}

\subsection*{પ્રશ્ન 2(બ) OR [4
ગુણ]}\label{uxaaauxab0uxab6uxaa8-2uxaac-or-4-uxa97uxaa3}

\textbf{PaaS વિગતે સમજાવો.}

\begin{solutionbox}

\textbf{Platform as a Service (PaaS)} એ ક્લાઉડ-બેઝ્ડ પ્લેટફોર્મ છે જે ગ્રાહકોને
અંતર્ગત ઇન્ફ્રાસ્ટ્રક્ચર સાથે વ્યવહાર કર્યા વગર એપ્લિકેશન ડેવલપ, ચલાવવા અને મેનેજ કરવાની
મંજૂરી આપે છે.

\textbf{PaaS ઘટકો:}

{\def\LTcaptype{none} % do not increment counter
\begin{longtable}[]{@{}ll@{}}
\toprule\noalign{}
ઘટક & વર્ણન \\
\midrule\noalign{}
\endhead
\bottomrule\noalign{}
\endlastfoot
\textbf{ડેવલપમેન્ટ ટૂલ્સ} & IDEs, debuggers, compilers \\
\textbf{રનટાઇમ એન્વાયરનમેન્ટ} & એપ્લિકેશન એક્ઝિક્યુશન પ્લેટફોર્મ \\
\textbf{ડેટાબેઝ મેનેજમેન્ટ} & બિલ્ટ-ઇન ડેટાબેઝ સેવાઓ \\
\textbf{મિડલવેર} & ઇન્ટિગ્રેશન અને કોમ્યુનિકેશન સેવાઓ \\
\end{longtable}
}

\textbf{PaaS આર્કિટેક્ચર:}

\begin{verbatim}
graph TB
    A[Applications] {-{-} B[PaaS Platform]}
    B {-{-} C[Development Tools]}
    B {-{-} D[Runtime Environment]}
    B {-{-} E[Database Services]}
    B {-{-} F[Middleware]}
    F {-{-} G[IaaS Infrastructure]}
\end{verbatim}

\textbf{ફાયદાઓ:}

\begin{itemize}
\tightlist
\item
  \textbf{ઝડપી} એપ્લિકેશન ડેવલપમેન્ટ
\item
  \textbf{ઘટેલી} જટિલતા
\item
  \textbf{બિલ્ટ-ઇન} સ્કેલેબિલિટી
\end{itemize}

\end{solutionbox}
\begin{mnemonicbox}
``DRDM'' - Development tools, Runtime, Database,
Middleware

\end{mnemonicbox}
\begin{center}\rule{0.5\linewidth}{0.5pt}\end{center}

\subsection*{પ્રશ્ન 2(ક) OR [7
ગુણ]}\label{uxaaauxab0uxab6uxaa8-2uxa95-or-7-uxa97uxaa3}

\textbf{DevSecOps વિગતે સમજાવો.}

\begin{solutionbox}

\textbf{DevSecOps} એ DevOps પ્રક્રિયામાં સિક્યુરિટી પ્રેક્ટિસ ઇન્ટિગ્રેટ કરે છે, જે
સમગ્ર ડેવલપમેન્ટ લાઇફસાઇકલ દરમિયાન સિક્યુરિટીને સહેજ જવાબદારી બનાવે છે.

\textbf{DevSecOps સિદ્ધાંતો:}

{\def\LTcaptype{none} % do not increment counter
\begin{longtable}[]{@{}ll@{}}
\toprule\noalign{}
સિદ્ધાંત & વર્ણન \\
\midrule\noalign{}
\endhead
\bottomrule\noalign{}
\endlastfoot
\textbf{Shift Left} & ડેવલપમેન્ટમાં વહેલી સિક્યુરિટી ઇન્ટિગ્રેટ કરવી \\
\textbf{ઓટોમેશન} & ઓટોમેટેડ સિક્યુરિટી ટેસ્ટિંગ અને કોમ્પ્લાયન્સ \\
\textbf{કોલેબોરેશન} & સિક્યુરિટી ટીમો ડેવલપમેન્ટ અને ઓપરેશન સાથે કામ કરે છે \\
\textbf{સતત મોનિટરિંગ} & ચાલુ સિક્યુરિટી મૂલ્યાંકન \\
\end{longtable}
}

\textbf{DevSecOps પાઇપલાઇન:}

\begin{center}
\textbf{Mermaid Diagram (Code)}
\begin{verbatim}
{Shaded}
{Highlighting}[]
graph LR
    A[Plan] {-{-}{} B[Code]}
    B {-{-}{} C[Build + Security Scan]}
    C {-{-}{} D[Test + Security Test]}
    D {-{-}{} E[Deploy + Security Config]}
    E {-{-}{} F[Monitor + Security Monitor]}
    F {-{-}{} A}
{Highlighting}
{Shaded}
\end{verbatim}
\end{center}

\textbf{સિક્યુરિટી ઇન્ટિગ્રેશન પોઇન્ટ્સ:}

\begin{itemize}
\tightlist
\item
  ડેવલપમેન્ટ દરમિયાન \textbf{કોડ એનાલિસિસ}
\item
  CI/CD પાઇપલાઇનમાં \textbf{વલ્નરેબિલિટી સ્કેનિંગ}
\item
  ડિપ્લોયમેન્ટ પહેલાં \textbf{કોમ્પ્લાયન્સ ચેક}
\item
  પ્રોડક્શનમાં \textbf{રનટાઇમ પ્રોટેક્શન}
\end{itemize}

\textbf{ફાયદાઓ:}

\begin{itemize}
\tightlist
\item
  \textbf{વહેલી} વલ્નરેબિલિટી ડિટેક્શન
\item
  \textbf{ઝડપી} સિક્યુરિટી ફિક્સ
\item
  \textbf{ઘટેલો} સિક્યુરિટી ડેટ
\item
  \textbf{સુધારેલ} કોમ્પ્લાયન્સ
\end{itemize}

\end{solutionbox}
\begin{mnemonicbox}
``SACM'' - Shift left, Automation, Collaboration,
Monitoring

\end{mnemonicbox}
\begin{center}\rule{0.5\linewidth}{0.5pt}\end{center}

\subsection*{પ્રશ્ન 3(અ) [3
ગુણ]}\label{uxaaauxab0uxab6uxaa8-3uxa85-3-uxa97uxaa3}

\textbf{Edge Computing મહત્વનું કેમ છે?}

\begin{solutionbox}

\textbf{Edge Computing નું મહત્વ:}

{\def\LTcaptype{none} % do not increment counter
\begin{longtable}[]{@{}ll@{}}
\toprule\noalign{}
ફાયદો & વર્ણન \\
\midrule\noalign{}
\endhead
\bottomrule\noalign{}
\endlastfoot
\textbf{ઘટાડેલ લેટન્સી} & સ્રોતની નજીક ડેટા પ્રોસેસિંગ \\
\textbf{બેન્ડવિડ્થ ઓપ્ટિમાઇઝેશન} & ક્લાઉડ પર ઓછા ડેટા ટ્રાન્સમિશન \\
\textbf{રિયલ-ટાઇમ પ્રોસેસિંગ} & ક્રિટિકલ એપ્લિકેશન માટે તત્કાલ પ્રતિસાદ \\
\textbf{ડેટા પ્રાઇવેસી} & સ્થાનિક પ્રોસેસિંગ સંવેદનશીલ ડેટાને સ્થાનિક રાખે છે \\
\end{longtable}
}

\end{solutionbox}
\begin{mnemonicbox}
``RBRD'' - Reduced latency, Bandwidth optimization,
Real-time processing, Data privacy

\end{mnemonicbox}
\begin{center}\rule{0.5\linewidth}{0.5pt}\end{center}

\subsection*{પ્રશ્ન 3(બ) [4
ગુણ]}\label{uxaaauxab0uxab6uxaa8-3uxaac-4-uxa97uxaa3}

\textbf{ડેટા સેન્ટર વ્યાખ્યાયિત કરો. ડેટા સેન્ટરના પ્રકારોની યાદી આપો. કોઈ એક
સમજાવો.}

\begin{solutionbox}

\textbf{ડેટા સેન્ટર} એ IT ઓપરેશન માટે કમ્પ્યુટર સિસ્ટમ, સ્ટોરેજ સિસ્ટમ, નેટવર્કિંગ
સાધનો અને સહાયક ઇન્ફ્રાસ્ટ્રક્ચર રાખતી સુવિધા છે.

\textbf{ડેટા સેન્ટરના પ્રકારો:}

{\def\LTcaptype{none} % do not increment counter
\begin{longtable}[]{@{}ll@{}}
\toprule\noalign{}
પ્રકાર & વર્ણન \\
\midrule\noalign{}
\endhead
\bottomrule\noalign{}
\endlastfoot
\textbf{એન્ટરપ્રાઇઝ} & સંસ્થાઓ દ્વારા માલિકી ધરાવતા ખાનગી ડેટા સેન્ટર \\
\textbf{કોલોકેશન} & બહુવિધ ભાડૂતોને જગ્યા ભાડે આપતી સહેજ સુવિધા \\
\textbf{હાઇપરસ્કેલ} & ક્લાઉડ પ્રદાતાઓ માટે મોટા પાયે સુવિધાઓ \\
\textbf{એજ} & અંતિમ વપરાશકર્તાઓની નજીક નાની સુવિધાઓ \\
\end{longtable}
}

\textbf{એન્ટરપ્રાઇઝ ડેટા સેન્ટર (વિગતવાર):}

\begin{itemize}
\tightlist
\item
  ઇન્ફ્રાસ્ટ્રક્ચર પર \textbf{સંપૂર્ણ નિયંત્રણ}
\item
  સંસ્થાની જરૂરિયાતો માટે \textbf{કસ્ટમાઇઝ્ડ}
\item
  \textbf{ઉચ્ચ સુરક્ષા} અને કોમ્પ્લાયન્સ
\item
  \textbf{નોંધપાત્ર} મૂડી રોકાણ જરૂરી
\end{itemize}

\textbf{ડેટા સેન્ટર આર્કિટેક્ચર:}

\begin{verbatim}
┌─────────────────────────────────────────┐
│              Data Center                │
│  ┌─────────┐  ┌─────────┐  ┌─────────┐  │
│  │ Server  │  │ Storage │  │ Network │  │
│  │  Racks  │  │  Systems│  │ Equip.  │  │
│  └─────────┘  └─────────┘  └─────────┘  │
│  ┌─────────────────────────────────────┐ │
│  │     Power \& Cooling Systems         │ │
│  └─────────────────────────────────────┘ │
└─────────────────────────────────────────┘
\end{verbatim}

\end{solutionbox}
\begin{mnemonicbox}
``ECHE'' - Enterprise, Colocation, Hyperscale, Edge

\end{mnemonicbox}
\begin{center}\rule{0.5\linewidth}{0.5pt}\end{center}

\subsection*{પ્રશ્ન 3(ક) [7
ગુણ]}\label{uxaaauxab0uxab6uxaa8-3uxa95-7-uxa97uxaa3}

\textbf{ક્લાઉડ ડેટાબેઝના પ્રકારો વિગતે સમજાવો.}

\begin{solutionbox}

\textbf{ક્લાઉડ ડેટાબેઝના પ્રકારો:}

\textbf{1. SQL ડેટાબેઝ (રિલેશનલ):}

\begin{itemize}
\tightlist
\item
  \textbf{સ્ટ્રક્ચર:} પૂર્વ-નિર્ધારિત સ્કીમા સાથે ટેબલ-આધારિત
\item
  \textbf{ACID ગુણધર્મો:} ડેટા સુસંગતતા સુનિશ્ચિત કરે છે
\item
  \textbf{ઉદાહરણો:} Amazon RDS, Google Cloud SQL
\end{itemize}

\textbf{2. NoSQL ડેટાબેઝ:}

{\def\LTcaptype{none} % do not increment counter
\begin{longtable}[]{@{}lll@{}}
\toprule\noalign{}
NoSQL પ્રકાર & વર્ણન & ઉપયોગ કેસ \\
\midrule\noalign{}
\endhead
\bottomrule\noalign{}
\endlastfoot
\textbf{ડોક્યુમેન્ટ} & JSON જેવા દસ્તાવેજો & કન્ટેન્ટ મેનેજમેન્ટ, કેટાલોગ \\
\textbf{કી-વેલ્યુ} & સરળ કી-વેલ્યુ જોડી & સેશન મેનેજમેન્ટ, કેશિંગ \\
\textbf{કોલમ-ફેમિલી} & વાઇડ કોલમ સ્ટોરેજ & એનાલિટિક્સ, ટાઇમ-સીરીઝ ડેટા \\
\textbf{ગ્રાફ} & નોડ્સ અને સંબંધો & સોશિયલ નેટવર્ક, રેકમેન્ડેશન \\
\end{longtable}
}

\textbf{ડેટાબેઝ તુલના:}

\begin{verbatim}
graph TB
    A[Cloud Databases] {-{-} B[SQL/Relational]}
    A {-{-} C[NoSQL]}
    B {-{-} D[MySQL, PostgreSQL]}
    C {-{-} E[Document {-} MongoDB]}
    C {-{-} F[Key{-}Value {-} Redis]}
    C {-{-} G[Column {-} Cassandra]}
    C {-{-} H[Graph {-} Neo4j]}
\end{verbatim}

\textbf{પસંદગીના માપદંડો:}

\begin{itemize}
\tightlist
\item
  \textbf{ડેટા સ્ટ્રક્ચર} આવશ્યકતાઓ
\item
  \textbf{સ્કેલેબિલિટી} જરૂરિયાતો
\item
  \textbf{કોન્સિસ્ટન્સી} આવશ્યકતાઓ
\item
  \textbf{પ્રદર્શન} અપેક્ષાઓ
\end{itemize}

\textbf{ફાયદાઓ:}

\begin{itemize}
\tightlist
\item
  \textbf{મેનેજ્ડ} સેવાઓ ઓપરેશનલ ઓવરહેડ ઘટાડે છે
\item
  \textbf{ઓટોમેટિક} સ્કેલિંગ અને બેકઅપ
\item
  \textbf{ગ્લોબલ} ડિસ્ટ્રિબ્યુશન ક્ષમતાઓ
\item
  \textbf{કોસ્ટ-ઇફેક્ટિવ} પે-પર-યુઝ મોડલ
\end{itemize}

\end{solutionbox}
\begin{mnemonicbox}
``DKCG'' - Document, Key-value, Column-family, Graph

\end{mnemonicbox}
\begin{center}\rule{0.5\linewidth}{0.5pt}\end{center}

\subsection*{પ્રશ્ન 3(અ) OR [3
ગુણ]}\label{uxaaauxab0uxab6uxaa8-3uxa85-or-3-uxa97uxaa3}

\textbf{ક્લાઉડ કમ્પ્યુટિંગમાં મશીન લર્નિંગની ભૂમિકા શું છે? તે સમજાવો.}

\begin{solutionbox}

\textbf{ક્લાઉડ કમ્પ્યુટિંગમાં મશીન લર્નિંગની ભૂમિકા:}

{\def\LTcaptype{none} % do not increment counter
\begin{longtable}[]{@{}ll@{}}
\toprule\noalign{}
ભૂમિકા & વર્ણન \\
\midrule\noalign{}
\endhead
\bottomrule\noalign{}
\endlastfoot
\textbf{રિસોર્સ ઓપ્ટિમાઇઝેશન} & રિસોર્સ ફાળવણીની આગાહી અને ઓપ્ટિમાઇઝેશન \\
\textbf{સિક્યુરિટી એન્હાન્સમેન્ટ} & અસામાન્યતા અને ધમકીઓ શોધવા \\
\textbf{કોસ્ટ મેનેજમેન્ટ} & ખર્ચ અને વપરાશ પેટર્ન ઓપ્ટિમાઇઝ કરવા \\
\textbf{પ્રદર્શન મોનિટરિંગ} & સિસ્ટમ નિષ્ફળતાની આગાહી અને અટકાવવી \\
\end{longtable}
}

\end{solutionbox}
\begin{mnemonicbox}
``RSCP'' - Resource optimization, Security
enhancement, Cost management, Performance monitoring

\end{mnemonicbox}
\begin{center}\rule{0.5\linewidth}{0.5pt}\end{center}

\subsection*{પ્રશ્ન 3(બ) OR [4
ગુણ]}\label{uxaaauxab0uxab6uxaa8-3uxaac-or-4-uxa97uxaa3}

\textbf{ક્લાઉડ સ્કેલેબિલિટી શું છે? વિગતે સમજાવો.}

\begin{solutionbox}

\textbf{ક્લાઉડ સ્કેલેબિલિટી} એ પ્રદર્શનને અસર કર્યા વગર માંગ પર આધારિત કમ્પ્યુટિંગ
રિસોર્સ ગતિશીલ રીતે વધારવા અથવા ઘટાડવાની ક્ષમતા છે.

\textbf{સ્કેલેબિલિટી પ્રકારો:}

{\def\LTcaptype{none} % do not increment counter
\begin{longtable}[]{@{}
  >{\raggedright\arraybackslash}p{(\linewidth - 4\tabcolsep) * \real{0.2727}}
  >{\raggedright\arraybackslash}p{(\linewidth - 4\tabcolsep) * \real{0.3636}}
  >{\raggedright\arraybackslash}p{(\linewidth - 4\tabcolsep) * \real{0.3636}}@{}}
\toprule\noalign{}
\begin{minipage}[b]{\linewidth}\raggedright
પ્રકાર
\end{minipage} & \begin{minipage}[b]{\linewidth}\raggedright
વર્ણન
\end{minipage} & \begin{minipage}[b]{\linewidth}\raggedright
પદ્ધતિ
\end{minipage} \\
\midrule\noalign{}
\endhead
\bottomrule\noalign{}
\endlastfoot
\textbf{વર્ટિકલ (સ્કેલ અપ)} & હાલના મશીનમાં વધુ પાવર ઉમેરવો & CPU, RAM, સ્ટોરેજ
અપગ્રેડ \\
\textbf{હોરિઝોન્ટલ (સ્કેલ આઉટ)} & રિસોર્સ પુલમાં વધુ મશીનો ઉમેરવા & લોડ
ડિસ્ટ્રિબ્યુશન \\
\end{longtable}
}

\textbf{સ્કેલેબિલિટી પ્રક્રિયા:}

\begin{center}
\textbf{Mermaid Diagram (Code)}
\begin{verbatim}
{Shaded}
{Highlighting}[]
graph LR
    A[Monitor Load] {-{-}{} B\{High Load?\}}
    B {-{-}{}|Yes| C[Scale Out/Up]}
    B {-{-}{}|No| D\{Low Load?\}}
    D {-{-}{}|Yes| E[Scale In/Down]}
    D {-{-}{}|No| A}
    C {-{-}{} A}
    E {-{-}{} A}
{Highlighting}
{Shaded}
\end{verbatim}
\end{center}

\textbf{ફાયદાઓ:}

\begin{itemize}
\tightlist
\item
  ગતિશીલ રિસોર્સ ફાળવણી દ્વારા \textbf{કોસ્ટ કાર્યક્ષમતા}
\item
  પીક લોડ દરમિયાન \textbf{પ્રદર્શન} જાળવણી
\item
  \textbf{ઉપલબ્ધતા} સુધારો
\end{itemize}

\end{solutionbox}
\begin{mnemonicbox}
``VH'' - Vertical scaling, Horizontal scaling

\end{mnemonicbox}
\begin{center}\rule{0.5\linewidth}{0.5pt}\end{center}

\subsection*{પ્રશ્ન 3(ક) OR [7
ગુણ]}\label{uxaaauxab0uxab6uxaa8-3uxa95-or-7-uxa97uxaa3}

\textbf{ડેટા કોન્સિસ્ટન્સી અને ડ્યુરેબિલિટી વિગતે સમજાવો.}

\begin{solutionbox}

\textbf{ડેટા કોન્સિસ્ટન્સી} એ ખાતરી કરે છે કે બધા નોડ્સ વિતરિત સિસ્ટમમાં એક જ સમયે
સમાન ડેટા જુએ.

\textbf{ડેટા ડ્યુરેબિલિટી} એ સિસ્ટમ નિષ્ફળતાના કિસ્સામાં પણ ડેટા પર્સિસ્ટન્સની ગેરંટી
આપે છે.

\textbf{કોન્સિસ્ટન્સી મોડલ્સ:}

{\def\LTcaptype{none} % do not increment counter
\begin{longtable}[]{@{}lll@{}}
\toprule\noalign{}
મોડલ & વર્ણન & ઉપયોગ કેસ \\
\midrule\noalign{}
\endhead
\bottomrule\noalign{}
\endlastfoot
\textbf{સ્ટ્રોંગ} & બધા રીડ્સ સૌથી તાજેતરના લેખન મેળવે છે & ફાઇનાન્શિયલ સિસ્ટમ \\
\textbf{ઇવેન્ચ્યુઅલ} & સમય સાથે સિસ્ટમ કોન્સિસ્ટન્ટ બને છે & સોશિયલ મીડિયા \\
\textbf{વીક} & કોન્સિસ્ટન્સી ક્યારે થશે તેની કોઈ ગેરંટી નથી & ગેમિંગ, રિયલ-ટાઇમ \\
\end{longtable}
}

\textbf{ડ્યુરેબિલિટી મેકેનિઝમ્સ:}

{\def\LTcaptype{none} % do not increment counter
\begin{longtable}[]{@{}ll@{}}
\toprule\noalign{}
મેકેનિઝમ & વર્ણન \\
\midrule\noalign{}
\endhead
\bottomrule\noalign{}
\endlastfoot
\textbf{રેપ્લિકેશન} & વિવિધ સ્થાનોમાં બહુવિધ કોપીઝ \\
\textbf{બેકઅપ} & નિયમિત ડેટા સ્નેપશોટ \\
\textbf{રિડન્ડન્સી} & RAID, erasure coding \\
\textbf{વર્ઝનિંગ} & ડેટાના બહુવિધ વર્ઝન \\
\end{longtable}
}

\textbf{CAP થિયોરમ:}

\begin{verbatim}
graph TB
    A[CAP Theorem] {-{-} B[Consistency]}
    A {-{-} C[Availability]}
    A {-{-} D[Partition Tolerance]}
    E[Note: Can only guarantee 2 of 3]
\end{verbatim}

\textbf{અમલીકરણ વ્યૂહરચનાઓ:}

\begin{itemize}
\tightlist
\item
  ડ્યુરેબિલિટી માટે \textbf{મલ્ટી-રીજન} રેપ્લિકેશન
\item
  ઉપલબ્ધતા માટે \textbf{કોરમ-આધારિત} કોન્સિસ્ટન્સી
\item
  ડેટા ઇન્ટિગ્રિટી માટે \textbf{ચેકસમ્સ}
\item
  રિકવરી માટે \textbf{ટ્રાન્ઝેક્શન લોગ્સ}
\end{itemize}

\end{solutionbox}
\begin{mnemonicbox}
``SEWR'' - Strong consistency, Eventual consistency,
Weak consistency, Replication strategies

\end{mnemonicbox}
\begin{center}\rule{0.5\linewidth}{0.5pt}\end{center}

\subsection*{પ્રશ્ન 4(અ) [3
ગુણ]}\label{uxaaauxab0uxab6uxaa8-4uxa85-3-uxa97uxaa3}

\textbf{ડેટા સ્કેલિંગની ભૂમિકા લખો.}

\begin{solutionbox}

\textbf{ડેટા સ્કેલિંગની ભૂમિકા:}

{\def\LTcaptype{none} % do not increment counter
\begin{longtable}[]{@{}ll@{}}
\toprule\noalign{}
ભૂમિકા & વર્ણન \\
\midrule\noalign{}
\endhead
\bottomrule\noalign{}
\endlastfoot
\textbf{પ્રદર્શન જાળવણી} & વધેલા ડેટા વોલ્યુમને કાર્યક્ષમ રીતે હેન્ડલ કરવું \\
\textbf{સ્ટોરેજ ઓપ્ટિમાઇઝેશન} & બહુવિધ સિસ્ટમ્સમાં ડેટા વિતરણ \\
\textbf{ક્વેરી પ્રદર્શન} & ઝડપી ડેટા રિટ્રીવલ સ્પીડ જાળવવી \\
\textbf{કોસ્ટ મેનેજમેન્ટ} & સ્ટોરેજ કોસ્ટ સાથે પ્રદર્શનનું સંતુલન \\
\end{longtable}
}

\end{solutionbox}
\begin{mnemonicbox}
``PSQC'' - Performance, Storage optimization, Query
performance, Cost management

\end{mnemonicbox}
\begin{center}\rule{0.5\linewidth}{0.5pt}\end{center}

\subsection*{પ્રશ્ન 4(બ) [4
ગુણ]}\label{uxaaauxab0uxab6uxaa8-4uxaac-4-uxa97uxaa3}

\textbf{Kubernetes વ્યાખ્યાયિત કરો. કારણ સાથે સમજાવો: Kubernetes એ cloud
computing નો આવશ્યક ભાગ છે.}

\begin{solutionbox}

\textbf{Kubernetes} એ ઓપન-સોર્સ કન્ટેનર ઓર્કેસ્ટ્રેશન પ્લેટફોર્મ છે જે કન્ટેનરાઇઝ્ડ
એપ્લિકેશનોના ડિપ્લોયમેન્ટ, સ્કેલિંગ અને મેનેજમેન્ટને ઓટોમેટ કરે છે.

\textbf{Kubernetes ક્લાઉડ કમ્પ્યુટિંગ માટે કેમ આવશ્યક છે:}

{\def\LTcaptype{none} % do not increment counter
\begin{longtable}[]{@{}ll@{}}
\toprule\noalign{}
કારણ & સમજાવટ \\
\midrule\noalign{}
\endhead
\bottomrule\noalign{}
\endlastfoot
\textbf{કન્ટેનર ઓર્કેસ્ટ્રેશન} & ક્લસ્ટર્સમાં બહુવિધ કન્ટેનરોનું સંચાલન \\
\textbf{ઓટો-સ્કેલિંગ} & માંગ પર આધારિત રિસોર્સ ગતિશીલ ગોઠવણી \\
\textbf{સર્વિસ ડિસ્કવરી} & ઓટોમેટિક લોડ બેલેન્સિંગ અને નેટવર્કિંગ \\
\textbf{સેલ્ફ-હીલિંગ} & નિષ્ફળ કન્ટેનર્સને ઓટોમેટિક રીતે બદલવા \\
\end{longtable}
}

\textbf{Kubernetes આર્કિટેક્ચર:}

\begin{verbatim}
graph TB
    A[Master Node] {-{-} B[API Server]}
    A {-{-} C[Controller Manager]}
    A {-{-} D[Scheduler]}
    E[Worker Node 1] {-{-} F[Kubelet]}
    E {-{-} G[Pods]}
    H[Worker Node 2] {-{-} I[Kubelet]}
    H {-{-} J[Pods]}
\end{verbatim}

\textbf{આવશ્યક ફાયદાઓ:}

\begin{itemize}
\tightlist
\item
  ક્લાઉડ પ્રોવાઇડર્સમાં \textbf{પ્લેટફોર્મ સ્વતંત્રતા}
\item
  કન્ટેનર ડેન્સિટી દ્વારા \textbf{રિસોર્સ કાર્યક્ષમતા}
\item
  CI/CD પાઇપલાઇન્સ સાથે \textbf{DevOps ઇન્ટિગ્રેશન}
\end{itemize}

\end{solutionbox}
\begin{mnemonicbox}
``CASS'' - Container orchestration, Auto-scaling,
Service discovery, Self-healing

\end{mnemonicbox}
\begin{center}\rule{0.5\linewidth}{0.5pt}\end{center}

\subsection*{પ્રશ્ન 4(ક) [7
ગુણ]}\label{uxaaauxab0uxab6uxaa8-4uxa95-7-uxa97uxaa3}

\textbf{ડેટા સેન્ટર નેટવર્ક ટોપોલોજીઝ સમજાવો.}

\begin{solutionbox}

\textbf{ડેટા સેન્ટર નેટવર્ક ટોપોલોજીઝ} એ ડેટા સેન્ટરની અંદર નેટવર્ક ઘટકો કેવી રીતે
એકબીજા સાથે જોડાયેલા છે તે વ્યાખ્યાયિત કરે છે.

\textbf{સામાન્ય ટોપોલોજીઝ:}

{\def\LTcaptype{none} % do not increment counter
\begin{longtable}[]{@{}
  >{\raggedright\arraybackslash}p{(\linewidth - 6\tabcolsep) * \real{0.2571}}
  >{\raggedright\arraybackslash}p{(\linewidth - 6\tabcolsep) * \real{0.2286}}
  >{\raggedright\arraybackslash}p{(\linewidth - 6\tabcolsep) * \real{0.2571}}
  >{\raggedright\arraybackslash}p{(\linewidth - 6\tabcolsep) * \real{0.2571}}@{}}
\toprule\noalign{}
\begin{minipage}[b]{\linewidth}\raggedright
ટોપોલોજી
\end{minipage} & \begin{minipage}[b]{\linewidth}\raggedright
વર્ણન
\end{minipage} & \begin{minipage}[b]{\linewidth}\raggedright
ફાયદાઓ
\end{minipage} & \begin{minipage}[b]{\linewidth}\raggedright
નુકસાન
\end{minipage} \\
\midrule\noalign{}
\endhead
\bottomrule\noalign{}
\endlastfoot
\textbf{થ્રી-ટાયર} & કોર, એગ્રિગેશન, એક્સેસ લેયર & સરળ, હાયરાર્કિકલ & મર્યાદિત
સ્કેલેબિલિટી \\
\textbf{સ્પાઇન-લીફ} & નોન-બ્લોકિંગ, ફ્લેટ આર્કિટેક્ચર & ઉચ્ચ બેન્ડવિડ્થ, સ્કેલેબલ &
જટિલ કોન્ફિગરેશન \\
\textbf{ફેટ ટ્રી} & બહુવિધ પાથ સાથે ટ્રી સ્ટ્રક્ચર & સારી ફોલ્ટ ટોલરન્સ &
ઓવરસબસ્ક્રિપ્શન સમસ્યાઓ \\
\end{longtable}
}

\textbf{સ્પાઇન-લીફ આર્કિટેક્ચર:}

\begin{verbatim}
graph TB
    S1[Spine 1] {-{-}{-} L1[Leaf 1]}
    S1 {-{-}{-} L2[Leaf 2]}
    S1 {-{-}{-} L3[Leaf 3]}
    S2[Spine 2] {-{-}{-} L1}
    S2 {-{-}{-} L2}
    S2 {-{-}{-} L3}
    L1 {-{-}{-} A1[Server 1]}
    L2 {-{-}{-} A2[Server 2]}
    L3 {-{-}{-} A3[Server 3]}
\end{verbatim}

\textbf{આધુનિક ટ્રેન્ડ્સ:}

\begin{itemize}
\tightlist
\item
  પ્રોગ્રામેબલ નેટવર્ક માટે \textbf{Software-Defined Networking (SDN)}
\item
  લવચીક સેવાઓ માટે \textbf{Network Function Virtualization (NFV)}
\item
  વધારેલી સુરક્ષા માટે \textbf{માઇક્રો-સેગમેન્ટેશન}
\end{itemize}

\textbf{પસંદગીના માપદંડો:}

\begin{itemize}
\tightlist
\item
  \textbf{બેન્ડવિડ્થ} આવશ્યકતાઓ
\item
  \textbf{લેટન્સી} સંવેદનશીલતા
\item
  \textbf{સ્કેલેબિલિટી} જરૂરિયાતો
\item
  \textbf{કોસ્ટ} વિચારણાઓ
\end{itemize}

\textbf{આધુનિક ટોપોલોજીઝના ફાયદાઓ:}

\begin{itemize}
\tightlist
\item
  \textbf{નોન-બ્લોકિંગ} કોમ્યુનિકેશન પાથ
\item
  \textbf{ઇક્વલ-કોસ્ટ} મલ્ટિ-પાથ રાઉટિંગ
\item
  \textbf{હોરિઝોન્ટલ} સ્કેલિંગ ક્ષમતા
\item
  \textbf{ઘટાડેલ} નેટવર્ક કન્જેશન
\end{itemize}

\end{solutionbox}
\begin{mnemonicbox}
``TSF'' - Three-tier, Spine-leaf, Fat tree

\end{mnemonicbox}
\begin{center}\rule{0.5\linewidth}{0.5pt}\end{center}

\subsection*{પ્રશ્ન 4(અ) OR [3
ગુણ]}\label{uxaaauxab0uxab6uxaa8-4uxa85-or-3-uxa97uxaa3}

\textbf{ક્લાઉડમાં ફાઇલ સ્ટોરેજ સમજાવો.}

\begin{solutionbox}

\textbf{ક્લાઉડમાં ફાઇલ સ્ટોરેજ સમજાવો.}

\end{solutionbox}
\begin{solutionbox}

\textbf{ક્લાઉડ ફાઇલ સ્ટોરેજ} એ પરંપરાગત ફાઇલ સિસ્ટમ જેવું જ હાયરાર્કિકલ ફાઇલ
સિસ્ટમ એક્સેસ નેટવર્ક પર પ્રદાન કરે છે.

\textbf{લાક્ષણિકતાઓ:}

{\def\LTcaptype{none} % do not increment counter
\begin{longtable}[]{@{}ll@{}}
\toprule\noalign{}
વિશેષતા & વર્ણન \\
\midrule\noalign{}
\endhead
\bottomrule\noalign{}
\endlastfoot
\textbf{હાયરાર્કિકલ સ્ટ્રક્ચર} & ફોલ્ડર અને સબફોલ્ડર સંગઠન \\
\textbf{POSIX કોમ્પ્લાયન્સ} & સ્ટાન્ડર્ડ ફાઇલ સિસ્ટમ ઇન્ટરફેસ \\
\textbf{નેટવર્ક એક્સેસ} & SMB, NFS પ્રોટોકોલ સપોર્ટ \\
\textbf{શેર્ડ એક્સેસ} & બહુવિધ યુઝર્સ એક સાથે એક્સેસ કરી શકે છે \\
\end{longtable}
}

\end{solutionbox}
\begin{mnemonicbox}
``HPNS'' - Hierarchical, POSIX-compliant, Network
access, Shared access

\end{mnemonicbox}
\begin{center}\rule{0.5\linewidth}{0.5pt}\end{center}

\subsection*{પ્રશ્ન 4(બ) OR [4
ગુણ]}\label{uxaaauxab0uxab6uxaa8-4uxaac-or-4-uxa97uxaa3}

\textbf{સર્વરલેસ કમ્પ્યુટિંગ સમજાવો.}

\begin{solutionbox}

\textbf{સર્વરલેસ કમ્પ્યુટિંગ} એ ક્લાઉડ કમ્પ્યુટિંગ મોડલ છે જ્યાં ક્લાઉડ પ્રોવાઇડર્સ
ઓટોમેટિક રીતે સર્વર ઇન્ફ્રાસ્ટ્રક્ચરનું સંચાલન કરે છે, જે ડેવલપર્સને કોડ પર ધ્યાન આપવાની
મંજૂરી આપે છે.

\textbf{મુખ્ય વિશેષતાઓ:}

{\def\LTcaptype{none} % do not increment counter
\begin{longtable}[]{@{}ll@{}}
\toprule\noalign{}
વિશેષતા & વર્ણન \\
\midrule\noalign{}
\endhead
\bottomrule\noalign{}
\endlastfoot
\textbf{ઇવેન્ટ-ડ્રિવન} & ઇવેન્ટ્સ દ્વારા ટ્રિગર થતા ફંક્શન્સ \\
\textbf{ઓટો-સ્કેલિંગ} & ઓટોમેટિક રિસોર્સ ફાળવણી \\
\textbf{પે-પર-એક્ઝિક્યુશન} & વાસ્તવિક ઉપયોગ પર આધારિત બિલિંગ \\
\textbf{સ્ટેટલેસ} & ફંક્શન્સ સ્ટેટ જાળવતા નથી \\
\end{longtable}
}

\textbf{સર્વરલેસ આર્કિટેક્ચર:}

\begin{center}
\textbf{Mermaid Diagram (Code)}
\begin{verbatim}
{Shaded}
{Highlighting}[]
graph LR
    A[Event Source] {-{-}{} B[Function Trigger]}
    B {-{-}{} C[Function Execution]}
    C {-{-}{} D[Response]}
    E[Cloud Provider] {-{-}{} F[Infrastructure Management]}
{Highlighting}
{Shaded}
\end{verbatim}
\end{center}

\textbf{ફાયદાઓ:}

\begin{itemize}
\tightlist
\item
  \textbf{કોઈ સર્વર મેનેજમેન્ટ} જરૂરી નથી
\item
  વેરિયેબલ વર્કલોડ માટે \textbf{કોસ્ટ કાર્યક્ષમતા}
\item
  \textbf{ઝડપી સ્કેલિંગ} ક્ષમતાઓ
\end{itemize}

\end{solutionbox}
\begin{mnemonicbox}
``EAPS'' - Event-driven, Auto-scaling,
Pay-per-execution, Stateless

\end{mnemonicbox}
\begin{center}\rule{0.5\linewidth}{0.5pt}\end{center}

\subsection*{પ્રશ્ન 4(ક) OR [7
ગુણ]}\label{uxaaauxab0uxab6uxaa8-4uxa95-or-7-uxa97uxaa3}

\textbf{SDN (Software Defined Networking) આર્કિટેક્ચર સમજાવો.}

\begin{solutionbox}

\textbf{Software Defined Networking (SDN)} એ નેટવર્ક કંટ્રોલ પ્લેનને ડેટા પ્લેનથી
અલગ કરે છે, જે સોફ્ટવેર દ્વારા કેન્દ્રીકૃત નેટવર્ક મેનેજમેન્ટને સક્ષમ બનાવે છે.

\textbf{SDN આર્કિટેક્ચર લેયર્સ:}

{\def\LTcaptype{none} % do not increment counter
\begin{longtable}[]{@{}lll@{}}
\toprule\noalign{}
લેયર & કાર્ય & ઘટકો \\
\midrule\noalign{}
\endhead
\bottomrule\noalign{}
\endlastfoot
\textbf{એપ્લિકેશન લેયર} & નેટવર્ક એપ્લિકેશન અને સેવાઓ & ફાયરવૉલ, લોડ બેલેન્સર \\
\textbf{કંટ્રોલ લેયર} & કેન્દ્રીકૃત નેટવર્ક ઇન્ટેલિજન્સ & SDN કંટ્રોલર \\
\textbf{ઇન્ફ્રાસ્ટ્રક્ચર લેયર} & નેટવર્ક ફોરવર્ડિંગ ઉપકરણો & સ્વિચ, રાઉટર \\
\end{longtable}
}

\textbf{SDN આર્કિટેક્ચર ડાયાગ્રામ:}

\begin{center}
\textbf{Mermaid Diagram (Code)}
\begin{verbatim}
{Shaded}
{Highlighting}[]
graph LR
    A[Application Layer] {-{-}{} B[Northbound APIs]}
    B {-{-}{} C[SDN Controller]}
    C {-{-}{} D[Southbound APIs]}
    D {-{-}{} E[Infrastructure Layer]}
    
    F[Network Apps] {-{-}{} A}
    G[OpenFlow Switches] {-{-}{} E}
{Highlighting}
{Shaded}
\end{verbatim}
\end{center}

\textbf{મુખ્ય પ્રોટોકોલ્સ:}

\begin{itemize}
\tightlist
\item
  \textbf{OpenFlow:} કંટ્રોલર અને સ્વિચ વચ્ચે કોમ્યુનિકેશન
\item
  \textbf{NETCONF:} નેટવર્ક કોન્ફિગરેશન પ્રોટોકોલ
\item
  \textbf{REST APIs:} નોર્થબાઉન્ડ એપ્લિકેશન ઇન્ટરફેસ
\end{itemize}

\textbf{SDN ફાયદાઓ:}

{\def\LTcaptype{none} % do not increment counter
\begin{longtable}[]{@{}ll@{}}
\toprule\noalign{}
ફાયદો & વર્ણન \\
\midrule\noalign{}
\endhead
\bottomrule\noalign{}
\endlastfoot
\textbf{કેન્દ્રીકૃત નિયંત્રણ} & નેટવર્ક મેનેજમેન્ટનું એક બિંદુ \\
\textbf{પ્રોગ્રામેબિલિટી} & સોફ્ટવેર-આધારિત નેટવર્ક કોન્ફિગરેશન \\
\textbf{લવચીકતા} & ગતિશીલ નેટવર્ક રિકોન્ફિગરેશન \\
\textbf{કોસ્ટ રિડક્શન} & કમોડિટી હાર્ડવેર ઉપયોગ \\
\end{longtable}
}

\textbf{ઉપયોગ કેસ:}

\begin{itemize}
\tightlist
\item
  \textbf{ડેટા સેન્ટર} નેટવર્કિંગ
\item
  \textbf{કેમ્પસ} નેટવર્ક
\item
  \textbf{વાઇડ એરિયા} નેટવર્ક
\item
  \textbf{નેટવર્ક ફંક્શન} વર્ચ્યુઅલાઇઝેશન
\end{itemize}

\textbf{પડકારો:}

\begin{itemize}
\tightlist
\item
  \textbf{સિંગલ પોઇન્ટ} ઓફ ફેલ્યોર (કંટ્રોલર)
\item
  \textbf{સ્કેલેબિલિટી} ચિંતાઓ
\item
  \textbf{સિક્યુરિટી} વિચારણાઓ
\item
  \textbf{વેન્ડર} ઇન્ટરઓપરેબિલિટી
\end{itemize}

\end{solutionbox}
\begin{mnemonicbox}
``ACI'' - Application layer, Control layer,
Infrastructure layer

\end{mnemonicbox}
\begin{center}\rule{0.5\linewidth}{0.5pt}\end{center}

\subsection*{પ્રશ્ન 5(અ) [3
ગુણ]}\label{uxaaauxab0uxab6uxaa8-5uxa85-3-uxa97uxaa3}

\textbf{Infrastructure as Code (IaC) વિગતે સમજાવો.}

\begin{solutionbox}

\textbf{Infrastructure as Code (IaC)} એ મેન્યુઅલ પ્રક્રિયાઓને બદલે મશીન-રીડેબલ
ડેફિનિશન ફાઇલો દ્વારા કમ્પ્યુટિંગ ઇન્ફ્રાસ્ટ્રક્ચરનું સંચાલન અને પ્રોવિઝન કરે છે.

\textbf{IaC લાક્ષણિકતાઓ:}

{\def\LTcaptype{none} % do not increment counter
\begin{longtable}[]{@{}ll@{}}
\toprule\noalign{}
લાક્ષણિકતા & વર્ણન \\
\midrule\noalign{}
\endhead
\bottomrule\noalign{}
\endlastfoot
\textbf{વર્ઝન કંટ્રોલ} & રિપોઝિટરીમાં સ્ટોર થતી ઇન્ફ્રાસ્ટ્રક્ચર ડેફિનિશન \\
\textbf{ઓટોમેશન} & ઓટોમેટેડ ડિપ્લોયમેન્ટ અને મેનેજમેન્ટ \\
\textbf{કોન્સિસ્ટન્સી} & ડિપ્લોયમેન્ટ્સમાં સમાન વાતાવરણ \\
\textbf{રિપીટેબિલિટી} & પુનઃઉત્પાદનક્ષમ ઇન્ફ્રાસ્ટ્રક્ચર સેટઅપ \\
\end{longtable}
}

\end{solutionbox}
\begin{mnemonicbox}
``VACR'' - Version control, Automation, Consistency,
Repeatability

\end{mnemonicbox}
\begin{center}\rule{0.5\linewidth}{0.5pt}\end{center}

\subsection*{પ્રશ્ન 5(બ) [4
ગુણ]}\label{uxaaauxab0uxab6uxaa8-5uxaac-4-uxa97uxaa3}

\textbf{SLA નું ફુલ ફોર્મ આપો અને વિગતે સમજાવો.}

\begin{solutionbox}

\textbf{SLA - Service Level Agreement}

\textbf{SLA ડેફિનિશન:} સર્વિસ પ્રોવાઇડર અને ગ્રાહક વચ્ચેનો કરાર જે અપેક્ષિત સર્વિસ
લેવલ અને પ્રદર્શન મેટ્રિક્સ વ્યાખ્યાયિત કરે છે.

\textbf{SLA ઘટકો:}

{\def\LTcaptype{none} % do not increment counter
\begin{longtable}[]{@{}ll@{}}
\toprule\noalign{}
ઘટક & વર્ણન \\
\midrule\noalign{}
\endhead
\bottomrule\noalign{}
\endlastfoot
\textbf{ઉપલબ્ધતા} & અપટાઇમ ટકાવારી (99.9\%, 99.99\%) \\
\textbf{પ્રદર્શન} & રિસ્પોન્સ ટાઇમ, થ્રુપુટ મેટ્રિક્સ \\
\textbf{સપોર્ટ} & સમસ્યાઓ માટે રિસ્પોન્સ ટાઇમ \\
\textbf{પેનાલ્ટીઝ} & SLA ઉલ્લંઘન માટે વળતર \\
\end{longtable}
}

\textbf{SLA મેટ્રિક્સ:}

\begin{verbatim}
┌─────────────────┐    ┌─────────────────┐
│   Availability  │    │   Performance   │
│     99.99\%      │    │   { 200ms       │}
└─────────────────┘    └─────────────────┘
         │                       │
         └───────────┬───────────┘
                     │
            ┌─────────────────┐
            │       SLA       │
            │   Requirements  │
            └─────────────────┘
\end{verbatim}

\textbf{ફાયદાઓ:}

\begin{itemize}
\tightlist
\item
  બન્ને પક્ષો માટે \textbf{સ્પષ્ટ અપેક્ષાઓ}
\item
  \textbf{પ્રદર્શન} મેઝરમેન્ટ સ્ટાન્ડર્ડ્સ
\item
  પેનાલ્ટીઝ દ્વારા \textbf{રિસ્ક મિટિગેશન}
\end{itemize}

\end{solutionbox}
\begin{mnemonicbox}
``APSP'' - Availability, Performance, Support,
Penalties

\end{mnemonicbox}
\begin{center}\rule{0.5\linewidth}{0.5pt}\end{center}

\subsection*{પ્રશ્ન 5(ક) [7
ગુણ]}\label{uxaaauxab0uxab6uxaa8-5uxa95-7-uxa97uxaa3}

\textbf{હાઇપરવાઇઝર્સ વિગતે સમજાવો.}

\begin{solutionbox}

\textbf{હાઇપરવાઇઝર} (વર્ચ્યુઅલ મશીન મોનિટર) એ સોફ્ટવેર છે જે ભૌતિક હાર્ડવેરને અમૂર્ત
બનાવીને વર્ચ્યુઅલ મશીનો બનાવે અને મેનેજ કરે છે.

\textbf{હાઇપરવાઇઝરના પ્રકારો:}

{\def\LTcaptype{none} % do not increment counter
\begin{longtable}[]{@{}
  >{\raggedright\arraybackslash}p{(\linewidth - 6\tabcolsep) * \real{0.1579}}
  >{\raggedright\arraybackslash}p{(\linewidth - 6\tabcolsep) * \real{0.2105}}
  >{\raggedright\arraybackslash}p{(\linewidth - 6\tabcolsep) * \real{0.2368}}
  >{\raggedright\arraybackslash}p{(\linewidth - 6\tabcolsep) * \real{0.3947}}@{}}
\toprule\noalign{}
\begin{minipage}[b]{\linewidth}\raggedright
પ્રકાર
\end{minipage} & \begin{minipage}[b]{\linewidth}\raggedright
વર્ણન
\end{minipage} & \begin{minipage}[b]{\linewidth}\raggedright
ઉદાહરણો
\end{minipage} & \begin{minipage}[b]{\linewidth}\raggedright
લાક્ષણિકતાઓ
\end{minipage} \\
\midrule\noalign{}
\endhead
\bottomrule\noalign{}
\endlastfoot
\textbf{ટાઇપ 1 (બેર મેટલ)} & સીધું હાર્ડવેર પર ચાલે છે & VMware vSphere, Hyper-V
& બહેતર પ્રદર્શન, એન્ટરપ્રાઇઝ ઉપયોગ \\
\textbf{ટાઇપ 2 (હોસ્ટેડ)} & હોસ્ટ ઓપરેટિંગ સિસ્ટમ પર ચાલે છે & VirtualBox,
VMware Workstation & સરળ સેટઅપ, ડેસ્કટોપ ઉપયોગ \\
\end{longtable}
}

\textbf{હાઇપરવાઇઝર આર્કિટેક્ચર:}

\begin{verbatim}
graph TB
    subgraph "Type 1 {- Bare Metal"}
        A[Physical Hardware] {-{-} B[Type 1 Hypervisor]}
        B {-{-} C[VM1]}
        B {-{-} D[VM2]}
        B {-{-} E[VM3]}
    end
    
    subgraph "Type 2 {- Hosted"}
        F[Physical Hardware] {-{-} G[Host OS]}
        G {-{-} H[Type 2 Hypervisor]}
        H {-{-} I[VM1]}
        H {-{-} J[VM2]}
    end
\end{verbatim}

\textbf{હાઇપરવાઇઝર કાર્યો:}

{\def\LTcaptype{none} % do not increment counter
\begin{longtable}[]{@{}ll@{}}
\toprule\noalign{}
કાર્ય & વર્ણન \\
\midrule\noalign{}
\endhead
\bottomrule\noalign{}
\endlastfoot
\textbf{રિસોર્સ ફાળવણી} & CPU, મેમરી, સ્ટોરેજ વિતરણ \\
\textbf{આઇસોલેશન} & અલગ VM વાતાવરણ \\
\textbf{હાર્ડવેર અમૂર્તીકરણ} & વર્ચ્યુઅલ હાર્ડવેર પ્રેઝન્ટેશન \\
\textbf{VM લાઇફસાઇકલ મેનેજમેન્ટ} & VM બનાવવા, શરૂ કરવા, બંધ કરવા, ડિલીટ
કરવા \\
\end{longtable}
}

\textbf{વર્ચ્યુઅલાઇઝેશન તકનીકો:}

\begin{itemize}
\tightlist
\item
  \textbf{હાર્ડવેર-એસિસ્ટેડ} વર્ચ્યુઅલાઇઝેશન (Intel VT-x, AMD-V)
\item
  સુધારેલ પ્રદર્શન માટે \textbf{પેરાવર્ચ્યુઅલાઇઝેશન}
\item
  સુસંગતતા માટે \textbf{બાઇનરી ટ્રાન્સલેશન}
\end{itemize}

\textbf{પ્રદર્શન વિચારણાઓ:}

\begin{itemize}
\tightlist
\item
  વર્ચ્યુઅલાઇઝેશન લેયરથી \textbf{CPU ઓવરહેડ}
\item
  વર્ચ્યુઅલ મેમરી સાથે \textbf{મેમરી મેનેજમેન્ટ}
\item
  સ્ટોરેજ અને નેટવર્ક માટે \textbf{I/O ઓપ્ટિમાઇઝેશન}
\item
  VM વચ્ચે \textbf{રિસોર્સ શેડ્યુલિંગ}
\end{itemize}

\textbf{ફાયદાઓ:}

\begin{itemize}
\tightlist
\item
  હાર્ડવેર કોસ્ટ ઘટાડીને \textbf{સર્વર કન્સોલિડેશન}
\item
  VM સ્નેપશોટ દ્વારા \textbf{ડિઝાસ્ટર રિકવરી}
\item
  ઝડપી પ્રોવિઝનિંગ \textbf{ટેસ્ટિંગ એન્વાયરનમેન્ટ}
\item
  લીગેસી એપ્લિકેશન \textbf{સપોર્ટ}
\end{itemize}

\textbf{પડકારો:}

\begin{itemize}
\tightlist
\item
  બેર મેટલ સરખામણીમાં \textbf{પ્રદર્શન ઓવરહેડ}
\item
  મેનેજમેન્ટમાં \textbf{જટિલતા}
\item
  એન્ટરપ્રાઇઝ હાઇપરવાઇઝર્સ માટે \textbf{લાઇસન્સિંગ કોસ્ટ}
\item
  શેર્ડ રિસોર્સીસ માટે \textbf{સિક્યુરિટી} વિચારણાઓ
\end{itemize}

\end{solutionbox}
\begin{mnemonicbox}
``RAIH'' - Resource allocation, isolation, Hardware
abstraction

\end{mnemonicbox}
\begin{center}\rule{0.5\linewidth}{0.5pt}\end{center}

\subsection*{પ્રશ્ન 5(અ) OR [3
ગુણ]}\label{uxaaauxab0uxab6uxaa8-5uxa85-or-3-uxa97uxaa3}

\textbf{ડેટા સેન્ટર્સમાં ઓટોમેશન શું છે? વિગતે સમજાવો.}

\begin{solutionbox}

\textbf{ડેટા સેન્ટર ઓટોમેશન} એ મેન્યુઅલ હસ્તક્ષેપ વગર નિયમિત કાર્યો ઓટોમેટિક રીતે
કરવા માટે સોફ્ટવેર અને ટેકનોલોજીઝનો ઉપયોગ છે.

\textbf{ઓટોમેશન વિસ્તારો:}

{\def\LTcaptype{none} % do not increment counter
\begin{longtable}[]{@{}ll@{}}
\toprule\noalign{}
વિસ્તાર & વર્ણન \\
\midrule\noalign{}
\endhead
\bottomrule\noalign{}
\endlastfoot
\textbf{પ્રોવિઝનિંગ} & ઓટોમેટિક સર્વર અને સર્વિસ ડિપ્લોયમેન્ટ \\
\textbf{મોનિટરિંગ} & સતત પ્રદર્શન અને હેલ્થ ટ્રેકિંગ \\
\textbf{સ્કેલિંગ} & ગતિશીલ રિસોર્સ ગોઠવણી \\
\textbf{મેઇન્ટેનન્સ} & ઓટોમેટેડ પેચિંગ અને અપડેટ્સ \\
\end{longtable}
}

\end{solutionbox}
\begin{mnemonicbox}
``PMSM'' - Provisioning, Monitoring, Scaling,
Maintenance

\end{mnemonicbox}
\begin{center}\rule{0.5\linewidth}{0.5pt}\end{center}

\subsection*{પ્રશ્ન 5(બ) OR [4
ગુણ]}\label{uxaaauxab0uxab6uxaa8-5uxaac-or-4-uxa97uxaa3}

\textbf{ક્લાઉડમાં ડેટા સિક્યુરિટી શું છે? વિગતે સમજાવો.}

\begin{solutionbox}

\textbf{ક્લાઉડ ડેટા સિક્યુરિટી} એ ક્લાઉડ વાતાવરણમાં સ્ટોર, પ્રોસેસ અને ટ્રાન્સમિટ
થતા ડેટાને અનધિકૃત એક્સેસ, ભ્રષ્ટાચાર અને ચોરીથી સુરક્ષિત રાખવાનો સમાવેશ કરે છે.

\textbf{સિક્યુરિટી પગલાં:}

{\def\LTcaptype{none} % do not increment counter
\begin{longtable}[]{@{}ll@{}}
\toprule\noalign{}
પગલું & વર્ણન \\
\midrule\noalign{}
\endhead
\bottomrule\noalign{}
\endlastfoot
\textbf{એન્ક્રિપ્શન} & રેસ્ટ અને ટ્રાન્ઝિટમાં ડેટા પ્રોટેક્શન \\
\textbf{એક્સેસ કંટ્રોલ} & યુઝર ઓથેન્ટિકેશન અને ઓથરાઇઝેશન \\
\textbf{બેકઅપ એન્ડ રિકવરી} & નુકસાન સામે ડેટા પ્રોટેક્શન \\
\textbf{કોમ્પ્લાયન્સ} & નિયમનકારી આવશ્યકતાઓનું પાલન \\
\end{longtable}
}

\textbf{સિક્યુરિટી અમલીકરણ:}

\begin{verbatim}
┌─────────────┐    ┌─────────────┐    ┌─────────────┐
│ Encryption  │    │   Access    │    │   Backup    │
│             │    │  Controls   │    │             │
│ AES{-256     │    │ IAM/RBAC    │    │ 3{-}2{-}1 Rule  │}
└─────────────┘    └─────────────┘    └─────────────┘
       │                    │                    │
       └────────────────────┼────────────────────┘
                            │
                   ┌─────────────┐
                   │    Data     │
                   │  Security   │
                   └─────────────┘
\end{verbatim}

\textbf{બેસ્ટ પ્રેક્ટિસ:}

\begin{itemize}
\tightlist
\item
  \textbf{ઝીરો-ટ્રસ્ટ} સિક્યુરિટી મોડલ
\item
  \textbf{નિયમિત} સિક્યુરિટી ઓડિટ
\item
  \textbf{ડેટા ક્લાસિફિકેશન} અને હેન્ડલિંગ
\end{itemize}

\end{solutionbox}
\begin{mnemonicbox}
``EABC'' - Encryption, Access controls, Backup,
Compliance

\end{mnemonicbox}
\begin{center}\rule{0.5\linewidth}{0.5pt}\end{center}

\subsection*{પ્રશ્ન 5(ક) OR [7
ગુણ]}\label{uxaaauxab0uxab6uxaa8-5uxa95-or-7-uxa97uxaa3}

\textbf{વર્ચ્યુઅલ મશીન્સ શું છે? વર્ચ્યુઅલ મશીન્સ બનાવવા અને મેનેજ કરવાના સ્ટેપ્સ
સમજાવો.}

\begin{solutionbox}

\textbf{વર્ચ્યુઅલ મશીન (VM)} એ ભૌતિક કમ્પ્યુટરના સોફ્ટવેર-આધારિત એમ્યુલેશન છે જે અલગ
વાતાવરણમાં ઓપરેટિંગ સિસ્ટમ અને એપ્લિકેશન ચલાવે છે.

\textbf{VM ઘટકો:}

{\def\LTcaptype{none} % do not increment counter
\begin{longtable}[]{@{}ll@{}}
\toprule\noalign{}
ઘટક & વર્ણન \\
\midrule\noalign{}
\endhead
\bottomrule\noalign{}
\endlastfoot
\textbf{વર્ચ્યુઅલ CPU} & એમ્યુલેટેડ પ્રોસેસર કોર્સ \\
\textbf{વર્ચ્યુઅલ મેમરી} & VM માટે ફાળવેલ RAM \\
\textbf{વર્ચ્યુઅલ સ્ટોરેજ} & વર્ચ્યુઅલ હાર્ડ ડિસ્ક \\
\textbf{વર્ચ્યુઅલ નેટવર્ક} & નેટવર્ક ઇન્ટરફેસ એમ્યુલેશન \\
\end{longtable}
}

\textbf{વર્ચ્યુઅલ મશીન બનાવવાના સ્ટેપ્સ:}

\textbf{1. પ્લાનિંગ ફેઝ:}

\begin{itemize}
\tightlist
\item
  \textbf{રિસોર્સ એસેસમેન્ટ:} CPU, RAM, સ્ટોરેજ આવશ્યકતાઓ નક્કી કરવી
\item
  \textbf{OS પસંદગી:} ગેસ્ટ ઓપરેટિંગ સિસ્ટમ પસંદ કરવું
\item
  \textbf{નેટવર્ક કોન્ફિગરેશન:} IP એડ્રેસિંગ અને કનેક્ટિવિટી પ્લાન કરવી
\end{itemize}

\textbf{2. VM બનાવવાની પ્રક્રિયા:}

\begin{center}
\textbf{Mermaid Diagram (Code)}
\begin{verbatim}
{Shaded}
{Highlighting}[]
graph LR
    A[Select Hypervisor] {-{-}{} B[Create VM]}
    B {-{-}{} C[Allocate Resources]}
    C {-{-}{} D[Install OS]}
    D {-{-}{} E[Configure Network]}
    E {-{-}{} F[Install Applications]}
{Highlighting}
{Shaded}
\end{verbatim}
\end{center}

\textbf{3. વિગતવાર બનાવવાના સ્ટેપ્સ:}

{\def\LTcaptype{none} % do not increment counter
\begin{longtable}[]{@{}lll@{}}
\toprule\noalign{}
સ્ટેપ & એક્શન & વિગતો \\
\midrule\noalign{}
\endhead
\bottomrule\noalign{}
\endlastfoot
\textbf{1} & \textbf{VM કન્ટેનર બનાવવું} & VM નામ અને સ્થાન વ્યાખ્યાયિત કરવું \\
\textbf{2} & \textbf{CPU ફાળવવું} & વર્ચ્યુઅલ પ્રોસેસર કોર્સ એસાઇન કરવા \\
\textbf{3} & \textbf{મેમરી એસાઇન કરવી} & RAM ફાળવવી (2GB-16GB સામાન્ય) \\
\textbf{4} & \textbf{સ્ટોરેજ બનાવવું} & વર્ચ્યુઅલ હાર્ડ ડિસ્ક સેટ કરવી \\
\textbf{5} & \textbf{નેટવર્ક સેટઅપ} & વર્ચ્યુઅલ નેટવર્ક એડેપ્ટર કોન્ફિગર કરવું \\
\textbf{6} & \textbf{OS ઇન્સ્ટોલેશન} & ગેસ્ટ ઓપરેટિંગ સિસ્ટમ ઇન્સ્ટોલ કરવું \\
\end{longtable}
}

\textbf{VM મેનેજમેન્ટ ઓપરેશન્સ:}

\textbf{પાવર મેનેજમેન્ટ:}

\begin{itemize}
\tightlist
\item
  \textbf{સ્ટાર્ટ/સ્ટોપ:} VM પાવર સ્ટેટ કંટ્રોલ કરવું
\item
  \textbf{સસ્પેન્ડ/રિઝ્યુમ:} VM એક્ઝિક્યુશન પોઝ અને રિઝ્યુમ કરવું
\item
  \textbf{રીસેટ:} VM ને ફોર્સ રીસ્ટાર્ટ કરવું
\end{itemize}

\textbf{રિસોર્સ મેનેજમેન્ટ:}

\begin{itemize}
\tightlist
\item
  \textbf{હોટ-એડ CPU/મેમરી:} શટડાઉન વગર રિસોર્સ ઉમેરવા
\item
  \textbf{સ્ટોરેજ એક્સપાન્શન:} ડિસ્ક કેપાસિટી વધારવી
\item
  \textbf{નેટવર્ક રિકોન્ફિગરેશન:} નેટવર્ક સેટિંગ્સ બદલવી
\end{itemize}

\textbf{મેઇન્ટેનન્સ ઓપરેશન્સ:}

{\def\LTcaptype{none} % do not increment counter
\begin{longtable}[]{@{}lll@{}}
\toprule\noalign{}
ઓપરેશન & હેતુ & આવર્તન \\
\midrule\noalign{}
\endhead
\bottomrule\noalign{}
\endlastfoot
\textbf{સ્નેપશોટ્સ} & પૉઇન્ટ-ઇન-ટાઇમ બેકઅપ & મોટા ફેરફારો પહેલાં \\
\textbf{ક્લોનિંગ} & સમાન કોપીઝ બનાવવા & સ્કેલિંગ/ટેસ્ટિંગ માટે \\
\textbf{માઇગ્રેશન} & હોસ્ટ્સ વચ્ચે VM ખસેડવું & મેઇન્ટેનન્સ માટે \\
\textbf{બેકઅપ} & ડેટા પ્રોટેક્શન & દૈનિક/સાપ્તાહિક \\
\end{longtable}
}

\textbf{VM લાઇફસાઇકલ મેનેજમેન્ટ:}

\begin{center}
\textbf{Mermaid Diagram (Code)}
\begin{verbatim}
{Shaded}
{Highlighting}[]
graph LR
    A[Create VM] {-{-}{} B[Configure VM]}
    B {-{-}{} C[Deploy Applications]}
    C {-{-}{} D[Monitor Performance]}
    D {-{-}{} E\{Maintenance Needed?\}}
    E {-{-}{}|Yes| F[Update/Patch]}
    E {-{-}{}|No| D}
    F {-{-}{} G\{End of Life?\}}
    G {-{-}{}|No| D}
    G {-{-}{}|Yes| H[Decommission VM]}
{Highlighting}
{Shaded}
\end{verbatim}
\end{center}

\textbf{બેસ્ટ પ્રેક્ટિસ:}

\begin{itemize}
\tightlist
\item
  \textbf{નિયમિત બેકઅપ} અને સ્નેપશોટ મેનેજમેન્ટ
\item
  ઓપ્ટિમાઇઝેશન માટે \textbf{રિસોર્સ મોનિટરિંગ}
\item
  \textbf{સિક્યુરિટી પેચિંગ} અને અપડેટ્સ
\item
  વર્કલોડ આધારિત \textbf{પ્રદર્શન ટ્યુનિંગ}
\end{itemize}

\textbf{મોનિટરિંગ અને ટ્રબલશૂટિંગ:}

\begin{itemize}
\tightlist
\item
  \textbf{પ્રદર્શન મેટ્રિક્સ:} CPU, મેમરી, ડિસ્ક I/O
\item
  \textbf{ઇવેન્ટ લોગ્સ:} સિસ્ટમ અને એપ્લિકેશન ઇવેન્ટ્સ
\item
  \textbf{નેટવર્ક કનેક્ટિવિટી:} પિંગ, ટ્રેસરાઉટ ટેસ્ટ્સ
\item
  \textbf{રિસોર્સ યુટિલાઇઝેશન:} કેપાસિટી પ્લાનિંગ
\end{itemize}

\textbf{VM સિક્યુરિટી:}

\begin{itemize}
\tightlist
\item
  \textbf{ગેસ્ટ OS હાર્ડનિંગ:} બિનજરૂરી સર્વિસ દૂર કરવી
\item
  \textbf{નેટવર્ક આઇસોલેશન:} VLAN સેગમેન્ટેશન
\item
  \textbf{એક્સેસ કંટ્રોલ:} યુઝર ઓથેન્ટિકેશન
\item
  \textbf{એન્ટીવાઇરસ પ્રોટેક્શન:} મેલવેર સ્કેનિંગ
\end{itemize}

\end{solutionbox}
\begin{mnemonicbox}
``CVMN'' - CPU, Virtual memory, Network, Storage

\end{mnemonicbox}

\end{document}
