\documentclass[10pt,a4paper]{article}

% content/resources/templates/preamble.tex
\usepackage[margin=0.6in]{geometry}
\author{Milav Dabgar}
\usepackage{amsmath,amssymb,amsthm}
\usepackage{booktabs}
\usepackage{multirow}
\usepackage{xcolor}
\usepackage{tcolorbox}
\tcbuselibrary{breakable,skins}
\usepackage[colorlinks=true,linkcolor=blue]{hyperref}
\usepackage{titlesec}
\usepackage{enumitem}
\usepackage{tikz}
\usepackage{pgfplots}
\usepackage{circuitikz}
\usepackage[version=4]{mhchem}
\usepackage{longtable}
\usepackage{array}
\usepackage{float}
\usepackage{caption}
\usepackage{listings}

\lstset{
  basicstyle=\small\ttfamily,
  breaklines=true,
  breakatwhitespace=false,
  postbreak=\mbox{\textcolor{red}{$\hookrightarrow$}\space},
  float=false,
  numbers=left,
  numberstyle=\tiny\color{gray},
  numbersep=10pt,
  xleftmargin=2em,
  keywordstyle=\color{blue},
  commentstyle=\color{green!60!black},
  stringstyle=\color{purple},
  backgroundcolor=\color{gray!5},
  showstringspaces=false,
  tabsize=2,
  captionpos=b,
  keepspaces=true,
  columns=flexible
}

\pgfplotsset{compat=1.18}
\usetikzlibrary{shapes,arrows,positioning,calc,patterns,decorations.pathmorphing,decorations.markings,arrows.meta}

% Color scheme
\definecolor{headcolor}{RGB}{0,102,204}
\definecolor{keycolor}{RGB}{220,20,60}
\definecolor{solutioncolor}{RGB}{34,139,34}
\definecolor{mnemoniccolor}{RGB}{148,0,211}
\definecolor{codecolor}{RGB}{0,0,100}

% Spacing
\setlength{\parskip}{3pt}
\setlist[itemize]{nosep}
\setlist[enumerate]{nosep}

% Title formatting
\titleformat{\section}{\Large\bfseries\color{headcolor}}{\thesection}{1em}{}
\titleformat{\subsection}{\large\bfseries\color{headcolor}}{\thesubsection}{1em}{}

% Pandoc tightlist compatibility
\providecommand{\tightlist}{%
  \setlength{\itemsep}{0pt}\setlength{\parskip}{0pt}}

% Pandoc longtable compatibility
\newcounter{none}
\def\thenone{}


% content/resources/templates/english-boxes.tex
% This file is currently empty - it exists to maintain consistency with the import structure.
% Add custom environments here if needed in the future.


\begin{document}

\begin{center}
{\Huge\bfseries\color{headcolor} Subject Name Solutions}\\[5pt]
{\LARGE 4361602 -- Winter 2024}\\[3pt]
{\large Semester 1 Study Material}\\[3pt]
{\normalsize\textit{Detailed Solutions and Explanations}}
\end{center}

\vspace{10pt}

\subsection*{Question 1(a) [3 marks]}\label{q1a}

\textbf{Define cloud computing and state it's desirable features.}

\begin{solutionbox}

\textbf{Cloud Computing} is a technology that delivers computing
services like servers, storage, databases, and software over the
internet, allowing users to access resources on-demand without owning
physical infrastructure.

\textbf{Desirable Features}:

{\def\LTcaptype{none} % do not increment counter
\begin{longtable}[]{@{}
  >{\raggedright\arraybackslash}p{(\linewidth - 2\tabcolsep) * \real{0.4091}}
  >{\raggedright\arraybackslash}p{(\linewidth - 2\tabcolsep) * \real{0.5909}}@{}}
\toprule\noalign{}
\begin{minipage}[b]{\linewidth}\raggedright
Feature
\end{minipage} & \begin{minipage}[b]{\linewidth}\raggedright
Description
\end{minipage} \\
\midrule\noalign{}
\endhead
\bottomrule\noalign{}
\endlastfoot
\textbf{On-demand self-service} & Users can access resources instantly
without human interaction \\
\textbf{Broad network access} & Services available over network through
standard platforms \\
\textbf{Resource pooling} & Computing resources are pooled to serve
multiple users \\
\textbf{Rapid elasticity} & Resources can be scaled up or down
quickly \\
\textbf{Measured service} & Usage is monitored and billed
automatically \\
\end{longtable}
}

\end{solutionbox}
\begin{mnemonicbox}
``On-Demand Broad Resources Rapidly Measured''

\end{mnemonicbox}
\begin{center}\rule{0.5\linewidth}{0.5pt}\end{center}

\subsection*{Question 1(b) [4 marks]}\label{q1b}

\textbf{Draw and explain cloud architecture.}

\begin{solutionbox}

\begin{center}
\textbf{Mermaid Diagram (Code)}
\begin{verbatim}
{Shaded}
{Highlighting}[]
graph LR
    A[Client Layer{br/{}Web Browser, Mobile Apps] {-}{-}{} B[Internet]}
    B {-{-}{} C[Cloud Service Provider]}
    C {-{-}{} D[Frontend Platform{}br/{}User Interface]}
    D {-{-}{} E[Backend Platform]}
    E {-{-}{} F[IaaS {-} Infrastructure]}
    E {-{-}{} G[PaaS {-} Platform]}
    E {-{-}{} H[SaaS {-} Software]}
    F {-{-}{} I[Physical Infrastructure{}br/{}Servers, Storage, Network]}
{Highlighting}
{Shaded}
\end{verbatim}
\end{center}

\textbf{Cloud Architecture Components}:

\begin{itemize}
\tightlist
\item
  \textbf{Client Layer}: End-user devices accessing cloud services
\item
  \textbf{Internet}: Network connection medium
\item
  \textbf{Frontend}: User interface and service management
\item
  \textbf{Backend}: Core processing and resource management
\item
  \textbf{Service Models}: IaaS, PaaS, SaaS layers
\item
  \textbf{Physical Infrastructure}: Hardware resources in data centers
\end{itemize}

\end{solutionbox}
\begin{mnemonicbox}
``Clients Connect Through Frontend Backend Services
Infrastructure''

\end{mnemonicbox}
\begin{center}\rule{0.5\linewidth}{0.5pt}\end{center}

\subsection*{Question 1(c) [7 marks]}\label{q1c}

\textbf{Explain the cloud service models in detail.}

\begin{solutionbox}

{\def\LTcaptype{none} % do not increment counter
\begin{longtable}[]{@{}
  >{\raggedright\arraybackslash}p{(\linewidth - 6\tabcolsep) * \real{0.2885}}
  >{\raggedright\arraybackslash}p{(\linewidth - 6\tabcolsep) * \real{0.2500}}
  >{\raggedright\arraybackslash}p{(\linewidth - 6\tabcolsep) * \real{0.1923}}
  >{\raggedright\arraybackslash}p{(\linewidth - 6\tabcolsep) * \real{0.2692}}@{}}
\toprule\noalign{}
\begin{minipage}[b]{\linewidth}\raggedright
Service Model
\end{minipage} & \begin{minipage}[b]{\linewidth}\raggedright
Description
\end{minipage} & \begin{minipage}[b]{\linewidth}\raggedright
Examples
\end{minipage} & \begin{minipage}[b]{\linewidth}\raggedright
User Control
\end{minipage} \\
\midrule\noalign{}
\endhead
\bottomrule\noalign{}
\endlastfoot
\textbf{IaaS} & Infrastructure as a Service - Virtual machines, storage,
networks & AWS EC2, Google Compute Engine & High - OS, Runtime, Apps \\
\textbf{PaaS} & Platform as a Service - Development platform with tools
& Google App Engine, Heroku & Medium - Apps and Data \\
\textbf{SaaS} & Software as a Service - Ready-to-use applications &
Gmail, Office 365, Salesforce & Low - Only Data \\
\end{longtable}
}

\textbf{Detailed Explanation}:

\begin{itemize}
\item
  \textbf{IaaS (Infrastructure as a Service)}: Provides virtualized
  computing resources including virtual machines, storage, and
  networking. Users have complete control over operating systems and
  applications.
\item
  \textbf{PaaS (Platform as a Service)}: Offers a development platform
  with programming tools, database management, and middleware.
  Developers focus on application logic without infrastructure
  management.
\item
  \textbf{SaaS (Software as a Service)}: Delivers complete applications
  over the internet. Users simply access the software through web
  browsers without installation or maintenance.
\end{itemize}

\end{solutionbox}
\begin{mnemonicbox}
``Infrastructure Platforms Software - Increasing
Abstraction''

\end{mnemonicbox}
\begin{center}\rule{0.5\linewidth}{0.5pt}\end{center}

\subsection*{Question 1(c OR) [7
marks]}\label{question-1c-or-7-marks}

\textbf{Explain service level agreement (SLA) in cloud computing with
example.}

\begin{solutionbox}

\textbf{Service Level Agreement (SLA)} is a contract between cloud
service provider and customer that defines the expected level of
service, performance metrics, and penalties for non-compliance.

\textbf{Key Components}:

{\def\LTcaptype{none} % do not increment counter
\begin{longtable}[]{@{}
  >{\raggedright\arraybackslash}p{(\linewidth - 4\tabcolsep) * \real{0.3333}}
  >{\raggedright\arraybackslash}p{(\linewidth - 4\tabcolsep) * \real{0.3939}}
  >{\raggedright\arraybackslash}p{(\linewidth - 4\tabcolsep) * \real{0.2727}}@{}}
\toprule\noalign{}
\begin{minipage}[b]{\linewidth}\raggedright
Component
\end{minipage} & \begin{minipage}[b]{\linewidth}\raggedright
Description
\end{minipage} & \begin{minipage}[b]{\linewidth}\raggedright
Example
\end{minipage} \\
\midrule\noalign{}
\endhead
\bottomrule\noalign{}
\endlastfoot
\textbf{Availability} & Uptime guarantee & 99.9\% uptime \\
\textbf{Performance} & Response time metrics & \textless200ms response
time \\
\textbf{Security} & Data protection standards & ISO 27001 compliance \\
\textbf{Support} & Help desk response time & 24/7 support, 4-hour
response \\
\textbf{Penalties} & Compensation for failures & Service credits for
downtime \\
\end{longtable}
}

\textbf{Example - AWS SLA}:

\begin{itemize}
\tightlist
\item
  \textbf{EC2 SLA}: 99.99\% monthly uptime
\item
  \textbf{S3 SLA}: 99.9\% availability, 99.999999999\% durability
\item
  \textbf{Penalty}: 10\% service credit if availability drops below
  threshold
\end{itemize}

\textbf{Benefits}:

\begin{itemize}
\tightlist
\item
  \textbf{Accountability}: Clear expectations for both parties
\item
  \textbf{Quality assurance}: Guaranteed service levels
\item
  \textbf{Risk mitigation}: Compensation for service failures
\end{itemize}

\end{solutionbox}
\begin{mnemonicbox}
``Availability Performance Security Support
Penalties''

\end{mnemonicbox}
\begin{center}\rule{0.5\linewidth}{0.5pt}\end{center}

\subsection*{Question 2(a) [3 marks]}\label{q2a}

\textbf{Define virtualization. Give characteristics of virtualization.}

\begin{solutionbox}

\textbf{Virtualization} is a technology that creates virtual versions of
computing resources like servers, storage, or networks, allowing
multiple virtual instances to run on single physical hardware.

\textbf{Characteristics}:

\begin{itemize}
\tightlist
\item
  \textbf{Resource sharing}: Multiple VMs share physical hardware
  efficiently
\item
  \textbf{Isolation}: Virtual machines operate independently without
  interference
\item
  \textbf{Portability}: VMs can be moved between different physical
  hosts
\item
  \textbf{Scalability}: Resources can be allocated dynamically as needed
\item
  \textbf{Cost efficiency}: Reduces hardware requirements and
  operational costs
\end{itemize}

\end{solutionbox}
\begin{mnemonicbox}
``Resources Isolated Portable Scalable
Cost-effective''

\end{mnemonicbox}
\begin{center}\rule{0.5\linewidth}{0.5pt}\end{center}

\subsection*{Question 2(b) [4 marks]}\label{q2b}

\textbf{Distinguish between paravirtualization and full virtualization.}

\begin{solutionbox}

{\def\LTcaptype{none} % do not increment counter
\begin{longtable}[]{@{}
  >{\raggedright\arraybackslash}p{(\linewidth - 4\tabcolsep) * \real{0.1739}}
  >{\raggedright\arraybackslash}p{(\linewidth - 4\tabcolsep) * \real{0.4130}}
  >{\raggedright\arraybackslash}p{(\linewidth - 4\tabcolsep) * \real{0.4130}}@{}}
\toprule\noalign{}
\begin{minipage}[b]{\linewidth}\raggedright
Aspect
\end{minipage} & \begin{minipage}[b]{\linewidth}\raggedright
Paravirtualization
\end{minipage} & \begin{minipage}[b]{\linewidth}\raggedright
Full Virtualization
\end{minipage} \\
\midrule\noalign{}
\endhead
\bottomrule\noalign{}
\endlastfoot
\textbf{Guest OS Modification} & Modified to communicate with hypervisor
& No modification needed \\
\textbf{Performance} & Higher performance & Slightly lower
performance \\
\textbf{Hardware Support} & Doesn't require special hardware & Requires
hardware virtualization support \\
\textbf{Compatibility} & Limited OS compatibility & Supports any OS \\
\textbf{Examples} & Xen, VMware ESX & VMware Workstation, VirtualBox \\
\end{longtable}
}

\textbf{Key Differences}:

\begin{itemize}
\tightlist
\item
  \textbf{Paravirtualization} requires guest OS to be aware of
  virtualization and cooperate with hypervisor
\item
  \textbf{Full Virtualization} completely emulates hardware, making
  guest OS unaware of virtualization
\end{itemize}

\end{solutionbox}
\begin{mnemonicbox}
``Para Cooperates, Full Emulates''

\end{mnemonicbox}
\begin{center}\rule{0.5\linewidth}{0.5pt}\end{center}

\subsection*{Question 2(c) [7 marks]}\label{q2c}

\textbf{Define hypervisors. Explain Type 1 and Type 2 hypervisors.}

\begin{solutionbox}

\textbf{Hypervisor} is software that creates and manages virtual
machines by abstracting physical hardware and allocating resources to
multiple VMs.

\begin{verbatim}
graph TB
    subgraph "Type 1 Hypervisor"
        A1[VM1] 
        A2[VM2]
        A3[VM3]
        A4[Type 1 Hypervisor{br/Bare Metal]}
        A5[Physical Hardware]
        A1 {-{-} A4}
        A2 {-{-} A4}
        A3 {-{-} A4}
        A4 {-{-} A5}
    end
    
    subgraph "Type 2 Hypervisor"
        B1[VM1]
        B2[VM2]
        B3[Type 2 Hypervisor{br/Hosted]}
        B4[Host Operating System]
        B5[Physical Hardware]
        B1 {-{-} B3}
        B2 {-{-} B3}
        B3 {-{-} B4}
        B4 {-{-} B5}
    end
\end{verbatim}

\textbf{Comparison}:

{\def\LTcaptype{none} % do not increment counter
\begin{longtable}[]{@{}
  >{\raggedright\arraybackslash}p{(\linewidth - 4\tabcolsep) * \real{0.2000}}
  >{\raggedright\arraybackslash}p{(\linewidth - 4\tabcolsep) * \real{0.4222}}
  >{\raggedright\arraybackslash}p{(\linewidth - 4\tabcolsep) * \real{0.3778}}@{}}
\toprule\noalign{}
\begin{minipage}[b]{\linewidth}\raggedright
Feature
\end{minipage} & \begin{minipage}[b]{\linewidth}\raggedright
Type 1 (Bare Metal)
\end{minipage} & \begin{minipage}[b]{\linewidth}\raggedright
Type 2 (Hosted)
\end{minipage} \\
\midrule\noalign{}
\endhead
\bottomrule\noalign{}
\endlastfoot
\textbf{Installation} & Directly on hardware & On host operating
system \\
\textbf{Performance} & Higher performance & Lower performance \\
\textbf{Use Case} & Enterprise, data centers & Desktop virtualization,
testing \\
\textbf{Examples} & VMware vSphere, Hyper-V & VMware Workstation,
VirtualBox \\
\textbf{Resource Overhead} & Lower overhead & Higher overhead \\
\end{longtable}
}

\textbf{Type 1 Advantages}: Better performance, direct hardware access,
enterprise-grade security \textbf{Type 2 Advantages}: Easier setup, runs
alongside host OS, good for development

\end{solutionbox}
\begin{mnemonicbox}
``Type 1 Bare Metal, Type 2 Hosted''

\end{mnemonicbox}
\begin{center}\rule{0.5\linewidth}{0.5pt}\end{center}

\subsection*{Question 2(a OR) [3
marks]}\label{question-2a-or-3-marks}

\textbf{List out types of virtualization and explain any one in brief.}

\begin{solutionbox}

\textbf{Types of Virtualization}:

\begin{itemize}
\tightlist
\item
  Server Virtualization
\item
  Storage Virtualization\\
\item
  Network Virtualization
\item
  Desktop Virtualization
\item
  Application Virtualization
\item
  Memory Virtualization
\end{itemize}

\textbf{Server Virtualization (Detailed)}: Server virtualization creates
multiple virtual servers on single physical server. Each virtual server
operates independently with its own operating system and applications.

\textbf{Benefits}:

\begin{itemize}
\tightlist
\item
  \textbf{Resource optimization}: Better hardware utilization
\item
  \textbf{Cost reduction}: Fewer physical servers needed
\item
  \textbf{Flexibility}: Easy VM migration and scaling
\end{itemize}

\end{solutionbox}
\begin{mnemonicbox}
``Server Storage Network Desktop Application Memory''

\end{mnemonicbox}
\begin{center}\rule{0.5\linewidth}{0.5pt}\end{center}

\subsection*{Question 2(b OR) [4
marks]}\label{question-2b-or-4-marks}

\textbf{Describe hardware and software virtualization.}

\begin{solutionbox}

{\def\LTcaptype{none} % do not increment counter
\begin{longtable}[]{@{}
  >{\raggedright\arraybackslash}p{(\linewidth - 4\tabcolsep) * \real{0.1071}}
  >{\raggedright\arraybackslash}p{(\linewidth - 4\tabcolsep) * \real{0.4286}}
  >{\raggedright\arraybackslash}p{(\linewidth - 4\tabcolsep) * \real{0.4643}}@{}}
\toprule\noalign{}
\begin{minipage}[b]{\linewidth}\raggedright
Type
\end{minipage} & \begin{minipage}[b]{\linewidth}\raggedright
Hardware Virtualization
\end{minipage} & \begin{minipage}[b]{\linewidth}\raggedright
Software Virtualization
\end{minipage} \\
\midrule\noalign{}
\endhead
\bottomrule\noalign{}
\endlastfoot
\textbf{Method} & Uses CPU virtualization features & Pure software
emulation \\
\textbf{Performance} & Near-native performance & Slower due to
emulation \\
\textbf{CPU Support} & Requires Intel VT-x or AMD-V & Works on any
CPU \\
\textbf{Guest OS} & Unmodified OS can run & May require OS
modifications \\
\textbf{Examples} & VMware vSphere, KVM & QEMU, VMware Workstation
(software mode) \\
\end{longtable}
}

\textbf{Hardware Virtualization}: Leverages CPU virtualization
extensions to directly execute guest instructions, providing better
performance and security isolation.

\textbf{Software Virtualization}: Uses binary translation to convert
guest instructions to host-compatible instructions, offering broader
compatibility but with performance overhead.

\end{solutionbox}
\begin{mnemonicbox}
``Hardware Fast, Software Compatible''

\end{mnemonicbox}
\begin{center}\rule{0.5\linewidth}{0.5pt}\end{center}

\subsection*{Question 2(c OR) [7
marks]}\label{question-2c-or-7-marks}

\textbf{Explain the process of creating and managing virtual machines.}

\begin{solutionbox}

\textbf{VM Creation Process}:

\begin{verbatim}
flowchart TD
    A[Plan VM Requirements] {-{-} B[Select Hypervisor Platform]}
    B {-{-} C[Allocate Resourcesbr/CPU, RAM, Storage]}
    C {-{-} D[Create Virtual Disk]}
    D {-{-} E[Configure Network Settings]}
    E {-{-} F[Install Guest OS]}
    F {-{-} G[Install VM Tools/Drivers]}
    G {-{-} H[Configure VM Settings]}
    H {-{-} I[Create VM Snapshot]}
\end{verbatim}

\textbf{Detailed Steps}:

\begin{enumerate}
\tightlist
\item
  \textbf{Planning}: Determine CPU cores, RAM, storage, and network
  requirements
\item
  \textbf{Resource Allocation}: Assign physical resources to virtual
  machine
\item
  \textbf{Storage Setup}: Create virtual disks (VMDK, VHD, QCOW2
  formats)
\item
  \textbf{Network Configuration}: Set up virtual network adapters and
  connectivity
\item
  \textbf{OS Installation}: Install operating system using ISO or
  network boot
\item
  \textbf{Tools Installation}: Install hypervisor-specific tools for
  better integration
\item
  \textbf{Management Tasks}: Monitor performance, create snapshots,
  backup VMs
\end{enumerate}

\textbf{VM Management Operations}:

\begin{itemize}
\tightlist
\item
  \textbf{Start/Stop/Restart}: Power operations
\item
  \textbf{Snapshot Management}: Create, restore, delete snapshots
\item
  \textbf{Resource Scaling}: Add/remove CPU, memory, storage
\item
  \textbf{Migration}: Move VMs between hosts
\item
  \textbf{Backup/Recovery}: Data protection strategies
\end{itemize}

\end{solutionbox}
\begin{mnemonicbox}
``Plan Select Allocate Create Configure Install
Manage''

\end{mnemonicbox}
\begin{center}\rule{0.5\linewidth}{0.5pt}\end{center}

\subsection*{Question 3(a) [3 marks]}\label{q3a}

\textbf{Define Data Center. Describe any two types of data centers.}

\begin{solutionbox}

\textbf{Data Center} is a facility that houses computer systems,
networking equipment, and storage systems along with supporting
infrastructure like power, cooling, and security systems.

\textbf{Types of Data Centers}:

{\def\LTcaptype{none} % do not increment counter
\begin{longtable}[]{@{}
  >{\raggedright\arraybackslash}p{(\linewidth - 4\tabcolsep) * \real{0.1667}}
  >{\raggedright\arraybackslash}p{(\linewidth - 4\tabcolsep) * \real{0.3611}}
  >{\raggedright\arraybackslash}p{(\linewidth - 4\tabcolsep) * \real{0.4722}}@{}}
\toprule\noalign{}
\begin{minipage}[b]{\linewidth}\raggedright
Type
\end{minipage} & \begin{minipage}[b]{\linewidth}\raggedright
Description
\end{minipage} & \begin{minipage}[b]{\linewidth}\raggedright
Characteristics
\end{minipage} \\
\midrule\noalign{}
\endhead
\bottomrule\noalign{}
\endlastfoot
\textbf{Enterprise Data Center} & Owned and operated by single
organization & Private, customized, high security \\
\textbf{Colocation Data Center} & Shared facility renting space to
multiple clients & Shared infrastructure, cost-effective \\
\end{longtable}
}

\textbf{Enterprise Data Center}:

\begin{itemize}
\tightlist
\item
  Built and managed by organization for internal use
\item
  Complete control over infrastructure and security
\item
  Higher initial investment but customized solutions
\end{itemize}

\textbf{Colocation Data Center}:

\begin{itemize}
\tightlist
\item
  Third-party facility providing space, power, and cooling
\item
  Multiple organizations share common infrastructure
\item
  Lower costs and professional management
\end{itemize}

\end{solutionbox}
\begin{mnemonicbox}
``Enterprise Private, Colocation Shared''

\end{mnemonicbox}
\begin{center}\rule{0.5\linewidth}{0.5pt}\end{center}

\subsection*{Question 3(b) [4 marks]}\label{q3b}

\textbf{Differentiate between scalability and elasticity in cloud data
center.}

\begin{solutionbox}

{\def\LTcaptype{none} % do not increment counter
\begin{longtable}[]{@{}
  >{\raggedright\arraybackslash}p{(\linewidth - 4\tabcolsep) * \real{0.2500}}
  >{\raggedright\arraybackslash}p{(\linewidth - 4\tabcolsep) * \real{0.3750}}
  >{\raggedright\arraybackslash}p{(\linewidth - 4\tabcolsep) * \real{0.3750}}@{}}
\toprule\noalign{}
\begin{minipage}[b]{\linewidth}\raggedright
Aspect
\end{minipage} & \begin{minipage}[b]{\linewidth}\raggedright
Scalability
\end{minipage} & \begin{minipage}[b]{\linewidth}\raggedright
Elasticity
\end{minipage} \\
\midrule\noalign{}
\endhead
\bottomrule\noalign{}
\endlastfoot
\textbf{Definition} & Ability to handle increased workload & Automatic
scaling based on demand \\
\textbf{Response} & Manual or planned scaling & Automatic and rapid
response \\
\textbf{Direction} & Usually upward scaling & Both up and down
scaling \\
\textbf{Time Frame} & Long-term capacity planning & Real-time demand
response \\
\textbf{Resource Usage} & May have unused resources & Optimal resource
utilization \\
\end{longtable}
}

\textbf{Key Differences}:

\begin{itemize}
\tightlist
\item
  \textbf{Scalability} focuses on capacity to grow, while
  \textbf{Elasticity} emphasizes automatic adjustment
\item
  \textbf{Scalability} requires human intervention, \textbf{Elasticity}
  is automated
\item
  \textbf{Scalability} is strategic planning, \textbf{Elasticity} is
  operational efficiency
\end{itemize}

\textbf{Examples}:

\begin{itemize}
\tightlist
\item
  \textbf{Scalability}: Adding more servers during expected traffic
  increase
\item
  \textbf{Elasticity}: Auto-scaling groups that add/remove instances
  based on CPU usage
\end{itemize}

\end{solutionbox}
\begin{mnemonicbox}
``Scalability Plans, Elasticity Adapts''

\end{mnemonicbox}
\begin{center}\rule{0.5\linewidth}{0.5pt}\end{center}

\subsection*{Question 3(c) [7 marks]}\label{q3c}

\textbf{Explain SDN (Software-Defined Networking) in data center with
diagram.}

\begin{solutionbox}

\begin{verbatim}
graph TB
    subgraph "SDN Architecture"
        A[Applications Layer{br/Network Apps, Services]}
        B[Control Layer{br/SDN Controllerbr/OpenFlow Protocol]}
        C[Infrastructure Layer{br/OpenFlow Switches]}
        D[Physical Network Infrastructure]
        
        A {-.{-}|Northbound API| B}
        B {-.{-}|Southbound APIbr/OpenFlow| C}
        C {-{-} D}
    end
\end{verbatim}

\textbf{SDN Components}:

{\def\LTcaptype{none} % do not increment counter
\begin{longtable}[]{@{}
  >{\raggedright\arraybackslash}p{(\linewidth - 4\tabcolsep) * \real{0.2593}}
  >{\raggedright\arraybackslash}p{(\linewidth - 4\tabcolsep) * \real{0.3704}}
  >{\raggedright\arraybackslash}p{(\linewidth - 4\tabcolsep) * \real{0.3704}}@{}}
\toprule\noalign{}
\begin{minipage}[b]{\linewidth}\raggedright
Layer
\end{minipage} & \begin{minipage}[b]{\linewidth}\raggedright
Function
\end{minipage} & \begin{minipage}[b]{\linewidth}\raggedright
Examples
\end{minipage} \\
\midrule\noalign{}
\endhead
\bottomrule\noalign{}
\endlastfoot
\textbf{Application Layer} & Network applications and services & Load
balancers, firewalls, monitoring \\
\textbf{Control Layer} & Centralized network control and management &
OpenDaylight, ONOS, Floodlight \\
\textbf{Infrastructure Layer} & Forwarding devices controlled by
controller & OpenFlow switches, routers \\
\end{longtable}
}

\textbf{Key Features}:

\begin{itemize}
\tightlist
\item
  \textbf{Centralized Control}: Single point of network management
\item
  \textbf{Programmability}: Network behavior defined through software
\item
  \textbf{Abstraction}: Separation of control and data planes
\item
  \textbf{Dynamic Configuration}: Real-time network policy changes
\end{itemize}

\textbf{Benefits in Data Centers}:

\begin{itemize}
\tightlist
\item
  \textbf{Flexibility}: Easy network configuration changes
\item
  \textbf{Automation}: Programmable network management
\item
  \textbf{Cost Reduction}: Commodity hardware usage
\item
  \textbf{Innovation}: Rapid deployment of new services
\end{itemize}

\end{solutionbox}
\begin{mnemonicbox}
``Applications Control Infrastructure - Programmable
Networks''

\end{mnemonicbox}
\begin{center}\rule{0.5\linewidth}{0.5pt}\end{center}

\subsection*{Question 3(a OR) [3
marks]}\label{question-3a-or-3-marks}

\textbf{Identify and describe the key components of a data center.}

\begin{solutionbox}

\textbf{Key Data Center Components}:

\begin{itemize}
\tightlist
\item
  \textbf{Servers}: Computing resources running applications and
  services
\item
  \textbf{Storage Systems}: Data storage arrays (SAN, NAS, DAS)
\item
  \textbf{Network Equipment}: Switches, routers, load balancers for
  connectivity
\item
  \textbf{Power Infrastructure}: UPS, generators, PDUs for reliable
  power
\item
  \textbf{Cooling Systems}: HVAC systems maintaining optimal temperature
\item
  \textbf{Security Systems}: Physical and logical access controls
\end{itemize}

\textbf{Critical Infrastructure}: Each component is essential for data
center operation, with redundancy built-in for high availability and
disaster recovery.

\end{solutionbox}
\begin{mnemonicbox}
``Servers Store Network Power Cool Secure''

\end{mnemonicbox}
\begin{center}\rule{0.5\linewidth}{0.5pt}\end{center}

\subsection*{Question 3(b OR) [4
marks]}\label{question-3b-or-4-marks}

\textbf{List data center network topologies and explain any one of
them.}

\begin{solutionbox}

\textbf{Data Center Network Topologies}:

\begin{itemize}
\tightlist
\item
  Three-tier Architecture
\item
  Spine-Leaf Architecture\\
\item
  Fat Tree Topology
\item
  Mesh Topology
\end{itemize}

\textbf{Spine-Leaf Architecture (Detailed)}:

\begin{verbatim}
    +{-{-}{-}{-}{-}{-}{-}+    +{-}{-}{-}{-}{-}{-}{-}+    +{-}{-}{-}{-}{-}{-}{-}+}
    | Leaf1 |    | Leaf2 |    | Leaf3 |
    +{-{-}{-}{-}{-}{-}{-}+    +{-}{-}{-}{-}{-}{-}{-}+    +{-}{-}{-}{-}{-}{-}{-}+}
       | |         | |         | |
       | +{-{-}{-}{-}{-}{-}{-}{-}{-}+ +{-}{-}{-}{-}{-}{-}{-}{-}{-}+ |}
       |             |           |
    +{-{-}{-}{-}{-}{-}{-}+    +{-}{-}{-}{-}{-}{-}{-}+    +{-}{-}{-}{-}{-}{-}{-}+}
    |Spine1 |    |Spine2 |    |Spine3 |
    +{-{-}{-}{-}{-}{-}{-}+    +{-}{-}{-}{-}{-}{-}{-}+    +{-}{-}{-}{-}{-}{-}{-}+}
\end{verbatim}

\textbf{Characteristics}:

\begin{itemize}
\tightlist
\item
  \textbf{Leaf switches} connect to servers and storage
\item
  \textbf{Spine switches} provide inter-leaf connectivity\\
\item
  \textbf{No leaf-to-leaf connections} - all traffic goes through spine
\item
  \textbf{Equal path lengths} between any two endpoints
\item
  \textbf{High bandwidth} and \textbf{low latency} design
\end{itemize}

\end{solutionbox}
\begin{mnemonicbox}
``Three Spine Fat Mesh''

\end{mnemonicbox}
\begin{center}\rule{0.5\linewidth}{0.5pt}\end{center}

\subsection*{Question 3(c OR) [7
marks]}\label{question-3c-or-7-marks}

\textbf{Explain Infrastructure as Code (IaC) with its popular automation
tools.}

\begin{solutionbox}

\textbf{Infrastructure as Code (IaC)} is the practice of managing and
provisioning computing infrastructure through machine-readable
definition files rather than manual processes.

\textbf{Key Principles}:

{\def\LTcaptype{none} % do not increment counter
\begin{longtable}[]{@{}
  >{\raggedright\arraybackslash}p{(\linewidth - 4\tabcolsep) * \real{0.3235}}
  >{\raggedright\arraybackslash}p{(\linewidth - 4\tabcolsep) * \real{0.3824}}
  >{\raggedright\arraybackslash}p{(\linewidth - 4\tabcolsep) * \real{0.2941}}@{}}
\toprule\noalign{}
\begin{minipage}[b]{\linewidth}\raggedright
Principle
\end{minipage} & \begin{minipage}[b]{\linewidth}\raggedright
Description
\end{minipage} & \begin{minipage}[b]{\linewidth}\raggedright
Benefits
\end{minipage} \\
\midrule\noalign{}
\endhead
\bottomrule\noalign{}
\endlastfoot
\textbf{Declarative} & Define desired state, not steps & Predictable
outcomes \\
\textbf{Version Control} & Infrastructure definitions in Git & Change
tracking, rollback \\
\textbf{Automation} & Automated deployment and updates & Reduced human
errors \\
\textbf{Consistency} & Same configuration across environments & Reliable
deployments \\
\end{longtable}
}

\textbf{Popular IaC Tools}:

{\def\LTcaptype{none} % do not increment counter
\begin{longtable}[]{@{}
  >{\raggedright\arraybackslash}p{(\linewidth - 6\tabcolsep) * \real{0.1714}}
  >{\raggedright\arraybackslash}p{(\linewidth - 6\tabcolsep) * \real{0.1714}}
  >{\raggedright\arraybackslash}p{(\linewidth - 6\tabcolsep) * \real{0.3714}}
  >{\raggedright\arraybackslash}p{(\linewidth - 6\tabcolsep) * \real{0.2857}}@{}}
\toprule\noalign{}
\begin{minipage}[b]{\linewidth}\raggedright
Tool
\end{minipage} & \begin{minipage}[b]{\linewidth}\raggedright
Type
\end{minipage} & \begin{minipage}[b]{\linewidth}\raggedright
Description
\end{minipage} & \begin{minipage}[b]{\linewidth}\raggedright
Use Case
\end{minipage} \\
\midrule\noalign{}
\endhead
\bottomrule\noalign{}
\endlastfoot
\textbf{Terraform} & Declarative & Multi-cloud infrastructure
provisioning & Cross-platform deployments \\
\textbf{Ansible} & Imperative & Configuration management and automation
& Server configuration \\
\textbf{CloudFormation} & Declarative & AWS-specific infrastructure
templates & AWS resource management \\
\textbf{Puppet} & Declarative & Configuration management & Enterprise
automation \\
\textbf{Chef} & Imperative & Infrastructure automation platform &
Complex deployments \\
\end{longtable}
}

\textbf{IaC Benefits}:

\begin{itemize}
\tightlist
\item
  \textbf{Speed}: Faster deployment and scaling
\item
  \textbf{Consistency}: Identical environments across stages
\item
  \textbf{Cost Control}: Resource optimization and tracking
\item
  \textbf{Reliability}: Reduced configuration drift
\item
  \textbf{Collaboration}: Shared infrastructure definitions
\end{itemize}

\textbf{Implementation Example}:

\begin{verbatim}
# Terraform example
resource "aws_instance" "web_server" {
  ami           = "ami-12345678"
  instance_type = "t2.micro"
  tags = {
    Name = "WebServer"
  }
}
\end{verbatim}

\end{solutionbox}
\begin{mnemonicbox}
``Terraform Ansible CloudFormation Puppet Chef''

\end{mnemonicbox}
\begin{center}\rule{0.5\linewidth}{0.5pt}\end{center}

\subsection*{Question 4(a) [3 marks]}\label{q4a}

\textbf{Define cloud storage. Write example of cloud storage services.}

\begin{solutionbox}

\textbf{Cloud Storage} is a service that allows users to store, access,
and manage data on remote servers over the internet instead of local
storage devices.

\textbf{Examples of Cloud Storage Services}:

{\def\LTcaptype{none} % do not increment counter
\begin{longtable}[]{@{}
  >{\raggedright\arraybackslash}p{(\linewidth - 6\tabcolsep) * \real{0.2857}}
  >{\raggedright\arraybackslash}p{(\linewidth - 6\tabcolsep) * \real{0.2571}}
  >{\raggedright\arraybackslash}p{(\linewidth - 6\tabcolsep) * \real{0.1714}}
  >{\raggedright\arraybackslash}p{(\linewidth - 6\tabcolsep) * \real{0.2857}}@{}}
\toprule\noalign{}
\begin{minipage}[b]{\linewidth}\raggedright
Provider
\end{minipage} & \begin{minipage}[b]{\linewidth}\raggedright
Service
\end{minipage} & \begin{minipage}[b]{\linewidth}\raggedright
Type
\end{minipage} & \begin{minipage}[b]{\linewidth}\raggedright
Use Case
\end{minipage} \\
\midrule\noalign{}
\endhead
\bottomrule\noalign{}
\endlastfoot
\textbf{Amazon} & S3 (Simple Storage Service) & Object Storage & Web
applications, backup \\
\textbf{Google} & Google Drive & File Storage & Personal,
collaboration \\
\textbf{Microsoft} & Azure Blob Storage & Object Storage & Enterprise
applications \\
\textbf{Dropbox} & Dropbox & File Sync & File sharing, sync \\
\textbf{iCloud} & Apple iCloud & Personal Cloud & iOS device backup \\
\end{longtable}
}

\textbf{Key Benefits}: Accessibility, scalability, cost-effectiveness,
automatic backup

\end{solutionbox}
\begin{mnemonicbox}
``Amazon Google Microsoft Dropbox Apple''

\end{mnemonicbox}
\begin{center}\rule{0.5\linewidth}{0.5pt}\end{center}

\subsection*{Question 4(b) [4 marks]}\label{q4b}

\textbf{Differentiate between data consistency and durability.}

\begin{solutionbox}

{\def\LTcaptype{none} % do not increment counter
\begin{longtable}[]{@{}
  >{\raggedright\arraybackslash}p{(\linewidth - 4\tabcolsep) * \real{0.1860}}
  >{\raggedright\arraybackslash}p{(\linewidth - 4\tabcolsep) * \real{0.4186}}
  >{\raggedright\arraybackslash}p{(\linewidth - 4\tabcolsep) * \real{0.3953}}@{}}
\toprule\noalign{}
\begin{minipage}[b]{\linewidth}\raggedright
Aspect
\end{minipage} & \begin{minipage}[b]{\linewidth}\raggedright
Data Consistency
\end{minipage} & \begin{minipage}[b]{\linewidth}\raggedright
Data Durability
\end{minipage} \\
\midrule\noalign{}
\endhead
\bottomrule\noalign{}
\endlastfoot
\textbf{Definition} & All nodes see same data simultaneously & Data
persists despite system failures \\
\textbf{Focus} & Data accuracy and synchronization & Data preservation
and recovery \\
\textbf{Challenge} & Concurrent access conflicts & Hardware failures,
disasters \\
\textbf{Solutions} & ACID properties, eventual consistency &
Replication, backups, redundancy \\
\textbf{Examples} & Bank transactions, inventory updates & File backups,
disaster recovery \\
\end{longtable}
}

\textbf{Data Consistency}: Ensures all database nodes contain identical
data at any given time, crucial for applications requiring real-time
accuracy.

\textbf{Data Durability}: Guarantees that committed data remains
available even after system crashes, power failures, or hardware
malfunctions.

\textbf{Trade-offs}: Strong consistency may impact performance, while
high durability requires additional storage costs.

\end{solutionbox}
\begin{mnemonicbox}
``Consistency Synchronizes, Durability Survives''

\end{mnemonicbox}
\begin{center}\rule{0.5\linewidth}{0.5pt}\end{center}

\subsection*{Question 4(c) [7 marks]}\label{q4c}

\textbf{Explain types of cloud storage in detail.}

\begin{solutionbox}

{\def\LTcaptype{none} % do not increment counter
\begin{longtable}[]{@{}
  >{\raggedright\arraybackslash}p{(\linewidth - 6\tabcolsep) * \real{0.2917}}
  >{\raggedright\arraybackslash}p{(\linewidth - 6\tabcolsep) * \real{0.2708}}
  >{\raggedright\arraybackslash}p{(\linewidth - 6\tabcolsep) * \real{0.2292}}
  >{\raggedright\arraybackslash}p{(\linewidth - 6\tabcolsep) * \real{0.2083}}@{}}
\toprule\noalign{}
\begin{minipage}[b]{\linewidth}\raggedright
Storage Type
\end{minipage} & \begin{minipage}[b]{\linewidth}\raggedright
Description
\end{minipage} & \begin{minipage}[b]{\linewidth}\raggedright
Use Cases
\end{minipage} & \begin{minipage}[b]{\linewidth}\raggedright
Examples
\end{minipage} \\
\midrule\noalign{}
\endhead
\bottomrule\noalign{}
\endlastfoot
\textbf{Object Storage} & Stores files as objects with metadata & Web
apps, content distribution & Amazon S3, Google Cloud Storage \\
\textbf{Block Storage} & Raw block-level storage for databases &
High-performance databases & Amazon EBS, Azure Disk \\
\textbf{File Storage} & Traditional hierarchical file system & File
sharing, content management & Amazon EFS, Azure Files \\
\end{longtable}
}

\textbf{Detailed Explanation}:

\textbf{Object Storage}:

\begin{itemize}
\tightlist
\item
  \textbf{Structure}: Flat namespace with unique object identifiers
\item
  \textbf{Scalability}: Virtually unlimited capacity
\item
  \textbf{Access}: REST APIs, web interfaces
\item
  \textbf{Benefits}: Cost-effective, globally accessible, metadata
  support
\end{itemize}

\textbf{Block Storage}:

\begin{itemize}
\tightlist
\item
  \textbf{Structure}: Raw storage blocks attached to compute instances
\item
  \textbf{Performance}: High IOPS, low latency
\item
  \textbf{Access}: Direct block-level access
\item
  \textbf{Benefits}: High performance, database optimization
\end{itemize}

\textbf{File Storage}:

\begin{itemize}
\tightlist
\item
  \textbf{Structure}: Traditional directory/folder hierarchy
\item
  \textbf{Sharing}: Multi-user concurrent access
\item
  \textbf{Access}: Standard file system protocols (NFS, SMB)
\item
  \textbf{Benefits}: Familiar interface, application compatibility
\end{itemize}

\textbf{Selection Criteria}:

\begin{itemize}
\tightlist
\item
  \textbf{Performance requirements}: Block for databases, Object for web
\item
  \textbf{Access patterns}: File for shared access, Object for web apps
\item
  \textbf{Cost considerations}: Object cheapest, Block most expensive
\end{itemize}

\end{solutionbox}
\begin{mnemonicbox}
``Objects Scale, Blocks Perform, Files Share''

\end{mnemonicbox}
\begin{center}\rule{0.5\linewidth}{0.5pt}\end{center}

\subsection*{Question 4(a OR) [3
marks]}\label{question-4a-or-3-marks}

\textbf{Define cloud databases. Write example of cloud database
services.}

\begin{solutionbox}

\textbf{Cloud Databases} are database services hosted and managed by
cloud providers, offering scalability, high availability, and reduced
administration overhead.

\textbf{Examples of Cloud Database Services}:

{\def\LTcaptype{none} % do not increment counter
\begin{longtable}[]{@{}
  >{\raggedright\arraybackslash}p{(\linewidth - 6\tabcolsep) * \real{0.2857}}
  >{\raggedright\arraybackslash}p{(\linewidth - 6\tabcolsep) * \real{0.2571}}
  >{\raggedright\arraybackslash}p{(\linewidth - 6\tabcolsep) * \real{0.1714}}
  >{\raggedright\arraybackslash}p{(\linewidth - 6\tabcolsep) * \real{0.2857}}@{}}
\toprule\noalign{}
\begin{minipage}[b]{\linewidth}\raggedright
Provider
\end{minipage} & \begin{minipage}[b]{\linewidth}\raggedright
Service
\end{minipage} & \begin{minipage}[b]{\linewidth}\raggedright
Type
\end{minipage} & \begin{minipage}[b]{\linewidth}\raggedright
Features
\end{minipage} \\
\midrule\noalign{}
\endhead
\bottomrule\noalign{}
\endlastfoot
\textbf{Amazon} & RDS (Relational Database Service) & SQL & MySQL,
PostgreSQL, Oracle \\
\textbf{Google} & Cloud SQL & SQL & Managed MySQL, PostgreSQL \\
\textbf{Microsoft} & Azure SQL Database & SQL & SQL Server in cloud \\
\textbf{MongoDB} & Atlas & NoSQL & Managed MongoDB \\
\textbf{Amazon} & DynamoDB & NoSQL & Key-value, document store \\
\end{longtable}
}

\textbf{Benefits}: Automatic scaling, backup management, security
updates, global availability

\end{solutionbox}
\begin{mnemonicbox}
``Amazon Google Microsoft MongoDB''

\end{mnemonicbox}
\begin{center}\rule{0.5\linewidth}{0.5pt}\end{center}

\subsection*{Question 4(b OR) [4
marks]}\label{question-4b-or-4-marks}

\textbf{Describe data scaling and replication.}

\begin{solutionbox}

\textbf{Data Scaling}:

{\def\LTcaptype{none} % do not increment counter
\begin{longtable}[]{@{}
  >{\raggedright\arraybackslash}p{(\linewidth - 6\tabcolsep) * \real{0.3111}}
  >{\raggedright\arraybackslash}p{(\linewidth - 6\tabcolsep) * \real{0.2889}}
  >{\raggedright\arraybackslash}p{(\linewidth - 6\tabcolsep) * \real{0.1778}}
  >{\raggedright\arraybackslash}p{(\linewidth - 6\tabcolsep) * \real{0.2222}}@{}}
\toprule\noalign{}
\begin{minipage}[b]{\linewidth}\raggedright
Scaling Type
\end{minipage} & \begin{minipage}[b]{\linewidth}\raggedright
Description
\end{minipage} & \begin{minipage}[b]{\linewidth}\raggedright
Method
\end{minipage} & \begin{minipage}[b]{\linewidth}\raggedright
Benefits
\end{minipage} \\
\midrule\noalign{}
\endhead
\bottomrule\noalign{}
\endlastfoot
\textbf{Vertical Scaling} & Increase server capacity & Add CPU, RAM,
storage & Simple, no code changes \\
\textbf{Horizontal Scaling} & Add more servers & Distribute across nodes
& Better fault tolerance \\
\end{longtable}
}

\textbf{Data Replication}:

{\def\LTcaptype{none} % do not increment counter
\begin{longtable}[]{@{}
  >{\raggedright\arraybackslash}p{(\linewidth - 6\tabcolsep) * \real{0.3333}}
  >{\raggedright\arraybackslash}p{(\linewidth - 6\tabcolsep) * \real{0.2407}}
  >{\raggedright\arraybackslash}p{(\linewidth - 6\tabcolsep) * \real{0.1852}}
  >{\raggedright\arraybackslash}p{(\linewidth - 6\tabcolsep) * \real{0.2407}}@{}}
\toprule\noalign{}
\begin{minipage}[b]{\linewidth}\raggedright
Replication Type
\end{minipage} & \begin{minipage}[b]{\linewidth}\raggedright
Description
\end{minipage} & \begin{minipage}[b]{\linewidth}\raggedright
Use Case
\end{minipage} & \begin{minipage}[b]{\linewidth}\raggedright
Consistency
\end{minipage} \\
\midrule\noalign{}
\endhead
\bottomrule\noalign{}
\endlastfoot
\textbf{Master-Slave} & One write node, multiple read nodes & Read-heavy
workloads & Eventual consistency \\
\textbf{Master-Master} & Multiple write nodes & High availability &
Conflict resolution needed \\
\textbf{Peer-to-Peer} & All nodes equal & Distributed systems & Complex
consistency \\
\end{longtable}
}

\textbf{Key Benefits}:

\begin{itemize}
\tightlist
\item
  \textbf{Scaling}: Handle increased load and data volume
\item
  \textbf{Replication}: Improve availability and disaster recovery
\item
  \textbf{Performance}: Distribute load across multiple systems
\item
  \textbf{Fault Tolerance}: Continue operations despite failures
\end{itemize}

\end{solutionbox}
\begin{mnemonicbox}
``Vertical Horizontal, Master Slave Peer''

\end{mnemonicbox}
\begin{center}\rule{0.5\linewidth}{0.5pt}\end{center}

\subsection*{Question 4(c OR) [7
marks]}\label{question-4c-or-7-marks}

\textbf{Explain types of cloud databases.}

\begin{solutionbox}

{\def\LTcaptype{none} % do not increment counter
\begin{longtable}[]{@{}
  >{\raggedright\arraybackslash}p{(\linewidth - 6\tabcolsep) * \real{0.3061}}
  >{\raggedright\arraybackslash}p{(\linewidth - 6\tabcolsep) * \real{0.2653}}
  >{\raggedright\arraybackslash}p{(\linewidth - 6\tabcolsep) * \real{0.2041}}
  >{\raggedright\arraybackslash}p{(\linewidth - 6\tabcolsep) * \real{0.2245}}@{}}
\toprule\noalign{}
\begin{minipage}[b]{\linewidth}\raggedright
Database Type
\end{minipage} & \begin{minipage}[b]{\linewidth}\raggedright
Description
\end{minipage} & \begin{minipage}[b]{\linewidth}\raggedright
Examples
\end{minipage} & \begin{minipage}[b]{\linewidth}\raggedright
Use Cases
\end{minipage} \\
\midrule\noalign{}
\endhead
\bottomrule\noalign{}
\endlastfoot
\textbf{Relational (SQL)} & Structured data with ACID properties &
MySQL, PostgreSQL, Oracle & Financial systems, ERP \\
\textbf{Document} & JSON-like document storage & MongoDB, CouchDB &
Content management, catalogs \\
\textbf{Key-Value} & Simple key-value pairs & Redis, DynamoDB & Caching,
session storage \\
\textbf{Column-Family} & Wide-column storage & Cassandra, HBase &
Time-series, IoT data \\
\textbf{Graph} & Nodes and relationships & Neo4j, Amazon Neptune &
Social networks, recommendations \\
\end{longtable}
}

\textbf{SQL vs NoSQL Comparison}:

{\def\LTcaptype{none} % do not increment counter
\begin{longtable}[]{@{}lll@{}}
\toprule\noalign{}
Aspect & SQL Databases & NoSQL Databases \\
\midrule\noalign{}
\endhead
\bottomrule\noalign{}
\endlastfoot
\textbf{Schema} & Fixed schema & Flexible schema \\
\textbf{Scaling} & Vertical scaling & Horizontal scaling \\
\textbf{ACID} & Full ACID compliance & BASE properties \\
\textbf{Queries} & SQL language & Various query methods \\
\textbf{Consistency} & Strong consistency & Eventual consistency \\
\end{longtable}
}

\textbf{Selection Criteria}:

\begin{itemize}
\tightlist
\item
  \textbf{Data Structure}: Structured data \rightarrow SQL, Unstructured \rightarrow NoSQL\\
\item
  \textbf{Scalability}: Horizontal scaling \rightarrow NoSQL
\item
  \textbf{Consistency}: Strong consistency \rightarrow SQL
\item
  \textbf{Complexity}: Complex queries \rightarrow SQL, Simple access \rightarrow NoSQL
\end{itemize}

\textbf{Cloud Database Services}:

\begin{itemize}
\tightlist
\item
  \textbf{Amazon}: RDS (SQL), DynamoDB (NoSQL), DocumentDB (Document)
\item
  \textbf{Google}: Cloud SQL, Firestore, BigTable
\item
  \textbf{Microsoft}: Azure SQL, Cosmos DB
\end{itemize}

\end{solutionbox}
\begin{mnemonicbox}
``Relational Document Key Column Graph''

\end{mnemonicbox}
\begin{center}\rule{0.5\linewidth}{0.5pt}\end{center}

\subsection*{Question 5(a) [3 marks]}\label{q5a}

\textbf{Define cloud security. List out various Challenges for Cloud
Security.}

\begin{solutionbox}

\textbf{Cloud Security} refers to the policies, technologies,
applications, and controls utilized to protect virtualized IP, data,
applications, services, and infrastructure associated with cloud
computing.

\textbf{Cloud Security Challenges}:

\begin{itemize}
\tightlist
\item
  \textbf{Data breaches and privacy concerns}
\item
  \textbf{Identity and access management complexity}\\
\item
  \textbf{Insider threats and privileged user access}
\item
  \textbf{Compliance and regulatory requirements}
\item
  \textbf{Shared responsibility model confusion}
\item
  \textbf{API security vulnerabilities}
\end{itemize}

\textbf{Key Challenge Areas}: Each challenge requires specific security
strategies and tools to mitigate risks and ensure data protection in
cloud environments.

\end{solutionbox}
\begin{mnemonicbox}
``Data Identity Insider Compliance Shared API''

\end{mnemonicbox}
\begin{center}\rule{0.5\linewidth}{0.5pt}\end{center}

\subsection*{Question 5(b) [4 marks]}\label{q5b}

\textbf{Write a short note on Identity Management and Access Control.}

\begin{solutionbox}

\textbf{Identity and Access Management (IAM)}:

{\def\LTcaptype{none} % do not increment counter
\begin{longtable}[]{@{}
  >{\raggedright\arraybackslash}p{(\linewidth - 4\tabcolsep) * \real{0.3235}}
  >{\raggedright\arraybackslash}p{(\linewidth - 4\tabcolsep) * \real{0.3824}}
  >{\raggedright\arraybackslash}p{(\linewidth - 4\tabcolsep) * \real{0.2941}}@{}}
\toprule\noalign{}
\begin{minipage}[b]{\linewidth}\raggedright
Component
\end{minipage} & \begin{minipage}[b]{\linewidth}\raggedright
Description
\end{minipage} & \begin{minipage}[b]{\linewidth}\raggedright
Function
\end{minipage} \\
\midrule\noalign{}
\endhead
\bottomrule\noalign{}
\endlastfoot
\textbf{Authentication} & Verify user identity & Username/password, MFA,
biometrics \\
\textbf{Authorization} & Grant appropriate permissions & Role-based
access control (RBAC) \\
\textbf{Accounting} & Track user activities & Audit logs, compliance
reporting \\
\end{longtable}
}

\textbf{Access Control Models}:

\begin{itemize}
\tightlist
\item
  \textbf{Role-Based Access Control (RBAC)}: Users assigned roles with
  specific permissions
\item
  \textbf{Attribute-Based Access Control (ABAC)}: Dynamic permissions
  based on attributes
\item
  \textbf{Mandatory Access Control (MAC)}: System-enforced security
  policies
\end{itemize}

\textbf{Best Practices}:

\begin{itemize}
\tightlist
\item
  \textbf{Principle of least privilege}: Minimum necessary access
\item
  \textbf{Multi-factor authentication}: Enhanced security verification
\item
  \textbf{Regular access reviews}: Periodic permission audits
\item
  \textbf{Zero trust model}: Verify every access request
\end{itemize}

\end{solutionbox}
\begin{mnemonicbox}
``Authenticate Authorize Account''

\end{mnemonicbox}
\begin{center}\rule{0.5\linewidth}{0.5pt}\end{center}

\subsection*{Question 5(c) [7 marks]}\label{q5c}

\textbf{Explain the technologies used for data security in cloud.}

\begin{solutionbox}

{\def\LTcaptype{none} % do not increment counter
\begin{longtable}[]{@{}
  >{\raggedright\arraybackslash}p{(\linewidth - 6\tabcolsep) * \real{0.2400}}
  >{\raggedright\arraybackslash}p{(\linewidth - 6\tabcolsep) * \real{0.1800}}
  >{\raggedright\arraybackslash}p{(\linewidth - 6\tabcolsep) * \real{0.2600}}
  >{\raggedright\arraybackslash}p{(\linewidth - 6\tabcolsep) * \real{0.3200}}@{}}
\toprule\noalign{}
\begin{minipage}[b]{\linewidth}\raggedright
Technology
\end{minipage} & \begin{minipage}[b]{\linewidth}\raggedright
Purpose
\end{minipage} & \begin{minipage}[b]{\linewidth}\raggedright
Description
\end{minipage} & \begin{minipage}[b]{\linewidth}\raggedright
Implementation
\end{minipage} \\
\midrule\noalign{}
\endhead
\bottomrule\noalign{}
\endlastfoot
\textbf{Encryption} & Data protection & Converts data to unreadable
format & AES-256, RSA encryption \\
\textbf{Key Management} & Secure key storage & Centralized key lifecycle
management & AWS KMS, Azure Key Vault \\
\textbf{Digital Signatures} & Data integrity & Verify data authenticity
& PKI certificates \\
\textbf{Access Controls} & Permission management & Role-based access
restrictions & IAM policies, RBAC \\
\textbf{Network Security} & Traffic protection & Secure data
transmission & VPN, TLS/SSL, firewalls \\
\textbf{Data Loss Prevention} & Prevent data leaks & Monitor and control
data movement & DLP tools, content inspection \\
\textbf{Backup \& Recovery} & Data availability & Disaster recovery
planning & Automated backups, replication \\
\end{longtable}
}

\textbf{Security Implementation Layers}:

\begin{center}
\textbf{Mermaid Diagram (Code)}
\begin{verbatim}
{Shaded}
{Highlighting}[]
graph LR
    A[Application Security{br/{}Code security, input validation] }
    B[Data Security{br/{}Encryption, tokenization]}
    C[Network Security{br/{}Firewalls, VPN, SSL/TLS]}
    D[Infrastructure Security{br/{}Physical security, hypervisor]}
    
    A {-{-}{} B {-}{-}{} C {-}{-}{} D}
{Highlighting}
{Shaded}
\end{verbatim}
\end{center}

\textbf{Key Security Practices}:

\begin{itemize}
\tightlist
\item
  \textbf{Data at Rest}: Encrypt stored data using strong encryption
  algorithms
\item
  \textbf{Data in Transit}: Secure transmission using TLS/SSL
  protocols\\
\item
  \textbf{Data in Use}: Protect data during processing with secure
  enclaves
\item
  \textbf{Key Rotation}: Regular cryptographic key updates
\item
  \textbf{Compliance}: Meet regulatory requirements (GDPR, HIPAA, SOX)
\end{itemize}

\textbf{Emerging Technologies}:

\begin{itemize}
\tightlist
\item
  \textbf{Homomorphic Encryption}: Compute on encrypted data
\item
  \textbf{Zero-Knowledge Proofs}: Verify without revealing data
\item
  \textbf{Confidential Computing}: Protect data during processing
\end{itemize}

\end{solutionbox}
\begin{mnemonicbox}
``Encrypt Keys Sign Control Network Prevent Backup''

\end{mnemonicbox}
\begin{center}\rule{0.5\linewidth}{0.5pt}\end{center}

\subsection*{Question 5(a OR) [3
marks]}\label{question-5a-or-3-marks}

\textbf{Define serverless computing. List out advantages of serverless
computing.}

\begin{solutionbox}

\textbf{Serverless Computing} is a cloud execution model where cloud
providers dynamically manage server allocation and scaling, allowing
developers to focus solely on code without server management.

\textbf{Advantages of Serverless Computing}:

\begin{itemize}
\tightlist
\item
  \textbf{No server management}: Cloud provider handles infrastructure
\item
  \textbf{Automatic scaling}: Scales up/down based on demand
  automatically\\
\item
  \textbf{Pay-per-use pricing}: Only pay for actual execution time
\item
  \textbf{Faster development}: Focus on business logic, not
  infrastructure
\item
  \textbf{High availability}: Built-in fault tolerance and redundancy
\item
  \textbf{Reduced operational overhead}: No patching, monitoring servers
\end{itemize}

\textbf{Popular Examples}: AWS Lambda, Azure Functions, Google Cloud
Functions

\end{solutionbox}
\begin{mnemonicbox}
``No Automatic Pay Faster High Reduced''

\end{mnemonicbox}
\begin{center}\rule{0.5\linewidth}{0.5pt}\end{center}

\subsection*{Question 5(b OR) [4
marks]}\label{question-5b-or-4-marks}

\textbf{Differentiate between edge and fog computing.}

\begin{solutionbox}

{\def\LTcaptype{none} % do not increment counter
\begin{longtable}[]{@{}
  >{\raggedright\arraybackslash}p{(\linewidth - 4\tabcolsep) * \real{0.2051}}
  >{\raggedright\arraybackslash}p{(\linewidth - 4\tabcolsep) * \real{0.4103}}
  >{\raggedright\arraybackslash}p{(\linewidth - 4\tabcolsep) * \real{0.3846}}@{}}
\toprule\noalign{}
\begin{minipage}[b]{\linewidth}\raggedright
Aspect
\end{minipage} & \begin{minipage}[b]{\linewidth}\raggedright
Edge Computing
\end{minipage} & \begin{minipage}[b]{\linewidth}\raggedright
Fog Computing
\end{minipage} \\
\midrule\noalign{}
\endhead
\bottomrule\noalign{}
\endlastfoot
\textbf{Location} & At network edge, close to devices & Between cloud
and edge devices \\
\textbf{Processing} & Local processing on edge devices & Distributed
processing across nodes \\
\textbf{Latency} & Ultra-low latency & Low to medium latency \\
\textbf{Connectivity} & Direct device connection & Hierarchical network
structure \\
\textbf{Use Cases} & IoT sensors, autonomous vehicles & Smart cities,
industrial automation \\
\textbf{Examples} & Smartphone apps, smart cameras & Router-based
processing, gateways \\
\end{longtable}
}

\textbf{Key Differences}:

\begin{itemize}
\tightlist
\item
  \textbf{Edge} brings compute directly to data source
\item
  \textbf{Fog} creates a distributed computing layer
\item
  \textbf{Edge} optimizes for immediate response
\item
  \textbf{Fog} provides broader area coverage
\end{itemize}

\textbf{Benefits of Both}:

\begin{itemize}
\tightlist
\item
  Reduced bandwidth usage to cloud
\item
  Improved response times
\item
  Enhanced privacy and security
\item
  Better reliability for critical applications
\end{itemize}

\end{solutionbox}
\begin{mnemonicbox}
``Edge Direct, Fog Distributed''

\end{mnemonicbox}
\begin{center}\rule{0.5\linewidth}{0.5pt}\end{center}

\subsection*{Question 5(c OR) [7
marks]}\label{question-5c-or-7-marks}

\textbf{Define Containers. Explain steps to create image and execute the
docker container with example.}

\begin{solutionbox}

\textbf{Containers} are lightweight, portable packages that include
application code, runtime, system tools, libraries, and settings needed
to run an application consistently across different environments.

\textbf{Docker Container Creation Steps}:

\begin{verbatim}
flowchart LR
    A[Write Dockerfile] {-{-} B[Build Docker Image]}
    B {-{-} C[Run Docker Container]}
    C {-{-} D[Manage Container Lifecycle]}
    
    A1[FROM base\_image{br/COPY app\_filesbr/RUN install\_commandsbr/CMD start\_command] {-}{-} A}
    B1[docker build {-t image\_name .] {-}{-} B}
    C1[docker run {-p port:port image\_name] {-}{-} C}
    D1[docker ps{br/docker stopbr/docker start] {-}{-} D}
\end{verbatim}

\textbf{Step-by-Step Process}:

\textbf{1. Create Dockerfile}:

\begin{verbatim}
\# Base image
FROM node:14{-alpine}

\# Set working directory
WORKDIR /app

\# Copy package files
COPY package*.json ./

\# Install dependencies
RUN npm install

\# Copy application code
COPY . .

\# Expose port
EXPOSE 3000

\# Start command
CMD ["npm", "start"]
\end{verbatim}

\textbf{2. Build Docker Image}:

\begin{verbatim}
\# Build image from Dockerfile
docker build {-t} my{-web{-}app:latest .}

\# List images
docker images
\end{verbatim}

\textbf{3. Run Docker Container}:

\begin{verbatim}
\# Run container with port mapping
docker run {-d} {-p} 8080:3000 {-{-}name} web{-app my{-}web{-}app:latest}

\# Check running containers
docker ps
\end{verbatim}

\textbf{4. Container Management}:

{\def\LTcaptype{none} % do not increment counter
\begin{longtable}[]{@{}
  >{\raggedright\arraybackslash}p{(\linewidth - 4\tabcolsep) * \real{0.3333}}
  >{\raggedright\arraybackslash}p{(\linewidth - 4\tabcolsep) * \real{0.3333}}
  >{\raggedright\arraybackslash}p{(\linewidth - 4\tabcolsep) * \real{0.3333}}@{}}
\toprule\noalign{}
\begin{minipage}[b]{\linewidth}\raggedright
Command
\end{minipage} & \begin{minipage}[b]{\linewidth}\raggedright
Purpose
\end{minipage} & \begin{minipage}[b]{\linewidth}\raggedright
Example
\end{minipage} \\
\midrule\noalign{}
\endhead
\bottomrule\noalign{}
\endlastfoot
\textbf{docker ps} & List running containers &
\texttt{docker\ ps\ -a} \\
\textbf{docker stop} & Stop container &
\texttt{docker\ stop\ web-app} \\
\textbf{docker start} & Start stopped container &
\texttt{docker\ start\ web-app} \\
\textbf{docker logs} & View container logs &
\texttt{docker\ logs\ web-app} \\
\textbf{docker exec} & Execute command in container &
\texttt{docker\ exec\ -it\ web-app\ /bin/sh} \\
\end{longtable}
}

\textbf{Container Benefits}:

\begin{itemize}
\tightlist
\item
  \textbf{Portability}: Run anywhere Docker is installed
\item
  \textbf{Consistency}: Same environment across development/production
\item
  \textbf{Isolation}: Applications run independently
\item
  \textbf{Efficiency}: Share OS kernel, lightweight compared to VMs
\item
  \textbf{Scalability}: Easy horizontal scaling with orchestration
\end{itemize}

\textbf{Docker vs VM Comparison}:

\begin{verbatim}
    Docker Containers              Virtual Machines
    +{-{-}{-}{-}{-}{-}{-}{-}{-}{-}{-}{-}{-}{-}{-}{-}{-}{-}{-}+          +{-}{-}{-}{-}{-}{-}{-}{-}{-}{-}{-}{-}{-}{-}{-}{-}{-}{-}{-}+}
    |   App A   App B   |          |   App A   App B   |
    |  Runtime Runtime  |          |   OS A    OS B    |
    +{-{-}{-}{-}{-}{-}{-}{-}{-}{-}{-}{-}{-}{-}{-}{-}{-}{-}{-}+          +{-}{-}{-}{-}{-}{-}{-}{-}{-}{-}{-}{-}{-}{-}{-}{-}{-}{-}{-}+}
    |   Docker Engine   |          |    Hypervisor     |
    +{-{-}{-}{-}{-}{-}{-}{-}{-}{-}{-}{-}{-}{-}{-}{-}{-}{-}{-}+          +{-}{-}{-}{-}{-}{-}{-}{-}{-}{-}{-}{-}{-}{-}{-}{-}{-}{-}{-}+}
    |    Host OS        |          |     Host OS       |
    +{-{-}{-}{-}{-}{-}{-}{-}{-}{-}{-}{-}{-}{-}{-}{-}{-}{-}{-}+          +{-}{-}{-}{-}{-}{-}{-}{-}{-}{-}{-}{-}{-}{-}{-}{-}{-}{-}{-}+}
    |    Hardware       |          |     Hardware      |
    +{-{-}{-}{-}{-}{-}{-}{-}{-}{-}{-}{-}{-}{-}{-}{-}{-}{-}{-}+          +{-}{-}{-}{-}{-}{-}{-}{-}{-}{-}{-}{-}{-}{-}{-}{-}{-}{-}{-}+}
\end{verbatim}

\textbf{Common Docker Commands}:

\begin{itemize}
\tightlist
\item
  \textbf{Image Management}: \texttt{docker\ pull},
  \texttt{docker\ push}, \texttt{docker\ rmi}
\item
  \textbf{Container Operations}: \texttt{docker\ create},
  \texttt{docker\ kill}, \texttt{docker\ rm}
\item
  \textbf{System Info}: \texttt{docker\ info}, \texttt{docker\ version},
  \texttt{docker\ system\ df}
\end{itemize}

\textbf{Example Use Case}: A web application with Node.js backend can be
containerized to ensure consistent deployment across development,
testing, and production environments, eliminating ``works on my
machine'' issues.

\textbf{Container Orchestration}: For production deployments, use
orchestration tools like:

\begin{itemize}
\tightlist
\item
  \textbf{Kubernetes}: Advanced container orchestration
\item
  \textbf{Docker Swarm}: Native Docker clustering
\item
  \textbf{Amazon ECS}: AWS container service
\end{itemize}

\end{solutionbox}
\begin{mnemonicbox}
``Create Build Run Manage - Dockerfile Commands
Lifecycle''

\end{mnemonicbox}

\end{document}
