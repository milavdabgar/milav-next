\documentclass[10pt,a4paper]{article}

% content/resources/templates/preamble.tex
\usepackage[margin=0.6in]{geometry}
\author{Milav Dabgar}
\usepackage{amsmath,amssymb,amsthm}
\usepackage{booktabs}
\usepackage{multirow}
\usepackage{xcolor}
\usepackage{tcolorbox}
\tcbuselibrary{breakable,skins}
\usepackage[colorlinks=true,linkcolor=blue]{hyperref}
\usepackage{titlesec}
\usepackage{enumitem}
\usepackage{tikz}
\usepackage{pgfplots}
\usepackage{circuitikz}
\usepackage[version=4]{mhchem}
\usepackage{longtable}
\usepackage{array}
\usepackage{float}
\usepackage{caption}
\usepackage{listings}

\lstset{
  basicstyle=\small\ttfamily,
  breaklines=true,
  breakatwhitespace=false,
  postbreak=\mbox{\textcolor{red}{$\hookrightarrow$}\space},
  float=false,
  numbers=left,
  numberstyle=\tiny\color{gray},
  numbersep=10pt,
  xleftmargin=2em,
  keywordstyle=\color{blue},
  commentstyle=\color{green!60!black},
  stringstyle=\color{purple},
  backgroundcolor=\color{gray!5},
  showstringspaces=false,
  tabsize=2,
  captionpos=b,
  keepspaces=true,
  columns=flexible
}

\pgfplotsset{compat=1.18}
\usetikzlibrary{shapes,arrows,positioning,calc,patterns,decorations.pathmorphing,decorations.markings,arrows.meta}

% Color scheme
\definecolor{headcolor}{RGB}{0,102,204}
\definecolor{keycolor}{RGB}{220,20,60}
\definecolor{solutioncolor}{RGB}{34,139,34}
\definecolor{mnemoniccolor}{RGB}{148,0,211}
\definecolor{codecolor}{RGB}{0,0,100}

% Spacing
\setlength{\parskip}{3pt}
\setlist[itemize]{nosep}
\setlist[enumerate]{nosep}

% Title formatting
\titleformat{\section}{\Large\bfseries\color{headcolor}}{\thesection}{1em}{}
\titleformat{\subsection}{\large\bfseries\color{headcolor}}{\thesubsection}{1em}{}

% Pandoc tightlist compatibility
\providecommand{\tightlist}{%
  \setlength{\itemsep}{0pt}\setlength{\parskip}{0pt}}

% Pandoc longtable compatibility
\newcounter{none}
\def\thenone{}


% content/resources/templates/english-boxes.tex
% This file is currently empty - it exists to maintain consistency with the import structure.
% Add custom environments here if needed in the future.


\begin{document}

\begin{center}
{\Huge\bfseries\color{headcolor} Subject Name Solutions}\\[5pt]
{\LARGE 4361602 -- Summer 2025}\\[3pt]
{\large Semester 1 Study Material}\\[3pt]
{\normalsize\textit{Detailed Solutions and Explanations}}
\end{center}

\vspace{10pt}

\subsection*{Question 1(a) [3 marks]}\label{q1a}

\textbf{Define Cloud Computing. Explain Applications of cloud
computing.}

\begin{solutionbox}

\textbf{Cloud Computing} is the delivery of computing services including
servers, storage, databases, networking, software, analytics, and
intelligence over the Internet (``the cloud'') to offer faster
innovation, flexible resources, and economies of scale.

\textbf{Applications of Cloud Computing:}

{\def\LTcaptype{none} % do not increment counter
\begin{longtable}[]{@{}
  >{\raggedright\arraybackslash}p{(\linewidth - 2\tabcolsep) * \real{0.5000}}
  >{\raggedright\arraybackslash}p{(\linewidth - 2\tabcolsep) * \real{0.5000}}@{}}
\toprule\noalign{}
\begin{minipage}[b]{\linewidth}\raggedright
Application
\end{minipage} & \begin{minipage}[b]{\linewidth}\raggedright
Description
\end{minipage} \\
\midrule\noalign{}
\endhead
\bottomrule\noalign{}
\endlastfoot
\textbf{Data Storage} & Storing files and documents online \\
\textbf{Web Applications} & Running software applications via web
browsers \\
\textbf{Email Services} & Gmail, Outlook hosted on cloud \\
\textbf{Backup \& Recovery} & Automatic data backup and disaster
recovery \\
\end{longtable}
}

\end{solutionbox}
\begin{mnemonicbox}
``SWEB'' - Storage, Web apps, Email, Backup

\end{mnemonicbox}
\begin{center}\rule{0.5\linewidth}{0.5pt}\end{center}

\subsection*{Question 1(b) [4 marks]}\label{q1b}

\textbf{What is Cloud Storage Solutions? Explain Object storage in
detail.}

\begin{solutionbox}

\textbf{Cloud Storage Solutions} are online services that provide data
storage, management, and access through internet-connected devices.

\textbf{Object Storage Details:}

{\def\LTcaptype{none} % do not increment counter
\begin{longtable}[]{@{}ll@{}}
\toprule\noalign{}
Feature & Description \\
\midrule\noalign{}
\endhead
\bottomrule\noalign{}
\endlastfoot
\textbf{Structure} & Stores data as objects in buckets/containers \\
\textbf{Metadata} & Each object contains data, metadata, and unique
ID \\
\textbf{Scalability} & Virtually unlimited storage capacity \\
\textbf{Access} & RESTful APIs for programmatic access \\
\end{longtable}
}

\textbf{Diagram:}

\begin{verbatim}
┌─────────────────┐    ┌─────────────────┐    ┌─────────────────┐
│    Object 1     │    │    Object 2     │    │    Object 3     │
│                 │    │                 │    │                 │
│ Data + Metadata │    │ Data + Metadata │    │ Data + Metadata │
│ Unique ID: 001  │    │ Unique ID: 002  │    │ Unique ID: 003  │
└─────────────────┘    └─────────────────┘    └─────────────────┘
        │                       │                       │
        └───────────────────────┼───────────────────────┘
                                │
                    ┌─────────────────┐
                    │     Bucket      │
                    │   (Container)   │
                    └─────────────────┘
\end{verbatim}

\end{solutionbox}
\begin{mnemonicbox}
``SMAR'' - Scalable, Metadata-rich, API-accessible,
Resilient

\end{mnemonicbox}
\begin{center}\rule{0.5\linewidth}{0.5pt}\end{center}

\subsection*{Question 1(c) [7 marks]}\label{q1c}

\textbf{Explain Hardware virtualization and Software Virtualization in
detail.}

\begin{solutionbox}

\textbf{Hardware Virtualization:}

\begin{itemize}
\tightlist
\item
  \textbf{Physical layer abstraction} creating virtual versions of
  physical hardware components
\item
  \textbf{Hypervisor} manages multiple virtual machines on single
  physical server
\end{itemize}

\textbf{Software Virtualization:}

\begin{itemize}
\tightlist
\item
  \textbf{Application layer abstraction} allowing software to run in
  isolated environments
\item
  \textbf{Runtime environments} provide compatibility across different
  platforms
\end{itemize}

\textbf{Comparison Table:}

{\def\LTcaptype{none} % do not increment counter
\begin{longtable}[]{@{}lll@{}}
\toprule\noalign{}
Aspect & Hardware Virtualization & Software Virtualization \\
\midrule\noalign{}
\endhead
\bottomrule\noalign{}
\endlastfoot
\textbf{Level} & Hardware/OS level & Application level \\
\textbf{Performance} & Near-native & Slight overhead \\
\textbf{Resource Usage} & High & Moderate \\
\textbf{Isolation} & Complete & Application-specific \\
\end{longtable}
}

\textbf{Architecture Diagram:}

\begin{verbatim}
graph TB
    A[Physical Hardware] {-{-} B[Hypervisor]}
    B {-{-} C[VM1 {-} OS + Apps]}
    B {-{-} D[VM2 {-} OS + Apps]}
    B {-{-} E[VM3 {-} OS + Apps]}
    
    F[Host OS] {-{-} G[Software Virtualization Layer]}
    G {-{-} H[App Container 1]}
    G {-{-} I[App Container 2]}
    G {-{-} J[App Container 3]}
\end{verbatim}

\end{solutionbox}
\begin{mnemonicbox}
``HAPI'' - Hardware abstraction, Application
isolation, Performance consideration, Infrastructure management

\end{mnemonicbox}
\begin{center}\rule{0.5\linewidth}{0.5pt}\end{center}

\subsection*{Question 1(c) OR [7
marks]}\label{q1c}

\textbf{What is Cloud virtualization? Explain Characteristics of
virtualization.}

\begin{solutionbox}

\textbf{Cloud Virtualization} is the process of creating virtual
versions of computing resources (servers, storage, networks) that can be
dynamically allocated and managed in cloud environments.

\textbf{Characteristics of Virtualization:}

{\def\LTcaptype{none} % do not increment counter
\begin{longtable}[]{@{}
  >{\raggedright\arraybackslash}p{(\linewidth - 2\tabcolsep) * \real{0.5517}}
  >{\raggedright\arraybackslash}p{(\linewidth - 2\tabcolsep) * \real{0.4483}}@{}}
\toprule\noalign{}
\begin{minipage}[b]{\linewidth}\raggedright
Characteristic
\end{minipage} & \begin{minipage}[b]{\linewidth}\raggedright
Description
\end{minipage} \\
\midrule\noalign{}
\endhead
\bottomrule\noalign{}
\endlastfoot
\textbf{Resource Pooling} & Multiple physical resources combined into
pools \\
\textbf{Isolation} & Virtual machines operate independently \\
\textbf{Elasticity} & Dynamic scaling based on demand \\
\textbf{Efficiency} & Better hardware utilization \\
\end{longtable}
}

\textbf{Benefits:}

\begin{itemize}
\tightlist
\item
  \textbf{Cost reduction} through hardware consolidation
\item
  \textbf{Flexibility} in resource allocation
\item
  \textbf{Scalability} for growing demands
\item
  \textbf{Management} simplified through centralization
\end{itemize}

\textbf{Virtualization Stack:}

\begin{verbatim}
graph BT
    A[Physical Hardware] {-{-} B[Hypervisor/VMM]}
    B {-{-} C[Virtual Machine 1]}
    B {-{-} D[Virtual Machine 2]}
    B {-{-} E[Virtual Machine 3]}
    C {-{-} F[Guest OS 1]}
    D {-{-} G[Guest OS 2]}
    E {-{-} H[Guest OS 3]}
\end{verbatim}

\end{solutionbox}
\begin{mnemonicbox}
``RIEM'' - Resource pooling, Isolation, Elasticity,
Management

\end{mnemonicbox}
\begin{center}\rule{0.5\linewidth}{0.5pt}\end{center}

\subsection*{Question 2(a) [3 marks]}\label{q2a}

\textbf{Which are Cloud security challenges?}

\begin{solutionbox}

\textbf{Cloud Security Challenges:}

{\def\LTcaptype{none} % do not increment counter
\begin{longtable}[]{@{}
  >{\raggedright\arraybackslash}p{(\linewidth - 2\tabcolsep) * \real{0.4583}}
  >{\raggedright\arraybackslash}p{(\linewidth - 2\tabcolsep) * \real{0.5417}}@{}}
\toprule\noalign{}
\begin{minipage}[b]{\linewidth}\raggedright
Challenge
\end{minipage} & \begin{minipage}[b]{\linewidth}\raggedright
Description
\end{minipage} \\
\midrule\noalign{}
\endhead
\bottomrule\noalign{}
\endlastfoot
\textbf{Data Breaches} & Unauthorized access to sensitive information \\
\textbf{Access Management} & Controlling user permissions and
authentication \\
\textbf{Compliance} & Meeting regulatory and industry standards \\
\textbf{Vendor Lock-in} & Dependency on specific cloud provider \\
\end{longtable}
}

\end{solutionbox}
\begin{mnemonicbox}
``DACV'' - Data breaches, Access control, Compliance,
Vendor dependency

\end{mnemonicbox}
\begin{center}\rule{0.5\linewidth}{0.5pt}\end{center}

\subsection*{Question 2(b) [4 marks]}\label{q2b}

\textbf{Explain IaaS in detail.}

\begin{solutionbox}

\textbf{Infrastructure as a Service (IaaS)} provides virtualized
computing infrastructure over the internet, including servers, storage,
and networking.

\textbf{IaaS Components:}

{\def\LTcaptype{none} % do not increment counter
\begin{longtable}[]{@{}ll@{}}
\toprule\noalign{}
Component & Description \\
\midrule\noalign{}
\endhead
\bottomrule\noalign{}
\endlastfoot
\textbf{Compute} & Virtual machines and processing power \\
\textbf{Storage} & Block, file, and object storage \\
\textbf{Networking} & Virtual networks, load balancers, firewalls \\
\textbf{Management} & Monitoring, security, and backup tools \\
\end{longtable}
}

\textbf{IaaS Architecture:}

\begin{verbatim}
graph TB
    A[User/Customer] {-{-} B[IaaS Management Portal]}
    B {-{-} C[Compute Resources]}
    B {-{-} D[Storage Resources]}
    B {-{-} E[Network Resources]}
    C {-{-} F[Physical Servers]}
    D {-{-} G[Storage Arrays]}
    E {-{-} H[Network Infrastructure]}
\end{verbatim}

\textbf{Benefits:}

\begin{itemize}
\tightlist
\item
  \textbf{Pay-per-use} pricing model
\item
  \textbf{Scalability} on demand
\item
  \textbf{Reduced} capital expenditure
\end{itemize}

\end{solutionbox}
\begin{mnemonicbox}
``CSNM'' - Compute, Storage, Network, Management

\end{mnemonicbox}
\begin{center}\rule{0.5\linewidth}{0.5pt}\end{center}

\subsection*{Question 2(c) [7 marks]}\label{q2c}

\textbf{Explain Identity and access management in detail.}

\begin{solutionbox}

\textbf{Identity and Access Management (IAM)} is a framework for
managing digital identities and controlling access to resources in cloud
environments.

\textbf{IAM Components:}

{\def\LTcaptype{none} % do not increment counter
\begin{longtable}[]{@{}ll@{}}
\toprule\noalign{}
Component & Function \\
\midrule\noalign{}
\endhead
\bottomrule\noalign{}
\endlastfoot
\textbf{Authentication} & Verifying user identity \\
\textbf{Authorization} & Determining access permissions \\
\textbf{User Management} & Creating, modifying, deleting user
accounts \\
\textbf{Role-Based Access} & Assigning permissions based on roles \\
\end{longtable}
}

\textbf{IAM Process Flow:}

\begin{center}
\textbf{Mermaid Diagram (Code)}
\begin{verbatim}
{Shaded}
{Highlighting}[]
graph LR
    A[User Request] {-{-}{} B[Authentication]}
    B {-{-}{} C\{Valid Identity?\}}
    C {-{-}{}|Yes| D[Authorization Check]}
    C {-{-}{}|No| E[Access Denied]}
    D {-{-}{} F\{Permission Granted?\}}
    F {-{-}{}|Yes| G[Resource Access]}
    F {-{-}{}|No| H[Access Denied]}
{Highlighting}
{Shaded}
\end{verbatim}
\end{center}

\textbf{Key Features:}

\begin{itemize}
\tightlist
\item
  \textbf{Single Sign-On (SSO)} for seamless access
\item
  \textbf{Multi-Factor Authentication (MFA)} for enhanced security
\item
  \textbf{Policy Management} for access control
\item
  \textbf{Audit Logging} for compliance tracking
\end{itemize}

\textbf{Security Benefits:}

\begin{itemize}
\tightlist
\item
  \textbf{Centralized} identity management
\item
  \textbf{Reduced} security risks
\item
  \textbf{Compliance} with regulations
\item
  \textbf{Improved} user experience
\end{itemize}

\end{solutionbox}
\begin{mnemonicbox}
``AURU'' - Authentication, Authorization, User
management, Role-based access

\end{mnemonicbox}
\begin{center}\rule{0.5\linewidth}{0.5pt}\end{center}

\subsection*{Question 2(a) OR [3
marks]}\label{q2a}

\textbf{Need for Access control and authentication in cloud.}

\begin{solutionbox}

\textbf{Need for Access Control and Authentication:}

{\def\LTcaptype{none} % do not increment counter
\begin{longtable}[]{@{}ll@{}}
\toprule\noalign{}
Need & Reason \\
\midrule\noalign{}
\endhead
\bottomrule\noalign{}
\endlastfoot
\textbf{Data Protection} & Prevent unauthorized access to sensitive
data \\
\textbf{Regulatory Compliance} & Meet legal and industry requirements \\
\textbf{Resource Security} & Control who can use cloud resources \\
\textbf{Cost Management} & Prevent unauthorized resource usage \\
\end{longtable}
}

\end{solutionbox}
\begin{mnemonicbox}
``DRRC'' - Data protection, Regulatory compliance,
Resource security, Cost management

\end{mnemonicbox}
\begin{center}\rule{0.5\linewidth}{0.5pt}\end{center}

\subsection*{Question 2(b) OR [4
marks]}\label{q2b}

\textbf{Explain PaaS in detail.}

\begin{solutionbox}

\textbf{Platform as a Service (PaaS)} provides a cloud-based platform
allowing customers to develop, run, and manage applications without
dealing with underlying infrastructure.

\textbf{PaaS Components:}

{\def\LTcaptype{none} % do not increment counter
\begin{longtable}[]{@{}ll@{}}
\toprule\noalign{}
Component & Description \\
\midrule\noalign{}
\endhead
\bottomrule\noalign{}
\endlastfoot
\textbf{Development Tools} & IDEs, debuggers, compilers \\
\textbf{Runtime Environment} & Application execution platform \\
\textbf{Database Management} & Built-in database services \\
\textbf{Middleware} & Integration and communication services \\
\end{longtable}
}

\textbf{PaaS Architecture:}

\begin{verbatim}
graph TB
    A[Applications] {-{-} B[PaaS Platform]}
    B {-{-} C[Development Tools]}
    B {-{-} D[Runtime Environment]}
    B {-{-} E[Database Services]}
    B {-{-} F[Middleware]}
    F {-{-} G[IaaS Infrastructure]}
\end{verbatim}

\textbf{Benefits:}

\begin{itemize}
\tightlist
\item
  \textbf{Faster} application development
\item
  \textbf{Reduced} complexity
\item
  \textbf{Built-in} scalability
\end{itemize}

\end{solutionbox}
\begin{mnemonicbox}
``DRDM'' - Development tools, Runtime, Database,
Middleware

\end{mnemonicbox}
\begin{center}\rule{0.5\linewidth}{0.5pt}\end{center}

\subsection*{Question 2(c) OR [7
marks]}\label{q2c}

\textbf{Explain DevSecOps in detail.}

\begin{solutionbox}

\textbf{DevSecOps} integrates security practices into the DevOps
process, making security a shared responsibility throughout the
development lifecycle.

\textbf{DevSecOps Principles:}

{\def\LTcaptype{none} % do not increment counter
\begin{longtable}[]{@{}
  >{\raggedright\arraybackslash}p{(\linewidth - 2\tabcolsep) * \real{0.4583}}
  >{\raggedright\arraybackslash}p{(\linewidth - 2\tabcolsep) * \real{0.5417}}@{}}
\toprule\noalign{}
\begin{minipage}[b]{\linewidth}\raggedright
Principle
\end{minipage} & \begin{minipage}[b]{\linewidth}\raggedright
Description
\end{minipage} \\
\midrule\noalign{}
\endhead
\bottomrule\noalign{}
\endlastfoot
\textbf{Shift Left} & Integrate security early in development \\
\textbf{Automation} & Automated security testing and compliance \\
\textbf{Collaboration} & Security teams work with development and
operations \\
\textbf{Continuous Monitoring} & Ongoing security assessment \\
\end{longtable}
}

\textbf{DevSecOps Pipeline:}

\begin{center}
\textbf{Mermaid Diagram (Code)}
\begin{verbatim}
{Shaded}
{Highlighting}[]
graph LR
    A[Plan] {-{-}{} B[Code]}
    B {-{-}{} C[Build + Security Scan]}
    C {-{-}{} D[Test + Security Test]}
    D {-{-}{} E[Deploy + Security Config]}
    E {-{-}{} F[Monitor + Security Monitor]}
    F {-{-}{} A}
{Highlighting}
{Shaded}
\end{verbatim}
\end{center}

\textbf{Security Integration Points:}

\begin{itemize}
\tightlist
\item
  \textbf{Code Analysis} during development
\item
  \textbf{Vulnerability Scanning} in CI/CD pipeline
\item
  \textbf{Compliance Checks} before deployment
\item
  \textbf{Runtime Protection} in production
\end{itemize}

\textbf{Benefits:}

\begin{itemize}
\tightlist
\item
  \textbf{Early} vulnerability detection
\item
  \textbf{Faster} security fixes
\item
  \textbf{Reduced} security debt
\item
  \textbf{Improved} compliance
\end{itemize}

\end{solutionbox}
\begin{mnemonicbox}
``SACM'' - Shift left, Automation, Collaboration,
Monitoring

\end{mnemonicbox}
\begin{center}\rule{0.5\linewidth}{0.5pt}\end{center}

\subsection*{Question 3(a) [3 marks]}\label{q3a}

\textbf{Why is Edge Computing important?}

\begin{solutionbox}

\textbf{Importance of Edge Computing:}

{\def\LTcaptype{none} % do not increment counter
\begin{longtable}[]{@{}
  >{\raggedright\arraybackslash}p{(\linewidth - 2\tabcolsep) * \real{0.4091}}
  >{\raggedright\arraybackslash}p{(\linewidth - 2\tabcolsep) * \real{0.5909}}@{}}
\toprule\noalign{}
\begin{minipage}[b]{\linewidth}\raggedright
Benefit
\end{minipage} & \begin{minipage}[b]{\linewidth}\raggedright
Description
\end{minipage} \\
\midrule\noalign{}
\endhead
\bottomrule\noalign{}
\endlastfoot
\textbf{Reduced Latency} & Processing data closer to source \\
\textbf{Bandwidth Optimization} & Less data transmission to cloud \\
\textbf{Real-time Processing} & Immediate response for critical
applications \\
\textbf{Data Privacy} & Local processing keeps sensitive data local \\
\end{longtable}
}

\end{solutionbox}
\begin{mnemonicbox}
``RBRD'' - Reduced latency, Bandwidth optimization,
Real-time processing, Data privacy

\end{mnemonicbox}
\begin{center}\rule{0.5\linewidth}{0.5pt}\end{center}

\subsection*{Question 3(b) [4 marks]}\label{q3b}

\textbf{Define Data Center. List types of Data center. Explain anyone.}

\begin{solutionbox}

\textbf{Data Center} is a facility housing computer systems, storage
systems, networking equipment, and supporting infrastructure for IT
operations.

\textbf{Types of Data Centers:}

{\def\LTcaptype{none} % do not increment counter
\begin{longtable}[]{@{}ll@{}}
\toprule\noalign{}
Type & Description \\
\midrule\noalign{}
\endhead
\bottomrule\noalign{}
\endlastfoot
\textbf{Enterprise} & Private data centers owned by organizations \\
\textbf{Colocation} & Shared facility renting space to multiple
tenants \\
\textbf{Hyperscale} & Large-scale facilities for cloud providers \\
\textbf{Edge} & Small facilities closer to end users \\
\end{longtable}
}

\textbf{Enterprise Data Center (Detailed):}

\begin{itemize}
\tightlist
\item
  \textbf{Complete control} over infrastructure
\item
  \textbf{Customized} to organization needs
\item
  \textbf{High security} and compliance
\item
  \textbf{Significant} capital investment required
\end{itemize}

\textbf{Data Center Architecture:}

\begin{verbatim}
┌─────────────────────────────────────────┐
│              Data Center                │
│  ┌─────────┐  ┌─────────┐  ┌─────────┐  │
│  │ Server  │  │ Storage │  │ Network │  │
│  │  Racks  │  │  Systems│  │ Equip.  │  │
│  └─────────┘  └─────────┘  └─────────┘  │
│  ┌─────────────────────────────────────┐ │
│  │     Power \& Cooling Systems         │ │
│  └─────────────────────────────────────┘ │
└─────────────────────────────────────────┘
\end{verbatim}

\end{solutionbox}
\begin{mnemonicbox}
``ECHE'' - Enterprise, Colocation, Hyperscale, Edge

\end{mnemonicbox}
\begin{center}\rule{0.5\linewidth}{0.5pt}\end{center}

\subsection*{Question 3(c) [7 marks]}\label{q3c}

\textbf{Explain types of cloud databases in detail.}

\begin{solutionbox}

\textbf{Types of Cloud Databases:}

\textbf{1. SQL Databases (Relational):}

\begin{itemize}
\tightlist
\item
  \textbf{Structure:} Table-based with predefined schema
\item
  \textbf{ACID Properties:} Ensure data consistency
\item
  \textbf{Examples:} Amazon RDS, Google Cloud SQL
\end{itemize}

\textbf{2. NoSQL Databases:}

{\def\LTcaptype{none} % do not increment counter
\begin{longtable}[]{@{}
  >{\raggedright\arraybackslash}p{(\linewidth - 4\tabcolsep) * \real{0.3333}}
  >{\raggedright\arraybackslash}p{(\linewidth - 4\tabcolsep) * \real{0.3611}}
  >{\raggedright\arraybackslash}p{(\linewidth - 4\tabcolsep) * \real{0.3056}}@{}}
\toprule\noalign{}
\begin{minipage}[b]{\linewidth}\raggedright
NoSQL Type
\end{minipage} & \begin{minipage}[b]{\linewidth}\raggedright
Description
\end{minipage} & \begin{minipage}[b]{\linewidth}\raggedright
Use Cases
\end{minipage} \\
\midrule\noalign{}
\endhead
\bottomrule\noalign{}
\endlastfoot
\textbf{Document} & JSON-like documents & Content management,
catalogs \\
\textbf{Key-Value} & Simple key-value pairs & Session management,
caching \\
\textbf{Column-Family} & Wide column storage & Analytics, time-series
data \\
\textbf{Graph} & Nodes and relationships & Social networks,
recommendations \\
\end{longtable}
}

\textbf{Database Comparison:}

\begin{verbatim}
graph TB
    A[Cloud Databases] {-{-} B[SQL/Relational]}
    A {-{-} C[NoSQL]}
    B {-{-} D[MySQL, PostgreSQL]}
    C {-{-} E[Document {-} MongoDB]}
    C {-{-} F[Key{-}Value {-} Redis]}
    C {-{-} G[Column {-} Cassandra]}
    C {-{-} H[Graph {-} Neo4j]}
\end{verbatim}

\textbf{Selection Criteria:}

\begin{itemize}
\tightlist
\item
  \textbf{Data Structure} requirements
\item
  \textbf{Scalability} needs
\item
  \textbf{Consistency} requirements
\item
  \textbf{Performance} expectations
\end{itemize}

\textbf{Benefits:}

\begin{itemize}
\tightlist
\item
  \textbf{Managed} services reduce operational overhead
\item
  \textbf{Automatic} scaling and backup
\item
  \textbf{Global} distribution capabilities
\item
  \textbf{Cost-effective} pay-per-use model
\end{itemize}

\end{solutionbox}
\begin{mnemonicbox}
``DKCG'' - Document, Key-value, Column-family, Graph

\end{mnemonicbox}
\begin{center}\rule{0.5\linewidth}{0.5pt}\end{center}

\subsection*{Question 3(a) OR [3
marks]}\label{q3a}

\textbf{What is the Role of Machine Learning in Cloud Computing? Explain
it.}

\begin{solutionbox}

\textbf{Role of Machine Learning in Cloud Computing:}

{\def\LTcaptype{none} % do not increment counter
\begin{longtable}[]{@{}ll@{}}
\toprule\noalign{}
Role & Description \\
\midrule\noalign{}
\endhead
\bottomrule\noalign{}
\endlastfoot
\textbf{Resource Optimization} & Predict and optimize resource
allocation \\
\textbf{Security Enhancement} & Detect anomalies and threats \\
\textbf{Cost Management} & Optimize spending and usage patterns \\
\textbf{Performance Monitoring} & Predict and prevent system failures \\
\end{longtable}
}

\end{solutionbox}
\begin{mnemonicbox}
``RSCP'' - Resource optimization, Security
enhancement, Cost management, Performance monitoring

\end{mnemonicbox}
\begin{center}\rule{0.5\linewidth}{0.5pt}\end{center}

\subsection*{Question 3(b) OR [4
marks]}\label{q3b}

\textbf{What is Cloud Scalability? Explain in detail.}

\begin{solutionbox}

\textbf{Cloud Scalability} is the ability to increase or decrease
computing resources dynamically based on demand without affecting
performance.

\textbf{Types of Scalability:}

{\def\LTcaptype{none} % do not increment counter
\begin{longtable}[]{@{}
  >{\raggedright\arraybackslash}p{(\linewidth - 4\tabcolsep) * \real{0.2222}}
  >{\raggedright\arraybackslash}p{(\linewidth - 4\tabcolsep) * \real{0.4815}}
  >{\raggedright\arraybackslash}p{(\linewidth - 4\tabcolsep) * \real{0.2963}}@{}}
\toprule\noalign{}
\begin{minipage}[b]{\linewidth}\raggedright
Type
\end{minipage} & \begin{minipage}[b]{\linewidth}\raggedright
Description
\end{minipage} & \begin{minipage}[b]{\linewidth}\raggedright
Method
\end{minipage} \\
\midrule\noalign{}
\endhead
\bottomrule\noalign{}
\endlastfoot
\textbf{Vertical (Scale Up)} & Adding more power to existing machine &
CPU, RAM, Storage upgrade \\
\textbf{Horizontal (Scale Out)} & Adding more machines to resource pool
& Load distribution \\
\end{longtable}
}

\textbf{Scalability Process:}

\begin{center}
\textbf{Mermaid Diagram (Code)}
\begin{verbatim}
{Shaded}
{Highlighting}[]
graph LR
    A[Monitor Load] {-{-}{} B\{High Load?\}}
    B {-{-}{}|Yes| C[Scale Out/Up]}
    B {-{-}{}|No| D\{Low Load?\}}
    D {-{-}{}|Yes| E[Scale In/Down]}
    D {-{-}{}|No| A}
    C {-{-}{} A}
    E {-{-}{} A}
{Highlighting}
{Shaded}
\end{verbatim}
\end{center}

\textbf{Benefits:}

\begin{itemize}
\tightlist
\item
  \textbf{Cost efficiency} through dynamic resource allocation
\item
  \textbf{Performance} maintenance during peak loads
\item
  \textbf{Availability} improvement
\end{itemize}

\end{solutionbox}
\begin{mnemonicbox}
``VH'' - Vertical scaling, Horizontal scaling

\end{mnemonicbox}
\begin{center}\rule{0.5\linewidth}{0.5pt}\end{center}

\subsection*{Question 3(c) OR [7
marks]}\label{q3c}

\textbf{Explain Data consistency and durability in detail.}

\begin{solutionbox}

\textbf{Data Consistency} ensures all nodes see the same data
simultaneously in distributed systems.

\textbf{Data Durability} guarantees data persistence even in case of
system failures.

\textbf{Consistency Models:}

{\def\LTcaptype{none} % do not increment counter
\begin{longtable}[]{@{}
  >{\raggedright\arraybackslash}p{(\linewidth - 4\tabcolsep) * \real{0.2333}}
  >{\raggedright\arraybackslash}p{(\linewidth - 4\tabcolsep) * \real{0.4333}}
  >{\raggedright\arraybackslash}p{(\linewidth - 4\tabcolsep) * \real{0.3333}}@{}}
\toprule\noalign{}
\begin{minipage}[b]{\linewidth}\raggedright
Model
\end{minipage} & \begin{minipage}[b]{\linewidth}\raggedright
Description
\end{minipage} & \begin{minipage}[b]{\linewidth}\raggedright
Use Case
\end{minipage} \\
\midrule\noalign{}
\endhead
\bottomrule\noalign{}
\endlastfoot
\textbf{Strong} & All reads get most recent write & Financial systems \\
\textbf{Eventual} & System becomes consistent over time & Social
media \\
\textbf{Weak} & No guarantees about when consistency occurs & Gaming,
real-time \\
\end{longtable}
}

\textbf{Durability Mechanisms:}

{\def\LTcaptype{none} % do not increment counter
\begin{longtable}[]{@{}ll@{}}
\toprule\noalign{}
Mechanism & Description \\
\midrule\noalign{}
\endhead
\bottomrule\noalign{}
\endlastfoot
\textbf{Replication} & Multiple copies across different locations \\
\textbf{Backup} & Regular data snapshots \\
\textbf{Redundancy} & RAID, erasure coding \\
\textbf{Versioning} & Multiple versions of data \\
\end{longtable}
}

\textbf{CAP Theorem:}

\begin{verbatim}
graph TB
    A[CAP Theorem] {-{-} B[Consistency]}
    A {-{-} C[Availability]}
    A {-{-} D[Partition Tolerance]}
    E[Note: Can only guarantee 2 of 3]
\end{verbatim}

\textbf{Implementation Strategies:}

\begin{itemize}
\tightlist
\item
  \textbf{Multi-region} replication for durability
\item
  \textbf{Quorum-based} consistency for availability
\item
  \textbf{Checksums} for data integrity
\item
  \textbf{Transaction logs} for recovery
\end{itemize}

\end{solutionbox}
\begin{mnemonicbox}
``SEWR'' - Strong consistency, Eventual consistency,
Weak consistency, Replication strategies

\end{mnemonicbox}
\begin{center}\rule{0.5\linewidth}{0.5pt}\end{center}

\subsection*{Question 4(a) [3 marks]}\label{q4a}

\textbf{State the role of Data scaling.}

\begin{solutionbox}

\textbf{Role of Data Scaling:}

{\def\LTcaptype{none} % do not increment counter
\begin{longtable}[]{@{}
  >{\raggedright\arraybackslash}p{(\linewidth - 2\tabcolsep) * \real{0.3158}}
  >{\raggedright\arraybackslash}p{(\linewidth - 2\tabcolsep) * \real{0.6842}}@{}}
\toprule\noalign{}
\begin{minipage}[b]{\linewidth}\raggedright
Role
\end{minipage} & \begin{minipage}[b]{\linewidth}\raggedright
Description
\end{minipage} \\
\midrule\noalign{}
\endhead
\bottomrule\noalign{}
\endlastfoot
\textbf{Performance Maintenance} & Handle increased data volume
efficiently \\
\textbf{Storage Optimization} & Distribute data across multiple
systems \\
\textbf{Query Performance} & Maintain fast data retrieval speeds \\
\textbf{Cost Management} & Balance performance with storage costs \\
\end{longtable}
}

\end{solutionbox}
\begin{mnemonicbox}
``PSQC'' - Performance, Storage optimization, Query
performance, Cost management

\end{mnemonicbox}
\begin{center}\rule{0.5\linewidth}{0.5pt}\end{center}

\subsection*{Question 4(b) [4 marks]}\label{q4b}

\textbf{Define Kubernetes. Explain with reason: Kubernetes is an
essential component of cloud computing.}

\begin{solutionbox}

\textbf{Kubernetes} is an open-source container orchestration platform
that automates deployment, scaling, and management of containerized
applications.

\textbf{Why Kubernetes is Essential for Cloud Computing:}

{\def\LTcaptype{none} % do not increment counter
\begin{longtable}[]{@{}
  >{\raggedright\arraybackslash}p{(\linewidth - 2\tabcolsep) * \real{0.3810}}
  >{\raggedright\arraybackslash}p{(\linewidth - 2\tabcolsep) * \real{0.6190}}@{}}
\toprule\noalign{}
\begin{minipage}[b]{\linewidth}\raggedright
Reason
\end{minipage} & \begin{minipage}[b]{\linewidth}\raggedright
Explanation
\end{minipage} \\
\midrule\noalign{}
\endhead
\bottomrule\noalign{}
\endlastfoot
\textbf{Container Orchestration} & Manages multiple containers across
clusters \\
\textbf{Auto-scaling} & Dynamically adjusts resources based on demand \\
\textbf{Service Discovery} & Automatic load balancing and networking \\
\textbf{Self-healing} & Automatically replaces failed containers \\
\end{longtable}
}

\textbf{Kubernetes Architecture:}

\begin{verbatim}
graph TB
    A[Master Node] {-{-} B[API Server]}
    A {-{-} C[Controller Manager]}
    A {-{-} D[Scheduler]}
    E[Worker Node 1] {-{-} F[Kubelet]}
    E {-{-} G[Pods]}
    H[Worker Node 2] {-{-} I[Kubelet]}
    H {-{-} J[Pods]}
\end{verbatim}

\textbf{Essential Benefits:}

\begin{itemize}
\tightlist
\item
  \textbf{Platform independence} across cloud providers
\item
  \textbf{Resource efficiency} through container density
\item
  \textbf{DevOps integration} with CI/CD pipelines
\end{itemize}

\end{solutionbox}
\begin{mnemonicbox}
``CASS'' - Container orchestration, Auto-scaling,
Service discovery, Self-healing

\end{mnemonicbox}
\begin{center}\rule{0.5\linewidth}{0.5pt}\end{center}

\subsection*{Question 4(c) [7 marks]}\label{q4c}

\textbf{Explain Data center network topologies.}

\begin{solutionbox}

\textbf{Data Center Network Topologies} define how network components
are interconnected within a data center.

\textbf{Common Topologies:}

{\def\LTcaptype{none} % do not increment counter
\begin{longtable}[]{@{}
  >{\raggedright\arraybackslash}p{(\linewidth - 6\tabcolsep) * \real{0.2000}}
  >{\raggedright\arraybackslash}p{(\linewidth - 6\tabcolsep) * \real{0.2600}}
  >{\raggedright\arraybackslash}p{(\linewidth - 6\tabcolsep) * \real{0.2400}}
  >{\raggedright\arraybackslash}p{(\linewidth - 6\tabcolsep) * \real{0.3000}}@{}}
\toprule\noalign{}
\begin{minipage}[b]{\linewidth}\raggedright
Topology
\end{minipage} & \begin{minipage}[b]{\linewidth}\raggedright
Description
\end{minipage} & \begin{minipage}[b]{\linewidth}\raggedright
Advantages
\end{minipage} & \begin{minipage}[b]{\linewidth}\raggedright
Disadvantages
\end{minipage} \\
\midrule\noalign{}
\endhead
\bottomrule\noalign{}
\endlastfoot
\textbf{Three-Tier} & Core, Aggregation, Access layers & Simple,
hierarchical & Limited scalability \\
\textbf{Spine-Leaf} & Non-blocking, flat architecture & High bandwidth,
scalable & Complex configuration \\
\textbf{Fat Tree} & Tree structure with multiple paths & Good fault
tolerance & Oversubscription issues \\
\end{longtable}
}

\textbf{Spine-Leaf Architecture:}

\begin{verbatim}
graph TB
    S1[Spine 1] {-{-}{-} L1[Leaf 1]}
    S1 {-{-}{-} L2[Leaf 2]}
    S1 {-{-}{-} L3[Leaf 3]}
    S2[Spine 2] {-{-}{-} L1}
    S2 {-{-}{-} L2}
    S2 {-{-}{-} L3}
    L1 {-{-}{-} A1[Server 1]}
    L2 {-{-}{-} A2[Server 2]}
    L3 {-{-}{-} A3[Server 3]}
\end{verbatim}

\textbf{Modern Trends:}

\begin{itemize}
\tightlist
\item
  \textbf{Software-Defined Networking (SDN)} for programmable networks
\item
  \textbf{Network Function Virtualization (NFV)} for flexible services
\item
  \textbf{Micro-segmentation} for enhanced security
\end{itemize}

\textbf{Selection Criteria:}

\begin{itemize}
\tightlist
\item
  \textbf{Bandwidth} requirements
\item
  \textbf{Latency} sensitivity
\item
  \textbf{Scalability} needs
\item
  \textbf{Cost} considerations
\end{itemize}

\textbf{Benefits of Modern Topologies:}

\begin{itemize}
\tightlist
\item
  \textbf{Non-blocking} communication paths
\item
  \textbf{Equal-cost} multi-path routing
\item
  \textbf{Horizontal} scaling capability
\item
  \textbf{Reduced} network congestion
\end{itemize}

\end{solutionbox}
\begin{mnemonicbox}
``TSF'' - Three-tier, Spine-leaf, Fat tree

\end{mnemonicbox}
\begin{center}\rule{0.5\linewidth}{0.5pt}\end{center}

\subsection*{Question 4(a) OR [3
marks]}\label{q4a}

\textbf{Explain file storage in the cloud.}

\begin{solutionbox}

\textbf{Cloud File Storage} provides hierarchical file system access
over the network, similar to traditional file systems.

\textbf{Characteristics:}

{\def\LTcaptype{none} % do not increment counter
\begin{longtable}[]{@{}ll@{}}
\toprule\noalign{}
Feature & Description \\
\midrule\noalign{}
\endhead
\bottomrule\noalign{}
\endlastfoot
\textbf{Hierarchical Structure} & Folders and subfolders organization \\
\textbf{POSIX Compliance} & Standard file system interface \\
\textbf{Network Access} & SMB, NFS protocol support \\
\textbf{Shared Access} & Multiple users can access simultaneously \\
\end{longtable}
}

\end{solutionbox}
\begin{mnemonicbox}
``HPNS'' - Hierarchical, POSIX-compliant, Network
access, Shared access

\end{mnemonicbox}
\begin{center}\rule{0.5\linewidth}{0.5pt}\end{center}

\subsection*{Question 4(b) OR [4
marks]}\label{q4b}

\textbf{Explain Serverless Computing.}

\begin{solutionbox}

\textbf{Serverless Computing} is a cloud computing model where cloud
providers automatically manage server infrastructure, allowing
developers to focus on code.

\textbf{Key Features:}

{\def\LTcaptype{none} % do not increment counter
\begin{longtable}[]{@{}ll@{}}
\toprule\noalign{}
Feature & Description \\
\midrule\noalign{}
\endhead
\bottomrule\noalign{}
\endlastfoot
\textbf{Event-Driven} & Functions triggered by events \\
\textbf{Auto-Scaling} & Automatic resource allocation \\
\textbf{Pay-per-Execution} & Billing based on actual usage \\
\textbf{Stateless} & Functions don't maintain state \\
\end{longtable}
}

\textbf{Serverless Architecture:}

\begin{center}
\textbf{Mermaid Diagram (Code)}
\begin{verbatim}
{Shaded}
{Highlighting}[]
graph LR
    A[Event Source] {-{-}{} B[Function Trigger]}
    B {-{-}{} C[Function Execution]}
    C {-{-}{} D[Response]}
    E[Cloud Provider] {-{-}{} F[Infrastructure Management]}
{Highlighting}
{Shaded}
\end{verbatim}
\end{center}

\textbf{Benefits:}

\begin{itemize}
\tightlist
\item
  \textbf{No server management} required
\item
  \textbf{Cost efficiency} for variable workloads
\item
  \textbf{Rapid scaling} capabilities
\end{itemize}

\end{solutionbox}
\begin{mnemonicbox}
``EAPS'' - Event-driven, Auto-scaling,
Pay-per-execution, Stateless

\end{mnemonicbox}
\begin{center}\rule{0.5\linewidth}{0.5pt}\end{center}

\subsection*{Question 4(c) OR [7
marks]}\label{q4c}

\textbf{Explain SDN (Software Defined Networking) architecture.}

\begin{solutionbox}

\textbf{Software Defined Networking (SDN)} separates network control
plane from data plane, enabling centralized network management through
software.

\textbf{SDN Architecture Layers:}

{\def\LTcaptype{none} % do not increment counter
\begin{longtable}[]{@{}
  >{\raggedright\arraybackslash}p{(\linewidth - 4\tabcolsep) * \real{0.2414}}
  >{\raggedright\arraybackslash}p{(\linewidth - 4\tabcolsep) * \real{0.3448}}
  >{\raggedright\arraybackslash}p{(\linewidth - 4\tabcolsep) * \real{0.4138}}@{}}
\toprule\noalign{}
\begin{minipage}[b]{\linewidth}\raggedright
Layer
\end{minipage} & \begin{minipage}[b]{\linewidth}\raggedright
Function
\end{minipage} & \begin{minipage}[b]{\linewidth}\raggedright
Components
\end{minipage} \\
\midrule\noalign{}
\endhead
\bottomrule\noalign{}
\endlastfoot
\textbf{Application Layer} & Network applications and services &
Firewalls, Load balancers \\
\textbf{Control Layer} & Centralized network intelligence & SDN
Controller \\
\textbf{Infrastructure Layer} & Network forwarding devices & Switches,
Routers \\
\end{longtable}
}

\textbf{SDN Architecture Diagram:}

\begin{center}
\textbf{Mermaid Diagram (Code)}
\begin{verbatim}
{Shaded}
{Highlighting}[]
graph LR
    A[Application Layer] {-{-}{} B[Northbound APIs]}
    B {-{-}{} C[SDN Controller]}
    C {-{-}{} D[Southbound APIs]}
    D {-{-}{} E[Infrastructure Layer]}
    
    F[Network Apps] {-{-}{} A}
    G[OpenFlow Switches] {-{-}{} E}
{Highlighting}
{Shaded}
\end{verbatim}
\end{center}

\textbf{Key Protocols:}

\begin{itemize}
\tightlist
\item
  \textbf{OpenFlow:} Communication between controller and switches
\item
  \textbf{NETCONF:} Network configuration protocol
\item
  \textbf{REST APIs:} Northbound application interfaces
\end{itemize}

\textbf{SDN Benefits:}

{\def\LTcaptype{none} % do not increment counter
\begin{longtable}[]{@{}ll@{}}
\toprule\noalign{}
Benefit & Description \\
\midrule\noalign{}
\endhead
\bottomrule\noalign{}
\endlastfoot
\textbf{Centralized Control} & Single point of network management \\
\textbf{Programmability} & Software-based network configuration \\
\textbf{Flexibility} & Dynamic network reconfiguration \\
\textbf{Cost Reduction} & Commodity hardware usage \\
\end{longtable}
}

\textbf{Use Cases:}

\begin{itemize}
\tightlist
\item
  \textbf{Data center} networking
\item
  \textbf{Campus} networks
\item
  \textbf{Wide area} networks
\item
  \textbf{Network function} virtualization
\end{itemize}

\textbf{Challenges:}

\begin{itemize}
\tightlist
\item
  \textbf{Single point} of failure (controller)
\item
  \textbf{Scalability} concerns
\item
  \textbf{Security} considerations
\item
  \textbf{Vendor} interoperability
\end{itemize}

\end{solutionbox}
\begin{mnemonicbox}
``ACI'' - Application layer, Control layer,
Infrastructure layer

\end{mnemonicbox}
\begin{center}\rule{0.5\linewidth}{0.5pt}\end{center}

\subsection*{Question 5(a) [3 marks]}\label{q5a}

\textbf{Explain Infrastructure as Code (IaC) in Detail.}

\begin{solutionbox}

\textbf{Infrastructure as Code (IaC)} manages and provisions computing
infrastructure through machine-readable definition files rather than
manual processes.

\textbf{IaC Characteristics:}

{\def\LTcaptype{none} % do not increment counter
\begin{longtable}[]{@{}
  >{\raggedright\arraybackslash}p{(\linewidth - 2\tabcolsep) * \real{0.5517}}
  >{\raggedright\arraybackslash}p{(\linewidth - 2\tabcolsep) * \real{0.4483}}@{}}
\toprule\noalign{}
\begin{minipage}[b]{\linewidth}\raggedright
Characteristic
\end{minipage} & \begin{minipage}[b]{\linewidth}\raggedright
Description
\end{minipage} \\
\midrule\noalign{}
\endhead
\bottomrule\noalign{}
\endlastfoot
\textbf{Version Control} & Infrastructure definitions stored in
repositories \\
\textbf{Automation} & Automated deployment and management \\
\textbf{Consistency} & Identical environments across deployments \\
\textbf{Repeatability} & Reproducible infrastructure setups \\
\end{longtable}
}

\end{solutionbox}
\begin{mnemonicbox}
``VACR'' - Version control, Automation, Consistency,
Repeatability

\end{mnemonicbox}
\begin{center}\rule{0.5\linewidth}{0.5pt}\end{center}

\subsection*{Question 5(b) [4 marks]}\label{q5b}

\textbf{Give full form of SLA. Explain in detail.}

\begin{solutionbox}

\textbf{SLA - Service Level Agreement}

\textbf{SLA Definition:} A contract between service provider and
customer defining expected service levels and performance metrics.

\textbf{SLA Components:}

{\def\LTcaptype{none} % do not increment counter
\begin{longtable}[]{@{}ll@{}}
\toprule\noalign{}
Component & Description \\
\midrule\noalign{}
\endhead
\bottomrule\noalign{}
\endlastfoot
\textbf{Availability} & Uptime percentage (99.9\%, 99.99\%) \\
\textbf{Performance} & Response time, throughput metrics \\
\textbf{Support} & Response time for issues \\
\textbf{Penalties} & Compensation for SLA violations \\
\end{longtable}
}

\textbf{SLA Metrics:}

\begin{verbatim}
┌─────────────────┐    ┌─────────────────┐
│   Availability  │    │   Performance   │
│     99.99\%      │    │   { 200ms       │}
└─────────────────┘    └─────────────────┘
         │                       │
         └───────────┬───────────┘
                     │
            ┌─────────────────┐
            │       SLA       │
            │   Requirements  │
            └─────────────────┘
\end{verbatim}

\textbf{Benefits:}

\begin{itemize}
\tightlist
\item
  \textbf{Clear expectations} for both parties
\item
  \textbf{Performance} measurement standards
\item
  \textbf{Risk mitigation} through penalties
\end{itemize}

\end{solutionbox}
\begin{mnemonicbox}
``APSP'' - Availability, Performance, Support,
Penalties

\end{mnemonicbox}
\begin{center}\rule{0.5\linewidth}{0.5pt}\end{center}

\subsection*{Question 5(c) [7 marks]}\label{q5c}

\textbf{Explain Hypervisors in detail.}

\begin{solutionbox}

\textbf{Hypervisor} (Virtual Machine Monitor) is software that creates
and manages virtual machines by abstracting physical hardware.

\textbf{Types of Hypervisors:}

{\def\LTcaptype{none} % do not increment counter
\begin{longtable}[]{@{}
  >{\raggedright\arraybackslash}p{(\linewidth - 6\tabcolsep) * \real{0.1304}}
  >{\raggedright\arraybackslash}p{(\linewidth - 6\tabcolsep) * \real{0.2826}}
  >{\raggedright\arraybackslash}p{(\linewidth - 6\tabcolsep) * \real{0.2174}}
  >{\raggedright\arraybackslash}p{(\linewidth - 6\tabcolsep) * \real{0.3696}}@{}}
\toprule\noalign{}
\begin{minipage}[b]{\linewidth}\raggedright
Type
\end{minipage} & \begin{minipage}[b]{\linewidth}\raggedright
Description
\end{minipage} & \begin{minipage}[b]{\linewidth}\raggedright
Examples
\end{minipage} & \begin{minipage}[b]{\linewidth}\raggedright
Characteristics
\end{minipage} \\
\midrule\noalign{}
\endhead
\bottomrule\noalign{}
\endlastfoot
\textbf{Type 1 (Bare Metal)} & Runs directly on hardware & VMware
vSphere, Hyper-V & Better performance, enterprise use \\
\textbf{Type 2 (Hosted)} & Runs on host operating system & VirtualBox,
VMware Workstation & Easier setup, desktop use \\
\end{longtable}
}

\textbf{Hypervisor Architecture:}

\begin{verbatim}
graph TB
    subgraph "Type 1 {- Bare Metal"}
        A[Physical Hardware] {-{-} B[Type 1 Hypervisor]}
        B {-{-} C[VM1]}
        B {-{-} D[VM2]}
        B {-{-} E[VM3]}
    end
    
    subgraph "Type 2 {- Hosted"}
        F[Physical Hardware] {-{-} G[Host OS]}
        G {-{-} H[Type 2 Hypervisor]}
        H {-{-} I[VM1]}
        H {-{-} J[VM2]}
    end
\end{verbatim}

\textbf{Hypervisor Functions:}

{\def\LTcaptype{none} % do not increment counter
\begin{longtable}[]{@{}ll@{}}
\toprule\noalign{}
Function & Description \\
\midrule\noalign{}
\endhead
\bottomrule\noalign{}
\endlastfoot
\textbf{Resource Allocation} & CPU, memory, storage distribution \\
\textbf{Isolation} & Separate VM environments \\
\textbf{Hardware Abstraction} & Virtual hardware presentation \\
\textbf{VM Lifecycle Management} & Create, start, stop, delete VMs \\
\end{longtable}
}

\textbf{Virtualization Techniques:}

\begin{itemize}
\tightlist
\item
  \textbf{Hardware-assisted} virtualization (Intel VT-x, AMD-V)
\item
  \textbf{Paravirtualization} for improved performance
\item
  \textbf{Binary translation} for compatibility
\end{itemize}

\textbf{Performance Considerations:}

\begin{itemize}
\tightlist
\item
  \textbf{CPU overhead} from virtualization layer
\item
  \textbf{Memory management} with virtual memory
\item
  \textbf{I/O optimization} for storage and network
\item
  \textbf{Resource scheduling} among VMs
\end{itemize}

\textbf{Benefits:}

\begin{itemize}
\tightlist
\item
  \textbf{Server consolidation} reducing hardware costs
\item
  \textbf{Disaster recovery} through VM snapshots
\item
  \textbf{Testing environments} quick provisioning
\item
  \textbf{Legacy application} support
\end{itemize}

\textbf{Challenges:}

\begin{itemize}
\tightlist
\item
  \textbf{Performance overhead} compared to bare metal
\item
  \textbf{Complexity} in management
\item
  \textbf{Licensing costs} for enterprise hypervisors
\item
  \textbf{Security} considerations for shared resources
\end{itemize}

\end{solutionbox}
\begin{mnemonicbox}
``RAIH'' - Resource allocation, isolation, Hardware
abstraction

\end{mnemonicbox}
\begin{center}\rule{0.5\linewidth}{0.5pt}\end{center}

\subsection*{Question 5(a) OR [3
marks]}\label{q5a}

\textbf{What is Automation in Data Centers? Explain in detail.}

\begin{solutionbox}

\textbf{Data Center Automation} uses software and technologies to
perform routine tasks automatically without manual intervention.

\textbf{Automation Areas:}

{\def\LTcaptype{none} % do not increment counter
\begin{longtable}[]{@{}ll@{}}
\toprule\noalign{}
Area & Description \\
\midrule\noalign{}
\endhead
\bottomrule\noalign{}
\endlastfoot
\textbf{Provisioning} & Automatic server and service deployment \\
\textbf{Monitoring} & Continuous performance and health tracking \\
\textbf{Scaling} & Dynamic resource adjustment \\
\textbf{Maintenance} & Automated patching and updates \\
\end{longtable}
}

\end{solutionbox}
\begin{mnemonicbox}
``PMSM'' - Provisioning, Monitoring, Scaling,
Maintenance

\end{mnemonicbox}
\begin{center}\rule{0.5\linewidth}{0.5pt}\end{center}

\subsection*{Question 5(b) OR [4
marks]}\label{q5b}

\textbf{What is Data Security in Cloud? Explain in detail.}

\begin{solutionbox}

\textbf{Cloud Data Security} involves protecting data stored, processed,
and transmitted in cloud environments from unauthorized access,
corruption, and theft.

\textbf{Security Measures:}

{\def\LTcaptype{none} % do not increment counter
\begin{longtable}[]{@{}ll@{}}
\toprule\noalign{}
Measure & Description \\
\midrule\noalign{}
\endhead
\bottomrule\noalign{}
\endlastfoot
\textbf{Encryption} & Data protection at rest and in transit \\
\textbf{Access Controls} & User authentication and authorization \\
\textbf{Backup \& Recovery} & Data protection against loss \\
\textbf{Compliance} & Adherence to regulatory requirements \\
\end{longtable}
}

\textbf{Security Implementation:}

\begin{verbatim}
┌─────────────┐    ┌─────────────┐    ┌─────────────┐
│ Encryption  │    │   Access    │    │   Backup    │
│             │    │  Controls   │    │             │
│ AES{-256     │    │ IAM/RBAC    │    │ 3{-}2{-}1 Rule  │}
└─────────────┘    └─────────────┘    └─────────────┘
       │                    │                    │
       └────────────────────┼────────────────────┘
                            │
                   ┌─────────────┐
                   │    Data     │
                   │  Security   │
                   └─────────────┘
\end{verbatim}

\textbf{Best Practices:}

\begin{itemize}
\tightlist
\item
  \textbf{Zero-trust} security model
\item
  \textbf{Regular} security audits
\item
  \textbf{Data classification} and handling
\end{itemize}

\end{solutionbox}
\begin{mnemonicbox}
``EABC'' - Encryption, Access controls, Backup,
Compliance

\end{mnemonicbox}
\begin{center}\rule{0.5\linewidth}{0.5pt}\end{center}

\subsection*{Question 5(c) OR [7
marks]}\label{q5c}

\textbf{What is Virtual Machines? Explain Steps to Create and manage
Virtual machines.}

\begin{solutionbox}

\textbf{Virtual Machine (VM)} is a software-based emulation of a
physical computer that runs an operating system and applications in an
isolated environment.

\textbf{VM Components:}

{\def\LTcaptype{none} % do not increment counter
\begin{longtable}[]{@{}ll@{}}
\toprule\noalign{}
Component & Description \\
\midrule\noalign{}
\endhead
\bottomrule\noalign{}
\endlastfoot
\textbf{Virtual CPU} & Emulated processor cores \\
\textbf{Virtual Memory} & Allocated RAM for VM \\
\textbf{Virtual Storage} & Virtual hard disks \\
\textbf{Virtual Network} & Network interface emulation \\
\end{longtable}
}

\textbf{Steps to Create Virtual Machine:}

\textbf{1. Planning Phase:}

\begin{itemize}
\tightlist
\item
  \textbf{Resource Assessment:} Determine CPU, RAM, storage requirements
\item
  \textbf{OS Selection:} Choose guest operating system
\item
  \textbf{Network Configuration:} Plan IP addressing and connectivity
\end{itemize}

\textbf{2. VM Creation Process:}

\begin{center}
\textbf{Mermaid Diagram (Code)}
\begin{verbatim}
{Shaded}
{Highlighting}[]
graph LR
    A[Select Hypervisor] {-{-}{} B[Create VM]}
    B {-{-}{} C[Allocate Resources]}
    C {-{-}{} D[Install OS]}
    D {-{-}{} E[Configure Network]}
    E {-{-}{} F[Install Applications]}
{Highlighting}
{Shaded}
\end{verbatim}
\end{center}

\textbf{3. Detailed Creation Steps:}

{\def\LTcaptype{none} % do not increment counter
\begin{longtable}[]{@{}lll@{}}
\toprule\noalign{}
Step & Action & Details \\
\midrule\noalign{}
\endhead
\bottomrule\noalign{}
\endlastfoot
\textbf{1} & \textbf{Create VM Container} & Define VM name and
location \\
\textbf{2} & \textbf{Allocate CPU} & Assign virtual processor cores \\
\textbf{3} & \textbf{Assign Memory} & Allocate RAM (2GB-16GB typical) \\
\textbf{4} & \textbf{Create Storage} & Set up virtual hard disk \\
\textbf{5} & \textbf{Network Setup} & Configure virtual network
adapter \\
\textbf{6} & \textbf{OS Installation} & Install guest operating
system \\
\end{longtable}
}

\textbf{VM Management Operations:}

\textbf{Power Management:}

\begin{itemize}
\tightlist
\item
  \textbf{Start/Stop:} Control VM power state
\item
  \textbf{Suspend/Resume:} Pause and resume VM execution
\item
  \textbf{Reset:} Force restart VM
\end{itemize}

\textbf{Resource Management:}

\begin{itemize}
\tightlist
\item
  \textbf{Hot-add CPU/Memory:} Add resources without shutdown
\item
  \textbf{Storage Expansion:} Increase disk capacity
\item
  \textbf{Network Reconfiguration:} Modify network settings
\end{itemize}

\textbf{Maintenance Operations:}

{\def\LTcaptype{none} % do not increment counter
\begin{longtable}[]{@{}lll@{}}
\toprule\noalign{}
Operation & Purpose & Frequency \\
\midrule\noalign{}
\endhead
\bottomrule\noalign{}
\endlastfoot
\textbf{Snapshots} & Point-in-time backup & Before major changes \\
\textbf{Cloning} & Create identical copies & For scaling/testing \\
\textbf{Migration} & Move VM between hosts & For maintenance \\
\textbf{Backup} & Data protection & Daily/Weekly \\
\end{longtable}
}

\textbf{VM Lifecycle Management:}

\begin{center}
\textbf{Mermaid Diagram (Code)}
\begin{verbatim}
{Shaded}
{Highlighting}[]
graph LR
    A[Create VM] {-{-}{} B[Configure VM]}
    B {-{-}{} C[Deploy Applications]}
    C {-{-}{} D[Monitor Performance]}
    D {-{-}{} E\{Maintenance Needed?\}}
    E {-{-}{}|Yes| F[Update/Patch]}
    E {-{-}{}|No| D}
    F {-{-}{} G\{End of Life?\}}
    G {-{-}{}|No| D}
    G {-{-}{}|Yes| H[Decommission VM]}
{Highlighting}
{Shaded}
\end{verbatim}
\end{center}

\textbf{Best Practices:}

\begin{itemize}
\tightlist
\item
  \textbf{Regular backups} and snapshot management
\item
  \textbf{Resource monitoring} for optimization
\item
  \textbf{Security patching} and updates
\item
  \textbf{Performance tuning} based on workload
\end{itemize}

\textbf{Monitoring and Troubleshooting:}

\begin{itemize}
\tightlist
\item
  \textbf{Performance metrics:} CPU, memory, disk I/O
\item
  \textbf{Event logs:} System and application events
\item
  \textbf{Network connectivity:} Ping, traceroute tests
\item
  \textbf{Resource utilization:} Capacity planning
\end{itemize}

\textbf{VM Security:}

\begin{itemize}
\tightlist
\item
  \textbf{Guest OS hardening:} Remove unnecessary services
\item
  \textbf{Network isolation:} VLAN segmentation
\item
  \textbf{Access control:} User authentication
\item
  \textbf{Antivirus protection:} Malware scanning
\end{itemize}

\end{solutionbox}
\begin{mnemonicbox}
``CVMN'' - CPU, Virtual memory, Network, Storage

\end{mnemonicbox}

\end{document}
