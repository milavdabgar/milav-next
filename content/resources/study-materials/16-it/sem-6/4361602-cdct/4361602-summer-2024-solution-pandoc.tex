\documentclass[10pt,a4paper]{article}

% content/resources/templates/preamble.tex
\usepackage[margin=0.6in]{geometry}
\author{Milav Dabgar}
\usepackage{amsmath,amssymb,amsthm}
\usepackage{booktabs}
\usepackage{multirow}
\usepackage{xcolor}
\usepackage{tcolorbox}
\tcbuselibrary{breakable,skins}
\usepackage[colorlinks=true,linkcolor=blue]{hyperref}
\usepackage{titlesec}
\usepackage{enumitem}
\usepackage{tikz}
\usepackage{pgfplots}
\usepackage{circuitikz}
\usepackage[version=4]{mhchem}
\usepackage{longtable}
\usepackage{array}
\usepackage{float}
\usepackage{caption}
\usepackage{listings}

\lstset{
  basicstyle=\small\ttfamily,
  breaklines=true,
  breakatwhitespace=false,
  postbreak=\mbox{\textcolor{red}{$\hookrightarrow$}\space},
  float=false,
  numbers=left,
  numberstyle=\tiny\color{gray},
  numbersep=10pt,
  xleftmargin=2em,
  keywordstyle=\color{blue},
  commentstyle=\color{green!60!black},
  stringstyle=\color{purple},
  backgroundcolor=\color{gray!5},
  showstringspaces=false,
  tabsize=2,
  captionpos=b,
  keepspaces=true,
  columns=flexible
}

\pgfplotsset{compat=1.18}
\usetikzlibrary{shapes,arrows,positioning,calc,patterns,decorations.pathmorphing,decorations.markings,arrows.meta}

% Color scheme
\definecolor{headcolor}{RGB}{0,102,204}
\definecolor{keycolor}{RGB}{220,20,60}
\definecolor{solutioncolor}{RGB}{34,139,34}
\definecolor{mnemoniccolor}{RGB}{148,0,211}
\definecolor{codecolor}{RGB}{0,0,100}

% Spacing
\setlength{\parskip}{3pt}
\setlist[itemize]{nosep}
\setlist[enumerate]{nosep}

% Title formatting
\titleformat{\section}{\Large\bfseries\color{headcolor}}{\thesection}{1em}{}
\titleformat{\subsection}{\large\bfseries\color{headcolor}}{\thesubsection}{1em}{}

% Pandoc tightlist compatibility
\providecommand{\tightlist}{%
  \setlength{\itemsep}{0pt}\setlength{\parskip}{0pt}}

% Pandoc longtable compatibility
\newcounter{none}
\def\thenone{}


% content/resources/templates/english-boxes.tex
% This file is currently empty - it exists to maintain consistency with the import structure.
% Add custom environments here if needed in the future.


\begin{document}

\begin{center}
{\Huge\bfseries\color{headcolor} Subject Name Solutions}\\[5pt]
{\LARGE 4361602 -- Summer 2024}\\[3pt]
{\large Semester 1 Study Material}\\[3pt]
{\normalsize\textit{Detailed Solutions and Explanations}}
\end{center}

\vspace{10pt}

\subsection*{Question 1(a) [3 marks]}\label{q1a}

\textbf{Define Cloud computing. Explain any two advantages of using
cloud computing.}

\begin{solutionbox}

\textbf{Cloud Computing} is the delivery of computing services over the
internet including servers, storage, databases, and software.


{\def\LTcaptype{none} % do not increment counter
\vspace{-5pt}
\captionof{table}{Cloud Computing Advantages}
\vspace{-10pt}
\begin{longtable}[]{@{}ll@{}}
\toprule\noalign{}
Advantage & Description \\
\midrule\noalign{}
\endhead
\bottomrule\noalign{}
\endlastfoot
\textbf{Cost-Effective} & No upfront hardware costs, pay-as-you-use
model \\
\textbf{Scalability} & Resources can be scaled up/down based on
demand \\
\end{longtable}
}

\end{solutionbox}
\begin{mnemonicbox}
``Cloud Saves Cash'' (Cost-effective, Scalable)

\end{mnemonicbox}
\begin{center}\rule{0.5\linewidth}{0.5pt}\end{center}

\subsection*{Question 1(b) [4 marks]}\label{q1b}

\textbf{List the cloud service models. Justify: Infrastructure as a
service model is the base of cloud computing structure.}

\begin{solutionbox}


{\def\LTcaptype{none} % do not increment counter
\vspace{-5pt}
\captionof{table}{Cloud Service Models}
\vspace{-10pt}
\begin{longtable}[]{@{}
  >{\raggedright\arraybackslash}p{(\linewidth - 4\tabcolsep) * \real{0.2258}}
  >{\raggedright\arraybackslash}p{(\linewidth - 4\tabcolsep) * \real{0.3548}}
  >{\raggedright\arraybackslash}p{(\linewidth - 4\tabcolsep) * \real{0.4194}}@{}}
\toprule\noalign{}
\begin{minipage}[b]{\linewidth}\raggedright
Model
\end{minipage} & \begin{minipage}[b]{\linewidth}\raggedright
Full Form
\end{minipage} & \begin{minipage}[b]{\linewidth}\raggedright
Description
\end{minipage} \\
\midrule\noalign{}
\endhead
\bottomrule\noalign{}
\endlastfoot
\textbf{IaaS} & Infrastructure as a Service & Virtual machines, storage,
networks \\
\textbf{PaaS} & Platform as a Service & Development platforms and
tools \\
\textbf{SaaS} & Software as a Service & Ready-to-use applications \\
\end{longtable}
}

\textbf{Justification}: IaaS is the foundation because it provides basic
computing infrastructure (servers, storage, networking) upon which PaaS
and SaaS are built.

\end{solutionbox}
\begin{mnemonicbox}
``I Pay for Software'' (IaaS, PaaS, SaaS)

\end{mnemonicbox}
\begin{center}\rule{0.5\linewidth}{0.5pt}\end{center}

\subsection*{Question 1(c) [7 marks]}\label{q1c}

\textbf{Differentiate between edge and fog computing.}

\begin{solutionbox}


{\def\LTcaptype{none} % do not increment counter
\vspace{-5pt}
\captionof{table}{Edge vs Fog Computing}
\vspace{-10pt}
\begin{longtable}[]{@{}
  >{\raggedright\arraybackslash}p{(\linewidth - 4\tabcolsep) * \real{0.1739}}
  >{\raggedright\arraybackslash}p{(\linewidth - 4\tabcolsep) * \real{0.4130}}
  >{\raggedright\arraybackslash}p{(\linewidth - 4\tabcolsep) * \real{0.4130}}@{}}
\toprule\noalign{}
\begin{minipage}[b]{\linewidth}\raggedright
Aspect
\end{minipage} & \begin{minipage}[b]{\linewidth}\raggedright
\textbf{Edge Computing}
\end{minipage} & \begin{minipage}[b]{\linewidth}\raggedright
\textbf{Fog Computing}
\end{minipage} \\
\midrule\noalign{}
\endhead
\bottomrule\noalign{}
\endlastfoot
\textbf{Location} & At device level (endpoints) & Between cloud and
edge \\
\textbf{Latency} & Ultra-low (milliseconds) & Low (few seconds) \\
\textbf{Processing} & Limited local processing & Distributed
processing \\
\textbf{Storage} & Minimal local storage & Moderate storage capacity \\
\textbf{Use Cases} & IoT sensors, autonomous vehicles & Smart cities,
industrial IoT \\
\end{longtable}
}

\textbf{Diagram:}

\begin{center}
\textbf{Mermaid Diagram (Code)}
\begin{verbatim}
{Shaded}
{Highlighting}[]
graph LR
    A[Cloud Data Center] {-{-}{} B[Fog Layer]}
    B {-{-}{} C[Edge Devices]}
    B {-{-}{} D[Edge Devices]}
    B {-{-}{} E[Edge Devices]}
{Highlighting}
{Shaded}
\end{verbatim}
\end{center}

\end{solutionbox}
\begin{mnemonicbox}
``Edge is Extremely close, Fog is Further''

\end{mnemonicbox}
\begin{center}\rule{0.5\linewidth}{0.5pt}\end{center}

\subsection*{Question 1(c) OR [7
marks]}\label{q1c}

\textbf{Explain distributed ledger technology used in cloud computing.}

\begin{solutionbox}

\textbf{Distributed Ledger Technology (DLT)} is a decentralized database
spread across multiple nodes in cloud computing.

\textbf{Key Features:}

\begin{itemize}
\tightlist
\item
  \textbf{Decentralization}: No single point of failure
\item
  \textbf{Immutability}: Records cannot be altered once added
\item
  \textbf{Transparency}: All participants can view transactions
\item
  \textbf{Consensus}: Agreement required for new entries
\end{itemize}


{\def\LTcaptype{none} % do not increment counter
\vspace{-5pt}
\captionof{table}{DLT Benefits in Cloud}
\vspace{-10pt}
\begin{longtable}[]{@{}ll@{}}
\toprule\noalign{}
Benefit & Description \\
\midrule\noalign{}
\endhead
\bottomrule\noalign{}
\endlastfoot
\textbf{Security} & Enhanced data protection through cryptography \\
\textbf{Trust} & Eliminates need for intermediaries \\
\textbf{Audit Trail} & Complete transaction history \\
\end{longtable}
}

\end{solutionbox}
\begin{mnemonicbox}
``DLT Delivers Trusted Security''

\end{mnemonicbox}
\begin{center}\rule{0.5\linewidth}{0.5pt}\end{center}

\subsection*{Question 2(a) [3 marks]}\label{q2a}

\textbf{List and explain the major components of virtualization
environment.}

\begin{solutionbox}


{\def\LTcaptype{none} % do not increment counter
\vspace{-5pt}
\captionof{table}{Virtualization Components}
\vspace{-10pt}
\begin{longtable}[]{@{}ll@{}}
\toprule\noalign{}
Component & Description \\
\midrule\noalign{}
\endhead
\bottomrule\noalign{}
\endlastfoot
\textbf{Hypervisor} & Software managing virtual machines \\
\textbf{Virtual Machines} & Isolated computing environments \\
\textbf{Host OS} & Operating system running hypervisor \\
\end{longtable}
}

\end{solutionbox}
\begin{mnemonicbox}
``Hypervisor Handles Virtual Machines''

\end{mnemonicbox}
\begin{center}\rule{0.5\linewidth}{0.5pt}\end{center}

\subsection*{Question 2(b) [4 marks]}\label{q2b}

\textbf{Justify with example: Renting resources on cloud is more
beneficial than actually buying them for small and midcap companies.}

\begin{solutionbox}

\textbf{Benefits of Cloud Renting:}

\begin{itemize}
\tightlist
\item
  \textbf{Lower Initial Cost}: No upfront investment in hardware
\item
  \textbf{Flexibility}: Scale resources based on demand
\item
  \textbf{Maintenance-Free}: Provider handles updates and repairs
\end{itemize}

\textbf{Example}: A startup needs servers during peak season only.
Buying costs ₹10 lakhs, while cloud renting costs ₹50,000 for 3 months
usage.

\end{solutionbox}
\begin{mnemonicbox}
``Rent for Flexibility, Buy for Permanency''

\end{mnemonicbox}
\begin{center}\rule{0.5\linewidth}{0.5pt}\end{center}

\subsection*{Question 2(c) [7 marks]}\label{q2c}

\textbf{Explain Hypervisor with its types.}

\begin{solutionbox}

\textbf{Hypervisor} is software that creates and manages virtual
machines by abstracting hardware resources.


{\def\LTcaptype{none} % do not increment counter
\vspace{-5pt}
\captionof{table}{Hypervisor Types}
\vspace{-10pt}
\begin{longtable}[]{@{}
  >{\raggedright\arraybackslash}p{(\linewidth - 6\tabcolsep) * \real{0.1714}}
  >{\raggedright\arraybackslash}p{(\linewidth - 6\tabcolsep) * \real{0.1714}}
  >{\raggedright\arraybackslash}p{(\linewidth - 6\tabcolsep) * \real{0.3714}}
  >{\raggedright\arraybackslash}p{(\linewidth - 6\tabcolsep) * \real{0.2857}}@{}}
\toprule\noalign{}
\begin{minipage}[b]{\linewidth}\raggedright
Type
\end{minipage} & \begin{minipage}[b]{\linewidth}\raggedright
Name
\end{minipage} & \begin{minipage}[b]{\linewidth}\raggedright
Description
\end{minipage} & \begin{minipage}[b]{\linewidth}\raggedright
Examples
\end{minipage} \\
\midrule\noalign{}
\endhead
\bottomrule\noalign{}
\endlastfoot
\textbf{Type 1} & Bare Metal & Runs directly on hardware & VMware ESXi,
Hyper-V \\
\textbf{Type 2} & Hosted & Runs on host operating system & VirtualBox,
VMware Workstation \\
\end{longtable}
}

\textbf{Diagram:}

\begin{verbatim}
Type 1 (Bare Metal)          Type 2 (Hosted)
┌─────────────────┐          ┌─────────────────┐
│    VM1  │  VM2  │          │    VM1  │  VM2  │
├─────────┼───────┤          ├─────────┼───────┤
│   Hypervisor    │          │   Hypervisor    │
├─────────────────┤          ├─────────────────┤
│    Hardware     │          │    Host OS      │
└─────────────────┘          ├─────────────────┤
                             │    Hardware     │
                             └─────────────────┘
\end{verbatim}

\end{solutionbox}
\begin{mnemonicbox}
``Type 1 Touches Hardware, Type 2 Touches OS''

\end{mnemonicbox}
\begin{center}\rule{0.5\linewidth}{0.5pt}\end{center}

\subsection*{Question 2(a) OR [3
marks]}\label{q2a}

\textbf{State the advantages of using virtualization. Explain any one.}

\begin{solutionbox}

\textbf{Virtualization Advantages:}

\begin{itemize}
\tightlist
\item
  \textbf{Resource Optimization}: Better hardware utilization
\item
  \textbf{Cost Reduction}: Fewer physical servers needed
\item
  \textbf{Isolation}: Applications run independently
\end{itemize}

\textbf{Resource Optimization}: Multiple virtual machines can run on
single physical server, utilizing 80-90\% of hardware capacity instead
of typical 15-20\%.

\end{solutionbox}
\begin{mnemonicbox}
``Virtualization Optimizes Resources''

\end{mnemonicbox}
\begin{center}\rule{0.5\linewidth}{0.5pt}\end{center}

\subsection*{Question 2(b) OR [4
marks]}\label{q2b}

\textbf{Explain Application-level virtualization.}

\begin{solutionbox}

\textbf{Application-level virtualization} allows applications to run in
isolated environments without installing them on the host OS.


{\def\LTcaptype{none} % do not increment counter
\vspace{-5pt}
\captionof{table}{Application Virtualization Features}
\vspace{-10pt}
\begin{longtable}[]{@{}ll@{}}
\toprule\noalign{}
Feature & Description \\
\midrule\noalign{}
\endhead
\bottomrule\noalign{}
\endlastfoot
\textbf{Isolation} & Apps don't interfere with each other \\
\textbf{Portability} & Apps run on different OS without modification \\
\textbf{Security} & Sandboxed execution environment \\
\end{longtable}
}

\textbf{Example}: Docker containers running applications with their
dependencies packaged together.

\end{solutionbox}
\begin{mnemonicbox}
``Apps Are Isolated and Portable''

\end{mnemonicbox}
\begin{center}\rule{0.5\linewidth}{0.5pt}\end{center}

\subsection*{Question 2(c) OR [7
marks]}\label{q2c}

\textbf{Explain hardware virtualization in cloud.}

\begin{solutionbox}

\textbf{Hardware virtualization} creates virtual versions of physical
hardware components in cloud environments.

\textbf{Key Components:}

\begin{itemize}
\tightlist
\item
  \textbf{CPU Virtualization}: Multiple VMs share physical processor
\item
  \textbf{Memory Virtualization}: Virtual memory allocation to VMs
\item
  \textbf{Storage Virtualization}: Abstract storage resources
\item
  \textbf{Network Virtualization}: Virtual network interfaces
\end{itemize}


{\def\LTcaptype{none} % do not increment counter
\vspace{-5pt}
\captionof{table}{Hardware Virtualization Benefits}
\vspace{-10pt}
\begin{longtable}[]{@{}ll@{}}
\toprule\noalign{}
Benefit & Description \\
\midrule\noalign{}
\endhead
\bottomrule\noalign{}
\endlastfoot
\textbf{Resource Sharing} & Multiple VMs use same hardware \\
\textbf{Isolation} & VMs operate independently \\
\textbf{Migration} & VMs can move between hosts \\
\end{longtable}
}

\end{solutionbox}
\begin{mnemonicbox}
``Hardware Hosts Multiple Virtual Machines''

\end{mnemonicbox}
\begin{center}\rule{0.5\linewidth}{0.5pt}\end{center}

\subsection*{Question 3(a) [3 marks]}\label{q3a}

\textbf{Define Data Center. List types of Data center.}

\begin{solutionbox}

\textbf{Data Center} is a facility housing computing and networking
equipment to store, process, and distribute data.


{\def\LTcaptype{none} % do not increment counter
\vspace{-5pt}
\captionof{table}{Data Center Types}
\vspace{-10pt}
\begin{longtable}[]{@{}ll@{}}
\toprule\noalign{}
Type & Description \\
\midrule\noalign{}
\endhead
\bottomrule\noalign{}
\endlastfoot
\textbf{Enterprise} & Private data centers for organizations \\
\textbf{Colocation} & Shared facilities for multiple clients \\
\textbf{Cloud} & Virtualized, scalable data centers \\
\end{longtable}
}

\end{solutionbox}
\begin{mnemonicbox}
``Enterprise, Colocation, Cloud Centers''

\end{mnemonicbox}
\begin{center}\rule{0.5\linewidth}{0.5pt}\end{center}

\subsection*{Question 3(b) [4 marks]}\label{q3b}

\textbf{Why data centre automation is important?}

\begin{solutionbox}

\textbf{Data Center Automation Benefits:}

\begin{itemize}
\tightlist
\item
  \textbf{Efficiency}: Reduces manual tasks and errors
\item
  \textbf{Cost Savings}: Lower operational expenses
\item
  \textbf{Scalability}: Quick resource provisioning
\item
  \textbf{Reliability}: Consistent operations and monitoring
\end{itemize}


{\def\LTcaptype{none} % do not increment counter
\vspace{-5pt}
\captionof{table}{Automation Areas}
\vspace{-10pt}
\begin{longtable}[]{@{}ll@{}}
\toprule\noalign{}
Area & Benefit \\
\midrule\noalign{}
\endhead
\bottomrule\noalign{}
\endlastfoot
\textbf{Provisioning} & Faster server deployment \\
\textbf{Monitoring} & Real-time performance tracking \\
\textbf{Maintenance} & Automated updates and patches \\
\end{longtable}
}

\end{solutionbox}
\begin{mnemonicbox}
``Automation Enhances Efficiency''

\end{mnemonicbox}
\begin{center}\rule{0.5\linewidth}{0.5pt}\end{center}

\subsection*{Question 3(c) [7 marks]}\label{q3c}

\textbf{Explain SDN (Software Defined Networking) architecture.}

\begin{solutionbox}

\textbf{SDN} separates network control plane from data plane, enabling
centralized network management.

\textbf{SDN Architecture Layers:}

\begin{center}
\textbf{Mermaid Diagram (Code)}
\begin{verbatim}
{Shaded}
{Highlighting}[]
graph LR
    A[Application Layer] {-{-}{} B[Control Layer]}
    B {-{-}{} C[Infrastructure Layer]}
    A {-.{-}{}|Northbound API| B}
    B {-.{-}{}|Southbound API| C}
{Highlighting}
{Shaded}
\end{verbatim}
\end{center}


{\def\LTcaptype{none} % do not increment counter
\vspace{-5pt}
\captionof{table}{SDN Components}
\vspace{-10pt}
\begin{longtable}[]{@{}ll@{}}
\toprule\noalign{}
Component & Function \\
\midrule\noalign{}
\endhead
\bottomrule\noalign{}
\endlastfoot
\textbf{Controller} & Centralized network control \\
\textbf{Switches} & Forward packets based on controller \\
\textbf{Applications} & Network services and policies \\
\end{longtable}
}

\textbf{Benefits:}

\begin{itemize}
\tightlist
\item
  \textbf{Centralized Control}: Single point of network management
\item
  \textbf{Programmability}: Dynamic network configuration
\item
  \textbf{Flexibility}: Easy policy implementation
\end{itemize}

\end{solutionbox}
\begin{mnemonicbox}
``SDN Separates Control from Data''

\end{mnemonicbox}
\begin{center}\rule{0.5\linewidth}{0.5pt}\end{center}

\subsection*{Question 3(a) OR [3
marks]}\label{q3a}

\textbf{Define the following: (i) Cloud Elasticity (ii) Cloud
Scalability}

\begin{solutionbox}


{\def\LTcaptype{none} % do not increment counter
\vspace{-5pt}
\captionof{table}{Cloud Elasticity vs Scalability}
\vspace{-10pt}
\begin{longtable}[]{@{}
  >{\raggedright\arraybackslash}p{(\linewidth - 2\tabcolsep) * \real{0.3333}}
  >{\raggedright\arraybackslash}p{(\linewidth - 2\tabcolsep) * \real{0.6667}}@{}}
\toprule\noalign{}
\begin{minipage}[b]{\linewidth}\raggedright
Term
\end{minipage} & \begin{minipage}[b]{\linewidth}\raggedright
Definition
\end{minipage} \\
\midrule\noalign{}
\endhead
\bottomrule\noalign{}
\endlastfoot
\textbf{Cloud Elasticity} & Automatic resource adjustment based on
demand \\
\textbf{Cloud Scalability} & Ability to handle increased workload by
adding resources \\
\end{longtable}
}

\textbf{Key Difference}: Elasticity is automatic, scalability can be
manual or automatic.

\end{solutionbox}
\begin{mnemonicbox}
``Elasticity is Automatic, Scalability is Adaptable''

\end{mnemonicbox}
\begin{center}\rule{0.5\linewidth}{0.5pt}\end{center}

\subsection*{Question 3(b) OR [4
marks]}\label{q3b}

\textbf{Explain with reason: Vendor lock-in is a major problem in cloud
computing services.}

\begin{solutionbox}

\textbf{Vendor Lock-in} occurs when switching cloud providers becomes
difficult due to dependency on specific services.

\textbf{Problems:}

\begin{itemize}
\tightlist
\item
  \textbf{High Migration Costs}: Data transfer and application
  modification expenses
\item
  \textbf{Limited Flexibility}: Restricted choice of providers
\item
  \textbf{Dependency}: Reliance on single vendor's technologies
\end{itemize}

\textbf{Example}: Using AWS-specific services makes migration to Google
Cloud expensive and complex.

\end{solutionbox}
\begin{mnemonicbox}
``Lock-in Limits Liberty''

\end{mnemonicbox}
\begin{center}\rule{0.5\linewidth}{0.5pt}\end{center}

\subsection*{Question 3(c) OR [7
marks]}\label{q3c}

\textbf{Explain Infrastructure as Code (IaC) with its different
approaches.}

\begin{solutionbox}

\textbf{Infrastructure as Code (IaC)} manages infrastructure through
code rather than manual processes.


{\def\LTcaptype{none} % do not increment counter
\vspace{-5pt}
\captionof{table}{IaC Approaches}
\vspace{-10pt}
\begin{longtable}[]{@{}
  >{\raggedright\arraybackslash}p{(\linewidth - 4\tabcolsep) * \real{0.3333}}
  >{\raggedright\arraybackslash}p{(\linewidth - 4\tabcolsep) * \real{0.4333}}
  >{\raggedright\arraybackslash}p{(\linewidth - 4\tabcolsep) * \real{0.2333}}@{}}
\toprule\noalign{}
\begin{minipage}[b]{\linewidth}\raggedright
Approach
\end{minipage} & \begin{minipage}[b]{\linewidth}\raggedright
Description
\end{minipage} & \begin{minipage}[b]{\linewidth}\raggedright
Tools
\end{minipage} \\
\midrule\noalign{}
\endhead
\bottomrule\noalign{}
\endlastfoot
\textbf{Declarative} & Define desired end state & Terraform, ARM
templates \\
\textbf{Imperative} & Define step-by-step instructions & Scripts,
Ansible \\
\textbf{Hybrid} & Combination of both approaches & Pulumi \\
\end{longtable}
}

\textbf{Benefits:}

\begin{itemize}
\tightlist
\item
  \textbf{Consistency}: Repeatable infrastructure deployment
\item
  \textbf{Version Control}: Track infrastructure changes
\item
  \textbf{Automation}: Reduce manual configuration errors
\end{itemize}

\textbf{Diagram:}

\begin{center}
\textbf{Mermaid Diagram (Code)}
\begin{verbatim}
{Shaded}
{Highlighting}[]
graph LR
    A[Code] {-{-}{} B[IaC Tool]}
    B {-{-}{} C[Cloud Provider]}
    C {-{-}{} D[Infrastructure]}
{Highlighting}
{Shaded}
\end{verbatim}
\end{center}

\end{solutionbox}
\begin{mnemonicbox}
``IaC Codes Infrastructure''

\end{mnemonicbox}
\begin{center}\rule{0.5\linewidth}{0.5pt}\end{center}

\subsection*{Question 4(a) [3 marks]}\label{q4a}

\textbf{Define cloud storage. List the major cloud storage solutions.}

\begin{solutionbox}

\textbf{Cloud Storage} is a service that stores data on remote servers
accessible via internet.


{\def\LTcaptype{none} % do not increment counter
\vspace{-5pt}
\captionof{table}{Major Cloud Storage Solutions}
\vspace{-10pt}
\begin{longtable}[]{@{}lll@{}}
\toprule\noalign{}
Provider & Service & Type \\
\midrule\noalign{}
\endhead
\bottomrule\noalign{}
\endlastfoot
\textbf{Amazon} & S3 & Object Storage \\
\textbf{Google} & Cloud Storage & Object Storage \\
\textbf{Microsoft} & Azure Blob & Object Storage \\
\end{longtable}
}

\end{solutionbox}
\begin{mnemonicbox}
``Amazon, Google, Microsoft Store Objects''

\end{mnemonicbox}
\begin{center}\rule{0.5\linewidth}{0.5pt}\end{center}

\subsection*{Question 4(b) [4 marks]}\label{q4b}

\textbf{Justify with example: Data consistency is an essential feature
of cloud storage}

\begin{solutionbox}

\textbf{Data Consistency} ensures all copies of data across distributed
systems show the same value.

\textbf{Importance:}

\begin{itemize}
\tightlist
\item
  \textbf{Reliability}: Users get correct data always
\item
  \textbf{Integrity}: Prevents data corruption
\item
  \textbf{Synchronization}: Multiple users see same information
\end{itemize}

\textbf{Example}: In banking system, account balance must be consistent
across all ATMs and branches to prevent double spending.

\end{solutionbox}
\begin{mnemonicbox}
``Consistency Creates Confidence''

\end{mnemonicbox}
\begin{center}\rule{0.5\linewidth}{0.5pt}\end{center}

\subsection*{Question 4(c) [7 marks]}\label{q4c}

\textbf{Explain types of cloud databases in detail.}

\begin{solutionbox}


{\def\LTcaptype{none} % do not increment counter
\vspace{-5pt}
\captionof{table}{Cloud Database Types}
\vspace{-10pt}
\begin{longtable}[]{@{}
  >{\raggedright\arraybackslash}p{(\linewidth - 6\tabcolsep) * \real{0.1500}}
  >{\raggedright\arraybackslash}p{(\linewidth - 6\tabcolsep) * \real{0.3250}}
  >{\raggedright\arraybackslash}p{(\linewidth - 6\tabcolsep) * \real{0.2500}}
  >{\raggedright\arraybackslash}p{(\linewidth - 6\tabcolsep) * \real{0.2750}}@{}}
\toprule\noalign{}
\begin{minipage}[b]{\linewidth}\raggedright
Type
\end{minipage} & \begin{minipage}[b]{\linewidth}\raggedright
Description
\end{minipage} & \begin{minipage}[b]{\linewidth}\raggedright
Examples
\end{minipage} & \begin{minipage}[b]{\linewidth}\raggedright
Use Cases
\end{minipage} \\
\midrule\noalign{}
\endhead
\bottomrule\noalign{}
\endlastfoot
\textbf{SQL Databases} & Relational databases with ACID properties &
Amazon RDS, Azure SQL & Transaction processing \\
\textbf{NoSQL Databases} & Non-relational, flexible schema & MongoDB
Atlas, DynamoDB & Big data, real-time web apps \\
\textbf{In-Memory} & Data stored in RAM for speed & Redis, Memcached &
Caching, real-time analytics \\
\textbf{Graph Databases} & Relationship-focused data storage & Neo4j,
Amazon Neptune & Social networks, recommendations \\
\end{longtable}
}

\textbf{SQL vs NoSQL Comparison:}

\begin{center}
\textbf{Mermaid Diagram (Code)}
\begin{verbatim}
{Shaded}
{Highlighting}[]
graph LR
    A[Structured Data] {-{-}{} B[SQL Database]}
    C[Unstructured Data] {-{-}{} D[NoSQL Database]}
    B {-{-}{} E[ACID Compliance]}
    D {-{-}{} F[High Scalability]}
{Highlighting}
{Shaded}
\end{verbatim}
\end{center}

\end{solutionbox}
\begin{mnemonicbox}
``SQL for Structure, NoSQL for Scale''

\end{mnemonicbox}
\begin{center}\rule{0.5\linewidth}{0.5pt}\end{center}

\subsection*{Question 4(a) OR [3
marks]}\label{q4a}

\textbf{Define database services in cloud. List the major features of
database services}

\begin{solutionbox}

\textbf{Cloud Database Services} are managed database solutions provided
by cloud vendors.


{\def\LTcaptype{none} % do not increment counter
\vspace{-5pt}
\captionof{table}{Major Features}
\vspace{-10pt}
\begin{longtable}[]{@{}ll@{}}
\toprule\noalign{}
Feature & Description \\
\midrule\noalign{}
\endhead
\bottomrule\noalign{}
\endlastfoot
\textbf{Auto-scaling} & Automatic resource adjustment \\
\textbf{Backup \& Recovery} & Automated data protection \\
\textbf{High Availability} & 99.9\% uptime guarantee \\
\end{longtable}
}

\end{solutionbox}
\begin{mnemonicbox}
``Databases Auto-scale, Backup, and stay Available''

\end{mnemonicbox}
\begin{center}\rule{0.5\linewidth}{0.5pt}\end{center}

\subsection*{Question 4(b) OR [4
marks]}\label{q4b}

\textbf{Justify with example: Data durability is an essential feature of
cloud storage.}

\begin{solutionbox}

\textbf{Data Durability} ensures data persists over time without loss or
corruption.

\textbf{Importance:}

\begin{itemize}
\tightlist
\item
  \textbf{Data Protection}: Prevents permanent data loss
\item
  \textbf{Business Continuity}: Critical for operations
\item
  \textbf{Compliance}: Required by regulations
\end{itemize}

\textbf{Example}: Amazon S3 provides 99.999999999\% (11 9's) durability
by storing data across multiple facilities and creating multiple copies.

\end{solutionbox}
\begin{mnemonicbox}
``Durability Delivers Data Protection''

\end{mnemonicbox}
\begin{center}\rule{0.5\linewidth}{0.5pt}\end{center}

\subsection*{Question 4(c) OR [7
marks]}\label{q4c}

\textbf{Explain data scaling and replication in detail.}

\begin{solutionbox}

\textbf{Data Scaling} is the ability to handle increased data load by
adding resources.


{\def\LTcaptype{none} % do not increment counter
\vspace{-5pt}
\captionof{table}{Scaling Types}
\vspace{-10pt}
\begin{longtable}[]{@{}
  >{\raggedright\arraybackslash}p{(\linewidth - 4\tabcolsep) * \real{0.2222}}
  >{\raggedright\arraybackslash}p{(\linewidth - 4\tabcolsep) * \real{0.4815}}
  >{\raggedright\arraybackslash}p{(\linewidth - 4\tabcolsep) * \real{0.2963}}@{}}
\toprule\noalign{}
\begin{minipage}[b]{\linewidth}\raggedright
Type
\end{minipage} & \begin{minipage}[b]{\linewidth}\raggedright
Description
\end{minipage} & \begin{minipage}[b]{\linewidth}\raggedright
Method
\end{minipage} \\
\midrule\noalign{}
\endhead
\bottomrule\noalign{}
\endlastfoot
\textbf{Vertical Scaling} & Adding more power to existing machine &
Increase CPU, RAM \\
\textbf{Horizontal Scaling} & Adding more machines & Add more servers \\
\end{longtable}
}

\textbf{Data Replication} creates copies of data across multiple
locations.


{\def\LTcaptype{none} % do not increment counter
\vspace{-5pt}
\captionof{table}{Replication Types}
\vspace{-10pt}
\begin{longtable}[]{@{}lll@{}}
\toprule\noalign{}
Type & Description & Use Case \\
\midrule\noalign{}
\endhead
\bottomrule\noalign{}
\endlastfoot
\textbf{Synchronous} & Real-time data copying & Critical applications \\
\textbf{Asynchronous} & Delayed data copying & Backup systems \\
\end{longtable}
}

\textbf{Diagram:}

\begin{center}
\textbf{Mermaid Diagram (Code)}
\begin{verbatim}
{Shaded}
{Highlighting}[]
graph TD
    A[Master Database] {-{-}{} B[Replica 1]}
    A {-{-}{} C[Replica 2]}
    A {-{-}{} D[Replica 3]}
{Highlighting}
{Shaded}
\end{verbatim}
\end{center}

\end{solutionbox}
\begin{mnemonicbox}
``Scale Up or Scale Out, Replicate for Reliability''

\end{mnemonicbox}
\begin{center}\rule{0.5\linewidth}{0.5pt}\end{center}

\subsection*{Question 5(a) [3 marks]}\label{q5a}

\textbf{Justify: Authentication and access control are two different
aspects of security in cloud computing.}

\begin{solutionbox}


{\def\LTcaptype{none} % do not increment counter
\vspace{-5pt}
\captionof{table}{Authentication vs Access Control}
\vspace{-10pt}
\begin{longtable}[]{@{}lll@{}}
\toprule\noalign{}
Aspect & \textbf{Authentication} & \textbf{Access Control} \\
\midrule\noalign{}
\endhead
\bottomrule\noalign{}
\endlastfoot
\textbf{Purpose} & Verify user identity & Determine permissions \\
\textbf{Question} & ``Who are you?'' & ``What can you do?'' \\
\textbf{Methods} & Passwords, biometrics & Roles, policies \\
\end{longtable}
}

\textbf{Justification}: Authentication verifies identity first, then
access control determines what authenticated user can access.

\end{solutionbox}
\begin{mnemonicbox}
``Authenticate first, Authorize second''

\end{mnemonicbox}
\begin{center}\rule{0.5\linewidth}{0.5pt}\end{center}

\subsection*{Question 5(b) [4 marks]}\label{q5b}

\textbf{State the role of machine learning in the cloud. Justify: Cloud
computing aids in the task of machine learning.}

\begin{solutionbox}

\textbf{ML Role in Cloud:}

\begin{itemize}
\tightlist
\item
  \textbf{Data Processing}: Handle large datasets efficiently
\item
  \textbf{Model Training}: Scalable computing for complex algorithms
\item
  \textbf{Deployment}: Easy model hosting and serving
\end{itemize}

\textbf{Justification}: Cloud provides necessary computational power,
storage, and tools that make ML accessible without huge infrastructure
investment.


{\def\LTcaptype{none} % do not increment counter
\vspace{-5pt}
\captionof{table}{Cloud ML Benefits}
\vspace{-10pt}
\begin{longtable}[]{@{}ll@{}}
\toprule\noalign{}
Benefit & Description \\
\midrule\noalign{}
\endhead
\bottomrule\noalign{}
\endlastfoot
\textbf{Scalability} & Handle massive datasets \\
\textbf{Cost-Effective} & Pay-per-use model \\
\textbf{Accessibility} & Pre-built ML services \\
\end{longtable}
}

\end{solutionbox}
\begin{mnemonicbox}
``Cloud Computes ML Models''

\end{mnemonicbox}
\begin{center}\rule{0.5\linewidth}{0.5pt}\end{center}

\subsection*{Question 5(c) [7 marks]}\label{q5c}

\textbf{Explain cloud security challenges.}

\begin{solutionbox}


{\def\LTcaptype{none} % do not increment counter
\vspace{-5pt}
\captionof{table}{Major Cloud Security Challenges}
\vspace{-10pt}
\begin{longtable}[]{@{}
  >{\raggedright\arraybackslash}p{(\linewidth - 4\tabcolsep) * \real{0.3333}}
  >{\raggedright\arraybackslash}p{(\linewidth - 4\tabcolsep) * \real{0.3939}}
  >{\raggedright\arraybackslash}p{(\linewidth - 4\tabcolsep) * \real{0.2727}}@{}}
\toprule\noalign{}
\begin{minipage}[b]{\linewidth}\raggedright
Challenge
\end{minipage} & \begin{minipage}[b]{\linewidth}\raggedright
Description
\end{minipage} & \begin{minipage}[b]{\linewidth}\raggedright
Impact
\end{minipage} \\
\midrule\noalign{}
\endhead
\bottomrule\noalign{}
\endlastfoot
\textbf{Data Breaches} & Unauthorized access to sensitive data &
Financial loss, reputation damage \\
\textbf{Identity Management} & Managing user access and permissions &
Security vulnerabilities \\
\textbf{Compliance} & Meeting regulatory requirements & Legal issues,
penalties \\
\textbf{Multi-tenancy} & Shared resources among users & Data isolation
concerns \\
\textbf{Vendor Lock-in} & Dependency on single provider & Limited
security options \\
\end{longtable}
}

\textbf{Security Layers:}

\begin{center}
\textbf{Mermaid Diagram (Code)}
\begin{verbatim}
{Shaded}
{Highlighting}[]
graph LR
    A[Application Security] {-{-}{} B[Data Security]}
    B {-{-}{} C[Network Security]}
    C {-{-}{} D[Infrastructure Security]}
{Highlighting}
{Shaded}
\end{verbatim}
\end{center}

\textbf{Mitigation Strategies:}

\begin{itemize}
\tightlist
\item
  \textbf{Encryption}: Protect data in transit and at rest
\item
  \textbf{Monitoring}: Continuous security assessment
\item
  \textbf{Access Controls}: Role-based permissions
\end{itemize}

\end{solutionbox}
\begin{mnemonicbox}
``Data, Identity, Compliance Challenges''

\end{mnemonicbox}
\begin{center}\rule{0.5\linewidth}{0.5pt}\end{center}

\subsection*{Question 5(a) OR [3
marks]}\label{q5a}

\textbf{State the role of identity access management.}

\begin{solutionbox}

\textbf{Identity Access Management (IAM)} controls who can access what
resources in cloud systems.


{\def\LTcaptype{none} % do not increment counter
\vspace{-5pt}
\captionof{table}{IAM Functions}
\vspace{-10pt}
\begin{longtable}[]{@{}ll@{}}
\toprule\noalign{}
Function & Description \\
\midrule\noalign{}
\endhead
\bottomrule\noalign{}
\endlastfoot
\textbf{Authentication} & Verify user identity \\
\textbf{Authorization} & Grant appropriate permissions \\
\textbf{Audit} & Track access activities \\
\end{longtable}
}

\end{solutionbox}
\begin{mnemonicbox}
``IAM Identifies, Authorizes, Audits''

\end{mnemonicbox}
\begin{center}\rule{0.5\linewidth}{0.5pt}\end{center}

\subsection*{Question 5(b) OR [4
marks]}\label{q5b}

\textbf{Define Kubernetes. Explain with reason: Kubernetes is an
essential component of cloud computing.}

\begin{solutionbox}

\textbf{Kubernetes} is an open-source container orchestration platform
that automates deployment, scaling, and management of applications.

\textbf{Justification}: Kubernetes is essential because it:

\begin{itemize}
\tightlist
\item
  \textbf{Automates Deployment}: Simplifies application management
\item
  \textbf{Ensures Scalability}: Handles varying workloads automatically
\item
  \textbf{Provides Reliability}: Self-healing capabilities
\end{itemize}


{\def\LTcaptype{none} % do not increment counter
\vspace{-5pt}
\captionof{table}{Kubernetes Benefits}
\vspace{-10pt}
\begin{longtable}[]{@{}ll@{}}
\toprule\noalign{}
Benefit & Description \\
\midrule\noalign{}
\endhead
\bottomrule\noalign{}
\endlastfoot
\textbf{Portability} & Run anywhere consistently \\
\textbf{Efficiency} & Optimal resource utilization \\
\textbf{Automation} & Reduces manual operations \\
\end{longtable}
}

\end{solutionbox}
\begin{mnemonicbox}
``Kubernetes Orchestrates Containers''

\end{mnemonicbox}
\begin{center}\rule{0.5\linewidth}{0.5pt}\end{center}

\subsection*{Question 5(c) OR [7
marks]}\label{q5c}

\textbf{Explain DevSecOps (Development Security and Operations).}

\begin{solutionbox}

\textbf{DevSecOps} integrates security practices into DevOps pipeline
from development to deployment.

\textbf{Traditional vs DevSecOps:}

\begin{center}
\textbf{Mermaid Diagram (Code)}
\begin{verbatim}
{Shaded}
{Highlighting}[]
graph LR
    A[Development] {-{-}{} B[Security Testing]}
    B {-{-}{} C[Operations]}
    D[DevSecOps: Security Integrated Throughout]
{Highlighting}
{Shaded}
\end{verbatim}
\end{center}


{\def\LTcaptype{none} % do not increment counter
\vspace{-5pt}
\captionof{table}{DevSecOps Principles}
\vspace{-10pt}
\begin{longtable}[]{@{}
  >{\raggedright\arraybackslash}p{(\linewidth - 4\tabcolsep) * \real{0.2750}}
  >{\raggedright\arraybackslash}p{(\linewidth - 4\tabcolsep) * \real{0.3250}}
  >{\raggedright\arraybackslash}p{(\linewidth - 4\tabcolsep) * \real{0.4000}}@{}}
\toprule\noalign{}
\begin{minipage}[b]{\linewidth}\raggedright
Principle
\end{minipage} & \begin{minipage}[b]{\linewidth}\raggedright
Description
\end{minipage} & \begin{minipage}[b]{\linewidth}\raggedright
Implementation
\end{minipage} \\
\midrule\noalign{}
\endhead
\bottomrule\noalign{}
\endlastfoot
\textbf{Shift Left} & Early security testing & Security in code
review \\
\textbf{Automation} & Automated security scans & CI/CD security tools \\
\textbf{Collaboration} & Security as shared responsibility & Cross-team
security training \\
\textbf{Continuous Monitoring} & Ongoing security assessment & Real-time
threat detection \\
\end{longtable}
}

\textbf{Benefits:}

\begin{itemize}
\tightlist
\item
  \textbf{Faster Delivery}: Security doesn't slow development
\item
  \textbf{Reduced Risks}: Early vulnerability detection
\item
  \textbf{Cost Savings}: Fix issues before production
\end{itemize}

\textbf{Tools:}

\begin{itemize}
\tightlist
\item
  \textbf{SAST}: Static Application Security Testing
\item
  \textbf{DAST}: Dynamic Application Security Testing
\item
  \textbf{Container Scanning}: Docker security tools
\end{itemize}

\end{solutionbox}
\begin{mnemonicbox}
``DevSecOps Develops Securely from Start''

\end{mnemonicbox}

\end{document}
