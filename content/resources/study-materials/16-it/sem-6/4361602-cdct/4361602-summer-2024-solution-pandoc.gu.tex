\documentclass[10pt,a4paper]{article}

% content/resources/templates/preamble.tex
\usepackage[margin=0.6in]{geometry}
\author{Milav Dabgar}
\usepackage{amsmath,amssymb,amsthm}
\usepackage{booktabs}
\usepackage{multirow}
\usepackage{xcolor}
\usepackage{tcolorbox}
\tcbuselibrary{breakable,skins}
\usepackage[colorlinks=true,linkcolor=blue]{hyperref}
\usepackage{titlesec}
\usepackage{enumitem}
\usepackage{tikz}
\usepackage{pgfplots}
\usepackage{circuitikz}
\usepackage[version=4]{mhchem}
\usepackage{longtable}
\usepackage{array}
\usepackage{float}
\usepackage{caption}
\usepackage{listings}

\lstset{
  basicstyle=\small\ttfamily,
  breaklines=true,
  breakatwhitespace=false,
  postbreak=\mbox{\textcolor{red}{$\hookrightarrow$}\space},
  float=false,
  numbers=left,
  numberstyle=\tiny\color{gray},
  numbersep=10pt,
  xleftmargin=2em,
  keywordstyle=\color{blue},
  commentstyle=\color{green!60!black},
  stringstyle=\color{purple},
  backgroundcolor=\color{gray!5},
  showstringspaces=false,
  tabsize=2,
  captionpos=b,
  keepspaces=true,
  columns=flexible
}

\pgfplotsset{compat=1.18}
\usetikzlibrary{shapes,arrows,positioning,calc,patterns,decorations.pathmorphing,decorations.markings,arrows.meta}

% Color scheme
\definecolor{headcolor}{RGB}{0,102,204}
\definecolor{keycolor}{RGB}{220,20,60}
\definecolor{solutioncolor}{RGB}{34,139,34}
\definecolor{mnemoniccolor}{RGB}{148,0,211}
\definecolor{codecolor}{RGB}{0,0,100}

% Spacing
\setlength{\parskip}{3pt}
\setlist[itemize]{nosep}
\setlist[enumerate]{nosep}

% Title formatting
\titleformat{\section}{\Large\bfseries\color{headcolor}}{\thesection}{1em}{}
\titleformat{\subsection}{\large\bfseries\color{headcolor}}{\thesubsection}{1em}{}

% Pandoc tightlist compatibility
\providecommand{\tightlist}{%
  \setlength{\itemsep}{0pt}\setlength{\parskip}{0pt}}

% Pandoc longtable compatibility
\newcounter{none}
\def\thenone{}


% content/resources/templates/gujarati-boxes.tex
\usepackage{fontspec}
\usepackage{polyglossia}

% Set Gujarati as main language (document is primarily in Gujarati)
% Note: gloss-gujarati.ldf doesn't exist in polyglossia, but it will use hyphenation patterns
\setdefaultlanguage{gujarati}
\setotherlanguage{english}

% Configure Gujarati font properly
% Use Language=Default to prevent polyglossia from trying to add language-specific features
% that don't exist for Gujarati, which causes "empty feature" warnings
\newfontfamily\gujaratifont[Script=Gujarati,AutoFakeBold=2.5,AutoFakeSlant=0.3]{Noto Sans Gujarati}
\setmainfont[Script=Gujarati,AutoFakeBold=2.5,AutoFakeSlant=0.3]{Noto Sans Gujarati}
% Use Noto Sans Gujarati for monospace to support Gujarati in text
\setmonofont[Scale=0.9]{Noto Sans Gujarati}

% Configure English to use the same font
\newfontfamily\englishfont[Script=Gujarati,AutoFakeBold=2.5,AutoFakeSlant=0.3]{Noto Sans Gujarati}

% Translations for polyglossia
\gappto\captionsgujarati{
  \renewcommand{\tablename}{કોષ્ટક}
  \renewcommand{\figurename}{આકૃતિ}
}

% Helper for TikZ nodes to ensure Gujarati font
\newcommand{\gu}[1]{{\gujaratifont #1}}

% Custom environments
\newtcolorbox{solutionbox}{
    breakable,
    enhanced,
    colback=solutioncolor!5!white,
    colframe=solutioncolor!75!black,
    fonttitle=\bfseries,
    title=જવાબ
}

\newtcolorbox{solutionboxnobreak}{
 colback=solutioncolor!5!white,
 colframe=solutioncolor!75!black,
 fonttitle=\bfseries,
 title=જવાબ
}

\newtcolorbox{keyformula}{
 breakable,
 enhanced,
 colback=keycolor!5!white,
 colframe=keycolor!75!black,
 fonttitle=\bfseries,
 title=રાસાયણિક સમીકરણ/સૂત્ર
}

\newtcolorbox{mnemonicbox}{
 breakable,
 enhanced,
 colback=mnemoniccolor!5!white,
 colframe=mnemoniccolor!75!black,
 fonttitle=\bfseries,
 title=મેમરી ટ્રીક
}


\begin{document}

\begin{center}
{\Huge\bfseries\color{headcolor} Subject Name (Gujarati)}\\[5pt]
{\LARGE 4361602 -- Summer 2024}\\[3pt]
{\large Semester 1 Study Material}\\[3pt]
{\normalsize\textit{Detailed Solutions and Explanations}}
\end{center}

\vspace{10pt}

\subsection*{પ્રશ્ન 1(અ) [3
ગુણ]}\label{uxaaauxab0uxab6uxaa8-1uxa85-3-uxa97uxaa3}

\textbf{Cloud computing ની વ્યાખ્યા આપો. Cloud computing ઉપયોગ કરવાના
કોઈપણ બે ફાયદા સમજાવો.}

\begin{solutionbox}

\textbf{Cloud Computing} એ ઈન્ટરનેટ દ્વારા computing services જેમ કે servers,
storage, databases અને software પ્રદાન કરવાની સેવા છે.

\textbf{ટેબલ: Cloud Computing ના ફાયદા}

{\def\LTcaptype{none} % do not increment counter
\begin{longtable}[]{@{}ll@{}}
\toprule\noalign{}
ફાયદો & વર્ણન \\
\midrule\noalign{}
\endhead
\bottomrule\noalign{}
\endlastfoot
\textbf{કિંમત અસરકારક} & કોઈ upfront hardware ખર્ચ નથી, ઉપયોગ પ્રમાણે
ચુકવણી \\
\textbf{સ્કેલેબિલિટી} & માંગ પ્રમાણે resources વધારી કે ઘટાડી શકાય \\
\end{longtable}
}

\end{solutionbox}
\begin{mnemonicbox}
``Cloud Saves Cash'' (કિંમત અસરકારક, સ્કેલેબલ)

\end{mnemonicbox}
\begin{center}\rule{0.5\linewidth}{0.5pt}\end{center}

\subsection*{પ્રશ્ન 1(બ) [4
ગુણ]}\label{uxaaauxab0uxab6uxaa8-1uxaac-4-uxa97uxaa3}

\textbf{Cloud service models ની યાદી બનાવો. જસ્ટિફાઈ કરો: Infrastructure
as a service model એ cloud computing structure નો આધાર છે.}

\begin{solutionbox}

\textbf{ટેબલ: Cloud Service Models}

{\def\LTcaptype{none} % do not increment counter
\begin{longtable}[]{@{}
  >{\raggedright\arraybackslash}p{(\linewidth - 4\tabcolsep) * \real{0.2727}}
  >{\raggedright\arraybackslash}p{(\linewidth - 4\tabcolsep) * \real{0.4091}}
  >{\raggedright\arraybackslash}p{(\linewidth - 4\tabcolsep) * \real{0.3182}}@{}}
\toprule\noalign{}
\begin{minipage}[b]{\linewidth}\raggedright
મોડલ
\end{minipage} & \begin{minipage}[b]{\linewidth}\raggedright
પૂરું નામ
\end{minipage} & \begin{minipage}[b]{\linewidth}\raggedright
વર્ણન
\end{minipage} \\
\midrule\noalign{}
\endhead
\bottomrule\noalign{}
\endlastfoot
\textbf{IaaS} & Infrastructure as a Service & Virtual machines, storage,
networks \\
\textbf{PaaS} & Platform as a Service & Development platforms અને
tools \\
\textbf{SaaS} & Software as a Service & તૈયાર applications \\
\end{longtable}
}

\textbf{જસ્ટિફિકેશન}: IaaS એ foundation છે કારણ કે તે basic computing
infrastructure (servers, storage, networking) પ્રદાન કરે છે જેના ઉપર PaaS અને
SaaS બનાવવામાં આવે છે.

\end{solutionbox}
\begin{mnemonicbox}
``I Pay for Software'' (IaaS, PaaS, SaaS)

\end{mnemonicbox}
\begin{center}\rule{0.5\linewidth}{0.5pt}\end{center}

\subsection*{પ્રશ્ન 1(ક) [7
ગુણ]}\label{uxaaauxab0uxab6uxaa8-1uxa95-7-uxa97uxaa3}

\textbf{Edge અને fog computing વચ્ચે તફાવત કરો.}

\begin{solutionbox}

\textbf{ટેબલ: Edge vs Fog Computing}

{\def\LTcaptype{none} % do not increment counter
\begin{longtable}[]{@{}
  >{\raggedright\arraybackslash}p{(\linewidth - 4\tabcolsep) * \real{0.1364}}
  >{\raggedright\arraybackslash}p{(\linewidth - 4\tabcolsep) * \real{0.4318}}
  >{\raggedright\arraybackslash}p{(\linewidth - 4\tabcolsep) * \real{0.4318}}@{}}
\toprule\noalign{}
\begin{minipage}[b]{\linewidth}\raggedright
પાસું
\end{minipage} & \begin{minipage}[b]{\linewidth}\raggedright
\textbf{Edge Computing}
\end{minipage} & \begin{minipage}[b]{\linewidth}\raggedright
\textbf{Fog Computing}
\end{minipage} \\
\midrule\noalign{}
\endhead
\bottomrule\noalign{}
\endlastfoot
\textbf{સ્થાન} & Device level પર (endpoints) & Cloud અને edge વચ્ચે \\
\textbf{Latency} & અત્યંત ઓછી (milliseconds) & ઓછી (થોડી seconds) \\
\textbf{Processing} & મર્યાદિત local processing & વિતરિત processing \\
\textbf{Storage} & ન્યૂનતમ local storage & મધ્યમ storage capacity \\
\textbf{ઉપયોગ} & IoT sensors, autonomous vehicles & Smart cities,
industrial IoT \\
\end{longtable}
}

\textbf{ડાયાગ્રામ:}

\begin{center}
\textbf{Mermaid Diagram (Code)}
\begin{verbatim}
{Shaded}
{Highlighting}[]
graph LR
    A[Cloud Data Center] {-{-}{} B[Fog Layer]}
    B {-{-}{} C[Edge Devices]}
    B {-{-}{} D[Edge Devices]}
    B {-{-}{} E[Edge Devices]}
{Highlighting}
{Shaded}
\end{verbatim}
\end{center}

\end{solutionbox}
\begin{mnemonicbox}
``Edge is Extremely close, Fog is Further''

\end{mnemonicbox}
\begin{center}\rule{0.5\linewidth}{0.5pt}\end{center}

\subsection*{પ્રશ્ન 1(ક) OR [7
ગુણ]}\label{uxaaauxab0uxab6uxaa8-1uxa95-or-7-uxa97uxaa3}

\textbf{Cloud computing માં વપરાતી distributed ledger technology સમજાવો.}

\begin{solutionbox}

\textbf{Distributed Ledger Technology (DLT)} એ cloud computing માં
multiple nodes પર ફેલાયેલ decentralized database છે.

\textbf{મુખ્ય લક્ષણો:}

\begin{itemize}
\tightlist
\item
  \textbf{Decentralization}: કોઈ single point of failure નથી
\item
  \textbf{Immutability}: એકવાર add કર્યા પછી records બદલી શકાતા નથી
\item
  \textbf{Transparency}: બધા participants transactions જોઈ શકે છે
\item
  \textbf{Consensus}: નવી entries માટે agreement જરૂરી
\end{itemize}

\textbf{ટેબલ: Cloud માં DLT ના ફાયદા}

{\def\LTcaptype{none} % do not increment counter
\begin{longtable}[]{@{}ll@{}}
\toprule\noalign{}
ફાયદો & વર્ણન \\
\midrule\noalign{}
\endhead
\bottomrule\noalign{}
\endlastfoot
\textbf{સુરક્ષા} & Cryptography દ્વારા વધુ data protection \\
\textbf{વિશ્વાસ} & Intermediaries ની જરૂર નથી \\
\textbf{Audit Trail} & સંપૂર્ણ transaction history \\
\end{longtable}
}

\end{solutionbox}
\begin{mnemonicbox}
``DLT Delivers Trusted Security''

\end{mnemonicbox}
\begin{center}\rule{0.5\linewidth}{0.5pt}\end{center}

\subsection*{પ્રશ્ન 2(અ) [3
ગુણ]}\label{uxaaauxab0uxab6uxaa8-2uxa85-3-uxa97uxaa3}

\textbf{Virtualization environment ના મુખ્ય components ની યાદી બનાવો અને
સમજાવો.}

\begin{solutionbox}

\textbf{ટેબલ: Virtualization Components}

{\def\LTcaptype{none} % do not increment counter
\begin{longtable}[]{@{}ll@{}}
\toprule\noalign{}
Component & વર્ણન \\
\midrule\noalign{}
\endhead
\bottomrule\noalign{}
\endlastfoot
\textbf{Hypervisor} & Virtual machines manage કરતું software \\
\textbf{Virtual Machines} & અલગ computing environments \\
\textbf{Host OS} & Hypervisor ચલાવતું operating system \\
\end{longtable}
}

\end{solutionbox}
\begin{mnemonicbox}
``Hypervisor Handles Virtual Machines''

\end{mnemonicbox}
\begin{center}\rule{0.5\linewidth}{0.5pt}\end{center}

\subsection*{પ્રશ્ન 2(બ) [4
ગુણ]}\label{uxaaauxab0uxab6uxaa8-2uxaac-4-uxa97uxaa3}

\textbf{ઉદાહરણ સાથે વાજબી જવાબ આપો: Small અને midcap કંપનીઓ માટે resources
ખરીદવા કરતાં cloud ઉપર ભાડે લેવા વધુ હિતાવહ છે.}

\begin{solutionbox}

\textbf{Cloud Renting ના ફાયદા:}

\begin{itemize}
\tightlist
\item
  \textbf{ઓછો પ્રારંભિક ખર્ચ}: Hardware માં upfront investment નહીં
\item
  \textbf{લવચીકતા}: માંગ પ્રમાણે resources scale કરી શકાય
\item
  \textbf{Maintenance-Free}: Provider updates અને repairs સંભાળે છે
\end{itemize}

\textbf{ઉદાહરણ}: એક startup ને peak season દરમિયાન જ servers જોઈએ છે.
ખરીદવાનો ખર્ચ ₹10 લાખ છે, જ્યારે cloud renting નો ખર્ચ 3 મહિનાના ઉપયોગ માટે
₹50,000 છે.

\end{solutionbox}
\begin{mnemonicbox}
``Rent for Flexibility, Buy for Permanency''

\end{mnemonicbox}
\begin{center}\rule{0.5\linewidth}{0.5pt}\end{center}

\subsection*{પ્રશ્ન 2(ક) [7
ગુણ]}\label{uxaaauxab0uxab6uxaa8-2uxa95-7-uxa97uxaa3}

\textbf{Hypervisor ને તેના પ્રકારો સાથે સમજાવો.}

\begin{solutionbox}

\textbf{Hypervisor} એ software છે જે hardware resources ને abstract કરીને
virtual machines બનાવે અને manage કરે છે.

\textbf{ટેબલ: Hypervisor પ્રકારો}

{\def\LTcaptype{none} % do not increment counter
\begin{longtable}[]{@{}
  >{\raggedright\arraybackslash}p{(\linewidth - 6\tabcolsep) * \real{0.2308}}
  >{\raggedright\arraybackslash}p{(\linewidth - 6\tabcolsep) * \real{0.1923}}
  >{\raggedright\arraybackslash}p{(\linewidth - 6\tabcolsep) * \real{0.2692}}
  >{\raggedright\arraybackslash}p{(\linewidth - 6\tabcolsep) * \real{0.3077}}@{}}
\toprule\noalign{}
\begin{minipage}[b]{\linewidth}\raggedright
પ્રકાર
\end{minipage} & \begin{minipage}[b]{\linewidth}\raggedright
નામ
\end{minipage} & \begin{minipage}[b]{\linewidth}\raggedright
વર્ણન
\end{minipage} & \begin{minipage}[b]{\linewidth}\raggedright
ઉદાહરણો
\end{minipage} \\
\midrule\noalign{}
\endhead
\bottomrule\noalign{}
\endlastfoot
\textbf{Type 1} & Bare Metal & Hardware પર સીધું ચાલે છે & VMware ESXi,
Hyper-V \\
\textbf{Type 2} & Hosted & Host operating system પર ચાલે છે & VirtualBox,
VMware Workstation \\
\end{longtable}
}

\textbf{ડાયાગ્રામ:}

\begin{verbatim}
Type 1 (Bare Metal)          Type 2 (Hosted)
┌─────────────────┐          ┌─────────────────┐
│    VM1  │  VM2  │          │    VM1  │  VM2  │
├─────────┼───────┤          ├─────────┼───────┤
│   Hypervisor    │          │   Hypervisor    │
├─────────────────┤          ├─────────────────┤
│    Hardware     │          │    Host OS      │
└─────────────────┘          ├─────────────────┤
                             │    Hardware     │
                             └─────────────────┘
\end{verbatim}

\end{solutionbox}
\begin{mnemonicbox}
``Type 1 Touches Hardware, Type 2 Touches OS''

\end{mnemonicbox}
\begin{center}\rule{0.5\linewidth}{0.5pt}\end{center}

\subsection*{પ્રશ્ન 2(અ) OR [3
ગુણ]}\label{uxaaauxab0uxab6uxaa8-2uxa85-or-3-uxa97uxaa3}

\textbf{Virtualization ઉપયોગ કરવાના ફાયદાઓની યાદી બનાવો. કોઈપણ એક
સમજાવો.}

\begin{solutionbox}

\textbf{Virtualization ના ફાયદા:}

\begin{itemize}
\tightlist
\item
  \textbf{Resource Optimization}: બહેતર hardware ઉપયોગ
\item
  \textbf{ખર્ચમાં ઘટાડો}: ઓછા physical servers જોઈએ
\item
  \textbf{Isolation}: Applications સ્વતંત્ર રીતે ચાલે છે
\end{itemize}

\textbf{Resource Optimization}: એક physical server પર ઘણા virtual
machines ચાલી શકે છે, hardware capacity નો 80-90\% ઉપયોગ થાય છે સામાન્ય
15-20\% ને બદલે.

\end{solutionbox}
\begin{mnemonicbox}
``Virtualization Optimizes Resources''

\end{mnemonicbox}
\begin{center}\rule{0.5\linewidth}{0.5pt}\end{center}

\subsection*{પ્રશ્ન 2(બ) OR [4
ગુણ]}\label{uxaaauxab0uxab6uxaa8-2uxaac-or-4-uxa97uxaa3}

\textbf{Application-level virtualization સમજાવો.}

\begin{solutionbox}

\textbf{Application-level virtualization} applications ને host OS પર
install કર્યા વિના isolated environments માં ચલાવવાની મંજૂરી આપે છે.

\textbf{ટેબલ: Application Virtualization Features}

{\def\LTcaptype{none} % do not increment counter
\begin{longtable}[]{@{}ll@{}}
\toprule\noalign{}
લક્ષણ & વર્ણન \\
\midrule\noalign{}
\endhead
\bottomrule\noalign{}
\endlastfoot
\textbf{Isolation} & Apps એકબીજાને અસર કરતા નથી \\
\textbf{Portability} & Apps વિવિધ OS પર modification વિના ચાલે છે \\
\textbf{Security} & Sandboxed execution environment \\
\end{longtable}
}

\textbf{ઉદાહરણ}: Docker containers જે applications ને તેમની dependencies
સાથે package કરીને ચલાવે છે.

\end{solutionbox}
\begin{mnemonicbox}
``Apps Are Isolated and Portable''

\end{mnemonicbox}
\begin{center}\rule{0.5\linewidth}{0.5pt}\end{center}

\subsection*{પ્રશ્ન 2(ક) OR [7
ગુણ]}\label{uxaaauxab0uxab6uxaa8-2uxa95-or-7-uxa97uxaa3}

\textbf{Cloud માં hardware virtualization સમજાવો.}

\begin{solutionbox}

\textbf{Hardware virtualization} cloud environments માં physical hardware
components ના virtual versions બનાવે છે.

\textbf{મુખ્ય Components:}

\begin{itemize}
\tightlist
\item
  \textbf{CPU Virtualization}: ઘણા VMs physical processor share કરે છે
\item
  \textbf{Memory Virtualization}: VMs ને virtual memory allocation
\item
  \textbf{Storage Virtualization}: Storage resources ને abstract કરે છે
\item
  \textbf{Network Virtualization}: Virtual network interfaces
\end{itemize}

\textbf{ટેબલ: Hardware Virtualization ના ફાયદા}

{\def\LTcaptype{none} % do not increment counter
\begin{longtable}[]{@{}ll@{}}
\toprule\noalign{}
ફાયદો & વર્ણન \\
\midrule\noalign{}
\endhead
\bottomrule\noalign{}
\endlastfoot
\textbf{Resource Sharing} & ઘણા VMs સમાન hardware વાપરે છે \\
\textbf{Isolation} & VMs સ્વતંત્ર રીતે કામ કરે છે \\
\textbf{Migration} & VMs hosts વચ્ચે move કરી શકાય છે \\
\end{longtable}
}

\end{solutionbox}
\begin{mnemonicbox}
``Hardware Hosts Multiple Virtual Machines''

\end{mnemonicbox}
\begin{center}\rule{0.5\linewidth}{0.5pt}\end{center}

\subsection*{પ્રશ્ન 3(અ) [3
ગુણ]}\label{uxaaauxab0uxab6uxaa8-3uxa85-3-uxa97uxaa3}

\textbf{Data Center ની વ્યાખ્યા આપો. Data center ના પ્રકારોની યાદી બનાવો.}

\begin{solutionbox}

\textbf{Data Center} એ computing અને networking equipment રાખવાની સુવિધા છે
જે data store, process અને distribute કરે છે.

\textbf{ટેબલ: Data Center પ્રકારો}

{\def\LTcaptype{none} % do not increment counter
\begin{longtable}[]{@{}ll@{}}
\toprule\noalign{}
પ્રકાર & વર્ણન \\
\midrule\noalign{}
\endhead
\bottomrule\noalign{}
\endlastfoot
\textbf{Enterprise} & Organizations માટે private data centers \\
\textbf{Colocation} & ઘણા clients માટે shared facilities \\
\textbf{Cloud} & Virtualized, scalable data centers \\
\end{longtable}
}

\end{solutionbox}
\begin{mnemonicbox}
``Enterprise, Colocation, Cloud Centers''

\end{mnemonicbox}
\begin{center}\rule{0.5\linewidth}{0.5pt}\end{center}

\subsection*{પ્રશ્ન 3(બ) [4
ગુણ]}\label{uxaaauxab0uxab6uxaa8-3uxaac-4-uxa97uxaa3}

\textbf{Data centre automation કેમ મહત્વનું છે?}

\begin{solutionbox}

\textbf{Data Center Automation ના ફાયદા:}

\begin{itemize}
\tightlist
\item
  \textbf{કાર્યક્ષમતા}: Manual tasks અને errors ઘટાડે છે
\item
  \textbf{ખર્ચમાં બચત}: ઓછા operational expenses
\item
  \textbf{Scalability}: ઝડપી resource provisioning
\item
  \textbf{વિશ્વસનીયતા}: સતત operations અને monitoring
\end{itemize}

\textbf{ટેબલ: Automation ક્ષેત્રો}

{\def\LTcaptype{none} % do not increment counter
\begin{longtable}[]{@{}ll@{}}
\toprule\noalign{}
ક્ષેત્ર & ફાયદો \\
\midrule\noalign{}
\endhead
\bottomrule\noalign{}
\endlastfoot
\textbf{Provisioning} & ઝડપી server deployment \\
\textbf{Monitoring} & Real-time performance tracking \\
\textbf{Maintenance} & Automated updates અને patches \\
\end{longtable}
}

\end{solutionbox}
\begin{mnemonicbox}
``Automation Enhances Efficiency''

\end{mnemonicbox}
\begin{center}\rule{0.5\linewidth}{0.5pt}\end{center}

\subsection*{પ્રશ્ન 3(ક) [7
ગુણ]}\label{uxaaauxab0uxab6uxaa8-3uxa95-7-uxa97uxaa3}

\textbf{SDN (Software Defined Networking) આર્કિટેક્ચર સમજાવો.}

\begin{solutionbox}

\textbf{SDN} network control plane ને data plane થી અલગ કરે છે, centralized
network management શક્ય બનાવે છે.

\textbf{SDN Architecture Layers:}

\begin{center}
\textbf{Mermaid Diagram (Code)}
\begin{verbatim}
{Shaded}
{Highlighting}[]
graph LR
    A[Application Layer] {-{-}{} B[Control Layer]}
    B {-{-}{} C[Infrastructure Layer]}
    A {-.{-}{}|Northbound API| B}
    B {-.{-}{}|Southbound API| C}
{Highlighting}
{Shaded}
\end{verbatim}
\end{center}

\textbf{ટેબલ: SDN Components}

{\def\LTcaptype{none} % do not increment counter
\begin{longtable}[]{@{}ll@{}}
\toprule\noalign{}
Component & કાર્ય \\
\midrule\noalign{}
\endhead
\bottomrule\noalign{}
\endlastfoot
\textbf{Controller} & Centralized network control \\
\textbf{Switches} & Controller આધારિત packets forward કરે છે \\
\textbf{Applications} & Network services અને policies \\
\end{longtable}
}

\textbf{ફાયદા:}

\begin{itemize}
\tightlist
\item
  \textbf{Centralized Control}: Single point of network management
\item
  \textbf{Programmability}: Dynamic network configuration
\item
  \textbf{Flexibility}: સરળ policy implementation
\end{itemize}

\end{solutionbox}
\begin{mnemonicbox}
``SDN Separates Control from Data''

\end{mnemonicbox}
\begin{center}\rule{0.5\linewidth}{0.5pt}\end{center}

\subsection*{પ્રશ્ન 3(અ) OR [3
ગુણ]}\label{uxaaauxab0uxab6uxaa8-3uxa85-or-3-uxa97uxaa3}

\textbf{વ્યાખ્યાયિત કરો: (1) Cloud Elasticity (2) Cloud Scalability}

\begin{solutionbox}

\textbf{ટેબલ: Cloud Elasticity vs Scalability}

{\def\LTcaptype{none} % do not increment counter
\begin{longtable}[]{@{}
  >{\raggedright\arraybackslash}p{(\linewidth - 2\tabcolsep) * \real{0.3846}}
  >{\raggedright\arraybackslash}p{(\linewidth - 2\tabcolsep) * \real{0.6154}}@{}}
\toprule\noalign{}
\begin{minipage}[b]{\linewidth}\raggedright
શબ્દ
\end{minipage} & \begin{minipage}[b]{\linewidth}\raggedright
વ્યાખ્યા
\end{minipage} \\
\midrule\noalign{}
\endhead
\bottomrule\noalign{}
\endlastfoot
\textbf{Cloud Elasticity} & માંગ આધારિત automatic resource adjustment \\
\textbf{Cloud Scalability} & Resources add કરીને વધતી workload handle
કરવાની ક્ષમતા \\
\end{longtable}
}

\textbf{મુખ્ય તફાવત}: Elasticity automatic છે, scalability manual કે
automatic હોઈ શકે છે.

\end{solutionbox}
\begin{mnemonicbox}
``Elasticity is Automatic, Scalability is
Adaptable''

\end{mnemonicbox}
\begin{center}\rule{0.5\linewidth}{0.5pt}\end{center}

\subsection*{પ્રશ્ન 3(બ) OR [4
ગુણ]}\label{uxaaauxab0uxab6uxaa8-3uxaac-or-4-uxa97uxaa3}

\textbf{કારણ સાથે સમજાવો: Cloud computing માં Vendor lock-in એ એક મોટી
સમસ્યા છે.}

\begin{solutionbox}

\textbf{Vendor Lock-in} ત્યારે થાય છે જ્યારે specific services પર dependency
ના કારણે cloud providers બદલવું મુશ્કેલ બને છે.

\textbf{સમસ્યાઓ:}

\begin{itemize}
\tightlist
\item
  \textbf{ઊંચા Migration Costs}: Data transfer અને application
  modification ખર્ચ
\item
  \textbf{મર્યાદિત લવચીકતા}: Providers ની મર્યાદિત પસંદગી
\item
  \textbf{Dependency}: Single vendor ની technologies પર આધારિતતા
\end{itemize}

\textbf{ઉદાહરણ}: AWS-specific services વાપરવાથી Google Cloud પર
migration મોંઘું અને જટિલ બને છે.

\end{solutionbox}
\begin{mnemonicbox}
``Lock-in Limits Liberty''

\end{mnemonicbox}
\begin{center}\rule{0.5\linewidth}{0.5pt}\end{center}

\subsection*{પ્રશ્ન 3(ક) OR [7
ગુણ]}\label{uxaaauxab0uxab6uxaa8-3uxa95-or-7-uxa97uxaa3}

\textbf{Infrastructure as Code (IaC) ને તેના different approaches સાથે
સમજાવો.}

\begin{solutionbox}

\textbf{Infrastructure as Code (IaC)} manual processes ને બદલે code દ્વારા
infrastructure manage કરે છે.

\textbf{ટેબલ: IaC Approaches}

{\def\LTcaptype{none} % do not increment counter
\begin{longtable}[]{@{}
  >{\raggedright\arraybackslash}p{(\linewidth - 4\tabcolsep) * \real{0.4167}}
  >{\raggedright\arraybackslash}p{(\linewidth - 4\tabcolsep) * \real{0.2917}}
  >{\raggedright\arraybackslash}p{(\linewidth - 4\tabcolsep) * \real{0.2917}}@{}}
\toprule\noalign{}
\begin{minipage}[b]{\linewidth}\raggedright
Approach
\end{minipage} & \begin{minipage}[b]{\linewidth}\raggedright
વર્ણન
\end{minipage} & \begin{minipage}[b]{\linewidth}\raggedright
Tools
\end{minipage} \\
\midrule\noalign{}
\endhead
\bottomrule\noalign{}
\endlastfoot
\textbf{Declarative} & ઇચ્છિત end state define કરે છે & Terraform, ARM
templates \\
\textbf{Imperative} & Step-by-step instructions define કરે છે & Scripts,
Ansible \\
\textbf{Hybrid} & બંને approaches નું combination & Pulumi \\
\end{longtable}
}

\textbf{ફાયદા:}

\begin{itemize}
\tightlist
\item
  \textbf{Consistency}: Repeatable infrastructure deployment
\item
  \textbf{Version Control}: Infrastructure changes track કરે છે
\item
  \textbf{Automation}: Manual configuration errors ઘટાડે છે
\end{itemize}

\textbf{ડાયાગ્રામ:}

\begin{center}
\textbf{Mermaid Diagram (Code)}
\begin{verbatim}
{Shaded}
{Highlighting}[]
graph LR
    A[Code] {-{-}{} B[IaC Tool]}
    B {-{-}{} C[Cloud Provider]}
    C {-{-}{} D[Infrastructure]}
{Highlighting}
{Shaded}
\end{verbatim}
\end{center}

\end{solutionbox}
\begin{mnemonicbox}
``IaC Codes Infrastructure''

\end{mnemonicbox}
\begin{center}\rule{0.5\linewidth}{0.5pt}\end{center}

\subsection*{પ્રશ્ન 4(અ) [3
ગુણ]}\label{uxaaauxab0uxab6uxaa8-4uxa85-3-uxa97uxaa3}

\textbf{Cloud storage ની વ્યાખ્યા આપો. મુખ્ય cloud storage સોલ્યુશન્સ આપતી
સર્વિસીસની યાદી બનાવો.}

\begin{solutionbox}

\textbf{Cloud Storage} એ ઈન્ટરનેટ દ્વારા accessible remote servers પર data
store કરવાની સેવા છે.

\textbf{ટેબલ: મુખ્ય Cloud Storage Solutions}

{\def\LTcaptype{none} % do not increment counter
\begin{longtable}[]{@{}lll@{}}
\toprule\noalign{}
Provider & Service & પ્રકાર \\
\midrule\noalign{}
\endhead
\bottomrule\noalign{}
\endlastfoot
\textbf{Amazon} & S3 & Object Storage \\
\textbf{Google} & Cloud Storage & Object Storage \\
\textbf{Microsoft} & Azure Blob & Object Storage \\
\end{longtable}
}

\end{solutionbox}
\begin{mnemonicbox}
``Amazon, Google, Microsoft Store Objects''

\end{mnemonicbox}
\begin{center}\rule{0.5\linewidth}{0.5pt}\end{center}

\subsection*{પ્રશ્ન 4(બ) [4
ગુણ]}\label{uxaaauxab0uxab6uxaa8-4uxaac-4-uxa97uxaa3}

\textbf{ઉદાહરણ સાથે સમર્થન આપો: Data consistency એ cloud storage ની આવશ્યક
વિશેષતા છે.}

\begin{solutionbox}

\textbf{Data Consistency} ખાતરી કરે છે કે distributed systems માં data ની
બધી copies સમાન value બતાવે છે.

\textbf{મહત્વ:}

\begin{itemize}
\tightlist
\item
  \textbf{વિશ્વસનીયતા}: Users ને હંમેશા સાચો data મળે છે
\item
  \textbf{Integrity}: Data corruption અટકાવે છે
\item
  \textbf{Synchronization}: ઘણા users સમાન information જુએ છે
\end{itemize}

\textbf{ઉદાહરણ}: Banking system માં account balance બધા ATMs અને branches
માં consistent હોવું જોઈએ double spending અટકાવવા માટે.

\end{solutionbox}
\begin{mnemonicbox}
``Consistency Creates Confidence''

\end{mnemonicbox}
\begin{center}\rule{0.5\linewidth}{0.5pt}\end{center}

\subsection*{પ્રશ્ન 4(ક) [7
ગુણ]}\label{uxaaauxab0uxab6uxaa8-4uxa95-7-uxa97uxaa3}

\textbf{Cloud databases ના પ્રકારો વિગતવાર સમજાવો.}

\begin{solutionbox}

\textbf{ટેબલ: Cloud Database પ્રકારો}

{\def\LTcaptype{none} % do not increment counter
\begin{longtable}[]{@{}
  >{\raggedright\arraybackslash}p{(\linewidth - 6\tabcolsep) * \real{0.2143}}
  >{\raggedright\arraybackslash}p{(\linewidth - 6\tabcolsep) * \real{0.2500}}
  >{\raggedright\arraybackslash}p{(\linewidth - 6\tabcolsep) * \real{0.3214}}
  >{\raggedright\arraybackslash}p{(\linewidth - 6\tabcolsep) * \real{0.2143}}@{}}
\toprule\noalign{}
\begin{minipage}[b]{\linewidth}\raggedright
પ્રકાર
\end{minipage} & \begin{minipage}[b]{\linewidth}\raggedright
વર્ણન
\end{minipage} & \begin{minipage}[b]{\linewidth}\raggedright
ઉદાહરણો
\end{minipage} & \begin{minipage}[b]{\linewidth}\raggedright
ઉપયોગ
\end{minipage} \\
\midrule\noalign{}
\endhead
\bottomrule\noalign{}
\endlastfoot
\textbf{SQL Databases} & ACID properties સાથે relational databases &
Amazon RDS, Azure SQL & Transaction processing \\
\textbf{NoSQL Databases} & Non-relational, flexible schema & MongoDB
Atlas, DynamoDB & Big data, real-time web apps \\
\textbf{In-Memory} & ઝડપ માટે RAM માં data stored & Redis, Memcached &
Caching, real-time analytics \\
\textbf{Graph Databases} & Relationship-focused data storage & Neo4j,
Amazon Neptune & Social networks, recommendations \\
\end{longtable}
}

\textbf{SQL vs NoSQL તુલના:}

\begin{center}
\textbf{Mermaid Diagram (Code)}
\begin{verbatim}
{Shaded}
{Highlighting}[]
graph LR
    A[Structured Data] {-{-}{} B[SQL Database]}
    C[Unstructured Data] {-{-}{} D[NoSQL Database]}
    B {-{-}{} E[ACID Compliance]}
    D {-{-}{} F[High Scalability]}
{Highlighting}
{Shaded}
\end{verbatim}
\end{center}

\end{solutionbox}
\begin{mnemonicbox}
``SQL for Structure, NoSQL for Scale''

\end{mnemonicbox}
\begin{center}\rule{0.5\linewidth}{0.5pt}\end{center}

\subsection*{પ્રશ્ન 4(અ) OR [3
ગુણ]}\label{uxaaauxab0uxab6uxaa8-4uxa85-or-3-uxa97uxaa3}

\textbf{Cloud માં database services ની વ્યાખ્યા આપો. Database services ના
મુખ્ય લક્ષણોની યાદી બનાવો.}

\begin{solutionbox}

\textbf{Cloud Database Services} એ cloud vendors દ્વારા પ્રદાન કરવામાં આવતા
managed database solutions છે.

\textbf{ટેબલ: મુખ્ય લક્ષણો}

{\def\LTcaptype{none} % do not increment counter
\begin{longtable}[]{@{}ll@{}}
\toprule\noalign{}
લક્ષણ & વર્ણન \\
\midrule\noalign{}
\endhead
\bottomrule\noalign{}
\endlastfoot
\textbf{Auto-scaling} & Automatic resource adjustment \\
\textbf{Backup \& Recovery} & Automated data protection \\
\textbf{High Availability} & 99.9\% uptime guarantee \\
\end{longtable}
}

\end{solutionbox}
\begin{mnemonicbox}
``Databases Auto-scale, Backup, and stay Available''

\end{mnemonicbox}
\begin{center}\rule{0.5\linewidth}{0.5pt}\end{center}

\subsection*{પ્રશ્ન 4(બ) OR [4
ગુણ]}\label{uxaaauxab0uxab6uxaa8-4uxaac-or-4-uxa97uxaa3}

\textbf{ઉદાહરણ સાથે સમર્થન આપો: Data durability એ cloud storage ની આવશ્યક
વિશેષતા છે.}

\begin{solutionbox}

\textbf{Data Durability} ખાતરી કરે છે કે data loss કે corruption વિના સમય
સાથે ટકી રહે છે.

\textbf{મહત્વ:}

\begin{itemize}
\tightlist
\item
  \textbf{Data Protection}: કાયમી data loss અટકાવે છે
\item
  \textbf{Business Continuity}: Operations માટે જરૂરી છે
\item
  \textbf{Compliance}: Regulations દ્વારા જરૂરી છે
\end{itemize}

\textbf{ઉદાહરણ}: Amazon S3 multiple facilities માં data store કરીને અને
multiple copies બનાવીને 99.999999999\% (11 9's) durability પ્રદાન કરે છે.

\end{solutionbox}
\begin{mnemonicbox}
``Durability Delivers Data Protection''

\end{mnemonicbox}
\begin{center}\rule{0.5\linewidth}{0.5pt}\end{center}

\subsection*{પ્રશ્ન 4(ક) OR [7
ગુણ]}\label{uxaaauxab0uxab6uxaa8-4uxa95-or-7-uxa97uxaa3}

\textbf{Data scaling અને replication વિગતવાર સમજાવો.}

\begin{solutionbox}

\textbf{Data Scaling} એ resources add કરીને વધતા data load ને handle
કરવાની ક્ષમતા છે.

\textbf{ટેબલ: Scaling પ્રકારો}

{\def\LTcaptype{none} % do not increment counter
\begin{longtable}[]{@{}
  >{\raggedright\arraybackslash}p{(\linewidth - 4\tabcolsep) * \real{0.3158}}
  >{\raggedright\arraybackslash}p{(\linewidth - 4\tabcolsep) * \real{0.3684}}
  >{\raggedright\arraybackslash}p{(\linewidth - 4\tabcolsep) * \real{0.3158}}@{}}
\toprule\noalign{}
\begin{minipage}[b]{\linewidth}\raggedright
પ્રકાર
\end{minipage} & \begin{minipage}[b]{\linewidth}\raggedright
વર્ણન
\end{minipage} & \begin{minipage}[b]{\linewidth}\raggedright
પદ્ધતિ
\end{minipage} \\
\midrule\noalign{}
\endhead
\bottomrule\noalign{}
\endlastfoot
\textbf{Vertical Scaling} & વર્તમાન machine માં વધુ power add કરવું & CPU,
RAM વધારવું \\
\textbf{Horizontal Scaling} & વધુ machines add કરવા & વધુ servers add
કરવા \\
\end{longtable}
}

\textbf{Data Replication} ઘણા સ્થળોએ data ની copies બનાવે છે.

\textbf{ટેબલ: Replication પ્રકારો}

{\def\LTcaptype{none} % do not increment counter
\begin{longtable}[]{@{}lll@{}}
\toprule\noalign{}
પ્રકાર & વર્ણન & ઉપયોગ \\
\midrule\noalign{}
\endhead
\bottomrule\noalign{}
\endlastfoot
\textbf{Synchronous} & Real-time data copying & Critical applications \\
\textbf{Asynchronous} & Delayed data copying & Backup systems \\
\end{longtable}
}

\textbf{ડાયાગ્રામ:}

\begin{center}
\textbf{Mermaid Diagram (Code)}
\begin{verbatim}
{Shaded}
{Highlighting}[]
graph TD
    A[Master Database] {-{-}{} B[Replica 1]}
    A {-{-}{} C[Replica 2]}
    A {-{-}{} D[Replica 3]}
{Highlighting}
{Shaded}
\end{verbatim}
\end{center}

\end{solutionbox}
\begin{mnemonicbox}
``Scale Up or Scale Out, Replicate for Reliability''

\end{mnemonicbox}
\begin{center}\rule{0.5\linewidth}{0.5pt}\end{center}

\subsection*{પ્રશ્ન 5(અ) [3
ગુણ]}\label{uxaaauxab0uxab6uxaa8-5uxa85-3-uxa97uxaa3}

\textbf{જસ્ટિફાઈ કરો: Cloud computing માં authentication અને access control
એ સુરક્ષાના બે અલગ અલગ પાસાઓ છે.}

\begin{solutionbox}

\textbf{ટેબલ: Authentication vs Access Control}

{\def\LTcaptype{none} % do not increment counter
\begin{longtable}[]{@{}lll@{}}
\toprule\noalign{}
પાસું & \textbf{Authentication} & \textbf{Access Control} \\
\midrule\noalign{}
\endhead
\bottomrule\noalign{}
\endlastfoot
\textbf{હેતુ} & User identity verify કરવું & Permissions નક્કી કરવા \\
\textbf{પ્રશ્ન} & ``તમે કોણ છો?'' & ``તમે શું કરી શકો છો?'' \\
\textbf{પદ્ધતિઓ} & Passwords, biometrics & Roles, policies \\
\end{longtable}
}

\textbf{જસ્ટિફિકેશન}: Authentication પહેલા identity verify કરે છે, પછી access
control નક્કી કરે છે કે authenticated user શું access કરી શકે છે.

\end{solutionbox}
\begin{mnemonicbox}
``Authenticate first, Authorize second''

\end{mnemonicbox}
\begin{center}\rule{0.5\linewidth}{0.5pt}\end{center}

\subsection*{પ્રશ્ન 5(બ) [4
ગુણ]}\label{uxaaauxab0uxab6uxaa8-5uxaac-4-uxa97uxaa3}

\textbf{Cloud માં machine learning ની ભૂમિકા જણાવો. જસ્ટિફાઈ કરો: Cloud
computing એ machine learning ના કાર્યમાં મદદ કરે છે.}

\begin{solutionbox}

\textbf{Cloud માં ML ની ભૂમિકા:}

\begin{itemize}
\tightlist
\item
  \textbf{Data Processing}: મોટા datasets ને કાર્યક્ષમ રીતે handle કરે છે
\item
  \textbf{Model Training}: જટિલ algorithms માટે scalable computing
\item
  \textbf{Deployment}: સરળ model hosting અને serving
\end{itemize}

\textbf{જસ્ટિફિકેશન}: Cloud જરૂરી computational power, storage અને tools
પ્રદાન કરે છે જે મોટા infrastructure investment વિના ML ને accessible બનાવે છે.

\textbf{ટેબલ: Cloud ML ના ફાયદા}

{\def\LTcaptype{none} % do not increment counter
\begin{longtable}[]{@{}ll@{}}
\toprule\noalign{}
ફાયદો & વર્ણન \\
\midrule\noalign{}
\endhead
\bottomrule\noalign{}
\endlastfoot
\textbf{Scalability} & વિશાળ datasets handle કરે છે \\
\textbf{Cost-Effective} & Pay-per-use model \\
\textbf{Accessibility} & Pre-built ML services \\
\end{longtable}
}

\end{solutionbox}
\begin{mnemonicbox}
``Cloud Computes ML Models''

\end{mnemonicbox}
\begin{center}\rule{0.5\linewidth}{0.5pt}\end{center}

\subsection*{પ્રશ્ન 5(ક) [7
ગુણ]}\label{uxaaauxab0uxab6uxaa8-5uxa95-7-uxa97uxaa3}

\textbf{Cloud માં security પડકારો સમજાવો.}

\begin{solutionbox}

\textbf{ટેબલ: મુખ્ય Cloud Security Challenges}

{\def\LTcaptype{none} % do not increment counter
\begin{longtable}[]{@{}
  >{\raggedright\arraybackslash}p{(\linewidth - 4\tabcolsep) * \real{0.3684}}
  >{\raggedright\arraybackslash}p{(\linewidth - 4\tabcolsep) * \real{0.3684}}
  >{\raggedright\arraybackslash}p{(\linewidth - 4\tabcolsep) * \real{0.2632}}@{}}
\toprule\noalign{}
\begin{minipage}[b]{\linewidth}\raggedright
પડકાર
\end{minipage} & \begin{minipage}[b]{\linewidth}\raggedright
વર્ણન
\end{minipage} & \begin{minipage}[b]{\linewidth}\raggedright
અસર
\end{minipage} \\
\midrule\noalign{}
\endhead
\bottomrule\noalign{}
\endlastfoot
\textbf{Data Breaches} & Sensitive data માં unauthorized access & નાણાકીય
નુકસાન, reputation damage \\
\textbf{Identity Management} & User access અને permissions manage કરવા &
Security vulnerabilities \\
\textbf{Compliance} & Regulatory requirements પૂરા કરવા & કાનૂની મુદ્દાઓ,
penalties \\
\textbf{Multi-tenancy} & Users વચ્ચે shared resources & Data isolation
concerns \\
\textbf{Vendor Lock-in} & Single provider પર dependency & મર્યાદિત
security options \\
\end{longtable}
}

\textbf{Security Layers:}

\begin{center}
\textbf{Mermaid Diagram (Code)}
\begin{verbatim}
{Shaded}
{Highlighting}[]
graph LR
    A[Application Security] {-{-}{} B[Data Security]}
    B {-{-}{} C[Network Security]}
    C {-{-}{} D[Infrastructure Security]}
{Highlighting}
{Shaded}
\end{verbatim}
\end{center}

\textbf{Mitigation Strategies:}

\begin{itemize}
\tightlist
\item
  \textbf{Encryption}: Transit અને rest માં data protect કરવું
\item
  \textbf{Monitoring}: સતત security assessment
\item
  \textbf{Access Controls}: Role-based permissions
\end{itemize}

\end{solutionbox}
\begin{mnemonicbox}
``Data, Identity, Compliance Challenges''

\end{mnemonicbox}
\begin{center}\rule{0.5\linewidth}{0.5pt}\end{center}

\subsection*{પ્રશ્ન 5(અ) OR [3
ગુણ]}\label{uxaaauxab0uxab6uxaa8-5uxa85-or-3-uxa97uxaa3}

\textbf{Identity access management ની ભૂમિકા જણાવો.}

\begin{solutionbox}

\textbf{Identity Access Management (IAM)} cloud systems માં કોણ કયા
resources access કરી શકે છે તે control કરે છે.

\textbf{ટેબલ: IAM કાર્યો}

{\def\LTcaptype{none} % do not increment counter
\begin{longtable}[]{@{}ll@{}}
\toprule\noalign{}
કાર્ય & વર્ણન \\
\midrule\noalign{}
\endhead
\bottomrule\noalign{}
\endlastfoot
\textbf{Authentication} & User identity verify કરવું \\
\textbf{Authorization} & યોગ્ય permissions આપવા \\
\textbf{Audit} & Access activities track કરવા \\
\end{longtable}
}

\end{solutionbox}
\begin{mnemonicbox}
``IAM Identifies, Authorizes, Audits''

\end{mnemonicbox}
\begin{center}\rule{0.5\linewidth}{0.5pt}\end{center}

\subsection*{પ્રશ્ન 5(બ) OR [4
ગુણ]}\label{uxaaauxab0uxab6uxaa8-5uxaac-or-4-uxa97uxaa3}

\textbf{Kubernetes ની વ્યાખ્યા આપો. કારણ સાથે સમજાવો: Kubernetes એ cloud
computing નો આવશ્યક ભાગ છે.}

\begin{solutionbox}

\textbf{Kubernetes} એ open-source container orchestration platform છે જે
applications ની deployment, scaling અને management automate કરે છે.

\textbf{જસ્ટિફિકેશન}: Kubernetes આવશ્યક છે કારણ કે તે:

\begin{itemize}
\tightlist
\item
  \textbf{Automates Deployment}: Application management સરળ બનાવે છે
\item
  \textbf{Ensures Scalability}: વિવિધ workloads automatically handle કરે
  છે
\item
  \textbf{Provides Reliability}: Self-healing capabilities છે
\end{itemize}

\textbf{ટેબલ: Kubernetes ના ફાયદા}

{\def\LTcaptype{none} % do not increment counter
\begin{longtable}[]{@{}ll@{}}
\toprule\noalign{}
ફાયદો & વર્ણન \\
\midrule\noalign{}
\endhead
\bottomrule\noalign{}
\endlastfoot
\textbf{Portability} & કોઈપણ જગ્યાએ consistently ચાલે છે \\
\textbf{Efficiency} & Optimal resource utilization \\
\textbf{Automation} & Manual operations ઘટાડે છે \\
\end{longtable}
}

\end{solutionbox}
\begin{mnemonicbox}
``Kubernetes Orchestrates Containers''

\end{mnemonicbox}
\begin{center}\rule{0.5\linewidth}{0.5pt}\end{center}

\subsection*{પ્રશ્ન 5(ક) OR [7
ગુણ]}\label{uxaaauxab0uxab6uxaa8-5uxa95-or-7-uxa97uxaa3}

\textbf{DevSecOps (Development Security and Operations) સમજાવો.}

\begin{solutionbox}

\textbf{DevSecOps} development થી deployment સુધી DevOps pipeline માં
security practices integrate કરે છે.

\textbf{Traditional vs DevSecOps:}

\begin{center}
\textbf{Mermaid Diagram (Code)}
\begin{verbatim}
{Shaded}
{Highlighting}[]
graph LR
    A[Development] {-{-}{} B[Security Testing]}
    B {-{-}{} C[Operations]}
    D[DevSecOps: Security Integrated Throughout]
{Highlighting}
{Shaded}
\end{verbatim}
\end{center}

\textbf{ટેબલ: DevSecOps સિદ્ધાંતો}

{\def\LTcaptype{none} % do not increment counter
\begin{longtable}[]{@{}
  >{\raggedright\arraybackslash}p{(\linewidth - 4\tabcolsep) * \real{0.3600}}
  >{\raggedright\arraybackslash}p{(\linewidth - 4\tabcolsep) * \real{0.2800}}
  >{\raggedright\arraybackslash}p{(\linewidth - 4\tabcolsep) * \real{0.3600}}@{}}
\toprule\noalign{}
\begin{minipage}[b]{\linewidth}\raggedright
સિદ્ધાંત
\end{minipage} & \begin{minipage}[b]{\linewidth}\raggedright
વર્ણન
\end{minipage} & \begin{minipage}[b]{\linewidth}\raggedright
અમલીકરણ
\end{minipage} \\
\midrule\noalign{}
\endhead
\bottomrule\noalign{}
\endlastfoot
\textbf{Shift Left} & પ્રારંભિક security testing & Code review માં
security \\
\textbf{Automation} & Automated security scans & CI/CD security tools \\
\textbf{Collaboration} & Security shared responsibility તરીકે &
Cross-team security training \\
\textbf{Continuous Monitoring} & સતત security assessment & Real-time
threat detection \\
\end{longtable}
}

\textbf{ફાયદા:}

\begin{itemize}
\tightlist
\item
  \textbf{ઝડપી Delivery}: Security development ધીમું કરતું નથી
\item
  \textbf{ઘટાડેલ Risks}: પ્રારંભિક vulnerability detection
\item
  \textbf{ખર્ચમાં બચત}: Production પહેલા issues fix કરવા
\end{itemize}

\textbf{Tools:}

\begin{itemize}
\tightlist
\item
  \textbf{SAST}: Static Application Security Testing
\item
  \textbf{DAST}: Dynamic Application Security Testing
\item
  \textbf{Container Scanning}: Docker security tools
\end{itemize}

\end{solutionbox}
\begin{mnemonicbox}
``DevSecOps Develops Securely from Start''

\end{mnemonicbox}

\end{document}
