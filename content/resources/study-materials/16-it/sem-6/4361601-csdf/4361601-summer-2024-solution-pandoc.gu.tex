\documentclass[10pt,a4paper]{article}

% content/resources/templates/preamble.tex
\usepackage[margin=0.6in]{geometry}
\author{Milav Dabgar}
\usepackage{amsmath,amssymb,amsthm}
\usepackage{booktabs}
\usepackage{multirow}
\usepackage{xcolor}
\usepackage{tcolorbox}
\tcbuselibrary{breakable,skins}
\usepackage[colorlinks=true,linkcolor=blue]{hyperref}
\usepackage{titlesec}
\usepackage{enumitem}
\usepackage{tikz}
\usepackage{pgfplots}
\usepackage{circuitikz}
\usepackage[version=4]{mhchem}
\usepackage{longtable}
\usepackage{array}
\usepackage{float}
\usepackage{caption}
\usepackage{listings}

\lstset{
  basicstyle=\small\ttfamily,
  breaklines=true,
  breakatwhitespace=false,
  postbreak=\mbox{\textcolor{red}{$\hookrightarrow$}\space},
  float=false,
  numbers=left,
  numberstyle=\tiny\color{gray},
  numbersep=10pt,
  xleftmargin=2em,
  keywordstyle=\color{blue},
  commentstyle=\color{green!60!black},
  stringstyle=\color{purple},
  backgroundcolor=\color{gray!5},
  showstringspaces=false,
  tabsize=2,
  captionpos=b,
  keepspaces=true,
  columns=flexible
}

\pgfplotsset{compat=1.18}
\usetikzlibrary{shapes,arrows,positioning,calc,patterns,decorations.pathmorphing,decorations.markings,arrows.meta}

% Color scheme
\definecolor{headcolor}{RGB}{0,102,204}
\definecolor{keycolor}{RGB}{220,20,60}
\definecolor{solutioncolor}{RGB}{34,139,34}
\definecolor{mnemoniccolor}{RGB}{148,0,211}
\definecolor{codecolor}{RGB}{0,0,100}

% Spacing
\setlength{\parskip}{3pt}
\setlist[itemize]{nosep}
\setlist[enumerate]{nosep}

% Title formatting
\titleformat{\section}{\Large\bfseries\color{headcolor}}{\thesection}{1em}{}
\titleformat{\subsection}{\large\bfseries\color{headcolor}}{\thesubsection}{1em}{}

% Pandoc tightlist compatibility
\providecommand{\tightlist}{%
  \setlength{\itemsep}{0pt}\setlength{\parskip}{0pt}}

% Pandoc longtable compatibility
\newcounter{none}
\def\thenone{}


% content/resources/templates/gujarati-boxes.tex
\usepackage{fontspec}
\usepackage{polyglossia}

% Set Gujarati as main language (document is primarily in Gujarati)
% Note: gloss-gujarati.ldf doesn't exist in polyglossia, but it will use hyphenation patterns
\setdefaultlanguage{gujarati}
\setotherlanguage{english}

% Configure Gujarati font properly
% Use Language=Default to prevent polyglossia from trying to add language-specific features
% that don't exist for Gujarati, which causes "empty feature" warnings
\newfontfamily\gujaratifont[Script=Gujarati,AutoFakeBold=2.5,AutoFakeSlant=0.3]{Noto Sans Gujarati}
\setmainfont[Script=Gujarati,AutoFakeBold=2.5,AutoFakeSlant=0.3]{Noto Sans Gujarati}
% Use Noto Sans Gujarati for monospace to support Gujarati in text
\setmonofont[Scale=0.9]{Noto Sans Gujarati}

% Configure English to use the same font
\newfontfamily\englishfont[Script=Gujarati,AutoFakeBold=2.5,AutoFakeSlant=0.3]{Noto Sans Gujarati}

% Translations for polyglossia
\gappto\captionsgujarati{
  \renewcommand{\tablename}{કોષ્ટક}
  \renewcommand{\figurename}{આકૃતિ}
}

% Helper for TikZ nodes to ensure Gujarati font
\newcommand{\gu}[1]{{\gujaratifont #1}}

% Custom environments
\newtcolorbox{solutionbox}{
    breakable,
    enhanced,
    colback=solutioncolor!5!white,
    colframe=solutioncolor!75!black,
    fonttitle=\bfseries,
    title=જવાબ
}

\newtcolorbox{solutionboxnobreak}{
 colback=solutioncolor!5!white,
 colframe=solutioncolor!75!black,
 fonttitle=\bfseries,
 title=જવાબ
}

\newtcolorbox{keyformula}{
 breakable,
 enhanced,
 colback=keycolor!5!white,
 colframe=keycolor!75!black,
 fonttitle=\bfseries,
 title=રાસાયણિક સમીકરણ/સૂત્ર
}

\newtcolorbox{mnemonicbox}{
 breakable,
 enhanced,
 colback=mnemoniccolor!5!white,
 colframe=mnemoniccolor!75!black,
 fonttitle=\bfseries,
 title=મેમરી ટ્રીક
}


\begin{document}

\begin{center}
{\Huge\bfseries\color{headcolor} Subject Name (Gujarati)}\\[5pt]
{\LARGE 4361601 -- Summer 2024}\\[3pt]
{\large Semester 1 Study Material}\\[3pt]
{\normalsize\textit{Detailed Solutions and Explanations}}
\end{center}

\vspace{10pt}

\subsection*{પ્રશ્ન 1(અ) [3
ગુણ]}\label{uxaaauxab0uxab6uxaa8-1uxa85-3-uxa97uxaa3}

\textbf{ઉદાહરણ સાથે CIA ત્રિપુટીનું વર્ણન કરો.}

\begin{solutionbox}

\textbf{CIA ત્રિપુટી તુલના કોષ્ટક:}

{\def\LTcaptype{none} % do not increment counter
\begin{longtable}[]{@{}
  >{\raggedright\arraybackslash}p{(\linewidth - 4\tabcolsep) * \real{0.3438}}
  >{\raggedright\arraybackslash}p{(\linewidth - 4\tabcolsep) * \real{0.3750}}
  >{\raggedright\arraybackslash}p{(\linewidth - 4\tabcolsep) * \real{0.2812}}@{}}
\toprule\noalign{}
\begin{minipage}[b]{\linewidth}\raggedright
ઘટક
\end{minipage} & \begin{minipage}[b]{\linewidth}\raggedright
વ્યાખ્યા
\end{minipage} & \begin{minipage}[b]{\linewidth}\raggedright
ઉદાહરણ
\end{minipage} \\
\midrule\noalign{}
\endhead
\bottomrule\noalign{}
\endlastfoot
\textbf{ગુપ્તતા (Confidentiality)} & ડેટા માત્ર અધિકૃત વપરાશકર્તાઓને જ ઉપલબ્ધ
હોય & બેંક એકાઉન્ટની વિગતો માત્ર એકાઉન્ટ ધારકને જ દેખાવી જોઈએ \\
\textbf{અખંડતા (Integrity)} & ડેટા સચોટ અને અપરિવર્તિત રહે & મેડિકલ રેકોર્ડ
અધિકૃતતા વિના બદલાવા જોઈએ નહીં \\
\textbf{ઉપલબ્ધતા (Availability)} & સિસ્ટમ અને ડેટા જરૂર પડે ત્યારે ઉપલબ્ધ હોય &
ATM સેવાઓ ગ્રાહકો માટે 24/7 ઉપલબ્ધ હોવી જોઈએ \\
\end{longtable}
}

\end{solutionbox}
\begin{mnemonicbox}
``ગુઆ'' - ગુપ્તતા, અખંડતા, ઉપલબ્ધતા

\end{mnemonicbox}
\begin{center}\rule{0.5\linewidth}{0.5pt}\end{center}

\subsection*{પ્રશ્ન 1(બ) [4
ગુણ]}\label{uxaaauxab0uxab6uxaa8-1uxaac-4-uxa97uxaa3}

\textbf{પબ્લિક કી અને પ્રાઇવેટ કી ક્રિપ્ટોગ્રાફી સમજાવો.}

\begin{solutionbox}

\textbf{મુખ્ય તફાવતો કોષ્ટક:}

{\def\LTcaptype{none} % do not increment counter
\begin{longtable}[]{@{}lll@{}}
\toprule\noalign{}
પાસું & Public Key Cryptography & Private Key Cryptography \\
\midrule\noalign{}
\endhead
\bottomrule\noalign{}
\endlastfoot
\textbf{વપરાતી કી} & બે કી (public + private) & એક શેર કરેલી કી \\
\textbf{કી વિતરણ} & Public કી ખુલ્લેઆમ શેર કરી શકાય & કી ગુપ્ત રીતે શેર કરવી
પડે \\
\textbf{ઝડપ} & ધીમી encryption/decryption & ઝડપી operations \\
\textbf{સુરક્ષા} & વધુ સુરક્ષિત, કી શેરિંગ સમસ્યા નથી & ઓછી સુરક્ષા કી વિતરણને
કારણે \\
\end{longtable}
}

\textbf{મુખ્ય મુદ્દાઓ:}

\begin{itemize}
\tightlist
\item
  \textbf{Public Key}: asymmetric encryption નો ઉપયોગ કરે છે
\item
  \textbf{Private Key}: symmetric encryption નો ઉપયોગ કરે છે
\item
  \textbf{Digital Signatures}: Public કી non-repudiation શક્ય બનાવે છે
\item
  \textbf{કી મેનેજમેન્ટ}: Private કી સુરક્ષિત વિતરણની જરૂર છે
\end{itemize}

\end{solutionbox}
\begin{mnemonicbox}
``PASS'' - Public Asymmetric, Symmetric Secret

\end{mnemonicbox}
\begin{center}\rule{0.5\linewidth}{0.5pt}\end{center}

\subsection*{પ્રશ્ન 1(ક) [7
ગુણ]}\label{uxaaauxab0uxab6uxaa8-1uxa95-7-uxa97uxaa3}

\textbf{વિવિધ સિક્યુરિટી સર્વિસ અને સિક્યુરિટી મેકેનિઝમ સમજાવો.}

\begin{solutionbox}

\textbf{સિક્યુરિટી સર્વિસ કોષ્ટક:}

{\def\LTcaptype{none} % do not increment counter
\begin{longtable}[]{@{}
  >{\raggedright\arraybackslash}p{(\linewidth - 4\tabcolsep) * \real{0.2500}}
  >{\raggedright\arraybackslash}p{(\linewidth - 4\tabcolsep) * \real{0.2500}}
  >{\raggedright\arraybackslash}p{(\linewidth - 4\tabcolsep) * \real{0.5000}}@{}}
\toprule\noalign{}
\begin{minipage}[b]{\linewidth}\raggedright
સર્વિસ
\end{minipage} & \begin{minipage}[b]{\linewidth}\raggedright
હેતુ
\end{minipage} & \begin{minipage}[b]{\linewidth}\raggedright
મેકેનિઝમ ઉદાહરણ
\end{minipage} \\
\midrule\noalign{}
\endhead
\bottomrule\noalign{}
\endlastfoot
\textbf{Authentication} & વપરાશકર્તાની ઓળખ ચકાસવી & Passwords,
Biometrics \\
\textbf{Authorization} & પ્રવેશ પરવાનગીઓ નિયંત્રિત કરવી & Access Control
Lists \\
\textbf{Confidentiality} & ડેટાની ગોપનીયતા સુરક્ષિત કરવી & Encryption (AES,
RSA) \\
\textbf{Integrity} & ડેટાની સચોટતા સુનિશ્ચિત કરવી & Digital signatures,
Hashing \\
\textbf{Non-repudiation} & ક્રિયાઓના ઇનકારને અટકાવવો & Digital
certificates \\
\textbf{Availability} & સેવાની પહોંચ સુનિશ્ચિત કરવી & Firewalls, Backup
systems \\
\end{longtable}
}

\textbf{સિક્યુરિટી મેકેનિઝમ:}

\begin{itemize}
\tightlist
\item
  \textbf{Encryption}: plaintext ને ciphertext માં ફેરવે છે
\item
  \textbf{Digital Signatures}: authentication અને integrity પૂરી પાડે છે
\item
  \textbf{Access Control}: અનધિકૃત પ્રવેશ પર પ્રતિબંધ મૂકે છે
\item
  \textbf{Audit Trails}: સિક્યુરિટી ઇવેન્ટ્સ મોનિટર અને લોગ કરે છે
\end{itemize}

\end{solutionbox}
\begin{mnemonicbox}
``ACIANA'' - Authentication, Confidentiality,
Integrity, Authorization, Non-repudiation, Availability

\end{mnemonicbox}
\begin{center}\rule{0.5\linewidth}{0.5pt}\end{center}

\subsection*{પ્રશ્ન 1(ક) OR [7
ગુણ]}\label{uxaaauxab0uxab6uxaa8-1uxa95-or-7-uxa97uxaa3}

\textbf{MD5 હેશિંગ અલ્ગોરિધમ સમજાવો.}

\begin{solutionbox}

\textbf{MD5 અલ્ગોરિધમ પ્રક્રિયા:}

\begin{verbatim}
flowchart LR
    A[Input Message] {-{-} B[Padding]}
    B {-{-} C[Append Length]}
    C {-{-} D[Initialize MD Buffer]}
    D {-{-} E[Process in 512{-}bit blocks]}
    E {-{-} F[128{-}bit Hash Output]}
\end{verbatim}

\textbf{MD5 લાક્ષણિકતાઓ કોષ્ટક:}

{\def\LTcaptype{none} % do not increment counter
\begin{longtable}[]{@{}ll@{}}
\toprule\noalign{}
ગુણધર્મ & મૂલ્ય \\
\midrule\noalign{}
\endhead
\bottomrule\noalign{}
\endlastfoot
\textbf{હેશ સાઇઝ} & 128 bits (16 bytes) \\
\textbf{બ્લોક સાઇઝ} & 512 bits \\
\textbf{રાઉન્ડ્સ} & 64 rounds \\
\textbf{સુરક્ષા સ્થિતિ} & ક્રિપ્ટોગ્રાફિકલી ભાંગી ગયેલ \\
\end{longtable}
}

\textbf{મુખ્ય લક્ષણો:}

\begin{itemize}
\tightlist
\item
  \textbf{One-way Function}: હેશથી મૂળ માં પાછા ફેરવી શકાતું નથી
\item
  \textbf{નિશ્ચિત આઉટપુટ}: હંમેશા 128-bit હેશ ઉત્પન્ન કરે છે
\item
  \textbf{Avalanche Effect}: નાનો ઇનપુટ ફેરફાર મોટો આઉટપુટ ફેરફાર બનાવે છે
\item
  \textbf{Collision Vulnerable}: ઘણા ઇનપુટ્સ સમાન હેશ ઉત્પન્ન કરી શકે છે
\end{itemize}

\end{solutionbox}
\begin{mnemonicbox}
``MD5 FORB'' - Message Digest 5, Fixed Output,
Rounds 64, Broken security

\end{mnemonicbox}
\begin{center}\rule{0.5\linewidth}{0.5pt}\end{center}

\subsection*{પ્રશ્ન 2(અ) [3
ગુણ]}\label{uxaaauxab0uxab6uxaa8-2uxa85-3-uxa97uxaa3}

\textbf{ફાયરવોલ શું છે? ફાયરવોલના પ્રકારોની યાદી આપો.}

\begin{solutionbox}

\textbf{ફાયરવોલ વ્યાખ્યા:} નેટવર્ક સિક્યુરિટી ઉપકરણ જે પૂર્વનિર્ધારિત નિયમોના આધારે
આવતા/જતા ટ્રાફિકને મોનિટર અને નિયંત્રિત કરે છે.

\textbf{ફાયરવોલ પ્રકારો કોષ્ટક:}

{\def\LTcaptype{none} % do not increment counter
\begin{longtable}[]{@{}lll@{}}
\toprule\noalign{}
પ્રકાર & ઓપરેશન લેવલ & ઉદાહરણ \\
\midrule\noalign{}
\endhead
\bottomrule\noalign{}
\endlastfoot
\textbf{Packet Filtering} & Network Layer & iptables \\
\textbf{Stateful Inspection} & Session Layer & Cisco ASA \\
\textbf{Application Gateway} & Application Layer & Proxy servers \\
\textbf{Next-Gen Firewall} & Multiple Layers & Palo Alto \\
\end{longtable}
}

\end{solutionbox}
\begin{mnemonicbox}
``PSAN'' - Packet, Stateful, Application, Next-gen

\end{mnemonicbox}
\begin{center}\rule{0.5\linewidth}{0.5pt}\end{center}

\subsection*{પ્રશ્ન 2(બ) [4
ગુણ]}\label{uxaaauxab0uxab6uxaa8-2uxaac-4-uxa97uxaa3}

\textbf{વ્યાખ્યાયિત કરો: HTTPS અને HTTPS ના કાર્યનું વર્ણન કરો.}

\begin{solutionbox}

\textbf{HTTPS વ્યાખ્યા:} HTTP Secure - SSL/TLS protocols નો ઉપયોગ કરીને
HTTP નું એન્ક્રિપ્ટેડ વર્ઝન.

\textbf{HTTPS કાર્ય પ્રક્રિયા:}

\begin{verbatim}
sequenceDiagram
    participant C as Client
    participant S as Server
    C{-S: 1. HTTPS Request}
    S{-C: 2. SSL Certificate}
    C{-S: 3. Verify \& Send Session Key}
    S{-C: 4. Encrypted Communication}
\end{verbatim}

\textbf{મુખ્ય ઘટકો:}

\begin{itemize}
\tightlist
\item
  \textbf{SSL/TLS}: એન્ક્રિપ્શન લેયર પૂરી પાડે છે
\item
  \textbf{Digital Certificates}: સર્વર આઇડેન્ટિટી ચકાસે છે
\item
  \textbf{Port 443}: ડિફોલ્ટ HTTPS પોર્ટ
\item
  \textbf{End-to-end Encryption}: ટ્રાન્ઝિટમાં ડેટાની સુરક્ષા કરે છે
\end{itemize}

\end{solutionbox}
\begin{mnemonicbox}
``HTTPS SDP4'' - Secure, Digital certs, Port 443

\end{mnemonicbox}
\begin{center}\rule{0.5\linewidth}{0.5pt}\end{center}

\subsection*{પ્રશ્ન 2(ક) [7
ગુણ]}\label{uxaaauxab0uxab6uxaa8-2uxa95-7-uxa97uxaa3}

\textbf{Active attack અને passive attack ની વિગતવાર સમજૂતી આપો.}

\begin{solutionbox}

\textbf{હુમલા પ્રકારોની તુલના:}

{\def\LTcaptype{none} % do not increment counter
\begin{longtable}[]{@{}lll@{}}
\toprule\noalign{}
પાસું & Active Attack & Passive Attack \\
\midrule\noalign{}
\endhead
\bottomrule\noalign{}
\endlastfoot
\textbf{શોધ} & સરળતાથી શોધી શકાય છે & શોધવું મુશ્કેલ \\
\textbf{સિસ્ટમ પર અસર} & સિસ્ટમ/ડેટામાં ફેરફાર કરે છે & માત્ર ડેટાનું અવલોકન કરે
છે \\
\textbf{ઉદાહરણો} & DoS, Man-in-middle & Eavesdropping, Traffic
analysis \\
\textbf{અટકાવવાની રીત} & Firewalls, IDS & Encryption, Physical
security \\
\end{longtable}
}

\textbf{Active Attack પ્રકારો:}

\begin{itemize}
\tightlist
\item
  \textbf{Masquerade}: અધિકૃત વપરાશકર્તાની નકલ કરવી
\item
  \textbf{Replay}: માન્ય ડેટા ટ્રાન્સમિશનને ફરીથી મોકલવું
\item
  \textbf{Modification}: સંદેશાની સામગ્રીમાં ફેરફાર કરવો
\item
  \textbf{Denial of Service}: કાયદેસર પ્રવેશને અટકાવવો
\end{itemize}

\textbf{Passive Attack પ્રકારો:}

\begin{itemize}
\tightlist
\item
  \textbf{Traffic Analysis}: કમ્યુનિકેશન પેટર્નનો અભ્યાસ
\item
  \textbf{Eavesdropping}: કમ્યુનિકેશનની મોનિટરિંગ
\item
  \textbf{Footprinting}: સિસ્ટમ માહિતી એકત્રિત કરવી
\end{itemize}

\end{solutionbox}
\begin{mnemonicbox}
``Active MRMD, Passive TEF'' -
Masquerade/Replay/Modify/DoS, Traffic/Eavesdrop/Footprint

\end{mnemonicbox}
\begin{center}\rule{0.5\linewidth}{0.5pt}\end{center}

\subsection*{પ્રશ્ન 2(અ) OR [3
ગુણ]}\label{uxaaauxab0uxab6uxaa8-2uxa85-or-3-uxa97uxaa3}

\textbf{Digital signature શું છે? તેના ગુણધર્મો સમજાવો.}

\begin{solutionbox}

\textbf{Digital Signature:} ક્રિપ્ટોગ્રાફિક મેકેનિઝમ જે authentication,
integrity, અને non-repudiation પૂરી પાડે છે.

\textbf{ગુણધર્મો કોષ્ટક:}

{\def\LTcaptype{none} % do not increment counter
\begin{longtable}[]{@{}ll@{}}
\toprule\noalign{}
ગુણધર્મ & વર્ણન \\
\midrule\noalign{}
\endhead
\bottomrule\noalign{}
\endlastfoot
\textbf{Authentication} & મોકલનારની ઓળખ ચકાસે છે \\
\textbf{Integrity} & સંદેશો અપરિવર્તિત છે તેની ખાતરી કરે છે \\
\textbf{Non-repudiation} & મોકલનારનો ઇનકાર અટકાવે છે \\
\textbf{Unforgeable} & Private કી વિના બનાવી શકાતું નથી \\
\end{longtable}
}

\end{solutionbox}
\begin{mnemonicbox}
``AINU'' - Authentication, Integrity,
Non-repudiation, Unforgeable

\end{mnemonicbox}
\begin{center}\rule{0.5\linewidth}{0.5pt}\end{center}

\subsection*{પ્રશ્ન 2(બ) OR [4
ગુણ]}\label{uxaaauxab0uxab6uxaa8-2uxaac-or-4-uxa97uxaa3}

\textbf{વ્યાખ્યાયિત કરો: ટ્રોજન્સ, રૂટકિટ, બેકડોર્સ, કીલોગર}

\begin{solutionbox}

\textbf{મેલવેર પ્રકારો કોષ્ટક:}

{\def\LTcaptype{none} % do not increment counter
\begin{longtable}[]{@{}
  >{\raggedright\arraybackslash}p{(\linewidth - 4\tabcolsep) * \real{0.1667}}
  >{\raggedright\arraybackslash}p{(\linewidth - 4\tabcolsep) * \real{0.3333}}
  >{\raggedright\arraybackslash}p{(\linewidth - 4\tabcolsep) * \real{0.5000}}@{}}
\toprule\noalign{}
\begin{minipage}[b]{\linewidth}\raggedright
પ્રકાર
\end{minipage} & \begin{minipage}[b]{\linewidth}\raggedright
વ્યાખ્યા
\end{minipage} & \begin{minipage}[b]{\linewidth}\raggedright
મુખ્ય કાર્ય
\end{minipage} \\
\midrule\noalign{}
\endhead
\bottomrule\noalign{}
\endlastfoot
\textbf{Trojans} & કાયદેસર સોફ્ટવેરના વેશમાં દુષ્ટ કોડ & અનધિકૃત પ્રવેશ પૂરો
પાડવો \\
\textbf{Rootkit} & અન્ય મેલવેરની હાજરી છુપાવતું સોફ્ટવેર & દુષ્ટ પ્રવૃત્તિઓ છુપાવવી \\
\textbf{Backdoors} & સુરક્ષાને બાયપાસ કરતું ગુપ્ત પ્રવેશદ્વાર & દૂરસ્થ અનધિકૃત
પ્રવેશ \\
\textbf{Keylogger} & વપરાશકર્તાના કીસ્ટ્રોક રેકોર્ડ કરે છે & પાસવર્ડ/સંવેદનશીલ
ડેટાની ચોરી \\
\end{longtable}
}

\end{solutionbox}
\begin{mnemonicbox}
``TRBK'' - Trojans છુપાવે, Rootkits ગુપ્ત કરે, Backdoors
બાયપાસ કરે, Keyloggers રેકોર્ડ કરે

\end{mnemonicbox}
\begin{center}\rule{0.5\linewidth}{0.5pt}\end{center}

\subsection*{પ્રશ્ન 2(ક) OR [7
ગુણ]}\label{uxaaauxab0uxab6uxaa8-2uxa95-or-7-uxa97uxaa3}

\textbf{Secure Socket Layer સમજાવો.}

\begin{solutionbox}

\textbf{SSL આર્કિટેક્ચર:}

\begin{center}
\textbf{Mermaid Diagram (Code)}
\begin{verbatim}
{Shaded}
{Highlighting}[]
graph LR
    A[Application Layer] {-{-}{} B[SSL Record Protocol]}
    B {-{-}{} C[SSL Handshake Protocol]}
    B {-{-}{} D[SSL Change Cipher]}
    B {-{-}{} E[SSL Alert Protocol]}
    C {-{-}{} F[TCP Layer]}
    D {-{-}{} F}
    E {-{-}{} F}
{Highlighting}
{Shaded}
\end{verbatim}
\end{center}

\textbf{SSL ઘટકો કોષ્ટક:}

{\def\LTcaptype{none} % do not increment counter
\begin{longtable}[]{@{}ll@{}}
\toprule\noalign{}
ઘટક & કાર્ય \\
\midrule\noalign{}
\endhead
\bottomrule\noalign{}
\endlastfoot
\textbf{Record Protocol} & મૂળભૂત સુરક્ષા સેવાઓ પૂરી પાડે છે \\
\textbf{Handshake Protocol} & સુરક્ષા પેરામીટર્સ સ્થાપિત કરે છે \\
\textbf{Change Cipher} & એન્ક્રિપ્શન ફેરફારોનો સંકેત આપે છે \\
\textbf{Alert Protocol} & એરર સ્થિતિઓ સંભાળે છે \\
\end{longtable}
}

\textbf{SSL પ્રક્રિયા:}

\begin{itemize}
\tightlist
\item
  \textbf{Handshake}: સુરક્ષા પેરામીટર્સની વાતચીત
\item
  \textbf{Authentication}: સર્વર આઇડેન્ટિટી ચકાસવી
\item
  \textbf{Key Exchange}: સેશન કી સ્થાપિત કરવી
\item
  \textbf{Encryption}: સુરક્ષિત ડેટા ટ્રાન્સમિશન
\end{itemize}

\end{solutionbox}
\begin{mnemonicbox}
``SSL RHCA-HAKE'' - Record/Handshake/Change/Alert,
Handshake/Auth/Key/Encrypt

\end{mnemonicbox}
\begin{center}\rule{0.5\linewidth}{0.5pt}\end{center}

\subsection*{પ્રશ્ન 3(અ) [3
ગુણ]}\label{uxaaauxab0uxab6uxaa8-3uxa85-3-uxa97uxaa3}

\textbf{સાયબર ક્રાઇમ અને સાયબર ક્રિમિનલને વિગતવાર સમજાવો.}

\begin{solutionbox}

\textbf{વ્યાખ્યાઓ કોષ્ટક:}

{\def\LTcaptype{none} % do not increment counter
\begin{longtable}[]{@{}ll@{}}
\toprule\noalign{}
શબ્દ & વ્યાખ્યા \\
\midrule\noalign{}
\endhead
\bottomrule\noalign{}
\endlastfoot
\textbf{સાયબર ક્રાઇમ} & કમ્પ્યુટર/ઇન્ટરનેટનો ઉપયોગ કરીને કરાતી ગુનાહિત
પ્રવૃત્તિઓ \\
\textbf{સાયબર ક્રિમિનલ} & ડિજિટલ ટેકનોલોજીનો ઉપયોગ કરીને ગુના કરતી વ્યક્તિ \\
\end{longtable}
}

\textbf{સાયબર ક્રિમિનલ પ્રકારો:}

\begin{itemize}
\tightlist
\item
  \textbf{Script Kiddies}: ઊંડા જ્ઞાન વિના હાલના ટૂલ્સનો ઉપયોગ કરે છે
\item
  \textbf{Hacktivists}: રાજકીય/સામાજિક કારણોથી પ્રેરિત
\item
  \textbf{Organized Crime}: વ્યાવસાયિક ગુનાહિત જૂથો
\item
  \textbf{State-sponsored}: સરકાર દ્વારા સમર્થિત હુમલાખોરો
\end{itemize}

\end{solutionbox}
\begin{mnemonicbox}
``SSHT'' - Script kiddies, State-sponsored,
Hacktivists, Teams organized

\end{mnemonicbox}
\begin{center}\rule{0.5\linewidth}{0.5pt}\end{center}

\subsection*{પ્રશ્ન 3(બ) [4
ગુણ]}\label{uxaaauxab0uxab6uxaa8-3uxaac-4-uxa97uxaa3}

\textbf{સાયબર સ્ટોકિંગ અને સાયબર બુલિંગનું વિગતવાર વર્ણન કરો.}

\begin{solutionbox}

\textbf{તુલના કોષ્ટક:}

{\def\LTcaptype{none} % do not increment counter
\begin{longtable}[]{@{}
  >{\raggedright\arraybackslash}p{(\linewidth - 4\tabcolsep) * \real{0.2000}}
  >{\raggedright\arraybackslash}p{(\linewidth - 4\tabcolsep) * \real{0.4000}}
  >{\raggedright\arraybackslash}p{(\linewidth - 4\tabcolsep) * \real{0.4000}}@{}}
\toprule\noalign{}
\begin{minipage}[b]{\linewidth}\raggedright
પાસું
\end{minipage} & \begin{minipage}[b]{\linewidth}\raggedright
સાયબર સ્ટોકિંગ
\end{minipage} & \begin{minipage}[b]{\linewidth}\raggedright
સાયબર બુલિંગ
\end{minipage} \\
\midrule\noalign{}
\endhead
\bottomrule\noalign{}
\endlastfoot
\textbf{લક્ષ્ય} & ચોક્કસ વ્યક્તિ (મોટે ભાગે પુખ્ત) & મોટે ભાગે બાળકો/સાથીદારો \\
\textbf{અવધિ} & લાંબા ગાળાની પરેશાની & એક વખતની અથવા પુનરાવર્તિત હોઈ શકે \\
\textbf{હેતુ} & ધાક, નિયંત્રણ & અપમાન, સામાજિક બહિષ્કાર \\
\textbf{પદ્ધતિઓ} & મોનિટરિંગ, ધમકીભર્યા સંદેશાઓ & સોશિયલ મીડિયા પરેશાની, અફવાઓ
ફેલાવવી \\
\end{longtable}
}

\textbf{સામાન્ય લાક્ષણિકતાઓ:}

\begin{itemize}
\tightlist
\item
  \textbf{ડિજિટલ પ્લેટફોર્મ}: સોશિયલ મીડિયા, ઇમેઇલ, મેસેજિંગ એપ્સ
\item
  \textbf{અનામી}: ગુનેગારો મોટે ભાગે ઓળખ છુપાવે છે
\item
  \textbf{માનસિક અસર}: ભાવનાત્મક તકલીફ પહોંચાડે છે
\item
  \textbf{કાયદેસરી પરિણામો}: સાયબર કાયદાઓનું ઉલ્લંઘન કરે છે
\end{itemize}

\end{solutionbox}
\begin{mnemonicbox}
``STAL-BULL DPAL'' - Digital platforms,
Psychological impact, Anonymity, Legal issues

\end{mnemonicbox}
\begin{center}\rule{0.5\linewidth}{0.5pt}\end{center}

\subsection*{પ્રશ્ન 3(ક) [7
ગુણ]}\label{uxaaauxab0uxab6uxaa8-3uxa95-7-uxa97uxaa3}

\textbf{સાયબર ક્રાઇમમાં પ્રોપર્ટી બેઇઝ ક્લાસિફિકેશન સમજાવો.}

\begin{solutionbox}

\textbf{પ્રોપર્ટી-આધારિત સાયબર ક્રાઇમ વર્ગીકરણ:}

{\def\LTcaptype{none} % do not increment counter
\begin{longtable}[]{@{}
  >{\raggedright\arraybackslash}p{(\linewidth - 4\tabcolsep) * \real{0.3529}}
  >{\raggedright\arraybackslash}p{(\linewidth - 4\tabcolsep) * \real{0.3824}}
  >{\raggedright\arraybackslash}p{(\linewidth - 4\tabcolsep) * \real{0.2647}}@{}}
\toprule\noalign{}
\begin{minipage}[b]{\linewidth}\raggedright
ગુનો પ્રકાર
\end{minipage} & \begin{minipage}[b]{\linewidth}\raggedright
વર્ણન
\end{minipage} & \begin{minipage}[b]{\linewidth}\raggedright
ઉદાહરણ
\end{minipage} \\
\midrule\noalign{}
\endhead
\bottomrule\noalign{}
\endlastfoot
\textbf{Credit Card Fraud} & પેમેન્ટ કાર્ડનો અનધિકૃત ઉપયોગ & ચોરાયેલા કાર્ડથી
ઓનલાઇન ખરીદારી \\
\textbf{Software Piracy} & સોફ્ટવેરની ગેરકાયદેસર કોપીઇંગ/વિતરણ & કોપીરાઇટ
સોફ્ટવેર ડાઉનલોડ કરવું \\
\textbf{Copyright Infringement} & બૌદ્ધિક સંપત્તિ અધિકારોનું ઉલ્લંઘન &
ફિલ્મો/સંગીતની ગેરકાયદેસર શેરિંગ \\
\textbf{Trademark Violations} & રજિસ્ટર્ડ ટ્રેડમાર્કનો દુરુપયોગ & બનાવટી બ્રાન્ડ
વેબસાઇટ્સ બનાવવી \\
\end{longtable}
}

\textbf{અસર મૂલ્યાંકન:}

\begin{itemize}
\tightlist
\item
  \textbf{નાણાકીય નુકસાન}: સીધો નાણાકીય નુકસાન
\item
  \textbf{બૌદ્ધિક સંપત્તિ ચોરી}: સ્પર્ધાત્મક લાભનું નુકસાન
\item
  \textbf{બ્રાન્ડ પ્રતિષ્ઠા}: કંપનીની છબીને નુકસાન
\item
  \textbf{કાયદેસરી ખર્ચ}: કાર્યવાહી/સંરક્ષણનો ખર્ચ
\end{itemize}

\textbf{અટકાવવાના પગલાં:}

\begin{itemize}
\tightlist
\item
  \textbf{Digital Rights Management}: કોપીરાઇટ સામગ્રીની સુરક્ષા
\item
  \textbf{સુરક્ષિત પેમેન્ટ સિસ્ટમ}: છેતરપિંડી શોધ લાગુ કરવી
\item
  \textbf{કાયદેસરી અમલીકરણ}: ઉલ્લંઘન કરનારાઓ સામે કાર્યવાહી
\item
  \textbf{જનજાગૃતિ}: કાયદેસર સોફ્ટવેર વિશે શિક્ષિત કરવું
\end{itemize}

\end{solutionbox}
\begin{mnemonicbox}
``CSCT-FILP'' - Credit/Software/Copyright/Trademark,
Financial/Intellectual/Legal/Public

\end{mnemonicbox}
\begin{center}\rule{0.5\linewidth}{0.5pt}\end{center}

\subsection*{પ્રશ્ન 3(અ) OR [3
ગુણ]}\label{uxaaauxab0uxab6uxaa8-3uxa85-or-3-uxa97uxaa3}

\textbf{ડેટા ડિડલિંગ સમજાવો.}

\begin{solutionbox}

\textbf{ડેટા ડિડલિંગ વ્યાખ્યા:} કમ્પ્યુટર સિસ્ટમમાં ઇનપુટ પહેલાં/દરમિયાન ડેટાની
અનધિકૃત ફેરબદલી.

\textbf{લાક્ષણિકતાઓ કોષ્ટક:}

{\def\LTcaptype{none} % do not increment counter
\begin{longtable}[]{@{}ll@{}}
\toprule\noalign{}
પાસું & વિગતો \\
\midrule\noalign{}
\endhead
\bottomrule\noalign{}
\endlastfoot
\textbf{પદ્ધતિ} & ડેટા વેલ્યુઝમાં સહેજ ફેરફાર \\
\textbf{શોધ} & શોધવું ખૂબ મુશ્કેલ \\
\textbf{લક્ષ્ય} & નાણાકીય/સંવેદનશીલ ડેટા \\
\textbf{અસર} & સંચિત નોંધપાત્ર નુકસાન \\
\end{longtable}
}

\end{solutionbox}
\begin{mnemonicbox}
``DIDDL'' - Data alteration, Input manipulation,
Difficult detection, Dollar losses

\end{mnemonicbox}
\begin{center}\rule{0.5\linewidth}{0.5pt}\end{center}

\subsection*{પ્રશ્ન 3(બ) OR [4
ગુણ]}\label{uxaaauxab0uxab6uxaa8-3uxaac-or-4-uxa97uxaa3}

\textbf{સાયબર સ્પાયિંગ અને સાયબર ટેરરિઝમ સમજાવો.}

\begin{solutionbox}

\textbf{તુલના કોષ્ટક:}

{\def\LTcaptype{none} % do not increment counter
\begin{longtable}[]{@{}lll@{}}
\toprule\noalign{}
પાસું & સાયબર સ્પાયિંગ & સાયબર ટેરરિઝમ \\
\midrule\noalign{}
\endhead
\bottomrule\noalign{}
\endlastfoot
\textbf{હેતુ} & ગુપ્ત માહિતી એકત્રિત કરવી & ભય/અવ્યવસ્થા ફેલાવવી \\
\textbf{લક્ષ્યો} & સરકાર, કોર્પોરેશન્સ & મહત્વપૂર્ણ ઇન્ફ્રાસ્ટ્રક્ચર \\
\textbf{પદ્ધતિઓ} & ગુપ્તતા, લાંબા ગાળાની ઘૂસણખોરી & વિનાશક હુમલાઓ \\
\textbf{અસર} & માહિતીની ચોરી & ભૌતિક/આર્થિક નુકસાન \\
\end{longtable}
}

\textbf{મુખ્ય લાક્ષણિકતાઓ:}

\begin{itemize}
\tightlist
\item
  \textbf{સાયબર સ્પાયિંગ}: રાજ્ય-પ્રાયોજિત, કોર્પોરેટ જાસૂસી
\item
  \textbf{સાયબર ટેરરિઝમ}: વિચારધારાથી પ્રેરિત, વ્યાપક વિક્ષેપ
\item
  \textbf{સામાન્ય ટૂલ્સ}: મેલવેર, સામાજિક એન્જિનિયરિંગ, ઝીરો-ડે એક્સપ્લોઇટ્સ
\end{itemize}

\end{solutionbox}
\begin{mnemonicbox}
``SPY-TER IGSD'' -
Intelligence/Government/Stealth/Disruption, Terror/Economic/Rapid/Damage

\end{mnemonicbox}
\begin{center}\rule{0.5\linewidth}{0.5pt}\end{center}

\subsection*{પ્રશ્ન 3(ક) OR [7
ગુણ]}\label{uxaaauxab0uxab6uxaa8-3uxa95-or-7-uxa97uxaa3}

\textbf{સાયબર કાયદાના કલમ 65 અને કલમ 66 સમજાવો.}

\begin{solutionbox}

\textbf{IT એક્ટ 2008 કલમો:}

{\def\LTcaptype{none} % do not increment counter
\begin{longtable}[]{@{}
  >{\raggedright\arraybackslash}p{(\linewidth - 4\tabcolsep) * \real{0.3000}}
  >{\raggedright\arraybackslash}p{(\linewidth - 4\tabcolsep) * \real{0.3000}}
  >{\raggedright\arraybackslash}p{(\linewidth - 4\tabcolsep) * \real{0.4000}}@{}}
\toprule\noalign{}
\begin{minipage}[b]{\linewidth}\raggedright
કલમ
\end{minipage} & \begin{minipage}[b]{\linewidth}\raggedright
ગુનો
\end{minipage} & \begin{minipage}[b]{\linewidth}\raggedright
સજા
\end{minipage} \\
\midrule\noalign{}
\endhead
\bottomrule\noalign{}
\endlastfoot
\textbf{કલમ 65} & કમ્પ્યુટર સોર્સ કોડ સાથે છેડછાડ & 3 વર્ષ સુધીની જેલ અથવા ₹2 લાખ
સુધીનો દંડ \\
\textbf{કલમ 66} & કમ્પ્યુટર સંબંધિત ગુનાઓ & 3 વર્ષ સુધીની જેલ અથવા ₹5 લાખ સુધીનો
દંડ \\
\end{longtable}
}

\textbf{કલમ 65 વિગતો:}

\begin{itemize}
\tightlist
\item
  \textbf{અવકાશ}: જાણીજોઈને કમ્પ્યુટર સોર્સ કોડ છુપાવવો, નાશ કરવો, બદલવો
\item
  \textbf{આશય}: જ્યારે કમ્પ્યુટર સોર્સ કોડ કાયદા દ્વારા રાખવો/જાળવવો જરૂરી હોય
\item
  \textbf{લાગુ}: આવશ્યક સોફ્ટવેર સિસ્ટમ્સની અખંડતાનું રક્ષણ કરે છે
\end{itemize}

\textbf{કલમ 66 વિગતો:}

\begin{itemize}
\tightlist
\item
  \textbf{કમ્પ્યુટર હેકિંગ}: કમ્પ્યુટર સિસ્ટમ્સમાં અનધિકૃત પ્રવેશ
\item
  \textbf{ડેટા ચોરી}: બેઇમાનીથી ડેટા ડાઉનલોડ, કોપી, એક્સટ્રેક્ટ કરવું
\item
  \textbf{સિસ્ટમ નુકસાન}: માહિતી નાશ, ડિલીટ, બદલવી
\item
  \textbf{સેવા વિક્ષેપ}: અધિકૃત વ્યક્તિઓને પ્રવેશ ન આપવો
\end{itemize}

\end{solutionbox}
\begin{mnemonicbox}
``65-66 CDHD'' - Code tampering, Damage, Hacking,
Data theft

\end{mnemonicbox}
\begin{center}\rule{0.5\linewidth}{0.5pt}\end{center}

\subsection*{પ્રશ્ન 4(અ) [3
ગુણ]}\label{uxaaauxab0uxab6uxaa8-4uxa85-3-uxa97uxaa3}

\textbf{હેકિંગ શું છે? હેકર્સના પ્રકારોની યાદી બનાવો.}

\begin{solutionbox}

\textbf{હેકિંગ વ્યાખ્યા:} નબળાઈઓનો ફાયદો ઉઠાવવા માટે કમ્પ્યુટર સિસ્ટમ્સ/નેટવર્ક્સમાં
અનધિકૃત પ્રવેશ.

\textbf{હેકર પ્રકારો કોષ્ટક:}

{\def\LTcaptype{none} % do not increment counter
\begin{longtable}[]{@{}lll@{}}
\toprule\noalign{}
પ્રકાર & પ્રેરણા & પ્રવૃત્તિ \\
\midrule\noalign{}
\endhead
\bottomrule\noalign{}
\endlastfoot
\textbf{White Hat} & સુરક્ષા સુધારણા & નૈતિક પેનિટ્રેશન ટેસ્ટિંગ \\
\textbf{Black Hat} & દુષ્ટ ઇરાદો & ગુનાહિત પ્રવૃત્તિઓ \\
\textbf{Grey Hat} & મિશ્ર હેતુઓ & અનધિકૃત પરંતુ બિન-દુષ્ટ \\
\textbf{Script Kiddie} & માન્યતા & હાલના ટૂલ્સનો ઉપયોગ \\
\end{longtable}
}

\end{solutionbox}
\begin{mnemonicbox}
``WBGS Hat'' - White, Black, Grey, Script kiddie

\end{mnemonicbox}
\begin{center}\rule{0.5\linewidth}{0.5pt}\end{center}

\subsection*{પ્રશ્ન 4(બ) [4
ગુણ]}\label{uxaaauxab0uxab6uxaa8-4uxaac-4-uxa97uxaa3}

\textbf{હેકિંગની વલ્નેરેબિલિટી અને 0-દિવસની પરિભાષા સમજાવો.}

\begin{solutionbox}

\textbf{પરિભાષા કોષ્ટક:}

{\def\LTcaptype{none} % do not increment counter
\begin{longtable}[]{@{}lll@{}}
\toprule\noalign{}
શબ્દ & વ્યાખ્યા & જોખમ સ્તર \\
\midrule\noalign{}
\endhead
\bottomrule\noalign{}
\endlastfoot
\textbf{Vulnerability} & શોષણ કરી શકાય તેવી સુરક્ષા નબળાઈ & મધ્યમ-ઉચ્ચ \\
\textbf{0-Day Vulnerability} & અજ્ઞાત સુરક્ષા ખામી & ગંભીર \\
\textbf{0-Day Exploit} & 0-day vulnerability માટે હુમલો કોડ & ગંભીર \\
\textbf{0-Day Attack} & 0-day નો સક્રિય શોષણ & ગંભીર \\
\end{longtable}
}

\textbf{મુખ્ય લાક્ષણિકતાઓ:}

\begin{itemize}
\tightlist
\item
  \textbf{વિક્રેતાઓને અજ્ઞાત}: કોઈ પેચ ઉપલબ્ધ નથી
\item
  \textbf{ઉચ્ચ મૂલ્ય}: ડાર્ક માર્કેટમાં વેચાય છે
\item
  \textbf{છુપી}: શોધવું મુશ્કેલ
\item
  \textbf{સમય-નિર્ણાયક}: જાહેર થયા પછી મૂલ્ય ઘટે છે
\end{itemize}

\end{solutionbox}
\begin{mnemonicbox}
``0-Day UHST'' - Unknown, High-value, Stealthy,
Time-critical

\end{mnemonicbox}
\begin{center}\rule{0.5\linewidth}{0.5pt}\end{center}

\subsection*{પ્રશ્ન 4(ક) [7
ગુણ]}\label{uxaaauxab0uxab6uxaa8-4uxa95-7-uxa97uxaa3}

\textbf{હેકિંગના પાંચ સ્ટેપ્સ સમજાવો.}

\begin{solutionbox}

\textbf{હેકિંગ પ્રક્રિયા ફ્લો:}

\begin{verbatim}
flowchart LR
    A[Information Gathering] {-{-} B[Scanning]}
    B {-{-} C[Gaining Access]}
    C {-{-} D[Maintaining Access]}
    D {-{-} E[Covering Tracks]}
\end{verbatim}

\textbf{પાંચ સ્ટેપ્સ વિગતવાર:}

{\def\LTcaptype{none} % do not increment counter
\begin{longtable}[]{@{}
  >{\raggedright\arraybackslash}p{(\linewidth - 4\tabcolsep) * \real{0.1818}}
  >{\raggedright\arraybackslash}p{(\linewidth - 4\tabcolsep) * \real{0.2727}}
  >{\raggedright\arraybackslash}p{(\linewidth - 4\tabcolsep) * \real{0.5455}}@{}}
\toprule\noalign{}
\begin{minipage}[b]{\linewidth}\raggedright
સ્ટેપ
\end{minipage} & \begin{minipage}[b]{\linewidth}\raggedright
હેતુ
\end{minipage} & \begin{minipage}[b]{\linewidth}\raggedright
ટૂલ્સ/તકનીકો
\end{minipage} \\
\midrule\noalign{}
\endhead
\bottomrule\noalign{}
\endlastfoot
\textbf{1. માહિતી એકત્રીકરણ} & લક્ષ્ય માહિતી એકત્રિત કરવી & OSINT, સામાજિક
એન્જિનિયરિંગ \\
\textbf{2. સ્કેનિંગ} & જીવંત સિસ્ટમ્સ, પોર્ટ્સ ઓળખવા & Nmap, પોર્ટ સ્કેનર્સ \\
\textbf{3. પ્રવેશ મેળવવો} & નબળાઈઓનો શોષણ કરવો & Metasploit, કસ્ટમ
એક્સપ્લોઇટ્સ \\
\textbf{4. પ્રવેશ જાળવવો} & સતત હાજરી સ્થાપિત કરવી & બેકડોર્સ, રૂટકિટ્સ \\
\textbf{5. નિશાનો છુપાવવા} & પુરાવાઓ દૂર કરવા & લોગ ડિલીશન, ફાઇલ સફાઈ \\
\end{longtable}
}

\textbf{દરેક સ્ટેપની વિગતો:}

\begin{itemize}
\tightlist
\item
  \textbf{માહિતી એકત્રીકરણ}: નિષ્ક્રિય/સક્રિય જાસૂસી
\item
  \textbf{સ્કેનિંગ}: નેટવર્ક મેપિંગ, વલ્નેરેબિલિટી મૂલ્યાંકન
\item
  \textbf{પ્રવેશ મેળવવો}: પાસવર્ડ હુમલાઓ, બફર ઓવરફ્લો
\item
  \textbf{પ્રવેશ જાળવવો}: વિશેષાધિકાર વૃદ્ધિ, બેકડોર ઇન્સ્ટોલેશન
\item
  \textbf{નિશાનો છુપાવવા}: એન્ટિ-ફોરેન્સિક્સ તકનીકો
\end{itemize}

\end{solutionbox}
\begin{mnemonicbox}
``ISGMC'' - Information, Scanning, Gaining,
Maintaining, Covering

\end{mnemonicbox}
\begin{center}\rule{0.5\linewidth}{0.5pt}\end{center}

\subsection*{પ્રશ્ન 4(અ) OR [3
ગુણ]}\label{uxaaauxab0uxab6uxaa8-4uxa85-or-3-uxa97uxaa3}

\textbf{કાલી લિનક્સના કોઈપણ ત્રણ બેઝિક કમાન્ડ યોગ્ય ઉદાહરણ સાથે સમજાવો.}

\begin{solutionbox}

\textbf{કાલી લિનક્સ કમાન્ડ્સ કોષ્ટક:}

{\def\LTcaptype{none} % do not increment counter
\begin{longtable}[]{@{}lll@{}}
\toprule\noalign{}
કમાન્ડ & હેતુ & ઉદાહરણ \\
\midrule\noalign{}
\endhead
\bottomrule\noalign{}
\endlastfoot
\textbf{nmap} & નેટવર્ક સ્કેનિંગ & \texttt{nmap\ -sS\ 192.168.1.1} \\
\textbf{netcat} & નેટવર્ક યુટિલિટી & \texttt{nc\ -l\ -p\ 4444} \\
\textbf{john} & પાસવર્ડ ક્રેકિંગ &
\texttt{john\ -\/-wordlist=passwords.txt\ hashes.txt} \\
\end{longtable}
}

\textbf{કમાન્ડ વિગતો:}

\begin{itemize}
\tightlist
\item
  \textbf{nmap}: લક્ષ્ય IP પર સ્ટેલ્થ SYN સ્કેન
\item
  \textbf{netcat}: કનેક્શન માટે પોર્ટ 4444 પર સાંભળો
\item
  \textbf{john}: પાસવર્ડ હેશ પર ડિક્શનરી એટેક
\end{itemize}

\end{solutionbox}
\begin{mnemonicbox}
``NNJ'' - Nmap સ્કેન કરે, Netcat સાંભળે, John ક્રેક કરે

\end{mnemonicbox}
\begin{center}\rule{0.5\linewidth}{0.5pt}\end{center}

\subsection*{પ્રશ્ન 4(બ) OR [4
ગુણ]}\label{uxaaauxab0uxab6uxaa8-4uxaac-or-4-uxa97uxaa3}

\textbf{સેશન હાઇજેકિંગનું વિગતવાર વર્ણન કરો.}

\begin{solutionbox}

\textbf{સેશન હાઇજેકિંગ પ્રક્રિયા:}

\begin{verbatim}
sequenceDiagram
    participant U as User
    participant A as Attacker
    participant S as Server
    U{-S: 1. Login \& Get Session ID}
    A{-A: 2. Capture Session ID}
    A{-S: 3. Use Stolen Session ID}
    S{-A: 4. Grant Access}
\end{verbatim}

\textbf{પ્રકારો અને પદ્ધતિઓ:}

\begin{itemize}
\tightlist
\item
  \textbf{Active Hijacking}: હુમલાખોર સક્રિયપણે ભાગ લે છે
\item
  \textbf{Passive Hijacking}: સેશન્સનું મોનિટર અને કેપ્ચર કરે છે
\item
  \textbf{Network Level}: IP spoofing, ARP poisoning
\item
  \textbf{Application Level}: Session ID અનુમાન, XSS
\end{itemize}

\textbf{અટકાવવાના પગલાં:}

\begin{itemize}
\tightlist
\item
  \textbf{HTTPS}: સેશન ડેટા એન્ક્રિપ્ટ કરવો
\item
  \textbf{સેશન ટાઇમઆઉટ્સ}: સેશનની અવધિ મર્યાદિત કરવી
\item
  \textbf{IP બાઇન્ડિંગ}: સેશન્સને IP એડ્રેસ સાથે બાંધવા
\item
  \textbf{મજબૂત સેશન IDs}: અણધારી ટોકન્સનો ઉપયોગ
\end{itemize}

\end{solutionbox}
\begin{mnemonicbox}
``APNA-HSIS'' - Active/Passive/Network/Application,
HTTPS/Strong/IP/Session

\end{mnemonicbox}
\begin{center}\rule{0.5\linewidth}{0.5pt}\end{center}

\subsection*{પ્રશ્ન 4(ક) OR [7
ગુણ]}\label{uxaaauxab0uxab6uxaa8-4uxa95-or-7-uxa97uxaa3}

\textbf{રિમોટ એડમિનિસ્ટ્રેશન ટૂલ્સ સમજાવો.}

\begin{solutionbox}

\textbf{RAT વ્યાખ્યા:} કમ્પ્યુટર સિસ્ટમ્સના દૂરસ્થ નિયંત્રણની મંજૂરી આપતું સોફ્ટવેર, મોટે
ભાગે દુષ્ટતાથી વપરાય છે.

\textbf{RAT કાર્યક્ષમતા કોષ્ટક:}

{\def\LTcaptype{none} % do not increment counter
\begin{longtable}[]{@{}lll@{}}
\toprule\noalign{}
કાર્ય & વર્ણન & જોખમ સ્તર \\
\midrule\noalign{}
\endhead
\bottomrule\noalign{}
\endlastfoot
\textbf{સ્ક્રીન કેપ્ચર} & દૂરસ્થ સ્ક્રીનશોટ લેવા & મધ્યમ \\
\textbf{કીલોગિંગ} & કીસ્ટ્રોક રેકોર્ડ કરવા & ઉચ્ચ \\
\textbf{ફાઇલ ટ્રાન્સફર} & ફાઇલ અપલોડ/ડાઉનલોડ & ઉચ્ચ \\
\textbf{કેમેરા એક્સેસ} & વેબકેમ/માઇક્રોફોન સક્રિય કરવા & ગંભીર \\
\end{longtable}
}

\textbf{કાયદેસર વિ. દુષ્ટ ઉપયોગ:}

{\def\LTcaptype{none} % do not increment counter
\begin{longtable}[]{@{}lll@{}}
\toprule\noalign{}
પાસું & કાયદેસર & દુષ્ટ \\
\midrule\noalign{}
\endhead
\bottomrule\noalign{}
\endlastfoot
\textbf{હેતુ} & IT સપોર્ટ, એડમિનિસ્ટ્રેશન & જાસૂસી, ચોરી \\
\textbf{સંમતિ} & વપરાશકર્તા જાગરૂક અને સંમત & જ્ઞાન વિના ઇન્સ્ટોલ \\
\textbf{પ્રવેશ} & માત્ર અધિકૃત કર્મચારીઓ & અનધિકૃત હુમલાખોરો \\
\end{longtable}
}

\textbf{શોધ અને અટકાવવું:}

\begin{itemize}
\tightlist
\item
  \textbf{એન્ટિવાયરસ}: જાણીતા RAT સિગ્નેચર શોધવા
\item
  \textbf{નેટવર્ક મોનિટરિંગ}: અસામાન્ય આઉટબાઉન્ડ કનેક્શન્સ
\item
  \textbf{વપરાશકર્તા શિક્ષણ}: શંકાસ્પદ ડાઉનલોડ્સ ટાળવા
\item
  \textbf{ફાયરવોલ નિયમો}: અનધિકૃત કનેક્શન્સ બ્લોક કરવા
\end{itemize}

\textbf{સામાન્ય RATs:}

\begin{itemize}
\tightlist
\item
  \textbf{TeamViewer}: કાયદેસર દૂરસ્થ પ્રવેશ
\item
  \textbf{DarkComet}: દુષ્ટ RAT
\item
  \textbf{Poison Ivy}: અદ્યતન સતત ધમકી ટૂલ
\end{itemize}

\end{solutionbox}
\begin{mnemonicbox}
``RAT SKFC-ANUM'' - Screen/Key/File/Camera,
Antivirus/Network/User/Monitoring

\end{mnemonicbox}
\begin{center}\rule{0.5\linewidth}{0.5pt}\end{center}

\subsection*{પ્રશ્ન 5(અ) [3
ગુણ]}\label{uxaaauxab0uxab6uxaa8-5uxa85-3-uxa97uxaa3}

\textbf{મોબાઇલ ફોરેન્સિક્સ સમજાવો.}

\begin{solutionbox}

\textbf{મોબાઇલ ફોરેન્સિક્સ વ્યાખ્યા:} વૈજ્ઞાનિક રીતે સ્વીકૃત પદ્ધતિઓનો ઉપયોગ કરીને
મોબાઇલ ઉપકરણોમાંથી ડિજિટલ પુરાવા પુનઃપ્રાપ્ત કરવાની પ્રક્રિયા.

\textbf{મુખ્ય પાસાઓ કોષ્ટક:}

{\def\LTcaptype{none} % do not increment counter
\begin{longtable}[]{@{}ll@{}}
\toprule\noalign{}
પાસું & વર્ણન \\
\midrule\noalign{}
\endhead
\bottomrule\noalign{}
\endlastfoot
\textbf{ડેટા પ્રકારો} & કોલ લોગ્સ, SMS, ફોટો, એપ ડેટા \\
\textbf{પડકારો} & એન્ક્રિપ્શન, એન્ટિ-ફોરેન્સિક્સ, OS ની વિવિધતા \\
\textbf{ટૂલ્સ} & Cellebrite, XRY, Oxygen Suite \\
\textbf{કાયદેસર} & કસ્ટડી ચેન, કોર્ટ સ્વીકાર્યતા \\
\end{longtable}
}

\end{solutionbox}
\begin{mnemonicbox}
``DCTL'' - Data types, Challenges, Tools, Legal
requirements

\end{mnemonicbox}
\begin{center}\rule{0.5\linewidth}{0.5pt}\end{center}

\subsection*{પ્રશ્ન 5(બ) [4
ગુણ]}\label{uxaaauxab0uxab6uxaa8-5uxaac-4-uxa97uxaa3}

\textbf{ડિજિટલ ફોરેન્સિક્સ શું છે? ડિજિટલ ફોરેન્સિક્સના ફાયદાઓ લખો.}

\begin{solutionbox}

\textbf{ડિજિટલ ફોરેન્સિક્સ વ્યાખ્યા:} કાયદેસરી કાર્યવાહી માટે પુરાવાઓ પુનઃપ્રાપ્ત અને
વિશ્લેષણ કરવા માટે ડિજિટલ ઉપકરણોની વૈજ્ઞાનિક તપાસ.

\textbf{ફાયદાઓ કોષ્ટક:}

{\def\LTcaptype{none} % do not increment counter
\begin{longtable}[]{@{}ll@{}}
\toprule\noalign{}
ફાયદો & વર્ણન \\
\midrule\noalign{}
\endhead
\bottomrule\noalign{}
\endlastfoot
\textbf{પુરાવા પુનઃપ્રાપ્તિ} & ડિલીટ/છુપાયેલ ડેટા પુનઃપ્રાપ્ત કરવો \\
\textbf{ગુના ઉકેલ} & કેસો માટે મહત્વપૂર્ણ પુરાવા પૂરા પાડવા \\
\textbf{ખર્ચ અસરકારક} & પરંપરાગત તપાસ કરતાં સસ્તું \\
\textbf{સચોટ પરિણામો} & વૈજ્ઞાનિક પદ્ધતિઓ વિશ્વસનીયતા સુનિશ્ચિત કરે છે \\
\end{longtable}
}

\textbf{વધારાના ફાયદાઓ:}

\begin{itemize}
\tightlist
\item
  \textbf{સમય કાર્યક્ષમ}: મેન્યુઅલ તપાસ કરતાં ઝડપી
\item
  \textbf{બિન-વિનાશક}: મૂળ પુરાવાઓ સાચવે છે
\item
  \textbf{વ્યાપક}: બહુવિધ ડેટા સ્ત્રોતોનું વિશ્લેષણ કરે છે
\item
  \textbf{કોર્ટ સ્વીકાર્ય}: કાયદેસર રીતે સ્વીકાર્ય પુરાવા
\end{itemize}

\end{solutionbox}
\begin{mnemonicbox}
``ECCA-TNCA'' - Evidence/Crime/Cost/Accurate,
Time/Non-destructive/Comprehensive/Admissible

\end{mnemonicbox}
\begin{center}\rule{0.5\linewidth}{0.5pt}\end{center}

\subsection*{પ્રશ્ન 5(ક) [7
ગુણ]}\label{uxaaauxab0uxab6uxaa8-5uxa95-7-uxa97uxaa3}

\textbf{ડિજિટલ ફોરેન્સિક્સ માં લોકાર્ડના પ્રિન્સિપલ ઓફ એક્સચેન્જને વિગતવાર વર્ણન
કરો.}

\begin{solutionbox}

\textbf{લોકાર્ડનો સિદ્ધાંત:} ``દરેક સંપર્ક નિશાન છોડે છે'' - વસ્તુઓ વચ્ચેની કોઈપણ
ક્રિયા સામગ્રીના વિનિમયમાં પરિણમે છે.

\textbf{ડિજિટલ એપ્લિકેશન:}

\begin{center}
\textbf{Mermaid Diagram (Code)}
\begin{verbatim}
{Shaded}
{Highlighting}[]
graph TD
    A[User Action] {-{-}{} B[Digital Traces]}
    B {-{-}{} C[Log Files]}
    B {-{-}{} D[Registry Entries]}
    B {-{-}{} E[File Metadata]}
    B {-{-}{} F[Network Traffic]}
{Highlighting}
{Shaded}
\end{verbatim}
\end{center}

\textbf{ડિજિટલ નિશાનો કોષ્ટક:}

{\def\LTcaptype{none} % do not increment counter
\begin{longtable}[]{@{}lll@{}}
\toprule\noalign{}
ક્રિયા & ડિજિટલ નિશાન & સ્થાન \\
\midrule\noalign{}
\endhead
\bottomrule\noalign{}
\endlastfoot
\textbf{ફાઇલ એક્સેસ} & એક્સેસ ટાઇમસ્ટેમ્પ્સ & ફાઇલ સિસ્ટમ મેટાડેટા \\
\textbf{વેબ બ્રાઉઝિંગ} & બ્રાઉઝર હિસ્ટરી & બ્રાઉઝર ડેટાબેસ \\
\textbf{ઇમેઇલ મોકલવો} & ઇમેઇલ હેડર્સ & મેઇલ સર્વર લોગ્સ \\
\textbf{USB કનેક્શન} & ઉપકરણ રજિસ્ટ્રી & Windows રજિસ્ટ્રી \\
\end{longtable}
}

\textbf{ફોરેન્સિક અસરો:}

\begin{itemize}
\tightlist
\item
  \textbf{સ્થાયિત્વ}: ડિજિટલ નિશાનો મોટે ભાગે વધુ લાંબા સમય ટકે છે
\item
  \textbf{સચોટતા}: ચોક્કસ ટાઇમસ્ટેમ્પ્સ અને ડેટા
\item
  \textbf{માત્રા}: મોટી માત્રામાં ટ્રેસ પુરાવા
\item
  \textbf{પુનઃપ્રાપ્તિ}: ડિલીટ થયેલ ડેટા પુનઃપ્રાપ્ત કરી શકાય છે
\end{itemize}

\textbf{પુરાવા પ્રકારો:}

\begin{itemize}
\tightlist
\item
  \textbf{કાલાનુક્રમિક}: ક્રિયાઓ ક્યારે થઈ
\item
  \textbf{અવકાશીય}: ક્રિયાઓ ક્યાં થઈ
\item
  \textbf{સંબંધીય}: એન્ટિટી વચ્ચેના જોડાણો
\item
  \textbf{વર્તણૂકીય}: વપરાશકર્તા પ્રવૃત્તિના પેટર્ન
\end{itemize}

\textbf{એપ્લિકેશન્સ:}

\begin{itemize}
\tightlist
\item
  \textbf{ગુનાહિત કેસો}: હાજરી/ક્રિયાઓ સાબિત કરવી
\item
  \textbf{સિવિલ મુકદ્દમાઓ}: વ્યવસાયિક વિવાદો
\item
  \textbf{આંતરિક તપાસ}: કર્મચારીઓની ગેરવર્તણૂક
\item
  \textbf{ઘટના પ્રતિભાવ}: સુરક્ષા ભંગ વિશ્લેષણ
\end{itemize}

\end{solutionbox}
\begin{mnemonicbox}
``LOCARD PVAR-TREB'' -
Persistence/Volume/Accuracy/Recovery,
Temporal/Relational/Evidence/Behavioral

\end{mnemonicbox}
\begin{center}\rule{0.5\linewidth}{0.5pt}\end{center}

\subsection*{પ્રશ્ન 5(અ) OR [3
ગુણ]}\label{uxaaauxab0uxab6uxaa8-5uxa85-or-3-uxa97uxaa3}

\textbf{નેટવર્ક ફોરેન્સિક્સ સમજાવો.}

\begin{solutionbox}

\textbf{નેટવર્ક ફોરેન્સિક્સ વ્યાખ્યા:} માહિતી અને પુરાવા એકત્રિત કરવા માટે નેટવર્ક
ટ્રાફિકનું મોનિટરિંગ અને વિશ્લેષણ.

\textbf{મુખ્ય ઘટકો કોષ્ટક:}

{\def\LTcaptype{none} % do not increment counter
\begin{longtable}[]{@{}ll@{}}
\toprule\noalign{}
ઘટક & કાર્ય \\
\midrule\noalign{}
\endhead
\bottomrule\noalign{}
\endlastfoot
\textbf{પેકેટ કેપ્ચર} & નેટવર્ક ટ્રાફિક રેકોર્ડ કરવો \\
\textbf{ટ્રાફિક વિશ્લેષણ} & કમ્યુનિકેશન પેટર્નનું પરીક્ષણ \\
\textbf{પ્રોટોકોલ વિશ્લેષણ} & નેટવર્ક પ્રોટોકોલ્સ ડીકોડ કરવા \\
\textbf{ટાઇમલાઇન બનાવવી} & ઘટનાઓનો ક્રમ સ્થાપિત કરવો \\
\end{longtable}
}

\end{solutionbox}
\begin{mnemonicbox}
``PTTP'' - Packet capture, Traffic analysis,
Timeline, Protocol analysis

\end{mnemonicbox}
\begin{center}\rule{0.5\linewidth}{0.5pt}\end{center}

\subsection*{પ્રશ્ન 5(બ) OR [4
ગુણ]}\label{uxaaauxab0uxab6uxaa8-5uxaac-or-4-uxa97uxaa3}

\textbf{ડિજિટલ ફોરેન્સિક તપાસમાં પુરાવા તરીકે CCTV શા માટે મહત્વની ભૂમિકા ભજવે છે
તે સમજાવો.}

\begin{solutionbox}

\textbf{CCTV પુરાવાનું મૂલ્ય:}

{\def\LTcaptype{none} % do not increment counter
\begin{longtable}[]{@{}ll@{}}
\toprule\noalign{}
પાસું & મહત્વ \\
\midrule\noalign{}
\endhead
\bottomrule\noalign{}
\endlastfoot
\textbf{દ્રશ્ય પુરાવો} & ઘટનાઓના સીધા પુરાવા \\
\textbf{ટાઇમસ્ટેમ્પ} & ચોક્કસ સમય સહસંબંધ \\
\textbf{સ્થાન ચકાસણી} & ઘટના સ્થળે હાજરી સાબિત કરે છે \\
\textbf{વર્તણૂક વિશ્લેષણ} & ક્રિયાઓ અને ઇરાદો દર્શાવે છે \\
\end{longtable}
}

\textbf{ડિજિટલ ફોરેન્સિક્સ એકીકરણ:}

\begin{itemize}
\tightlist
\item
  \textbf{મેટાડેટા નિષ્કર્ષણ}: કેમેરા સેટિંગ્સ, ટાઇમસ્ટેમ્પ્સ
\item
  \textbf{વીડિયો સુધારણા}: છબીની ગુણવત્તા સુધારવી
\item
  \textbf{ફોર્મેટ વિશ્લેષણ}: કમ્પ્રેશન આર્ટિફેક્ટ્સ સમજવા
\item
  \textbf{પ્રમાણીકરણ}: વીડિયોની અખંડતા ચકાસવી
\end{itemize}

\textbf{કાયદેસરી વિચારણાઓ:}

\begin{itemize}
\tightlist
\item
  \textbf{કસ્ટડી ચેન}: પુરાવાની અખંડતા જાળવવી
\item
  \textbf{કોર્ટ સ્વીકાર્યતા}: કાયદેસર પ્રક્રિયાઓ અનુસરવી
\item
  \textbf{ગોપનીયતા અધિકારો}: સર્વેલન્સ કાયદાઓનું સન્માન કરવું
\item
  \textbf{તકનીકી માન્યતા}: પ્રામાણિકતા સાબિત કરવી
\end{itemize}

\end{solutionbox}
\begin{mnemonicbox}
``VTLB-MFAC'' - Visual/Timestamp/Location/Behavior,
Metadata/Format/Authentication/Chain

\end{mnemonicbox}
\begin{center}\rule{0.5\linewidth}{0.5pt}\end{center}

\subsection*{પ્રશ્ન 5(ક) OR [7
ગુણ]}\label{uxaaauxab0uxab6uxaa8-5uxa95-or-7-uxa97uxaa3}

\textbf{ડિજિટલ ફોરેન્સિક તપાસના તબક્કાઓ સમજાવો.}

\begin{solutionbox}

\textbf{ડિજિટલ ફોરેન્સિક તપાસના તબક્કાઓ:}

\begin{verbatim}
flowchart LR
    A[Identification] {-{-} B[Preservation]}
    B {-{-} C[Analysis]}
    C {-{-} D[Documentation]}
    D {-{-} E[Presentation]}
\end{verbatim}

\textbf{તબક્કાઓની વિગતો કોષ્ટક:}

{\def\LTcaptype{none} % do not increment counter
\begin{longtable}[]{@{}
  >{\raggedright\arraybackslash}p{(\linewidth - 4\tabcolsep) * \real{0.2059}}
  >{\raggedright\arraybackslash}p{(\linewidth - 4\tabcolsep) * \real{0.3529}}
  >{\raggedright\arraybackslash}p{(\linewidth - 4\tabcolsep) * \real{0.4412}}@{}}
\toprule\noalign{}
\begin{minipage}[b]{\linewidth}\raggedright
તબક્કો
\end{minipage} & \begin{minipage}[b]{\linewidth}\raggedright
પ્રવૃત્તિઓ
\end{minipage} & \begin{minipage}[b]{\linewidth}\raggedright
ટૂલ્સ/પદ્ધતિઓ
\end{minipage} \\
\midrule\noalign{}
\endhead
\bottomrule\noalign{}
\endlastfoot
\textbf{ઓળખ} & સંભવિત પુરાવા સ્ત્રોતો શોધવા & પ્રારંભિક મૂલ્યાંકન, સીન સર્વે \\
\textbf{સંરક્ષણ} & ફેરફાર વિના પુરાવા સુરક્ષિત કરવા & ઇમેજિંગ, હેશ ચકાસણી \\
\textbf{વિશ્લેષણ} & સંબંધિત ડેટા માટે પુરાવાઓનું પરીક્ષણ & ફોરેન્સિક સોફ્ટવેર, મેન્યુઅલ
સમીક્ષા \\
\textbf{દસ્તાવેજીકરણ} & શોધો અને પ્રક્રિયાઓ રેકોર્ડ કરવી & રિપોર્ટ્સ, સ્ક્રીનશોટ્સ,
લોગ્સ \\
\textbf{રજૂઆત} & હિતધારકોને શોધો રજૂ કરવા & કોર્ટ સાક્ષ્ય, નિષ્ણાત રિપોર્ટ્સ \\
\end{longtable}
}

\textbf{વિગતવાર પ્રવૃત્તિઓ:}

\textbf{1. ઓળખ તબક્કો:}

\begin{itemize}
\tightlist
\item
  \textbf{પુરાવા સ્ત્રોતો}: કમ્પ્યુટર્સ, ફોન્સ, સર્વર્સ, નેટવર્ક લોગ્સ
\item
  \textbf{અવકાશ વ્યાખ્યા}: તપાસની સીમાઓ નક્કી કરવી
\item
  \textbf{કાયદેસર અધિકાર}: વોરંટ/પરવાનગીઓ મેળવવી
\item
  \textbf{પ્રારંભિક ફોટોગ્રાફી}: સીનની સ્થિતિ દસ્તાવેજીકરણ
\end{itemize}

\textbf{2. સંરક્ષણ તબક્કો:}

\begin{itemize}
\tightlist
\item
  \textbf{બિટ-બાય-બિટ ઇમેજિંગ}: ચોક્કસ કોપીઓ બનાવવી
\item
  \textbf{હેશ ગણતરી}: અખંડતા ચકાસવી (MD5, SHA)
\item
  \textbf{કસ્ટડી ચેન}: પુરાવા ટ્રેઇલ જાળવવી
\item
  \textbf{રાઇટ પ્રોટેક્શન}: પુરાવા ફેરફાર અટકાવવો
\end{itemize}

\textbf{3. વિશ્લેષણ તબક્કો:}

\begin{itemize}
\tightlist
\item
  \textbf{ડેટા પુનઃપ્રાપ્તિ}: ડિલીટ થયેલી ફાઇલો પુનઃપ્રાપ્ત કરવી
\item
  \textbf{કીવર્ડ શોધ}: સંબંધિત માહિતી શોધવી
\item
  \textbf{ટાઇમલાઇન વિશ્લેષણ}: ઘટનાઓનું પુનર્નિર્માણ કરવું
\item
  \textbf{પેટર્ન ઓળખ}: શંકાસ્પદ પ્રવૃત્તિઓ ઓળખવી
\end{itemize}

\textbf{4. દસ્તાવેજીકરણ તબક્કો:}

\begin{itemize}
\tightlist
\item
  \textbf{પદ્ધતિ રેકોર્ડિંગ}: ઉપયોગ કરેલી પ્રક્રિયાઓ દસ્તાવેજીકરણ
\item
  \textbf{પુરાવા કેટેલોગિંગ}: બધા શોધો સૂચિબદ્ધ કરવા
\item
  \textbf{સ્ક્રીનશોટ કેપ્ચર}: દ્રશ્ય પુરાવા દસ્તાવેજીકરણ
\item
  \textbf{રિપોર્ટ તૈયારી}: વ્યાપક તપાસ રિપોર્ટ
\end{itemize}

\textbf{5. રજૂઆત તબક્કો:}

\begin{itemize}
\tightlist
\item
  \textbf{નિષ્ણાત સાક્ષ્ય}: કોર્ટમાં હાજરી
\item
  \textbf{દ્રશ્ય સહાયતા}: ચાર્ટ્સ, આકૃતિઓ, પ્રદર્શન
\item
  \textbf{તકનીકી અનુવાદ}: જટિલ વિભાવનાઓ સમજાવવી
\item
  \textbf{ક્રોસ-એક્ઝામિનેશન}: બચાવ પક્ષના પ્રશ્નોના જવાબ
\end{itemize}

\textbf{ગુણવત્તા ખાતરી:}

\begin{itemize}
\tightlist
\item
  \textbf{પીઅર રિવ્યુ}: બીજા પરીક્ષકની ચકાસણી
\item
  \textbf{ટૂલ માન્યતા}: સોફ્ટવેરની સચોટતા સુનિશ્ચિત કરવી
\item
  \textbf{પ્રક્રિયા પાલન}: માનક પ્રોટોકોલ્સ અનુસરવા
\item
  \textbf{સતત તાલીમ}: કુશળતા વર્તમાન રાખવી
\end{itemize}

\textbf{કાયદેસરી વિચારણાઓ:}

\begin{itemize}
\tightlist
\item
  \textbf{સ્વીકાર્યતા નિયમો}: કોર્ટના ધોરણો પૂરા કરવા
\item
  \textbf{ગોપનીયતા સુરક્ષા}: વ્યક્તિગત અધિકારોનું સન્માન કરવું
\item
  \textbf{આંતરરાષ્ટ્રીય કાયદો}: ક્રોસ-બોર્ડર તપાસ
\item
  \textbf{વ્યાવસાયિક નીતિશાસ્ત્ર}: નિષ્પક્ષતા જાળવવી
\end{itemize}

\end{solutionbox}
\begin{mnemonicbox}
``IPADP-ESLR-HTVC-MSCR-ETVI'' -
Identification/Preservation/Analysis/Documentation/Presentation વિગતવાર
પેટા-પ્રવૃત્તિઓ સાથે

\end{mnemonicbox}

\end{document}
