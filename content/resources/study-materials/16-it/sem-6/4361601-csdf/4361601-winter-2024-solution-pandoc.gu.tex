\documentclass[10pt,a4paper]{article}

% content/resources/templates/preamble.tex
\usepackage[margin=0.6in]{geometry}
\author{Milav Dabgar}
\usepackage{amsmath,amssymb,amsthm}
\usepackage{booktabs}
\usepackage{multirow}
\usepackage{xcolor}
\usepackage{tcolorbox}
\tcbuselibrary{breakable,skins}
\usepackage[colorlinks=true,linkcolor=blue]{hyperref}
\usepackage{titlesec}
\usepackage{enumitem}
\usepackage{tikz}
\usepackage{pgfplots}
\usepackage{circuitikz}
\usepackage[version=4]{mhchem}
\usepackage{longtable}
\usepackage{array}
\usepackage{float}
\usepackage{caption}
\usepackage{listings}

\lstset{
  basicstyle=\small\ttfamily,
  breaklines=true,
  breakatwhitespace=false,
  postbreak=\mbox{\textcolor{red}{$\hookrightarrow$}\space},
  float=false,
  numbers=left,
  numberstyle=\tiny\color{gray},
  numbersep=10pt,
  xleftmargin=2em,
  keywordstyle=\color{blue},
  commentstyle=\color{green!60!black},
  stringstyle=\color{purple},
  backgroundcolor=\color{gray!5},
  showstringspaces=false,
  tabsize=2,
  captionpos=b,
  keepspaces=true,
  columns=flexible
}

\pgfplotsset{compat=1.18}
\usetikzlibrary{shapes,arrows,positioning,calc,patterns,decorations.pathmorphing,decorations.markings,arrows.meta}

% Color scheme
\definecolor{headcolor}{RGB}{0,102,204}
\definecolor{keycolor}{RGB}{220,20,60}
\definecolor{solutioncolor}{RGB}{34,139,34}
\definecolor{mnemoniccolor}{RGB}{148,0,211}
\definecolor{codecolor}{RGB}{0,0,100}

% Spacing
\setlength{\parskip}{3pt}
\setlist[itemize]{nosep}
\setlist[enumerate]{nosep}

% Title formatting
\titleformat{\section}{\Large\bfseries\color{headcolor}}{\thesection}{1em}{}
\titleformat{\subsection}{\large\bfseries\color{headcolor}}{\thesubsection}{1em}{}

% Pandoc tightlist compatibility
\providecommand{\tightlist}{%
  \setlength{\itemsep}{0pt}\setlength{\parskip}{0pt}}

% Pandoc longtable compatibility
\newcounter{none}
\def\thenone{}


% content/resources/templates/gujarati-boxes.tex
\usepackage{fontspec}
\usepackage{polyglossia}

% Set Gujarati as main language (document is primarily in Gujarati)
% Note: gloss-gujarati.ldf doesn't exist in polyglossia, but it will use hyphenation patterns
\setdefaultlanguage{gujarati}
\setotherlanguage{english}

% Configure Gujarati font properly
% Use Language=Default to prevent polyglossia from trying to add language-specific features
% that don't exist for Gujarati, which causes "empty feature" warnings
\newfontfamily\gujaratifont[Script=Gujarati,AutoFakeBold=2.5,AutoFakeSlant=0.3]{Noto Sans Gujarati}
\setmainfont[Script=Gujarati,AutoFakeBold=2.5,AutoFakeSlant=0.3]{Noto Sans Gujarati}
% Use Noto Sans Gujarati for monospace to support Gujarati in text
\setmonofont[Scale=0.9]{Noto Sans Gujarati}

% Configure English to use the same font
\newfontfamily\englishfont[Script=Gujarati,AutoFakeBold=2.5,AutoFakeSlant=0.3]{Noto Sans Gujarati}

% Translations for polyglossia
\gappto\captionsgujarati{
  \renewcommand{\tablename}{કોષ્ટક}
  \renewcommand{\figurename}{આકૃતિ}
}

% Helper for TikZ nodes to ensure Gujarati font
\newcommand{\gu}[1]{{\gujaratifont #1}}

% Custom environments
\newtcolorbox{solutionbox}{
    breakable,
    enhanced,
    colback=solutioncolor!5!white,
    colframe=solutioncolor!75!black,
    fonttitle=\bfseries,
    title=જવાબ
}

\newtcolorbox{solutionboxnobreak}{
 colback=solutioncolor!5!white,
 colframe=solutioncolor!75!black,
 fonttitle=\bfseries,
 title=જવાબ
}

\newtcolorbox{keyformula}{
 breakable,
 enhanced,
 colback=keycolor!5!white,
 colframe=keycolor!75!black,
 fonttitle=\bfseries,
 title=રાસાયણિક સમીકરણ/સૂત્ર
}

\newtcolorbox{mnemonicbox}{
 breakable,
 enhanced,
 colback=mnemoniccolor!5!white,
 colframe=mnemoniccolor!75!black,
 fonttitle=\bfseries,
 title=મેમરી ટ્રીક
}


\begin{document}

\begin{center}
{\Huge\bfseries\color{headcolor} Subject Name (Gujarati)}\\[5pt]
{\LARGE 4361601 -- Winter 2024}\\[3pt]
{\large Semester 1 Study Material}\\[3pt]
{\normalsize\textit{Detailed Solutions and Explanations}}
\end{center}

\vspace{10pt}

\subsection*{પ્રશ્ન 1(a) [3
ગુણ]}\label{q1a}

\textbf{દરેક પ્રશ્નોના જવાબ આપો}

\textbf{i) ઇન્ફર્મેશન સિક્યુરિટી શું છે?}

\begin{solutionbox}
Information Security એ ડિજિટલ ડેટાને અનધિકૃત પ્રવેશ, ઉપયોગ,
જાહેરાત, વિક્ષેપ, ફેરફાર અથવા વિનાશથી સુરક્ષિત રાખે છે.

\textbf{મુખ્ય ઘટકો:}

\begin{itemize}
\tightlist
\item
  \textbf{ગુપ્તિયતા (Confidentiality)}: માત્ર અધિકૃત વપરાશકર્તાઓ ડેટા પ્રાપ્ત
  કરી શકે
\item
  \textbf{અખંડિતતા (Integrity)}: ડેટા ચોક્કસ અને સંપૂર્ણ રહે
\item
  \textbf{ઉપલબ્ધતા (Availability)}: જરૂર પડે ત્યારે ડેટા મળી શકે
\end{itemize}

\end{solutionbox}
\begin{mnemonicbox}
``CIA ડેટાને સુરક્ષિત રાખે છે''

\textbf{ii) હેકર્સના પ્રકારોની યાદી લખો}

\end{mnemonicbox}
\begin{solutionbox}

{\def\LTcaptype{none} % do not increment counter
\begin{longtable}[]{@{}lll@{}}
\toprule\noalign{}
હેકર પ્રકાર & વિવરણ & હેતુ \\
\midrule\noalign{}
\endhead
\bottomrule\noalign{}
\endlastfoot
White Hat & નૈતિક હેકર્સ & સારા ઈરાદા \\
Black Hat & દુષ્ટ હેકર્સ & ગુનાહિત પ્રવૃત્તિઓ \\
Gray Hat & બન્નેનું મિશ્રણ & તટસ્થ હેતુ \\
Script Kiddies & હાલના tools વાપરે & મર્યાદિત કુશળતા \\
\end{longtable}
}

\textbf{iii) Kali Linux માટે default username અને password શું હોય છે?}

\end{solutionbox}
\begin{solutionbox}

\begin{itemize}
\tightlist
\item
  \textbf{Username}: kali
\item
  \textbf{Password}: kali (નવા versions માં root/toor થી બદલાયું)
\end{itemize}

\end{solutionbox}
\begin{center}\rule{0.5\linewidth}{0.5pt}\end{center}

\subsection*{પ્રશ્ન 1(b) [4
ગુણ]}\label{q1b}

\textbf{CIA triad ઉદાહરણ સાથે સમજાવો}

\begin{solutionbox}
CIA Triad એ information security નો પાયો છે જેમાં ત્રણ મુખ્ય
સિદ્ધાંતો છે:

{\def\LTcaptype{none} % do not increment counter
\begin{longtable}[]{@{}
  >{\raggedright\arraybackslash}p{(\linewidth - 4\tabcolsep) * \real{0.3448}}
  >{\raggedright\arraybackslash}p{(\linewidth - 4\tabcolsep) * \real{0.3103}}
  >{\raggedright\arraybackslash}p{(\linewidth - 4\tabcolsep) * \real{0.3448}}@{}}
\toprule\noalign{}
\begin{minipage}[b]{\linewidth}\raggedright
સિદ્ધાંત
\end{minipage} & \begin{minipage}[b]{\linewidth}\raggedright
વ્યાખ્યા
\end{minipage} & \begin{minipage}[b]{\linewidth}\raggedright
ઉદાહરણ
\end{minipage} \\
\midrule\noalign{}
\endhead
\bottomrule\noalign{}
\endlastfoot
\textbf{ગુપ્તિયતા (Confidentiality)} & માત્ર અધિકૃત વપરાશકર્તાઓ ડેટા પ્રાપ્ત કરી
શકે & Password protection, encryption \\
\textbf{અખંડિતતા (Integrity)} & ડેટા ચોક્કસ અને અપરિવર્તિત રહે & Digital
signatures, checksums \\
\textbf{ઉપલબ્ધતા (Availability)} & જરૂર પડે ત્યારે ડેટા મળી શકે & Backup
systems, redundancy \\
\end{longtable}
}

\textbf{વાસ્તવિક ઉદાહરણ}: બેંકિંગ સિસ્ટમ login credentials દ્વારા ગુપ્તિયતા,
transaction verification દ્વારા અખંડિતતા, અને 24/7 સેવા દ્વારા ઉપલબ્ધતા જાળવે
છે.

\end{solutionbox}
\begin{mnemonicbox}
``CIA માહિતીને ગુપ્ત એજન્ટની જેમ સુરક્ષિત રાખે છે''

\end{mnemonicbox}
\begin{center}\rule{0.5\linewidth}{0.5pt}\end{center}

\subsection*{પ્રશ્ન 1(c) [7
ગુણ]}\label{q1c}

\textbf{MD5 hashing algorithm સમજાવો}

\begin{solutionbox}
MD5 (Message Digest 5) એ cryptographic hash function છે જે
128-bit hash values બનાવે છે.

\textbf{MD5 પ્રક્રિયા કોષ્ટક:}

{\def\LTcaptype{none} % do not increment counter
\begin{longtable}[]{@{}lll@{}}
\toprule\noalign{}
પગલું & પ્રક્રિયા & વિગતો \\
\midrule\noalign{}
\endhead
\bottomrule\noalign{}
\endlastfoot
1 & \textbf{Padding} & લેન્થ ≡ 448 (mod 512) બનાવવા bits ઉમેરો \\
2 & \textbf{Length Addition} & 64-bit લેન્થ append કરો \\
3 & \textbf{Initialize} & ચાર 32-bit variables સેટ કરો \\
4 & \textbf{Processing} & ચાર rounds ના operations \\
5 & \textbf{Output} & 128-bit hash value \\
\end{longtable}
}

\begin{verbatim}
flowchart LR
    A[Input Message] {-{-} B[Padding]}
    B {-{-} C[Append Length]}
    C {-{-} D[Initialize MD Buffer]}
    D {-{-} E[Process in 512{-}bit Blocks]}
    E {-{-} F[128{-}bit Hash Output]}
\end{verbatim}

\textbf{મુખ્ય લક્ષણો:}

\begin{itemize}
\tightlist
\item
  \textbf{નિશ્ચિત Output}: હંમેશા 128 bits
\item
  \textbf{એક-તરફી}: hash ને original માં પાછું ફેરવી શકાતું નથી
\item
  \textbf{Collision Prone}: આક્રમણો માટે સંવેદનશીલ
\end{itemize}

\end{solutionbox}
\begin{mnemonicbox}
``MD5 ડેટાને 5-પગલાંના hash માં બનાવે છે''

\end{mnemonicbox}
\begin{center}\rule{0.5\linewidth}{0.5pt}\end{center}

\subsection*{પ્રશ્ન 1(c) OR [7
ગુણ]}\label{q1c}

\textbf{SHA algorithm સમજાવો}

\begin{solutionbox}
SHA (Secure Hash Algorithm) એ NSA દ્વારા ડિઝાઇન કરાયેલ
cryptographic hash functions નું પરિવાર છે.

\textbf{SHA વેરિઅન્ટ્સ તુલના:}

{\def\LTcaptype{none} % do not increment counter
\begin{longtable}[]{@{}llll@{}}
\toprule\noalign{}
સંસ્કરણ & Output Size & Block Size & Security Level \\
\midrule\noalign{}
\endhead
\bottomrule\noalign{}
\endlastfoot
SHA-1 & 160 bits & 512 bits & Deprecated \\
SHA-256 & 256 bits & 512 bits & મજબૂત \\
SHA-512 & 512 bits & 1024 bits & ખૂબ મજબૂત \\
\end{longtable}
}

\begin{verbatim}
flowchart LR
    A[Message] {-{-} B[Pre{-}processing]}
    B {-{-} C[Hash Computation]}
    C {-{-} D[Final Hash]}
    
    B {-{-} B1[Padding]}
    B {-{-} B2[Parsing]}
    
    C {-{-} C1[Initialize Hash Values]}
    C {-{-} C2[Process Message Blocks]}
    C {-{-} C3[Compute Intermediate Hash]}
\end{verbatim}

\textbf{SHA-256 પ્રક્રિયા:}

\begin{itemize}
\tightlist
\item
  \textbf{Preprocessing}: Message ને padding અને parsing
\item
  \textbf{Hash Computation}: 64 rounds ના operations
\item
  \textbf{Final Hash}: 256-bit output
\end{itemize}

\textbf{MD5 કરતાં ફાયદા:}

\begin{itemize}
\tightlist
\item
  \textbf{મજબૂત સુરક્ષા}: Collision attacks સામે પ્રતિરોધક
\item
  \textbf{મોટું Output}: સુરક્ષા માટે વધુ bits
\item
  \textbf{સરકારી માનક}: NIST દ્વારા મંજૂર
\end{itemize}

\end{solutionbox}
\begin{mnemonicbox}
``SHA બધા ડેટાને સુરક્ષિત રીતે hash કરે છે''

\end{mnemonicbox}
\begin{center}\rule{0.5\linewidth}{0.5pt}\end{center}

\subsection*{પ્રશ્ન 2(a) [3
ગુણ]}\label{q2a}

\textbf{વાઈરસ શું છે? વાઈરસની લાઈફ સાયકલ સમજાવો.}

\begin{solutionbox}
Computer virus એ દુષ્ટ software છે જે પોતાની નકલો અન્ય
programs અથવા files માં મૂકીને પ્રતિકૃતિ બનાવે છે.

\textbf{વાઈરસ લાઈફ સાયકલ:}

\begin{center}
\textbf{Mermaid Diagram (Code)}
\begin{verbatim}
{Shaded}
{Highlighting}[]
graph LR
    A[Dormant Phase] {-{-}{} B[Propagation Phase]}
    B {-{-}{} C[Triggering Phase]}
    C {-{-}{} D[Execution Phase]}
    D {-{-}{} A}
{Highlighting}
{Shaded}
\end{verbatim}
\end{center}

\textbf{તબક્કાની વિગતો:}

\begin{itemize}
\tightlist
\item
  \textbf{Dormant}: વાઈરસ નિષ્ક્રિય રહે છે
\item
  \textbf{Propagation}: અન્ય સિસ્ટમ્સમાં પોતાની નકલ કરે છે
\item
  \textbf{Triggering}: ચોક્કસ પરિસ્થિતિઓ દ્વારા સક્રિય થાય છે
\item
  \textbf{Execution}: દુષ્ટ પ્રવૃત્તિઓ કરે છે
\end{itemize}

\end{solutionbox}
\begin{mnemonicbox}
``વાઈરસ ડાન્સ કરે, ફેલાવે, ચાલુ કરે, ચલાવે''

\end{mnemonicbox}
\begin{center}\rule{0.5\linewidth}{0.5pt}\end{center}

\subsection*{પ્રશ્ન 2(b) [4
ગુણ]}\label{q2b}

\textbf{દરેક પ્રશ્નોના જવાબ આપો}

\textbf{i) Private key અને Public Key cryptography વચ્ચેના તફાવત જણાવો}

\begin{solutionbox}

{\def\LTcaptype{none} % do not increment counter
\begin{longtable}[]{@{}lll@{}}
\toprule\noalign{}
પાસાં & Private Key & Public Key \\
\midrule\noalign{}
\endhead
\bottomrule\noalign{}
\endlastfoot
\textbf{Keys} & એક જ shared key & Key pair (public/private) \\
\textbf{ઝડપ} & ઝડપી encryption & ધીમી encryption \\
\textbf{Key Distribution} & મુશ્કેલ & સરળ વિતરણ \\
\textbf{Scalability} & વિશાળ networks માટે ખરાબ & સારી scalability \\
\end{longtable}
}

\textbf{ii) Database Forensics ની વ્યાખ્યા આપો અને Database Forensics
દરમ્યાન કરવામાં આવતી વિવિધ પ્રવૃત્તિઓની યાદી લખો.}

\end{solutionbox}
\begin{solutionbox}
Database forensics એ કાનૂની કાર્યવાહી માટે ડિજિટલ પુરાવા
મેળવવા database systems ની તપાસ કરે છે.

\textbf{કરવામાં આવતી પ્રવૃત્તિઓ:}

\begin{itemize}
\tightlist
\item
  \textbf{Log Analysis}: Transaction logs ની તપાસ
\item
  \textbf{Metadata Extraction}: Database structure ની પુનઃપ્રાપ્તિ
\item
  \textbf{Deleted Data Recovery}: દૂર કરેલા records ની પુનઃપ્રાપ્તિ
\item
  \textbf{Timeline Analysis}: ડેટા modifications ને track કરવું
\end{itemize}

\end{solutionbox}
\begin{center}\rule{0.5\linewidth}{0.5pt}\end{center}

\subsection*{પ્રશ્ન 2(c) [7
ગુણ]}\label{q2c}

\textbf{Proxy server વિશે સમજાવો અને શા માટે તેની જરૂરિયાત છે?}

\begin{solutionbox}
Proxy server એ client અને server વચ્ચે મધ્યસ્થી તરીકે કામ કરે છે,
requests અને responses ને forward કરે છે.

\textbf{Proxy Server આર્કિટેક્ચર:}

\begin{verbatim}
sequenceDiagram
    participant C as Client
    participant P as Proxy Server
    participant S as Target Server
    
    C{-P: Request}
    P{-S: Forward Request}
    S{-P: Response}
    P{-C: Forward Response}
\end{verbatim}

\textbf{Proxy Servers ના પ્રકારો:}

{\def\LTcaptype{none} % do not increment counter
\begin{longtable}[]{@{}lll@{}}
\toprule\noalign{}
પ્રકાર & કાર્ય & ઉપયોગ \\
\midrule\noalign{}
\endhead
\bottomrule\noalign{}
\endlastfoot
\textbf{Forward Proxy} & Client-side મધ્યસ્થી & Web filtering \\
\textbf{Reverse Proxy} & Server-side મધ્યસ્થી & Load balancing \\
\textbf{Transparent Proxy} & Client ને અદ્રશ્ય & Content caching \\
\end{longtable}
}

\textbf{Proxy Servers ની જરૂરિયાત:}

\begin{itemize}
\tightlist
\item
  \textbf{સુરક્ષા}: Client IP addresses છુપાવે છે
\item
  \textbf{Performance}: વારંવાર access કરવામાં આવતું content cache કરે છે
\item
  \textbf{Control}: Traffic ને filter અને monitor કરે છે
\item
  \textbf{Anonymity}: વપરાશકર્તાની privacy સુરક્ષિત રાખે છે
\end{itemize}

\textbf{ફાયદા:}

\begin{itemize}
\tightlist
\item
  \textbf{Bandwidth Saving}: Caching દ્વારા traffic ઘટે છે
\item
  \textbf{Access Control}: અનિચ્છિત sites ને block કરે છે
\item
  \textbf{Load Distribution}: Server requests ને balance કરે છે
\end{itemize}

\end{solutionbox}
\begin{mnemonicbox}
``Proxy Privacy અને Performance ને સુરક્ષિત રાખે છે''

\end{mnemonicbox}
\begin{center}\rule{0.5\linewidth}{0.5pt}\end{center}

\subsection*{પ્રશ્ન 2(a) OR [3
ગુણ]}\label{q2a}

\textbf{વ્યાખ્યા આપો: Trojans, Rootkit, Backdoors, Keylogger}

\begin{solutionbox}

{\def\LTcaptype{none} % do not increment counter
\begin{longtable}[]{@{}ll@{}}
\toprule\noalign{}
Malware પ્રકાર & વ્યાખ્યા \\
\midrule\noalign{}
\endhead
\bottomrule\noalign{}
\endlastfoot
\textbf{Trojans} & કાયદેસર programs તરીકે વેશમાં રહેલા દુષ્ટ software \\
\textbf{Rootkit} & સિસ્ટમમાં malware ની હાજરી છુપાવતા software \\
\textbf{Backdoors} & સામાન્ય authentication ને bypass કરતા ગુપ્ત પ્રવેશ
બિંદુઓ \\
\textbf{Keylogger} & Passwords ચોરવા keystrokes record કરતા software \\
\end{longtable}
}

\end{solutionbox}
\begin{mnemonicbox}
``TRBK - Trojans, Rootkits, Backdoors આક્રમણ કરતા રહે
છે''

\end{mnemonicbox}
\begin{center}\rule{0.5\linewidth}{0.5pt}\end{center}

\subsection*{પ્રશ્ન 2(b) OR [4
ગુણ]}\label{q2b}

\textbf{દરેક પ્રશ્નોના જવાબ આપો}

\textbf{i) Firewall ના ફાયદા અને ગેરફાયદા જણાવો.}

\begin{solutionbox}

{\def\LTcaptype{none} % do not increment counter
\begin{longtable}[]{@{}ll@{}}
\toprule\noalign{}
ફાયદા & ગેરફાયદા \\
\midrule\noalign{}
\endhead
\bottomrule\noalign{}
\endlastfoot
\textbf{Network Protection} & \textbf{Performance Impact} \\
\textbf{Access Control} & \textbf{Configuration Complexity} \\
\textbf{Traffic Monitoring} & \textbf{Cannot Stop All Attacks} \\
\textbf{Log Generation} & \textbf{Maintenance Required} \\
\end{longtable}
}

\textbf{ii) ડિજિટલ પુરાવાઓને સાચવવા માટેના મહત્વપૂર્ણ પગલાઓની યાદી બનાવો.}

\end{solutionbox}
\begin{solutionbox}

\begin{itemize}
\tightlist
\item
  \textbf{Identification}: સંભવિત પુરાવા શોધવા
\item
  \textbf{Documentation}: પુરાવાની વિગતો record કરવી
\item
  \textbf{Collection}: પુરાવાને સુરક્ષિત રીતે એકત્રિત કરવા
\item
  \textbf{Preservation}: પુરાવાની અખંડિતતા જાળવવી
\item
  \textbf{Chain of Custody}: પુરાવાના handling ને track કરવું
\end{itemize}

\end{solutionbox}
\begin{center}\rule{0.5\linewidth}{0.5pt}\end{center}

\subsection*{પ્રશ્ન 2(c) OR [7
ગુણ]}\label{q2c}

\textbf{IP Security Architecture સમજાવો.}

\begin{solutionbox}
IPSec એ IP communications માટે network layer પર security
services પૂરી પાડે છે.

\textbf{IPSec આર્કિટેક્ચર ઘટકો:}

\begin{verbatim}
graph TB
    A[IPSec Architecture] {-{-} B[Security Protocols]}
    A {-{-} C[Security Associations]}
    A {-{-} D[Key Management]}
    
    B {-{-} B1[AH {-} Authentication Header]}
    B {-{-} B2[ESP {-} Encapsulating Security Payload]}
    
    C {-{-} C1[SAD {-} Security Association Database]}
    C {-{-} C2[SPD {-} Security Policy Database]}
    
    D {-{-} D1[IKE {-} Internet Key Exchange]}
\end{verbatim}

\textbf{Security Services:}

{\def\LTcaptype{none} % do not increment counter
\begin{longtable}[]{@{}lll@{}}
\toprule\noalign{}
Service & Protocol & કાર્ય \\
\midrule\noalign{}
\endhead
\bottomrule\noalign{}
\endlastfoot
\textbf{Authentication} & AH & Packet origin ને verify કરવું \\
\textbf{Confidentiality} & ESP & Packet data ને encrypt કરવું \\
\textbf{Integrity} & બન્ને & Modifications detect કરવા \\
\textbf{Anti-replay} & બન્ને & Replay attacks ને અટકાવવા \\
\end{longtable}
}

\textbf{IPSec Modes:}

\begin{itemize}
\tightlist
\item
  \textbf{Transport Mode}: માત્ર payload ને protect કરે છે
\item
  \textbf{Tunnel Mode}: સંપૂર્ણ IP packet ને protect કરે છે
\end{itemize}

\textbf{મુખ્ય ઘટકો:}

\begin{itemize}
\tightlist
\item
  \textbf{Security Association (SA)}: Security parameters
\item
  \textbf{Security Policy Database (SPD)}: Security policies
\item
  \textbf{Key Management}: Automated key exchange
\end{itemize}

\end{solutionbox}
\begin{mnemonicbox}
``IPSec સંપૂર્ણ રીતે Protection, Security, Encryption ને
integrate કરે છે''

\end{mnemonicbox}
\begin{center}\rule{0.5\linewidth}{0.5pt}\end{center}

\subsection*{પ્રશ્ન 3(a) [3
ગુણ]}\label{q3a}

\textbf{સાયબર ક્રાઈમના પ્રકારો લખો અને કોઈપણ એક વિશે સમજાવો}

\begin{solutionbox}

\textbf{સાયબર ક્રાઈમ પ્રકારો:}

\begin{itemize}
\tightlist
\item
  \textbf{Financial Crimes}: Credit card fraud, online banking theft
\item
  \textbf{Identity Theft}: વ્યક્તિગત માહિતી ચોરી
\item
  \textbf{Cyber Bullying}: Online harassment
\item
  \textbf{Data Breach}: અનધિકૃત ડેટા access
\end{itemize}

\textbf{Email Bombing (વિગતવાર સમજૂતી):} Email bombing માં victim ના
mailbox અને server resources ને overwhelm કરવા માટે મોટી માત્રામાં emails
મોકલવામાં આવે છે.

\textbf{Attack Process:}

\begin{itemize}
\tightlist
\item
  \textbf{Target Selection}: Victim email પસંદ કરવું
\item
  \textbf{Volume Generation}: હજારો emails મોકલવા
\item
  \textbf{Resource Exhaustion}: Mail server ને overwhelm કરવું
\item
  \textbf{Service Disruption}: Email ને unusable બનાવવું
\end{itemize}

\end{solutionbox}
\begin{mnemonicbox}
``સાયબર ક્રાઈમ્સ સતત અંધાધૂંધ મચાવે છે''

\end{mnemonicbox}
\begin{center}\rule{0.5\linewidth}{0.5pt}\end{center}

\subsection*{પ્રશ્ન 3(b) [4
ગુણ]}\label{q3b}

\textbf{વ્યાખ્યા આપો: Web Jacking, Data Diddling, DoS Attack અને DDoS
Attack}

\begin{solutionbox}

{\def\LTcaptype{none} % do not increment counter
\begin{longtable}[]{@{}
  >{\raggedright\arraybackslash}p{(\linewidth - 2\tabcolsep) * \real{0.6000}}
  >{\raggedright\arraybackslash}p{(\linewidth - 2\tabcolsep) * \real{0.4000}}@{}}
\toprule\noalign{}
\begin{minipage}[b]{\linewidth}\raggedright
Attack પ્રકાર
\end{minipage} & \begin{minipage}[b]{\linewidth}\raggedright
વ્યાખ્યા
\end{minipage} \\
\midrule\noalign{}
\endhead
\bottomrule\noalign{}
\endlastfoot
\textbf{Web Jacking} & Content બદલીને website પર અનધિકૃત control \\
\textbf{Data Diddling} & Processing પહેલાં ડેટાનું અનધિકૃત modification \\
\textbf{DoS Attack} & Service ને unavailable બનાવવા single source
attack \\
\textbf{DDoS Attack} & Target system ને overwhelm કરવા multiple sources
attack \\
\end{longtable}
}

\textbf{Attack Comparison:}

\begin{center}
\textbf{Mermaid Diagram (Code)}
\begin{verbatim}
{Shaded}
{Highlighting}[]
graph LR
    A[DoS Attack] {-{-}{} B[Single Attacker]}
    C[DDoS Attack] {-{-}{} D[Multiple Attackers]}
    B {-{-}{} E[Target Server]}
    D {-{-}{} E}
{Highlighting}
{Shaded}
\end{verbatim}
\end{center}

\end{solutionbox}
\begin{center}\rule{0.5\linewidth}{0.5pt}\end{center}

\subsection*{પ્રશ્ન 3(c) [7
ગુણ]}\label{q3c}

\textbf{Man in the middle attack યોગ્ય ઉદાહરણ સાથે સમજાવો.}

\begin{solutionbox}
Man-in-the-Middle (MITM) attack ત્યારે થાય છે જ્યારે આક્રમણકર્તા
બે પક્ષો વચ્ચેના communications ને ગુપ્ત રીતે intercept કરે અને relay કરે છે.

\textbf{MITM Attack Process:}

\begin{verbatim}
sequenceDiagram
    participant A as Alice
    participant M as Attacker (Mallory)
    participant B as Bob
    
    A{-M: Bob ને Message}
    M{-M: Intercept \& Read}
    M{-B: Modified/Original Message}
    B{-M: Alice ને Response}
    M{-M: Intercept \& Read}
    M{-A: Modified/Original Response}
\end{verbatim}

\textbf{Attack પ્રકારો:}

{\def\LTcaptype{none} % do not increment counter
\begin{longtable}[]{@{}lll@{}}
\toprule\noalign{}
પ્રકાર & પદ્ધતિ & ઉદાહરણ \\
\midrule\noalign{}
\endhead
\bottomrule\noalign{}
\endlastfoot
\textbf{Wi-Fi Eavesdropping} & Fake hotspots & Coffee shop Wi-Fi \\
\textbf{Email Hijacking} & Compromised accounts & Business email \\
\textbf{DNS Spoofing} & Fake DNS responses & Fake sites પર redirect \\
\textbf{HTTPS Spoofing} & Fake certificates & Banking websites \\
\end{longtable}
}

\textbf{વાસ્તવિક ઉદાહરણ - Wi-Fi Attack:}

\begin{enumerate}
\tightlist
\item
  આક્રમણકર્તા fake ``Free\_WiFi'' hotspot બનાવે છે
\item
  Victim malicious network સાથે connect થાય છે
\item
  બધો traffic આક્રમણકર્તા મારફતે જાય છે
\item
  Passwords જેવા sensitive data ચોરાય છે
\end{enumerate}

\textbf{Prevention Measures:}

\begin{itemize}
\tightlist
\item
  \textbf{HTTPS ઉપયોગ}: Encrypted connections
\item
  \textbf{VPN Usage}: વધારાનું encryption layer
\item
  \textbf{Certificate Verification}: SSL certificates check કરવા
\item
  \textbf{Secure Networks}: Sensitive tasks માટે public Wi-Fi ટાળવું
\end{itemize}

\end{solutionbox}
\begin{mnemonicbox}
``MITM દુષ્ટતાથી Intercept કરે, Messages ને Tamper કરે''

\end{mnemonicbox}
\begin{center}\rule{0.5\linewidth}{0.5pt}\end{center}

\subsection*{પ્રશ્ન 3(a) OR [3
ગુણ]}\label{q3a}

\textbf{Salami attack વિશે સમજાવો}

\begin{solutionbox}
Salami attack માં detection ટાળવા માટે ઘણા accounts માંથી
નાની રકમો ચોરવામાં આવે છે.

\textbf{Attack Mechanism:}

\begin{itemize}
\tightlist
\item
  \textbf{નાની રકમો}: Currency ના fractions ચોરવા
\item
  \textbf{વિશાળ પાયે}: હજારો accounts ને target કરવા
\item
  \textbf{Rounding Errors}: Calculation differences નો ફાયદો ઉઠાવવો
\item
  \textbf{Accumulation}: નાની ચોરીઓ મોટો નફો બનાવે છે
\end{itemize}

\textbf{ઉદાહરણ}: Banking system interest ને nearest cent સુધી round કરે છે.
આક્રમણકર્તા લાખો accounts માંથી બાકી રહેલા fractions collect કરે છે.

\end{solutionbox}
\begin{mnemonicbox}
``Salami નાના ટુકડા કરે, મોટી રકમ ચોરે''

\end{mnemonicbox}
\begin{center}\rule{0.5\linewidth}{0.5pt}\end{center}

\subsection*{પ્રશ્ન 3(b) OR [4
ગુણ]}\label{q3b}

\textbf{Cyber bullying, Phishing, spyware અને logic bomb ની વ્યાખ્યા આપો.}

\begin{solutionbox}

{\def\LTcaptype{none} % do not increment counter
\begin{longtable}[]{@{}ll@{}}
\toprule\noalign{}
શબ્દ & વ્યાખ્યા \\
\midrule\noalign{}
\endhead
\bottomrule\noalign{}
\endlastfoot
\textbf{Cyber Bullying} & Emotional distress પેદા કરતું online
harassment \\
\textbf{Phishing} & Sensitive information મેળવવાના fraudulent પ્રયત્નો \\
\textbf{Spyware} & User activities ને ગુપ્ત રીતે monitor કરતા software \\
\textbf{Logic Bomb} & ચોક્કસ conditions દ્વારા trigger થતા malicious
code \\
\end{longtable}
}

\end{solutionbox}
\begin{center}\rule{0.5\linewidth}{0.5pt}\end{center}

\subsection*{પ્રશ્ન 3(c) OR [7
ગુણ]}\label{q3c}

\textbf{Ransomware વિશે વિસ્તારપૂર્વક સમજાવો}

\begin{solutionbox}
Ransomware victim ની files ને encrypt કરે છે અને decryption
key માટે payment ની માંગ કરે છે.

\textbf{Ransomware Attack Process:}

\begin{verbatim}
flowchart LR
    A[Initial Infection] {-{-} B[File Encryption]}
    B {-{-} C[Ransom Demand]}
    C {-{-} D[Payment Request]}
    D {-{-} E\{Payment Made?\}}
    E {-{-}|હા| F[Decryption Key]}
    E {-{-}|ના| G[Files Remain Encrypted]}
\end{verbatim}

\textbf{Ransomware પ્રકારો:}

{\def\LTcaptype{none} % do not increment counter
\begin{longtable}[]{@{}lll@{}}
\toprule\noalign{}
પ્રકાર & વર્તન & ઉદાહરણ \\
\midrule\noalign{}
\endhead
\bottomrule\noalign{}
\endlastfoot
\textbf{Crypto Ransomware} & Files ને encrypt કરે & WannaCry \\
\textbf{Locker Ransomware} & System access lock કરે & Police-themed \\
\textbf{Scareware} & Fake threats & Fake antivirus \\
\textbf{Doxware} & Data publication ની ધમકી & Personal photos \\
\end{longtable}
}

\textbf{Attack Vectors:}

\begin{itemize}
\tightlist
\item
  \textbf{Email Attachments}: દુષ્ટ documents
\item
  \textbf{Drive-by Downloads}: Compromised websites
\item
  \textbf{Exploit Kits}: Vulnerability exploitation
\item
  \textbf{RDP Attacks}: Remote desktop compromise
\end{itemize}

\textbf{Prevention Strategies:}

\begin{itemize}
\tightlist
\item
  \textbf{નિયમિત Backups}: Offline data copies
\item
  \textbf{Security Updates}: Vulnerabilities ને patch કરવા
\item
  \textbf{Email Filtering}: દુષ્ટ attachments ને block કરવા
\item
  \textbf{User Training}: Threats ને ઓળખવા
\item
  \textbf{Network Segmentation}: Spread ને limit કરવા
\end{itemize}

\textbf{Impact Assessment:}

\begin{itemize}
\tightlist
\item
  \textbf{નાણાકીય નુકસાન}: Ransom payments અને downtime
\item
  \textbf{ડેટા Loss}: કાયમ માટે encrypted files
\item
  \textbf{પ્રતિષ્ઠાને નુકસાન}: Customer trust loss
\item
  \textbf{Operational Disruption}: Business shutdown
\end{itemize}

\end{solutionbox}
\begin{mnemonicbox}
``Ransomware ખરેખર Recovery ને બરબાદ કરે, મજબૂત Response
જોઈએ''

\end{mnemonicbox}
\begin{center}\rule{0.5\linewidth}{0.5pt}\end{center}

\subsection*{પ્રશ્ન 4(a) [3
ગુણ]}\label{q4a}

\textbf{Kali Linux ના કોઈપણ ૬ મૂળભૂત commands લખો}

\begin{solutionbox}

{\def\LTcaptype{none} % do not increment counter
\begin{longtable}[]{@{}ll@{}}
\toprule\noalign{}
Command & કાર્ય \\
\midrule\noalign{}
\endhead
\bottomrule\noalign{}
\endlastfoot
\textbf{ls} & Directory contents list કરવા \\
\textbf{cd} & Directory બદલવા \\
\textbf{pwd} & Working directory print કરવા \\
\textbf{mkdir} & Directory બનાવવા \\
\textbf{cp} & Files copy કરવા \\
\textbf{nmap} & Network scanning \\
\end{longtable}
}

\end{solutionbox}
\begin{mnemonicbox}
``Linux Commands Navigation ને શક્ય બનાવે છે''

\end{mnemonicbox}
\begin{center}\rule{0.5\linewidth}{0.5pt}\end{center}

\subsection*{પ્રશ્ન 4(b) [4
ગુણ]}\label{q4b}

\textbf{Zero day attack ઉદાહરણ આપી સમજાવો}

\begin{solutionbox}
Zero-day attack એ અજ્ઞાત vulnerability નો ઉપયોગ કરીને
security patch ઉપલબ્ધ થાય તે પહેલાં હુમલો કરે છે.

\textbf{Zero-Day Timeline:}

\begin{verbatim}
timeline
    title Zero{-Day Attack Timeline}
    
    Day 0 : Vulnerability શોધાયું
          : Exploit બનાવ્યું
    
    Day 1{-X : Attack શરૂ કર્યું}
            : Systems સાથે ચેડાં કર્યા
    
    Day X+1 : Vulnerability જાહેર કર્યું
            : Patch Development
    
    Day X+Y : Patch રિલીઝ કર્યું
            : Systems અપડેટ કર્યા
\end{verbatim}

\textbf{ઉદાહરણ - Stuxnet Worm:}

\begin{itemize}
\tightlist
\item
  \textbf{Target}: ઈરાની પરમાણુ સુવિધાઓ
\item
  \textbf{Exploit}: Windows zero-day vulnerabilities
\item
  \textbf{Impact}: Centrifuges ને ભૌતિક નુકસાન
\item
  \textbf{Duration}: શોધાય તે પહેલાં મહિનાઓ સુધી સક્રિય
\end{itemize}

\textbf{લાક્ષણિકતાઓ:}

\begin{itemize}
\tightlist
\item
  \textbf{અજ્ઞાત Vulnerability}: હાલના patches નથી
\item
  \textbf{ઉચ્ચ સફળતા દર}: કોઈ defenses તૈયાર નથી
\item
  \textbf{કિંમતી}: Dark markets માં મોંઘા
\item
  \textbf{મર્યાદિત આયુષ્ય}: શોધાયા પછી patch થઈ જાય
\end{itemize}

\end{solutionbox}
\begin{mnemonicbox}
``Zero-day કોઈ જાણે તે પહેલાં માર્યા કરે''

\end{mnemonicbox}
\begin{center}\rule{0.5\linewidth}{0.5pt}\end{center}

\subsection*{પ્રશ્ન 4(c) [7
ગુણ]}\label{q4c}

\textbf{Remote Access Tools સમજાવો અને કેવી રીતે આપણે RAT થી system નું રક્ષણ
કરી શકીએ છે?}

\begin{solutionbox}
Remote Access Tool (RAT) એ computer systems ના remote
control ની મંજૂરી આપે છે, ઘણીવાર દુષ્ટતાપૂર્વક વપરાય છે.

\textbf{RAT Functionality:}

\begin{verbatim}
graph TB
    A[RAT Server on Victim] {-{-} B[File Access]}
    A {-{-} C[Screen Capture]}
    A {-{-} D[Keylogging]}
    A {-{-} E[Camera/Mic Access]}
    A {-{-} F[System Control]}
    
    G[Attacker Client] {-{-} A}
\end{verbatim}

\textbf{સામાન્ય RATs:}

{\def\LTcaptype{none} % do not increment counter
\begin{longtable}[]{@{}lll@{}}
\toprule\noalign{}
RAT નામ & Features & Detection મુશ્કેલી \\
\midrule\noalign{}
\endhead
\bottomrule\noalign{}
\endlastfoot
\textbf{DarkComet} & સંપૂર્ણ system control & મધ્યમ \\
\textbf{Poison Ivy} & Stealth operations & ઉચ્ચ \\
\textbf{Back Orifice} & Windows targeting & નીચી \\
\textbf{NetBus} & સરળ interface & નીચી \\
\end{longtable}
}

\textbf{RAT Infection પદ્ધતિઓ:}

\begin{itemize}
\tightlist
\item
  \textbf{Email Attachments}: Trojanized files
\item
  \textbf{Software Bundling}: કાયદેસર software માં છુપાયેલ
\item
  \textbf{Drive-by Downloads}: દુષ્ટ websites
\item
  \textbf{Social Engineering}: વપરાશકર્તાઓને installation માટે છેતરવા
\end{itemize}

\textbf{સુરક્ષા વ્યૂહરચનાઓ:}

\textbf{ટેકનિકલ પગલાં:}

\begin{itemize}
\tightlist
\item
  \textbf{Antivirus Software}: Real-time scanning
\item
  \textbf{Firewall Rules}: અનધિકૃત connections ને block કરવા
\item
  \textbf{Network Monitoring}: અસામાન્ય traffic detect કરવા
\item
  \textbf{System Updates}: Vulnerabilities ને patch કરવા
\end{itemize}

\textbf{વર્તણૂકીય પગલાં:}

\begin{itemize}
\tightlist
\item
  \textbf{Email સાવધાની}: Attachments ને verify કરવા
\item
  \textbf{Download Sources}: માત્ર વિશ્વસનીય sites વાપરવી
\item
  \textbf{નિયમિત Scans}: સમયાંતરે malware checks
\item
  \textbf{User Training}: Threats ને ઓળખવા
\end{itemize}

\textbf{Detection ના સંકેતો:}

\begin{itemize}
\tightlist
\item
  \textbf{ધીમી Performance}: અસામાન્ય system lag
\item
  \textbf{Network Activity}: અનપેક્ષિત connections
\item
  \textbf{File Changes}: બદલાયેલી અથવા નવી files
\item
  \textbf{વિચિત્ર વર્તન}: અનપેક્ષિત system actions
\end{itemize}

\textbf{Incident Response:}

\begin{enumerate}
\tightlist
\item
  \textbf{System ને Isolate કરવું}: Network થી disconnect કરવું
\item
  \textbf{પુરાવા Document કરવા}: દુષ્ટ activity record કરવી
\item
  \textbf{System સાફ કરવું}: RAT ને સંપૂર્ણ remove કરવું
\item
  \textbf{Data Restore કરવું}: સાફ backups માંથી
\item
  \textbf{Security મજબૂત કરવી}: Defenses સુધારવા
\end{enumerate}

\end{solutionbox}
\begin{mnemonicbox}
``RATs દૂરથી Access કરે, મજબૂત Response જોઈએ''

\end{mnemonicbox}
\begin{center}\rule{0.5\linewidth}{0.5pt}\end{center}

\subsection*{પ્રશ્ન 4(a) OR [3
ગુણ]}\label{q4a}

\textbf{Hacking તેમજ Blackhat અને White hat hacker વિશે ટૂંકમાં સમજાવો}

\begin{solutionbox}

{\def\LTcaptype{none} % do not increment counter
\begin{longtable}[]{@{}ll@{}}
\toprule\noalign{}
શબ્દ & વ્યાખ્યા \\
\midrule\noalign{}
\endhead
\bottomrule\noalign{}
\endlastfoot
\textbf{Hacking} & Systems અથવા networks માં અનધિકૃત પ્રવેશ મેળવવો \\
\textbf{Black Hat} & ગુનાહિત હેતુ સાથે દુષ્ટ hackers \\
\textbf{White Hat} & Security સુધારવામાં મદદ કરતા નૈતિક hackers \\
\end{longtable}
}

\textbf{તુલના:}

\begin{itemize}
\tightlist
\item
  \textbf{હેતુ}: White hat મદદ કરે, Black hat નુકસાન કરે
\item
  \textbf{અધિકૃતતા}: White hat ને permission હોય છે
\item
  \textbf{હેતુ}: White hat સુરક્ષા આપે, Black hat exploit કરે
\end{itemize}

\end{solutionbox}
\begin{mnemonicbox}
``Hats અલગ અલગ Hacking ટેવો ધરાવે છે''

\end{mnemonicbox}
\begin{center}\rule{0.5\linewidth}{0.5pt}\end{center}

\subsection*{પ્રશ્ન 4(b) OR [4
ગુણ]}\label{q4b}

\textbf{Port Scanning શું છે? કોઈપણ બે Port Scanning techniques સમજાવો}

\begin{solutionbox}
Port scanning એ target systems પર open ports અને services
શોધે છે.

\textbf{Port Scanning Techniques:}

{\def\LTcaptype{none} % do not increment counter
\begin{longtable}[]{@{}lll@{}}
\toprule\noalign{}
Technique & પદ્ધતિ & Stealth Level \\
\midrule\noalign{}
\endhead
\bottomrule\noalign{}
\endlastfoot
\textbf{TCP Connect} & સંપૂર્ણ connection & નીચી stealth \\
\textbf{SYN Scan} & અર્ધ-ખુલ્લું connection & ઉચ્ચી stealth \\
\end{longtable}
}

\textbf{TCP Connect Scan:}

\begin{itemize}
\tightlist
\item
  સંપૂર્ણ TCP handshake પૂર્ણ કરે
\item
  વિશ્વસનીય પણ સહેલાઈથી detect થાય
\item
  Target systems દ્વારા log થાય
\end{itemize}

\textbf{SYN Scan (Half-Open):}

\begin{itemize}
\tightlist
\item
  SYN મોકલે, SYN-ACK મળે, RST મોકલે
\item
  Stealthy, ઘણીવાર unlogged
\item
  Connect scan કરતાં ઝડપી
\end{itemize}

\end{solutionbox}
\begin{mnemonicbox}
``Port Scanning સિસ્ટમ Services ને Probe કરે''

\end{mnemonicbox}
\begin{center}\rule{0.5\linewidth}{0.5pt}\end{center}

\subsection*{પ્રશ્ન 4(c) OR [7
ગુણ]}\label{q4c}

\textbf{Hacking માટેની પ્રક્રિયા વિસ્તારપૂર્વક સમજાવો}

\begin{solutionbox}
Hacking એ અનધિકૃત system access મેળવવા માટે વ્યવસ્થિત
પાંચ-તબક્કાની પદ્ધતિ અનુસરે છે.

\textbf{Hacking ના પાંચ તબક્કા:}

\begin{verbatim}
flowchart TD
    A[Information Gathering] {-{-} B[Scanning]}
    B {-{-} C[Gaining Access]}
    C {-{-} D[Maintaining Access]}
    D {-{-} E[Covering Tracks]}
    E {-{-} A}
\end{verbatim}

\textbf{તબક્કાની વિગતો:}

\textbf{1. Information Gathering (Reconnaissance):}

\begin{itemize}
\tightlist
\item
  \textbf{Passive}: OSINT, social media research
\item
  \textbf{Active}: Network queries, DNS lookups
\item
  \textbf{Tools}: Google dorking, Whois, social engineering
\end{itemize}

\textbf{2. Scanning:}

\begin{itemize}
\tightlist
\item
  \textbf{Network Scanning}: Live hosts શોધવા
\item
  \textbf{Port Scanning}: Open services શોધવા
\item
  \textbf{Vulnerability Scanning}: Weaknesses ઓળખવા
\item
  \textbf{Tools}: Nmap, Nessus, OpenVAS
\end{itemize}

\textbf{3. Gaining Access:}

\begin{itemize}
\tightlist
\item
  \textbf{Exploit Vulnerabilities}: શોધાયેલા weaknesses વાપરવા
\item
  \textbf{Password Attacks}: Brute force, dictionary
\item
  \textbf{Social Engineering}: Humans ને manipulate કરવા
\item
  \textbf{Tools}: Metasploit, custom exploits
\end{itemize}

\textbf{4. Maintaining Access:}

\begin{itemize}
\tightlist
\item
  \textbf{Install Backdoors}: સતત access સુનિશ્ચિત કરવા
\item
  \textbf{Create User Accounts}: છુપાયેલ administrator
\item
  \textbf{Rootkits}: હાજરી છુપાવવા
\item
  \textbf{Tools}: Netcat, custom backdoors
\end{itemize}

\textbf{5. Covering Tracks:}

\begin{itemize}
\tightlist
\item
  \textbf{Log Deletion}: પુરાવા દૂર કરવા
\item
  \textbf{File Hiding}: દુષ્ટ files છુપાવવા
\item
  \textbf{Process Hiding}: ચાલતા programs છુપાવવા
\item
  \textbf{Tools}: Log cleaners, steganography
\end{itemize}

\textbf{વિગતવાર પ્રક્રિયા Flow:}

{\def\LTcaptype{none} % do not increment counter
\begin{longtable}[]{@{}llll@{}}
\toprule\noalign{}
તબક્કો & પ્રવૃત્તિઓ & સમય & જોખમ સ્તર \\
\midrule\noalign{}
\endhead
\bottomrule\noalign{}
\endlastfoot
\textbf{Reconnaissance} & Passive info gathering & દિવસો/અઠવાડિયા &
નીચું \\
\textbf{Scanning} & Active probing & કલાકો/દિવસો & મધ્યમ \\
\textbf{Gaining Access} & Exploitation & મિનિટો/કલાકો & ઉચ્ચું \\
\textbf{Maintaining Access} & Persistence & ચાલુ & મધ્યમ \\
\textbf{Covering Tracks} & Evidence removal & કલાકો & ઉચ્ચું \\
\end{longtable}
}

\textbf{કાયદેસર vs ગેરકાયદેસર Hacking:}

\begin{itemize}
\tightlist
\item
  \textbf{Ethical Hacking}: અધિકૃત penetration testing
\item
  \textbf{Malicious Hacking}: અનધિકૃત ગુનાહિત પ્રવૃત્તિ
\item
  \textbf{Bug Bounty}: કાયદેસર vulnerability discovery
\end{itemize}

\end{solutionbox}
\begin{mnemonicbox}
``Hackers તપાસ કરે, Scan કરે, પ્રવેશ મેળવે, જાળવે, છુપાવે''

\end{mnemonicbox}
\begin{center}\rule{0.5\linewidth}{0.5pt}\end{center}

\subsection*{પ્રશ્ન 5(a) [3
ગુણ]}\label{q5a}

\textbf{Locard's principal લખો અને તે સાયબર ક્રાઈમ સાથે કેવી રીતે સંબંધિત છે તે
સમજાવો?}

\begin{solutionbox}
Locard's Principle કહે છે કે ``દરેક સંપર્ક નિશાન છોડે છે'' -
forensic science નો મૂળભૂત સિદ્ધાંત.

\textbf{Digital Application:}

\begin{itemize}
\tightlist
\item
  \textbf{Log Files}: સિસ્ટમ પ્રવૃત્તિઓ record થાય છે
\item
  \textbf{Network Traffic}: Communication traces
\item
  \textbf{File Metadata}: બનાવટ, ફેરફારના સમય
\item
  \textbf{Memory Dumps}: Runtime evidence
\end{itemize}

\textbf{Cybercrime સાથે સંબંધ:} ડિજિટલ પ્રવૃત્તિઓ electronic traces બનાવે છે જેનું
વિશ્લેષણ કરીને investigators ગુનાહિત પ્રવૃત્તિઓનું પુનર્નિર્માણ કરી શકે છે.

\end{solutionbox}
\begin{mnemonicbox}
``Locard નો કાયદો: લાસ્ટિંગ Logs છોડે છે''

\end{mnemonicbox}
\begin{center}\rule{0.5\linewidth}{0.5pt}\end{center}

\subsection*{પ્રશ્ન 5(b) [4
ગુણ]}\label{q5b}

\textbf{Software forensics શું છે? તે સાયબર ક્રાઈમમાં કેવી રીતે યોગદાન આપી રહ્યું
છે?}

\begin{solutionbox}
Software forensics એ authorship નક્કી કરવા, plagiarism
detect કરવા, અથવા malicious code ની તપાસ કરવા software artifacts નું
વિશ્લેષણ કરે છે.

\textbf{Software Forensics Applications:}

{\def\LTcaptype{none} % do not increment counter
\begin{longtable}[]{@{}lll@{}}
\toprule\noalign{}
Application & હેતુ & Cybercrime ઉપયોગ \\
\midrule\noalign{}
\endhead
\bottomrule\noalign{}
\endlastfoot
\textbf{Code Analysis} & Programmer ઓળખવા & Malware attribution \\
\textbf{Binary Analysis} & Reverse engineering & Attacks સમજવા \\
\textbf{License Compliance} & Software piracy & IP theft cases \\
\textbf{Plagiarism Detection} & Academic integrity & Copyright
violation \\
\end{longtable}
}

\textbf{Cybercrime Investigation માં યોગદાન:}

\begin{itemize}
\tightlist
\item
  \textbf{Malware Attribution}: Code ને specific authors સાથે link કરવું
\item
  \textbf{Attack Reconstruction}: Attacks કેવી રીતે થયા તે સમજવું
\item
  \textbf{Evidence Collection}: ડિજિટલ proof એકત્રિત કરવા
\item
  \textbf{Pattern Recognition}: પુનરાવર્તિત ગુનેગારો ઓળખવા
\end{itemize}

\end{solutionbox}
\begin{center}\rule{0.5\linewidth}{0.5pt}\end{center}

\subsection*{પ્રશ્ન 5(c) [7
ગુણ]}\label{q5c}

\textbf{Drive imaging, Chain of custody અને hash values વિસ્તારપૂર્વક
સમજાવો.}

\begin{solutionbox}

\textbf{Drive Imaging:} Storage device નું ભૌતિક bit-by-bit copy જે deleted
files અને slack space સહિત બધો ડેટા સાચવે છે.

\textbf{Imaging Process:}

\begin{verbatim}
flowchart LR
    A[Original Drive] {-{-} B[Imaging Tool]}
    B {-{-} C[Forensic Image]}
    C {-{-} D[Hash Verification]}
    D {-{-} E[Analysis]}
\end{verbatim}

\textbf{Chain of Custody:} પુરાવાને seizure થી court presentation સુધી
handling track કરતું documentation.

\textbf{Chain of Custody Elements:}

{\def\LTcaptype{none} % do not increment counter
\begin{longtable}[]{@{}ll@{}}
\toprule\noalign{}
Element & વિગતો \\
\midrule\noalign{}
\endhead
\bottomrule\noalign{}
\endlastfoot
\textbf{કોણ} & પુરાવા handle કરતી વ્યક્તિ \\
\textbf{શું} & પુરાવાનું વર્ણન \\
\textbf{ક્યારે} & તારીખ અને સમય \\
\textbf{ક્યાં} & પુરાવાનું સ્થાન \\
\textbf{શા માટે} & Handling નું કારણ \\
\end{longtable}
}

\textbf{Hash Values:} ડેટા integrity verify કરવા unique fingerprints
બનાવતા ગાણિતિક algorithms.

\textbf{સામાન્ય Hash Algorithms:}

{\def\LTcaptype{none} % do not increment counter
\begin{longtable}[]{@{}lll@{}}
\toprule\noalign{}
Algorithm & Output Size & ઉપયોગ \\
\midrule\noalign{}
\endhead
\bottomrule\noalign{}
\endlastfoot
\textbf{MD5} & 128 bits & ઝડપી verification \\
\textbf{SHA-1} & 160 bits & Legacy systems \\
\textbf{SHA-256} & 256 bits & આધુનિક માનક \\
\end{longtable}
}

\textbf{Forensic Implementation:}

\begin{enumerate}
\tightlist
\item
  \textbf{Image બનાવવું}: Bit-by-bit copy
\item
  \textbf{Hash Generate કરવું}: Original drive hash calculate કરવું
\item
  \textbf{Integrity Verify કરવી}: Image hash compare કરવું
\item
  \textbf{Process Document કરવી}: Chain of custody
\item
  \textbf{સુરક્ષિત Analysis}: Copy પર જ કામ કરવું
\end{enumerate}

\textbf{Digital Forensics માં મહત્વ:}

\begin{itemize}
\tightlist
\item
  \textbf{Data Integrity}: પુરાવાની authenticity સુનિશ્ચિત કરે
\item
  \textbf{Legal Admissibility}: Court verified પુરાવા સ્વીકારે
\item
  \textbf{Non-Repudiation}: ડેટા અપરિવર્તિત હોવાનું સાબિત કરે
\item
  \textbf{Forensic Soundness}: પુરાવાની ગુણવત્તા જાળવે
\end{itemize}

\end{solutionbox}
\begin{mnemonicbox}
``Drive Images ડિજિટલ ડેટાને નિશ્ચિતપણે Document કરે''

\end{mnemonicbox}
\begin{center}\rule{0.5\linewidth}{0.5pt}\end{center}

\subsection*{પ્રશ્ન 5(a) OR [3
ગુણ]}\label{q5a}

\textbf{Malware analysis ના ચાર તબક્કાઓને ટૂંકમાં સમજાવો.}

\begin{solutionbox}

\textbf{Malware Analysis તબક્કાઓ:}

{\def\LTcaptype{none} % do not increment counter
\begin{longtable}[]{@{}
  >{\raggedright\arraybackslash}p{(\linewidth - 4\tabcolsep) * \real{0.2759}}
  >{\raggedright\arraybackslash}p{(\linewidth - 4\tabcolsep) * \real{0.2414}}
  >{\raggedright\arraybackslash}p{(\linewidth - 4\tabcolsep) * \real{0.4828}}@{}}
\toprule\noalign{}
\begin{minipage}[b]{\linewidth}\raggedright
તબક્કો
\end{minipage} & \begin{minipage}[b]{\linewidth}\raggedright
વર્ણન
\end{minipage} & \begin{minipage}[b]{\linewidth}\raggedright
વપરાતા Tools
\end{minipage} \\
\midrule\noalign{}
\endhead
\bottomrule\noalign{}
\endlastfoot
\textbf{Static Analysis} & Execution વગર તપાસ & Hex editors,
disassemblers \\
\textbf{Dynamic Analysis} & Runtime behavior નિરીક્ષણ & Sandboxes,
debuggers \\
\textbf{Code Analysis} & Source reverse engineer & IDA Pro, Ghidra \\
\textbf{Network Analysis} & Communications monitor & Wireshark,
tcpdump \\
\end{longtable}
}

\end{solutionbox}
\begin{mnemonicbox}
``Static, Dynamic, Code, Network - SDCN''

\end{mnemonicbox}
\begin{center}\rule{0.5\linewidth}{0.5pt}\end{center}

\subsection*{પ્રશ્ન 5(b) OR [4
ગુણ]}\label{q5b}

\textbf{Network forensic કેવી રીતે કાર્ય કરે છે?}

\begin{solutionbox}
Network forensics એ security incidents ની તપાસ કરવા
network traffic ને capture, record અને analyze કરે છે.

\textbf{Network Forensics Process:}

\begin{verbatim}
flowchart LR
    A[Traffic Capture] {-{-} B[Data Storage]}
    B {-{-} C[Analysis]}
    C {-{-} D[Evidence Extraction]}
    D {-{-} E[Reporting]}
\end{verbatim}

\textbf{મુખ્ય કાર્યો:}

\begin{itemize}
\tightlist
\item
  \textbf{Packet Capture}: Network communications record કરવા
\item
  \textbf{Protocol Analysis}: Communication protocols ની તપાસ
\item
  \textbf{Flow Analysis}: ડેટા movement patterns track કરવા
\item
  \textbf{Content Inspection}: Payload data નું વિશ્લેષણ
\end{itemize}

\textbf{Tools અને Techniques:}

\begin{itemize}
\tightlist
\item
  \textbf{Network Taps}: Hardware monitoring
\item
  \textbf{Packet Analyzers}: Software inspection
\item
  \textbf{Flow Collectors}: Traffic summarization
\item
  \textbf{SIEM Systems}: Correlation અને alerting
\end{itemize}

\end{solutionbox}
\begin{center}\rule{0.5\linewidth}{0.5pt}\end{center}

\subsection*{પ્રશ્ન 5(c) OR [7
ગુણ]}\label{q5c}

\textbf{Digital forensic investigation ની પ્રક્રિયા સમજાવો}

\begin{solutionbox}
Digital forensic investigation એ ડિજિટલ પુરાવા collect,
preserve, analyze અને present કરવા વ્યવસ્થિત પદ્ધતિ અનુસરે છે.

\textbf{Investigation Process તબક્કાઓ:}

\begin{verbatim}
flowchart LR
    A[Identification] {-{-} B[Preservation]}
    B {-{-} C[Collection]}
    C {-{-} D[Examination]}
    D {-{-} E[Analysis]}
    E {-{-} F[Presentation]}
\end{verbatim}

\textbf{વિગતવાર પ્રક્રિયા:}

\textbf{1. Identification તબક્કો:}

\begin{itemize}
\tightlist
\item
  \textbf{Evidence Location}: સંભવિત ડિજિટલ પુરાવા શોધવા
\item
  \textbf{Scope Definition}: તપાસની સીમાઓ નક્કી કરવી
\item
  \textbf{Resource Planning}: કર્મચારીઓ અને tools allocate કરવા
\item
  \textbf{Legal Considerations}: જરૂરી warrants મેળવવા
\end{itemize}

\textbf{2. Preservation તબક્કો:}

\begin{itemize}
\tightlist
\item
  \textbf{Scene Securing}: પુરાવા contamination અટકાવવા
\item
  \textbf{System Isolation}: Networks થી disconnect કરવું
\item
  \textbf{Evidence Documentation}: ફોટોગ્રાફ અને catalog
\item
  \textbf{Chain of Custody}: Documentation trail શરૂ કરવી
\end{itemize}

\textbf{3. Collection તબક્કો:}

\begin{itemize}
\tightlist
\item
  \textbf{Imaging Process}: Forensic copies બનાવવી
\item
  \textbf{Hash Generation}: ડેટા integrity સુનિશ્ચિત કરવી
\item
  \textbf{Metadata Capture}: File properties record કરવા
\item
  \textbf{Live Data Collection}: Volatile information capture કરવા
\end{itemize}

\textbf{4. Examination તબક્કો:}

\begin{itemize}
\tightlist
\item
  \textbf{Data Recovery}: Deleted files retrieve કરવી
\item
  \textbf{File System Analysis}: Storage structures ની તપાસ
\item
  \textbf{Timeline Creation}: Event sequence સ્થાપિત કરવું
\item
  \textbf{Keyword Searching}: સંબંધિત content શોધવા
\end{itemize}

\textbf{5. Analysis તબક્કો:}

\begin{itemize}
\tightlist
\item
  \textbf{Evidence Correlation}: સંબંધિત findings link કરવા
\item
  \textbf{Pattern Recognition}: Trends ઓળખવા
\item
  \textbf{Hypothesis Testing}: Theories validate કરવા
\item
  \textbf{Timeline Analysis}: Events reconstruct કરવા
\end{itemize}

\textbf{6. Presentation તબક્કો:}

\begin{itemize}
\tightlist
\item
  \textbf{Report Writing}: Findings document કરવા
\item
  \textbf{Evidence Preparation}: Court માટે organize કરવા
\item
  \textbf{Expert Testimony}: કાનૂની કાર્યવાહીમાં present કરવા
\item
  \textbf{Visualization}: Demonstrative aids બનાવવા
\end{itemize}

\textbf{Investigation સિદ્ધાંતો:}

{\def\LTcaptype{none} % do not increment counter
\begin{longtable}[]{@{}lll@{}}
\toprule\noalign{}
સિદ્ધાંત & વર્ણન & મહત્વ \\
\midrule\noalign{}
\endhead
\bottomrule\noalign{}
\endlastfoot
\textbf{Reliability} & પુરાવા ભરોસાપાત્ર હોવા જોઈએ & Court acceptance \\
\textbf{Repeatability} & પરિણામો reproduce થઈ શકે & Scientific
validity \\
\textbf{Integrity} & ડેટા અપરિવર્તિત રહે & Legal admissibility \\
\textbf{Documentation} & સંપૂર્ણ રેકોર્ડ keeping & Audit trail \\
\end{longtable}
}

\textbf{મુખ્ય પડકારો:}

\begin{itemize}
\tightlist
\item
  \textbf{Encryption}: Password-protected ડેટા
\item
  \textbf{Anti-Forensics}: પુરાવા છુપાવવાની techniques
\item
  \textbf{Volume}: મોટી માત્રામાં ડેટા
\item
  \textbf{Technology}: ઝડપથી બદલાતા systems
\end{itemize}

\textbf{Best Practices:}

\begin{itemize}
\tightlist
\item
  \textbf{Standard Procedures}: સ્થાપિત protocols અનુસરવા
\item
  \textbf{Tool Validation}: Tested forensic tools વાપરવા
\item
  \textbf{Continuous Training}: Technology સાથે current રહેવું
\item
  \textbf{Quality Assurance}: Peer review processes
\end{itemize}

\textbf{કાનૂની ફ્રેમવર્ક:}

\begin{itemize}
\tightlist
\item
  \textbf{Evidence Rules}: Admissibility requirements
\item
  \textbf{Privacy Laws}: ડેટા protection compliance
\item
  \textbf{Chain of Custody}: અખંડ documentation
\item
  \textbf{Expert Qualifications}: Forensic examiner credentials
\end{itemize}

\end{solutionbox}
\begin{mnemonicbox}
``ડિજિટલ તપાસ: ઓળખો, સાચવો, એકત્રિત કરો, તપાસો,
વિશ્લેષણ કરો, રજૂ કરો''

\end{mnemonicbox}

\end{document}
