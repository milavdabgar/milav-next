\documentclass[10pt,a4paper]{article}

% content/resources/templates/preamble.tex
\usepackage[margin=0.6in]{geometry}
\author{Milav Dabgar}
\usepackage{amsmath,amssymb,amsthm}
\usepackage{booktabs}
\usepackage{multirow}
\usepackage{xcolor}
\usepackage{tcolorbox}
\tcbuselibrary{breakable,skins}
\usepackage[colorlinks=true,linkcolor=blue]{hyperref}
\usepackage{titlesec}
\usepackage{enumitem}
\usepackage{tikz}
\usepackage{pgfplots}
\usepackage{circuitikz}
\usepackage[version=4]{mhchem}
\usepackage{longtable}
\usepackage{array}
\usepackage{float}
\usepackage{caption}
\usepackage{listings}

\lstset{
  basicstyle=\small\ttfamily,
  breaklines=true,
  breakatwhitespace=false,
  postbreak=\mbox{\textcolor{red}{$\hookrightarrow$}\space},
  float=false,
  numbers=left,
  numberstyle=\tiny\color{gray},
  numbersep=10pt,
  xleftmargin=2em,
  keywordstyle=\color{blue},
  commentstyle=\color{green!60!black},
  stringstyle=\color{purple},
  backgroundcolor=\color{gray!5},
  showstringspaces=false,
  tabsize=2,
  captionpos=b,
  keepspaces=true,
  columns=flexible
}

\pgfplotsset{compat=1.18}
\usetikzlibrary{shapes,arrows,positioning,calc,patterns,decorations.pathmorphing,decorations.markings,arrows.meta}

% Color scheme
\definecolor{headcolor}{RGB}{0,102,204}
\definecolor{keycolor}{RGB}{220,20,60}
\definecolor{solutioncolor}{RGB}{34,139,34}
\definecolor{mnemoniccolor}{RGB}{148,0,211}
\definecolor{codecolor}{RGB}{0,0,100}

% Spacing
\setlength{\parskip}{3pt}
\setlist[itemize]{nosep}
\setlist[enumerate]{nosep}

% Title formatting
\titleformat{\section}{\Large\bfseries\color{headcolor}}{\thesection}{1em}{}
\titleformat{\subsection}{\large\bfseries\color{headcolor}}{\thesubsection}{1em}{}

% Pandoc tightlist compatibility
\providecommand{\tightlist}{%
  \setlength{\itemsep}{0pt}\setlength{\parskip}{0pt}}

% Pandoc longtable compatibility
\newcounter{none}
\def\thenone{}


% content/resources/templates/english-boxes.tex
% This file is currently empty - it exists to maintain consistency with the import structure.
% Add custom environments here if needed in the future.


\begin{document}

\begin{center}
{\Huge\bfseries\color{headcolor} Subject Name Solutions}\\[5pt]
{\LARGE 4361601 -- Summer 2024}\\[3pt]
{\large Semester 1 Study Material}\\[3pt]
{\normalsize\textit{Detailed Solutions and Explanations}}
\end{center}

\vspace{10pt}

\subsection*{Question 1(a) [3 marks]}\label{q1a}

\textbf{Describe CIA triad with example.}

\begin{solutionbox}

\textbf{CIA Triad Comparison Table:}

{\def\LTcaptype{none} % do not increment counter
\begin{longtable}[]{@{}
  >{\raggedright\arraybackslash}p{(\linewidth - 4\tabcolsep) * \real{0.3438}}
  >{\raggedright\arraybackslash}p{(\linewidth - 4\tabcolsep) * \real{0.3750}}
  >{\raggedright\arraybackslash}p{(\linewidth - 4\tabcolsep) * \real{0.2812}}@{}}
\toprule\noalign{}
\begin{minipage}[b]{\linewidth}\raggedright
Component
\end{minipage} & \begin{minipage}[b]{\linewidth}\raggedright
Definition
\end{minipage} & \begin{minipage}[b]{\linewidth}\raggedright
Example
\end{minipage} \\
\midrule\noalign{}
\endhead
\bottomrule\noalign{}
\endlastfoot
\textbf{Confidentiality} & Ensures data is accessible only to authorized
users & Bank account details should only be viewed by account holder \\
\textbf{Integrity} & Ensures data remains accurate and unmodified &
Medical records must not be altered without authorization \\
\textbf{Availability} & Ensures systems and data are accessible when
needed & ATM services must be available 24/7 for customers \\
\end{longtable}
}

\end{solutionbox}
\begin{mnemonicbox}
``Can I Access'' - Confidentiality, Integrity,
Availability

\end{mnemonicbox}
\begin{center}\rule{0.5\linewidth}{0.5pt}\end{center}

\subsection*{Question 1(b) [4 marks]}\label{q1b}

\textbf{Explain Public key and Private Key cryptography.}

\begin{solutionbox}

\textbf{Key Differences Table:}

{\def\LTcaptype{none} % do not increment counter
\begin{longtable}[]{@{}
  >{\raggedright\arraybackslash}p{(\linewidth - 4\tabcolsep) * \real{0.1404}}
  >{\raggedright\arraybackslash}p{(\linewidth - 4\tabcolsep) * \real{0.4211}}
  >{\raggedright\arraybackslash}p{(\linewidth - 4\tabcolsep) * \real{0.4386}}@{}}
\toprule\noalign{}
\begin{minipage}[b]{\linewidth}\raggedright
Aspect
\end{minipage} & \begin{minipage}[b]{\linewidth}\raggedright
Public Key Cryptography
\end{minipage} & \begin{minipage}[b]{\linewidth}\raggedright
Private Key Cryptography
\end{minipage} \\
\midrule\noalign{}
\endhead
\bottomrule\noalign{}
\endlastfoot
\textbf{Keys Used} & Two keys (public + private) & Single shared key \\
\textbf{Key Distribution} & Public key can be shared openly & Key must
be shared secretly \\
\textbf{Speed} & Slower encryption/decryption & Faster operations \\
\textbf{Security} & Higher security, no key sharing problem & Lower
security due to key distribution \\
\end{longtable}
}

\textbf{Key Points:}

\begin{itemize}
\tightlist
\item
  \textbf{Public Key}: Uses asymmetric encryption with key pairs
\item
  \textbf{Private Key}: Uses symmetric encryption with shared secrets
\item
  \textbf{Digital Signatures}: Public key enables non-repudiation
\item
  \textbf{Key Management}: Private key requires secure distribution
\end{itemize}

\end{solutionbox}
\begin{mnemonicbox}
``PASS'' - Public Asymmetric, Symmetric Secret

\end{mnemonicbox}
\begin{center}\rule{0.5\linewidth}{0.5pt}\end{center}

\subsection*{Question 1(c) [7 marks]}\label{q1c}

\textbf{Explain various security services and security mechanism.}

\begin{solutionbox}

\textbf{Security Services Table:}

{\def\LTcaptype{none} % do not increment counter
\begin{longtable}[]{@{}
  >{\raggedright\arraybackslash}p{(\linewidth - 4\tabcolsep) * \real{0.2500}}
  >{\raggedright\arraybackslash}p{(\linewidth - 4\tabcolsep) * \real{0.2500}}
  >{\raggedright\arraybackslash}p{(\linewidth - 4\tabcolsep) * \real{0.5000}}@{}}
\toprule\noalign{}
\begin{minipage}[b]{\linewidth}\raggedright
Service
\end{minipage} & \begin{minipage}[b]{\linewidth}\raggedright
Purpose
\end{minipage} & \begin{minipage}[b]{\linewidth}\raggedright
Mechanism Example
\end{minipage} \\
\midrule\noalign{}
\endhead
\bottomrule\noalign{}
\endlastfoot
\textbf{Authentication} & Verify user identity & Passwords,
Biometrics \\
\textbf{Authorization} & Control access permissions & Access Control
Lists \\
\textbf{Confidentiality} & Protect data privacy & Encryption (AES,
RSA) \\
\textbf{Integrity} & Ensure data accuracy & Digital signatures,
Hashing \\
\textbf{Non-repudiation} & Prevent denial of actions & Digital
certificates \\
\textbf{Availability} & Ensure service accessibility & Firewalls, Backup
systems \\
\end{longtable}
}

\textbf{Security Mechanisms:}

\begin{itemize}
\tightlist
\item
  \textbf{Encryption}: Transforms plaintext to ciphertext
\item
  \textbf{Digital Signatures}: Provides authentication and integrity
\item
  \textbf{Access Control}: Restricts unauthorized access
\item
  \textbf{Audit Trails}: Monitor and log security events
\end{itemize}

\end{solutionbox}
\begin{mnemonicbox}
``ACIANA'' - Authentication, Confidentiality,
Integrity, Authorization, Non-repudiation, Availability

\end{mnemonicbox}
\begin{center}\rule{0.5\linewidth}{0.5pt}\end{center}

\subsection*{Question 1(c) OR [7
marks]}\label{q1c}

\textbf{Explain MD5 hashing algorithm.}

\begin{solutionbox}

\textbf{MD5 Algorithm Process:}

\begin{verbatim}
flowchart LR
    A[Input Message] {-{-} B[Padding]}
    B {-{-} C[Append Length]}
    C {-{-} D[Initialize MD Buffer]}
    D {-{-} E[Process in 512{-}bit blocks]}
    E {-{-} F[128{-}bit Hash Output]}
\end{verbatim}

\textbf{MD5 Characteristics Table:}

{\def\LTcaptype{none} % do not increment counter
\begin{longtable}[]{@{}ll@{}}
\toprule\noalign{}
Property & Value \\
\midrule\noalign{}
\endhead
\bottomrule\noalign{}
\endlastfoot
\textbf{Hash Size} & 128 bits (16 bytes) \\
\textbf{Block Size} & 512 bits \\
\textbf{Rounds} & 64 rounds \\
\textbf{Security Status} & Cryptographically broken \\
\end{longtable}
}

\textbf{Key Features:}

\begin{itemize}
\tightlist
\item
  \textbf{One-way Function}: Cannot reverse hash to original
\item
  \textbf{Fixed Output}: Always produces 128-bit hash
\item
  \textbf{Avalanche Effect}: Small input change creates large output
  change
\item
  \textbf{Collision Vulnerable}: Multiple inputs can produce same hash
\end{itemize}

\end{solutionbox}
\begin{mnemonicbox}
``MD5 FORB'' - Message Digest 5, Fixed Output, Rounds
64, Broken security

\end{mnemonicbox}
\begin{center}\rule{0.5\linewidth}{0.5pt}\end{center}

\subsection*{Question 2(a) [3 marks]}\label{q2a}

\textbf{What is firewall? List out types of firewall.}

\begin{solutionbox}

\textbf{Firewall Definition:} Network security device that monitors and
controls incoming/outgoing traffic based on predetermined rules.

\textbf{Firewall Types Table:}

{\def\LTcaptype{none} % do not increment counter
\begin{longtable}[]{@{}lll@{}}
\toprule\noalign{}
Type & Operation Level & Example \\
\midrule\noalign{}
\endhead
\bottomrule\noalign{}
\endlastfoot
\textbf{Packet Filtering} & Network Layer & iptables \\
\textbf{Stateful Inspection} & Session Layer & Cisco ASA \\
\textbf{Application Gateway} & Application Layer & Proxy servers \\
\textbf{Next-Gen Firewall} & Multiple Layers & Palo Alto \\
\end{longtable}
}

\end{solutionbox}
\begin{mnemonicbox}
``PSAN'' - Packet, Stateful, Application, Next-gen

\end{mnemonicbox}
\begin{center}\rule{0.5\linewidth}{0.5pt}\end{center}

\subsection*{Question 2(b) [4 marks]}\label{q2b}

\textbf{Define: HTTPS and describe working of HTTPS.}

\begin{solutionbox}

\textbf{HTTPS Definition:} HTTP Secure - encrypted version of HTTP using
SSL/TLS protocols.

\textbf{HTTPS Working Process:}

\begin{verbatim}
sequenceDiagram
    participant C as Client
    participant S as Server
    C{-S: 1. HTTPS Request}
    S{-C: 2. SSL Certificate}
    C{-S: 3. Verify \& Send Session Key}
    S{-C: 4. Encrypted Communication}
\end{verbatim}

\textbf{Key Components:}

\begin{itemize}
\tightlist
\item
  \textbf{SSL/TLS}: Provides encryption layer
\item
  \textbf{Digital Certificates}: Verify server identity
\item
  \textbf{Port 443}: Default HTTPS port
\item
  \textbf{End-to-end Encryption}: Protects data in transit
\end{itemize}

\end{solutionbox}
\begin{mnemonicbox}
``HTTPS SDP4'' - Secure, Digital certs, Port 443

\end{mnemonicbox}
\begin{center}\rule{0.5\linewidth}{0.5pt}\end{center}

\subsection*{Question 2(c) [7 marks]}\label{q2c}

\textbf{Give explanation of active attack and passive attack in detail.}

\begin{solutionbox}

\textbf{Attack Types Comparison:}

{\def\LTcaptype{none} % do not increment counter
\begin{longtable}[]{@{}lll@{}}
\toprule\noalign{}
Aspect & Active Attack & Passive Attack \\
\midrule\noalign{}
\endhead
\bottomrule\noalign{}
\endlastfoot
\textbf{Detection} & Easily detectable & Difficult to detect \\
\textbf{System Impact} & Modifies system/data & Only observes data \\
\textbf{Examples} & DoS, Man-in-middle & Eavesdropping, Traffic
analysis \\
\textbf{Prevention} & Firewalls, IDS & Encryption, Physical security \\
\end{longtable}
}

\textbf{Active Attack Types:}

\begin{itemize}
\tightlist
\item
  \textbf{Masquerade}: Impersonating authorized user
\item
  \textbf{Replay}: Retransmitting valid data transmissions
\item
  \textbf{Modification}: Altering message contents
\item
  \textbf{Denial of Service}: Preventing legitimate access
\end{itemize}

\textbf{Passive Attack Types:}

\begin{itemize}
\tightlist
\item
  \textbf{Traffic Analysis}: Studying communication patterns
\item
  \textbf{Eavesdropping}: Monitoring communications
\item
  \textbf{Footprinting}: Gathering system information
\end{itemize}

\end{solutionbox}
\begin{mnemonicbox}
``Active MRMD, Passive TEF'' -
Masquerade/Replay/Modify/DoS, Traffic/Eavesdrop/Footprint

\end{mnemonicbox}
\begin{center}\rule{0.5\linewidth}{0.5pt}\end{center}

\subsection*{Question 2(a) OR [3
marks]}\label{q2a}

\textbf{What is digital signature? Explain digital signature
properties.}

\begin{solutionbox}

\textbf{Digital Signature:} Cryptographic mechanism providing
authentication, integrity, and non-repudiation.

\textbf{Properties Table:}

{\def\LTcaptype{none} % do not increment counter
\begin{longtable}[]{@{}ll@{}}
\toprule\noalign{}
Property & Description \\
\midrule\noalign{}
\endhead
\bottomrule\noalign{}
\endlastfoot
\textbf{Authentication} & Verifies sender identity \\
\textbf{Integrity} & Ensures message unchanged \\
\textbf{Non-repudiation} & Prevents sender denial \\
\textbf{Unforgeable} & Cannot be created without private key \\
\end{longtable}
}

\end{solutionbox}
\begin{mnemonicbox}
``AINU'' - Authentication, Integrity,
Non-repudiation, Unforgeable

\end{mnemonicbox}
\begin{center}\rule{0.5\linewidth}{0.5pt}\end{center}

\subsection*{Question 2(b) OR [4
marks]}\label{q2b}

\textbf{Define: Trojans, Rootkit, Backdoors, Keylogger}

\begin{solutionbox}

\textbf{Malware Types Table:}

{\def\LTcaptype{none} % do not increment counter
\begin{longtable}[]{@{}
  >{\raggedright\arraybackslash}p{(\linewidth - 4\tabcolsep) * \real{0.1667}}
  >{\raggedright\arraybackslash}p{(\linewidth - 4\tabcolsep) * \real{0.3333}}
  >{\raggedright\arraybackslash}p{(\linewidth - 4\tabcolsep) * \real{0.5000}}@{}}
\toprule\noalign{}
\begin{minipage}[b]{\linewidth}\raggedright
Type
\end{minipage} & \begin{minipage}[b]{\linewidth}\raggedright
Definition
\end{minipage} & \begin{minipage}[b]{\linewidth}\raggedright
Primary Function
\end{minipage} \\
\midrule\noalign{}
\endhead
\bottomrule\noalign{}
\endlastfoot
\textbf{Trojans} & Malicious code disguised as legitimate software &
Provide unauthorized access \\
\textbf{Rootkit} & Software hiding presence of other malware & Conceal
malicious activities \\
\textbf{Backdoors} & Secret entry point bypassing security & Remote
unauthorized access \\
\textbf{Keylogger} & Records user keystrokes & Steal passwords/sensitive
data \\
\end{longtable}
}

\end{solutionbox}
\begin{mnemonicbox}
``TRBK'' - Trojans hide, Rootkits conceal, Backdoors
bypass, Keyloggers record

\end{mnemonicbox}
\begin{center}\rule{0.5\linewidth}{0.5pt}\end{center}

\subsection*{Question 2(c) OR [7
marks]}\label{q2c}

\textbf{Explain Secure Socket Layer.}

\begin{solutionbox}

\textbf{SSL Architecture:}

\begin{center}
\textbf{Mermaid Diagram (Code)}
\begin{verbatim}
{Shaded}
{Highlighting}[]
graph LR
    A[Application Layer] {-{-}{} B[SSL Record Protocol]}
    B {-{-}{} C[SSL Handshake Protocol]}
    B {-{-}{} D[SSL Change Cipher]}
    B {-{-}{} E[SSL Alert Protocol]}
    C {-{-}{} F[TCP Layer]}
    D {-{-}{} F}
    E {-{-}{} F}
{Highlighting}
{Shaded}
\end{verbatim}
\end{center}

\textbf{SSL Components Table:}

{\def\LTcaptype{none} % do not increment counter
\begin{longtable}[]{@{}ll@{}}
\toprule\noalign{}
Component & Function \\
\midrule\noalign{}
\endhead
\bottomrule\noalign{}
\endlastfoot
\textbf{Record Protocol} & Provides basic security services \\
\textbf{Handshake Protocol} & Establishes security parameters \\
\textbf{Change Cipher} & Signals encryption changes \\
\textbf{Alert Protocol} & Handles error conditions \\
\end{longtable}
}

\textbf{SSL Process:}

\begin{itemize}
\tightlist
\item
  \textbf{Handshake}: Negotiate security parameters
\item
  \textbf{Authentication}: Verify server identity
\item
  \textbf{Key Exchange}: Establish session keys
\item
  \textbf{Encryption}: Secure data transmission
\end{itemize}

\end{solutionbox}
\begin{mnemonicbox}
``SSL RHCA-HAKE'' - Record/Handshake/Change/Alert,
Handshake/Auth/Key/Encrypt

\end{mnemonicbox}
\begin{center}\rule{0.5\linewidth}{0.5pt}\end{center}

\subsection*{Question 3(a) [3 marks]}\label{q3a}

\textbf{Explain in detail cybercrime and cybercriminal.}

\begin{solutionbox}

\textbf{Definitions Table:}

{\def\LTcaptype{none} % do not increment counter
\begin{longtable}[]{@{}
  >{\raggedright\arraybackslash}p{(\linewidth - 2\tabcolsep) * \real{0.3333}}
  >{\raggedright\arraybackslash}p{(\linewidth - 2\tabcolsep) * \real{0.6667}}@{}}
\toprule\noalign{}
\begin{minipage}[b]{\linewidth}\raggedright
Term
\end{minipage} & \begin{minipage}[b]{\linewidth}\raggedright
Definition
\end{minipage} \\
\midrule\noalign{}
\endhead
\bottomrule\noalign{}
\endlastfoot
\textbf{Cybercrime} & Criminal activities carried out using
computers/internet \\
\textbf{Cybercriminal} & Individual who commits crimes using digital
technology \\
\end{longtable}
}

\textbf{Cybercriminal Types:}

\begin{itemize}
\tightlist
\item
  \textbf{Script Kiddies}: Use existing tools without deep knowledge
\item
  \textbf{Hacktivists}: Motivated by political/social causes
\item
  \textbf{Organized Crime}: Professional criminal groups
\item
  \textbf{State-sponsored}: Government-backed attackers
\end{itemize}

\end{solutionbox}
\begin{mnemonicbox}
``SSHT'' - Script kiddies, State-sponsored,
Hacktivists, Teams organized

\end{mnemonicbox}
\begin{center}\rule{0.5\linewidth}{0.5pt}\end{center}

\subsection*{Question 3(b) [4 marks]}\label{q3b}

\textbf{Describe cyber stalking and cyber bullying in detail.}

\begin{solutionbox}

\textbf{Comparison Table:}

{\def\LTcaptype{none} % do not increment counter
\begin{longtable}[]{@{}
  >{\raggedright\arraybackslash}p{(\linewidth - 4\tabcolsep) * \real{0.2000}}
  >{\raggedright\arraybackslash}p{(\linewidth - 4\tabcolsep) * \real{0.4000}}
  >{\raggedright\arraybackslash}p{(\linewidth - 4\tabcolsep) * \real{0.4000}}@{}}
\toprule\noalign{}
\begin{minipage}[b]{\linewidth}\raggedright
Aspect
\end{minipage} & \begin{minipage}[b]{\linewidth}\raggedright
Cyber Stalking
\end{minipage} & \begin{minipage}[b]{\linewidth}\raggedright
Cyber Bullying
\end{minipage} \\
\midrule\noalign{}
\endhead
\bottomrule\noalign{}
\endlastfoot
\textbf{Target} & Specific individual (often adult) & Often
minors/peers \\
\textbf{Duration} & Long-term harassment & Can be one-time or
repeated \\
\textbf{Intent} & Intimidation, control & Humiliation, social
exclusion \\
\textbf{Methods} & Monitoring, threatening messages & Social media
harassment, spreading rumors \\
\end{longtable}
}

\textbf{Common Characteristics:}

\begin{itemize}
\tightlist
\item
  \textbf{Digital Platforms}: Social media, email, messaging apps
\item
  \textbf{Anonymity}: Perpetrators often hide identity
\item
  \textbf{Psychological Impact}: Causes emotional distress
\item
  \textbf{Legal Consequences}: Violates cyber laws
\end{itemize}

\end{solutionbox}
\begin{mnemonicbox}
``STAL-BULL DPAL'' - Digital platforms, Psychological
impact, Anonymity, Legal issues

\end{mnemonicbox}
\begin{center}\rule{0.5\linewidth}{0.5pt}\end{center}

\subsection*{Question 3(c) [7 marks]}\label{q3c}

\textbf{Explain Property based classification in cybercrime.}

\begin{solutionbox}

\textbf{Property-Based Cybercrime Classification:}

{\def\LTcaptype{none} % do not increment counter
\begin{longtable}[]{@{}
  >{\raggedright\arraybackslash}p{(\linewidth - 4\tabcolsep) * \real{0.3529}}
  >{\raggedright\arraybackslash}p{(\linewidth - 4\tabcolsep) * \real{0.3824}}
  >{\raggedright\arraybackslash}p{(\linewidth - 4\tabcolsep) * \real{0.2647}}@{}}
\toprule\noalign{}
\begin{minipage}[b]{\linewidth}\raggedright
Crime Type
\end{minipage} & \begin{minipage}[b]{\linewidth}\raggedright
Description
\end{minipage} & \begin{minipage}[b]{\linewidth}\raggedright
Example
\end{minipage} \\
\midrule\noalign{}
\endhead
\bottomrule\noalign{}
\endlastfoot
\textbf{Credit Card Fraud} & Unauthorized use of payment cards & Online
shopping with stolen cards \\
\textbf{Software Piracy} & Illegal copying/distribution of software &
Downloading copyrighted software \\
\textbf{Copyright Infringement} & Violating intellectual property rights
& Sharing movies/music illegally \\
\textbf{Trademark Violations} & Misusing registered trademarks &
Creating fake brand websites \\
\end{longtable}
}

\textbf{Impact Assessment:}

\begin{itemize}
\tightlist
\item
  \textbf{Financial Loss}: Direct monetary damage
\item
  \textbf{Intellectual Property Theft}: Loss of competitive advantage
\item
  \textbf{Brand Reputation}: Damage to company image
\item
  \textbf{Legal Costs}: Expenses for prosecution/defense
\end{itemize}

\textbf{Prevention Measures:}

\begin{itemize}
\tightlist
\item
  \textbf{Digital Rights Management}: Protect copyrighted content
\item
  \textbf{Secure Payment Systems}: Implement fraud detection
\item
  \textbf{Legal Enforcement}: Prosecute violators
\item
  \textbf{Public Awareness}: Educate about legitimate software
\end{itemize}

\end{solutionbox}
\begin{mnemonicbox}
``CSCT-FILP'' - Credit/Software/Copyright/Trademark,
Financial/Intellectual/Legal/Public

\end{mnemonicbox}
\begin{center}\rule{0.5\linewidth}{0.5pt}\end{center}

\subsection*{Question 3(a) OR [3
marks]}\label{q3a}

\textbf{Explain Data diddling.}

\begin{solutionbox}

\textbf{Data Diddling Definition:} Unauthorized alteration of data
before/during input into computer system.

\textbf{Characteristics Table:}

{\def\LTcaptype{none} % do not increment counter
\begin{longtable}[]{@{}ll@{}}
\toprule\noalign{}
Aspect & Details \\
\midrule\noalign{}
\endhead
\bottomrule\noalign{}
\endlastfoot
\textbf{Method} & Changing data values slightly \\
\textbf{Detection} & Very difficult to detect \\
\textbf{Target} & Financial/sensitive data \\
\textbf{Impact} & Cumulative significant loss \\
\end{longtable}
}

\end{solutionbox}
\begin{mnemonicbox}
``DIDDL'' - Data alteration, Input manipulation,
Difficult detection, Dollar losses

\end{mnemonicbox}
\begin{center}\rule{0.5\linewidth}{0.5pt}\end{center}

\subsection*{Question 3(b) OR [4
marks]}\label{q3b}

\textbf{Explain cyber spying and cyber terrorism.}

\begin{solutionbox}

\textbf{Comparison Table:}

{\def\LTcaptype{none} % do not increment counter
\begin{longtable}[]{@{}lll@{}}
\toprule\noalign{}
Aspect & Cyber Spying & Cyber Terrorism \\
\midrule\noalign{}
\endhead
\bottomrule\noalign{}
\endlastfoot
\textbf{Purpose} & Intelligence gathering & Cause fear/disruption \\
\textbf{Targets} & Government, corporations & Critical infrastructure \\
\textbf{Methods} & Stealth, long-term infiltration & Destructive
attacks \\
\textbf{Impact} & Information theft & Physical/economic damage \\
\end{longtable}
}

\textbf{Key Characteristics:}

\begin{itemize}
\tightlist
\item
  \textbf{Cyber Spying}: State-sponsored, corporate espionage
\item
  \textbf{Cyber Terrorism}: Ideologically motivated, mass disruption
\item
  \textbf{Common Tools}: Malware, social engineering, zero-day exploits
\end{itemize}

\end{solutionbox}
\begin{mnemonicbox}
``SPY-TER IGSD'' -
Intelligence/Government/Stealth/Disruption, Terror/Economic/Rapid/Damage

\end{mnemonicbox}
\begin{center}\rule{0.5\linewidth}{0.5pt}\end{center}

\subsection*{Question 3(c) OR [7
marks]}\label{q3c}

\textbf{Explain article section 65 and section 66 of cyber law.}

\begin{solutionbox}

\textbf{IT Act 2008 Sections:}

{\def\LTcaptype{none} % do not increment counter
\begin{longtable}[]{@{}
  >{\raggedright\arraybackslash}p{(\linewidth - 4\tabcolsep) * \real{0.3000}}
  >{\raggedright\arraybackslash}p{(\linewidth - 4\tabcolsep) * \real{0.3000}}
  >{\raggedright\arraybackslash}p{(\linewidth - 4\tabcolsep) * \real{0.4000}}@{}}
\toprule\noalign{}
\begin{minipage}[b]{\linewidth}\raggedright
Section
\end{minipage} & \begin{minipage}[b]{\linewidth}\raggedright
Offense
\end{minipage} & \begin{minipage}[b]{\linewidth}\raggedright
Punishment
\end{minipage} \\
\midrule\noalign{}
\endhead
\bottomrule\noalign{}
\endlastfoot
\textbf{Section 65} & Computer source code tampering & Up to 3 years
imprisonment or fine up to ₹2 lakh \\
\textbf{Section 66} & Computer-related offenses & Up to 3 years
imprisonment or fine up to ₹5 lakh \\
\end{longtable}
}

\textbf{Section 65 Details:}

\begin{itemize}
\tightlist
\item
  \textbf{Scope}: Knowingly/intentionally concealing, destroying,
  altering computer source code
\item
  \textbf{Intent}: When computer source code required to be
  kept/maintained by law
\item
  \textbf{Application}: Protects integrity of essential software systems
\end{itemize}

\textbf{Section 66 Details:}

\begin{itemize}
\tightlist
\item
  \textbf{Computer Hacking}: Unauthorized access to computer systems
\item
  \textbf{Data Theft}: Downloading, copying, extracting data dishonestly
\item
  \textbf{System Damage}: Destroying, deleting, altering information
\item
  \textbf{Service Disruption}: Denying access to authorized persons
\end{itemize}

\end{solutionbox}
\begin{mnemonicbox}
``65-66 CDHD'' - Code tampering, Damage, Hacking,
Data theft

\end{mnemonicbox}
\begin{center}\rule{0.5\linewidth}{0.5pt}\end{center}

\subsection*{Question 4(a) [3 marks]}\label{q4a}

\textbf{What is Hacking? List out types of Hackers.}

\begin{solutionbox}

\textbf{Hacking Definition:} Unauthorized access to computer
systems/networks to exploit vulnerabilities.

\textbf{Hacker Types Table:}

{\def\LTcaptype{none} % do not increment counter
\begin{longtable}[]{@{}lll@{}}
\toprule\noalign{}
Type & Motivation & Activity \\
\midrule\noalign{}
\endhead
\bottomrule\noalign{}
\endlastfoot
\textbf{White Hat} & Security improvement & Ethical penetration
testing \\
\textbf{Black Hat} & Malicious intent & Criminal activities \\
\textbf{Grey Hat} & Mixed motives & Unauthorized but non-malicious \\
\textbf{Script Kiddie} & Recognition & Using existing tools \\
\end{longtable}
}

\end{solutionbox}
\begin{mnemonicbox}
``WBGS Hat'' - White, Black, Grey, Script kiddie

\end{mnemonicbox}
\begin{center}\rule{0.5\linewidth}{0.5pt}\end{center}

\subsection*{Question 4(b) [4 marks]}\label{q4b}

\textbf{Explain Vulnerability and 0-Day terminology of Hacking.}

\begin{solutionbox}

\textbf{Terminology Table:}

{\def\LTcaptype{none} % do not increment counter
\begin{longtable}[]{@{}
  >{\raggedright\arraybackslash}p{(\linewidth - 4\tabcolsep) * \real{0.2000}}
  >{\raggedright\arraybackslash}p{(\linewidth - 4\tabcolsep) * \real{0.4000}}
  >{\raggedright\arraybackslash}p{(\linewidth - 4\tabcolsep) * \real{0.4000}}@{}}
\toprule\noalign{}
\begin{minipage}[b]{\linewidth}\raggedright
Term
\end{minipage} & \begin{minipage}[b]{\linewidth}\raggedright
Definition
\end{minipage} & \begin{minipage}[b]{\linewidth}\raggedright
Risk Level
\end{minipage} \\
\midrule\noalign{}
\endhead
\bottomrule\noalign{}
\endlastfoot
\textbf{Vulnerability} & Security weakness that can be exploited &
Medium-High \\
\textbf{0-Day Vulnerability} & Unknown security flaw & Critical \\
\textbf{0-Day Exploit} & Attack code for 0-day vulnerability &
Critical \\
\textbf{0-Day Attack} & Active exploitation of 0-day & Critical \\
\end{longtable}
}

\textbf{Key Characteristics:}

\begin{itemize}
\tightlist
\item
  \textbf{Unknown to Vendors}: No patches available
\item
  \textbf{High Value}: Sold in dark markets
\item
  \textbf{Stealthy}: Difficult to detect
\item
  \textbf{Time-Critical}: Value decreases after disclosure
\end{itemize}

\end{solutionbox}
\begin{mnemonicbox}
``0-Day UHST'' - Unknown, High-value, Stealthy,
Time-critical

\end{mnemonicbox}
\begin{center}\rule{0.5\linewidth}{0.5pt}\end{center}

\subsection*{Question 4(c) [7 marks]}\label{q4c}

\textbf{Explain Five Steps of Hacking.}

\begin{solutionbox}

\textbf{Hacking Process Flow:}

\begin{verbatim}
flowchart LR
    A[Information Gathering] {-{-} B[Scanning]}
    B {-{-} C[Gaining Access]}
    C {-{-} D[Maintaining Access]}
    D {-{-} E[Covering Tracks]}
\end{verbatim}

\textbf{Five Steps Detailed:}

{\def\LTcaptype{none} % do not increment counter
\begin{longtable}[]{@{}
  >{\raggedright\arraybackslash}p{(\linewidth - 4\tabcolsep) * \real{0.1818}}
  >{\raggedright\arraybackslash}p{(\linewidth - 4\tabcolsep) * \real{0.2727}}
  >{\raggedright\arraybackslash}p{(\linewidth - 4\tabcolsep) * \real{0.5455}}@{}}
\toprule\noalign{}
\begin{minipage}[b]{\linewidth}\raggedright
Step
\end{minipage} & \begin{minipage}[b]{\linewidth}\raggedright
Purpose
\end{minipage} & \begin{minipage}[b]{\linewidth}\raggedright
Tools/Techniques
\end{minipage} \\
\midrule\noalign{}
\endhead
\bottomrule\noalign{}
\endlastfoot
\textbf{1. Information Gathering} & Collect target information & OSINT,
Social engineering \\
\textbf{2. Scanning} & Identify live systems, ports & Nmap, Port
scanners \\
\textbf{3. Gaining Access} & Exploit vulnerabilities & Metasploit,
Custom exploits \\
\textbf{4. Maintaining Access} & Establish persistent presence &
Backdoors, Rootkits \\
\textbf{5. Covering Tracks} & Remove evidence & Log deletion, File
cleanup \\
\end{longtable}
}

\textbf{Each Step Details:}

\begin{itemize}
\tightlist
\item
  \textbf{Information Gathering}: Passive/Active reconnaissance
\item
  \textbf{Scanning}: Network mapping, vulnerability assessment
\item
  \textbf{Gaining Access}: Password attacks, buffer overflows
\item
  \textbf{Maintaining Access}: Privilege escalation, backdoor
  installation
\item
  \textbf{Covering Tracks}: Anti-forensics techniques
\end{itemize}

\end{solutionbox}
\begin{mnemonicbox}
``ISGMC'' - Information, Scanning, Gaining,
Maintaining, Covering

\end{mnemonicbox}
\begin{center}\rule{0.5\linewidth}{0.5pt}\end{center}

\subsection*{Question 4(a) OR [3
marks]}\label{q4a}

\textbf{Explain any three basic commands of kali Linux with suitable
example.}

\begin{solutionbox}

\textbf{Kali Linux Commands Table:}

{\def\LTcaptype{none} % do not increment counter
\begin{longtable}[]{@{}
  >{\raggedright\arraybackslash}p{(\linewidth - 4\tabcolsep) * \real{0.3333}}
  >{\raggedright\arraybackslash}p{(\linewidth - 4\tabcolsep) * \real{0.3333}}
  >{\raggedright\arraybackslash}p{(\linewidth - 4\tabcolsep) * \real{0.3333}}@{}}
\toprule\noalign{}
\begin{minipage}[b]{\linewidth}\raggedright
Command
\end{minipage} & \begin{minipage}[b]{\linewidth}\raggedright
Purpose
\end{minipage} & \begin{minipage}[b]{\linewidth}\raggedright
Example
\end{minipage} \\
\midrule\noalign{}
\endhead
\bottomrule\noalign{}
\endlastfoot
\textbf{nmap} & Network scanning & \texttt{nmap\ -sS\ 192.168.1.1} \\
\textbf{netcat} & Network utility & \texttt{nc\ -l\ -p\ 4444} \\
\textbf{john} & Password cracking &
\texttt{john\ -\/-wordlist=passwords.txt\ hashes.txt} \\
\end{longtable}
}

\textbf{Command Details:}

\begin{itemize}
\tightlist
\item
  \textbf{nmap}: Stealth SYN scan on target IP
\item
  \textbf{netcat}: Listen on port 4444 for connections
\item
  \textbf{john}: Dictionary attack on password hashes
\end{itemize}

\end{solutionbox}
\begin{mnemonicbox}
``NNJ'' - Nmap scans, Netcat listens, John cracks

\end{mnemonicbox}
\begin{center}\rule{0.5\linewidth}{0.5pt}\end{center}

\subsection*{Question 4(b) OR [4
marks]}\label{q4b}

\textbf{Describe Session Hijacking in detail.}

\begin{solutionbox}

\textbf{Session Hijacking Process:}

\begin{verbatim}
sequenceDiagram
    participant U as User
    participant A as Attacker
    participant S as Server
    U{-S: 1. Login \& Get Session ID}
    A{-A: 2. Capture Session ID}
    A{-S: 3. Use Stolen Session ID}
    S{-A: 4. Grant Access}
\end{verbatim}

\textbf{Types and Methods:}

\begin{itemize}
\tightlist
\item
  \textbf{Active Hijacking}: Attacker actively participates
\item
  \textbf{Passive Hijacking}: Monitor and capture sessions
\item
  \textbf{Network Level}: IP spoofing, ARP poisoning
\item
  \textbf{Application Level}: Session ID prediction, XSS
\end{itemize}

\textbf{Prevention Measures:}

\begin{itemize}
\tightlist
\item
  \textbf{HTTPS}: Encrypt session data
\item
  \textbf{Session Timeouts}: Limit session duration
\item
  \textbf{IP Binding}: Tie sessions to IP addresses
\item
  \textbf{Strong Session IDs}: Use unpredictable tokens
\end{itemize}

\end{solutionbox}
\begin{mnemonicbox}
``APNA-HSIS'' - Active/Passive/Network/Application,
HTTPS/Strong/IP/Session

\end{mnemonicbox}
\begin{center}\rule{0.5\linewidth}{0.5pt}\end{center}

\subsection*{Question 4(c) OR [7
marks]}\label{q4c}

\textbf{Explain Remote Administration Tools.}

\begin{solutionbox}

\textbf{RAT Definition:} Software allowing remote control of computer
systems, often used maliciously.

\textbf{RAT Functionality Table:}

{\def\LTcaptype{none} % do not increment counter
\begin{longtable}[]{@{}lll@{}}
\toprule\noalign{}
Function & Description & Risk Level \\
\midrule\noalign{}
\endhead
\bottomrule\noalign{}
\endlastfoot
\textbf{Screen Capture} & Take screenshots remotely & Medium \\
\textbf{Keylogging} & Record keystrokes & High \\
\textbf{File Transfer} & Upload/download files & High \\
\textbf{Camera Access} & Activate webcam/microphone & Critical \\
\end{longtable}
}

\textbf{Legitimate vs Malicious Use:}

{\def\LTcaptype{none} % do not increment counter
\begin{longtable}[]{@{}
  >{\raggedright\arraybackslash}p{(\linewidth - 4\tabcolsep) * \real{0.2581}}
  >{\raggedright\arraybackslash}p{(\linewidth - 4\tabcolsep) * \real{0.3871}}
  >{\raggedright\arraybackslash}p{(\linewidth - 4\tabcolsep) * \real{0.3548}}@{}}
\toprule\noalign{}
\begin{minipage}[b]{\linewidth}\raggedright
Aspect
\end{minipage} & \begin{minipage}[b]{\linewidth}\raggedright
Legitimate
\end{minipage} & \begin{minipage}[b]{\linewidth}\raggedright
Malicious
\end{minipage} \\
\midrule\noalign{}
\endhead
\bottomrule\noalign{}
\endlastfoot
\textbf{Purpose} & IT support, administration & Espionage, theft \\
\textbf{Consent} & User aware and consenting & Installed without
knowledge \\
\textbf{Access} & Authorized personnel only & Unauthorized attackers \\
\end{longtable}
}

\textbf{Detection and Prevention:}

\begin{itemize}
\tightlist
\item
  \textbf{Antivirus}: Detect known RAT signatures
\item
  \textbf{Network Monitoring}: Unusual outbound connections
\item
  \textbf{User Education}: Avoid suspicious downloads
\item
  \textbf{Firewall Rules}: Block unauthorized connections
\end{itemize}

\textbf{Common RATs:}

\begin{itemize}
\tightlist
\item
  \textbf{TeamViewer}: Legitimate remote access
\item
  \textbf{DarkComet}: Malicious RAT
\item
  \textbf{Poison Ivy}: Advanced persistent threat tool
\end{itemize}

\end{solutionbox}
\begin{mnemonicbox}
``RAT SKFC-ANUM'' - Screen/Key/File/Camera,
Antivirus/Network/User/Monitoring

\end{mnemonicbox}
\begin{center}\rule{0.5\linewidth}{0.5pt}\end{center}

\subsection*{Question 5(a) [3 marks]}\label{q5a}

\textbf{Explain Mobile forensics.}

\begin{solutionbox}

\textbf{Mobile Forensics Definition:} Process of recovering digital
evidence from mobile devices using scientifically accepted methods.

\textbf{Key Aspects Table:}

{\def\LTcaptype{none} % do not increment counter
\begin{longtable}[]{@{}ll@{}}
\toprule\noalign{}
Aspect & Description \\
\midrule\noalign{}
\endhead
\bottomrule\noalign{}
\endlastfoot
\textbf{Data Types} & Call logs, SMS, photos, app data \\
\textbf{Challenges} & Encryption, anti-forensics, variety of OS \\
\textbf{Tools} & Cellebrite, XRY, Oxygen Suite \\
\textbf{Legal} & Chain of custody, court admissibility \\
\end{longtable}
}

\end{solutionbox}
\begin{mnemonicbox}
``DCTL'' - Data types, Challenges, Tools, Legal
requirements

\end{mnemonicbox}
\begin{center}\rule{0.5\linewidth}{0.5pt}\end{center}

\subsection*{Question 5(b) [4 marks]}\label{q5b}

\textbf{What is Digital forensics? Write down advantages of Digital
forensics.}

\begin{solutionbox}

\textbf{Digital Forensics Definition:} Scientific examination of digital
devices to recover and analyze evidence for legal proceedings.

\textbf{Advantages Table:}

{\def\LTcaptype{none} % do not increment counter
\begin{longtable}[]{@{}ll@{}}
\toprule\noalign{}
Advantage & Description \\
\midrule\noalign{}
\endhead
\bottomrule\noalign{}
\endlastfoot
\textbf{Evidence Recovery} & Retrieve deleted/hidden data \\
\textbf{Crime Solving} & Provide crucial evidence for cases \\
\textbf{Cost Effective} & Cheaper than traditional investigation \\
\textbf{Accurate Results} & Scientific methods ensure reliability \\
\end{longtable}
}

\textbf{Additional Benefits:}

\begin{itemize}
\tightlist
\item
  \textbf{Time Efficient}: Faster than manual investigation
\item
  \textbf{Non-destructive}: Preserves original evidence
\item
  \textbf{Comprehensive}: Analyzes multiple data sources
\item
  \textbf{Court Acceptable}: Legally admissible evidence
\end{itemize}

\end{solutionbox}
\begin{mnemonicbox}
``ECCA-TNCA'' - Evidence/Crime/Cost/Accurate,
Time/Non-destructive/Comprehensive/Admissible

\end{mnemonicbox}
\begin{center}\rule{0.5\linewidth}{0.5pt}\end{center}

\subsection*{Question 5(c) [7 marks]}\label{q5c}

\textbf{Describe in detail Locard's Principle of exchange in Digital
Forensics.}

\begin{solutionbox}

\textbf{Locard's Principle:} ``Every contact leaves a trace'' - any
interaction between objects results in exchange of materials.

\textbf{Digital Application:}

\begin{center}
\textbf{Mermaid Diagram (Code)}
\begin{verbatim}
{Shaded}
{Highlighting}[]
graph TD
    A[User Action] {-{-}{} B[Digital Traces]}
    B {-{-}{} C[Log Files]}
    B {-{-}{} D[Registry Entries]}
    B {-{-}{} E[File Metadata]}
    B {-{-}{} F[Network Traffic]}
{Highlighting}
{Shaded}
\end{verbatim}
\end{center}

\textbf{Digital Traces Table:}

{\def\LTcaptype{none} % do not increment counter
\begin{longtable}[]{@{}lll@{}}
\toprule\noalign{}
Action & Digital Trace & Location \\
\midrule\noalign{}
\endhead
\bottomrule\noalign{}
\endlastfoot
\textbf{File Access} & Access timestamps & File system metadata \\
\textbf{Web Browsing} & Browser history & Browser databases \\
\textbf{Email Sending} & Email headers & Mail server logs \\
\textbf{USB Connection} & Device registry & Windows registry \\
\end{longtable}
}

\textbf{Forensic Implications:}

\begin{itemize}
\tightlist
\item
  \textbf{Persistence}: Digital traces often persist longer
\item
  \textbf{Accuracy}: Precise timestamps and data
\item
  \textbf{Volume}: Large amounts of trace evidence
\item
  \textbf{Recovery}: Deleted data can be recovered
\end{itemize}

\textbf{Evidence Types:}

\begin{itemize}
\tightlist
\item
  \textbf{Temporal}: When actions occurred
\item
  \textbf{Spatial}: Where actions took place
\item
  \textbf{Relational}: Connections between entities
\item
  \textbf{Behavioral}: Patterns of user activity
\end{itemize}

\textbf{Applications:}

\begin{itemize}
\tightlist
\item
  \textbf{Criminal Cases}: Prove presence/actions
\item
  \textbf{Civil Litigation}: Business disputes
\item
  \textbf{Internal Investigations}: Employee misconduct
\item
  \textbf{Incident Response}: Security breach analysis
\end{itemize}

\end{solutionbox}
\begin{mnemonicbox}
``LOCARD PVAR-TREB'' -
Persistence/Volume/Accuracy/Recovery,
Temporal/Relational/Evidence/Behavioral

\end{mnemonicbox}
\begin{center}\rule{0.5\linewidth}{0.5pt}\end{center}

\subsection*{Question 5(a) OR [3
marks]}\label{q5a}

\textbf{Explain Network forensics.}

\begin{solutionbox}

\textbf{Network Forensics Definition:} Monitoring and analysis of
network traffic to gather information and evidence.

\textbf{Key Components Table:}

{\def\LTcaptype{none} % do not increment counter
\begin{longtable}[]{@{}ll@{}}
\toprule\noalign{}
Component & Function \\
\midrule\noalign{}
\endhead
\bottomrule\noalign{}
\endlastfoot
\textbf{Packet Capture} & Record network traffic \\
\textbf{Traffic Analysis} & Examine communication patterns \\
\textbf{Protocol Analysis} & Decode network protocols \\
\textbf{Timeline Creation} & Establish sequence of events \\
\end{longtable}
}

\end{solutionbox}
\begin{mnemonicbox}
``PTTP'' - Packet capture, Traffic analysis,
Timeline, Protocol analysis

\end{mnemonicbox}
\begin{center}\rule{0.5\linewidth}{0.5pt}\end{center}

\subsection*{Question 5(b) OR [4
marks]}\label{q5b}

\textbf{Explain why CCTV plays an important role as evidence in digital
forensics investigations.}

\begin{solutionbox}

\textbf{CCTV Evidence Value:}

{\def\LTcaptype{none} % do not increment counter
\begin{longtable}[]{@{}ll@{}}
\toprule\noalign{}
Aspect & Importance \\
\midrule\noalign{}
\endhead
\bottomrule\noalign{}
\endlastfoot
\textbf{Visual Proof} & Direct evidence of events \\
\textbf{Timestamp} & Precise time correlation \\
\textbf{Location Verification} & Proves presence at scene \\
\textbf{Behavior Analysis} & Shows actions and intent \\
\end{longtable}
}

\textbf{Digital Forensics Integration:}

\begin{itemize}
\tightlist
\item
  \textbf{Metadata Extraction}: Camera settings, timestamps
\item
  \textbf{Video Enhancement}: Improve image quality
\item
  \textbf{Format Analysis}: Understand compression artifacts
\item
  \textbf{Authentication}: Verify video integrity
\end{itemize}

\textbf{Legal Considerations:}

\begin{itemize}
\tightlist
\item
  \textbf{Chain of Custody}: Maintain evidence integrity
\item
  \textbf{Court Admissibility}: Follow legal procedures
\item
  \textbf{Privacy Rights}: Respect surveillance laws
\item
  \textbf{Technical Validation}: Prove authenticity
\end{itemize}

\end{solutionbox}
\begin{mnemonicbox}
``VTLB-MFAC'' - Visual/Timestamp/Location/Behavior,
Metadata/Format/Authentication/Chain

\end{mnemonicbox}
\begin{center}\rule{0.5\linewidth}{0.5pt}\end{center}

\subsection*{Question 5(c) OR [7
marks]}\label{q5c}

\textbf{Explain phases of Digital forensic investigation.}

\begin{solutionbox}

\textbf{Digital Forensic Investigation Phases:}

\begin{verbatim}
flowchart LR
    A[Identification] {-{-} B[Preservation]}
    B {-{-} C[Analysis]}
    C {-{-} D[Documentation]}
    D {-{-} E[Presentation]}
\end{verbatim}

\textbf{Phase Details Table:}

{\def\LTcaptype{none} % do not increment counter
\begin{longtable}[]{@{}
  >{\raggedright\arraybackslash}p{(\linewidth - 4\tabcolsep) * \real{0.2059}}
  >{\raggedright\arraybackslash}p{(\linewidth - 4\tabcolsep) * \real{0.3529}}
  >{\raggedright\arraybackslash}p{(\linewidth - 4\tabcolsep) * \real{0.4412}}@{}}
\toprule\noalign{}
\begin{minipage}[b]{\linewidth}\raggedright
Phase
\end{minipage} & \begin{minipage}[b]{\linewidth}\raggedright
Activities
\end{minipage} & \begin{minipage}[b]{\linewidth}\raggedright
Tools/Methods
\end{minipage} \\
\midrule\noalign{}
\endhead
\bottomrule\noalign{}
\endlastfoot
\textbf{Identification} & Locate potential evidence sources & Initial
assessment, Scene survey \\
\textbf{Preservation} & Secure evidence without alteration & Imaging,
Hash verification \\
\textbf{Analysis} & Examine evidence for relevant data & Forensic
software, Manual review \\
\textbf{Documentation} & Record findings and procedures & Reports,
Screenshots, Logs \\
\textbf{Presentation} & Present findings to stakeholders & Court
testimony, Expert reports \\
\end{longtable}
}

\textbf{Detailed Activities:}

\textbf{1. Identification Phase:}

\begin{itemize}
\tightlist
\item
  \textbf{Evidence Sources}: Computers, phones, servers, network logs
\item
  \textbf{Scope Definition}: Determine investigation boundaries
\item
  \textbf{Legal Authorization}: Obtain warrants/permissions
\item
  \textbf{Initial Photography}: Document scene condition
\end{itemize}

\textbf{2. Preservation Phase:}

\begin{itemize}
\tightlist
\item
  \textbf{Bit-by-bit Imaging}: Create exact copies
\item
  \textbf{Hash Calculation}: Verify integrity (MD5, SHA)
\item
  \textbf{Chain of Custody}: Maintain evidence trail
\item
  \textbf{Write Protection}: Prevent evidence modification
\end{itemize}

\textbf{3. Analysis Phase:}

\begin{itemize}
\tightlist
\item
  \textbf{Data Recovery}: Retrieve deleted files
\item
  \textbf{Keyword Searching}: Find relevant information
\item
  \textbf{Timeline Analysis}: Reconstruct events
\item
  \textbf{Pattern Recognition}: Identify suspicious activities
\end{itemize}

\textbf{4. Documentation Phase:}

\begin{itemize}
\tightlist
\item
  \textbf{Methodology Recording}: Document procedures used
\item
  \textbf{Evidence Cataloging}: List all findings
\item
  \textbf{Screenshot Capture}: Visual evidence documentation
\item
  \textbf{Report Preparation}: Comprehensive investigation report
\end{itemize}

\textbf{5. Presentation Phase:}

\begin{itemize}
\tightlist
\item
  \textbf{Expert Testimony}: Court appearances
\item
  \textbf{Visual Aids}: Charts, diagrams, demonstrations
\item
  \textbf{Technical Translation}: Explain complex concepts
\item
  \textbf{Cross-examination}: Answer defense questions
\end{itemize}

\textbf{Quality Assurance:}

\begin{itemize}
\tightlist
\item
  \textbf{Peer Review}: Second examiner verification
\item
  \textbf{Tool Validation}: Ensure software accuracy
\item
  \textbf{Procedure Adherence}: Follow standard protocols
\item
  \textbf{Continuous Training}: Keep skills current
\end{itemize}

\textbf{Legal Considerations:}

\begin{itemize}
\tightlist
\item
  \textbf{Admissibility Rules}: Meet court standards
\item
  \textbf{Privacy Protection}: Respect individual rights
\item
  \textbf{International Law}: Cross-border investigations
\item
  \textbf{Professional Ethics}: Maintain objectivity
\end{itemize}

\end{solutionbox}
\begin{mnemonicbox}
``IPADP-ESLR-HTVC-MSCR-ETVI'' -
Identification/Preservation/Analysis/Documentation/Presentation with
detailed sub-activities

\end{mnemonicbox}

\end{document}
