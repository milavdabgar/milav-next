\documentclass[10pt,a4paper]{article}

% content/resources/templates/preamble.tex
\usepackage[margin=0.6in]{geometry}
\author{Milav Dabgar}
\usepackage{amsmath,amssymb,amsthm}
\usepackage{booktabs}
\usepackage{multirow}
\usepackage{xcolor}
\usepackage{tcolorbox}
\tcbuselibrary{breakable,skins}
\usepackage[colorlinks=true,linkcolor=blue]{hyperref}
\usepackage{titlesec}
\usepackage{enumitem}
\usepackage{tikz}
\usepackage{pgfplots}
\usepackage{circuitikz}
\usepackage[version=4]{mhchem}
\usepackage{longtable}
\usepackage{array}
\usepackage{float}
\usepackage{caption}
\usepackage{listings}

\lstset{
  basicstyle=\small\ttfamily,
  breaklines=true,
  breakatwhitespace=false,
  postbreak=\mbox{\textcolor{red}{$\hookrightarrow$}\space},
  float=false,
  numbers=left,
  numberstyle=\tiny\color{gray},
  numbersep=10pt,
  xleftmargin=2em,
  keywordstyle=\color{blue},
  commentstyle=\color{green!60!black},
  stringstyle=\color{purple},
  backgroundcolor=\color{gray!5},
  showstringspaces=false,
  tabsize=2,
  captionpos=b,
  keepspaces=true,
  columns=flexible
}

\pgfplotsset{compat=1.18}
\usetikzlibrary{shapes,arrows,positioning,calc,patterns,decorations.pathmorphing,decorations.markings,arrows.meta}

% Color scheme
\definecolor{headcolor}{RGB}{0,102,204}
\definecolor{keycolor}{RGB}{220,20,60}
\definecolor{solutioncolor}{RGB}{34,139,34}
\definecolor{mnemoniccolor}{RGB}{148,0,211}
\definecolor{codecolor}{RGB}{0,0,100}

% Spacing
\setlength{\parskip}{3pt}
\setlist[itemize]{nosep}
\setlist[enumerate]{nosep}

% Title formatting
\titleformat{\section}{\Large\bfseries\color{headcolor}}{\thesection}{1em}{}
\titleformat{\subsection}{\large\bfseries\color{headcolor}}{\thesubsection}{1em}{}

% Pandoc tightlist compatibility
\providecommand{\tightlist}{%
  \setlength{\itemsep}{0pt}\setlength{\parskip}{0pt}}

% Pandoc longtable compatibility
\newcounter{none}
\def\thenone{}


% content/resources/templates/english-boxes.tex
% This file is currently empty - it exists to maintain consistency with the import structure.
% Add custom environments here if needed in the future.


\begin{document}

\begin{center}
{\Huge\bfseries\color{headcolor} Subject Name Solutions}\\[5pt]
{\LARGE 4361601 -- Winter 2024}\\[3pt]
{\large Semester 1 Study Material}\\[3pt]
{\normalsize\textit{Detailed Solutions and Explanations}}
\end{center}

\vspace{10pt}

\subsection*{Question 1(a) [3 marks]}\label{q1a}

\begin{solutionbox}

\textbf{i) What is Information Security?}

\end{solutionbox}
\begin{solutionbox}
Information Security protects digital data from
unauthorized access, use, disclosure, disruption, modification, or
destruction.

\textbf{Key Components:}

\begin{itemize}
\tightlist
\item
  \textbf{Confidentiality}: Data accessible only to authorized users
\item
  \textbf{Integrity}: Data remains accurate and complete
\item
  \textbf{Availability}: Data accessible when needed
\end{itemize}

\end{solutionbox}
\begin{mnemonicbox}
``CIA keeps data safe''

\textbf{ii) List Types of hackers}

\end{mnemonicbox}
\begin{solutionbox}

{\def\LTcaptype{none} % do not increment counter
\begin{longtable}[]{@{}lll@{}}
\toprule\noalign{}
Hacker Type & Description & Intent \\
\midrule\noalign{}
\endhead
\bottomrule\noalign{}
\endlastfoot
White Hat & Ethical hackers & Good intentions \\
Black Hat & Malicious hackers & Criminal activities \\
Gray Hat & Mix of both & Neutral motives \\
Script Kiddies & Use existing tools & Limited skills \\
\end{longtable}
}

\textbf{iii) What is the default username and password for Kali Linux?}

\end{solutionbox}
\begin{solutionbox}

\begin{itemize}
\tightlist
\item
  \textbf{Username}: kali
\item
  \textbf{Password}: kali (changed from root/toor in newer versions)
\end{itemize}

\end{solutionbox}
\begin{center}\rule{0.5\linewidth}{0.5pt}\end{center}

\subsection*{Question 1(b) [4 marks]}\label{q1b}

\textbf{Describe CIA triad with example.}

\begin{solutionbox}
CIA Triad is the foundation of information security
with three core principles:

{\def\LTcaptype{none} % do not increment counter
\begin{longtable}[]{@{}
  >{\raggedright\arraybackslash}p{(\linewidth - 4\tabcolsep) * \real{0.3438}}
  >{\raggedright\arraybackslash}p{(\linewidth - 4\tabcolsep) * \real{0.3750}}
  >{\raggedright\arraybackslash}p{(\linewidth - 4\tabcolsep) * \real{0.2812}}@{}}
\toprule\noalign{}
\begin{minipage}[b]{\linewidth}\raggedright
Principle
\end{minipage} & \begin{minipage}[b]{\linewidth}\raggedright
Definition
\end{minipage} & \begin{minipage}[b]{\linewidth}\raggedright
Example
\end{minipage} \\
\midrule\noalign{}
\endhead
\bottomrule\noalign{}
\endlastfoot
\textbf{Confidentiality} & Data accessible only to authorized users &
Password protection, encryption \\
\textbf{Integrity} & Data remains accurate and unmodified & Digital
signatures, checksums \\
\textbf{Availability} & Data accessible when needed & Backup systems,
redundancy \\
\end{longtable}
}

\textbf{Real-world Example}: Banking system maintains confidentiality
through login credentials, integrity through transaction verification,
and availability through 24/7 service.

\end{solutionbox}
\begin{mnemonicbox}
``CIA protects information like secret agents''

\end{mnemonicbox}
\begin{center}\rule{0.5\linewidth}{0.5pt}\end{center}

\subsection*{Question 1(c) [7 marks]}\label{q1c}

\textbf{Explain MD5 hashing algorithm}

\begin{solutionbox}
MD5 (Message Digest 5) is a cryptographic hash function
producing 128-bit hash values.

\textbf{MD5 Process Table:}

{\def\LTcaptype{none} % do not increment counter
\begin{longtable}[]{@{}lll@{}}
\toprule\noalign{}
Step & Process & Details \\
\midrule\noalign{}
\endhead
\bottomrule\noalign{}
\endlastfoot
1 & \textbf{Padding} & Add bits to make length ≡ 448 (mod 512) \\
2 & \textbf{Length Addition} & Append 64-bit length \\
3 & \textbf{Initialize} & Set four 32-bit variables \\
4 & \textbf{Processing} & Four rounds of operations \\
5 & \textbf{Output} & 128-bit hash value \\
\end{longtable}
}

\begin{verbatim}
flowchart LR
    A[Input Message] {-{-} B[Padding]}
    B {-{-} C[Append Length]}
    C {-{-} D[Initialize MD Buffer]}
    D {-{-} E[Process in 512{-}bit Blocks]}
    E {-{-} F[128{-}bit Hash Output]}
\end{verbatim}

\textbf{Key Features:}

\begin{itemize}
\tightlist
\item
  \textbf{Fixed Output}: Always 128 bits
\item
  \textbf{One-way}: Cannot reverse hash to original
\item
  \textbf{Collision Prone}: Vulnerable to attacks
\end{itemize}

\end{solutionbox}
\begin{mnemonicbox}
``MD5 Makes Data into 5-step hash''

\end{mnemonicbox}
\begin{center}\rule{0.5\linewidth}{0.5pt}\end{center}

\subsection*{Question 1(c) OR [7
marks]}\label{q1c}

\textbf{Explain SHA algorithm}

\begin{solutionbox}
SHA (Secure Hash Algorithm) is a family of
cryptographic hash functions designed by NSA.

\textbf{SHA Variants Comparison:}

{\def\LTcaptype{none} % do not increment counter
\begin{longtable}[]{@{}llll@{}}
\toprule\noalign{}
Version & Output Size & Block Size & Security Level \\
\midrule\noalign{}
\endhead
\bottomrule\noalign{}
\endlastfoot
SHA-1 & 160 bits & 512 bits & Deprecated \\
SHA-256 & 256 bits & 512 bits & Strong \\
SHA-512 & 512 bits & 1024 bits & Very Strong \\
\end{longtable}
}

\begin{verbatim}
flowchart LR
    A[Message] {-{-} B[Pre{-}processing]}
    B {-{-} C[Hash Computation]}
    C {-{-} D[Final Hash]}
    
    B {-{-} B1[Padding]}
    B {-{-} B2[Parsing]}
    
    C {-{-} C1[Initialize Hash Values]}
    C {-{-} C2[Process Message Blocks]}
    C {-{-} C3[Compute Intermediate Hash]}
\end{verbatim}

\textbf{SHA-256 Process:}

\begin{itemize}
\tightlist
\item
  \textbf{Preprocessing}: Padding and parsing message
\item
  \textbf{Hash Computation}: 64 rounds of operations
\item
  \textbf{Final Hash}: 256-bit output
\end{itemize}

\textbf{Advantages over MD5:}

\begin{itemize}
\tightlist
\item
  \textbf{Stronger Security}: Resistant to collision attacks
\item
  \textbf{Larger Output}: More bits for security
\item
  \textbf{Government Standard}: NIST approved
\end{itemize}

\end{solutionbox}
\begin{mnemonicbox}
``SHA Securely Hashes All data''

\end{mnemonicbox}
\begin{center}\rule{0.5\linewidth}{0.5pt}\end{center}

\subsection*{Question 2(a) [3 marks]}\label{q2a}

\textbf{What is virus? Explain Virus Life cycle.}

\begin{solutionbox}
Computer virus is malicious software that replicates by
inserting copies into other programs or files.

\textbf{Virus Life Cycle:}

\begin{center}
\textbf{Mermaid Diagram (Code)}
\begin{verbatim}
{Shaded}
{Highlighting}[]
graph LR
    A[Dormant Phase] {-{-}{} B[Propagation Phase]}
    B {-{-}{} C[Triggering Phase]}
    C {-{-}{} D[Execution Phase]}
    D {-{-}{} A}
{Highlighting}
{Shaded}
\end{verbatim}
\end{center}

\textbf{Phase Details:}

\begin{itemize}
\tightlist
\item
  \textbf{Dormant}: Virus remains inactive
\item
  \textbf{Propagation}: Copies itself to other systems
\item
  \textbf{Triggering}: Activated by specific conditions
\item
  \textbf{Execution}: Performs malicious activities
\end{itemize}

\end{solutionbox}
\begin{mnemonicbox}
``Viruses Dance, Propagate, Trigger, Execute''

\end{mnemonicbox}
\begin{center}\rule{0.5\linewidth}{0.5pt}\end{center}

\subsection*{Question 2(b) [4 marks]}\label{q2b}

\begin{solutionbox}

\textbf{i) Difference between Private key and Public Key cryptography}

\end{solutionbox}
\begin{solutionbox}

{\def\LTcaptype{none} % do not increment counter
\begin{longtable}[]{@{}lll@{}}
\toprule\noalign{}
Aspect & Private Key & Public Key \\
\midrule\noalign{}
\endhead
\bottomrule\noalign{}
\endlastfoot
\textbf{Keys} & Single shared key & Key pair (public/private) \\
\textbf{Speed} & Fast encryption & Slower encryption \\
\textbf{Key Distribution} & Difficult & Easy distribution \\
\textbf{Scalability} & Poor for large networks & Good scalability \\
\end{longtable}
}

\textbf{ii) Define database forensics and list different kind of
activities performed during database forensics.}

\end{solutionbox}
\begin{solutionbox}
Database forensics examines database systems to extract
digital evidence for legal proceedings.

\textbf{Activities Performed:}

\begin{itemize}
\tightlist
\item
  \textbf{Log Analysis}: Examining transaction logs
\item
  \textbf{Metadata Extraction}: Recovering database structure
\item
  \textbf{Deleted Data Recovery}: Retrieving removed records
\item
  \textbf{Timeline Analysis}: Tracking data modifications
\end{itemize}

\end{solutionbox}
\begin{center}\rule{0.5\linewidth}{0.5pt}\end{center}

\subsection*{Question 2(c) [7 marks]}\label{q2c}

\textbf{Explain proxy server in details and why we need it?}

\begin{solutionbox}
Proxy server acts as intermediary between client and
server, forwarding requests and responses.

\textbf{Proxy Server Architecture:}

\begin{verbatim}
sequenceDiagram
    participant C as Client
    participant P as Proxy Server
    participant S as Target Server
    
    C{-P: Request}
    P{-S: Forward Request}
    S{-P: Response}
    P{-C: Forward Response}
\end{verbatim}

\textbf{Types of Proxy Servers:}

{\def\LTcaptype{none} % do not increment counter
\begin{longtable}[]{@{}lll@{}}
\toprule\noalign{}
Type & Function & Use Case \\
\midrule\noalign{}
\endhead
\bottomrule\noalign{}
\endlastfoot
\textbf{Forward Proxy} & Client-side intermediary & Web filtering \\
\textbf{Reverse Proxy} & Server-side intermediary & Load balancing \\
\textbf{Transparent Proxy} & Invisible to client & Content caching \\
\end{longtable}
}

\textbf{Why We Need Proxy Servers:}

\begin{itemize}
\tightlist
\item
  \textbf{Security}: Hide client IP addresses
\item
  \textbf{Performance}: Cache frequently accessed content
\item
  \textbf{Control}: Filter and monitor traffic
\item
  \textbf{Anonymity}: Protect user privacy
\end{itemize}

\textbf{Benefits:}

\begin{itemize}
\tightlist
\item
  \textbf{Bandwidth Saving}: Caching reduces traffic
\item
  \textbf{Access Control}: Block unwanted sites
\item
  \textbf{Load Distribution}: Balance server requests
\end{itemize}

\end{solutionbox}
\begin{mnemonicbox}
``Proxy Protects Privacy and Performance''

\end{mnemonicbox}
\begin{center}\rule{0.5\linewidth}{0.5pt}\end{center}

\subsection*{Question 2(a) OR [3
marks]}\label{q2a}

\textbf{Define: Trojans, Rootkit, Backdoors, Keylogger}

\begin{solutionbox}

{\def\LTcaptype{none} % do not increment counter
\begin{longtable}[]{@{}ll@{}}
\toprule\noalign{}
Malware Type & Definition \\
\midrule\noalign{}
\endhead
\bottomrule\noalign{}
\endlastfoot
\textbf{Trojans} & Malicious software disguised as legitimate
programs \\
\textbf{Rootkit} & Software hiding presence of malware in system \\
\textbf{Backdoors} & Secret entry points bypassing normal
authentication \\
\textbf{Keylogger} & Software recording keystrokes to steal passwords \\
\end{longtable}
}

\end{solutionbox}
\begin{mnemonicbox}
``TRBK - Trojans, Rootkits, Backdoors Keep
attacking''

\end{mnemonicbox}
\begin{center}\rule{0.5\linewidth}{0.5pt}\end{center}

\subsection*{Question 2(b) OR [4
marks]}\label{q2b}

\begin{solutionbox}

\textbf{i) Write advantages and disadvantages of firewall.}

\end{solutionbox}
\begin{solutionbox}

{\def\LTcaptype{none} % do not increment counter
\begin{longtable}[]{@{}ll@{}}
\toprule\noalign{}
Advantages & Disadvantages \\
\midrule\noalign{}
\endhead
\bottomrule\noalign{}
\endlastfoot
\textbf{Network Protection} & \textbf{Performance Impact} \\
\textbf{Access Control} & \textbf{Configuration Complexity} \\
\textbf{Traffic Monitoring} & \textbf{Cannot Stop All Attacks} \\
\textbf{Log Generation} & \textbf{Maintenance Required} \\
\end{longtable}
}

\textbf{ii) List critical steps in preserving digital evidence.}

\end{solutionbox}
\begin{solutionbox}

\begin{itemize}
\tightlist
\item
  \textbf{Identification}: Locate potential evidence
\item
  \textbf{Documentation}: Record evidence details
\item
  \textbf{Collection}: Gather evidence safely
\item
  \textbf{Preservation}: Maintain evidence integrity
\item
  \textbf{Chain of Custody}: Track evidence handling
\end{itemize}

\end{solutionbox}
\begin{center}\rule{0.5\linewidth}{0.5pt}\end{center}

\subsection*{Question 2(c) OR [7
marks]}\label{q2c}

\textbf{Explain IP Security Architecture.}

\begin{solutionbox}
IPSec provides security services at network layer for
IP communications.

\textbf{IPSec Architecture Components:}

\begin{verbatim}
graph TB
    A[IPSec Architecture] {-{-} B[Security Protocols]}
    A {-{-} C[Security Associations]}
    A {-{-} D[Key Management]}
    
    B {-{-} B1[AH {-} Authentication Header]}
    B {-{-} B2[ESP {-} Encapsulating Security Payload]}
    
    C {-{-} C1[SAD {-} Security Association Database]}
    C {-{-} C2[SPD {-} Security Policy Database]}
    
    D {-{-} D1[IKE {-} Internet Key Exchange]}
\end{verbatim}

\textbf{Security Services:}

{\def\LTcaptype{none} % do not increment counter
\begin{longtable}[]{@{}lll@{}}
\toprule\noalign{}
Service & Protocol & Function \\
\midrule\noalign{}
\endhead
\bottomrule\noalign{}
\endlastfoot
\textbf{Authentication} & AH & Verify packet origin \\
\textbf{Confidentiality} & ESP & Encrypt packet data \\
\textbf{Integrity} & Both & Detect modifications \\
\textbf{Anti-replay} & Both & Prevent replay attacks \\
\end{longtable}
}

\textbf{IPSec Modes:}

\begin{itemize}
\tightlist
\item
  \textbf{Transport Mode}: Protects payload only
\item
  \textbf{Tunnel Mode}: Protects entire IP packet
\end{itemize}

\textbf{Key Components:}

\begin{itemize}
\tightlist
\item
  \textbf{Security Association (SA)}: Security parameters
\item
  \textbf{Security Policy Database (SPD)}: Security policies
\item
  \textbf{Key Management}: Automated key exchange
\end{itemize}

\end{solutionbox}
\begin{mnemonicbox}
``IPSec Integrates Protection, Security, Encryption
Completely''

\end{mnemonicbox}
\begin{center}\rule{0.5\linewidth}{0.5pt}\end{center}

\subsection*{Question 3(a) [3 marks]}\label{q3a}

\textbf{List out various types of cybercrime and explain anyone.}

\begin{solutionbox}

\textbf{Cybercrime Types:}

\begin{itemize}
\tightlist
\item
  \textbf{Financial Crimes}: Credit card fraud, online banking theft
\item
  \textbf{Identity Theft}: Stealing personal information
\item
  \textbf{Cyber Bullying}: Online harassment
\item
  \textbf{Data Breach}: Unauthorized data access
\end{itemize}

\textbf{Email Bombing (Detailed Explanation):} Email bombing involves
sending large volumes of emails to overwhelm victim's mailbox and server
resources.

\textbf{Attack Process:}

\begin{itemize}
\tightlist
\item
  \textbf{Target Selection}: Choose victim email
\item
  \textbf{Volume Generation}: Send thousands of emails
\item
  \textbf{Resource Exhaustion}: Overwhelm mail server
\item
  \textbf{Service Disruption}: Make email unusable
\end{itemize}

\end{solutionbox}
\begin{mnemonicbox}
``Cyber Crimes Create Chaos Constantly''

\end{mnemonicbox}
\begin{center}\rule{0.5\linewidth}{0.5pt}\end{center}

\subsection*{Question 3(b) [4 marks]}\label{q3b}

\textbf{Define Web Jacking, Data Diddling, Dos Attack and DDOS Attack}

\begin{solutionbox}

{\def\LTcaptype{none} % do not increment counter
\begin{longtable}[]{@{}
  >{\raggedright\arraybackslash}p{(\linewidth - 2\tabcolsep) * \real{0.5200}}
  >{\raggedright\arraybackslash}p{(\linewidth - 2\tabcolsep) * \real{0.4800}}@{}}
\toprule\noalign{}
\begin{minipage}[b]{\linewidth}\raggedright
Attack Type
\end{minipage} & \begin{minipage}[b]{\linewidth}\raggedright
Definition
\end{minipage} \\
\midrule\noalign{}
\endhead
\bottomrule\noalign{}
\endlastfoot
\textbf{Web Jacking} & Unauthorized control of website by changing
content \\
\textbf{Data Diddling} & Unauthorized modification of data before
processing \\
\textbf{DoS Attack} & Single source attack to make service
unavailable \\
\textbf{DDoS Attack} & Multiple sources attack to overwhelm target
system \\
\end{longtable}
}

\textbf{Attack Comparison:}

\begin{center}
\textbf{Mermaid Diagram (Code)}
\begin{verbatim}
{Shaded}
{Highlighting}[]
graph LR
    A[DoS Attack] {-{-}{} B[Single Attacker]}
    C[DDoS Attack] {-{-}{} D[Multiple Attackers]}
    B {-{-}{} E[Target Server]}
    D {-{-}{} E}
{Highlighting}
{Shaded}
\end{verbatim}
\end{center}

\end{solutionbox}
\begin{center}\rule{0.5\linewidth}{0.5pt}\end{center}

\subsection*{Question 3(c) [7 marks]}\label{q3c}

\textbf{Explain Main in the middle attack with suitable examples.}

\begin{solutionbox}
Man-in-the-Middle (MITM) attack occurs when attacker
secretly intercepts and relays communications between two parties.

\textbf{MITM Attack Process:}

\begin{verbatim}
sequenceDiagram
    participant A as Alice
    participant M as Attacker (Mallory)
    participant B as Bob
    
    A{-M: Message to Bob}
    M{-M: Intercept \& Read}
    M{-B: Modified/Original Message}
    B{-M: Response to Alice}
    M{-M: Intercept \& Read}
    M{-A: Modified/Original Response}
\end{verbatim}

\textbf{Attack Types:}

{\def\LTcaptype{none} % do not increment counter
\begin{longtable}[]{@{}lll@{}}
\toprule\noalign{}
Type & Method & Example \\
\midrule\noalign{}
\endhead
\bottomrule\noalign{}
\endlastfoot
\textbf{Wi-Fi Eavesdropping} & Fake hotspots & Coffee shop Wi-Fi \\
\textbf{Email Hijacking} & Compromised accounts & Business email \\
\textbf{DNS Spoofing} & Fake DNS responses & Redirect to fake sites \\
\textbf{HTTPS Spoofing} & Fake certificates & Banking websites \\
\end{longtable}
}

\textbf{Real Example - Wi-Fi Attack:}

\begin{enumerate}
\tightlist
\item
  Attacker creates fake ``Free\_WiFi'' hotspot
\item
  Victim connects to malicious network
\item
  All traffic passes through attacker
\item
  Sensitive data like passwords stolen
\end{enumerate}

\textbf{Prevention Measures:}

\begin{itemize}
\tightlist
\item
  \textbf{Use HTTPS}: Encrypted connections
\item
  \textbf{VPN Usage}: Additional encryption layer
\item
  \textbf{Certificate Verification}: Check SSL certificates
\item
  \textbf{Secure Networks}: Avoid public Wi-Fi for sensitive tasks
\end{itemize}

\end{solutionbox}
\begin{mnemonicbox}
``MITM Maliciously Intercepts, Tampers Messages''

\end{mnemonicbox}
\begin{center}\rule{0.5\linewidth}{0.5pt}\end{center}

\subsection*{Question 3(a) OR [3
marks]}\label{q3a}

\textbf{Explain Salami attack in detail}

\begin{solutionbox}
Salami attack involves stealing small amounts of money
from many accounts to avoid detection.

\textbf{Attack Mechanism:}

\begin{itemize}
\tightlist
\item
  \textbf{Small Amounts}: Steal fractions of currency
\item
  \textbf{Large Scale}: Target thousands of accounts
\item
  \textbf{Rounding Errors}: Exploit calculation differences
\item
  \textbf{Accumulation}: Small thefts create large profit
\end{itemize}

\textbf{Example}: Banking system rounds interest to nearest cent.
Attacker collects remaining fractions from millions of accounts.

\end{solutionbox}
\begin{mnemonicbox}
``Salami Slices Small, Steals Significantly''

\end{mnemonicbox}
\begin{center}\rule{0.5\linewidth}{0.5pt}\end{center}

\subsection*{Question 3(b) OR [4
marks]}\label{q3b}

\textbf{Define Cyber bullying, Phishing, spyware and logic bomb}

\begin{solutionbox}

{\def\LTcaptype{none} % do not increment counter
\begin{longtable}[]{@{}ll@{}}
\toprule\noalign{}
Term & Definition \\
\midrule\noalign{}
\endhead
\bottomrule\noalign{}
\endlastfoot
\textbf{Cyber Bullying} & Online harassment causing emotional
distress \\
\textbf{Phishing} & Fraudulent attempts to obtain sensitive
information \\
\textbf{Spyware} & Software secretly monitoring user activities \\
\textbf{Logic Bomb} & Malicious code triggered by specific conditions \\
\end{longtable}
}

\end{solutionbox}
\begin{center}\rule{0.5\linewidth}{0.5pt}\end{center}

\subsection*{Question 3(c) OR [7
marks]}\label{q3c}

\textbf{Explain ransomware in detail?}

\begin{solutionbox}
Ransomware encrypts victim's files and demands payment
for decryption key.

\textbf{Ransomware Attack Process:}

\begin{verbatim}
flowchart LR
    A[Initial Infection] {-{-} B[File Encryption]}
    B {-{-} C[Ransom Demand]}
    C {-{-} D[Payment Request]}
    D {-{-} E\{Payment Made?\}}
    E {-{-}|Yes| F[Decryption Key]}
    E {-{-}|No| G[Files Remain Encrypted]}
\end{verbatim}

\textbf{Ransomware Types:}

{\def\LTcaptype{none} % do not increment counter
\begin{longtable}[]{@{}lll@{}}
\toprule\noalign{}
Type & Behavior & Example \\
\midrule\noalign{}
\endhead
\bottomrule\noalign{}
\endlastfoot
\textbf{Crypto Ransomware} & Encrypts files & WannaCry \\
\textbf{Locker Ransomware} & Locks system access & Police-themed \\
\textbf{Scareware} & Fake threats & Fake antivirus \\
\textbf{Doxware} & Threatens data publication & Personal photos \\
\end{longtable}
}

\textbf{Attack Vectors:}

\begin{itemize}
\tightlist
\item
  \textbf{Email Attachments}: Malicious documents
\item
  \textbf{Drive-by Downloads}: Compromised websites
\item
  \textbf{Exploit Kits}: Vulnerability exploitation
\item
  \textbf{RDP Attacks}: Remote desktop compromise
\end{itemize}

\textbf{Prevention Strategies:}

\begin{itemize}
\tightlist
\item
  \textbf{Regular Backups}: Offline data copies
\item
  \textbf{Security Updates}: Patch vulnerabilities
\item
  \textbf{Email Filtering}: Block malicious attachments
\item
  \textbf{User Training}: Recognize threats
\item
  \textbf{Network Segmentation}: Limit spread
\end{itemize}

\textbf{Impact Assessment:}

\begin{itemize}
\tightlist
\item
  \textbf{Financial Loss}: Ransom payments and downtime
\item
  \textbf{Data Loss}: Permanently encrypted files
\item
  \textbf{Reputation Damage}: Customer trust loss
\item
  \textbf{Operational Disruption}: Business shutdown
\end{itemize}

\end{solutionbox}
\begin{mnemonicbox}
``Ransomware Really Ruins Recovery, Requires Robust
Response''

\end{mnemonicbox}
\begin{center}\rule{0.5\linewidth}{0.5pt}\end{center}

\subsection*{Question 4(a) [3 marks]}\label{q4a}

\textbf{List out any six basic kali Linux commands.}

\begin{solutionbox}

{\def\LTcaptype{none} % do not increment counter
\begin{longtable}[]{@{}ll@{}}
\toprule\noalign{}
Command & Function \\
\midrule\noalign{}
\endhead
\bottomrule\noalign{}
\endlastfoot
\textbf{ls} & List directory contents \\
\textbf{cd} & Change directory \\
\textbf{pwd} & Print working directory \\
\textbf{mkdir} & Create directory \\
\textbf{cp} & Copy files \\
\textbf{nmap} & Network scanning \\
\end{longtable}
}

\end{solutionbox}
\begin{mnemonicbox}
``Linux Commands Make Navigation Possible''

\end{mnemonicbox}
\begin{center}\rule{0.5\linewidth}{0.5pt}\end{center}

\subsection*{Question 4(b) [4 marks]}\label{q4b}

\textbf{Explain Zero day attack with example}

\begin{solutionbox}
Zero-day attack exploits unknown vulnerability before
security patch is available.

\textbf{Zero-Day Timeline:}

\begin{verbatim}
timeline
    title Zero{-Day Attack Timeline}
    
    Day 0 : Vulnerability Discovered
          : Exploit Created
    
    Day 1{-X : Attack Launched}
            : Systems Compromised
    
    Day X+1 : Vulnerability Disclosed
            : Patch Development
    
    Day X+Y : Patch Released
            : Systems Updated
\end{verbatim}

\textbf{Example - Stuxnet Worm:}

\begin{itemize}
\tightlist
\item
  \textbf{Target}: Iranian nuclear facilities
\item
  \textbf{Exploit}: Windows zero-day vulnerabilities
\item
  \textbf{Impact}: Physical damage to centrifuges
\item
  \textbf{Duration}: Active for months before detection
\end{itemize}

\textbf{Characteristics:}

\begin{itemize}
\tightlist
\item
  \textbf{Unknown Vulnerability}: No existing patches
\item
  \textbf{High Success Rate}: No defenses prepared
\item
  \textbf{Valuable}: Expensive in dark markets
\item
  \textbf{Limited Lifespan}: Once discovered, patched
\end{itemize}

\end{solutionbox}
\begin{mnemonicbox}
``Zero-day Zaps before Anyone Notices''

\end{mnemonicbox}
\begin{center}\rule{0.5\linewidth}{0.5pt}\end{center}

\subsection*{Question 4(c) [7 marks]}\label{q4c}

\textbf{Explain Remote Access Tools and how we protect system from RAT?}

\begin{solutionbox}
Remote Access Tool (RAT) allows remote control of
computer systems, often used maliciously.

\textbf{RAT Functionality:}

\begin{verbatim}
graph TB
    A[RAT Server on Victim] {-{-} B[File Access]}
    A {-{-} C[Screen Capture]}
    A {-{-} D[Keylogging]}
    A {-{-} E[Camera/Mic Access]}
    A {-{-} F[System Control]}
    
    G[Attacker Client] {-{-} A}
\end{verbatim}

\textbf{Common RATs:}

{\def\LTcaptype{none} % do not increment counter
\begin{longtable}[]{@{}lll@{}}
\toprule\noalign{}
RAT Name & Features & Detection Difficulty \\
\midrule\noalign{}
\endhead
\bottomrule\noalign{}
\endlastfoot
\textbf{DarkComet} & Full system control & Medium \\
\textbf{Poison Ivy} & Stealth operations & High \\
\textbf{Back Orifice} & Windows targeting & Low \\
\textbf{NetBus} & Simple interface & Low \\
\end{longtable}
}

\textbf{RAT Infection Methods:}

\begin{itemize}
\tightlist
\item
  \textbf{Email Attachments}: Trojanized files
\item
  \textbf{Software Bundling}: Hidden in legitimate software
\item
  \textbf{Drive-by Downloads}: Malicious websites
\item
  \textbf{Social Engineering}: Trick users into installation
\end{itemize}

\textbf{Protection Strategies:}

\textbf{Technical Measures:}

\begin{itemize}
\tightlist
\item
  \textbf{Antivirus Software}: Real-time scanning
\item
  \textbf{Firewall Rules}: Block unauthorized connections
\item
  \textbf{Network Monitoring}: Detect unusual traffic
\item
  \textbf{System Updates}: Patch vulnerabilities
\end{itemize}

\textbf{Behavioral Measures:}

\begin{itemize}
\tightlist
\item
  \textbf{Email Caution}: Verify attachments
\item
  \textbf{Download Sources}: Use trusted sites only
\item
  \textbf{Regular Scans}: Periodic malware checks
\item
  \textbf{User Training}: Recognize threats
\end{itemize}

\textbf{Detection Signs:}

\begin{itemize}
\tightlist
\item
  \textbf{Slow Performance}: Unusual system lag
\item
  \textbf{Network Activity}: Unexpected connections
\item
  \textbf{File Changes}: Modified or new files
\item
  \textbf{Strange Behavior}: Unexpected system actions
\end{itemize}

\textbf{Incident Response:}

\begin{enumerate}
\tightlist
\item
  \textbf{Isolate System}: Disconnect from network
\item
  \textbf{Document Evidence}: Record malicious activity
\item
  \textbf{Clean System}: Remove RAT completely
\item
  \textbf{Restore Data}: From clean backups
\item
  \textbf{Strengthen Security}: Improve defenses
\end{enumerate}

\end{solutionbox}
\begin{mnemonicbox}
``RATs Remotely Access, Require Robust Response''

\end{mnemonicbox}
\begin{center}\rule{0.5\linewidth}{0.5pt}\end{center}

\subsection*{Question 4(a) OR [3
marks]}\label{q4a}

\textbf{Describe Hacking, Blackhat, and White hat hacker in short.}

\begin{solutionbox}

{\def\LTcaptype{none} % do not increment counter
\begin{longtable}[]{@{}ll@{}}
\toprule\noalign{}
Term & Definition \\
\midrule\noalign{}
\endhead
\bottomrule\noalign{}
\endlastfoot
\textbf{Hacking} & Gaining unauthorized access to systems or networks \\
\textbf{Black Hat} & Malicious hackers with criminal intent \\
\textbf{White Hat} & Ethical hackers helping improve security \\
\end{longtable}
}

\textbf{Comparison:}

\begin{itemize}
\tightlist
\item
  \textbf{Intent}: White hat helps, Black hat harms
\item
  \textbf{Authorization}: White hat has permission
\item
  \textbf{Purpose}: White hat protects, Black hat exploits
\end{itemize}

\end{solutionbox}
\begin{mnemonicbox}
``Hats Have Different Hacking Habits''

\end{mnemonicbox}
\begin{center}\rule{0.5\linewidth}{0.5pt}\end{center}

\subsection*{Question 4(b) OR [4
marks]}\label{q4b}

\textbf{What is Port Scanning? Explain any two port scanning
techniques.}

\begin{solutionbox}
Port scanning discovers open ports and services on
target systems.

\textbf{Port Scanning Techniques:}

{\def\LTcaptype{none} % do not increment counter
\begin{longtable}[]{@{}lll@{}}
\toprule\noalign{}
Technique & Method & Stealth Level \\
\midrule\noalign{}
\endhead
\bottomrule\noalign{}
\endlastfoot
\textbf{TCP Connect} & Full connection & Low stealth \\
\textbf{SYN Scan} & Half-open connection & High stealth \\
\end{longtable}
}

\textbf{TCP Connect Scan:}

\begin{itemize}
\tightlist
\item
  Completes full TCP handshake
\item
  Reliable but easily detected
\item
  Logged by target systems
\end{itemize}

\textbf{SYN Scan (Half-Open):}

\begin{itemize}
\tightlist
\item
  Sends SYN, receives SYN-ACK, sends RST
\item
  Stealthy, often unlogged
\item
  Faster than connect scan
\end{itemize}

\end{solutionbox}
\begin{mnemonicbox}
``Port Scanning Probes System Services''

\end{mnemonicbox}
\begin{center}\rule{0.5\linewidth}{0.5pt}\end{center}

\subsection*{Question 4(c) OR [7
marks]}\label{q4c}

\textbf{Explain the hacking process in detail.}

\begin{solutionbox}
Hacking follows systematic five-phase methodology for
gaining unauthorized system access.

\textbf{Five Phases of Hacking:}

\begin{verbatim}
flowchart TD
    A[Information Gathering] {-{-} B[Scanning]}
    B {-{-} C[Gaining Access]}
    C {-{-} D[Maintaining Access]}
    D {-{-} E[Covering Tracks]}
    E {-{-} A}
\end{verbatim}

\textbf{Phase Details:}

\textbf{1. Information Gathering (Reconnaissance):}

\begin{itemize}
\tightlist
\item
  \textbf{Passive}: OSINT, social media research
\item
  \textbf{Active}: Network queries, DNS lookups
\item
  \textbf{Tools}: Google dorking, Whois, social engineering
\end{itemize}

\textbf{2. Scanning:}

\begin{itemize}
\tightlist
\item
  \textbf{Network Scanning}: Discover live hosts
\item
  \textbf{Port Scanning}: Find open services
\item
  \textbf{Vulnerability Scanning}: Identify weaknesses
\item
  \textbf{Tools}: Nmap, Nessus, OpenVAS
\end{itemize}

\textbf{3. Gaining Access:}

\begin{itemize}
\tightlist
\item
  \textbf{Exploit Vulnerabilities}: Use discovered weaknesses
\item
  \textbf{Password Attacks}: Brute force, dictionary
\item
  \textbf{Social Engineering}: Manipulate humans
\item
  \textbf{Tools}: Metasploit, custom exploits
\end{itemize}

\textbf{4. Maintaining Access:}

\begin{itemize}
\tightlist
\item
  \textbf{Install Backdoors}: Ensure continued access
\item
  \textbf{Create User Accounts}: Hidden administrator
\item
  \textbf{Rootkits}: Hide presence
\item
  \textbf{Tools}: Netcat, custom backdoors
\end{itemize}

\textbf{5. Covering Tracks:}

\begin{itemize}
\tightlist
\item
  \textbf{Log Deletion}: Remove evidence
\item
  \textbf{File Hiding}: Conceal malicious files
\item
  \textbf{Process Hiding}: Hide running programs
\item
  \textbf{Tools}: Log cleaners, steganography
\end{itemize}

\textbf{Detailed Process Flow:}

{\def\LTcaptype{none} % do not increment counter
\begin{longtable}[]{@{}llll@{}}
\toprule\noalign{}
Phase & Activities & Duration & Risk Level \\
\midrule\noalign{}
\endhead
\bottomrule\noalign{}
\endlastfoot
\textbf{Reconnaissance} & Passive info gathering & Days/Weeks & Low \\
\textbf{Scanning} & Active probing & Hours/Days & Medium \\
\textbf{Gaining Access} & Exploitation & Minutes/Hours & High \\
\textbf{Maintaining Access} & Persistence & Ongoing & Medium \\
\textbf{Covering Tracks} & Evidence removal & Hours & High \\
\end{longtable}
}

\textbf{Legal vs Illegal Hacking:}

\begin{itemize}
\tightlist
\item
  \textbf{Ethical Hacking}: Authorized penetration testing
\item
  \textbf{Malicious Hacking}: Unauthorized criminal activity
\item
  \textbf{Bug Bounty}: Legal vulnerability discovery
\end{itemize}

\end{solutionbox}
\begin{mnemonicbox}
``Hackers Investigate, Scan, Gain, Maintain, Cover''

\end{mnemonicbox}
\begin{center}\rule{0.5\linewidth}{0.5pt}\end{center}

\subsection*{Question 5(a) [3 marks]}\label{q5a}

\textbf{Write Locards's principal and explain how it is related to
cybercrime?}

\begin{solutionbox}
Locard's Principle states ``Every contact leaves a
trace'' - fundamental principle in forensic science.

\textbf{Digital Application:}

\begin{itemize}
\tightlist
\item
  \textbf{Log Files}: System activities recorded
\item
  \textbf{Network Traffic}: Communication traces
\item
  \textbf{File Metadata}: Creation, modification times
\item
  \textbf{Memory Dumps}: Runtime evidence
\end{itemize}

\textbf{Cybercrime Relevance:} Digital activities create electronic
traces that investigators can analyze to reconstruct criminal
activities.

\end{solutionbox}
\begin{mnemonicbox}
``Locard's Law: Leave Lasting Logs''

\end{mnemonicbox}
\begin{center}\rule{0.5\linewidth}{0.5pt}\end{center}

\subsection*{Question 5(b) [4 marks]}\label{q5b}

\textbf{What is software forensics? How it is contributing in
cybercrime?}

\begin{solutionbox}
Software forensics analyzes software artifacts to
determine authorship, detect plagiarism, or investigate malicious code.

\textbf{Software Forensics Applications:}

{\def\LTcaptype{none} % do not increment counter
\begin{longtable}[]{@{}lll@{}}
\toprule\noalign{}
Application & Purpose & Cybercrime Use \\
\midrule\noalign{}
\endhead
\bottomrule\noalign{}
\endlastfoot
\textbf{Code Analysis} & Identify programmer & Malware attribution \\
\textbf{Binary Analysis} & Reverse engineering & Understand attacks \\
\textbf{License Compliance} & Software piracy & IP theft cases \\
\textbf{Plagiarism Detection} & Academic integrity & Copyright
violation \\
\end{longtable}
}

\textbf{Contribution to Cybercrime Investigation:}

\begin{itemize}
\tightlist
\item
  \textbf{Malware Attribution}: Link code to specific authors
\item
  \textbf{Attack Reconstruction}: Understand how attacks occurred
\item
  \textbf{Evidence Collection}: Gather digital proof
\item
  \textbf{Pattern Recognition}: Identify repeat offenders
\end{itemize}

\end{solutionbox}
\begin{center}\rule{0.5\linewidth}{0.5pt}\end{center}

\subsection*{Question 5(c) [7 marks]}\label{q5c}

\textbf{Explain in detail: Drive imaging, Chain of custody and hash
values}

\begin{solutionbox}

\textbf{Drive Imaging:} Physical bit-by-bit copy of storage device
preserving all data including deleted files and slack space.

\textbf{Imaging Process:}

\begin{verbatim}
flowchart LR
    A[Original Drive] {-{-} B[Imaging Tool]}
    B {-{-} C[Forensic Image]}
    C {-{-} D[Hash Verification]}
    D {-{-} E[Analysis]}
\end{verbatim}

\textbf{Chain of Custody:} Documentation tracking evidence handling from
seizure to court presentation.

\textbf{Chain of Custody Elements:}

{\def\LTcaptype{none} % do not increment counter
\begin{longtable}[]{@{}ll@{}}
\toprule\noalign{}
Element & Details \\
\midrule\noalign{}
\endhead
\bottomrule\noalign{}
\endlastfoot
\textbf{Who} & Person handling evidence \\
\textbf{What} & Evidence description \\
\textbf{When} & Date and time \\
\textbf{Where} & Location of evidence \\
\textbf{Why} & Reason for handling \\
\end{longtable}
}

\textbf{Hash Values:} Mathematical algorithms creating unique
fingerprints to verify data integrity.

\textbf{Common Hash Algorithms:}

{\def\LTcaptype{none} % do not increment counter
\begin{longtable}[]{@{}lll@{}}
\toprule\noalign{}
Algorithm & Output Size & Use Case \\
\midrule\noalign{}
\endhead
\bottomrule\noalign{}
\endlastfoot
\textbf{MD5} & 128 bits & Quick verification \\
\textbf{SHA-1} & 160 bits & Legacy systems \\
\textbf{SHA-256} & 256 bits & Modern standard \\
\end{longtable}
}

\textbf{Forensic Implementation:}

\begin{enumerate}
\tightlist
\item
  \textbf{Create Image}: Bit-by-bit copy
\item
  \textbf{Generate Hash}: Calculate original drive hash
\item
  \textbf{Verify Integrity}: Compare image hash
\item
  \textbf{Document Process}: Chain of custody
\item
  \textbf{Analyze Safely}: Work on copy only
\end{enumerate}

\textbf{Importance in Digital Forensics:}

\begin{itemize}
\tightlist
\item
  \textbf{Data Integrity}: Ensures evidence authenticity
\item
  \textbf{Legal Admissibility}: Court accepts verified evidence
\item
  \textbf{Non-Repudiation}: Proves data unchanged
\item
  \textbf{Forensic Soundness}: Maintains evidence quality
\end{itemize}

\end{solutionbox}
\begin{mnemonicbox}
``Drive Images Document Digital Data Definitively''

\end{mnemonicbox}
\begin{center}\rule{0.5\linewidth}{0.5pt}\end{center}

\subsection*{Question 5(a) OR [3
marks]}\label{q5a}

\textbf{Explain four stage of malware analysis in short.}

\begin{solutionbox}

\textbf{Malware Analysis Stages:}

{\def\LTcaptype{none} % do not increment counter
\begin{longtable}[]{@{}
  >{\raggedright\arraybackslash}p{(\linewidth - 4\tabcolsep) * \real{0.2188}}
  >{\raggedright\arraybackslash}p{(\linewidth - 4\tabcolsep) * \real{0.4062}}
  >{\raggedright\arraybackslash}p{(\linewidth - 4\tabcolsep) * \real{0.3750}}@{}}
\toprule\noalign{}
\begin{minipage}[b]{\linewidth}\raggedright
Stage
\end{minipage} & \begin{minipage}[b]{\linewidth}\raggedright
Description
\end{minipage} & \begin{minipage}[b]{\linewidth}\raggedright
Tools Used
\end{minipage} \\
\midrule\noalign{}
\endhead
\bottomrule\noalign{}
\endlastfoot
\textbf{Static Analysis} & Examine without execution & Hex editors,
disassemblers \\
\textbf{Dynamic Analysis} & Observe runtime behavior & Sandboxes,
debuggers \\
\textbf{Code Analysis} & Reverse engineer source & IDA Pro, Ghidra \\
\textbf{Network Analysis} & Monitor communications & Wireshark,
tcpdump \\
\end{longtable}
}

\end{solutionbox}
\begin{mnemonicbox}
``Static, Dynamic, Code, Network - SDCN''

\end{mnemonicbox}
\begin{center}\rule{0.5\linewidth}{0.5pt}\end{center}

\subsection*{Question 5(b) OR [4
marks]}\label{q5b}

\textbf{How does network forensic functions?}

\begin{solutionbox}
Network forensics captures, records, and analyzes
network traffic to investigate security incidents.

\textbf{Network Forensics Process:}

\begin{verbatim}
flowchart LR
    A[Traffic Capture] {-{-} B[Data Storage]}
    B {-{-} C[Analysis]}
    C {-{-} D[Evidence Extraction]}
    D {-{-} E[Reporting]}
\end{verbatim}

\textbf{Key Functions:}

\begin{itemize}
\tightlist
\item
  \textbf{Packet Capture}: Record network communications
\item
  \textbf{Protocol Analysis}: Examine communication protocols
\item
  \textbf{Flow Analysis}: Track data movement patterns
\item
  \textbf{Content Inspection}: Analyze payload data
\end{itemize}

\textbf{Tools and Techniques:}

\begin{itemize}
\tightlist
\item
  \textbf{Network Taps}: Hardware monitoring
\item
  \textbf{Packet Analyzers}: Software inspection
\item
  \textbf{Flow Collectors}: Traffic summarization
\item
  \textbf{SIEM Systems}: Correlation and alerting
\end{itemize}

\end{solutionbox}
\begin{center}\rule{0.5\linewidth}{0.5pt}\end{center}

\subsection*{Question 5(c) OR [7
marks]}\label{q5c}

\textbf{Explain digital forensic investigation process}

\begin{solutionbox}
Digital forensic investigation follows systematic
methodology to collect, preserve, analyze, and present digital evidence.

\textbf{Investigation Process Phases:}

\begin{verbatim}
flowchart LR
    A[Identification] {-{-} B[Preservation]}
    B {-{-} C[Collection]}
    C {-{-} D[Examination]}
    D {-{-} E[Analysis]}
    E {-{-} F[Presentation]}
\end{verbatim}

\textbf{Detailed Process:}

\textbf{1. Identification Phase:}

\begin{itemize}
\tightlist
\item
  \textbf{Evidence Location}: Find potential digital evidence
\item
  \textbf{Scope Definition}: Determine investigation boundaries
\item
  \textbf{Resource Planning}: Allocate personnel and tools
\item
  \textbf{Legal Considerations}: Obtain necessary warrants
\end{itemize}

\textbf{2. Preservation Phase:}

\begin{itemize}
\tightlist
\item
  \textbf{Scene Securing}: Prevent evidence contamination
\item
  \textbf{System Isolation}: Disconnect from networks
\item
  \textbf{Evidence Documentation}: Photograph and catalog
\item
  \textbf{Chain of Custody}: Begin documentation trail
\end{itemize}

\textbf{3. Collection Phase:}

\begin{itemize}
\tightlist
\item
  \textbf{Imaging Process}: Create forensic copies
\item
  \textbf{Hash Generation}: Ensure data integrity
\item
  \textbf{Metadata Capture}: Record file properties
\item
  \textbf{Live Data Collection}: Capture volatile information
\end{itemize}

\textbf{4. Examination Phase:}

\begin{itemize}
\tightlist
\item
  \textbf{Data Recovery}: Retrieve deleted files
\item
  \textbf{File System Analysis}: Examine storage structures
\item
  \textbf{Timeline Creation}: Establish event sequence
\item
  \textbf{Keyword Searching}: Find relevant content
\end{itemize}

\textbf{5. Analysis Phase:}

\begin{itemize}
\tightlist
\item
  \textbf{Evidence Correlation}: Link related findings
\item
  \textbf{Pattern Recognition}: Identify trends
\item
  \textbf{Hypothesis Testing}: Validate theories
\item
  \textbf{Timeline Analysis}: Reconstruct events
\end{itemize}

\textbf{6. Presentation Phase:}

\begin{itemize}
\tightlist
\item
  \textbf{Report Writing}: Document findings
\item
  \textbf{Evidence Preparation}: Organize for court
\item
  \textbf{Expert Testimony}: Present in legal proceedings
\item
  \textbf{Visualization}: Create demonstrative aids
\end{itemize}

\textbf{Investigation Principles:}

{\def\LTcaptype{none} % do not increment counter
\begin{longtable}[]{@{}lll@{}}
\toprule\noalign{}
Principle & Description & Importance \\
\midrule\noalign{}
\endhead
\bottomrule\noalign{}
\endlastfoot
\textbf{Reliability} & Evidence must be dependable & Court acceptance \\
\textbf{Repeatability} & Results can be reproduced & Scientific
validity \\
\textbf{Integrity} & Data remains unchanged & Legal admissibility \\
\textbf{Documentation} & Complete record keeping & Audit trail \\
\end{longtable}
}

\textbf{Key Challenges:}

\begin{itemize}
\tightlist
\item
  \textbf{Encryption}: Password-protected data
\item
  \textbf{Anti-Forensics}: Evidence hiding techniques
\item
  \textbf{Volume}: Large amounts of data
\item
  \textbf{Technology}: Rapidly changing systems
\end{itemize}

\textbf{Best Practices:}

\begin{itemize}
\tightlist
\item
  \textbf{Standard Procedures}: Follow established protocols
\item
  \textbf{Tool Validation}: Use tested forensic tools
\item
  \textbf{Continuous Training}: Stay current with technology
\item
  \textbf{Quality Assurance}: Peer review processes
\end{itemize}

\textbf{Legal Framework:}

\begin{itemize}
\tightlist
\item
  \textbf{Evidence Rules}: Admissibility requirements
\item
  \textbf{Privacy Laws}: Data protection compliance
\item
  \textbf{Chain of Custody}: Unbroken documentation
\item
  \textbf{Expert Qualifications}: Forensic examiner credentials
\end{itemize}

\end{solutionbox}
\begin{mnemonicbox}
``Digital Investigation: Identify, Preserve, Collect,
Examine, Analyze, Present''

\end{mnemonicbox}

\end{document}
