\documentclass[10pt,a4paper]{article}

% content/resources/templates/preamble.tex
\usepackage[margin=0.6in]{geometry}
\author{Milav Dabgar}
\usepackage{amsmath,amssymb,amsthm}
\usepackage{booktabs}
\usepackage{multirow}
\usepackage{xcolor}
\usepackage{tcolorbox}
\tcbuselibrary{breakable,skins}
\usepackage[colorlinks=true,linkcolor=blue]{hyperref}
\usepackage{titlesec}
\usepackage{enumitem}
\usepackage{tikz}
\usepackage{pgfplots}
\usepackage{circuitikz}
\usepackage[version=4]{mhchem}
\usepackage{longtable}
\usepackage{array}
\usepackage{float}
\usepackage{caption}
\usepackage{listings}

\lstset{
  basicstyle=\small\ttfamily,
  breaklines=true,
  breakatwhitespace=false,
  postbreak=\mbox{\textcolor{red}{$\hookrightarrow$}\space},
  float=false,
  numbers=left,
  numberstyle=\tiny\color{gray},
  numbersep=10pt,
  xleftmargin=2em,
  keywordstyle=\color{blue},
  commentstyle=\color{green!60!black},
  stringstyle=\color{purple},
  backgroundcolor=\color{gray!5},
  showstringspaces=false,
  tabsize=2,
  captionpos=b,
  keepspaces=true,
  columns=flexible
}

\pgfplotsset{compat=1.18}
\usetikzlibrary{shapes,arrows,positioning,calc,patterns,decorations.pathmorphing,decorations.markings,arrows.meta}

% Color scheme
\definecolor{headcolor}{RGB}{0,102,204}
\definecolor{keycolor}{RGB}{220,20,60}
\definecolor{solutioncolor}{RGB}{34,139,34}
\definecolor{mnemoniccolor}{RGB}{148,0,211}
\definecolor{codecolor}{RGB}{0,0,100}

% Spacing
\setlength{\parskip}{3pt}
\setlist[itemize]{nosep}
\setlist[enumerate]{nosep}

% Title formatting
\titleformat{\section}{\Large\bfseries\color{headcolor}}{\thesection}{1em}{}
\titleformat{\subsection}{\large\bfseries\color{headcolor}}{\thesubsection}{1em}{}

% Pandoc tightlist compatibility
\providecommand{\tightlist}{%
  \setlength{\itemsep}{0pt}\setlength{\parskip}{0pt}}

% Pandoc longtable compatibility
\newcounter{none}
\def\thenone{}


% content/resources/templates/gujarati-boxes.tex
\usepackage{fontspec}
\usepackage{polyglossia}

% Set Gujarati as main language (document is primarily in Gujarati)
% Note: gloss-gujarati.ldf doesn't exist in polyglossia, but it will use hyphenation patterns
\setdefaultlanguage{gujarati}
\setotherlanguage{english}

% Configure Gujarati font properly
% Use Language=Default to prevent polyglossia from trying to add language-specific features
% that don't exist for Gujarati, which causes "empty feature" warnings
\newfontfamily\gujaratifont[Script=Gujarati,AutoFakeBold=2.5,AutoFakeSlant=0.3]{Noto Sans Gujarati}
\setmainfont[Script=Gujarati,AutoFakeBold=2.5,AutoFakeSlant=0.3]{Noto Sans Gujarati}
% Use Noto Sans Gujarati for monospace to support Gujarati in text
\setmonofont[Scale=0.9]{Noto Sans Gujarati}

% Configure English to use the same font
\newfontfamily\englishfont[Script=Gujarati,AutoFakeBold=2.5,AutoFakeSlant=0.3]{Noto Sans Gujarati}

% Translations for polyglossia
\gappto\captionsgujarati{
  \renewcommand{\tablename}{કોષ્ટક}
  \renewcommand{\figurename}{આકૃતિ}
}

% Helper for TikZ nodes to ensure Gujarati font
\newcommand{\gu}[1]{{\gujaratifont #1}}

% Custom environments
\newtcolorbox{solutionbox}{
    breakable,
    enhanced,
    colback=solutioncolor!5!white,
    colframe=solutioncolor!75!black,
    fonttitle=\bfseries,
    title=જવાબ
}

\newtcolorbox{solutionboxnobreak}{
 colback=solutioncolor!5!white,
 colframe=solutioncolor!75!black,
 fonttitle=\bfseries,
 title=જવાબ
}

\newtcolorbox{keyformula}{
 breakable,
 enhanced,
 colback=keycolor!5!white,
 colframe=keycolor!75!black,
 fonttitle=\bfseries,
 title=રાસાયણિક સમીકરણ/સૂત્ર
}

\newtcolorbox{mnemonicbox}{
 breakable,
 enhanced,
 colback=mnemoniccolor!5!white,
 colframe=mnemoniccolor!75!black,
 fonttitle=\bfseries,
 title=મેમરી ટ્રીક
}


\begin{document}

\begin{center}
{\Huge\bfseries\color{headcolor} Subject Name (Gujarati)}\\[5pt]
{\LARGE 4361601 -- Summer 2025}\\[3pt]
{\large Semester 1 Study Material}\\[3pt]
{\normalsize\textit{Detailed Solutions and Explanations}}
\end{center}

\vspace{10pt}

\subsection*{પ્રશ્ન 1(અ) [3
ગુણ]}\label{uxaaauxab0uxab6uxaa8-1uxa85-3-uxa97uxaa3}

\textbf{Public key અને Private Key cryptography વચ્ચેનો તફાવત આપો.}

\begin{solutionbox}

{\def\LTcaptype{none} % do not increment counter
\begin{longtable}[]{@{}
  >{\raggedright\arraybackslash}p{(\linewidth - 4\tabcolsep) * \real{0.1404}}
  >{\raggedright\arraybackslash}p{(\linewidth - 4\tabcolsep) * \real{0.4386}}
  >{\raggedright\arraybackslash}p{(\linewidth - 4\tabcolsep) * \real{0.4211}}@{}}
\toprule\noalign{}
\begin{minipage}[b]{\linewidth}\raggedright
પાસાં
\end{minipage} & \begin{minipage}[b]{\linewidth}\raggedright
Private Key Cryptography
\end{minipage} & \begin{minipage}[b]{\linewidth}\raggedright
Public Key Cryptography
\end{minipage} \\
\midrule\noalign{}
\endhead
\bottomrule\noalign{}
\endlastfoot
\textbf{Key Management} & એક જ key encryption/decryption માટે & અલગ keys
encryption/decryption માટે \\
\textbf{Key Distribution} & સુરક્ષિત channel જરૂરી & સુરક્ષિત channel જરૂરી
નથી \\
\textbf{Speed} & ઝડપી processing & Private key કરતાં ધીમી \\
\textbf{Security Level} & key ગુપ્ત રાખવાથી ઉચ્ચ & ગાણિતિક સુરક્ષા ઉચ્ચ \\
\textbf{ઉદાહરણ} & DES, AES & RSA, ECC \\
\end{longtable}
}

\end{solutionbox}
\begin{mnemonicbox}
``Private Personal, Public Pair''

\end{mnemonicbox}
\begin{center}\rule{0.5\linewidth}{0.5pt}\end{center}

\subsection*{પ્રશ્ન 1(બ) [4
ગુણ]}\label{uxaaauxab0uxab6uxaa8-1uxaac-4-uxa97uxaa3}

\textbf{CIA Triad સમજાવો.}

\begin{solutionbox}

CIA Triad એ માહિતી સુરક્ષાનો પાયો છે જેમાં ત્રણ મુખ્ય સિદ્ધાંતો છે:

\textbf{આકૃતિ:}

\begin{center}
\textbf{Mermaid Diagram (Code)}
\begin{verbatim}
{Shaded}
{Highlighting}[]
graph TD
    A[CIA Triad] {-{-}{} B[Confidentiality]}
    A {-{-}{} C[Integrity]}
    A {-{-}{} D[Availability]}
    B {-{-}{} E[ડેટા ગોપનીયતા]}
    C {-{-}{} F[ડેટા ચોકસાઈ]}
    D {-{-}{} G[સિસ્ટમ ઍક્સેસ]}
{Highlighting}
{Shaded}
\end{verbatim}
\end{center}

\begin{itemize}
\tightlist
\item
  \textbf{Confidentiality (ગોપનીયતા)}: ડેટા ફક્ત અધિકૃત વપરાશકર્તાઓ માટે
  ઉપલબ્ધ હોય
\item
  \textbf{Integrity (અખંડિતતા)}: ડેટાની સચોટતા અને સંપૂર્ણતા જાળવે
\item
  \textbf{Availability (ઉપલબ્ધતા)}: જરૂર પડે ત્યારે સિસ્ટમ્સ ઉપલબ્ધ હોય
\end{itemize}

\end{solutionbox}
\begin{mnemonicbox}
``Can I Access'' (Confidentiality, Integrity,
Availability)

\end{mnemonicbox}
\begin{center}\rule{0.5\linewidth}{0.5pt}\end{center}

\subsection*{પ્રશ્ન 1(ક) [7
ગુણ]}\label{uxaaauxab0uxab6uxaa8-1uxa95-7-uxa97uxaa3}

\textbf{Md5 અલ્ગોરિધમના પગલાં સમજાવો.}

\begin{solutionbox}

MD5 (Message Digest 5) એ 128-bit hash value બનાવતું cryptographic hash
function છે.

\textbf{અલ્ગોરિધમ પગલાં:}

{\def\LTcaptype{none} % do not increment counter
\begin{longtable}[]{@{}lll@{}}
\toprule\noalign{}
પગલું & પ્રક્રિયા & વર્ણન \\
\midrule\noalign{}
\endhead
\bottomrule\noalign{}
\endlastfoot
1 & \textbf{Padding} & message length ≡ 448 (mod 512) બનાવવા bits
ઉમેરવા \\
2 & \textbf{Length Addition} & મૂળ message ની 64-bit length ઉમેરવી \\
3 & \textbf{Initialize Buffers} & ચાર 32-bit buffers (A, B, C, D) સેટ
કરવા \\
4 & \textbf{Process Blocks} & 512-bit blocks માં message process કરવો \\
5 & \textbf{Round Functions} & 16 operations ના 4 rounds લાગુ કરવા \\
\end{longtable}
}

\textbf{કોડ બ્લોક:}

\begin{verbatim}
\# MD5 Processing Steps
def md5\_process():
    \# Step 1: Padding
    padded\_message = original + padding\_bits
    \# Step 2: Process in 512{-bit chunks  }
    for chunk in chunks:
        \# Step 3: Apply round functions
        result = round\_functions(chunk)
    return final\_hash
\end{verbatim}

\begin{itemize}
\tightlist
\item
  \textbf{Round 1}: F(X,Y,Z) = (X\wedgeY) \vee (\negX\wedgeZ)
\item
  \textbf{Round 2}: G(X,Y,Z) = (X\wedgeZ) \vee (Y\wedge\negZ)
\item
  \textbf{Round 3}: H(X,Y,Z) = X\oplusY\oplusZ
\item
  \textbf{Round 4}: I(X,Y,Z) = Y\oplus(X\vee\negZ)
\end{itemize}

\end{solutionbox}
\begin{mnemonicbox}
``My Data Needs Proper Processing''

\end{mnemonicbox}
\begin{center}\rule{0.5\linewidth}{0.5pt}\end{center}

\subsection*{પ્રશ્ન 1(ક અથવા) [7
ગુણ]}\label{uxaaauxab0uxab6uxaa8-1uxa95-uxa85uxaa5uxab5-7-uxa97uxaa3}

\textbf{RSA ના શોધકોની યાદી બનાવો. RSA અલ્ગોરિધમના સ્ટેપ્સ લખો.}

\begin{solutionbox}

\textbf{RSA શોધકો:}

\begin{itemize}
\tightlist
\item
  \textbf{Ron Rivest} (MIT)
\item
  \textbf{Adi Shamir} (MIT)
\item
  \textbf{Leonard Adleman} (MIT)
\end{itemize}

\textbf{RSA અલ્ગોરિધમ પગલાં:}

{\def\LTcaptype{none} % do not increment counter
\begin{longtable}[]{@{}lll@{}}
\toprule\noalign{}
પગલું & પ્રક્રિયા & સૂત્ર \\
\midrule\noalign{}
\endhead
\bottomrule\noalign{}
\endlastfoot
1 & \textbf{Primes પસંદ કરો} & p, q (મોટા primes) પસંદ કરો \\
2 & \textbf{n ગણતરી} & n = p \times q \\
3 & \textbf{φ(n) ગણતરી} & φ(n) = (p-1) \times (q-1) \\
4 & \textbf{e પસંદ કરો} & gcd(e, φ(n)) = 1 \\
5 & \textbf{d ગણતરી} & d \times e ≡ 1 (mod φ(n)) \\
6 & \textbf{Encryption} & C = M\^{}e mod n \\
7 & \textbf{Decryption} & M = C\^{}d mod n \\
\end{longtable}
}

\textbf{Key Pairs:}

\begin{itemize}
\tightlist
\item
  \textbf{Public Key}: (n, e)
\item
  \textbf{Private Key}: (n, d)
\end{itemize}

\end{solutionbox}
\begin{mnemonicbox}
``RSA: Rivest Shamir Adleman''

\end{mnemonicbox}
\begin{center}\rule{0.5\linewidth}{0.5pt}\end{center}

\subsection*{પ્રશ્ન 2(અ) [3
ગુણ]}\label{uxaaauxab0uxab6uxaa8-2uxa85-3-uxa97uxaa3}

\textbf{વ્યાખ્યા આપો: Firewall. Firewall ની મર્યાદાઓની યાદી બનાવો.}

\begin{solutionbox}

\textbf{વ્યાખ્યા:} Firewall એ network security device છે જે પૂર્વ-નિર્ધારિત
સુરક્ષા નિયમોના આધારે આવતા/જતા network traffic ને monitor અને control કરે છે.

\textbf{મર્યાદાઓ:}

{\def\LTcaptype{none} % do not increment counter
\begin{longtable}[]{@{}
  >{\raggedright\arraybackslash}p{(\linewidth - 2\tabcolsep) * \real{0.4800}}
  >{\raggedright\arraybackslash}p{(\linewidth - 2\tabcolsep) * \real{0.5200}}@{}}
\toprule\noalign{}
\begin{minipage}[b]{\linewidth}\raggedright
મર્યાદા
\end{minipage} & \begin{minipage}[b]{\linewidth}\raggedright
વર્ણન
\end{minipage} \\
\midrule\noalign{}
\endhead
\bottomrule\noalign{}
\endlastfoot
\textbf{આંતરિક ધમકીઓ} & insider attacks થી સુરક્ષા આપી શકતી નથી \\
\textbf{Application Layer} & application-specific attacks સામે મર્યાદિત
સુરક્ષા \\
\textbf{Performance} & network traffic ધીમી કરી શકે છે \\
\textbf{Configuration} & યોગ્ય setup અને maintenance જરૂરી \\
\textbf{Encrypted Traffic} & encrypted content ને અસરકારક રીતે inspect કરી
શકતી નથી \\
\end{longtable}
}

\end{solutionbox}
\begin{mnemonicbox}
``Fire Walls Limit Internal Protection''

\end{mnemonicbox}
\begin{center}\rule{0.5\linewidth}{0.5pt}\end{center}

\subsection*{પ્રશ્ન 2(બ) [4
ગુણ]}\label{uxaaauxab0uxab6uxaa8-2uxaac-4-uxa97uxaa3}

\textbf{IPsec Tunnel Mode અને Transport mode નું સ્કેચ કરો.}

\begin{solutionbox}

\textbf{IPsec Modes Comparison:}

\begin{verbatim}
Transport Mode:
+{-{-}{-}{-}{-}{-}{-}{-}{-}{-}+{-}{-}{-}{-}{-}{-}{-}{-}{-}{-}+{-}{-}{-}{-}{-}{-}{-}{-}{-}{-}+}
| Original | IPsec    | Original |
| IP Header| Header   | Payload  |
+{-{-}{-}{-}{-}{-}{-}{-}{-}{-}+{-}{-}{-}{-}{-}{-}{-}{-}{-}{-}+{-}{-}{-}{-}{-}{-}{-}{-}{-}{-}+}

Tunnel Mode:
+{-{-}{-}{-}{-}{-}{-}{-}{-}{-}+{-}{-}{-}{-}{-}{-}{-}{-}{-}{-}+{-}{-}{-}{-}{-}{-}{-}{-}{-}{-}+{-}{-}{-}{-}{-}{-}{-}{-}{-}{-}+}
| New IP   | IPsec    | Original | Original |
| Header   | Header   | IP Header| Payload  |
+{-{-}{-}{-}{-}{-}{-}{-}{-}{-}+{-}{-}{-}{-}{-}{-}{-}{-}{-}{-}+{-}{-}{-}{-}{-}{-}{-}{-}{-}{-}+{-}{-}{-}{-}{-}{-}{-}{-}{-}{-}+}
\end{verbatim}

\textbf{મુખ્ય તફાવતો:}

{\def\LTcaptype{none} % do not increment counter
\begin{longtable}[]{@{}lll@{}}
\toprule\noalign{}
પાસું & Transport Mode & Tunnel Mode \\
\midrule\noalign{}
\endhead
\bottomrule\noalign{}
\endlastfoot
\textbf{સુરક્ષા} & ફક્ત Payload & સંપૂર્ણ packet \\
\textbf{ઉપયોગ} & End-to-end & Gateway-to-gateway \\
\textbf{Overhead} & ઓછું & વધારે \\
\textbf{IP Header} & મૂળ જાળવાયેલું & નવું header ઉમેર્યું \\
\end{longtable}
}

\end{solutionbox}
\begin{mnemonicbox}
``Transport Travels, Tunnel Total''

\end{mnemonicbox}
\begin{center}\rule{0.5\linewidth}{0.5pt}\end{center}

\subsection*{પ્રશ્ન 2(ક) [7
ગુણ]}\label{uxaaauxab0uxab6uxaa8-2uxa95-7-uxa97uxaa3}

\textbf{વિવિધ પ્રકારના Active અને Passive attacks નું વિગતવાર વર્ણન કરો.}

\begin{solutionbox}

\textbf{Attack વર્ગીકરણ:}

\begin{center}
\textbf{Mermaid Diagram (Code)}
\begin{verbatim}
{Shaded}
{Highlighting}[]
graph TD
    A[Network Attacks] {-{-}{} B[Active Attacks]}
    A {-{-}{} C[Passive Attacks]}
    B {-{-}{} D[Modification]}
    B {-{-}{} E[Fabrication]}
    B {-{-}{} F[Interruption]}
    C {-{-}{} G[Eavesdropping]}
    C {-{-}{} H[Traffic Analysis]}
{Highlighting}
{Shaded}
\end{verbatim}
\end{center}

\textbf{Active Attacks:}

{\def\LTcaptype{none} % do not increment counter
\begin{longtable}[]{@{}lll@{}}
\toprule\noalign{}
પ્રકાર & વર્ણન & ઉદાહરણ \\
\midrule\noalign{}
\endhead
\bottomrule\noalign{}
\endlastfoot
\textbf{Masquerade} & અન્ય entity નો નકલી અવતાર & Fake identity \\
\textbf{Replay} & captured data ને ફરીથી transmit કરવું & Session replay \\
\textbf{Modification} & message content ને બદલવું & Data tampering \\
\textbf{DoS} & service availability નો ઇનકાર & Server flooding \\
\end{longtable}
}

\textbf{Passive Attacks:}

{\def\LTcaptype{none} % do not increment counter
\begin{longtable}[]{@{}
  >{\raggedright\arraybackslash}p{(\linewidth - 4\tabcolsep) * \real{0.2222}}
  >{\raggedright\arraybackslash}p{(\linewidth - 4\tabcolsep) * \real{0.4815}}
  >{\raggedright\arraybackslash}p{(\linewidth - 4\tabcolsep) * \real{0.2963}}@{}}
\toprule\noalign{}
\begin{minipage}[b]{\linewidth}\raggedright
પ્રકાર
\end{minipage} & \begin{minipage}[b]{\linewidth}\raggedright
વર્ણન
\end{minipage} & \begin{minipage}[b]{\linewidth}\raggedright
અસર
\end{minipage} \\
\midrule\noalign{}
\endhead
\bottomrule\noalign{}
\endlastfoot
\textbf{Eavesdropping} & communications સાંભળવું & Data theft \\
\textbf{Traffic Analysis} & communication patterns નું analysis & Privacy
breach \\
\textbf{Monitoring} & network activity નું observation & Information
gathering \\
\end{longtable}
}

\begin{itemize}
\tightlist
\item
  \textbf{Active attacks} system resources અથવા data ને modify કરે છે
\item
  \textbf{Passive attacks} માહિતી observe અને collect કરે છે
\item
  \textbf{Detection}: Active attacks passive કરતાં વધારે detect થાય છે
\end{itemize}

\end{solutionbox}
\begin{mnemonicbox}
``Active Acts, Passive Peeks''

\end{mnemonicbox}
\begin{center}\rule{0.5\linewidth}{0.5pt}\end{center}

\subsection*{પ્રશ્ન 2(અ અથવા) [3
ગુણ]}\label{uxaaauxab0uxab6uxaa8-2uxa85-uxa85uxaa5uxab5-3-uxa97uxaa3}

\textbf{વ્યાખ્યા આપો: Digital Signature. Digital Signature ના વિવિધ
એપ્લિકેશન ક્ષેત્રોરની ચર્ચા કરો.}

\begin{solutionbox}

\textbf{વ્યાખ્યા:} Digital Signature એ cryptographic technique છે જે public
key cryptography ના ઉપયોગથી digital messages અથવા documents ની
authenticity અને integrity ને validate કરે છે.

\textbf{એપ્લિકેશન ક્ષેત્રો:}

{\def\LTcaptype{none} % do not increment counter
\begin{longtable}[]{@{}ll@{}}
\toprule\noalign{}
ક્ષેત્ર & ઉપયોગ \\
\midrule\noalign{}
\endhead
\bottomrule\noalign{}
\endlastfoot
\textbf{E-commerce} & Online transactions, contracts \\
\textbf{Banking} & Electronic fund transfers, cheques \\
\textbf{Government} & Digital certificates, સરકારી documents \\
\textbf{Healthcare} & Patient records, prescriptions \\
\textbf{Legal} & Electronic contracts, court documents \\
\end{longtable}
}

\end{solutionbox}
\begin{mnemonicbox}
``Digital Documents Demand Authentic Approval''

\end{mnemonicbox}
\begin{center}\rule{0.5\linewidth}{0.5pt}\end{center}

\subsection*{પ્રશ્ન 2(બ અથવા) [4
ગુણ]}\label{uxaaauxab0uxab6uxaa8-2uxaac-uxa85uxaa5uxab5-4-uxa97uxaa3}

\textbf{HTTP અને HTTPS વચ્ચેનો તફાવત આપો.}

\begin{solutionbox}

{\def\LTcaptype{none} % do not increment counter
\begin{longtable}[]{@{}
  >{\raggedright\arraybackslash}p{(\linewidth - 4\tabcolsep) * \real{0.4583}}
  >{\raggedright\arraybackslash}p{(\linewidth - 4\tabcolsep) * \real{0.2500}}
  >{\raggedright\arraybackslash}p{(\linewidth - 4\tabcolsep) * \real{0.2917}}@{}}
\toprule\noalign{}
\begin{minipage}[b]{\linewidth}\raggedright
પેરામીટર
\end{minipage} & \begin{minipage}[b]{\linewidth}\raggedright
HTTP
\end{minipage} & \begin{minipage}[b]{\linewidth}\raggedright
HTTPS
\end{minipage} \\
\midrule\noalign{}
\endhead
\bottomrule\noalign{}
\endlastfoot
\textbf{સુરક્ષા} & કોઈ encryption નથી & SSL/TLS encryption \\
\textbf{Port} & 80 & 443 \\
\textbf{Protocol} & Hypertext Transfer Protocol & HTTP + SSL/TLS \\
\textbf{ડેટા સુરક્ષા} & Plain text & Encrypted \\
\textbf{Authentication} & Server verification નથી & Server certificate
validation \\
\textbf{Speed} & વધારે ઝડપી & થોડી ધીમી \\
\textbf{URL Prefix} & http:// & https:// \\
\end{longtable}
}

\textbf{આકૃતિ:}

\begin{verbatim}
HTTP:
Client {-{-}{-}{-}Plain Text{-}{-}{-}{-} Server}

HTTPS:
Client {-{-}{-}{-}Encrypted{-}{-}{-}{-}{-} Server}
       {{-}{-}{-}Certificate{-}{-}{-}{-}}
\end{verbatim}

\end{solutionbox}
\begin{mnemonicbox}
``HTTPS Has Security''

\end{mnemonicbox}
\begin{center}\rule{0.5\linewidth}{0.5pt}\end{center}

\subsection*{પ્રશ્ન 2(ક અથવા) [7
ગુણ]}\label{uxaaauxab0uxab6uxaa8-2uxa95-uxa85uxaa5uxab5-7-uxa97uxaa3}

\textbf{વ્યાખ્યા આપો: Malicious software. Virus, Worm, Keylogger, Trojans
ને વિગતવાર સમજાવો.}

\begin{solutionbox}

\textbf{વ્યાખ્યા:} Malicious software (Malware) એ એવા software છે જે
computer systems ને નુકસાન પહોંચાડવા, exploit કરવા અથવા unauthorized access
મેળવવા માટે design કરવામાં આવે છે.

\textbf{Malware ના પ્રકારો:}

{\def\LTcaptype{none} % do not increment counter
\begin{longtable}[]{@{}
  >{\raggedright\arraybackslash}p{(\linewidth - 4\tabcolsep) * \real{0.1875}}
  >{\raggedright\arraybackslash}p{(\linewidth - 4\tabcolsep) * \real{0.5000}}
  >{\raggedright\arraybackslash}p{(\linewidth - 4\tabcolsep) * \real{0.3125}}@{}}
\toprule\noalign{}
\begin{minipage}[b]{\linewidth}\raggedright
પ્રકાર
\end{minipage} & \begin{minipage}[b]{\linewidth}\raggedright
લક્ષણો
\end{minipage} & \begin{minipage}[b]{\linewidth}\raggedright
વર્તન
\end{minipage} \\
\midrule\noalign{}
\endhead
\bottomrule\noalign{}
\endlastfoot
\textbf{Virus} & Host file જરૂરી & Programs સાથે attach થાય, execute થતાં
spread થાય \\
\textbf{Worm} & Self-replicating & Networks દ્વારા સ્વતંત્ર રીતે spread
થાય \\
\textbf{Keylogger} & Keystrokes record કરે & Passwords અને sensitive data
steal કરે \\
\textbf{Trojan} & Legitimate તરીકે disguise & Attackers ને backdoor access
આપે \\
\end{longtable}
}

\textbf{વિગતવાર સમજૂતી:}

\textbf{Virus:}

\begin{itemize}
\tightlist
\item
  Execute થવા માટે host program જરૂરી
\item
  Infected files દ્વારા spread થાય
\item
  Data corrupt અથવા delete કરી શકે
\end{itemize}

\textbf{Worm:}

\begin{itemize}
\tightlist
\item
  Self-propagating malware
\item
  Network vulnerabilities exploit કરે
\item
  Network bandwidth consume કરે
\end{itemize}

\textbf{Keylogger:}

\begin{itemize}
\tightlist
\item
  User keystrokes record કરે
\item
  Login credentials capture કરે
\item
  Hardware અથવા software-based હોઈ શકે
\end{itemize}

\textbf{Trojan:}

\begin{itemize}
\tightlist
\item
  Legitimate software તરીકે દેખાય
\item
  Remote access માટે backdoor બનાવે
\item
  Self-replicate થતું નથી
\end{itemize}

\end{solutionbox}
\begin{mnemonicbox}
``Viruses Visit, Worms Wander, Keys Captured,
Trojans Trick''

\end{mnemonicbox}
\begin{center}\rule{0.5\linewidth}{0.5pt}\end{center}

\subsection*{પ્રશ્ન 3(અ) [3
ગુણ]}\label{uxaaauxab0uxab6uxaa8-3uxa85-3-uxa97uxaa3}

\textbf{વ્યાખ્યા આપો: Cybercrime. Cyber Law ની જરૂરિયાતો વિશે ચર્ચા કરો.}

\begin{solutionbox}

\textbf{વ્યાખ્યા:} Cybercrime એ computers, networks અથવા digital devices ને
tools અથવા targets તરીકે ઉપયોગ કરીને કરવામાં આવતી ગુનાહિત પ્રવૃત્તિઓ છે.

\textbf{Cyber Law ની જરૂરિયાતો:}

{\def\LTcaptype{none} % do not increment counter
\begin{longtable}[]{@{}ll@{}}
\toprule\noalign{}
જરૂરિયાત & સમર્થન \\
\midrule\noalign{}
\endhead
\bottomrule\noalign{}
\endlastfoot
\textbf{કાનૂની માળખું} & Cyber અપરાધોની સ્પષ્ટ વ્યાખ્યા સ્થાપિત કરવી \\
\textbf{અધિકારક્ષેત્ર} & ભૌગોલિક સીમાઓમાં સત્તાની વ્યાખ્યા \\
\textbf{પુરાવા} & Digital evidence collection માટે guidelines \\
\textbf{સજા} & Cybercriminals માટે deterrent પગલાં \\
\textbf{સુરક્ષા} & વ્યક્તિગત અને સંસ્થાકીય અધિકારોનું રક્ષણ \\
\end{longtable}
}

\end{solutionbox}
\begin{mnemonicbox}
``Cyber Laws Create Legal Protection''

\end{mnemonicbox}
\begin{center}\rule{0.5\linewidth}{0.5pt}\end{center}

\subsection*{પ્રશ્ન 3(બ) [4
ગુણ]}\label{uxaaauxab0uxab6uxaa8-3uxaac-4-uxa97uxaa3}

\textbf{Cyber spying અને Cyber theft સમજાવો.}

\begin{solutionbox}

\textbf{Cyber Spying:}

\begin{itemize}
\tightlist
\item
  \textbf{વ્યાખ્યા}: Digital communications અને activities ની unauthorized
  surveillance
\item
  \textbf{પદ્ધતિઓ}: Malware, phishing, social engineering
\item
  \textbf{લક્ષ્યો}: Government, corporate secrets, personal data
\item
  \textbf{અસર}: National security threats, competitive disadvantage
\end{itemize}

\textbf{Cyber Theft:}

\begin{itemize}
\tightlist
\item
  \textbf{વ્યાખ્યા}: Digital assets અથવા information નું unauthorized taking
\item
  \textbf{પ્રકારો}: Identity theft, financial fraud, intellectual
  property theft
\item
  \textbf{પદ્ધતિઓ}: Hacking, social engineering, insider threats
\item
  \textbf{પરિણામો}: Financial loss, reputation damage
\end{itemize}

\textbf{તુલના કોષ્ટક:}

{\def\LTcaptype{none} % do not increment counter
\begin{longtable}[]{@{}lll@{}}
\toprule\noalign{}
પાસું & Cyber Spying & Cyber Theft \\
\midrule\noalign{}
\endhead
\bottomrule\noalign{}
\endlastfoot
\textbf{હેતુ} & Information gathering & Asset acquisition \\
\textbf{Detection} & ઘણીવાર undetected & Notice થઈ શકે \\
\textbf{અવધિ} & Long-term monitoring & One-time અથવા periodic \\
\textbf{પ્રેરણા} & Intelligence/espionage & Financial gain \\
\end{longtable}
}

\end{solutionbox}
\begin{mnemonicbox}
``Spies Spy, Thieves Take''

\end{mnemonicbox}
\begin{center}\rule{0.5\linewidth}{0.5pt}\end{center}

\subsection*{પ્રશ્ન 3(ક) [7
ગુણ]}\label{uxaaauxab0uxab6uxaa8-3uxa95-7-uxa97uxaa3}

\textbf{Cyber Law ની કલમ 66 સમજાવો.}

\begin{solutionbox}

\textbf{કલમ 66 - Computer Related Offences (IT Act 2008):}

\textbf{મુખ્ય જોગવાઈઓ:}

{\def\LTcaptype{none} % do not increment counter
\begin{longtable}[]{@{}
  >{\raggedright\arraybackslash}p{(\linewidth - 4\tabcolsep) * \real{0.3824}}
  >{\raggedright\arraybackslash}p{(\linewidth - 4\tabcolsep) * \real{0.2647}}
  >{\raggedright\arraybackslash}p{(\linewidth - 4\tabcolsep) * \real{0.3529}}@{}}
\toprule\noalign{}
\begin{minipage}[b]{\linewidth}\raggedright
પેટા-કલમ
\end{minipage} & \begin{minipage}[b]{\linewidth}\raggedright
અપરાધ
\end{minipage} & \begin{minipage}[b]{\linewidth}\raggedright
સજા
\end{minipage} \\
\midrule\noalign{}
\endhead
\bottomrule\noalign{}
\endlastfoot
\textbf{66(1)} & બેઈમાનીથી/છેતરપિંડીથી computer resource damage & 3 વર્ષ સુધી
કેદ + ₹5 લાખ સુધી દંડ \\
\textbf{66A} & અપમાનજનક સંદેશા મોકલવા & 3 વર્ષ સુધી + દંડ \\
\textbf{66B} & ચોરેલા computer resource receive કરવા & 3 વર્ષ + ₹1 લાખ સુધી
દંડ \\
\textbf{66C} & Identity theft & 3 વર્ષ + ₹1 લાખ સુધી દંડ \\
\textbf{66D} & Computer વાપરીને personation દ્વારા છેતરપિંડી & 3 વર્ષ + ₹1
લાખ સુધી દંડ \\
\textbf{66E} & Privacy નું ઉલ્લંઘન & 3 વર્ષ + ₹2 લાખ સુધી દંડ \\
\textbf{66F} & Cyber terrorism & આજીવન કેદ \\
\end{longtable}
}

\textbf{વિગતવાર કવરેજ:}

\textbf{કલમ 66 મુખ્ય અપરાધો:}

\begin{itemize}
\tightlist
\item
  \textbf{Hacking}: Computer systems માં unauthorized access
\item
  \textbf{Data Theft}: પરવાનગી વિના data steal અથવા copy કરવું
\item
  \textbf{System Damage}: Computer data destroy અથવા alter કરવું
\item
  \textbf{વાયરસ પ્રવેશ}: Malicious code દાખલ કરવું
\end{itemize}

\textbf{જરૂરી તત્વો:}

\begin{itemize}
\tightlist
\item
  \textbf{ઈરાદો}: બેઈમાન અથવા છેતરપિંડીનો ઈરાદો
\item
  \textbf{પ્રવેશ}: માલિકની પરવાનગી વિના
\item
  \textbf{નુકસાન}: System અથવા data ને હાનિ પહોંચાડવી
\item
  \textbf{જાણકારી}: Unauthorized access ની જાણકારી
\end{itemize}

\textbf{કાનૂની માળખું:}

\begin{itemize}
\tightlist
\item
  \textbf{Cognizable}: Police warrant વિના arrest કરી શકે
\item
  \textbf{Non-bailable}: Court ના વિવેકબુદ્ધિથી bail
\item
  \textbf{પુરાવા}: Digital evidence court માં admissible
\end{itemize}

\end{solutionbox}
\begin{mnemonicbox}
``Section 66 Stops Cyber Sins''

\end{mnemonicbox}
\begin{center}\rule{0.5\linewidth}{0.5pt}\end{center}

\subsection*{પ્રશ્ન 3(અ અથવા) [3
ગુણ]}\label{uxaaauxab0uxab6uxaa8-3uxa85-uxa85uxaa5uxab5-3-uxa97uxaa3}

\textbf{Cyber terrorism સમજાવો.}

\begin{solutionbox}

\textbf{વ્યાખ્યા:} Cyber terrorism એ રાજકીય, ધાર્મિક અથવા વૈચારિક હેતુઓ માટે
ભય, વિક્ષેપ અથવા નુકસાન સર્જવા માટે digital technologies નો ઉપયોગ છે.

\textbf{લક્ષણો:}

{\def\LTcaptype{none} % do not increment counter
\begin{longtable}[]{@{}ll@{}}
\toprule\noalign{}
પાસું & વર્ણન \\
\midrule\noalign{}
\endhead
\bottomrule\noalign{}
\endlastfoot
\textbf{લક્ષ્ય} & Critical infrastructure, government systems \\
\textbf{પદ્ધતિ} & DDoS attacks, system infiltration, data destruction \\
\textbf{પ્રેરણા} & Political, religious, ideological goals \\
\textbf{અસર} & Public fear, economic disruption, national security \\
\end{longtable}
}

\textbf{ઉદાહરણો:}

\begin{itemize}
\tightlist
\item
  Power grid પર attacks
\item
  Transportation system disruption
\item
  Financial system targeting
\end{itemize}

\end{solutionbox}
\begin{mnemonicbox}
``Terror Through Technology''

\end{mnemonicbox}
\begin{center}\rule{0.5\linewidth}{0.5pt}\end{center}

\subsection*{પ્રશ્ન 3(બ અથવા) [4
ગુણ]}\label{uxaaauxab0uxab6uxaa8-3uxaac-uxa85uxaa5uxab5-4-uxa97uxaa3}

\textbf{Cyber bullying \& Cyber stalking સમજાવો.}

\begin{solutionbox}

\textbf{Cyber Bullying:}

\begin{itemize}
\tightlist
\item
  \textbf{વ્યાખ્યા}: અન્યોને harass, intimidate અથવા harm કરવા માટે digital
  platforms નો ઉપયોગ
\item
  \textbf{પ્લેટફોર્મ}: Social media, messaging apps, online forums
\item
  \textbf{લક્ષણો}: Repetitive, intentional harm, power imbalance
\item
  \textbf{અસર}: Psychological trauma, depression, social isolation
\end{itemize}

\textbf{Cyber Stalking:}

\begin{itemize}
\tightlist
\item
  \textbf{વ્યાખ્યા}: ભય અથવા emotional distress ઉત્પન્ન કરતું persistent
  online harassment
\item
  \textbf{પદ્ધતિઓ}: Unwanted messages, tracking, identity theft
\item
  \textbf{અવધિ}: Long-term, continuous behavior
\item
  \textbf{કાનૂની}: ઘણા jurisdictions માં criminal offense
\end{itemize}

\textbf{તુલના:}

{\def\LTcaptype{none} % do not increment counter
\begin{longtable}[]{@{}lll@{}}
\toprule\noalign{}
પાસું & Cyber Bullying & Cyber Stalking \\
\midrule\noalign{}
\endhead
\bottomrule\noalign{}
\endlastfoot
\textbf{અવધિ} & Episodes & Persistent \\
\textbf{વયજૂથ} & મુખ્યત્વે minors & બધી ઉંમર \\
\textbf{પ્રેરણા} & Social dominance & Obsession/control \\
\textbf{પ્લેટફોર્મ} & Public/semi-public & Private/public \\
\end{longtable}
}

\end{solutionbox}
\begin{mnemonicbox}
``Bullies Bother, Stalkers Stalk''

\end{mnemonicbox}
\begin{center}\rule{0.5\linewidth}{0.5pt}\end{center}

\subsection*{પ્રશ્ન 3(ક અથવા) [7
ગુણ]}\label{uxaaauxab0uxab6uxaa8-3uxa95-uxa85uxaa5uxab5-7-uxa97uxaa3}

\textbf{Cyber Law ની કલમ 67 સમજાવો.}

\begin{solutionbox}

\textbf{કલમ 67 - અશ્લીલ માહિતી પ્રકાશિત કરવું (IT Act 2008):}

\textbf{મુખ્ય જોગવાઈઓ:}

{\def\LTcaptype{none} % do not increment counter
\begin{longtable}[]{@{}
  >{\raggedright\arraybackslash}p{(\linewidth - 4\tabcolsep) * \real{0.3000}}
  >{\raggedright\arraybackslash}p{(\linewidth - 4\tabcolsep) * \real{0.3000}}
  >{\raggedright\arraybackslash}p{(\linewidth - 4\tabcolsep) * \real{0.4000}}@{}}
\toprule\noalign{}
\begin{minipage}[b]{\linewidth}\raggedright
કલમ
\end{minipage} & \begin{minipage}[b]{\linewidth}\raggedright
વિષય-વસ્તુ
\end{minipage} & \begin{minipage}[b]{\linewidth}\raggedright
સજા
\end{minipage} \\
\midrule\noalign{}
\endhead
\bottomrule\noalign{}
\endlastfoot
\textbf{67} & અશ્લીલ સામગ્રી પ્રકાશિત કરવી & પ્રથમ દોષિત: 3 વર્ષ + ₹5 લાખ
દંડ \\
\textbf{67A} & લૈંગિક સ્પષ્ટ સામગ્રી & 5 વર્ષ સુધી + ₹10 લાખ દંડ \\
\textbf{67B} & બાળ અશ્લીલતા & પ્રથમ: 5 વર્ષ + ₹10 લાખ, આવર્તક: 7 વર્ષ + ₹10
લાખ \\
\textbf{67C} & મધ્યવર્તી જવાબદારી & ગેરકાયદેસર content remove કરવામાં
નિષ્ફળતા \\
\end{longtable}
}

\textbf{મુખ્ય તત્વો:}

\textbf{કલમ 67 - અશ્લીલતા:}

\begin{itemize}
\tightlist
\item
  \textbf{પ્રકાશન}: Electronic form માં ઉપલબ્ધ કરાવવું
\item
  \textbf{વિષય-વસ્તુ}: કામુક, લૈંગિક સ્પષ્ટ સામગ્રી
\item
  \textbf{માધ્યમ}: Website, email, social media
\item
  \textbf{ઈરાદો}: દર્શકોને corrupt અથવા deprave કરવાનો
\end{itemize}

\textbf{કલમ 67A - લૈંગિક સ્પષ્ટ:}

\begin{itemize}
\tightlist
\item
  સામાન્ય અશ્લીલતા કરતાં \textbf{વધારે સજા}
\item
  સ્પષ્ટ sexual content માટે \textbf{વ્યાપક અવકાશ}
\item
  \textbf{વ્યાવસાયિક હેતુ} aggravating factor તરીકે ગણાય
\end{itemize}

\textbf{કલમ 67B - બાળ સુરક્ષા:}

\begin{itemize}
\tightlist
\item
  બાળ શોષણ માટે \textbf{શૂન્ય સહનશીલતા}
\item
  Possession અને distribution માટે \textbf{કડક જવાબદારી}
\item
  ગંભીરતા દર્શાવતી \textbf{ઉચ્ચ સજાઓ}
\item
  Platforms માટે \textbf{વય ચકાસણી} જરૂરિયાતો
\end{itemize}

\textbf{ઉપલબ્ધ બચાવ:}

\begin{itemize}
\tightlist
\item
  \textbf{વૈજ્ઞાનિક/શિક્ષણિક} હેતુ
\item
  \textbf{કલાત્મક ગુણવત્તા} ની ધ્યાનમાં લેવાઈ
\item
  કેટલાક કિસ્સાઓમાં \textbf{ખાનગી જોવાઈ}
\item
  Content ના \textbf{સ્વભાવ વિશે જાણકારીનો} અભાવ
\end{itemize}

\textbf{ડિજિટલ પુરાવાની જરૂરિયાતો:}

\begin{itemize}
\tightlist
\item
  \textbf{Chain of custody} ની જાળવણી
\item
  \textbf{તકનીકી અધિકૃતતા} નો પુરાવો
\item
  \textbf{સ્રોત ઓળખ} પદ્ધતિઓ
\item
  Electronic evidence નું \textbf{સંરક્ષણ}
\end{itemize}

\end{solutionbox}
\begin{mnemonicbox}
``Section 67 Stops Shameful Sharing''

\end{mnemonicbox}
\begin{center}\rule{0.5\linewidth}{0.5pt}\end{center}

\subsection*{પ્રશ્ન 4(અ) [3
ગુણ]}\label{uxaaauxab0uxab6uxaa8-4uxa85-3-uxa97uxaa3}

\textbf{હેકર્સના પ્રકારોની ચર્ચા કરો.}

\begin{solutionbox}

\textbf{હેકર વર્ગીકરણ:}

{\def\LTcaptype{none} % do not increment counter
\begin{longtable}[]{@{}lll@{}}
\toprule\noalign{}
પ્રકાર & પ્રેરણા & પ્રવૃત્તિઓ \\
\midrule\noalign{}
\endhead
\bottomrule\noalign{}
\endlastfoot
\textbf{White Hat} & નૈતિક સુરક્ષા પરીક્ષણ & અધિકૃત penetration testing \\
\textbf{Black Hat} & દુર્ભાવનાપૂર્ણ ઈરાદો & ગેરકાયદેસર system breaking \\
\textbf{Gray Hat} & મિશ્ર પ્રેરણાઓ & Unauthorized પણ non-malicious \\
\textbf{Script Kiddie} & માન્યતા/મજા & હાલના tools નો ઉપયોગ \\
\textbf{Hacktivist} & રાજકીય/સામાજિક કારણો & Hacking દ્વારા વિરોધ \\
\end{longtable}
}

\textbf{વિગતવાર પ્રકારો:}

\begin{itemize}
\tightlist
\item
  \textbf{White Hat}: નૈતિક hackers, સુરક્ષા વ્યાવસાયિકો
\item
  \textbf{Black Hat}: નફો અથવા નુકસાન શોધતા cybercriminals
\item
  \textbf{Gray Hat}: નૈતિક અને દુર્ભાવનાપૂર્ણ વચ્ચે
\end{itemize}

\end{solutionbox}
\begin{mnemonicbox}
``Hats Have Hacker Hierarchy''

\end{mnemonicbox}
\begin{center}\rule{0.5\linewidth}{0.5pt}\end{center}

\subsection*{પ્રશ્ન 4(બ) [4
ગુણ]}\label{uxaaauxab0uxab6uxaa8-4uxaac-4-uxa97uxaa3}

\textbf{RAT સમજાવો.}

\begin{solutionbox}

\textbf{RAT (Remote Administration Tool):}

\textbf{વ્યાખ્યા:} Software જે computer system ના remote control ની મંજૂરી આપે
છે, ઘણીવાર unauthorized access માટે દુર્ભાવનાપૂર્ણ રીતે ઉપયોગ થાય છે.

\textbf{લક્ષણો:}

{\def\LTcaptype{none} % do not increment counter
\begin{longtable}[]{@{}ll@{}}
\toprule\noalign{}
ફીચર & વર્ણન \\
\midrule\noalign{}
\endhead
\bottomrule\noalign{}
\endlastfoot
\textbf{Remote Control} & અંતરથી સંપૂર્ણ system access \\
\textbf{Stealth Mode} & User detection થી છુપાયેલું \\
\textbf{Data Theft} & ફાઈલ access અને transfer ક્ષમતાઓ \\
\textbf{Keylogging} & Keystroke recording \\
\textbf{Screen Capture} & Desktop monitoring \\
\end{longtable}
}

\textbf{સામાન્ય RATs:}

\begin{itemize}
\tightlist
\item
  \textbf{BackOrifice}
\item
  \textbf{NetBus}
\item
  \textbf{DarkComet}
\item
  \textbf{Poison Ivy}
\end{itemize}

\textbf{Detection પદ્ધતિઓ:}

\begin{itemize}
\tightlist
\item
  Antivirus software
\item
  Network monitoring
\item
  Process analysis
\item
  Behavioral detection
\end{itemize}

\end{solutionbox}
\begin{mnemonicbox}
``RATs Run Remote Access Tactics''

\end{mnemonicbox}
\begin{center}\rule{0.5\linewidth}{0.5pt}\end{center}

\subsection*{પ્રશ્ન 4(ક) [7
ગુણ]}\label{uxaaauxab0uxab6uxaa8-4uxa95-7-uxa97uxaa3}

\textbf{હેકિંગના પાંચ સ્ટેપ્સ સમજાવો.}

\begin{solutionbox}

\textbf{પાંચ-તબક્કાની હેકિંગ પદ્ધતિ:}

\begin{center}
\textbf{Mermaid Diagram (Code)}
\begin{verbatim}
{Shaded}
{Highlighting}[]
graph LR
    A[1. Reconnaissance] {-{-}{} B[2. Scanning]}
    B {-{-}{} C[3. Gaining Access]}
    C {-{-}{} D[4. Maintaining Access]}
    D {-{-}{} E[5. Covering Tracks]}
{Highlighting}
{Shaded}
\end{verbatim}
\end{center}

\textbf{વિગતવાર પગલાં:}

{\def\LTcaptype{none} % do not increment counter
\begin{longtable}[]{@{}
  >{\raggedright\arraybackslash}p{(\linewidth - 6\tabcolsep) * \real{0.2000}}
  >{\raggedright\arraybackslash}p{(\linewidth - 6\tabcolsep) * \real{0.2571}}
  >{\raggedright\arraybackslash}p{(\linewidth - 6\tabcolsep) * \real{0.3429}}
  >{\raggedright\arraybackslash}p{(\linewidth - 6\tabcolsep) * \real{0.2000}}@{}}
\toprule\noalign{}
\begin{minipage}[b]{\linewidth}\raggedright
તબક્કો
\end{minipage} & \begin{minipage}[b]{\linewidth}\raggedright
હેતુ
\end{minipage} & \begin{minipage}[b]{\linewidth}\raggedright
તકનીકો
\end{minipage} & \begin{minipage}[b]{\linewidth}\raggedright
સાધનો
\end{minipage} \\
\midrule\noalign{}
\endhead
\bottomrule\noalign{}
\endlastfoot
\textbf{1. Reconnaissance} & માહિતી એકત્રીકરણ & OSINT, Social Engineering
& Google, Shodan, WHOIS \\
\textbf{2. Scanning} & Vulnerabilities ઓળખવી & Port scanning, Network
mapping & Nmap, Nessus \\
\textbf{3. Gaining Access} & Vulnerabilities નો દુરુપયોગ & Password
attacks, Code injection & Metasploit, Hydra \\
\textbf{4. Maintaining Access} & સતત નિયંત્રણ & Backdoors, Rootkits &
RATs, Trojans \\
\textbf{5. Covering Tracks} & પુરાવા છુપાવવા & Log deletion, Steganography
& CCleaner, File wipers \\
\end{longtable}
}

\textbf{તબક્કો 1 - Reconnaissance:}

\begin{itemize}
\tightlist
\item
  \textbf{Passive}: જાહેર માહિતી એકત્રીકરણ
\item
  \textbf{Active}: પ્રત્યક્ષ target interaction
\item
  \textbf{લક્ષ્ય}: Target infrastructure નું mapping
\end{itemize}

\textbf{તબક્કો 2 - Scanning:}

\begin{itemize}
\tightlist
\item
  \textbf{Network scanning}: Live system identification
\item
  \textbf{Port scanning}: Service discovery\\
\item
  \textbf{Vulnerability scanning}: Weakness identification
\end{itemize}

\textbf{તબક્કો 3 - Gaining Access:}

\begin{itemize}
\tightlist
\item
  \textbf{Exploitation}: Vulnerability utilization
\item
  \textbf{Authentication attacks}: Password cracking
\item
  \textbf{Privilege escalation}: Higher access levels
\end{itemize}

\textbf{તબક્કો 4 - Maintaining Access:}

\begin{itemize}
\tightlist
\item
  \textbf{Backdoor installation}: ભવિષ્ય access
\item
  \textbf{System modification}: Persistence mechanisms
\item
  \textbf{Data collection}: Information harvesting
\end{itemize}

\textbf{તબક્કો 5 - Covering Tracks:}

\begin{itemize}
\tightlist
\item
  \textbf{Log manipulation}: Evidence removal
\item
  \textbf{File deletion}: Trace elimination
\item
  \textbf{Timeline modification}: Activity concealment
\end{itemize}

\end{solutionbox}
\begin{mnemonicbox}
``Real Smart Guys Make Choices'' (Reconnaissance,
Scanning, Gaining, Maintaining, Covering)

\end{mnemonicbox}
\begin{center}\rule{0.5\linewidth}{0.5pt}\end{center}

\subsection*{પ્રશ્ન 4(અ અથવા) [3
ગુણ]}\label{uxaaauxab0uxab6uxaa8-4uxa85-uxa85uxaa5uxab5-3-uxa97uxaa3}

\textbf{Brute force attack સમજાવો.}

\begin{solutionbox}

\textbf{વ્યાખ્યા:} Brute force attack એ trial-and-error પદ્ધતિ છે જે બધા
સંભવિત combinations ને વ્યવસ્થિત રીતે try કરીને encrypted data ને decode કરવા
માટે ઉપયોગ થાય છે.

\textbf{લક્ષણો:}

{\def\LTcaptype{none} % do not increment counter
\begin{longtable}[]{@{}ll@{}}
\toprule\noalign{}
પાસું & વર્ણન \\
\midrule\noalign{}
\endhead
\bottomrule\noalign{}
\endlastfoot
\textbf{પદ્ધતિ} & Exhaustive key search \\
\textbf{સમય} & Computationally intensive \\
\textbf{સફળતા} & બાંયધરી આપેલી પણ સમય લેવાડતી \\
\textbf{લક્ષ્ય} & Passwords, encryption keys \\
\textbf{સાધનો} & Automated software \\
\end{longtable}
}

\textbf{પ્રકારો:}

\begin{itemize}
\tightlist
\item
  \textbf{Simple Brute Force}: બધા સંભવિત combinations
\item
  \textbf{Dictionary Attack}: સામાન્ય passwords
\item
  \textbf{Hybrid Attack}: Dictionary + variations
\end{itemize}

\end{solutionbox}
\begin{mnemonicbox}
``Brute Force Breaks By Trying''

\end{mnemonicbox}
\begin{center}\rule{0.5\linewidth}{0.5pt}\end{center}

\subsection*{પ્રશ્ન 4(બ અથવા) [4
ગુણ]}\label{uxaaauxab0uxab6uxaa8-4uxaac-uxa85uxaa5uxab5-4-uxa97uxaa3}

\textbf{વ્યાખ્યા આપો: Vulnerability, Threat, Exploit}

\begin{solutionbox}

\textbf{સુરક્ષા પરિભાષા:}

{\def\LTcaptype{none} % do not increment counter
\begin{longtable}[]{@{}
  >{\raggedright\arraybackslash}p{(\linewidth - 4\tabcolsep) * \real{0.2222}}
  >{\raggedright\arraybackslash}p{(\linewidth - 4\tabcolsep) * \real{0.4444}}
  >{\raggedright\arraybackslash}p{(\linewidth - 4\tabcolsep) * \real{0.3333}}@{}}
\toprule\noalign{}
\begin{minipage}[b]{\linewidth}\raggedright
શબ્દ
\end{minipage} & \begin{minipage}[b]{\linewidth}\raggedright
વ્યાખ્યા
\end{minipage} & \begin{minipage}[b]{\linewidth}\raggedright
ઉદાહરણ
\end{minipage} \\
\midrule\noalign{}
\endhead
\bottomrule\noalign{}
\endlastfoot
\textbf{Vulnerability} & System/software માં નબળાઈ & Unpatched software
bug \\
\textbf{Threat} & Asset માટે સંભવિત ખતરો & Malicious hacker \\
\textbf{Exploit} & Vulnerability નો ફાયદો ઉઠાવતો code & Buffer overflow
attack \\
\end{longtable}
}

\textbf{સંબંધ:}

\begin{verbatim}
Threat {-{-}{-}{-}uses{-}{-}{-}{-} Exploit {-}{-}{-}{-}targets{-}{-}{-}{-} Vulnerability}
   |                    |                        |
   v                    v                        v
Hacker              Attack Code            System Weakness
\end{verbatim}

\textbf{ઉદાહરણો:}

\begin{itemize}
\tightlist
\item
  \textbf{Vulnerability}: SQL injection ખામી
\item
  \textbf{Threat}: Cybercriminal
\item
  \textbf{Exploit}: SQL injection payload
\end{itemize}

\textbf{જોખમ સૂત્ર:} Risk = Threat \times Vulnerability \times Asset Value

\end{solutionbox}
\begin{mnemonicbox}
``Threats Target Vulnerable Exploits''

\end{mnemonicbox}
\begin{center}\rule{0.5\linewidth}{0.5pt}\end{center}

\subsection*{પ્રશ્ન 4(ક અથવા) [7
ગુણ]}\label{uxaaauxab0uxab6uxaa8-4uxa95-uxa85uxaa5uxab5-7-uxa97uxaa3}

\textbf{kali Linux ના કોઈપણ ત્રણ કમાન્ડ ઉદાહરણ આપીને સમજાવો.}

\begin{solutionbox}

\textbf{આવશ્યક Kali Linux કમાન્ડ્સ:}

\textbf{1. NMAP (Network Mapper):}

\begin{verbatim}
\# Port scanning
nmap {-sS} target\_ip
nmap {-A} {-T4} 192.168.1.1
\end{verbatim}

{\def\LTcaptype{none} % do not increment counter
\begin{longtable}[]{@{}lll@{}}
\toprule\noalign{}
વિકલ્પ & હેતુ & ઉદાહરણ \\
\midrule\noalign{}
\endhead
\bottomrule\noalign{}
\endlastfoot
\textbf{-sS} & SYN scan & nmap -sS 192.168.1.1 \\
\textbf{-A} & Aggressive scan & nmap -A target.com \\
\textbf{-p} & Specific ports & nmap -p 80,443 target.com \\
\end{longtable}
}

\textbf{2. Metasploit:}

\begin{verbatim}
\# Metasploit શરૂ કરો
msfconsole
\# Exploits શોધો
search apache
\# Exploit ઉપયોગ કરો
use exploit/windows/smb/ms17\_010\_eternalblue
\end{verbatim}

\textbf{કમાન્ડ્સ:}

\begin{itemize}
\tightlist
\item
  \textbf{search}: Exploits/payloads શોધવા
\item
  \textbf{use}: Module પસંદ કરવું
\item
  \textbf{set}: Options configure કરવા
\item
  \textbf{exploit}: Attack લોંચ કરવા
\end{itemize}

\textbf{3. Wireshark:}

\begin{verbatim}
\# Command line version
tshark {-i} eth0
\# Traffic filter કરો
tshark {-i} eth0 {-f} "port 80"
\end{verbatim}

\textbf{ફીચર્સ:}

\begin{itemize}
\tightlist
\item
  \textbf{Packet capture}: Real-time network monitoring
\item
  \textbf{Protocol analysis}: Deep packet inspection\\
\item
  \textbf{Filter options}: Targeted traffic analysis
\item
  \textbf{GUI interface}: User-friendly analysis
\end{itemize}

\textbf{વધારાની કમાન્ડ્સ:}

\textbf{4. Hydra (Password Cracking):}

\begin{verbatim}
hydra {-l} admin {-P} passwords.txt ssh://192.168.1.1
\end{verbatim}

\textbf{5. John the Ripper:}

\begin{verbatim}
john {-{-}wordlist}=rockyou.txt hashes.txt
\end{verbatim}

\textbf{6. Aircrack-ng (WiFi Security):}

\begin{verbatim}
airmon{-ng} start wlan0
airodump{-ng} wlan0mon
\end{verbatim}

\textbf{કમાન્ડ કેટેગરીઝ:}

{\def\LTcaptype{none} % do not increment counter
\begin{longtable}[]{@{}lll@{}}
\toprule\noalign{}
કેટેગરી & સાધનો & હેતુ \\
\midrule\noalign{}
\endhead
\bottomrule\noalign{}
\endlastfoot
\textbf{Network Scanning} & nmap, masscan & Host/port discovery \\
\textbf{Vulnerability Assessment} & OpenVAS, Nessus & Security
scanning \\
\textbf{Exploitation} & Metasploit, SQLmap & Vulnerability
exploitation \\
\textbf{Password Attacks} & Hydra, John & Credential cracking \\
\textbf{Wireless Security} & Aircrack-ng & WiFi penetration testing \\
\end{longtable}
}

\end{solutionbox}
\begin{mnemonicbox}
``Network Maps Make Security''

\end{mnemonicbox}
\begin{center}\rule{0.5\linewidth}{0.5pt}\end{center}

\subsection*{પ્રશ્ન 5(અ) [3
ગુણ]}\label{uxaaauxab0uxab6uxaa8-5uxa85-3-uxa97uxaa3}

\textbf{Digital Forensics ની શાખાઓની સૂચિ બનાવો}

\begin{solutionbox}

\textbf{Digital Forensics શાખાઓ:}

{\def\LTcaptype{none} % do not increment counter
\begin{longtable}[]{@{}
  >{\raggedright\arraybackslash}p{(\linewidth - 4\tabcolsep) * \real{0.2353}}
  >{\raggedright\arraybackslash}p{(\linewidth - 4\tabcolsep) * \real{0.3529}}
  >{\raggedright\arraybackslash}p{(\linewidth - 4\tabcolsep) * \real{0.4118}}@{}}
\toprule\noalign{}
\begin{minipage}[b]{\linewidth}\raggedright
શાખા
\end{minipage} & \begin{minipage}[b]{\linewidth}\raggedright
ફોકસ વિસ્તાર
\end{minipage} & \begin{minipage}[b]{\linewidth}\raggedright
એપ્લિકેશન્સ
\end{minipage} \\
\midrule\noalign{}
\endhead
\bottomrule\noalign{}
\endlastfoot
\textbf{Computer Forensics} & Desktop/laptop systems & Hard drive
analysis \\
\textbf{Network Forensics} & Network traffic analysis & Intrusion
investigation \\
\textbf{Mobile Forensics} & Smartphones/tablets & Call logs, messages \\
\textbf{Database Forensics} & Database systems & Data integrity
verification \\
\textbf{Malware Forensics} & Malicious software & Malware analysis \\
\textbf{Email Forensics} & Email communications & Email header
analysis \\
\textbf{Memory Forensics} & RAM analysis & Live system investigation \\
\end{longtable}
}

\textbf{વિશેષિત વિસ્તારો:}

\begin{itemize}
\tightlist
\item
  \textbf{Cloud Forensics}
\item
  \textbf{IoT Forensics}
\item
  \textbf{Blockchain Forensics}
\end{itemize}

\end{solutionbox}
\begin{mnemonicbox}
``Digital Detectives Discover Many Clues''

\end{mnemonicbox}
\begin{center}\rule{0.5\linewidth}{0.5pt}\end{center}

\subsection*{પ્રશ્ન 5(બ) [4
ગુણ]}\label{uxaaauxab0uxab6uxaa8-5uxaac-4-uxa97uxaa3}

\textbf{Digital Forensics માં લોકાર્ડના વિનિમયના સિદ્ધાંતની ચર્ચા કરો.}

\begin{solutionbox}

\textbf{લોકાર્ડનો વિનિમય સિદ્ધાંત:}

\textbf{મૂળ સિદ્ધાંત:} ``દરેક સંપર્ક નિશાન છોડે છે''

\textbf{ડિજિટલ એપ્લિકેશન:}

{\def\LTcaptype{none} % do not increment counter
\begin{longtable}[]{@{}lll@{}}
\toprule\noalign{}
ડિજિટલ પ્રવૃત્તિ & છોડવામાં આવેલ નિશાન & સ્થાન \\
\midrule\noalign{}
\endhead
\bottomrule\noalign{}
\endlastfoot
\textbf{File Access} & Access timestamps & File metadata \\
\textbf{Web Browsing} & Browser history, cookies & Browser cache \\
\textbf{Email Communication} & Headers, logs & Mail servers \\
\textbf{Network Activity} & Connection logs & Network devices \\
\textbf{USB Usage} & Device artifacts & Registry/logs \\
\end{longtable}
}

\textbf{ડિજિટલ પુરાવાના નિશાનો:}

\textbf{સિસ્ટમ સ્તર:}

\begin{itemize}
\tightlist
\item
  \textbf{Registry entries}: સિસ્ટમ ફેરફારો
\item
  \textbf{Log files}: પ્રવૃત્તિ રેકોર્ડ્સ
\item
  \textbf{Temporary files}: Process artifacts
\item
  \textbf{Metadata}: ફાઈલ માહિતી
\end{itemize}

\textbf{નેટવર્ક સ્તર:}

\begin{itemize}
\tightlist
\item
  \textbf{Router logs}: Traffic records
\item
  \textbf{Firewall logs}: Connection attempts
\item
  \textbf{DNS queries}: Website visits
\item
  \textbf{Packet captures}: Communication content
\end{itemize}

\textbf{એપ્લિકેશન સ્તર:}

\begin{itemize}
\tightlist
\item
  \textbf{Browser artifacts}: Web activity
\item
  \textbf{Application logs}: Software usage
\item
  \textbf{Database changes}: Data modifications
\item
  \textbf{Cache files}: Temporary storage
\end{itemize}

\textbf{ફોરેન્સિક અસરો:}

\begin{itemize}
\tightlist
\item
  \textbf{સંપૂર્ણ ગુનો નથી}: ડિજિટલ નિશાનો હંમેશા અસ્તિત્વમાં છે
\item
  \textbf{પુરાવાનું સ્થાન}: અનેક સ્રોતો ઉપલબ્ધ
\item
  \textbf{સમર્થન}: અનેક નિશાન validation
\item
  \textbf{Timeline પુનર્નિર્માણ}: પ્રવૃત્તિ ક્રમ
\end{itemize}

\end{solutionbox}
\begin{mnemonicbox}
``Every Exchange Exists Electronically''

\end{mnemonicbox}
\begin{center}\rule{0.5\linewidth}{0.5pt}\end{center}

\subsection*{પ્રશ્ન 5(ક) [7
ગુણ]}\label{uxaaauxab0uxab6uxaa8-5uxa95-7-uxa97uxaa3}

\textbf{Digital Evidence સાચવવા માટેના મહત્વના પગલાઓની યાદી બનાવો.}

\begin{solutionbox}

\textbf{ડિજિટલ પુરાવા સંરક્ષણ પ્રક્રિયા:}

\begin{center}
\textbf{Mermaid Diagram (Code)}
\begin{verbatim}
{Shaded}
{Highlighting}[]
graph LR
    A[Digital Evidence] {-{-}{} B[Identification]}
    B {-{-}{} C[Collection]}
    C {-{-}{} D[Preservation]}
    D {-{-}{} E[Analysis]}
    E {-{-}{} F[Presentation]}
{Highlighting}
{Shaded}
\end{verbatim}
\end{center}

\textbf{મહત્વપૂર્ણ સંરક્ષણ પગલાં:}

{\def\LTcaptype{none} % do not increment counter
\begin{longtable}[]{@{}
  >{\raggedright\arraybackslash}p{(\linewidth - 6\tabcolsep) * \real{0.1935}}
  >{\raggedright\arraybackslash}p{(\linewidth - 6\tabcolsep) * \real{0.2903}}
  >{\raggedright\arraybackslash}p{(\linewidth - 6\tabcolsep) * \real{0.2903}}
  >{\raggedright\arraybackslash}p{(\linewidth - 6\tabcolsep) * \real{0.2258}}@{}}
\toprule\noalign{}
\begin{minipage}[b]{\linewidth}\raggedright
પગલું
\end{minipage} & \begin{minipage}[b]{\linewidth}\raggedright
પ્રક્રિયા
\end{minipage} & \begin{minipage}[b]{\linewidth}\raggedright
હેતુ
\end{minipage} & \begin{minipage}[b]{\linewidth}\raggedright
સાધનો
\end{minipage} \\
\midrule\noalign{}
\endhead
\bottomrule\noalign{}
\endlastfoot
\textbf{1. ઓળખ} & સંભવિત પુરાવા શોધવા & અવકાશ નક્કી કરવો & દ્રશ્ય નિરીક્ષણ \\
\textbf{2. દસ્તાવેજીકરણ} & દ્રશ્ય વિગતો record કરવી & Chain of custody જાળવવું
& ફોટોગ્રાફી, નોંધો \\
\textbf{3. અલગીકરણ} & દૂષણ અટકાવવું & અખંડિતતા જાળવવી & Network
disconnection \\
\textbf{4. Imaging} & Bit-by-bit copy બનાવવી & મૂળ સાચવવું & dd, FTK
Imager \\
\textbf{5. Hashing} & Integrity checks બનાવવા & અધિકૃતતા ચકાસવી & MD5,
SHA-256 \\
\textbf{6. સંગ્રહ} & સુરક્ષિત પુરાવા સંગ્રહ & છેડછાડ અટકાવવી & Write-protected
media \\
\textbf{7. Chain of Custody} & Handling દસ્તાવેજીકરણ & કાનૂની સ્વીકાર્યતા &
Forensic forms \\
\end{longtable}
}

\textbf{વિગતવાર સંરક્ષણ પદ્ધતિઓ:}

\textbf{ભૌતિક સંરક્ષણ:}

\begin{itemize}
\tightlist
\item
  \textbf{Power management}: યોગ્ય shutdown procedures
\item
  \textbf{Hardware protection}: Anti-static પગલાં
\item
  \textbf{પર્યાવરણીય નિયંત્રણ}: તાપમાન/ભેજ
\item
  \textbf{પ્રવેશ પ્રતિબંધ}: અધિકૃત કર્મચારીઓ માત્ર
\end{itemize}

\textbf{તર્કસંગત સંરક્ષણ:}

\begin{itemize}
\tightlist
\item
  \textbf{Bit-stream imaging}: હૂબહૂ disk copies
\item
  \textbf{Hash verification}: અખંડિતતા પુષ્ટિ
\item
  \textbf{Write blocking}: ફેરફારો અટકાવવા
\item
  \textbf{Metadata preservation}: Timestamp સુરક્ષા
\end{itemize}

\textbf{કાનૂની સંરક્ષણ:}

\begin{itemize}
\tightlist
\item
  \textbf{દસ્તાવેજીકરણ ધોરણો}: વિગતવાર રેકોર્ડ્સ
\item
  \textbf{Chain of custody}: Handling log
\item
  \textbf{પ્રામાણિકતા}: પુરાવા ચકાસણી
\item
  \textbf{સ્વીકાર્યતા}: કોર્ટ જરૂરિયાતો
\end{itemize}

\textbf{શ્રેષ્ઠ પ્રથાઓ:}

\textbf{કરવા જેવું:}

\begin{itemize}
\tightlist
\item
  પુરાવાની \textbf{અનેક નકલો બનાવવી}
\item
  \textbf{Forensically sound સાધનો} ઉપયોગ કરવા
\item
  \textbf{દરેક ક્રિયા નોંધવી}
\item
  \textbf{Chain of custody જાળવવું}
\item
  Hash સાથે \textbf{અખંડિતતા ચકાસવી}
\end{itemize}

\textbf{ન કરવા જેવું:}

\begin{itemize}
\tightlist
\item
  \textbf{કદી મૂળ પુરાવા પર કામ ન કરવું}
\item
  દ્રશ્યનું \textbf{દૂષણ ટાળવું}
\item
  Suspect systems ને \textbf{power on ન કરવા}
\item
  પુરાવાને \textbf{modify ન કરવા}
\item
  \textbf{Chain of custody તોડવું નહીં}
\end{itemize}

\textbf{ગુણવત્તા ખાતરી:}

{\def\LTcaptype{none} % do not increment counter
\begin{longtable}[]{@{}lll@{}}
\toprule\noalign{}
ચેક & ચકાસણી પદ્ધતિ & આવર્તન \\
\midrule\noalign{}
\endhead
\bottomrule\noalign{}
\endlastfoot
\textbf{Hash Validation} & Original vs copy સરખામણી & પહેલાં/પછી
operations \\
\textbf{Tool Calibration} & Tool accuracy ચકાસવી & Regular intervals \\
\textbf{Process Review} & Procedures audit કરવી & Case completion \\
\textbf{Documentation Check} & સંપૂર્ણતા ચકાસવી & દરેક પગલે \\
\end{longtable}
}

\textbf{કાનૂની વિચારણાઓ:}

\begin{itemize}
\tightlist
\item
  \textbf{સ્વીકાર્યતા જરૂરિયાતો}: કોર્ટ ધોરણો
\item
  \textbf{નિષ્ણાત સાક્ષી}: તકનીકી સમજૂતી
\item
  \textbf{ઉલટ-સવાલ}: પ્રક્રિયા validation
\item
  \textbf{ધોરણ અનુપાલન}: ઉદ્યોગ શ્રેષ્ઠ પ્રથાઓ
\end{itemize}

\end{solutionbox}
\begin{mnemonicbox}
``Proper Preservation Prevents Problems''

\end{mnemonicbox}
\begin{center}\rule{0.5\linewidth}{0.5pt}\end{center}

\subsection*{પ્રશ્ન 5(અ અથવા) [3
ગુણ]}\label{uxaaauxab0uxab6uxaa8-5uxa85-uxa85uxaa5uxab5-3-uxa97uxaa3}

\textbf{Malware forensics સમજાવો.}

\begin{solutionbox}

\textbf{વ્યાખ્યા:} Malware forensics માં infected systems પર તેના વર્તન, મૂળ
અને અસરને સમજવા માટે malicious software નું analysis કરવામાં આવે છે.

\textbf{મુખ્ય ઘટકો:}

{\def\LTcaptype{none} % do not increment counter
\begin{longtable}[]{@{}ll@{}}
\toprule\noalign{}
ઘટક & વર્ણન \\
\midrule\noalign{}
\endhead
\bottomrule\noalign{}
\endlastfoot
\textbf{Static Analysis} & Execution વિના malware ની તપાસ \\
\textbf{Dynamic Analysis} & Controlled environment માં malware ચલાવવું \\
\textbf{Code Analysis} & Malware code નું reverse engineering \\
\textbf{Behavioral Analysis} & Malware actions નો અભ્યાસ \\
\end{longtable}
}

\textbf{પ્રક્રિયા:}

\begin{itemize}
\tightlist
\item
  \textbf{Sample collection}: Malware acquisition
\item
  \textbf{Isolation}: Sandbox environment
\item
  \textbf{Analysis}: Behavior observation
\item
  \textbf{Reporting}: Findings documentation
\end{itemize}

\end{solutionbox}
\begin{mnemonicbox}
``Malware Makes Mysteries''

\end{mnemonicbox}
\begin{center}\rule{0.5\linewidth}{0.5pt}\end{center}

\subsection*{પ્રશ્ન 5(બ અથવા) [4
ગુણ]}\label{uxaaauxab0uxab6uxaa8-5uxaac-uxa85uxaa5uxab5-4-uxa97uxaa3}

\textbf{Digital Forensics તપાસમાં પુરાવા તરીકે CCTV શા માટે મહત્વની ભૂમિકા ભજવે
છે તે સમજાવો.}

\begin{solutionbox}

\textbf{Digital Forensics માં CCTV:}

\textbf{CCTV પુરાવાનું મહત્વ:}

{\def\LTcaptype{none} % do not increment counter
\begin{longtable}[]{@{}lll@{}}
\toprule\noalign{}
ભૂમિકા & વર્ણન & ફાયદો \\
\midrule\noalign{}
\endhead
\bottomrule\noalign{}
\endlastfoot
\textbf{દ્રશ્ય દસ્તાવેજીકરણ} & વાસ્તવિક ઘટનાઓ record કરે & Objective પુરાવા \\
\textbf{Timeline સ્થાપના} & પ્રવૃત્તિઓ timestamps કરે & કાલક્રમિક ક્રમ \\
\textbf{ઓળખ ચકાસણી} & Suspect images capture કરે & વ્યક્તિ ઓળખ \\
\textbf{સમર્થન} & અન્ય પુરાવાઓને support કરે & કેસ મજબૂત બનાવે \\
\end{longtable}
}

\textbf{ડિજિટલ પુરાવા ગુણધર્મો:}

\textbf{તકનીકી પાસાઓ:}

\begin{itemize}
\tightlist
\item
  \textbf{Metadata preservation}: Timestamp, camera ID, settings
\item
  \textbf{Chain of custody}: સુરક્ષિત handling procedures
\item
  \textbf{Format integrity}: મૂળ file structure maintenance
\item
  \textbf{Authentication}: Digital signatures, hash values
\end{itemize}

\textbf{ફોરેન્સિક મૂલ્ય:}

\begin{itemize}
\tightlist
\item
  \textbf{Real-time documentation}: Live incident recording
\item
  \textbf{Unbiased testimony}: યાંત્રિક સાક્ષી
\item
  \textbf{High resolution}: સ્પષ્ટ image quality
\item
  \textbf{Audio capture}: વધારાના sensory પુરાવા
\end{itemize}

\textbf{Analysis પદ્ધતિઓ:}

\begin{itemize}
\tightlist
\item
  \textbf{Frame-by-frame examination}: વિગતવાર scrutiny
\item
  \textbf{Enhancement techniques}: Image improvement
\item
  \textbf{Comparison analysis}: Multiple angle correlation
\item
  \textbf{Motion tracking}: Subject movement patterns
\end{itemize}

\textbf{કાનૂની સ્વીકાર્યતા:}

\begin{itemize}
\tightlist
\item
  \textbf{Authenticity verification}: Chain of custody
\item
  \textbf{Technical validation}: Equipment calibration
\item
  \textbf{Expert testimony}: Forensic analysis explanation
\item
  \textbf{Standard compliance}: Industry best practices
\end{itemize}

\end{solutionbox}
\begin{mnemonicbox}
``CCTV Captures Criminal Conduct Clearly''

\end{mnemonicbox}
\begin{center}\rule{0.5\linewidth}{0.5pt}\end{center}

\subsection*{પ્રશ્ન 5(ક અથવા) [7
ગુણ]}\label{uxaaauxab0uxab6uxaa8-5uxa95-uxa85uxaa5uxab5-7-uxa97uxaa3}

\textbf{Digital forensic તપાસના તબક્કાઓ સમજાવો.}

\begin{solutionbox}

\textbf{Digital Forensic તપાસ પ્રક્રિયા:}

\begin{center}
\textbf{Mermaid Diagram (Code)}
\begin{verbatim}
{Shaded}
{Highlighting}[]
graph LR
    A[Incident Response] {-{-}{} B[Evidence Identification]}
    B {-{-}{} C[Evidence Collection]}
    C {-{-}{} D[Evidence Preservation]}
    D {-{-}{} E[Evidence Analysis]}
    E {-{-}{} F[Documentation]}
    F {-{-}{} G[Presentation]}
{Highlighting}
{Shaded}
\end{verbatim}
\end{center}

\textbf{તબક્કાવાર વિભાજન:}

{\def\LTcaptype{none} % do not increment counter
\begin{longtable}[]{@{}llll@{}}
\toprule\noalign{}
તબક્કો & હેતુ & પ્રવૃત્તિઓ & આઉટપુટ \\
\midrule\noalign{}
\endhead
\bottomrule\noalign{}
\endlastfoot
\textbf{1. તૈયારી} & તત્પરતા સ્થાપના & Tool setup, training & Forensic
kit \\
\textbf{2. ઓળખ} & પુરાવાનું સ્થાન & Survey, documentation & Evidence list \\
\textbf{3. સંગ્રહ} & પુરાવા પ્રાપ્તિ & Imaging, copying & Digital copies \\
\textbf{4. સંરક્ષણ} & અખંડિતતા જાળવણી & Hashing, storage & Verified
evidence \\
\textbf{5. વિશ્લેષણ} & ડેટા તપાસ & Investigation, correlation & Findings \\
\textbf{6. પ્રસ્તુતિ} & પરિણામો સંપ્રેષણ & Reporting, testimony & Final
report \\
\end{longtable}
}

\textbf{વિગતવાર તબક્કો વિશ્લેષણ:}

\textbf{તબક્કો 1 - તૈયારી:}

\begin{itemize}
\tightlist
\item
  \textbf{Tool readiness}: Forensic software installation
\item
  \textbf{Hardware setup}: Write blockers, imaging devices
\item
  \textbf{Documentation templates}: Chain of custody forms
\item
  \textbf{Team preparation}: Role assignments, training
\item
  \textbf{Legal preparation}: Warrant requirements, permissions
\end{itemize}

\textbf{તબક્કો 2 - ઓળખ:}

\begin{itemize}
\tightlist
\item
  \textbf{Scene survey}: Evidence location mapping
\item
  \textbf{Device inventory}: System identification
\item
  \textbf{Volatile evidence}: Memory, network connections
\item
  \textbf{Priority assessment}: Critical evidence first
\item
  \textbf{Photography}: Scene documentation
\end{itemize}

\textbf{તબક્કો 3 - સંગ્રહ:}

\begin{itemize}
\tightlist
\item
  \textbf{Live system analysis}: Memory acquisition
\item
  \textbf{Disk imaging}: Bit-for-bit copies
\item
  \textbf{Network evidence}: Log files, packet captures
\item
  \textbf{Mobile devices}: Physical/logical extraction
\item
  \textbf{Cloud evidence}: Remote data acquisition
\end{itemize}

\textbf{તબક્કો 4 - સંરક્ષણ:}

\begin{itemize}
\tightlist
\item
  \textbf{Hash generation}: MD5, SHA-256 checksums
\item
  \textbf{Write protection}: Hardware/software blocking
\item
  \textbf{Storage security}: Tamper-evident containers
\item
  \textbf{Chain of custody}: Handling documentation
\item
  \textbf{Backup creation}: Multiple evidence copies
\end{itemize}

\textbf{તબક્કો 5 - વિશ્લેષણ:}

\begin{itemize}
\tightlist
\item
  \textbf{File system examination}: Directory structure analysis
\item
  \textbf{Deleted data recovery}: Unallocated space searching
\item
  \textbf{Timeline creation}: Event chronology
\item
  \textbf{Keyword searching}: Relevant content identification
\item
  \textbf{Pattern recognition}: Behavioral analysis
\end{itemize}

\textbf{તબક્કો 6 - પ્રસ્તુતિ:}

\begin{itemize}
\tightlist
\item
  \textbf{Report writing}: Findings documentation
\item
  \textbf{Visual aids}: Charts, diagrams, screenshots
\item
  \textbf{Expert testimony}: Court presentation
\item
  \textbf{Peer review}: Quality assurance
\item
  \textbf{Archive maintenance}: Case file storage
\end{itemize}

\textbf{શ્રેષ્ઠ પ્રથાઓ:}

\textbf{તકનીકી ધોરણો:}

\begin{itemize}
\tightlist
\item
  \textbf{Tool validation}: Regular calibration
\item
  \textbf{Methodology consistency}: Standard procedures
\item
  \textbf{Quality control}: Verification checks
\item
  \textbf{Documentation completeness}: Detailed records
\end{itemize}

\textbf{કાનૂની જરૂરિયાતો:}

\begin{itemize}
\tightlist
\item
  \textbf{Admissibility standards}: Court requirements
\item
  \textbf{Chain of custody}: Unbroken documentation
\item
  \textbf{Expert qualifications}: Professional certification
\item
  \textbf{Cross-examination preparation}: Defense against challenges
\end{itemize}

\textbf{ગુણવત્તા ખાતરી:}

{\def\LTcaptype{none} % do not increment counter
\begin{longtable}[]{@{}lll@{}}
\toprule\noalign{}
ચેક પોઈન્ટ & ચકાસણી & દસ્તાવેજીકરણ \\
\midrule\noalign{}
\endhead
\bottomrule\noalign{}
\endlastfoot
\textbf{Evidence integrity} & Hash comparison & Verification logs \\
\textbf{Tool reliability} & Calibration tests & Certification records \\
\textbf{Process compliance} & Standard adherence & Procedure
checklists \\
\textbf{Report accuracy} & Peer review & Review signatures \\
\end{longtable}
}

\textbf{સામાન્ય પડકારો:}

\begin{itemize}
\tightlist
\item
  \textbf{Encryption}: Data protection barriers
\item
  \textbf{Anti-forensics}: Evidence hiding techniques
\item
  \textbf{Volume}: Large data sets
\item
  \textbf{Volatility}: Temporary evidence
\item
  \textbf{Legal complexity}: Jurisdiction issues
\end{itemize}

\textbf{સફળતાના પરિબળો:}

\begin{itemize}
\tightlist
\item
  \textbf{Systematic approach}: Methodical investigation
\item
  \textbf{Technical expertise}: Skilled personnel
\item
  \textbf{Proper tools}: Adequate resources
\item
  \textbf{Legal knowledge}: Compliance understanding
\item
  \textbf{Documentation discipline}: Thorough records
\end{itemize}

\end{solutionbox}
\begin{mnemonicbox}
``Proper Planning Prevents Poor Performance''
(Preparation, Preservation, Processing, Presentation, Proof)

\end{mnemonicbox}

\end{document}
