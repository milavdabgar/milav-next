\documentclass[10pt,a4paper]{article}

% content/resources/templates/preamble.tex
\usepackage[margin=0.6in]{geometry}
\author{Milav Dabgar}
\usepackage{amsmath,amssymb,amsthm}
\usepackage{booktabs}
\usepackage{multirow}
\usepackage{xcolor}
\usepackage{tcolorbox}
\tcbuselibrary{breakable,skins}
\usepackage[colorlinks=true,linkcolor=blue]{hyperref}
\usepackage{titlesec}
\usepackage{enumitem}
\usepackage{tikz}
\usepackage{pgfplots}
\usepackage{circuitikz}
\usepackage[version=4]{mhchem}
\usepackage{longtable}
\usepackage{array}
\usepackage{float}
\usepackage{caption}
\usepackage{listings}

\lstset{
  basicstyle=\small\ttfamily,
  breaklines=true,
  breakatwhitespace=false,
  postbreak=\mbox{\textcolor{red}{$\hookrightarrow$}\space},
  float=false,
  numbers=left,
  numberstyle=\tiny\color{gray},
  numbersep=10pt,
  xleftmargin=2em,
  keywordstyle=\color{blue},
  commentstyle=\color{green!60!black},
  stringstyle=\color{purple},
  backgroundcolor=\color{gray!5},
  showstringspaces=false,
  tabsize=2,
  captionpos=b,
  keepspaces=true,
  columns=flexible
}

\pgfplotsset{compat=1.18}
\usetikzlibrary{shapes,arrows,positioning,calc,patterns,decorations.pathmorphing,decorations.markings,arrows.meta}

% Color scheme
\definecolor{headcolor}{RGB}{0,102,204}
\definecolor{keycolor}{RGB}{220,20,60}
\definecolor{solutioncolor}{RGB}{34,139,34}
\definecolor{mnemoniccolor}{RGB}{148,0,211}
\definecolor{codecolor}{RGB}{0,0,100}

% Spacing
\setlength{\parskip}{3pt}
\setlist[itemize]{nosep}
\setlist[enumerate]{nosep}

% Title formatting
\titleformat{\section}{\Large\bfseries\color{headcolor}}{\thesection}{1em}{}
\titleformat{\subsection}{\large\bfseries\color{headcolor}}{\thesubsection}{1em}{}

% Pandoc tightlist compatibility
\providecommand{\tightlist}{%
  \setlength{\itemsep}{0pt}\setlength{\parskip}{0pt}}

% Pandoc longtable compatibility
\newcounter{none}
\def\thenone{}


% content/resources/templates/english-boxes.tex
% This file is currently empty - it exists to maintain consistency with the import structure.
% Add custom environments here if needed in the future.


\begin{document}

\begin{center}
{\Huge\bfseries\color{headcolor} Subject Name Solutions}\\[5pt]
{\LARGE 4361601 -- Summer 2025}\\[3pt]
{\large Semester 1 Study Material}\\[3pt]
{\normalsize\textit{Detailed Solutions and Explanations}}
\end{center}

\vspace{10pt}

\subsection*{Question 1(a) [3 marks]}\label{q1a}

\textbf{Give comparison between Public key and Private Key
cryptography.}

\begin{solutionbox}

{\def\LTcaptype{none} % do not increment counter
\begin{longtable}[]{@{}
  >{\raggedright\arraybackslash}p{(\linewidth - 4\tabcolsep) * \real{0.1404}}
  >{\raggedright\arraybackslash}p{(\linewidth - 4\tabcolsep) * \real{0.4386}}
  >{\raggedright\arraybackslash}p{(\linewidth - 4\tabcolsep) * \real{0.4211}}@{}}
\toprule\noalign{}
\begin{minipage}[b]{\linewidth}\raggedright
Aspect
\end{minipage} & \begin{minipage}[b]{\linewidth}\raggedright
Private Key Cryptography
\end{minipage} & \begin{minipage}[b]{\linewidth}\raggedright
Public Key Cryptography
\end{minipage} \\
\midrule\noalign{}
\endhead
\bottomrule\noalign{}
\endlastfoot
\textbf{Key Management} & Same key for encryption/decryption & Different
keys for encryption/decryption \\
\textbf{Key Distribution} & Secure channel required & No secure channel
needed \\
\textbf{Speed} & Fast processing & Slower than private key \\
\textbf{Security Level} & High if key is secret & High mathematical
security \\
\textbf{Example} & DES, AES & RSA, ECC \\
\end{longtable}
}

\end{solutionbox}
\begin{mnemonicbox}
``Private Personal, Public Pair''

\end{mnemonicbox}
\begin{center}\rule{0.5\linewidth}{0.5pt}\end{center}

\subsection*{Question 1(b) [4 marks]}\label{q1b}

\textbf{Explain CIA Triad in detail.}

\begin{solutionbox}

CIA Triad is the foundation of information security with three core
principles:

\textbf{Diagram:}

\begin{center}
\textbf{Mermaid Diagram (Code)}
\begin{verbatim}
{Shaded}
{Highlighting}[]
graph TD
    A[CIA Triad] {-{-}{} B[Confidentiality]}
    A {-{-}{} C[Integrity]}
    A {-{-}{} D[Availability]}
    B {-{-}{} E[Data Privacy]}
    C {-{-}{} F[Data Accuracy]}
    D {-{-}{} G[System Access]}
{Highlighting}
{Shaded}
\end{verbatim}
\end{center}

\begin{itemize}
\tightlist
\item
  \textbf{Confidentiality}: Ensures data is accessible only to
  authorized users
\item
  \textbf{Integrity}: Maintains accuracy and completeness of data
\item
  \textbf{Availability}: Ensures systems are accessible when needed
\end{itemize}

\end{solutionbox}
\begin{mnemonicbox}
``Can I Access'' (Confidentiality, Integrity,
Availability)

\end{mnemonicbox}
\begin{center}\rule{0.5\linewidth}{0.5pt}\end{center}

\subsection*{Question 1(c) [7 marks]}\label{q1c}

\textbf{Explain Md5 algorithm steps.}

\begin{solutionbox}

MD5 (Message Digest 5) is a cryptographic hash function producing
128-bit hash value.

\textbf{Algorithm Steps:}

{\def\LTcaptype{none} % do not increment counter
\begin{longtable}[]{@{}lll@{}}
\toprule\noalign{}
Step & Process & Description \\
\midrule\noalign{}
\endhead
\bottomrule\noalign{}
\endlastfoot
1 & \textbf{Padding} & Add bits to make message length ≡ 448 (mod
512) \\
2 & \textbf{Length Addition} & Append 64-bit length of original
message \\
3 & \textbf{Initialize Buffers} & Set four 32-bit buffers (A, B, C,
D) \\
4 & \textbf{Process Blocks} & Process message in 512-bit blocks \\
5 & \textbf{Round Functions} & Apply 4 rounds of 16 operations each \\
\end{longtable}
}

\textbf{Code Block:}

\begin{verbatim}
\# MD5 Processing Steps
def md5\_process():
    \# Step 1: Padding
    padded\_message = original + padding\_bits
    \# Step 2: Process in 512{-bit chunks  }
    for chunk in chunks:
        \# Step 3: Apply round functions
        result = round\_functions(chunk)
    return final\_hash
\end{verbatim}

\begin{itemize}
\tightlist
\item
  \textbf{Round 1}: F(X,Y,Z) = (X\wedgeY) \vee (\negX\wedgeZ)
\item
  \textbf{Round 2}: G(X,Y,Z) = (X\wedgeZ) \vee (Y\wedge\negZ)
\item
  \textbf{Round 3}: H(X,Y,Z) = X\oplusY\oplusZ
\item
  \textbf{Round 4}: I(X,Y,Z) = Y\oplus(X\vee\negZ)
\end{itemize}

\end{solutionbox}
\begin{mnemonicbox}
``My Data Needs Proper Processing'' (Message, Digest,
Needs, Proper, Processing)

\end{mnemonicbox}
\begin{center}\rule{0.5\linewidth}{0.5pt}\end{center}

\subsection*{Question 1(c OR) [7
marks]}\label{question-1c-or-7-marks}

\textbf{List inventors of RSA. Write steps of RSA algorithm.}

\begin{solutionbox}

\textbf{RSA Inventors:}

\begin{itemize}
\tightlist
\item
  \textbf{Ron Rivest} (MIT)
\item
  \textbf{Adi Shamir} (MIT)
\item
  \textbf{Leonard Adleman} (MIT)
\end{itemize}

\textbf{RSA Algorithm Steps:}

{\def\LTcaptype{none} % do not increment counter
\begin{longtable}[]{@{}lll@{}}
\toprule\noalign{}
Step & Process & Formula \\
\midrule\noalign{}
\endhead
\bottomrule\noalign{}
\endlastfoot
1 & \textbf{Select Primes} & Choose p, q (large primes) \\
2 & \textbf{Calculate n} & n = p \times q \\
3 & \textbf{Calculate φ(n)} & φ(n) = (p-1) \times (q-1) \\
4 & \textbf{Choose e} & gcd(e, φ(n)) = 1 \\
5 & \textbf{Calculate d} & d \times e ≡ 1 (mod φ(n)) \\
6 & \textbf{Encryption} & C = M\^{}e mod n \\
7 & \textbf{Decryption} & M = C\^{}d mod n \\
\end{longtable}
}

\textbf{Key Pairs:}

\begin{itemize}
\tightlist
\item
  \textbf{Public Key}: (n, e)
\item
  \textbf{Private Key}: (n, d)
\end{itemize}

\end{solutionbox}
\begin{mnemonicbox}
``RSA: Rivest Shamir Adleman''

\end{mnemonicbox}
\begin{center}\rule{0.5\linewidth}{0.5pt}\end{center}

\subsection*{Question 2(a) [3 marks]}\label{q2a}

\textbf{Define: Firewall. List limitations of firewall.}

\begin{solutionbox}

\textbf{Definition:} Firewall is a network security device that monitors
and controls incoming/outgoing network traffic based on predetermined
security rules.

\textbf{Limitations:}

{\def\LTcaptype{none} % do not increment counter
\begin{longtable}[]{@{}
  >{\raggedright\arraybackslash}p{(\linewidth - 2\tabcolsep) * \real{0.4800}}
  >{\raggedright\arraybackslash}p{(\linewidth - 2\tabcolsep) * \real{0.5200}}@{}}
\toprule\noalign{}
\begin{minipage}[b]{\linewidth}\raggedright
Limitation
\end{minipage} & \begin{minipage}[b]{\linewidth}\raggedright
Description
\end{minipage} \\
\midrule\noalign{}
\endhead
\bottomrule\noalign{}
\endlastfoot
\textbf{Internal Threats} & Cannot protect against insider attacks \\
\textbf{Application Layer} & Limited protection against
application-specific attacks \\
\textbf{Performance} & Can slow down network traffic \\
\textbf{Configuration} & Requires proper setup and maintenance \\
\textbf{Encrypted Traffic} & Cannot inspect encrypted content
effectively \\
\end{longtable}
}

\end{solutionbox}
\begin{mnemonicbox}
``Fire Walls Limit Internal Protection''

\end{mnemonicbox}
\begin{center}\rule{0.5\linewidth}{0.5pt}\end{center}

\subsection*{Question 2(b) [4 marks]}\label{q2b}

\textbf{Sketch IPsec Tunnel Mode and Transport mode.}

\begin{solutionbox}

\textbf{IPsec Modes Comparison:}

\begin{verbatim}
Transport Mode:
+{-{-}{-}{-}{-}{-}{-}{-}{-}{-}+{-}{-}{-}{-}{-}{-}{-}{-}{-}{-}+{-}{-}{-}{-}{-}{-}{-}{-}{-}{-}+}
| Original | IPsec    | Original |
| IP Header| Header   | Payload  |
+{-{-}{-}{-}{-}{-}{-}{-}{-}{-}+{-}{-}{-}{-}{-}{-}{-}{-}{-}{-}+{-}{-}{-}{-}{-}{-}{-}{-}{-}{-}+}

Tunnel Mode:
+{-{-}{-}{-}{-}{-}{-}{-}{-}{-}+{-}{-}{-}{-}{-}{-}{-}{-}{-}{-}+{-}{-}{-}{-}{-}{-}{-}{-}{-}{-}+{-}{-}{-}{-}{-}{-}{-}{-}{-}{-}+}
| New IP   | IPsec    | Original | Original |
| Header   | Header   | IP Header| Payload  |
+{-{-}{-}{-}{-}{-}{-}{-}{-}{-}+{-}{-}{-}{-}{-}{-}{-}{-}{-}{-}+{-}{-}{-}{-}{-}{-}{-}{-}{-}{-}+{-}{-}{-}{-}{-}{-}{-}{-}{-}{-}+}
\end{verbatim}

\textbf{Key Differences:}

{\def\LTcaptype{none} % do not increment counter
\begin{longtable}[]{@{}lll@{}}
\toprule\noalign{}
Aspect & Transport Mode & Tunnel Mode \\
\midrule\noalign{}
\endhead
\bottomrule\noalign{}
\endlastfoot
\textbf{Protection} & Payload only & Entire packet \\
\textbf{Use Case} & End-to-end & Gateway-to-gateway \\
\textbf{Overhead} & Lower & Higher \\
\textbf{IP Header} & Original preserved & New header added \\
\end{longtable}
}

\end{solutionbox}
\begin{mnemonicbox}
``Transport Travels, Tunnel Total''

\end{mnemonicbox}
\begin{center}\rule{0.5\linewidth}{0.5pt}\end{center}

\subsection*{Question 2(c) [7 marks]}\label{q2c}

\textbf{Explain various types of Active \& Passive attacks in detail.}

\begin{solutionbox}

\textbf{Attack Classification:}

\begin{center}
\textbf{Mermaid Diagram (Code)}
\begin{verbatim}
{Shaded}
{Highlighting}[]
graph TD
    A[Network Attacks] {-{-}{} B[Active Attacks]}
    A {-{-}{} C[Passive Attacks]}
    B {-{-}{} D[Modification]}
    B {-{-}{} E[Fabrication]}
    B {-{-}{} F[Interruption]}
    C {-{-}{} G[Eavesdropping]}
    C {-{-}{} H[Traffic Analysis]}
{Highlighting}
{Shaded}
\end{verbatim}
\end{center}

\textbf{Active Attacks:}

{\def\LTcaptype{none} % do not increment counter
\begin{longtable}[]{@{}lll@{}}
\toprule\noalign{}
Type & Description & Example \\
\midrule\noalign{}
\endhead
\bottomrule\noalign{}
\endlastfoot
\textbf{Masquerade} & Impersonating another entity & Fake identity \\
\textbf{Replay} & Retransmitting captured data & Session replay \\
\textbf{Modification} & Altering message content & Data tampering \\
\textbf{DoS} & Denying service availability & Server flooding \\
\end{longtable}
}

\textbf{Passive Attacks:}

{\def\LTcaptype{none} % do not increment counter
\begin{longtable}[]{@{}
  >{\raggedright\arraybackslash}p{(\linewidth - 4\tabcolsep) * \real{0.2222}}
  >{\raggedright\arraybackslash}p{(\linewidth - 4\tabcolsep) * \real{0.4815}}
  >{\raggedright\arraybackslash}p{(\linewidth - 4\tabcolsep) * \real{0.2963}}@{}}
\toprule\noalign{}
\begin{minipage}[b]{\linewidth}\raggedright
Type
\end{minipage} & \begin{minipage}[b]{\linewidth}\raggedright
Description
\end{minipage} & \begin{minipage}[b]{\linewidth}\raggedright
Impact
\end{minipage} \\
\midrule\noalign{}
\endhead
\bottomrule\noalign{}
\endlastfoot
\textbf{Eavesdropping} & Listening to communications & Data theft \\
\textbf{Traffic Analysis} & Analyzing communication patterns & Privacy
breach \\
\textbf{Monitoring} & Observing network activity & Information
gathering \\
\end{longtable}
}

\begin{itemize}
\tightlist
\item
  \textbf{Active attacks} modify system resources or data
\item
  \textbf{Passive attacks} observe and collect information
\item
  \textbf{Detection}: Active attacks easier to detect than passive
\end{itemize}

\end{solutionbox}
\begin{mnemonicbox}
``Active Acts, Passive Peeks''

\end{mnemonicbox}
\begin{center}\rule{0.5\linewidth}{0.5pt}\end{center}

\subsection*{Question 2(a OR) [3
marks]}\label{question-2a-or-3-marks}

\textbf{Define: Digital Signature. Also discuss various application
areas of Digital Signature.}

\begin{solutionbox}

\textbf{Definition:} Digital Signature is a cryptographic technique that
validates authenticity and integrity of digital messages or documents
using public key cryptography.

\textbf{Application Areas:}

{\def\LTcaptype{none} % do not increment counter
\begin{longtable}[]{@{}ll@{}}
\toprule\noalign{}
Area & Use Case \\
\midrule\noalign{}
\endhead
\bottomrule\noalign{}
\endlastfoot
\textbf{E-commerce} & Online transactions, contracts \\
\textbf{Banking} & Electronic fund transfers, cheques \\
\textbf{Government} & Digital certificates, official documents \\
\textbf{Healthcare} & Patient records, prescriptions \\
\textbf{Legal} & Electronic contracts, court documents \\
\end{longtable}
}

\end{solutionbox}
\begin{mnemonicbox}
``Digital Documents Demand Authentic Approval''

\end{mnemonicbox}
\begin{center}\rule{0.5\linewidth}{0.5pt}\end{center}

\subsection*{Question 2(b OR) [4
marks]}\label{question-2b-or-4-marks}

\textbf{Differentiate HTTP \& HTTPS.}

\begin{solutionbox}

{\def\LTcaptype{none} % do not increment counter
\begin{longtable}[]{@{}
  >{\raggedright\arraybackslash}p{(\linewidth - 4\tabcolsep) * \real{0.4583}}
  >{\raggedright\arraybackslash}p{(\linewidth - 4\tabcolsep) * \real{0.2500}}
  >{\raggedright\arraybackslash}p{(\linewidth - 4\tabcolsep) * \real{0.2917}}@{}}
\toprule\noalign{}
\begin{minipage}[b]{\linewidth}\raggedright
Parameter
\end{minipage} & \begin{minipage}[b]{\linewidth}\raggedright
HTTP
\end{minipage} & \begin{minipage}[b]{\linewidth}\raggedright
HTTPS
\end{minipage} \\
\midrule\noalign{}
\endhead
\bottomrule\noalign{}
\endlastfoot
\textbf{Security} & No encryption & SSL/TLS encryption \\
\textbf{Port} & 80 & 443 \\
\textbf{Protocol} & Hypertext Transfer Protocol & HTTP + SSL/TLS \\
\textbf{Data Protection} & Plain text & Encrypted \\
\textbf{Authentication} & No server verification & Server certificate
validation \\
\textbf{Speed} & Faster & Slightly slower \\
\textbf{URL Prefix} & http:// & https:// \\
\end{longtable}
}

\textbf{Diagram:}

\begin{verbatim}
HTTP:
Client {-{-}{-}{-}Plain Text{-}{-}{-}{-} Server}

HTTPS:
Client {-{-}{-}{-}Encrypted{-}{-}{-}{-}{-} Server}
       {{-}{-}{-}Certificate{-}{-}{-}{-}}
\end{verbatim}

\end{solutionbox}
\begin{mnemonicbox}
``HTTPS Has Security''

\end{mnemonicbox}
\begin{center}\rule{0.5\linewidth}{0.5pt}\end{center}

\subsection*{Question 2(c OR) [7
marks]}\label{question-2c-or-7-marks}

\textbf{Define: Malicious software. Explain Virus, Worm, Keylogger,
Trojans in detail.}

\begin{solutionbox}

\textbf{Definition:} Malicious software (Malware) is any software
designed to harm, exploit, or gain unauthorized access to computer
systems.

\textbf{Types of Malware:}

{\def\LTcaptype{none} % do not increment counter
\begin{longtable}[]{@{}
  >{\raggedright\arraybackslash}p{(\linewidth - 4\tabcolsep) * \real{0.1875}}
  >{\raggedright\arraybackslash}p{(\linewidth - 4\tabcolsep) * \real{0.5000}}
  >{\raggedright\arraybackslash}p{(\linewidth - 4\tabcolsep) * \real{0.3125}}@{}}
\toprule\noalign{}
\begin{minipage}[b]{\linewidth}\raggedright
Type
\end{minipage} & \begin{minipage}[b]{\linewidth}\raggedright
Characteristics
\end{minipage} & \begin{minipage}[b]{\linewidth}\raggedright
Behavior
\end{minipage} \\
\midrule\noalign{}
\endhead
\bottomrule\noalign{}
\endlastfoot
\textbf{Virus} & Requires host file & Attaches to programs, spreads when
executed \\
\textbf{Worm} & Self-replicating & Spreads independently through
networks \\
\textbf{Keylogger} & Records keystrokes & Steals passwords and sensitive
data \\
\textbf{Trojan} & Disguised as legitimate & Provides backdoor access to
attackers \\
\end{longtable}
}

\textbf{Detailed Explanation:}

\textbf{Virus:}

\begin{itemize}
\tightlist
\item
  Requires host program to execute
\item
  Spreads through infected files
\item
  Can corrupt or delete data
\end{itemize}

\textbf{Worm:}

\begin{itemize}
\tightlist
\item
  Self-propagating malware
\item
  Exploits network vulnerabilities
\item
  Consumes network bandwidth
\end{itemize}

\textbf{Keylogger:}

\begin{itemize}
\tightlist
\item
  Records user keystrokes
\item
  Captures login credentials
\item
  Can be hardware or software-based
\end{itemize}

\textbf{Trojan:}

\begin{itemize}
\tightlist
\item
  Appears as legitimate software
\item
  Creates backdoor for remote access
\item
  Does not self-replicate
\end{itemize}

\end{solutionbox}
\begin{mnemonicbox}
``Viruses Visit, Worms Wander, Keys Captured, Trojans
Trick''

\end{mnemonicbox}
\begin{center}\rule{0.5\linewidth}{0.5pt}\end{center}

\subsection*{Question 3(a) [3 marks]}\label{q3a}

\textbf{Define: Cybercrime. Also discuss needs of Cyber Law.}

\begin{solutionbox}

\textbf{Definition:} Cybercrime refers to criminal activities carried
out using computers, networks, or digital devices as tools or targets.

\textbf{Needs of Cyber Law:}

{\def\LTcaptype{none} % do not increment counter
\begin{longtable}[]{@{}ll@{}}
\toprule\noalign{}
Need & Justification \\
\midrule\noalign{}
\endhead
\bottomrule\noalign{}
\endlastfoot
\textbf{Legal Framework} & Establish clear definitions of cyber
offenses \\
\textbf{Jurisdiction} & Define authority across geographical
boundaries \\
\textbf{Evidence} & Guidelines for digital evidence collection \\
\textbf{Punishment} & Deterrent measures for cybercriminals \\
\textbf{Protection} & Safeguard individual and organizational rights \\
\end{longtable}
}

\end{solutionbox}
\begin{mnemonicbox}
``Cyber Laws Create Legal Protection''

\end{mnemonicbox}
\begin{center}\rule{0.5\linewidth}{0.5pt}\end{center}

\subsection*{Question 3(b) [4 marks]}\label{q3b}

\textbf{Explain Cyber spying and Cyber theft.}

\begin{solutionbox}

\textbf{Cyber Spying:}

\begin{itemize}
\tightlist
\item
  \textbf{Definition}: Unauthorized surveillance of digital
  communications and activities
\item
  \textbf{Methods}: Malware, phishing, social engineering
\item
  \textbf{Targets}: Government, corporate secrets, personal data
\item
  \textbf{Impact}: National security threats, competitive disadvantage
\end{itemize}

\textbf{Cyber Theft:}

\begin{itemize}
\tightlist
\item
  \textbf{Definition}: Unauthorized taking of digital assets or
  information
\item
  \textbf{Types}: Identity theft, financial fraud, intellectual property
  theft
\item
  \textbf{Methods}: Hacking, social engineering, insider threats
\item
  \textbf{Consequences}: Financial loss, reputation damage
\end{itemize}

\textbf{Comparison Table:}

{\def\LTcaptype{none} % do not increment counter
\begin{longtable}[]{@{}lll@{}}
\toprule\noalign{}
Aspect & Cyber Spying & Cyber Theft \\
\midrule\noalign{}
\endhead
\bottomrule\noalign{}
\endlastfoot
\textbf{Purpose} & Information gathering & Asset acquisition \\
\textbf{Detection} & Often undetected & May be noticed \\
\textbf{Duration} & Long-term monitoring & One-time or periodic \\
\textbf{Motivation} & Intelligence/espionage & Financial gain \\
\end{longtable}
}

\end{solutionbox}
\begin{mnemonicbox}
``Spies Spy, Thieves Take''

\end{mnemonicbox}
\begin{center}\rule{0.5\linewidth}{0.5pt}\end{center}

\subsection*{Question 3(c) [7 marks]}\label{q3c}

\textbf{Explain article section 66 of cyber law.}

\begin{solutionbox}

\textbf{Section 66 - Computer Related Offences (IT Act 2008):}

\textbf{Key Provisions:}

{\def\LTcaptype{none} % do not increment counter
\begin{longtable}[]{@{}
  >{\raggedright\arraybackslash}p{(\linewidth - 4\tabcolsep) * \real{0.3824}}
  >{\raggedright\arraybackslash}p{(\linewidth - 4\tabcolsep) * \real{0.2647}}
  >{\raggedright\arraybackslash}p{(\linewidth - 4\tabcolsep) * \real{0.3529}}@{}}
\toprule\noalign{}
\begin{minipage}[b]{\linewidth}\raggedright
Sub-section
\end{minipage} & \begin{minipage}[b]{\linewidth}\raggedright
Offense
\end{minipage} & \begin{minipage}[b]{\linewidth}\raggedright
Punishment
\end{minipage} \\
\midrule\noalign{}
\endhead
\bottomrule\noalign{}
\endlastfoot
\textbf{66(1)} & Dishonestly/fraudulently computer resource damage & Up
to 3 years imprisonment + fine up to ₹5 lakh \\
\textbf{66A} & Sending offensive messages & Up to 3 years + fine \\
\textbf{66B} & Receiving stolen computer resource & Up to 3 years + fine
up to ₹1 lakh \\
\textbf{66C} & Identity theft & Up to 3 years + fine up to ₹1 lakh \\
\textbf{66D} & Cheating by personation using computer & Up to 3 years +
fine up to ₹1 lakh \\
\textbf{66E} & Violation of privacy & Up to 3 years + fine up to ₹2
lakh \\
\textbf{66F} & Cyber terrorism & Life imprisonment \\
\end{longtable}
}

\textbf{Detailed Coverage:}

\textbf{Section 66 Main Offenses:}

\begin{itemize}
\tightlist
\item
  \textbf{Hacking}: Unauthorized access to computer systems
\item
  \textbf{Data Theft}: Stealing or copying data without permission
\item
  \textbf{System Damage}: Destroying or altering computer data
\item
  \textbf{Virus Introduction}: Introducing malicious code
\end{itemize}

\textbf{Elements Required:}

\begin{itemize}
\tightlist
\item
  \textbf{Intent}: Dishonest or fraudulent intention
\item
  \textbf{Access}: Without permission of owner
\item
  \textbf{Damage}: Causing harm to system or data
\item
  \textbf{Knowledge}: Awareness of unauthorized access
\end{itemize}

\textbf{Legal Framework:}

\begin{itemize}
\tightlist
\item
  \textbf{Cognizable}: Police can arrest without warrant
\item
  \textbf{Non-bailable}: Bail at court's discretion
\item
  \textbf{Evidence}: Digital evidence admissible in court
\end{itemize}

\end{solutionbox}
\begin{mnemonicbox}
``Section 66 Stops Cyber Sins''

\end{mnemonicbox}
\begin{center}\rule{0.5\linewidth}{0.5pt}\end{center}

\subsection*{Question 3(a OR) [3
marks]}\label{question-3a-or-3-marks}

\textbf{Explain Cyber terrorism.}

\begin{solutionbox}

\textbf{Definition:} Cyber terrorism involves the use of digital
technologies to create fear, disruption, or harm for political,
religious, or ideological purposes.

\textbf{Characteristics:}

{\def\LTcaptype{none} % do not increment counter
\begin{longtable}[]{@{}ll@{}}
\toprule\noalign{}
Aspect & Description \\
\midrule\noalign{}
\endhead
\bottomrule\noalign{}
\endlastfoot
\textbf{Target} & Critical infrastructure, government systems \\
\textbf{Method} & DDoS attacks, system infiltration, data destruction \\
\textbf{Motivation} & Political, religious, ideological goals \\
\textbf{Impact} & Public fear, economic disruption, national security \\
\end{longtable}
}

\textbf{Examples:}

\begin{itemize}
\tightlist
\item
  Power grid attacks
\item
  Transportation system disruption
\item
  Financial system targeting
\end{itemize}

\end{solutionbox}
\begin{mnemonicbox}
``Terror Through Technology''

\end{mnemonicbox}
\begin{center}\rule{0.5\linewidth}{0.5pt}\end{center}

\subsection*{Question 3(b OR) [4
marks]}\label{question-3b-or-4-marks}

\textbf{Explain Cyber bullying \& Cyber stalking.}

\begin{solutionbox}

\textbf{Cyber Bullying:}

\begin{itemize}
\tightlist
\item
  \textbf{Definition}: Using digital platforms to harass, intimidate, or
  harm others
\item
  \textbf{Platforms}: Social media, messaging apps, online forums
\item
  \textbf{Characteristics}: Repetitive, intentional harm, power
  imbalance
\item
  \textbf{Impact}: Psychological trauma, depression, social isolation
\end{itemize}

\textbf{Cyber Stalking:}

\begin{itemize}
\tightlist
\item
  \textbf{Definition}: Persistent online harassment causing fear or
  emotional distress
\item
  \textbf{Methods}: Unwanted messages, tracking, identity theft
\item
  \textbf{Duration}: Long-term, continuous behavior
\item
  \textbf{Legal}: Criminal offense in many jurisdictions
\end{itemize}

\textbf{Comparison:}

{\def\LTcaptype{none} % do not increment counter
\begin{longtable}[]{@{}lll@{}}
\toprule\noalign{}
Aspect & Cyber Bullying & Cyber Stalking \\
\midrule\noalign{}
\endhead
\bottomrule\noalign{}
\endlastfoot
\textbf{Duration} & Episodes & Persistent \\
\textbf{Age Group} & Mainly minors & All ages \\
\textbf{Motivation} & Social dominance & Obsession/control \\
\textbf{Platform} & Public/semi-public & Private/public \\
\end{longtable}
}

\end{solutionbox}
\begin{mnemonicbox}
``Bullies Bother, Stalkers Stalk''

\end{mnemonicbox}
\begin{center}\rule{0.5\linewidth}{0.5pt}\end{center}

\subsection*{Question 3(c OR) [7
marks]}\label{question-3c-or-7-marks}

\textbf{Explain article section 67 of cyber law.}

\begin{solutionbox}

\textbf{Section 67 - Publishing Obscene Information (IT Act 2008):}

\textbf{Main Provisions:}

{\def\LTcaptype{none} % do not increment counter
\begin{longtable}[]{@{}
  >{\raggedright\arraybackslash}p{(\linewidth - 4\tabcolsep) * \real{0.3000}}
  >{\raggedright\arraybackslash}p{(\linewidth - 4\tabcolsep) * \real{0.3000}}
  >{\raggedright\arraybackslash}p{(\linewidth - 4\tabcolsep) * \real{0.4000}}@{}}
\toprule\noalign{}
\begin{minipage}[b]{\linewidth}\raggedright
Section
\end{minipage} & \begin{minipage}[b]{\linewidth}\raggedright
Content
\end{minipage} & \begin{minipage}[b]{\linewidth}\raggedright
Punishment
\end{minipage} \\
\midrule\noalign{}
\endhead
\bottomrule\noalign{}
\endlastfoot
\textbf{67} & Publishing obscene material & First conviction: 3 years +
₹5 lakh fine \\
\textbf{67A} & Sexually explicit material & Up to 5 years + ₹10 lakh
fine \\
\textbf{67B} & Child pornography & First: 5 years + ₹10 lakh,
Subsequent: 7 years + ₹10 lakh \\
\textbf{67C} & Intermediate liability** & Failure to remove illegal
content \\
\end{longtable}
}

\textbf{Key Elements:}

\textbf{Section 67 - Obscenity:}

\begin{itemize}
\tightlist
\item
  \textbf{Publishing}: Making available in electronic form
\item
  \textbf{Content}: Lascivious, sexually explicit material
\item
  \textbf{Medium}: Website, email, social media
\item
  \textbf{Intent}: Corrupt or deprave viewers
\end{itemize}

\textbf{Section 67A - Sexually Explicit:}

\begin{itemize}
\tightlist
\item
  \textbf{Enhanced punishment} for explicit sexual content
\item
  \textbf{Broader scope} than general obscenity
\item
  \textbf{Commercial purpose} considered aggravating factor
\end{itemize}

\textbf{Section 67B - Child Protection:}

\begin{itemize}
\tightlist
\item
  \textbf{Zero tolerance} for child exploitation
\item
  \textbf{Strict liability} for possession and distribution
\item
  \textbf{Higher penalties} reflecting seriousness
\item
  \textbf{Age verification} requirements for platforms
\end{itemize}

\textbf{Defenses Available:}

\begin{itemize}
\tightlist
\item
  \textbf{Scientific/educational} purpose
\item
  \textbf{Artistic merit} consideration
\item
  \textbf{Private viewing} in some cases
\item
  \textbf{Lack of knowledge} about content nature
\end{itemize}

\textbf{Digital Evidence Requirements:}

\begin{itemize}
\tightlist
\item
  \textbf{Chain of custody} maintenance
\item
  \textbf{Technical authenticity} proof
\item
  \textbf{Source identification} methods
\item
  \textbf{Preservation} of electronic evidence
\end{itemize}

\end{solutionbox}
\begin{mnemonicbox}
``Section 67 Stops Shameful Sharing''

\end{mnemonicbox}
\begin{center}\rule{0.5\linewidth}{0.5pt}\end{center}

\subsection*{Question 4(a) [3 marks]}\label{q4a}

\textbf{Discuss types of Hackers.}

\begin{solutionbox}

\textbf{Hacker Classification:}

{\def\LTcaptype{none} % do not increment counter
\begin{longtable}[]{@{}
  >{\raggedright\arraybackslash}p{(\linewidth - 4\tabcolsep) * \real{0.2000}}
  >{\raggedright\arraybackslash}p{(\linewidth - 4\tabcolsep) * \real{0.4000}}
  >{\raggedright\arraybackslash}p{(\linewidth - 4\tabcolsep) * \real{0.4000}}@{}}
\toprule\noalign{}
\begin{minipage}[b]{\linewidth}\raggedright
Type
\end{minipage} & \begin{minipage}[b]{\linewidth}\raggedright
Motivation
\end{minipage} & \begin{minipage}[b]{\linewidth}\raggedright
Activities
\end{minipage} \\
\midrule\noalign{}
\endhead
\bottomrule\noalign{}
\endlastfoot
\textbf{White Hat} & Ethical security testing & Authorized penetration
testing \\
\textbf{Black Hat} & Malicious intent & Illegal system breaking \\
\textbf{Gray Hat} & Mixed motivations & Unauthorized but
non-malicious \\
\textbf{Script Kiddie} & Recognition/fun & Using existing tools \\
\textbf{Hacktivist} & Political/social causes & Protest through
hacking \\
\end{longtable}
}

\textbf{Detailed Types:}

\begin{itemize}
\tightlist
\item
  \textbf{White Hat}: Ethical hackers, security professionals
\item
  \textbf{Black Hat}: Cybercriminals seeking profit or damage
\item
  \textbf{Gray Hat}: Between ethical and malicious
\end{itemize}

\end{solutionbox}
\begin{mnemonicbox}
``Hats Have Hacker Hierarchy''

\end{mnemonicbox}
\begin{center}\rule{0.5\linewidth}{0.5pt}\end{center}

\subsection*{Question 4(b) [4 marks]}\label{q4b}

\textbf{Explain RAT.}

\begin{solutionbox}

\textbf{RAT (Remote Administration Tool):}

\textbf{Definition:} Software that allows remote control of a computer
system, often used maliciously for unauthorized access.

\textbf{Characteristics:}

{\def\LTcaptype{none} % do not increment counter
\begin{longtable}[]{@{}ll@{}}
\toprule\noalign{}
Feature & Description \\
\midrule\noalign{}
\endhead
\bottomrule\noalign{}
\endlastfoot
\textbf{Remote Control} & Complete system access from distance \\
\textbf{Stealth Mode} & Hidden from user detection \\
\textbf{Data Theft} & File access and transfer capabilities \\
\textbf{Keylogging} & Keystroke recording \\
\textbf{Screen Capture} & Desktop monitoring \\
\end{longtable}
}

\textbf{Common RATs:}

\begin{itemize}
\tightlist
\item
  \textbf{BackOrifice}
\item
  \textbf{NetBus}
\item
  \textbf{DarkComet}
\item
  \textbf{Poison Ivy}
\end{itemize}

\textbf{Detection Methods:}

\begin{itemize}
\tightlist
\item
  Antivirus software
\item
  Network monitoring
\item
  Process analysis
\item
  Behavioral detection
\end{itemize}

\end{solutionbox}
\begin{mnemonicbox}
``RATs Run Remote Access Tactics''

\end{mnemonicbox}
\begin{center}\rule{0.5\linewidth}{0.5pt}\end{center}

\subsection*{Question 4(c) [7 marks]}\label{q4c}

\textbf{Explain Five Steps of Hacking.}

\begin{solutionbox}

\textbf{The Five-Phase Hacking Methodology:}

\begin{center}
\textbf{Mermaid Diagram (Code)}
\begin{verbatim}
{Shaded}
{Highlighting}[]
graph LR
    A[1. Reconnaissance] {-{-}{} B[2. Scanning]}
    B {-{-}{} C[3. Gaining Access]}
    C {-{-}{} D[4. Maintaining Access]}
    D {-{-}{} E[5. Covering Tracks]}
{Highlighting}
{Shaded}
\end{verbatim}
\end{center}

\textbf{Detailed Steps:}

{\def\LTcaptype{none} % do not increment counter
\begin{longtable}[]{@{}
  >{\raggedright\arraybackslash}p{(\linewidth - 6\tabcolsep) * \real{0.2000}}
  >{\raggedright\arraybackslash}p{(\linewidth - 6\tabcolsep) * \real{0.2571}}
  >{\raggedright\arraybackslash}p{(\linewidth - 6\tabcolsep) * \real{0.3429}}
  >{\raggedright\arraybackslash}p{(\linewidth - 6\tabcolsep) * \real{0.2000}}@{}}
\toprule\noalign{}
\begin{minipage}[b]{\linewidth}\raggedright
Phase
\end{minipage} & \begin{minipage}[b]{\linewidth}\raggedright
Purpose
\end{minipage} & \begin{minipage}[b]{\linewidth}\raggedright
Techniques
\end{minipage} & \begin{minipage}[b]{\linewidth}\raggedright
Tools
\end{minipage} \\
\midrule\noalign{}
\endhead
\bottomrule\noalign{}
\endlastfoot
\textbf{1. Reconnaissance} & Information Gathering & OSINT, Social
Engineering & Google, Shodan, WHOIS \\
\textbf{2. Scanning} & Identify Vulnerabilities & Port scanning, Network
mapping & Nmap, Nessus \\
\textbf{3. Gaining Access} & Exploit Vulnerabilities & Password attacks,
Code injection & Metasploit, Hydra \\
\textbf{4. Maintaining Access} & Persistent Control & Backdoors,
Rootkits & RATs, Trojans \\
\textbf{5. Covering Tracks} & Hide Evidence & Log deletion,
Steganography & CCleaner, File wipers \\
\end{longtable}
}

\textbf{Phase 1 - Reconnaissance:}

\begin{itemize}
\tightlist
\item
  \textbf{Passive}: Public information gathering
\item
  \textbf{Active}: Direct target interaction
\item
  \textbf{Goal}: Map target infrastructure
\end{itemize}

\textbf{Phase 2 - Scanning:}

\begin{itemize}
\tightlist
\item
  \textbf{Network scanning}: Live system identification
\item
  \textbf{Port scanning}: Service discovery\\
\item
  \textbf{Vulnerability scanning}: Weakness identification
\end{itemize}

\textbf{Phase 3 - Gaining Access:}

\begin{itemize}
\tightlist
\item
  \textbf{Exploitation}: Vulnerability utilization
\item
  \textbf{Authentication attacks}: Password cracking
\item
  \textbf{Privilege escalation}: Higher access levels
\end{itemize}

\textbf{Phase 4 - Maintaining Access:}

\begin{itemize}
\tightlist
\item
  \textbf{Backdoor installation}: Future access
\item
  \textbf{System modification}: Persistence mechanisms
\item
  \textbf{Data collection}: Information harvesting
\end{itemize}

\textbf{Phase 5 - Covering Tracks:}

\begin{itemize}
\tightlist
\item
  \textbf{Log manipulation}: Evidence removal
\item
  \textbf{File deletion}: Trace elimination
\item
  \textbf{Timeline modification}: Activity concealment
\end{itemize}

\end{solutionbox}
\begin{mnemonicbox}
``Real Smart Guys Make Choices'' (Reconnaissance,
Scanning, Gaining, Maintaining, Covering)

\end{mnemonicbox}
\begin{center}\rule{0.5\linewidth}{0.5pt}\end{center}

\subsection*{Question 4(a OR) [3
marks]}\label{question-4a-or-3-marks}

\textbf{Explain Brute force attack.}

\begin{solutionbox}

\textbf{Definition:} Brute force attack is a trial-and-error method used
to decode encrypted data by systematically trying all possible
combinations.

\textbf{Characteristics:}

{\def\LTcaptype{none} % do not increment counter
\begin{longtable}[]{@{}ll@{}}
\toprule\noalign{}
Aspect & Description \\
\midrule\noalign{}
\endhead
\bottomrule\noalign{}
\endlastfoot
\textbf{Method} & Exhaustive key search \\
\textbf{Time} & Computationally intensive \\
\textbf{Success} & Guaranteed but time-consuming \\
\textbf{Target} & Passwords, encryption keys \\
\textbf{Tools} & Automated software \\
\end{longtable}
}

\textbf{Types:}

\begin{itemize}
\tightlist
\item
  \textbf{Simple Brute Force}: All possible combinations
\item
  \textbf{Dictionary Attack}: Common passwords
\item
  \textbf{Hybrid Attack}: Dictionary + variations
\end{itemize}

\end{solutionbox}
\begin{mnemonicbox}
``Brute Force Breaks By Trying''

\end{mnemonicbox}
\begin{center}\rule{0.5\linewidth}{0.5pt}\end{center}

\subsection*{Question 4(b OR) [4
marks]}\label{question-4b-or-4-marks}

\textbf{Define: Vulnerability, Threat, Exploit}

\begin{solutionbox}

\textbf{Security Terminology:}

{\def\LTcaptype{none} % do not increment counter
\begin{longtable}[]{@{}
  >{\raggedright\arraybackslash}p{(\linewidth - 4\tabcolsep) * \real{0.2222}}
  >{\raggedright\arraybackslash}p{(\linewidth - 4\tabcolsep) * \real{0.4444}}
  >{\raggedright\arraybackslash}p{(\linewidth - 4\tabcolsep) * \real{0.3333}}@{}}
\toprule\noalign{}
\begin{minipage}[b]{\linewidth}\raggedright
Term
\end{minipage} & \begin{minipage}[b]{\linewidth}\raggedright
Definition
\end{minipage} & \begin{minipage}[b]{\linewidth}\raggedright
Example
\end{minipage} \\
\midrule\noalign{}
\endhead
\bottomrule\noalign{}
\endlastfoot
\textbf{Vulnerability} & Weakness in system/software & Unpatched
software bug \\
\textbf{Threat} & Potential danger to asset & Malicious hacker \\
\textbf{Exploit} & Code taking advantage of vulnerability & Buffer
overflow attack \\
\end{longtable}
}

\textbf{Relationship:}

\begin{verbatim}
Threat {-{-}{-}{-}uses{-}{-}{-}{-} Exploit {-}{-}{-}{-}targets{-}{-}{-}{-} Vulnerability}
   |                    |                        |
   v                    v                        v
Hacker              Attack Code            System Weakness
\end{verbatim}

\textbf{Examples:}

\begin{itemize}
\tightlist
\item
  \textbf{Vulnerability}: SQL injection flaw
\item
  \textbf{Threat}: Cybercriminal
\item
  \textbf{Exploit}: SQL injection payload
\end{itemize}

\textbf{Risk Formula:} Risk = Threat \times Vulnerability \times Asset Value

\end{solutionbox}
\begin{mnemonicbox}
``Threats Target Vulnerable Exploits''

\end{mnemonicbox}
\begin{center}\rule{0.5\linewidth}{0.5pt}\end{center}

\subsection*{Question 4(c OR) [7
marks]}\label{question-4c-or-7-marks}

\textbf{Explain any three basic commands of kali Linux with suitable
example.}

\begin{solutionbox}

\textbf{Essential Kali Linux Commands:}

\textbf{1. NMAP (Network Mapper):}

\begin{verbatim}
\# Port scanning
nmap {-sS} target\_ip
nmap {-A} {-T4} 192.168.1.1
\end{verbatim}

{\def\LTcaptype{none} % do not increment counter
\begin{longtable}[]{@{}lll@{}}
\toprule\noalign{}
Option & Purpose & Example \\
\midrule\noalign{}
\endhead
\bottomrule\noalign{}
\endlastfoot
\textbf{-sS} & SYN scan & nmap -sS 192.168.1.1 \\
\textbf{-A} & Aggressive scan & nmap -A target.com \\
\textbf{-p} & Specific ports & nmap -p 80,443 target.com \\
\end{longtable}
}

\textbf{2. Metasploit:}

\begin{verbatim}
\# Start Metasploit
msfconsole
\# Search exploits
search apache
\# Use exploit
use exploit/windows/smb/ms17\_010\_eternalblue
\end{verbatim}

\textbf{Commands:}

\begin{itemize}
\tightlist
\item
  \textbf{search}: Find exploits/payloads
\item
  \textbf{use}: Select module
\item
  \textbf{set}: Configure options
\item
  \textbf{exploit}: Launch attack
\end{itemize}

\textbf{3. Wireshark:}

\begin{verbatim}
\# Command line version
tshark {-i} eth0
\# Filter traffic
tshark {-i} eth0 {-f} "port 80"
\end{verbatim}

\textbf{Features:}

\begin{itemize}
\tightlist
\item
  \textbf{Packet capture}: Real-time network monitoring
\item
  \textbf{Protocol analysis}: Deep packet inspection\\
\item
  \textbf{Filter options}: Targeted traffic analysis
\item
  \textbf{GUI interface}: User-friendly analysis
\end{itemize}

\textbf{Additional Commands:}

\textbf{4. Hydra (Password Cracking):}

\begin{verbatim}
hydra {-l} admin {-P} passwords.txt ssh://192.168.1.1
\end{verbatim}

\textbf{5. John the Ripper:}

\begin{verbatim}
john {-{-}wordlist}=rockyou.txt hashes.txt
\end{verbatim}

\textbf{6. Aircrack-ng (WiFi Security):}

\begin{verbatim}
airmon{-ng} start wlan0
airodump{-ng} wlan0mon
\end{verbatim}

\textbf{Command Categories:}

{\def\LTcaptype{none} % do not increment counter
\begin{longtable}[]{@{}lll@{}}
\toprule\noalign{}
Category & Tools & Purpose \\
\midrule\noalign{}
\endhead
\bottomrule\noalign{}
\endlastfoot
\textbf{Network Scanning} & nmap, masscan & Host/port discovery \\
\textbf{Vulnerability Assessment} & OpenVAS, Nessus & Security
scanning \\
\textbf{Exploitation} & Metasploit, SQLmap & Vulnerability
exploitation \\
\textbf{Password Attacks} & Hydra, John & Credential cracking \\
\textbf{Wireless Security} & Aircrack-ng & WiFi penetration testing \\
\end{longtable}
}

\end{solutionbox}
\begin{mnemonicbox}
``Network Maps Make Security''

\end{mnemonicbox}
\begin{center}\rule{0.5\linewidth}{0.5pt}\end{center}

\subsection*{Question 5(a) [3 marks]}\label{q5a}

\textbf{List the branches of Digital Forensics.}

\begin{solutionbox}

\textbf{Digital Forensics Branches:}

{\def\LTcaptype{none} % do not increment counter
\begin{longtable}[]{@{}
  >{\raggedright\arraybackslash}p{(\linewidth - 4\tabcolsep) * \real{0.2353}}
  >{\raggedright\arraybackslash}p{(\linewidth - 4\tabcolsep) * \real{0.3529}}
  >{\raggedright\arraybackslash}p{(\linewidth - 4\tabcolsep) * \real{0.4118}}@{}}
\toprule\noalign{}
\begin{minipage}[b]{\linewidth}\raggedright
Branch
\end{minipage} & \begin{minipage}[b]{\linewidth}\raggedright
Focus Area
\end{minipage} & \begin{minipage}[b]{\linewidth}\raggedright
Applications
\end{minipage} \\
\midrule\noalign{}
\endhead
\bottomrule\noalign{}
\endlastfoot
\textbf{Computer Forensics} & Desktop/laptop systems & Hard drive
analysis \\
\textbf{Network Forensics} & Network traffic analysis & Intrusion
investigation \\
\textbf{Mobile Forensics} & Smartphones/tablets & Call logs, messages \\
\textbf{Database Forensics} & Database systems & Data integrity
verification \\
\textbf{Malware Forensics} & Malicious software & Malware analysis \\
\textbf{Email Forensics} & Email communications & Email header
analysis \\
\textbf{Memory Forensics} & RAM analysis & Live system investigation \\
\end{longtable}
}

\textbf{Specialized Areas:}

\begin{itemize}
\tightlist
\item
  \textbf{Cloud Forensics}
\item
  \textbf{IoT Forensics}
\item
  \textbf{Blockchain Forensics}
\end{itemize}

\end{solutionbox}
\begin{mnemonicbox}
``Digital Detectives Discover Many Clues''

\end{mnemonicbox}
\begin{center}\rule{0.5\linewidth}{0.5pt}\end{center}

\subsection*{Question 5(b) [4 marks]}\label{q5b}

\textbf{Discuss Locard's Principle of Exchange in Digital Forensics.}

\begin{solutionbox}

\textbf{Locard's Exchange Principle:}

\textbf{Original Principle:} ``Every contact leaves a trace''

\textbf{Digital Application:}

{\def\LTcaptype{none} % do not increment counter
\begin{longtable}[]{@{}lll@{}}
\toprule\noalign{}
Digital Activity & Trace Left & Location \\
\midrule\noalign{}
\endhead
\bottomrule\noalign{}
\endlastfoot
\textbf{File Access} & Access timestamps & File metadata \\
\textbf{Web Browsing} & Browser history, cookies & Browser cache \\
\textbf{Email Communication} & Headers, logs & Mail servers \\
\textbf{Network Activity} & Connection logs & Network devices \\
\textbf{USB Usage} & Device artifacts & Registry/logs \\
\end{longtable}
}

\textbf{Digital Evidence Traces:}

\textbf{System Level:}

\begin{itemize}
\tightlist
\item
  \textbf{Registry entries}: System changes
\item
  \textbf{Log files}: Activity records
\item
  \textbf{Temporary files}: Process artifacts
\item
  \textbf{Metadata}: File information
\end{itemize}

\textbf{Network Level:}

\begin{itemize}
\tightlist
\item
  \textbf{Router logs}: Traffic records
\item
  \textbf{Firewall logs}: Connection attempts
\item
  \textbf{DNS queries}: Website visits
\item
  \textbf{Packet captures}: Communication content
\end{itemize}

\textbf{Application Level:}

\begin{itemize}
\tightlist
\item
  \textbf{Browser artifacts}: Web activity
\item
  \textbf{Application logs}: Software usage
\item
  \textbf{Database changes}: Data modifications
\item
  \textbf{Cache files}: Temporary storage
\end{itemize}

\textbf{Forensic Implications:}

\begin{itemize}
\tightlist
\item
  \textbf{No perfect crime}: Digital traces always exist
\item
  \textbf{Evidence location}: Multiple sources available
\item
  \textbf{Corroboration}: Multiple trace validation
\item
  \textbf{Timeline reconstruction}: Activity sequencing
\end{itemize}

\end{solutionbox}
\begin{mnemonicbox}
``Every Exchange Exists Electronically''

\end{mnemonicbox}
\begin{center}\rule{0.5\linewidth}{0.5pt}\end{center}

\subsection*{Question 5(c) [7 marks]}\label{q5c}

\textbf{List the critical steps in preserving Digital Evidence.}

\begin{solutionbox}

\textbf{Digital Evidence Preservation Process:}

\begin{center}
\textbf{Mermaid Diagram (Code)}
\begin{verbatim}
{Shaded}
{Highlighting}[]
graph LR
    A[Digital Evidence] {-{-}{} B[Identification]}
    B {-{-}{} C[Collection]}
    C {-{-}{} D[Preservation]}
    D {-{-}{} E[Analysis]}
    E {-{-}{} F[Presentation]}
{Highlighting}
{Shaded}
\end{verbatim}
\end{center}

\textbf{Critical Preservation Steps:}

{\def\LTcaptype{none} % do not increment counter
\begin{longtable}[]{@{}
  >{\raggedright\arraybackslash}p{(\linewidth - 6\tabcolsep) * \real{0.1935}}
  >{\raggedright\arraybackslash}p{(\linewidth - 6\tabcolsep) * \real{0.2903}}
  >{\raggedright\arraybackslash}p{(\linewidth - 6\tabcolsep) * \real{0.2903}}
  >{\raggedright\arraybackslash}p{(\linewidth - 6\tabcolsep) * \real{0.2258}}@{}}
\toprule\noalign{}
\begin{minipage}[b]{\linewidth}\raggedright
Step
\end{minipage} & \begin{minipage}[b]{\linewidth}\raggedright
Process
\end{minipage} & \begin{minipage}[b]{\linewidth}\raggedright
Purpose
\end{minipage} & \begin{minipage}[b]{\linewidth}\raggedright
Tools
\end{minipage} \\
\midrule\noalign{}
\endhead
\bottomrule\noalign{}
\endlastfoot
\textbf{1. Identification} & Locate potential evidence & Determine scope
& Visual inspection \\
\textbf{2. Documentation} & Record scene details & Maintain chain of
custody & Photography, notes \\
\textbf{3. Isolation} & Prevent contamination & Preserve integrity &
Network disconnection \\
\textbf{4. Imaging} & Create bit-by-bit copy & Preserve original & dd,
FTK Imager \\
\textbf{5. Hashing} & Generate integrity checks & Verify authenticity &
MD5, SHA-256 \\
\textbf{6. Storage} & Secure evidence storage & Prevent tampering &
Write-protected media \\
\textbf{7. Chain of Custody} & Document handling & Legal admissibility &
Forensic forms \\
\end{longtable}
}

\textbf{Detailed Preservation Methods:}

\textbf{Physical Preservation:}

\begin{itemize}
\tightlist
\item
  \textbf{Power management}: Proper shutdown procedures
\item
  \textbf{Hardware protection}: Anti-static measures
\item
  \textbf{Environmental control}: Temperature/humidity
\item
  \textbf{Access restriction}: Authorized personnel only
\end{itemize}

\textbf{Logical Preservation:}

\begin{itemize}
\tightlist
\item
  \textbf{Bit-stream imaging}: Exact disk copies
\item
  \textbf{Hash verification}: Integrity confirmation
\item
  \textbf{Write blocking}: Prevent modifications
\item
  \textbf{Metadata preservation}: Timestamp protection
\end{itemize}

\textbf{Legal Preservation:}

\begin{itemize}
\tightlist
\item
  \textbf{Documentation standards}: Detailed records
\item
  \textbf{Chain of custody}: Handling log
\item
  \textbf{Authentication}: Evidence verification
\item
  \textbf{Admissibility}: Court requirements
\end{itemize}

\textbf{Best Practices:}

\textbf{Do's:}

\begin{itemize}
\tightlist
\item
  \textbf{Create multiple copies} of evidence
\item
  \textbf{Use forensically sound tools}
\item
  \textbf{Document every action}
\item
  \textbf{Maintain chain of custody}
\item
  \textbf{Verify integrity} with hashes
\end{itemize}

\textbf{Don'ts:}

\begin{itemize}
\tightlist
\item
  \textbf{Never work on original} evidence
\item
  \textbf{Avoid contamination} of scene
\item
  \textbf{Don't power on} suspect systems
\item
  \textbf{Never modify} evidence
\item
  \textbf{Don't break} chain of custody
\end{itemize}

\textbf{Quality Assurance:}

{\def\LTcaptype{none} % do not increment counter
\begin{longtable}[]{@{}
  >{\raggedright\arraybackslash}p{(\linewidth - 4\tabcolsep) * \real{0.1892}}
  >{\raggedright\arraybackslash}p{(\linewidth - 4\tabcolsep) * \real{0.5135}}
  >{\raggedright\arraybackslash}p{(\linewidth - 4\tabcolsep) * \real{0.2973}}@{}}
\toprule\noalign{}
\begin{minipage}[b]{\linewidth}\raggedright
Check
\end{minipage} & \begin{minipage}[b]{\linewidth}\raggedright
Verification Method
\end{minipage} & \begin{minipage}[b]{\linewidth}\raggedright
Frequency
\end{minipage} \\
\midrule\noalign{}
\endhead
\bottomrule\noalign{}
\endlastfoot
\textbf{Hash Validation} & Compare original vs copy & Before/after
operations \\
\textbf{Tool Calibration} & Verify tool accuracy & Regular intervals \\
\textbf{Process Review} & Audit procedures & Case completion \\
\textbf{Documentation Check} & Verify completeness & Each step \\
\end{longtable}
}

\textbf{Legal Considerations:}

\begin{itemize}
\tightlist
\item
  \textbf{Admissibility requirements}: Court standards
\item
  \textbf{Expert testimony}: Technical explanation
\item
  \textbf{Cross-examination}: Process validation
\item
  \textbf{Standard compliance}: Industry best practices
\end{itemize}

\end{solutionbox}
\begin{mnemonicbox}
``Proper Preservation Prevents Problems'' (Plan,
Preserve, Protect, Prove)

\end{mnemonicbox}
\begin{center}\rule{0.5\linewidth}{0.5pt}\end{center}

\subsection*{Question 5(a OR) [3
marks]}\label{question-5a-or-3-marks}

\textbf{Explain Malware forensics.}

\begin{solutionbox}

\textbf{Definition:} Malware forensics involves the analysis of
malicious software to understand its behavior, origin, and impact on
infected systems.

\textbf{Key Components:}

{\def\LTcaptype{none} % do not increment counter
\begin{longtable}[]{@{}ll@{}}
\toprule\noalign{}
Component & Description \\
\midrule\noalign{}
\endhead
\bottomrule\noalign{}
\endlastfoot
\textbf{Static Analysis} & Examining malware without execution \\
\textbf{Dynamic Analysis} & Running malware in controlled environment \\
\textbf{Code Analysis} & Reverse engineering malware code \\
\textbf{Behavioral Analysis} & Studying malware actions \\
\end{longtable}
}

\textbf{Process:}

\begin{itemize}
\tightlist
\item
  \textbf{Sample collection}: Malware acquisition
\item
  \textbf{Isolation}: Sandbox environment
\item
  \textbf{Analysis}: Behavior observation
\item
  \textbf{Reporting}: Findings documentation
\end{itemize}

\end{solutionbox}
\begin{mnemonicbox}
``Malware Makes Mysteries''

\end{mnemonicbox}
\begin{center}\rule{0.5\linewidth}{0.5pt}\end{center}

\subsection*{Question 5(b OR) [4
marks]}\label{question-5b-or-4-marks}

\textbf{Explain why CCTV plays an important role as evidence in digital
forensics investigations.}

\begin{solutionbox}

\textbf{CCTV in Digital Forensics:}

\textbf{Importance of CCTV Evidence:}

{\def\LTcaptype{none} % do not increment counter
\begin{longtable}[]{@{}
  >{\raggedright\arraybackslash}p{(\linewidth - 4\tabcolsep) * \real{0.2143}}
  >{\raggedright\arraybackslash}p{(\linewidth - 4\tabcolsep) * \real{0.4643}}
  >{\raggedright\arraybackslash}p{(\linewidth - 4\tabcolsep) * \real{0.3214}}@{}}
\toprule\noalign{}
\begin{minipage}[b]{\linewidth}\raggedright
Role
\end{minipage} & \begin{minipage}[b]{\linewidth}\raggedright
Description
\end{minipage} & \begin{minipage}[b]{\linewidth}\raggedright
Benefit
\end{minipage} \\
\midrule\noalign{}
\endhead
\bottomrule\noalign{}
\endlastfoot
\textbf{Visual Documentation} & Records actual events & Objective
evidence \\
\textbf{Timeline Establishment} & Timestamps activities & Chronological
sequence \\
\textbf{Identity Verification} & Captures suspect images & Person
identification \\
\textbf{Corroboration} & Supports other evidence & Strengthens case \\
\end{longtable}
}

\textbf{Digital Evidence Properties:}

\textbf{Technical Aspects:}

\begin{itemize}
\tightlist
\item
  \textbf{Metadata preservation}: Timestamp, camera ID, settings
\item
  \textbf{Chain of custody}: Secure handling procedures
\item
  \textbf{Format integrity}: Original file structure maintenance
\item
  \textbf{Authentication}: Digital signatures, hash values
\end{itemize}

\textbf{Forensic Value:}

\begin{itemize}
\tightlist
\item
  \textbf{Real-time documentation}: Live incident recording
\item
  \textbf{Unbiased testimony}: Mechanical witness
\item
  \textbf{High resolution}: Clear image quality
\item
  \textbf{Audio capture}: Additional sensory evidence
\end{itemize}

\textbf{Analysis Methods:}

\begin{itemize}
\tightlist
\item
  \textbf{Frame-by-frame examination}: Detailed scrutiny
\item
  \textbf{Enhancement techniques}: Image improvement
\item
  \textbf{Comparison analysis}: Multiple angle correlation
\item
  \textbf{Motion tracking}: Subject movement patterns
\end{itemize}

\textbf{Legal Admissibility:}

\begin{itemize}
\tightlist
\item
  \textbf{Authenticity verification}: Chain of custody
\item
  \textbf{Technical validation}: Equipment calibration
\item
  \textbf{Expert testimony}: Forensic analysis explanation
\item
  \textbf{Standard compliance}: Industry best practices
\end{itemize}

\end{solutionbox}
\begin{mnemonicbox}
``CCTV Captures Criminal Conduct Clearly''

\end{mnemonicbox}
\begin{center}\rule{0.5\linewidth}{0.5pt}\end{center}

\subsection*{Question 5(c OR) [7
marks]}\label{question-5c-or-7-marks}

\textbf{Explain phases of Digital forensic investigation.}

\begin{solutionbox}

\textbf{Digital Forensic Investigation Process:}

\begin{center}
\textbf{Mermaid Diagram (Code)}
\begin{verbatim}
{Shaded}
{Highlighting}[]
graph LR
    A[Incident Response] {-{-}{} B[Evidence Identification]}
    B {-{-}{} C[Evidence Collection]}
    C {-{-}{} D[Evidence Preservation]}
    D {-{-}{} E[Evidence Analysis]}
    E {-{-}{} F[Documentation]}
    F {-{-}{} G[Presentation]}
{Highlighting}
{Shaded}
\end{verbatim}
\end{center}

\textbf{Phase-wise Breakdown:}

{\def\LTcaptype{none} % do not increment counter
\begin{longtable}[]{@{}
  >{\raggedright\arraybackslash}p{(\linewidth - 6\tabcolsep) * \real{0.1842}}
  >{\raggedright\arraybackslash}p{(\linewidth - 6\tabcolsep) * \real{0.2895}}
  >{\raggedright\arraybackslash}p{(\linewidth - 6\tabcolsep) * \real{0.3158}}
  >{\raggedright\arraybackslash}p{(\linewidth - 6\tabcolsep) * \real{0.2105}}@{}}
\toprule\noalign{}
\begin{minipage}[b]{\linewidth}\raggedright
Phase
\end{minipage} & \begin{minipage}[b]{\linewidth}\raggedright
Objective
\end{minipage} & \begin{minipage}[b]{\linewidth}\raggedright
Activities
\end{minipage} & \begin{minipage}[b]{\linewidth}\raggedright
Output
\end{minipage} \\
\midrule\noalign{}
\endhead
\bottomrule\noalign{}
\endlastfoot
\textbf{1. Preparation} & Readiness establishment & Tool setup, training
& Forensic kit \\
\textbf{2. Identification} & Evidence location & Survey, documentation &
Evidence list \\
\textbf{3. Collection} & Evidence acquisition & Imaging, copying &
Digital copies \\
\textbf{4. Preservation} & Integrity maintenance & Hashing, storage &
Verified evidence \\
\textbf{5. Analysis} & Data examination & Investigation, correlation &
Findings \\
\textbf{6. Presentation} & Results communication & Reporting, testimony
& Final report \\
\end{longtable}
}

\textbf{Detailed Phase Analysis:}

\textbf{Phase 1 - Preparation:}

\begin{itemize}
\tightlist
\item
  \textbf{Tool readiness}: Forensic software installation
\item
  \textbf{Hardware setup}: Write blockers, imaging devices
\item
  \textbf{Documentation templates}: Chain of custody forms
\item
  \textbf{Team preparation}: Role assignments, training
\item
  \textbf{Legal preparation}: Warrant requirements, permissions
\end{itemize}

\textbf{Phase 2 - Identification:}

\begin{itemize}
\tightlist
\item
  \textbf{Scene survey}: Evidence location mapping
\item
  \textbf{Device inventory}: System identification
\item
  \textbf{Volatile evidence}: Memory, network connections
\item
  \textbf{Priority assessment}: Critical evidence first
\item
  \textbf{Photography}: Scene documentation
\end{itemize}

\textbf{Phase 3 - Collection:}

\begin{itemize}
\tightlist
\item
  \textbf{Live system analysis}: Memory acquisition
\item
  \textbf{Disk imaging}: Bit-for-bit copies
\item
  \textbf{Network evidence}: Log files, packet captures
\item
  \textbf{Mobile devices}: Physical/logical extraction
\item
  \textbf{Cloud evidence}: Remote data acquisition
\end{itemize}

\textbf{Phase 4 - Preservation:}

\begin{itemize}
\tightlist
\item
  \textbf{Hash generation}: MD5, SHA-256 checksums
\item
  \textbf{Write protection}: Hardware/software blocking
\item
  \textbf{Storage security}: Tamper-evident containers
\item
  \textbf{Chain of custody}: Handling documentation
\item
  \textbf{Backup creation}: Multiple evidence copies
\end{itemize}

\textbf{Phase 5 - Analysis:}

\begin{itemize}
\tightlist
\item
  \textbf{File system examination}: Directory structure analysis
\item
  \textbf{Deleted data recovery}: Unallocated space searching
\item
  \textbf{Timeline creation}: Event chronology
\item
  \textbf{Keyword searching}: Relevant content identification
\item
  \textbf{Pattern recognition}: Behavioral analysis
\end{itemize}

\textbf{Phase 6 - Presentation:}

\begin{itemize}
\tightlist
\item
  \textbf{Report writing}: Findings documentation
\item
  \textbf{Visual aids}: Charts, diagrams, screenshots
\item
  \textbf{Expert testimony}: Court presentation
\item
  \textbf{Peer review}: Quality assurance
\item
  \textbf{Archive maintenance}: Case file storage
\end{itemize}

\textbf{Best Practices:}

\textbf{Technical Standards:}

\begin{itemize}
\tightlist
\item
  \textbf{Tool validation}: Regular calibration
\item
  \textbf{Methodology consistency}: Standard procedures
\item
  \textbf{Quality control}: Verification checks
\item
  \textbf{Documentation completeness}: Detailed records
\end{itemize}

\textbf{Legal Requirements:}

\begin{itemize}
\tightlist
\item
  \textbf{Admissibility standards}: Court requirements
\item
  \textbf{Chain of custody}: Unbroken documentation
\item
  \textbf{Expert qualifications}: Professional certification
\item
  \textbf{Cross-examination preparation}: Defense against challenges
\end{itemize}

\textbf{Quality Assurance:}

{\def\LTcaptype{none} % do not increment counter
\begin{longtable}[]{@{}lll@{}}
\toprule\noalign{}
Check Point & Verification & Documentation \\
\midrule\noalign{}
\endhead
\bottomrule\noalign{}
\endlastfoot
\textbf{Evidence integrity} & Hash comparison & Verification logs \\
\textbf{Tool reliability} & Calibration tests & Certification records \\
\textbf{Process compliance} & Standard adherence & Procedure
checklists \\
\textbf{Report accuracy} & Peer review & Review signatures \\
\end{longtable}
}

\textbf{Common Challenges:}

\begin{itemize}
\tightlist
\item
  \textbf{Encryption}: Data protection barriers
\item
  \textbf{Anti-forensics}: Evidence hiding techniques
\item
  \textbf{Volume}: Large data sets
\item
  \textbf{Volatility}: Temporary evidence
\item
  \textbf{Legal complexity}: Jurisdiction issues
\end{itemize}

\textbf{Success Factors:}

\begin{itemize}
\tightlist
\item
  \textbf{Systematic approach}: Methodical investigation
\item
  \textbf{Technical expertise}: Skilled personnel
\item
  \textbf{Proper tools}: Adequate resources
\item
  \textbf{Legal knowledge}: Compliance understanding
\item
  \textbf{Documentation discipline}: Thorough records
\end{itemize}

\end{solutionbox}
\begin{mnemonicbox}
``Proper Planning Prevents Poor Performance''
(Preparation, Preservation, Processing, Presentation, Proof)

\end{mnemonicbox}

\end{document}
