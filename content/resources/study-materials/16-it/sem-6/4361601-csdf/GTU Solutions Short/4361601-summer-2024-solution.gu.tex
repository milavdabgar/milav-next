\documentclass{article}

% content/resources/templates/preamble.tex
\usepackage[margin=0.6in]{geometry}
\author{Milav Dabgar}
\usepackage{amsmath,amssymb,amsthm}
\usepackage{booktabs}
\usepackage{multirow}
\usepackage{xcolor}
\usepackage{tcolorbox}
\tcbuselibrary{breakable,skins}
\usepackage[colorlinks=true,linkcolor=blue]{hyperref}
\usepackage{titlesec}
\usepackage{enumitem}
\usepackage{tikz}
\usepackage{pgfplots}
\usepackage{circuitikz}
\usepackage[version=4]{mhchem}
\usepackage{longtable}
\usepackage{array}
\usepackage{float}
\usepackage{caption}
\usepackage{listings}

\lstset{
  basicstyle=\small\ttfamily,
  breaklines=true,
  breakatwhitespace=false,
  postbreak=\mbox{\textcolor{red}{$\hookrightarrow$}\space},
  float=false,
  numbers=left,
  numberstyle=\tiny\color{gray},
  numbersep=10pt,
  xleftmargin=2em,
  keywordstyle=\color{blue},
  commentstyle=\color{green!60!black},
  stringstyle=\color{purple},
  backgroundcolor=\color{gray!5},
  showstringspaces=false,
  tabsize=2,
  captionpos=b,
  keepspaces=true,
  columns=flexible
}

\pgfplotsset{compat=1.18}
\usetikzlibrary{shapes,arrows,positioning,calc,patterns,decorations.pathmorphing,decorations.markings,arrows.meta}

% Color scheme
\definecolor{headcolor}{RGB}{0,102,204}
\definecolor{keycolor}{RGB}{220,20,60}
\definecolor{solutioncolor}{RGB}{34,139,34}
\definecolor{mnemoniccolor}{RGB}{148,0,211}
\definecolor{codecolor}{RGB}{0,0,100}

% Spacing
\setlength{\parskip}{3pt}
\setlist[itemize]{nosep}
\setlist[enumerate]{nosep}

% Title formatting
\titleformat{\section}{\Large\bfseries\color{headcolor}}{\thesection}{1em}{}
\titleformat{\subsection}{\large\bfseries\color{headcolor}}{\thesubsection}{1em}{}

% Pandoc tightlist compatibility
\providecommand{\tightlist}{%
  \setlength{\itemsep}{0pt}\setlength{\parskip}{0pt}}

% Pandoc longtable compatibility
\newcounter{none}
\def\thenone{}


% content/resources/templates/gujarati-boxes.tex
\usepackage{fontspec}
\usepackage{polyglossia}

% Set Gujarati as main language (document is primarily in Gujarati)
% Note: gloss-gujarati.ldf doesn't exist in polyglossia, but it will use hyphenation patterns
\setdefaultlanguage{gujarati}
\setotherlanguage{english}

% Configure Gujarati font properly
% Use Language=Default to prevent polyglossia from trying to add language-specific features
% that don't exist for Gujarati, which causes "empty feature" warnings
\newfontfamily\gujaratifont[Script=Gujarati,AutoFakeBold=2.5,AutoFakeSlant=0.3]{Noto Sans Gujarati}
\setmainfont[Script=Gujarati,AutoFakeBold=2.5,AutoFakeSlant=0.3]{Noto Sans Gujarati}
% Use Noto Sans Gujarati for monospace to support Gujarati in text
\setmonofont[Scale=0.9]{Noto Sans Gujarati}

% Configure English to use the same font
\newfontfamily\englishfont[Script=Gujarati,AutoFakeBold=2.5,AutoFakeSlant=0.3]{Noto Sans Gujarati}

% Translations for polyglossia
\gappto\captionsgujarati{
  \renewcommand{\tablename}{કોષ્ટક}
  \renewcommand{\figurename}{આકૃતિ}
}

% Helper for TikZ nodes to ensure Gujarati font
\newcommand{\gu}[1]{{\gujaratifont #1}}

% Custom environments
\newtcolorbox{solutionbox}{
    breakable,
    enhanced,
    colback=solutioncolor!5!white,
    colframe=solutioncolor!75!black,
    fonttitle=\bfseries,
    title=જવાબ
}

\newtcolorbox{solutionboxnobreak}{
 colback=solutioncolor!5!white,
 colframe=solutioncolor!75!black,
 fonttitle=\bfseries,
 title=જવાબ
}

\newtcolorbox{keyformula}{
 breakable,
 enhanced,
 colback=keycolor!5!white,
 colframe=keycolor!75!black,
 fonttitle=\bfseries,
 title=રાસાયણિક સમીકરણ/સૂત્ર
}

\newtcolorbox{mnemonicbox}{
 breakable,
 enhanced,
 colback=mnemoniccolor!5!white,
 colframe=mnemoniccolor!75!black,
 fonttitle=\bfseries,
 title=મેમરી ટ્રીક
}


% Custom commands for GTU solutions
% This file defines semantic commands for consistent formatting

% Question command with automatic formatting
\newcommand{\question}[2]{%
  \section*{Question #1}%
  \textbf{#2}%
}

% OR question variant
\newcommand{\questionor}[2]{%
  \section*{Question #1 OR}%
  \textbf{#2}%
}

% Proper table environment with caption
\newenvironment{answertable}[1]{%
  \begin{table}[htbp]
  \centering
  \caption{#1}
}{%
  \end{table}
}

% Proper figure environment for diagrams
\newenvironment{answerdiagram}[1]{%
  \begin{figure}[htbp]
  \centering
  \caption{#1}
}{%
  \end{figure}
}

% Semantic markup for key terms
\newcommand{\keyword}[1]{\textbf{#1}}
\newcommand{\code}[1]{\texttt{#1}}
\newcommand{\classname}[1]{\texttt{#1}}
\newcommand{\methodname}[1]{\texttt{#1}}

% Proper quotation marks
\newcommand{\mnemonic}[1]{``#1''}


\title{Cyber Security and Digital Forensics (4361601) - Summer 2024 Solution}
\date{May 14, 2024}

\begin{document}
\maketitle

\questionmarks{1(a)}{3}{Describe CIA triad with example.}
\begin{solutionbox}
    \begin{answertable}{CIA ત્રિપુટી તુલના કોષ્ટક}
    \begin{tabulary}{\textwidth}{|L|L|L|}
        \hline
        \textbf{ઘટક} & \textbf{વ્યાખ્યા} & \textbf{ઉદાહરણ} \\
        \hline
        \textbf{ગુપ્તતા (Confidentiality)} & ડેટા માત્ર અધિકૃત વપરાશકર્તાઓને જ ઉપલબ્ધ હોય & બેંક એકાઉન્ટની વિગતો માત્ર એકાઉન્ટ ધારકને જ દેખાવી જોઈએ \\
        \hline
        \textbf{અખંડતા (Integrity)} & ડેટા સચોટ અને અપરિવર્તિત રહે & મેડિકલ રેકોર્ડ અધિકૃતતા વિના બદલાવા જોઈએ નહીં \\
        \hline
        \textbf{ઉપલબ્ધતા (Availability)} & સિસ્ટમ અને ડેટા જરૂર પડે ત્યારે ઉપલબ્ધ હોય & ATM સેવાઓ ગ્રાહકો માટે 24/7 ઉપલબ્ધ હોવી જોઈએ \\
        \hline
    \end{tabulary}
    \end{answertable}

    \begin{mnemonicbox}
    "ગુઆ" - ગુપ્તતા, અખંડતા, ઉપલબ્ધતા
    \end{mnemonicbox}
\end{solutionbox}

\questionmarks{1(b)}{4}{Explain Public key and Private Key cryptography.}
\begin{solutionbox}
    \begin{answertable}{મુખ્ય તફાવતો કોષ્ટક}
    \begin{tabulary}{\textwidth}{|L|L|L|}
        \hline
        \textbf{પાસું} & \textbf{Public Key Cryptography} & \textbf{Private Key Cryptography} \\
        \hline
        \textbf{વપરાતી કી} & બે કી (public + private) & એક શેર કરેલી કી \\
        \hline
        \textbf{કી વિતરણ} & Public કી ખુલ્લેઆમ શેર કરી શકાય & કી ગુપ્ત રીતે શેર કરવી પડે \\
        \hline
        \textbf{ઝડપ} & ધીમી encryption/decryption & ઝડપી operations \\
        \hline
        \textbf{સુરક્ષા} & વધુ સુરક્ષિત, કી શેરિંગ સમસ્યા નથી & ઓછી સુરક્ષા કી વિતરણને કારણે \\
        \hline
    \end{tabulary}
    \end{answertable}

    \textbf{મુખ્ય મુદ્દાઓ:}

    \begin{itemize}
        \item \keyword{Public Key}: asymmetric encryption નો ઉપયોગ કરે છે
        \item \keyword{Private Key}: symmetric encryption નો ઉપયોગ કરે છે
        \item \keyword{Digital Signatures}: Public કી non-repudiation શક્ય બનાવે છે
        \item \keyword{કી મેનેજમેન્ટ}: Private કી સુરક્ષિત વિતરણની જરૂર છે
    \end{itemize}

    \begin{mnemonicbox}
    "PASS" - Public Asymmetric, Symmetric Secret
    \end{mnemonicbox}
\end{solutionbox}

\questionmarks{1(c)}{7}{Explain various security services and security mechanism.}
\begin{solutionbox}
    \begin{answertable}{સિક્યુરિટી સર્વિસ કોષ્ટક}
    \begin{tabulary}{\textwidth}{|L|L|L|}
        \hline
        \textbf{સર્વિસ} & \textbf{હેતુ} & \textbf{મેકેનિઝમ ઉદાહરણ} \\
        \hline
        \textbf{Authentication} & વપરાશકર્તાની ઓળખ ચકાસવી & Passwords, Biometrics \\
        \hline
        \textbf{Authorization} & પ્રવેશ પરવાનગીઓ નિયંત્રિત કરવી & Access Control Lists \\
        \hline
        \textbf{Confidentiality} & ડેટાની ગોપનીયતા સુરક્ષિત કરવી & Encryption (AES, RSA) \\
        \hline
        \textbf{Integrity} & ડેટાની સચોટતા સુનિશ્ચિત કરવી & Digital signatures, Hashing \\
        \hline
        \textbf{Non-repudiation} & ક્રિયાઓના ઇનકારને અટકાવવો & Digital certificates \\
        \hline
        \textbf{Availability} & સેવાની પહોંચ સુનિશ્ચિત કરવી & Firewalls, Backup systems \\
        \hline
    \end{tabulary}
    \end{answertable}

    \textbf{સિક્યુરિટી મેકેનિઝમ:}

    \begin{itemize}
        \item \keyword{Encryption}: plaintext ને ciphertext માં ફેરવે છે
        \item \keyword{Digital Signatures}: authentication અને integrity પૂરી પાડે છે
        \item \keyword{Access Control}: અનધિકૃત પ્રવેશ પર પ્રતિબંધ મૂકે છે
        \item \keyword{Audit Trails}: સિક્યુરિટી ઇવેન્ટ્સ મોનિટર અને લોગ કરે છે
    \end{itemize}

    \begin{mnemonicbox}
    "ACIANA" - Authentication, Confidentiality, Integrity, Authorization, Non-repudiation, Availability
    \end{mnemonicbox}
\end{solutionbox}

\orquestionmarks{1(c)}{7}{Explain MD5 hashing algorithm.}
\begin{solutionbox}
    \textbf{MD5 અલ્ગોરિધમ પ્રક્રિયા:}

    \begin{center}
    \begin{tikzpicture}[node distance=2cm, auto]
        \node [gtu block] (input) {Input Message};
        \node [gtu block, right of=input, xshift=1cm] (pad) {Padding};
        \node [gtu block, right of=pad, xshift=1cm] (len) {Append Length};
        \node [gtu block, below of=input] (init) {Initialize MD Buffer};
        \node [gtu block, right of=init, xshift=1cm] (proc) {Process (512-bit blocks)};
        \node [gtu block, right of=proc, xshift=1cm] (out) {128-bit Hash Output};

        \draw [gtu arrow] (input) -- (pad);
        \draw [gtu arrow] (pad) -- (len);
        \draw [gtu arrow] (len) -- (init);
        \draw [gtu arrow] (init) -- (proc);
        \draw [gtu arrow] (proc) -- (out);
    \end{tikzpicture}
    \end{center}

    \begin{answertable}{MD5 લાક્ષણિકતાઓ કોષ્ટક}
    \begin{tabulary}{\textwidth}{|L|L|}
        \hline
        \textbf{ગુણધર્મ} & \textbf{મૂલ્ય} \\
        \hline
        \textbf{હેશ સાઇઝ} & 128 bits (16 bytes) \\
        \hline
        \textbf{બ્લોક સાઇઝ} & 512 bits \\
        \hline
        \textbf{રાઉન્ડ્સ} & 64 rounds \\
        \hline
        \textbf{સુરક્ષા સ્થિતિ} & ક્રિપ્ટોગ્રાફિકલી ભાંગી ગયેલ | \\
        \hline
    \end{tabulary}
    \end{answertable}

    \textbf{મુખ્ય લક્ષણો:}

    \begin{itemize}
        \item \keyword{One-way Function}: હેશથી મૂળ માં પાછા ફેરવી શકાતું નથી
        \item \keyword{નિશ્ચિત આઉટપુટ}: હંમેશા 128-bit હેશ ઉત્પન્ન કરે છે
        \item \keyword{Avalanche Effect}: નાનો ઇનપુટ ફેરફાર મોટો આઉટપુટ ફેરફાર બનાવે છે
        \item \keyword{Collision Vulnerable}: ઘણા ઇનપુટ્સ સમાન હેશ ઉત્પન્ન કરી શકે છે
    \end{itemize}

    \begin{mnemonicbox}
    "MD5 FORB" - Message Digest 5, Fixed Output, Rounds 64, Broken security
    \end{mnemonicbox}
\end{solutionbox}

\questionmarks{2(a)}{3}{What is firewall? List out types of firewall.}
\begin{solutionbox}
    \textbf{ફાયરવોલ વ્યાખ્યા:} નેટવર્ક સિક્યુરિટી ઉપકરણ જે પૂર્વનિર્ધારિત નિયમોના આધારે આવતા/જતા ટ્રાફિકને મોનિટર અને નિયંત્રિત કરે છે.

    \begin{answertable}{ફાયરવોલ પ્રકારો કોષ્ટક}
    \begin{tabulary}{\textwidth}{|L|L|L|}
        \hline
        \textbf{પ્રકાર} & \textbf{ઓપરેશન લેવલ} & \textbf{ઉદાહરણ} \\
        \hline
        \textbf{Packet Filtering} & Network Layer & iptables \\
        \hline
        \textbf{Stateful Inspection} & Session Layer & Cisco ASA \\
        \hline
        \textbf{Application Gateway} & Application Layer & Proxy servers \\
        \hline
        \textbf{Next-Gen Firewall} & Multiple Layers & Palo Alto \\
        \hline
    \end{tabulary}
    \end{answertable}

    \begin{mnemonicbox}
    "PSAN" - Packet, Stateful, Application, Next-gen
    \end{mnemonicbox}
\end{solutionbox}

\questionmarks{2(b)}{4}{Define: HTTPS and describe working of HTTPS.}
\begin{solutionbox}
    \textbf{HTTPS વ્યાખ્યા:} HTTP Secure - SSL/TLS protocols નો ઉપયોગ કરીને HTTP નું એન્ક્રિપ્ટેડ વર્ઝન.

    \textbf{HTTPS કાર્ય પ્રક્રિયા:}

    \begin{center}
    \begin{tikzpicture}[node distance=1.5cm, auto]
        \node [gtu block] (client) {Client};
        \node [gtu block, right of=client, xshift=4cm] (server) {Server};

        \draw [gtu arrow] (client) -- node[above] {1. HTTPS Request} (server);
        \draw [gtu arrow] (server) -- node[below, align=center] {2. SSL Certificate} (client);
        \draw [gtu arrow] ([yshift=-1cm]client.east) -- node[above, align=center] {3. Verify \& Send Key} ([yshift=-1cm]server.west);
        \draw [gtu arrow] ([yshift=-2cm]server.west) -- node[above] {4. Encrypted Data} ([yshift=-2cm]client.east);
    \end{tikzpicture}
    \end{center}

    \textbf{મુખ્ય ઘટકો:}

    \begin{itemize}
        \item \keyword{SSL/TLS}: એન્ક્રિપ્શન લેયર પૂરી પાડે છે
        \item \keyword{Digital Certificates}: સર્વર આઇડેન્ટિટી ચકાસે છે
        \item \keyword{Port 443}: ડિફોલ્ટ HTTPS પોર્ટ
        \item \keyword{End-to-end Encryption}: ટ્રાન્ઝિટમાં ડેટાની સુરક્ષા કરે છે
    \end{itemize}

    \begin{mnemonicbox}
    "HTTPS SDP4" - Secure, Digital certs, Port 443
    \end{mnemonicbox}
\end{solutionbox}

\questionmarks{2(c)}{7}{Give explanation of active attack and passive attack in detail.}
\begin{solutionbox}
    \begin{answertable}{હુમલા પ્રકારોની તુલના}
    \begin{tabulary}{\textwidth}{|L|L|L|}
        \hline
        \textbf{પાસું} & \textbf{Active Attack} & \textbf{Passive Attack} \\
        \hline
        \textbf{શોધ} & સરળતાથી શોધી શકાય છે | & શોધવું મુશ્કેલ \\
        \hline
        \textbf{સિસ્ટમ પર અસર} & સિસ્ટમ/ડેટામાં ફેરફાર કરે છે | & માત્ર ડેટાનું અવલોકન કરે છે \\
        \hline
        \textbf{ઉદાહરણો} & DoS, Man-in-middle & Eavesdropping, Traffic analysis \\
        \hline
        \textbf{અટકાવવાની રીત} & Firewalls, IDS & Encryption, Physical security \\
        \hline
    \end{tabulary}
    \end{answertable}

    \textbf{Active Attack પ્રકારો:}

    \begin{itemize}
        \item \keyword{Masquerade}: અધિકૃત વપરાશકર્તાની નકલ કરવી
        \item \keyword{Replay}: માન્ય ડેટા ટ્રાન્સમિશનને ફરીથી મોકલવું
        \item \keyword{Modification}: સંદેશાની સામગ્રીમાં ફેરફાર કરવો
        \item \keyword{Denial of Service}: કાયદેસર પ્રવેશને અટકાવવો
    \end{itemize}

    \textbf{Passive Attack પ્રકારો:}

    \begin{itemize}
        \item \keyword{Traffic Analysis}: કમ્યુનિકેશન પેટર્નનો અભ્યાસ
        \item \keyword{Eavesdropping}: કમ્યુનિકેશનની મોનિટરિંગ
        \item \keyword{Footprinting}: સિસ્ટમ માહિતી એકત્રિત કરવી
    \end{itemize}

    \begin{mnemonicbox}
    "Active MRMD, Passive TEF" - Masquerade/Replay/Modify/DoS, Traffic/Eavesdrop/Footprint
    \end{mnemonicbox}
\end{solutionbox}

\orquestionmarks{2(a)}{3}{What is digital signature? Explain digital signature properties.}
\begin{solutionbox}
    \textbf{Digital Signature:} ક્રિપ્ટોગ્રાફિક મેકેનિઝમ જે authentication, integrity, અને non-repudiation પૂરી પાડે છે.

    \begin{answertable}{ગુણધર્મો કોષ્ટક}
    \begin{tabulary}{\textwidth}{|L|L|}
        \hline
        \textbf{ગુણધર્મ} & \textbf{વર્ણન} \\
        \hline
        \textbf{Authentication} & મોકલનારની ઓળખ ચકાસે છે \\
        \hline
        \textbf{Integrity} & સંદેશો અપરિવર્તિત છે તેની ખાતરી કરે છે \\
        \hline
        \textbf{Non-repudiation} & મોકલનારનો ઇનકાર અટકાવે છે \\
        \hline
        \textbf{Unforgeable} & Private કી વિના બનાવી શકાતું નથી \\
        \hline
    \end{tabulary}
    \end{answertable}

    \begin{mnemonicbox}
    "AINU" - Authentication, Integrity, Non-repudiation, Unforgeable
    \end{mnemonicbox}
\end{solutionbox}

\orquestionmarks{2(b)}{4}{Define: Trojans, Rootkit, Backdoors, Keylogger}
\begin{solutionbox}
    \begin{answertable}{મેલવેર પ્રકારો કોષ્ટક}
    \begin{tabulary}{\textwidth}{|L|L|L|}
        \hline
        \textbf{પ્રકાર} & \textbf{વ્યાખ્યા} & \textbf{મુખ્ય કાર્ય} \\
        \hline
        \textbf{Trojans} & કાયદેસર સોફ્ટવેરના વેશમાં મુકાયેલ દુષ્ટ કોડ & અનધિકૃત પ્રવેશ પૂરો પાડવો \\
        \hline
        \textbf{Rootkit} & અન્ય મેલવેરની હાજરી છુપાવતું સોફ્ટવેર & દુષ્ટ પ્રવૃત્તિઓ છુપાવવી \\
        \hline
        \textbf{Backdoors} & સુરક્ષાને બાયપાસ કરતું ગુપ્ત પ્રવેશદ્વાર & દૂરસ્થ અનધિકૃત પ્રવેશ \\
        \hline
        \textbf{Keylogger} & વપરાશકર્તાના કીસ્ટ્રોક રેકોર્ડ કરે છે & પાસવર્ડ/સંવેદનશીલ ડેટાની ચોરી \\
        \hline
    \end{tabulary}
    \end{answertable}

    \begin{mnemonicbox}
    "TRBK" - Trojans છુપાવે, Rootkits ગુપ્ત કરે, Backdoors બાયપાસ કરે, Keyloggers રેકોર્ડ કરે
    \end{mnemonicbox}
\end{solutionbox}

\orquestionmarks{2(c)}{7}{Explain Secure Socket Layer.}
\begin{solutionbox}
    \textbf{SSL આર્કિટેક્ચર:}

    \begin{center}
    \begin{tikzpicture}[node distance=3cm, auto]
        \node [gtu block] (record) {SSL Record Protocol};
        \node [gtu block, above of=record] (app) {Application Layer};
        \node [gtu block, below of=record] (tcp) {TCP Layer};
        
        \node [gtu block, right of=record, xshift=1cm] (handshake) {Handshake Protocol};
        \node [gtu block, above of=handshake, yshift=-1.5cm] (change) {Change Cipher Spec};
        \node [gtu block, below of=handshake, yshift=1.5cm] (alert) {Alert Protocol};

        \draw [gtu arrow] (app) -- (record);
        \draw [gtu arrow] (record) -- (tcp);
        
        \draw [gtu arrow] (handshake) -- (record);
        \draw [gtu arrow] (change) -- (record);
        \draw [gtu arrow] (alert) -- (record);
    \end{tikzpicture}
    \end{center}

    \begin{answertable}{SSL ઘટકો કોષ્ટક}
    \begin{tabulary}{\textwidth}{|L|L|}
        \hline
        \textbf{ઘટક} & \textbf{કાર્ય} \\
        \hline
        \textbf{Record Protocol} & મૂળભૂત સુરક્ષા સેવાઓ પૂરી પાડે છે \\
        \hline
        \textbf{Handshake Protocol} & સુરક્ષા પેરામીટર્સ સ્થાપિત કરે છે \\
        \hline
        \textbf{Change Cipher} & એન્ક્રિપ્શન ફેરફારોનો સંકેત આપે છે \\
        \hline
        \textbf{Alert Protocol} & એરર સ્થિતિઓ સંભાળે છે \\
        \hline
    \end{tabulary}
    \end{answertable}

    \textbf{SSL પ્રક્રિયા:}

    \begin{itemize}
        \item \keyword{Handshake}: સુરક્ષા પેરામીટર્સની વાતચીત
        \item \keyword{Authentication}: સર્વર આઇડેન્ટિટી ચકાસવી
        \item \keyword{Key Exchange}: સેશન કી સ્થાપિત કરવી
        \item \keyword{Encryption}: સુરક્ષિત ડેટા ટ્રાન્સમિશન
    \end{itemize}

    \begin{mnemonicbox}
    "SSL RHCA-HAKE" - Record/Handshake/Change/Alert, Handshake/Auth/Key/Encrypt
    \end{mnemonicbox}
\end{solutionbox}

\questionmarks{3(a)}{3}{Explain in detail cybercrime and cybercriminal.}
\begin{solutionbox}
    \begin{answertable}{વ્યાખ્યાઓ કોષ્ટક}
    \begin{tabulary}{\textwidth}{|L|L|}
        \hline
        \textbf{શબ્દ} & \textbf{વ્યાખ્યા} \\
        \hline
        \textbf{સાયબર ક્રાઇમ} & કમ્પ્યુટર/ઇન્ટરનેટનો ઉપયોગ કરીને કરાતી ગુનાહિત પ્રવૃત્તિઓ \\
        \hline
        \textbf{સાયબર ક્રિમિનલ} & ડિજિટલ ટેકનોલોજીનો ઉપયોગ કરીને ગુના કરતી વ્યક્તિ | \\
        \hline
    \end{tabulary}
    \end{answertable}

    \textbf{સાયબર ક્રિમિનલ પ્રકારો:}

    \begin{itemize}
        \item \keyword{Script Kiddies}: ઊંડા જ્ઞાન વિના હાલના ટૂલ્સનો ઉપયોગ કરે છે
        \item \keyword{Hacktivists}: રાજકીય/સામાજિક કારણોથી પ્રેરિત
        \item \keyword{Organized Crime}: વ્યાવસાયિક ગુનાહિત જૂથો
        \item \keyword{State-sponsored}: સરકાર દ્વારા સમર્થિત હુમલાખોરો
    \end{itemize}

    \begin{mnemonicbox}
    "SSHT" - Script kiddies, State-sponsored, Hacktivists, Teams organized
    \end{mnemonicbox}
\end{solutionbox}

\questionmarks{3(b)}{4}{Describe cyber stalking and cyber bullying in detail.}
\begin{solutionbox}
    \begin{answertable}{તુલના કોષ્ટક}
    \begin{tabulary}{\textwidth}{|L|L|L|}
        \hline
        \textbf{પાસું} & \textbf{સાયબર સ્ટોકિંગ} & \textbf{સાયબર બુલિંગ} \\
        \hline
        \textbf{લક્ષ્ય} & ચોક્કસ વ્યક્તિ (મોટે ભાગે પુખ્ત) & મોટે ભાગે બાળકો/સાથીદારો \\
        \hline
        \textbf{અવધિ} & લાંબા ગાળાની પરેશાની & એક વખતની અથવા પુનરાવર્તિત હોઈ શકે \\
        \hline
        \textbf{હેતુ} & ધાક, નિયંત્રણ & અપમાન, સામાજિક બહિષ્કાર \\
        \hline
        \textbf{પદ્ધતિઓ} & મોનિટરિંગ, ધમકીભર્યા સંદેશાઓ & સોશિયલ મીડિયા પરેશાની, અફવાઓ ફેલાવવી \\
        \hline
    \end{tabulary}
    \end{answertable}

    \textbf{સામાન્ય લાક્ષણિકતાઓ:}

    \begin{itemize}
        \item \keyword{ડિજિટલ પ્લેટફોર્મ}: સોશિયલ મીડિયા, ઇમેઇલ, મેસેજિંગ એપ્સ
        \item \keyword{અનામી}: ગુનેગારો મોટે ભાગે ઓળખ છુપાવે છે
        \item \keyword{માનસિક અસર}: ભાવનાત્મક તકલીફ પહોંચાડે છે
        \item \keyword{કાયદેસરી પરિણામો}: સાયબર કાયદાઓનું ઉલ્લંઘન કરે છે
    \end{itemize}

    \begin{mnemonicbox}
    "STAL-BULL DPAL" - Digital platforms, Psychological impact, Anonymity, Legal issues
    \end{mnemonicbox}
\end{solutionbox}

\questionmarks{3(c)}{7}{Explain Property based classification in cybercrime.}
\begin{solutionbox}
    \begin{answertable}{પ્રોપર્ટી-આધારિત સાયબર ક્રાઇમ વર્ગીકરણ}
    \begin{tabulary}{\textwidth}{|L|L|L|}
        \hline
        \textbf{ગુનો પ્રકાર} & \textbf{વર્ણન} & \textbf{ઉદાહરણ} \\
        \hline
        \textbf{Credit Card Fraud} & સિસ્ટમ અનધિકૃત ઉપયોગ & ચોરાયેલા કાર્ડથી ઓનલાઇન ખરીદારી \\
        \hline
        \textbf{Software Piracy} & સોફ્ટવેરની ગેરકાયદેસર કોપીઇંગ/વિતરણ & કોપીરાઇટ સોફ્ટવેર ડાઉનલોડ કરવું \\
        \hline
        \textbf{Copyright Infringement} & બૌદ્ધિક સંપત્તિ અધિકારોનું ઉલ્લંઘન & ફિલ્મો/સંગીતની ગેરકાયદેસર શેરિંગ | \\
        \hline
        \textbf{Trademark Violations} & રજિસ્ટર્ડ ટ્રેડમાર્કનો દુરુપયોગ & બનાવટી બ્રાન્ડ વેબસાઇટ્સ બનાવવી \\
        \hline
    \end{tabulary}
    \end{answertable}

    \textbf{અસર મૂલ્યાંકન:}

    \begin{itemize}
        \item \keyword{નાણાકીય નુકસાન}: સીધો નાણાકીય નુકસાન
        \item \keyword{બૌદ્ધિક સંપત્તિ ચોરી}: સ્પર્ધાત્મક લાભનું નુકસાન
        \item \keyword{બ્રાન્ડ પ્રતિષ્ઠા}: કંપનીની છબીને નુકસાન
        \item \keyword{કાયદેસરી ખર્ચ}: કાર્યવાહી/સંરક્ષણનો ખર્ચ
    \end{itemize}

    \textbf{અટકાવવાના પગલાં:}

    \begin{itemize}
        \item \keyword{Digital Rights Management}: કોપીરાઇટ સામગ્રીની સુરક્ષા
        \item \keyword{સુરક્ષિત પેમેન્ટ સિસ્ટમ}: છેતરપિંડી શોધ લાગુ કરવી
        \item \keyword{કાયદેસરી અમલીકરણ}: ઉલ્લંઘન કરનારાઓ સામે કાર્યવાહી
        \item \keyword{જનજાગૃતિ}: કાયદેસર સોફ્ટવેર વિશે શિક્ષિત કરવું
    \end{itemize}

    \begin{mnemonicbox}
    "CSCT-FILP" - Credit/Software/Copyright/Trademark, Financial/Intellectual/Legal/Public
    \end{mnemonicbox}
\end{solutionbox}

\orquestionmarks{3(a)}{3}{Explain Data diddling.}
\begin{solutionbox}
    \textbf{ડેટા ડિડલિંગ વ્યાખ્યા:} કમ્પ્યુટર સિસ્ટમમાં ઇનપુટ પહેલાં/દરમિયાન ડેટાની અનધિકૃત ફેરબદલી.

    \begin{answertable}{લાક્ષણિકતાઓ કોષ્ટક}
    \begin{tabulary}{\textwidth}{|L|L|}
        \hline
        \textbf{પાસું} & \textbf{વિગતો} \\
        \hline
        \textbf{પદ્ધતિ} & ડેટા વેલ્યુઝમાં સહેજ ફેરફાર \\
        \hline
        \textbf{શોધ} & શોધવું ખૂબ મુશ્કેલ \\
        \hline
        \textbf{લક્ષ્ય} & નાણાકીય/સંવેદનશીલ ડેટા \\
        \hline
        \textbf{અસર} & સંચિત નોંધપાત્ર નુકસાન \\
        \hline
    \end{tabulary}
    \end{answertable}

    \begin{mnemonicbox}
    "DIDDL" - Data alteration, Input manipulation, Difficult detection, Dollar losses
    \end{mnemonicbox}
\end{solutionbox}

\orquestionmarks{3(b)}{4}{Explain cyber spying and cyber terrorism.}
\begin{solutionbox}
    \begin{answertable}{તુલના કોષ્ટક}
    \begin{tabulary}{\textwidth}{|L|L|L|}
        \hline
        \textbf{પાસું} & \textbf{સાયબર સ્પાયિંગ} & \textbf{સાયબર ટેરરિઝમ} \\
        \hline
        \textbf{હેતુ} & ગુપ્ત માહિતી એકત્રિત કરવી & ભય/અવ્યવસ્થા ફેલાવવી \\
        \hline
        \textbf{લક્ષ્યો} & સરકાર, કોર્પોરેશન્સ & મહત્વપૂર્ણ ઇન્ફ્રાસ્ટ્રક્ચર \\
        \hline
        \textbf{પદ્ધતિઓ} & ગુપ્તતા, લાંબા ગાળાની ઘૂસણખોરી & વિનાશક હુમલાઓ \\
        \hline
        \textbf{અસર} & માહિતીની ચોરી & ભૌતિક/આર્થિક નુકસાન \\
        \hline
    \end{tabulary}
    \end{answertable}

    \textbf{મુખ્ય લાક્ષણિકતાઓ:}

    \begin{itemize}
        \item \keyword{સાયબર સ્પાયિંગ}: રાજ્ય-પ્રાયોજિત, કોર્પોરેટ જાસૂસી
        \item \keyword{સાયબર ટેરરિઝમ}: વિચારધારાથી પ્રેરિત, વ્યાપક વિક્ષેપ
        \item \keyword{સામાન્ય ટૂલ્સ}: મેલવેર, સામાજિક એન્જિનિયરિંગ, ઝીરો-ડે એક્સપ્લોઇટ્સ
    \end{itemize}

    \begin{mnemonicbox}
    "SPY-TER IGSD" - Intelligence/Government/Stealth/Disruption, Terror/Economic/Rapid/Damage
    \end{mnemonicbox}
\end{solutionbox}

\orquestionmarks{3(c)}{7}{Explain article section 65 and section 66 of cyber law.}
\begin{solutionbox}
    \begin{answertable}{IT એક્ટ 2008 કલમો}
    \begin{tabulary}{\textwidth}{|L|L|L|}
        \hline
        \textbf{કલમ} & \textbf{ગુનો} & \textbf{સજા} \\
        \hline
        \textbf{કલમ 65} & કમ્પ્યુટર સોર્સ કોડ સાથે છેડછાડ & 3 વર્ષ સુધીની જેલ અથવા ₹2 લાખ સુધીનો દંડ \\
        \hline
        \textbf{કલમ 66} & કમ્પ્યુટર સંબંધિત ગુનાઓ & 3 વર્ષ સુધીની જેલ અથવા ₹5 લાખ સુધીનો દંડ \\
        \hline
    \end{tabulary}
    \end{answertable}

    \textbf{કલમ 65 વિગતો:}

    \begin{itemize}
        \item \keyword{અવકાશ}: જાણીજોઈને કમ્પ્યુટર સોર્સ કોડ છુપાવવો, નાશ કરવો, બદલવો
        \item \keyword{આશય}: જ્યારે કમ્પ્યુટર સોર્સ કોડ કાયદા દ્વારા રાખવો/જાળવવો જરૂરી હોય
        \item \keyword{લાગુ}: આવશ્યક સોફ્ટવેર સિસ્ટમ્સની અખંડતાનું રક્ષણ કરે છે
    \end{itemize}

    \textbf{કલમ 66 વિગતો:}

    \begin{itemize}
        \item \keyword{કમ્પ્યુટર હેકિંગ}: કમ્પ્યુટર સિસ્ટમ્સમાં અનધિકૃત પ્રવેશ
        \item \keyword{ડેટા ચોરી}: બેઇમાનીથી ડેટા ડાઉનલોડ, કોપી, એક્સટ્રેક્ટ કરવું
        \item \keyword{સિસ્ટમ નુકસાન}: માહિતી નાશ, ડિલીટ, બદલવી
        \item \keyword{સેવા વિક્ષેપ}: અધિકૃત વ્યક્તિઓને પ્રવેશ ન આપવો
    \end{itemize}

    \begin{mnemonicbox}
    "65-66 CDHD" - Code tampering, Damage, Hacking, Data theft
    \end{mnemonicbox}
\end{solutionbox}

\questionmarks{4(a)}{3}{What is Hacking? List out types of Hackers.}
\begin{solutionbox}
    \textbf{હેકિંગ વ્યાખ્યા:} નબળાઈઓનો ફાયદો ઉઠાવવા માટે કમ્પ્યુટર સિસ્ટમ્સ/નેટવર્ક્સમાં અનધિકૃત પ્રવેશ.

    \begin{answertable}{હેકર પ્રકારો કોષ્ટક}
    \begin{tabulary}{\textwidth}{|L|L|L|}
        \hline
        \textbf{પ્રકાર} & \textbf{પ્રેરણા} & \textbf{પ્રવૃત્તિ} \\
        \hline
        \textbf{White Hat} & સુરક્ષા સુધારણા & નૈતિક પેનિટ્રેશન ટેસ્ટિંગ \\
        \hline
        \textbf{Black Hat} & દુષ્ટ ઇરાદો & ગુનાહિત પ્રવૃત્તિઓ \\
        \hline
        \textbf{Grey Hat} & મિશ્ર હેતુઓ & અનધિકૃત પરંતુ બિન-દુષ્ટ \\
        \hline
        \textbf{Script Kiddie} & માન્યતા & હાલના ટૂલ્સનો ઉપયોગ \\
        \hline
    \end{tabulary}
    \end{answertable}

    \begin{mnemonicbox}
    "WBGS Hat" - White, Black, Grey, Script kiddie
    \end{mnemonicbox}
\end{solutionbox}

\questionmarks{4(b)}{4}{Explain Vulnerability and 0-Day terminology of Hacking.}
\begin{solutionbox}
    \begin{answertable}{પરિભાષા કોષ્ટક}
    \begin{tabulary}{\textwidth}{|L|L|L|}
        \hline
        \textbf{શબ્દ} & \textbf{વ્યાખ્યા} & \textbf{જોખમ સ્તર} \\
        \hline
        \textbf{Vulnerability} & શોષણ કરી શકાય તેવી સુરક્ષા નબળાઈ & મધ્યમ-ઉચ્ચ \\
        \hline
        \textbf{0-Day Vulnerability} & અજ્ઞાત સુરક્ષા ખામી & ગંભીર \\
        \hline
        \textbf{0-Day Exploit} & 0-day vulnerability માટે હુમલો કોડ & ગંભીર \\
        \hline
        \textbf{0-Day Attack} & 0-day નો સક્રિય શોષણ & ગંભીર \\
        \hline
    \end{tabulary}
    \end{answertable}

    \textbf{મુખ્ય લાક્ષણિકતાઓ:}

    \begin{itemize}
        \item \keyword{વિક્રેતાઓને અજ્ઞાત}: કોઈ પેચ ઉપલબ્ધ નથી
        \item \keyword{ઉચ્ચ મૂલ્ય}: ડાર્ક માર્કેટમાં વેચાય છે
        \item \keyword{છુપી}: શોધવું મુશ્કેલ
        \item \keyword{સમય-નિર્ણાયક}: જાહેર થયા પછી મૂલ્ય ઘટે છે
    \end{itemize}

    \begin{mnemonicbox}
    "0-Day UHST" - Unknown, High-value, Stealthy, Time-critical
    \end{mnemonicbox}
\end{solutionbox}

\questionmarks{4(c)}{7}{Explain Five Steps of Hacking.}
\begin{solutionbox}
    \textbf{હેકિંગ પ્રક્રિયા ફ્લો:}

    \begin{center}
    \begin{tikzpicture}[node distance=1.8cm, auto]
        \node [gtu block] (info) {Information Gathering};
        \node [gtu block, right of=info, xshift=1.5cm] (scan) {Scanning};
        \node [gtu block, right of=scan, xshift=1.5cm] (gain) {Gaining Access};
        \node [gtu block, below of=info] (maint) {Maintaining Access};
        \node [gtu block, below of=scan] (cover) {Covering Tracks};

        \draw [gtu arrow] (info) -- (scan);
        \draw [gtu arrow] (scan) -- (gain);
        \draw [gtu arrow] (gain) -- (maint);
        \draw [gtu arrow] (maint) -- (cover);
    \end{tikzpicture}
    \end{center}

    \begin{answertable}{પાંચ સ્ટેપ્સ વિગતવાર}
    \begin{tabulary}{\textwidth}{|L|L|L|}
        \hline
        \textbf{સ્ટેપ} & \textbf{હેતુ} & \textbf{ટૂલ્સ/તકનીકો} \\
        \hline
        \textbf{1. માહિતી એકત્રીકરણ} & લક્ષ્ય માહિતી એકત્રિત કરવી & OSINT, સામાજિક એન્જિનિયરિંગ \\
        \hline
        \textbf{2. સ્કેનિંગ} & જીવંત સિસ્ટમ્સ, પોર્ટ્સ ઓળખવા & Nmap, પોર્ટ સ્કેનર્સ \\
        \hline
        \textbf{3. પ્રવેશ મેળવવો} & નબળાઈઓનો શોષણ કરવો & Metasploit, કસ્ટમ એક્સપ્લોઇટ્સ \\
        \hline
        \textbf{4. પ્રવેશ જાળવવો} & સતત હાજરી સ્થાપિત કરવી & બેકડોર્સ, રૂટકિટ્સ \\
        \hline
        \textbf{5. નિશાનો છુપાવવા} & પુરાવાઓ દૂર કરવા & લોગ ડિલીશન, ફાઇલ સફાઈ \\
        \hline
    \end{tabulary}
    \end{answertable}

    \textbf{દરેક સ્ટેપની વિગતો:}

    \begin{itemize}
        \item \keyword{માહિતી એકત્રીકરણ}: નિષ્ક્રિય/સક્રિય જાસૂસી
        \item \keyword{સ્કેનિંગ}: નેટવર્ક મેપિંગ, વલ્નેરેબિલિટી મૂલ્યાંકન
        \item \keyword{પ્રવેશ મેળવવો}: પાસવર્ડ હુમલાઓ, બફર ઓવરફ્લો
        \item \keyword{પ્રવેશ જાળવવો}: વિશેષાધિકાર વૃદ્ધિ, બેકડોર ઇન્સ્ટોલેશન
        \item \keyword{નિશાનો છુપાવવા}: એન્ટિ-ફોરેન્સિક્સ તકનીકો
    \end{itemize}

    \begin{mnemonicbox}
    "ISGMC" - Information, Scanning, Gaining, Maintaining, Covering
    \end{mnemonicbox}
\end{solutionbox}

\orquestionmarks{4(a)}{3}{Explain any three basic commands of kali Linux with suitable example.}
\begin{solutionbox}
    \begin{answertable}{કાલી લિનક્સ કમાન્ડ્સ કોષ્ટક}
    \begin{tabulary}{\textwidth}{|L|L|L|}
        \hline
        \textbf{કમાન્ડ} & \textbf{હેતુ} & \textbf{ઉદાહરણ} \\
        \hline
        \textbf{nmap} & નેટવર્ક સ્કેનિંગ & \code{nmap -sS 192.168.1.1} \\
        \hline
        \textbf{netcat} & નેટવર્ક યુટિલિટી & \code{nc -l -p 4444} \\
        \hline
        \textbf{john} & પાસવર્ડ ક્રેકિંગ & \code{john --wordlist=pw.txt hashes.txt} \\
        \hline
    \end{tabulary}
    \end{answertable}

    \textbf{કમાન્ડ વિગતો:}

    \begin{itemize}
        \item \keyword{nmap}: લક્ષ્ય IP પર સ્ટેલ્થ SYN સ્કેન
        \item \keyword{netcat}: કનેક્શન માટે પોર્ટ 4444 પર સાંભળો
        \item \keyword{john}: પાસવર્ડ હેશ પર ડિક્શનરી એટેક
    \end{itemize}

    \begin{mnemonicbox}
    "NNJ" - Nmap સ્કેન કરે, Netcat સાંભળે, John ક્રેક કરે
    \end{mnemonicbox}
\end{solutionbox}

\orquestionmarks{4(b)}{4}{Describe Session Hijacking in detail.}
\begin{solutionbox}
    \textbf{સેશન હાઇજેકિંગ પ્રક્રિયા:}

    \begin{center}
    \begin{tikzpicture}[node distance=1.5cm, auto]
        \node [gtu block] (user) {User};
        \node [gtu block, right of=user, xshift=2cm] (server) {Server};
        \node [gtu block, below of=user] (attacker) {Attacker};

        \draw [gtu arrow] (user) -- node[above, font=\footnotesize] {1. Login, Get ID} (server);
        \draw [gtu arrow, dashed] (user) -- node[left, font=\footnotesize] {2. Sniff ID} (attacker);
        \draw [gtu arrow] (attacker) -- node[below, font=\footnotesize] {3. Use Stolen ID} (server);
        \draw [gtu arrow] (server) -- node[right, font=\footnotesize] {4. Grant Access} (attacker);
    \end{tikzpicture}
    \end{center}

    \textbf{પ્રકારો અને પદ્ધતિઓ:}

    \begin{itemize}
        \item \keyword{Active Hijacking}: હુમલાખોર સક્રિયપણે ભાગ લે છે
        \item \keyword{Passive Hijacking}: સેશન્સનું મોનિટર અને કેપ્ચર કરે છે
        \item \keyword{Network Level}: IP spoofing, ARP poisoning
        \item \keyword{Application Level}: Session ID અનુમાન, XSS
    \end{itemize}

    \textbf{અટકાવવાના પગલાં:}

    \begin{itemize}
        \item \keyword{HTTPS}: સેશન ડેટા એન્ક્રિપ્ટ કરવો
        \item \keyword{સેશન ટાઇમઆઉટ્સ}: સેશનની અવધિ મર્યાદિત કરવી
        \item \keyword{IP બાઇન્ડિંગ}: સેશન્સને IP એડ્રેસ સાથે બાંધવા
        \item \keyword{મજબૂત સેશન IDs}: અણધારી ટોકન્સનો ઉપયોગ
    \end{itemize}

    \begin{mnemonicbox}
    "APNA-HSIS" - Active/Passive/Network/Application, HTTPS/Strong/IP/Session
    \end{mnemonicbox}
\end{solutionbox}

\orquestionmarks{4(c)}{7}{Explain Remote Administration Tools.}
\begin{solutionbox}
    \textbf{RAT વ્યાખ્યા:} કમ્પ્યુટર સિસ્ટમ્સના દૂરસ્થ નિયંત્રણની મંજૂરી આપતું સોફ્ટવેર, મોટે ભાગે દુષ્ટતાથી વપરાય છે.

    \begin{answertable}{RAT કાર્યક્ષમતા કોષ્ટક}
    \begin{tabulary}{\textwidth}{|L|L|L|}
        \hline
        \textbf{કાર્ય} & \textbf{વર્ણન} & \textbf{જોખમ સ્તર} \\
        \hline
        \textbf{સ્ક્રીન કેપ્ચર} & દૂરસ્થ સ્ક્રીનશોટ લેવા & મધ્યમ \\
        \hline
        \textbf{કીલોગિંગ} & કીસ્ટ્રોક રેકોર્ડ કરવા & ઉચ્ચ \\
        \hline
        \textbf{ફાઇલ ટ્રાન્સફર} & ફાઇલ અપલોડ/ડાઉનલોડ & ઉચ્ચ \\
        \hline
        \textbf{કેમેરા એક્સેસ} & વેબકેમ/માઇક્રોફોન સક્રિય કરવા & ગંભીર \\
        \hline
    \end{tabulary}
    \end{answertable}

    \begin{answertable}{કાયદેસર વિ. દુષ્ટ ઉપયોગ}
    \begin{tabulary}{\textwidth}{|L|L|L|}
        \hline
        \textbf{પાસું} & \textbf{કાયદેસર} & \textbf{દુષ્ટ} \\
        \hline
        \textbf{હેતુ} & IT સપોર્ટ, એડમિનિસ્ટ્રેશન & જાસૂસી, ચોરી \\
        \hline
        \textbf{સંમતિ} & વપરાશકર્તા જાગરૂક અને સંમત & જ્ઞાન વિના ઇન્સ્ટોલ | \\
        \hline
        \textbf{પ્રવેશ} & માત્ર અધિકૃત કર્મચારીઓ & અનધિકૃત હુમલાખોરો \\
        \hline
    \end{tabulary}
    \end{answertable}

    \textbf{શોધ અને અટકાવવું:}

    \begin{itemize}
        \item \keyword{એન્ટિવાયરસ}: જાણીતા RAT સિગ્નેચર શોધવા
        \item \keyword{નેટવર્ક મોનિટરિંગ}: અસામાન્ય આઉટબાઉન્ડ કનેક્શન્સ
        \item \keyword{વપરાશકર્તા શિક્ષણ}: શંકાસ્પદ ડાઉનલોડ્સ ટાળવા
        \item \keyword{ફાયરવોલ નિયમો}: અનધિકૃત કનેક્શન્સ બ્લોક કરવા
    \end{itemize}

    \textbf{સામાન્ય RATs:}

    \begin{itemize}
        \item \keyword{TeamViewer}: કાયદેસર દૂરસ્થ પ્રવેશ
        \item \keyword{DarkComet}: દુષ્ટ RAT
        \item \keyword{Poison Ivy}: અદ્યતન સતત ધમકી ટૂલ
    \end{itemize}

    \begin{mnemonicbox}
    "RAT SKFC-ANUM" - Screen/Key/File/Camera, Antivirus/Network/User/Monitoring
    \end{mnemonicbox}
\end{solutionbox}

\questionmarks{5(a)}{3}{Explain Mobile forensics.}
\begin{solutionbox}
    \textbf{મોબાઇલ ફોરેન્સિક્સ વ્યાખ્યા:} વૈજ્ઞાનિક રીતે સ્વીકૃત પદ્ધતિઓનો ઉપયોગ કરીને મોબાઇલ ઉપકરણોમાંથી ડિજિટલ પુરાવા પુનઃપ્રાપ્ત કરવાની પ્રક્રિયા.

    \begin{answertable}{મુખ્ય પાસાઓ કોષ્ટક}
    \begin{tabulary}{\textwidth}{|L|L|}
        \hline
        \textbf{પાસું} & \textbf{વર્ણન} \\
        \hline
        \textbf{ડેટા પ્રકારો} & કોલ લોગ્સ, SMS, ફોટો, એપ ડેટા \\
        \hline
        \textbf{પડકારો} & એન્ક્રિપ્શન, એન્ટિ-ફોરેન્સિક્સ, OS ની વિવિધતા \\
        \hline
        \textbf{ટૂલ્સ} & Cellebrite, XRY, Oxygen Suite \\
        \hline
        \textbf{કાયદેસર} & કસ્ટડી ચેન, કોર્ટ સ્વીકાર્યતા \\
        \hline
    \end{tabulary}
    \end{answertable}

    \begin{mnemonicbox}
    "DCTL" - Data types, Challenges, Tools, Legal requirements
    \end{mnemonicbox}
\end{solutionbox}

\questionmarks{5(b)}{4}{What is Digital forensics? Write down advantages of Digital forensics.}
\begin{solutionbox}
    \textbf{ડિજિટલ ફોરેન્સિક્સ વ્યાખ્યા:} કાયદેસરી કાર્યવાહી માટે પુરાવાઓ પુનઃપ્રાપ્ત અને વિશ્લેષણ કરવા માટે ડિજિટલ ઉપકરણોની વૈજ્ઞાનિક તપાસ.

    \begin{answertable}{ફાયદાઓ કોષ્ટક}
    \begin{tabulary}{\textwidth}{|L|L|}
        \hline
        \textbf{ફાયદો} & \textbf{વર્ણન} \\
        \hline
        \textbf{પુરાવા પુનઃપ્રાપ્તિ} & ડિલીટ/છુપાયેલ ડેટા પુનઃપ્રાપ્ત કરવો \\
        \hline
        \textbf{ગુના ઉકેલ} & કેસો માટે મહત્વપૂર્ણ પુરાવા પૂરા પાડવા \\
        \hline
        \textbf{ખર્ચ અસરકારક} & પરંપરાગત તપાસ કરતાં સસ્તું \\
        \hline
        \textbf{સચોટ પરિણામો} & વૈજ્ઞાનિક પદ્ધતિઓ વિશ્વસનીયતા સુનિશ્ચિત કરે છે \\
        \hline
    \end{tabulary}
    \end{answertable}

    \textbf{વધારાના ફાયદાઓ:}

    \begin{itemize}
        \item \keyword{સમય કાર્યક્ષમ}: મેન્યુઅલ તપાસ કરતાં ઝડપી
        \item \keyword{બિન-વિનાશક}: મૂળ પુરાવાઓ સાચવે છે
        \item \keyword{વ્યાપક}: બહુવિધ ડેટા સ્ત્રોતોનું વિશ્લેષણ કરે છે
        \item \keyword{કોર્ટ સ્વીકાર્ય}: કાયદેસર રીતે સ્વીકાર્ય પુરાવા
    \end{itemize}

    \begin{mnemonicbox}
    "ECCA-TNCA" - Evidence/Crime/Cost/Accurate, Time/Non-destructive/Comprehensive/Admissible
    \end{mnemonicbox}
\end{solutionbox}

\questionmarks{5(c)}{7}{Describe in detail Locard's Principle of exchange in Digital Forensics.}
\begin{solutionbox}
    \textbf{લોકાર્ડનો સિદ્ધાંત:} "દરેક સંપર્ક નિશાન છોડે છે" - વસ્તુઓ વચ્ચેની કોઈપણ ક્રિયા સામગ્રીના વિનિમયમાં પરિણમે છે.

    \textbf{ડિજિટલ એપ્લિકેશન:}

    \begin{center}
    \begin{tikzpicture}[node distance=1.5cm, auto]
        \node [gtu block] (action) {User Action};
        \node [gtu block, right of=action, xshift=2cm] (traces) {Digital Traces};
        
        \node [gtu block, above of=traces, xshift=3cm] (log) {Log Files};
        \node [gtu block, below of=log] (reg) {Registry Entries};
        \node [gtu block, below of=reg] (meta) {File Metadata};
        \node [gtu block, below of=meta] (net) {Network Traffic};

        \draw [gtu arrow] (action) -- (traces);
        \draw [gtu arrow] (traces) -- (log);
        \draw [gtu arrow] (traces) -- (reg);
        \draw [gtu arrow] (traces) -- (meta);
        \draw [gtu arrow] (traces) -- (net);
    \end{tikzpicture}
    \end{center}

    \begin{answertable}{ડિજિટલ નિશાનો કોષ્ટક}
    \begin{tabulary}{\textwidth}{|L|L|L|}
        \hline
        \textbf{ક્રિયા} & \textbf{ડિજિટલ નિશાન} & \textbf{સ્થાન} \\
        \hline
        \textbf{ફાઇલ એક્સેસ} & એક્સેસ ટાઇમસ્ટેમ્પ્સ & ફાઇલ સિસ્ટમ મેટાડેટા \\
        \hline
        \textbf{વેબ બ્રાઉઝિંગ} & બ્રાઉઝર હિસ્ટરી & બ્રાઉઝર ડેટાબેસ \\
        \hline
        \textbf{ઇમેઇલ મોકલવો} & ઇમેઇલ હેડર્સ & મેઇલ સર્વર લોગ્સ \\
        \hline
        \textbf{USB કનેક્શન} & ઉપકરણ રજિસ્ટ્રી & Windows રજિસ્ટ્રી \\
        \hline
    \end{tabulary}
    \end{answertable}

    \textbf{ફોરેન્સિક અસરો:}

    \begin{itemize}
        \item \keyword{સ્થાયિત્વ}: ડિજિટલ નિશાનો મોટે ભાગે વધુ લાંબા સમય ટકે છે
        \item \keyword{સચોટતા}: ચોક્કસ ટાઇમસ્ટેમ્પ્સ અને ડેટા
        \item \keyword{માત્રા}: મોટી માત્રામાં ટ્રેસ પુરાવા
        \item \keyword{પુનઃપ્રાપ્તિ}: ડિલીટ થયેલ ડેટા પુનઃપ્રાપ્ત કરી શકાય છે
    \end{itemize}

    \textbf{પુરાવા પ્રકારો:}

    \begin{itemize}
        \item \keyword{કાલાનુક્રમિક}: ક્રિયાઓ ક્યારે થઈ
        \item \keyword{અવકાશીય}: ક્રિયાઓ ક્યાં થઈ
        \item \keyword{સંબંધીય}: એન્ટિટી વચ્ચેના જોડાણો
        \item \keyword{વર્તણૂકીય}: વપરાશકર્તા પ્રવૃત્તિના પેટર્ન
    \end{itemize}

    \textbf{એપ્લિકેશન્સ:}

    \begin{itemize}
        \item \keyword{ગુનાહિત કેસો}: હાજરી/ક્રિયાઓ સાબિત કરવી
        \item \keyword{સિવિલ મુકદ્દમાઓ}: વ્યવસાયિક વિવાદો
        \item \keyword{આંતરિક તપાસ}: કર્મચારીઓની ગેરવર્તણૂક
        \item \keyword{ઘટના પ્રતિભાવ}: સુરક્ષા ભંગ વિશ્લેષણ
    \end{itemize}

    \begin{mnemonicbox}
    "LOCARD PVAR-TREB" - Persistence/Volume/Accuracy/Recovery, Temporal/Relational/Evidence/Behavioral
    \end{mnemonicbox}
\end{solutionbox}

\orquestionmarks{5(a)}{3}{Explain Network forensics.}
\begin{solutionbox}
    \textbf{નેટવર્ક ફોરેન્સિક્સ વ્યાખ્યા:} માહિતી અને પુરાવા એકત્રિત કરવા માટે નેટવર્ક ટ્રાફિકનું મોનિટરિંગ અને વિશ્લેષણ.

    \begin{answertable}{મુખ્ય ઘટકો કોષ્ટક}
    \begin{tabulary}{\textwidth}{|L|L|}
        \hline
        \textbf{ઘટક} & \textbf{કાર્ય} \\
        \hline
        \textbf{પેકેટ કેપ્ચર} & નેટવર્ક ટ્રાફિક રેકોર્ડ કરવો \\
        \hline
        \textbf{ટ્રાફિક વિશ્લેષણ} & કમ્યુનિકેશન પેટર્નનું પરીક્ષણ \\
        \hline
        \textbf{પ્રોટોકોલ વિશ્લેષણ} & નેટવર્ક પ્રોટોકોલ્સ ડીકોડ કરવા \\
        \hline
        \textbf{ટાઇમલાઇન બનાવવી} & ઘટનાઓનો ક્રમ સ્થાપિત કરવો \\
        \hline
    \end{tabulary}
    \end{answertable}

    \begin{mnemonicbox}
    "PTTP" - Packet capture, Traffic analysis, Timeline, Protocol analysis
    \end{mnemonicbox}
\end{solutionbox}

\orquestionmarks{5(b)}{4}{Explain why CCTV plays an important role as evidence in digital forensics investigations.}
\begin{solutionbox}
    \begin{answertable}{CCTV પુરાવાનું મૂલ્ય}
    \begin{tabulary}{\textwidth}{|L|L|}
        \hline
        \textbf{પાસું} & \textbf{મહત્વ} \\
        \hline
        \textbf{દ્રશ્ય પુરાવો} & ઘટનાઓના સીધા પુરાવા \\
        \hline
        \textbf{ટાઇમસ્ટેમ્પ} & ચોક્કસ સમય સહસંબંધ \\
        \hline
        \textbf{સ્થાન ચકાસણી} & ઘટના સ્થળે હાજરી સાબિત કરે છે \\
        \hline
        \textbf{વર્તણૂક વિશ્લેષણ} & ક્રિયાઓ અને ઇરાદો દર્શાવે છે \\
        \hline
    \end{tabulary}
    \end{answertable}

    \textbf{ડિજિટલ ફોરેન્સિક્સ એકીકરણ:}

    \begin{itemize}
        \item \keyword{મેટાડેટા નિષ્કર્ષણ}: કેમેરા સેટિંગ્સ, ટાઇમસ્ટેમ્પ્સ
        \item \keyword{વીડિયો સુધારણા}: છબીની ગુણવત્તા સુધારવી
        \item \keyword{ફોર્મેટ વિશ્લેષણ}: કમ્પ્રેશન આર્ટિફેક્ટ્સ સમજવા
        \item \keyword{પ્રમાણીકરણ}: વીડિયોની અખંડતા ચકાસવી
    \end{itemize}

    \textbf{કાયદેસરી વિચારણાઓ:}

    \begin{itemize}
        \item \keyword{કસ્ટડી ચેન}: પુરાવાની અખંડતા જાળવવી
        \item \keyword{કોર્ટ સ્વીકાર્યતા}: કાયદેસર પ્રક્રિયાઓ અનુસરવી
        \item \keyword{ગોપનીયતા અધિકારો}: સર્વેલન્સ કાયદાઓનું સન્માન કરવું
        \item \keyword{તકનીકી માન્યતા}: પ્રામાણિકતા સાબિત કરવી
    \end{itemize}

    \begin{mnemonicbox}
    "VTLB-MFAC" - Visual/Timestamp/Location/Behavior, Metadata/Format/Authentication/Chain
    \end{mnemonicbox}
\end{solutionbox}

\orquestionmarks{5(c)}{7}{Explain phases of Digital forensic investigation.}
\begin{solutionbox}
    \textbf{ડિજિટલ ફોરેન્સિક તપાસના તબક્કાઓ:}

    \begin{center}
    \begin{tikzpicture}[node distance=2.2cm, auto]
        \node [gtu block] (id) {Identification};
        \node [gtu block, right of=id] (pres) {Preservation};
        \node [gtu block, right of=pres] (anal) {Analysis};
        \node [gtu block, right of=anal] (doc) {Documentation};
        \node [gtu block, right of=doc] (presen) {Presentation};

        \draw [gtu arrow] (id) -- (pres);
        \draw [gtu arrow] (pres) -- (anal);
        \draw [gtu arrow] (anal) -- (doc);
        \draw [gtu arrow] (doc) -- (presen);
    \end{tikzpicture}
    \end{center}

    \begin{answertable}{તબક્કાઓની વિગતો કોષ્ટક}
    \begin{tabulary}{\textwidth}{|L|L|L|}
        \hline
        \textbf{તબક્કો} & \textbf{પ્રવૃત્તિઓ} & \textbf{ટૂલ્સ/પદ્ધતિઓ} \\
        \hline
        \textbf{ઓળખ} & સંભવિત પુરાવા સ્ત્રોતો શોધવા & પ્રારંભિક મૂલ્યાંકન, સીન સર્વે \\
        \hline
        \textbf{સંરક્ષણ} & ફેરફાર વિના પુરાવા સુરક્ષિત કરવા & ઇમેજિંગ, હેશ ચકાસણી \\
        \hline
        \textbf{વિશ્લેષણ} & સંબંધિત ડેટા માટે પુરાવાઓનું પરીક્ષણ & ફોરેન્સિક સોફ્ટવેર, મેન્યુઅલ સમીક્ષા \\
        \hline
        \textbf{દસ્તાવેજીકરણ} & શોધો અને પ્રક્રિયાઓ રેકોર્ડ કરવી & રિપોર્ટ્સ, સ્ક્રીનશોટ્સ, લોગ્સ \\
        \hline
        \textbf{રજૂઆત} & હિતધારકોને શોધો રજૂ કરવા & કોર્ટ સાક્ષ્ય, નિષ્ણાત રિપોર્ટ્સ \\
        \hline
    \end{tabulary}
    \end{answertable}

    \textbf{વિગતવાર પ્રવૃત્તિઓ:}

    \textbf{1. ઓળખ તબક્કો:}
    \begin{itemize}
        \item \keyword{પુરાવા સ્ત્રોતો}: કમ્પ્યુટર્સ, ફોન્સ, સર્વર્સ, નેટવર્ક લોગ્સ
        \item \keyword{અવકાશ વ્યાખ્યા}: તપાસની સીમાઓ નક્કી કરવી
        \item \keyword{કાયદેસર અધિકાર}: વોરંટ/પરવાનગીઓ મેળવવી
        \item \keyword{પ્રારંભિક ફોટોગ્રાફી}: સીનની સ્થિતિ દસ્તાવેજીકરણ
    \end{itemize}

    \textbf{2. સંરક્ષણ તબક્કો:}
    \begin{itemize}
        \item \keyword{બિટ-બાય-બિટ ઇમેજિંગ}: ચોક્કસ કોપીઓ બનાવવી
        \item \keyword{હેશ ગણતરી}: અખંડતા ચકાસવી (MD5, SHA)
        \item \keyword{કસ્ટડી ચેન}: પુરાવા ટ્રેઇલ જાળવવી
        \item \keyword{રાઇટ પ્રોટેક્શન}: પુરાવા ફેરફાર અટકાવવો
    \end{itemize}

    \textbf{3. વિશ્લેષણ તબક્કો:}
    \begin{itemize}
        \item \keyword{ડેટા પુનઃપ્રાપ્તિ}: ડિલીટ થયેલી ફાઇલો પુનઃપ્રાપ્ત કરવી
        \item \keyword{કીવર્ડ શોધ}: સંબંધિત માહિતી શોધવી
        \item \keyword{ટાઇમલાઇન વિશ્લેષણ}: ઘટનાઓનું પુનર્નિર્માણ કરવું
        \item \keyword{પેટર્ન ઓળખ}: શંકાસ્પદ પ્રવૃત્તિઓ ઓળખવી
    \end{itemize}

    \textbf{4. દસ્તાવેજીકરણ તબક્કો:}
    \begin{itemize}
        \item \keyword{પદ્ધતિ રેકોર્ડિંગ}: ઉપયોગ કરેલી પ્રક્રિયાઓ દસ્તાવેજીકરણ
        \item \keyword{પુરાવા કેટેલોગિંગ}: બધા શોધો સૂચિબદ્ધ કરવા
        \item \keyword{સ્ક્રીનશોટ કેપ્ચર}: દ્રશ્ય પુરાવા દસ્તાવેજીકરણ
        \item \keyword{રિપોર્ટ તૈયારી}: વ્યાપક તપાસ રિપોર્ટ
    \end{itemize}

    \textbf{5. રજૂઆત તબક્કો:}
    \begin{itemize}
        \item \keyword{નિષ્ણાત સાક્ષ્ય}: કોર્ટમાં હાજરી
        \item \keyword{દ્રશ્ય સહાયતા}: ચાર્ટ્સ, આકૃતિઓ, પ્રદર્શન
        \item \keyword{તકનીકી અનુવાદ}: જટિલ વિભાવનાઓ સમજાવવી
        \item \keyword{ક્રોસ-એક્ઝામિનેશન}: બચાવ પક્ષના પ્રશ્નોના જવાબ
    \end{itemize}

    \textbf{ગુણવત્તા ખાતરી:}
    \begin{itemize}
        \item \keyword{પીઅર રિવ્યુ}: બીજા પરીક્ષકની ચકાસણી
        \item \keyword{ટૂલ માન્યતા}: સોફ્ટવેરની સચોટતા સુનિશ્ચિત કરવી
        \item \keyword{પ્રક્રિયા પાલન}: માનક પ્રોટોકોલ્સ અનુસરવા
        \item \keyword{સતત તાલીમ}: કુશળતા વર્તમાન રાખવી
    \end{itemize}

    \textbf{કાયદેસરી વિચારણાઓ:}
    \begin{itemize}
        \item \keyword{સ્વીકાર્યતા નિયમો}: કોર્ટના ધોરણો પૂરા કરવા
        \item \keyword{ગોપનીયતા સુરક્ષા}: વ્યક્તિગત અધિકારોનું સન્માન કરવું
        \item \keyword{આંતરરાષ્ટ્રીય કાયદો}: ક્રોસ-બોર્ડર તપાસ
        \item \keyword{વ્યાવસાયિક નીતિશાસ્ત્ર}: નિષ્પક્ષતા જાળવવી
    \end{itemize}

    \begin{mnemonicbox}
    "IPADP-ESLR-HTVC-MSCR-ETVI" - Identification/Preservation/Analysis/Documentation/Presentation વિગતવાર પેટા-પ્રવૃત્તિઓ સાથે
    \end{mnemonicbox}
\end{solutionbox}

\end{document}
