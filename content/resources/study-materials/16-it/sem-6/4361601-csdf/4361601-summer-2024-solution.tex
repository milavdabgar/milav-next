\documentclass{article}

% content/resources/templates/preamble.tex
\usepackage[margin=0.6in]{geometry}
\author{Milav Dabgar}
\usepackage{amsmath,amssymb,amsthm}
\usepackage{booktabs}
\usepackage{multirow}
\usepackage{xcolor}
\usepackage{tcolorbox}
\tcbuselibrary{breakable,skins}
\usepackage[colorlinks=true,linkcolor=blue]{hyperref}
\usepackage{titlesec}
\usepackage{enumitem}
\usepackage{tikz}
\usepackage{pgfplots}
\usepackage{circuitikz}
\usepackage[version=4]{mhchem}
\usepackage{longtable}
\usepackage{array}
\usepackage{float}
\usepackage{caption}
\usepackage{listings}

\lstset{
  basicstyle=\small\ttfamily,
  breaklines=true,
  breakatwhitespace=false,
  postbreak=\mbox{\textcolor{red}{$\hookrightarrow$}\space},
  float=false,
  numbers=left,
  numberstyle=\tiny\color{gray},
  numbersep=10pt,
  xleftmargin=2em,
  keywordstyle=\color{blue},
  commentstyle=\color{green!60!black},
  stringstyle=\color{purple},
  backgroundcolor=\color{gray!5},
  showstringspaces=false,
  tabsize=2,
  captionpos=b,
  keepspaces=true,
  columns=flexible
}

\pgfplotsset{compat=1.18}
\usetikzlibrary{shapes,arrows,positioning,calc,patterns,decorations.pathmorphing,decorations.markings,arrows.meta}

% Color scheme
\definecolor{headcolor}{RGB}{0,102,204}
\definecolor{keycolor}{RGB}{220,20,60}
\definecolor{solutioncolor}{RGB}{34,139,34}
\definecolor{mnemoniccolor}{RGB}{148,0,211}
\definecolor{codecolor}{RGB}{0,0,100}

% Spacing
\setlength{\parskip}{3pt}
\setlist[itemize]{nosep}
\setlist[enumerate]{nosep}

% Title formatting
\titleformat{\section}{\Large\bfseries\color{headcolor}}{\thesection}{1em}{}
\titleformat{\subsection}{\large\bfseries\color{headcolor}}{\thesubsection}{1em}{}

% Pandoc tightlist compatibility
\providecommand{\tightlist}{%
  \setlength{\itemsep}{0pt}\setlength{\parskip}{0pt}}

% Pandoc longtable compatibility
\newcounter{none}
\def\thenone{}


% content/resources/templates/english-boxes.tex
% This file is currently empty - it exists to maintain consistency with the import structure.
% Add custom environments here if needed in the future.


% Custom commands for GTU solutions
% This file defines semantic commands for consistent formatting

% Question command with automatic formatting
\newcommand{\question}[2]{%
  \section*{Question #1}%
  \textbf{#2}%
}

% OR question variant
\newcommand{\questionor}[2]{%
  \section*{Question #1 OR}%
  \textbf{#2}%
}

% Proper table environment with caption
\newenvironment{answertable}[1]{%
  \begin{table}[htbp]
  \centering
  \caption{#1}
}{%
  \end{table}
}

% Proper figure environment for diagrams
\newenvironment{answerdiagram}[1]{%
  \begin{figure}[htbp]
  \centering
  \caption{#1}
}{%
  \end{figure}
}

% Semantic markup for key terms
\newcommand{\keyword}[1]{\textbf{#1}}
\newcommand{\code}[1]{\texttt{#1}}
\newcommand{\classname}[1]{\texttt{#1}}
\newcommand{\methodname}[1]{\texttt{#1}}

% Proper quotation marks
\newcommand{\mnemonic}[1]{``#1''}


\title{Cyber Security and Digital Forensics (4361601) - Summer 2024 Solution}
\date{May 14, 2024}

\begin{document}
\maketitle

\questionmarks{1(a)}{3}{Describe CIA triad with example.}
\begin{solutionbox}
    \begin{answertable}{CIA Triad Comparison Table}
    \begin{tabulary}{\textwidth}{|L|L|L|}
        \hline
        \textbf{Component} & \textbf{Definition} & \textbf{Example} \\
        \hline
        \textbf{Confidentiality} & Ensures data is accessible only to authorized users & Bank account details should only be viewed by account holder \\
        \hline
        \textbf{Integrity} & Ensures data remains accurate and unmodified & Medical records must not be altered without authorization \\
        \hline
        \textbf{Availability} & Ensures systems and data are accessible when needed & ATM services must be available 24/7 for customers \\
        \hline
    \end{tabulary}
    \end{answertable}

    \begin{mnemonicbox}
    "Can I Access" - Confidentiality, Integrity, Availability
    \end{mnemonicbox}
\end{solutionbox}

\questionmarks{1(b)}{4}{Explain Public key and Private Key cryptography.}
\begin{solutionbox}
    \begin{answertable}{Key Differences Table}
    \begin{tabulary}{\textwidth}{|L|L|L|}
        \hline
        \textbf{Aspect} & \textbf{Public Key Cryptography} & \textbf{Private Key Cryptography} \\
        \hline
        \textbf{Keys Used} & Two keys (public + private) & Single shared key \\
        \hline
        \textbf{Key Distribution} & Public key can be shared openly & Key must be shared secretly \\
        \hline
        \textbf{Speed} & Slower encryption/decryption & Faster operations \\
        \hline
        \textbf{Security} & Higher security, no key sharing problem & Lower security due to key distribution \\
        \hline
    \end{tabulary}
    \end{answertable}

    \textbf{Key Points:}

    \begin{itemize}
        \item \keyword{Public Key}: Uses asymmetric encryption with key pairs
        \item \keyword{Private Key}: Uses symmetric encryption with shared secrets
        \item \keyword{Digital Signatures}: Public key enables non-repudiation
        \item \keyword{Key Management}: Private key requires secure distribution
    \end{itemize}

    \begin{mnemonicbox}
    "PASS" - Public Asymmetric, Symmetric Secret
    \end{mnemonicbox}
\end{solutionbox}

\questionmarks{1(c)}{7}{Explain various security services and security mechanism.}
\begin{solutionbox}
    \begin{answertable}{Security Services Table}
    \begin{tabulary}{\textwidth}{|L|L|L|}
        \hline
        \textbf{Service} & \textbf{Purpose} & \textbf{Mechanism Example} \\
        \hline
        \textbf{Authentication} & Verify user identity & Passwords, Biometrics \\
        \hline
        \textbf{Authorization} & Control access permissions & Access Control Lists \\
        \hline
        \textbf{Confidentiality} & Protect data privacy & Encryption (AES, RSA) \\
        \hline
        \textbf{Integrity} & Ensure data accuracy & Digital signatures, Hashing \\
        \hline
        \textbf{Non-repudiation} & Prevent denial of actions & Digital certificates \\
        \hline
        \textbf{Availability} & Ensure service accessibility & Firewalls, Backup systems \\
        \hline
    \end{tabulary}
    \end{answertable}

    \textbf{Security Mechanisms:}

    \begin{itemize}
        \item \keyword{Encryption}: Transforms plaintext to ciphertext
        \item \keyword{Digital Signatures}: Provides authentication and integrity
        \item \keyword{Access Control}: Restricts unauthorized access
        \item \keyword{Audit Trails}: Monitor and log security events
    \end{itemize}

    \begin{mnemonicbox}
    "ACIANA" - Authentication, Confidentiality, Integrity, Authorization, Non-repudiation, Availability
    \end{mnemonicbox}
\end{solutionbox}

\orquestionmarks{1(c)}{7}{Explain MD5 hashing algorithm.}
\begin{solutionbox}
    \textbf{MD5 Algorithm Process:}

    \begin{center}
    \begin{tikzpicture}[node distance=2cm, auto]
        \node [gtu block] (input) {Input Message};
        \node [gtu block, right of=input, xshift=1cm] (pad) {Padding};
        \node [gtu block, right of=pad, xshift=1cm] (len) {Append Length};
        \node [gtu block, below of=input] (init) {Initialize MD Buffer};
        \node [gtu block, right of=init, xshift=1cm] (proc) {Process (512-bit blocks)};
        \node [gtu block, right of=proc, xshift=1cm] (out) {128-bit Hash Output};

        \draw [gtu arrow] (input) -- (pad);
        \draw [gtu arrow] (pad) -- (len);
        \draw [gtu arrow] (len) -- (init);
        \draw [gtu arrow] (init) -- (proc);
        \draw [gtu arrow] (proc) -- (out);
    \end{tikzpicture}
    \end{center}

    \begin{answertable}{MD5 Characteristics Table}
    \begin{tabulary}{\textwidth}{|L|L|}
        \hline
        \textbf{Property} & \textbf{Value} \\
        \hline
        \textbf{Hash Size} & 128 bits (16 bytes) \\
        \hline
        \textbf{Block Size} & 512 bits \\
        \hline
        \textbf{Rounds} & 64 rounds \\
        \hline
        \textbf{Security Status} & Cryptographically broken \\
        \hline
    \end{tabulary}
    \end{answertable}

    \textbf{Key Features:}

    \begin{itemize}
        \item \keyword{One-way Function}: Cannot reverse hash to original
        \item \keyword{Fixed Output}: Always produces 128-bit hash
        \item \keyword{Avalanche Effect}: Small input change creates large output change
        \item \keyword{Collision Vulnerable}: Multiple inputs can produce same hash
    \end{itemize}

    \begin{mnemonicbox}
    "MD5 FORB" - Message Digest 5, Fixed Output, Rounds 64, Broken security
    \end{mnemonicbox}
\end{solutionbox}

\questionmarks{2(a)}{3}{What is firewall? List out types of firewall.}
\begin{solutionbox}
    \textbf{Firewall Definition:} Network security device that monitors and controls incoming/outgoing traffic based on predetermined rules.

    \begin{answertable}{Firewall Types Table}
    \begin{tabulary}{\textwidth}{|L|L|L|}
        \hline
        \textbf{Type} & \textbf{Operation Level} & \textbf{Example} \\
        \hline
        \textbf{Packet Filtering} & Network Layer & iptables \\
        \hline
        \textbf{Stateful Inspection} & Session Layer & Cisco ASA \\
        \hline
        \textbf{Application Gateway} & Application Layer & Proxy servers \\
        \hline
        \textbf{Next-Gen Firewall} & Multiple Layers & Palo Alto \\
        \hline
    \end{tabulary}
    \end{answertable}

    \begin{mnemonicbox}
    "PSAN" - Packet, Stateful, Application, Next-gen
    \end{mnemonicbox}
\end{solutionbox}

\questionmarks{2(b)}{4}{Define: HTTPS and describe working of HTTPS.}
\begin{solutionbox}
    \textbf{HTTPS Definition:} HTTP Secure - encrypted version of HTTP using SSL/TLS protocols.

    \textbf{HTTPS Working Process:}

    \begin{center}
    \begin{tikzpicture}[node distance=1.5cm, auto]
        \node [gtu block] (client) {Client};
        \node [gtu block, right of=client, xshift=4cm] (server) {Server};

        \draw [gtu arrow] (client) -- node[above] {1. HTTPS Request} (server);
        \draw [gtu arrow] (server) -- node[below, align=center] {2. SSL Certificate} (client);
        \draw [gtu arrow] ([yshift=-1cm]client.east) -- node[above, align=center] {3. Verify \& Send Key} ([yshift=-1cm]server.west);
        \draw [gtu arrow] ([yshift=-2cm]server.west) -- node[above] {4. Encrypted Data} ([yshift=-2cm]client.east);
    \end{tikzpicture}
    \end{center}

    \textbf{Key Components:}

    \begin{itemize}
        \item \keyword{SSL/TLS}: Provides encryption layer
        \item \keyword{Digital Certificates}: Verify server identity
        \item \keyword{Port 443}: Default HTTPS port
        \item \keyword{End-to-end Encryption}: Protects data in transit
    \end{itemize}

    \begin{mnemonicbox}
    "HTTPS SDP4" - Secure, Digital certs, Port 443
    \end{mnemonicbox}
\end{solutionbox}

\questionmarks{2(c)}{7}{Give explanation of active attack and passive attack in detail.}
\begin{solutionbox}
    \begin{answertable}{Attack Types Comparison}
    \begin{tabulary}{\textwidth}{|L|L|L|}
        \hline
        \textbf{Aspect} & \textbf{Active Attack} & \textbf{Passive Attack} \\
        \hline
        \textbf{Detection} & Easily detectable & Difficult to detect \\
        \hline
        \textbf{System Impact} & Modifies system/data & Only observes data \\
        \hline
        \textbf{Examples} & DoS, Man-in-middle & Eavesdropping, Traffic analysis \\
        \hline
        \textbf{Prevention} & Firewalls, IDS & Encryption, Physical security \\
        \hline
    \end{tabulary}
    \end{answertable}

    \textbf{Active Attack Types:}

    \begin{itemize}
        \item \keyword{Masquerade}: Impersonating authorized user
        \item \keyword{Replay}: Retransmitting valid data transmissions
        \item \keyword{Modification}: Altering message contents
        \item \keyword{Denial of Service}: Preventing legitimate access
    \end{itemize}

    \textbf{Passive Attack Types:}

    \begin{itemize}
        \item \keyword{Traffic Analysis}: Studying communication patterns
        \item \keyword{Eavesdropping}: Monitoring communications
        \item \keyword{Footprinting}: Gathering system information
    \end{itemize}

    \begin{mnemonicbox}
    "Active MRMD, Passive TEF" - Masquerade/Replay/Modify/DoS, Traffic/Eavesdrop/Footprint
    \end{mnemonicbox}
\end{solutionbox}

\orquestionmarks{2(a)}{3}{What is digital signature? Explain digital signature properties.}
\begin{solutionbox}
    \textbf{Digital Signature:} Cryptographic mechanism providing authentication, integrity, and non-repudiation.

    \begin{answertable}{Properties Table}
    \begin{tabulary}{\textwidth}{|L|L|}
        \hline
        \textbf{Property} & \textbf{Description} \\
        \hline
        \textbf{Authentication} & Verifies sender identity \\
        \hline
        \textbf{Integrity} & Ensures message unchanged \\
        \hline
        \textbf{Non-repudiation} & Prevents sender denial \\
        \hline
        \textbf{Unforgeable} & Cannot be created without private key \\
        \hline
    \end{tabulary}
    \end{answertable}

    \begin{mnemonicbox}
    "AINU" - Authentication, Integrity, Non-repudiation, Unforgeable
    \end{mnemonicbox}
\end{solutionbox}

\orquestionmarks{2(b)}{4}{Define: Trojans, Rootkit, Backdoors, Keylogger}
\begin{solutionbox}
    \begin{answertable}{Malware Types Table}
    \begin{tabulary}{\textwidth}{|L|L|L|}
        \hline
        \textbf{Type} & \textbf{Definition} & \textbf{Primary Function} \\
        \hline
        \textbf{Trojans} & Malicious code disguised as legitimate software & Provide unauthorized access \\
        \hline
        \textbf{Rootkit} & Software hiding presence of other malware & Conceal malicious activities \\
        \hline
        \textbf{Backdoors} & Secret entry point bypassing security & Remote unauthorized access \\
        \hline
        \textbf{Keylogger} & Records user keystrokes & Steal passwords/sensitive data \\
        \hline
    \end{tabulary}
    \end{answertable}

    \begin{mnemonicbox}
    "TRBK" - Trojans hide, Rootkits conceal, Backdoors bypass, Keyloggers record
    \end{mnemonicbox}
\end{solutionbox}

\orquestionmarks{2(c)}{7}{Explain Secure Socket Layer.}
\begin{solutionbox}
    \textbf{SSL Architecture:}

    \begin{center}
    \begin{tikzpicture}[node distance=3cm, auto]
        \node [gtu block] (record) {SSL Record Protocol};
        \node [gtu block, above of=record] (app) {Application Layer};
        \node [gtu block, below of=record] (tcp) {TCP Layer};
        
        \node [gtu block, right of=record, xshift=1cm] (handshake) {Handshake Protocol};
        \node [gtu block, above of=handshake, yshift=-1.5cm] (change) {Change Cipher Spec};
        \node [gtu block, below of=handshake, yshift=1.5cm] (alert) {Alert Protocol};

        \draw [gtu arrow] (app) -- (record);
        \draw [gtu arrow] (record) -- (tcp);
        
        \draw [gtu arrow] (handshake) -- (record);
        \draw [gtu arrow] (change) -- (record);
        \draw [gtu arrow] (alert) -- (record);
    \end{tikzpicture}
    \end{center}

    \begin{answertable}{SSL Components Table}
    \begin{tabulary}{\textwidth}{|L|L|}
        \hline
        \textbf{Component} & \textbf{Function} \\
        \hline
        \textbf{Record Protocol} & Provides basic security services \\
        \hline
        \textbf{Handshake Protocol} & Establishes security parameters \\
        \hline
        \textbf{Change Cipher} & Signals encryption changes \\
        \hline
        \textbf{Alert Protocol} & Handles error conditions \\
        \hline
    \end{tabulary}
    \end{answertable}

    \textbf{SSL Process:}

    \begin{itemize}
        \item \keyword{Handshake}: Negotiate security parameters
        \item \keyword{Authentication}: Verify server identity
        \item \keyword{Key Exchange}: Establish session keys
        \item \keyword{Encryption}: Secure data transmission
    \end{itemize}

    \begin{mnemonicbox}
    "SSL RHCA-HAKE" - Record/Handshake/Change/Alert, Handshake/Auth/Key/Encrypt
    \end{mnemonicbox}
\end{solutionbox}

\questionmarks{3(a)}{3}{Explain in detail cybercrime and cybercriminal.}
\begin{solutionbox}
    \begin{answertable}{Definitions Table}
    \begin{tabulary}{\textwidth}{|L|L|}
        \hline
        \textbf{Term} & \textbf{Definition} \\
        \hline
        \textbf{Cybercrime} & Criminal activities carried out using computers/internet \\
        \hline
        \textbf{Cybercriminal} & Individual who commits crimes using digital technology \\
        \hline
    \end{tabulary}
    \end{answertable}

    \textbf{Cybercriminal Types:}

    \begin{itemize}
        \item \keyword{Script Kiddies}: Use existing tools without deep knowledge
        \item \keyword{Hacktivists}: Motivated by political/social causes
        \item \keyword{Organized Crime}: Professional criminal groups
        \item \keyword{State-sponsored}: Government-backed attackers
    \end{itemize}

    \begin{mnemonicbox}
    "SSHT" - Script kiddies, State-sponsored, Hacktivists, Teams organized
    \end{mnemonicbox}
\end{solutionbox}

\questionmarks{3(b)}{4}{Describe cyber stalking and cyber bullying in detail.}
\begin{solutionbox}
    \begin{answertable}{Comparison Table}
    \begin{tabulary}{\textwidth}{|L|L|L|}
        \hline
        \textbf{Aspect} & \textbf{Cyber Stalking} & \textbf{Cyber Bullying} \\
        \hline
        \textbf{Target} & Specific individual (often adult) & Often minors/peers \\
        \hline
        \textbf{Duration} & Long-term harassment & Can be one-time or repeated \\
        \hline
        \textbf{Intent} & Intimidation, control & Humiliation, social exclusion \\
        \hline
        \textbf{Methods} & Monitoring, threatening messages & Social media harassment, spreading rumors \\
        \hline
    \end{tabulary}
    \end{answertable}

    \textbf{Common Characteristics:}

    \begin{itemize}
        \item \keyword{Digital Platforms}: Social media, email, messaging apps
        \item \keyword{Anonymity}: Perpetrators often hide identity
        \item \keyword{Psychological Impact}: Causes emotional distress
        \item \keyword{Legal Consequences}: Violates cyber laws
    \end{itemize}

    \begin{mnemonicbox}
    "STAL-BULL DPAL" - Digital platforms, Psychological impact, Anonymity, Legal issues
    \end{mnemonicbox}
\end{solutionbox}

\questionmarks{3(c)}{7}{Explain Property based classification in cybercrime.}
\begin{solutionbox}
    \begin{answertable}{Property-Based Cybercrime Classification}
    \begin{tabulary}{\textwidth}{|L|L|L|}
        \hline
        \textbf{Crime Type} & \textbf{Description} & \textbf{Example} \\
        \hline
        \textbf{Credit Card Fraud} & Unauthorized use of payment cards & Online shopping with stolen cards \\
        \hline
        \textbf{Software Piracy} & Illegal copying/distribution of software & Downloading copyrighted software \\
        \hline
        \textbf{Copyright Infringement} & Violating intellectual property rights & Sharing movies/music illegally \\
        \hline
        \textbf{Trademark Violations} & Misusing registered trademarks & Creating fake brand websites \\
        \hline
    \end{tabulary}
    \end{answertable}

    \textbf{Impact Assessment:}

    \begin{itemize}
        \item \keyword{Financial Loss}: Direct monetary damage
        \item \keyword{Intellectual Property Theft}: Loss of competitive advantage
        \item \keyword{Brand Reputation}: Damage to company image
        \item \keyword{Legal Costs}: Expenses for prosecution/defense
    \end{itemize}

    \textbf{Prevention Measures:}

    \begin{itemize}
        \item \keyword{Digital Rights Management}: Protect copyrighted content
        \item \keyword{Secure Payment Systems}: Implement fraud detection
        \item \keyword{Legal Enforcement}: Prosecute violators
        \item \keyword{Public Awareness}: Educate about legitimate software
    \end{itemize}

    \begin{mnemonicbox}
    "CSCT-FILP" - Credit/Software/Copyright/Trademark, Financial/Intellectual/Legal/Public
    \end{mnemonicbox}
\end{solutionbox}

\orquestionmarks{3(a)}{3}{Explain Data diddling.}
\begin{solutionbox}
    \textbf{Data Diddling Definition:} Unauthorized alteration of data before/during input into computer system.

    \begin{answertable}{Characteristics Table}
    \begin{tabulary}{\textwidth}{|L|L|}
        \hline
        \textbf{Aspect} & \textbf{Details} \\
        \hline
        \textbf{Method} & Changing data values slightly \\
        \hline
        \textbf{Detection} & Very difficult to detect \\
        \hline
        \textbf{Target} & Financial/sensitive data \\
        \hline
        \textbf{Impact} & Cumulative significant loss \\
        \hline
    \end{tabulary}
    \end{answertable}

    \begin{mnemonicbox}
    "DIDDL" - Data alteration, Input manipulation, Difficult detection, Dollar losses
    \end{mnemonicbox}
\end{solutionbox}

\orquestionmarks{3(b)}{4}{Explain cyber spying and cyber terrorism.}
\begin{solutionbox}
    \begin{answertable}{Comparison Table}
    \begin{tabulary}{\textwidth}{|L|L|L|}
        \hline
        \textbf{Aspect} & \textbf{Cyber Spying} & \textbf{Cyber Terrorism} \\
        \hline
        \textbf{Purpose} & Intelligence gathering & Cause fear/disruption \\
        \hline
        \textbf{Targets} & Government, corporations & Critical infrastructure \\
        \hline
        \textbf{Methods} & Stealth, long-term infiltration & Destructive attacks \\
        \hline
        \textbf{Impact} & Information theft & Physical/economic damage \\
        \hline
    \end{tabulary}
    \end{answertable}

    \textbf{Key Characteristics:}

    \begin{itemize}
        \item \keyword{Cyber Spying}: State-sponsored, corporate espionage
        \item \keyword{Cyber Terrorism}: Ideologically motivated, mass disruption
        \item \keyword{Common Tools}: Malware, social engineering, zero-day exploits
    \end{itemize}

    \begin{mnemonicbox}
    "SPY-TER IGSD" - Intelligence/Government/Stealth/Disruption, Terror/Economic/Rapid/Damage
    \end{mnemonicbox}
\end{solutionbox}

\orquestionmarks{3(c)}{7}{Explain article section 65 and section 66 of cyber law.}
\begin{solutionbox}
    \begin{answertable}{IT Act 2008 Sections}
    \begin{tabulary}{\textwidth}{|L|L|L|}
        \hline
        \textbf{Section} & \textbf{Offense} & \textbf{Punishment} \\
        \hline
        \textbf{Section 65} & Computer source code tampering & Up to 3 years imprisonment or fine up to ₹2 lakh \\
        \hline
        \textbf{Section 66} & Computer-related offenses & Up to 3 years imprisonment or fine up to ₹5 lakh \\
        \hline
    \end{tabulary}
    \end{answertable}

    \textbf{Section 65 Details:}

    \begin{itemize}
        \item \keyword{Scope}: Knowingly/intentionally concealing, destroying, altering computer source code
        \item \keyword{Intent}: When computer source code required to be kept/maintained by law
        \item \keyword{Application}: Protects integrity of essential software systems
    \end{itemize}

    \textbf{Section 66 Details:}

    \begin{itemize}
        \item \keyword{Computer Hacking}: Unauthorized access to computer systems
        \item \keyword{Data Theft}: Downloading, copying, extracting data dishonestly
        \item \keyword{System Damage}: Destroying, deleting, altering information
        \item \keyword{Service Disruption}: Denying access to authorized persons
    \end{itemize}

    \begin{mnemonicbox}
    "65-66 CDHD" - Code tampering, Damage, Hacking, Data theft
    \end{mnemonicbox}
\end{solutionbox}

\questionmarks{4(a)}{3}{What is Hacking? List out types of Hackers.}
\begin{solutionbox}
    \textbf{Hacking Definition:} Unauthorized access to computer systems/networks to exploit vulnerabilities.

    \begin{answertable}{Hacker Types Table}
    \begin{tabulary}{\textwidth}{|L|L|L|}
        \hline
        \textbf{Type} & \textbf{Motivation} & \textbf{Activity} \\
        \hline
        \textbf{White Hat} & Security improvement & Ethical penetration testing \\
        \hline
        \textbf{Black Hat} & Malicious intent & Criminal activities \\
        \hline
        \textbf{Grey Hat} & Mixed motives & Unauthorized but non-malicious \\
        \hline
        \textbf{Script Kiddie} & Recognition & Using existing tools \\
        \hline
    \end{tabulary}
    \end{answertable}

    \begin{mnemonicbox}
    "WBGS Hat" - White, Black, Grey, Script kiddie
    \end{mnemonicbox}
\end{solutionbox}

\questionmarks{4(b)}{4}{Explain Vulnerability and 0-Day terminology of Hacking.}
\begin{solutionbox}
    \begin{answertable}{Terminology Table}
    \begin{tabulary}{\textwidth}{|L|L|L|}
        \hline
        \textbf{Term} & \textbf{Definition} & \textbf{Risk Level} \\
        \hline
        \textbf{Vulnerability} & Security weakness that can be exploited & Medium-High \\
        \hline
        \textbf{0-Day Vulnerability} & Unknown security flaw & Critical \\
        \hline
        \textbf{0-Day Exploit} & Attack code for 0-day vulnerability & Critical \\
        \hline
        \textbf{0-Day Attack} & Active exploitation of 0-day & Critical \\
        \hline
    \end{tabulary}
    \end{answertable}

    \textbf{Key Characteristics:}

    \begin{itemize}
        \item \keyword{Unknown to Vendors}: No patches available
        \item \keyword{High Value}: Sold in dark markets
        \item \keyword{Stealthy}: Difficult to detect
        \item \keyword{Time-Critical}: Value decreases after disclosure
    \end{itemize}

    \begin{mnemonicbox}
    "0-Day UHST" - Unknown, High-value, Stealthy, Time-critical
    \end{mnemonicbox}
\end{solutionbox}

\questionmarks{4(c)}{7}{Explain Five Steps of Hacking.}
\begin{solutionbox}
    \textbf{Hacking Process Flow:}

    \begin{center}
    \begin{tikzpicture}[node distance=1.8cm, auto]
        \node [gtu block] (info) {Information Gathering};
        \node [gtu block, right of=info, xshift=1.5cm] (scan) {Scanning};
        \node [gtu block, right of=scan, xshift=1.5cm] (gain) {Gaining Access};
        \node [gtu block, below of=info] (maint) {Maintaining Access};
        \node [gtu block, below of=scan] (cover) {Covering Tracks};

        \draw [gtu arrow] (info) -- (scan);
        \draw [gtu arrow] (scan) -- (gain);
        \draw [gtu arrow] (gain) -- (maint);
        \draw [gtu arrow] (maint) -- (cover);
    \end{tikzpicture}
    \end{center}

    \begin{answertable}{Five Steps Detailed}
    \begin{tabulary}{\textwidth}{|L|L|L|}
        \hline
        \textbf{Step} & \textbf{Purpose} & \textbf{Tools/Techniques} \\
        \hline
        \textbf{1. Information Gathering} & Collect target information & OSINT, Social engineering \\
        \hline
        \textbf{2. Scanning} & Identify live systems, ports & Nmap, Port scanners \\
        \hline
        \textbf{3. Gaining Access} & Exploit vulnerabilities & Metasploit, Custom exploits \\
        \hline
        \textbf{4. Maintaining Access} & Establish persistent presence & Backdoors, Rootkits \\
        \hline
        \textbf{5. Covering Tracks} & Remove evidence & Log deletion, File cleanup \\
        \hline
    \end{tabulary}
    \end{answertable}

    \textbf{Each Step Details:}

    \begin{itemize}
        \item \keyword{Information Gathering}: Passive/Active reconnaissance
        \item \keyword{Scanning}: Network mapping, vulnerability assessment
        \item \keyword{Gaining Access}: Password attacks, buffer overflows
        \item \keyword{Maintaining Access}: Privilege escalation, backdoor installation
        \item \keyword{Covering Tracks}: Anti-forensics techniques
    \end{itemize}

    \begin{mnemonicbox}
    "ISGMC" - Information, Scanning, Gaining, Maintaining, Covering
    \end{mnemonicbox}
\end{solutionbox}

\orquestionmarks{4(a)}{3}{Explain any three basic commands of kali Linux with suitable example.}
\begin{solutionbox}
    \begin{answertable}{Kali Linux Commands Table}
    \begin{tabulary}{\textwidth}{|L|L|L|}
        \hline
        \textbf{Command} & \textbf{Purpose} & \textbf{Example} \\
        \hline
        \textbf{nmap} & Network scanning & \code{nmap -sS 192.168.1.1} \\
        \hline
        \textbf{netcat} & Network utility & \code{nc -l -p 4444} \\
        \hline
        \textbf{john} & Password cracking & \code{john --wordlist=passwords.txt hashes.txt} \\
        \hline
    \end{tabulary}
    \end{answertable}

    \textbf{Command Details:}

    \begin{itemize}
        \item \keyword{nmap}: Stealth SYN scan on target IP
        \item \keyword{netcat}: Listen on port 4444 for connections
        \item \keyword{john}: Dictionary attack on password hashes
    \end{itemize}

    \begin{mnemonicbox}
    "NNJ" - Nmap scans, Netcat listens, John cracks
    \end{mnemonicbox}
\end{solutionbox}

\orquestionmarks{4(b)}{4}{Describe Session Hijacking in detail.}
\begin{solutionbox}
    \textbf{Session Hijacking Process:}

    \begin{center}
    \begin{tikzpicture}[node distance=1.5cm, auto]
        \node [gtu block] (user) {User};
        \node [gtu block, right of=user, xshift=2cm] (server) {Server};
        \node [gtu block, below of=user] (attacker) {Attacker};

        \draw [gtu arrow] (user) -- node[above, font=\footnotesize] {1. Login, Get ID} (server);
        \draw [gtu arrow, dashed] (user) -- node[left, font=\footnotesize] {2. Sniff ID} (attacker);
        \draw [gtu arrow] (attacker) -- node[below, font=\footnotesize] {3. Use Stolen ID} (server);
        \draw [gtu arrow] (server) -- node[right, font=\footnotesize] {4. Grant Access} (attacker);
    \end{tikzpicture}
    \end{center}

    \textbf{Types and Methods:}

    \begin{itemize}
        \item \keyword{Active Hijacking}: Attacker actively participates
        \item \keyword{Passive Hijacking}: Monitor and capture sessions
        \item \keyword{Network Level}: IP spoofing, ARP poisoning
        \item \keyword{Application Level}: Session ID prediction, XSS
    \end{itemize}

    \textbf{Prevention Measures:}

    \begin{itemize}
        \item \keyword{HTTPS}: Encrypt session data
        \item \keyword{Session Timeouts}: Limit session duration
        \item \keyword{IP Binding}: Tie sessions to IP addresses
        \item \keyword{Strong Session IDs}: Use unpredictable tokens
    \end{itemize}

    \begin{mnemonicbox}
    "APNA-HSIS" - Active/Passive/Network/Application, HTTPS/Strong/IP/Session
    \end{mnemonicbox}
\end{solutionbox}

\orquestionmarks{4(c)}{7}{Explain Remote Administration Tools.}
\begin{solutionbox}
    \textbf{RAT Definition:} Software allowing remote control of computer systems, often used maliciously.

    \begin{answertable}{RAT Functionality Table}
    \begin{tabulary}{\textwidth}{|L|L|L|}
        \hline
        \textbf{Function} & \textbf{Description} & \textbf{Risk Level} \\
        \hline
        \textbf{Screen Capture} & Take screenshots remotely & Medium \\
        \hline
        \textbf{Keylogging} & Record keystrokes & High \\
        \hline
        \textbf{File Transfer} & Upload/download files & High \\
        \hline
        \textbf{Camera Access} & Activate webcam/microphone & Critical \\
        \hline
    \end{tabulary}
    \end{answertable}

    \begin{answertable}{Legitimate vs Malicious Use}
    \begin{tabulary}{\textwidth}{|L|L|L|}
        \hline
        \textbf{Aspect} & \textbf{Legitimate} & \textbf{Malicious} \\
        \hline
        \textbf{Purpose} & IT support, administration & Espionage, theft \\
        \hline
        \textbf{Consent} & User aware and consenting & Installed without knowledge \\
        \hline
        \textbf{Access} & Authorized personnel only & Unauthorized attackers \\
        \hline
    \end{tabulary}
    \end{answertable}

    \textbf{Detection and Prevention:}

    \begin{itemize}
        \item \keyword{Antivirus}: Detect known RAT signatures
        \item \keyword{Network Monitoring}: Unusual outbound connections
        \item \keyword{User Education}: Avoid suspicious downloads
        \item \keyword{Firewall Rules}: Block unauthorized connections
    \end{itemize}

    \textbf{Common RATs:}

    \begin{itemize}
        \item \keyword{TeamViewer}: Legitimate remote access
        \item \keyword{DarkComet}: Malicious RAT
        \item \keyword{Poison Ivy}: Advanced persistent threat tool
    \end{itemize}

    \begin{mnemonicbox}
    "RAT SKFC-ANUM" - Screen/Key/File/Camera, Antivirus/Network/User/Monitoring
    \end{mnemonicbox}
\end{solutionbox}

\questionmarks{5(a)}{3}{Explain Mobile forensics.}
\begin{solutionbox}
    \textbf{Mobile Forensics Definition:} Process of recovering digital evidence from mobile devices using scientifically accepted methods.

    \begin{answertable}{Key Aspects Table}
    \begin{tabulary}{\textwidth}{|L|L|}
        \hline
        \textbf{Aspect} & \textbf{Description} \\
        \hline
        \textbf{Data Types} & Call logs, SMS, photos, app data \\
        \hline
        \textbf{Challenges} & Encryption, anti-forensics, variety of OS \\
        \hline
        \textbf{Tools} & Cellebrite, XRY, Oxygen Suite \\
        \hline
        \textbf{Legal} & Chain of custody, court admissibility \\
        \hline
    \end{tabulary}
    \end{answertable}

    \begin{mnemonicbox}
    "DCTL" - Data types, Challenges, Tools, Legal requirements
    \end{mnemonicbox}
\end{solutionbox}

\questionmarks{5(b)}{4}{What is Digital forensics? Write down advantages of Digital forensics.}
\begin{solutionbox}
    \textbf{Digital Forensics Definition:} Scientific examination of digital devices to recover and analyze evidence for legal proceedings.

    \begin{answertable}{Advantages Table}
    \begin{tabulary}{\textwidth}{|L|L|}
        \hline
        \textbf{Advantage} & \textbf{Description} \\
        \hline
        \textbf{Evidence Recovery} & Retrieve deleted/hidden data \\
        \hline
        \textbf{Crime Solving} & Provide crucial evidence for cases \\
        \hline
        \textbf{Cost Effective} & Cheaper than traditional investigation \\
        \hline
        \textbf{Accurate Results} & Scientific methods ensure reliability \\
        \hline
    \end{tabulary}
    \end{answertable}

    \textbf{Additional Benefits:}

    \begin{itemize}
        \item \keyword{Time Efficient}: Faster than manual investigation
        \item \keyword{Non-destructive}: Preserves original evidence
        \item \keyword{Comprehensive}: Analyzes multiple data sources
        \item \keyword{Court Acceptable}: Legally admissible evidence
    \end{itemize}

    \begin{mnemonicbox}
    "ECCA-TNCA" - Evidence/Crime/Cost/Accurate, Time/Non-destructive/Comprehensive/Admissible
    \end{mnemonicbox}
\end{solutionbox}

\questionmarks{5(c)}{7}{Describe in detail Locard's Principle of exchange in Digital Forensics.}
\begin{solutionbox}
    \textbf{Locard's Principle:} "Every contact leaves a trace" - any interaction between objects results in exchange of materials.

    \textbf{Digital Application:}

    \begin{center}
    \begin{tikzpicture}[node distance=1.5cm, auto]
        \node [gtu block] (action) {User Action};
        \node [gtu block, right of=action, xshift=2cm] (traces) {Digital Traces};
        
        \node [gtu block, above of=traces, xshift=3cm] (log) {Log Files};
        \node [gtu block, below of=log] (reg) {Registry Entries};
        \node [gtu block, below of=reg] (meta) {File Metadata};
        \node [gtu block, below of=meta] (net) {Network Traffic};

        \draw [gtu arrow] (action) -- (traces);
        \draw [gtu arrow] (traces) -- (log);
        \draw [gtu arrow] (traces) -- (reg);
        \draw [gtu arrow] (traces) -- (meta);
        \draw [gtu arrow] (traces) -- (net);
    \end{tikzpicture}
    \end{center}

    \begin{answertable}{Digital Traces Table}
    \begin{tabulary}{\textwidth}{|L|L|L|}
        \hline
        \textbf{Action} & \textbf{Digital Trace} & \textbf{Location} \\
        \hline
        \textbf{File Access} & Access timestamps & File system metadata \\
        \hline
        \textbf{Web Browsing} & Browser history & Browser databases \\
        \hline
        \textbf{Email Sending} & Email headers & Mail server logs \\
        \hline
        \textbf{USB Connection} & Device registry & Windows registry \\
        \hline
    \end{tabulary}
    \end{answertable}

    \textbf{Forensic Implications:}

    \begin{itemize}
        \item \keyword{Persistence}: Digital traces often persist longer
        \item \keyword{Accuracy}: Precise timestamps and data
        \item \keyword{Volume}: Large amounts of trace evidence
        \item \keyword{Recovery}: Deleted data can be recovered
    \end{itemize}

    \textbf{Evidence Types:}

    \begin{itemize}
        \item \keyword{Temporal}: When actions occurred
        \item \keyword{Spatial}: Where actions took place
        \item \keyword{Relational}: Connections between entities
        \item \keyword{Behavioral}: Patterns of user activity
    \end{itemize}

    \textbf{Applications:}

    \begin{itemize}
        \item \keyword{Criminal Cases}: Prove presence/actions
        \item \keyword{Civil Litigation}: Business disputes
        \item \keyword{Internal Investigations}: Employee misconduct
        \item \keyword{Incident Response}: Security breach analysis
    \end{itemize}

    \begin{mnemonicbox}
    "LOCARD PVAR-TREB" - Persistence/Volume/Accuracy/Recovery, Temporal/Relational/Evidence/Behavioral
    \end{mnemonicbox}
\end{solutionbox}

\orquestionmarks{5(a)}{3}{Explain Network forensics.}
\begin{solutionbox}
    \textbf{Network Forensics Definition:} Monitoring and analysis of network traffic to gather information and evidence.

    \begin{answertable}{Key Components Table}
    \begin{tabulary}{\textwidth}{|L|L|}
        \hline
        \textbf{Component} & \textbf{Function} \\
        \hline
        \textbf{Packet Capture} & Record network traffic \\
        \hline
        \textbf{Traffic Analysis} & Examine communication patterns \\
        \hline
        \textbf{Protocol Analysis} & Decode network protocols \\
        \hline
        \textbf{Timeline Creation} & Establish sequence of events \\
        \hline
    \end{tabulary}
    \end{answertable}

    \begin{mnemonicbox}
    "PTTP" - Packet capture, Traffic analysis, Timeline, Protocol analysis
    \end{mnemonicbox}
\end{solutionbox}

\orquestionmarks{5(b)}{4}{Explain why CCTV plays an important role as evidence in digital forensics investigations.}
\begin{solutionbox}
    \begin{answertable}{CCTV Evidence Value}
    \begin{tabulary}{\textwidth}{|L|L|}
        \hline
        \textbf{Aspect} & \textbf{Importance} \\
        \hline
        \textbf{Visual Proof} & Direct evidence of events \\
        \hline
        \textbf{Timestamp} & Precise time correlation \\
        \hline
        \textbf{Location Verification} & Proves presence at scene \\
        \hline
        \textbf{Behavior Analysis} & Shows actions and intent \\
        \hline
    \end{tabulary}
    \end{answertable}

    \textbf{Digital Forensics Integration:}

    \begin{itemize}
        \item \keyword{Metadata Extraction}: Camera settings, timestamps
        \item \keyword{Video Enhancement}: Improve image quality
        \item \keyword{Format Analysis}: Understand compression artifacts
        \item \keyword{Authentication}: Verify video integrity
    \end{itemize}

    \textbf{Legal Considerations:}

    \begin{itemize}
        \item \keyword{Chain of Custody}: Maintain evidence integrity
        \item \keyword{Court Admissibility}: Follow legal procedures
        \item \keyword{Privacy Rights}: Respect surveillance laws
        \item \keyword{Technical Validation}: Prove authenticity
    \end{itemize}

    \begin{mnemonicbox}
    "VTLB-MFAC" - Visual/Timestamp/Location/Behavior, Metadata/Format/Authentication/Chain
    \end{mnemonicbox}
\end{solutionbox}

\orquestionmarks{5(c)}{7}{Explain phases of Digital forensic investigation.}
\begin{solutionbox}
    \textbf{Digital Forensic Investigation Phases:}

    \begin{center}
    \begin{tikzpicture}[node distance=2.2cm, auto]
        \node [gtu block] (id) {Identification};
        \node [gtu block, right of=id] (pres) {Preservation};
        \node [gtu block, right of=pres] (anal) {Analysis};
        \node [gtu block, right of=anal] (doc) {Documentation};
        \node [gtu block, right of=doc] (presen) {Presentation};

        \draw [gtu arrow] (id) -- (pres);
        \draw [gtu arrow] (pres) -- (anal);
        \draw [gtu arrow] (anal) -- (doc);
        \draw [gtu arrow] (doc) -- (presen);
    \end{tikzpicture}
    \end{center}

    \begin{answertable}{Phase Details Table}
    \begin{tabulary}{\textwidth}{|L|L|L|}
        \hline
        \textbf{Phase} & \textbf{Activities} & \textbf{Tools/Methods} \\
        \hline
        \textbf{Identification} & Locate potential evidence sources & Initial assessment, Scene survey \\
        \hline
        \textbf{Preservation} & Secure evidence without alteration & Imaging, Hash verification \\
        \hline
        \textbf{Analysis} & Examine evidence for relevant data & Forensic software, Manual review \\
        \hline
        \textbf{Documentation} & Record findings and procedures & Reports, Screenshots, Logs \\
        \hline
        \textbf{Presentation} & Present findings to stakeholders & Court testimony, Expert reports \\
        \hline
    \end{tabulary}
    \end{answertable}

    \textbf{Detailed Activities:}

    \textbf{1. Identification Phase:}
    \begin{itemize}
        \item \keyword{Evidence Sources}: Computers, phones, servers, network logs
        \item \keyword{Scope Definition}: Determine investigation boundaries
        \item \keyword{Legal Authorization}: Obtain warrants/permissions
        \item \keyword{Initial Photography}: Document scene condition
    \end{itemize}

    \textbf{2. Preservation Phase:}
    \begin{itemize}
        \item \keyword{Bit-by-bit Imaging}: Create exact copies
        \item \keyword{Hash Calculation}: Verify integrity (MD5, SHA)
        \item \keyword{Chain of Custody}: Maintain evidence trail
        \item \keyword{Write Protection}: Prevent evidence modification
    \end{itemize}

    \textbf{3. Analysis Phase:}
    \begin{itemize}
        \item \keyword{Data Recovery}: Retrieve deleted files
        \item \keyword{Keyword Searching}: Find relevant information
        \item \keyword{Timeline Analysis}: Reconstruct events
        \item \keyword{Pattern Recognition}: Identify suspicious activities
    \end{itemize}

    \textbf{4. Documentation Phase:}
    \begin{itemize}
        \item \keyword{Methodology Recording}: Document procedures used
        \item \keyword{Evidence Cataloging}: List all findings
        \item \keyword{Screenshot Capture}: Visual evidence documentation
        \item \keyword{Report Preparation}: Comprehensive investigation report
    \end{itemize}

    \textbf{5. Presentation Phase:}
    \begin{itemize}
        \item \keyword{Expert Testimony}: Court appearances
        \item \keyword{Visual Aids}: Charts, diagrams, demonstrations
        \item \keyword{Technical Translation}: Explain complex concepts
        \item \keyword{Cross-examination}: Answer defense questions
    \end{itemize}

    \textbf{Quality Assurance:}
    \begin{itemize}
        \item \keyword{Peer Review}: Second examiner verification
        \item \keyword{Tool Validation}: Ensure software accuracy
        \item \keyword{Procedure Adherence}: Follow standard protocols
        \item \keyword{Continuous Training}: Keep skills current
    \end{itemize}

    \textbf{Legal Considerations:}
    \begin{itemize}
        \item \keyword{Admissibility Rules}: Meet court standards
        \item \keyword{Privacy Protection}: Respect individual rights
        \item \keyword{International Law}: Cross-border investigations
        \item \keyword{Professional Ethics}: Maintain objectivity
    \end{itemize}

    \begin{mnemonicbox}
    "IPADP-ESLR-HTVC-MSCR-ETVI" - Identification/Preservation/Analysis/Documentation/Presentation with detailed sub-activities
    \end{mnemonicbox}
\end{solutionbox}

\end{document}
