\documentclass[10pt,a4paper]{article}

% content/resources/templates/preamble.tex
\usepackage[margin=0.6in]{geometry}
\author{Milav Dabgar}
\usepackage{amsmath,amssymb,amsthm}
\usepackage{booktabs}
\usepackage{multirow}
\usepackage{xcolor}
\usepackage{tcolorbox}
\tcbuselibrary{breakable,skins}
\usepackage[colorlinks=true,linkcolor=blue]{hyperref}
\usepackage{titlesec}
\usepackage{enumitem}
\usepackage{tikz}
\usepackage{pgfplots}
\usepackage{circuitikz}
\usepackage[version=4]{mhchem}
\usepackage{longtable}
\usepackage{array}
\usepackage{float}
\usepackage{caption}
\usepackage{listings}

\lstset{
  basicstyle=\small\ttfamily,
  breaklines=true,
  breakatwhitespace=false,
  postbreak=\mbox{\textcolor{red}{$\hookrightarrow$}\space},
  float=false,
  numbers=left,
  numberstyle=\tiny\color{gray},
  numbersep=10pt,
  xleftmargin=2em,
  keywordstyle=\color{blue},
  commentstyle=\color{green!60!black},
  stringstyle=\color{purple},
  backgroundcolor=\color{gray!5},
  showstringspaces=false,
  tabsize=2,
  captionpos=b,
  keepspaces=true,
  columns=flexible
}

\pgfplotsset{compat=1.18}
\usetikzlibrary{shapes,arrows,positioning,calc,patterns,decorations.pathmorphing,decorations.markings,arrows.meta}

% Color scheme
\definecolor{headcolor}{RGB}{0,102,204}
\definecolor{keycolor}{RGB}{220,20,60}
\definecolor{solutioncolor}{RGB}{34,139,34}
\definecolor{mnemoniccolor}{RGB}{148,0,211}
\definecolor{codecolor}{RGB}{0,0,100}

% Spacing
\setlength{\parskip}{3pt}
\setlist[itemize]{nosep}
\setlist[enumerate]{nosep}

% Title formatting
\titleformat{\section}{\Large\bfseries\color{headcolor}}{\thesection}{1em}{}
\titleformat{\subsection}{\large\bfseries\color{headcolor}}{\thesubsection}{1em}{}

% Pandoc tightlist compatibility
\providecommand{\tightlist}{%
  \setlength{\itemsep}{0pt}\setlength{\parskip}{0pt}}

% Pandoc longtable compatibility
\newcounter{none}
\def\thenone{}


% content/resources/templates/gujarati-boxes.tex
\usepackage{fontspec}
\usepackage{polyglossia}

% Set Gujarati as main language (document is primarily in Gujarati)
% Note: gloss-gujarati.ldf doesn't exist in polyglossia, but it will use hyphenation patterns
\setdefaultlanguage{gujarati}
\setotherlanguage{english}

% Configure Gujarati font properly
% Use Language=Default to prevent polyglossia from trying to add language-specific features
% that don't exist for Gujarati, which causes "empty feature" warnings
\newfontfamily\gujaratifont[Script=Gujarati,AutoFakeBold=2.5,AutoFakeSlant=0.3]{Noto Sans Gujarati}
\setmainfont[Script=Gujarati,AutoFakeBold=2.5,AutoFakeSlant=0.3]{Noto Sans Gujarati}
% Use Noto Sans Gujarati for monospace to support Gujarati in text
\setmonofont[Scale=0.9]{Noto Sans Gujarati}

% Configure English to use the same font
\newfontfamily\englishfont[Script=Gujarati,AutoFakeBold=2.5,AutoFakeSlant=0.3]{Noto Sans Gujarati}

% Translations for polyglossia
\gappto\captionsgujarati{
  \renewcommand{\tablename}{કોષ્ટક}
  \renewcommand{\figurename}{આકૃતિ}
}

% Helper for TikZ nodes to ensure Gujarati font
\newcommand{\gu}[1]{{\gujaratifont #1}}

% Custom environments
\newtcolorbox{solutionbox}{
    breakable,
    enhanced,
    colback=solutioncolor!5!white,
    colframe=solutioncolor!75!black,
    fonttitle=\bfseries,
    title=જવાબ
}

\newtcolorbox{solutionboxnobreak}{
 colback=solutioncolor!5!white,
 colframe=solutioncolor!75!black,
 fonttitle=\bfseries,
 title=જવાબ
}

\newtcolorbox{keyformula}{
 breakable,
 enhanced,
 colback=keycolor!5!white,
 colframe=keycolor!75!black,
 fonttitle=\bfseries,
 title=રાસાયણિક સમીકરણ/સૂત્ર
}

\newtcolorbox{mnemonicbox}{
 breakable,
 enhanced,
 colback=mnemoniccolor!5!white,
 colframe=mnemoniccolor!75!black,
 fonttitle=\bfseries,
 title=મેમરી ટ્રીક
}


\begin{document}

\begin{center}
{\Huge\bfseries\color{headcolor} Subject Name (Gujarati)}\\[5pt]
{\LARGE 4361603 -- Summer 2024}\\[3pt]
{\large Semester 1 Study Material}\\[3pt]
{\normalsize\textit{Detailed Solutions and Explanations}}
\end{center}

\vspace{10pt}

\subsection*{પ્રશ્ન 1(અ) [3
ગુણ]}\label{uxaaauxab0uxab6uxaa8-1uxa85-3-uxa97uxaa3}

\textbf{Distributed ledger systems ના ઉપયોગમાં લેવાના ફાયદાઓ સમજાવો.}

\begin{solutionbox}

\textbf{ટેબલ: Distributed Ledger Systems ના ફાયદાઓ}

{\def\LTcaptype{none} % do not increment counter
\begin{longtable}[]{@{}ll@{}}
\toprule\noalign{}
ફાયદો & વર્ણન \\
\midrule\noalign{}
\endhead
\bottomrule\noalign{}
\endlastfoot
\textbf{પારદર્શિતા} & બધા સહભાગીઓ transaction history જોઈ શકે છે \\
\textbf{સુરક્ષા} & Cryptographic સુરક્ષા છેડછાડ સામે \\
\textbf{વિકેન્દ્રીકરણ} & એક જ નિયંત્રણ અથવા નિષ્ફળતાનું બિંદુ નથી \\
\textbf{અપરિવર્તનીયતા} & એકવાર confirm થયા પછી records બદલી શકાતા નથી \\
\end{longtable}
}

\end{solutionbox}
\begin{mnemonicbox}
``T-S-D-I'' (Transparent, Secure, Decentralized,
Immutable)

\end{mnemonicbox}
\subsection*{પ્રશ્ન 1(બ) [4
ગુણ]}\label{uxaaauxab0uxab6uxaa8-1uxaac-4-uxa97uxaa3}

\textbf{વ્યાખ્યાયિત કરો: 1) Blockchain 2) Distributed systems}

\begin{solutionbox}

\textbf{ટેબલ: મુખ્ય વ્યાખ્યાઓ}

{\def\LTcaptype{none} % do not increment counter
\begin{longtable}[]{@{}
  >{\raggedright\arraybackslash}p{(\linewidth - 2\tabcolsep) * \real{0.4167}}
  >{\raggedright\arraybackslash}p{(\linewidth - 2\tabcolsep) * \real{0.5833}}@{}}
\toprule\noalign{}
\begin{minipage}[b]{\linewidth}\raggedright
શબ્દ
\end{minipage} & \begin{minipage}[b]{\linewidth}\raggedright
વ્યાખ્યા
\end{minipage} \\
\midrule\noalign{}
\endhead
\bottomrule\noalign{}
\endlastfoot
\textbf{Blockchain} & Transaction data ધરાવતા blocks ની chain,
cryptographic hashes દ્વારા જોડાયેલ \\
\textbf{Distributed Systems} & સ્વતંત્ર computers નું network એક single
system તરીકે કામ કરતું \\
\end{longtable}
}

\textbf{મુખ્ય લાક્ષણિકતાઓ}:

\begin{itemize}
\tightlist
\item
  \textbf{Blockchain}: Hash pointers, consensus mechanisms, અને merkle
  trees વાપરે છે
\item
  \textbf{Distributed Systems}: Fault tolerance, scalability, અને
  resource sharing
\end{itemize}

\end{solutionbox}
\begin{mnemonicbox}
Blockchain માટે ``Chain-Hash-Consensus'', Distributed
માટે ``Network-Independent-Together''

\end{mnemonicbox}
\subsection*{પ્રશ્ન 1(ક) [7
ગુણ]}\label{uxaaauxab0uxab6uxaa8-1uxa95-7-uxa97uxaa3}

\textbf{Blockchain network વડે CAP theorem વર્ણવો.}

\begin{solutionbox}

\textbf{ટેબલ: CAP Theorem ના ઘટકો}

{\def\LTcaptype{none} % do not increment counter
\begin{longtable}[]{@{}
  >{\raggedright\arraybackslash}p{(\linewidth - 4\tabcolsep) * \real{0.2581}}
  >{\raggedright\arraybackslash}p{(\linewidth - 4\tabcolsep) * \real{0.1935}}
  >{\raggedright\arraybackslash}p{(\linewidth - 4\tabcolsep) * \real{0.5484}}@{}}
\toprule\noalign{}
\begin{minipage}[b]{\linewidth}\raggedright
ગુણધર્મ
\end{minipage} & \begin{minipage}[b]{\linewidth}\raggedright
વર્ણન
\end{minipage} & \begin{minipage}[b]{\linewidth}\raggedright
Blockchain સંદર્ભ
\end{minipage} \\
\midrule\noalign{}
\endhead
\bottomrule\noalign{}
\endlastfoot
\textbf{Consistency} & બધા nodes એ જ data જુએ છે & બધા nodes પાસે સમાન
ledger \\
\textbf{Availability} & System કાર્યરત રહે છે & Network accessible રહે છે \\
\textbf{Partition Tolerance} & Network failures છતાં કામ કરે છે & Node
disconnections દર્મિયાન ચાલુ રહે છે \\
\end{longtable}
}

\textbf{આકૃતિ:}

\begin{center}
\textbf{Mermaid Diagram (Code)}
\begin{verbatim}
{Shaded}
{Highlighting}[]
graph TD
    A[CAP Theorem] {-{-}{} B[Consistency]}
    A {-{-}{} C[Availability] }
    A {-{-}{} D[Partition Tolerance]}
    B {-{-}{} E[Bitcoin C+P ને પ્રાધાન્ય આપે છે]}
    C {-{-}{} F[કેટલાક systems A+P પસંદ કરે છે]}
    D {-{-}{} G[Blockchain માટે જરૂરી]}
{Highlighting}
{Shaded}
\end{verbatim}
\end{center}

\textbf{મુખ્ય મુદ્દાઓ}:

\begin{itemize}
\tightlist
\item
  \textbf{Trade-off}: 3 માંથી માત્ર 2 properties એક સાથે મેળવી શકાય
\item
  \textbf{Blockchain Choice}: મોટાભાગના blockchains Consistency +
  Partition Tolerance પસંદ કરે છે
\item
  \textbf{ઉદાહરણ}: Bitcoin અસ્થાયી રૂપે unavailable બની શકે પણ consistency
  જાળવે છે
\end{itemize}

\end{solutionbox}
\begin{mnemonicbox}
``CAP-2-out-of-3'' (3 માંથી કોઈ પણ 2 Properties પસંદ
કરો)

\end{mnemonicbox}
\subsection*{પ્રશ્ન 1(ક) OR [7
ગુણ]}\label{uxaaauxab0uxab6uxaa8-1uxa95-or-7-uxa97uxaa3}

\textbf{Blockchain network ની ઉપયોગિતાઓ યાદી બનાવો અને સમજાવો.}

\begin{solutionbox}

\textbf{ટેબલ: Blockchain Applications}

{\def\LTcaptype{none} % do not increment counter
\begin{longtable}[]{@{}
  >{\raggedright\arraybackslash}p{(\linewidth - 4\tabcolsep) * \real{0.4483}}
  >{\raggedright\arraybackslash}p{(\linewidth - 4\tabcolsep) * \real{0.2414}}
  >{\raggedright\arraybackslash}p{(\linewidth - 4\tabcolsep) * \real{0.3103}}@{}}
\toprule\noalign{}
\begin{minipage}[b]{\linewidth}\raggedright
Application
\end{minipage} & \begin{minipage}[b]{\linewidth}\raggedright
વર્ણન
\end{minipage} & \begin{minipage}[b]{\linewidth}\raggedright
ઉદાહરણ
\end{minipage} \\
\midrule\noalign{}
\endhead
\bottomrule\noalign{}
\endlastfoot
\textbf{Cryptocurrency} & Digital money transactions & Bitcoin,
Ethereum \\
\textbf{Supply Chain} & ઉત્પાદનો ને origin થી track કરવું & Walmart food
tracing \\
\textbf{Healthcare} & સુરક્ષિત patient records & Medical data sharing \\
\textbf{Voting} & પારદર્શી elections & Estonia e-voting \\
\textbf{Real Estate} & Property ownership records & Land registries \\
\end{longtable}
}

\textbf{મુખ્ય ફાયદાઓ}:

\begin{itemize}
\tightlist
\item
  \textbf{પારદર્શિતા}: બધા transactions સહભાગીઓને દૃશ્યમાન
\item
  \textbf{સુરક્ષા}: છેતરપિંડી સામે cryptographic સુરક્ષા
\item
  \textbf{કાર્યક્ષમતા}: મધ્યસ્થીઓ અને ખર્ચમાં ઘટાડો
\end{itemize}

\end{solutionbox}
\begin{mnemonicbox}
``C-S-H-V-R'' (Crypto, Supply, Health, Vote, Real
estate)

\end{mnemonicbox}
\subsection*{પ્રશ્ન 2(અ) [3
ગુણ]}\label{uxaaauxab0uxab6uxaa8-2uxa85-3-uxa97uxaa3}

\textbf{Permissionless blockchain ની વ્યાખ્યા કરો અને સમજાવો.}

\begin{solutionbox}

\textbf{વ્યાખ્યા}: એક blockchain જ્યાં કોઈપણ વ્યક્તિ કેન્દ્રીય સત્તાધિકારીની
પરવાનગી વગર ભાગ લઈ શકે છે.

\textbf{ટેબલ: Permissionless Blockchain ની લાક્ષણિકતાઓ}

{\def\LTcaptype{none} % do not increment counter
\begin{longtable}[]{@{}ll@{}}
\toprule\noalign{}
લાક્ષણિકતા & વર્ણન \\
\midrule\noalign{}
\endhead
\bottomrule\noalign{}
\endlastfoot
\textbf{ખુલ્લી પહોંચ} & કોઈપણ join કરી અને ભાગ લઈ શકે છે \\
\textbf{જાહેર ચકાસણી} & બધા transactions જાહેરમાં verifiable છે \\
\textbf{વિકેન્દ્રીકૃત} & કોઈ કેન્દ્રીય નિયંત્રણ સત્તા નથી \\
\end{longtable}
}

\textbf{મુખ્ય લાક્ષણિકતાઓ}:

\begin{itemize}
\tightlist
\item
  \textbf{Consensus}: Proof-of-work અથવા proof-of-stake વાપરે છે
\item
  \textbf{ઉદાહરણો}: Bitcoin, Ethereum mainnet
\end{itemize}

\end{solutionbox}
\begin{mnemonicbox}
``Open-Public-Decentralized'' (OPD)

\end{mnemonicbox}
\subsection*{પ્રશ્ન 2(બ) [4
ગુણ]}\label{uxaaauxab0uxab6uxaa8-2uxaac-4-uxa97uxaa3}

\textbf{Blockchain ના data structure ની આકૃતિ દોરો અને સંક્ષિપ્તમાં સમજૂતી
આપો.}

\begin{solutionbox}

\textbf{આકૃતિ: Blockchain Data Structure}

\begin{verbatim}
+{-{-}{-}{-}{-}{-}{-}{-}{-}{-}{-}{-}{-}{-}{-}{-}{-}{-}{-}+    +{-}{-}{-}{-}{-}{-}{-}{-}{-}{-}{-}{-}{-}{-}{-}{-}{-}{-}{-}+    +{-}{-}{-}{-}{-}{-}{-}{-}{-}{-}{-}{-}{-}{-}{-}{-}{-}{-}{-}+}
|     Block 1       |    |     Block 2       |    |     Block 3       |
|{-{-}{-}{-}{-}{-}{-}{-}{-}{-}{-}{-}{-}{-}{-}{-}{-}{-}{-}|    |{-}{-}{-}{-}{-}{-}{-}{-}{-}{-}{-}{-}{-}{-}{-}{-}{-}{-}{-}|    |{-}{-}{-}{-}{-}{-}{-}{-}{-}{-}{-}{-}{-}{-}{-}{-}{-}{-}{-}|}
| Previous Hash: 0  |{-{-}{-}| Previous Hash: H1 |{-}{-}{-}| Previous Hash: H2 |}
| Merkle Root: MR1  |    | Merkle Root: MR2  |    | Merkle Root: MR3  |
| Timestamp: T1     |    | Timestamp: T2     |    | Timestamp: T3     |
| Nonce: N1         |    | Nonce: N2         |    | Nonce: N3         |
| Transactions: TX1 |    | Transactions: TX2 |    | Transactions: TX3 |
+{-{-}{-}{-}{-}{-}{-}{-}{-}{-}{-}{-}{-}{-}{-}{-}{-}{-}{-}+    +{-}{-}{-}{-}{-}{-}{-}{-}{-}{-}{-}{-}{-}{-}{-}{-}{-}{-}{-}+    +{-}{-}{-}{-}{-}{-}{-}{-}{-}{-}{-}{-}{-}{-}{-}{-}{-}{-}{-}+}
\end{verbatim}

\textbf{મુખ્ય ઘટકો}:

\begin{itemize}
\tightlist
\item
  \textbf{Previous Hash}: Blocks ને એકસાથે જોડે છે અને chain બનાવે છે
\item
  \textbf{Merkle Root}: Block માં બધા transactions નો સારાંશ
\item
  \textbf{Timestamp}: Block ક્યારે બનાવ્યો તે સમય
\item
  \textbf{Nonce}: Proof-of-work માટે એકવાર વાપરાતો આંકડો
\end{itemize}

\end{solutionbox}
\begin{mnemonicbox}
``P-M-T-N'' (Previous, Merkle, Time, Nonce)

\end{mnemonicbox}
\subsection*{પ્રશ્ન 2(ક) [7
ગુણ]}\label{uxaaauxab0uxab6uxaa8-2uxa95-7-uxa97uxaa3}

\textbf{Blockchain ના core components ની સમજૂતી યોગ્ય આકૃતિ સાથે આપો.}

\begin{solutionbox}

\textbf{ટેબલ: Blockchain ના મુખ્ય ઘટકો}

{\def\LTcaptype{none} % do not increment counter
\begin{longtable}[]{@{}
  >{\raggedright\arraybackslash}p{(\linewidth - 4\tabcolsep) * \real{0.3333}}
  >{\raggedright\arraybackslash}p{(\linewidth - 4\tabcolsep) * \real{0.3333}}
  >{\raggedright\arraybackslash}p{(\linewidth - 4\tabcolsep) * \real{0.3333}}@{}}
\toprule\noalign{}
\begin{minipage}[b]{\linewidth}\raggedright
ઘટક
\end{minipage} & \begin{minipage}[b]{\linewidth}\raggedright
કાર્ય
\end{minipage} & \begin{minipage}[b]{\linewidth}\raggedright
હેતુ
\end{minipage} \\
\midrule\noalign{}
\endhead
\bottomrule\noalign{}
\endlastfoot
\textbf{Blocks} & Data containers & Transaction માહિતી સ્ટોર કરવા \\
\textbf{Hash Functions} & Digital fingerprints બનાવવા & Data integrity
સુનિશ્ચિત કરવા \\
\textbf{Merkle Trees} & Transaction summaries & કાર્યક્ષમ verification \\
\textbf{Consensus Mechanism} & Agreement protocol & નવા blocks validate
કરવા \\
\textbf{Digital Signatures} & Identity verification & Transactions
authenticate કરવા \\
\end{longtable}
}

\textbf{આકૃતિ: Merkle Tree Structure}

\begin{center}
\textbf{Mermaid Diagram (Code)}
\begin{verbatim}
{Shaded}
{Highlighting}[]
graph TD
    A[Root Hash] {-{-}{} B[Hash AB]}
    A {-{-}{} C[Hash CD]}
    B {-{-}{} D[Hash A]}
    B {-{-}{} E[Hash B]}
    C {-{-}{} F[Hash C]}
    C {-{-}{} G[Hash D]}
    D {-{-}{} H[Transaction A]}
    E {-{-}{} I[Transaction B]}
    F {-{-}{} J[Transaction C]}
    G {-{-}{} K[Transaction D]}
{Highlighting}
{Shaded}
\end{verbatim}
\end{center}

\textbf{મુખ્ય મુદ્દાઓ}:

\begin{itemize}
\tightlist
\item
  \textbf{અપરિવર્તનીયતા}: Hash functions છેડછાડ શોધી શકાય એવું બનાવે છે
\item
  \textbf{કાર્યક્ષમતા}: Merkle trees ઝડપી verification માટે પરવાનગી આપે છે
\item
  \textbf{વિકેન્દ્રીકરણ}: Consensus mechanisms કેન્દ્રીય સત્તાને દૂર કરે છે
\end{itemize}

\end{solutionbox}
\begin{mnemonicbox}
``B-H-M-C-D'' (Blocks, Hash, Merkle, Consensus,
Digital)

\end{mnemonicbox}
\subsection*{પ્રશ્ન 2(અ) OR [3
ગુણ]}\label{uxaaauxab0uxab6uxaa8-2uxa85-or-3-uxa97uxaa3}

\textbf{Permissioned blockchain ની વ્યાખ્યા કરો અને સમજાવો.}

\begin{solutionbox}

\textbf{વ્યાખ્યા}: એક blockchain જ્યાં ભાગ લેવા માટે શાસન સત્તાધિકારીની સ્પષ્ટ
પરવાનગીની જરૂર હોય છે.

\textbf{ટેબલ: Permissioned Blockchain ની લાક્ષણિકતાઓ}

{\def\LTcaptype{none} % do not increment counter
\begin{longtable}[]{@{}ll@{}}
\toprule\noalign{}
લાક્ષણિકતા & વર્ણન \\
\midrule\noalign{}
\endhead
\bottomrule\noalign{}
\endlastfoot
\textbf{પ્રતિબંધિત પહોંચ} & માત્ર અધિકૃત users ભાગ લઈ શકે છે \\
\textbf{ખાનગી Network} & નિયંત્રિત membership \\
\textbf{કેન્દ્રીકૃત નિયંત્રણ} & શાસન સંસ્થા permissions વ્યવસ્થાપિત કરે છે \\
\end{longtable}
}

\textbf{મુખ્ય લાક્ષણિકતાઓ}:

\begin{itemize}
\tightlist
\item
  \textbf{ગોપનીયતા}: સંવેદનશીલ data માટે વધારેલી ગુપ્તતા
\item
  \textbf{પ્રદર્શન}: ઓછા validators ને કારણે ઝડપી transactions
\item
  \textbf{ઉદાહરણો}: Hyperledger Fabric, R3 Corda
\end{itemize}

\end{solutionbox}
\begin{mnemonicbox}
``Restricted-Private-Centralized'' (RPC)

\end{mnemonicbox}
\subsection*{પ્રશ્ન 2(બ) OR [4
ગુણ]}\label{uxaaauxab0uxab6uxaa8-2uxaac-or-4-uxa97uxaa3}

\textbf{Wallet ના પ્રકાર blockchain ના સંદર્ભમાં સમજાવો. તેમજ ચોક્કસ જરૂરિયાત
માટે Wallet પસંદ કરતી વખતે ધ્યાનમાં લેવાના પરિબળોની ચર્ચા કરો.}

\begin{solutionbox}

\textbf{ટેબલ: Blockchain Wallets ના પ્રકારો}

{\def\LTcaptype{none} % do not increment counter
\begin{longtable}[]{@{}lll@{}}
\toprule\noalign{}
Wallet પ્રકાર & વર્ણન & સુરક્ષા સ્તર \\
\midrule\noalign{}
\endhead
\bottomrule\noalign{}
\endlastfoot
\textbf{Hot Wallets} & Internet સાથે જોડાયેલ & મધ્યમ \\
\textbf{Cold Wallets} & Offline storage & ઊંચું \\
\textbf{Hardware Wallets} & ભૌતિક devices & ખૂબ ઊંચું \\
\textbf{Paper Wallets} & છાપેલી keys & ઊંચું (જો સુરક્ષિત રીતે સંગ્રહિત) \\
\end{longtable}
}

\textbf{પસંદગીના પરિબળો}:

\begin{itemize}
\tightlist
\item
  \textbf{સુરક્ષા જરૂરિયાતો}: ઊંચું મૂલ્ય બહેતર સુરક્ષાની જરૂર પાડે છે
\item
  \textbf{ઉપયોગની આવર્તન}: નિયમિત ઉપયોગ hot wallets ને તરફેણ કરે છે
\item
  \textbf{તકનીકી કુશળતા}: શરૂઆતીઓ માટે સરળ wallets
\end{itemize}

\end{solutionbox}
\begin{mnemonicbox}
``H-C-H-P'' (Hot, Cold, Hardware, Paper)

\end{mnemonicbox}
\subsection*{પ્રશ્ન 2(ક) OR [7
ગુણ]}\label{uxaaauxab0uxab6uxaa8-2uxa95-or-7-uxa97uxaa3}

\textbf{Sidechain ને યોગ્ય આકૃતિ સાથે વિગતવાર સમજાવો.}

\begin{solutionbox}

\textbf{વ્યાખ્યા}: એક અલગ blockchain જે two-way peg વાપરીને parent
blockchain સાથે જોડાયેલ છે.

\textbf{આકૃતિ: Sidechain Architecture}

\begin{center}
\textbf{Mermaid Diagram (Code)}
\begin{verbatim}
{Shaded}
{Highlighting}[]
graph LR
    A[Main Chain] {{-}{-}{} B[Two{-}Way Peg]}
    B {{-}{-}{} C[Sidechain]}
    A {-{-}{} D[Bitcoin/Ethereum]}
    C {-{-}{} E[વિશિષ્ટ કાર્યો]}
    E {-{-}{} F[Smart Contracts]}
    E {-{-}{} G[Privacy Features]}
    E {-{-}{} H[ઝડપી Transactions]}
{Highlighting}
{Shaded}
\end{verbatim}
\end{center}

\textbf{ટેબલ: Sidechain ના ફાયદાઓ}

{\def\LTcaptype{none} % do not increment counter
\begin{longtable}[]{@{}ll@{}}
\toprule\noalign{}
ફાયદો & વર્ણન \\
\midrule\noalign{}
\endhead
\bottomrule\noalign{}
\endlastfoot
\textbf{માપનીયતા} & Main chain પરનો લોડ ઘટાડે છે \\
\textbf{પ્રયોગશીલતા} & નવી features સુરક્ષિત રીતે test કરે છે \\
\textbf{વિશિષ્ટ કાર્યો} & કસ્ટમ applications \\
\textbf{Interoperability} & વિવિધ blockchains ને જોડે છે \\
\end{longtable}
}

\textbf{મુખ્ય પદ્ધતિઓ}:

\begin{itemize}
\tightlist
\item
  \textbf{Two-Way Peg}: Chains વચ્ચે asset transfer માટે પરવાનગી આપે છે
\item
  \textbf{SPV Proofs}: Simplified payment verification
\item
  \textbf{Federated Control}: બહુવિધ parties transfers નું વ્યવસ્થાપન કરે છે
\end{itemize}

\end{solutionbox}
\begin{mnemonicbox}
``S-E-S-I'' (Scalability, Experimentation,
Specialized, Interoperability)

\end{mnemonicbox}
\subsection*{પ્રશ્ન 3(અ) [3
ગુણ]}\label{uxaaauxab0uxab6uxaa8-3uxa85-3-uxa97uxaa3}

\textbf{Blockchain network માં transaction ના સંદર્ભમાં ``Confirmation'' અને
``Finality'' ને વ્યાખ્યાયિત કરો.}

\begin{solutionbox}

\textbf{ટેબલ: Transaction States}

{\def\LTcaptype{none} % do not increment counter
\begin{longtable}[]{@{}ll@{}}
\toprule\noalign{}
શબ્દ & વ્યાખ્યા \\
\midrule\noalign{}
\endhead
\bottomrule\noalign{}
\endlastfoot
\textbf{Confirmation} & Transaction block ની ઉપર બનાવાયેલા blocks ની
સંખ્યા \\
\textbf{Finality} & જ્યાં transaction અપરિવર્તનીય બને છે તે બિંદુ \\
\end{longtable}
}

\textbf{મુખ્ય મુદ્દાઓ}:

\begin{itemize}
\tightlist
\item
  \textbf{Confirmation Count}: વધુ confirmations = વધુ સુરક્ષા
\item
  \textbf{Bitcoin Standard}: ઊંચા મૂલ્યના transactions માટે 6 confirmations
\item
  \textbf{Finality પ્રકારો}: Probabilistic (Bitcoin) vs Absolute (કેટલાક
  PoS systems)
\end{itemize}

\end{solutionbox}
\begin{mnemonicbox}
Confirmation માટે ``Count-Blocks-Security'', Finality
માટે ``Irreversible-Point''

\end{mnemonicbox}
\subsection*{પ્રશ્ન 3(બ) [4
ગુણ]}\label{uxaaauxab0uxab6uxaa8-3uxaac-4-uxa97uxaa3}

\textbf{Proof of Work અને Proof of Stake નો તફાવત આપો.}

\begin{solutionbox}

\textbf{ટેબલ: PoW vs PoS સરખામણી}

{\def\LTcaptype{none} % do not increment counter
\begin{longtable}[]{@{}lll@{}}
\toprule\noalign{}
પાસું & Proof of Work (PoW) & Proof of Stake (PoS) \\
\midrule\noalign{}
\endhead
\bottomrule\noalign{}
\endlastfoot
\textbf{સંસાધન} & Computational power & Stake ownership \\
\textbf{Energy Use} & ઊંચું & નીચું \\
\textbf{સુરક્ષા} & Hash rate dependent & Stake dependent \\
\textbf{Rewards} & Mining rewards & Staking rewards \\
\textbf{ઉદાહરણો} & Bitcoin, Ethereum (જૂનું) & Ethereum 2.0, Cardano \\
\end{longtable}
}

\textbf{મુખ્ય તફાવતો}:

\begin{itemize}
\tightlist
\item
  \textbf{પદ્ધતિ}: PoW mining વાપરે, PoS validators વાપરે
\item
  \textbf{પર્યાવરણીય અસર}: PoS વધુ પર્યાવરણ-મિત્ર છે
\item
  \textbf{પ્રવેશ અવરોધો}: PoS પ્રારંભિક stake જરૂરે, PoW hardware જરૂરે
\end{itemize}

\end{solutionbox}
\begin{mnemonicbox}
``Work-vs-Stake'' (Computational Work vs Financial
Stake)

\end{mnemonicbox}
\subsection*{પ્રશ્ન 3(ક) [7
ગુણ]}\label{uxaaauxab0uxab6uxaa8-3uxa95-7-uxa97uxaa3}

\textbf{Blockchain network ના સંદર્ભમાં 51\% attack સમજાવો.}

\begin{solutionbox}

\textbf{વ્યાખ્યા}: એક attack જ્યાં એક જ entity network ના mining power અથવા
stake ના 50\% થી વધુ પર નિયંત્રણ રાખે છે.

\textbf{આકૃતિ: 51\% Attack Scenario}

\begin{center}
\textbf{Mermaid Diagram (Code)}
\begin{verbatim}
{Shaded}
{Highlighting}[]
graph LR
    A[Network Hash Rate] {-{-}{} B[Honest Miners 49\%]}
    A {-{-}{} C[Attacker 51\%]}
    C {-{-}{} D[લાંબી Chain બનાવી શકે છે]}
    D {-{-}{} E[Double Spending]}
    D {-{-}{} F[Transaction Reversal]}
    D {-{-}{} G[Block Withholding]}
{Highlighting}
{Shaded}
\end{verbatim}
\end{center}

\textbf{ટેબલ: Attack ની ક્ષમતાઓ અને મર્યાદાઓ}

{\def\LTcaptype{none} % do not increment counter
\begin{longtable}[]{@{}ll@{}}
\toprule\noalign{}
કરી શકે છે & કરી શકતું નથી \\
\midrule\noalign{}
\endhead
\bottomrule\noalign{}
\endlastfoot
પોતાના coins double spend કરવું & બીજાના coins ચોરી કરવું \\
તાજેતરના transactions reverse કરવું & કંઈ પણ માંથી coins બનાવવું \\
ચોક્કસ transactions block કરવું & Consensus rules બદલવા \\
Blockchain fork કરવું & Private keys ને access કરવા \\
\end{longtable}
}

\textbf{રોકથામના પગલાં}:

\begin{itemize}
\tightlist
\item
  \textbf{વૈવિધ્યસભર Mining}: બહુવિધ mining pools ને પ્રોત્સાહન આપવું
\item
  \textbf{Checkpoint Systems}: સમયાંતરે finality markers
\item
  \textbf{આર્થિક પ્રોત્સાહનો}: Attacks ને અલાભકારક બનાવવા
\end{itemize}

\textbf{અસર}:

\begin{itemize}
\tightlist
\item
  \textbf{Network વિક્ષેપ}: અસ્થાયી સેવા વિક્ષેપ
\item
  \textbf{આર્થિક નુકસાન}: ઘટેલો વિશ્વાસ અને મૂલ્ય
\item
  \textbf{પુનઃપ્રાપ્તિ}: Attack સમાપ્ત થયા પછી network સામાન્યત: સ્વસ્થ થાય છે
\end{itemize}

\end{solutionbox}
\begin{mnemonicbox}
``Majority-Control-Attack'' (51\% = Majority Control
= Attack Power)

\end{mnemonicbox}
\subsection*{પ્રશ્ન 3(અ) OR [3
ગુણ]}\label{uxaaauxab0uxab6uxaa8-3uxa85-or-3-uxa97uxaa3}

\textbf{``Hard fork'' અને ``Soft fork'' ની વ્યાખ્યા આપો.}

\begin{solutionbox}

\textbf{ટેબલ: Fork પ્રકારો}

{\def\LTcaptype{none} % do not increment counter
\begin{longtable}[]{@{}lll@{}}
\toprule\noalign{}
Fork પ્રકાર & વ્યાખ્યા & સુસંગતતા \\
\midrule\noalign{}
\endhead
\bottomrule\noalign{}
\endlastfoot
\textbf{Hard Fork} & Non-backward compatible protocol change & સુસંગત
નથી \\
\textbf{Soft Fork} & Backward compatible protocol change & સુસંગત છે \\
\end{longtable}
}

\textbf{મુખ્ય લાક્ષણિકતાઓ}:

\begin{itemize}
\tightlist
\item
  \textbf{Hard Fork}: નવી blockchain branch બનાવે છે, બધા nodes ને upgrade
  જરૂરી
\item
  \textbf{Soft Fork}: Rules ને tight કરે છે, જૂના nodes હજી પણ operate કરી
  શકે છે
\end{itemize}

\textbf{ઉદાહરણો}:

\begin{itemize}
\tightlist
\item
  \textbf{Hard Fork}: Bitcoin Cash નો Bitcoin માંથી વિભાજન
\item
  \textbf{Soft Fork}: Bitcoin માં SegWit activation
\end{itemize}

\end{solutionbox}
\begin{mnemonicbox}
``Hard-Breaks-Compatibility'' vs
``Soft-Keeps-Compatibility''

\end{mnemonicbox}
\subsection*{પ્રશ્ન 3(બ) OR [4
ગુણ]}\label{uxaaauxab0uxab6uxaa8-3uxaac-or-4-uxa97uxaa3}

\textbf{વિવિધ પ્રકારના consensus mechanisms ની યાદી બનાવો અને કોઈ પણ એકને
વિગતવાર સમજાવો.}

\begin{solutionbox}

\textbf{ટેબલ: Consensus Mechanisms}

{\def\LTcaptype{none} % do not increment counter
\begin{longtable}[]{@{}lll@{}}
\toprule\noalign{}
પદ્ધતિ & વર્ણન & Energy Use \\
\midrule\noalign{}
\endhead
\bottomrule\noalign{}
\endlastfoot
\textbf{Proof of Work} & Computational puzzle solving & ઊંચું \\
\textbf{Proof of Stake} & Stake-based validation & નીચું \\
\textbf{Delegated PoS} & મત આપેલા પ્રતિનિધિઓ validate કરે છે & ખૂબ નીચું \\
\textbf{Proof of Authority} & પૂર્વ-મંજૂર validators & ન્યૂનતમ \\
\end{longtable}
}

\textbf{વિગતવાર સમજૂતી - Proof of Stake (PoS)}:

\textbf{પ્રક્રિયા}:

\begin{itemize}
\tightlist
\item
  \textbf{Validator Selection}: Stake amount અને randomization આધારે
\item
  \textbf{Block Creation}: પસંદ કરાયેલ validator નવો block propose કરે છે
\item
  \textbf{Validation}: બીજા validators block verify કરે છે અને attest કરે છે
\item
  \textbf{Rewards}: Validators fees અને નવા tokens મેળવે છે
\end{itemize}

\textbf{ફાયદાઓ}: ઓછું energy consumption, ઘટેલું centralization risk
\textbf{નુકસાનો}: ``Nothing at stake'' problem, પ્રારંભિક વિતરણ સમસ્યાઓ

\end{solutionbox}
\begin{mnemonicbox}
``Stake-Select-Validate-Reward'' (PoS Process)

\end{mnemonicbox}
\subsection*{પ્રશ્ન 3(ક) OR [7
ગુણ]}\label{uxaaauxab0uxab6uxaa8-3uxa95-or-7-uxa97uxaa3}

\textbf{Blockchain network ના સંદર્ભમાં sybil attack સમજાવો.}

\begin{solutionbox}

\textbf{વ્યાખ્યા}: એક attack જ્યાં એક જ શત્રુ network માં અપ્રમાણસર પ્રભાવ મેળવવા
માટે બહુવિધ નકલી identities બનાવે છે.

\textbf{આકૃતિ: Sybil Attack Structure}

\begin{center}
\textbf{Mermaid Diagram (Code)}
\begin{verbatim}
{Shaded}
{Highlighting}[]
graph LR
    A[Attacker] {-{-}{} B[Fake Identity 1]}
    A {-{-}{} C[Fake Identity 2]}
    A {-{-}{} D[Fake Identity 3]}
    A {-{-}{} E[Fake Identity N]}
    B {-{-}{} F[Network Influence]}
    C {-{-}{} F}
    D {-{-}{} F}
    E {-{-}{} F}
    F {-{-}{} G[Consensus Manipulation]}
{Highlighting}
{Shaded}
\end{verbatim}
\end{center}

\textbf{ટેબલ: Attack પદ્ધતિઓ અને બચાવો}

{\def\LTcaptype{none} % do not increment counter
\begin{longtable}[]{@{}
  >{\raggedright\arraybackslash}p{(\linewidth - 4\tabcolsep) * \real{0.5200}}
  >{\raggedright\arraybackslash}p{(\linewidth - 4\tabcolsep) * \real{0.2800}}
  >{\raggedright\arraybackslash}p{(\linewidth - 4\tabcolsep) * \real{0.2000}}@{}}
\toprule\noalign{}
\begin{minipage}[b]{\linewidth}\raggedright
Attack પદ્ધતિ
\end{minipage} & \begin{minipage}[b]{\linewidth}\raggedright
વર્ણન
\end{minipage} & \begin{minipage}[b]{\linewidth}\raggedright
બચાવ
\end{minipage} \\
\midrule\noalign{}
\endhead
\bottomrule\noalign{}
\endlastfoot
\textbf{Identity Flooding} & ઘણી નકલી nodes બનાવવી & Proof of
Work/Stake \\
\textbf{Routing Manipulation} & Network paths નિયંત્રિત કરવા & Reputation
systems \\
\textbf{Consensus Disruption} & Voting પ્રભાવિત કરવું & Resource
requirements \\
\end{longtable}
}

\textbf{Blockchain પર અસર}:

\begin{itemize}
\tightlist
\item
  \textbf{Network Partitioning}: Honest nodes ને અલગ પાડવા
\item
  \textbf{Double Spending}: છેતરપિંડીવાળા transactions ને સહાય કરવી
\item
  \textbf{Consensus Failure}: Network agreement અટકાવવું
\end{itemize}

\textbf{રોકથામ પદ્ધતિઓ}:

\begin{itemize}
\tightlist
\item
  \textbf{Resource Requirements}: PoW/PoS attacks ને મોંઘા બનાવે છે
\item
  \textbf{Identity Verification}: KYC/AML પ્રક્રિયાઓ
\item
  \textbf{Network Monitoring}: શંકાસ્પદ વર્તન patterns શોધવા
\item
  \textbf{Reputation Systems}: સમય સાથે node behavior track કરવું
\end{itemize}

\textbf{વાસ્તવિક ઉદાહરણો}:

\begin{itemize}
\tightlist
\item
  \textbf{P2P Networks}: BitTorrent, Gnutella vulnerabilities
\item
  \textbf{Social Networks}: Fake account creation
\item
  \textbf{Blockchain}: Permissionless networks માટે સંભવિત ખતરો
\end{itemize}

\end{solutionbox}
\begin{mnemonicbox}
``Single-Multiple-Influence'' (Single Attacker,
Multiple Identities, Network Influence)

\end{mnemonicbox}
\subsection*{પ્રશ્ન 4(અ) [3
ગુણ]}\label{uxaaauxab0uxab6uxaa8-4uxa85-3-uxa97uxaa3}

\textbf{``Merkle Tree'' અને ``Hyperledger'' ને વ્યાખ્યાયિત કરો.}

\begin{solutionbox}

\textbf{ટેબલ: મુખ્ય વ્યાખ્યાઓ}

{\def\LTcaptype{none} % do not increment counter
\begin{longtable}[]{@{}
  >{\raggedright\arraybackslash}p{(\linewidth - 2\tabcolsep) * \real{0.4167}}
  >{\raggedright\arraybackslash}p{(\linewidth - 2\tabcolsep) * \real{0.5833}}@{}}
\toprule\noalign{}
\begin{minipage}[b]{\linewidth}\raggedright
શબ્દ
\end{minipage} & \begin{minipage}[b]{\linewidth}\raggedright
વ્યાખ્યા
\end{minipage} \\
\midrule\noalign{}
\endhead
\bottomrule\noalign{}
\endlastfoot
\textbf{Merkle Tree} & Hashes નો binary tree જે બધા transactions ને
કાર્યક્ષમ રીતે સારાંશિત કરે છે \\
\textbf{Hyperledger} & Linux Foundation દ્વારા hosted open-source
blockchain platform \\
\end{longtable}
}

\textbf{મુખ્ય લાક્ષણિકતાઓ}:

\begin{itemize}
\tightlist
\item
  \textbf{Merkle Tree}: સંપૂર્ણ blockchain download કર્યા વગર કાર્યક્ષમ
  verification સક્ષમ કરે છે
\item
  \textbf{Hyperledger}: Enterprise-focused, modular architecture, બહુવિધ
  frameworks
\end{itemize}

\end{solutionbox}
\begin{mnemonicbox}
Merkle માટે ``Tree-Hash-Efficient'', Hyperledger માટે
``Enterprise-Modular-Linux''

\end{mnemonicbox}
\subsection*{પ્રશ્ન 4(બ) [4
ગુણ]}\label{uxaaauxab0uxab6uxaa8-4uxaac-4-uxa97uxaa3}

\textbf{Classic Byzantine generals problem ને વિગતવાર સમજાવો.}

\begin{solutionbox}

\textbf{પરિસ્થિતિ}: બહુવિધ generals એ શહેર પર હુમલાનું સંકલન કરવું જોઈએ, પરંતુ
કેટલાક દગાબાજ હોઈ શકે છે.

\textbf{ટેબલ: Problem ના ઘટકો}

{\def\LTcaptype{none} % do not increment counter
\begin{longtable}[]{@{}ll@{}}
\toprule\noalign{}
ઘટક & વર્ણન \\
\midrule\noalign{}
\endhead
\bottomrule\noalign{}
\endlastfoot
\textbf{Generals} & Network nodes/સહભાગીઓ \\
\textbf{Messages} & Transactions/communications \\
\textbf{Traitors} & દુર્ભાવનાપૂર્ણ/ખરાબ nodes \\
\textbf{Consensus} & કાર્ય પર સમજૂતી \\
\end{longtable}
}

\textbf{સોલ્યુશન જરૂરિયાતો}:

\begin{itemize}
\tightlist
\item
  \textbf{Agreement}: બધા પ્રામાણિક generals એ જ કાર્યનો નિર્ણય લે
\item
  \textbf{Validity}: જો બધા પ્રામાણિક generals હુમલો કરવા માગે તો તેઓએ હુમલો
  કરવો જોઈએ
\item
  \textbf{Termination}: મર્યાદિત સમયમાં નિર્ણય લેવાયો હોવો જોઈએ
\end{itemize}

\textbf{Blockchain સુસંગતતા}: દુર્ભાવનાપૂર્ણ nodes છતાં network agreement
સુનિશ્ચિત કરે છે

\end{solutionbox}
\begin{mnemonicbox}
``Generals-Messages-Traitors-Consensus'' (GMTC)

\end{mnemonicbox}
\subsection*{પ્રશ્ન 4(ક) [7
ગુણ]}\label{uxaaauxab0uxab6uxaa8-4uxa95-7-uxa97uxaa3}

\textbf{Merkle tree creation ની પ્રક્રિયા યોગ્ય ઉદાહરણ અને આકૃતિ સાથે સમજાવો.}

\begin{solutionbox}

\textbf{પ્રક્રિયાના પગલાં}:

\begin{enumerate}
\tightlist
\item
  દરેક transaction ને વ્યક્તિગત રીતે hash કરો
\item
  Hashes ને જોડો અને pairs ને hash કરો
\item
  એક જ root hash બાકી રહેવા સુધી ચાલુ રાખો
\end{enumerate}

\textbf{ઉદાહરણ: 4 Transactions}

\begin{verbatim}
                    Root Hash
                   /           {}
              Hash(AB)         Hash(CD)
             /        {       /        }
        Hash(A)    Hash(B) Hash(C)  Hash(D)
           |          |       |        |
         Tx A       Tx B    Tx C     Tx D
\end{verbatim}

\textbf{ટેબલ: Merkle Tree ના ફાયદાઓ}

{\def\LTcaptype{none} % do not increment counter
\begin{longtable}[]{@{}ll@{}}
\toprule\noalign{}
ફાયદો & વર્ણન \\
\midrule\noalign{}
\endhead
\bottomrule\noalign{}
\endlastfoot
\textbf{કાર્યક્ષમતા} & સંપૂર્ણ data વગર transactions verify કરો \\
\textbf{સુરક્ષા} & કોઈપણ બદલાવ root hash ને અસર કરે છે \\
\textbf{માપનીયતા} & Log(n) verification complexity \\
\end{longtable}
}

\textbf{Verification પ્રક્રિયા}:

\begin{itemize}
\tightlist
\item
  Tx A verify કરવા માટે: Hash(B), Hash(CD), અને Root Hash જરૂરી
\item
  Path verification: Hash(A) + Hash(B) = Hash(AB)
\item
  Hash(AB) + Hash(CD) = Root Hash
\end{itemize}

\textbf{Applications}:

\begin{itemize}
\tightlist
\item
  \textbf{Bitcoin}: Block headers માં Merkle root છે
\item
  \textbf{SPV Clients}: Light wallets Merkle proofs વાપરે છે
\item
  \textbf{Git}: Version control system સમાન structure વાપરે છે
\end{itemize}

\end{solutionbox}
\begin{mnemonicbox}
``Hash-Pair-Repeat-Root'' (Merkle Tree Creation
Process)

\end{mnemonicbox}
\subsection*{પ્રશ્ન 4(અ) OR [3
ગુણ]}\label{uxaaauxab0uxab6uxaa8-4uxa85-or-3-uxa97uxaa3}

\textbf{Hyperledger projects ના વિવિધ પ્રકારની યાદી બનાવો.}

\begin{solutionbox}

\textbf{ટેબલ: Hyperledger Projects}

{\def\LTcaptype{none} % do not increment counter
\begin{longtable}[]{@{}lll@{}}
\toprule\noalign{}
Project & પ્રકાર & હેતુ \\
\midrule\noalign{}
\endhead
\bottomrule\noalign{}
\endlastfoot
\textbf{Fabric} & Framework & Permissioned blockchain platform \\
\textbf{Sawtooth} & Framework & Modular blockchain suite \\
\textbf{Iroha} & Framework & Mobile/web માટે સરળ blockchain \\
\textbf{Burrow} & Framework & Ethereum Virtual Machine \\
\textbf{Caliper} & Tool & Blockchain performance benchmark \\
\textbf{Composer} & Tool & Business network development \\
\end{longtable}
}

\textbf{શ્રેણીઓ}:

\begin{itemize}
\tightlist
\item
  \textbf{Frameworks}: મુખ્ય blockchain platforms
\item
  \textbf{Tools}: Development અને testing utilities
\end{itemize}

\end{solutionbox}
\begin{mnemonicbox}
``F-S-I-B-C-C'' (Fabric, Sawtooth, Iroha, Burrow,
Caliper, Composer)

\end{mnemonicbox}
\subsection*{પ્રશ્ન 4(બ) OR [4
ગુણ]}\label{uxaaauxab0uxab6uxaa8-4uxaac-or-4-uxa97uxaa3}

\textbf{Practical Byzantine Fault Tolerance algorithm વિગતવાર સમજાવો.}

\begin{solutionbox}

\textbf{વ્યાખ્યા}: Consensus algorithm જે 1/3 સુધી nodes ખરાબ અથવા
દુર્ભાવનાપૂર્ણ હોય તો પણ યોગ્ય રીતે કામ કરે છે.

\textbf{ટેબલ: PBFT Phases}

{\def\LTcaptype{none} % do not increment counter
\begin{longtable}[]{@{}
  >{\raggedright\arraybackslash}p{(\linewidth - 4\tabcolsep) * \real{0.3684}}
  >{\raggedright\arraybackslash}p{(\linewidth - 4\tabcolsep) * \real{0.3684}}
  >{\raggedright\arraybackslash}p{(\linewidth - 4\tabcolsep) * \real{0.2632}}@{}}
\toprule\noalign{}
\begin{minipage}[b]{\linewidth}\raggedright
Phase
\end{minipage} & \begin{minipage}[b]{\linewidth}\raggedright
વર્ણન
\end{minipage} & \begin{minipage}[b]{\linewidth}\raggedright
હેતુ
\end{minipage} \\
\midrule\noalign{}
\endhead
\bottomrule\noalign{}
\endlastfoot
\textbf{Pre-prepare} & Primary request broadcast કરે છે & Consensus શરૂ
કરવું \\
\textbf{Prepare} & Nodes validate કરે છે અને broadcast કરે છે & Proposal
verify કરવું \\
\textbf{Commit} & Nodes નિર્ણય પર commit કરે છે & Agreement finalize કરવું \\
\end{longtable}
}

\textbf{Algorithm પગલાં}:

\begin{enumerate}
\tightlist
\item
  Client primary replica ને request મોકલે છે
\item
  Primary pre-prepare message broadcast કરે છે
\item
  Valid હોય તો backups prepare messages મોકલે છે
\item
  2f+1 prepares મળ્યા પછી commit મોકલે છે
\item
  2f+1 commits મળ્યા પછી execute કરે છે
\end{enumerate}

\textbf{મુખ્ય ગુણધર્મો}:

\begin{itemize}
\tightlist
\item
  \textbf{Safety}: ક્યારેય અસંગત પરિણામો ઉત્પન્ન કરતું નથી
\item
  \textbf{Liveness}: આખરે પરિણામો ઉત્પન્ન કરે છે
\item
  \textbf{Fault Tolerance}: f \textless{} n/3 ખરાબ nodes સાથે કામ કરે છે
\end{itemize}

\end{solutionbox}
\begin{mnemonicbox}
``Pre-Prepare-Commit'' (PBFT ના 3 Phases)

\end{mnemonicbox}
\subsection*{પ્રશ્ન 4(ક) OR [7
ગુણ]}\label{uxaaauxab0uxab6uxaa8-4uxa95-or-7-uxa97uxaa3}

\textbf{``Eventual consistency is evident in the context of bitcoin.''
આપેલ વાક્યને પુરવાર કરો.}

\begin{solutionbox}

\textbf{વ્યાખ્યા}: Eventual consistency નો અર્થ છે કે system સમય સાથે
consistent બનશે, ભલે તે અસ્થાયી રૂપે inconsistent હોય.

\textbf{Bitcoin Implementation}:

\textbf{ટેબલ: Bitcoin Consistency Mechanisms}

{\def\LTcaptype{none} % do not increment counter
\begin{longtable}[]{@{}
  >{\raggedright\arraybackslash}p{(\linewidth - 4\tabcolsep) * \real{0.3684}}
  >{\raggedright\arraybackslash}p{(\linewidth - 4\tabcolsep) * \real{0.3684}}
  >{\raggedright\arraybackslash}p{(\linewidth - 4\tabcolsep) * \real{0.2632}}@{}}
\toprule\noalign{}
\begin{minipage}[b]{\linewidth}\raggedright
પદ્ધતિ
\end{minipage} & \begin{minipage}[b]{\linewidth}\raggedright
વર્ણન
\end{minipage} & \begin{minipage}[b]{\linewidth}\raggedright
હેતુ
\end{minipage} \\
\midrule\noalign{}
\endhead
\bottomrule\noalign{}
\endlastfoot
\textbf{Chain Reorganization} & લાંબી chain સાથે ટૂંકી chain replace કરવી &
Consensus જાળવવું \\
\textbf{Confirmation Delays} & બહુવિધ blocks માટે રાહ જોવી & વિશ્વસનીયતા
વધારવી \\
\textbf{Fork Resolution} & સૌથી લાંબી chain જીતે છે & સંઘર્ષો ઉકેલવા \\
\end{longtable}
}

\textbf{Eventual Consistency દર્શાવતા દૃશ્યો}:

\begin{enumerate}
\tightlist
\item
  \textbf{અસ્થાયી Forks}: જ્યારે બે miners એકસાથે blocks શોધે છે
\item
  \textbf{Network Partitions}: અલગ પડેલા nodes જુદા જુદા views હોઈ શકે છે
\item
  \textbf{Double Spending Attempts}: વિવિધ blocks માં સંઘર્ષ કરતા
  transactions
\end{enumerate}

\textbf{Resolution પ્રક્રિયા}:

\begin{itemize}
\tightlist
\item
  \textbf{Mining ચાલુ રહે છે}: Miners તેમની પસંદીદા chain પર build કરે છે
\item
  \textbf{Longest Chain Rule}: Network સૌથી વધુ work વાળી chain અપનાવે છે
\item
  \textbf{Automatic Convergence}: બધા nodes આખરે સહમત થાય છે
\end{itemize}

\textbf{આકૃતિ: Fork Resolution}

\begin{center}
\textbf{Mermaid Diagram (Code)}
\begin{verbatim}
{Shaded}
{Highlighting}[]
graph LR
    A[Block N] {-{-}{} B[Block N+1a]}
    A {-{-}{} C[Block N+1b]}
    B {-{-}{} D[Block N+2a]}
    C {-{-}{} E[મૃત્યુ પામે {-} ટૂંકી Chain]}
    D {-{-}{} F[Main Chain બને છે]}
{Highlighting}
{Shaded}
\end{verbatim}
\end{center}

\textbf{વાજબીપણાના મુદ્દાઓ}:

\begin{itemize}
\tightlist
\item
  \textbf{Probabilistic Finality}: લાંબો confirmation time = વધારે
  વિશ્વસનીયતા
\item
  \textbf{તાત્કાલિક Consistency નથી}: નવા transactions તરત final નથી
\item
  \textbf{Convergence Guarantee}: Network આખરે એક જ chain પર સહમત થશે
\item
  \textbf{Time-based Resolution}: સમય સાથે consistency સુધરે છે
\end{itemize}

\textbf{વ્યવહારિક અસરો}:

\begin{itemize}
\tightlist
\item
  \textbf{Merchant Waiting}: Payment accept કરતા પહેલાં confirmations માટે
  રાહ જોવી
\item
  \textbf{Exchange Policies}: વિવિધ રકમો માટે વિવિધ confirmation
  requirements
\item
  \textbf{Risk Management}: Transaction value આધારે speed vs security
  સંતુલિત કરવું
\end{itemize}

\end{solutionbox}
\begin{mnemonicbox}
``Time-Brings-Consistency'' (Eventual Consistency =
Time + Convergence)

\end{mnemonicbox}
\subsection*{પ્રશ્ન 5(અ) [3
ગુણ]}\label{uxaaauxab0uxab6uxaa8-5uxa85-3-uxa97uxaa3}

\textbf{ERC 20 ના ફાયદાઓ સમજાવો.}

\begin{solutionbox}

\textbf{ટેબલ: ERC-20 Token ના ફાયદાઓ}

{\def\LTcaptype{none} % do not increment counter
\begin{longtable}[]{@{}ll@{}}
\toprule\noalign{}
ફાયદો & વર્ણન \\
\midrule\noalign{}
\endhead
\bottomrule\noalign{}
\endlastfoot
\textbf{માનકીકરણ} & બધા tokens માટે સામાન્ય interface \\
\textbf{Interoperability} & બધા Ethereum wallets/exchanges સાથે કામ કરે
છે \\
\textbf{તરલતા} & સરળ trading અને exchange \\
\end{longtable}
}

\textbf{મુખ્ય ફાયદાઓ}:

\begin{itemize}
\tightlist
\item
  \textbf{Developer Friendly}: સરળ implementation standard
\item
  \textbf{Market Adoption}: platforms પર વ્યાપક રીતે supported
\item
  \textbf{Smart Contract Integration}: સરળ DeFi integration
\end{itemize}

\end{solutionbox}
\begin{mnemonicbox}
``Standard-Interoperable-Liquid'' (SIL)

\end{mnemonicbox}
\subsection*{પ્રશ્ન 5(બ) [4
ગુણ]}\label{uxaaauxab0uxab6uxaa8-5uxaac-4-uxa97uxaa3}

\textbf{Smart-contract ની working mechanism વિગતવાર સમજાવો.}

\begin{solutionbox}

\textbf{ટેબલ: Smart Contract Workflow}

{\def\LTcaptype{none} % do not increment counter
\begin{longtable}[]{@{}ll@{}}
\toprule\noalign{}
પગલું & વર્ણન \\
\midrule\noalign{}
\endhead
\bottomrule\noalign{}
\endlastfoot
\textbf{Code Deployment} & Contract blockchain પર upload કરવું \\
\textbf{Trigger Conditions} & પૂર્વ-નિર્ધારિત conditions નું monitoring \\
\textbf{Automatic Execution} & Conditions મળ્યે contract execute થાય છે \\
\textbf{State Update} & Blockchain state modify થાય છે \\
\end{longtable}
}

\textbf{કાર્ય પ્રક્રિયા}:

\begin{enumerate}
\tightlist
\item
  \textbf{Development}: Solidity/Vyper માં contract લખવું
\item
  \textbf{Compilation}: Bytecode માં convert કરવું
\item
  \textbf{Deployment}: Blockchain network પર upload કરવું
\item
  \textbf{Execution}: Transactions અથવા events દ્વારા trigger થવું
\end{enumerate}

\end{solutionbox}
\begin{mnemonicbox}
``Deploy-Trigger-Execute-Update'' (DTEU)

\end{mnemonicbox}
\subsection*{પ્રશ્ન 5(ક) [7
ગુણ]}\label{uxaaauxab0uxab6uxaa8-5uxa95-7-uxa97uxaa3}

\textbf{Smart-contract શું છે? Smart-contract ની વિશેષતા અને ઉપયોગીતા વિગતવાર
સમજાવો.}

\begin{solutionbox}

\textbf{વ્યાખ્યા}: Self-executing contracts જેના terms સીધા code માં લખેલા
હોય છે, blockchain પર ચાલે છે.

\textbf{ટેબલ: Smart Contract વિશેષતાઓ}

{\def\LTcaptype{none} % do not increment counter
\begin{longtable}[]{@{}lll@{}}
\toprule\noalign{}
વિશેષતા & વર્ણન & ફાયદો \\
\midrule\noalign{}
\endhead
\bottomrule\noalign{}
\endlastfoot
\textbf{સ્વાયત્ત} & મધ્યસ્થીઓ વગર execute થાય છે & ખર્ચમાં ઘટાડો \\
\textbf{પારદર્શી} & Code blockchain પર દૃશ્યમાન છે & વિશ્વાસ નિર્માણ \\
\textbf{અપરિવર્તનશીલ} & Deploy થયા પછી બદલાઈ શકતું નથી & સુરક્ષા \\
\textbf{નિર્ધારિત} & સમાન input સમાન output આપે છે & અનુમાનિતતા \\
\end{longtable}
}

\textbf{આકૃતિ: Smart Contract Architecture}

\begin{center}
\textbf{Mermaid Diagram (Code)}
\begin{verbatim}
{Shaded}
{Highlighting}[]
graph LR
    A[Smart Contract] {-{-}{} B[Trigger Conditions]}
    B {-{-}{} C[Automatic Execution]}
    C {-{-}{} D[State Changes]}
    D {-{-}{} E[Event Emissions]}
    A {-{-}{} F[External Calls]}
    F {-{-}{} G[Other Contracts]}
{Highlighting}
{Shaded}
\end{verbatim}
\end{center}

\textbf{ઉપયોગિતાઓ}:

\textbf{ટેબલ: Smart Contract Applications}

{\def\LTcaptype{none} % do not increment counter
\begin{longtable}[]{@{}lll@{}}
\toprule\noalign{}
ક્ષેત્ર & ઉપયોગ & ઉદાહરણ \\
\midrule\noalign{}
\endhead
\bottomrule\noalign{}
\endlastfoot
\textbf{Finance} & Automated lending & DeFi protocols \\
\textbf{Insurance} & Claim processing & Flight delay insurance \\
\textbf{Supply Chain} & Product tracking & Food provenance \\
\textbf{Real Estate} & Property transfers & Automated escrow \\
\textbf{Gaming} & Digital assets & NFT marketplaces \\
\end{longtable}
}

\textbf{ફાયદાઓ}:

\begin{itemize}
\tightlist
\item
  \textbf{કાર્યક્ષમતા}: ઘટેલો processing time અને costs
\item
  \textbf{વિશ્વાસ}: Trusted third parties ની જરૂર નથી
\item
  \textbf{ચોકસાઈ}: માનવીય ભૂલો દૂર કરે છે
\item
  \textbf{વૈશ્વિક પહોંચ}: 24/7 વિશ્વવ્યાપી ઉપલબ્ધ
\end{itemize}

\textbf{મર્યાદાઓ}:

\begin{itemize}
\tightlist
\item
  \textbf{અપરિવર્તનશીલતા}: Deployment પછી bugs ઠીક કરવા મુશ્કેલ
\item
  \textbf{Oracle Problem}: બાહ્ય data sources ની જરૂર
\item
  \textbf{Gas Costs}: Execution costs ઊંચા હોઈ શકે છે
\item
  \textbf{જટિલતા}: તકનીકી નિપુણતા જરૂરી
\end{itemize}

\textbf{Development વિચારણા}:

\begin{itemize}
\tightlist
\item
  \textbf{Security Audits}: Deployment પહેલાં આવશ્યક
\item
  \textbf{Testing}: Testnets પર વ્યાપક testing
\item
  \textbf{Upgradability}: Updates માટે design patterns
\item
  \textbf{Gas Optimization}: Execution costs ઘટાડવા
\end{itemize}

\end{solutionbox}
\begin{mnemonicbox}
વિશેષતાઓ માટે
``Auto-Transparent-Immutable-Deterministic'' (ATID)

\end{mnemonicbox}
\subsection*{પ્રશ્ન 5(અ) OR [3
ગુણ]}\label{uxaaauxab0uxab6uxaa8-5uxa85-or-3-uxa97uxaa3}

\textbf{ERC 20 ના ગેરફાયદાઓ સમજાવો.}

\begin{solutionbox}

\textbf{ટેબલ: ERC-20 Token ના ગેરફાયદાઓ}

{\def\LTcaptype{none} % do not increment counter
\begin{longtable}[]{@{}ll@{}}
\toprule\noalign{}
ગેરફાયદો & વર્ણન \\
\midrule\noalign{}
\endhead
\bottomrule\noalign{}
\endlastfoot
\textbf{મર્યાદિત કાર્યક્ષમતા} & માત્ર બુનિયાદી token operations \\
\textbf{Built-in Security નથી} & સામાન્ય attacks માટે vulnerable \\
\textbf{Gas Dependency} & Transactions માટે ETH જરૂરી \\
\end{longtable}
}

\textbf{મુખ્ય સમસ્યાઓ}:

\begin{itemize}
\tightlist
\item
  \textbf{Transfer મર્યાદાઓ}: જટિલ transfers handle કરી શકતું નથી
\item
  \textbf{Approval Risks}: Double spending vulnerabilities
\item
  \textbf{Network Congestion}: Peak times દરમિયાન ઊંચી fees
\end{itemize}

\end{solutionbox}
\begin{mnemonicbox}
``Limited-Vulnerable-Dependent'' (LVD)

\end{mnemonicbox}
\subsection*{પ્રશ્ન 5(બ) OR [4
ગુણ]}\label{uxaaauxab0uxab6uxaa8-5uxaac-or-4-uxa97uxaa3}

\textbf{Decentralized Autonomous Organization (DAO) ના Launching માટેના
steps વર્ણવો.}

\begin{solutionbox}

\textbf{ટેબલ: DAO Launch Steps}

{\def\LTcaptype{none} % do not increment counter
\begin{longtable}[]{@{}ll@{}}
\toprule\noalign{}
પગલું & વર્ણન \\
\midrule\noalign{}
\endhead
\bottomrule\noalign{}
\endlastfoot
\textbf{Concept Design} & હેતુ અને governance rules વ્યાખ્યાયિત કરવા \\
\textbf{Smart Contract Development} & Governance mechanisms code કરવા \\
\textbf{Token Distribution} & Voting rights વહેંચવા \\
\textbf{Community Building} & સભ્યો અને contributors આકર્ષવા \\
\end{longtable}
}

\textbf{વિગતવાર પ્રક્રિયા}:

\begin{enumerate}
\tightlist
\item
  \textbf{Whitepaper Creation}: Vision અને tokenomics નું document
\item
  \textbf{Technical Implementation}: Governance contracts deploy કરવા
\item
  \textbf{Initial Funding}: Token sales દ્વારા capital raise કરવું
\item
  \textbf{Operations Launch}: વિકેન્દ્રીકૃત operations શરૂ કરવા
\end{enumerate}

\end{solutionbox}
\begin{mnemonicbox}
``Design-Develop-Distribute-Deploy'' (4D Launch)

\end{mnemonicbox}
\subsection*{પ્રશ્ન 5(ક) OR [7
ગુણ]}\label{uxaaauxab0uxab6uxaa8-5uxa95-or-7-uxa97uxaa3}

\textbf{Decentralized Autonomous Organization (DAO) શું છે? તેના ફાયદાઓ અને
ગેરફાયદાઓ વિગતવાર સમજાવો.}

\begin{solutionbox}

\textbf{વ્યાખ્યા}: Blockchain-based organization જે પરંપરાગત management ને
બદલે smart contracts અને token holders દ્વારા સંચાલિત થાય છે.

\textbf{ટેબલ: DAO Structure}

{\def\LTcaptype{none} % do not increment counter
\begin{longtable}[]{@{}
  >{\raggedright\arraybackslash}p{(\linewidth - 4\tabcolsep) * \real{0.2941}}
  >{\raggedright\arraybackslash}p{(\linewidth - 4\tabcolsep) * \real{0.4118}}
  >{\raggedright\arraybackslash}p{(\linewidth - 4\tabcolsep) * \real{0.2941}}@{}}
\toprule\noalign{}
\begin{minipage}[b]{\linewidth}\raggedright
ઘટક
\end{minipage} & \begin{minipage}[b]{\linewidth}\raggedright
વર્ણન
\end{minipage} & \begin{minipage}[b]{\linewidth}\raggedright
કાર્ય
\end{minipage} \\
\midrule\noalign{}
\endhead
\bottomrule\noalign{}
\endlastfoot
\textbf{Smart Contracts} & Code માં governance rules & Automated decision
execution \\
\textbf{Tokens} & Voting rights અને ownership & લોકશાહી ભાગીદારી \\
\textbf{Proposals} & સૂચિત બદલાવો અથવા ક્રિયાઓ & Community-driven
initiatives \\
\textbf{Treasury} & સહેજ funds & Resource allocation \\
\end{longtable}
}

\textbf{આકૃતિ: DAO Governance Flow}

\begin{center}
\textbf{Mermaid Diagram (Code)}
\begin{verbatim}
{Shaded}
{Highlighting}[]
graph TD
    A[Token Holders] {-{-}{} B[Submit Proposals]}
    B {-{-}{} C[Community Discussion]}
    C {-{-}{} D[Voting Period]}
    D {-{-}{} E[Passed હોય તો Execution]}
    E {-{-}{} F[Smart Contract Updates]}
    F {-{-}{} G[Treasury Actions]}
{Highlighting}
{Shaded}
\end{verbatim}
\end{center}

\textbf{ફાયદાઓ}:

\textbf{ટેબલ: DAO ના ફાયદાઓ}

{\def\LTcaptype{none} % do not increment counter
\begin{longtable}[]{@{}lll@{}}
\toprule\noalign{}
ફાયદો & વર્ણન & અસર \\
\midrule\noalign{}
\endhead
\bottomrule\noalign{}
\endlastfoot
\textbf{વિકેન્દ્રીકરણ} & નિયંત્રણનું એક જ બિંદુ નથી & ભ્રષ્ટાચાર જોખમ ઘટાડે છે \\
\textbf{પારદર્શિતા} & બધા નિર્ણયો blockchain પર & વધારેલી જવાબદારી \\
\textbf{વૈશ્વિક ભાગીદારી} & કોઈપણ join કરી શકે છે & વિવિધ દ્રષ્ટિકોણો \\
\textbf{કાર્યક્ષમતા} & Automated execution & ઝડપી નિર્ણય implementation \\
\textbf{લોકશાહી Governance} & Token-based voting & વાજબી પ્રતિનિધિત્વ \\
\end{longtable}
}

\textbf{ગેરફાયદાઓ}:

\textbf{ટેબલ: DAO Challenges}

{\def\LTcaptype{none} % do not increment counter
\begin{longtable}[]{@{}
  >{\raggedright\arraybackslash}p{(\linewidth - 4\tabcolsep) * \real{0.4286}}
  >{\raggedright\arraybackslash}p{(\linewidth - 4\tabcolsep) * \real{0.3333}}
  >{\raggedright\arraybackslash}p{(\linewidth - 4\tabcolsep) * \real{0.2381}}@{}}
\toprule\noalign{}
\begin{minipage}[b]{\linewidth}\raggedright
ગેરફાયદો
\end{minipage} & \begin{minipage}[b]{\linewidth}\raggedright
વર્ણન
\end{minipage} & \begin{minipage}[b]{\linewidth}\raggedright
જોખમ
\end{minipage} \\
\midrule\noalign{}
\endhead
\bottomrule\noalign{}
\endlastfoot
\textbf{તકનીકી જટિલતા} & Smart contract bugs & System failures \\
\textbf{કાનૂની અનિશ્ચિતતા} & અસ્પષ્ટ regulatory status & Compliance
issues \\
\textbf{સંકલન સમસ્યાઓ} & મુશ્કેલ નિર્ણય લેવું & ધીમી પ્રગતિ \\
\textbf{Token Concentration} & શ્રીમંત holders votes control કરે છે &
Centralization જોખમ \\
\textbf{સુરક્ષા Vulnerabilities} & Code exploits શક્ય છે & નાણાકીય નુકસાન \\
\end{longtable}
}

\textbf{DAO ના પ્રકારો}:

\begin{itemize}
\tightlist
\item
  \textbf{Investment DAOs}: સામૂહિક investment નિર્ણયો
\item
  \textbf{Protocol DAOs}: Blockchain protocol governance
\item
  \textbf{Social DAOs}: Community-driven organizations
\item
  \textbf{Collector DAOs}: NFT અને art collecting
\end{itemize}

\textbf{સફળતાના પરિબળો}:

\begin{itemize}
\tightlist
\item
  \textbf{સ્પષ્ટ હેતુ}: સુ-વ્યાખ્યાયિત mission અને goals
\item
  \textbf{મજબૂત Governance}: અસરકારક voting mechanisms
\item
  \textbf{Community Engagement}: સક્રિય સભ્ય ભાગીદારી
\item
  \textbf{તકનીકી સુરક્ષા}: Audited smart contracts
\item
  \textbf{કાનૂની Compliance}: લાગુ પડતી જગ્યાએ regulatory compliance
\end{itemize}

\textbf{નોંધપાત્ર ઉદાહરણો}:

\begin{itemize}
\tightlist
\item
  \textbf{MakerDAO}: Decentralized finance protocol
\item
  \textbf{Uniswap}: Decentralized exchange governance
\item
  \textbf{Compound}: Money market protocol
\end{itemize}

\textbf{ભવિષ્યની દ્રષ્ટિ}:

\begin{itemize}
\tightlist
\item
  \textbf{Regulatory Clarity}: વિકસિત થતા કાનૂની frameworks
\item
  \textbf{તકનીકી સુધારાઓ}: બહેતર governance tools
\item
  \textbf{Mainstream Adoption}: વધતી corporate interest
\item
  \textbf{Integration}: Hybrid traditional-DAO models
\end{itemize}

\end{solutionbox}
\begin{mnemonicbox}
``Decentralized-Autonomous-Organization'' (DAO =
Democratic Automated Ownership)

\end{mnemonicbox}

\end{document}
