\documentclass{article}

% content/resources/templates/preamble.tex
\usepackage[margin=0.6in]{geometry}
\author{Milav Dabgar}
\usepackage{amsmath,amssymb,amsthm}
\usepackage{booktabs}
\usepackage{multirow}
\usepackage{xcolor}
\usepackage{tcolorbox}
\tcbuselibrary{breakable,skins}
\usepackage[colorlinks=true,linkcolor=blue]{hyperref}
\usepackage{titlesec}
\usepackage{enumitem}
\usepackage{tikz}
\usepackage{pgfplots}
\usepackage{circuitikz}
\usepackage[version=4]{mhchem}
\usepackage{longtable}
\usepackage{array}
\usepackage{float}
\usepackage{caption}
\usepackage{listings}

\lstset{
  basicstyle=\small\ttfamily,
  breaklines=true,
  breakatwhitespace=false,
  postbreak=\mbox{\textcolor{red}{$\hookrightarrow$}\space},
  float=false,
  numbers=left,
  numberstyle=\tiny\color{gray},
  numbersep=10pt,
  xleftmargin=2em,
  keywordstyle=\color{blue},
  commentstyle=\color{green!60!black},
  stringstyle=\color{purple},
  backgroundcolor=\color{gray!5},
  showstringspaces=false,
  tabsize=2,
  captionpos=b,
  keepspaces=true,
  columns=flexible
}

\pgfplotsset{compat=1.18}
\usetikzlibrary{shapes,arrows,positioning,calc,patterns,decorations.pathmorphing,decorations.markings,arrows.meta}

% Color scheme
\definecolor{headcolor}{RGB}{0,102,204}
\definecolor{keycolor}{RGB}{220,20,60}
\definecolor{solutioncolor}{RGB}{34,139,34}
\definecolor{mnemoniccolor}{RGB}{148,0,211}
\definecolor{codecolor}{RGB}{0,0,100}

% Spacing
\setlength{\parskip}{3pt}
\setlist[itemize]{nosep}
\setlist[enumerate]{nosep}

% Title formatting
\titleformat{\section}{\Large\bfseries\color{headcolor}}{\thesection}{1em}{}
\titleformat{\subsection}{\large\bfseries\color{headcolor}}{\thesubsection}{1em}{}

% Pandoc tightlist compatibility
\providecommand{\tightlist}{%
  \setlength{\itemsep}{0pt}\setlength{\parskip}{0pt}}

% Pandoc longtable compatibility
\newcounter{none}
\def\thenone{}


% content/resources/templates/gujarati-boxes.tex
\usepackage{fontspec}
\usepackage{polyglossia}

% Set Gujarati as main language (document is primarily in Gujarati)
% Note: gloss-gujarati.ldf doesn't exist in polyglossia, but it will use hyphenation patterns
\setdefaultlanguage{gujarati}
\setotherlanguage{english}

% Configure Gujarati font properly
% Use Language=Default to prevent polyglossia from trying to add language-specific features
% that don't exist for Gujarati, which causes "empty feature" warnings
\newfontfamily\gujaratifont[Script=Gujarati,AutoFakeBold=2.5,AutoFakeSlant=0.3]{Noto Sans Gujarati}
\setmainfont[Script=Gujarati,AutoFakeBold=2.5,AutoFakeSlant=0.3]{Noto Sans Gujarati}
% Use Noto Sans Gujarati for monospace to support Gujarati in text
\setmonofont[Scale=0.9]{Noto Sans Gujarati}

% Configure English to use the same font
\newfontfamily\englishfont[Script=Gujarati,AutoFakeBold=2.5,AutoFakeSlant=0.3]{Noto Sans Gujarati}

% Translations for polyglossia
\gappto\captionsgujarati{
  \renewcommand{\tablename}{કોષ્ટક}
  \renewcommand{\figurename}{આકૃતિ}
}

% Helper for TikZ nodes to ensure Gujarati font
\newcommand{\gu}[1]{{\gujaratifont #1}}

% Custom environments
\newtcolorbox{solutionbox}{
    breakable,
    enhanced,
    colback=solutioncolor!5!white,
    colframe=solutioncolor!75!black,
    fonttitle=\bfseries,
    title=જવાબ
}

\newtcolorbox{solutionboxnobreak}{
 colback=solutioncolor!5!white,
 colframe=solutioncolor!75!black,
 fonttitle=\bfseries,
 title=જવાબ
}

\newtcolorbox{keyformula}{
 breakable,
 enhanced,
 colback=keycolor!5!white,
 colframe=keycolor!75!black,
 fonttitle=\bfseries,
 title=રાસાયણિક સમીકરણ/સૂત્ર
}

\newtcolorbox{mnemonicbox}{
 breakable,
 enhanced,
 colback=mnemoniccolor!5!white,
 colframe=mnemoniccolor!75!black,
 fonttitle=\bfseries,
 title=મેમરી ટ્રીક
}


% Custom commands for GTU solutions
% This file defines semantic commands for consistent formatting

% Question command with automatic formatting
\newcommand{\question}[2]{%
  \section*{Question #1}%
  \textbf{#2}%
}

% OR question variant
\newcommand{\questionor}[2]{%
  \section*{Question #1 OR}%
  \textbf{#2}%
}

% Proper table environment with caption
\newenvironment{answertable}[1]{%
  \begin{table}[htbp]
  \centering
  \caption{#1}
}{%
  \end{table}
}

% Proper figure environment for diagrams
\newenvironment{answerdiagram}[1]{%
  \begin{figure}[htbp]
  \centering
  \caption{#1}
}{%
  \end{figure}
}

% Semantic markup for key terms
\newcommand{\keyword}[1]{\textbf{#1}}
\newcommand{\code}[1]{\texttt{#1}}
\newcommand{\classname}[1]{\texttt{#1}}
\newcommand{\methodname}[1]{\texttt{#1}}

% Proper quotation marks
\newcommand{\mnemonic}[1]{``#1''}


\title{Foundation of Blockchain (4361603) - Summer 2025 Gujarati Solution}
\date{May 14, 2025}

\begin{document}
\maketitle

\questionmarks{1(અ)}{3}{બ્લોકચેનમાં Private key અને Public key નો તફાવત આપો.}

\begin{tabulary}{\linewidth}{L L L}
    \toprule
    \textbf{બાબત} & \textbf{Private Key} & \textbf{Public Key} \\
    \midrule
    \textbf{હેતુ} & Transaction sign કરવા માટે & Verification માટે ઉપયોગ \\
    \textbf{શેરિંગ} & ગુપ્ત રાખવી જોઈએ & બધાને આપી શકાય \\
    \textbf{કામ} & Data decrypt કરે, signature બનાવે & Data encrypt કરે, signature verify કરે \\
    \textbf{માલિકી} & ફક્ત માલિક જ જાણે & બધા access કરી શકે \\
    \bottomrule
\end{tabulary}

\begin{itemize}
    \item \keyword{Private Key}: ગુપ્ત mathematical code જે ownership સાબિત કરે
    \item \keyword{Public Key}: ખુલ્લું address જેથી બીજા transaction મોકલી શકે
    \item \keyword{સુરક્ષા}: Private key ગુમાવવી = પૈસા હંમેશ માટે ગુમાવવા
\end{itemize}

\begin{mnemonicbox}
Private છે Personal, Public છે Posted
\end{mnemonicbox}

\questionmarks{1(બ)}{4}{Distributed Ledger ને વિગતવાર સમજાવો.}

\textbf{Distributed Ledger} એ database છે જે ઘણી જગ્યાએ અને ઘણા લોકોમાં વહેંચાયેલું હોય છે.

\begin{tabulary}{\linewidth}{L L}
    \toprule
    \textbf{લક્ષણ} & \textbf{વર્ણન} \\
    \midrule
    \textbf{Decentralized} & કોઈ એક control point નથી \\
    \textbf{Synchronized} & બધી copies updated રહે છે \\
    \textbf{Transparent} & બધા participants જોઈ શકે છે \\
    \textbf{Immutable} & સહેલાઈથી બદલાતું નથી \\
    \bottomrule
\end{tabulary}

\begin{figure}[H]
    \centering
    \begin{tikzpicture}[node distance=2cm, auto]
        \node [gtu block] (DL) {Distributed Ledger};
        \node [gtu block, above=of DL] (P2) {Participant 2};
        \node [gtu block, left=of P2] (P1) {Participant 1};
        \node [gtu block, right=of P2] (P3) {Participant 3};
        \node [gtu block, below=of DL] (SC2) {Synchronized Copy 2};
        \node [gtu block, left=of SC2] (SC1) {Synchronized Copy 1};
        \node [gtu block, right=of SC2] (SC3) {Synchronized Copy 3};

        \draw [gtu arrow] (P1) -- (DL);
        \draw [gtu arrow] (P2) -- (DL);
        \draw [gtu arrow] (P3) -- (DL);
        \draw [gtu arrow] (DL) -- (SC1);
        \draw [gtu arrow] (DL) -- (SC2);
        \draw [gtu arrow] (DL) -- (SC3);
    \end{tikzpicture}
    \caption{Distributed Ledger System}
\end{figure}

\begin{itemize}
    \item \keyword{ફાયદા}: Intermediaries નાબૂદ કરે, trust વધારે, fraud ઓછું
    \item \keyword{કામ}: બધા participants પાસે records ની identical copies હોય
\end{itemize}

\begin{mnemonicbox}
Distributed = વિભાજિત પણ સમાન
\end{mnemonicbox}

\questionmarks{1(ક)}{7}{Blockchain વ્યાખ્યાયિત કરો. Blockchain ની એપ્લિકેશનો અને મર્યાદાઓનાં વર્ણન કરો.}

\keyword{Blockchain વ્યાખ્યા}: Transaction records ધરાવતા blocks નો chain જે cryptography વાપરીને જોડાયેલા હોય.

\textbf{એપ્લિકેશન કોષ્ટક:}

\begin{tabulary}{\linewidth}{L L L}
    \toprule
    \textbf{ક્ષેત્ર} & \textbf{એપ્લિકેશન} & \textbf{ફાયદો} \\
    \midrule
    \textbf{Finance} & Cryptocurrency, payments & ઝડપી, સસ્તી transfers \\
    \textbf{Healthcare} & Patient records & સુરક્ષિત, accessible data \\
    \textbf{Supply Chain} & Product tracking & પારદર્શિતા, authenticity \\
    \textbf{Real Estate} & Property records & Fraud prevention \\
    \textbf{Voting} & Digital elections & પારદર્શી, tamper-proof \\
    \bottomrule
\end{tabulary}

\textbf{મર્યાદાઓ કોષ્ટક:}

\begin{tabulary}{\linewidth}{L L}
    \toprule
    \textbf{મર્યાદા} & \textbf{અસર} \\
    \midrule
    \textbf{Scalability} & ધીમી transaction processing \\
    \textbf{Energy Usage} & વધુ electricity વપરાશ \\
    \textbf{Complexity} & Users માટે સમજવું મુશ્કેલ \\
    \textbf{Regulation} & કાયદાકીય અસ્પષ્ટતા \\
    \textbf{Storage} & વધતો data size ની સમસ્યા \\
    \bottomrule
\end{tabulary}

\begin{figure}[H]
    \centering
    \begin{tikzpicture}[node distance=1.5cm, auto]
        \node [gtu block] (B1) {Block 1};
        \node [gtu block, right=of B1] (B2) {Block 2};
        \node [gtu block, right=of B2] (B3) {Block 3};
        \node [gtu block, right=of B3] (B4) {Block 4};

        \node [gtu state, below=0.5cm of B1] (H1) {Hash};
        \node [gtu state, below=0.5cm of B2] (H2) {Hash};
        \node [gtu state, below=0.5cm of B3] (H3) {Hash};
        \node [gtu state, below=0.5cm of B4] (H4) {Hash};

        \draw [gtu arrow] (B1) -- (B2);
        \draw [gtu arrow] (B2) -- (B3);
        \draw [gtu arrow] (B3) -- (B4);
        
        \draw [gtu arrow] (H1) -- (B1);
        \draw [gtu arrow] (H2) -- (B2);
        \draw [gtu arrow] (H3) -- (B3);
        \draw [gtu arrow] (H4) -- (B4);
    \end{tikzpicture}
    \caption{Blockchain Architecture}
\end{figure}

\begin{itemize}
    \item \keyword{સુરક્ષા}: Cryptographic linking થી tampering મુશ્કેલ
    \item \keyword{પારદર્શિતા}: બધા transactions network participants ને દેખાય
\end{itemize}

\begin{mnemonicbox}
Blocks Chained = Blockchain, Apps ઘણી = Limits ઘણી
\end{mnemonicbox}

\orquestionmarks{1(ક)}{7}{ટૂંકી નોંધ લખો: બ્લોકચેનમાં CAP Theorem}

\keyword{CAP Theorem} કહે છે કે distributed systems એ 3 properties માંથી માત્ર 2 જ simultaneously guarantee કરી શકે.

\textbf{CAP Components કોષ્ટક:}

\begin{tabulary}{\linewidth}{L L L}
    \toprule
    \textbf{Property} & \textbf{વર્ણન} & \textbf{ઉદાહરણ} \\
    \midrule
    \textbf{Consistency} & બધા nodes પાસે same data & બર્યાને જગ્યાએ same balance દેખાય \\
    \textbf{Availability} & System હંમેશા response આપે & Network કદી down ન જાય \\
    \textbf{Partition Tolerance} & Network failures છતાં કામ કરે & Nodes disconnect થયા છતાં function કરે \\
    \bottomrule
\end{tabulary}

\begin{figure}[H]
    \centering
    \begin{tikzpicture}[node distance=1.5cm, auto]
        \node [gtu block] (CAP) {CAP Theorem};
        \node [gtu state, below left=of CAP] (C) {Consistency};
        \node [gtu state, below=of CAP] (A) {Availability};
        \node [gtu state, below right=of CAP] (P) {Partition Tolerance};

        \node [gtu block, below=2cm of C] (BTC) {Bitcoin};
        \node [gtu block, below=2cm of P] (PVT) {Private Blockchain};

        \draw [gtu arrow] (CAP) -- (C);
        \draw [gtu arrow] (CAP) -- (A);
        \draw [gtu arrow] (CAP) -- (P);

        \draw [gtu arrow] (BTC) -- (C);
        \draw [gtu arrow] (BTC) -- (P);
        \draw [gtu arrow] (PVT) -- (C);
        \draw [gtu arrow] (PVT) -- (A);
    \end{tikzpicture}
    \caption{CAP Theorem and Blockchain Trade-offs}
\end{figure}

\textbf{વાસ્તવિક ઉપયોગ:}

\begin{tabulary}{\linewidth}{L L L}
    \toprule
    \textbf{Blockchain Type} & \textbf{પસંદ કરે} & \textbf{ત્યાગ કરે} \\
    \midrule
    \textbf{Bitcoin} & Consistency + Partition & Availability \\
    \textbf{Ethereum} & Consistency + Partition & Availability \\
    \textbf{Private Networks} & Consistency + Availability & Partition Tolerance \\
    \bottomrule
\end{tabulary}

\begin{itemize}
    \item \keyword{અસર}: Blockchain designers એ કયા property sacrifice કરવી તે choose કરવું પડે
    \item \keyword{Trade-off}: Distributed networks માં perfect systems અશક્ય
\end{itemize}

\begin{mnemonicbox}
કમ્પ્લીટ સિસ્ટમ શક્ય નથી - 3 માંથી 2 જ પસંદ કરો
\end{mnemonicbox}

\questionmarks{2(અ)}{3}{બ્લોકચેનનાં Data Structure સમજાવો.}

\textbf{Blockchain Data Structure} transaction data ધરાવતા linked blocks ધાયેલું હોય છે.

\begin{tabulary}{\linewidth}{L L}
    \toprule
    \textbf{Component} & \textbf{હેતુ} \\
    \midrule
    \textbf{Block Header} & Metadata રાખે છે \\
    \textbf{Previous Hash} & Previous block સાથે link કરે \\
    \textbf{Merkle Root} & બધા transactions નો summary \\
    \textbf{Timestamp} & Block કયારે બન્યો તેની માહિતી \\
    \textbf{Transactions} & વાસ્તવિક data/transfers \\
    \bottomrule
\end{tabulary}

\begin{figure}[H]
    \centering
    \begin{tikzpicture}[node distance=0cm, outer sep=0pt]
        \node [draw, rectangle, minimum width=4cm, minimum height=1cm, fill=gray!10] (Head) {\textbf{Block Header}};
        \node [draw, rectangle, minimum width=4cm, minimum height=0.7cm, below=0pt of Head] (Prev) {Previous Hash};
        \node [draw, rectangle, minimum width=4cm, minimum height=0.7cm, below=0pt of Prev] (Merkle) {Merkle Root};
        \node [draw, rectangle, minimum width=4cm, minimum height=0.7cm, below=0pt of Merkle] (Time) {Timestamp};
        \node [draw, rectangle, minimum width=4cm, minimum height=0.7cm, below=0pt of Time] (Nonce) {Nonce};
        \node [draw, rectangle, minimum width=4cm, minimum height=1.5cm, below=0pt of Nonce, fill=green!10, align=center] (Tx) {\textbf{Transactions}\\ {[TX1, TX2, TX3]}};
        
        \node [draw, rectangle, minimum width=4.2cm, minimum height=6cm, below=0pt of Head, yshift=2.9cm] (Block) {};
    \end{tikzpicture}
    \caption{Structure of a Block}
\end{figure}

\begin{itemize}
    \item \keyword{Linking}: દરેક block previous block ને hash વાપરીને point કરે
    \item \keyword{Integrity}: એક block બદલાવવાથી આખી chain ટૂટી જાય
\end{itemize}

\begin{mnemonicbox}
Header હોય છે, Transactions વાત કરે છે
\end{mnemonicbox}

\questionmarks{2(બ)}{4}{Decentralization ના ફાયદા શું છે?}

\textbf{Decentralization ફાયદા:}

\begin{tabulary}{\linewidth}{L L}
    \toprule
    \textbf{ફાયદો} & \textbf{સમજૂતી} \\
    \midrule
    \textbf{No Single Point of Failure} & એક node fail થયા છતાં network ચાલુ રહે \\
    \textbf{Censorship Resistance} & કોઈ authority transactions block કરી શકે નહિ \\
    \textbf{Transparency} & બધા participants સમાન માહિતી જુએ છે \\
    \textbf{Reduced Costs} & Intermediary fees નાબૂદ થાય \\
    \textbf{Trust} & Central authority પર trust કરવાની જરૂર નથી \\
    \bottomrule
\end{tabulary}

\begin{figure}[H]
    \centering
    \begin{tikzpicture}[node distance=1.5cm]
        % Centralized
        \node [gtu state] (C) {Central};
        \node [gtu block, below left=of C] (U1) {User 1};
        \node [gtu block, below=of C] (U2) {User 2};
        \node [gtu block, below right=of C] (U3) {User 3};
        
        \draw [gtu arrow] (C) -- (U1);
        \draw [gtu arrow] (C) -- (U2);
        \draw [gtu arrow] (C) -- (U3);
        
        \node [below=0.5cm of U2] {Centralized};

        % Decentralized - shifted right
        \node [gtu block, right=5cm of U1] (D1) {User 1};
        \node [gtu block, right=of D1] (D2) {User 2};
        \node [gtu block, right=of D2] (D3) {User 3};
        
        \draw [gtu arrow, <->] (D1) -- (D2);
        \draw [gtu arrow, <->] (D2) -- (D3);
        \draw [gtu arrow, <->, bend left=45] (D1) to (D3);
        
        \node [below=0.5cm of D2] {Decentralized};
    \end{tikzpicture}
    \caption{Centralized vs. Decentralized Networks}
\end{figure}

\begin{itemize}
    \item \keyword{સુરક્ષા}: Multiple copies થી data loss અટકે
    \item \keyword{લોકશાહી}: બધા participants ને સમાન અધિકાર
    \item \keyword{મજબૂતાઈ}: Individual failures સામે system ટકે
\end{itemize}

\begin{mnemonicbox}
વિકેન્દ્રિત = ટકાઉ, લોકશાહી, પ્રત્યક્ષ
\end{mnemonicbox}

\questionmarks{2(ક)}{7}{Public બ્લોકચેન અને Private બ્લોકચેન વચ્ચે તફાવત કરો.}

\keyword{વ્યાપક સરખામણી}:

\begin{tabulary}{\linewidth}{L L L}
    \toprule
    \textbf{બાબત} & \textbf{Public Blockchain} & \textbf{Private Blockchain} \\
    \midrule
    \textbf{Access} & બધા માટે ખુલ્લું & ખાસ users માટે મર્યાદિત \\
    \textbf{Permission} & Permission ની જરૂર નથી & Permission આવશ્યક \\
    \textbf{Control} & Decentralized & Centralized control \\
    \textbf{Speed} & ધીમું (consensus જરૂરી) & ઝડપી (ઓછા validators) \\
    \textbf{Security} & ઊંચી (ઘણા validators) & મધ્યમ (ઓછા validators) \\
    \textbf{Cost} & Transaction fees જરૂરી & ઓછી operational costs \\
    \textbf{Transparency} & સંપૂર્ણ પારદર્શિતા & મર્યાદિત પારદર્શિતા \\
    \textbf{ઉદાહરણ} & Bitcoin, Ethereum & Hyperledger, R3 Corda \\
    \bottomrule
\end{tabulary}

\begin{figure}[H]
    \centering
    \begin{tikzpicture}[node distance=1.5cm]
        % Public
        \node [gtu block] (PubNet) {Global Network};
        \node [gtu state, above left=of PubNet] (Any1) {Anyone};
        \node [gtu state, above=of PubNet] (Any2) {Anyone};
        \node [gtu state, above right=of PubNet] (Any3) {Anyone};
        
        \draw [gtu arrow] (Any1) -- (PubNet);
        \draw [gtu arrow] (Any2) -- (PubNet);
        \draw [gtu arrow] (Any3) -- (PubNet);
        
        \node [below=0.2cm of PubNet] {Public Blockchain};

        % Private
        \node [gtu block, right=6cm of PubNet] (PrivNet) {Private Network};
        \node [gtu state, above left=of PrivNet, fill=blue!10] (Auth1) {User 1};
        \node [gtu state, above=of PrivNet, fill=blue!10] (Auth2) {User 2};
        \node [gtu state, above right=of PrivNet, fill=blue!10] (Auth3) {User 3};
        
        \draw [gtu arrow] (Auth1) -- (PrivNet);
        \draw [gtu arrow] (Auth2) -- (PrivNet);
        \draw [gtu arrow] (Auth3) -- (PrivNet);
        
        \node [below=0.2cm of PrivNet] {Private Blockchain};
    \end{tikzpicture}
    \caption{Public vs Private Architecture}
\end{figure}

\begin{itemize}
    \item \keyword{Trade-offs}: Public વધુ security આપે, Private વધુ control આપે
    \item \keyword{પસંદગી}: Transparency vs. privacy ની જરૂરિયાત પર નિર્ભર
\end{itemize}

\begin{mnemonicbox}
Public = લોકોનું, Private = મંજૂરીવાળું
\end{mnemonicbox}

\orquestionmarks{2(અ)}{3}{યોગ્ય આકૃતિ સાથે બ્લોક ચેઇનના Core Components નાં વર્ણન કરો.}

\textbf{મુખ્ય Components:}

\begin{tabulary}{\linewidth}{L L}
    \toprule
    \textbf{Component} & \textbf{કામ} \\
    \midrule
    \textbf{Blocks} & Transaction data store કરે \\
    \textbf{Hash Functions} & Unique fingerprints બનાવે \\
    \textbf{Digital Signatures} & Transaction authenticity verify કરે \\
    \textbf{Consensus Mechanism} & Valid transactions પર સંમતિ કરે \\
    \textbf{Peer-to-Peer Network} & બધા participants ને connect કરે \\
    \bottomrule
\end{tabulary}

\begin{figure}[H]
    \centering
    \begin{tikzpicture}[node distance=1cm, auto]
        \node [gtu block] (P2P) {P2P Network};
        \node [gtu block, right=of P2P] (Con) {Consensus};
        \node [gtu block, right=of Con] (Create) {Block Creation};
        \node [gtu block, below=of Create] (Hash) {Hash Functions};
        \node [gtu block, left=of Hash] (Sign) {Signatures};
        \node [gtu block, left=of Sign] (Valid) {Validation};
        
        \draw [gtu arrow] (P2P) -- (Con);
        \draw [gtu arrow] (Con) -- (Create);
        \draw [gtu arrow] (Create) -- (Hash);
        \draw [gtu arrow] (Hash) -- (Sign);
        \draw [gtu arrow] (Sign) -- (Valid);
        \draw [gtu arrow] (Valid) -- (P2P);
    \end{tikzpicture}
    \caption{Blockchain Core Components Interaction}
\end{figure}

\begin{itemize}
    \item \keyword{એકીકરણ}: બધા components મળીને security માટે કામ કરે
    \item \keyword{હેતુ}: દરેક component ખાસ blockchain function serve કરે
\end{itemize}

\begin{mnemonicbox}
Blocks બનાવે, Hash પકડે, Signatures સુરક્ષિત કરે
\end{mnemonicbox}

\orquestionmarks{2(બ)}{4}{Permissioned blockchain ને વ્યાખ્યાયિત કરો અને વિગતવાર સમજાવો.}

\keyword{Permissioned Blockchain વ્યાખ્યા}: એવી blockchain જેમાં participation માટે network administrators પાસેથી સ્પષ્ટ permission જરૂરી હોય.

\begin{tabulary}{\linewidth}{L L}
    \toprule
    \textbf{લક્ષણ} & \textbf{વર્ણન} \\
    \midrule
    \textbf{Access Control} & ફક્ત approved users જ join કરી શકે \\
    \textbf{Validation Rights} & પસંદગીના nodes જ transactions validate કરે \\
    \textbf{Governance} & Central authority network manage કરે \\
    \textbf{Privacy} & Transaction details private હોઈ શકે \\
    \bottomrule
\end{tabulary}

\begin{figure}[H]
    \centering
    \begin{tikzpicture}[node distance=1.5cm]
        \node [gtu block] (Admin) {Network Admin};
        \node [gtu state, below=of Admin] (Full) {Full Access};
        \node [gtu state, left=of Full] (RW) {Read/Write};
        \node [gtu state, right=of Full] (Read) {Read Only};
        \node [gtu state, right=2.5cm of Full] (No) {No Access};
        
        \draw [gtu arrow] (Admin) -- (Full);
        \draw [gtu arrow] (Admin) -- (RW);
        \draw [gtu arrow] (Admin) -- (Read);
        \draw [gtu arrow] (Admin) -- (No);
    \end{tikzpicture}
    \caption{Permission Levels in Permissioned Blockchain}
\end{figure}

\begin{itemize}
    \item \keyword{ફાયદા}: બહેતર privacy, regulatory compliance, ઝડપી processing
    \item \keyword{ગેરફાયદા}: ઓછું decentralized, administrators પર trust આવશ્યક
\end{itemize}

\begin{mnemonicbox}
Permission = Participation માટે મંજૂરી
\end{mnemonicbox}

\orquestionmarks{2(ક)}{7}{Sidechain ને સંક્ષિપ્તમાં સમજાવો.}

\keyword{Sidechain વ્યાખ્યા}: Main blockchain સાથે connected અલગ blockchain જે chains વચ્ચે asset transfer કરવાની સુવિધા આપે.

\begin{figure}[H]
    \centering
    \begin{tikzpicture}[node distance=2cm]
        \node [gtu block, minimum width=3cm] (Main) {Main Chain};
        \node [gtu block, above right=of Main] (Side1) {Sidechain 1};
        \node [gtu block, right=of Main] (Side2) {Sidechain 2};
        \node [gtu block, below right=of Main] (Side3) {Sidechain 3};
        
        \draw [gtu arrow, <->] (Main) -- (Side1);
        \draw [gtu arrow, <->] (Main) -- (Side2);
        \draw [gtu arrow, <->] (Main) -- (Side3);
        
        \node [right=0.2cm of Side1] {Specific Purpose 1};
        \node [right=0.2cm of Side2] {Specific Purpose 2};
        \node [right=0.2cm of Side3] {Specific Purpose 3};
    \end{tikzpicture}
    \caption{Sidechain Architecture}
\end{figure}

\textbf{ફાયદા અને લક્ષણો:}

\begin{tabulary}{\linewidth}{L L}
    \toprule
    \textbf{બાબત} & \textbf{ફાયદો} \\
    \midrule
    \textbf{Scalability} & Main chain નો load ઘટાડે \\
    \textbf{Experimentation} & નવા features સુરક્ષિત રીતે test કરે \\
    \textbf{Specialization} & ખાસ use cases માટે optimized \\
    \textbf{Interoperability} & અલગ અલગ blockchains ને connect કરે \\
    \bottomrule
\end{tabulary}

\textbf{Transfer Process:}

\begin{enumerate}
    \item \textbf{Lock}: Main chain પર assets lock કરાય
    \item \textbf{Proof}: Cryptographic proof generate કરાય
    \item \textbf{Release}: Sidechain પર equivalent assets release કરાય
    \item \textbf{Use}: Sidechain પર assets ઉપયોગ કરાય
    \item \textbf{Return}: Assets પાછા લાવવા માટે reverse process
\end{enumerate}

\textbf{વાસ્તવિક ઉદાહરણો:}

\begin{tabulary}{\linewidth}{L L}
    \toprule
    \textbf{Sidechain} & \textbf{હેતુ} \\
    \midrule
    \textbf{Lightning Network} & ઝડપી Bitcoin payments \\
    \textbf{Plasma} & Ethereum scaling \\
    \textbf{Liquid} & Bitcoin trading \\
    \bottomrule
\end{tabulary}

\begin{itemize}
    \item \keyword{સુરક્ષા}: Secure main chain સાથેનું connection જાળવે
    \item \keyword{લવચિકતા}: દરેક sidechain ના અલગ rules હોઈ શકે
\end{itemize}

\begin{mnemonicbox}
Side સહાય કરે, Main જાળવે
\end{mnemonicbox}

\questionmarks{3(અ)}{3}{Consensus Mechanism ને વ્યાખ્યાયિત કરો અને કોઈપણ એકને વિગતવાર સમજાવો.}

\keyword{Consensus Mechanism વ્યાખ્યા}: એક protocol જે ખાતરી કરે કે બધા network participants blockchain ની current state પર સંમત હોય.

\textbf{Proof of Work (PoW) સમજૂતી:}

\begin{tabulary}{\linewidth}{L L}
    \toprule
    \textbf{Component} & \textbf{કામ} \\
    \midrule
    \textbf{Mining} & જટિલ mathematical puzzles solve કરવું \\
    \textbf{Competition} & Miners વચ્ચે પહેલા solve કરવાની સ્પર્ધા \\
    \textbf{Verification} & Network solution verify કરે \\
    \textbf{Reward} & Winner ને cryptocurrency reward મળે \\
    \bottomrule
\end{tabulary}

\begin{figure}[H]
    \centering
    \begin{tikzpicture}[node distance=1.5cm, auto]
        \node [gtu state] (New) {New Transaction};
        \node [gtu block, right=of New] (Col) {Miners Collect};
        \node [gtu block, right=of Col] (Create) {Create Block};
        \node [gtu block, below=of Create] (Solve) {Solve Puzzle};
        \node [gtu state, left=of Solve] (Win) {First Wins};
        \node [gtu block, left=of Win] (Add) {Block Added};
        
        \draw [gtu arrow] (New) -- (Col);
        \draw [gtu arrow] (Col) -- (Create);
        \draw [gtu arrow] (Create) -- (Solve);
        \draw [gtu arrow] (Solve) -- (Win);
        \draw [gtu arrow] (Win) -- (Add);
    \end{tikzpicture}
    \caption{Proof of Work Process}
\end{figure}

\begin{itemize}
    \item \keyword{સુરક્ષા}: Computational work થી tampering મોંઘું બને
    \item \keyword{ઉદાહરણ}: Bitcoin Proof of Work consensus વાપરે
\end{itemize}

\begin{mnemonicbox}
Consensus = સામાન્ય બુદ્ધિ, Work = જીત
\end{mnemonicbox}

\questionmarks{3(બ)}{4}{બ્લોકચેનમાં Forking શા માટે જરૂરી છે? બ્લોકચેનમાં વિવિધ પ્રકારના Forks ની યાદી બનાવો.}

\textbf{Forking કેમ જરૂરી:}

\begin{tabulary}{\linewidth}{L L}
    \toprule
    \textbf{કારણ} & \textbf{હેતુ} \\
    \midrule
    \textbf{Upgrades} & Blockchain માં નવા features add કરવા \\
    \textbf{Bug Fixes} & Security vulnerabilities સુધારવા \\
    \textbf{Rule Changes} & Consensus rules modify કરવા \\
    \textbf{Community Disagreement} & Consensus ન મળે ત્યારે split કરવા \\
    \bottomrule
\end{tabulary}

\textbf{Forks ના પ્રકારો:}

\begin{tabulary}{\linewidth}{L L L}
    \toprule
    \textbf{Fork Type} & \textbf{વર્ણન} & \textbf{Compatibility} \\
    \midrule
    \textbf{Soft Fork} & Rules tight કરે & Backward compatible \\
    \textbf{Hard Fork} & Rules સંપૂર્ણ બદલે & Backward compatible નથી \\
    \textbf{Accidental Fork} & અસ્થાયી split & આપોઆપ resolve થાય \\
    \textbf{Contentious Fork} & Community disagreement & કાયમી split \\
    \bottomrule
\end{tabulary}

\begin{figure}[H]
    \centering
    \begin{tikzpicture}[node distance=1.5cm]
        \node [gtu block] (Org) {Original Chain};
        \node [gtu block, right=of Org] (Fork) {Fork Point};
        
        \node [gtu block, above right=of Fork] (Soft) {Soft Fork};
        \node [gtu block, right=of Soft] (Valid) {Old Nodes Valid};
        
        \node [gtu block, below right=of Fork] (Hard) {Hard Fork};
        \node [gtu block, right=of Hard] (Invalid) {Old Nodes Invalid};
        
        \draw [gtu arrow] (Org) -- (Fork);
        \draw [gtu arrow] (Fork) -- (Soft);
        \draw [gtu arrow] (Soft) -- (Valid);
        \draw [gtu arrow] (Fork) -- (Hard);
        \draw [gtu arrow] (Hard) -- (Invalid);
    \end{tikzpicture}
    \caption{Soft vs Hard Fork}
\end{figure}

\begin{itemize}
    \item \keyword{અસર}: Forks થી નવી cryptocurrencies બની શકે
    \item \keyword{ઉદાહરણો}: Bitcoin Cash (hard fork), Ethereum updates (soft forks)
\end{itemize}

\begin{mnemonicbox}
Fork = ભવિષ્યના વિકલ્પો, Rules જાળવાય
\end{mnemonicbox}

\questionmarks{3(ક)}{7}{Bitcoin Mining શું છે? Bitcoin Mining નાં કામકાજ, મુશ્કેલી અને ફાયદાઓ વિશે વિગતવાર જણાવો.}

\keyword{Bitcoin Mining વ્યાખ્યા}: Computational puzzles solve કરીને Bitcoin blockchain માં નવા transactions add કરવાની પ્રક્રિયા.

\textbf{Mining Process:}

\begin{enumerate}
    \item \textbf{Collection}: Pending transactions ભેગા કરવા (Mempool માંથી)
    \item \textbf{Block Creation}: નવો block બનાવવો અને transactions સામેલ કરવા
    \item \textbf{Puzzle Solving}: સાચો nonce શોધવો (Trial and error)
    \item \textbf{Verification}: Network solution check કરે અને block validate કરે
    \item \textbf{Addition}: Chain માં block add કરવો (કાયમી record)
    \item \textbf{Reward}: Miner ને Bitcoin મળે (હાલમાં 6.25 BTC)
\end{enumerate}

\begin{figure}[H]
    \centering
    \begin{tikzpicture}[node distance=1.5cm, auto]
        \node [gtu state] (Trans) {Transactions};
        \node [gtu block, right=of Trans] (Header) {Create Header};
        \node [gtu block, right=of Header] (Nonce) {Guess Nonce};
        \node [gtu block, below=of Nonce] (Hash) {Calc Hash};
        \node [draw, diamond, aspect=2, below=of Hash] (Check) {Hash < Target?};
        
        \node [gtu block, left=of Check] (Broadcast) {Broadcast};
        \node [gtu block, left=of Broadcast] (Add) {Add Block};
        
        \draw [gtu arrow] (Trans) -- (Header);
        \draw [gtu arrow] (Header) -- (Nonce);
        \draw [gtu arrow] (Nonce) -- (Hash);
        \draw [gtu arrow] (Hash) -- (Check);
        \draw [gtu arrow] (Check) -- node[above] {Yes} (Broadcast);
        \draw [gtu arrow] (Broadcast) -- (Add);
        \draw [gtu arrow] (Check.east) -- ++(1,0) |- node[near start, right] {No} (Nonce);
    \end{tikzpicture}
    \caption{Bitcoin Mining Workflow}
\end{figure}

\textbf{Difficulty Adjustment:}

\begin{tabulary}{\linewidth}{L L}
    \toprule
    \textbf{બાબત} & \textbf{પદ્ધતિ} \\
    \midrule
    \textbf{Target Time} & દરેક block માટે 10 મિનિટ \\
    \textbf{Adjustment Period} & દરેક 2016 blocks (~2 અઠવાડિયા) \\
    \textbf{Auto-Regulation} & Blocks ઝડપી આવે તો વધારે \\
    \textbf{હેતુ} & Consistent block time જાળવવું \\
    \bottomrule
\end{tabulary}

\textbf{Mining ના ફાયદા:}

\begin{itemize}
    \item \textbf{Financial Reward}: Successful mining માટે Bitcoin કમાવવું
    \item \textbf{Network Security}: વધુ miners = વધુ secure network
    \item \textbf{Transaction Processing}: Bitcoin transfers શક્ય બનાવવું
    \item \textbf{Decentralization}: Central authority ની જરૂર નથી
\end{itemize}

\begin{mnemonicbox}
Mining = પૈસા, Math, Maintenance
\end{mnemonicbox}

\orquestionmarks{3(અ)}{3}{Soft fork અને Hard fork નો તફાવત આપો.}

\textbf{Fork સરખામણી:}

\begin{tabulary}{\linewidth}{L L L}
    \toprule
    \textbf{બાબત} & \textbf{Soft Fork} & \textbf{Hard Fork} \\
    \midrule
    \textbf{Compatibility} & Backward compatible & Backward compatible નથી \\
    \textbf{Rules} & Rules વધુ સખત બનાવે & Rules સંપૂર્ણ બદલે \\
    \textbf{Node Updates} & જૂના nodes માટે વૈકલ્પિક & બધા nodes માટે ફરજિયાત \\
    \textbf{Chain Split} & કાયમી split નથી & કાયમી split કરી શકે \\
    \textbf{Consensus} & Implement કરવું સરળ & Majority agreement જરૂરી \\
    \textbf{ઉદાહરણો} & SegWit (Bitcoin) & Bitcoin Cash, Ethereum Classic \\
    \bottomrule
\end{tabulary}

\begin{figure}[H]
    \centering
    \begin{tikzpicture}[node distance=1.5cm]
        \node [gtu block] (Org) {Original Chain};
        \node [gtu block, right=of Org] (Fork) {Fork Point};
        
        \node [gtu block, above right=of Fork] (Soft) {Soft Fork};
        \node [gtu block, right=of Soft] (Valid) {જૂના nodes હજુ valid};
        \node [gtu block, right=of Valid] (Single) {એક જ chain ચાલુ};
        
        \node [gtu block, below right=of Fork] (Hard) {Hard Fork};
        \node [gtu block, right=of Hard] (Invalid) {જૂના nodes incompatible};
        \node [gtu block, right=of Invalid] (Split) {બે અલગ chains};
        
        \draw [gtu arrow] (Org) -- (Fork);
        \draw [gtu arrow] (Fork) -- (Soft);
        \draw [gtu arrow] (Soft) -- (Valid);
        \draw [gtu arrow] (Valid) -- (Single);
        
        \draw [gtu arrow] (Fork) -- (Hard);
        \draw [gtu arrow] (Hard) -- (Invalid);
        \draw [gtu arrow] (Invalid) -- (Split);
    \end{tikzpicture}
    \caption{Soft Fork vs Hard Fork Outcome}
\end{figure}

\begin{itemize}
    \item \keyword{જોખમ}: Hard forks community split કરી શકે અને competing currencies બનાવી શકે
    \item \keyword{સુરક્ષા}: Soft forks સામાન્ય રીતે સુરક્ષિત અને ઓછા disruptive
\end{itemize}

\begin{mnemonicbox}
Soft = સમાન દિશા, Hard = મોટો તફાવત
\end{mnemonicbox}

\orquestionmarks{3(બ)}{4}{બ્લોકચેનની દુનિયામાં Finality નાં શું મહત્વ છે?}

\keyword{Finality વ્યાખ્યા}: એક વાર transaction confirm થઈ ગયા પછી તે reverse કે alter ન થઈ શકે તેની ગેરંટી.

\textbf{મહત્વ:}

\begin{tabulary}{\linewidth}{L L}
    \toprule
    \textbf{બાબત} & \textbf{મહત્વ} \\
    \midrule
    \textbf{Trust} & Users ને વિશ્વાસ કે transactions કાયમી છે \\
    \textbf{Business Use} & Companies completed transactions પર ભરોસો કરી શકે \\
    \textbf{Legal Certainty} & Courts blockchain records enforce કરી શકે \\
    \textbf{Settlement} & Financial institutions payments clear કરી શકે \\
    \bottomrule
\end{tabulary}

\begin{figure}[H]
    \centering
    \begin{tikzpicture}[node distance=1.5cm, auto]
        \node [gtu state] (Sub) {Submitted};
        \node [gtu block, right=of Sub] (Conf1) {1st Confirm};
        \node [gtu block, right=of Conf1] (ConfN) {N Confirms};
        \node [gtu block, below=of ConfN] (Prob) {Probabilistic};
        \node [gtu block, left=of Prob] (Final) {Practical Finality};
        
        \draw [gtu arrow] (Sub) -- (Conf1);
        \draw [gtu arrow] (Conf1) -- (ConfN);
        \draw [gtu arrow] (ConfN) -- (Prob);
        \draw [gtu arrow] (Prob) -- (Final);
    \end{tikzpicture}
    \caption{Consensus and Finality Process}
\end{figure}

\begin{itemize}
    \item \keyword{Bitcoin}: 6 confirmations સામાન્ય રીતે final ગણાય
    \item \keyword{Ethereum}: Proof of Stake સાથે ઝડપી finality તરફ જતું
\end{itemize}

\begin{mnemonicbox}
Final = હંમેશ માટે, મહત્વપૂર્ણ = પાછું ન બદલાય
\end{mnemonicbox}

\orquestionmarks{3(ક)}{7}{બ્લોકચેનમાં 51\% attack શું છે? ટૂંકમાં સમજાવો.}

\keyword{51\% Attack વ્યાખ્યા}: જ્યારે કોઈ એક entity network ની 50\% થી વધુ mining power અથવા validators ને control કરે અને blockchain manipulate કરી શકે.

\textbf{Attack પદ્ધતિ:}

\begin{enumerate}
    \item \textbf{Control}: >50\% mining power મેળવવું
    \item \textbf{Double Spend}: ગુપ્ત chain બનાવવી (alternative history)
    \item \textbf{Execute}: લાંબી chain release કરવી
    \item \textbf{Profit}: Coins બે વાર spend કરવા
\end{enumerate}

\begin{figure}[H]
    \centering
    \begin{tikzpicture}[node distance=1.5cm]
        \node [gtu block] (B_N) {Block N};
        
        % Honest Chain
        \node [gtu block, above right=2cm of B_N] (Honest1) {Block N+1};
        \node [right=0.2cm of Honest1] {Honest Chain છોડી દેવાય};
        
        % Attacker Chain
        \node [gtu block, below right=2cm of B_N, fill=red!10] (Bad1) {Block N'+1};
        \node [gtu block, right=of Bad1, fill=red!10] (Bad2) {Block N'+2 - લાંબી Chain};
        \node [right=0.2cm of Bad2] {Attacker Chain Accepted};

        \draw [gtu arrow] (B_N) -- (Honest1);
        \draw [gtu arrow] (B_N) -- (Bad1);
        \draw [gtu arrow] (Bad1) -- (Bad2);
    \end{tikzpicture}
    \caption{51\% Attack: Longest Chain Rule Abuse}
\end{figure}

\textbf{બચાવના પદ્ધતિઓ:}

\begin{tabulary}{\linewidth}{L L}
    \toprule
    \textbf{પદ્ધતિ} & \textbf{કેવી રીતે મદદ કરે} \\
    \midrule
    \textbf{Decentralization} & Mining ઘણા participants માં વહેંચવું \\
    \textbf{High Hash Rate} & Attack ને economically અશક્ય બનાવવું \\
    \textbf{Proof of Stake} & Attackers ના staked coins ગુમાવવા \\
    \bottomrule
\end{tabulary}

\begin{mnemonicbox}
51\% = બહુમતીની બદમાશી, Control = કોલાહલ
\end{mnemonicbox}

\questionmarks{4(અ)}{3}{વિવિધ પ્રકારના Hyperledger પ્રોજેક્ટ્સનાં વર્ણન કરો.}

\keyword{Hyperledger Project Types}:

\begin{tabulary}{\linewidth}{L L L}
    \toprule
    \textbf{Project} & \textbf{હેતુ} & \textbf{Use Case} \\
    \midrule
    \textbf{Fabric} & Modular blockchain platform & Enterprise applications \\
    \textbf{Sawtooth} & Scalable blockchain suite & Supply chain, IoT \\
    \textbf{Iroha} & Mobile-focused blockchain & Identity management \\
    \textbf{Indy} & Digital identity platform & Self-sovereign identity \\
    \textbf{Besu} & Ethereum-compatible client & Public/private Ethereum \\
    \textbf{Burrow} & Smart contract platform & Permissioned networks \\
    \bottomrule
\end{tabulary}

\begin{figure}[H]
    \centering
    \begin{tikzpicture}[node distance=1.5cm, auto]
        \node [gtu block] (HL) {Hyperledger};
        \node [gtu block, below left=of HL] (Frame) {Frameworks};
        \node [gtu block, below right=of HL] (Tools) {Tools};
        
        \node [gtu state, below=of Frame, align=center] (F_List) {Fabric\\Sawtooth\\Iroha\\Indy};
        \node [gtu state, below=of Tools, align=center] (T_List) {Caliper\\Composer\\Explorer};
        
        \draw [gtu arrow] (HL) -- (Frame);
        \draw [gtu arrow] (HL) -- (Tools);
        \draw [gtu arrow] (Frame) -- (F_List);
        \draw [gtu arrow] (Tools) -- (T_List);
    \end{tikzpicture}
    \caption{Hyperledger Ecosystem}
\end{figure}

\begin{itemize}
    \item \keyword{ફોકસ}: Enterprise અને business blockchain solutions
    \item \keyword{Open Source}: બધા projects મુફતમાં ઉપલબ્ધ
\end{itemize}

\begin{mnemonicbox}
Hyperledger = High Performance, Low Publicity
\end{mnemonicbox}

\questionmarks{4(બ)}{4}{Blockchain અને Bitcoin નો તફાવત આપો.}

\textbf{વ્યાપક સરખામણી:}

\begin{tabulary}{\linewidth}{L L L}
    \toprule
    \textbf{બાબત} & \textbf{Blockchain} & \textbf{Bitcoin} \\
    \midrule
    \textbf{વ્યાખ્યા} & Technology/Platform & Digital Currency \\
    \textbf{અવકાશ} & વ્યાપક concept & Specific application \\
    \textbf{હેતુ} & Record keeping system & Peer-to-peer payments \\
    \textbf{Applications} & ઘણા industries & મુખ્યત્વે financial \\
    \textbf{લવચિકતા} & Customize કરી શકાય & Fixed protocol \\
    \bottomrule
\end{tabulary}

\begin{figure}[H]
    \centering
    \begin{tikzpicture}[node distance=1.5cm]
        \node [gtu block] (Tech) {Blockchain Technology};
        \node [gtu block, below=of Tech] (BTC) {Bitcoin Cryptocurrency};
        \node [gtu block, right=of BTC] (ETH) {Ethereum Platform};
        \node [gtu block, left=of BTC] (Apps) {Supply Chain Apps};
        
        \draw [gtu arrow] (Tech) -- (BTC);
        \draw [gtu arrow] (Tech) -- (ETH);
        \draw [gtu arrow] (Tech) -- (Apps);
    \end{tikzpicture}
    \caption{Blockchain vs Bitcoin Relationship}
\end{figure}

\begin{itemize}
    \item \keyword{સમાનતા}: Blockchain ઈન્ટરનેટ જેવું, Bitcoin email જેવું
    \item \keyword{નિર્ભરતા}: Bitcoin ને blockchain જોઈએ, પણ blockchain ને Bitcoin જરૂરી નથી
\end{itemize}

\begin{mnemonicbox}
Blockchain = Building Block, Bitcoin = Specific Brick
\end{mnemonicbox}

\questionmarks{4(ક)}{7}{ટૂંકી નોંધ લખો: Merkle Tree}

\keyword{Merkle Tree વ્યાખ્યા}: Binary tree structure જેમાં દરેક leaf transaction hash દર્શાવે અને દરેક internal node તેના children નો hash ધરાવે.

\textbf{Structure અને Components:}

\begin{itemize}
    \item \keyword{Leaf Nodes}: Individual transaction hashes
    \item \keyword{Internal Nodes}: બે child nodes નો hash
    \item \keyword{Root Hash}: આખા tree નો single hash
\end{itemize}

\begin{figure}[H]
    \centering
    \begin{tikzpicture}[level distance=1.5cm, sibling distance=2.5cm, every node/.style={gtu block, minimum width=1.5cm}]
        \node {Root Hash}
            child {node {Hash AB}
                child {node {Hash A}
                    child {node {TX A}}
                }
                child {node {Hash B}
                    child {node {TX B}}
                }
            }
            child {node {Hash CD}
                child {node {Hash C}
                    child {node {TX C}}
                }
                child {node {Hash D}
                    child {node {TX D}}
                }
            };
    \end{tikzpicture}
    \caption{Merkle Tree Structure}
\end{figure}

\textbf{ફાયદા:}

\begin{itemize}
    \item \keyword{Efficiency}: બધા data download કર્યા વગર ઝડપી verification
    \item \keyword{Security}: કોઈપણ change તુરંત detect થાય
    \item \keyword{Storage}: Block header માં ફક્ત root hash જરૂરી
\end{itemize}

\begin{figure}[H]
    \centering
    \begin{tikzpicture}[node distance=1.5cm, auto]
        \node [gtu state] (Tx) {Tx};
        \node [gtu block, right=of Tx] (Path) {Get Merkle Path};
        \node [gtu block, right=of Path] (Comp) {Compute Root};
        \node [gtu block, below=of Comp] (Check) {Compare Stored Root};
        \node [gtu state, left=of Check] (Res) {Valid/Invalid};
        
        \draw [gtu arrow] (Tx) -- (Path);
        \draw [gtu arrow] (Path) -- (Comp);
        \draw [gtu arrow] (Comp) -- (Check);
        \draw [gtu arrow] (Check) -- (Res);
    \end{tikzpicture}
    \caption{Verification Process}
\end{figure}

\begin{mnemonicbox}
Merkle = Many Made One, Tree = Trustworthy
\end{mnemonicbox}

\orquestionmarks{4(અ)}{3}{Hash pointer વિશે ટૂંકમાં ચર્ચા કરો અને Merkle tree માં તેનો ઉપયોગ કેવી રીતે થાય છે.}

\keyword{Hash Pointer વ્યાખ્યા}: Data structure જેમાં data નું location અને તે data નો cryptographic hash બંને હોય.

\begin{figure}[H]
    \centering
    \begin{tikzpicture}[node distance=2cm, auto]
        \node [draw, rectangle, minimum height=1cm] (Ptr1) {Pointer};
        \node [draw, rectangle, minimum height=1cm, right=0pt of Ptr1] (Ptr2) {Hash};
        \node [draw, cloud, right=of Ptr2, aspect=2] (Data) {Data};
        \draw [->, thick] (Ptr1.west) to[out=180,in=180, looseness=2] (Data.west);
        \draw [->, dashed] (Data) -- (Ptr2.south);
    \end{tikzpicture}
    \caption{Hash Pointer Concept}
\end{figure}

\textbf{Merkle Tree માં ઉપયોગ:}

\begin{itemize}
    \item \keyword{Leaf Level}: Transaction ને point કરે, transaction hash ધરાવે
    \item \keyword{Internal Nodes}: Children ને point કરે, combined hash ધરાવે
    \item \keyword{Root}: Tree structure ને point કરે, overall hash ધરાવે
\end{itemize}

\begin{mnemonicbox}
Hash Pointer = સ્થાન + Verification
\end{mnemonicbox}

\orquestionmarks{4(બ)}{4}{બ્લોકચેનમાં Hashing શું છે? Bitcoin માં તે કેવી રીતે ઉપયોગી છે?}

\keyword{Hashing વ્યાખ્યા}: Mathematical function જે input data ને fixed-size characters ના string માં convert કરે.

\begin{tabulary}{\linewidth}{L L}
    \toprule
    \textbf{Property} & \textbf{વર્ણન} \\
    \midrule
    \textbf{Deterministic} & સમાન input હંમેશા સમાન output આપે \\
    \textbf{Fixed Size} & Output હંમેશા સમાન length (SHA-256 માટે 256 bits) \\
    \textbf{Avalanche Effect} & નાનો input change = સંપૂર્ણ અલગ output \\
    \textbf{One-way} & Original input શોધવા માટે reverse કરી શકાતું નથી \\
    \bottomrule
\end{tabulary}

\begin{figure}[H]
    \centering
    \begin{tikzpicture}[node distance=2cm]
        \node [gtu block] (Input1) {Input Data};
        \node [gtu state, right=3cm of Input1] (Hash1) {Hash Output 1};
        \draw [gtu arrow, ->] (Input1) -- node[above] {SHA-256} (Hash1);
        
        \node [gtu block, below=of Input1] (Input2) {Input માં નાનો Change};
        \node [gtu state, right=3cm of Input2] (Hash2) {સંપૂર્ણ અલગ Hash};
        \draw [gtu arrow, ->] (Input2) -- node[above] {SHA-256} (Hash2);
    \end{tikzpicture}
    \caption{Avalanche Effect in Hashing}
\end{figure}

\begin{itemize}
    \item \keyword{Algorithm}: Bitcoin SHA-256 hashing વાપરે
    \item \keyword{સુરક્ષા}: Blockchain ને tamper-evident બનાવે
\end{itemize}

\begin{mnemonicbox}
Hash = Fingerprint, Bitcoin = Hashing પર આધારિત
\end{mnemonicbox}

\orquestionmarks{4(ક)}{7}{Classic Byzantine generals problem અને Practical Byzantine Fault Tolerance ને વિગતવાર સમજાવો.}

\textbf{Byzantine Generals Problem}: Distributed systems માં unreliable participants સાથે consensus achieve કરવાની સમસ્યા.

\begin{figure}[H]
    \centering
    \begin{tikzpicture}[node distance=2cm]
        \node [gtu block] (City) {City Under Siege};
        \node [gtu state, above=of City] (GeneralA) {General A (Honest)};
        \node [gtu state, right=of City, fill=red!10] (GeneralB) {General B (Traitor)};
        \node [gtu state, left=of City] (GeneralC) {General C (Honest)};
        
        \draw [gtu arrow] (GeneralA) -- node[left] {Attack} (City);
        \draw [gtu arrow] (GeneralC) -- node[right] {Attack} (City);
        \draw [gtu arrow] (GeneralB) -- node[left] {Retreat (to C)} (City);
        
        \draw [gtu arrow, dashed] (GeneralB) -- node[above] {Attack} (GeneralA);
    \end{tikzpicture}
    \caption{Byzantine Generals Problem}
\end{figure}

\textbf{Practical Byzantine Fault Tolerance (pBFT):}

\begin{itemize}
    \item \textbf{Pre-prepare}: Leader proposal broadcast કરે
    \item \textbf{Prepare}: Nodes validate કરે અને agreement broadcast કરે
    \item \textbf{Commit}: Nodes decision પર commit કરે
\end{itemize}

\begin{figure}[H]
    \centering
    \begin{tikzpicture}[node distance=1.5cm, auto]
        \node [gtu state] (Client) {Client};
        \node [gtu block, right=of Client] (Primary) {Primary};
        \node [gtu block, right=of Primary] (Backup) {Backups};
        \node [gtu block, below=of Backup] (Commit) {Commit};
        \node [gtu state, left=of Commit] (Reply) {Reply};
        
        \draw [gtu arrow] (Client) -- (Primary);
        \draw [gtu arrow] (Primary) -- (Backup);
        \draw [gtu arrow] (Backup) -- (Commit);
        \draw [gtu arrow] (Commit) -- (Reply);
    \end{tikzpicture}
    \caption{pBFT Process Flow}
\end{figure}

\begin{mnemonicbox}
Byzantine = Bad actors, pBFT = Practical Fix
\end{mnemonicbox}

\questionmarks{5(અ)}{3}{બ્લોકચેનમાં cryptocurrency wallets ની યાદી બનાવો અને સમજાવો.}

\textbf{Cryptocurrency Wallet પ્રકારો:}

\begin{tabulary}{\linewidth}{L L L}
    \toprule
    \textbf{Wallet Type} & \textbf{વર્ણન} & \textbf{Security Level} \\
    \midrule
    \textbf{Hardware Wallet} & Keys store કરતા physical device & ખૂબ ઊંચી \\
    \textbf{Software Wallet} & Computer/phone પર application & મધ્યમ થી ઊંચી \\
    \textbf{Paper Wallet} & કાગળ પર છપાયેલી keys & ઊંચી (સુરક્ષિત રીતે stored હોય તો) \\
    \textbf{Web Wallet} & Online wallet service & મધ્યમ \\
    \bottomrule
\end{tabulary}

\begin{figure}[H]
    \centering
    \begin{tikzpicture}[node distance=1.5cm, auto]
        \node [gtu block] (Wallet) {Cryptocurrency Wallet};
        \node [gtu block, below=of Wallet] (Store) {Private Keys Store કરે};
        \node [gtu block, left=of Store] (Gen) {Addresses Generate કરે};
        \node [gtu block, right=of Store] (Sign) {Transactions Sign કરે};
        \node [gtu block, below=of Store] (Check) {Balances Check કરે};
        \node [gtu block, right=of Check] (Send) {Crypto Send/Receive કરે};
        
        \draw [gtu arrow] (Wallet) -- (Store);
        \draw [gtu arrow] (Wallet) -- (Gen);
        \draw [gtu arrow] (Wallet) -- (Sign);
        \draw [gtu arrow] (Wallet) -- (Check);
        \draw [gtu arrow] (Wallet) -- (Send);
    \end{tikzpicture}
    \caption{Functions of a Wallet}
\end{figure}

\begin{mnemonicbox}
Wallet = Key Keeper, Not Coin Container
\end{mnemonicbox}

\questionmarks{5(બ)}{4}{ERC-20 ટોકનના ફાયદા અને ગેરફાયદા લખો.}

\keyword{ERC-20 Token વ્યાખ્યા}: Ethereum blockchain પર tokens બનાવવા માટેનો standard protocol.

\textbf{ફાયદા:}

\begin{tabulary}{\linewidth}{L L}
    \toprule
    \textbf{ફાયદો} & \textbf{લાભ} \\
    \midrule
    \textbf{Standardization} & બધા tokens સમાન રીતે કામ કરે \\
    \textbf{Interoperability} & બધા Ethereum wallets સાથે compatible \\
    \textbf{Easy Development} & નવા tokens બનાવવા સરળ \\
    \textbf{Wide Support} & Exchanges અને services દ્વારા support \\
    \bottomrule
\end{tabulary}

\textbf{ગેરફાયદા:}

\begin{tabulary}{\linewidth}{L L}
    \toprule
    \textbf{ગેરફાયદા} & \textbf{સમસ્યા} \\
    \midrule
    \textbf{Gas Fees} & Network congestion દરમિયાન મોંઘા transactions \\
    \textbf{Scalability} & Ethereum ની transaction throughput દ્વારા મર્યાદિત \\
    \textbf{Security Risks} & Smart contract bugs થી token loss \\
    \textbf{Centralization} & ઘણા tokens નું centralized control \\
    \bottomrule
\end{tabulary}

\begin{mnemonicbox}
ERC-20 = Easy અને Expensive
\end{mnemonicbox}

\questionmarks{5(ક)}{7}{dApps નો ઉપયોગ શેના માટે થાય છે? dApps ના ફાયદા અને ગેરફાયદા સમજાવો.}

\keyword{dApps વ્યાખ્યા}: Decentralized Applications જે blockchain networks પર central authority વગર run થાય.

\textbf{dApps ઉપયોગ વર્ગીકરણ:}

\begin{tabulary}{\linewidth}{L L L}
    \toprule
    \textbf{વર્ગ} & \textbf{ઉદાહરણો} & \textbf{હેતુ} \\
    \midrule
    \textbf{DeFi} & Uniswap, Compound & Financial services \\
    \textbf{Gaming} & CryptoKitties & Blockchain games \\
    \textbf{Social Media} & Steemit & Censorship-resistant platforms \\
    \textbf{Marketplaces} & OpenSea & NFT trading \\
    \bottomrule
\end{tabulary}

\begin{figure}[H]
    \centering
    \begin{tikzpicture}[node distance=1.5cm, auto]
        \node [gtu state] (User) {Frontend UI};
        \node [gtu block, right=of User] (Web3) {Web3 Connection};
        \node [gtu block, right=of Web3] (Smart) {Smart Contracts};
        \node [gtu block, right=of Smart] (Chain) {Blockchain};
        \node [gtu block, below=of Smart] (Store) {Distributed Storage};
        
        \draw [gtu arrow] (User) -- (Web3);
        \draw [gtu arrow] (Web3) -- (Smart);
        \draw [gtu arrow] (Smart) -- (Chain);
        \draw [gtu arrow] (Smart) -- (Store);
    \end{tikzpicture}
    \caption{dApp Architecture}
\end{figure}

\textbf{ફાયદા:}

\begin{itemize}
    \item \keyword{Censorship Resistance}: કોઈ એક control point નથી
    \item \keyword{Transparency}: Code અને data publicly verifiable
    \item \keyword{No Downtime}: ઘણા nodes માં distributed
\end{itemize}

\textbf{ગેરફાયદા:}

\begin{itemize}
    \item \keyword{Poor User Experience}: જટિલ interfaces, ધીમા transactions
    \item \keyword{High Costs}: દરેક interaction માટે gas fees
    \item \keyword{Immutable Bugs}: Smart contract errors સહેલાઈથી fix ન કરી શકાય
\end{itemize}

\begin{mnemonicbox}
dApps = Decentralized but Difficult
\end{mnemonicbox}

\orquestionmarks{5(અ)}{3}{Tokenized અને token less બ્લોકચેનને વિગતવાર સમજાવો.}

\textbf{સરખામણી કોષ્ટક:}

\begin{tabulary}{\linewidth}{L L L}
    \toprule
    \textbf{બાબત} & \textbf{Tokenized} & \textbf{Token-less} \\
    \midrule
    \textbf{Incentive Model} & Economic rewards & Permission-based \\
    \textbf{Access} & Tokens હોય તો કોઈપણ & Restricted access \\
    \textbf{Governance} & Token holder voting & Centralized control \\
    \textbf{Use Case} & Public networks & Private/enterprise \\
    \textbf{Security} & Economic game theory & Traditional security \\
    \bottomrule
\end{tabulary}

\begin{figure}[H]
    \centering
    \begin{tikzpicture}[node distance=1.5cm]
        % Tokenized
        \node [gtu block] (Token) {Token Rewards};
        \node [gtu block, right=of Token] (Miner) {Miners};
        \node [gtu block, right=of Miner] (Sec) {Secure Network};
        \node [gtu block, right=of Sec] (Pub) {Public Access};
        
        \draw [gtu arrow] (Token) -- (Miner);
        \draw [gtu arrow] (Miner) -- (Sec);
        \draw [gtu arrow] (Sec) -- (Pub);
        
        \node [below=0.2cm of Miner] {Tokenized Blockchain};

        % Token-less
        \node [gtu block, below=2cm of Token] (Perm) {Permission System};
        \node [gtu block, right=of Perm] (Auth) {Auth Nodes};
        \node [gtu block, right=of Auth] (Sec2) {Secure Network};
        \node [gtu block, right=of Sec2] (Priv) {Private Access};
        
        \draw [gtu arrow] (Perm) -- (Auth);
        \draw [gtu arrow] (Auth) -- (Sec2);
        \draw [gtu arrow] (Sec2) -- (Priv);
        
        \node [below=0.2cm of Auth] {Token-less Blockchain};
    \end{tikzpicture}
    \caption{Tokenized vs Token-less Architecture}
\end{figure}

\begin{mnemonicbox}
Token = Public Participation, Token-less = Private Permission
\end{mnemonicbox}

\orquestionmarks{5(બ)}{4}{Hyperledger ના ફાયદા અને ગેરફાયદા લખો.}

\keyword{Hyperledger વ્યાખ્યા}: Enterprise-grade blockchain solutions develop કરવા માટેનું open-source collaborative framework.

\textbf{ફાયદા:}

\begin{itemize}
    \item \keyword{Enterprise Focus}: Business use cases માટે design
    \item \keyword{Modular Architecture}: જરૂર પ્રમાણે components customize કરી શકાય
    \item \keyword{Privacy}: Confidential transactions શક્ય
    \item \keyword{Permissioned Network}: કોણ participate કરી શકે તેનું control
\end{itemize}

\textbf{ગેરફાયદા:}

\begin{itemize}
    \item \keyword{Centralization}: Public blockchains કરતાં ઓછું decentralized
    \item \keyword{Complexity}: Implement કરવા માટે technical expertise જરૂરી
    \item \keyword{No Token Economy}: Crypto incentives leverage કરી શકાતું નથી
\end{itemize}

\begin{mnemonicbox}
Hyperledger = High Performance, Low Publicity
\end{mnemonicbox}

\orquestionmarks{5(ક)}{7}{Smart contract સમજાવો. Smart contract ની વિવિધ એપ્લિકેશન્સ લખો.}

\keyword{Smart Contract વ્યાખ્યા}: Self-executing contracts જેના terms સીધા code માં લખાયેલા હોય અને blockchain પર આપોઆપ enforce થાય.

\begin{figure}[H]
    \centering
    \begin{tikzpicture}[node distance=1.5cm, auto]
        \node [gtu block] (Create) {Contract Created};
        \node [gtu block, right=of Create] (Deploy) {Blockchain પર Deployed};
        \node [gtu block, below=of Deploy] (Monitor) {Conditions Monitored};
        \node [draw, diamond, aspect=2, below=of Monitor] (Check) {Conditions Met?};
        
        \node [gtu block, left=of Check] (Exec) {Contract Executes};
        \node [gtu block, above=of Exec] (Settle) {Automatic Settlement};
        
        \draw [gtu arrow] (Create) -- (Deploy);
        \draw [gtu arrow] (Deploy) -- (Monitor);
        \draw [gtu arrow] (Monitor) -- (Check);
        \draw [gtu arrow] (Check) -- node[above] {Yes} (Exec);
        \draw [gtu arrow] (Exec) -- (Settle);
        
        \draw [gtu arrow] (Check.east) -- ++(1,0) |- node[near start, right] {No} (Monitor);
    \end{tikzpicture}
    \caption{Smart Contract Workflow}
\end{figure}

\textbf{Industry પ્રમાણે Applications:}

\begin{tabulary}{\linewidth}{L L L}
    \toprule
    \textbf{Industry} & \textbf{Application} & \textbf{ફાયદો} \\
    \midrule
    \textbf{Finance} & Automated loans & ઝડપી, ઓછી costs \\
    \textbf{Real Estate} & Property transfers & ફ્રોડ ઘટાડવું \\
    \textbf{Supply Chain} & Product tracking & પારદર્શિતા \\
    \textbf{Healthcare} & Insurance claims & Privacy protection \\
    \bottomrule
\end{tabulary}

\begin{mnemonicbox}
Smart Contract = Self-executing, Solves Problems
\end{mnemonicbox}

\end{document}
