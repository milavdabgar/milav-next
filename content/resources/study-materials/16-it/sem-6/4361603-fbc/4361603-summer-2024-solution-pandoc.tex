\documentclass[10pt,a4paper]{article}

% content/resources/templates/preamble.tex
\usepackage[margin=0.6in]{geometry}
\author{Milav Dabgar}
\usepackage{amsmath,amssymb,amsthm}
\usepackage{booktabs}
\usepackage{multirow}
\usepackage{xcolor}
\usepackage{tcolorbox}
\tcbuselibrary{breakable,skins}
\usepackage[colorlinks=true,linkcolor=blue]{hyperref}
\usepackage{titlesec}
\usepackage{enumitem}
\usepackage{tikz}
\usepackage{pgfplots}
\usepackage{circuitikz}
\usepackage[version=4]{mhchem}
\usepackage{longtable}
\usepackage{array}
\usepackage{float}
\usepackage{caption}
\usepackage{listings}

\lstset{
  basicstyle=\small\ttfamily,
  breaklines=true,
  breakatwhitespace=false,
  postbreak=\mbox{\textcolor{red}{$\hookrightarrow$}\space},
  float=false,
  numbers=left,
  numberstyle=\tiny\color{gray},
  numbersep=10pt,
  xleftmargin=2em,
  keywordstyle=\color{blue},
  commentstyle=\color{green!60!black},
  stringstyle=\color{purple},
  backgroundcolor=\color{gray!5},
  showstringspaces=false,
  tabsize=2,
  captionpos=b,
  keepspaces=true,
  columns=flexible
}

\pgfplotsset{compat=1.18}
\usetikzlibrary{shapes,arrows,positioning,calc,patterns,decorations.pathmorphing,decorations.markings,arrows.meta}

% Color scheme
\definecolor{headcolor}{RGB}{0,102,204}
\definecolor{keycolor}{RGB}{220,20,60}
\definecolor{solutioncolor}{RGB}{34,139,34}
\definecolor{mnemoniccolor}{RGB}{148,0,211}
\definecolor{codecolor}{RGB}{0,0,100}

% Spacing
\setlength{\parskip}{3pt}
\setlist[itemize]{nosep}
\setlist[enumerate]{nosep}

% Title formatting
\titleformat{\section}{\Large\bfseries\color{headcolor}}{\thesection}{1em}{}
\titleformat{\subsection}{\large\bfseries\color{headcolor}}{\thesubsection}{1em}{}

% Pandoc tightlist compatibility
\providecommand{\tightlist}{%
  \setlength{\itemsep}{0pt}\setlength{\parskip}{0pt}}

% Pandoc longtable compatibility
\newcounter{none}
\def\thenone{}


% content/resources/templates/english-boxes.tex
% This file is currently empty - it exists to maintain consistency with the import structure.
% Add custom environments here if needed in the future.


\begin{document}

\begin{center}
{\Huge\bfseries\color{headcolor} Subject Name Solutions}\\[5pt]
{\LARGE 4361603 -- Summer 2024}\\[3pt]
{\large Semester 1 Study Material}\\[3pt]
{\normalsize\textit{Detailed Solutions and Explanations}}
\end{center}

\vspace{10pt}

\subsection*{Question 1(a) [3 marks]}\label{q1a}

\textbf{Explain benefits of using distributed ledger systems.}

\begin{solutionbox}


{\def\LTcaptype{none} % do not increment counter
\vspace{-5pt}
\captionof{table}{Benefits of Distributed Ledger Systems}
\vspace{-10pt}
\begin{longtable}[]{@{}ll@{}}
\toprule\noalign{}
Benefit & Description \\
\midrule\noalign{}
\endhead
\bottomrule\noalign{}
\endlastfoot
\textbf{Transparency} & All participants can view transaction history \\
\textbf{Security} & Cryptographic protection against tampering \\
\textbf{Decentralization} & No single point of failure or control \\
\textbf{Immutability} & Records cannot be altered once confirmed \\
\end{longtable}
}

\end{solutionbox}
\begin{mnemonicbox}
``T-S-D-I'' (Transparent, Secure, Decentralized,
Immutable)

\end{mnemonicbox}
\subsection*{Question 1(b) [4 marks]}\label{q1b}

\textbf{Define: 1) Blockchain 2) Distributed systems}

\begin{solutionbox}


{\def\LTcaptype{none} % do not increment counter
\vspace{-5pt}
\captionof{table}{Key Definitions}
\vspace{-10pt}
\begin{longtable}[]{@{}
  >{\raggedright\arraybackslash}p{(\linewidth - 2\tabcolsep) * \real{0.3333}}
  >{\raggedright\arraybackslash}p{(\linewidth - 2\tabcolsep) * \real{0.6667}}@{}}
\toprule\noalign{}
\begin{minipage}[b]{\linewidth}\raggedright
Term
\end{minipage} & \begin{minipage}[b]{\linewidth}\raggedright
Definition
\end{minipage} \\
\midrule\noalign{}
\endhead
\bottomrule\noalign{}
\endlastfoot
\textbf{Blockchain} & A chain of blocks containing transaction data,
linked using cryptographic hashes \\
\textbf{Distributed Systems} & Network of independent computers working
together as a single system \\
\end{longtable}
}

\textbf{Key Features}:

\begin{itemize}
\tightlist
\item
  \textbf{Blockchain}: Uses hash pointers, consensus mechanisms, and
  merkle trees
\item
  \textbf{Distributed Systems}: Fault tolerance, scalability, and
  resource sharing
\end{itemize}

\end{solutionbox}
\begin{mnemonicbox}
``Chain-Hash-Consensus'' for Blockchain,
``Network-Independent-Together'' for Distributed

\end{mnemonicbox}
\subsection*{Question 1(c) [7 marks]}\label{q1c}

\textbf{Illustrate CAP theorem with the help of Blockchain network.}

\begin{solutionbox}


{\def\LTcaptype{none} % do not increment counter
\vspace{-5pt}
\captionof{table}{CAP Theorem Components}
\vspace{-10pt}
\begin{longtable}[]{@{}
  >{\raggedright\arraybackslash}p{(\linewidth - 4\tabcolsep) * \real{0.2381}}
  >{\raggedright\arraybackslash}p{(\linewidth - 4\tabcolsep) * \real{0.3095}}
  >{\raggedright\arraybackslash}p{(\linewidth - 4\tabcolsep) * \real{0.4524}}@{}}
\toprule\noalign{}
\begin{minipage}[b]{\linewidth}\raggedright
Property
\end{minipage} & \begin{minipage}[b]{\linewidth}\raggedright
Description
\end{minipage} & \begin{minipage}[b]{\linewidth}\raggedright
Blockchain Context
\end{minipage} \\
\midrule\noalign{}
\endhead
\bottomrule\noalign{}
\endlastfoot
\textbf{Consistency} & All nodes see same data & All nodes have
identical ledger \\
\textbf{Availability} & System remains operational & Network stays
accessible \\
\textbf{Partition Tolerance} & Works despite network failures &
Continues during node disconnections \\
\end{longtable}
}

\textbf{Diagram:}

\begin{center}
\textbf{Mermaid Diagram (Code)}
\begin{verbatim}
{Shaded}
{Highlighting}[]
graph TD
    A[CAP Theorem] {-{-}{} B[Consistency]}
    A {-{-}{} C[Availability] }
    A {-{-}{} D[Partition Tolerance]}
    B {-{-}{} E[Bitcoin prioritizes C+P]}
    C {-{-}{} F[Some systems choose A+P]}
    D {-{-}{} G[Essential for blockchain]}
{Highlighting}
{Shaded}
\end{verbatim}
\end{center}

\textbf{Key Points}:

\begin{itemize}
\tightlist
\item
  \textbf{Trade-off}: Can only achieve 2 out of 3 properties
  simultaneously
\item
  \textbf{Blockchain Choice}: Most blockchains choose Consistency +
  Partition Tolerance
\item
  \textbf{Example}: Bitcoin may become temporarily unavailable but
  maintains consistency
\end{itemize}

\end{solutionbox}
\begin{mnemonicbox}
``CAP-2-out-of-3'' (Choose Any 2 Properties out of 3)

\end{mnemonicbox}
\subsection*{Question 1(c) OR [7
marks]}\label{q1c}

\textbf{List and explain applications of blockchain network.}

\begin{solutionbox}


{\def\LTcaptype{none} % do not increment counter
\vspace{-5pt}
\captionof{table}{Blockchain Applications}
\vspace{-10pt}
\begin{longtable}[]{@{}lll@{}}
\toprule\noalign{}
Application & Description & Example \\
\midrule\noalign{}
\endhead
\bottomrule\noalign{}
\endlastfoot
\textbf{Cryptocurrency} & Digital money transactions & Bitcoin,
Ethereum \\
\textbf{Supply Chain} & Track products from origin & Walmart food
tracing \\
\textbf{Healthcare} & Secure patient records & Medical data sharing \\
\textbf{Voting} & Transparent elections & Estonia e-voting \\
\textbf{Real Estate} & Property ownership records & Land registries \\
\end{longtable}
}

\textbf{Key Benefits}:

\begin{itemize}
\tightlist
\item
  \textbf{Transparency}: All transactions visible to participants
\item
  \textbf{Security}: Cryptographic protection against fraud
\item
  \textbf{Efficiency}: Reduced intermediaries and costs
\end{itemize}

\end{solutionbox}
\begin{mnemonicbox}
``C-S-H-V-R'' (Crypto, Supply, Health, Vote, Real
estate)

\end{mnemonicbox}
\subsection*{Question 2(a) [3 marks]}\label{q2a}

\textbf{Define and explain a permissionless blockchain in detail.}

\begin{solutionbox}

\textbf{Definition}: A blockchain where anyone can participate without
requiring permission from a central authority.


{\def\LTcaptype{none} % do not increment counter
\vspace{-5pt}
\captionof{table}{Permissionless Blockchain Features}
\vspace{-10pt}
\begin{longtable}[]{@{}ll@{}}
\toprule\noalign{}
Feature & Description \\
\midrule\noalign{}
\endhead
\bottomrule\noalign{}
\endlastfoot
\textbf{Open Access} & Anyone can join and participate \\
\textbf{Public Verification} & All transactions are publicly
verifiable \\
\textbf{Decentralized} & No central controlling authority \\
\end{longtable}
}

\textbf{Key Characteristics}:

\begin{itemize}
\tightlist
\item
  \textbf{Consensus}: Uses proof-of-work or proof-of-stake
\item
  \textbf{Examples}: Bitcoin, Ethereum mainnet
\end{itemize}

\end{solutionbox}
\begin{mnemonicbox}
``Open-Public-Decentralized'' (OPD)

\end{mnemonicbox}
\subsection*{Question 2(b) [4 marks]}\label{q2b}

\textbf{Draw a figure and provide a brief explanation of a data
structure of a blockchain.}

\begin{solutionbox}

\textbf{Diagram: Blockchain Data Structure}

\begin{verbatim}
+{-{-}{-}{-}{-}{-}{-}{-}{-}{-}{-}{-}{-}{-}{-}{-}{-}{-}{-}+    +{-}{-}{-}{-}{-}{-}{-}{-}{-}{-}{-}{-}{-}{-}{-}{-}{-}{-}{-}+    +{-}{-}{-}{-}{-}{-}{-}{-}{-}{-}{-}{-}{-}{-}{-}{-}{-}{-}{-}+}
|     Block 1       |    |     Block 2       |    |     Block 3       |
|{-{-}{-}{-}{-}{-}{-}{-}{-}{-}{-}{-}{-}{-}{-}{-}{-}{-}{-}|    |{-}{-}{-}{-}{-}{-}{-}{-}{-}{-}{-}{-}{-}{-}{-}{-}{-}{-}{-}|    |{-}{-}{-}{-}{-}{-}{-}{-}{-}{-}{-}{-}{-}{-}{-}{-}{-}{-}{-}|}
| Previous Hash: 0  |{-{-}{-}| Previous Hash: H1 |{-}{-}{-}| Previous Hash: H2 |}
| Merkle Root: MR1  |    | Merkle Root: MR2  |    | Merkle Root: MR3  |
| Timestamp: T1     |    | Timestamp: T2     |    | Timestamp: T3     |
| Nonce: N1         |    | Nonce: N2         |    | Nonce: N3         |
| Transactions: TX1 |    | Transactions: TX2 |    | Transactions: TX3 |
+{-{-}{-}{-}{-}{-}{-}{-}{-}{-}{-}{-}{-}{-}{-}{-}{-}{-}{-}+    +{-}{-}{-}{-}{-}{-}{-}{-}{-}{-}{-}{-}{-}{-}{-}{-}{-}{-}{-}+    +{-}{-}{-}{-}{-}{-}{-}{-}{-}{-}{-}{-}{-}{-}{-}{-}{-}{-}{-}+}
\end{verbatim}

\textbf{Key Components}:

\begin{itemize}
\tightlist
\item
  \textbf{Previous Hash}: Links blocks together creating chain
\item
  \textbf{Merkle Root}: Summary of all transactions in block
\item
  \textbf{Timestamp}: When block was created
\item
  \textbf{Nonce}: Number used once for proof-of-work
\end{itemize}

\end{solutionbox}
\begin{mnemonicbox}
``P-M-T-N'' (Previous, Merkle, Time, Nonce)

\end{mnemonicbox}
\subsection*{Question 2(c) [7 marks]}\label{q2c}

\textbf{Explain the core components of blockchain with suitable
diagrams.}

\begin{solutionbox}


{\def\LTcaptype{none} % do not increment counter
\vspace{-5pt}
\captionof{table}{Core Components of Blockchain}
\vspace{-10pt}
\begin{longtable}[]{@{}
  >{\raggedright\arraybackslash}p{(\linewidth - 4\tabcolsep) * \real{0.3667}}
  >{\raggedright\arraybackslash}p{(\linewidth - 4\tabcolsep) * \real{0.3333}}
  >{\raggedright\arraybackslash}p{(\linewidth - 4\tabcolsep) * \real{0.3000}}@{}}
\toprule\noalign{}
\begin{minipage}[b]{\linewidth}\raggedright
Component
\end{minipage} & \begin{minipage}[b]{\linewidth}\raggedright
Function
\end{minipage} & \begin{minipage}[b]{\linewidth}\raggedright
Purpose
\end{minipage} \\
\midrule\noalign{}
\endhead
\bottomrule\noalign{}
\endlastfoot
\textbf{Blocks} & Data containers & Store transaction information \\
\textbf{Hash Functions} & Create digital fingerprints & Ensure data
integrity \\
\textbf{Merkle Trees} & Transaction summaries & Efficient
verification \\
\textbf{Consensus Mechanism} & Agreement protocol & Validate new
blocks \\
\textbf{Digital Signatures} & Identity verification & Authenticate
transactions \\
\end{longtable}
}

\textbf{Diagram: Merkle Tree Structure}

\begin{center}
\textbf{Mermaid Diagram (Code)}
\begin{verbatim}
{Shaded}
{Highlighting}[]
graph TD
    A[Root Hash] {-{-}{} B[Hash AB]}
    A {-{-}{} C[Hash CD]}
    B {-{-}{} D[Hash A]}
    B {-{-}{} E[Hash B]}
    C {-{-}{} F[Hash C]}
    C {-{-}{} G[Hash D]}
    D {-{-}{} H[Transaction A]}
    E {-{-}{} I[Transaction B]}
    F {-{-}{} J[Transaction C]}
    G {-{-}{} K[Transaction D]}
{Highlighting}
{Shaded}
\end{verbatim}
\end{center}

\textbf{Key Points}:

\begin{itemize}
\tightlist
\item
  \textbf{Immutability}: Hash functions make tampering detectable
\item
  \textbf{Efficiency}: Merkle trees allow fast verification
\item
  \textbf{Decentralization}: Consensus mechanisms eliminate central
  authority
\end{itemize}

\end{solutionbox}
\begin{mnemonicbox}
``B-H-M-C-D'' (Blocks, Hash, Merkle, Consensus,
Digital)

\end{mnemonicbox}
\subsection*{Question 2(a) OR [3
marks]}\label{q2a}

\textbf{Define and explain permissioned blockchain in detail.}

\begin{solutionbox}

\textbf{Definition}: A blockchain where participation requires explicit
permission from a governing authority.


{\def\LTcaptype{none} % do not increment counter
\vspace{-5pt}
\captionof{table}{Permissioned Blockchain Features}
\vspace{-10pt}
\begin{longtable}[]{@{}ll@{}}
\toprule\noalign{}
Feature & Description \\
\midrule\noalign{}
\endhead
\bottomrule\noalign{}
\endlastfoot
\textbf{Restricted Access} & Only authorized users can participate \\
\textbf{Private Network} & Controlled membership \\
\textbf{Centralized Control} & Governing body manages permissions \\
\end{longtable}
}

\textbf{Key Characteristics}:

\begin{itemize}
\tightlist
\item
  \textbf{Privacy}: Enhanced confidentiality for sensitive data
\item
  \textbf{Performance}: Faster transactions due to fewer validators
\item
  \textbf{Examples}: Hyperledger Fabric, R3 Corda
\end{itemize}

\end{solutionbox}
\begin{mnemonicbox}
``Restricted-Private-Centralized'' (RPC)

\end{mnemonicbox}
\subsection*{Question 2(b) OR [4
marks]}\label{q2b}

\textbf{Explain types of wallets in the context of blockchain. Also
discuss the factors to be considered while selecting wallet for the
specific need.}

\begin{solutionbox}


{\def\LTcaptype{none} % do not increment counter
\vspace{-5pt}
\captionof{table}{Types of Blockchain Wallets}
\vspace{-10pt}
\begin{longtable}[]{@{}lll@{}}
\toprule\noalign{}
Wallet Type & Description & Security Level \\
\midrule\noalign{}
\endhead
\bottomrule\noalign{}
\endlastfoot
\textbf{Hot Wallets} & Connected to internet & Medium \\
\textbf{Cold Wallets} & Offline storage & High \\
\textbf{Hardware Wallets} & Physical devices & Very High \\
\textbf{Paper Wallets} & Printed keys & High (if stored safely) \\
\end{longtable}
}

\textbf{Selection Factors}:

\begin{itemize}
\tightlist
\item
  \textbf{Security Requirements}: Higher value needs better security
\item
  \textbf{Frequency of Use}: Regular use favors hot wallets
\item
  \textbf{Technical Expertise}: Simple wallets for beginners
\end{itemize}

\end{solutionbox}
\begin{mnemonicbox}
``H-C-H-P'' (Hot, Cold, Hardware, Paper)

\end{mnemonicbox}
\subsection*{Question 2(c) OR [7
marks]}\label{q2c}

\textbf{Explain sidechain in detail with suitable diagrams.}

\begin{solutionbox}

\textbf{Definition}: A separate blockchain that is attached to a parent
blockchain using a two-way peg.

\textbf{Diagram: Sidechain Architecture}

\begin{center}
\textbf{Mermaid Diagram (Code)}
\begin{verbatim}
{Shaded}
{Highlighting}[]
graph LR
    A[Main Chain] {{-}{-}{} B[Two{-}Way Peg]}
    B {{-}{-}{} C[Sidechain]}
    A {-{-}{} D[Bitcoin/Ethereum]}
    C {-{-}{} E[Specialized Functions]}
    E {-{-}{} F[Smart Contracts]}
    E {-{-}{} G[Privacy Features]}
    E {-{-}{} H[Faster Transactions]}
{Highlighting}
{Shaded}
\end{verbatim}
\end{center}


{\def\LTcaptype{none} % do not increment counter
\vspace{-5pt}
\captionof{table}{Sidechain Benefits}
\vspace{-10pt}
\begin{longtable}[]{@{}ll@{}}
\toprule\noalign{}
Benefit & Description \\
\midrule\noalign{}
\endhead
\bottomrule\noalign{}
\endlastfoot
\textbf{Scalability} & Reduces load on main chain \\
\textbf{Experimentation} & Test new features safely \\
\textbf{Specialized Functions} & Custom applications \\
\textbf{Interoperability} & Connect different blockchains \\
\end{longtable}
}

\textbf{Key Mechanisms}:

\begin{itemize}
\tightlist
\item
  \textbf{Two-Way Peg}: Allows asset transfer between chains
\item
  \textbf{SPV Proofs}: Simplified payment verification
\item
  \textbf{Federated Control}: Multiple parties manage transfers
\end{itemize}

\end{solutionbox}
\begin{mnemonicbox}
``S-E-S-I'' (Scalability, Experimentation,
Specialized, Interoperability)

\end{mnemonicbox}
\subsection*{Question 3(a) [3 marks]}\label{q3a}

\textbf{With respect to transaction in a blockchain network, define the
terms ``Confirmation'' and ``Finality''.}

\begin{solutionbox}


{\def\LTcaptype{none} % do not increment counter
\vspace{-5pt}
\captionof{table}{Transaction States}
\vspace{-10pt}
\begin{longtable}[]{@{}
  >{\raggedright\arraybackslash}p{(\linewidth - 2\tabcolsep) * \real{0.3333}}
  >{\raggedright\arraybackslash}p{(\linewidth - 2\tabcolsep) * \real{0.6667}}@{}}
\toprule\noalign{}
\begin{minipage}[b]{\linewidth}\raggedright
Term
\end{minipage} & \begin{minipage}[b]{\linewidth}\raggedright
Definition
\end{minipage} \\
\midrule\noalign{}
\endhead
\bottomrule\noalign{}
\endlastfoot
\textbf{Confirmation} & Number of blocks built on top of transaction
block \\
\textbf{Finality} & Point where transaction becomes irreversible \\
\end{longtable}
}

\textbf{Key Points}:

\begin{itemize}
\tightlist
\item
  \textbf{Confirmation Count}: More confirmations = higher security
\item
  \textbf{Bitcoin Standard}: 6 confirmations for high-value transactions
\item
  \textbf{Finality Types}: Probabilistic (Bitcoin) vs Absolute (some PoS
  systems)
\end{itemize}

\end{solutionbox}
\begin{mnemonicbox}
``Count-Blocks-Security'' for Confirmation,
``Irreversible-Point'' for Finality

\end{mnemonicbox}
\subsection*{Question 3(b) [4 marks]}\label{q3b}

\textbf{Differentiate Proof of Work and Proof of Stake.}

\begin{solutionbox}


{\def\LTcaptype{none} % do not increment counter
\vspace{-5pt}
\captionof{table}{PoW vs PoS Comparison}
\vspace{-10pt}
\begin{longtable}[]{@{}lll@{}}
\toprule\noalign{}
Aspect & Proof of Work (PoW) & Proof of Stake (PoS) \\
\midrule\noalign{}
\endhead
\bottomrule\noalign{}
\endlastfoot
\textbf{Resource} & Computational power & Stake ownership \\
\textbf{Energy Use} & High & Low \\
\textbf{Security} & Hash rate dependent & Stake dependent \\
\textbf{Rewards} & Mining rewards & Staking rewards \\
\textbf{Examples} & Bitcoin, Ethereum (old) & Ethereum 2.0, Cardano \\
\end{longtable}
}

\textbf{Key Differences}:

\begin{itemize}
\tightlist
\item
  \textbf{Mechanism}: PoW uses mining, PoS uses validators
\item
  \textbf{Environmental Impact}: PoS is more eco-friendly
\item
  \textbf{Barriers to Entry}: PoS requires initial stake, PoW needs
  hardware
\end{itemize}

\end{solutionbox}
\begin{mnemonicbox}
``Work-vs-Stake'' (Computational Work vs Financial
Stake)

\end{mnemonicbox}
\subsection*{Question 3(c) [7 marks]}\label{q3c}

\textbf{With respect to blockchain network, explain 51\% attack.}

\begin{solutionbox}

\textbf{Definition}: An attack where a single entity controls more than
50\% of the network's mining power or stake.

\textbf{Diagram: 51\% Attack Scenario}

\begin{center}
\textbf{Mermaid Diagram (Code)}
\begin{verbatim}
{Shaded}
{Highlighting}[]
graph LR
    A[Network Hash Rate] {-{-}{} B[Honest Miners 49\%]}
    A {-{-}{} C[Attacker 51\%]}
    C {-{-}{} D[Can Create Longer Chain]}
    D {-{-}{} E[Double Spending]}
    D {-{-}{} F[Transaction Reversal]}
    D {-{-}{} G[Block Withholding]}
{Highlighting}
{Shaded}
\end{verbatim}
\end{center}


{\def\LTcaptype{none} % do not increment counter
\vspace{-5pt}
\captionof{table}{Attack Capabilities and Limitations}
\vspace{-10pt}
\begin{longtable}[]{@{}ll@{}}
\toprule\noalign{}
Can Do & Cannot Do \\
\midrule\noalign{}
\endhead
\bottomrule\noalign{}
\endlastfoot
Double spend own coins & Steal others' coins \\
Reverse recent transactions & Create coins from nothing \\
Block specific transactions & Change consensus rules \\
Fork the blockchain & Access private keys \\
\end{longtable}
}

\textbf{Prevention Measures}:

\begin{itemize}
\tightlist
\item
  \textbf{Diversified Mining}: Encourage multiple mining pools
\item
  \textbf{Checkpoint Systems}: Periodic finality markers
\item
  \textbf{Economic Incentives}: Make attacks unprofitable
\end{itemize}

\textbf{Impact}:

\begin{itemize}
\tightlist
\item
  \textbf{Network Disruption}: Temporary service interruption
\item
  \textbf{Economic Loss}: Reduced trust and value
\item
  \textbf{Recovery}: Network usually recovers after attack ends
\end{itemize}

\end{solutionbox}
\begin{mnemonicbox}
``Majority-Control-Attack'' (51\% = Majority Control
= Attack Power)

\end{mnemonicbox}
\subsection*{Question 3(a) OR [3
marks]}\label{q3a}

\textbf{Define the terms ``Hard fork'' and ``Soft fork''}

\begin{solutionbox}


{\def\LTcaptype{none} % do not increment counter
\vspace{-5pt}
\captionof{table}{Fork Types}
\vspace{-10pt}
\begin{longtable}[]{@{}
  >{\raggedright\arraybackslash}p{(\linewidth - 4\tabcolsep) * \real{0.2895}}
  >{\raggedright\arraybackslash}p{(\linewidth - 4\tabcolsep) * \real{0.3158}}
  >{\raggedright\arraybackslash}p{(\linewidth - 4\tabcolsep) * \real{0.3947}}@{}}
\toprule\noalign{}
\begin{minipage}[b]{\linewidth}\raggedright
Fork Type
\end{minipage} & \begin{minipage}[b]{\linewidth}\raggedright
Definition
\end{minipage} & \begin{minipage}[b]{\linewidth}\raggedright
Compatibility
\end{minipage} \\
\midrule\noalign{}
\endhead
\bottomrule\noalign{}
\endlastfoot
\textbf{Hard Fork} & Non-backward compatible protocol change & Not
compatible \\
\textbf{Soft Fork} & Backward compatible protocol change & Compatible \\
\end{longtable}
}

\textbf{Key Characteristics}:

\begin{itemize}
\tightlist
\item
  \textbf{Hard Fork}: Creates new blockchain branch, requires all nodes
  to upgrade
\item
  \textbf{Soft Fork}: Tightens rules, old nodes can still operate
\end{itemize}

\textbf{Examples}:

\begin{itemize}
\tightlist
\item
  \textbf{Hard Fork}: Bitcoin Cash split from Bitcoin
\item
  \textbf{Soft Fork}: SegWit activation in Bitcoin
\end{itemize}

\end{solutionbox}
\begin{mnemonicbox}
``Hard-Breaks-Compatibility'' vs
``Soft-Keeps-Compatibility''

\end{mnemonicbox}
\subsection*{Question 3(b) OR [4
marks]}\label{q3b}

\textbf{List various types of consensus mechanisms and explain any one
in detail.}

\begin{solutionbox}


{\def\LTcaptype{none} % do not increment counter
\vspace{-5pt}
\captionof{table}{Consensus Mechanisms}
\vspace{-10pt}
\begin{longtable}[]{@{}lll@{}}
\toprule\noalign{}
Mechanism & Description & Energy Use \\
\midrule\noalign{}
\endhead
\bottomrule\noalign{}
\endlastfoot
\textbf{Proof of Work} & Computational puzzle solving & High \\
\textbf{Proof of Stake} & Stake-based validation & Low \\
\textbf{Delegated PoS} & Voted representatives validate & Very Low \\
\textbf{Proof of Authority} & Pre-approved validators & Minimal \\
\end{longtable}
}

\textbf{Detailed Explanation - Proof of Stake (PoS)}:

\textbf{Process}:

\begin{itemize}
\tightlist
\item
  \textbf{Validator Selection}: Based on stake amount and randomization
\item
  \textbf{Block Creation}: Selected validator proposes new block
\item
  \textbf{Validation}: Other validators verify and attest to block
\item
  \textbf{Rewards}: Validators earn fees and new tokens
\end{itemize}

\textbf{Advantages}: Lower energy consumption, reduced centralization
risk \textbf{Disadvantages}: ``Nothing at stake'' problem, initial
distribution issues

\end{solutionbox}
\begin{mnemonicbox}
``Stake-Select-Validate-Reward'' (PoS Process)

\end{mnemonicbox}
\subsection*{Question 3(c) OR [7
marks]}\label{q3c}

\textbf{With respect to blockchain network, explain sybil attack.}

\begin{solutionbox}

\textbf{Definition}: An attack where a single adversary creates multiple
fake identities to gain disproportionate influence in the network.

\textbf{Diagram: Sybil Attack Structure}

\begin{center}
\textbf{Mermaid Diagram (Code)}
\begin{verbatim}
{Shaded}
{Highlighting}[]
graph TD
    A[Attacker] {-{-}{} B[Fake Identity 1]}
    A {-{-}{} C[Fake Identity 2]}
    A {-{-}{} D[Fake Identity 3]}
    A {-{-}{} E[Fake Identity N]}
    B {-{-}{} F[Network Influence]}
    C {-{-}{} F}
    D {-{-}{} F}
    E {-{-}{} F}
    F {-{-}{} G[Consensus Manipulation]}
{Highlighting}
{Shaded}
\end{verbatim}
\end{center}


{\def\LTcaptype{none} % do not increment counter
\vspace{-5pt}
\captionof{table}{Attack Methods and Defenses}
\vspace{-10pt}
\begin{longtable}[]{@{}
  >{\raggedright\arraybackslash}p{(\linewidth - 4\tabcolsep) * \real{0.4054}}
  >{\raggedright\arraybackslash}p{(\linewidth - 4\tabcolsep) * \real{0.3514}}
  >{\raggedright\arraybackslash}p{(\linewidth - 4\tabcolsep) * \real{0.2432}}@{}}
\toprule\noalign{}
\begin{minipage}[b]{\linewidth}\raggedright
Attack Method
\end{minipage} & \begin{minipage}[b]{\linewidth}\raggedright
Description
\end{minipage} & \begin{minipage}[b]{\linewidth}\raggedright
Defense
\end{minipage} \\
\midrule\noalign{}
\endhead
\bottomrule\noalign{}
\endlastfoot
\textbf{Identity Flooding} & Create many fake nodes & Proof of
Work/Stake \\
\textbf{Routing Manipulation} & Control network paths & Reputation
systems \\
\textbf{Consensus Disruption} & Influence voting & Resource
requirements \\
\end{longtable}
}

\textbf{Impact on Blockchain}:

\begin{itemize}
\tightlist
\item
  \textbf{Network Partitioning}: Isolate honest nodes
\item
  \textbf{Double Spending}: Facilitate fraudulent transactions
\item
  \textbf{Consensus Failure}: Prevent network agreement
\end{itemize}

\textbf{Prevention Mechanisms}:

\begin{itemize}
\tightlist
\item
  \textbf{Resource Requirements}: PoW/PoS make attacks expensive
\item
  \textbf{Identity Verification}: KYC/AML procedures
\item
  \textbf{Network Monitoring}: Detect suspicious behavior patterns
\item
  \textbf{Reputation Systems}: Track node behavior over time
\end{itemize}

\textbf{Real-world Examples}:

\begin{itemize}
\tightlist
\item
  \textbf{P2P Networks}: BitTorrent, Gnutella vulnerabilities
\item
  \textbf{Social Networks}: Fake account creation
\item
  \textbf{Blockchain}: Potential threat to permissionless networks
\end{itemize}

\end{solutionbox}
\begin{mnemonicbox}
``Single-Multiple-Influence'' (Single Attacker,
Multiple Identities, Network Influence)

\end{mnemonicbox}
\subsection*{Question 4(a) [3 marks]}\label{q4a}

\textbf{Define the terms ``Merkle Tree'' and ``Hyperledger''.}

\begin{solutionbox}


{\def\LTcaptype{none} % do not increment counter
\vspace{-5pt}
\captionof{table}{Key Definitions}
\vspace{-10pt}
\begin{longtable}[]{@{}
  >{\raggedright\arraybackslash}p{(\linewidth - 2\tabcolsep) * \real{0.3333}}
  >{\raggedright\arraybackslash}p{(\linewidth - 2\tabcolsep) * \real{0.6667}}@{}}
\toprule\noalign{}
\begin{minipage}[b]{\linewidth}\raggedright
Term
\end{minipage} & \begin{minipage}[b]{\linewidth}\raggedright
Definition
\end{minipage} \\
\midrule\noalign{}
\endhead
\bottomrule\noalign{}
\endlastfoot
\textbf{Merkle Tree} & Binary tree of hashes that efficiently summarizes
all transactions \\
\textbf{Hyperledger} & Open-source blockchain platform hosted by Linux
Foundation \\
\end{longtable}
}

\textbf{Key Features}:

\begin{itemize}
\tightlist
\item
  \textbf{Merkle Tree}: Enables efficient verification without
  downloading full blockchain
\item
  \textbf{Hyperledger}: Enterprise-focused, modular architecture,
  multiple frameworks
\end{itemize}

\end{solutionbox}
\begin{mnemonicbox}
``Tree-Hash-Efficient'' for Merkle,
``Enterprise-Modular-Linux'' for Hyperledger

\end{mnemonicbox}
\subsection*{Question 4(b) [4 marks]}\label{q4b}

\textbf{Explain classic Byzantine generals problem in detail.}

\begin{solutionbox}

\textbf{Scenario}: Multiple generals must coordinate attack on a city,
but some may be traitors.


{\def\LTcaptype{none} % do not increment counter
\vspace{-5pt}
\captionof{table}{Problem Components}
\vspace{-10pt}
\begin{longtable}[]{@{}ll@{}}
\toprule\noalign{}
Component & Description \\
\midrule\noalign{}
\endhead
\bottomrule\noalign{}
\endlastfoot
\textbf{Generals} & Network nodes/participants \\
\textbf{Messages} & Transactions/communications \\
\textbf{Traitors} & Malicious/faulty nodes \\
\textbf{Consensus} & Agreement on action \\
\end{longtable}
}

\textbf{Solution Requirements}:

\begin{itemize}
\tightlist
\item
  \textbf{Agreement}: All honest generals decide on same action
\item
  \textbf{Validity}: If all honest generals want to attack, they should
  attack
\item
  \textbf{Termination}: Decision must be reached in finite time
\end{itemize}

\textbf{Blockchain Relevance}: Ensures network agreement despite
malicious nodes

\end{solutionbox}
\begin{mnemonicbox}
``Generals-Messages-Traitors-Consensus'' (GMTC)

\end{mnemonicbox}
\subsection*{Question 4(c) [7 marks]}\label{q4c}

\textbf{Explain the process of Merkle tree creation with suitable
example and supporting diagrams.}

\begin{solutionbox}

\textbf{Process Steps}:

\begin{enumerate}
\tightlist
\item
  Hash each transaction individually
\item
  Pair hashes and hash the pairs
\item
  Continue until single root hash remains
\end{enumerate}

\textbf{Example: 4 Transactions}

\begin{verbatim}
                    Root Hash
                   /           {}
              Hash(AB)         Hash(CD)
             /        {       /        }
        Hash(A)    Hash(B) Hash(C)  Hash(D)
           |          |       |        |
         Tx A       Tx B    Tx C     Tx D
\end{verbatim}


{\def\LTcaptype{none} % do not increment counter
\vspace{-5pt}
\captionof{table}{Merkle Tree Benefits}
\vspace{-10pt}
\begin{longtable}[]{@{}ll@{}}
\toprule\noalign{}
Benefit & Description \\
\midrule\noalign{}
\endhead
\bottomrule\noalign{}
\endlastfoot
\textbf{Efficiency} & Verify transactions without full data \\
\textbf{Security} & Any change affects root hash \\
\textbf{Scalability} & Log(n) verification complexity \\
\end{longtable}
}

\textbf{Verification Process}:

\begin{itemize}
\tightlist
\item
  To verify Tx A: Need Hash(B), Hash(CD), and Root Hash
\item
  Path verification: Hash(A) + Hash(B) = Hash(AB)
\item
  Hash(AB) + Hash(CD) = Root Hash
\end{itemize}

\textbf{Applications}:

\begin{itemize}
\tightlist
\item
  \textbf{Bitcoin}: Block headers contain Merkle root
\item
  \textbf{SPV Clients}: Light wallets use Merkle proofs
\item
  \textbf{Git}: Version control system uses similar structure
\end{itemize}

\end{solutionbox}
\begin{mnemonicbox}
``Hash-Pair-Repeat-Root'' (Merkle Tree Creation
Process)

\end{mnemonicbox}
\subsection*{Question 4(a) OR [3
marks]}\label{q4a}

\textbf{List various types of Hyperledger projects.}

\begin{solutionbox}


{\def\LTcaptype{none} % do not increment counter
\vspace{-5pt}
\captionof{table}{Hyperledger Projects}
\vspace{-10pt}
\begin{longtable}[]{@{}lll@{}}
\toprule\noalign{}
Project & Type & Purpose \\
\midrule\noalign{}
\endhead
\bottomrule\noalign{}
\endlastfoot
\textbf{Fabric} & Framework & Permissioned blockchain platform \\
\textbf{Sawtooth} & Framework & Modular blockchain suite \\
\textbf{Iroha} & Framework & Simple blockchain for mobile/web \\
\textbf{Burrow} & Framework & Ethereum Virtual Machine \\
\textbf{Caliper} & Tool & Blockchain performance benchmark \\
\textbf{Composer} & Tool & Business network development \\
\end{longtable}
}

\textbf{Categories}:

\begin{itemize}
\tightlist
\item
  \textbf{Frameworks}: Core blockchain platforms
\item
  \textbf{Tools}: Development and testing utilities
\end{itemize}

\end{solutionbox}
\begin{mnemonicbox}
``F-S-I-B-C-C'' (Fabric, Sawtooth, Iroha, Burrow,
Caliper, Composer)

\end{mnemonicbox}
\subsection*{Question 4(b) OR [4
marks]}\label{q4b}

\textbf{Explain Practical Byzantine Fault Tolerance algorithm in
detail.}

\begin{solutionbox}

\textbf{Definition}: Consensus algorithm that works correctly even when
up to 1/3 of nodes are faulty or malicious.


{\def\LTcaptype{none} % do not increment counter
\vspace{-5pt}
\captionof{table}{PBFT Phases}
\vspace{-10pt}
\begin{longtable}[]{@{}lll@{}}
\toprule\noalign{}
Phase & Description & Purpose \\
\midrule\noalign{}
\endhead
\bottomrule\noalign{}
\endlastfoot
\textbf{Pre-prepare} & Primary broadcasts request & Initiate
consensus \\
\textbf{Prepare} & Nodes validate and broadcast & Verify proposal \\
\textbf{Commit} & Nodes commit to decision & Finalize agreement \\
\end{longtable}
}

\textbf{Algorithm Steps}:

\begin{enumerate}
\tightlist
\item
  Client sends request to primary replica
\item
  Primary broadcasts pre-prepare message
\item
  Backups send prepare messages if valid
\item
  After receiving 2f+1 prepares, send commit
\item
  Execute after receiving 2f+1 commits
\end{enumerate}

\textbf{Key Properties}:

\begin{itemize}
\tightlist
\item
  \textbf{Safety}: Never produces inconsistent results
\item
  \textbf{Liveness}: Eventually produces results
\item
  \textbf{Fault Tolerance}: Works with f \textless{} n/3 faulty nodes
\end{itemize}

\end{solutionbox}
\begin{mnemonicbox}
``Pre-Prepare-Commit'' (3 Phases of PBFT)

\end{mnemonicbox}
\subsection*{Question 4(c) OR [7
marks]}\label{q4c}

\textbf{``Eventual consistency is evident in the context of bitcoin.''
Justify the given statement.}

\begin{solutionbox}

\textbf{Definition}: Eventual consistency means the system will become
consistent over time, even if it's temporarily inconsistent.

\textbf{Bitcoin Implementation}:


{\def\LTcaptype{none} % do not increment counter
\vspace{-5pt}
\captionof{table}{Bitcoin Consistency Mechanisms}
\vspace{-10pt}
\begin{longtable}[]{@{}
  >{\raggedright\arraybackslash}p{(\linewidth - 4\tabcolsep) * \real{0.3333}}
  >{\raggedright\arraybackslash}p{(\linewidth - 4\tabcolsep) * \real{0.3939}}
  >{\raggedright\arraybackslash}p{(\linewidth - 4\tabcolsep) * \real{0.2727}}@{}}
\toprule\noalign{}
\begin{minipage}[b]{\linewidth}\raggedright
Mechanism
\end{minipage} & \begin{minipage}[b]{\linewidth}\raggedright
Description
\end{minipage} & \begin{minipage}[b]{\linewidth}\raggedright
Purpose
\end{minipage} \\
\midrule\noalign{}
\endhead
\bottomrule\noalign{}
\endlastfoot
\textbf{Chain Reorganization} & Replace shorter chain with longer &
Maintain consensus \\
\textbf{Confirmation Delays} & Wait for multiple blocks & Increase
certainty \\
\textbf{Fork Resolution} & Longest chain wins & Resolve conflicts \\
\end{longtable}
}

\textbf{Scenarios Demonstrating Eventual Consistency}:

\begin{enumerate}
\tightlist
\item
  \textbf{Temporary Forks}: When two miners find blocks simultaneously
\item
  \textbf{Network Partitions}: Isolated nodes may have different views
\item
  \textbf{Double Spending Attempts}: Conflicting transactions in
  different blocks
\end{enumerate}

\textbf{Resolution Process}:

\begin{itemize}
\tightlist
\item
  \textbf{Mining Continues}: Miners build on their preferred chain
\item
  \textbf{Longest Chain Rule}: Network adopts chain with most work
\item
  \textbf{Automatic Convergence}: All nodes eventually agree
\end{itemize}

\textbf{Diagram: Fork Resolution}

\begin{center}
\textbf{Mermaid Diagram (Code)}
\begin{verbatim}
{Shaded}
{Highlighting}[]
graph LR
    A[Block N] {-{-}{} B[Block N+1a]}
    A {-{-}{} C[Block N+1b]}
    B {-{-}{} D[Block N+2a]}
    C {-{-}{} E[Dies {-} Shorter Chain]}
    D {-{-}{} F[Becomes Main Chain]}
{Highlighting}
{Shaded}
\end{verbatim}
\end{center}

\textbf{Justification Points}:

\begin{itemize}
\tightlist
\item
  \textbf{Probabilistic Finality}: Longer confirmation time = higher
  certainty
\item
  \textbf{No Immediate Consistency}: New transactions aren't instantly
  final
\item
  \textbf{Convergence Guarantee}: Network will eventually agree on
  single chain
\item
  \textbf{Time-based Resolution}: Consistency improves with time
\end{itemize}

\textbf{Practical Implications}:

\begin{itemize}
\tightlist
\item
  \textbf{Merchant Waiting}: Wait for confirmations before accepting
  payment
\item
  \textbf{Exchange Policies}: Different confirmation requirements for
  different amounts
\item
  \textbf{Risk Management}: Balance speed vs security based on
  transaction value
\end{itemize}

\end{solutionbox}
\begin{mnemonicbox}
``Time-Brings-Consistency'' (Eventual Consistency =
Time + Convergence)

\end{mnemonicbox}
\subsection*{Question 5(a) [3 marks]}\label{q5a}

\textbf{Explain advantages of ERC 20.}

\begin{solutionbox}


{\def\LTcaptype{none} % do not increment counter
\vspace{-5pt}
\captionof{table}{ERC-20 Token Advantages}
\vspace{-10pt}
\begin{longtable}[]{@{}ll@{}}
\toprule\noalign{}
Advantage & Description \\
\midrule\noalign{}
\endhead
\bottomrule\noalign{}
\endlastfoot
\textbf{Standardization} & Common interface for all tokens \\
\textbf{Interoperability} & Works with all Ethereum wallets/exchanges \\
\textbf{Liquidity} & Easy trading and exchange \\
\end{longtable}
}

\textbf{Key Benefits}:

\begin{itemize}
\tightlist
\item
  \textbf{Developer Friendly}: Simple implementation standard
\item
  \textbf{Market Adoption}: Widely supported across platforms
\item
  \textbf{Smart Contract Integration}: Easy DeFi integration
\end{itemize}

\end{solutionbox}
\begin{mnemonicbox}
``Standard-Interoperable-Liquid'' (SIL)

\end{mnemonicbox}
\subsection*{Question 5(b) [4 marks]}\label{q5b}

\textbf{Describe working mechanism of a smart-contract in detail.}

\begin{solutionbox}


{\def\LTcaptype{none} % do not increment counter
\vspace{-5pt}
\captionof{table}{Smart Contract Workflow}
\vspace{-10pt}
\begin{longtable}[]{@{}ll@{}}
\toprule\noalign{}
Step & Description \\
\midrule\noalign{}
\endhead
\bottomrule\noalign{}
\endlastfoot
\textbf{Code Deployment} & Contract uploaded to blockchain \\
\textbf{Trigger Conditions} & Predefined conditions monitored \\
\textbf{Automatic Execution} & Contract executes when conditions met \\
\textbf{State Update} & Blockchain state modified \\
\end{longtable}
}

\textbf{Working Process}:

\begin{enumerate}
\tightlist
\item
  \textbf{Development}: Write contract in Solidity/Vyper
\item
  \textbf{Compilation}: Convert to bytecode
\item
  \textbf{Deployment}: Upload to blockchain network
\item
  \textbf{Execution}: Triggered by transactions or events
\end{enumerate}

\end{solutionbox}
\begin{mnemonicbox}
``Deploy-Trigger-Execute-Update'' (DTEU)

\end{mnemonicbox}
\subsection*{Question 5(c) [7 marks]}\label{q5c}

\textbf{What is smart-contract? Explain features and applications of
smart-contract in detail.}

\begin{solutionbox}

\textbf{Definition}: Self-executing contracts with terms directly
written into code, running on blockchain.


{\def\LTcaptype{none} % do not increment counter
\vspace{-5pt}
\captionof{table}{Smart Contract Features}
\vspace{-10pt}
\begin{longtable}[]{@{}lll@{}}
\toprule\noalign{}
Feature & Description & Benefit \\
\midrule\noalign{}
\endhead
\bottomrule\noalign{}
\endlastfoot
\textbf{Autonomous} & Executes without intermediaries & Cost
reduction \\
\textbf{Transparent} & Code visible on blockchain & Trust building \\
\textbf{Immutable} & Cannot be changed once deployed & Security \\
\textbf{Deterministic} & Same input produces same output &
Predictability \\
\end{longtable}
}

\textbf{Diagram: Smart Contract Architecture}

\begin{center}
\textbf{Mermaid Diagram (Code)}
\begin{verbatim}
{Shaded}
{Highlighting}[]
graph LR
    A[Smart Contract] {-{-}{} B[Trigger Conditions]}
    B {-{-}{} C[Automatic Execution]}
    C {-{-}{} D[State Changes]}
    D {-{-}{} E[Event Emissions]}
    A {-{-}{} F[External Calls]}
    F {-{-}{} G[Other Contracts]}
{Highlighting}
{Shaded}
\end{verbatim}
\end{center}

\textbf{Applications}:


{\def\LTcaptype{none} % do not increment counter
\vspace{-5pt}
\captionof{table}{Smart Contract Applications}
\vspace{-10pt}
\begin{longtable}[]{@{}lll@{}}
\toprule\noalign{}
Domain & Use Case & Example \\
\midrule\noalign{}
\endhead
\bottomrule\noalign{}
\endlastfoot
\textbf{Finance} & Automated lending & DeFi protocols \\
\textbf{Insurance} & Claim processing & Flight delay insurance \\
\textbf{Supply Chain} & Product tracking & Food provenance \\
\textbf{Real Estate} & Property transfers & Automated escrow \\
\textbf{Gaming} & Digital assets & NFT marketplaces \\
\end{longtable}
}

\textbf{Advantages}:

\begin{itemize}
\tightlist
\item
  \textbf{Efficiency}: Reduced processing time and costs
\item
  \textbf{Trust}: No need for trusted third parties
\item
  \textbf{Accuracy}: Eliminates human errors
\item
  \textbf{Global Access}: Available 24/7 worldwide
\end{itemize}

\textbf{Limitations}:

\begin{itemize}
\tightlist
\item
  \textbf{Immutability}: Difficult to fix bugs after deployment
\item
  \textbf{Oracle Problem}: Need external data sources
\item
  \textbf{Gas Costs}: Execution costs can be high
\item
  \textbf{Complexity}: Requires technical expertise
\end{itemize}

\textbf{Development Considerations}:

\begin{itemize}
\tightlist
\item
  \textbf{Security Audits}: Essential before deployment
\item
  \textbf{Testing}: Extensive testing on testnets
\item
  \textbf{Upgradability}: Design patterns for updates
\item
  \textbf{Gas Optimization}: Minimize execution costs
\end{itemize}

\end{solutionbox}
\begin{mnemonicbox}
``Auto-Transparent-Immutable-Deterministic'' (ATID)
for features

\end{mnemonicbox}
\subsection*{Question 5(a) OR [3
marks]}\label{q5a}

\textbf{Explain disadvantages of ERC20.}

\begin{solutionbox}


{\def\LTcaptype{none} % do not increment counter
\vspace{-5pt}
\captionof{table}{ERC-20 Token Disadvantages}
\vspace{-10pt}
\begin{longtable}[]{@{}ll@{}}
\toprule\noalign{}
Disadvantage & Description \\
\midrule\noalign{}
\endhead
\bottomrule\noalign{}
\endlastfoot
\textbf{Limited Functionality} & Only basic token operations \\
\textbf{No Built-in Security} & Vulnerable to common attacks \\
\textbf{Gas Dependency} & Requires ETH for transactions \\
\end{longtable}
}

\textbf{Key Issues}:

\begin{itemize}
\tightlist
\item
  \textbf{Transfer Limitations}: Cannot handle complex transfers
\item
  \textbf{Approval Risks}: Double spending vulnerabilities
\item
  \textbf{Network Congestion}: High fees during peak times
\end{itemize}

\end{solutionbox}
\begin{mnemonicbox}
``Limited-Vulnerable-Dependent'' (LVD)

\end{mnemonicbox}
\subsection*{Question 5(b) OR [4
marks]}\label{q5b}

\textbf{Describe steps for Launching of a Decentralized Autonomous
Organization (DAO)?}

\begin{solutionbox}


{\def\LTcaptype{none} % do not increment counter
\vspace{-5pt}
\captionof{table}{DAO Launch Steps}
\vspace{-10pt}
\begin{longtable}[]{@{}ll@{}}
\toprule\noalign{}
Step & Description \\
\midrule\noalign{}
\endhead
\bottomrule\noalign{}
\endlastfoot
\textbf{Concept Design} & Define purpose and governance rules \\
\textbf{Smart Contract Development} & Code governance mechanisms \\
\textbf{Token Distribution} & Allocate voting rights \\
\textbf{Community Building} & Attract members and contributors \\
\end{longtable}
}

\textbf{Detailed Process}:

\begin{enumerate}
\tightlist
\item
  \textbf{Whitepaper Creation}: Document vision and tokenomics
\item
  \textbf{Technical Implementation}: Deploy governance contracts
\item
  \textbf{Initial Funding}: Raise capital through token sales
\item
  \textbf{Operations Launch}: Begin decentralized operations
\end{enumerate}

\end{solutionbox}
\begin{mnemonicbox}
``Design-Develop-Distribute-Deploy'' (4D Launch)

\end{mnemonicbox}
\subsection*{Question 5(c) OR [7
marks]}\label{q5c}

\textbf{What is Decentralized Autonomous Organization (DAO)? Explain its
advantages and disadvantages in detail.}

\begin{solutionbox}

\textbf{Definition}: A blockchain-based organization governed by smart
contracts and token holders rather than traditional management.


{\def\LTcaptype{none} % do not increment counter
\vspace{-5pt}
\captionof{table}{DAO Structure}
\vspace{-10pt}
\begin{longtable}[]{@{}
  >{\raggedright\arraybackslash}p{(\linewidth - 4\tabcolsep) * \real{0.3235}}
  >{\raggedright\arraybackslash}p{(\linewidth - 4\tabcolsep) * \real{0.3824}}
  >{\raggedright\arraybackslash}p{(\linewidth - 4\tabcolsep) * \real{0.2941}}@{}}
\toprule\noalign{}
\begin{minipage}[b]{\linewidth}\raggedright
Component
\end{minipage} & \begin{minipage}[b]{\linewidth}\raggedright
Description
\end{minipage} & \begin{minipage}[b]{\linewidth}\raggedright
Function
\end{minipage} \\
\midrule\noalign{}
\endhead
\bottomrule\noalign{}
\endlastfoot
\textbf{Smart Contracts} & Governance rules in code & Automated decision
execution \\
\textbf{Tokens} & Voting rights and ownership & Democratic
participation \\
\textbf{Proposals} & Suggested changes or actions & Community-driven
initiatives \\
\textbf{Treasury} & Shared funds & Resource allocation \\
\end{longtable}
}

\textbf{Diagram: DAO Governance Flow}

\begin{center}
\textbf{Mermaid Diagram (Code)}
\begin{verbatim}
{Shaded}
{Highlighting}[]
graph LR
    A[Token Holders] {-{-}{} B[Submit Proposals]}
    B {-{-}{} C[Community Discussion]}
    C {-{-}{} D[Voting Period]}
    D {-{-}{} E[Execution if Passed]}
    E {-{-}{} F[Smart Contract Updates]}
    F {-{-}{} G[Treasury Actions]}
{Highlighting}
{Shaded}
\end{verbatim}
\end{center}

\textbf{Advantages}:


{\def\LTcaptype{none} % do not increment counter
\vspace{-5pt}
\captionof{table}{DAO Benefits}
\vspace{-10pt}
\begin{longtable}[]{@{}
  >{\raggedright\arraybackslash}p{(\linewidth - 4\tabcolsep) * \real{0.3438}}
  >{\raggedright\arraybackslash}p{(\linewidth - 4\tabcolsep) * \real{0.4062}}
  >{\raggedright\arraybackslash}p{(\linewidth - 4\tabcolsep) * \real{0.2500}}@{}}
\toprule\noalign{}
\begin{minipage}[b]{\linewidth}\raggedright
Advantage
\end{minipage} & \begin{minipage}[b]{\linewidth}\raggedright
Description
\end{minipage} & \begin{minipage}[b]{\linewidth}\raggedright
Impact
\end{minipage} \\
\midrule\noalign{}
\endhead
\bottomrule\noalign{}
\endlastfoot
\textbf{Decentralization} & No single point of control & Reduced
corruption risk \\
\textbf{Transparency} & All decisions on blockchain & Enhanced
accountability \\
\textbf{Global Participation} & Anyone can join & Diverse
perspectives \\
\textbf{Efficiency} & Automated execution & Faster decision
implementation \\
\textbf{Democratic Governance} & Token-based voting & Fair
representation \\
\end{longtable}
}

\textbf{Disadvantages}:


{\def\LTcaptype{none} % do not increment counter
\vspace{-5pt}
\captionof{table}{DAO Challenges}
\vspace{-10pt}
\begin{longtable}[]{@{}
  >{\raggedright\arraybackslash}p{(\linewidth - 4\tabcolsep) * \real{0.4242}}
  >{\raggedright\arraybackslash}p{(\linewidth - 4\tabcolsep) * \real{0.3939}}
  >{\raggedright\arraybackslash}p{(\linewidth - 4\tabcolsep) * \real{0.1818}}@{}}
\toprule\noalign{}
\begin{minipage}[b]{\linewidth}\raggedright
Disadvantage
\end{minipage} & \begin{minipage}[b]{\linewidth}\raggedright
Description
\end{minipage} & \begin{minipage}[b]{\linewidth}\raggedright
Risk
\end{minipage} \\
\midrule\noalign{}
\endhead
\bottomrule\noalign{}
\endlastfoot
\textbf{Technical Complexity} & Smart contract bugs & System failures \\
\textbf{Legal Uncertainty} & Unclear regulatory status & Compliance
issues \\
\textbf{Coordination Problems} & Difficult decision making & Slow
progress \\
\textbf{Token Concentration} & Wealthy holders control votes &
Centralization risk \\
\textbf{Security Vulnerabilities} & Code exploits possible & Financial
losses \\
\end{longtable}
}

\textbf{Types of DAOs}:

\begin{itemize}
\tightlist
\item
  \textbf{Investment DAOs}: Collective investment decisions
\item
  \textbf{Protocol DAOs}: Blockchain protocol governance
\item
  \textbf{Social DAOs}: Community-driven organizations
\item
  \textbf{Collector DAOs}: NFT and art collecting
\end{itemize}

\textbf{Success Factors}:

\begin{itemize}
\tightlist
\item
  \textbf{Clear Purpose}: Well-defined mission and goals
\item
  \textbf{Robust Governance}: Effective voting mechanisms
\item
  \textbf{Community Engagement}: Active member participation
\item
  \textbf{Technical Security}: Audited smart contracts
\item
  \textbf{Legal Compliance}: Regulatory compliance where applicable
\end{itemize}

\textbf{Notable Examples}:

\begin{itemize}
\tightlist
\item
  \textbf{MakerDAO}: Decentralized finance protocol
\item
  \textbf{Uniswap}: Decentralized exchange governance
\item
  \textbf{Compound}: Money market protocol
\end{itemize}

\textbf{Future Outlook}:

\begin{itemize}
\tightlist
\item
  \textbf{Regulatory Clarity}: Evolving legal frameworks
\item
  \textbf{Technical Improvements}: Better governance tools
\item
  \textbf{Mainstream Adoption}: Growing corporate interest
\item
  \textbf{Integration}: Hybrid traditional-DAO models
\end{itemize}

\end{solutionbox}
\begin{mnemonicbox}
``Decentralized-Autonomous-Organization'' (DAO =
Democratic Automated Ownership)

\end{mnemonicbox}

\end{document}
