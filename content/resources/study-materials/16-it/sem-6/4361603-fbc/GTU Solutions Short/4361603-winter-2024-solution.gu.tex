\documentclass{article}

% content/resources/templates/preamble.tex
\usepackage[margin=0.6in]{geometry}
\author{Milav Dabgar}
\usepackage{amsmath,amssymb,amsthm}
\usepackage{booktabs}
\usepackage{multirow}
\usepackage{xcolor}
\usepackage{tcolorbox}
\tcbuselibrary{breakable,skins}
\usepackage[colorlinks=true,linkcolor=blue]{hyperref}
\usepackage{titlesec}
\usepackage{enumitem}
\usepackage{tikz}
\usepackage{pgfplots}
\usepackage{circuitikz}
\usepackage[version=4]{mhchem}
\usepackage{longtable}
\usepackage{array}
\usepackage{float}
\usepackage{caption}
\usepackage{listings}

\lstset{
  basicstyle=\small\ttfamily,
  breaklines=true,
  breakatwhitespace=false,
  postbreak=\mbox{\textcolor{red}{$\hookrightarrow$}\space},
  float=false,
  numbers=left,
  numberstyle=\tiny\color{gray},
  numbersep=10pt,
  xleftmargin=2em,
  keywordstyle=\color{blue},
  commentstyle=\color{green!60!black},
  stringstyle=\color{purple},
  backgroundcolor=\color{gray!5},
  showstringspaces=false,
  tabsize=2,
  captionpos=b,
  keepspaces=true,
  columns=flexible
}

\pgfplotsset{compat=1.18}
\usetikzlibrary{shapes,arrows,positioning,calc,patterns,decorations.pathmorphing,decorations.markings,arrows.meta}

% Color scheme
\definecolor{headcolor}{RGB}{0,102,204}
\definecolor{keycolor}{RGB}{220,20,60}
\definecolor{solutioncolor}{RGB}{34,139,34}
\definecolor{mnemoniccolor}{RGB}{148,0,211}
\definecolor{codecolor}{RGB}{0,0,100}

% Spacing
\setlength{\parskip}{3pt}
\setlist[itemize]{nosep}
\setlist[enumerate]{nosep}

% Title formatting
\titleformat{\section}{\Large\bfseries\color{headcolor}}{\thesection}{1em}{}
\titleformat{\subsection}{\large\bfseries\color{headcolor}}{\thesubsection}{1em}{}

% Pandoc tightlist compatibility
\providecommand{\tightlist}{%
  \setlength{\itemsep}{0pt}\setlength{\parskip}{0pt}}

% Pandoc longtable compatibility
\newcounter{none}
\def\thenone{}


% content/resources/templates/gujarati-boxes.tex
\usepackage{fontspec}
\usepackage{polyglossia}

% Set Gujarati as main language (document is primarily in Gujarati)
% Note: gloss-gujarati.ldf doesn't exist in polyglossia, but it will use hyphenation patterns
\setdefaultlanguage{gujarati}
\setotherlanguage{english}

% Configure Gujarati font properly
% Use Language=Default to prevent polyglossia from trying to add language-specific features
% that don't exist for Gujarati, which causes "empty feature" warnings
\newfontfamily\gujaratifont[Script=Gujarati,AutoFakeBold=2.5,AutoFakeSlant=0.3]{Noto Sans Gujarati}
\setmainfont[Script=Gujarati,AutoFakeBold=2.5,AutoFakeSlant=0.3]{Noto Sans Gujarati}
% Use Noto Sans Gujarati for monospace to support Gujarati in text
\setmonofont[Scale=0.9]{Noto Sans Gujarati}

% Configure English to use the same font
\newfontfamily\englishfont[Script=Gujarati,AutoFakeBold=2.5,AutoFakeSlant=0.3]{Noto Sans Gujarati}

% Translations for polyglossia
\gappto\captionsgujarati{
  \renewcommand{\tablename}{કોષ્ટક}
  \renewcommand{\figurename}{આકૃતિ}
}

% Helper for TikZ nodes to ensure Gujarati font
\newcommand{\gu}[1]{{\gujaratifont #1}}

% Custom environments
\newtcolorbox{solutionbox}{
    breakable,
    enhanced,
    colback=solutioncolor!5!white,
    colframe=solutioncolor!75!black,
    fonttitle=\bfseries,
    title=જવાબ
}

\newtcolorbox{solutionboxnobreak}{
 colback=solutioncolor!5!white,
 colframe=solutioncolor!75!black,
 fonttitle=\bfseries,
 title=જવાબ
}

\newtcolorbox{keyformula}{
 breakable,
 enhanced,
 colback=keycolor!5!white,
 colframe=keycolor!75!black,
 fonttitle=\bfseries,
 title=રાસાયણિક સમીકરણ/સૂત્ર
}

\newtcolorbox{mnemonicbox}{
 breakable,
 enhanced,
 colback=mnemoniccolor!5!white,
 colframe=mnemoniccolor!75!black,
 fonttitle=\bfseries,
 title=મેમરી ટ્રીક
}


% Custom commands for GTU solutions
% This file defines semantic commands for consistent formatting

% Question command with automatic formatting
\newcommand{\question}[2]{%
  \section*{Question #1}%
  \textbf{#2}%
}

% OR question variant
\newcommand{\questionor}[2]{%
  \section*{Question #1 OR}%
  \textbf{#2}%
}

% Proper table environment with caption
\newenvironment{answertable}[1]{%
  \begin{table}[htbp]
  \centering
  \caption{#1}
}{%
  \end{table}
}

% Proper figure environment for diagrams
\newenvironment{answerdiagram}[1]{%
  \begin{figure}[htbp]
  \centering
  \caption{#1}
}{%
  \end{figure}
}

% Semantic markup for key terms
\newcommand{\keyword}[1]{\textbf{#1}}
\newcommand{\code}[1]{\texttt{#1}}
\newcommand{\classname}[1]{\texttt{#1}}
\newcommand{\methodname}[1]{\texttt{#1}}

% Proper quotation marks
\newcommand{\mnemonic}[1]{``#1''}


\title{Foundation of Blockchain (4361603) - Winter 2024 Solution}
\date{November 25, 2024}

\begin{document}
\maketitle

\questionmarks{1(a)}{3}{Short Note on: Distributed Ledger}

\begin{solutionbox}
\textbf{જવાબ}:

\begin{center}
\captionof{table}{Distributed Ledger Features}
\begin{tabulary}{\linewidth}{L L}
    \toprule
    \textbf{પાસું} & \textbf{વર્ણન} \\
    \midrule
    \textbf{વ્યાખ્યા} & ડેટાબેઝ જે અનેક કોમ્પ્યુટર્સ પર ફેલાયેલો છે \\
    \textbf{સંગ્રહ} & ડેટા અનેક સ્થાનો પર સંગ્રહિત થાય છે \\
    \textbf{નિયંત્રણ} & કોઈ એક સત્તાધિકારી તેની માલિકી ધરાવતું નથી \\
    \textbf{અપડેટ્સ} & બધા નકલો એક સાથે અપડેટ થાય છે \\
    \bottomrule
\end{tabulary}
\end{center}

\begin{itemize}
    \item \keyword{વિકેન્દ્રીકૃત}: કોઈ કેન્દ્રીય સર્વરની જરૂર નથી
    \item \keyword{પારદર્શક}: બધા સહભાગીઓ વ્યવહારો જોઈ શકે છે
    \item \keyword{સુરક્ષિત}: સુરક્ષા માટે ક્રિપ્ટોગ્રાફીનો ઉપયોગ કરે છે
\end{itemize}
\end{solutionbox}

\begin{mnemonicbox}
ડેટા સંગ્રહિત પારદર્શક રીતે સુરક્ષિત (DSTS)
\end{mnemonicbox}

\questionmarks{1(b)}{4}{Describe the applications of Blockchain.}

\begin{solutionbox}
\textbf{જવાબ}:

\begin{center}
\captionof{table}{Blockchain Applications}
\begin{tabulary}{\linewidth}{L L L}
    \toprule
    \textbf{એપ્લિકેશન} & \textbf{ઉપયોગ} & \textbf{ફાયદો} \\
    \midrule
    \textbf{Cryptocurrency} & Bitcoin જેવું ડિજિટલ નાણું & સુરક્ષિત ચૂકવણી \\
    \textbf{Supply Chain} & સ્ત્રોતથી પ્રોડક્ટ્સ ટ્રેક કરવા & નકલી માલ અટકાવવા \\
    \textbf{Healthcare} & મેડિકલ રેકોર્ડ્સ સંગ્રહવા & ડેટા સુરક્ષા \\
    \textbf{Voting} & ઇલેક્ટ્રોનિક વોટિંગ સિસ્ટમ & પારદર્શક ચૂંટણીઓ \\
    \textbf{Real Estate} & મિલકત રેકોર્ડ્સ & છેતરપિંડી અટકાવવા \\
    \bottomrule
\end{tabulary}
\end{center}

\begin{itemize}
    \item \keyword{Finance}: ઝડપી આંતરરાષ્ટ્રીય ચૂકવણી
    \item \keyword{Identity}: ડિજિટલ ID ચકાસણી
    \item \keyword{Smart Contracts}: સ્વચાલિત કરારો
\end{itemize}
\end{solutionbox}

\begin{mnemonicbox}
પૈસા, દવા, મતદાન, મિલકત (MMVP)
\end{mnemonicbox}

\questionmarks{1(c)}{7}{Explain Asymmetric Encryption Model with example.}

\begin{solutionbox}
\textbf{જવાબ}:

\textbf{Asymmetric Encryption Process}

\begin{center}
    \begin{tikzpicture}[node distance=1.5cm, auto]
        \node [gtu state] (Sender) {મોકલનાર};
        \node [gtu block, right=of Sender] (PubKey) {Public Key};
        \node [gtu block, right=of PubKey] (Encrypt) {સંદેશ એન્ક્રિપ્ટ};
        \node [gtu block, below=of Encrypt] (Data) {એન્ક્રિપ્ટેડ ડેટા};
        \node [gtu block, left=of Data] (Receiver) {મેળવનાર};
        \node [gtu block, left=of Receiver] (PrivKey) {Private Key};
        \node [gtu block, below=of PrivKey] (Decrypt) {સંદેશ ડિક્રિપ્ટ};
        \node [gtu state, right=of Decrypt] (Msg) {મૂળ સંદેશ};

        \draw [gtu arrow] (Sender) -- (PubKey);
        \draw [gtu arrow] (PubKey) -- (Encrypt);
        \draw [gtu arrow] (Encrypt) -- (Data);
        \draw [gtu arrow] (Data) -- (Receiver);
        \draw [gtu arrow] (Receiver) -- (PrivKey);
        \draw [gtu arrow] (PrivKey) -- (Decrypt);
        \draw [gtu arrow] (Decrypt) -- (Msg);
    \end{tikzpicture}
    \captionof{figure}{Asymmetric Encryption પ્રક્રિયા}
\end{center}

\begin{center}
\captionof{table}{Key તફાવત}
\begin{tabulary}{\linewidth}{L L L L}
    \toprule
    \textbf{Key પ્રકાર} & \textbf{હેતુ} & \textbf{શેરિંગ} & \textbf{ઉદાહરણ} \\
    \midrule
    \textbf{Public Key} & Encryption & ખુલ્લેઆમ શેર થાય છે & RSA Public Key \\
    \textbf{Private Key} & Decryption & ગુપ્ત રાખવામાં આવે છે & RSA Private Key \\
    \bottomrule
\end{tabulary}
\end{center}

\textbf{ઉદાહરણ પ્રક્રિયા:}

\begin{enumerate}
    \item Alice Bob ને સંદેશ મોકલવા માંગે છે
    \item Alice Bob ની public key વાપરીને encrypt કરે છે
    \item ફક્ત Bob ની private key decrypt કરી શકે છે
    \item Bob સંદેશ મેળવે છે અને decrypt કરે છે
\end{enumerate}

\begin{itemize}
    \item \keyword{સુરક્ષા}: Public key જાણીતી હોવા છતાં ડેટા સુરક્ષિત રહે છે
    \item \keyword{પ્રમાણીકરણ}: મોકલનારની ઓળખ સાબિત કરે છે
    \item \keyword{Non-repudiation}: મોકલનાર મોકલવાનો ઇનકાર કરી શકતો નથી
\end{itemize}
\end{solutionbox}

\begin{mnemonicbox}
જાહેર એન્ક્રિપ્ટ, ખાનગી ડિક્રિપ્ટ (PEPD)
\end{mnemonicbox}

\orquestionmarks{1(c)}{7}{Explain Consistency, Availability and Partition Tolerance (CAP) theorem in Blockchain.}

\begin{solutionbox}
\textbf{જવાબ}:

\textbf{CAP Theorem ત્રિકોણ}

\begin{center}
    \begin{tikzpicture}[node distance=2.5cm, auto]
        \node [gtu state] (C) {Consistency};
        \node [gtu state, below left=of C] (A) {Availability};
        \node [gtu state, below right=of C] (P) {Partition Tolerance};
        
        \path (C) -- node[midway] (CA) {} (A);
        \path (A) -- node[midway] (AP) {} (P);
        \path (P) -- node[midway] (PC) {} (C);
        
        \node [gtu block, align=center] at (barycentric cs:C=1,A=1,P=1) {CAP Theorem};
        
        \node [below=0.2cm of C, font=\footnotesize] {બધા નોડ્સ સમાન ડેટા જુએ છે};
        \node [below=0.2cm of A, font=\footnotesize] {સિસ્ટમ હંમેશા પ્રતિક્રિયા આપે છે};
        \node [below=0.2cm of P, font=\footnotesize] {નિષ્ફળતા છતાં કામ કરે છે};

        \draw [gtu arrow, <->] (C) -- (A);
        \draw [gtu arrow, <->] (A) -- (P);
        \draw [gtu arrow, <->] (P) -- (C);
    \end{tikzpicture}
    \captionof{figure}{CAP Theorem ગુણધર્મો}
\end{center}

\begin{center}
\captionof{table}{CAP ગુણધર્મો}
\begin{tabulary}{\linewidth}{L L L}
    \toprule
    \textbf{ગુણધર્મ} & \textbf{વ્યાખ્યા} & \textbf{Blockchain ફોકસ} \\
    \midrule
    \textbf{Consistency} & બધા નોડ્સ પાસે સમાન ડેટા હોય છે & મધ્યમ અગ્રતા \\
    \textbf{Availability} & સિસ્ટમ હંમેશા પ્રતિક્રિયા આપે છે & ઉચ્ચ અગ્રતા \\
    \textbf{Partition Tolerance} & નેટવર્ક વિભાજન છતાં કામ કરે છે & ઉચ્ચ અગ્રતા \\
    \bottomrule
\end{tabulary}
\end{center}

\textbf{મુખ્ય મુદ્દાઓ:}

\begin{itemize}
    \item \keyword{Trade-off}: 3 માંથી માત્ર 2 ગુણધર્મો પૂરા કરી શકાય
    \item \keyword{Blockchain પસંદગી}: સામાન્ય રીતે Availability + Partition Tolerance પસંદ કરે છે
    \item \keyword{વાસ્તવિક ઉદાહરણ}: Bitcoin C કરતાં AP પસંદ કરે છે (eventual consistency)
\end{itemize}
\end{solutionbox}

\begin{mnemonicbox}
કોઈપણ બે પસંદ કરો (CAT)
\end{mnemonicbox}

\questionmarks{2(a)}{3}{Define: Public key, Private key, Digital Signature.}

\begin{solutionbox}
\textbf{જવાબ}:

\begin{center}
\captionof{table}{Cryptographic ઘટકો}
\begin{tabulary}{\linewidth}{L L L}
    \toprule
    \textbf{ઘટક} & \textbf{વ્યાખ્યા} & \textbf{ઉપયોગ} \\
    \midrule
    \textbf{Public Key} & ખુલ્લેઆમ શેર થતી encryption key & ડેટા એન્ક્રિપ્ટ, સહી ચકાસણી \\
    \textbf{Private Key} & માલિક દ્વારા ગુપ્ત રખાતી key & ડેટા ડિક્રિપ્ટ, સહી કરવી \\
    \textbf{Digital Signature} & સંદેશનો એન્ક્રિપ્ટેડ હેશ & અધિકૃતતા અને અખંડિતતા સાબિત કરવી \\
    \bottomrule
\end{tabulary}
\end{center}
\end{solutionbox}

\begin{mnemonicbox}
જાહેર રક્ષણ કરે, ખાનગી સાબિત કરે (PPPP)
\end{mnemonicbox}

\questionmarks{2(b)}{4}{Explain Public blockchain with its advantage and disadvantage.}

\begin{solutionbox}
\textbf{જવાબ}:

\begin{center}
\captionof{table}{Public Blockchain વિશ્લેષણ}
\begin{tabulary}{\linewidth}{L L}
    \toprule
    \textbf{પાસું} & \textbf{વિગતો} \\
    \midrule
    \textbf{વ્યાખ્યા} & દરેક માટે ખુલ્લું નેટવર્ક \\
    \textbf{ઉદાહરણો} & Bitcoin, Ethereum \\
    \bottomrule
\end{tabulary}
\end{center}

\textbf{ફાયદા:}

\begin{itemize}
    \item \keyword{પારદર્શિતા}: બધા વ્યવહારો જોઈ શકાય છે
    \item \keyword{વિકેન્દ્રીકરણ}: કોઈ એકનું નિયંત્રણ નથી
    \item \keyword{સુરક્ષા}: ઘણા નોડ્સ માન્ય કરે છે
\end{itemize}

\textbf{ગેરફાયદા:}

\begin{itemize}
    \item \keyword{ઝડપ}: ધીમી પ્રક્રિયા
    \item \keyword{ઊર્જા}: ઉચ્ચ વીજળી વપરાશ
    \item \keyword{સ્કેલેબિલિટી}: પ્રતિ સેકન્ડ મર્યાદિત વ્યવહારો
\end{itemize}
\end{solutionbox}

\begin{mnemonicbox}
પારદર્શક પણ ધીમું (TBS)
\end{mnemonicbox}

\questionmarks{2(c)}{7}{Describe Core components of Blockchain.}

\begin{solutionbox}
\textbf{જવાબ}:

\textbf{Blockchain માળખું}

\begin{center}
    \begin{tikzpicture}[node distance=1.5cm, auto]
        \node [gtu block] (BN) {Block N};
        \node [gtu block, left=of BN] (BNm1) {Block N-1};
        \node [gtu block, right=of BN] (BNp1) {Block N+1};
        
        \node [gtu block, below=of BN, align=center] (Details) {Block Header\\Transactions};
        
        \node [gtu interval, below=of Details, align=center] (Components) {Previous Hash\\Merkle Root\\Timestamp\\Nonce};
        
        \draw [gtu arrow] (BNm1) -- (BN);
        \draw [gtu arrow] (BN) -- (BNp1);
        \draw [gtu arrow] (BN) -- (Details);
        \draw [gtu arrow] (Details) -- (Components);
    \end{tikzpicture}
    \captionof{figure}{Core Blockchain ઘટકો}
\end{center}

\begin{center}
\captionof{table}{Core ઘટકો}
\begin{tabulary}{\linewidth}{L L L}
    \toprule
    \textbf{ઘટક} & \textbf{કાર્ય} & \textbf{મહત્વ} \\
    \midrule
    \textbf{Block} & વ્યવહારો માટે કન્ટેનર & ડેટા સંગ્રહ \\
    \textbf{Hash} & અનન્ય ઓળખકર્તા & સુરક્ષા \\
    \textbf{Merkle Tree} & વ્યવહાર સારાંશ & ચકાસણી \\
    \textbf{Nonce} & માઇનિંગ નંબર & Proof of work \\
    \textbf{Timestamp} & સમય રેકોર્ડ & કાલક્રમિક ક્રમ \\
    \textbf{Previous Hash} & પાછલા બ્લોક સાથે લિંક & સાંકળ અખંડિતતા \\
    \bottomrule
\end{tabulary}
\end{center}

\begin{itemize}
    \item \keyword{અફરતા}: ભૂતકાળના રેકોર્ડ્સ બદલી શકાતા નથી
    \item \keyword{પારદર્શિતા}: બધો ડેટા દૃશ્યમાન
    \item \keyword{સહમતિ (Consensus)}: નેટવર્ક માન્યતા પર સંમત થાય છે
\end{itemize}
\end{solutionbox}

\begin{mnemonicbox}
બ્લોક્સ હેશ મર્કલી નોન્સ સમય પાછલો (BHMNTP)
\end{mnemonicbox}

\orquestionmarks{2(a)}{3}{Short Note on: SideChain}

\begin{solutionbox}
\textbf{જવાબ}:

\begin{center}
\captionof{table}{SideChain લક્ષણો}
\begin{tabulary}{\linewidth}{L L}
    \toprule
    \textbf{લક્ષણ} & \textbf{વર્ણન} \\
    \midrule
    \textbf{વ્યાખ્યા} & મુખ્ય બ્લોકચેન સાથે જોડાયેલ અલગ બ્લોકચેન \\
    \textbf{હેતુ} & મુખ્ય બ્લોકચેન કાર્યક્ષમતા વધારવી \\
    \textbf{જોડાણ} & Two-way peg મિકેનિઝમ \\
    \bottomrule
\end{tabulary}
\end{center}

\begin{itemize}
    \item \keyword{સ્કેલેબિલિટી}: મુખ્ય ચેઇનનો ભાર ઘટાડે છે
    \item \keyword{લવચીકતા}: કસ્ટમ ફીચર્સ શક્ય છે
    \item \keyword{સુરક્ષા}: મુખ્ય ચેઇન સુરક્ષા વારસામાં મેળવે છે
\end{itemize}
\end{solutionbox}

\begin{mnemonicbox}
અલગ સાઈડ સ્કેલ (SSS)
\end{mnemonicbox}

\orquestionmarks{2(b)}{4}{Explain Private blockchain with its advantage and disadvantage.}

\begin{solutionbox}
\textbf{જવાબ}:

\begin{center}
\captionof{table}{Private Blockchain વિશ્લેષણ}
\begin{tabulary}{\linewidth}{L L}
    \toprule
    \textbf{પાસું} & \textbf{વિગતો} \\
    \midrule
    \textbf{વ્યાખ્યા} & નિયંત્રિત એક્સેસ સાથે પ્રતિબંધિત નેટવર્ક \\
    \textbf{નિયંત્રણ} & એક જ સંસ્થા સંચાલન કરે છે \\
    \bottomrule
\end{tabulary}
\end{center}

\textbf{ફાયદા:}

\begin{itemize}
    \item \keyword{ઝડપ}: ઝડપી વ્યવહારો
    \item \keyword{ગોપનીયતા}: નિયંત્રિત ડેટા એક્સેસ
    \item \keyword{કાર્યક્ષમતા}: ઓછો ઊર્જા વપરાશ
    \item \keyword{Compliance}: નિયામક આવશ્યકતાઓ પૂરી કરે છે
\end{itemize}

\textbf{ગેરફાયદા:}

\begin{itemize}
    \item \keyword{કેન્દ્રીકરણ}: નિયંત્રણનું એક બિંદુ
    \item \keyword{વિશ્વાસ}: નિયંત્રિત સંસ્થા પર આધાર રાખે છે
    \item \keyword{મર્યાદિત}: ઓછા સહભાગીઓ
\end{itemize}
\end{solutionbox}

\begin{mnemonicbox}
ઝડપી ખાનગી નિયંત્રિત (FPC)
\end{mnemonicbox}

\orquestionmarks{2(c)}{7}{Explain Data structure of Blockchain.}

\begin{solutionbox}
\textbf{જવાબ}:

\textbf{Blockchain ડેટા સ્ટ્રક્ચર}

\begin{center}
    \begin{tikzpicture}[node distance=0cm, outer sep=0pt]
        % Block 1
        \node [draw, rectangle, minimum width=3.5cm, minimum height=1cm, fill=gray!10] (Head1) {\textbf{Block 1}};
        \node [draw, rectangle, minimum width=3.5cm, minimum height=0.7cm, below=0pt of Head1] (Prev1) {Has Prev: 0000};
        \node [draw, rectangle, minimum width=3.5cm, minimum height=0.7cm, below=0pt of Prev1] (Merkle1) {Merkle Root};
        \node [draw, rectangle, minimum width=3.5cm, minimum height=0.7cm, below=0pt of Merkle1] (Nonce1) {Nonce};
        \node [draw, rectangle, minimum width=3.5cm, minimum height=1.5cm, below=0pt of Nonce1, align=center] (Tx1) {Tx 1\\Tx 2\\Tx 3};
        
        \node [draw, rectangle, minimum width=3.7cm, minimum height=5.5cm, below=0pt of Head1, yshift=2.7cm] (Block1) {};

        % Block 2
        \node [draw, rectangle, minimum width=3.5cm, minimum height=1cm, fill=gray!10, right=1cm of Head1] (Head2) {\textbf{Block 2}};
        \node [draw, rectangle, minimum width=3.5cm, minimum height=0.7cm, below=0pt of Head2] (Prev2) {Has Prev: Hash(B1)};
        \node [draw, rectangle, minimum width=3.5cm, minimum height=0.7cm, below=0pt of Prev2] (Merkle2) {Merkle Root};
        \node [draw, rectangle, minimum width=3.5cm, minimum height=0.7cm, below=0pt of Merkle2] (Nonce2) {Nonce};
        \node [draw, rectangle, minimum width=3.5cm, minimum height=1.5cm, below=0pt of Nonce2, align=center] (Tx2) {Tx 4\\Tx 5\\Tx 6};
        
        \node [draw, rectangle, minimum width=3.7cm, minimum height=5.5cm, below=0pt of Head2, yshift=2.7cm] (Block2) {};

        % Arrows
        \draw [thick, ->] (Prev2.west) -- ++(-0.5,0) |- (Block1.east);
    \end{tikzpicture}
    \captionof{figure}{Blockchain Linked List માળખું}
\end{center}

\begin{center}
\captionof{table}{Data Structure તત્વો}
\begin{tabulary}{\linewidth}{L L L}
    \toprule
    \textbf{તત્વ} & \textbf{હેતુ} & \textbf{સાઈઝ} \\
    \midrule
    \textbf{Block Header} & મેટાડેટા ધરાવે છે & Fixed size \\
    \textbf{Transaction List} & વાસ્તવિક ડેટા & Variable size \\
    \textbf{Hash Pointer} & બ્લોક્સ લિંક કરે છે & 256 bits \\
    \textbf{Merkle Tree} & વ્યવહાર સારાંશ & Logarithmic \\
    \bottomrule
\end{tabulary}
\end{center}

\textbf{મુખ્ય લક્ષણો:}

\begin{itemize}
    \item \keyword{રૈખિક માળખું}: બ્લોક્સ ક્રમમાં જોડાયેલા
    \item \keyword{Hash Linking}: દરેક બ્લોક પાછલા નો સંદર્ભ આપે છે
    \item \keyword{Merkle Trees}: કાર્યક્ષમ વ્યવહાર ચકાસણી
    \item \keyword{અફર}: શોધાયા વગર બદલી શકાતું નથી
\end{itemize}
\end{solutionbox}

\begin{mnemonicbox}
રૈખિક હેશ મર્કલી અફર (LHMI)
\end{mnemonicbox}

\questionmarks{3(a)}{3}{Short Note on: Consensus Mechanism in Blockchain.}

\begin{solutionbox}
\textbf{જવાબ}:

\begin{center}
\captionof{table}{Consensus Mechanism}
\begin{tabulary}{\linewidth}{L L}
    \toprule
    \textbf{પાસું} & \textbf{વર્ણન} \\
    \midrule
    \textbf{હેતુ} & નેટવર્ક સ્થિતિ પર સંમત થવું \\
    \textbf{જરૂરિયાત} & Double spending અટકાવવું \\
    \textbf{પ્રકારો} & PoW, PoS, DPoS \\
    \bottomrule
\end{tabulary}
\end{center}

\begin{itemize}
    \item \keyword{સંમતિ}: બધા નોડ્સ સંમત થવા જોઈએ
    \item \keyword{વિકેન્દ્રીકરણ}: કોઈ કેન્દ્રીય સત્તા નથી
    \item \keyword{સુરક્ષા}: દૂષિત પ્રવૃત્તિઓ અટકાવે છે
\end{itemize}
\end{solutionbox}

\begin{mnemonicbox}
સંમતિ સુરક્ષા અટકાવે (APS)
\end{mnemonicbox}

\questionmarks{3(b)}{4}{Compare Hard Fork and Soft Fork in Blockchain.}

\begin{solutionbox}
\textbf{જવાબ}:

\begin{center}
\captionof{table}{Fork તફાવત}
\begin{tabulary}{\linewidth}{L L L}
    \toprule
    \textbf{લક્ષણ} & \textbf{Hard Fork} & \textbf{Soft Fork} \\
    \midrule
    \textbf{સુસંગતતા} & પછાત સુસંગત નથી (Not backward compatible) & પછાત સુસંગત (Backward compatible) \\
    \textbf{નિયમો} & નવા નિયમો બનાવે છે & હાલના નિયમો કડક બનાવે છે \\
    \textbf{અપગ્રેડ} & બધા નોડ્સ અપગ્રેડ કરવા પડે & વૈકલ્પિક અપગ્રેડ \\
    \textbf{પરિણામ} & બે અલગ ચેઇન & એક ચેઇન ચાલુ રહે \\
    \textbf{ઉદાહરણ} & Ethereum થી Ethereum Classic & Bitcoin SegWit \\
    \bottomrule
\end{tabulary}
\end{center}

\textbf{મુખ્ય તફાવતો:}

\begin{itemize}
    \item \keyword{Hard Fork}: બ્લોકચેનમાં કાયમી વિભાજન
    \item \keyword{Soft Fork}: કામચલાઉ પ્રતિબંધ જે કાયમી બને છે
\end{itemize}
\end{solutionbox}

\begin{mnemonicbox}
હાર્ડ સ્પ્લિટ, સોફ્ટ પ્રતિબંધ (HSSR)
\end{mnemonicbox}

\questionmarks{3(c)}{7}{What is Proof of Work? How does it work? Explain with example.}

\begin{solutionbox}
\textbf{જવાબ}:

\textbf{Proof of Work પ્રક્રિયા}

\begin{center}
    \begin{tikzpicture}[node distance=1.5cm, auto]
        \node [gtu state] (New) {નવા વ્યવહારો};
        \node [gtu block, right=of New] (Create) {બ્લોક બનાવો};
        \node [gtu block, right=of Create] (Hash) {Hash ગણતરી};
        \node [draw, diamond, aspect=2, below=of Hash] (Check) {Hash < Target?};
        
        \node [gtu block, left=of Check] (Change) {Nonce બદલો};
        \node [gtu block, right=of Check] (Valid) {બ્લોક માન્ય};
        \node [gtu block, below=of Valid] (Add) {બ્લોકચેનમાં ઉમેરો};
        
        \draw [gtu arrow] (New) -- (Create);
        \draw [gtu arrow] (Create) -- (Hash);
        \draw [gtu arrow] (Hash) -- (Check);
        \draw [gtu arrow] (Check) -- node[above] {હા} (Valid);
        \draw [gtu arrow] (Check.west) -- node[above] {ના} (Change);
        \draw [gtu arrow] (Change.north) |- (Hash.west);
        \draw [gtu arrow] (Valid) -- (Add);
    \end{tikzpicture}
    \captionof{figure}{Mining અને PoW વર્કફ્લો}
\end{center}

\begin{center}
\captionof{table}{PoW ઘટકો}
\begin{tabulary}{\linewidth}{L L L}
    \toprule
    \textbf{ઘટક} & \textbf{કાર્ય} & \textbf{ઉદાહરણ} \\
    \midrule
    \textbf{Hash Function} & અનન્ય ફિંગરપ્રિન્ટ બનાવે છે & SHA-256 \\
    \textbf{Nonce} & Hash બદલવા માટે રેન્ડમ નંબર & 12345 \\
    \textbf{Difficulty} & જરૂરી અગ્રણી શૂન્ય (zeros) & 4 શૂન્ય \\
    \textbf{Mining} & કમ્પ્યુટિંગ પ્રક્રિયા & Bitcoin mining \\
    \bottomrule
\end{tabulary}
\end{center}

\textbf{કાર્ય પ્રક્રિયા:}

\begin{enumerate}
    \item પેન્ડિંગ વ્યવહારો એકત્રિત કરો
    \item વ્યવહારો સાથે બ્લોક બનાવો
    \item વિવિધ nonce મૂલ્યો અજમાવો
    \item વારંવાર hash ગણતરી કરો
    \item જરૂરી શૂન્ય સાથે hash શોધો
    \item નેટવર્ક પર માન્ય બ્લોક પ્રસારિત કરો
\end{enumerate}

\textbf{Bitcoin ઉદાહરણ:}

\begin{itemize}
    \item \keyword{Target}: Hash ચોક્કસ શૂન્યથી શરૂ થવું જોઈએ
    \item \keyword{Time}: ~10 મિનિટ પ્રતિ બ્લોક
    \item \keyword{Reward}: 6.25 BTC (2024 મુજબ)
\end{itemize}
\end{solutionbox}

\begin{mnemonicbox}
અજમાવો ગણતરી કરો શૂન્ય સુધી (TCUZ)
\end{mnemonicbox}

\orquestionmarks{3(a)}{3}{Short Note on: Block Rewards in Blockchain.}

\begin{solutionbox}
\textbf{જવાબ}:

\begin{center}
\captionof{table}{Block Rewards વિશ્લેષણ}
\begin{tabulary}{\linewidth}{L L}
    \toprule
    \textbf{લક્ષણ} & \textbf{વર્ણન} \\
    \midrule
    \textbf{હેતુ} & Miners ને પ્રોત્સાહન આપવા \\
    \textbf{ઘટકો} & Block reward + transaction fees \\
    \textbf{Bitcoin} & 50 BTC થી શરૂ, દર 4 વર્ષે અડધું \\
    \bottomrule
\end{tabulary}
\end{center}

\begin{itemize}
    \item \keyword{પ્રેરણા}: નેટવર્ક સહભાગિતાને પ્રોત્સાહન આપે છે
    \item \keyword{અડધું કરવું}: સમય સાથે ફુગાવો ઘટાડે છે
    \item \keyword{ફી}: Miners માટે વધારાની આવક
\end{itemize}
\end{solutionbox}

\begin{mnemonicbox}
Miners પ્રેરિત પૈસા (MPP)
\end{mnemonicbox}

\orquestionmarks{3(b)}{4}{What is 51\% attack and how does it work?}

\begin{solutionbox}
\textbf{જવાબ}:

\begin{center}
\captionof{table}{51\% Attack વિશ્લેષણ}
\begin{tabulary}{\linewidth}{L L}
    \toprule
    \textbf{પાસું} & \textbf{વિગતો} \\
    \midrule
    \textbf{વ્યાખ્યા} & બહુમતી mining power નિયંત્રિત કરવું \\
    \textbf{મર્યાદા} & 50\% થી વધુ નેટવર્ક hash rate \\
    \textbf{ક્ષમતા} & Transactions ઉલટાવી શકે છે \\
    \textbf{મર્યાદા} & બીજાના coins ચોરી શકતો નથી \\
    \bottomrule
\end{tabulary}
\end{center}

\textbf{તે કેવી રીતે કામ કરે છે:}

\begin{enumerate}
    \item આક્રમણકારી બહુમતી mining power મેળવે છે
    \item ખાનગી blockchain fork બનાવે છે
    \item પ્રામાણિક નેટવર્ક કરતાં ઝડપથી mine કરે છે
    \item નેટવર્ક પર લાંબી chain છોડે છે
    \item નેટવર્ક લાંબી chain ને માન્ય તરીકે સ્વીકારે છે
\end{enumerate}

\begin{center}
    \begin{tikzpicture}[node distance=1.5cm]
        \node [gtu block] (B_N) {Block N};
        
        % Honest Chain
        \node [gtu block, above right=1cm and 2cm of B_N] (Honest1) {Block N+1};
        \node [right=0.2cm of Honest1] {પ્રામાણિક Chain (ત્યજી દેવાયેલ)};
        
        % Attacker Chain
        \node [gtu block, below right=1cm and 2cm of B_N, fill=red!10] (Bad1) {Block N'+1};
        \node [gtu block, right=of Bad1, fill=red!10] (Bad2) {Block N'+2 (લાંબી)};
        \node [right=0.2cm of Bad2] {આક્રમણકારી Chain (સ્વીકૃત)};

        \draw [gtu arrow] (B_N) -- (Honest1);
        \draw [gtu arrow] (B_N) -- (Bad1);
        \draw [gtu arrow] (Bad1) -- (Bad2);
    \end{tikzpicture}
    \captionof{figure}{51\% Attack મિકેનિઝમ}
\end{center}

\begin{itemize}
    \item \keyword{ડબલ ખર્ચ}: સમાન coins બે વાર ખર્ચ કરવા
    \item \keyword{Transaction ઉલટાવવા}: પુષ્ટિ થયેલા transactions રદ કરવા
\end{itemize}
\end{solutionbox}

\begin{mnemonicbox}
બહુમતી નિયંત્રણ Chain (BNC)
\end{mnemonicbox}

\orquestionmarks{3(c)}{7}{What is Proof of Stake? How does it work? Explain with example.}

\begin{solutionbox}
\textbf{જવાબ}:

\textbf{Proof of Stake પ્રક્રિયા}

\begin{center}
    \begin{tikzpicture}[node distance=1.5cm, auto]
        \node [gtu state] (Stake) {Validators Stake કરે છે};
        \node [gtu block, right=of Stake] (Select) {રેન્ડમ પસંદગી};
        \node [gtu block, right=of Select] (Prop) {Block સૂચવે છે};
        \node [gtu block, below=of Prop] (Vote) {Validators મત આપે છે};
        \node [draw, diamond, aspect=2, below=of Vote] (Cons) {બહુમતી સંમત?};
        
        \node [gtu block, left=of Cons] (Add) {Block ઉમેરાય છે};
        \node [gtu state, below=of Add] (Reward) {પુરસ્કાર};
        \node [gtu state, right=of Cons] (Slash) {દંડ};
        
        \draw [gtu arrow] (Stake) -- (Select);
        \draw [gtu arrow] (Select) -- (Prop);
        \draw [gtu arrow] (Prop) -- (Vote);
        \draw [gtu arrow] (Vote) -- (Cons);
        \draw [gtu arrow] (Cons) -- node[above] {હા} (Add);
        \draw [gtu arrow] (Add) -- (Reward);
        \draw [gtu arrow] (Cons) -- node[above] {ના} (Slash);
    \end{tikzpicture}
    \captionof{figure}{Proof of Stake વર્કફ્લો}
\end{center}

\begin{center}
\captionof{table}{PoS vs PoW}
\begin{tabulary}{\linewidth}{L L L}
    \toprule
    \textbf{લક્ષણ} & \textbf{Proof of Stake} & \textbf{Proof of Work} \\
    \midrule
    \textbf{ઊર્જા} & ઓછો વપરાશ & વધુ વપરાશ \\
    \textbf{પસંદગી} & Stake આધારિત & Computing power \\
    \textbf{હાર્ડવેર} & સામાન્ય કમ્પ્યુટર & વિશેષ miners \\
    \textbf{ઝડપ} & ઝડપી & ધીમી \\
    \bottomrule
\end{tabulary}
\end{center}

\textbf{Ethereum ઉદાહરણ:}

\begin{itemize}
    \item \keyword{લઘુત્તમ Stake}: 32 ETH જરૂરી
    \item \keyword{દંડ}: દુષ્ટ વર્તન માટે slashing
    \item \keyword{ફાયદા}: ઊર્જા કાર્યક્ષમ અને સ્કેલેબલ
\end{itemize}
\end{solutionbox}

\begin{mnemonicbox}
Stake પસંદ Validate પુરસ્કાર (SPVP)
\end{mnemonicbox}

\questionmarks{4(a)}{3}{Describe Byzantine Fault Tolerance.}

\begin{solutionbox}
\textbf{જવાબ}:

\begin{center}
\captionof{table}{Byzantine Fault Tolerance}
\begin{tabulary}{\linewidth}{L L}
    \toprule
    \textbf{પાસું} & \textbf{વર્ણન} \\
    \midrule
    \textbf{સમસ્યા} & કેટલાક nodes દુષ્ટ રીતે વર્તે છે \\
    \textbf{સહનશીલતા} & ખામીયુક્ત nodes છતાં સિસ્ટમ કામ કરે છે \\
    \textbf{આવશ્યકતા} & 1/3 થી ઓછા nodes ખામીયુક્ત હોઈ શકે છે \\
    \bottomrule
\end{tabulary}
\end{center}

\begin{itemize}
    \item \keyword{સર્વસંમતિ}: પ્રામાણિક nodes સંમત થવા જોઈએ
    \item \keyword{પ્રતિકાર}: નેટવર્ક હુમલાઓમાં ટકી રહે છે
\end{itemize}
\end{solutionbox}

\begin{mnemonicbox}
ખામીયુક્ત Nodes સહન (KNS)
\end{mnemonicbox}

\questionmarks{4(b)}{4}{How smart contract works in blockchain?}

\begin{solutionbox}
\textbf{જવાબ}:

\textbf{Smart Contract અમલીકરણ}

\begin{center}
    \begin{tikzpicture}[node distance=1.5cm, auto]
        \node [gtu block] (Create) {Contract બનાવાયેલ};
        \node [gtu block, right=of Create] (Deploy) {Blockchain પર Deploy};
        \node [gtu block, below=of Deploy] (Cond) {શરતો પૂરી થાય છે};
        \node [gtu state, left=of Cond] (Exec) {સ્વચાલિત અમલીકરણ};
        \node [gtu state, left=of Exec] (Rec) {પરિણામો રેકોર્ડ};
        
        \draw [gtu arrow] (Create) -- (Deploy);
        \draw [gtu arrow] (Deploy) -- (Cond);
        \draw [gtu arrow] (Cond) -- (Exec);
        \draw [gtu arrow] (Exec) -- (Rec);
    \end{tikzpicture}
    \captionof{figure}{Smart Contract જીવનચક્ર}
\end{center}

\textbf{કાર્ય પ્રક્રિયા:}

\begin{enumerate}
    \item \textbf{નિર્માણ}: Developer contract code લખે છે
    \item \textbf{Deployment}: Contract બ્લોકચેન પર સંગ્રહિત થાય છે
    \item \textbf{Trigger}: બાહ્ય ઘટના contract સક્રિય કરે છે
    \item \textbf{અમલીકરણ}: Code સ્વચાલિત રીતે ચાલે છે
\end{enumerate}
\end{solutionbox}

\begin{mnemonicbox}
Code સ્વચાલિત અમલ (CSA)
\end{mnemonicbox}

\questionmarks{4(c)}{7}{What is SHA-256 and what is the use of SHA-256 in Blockchain.}

\begin{solutionbox}
\textbf{જવાબ}:

\begin{center}
\captionof{table}{SHA-256 ગુણધર્મો}
\begin{tabulary}{\linewidth}{L L}
    \toprule
    \textbf{ગુણધર્મ} & \textbf{વર્ણન} \\
    \midrule
    \textbf{પૂરું નામ} & Secure Hash Algorithm 256-bit \\
    \textbf{આઉટપુટ} & હંમેશા 256 bits (64 hex characters) \\
    \textbf{ઇનપુટ} & કોઈ પણ કદનો ડેટા \\
    \textbf{પ્રકૃતિ} & એક-માર્ગીય function \\
    \bottomrule
\end{tabulary}
\end{center}

\textbf{બ્લોકચેનમાં SHA-256}

\begin{center}
    \begin{tikzpicture}[node distance=1.5cm, auto]
        \node [gtu block] (Data) {Block ડેટા};
        \node [gtu block, right=of Data] (Hash) {SHA-256 Hash};
        \node [gtu state, right=of Hash] (Out) {Block Hash};
        \node [gtu block, below=of Out] (Prev) {Previous Hash સંદર્ભ};
        \node [gtu block, left=of Prev] (Chain) {Chain અખંડિતતા};
        
        \draw [gtu arrow] (Data) -- (Hash);
        \draw [gtu arrow] (Hash) -- (Out);
        \draw [gtu arrow] (Out) -- (Prev);
        \draw [gtu arrow] (Prev) -- (Chain);
    \end{tikzpicture}
    \captionof{figure}{બ્લોકચેનમાં Hashing}
\end{center}

\textbf{બ્લોકચેનમાં ઉપયોગ:}

\begin{itemize}
    \item \keyword{Block Hashing}: ઉનિક block ઓળખકર્તા બનાવવા
    \item \keyword{Merkle Trees}: બધા transactions નો સારાંશ આપવા
    \item \keyword{Proof of Work}: Mining કઠિનતા લક્ષ્ય
    \item \keyword{Digital Signatures}: સુરક્ષિત transaction હસ્તાક્ષર
\end{itemize}
\end{solutionbox}

\begin{mnemonicbox}
Hash ઓળખે સુરક્ષિત કરે સાબિત કરે (HOSK)
\end{mnemonicbox}

\orquestionmarks{4(a)}{3}{Explain Bitcoin and eventual consistency.}

\begin{solutionbox}
\textbf{જવાબ}:

\begin{center}
\captionof{table}{Bitcoin Consistency}
\begin{tabulary}{\linewidth}{L L}
    \toprule
    \textbf{ખ્યાલ} & \textbf{વર્ણન} \\
    \midrule
    \textbf{Eventual Consistency} & બધા nodes આખરે સંમત થાય છે \\
    \textbf{અસ્થાયી Forks} & અનેક માન્ય chains અસ્તિત્વ ધરાવે છે \\
    \textbf{ઉકેલ} & સૌથી લાંબી chain જીતે છે \\
    \bottomrule
\end{tabulary}
\end{center}

\begin{itemize}
    \item \keyword{સમય વિલંબ}: નેટવર્ક પ્રસારણમાં સમય લાગે છે
    \item \keyword{પુષ્ટિ}: વધુ blocks = વધુ નિશ્ચિતતા
    \item \keyword{અંતિમતા}: વ્યવહારિક રીતે અનુલટાવી શકાય તેવું બને છે
\end{itemize}
\end{solutionbox}

\begin{mnemonicbox}
આખરે દરેક સંમત (ADS)
\end{mnemonicbox}

\orquestionmarks{4(b)}{4}{Discuss types of smart contract in blockchain.}

\begin{solutionbox}
\textbf{જવાબ}:

\begin{center}
\captionof{table}{Smart Contract પ્રકારો}
\begin{tabulary}{\linewidth}{L L L}
    \toprule
    \textbf{પ્રકાર} & \textbf{કાર્ય} & \textbf{ઉદાહરણ} \\
    \midrule
    \textbf{કાનૂની Contract} & કાનૂની રીતે બંધનકર્તા કરાર & Real estate ટ્રાન્સફર \\
    \textbf{Application Logic} & Decentralized app functions & Token એક્સચેન્જ \\
    \textbf{Decentralized Autonomous} & સ્વ-શાસિત સંસ્થાઓ & DAO મતદાન \\
    \textbf{Multi-signature} & અનેક મંજૂરીઓ જરૂરી & Escrow સેવાઓ \\
    \bottomrule
\end{tabulary}
\end{center}
\end{solutionbox}

\begin{mnemonicbox}
કાનૂની Logic સ્વાયત્ત બહુ (KLSB)
\end{mnemonicbox}

\orquestionmarks{4(c)}{7}{Define Merkle Tree and explain how it works in blockchain.}

\begin{solutionbox}
\textbf{જવાબ}:

\textbf{Merkle Tree રચના}

\begin{center}
    \begin{tikzpicture}[level distance=1.5cm, sibling distance=2.5cm, every node/.style={gtu block, minimum width=1.5cm}]
        \node {Root Hash (ABCD)}
            child {node {Hash(AB)}
                child {node {Hash(A)}
                    child {node {Tx A}}
                }
                child {node {Hash(B)}
                    child {node {Tx B}}
                }
            }
            child {node {Hash(CD)}
                child {node {Hash(C)}
                    child {node {Tx C}}
                }
                child {node {Hash(D)}
                    child {node {Tx D}}
                }
            };
    \end{tikzpicture}
    \captionof{figure}{Merkle Binary Tree}
\end{center}

\begin{center}
\captionof{table}{Merkle Tree ફાયદા}
\begin{tabulary}{\linewidth}{L L}
    \toprule
    \textbf{ફાયદો} & \textbf{વર્ણન} \\
    \midrule
    \textbf{કાર્યક્ષમતા} & બધો ડેટા ડાઉનલોડ કર્યા વિના transactions ચકાસો \\
    \textbf{સુરક્ષા} & કોઈ પણ ફેરફાર તરત શોધાય જાય છે \\
    \textbf{સ્કેલેબિલિટી} & Logarithmic ચકાસણી સમય \\
    \textbf{સંગ્રહ} & કોમ્પેક્ટ પ્રતિનિધિત્વ \\
    \bottomrule
\end{tabulary}
\end{center}

\textbf{કાર્ય પ્રક્રિયા:}

\begin{enumerate}
    \item \textbf{Hash Transactions}: દરેક transaction નો hash મેળવો
    \item \textbf{જોડી Hashing}: નજીકના hashes ને મિલાવો
    \item \textbf{પ્રક્રિયા પુનરાવર્તન}: એક root hash સુધી ચાલુ રાખો
    \item \textbf{Root સંગ્રહ}: ફક્ત root block header માં સંગ્રહિત કરો
\end{enumerate}
\end{solutionbox}

\begin{mnemonicbox}
Tree ગોઠવે ચકાસે કાર્યક્ષમ રીતે (TGCK)
\end{mnemonicbox}

\questionmarks{5(a)}{3}{Short Note on: Bitcoin Scripting}

\begin{solutionbox}
\textbf{જવાબ}:

\begin{center}
\captionof{table}{Bitcoin Scripting}
\begin{tabulary}{\linewidth}{L L}
    \toprule
    \textbf{લક્ષણ} & \textbf{વર્ણન} \\
    \midrule
    \textbf{ભાષા} & Stack-based programming ભાષા \\
    \textbf{હેતુ} & ખર્ચની શરતો વ્યાખ્યાયિત કરવી \\
    \textbf{અમલીકરણ} & Coins ખર્ચ કરવામાં આવે ત્યારે ચાલે છે \\
    \bottomrule
\end{tabulary}
\end{center}

\begin{itemize}
    \item \keyword{સરળ}: ફક્ત મૂળભૂત operations
    \item \keyword{સુરક્ષિત}: મર્યાદિત કાર્યક્ષમતા દુરુપયોગ અટકાવે છે
    \item \keyword{લવચીક}: વિવિધ transaction પ્રકારો શક્ય છે
\end{itemize}
\end{solutionbox}

\begin{mnemonicbox}
Stack વ્યાખ્યા ખર્ચ (SVK)
\end{mnemonicbox}

\questionmarks{5(b)}{4}{Explain Decentralized Applications (dApps) in Blockchain and how does it work?}

\begin{solutionbox}
\textbf{જવાબ}:

\begin{center}
\captionof{table}{dApp ઘટકો}
\begin{tabulary}{\linewidth}{L L}
    \toprule
    \textbf{ઘટક} & \textbf{કાર્ય} \\
    \midrule
    \textbf{Frontend} & User interface \\
    \textbf{Backend} & Blockchain પર smart contracts \\
    \textbf{Storage} & Decentralized storage systems \\
    \textbf{Network} & Peer-to-peer communication \\
    \bottomrule
\end{tabulary}
\end{center}

\textbf{dApp Architecture}

\begin{center}
    \begin{tikzpicture}[node distance=1.5cm, auto]
        \node [gtu state] (User) {User};
        \node [gtu block, right=of User] (UI) {Frontend UI};
        \node [gtu block, right=of UI] (Smart) {Smart Contract};
        \node [gtu block, below=of Smart] (Chain) {Blockchain};
        \node [gtu block, below=of UI] (P2P) {P2P Network};
        
        \draw [gtu arrow] (User) -- (UI);
        \draw [gtu arrow] (UI) -- (Smart);
        \draw [gtu arrow] (Smart) -- (Chain);
        \draw [gtu arrow] (UI) -- (P2P);
    \end{tikzpicture}
    \captionof{figure}{dApp વર્કિંગ મોડેલ}
\end{center}

\textbf{મુખ્ય લક્ષણો:}

\begin{itemize}
    \item \keyword{કોઈ કેન્દ્રીય સર્વર નથી}: વિતરિત નેટવર્ક પર ચાલે છે
    \item \keyword{Open Source}: Code જાહેરમાં ઉપલબ્ધ છે
    \item \keyword{સ્વાયત્ત}: કંપની નિયંત્રણ વિના કામ કરે છે
\end{itemize}
\end{solutionbox}

\begin{mnemonicbox}
વિકેન્દ્રીત Apps દરેક જગ્યાએ ચાલે (VADJ)
\end{mnemonicbox}

\questionmarks{5(c)}{7}{Explain Hyperledger with its advantages and disadvantages.}

\begin{solutionbox}
\textbf{જવાબ}:

\begin{center}
\captionof{table}{Hyperledger ઝાંખી}
\begin{tabulary}{\linewidth}{L L}
    \toprule
    \textbf{પાસું} & \textbf{વર્ણન} \\
    \midrule
    \textbf{પ્રકાર} & Private/Consortium blockchain platform \\
    \textbf{વિકાસકર્તા} & Linux Foundation \\
    \textbf{લક્ષ્ય} & Enterprise applications \\
    \textbf{Consensus} & Pluggable consensus mechanism \\
    \bottomrule
\end{tabulary}
\end{center}

\textbf{Hyperledger આર્કિટેક્ચર}

\begin{center}
    \begin{tikzpicture}[node distance=1.5cm, auto]
        \node [gtu block] (App) {Application Layer};
        \node [gtu block, below=of App, minimum width=5cm] (Fabric) {Hyperledger Fabric};
        
        \node [gtu interval, below=of Fabric, align=center] (Core) {Chaincode \\ Consensus \\ Membership};
        \node [gtu block, below=of Core] (Order) {Ordering Service};
        
        \draw [gtu arrow] (App) -- (Fabric);
        \draw [gtu arrow] (Fabric) -- (Core);
        \draw [gtu arrow] (Core) -- (Order);
    \end{tikzpicture}
    \captionof{figure}{Hyperledger Layered આર્કિટેક્ચર}
\end{center}

\textbf{ફાયદા:}

\begin{itemize}
    \item \keyword{પ્રદર્શન}: ઉચ્ચ transaction throughput
    \item \keyword{ગોપનીયતા}: ગુપ્ત transactions
    \item \keyword{મોડ્યુલર}: Pluggable components
\end{itemize}

\textbf{ગેરફાયદા:}

\begin{itemize}
    \item \keyword{કેન્દ્રીકરણ}: સંપૂર્ણ વિકેન્દ્રીકૃત નથી
    \item \keyword{જટિલતા}: સેટ કરવું મુશ્કેલ છે
    \item \keyword{ખર્ચ}: મોંઘું infrastructure
\end{itemize}
\end{solutionbox}

\begin{mnemonicbox}
ખાનગી પ્રદર્શન Enterprise (KPE)
\end{mnemonicbox}

\orquestionmarks{5(a)}{3}{Short Note on: Bitcoin Mining}

\begin{solutionbox}
\textbf{જવાબ}:

\begin{center}
\captionof{table}{Bitcoin Mining વિશ્લેષણ}
\begin{tabulary}{\linewidth}{L L}
    \toprule
    \textbf{પાસું} & \textbf{વર્ણન} \\
    \midrule
    \textbf{હેતુ} & Transactions ચકાસણી અને blocks બનાવવા \\
    \textbf{પ્રક્રિયા} & Cryptographic પઝલ હલ કરવા \\
    \textbf{પુરસ્કાર} & BTC + transaction fees \\
    \bottomrule
\end{tabulary}
\end{center}

\begin{itemize}
    \item \keyword{હાર્ડવેર}: વિશિષ્ટ ASIC miners
    \item \keyword{ઊર્જા} : ઉચ્ચ વીજળી વપરાશ
    \item \keyword{સ્પર્ધા}: વૈશ્વિક mining pools સ્પર્ધા કરે છે
\end{itemize}
\end{solutionbox}

\begin{mnemonicbox}
ચકાસણી હલ પુરસ્કાર (CHP)
\end{mnemonicbox}

\orquestionmarks{5(b)}{4}{Short Note on: Decentralized Autonomous Organization (DAO)}

\begin{solutionbox}
\textbf{જવાબ}:

\begin{center}
\captionof{table}{DAO લક્ષણો}
\begin{tabulary}{\linewidth}{L L}
    \toprule
    \textbf{લક્ષણ} & \textbf{વર્ણન} \\
    \midrule
    \textbf{ગવર્નન્સ} & સમુદાય-સંચાલિત નિર્ણયો \\
    \textbf{મતદાન} & Token-આધારિત મતદાન અધિકારો \\
    \textbf{સ્વચાલન} & Smart contracts નિર્ણયો અમલ કરે છે \\
    \textbf{પારદર્શિતા} & બધી પ્રવૃત્તિઓ બ્લોકચેન પર \\
    \bottomrule
\end{tabulary}
\end{center}

\textbf{મુખ્ય લાક્ષણિકતાઓ:}

\begin{itemize}
    \item \keyword{કોઈ કેન્દ્રીય સત્તા નથી}: સમુદાય નિયંત્રિત
    \item \keyword{Token માલિકી}: Tokens આધારે મતદાન શક્તિ
    \item \keyword{સ્વચાલિત અમલીકરણ}: મંજૂર પ્રસ્તાવો સ્વચાલિત અમલ થાય છે
\end{itemize}
\end{solutionbox}

\begin{mnemonicbox}
સમુદાય મત આપે સ્વચાલિત (SMS)
\end{mnemonicbox}

\orquestionmarks{5(c)}{7}{Explain ERC-20 with its advantages and disadvantages}

\begin{solutionbox}
\textbf{જવાબ}:

\begin{center}
\captionof{table}{ERC-20 Standard ઝાંખી}
\begin{tabulary}{\linewidth}{L L}
    \toprule
    \textbf{પાસું} & \textbf{વર્ણન} \\
    \midrule
    \textbf{પૂરું નામ} & Ethereum Request for Comments 20 \\
    \textbf{પ્રકાર} & Ethereum પર token standard \\
    \textbf{Functions} & માનકીકૃત token operations \\
    \textbf{સુસંગતતા} & બધા Ethereum wallets સાથે કામ કરે છે \\
    \bottomrule
\end{tabulary}
\end{center}

\textbf{ERC-20 Token Flow}

\begin{center}
    \begin{tikzpicture}[node distance=1.5cm, auto]
        \node [gtu block] (Contract) {Token Contract};
        \node [gtu block, right=of Contract] (Func) {Transfer Function};
        \node [gtu state, right=of Func] (Update) {Balances અપડેટ};
        \node [gtu state, below=of Update] (Emit) {Event Emit};
        \node [gtu block, left=of Emit] (Wallet) {Wallet અપડેટ};
        
        \draw [gtu arrow] (Contract) -- (Func);
        \draw [gtu arrow] (Func) -- (Update);
        \draw [gtu arrow] (Update) -- (Emit);
        \draw [gtu arrow] (Emit) -- (Wallet);
    \end{tikzpicture}
    \captionof{figure}{ERC-20 Execution Flow}
\end{center}

\textbf{જરૂરી Functions:}

\begin{center}
\captionof{table}{જરૂરી Functions}
\begin{tabulary}{\linewidth}{L L}
    \toprule
    \textbf{Function} & \textbf{હેતુ} \\
    \midrule
    \textbf{totalSupply()} & કુલ token supply પરત કરે \\
    \textbf{balanceOf()} & Account balance ચકાસે \\
    \textbf{transfer()} & Address પર tokens મોકલે \\
    \textbf{approve()} & વતી ખર્ચની મંજૂરી આપે \\
    \bottomrule
\end{tabulary}
\end{center}

\textbf{ફાયદા:}

\begin{itemize}
    \item \keyword{માનકીકરણ}: બધા tokens માટે એકસમાન interface
    \item \keyword{Interoperability}: કોઈ પણ Ethereum wallet/exchange સાથે કામ કરે છે
    \item \keyword{Liquidity}: Decentralized exchanges પર ટ્રેડ કરી શકાય છે
\end{itemize}

\textbf{ગેરફાયદા:}

\begin{itemize}
    \item \keyword{Gas Fees}: Ethereum transaction ખર્ચ
    \item \keyword{સ્કેલેબિલિટી}: નેટવર્ક congestion સમસ્યાઓ
    \item \keyword{સુરક્ષા}: Smart contract vulnerabilities
\end{itemize}
\end{solutionbox}

\begin{mnemonicbox}
Standard Tokens Trade Everywhere (STTE)
\end{mnemonicbox}

\end{document}
