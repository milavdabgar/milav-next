\documentclass[10pt,a4paper]{article}

% content/resources/templates/preamble.tex
\usepackage[margin=0.6in]{geometry}
\author{Milav Dabgar}
\usepackage{amsmath,amssymb,amsthm}
\usepackage{booktabs}
\usepackage{multirow}
\usepackage{xcolor}
\usepackage{tcolorbox}
\tcbuselibrary{breakable,skins}
\usepackage[colorlinks=true,linkcolor=blue]{hyperref}
\usepackage{titlesec}
\usepackage{enumitem}
\usepackage{tikz}
\usepackage{pgfplots}
\usepackage{circuitikz}
\usepackage[version=4]{mhchem}
\usepackage{longtable}
\usepackage{array}
\usepackage{float}
\usepackage{caption}
\usepackage{listings}

\lstset{
  basicstyle=\small\ttfamily,
  breaklines=true,
  breakatwhitespace=false,
  postbreak=\mbox{\textcolor{red}{$\hookrightarrow$}\space},
  float=false,
  numbers=left,
  numberstyle=\tiny\color{gray},
  numbersep=10pt,
  xleftmargin=2em,
  keywordstyle=\color{blue},
  commentstyle=\color{green!60!black},
  stringstyle=\color{purple},
  backgroundcolor=\color{gray!5},
  showstringspaces=false,
  tabsize=2,
  captionpos=b,
  keepspaces=true,
  columns=flexible
}

\pgfplotsset{compat=1.18}
\usetikzlibrary{shapes,arrows,positioning,calc,patterns,decorations.pathmorphing,decorations.markings,arrows.meta}

% Color scheme
\definecolor{headcolor}{RGB}{0,102,204}
\definecolor{keycolor}{RGB}{220,20,60}
\definecolor{solutioncolor}{RGB}{34,139,34}
\definecolor{mnemoniccolor}{RGB}{148,0,211}
\definecolor{codecolor}{RGB}{0,0,100}

% Spacing
\setlength{\parskip}{3pt}
\setlist[itemize]{nosep}
\setlist[enumerate]{nosep}

% Title formatting
\titleformat{\section}{\Large\bfseries\color{headcolor}}{\thesection}{1em}{}
\titleformat{\subsection}{\large\bfseries\color{headcolor}}{\thesubsection}{1em}{}

% Pandoc tightlist compatibility
\providecommand{\tightlist}{%
  \setlength{\itemsep}{0pt}\setlength{\parskip}{0pt}}

% Pandoc longtable compatibility
\newcounter{none}
\def\thenone{}


% content/resources/templates/gujarati-boxes.tex
\usepackage{fontspec}
\usepackage{polyglossia}

% Set Gujarati as main language (document is primarily in Gujarati)
% Note: gloss-gujarati.ldf doesn't exist in polyglossia, but it will use hyphenation patterns
\setdefaultlanguage{gujarati}
\setotherlanguage{english}

% Configure Gujarati font properly
% Use Language=Default to prevent polyglossia from trying to add language-specific features
% that don't exist for Gujarati, which causes "empty feature" warnings
\newfontfamily\gujaratifont[Script=Gujarati,AutoFakeBold=2.5,AutoFakeSlant=0.3]{Noto Sans Gujarati}
\setmainfont[Script=Gujarati,AutoFakeBold=2.5,AutoFakeSlant=0.3]{Noto Sans Gujarati}
% Use Noto Sans Gujarati for monospace to support Gujarati in text
\setmonofont[Scale=0.9]{Noto Sans Gujarati}

% Configure English to use the same font
\newfontfamily\englishfont[Script=Gujarati,AutoFakeBold=2.5,AutoFakeSlant=0.3]{Noto Sans Gujarati}

% Translations for polyglossia
\gappto\captionsgujarati{
  \renewcommand{\tablename}{કોષ્ટક}
  \renewcommand{\figurename}{આકૃતિ}
}

% Helper for TikZ nodes to ensure Gujarati font
\newcommand{\gu}[1]{{\gujaratifont #1}}

% Custom environments
\newtcolorbox{solutionbox}{
    breakable,
    enhanced,
    colback=solutioncolor!5!white,
    colframe=solutioncolor!75!black,
    fonttitle=\bfseries,
    title=જવાબ
}

\newtcolorbox{solutionboxnobreak}{
 colback=solutioncolor!5!white,
 colframe=solutioncolor!75!black,
 fonttitle=\bfseries,
 title=જવાબ
}

\newtcolorbox{keyformula}{
 breakable,
 enhanced,
 colback=keycolor!5!white,
 colframe=keycolor!75!black,
 fonttitle=\bfseries,
 title=રાસાયણિક સમીકરણ/સૂત્ર
}

\newtcolorbox{mnemonicbox}{
 breakable,
 enhanced,
 colback=mnemoniccolor!5!white,
 colframe=mnemoniccolor!75!black,
 fonttitle=\bfseries,
 title=મેમરી ટ્રીક
}


\begin{document}

\begin{center}
{\Huge\bfseries\color{headcolor} Subject Name (Gujarati)}\\[5pt]
{\LARGE 4361603 -- Summer 2025}\\[3pt]
{\large Semester 1 Study Material}\\[3pt]
{\normalsize\textit{Detailed Solutions and Explanations}}
\end{center}

\vspace{10pt}

\subsection*{પ્રશ્ન 1(અ) [3
ગુણ]}\label{uxaaauxab0uxab6uxaa8-1uxa85-3-uxa97uxaa3}

\textbf{બ્લોકચેનમાં Private key અને Public key નો તફાવત આપો.}

\begin{solutionbox}

{\def\LTcaptype{none} % do not increment counter
\begin{longtable}[]{@{}
  >{\raggedright\arraybackslash}p{(\linewidth - 4\tabcolsep) * \real{0.2326}}
  >{\raggedright\arraybackslash}p{(\linewidth - 4\tabcolsep) * \real{0.3953}}
  >{\raggedright\arraybackslash}p{(\linewidth - 4\tabcolsep) * \real{0.3721}}@{}}
\toprule\noalign{}
\begin{minipage}[b]{\linewidth}\raggedright
\textbf{બાબત}
\end{minipage} & \begin{minipage}[b]{\linewidth}\raggedright
\textbf{Private Key}
\end{minipage} & \begin{minipage}[b]{\linewidth}\raggedright
\textbf{Public Key}
\end{minipage} \\
\midrule\noalign{}
\endhead
\bottomrule\noalign{}
\endlastfoot
\textbf{હેતુ} & Transaction sign કરવા માટે & Verification માટે ઉપયોગ \\
\textbf{શેરિંગ} & ગુપ્ત રાખવી જોઈએ & બધાને આપી શકાય \\
\textbf{કામ} & Data decrypt કરે, signature બનાવે & Data encrypt કરે,
signature verify કરે \\
\textbf{માલિકી} & ફક્ત માલિક જ જાણે & બધા access કરી શકે \\
\end{longtable}
}

\begin{itemize}
\tightlist
\item
  \textbf{Private Key}: ગુપ્ત mathematical code જે ownership સાબિત કરે
\item
  \textbf{Public Key}: ખુલ્લું address જેથી બીજા transaction મોકલી શકે
\item
  \textbf{સુરક્ષા}: Private key ગુમાવવી = પૈસા હંમેશ માટે ગુમાવવા
\end{itemize}

\end{solutionbox}
\begin{mnemonicbox}
``Private છે Personal, Public છે Posted''

\end{mnemonicbox}
\begin{center}\rule{0.5\linewidth}{0.5pt}\end{center}

\subsection*{પ્રશ્ન 1(બ) [4
ગુણ]}\label{uxaaauxab0uxab6uxaa8-1uxaac-4-uxa97uxaa3}

\textbf{Distributed Ledger ને વિગતવાર સમજાવો.}

\begin{solutionbox}

\textbf{Distributed Ledger} એ database છે જે ઘણી જગ્યાએ અને ઘણા લોકોમાં
વહેંચાયેલું હોય છે.

\textbf{મુખ્ય લક્ષણો:}

{\def\LTcaptype{none} % do not increment counter
\begin{longtable}[]{@{}ll@{}}
\toprule\noalign{}
\textbf{લક્ષણ} & \textbf{વર્ણન} \\
\midrule\noalign{}
\endhead
\bottomrule\noalign{}
\endlastfoot
\textbf{Decentralized} & કોઈ એક control point નથી \\
\textbf{Synchronized} & બધી copies updated રહે છે \\
\textbf{Transparent} & બધા participants જોઈ શકે છે \\
\textbf{Immutable} & સહેલાઈથી બદલાતું નથી \\
\end{longtable}
}

\textbf{આકૃતિ:}

\begin{center}
\textbf{Mermaid Diagram (Code)}
\begin{verbatim}
{Shaded}
{Highlighting}[]
graph TD
    A[Participant 1] {-{-}{} D[Distributed Ledger]}
    B[Participant 2] {-{-}{} D}
    C[Participant 3] {-{-}{} D}
    D {-{-}{} E[Synchronized Copy 1]}
    D {-{-}{} F[Synchronized Copy 2]}
    D {-{-}{} G[Synchronized Copy 3]}
{Highlighting}
{Shaded}
\end{verbatim}
\end{center}

\begin{itemize}
\tightlist
\item
  \textbf{ફાયદા}: Intermediaries નાબૂદ કરે, trust વધારે, fraud ઓછું
\item
  \textbf{કામ}: બધા participants પાસે records ની identical copies હોય
\end{itemize}

\end{solutionbox}
\begin{mnemonicbox}
``Distributed = વિભાજિત પણ સમાન''

\end{mnemonicbox}
\begin{center}\rule{0.5\linewidth}{0.5pt}\end{center}

\subsection*{પ્રશ્ન 1(ક) [7
ગુણ]}\label{uxaaauxab0uxab6uxaa8-1uxa95-7-uxa97uxaa3}

\textbf{Blockchain વ્યાખ્યાયિત કરો. Blockchain ની એપ્લિકેશનો અને મર્યાદાઓનાં
વર્ણન કરો.}

\begin{solutionbox}

\textbf{Blockchain વ્યાખ્યા}: Transaction records ધરાવતા blocks નો chain જે
cryptography વાપરીને જોડાયેલા હોય.

\textbf{એપ્લિકેશન કોષ્ટક:}

{\def\LTcaptype{none} % do not increment counter
\begin{longtable}[]{@{}lll@{}}
\toprule\noalign{}
\textbf{ક્ષેત્ર} & \textbf{એપ્લિકેશન} & \textbf{ફાયદો} \\
\midrule\noalign{}
\endhead
\bottomrule\noalign{}
\endlastfoot
\textbf{Finance} & Cryptocurrency, payments & ઝડપી, સસ્તી transfers \\
\textbf{Healthcare} & Patient records & સુરક્ષિત, accessible data \\
\textbf{Supply Chain} & Product tracking & પારદર્શિતા, authenticity \\
\textbf{Real Estate} & Property records & Fraud prevention \\
\textbf{Voting} & Digital elections & પારદર્શી, tamper-proof \\
\end{longtable}
}

\textbf{મર્યાદાઓ કોષ્ટક:}

{\def\LTcaptype{none} % do not increment counter
\begin{longtable}[]{@{}ll@{}}
\toprule\noalign{}
\textbf{મર્યાદા} & \textbf{અસર} \\
\midrule\noalign{}
\endhead
\bottomrule\noalign{}
\endlastfoot
\textbf{Scalability} & ધીમી transaction processing \\
\textbf{Energy Usage} & વધુ electricity વપરાશ \\
\textbf{Complexity} & Users માટે સમજવું મુશ્કેલ \\
\textbf{Regulation} & કાયદાકીય અસ્પષ્ટતા \\
\textbf{Storage} & વધતો data size ની સમસ્યા \\
\end{longtable}
}

\textbf{આર્કિટેક્ચર આકૃતિ:}

\begin{center}
\textbf{Mermaid Diagram (Code)}
\begin{verbatim}
{Shaded}
{Highlighting}[]
graph LR
    A[Block 1] {-{-}{} B[Block 2]}
    B {-{-}{} C[Block 3]}
    C {-{-}{} D[Block 4]}
    
    A1[Hash] {-{-}{} A}
    B1[Hash] {-{-}{} B}
    C1[Hash] {-{-}{} C}
    D1[Hash] {-{-}{} D}
{Highlighting}
{Shaded}
\end{verbatim}
\end{center}

\begin{itemize}
\tightlist
\item
  \textbf{સુરક્ષા}: Cryptographic linking થી tampering મુશ્કેલ
\item
  \textbf{પારદર્શિતા}: બધા transactions network participants ને દેખાય
\end{itemize}

\end{solutionbox}
\begin{mnemonicbox}
``Blocks Chained = Blockchain, Apps ઘણી = Limits
ઘણી''

\end{mnemonicbox}
\begin{center}\rule{0.5\linewidth}{0.5pt}\end{center}

\subsection*{પ્રશ્ન 1(ક) OR [7
ગુણ]}\label{uxaaauxab0uxab6uxaa8-1uxa95-or-7-uxa97uxaa3}

\textbf{ટૂંકી નોંધ લખો: બ્લોકચેનમાં CAP Theorem}

\begin{solutionbox}

\textbf{CAP Theorem} કહે છે કે distributed systems એ 3 properties માંથી માત્ર
2 જ simultaneously guarantee કરી શકે.

\textbf{CAP Components કોષ્ટક:}

{\def\LTcaptype{none} % do not increment counter
\begin{longtable}[]{@{}
  >{\raggedright\arraybackslash}p{(\linewidth - 4\tabcolsep) * \real{0.3714}}
  >{\raggedright\arraybackslash}p{(\linewidth - 4\tabcolsep) * \real{0.2857}}
  >{\raggedright\arraybackslash}p{(\linewidth - 4\tabcolsep) * \real{0.3429}}@{}}
\toprule\noalign{}
\begin{minipage}[b]{\linewidth}\raggedright
\textbf{Property}
\end{minipage} & \begin{minipage}[b]{\linewidth}\raggedright
\textbf{વર્ણન}
\end{minipage} & \begin{minipage}[b]{\linewidth}\raggedright
\textbf{ઉદાહરણ}
\end{minipage} \\
\midrule\noalign{}
\endhead
\bottomrule\noalign{}
\endlastfoot
\textbf{Consistency} & બધા nodes પાસે same data & બર્યાને જગ્યાએ same
balance દેખાય \\
\textbf{Availability} & System હંમેશા response આપે & Network કદી down ન
જાય \\
\textbf{Partition Tolerance} & Network failures છતાં કામ કરે & Nodes
disconnect થયા છતાં function કરે \\
\end{longtable}
}

\textbf{Blockchain Trade-offs:}

\begin{center}
\textbf{Mermaid Diagram (Code)}
\begin{verbatim}
{Shaded}
{Highlighting}[]
graph TD
    A[CAP Theorem] {-{-}{} B[Consistency]}
    A {-{-}{} C[Availability]}
    A {-{-}{} D[Partition Tolerance]}
    
    E[Bitcoin] {-{-}{} B}
    E {-{-}{} D}
    F[Private Blockchain] {-{-}{} B}
    F {-{-}{} C}
{Highlighting}
{Shaded}
\end{verbatim}
\end{center}

\textbf{વાસ્તવિક ઉપયોગ:}

{\def\LTcaptype{none} % do not increment counter
\begin{longtable}[]{@{}
  >{\raggedright\arraybackslash}p{(\linewidth - 4\tabcolsep) * \real{0.4222}}
  >{\raggedright\arraybackslash}p{(\linewidth - 4\tabcolsep) * \real{0.2889}}
  >{\raggedright\arraybackslash}p{(\linewidth - 4\tabcolsep) * \real{0.2889}}@{}}
\toprule\noalign{}
\begin{minipage}[b]{\linewidth}\raggedright
\textbf{Blockchain Type}
\end{minipage} & \begin{minipage}[b]{\linewidth}\raggedright
\textbf{પસંદ કરે}
\end{minipage} & \begin{minipage}[b]{\linewidth}\raggedright
\textbf{ત્યાગ કરે}
\end{minipage} \\
\midrule\noalign{}
\endhead
\bottomrule\noalign{}
\endlastfoot
\textbf{Bitcoin} & Consistency + Partition & Availability \\
\textbf{Ethereum} & Consistency + Partition & Availability \\
\textbf{Private Networks} & Consistency + Availability & Partition
Tolerance \\
\end{longtable}
}

\begin{itemize}
\tightlist
\item
  \textbf{અસર}: Blockchain designers એ કયા property sacrifice કરવી તે
  choose કરવું પડે
\item
  \textbf{Trade-off}: Distributed networks માં perfect systems અશક્ય
\end{itemize}

\end{solutionbox}
\begin{mnemonicbox}
``કમ્પ્લીટ સિસ્ટમ શક્ય નથી - 3 માંથી 2 જ પસંદ કરો''

\end{mnemonicbox}
\begin{center}\rule{0.5\linewidth}{0.5pt}\end{center}

\subsection*{પ્રશ્ન 2(અ) [3
ગુણ]}\label{uxaaauxab0uxab6uxaa8-2uxa85-3-uxa97uxaa3}

\textbf{બ્લોકચેનનાં Data Structure સમજાવો.}

\begin{solutionbox}

\textbf{Blockchain Data Structure} transaction data ધરાવતા linked blocks
ધાયેલું હોય છે.

\textbf{Block Structure કોષ્ટક:}

{\def\LTcaptype{none} % do not increment counter
\begin{longtable}[]{@{}ll@{}}
\toprule\noalign{}
\textbf{Component} & \textbf{હેતુ} \\
\midrule\noalign{}
\endhead
\bottomrule\noalign{}
\endlastfoot
\textbf{Block Header} & Metadata રાખે છે \\
\textbf{Previous Hash} & Previous block સાથે link કરે \\
\textbf{Merkle Root} & બધા transactions નો summary \\
\textbf{Timestamp} & Block કયારે બન્યો તેની માહિતી \\
\textbf{Transactions} & વાસ્તવિક data/transfers \\
\end{longtable}
}

\textbf{Visual Structure:}

\begin{verbatim}
+{-{-}{-}{-}{-}{-}{-}{-}{-}{-}{-}{-}{-}{-}{-}{-}{-}{-}+}
|   Block Header   |
|{-{-}{-}{-}{-}{-}{-}{-}{-}{-}{-}{-}{-}{-}{-}{-}{-}{-}|}
| Previous Hash    |
| Merkle Root      |
| Timestamp        |
| Nonce            |
+{-{-}{-}{-}{-}{-}{-}{-}{-}{-}{-}{-}{-}{-}{-}{-}{-}{-}+}
|   Transactions   |
|  [TX1, TX2, TX3] |
+{-{-}{-}{-}{-}{-}{-}{-}{-}{-}{-}{-}{-}{-}{-}{-}{-}{-}+}
\end{verbatim}

\begin{itemize}
\tightlist
\item
  \textbf{Linking}: દરેક block previous block ને hash વાપરીને point કરે
\item
  \textbf{Integrity}: એક block બદલાવવાથી આખી chain ટૂટી જાય
\end{itemize}

\end{solutionbox}
\begin{mnemonicbox}
``Header હોય છે, Transactions વાત કરે છે''

\end{mnemonicbox}
\begin{center}\rule{0.5\linewidth}{0.5pt}\end{center}

\subsection*{પ્રશ્ન 2(બ) [4
ગુણ]}\label{uxaaauxab0uxab6uxaa8-2uxaac-4-uxa97uxaa3}

\textbf{Decentralization ના ફાયદા શું છે?}

\begin{solutionbox}

\textbf{Decentralization ફાયદા:}

{\def\LTcaptype{none} % do not increment counter
\begin{longtable}[]{@{}
  >{\raggedright\arraybackslash}p{(\linewidth - 2\tabcolsep) * \real{0.4545}}
  >{\raggedright\arraybackslash}p{(\linewidth - 2\tabcolsep) * \real{0.5455}}@{}}
\toprule\noalign{}
\begin{minipage}[b]{\linewidth}\raggedright
\textbf{ફાયદો}
\end{minipage} & \begin{minipage}[b]{\linewidth}\raggedright
\textbf{સમજૂતી}
\end{minipage} \\
\midrule\noalign{}
\endhead
\bottomrule\noalign{}
\endlastfoot
\textbf{No Single Point of Failure} & એક node fail થયા છતાં network ચાલુ
રહે \\
\textbf{Censorship Resistance} & કોઈ authority transactions block કરી શકે
નહિ \\
\textbf{Transparency} & બધા participants સમાન માહિતી જુએ છે \\
\textbf{Reduced Costs} & Intermediary fees નાબૂદ થાય \\
\textbf{Trust} & Central authority પર trust કરવાની જરૂર નથી \\
\end{longtable}
}

\textbf{સરખામણી આકૃતિ:}

\begin{center}
\textbf{Mermaid Diagram (Code)}
\begin{verbatim}
{Shaded}
{Highlighting}[]
graph TD
    subgraph Centralized
        A[Central Authority] {-{-}{} B[User 1]}
        A {-{-}{} C[User 2]}
        A {-{-}{} D[User 3]}
    end
    
    subgraph Decentralized
            direction LR
        E[User 1] {-{-}{} F[User 2]}
        F {-{-}{} G[User 3]}
        G {-{-}{} E}
    end
{Highlighting}
{Shaded}
\end{verbatim}
\end{center}

\begin{itemize}
\tightlist
\item
  \textbf{સુરક્ષા}: Multiple copies થી data loss અટકે
\item
  \textbf{લોકશાહી}: બધા participants ને સમાન અધિકાર
\item
  \textbf{મજબૂતાઈ}: Individual failures સામે system ટકે
\end{itemize}

\end{solutionbox}
\begin{mnemonicbox}
``વિકેન્દ્રિત = ટકાઉ, લોકશાહી, પ્રત્યક્ષ''

\end{mnemonicbox}
\begin{center}\rule{0.5\linewidth}{0.5pt}\end{center}

\subsection*{પ્રશ્ન 2(ક) [7
ગુણ]}\label{uxaaauxab0uxab6uxaa8-2uxa95-7-uxa97uxaa3}

\textbf{Public બ્લોકચેન અને Private બ્લોકચેન વચ્ચે તફાવત કરો.}

\begin{solutionbox}

\textbf{વ્યાપક સરખામણી:}

{\def\LTcaptype{none} % do not increment counter
\begin{longtable}[]{@{}lll@{}}
\toprule\noalign{}
\textbf{બાબત} & \textbf{Public Blockchain} & \textbf{Private
Blockchain} \\
\midrule\noalign{}
\endhead
\bottomrule\noalign{}
\endlastfoot
\textbf{Access} & બધા માટે ખુલ્લું & ખાસ users માટે મર્યાદિત \\
\textbf{Permission} & Permission ની જરૂર નથી & Permission આવશ્યક \\
\textbf{Control} & Decentralized & Centralized control \\
\textbf{Speed} & ધીમું (consensus જરૂરી) & ઝડપી (ઓછા validators) \\
\textbf{Security} & ઊંચી (ઘણા validators) & મધ્યમ (ઓછા validators) \\
\textbf{Cost} & Transaction fees જરૂરી & ઓછી operational costs \\
\textbf{Transparency} & સંપૂર્ણ પારદર્શિતા & મર્યાદિત પારદર્શિતા \\
\textbf{ઉદાહરણ} & Bitcoin, Ethereum & Hyperledger, R3 Corda \\
\end{longtable}
}

\textbf{Network આર્કિટેક્ચર:}

\begin{center}
\textbf{Mermaid Diagram (Code)}
\begin{verbatim}
{Shaded}
{Highlighting}[]
graph TD
    subgraph "Public Blockchain"
        A[કોઈપણ] {-{-}{} B[Global Network]}
        C[કોઈપણ] {-{-}{} B}
        D[કોઈપણ] {-{-}{} B}
    end
    
    subgraph "Private Blockchain"
        E[Authorized User 1] {-{-}{} F[Private Network]}
        G[Authorized User 2] {-{-}{} F}
        H[Authorized User 3] {-{-}{} F}
    end
{Highlighting}
{Shaded}
\end{verbatim}
\end{center}

\textbf{ઉપયોગ કેસો:}

{\def\LTcaptype{none} % do not increment counter
\begin{longtable}[]{@{}ll@{}}
\toprule\noalign{}
\textbf{પ્રકાર} & \textbf{શ્રેષ્ઠ છે} \\
\midrule\noalign{}
\endhead
\bottomrule\noalign{}
\endlastfoot
\textbf{Public} & Cryptocurrencies, public records \\
\textbf{Private} & Banking, supply chain, healthcare \\
\end{longtable}
}

\begin{itemize}
\tightlist
\item
  \textbf{Trade-offs}: Public વધુ security આપે, Private વધુ control આપે
\item
  \textbf{પસંદગી}: Transparency vs.~privacy ની જરૂરિયાત પર નિર્ભર
\end{itemize}

\end{solutionbox}
\begin{mnemonicbox}
``Public = લોકોનું, Private = મંજૂરીવાળું''

\end{mnemonicbox}
\begin{center}\rule{0.5\linewidth}{0.5pt}\end{center}

\subsection*{પ્રશ્ન 2(અ) OR [3
ગુણ]}\label{uxaaauxab0uxab6uxaa8-2uxa85-or-3-uxa97uxaa3}

\textbf{યોગ્ય આકૃતિ સાથે બ્લોક ચેઇનના Core Components નાં વર્ણન કરો.}

\begin{solutionbox}

\textbf{મુખ્ય Components:}

{\def\LTcaptype{none} % do not increment counter
\begin{longtable}[]{@{}ll@{}}
\toprule\noalign{}
\textbf{Component} & \textbf{કામ} \\
\midrule\noalign{}
\endhead
\bottomrule\noalign{}
\endlastfoot
\textbf{Blocks} & Transaction data store કરે \\
\textbf{Hash Functions} & Unique fingerprints બનાવે \\
\textbf{Digital Signatures} & Transaction authenticity verify કરે \\
\textbf{Consensus Mechanism} & Valid transactions પર સંમતિ કરે \\
\textbf{Peer-to-Peer Network} & બધા participants ને connect કરે \\
\end{longtable}
}

\textbf{System આર્કિટેક્ચર:}

\begin{center}
\textbf{Mermaid Diagram (Code)}
\begin{verbatim}
{Shaded}
{Highlighting}[]
graph TD
    A[Peer{-to{-}Peer Network] {-}{-}{} B[Consensus Mechanism]}
    B {-{-}{} C[Block Creation]}
    C {-{-}{} D[Hash Functions]}
    D {-{-}{} E[Digital Signatures]}
    E {-{-}{} F[Transaction Validation]}
    F {-{-}{} G[Block Addition]}
    G {-{-}{} H[Blockchain Updated]}
{Highlighting}
{Shaded}
\end{verbatim}
\end{center}

\begin{itemize}
\tightlist
\item
  \textbf{એકીકરણ}: બધા components મળીને security માટે કામ કરે
\item
  \textbf{હેતુ}: દરેક component ખાસ blockchain function serve કરે
\end{itemize}

\end{solutionbox}
\begin{mnemonicbox}
``Blocks બનાવે, Hash પકડે, Signatures સુરક્ષિત કરે''

\end{mnemonicbox}
\begin{center}\rule{0.5\linewidth}{0.5pt}\end{center}

\subsection*{પ્રશ્ન 2(બ) OR [4
ગુણ]}\label{uxaaauxab0uxab6uxaa8-2uxaac-or-4-uxa97uxaa3}

\textbf{Permissioned blockchain ને વ્યાખ્યાયિત કરો અને વિગતવાર સમજાવો.}

\begin{solutionbox}

\textbf{Permissioned Blockchain વ્યાખ્યા}: એવી blockchain જેમાં
participation માટે network administrators પાસેથી સ્પષ્ટ permission જરૂરી હોય.

\textbf{લક્ષણો કોષ્ટક:}

{\def\LTcaptype{none} % do not increment counter
\begin{longtable}[]{@{}ll@{}}
\toprule\noalign{}
\textbf{લક્ષણ} & \textbf{વર્ણન} \\
\midrule\noalign{}
\endhead
\bottomrule\noalign{}
\endlastfoot
\textbf{Access Control} & ફક્ત approved users જ join કરી શકે \\
\textbf{Validation Rights} & પસંદગીના nodes જ transactions validate કરે \\
\textbf{Governance} & Central authority network manage કરે \\
\textbf{Privacy} & Transaction details private હોઈ શકે \\
\end{longtable}
}

\textbf{Permission સ્તરો:}

\begin{center}
\textbf{Mermaid Diagram (Code)}
\begin{verbatim}
{Shaded}
{Highlighting}[]
graph TD
    A[Network Administrator] {-{-}{} B[Full Access]}
    A {-{-}{} C[Read/Write Access]}
    A {-{-}{} D[Read Only Access]}
    A {-{-}{} E[No Access]}
    
    B {-{-}{} F[Blocks validate કરી શકે]}
    C {-{-}{} G[Transactions submit કરી શકે]}
    D {-{-}{} H[ફક્ત data જોઈ શકે]}
    E {-{-}{} I[Network થી blocked]}
{Highlighting}
{Shaded}
\end{verbatim}
\end{center}

\begin{itemize}
\tightlist
\item
  \textbf{ફાયદા}: બહેતર privacy, regulatory compliance, ઝડપી processing
\item
  \textbf{ગેરફાયદા}: ઓછું decentralized, administrators પર trust આવશ્યક
\end{itemize}

\end{solutionbox}
\begin{mnemonicbox}
``Permission = Participation માટે મંજૂરી''

\end{mnemonicbox}
\begin{center}\rule{0.5\linewidth}{0.5pt}\end{center}

\subsection*{પ્રશ્ન 2(ક) OR [7
ગુણ]}\label{uxaaauxab0uxab6uxaa8-2uxa95-or-7-uxa97uxaa3}

\textbf{Sidechain ને સંક્ષિપ્તમાં સમજાવો.}

\begin{solutionbox}

\textbf{Sidechain વ્યાખ્યા}: Main blockchain સાથે connected અલગ blockchain
જે chains વચ્ચે asset transfer કરવાની સુવિધા આપે.

\textbf{Sidechain આર્કિટેક્ચર:}

\begin{center}
\textbf{Mermaid Diagram (Code)}
\begin{verbatim}
{Shaded}
{Highlighting}[]
graph LR
    A[Main Chain] {{-}{-}{} B[Sidechain 1]}
    A {{-}{-}{} C[Sidechain 2]}
    A {{-}{-}{} D[Sidechain 3]}
    
    B {-{-}{} E[Specific Purpose 1]}
    C {-{-}{} F[Specific Purpose 2]}
    D {-{-}{} G[Specific Purpose 3]}
{Highlighting}
{Shaded}
\end{verbatim}
\end{center}

\textbf{ફાયદા અને લક્ષણો:}

{\def\LTcaptype{none} % do not increment counter
\begin{longtable}[]{@{}ll@{}}
\toprule\noalign{}
\textbf{બાબત} & \textbf{ફાયદો} \\
\midrule\noalign{}
\endhead
\bottomrule\noalign{}
\endlastfoot
\textbf{Scalability} & Main chain નો load ઘટાડે \\
\textbf{Experimentation} & નવા features સુરક્ષિત રીતે test કરે \\
\textbf{Specialization} & ખાસ use cases માટે optimized \\
\textbf{Interoperability} & અલગ અલગ blockchains ને connect કરે \\
\end{longtable}
}

\textbf{Transfer Process:}

{\def\LTcaptype{none} % do not increment counter
\begin{longtable}[]{@{}ll@{}}
\toprule\noalign{}
\textbf{પગલું} & \textbf{ક્રિયા} \\
\midrule\noalign{}
\endhead
\bottomrule\noalign{}
\endlastfoot
\textbf{1. Lock} & Main chain પર assets lock કરાય \\
\textbf{2. Proof} & Cryptographic proof generate કરાય \\
\textbf{3. Release} & Sidechain પર equivalent assets release કરાય \\
\textbf{4. Use} & Sidechain પર assets ઉપયોગ કરાય \\
\textbf{5. Return} & Assets પાછા લાવવા માટે reverse process \\
\end{longtable}
}

\textbf{વાસ્તવિક ઉદાહરણો:}

{\def\LTcaptype{none} % do not increment counter
\begin{longtable}[]{@{}ll@{}}
\toprule\noalign{}
\textbf{Sidechain} & \textbf{હેતુ} \\
\midrule\noalign{}
\endhead
\bottomrule\noalign{}
\endlastfoot
\textbf{Lightning Network} & ઝડપી Bitcoin payments \\
\textbf{Plasma} & Ethereum scaling \\
\textbf{Liquid} & Bitcoin trading \\
\end{longtable}
}

\begin{itemize}
\tightlist
\item
  \textbf{સુરક્ષા}: Secure main chain સાથેનું connection જાળવે
\item
  \textbf{લવચિકતા}: દરેક sidechain ના અલગ rules હોઈ શકે
\item
  \textbf{નવીનતા}: Blockchain ecosystem વિસ્તરણ માટે
\end{itemize}

\end{solutionbox}
\begin{mnemonicbox}
``Side સહાય કરે, Main જાળવે''

\end{mnemonicbox}
\begin{center}\rule{0.5\linewidth}{0.5pt}\end{center}

\subsection*{પ્રશ્ન 3(અ) [3
ગુણ]}\label{uxaaauxab0uxab6uxaa8-3uxa85-3-uxa97uxaa3}

\textbf{Consensus Mechanism ને વ્યાખ્યાયિત કરો અને કોઈપણ એકને વિગતવાર
સમજાવો.}

\begin{solutionbox}

\textbf{Consensus Mechanism વ્યાખ્યા}: એક protocol જે ખાતરી કરે કે બધા
network participants blockchain ની current state પર સંમત હોય.

\textbf{Proof of Work (PoW) સમજૂતી:}

{\def\LTcaptype{none} % do not increment counter
\begin{longtable}[]{@{}ll@{}}
\toprule\noalign{}
\textbf{Component} & \textbf{કામ} \\
\midrule\noalign{}
\endhead
\bottomrule\noalign{}
\endlastfoot
\textbf{Mining} & જટિલ mathematical puzzles solve કરવું \\
\textbf{Competition} & Miners વચ્ચે પહેલા solve કરવાની સ્પર્ધા \\
\textbf{Verification} & Network solution verify કરે \\
\textbf{Reward} & Winner ને cryptocurrency reward મળે \\
\end{longtable}
}

\textbf{PoW Process:}

\begin{center}
\textbf{Mermaid Diagram (Code)}
\begin{verbatim}
{Shaded}
{Highlighting}[]
graph LR
    A[New Transaction] {-{-}{} B[Miners Collect Transactions]}
    B {-{-}{} C[Create Block]}
    C {-{-}{} D[Solve Mathematical Puzzle]}
    D {-{-}{} E[First Solution Wins]}
    E {-{-}{} F[Block Added to Chain]}
    F {-{-}{} G[Miner Gets Reward]}
{Highlighting}
{Shaded}
\end{verbatim}
\end{center}

\begin{itemize}
\tightlist
\item
  \textbf{સુરક્ષા}: Computational work થી tampering મોંઘું બને
\item
  \textbf{ઉદાહરણ}: Bitcoin Proof of Work consensus વાપરે
\end{itemize}

\end{solutionbox}
\begin{mnemonicbox}
``Consensus = સામાન્ય બુદ્ધિ, Work = જીત''

\end{mnemonicbox}
\begin{center}\rule{0.5\linewidth}{0.5pt}\end{center}

\subsection*{પ્રશ્ન 3(બ) [4
ગુણ]}\label{uxaaauxab0uxab6uxaa8-3uxaac-4-uxa97uxaa3}

\textbf{બ્લોકચેનમાં Forking શા માટે જરૂરી છે? બ્લોકચેનમાં વિવિધ પ્રકારના Forks ની
યાદી બનાવો.}

\begin{solutionbox}

\textbf{Forking કેમ જરૂરી:}

{\def\LTcaptype{none} % do not increment counter
\begin{longtable}[]{@{}ll@{}}
\toprule\noalign{}
\textbf{કારણ} & \textbf{હેતુ} \\
\midrule\noalign{}
\endhead
\bottomrule\noalign{}
\endlastfoot
\textbf{Upgrades} & Blockchain માં નવા features add કરવા \\
\textbf{Bug Fixes} & Security vulnerabilities સુધારવા \\
\textbf{Rule Changes} & Consensus rules modify કરવા \\
\textbf{Community Disagreement} & Consensus ન મળે ત્યારે split કરવા \\
\end{longtable}
}

\textbf{Forks ના પ્રકારો:}

{\def\LTcaptype{none} % do not increment counter
\begin{longtable}[]{@{}lll@{}}
\toprule\noalign{}
\textbf{Fork Type} & \textbf{વર્ણન} & \textbf{Compatibility} \\
\midrule\noalign{}
\endhead
\bottomrule\noalign{}
\endlastfoot
\textbf{Soft Fork} & Rules tight કરે & Backward compatible \\
\textbf{Hard Fork} & Rules સંપૂર્ણ બદલે & Backward compatible નથી \\
\textbf{Accidental Fork} & અસ્થાયી split & આપોઆપ resolve થાય \\
\textbf{Contentious Fork} & Community disagreement & કાયમી split \\
\end{longtable}
}

\textbf{Fork વિઝ્યુઅલાઈઝેશન:}

\begin{center}
\textbf{Mermaid Diagram (Code)}
\begin{verbatim}
{Shaded}
{Highlighting}[]
graph LR
    A[Original Chain] {-{-}{} B[Block N]}
    B {-{-}{} C[Soft Fork {-} Tighter Rules]}
    B {-{-}{} D[Hard Fork {-} New Rules]}
    
    C {-{-}{} E[જૂના nodes હજુ કામ કરે]}
    D {-{-}{} F[જૂના nodes reject થાય]}
{Highlighting}
{Shaded}
\end{verbatim}
\end{center}

\begin{itemize}
\tightlist
\item
  \textbf{અસર}: Forks થી નવી cryptocurrencies બની શકે
\item
  \textbf{ઉદાહરણો}: Bitcoin Cash (hard fork), Ethereum updates (soft
  forks)
\end{itemize}

\end{solutionbox}
\begin{mnemonicbox}
``Fork = ભવિષ્યના વિકલ્પો, Rules જાળવાય''

\end{mnemonicbox}
\begin{center}\rule{0.5\linewidth}{0.5pt}\end{center}

\subsection*{પ્રશ્ન 3(ક) [7
ગુણ]}\label{uxaaauxab0uxab6uxaa8-3uxa95-7-uxa97uxaa3}

\textbf{Bitcoin Mining શું છે? Bitcoin Mining નાં કામકાજ, મુશ્કેલી અને ફાયદાઓ
વિશે વિગતવાર જણાવો.}

\begin{solutionbox}

\textbf{Bitcoin Mining વ્યાખ્યા}: Computational puzzles solve કરીને Bitcoin
blockchain માં નવા transactions add કરવાની પ્રક્રિયા.

\textbf{Mining Process:}

{\def\LTcaptype{none} % do not increment counter
\begin{longtable}[]{@{}lll@{}}
\toprule\noalign{}
\textbf{પગલું} & \textbf{ક્રિયા} & \textbf{વિગતો} \\
\midrule\noalign{}
\endhead
\bottomrule\noalign{}
\endlastfoot
\textbf{1. Collection} & Pending transactions ભેગા કરવા & Mempool માંથી \\
\textbf{2. Block Creation} & નવો block બનાવવો & Transactions સામેલ
કરવા \\
\textbf{3. Puzzle Solving} & સાચો nonce શોધવો & Trial and error \\
\textbf{4. Verification} & Network solution check કરે & Block validate
કરે \\
\textbf{5. Addition} & Chain માં block add કરવો & કાયમી record \\
\textbf{6. Reward} & Miner ને Bitcoin મળે & હાલમાં 6.25 BTC \\
\end{longtable}
}

\textbf{Mining Workflow:}

\begin{center}
\textbf{Mermaid Diagram (Code)}
\begin{verbatim}
{Shaded}
{Highlighting}[]
graph LR
    A[Pending Transactions] {-{-}{} B[Miners Collect]}
    B {-{-}{} C[Create Block Header]}
    C {-{-}{} D[Guess Nonce Value]}
    D {-{-}{} E[Calculate Hash]}
    E {-{-}{} F\{Hash {} Target?\}}
    F {-{-}{}|ના| D}
    F {-{-}{}|હા| G[Broadcast Solution]}
    G {-{-}{} H[Network Validates]}
    H {-{-}{} I[Block Added + Reward]}
{Highlighting}
{Shaded}
\end{verbatim}
\end{center}

\textbf{Difficulty Adjustment:}

{\def\LTcaptype{none} % do not increment counter
\begin{longtable}[]{@{}ll@{}}
\toprule\noalign{}
\textbf{બાબત} & \textbf{પદ્ધતિ} \\
\midrule\noalign{}
\endhead
\bottomrule\noalign{}
\endlastfoot
\textbf{Target Time} & દરેક block માટે 10 મિનિટ \\
\textbf{Adjustment Period} & દરેક 2016 blocks (\textasciitilde2
અઠવાડિયા) \\
\textbf{Auto-Regulation} & Blocks ઝડપી આવે તો વધારે \\
\textbf{હેતુ} & Consistent block time જાળવવું \\
\end{longtable}
}

\textbf{Mining ના ફાયદા:}

{\def\LTcaptype{none} % do not increment counter
\begin{longtable}[]{@{}ll@{}}
\toprule\noalign{}
\textbf{ફાયદો} & \textbf{વર્ણન} \\
\midrule\noalign{}
\endhead
\bottomrule\noalign{}
\endlastfoot
\textbf{Financial Reward} & Successful mining માટે Bitcoin કમાવવું \\
\textbf{Network Security} & વધુ miners = વધુ secure network \\
\textbf{Transaction Processing} & Bitcoin transfers શક્ય બનાવવું \\
\textbf{Decentralization} & Central authority ની જરૂર નથી \\
\end{longtable}
}

\begin{itemize}
\tightlist
\item
  \textbf{Energy}: Mining માં નોંધપાત્ર electricity જરૂરી
\item
  \textbf{Competition}: વધુ miners સાથે difficulty વધે
\item
  \textbf{Hardware}: Specialized ASIC miners સૌથી કાર્યક્ષમ
\end{itemize}

\end{solutionbox}
\begin{mnemonicbox}
``Mining = પૈસા, Math, Maintenance''

\end{mnemonicbox}
\begin{center}\rule{0.5\linewidth}{0.5pt}\end{center}

\subsection*{પ્રશ્ન 3(અ) OR [3
ગુણ]}\label{uxaaauxab0uxab6uxaa8-3uxa85-or-3-uxa97uxaa3}

\textbf{Soft fork અને Hard fork નો તફાવત આપો.}

\begin{solutionbox}

\textbf{Fork સરખામણી:}

{\def\LTcaptype{none} % do not increment counter
\begin{longtable}[]{@{}lll@{}}
\toprule\noalign{}
\textbf{બાબત} & \textbf{Soft Fork} & \textbf{Hard Fork} \\
\midrule\noalign{}
\endhead
\bottomrule\noalign{}
\endlastfoot
\textbf{Compatibility} & Backward compatible & Backward compatible
નથી \\
\textbf{Rules} & Rules વધુ સખત બનાવે & Rules સંપૂર્ણ બદલે \\
\textbf{Node Updates} & જૂના nodes માટે વૈકલ્પિક & બધા nodes માટે ફરજિયાત \\
\textbf{Chain Split} & કાયમી split નથી & કાયમી split કરી શકે \\
\textbf{Consensus} & Implement કરવું સરળ & Majority agreement જરૂરી \\
\textbf{ઉદાહરણો} & SegWit (Bitcoin) & Bitcoin Cash, Ethereum Classic \\
\end{longtable}
}

\textbf{વિઝ્યુઅલ રજૂઆત:}

\begin{center}
\textbf{Mermaid Diagram (Code)}
\begin{verbatim}
{Shaded}
{Highlighting}[]
graph LR
    A[Original Blockchain] {-{-}{} B[Fork Point]}
    B {-{-}{} C[Soft Fork {-} સખત Rules]}
    B {-{-}{} D[Hard Fork {-} નવા Rules]}
    
    C {-{-}{} E[જૂના nodes હજુ valid]}
    D {-{-}{} F[જૂના nodes incompatible]}
    
    E {-{-}{} G[એક જ chain ચાલુ]}
    F {-{-}{} H[બે અલગ chains]}
{Highlighting}
{Shaded}
\end{verbatim}
\end{center}

\begin{itemize}
\tightlist
\item
  \textbf{જોખમ}: Hard forks community split કરી શકે અને competing
  currencies બનાવી શકે
\item
  \textbf{સુરક્ષા}: Soft forks સામાન્ય રીતે સુરક્ષિત અને ઓછા disruptive
\end{itemize}

\end{solutionbox}
\begin{mnemonicbox}
``Soft = સમાન દિશા, Hard = મોટો તફાવત''

\end{mnemonicbox}
\begin{center}\rule{0.5\linewidth}{0.5pt}\end{center}

\subsection*{પ્રશ્ન 3(બ) OR [4
ગુણ]}\label{uxaaauxab0uxab6uxaa8-3uxaac-or-4-uxa97uxaa3}

\textbf{બ્લોકચેનની દુનિયામાં Finality નાં શું મહત્વ છે?}

\begin{solutionbox}

\textbf{Finality વ્યાખ્યા}: એક વાર transaction confirm થઈ ગયા પછી તે
reverse કે alter ન થઈ શકે તેની ગેરંટી.

\textbf{મહત્વ કોષ્ટક:}

{\def\LTcaptype{none} % do not increment counter
\begin{longtable}[]{@{}ll@{}}
\toprule\noalign{}
\textbf{બાબત} & \textbf{મહત્વ} \\
\midrule\noalign{}
\endhead
\bottomrule\noalign{}
\endlastfoot
\textbf{Trust} & Users ને વિશ્વાસ કે transactions કાયમી છે \\
\textbf{Business Use} & Companies completed transactions પર ભરોસો કરી
શકે \\
\textbf{Legal Certainty} & Courts blockchain records enforce કરી શકે \\
\textbf{Settlement} & Financial institutions payments clear કરી શકે \\
\end{longtable}
}

\textbf{Finality ના પ્રકારો:}

{\def\LTcaptype{none} % do not increment counter
\begin{longtable}[]{@{}lll@{}}
\toprule\noalign{}
\textbf{પ્રકાર} & \textbf{વર્ણન} & \textbf{સમય} \\
\midrule\noalign{}
\endhead
\bottomrule\noalign{}
\endlastfoot
\textbf{Probabilistic} & સમય સાથે વધુ certain બને & Bitcoin:
\textasciitilde1 કલાક \\
\textbf{Absolute} & તુરંત guarantee & કેટલીક private chains \\
\textbf{Economic} & Reversal ની કિંમત ખૂબ વધારે & Network પ્રમાણે વિવિધ \\
\end{longtable}
}

\textbf{Finality Process:}

\begin{center}
\textbf{Mermaid Diagram (Code)}
\begin{verbatim}
{Shaded}
{Highlighting}[]
graph LR
    A[Transaction Submitted] {-{-}{} B[First Confirmation]}
    B {-{-}{} C[Multiple Confirmations]}
    C {-{-}{} D[Probabilistic Finality]}
    D {-{-}{} E[Practical Finality]}
{Highlighting}
{Shaded}
\end{verbatim}
\end{center}

\begin{itemize}
\tightlist
\item
  \textbf{Bitcoin}: 6 confirmations સામાન્ય રીતે final ગણાય
\item
  \textbf{Ethereum}: Proof of Stake સાથે ઝડપી finality તરફ જતું
\item
  \textbf{પડકાર}: Speed અને security વચ્ચે સંતુલન
\end{itemize}

\end{solutionbox}
\begin{mnemonicbox}
``Final = હંમેશ માટે, મહત્વપૂર્ણ = પાછું ન બદલાય''

\end{mnemonicbox}
\begin{center}\rule{0.5\linewidth}{0.5pt}\end{center}

\subsection*{પ્રશ્ન 3(ક) OR [7
ગુણ]}\label{uxaaauxab0uxab6uxaa8-3uxa95-or-7-uxa97uxaa3}

\textbf{બ્લોકચેનમાં 51\% attack શું છે? ટૂંકમાં સમજાવો.}

\begin{solutionbox}

\textbf{51\% Attack વ્યાખ્યા}: જ્યારે કોઈ એક entity network ની 50\% થી વધુ
mining power અથવા validators ને control કરે અને blockchain manipulate કરી
શકે.

\textbf{Attack પદ્ધતિ:}

{\def\LTcaptype{none} % do not increment counter
\begin{longtable}[]{@{}
  >{\raggedright\arraybackslash}p{(\linewidth - 4\tabcolsep) * \real{0.2683}}
  >{\raggedright\arraybackslash}p{(\linewidth - 4\tabcolsep) * \real{0.5122}}
  >{\raggedright\arraybackslash}p{(\linewidth - 4\tabcolsep) * \real{0.2195}}@{}}
\toprule\noalign{}
\begin{minipage}[b]{\linewidth}\raggedright
\textbf{પગલું}
\end{minipage} & \begin{minipage}[b]{\linewidth}\raggedright
\textbf{Attacker ની ક્રિયા}
\end{minipage} & \begin{minipage}[b]{\linewidth}\raggedright
\textbf{અસર}
\end{minipage} \\
\midrule\noalign{}
\endhead
\bottomrule\noalign{}
\endlastfoot
\textbf{1. Control} & \textgreater50\% mining power મેળવવું & Network પર
dominance \\
\textbf{2. Double Spend} & ગુપ્ત chain બનાવવી & Alternative history તૈયાર
કરવી \\
\textbf{3. Execute} & લાંબી chain release કરવી & Network fake version
accept કરે \\
\textbf{4. Profit} & Coins બે વાર spend કરવા & Victims પાસેથી ચોરી \\
\end{longtable}
}

\textbf{Attack વિઝ્યુઅલાઈઝેશન:}

\begin{center}
\textbf{Mermaid Diagram (Code)}
\begin{verbatim}
{Shaded}
{Highlighting}[]
graph LR
    A[Honest Chain] {-{-}{} B[Block N]}
    C[Attacker{s Secret Chain] {-}{-}{} D[Block N{}]}
    
    B {-{-}{} E[Block N+1]}
    D {-{-}{} F[Block N{}+1]}
    D {-{-}{} G[Block N{}+2 {-} લાંબી Chain]}
    
    G {-{-}{} H[Network Attacker ની Chain Accept કરે]}
    E {-{-}{} I[Honest Chain છોડી દેવાય]}
{Highlighting}
{Shaded}
\end{verbatim}
\end{center}

\textbf{શક્ય Attacks:}

{\def\LTcaptype{none} % do not increment counter
\begin{longtable}[]{@{}ll@{}}
\toprule\noalign{}
\textbf{Attack Type} & \textbf{વર્ણન} \\
\midrule\noalign{}
\endhead
\bottomrule\noalign{}
\endlastfoot
\textbf{Double Spending} & સમાન coins બે વાર spend કરવા \\
\textbf{Transaction Reversal} & Confirmed transactions cancel કરવા \\
\textbf{Mining Monopoly} & બીજા miners નું કામ block કરવું \\
\textbf{Censorship} & ખાસ transactions prevent કરવા \\
\end{longtable}
}

\textbf{બચાવના પદ્ધતિઓ:}

{\def\LTcaptype{none} % do not increment counter
\begin{longtable}[]{@{}ll@{}}
\toprule\noalign{}
\textbf{પદ્ધતિ} & \textbf{કેવી રીતે મદદ કરે} \\
\midrule\noalign{}
\endhead
\bottomrule\noalign{}
\endlastfoot
\textbf{Decentralization} & Mining ઘણા participants માં વહેંચવું \\
\textbf{High Hash Rate} & Attack ને economically અશક્ય બનાવવું \\
\textbf{Proof of Stake} & Attackers ના staked coins ગુમાવવા \\
\textbf{Monitoring} & Suspicious mining activity detect કરવી \\
\end{longtable}
}

\textbf{વાસ્તવિક ઉદાહરણો:}

{\def\LTcaptype{none} % do not increment counter
\begin{longtable}[]{@{}ll@{}}
\toprule\noalign{}
\textbf{Blockchain} & \textbf{સ્થિતિ} \\
\midrule\noalign{}
\endhead
\bottomrule\noalign{}
\endlastfoot
\textbf{Bitcoin} & કદી સફળ attack નથી થયો \\
\textbf{Ethereum Classic} & ઘણી વખત attack થયો \\
\textbf{નાની Altcoins} & Low hash rate થી વધુ vulnerable \\
\end{longtable}
}

\begin{itemize}
\tightlist
\item
  \textbf{કિંમત}: મુખ્ય networks પર attack અત્યંત મોંઘું
\item
  \textbf{શોધ}: Attacks સામાન્ય રીતે ઝડપથી detect થાય
\item
  \textbf{Recovery}: Networks countermeasures implement કરી શકે
\end{itemize}

\end{solutionbox}
\begin{mnemonicbox}
``51\% = બહુમતીની બદમાશી, Control = કોલાહલ''

\end{mnemonicbox}
\begin{center}\rule{0.5\linewidth}{0.5pt}\end{center}

\subsection*{પ્રશ્ન 4(અ) [3
ગુણ]}\label{uxaaauxab0uxab6uxaa8-4uxa85-3-uxa97uxaa3}

\textbf{વિવિધ પ્રકારના Hyperledger પ્રોજેક્ટ્સનાં વર્ણન કરો.}

\begin{solutionbox}

\textbf{Hyperledger Project Types:}

{\def\LTcaptype{none} % do not increment counter
\begin{longtable}[]{@{}lll@{}}
\toprule\noalign{}
\textbf{Project} & \textbf{હેતુ} & \textbf{Use Case} \\
\midrule\noalign{}
\endhead
\bottomrule\noalign{}
\endlastfoot
\textbf{Fabric} & Modular blockchain platform & Enterprise
applications \\
\textbf{Sawtooth} & Scalable blockchain suite & Supply chain, IoT \\
\textbf{Iroha} & Mobile-focused blockchain & Identity management \\
\textbf{Indy} & Digital identity platform & Self-sovereign identity \\
\textbf{Besu} & Ethereum-compatible client & Public/private Ethereum \\
\textbf{Burrow} & Smart contract platform & Permissioned networks \\
\end{longtable}
}

\textbf{Project વર્ગીકરણ:}

\begin{center}
\textbf{Mermaid Diagram (Code)}
\begin{verbatim}
{Shaded}
{Highlighting}[]
graph TD
    A[Hyperledger Projects] {-{-}{} B[Frameworks]}
    A {-{-}{} C[Tools]}
    
    B {-{-}{} D[Fabric {-} Enterprise]}
    B {-{-}{} E[Sawtooth {-} Scalable]}
    B {-{-}{} F[Iroha {-} Mobile]}
    
    C {-{-}{} G[Caliper {-} Performance]}
    C {-{-}{} H[Composer {-} Development]}
    C {-{-}{} I[Explorer {-} Monitoring]}
{Highlighting}
{Shaded}
\end{verbatim}
\end{center}

\begin{itemize}
\tightlist
\item
  \textbf{ફોકસ}: Enterprise અને business blockchain solutions
\item
  \textbf{Open Source}: બધા projects મુફતમાં ઉપલબ્ધ
\end{itemize}

\end{solutionbox}
\begin{mnemonicbox}
``Hyper = High Performance, Ledger = Large
Enterprise''

\end{mnemonicbox}
\begin{center}\rule{0.5\linewidth}{0.5pt}\end{center}

\subsection*{પ્રશ્ન 4(બ) [4
ગુણ]}\label{uxaaauxab0uxab6uxaa8-4uxaac-4-uxa97uxaa3}

\textbf{Blockchain અને Bitcoin નો તફાવત આપો.}

\begin{solutionbox}

\textbf{વ્યાપક સરખામણી:}

{\def\LTcaptype{none} % do not increment counter
\begin{longtable}[]{@{}lll@{}}
\toprule\noalign{}
\textbf{બાબત} & \textbf{Blockchain} & \textbf{Bitcoin} \\
\midrule\noalign{}
\endhead
\bottomrule\noalign{}
\endlastfoot
\textbf{વ્યાખ્યા} & Technology/Platform & Digital Currency \\
\textbf{અવકાશ} & વ્યાપક concept & Specific application \\
\textbf{હેતુ} & Record keeping system & Peer-to-peer payments \\
\textbf{Applications} & ઘણા industries & મુખ્યત્વે financial \\
\textbf{લવચિકતા} & Customize કરી શકાય & Fixed protocol \\
\textbf{સર્જક} & ઘણા contributors & Satoshi Nakamoto \\
\textbf{શરૂઆત} & Concept સમય સાથે વિકસ્યો & 2009 માં launch \\
\end{longtable}
}

\textbf{સંબંધ આકૃતિ:}

\begin{center}
\textbf{Mermaid Diagram (Code)}
\begin{verbatim}
{Shaded}
{Highlighting}[]
graph TD
    A[Blockchain Technology] {-{-}{} B[Bitcoin Cryptocurrency]}
    A {-{-}{} C[Ethereum Platform]}
    A {-{-}{} D[Supply Chain Apps]}
    A {-{-}{} E[Healthcare Records]}
    
    B {-{-}{} F[Digital Payments]}
    B {-{-}{} G[Store of Value]}
{Highlighting}
{Shaded}
\end{verbatim}
\end{center}

\textbf{મુખ્ય તફાવતો:}

{\def\LTcaptype{none} % do not increment counter
\begin{longtable}[]{@{}lll@{}}
\toprule\noalign{}
\textbf{વર્ગ} & \textbf{Blockchain} & \textbf{Bitcoin} \\
\midrule\noalign{}
\endhead
\bottomrule\noalign{}
\endlastfoot
\textbf{પ્રકાર} & Infrastructure & Application \\
\textbf{ઉપયોગ} & બહુવિધ હેતુઓ & ફક્ત currency \\
\textbf{ફેરફારો} & બદલી શકાય & Protocol fixed \\
\end{longtable}
}

\begin{itemize}
\tightlist
\item
  \textbf{સમાનતા}: Blockchain ઈન્ટરનેટ જેવું, Bitcoin email જેવું
\item
  \textbf{નિર્ભરતા}: Bitcoin ને blockchain જોઈએ, પણ blockchain ને Bitcoin
  જરૂરી નથી
\end{itemize}

\end{solutionbox}
\begin{mnemonicbox}
``Blockchain = બિલ્ડિંગ બ્લોક, Bitcoin = ખાસ ઈંટ''

\end{mnemonicbox}
\begin{center}\rule{0.5\linewidth}{0.5pt}\end{center}

\subsection*{પ્રશ્ન 4(ક) [7
ગુણ]}\label{uxaaauxab0uxab6uxaa8-4uxa95-7-uxa97uxaa3}

\textbf{ટૂંકી નોંધ લખો: Merkle Tree}

\begin{solutionbox}

\textbf{Merkle Tree વ્યાખ્યા}: Binary tree structure જેમાં દરેક leaf
transaction hash દર્શાવે અને દરેક internal node તેના children નો hash ધરાવે.

\textbf{Structure અને Components:}

{\def\LTcaptype{none} % do not increment counter
\begin{longtable}[]{@{}ll@{}}
\toprule\noalign{}
\textbf{Component} & \textbf{વર્ણન} \\
\midrule\noalign{}
\endhead
\bottomrule\noalign{}
\endlastfoot
\textbf{Leaf Nodes} & Individual transaction hashes \\
\textbf{Internal Nodes} & બે child nodes નો hash \\
\textbf{Root Hash} & આખા tree નો single hash \\
\textbf{Path} & Leaf થી root સુધીનો route \\
\end{longtable}
}

\textbf{Merkle Tree આકૃતિ:}

\begin{verbatim}
                    Root Hash
                   /         {}
              Hash AB       Hash CD
             /       {     /       }
        Hash A   Hash B Hash C   Hash D
          |        |      |        |
        TX A     TX B   TX C     TX D
\end{verbatim}

\textbf{બાંધકામ પ્રક્રિયા:}

{\def\LTcaptype{none} % do not increment counter
\begin{longtable}[]{@{}ll@{}}
\toprule\noalign{}
\textbf{પગલું} & \textbf{ક્રિયા} \\
\midrule\noalign{}
\endhead
\bottomrule\noalign{}
\endlastfoot
\textbf{1} & દરેક transaction ને અલગ અલગ hash કરવું \\
\textbf{2} & Hashes ને pair કરીને together hash કરવા \\
\textbf{3} & Single root સુધી pairing ચાલુ રાખવું \\
\textbf{4} & Block header માં root hash store કરવો \\
\end{longtable}
}

\textbf{ફાયદા કોષ્ટક:}

{\def\LTcaptype{none} % do not increment counter
\begin{longtable}[]{@{}ll@{}}
\toprule\noalign{}
\textbf{ફાયદો} & \textbf{સમજૂતી} \\
\midrule\noalign{}
\endhead
\bottomrule\noalign{}
\endlastfoot
\textbf{Efficiency} & બધા data download કર્યા વગર ઝડપી verification \\
\textbf{Security} & કોઈપણ change તુરંત detect થાય \\
\textbf{Scalability} & Verification time constant રહે \\
\textbf{Storage} & Block header માં ફક્ત root hash જરૂરી \\
\end{longtable}
}

\textbf{Verification Process:}

\begin{center}
\textbf{Mermaid Diagram (Code)}
\begin{verbatim}
{Shaded}
{Highlighting}[]
graph LR
    A[Transaction to Verify] {-{-}{} B[Get Merkle Path]}
    B {-{-}{} C[Hash with Sibling Nodes]}
    C {-{-}{} D[Compute Path to Root]}
    D {-{-}{} E[Compare with Stored Root]}
    E {-{-}{} F\{Match?\}}
    F {-{-}{}|હા| G[Valid Transaction]}
    F {-{-}{}|ના| H[Invalid Transaction]}
{Highlighting}
{Shaded}
\end{verbatim}
\end{center}

\textbf{વાસ્તવિક ઉપયોગ:}

{\def\LTcaptype{none} % do not increment counter
\begin{longtable}[]{@{}ll@{}}
\toprule\noalign{}
\textbf{Use Case} & \textbf{Application} \\
\midrule\noalign{}
\endhead
\bottomrule\noalign{}
\endlastfoot
\textbf{Bitcoin} & Transaction verification \\
\textbf{Git} & Version control \\
\textbf{IPFS} & Distributed storage \\
\textbf{Certificate Transparency} & SSL certificate logs \\
\end{longtable}
}

\begin{itemize}
\tightlist
\item
  \textbf{શોધકર્તા}: Ralph Merkle (1988) ના નામ પરથી
\item
  \textbf{કાર્યક્ષમતા}: O(log n) complexity સાથે verification
\item
  \textbf{સુરક્ષા}: કોઈપણ transaction સાથે છેડછાડ કરવાથી root hash બદલાય
\end{itemize}

\end{solutionbox}
\begin{mnemonicbox}
``Merkle = ઘણા મળીને એક, Tree = વિશ્વસનીય''

\end{mnemonicbox}
\begin{center}\rule{0.5\linewidth}{0.5pt}\end{center}

\subsection*{પ્રશ્ન 4(અ) OR [3
ગુણ]}\label{uxaaauxab0uxab6uxaa8-4uxa85-or-3-uxa97uxaa3}

\textbf{Hash pointer વિશે ટૂંકમાં ચર્ચા કરો અને Merkle tree માં તેનો ઉપયોગ કેવી
રીતે થાય છે.}

\begin{solutionbox}

\textbf{Hash Pointer વ્યાખ્યા}: Data structure જેમાં data નું location અને તે
data નો cryptographic hash બંને હોય.

\textbf{Components:}

{\def\LTcaptype{none} % do not increment counter
\begin{longtable}[]{@{}ll@{}}
\toprule\noalign{}
\textbf{Component} & \textbf{હેતુ} \\
\midrule\noalign{}
\endhead
\bottomrule\noalign{}
\endlastfoot
\textbf{Pointer} & Data ક્યાં stored છે તે બતાવે \\
\textbf{Hash} & Data બદલાયો નથી તે સાબિત કરે \\
\textbf{Combination} & Data ને integrity check સાથે link કરે \\
\end{longtable}
}

\textbf{Merkle Tree માં Hash Pointer:}

\begin{verbatim}
        Root Hash Pointer
       /                 {}
   Hash Ptr AB        Hash Ptr CD
   /         {        /         }
Hash A     Hash B  Hash C     Hash D
  |          |       |          |
 TX A       TX B    TX C       TX D
\end{verbatim}

\textbf{Merkle Tree માં ઉપયોગ:}

{\def\LTcaptype{none} % do not increment counter
\begin{longtable}[]{@{}ll@{}}
\toprule\noalign{}
\textbf{Level} & \textbf{Hash Pointer Function} \\
\midrule\noalign{}
\endhead
\bottomrule\noalign{}
\endlastfoot
\textbf{Leaf Level} & Transaction ને point કરે, transaction hash ધરાવે \\
\textbf{Internal Nodes} & Children ને point કરે, combined hash ધરાવે \\
\textbf{Root} & Tree structure ને point કરે, overall hash ધરાવે \\
\end{longtable}
}

\begin{itemize}
\tightlist
\item
  \textbf{Verification}: Tree structure માં કોઈપણ change detect કરી શકે
\item
  \textbf{Navigation}: Tree structure ની કાર્યક્ષમ traversal માટે
\end{itemize}

\end{solutionbox}
\begin{mnemonicbox}
``Hash Pointer = સ્થાન + Verification''

\end{mnemonicbox}
\begin{center}\rule{0.5\linewidth}{0.5pt}\end{center}

\subsection*{પ્રશ્ન 4(બ) OR [4
ગુણ]}\label{uxaaauxab0uxab6uxaa8-4uxaac-or-4-uxa97uxaa3}

\textbf{બ્લોકચેનમાં Hashing શું છે? Bitcoin માં તે કેવી રીતે ઉપયોગી છે?}

\begin{solutionbox}

\textbf{Hashing વ્યાખ્યા}: Mathematical function જે input data ને fixed-size
characters ના string માં convert કરે.

\textbf{Hashing Properties:}

{\def\LTcaptype{none} % do not increment counter
\begin{longtable}[]{@{}ll@{}}
\toprule\noalign{}
\textbf{Property} & \textbf{વર્ણન} \\
\midrule\noalign{}
\endhead
\bottomrule\noalign{}
\endlastfoot
\textbf{Deterministic} & સમાન input હંમેશા સમાન output આપે \\
\textbf{Fixed Size} & Output હંમેશા સમાન length (SHA-256 માટે 256 bits) \\
\textbf{Avalanche Effect} & નાનો input change = સંપૂર્ણ અલગ output \\
\textbf{One-way} & Original input શોધવા માટે reverse કરી શકાતું નથી \\
\textbf{Collision Resistant} & સમાન output આપતા બે inputs શોધવું અત્યંત
મુશ્કેલ \\
\end{longtable}
}

\textbf{Bitcoin માં ઉપયોગ:}

{\def\LTcaptype{none} % do not increment counter
\begin{longtable}[]{@{}ll@{}}
\toprule\noalign{}
\textbf{Use Case} & \textbf{હેતુ} \\
\midrule\noalign{}
\endhead
\bottomrule\noalign{}
\endlastfoot
\textbf{Block Linking} & દરેક block માં previous block નો hash હોય \\
\textbf{Mining} & Difficulty requirement પૂરો કરતો hash શોધવો \\
\textbf{Transaction IDs} & દરેક transaction માટે unique identifier \\
\textbf{Merkle Root} & Block માં બધા transactions નો summary \\
\textbf{Addresses} & Public keys માંથી Bitcoin addresses બનાવવા \\
\end{longtable}
}

\textbf{Hashing Process:}

\begin{center}
\textbf{Mermaid Diagram (Code)}
\begin{verbatim}
{Shaded}
{Highlighting}[]
graph LR
    A[Input Data] {-{-}{} B[SHA{-}256 Function]}
    B {-{-}{} C[256{-}bit Hash Output]}
    
    D[Input માં નાનો Change] {-{-}{} E[SHA{-}256 Function]}
    E {-{-}{} F[સંપૂર્ણ અલગ Hash]}
{Highlighting}
{Shaded}
\end{verbatim}
\end{center}

\begin{itemize}
\tightlist
\item
  \textbf{Algorithm}: Bitcoin SHA-256 hashing વાપરે
\item
  \textbf{સુરક્ષા}: Blockchain ને tamper-evident બનાવે
\item
  \textbf{કાર્યક્ષમતા}: Compute અને verify કરવું ઝડપી
\end{itemize}

\end{solutionbox}
\begin{mnemonicbox}
``Hash = Fingerprint, Bitcoin = Hashing પર આધારિત''

\end{mnemonicbox}
\begin{center}\rule{0.5\linewidth}{0.5pt}\end{center}

\subsection*{પ્રશ્ન 4(ક) OR [7
ગુણ]}\label{uxaaauxab0uxab6uxaa8-4uxa95-or-7-uxa97uxaa3}

\textbf{Classic Byzantine generals problem અને Practical Byzantine Fault
Tolerance ને વિગતવાર સમજાવો.}

\begin{solutionbox}

\textbf{Byzantine Generals Problem}: Distributed systems માં unreliable
participants સાથે consensus achieve કરવાની classic computer science
સમસ્યા.

\textbf{સમસ્યાનું Scenario:}

{\def\LTcaptype{none} % do not increment counter
\begin{longtable}[]{@{}ll@{}}
\toprule\noalign{}
\textbf{Element} & \textbf{વર્ણન} \\
\midrule\noalign{}
\endhead
\bottomrule\noalign{}
\endlastfoot
\textbf{Generals} & Network nodes દર્શાવે \\
\textbf{City} & System state દર્શાવે \\
\textbf{Attack Plan} & Consensus decision દર્શાવે \\
\textbf{Traitors} & Malicious/faulty nodes દર્શાવે \\
\textbf{Communication} & Nodes વચ્ચે messages \\
\end{longtable}
}

\textbf{સમસ્યા વિઝ્યુઅલાઈઝેશન:}

\begin{center}
\textbf{Mermaid Diagram (Code)}
\begin{verbatim}
{Shaded}
{Highlighting}[]
graph TD
    A[General A {- Honest] {-}{-}{} D[City Under Siege]}
    B[General B {- Traitor] {-}{-}{} D}
    C[General C {- Honest] {-}{-}{} D}
    E[General D {- Honest] {-}{-}{} D}
    
    A {-{-}{} F[Vote: Attack]}
    B {-{-}{} G[Vote: Attack to A, Retreat to C]}
    C {-{-}{} H[Vote: Attack]}
    E {-{-}{} I[Vote: Attack]}
{Highlighting}
{Shaded}
\end{verbatim}
\end{center}

\textbf{Practical Byzantine Fault Tolerance (pBFT):}

\textbf{pBFT Algorithm ના તબક્કાઓ:}

{\def\LTcaptype{none} % do not increment counter
\begin{longtable}[]{@{}
  >{\raggedright\arraybackslash}p{(\linewidth - 4\tabcolsep) * \real{0.3548}}
  >{\raggedright\arraybackslash}p{(\linewidth - 4\tabcolsep) * \real{0.3548}}
  >{\raggedright\arraybackslash}p{(\linewidth - 4\tabcolsep) * \real{0.2903}}@{}}
\toprule\noalign{}
\begin{minipage}[b]{\linewidth}\raggedright
\textbf{તબક્કો}
\end{minipage} & \begin{minipage}[b]{\linewidth}\raggedright
\textbf{ક્રિયા}
\end{minipage} & \begin{minipage}[b]{\linewidth}\raggedright
\textbf{હેતુ}
\end{minipage} \\
\midrule\noalign{}
\endhead
\bottomrule\noalign{}
\endlastfoot
\textbf{Pre-prepare} & Leader proposal broadcast કરે & Consensus round શરૂ
કરવો \\
\textbf{Prepare} & Nodes validate કરે અને agreement broadcast કરે & બધાને
proposal દેખાડવું \\
\textbf{Commit} & Nodes decision પર commit કરે & Consensus finalize
કરવું \\
\end{longtable}
}

\textbf{pBFT Process Flow:}

\begin{verbatim}
sequenceDiagram
    participant C as Client
    participant P as Primary Node
    participant B1 as Backup Node 1
    participant B2 as Backup Node 2
    
    C{-P: Request}
    P{-B1: Pre{-}prepare}
    P{-B2: Pre{-}prepare}
    B1{-B2: Prepare}
    B2{-B1: Prepare}
    B1{-B2: Commit}
    B2{-B1: Commit}
    P{-C: Reply}
\end{verbatim}

\textbf{Fault Tolerance:}

{\def\LTcaptype{none} % do not increment counter
\begin{longtable}[]{@{}ll@{}}
\toprule\noalign{}
\textbf{બાબત} & \textbf{ક્ષમતા} \\
\midrule\noalign{}
\endhead
\bottomrule\noalign{}
\endlastfoot
\textbf{Maximum Faulty Nodes} & 1/3 સુધી faulty nodes સહન કરી શકે \\
\textbf{Network Requirement} & Synchronous અથવા partially synchronous \\
\textbf{Message Complexity} & દરેક consensus માટે O(n^{2}) messages \\
\textbf{Finality} & તુરંત finality મળે \\
\end{longtable}
}

\textbf{Applications:}

{\def\LTcaptype{none} % do not increment counter
\begin{longtable}[]{@{}ll@{}}
\toprule\noalign{}
\textbf{System} & \textbf{ઉપયોગ} \\
\midrule\noalign{}
\endhead
\bottomrule\noalign{}
\endlastfoot
\textbf{Hyperledger Fabric} & Consensus mechanism \\
\textbf{Tendermint} & Byzantine fault tolerant consensus \\
\textbf{Zilliqa} & Practical Byzantine fault tolerance \\
\end{longtable}
}

\begin{itemize}
\tightlist
\item
  \textbf{ફાયદો}: ઝડપી finality, permissioned networks માટે સારું
\item
  \textbf{મર્યાદા}: ઊંચો communication overhead, સારી રીતે scale કરતું નથી
\end{itemize}

\end{solutionbox}
\begin{mnemonicbox}
``Byzantine = ખરાબ અભિનેતા, pBFT = વ્યાવહારિક ઉકેલ''

\end{mnemonicbox}
\begin{center}\rule{0.5\linewidth}{0.5pt}\end{center}

\subsection*{પ્રશ્ન 5(અ) [3
ગુણ]}\label{uxaaauxab0uxab6uxaa8-5uxa85-3-uxa97uxaa3}

\textbf{બ્લોકચેનમાં cryptocurrency wallets ની યાદી બનાવો અને સમજાવો.}

\begin{solutionbox}

\textbf{Cryptocurrency Wallet પ્રકારો:}

{\def\LTcaptype{none} % do not increment counter
\begin{longtable}[]{@{}
  >{\raggedright\arraybackslash}p{(\linewidth - 4\tabcolsep) * \real{0.3696}}
  >{\raggedright\arraybackslash}p{(\linewidth - 4\tabcolsep) * \real{0.2174}}
  >{\raggedright\arraybackslash}p{(\linewidth - 4\tabcolsep) * \real{0.4130}}@{}}
\toprule\noalign{}
\begin{minipage}[b]{\linewidth}\raggedright
\textbf{Wallet Type}
\end{minipage} & \begin{minipage}[b]{\linewidth}\raggedright
\textbf{વર્ણન}
\end{minipage} & \begin{minipage}[b]{\linewidth}\raggedright
\textbf{Security Level}
\end{minipage} \\
\midrule\noalign{}
\endhead
\bottomrule\noalign{}
\endlastfoot
\textbf{Hardware Wallet} & Keys store કરતા physical device & ખૂબ ઊંચી \\
\textbf{Software Wallet} & Computer/phone પર application & મધ્યમ થી
ઊંચી \\
\textbf{Paper Wallet} & કાગળ પર છપાયેલી keys & ઊંચી (સુરક્ષિત રીતે stored હોય
તો) \\
\textbf{Web Wallet} & Online wallet service & મધ્યમ \\
\textbf{Brain Wallet} & યાદ રાખેલી keys & વિવિધ \\
\end{longtable}
}

\textbf{Storage પદ્ધતિઓ:}

{\def\LTcaptype{none} % do not increment counter
\begin{longtable}[]{@{}lll@{}}
\toprule\noalign{}
\textbf{પદ્ધતિ} & \textbf{Accessibility} & \textbf{Security} \\
\midrule\noalign{}
\endhead
\bottomrule\noalign{}
\endlastfoot
\textbf{Hot Wallet} & હંમેશા online & ઓછી security \\
\textbf{Cold Wallet} & Offline storage & વધુ security \\
\end{longtable}
}

\textbf{Wallet કામો:}

\begin{center}
\textbf{Mermaid Diagram (Code)}
\begin{verbatim}
{Shaded}
{Highlighting}[]
graph TD
    A[Cryptocurrency Wallet] {-{-}{} B[Private Keys Store કરે]}
    A {-{-}{} C[Addresses Generate કરે]}
    A {-{-}{} D[Transactions Sign કરે]}
    A {-{-}{} E[Balances Check કરે]}
    A {-{-}{} F[Crypto Send/Receive કરે]}
{Highlighting}
{Shaded}
\end{verbatim}
\end{center}

\begin{itemize}
\tightlist
\item
  \textbf{મુખ્ય મુદ્દો}: Wallets coins store કરતા નથી, coins access કરવાની
  keys store કરે
\item
  \textbf{Backup}: હંમેશા wallet seed phrase નો backup રાખવો
\end{itemize}

\end{solutionbox}
\begin{mnemonicbox}
``Wallet = Key Keeper, Coin Container નથી''

\end{mnemonicbox}
\begin{center}\rule{0.5\linewidth}{0.5pt}\end{center}

\subsection*{પ્રશ્ન 5(બ) [4
ગુણ]}\label{uxaaauxab0uxab6uxaa8-5uxaac-4-uxa97uxaa3}

\textbf{ERC-20 ટોકનના ફાયદા અને ગેરફાયદા લખો.}

\begin{solutionbox}

\textbf{ERC-20 Token વ્યાખ્યા}: Ethereum blockchain પર tokens બનાવવા માટેનો
standard protocol.

\textbf{ફાયદા:}

{\def\LTcaptype{none} % do not increment counter
\begin{longtable}[]{@{}ll@{}}
\toprule\noalign{}
\textbf{ફાયદો} & \textbf{લાભ} \\
\midrule\noalign{}
\endhead
\bottomrule\noalign{}
\endlastfoot
\textbf{Standardization} & બધા tokens સમાન રીતે કામ કરે \\
\textbf{Interoperability} & બધા Ethereum wallets સાથે compatible \\
\textbf{Easy Development} & નવા tokens બનાવવા સરળ \\
\textbf{Wide Support} & Exchanges અને services દ્વારા support \\
\textbf{Smart Contract Integration} & DeFi protocols સાથે interact કરી
શકે \\
\end{longtable}
}

\textbf{ગેરફાયદા:}

{\def\LTcaptype{none} % do not increment counter
\begin{longtable}[]{@{}ll@{}}
\toprule\noalign{}
\textbf{ગેરફાયદા} & \textbf{સમસ્યા} \\
\midrule\noalign{}
\endhead
\bottomrule\noalign{}
\endlastfoot
\textbf{Gas Fees} & Network congestion દરમિયાન મોંઘા transactions \\
\textbf{Scalability} & Ethereum ની transaction throughput દ્વારા
મર્યાદિત \\
\textbf{Security Risks} & Smart contract bugs થી token loss \\
\textbf{Centralization} & ઘણા tokens નું centralized control \\
\textbf{Environmental Impact} & ઊંચો energy consumption \\
\end{longtable}
}

\textbf{સરખામણી કોષ્ટક:}

{\def\LTcaptype{none} % do not increment counter
\begin{longtable}[]{@{}lll@{}}
\toprule\noalign{}
\textbf{બાબત} & \textbf{ફાયદો} & \textbf{ગેરફાયદા} \\
\midrule\noalign{}
\endhead
\bottomrule\noalign{}
\endlastfoot
\textbf{Adoption} & બહોળો સ્વીકાર & Market oversaturation \\
\textbf{Development} & બનાવવા સરળ & Scam tokens બનાવવા પણ સરળ \\
\textbf{Functionality} & Standard features & મર્યાદિત customization \\
\end{longtable}
}

\begin{itemize}
\tightlist
\item
  \textbf{ઉપયોગ}: Cryptocurrency tokens બનાવવા માટે સૌથી લોકપ્રિય standard
\item
  \textbf{ઉદાહરણો}: USDT, LINK, UNI એ ERC-20 tokens છે
\end{itemize}

\end{solutionbox}
\begin{mnemonicbox}
``ERC-20 = Easy અને Expensive''

\end{mnemonicbox}
\begin{center}\rule{0.5\linewidth}{0.5pt}\end{center}

\subsection*{પ્રશ્ન 5(ક) [7
ગુણ]}\label{uxaaauxab0uxab6uxaa8-5uxa95-7-uxa97uxaa3}

\textbf{dApps નો ઉપયોગ શેના માટે થાય છે? dApps ના ફાયદા અને ગેરફાયદા સમજાવો.}

\begin{solutionbox}

\textbf{dApps વ્યાખ્યા}: Decentralized Applications જે blockchain networks
પર central authority વગર run થાય.

\textbf{dApps ઉપયોગ વર્ગીકરણ:}

{\def\LTcaptype{none} % do not increment counter
\begin{longtable}[]{@{}lll@{}}
\toprule\noalign{}
\textbf{વર્ગ} & \textbf{ઉદાહરણો} & \textbf{હેતુ} \\
\midrule\noalign{}
\endhead
\bottomrule\noalign{}
\endlastfoot
\textbf{DeFi} & Uniswap, Compound & Financial services \\
\textbf{Gaming} & CryptoKitties, Axie Infinity & Blockchain games \\
\textbf{Social Media} & Steemit, Minds & Censorship-resistant
platforms \\
\textbf{Marketplaces} & OpenSea, Rarible & NFT trading \\
\textbf{Governance} & Aragon, DAOstack & Decentralized organizations \\
\textbf{Storage} & Filecoin, Storj & Distributed file storage \\
\end{longtable}
}

\textbf{dApp આર્કિટેક્ચર:}

\begin{center}
\textbf{Mermaid Diagram (Code)}
\begin{verbatim}
{Shaded}
{Highlighting}[]
graph LR
    A[Frontend {- User Interface] {-}{-}{} B[Web3 Connection]}
    B {-{-}{} C[Smart Contracts]}
    C {-{-}{} D[Blockchain Network]}
    D {-{-}{} E[Distributed Storage]}
    
    F[Traditional App] {-{-}{} G[Central Server]}
    G {-{-}{} H[Central Database]}
{Highlighting}
{Shaded}
\end{verbatim}
\end{center}

\textbf{ફાયદા:}

{\def\LTcaptype{none} % do not increment counter
\begin{longtable}[]{@{}ll@{}}
\toprule\noalign{}
\textbf{ફાયદો} & \textbf{વર્ણન} \\
\midrule\noalign{}
\endhead
\bottomrule\noalign{}
\endlastfoot
\textbf{Censorship Resistance} & કોઈ એક control point નથી \\
\textbf{Transparency} & Code અને data publicly verifiable \\
\textbf{Global Access} & વિશ્વભરમાં restrictions વગર ઉપલબ્ધ \\
\textbf{No Downtime} & ઘણા nodes માં distributed \\
\textbf{User Ownership} & Users પોતાના data અને assets control કરે \\
\textbf{Trustless} & Intermediaries પર trust કરવાની જરૂર નથી \\
\end{longtable}
}

\textbf{ગેરફાયદા:}

{\def\LTcaptype{none} % do not increment counter
\begin{longtable}[]{@{}ll@{}}
\toprule\noalign{}
\textbf{ગેરફાયદા} & \textbf{વર્ણન} \\
\midrule\noalign{}
\endhead
\bottomrule\noalign{}
\endlastfoot
\textbf{Poor User Experience} & જટિલ interfaces, ધીમા transactions \\
\textbf{Scalability Issues} & મર્યાદિત transaction throughput \\
\textbf{High Costs} & દરેક interaction માટે gas fees \\
\textbf{Technical Complexity} & Non-technical users માટે મુશ્કેલ \\
\textbf{Regulatory Uncertainty} & અસ્પષ્ટ કાયદાકીય સ્થિતિ \\
\textbf{Energy Consumption} & ઊંચો environmental impact \\
\textbf{Immutable Bugs} & Smart contract errors સહેલાઈથી fix ન કરી
શકાય \\
\end{longtable}
}

\textbf{Development પડકારો:}

{\def\LTcaptype{none} % do not increment counter
\begin{longtable}[]{@{}ll@{}}
\toprule\noalign{}
\textbf{પડકાર} & \textbf{અસર} \\
\midrule\noalign{}
\endhead
\bottomrule\noalign{}
\endlastfoot
\textbf{Gas Optimization} & Transaction costs minimize કરવા જોઈએ \\
\textbf{Security Auditing} & Hacks અટકાવવા માટે જરૂરી \\
\textbf{User Onboarding} & Mainstream users આકર્ષવું મુશ્કેલ \\
\textbf{Scalability Solutions} & Layer 2 અથવા alternative chains જોઈએ \\
\end{longtable}
}

\textbf{લોકપ્રિય dApp Platforms:}

{\def\LTcaptype{none} % do not increment counter
\begin{longtable}[]{@{}ll@{}}
\toprule\noalign{}
\textbf{Platform} & \textbf{લક્ષણો} \\
\midrule\noalign{}
\endhead
\bottomrule\noalign{}
\endlastfoot
\textbf{Ethereum} & સૌથી વધુ સ્થાપિત, સૌથી વધુ fees \\
\textbf{Binance Smart Chain} & ઓછી fees, વધુ centralized \\
\textbf{Polygon} & Ethereum Layer 2, ઝડપી અને સસ્તું \\
\textbf{Solana} & ઊંચી throughput, નવું ecosystem \\
\end{longtable}
}

\begin{itemize}
\tightlist
\item
  \textbf{ભવિષ્ય}: બહેતર user experience અને ઓછી costs તરફ જતું
\item
  \textbf{અપનાવણ}: હજુ પણ early stage પણ ઝડપથી વધી રહ્યું
\end{itemize}

\end{solutionbox}
\begin{mnemonicbox}
``dApps = Decentralized પણ મુશ્કેલ''

\end{mnemonicbox}
\begin{center}\rule{0.5\linewidth}{0.5pt}\end{center}

\subsection*{પ્રશ્ન 5(અ) OR [3
ગુણ]}\label{uxaaauxab0uxab6uxaa8-5uxa85-or-3-uxa97uxaa3}

\textbf{Tokenized અને token less બ્લોકચેનને વિગતવાર સમજાવો.}

\begin{solutionbox}

\textbf{Tokenized Blockchain:}

{\def\LTcaptype{none} % do not increment counter
\begin{longtable}[]{@{}ll@{}}
\toprule\noalign{}
\textbf{લક્ષણ} & \textbf{વર્ણન} \\
\midrule\noalign{}
\endhead
\bottomrule\noalign{}
\endlastfoot
\textbf{વ્યાખ્યા} & Native cryptocurrency token સાથેની blockchain \\
\textbf{Token હેતુ} & Network participation incentivize કરવો \\
\textbf{ઉદાહરણો} & Bitcoin (BTC), Ethereum (ETH) \\
\textbf{કામ} & Transaction fees ચૂકવવા, miners/validators ને reward
આપવા \\
\end{longtable}
}

\textbf{Token-less Blockchain:}

{\def\LTcaptype{none} % do not increment counter
\begin{longtable}[]{@{}ll@{}}
\toprule\noalign{}
\textbf{લક્ષણ} & \textbf{વર્ણન} \\
\midrule\noalign{}
\endhead
\bottomrule\noalign{}
\endlastfoot
\textbf{વ્યાખ્યા} & Native cryptocurrency વગરની blockchain \\
\textbf{Access} & Permission-based participation \\
\textbf{ઉદાહરણો} & Hyperledger Fabric, R3 Corda \\
\textbf{કામ} & Record keeping, process automation \\
\end{longtable}
}

\textbf{સરખામણી કોષ્ટક:}

{\def\LTcaptype{none} % do not increment counter
\begin{longtable}[]{@{}lll@{}}
\toprule\noalign{}
\textbf{બાબત} & \textbf{Tokenized} & \textbf{Token-less} \\
\midrule\noalign{}
\endhead
\bottomrule\noalign{}
\endlastfoot
\textbf{Incentive Model} & Economic rewards & Permission-based \\
\textbf{Access} & Tokens હોય તો કોઈપણ & Restricted access \\
\textbf{Governance} & Token holder voting & Centralized control \\
\textbf{Use Case} & Public networks & Private/enterprise \\
\textbf{Security} & Economic game theory & Traditional security \\
\end{longtable}
}

\textbf{આર્કિટેક્ચર તફાવતો:}

\begin{center}
\textbf{Mermaid Diagram (Code)}
\begin{verbatim}
{Shaded}
{Highlighting}[]
graph TD
    subgraph "Tokenized Blockchain"
        A[Token Rewards] {-{-}{} B[Miners/Validators]}
        B {-{-}{} C[Secure Network]}
        C {-{-}{} D[Public Access]}
    end
    
    subgraph "Token{-less Blockchain"}
        E[Permission System] {-{-}{} F[Authorized Nodes]}
        F {-{-}{} G[Secure Network]}
        G {-{-}{} H[Private Access]}
    end
{Highlighting}
{Shaded}
\end{verbatim}
\end{center}

\begin{itemize}
\tightlist
\item
  \textbf{પસંદગી}: Public participation જોઈએ કે private control તેના પર
  નિર્ભર
\item
  \textbf{ટ્રેન્ડ}: મોટાભાગની public blockchains tokenized, મોટાભાગની
  private token-less
\end{itemize}

\end{solutionbox}
\begin{mnemonicbox}
``Token = Public Participation, Token-less = Private
Permission''

\end{mnemonicbox}
\begin{center}\rule{0.5\linewidth}{0.5pt}\end{center}

\subsection*{પ્રશ્ન 5(બ) OR [4
ગુણ]}\label{uxaaauxab0uxab6uxaa8-5uxaac-or-4-uxa97uxaa3}

\textbf{Hyperledger ના ફાયદા અને ગેરફાયદા લખો.}

\begin{solutionbox}

\textbf{Hyperledger વ્યાખ્યા}: Enterprise-grade blockchain solutions
develop કરવા માટેનું open-source collaborative framework.

\textbf{ફાયદા:}

{\def\LTcaptype{none} % do not increment counter
\begin{longtable}[]{@{}ll@{}}
\toprule\noalign{}
\textbf{ફાયદો} & \textbf{વર્ણન} \\
\midrule\noalign{}
\endhead
\bottomrule\noalign{}
\endlastfoot
\textbf{Enterprise Focus} & Business use cases માટે design \\
\textbf{Modular Architecture} & જરૂર પ્રમાણે components customize કરી
શકાય \\
\textbf{Privacy} & Confidential transactions શક્ય \\
\textbf{Performance} & ઊંચી transaction throughput \\
\textbf{Governance} & Professional development standards \\
\textbf{No Cryptocurrency} & Regulatory crypto issues ટાળે \\
\textbf{Permissioned Network} & કોણ participate કરી શકે તેનું control \\
\end{longtable}
}

\textbf{ગેરફાયદા:}

{\def\LTcaptype{none} % do not increment counter
\begin{longtable}[]{@{}ll@{}}
\toprule\noalign{}
\textbf{ગેરફાયદા} & \textbf{વર્ણન} \\
\midrule\noalign{}
\endhead
\bottomrule\noalign{}
\endlastfoot
\textbf{Centralization} & Public blockchains કરતાં ઓછું decentralized \\
\textbf{Complexity} & Implement કરવા માટે technical expertise જરૂરી \\
\textbf{Limited Adoption} & Ethereum કરતાં નાનું ecosystem \\
\textbf{Vendor Lock-in} & Specific technology providers પર નિર્ભરતા \\
\textbf{Scalability} & હજુ પણ કેટલીક scaling challenges \\
\textbf{No Token Economy} & Crypto incentives leverage કરી શકાતું નથી \\
\end{longtable}
}

\textbf{Hyperledger Projects સરખામણી:}

{\def\LTcaptype{none} % do not increment counter
\begin{longtable}[]{@{}lll@{}}
\toprule\noalign{}
\textbf{Project} & \textbf{શક્તિઓ} & \textbf{મર્યાદાઓ} \\
\midrule\noalign{}
\endhead
\bottomrule\noalign{}
\endlastfoot
\textbf{Fabric} & Mature, flexible & Complex setup \\
\textbf{Sawtooth} & Scalable & ઓછું documentation \\
\textbf{Iroha} & Simple, mobile-friendly & મર્યાદિત features \\
\end{longtable}
}

\textbf{Use Case યોગ્યતા:}

{\def\LTcaptype{none} % do not increment counter
\begin{longtable}[]{@{}ll@{}}
\toprule\noalign{}
\textbf{સારું છે} & \textbf{આદર્શ નથી} \\
\midrule\noalign{}
\endhead
\bottomrule\noalign{}
\endlastfoot
\textbf{Supply chain tracking} & Public cryptocurrencies \\
\textbf{Healthcare records} & સંપૂર્ણ decentralized systems \\
\textbf{Banking consortiums} & High-frequency trading \\
\textbf{Government systems} & Anonymous transactions \\
\end{longtable}
}

\begin{itemize}
\tightlist
\item
  \textbf{લક્ષ્ય}: મોટી enterprises અને consortiums
\item
  \textbf{સપોર્ટ}: Linux Foundation દ્વારા backed
\end{itemize}

\end{solutionbox}
\begin{mnemonicbox}
``Hyperledger = High Performance, Low Publicity''

\end{mnemonicbox}
\begin{center}\rule{0.5\linewidth}{0.5pt}\end{center}

\subsection*{પ્રશ્ન 5(ક) OR [7
ગુણ]}\label{uxaaauxab0uxab6uxaa8-5uxa95-or-7-uxa97uxaa3}

\textbf{Smart contract સમજાવો. Smart contract ની વિવિધ એપ્લિકેશન્સ લખો.}

\begin{solutionbox}

\textbf{Smart Contract વ્યાખ્યા}: Self-executing contracts જેના terms સીધા
code માં લખાયેલા હોય અને blockchain પર આપોઆપ enforce થાય.

\textbf{મુખ્ય લક્ષણો:}

{\def\LTcaptype{none} % do not increment counter
\begin{longtable}[]{@{}ll@{}}
\toprule\noalign{}
\textbf{લક્ષણ} & \textbf{વર્ણન} \\
\midrule\noalign{}
\endhead
\bottomrule\noalign{}
\endlastfoot
\textbf{Automated} & Conditions પૂરી થાય ત્યારે આપોઆપ execute \\
\textbf{Immutable} & Deployment પછી બદલી શકાતું નથી \\
\textbf{Transparent} & Code publicly visible \\
\textbf{Trustless} & Intermediaries ની જરૂર નથી \\
\textbf{Deterministic} & સમાન input હંમેશા સમાન output \\
\end{longtable}
}

\textbf{Smart Contract Workflow:}

\begin{center}
\textbf{Mermaid Diagram (Code)}
\begin{verbatim}
{Shaded}
{Highlighting}[]
graph LR
    A[Contract Created] {-{-}{} B[Blockchain પર Deployed]}
    B {-{-}{} C[Conditions Monitored]}
    C {-{-}{} D\{Conditions Met?\}}
    D {-{-}{}|હા| E[Contract Executes]}
    D {-{-}{}|ના| F[Monitoring ચાલુ]}
    E {-{-}{} G[Automatic Settlement]}
    F {-{-}{} C}
{Highlighting}
{Shaded}
\end{verbatim}
\end{center}

\textbf{Industry પ્રમાણે Applications:}

{\def\LTcaptype{none} % do not increment counter
\begin{longtable}[]{@{}
  >{\raggedright\arraybackslash}p{(\linewidth - 4\tabcolsep) * \real{0.3415}}
  >{\raggedright\arraybackslash}p{(\linewidth - 4\tabcolsep) * \real{0.4146}}
  >{\raggedright\arraybackslash}p{(\linewidth - 4\tabcolsep) * \real{0.2439}}@{}}
\toprule\noalign{}
\begin{minipage}[b]{\linewidth}\raggedright
\textbf{Industry}
\end{minipage} & \begin{minipage}[b]{\linewidth}\raggedright
\textbf{Application}
\end{minipage} & \begin{minipage}[b]{\linewidth}\raggedright
\textbf{ફાયદો}
\end{minipage} \\
\midrule\noalign{}
\endhead
\bottomrule\noalign{}
\endlastfoot
\textbf{Finance} & Automated loans, insurance claims & ઝડપી processing,
ઓછી costs \\
\textbf{Real Estate} & Property transfers, rental agreements & ફ્રોડ
ઘટાડવું, instant settlements \\
\textbf{Supply Chain} & Product tracking, quality assurance & પારદર્શિતા,
automated compliance \\
\textbf{Healthcare} & Patient consent, insurance claims & Privacy
protection, automated payouts \\
\textbf{Entertainment} & Royalty distribution, content licensing & Fair
payment, transparent accounting \\
\textbf{Gaming} & In-game assets, tournaments & True ownership,
automated prizes \\
\end{longtable}
}

\textbf{ખાસ Smart Contract ઉદાહરણો:}

{\def\LTcaptype{none} % do not increment counter
\begin{longtable}[]{@{}lll@{}}
\toprule\noalign{}
\textbf{Application} & \textbf{Function} & \textbf{Platform} \\
\midrule\noalign{}
\endhead
\bottomrule\noalign{}
\endlastfoot
\textbf{Uniswap} & Automated token trading & Ethereum \\
\textbf{Compound} & Lending અને borrowing & Ethereum \\
\textbf{CryptoKitties} & Digital pet ownership & Ethereum \\
\textbf{Chainlink} & Oracle data feeds & Multiple platforms \\
\textbf{Aave} & Flash loans & Ethereum \\
\end{longtable}
}

\textbf{Development Platforms:}

{\def\LTcaptype{none} % do not increment counter
\begin{longtable}[]{@{}lll@{}}
\toprule\noalign{}
\textbf{Platform} & \textbf{Language} & \textbf{લક્ષણો} \\
\midrule\noalign{}
\endhead
\bottomrule\noalign{}
\endlastfoot
\textbf{Ethereum} & Solidity & સૌથી mature ecosystem \\
\textbf{Binance Smart Chain} & Solidity & ઓછી fees, ઝડપી \\
\textbf{Cardano} & Plutus & Academic approach \\
\textbf{Solana} & Rust & High performance \\
\end{longtable}
}

\textbf{ફાયદા:}

{\def\LTcaptype{none} % do not increment counter
\begin{longtable}[]{@{}lll@{}}
\toprule\noalign{}
\textbf{ફાયદો} & \textbf{Traditional Contract} & \textbf{Smart
Contract} \\
\midrule\noalign{}
\endhead
\bottomrule\noalign{}
\endlastfoot
\textbf{Speed} & દિવસો થી અઠવાડિયા & મિનિટો થી કલાકો \\
\textbf{Cost} & ઊંચી legal fees & ઓછી gas fees \\
\textbf{Trust} & Intermediaries જરૂરી & Trustless execution \\
\textbf{Accuracy} & Human error શક્ય & Coded precision \\
\end{longtable}
}

\textbf{મર્યાદાઓ:}

{\def\LTcaptype{none} % do not increment counter
\begin{longtable}[]{@{}ll@{}}
\toprule\noalign{}
\textbf{મર્યાદા} & \textbf{વર્ણન} \\
\midrule\noalign{}
\endhead
\bottomrule\noalign{}
\endlastfoot
\textbf{Code Bugs} & Errors થી financial loss \\
\textbf{Oracle Problem} & Real-world data મેળવવાની મુશ્કેલી \\
\textbf{Immutability} & Deployment પછી fix કરવું મુશ્કેલ \\
\textbf{Gas Costs} & Congested networks પર મોંઘું \\
\textbf{Legal Status} & અસ્પષ્ટ regulatory framework \\
\end{longtable}
}

\textbf{વાસ્તવિક અસર:}

{\def\LTcaptype{none} % do not increment counter
\begin{longtable}[]{@{}ll@{}}
\toprule\noalign{}
\textbf{ક્ષેત્ર} & \textbf{પરિવર્તન} \\
\midrule\noalign{}
\endhead
\bottomrule\noalign{}
\endlastfoot
\textbf{DeFi} & Smart contracts માં \$100+ billion locked \\
\textbf{NFTs} & નવા digital ownership models \\
\textbf{DAOs} & Decentralized governance systems \\
\textbf{Insurance} & Parametric insurance products \\
\end{longtable}
}

\begin{itemize}
\tightlist
\item
  \textbf{ભવિષ્ય}: IoT, AI અને traditional business systems સાથે
  integration
\item
  \textbf{વિકાસ}: વધુ user-friendly development tools તરફ જતું
\end{itemize}

\end{solutionbox}
\begin{mnemonicbox}
``Smart Contract = Self-executing, સમસ્યાઓ હલ કરે''

\end{mnemonicbox}

\end{document}
