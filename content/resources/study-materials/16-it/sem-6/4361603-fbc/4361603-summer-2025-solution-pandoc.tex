\documentclass[10pt,a4paper]{article}

% content/resources/templates/preamble.tex
\usepackage[margin=0.6in]{geometry}
\author{Milav Dabgar}
\usepackage{amsmath,amssymb,amsthm}
\usepackage{booktabs}
\usepackage{multirow}
\usepackage{xcolor}
\usepackage{tcolorbox}
\tcbuselibrary{breakable,skins}
\usepackage[colorlinks=true,linkcolor=blue]{hyperref}
\usepackage{titlesec}
\usepackage{enumitem}
\usepackage{tikz}
\usepackage{pgfplots}
\usepackage{circuitikz}
\usepackage[version=4]{mhchem}
\usepackage{longtable}
\usepackage{array}
\usepackage{float}
\usepackage{caption}
\usepackage{listings}

\lstset{
  basicstyle=\small\ttfamily,
  breaklines=true,
  breakatwhitespace=false,
  postbreak=\mbox{\textcolor{red}{$\hookrightarrow$}\space},
  float=false,
  numbers=left,
  numberstyle=\tiny\color{gray},
  numbersep=10pt,
  xleftmargin=2em,
  keywordstyle=\color{blue},
  commentstyle=\color{green!60!black},
  stringstyle=\color{purple},
  backgroundcolor=\color{gray!5},
  showstringspaces=false,
  tabsize=2,
  captionpos=b,
  keepspaces=true,
  columns=flexible
}

\pgfplotsset{compat=1.18}
\usetikzlibrary{shapes,arrows,positioning,calc,patterns,decorations.pathmorphing,decorations.markings,arrows.meta}

% Color scheme
\definecolor{headcolor}{RGB}{0,102,204}
\definecolor{keycolor}{RGB}{220,20,60}
\definecolor{solutioncolor}{RGB}{34,139,34}
\definecolor{mnemoniccolor}{RGB}{148,0,211}
\definecolor{codecolor}{RGB}{0,0,100}

% Spacing
\setlength{\parskip}{3pt}
\setlist[itemize]{nosep}
\setlist[enumerate]{nosep}

% Title formatting
\titleformat{\section}{\Large\bfseries\color{headcolor}}{\thesection}{1em}{}
\titleformat{\subsection}{\large\bfseries\color{headcolor}}{\thesubsection}{1em}{}

% Pandoc tightlist compatibility
\providecommand{\tightlist}{%
  \setlength{\itemsep}{0pt}\setlength{\parskip}{0pt}}

% Pandoc longtable compatibility
\newcounter{none}
\def\thenone{}


% content/resources/templates/english-boxes.tex
% This file is currently empty - it exists to maintain consistency with the import structure.
% Add custom environments here if needed in the future.


\begin{document}

\begin{center}
{\Huge\bfseries\color{headcolor} Subject Name Solutions}\\[5pt]
{\LARGE 4361603 -- Summer 2025}\\[3pt]
{\large Semester 1 Study Material}\\[3pt]
{\normalsize\textit{Detailed Solutions and Explanations}}
\end{center}

\vspace{10pt}

\subsection*{Question 1(a) [3 marks]}\label{q1a}

\textbf{Differentiate between Private key and Public key in Blockchain.}

\begin{solutionbox}

{\def\LTcaptype{none} % do not increment counter
\begin{longtable}[]{@{}
  >{\raggedright\arraybackslash}p{(\linewidth - 4\tabcolsep) * \real{0.2667}}
  >{\raggedright\arraybackslash}p{(\linewidth - 4\tabcolsep) * \real{0.3778}}
  >{\raggedright\arraybackslash}p{(\linewidth - 4\tabcolsep) * \real{0.3556}}@{}}
\toprule\noalign{}
\begin{minipage}[b]{\linewidth}\raggedright
\textbf{Aspect}
\end{minipage} & \begin{minipage}[b]{\linewidth}\raggedright
\textbf{Private Key}
\end{minipage} & \begin{minipage}[b]{\linewidth}\raggedright
\textbf{Public Key}
\end{minipage} \\
\midrule\noalign{}
\endhead
\bottomrule\noalign{}
\endlastfoot
\textbf{Purpose} & Used for signing transactions & Used for
verification \\
\textbf{Sharing} & Must be kept secret & Can be shared publicly \\
\textbf{Function} & Decrypts data, creates signatures & Encrypts data,
verifies signatures \\
\textbf{Ownership} & Only owner knows it & Everyone can access it \\
\end{longtable}
}

\begin{itemize}
\tightlist
\item
  \textbf{Private Key}: Secret mathematical code that proves ownership
\item
  \textbf{Public Key}: Open address that others use to send transactions
\item
  \textbf{Security}: Private key loss = permanent fund loss
\end{itemize}

\end{solutionbox}
\begin{mnemonicbox}
``Private is Personal, Public is Posted''

\end{mnemonicbox}
\begin{center}\rule{0.5\linewidth}{0.5pt}\end{center}

\subsection*{Question 1(b) [4 marks]}\label{q1b}

\textbf{Explain Distributed Ledger in detail.}

\begin{solutionbox}

\textbf{Distributed Ledger} is a database spread across multiple
locations and participants.


{\def\LTcaptype{none} % do not increment counter
\vspace{-5pt}
\captionof{table}{Key Features}
\vspace{-10pt}
\begin{longtable}[]{@{}ll@{}}
\toprule\noalign{}
\textbf{Feature} & \textbf{Description} \\
\midrule\noalign{}
\endhead
\bottomrule\noalign{}
\endlastfoot
\textbf{Decentralized} & No single control point \\
\textbf{Synchronized} & All copies stay updated \\
\textbf{Transparent} & All participants can view \\
\textbf{Immutable} & Cannot be easily changed \\
\end{longtable}
}

\textbf{Diagram:}

\begin{center}
\textbf{Mermaid Diagram (Code)}
\begin{verbatim}
{Shaded}
{Highlighting}[]
graph TD
    A[Participant 1] {-{-}{} D[Distributed Ledger]}
    B[Participant 2] {-{-}{} D}
    C[Participant 3] {-{-}{} D}
    D {-{-}{} E[Synchronized Copy 1]}
    D {-{-}{} F[Synchronized Copy 2]}
    D {-{-}{} G[Synchronized Copy 3]}
{Highlighting}
{Shaded}
\end{verbatim}
\end{center}

\begin{itemize}
\tightlist
\item
  \textbf{Benefits}: Eliminates intermediaries, increases trust, reduces
  fraud
\item
  \textbf{Working}: All participants maintain identical copies of
  records
\end{itemize}

\end{solutionbox}
\begin{mnemonicbox}
``Distributed = Divided but Identical''

\end{mnemonicbox}
\begin{center}\rule{0.5\linewidth}{0.5pt}\end{center}

\subsection*{Question 1(c) [7 marks]}\label{q1c}

\textbf{Define Blockchain. Describe applications and limits of
Blockchain.}

\begin{solutionbox}

\textbf{Blockchain Definition}: A chain of blocks containing transaction
records, linked using cryptography.

\textbf{Applications Table:}

{\def\LTcaptype{none} % do not increment counter
\begin{longtable}[]{@{}lll@{}}
\toprule\noalign{}
\textbf{Sector} & \textbf{Application} & \textbf{Benefit} \\
\midrule\noalign{}
\endhead
\bottomrule\noalign{}
\endlastfoot
\textbf{Finance} & Cryptocurrency, payments & Faster, cheaper
transfers \\
\textbf{Healthcare} & Patient records & Secure, accessible data \\
\textbf{Supply Chain} & Product tracking & Transparency, authenticity \\
\textbf{Real Estate} & Property records & Fraud prevention \\
\textbf{Voting} & Digital elections & Transparent, tamper-proof \\
\end{longtable}
}

\textbf{Limits Table:}

{\def\LTcaptype{none} % do not increment counter
\begin{longtable}[]{@{}ll@{}}
\toprule\noalign{}
\textbf{Limitation} & \textbf{Impact} \\
\midrule\noalign{}
\endhead
\bottomrule\noalign{}
\endlastfoot
\textbf{Scalability} & Slow transaction processing \\
\textbf{Energy Usage} & High electricity consumption \\
\textbf{Complexity} & Difficult for users to understand \\
\textbf{Regulation} & Legal uncertainty \\
\textbf{Storage} & Growing data size problems \\
\end{longtable}
}

\textbf{Architecture Diagram:}

\begin{center}
\textbf{Mermaid Diagram (Code)}
\begin{verbatim}
{Shaded}
{Highlighting}[]
graph LR
    A[Block 1] {-{-}{} B[Block 2]}
    B {-{-}{} C[Block 3]}
    C {-{-}{} D[Block 4]}
    
    A1[Hash] {-{-}{} A}
    B1[Hash] {-{-}{} B}
    C1[Hash] {-{-}{} C}
    D1[Hash] {-{-}{} D}
{Highlighting}
{Shaded}
\end{verbatim}
\end{center}

\begin{itemize}
\tightlist
\item
  \textbf{Security}: Cryptographic linking makes tampering difficult
\item
  \textbf{Transparency}: All transactions visible to network
  participants
\end{itemize}

\end{solutionbox}
\begin{mnemonicbox}
``Blocks Chained = Blockchain, Apps Many = Limits
Many''

\end{mnemonicbox}
\begin{center}\rule{0.5\linewidth}{0.5pt}\end{center}

\subsection*{Question 1(c) OR [7
marks]}\label{q1c}

\textbf{Write a short note on: CAP Theorem in Blockchain}

\begin{solutionbox}

\textbf{CAP Theorem} states that distributed systems can only guarantee
2 out of 3 properties simultaneously.

\textbf{CAP Components Table:}

{\def\LTcaptype{none} % do not increment counter
\begin{longtable}[]{@{}
  >{\raggedright\arraybackslash}p{(\linewidth - 4\tabcolsep) * \real{0.3182}}
  >{\raggedright\arraybackslash}p{(\linewidth - 4\tabcolsep) * \real{0.3864}}
  >{\raggedright\arraybackslash}p{(\linewidth - 4\tabcolsep) * \real{0.2955}}@{}}
\toprule\noalign{}
\begin{minipage}[b]{\linewidth}\raggedright
\textbf{Property}
\end{minipage} & \begin{minipage}[b]{\linewidth}\raggedright
\textbf{Description}
\end{minipage} & \begin{minipage}[b]{\linewidth}\raggedright
\textbf{Example}
\end{minipage} \\
\midrule\noalign{}
\endhead
\bottomrule\noalign{}
\endlastfoot
\textbf{Consistency} & All nodes have same data & Same balance shown
everywhere \\
\textbf{Availability} & System always responds & Network never goes
down \\
\textbf{Partition Tolerance} & Works despite network failures &
Functions even if nodes disconnect \\
\end{longtable}
}

\textbf{Blockchain Trade-offs:}

\begin{center}
\textbf{Mermaid Diagram (Code)}
\begin{verbatim}
{Shaded}
{Highlighting}[]
graph TD
    A[CAP Theorem] {-{-}{} B[Consistency]}
    A {-{-}{} C[Availability]}
    A {-{-}{} D[Partition Tolerance]}
    
    E[Bitcoin] {-{-}{} B}
    E {-{-}{} D}
    F[Private Blockchain] {-{-}{} B}
    F {-{-}{} C}
{Highlighting}
{Shaded}
\end{verbatim}
\end{center}

\textbf{Real-world Applications:}

{\def\LTcaptype{none} % do not increment counter
\begin{longtable}[]{@{}
  >{\raggedright\arraybackslash}p{(\linewidth - 4\tabcolsep) * \real{0.4200}}
  >{\raggedright\arraybackslash}p{(\linewidth - 4\tabcolsep) * \real{0.2600}}
  >{\raggedright\arraybackslash}p{(\linewidth - 4\tabcolsep) * \real{0.3200}}@{}}
\toprule\noalign{}
\begin{minipage}[b]{\linewidth}\raggedright
\textbf{Blockchain Type}
\end{minipage} & \begin{minipage}[b]{\linewidth}\raggedright
\textbf{Chooses}
\end{minipage} & \begin{minipage}[b]{\linewidth}\raggedright
\textbf{Sacrifices}
\end{minipage} \\
\midrule\noalign{}
\endhead
\bottomrule\noalign{}
\endlastfoot
\textbf{Bitcoin} & Consistency + Partition & Availability \\
\textbf{Ethereum} & Consistency + Partition & Availability \\
\textbf{Private Networks} & Consistency + Availability & Partition
Tolerance \\
\end{longtable}
}

\begin{itemize}
\tightlist
\item
  \textbf{Impact}: Blockchain designers must choose which property to
  sacrifice
\item
  \textbf{Trade-off}: Perfect systems impossible in distributed networks
\end{itemize}

\end{solutionbox}
\begin{mnemonicbox}
``Can't Always Please - Choose 2 of 3''

\end{mnemonicbox}
\begin{center}\rule{0.5\linewidth}{0.5pt}\end{center}

\subsection*{Question 2(a) [3 marks]}\label{q2a}

\textbf{Explain Data Structure of a Blockchain.}

\begin{solutionbox}

\textbf{Blockchain Data Structure} consists of linked blocks containing
transaction data.

\textbf{Block Structure Table:}

{\def\LTcaptype{none} % do not increment counter
\begin{longtable}[]{@{}ll@{}}
\toprule\noalign{}
\textbf{Component} & \textbf{Purpose} \\
\midrule\noalign{}
\endhead
\bottomrule\noalign{}
\endlastfoot
\textbf{Block Header} & Contains metadata \\
\textbf{Previous Hash} & Links to previous block \\
\textbf{Merkle Root} & Summary of all transactions \\
\textbf{Timestamp} & When block was created \\
\textbf{Transactions} & Actual data/transfers \\
\end{longtable}
}

\textbf{Visual Structure:}

\begin{verbatim}
+{-{-}{-}{-}{-}{-}{-}{-}{-}{-}{-}{-}{-}{-}{-}{-}{-}{-}+}
|   Block Header   |
|{-{-}{-}{-}{-}{-}{-}{-}{-}{-}{-}{-}{-}{-}{-}{-}{-}{-}|}
| Previous Hash    |
| Merkle Root      |
| Timestamp        |
| Nonce            |
+{-{-}{-}{-}{-}{-}{-}{-}{-}{-}{-}{-}{-}{-}{-}{-}{-}{-}+}
|   Transactions   |
|  [TX1, TX2, TX3] |
+{-{-}{-}{-}{-}{-}{-}{-}{-}{-}{-}{-}{-}{-}{-}{-}{-}{-}+}
\end{verbatim}

\begin{itemize}
\tightlist
\item
  \textbf{Linking}: Each block points to previous block using hash
\item
  \textbf{Integrity}: Changing one block breaks the entire chain
\end{itemize}

\end{solutionbox}
\begin{mnemonicbox}
``Header Holds, Transactions Tell''

\end{mnemonicbox}
\begin{center}\rule{0.5\linewidth}{0.5pt}\end{center}

\subsection*{Question 2(b) [4 marks]}\label{q2b}

\textbf{What are the benefits of Decentralization?}

\begin{solutionbox}

\textbf{Decentralization Benefits:}

{\def\LTcaptype{none} % do not increment counter
\begin{longtable}[]{@{}ll@{}}
\toprule\noalign{}
\textbf{Benefit} & \textbf{Explanation} \\
\midrule\noalign{}
\endhead
\bottomrule\noalign{}
\endlastfoot
\textbf{No Single Point of Failure} & Network continues if one node
fails \\
\textbf{Censorship Resistance} & No authority can block transactions \\
\textbf{Transparency} & All participants see same information \\
\textbf{Reduced Costs} & Eliminates intermediary fees \\
\textbf{Trust} & No need to trust central authority \\
\end{longtable}
}

\textbf{Comparison Diagram:}

\begin{center}
\textbf{Mermaid Diagram (Code)}
\begin{verbatim}
{Shaded}
{Highlighting}[]
graph TD
    subgraph Centralized
        A[Central Authority] {-{-}{} B[User 1]}
        A {-{-}{} C[User 2]}
        A {-{-}{} D[User 3]}
    end
    
    subgraph Decentralized
            direction LR
        E[User 1] {-{-}{} F[User 2]}
        F {-{-}{} G[User 3]}
        G {-{-}{} E}
    end
{Highlighting}
{Shaded}
\end{verbatim}
\end{center}

\begin{itemize}
\tightlist
\item
  \textbf{Security}: Multiple copies prevent data loss
\item
  \textbf{Democracy}: All participants have equal rights
\item
  \textbf{Resilience}: System survives individual failures
\end{itemize}

\end{solutionbox}
\begin{mnemonicbox}
``Distributed = Durable, Democratic, Direct''

\end{mnemonicbox}
\begin{center}\rule{0.5\linewidth}{0.5pt}\end{center}

\subsection*{Question 2(c) [7 marks]}\label{q2c}

\textbf{Differentiate between Public Blockchain and Private Blockchain.}

\begin{solutionbox}

\textbf{Comprehensive Comparison:}

{\def\LTcaptype{none} % do not increment counter
\begin{longtable}[]{@{}lll@{}}
\toprule\noalign{}
\textbf{Aspect} & \textbf{Public Blockchain} & \textbf{Private
Blockchain} \\
\midrule\noalign{}
\endhead
\bottomrule\noalign{}
\endlastfoot
\textbf{Access} & Open to everyone & Restricted to specific users \\
\textbf{Permission} & Permissionless & Requires permission \\
\textbf{Control} & Decentralized & Centralized control \\
\textbf{Speed} & Slower (consensus needed) & Faster (fewer
validators) \\
\textbf{Security} & High (many validators) & Medium (fewer
validators) \\
\textbf{Cost} & Transaction fees required & Lower operational costs \\
\textbf{Transparency} & Fully transparent & Limited transparency \\
\textbf{Examples} & Bitcoin, Ethereum & Hyperledger, R3 Corda \\
\end{longtable}
}

\textbf{Network Architecture:}

\begin{center}
\textbf{Mermaid Diagram (Code)}
\begin{verbatim}
{Shaded}
{Highlighting}[]
graph TD
    subgraph "Public Blockchain"
        A[Anyone] {-{-}{} B[Global Network]}
        C[Anyone] {-{-}{} B}
        D[Anyone] {-{-}{} B}
    end
    
    subgraph "Private Blockchain"
        E[Authorized User 1] {-{-}{} F[Private Network]}
        G[Authorized User 2] {-{-}{} F}
        H[Authorized User 3] {-{-}{} F}
    end
{Highlighting}
{Shaded}
\end{verbatim}
\end{center}

\textbf{Use Cases:}

{\def\LTcaptype{none} % do not increment counter
\begin{longtable}[]{@{}ll@{}}
\toprule\noalign{}
\textbf{Type} & \textbf{Best For} \\
\midrule\noalign{}
\endhead
\bottomrule\noalign{}
\endlastfoot
\textbf{Public} & Cryptocurrencies, public records \\
\textbf{Private} & Banking, supply chain, healthcare \\
\end{longtable}
}

\begin{itemize}
\tightlist
\item
  \textbf{Trade-offs}: Public offers more security, Private offers more
  control
\item
  \textbf{Choice}: Depends on transparency vs.~privacy needs
\end{itemize}

\end{solutionbox}
\begin{mnemonicbox}
``Public = People's, Private = Permitted''

\end{mnemonicbox}
\begin{center}\rule{0.5\linewidth}{0.5pt}\end{center}

\subsection*{Question 2(a) OR [3
marks]}\label{q2a}

\textbf{Describe Core Components of Block Chain with suitable diagram.}

\begin{solutionbox}

\textbf{Core Components:}

{\def\LTcaptype{none} % do not increment counter
\begin{longtable}[]{@{}ll@{}}
\toprule\noalign{}
\textbf{Component} & \textbf{Function} \\
\midrule\noalign{}
\endhead
\bottomrule\noalign{}
\endlastfoot
\textbf{Blocks} & Store transaction data \\
\textbf{Hash Functions} & Create unique fingerprints \\
\textbf{Digital Signatures} & Verify transaction authenticity \\
\textbf{Consensus Mechanism} & Agree on valid transactions \\
\textbf{Peer-to-Peer Network} & Connect all participants \\
\end{longtable}
}

\textbf{System Architecture:}

\begin{center}
\textbf{Mermaid Diagram (Code)}
\begin{verbatim}
{Shaded}
{Highlighting}[]
graph LR
    A[Peer{-to{-}Peer Network] {-}{-}{} B[Consensus Mechanism]}
    B {-{-}{} C[Block Creation]}
    C {-{-}{} D[Hash Functions]}
    D {-{-}{} E[Digital Signatures]}
    E {-{-}{} F[Transaction Validation]}
    F {-{-}{} G[Block Addition]}
    G {-{-}{} H[Blockchain Updated]}
{Highlighting}
{Shaded}
\end{verbatim}
\end{center}

\begin{itemize}
\tightlist
\item
  \textbf{Integration}: All components work together for security
\item
  \textbf{Purpose}: Each component serves specific blockchain function
\end{itemize}

\end{solutionbox}
\begin{mnemonicbox}
``Blocks Build, Hash Holds, Signatures Secure''

\end{mnemonicbox}
\begin{center}\rule{0.5\linewidth}{0.5pt}\end{center}

\subsection*{Question 2(b) OR [4
marks]}\label{q2b}

\textbf{Define and explain permissioned blockchain in detail.}

\begin{solutionbox}

\textbf{Permissioned Blockchain Definition}: A blockchain where
participation requires explicit permission from network administrators.

\textbf{Characteristics Table:}

{\def\LTcaptype{none} % do not increment counter
\begin{longtable}[]{@{}ll@{}}
\toprule\noalign{}
\textbf{Feature} & \textbf{Description} \\
\midrule\noalign{}
\endhead
\bottomrule\noalign{}
\endlastfoot
\textbf{Access Control} & Only approved users can join \\
\textbf{Validation Rights} & Selected nodes validate transactions \\
\textbf{Governance} & Central authority manages network \\
\textbf{Privacy} & Transaction details can be private \\
\end{longtable}
}

\textbf{Permission Levels:}

\begin{center}
\textbf{Mermaid Diagram (Code)}
\begin{verbatim}
{Shaded}
{Highlighting}[]
graph TD
    A[Network Administrator] {-{-}{} B[Full Access]}
    A {-{-}{} C[Read/Write Access]}
    A {-{-}{} D[Read Only Access]}
    A {-{-}{} E[No Access]}
    
    B {-{-}{} F[Can validate blocks]}
    C {-{-}{} G[Can submit transactions]}
    D {-{-}{} H[Can view data only]}
    E {-{-}{} I[Blocked from network]}
{Highlighting}
{Shaded}
\end{verbatim}
\end{center}

\begin{itemize}
\tightlist
\item
  \textbf{Benefits}: Better privacy, regulatory compliance, faster
  processing
\item
  \textbf{Drawbacks}: Less decentralized, requires trust in
  administrators
\end{itemize}

\end{solutionbox}
\begin{mnemonicbox}
``Permission = Participation Permitted''

\end{mnemonicbox}
\begin{center}\rule{0.5\linewidth}{0.5pt}\end{center}

\subsection*{Question 2(c) OR [7
marks]}\label{q2c}

\textbf{Explain sidechain in brief.}

\begin{solutionbox}

\textbf{Sidechain Definition}: A separate blockchain connected to main
blockchain, allowing asset transfer between chains.

\textbf{Sidechain Architecture:}

\begin{center}
\textbf{Mermaid Diagram (Code)}
\begin{verbatim}
{Shaded}
{Highlighting}[]
graph LR
    A[Main Chain] {{-}{-}{} B[Sidechain 1]}
    A {{-}{-}{} C[Sidechain 2]}
    A {{-}{-}{} D[Sidechain 3]}
    
    B {-{-}{} E[Specific Purpose 1]}
    C {-{-}{} F[Specific Purpose 2]}
    D {-{-}{} G[Specific Purpose 3]}
{Highlighting}
{Shaded}
\end{verbatim}
\end{center}

\textbf{Benefits and Features:}

{\def\LTcaptype{none} % do not increment counter
\begin{longtable}[]{@{}ll@{}}
\toprule\noalign{}
\textbf{Aspect} & \textbf{Benefit} \\
\midrule\noalign{}
\endhead
\bottomrule\noalign{}
\endlastfoot
\textbf{Scalability} & Reduces main chain load \\
\textbf{Experimentation} & Test new features safely \\
\textbf{Specialization} & Optimized for specific use cases \\
\textbf{Interoperability} & Connect different blockchains \\
\end{longtable}
}

\textbf{Transfer Process:}

{\def\LTcaptype{none} % do not increment counter
\begin{longtable}[]{@{}ll@{}}
\toprule\noalign{}
\textbf{Step} & \textbf{Action} \\
\midrule\noalign{}
\endhead
\bottomrule\noalign{}
\endlastfoot
\textbf{1. Lock} & Assets locked on main chain \\
\textbf{2. Proof} & Cryptographic proof generated \\
\textbf{3. Release} & Equivalent assets released on sidechain \\
\textbf{4. Use} & Assets used on sidechain \\
\textbf{5. Return} & Reverse process to return assets \\
\end{longtable}
}

\textbf{Real Examples:}

{\def\LTcaptype{none} % do not increment counter
\begin{longtable}[]{@{}ll@{}}
\toprule\noalign{}
\textbf{Sidechain} & \textbf{Purpose} \\
\midrule\noalign{}
\endhead
\bottomrule\noalign{}
\endlastfoot
\textbf{Lightning Network} & Fast Bitcoin payments \\
\textbf{Plasma} & Ethereum scaling \\
\textbf{Liquid} & Bitcoin trading \\
\end{longtable}
}

\begin{itemize}
\tightlist
\item
  \textbf{Security}: Maintains connection to secure main chain
\item
  \textbf{Flexibility}: Each sidechain can have different rules
\item
  \textbf{Innovation}: Allows blockchain ecosystem expansion
\end{itemize}

\end{solutionbox}
\begin{mnemonicbox}
``Side Supports, Main Maintains''

\end{mnemonicbox}
\begin{center}\rule{0.5\linewidth}{0.5pt}\end{center}

\subsection*{Question 3(a) [3 marks]}\label{q3a}

\textbf{Define Consensus Mechanism and explain any one in detail.}

\begin{solutionbox}

\textbf{Consensus Mechanism Definition}: A protocol that ensures all
network participants agree on the blockchain's current state.

\textbf{Proof of Work (PoW) Explanation:}

{\def\LTcaptype{none} % do not increment counter
\begin{longtable}[]{@{}ll@{}}
\toprule\noalign{}
\textbf{Component} & \textbf{Function} \\
\midrule\noalign{}
\endhead
\bottomrule\noalign{}
\endlastfoot
\textbf{Mining} & Solving complex mathematical puzzles \\
\textbf{Competition} & Miners compete to solve first \\
\textbf{Verification} & Network verifies solution \\
\textbf{Reward} & Winner gets cryptocurrency reward \\
\end{longtable}
}

\textbf{PoW Process:}

\begin{center}
\textbf{Mermaid Diagram (Code)}
\begin{verbatim}
{Shaded}
{Highlighting}[]
graph LR
    A[New Transaction] {-{-}{} B[Miners Collect Transactions]}
    B {-{-}{} C[Create Block]}
    C {-{-}{} D[Solve Mathematical Puzzle]}
    D {-{-}{} E[First Solution Wins]}
    E {-{-}{} F[Block Added to Chain]}
    F {-{-}{} G[Miner Gets Reward]}
{Highlighting}
{Shaded}
\end{verbatim}
\end{center}

\begin{itemize}
\tightlist
\item
  \textbf{Security}: Computational work makes tampering expensive
\item
  \textbf{Example}: Bitcoin uses Proof of Work consensus
\end{itemize}

\end{solutionbox}
\begin{mnemonicbox}
``Consensus = Common Sense, Work = Win''

\end{mnemonicbox}
\begin{center}\rule{0.5\linewidth}{0.5pt}\end{center}

\subsection*{Question 3(b) [4 marks]}\label{q3b}

\textbf{Why is Forking needed in Blockchain? List various types of Forks
in Blockchain.}

\begin{solutionbox}

\textbf{Why Forking is Needed:}

{\def\LTcaptype{none} % do not increment counter
\begin{longtable}[]{@{}ll@{}}
\toprule\noalign{}
\textbf{Reason} & \textbf{Purpose} \\
\midrule\noalign{}
\endhead
\bottomrule\noalign{}
\endlastfoot
\textbf{Upgrades} & Add new features to blockchain \\
\textbf{Bug Fixes} & Correct security vulnerabilities \\
\textbf{Rule Changes} & Modify consensus rules \\
\textbf{Community Disagreement} & Split when no consensus reached \\
\end{longtable}
}

\textbf{Types of Forks:}

{\def\LTcaptype{none} % do not increment counter
\begin{longtable}[]{@{}lll@{}}
\toprule\noalign{}
\textbf{Fork Type} & \textbf{Description} & \textbf{Compatibility} \\
\midrule\noalign{}
\endhead
\bottomrule\noalign{}
\endlastfoot
\textbf{Soft Fork} & Tightens rules & Backward compatible \\
\textbf{Hard Fork} & Changes rules completely & Not backward
compatible \\
\textbf{Accidental Fork} & Temporary split & Resolves automatically \\
\textbf{Contentious Fork} & Community disagreement & Permanent split \\
\end{longtable}
}

\textbf{Fork Visualization:}

\begin{center}
\textbf{Mermaid Diagram (Code)}
\begin{verbatim}
{Shaded}
{Highlighting}[]
graph LR
    A[Original Chain] {-{-}{} B[Block N]}
    B {-{-}{} C[Soft Fork {-} Tighter Rules]}
    B {-{-}{} D[Hard Fork {-} New Rules]}
    
    C {-{-}{} E[Old nodes still work]}
    D {-{-}{} F[Old nodes rejected]}
{Highlighting}
{Shaded}
\end{verbatim}
\end{center}

\begin{itemize}
\tightlist
\item
  \textbf{Impact}: Forks can create new cryptocurrencies
\item
  \textbf{Examples}: Bitcoin Cash (hard fork), Ethereum updates (soft
  forks)
\end{itemize}

\end{solutionbox}
\begin{mnemonicbox}
``Fork = Future Options, Rules Kept''

\end{mnemonicbox}
\begin{center}\rule{0.5\linewidth}{0.5pt}\end{center}

\subsection*{Question 3(c) [7 marks]}\label{q3c}

\textbf{What is Bitcoin Mining? Explain working, difficulty and benefits
of Bitcoin mining in detail.}

\begin{solutionbox}

\textbf{Bitcoin Mining Definition}: Process of adding new transactions
to Bitcoin blockchain by solving computational puzzles.

\textbf{Mining Process:}

{\def\LTcaptype{none} % do not increment counter
\begin{longtable}[]{@{}lll@{}}
\toprule\noalign{}
\textbf{Step} & \textbf{Action} & \textbf{Details} \\
\midrule\noalign{}
\endhead
\bottomrule\noalign{}
\endlastfoot
\textbf{1. Collection} & Gather pending transactions & From mempool \\
\textbf{2. Block Creation} & Form new block & Include transactions \\
\textbf{3. Puzzle Solving} & Find correct nonce & Trial and error \\
\textbf{4. Verification} & Network checks solution & Validates block \\
\textbf{5. Addition} & Add block to chain & Permanent record \\
\textbf{6. Reward} & Miner gets Bitcoin & Currently 6.25 BTC \\
\end{longtable}
}

\textbf{Mining Workflow:}

\begin{center}
\textbf{Mermaid Diagram (Code)}
\begin{verbatim}
{Shaded}
{Highlighting}[]
graph LR
    A[Pending Transactions] {-{-}{} B[Miners Collect]}
    B {-{-}{} C[Create Block Header]}
    C {-{-}{} D[Guess Nonce Value]}
    D {-{-}{} E[Calculate Hash]}
    E {-{-}{} F\{Hash {} Target?\}}
    F {-{-}{}|No| D}
    F {-{-}{}|Yes| G[Broadcast Solution]}
    G {-{-}{} H[Network Validates]}
    H {-{-}{} I[Block Added + Reward]}
{Highlighting}
{Shaded}
\end{verbatim}
\end{center}

\textbf{Difficulty Adjustment:}

{\def\LTcaptype{none} % do not increment counter
\begin{longtable}[]{@{}ll@{}}
\toprule\noalign{}
\textbf{Aspect} & \textbf{Mechanism} \\
\midrule\noalign{}
\endhead
\bottomrule\noalign{}
\endlastfoot
\textbf{Target Time} & 10 minutes per block \\
\textbf{Adjustment Period} & Every 2016 blocks (\textasciitilde2
weeks) \\
\textbf{Auto-Regulation} & Increases if blocks too fast \\
\textbf{Purpose} & Maintain consistent block time \\
\end{longtable}
}

\textbf{Benefits of Mining:}

{\def\LTcaptype{none} % do not increment counter
\begin{longtable}[]{@{}ll@{}}
\toprule\noalign{}
\textbf{Benefit} & \textbf{Description} \\
\midrule\noalign{}
\endhead
\bottomrule\noalign{}
\endlastfoot
\textbf{Financial Reward} & Earn Bitcoin for successful mining \\
\textbf{Network Security} & More miners = more secure network \\
\textbf{Transaction Processing} & Enables Bitcoin transfers \\
\textbf{Decentralization} & No central authority needed \\
\end{longtable}
}

\begin{itemize}
\tightlist
\item
  \textbf{Energy}: Mining requires significant electricity
\item
  \textbf{Competition}: Difficulty increases with more miners
\item
  \textbf{Hardware}: Specialized ASIC miners most efficient
\end{itemize}

\end{solutionbox}
\begin{mnemonicbox}
``Mining = Money, Math, Maintenance''

\end{mnemonicbox}
\begin{center}\rule{0.5\linewidth}{0.5pt}\end{center}

\subsection*{Question 3(a) OR [3
marks]}\label{q3a}

\textbf{Differentiate Soft fork and Hard fork.}

\begin{solutionbox}

\textbf{Fork Comparison:}

{\def\LTcaptype{none} % do not increment counter
\begin{longtable}[]{@{}lll@{}}
\toprule\noalign{}
\textbf{Aspect} & \textbf{Soft Fork} & \textbf{Hard Fork} \\
\midrule\noalign{}
\endhead
\bottomrule\noalign{}
\endlastfoot
\textbf{Compatibility} & Backward compatible & Not backward
compatible \\
\textbf{Rules} & Makes rules stricter & Changes rules completely \\
\textbf{Node Updates} & Optional for old nodes & Mandatory for all
nodes \\
\textbf{Chain Split} & No permanent split & Can create permanent
split \\
\textbf{Consensus} & Easier to implement & Requires majority
agreement \\
\textbf{Examples} & SegWit (Bitcoin) & Bitcoin Cash, Ethereum Classic \\
\end{longtable}
}

\textbf{Visual Representation:}

\begin{center}
\textbf{Mermaid Diagram (Code)}
\begin{verbatim}
{Shaded}
{Highlighting}[]
graph LR
    A[Original Blockchain] {-{-}{} B[Fork Point]}
    B {-{-}{} C[Soft Fork {-} Stricter Rules]}
    B {-{-}{} D[Hard Fork {-} New Rules]}
    
    C {-{-}{} E[Old nodes still valid]}
    D {-{-}{} F[Old nodes incompatible]}
    
    E {-{-}{} G[Single chain continues]}
    F {-{-}{} H[Two separate chains]}
{Highlighting}
{Shaded}
\end{verbatim}
\end{center}

\begin{itemize}
\tightlist
\item
  \textbf{Risk}: Hard forks can split community and create competing
  currencies
\item
  \textbf{Safety}: Soft forks are generally safer and less disruptive
\end{itemize}

\end{solutionbox}
\begin{mnemonicbox}
``Soft = Same Direction, Hard = Huge Difference''

\end{mnemonicbox}
\begin{center}\rule{0.5\linewidth}{0.5pt}\end{center}

\subsection*{Question 3(b) OR [4
marks]}\label{q3b}

\textbf{What is the importance of Finality in the World of Blockchain?}

\begin{solutionbox}

\textbf{Finality Definition}: The guarantee that once a transaction is
confirmed, it cannot be reversed or altered.

\textbf{Importance Table:}

{\def\LTcaptype{none} % do not increment counter
\begin{longtable}[]{@{}ll@{}}
\toprule\noalign{}
\textbf{Aspect} & \textbf{Importance} \\
\midrule\noalign{}
\endhead
\bottomrule\noalign{}
\endlastfoot
\textbf{Trust} & Users confident transactions are permanent \\
\textbf{Business Use} & Companies can rely on completed transactions \\
\textbf{Legal Certainty} & Courts can enforce blockchain records \\
\textbf{Settlement} & Financial institutions can clear payments \\
\end{longtable}
}

\textbf{Types of Finality:}

{\def\LTcaptype{none} % do not increment counter
\begin{longtable}[]{@{}
  >{\raggedright\arraybackslash}p{(\linewidth - 4\tabcolsep) * \real{0.2703}}
  >{\raggedright\arraybackslash}p{(\linewidth - 4\tabcolsep) * \real{0.4595}}
  >{\raggedright\arraybackslash}p{(\linewidth - 4\tabcolsep) * \real{0.2703}}@{}}
\toprule\noalign{}
\begin{minipage}[b]{\linewidth}\raggedright
\textbf{Type}
\end{minipage} & \begin{minipage}[b]{\linewidth}\raggedright
\textbf{Description}
\end{minipage} & \begin{minipage}[b]{\linewidth}\raggedright
\textbf{Time}
\end{minipage} \\
\midrule\noalign{}
\endhead
\bottomrule\noalign{}
\endlastfoot
\textbf{Probabilistic} & Becomes more certain over time & Bitcoin:
\textasciitilde1 hour \\
\textbf{Absolute} & Immediate guarantee & Some private chains \\
\textbf{Economic} & Cost of reversal too high & Varies by network \\
\end{longtable}
}

\textbf{Finality Process:}

\begin{center}
\textbf{Mermaid Diagram (Code)}
\begin{verbatim}
{Shaded}
{Highlighting}[]
graph LR
    A[Transaction Submitted] {-{-}{} B[First Confirmation]}
    B {-{-}{} C[Multiple Confirmations]}
    C {-{-}{} D[Probabilistic Finality]}
    D {-{-}{} E[Practical Finality]}
{Highlighting}
{Shaded}
\end{verbatim}
\end{center}

\begin{itemize}
\tightlist
\item
  \textbf{Bitcoin}: 6 confirmations generally considered final
\item
  \textbf{Ethereum}: Moving toward faster finality with Proof of Stake
\item
  \textbf{Challenge}: Balance between speed and security
\end{itemize}

\end{solutionbox}
\begin{mnemonicbox}
``Final = Forever, Important = Irreversible''

\end{mnemonicbox}
\begin{center}\rule{0.5\linewidth}{0.5pt}\end{center}

\subsection*{Question 3(c) OR [7
marks]}\label{q3c}

\textbf{What is a 51\% attack in Blockchain? Explain in brief.}

\begin{solutionbox}

\textbf{51\% Attack Definition}: When a single entity controls more than
50\% of network's mining power or validators, allowing them to
manipulate the blockchain.

\textbf{Attack Mechanism:}

{\def\LTcaptype{none} % do not increment counter
\begin{longtable}[]{@{}
  >{\raggedright\arraybackslash}p{(\linewidth - 4\tabcolsep) * \real{0.2326}}
  >{\raggedright\arraybackslash}p{(\linewidth - 4\tabcolsep) * \real{0.4884}}
  >{\raggedright\arraybackslash}p{(\linewidth - 4\tabcolsep) * \real{0.2791}}@{}}
\toprule\noalign{}
\begin{minipage}[b]{\linewidth}\raggedright
\textbf{Step}
\end{minipage} & \begin{minipage}[b]{\linewidth}\raggedright
\textbf{Attacker Action}
\end{minipage} & \begin{minipage}[b]{\linewidth}\raggedright
\textbf{Impact}
\end{minipage} \\
\midrule\noalign{}
\endhead
\bottomrule\noalign{}
\endlastfoot
\textbf{1. Control} & Gain \textgreater50\% mining power & Dominate
network \\
\textbf{2. Double Spend} & Create secret chain & Prepare alternative
history \\
\textbf{3. Execute} & Release longer chain & Network accepts fake
version \\
\textbf{4. Profit} & Spend coins twice & Steal from victims \\
\end{longtable}
}

\textbf{Attack Visualization:}

\begin{center}
\textbf{Mermaid Diagram (Code)}
\begin{verbatim}
{Shaded}
{Highlighting}[]
graph LR
    A[Honest Chain] {-{-}{} B[Block N]}
    C[Attacker{s Secret Chain] {-}{-}{} D[Block N{}]}
    
    B {-{-}{} E[Block N+1]}
    D {-{-}{} F[Block N{}+1]}
    D {-{-}{} G[Block N{}+2 {-} Longer Chain]}
    
    G {-{-}{} H[Network Accepts Attacker{}s Chain]}
    E {-{-}{} I[Honest Chain Abandoned]}
{Highlighting}
{Shaded}
\end{verbatim}
\end{center}

\textbf{Possible Attacks:}

{\def\LTcaptype{none} % do not increment counter
\begin{longtable}[]{@{}ll@{}}
\toprule\noalign{}
\textbf{Attack Type} & \textbf{Description} \\
\midrule\noalign{}
\endhead
\bottomrule\noalign{}
\endlastfoot
\textbf{Double Spending} & Spend same coins twice \\
\textbf{Transaction Reversal} & Cancel confirmed transactions \\
\textbf{Mining Monopoly} & Block other miners' work \\
\textbf{Censorship} & Prevent specific transactions \\
\end{longtable}
}

\textbf{Prevention Methods:}

{\def\LTcaptype{none} % do not increment counter
\begin{longtable}[]{@{}ll@{}}
\toprule\noalign{}
\textbf{Method} & \textbf{How It Helps} \\
\midrule\noalign{}
\endhead
\bottomrule\noalign{}
\endlastfoot
\textbf{Decentralization} & Spread mining across many participants \\
\textbf{High Hash Rate} & Make attack economically unfeasible \\
\textbf{Proof of Stake} & Attackers lose their staked coins \\
\textbf{Monitoring} & Detect suspicious mining activity \\
\end{longtable}
}

\textbf{Real Examples:}

{\def\LTcaptype{none} % do not increment counter
\begin{longtable}[]{@{}ll@{}}
\toprule\noalign{}
\textbf{Blockchain} & \textbf{Status} \\
\midrule\noalign{}
\endhead
\bottomrule\noalign{}
\endlastfoot
\textbf{Bitcoin} & Never successfully attacked \\
\textbf{Ethereum Classic} & Attacked multiple times \\
\textbf{Small Altcoins} & More vulnerable due to low hash rate \\
\end{longtable}
}

\begin{itemize}
\tightlist
\item
  \textbf{Cost}: Attacking major networks extremely expensive
\item
  \textbf{Detection}: Attacks usually detected quickly
\item
  \textbf{Recovery}: Networks can implement countermeasures
\end{itemize}

\end{solutionbox}
\begin{mnemonicbox}
``51\% = Majority Mischief, Control = Chaos''

\end{mnemonicbox}
\begin{center}\rule{0.5\linewidth}{0.5pt}\end{center}

\subsection*{Question 4(a) [3 marks]}\label{q4a}

\textbf{Describe various types of Hyperledger projects.}

\begin{solutionbox}

\textbf{Hyperledger Project Types:}

{\def\LTcaptype{none} % do not increment counter
\begin{longtable}[]{@{}lll@{}}
\toprule\noalign{}
\textbf{Project} & \textbf{Purpose} & \textbf{Use Case} \\
\midrule\noalign{}
\endhead
\bottomrule\noalign{}
\endlastfoot
\textbf{Fabric} & Modular blockchain platform & Enterprise
applications \\
\textbf{Sawtooth} & Scalable blockchain suite & Supply chain, IoT \\
\textbf{Iroha} & Mobile-focused blockchain & Identity management \\
\textbf{Indy} & Digital identity platform & Self-sovereign identity \\
\textbf{Besu} & Ethereum-compatible client & Public/private Ethereum \\
\textbf{Burrow} & Smart contract platform & Permissioned networks \\
\end{longtable}
}

\textbf{Project Categories:}

\begin{center}
\textbf{Mermaid Diagram (Code)}
\begin{verbatim}
{Shaded}
{Highlighting}[]
graph TD
    A[Hyperledger Projects] {-{-}{} B[Frameworks]}
    A {-{-}{} C[Tools]}
    
    B {-{-}{} D[Fabric {-} Enterprise]}
    B {-{-}{} E[Sawtooth {-} Scalable]}
    B {-{-}{} F[Iroha {-} Mobile]}
    
    C {-{-}{} G[Caliper {-} Performance]}
    C {-{-}{} H[Composer {-} Development]}
    C {-{-}{} I[Explorer {-} Monitoring]}
{Highlighting}
{Shaded}
\end{verbatim}
\end{center}

\begin{itemize}
\tightlist
\item
  \textbf{Focus}: Enterprise and business blockchain solutions
\item
  \textbf{Open Source}: All projects are freely available
\end{itemize}

\end{solutionbox}
\begin{mnemonicbox}
``Hyper = High Performance, Ledger = Large
Enterprise''

\end{mnemonicbox}
\begin{center}\rule{0.5\linewidth}{0.5pt}\end{center}

\subsection*{Question 4(b) [4 marks]}\label{q4b}

\textbf{Differentiate between Blockchain and Bitcoin.}

\begin{solutionbox}

\textbf{Comprehensive Comparison:}

{\def\LTcaptype{none} % do not increment counter
\begin{longtable}[]{@{}lll@{}}
\toprule\noalign{}
\textbf{Aspect} & \textbf{Blockchain} & \textbf{Bitcoin} \\
\midrule\noalign{}
\endhead
\bottomrule\noalign{}
\endlastfoot
\textbf{Definition} & Technology/Platform & Digital Currency \\
\textbf{Scope} & Broader concept & Specific application \\
\textbf{Purpose} & Record keeping system & Peer-to-peer payments \\
\textbf{Applications} & Many industries & Primarily financial \\
\textbf{Flexibility} & Can be customized & Fixed protocol \\
\textbf{Creator} & Multiple contributors & Satoshi Nakamoto \\
\textbf{Launch} & Concept evolved over time & Launched 2009 \\
\end{longtable}
}

\textbf{Relationship Diagram:}

\begin{center}
\textbf{Mermaid Diagram (Code)}
\begin{verbatim}
{Shaded}
{Highlighting}[]
graph TD
    A[Blockchain Technology] {-{-}{} B[Bitcoin Cryptocurrency]}
    A {-{-}{} C[Ethereum Platform]}
    A {-{-}{} D[Supply Chain Apps]}
    A {-{-}{} E[Healthcare Records]}
    
    B {-{-}{} F[Digital Payments]}
    B {-{-}{} G[Store of Value]}
{Highlighting}
{Shaded}
\end{verbatim}
\end{center}

\textbf{Key Differences:}

{\def\LTcaptype{none} % do not increment counter
\begin{longtable}[]{@{}lll@{}}
\toprule\noalign{}
\textbf{Category} & \textbf{Blockchain} & \textbf{Bitcoin} \\
\midrule\noalign{}
\endhead
\bottomrule\noalign{}
\endlastfoot
\textbf{Type} & Infrastructure & Application \\
\textbf{Usage} & Multiple purposes & Currency only \\
\textbf{Modifications} & Can be changed & Protocol fixed \\
\end{longtable}
}

\begin{itemize}
\tightlist
\item
  \textbf{Analogy}: Blockchain is like the internet, Bitcoin is like
  email
\item
  \textbf{Dependency}: Bitcoin needs blockchain, but blockchain doesn't
  need Bitcoin
\end{itemize}

\end{solutionbox}
\begin{mnemonicbox}
``Blockchain = Building Block, Bitcoin = Specific
Brick''

\end{mnemonicbox}
\begin{center}\rule{0.5\linewidth}{0.5pt}\end{center}

\subsection*{Question 4(c) [7 marks]}\label{q4c}

\textbf{Write a short note on: Merkle Tree}

\begin{solutionbox}

\textbf{Merkle Tree Definition}: A binary tree structure where each leaf
represents a transaction hash, and each internal node contains the hash
of its children.

\textbf{Structure and Components:}

{\def\LTcaptype{none} % do not increment counter
\begin{longtable}[]{@{}ll@{}}
\toprule\noalign{}
\textbf{Component} & \textbf{Description} \\
\midrule\noalign{}
\endhead
\bottomrule\noalign{}
\endlastfoot
\textbf{Leaf Nodes} & Individual transaction hashes \\
\textbf{Internal Nodes} & Hash of two child nodes \\
\textbf{Root Hash} & Single hash representing entire tree \\
\textbf{Path} & Route from leaf to root \\
\end{longtable}
}

\textbf{Merkle Tree Diagram:}

\begin{verbatim}
                    Root Hash
                   /         {}
              Hash AB       Hash CD
             /       {     /       }
        Hash A   Hash B Hash C   Hash D
          |        |      |        |
        TX A     TX B   TX C     TX D
\end{verbatim}

\textbf{Construction Process:}

{\def\LTcaptype{none} % do not increment counter
\begin{longtable}[]{@{}ll@{}}
\toprule\noalign{}
\textbf{Step} & \textbf{Action} \\
\midrule\noalign{}
\endhead
\bottomrule\noalign{}
\endlastfoot
\textbf{1} & Hash each transaction individually \\
\textbf{2} & Pair hashes and hash them together \\
\textbf{3} & Continue pairing until single root \\
\textbf{4} & Store root hash in block header \\
\end{longtable}
}

\textbf{Benefits Table:}

{\def\LTcaptype{none} % do not increment counter
\begin{longtable}[]{@{}ll@{}}
\toprule\noalign{}
\textbf{Benefit} & \textbf{Explanation} \\
\midrule\noalign{}
\endhead
\bottomrule\noalign{}
\endlastfoot
\textbf{Efficiency} & Quick verification without downloading all data \\
\textbf{Security} & Any change detected immediately \\
\textbf{Scalability} & Verification time stays constant \\
\textbf{Storage} & Only root hash needed in block header \\
\end{longtable}
}

\textbf{Verification Process:}

\begin{center}
\textbf{Mermaid Diagram (Code)}
\begin{verbatim}
{Shaded}
{Highlighting}[]
graph LR
    A[Transaction to Verify] {-{-}{} B[Get Merkle Path]}
    B {-{-}{} C[Hash with Sibling Nodes]}
    C {-{-}{} D[Compute Path to Root]}
    D {-{-}{} E[Compare with Stored Root]}
    E {-{-}{} F\{Match?\}}
    F {-{-}{}|Yes| G[Valid Transaction]}
    F {-{-}{}|No| H[Invalid Transaction]}
{Highlighting}
{Shaded}
\end{verbatim}
\end{center}

\textbf{Real-world Applications:}

{\def\LTcaptype{none} % do not increment counter
\begin{longtable}[]{@{}ll@{}}
\toprule\noalign{}
\textbf{Use Case} & \textbf{Application} \\
\midrule\noalign{}
\endhead
\bottomrule\noalign{}
\endlastfoot
\textbf{Bitcoin} & Transaction verification \\
\textbf{Git} & Version control \\
\textbf{IPFS} & Distributed storage \\
\textbf{Certificate Transparency} & SSL certificate logs \\
\end{longtable}
}

\begin{itemize}
\tightlist
\item
  \textbf{Inventor}: Named after Ralph Merkle (1988)
\item
  \textbf{Efficiency}: Allows verification with O(log n) complexity
\item
  \textbf{Security}: Tampering with any transaction changes root hash
\end{itemize}

\end{solutionbox}
\begin{mnemonicbox}
``Merkle = Many Made One, Tree = Trustworthy''

\end{mnemonicbox}
\begin{center}\rule{0.5\linewidth}{0.5pt}\end{center}

\subsection*{Question 4(a) OR [3
marks]}\label{q4a}

\textbf{Discuss briefly about Hash pointer and how it is used in Merkle
tree.}

\begin{solutionbox}

\textbf{Hash Pointer Definition}: A data structure containing both the
location of data and cryptographic hash of that data.

\textbf{Components:}

{\def\LTcaptype{none} % do not increment counter
\begin{longtable}[]{@{}ll@{}}
\toprule\noalign{}
\textbf{Component} & \textbf{Purpose} \\
\midrule\noalign{}
\endhead
\bottomrule\noalign{}
\endlastfoot
\textbf{Pointer} & Shows where data is stored \\
\textbf{Hash} & Proves data hasn't changed \\
\textbf{Combination} & Links data with integrity check \\
\end{longtable}
}

\textbf{Hash Pointer in Merkle Tree:}

\begin{verbatim}
        Root Hash Pointer
       /                 {}
   Hash Ptr AB        Hash Ptr CD
   /         {        /         }
Hash A     Hash B  Hash C     Hash D
  |          |       |          |
 TX A       TX B    TX C       TX D
\end{verbatim}

\textbf{Usage in Merkle Tree:}

{\def\LTcaptype{none} % do not increment counter
\begin{longtable}[]{@{}ll@{}}
\toprule\noalign{}
\textbf{Level} & \textbf{Hash Pointer Function} \\
\midrule\noalign{}
\endhead
\bottomrule\noalign{}
\endlastfoot
\textbf{Leaf Level} & Points to transaction, contains transaction
hash \\
\textbf{Internal Nodes} & Points to children, contains combined hash \\
\textbf{Root} & Points to tree structure, contains overall hash \\
\end{longtable}
}

\begin{itemize}
\tightlist
\item
  \textbf{Verification}: Can detect any change in tree structure
\item
  \textbf{Navigation}: Allows efficient traversal of tree
\end{itemize}

\end{solutionbox}
\begin{mnemonicbox}
``Hash Pointer = Location + Verification''

\end{mnemonicbox}
\begin{center}\rule{0.5\linewidth}{0.5pt}\end{center}

\subsection*{Question 4(b) OR [4
marks]}\label{q4b}

\textbf{What is Hashing in Blockchain? How it is useful in Bitcoin?}

\begin{solutionbox}

\textbf{Hashing Definition}: Mathematical function that converts input
data into fixed-size string of characters.

\textbf{Hashing Properties:}

{\def\LTcaptype{none} % do not increment counter
\begin{longtable}[]{@{}
  >{\raggedright\arraybackslash}p{(\linewidth - 2\tabcolsep) * \real{0.4516}}
  >{\raggedright\arraybackslash}p{(\linewidth - 2\tabcolsep) * \real{0.5484}}@{}}
\toprule\noalign{}
\begin{minipage}[b]{\linewidth}\raggedright
\textbf{Property}
\end{minipage} & \begin{minipage}[b]{\linewidth}\raggedright
\textbf{Description}
\end{minipage} \\
\midrule\noalign{}
\endhead
\bottomrule\noalign{}
\endlastfoot
\textbf{Deterministic} & Same input always produces same output \\
\textbf{Fixed Size} & Output always same length (256 bits for
SHA-256) \\
\textbf{Avalanche Effect} & Small input change = completely different
output \\
\textbf{One-way} & Cannot reverse to find original input \\
\textbf{Collision Resistant} & Extremely hard to find two inputs with
same output \\
\end{longtable}
}

\textbf{Bitcoin Usage:}

{\def\LTcaptype{none} % do not increment counter
\begin{longtable}[]{@{}ll@{}}
\toprule\noalign{}
\textbf{Use Case} & \textbf{Purpose} \\
\midrule\noalign{}
\endhead
\bottomrule\noalign{}
\endlastfoot
\textbf{Block Linking} & Each block contains hash of previous block \\
\textbf{Mining} & Find hash meeting difficulty requirement \\
\textbf{Transaction IDs} & Unique identifier for each transaction \\
\textbf{Merkle Root} & Summarize all transactions in block \\
\textbf{Addresses} & Create Bitcoin addresses from public keys \\
\end{longtable}
}

\textbf{Hashing Process:}

\begin{center}
\textbf{Mermaid Diagram (Code)}
\begin{verbatim}
{Shaded}
{Highlighting}[]
graph LR
    A[Input Data] {-{-}{} B[SHA{-}256 Function]}
    B {-{-}{} C[256{-}bit Hash Output]}
    
    D[Small Change in Input] {-{-}{} E[SHA{-}256 Function]}
    E {-{-}{} F[Completely Different Hash]}
{Highlighting}
{Shaded}
\end{verbatim}
\end{center}

\begin{itemize}
\tightlist
\item
  \textbf{Algorithm}: Bitcoin uses SHA-256 hashing
\item
  \textbf{Security}: Makes blockchain tamper-evident
\item
  \textbf{Efficiency}: Quick to compute and verify
\end{itemize}

\end{solutionbox}
\begin{mnemonicbox}
``Hash = Fingerprint, Bitcoin = Built on Hashing''

\end{mnemonicbox}
\begin{center}\rule{0.5\linewidth}{0.5pt}\end{center}

\subsection*{Question 4(c) OR [7
marks]}\label{q4c}

\textbf{Explain classic Byzantine generals problem and Practical
Byzantine Fault Tolerance in detail.}

\begin{solutionbox}

\textbf{Byzantine Generals Problem}: A classic computer science problem
about achieving consensus in distributed systems with potentially
unreliable participants.

\textbf{Problem Scenario:}

{\def\LTcaptype{none} % do not increment counter
\begin{longtable}[]{@{}ll@{}}
\toprule\noalign{}
\textbf{Element} & \textbf{Description} \\
\midrule\noalign{}
\endhead
\bottomrule\noalign{}
\endlastfoot
\textbf{Generals} & Represent network nodes \\
\textbf{City} & Represents the system state \\
\textbf{Attack Plan} & Represents consensus decision \\
\textbf{Traitors} & Represent malicious/faulty nodes \\
\textbf{Communication} & Messages between nodes \\
\end{longtable}
}

\textbf{Problem Visualization:}

\begin{center}
\textbf{Mermaid Diagram (Code)}
\begin{verbatim}
{Shaded}
{Highlighting}[]
graph TD
    A[General A {- Honest] {-}{-}{} D[City Under Siege]}
    B[General B {- Traitor] {-}{-}{} D}
    C[General C {- Honest] {-}{-}{} D}
    E[General D {- Honest] {-}{-}{} D}
    
    A {-{-}{} F[Vote: Attack]}
    B {-{-}{} G[Vote: Attack to A, Retreat to C]}
    C {-{-}{} H[Vote: Attack]}
    E {-{-}{} I[Vote: Attack]}
{Highlighting}
{Shaded}
\end{verbatim}
\end{center}

\textbf{Practical Byzantine Fault Tolerance (pBFT):}

\textbf{pBFT Algorithm Phases:}

{\def\LTcaptype{none} % do not increment counter
\begin{longtable}[]{@{}
  >{\raggedright\arraybackslash}p{(\linewidth - 4\tabcolsep) * \real{0.3056}}
  >{\raggedright\arraybackslash}p{(\linewidth - 4\tabcolsep) * \real{0.3333}}
  >{\raggedright\arraybackslash}p{(\linewidth - 4\tabcolsep) * \real{0.3611}}@{}}
\toprule\noalign{}
\begin{minipage}[b]{\linewidth}\raggedright
\textbf{Phase}
\end{minipage} & \begin{minipage}[b]{\linewidth}\raggedright
\textbf{Action}
\end{minipage} & \begin{minipage}[b]{\linewidth}\raggedright
\textbf{Purpose}
\end{minipage} \\
\midrule\noalign{}
\endhead
\bottomrule\noalign{}
\endlastfoot
\textbf{Pre-prepare} & Leader broadcasts proposal & Initiate consensus
round \\
\textbf{Prepare} & Nodes validate and broadcast agreement & Ensure
proposal is seen by all \\
\textbf{Commit} & Nodes commit to decision & Finalize consensus \\
\end{longtable}
}

\textbf{pBFT Process Flow:}

\begin{verbatim}
sequenceDiagram
    participant C as Client
    participant P as Primary Node
    participant B1 as Backup Node 1
    participant B2 as Backup Node 2
    
    C{-P: Request}
    P{-B1: Pre{-}prepare}
    P{-B2: Pre{-}prepare}
    B1{-B2: Prepare}
    B2{-B1: Prepare}
    B1{-B2: Commit}
    B2{-B1: Commit}
    P{-C: Reply}
\end{verbatim}

\textbf{Fault Tolerance:}

{\def\LTcaptype{none} % do not increment counter
\begin{longtable}[]{@{}ll@{}}
\toprule\noalign{}
\textbf{Aspect} & \textbf{Capability} \\
\midrule\noalign{}
\endhead
\bottomrule\noalign{}
\endlastfoot
\textbf{Maximum Faulty Nodes} & Can tolerate up to 1/3 faulty nodes \\
\textbf{Network Requirement} & Synchronous or partially synchronous \\
\textbf{Message Complexity} & O(n^{2}) messages per consensus \\
\textbf{Finality} & Immediate finality achieved \\
\end{longtable}
}

\textbf{Applications:}

{\def\LTcaptype{none} % do not increment counter
\begin{longtable}[]{@{}ll@{}}
\toprule\noalign{}
\textbf{System} & \textbf{Usage} \\
\midrule\noalign{}
\endhead
\bottomrule\noalign{}
\endlastfoot
\textbf{Hyperledger Fabric} & Consensus mechanism \\
\textbf{Tendermint} & Byzantine fault tolerant consensus \\
\textbf{Zilliqa} & Practical Byzantine fault tolerance \\
\end{longtable}
}

\begin{itemize}
\tightlist
\item
  \textbf{Advantage}: Fast finality, good for permissioned networks
\item
  \textbf{Limitation}: High communication overhead, doesn't scale well
\end{itemize}

\end{solutionbox}
\begin{mnemonicbox}
``Byzantine = Bad actors, pBFT = Practical Fix''

\end{mnemonicbox}
\begin{center}\rule{0.5\linewidth}{0.5pt}\end{center}

\subsection*{Question 5(a) [3 marks]}\label{q5a}

\textbf{List and explain cryptocurrency wallets in blockchain.}

\begin{solutionbox}

\textbf{Cryptocurrency Wallet Types:}

{\def\LTcaptype{none} % do not increment counter
\begin{longtable}[]{@{}lll@{}}
\toprule\noalign{}
\textbf{Wallet Type} & \textbf{Description} & \textbf{Security Level} \\
\midrule\noalign{}
\endhead
\bottomrule\noalign{}
\endlastfoot
\textbf{Hardware Wallet} & Physical device storing keys & Very High \\
\textbf{Software Wallet} & Application on computer/phone & Medium to
High \\
\textbf{Paper Wallet} & Keys printed on paper & High (if stored
safely) \\
\textbf{Web Wallet} & Online wallet service & Medium \\
\textbf{Brain Wallet} & Keys memorized & Variable \\
\end{longtable}
}

\textbf{Storage Methods:}

{\def\LTcaptype{none} % do not increment counter
\begin{longtable}[]{@{}lll@{}}
\toprule\noalign{}
\textbf{Method} & \textbf{Accessibility} & \textbf{Security} \\
\midrule\noalign{}
\endhead
\bottomrule\noalign{}
\endlastfoot
\textbf{Hot Wallet} & Always online & Lower security \\
\textbf{Cold Wallet} & Offline storage & Higher security \\
\end{longtable}
}

\textbf{Wallet Functions:}

\begin{center}
\textbf{Mermaid Diagram (Code)}
\begin{verbatim}
{Shaded}
{Highlighting}[]
graph TD
    A[Cryptocurrency Wallet] {-{-}{} B[Store Private Keys]}
    A {-{-}{} C[Generate Addresses]}
    A {-{-}{} D[Sign Transactions]}
    A {-{-}{} E[Check Balances]}
    A {-{-}{} F[Send/Receive Crypto]}
{Highlighting}
{Shaded}
\end{verbatim}
\end{center}

\begin{itemize}
\tightlist
\item
  \textbf{Key Point}: Wallets don't store coins, they store keys to
  access coins
\item
  \textbf{Backup}: Always backup wallet seed phrase
\end{itemize}

\end{solutionbox}
\begin{mnemonicbox}
``Wallet = Key Keeper, Not Coin Container''

\end{mnemonicbox}
\begin{center}\rule{0.5\linewidth}{0.5pt}\end{center}

\subsection*{Question 5(b) [4 marks]}\label{q5b}

\textbf{Write advantages and disadvantages of ERC-20 token.}

\begin{solutionbox}

\textbf{ERC-20 Token Definition}: Standard protocol for creating tokens
on Ethereum blockchain.

\textbf{Advantages:}

{\def\LTcaptype{none} % do not increment counter
\begin{longtable}[]{@{}ll@{}}
\toprule\noalign{}
\textbf{Advantage} & \textbf{Benefit} \\
\midrule\noalign{}
\endhead
\bottomrule\noalign{}
\endlastfoot
\textbf{Standardization} & All tokens work the same way \\
\textbf{Interoperability} & Compatible with all Ethereum wallets \\
\textbf{Easy Development} & Simple to create new tokens \\
\textbf{Wide Support} & Supported by exchanges and services \\
\textbf{Smart Contract Integration} & Can interact with DeFi
protocols \\
\end{longtable}
}

\textbf{Disadvantages:}

{\def\LTcaptype{none} % do not increment counter
\begin{longtable}[]{@{}ll@{}}
\toprule\noalign{}
\textbf{Disadvantage} & \textbf{Problem} \\
\midrule\noalign{}
\endhead
\bottomrule\noalign{}
\endlastfoot
\textbf{Gas Fees} & Expensive transactions during network congestion \\
\textbf{Scalability} & Limited by Ethereum's transaction throughput \\
\textbf{Security Risks} & Smart contract bugs can cause token loss \\
\textbf{Centralization} & Many tokens have centralized control \\
\textbf{Environmental Impact} & High energy consumption \\
\end{longtable}
}

\textbf{Comparison Table:}

{\def\LTcaptype{none} % do not increment counter
\begin{longtable}[]{@{}lll@{}}
\toprule\noalign{}
\textbf{Aspect} & \textbf{Advantage} & \textbf{Disadvantage} \\
\midrule\noalign{}
\endhead
\bottomrule\noalign{}
\endlastfoot
\textbf{Adoption} & Widely accepted & Market oversaturation \\
\textbf{Development} & Easy to create & Easy to create scam tokens \\
\textbf{Functionality} & Standard features & Limited customization \\
\end{longtable}
}

\begin{itemize}
\tightlist
\item
  \textbf{Usage}: Most popular standard for creating cryptocurrency
  tokens
\item
  \textbf{Examples}: USDT, LINK, UNI are ERC-20 tokens
\end{itemize}

\end{solutionbox}
\begin{mnemonicbox}
``ERC-20 = Easy and Expensive''

\end{mnemonicbox}
\begin{center}\rule{0.5\linewidth}{0.5pt}\end{center}

\subsection*{Question 5(c) [7 marks]}\label{q5c}

\textbf{What are dApps used for? Explain advantages and disadvantages of
dApps.}

\begin{solutionbox}

\textbf{dApps Definition}: Decentralized Applications that run on
blockchain networks without central authority.

\textbf{dApps Usage Categories:}

{\def\LTcaptype{none} % do not increment counter
\begin{longtable}[]{@{}lll@{}}
\toprule\noalign{}
\textbf{Category} & \textbf{Examples} & \textbf{Purpose} \\
\midrule\noalign{}
\endhead
\bottomrule\noalign{}
\endlastfoot
\textbf{DeFi} & Uniswap, Compound & Financial services \\
\textbf{Gaming} & CryptoKitties, Axie Infinity & Blockchain games \\
\textbf{Social Media} & Steemit, Minds & Censorship-resistant
platforms \\
\textbf{Marketplaces} & OpenSea, Rarible & NFT trading \\
\textbf{Governance} & Aragon, DAOstack & Decentralized organizations \\
\textbf{Storage} & Filecoin, Storj & Distributed file storage \\
\end{longtable}
}

\textbf{dApp Architecture:}

\begin{center}
\textbf{Mermaid Diagram (Code)}
\begin{verbatim}
{Shaded}
{Highlighting}[]
graph LR
    A[Frontend {- User Interface] {-}{-}{} B[Web3 Connection]}
    B {-{-}{} C[Smart Contracts]}
    C {-{-}{} D[Blockchain Network]}
    D {-{-}{} E[Distributed Storage]}
    
    F[Traditional App] {-{-}{} G[Central Server]}
    G {-{-}{} H[Central Database]}
{Highlighting}
{Shaded}
\end{verbatim}
\end{center}

\textbf{Advantages:}

{\def\LTcaptype{none} % do not increment counter
\begin{longtable}[]{@{}ll@{}}
\toprule\noalign{}
\textbf{Advantage} & \textbf{Description} \\
\midrule\noalign{}
\endhead
\bottomrule\noalign{}
\endlastfoot
\textbf{Censorship Resistance} & No single point of control \\
\textbf{Transparency} & Code and data publicly verifiable \\
\textbf{Global Access} & Available worldwide without restrictions \\
\textbf{No Downtime} & Distributed across many nodes \\
\textbf{User Ownership} & Users control their data and assets \\
\textbf{Trustless} & No need to trust intermediaries \\
\end{longtable}
}

\textbf{Disadvantages:}

{\def\LTcaptype{none} % do not increment counter
\begin{longtable}[]{@{}ll@{}}
\toprule\noalign{}
\textbf{Disadvantage} & \textbf{Description} \\
\midrule\noalign{}
\endhead
\bottomrule\noalign{}
\endlastfoot
\textbf{Poor User Experience} & Complex interfaces, slow transactions \\
\textbf{Scalability Issues} & Limited transaction throughput \\
\textbf{High Costs} & Gas fees for every interaction \\
\textbf{Technical Complexity} & Difficult for non-technical users \\
\textbf{Regulatory Uncertainty} & Unclear legal status \\
\textbf{Energy Consumption} & High environmental impact \\
\textbf{Immutable Bugs} & Cannot easily fix smart contract errors \\
\end{longtable}
}

\textbf{Development Challenges:}

{\def\LTcaptype{none} % do not increment counter
\begin{longtable}[]{@{}ll@{}}
\toprule\noalign{}
\textbf{Challenge} & \textbf{Impact} \\
\midrule\noalign{}
\endhead
\bottomrule\noalign{}
\endlastfoot
\textbf{Gas Optimization} & Must minimize transaction costs \\
\textbf{Security Auditing} & Critical to prevent hacks \\
\textbf{User Onboarding} & Difficult to attract mainstream users \\
\textbf{Scalability Solutions} & Need Layer 2 or alternative chains \\
\end{longtable}
}

\textbf{Popular dApp Platforms:}

{\def\LTcaptype{none} % do not increment counter
\begin{longtable}[]{@{}ll@{}}
\toprule\noalign{}
\textbf{Platform} & \textbf{Characteristics} \\
\midrule\noalign{}
\endhead
\bottomrule\noalign{}
\endlastfoot
\textbf{Ethereum} & Most established, highest fees \\
\textbf{Binance Smart Chain} & Lower fees, more centralized \\
\textbf{Polygon} & Ethereum Layer 2, faster and cheaper \\
\textbf{Solana} & High throughput, newer ecosystem \\
\end{longtable}
}

\begin{itemize}
\tightlist
\item
  \textbf{Future}: Moving toward better user experience and lower costs
\item
  \textbf{Adoption}: Still early stage but growing rapidly
\end{itemize}

\end{solutionbox}
\begin{mnemonicbox}
``dApps = Decentralized but Difficult''

\end{mnemonicbox}
\begin{center}\rule{0.5\linewidth}{0.5pt}\end{center}

\subsection*{Question 5(a) OR [3
marks]}\label{q5a}

\textbf{Explain tokenized and token less Blockchain in detail.}

\begin{solutionbox}

\textbf{Tokenized Blockchain:}

{\def\LTcaptype{none} % do not increment counter
\begin{longtable}[]{@{}ll@{}}
\toprule\noalign{}
\textbf{Feature} & \textbf{Description} \\
\midrule\noalign{}
\endhead
\bottomrule\noalign{}
\endlastfoot
\textbf{Definition} & Blockchain with native cryptocurrency token \\
\textbf{Token Purpose} & Incentivize network participation \\
\textbf{Examples} & Bitcoin (BTC), Ethereum (ETH) \\
\textbf{Function} & Pay transaction fees, reward miners/validators \\
\end{longtable}
}

\textbf{Token-less Blockchain:}

{\def\LTcaptype{none} % do not increment counter
\begin{longtable}[]{@{}ll@{}}
\toprule\noalign{}
\textbf{Feature} & \textbf{Description} \\
\midrule\noalign{}
\endhead
\bottomrule\noalign{}
\endlastfoot
\textbf{Definition} & Blockchain without native cryptocurrency \\
\textbf{Access} & Permission-based participation \\
\textbf{Examples} & Hyperledger Fabric, R3 Corda \\
\textbf{Function} & Record keeping, process automation \\
\end{longtable}
}

\textbf{Comparison Table:}

{\def\LTcaptype{none} % do not increment counter
\begin{longtable}[]{@{}lll@{}}
\toprule\noalign{}
\textbf{Aspect} & \textbf{Tokenized} & \textbf{Token-less} \\
\midrule\noalign{}
\endhead
\bottomrule\noalign{}
\endlastfoot
\textbf{Incentive Model} & Economic rewards & Permission-based \\
\textbf{Access} & Open to anyone with tokens & Restricted access \\
\textbf{Governance} & Token holder voting & Centralized control \\
\textbf{Use Case} & Public networks & Private/enterprise \\
\textbf{Security} & Economic game theory & Traditional security \\
\end{longtable}
}

\textbf{Architecture Differences:}

\begin{center}
\textbf{Mermaid Diagram (Code)}
\begin{verbatim}
{Shaded}
{Highlighting}[]
graph TD
    subgraph "Tokenized Blockchain"
        A[Token Rewards] {-{-}{} B[Miners/Validators]}
        B {-{-}{} C[Secure Network]}
        C {-{-}{} D[Public Access]}
    end
    
    subgraph "Token{-less Blockchain"}
        E[Permission System] {-{-}{} F[Authorized Nodes]}
        F {-{-}{} G[Secure Network]}
        G {-{-}{} H[Private Access]}
    end
{Highlighting}
{Shaded}
\end{verbatim}
\end{center}

\begin{itemize}
\tightlist
\item
  \textbf{Choice}: Depends on whether you need public participation or
  private control
\item
  \textbf{Trend}: Most public blockchains are tokenized, most private
  ones are token-less
\end{itemize}

\end{solutionbox}
\begin{mnemonicbox}
``Token = Public Participation, Token-less = Private
Permission''

\end{mnemonicbox}
\begin{center}\rule{0.5\linewidth}{0.5pt}\end{center}

\subsection*{Question 5(b) OR [4
marks]}\label{q5b}

\textbf{Write advantages and disadvantages of Hyperledger.}

\begin{solutionbox}

\textbf{Hyperledger Definition}: Open-source collaborative framework for
developing enterprise-grade blockchain solutions.

\textbf{Advantages:}

{\def\LTcaptype{none} % do not increment counter
\begin{longtable}[]{@{}ll@{}}
\toprule\noalign{}
\textbf{Advantage} & \textbf{Description} \\
\midrule\noalign{}
\endhead
\bottomrule\noalign{}
\endlastfoot
\textbf{Enterprise Focus} & Designed for business use cases \\
\textbf{Modular Architecture} & Customize components as needed \\
\textbf{Privacy} & Confidential transactions possible \\
\textbf{Performance} & Higher transaction throughput \\
\textbf{Governance} & Professional development standards \\
\textbf{No Cryptocurrency} & Avoids regulatory crypto issues \\
\textbf{Permissioned Network} & Control who can participate \\
\end{longtable}
}

\textbf{Disadvantages:}

{\def\LTcaptype{none} % do not increment counter
\begin{longtable}[]{@{}ll@{}}
\toprule\noalign{}
\textbf{Disadvantage} & \textbf{Description} \\
\midrule\noalign{}
\endhead
\bottomrule\noalign{}
\endlastfoot
\textbf{Centralization} & Less decentralized than public blockchains \\
\textbf{Complexity} & Requires technical expertise to implement \\
\textbf{Limited Adoption} & Smaller ecosystem compared to Ethereum \\
\textbf{Vendor Lock-in} & May depend on specific technology providers \\
\textbf{Scalability} & Still faces some scaling challenges \\
\textbf{No Token Economy} & Cannot leverage crypto incentives \\
\end{longtable}
}

\textbf{Hyperledger Projects Comparison:}

{\def\LTcaptype{none} % do not increment counter
\begin{longtable}[]{@{}lll@{}}
\toprule\noalign{}
\textbf{Project} & \textbf{Strengths} & \textbf{Limitations} \\
\midrule\noalign{}
\endhead
\bottomrule\noalign{}
\endlastfoot
\textbf{Fabric} & Mature, flexible & Complex setup \\
\textbf{Sawtooth} & Scalable & Less documentation \\
\textbf{Iroha} & Simple, mobile-friendly & Limited features \\
\end{longtable}
}

\textbf{Use Case Suitability:}

{\def\LTcaptype{none} % do not increment counter
\begin{longtable}[]{@{}ll@{}}
\toprule\noalign{}
\textbf{Good For} & \textbf{Not Ideal For} \\
\midrule\noalign{}
\endhead
\bottomrule\noalign{}
\endlastfoot
\textbf{Supply chain tracking} & Public cryptocurrencies \\
\textbf{Healthcare records} & Fully decentralized systems \\
\textbf{Banking consortiums} & High-frequency trading \\
\textbf{Government systems} & Anonymous transactions \\
\end{longtable}
}

\begin{itemize}
\tightlist
\item
  \textbf{Target}: Large enterprises and consortiums
\item
  \textbf{Support}: Backed by Linux Foundation
\end{itemize}

\end{solutionbox}
\begin{mnemonicbox}
``Hyperledger = High Performance, Low Publicity''

\end{mnemonicbox}
\begin{center}\rule{0.5\linewidth}{0.5pt}\end{center}

\subsection*{Question 5(c) OR [7
marks]}\label{q5c}

\textbf{Explain Smart contract. Write various applications of smart
contract.}

\begin{solutionbox}

\textbf{Smart Contract Definition}: Self-executing contracts with terms
directly written into code, automatically enforced on blockchain.

\textbf{Key Characteristics:}

{\def\LTcaptype{none} % do not increment counter
\begin{longtable}[]{@{}ll@{}}
\toprule\noalign{}
\textbf{Feature} & \textbf{Description} \\
\midrule\noalign{}
\endhead
\bottomrule\noalign{}
\endlastfoot
\textbf{Automated} & Executes automatically when conditions met \\
\textbf{Immutable} & Cannot be changed after deployment \\
\textbf{Transparent} & Code is publicly visible \\
\textbf{Trustless} & No intermediaries needed \\
\textbf{Deterministic} & Same input always produces same output \\
\end{longtable}
}

\textbf{Smart Contract Workflow:}

\begin{center}
\textbf{Mermaid Diagram (Code)}
\begin{verbatim}
{Shaded}
{Highlighting}[]
graph LR
    A[Contract Created] {-{-}{} B[Deployed to Blockchain]}
    B {-{-}{} C[Conditions Monitored]}
    C {-{-}{} D\{Conditions Met?\}}
    D {-{-}{}|Yes| E[Contract Executes]}
    D {-{-}{}|No| F[Continue Monitoring]}
    E {-{-}{} G[Automatic Settlement]}
    F {-{-}{} C}
{Highlighting}
{Shaded}
\end{verbatim}
\end{center}

\textbf{Applications by Industry:}

{\def\LTcaptype{none} % do not increment counter
\begin{longtable}[]{@{}
  >{\raggedright\arraybackslash}p{(\linewidth - 4\tabcolsep) * \real{0.3182}}
  >{\raggedright\arraybackslash}p{(\linewidth - 4\tabcolsep) * \real{0.3864}}
  >{\raggedright\arraybackslash}p{(\linewidth - 4\tabcolsep) * \real{0.2955}}@{}}
\toprule\noalign{}
\begin{minipage}[b]{\linewidth}\raggedright
\textbf{Industry}
\end{minipage} & \begin{minipage}[b]{\linewidth}\raggedright
\textbf{Application}
\end{minipage} & \begin{minipage}[b]{\linewidth}\raggedright
\textbf{Benefit}
\end{minipage} \\
\midrule\noalign{}
\endhead
\bottomrule\noalign{}
\endlastfoot
\textbf{Finance} & Automated loans, insurance claims & Faster
processing, lower costs \\
\textbf{Real Estate} & Property transfers, rental agreements & Reduced
fraud, instant settlements \\
\textbf{Supply Chain} & Product tracking, quality assurance &
Transparency, automated compliance \\
\textbf{Healthcare} & Patient consent, insurance claims & Privacy
protection, automated payouts \\
\textbf{Entertainment} & Royalty distribution, content licensing & Fair
payment, transparent accounting \\
\textbf{Gaming} & In-game assets, tournaments & True ownership,
automated prizes \\
\end{longtable}
}

\textbf{Specific Smart Contract Examples:}

{\def\LTcaptype{none} % do not increment counter
\begin{longtable}[]{@{}lll@{}}
\toprule\noalign{}
\textbf{Application} & \textbf{Function} & \textbf{Platform} \\
\midrule\noalign{}
\endhead
\bottomrule\noalign{}
\endlastfoot
\textbf{Uniswap} & Automated token trading & Ethereum \\
\textbf{Compound} & Lending and borrowing & Ethereum \\
\textbf{CryptoKitties} & Digital pet ownership & Ethereum \\
\textbf{Chainlink} & Oracle data feeds & Multiple platforms \\
\textbf{Aave} & Flash loans & Ethereum \\
\end{longtable}
}

\textbf{Development Platforms:}

{\def\LTcaptype{none} % do not increment counter
\begin{longtable}[]{@{}lll@{}}
\toprule\noalign{}
\textbf{Platform} & \textbf{Language} & \textbf{Features} \\
\midrule\noalign{}
\endhead
\bottomrule\noalign{}
\endlastfoot
\textbf{Ethereum} & Solidity & Most mature ecosystem \\
\textbf{Binance Smart Chain} & Solidity & Lower fees, faster \\
\textbf{Cardano} & Plutus & Academic approach \\
\textbf{Solana} & Rust & High performance \\
\end{longtable}
}

\textbf{Benefits:}

{\def\LTcaptype{none} % do not increment counter
\begin{longtable}[]{@{}lll@{}}
\toprule\noalign{}
\textbf{Benefit} & \textbf{Traditional Contract} & \textbf{Smart
Contract} \\
\midrule\noalign{}
\endhead
\bottomrule\noalign{}
\endlastfoot
\textbf{Speed} & Days to weeks & Minutes to hours \\
\textbf{Cost} & High legal fees & Low gas fees \\
\textbf{Trust} & Requires intermediaries & Trustless execution \\
\textbf{Accuracy} & Human error possible & Coded precision \\
\end{longtable}
}

\textbf{Limitations:}

{\def\LTcaptype{none} % do not increment counter
\begin{longtable}[]{@{}ll@{}}
\toprule\noalign{}
\textbf{Limitation} & \textbf{Description} \\
\midrule\noalign{}
\endhead
\bottomrule\noalign{}
\endlastfoot
\textbf{Code Bugs} & Errors can cause financial loss \\
\textbf{Oracle Problem} & Difficulty getting real-world data \\
\textbf{Immutability} & Hard to fix after deployment \\
\textbf{Gas Costs} & Can be expensive on congested networks \\
\textbf{Legal Status} & Unclear regulatory framework \\
\end{longtable}
}

\textbf{Real-world Impact:}

{\def\LTcaptype{none} % do not increment counter
\begin{longtable}[]{@{}ll@{}}
\toprule\noalign{}
\textbf{Sector} & \textbf{Transformation} \\
\midrule\noalign{}
\endhead
\bottomrule\noalign{}
\endlastfoot
\textbf{DeFi} & \$100+ billion locked in smart contracts \\
\textbf{NFTs} & New digital ownership models \\
\textbf{DAOs} & Decentralized governance systems \\
\textbf{Insurance} & Parametric insurance products \\
\end{longtable}
}

\begin{itemize}
\tightlist
\item
  \textbf{Future}: Integration with IoT, AI, and traditional business
  systems
\item
  \textbf{Evolution}: Moving toward more user-friendly development tools
\end{itemize}

\end{solutionbox}
\begin{mnemonicbox}
``Smart Contract = Self-executing, Solves Problems''

\end{mnemonicbox}

\end{document}
