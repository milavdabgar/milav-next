\documentclass[10pt,a4paper]{article}

% content/resources/templates/preamble.tex
\usepackage[margin=0.6in]{geometry}
\author{Milav Dabgar}
\usepackage{amsmath,amssymb,amsthm}
\usepackage{booktabs}
\usepackage{multirow}
\usepackage{xcolor}
\usepackage{tcolorbox}
\tcbuselibrary{breakable,skins}
\usepackage[colorlinks=true,linkcolor=blue]{hyperref}
\usepackage{titlesec}
\usepackage{enumitem}
\usepackage{tikz}
\usepackage{pgfplots}
\usepackage{circuitikz}
\usepackage[version=4]{mhchem}
\usepackage{longtable}
\usepackage{array}
\usepackage{float}
\usepackage{caption}
\usepackage{listings}

\lstset{
  basicstyle=\small\ttfamily,
  breaklines=true,
  breakatwhitespace=false,
  postbreak=\mbox{\textcolor{red}{$\hookrightarrow$}\space},
  float=false,
  numbers=left,
  numberstyle=\tiny\color{gray},
  numbersep=10pt,
  xleftmargin=2em,
  keywordstyle=\color{blue},
  commentstyle=\color{green!60!black},
  stringstyle=\color{purple},
  backgroundcolor=\color{gray!5},
  showstringspaces=false,
  tabsize=2,
  captionpos=b,
  keepspaces=true,
  columns=flexible
}

\pgfplotsset{compat=1.18}
\usetikzlibrary{shapes,arrows,positioning,calc,patterns,decorations.pathmorphing,decorations.markings,arrows.meta}

% Color scheme
\definecolor{headcolor}{RGB}{0,102,204}
\definecolor{keycolor}{RGB}{220,20,60}
\definecolor{solutioncolor}{RGB}{34,139,34}
\definecolor{mnemoniccolor}{RGB}{148,0,211}
\definecolor{codecolor}{RGB}{0,0,100}

% Spacing
\setlength{\parskip}{3pt}
\setlist[itemize]{nosep}
\setlist[enumerate]{nosep}

% Title formatting
\titleformat{\section}{\Large\bfseries\color{headcolor}}{\thesection}{1em}{}
\titleformat{\subsection}{\large\bfseries\color{headcolor}}{\thesubsection}{1em}{}

% Pandoc tightlist compatibility
\providecommand{\tightlist}{%
  \setlength{\itemsep}{0pt}\setlength{\parskip}{0pt}}

% Pandoc longtable compatibility
\newcounter{none}
\def\thenone{}


% content/resources/templates/english-boxes.tex
% This file is currently empty - it exists to maintain consistency with the import structure.
% Add custom environments here if needed in the future.


\begin{document}

\begin{center}
{\Huge\bfseries\color{headcolor} Subject Name Solutions}\\[5pt]
{\LARGE 1333202 -- Winter 2023}\\[3pt]
{\large Semester 1 Study Material}\\[3pt]
{\normalsize\textit{Detailed Solutions and Explanations}}
\end{center}

\vspace{10pt}

\subsection*{Question 1(a) [3 marks]}\label{q1a}

\textbf{Define Microprocessor.}

\begin{solutionbox}

A microprocessor is a single-chip CPU that contains all the arithmetic,
logic, and control circuitry required to perform the functions of a
digital computer's central processing unit.


{\def\LTcaptype{none} % do not increment counter
\vspace{-5pt}
\captionof{table}{Microprocessor Key Features}
\vspace{-10pt}
\begin{longtable}[]{@{}
  >{\raggedright\arraybackslash}p{(\linewidth - 2\tabcolsep) * \real{0.4091}}
  >{\raggedright\arraybackslash}p{(\linewidth - 2\tabcolsep) * \real{0.5909}}@{}}
\toprule\noalign{}
\begin{minipage}[b]{\linewidth}\raggedright
Feature
\end{minipage} & \begin{minipage}[b]{\linewidth}\raggedright
Description
\end{minipage} \\
\midrule\noalign{}
\endhead
\bottomrule\noalign{}
\endlastfoot
\textbf{Single Chip} & Complete CPU on one integrated circuit \\
\textbf{Processing Unit} & Executes instructions and performs
calculations \\
\textbf{Control Logic} & Manages system operations and data flow \\
\end{longtable}
}

\begin{itemize}
\tightlist
\item
  \textbf{Central Processing Unit}: Core component that executes
  instructions
\item
  \textbf{Integrated Circuit}: All functions combined on single silicon
  chip
\item
  \textbf{Programmable Device}: Can execute different programs based on
  stored instructions
\end{itemize}

\end{solutionbox}
\begin{mnemonicbox}
``Single Chip CPU = Smart Computer Processor Unit''

\end{mnemonicbox}
\subsection*{Question 1(b) [4 marks]}\label{q1b}

\textbf{Explain Flag register of microprocessor.}

\begin{solutionbox}

The Flag register stores status information about the result of
arithmetic and logical operations performed by the ALU.


{\def\LTcaptype{none} % do not increment counter
\vspace{-5pt}
\captionof{table}{8085 Flag Register Bits}
\vspace{-10pt}
\begin{longtable}[]{@{}
  >{\raggedright\arraybackslash}p{(\linewidth - 4\tabcolsep) * \real{0.2400}}
  >{\raggedright\arraybackslash}p{(\linewidth - 4\tabcolsep) * \real{0.4000}}
  >{\raggedright\arraybackslash}p{(\linewidth - 4\tabcolsep) * \real{0.3600}}@{}}
\toprule\noalign{}
\begin{minipage}[b]{\linewidth}\raggedright
Flag
\end{minipage} & \begin{minipage}[b]{\linewidth}\raggedright
Position
\end{minipage} & \begin{minipage}[b]{\linewidth}\raggedright
Purpose
\end{minipage} \\
\midrule\noalign{}
\endhead
\bottomrule\noalign{}
\endlastfoot
\textbf{S (Sign)} & Bit 7 & Indicates sign of result (1=negative,
0=positive) \\
\textbf{Z (Zero)} & Bit 6 & Set when result is zero \\
\textbf{AC (Auxiliary Carry)} & Bit 4 & Carry from bit 3 to bit 4 \\
\textbf{P (Parity)} & Bit 2 & Even parity flag \\
\textbf{CY (Carry)} & Bit 0 & Carry from MSB \\
\end{longtable}
}

\begin{itemize}
\tightlist
\item
  \textbf{Status Indicator}: Shows condition of last operation result\\
\item
  \textbf{Conditional Instructions}: Used for branching and decision
  making
\item
  \textbf{5 Active Flags}: Sign, Zero, Auxiliary Carry, Parity, and
  Carry flags
\end{itemize}

\end{solutionbox}
\begin{mnemonicbox}
``Flags Show Zero, Sign, Parity, Auxiliary, Carry''

\end{mnemonicbox}
\subsection*{Question 1(c) [7 marks]}\label{q1c}

\textbf{Explain format of instruction of microprocessor with example.}

\begin{solutionbox}

Microprocessor instructions consist of opcode and operand fields that
specify the operation and data locations.


{\def\LTcaptype{none} % do not increment counter
\vspace{-5pt}
\captionof{table}{8085 Instruction Format Types}
\vspace{-10pt}
\begin{longtable}[]{@{}llll@{}}
\toprule\noalign{}
Format & Size & Structure & Example \\
\midrule\noalign{}
\endhead
\bottomrule\noalign{}
\endlastfoot
\textbf{1-Byte} & 8 bits & Opcode only & MOV A,B \\
\textbf{2-Byte} & 16 bits & Opcode + 8-bit data & MVI A,05H \\
\textbf{3-Byte} & 24 bits & Opcode + 16-bit address & LDA 2000H \\
\end{longtable}
}

\textbf{Diagram:}

\includegraphics[width=1\linewidth,height=\textheight,keepaspectratio]{mermaid-801bbd8f.pdf}

\begin{itemize}
\tightlist
\item
  \textbf{Opcode Field}: Defines the operation to be performed (ADD,
  MOV, JMP)
\item
  \textbf{Operand Field}: Contains data, register, or memory address
  information
\item
  \textbf{Variable Length}: Instructions can be 1, 2, or 3 bytes long
\item
  \textbf{Addressing Modes}: Different ways to specify operand location
\end{itemize}

\end{solutionbox}
\begin{mnemonicbox}
``Opcode Operations + Operand Objects = Complete
Commands''

\end{mnemonicbox}
\subsection*{Question 1(c OR) [7
marks]}\label{question-1c-or-7-marks}

\textbf{Explain function of ALU, Control Unit and CPU of
Microprocessor.}

\begin{solutionbox}

The CPU consists of three main functional units that work together to
execute instructions.


{\def\LTcaptype{none} % do not increment counter
\vspace{-5pt}
\captionof{table}{CPU Components and Functions}
\vspace{-10pt}
\begin{longtable}[]{@{}lll@{}}
\toprule\noalign{}
Component & Primary Function & Key Operations \\
\midrule\noalign{}
\endhead
\bottomrule\noalign{}
\endlastfoot
\textbf{ALU} & Arithmetic \& Logic Operations & ADD, SUB, AND, OR,
XOR \\
\textbf{Control Unit} & Instruction Control & Fetch, Decode, Execute \\
\textbf{CPU} & Overall Processing & Coordinate all operations \\
\end{longtable}
}

\textbf{Diagram:}

\includegraphics[width=1\linewidth,height=\textheight,keepaspectratio]{mermaid-2715f80d.pdf}

\begin{itemize}
\tightlist
\item
  \textbf{ALU Functions}: Performs all arithmetic calculations and
  logical operations
\item
  \textbf{Control Unit Tasks}: Manages instruction execution cycle and
  generates control signals
\item
  \textbf{CPU Coordination}: Integrates ALU and Control Unit for
  complete processing
\end{itemize}

\end{solutionbox}
\begin{mnemonicbox}
``ALU Adds, Control Commands, CPU Coordinates''

\end{mnemonicbox}
\subsection*{Question 2(a) [3 marks]}\label{q2a}

\textbf{Explain function of ALE signal with diagram.}

\begin{solutionbox}

ALE (Address Latch Enable) signal is used to demultiplex the lower-order
address and data lines.


{\def\LTcaptype{none} % do not increment counter
\vspace{-5pt}
\captionof{table}{ALE Signal Functions}
\vspace{-10pt}
\begin{longtable}[]{@{}ll@{}}
\toprule\noalign{}
Function & Description \\
\midrule\noalign{}
\endhead
\bottomrule\noalign{}
\endlastfoot
\textbf{Address Latching} & Captures lower 8-bit address \\
\textbf{Demultiplexing} & Separates address from data \\
\textbf{Timing Control} & Provides timing reference \\
\end{longtable}
}

\textbf{Diagram:}

\begin{lstlisting}
    +--------+     ALE     +--------+
    |  8085  |------------>| Latch  |
    |        |             | 74373  |
    | AD0-7  |<----------->|        |
    +--------+             +--------+
                               |
                           A0-A7 (Address)
\end{lstlisting}

\begin{itemize}
\tightlist
\item
  \textbf{Active High Signal}: ALE goes high during T1 state
\item
  \textbf{External Latching}: Used with 74373 latch to hold address
\item
  \textbf{System Timing}: Provides reference for external devices
\end{itemize}

\end{solutionbox}
\begin{mnemonicbox}
``ALE Always Latches External Addresses''

\end{mnemonicbox}
\subsection*{Question 2(b) [4 marks]}\label{q2b}

\textbf{Compare microprocessor and microcontroller}

\begin{solutionbox}


{\def\LTcaptype{none} % do not increment counter
\vspace{-5pt}
\captionof{table}{Microprocessor vs Microcontroller Comparison}
\vspace{-10pt}
\begin{longtable}[]{@{}lll@{}}
\toprule\noalign{}
Parameter & Microprocessor & Microcontroller \\
\midrule\noalign{}
\endhead
\bottomrule\noalign{}
\endlastfoot
\textbf{Design} & General purpose & Application specific \\
\textbf{Memory} & External RAM/ROM & Internal RAM/ROM \\
\textbf{I/O Ports} & External interface & Built-in I/O ports \\
\textbf{Timers} & External & Built-in timers \\
\textbf{Cost} & Higher system cost & Lower system cost \\
\textbf{Power} & Higher consumption & Lower consumption \\
\end{longtable}
}

\begin{itemize}
\tightlist
\item
  \textbf{Integration Level}: Microcontroller has more integrated
  components
\item
  \textbf{Application Focus}: Microprocessor for computing,
  microcontroller for control
\item
  \textbf{System Complexity}: Microprocessor needs more external
  components
\item
  \textbf{Design Flexibility}: Microprocessor offers more expandability
\end{itemize}

\end{solutionbox}
\begin{mnemonicbox}
``Microprocessor = More Power, Microcontroller = More
Control''

\end{mnemonicbox}
\subsection*{Question 2(c) [7 marks]}\label{q2c}

\textbf{Draw \& explain block diagram of microprocessor.}

\begin{solutionbox}

The 8085 microprocessor consists of several functional blocks that work
together.

\textbf{Diagram:}

\includegraphics[width=1\linewidth,height=\textheight,keepaspectratio]{mermaid-64321ba8.pdf}


{\def\LTcaptype{none} % do not increment counter
\vspace{-5pt}
\captionof{table}{Block Functions}
\vspace{-10pt}
\begin{longtable}[]{@{}ll@{}}
\toprule\noalign{}
Block & Function \\
\midrule\noalign{}
\endhead
\bottomrule\noalign{}
\endlastfoot
\textbf{ALU} & Arithmetic and logical operations \\
\textbf{Register Array} & Temporary data storage (B,C,D,E,H,L) \\
\textbf{Control Unit} & Instruction execution control \\
\textbf{Address Buffer} & Drive address bus lines \\
\end{longtable}
}

\begin{itemize}
\tightlist
\item
  \textbf{Data Path}: Information flows between registers through
  internal bus
\item
  \textbf{Control Signals}: Generated by timing and control unit\\
\item
  \textbf{Bus Interface}: Connects to external memory and I/O devices
\item
  \textbf{Register Operations}: Temporary storage for operands and
  results
\end{itemize}

\end{solutionbox}
\begin{mnemonicbox}
``Blocks Build Better Processing Systems''

\end{mnemonicbox}
\subsection*{Question 2(a OR) [3
marks]}\label{question-2a-or-3-marks}

\textbf{Explain 16 bits registers of microprocessor.}

\begin{solutionbox}

The 8085 has three 16-bit registers formed by combining 8-bit register
pairs.


{\def\LTcaptype{none} % do not increment counter
\vspace{-5pt}
\captionof{table}{16-bit Registers}
\vspace{-10pt}
\begin{longtable}[]{@{}lll@{}}
\toprule\noalign{}
Register & Formation & Purpose \\
\midrule\noalign{}
\endhead
\bottomrule\noalign{}
\endlastfoot
\textbf{PC} & Single 16-bit & Program Counter - next instruction
address \\
\textbf{SP} & Single 16-bit & Stack Pointer - top of stack address \\
\textbf{HL} & H + L registers & Memory pointer - data address \\
\end{longtable}
}

\begin{itemize}
\tightlist
\item
  \textbf{Program Counter}: Automatically increments to next instruction
\item
  \textbf{Stack Pointer}: Points to last pushed data on stack\\
\item
  \textbf{HL Pair}: Most frequently used for memory addressing
\end{itemize}

\end{solutionbox}
\begin{mnemonicbox}
``PC Points Program, SP Stacks Properly, HL Holds
Location''

\end{mnemonicbox}
\subsection*{Question 2(b OR) [4
marks]}\label{question-2b-or-4-marks}

\textbf{Explain de-multiplexing lower order address and data lines with
diagram of microprocessor.}

\begin{solutionbox}

The 8085 multiplexes lower 8-bit address with data lines to reduce pin
count.


{\def\LTcaptype{none} % do not increment counter
\vspace{-5pt}
\captionof{table}{Multiplexed Lines}
\vspace{-10pt}
\begin{longtable}[]{@{}lll@{}}
\toprule\noalign{}
Lines & T1 State & T2-T4 States \\
\midrule\noalign{}
\endhead
\bottomrule\noalign{}
\endlastfoot
\textbf{AD0-AD7} & Lower Address A0-A7 & Data D0-D7 \\
\textbf{ALE Signal} & High & Low \\
\end{longtable}
}

\textbf{Diagram:}

\begin{lstlisting}
           8085
    +----------------+
    |                | ALE
    |      AD0-AD7   |---->+
    |                |     |
    +----------------+     |
            |              |
            |         +----v----+
            +-------->| 74373   |
                      | Latch   |
                      +---------+
                           |
                       A0-A7
\end{lstlisting}

\begin{itemize}
\tightlist
\item
  \textbf{Time Division}: Same lines carry address then data
\item
  \textbf{External Latch}: 74373 captures address when ALE is high
\item
  \textbf{Signal Separation}: Creates separate address and data buses
\end{itemize}

\end{solutionbox}
\begin{mnemonicbox}
``ALE Always Latches External Address Elegantly''

\end{mnemonicbox}
\subsection*{Question 2(c OR) [7
marks]}\label{question-2c-or-7-marks}

\textbf{Draw and explain pin diagram of 8085.}

\begin{solutionbox}

The 8085 is a 40-pin microprocessor with multiplexed address/data bus.

\textbf{Diagram:}

\begin{lstlisting}
        8085 Pin Diagram
    +-------------------+
X1  |1               40| Vcc
X2  |2               39| HOLD
RST |3               38| HLDA  
SOD |4               37| CLK
SID |5               36| RESET
TRAP|6               35| READY
RST7|7               34| IO/M*
RST6|8               33| S1
RST5|9               32| RD*
INTR|10              31| WR*
INTA|11              30| ALE
AD0 |12              29| S0
AD1 |13              28| A15
AD2 |14              27| A14
AD3 |15              26| A13
AD4 |16              25| A12
AD5 |17              24| A11
AD6 |18              23| A10
AD7 |19              22| A9
Vss |20              21| A8
    +-------------------+
\end{lstlisting}


{\def\LTcaptype{none} % do not increment counter
\vspace{-5pt}
\captionof{table}{Pin Groups}
\vspace{-10pt}
\begin{longtable}[]{@{}
  >{\raggedright\arraybackslash}p{(\linewidth - 4\tabcolsep) * \real{0.3043}}
  >{\raggedright\arraybackslash}p{(\linewidth - 4\tabcolsep) * \real{0.2609}}
  >{\raggedright\arraybackslash}p{(\linewidth - 4\tabcolsep) * \real{0.4348}}@{}}
\toprule\noalign{}
\begin{minipage}[b]{\linewidth}\raggedright
Group
\end{minipage} & \begin{minipage}[b]{\linewidth}\raggedright
Pins
\end{minipage} & \begin{minipage}[b]{\linewidth}\raggedright
Function
\end{minipage} \\
\midrule\noalign{}
\endhead
\bottomrule\noalign{}
\endlastfoot
\textbf{Address/Data} & AD0-AD7, A8-A15 & Memory addressing and data
transfer \\
\textbf{Control} & ALE, RD\emph{, WR}, IO/M* & Bus control signals \\
\textbf{Interrupts} & INTR, RST7-RST5, TRAP & Interrupt handling \\
\textbf{Power} & Vcc, Vss & Power supply connections \\
\end{longtable}
}

\begin{itemize}
\tightlist
\item
  \textbf{Multiplexed Bus}: AD0-AD7 carry both address and data
\item
  \textbf{Active Low Signals}: Signals with * are active low
\item
  \textbf{Crystal Connections}: X1, X2 for clock generation
\end{itemize}

\end{solutionbox}
\begin{mnemonicbox}
``Forty Pins Provide Perfect Processing Power''

\end{mnemonicbox}
\subsection*{Question 3(a) [3 marks]}\label{q3a}

\textbf{Draw clock and reset circuit of microcontroller}

\begin{solutionbox}

The 8051 requires external clock and reset circuits for proper
operation.

\textbf{Diagram:}

\begin{lstlisting}
Clock Circuit:
    +12MHz Crystal
    |
XTAL1 +---||---+ XTAL2
      |       |
     30pF    30pF
      |       |
     GND     GND

Reset Circuit:
     +5V
      |
     10K
      |
RST --+---||---GND
          10µF
\end{lstlisting}


{\def\LTcaptype{none} % do not increment counter
\vspace{-5pt}
\captionof{table}{Circuit Components}
\vspace{-10pt}
\begin{longtable}[]{@{}lll@{}}
\toprule\noalign{}
Component & Value & Purpose \\
\midrule\noalign{}
\endhead
\bottomrule\noalign{}
\endlastfoot
\textbf{Crystal} & 11.0592 MHz & Clock generation \\
\textbf{Capacitors} & 30pF each & Crystal stabilization \\
\textbf{Reset Resistor} & 10KΩ & Pull-up for reset \\
\textbf{Reset Capacitor} & 10µF & Power-on reset delay \\
\end{longtable}
}

\begin{itemize}
\tightlist
\item
  \textbf{Clock Frequency}: Commonly 11.0592 MHz for serial
  communication
\item
  \textbf{Reset Duration}: Must be high for at least 2 machine cycles
\item
  \textbf{Power-on Reset}: Automatic reset when power is applied
\end{itemize}

\end{solutionbox}
\begin{mnemonicbox}
``Crystals Create Clock, Resistors Reset Reliably''

\end{mnemonicbox}
\subsection*{Question 3(b) [4 marks]}\label{q3b}

\textbf{Explain internal RAM of 8051.}

\begin{solutionbox}

The 8051 contains 256 bytes of internal RAM organized in different
sections.


{\def\LTcaptype{none} % do not increment counter
\vspace{-5pt}
\captionof{table}{Internal RAM Organization}
\vspace{-10pt}
\begin{longtable}[]{@{}lll@{}}
\toprule\noalign{}
Address Range & Size & Purpose \\
\midrule\noalign{}
\endhead
\bottomrule\noalign{}
\endlastfoot
\textbf{00H-1FH} & 32 bytes & Register Banks (4 banks \times 8 registers) \\
\textbf{20H-2FH} & 16 bytes & Bit-addressable area \\
\textbf{30H-7FH} & 80 bytes & General purpose RAM \\
\textbf{80H-FFH} & 128 bytes & Special Function Registers (SFRs) \\
\end{longtable}
}

\textbf{Diagram:}

\includegraphics[width=1\linewidth,height=\textheight,keepaspectratio]{mermaid-01904707.pdf}

\begin{itemize}
\tightlist
\item
  \textbf{Register Banks}: Four banks of 8 registers each (R0-R7)
\item
  \textbf{Bit Addressing}: Individual bits can be addressed in 20H-2FH
  area
\item
  \textbf{Stack Area}: Usually located in general purpose RAM area
\item
  \textbf{Direct Access}: All locations accessible through direct
  addressing
\end{itemize}

\end{solutionbox}
\begin{mnemonicbox}
``RAM Registers, Bits, General, Special Functions''

\end{mnemonicbox}
\subsection*{Question 3(c) [7 marks]}\label{q3c}

\textbf{Explain block diagram of 8051.}

\begin{solutionbox}

The 8051 microcontroller integrates CPU, memory, and I/O on a single
chip.

\textbf{Diagram:}

\includegraphics[width=1\linewidth,height=\textheight,keepaspectratio]{mermaid-6ae60545.pdf}


{\def\LTcaptype{none} % do not increment counter
\vspace{-5pt}
\captionof{table}{Major Blocks}
\vspace{-10pt}
\begin{longtable}[]{@{}ll@{}}
\toprule\noalign{}
Block & Function \\
\midrule\noalign{}
\endhead
\bottomrule\noalign{}
\endlastfoot
\textbf{CPU} & Instruction execution and control \\
\textbf{Memory} & 4KB ROM + 256B RAM \\
\textbf{Timers} & Two 16-bit timer/counters \\
\textbf{I/O Ports} & Four 8-bit bidirectional ports \\
\textbf{Serial Port} & Full-duplex UART \\
\textbf{Interrupts} & 5-source interrupt system \\
\end{longtable}
}

\begin{itemize}
\tightlist
\item
  \textbf{Harvard Architecture}: Separate program and data memory spaces
\item
  \textbf{Built-in Peripherals}: Timers, serial port, interrupts
  integrated
\item
  \textbf{Expandable}: External memory and I/O can be added
\item
  \textbf{Control Applications}: Optimized for embedded control tasks
\end{itemize}

\end{solutionbox}
\begin{mnemonicbox}
``Complete Control Chip Contains CPU, Memory, I/O''

\end{mnemonicbox}
\subsection*{Question 3(a OR) [3
marks]}\label{question-3a-or-3-marks}

\textbf{Explain function of DPTR and PC.}

\begin{solutionbox}

DPTR and PC are important 16-bit registers in 8051 for memory
addressing.


{\def\LTcaptype{none} % do not increment counter
\vspace{-5pt}
\captionof{table}{DPTR and PC Functions}
\vspace{-10pt}
\begin{longtable}[]{@{}lll@{}}
\toprule\noalign{}
Register & Full Form & Function \\
\midrule\noalign{}
\endhead
\bottomrule\noalign{}
\endlastfoot
\textbf{DPTR} & Data Pointer & Points to external data memory \\
\textbf{PC} & Program Counter & Points to next instruction address \\
\end{longtable}
}

\begin{itemize}
\tightlist
\item
  \textbf{DPTR Usage}: Accessing external RAM and lookup tables
\item
  \textbf{PC Function}: Automatically increments after instruction fetch
\item
  \textbf{16-bit Addressing}: Both can address 64KB memory space
\end{itemize}

\end{solutionbox}
\begin{mnemonicbox}
``DPTR Data Pointer, PC Program Counter''

\end{mnemonicbox}
\subsection*{Question 3(b OR) [4
marks]}\label{question-3b-or-4-marks}

\textbf{Explain different timer modes of microcontroller.}

\begin{solutionbox}

The 8051 has two timers with four different operating modes.


{\def\LTcaptype{none} % do not increment counter
\vspace{-5pt}
\captionof{table}{Timer Modes}
\vspace{-10pt}
\begin{longtable}[]{@{}lll@{}}
\toprule\noalign{}
Mode & Configuration & Purpose \\
\midrule\noalign{}
\endhead
\bottomrule\noalign{}
\endlastfoot
\textbf{Mode 0} & 13-bit timer & Compatible with 8048 \\
\textbf{Mode 1} & 16-bit timer & Maximum count capability \\
\textbf{Mode 2} & 8-bit auto-reload & Constant time intervals \\
\textbf{Mode 3} & Two 8-bit timers & Timer 0 split operation \\
\end{longtable}
}

\begin{itemize}
\tightlist
\item
  \textbf{Mode Selection}: Controlled by TMOD register bits
\item
  \textbf{Timer 0/1}: Both timers support modes 0, 1, 2
\item
  \textbf{Mode 3 Special}: Only Timer 0 can operate in mode 3
\item
  \textbf{Applications}: Delays, baud rate generation, event counting
\end{itemize}

\end{solutionbox}
\begin{mnemonicbox}
``Modes Make Timers Tremendously Versatile''

\end{mnemonicbox}
\subsection*{Question 3(c OR) [7
marks]}\label{question-3c-or-7-marks}

\textbf{Explain interrupts of microcontroller.}

\begin{solutionbox}

The 8051 has a 5-source interrupt system for handling external events.


{\def\LTcaptype{none} % do not increment counter
\vspace{-5pt}
\captionof{table}{8051 Interrupt Sources}
\vspace{-10pt}
\begin{longtable}[]{@{}llll@{}}
\toprule\noalign{}
Interrupt & Vector Address & Priority & Trigger \\
\midrule\noalign{}
\endhead
\bottomrule\noalign{}
\endlastfoot
\textbf{Reset} & 0000H & Highest & Power-on/External \\
\textbf{External 0} & 0003H & High & INT0 pin \\
\textbf{Timer 0} & 000BH & Medium & Timer 0 overflow \\
\textbf{External 1} & 0013H & Medium & INT1 pin \\
\textbf{Timer 1} & 001BH & Low & Timer 1 overflow \\
\textbf{Serial} & 0023H & Lowest & Serial communication \\
\end{longtable}
}

\textbf{Diagram:}

\includegraphics[width=1\linewidth,height=\textheight,keepaspectratio]{mermaid-0bc4460e.pdf}

\begin{itemize}
\tightlist
\item
  \textbf{Interrupt Enable}: IE register controls individual interrupt
  enables
\item
  \textbf{Priority Control}: IP register sets interrupt priorities
\item
  \textbf{Vector Addresses}: Each interrupt has fixed vector location
\item
  \textbf{Nested Interrupts}: Higher priority can interrupt lower
  priority
\end{itemize}

\end{solutionbox}
\begin{mnemonicbox}
``Five Interrupt Sources Serve System Efficiently''

\end{mnemonicbox}
\subsection*{Question 4(a) [3 marks]}\label{q4a}

\textbf{Explain data transfer instruction with example for 8051.}

\begin{solutionbox}

Data transfer instructions move data between registers, memory, and I/O
ports.


{\def\LTcaptype{none} % do not increment counter
\vspace{-5pt}
\captionof{table}{Data Transfer Instructions}
\vspace{-10pt}
\begin{longtable}[]{@{}lll@{}}
\toprule\noalign{}
Instruction & Example & Function \\
\midrule\noalign{}
\endhead
\bottomrule\noalign{}
\endlastfoot
\textbf{MOV} & MOV A,\#55H & Move immediate data to accumulator \\
\textbf{MOVX} & MOVX A,@DPTR & Move external RAM to accumulator \\
\textbf{MOVC} & MOVC A,@A+PC & Move code memory to accumulator \\
\end{longtable}
}

\begin{itemize}
\tightlist
\item
  \textbf{MOV Variants}: Register to register, immediate to register
\item
  \textbf{External Access}: MOVX for external RAM operations\\
\item
  \textbf{Code Access}: MOVC for reading program memory tables
\end{itemize}

\end{solutionbox}
\begin{mnemonicbox}
``MOV Moves data, MOVX eXternal, MOVC Code''

\end{mnemonicbox}
\subsection*{Question 4(b) [4 marks]}\label{q4b}

\textbf{List and explain different addressing modes of microcontroller.}

\begin{solutionbox}

The 8051 supports several addressing modes for flexible data access.


{\def\LTcaptype{none} % do not increment counter
\vspace{-5pt}
\captionof{table}{8051 Addressing Modes}
\vspace{-10pt}
\begin{longtable}[]{@{}lll@{}}
\toprule\noalign{}
Mode & Example & Description \\
\midrule\noalign{}
\endhead
\bottomrule\noalign{}
\endlastfoot
\textbf{Immediate} & MOV A,\#55H & Data specified in instruction \\
\textbf{Register} & MOV A,R0 & Use register contents \\
\textbf{Direct} & MOV A,30H & Direct memory address \\
\textbf{Indirect} & MOV A,@R0 & Address stored in register \\
\textbf{Indexed} & MOVC A,@A+DPTR & Base address plus offset \\
\end{longtable}
}

\begin{itemize}
\tightlist
\item
  \textbf{Immediate Mode}: Constant data included in instruction
\item
  \textbf{Register Mode}: Fastest execution using register file
\item
  \textbf{Direct Mode}: Access any internal RAM location
\item
  \textbf{Indirect Mode}: Pointer-based addressing for arrays
\item
  \textbf{Indexed Mode}: Table lookup and array access
\end{itemize}

\end{solutionbox}
\begin{mnemonicbox}
``Immediate, Register, Direct, Indirect, Indexed
Addressing''

\end{mnemonicbox}
\subsection*{Question 4(c) [7 marks]}\label{q4c}

\textbf{Write a program to copy block of 8 data starting from location
100h to 200h.}

\begin{solutionbox}

\textbf{Assembly Program:}

\begin{lstlisting}
ORG 0000H           ; Start address
MOV R0,#100H        ; Source address pointer
MOV R1,#200H        ; Destination address pointer  
MOV R2,#08H         ; Counter for 8 bytes

LOOP:
MOV A,@R0           ; Read data from source
MOV @R1,A           ; Write data to destination
INC R0              ; Increment source pointer
INC R1              ; Increment destination pointer
DJNZ R2,LOOP        ; Decrement counter and jump if not zero

END                 ; End of program
\end{lstlisting}


{\def\LTcaptype{none} % do not increment counter
\vspace{-5pt}
\captionof{table}{Register Usage}
\vspace{-10pt}
\begin{longtable}[]{@{}ll@{}}
\toprule\noalign{}
Register & Purpose \\
\midrule\noalign{}
\endhead
\bottomrule\noalign{}
\endlastfoot
\textbf{R0} & Source address pointer (100H) \\
\textbf{R1} & Destination address pointer (200H) \\
\textbf{R2} & Loop counter (8 bytes) \\
\textbf{A} & Temporary data storage \\
\end{longtable}
}

\begin{itemize}
\tightlist
\item
  \textbf{Indirect Addressing}: @R0 and @R1 for memory access
\item
  \textbf{Loop Control}: DJNZ instruction decrements and tests
\item
  \textbf{Block Transfer}: Copies 8 consecutive bytes efficiently
\end{itemize}

\end{solutionbox}
\begin{mnemonicbox}
``Read, Write, Increment, Decrement, Jump Loop''

\end{mnemonicbox}
\subsection*{Question 4(a OR) [3
marks]}\label{question-4a-or-3-marks}

\textbf{Write a program to add two bytes of data and store result in R0
register.}

\begin{solutionbox}

\textbf{Assembly Program:}

\begin{lstlisting}
ORG 0000H           ; Start address
MOV A,#25H          ; Load first byte
ADD A,#35H          ; Add second byte
MOV R0,A            ; Store result in R0
END                 ; End program
\end{lstlisting}


{\def\LTcaptype{none} % do not increment counter
\vspace{-5pt}
\captionof{table}{Operation Steps}
\vspace{-10pt}
\begin{longtable}[]{@{}lll@{}}
\toprule\noalign{}
Step & Instruction & Result \\
\midrule\noalign{}
\endhead
\bottomrule\noalign{}
\endlastfoot
1 & MOV A,\#25H & A = 25H \\
2 & ADD A,\#35H & A = 5AH \\
3 & MOV R0,A & R0 = 5AH \\
\end{longtable}
}

\begin{itemize}
\tightlist
\item
  \textbf{Addition Result}: 25H + 35H = 5AH
\item
  \textbf{Flag Effects}: Carry flag set if result \textgreater{} FFH
\end{itemize}

\end{solutionbox}
\begin{mnemonicbox}
``Move, Add, Move = Simple Addition''

\end{mnemonicbox}
\subsection*{Question 4(b OR) [4
marks]}\label{question-4b-or-4-marks}

\textbf{Explain indexed addressing mode with example.}

\begin{solutionbox}

Indexed addressing uses a base address plus an offset for memory access.


{\def\LTcaptype{none} % do not increment counter
\vspace{-5pt}
\captionof{table}{Indexed Addressing Details}
\vspace{-10pt}
\begin{longtable}[]{@{}lll@{}}
\toprule\noalign{}
Component & Description & Example \\
\midrule\noalign{}
\endhead
\bottomrule\noalign{}
\endlastfoot
\textbf{Base Address} & DPTR or PC register & DPTR = 1000H \\
\textbf{Index} & Accumulator contents & A = 05H \\
\textbf{Effective Address} & Base + Index & 1000H + 05H = 1005H \\
\end{longtable}
}

\textbf{Example:}

\begin{lstlisting}
MOV DPTR,#1000H     ; Base address
MOV A,#05H          ; Index value
MOVC A,@A+DPTR      ; Read from address 1005H
\end{lstlisting}

\begin{itemize}
\tightlist
\item
  \textbf{Table Access}: Ideal for lookup tables and arrays
\item
  \textbf{Program Memory}: MOVC reads from code memory only
\item
  \textbf{Dynamic Indexing}: Index can change during execution
\end{itemize}

\end{solutionbox}
\begin{mnemonicbox}
``Base + Index = Dynamic Access''

\end{mnemonicbox}
\subsection*{Question 4(c OR) [7
marks]}\label{question-4c-or-7-marks}

\textbf{Explain stack operation of microcontroller, PUSH and POP
instruction.}

\begin{solutionbox}

The stack is a LIFO memory structure used for temporary data storage.


{\def\LTcaptype{none} % do not increment counter
\vspace{-5pt}
\captionof{table}{Stack Operations}
\vspace{-10pt}
\begin{longtable}[]{@{}lll@{}}
\toprule\noalign{}
Operation & Instruction & Function \\
\midrule\noalign{}
\endhead
\bottomrule\noalign{}
\endlastfoot
\textbf{PUSH} & PUSH 30H & Store data on stack \\
\textbf{POP} & POP 30H & Retrieve data from stack \\
\textbf{Stack Pointer} & SP register & Points to top of stack \\
\end{longtable}
}

\textbf{Diagram:}

\begin{lstlisting}
Stack Operation:
    
Before PUSH:     After PUSH 30H:    After POP 30H:
SP \rightarrow 07H            SP \rightarrow 08H           SP \rightarrow 07H
     06H                 08H: 30H           06H
     05H                 07H: old           05H
     
Stack grows upward in memory
\end{lstlisting}

\textbf{Example Program:}

\begin{lstlisting}
MOV SP,#30H         ; Initialize stack pointer
PUSH ACC            ; Save accumulator
PUSH B              ; Save B register
POP B               ; Restore B register
POP ACC             ; Restore accumulator
\end{lstlisting}

\begin{itemize}
\tightlist
\item
  \textbf{LIFO Structure}: Last In, First Out data organization
\item
  \textbf{SP Auto-increment}: Stack pointer automatically adjusts
\item
  \textbf{Subroutine Calls}: Stack saves return addresses
\item
  \textbf{Register Preservation}: Save/restore register contents
\end{itemize}

\end{solutionbox}
\begin{mnemonicbox}
``PUSH Puts Up, Stack Holds, POP Pulls Out''

\end{mnemonicbox}
\subsection*{Question 5(a) [3 marks]}\label{q5a}

\textbf{Explain branching instruction with example.}

\begin{solutionbox}

Branching instructions alter program flow based on conditions or
unconditionally.


{\def\LTcaptype{none} % do not increment counter
\vspace{-5pt}
\captionof{table}{Branching Instructions}
\vspace{-10pt}
\begin{longtable}[]{@{}lll@{}}
\toprule\noalign{}
Type & Instruction & Example \\
\midrule\noalign{}
\endhead
\bottomrule\noalign{}
\endlastfoot
\textbf{Unconditional} & LJMP address & LJMP 2000H \\
\textbf{Conditional} & JZ address & JZ ZERO\_LABEL \\
\textbf{Call/Return} & LCALL address & LCALL SUBROUTINE \\
\end{longtable}
}

\textbf{Example:}

\begin{lstlisting}
MOV A,#00H          ; Load zero
JZ ZERO_FOUND       ; Jump if A is zero
LJMP CONTINUE       ; Jump to continue
ZERO_FOUND:
    MOV R0,#01H     ; Set flag
CONTINUE:
    NOP             ; Continue execution
\end{lstlisting}

\begin{itemize}
\tightlist
\item
  \textbf{Program Control}: Changes execution sequence
\item
  \textbf{Conditional Jumps}: Based on flag register status
\item
  \textbf{Address Range}: Can jump to any program memory location
\end{itemize}

\end{solutionbox}
\begin{mnemonicbox}
``Jump Changes Control Flow''

\end{mnemonicbox}
\subsection*{Question 5(b) [4 marks]}\label{q5b}

\textbf{Interface 8 leds with microcontroller and write a program to
turn on and off.}

\begin{solutionbox}

\textbf{Circuit Diagram:}

\begin{lstlisting}
8051        LEDs
P1.0 ----[330Ω]----LED1----+5V
P1.1 ----[330Ω]----LED2----+5V  
P1.2 ----[330Ω]----LED3----+5V
P1.3 ----[330Ω]----LED4----+5V
P1.4 ----[330Ω]----LED5----+5V
P1.5 ----[330Ω]----LED6----+5V
P1.6 ----[330Ω]----LED7----+5V
P1.7 ----[330Ω]----LED8----+5V
\end{lstlisting}

\textbf{Program:}

\begin{lstlisting}
ORG 0000H
MAIN:
    MOV P1,#0FFH        ; Turn ON all LEDs
    CALL DELAY          ; Wait
    MOV P1,#00H         ; Turn OFF all LEDs  
    CALL DELAY          ; Wait
    SJMP MAIN           ; Repeat

DELAY:
    MOV R0,#0FFH        ; Outer loop counter
LOOP1:
    MOV R1,#0FFH        ; Inner loop counter  
LOOP2:
    DJNZ R1,LOOP2       ; Inner delay loop
    DJNZ R0,LOOP1       ; Outer delay loop
    RET                 ; Return
END
\end{lstlisting}


{\def\LTcaptype{none} % do not increment counter
\vspace{-5pt}
\captionof{table}{Components}
\vspace{-10pt}
\begin{longtable}[]{@{}lll@{}}
\toprule\noalign{}
Component & Value & Purpose \\
\midrule\noalign{}
\endhead
\bottomrule\noalign{}
\endlastfoot
\textbf{Resistor} & 330Ω & Current limiting \\
\textbf{LEDs} & 8 pieces & Visual indicators \\
\textbf{Port} & P1 & 8-bit output port \\
\end{longtable}
}

\begin{itemize}
\tightlist
\item
  \textbf{Current Limiting}: Resistors protect LEDs from overcurrent
\item
  \textbf{Port Configuration}: P1 used as output port for LED control
\item
  \textbf{Delay Routine}: Creates visible ON/OFF timing
\end{itemize}

\end{solutionbox}
\begin{mnemonicbox}
``Port Controls LEDs with Resistance and Delay''

\end{mnemonicbox}
\subsection*{Question 5(c) [7 marks]}\label{q5c}

\textbf{Interface LCD with microcontroller and write a program to
display ``welcome''.}

\begin{solutionbox}

\textbf{Circuit Connections:}

\begin{lstlisting}
8051        16x2 LCD
P2.0 --------> D4
P2.1 --------> D5  
P2.2 --------> D6
P2.3 --------> D7
P1.0 --------> RS (Register Select)
P1.1 --------> EN (Enable)
GND  --------> R/W (Write mode)
\end{lstlisting}

\textbf{Program:}

\begin{lstlisting}
ORG 0000H
    CALL LCD_INIT       ; Initialize LCD
    CALL DISPLAY_MSG    ; Display message
    SJMP $              ; Stop here

LCD_INIT:
    MOV P2,#38H         ; Function set: 8-bit, 2-line
    CALL COMMAND
    MOV P2,#0EH         ; Display ON, Cursor ON
    CALL COMMAND  
    MOV P2,#01H         ; Clear display
    CALL COMMAND
    MOV P2,#06H         ; Entry mode set
    CALL COMMAND
    RET

DISPLAY_MSG:
    MOV DPTR,#MESSAGE   ; Point to message
NEXT_CHAR:
    CLR A
    MOVC A,@A+DPTR      ; Read character
    JZ DONE             ; If zero, end of string
    CALL SEND_CHAR      ; Send character to LCD
    INC DPTR            ; Next character
    SJMP NEXT_CHAR
DONE:
    RET

COMMAND:
    CLR P1.0            ; RS = 0 for command
    SETB P1.1           ; EN = 1
    CLR P1.1            ; EN = 0 (pulse)
    CALL DELAY
    RET

SEND_CHAR:
    MOV P2,A            ; Put character on data lines
    SETB P1.0           ; RS = 1 for data
    SETB P1.1           ; EN = 1
    CLR P1.1            ; EN = 0 (pulse)
    CALL DELAY
    RET

DELAY:
    MOV R0,#50          ; Delay routine
DELAY_LOOP:
    MOV R1,#255
DELAY_INNER:
    DJNZ R1,DELAY_INNER
    DJNZ R0,DELAY_LOOP
    RET

MESSAGE:
    DB "WELCOME",0       ; Message string with null terminator
END
\end{lstlisting}


{\def\LTcaptype{none} % do not increment counter
\vspace{-5pt}
\captionof{table}{LCD Interface Pins}
\vspace{-10pt}
\begin{longtable}[]{@{}lll@{}}
\toprule\noalign{}
8051 Pin & LCD Pin & Function \\
\midrule\noalign{}
\endhead
\bottomrule\noalign{}
\endlastfoot
\textbf{P2.0-P2.3} & D4-D7 & 4-bit data lines \\
\textbf{P1.0} & RS & Register select (0=command, 1=data) \\
\textbf{P1.1} & EN & Enable pulse \\
\textbf{GND} & R/W & Read/Write (tied to ground for write) \\
\end{longtable}
}

\begin{itemize}
\tightlist
\item
  \textbf{4-bit Mode}: Uses only upper 4 data lines to save pins
\item
  \textbf{Control Signals}: RS selects command/data, EN provides timing
  pulse
\item
  \textbf{Character Display}: Each character sent as ASCII code
\item
  \textbf{Initialization}: Required command sequence for proper
  operation
\end{itemize}

\end{solutionbox}
\begin{mnemonicbox}
``LCD Displays Characters with Commands and Data''

\end{mnemonicbox}
\subsection*{Question 5(a OR) [3
marks]}\label{question-5a-or-3-marks}

\textbf{Explain logical instruction with example.}

\begin{solutionbox}

Logical instructions perform bitwise operations on data.


{\def\LTcaptype{none} % do not increment counter
\vspace{-5pt}
\captionof{table}{Logical Instructions}
\vspace{-10pt}
\begin{longtable}[]{@{}lll@{}}
\toprule\noalign{}
Instruction & Example & Function \\
\midrule\noalign{}
\endhead
\bottomrule\noalign{}
\endlastfoot
\textbf{ANL} & ANL A,\#0FH & Bitwise AND operation \\
\textbf{ORL} & ORL A,\#F0H & Bitwise OR operation \\
\textbf{XRL} & XRL A,\#FFH & Bitwise XOR operation \\
\end{longtable}
}

\textbf{Example:}

\begin{lstlisting}
MOV A,#55H          ; A = 01010101B
ANL A,#0FH          ; A = 00000101B (mask upper bits)
ORL A,#F0H          ; A = 11110101B (set upper bits)
XRL A,#FFH          ; A = 00001010B (complement all bits)
\end{lstlisting}

\begin{itemize}
\tightlist
\item
  \textbf{Bit Manipulation}: Used for setting, clearing, and testing
  bits
\item
  \textbf{Masking Operations}: ANL clears unwanted bits
\item
  \textbf{Flag Effects}: Updates parity flag based on result
\end{itemize}

\end{solutionbox}
\begin{mnemonicbox}
``AND Masks, OR Sets, XOR Toggles''

\end{mnemonicbox}
\subsection*{Question 5(b OR) [4
marks]}\label{question-5b-or-4-marks}

\textbf{Interface 7 segment with microcontroller.}

\begin{solutionbox}

\textbf{Circuit Diagram:}

\begin{lstlisting}
8051          7-Segment Display
P1.0 ----[330Ω]----a
P1.1 ----[330Ω]----b  
P1.2 ----[330Ω]----c
P1.3 ----[330Ω]----d
P1.4 ----[330Ω]----e
P1.5 ----[330Ω]----f
P1.6 ----[330Ω]----g
P1.7 ----[330Ω]----dp (decimal point)
\end{lstlisting}

\textbf{Program to Display 0-9:}

\begin{lstlisting}
ORG 0000H
    MOV DPTR,#DIGIT_TABLE   ; Point to lookup table
    MOV R0,#0               ; Start with digit 0

MAIN_LOOP:
    MOV A,R0                ; Get current digit
    MOVC A,@A+DPTR          ; Get 7-segment code
    MOV P1,A                ; Display on 7-segment
    CALL DELAY              ; Wait 1 second
    INC R0                  ; Next digit
    CJNE R0,#10,MAIN_LOOP   ; Check if reached 10
    MOV R0,#0               ; Reset to 0
    SJMP MAIN_LOOP          ; Repeat

DIGIT_TABLE:
    DB 3FH, 06H, 5BH, 4FH, 66H    ; 0,1,2,3,4
    DB 6DH, 7DH, 07H, 7FH, 6FH    ; 5,6,7,8,9
END
\end{lstlisting}


{\def\LTcaptype{none} % do not increment counter
\vspace{-5pt}
\captionof{table}{7-Segment Codes}
\vspace{-10pt}
\begin{longtable}[]{@{}llll@{}}
\toprule\noalign{}
Digit & Hex Code & Binary & Segments Lit \\
\midrule\noalign{}
\endhead
\bottomrule\noalign{}
\endlastfoot
\textbf{0} & 3FH & 00111111 & a,b,c,d,e,f \\
\textbf{1} & 06H & 00000110 & b,c \\
\textbf{2} & 5BH & 01011011 & a,b,g,e,d \\
\end{longtable}
}

\begin{itemize}
\tightlist
\item
  \textbf{Common Cathode}: Segments light when port pin is high
\item
  \textbf{Current Limiting}: Resistors prevent segment damage
\item
  \textbf{Lookup Table}: Efficient storage of segment patterns
\end{itemize}

\end{solutionbox}
\begin{mnemonicbox}
``Seven Segments Show Digits Clearly''

\end{mnemonicbox}
\subsection*{Question 5(c OR) [7
marks]}\label{question-5c-or-7-marks}

\textbf{Interface LM 35 with microcontroller and explain block diagram
of temperature controller.}

\begin{solutionbox}

\textbf{Circuit Diagram:}

\begin{lstlisting}
LM35 Temperature Sensor Interface:

+5V ----+---- LM35 ----+---- ADC0804 ----+---- 8051
        |     (Vout)   |     (Vin)       |     (P1)
       GND             |                 |
                      GND                |
                                         |
Relay Control:                           |
8051 P3.0 ----[ULN2003]---- Relay -------+
                                         |
                                    Load (Heater/Fan)
\end{lstlisting}

\textbf{Temperature Controller Block Diagram:}

\includegraphics[width=1\linewidth,height=\textheight,keepaspectratio]{mermaid-95418e06.pdf}

\textbf{Control Program:}

\begin{lstlisting}
ORG 0000H
MAIN:
    CALL READ_TEMP      ; Read temperature from ADC
    CALL DISPLAY_TEMP   ; Show temperature on display
    CALL TEMP_CONTROL   ; Control heating/cooling
    CALL DELAY          ; Wait before next reading
    SJMP MAIN

READ_TEMP:
    CLR P2.0            ; Start ADC conversion
    SETB P2.0           ; Pulse to start
    JNB P2.1,$          ; Wait for conversion complete
    MOV A,P1            ; Read temperature data
    RET

TEMP_CONTROL:
    CJNE A,#30,CHECK_HIGH   ; Compare with setpoint (30^\circC)
CHECK_HIGH:
    JC TEMP_LOW             ; If A < 30, temperature is low
    SETB P3.0               ; Turn ON cooling (fan)
    CLR P3.1                ; Turn OFF heating
    RET
TEMP_LOW:
    CLR P3.0                ; Turn OFF cooling
    SETB P3.1               ; Turn ON heating
    RET
END
\end{lstlisting}


{\def\LTcaptype{none} % do not increment counter
\vspace{-5pt}
\captionof{table}{System Components}
\vspace{-10pt}
\begin{longtable}[]{@{}ll@{}}
\toprule\noalign{}
Component & Function \\
\midrule\noalign{}
\endhead
\bottomrule\noalign{}
\endlastfoot
\textbf{LM35} & Temperature sensor (10mV/^\circC) \\
\textbf{ADC0804} & Analog to digital converter \\
\textbf{8051} & Main controller \\
\textbf{Relay} & Switch high power loads \\
\textbf{Display} & Show current temperature \\
\end{longtable}
}

\begin{itemize}
\tightlist
\item
  \textbf{Temperature Sensing}: LM35 provides 10mV per degree Celsius
\item
  \textbf{ADC Conversion}: Converts analog voltage to digital value
\item
  \textbf{Control Logic}: Compares with setpoint and controls relay
\item
  \textbf{Feedback System}: Continuous monitoring and adjustment
\item
  \textbf{Safety Features}: Over-temperature protection possible
\end{itemize}

\end{solutionbox}
\begin{mnemonicbox}
``Sense, Convert, Compare, Control Temperature
Automatically''

\end{mnemonicbox}

\end{document}
