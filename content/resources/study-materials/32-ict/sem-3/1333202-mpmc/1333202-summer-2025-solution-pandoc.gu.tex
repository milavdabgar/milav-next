\documentclass[10pt,a4paper]{article}

% content/resources/templates/preamble.tex
\usepackage[margin=0.6in]{geometry}
\author{Milav Dabgar}
\usepackage{amsmath,amssymb,amsthm}
\usepackage{booktabs}
\usepackage{multirow}
\usepackage{xcolor}
\usepackage{tcolorbox}
\tcbuselibrary{breakable,skins}
\usepackage[colorlinks=true,linkcolor=blue]{hyperref}
\usepackage{titlesec}
\usepackage{enumitem}
\usepackage{tikz}
\usepackage{pgfplots}
\usepackage{circuitikz}
\usepackage[version=4]{mhchem}
\usepackage{longtable}
\usepackage{array}
\usepackage{float}
\usepackage{caption}
\usepackage{listings}

\lstset{
  basicstyle=\small\ttfamily,
  breaklines=true,
  breakatwhitespace=false,
  postbreak=\mbox{\textcolor{red}{$\hookrightarrow$}\space},
  float=false,
  numbers=left,
  numberstyle=\tiny\color{gray},
  numbersep=10pt,
  xleftmargin=2em,
  keywordstyle=\color{blue},
  commentstyle=\color{green!60!black},
  stringstyle=\color{purple},
  backgroundcolor=\color{gray!5},
  showstringspaces=false,
  tabsize=2,
  captionpos=b,
  keepspaces=true,
  columns=flexible
}

\pgfplotsset{compat=1.18}
\usetikzlibrary{shapes,arrows,positioning,calc,patterns,decorations.pathmorphing,decorations.markings,arrows.meta}

% Color scheme
\definecolor{headcolor}{RGB}{0,102,204}
\definecolor{keycolor}{RGB}{220,20,60}
\definecolor{solutioncolor}{RGB}{34,139,34}
\definecolor{mnemoniccolor}{RGB}{148,0,211}
\definecolor{codecolor}{RGB}{0,0,100}

% Spacing
\setlength{\parskip}{3pt}
\setlist[itemize]{nosep}
\setlist[enumerate]{nosep}

% Title formatting
\titleformat{\section}{\Large\bfseries\color{headcolor}}{\thesection}{1em}{}
\titleformat{\subsection}{\large\bfseries\color{headcolor}}{\thesubsection}{1em}{}

% Pandoc tightlist compatibility
\providecommand{\tightlist}{%
  \setlength{\itemsep}{0pt}\setlength{\parskip}{0pt}}

% Pandoc longtable compatibility
\newcounter{none}
\def\thenone{}


% content/resources/templates/gujarati-boxes.tex
\usepackage{fontspec}
\usepackage{polyglossia}

% Set Gujarati as main language (document is primarily in Gujarati)
% Note: gloss-gujarati.ldf doesn't exist in polyglossia, but it will use hyphenation patterns
\setdefaultlanguage{gujarati}
\setotherlanguage{english}

% Configure Gujarati font properly
% Use Language=Default to prevent polyglossia from trying to add language-specific features
% that don't exist for Gujarati, which causes "empty feature" warnings
\newfontfamily\gujaratifont[Script=Gujarati,AutoFakeBold=2.5,AutoFakeSlant=0.3]{Noto Sans Gujarati}
\setmainfont[Script=Gujarati,AutoFakeBold=2.5,AutoFakeSlant=0.3]{Noto Sans Gujarati}
% Use Noto Sans Gujarati for monospace to support Gujarati in text
\setmonofont[Scale=0.9]{Noto Sans Gujarati}

% Configure English to use the same font
\newfontfamily\englishfont[Script=Gujarati,AutoFakeBold=2.5,AutoFakeSlant=0.3]{Noto Sans Gujarati}

% Translations for polyglossia
\gappto\captionsgujarati{
  \renewcommand{\tablename}{કોષ્ટક}
  \renewcommand{\figurename}{આકૃતિ}
}

% Helper for TikZ nodes to ensure Gujarati font
\newcommand{\gu}[1]{{\gujaratifont #1}}

% Custom environments
\newtcolorbox{solutionbox}{
    breakable,
    enhanced,
    colback=solutioncolor!5!white,
    colframe=solutioncolor!75!black,
    fonttitle=\bfseries,
    title=જવાબ
}

\newtcolorbox{solutionboxnobreak}{
 colback=solutioncolor!5!white,
 colframe=solutioncolor!75!black,
 fonttitle=\bfseries,
 title=જવાબ
}

\newtcolorbox{keyformula}{
 breakable,
 enhanced,
 colback=keycolor!5!white,
 colframe=keycolor!75!black,
 fonttitle=\bfseries,
 title=રાસાયણિક સમીકરણ/સૂત્ર
}

\newtcolorbox{mnemonicbox}{
 breakable,
 enhanced,
 colback=mnemoniccolor!5!white,
 colframe=mnemoniccolor!75!black,
 fonttitle=\bfseries,
 title=મેમરી ટ્રીક
}


\begin{document}

\begin{center}
{\Huge\bfseries\color{headcolor} Subject Name (Gujarati)}\\[5pt]
{\LARGE 1333202 -- Summer 2025}\\[3pt]
{\large Semester 1 Study Material}\\[3pt]
{\normalsize\textit{Detailed Solutions and Explanations}}
\end{center}

\vspace{10pt}

\subsection*{પ્રશ્ન 1(અ) [3
ગુણ]}\label{uxaaauxab0uxab6uxaa8-1uxa85-3-uxa97uxaa3}

\textbf{8085 નું બસ ઓર્ગેનાઈઝેશન દોરો.}

\begin{solutionbox}

\begin{verbatim}
                    8085 MICROPROCESSOR
                    
    +{-{-}{-}{-}{-}{-}{-}+      16{-}bit         +{-}{-}{-}{-}{-}{-}{-}+}
    |       |{{-}{-}{-}{-}{-}{-}{-}{-}{-}{-}{-}{-}{-}{-}{-}{-}{-}{-}{-}| Memory|}
    |       |   Address Bus       |       |
    |  8085 |                     +{-{-}{-}{-}{-}{-}{-}+}
    |  CPU  |      8{-bit          }
    |       |{{-}{-}{-}{-}{-}{-}{-}{-}{-}{-}{-}{-}{-}{-}{-}{-}{-}{-}{-}+{-}{-}{-}{-}{-}{-}{-}+}
    |       |    Data Bus         | I/O   |
    +{-{-}{-}{-}{-}{-}{-}+                     | Ports |}
         \^{                        +{-}{-}{-}{-}{-}{-}{-}+}
         |
         | Control Bus
         v
    +{-{-}{-}{-}{-}{-}{-}{-}{-}{-}{-}+}
    | Control   |
    | Signals   |
    +{-{-}{-}{-}{-}{-}{-}{-}{-}{-}{-}+}
\end{verbatim}

\textbf{બસના પ્રકારો:}

\begin{itemize}
\tightlist
\item
  \textbf{Address Bus}: મેમરી એડ્રેસિંગ માટે 16-bit એકદિશીય બસ
\item
  \textbf{Data Bus}: ડેટા ટ્રાન્સફર માટે 8-bit દ્વિદિશીય બસ\\
\item
  \textbf{Control Bus}: RD, WR, ALE, IO/M જેવા કંટ્રોલ સિગ્નલ્સ
\end{itemize}

\end{solutionbox}
\begin{mnemonicbox}
``ADC - Address, Data, Control''

\end{mnemonicbox}
\subsection*{પ્રશ્ન 1(બ) [4
ગુણ]}\label{uxaaauxab0uxab6uxaa8-1uxaac-4-uxa97uxaa3}

\textbf{માઈક્રોપ્રોસેસરની માઈક્રોકંટ્રોલર સાથે સરખામણી કરો.}

\begin{solutionbox}

{\def\LTcaptype{none} % do not increment counter
\begin{longtable}[]{@{}lll@{}}
\toprule\noalign{}
લક્ષણ & માઈક્રોપ્રોસેસર & માઈક્રોકંટ્રોલર \\
\midrule\noalign{}
\endhead
\bottomrule\noalign{}
\endlastfoot
\textbf{આર્કિટેક્ચર} & બાહ્ય ઘટકોની જરૂર & એક જ ચિપ પર બધા ઘટકો \\
\textbf{મેમરી} & બાહ્ય RAM/ROM જરૂરી & આંતરિક RAM/ROM ઉપલબ્ધ \\
\textbf{કિંમત} & વધુ સિસ્ટમ કોસ્ટ & ઓછી સિસ્ટમ કોસ્ટ \\
\textbf{પાવર} & વધુ પાવર વપરાશ & ઓછો પાવર વપરાશ \\
\textbf{સાઈઝ} & મોટું સિસ્ટમ સાઈઝ & કોમ્પેક્ટ સિસ્ટમ \\
\textbf{ઉપયોગ} & સામાન્ય હેતુ કમ્પ્યુટિંગ & એમ્બેડેડ કંટ્રોલ એપ્લિકેશનો \\
\end{longtable}
}

\textbf{મુખ્ય મુદ્દાઓ:}

\begin{itemize}
\tightlist
\item
  \textbf{માઈક્રોપ્રોસેસર}: માત્ર CPU, બાહ્ય સપોર્ટ ચિપ્સ જરૂરી
\item
  \textbf{માઈક્રોકંટ્રોલર}: ચિપ પર સંપૂર્ણ કમ્પ્યુટર સિસ્ટમ
\end{itemize}

\end{solutionbox}
\begin{mnemonicbox}
``MICRO - Memory Internal, Compact, Reduced cost,
Optimized''

\end{mnemonicbox}
\subsection*{પ્રશ્ન 1(ક) [7
ગુણ]}\label{uxaaauxab0uxab6uxaa8-1uxa95-7-uxa97uxaa3}

\textbf{8085 માઈક્રોપ્રોસેસરના દરેક બ્લોક દોરો અને સમજાવો.}

\begin{solutionbox}

\begin{verbatim}
graph TB
    A[Accumulator] {-{-} ALU[Arithmetic Logic Unit]}
    B[Register Array] {-{-} ALU}
    ALU {-{-} F[Flag Register]}
    
    PC[Program Counter] {-{-} AB[Address Buffer]}
    SP[Stack Pointer] {-{-} AB}
    AB {-{-} ADDR[Address Bus 16{-}bit]}
    
    IR[Instruction Register] {-{-} ID[Instruction Decoder]}
    ID {-{-} CU[Control Unit]}
    CU {-{-} CB[Control Bus]}
    
    DB[Data Buffer] {{-}{-} DATA[Data Bus 8{-}bit]}
    
    T[Timing \& Control] {-{-} CU}
\end{verbatim}

\textbf{બ્લોકના કાર્યો:}

\begin{itemize}
\tightlist
\item
  \textbf{ALU}: અંકગણિત અને તાર્કિક ઓપરેશન કરે છે
\item
  \textbf{Accumulator}: ડેટા પ્રોસેસિંગ માટે પ્રાથમિક કામકાજ રજિસ્ટર
\item
  \textbf{Register Array}: B, C, D, E, H, L સામાન્ય હેતુ રજિસ્ટરો
\item
  \textbf{Program Counter}: આગળના instruction નું address ધરાવે છે
\item
  \textbf{Stack Pointer}: મેમરીમાં stack ના ટોપને પોઈન્ટ કરે છે
\item
  \textbf{Control Unit}: પ્રોસેસરના એકંદર ઓપરેશનને કંટ્રોલ કરે છે
\end{itemize}

\end{solutionbox}
\begin{mnemonicbox}
``APRIL - ALU, Program counter, Registers,
Instruction decoder, Logic control''

\end{mnemonicbox}
\subsection*{પ્રશ્ન 1(ક) OR [7
ગુણ]}\label{uxaaauxab0uxab6uxaa8-1uxa95-or-7-uxa97uxaa3}

\textbf{8085 માઈક્રોપ્રોસેસરનો પીન ડાયાગ્રામ દોરો અને કોઈ પણ 4 પીન સમજાવો.}

\begin{solutionbox}

\begin{verbatim}
                  8085 PIN DIAGRAM
                  
      X1    1 +{-{-}{-}{-}{-}{-}{-}+ 40  Vcc}
      X2    2 |       | 39  HOLD
    RESET   3 |       | 38  HLDA  
     SOD    4 |       | 37  CLK(OUT)
     SID    5 |  8085 | 36  RESET IN
    TRAP    6 |       | 35  READY
    RST7.5  7 |       | 34  IO/M
    RST6.5  8 |       | 33  S1
    RST5.5  9 |       | 32  RD
    INTR   10 |       | 31  WR
    INTA   11 |       | 30  ALE
   AD0{-7 12{-}19|       | 23{-}29 A8{-}A15}
     Vss   20 +{-{-}{-}{-}{-}{-}{-}+ 21  A15{-}A8}
\end{verbatim}

\textbf{પીન સમજાવટ:}

\begin{itemize}
\tightlist
\item
  \textbf{ALE (Pin 30)}: Address Latch Enable - multiplexed bus પર
  address અને data અલગ કરે છે
\item
  \textbf{RD (Pin 32)}: Read control signal - active low, read operation
  દર્શાવે છે
\item
  \textbf{WR (Pin 31)}: Write control signal - active low, write
  operation દર્શાવે છે
\item
  \textbf{RESET (Pin 36)}: Reset input - low થાય ત્યારે processor
  initialize કરે છે
\end{itemize}

\end{solutionbox}
\begin{mnemonicbox}
``ARWA - ALE, Read, Write, rAset''

\end{mnemonicbox}
\subsection*{પ્રશ્ન 2(અ) [3
ગુણ]}\label{uxaaauxab0uxab6uxaa8-2uxa85-3-uxa97uxaa3}

\textbf{વ્યાખ્યા આપો: (1) Opcode (2) Operand}

\begin{solutionbox}

\textbf{વ્યાખ્યાઓ:}

\begin{itemize}
\tightlist
\item
  \textbf{Opcode}: Operation Code - કરવાનું operation સ્પષ્ટ કરે છે (ADD, MOV,
  JMP)
\item
  \textbf{Operand}: જે ડેટા અથવા address પર operation કરવાનું છે
\end{itemize}

\textbf{ઉદાહરણ:}

\begin{verbatim}
MOV A, B
|   |  |
|   |  +-- Operand 2 (Source)  
|   +-- Operand 1 (Destination)
+-- Opcode
\end{verbatim}

\end{solutionbox}
\begin{mnemonicbox}
``OO - Operation + Operand''

\end{mnemonicbox}
\subsection*{પ્રશ્ન 2(બ) [4
ગુણ]}\label{uxaaauxab0uxab6uxaa8-2uxaac-4-uxa97uxaa3}

\textbf{RISC અને CISC વચ્ચે તફાવત આપો.}

\begin{solutionbox}

{\def\LTcaptype{none} % do not increment counter
\begin{longtable}[]{@{}lll@{}}
\toprule\noalign{}
લક્ષણ & RISC & CISC \\
\midrule\noalign{}
\endhead
\bottomrule\noalign{}
\endlastfoot
\textbf{Instructions} & સરળ, fixed format & જટિલ, variable format \\
\textbf{Execution} & Single cycle execution & Multiple cycle
execution \\
\textbf{Addressing} & થોડા addressing modes & ઘણા addressing modes \\
\textbf{Memory} & Load/Store architecture & Memory-to-memory
operations \\
\textbf{Compiler} & જટિલ compiler design & સરળ compiler design \\
\end{longtable}
}

\textbf{મુખ્ય મુદ્દાઓ:}

\begin{itemize}
\tightlist
\item
  \textbf{RISC}: Reduced Instruction Set Computer - સરળ, ઝડપી
\item
  \textbf{CISC}: Complex Instruction Set Computer - feature rich
\end{itemize}

\end{solutionbox}
\begin{mnemonicbox}
``RISC is SLIM - Simple, Load-store, Instruction
reduced, Memory efficient''

\end{mnemonicbox}
\subsection*{પ્રશ્ન 2(ક) [7
ગુણ]}\label{uxaaauxab0uxab6uxaa8-2uxa95-7-uxa97uxaa3}

\textbf{Von-Neumann અને Harvard Architecture વચ્ચે તફાવત આપો.}

\begin{solutionbox}

{\def\LTcaptype{none} % do not increment counter
\begin{longtable}[]{@{}
  >{\raggedright\arraybackslash}p{(\linewidth - 4\tabcolsep) * \real{0.2414}}
  >{\raggedright\arraybackslash}p{(\linewidth - 4\tabcolsep) * \real{0.4483}}
  >{\raggedright\arraybackslash}p{(\linewidth - 4\tabcolsep) * \real{0.3103}}@{}}
\toprule\noalign{}
\begin{minipage}[b]{\linewidth}\raggedright
લક્ષણ
\end{minipage} & \begin{minipage}[b]{\linewidth}\raggedright
Von-Neumann
\end{minipage} & \begin{minipage}[b]{\linewidth}\raggedright
Harvard
\end{minipage} \\
\midrule\noalign{}
\endhead
\bottomrule\noalign{}
\endlastfoot
\textbf{Memory} & data અને instructions માટે single memory & data અને
instructions માટે અલગ memory \\
\textbf{Bus Structure} & Single bus system & Dual bus system \\
\textbf{Access} & data અને instructions ને sequential access &
simultaneous access શક્ય \\
\textbf{Cost} & ઓછી કિંમત & વધુ કિંમત \\
\textbf{Speed} & bus conflicts કારણે ધીમું & parallel access કારણે ઝડપી \\
\textbf{Examples} & 8085, સામાન્ય computers & 8051, DSP processors \\
\end{longtable}
}

\begin{center}
\textbf{Mermaid Diagram (Code)}
\begin{verbatim}
{Shaded}
{Highlighting}[]
graph TD
    subgraph "Von{-Neumann"}
        CPU1[CPU] {{-}{-}{} MEM1[Combined Memory{}br/{}Data + Instructions]}
    end
    
    subgraph "Harvard"  
        CPU2[CPU] {{-}{-}{} IMEM[Instruction Memory]}
        CPU2 {{-}{-}{} DMEM[Data Memory]}
    end
{Highlighting}
{Shaded}
\end{verbatim}
\end{center}

\end{solutionbox}
\begin{mnemonicbox}
``VH - Von has one bus, Harvard has two''

\end{mnemonicbox}
\subsection*{પ્રશ્ન 2(અ) OR [3
ગુણ]}\label{uxaaauxab0uxab6uxaa8-2uxa85-or-3-uxa97uxaa3}

\textbf{વ્યાખ્યા આપો: (1) T-State (2) Instruction Cycle (3) Machine Cycle}

\begin{solutionbox}

\textbf{વ્યાખ્યાઓ:}

\begin{itemize}
\tightlist
\item
  \textbf{T-State}: Time state - મૂળભૂત timing unit, એક clock period
\item
  \textbf{Instruction Cycle}: એક instruction નું સંપૂર્ણ execution
\item
  \textbf{Machine Cycle}: એક memory operation માટે જરૂરી T-states નું જૂથ
\end{itemize}

\textbf{સંબંધ:}

\begin{verbatim}
Instruction Cycle = Multiple Machine Cycles
Machine Cycle = Multiple T-States (3-6 T-states)
\end{verbatim}

\end{solutionbox}
\begin{mnemonicbox}
``TIM - T-state, Instruction cycle, Machine cycle''

\end{mnemonicbox}
\subsection*{પ્રશ્ન 2(બ) OR [4
ગુણ]}\label{uxaaauxab0uxab6uxaa8-2uxaac-or-4-uxa97uxaa3}

\textbf{8085 ના Address અને Data Bus નું De-Multiplexing સમજાવો.}

\begin{solutionbox}

\begin{center}
\textbf{Mermaid Diagram (Code)}
\begin{verbatim}
{Shaded}
{Highlighting}[]
graph LR
    A[AD0{-AD7{}br/{}Multiplexed Bus] {-}{-}{} L[74LS373 Latch]}
    ALE[ALE Signal] {-{-}{} L}
    L {-{-}{} ADDR[A0{-}A7{}br/{}Address Lines]}
    A {-{-}{} DATA[D0{-}D7{}br/{}Data Lines]}
{Highlighting}
{Shaded}
\end{verbatim}
\end{center}

\textbf{પ્રક્રિયા:}

\begin{itemize}
\tightlist
\item
  \textbf{Step 1}: T1 દરમિયાન, AD0-AD7 માં lower 8-bit address હોય છે
\item
  \textbf{Step 2}: ALE high થાય છે, external latch માં address latch થાય છે
\item
  \textbf{Step 3}: બાકીના T-states માટે AD0-AD7 data bus બને છે
\end{itemize}

\textbf{જરૂરી ઘટકો:}

\begin{itemize}
\tightlist
\item
  \textbf{74LS373}: Address latching માટે Octal latch IC
\item
  \textbf{ALE}: Timing માટે Address Latch Enable signal
\end{itemize}

\end{solutionbox}
\begin{mnemonicbox}
``LAD - Latch Address with Data separation''

\end{mnemonicbox}
\subsection*{પ્રશ્ન 2(ક) OR [7
ગુણ]}\label{uxaaauxab0uxab6uxaa8-2uxa95-or-7-uxa97uxaa3}

\textbf{8085 નો Flag Register દોરો અને સમજાવો.}

\begin{solutionbox}

\begin{verbatim}
    D7   D6   D5   D4   D3   D2   D1   D0
   +{-{-}{-}{-}+{-}{-}{-}{-}+{-}{-}{-}{-}+{-}{-}{-}{-}+{-}{-}{-}{-}+{-}{-}{-}{-}+{-}{-}{-}{-}+{-}{-}{-}{-}+}
   | S  | Z  | X  | AC | X  | P  | X  | CY |
   +{-{-}{-}{-}+{-}{-}{-}{-}+{-}{-}{-}{-}+{-}{-}{-}{-}+{-}{-}{-}{-}+{-}{-}{-}{-}+{-}{-}{-}{-}+{-}{-}{-}{-}+}
\end{verbatim}

\textbf{Flag વર્ણન:}

\begin{itemize}
\tightlist
\item
  \textbf{CY (D0)}: Carry flag - carry આવે ત્યારે set થાય છે
\item
  \textbf{P (D2)}: Parity flag - even parity માટે set થાય છે
\item
  \textbf{AC (D4)}: Auxiliary carry - BCD operations માટે set થાય છે
\item
  \textbf{Z (D6)}: Zero flag - પરિણામ zero હોય ત્યારે set થાય છે
\item
  \textbf{S (D7)}: Sign flag - પરિણામ negative હોય ત્યારે set થાય છે
\end{itemize}

\textbf{Flag Operations:}

\begin{itemize}
\tightlist
\item
  \textbf{Conditional Jumps}: Flag status પર આધારિત (JZ, JC, JP)
\item
  \textbf{Arithmetic Results}: ALU operations પછી automatically update
  થાય છે
\end{itemize}

\end{solutionbox}
\begin{mnemonicbox}
``SZAPC - Sign, Zero, Auxiliary, Parity, Carry''

\end{mnemonicbox}
\subsection*{પ્રશ્ન 3(અ) [3
ગુણ]}\label{uxaaauxab0uxab6uxaa8-3uxa85-3-uxa97uxaa3}

\textbf{SFR એટલે શું? કોઈ પણ ત્રણ SFR ની યાદી બનાવો.}

\begin{solutionbox}

\textbf{SFR વ્યાખ્યા:} \textbf{Special Function Register} -
microcontroller માં વિશિષ્ટ કાર્યો સાથે dedicated registers

\textbf{ત્રણ SFRs:}

\begin{itemize}
\tightlist
\item
  \textbf{ACC (E0H)}: Accumulator register
\item
  \textbf{PSW (D0H)}: Program Status Word
\item
  \textbf{SP (81H)}: Stack Pointer register
\end{itemize}

\textbf{લાક્ષણિકતાઓ:}

\begin{itemize}
\tightlist
\item
  \textbf{Address Range}: Internal RAM માં 80H થી FFH
\item
  \textbf{Bit Addressable}: કેટલાક SFRs individual bit access આપે છે
\item
  \textbf{Function Specific}: દરેકનું dedicated hardware function હોય છે
\end{itemize}

\end{solutionbox}
\begin{mnemonicbox}
``APS - ACC, PSW, Stack Pointer''

\end{mnemonicbox}
\subsection*{પ્રશ્ન 3(બ) [4
ગુણ]}\label{uxaaauxab0uxab6uxaa8-3uxaac-4-uxa97uxaa3}

\textbf{Program Counter (PC) અને Data Pointer (DPTR) Register સમજાવો.}

\begin{solutionbox}

\textbf{Program Counter (PC):}

\begin{itemize}
\tightlist
\item
  \textbf{Size}: 16-bit register
\item
  \textbf{Function}: આગળના instruction નું address ધરાવે છે
\item
  \textbf{Auto-increment}: Instruction fetch પછી automatically increment
  થાય છે
\item
  \textbf{Range}: 0000H થી FFFFH
\end{itemize}

\textbf{Data Pointer (DPTR):}

\begin{itemize}
\tightlist
\item
  \textbf{Size}: 16-bit register (DPH + DPL)
\item
  \textbf{Function}: External data memory locations ને point કરે છે
\item
  \textbf{Usage}: External memory access માટે MOVX instructions સાથે વપરાય
  છે
\item
  \textbf{Components}: DPH (83H) અને DPL (82H)
\end{itemize}

\begin{verbatim}
PC:   +{-{-}{-}{-}{-}{-}{-}{-}+{-}{-}{-}{-}{-}{-}{-}{-}+}
      |   PCH  |   PCL  |  16{-bit}
      +{-{-}{-}{-}{-}{-}{-}{-}+{-}{-}{-}{-}{-}{-}{-}{-}+}

DPTR: +{-{-}{-}{-}{-}{-}{-}{-}+{-}{-}{-}{-}{-}{-}{-}{-}+}
      |   DPH  |   DPL  |  16{-bit  }
      +{-{-}{-}{-}{-}{-}{-}{-}+{-}{-}{-}{-}{-}{-}{-}{-}+}
      |  83H   |  82H   |
\end{verbatim}

\end{solutionbox}
\begin{mnemonicbox}
``PD - PC Points to Program, DPTR Points to Data''

\end{mnemonicbox}
\subsection*{પ્રશ્ન 3(ક) [7
ગુણ]}\label{uxaaauxab0uxab6uxaa8-3uxa95-7-uxa97uxaa3}

\textbf{8051 નું આર્કિટેક્ચર દોરો અને સમજાવો.}

\begin{solutionbox}

\begin{verbatim}
graph TB
    subgraph "8051 Architecture"
        ALU[8{-bit ALU]}
        ACC[Accumulator A]
        B[B Register]
        PSW[Program Status Word]
        
        SP[Stack Pointer]
        DPTR[Data Pointer DPTR]
        PC[Program Counter PC]
        
        ROM[4KB ROM{br/0000{-}0FFF]}
        RAM[128B RAM{br/00{-}7F]}
        SFR[SFR Area{br/80{-}FF]}
        
        P0[Port 0]
        P1[Port 1] 
        P2[Port 2]
        P3[Port 3]
        
        T0[Timer 0]
        T1[Timer 1]
        UART[Serial Port]
        INT[Interrupt Control]
    end
    
    ALU {-{-}{-} ACC}
    ALU {-{-}{-} B}
    ALU {-{-}{-} PSW}
\end{verbatim}

\textbf{આર્કિટેક્ચર ઘટકો:}

\begin{itemize}
\tightlist
\item
  \textbf{CPU}: Accumulator અને B register સાથે 8-bit ALU
\item
  \textbf{Memory}: 4KB internal ROM, 128B internal RAM
\item
  \textbf{I/O Ports}: ચાર 8-bit bidirectional ports (P0-P3)
\item
  \textbf{Timers}: બે 16-bit timers/counters (T0, T1)
\item
  \textbf{Serial Port}: Communication માટે full duplex UART
\item
  \textbf{Interrupts}: Priority levels સાથે 5 interrupt sources
\end{itemize}

\textbf{વિશેષ લક્ષણો:}

\begin{itemize}
\tightlist
\item
  \textbf{Boolean Processor}: Bit manipulation capabilities
\item
  \textbf{Addressing Modes}: 8 અલગ addressing modes
\item
  \textbf{Power Management}: Idle અને power-down modes
\end{itemize}

\end{solutionbox}
\begin{mnemonicbox}
``MIPTIS - Memory, I/O, Processor, Timers,
Interrupts, Serial''

\end{mnemonicbox}
\subsection*{પ્રશ્ન 3(અ) OR [3
ગુણ]}\label{uxaaauxab0uxab6uxaa8-3uxa85-or-3-uxa97uxaa3}

\textbf{8051 ની નીચેની પીન સમજાવો: (1) ALE (2) PSEN (3) XTAL1 \& XTAL2}

\begin{solutionbox}

\textbf{પીન કાર્યો:}

\begin{itemize}
\tightlist
\item
  \textbf{ALE (Pin 30)}: Address Latch Enable

  \begin{itemize}
  \tightlist
  \item
    Lower address byte latch કરવા માટે output pulse
  \item
    Oscillator frequency ના 1/6 પર active high signal
  \end{itemize}
\item
  \textbf{PSEN (Pin 29)}: Program Store Enable

  \begin{itemize}
  \tightlist
  \item
    External program memory read માટે active low output
  \item
    External EPROM ના OE pin સાથે જોડાય છે
  \end{itemize}
\item
  \textbf{XTAL1 \& XTAL2 (Pins 19, 18)}: Crystal connections

  \begin{itemize}
  \tightlist
  \item
    Clock generation માટે external crystal જોડાય છે
  \item
    સામાન્ય frequency: 11.0592 MHz અથવા 12 MHz
  \end{itemize}
\end{itemize}

\begin{verbatim}
Crystal Oscillator Connection:
    
    XTAL1 {-{-}{-}{-}[Crystal]{-}{-}{-}{-} XTAL2}
      |                      |
     [C1]                   [C2]
      |                      |
     GND                    GND
\end{verbatim}

\end{solutionbox}
\begin{mnemonicbox}
``APX - ALE latches Address, PSEN enables Program,
XTAL generates clocK''

\end{mnemonicbox}
\subsection*{પ્રશ્ન 3(બ) OR [4
ગુણ]}\label{uxaaauxab0uxab6uxaa8-3uxaac-or-4-uxa97uxaa3}

\textbf{8051 માઈક્રોકંટ્રોલરનું આંતરિક RAM ઓર્ગેનાઈઝેશન સમજાવો.}

\begin{solutionbox}

\begin{verbatim}
    8051 Internal RAM Organization (128 Bytes)
    
    7FH +{-{-}{-}{-}{-}{-}{-}{-}{-}{-}{-}{-}{-}{-}{-}{-}{-}{-}{-}{-}{-}{-}{-}{-}+}
        |    General Purpose     |  
        |    Scratch Pad Area    |  78H{-7FH (8 bytes)}
    78H +{-{-}{-}{-}{-}{-}{-}{-}{-}{-}{-}{-}{-}{-}{-}{-}{-}{-}{-}{-}{-}{-}{-}{-}+}
        |                        |
        |    General Purpose     |  30H{-77H (72 bytes)  }
        |    Data Memory         |
    30H +{-{-}{-}{-}{-}{-}{-}{-}{-}{-}{-}{-}{-}{-}{-}{-}{-}{-}{-}{-}{-}{-}{-}{-}+}
        |  Bank 3 (R0{-R7)        |  18H{-}1FH}
    20H +{-{-}{-}{-}{-}{-}{-}{-}{-}{-}{-}{-}{-}{-}{-}{-}{-}{-}{-}{-}{-}{-}{-}{-}+}
        |  Bank 2 (R0{-R7)        |  10H{-}17H  }
    18H +{-{-}{-}{-}{-}{-}{-}{-}{-}{-}{-}{-}{-}{-}{-}{-}{-}{-}{-}{-}{-}{-}{-}{-}+}
        |  Bank 1 (R0{-R7)        |  08H{-}0FH}
    10H +{-{-}{-}{-}{-}{-}{-}{-}{-}{-}{-}{-}{-}{-}{-}{-}{-}{-}{-}{-}{-}{-}{-}{-}+}
        |  Bank 0 (R0{-R7)        |  00H{-}07H}
    08H +{-{-}{-}{-}{-}{-}{-}{-}{-}{-}{-}{-}{-}{-}{-}{-}{-}{-}{-}{-}{-}{-}{-}{-}+}
        |  Default Register Bank |
    00H +{-{-}{-}{-}{-}{-}{-}{-}{-}{-}{-}{-}{-}{-}{-}{-}{-}{-}{-}{-}{-}{-}{-}{-}+}
\end{verbatim}

\textbf{RAM વિભાગો:}

\begin{itemize}
\tightlist
\item
  \textbf{Register Banks}: 4 banks \times 8 registers દરેક (00H-1FH)
\item
  \textbf{Bit Addressable}: Individual bit access સાથે 16 bytes (20H-2FH)
\item
  \textbf{General Purpose}: User data માટે 80 bytes (30H-7FH)
\item
  \textbf{Stack Area}: સામાન્યતે 08H થી ઉપર શરૂ થાય છે
\end{itemize}

\textbf{Addressing:}

\begin{itemize}
\tightlist
\item
  \textbf{Direct}: વાસ્તવિક address વાપરીને (MOV 30H, A)
\item
  \textbf{Indirect}: Register pointer વાપરીને (MOV @R0, A)
\end{itemize}

\end{solutionbox}
\begin{mnemonicbox}
``RBGS - Register banks, Bit addressable, General
purpose, Stack''

\end{mnemonicbox}
\subsection*{પ્રશ્ન 3(ક) OR [7
ગુણ]}\label{uxaaauxab0uxab6uxaa8-3uxa95-or-7-uxa97uxaa3}

\textbf{8051 નો પીન ડાયાગ્રામ દોરો અને કોઈ પણ 4 પીન સમજાવો.}

\begin{solutionbox}

\begin{verbatim}
                    8051 PIN DIAGRAM
                    
    P1.0     1 +{-{-}{-}{-}{-}{-}{-}+ 40  Vcc}
    P1.1     2 |       | 39  P0.0/AD0
    P1.2     3 |       | 38  P0.1/AD1  
    P1.3     4 |       | 37  P0.2/AD2
    P1.4     5 |  8051 | 36  P0.3/AD3
    P1.5     6 |       | 35  P0.4/AD4
    P1.6     7 |       | 34  P0.5/AD5
    P1.7     8 |       | 33  P0.6/AD6
    RESET    9 |       | 32  P0.7/AD7
   P3.0/RXD  10|       | 31  EA/VPP
   P3.1/TXD  11|       | 30  ALE/PROG
   P3.2/INT0 12|       | 29  PSEN
   P3.3/INT1 13|       | 28  P2.7/A15
   P3.4/T0   14|       | 27  P2.6/A14
   P3.5/T1   15|       | 26  P2.5/A13
   P3.6/WR   16|       | 25  P2.4/A12
   P3.7/RD   17|       | 24  P2.3/A11
   XTAL2     18|       | 23  P2.2/A10
   XTAL1     19|       | 22  P2.1/A9
    VSS      20+{-{-}{-}{-}{-}{-}{-}+ 21  P2.0/A8}
\end{verbatim}

\textbf{પીન સમજાવટ:}

\begin{itemize}
\tightlist
\item
  \textbf{RESET (Pin 9)}: Reset input - Active high, microcontroller
  initialize કરે છે
\item
  \textbf{EA/VPP (Pin 31)}: External Access - Program memory selection
  control કરે છે
\item
  \textbf{P0 (Pins 32-39)}: Port 0 - External memory માટે multiplexed
  address/data bus
\item
  \textbf{P2 (Pins 21-28)}: Port 2 - External memory માટે high-order
  address bus
\end{itemize}

\end{solutionbox}
\begin{mnemonicbox}
``REPP - REset, External Access, Port 0, Port 2''

\end{mnemonicbox}
\subsection*{પ્રશ્ન 4(અ) [3
ગુણ]}\label{uxaaauxab0uxab6uxaa8-4uxa85-3-uxa97uxaa3}

\textbf{R0 રજિસ્ટરમાં સ્ટોર થયેલ ડેટાને R1 રજિસ્ટરમાં સ્ટોર થયેલ ડેટા સાથે ગુણાકાર
કરો અને પરિણામ R2 રજિસ્ટરમાં(LSB) અને R3 રજિસ્ટરમાં(MSB) સ્ટોર કરવાનો પ્રોગ્રામ
લખો.}

\begin{solutionbox}

\begin{verbatim}
ORG 0000H
MOV R0, \#05H    ; પહેલો નંબર લોડ કરો
MOV R1, \#03H    ; બીજો નંબર લોડ કરો
MOV A, R0       ; R0 ને accumulator માં મૂકો
MOV B, R1       ; R1 ને B register માં મૂકો
MUL AB          ; A અને B નો ગુણાકાર કરો
MOV R2, A       ; LSB ને R2 માં સ્ટોર કરો
MOV R3, B       ; MSB ને R3 માં સ્ટોર કરો
END
\end{verbatim}

\textbf{પ્રોગ્રામ ફ્લો:}

\begin{itemize}
\tightlist
\item
  \textbf{Operands લોડ કરો} R0 અને R1 માં
\item
  \textbf{ટ્રાન્સફર કરો} ગુણાકાર માટે A અને B registers માં
\item
  \textbf{Execute કરો} MUL AB instruction
\item
  \textbf{સ્ટોર કરો} 16-bit પરિણામ (A=LSB, B=MSB)
\end{itemize}

\textbf{પરિણામ સ્ટોરેજ:}

\begin{itemize}
\tightlist
\item
  \textbf{R2}: Product ના lower 8 bits
\item
  \textbf{R3}: Product ના upper 8 bits
\end{itemize}

\end{solutionbox}
\begin{mnemonicbox}
``LTSE - Load, Transfer, multiply, Store result''

\end{mnemonicbox}
\subsection*{પ્રશ્ન 4(બ) [4
ગુણ]}\label{uxaaauxab0uxab6uxaa8-4uxaac-4-uxa97uxaa3}

\textbf{ડેટા ટ્રાન્સફર ઇન્સ્ટ્રકશનની યાદી આપો. કોઈ પણ બે ડેટા ટ્રાન્સફર ઇન્સ્ટ્રકશન
ઉદાહરણ સહિત સમજાવો.}

\begin{solutionbox}

\textbf{ડેટા ટ્રાન્સફર ઇન્સ્ટ્રકશન:}

{\def\LTcaptype{none} % do not increment counter
\begin{longtable}[]{@{}ll@{}}
\toprule\noalign{}
Instruction & કાર્ય \\
\midrule\noalign{}
\endhead
\bottomrule\noalign{}
\endlastfoot
MOV & Registers/memory વચ્ચે data move કરે છે \\
MOVX & External memory થી data move કરે છે \\
MOVC & Code byte ને accumulator માં move કરે છે \\
PUSH & Data ને stack પર push કરે છે \\
POP & Stack માંથી data pop કરે છે \\
XCH & Accumulator સાથે register exchange કરે છે \\
XCHD & Lower nibble exchange કરે છે \\
\end{longtable}
}

\textbf{વિગતવાર ઉદાહરણો:}

\textbf{1. MOV Instruction:}

\begin{verbatim}
MOV A, \#50H     ; Immediate data 50H ને accumulator માં લોડ કરો
MOV R0, A       ; Accumulator content ને R0 માં copy કરો
MOV 30H, A      ; Accumulator content ને address 30H પર સ્ટોર કરો
\end{verbatim}

\textbf{2. PUSH/POP Instructions:}

\begin{verbatim}
PUSH ACC        ; Accumulator ને stack પર push કરો
PUSH 00H        ; R0 content ને stack પર push કરો
POP 01H         ; Stack content ને R1 માં pop કરો
POP ACC         ; Stack content ને accumulator માં pop કરો
\end{verbatim}

\end{solutionbox}
\begin{mnemonicbox}
``Move Makes Programs Possible - MOV, MOVX, PUSH,
POP''

\end{mnemonicbox}
\subsection*{પ્રશ્ન 4(ક) [7
ગુણ]}\label{uxaaauxab0uxab6uxaa8-4uxa95-7-uxa97uxaa3}

\textbf{8051 ના એડ્રેસિંગ મોડ્સને વ્યાખ્યાયિત કરો અને સમજાવો.}

\begin{solutionbox}

\textbf{8051 એડ્રેસિંગ મોડ્સ:}

{\def\LTcaptype{none} % do not increment counter
\begin{longtable}[]{@{}
  >{\raggedright\arraybackslash}p{(\linewidth - 6\tabcolsep) * \real{0.1667}}
  >{\raggedright\arraybackslash}p{(\linewidth - 6\tabcolsep) * \real{0.2667}}
  >{\raggedright\arraybackslash}p{(\linewidth - 6\tabcolsep) * \real{0.3000}}
  >{\raggedright\arraybackslash}p{(\linewidth - 6\tabcolsep) * \real{0.2667}}@{}}
\toprule\noalign{}
\begin{minipage}[b]{\linewidth}\raggedright
મોડ
\end{minipage} & \begin{minipage}[b]{\linewidth}\raggedright
વર્ણન
\end{minipage} & \begin{minipage}[b]{\linewidth}\raggedright
ઉદાહરણ
\end{minipage} & \begin{minipage}[b]{\linewidth}\raggedright
ઉપયોગ
\end{minipage} \\
\midrule\noalign{}
\endhead
\bottomrule\noalign{}
\endlastfoot
\textbf{Immediate} & Data instruction નો ભાગ છે & MOV A, \#50H & સ્થિર
મૂલ્યો \\
\textbf{Register} & Register નો સીધો ઉપયોગ & MOV A, R0 & ઝડપી access \\
\textbf{Direct} & સીધું address વાપરે છે & MOV A, 30H & RAM locations \\
\textbf{Indirect} & Register ને pointer તરીકે વાપરે છે & MOV A, @R0 &
Dynamic addressing \\
\textbf{Indexed} & Base + offset addressing & MOVC A, @A+DPTR & Table
lookup \\
\textbf{Relative} & PC + offset & SJMP LOOP & Branch instructions \\
\textbf{Absolute} & Direct jump address & LJMP 1000H & Long jumps \\
\textbf{Bit} & Individual bit access & SETB P1.0 & Control operations \\
\end{longtable}
}

\textbf{વિગતવાર ઉદાહરણો:}

\begin{verbatim}
; Immediate Addressing
MOV A, \#25H         ; 25H ને A માં લોડ કરો

; Register Addressing  
MOV A, R1           ; R1 ને A માં copy કરો

; Direct Addressing
MOV A, 40H          ; Address 40H માંથી લોડ કરો

; Indirect Addressing
MOV R0, \#40H        ; R0 40H ને point કરે છે
MOV A, @R0          ; R0 દ્વારા pointed address માંથી લોડ કરો

; Indexed Addressing
MOV DPTR, \#TABLE    ; Table ને point કરો
MOV A, \#02H         ; Index value
MOVC A, @A+DPTR     ; TABLE+2 માંથી લોડ કરો
\end{verbatim}

\end{solutionbox}
\begin{mnemonicbox}
``IRIDRAB - Immediate, Register, Indirect, Direct,
Relative, Absolute, Bit''

\end{mnemonicbox}
\subsection*{પ્રશ્ન 4(અ) OR [3
ગુણ]}\label{uxaaauxab0uxab6uxaa8-4uxa85-or-3-uxa97uxaa3}

\textbf{R0 રજિસ્ટરમાં સ્ટોર થયેલ ડેટાનું 2's Complement શોધવાનો પ્રોગ્રામ લખો.}

\begin{solutionbox}

\begin{verbatim}
ORG 0000H
MOV R0, \#85H        ; ટેસ્ટ ડેટા લોડ કરો
MOV A, R0           ; ડેટાને accumulator માં copy કરો
CPL A               ; બધા bits complement કરો (1{s complement)}
INC A               ; 2{s complement માટે 1 ઉમેરો}
MOV R1, A           ; પરિણામ R1 માં સ્ટોર કરો
END
\end{verbatim}

\textbf{Algorithm:}

\begin{itemize}
\tightlist
\item
  \textbf{Step 1}: R0 માંથી ડેટાને accumulator માં લોડ કરો
\item
  \textbf{Step 2}: CPL A વાપરીને બધા bits complement કરો
\item
  \textbf{Step 3}: 2's complement માટે INC A વાપરીને 1 ઉમેરો
\item
  \textbf{Step 4}: પરિણામ પાછું સ્ટોર કરો
\end{itemize}

\textbf{ચકાસણી:}

\begin{verbatim}
મૂળ: 85H = 10000101B
1's Comp: 7AH = 01111010B  
2's Comp: 7BH = 01111011B
\end{verbatim}

\end{solutionbox}
\begin{mnemonicbox}
``CCI - Complement, aCd 1, Include result''

\end{mnemonicbox}
\subsection*{પ્રશ્ન 4(બ) OR [4
ગુણ]}\label{uxaaauxab0uxab6uxaa8-4uxaac-or-4-uxa97uxaa3}

\textbf{લોજિકલ ઇન્સ્ટ્રકશનની યાદી આપો. કોઈ પણ બે લોજિકલ ઇન્સ્ટ્રકશન ઉદાહરણ સહિત
સમજાવો.}

\begin{solutionbox}

\textbf{લોજિકલ ઇન્સ્ટ્રકશન:}

{\def\LTcaptype{none} % do not increment counter
\begin{longtable}[]{@{}ll@{}}
\toprule\noalign{}
Instruction & કાર્ય \\
\midrule\noalign{}
\endhead
\bottomrule\noalign{}
\endlastfoot
ANL & Logical AND operation \\
ORL & Logical OR operation \\
XRL & Logical XOR operation \\
CPL & Complement operation \\
RL/RLC & Rotate left \\
RR/RRC & Rotate right \\
SWAP & Swap nibbles \\
\end{longtable}
}

\textbf{વિગતવાર ઉદાહરણો:}

\textbf{1. ANL (AND Logic):}

\begin{verbatim}
MOV A, \#0F0H        ; A = 11110000B
ANL A, \#0AAH        ; 10101010B સાથે AND કરો
; પરિણામ:

A = 10100000B = A0H

\end{verbatim}

\textbf{ઉપયોગ}: વિશિષ્ટ bits masking, અનચાહતા bits clear કરવા

\textbf{2. ORL (OR Logic):}

\begin{verbatim}
MOV A, \#0F0H        ; A = 11110000B  
ORL A, \#00FH        ; 00001111B સાથે OR કરો
; પરિણામ:

A = 11111111B = FFH

\end{verbatim}

\textbf{ઉપયોગ}: વિશિષ્ટ bits setting, bit patterns combine કરવા

\end{solutionbox}
\begin{mnemonicbox}
``AXOR - AND masks, XOR toggles, OR sets, Rotate
shifts''

\end{mnemonicbox}
\subsection*{પ્રશ્ન 4(ક) OR [7
ગુણ]}\label{uxaaauxab0uxab6uxaa8-4uxa95-or-7-uxa97uxaa3}

\textbf{નીચેની ઇન્સ્ટ્રકશન સમજાવો: (1)ADDC (2) INC (3) DEC (4) JZ (5) SUBB
(6) NOP (7) RET}

\begin{solutionbox}

\textbf{ઇન્સ્ટ્રકશન સમજાવટ:}

\textbf{1. ADDC (Add with Carry):}

\begin{verbatim}
MOV A, \#80H
ADDC A, \#90H    ; A = A + 90H + Carry flag
\end{verbatim}

\textbf{કાર્ય}: Source, destination, અને carry flag ઉમેરે છે

\textbf{2. INC (Increment):}

\begin{verbatim}
INC A           ; A = A + 1
INC R0          ; R0 = R0 + 1  
INC 30H         ; (30H) = (30H) + 1
\end{verbatim}

\textbf{કાર્ય}: Operand માં 1 વધારે છે

\textbf{3. DEC (Decrement):}

\begin{verbatim}
DEC A           ; A = A {- 1}
DEC R1          ; R1 = R1 {- 1}
DEC 40H         ; (40H) = (40H) {- 1  }
\end{verbatim}

\textbf{કાર્ય}: Operand માંથી 1 ઓછું કરે છે

\textbf{4. JZ (Jump on Zero):}

\begin{verbatim}
DEC A
JZ ZERO\_LABEL   ; A = 0 હોય તો jump કરો
\end{verbatim}

\textbf{કાર્ય}: Zero flag set હોય ત્યારે conditional jump

\textbf{5. SUBB (Subtract with Borrow):}

\begin{verbatim}
MOV A, \#50H
SUBB A, \#30H    ; A = A {- 30H {-} Carry flag}
\end{verbatim}

\textbf{કાર્ય}: Accumulator માંથી source અને carry ઓછું કરે છે

\textbf{6. NOP (No Operation):}

\begin{verbatim}
NOP             ; કંઈ ન કરો, 1 cycle વાપરો
\end{verbatim}

\textbf{કાર્ય}: Timing delay આપે છે, placeholder

\textbf{7. RET (Return):}

\begin{verbatim}
CALL SUBROUTINE
...
SUBROUTINE:
    MOV A, \#10H
    RET         ; Caller ને પાછા જાઓ
\end{verbatim}

\textbf{કાર્ય}: Subroutine માંથી calling address પર પાછા જાય છે

\end{solutionbox}
\begin{mnemonicbox}
``AIDS NR - Add, Increment, Decrement, Subtract,
No-op, Return''

\end{mnemonicbox}
\subsection*{પ્રશ્ન 5(અ) [3
ગુણ]}\label{uxaaauxab0uxab6uxaa8-5uxa85-3-uxa97uxaa3}

\textbf{DJNZ અને CJNE ઇન્સ્ટ્રકશન યોગ્ય ઉદાહરણ સહિત સમજાવો.}

\begin{solutionbox}

\textbf{DJNZ (Decrement and Jump if Not Zero):}

\begin{verbatim}
MOV R0, \#05H        ; Counter initialize કરો
LOOP:
    MOV A, \#00H     ; કોઈ operation
    DJNZ R0, LOOP   ; R0 ઓછું કરો, zero નથી તો jump કરો
\end{verbatim}

\textbf{કાર્ય}: Decrement અને conditional jump operations combine કરે છે

\textbf{CJNE (Compare and Jump if Not Equal):}

\begin{verbatim}
MOV A, \#30H
CJNE A, \#30H, NOT\_EQUAL  ; A ને 30H સાથે compare કરો
MOV R0, \#01H             ; Equal case
SJMP CONTINUE
NOT\_EQUAL:
    MOV R0, \#00H         ; Not equal case
CONTINUE:
\end{verbatim}

\textbf{કાર્ય}: બે operands compare કરે છે અને સમાન નથી તો jump કરે છે

\textbf{ઉપયોગો:}

\begin{itemize}
\tightlist
\item
  \textbf{DJNZ}: Loop control, counting operations
\item
  \textbf{CJNE}: Decision making, condition checking
\end{itemize}

\end{solutionbox}
\begin{mnemonicbox}
``DC - Decrement count, Compare jump''

\end{mnemonicbox}
\subsection*{પ્રશ્ન 5(બ) [4
ગુણ]}\label{uxaaauxab0uxab6uxaa8-5uxaac-4-uxa97uxaa3}

\textbf{ટાઈમર 0 નો ઉપયોગ કરી 30 મિલી સેકંડનો ટાઈમ ડિલે જનરેટ કરવા માટે એસેમ્બલી
પ્રોગ્રામ બનાવો. ક્રિસ્ટલ ફ્રિકવન્સી 12 મેગા હર્ટઝ ગણતરીમાં લેવી.}

\begin{solutionbox}

\begin{verbatim}
ORG 0000H
MAIN:
    CALL DELAY\_30MS     ; 30ms delay call કરો
    SJMP MAIN           ; Repeat કરો

DELAY\_30MS:
    MOV TMOD, \#01H      ; Timer 0, Mode 1 (16{-bit)}
    MOV TH0, \#8AH       ; 30ms માટે high byte લોડ કરો
    MOV TL0, \#23H       ; Low byte લોડ કરો
    SETB TR0            ; Timer 0 start કરો
WAIT:
    JNB TF0, WAIT       ; Timer overflow માટે રાહ જુઓ
    CLR TR0             ; Timer stop કરો
    CLR TF0             ; Timer flag clear કરો
    RET
END
\end{verbatim}

\textbf{30ms delay માટે ગણતરી:}

\begin{verbatim}
Crystal Frequency = 12 MHz
Machine Cycle = 12/12 MHz = 1 µs
30ms માટે = 30,000 µs = 30,000 machine cycles

Timer Count = 65536 - 30000 = 35536 = 8A23H
TH0 = 8AH, TL0 = 23H
\end{verbatim}

\textbf{Timer Configuration:}

\begin{itemize}
\tightlist
\item
  \textbf{TMOD}: Timer mode register configuration
\item
  \textbf{TH0/TL0}: Timer 0 high/low byte registers
\item
  \textbf{TR0}: Timer 0 run control bit
\item
  \textbf{TF0}: Timer 0 overflow flag
\end{itemize}

\end{solutionbox}
\begin{mnemonicbox}
``CLSW - Calculate, Load, Start, Wait''

\end{mnemonicbox}
\subsection*{પ્રશ્ન 5(ક) [7
ગુણ]}\label{uxaaauxab0uxab6uxaa8-5uxa95-7-uxa97uxaa3}

\textbf{8051 માઈક્રોકંટ્રોલર સાથે LCD નો ઇન્ટરફેસિંગ ડાયાગ્રામ દોરો અને ઇન્ટરફેસિંગ
માટે જરૂરી LCD ની તમામ પીનો સમજાવો.}

\begin{solutionbox}

\begin{verbatim}
    8051 to LCD Interfacing (4{-bit mode)}
    
    8051                      16x2 LCD
    {-{-}{-}{-}                      {-}{-}{-}{-}{-}{-}{-}{-}}
    P2.7 {-{-}{-}{-}{-}{-}{-}{-}{-} D7  (Pin 14)}
    P2.6 {-{-}{-}{-}{-}{-}{-}{-}{-} D6  (Pin 13)  }
    P2.5 {-{-}{-}{-}{-}{-}{-}{-}{-} D5  (Pin 12)}
    P2.4 {-{-}{-}{-}{-}{-}{-}{-}{-} D4  (Pin 11)}
    
    P1.2 {-{-}{-}{-}{-}{-}{-}{-}{-} EN  (Pin 6)}
    P1.1 {-{-}{-}{-}{-}{-}{-}{-}{-} RW  (Pin 5)}
    P1.0 {-{-}{-}{-}{-}{-}{-}{-}{-} RS  (Pin 4)}
    
    +5V  {-{-}{-}{-}{-}{-}{-}{-}{-} VCC (Pin 2)}
    GND  {-{-}{-}{-}{-}{-}{-}{-}{-} VSS (Pin 1)}
    GND  {-{-}{-}{-}{-}{-}{-}{-}{-} VEE (Pin 3) [Contrast]}
\end{verbatim}

\textbf{LCD પીન કાર્યો:}

\begin{itemize}
\tightlist
\item
  \textbf{RS (Pin 4)}: Register Select - 0=Command, 1=Data
\item
  \textbf{RW (Pin 5)}: Read/Write - 0=Write, 1=Read\\
\item
  \textbf{EN (Pin 6)}: Enable - Data transfer માટે high to low pulse
\item
  \textbf{D4-D7 (Pins 11-14)}: Commands/data માટે 4-bit data lines
\end{itemize}

\textbf{ઇન્ટરફેસ જરૂરિયાતો:}

\begin{itemize}
\tightlist
\item
  \textbf{Power Supply}: VCC=+5V, VSS=GND, VEE=Contrast control
\item
  \textbf{Control Lines}: LCD control માટે 3 pins (RS, RW, EN)
\item
  \textbf{Data Lines}: 4-bit mode operation માટે 4 pins (D4-D7)
\end{itemize}

\textbf{મૂળભૂત LCD Commands:}

\begin{itemize}
\tightlist
\item
  \textbf{0x38}: Function set (8-bit, 2 lines)
\item
  \textbf{0x0E}: Display ON, cursor ON
\item
  \textbf{0x01}: Clear display
\item
  \textbf{0x80}: Set cursor to first line
\end{itemize}

\end{solutionbox}
\begin{mnemonicbox}
``REED - RS selects, RW reads, EN enables, Data
transfers''

\end{mnemonicbox}
\subsection*{પ્રશ્ન 5(અ) OR [3
ગુણ]}\label{uxaaauxab0uxab6uxaa8-5uxa85-or-3-uxa97uxaa3}

\textbf{65h મેમરી લોકેશન પર સ્ટોર થયેલ ડેટાનું 75h મેમરી લોકેશન પર સ્ટોર થયેલ ડેટા
સાથે OR ઓપરેશન કરો અને પરિણામ R6 રજિસ્ટરમાં સ્ટોર કરવાનો પ્રોગ્રામ લખો.}

\begin{solutionbox}

\begin{verbatim}
ORG 0000H
MOV 65H, \#0F0H      ; 65H પર ટેસ્ટ ડેટા સ્ટોર કરો
MOV 75H, \#0AAH      ; 75H પર ટેસ્ટ ડેટા સ્ટોર કરો

MOV A, 65H          ; 65H માંથી ડેટાને accumulator માં લોડ કરો
ORL A, 75H          ; 75H પરના ડેટા સાથે OR કરો
MOV R6, A           ; પરિણામ R6 register માં સ્ટોર કરો
END
\end{verbatim}

\textbf{ઓપરેશન વિગતો:}

\begin{itemize}
\tightlist
\item
  \textbf{Load}: Memory location 65H માંથી પહેલો operand
\item
  \textbf{OR}: 75H પરના બીજા operand સાથે logical OR કરો
\item
  \textbf{Store}: પરિણામ R6 register માં
\end{itemize}

\textbf{ઉદાહરણ ગણતરી:}

\begin{verbatim}
65H પરનો ડેટા: F0H = 11110000B
75H પરનો ડેટા: AAH = 10101010B  
OR પરિણામ:   FAH = 11111010B
\end{verbatim}

\end{solutionbox}
\begin{mnemonicbox}
``LOS - Load, OR, Store result''

\end{mnemonicbox}
\subsection*{પ્રશ્ન 5(બ) OR [4
ગુણ]}\label{uxaaauxab0uxab6uxaa8-5uxaac-or-4-uxa97uxaa3}

\textbf{P1.3 પર 2 કિલો હર્ટઝનો સ્કવેર વેવ જનરેટ કરવા માટે એસેમ્બલી પ્રોગ્રામ લખો.
ક્રિસ્ટલ ફ્રિકવન્સી 11.0592 મેગા હર્ટઝ ગણતરીમાં લેવી.}

\begin{solutionbox}

\begin{verbatim}
ORG 0000H
MAIN:
    SETB P1.3           ; P1.3 ને high કરો
    CALL DELAY\_250US    ; અડધા period માટે delay
    CLR P1.3            ; P1.3 ને low કરો
    CALL DELAY\_250US    ; અડધા period માટે delay
    SJMP MAIN           ; સતત repeat કરો

DELAY\_250US:
    MOV TMOD, \#01H      ; Timer 0, Mode 1
    MOV TH0, \#0FEH      ; High byte લોડ કરો
    MOV TL0, \#0CBH      ; Low byte લોડ કરો
    SETB TR0            ; Timer 0 start કરો
WAIT:
    JNB TF0, WAIT       ; Overflow માટે રાહ જુઓ
    CLR TR0             ; Timer stop કરો
    CLR TF0             ; Flag clear કરો
    RET
END
\end{verbatim}

\textbf{2KHz Square Wave માટે ગણતરી:}

\begin{verbatim}
Frequency = 2KHz, Period = 500µs
Half Period = 250µs

Crystal = 11.0592 MHz
Machine Cycle = 11.0592/12 = 0.921 MHz = 1.085µs

Timer Count = 250µs / 1.085µs = 230 cycles
Timer Value = 65536 - 230 = 65306 = FECBH
TH0 = FEH, TL0 = CBH
\end{verbatim}

\textbf{Square Wave Generation:}

\begin{itemize}
\tightlist
\item
  \textbf{High Period}: Pin high કરો, 250µs રાહ જુઓ
\item
  \textbf{Low Period}: Pin low કરો, 250µs રાહ જુઓ
\item
  \textbf{Frequency}: 1/(250µs + 250µs) = 2KHz
\end{itemize}

\end{solutionbox}
\begin{mnemonicbox}
``SCDW - Set high, Clear low, Delay, Wait''

\end{mnemonicbox}
\subsection*{પ્રશ્ન 5(ક) OR [7
ગુણ]}\label{uxaaauxab0uxab6uxaa8-5uxa95-or-7-uxa97uxaa3}

\textbf{8051 માઈક્રોકંટ્રોલર સાથે 7-Segment ડિસ્પ્લેનો ઇન્ટરફેસિંગ દોરો અને
સમજાવો.}

\begin{solutionbox}

\begin{verbatim}
    8051 to 7{-Segment Display Interfacing}
    
    8051 Port 1              7{-Segment Display}
    {-{-}{-}{-}{-}{-}{-}{-}{-}{-}{-}              {-}{-}{-}{-}{-}{-}{-}{-}{-}{-}{-}{-}{-}{-}{-}{-}{-}}
    P1.0 {-{-}{-}{-}[R]{-}{-}{-}{-} a  (Pin 7)     }
    P1.1 {-{-}{-}{-}[R]{-}{-}{-}{-} b  (Pin 6)      aaaa}
    P1.2 {-{-}{-}{-}[R]{-}{-}{-}{-} c  (Pin 4)     f    b}
    P1.3 {-{-}{-}{-}[R]{-}{-}{-}{-} d  (Pin 2)     f    b}
    P1.4 {-{-}{-}{-}[R]{-}{-}{-}{-} e  (Pin 1)      gggg}
    P1.5 {-{-}{-}{-}[R]{-}{-}{-}{-} f  (Pin 9)     e    c}
    P1.6 {-{-}{-}{-}[R]{-}{-}{-}{-} g  (Pin 10)    e    c}
    P1.7 {-{-}{-}{-}[R]{-}{-}{-}{-} dp (Pin 5)      dddd dp}
    
    [R] = Current limiting resistor (330Ω)
    
    For Common Cathode:
    Common pin (Pin 3,8) {-{-}{-} GND}
    
    For Common Anode:  
    Common pin (Pin 3,8) {-{-}{-} +5V}
\end{verbatim}

\textbf{ડિસ્પ્લે કોન્ફિગરેશન:}

{\def\LTcaptype{none} % do not increment counter
\begin{longtable}[]{@{}lll@{}}
\toprule\noalign{}
અક્ષર & Common Cathode કોડ & Common Anode કોડ \\
\midrule\noalign{}
\endhead
\bottomrule\noalign{}
\endlastfoot
0 & 3FH & C0H \\
1 & 06H & F9H \\
2 & 5BH & A4H \\
3 & 4FH & B0H \\
4 & 66H & 99H \\
5 & 6DH & 92H \\
6 & 7DH & 82H \\
7 & 07H & F8H \\
8 & 7FH & 80H \\
9 & 6FH & 90H \\
\end{longtable}
}

\textbf{નમૂના પ્રોગ્રામ:}

\begin{verbatim}
ORG 0000H
MOV DPTR, \#DIGIT\_TABLE  ; Lookup table ને point કરો
MOV A, \#05H             ; અંક 5 દર્શાવો
MOVC A, @A+DPTR         ; 7{-segment કોડ મેળવો}
MOV P1, A               ; Display ને મોકલો
SJMP $                  ; અહીં રહો

DIGIT\_TABLE:
DB 3FH, 06H, 5BH, 4FH, 66H  ; 0,1,2,3,4
DB 6DH, 7DH, 07H, 7FH, 6FH  ; 5,6,7,8,9
END
\end{verbatim}

\textbf{ઇન્ટરફેસ ઘટકો:}

\begin{itemize}
\tightlist
\item
  \textbf{Current Limiting Resistors}: LED current મર્યાદિત કરવા 330Ω
\item
  \textbf{Common Connection}: GND ને cathode અથવા +5V ને anode
\item
  \textbf{Data Lines}: Segments a-g અને decimal point માટે 8 bits
\end{itemize}

\textbf{મલ્ટિપલ ડિજિટ્સ માટે Multiplexing:}

\begin{itemize}
\tightlist
\item
  \textbf{Digit Select}: Digit selection માટે વધારાના pins
\item
  \textbf{Time Division}: Digits વચ્ચે ઝડપથી switch કરવું
\item
  \textbf{Persistence of Vision}: એકસાથે display નો ભ્રમ બનાવે છે
\end{itemize}

\end{solutionbox}
\begin{mnemonicbox}
``CRAM - Common connection, Resistors limit, Address
segments, Multiplex digits''

\end{mnemonicbox}

\end{document}
