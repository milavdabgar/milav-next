\documentclass[10pt,a4paper]{article}

% content/resources/templates/preamble.tex
\usepackage[margin=0.6in]{geometry}
\author{Milav Dabgar}
\usepackage{amsmath,amssymb,amsthm}
\usepackage{booktabs}
\usepackage{multirow}
\usepackage{xcolor}
\usepackage{tcolorbox}
\tcbuselibrary{breakable,skins}
\usepackage[colorlinks=true,linkcolor=blue]{hyperref}
\usepackage{titlesec}
\usepackage{enumitem}
\usepackage{tikz}
\usepackage{pgfplots}
\usepackage{circuitikz}
\usepackage[version=4]{mhchem}
\usepackage{longtable}
\usepackage{array}
\usepackage{float}
\usepackage{caption}
\usepackage{listings}

\lstset{
  basicstyle=\small\ttfamily,
  breaklines=true,
  breakatwhitespace=false,
  postbreak=\mbox{\textcolor{red}{$\hookrightarrow$}\space},
  float=false,
  numbers=left,
  numberstyle=\tiny\color{gray},
  numbersep=10pt,
  xleftmargin=2em,
  keywordstyle=\color{blue},
  commentstyle=\color{green!60!black},
  stringstyle=\color{purple},
  backgroundcolor=\color{gray!5},
  showstringspaces=false,
  tabsize=2,
  captionpos=b,
  keepspaces=true,
  columns=flexible
}

\pgfplotsset{compat=1.18}
\usetikzlibrary{shapes,arrows,positioning,calc,patterns,decorations.pathmorphing,decorations.markings,arrows.meta}

% Color scheme
\definecolor{headcolor}{RGB}{0,102,204}
\definecolor{keycolor}{RGB}{220,20,60}
\definecolor{solutioncolor}{RGB}{34,139,34}
\definecolor{mnemoniccolor}{RGB}{148,0,211}
\definecolor{codecolor}{RGB}{0,0,100}

% Spacing
\setlength{\parskip}{3pt}
\setlist[itemize]{nosep}
\setlist[enumerate]{nosep}

% Title formatting
\titleformat{\section}{\Large\bfseries\color{headcolor}}{\thesection}{1em}{}
\titleformat{\subsection}{\large\bfseries\color{headcolor}}{\thesubsection}{1em}{}

% Pandoc tightlist compatibility
\providecommand{\tightlist}{%
  \setlength{\itemsep}{0pt}\setlength{\parskip}{0pt}}

% Pandoc longtable compatibility
\newcounter{none}
\def\thenone{}


% content/resources/templates/english-boxes.tex
% This file is currently empty - it exists to maintain consistency with the import structure.
% Add custom environments here if needed in the future.


\begin{document}

\begin{center}
{\Huge\bfseries\color{headcolor} Subject Name Solutions}\\[5pt]
{\LARGE 1333202 -- Winter 2024}\\[3pt]
{\large Semester 1 Study Material}\\[3pt]
{\normalsize\textit{Detailed Solutions and Explanations}}
\end{center}

\vspace{10pt}

\subsection*{Question 1(a) [3 marks]}\label{q1a}

\textbf{List the features of 8051 Microcontroller.}

\begin{solutionbox}

The 8051 microcontroller has several important features:

{\def\LTcaptype{none} % do not increment counter
\begin{longtable}[]{@{}ll@{}}
\toprule\noalign{}
Feature & Description \\
\midrule\noalign{}
\endhead
\bottomrule\noalign{}
\endlastfoot
\textbf{CPU} & 8-bit CPU optimized for control applications \\
\textbf{Memory} & 4KB internal ROM, 128 bytes internal RAM \\
\textbf{I/O Ports} & 4 bidirectional 8-bit I/O ports (P0-P3) \\
\textbf{Timers} & Two 16-bit timer/counters (Timer 0 \& Timer 1) \\
\textbf{Interrupts} & 5 interrupt sources with 2 priority levels \\
\textbf{Serial Port} & Full duplex UART for serial communication \\
\end{longtable}
}

\end{solutionbox}
\begin{mnemonicbox}
``CPU Memory Input-Output Timers Interrupts Serial''
(C-MIT-IS)

\end{mnemonicbox}
\subsection*{Question 1(b) [4 marks]}\label{q1b}

\textbf{Define: Opcode, Operand, Instruction cycle, Machine cycle}

\begin{solutionbox}

{\def\LTcaptype{none} % do not increment counter
\begin{longtable}[]{@{}
  >{\raggedright\arraybackslash}p{(\linewidth - 2\tabcolsep) * \real{0.3333}}
  >{\raggedright\arraybackslash}p{(\linewidth - 2\tabcolsep) * \real{0.6667}}@{}}
\toprule\noalign{}
\begin{minipage}[b]{\linewidth}\raggedright
Term
\end{minipage} & \begin{minipage}[b]{\linewidth}\raggedright
Definition
\end{minipage} \\
\midrule\noalign{}
\endhead
\bottomrule\noalign{}
\endlastfoot
\textbf{Opcode} & Operation code that specifies the operation to be
performed \\
\textbf{Operand} & Data or address on which the operation is
performed \\
\textbf{Instruction Cycle} & Complete process of fetching, decoding and
executing an instruction \\
\textbf{Machine Cycle} & Time required to access memory or I/O device \\
\end{longtable}
}

\textbf{Diagram:}

\includegraphics[width=1\linewidth,height=\textheight,keepaspectratio]{mermaid-362e8fc2.pdf}

\end{solutionbox}
\begin{mnemonicbox}
``OOID'' - Opcode Operand Instruction-cycle
Data-cycle

\end{mnemonicbox}
\subsection*{Question 1(c) [7 marks]}\label{q1c}

\textbf{Compare Von Neumann and Harvard Architecture.}

\begin{solutionbox}

{\def\LTcaptype{none} % do not increment counter
\begin{longtable}[]{@{}
  >{\raggedright\arraybackslash}p{(\linewidth - 4\tabcolsep) * \real{0.3333}}
  >{\raggedright\arraybackslash}p{(\linewidth - 4\tabcolsep) * \real{0.3939}}
  >{\raggedright\arraybackslash}p{(\linewidth - 4\tabcolsep) * \real{0.2727}}@{}}
\toprule\noalign{}
\begin{minipage}[b]{\linewidth}\raggedright
Parameter
\end{minipage} & \begin{minipage}[b]{\linewidth}\raggedright
Von Neumann
\end{minipage} & \begin{minipage}[b]{\linewidth}\raggedright
Harvard
\end{minipage} \\
\midrule\noalign{}
\endhead
\bottomrule\noalign{}
\endlastfoot
\textbf{Memory Structure} & Single memory for program and data &
Separate memory for program and data \\
\textbf{Bus System} & Single bus system & Separate bus for program and
data \\
\textbf{Speed} & Slower due to bus conflicts & Faster simultaneous
access \\
\textbf{Cost} & Lower cost & Higher cost \\
\textbf{Complexity} & Simple design & Complex design \\
\textbf{Examples} & 8085, x86 processors & 8051, DSP processors \\
\end{longtable}
}

\textbf{Diagram:}

\includegraphics[width=1\linewidth,height=\textheight,keepaspectratio]{mermaid-23aa833f.pdf}

\end{solutionbox}
\begin{mnemonicbox}
``VSBSC vs HSDFC'' (Von-Single-Bus-Simple-Cheap vs
Harvard-Separate-Dual-Fast-Complex)

\end{mnemonicbox}
\subsection*{Question 1(c) OR [7
marks]}\label{q1c}

\textbf{Compare RISC and CISC.}

\begin{solutionbox}

{\def\LTcaptype{none} % do not increment counter
\begin{longtable}[]{@{}
  >{\raggedright\arraybackslash}p{(\linewidth - 4\tabcolsep) * \real{0.4783}}
  >{\raggedright\arraybackslash}p{(\linewidth - 4\tabcolsep) * \real{0.2609}}
  >{\raggedright\arraybackslash}p{(\linewidth - 4\tabcolsep) * \real{0.2609}}@{}}
\toprule\noalign{}
\begin{minipage}[b]{\linewidth}\raggedright
Parameter
\end{minipage} & \begin{minipage}[b]{\linewidth}\raggedright
RISC
\end{minipage} & \begin{minipage}[b]{\linewidth}\raggedright
CISC
\end{minipage} \\
\midrule\noalign{}
\endhead
\bottomrule\noalign{}
\endlastfoot
\textbf{Instruction Set} & Reduced, simple instructions & Complex
instruction set \\
\textbf{Instruction Size} & Fixed size instructions & Variable size
instructions \\
\textbf{Execution Time} & Single clock cycle per instruction & Multiple
clock cycles \\
\textbf{Memory Access} & Load/Store architecture & Memory-to-memory
operations \\
\textbf{Compiler} & Complex compiler required & Simple compiler \\
\textbf{Examples} & ARM, MIPS & 8085, x86 \\
\end{longtable}
}

\textbf{Diagram:}

\includegraphics[width=1\linewidth,height=\textheight,keepaspectratio]{mermaid-f8b1ac17.pdf}

\end{solutionbox}
\begin{mnemonicbox}
``RISC-SFS vs CISC-CSS'' (Simple-Fast-Complex vs
Complex-Slow-Simple)

\end{mnemonicbox}
\subsection*{Question 2(a) [3 marks]}\label{q2a}

\textbf{List the 16-bit Registers available in 8085 and Explain its
Function.}

\begin{solutionbox}

{\def\LTcaptype{none} % do not increment counter
\begin{longtable}[]{@{}ll@{}}
\toprule\noalign{}
Register & Function \\
\midrule\noalign{}
\endhead
\bottomrule\noalign{}
\endlastfoot
\textbf{PC (Program Counter)} & Points to next instruction address \\
\textbf{SP (Stack Pointer)} & Points to top of stack in memory \\
\textbf{BC, DE, HL} & General purpose register pairs for data storage \\
\end{longtable}
}

\begin{itemize}
\tightlist
\item
  \textbf{PC}: Automatically increments after each instruction fetch
\item
  \textbf{SP}: Decrements during PUSH, increments during POP
  operations\\
\item
  \textbf{Register Pairs}: Can store 16-bit addresses or data
\end{itemize}

\end{solutionbox}
\begin{mnemonicbox}
``PC SP BDH'' (Program-Counter Stack-Pointer
BC-DE-HL)

\end{mnemonicbox}
\subsection*{Question 2(b) [4 marks]}\label{q2b}

\textbf{Explain Address and Data Bus De-multiplexing in 8085.}

\begin{solutionbox}

De-multiplexing separates address and data signals from AD0-AD7 pins.

\textbf{Process:}

\begin{itemize}
\tightlist
\item
  \textbf{ALE (Address Latch Enable)} signal controls the process
\item
  During \textbf{T1 state}: AD0-AD7 contains lower 8-bit address
\item
  \textbf{ALE goes HIGH}: Address is latched in external latch (74LS373)
\item
  During \textbf{T2-T3}: AD0-AD7 becomes data bus
\end{itemize}

\textbf{Diagram:}

\begin{lstlisting}
     +-------+    ALE   +--------+
AD0-7|  8085 |--------->| 74LS373|----> A0-A7
     |       |          | Latch  |
     +-------+          +--------+
         |
         +---> D0-D7 (Data Bus)
\end{lstlisting}

\end{solutionbox}
\begin{mnemonicbox}
``ALE Latches Address Low''

\end{mnemonicbox}
\subsection*{Question 2(c) [7 marks]}\label{q2c}

\textbf{Explain Pin Diagram of 8085 with neat sketch.}

\begin{solutionbox}

The 8085 is a 40-pin microprocessor with the following pin
configuration:

{\def\LTcaptype{none} % do not increment counter
\begin{longtable}[]{@{}ll@{}}
\toprule\noalign{}
Pin Group & Function \\
\midrule\noalign{}
\endhead
\bottomrule\noalign{}
\endlastfoot
\textbf{AD0-AD7} & Multiplexed Address/Data bus (Lower 8-bit) \\
\textbf{A8-A15} & Higher order Address bus \\
\textbf{ALE} & Address Latch Enable signal \\
\textbf{RD, WR} & Read and Write control signals \\
\textbf{IO/M} & I/O or Memory operation indicator \\
\textbf{S0, S1} & Status signals \\
\end{longtable}
}

\textbf{Pin Diagram:}

\begin{lstlisting}
        +---\\_/---+
   X1 --|1      40|-- Vcc
   X2 --|2      39|-- HOLD  
RESET --|3      38|-- HLDA
  SOD --|4      37|-- CLK
  SID --|5 8085 36|-- RESET IN
 TRAP --|6      35|-- READY
RST7.5--|7      34|-- IO/M
RST6.5--|8      33|-- S1
RST5.5--|9      32|-- RD
 INTR --|10     31|-- WR
 INTA --|11     30|-- ALE
  AD0 --|12     29|-- S0
  AD1 --|13     28|-- A15
  AD2 --|14     27|-- A14
  AD3 --|15     26|-- A13
  AD4 --|16     25|-- A12
  AD5 --|17     24|-- A11
  AD6 --|18     23|-- A10
  AD7 --|19     22|-- A9
  Vss --|20     21|-- A8
        +---------+
\end{lstlisting}

\textbf{Key Features:}

\begin{itemize}
\tightlist
\item
  \textbf{40-pin DIP package}
\item
  \textbf{Multiplexed bus} reduces pin count
\item
  \textbf{Control signals} for timing and operation
\item
  \textbf{Interrupt pins} for external device communication
\end{itemize}

\end{solutionbox}
\begin{mnemonicbox}
``Address Data Control Power Interrupt'' (ADCPI)

\end{mnemonicbox}
\subsection*{Question 2(a) OR [3
marks]}\label{q2a}

\textbf{Explain Instruction Fetching Operation in 8085.}

\begin{solutionbox}

Instruction fetching is the first step in instruction cycle:

\textbf{Steps:}

\begin{enumerate}
\tightlist
\item
  \textbf{PC contents} placed on address bus (A0-A15)
\item
  \textbf{ALE signal} goes high to latch address
\item
  \textbf{RD signal} goes low to read memory
\item
  \textbf{Instruction} fetched from memory to data bus
\item
  \textbf{PC incremented} to point to next instruction
\end{enumerate}

\textbf{Timing:}

\begin{itemize}
\tightlist
\item
  Occurs during \textbf{T1 and T2} states of machine cycle
\item
  Takes \textbf{4 clock cycles} for simple instructions
\end{itemize}

\end{solutionbox}
\begin{mnemonicbox}
``PC ALE RD Fetch Increment'' (PARFI)

\end{mnemonicbox}
\subsection*{Question 2(b) OR [4
marks]}\label{q2b}

\textbf{Explain Flag Register of 8085.}

\begin{solutionbox}

The Flag Register stores status information after arithmetic/logical
operations:

{\def\LTcaptype{none} % do not increment counter
\begin{longtable}[]{@{}lll@{}}
\toprule\noalign{}
Bit & Flag & Function \\
\midrule\noalign{}
\endhead
\bottomrule\noalign{}
\endlastfoot
\textbf{D7} & \textbf{S (Sign)} & Set if result is negative \\
\textbf{D6} & \textbf{Z (Zero)} & Set if result is zero \\
\textbf{D5} & \textbf{-} & Not used \\
\textbf{D4} & \textbf{AC (Auxiliary Carry)} & Set if carry from bit 3 to
4 \\
\textbf{D3} & \textbf{-} & Not used \\
\textbf{D2} & \textbf{P (Parity)} & Set if result has even parity \\
\textbf{D1} & \textbf{-} & Not used \\
\textbf{D0} & \textbf{CY (Carry)} & Set if carry/borrow generated \\
\end{longtable}
}

\textbf{Diagram:}

\begin{lstlisting}
D7  D6  D5  D4  D3  D2  D1  D0
+---+---+---+---+---+---+---+---+
| S | Z | X | AC| X | P | X |CY |
+---+---+---+---+---+---+---+---+
\end{lstlisting}

\end{solutionbox}
\begin{mnemonicbox}
``S-Z-X-AC-X-P-X-CY''

\end{mnemonicbox}
\subsection*{Question 2(c) OR [7
marks]}\label{q2c}

\textbf{Explain Architecture of 8085 with neat sketch.}

\begin{solutionbox}

The 8085 architecture consists of several functional blocks:

\textbf{Major Components:}

\begin{itemize}
\tightlist
\item
  \textbf{ALU (Arithmetic Logic Unit)}: Performs arithmetic and logical
  operations
\item
  \textbf{Registers}: Store data and addresses temporarily
\item
  \textbf{Control Unit}: Generates control signals for operation
\item
  \textbf{Address/Data Bus}: Communicates with external devices
\end{itemize}

\textbf{Block Diagram:}

\includegraphics[width=1\linewidth,height=\textheight,keepaspectratio]{mermaid-a92886d0.pdf}

\textbf{Key Features:}

\begin{itemize}
\tightlist
\item
  \textbf{8-bit microprocessor} with 16-bit address bus
\item
  \textbf{Von Neumann architecture} with shared bus
\item
  \textbf{Register-based operations} for faster execution
\item
  \textbf{Interrupt capability} for real-time applications
\end{itemize}

\end{solutionbox}
\begin{mnemonicbox}
``ALU Registers Control Address Data'' (ARCAD)

\end{mnemonicbox}
\subsection*{Question 3(a) [3 marks]}\label{q3a}

\textbf{Explain Internal RAM Organization of 8051 Microcontroller.}

\begin{solutionbox}

The 8051 has 128 bytes of internal RAM organized as:

{\def\LTcaptype{none} % do not increment counter
\begin{longtable}[]{@{}ll@{}}
\toprule\noalign{}
Address Range & Purpose \\
\midrule\noalign{}
\endhead
\bottomrule\noalign{}
\endlastfoot
\textbf{00H-1FH} & Register Banks (4 banks of 8 registers each) \\
\textbf{20H-2FH} & Bit Addressable Area (16 bytes) \\
\textbf{30H-7FH} & General Purpose RAM (80 bytes) \\
\end{longtable}
}

\textbf{Organization:}

\begin{itemize}
\tightlist
\item
  \textbf{Bank 0}: 00H-07H (Default register bank)
\item
  \textbf{Bank 1}: 08H-0FH
\item
  \textbf{Bank 2}: 10H-17H
\item
  \textbf{Bank 3}: 18H-1FH
\end{itemize}

\textbf{Diagram:}

\begin{lstlisting}
7FH +----------------+
    | General Purpose|
    |      RAM       |
30H +----------------+
    | Bit Addressable|
    |     Area       |
20H +----------------+
    |  Register      |
    |   Banks        |
00H +----------------+
\end{lstlisting}

\end{solutionbox}
\begin{mnemonicbox}
``Register Bit General'' (RBG)

\end{mnemonicbox}
\subsection*{Question 3(b) [4 marks]}\label{q3b}

\textbf{Explain Function of Each bit of TMOD SFR of 8051
Microcontroller.}

\begin{solutionbox}

TMOD (Timer Mode) register controls the operation of Timer 0 and Timer
1:

{\def\LTcaptype{none} % do not increment counter
\begin{longtable}[]{@{}lll@{}}
\toprule\noalign{}
Bit & Name & Function \\
\midrule\noalign{}
\endhead
\bottomrule\noalign{}
\endlastfoot
\textbf{D7} & \textbf{GATE1} & Timer 1 gate control \\
\textbf{D6} & \textbf{C/T1} & Timer/Counter select for Timer 1 \\
\textbf{D5} & \textbf{M11} & Mode bit 1 for Timer 1 \\
\textbf{D4} & \textbf{M01} & Mode bit 0 for Timer 1 \\
\textbf{D3} & \textbf{GATE0} & Timer 0 gate control \\
\textbf{D2} & \textbf{C/T0} & Timer/Counter select for Timer 0 \\
\textbf{D1} & \textbf{M10} & Mode bit 1 for Timer 0 \\
\textbf{D0} & \textbf{M00} & Mode bit 0 for Timer 0 \\
\end{longtable}
}

\textbf{Bit Functions:}

\begin{itemize}
\tightlist
\item
  \textbf{GATE}: 1 = External gate control, 0 = Internal control
\item
  \textbf{C/T}: 1 = Counter mode, 0 = Timer mode
\item
  \textbf{M1,M0}: Timer operating modes (00=Mode0, 01=Mode1, 10=Mode2,
  11=Mode3)
\end{itemize}

\end{solutionbox}
\begin{mnemonicbox}
``GATE C/T Mode1 Mode0'' for each timer

\end{mnemonicbox}
\subsection*{Question 3(c) [7 marks]}\label{q3c}

\textbf{Explain Architecture of 8051 with neat sketch.}

\begin{solutionbox}

The 8051 microcontroller has Harvard architecture with separate program
and data memory:

\textbf{Key Components:}

\begin{itemize}
\tightlist
\item
  \textbf{8-bit CPU} with Boolean processor
\item
  \textbf{Internal ROM}: 4KB program memory
\item
  \textbf{Internal RAM}: 128 bytes data memory
\item
  \textbf{Four I/O Ports}: P0, P1, P2, P3 (8-bit each)
\item
  \textbf{Two Timers}: 16-bit Timer/Counter 0 and 1
\item
  \textbf{Serial Port}: Full duplex UART
\end{itemize}

\textbf{Architecture Diagram:}

\includegraphics[width=1\linewidth,height=\textheight,keepaspectratio]{mermaid-20549950.pdf}

\textbf{Special Features:}

\begin{itemize}
\tightlist
\item
  \textbf{Harvard Architecture}: Separate buses for program and data
\item
  \textbf{SFR (Special Function Registers)}: Control various peripherals
\item
  \textbf{Interrupt System}: 5 interrupt sources
\item
  \textbf{Power Saving Modes}: Idle and Power-down modes
\end{itemize}

\end{solutionbox}
\begin{mnemonicbox}
``CPU ROM RAM Ports Timers Serial Interrupts''
(CRRRPTI)

\end{mnemonicbox}
\subsection*{Question 3(a) OR [3
marks]}\label{q3a}

\textbf{Explain PSW SFR of 8051 Microcontroller.}

\begin{solutionbox}

PSW (Program Status Word) contains status flags and register bank
selection:

{\def\LTcaptype{none} % do not increment counter
\begin{longtable}[]{@{}lll@{}}
\toprule\noalign{}
Bit & Flag & Function \\
\midrule\noalign{}
\endhead
\bottomrule\noalign{}
\endlastfoot
\textbf{D7} & \textbf{CY} & Carry flag \\
\textbf{D6} & \textbf{AC} & Auxiliary carry flag \\
\textbf{D5} & \textbf{F0} & Flag 0 (user defined) \\
\textbf{D4} & \textbf{RS1} & Register bank select bit 1 \\
\textbf{D3} & \textbf{RS0} & Register bank select bit 0 \\
\textbf{D2} & \textbf{OV} & Overflow flag \\
\textbf{D1} & \textbf{-} & Reserved \\
\textbf{D0} & \textbf{P} & Parity flag \\
\end{longtable}
}

\textbf{Register Bank Selection:}

\begin{itemize}
\tightlist
\item
  \textbf{RS1=0, RS0=0}: Bank 0 (00H-07H)
\item
  \textbf{RS1=0, RS0=1}: Bank 1 (08H-0FH)
\item
  \textbf{RS1=1, RS0=0}: Bank 2 (10H-17H)
\item
  \textbf{RS1=1, RS0=1}: Bank 3 (18H-1FH)
\end{itemize}

\end{solutionbox}
\begin{mnemonicbox}
``CY AC F0 RS1 RS0 OV - P''

\end{mnemonicbox}
\subsection*{Question 3(b) OR [4
marks]}\label{q3b}

\textbf{Explain Function of Each bit of SCON SFR of 8051
Microcontroller.}

\begin{solutionbox}

SCON (Serial Control) register controls the serial port operation:

{\def\LTcaptype{none} % do not increment counter
\begin{longtable}[]{@{}lll@{}}
\toprule\noalign{}
Bit & Name & Function \\
\midrule\noalign{}
\endhead
\bottomrule\noalign{}
\endlastfoot
\textbf{D7} & \textbf{SM0} & Serial mode bit 0 \\
\textbf{D6} & \textbf{SM1} & Serial mode bit 1 \\
\textbf{D5} & \textbf{SM2} & Multiprocessor communication \\
\textbf{D4} & \textbf{REN} & Receive enable \\
\textbf{D3} & \textbf{TB8} & 9th bit to transmit \\
\textbf{D2} & \textbf{RB8} & 9th bit received \\
\textbf{D1} & \textbf{TI} & Transmit interrupt flag \\
\textbf{D0} & \textbf{RI} & Receive interrupt flag \\
\end{longtable}
}

\textbf{Serial Modes:}

\begin{itemize}
\tightlist
\item
  \textbf{Mode 0}: Shift register, fixed baud rate
\item
  \textbf{Mode 1}: 8-bit UART, variable baud rate\\
\item
  \textbf{Mode 2}: 9-bit UART, fixed baud rate
\item
  \textbf{Mode 3}: 9-bit UART, variable baud rate
\end{itemize}

\textbf{Control Functions:}

\begin{itemize}
\tightlist
\item
  \textbf{REN}: Must be set to enable reception
\item
  \textbf{TI/RI}: Set by hardware, cleared by software
\end{itemize}

\end{solutionbox}
\begin{mnemonicbox}
``SM0 SM1 SM2 REN TB8 RB8 TI RI''

\end{mnemonicbox}
\subsection*{Question 3(c) OR [7
marks]}\label{q3c}

\textbf{Explain Pin Diagram of 8051 with neat sketch.}

\begin{solutionbox}

The 8051 is available in 40-pin DIP package:

\textbf{Pin Groups:}

\begin{itemize}
\tightlist
\item
  \textbf{Ports 0-3}: I/O pins with dual functions
\item
  \textbf{Power}: VCC, VSS pins
\item
  \textbf{Crystal}: XTAL1, XTAL2 for clock
\item
  \textbf{Control}: RST, EA, ALE, PSEN
\end{itemize}

\textbf{Pin Diagram:}

\begin{lstlisting}
           +---\\_/---+
  P1.0   --|1       40|-- Vcc
  P1.1   --|2       39|-- P0.0/AD0
  P1.2   --|3       38|-- P0.1/AD1
  P1.3   --|4       37|-- P0.2/AD2
  P1.4   --|5  8051 36|-- P0.3/AD3
  P1.5   --|6       35|-- P0.4/AD4
  P1.6   --|7       34|-- P0.5/AD5
  P1.7   --|8       33|-- P0.6/AD6
   RST   --|9       32|-- P0.7/AD7
P3.0/RXD --|10      31|-- EA/VPP
P3.1/TXD --|11      30|-- ALE/PROG
P3.2/INT0--|12      29|-- PSEN
P3.3/INT1--|13      28|-- P2.7/A15
P3.4/T0  --|14      27|-- P2.6/A14
P3.5/T1  --|15      26|-- P2.5/A13
P3.6/WR  --|16      25|-- P2.4/A12
P3.7/RD  --|17      24|-- P2.3/A11
 XTAL2   --|18      23|-- P2.2/A10
 XTAL1   --|19      22|-- P2.1/A9
   Vss   --|20      21|-- P2.0/A8
           +----------+
\end{lstlisting}

\textbf{Port Functions:}

\begin{itemize}
\tightlist
\item
  \textbf{Port 0}: Multiplexed address/data bus
\item
  \textbf{Port 1}: General purpose I/O
\item
  \textbf{Port 2}: Higher order address bus
\item
  \textbf{Port 3}: Alternate functions (UART, interrupts, timers)
\end{itemize}

\end{solutionbox}
\begin{mnemonicbox}
``Port Power Crystal Control'' (PPCC)

\end{mnemonicbox}
\subsection*{Question 4(a) [3 marks]}\label{q4a}

\textbf{Write and Explain any Three Data Transfer Instructions of 8051
Microcontroller.}

\begin{solutionbox}

Data transfer instructions move data between registers, memory, and I/O:

{\def\LTcaptype{none} % do not increment counter
\begin{longtable}[]{@{}ll@{}}
\toprule\noalign{}
Instruction & Function \\
\midrule\noalign{}
\endhead
\bottomrule\noalign{}
\endlastfoot
\textbf{MOV A,R0} & Move contents of R0 to Accumulator \\
\textbf{MOV R1,\#50H} & Move immediate data 50H to R1 \\
\textbf{MOV 30H,A} & Move Accumulator contents to address 30H \\
\end{longtable}
}

\textbf{Code Examples:}

\begin{lstlisting}
MOV A,R0        ; A = R0
MOV R1,#50H     ; R1 = 50H  
MOV 30H,A       ; [30H] = A
\end{lstlisting}

\textbf{Key Features:}

\begin{itemize}
\tightlist
\item
  \textbf{No flags affected} during data transfer
\item
  \textbf{Various addressing modes} supported
\item
  \textbf{Single cycle execution} for most instructions
\end{itemize}

\end{solutionbox}
\begin{mnemonicbox}
``MOV Between Register Immediate Direct'' (MBRID)

\end{mnemonicbox}
\subsection*{Question 4(b) [4 marks]}\label{q4b}

\textbf{Write 8051 Assembly Language Program to Multiply Content of R0
and R1 and Store Result in R5 (Lower Byte) and R6 (Higher Byte).}

\begin{solutionbox}

\begin{lstlisting}
ORG 0000H           ; Origin at 0000H

START:
    MOV A,R0        ; Load R0 into Accumulator
    MOV B,R1        ; Load R1 into B register
    MUL AB          ; Multiply A and B
    MOV R5,A        ; Store lower byte in R5
    MOV R6,B        ; Store higher byte in R6
    
    SJMP $          ; Stop program

END                 ; End of program
\end{lstlisting}

\textbf{Program Flow:}

\begin{enumerate}
\tightlist
\item
  \textbf{Load multiplicand} from R0 to A
\item
  \textbf{Load multiplier} from R1 to B\\
\item
  \textbf{Execute multiplication} using MUL AB
\item
  \textbf{Store lower byte} of result in R5
\item
  \textbf{Store higher byte} of result in R6
\end{enumerate}

\textbf{Note:} MUL AB instruction automatically stores 16-bit result
with lower byte in A and higher byte in B.

\end{solutionbox}
\subsection*{Question 4(c) [7 marks]}\label{q4c}

\textbf{List Addressing Modes of 8051 Microcontroller and Explain each
with Example.}

\begin{solutionbox}

The 8051 supports several addressing modes:

{\def\LTcaptype{none} % do not increment counter
\begin{longtable}[]{@{}lll@{}}
\toprule\noalign{}
Mode & Description & Example \\
\midrule\noalign{}
\endhead
\bottomrule\noalign{}
\endlastfoot
\textbf{Immediate} & Data specified in instruction & MOV A,\#50H \\
\textbf{Register} & Register contains data & MOV A,R0 \\
\textbf{Direct} & Memory address specified & MOV A,30H \\
\textbf{Indirect} & Register contains address & MOV A,@R0 \\
\textbf{Indexed} & Base + offset addressing & MOVC A,@A+DPTR \\
\textbf{Relative} & PC + offset & SJMP LABEL \\
\textbf{Bit} & Bit-specific operations & SETB P1.0 \\
\end{longtable}
}

\textbf{Detailed Examples:}

\textbf{1. Immediate Addressing:}

\begin{lstlisting}
MOV A,#25H      ; A = 25H (immediate data)
\end{lstlisting}

\textbf{2. Register Addressing:}

\begin{lstlisting}
MOV A,R1        ; A = contents of R1
\end{lstlisting}

\textbf{3. Direct Addressing:}

\begin{lstlisting}
MOV A,40H       ; A = contents of memory location 40H
\end{lstlisting}

\textbf{4. Indirect Addressing:}

\begin{lstlisting}
MOV R0,#40H     ; R0 = 40H (address)
MOV A,@R0       ; A = contents of location pointed by R0
\end{lstlisting}

\end{solutionbox}
\begin{mnemonicbox}
``I-R-D-I-I-R-B'' (Immediate Register Direct Indirect
Indexed Relative Bit)

\end{mnemonicbox}
\subsection*{Question 4(a) OR [3
marks]}\label{q4a}

\textbf{Write and Explain any Three Logical Instructions 8051
Microcontroller.}

\begin{solutionbox}

Logical instructions perform bitwise operations:

{\def\LTcaptype{none} % do not increment counter
\begin{longtable}[]{@{}ll@{}}
\toprule\noalign{}
Instruction & Function \\
\midrule\noalign{}
\endhead
\bottomrule\noalign{}
\endlastfoot
\textbf{ANL A,R0} & AND Accumulator with R0 \\
\textbf{ORL A,\#0FH} & OR Accumulator with immediate data 0FH \\
\textbf{XRL A,30H} & XOR Accumulator with contents of address 30H \\
\end{longtable}
}

\textbf{Code Examples:}

\begin{lstlisting}
ANL A,R0        ; A = A AND R0
ORL A,#0FH      ; A = A OR 0FH
XRL A,30H       ; A = A XOR [30H]
\end{lstlisting}

\textbf{Applications:}

\begin{itemize}
\tightlist
\item
  \textbf{ANL}: Masking specific bits (clear unwanted bits)
\item
  \textbf{ORL}: Setting specific bits\\
\item
  \textbf{XRL}: Toggling bits, checksum calculations
\end{itemize}

\end{solutionbox}
\begin{mnemonicbox}
``AND OR XOR'' logical operations

\end{mnemonicbox}
\subsection*{Question 4(b) OR [4
marks]}\label{q4b}

\textbf{Write 8051 Assembly Language Program to Subtract Number Stored
in 2000h from 2001h and Store result in 2002h. All given memory
locations are External Memory locations.}

\begin{solutionbox}

\begin{lstlisting}
ORG 0000H           ; Origin at 0000H

START:
    MOV DPTR,#2001H ; Point to minuend address
    MOVX A,@DPTR    ; Load minuend from external memory
    MOV R0,A        ; Store minuend in R0
    
    MOV DPTR,#2000H ; Point to subtrahend address  
    MOVX A,@DPTR    ; Load subtrahend from external memory
    MOV R1,A        ; Store subtrahend in R1
    
    MOV A,R0        ; Load minuend into A
    CLR C           ; Clear carry flag
    SUBB A,R1       ; Subtract: A = R0 - R1
    
    MOV DPTR,#2002H ; Point to result address
    MOVX @DPTR,A    ; Store result in external memory
    
    SJMP $          ; Stop program

END                 ; End of program
\end{lstlisting}

\textbf{Program Steps:}

\begin{enumerate}
\tightlist
\item
  \textbf{Load minuend} from external memory 2001H
\item
  \textbf{Load subtrahend} from external memory 2000H
\item
  \textbf{Perform subtraction} using SUBB instruction
\item
  \textbf{Store result} in external memory location 2002H
\end{enumerate}

\textbf{Note:} MOVX instruction is used for external memory access.

\end{solutionbox}
\subsection*{Question 4(c) OR [7
marks]}\label{q4c}

\textbf{Explain Instructions: (i) RET (ii) PUSH (iii) CLR PSW.0 (iv) RLC
A (v) CJNE A,\#DATA,LABEL (vi) NOP (vii) ANL A,\#DATA}

\begin{solutionbox}

{\def\LTcaptype{none} % do not increment counter
\begin{longtable}[]{@{}
  >{\raggedright\arraybackslash}p{(\linewidth - 4\tabcolsep) * \real{0.3611}}
  >{\raggedright\arraybackslash}p{(\linewidth - 4\tabcolsep) * \real{0.2778}}
  >{\raggedright\arraybackslash}p{(\linewidth - 4\tabcolsep) * \real{0.3611}}@{}}
\toprule\noalign{}
\begin{minipage}[b]{\linewidth}\raggedright
Instruction
\end{minipage} & \begin{minipage}[b]{\linewidth}\raggedright
Function
\end{minipage} & \begin{minipage}[b]{\linewidth}\raggedright
Description
\end{minipage} \\
\midrule\noalign{}
\endhead
\bottomrule\noalign{}
\endlastfoot
\textbf{RET} & Return from subroutine & Pops PC from stack and returns
control \\
\textbf{PUSH 30H} & Push to stack & Pushes contents of address 30H to
stack \\
\textbf{CLR PSW.0} & Clear carry flag & Clears bit 0 of PSW (Carry
flag) \\
\textbf{RLC A} & Rotate left through carry & Rotates A left through
carry flag \\
\textbf{CJNE A,\#50H,NEXT} & Compare and jump & Jump to NEXT if A \neq
50H \\
\textbf{NOP} & No operation & Does nothing, consumes one cycle \\
\textbf{ANL A,\#0FH} & AND with immediate & A = A AND 0FH \\
\end{longtable}
}

\textbf{Detailed Explanations:}

\textbf{RET:} Used to return from subroutine calls

\begin{lstlisting}
CALL SUB1       ; Call subroutine
...
SUB1: 
    MOV A,#10H
    RET         ; Return to caller
\end{lstlisting}

\textbf{PUSH:} Saves data on stack

\begin{lstlisting}
PUSH ACC        ; Save accumulator on stack
\end{lstlisting}

\textbf{RLC A:} Bit rotation with carry

\begin{lstlisting}
CY <- A7 <- A6 <- A5 <- A4 <- A3 <- A2 <- A1 <- A0 <- CY
\end{lstlisting}

\textbf{CJNE:} Conditional branching

\begin{lstlisting}
CJNE A,#50H,NOT_EQUAL   ; If A\neq50H, jump to NOT_EQUAL
; A equals 50H
NOT_EQUAL:
; A not equal to 50H
\end{lstlisting}

\end{solutionbox}
\begin{mnemonicbox}
``Return Push Clear Rotate Compare No-op AND''
(RPCRNA)

\end{mnemonicbox}
\subsection*{Question 5(a) [3 marks]}\label{q5a}

\textbf{List the application of Microcontroller in various fields.}

\begin{solutionbox}

Microcontrollers are used in numerous applications across various
fields:

{\def\LTcaptype{none} % do not increment counter
\begin{longtable}[]{@{}ll@{}}
\toprule\noalign{}
Field & Applications \\
\midrule\noalign{}
\endhead
\bottomrule\noalign{}
\endlastfoot
\textbf{Consumer Electronics} & TV remotes, washing machines,
microwaves \\
\textbf{Automotive} & Engine control, ABS, airbag systems \\
\textbf{Industrial} & Process control, robotics, automation \\
\textbf{Medical} & Pacemakers, blood glucose meters, ventilators \\
\textbf{Communication} & Mobile phones, modems, routers \\
\textbf{Home Automation} & Smart thermostats, security systems,
lighting \\
\end{longtable}
}

\textbf{Key Advantages:}

\begin{itemize}
\tightlist
\item
  \textbf{Low cost} and \textbf{compact size}
\item
  \textbf{Low power consumption}
\item
  \textbf{Real-time operation}
\item
  \textbf{Easy interfacing} with sensors and actuators
\end{itemize}

\end{solutionbox}
\begin{mnemonicbox}
``Consumer Automotive Industrial Medical
Communication Home'' (CAIMCH)

\end{mnemonicbox}
\subsection*{Question 5(b) [4 marks]}\label{q5b}

\textbf{Interface Stepper Motor with 8051 Microcontroller and Explain in
brief.}

\begin{solutionbox}

Stepper motor interfacing requires driver circuit due to current
requirements:

\textbf{Interface Circuit:}

\begin{lstlisting}
    8051          ULN2003         Stepper Motor
P1.0 ----+---> Input1 -----> Coil A
P1.1 ----+---> Input2 -----> Coil B  
P1.2 ----+---> Input3 -----> Coil C
P1.3 ----+---> Input4 -----> Coil D
         |
        GND
\end{lstlisting}

\textbf{Control Sequence (Half-Step):}

{\def\LTcaptype{none} % do not increment counter
\begin{longtable}[]{@{}llllll@{}}
\toprule\noalign{}
Step & P1.3 & P1.2 & P1.1 & P1.0 & Binary \\
\midrule\noalign{}
\endhead
\bottomrule\noalign{}
\endlastfoot
1 & 0 & 0 & 0 & 1 & 01H \\
2 & 0 & 0 & 1 & 1 & 03H \\
3 & 0 & 0 & 1 & 0 & 02H \\
4 & 0 & 1 & 1 & 0 & 06H \\
5 & 0 & 1 & 0 & 0 & 04H \\
6 & 1 & 1 & 0 & 0 & 0CH \\
7 & 1 & 0 & 0 & 0 & 08H \\
8 & 1 & 0 & 0 & 1 & 09H \\
\end{longtable}
}

\textbf{Driver Circuit:}

\begin{itemize}
\tightlist
\item
  \textbf{ULN2003}: Darlington driver IC provides current amplification
\item
  \textbf{Protection diodes}: Protect against back EMF
\item
  \textbf{Common ground}: Between 8051 and motor supply
\end{itemize}

\end{solutionbox}
\begin{mnemonicbox}
``Step Sequence Driver Protection'' (SSDP)

\end{mnemonicbox}
\subsection*{Question 5(c) [7 marks]}\label{q5c}

\textbf{Draw interfacing circuit to interface 4 LED at port 2.0 to 2.3
of microcontroller 8051 and write assembly language program to flash
it.}

\begin{solutionbox}

\textbf{Interface Circuit:}

\begin{lstlisting}
8051 Pin    Resistor     LED
P2.0 -----> 330Ω -----> LED1 -----> GND
P2.1 -----> 330Ω -----> LED2 -----> GND  
P2.2 -----> 330Ω -----> LED3 -----> GND
P2.3 -----> 330Ω -----> LED4 -----> GND
\end{lstlisting}

\textbf{Assembly Program:}

\begin{lstlisting}
ORG 0000H               ; Start address

MAIN:
    MOV P2,#0FH         ; Turn ON all LEDs (P2.0-P2.3)
    CALL DELAY          ; Call delay subroutine
    MOV P2,#00H         ; Turn OFF all LEDs
    CALL DELAY          ; Call delay subroutine
    SJMP MAIN           ; Repeat flashing

DELAY:
    MOV R0,#255         ; Outer loop counter
LOOP1:
    MOV R1,#255         ; Inner loop counter  
LOOP2:
    DJNZ R1,LOOP2       ; Decrement and jump if not zero
    DJNZ R0,LOOP1       ; Decrement outer counter
    RET                 ; Return from delay

END                     ; End of program
\end{lstlisting}

\textbf{Circuit Components:}

\begin{itemize}
\tightlist
\item
  \textbf{Current limiting resistors}: 330Ω to limit LED current
\item
  \textbf{LEDs}: Connected in active HIGH configuration
\item
  \textbf{Common ground}: All LED cathodes connected to ground
\end{itemize}

\textbf{Program Operation:}

\begin{enumerate}
\tightlist
\item
  \textbf{Turn ON LEDs}: Set P2.0-P2.3 high
\item
  \textbf{Delay}: Wait for visible flash duration
\item
  \textbf{Turn OFF LEDs}: Clear P2.0-P2.3
\item
  \textbf{Repeat}: Continuous flashing loop
\end{enumerate}

\end{solutionbox}
\begin{mnemonicbox}
``Resistor LED Ground Program'' (RLGP)

\end{mnemonicbox}
\subsection*{Question 5(a) OR [3
marks]}\label{q5a}

\textbf{Draw Interfacing of Push button switch and LED with 8051
Microcontroller.}

\begin{solutionbox}

\textbf{Interface Circuit:}

\begin{lstlisting}
       +5V
        |
        ├─── 10KΩ ──── P1.0 (Input)
        |         |
      Switch      |
        |        GND
       GND
       
P1.1 ──── 330Ω ──── LED ──── GND
                   (Output)
\end{lstlisting}

\textbf{Circuit Description:}

\begin{itemize}
\tightlist
\item
  \textbf{Push Button}: Connected to P1.0 with pull-up resistor
\item
  \textbf{Pull-up Resistor}: 10KΩ ensures logic HIGH when switch open
\item
  \textbf{LED}: Connected to P1.1 through current limiting resistor
\item
  \textbf{Current Limiting}: 330Ω resistor protects LED
\end{itemize}

\textbf{Operation:}

\begin{itemize}
\tightlist
\item
  \textbf{Switch Open}: P1.0 = 1 (HIGH)
\item
  \textbf{Switch Pressed}: P1.0 = 0 (LOW)
\item
  \textbf{LED Control}: Through P1.1 pin
\end{itemize}

\end{solutionbox}
\begin{mnemonicbox}
``Pull-up Switch LED Current-limit'' (PSLC)

\end{mnemonicbox}
\subsection*{Question 5(b) OR [4
marks]}\label{q5b}

\textbf{Interface Relay with 8051 Microcontroller and Explain in brief.}

\begin{solutionbox}

\textbf{Interface Circuit:}

\begin{lstlisting}
8051    Transistor    Relay      Load
P1.0 ──┬─ 1KΩ ──┬─── BC547 ────┬──── Relay Coil ──── +12V
       |        |    (NPN)     |                    |
      GND      Base           Collector            NO/NC
                |               |                    |
               Emitter ────────GND               Load Device
                |
            Flyback Diode
            (1N4007)
\end{lstlisting}

\textbf{Components:}

\begin{itemize}
\tightlist
\item
  \textbf{Transistor BC547}: Switching element for relay coil
\item
  \textbf{Base Resistor}: 1KΩ limits base current
\item
  \textbf{Flyback Diode}: 1N4007 protects against back EMF
\item
  \textbf{Relay}: 12V DC relay with NO/NC contacts
\end{itemize}

\textbf{Operation:}

\begin{enumerate}
\tightlist
\item
  \textbf{Logic HIGH} on P1.0 \rightarrow Transistor ON \rightarrow Relay energized
\item
  \textbf{Logic LOW} on P1.0 \rightarrow Transistor OFF \rightarrow Relay de-energized
\item
  \textbf{Relay contacts} switch the load circuit
\end{enumerate}

\textbf{Protection:}

\begin{itemize}
\tightlist
\item
  \textbf{Flyback diode} prevents damage from relay coil's back EMF
\item
  \textbf{Current limiting} through base resistor
\end{itemize}

\end{solutionbox}
\begin{mnemonicbox}
``Transistor Resistor Diode Relay'' (TRDR)

\end{mnemonicbox}
\subsection*{Question 5(c) OR [7
marks]}\label{q5c}

\textbf{Interface 7 segment LED with 8051 Microcontroller and write
assembly language program to print 0 on it.}

\begin{solutionbox}

\textbf{Interface Circuit:}

\begin{lstlisting}
       a
    -------
f  |       |  b
   |   g   |
    -------
e  |       |  c
   |       |
    -------
       d

8051 Connections:
P1.0 ──── 330Ω ──── a segment
P1.1 ──── 330Ω ──── b segment  
P1.2 ──── 330Ω ──── c segment
P1.3 ──── 330Ω ──── d segment
P1.4 ──── 330Ω ──── e segment
P1.5 ──── 330Ω ──── f segment
P1.6 ──── 330Ω ──── g segment
P1.7 ──── 330Ω ──── dp (decimal point)

Common Cathode: All cathodes to GND
\end{lstlisting}

\textbf{7-Segment Code Table:}

{\def\LTcaptype{none} % do not increment counter
\begin{longtable}[]{@{}llll@{}}
\toprule\noalign{}
Digit & Display & gfedcba & Hex Code \\
\midrule\noalign{}
\endhead
\bottomrule\noalign{}
\endlastfoot
0 & Display 0 & 0111111 & 3FH \\
1 & Display 1 & 0000110 & 06H \\
2 & Display 2 & 1011011 & 5BH \\
\end{longtable}
}

\textbf{Assembly Program to Display `0':}

\begin{lstlisting}
ORG 0000H               ; Start address

MAIN:
    MOV P1,#3FH         ; Display '0' on 7-segment
                        ; a,b,c,d,e,f ON, g OFF
    SJMP MAIN           ; Keep displaying

END                     ; End of program
\end{lstlisting}

\textbf{Segment Pattern for `0':}

\begin{itemize}
\tightlist
\item
  \textbf{Segments ON}: a, b, c, d, e, f (bits 0-5 = 1)
\item
  \textbf{Segment OFF}: g (bit 6 = 0)
\item
  \textbf{Binary}: 00111111 = 3FH
\end{itemize}

\textbf{Circuit Features:}

\begin{itemize}
\tightlist
\item
  \textbf{Common Cathode}: All segment cathodes connected to ground
\item
  \textbf{Current Limiting}: 330Ω resistors for each segment
\item
  \textbf{Active HIGH}: Logic 1 turns ON segment
\end{itemize}

\textbf{Alternative Patterns:}

\begin{lstlisting}
; Other digits can be displayed using:
MOV P1,#06H         ; Display '1'
MOV P1,#5BH         ; Display '2'
\end{lstlisting}

\end{solutionbox}
\begin{mnemonicbox}
``Seven Segments Common Cathode Current-limit''
(SSCCC)

\end{mnemonicbox}

\end{document}
