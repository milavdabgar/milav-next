\documentclass[10pt,a4paper]{article}

% content/resources/templates/preamble.tex
\usepackage[margin=0.6in]{geometry}
\author{Milav Dabgar}
\usepackage{amsmath,amssymb,amsthm}
\usepackage{booktabs}
\usepackage{multirow}
\usepackage{xcolor}
\usepackage{tcolorbox}
\tcbuselibrary{breakable,skins}
\usepackage[colorlinks=true,linkcolor=blue]{hyperref}
\usepackage{titlesec}
\usepackage{enumitem}
\usepackage{tikz}
\usepackage{pgfplots}
\usepackage{circuitikz}
\usepackage[version=4]{mhchem}
\usepackage{longtable}
\usepackage{array}
\usepackage{float}
\usepackage{caption}
\usepackage{listings}

\lstset{
  basicstyle=\small\ttfamily,
  breaklines=true,
  breakatwhitespace=false,
  postbreak=\mbox{\textcolor{red}{$\hookrightarrow$}\space},
  float=false,
  numbers=left,
  numberstyle=\tiny\color{gray},
  numbersep=10pt,
  xleftmargin=2em,
  keywordstyle=\color{blue},
  commentstyle=\color{green!60!black},
  stringstyle=\color{purple},
  backgroundcolor=\color{gray!5},
  showstringspaces=false,
  tabsize=2,
  captionpos=b,
  keepspaces=true,
  columns=flexible
}

\pgfplotsset{compat=1.18}
\usetikzlibrary{shapes,arrows,positioning,calc,patterns,decorations.pathmorphing,decorations.markings,arrows.meta}

% Color scheme
\definecolor{headcolor}{RGB}{0,102,204}
\definecolor{keycolor}{RGB}{220,20,60}
\definecolor{solutioncolor}{RGB}{34,139,34}
\definecolor{mnemoniccolor}{RGB}{148,0,211}
\definecolor{codecolor}{RGB}{0,0,100}

% Spacing
\setlength{\parskip}{3pt}
\setlist[itemize]{nosep}
\setlist[enumerate]{nosep}

% Title formatting
\titleformat{\section}{\Large\bfseries\color{headcolor}}{\thesection}{1em}{}
\titleformat{\subsection}{\large\bfseries\color{headcolor}}{\thesubsection}{1em}{}

% Pandoc tightlist compatibility
\providecommand{\tightlist}{%
  \setlength{\itemsep}{0pt}\setlength{\parskip}{0pt}}

% Pandoc longtable compatibility
\newcounter{none}
\def\thenone{}


% content/resources/templates/english-boxes.tex
% This file is currently empty - it exists to maintain consistency with the import structure.
% Add custom environments here if needed in the future.


\begin{document}

\begin{center}
{\Huge\bfseries\color{headcolor} Subject Name Solutions}\\[5pt]
{\LARGE 1333202 -- Summer 2025}\\[3pt]
{\large Semester 1 Study Material}\\[3pt]
{\normalsize\textit{Detailed Solutions and Explanations}}
\end{center}

\vspace{10pt}

\subsection*{Question 1(A) [3 marks]}\label{q1a}

\textbf{Draw The Bus Organization Of 8085.}

\begin{solutionbox}

\begin{lstlisting}
                    8085 MICROPROCESSOR
                    
    +-------+      16-bit         +-------+
    |       |<------------------->| Memory|
    |       |   Address Bus       |       |
    |  8085 |                     +-------+
    |  CPU  |      8-bit          
    |       |<------------------->+-------+
    |       |    Data Bus         | I/O   |
    +-------+                     | Ports |
         ^                        +-------+
         |
         | Control Bus
         v
    +-----------+
    | Control   |
    | Signals   |
    +-----------+
\end{lstlisting}

\textbf{Bus Types:}

\begin{itemize}
\tightlist
\item
  \textbf{Address Bus}: 16-bit unidirectional bus for memory addressing
\item
  \textbf{Data Bus}: 8-bit bidirectional bus for data transfer\\
\item
  \textbf{Control Bus}: Control signals like RD, WR, ALE, IO/M
\end{itemize}

\end{solutionbox}
\begin{mnemonicbox}
``ADC - Address, Data, Control''

\end{mnemonicbox}
\subsection*{Question 1(B) [4 marks]}\label{q1b}

\textbf{Compare Microprocessor With Microcontroller.}

\begin{solutionbox}

{\def\LTcaptype{none} % do not increment counter
\begin{longtable}[]{@{}
  >{\raggedright\arraybackslash}p{(\linewidth - 4\tabcolsep) * \real{0.2195}}
  >{\raggedright\arraybackslash}p{(\linewidth - 4\tabcolsep) * \real{0.3659}}
  >{\raggedright\arraybackslash}p{(\linewidth - 4\tabcolsep) * \real{0.4146}}@{}}
\toprule\noalign{}
\begin{minipage}[b]{\linewidth}\raggedright
Feature
\end{minipage} & \begin{minipage}[b]{\linewidth}\raggedright
Microprocessor
\end{minipage} & \begin{minipage}[b]{\linewidth}\raggedright
Microcontroller
\end{minipage} \\
\midrule\noalign{}
\endhead
\bottomrule\noalign{}
\endlastfoot
\textbf{Architecture} & External components needed & All components on
single chip \\
\textbf{Memory} & External RAM/ROM required & Internal RAM/ROM
available \\
\textbf{Cost} & Higher system cost & Lower system cost \\
\textbf{Power} & Higher power consumption & Lower power consumption \\
\textbf{Size} & Larger system size & Compact system \\
\textbf{Applications} & General purpose computing & Embedded control
applications \\
\end{longtable}
}

\textbf{Key Points:}

\begin{itemize}
\tightlist
\item
  \textbf{Microprocessor}: CPU only, requires external support chips
\item
  \textbf{Microcontroller}: Complete computer system on chip
\end{itemize}

\end{solutionbox}
\begin{mnemonicbox}
``MICRO - Memory Internal, Compact, Reduced cost,
Optimized''

\end{mnemonicbox}
\subsection*{Question 1(C) [7 marks]}\label{q1c}

\textbf{Draw And Explain Each Block Of 8085 Microprocessor.}

\begin{solutionbox}

\includegraphics[width=1\linewidth,height=\textheight,keepaspectratio]{mermaid-29216edc.pdf}

\textbf{Block Functions:}

\begin{itemize}
\tightlist
\item
  \textbf{ALU}: Performs arithmetic and logical operations
\item
  \textbf{Accumulator}: Primary working register for data processing
\item
  \textbf{Register Array}: B, C, D, E, H, L general purpose registers
\item
  \textbf{Program Counter}: Holds address of next instruction
\item
  \textbf{Stack Pointer}: Points to top of stack in memory
\item
  \textbf{Control Unit}: Controls overall operation of processor
\end{itemize}

\end{solutionbox}
\begin{mnemonicbox}
``APRIL - ALU, Program counter, Registers,
Instruction decoder, Logic control''

\end{mnemonicbox}
\subsection*{Question 1(C) OR [7
marks]}\label{q1c}

\textbf{Draw Pin Diagram Of 8085 Microprocessor And Explain Any 4(Four)
Pins.}

\begin{solutionbox}

\begin{lstlisting}
                  8085 PIN DIAGRAM
                  
      X1    1 +-------+ 40  Vcc
      X2    2 |       | 39  HOLD
    RESET   3 |       | 38  HLDA  
     SOD    4 |       | 37  CLK(OUT)
     SID    5 |  8085 | 36  RESET IN
    TRAP    6 |       | 35  READY
    RST7.5  7 |       | 34  IO/M
    RST6.5  8 |       | 33  S1
    RST5.5  9 |       | 32  RD
    INTR   10 |       | 31  WR
    INTA   11 |       | 30  ALE
   AD0-7 12-19|       | 23-29 A8-A15
     Vss   20 +-------+ 21  A15-A8
\end{lstlisting}

\textbf{Pin Explanations:}

\begin{itemize}
\tightlist
\item
  \textbf{ALE (Pin 30)}: Address Latch Enable - separates address and
  data on multiplexed bus
\item
  \textbf{RD (Pin 32)}: Read control signal - active low, indicates read
  operation
\item
  \textbf{WR (Pin 31)}: Write control signal - active low, indicates
  write operation\\
\item
  \textbf{RESET (Pin 36)}: Reset input - initializes processor when low
\end{itemize}

\end{solutionbox}
\begin{mnemonicbox}
``ARWA - ALE, Read, Write, rAset''

\end{mnemonicbox}
\subsection*{Question 2(A) [3 marks]}\label{q2a}

\textbf{Define : (1) Opcode (2) Operand}

\begin{solutionbox}

\textbf{Definitions:}

\begin{itemize}
\tightlist
\item
  \textbf{Opcode}: Operation Code - specifies the operation to be
  performed (ADD, MOV, JMP)
\item
  \textbf{Operand}: Data or address on which operation is performed
\end{itemize}

\textbf{Example:}

\begin{lstlisting}
MOV A, B
|   |  |
|   |  +-- Operand 2 (Source)  
|   +-- Operand 1 (Destination)
+-- Opcode
\end{lstlisting}

\end{solutionbox}
\begin{mnemonicbox}
``OO - Operation + Operand''

\end{mnemonicbox}
\subsection*{Question 2(B) [4 marks]}\label{q2b}

\textbf{Give Differences Between RISC And CISC.}

\begin{solutionbox}

{\def\LTcaptype{none} % do not increment counter
\begin{longtable}[]{@{}lll@{}}
\toprule\noalign{}
Feature & RISC & CISC \\
\midrule\noalign{}
\endhead
\bottomrule\noalign{}
\endlastfoot
\textbf{Instructions} & Simple, fixed format & Complex, variable
format \\
\textbf{Execution} & Single cycle execution & Multiple cycle
execution \\
\textbf{Addressing} & Few addressing modes & Many addressing modes \\
\textbf{Memory} & Load/Store architecture & Memory-to-memory
operations \\
\textbf{Compiler} & Complex compiler design & Simple compiler design \\
\end{longtable}
}

\textbf{Key Points:}

\begin{itemize}
\tightlist
\item
  \textbf{RISC}: Reduced Instruction Set Computer - simpler, faster
\item
  \textbf{CISC}: Complex Instruction Set Computer - feature rich
\end{itemize}

\end{solutionbox}
\begin{mnemonicbox}
``RISC is SLIM - Simple, Load-store, Instruction
reduced, Memory efficient''

\end{mnemonicbox}
\subsection*{Question 2(C) [7 marks]}\label{q2c}

\textbf{Give Differences Between Von-Neumann \& Harvard Architecture.}

\begin{solutionbox}

{\def\LTcaptype{none} % do not increment counter
\begin{longtable}[]{@{}
  >{\raggedright\arraybackslash}p{(\linewidth - 4\tabcolsep) * \real{0.2903}}
  >{\raggedright\arraybackslash}p{(\linewidth - 4\tabcolsep) * \real{0.4194}}
  >{\raggedright\arraybackslash}p{(\linewidth - 4\tabcolsep) * \real{0.2903}}@{}}
\toprule\noalign{}
\begin{minipage}[b]{\linewidth}\raggedright
Feature
\end{minipage} & \begin{minipage}[b]{\linewidth}\raggedright
Von-Neumann
\end{minipage} & \begin{minipage}[b]{\linewidth}\raggedright
Harvard
\end{minipage} \\
\midrule\noalign{}
\endhead
\bottomrule\noalign{}
\endlastfoot
\textbf{Memory} & Single memory for data and instructions & Separate
memory for data and instructions \\
\textbf{Bus Structure} & Single bus system & Dual bus system \\
\textbf{Access} & Sequential access to data and instructions &
Simultaneous access possible \\
\textbf{Cost} & Lower cost & Higher cost \\
\textbf{Speed} & Slower due to bus conflicts & Faster parallel access \\
\textbf{Examples} & 8085, General computers & 8051, DSP processors \\
\end{longtable}
}

\includegraphics[width=1\linewidth,height=\textheight,keepaspectratio]{mermaid-9900f98e.pdf}

\end{solutionbox}
\begin{mnemonicbox}
``VH - Von has one bus, Harvard has two''

\end{mnemonicbox}
\subsection*{Question 2(A) OR [3
marks]}\label{q2a}

\textbf{Define : (1) T-State (2) Instruction Cycle (3) Machine Cycle}

\begin{solutionbox}

\textbf{Definitions:}

\begin{itemize}
\tightlist
\item
  \textbf{T-State}: Time state - basic timing unit, one clock period
\item
  \textbf{Instruction Cycle}: Complete execution of one instruction
\item
  \textbf{Machine Cycle}: Group of T-states required for one memory
  operation
\end{itemize}

\textbf{Relationship:}

\begin{lstlisting}
Instruction Cycle = Multiple Machine Cycles
Machine Cycle = Multiple T-States (3-6 T-states)
\end{lstlisting}

\end{solutionbox}
\begin{mnemonicbox}
``TIM - T-state, Instruction cycle, Machine cycle''

\end{mnemonicbox}
\subsection*{Question 2(B) OR [4
marks]}\label{q2b}

\textbf{Explain De-Multiplexing Of Address And Data Bus Of 8085.}

\begin{solutionbox}

\includegraphics[width=1\linewidth,height=\textheight,keepaspectratio]{mermaid-fd48dd71.pdf}

\textbf{Process:}

\begin{itemize}
\tightlist
\item
  \textbf{Step 1}: During T1, AD0-AD7 contains lower 8-bit address
\item
  \textbf{Step 2}: ALE goes high, latches address in external latch
\item
  \textbf{Step 3}: AD0-AD7 becomes data bus for remaining T-states
\end{itemize}

\textbf{Components Required:}

\begin{itemize}
\tightlist
\item
  \textbf{74LS373}: Octal latch IC for address latching
\item
  \textbf{ALE}: Address Latch Enable signal for timing
\end{itemize}

\end{solutionbox}
\begin{mnemonicbox}
``LAD - Latch Address with Data separation''

\end{mnemonicbox}
\subsection*{Question 2(C) OR [7
marks]}\label{q2c}

\textbf{Draw And Explain Flag Register Of 8085.}

\begin{solutionbox}

\begin{lstlisting}
    D7   D6   D5   D4   D3   D2   D1   D0
   +----+----+----+----+----+----+----+----+
   | S  | Z  | X  | AC | X  | P  | X  | CY |
   +----+----+----+----+----+----+----+----+
\end{lstlisting}

\textbf{Flag Descriptions:}

\begin{itemize}
\tightlist
\item
  \textbf{CY (D0)}: Carry flag - Set when carry occurs
\item
  \textbf{P (D2)}: Parity flag - Set for even parity\\
\item
  \textbf{AC (D4)}: Auxiliary carry - Set for BCD operations
\item
  \textbf{Z (D6)}: Zero flag - Set when result is zero
\item
  \textbf{S (D7)}: Sign flag - Set when result is negative
\end{itemize}

\textbf{Flag Operations:}

\begin{itemize}
\tightlist
\item
  \textbf{Conditional Jumps}: Based on flag status (JZ, JC, JP)
\item
  \textbf{Arithmetic Results}: Automatically updated after ALU
  operations
\end{itemize}

\end{solutionbox}
\begin{mnemonicbox}
``SZAPC - Sign, Zero, Auxiliary, Parity, Carry''

\end{mnemonicbox}
\subsection*{Question 3(A) [3 marks]}\label{q3a}

\textbf{What Is SFR ? List Out Any Three SFR.}

\begin{solutionbox}

\textbf{SFR Definition:} \textbf{Special Function Register} - Dedicated
registers with specific functions in microcontroller

\textbf{Three SFRs:}

\begin{itemize}
\tightlist
\item
  \textbf{ACC (E0H)}: Accumulator register
\item
  \textbf{PSW (D0H)}: Program Status Word\\
\item
  \textbf{SP (81H)}: Stack Pointer register
\end{itemize}

\textbf{Characteristics:}

\begin{itemize}
\tightlist
\item
  \textbf{Address Range}: 80H to FFH in internal RAM
\item
  \textbf{Bit Addressable}: Some SFRs allow individual bit access
\item
  \textbf{Function Specific}: Each has dedicated hardware function
\end{itemize}

\end{solutionbox}
\begin{mnemonicbox}
``APS - ACC, PSW, Stack Pointer''

\end{mnemonicbox}
\subsection*{Question 3(B) [4 marks]}\label{q3b}

\textbf{Explain Program Counter (PC) And Data Pointer (DPTR) Register.}

\begin{solutionbox}

\textbf{Program Counter (PC):}

\begin{itemize}
\tightlist
\item
  \textbf{Size}: 16-bit register
\item
  \textbf{Function}: Holds address of next instruction to be executed
\item
  \textbf{Auto-increment}: Automatically increments after instruction
  fetch
\item
  \textbf{Range}: 0000H to FFFFH
\end{itemize}

\textbf{Data Pointer (DPTR):}

\begin{itemize}
\tightlist
\item
  \textbf{Size}: 16-bit register (DPH + DPL)
\item
  \textbf{Function}: Points to external data memory locations
\item
  \textbf{Usage}: Used with MOVX instructions for external memory access
\item
  \textbf{Components}: DPH (83H) and DPL (82H)
\end{itemize}

\begin{lstlisting}
PC:   +--------+--------+
      |   PCH  |   PCL  |  16-bit
      +--------+--------+

DPTR: +--------+--------+
      |   DPH  |   DPL  |  16-bit  
      +--------+--------+
      |  83H   |  82H   |
\end{lstlisting}

\end{solutionbox}
\begin{mnemonicbox}
``PD - PC Points to Program, DPTR Points to Data''

\end{mnemonicbox}
\subsection*{Question 3(C) [7 marks]}\label{q3c}

\textbf{Draw And Explain Architecture Of 8051.}

\begin{solutionbox}

\includegraphics[width=1\linewidth,height=\textheight,keepaspectratio]{mermaid-e7dc6a65.pdf}

\textbf{Architecture Components:}

\begin{itemize}
\tightlist
\item
  \textbf{CPU}: 8-bit ALU with accumulator and B register
\item
  \textbf{Memory}: 4KB internal ROM, 128B internal RAM
\item
  \textbf{I/O Ports}: Four 8-bit bidirectional ports (P0-P3)
\item
  \textbf{Timers}: Two 16-bit timers/counters (T0, T1)
\item
  \textbf{Serial Port}: Full duplex UART for communication
\item
  \textbf{Interrupts}: 5 interrupt sources with priority levels
\end{itemize}

\textbf{Special Features:}

\begin{itemize}
\tightlist
\item
  \textbf{Boolean Processor}: Bit manipulation capabilities
\item
  \textbf{Addressing Modes}: 8 different addressing modes
\item
  \textbf{Power Management}: Idle and power-down modes
\end{itemize}

\end{solutionbox}
\begin{mnemonicbox}
``MIPTIS - Memory, I/O, Processor, Timers,
Interrupts, Serial''

\end{mnemonicbox}
\subsection*{Question 3(A) OR [3
marks]}\label{q3a}

\textbf{Explain Following Pins Of 8051: (1) ALE (2) PSEN (3) XTAL1 \&
XTAL2}

\begin{solutionbox}

\textbf{Pin Functions:}

\begin{itemize}
\tightlist
\item
  \textbf{ALE (Pin 30)}: Address Latch Enable

  \begin{itemize}
  \tightlist
  \item
    Output pulse for latching lower address byte
  \item
    Active high signal at 1/6 of oscillator frequency
  \end{itemize}
\item
  \textbf{PSEN (Pin 29)}: Program Store Enable

  \begin{itemize}
  \tightlist
  \item
    Active low output for external program memory read
  \item
    Connected to OE pin of external EPROM
  \end{itemize}
\item
  \textbf{XTAL1 \& XTAL2 (Pins 19, 18)}: Crystal connections

  \begin{itemize}
  \tightlist
  \item
    Connect external crystal for clock generation
  \item
    Typical frequency: 11.0592 MHz or 12 MHz
  \end{itemize}
\end{itemize}

\begin{lstlisting}
Crystal Oscillator Connection:
    
    XTAL1 ----[Crystal]---- XTAL2
      |                      |
     [C1]                   [C2]
      |                      |
     GND                    GND
\end{lstlisting}

\end{solutionbox}
\begin{mnemonicbox}
``APX - ALE latches Address, PSEN enables Program,
XTAL generates clocK''

\end{mnemonicbox}
\subsection*{Question 3(B) OR [4
marks]}\label{q3b}

\textbf{Describe Internal RAM Organization Of 8051 Microcontroller.}

\begin{solutionbox}

\begin{lstlisting}
    8051 Internal RAM Organization (128 Bytes)
    
    7FH +------------------------+
        |    General Purpose     |  
        |    Scratch Pad Area    |  78H-7FH (8 bytes)
    78H +------------------------+
        |                        |
        |    General Purpose     |  30H-77H (72 bytes)  
        |    Data Memory         |
    30H +------------------------+
        |  Bank 3 (R0-R7)        |  18H-1FH
    20H +------------------------+
        |  Bank 2 (R0-R7)        |  10H-17H  
    18H +------------------------+
        |  Bank 1 (R0-R7)        |  08H-0FH
    10H +------------------------+
        |  Bank 0 (R0-R7)        |  00H-07H
    08H +------------------------+
        |  Default Register Bank |
    00H +------------------------+
\end{lstlisting}

\textbf{RAM Sections:}

\begin{itemize}
\tightlist
\item
  \textbf{Register Banks}: 4 banks \times 8 registers each (00H-1FH)
\item
  \textbf{Bit Addressable}: 16 bytes with individual bit access
  (20H-2FH)\\
\item
  \textbf{General Purpose}: 80 bytes for user data (30H-7FH)
\item
  \textbf{Stack Area}: Usually starts from 08H upward
\end{itemize}

\textbf{Addressing:}

\begin{itemize}
\tightlist
\item
  \textbf{Direct}: Using actual address (MOV 30H, A)
\item
  \textbf{Indirect}: Using register pointer (MOV @R0, A)
\end{itemize}

\end{solutionbox}
\begin{mnemonicbox}
``RBGS - Register banks, Bit addressable, General
purpose, Stack''

\end{mnemonicbox}
\subsection*{Question 3(C) OR [7
marks]}\label{q3c}

\textbf{Draw Pin Diagram Of 8051 And Explain Any 04(Four) Pins.}

\begin{solutionbox}

\begin{lstlisting}
                    8051 PIN DIAGRAM
                    
    P1.0     1 +-------+ 40  Vcc
    P1.1     2 |       | 39  P0.0/AD0
    P1.2     3 |       | 38  P0.1/AD1  
    P1.3     4 |       | 37  P0.2/AD2
    P1.4     5 |  8051 | 36  P0.3/AD3
    P1.5     6 |       | 35  P0.4/AD4
    P1.6     7 |       | 34  P0.5/AD5
    P1.7     8 |       | 33  P0.6/AD6
    RESET    9 |       | 32  P0.7/AD7
   P3.0/RXD  10|       | 31  EA/VPP
   P3.1/TXD  11|       | 30  ALE/PROG
   P3.2/INT0 12|       | 29  PSEN
   P3.3/INT1 13|       | 28  P2.7/A15
   P3.4/T0   14|       | 27  P2.6/A14
   P3.5/T1   15|       | 26  P2.5/A13
   P3.6/WR   16|       | 25  P2.4/A12
   P3.7/RD   17|       | 24  P2.3/A11
   XTAL2     18|       | 23  P2.2/A10
   XTAL1     19|       | 22  P2.1/A9
    VSS      20+-------+ 21  P2.0/A8
\end{lstlisting}

\textbf{Pin Explanations:}

\begin{itemize}
\tightlist
\item
  \textbf{RESET (Pin 9)}: Reset input - Active high, initializes
  microcontroller
\item
  \textbf{EA/VPP (Pin 31)}: External Access - Controls program memory
  selection
\item
  \textbf{P0 (Pins 32-39)}: Port 0 - Multiplexed address/data bus for
  external memory
\item
  \textbf{P2 (Pins 21-28)}: Port 2 - High-order address bus for external
  memory
\end{itemize}

\end{solutionbox}
\begin{mnemonicbox}
``REPP - REset, External Access, Port 0, Port 2''

\end{mnemonicbox}
\subsection*{Question 4(A) [3 marks]}\label{q4a}

\textbf{Write A Program To Multiply Data Stored In R0 Register With Data
Stored In R1 Register. Store The Result In R2 Register (LSB) And R3
Register (MSB).}

\begin{solutionbox}

\begin{lstlisting}
ORG 0000H
MOV R0, #05H    ; Load first number
MOV R1, #03H    ; Load second number  
MOV A, R0       ; Move R0 to accumulator
MOV B, R1       ; Move R1 to B register
MUL AB          ; Multiply A and B
MOV R2, A       ; Store LSB in R2
MOV R3, B       ; Store MSB in R3
END
\end{lstlisting}

\textbf{Program Flow:}

\begin{itemize}
\tightlist
\item
  \textbf{Load operands} into R0 and R1
\item
  \textbf{Transfer} to A and B registers for multiplication
\item
  \textbf{Execute} MUL AB instruction
\item
  \textbf{Store} 16-bit result (A=LSB, B=MSB)
\end{itemize}

\textbf{Result Storage:}

\begin{itemize}
\tightlist
\item
  \textbf{R2}: Contains lower 8 bits of product
\item
  \textbf{R3}: Contains upper 8 bits of product
\end{itemize}

\end{solutionbox}
\begin{mnemonicbox}
``LTSE - Load, Transfer, multiply, Store result''

\end{mnemonicbox}
\subsection*{Question 4(B) [4 marks]}\label{q4b}

\textbf{List Out Data Transfer Instructions And Explain Any Two Data
Transfer Instructions With Suitable Examples.}

\begin{solutionbox}

\textbf{Data Transfer Instructions:}

{\def\LTcaptype{none} % do not increment counter
\begin{longtable}[]{@{}ll@{}}
\toprule\noalign{}
Instruction & Function \\
\midrule\noalign{}
\endhead
\bottomrule\noalign{}
\endlastfoot
MOV & Move data between registers/memory \\
MOVX & Move data to/from external memory \\
MOVC & Move code byte to accumulator \\
PUSH & Push data onto stack \\
POP & Pop data from stack \\
XCH & Exchange accumulator with register \\
XCHD & Exchange lower nibble \\
\end{longtable}
}

\textbf{Detailed Examples:}

\textbf{1. MOV Instruction:}

\begin{lstlisting}
MOV A, #50H     ; Load immediate data 50H into accumulator
MOV R0, A       ; Copy accumulator content to R0
MOV 30H, A      ; Store accumulator content at address 30H
\end{lstlisting}

\textbf{2. PUSH/POP Instructions:}

\begin{lstlisting}
PUSH ACC        ; Push accumulator onto stack
PUSH 00H        ; Push R0 content onto stack  
POP 01H         ; Pop stack content to R1
POP ACC         ; Pop stack content to accumulator
\end{lstlisting}

\end{solutionbox}
\begin{mnemonicbox}
``Move Makes Programs Possible - MOV, MOVX, PUSH,
POP''

\end{mnemonicbox}
\subsection*{Question 4(C) [7 marks]}\label{q4c}

\textbf{Define And Explain Addressing Modes Of 8051.}

\begin{solutionbox}

\textbf{8051 Addressing Modes:}

{\def\LTcaptype{none} % do not increment counter
\begin{longtable}[]{@{}
  >{\raggedright\arraybackslash}p{(\linewidth - 6\tabcolsep) * \real{0.1667}}
  >{\raggedright\arraybackslash}p{(\linewidth - 6\tabcolsep) * \real{0.3611}}
  >{\raggedright\arraybackslash}p{(\linewidth - 6\tabcolsep) * \real{0.2500}}
  >{\raggedright\arraybackslash}p{(\linewidth - 6\tabcolsep) * \real{0.2222}}@{}}
\toprule\noalign{}
\begin{minipage}[b]{\linewidth}\raggedright
Mode
\end{minipage} & \begin{minipage}[b]{\linewidth}\raggedright
Description
\end{minipage} & \begin{minipage}[b]{\linewidth}\raggedright
Example
\end{minipage} & \begin{minipage}[b]{\linewidth}\raggedright
Usage
\end{minipage} \\
\midrule\noalign{}
\endhead
\bottomrule\noalign{}
\endlastfoot
\textbf{Immediate} & Data is part of instruction & MOV A, \#50H &
Constant values \\
\textbf{Register} & Uses register directly & MOV A, R0 & Fast access \\
\textbf{Direct} & Uses direct address & MOV A, 30H & RAM locations \\
\textbf{Indirect} & Uses register as pointer & MOV A, @R0 & Dynamic
addressing \\
\textbf{Indexed} & Base + offset addressing & MOVC A, @A+DPTR & Table
lookup \\
\textbf{Relative} & PC + offset & SJMP LOOP & Branch instructions \\
\textbf{Absolute} & Direct jump address & LJMP 1000H & Long jumps \\
\textbf{Bit} & Individual bit access & SETB P1.0 & Control operations \\
\end{longtable}
}

\textbf{Detailed Examples:}

\begin{lstlisting}
; Immediate Addressing
MOV A, #25H         ; Load 25H into A

; Register Addressing  
MOV A, R1           ; Copy R1 to A

; Direct Addressing
MOV A, 40H          ; Load from address 40H

; Indirect Addressing
MOV R0, #40H        ; R0 points to 40H
MOV A, @R0          ; Load from address pointed by R0

; Indexed Addressing
MOV DPTR, #TABLE    ; Point to table
MOV A, #02H         ; Index value
MOVC A, @A+DPTR     ; Load from TABLE+2
\end{lstlisting}

\end{solutionbox}
\begin{mnemonicbox}
``IRIDRAB - Immediate, Register, Indirect, Direct,
Relative, Absolute, Bit''

\end{mnemonicbox}
\subsection*{Question 4(A) OR [3
marks]}\label{q4a}

\textbf{Write A Program To Find 2's Complement of Data Stored in R0
Register.}

\begin{solutionbox}

\begin{lstlisting}
ORG 0000H
MOV R0, #85H        ; Load test data
MOV A, R0           ; Copy data to accumulator
CPL A               ; Complement all bits (1's complement)
INC A               ; Add 1 to get 2's complement
MOV R1, A           ; Store result in R1
END
\end{lstlisting}

\textbf{Algorithm:}

\begin{itemize}
\tightlist
\item
  \textbf{Step 1}: Load data from R0 to accumulator
\item
  \textbf{Step 2}: Complement all bits using CPL A
\item
  \textbf{Step 3}: Add 1 using INC A for 2's complement
\item
  \textbf{Step 4}: Store result back
\end{itemize}

\textbf{Verification:}

\begin{lstlisting}
Original: 85H = 10000101B
1's Comp: 7AH = 01111010B  
2's Comp: 7BH = 01111011B
\end{lstlisting}

\end{solutionbox}
\begin{mnemonicbox}
``CCI - Complement, aCd 1, Include result''

\end{mnemonicbox}
\subsection*{Question 4(B) OR [4
marks]}\label{q4b}

\textbf{List Logical Instructions And Explain Any Two Logical
Instructions With Suitable Examples.}

\begin{solutionbox}

\textbf{Logical Instructions:}

{\def\LTcaptype{none} % do not increment counter
\begin{longtable}[]{@{}ll@{}}
\toprule\noalign{}
Instruction & Function \\
\midrule\noalign{}
\endhead
\bottomrule\noalign{}
\endlastfoot
ANL & Logical AND operation \\
ORL & Logical OR operation \\
XRL & Logical XOR operation \\
CPL & Complement operation \\
RL/RLC & Rotate left \\
RR/RRC & Rotate right \\
SWAP & Swap nibbles \\
\end{longtable}
}

\textbf{Detailed Examples:}

\textbf{1. ANL (AND Logic):}

\begin{lstlisting}
MOV A, #0F0H        ; A = 11110000B
ANL A, #0AAH        ; AND with 10101010B
; Result:

A = 10100000B = A0H

\end{lstlisting}

\textbf{Usage}: Masking specific bits, clearing unwanted bits

\textbf{2. ORL (OR Logic):}

\begin{lstlisting}
MOV A, #0F0H        ; A = 11110000B  
ORL A, #00FH        ; OR with 00001111B
; Result:

A = 11111111B = FFH

\end{lstlisting}

\textbf{Usage}: Setting specific bits, combining bit patterns

\end{solutionbox}
\begin{mnemonicbox}
``AXOR - AND masks, XOR toggles, OR sets, Rotate
shifts''

\end{mnemonicbox}
\subsection*{Question 4(C) OR [7
marks]}\label{q4c}

\textbf{Explain Following Instructions: (1)ADDC (2) INC (3) DEC (4) JZ
(5) SUBB (6) NOP (7) RET}

\begin{solutionbox}

\textbf{Instruction Explanations:}

\textbf{1. ADDC (Add with Carry):}

\begin{lstlisting}
MOV A, #80H
ADDC A, #90H    ; A = A + 90H + Carry flag
\end{lstlisting}

\textbf{Function}: Adds source, destination, and carry flag

\textbf{2. INC (Increment):}

\begin{lstlisting}
INC A           ; A = A + 1
INC R0          ; R0 = R0 + 1  
INC 30H         ; (30H) = (30H) + 1
\end{lstlisting}

\textbf{Function}: Increases operand by 1

\textbf{3. DEC (Decrement):}

\begin{lstlisting}
DEC A           ; A = A - 1
DEC R1          ; R1 = R1 - 1
DEC 40H         ; (40H) = (40H) - 1  
\end{lstlisting}

\textbf{Function}: Decreases operand by 1

\textbf{4. JZ (Jump on Zero):}

\begin{lstlisting}
DEC A
JZ ZERO_LABEL   ; Jump if A = 0
\end{lstlisting}

\textbf{Function}: Conditional jump when zero flag is set

\textbf{5. SUBB (Subtract with Borrow):}

\begin{lstlisting}
MOV A, #50H
SUBB A, #30H    ; A = A - 30H - Carry flag
\end{lstlisting}

\textbf{Function}: Subtracts source and carry from accumulator

\textbf{6. NOP (No Operation):}

\begin{lstlisting}
NOP             ; Do nothing, consume 1 cycle
\end{lstlisting}

\textbf{Function}: Provides timing delay, placeholder

\textbf{7. RET (Return):}

\begin{lstlisting}
CALL SUBROUTINE
...
SUBROUTINE:
    MOV A, #10H
    RET         ; Return to caller
\end{lstlisting}

\textbf{Function}: Returns from subroutine to calling address

\end{solutionbox}
\begin{mnemonicbox}
``AIDS NR - Add, Increment, Decrement, Subtract,
No-op, Return''

\end{mnemonicbox}
\subsection*{Question 5(A) [3 marks]}\label{q5a}

\textbf{Explain DJNZ And CJNE Instructions With Suitable Examples.}

\begin{solutionbox}

\textbf{DJNZ (Decrement and Jump if Not Zero):}

\begin{lstlisting}
MOV R0, #05H        ; Initialize counter
LOOP:
    MOV A, #00H     ; Some operation
    DJNZ R0, LOOP   ; Decrement R0, jump if not zero
\end{lstlisting}

\textbf{Function}: Combines decrement and conditional jump operations

\textbf{CJNE (Compare and Jump if Not Equal):}

\begin{lstlisting}
MOV A, #30H
CJNE A, #30H, NOT_EQUAL  ; Compare A with 30H
MOV R0, #01H             ; Equal case
SJMP CONTINUE
NOT_EQUAL:
    MOV R0, #00H         ; Not equal case
CONTINUE:
\end{lstlisting}

\textbf{Function}: Compares two operands and jumps if not equal

\textbf{Applications:}

\begin{itemize}
\tightlist
\item
  \textbf{DJNZ}: Loop control, counting operations
\item
  \textbf{CJNE}: Decision making, condition checking
\end{itemize}

\end{solutionbox}
\begin{mnemonicbox}
``DC - Decrement count, Compare jump''

\end{mnemonicbox}
\subsection*{Question 5(B) [4 marks]}\label{q5b}

\textbf{Write An Assembly Language Program To Generate The Time Delay Of
30ms Using Timer 0. Assume Crystal Frequency Is 12 MHz}

\begin{solutionbox}

\begin{lstlisting}
ORG 0000H
MAIN:
    CALL DELAY_30MS     ; Call 30ms delay
    SJMP MAIN           ; Repeat

DELAY_30MS:
    MOV TMOD, #01H      ; Timer 0, Mode 1 (16-bit)
    MOV TH0, #8AH       ; Load high byte for 30ms
    MOV TL0, #23H       ; Load low byte  
    SETB TR0            ; Start Timer 0
WAIT:
    JNB TF0, WAIT       ; Wait for timer overflow
    CLR TR0             ; Stop timer
    CLR TF0             ; Clear timer flag
    RET
END
\end{lstlisting}

\textbf{Calculation for 30ms delay:}

\begin{lstlisting}
Crystal Frequency = 12 MHz
Machine Cycle = 12/12 MHz = 1 µs
For 30ms = 30,000 µs = 30,000 machine cycles

Timer Count = 65536 - 30000 = 35536 = 8A23H
TH0 = 8AH, TL0 = 23H
\end{lstlisting}

\textbf{Timer Configuration:}

\begin{itemize}
\tightlist
\item
  \textbf{TMOD}: Timer mode register configuration
\item
  \textbf{TH0/TL0}: Timer 0 high/low byte registers
\item
  \textbf{TR0}: Timer 0 run control bit
\item
  \textbf{TF0}: Timer 0 overflow flag
\end{itemize}

\end{solutionbox}
\begin{mnemonicbox}
``CLSW - Calculate, Load, Start, Wait''

\end{mnemonicbox}
\subsection*{Question 5(C) [7 marks]}\label{q5c}

\textbf{Draw The Interfacing Diagram Of LCD With 8051. Explain Pins Of
LCD Which Are Necessary For Interfacing.}

\begin{solutionbox}

\begin{lstlisting}
    8051 to LCD Interfacing (4-bit mode)
    
    8051                      16x2 LCD
    ----                      --------
    P2.7 ---------> D7  (Pin 14)
    P2.6 ---------> D6  (Pin 13)  
    P2.5 ---------> D5  (Pin 12)
    P2.4 ---------> D4  (Pin 11)
    
    P1.2 ---------> EN  (Pin 6)
    P1.1 ---------> RW  (Pin 5)
    P1.0 ---------> RS  (Pin 4)
    
    +5V  ---------> VCC (Pin 2)
    GND  ---------> VSS (Pin 1)
    GND  ---------> VEE (Pin 3) [Contrast]
\end{lstlisting}

\textbf{LCD Pin Functions:}

\begin{itemize}
\tightlist
\item
  \textbf{RS (Pin 4)}: Register Select - 0=Command, 1=Data
\item
  \textbf{RW (Pin 5)}: Read/Write - 0=Write, 1=Read\\
\item
  \textbf{EN (Pin 6)}: Enable - High to low pulse for data transfer
\item
  \textbf{D4-D7 (Pins 11-14)}: 4-bit data lines for commands/data
\end{itemize}

\textbf{Interface Requirements:}

\begin{itemize}
\tightlist
\item
  \textbf{Power Supply}: VCC=+5V, VSS=GND, VEE=Contrast control
\item
  \textbf{Control Lines}: 3 pins (RS, RW, EN) for LCD control
\item
  \textbf{Data Lines}: 4 pins (D4-D7) for 4-bit mode operation
\end{itemize}

\textbf{Basic LCD Commands:}

\begin{itemize}
\tightlist
\item
  \textbf{0x38}: Function set (8-bit, 2 lines)
\item
  \textbf{0x0E}: Display ON, cursor ON
\item
  \textbf{0x01}: Clear display
\item
  \textbf{0x80}: Set cursor to first line
\end{itemize}

\end{solutionbox}
\begin{mnemonicbox}
``REED - RS selects, RW reads, EN enables, Data
transfers''

\end{mnemonicbox}
\subsection*{Question 5(A) OR [3
marks]}\label{q5a}

\textbf{Write A Program To Perform OR Operation On Data Stored In 65h
Memory Location With Data Stored In 75h Memory Location. Store The
Result In R6 Register.}

\begin{solutionbox}

\begin{lstlisting}
ORG 0000H
MOV 65H, #0F0H      ; Store test data at 65H
MOV 75H, #0AAH      ; Store test data at 75H

MOV A, 65H          ; Load data from 65H to accumulator
ORL A, 75H          ; OR with data at 75H
MOV R6, A           ; Store result in R6 register
END
\end{lstlisting}

\textbf{Operation Details:}

\begin{itemize}
\tightlist
\item
  \textbf{Load}: First operand from memory location 65H
\item
  \textbf{OR}: Perform logical OR with second operand at 75H
\item
  \textbf{Store}: Result in R6 register
\end{itemize}

\textbf{Example Calculation:}

\begin{lstlisting}
Data at 65H: F0H = 11110000B
Data at 75H: AAH = 10101010B  
OR Result:   FAH = 11111010B
\end{lstlisting}

\end{solutionbox}
\begin{mnemonicbox}
``LOS - Load, OR, Store result''

\end{mnemonicbox}
\subsection*{Question 5(B) OR [4
marks]}\label{q5b}

\textbf{Write An Assembly Language Program To Generate A Square Wave Of
2khz On P1.3. Crystal Frequency Is 11.0592 Mhz.}

\begin{solutionbox}

\begin{lstlisting}
ORG 0000H
MAIN:
    SETB P1.3           ; Set P1.3 high
    CALL DELAY_250US    ; Delay for half period
    CLR P1.3            ; Set P1.3 low
    CALL DELAY_250US    ; Delay for half period
    SJMP MAIN           ; Repeat continuously

DELAY_250US:
    MOV TMOD, #01H      ; Timer 0, Mode 1
    MOV TH0, #0FEH      ; Load high byte
    MOV TL0, #0CBH      ; Load low byte
    SETB TR0            ; Start Timer 0
WAIT:
    JNB TF0, WAIT       ; Wait for overflow
    CLR TR0             ; Stop timer
    CLR TF0             ; Clear flag
    RET
END
\end{lstlisting}

\textbf{Calculation for 2KHz Square Wave:}

\begin{lstlisting}
Frequency = 2KHz, Period = 500µs
Half Period = 250µs

Crystal = 11.0592 MHz
Machine Cycle = 11.0592/12 = 0.921 MHz = 1.085µs

Timer Count = 250µs / 1.085µs = 230 cycles
Timer Value = 65536 - 230 = 65306 = FECBH
TH0 = FEH, TL0 = CBH
\end{lstlisting}

\textbf{Square Wave Generation:}

\begin{itemize}
\tightlist
\item
  \textbf{High Period}: Set pin high, wait 250µs
\item
  \textbf{Low Period}: Set pin low, wait 250µs
\item
  \textbf{Frequency}: 1/(250µs + 250µs) = 2KHz
\end{itemize}

\end{solutionbox}
\begin{mnemonicbox}
``SCDW - Set high, Clear low, Delay, Wait''

\end{mnemonicbox}
\subsection*{Question 5(C) OR [7
marks]}\label{q5c}

\textbf{Draw \& Explain The Interfacing Of 7-Segment Display With 8051.}

\begin{solutionbox}

\begin{lstlisting}
    8051 to 7-Segment Display Interfacing
    
    8051 Port 1              7-Segment Display
    -----------              -----------------
    P1.0 ----[R]----> a  (Pin 7)     
    P1.1 ----[R]----> b  (Pin 6)      aaaa
    P1.2 ----[R]----> c  (Pin 4)     f    b
    P1.3 ----[R]----> d  (Pin 2)     f    b
    P1.4 ----[R]----> e  (Pin 1)      gggg
    P1.5 ----[R]----> f  (Pin 9)     e    c
    P1.6 ----[R]----> g  (Pin 10)    e    c
    P1.7 ----[R]----> dp (Pin 5)      dddd dp
    
    [R] = Current limiting resistor (330Ω)
    
    For Common Cathode:
    Common pin (Pin 3,8) ---> GND
    
    For Common Anode:  
    Common pin (Pin 3,8) ---> +5V
\end{lstlisting}

\textbf{Display Configuration:}

{\def\LTcaptype{none} % do not increment counter
\begin{longtable}[]{@{}lll@{}}
\toprule\noalign{}
Character & Common Cathode Code & Common Anode Code \\
\midrule\noalign{}
\endhead
\bottomrule\noalign{}
\endlastfoot
0 & 3FH & C0H \\
1 & 06H & F9H \\
2 & 5BH & A4H \\
3 & 4FH & B0H \\
4 & 66H & 99H \\
5 & 6DH & 92H \\
6 & 7DH & 82H \\
7 & 07H & F8H \\
8 & 7FH & 80H \\
9 & 6FH & 90H \\
\end{longtable}
}

\textbf{Sample Program:}

\begin{lstlisting}
ORG 0000H
MOV DPTR, #DIGIT_TABLE  ; Point to lookup table
MOV A, #05H             ; Display digit 5
MOVC A, @A+DPTR         ; Get 7-segment code
MOV P1, A               ; Send to display
SJMP $                  ; Stay here

DIGIT_TABLE:
DB 3FH, 06H, 5BH, 4FH, 66H  ; 0,1,2,3,4
DB 6DH, 7DH, 07H, 7FH, 6FH  ; 5,6,7,8,9
ENDs
\end{lstlisting}

\textbf{Interface Components:}

\begin{itemize}
\tightlist
\item
  \textbf{Current Limiting Resistors}: 330Ω to limit LED current
\item
  \textbf{Common Connection}: Cathode to GND or Anode to +5V
\item
  \textbf{Data Lines}: 8 bits for segments a-g and decimal point
\end{itemize}

\textbf{Multiplexing for Multiple Digits:}

\begin{itemize}
\tightlist
\item
  \textbf{Digit Select}: Additional pins for digit selection
\item
  \textbf{Time Division}: Rapidly switch between digits
\item
  \textbf{Persistence of Vision}: Creates illusion of simultaneous
  display
\end{itemize}

\end{solutionbox}
\begin{mnemonicbox}
``CRAM - Common connection, Resistors limit, Address
segments, Multiplex digits''

\end{mnemonicbox}

\end{document}
