\documentclass[10pt,a4paper]{article}

% content/resources/templates/preamble.tex
\usepackage[margin=0.6in]{geometry}
\author{Milav Dabgar}
\usepackage{amsmath,amssymb,amsthm}
\usepackage{booktabs}
\usepackage{multirow}
\usepackage{xcolor}
\usepackage{tcolorbox}
\tcbuselibrary{breakable,skins}
\usepackage[colorlinks=true,linkcolor=blue]{hyperref}
\usepackage{titlesec}
\usepackage{enumitem}
\usepackage{tikz}
\usepackage{pgfplots}
\usepackage{circuitikz}
\usepackage[version=4]{mhchem}
\usepackage{longtable}
\usepackage{array}
\usepackage{float}
\usepackage{caption}
\usepackage{listings}

\lstset{
  basicstyle=\small\ttfamily,
  breaklines=true,
  breakatwhitespace=false,
  postbreak=\mbox{\textcolor{red}{$\hookrightarrow$}\space},
  float=false,
  numbers=left,
  numberstyle=\tiny\color{gray},
  numbersep=10pt,
  xleftmargin=2em,
  keywordstyle=\color{blue},
  commentstyle=\color{green!60!black},
  stringstyle=\color{purple},
  backgroundcolor=\color{gray!5},
  showstringspaces=false,
  tabsize=2,
  captionpos=b,
  keepspaces=true,
  columns=flexible
}

\pgfplotsset{compat=1.18}
\usetikzlibrary{shapes,arrows,positioning,calc,patterns,decorations.pathmorphing,decorations.markings,arrows.meta}

% Color scheme
\definecolor{headcolor}{RGB}{0,102,204}
\definecolor{keycolor}{RGB}{220,20,60}
\definecolor{solutioncolor}{RGB}{34,139,34}
\definecolor{mnemoniccolor}{RGB}{148,0,211}
\definecolor{codecolor}{RGB}{0,0,100}

% Spacing
\setlength{\parskip}{3pt}
\setlist[itemize]{nosep}
\setlist[enumerate]{nosep}

% Title formatting
\titleformat{\section}{\Large\bfseries\color{headcolor}}{\thesection}{1em}{}
\titleformat{\subsection}{\large\bfseries\color{headcolor}}{\thesubsection}{1em}{}

% Pandoc tightlist compatibility
\providecommand{\tightlist}{%
  \setlength{\itemsep}{0pt}\setlength{\parskip}{0pt}}

% Pandoc longtable compatibility
\newcounter{none}
\def\thenone{}


% content/resources/templates/gujarati-boxes.tex
\usepackage{fontspec}
\usepackage{polyglossia}

% Set Gujarati as main language (document is primarily in Gujarati)
% Note: gloss-gujarati.ldf doesn't exist in polyglossia, but it will use hyphenation patterns
\setdefaultlanguage{gujarati}
\setotherlanguage{english}

% Configure Gujarati font properly
% Use Language=Default to prevent polyglossia from trying to add language-specific features
% that don't exist for Gujarati, which causes "empty feature" warnings
\newfontfamily\gujaratifont[Script=Gujarati,AutoFakeBold=2.5,AutoFakeSlant=0.3]{Noto Sans Gujarati}
\setmainfont[Script=Gujarati,AutoFakeBold=2.5,AutoFakeSlant=0.3]{Noto Sans Gujarati}
% Use Noto Sans Gujarati for monospace to support Gujarati in text
\setmonofont[Scale=0.9]{Noto Sans Gujarati}

% Configure English to use the same font
\newfontfamily\englishfont[Script=Gujarati,AutoFakeBold=2.5,AutoFakeSlant=0.3]{Noto Sans Gujarati}

% Translations for polyglossia
\gappto\captionsgujarati{
  \renewcommand{\tablename}{કોષ્ટક}
  \renewcommand{\figurename}{આકૃતિ}
}

% Helper for TikZ nodes to ensure Gujarati font
\newcommand{\gu}[1]{{\gujaratifont #1}}

% Custom environments
\newtcolorbox{solutionbox}{
    breakable,
    enhanced,
    colback=solutioncolor!5!white,
    colframe=solutioncolor!75!black,
    fonttitle=\bfseries,
    title=જવાબ
}

\newtcolorbox{solutionboxnobreak}{
 colback=solutioncolor!5!white,
 colframe=solutioncolor!75!black,
 fonttitle=\bfseries,
 title=જવાબ
}

\newtcolorbox{keyformula}{
 breakable,
 enhanced,
 colback=keycolor!5!white,
 colframe=keycolor!75!black,
 fonttitle=\bfseries,
 title=રાસાયણિક સમીકરણ/સૂત્ર
}

\newtcolorbox{mnemonicbox}{
 breakable,
 enhanced,
 colback=mnemoniccolor!5!white,
 colframe=mnemoniccolor!75!black,
 fonttitle=\bfseries,
 title=મેમરી ટ્રીક
}


\begin{document}

\begin{center}
{\Huge\bfseries\color{headcolor} Subject Name (Gujarati)}\\[5pt]
{\LARGE 1333202 -- Winter 2023}\\[3pt]
{\large Semester 1 Study Material}\\[3pt]
{\normalsize\textit{Detailed Solutions and Explanations}}
\end{center}

\vspace{10pt}

\subsection*{પ્રશ્ન 1(અ) [3
ગુણ]}\label{uxaaauxab0uxab6uxaa8-1uxa85-3-uxa97uxaa3}

\textbf{માઇક્રોપ્રોસેસર ની વ્યાખ્યા આપો.}

\begin{solutionbox}

માઇક્રોપ્રોસેસર એ એક સિંગલ ચિપ CPU છે જેમાં digital computer ના central
processing unit ના કાર્યો કરવા માટે જરૂરી બધા arithmetic, logic અને control
circuits હોય છે.


{\def\LTcaptype{none} % do not increment counter
\vspace{-5pt}
\captionof{table}{માઇક્રોપ્રોસેસર ની મુખ્ય વિશેષતાઓ}
\vspace{-10pt}
\begin{longtable}[]{@{}ll@{}}
\toprule\noalign{}
વિશેષતા & વર્ણન \\
\midrule\noalign{}
\endhead
\bottomrule\noalign{}
\endlastfoot
\textbf{Single Chip} & એક integrated circuit પર સંપૂર્ણ CPU \\
\textbf{Processing Unit} & instructions execute કરે છે અને calculations કરે
છે \\
\textbf{Control Logic} & system operations અને data flow ને manage કરે છે \\
\end{longtable}
}

\begin{itemize}
\tightlist
\item
  \textbf{Central Processing Unit}: મુખ્ય component જે instructions execute
  કરે છે
\item
  \textbf{Integrated Circuit}: બધા functions એક જ silicon chip પર
  combined
\item
  \textbf{Programmable Device}: stored instructions આધારે વિવિધ programs
  execute કરી શકે છે
\end{itemize}

\end{solutionbox}
\begin{mnemonicbox}
``Single Chip CPU = Smart Computer Processor Unit''

\end{mnemonicbox}
\subsection*{પ્રશ્ન 1(બ) [4
ગુણ]}\label{uxaaauxab0uxab6uxaa8-1uxaac-4-uxa97uxaa3}

\textbf{માઇક્રોપ્રોસેસર ના ફ્લેગ રેજિસ્ટર ને સમજાવો.}

\begin{solutionbox}

Flag register માં ALU દ્વારા કરવામાં આવેલા arithmetic અને logical operations
ના result વિશે status information store થાય છે.


{\def\LTcaptype{none} % do not increment counter
\vspace{-5pt}
\captionof{table}{8085 Flag Register Bits}
\vspace{-10pt}
\begin{longtable}[]{@{}
  >{\raggedright\arraybackslash}p{(\linewidth - 4\tabcolsep) * \real{0.2727}}
  >{\raggedright\arraybackslash}p{(\linewidth - 4\tabcolsep) * \real{0.4545}}
  >{\raggedright\arraybackslash}p{(\linewidth - 4\tabcolsep) * \real{0.2727}}@{}}
\toprule\noalign{}
\begin{minipage}[b]{\linewidth}\raggedright
Flag
\end{minipage} & \begin{minipage}[b]{\linewidth}\raggedright
Position
\end{minipage} & \begin{minipage}[b]{\linewidth}\raggedright
હેતુ
\end{minipage} \\
\midrule\noalign{}
\endhead
\bottomrule\noalign{}
\endlastfoot
\textbf{S (Sign)} & Bit 7 & Result નું sign દશાર્વે છે (1=negative,
0=positive) \\
\textbf{Z (Zero)} & Bit 6 & Result zero હોય ત્યારે set થાય છે \\
\textbf{AC (Auxiliary Carry)} & Bit 4 & Bit 3 થી bit 4 માં carry \\
\textbf{P (Parity)} & Bit 2 & Even parity flag \\
\textbf{CY (Carry)} & Bit 0 & MSB માંથી carry \\
\end{longtable}
}

\begin{itemize}
\tightlist
\item
  \textbf{Status Indicator}: છેલ્લા operation result ની condition બતાવે છે
\item
  \textbf{Conditional Instructions}: Branching અને decision making માટે
  ઉપયોગ થાય છે
\item
  \textbf{5 Active Flags}: Sign, Zero, Auxiliary Carry, Parity અને Carry
  flags
\end{itemize}

\end{solutionbox}
\begin{mnemonicbox}
``Flags Show Zero, Sign, Parity, Auxiliary, Carry''

\end{mnemonicbox}
\subsection*{પ્રશ્ન 1(ક) [7
ગુણ]}\label{uxaaauxab0uxab6uxaa8-1uxa95-7-uxa97uxaa3}

\textbf{માઇક્રોપ્રોસેસર નું instruction format ઉદાહરણ સાથે સમજાવો.}

\begin{solutionbox}

Microprocessor instructions માં opcode અને operand fields હોય છે જે
operation અને data locations specify કરે છે.


{\def\LTcaptype{none} % do not increment counter
\vspace{-5pt}
\captionof{table}{8085 Instruction Format Types}
\vspace{-10pt}
\begin{longtable}[]{@{}llll@{}}
\toprule\noalign{}
Format & Size & Structure & Example \\
\midrule\noalign{}
\endhead
\bottomrule\noalign{}
\endlastfoot
\textbf{1-Byte} & 8 bits & Opcode only & MOV A,B \\
\textbf{2-Byte} & 16 bits & Opcode + 8-bit data & MVI A,05H \\
\textbf{3-Byte} & 24 bits & Opcode + 16-bit address & LDA 2000H \\
\end{longtable}
}

\textbf{ડાયાગ્રામ:}

\includegraphics[width=1\linewidth,height=\textheight,keepaspectratio]{mermaid-515933fa.pdf}

\begin{itemize}
\tightlist
\item
  \textbf{Opcode Field}: કયું operation કરવું છે તે define કરે છે (ADD, MOV,
  JMP)
\item
  \textbf{Operand Field}: Data, register અથવા memory address information
  હોય છે
\item
  \textbf{Variable Length}: Instructions 1, 2 અથવા 3 bytes ની હોઈ શકે છે
\item
  \textbf{Addressing Modes}: Operand location specify કરવાની વિવિધ રીતો
\end{itemize}

\end{solutionbox}
\begin{mnemonicbox}
``Opcode Operations + Operand Objects = Complete
Commands''

\end{mnemonicbox}
\subsection*{પ્રશ્ન 1(ક OR) [7
ગુણ]}\label{uxaaauxab0uxab6uxaa8-1uxa95-or-7-uxa97uxaa3}

\textbf{માઇક્રોપ્રોસેસરમાં ALU, Control Unit અને CPU સમજાવો.}

\begin{solutionbox}

CPU માં ત્રણ મુખ્ય functional units છે જે instructions execute કરવા માટે સાથે
મળીને કામ કરે છે.


{\def\LTcaptype{none} % do not increment counter
\vspace{-5pt}
\captionof{table}{CPU Components અને Functions}
\vspace{-10pt}
\begin{longtable}[]{@{}lll@{}}
\toprule\noalign{}
Component & Primary Function & Key Operations \\
\midrule\noalign{}
\endhead
\bottomrule\noalign{}
\endlastfoot
\textbf{ALU} & Arithmetic \& Logic Operations & ADD, SUB, AND, OR,
XOR \\
\textbf{Control Unit} & Instruction Control & Fetch, Decode, Execute \\
\textbf{CPU} & Overall Processing & બધા operations coordinate કરે છે \\
\end{longtable}
}

\textbf{ડાયાગ્રામ:}

\includegraphics[width=1\linewidth,height=\textheight,keepaspectratio]{mermaid-2715f80d.pdf}

\begin{itemize}
\tightlist
\item
  \textbf{ALU Functions}: બધા arithmetic calculations અને logical
  operations કરે છે
\item
  \textbf{Control Unit Tasks}: Instruction execution cycle manage કરે છે
  અને control signals generate કરે છે
\item
  \textbf{CPU Coordination}: Complete processing માટે ALU અને Control Unit
  ને integrate કરે છે
\end{itemize}

\end{solutionbox}
\begin{mnemonicbox}
``ALU Adds, Control Commands, CPU Coordinates''

\end{mnemonicbox}
\subsection*{પ્રશ્ન 2(અ) [3
ગુણ]}\label{uxaaauxab0uxab6uxaa8-2uxa85-3-uxa97uxaa3}

\textbf{ALE signal નું કાર્ય સમજાવો.}

\begin{solutionbox}

ALE (Address Latch Enable) signal નો ઉપયોગ lower-order address અને data
lines ને demultiplex કરવા માટે થાય છે.


{\def\LTcaptype{none} % do not increment counter
\vspace{-5pt}
\captionof{table}{ALE Signal Functions}
\vspace{-10pt}
\begin{longtable}[]{@{}ll@{}}
\toprule\noalign{}
Function & વર્ણન \\
\midrule\noalign{}
\endhead
\bottomrule\noalign{}
\endlastfoot
\textbf{Address Latching} & Lower 8-bit address capture કરે છે \\
\textbf{Demultiplexing} & Address ને data થી separate કરે છે \\
\textbf{Timing Control} & Timing reference પ્રદાન કરે છે \\
\end{longtable}
}

\textbf{ડાયાગ્રામ:}

\begin{lstlisting}
    +--------+     ALE     +--------+
    |  8085  |------------>| Latch  |
    |        |             | 74373  |
    | AD0-7  |<----------->|        |
    +--------+             +--------+
                               |
                           A0-A7 (Address)
\end{lstlisting}

\begin{itemize}
\tightlist
\item
  \textbf{Active High Signal}: T1 state દરમિયાન ALE high જાય છે
\item
  \textbf{External Latching}: Address hold કરવા માટે 74373 latch સાથે
  ઉપયોગ થાય છે
\item
  \textbf{System Timing}: External devices માટે reference પ્રદાન કરે છે
\end{itemize}

\end{solutionbox}
\begin{mnemonicbox}
``ALE Always Latches External Addresses''

\end{mnemonicbox}
\subsection*{પ્રશ્ન 2(બ) [4
ગુણ]}\label{uxaaauxab0uxab6uxaa8-2uxaac-4-uxa97uxaa3}

\textbf{માઇક્રોપ્રોસેસર અને માઇક્રોકંટ્રોલર ની સરખામણી કરો}

\begin{solutionbox}


{\def\LTcaptype{none} % do not increment counter
\vspace{-5pt}
\captionof{table}{Microprocessor vs Microcontroller Comparison}
\vspace{-10pt}
\begin{longtable}[]{@{}lll@{}}
\toprule\noalign{}
Parameter & Microprocessor & Microcontroller \\
\midrule\noalign{}
\endhead
\bottomrule\noalign{}
\endlastfoot
\textbf{Design} & General purpose & Application specific \\
\textbf{Memory} & External RAM/ROM & Internal RAM/ROM \\
\textbf{I/O Ports} & External interface & Built-in I/O ports \\
\textbf{Timers} & External & Built-in timers \\
\textbf{Cost} & વધુ system cost & ઓછો system cost \\
\textbf{Power} & વધુ consumption & ઓછો consumption \\
\end{longtable}
}

\begin{itemize}
\tightlist
\item
  \textbf{Integration Level}: Microcontroller માં વધુ integrated
  components હોય છે
\item
  \textbf{Application Focus}: Microprocessor computing માટે,
  microcontroller control માટે
\item
  \textbf{System Complexity}: Microprocessor ને વધુ external components
  જોઈએ છે
\item
  \textbf{Design Flexibility}: Microprocessor વધુ expandability આપે છે
\end{itemize}

\end{solutionbox}
\begin{mnemonicbox}
``Microprocessor = More Power, Microcontroller =
More Control''

\end{mnemonicbox}
\subsection*{પ્રશ્ન 2(ક) [7
ગુણ]}\label{uxaaauxab0uxab6uxaa8-2uxa95-7-uxa97uxaa3}

\textbf{માઇક્રોપ્રોસેસરનો બ્લોક ડાયાગ્રામ દોરો અને સમજાવો.}

\begin{solutionbox}

8085 microprocessor માં કેટલાક functional blocks છે જે સાથે મળીને કામ કરે છે.

\textbf{ડાયાગ્રામ:}

\includegraphics[width=1\linewidth,height=\textheight,keepaspectratio]{mermaid-64321ba8.pdf}


{\def\LTcaptype{none} % do not increment counter
\vspace{-5pt}
\captionof{table}{Block Functions}
\vspace{-10pt}
\begin{longtable}[]{@{}ll@{}}
\toprule\noalign{}
Block & Function \\
\midrule\noalign{}
\endhead
\bottomrule\noalign{}
\endlastfoot
\textbf{ALU} & Arithmetic અને logical operations \\
\textbf{Register Array} & Temporary data storage (B,C,D,E,H,L) \\
\textbf{Control Unit} & Instruction execution control \\
\textbf{Address Buffer} & Address bus lines drive કરે છે \\
\end{longtable}
}

\begin{itemize}
\tightlist
\item
  \textbf{Data Path}: Internal bus દ્વારા registers વચ્ચે information flow
  થાય છે
\item
  \textbf{Control Signals}: Timing અને control unit દ્વારા generate થાય છે
\item
  \textbf{Bus Interface}: External memory અને I/O devices સાથે connect કરે
  છે
\item
  \textbf{Register Operations}: Operands અને results માટે temporary
  storage
\end{itemize}

\end{solutionbox}
\begin{mnemonicbox}
``Blocks Build Better Processing Systems''

\end{mnemonicbox}
\subsection*{પ્રશ્ન 2(અ OR) [3
ગુણ]}\label{uxaaauxab0uxab6uxaa8-2uxa85-or-3-uxa97uxaa3}

\textbf{માઇક્રોપ્રોસેસરના 16 bits registers સમજાવો.}

\begin{solutionbox}

8085 માં 8-bit register pairs ને combine કરીને બનેલા ત્રણ 16-bit registers છે.


{\def\LTcaptype{none} % do not increment counter
\vspace{-5pt}
\captionof{table}{16-bit Registers}
\vspace{-10pt}
\begin{longtable}[]{@{}lll@{}}
\toprule\noalign{}
Register & Formation & Purpose \\
\midrule\noalign{}
\endhead
\bottomrule\noalign{}
\endlastfoot
\textbf{PC} & Single 16-bit & Program Counter - next instruction
address \\
\textbf{SP} & Single 16-bit & Stack Pointer - stack ના top નું address \\
\textbf{HL} & H + L registers & Memory pointer - data address \\
\end{longtable}
}

\begin{itemize}
\tightlist
\item
  \textbf{Program Counter}: આપમેળે next instruction પર increment થાય છે
\item
  \textbf{Stack Pointer}: Stack પર last pushed data તરફ point કરે છે
\item
  \textbf{HL Pair}: Memory addressing માટે સૌથી વધુ વપરાતું
\end{itemize}

\end{solutionbox}
\begin{mnemonicbox}
``PC Points Program, SP Stacks Properly, HL Holds
Location''

\end{mnemonicbox}
\subsection*{પ્રશ્ન 2(બ OR) [4
ગુણ]}\label{uxaaauxab0uxab6uxaa8-2uxaac-or-4-uxa97uxaa3}

\textbf{માઇક્રોપ્રોસેસર માં lower order address અને data lines ને
de-multiplexing કરવાનું સમજાવો.}

\begin{solutionbox}

8085 pin count ઘટાડવા માટે lower 8-bit address ને data lines સાથે multiplex
કરે છે.


{\def\LTcaptype{none} % do not increment counter
\vspace{-5pt}
\captionof{table}{Multiplexed Lines}
\vspace{-10pt}
\begin{longtable}[]{@{}lll@{}}
\toprule\noalign{}
Lines & T1 State & T2-T4 States \\
\midrule\noalign{}
\endhead
\bottomrule\noalign{}
\endlastfoot
\textbf{AD0-AD7} & Lower Address A0-A7 & Data D0-D7 \\
\textbf{ALE Signal} & High & Low \\
\end{longtable}
}

\textbf{ડાયાગ્રામ:}

\begin{lstlisting}
           8085
    +----------------+
    |                | ALE
    |      AD0-AD7   |---->+
    |                |     |
    +----------------+     |
            |              |
            |         +----v----+
            +-------->| 74373   |
                      | Latch   |
                      +---------+
                           |
                       A0-A7
\end{lstlisting}

\begin{itemize}
\tightlist
\item
  \textbf{Time Division}: સમાન lines પહેલા address પછી data carry કરે છે
\item
  \textbf{External Latch}: ALE high હોય ત્યારે 74373 address capture કરે છે
\item
  \textbf{Signal Separation}: અલગ address અને data buses બનાવે છે
\end{itemize}

\end{solutionbox}
\begin{mnemonicbox}
``ALE Always Latches External Address Elegantly''

\end{mnemonicbox}
\subsection*{પ્રશ્ન 2(ક OR) [7
ગુણ]}\label{uxaaauxab0uxab6uxaa8-2uxa95-or-7-uxa97uxaa3}

\textbf{8085 નો pin diagram દોરો અને સમજાવો.}

\begin{solutionbox}

8085 એ multiplexed address/data bus વાળું 40-pin microprocessor છે.

\textbf{ડાયાગ્રામ:}

\begin{lstlisting}
        8085 Pin Diagram
    +-------------------+
X1  |1               40| Vcc
X2  |2               39| HOLD
RST |3               38| HLDA  
SOD |4               37| CLK
SID |5               36| RESET
TRAP|6               35| READY
RST7|7               34| IO/M*
RST6|8               33| S1
RST5|9               32| RD*
INTR|10              31| WR*
INTA|11              30| ALE
AD0 |12              29| S0
AD1 |13              28| A15
AD2 |14              27| A14
AD3 |15              26| A13
AD4 |16              25| A12
AD5 |17              24| A11
AD6 |18              23| A10
AD7 |19              22| A9
Vss |20              21| A8
    +-------------------+
\end{lstlisting}


{\def\LTcaptype{none} % do not increment counter
\vspace{-5pt}
\captionof{table}{Pin Groups}
\vspace{-10pt}
\begin{longtable}[]{@{}
  >{\raggedright\arraybackslash}p{(\linewidth - 4\tabcolsep) * \real{0.3043}}
  >{\raggedright\arraybackslash}p{(\linewidth - 4\tabcolsep) * \real{0.2609}}
  >{\raggedright\arraybackslash}p{(\linewidth - 4\tabcolsep) * \real{0.4348}}@{}}
\toprule\noalign{}
\begin{minipage}[b]{\linewidth}\raggedright
Group
\end{minipage} & \begin{minipage}[b]{\linewidth}\raggedright
Pins
\end{minipage} & \begin{minipage}[b]{\linewidth}\raggedright
Function
\end{minipage} \\
\midrule\noalign{}
\endhead
\bottomrule\noalign{}
\endlastfoot
\textbf{Address/Data} & AD0-AD7, A8-A15 & Memory addressing અને data
transfer \\
\textbf{Control} & ALE, RD\emph{, WR}, IO/M* & Bus control signals \\
\textbf{Interrupts} & INTR, RST7-RST5, TRAP & Interrupt handling \\
\textbf{Power} & Vcc, Vss & Power supply connections \\
\end{longtable}
}

\begin{itemize}
\tightlist
\item
  \textbf{Multiplexed Bus}: AD0-AD7 address અને data બંને carry કરે છે
\item
  \textbf{Active Low Signals}: * વાળા signals active low છે
\item
  \textbf{Crystal Connections}: Clock generation માટે X1, X2
\end{itemize}

\end{solutionbox}
\begin{mnemonicbox}
``Forty Pins Provide Perfect Processing Power''

\end{mnemonicbox}
\subsection*{પ્રશ્ન 3(અ) [3
ગુણ]}\label{uxaaauxab0uxab6uxaa8-3uxa85-3-uxa97uxaa3}

\textbf{માઇક્રોકંટ્રોલર ની clock અને reset circuit નો diagram દોરો}

\begin{solutionbox}

8051 ને proper operation માટે external clock અને reset circuits જોઈએ છે.

\textbf{ડાયાગ્રામ:}

\begin{lstlisting}
Clock Circuit:
    +12MHz Crystal
    |
XTAL1 +---||---+ XTAL2
      |       |
     30pF    30pF
      |       |
     GND     GND

Reset Circuit:
     +5V
      |
     10K
      |
RST --+---||---GND
          10µF
\end{lstlisting}


{\def\LTcaptype{none} % do not increment counter
\vspace{-5pt}
\captionof{table}{Circuit Components}
\vspace{-10pt}
\begin{longtable}[]{@{}lll@{}}
\toprule\noalign{}
Component & Value & Purpose \\
\midrule\noalign{}
\endhead
\bottomrule\noalign{}
\endlastfoot
\textbf{Crystal} & 11.0592 MHz & Clock generation \\
\textbf{Capacitors} & 30pF દરેક & Crystal stabilization \\
\textbf{Reset Resistor} & 10KΩ & Reset માટે pull-up \\
\textbf{Reset Capacitor} & 10µF & Power-on reset delay \\
\end{longtable}
}

\begin{itemize}
\tightlist
\item
  \textbf{Clock Frequency}: Serial communication માટે સામાન્ય રીતે 11.0592
  MHz
\item
  \textbf{Reset Duration}: ઓછામાં ઓછા 2 machine cycles માટે high હોવું જોઈએ
\item
  \textbf{Power-on Reset}: Power apply થાય ત્યારે automatic reset
\end{itemize}

\end{solutionbox}
\begin{mnemonicbox}
``Crystals Create Clock, Resistors Reset Reliably''

\end{mnemonicbox}
\subsection*{પ્રશ્ન 3(બ) [4
ગુણ]}\label{uxaaauxab0uxab6uxaa8-3uxaac-4-uxa97uxaa3}

\textbf{માઇક્રોકંટ્રોલર ની આંતરીક RAM સમજાવો.}

\begin{solutionbox}

8051 માં વિવિધ sections માં organize થયેલા 256 bytes નો internal RAM છે.


{\def\LTcaptype{none} % do not increment counter
\vspace{-5pt}
\captionof{table}{Internal RAM Organization}
\vspace{-10pt}
\begin{longtable}[]{@{}lll@{}}
\toprule\noalign{}
Address Range & Size & Purpose \\
\midrule\noalign{}
\endhead
\bottomrule\noalign{}
\endlastfoot
\textbf{00H-1FH} & 32 bytes & Register Banks (4 banks \times 8 registers) \\
\textbf{20H-2FH} & 16 bytes & Bit-addressable area \\
\textbf{30H-7FH} & 80 bytes & General purpose RAM \\
\textbf{80H-FFH} & 128 bytes & Special Function Registers (SFRs) \\
\end{longtable}
}

\textbf{ડાયાગ્રામ:}

\includegraphics[width=1\linewidth,height=\textheight,keepaspectratio]{mermaid-01904707.pdf}

\begin{itemize}
\tightlist
\item
  \textbf{Register Banks}: દરેકમાં 8 registers (R0-R7) વાળા ચાર banks
\item
  \textbf{Bit Addressing}: 20H-2FH area માં individual bits address કરી
  શકાય છે
\item
  \textbf{Stack Area}: સામાન્ય રીતે general purpose RAM area માં હોય છે
\item
  \textbf{Direct Access}: બધા locations direct addressing દ્વારા
  accessible છે
\end{itemize}

\end{solutionbox}
\begin{mnemonicbox}
``RAM Registers, Bits, General, Special Functions''

\end{mnemonicbox}
\subsection*{પ્રશ્ન 3(ક) [7
ગુણ]}\label{uxaaauxab0uxab6uxaa8-3uxa95-7-uxa97uxaa3}

\textbf{8051 નો બ્લોક ડાયાગ્રામ સમજાવો.}

\begin{solutionbox}

8051 microcontroller એક જ chip પર CPU, memory અને I/O integrate કરે છે.

\textbf{ડાયાગ્રામ:}

\includegraphics[width=1\linewidth,height=\textheight,keepaspectratio]{mermaid-6ae60545.pdf}


{\def\LTcaptype{none} % do not increment counter
\vspace{-5pt}
\captionof{table}{મુખ્ય Blocks}
\vspace{-10pt}
\begin{longtable}[]{@{}ll@{}}
\toprule\noalign{}
Block & Function \\
\midrule\noalign{}
\endhead
\bottomrule\noalign{}
\endlastfoot
\textbf{CPU} & Instruction execution અને control \\
\textbf{Memory} & 4KB ROM + 256B RAM \\
\textbf{Timers} & બે 16-bit timer/counters \\
\textbf{I/O Ports} & ચાર 8-bit bidirectional ports \\
\textbf{Serial Port} & Full-duplex UART \\
\textbf{Interrupts} & 5-source interrupt system \\
\end{longtable}
}

\begin{itemize}
\tightlist
\item
  \textbf{Harvard Architecture}: અલગ program અને data memory spaces
\item
  \textbf{Built-in Peripherals}: Timers, serial port, interrupts
  integrated
\item
  \textbf{Expandable}: External memory અને I/O add કરી શકાય છે
\item
  \textbf{Control Applications}: Embedded control tasks માટે optimized
\end{itemize}

\end{solutionbox}
\begin{mnemonicbox}
``Complete Control Chip Contains CPU, Memory, I/O''

\end{mnemonicbox}
\subsection*{પ્રશ્ન 3(અ OR) [3
ગુણ]}\label{uxaaauxab0uxab6uxaa8-3uxa85-or-3-uxa97uxaa3}

\textbf{DPTR અને PC નું કાર્ય સમજાવો.}

\begin{solutionbox}

DPTR અને PC એ memory addressing માટે 8051 માં મહત્વપૂર્ણ 16-bit registers છે.


{\def\LTcaptype{none} % do not increment counter
\vspace{-5pt}
\captionof{table}{DPTR અને PC Functions}
\vspace{-10pt}
\begin{longtable}[]{@{}lll@{}}
\toprule\noalign{}
Register & Full Form & Function \\
\midrule\noalign{}
\endhead
\bottomrule\noalign{}
\endlastfoot
\textbf{DPTR} & Data Pointer & External data memory તરફ point કરે છે \\
\textbf{PC} & Program Counter & Next instruction address તરફ point કરે
છે \\
\end{longtable}
}

\begin{itemize}
\tightlist
\item
  \textbf{DPTR Usage}: External RAM અને lookup tables access કરવા માટે
\item
  \textbf{PC Function}: Instruction fetch પછી આપમેળે increment થાય છે
\item
  \textbf{16-bit Addressing}: બંને 64KB memory space address કરી શકે છે
\end{itemize}

\end{solutionbox}
\begin{mnemonicbox}
``DPTR Data Pointer, PC Program Counter''

\end{mnemonicbox}
\subsection*{પ્રશ્ન 3(બ OR) [4
ગુણ]}\label{uxaaauxab0uxab6uxaa8-3uxaac-or-4-uxa97uxaa3}

\textbf{માઇક્રોકંટ્રોલરમાં timer ના અલગ અલગ modes સમજાવો.}

\begin{solutionbox}

8051 માં ચાર અલગ operating modes સાથે બે timers છે.


{\def\LTcaptype{none} % do not increment counter
\vspace{-5pt}
\captionof{table}{Timer Modes}
\vspace{-10pt}
\begin{longtable}[]{@{}lll@{}}
\toprule\noalign{}
Mode & Configuration & Purpose \\
\midrule\noalign{}
\endhead
\bottomrule\noalign{}
\endlastfoot
\textbf{Mode 0} & 13-bit timer & 8048 સાથે compatible \\
\textbf{Mode 1} & 16-bit timer & Maximum count capability \\
\textbf{Mode 2} & 8-bit auto-reload & Constant time intervals \\
\textbf{Mode 3} & બે 8-bit timers & Timer 0 split operation \\
\end{longtable}
}

\begin{itemize}
\tightlist
\item
  \textbf{Mode Selection}: TMOD register bits દ્વારા control થાય છે
\item
  \textbf{Timer 0/1}: બંને timers modes 0, 1, 2 support કરે છે
\item
  \textbf{Mode 3 Special}: ફક્ત Timer 0 જ mode 3 માં operate કરી શકે છે
\item
  \textbf{Applications}: Delays, baud rate generation, event counting
\end{itemize}

\end{solutionbox}
\begin{mnemonicbox}
``Modes Make Timers Tremendously Versatile''

\end{mnemonicbox}
\subsection*{પ્રશ્ન 3(ક OR) [7
ગુણ]}\label{uxaaauxab0uxab6uxaa8-3uxa95-or-7-uxa97uxaa3}

\textbf{માઇક્રોકંટ્રોલર ની interrupts સમજાવો.}

\begin{solutionbox}

8051 માં external events handle કરવા માટે 5-source interrupt system છે.


{\def\LTcaptype{none} % do not increment counter
\vspace{-5pt}
\captionof{table}{8051 Interrupt Sources}
\vspace{-10pt}
\begin{longtable}[]{@{}llll@{}}
\toprule\noalign{}
Interrupt & Vector Address & Priority & Trigger \\
\midrule\noalign{}
\endhead
\bottomrule\noalign{}
\endlastfoot
\textbf{Reset} & 0000H & સૌથી વધુ & Power-on/External \\
\textbf{External 0} & 0003H & વધુ & INT0 pin \\
\textbf{Timer 0} & 000BH & મધ્યમ & Timer 0 overflow \\
\textbf{External 1} & 0013H & મધ્યમ & INT1 pin \\
\textbf{Timer 1} & 001BH & ઓછું & Timer 1 overflow \\
\textbf{Serial} & 0023H & સૌથી ઓછું & Serial communication \\
\end{longtable}
}

\textbf{ડાયાગ્રામ:}

\includegraphics[width=1\linewidth,height=\textheight,keepaspectratio]{mermaid-0bc4460e.pdf}

\begin{itemize}
\tightlist
\item
  \textbf{Interrupt Enable}: IE register individual interrupt enables
  control કરે છે
\item
  \textbf{Priority Control}: IP register interrupt priorities set કરે છે
\item
  \textbf{Vector Addresses}: દરેક interrupt નું fixed vector location છે
\item
  \textbf{Nested Interrupts}: વધુ priority ઓછી priority ને interrupt કરી
  શકે છે
\end{itemize}

\end{solutionbox}
\begin{mnemonicbox}
``Five Interrupt Sources Serve System Efficiently''

\end{mnemonicbox}
\subsection*{પ્રશ્ન 4(અ) [3
ગુણ]}\label{uxaaauxab0uxab6uxaa8-4uxa85-3-uxa97uxaa3}

\textbf{8051 ની data transfer instruction ઉદાહરણ આપી સમજાવો.}

\begin{solutionbox}

Data transfer instructions registers, memory અને I/O ports વચ્ચે data move
કરે છે.


{\def\LTcaptype{none} % do not increment counter
\vspace{-5pt}
\captionof{table}{Data Transfer Instructions}
\vspace{-10pt}
\begin{longtable}[]{@{}lll@{}}
\toprule\noalign{}
Instruction & Example & Function \\
\midrule\noalign{}
\endhead
\bottomrule\noalign{}
\endlastfoot
\textbf{MOV} & MOV A,\#55H & Immediate data ને accumulator માં move કરે
છે \\
\textbf{MOVX} & MOVX A,@DPTR & External RAM ને accumulator માં move કરે
છે \\
\textbf{MOVC} & MOVC A,@A+PC & Code memory ને accumulator માં move કરે છે \\
\end{longtable}
}

\begin{itemize}
\tightlist
\item
  \textbf{MOV Variants}: Register to register, immediate to register
\item
  \textbf{External Access}: External RAM operations માટે MOVX
\item
  \textbf{Code Access}: Program memory tables read કરવા માટે MOVC
\end{itemize}

\end{solutionbox}
\begin{mnemonicbox}
``MOV Moves data, MOVX eXternal, MOVC Code''

\end{mnemonicbox}
\subsection*{પ્રશ્ન 4(બ) [4
ગુણ]}\label{uxaaauxab0uxab6uxaa8-4uxaac-4-uxa97uxaa3}

\textbf{માઇક્રોકંટ્રોલરના addressing modes નું list બનાવી સમજાવો.}

\begin{solutionbox}

8051 flexible data access માટે કેટલાક addressing modes support કરે છે.


{\def\LTcaptype{none} % do not increment counter
\vspace{-5pt}
\captionof{table}{8051 Addressing Modes}
\vspace{-10pt}
\begin{longtable}[]{@{}lll@{}}
\toprule\noalign{}
Mode & Example & વર્ણન \\
\midrule\noalign{}
\endhead
\bottomrule\noalign{}
\endlastfoot
\textbf{Immediate} & MOV A,\#55H & Instruction માં data specify કર્યો છે \\
\textbf{Register} & MOV A,R0 & Register contents ઉપયોગ કરે છે \\
\textbf{Direct} & MOV A,30H & Direct memory address \\
\textbf{Indirect} & MOV A,@R0 & Register માં stored address \\
\textbf{Indexed} & MOVC A,@A+DPTR & Base address plus offset \\
\end{longtable}
}

\begin{itemize}
\tightlist
\item
  \textbf{Immediate Mode}: Instruction માં constant data included છે
\item
  \textbf{Register Mode}: Register file ઉપયોગ કરીને સૌથી ઝડપી execution
\item
  \textbf{Direct Mode}: કોઈપણ internal RAM location access કરે છે
\item
  \textbf{Indirect Mode}: Arrays માટે pointer-based addressing
\item
  \textbf{Indexed Mode}: Table lookup અને array access
\end{itemize}

\end{solutionbox}
\begin{mnemonicbox}
``Immediate, Register, Direct, Indirect, Indexed
Addressing''

\end{mnemonicbox}
\subsection*{પ્રશ્ન 4(ક) [7
ગુણ]}\label{uxaaauxab0uxab6uxaa8-4uxa95-7-uxa97uxaa3}

\textbf{8 data block ને શરુઆત ના address location 100h થી 200h માં copy
કરવાનો program લખો.}

\begin{solutionbox}

\textbf{Assembly Program:}

\begin{lstlisting}
ORG 0000H           ; Start address
MOV R0,#100H        ; Source address pointer
MOV R1,#200H        ; Destination address pointer  
MOV R2,#08H         ; 8 bytes માટે counter

LOOP:
MOV A,@R0           ; Source માંથી data read કરો
MOV @R1,A           ; Destination માં data write કરો
INC R0              ; Source pointer increment કરો
INC R1              ; Destination pointer increment કરો
DJNZ R2,LOOP        ; Counter decrement કરો અને zero નથી તો jump

END                 ; Program નો end
\end{lstlisting}


{\def\LTcaptype{none} % do not increment counter
\vspace{-5pt}
\captionof{table}{Register Usage}
\vspace{-10pt}
\begin{longtable}[]{@{}ll@{}}
\toprule\noalign{}
Register & Purpose \\
\midrule\noalign{}
\endhead
\bottomrule\noalign{}
\endlastfoot
\textbf{R0} & Source address pointer (100H) \\
\textbf{R1} & Destination address pointer (200H) \\
\textbf{R2} & Loop counter (8 bytes) \\
\textbf{A} & Temporary data storage \\
\end{longtable}
}

\begin{itemize}
\tightlist
\item
  \textbf{Indirect Addressing}: Memory access માટે @R0 અને @R1
\item
  \textbf{Loop Control}: DJNZ instruction decrements અને tests કરે છે
\item
  \textbf{Block Transfer}: 8 consecutive bytes efficiently copy કરે છે
\end{itemize}

\end{solutionbox}
\begin{mnemonicbox}
``Read, Write, Increment, Decrement, Jump Loop''

\end{mnemonicbox}
\subsection*{પ્રશ્ન 4(અ OR) [3
ગુણ]}\label{uxaaauxab0uxab6uxaa8-4uxa85-or-3-uxa97uxaa3}

\textbf{બે data bytes ને ઉમેરો અને result ને R0 register માં save કરો.}

\begin{solutionbox}

\textbf{Assembly Program:}

\begin{lstlisting}
ORG 0000H           ; Start address
MOV A,#25H          ; પ્રથમ byte load કરો
ADD A,#35H          ; બીજો byte add કરો
MOV R0,A            ; Result ને R0 માં store કરો
END                 ; Program end
\end{lstlisting}


{\def\LTcaptype{none} % do not increment counter
\vspace{-5pt}
\captionof{table}{Operation Steps}
\vspace{-10pt}
\begin{longtable}[]{@{}lll@{}}
\toprule\noalign{}
Step & Instruction & Result \\
\midrule\noalign{}
\endhead
\bottomrule\noalign{}
\endlastfoot
1 & MOV A,\#25H & A = 25H \\
2 & ADD A,\#35H & A = 5AH \\
3 & MOV R0,A & R0 = 5AH \\
\end{longtable}
}

\begin{itemize}
\tightlist
\item
  \textbf{Addition Result}: 25H + 35H = 5AH
\item
  \textbf{Flag Effects}: Result \textgreater{} FFH હોય તો carry flag set
  થાય છે
\end{itemize}

\end{solutionbox}
\begin{mnemonicbox}
``Move, Add, Move = Simple Addition''

\end{mnemonicbox}
\subsection*{પ્રશ્ન 4(બ OR) [4
ગુણ]}\label{uxaaauxab0uxab6uxaa8-4uxaac-or-4-uxa97uxaa3}

\textbf{Indexed addressing mode ઉદાહરણ સાથે સમજાવો.}

\begin{solutionbox}

Indexed addressing memory access માટે base address plus offset ઉપયોગ કરે
છે.


{\def\LTcaptype{none} % do not increment counter
\vspace{-5pt}
\captionof{table}{Indexed Addressing Details}
\vspace{-10pt}
\begin{longtable}[]{@{}lll@{}}
\toprule\noalign{}
Component & વર્ણન & Example \\
\midrule\noalign{}
\endhead
\bottomrule\noalign{}
\endlastfoot
\textbf{Base Address} & DPTR અથવા PC register & DPTR = 1000H \\
\textbf{Index} & Accumulator contents & A = 05H \\
\textbf{Effective Address} & Base + Index & 1000H + 05H = 1005H \\
\end{longtable}
}

\textbf{Example:}

\begin{lstlisting}
MOV DPTR,#1000H     ; Base address
MOV A,#05H          ; Index value
MOVC A,@A+DPTR      ; Address 1005H માંથી read કરો
\end{lstlisting}

\begin{itemize}
\tightlist
\item
  \textbf{Table Access}: Lookup tables અને arrays માટે આદર્શ
\item
  \textbf{Program Memory}: MOVC ફક્ત code memory માંથી જ read કરે છે
\item
  \textbf{Dynamic Indexing}: Execution દરમિયાન index બદલાઈ શકે છે
\end{itemize}

\end{solutionbox}
\begin{mnemonicbox}
``Base + Index = Dynamic Access''

\end{mnemonicbox}
\subsection*{પ્રશ્ન 4(ક OR) [7
ગુણ]}\label{uxaaauxab0uxab6uxaa8-4uxa95-or-7-uxa97uxaa3}

\textbf{માઇક્રોકંટ્રોલરનું stack operation, PUSH અને POP instruction સમજાવો.}

\begin{solutionbox}

Stack એ temporary data storage માટે ઉપયોગમાં લેવાતું LIFO memory structure છે.


{\def\LTcaptype{none} % do not increment counter
\vspace{-5pt}
\captionof{table}{Stack Operations}
\vspace{-10pt}
\begin{longtable}[]{@{}lll@{}}
\toprule\noalign{}
Operation & Instruction & Function \\
\midrule\noalign{}
\endhead
\bottomrule\noalign{}
\endlastfoot
\textbf{PUSH} & PUSH 30H & Stack પર data store કરે છે \\
\textbf{POP} & POP 30H & Stack માંથી data retrieve કરે છે \\
\textbf{Stack Pointer} & SP register & Stack ના top તરફ point કરે છે \\
\end{longtable}
}

\textbf{ડાયાગ્રામ:}

\begin{lstlisting}
Stack Operation:
    
PUSH પહેલાં:        PUSH 30H પછી:      POP 30H પછી:
SP \rightarrow 07H            SP \rightarrow 08H           SP \rightarrow 07H
     06H                 08H: 30H           06H
     05H                 07H: old           05H
     
Stack memory માં upward grow કરે છે
\end{lstlisting}

\textbf{Example Program:}

\begin{lstlisting}
MOV SP,#30H         ; Stack pointer initialize કરો
PUSH ACC            ; Accumulator save કરો
PUSH B              ; B register save કરો
POP B               ; B register restore કરો
POP ACC             ; Accumulator restore કરો
\end{lstlisting}

\begin{itemize}
\tightlist
\item
  \textbf{LIFO Structure}: Last In, First Out data organization
\item
  \textbf{SP Auto-increment}: Stack pointer આપમેળે adjust થાય છે
\item
  \textbf{Subroutine Calls}: Stack return addresses save કરે છે
\item
  \textbf{Register Preservation}: Register contents save/restore કરે છે
\end{itemize}

\end{solutionbox}
\begin{mnemonicbox}
``PUSH Puts Up, Stack Holds, POP Pulls Out''

\end{mnemonicbox}
\subsection*{પ્રશ્ન 5(અ) [3
ગુણ]}\label{uxaaauxab0uxab6uxaa8-5uxa85-3-uxa97uxaa3}

\textbf{Branching instruction ઉદાહરણ સાથે સમજાવો.}

\begin{solutionbox}

Branching instructions conditions આધારે અથવા unconditionally program flow
alter કરે છે.


{\def\LTcaptype{none} % do not increment counter
\vspace{-5pt}
\captionof{table}{Branching Instructions}
\vspace{-10pt}
\begin{longtable}[]{@{}lll@{}}
\toprule\noalign{}
Type & Instruction & Example \\
\midrule\noalign{}
\endhead
\bottomrule\noalign{}
\endlastfoot
\textbf{Unconditional} & LJMP address & LJMP 2000H \\
\textbf{Conditional} & JZ address & JZ ZERO\_LABEL \\
\textbf{Call/Return} & LCALL address & LCALL SUBROUTINE \\
\end{longtable}
}

\textbf{Example:}

\begin{lstlisting}
MOV A,#00H          ; Zero load કરો
JZ ZERO_FOUND       ; A zero છે તો jump કરો
LJMP CONTINUE       ; Continue તરફ jump કરો
ZERO_FOUND:
    MOV R0,#01H     ; Flag set કરો
CONTINUE:
    NOP             ; Execution continue કરો
\end{lstlisting}

\begin{itemize}
\tightlist
\item
  \textbf{Program Control}: Execution sequence બદલે છે
\item
  \textbf{Conditional Jumps}: Flag register status આધારે
\item
  \textbf{Address Range}: કોઈપણ program memory location પર jump કરી શકે છે
\end{itemize}

\end{solutionbox}
\begin{mnemonicbox}
``Jump Changes Control Flow''

\end{mnemonicbox}
\subsection*{પ્રશ્ન 5(બ) [4
ગુણ]}\label{uxaaauxab0uxab6uxaa8-5uxaac-4-uxa97uxaa3}

\textbf{માઇક્રોકંટ્રોલર સાથે 8 LEDs ને interface કરો અને તેને on અને off કરવા
માટેનો program લખો.}

\begin{solutionbox}

\textbf{Circuit Diagram:}

\begin{lstlisting}
8051        LEDs
P1.0 ----[330Ω]----LED1----+5V
P1.1 ----[330Ω]----LED2----+5V  
P1.2 ----[330Ω]----LED3----+5V
P1.3 ----[330Ω]----LED4----+5V
P1.4 ----[330Ω]----LED5----+5V
P1.5 ----[330Ω]----LED6----+5V
P1.6 ----[330Ω]----LED7----+5V
P1.7 ----[330Ω]----LED8----+5V
\end{lstlisting}

\textbf{Program:}

\begin{lstlisting}
ORG 0000H
MAIN:
    MOV P1,#0FFH        ; બધા LEDs ON કરો
    CALL DELAY          ; Wait કરો
    MOV P1,#00H         ; બધા LEDs OFF કરો  
    CALL DELAY          ; Wait કરો
    SJMP MAIN           ; Repeat કરો

DELAY:
    MOV R0,#0FFH        ; Outer loop counter
LOOP1:
    MOV R1,#0FFH        ; Inner loop counter  
LOOP2:
    DJNZ R1,LOOP2       ; Inner delay loop
    DJNZ R0,LOOP1       ; Outer delay loop
    RET                 ; Return કરો
END
\end{lstlisting}


{\def\LTcaptype{none} % do not increment counter
\vspace{-5pt}
\captionof{table}{Components}
\vspace{-10pt}
\begin{longtable}[]{@{}lll@{}}
\toprule\noalign{}
Component & Value & Purpose \\
\midrule\noalign{}
\endhead
\bottomrule\noalign{}
\endlastfoot
\textbf{Resistor} & 330Ω & Current limiting \\
\textbf{LEDs} & 8 pieces & Visual indicators \\
\textbf{Port} & P1 & 8-bit output port \\
\end{longtable}
}

\begin{itemize}
\tightlist
\item
  \textbf{Current Limiting}: Resistors LEDs ને overcurrent થી protect કરે
  છે
\item
  \textbf{Port Configuration}: LED control માટે P1 ને output port તરીકે
  ઉપયોગ
\item
  \textbf{Delay Routine}: Visible ON/OFF timing બનાવે છે
\end{itemize}

\end{solutionbox}
\begin{mnemonicbox}
``Port Controls LEDs with Resistance and Delay''

\end{mnemonicbox}
\subsection*{પ્રશ્ન 5(ક) [7
ગુણ]}\label{uxaaauxab0uxab6uxaa8-5uxa95-7-uxa97uxaa3}

\textbf{માઇક્રોકંટ્રોલર સાથે LCD ને interface કરો અને ``welcome'' display
કરવાનો program લખો.}

\begin{solutionbox}

\textbf{Circuit Connections:}

\begin{lstlisting}
8051        16x2 LCD
P2.0 --------> D4
P2.1 --------> D5  
P2.2 --------> D6
P2.3 --------> D7
P1.0 --------> RS (Register Select)
P1.1 --------> EN (Enable)
GND  --------> R/W (Write mode)
\end{lstlisting}

\textbf{Program:}

\begin{lstlisting}
ORG 0000H
    CALL LCD_INIT       ; LCD initialize કરો
    CALL DISPLAY_MSG    ; Message display કરો
    SJMP $              ; અહીં stop કરો

LCD_INIT:
    MOV P2,#38H         ; Function set: 8-bit, 2-line
    CALL COMMAND
    MOV P2,#0EH         ; Display ON, Cursor ON
    CALL COMMAND  
    MOV P2,#01H         ; Display clear કરો
    CALL COMMAND
    MOV P2,#06H         ; Entry mode set
    CALL COMMAND
    RET

DISPLAY_MSG:
    MOV DPTR,#MESSAGE   ; Message તરફ point કરો
NEXT_CHAR:
    CLR A
    MOVC A,@A+DPTR      ; Character read કરો
    JZ DONE             ; Zero હોય તો string end
    CALL SEND_CHAR      ; LCD પર character send કરો
    INC DPTR            ; Next character
    SJMP NEXT_CHAR
DONE:
    RET

COMMAND:
    CLR P1.0            ; Command માટે RS = 0
    SETB P1.1           ; EN = 1
    CLR P1.1            ; EN = 0 (pulse)
    CALL DELAY
    RET

SEND_CHAR:
    MOV P2,A            ; Data lines પર character put કરો
    SETB P1.0           ; Data માટે RS = 1
    SETB P1.1           ; EN = 1
    CLR P1.1            ; EN = 0 (pulse)
    CALL DELAY
    RET

DELAY:
    MOV R0,#50          ; Delay routine
DELAY_LOOP:
    MOV R1,#255
DELAY_INNER:
    DJNZ R1,DELAY_INNER
    DJNZ R0,DELAY_LOOP
    RET

MESSAGE:
    DB "WELCOME",0       ; Null terminator સાથે message string
END
\end{lstlisting}


{\def\LTcaptype{none} % do not increment counter
\vspace{-5pt}
\captionof{table}{LCD Interface Pins}
\vspace{-10pt}
\begin{longtable}[]{@{}lll@{}}
\toprule\noalign{}
8051 Pin & LCD Pin & Function \\
\midrule\noalign{}
\endhead
\bottomrule\noalign{}
\endlastfoot
\textbf{P2.0-P2.3} & D4-D7 & 4-bit data lines \\
\textbf{P1.0} & RS & Register select (0=command, 1=data) \\
\textbf{P1.1} & EN & Enable pulse \\
\textbf{GND} & R/W & Read/Write (write માટે ground સાથે tied) \\
\end{longtable}
}

\begin{itemize}
\tightlist
\item
  \textbf{4-bit Mode}: Pins save કરવા માટે ફક્ત upper 4 data lines ઉપયોગ
  કરે છે
\item
  \textbf{Control Signals}: RS command/data select કરે છે, EN timing pulse
  આપે છે
\item
  \textbf{Character Display}: દરેક character ASCII code તરીકે send થાય છે
\item
  \textbf{Initialization}: Proper operation માટે જરૂરી command sequence
\end{itemize}

\end{solutionbox}
\begin{mnemonicbox}
``LCD Displays Characters with Commands and Data''

\end{mnemonicbox}
\subsection*{પ્રશ્ન 5(અ OR) [3
ગુણ]}\label{uxaaauxab0uxab6uxaa8-5uxa85-or-3-uxa97uxaa3}

\textbf{Logical instruction ઉદાહરણ સાથે સમજાવો.}

\begin{solutionbox}

Logical instructions data પર bitwise operations કરે છે.


{\def\LTcaptype{none} % do not increment counter
\vspace{-5pt}
\captionof{table}{Logical Instructions}
\vspace{-10pt}
\begin{longtable}[]{@{}lll@{}}
\toprule\noalign{}
Instruction & Example & Function \\
\midrule\noalign{}
\endhead
\bottomrule\noalign{}
\endlastfoot
\textbf{ANL} & ANL A,\#0FH & Bitwise AND operation \\
\textbf{ORL} & ORL A,\#F0H & Bitwise OR operation \\
\textbf{XRL} & XRL A,\#FFH & Bitwise XOR operation \\
\end{longtable}
}

\textbf{Example:}

\begin{lstlisting}
MOV A,#55H          ; A = 01010101B
ANL A,#0FH          ; A = 00000101B (upper bits mask કરો)
ORL A,#F0H          ; A = 11110101B (upper bits set કરો)
XRL A,#FFH          ; A = 00001010B (બધા bits complement કરો)
\end{lstlisting}

\begin{itemize}
\tightlist
\item
  \textbf{Bit Manipulation}: Bits setting, clearing અને testing માટે ઉપયોગ
  થાય છે
\item
  \textbf{Masking Operations}: ANL unwanted bits clear કરે છે
\item
  \textbf{Flag Effects}: Result આધારે parity flag update થાય છે
\end{itemize}

\end{solutionbox}
\begin{mnemonicbox}
``AND Masks, OR Sets, XOR Toggles''

\end{mnemonicbox}
\subsection*{પ્રશ્ન 5(બ OR) [4
ગુણ]}\label{uxaaauxab0uxab6uxaa8-5uxaac-or-4-uxa97uxaa3}

\textbf{માઇક્રોકંટ્રોલર સાથે 7 segment ને interface કરો.}

\begin{solutionbox}

\textbf{Circuit Diagram:}

\begin{lstlisting}
8051          7-Segment Display
P1.0 ----[330Ω]----a
P1.1 ----[330Ω]----b  
P1.2 ----[330Ω]----c
P1.3 ----[330Ω]----d
P1.4 ----[330Ω]----e
P1.5 ----[330Ω]----f
P1.6 ----[330Ω]----g
P1.7 ----[330Ω]----dp (decimal point)
\end{lstlisting}

\textbf{Program to Display 0-9:}

\begin{lstlisting}
ORG 0000H
    MOV DPTR,#DIGIT_TABLE   ; Lookup table તરફ point કરો
    MOV R0,#0               ; Digit 0 થી start કરો

MAIN_LOOP:
    MOV A,R0                ; Current digit get કરો
    MOVC A,@A+DPTR          ; 7-segment code get કરો
    MOV P1,A                ; 7-segment પર display કરો
    CALL DELAY              ; 1 second wait કરો
    INC R0                  ; Next digit
    CJNE R0,#10,MAIN_LOOP   ; 10 reached છે કે check કરો
    MOV R0,#0               ; 0 પર reset કરો
    SJMP MAIN_LOOP          ; Repeat કરો

DIGIT_TABLE:
    DB 3FH, 06H, 5BH, 4FH, 66H    ; 0,1,2,3,4
    DB 6DH, 7DH, 07H, 7FH, 6FH    ; 5,6,7,8,9
END
\end{lstlisting}


{\def\LTcaptype{none} % do not increment counter
\vspace{-5pt}
\captionof{table}{7-Segment Codes}
\vspace{-10pt}
\begin{longtable}[]{@{}llll@{}}
\toprule\noalign{}
Digit & Hex Code & Binary & Segments Lit \\
\midrule\noalign{}
\endhead
\bottomrule\noalign{}
\endlastfoot
\textbf{0} & 3FH & 00111111 & a,b,c,d,e,f \\
\textbf{1} & 06H & 00000110 & b,c \\
\textbf{2} & 5BH & 01011011 & a,b,g,e,d \\
\end{longtable}
}

\begin{itemize}
\tightlist
\item
  \textbf{Common Cathode}: Port pin high હોય ત્યારે segments light થાય છે
\item
  \textbf{Current Limiting}: Resistors segment damage અટકાવે છે
\item
  \textbf{Lookup Table}: Segment patterns નું efficient storage
\end{itemize}

\end{solutionbox}
\begin{mnemonicbox}
``Seven Segments Show Digits Clearly''

\end{mnemonicbox}
\subsection*{પ્રશ્ન 5(ક OR) [7
ગુણ]}\label{uxaaauxab0uxab6uxaa8-5uxa95-or-7-uxa97uxaa3}

\textbf{માઇક્રોકંટ્રોલર સાથે LM 35 ને interface કરો અને temperature controller
નો block diagram સમજાવો.}

\begin{solutionbox}

\textbf{Circuit Diagram:}

\begin{lstlisting}
LM35 Temperature Sensor Interface:

+5V ----+---- LM35 ----+---- ADC0804 ----+---- 8051
        |     (Vout)   |     (Vin)       |     (P1)
       GND             |                 |
                      GND                |
                                         |
Relay Control:                           |
8051 P3.0 ----[ULN2003]---- Relay -------+
                                         |
                                    Load (Heater/Fan)
\end{lstlisting}

\textbf{Temperature Controller Block Diagram:}

\includegraphics[width=1\linewidth,height=\textheight,keepaspectratio]{mermaid-95418e06.pdf}

\textbf{Control Program:}

\begin{lstlisting}
ORG 0000H
MAIN:
    CALL READ_TEMP      ; ADC માંથી temperature read કરો
    CALL DISPLAY_TEMP   ; Display પર temperature show કરો
    CALL TEMP_CONTROL   ; Heating/cooling control કરો
    CALL DELAY          ; Next reading પહેલાં wait કરો
    SJMP MAIN

READ_TEMP:
    CLR P2.0            ; ADC conversion start કરો
    SETB P2.0           ; Start માટે pulse
    JNB P2.1,$          ; Conversion complete થવાની wait કરો
    MOV A,P1            ; Temperature data read કરો
    RET

TEMP_CONTROL:
    CJNE A,#30,CHECK_HIGH   ; Setpoint (30^\circC) સાથે compare કરો
CHECK_HIGH:
    JC TEMP_LOW             ; A < 30 હોય તો temperature low છે
    SETB P3.0               ; Cooling (fan) ON કરો
    CLR P3.1                ; Heating OFF કરો
    RET
TEMP_LOW:
    CLR P3.0                ; Cooling OFF કરો
    SETB P3.1               ; Heating ON કરો
    RET
END
\end{lstlisting}


{\def\LTcaptype{none} % do not increment counter
\vspace{-5pt}
\captionof{table}{System Components}
\vspace{-10pt}
\begin{longtable}[]{@{}ll@{}}
\toprule\noalign{}
Component & Function \\
\midrule\noalign{}
\endhead
\bottomrule\noalign{}
\endlastfoot
\textbf{LM35} & Temperature sensor (10mV/^\circC) \\
\textbf{ADC0804} & Analog to digital converter \\
\textbf{8051} & Main controller \\
\textbf{Relay} & High power loads switch કરે છે \\
\textbf{Display} & Current temperature show કરે છે \\
\end{longtable}
}

\begin{itemize}
\tightlist
\item
  \textbf{Temperature Sensing}: LM35 દરેક degree Celsius માટે 10mV આપે છે
\item
  \textbf{ADC Conversion}: Analog voltage ને digital value માં convert કરે
  છે
\item
  \textbf{Control Logic}: Setpoint સાથે compare કરે છે અને relays control કરે
  છે
\item
  \textbf{Feedback System}: Continuous monitoring અને adjustment
\item
  \textbf{Safety Features}: Over-temperature protection શક્ય છે
\end{itemize}

\end{solutionbox}
\begin{mnemonicbox}
``Sense, Convert, Compare, Control Temperature
Automatically''

\end{mnemonicbox}

\end{document}
