\documentclass[10pt,a4paper]{article}

% content/resources/templates/preamble.tex
\usepackage[margin=0.6in]{geometry}
\author{Milav Dabgar}
\usepackage{amsmath,amssymb,amsthm}
\usepackage{booktabs}
\usepackage{multirow}
\usepackage{xcolor}
\usepackage{tcolorbox}
\tcbuselibrary{breakable,skins}
\usepackage[colorlinks=true,linkcolor=blue]{hyperref}
\usepackage{titlesec}
\usepackage{enumitem}
\usepackage{tikz}
\usepackage{pgfplots}
\usepackage{circuitikz}
\usepackage[version=4]{mhchem}
\usepackage{longtable}
\usepackage{array}
\usepackage{float}
\usepackage{caption}
\usepackage{listings}

\lstset{
  basicstyle=\small\ttfamily,
  breaklines=true,
  breakatwhitespace=false,
  postbreak=\mbox{\textcolor{red}{$\hookrightarrow$}\space},
  float=false,
  numbers=left,
  numberstyle=\tiny\color{gray},
  numbersep=10pt,
  xleftmargin=2em,
  keywordstyle=\color{blue},
  commentstyle=\color{green!60!black},
  stringstyle=\color{purple},
  backgroundcolor=\color{gray!5},
  showstringspaces=false,
  tabsize=2,
  captionpos=b,
  keepspaces=true,
  columns=flexible
}

\pgfplotsset{compat=1.18}
\usetikzlibrary{shapes,arrows,positioning,calc,patterns,decorations.pathmorphing,decorations.markings,arrows.meta}

% Color scheme
\definecolor{headcolor}{RGB}{0,102,204}
\definecolor{keycolor}{RGB}{220,20,60}
\definecolor{solutioncolor}{RGB}{34,139,34}
\definecolor{mnemoniccolor}{RGB}{148,0,211}
\definecolor{codecolor}{RGB}{0,0,100}

% Spacing
\setlength{\parskip}{3pt}
\setlist[itemize]{nosep}
\setlist[enumerate]{nosep}

% Title formatting
\titleformat{\section}{\Large\bfseries\color{headcolor}}{\thesection}{1em}{}
\titleformat{\subsection}{\large\bfseries\color{headcolor}}{\thesubsection}{1em}{}

% Pandoc tightlist compatibility
\providecommand{\tightlist}{%
  \setlength{\itemsep}{0pt}\setlength{\parskip}{0pt}}

% Pandoc longtable compatibility
\newcounter{none}
\def\thenone{}


% content/resources/templates/gujarati-boxes.tex
\usepackage{fontspec}
\usepackage{polyglossia}

% Set Gujarati as main language (document is primarily in Gujarati)
% Note: gloss-gujarati.ldf doesn't exist in polyglossia, but it will use hyphenation patterns
\setdefaultlanguage{gujarati}
\setotherlanguage{english}

% Configure Gujarati font properly
% Use Language=Default to prevent polyglossia from trying to add language-specific features
% that don't exist for Gujarati, which causes "empty feature" warnings
\newfontfamily\gujaratifont[Script=Gujarati,AutoFakeBold=2.5,AutoFakeSlant=0.3]{Noto Sans Gujarati}
\setmainfont[Script=Gujarati,AutoFakeBold=2.5,AutoFakeSlant=0.3]{Noto Sans Gujarati}
% Use Noto Sans Gujarati for monospace to support Gujarati in text
\setmonofont[Scale=0.9]{Noto Sans Gujarati}

% Configure English to use the same font
\newfontfamily\englishfont[Script=Gujarati,AutoFakeBold=2.5,AutoFakeSlant=0.3]{Noto Sans Gujarati}

% Translations for polyglossia
\gappto\captionsgujarati{
  \renewcommand{\tablename}{કોષ્ટક}
  \renewcommand{\figurename}{આકૃતિ}
}

% Helper for TikZ nodes to ensure Gujarati font
\newcommand{\gu}[1]{{\gujaratifont #1}}

% Custom environments
\newtcolorbox{solutionbox}{
    breakable,
    enhanced,
    colback=solutioncolor!5!white,
    colframe=solutioncolor!75!black,
    fonttitle=\bfseries,
    title=જવાબ
}

\newtcolorbox{solutionboxnobreak}{
 colback=solutioncolor!5!white,
 colframe=solutioncolor!75!black,
 fonttitle=\bfseries,
 title=જવાબ
}

\newtcolorbox{keyformula}{
 breakable,
 enhanced,
 colback=keycolor!5!white,
 colframe=keycolor!75!black,
 fonttitle=\bfseries,
 title=રાસાયણિક સમીકરણ/સૂત્ર
}

\newtcolorbox{mnemonicbox}{
 breakable,
 enhanced,
 colback=mnemoniccolor!5!white,
 colframe=mnemoniccolor!75!black,
 fonttitle=\bfseries,
 title=મેમરી ટ્રીક
}


\begin{document}

\begin{center}
{\Huge\bfseries\color{headcolor} Subject Name (Gujarati)}\\[5pt]
{\LARGE 1333204 -- Winter 2023}\\[3pt]
{\large Semester 1 Study Material}\\[3pt]
{\normalsize\textit{Detailed Solutions and Explanations}}
\end{center}

\vspace{10pt}

\subsection*{પ્રશ્ન 1(અ) [3
ગુણ]}\label{uxaaauxab0uxab6uxaa8-1uxa85-3-uxa97uxaa3}

\textbf{વ્યાખ્યા આપો: ફિલ્ડ, રેકોર્ડ, મેટાડેટા}

\begin{solutionbox}

{\def\LTcaptype{none} % do not increment counter
\begin{longtable}[]{@{}
  >{\raggedright\arraybackslash}p{(\linewidth - 2\tabcolsep) * \real{0.3333}}
  >{\raggedright\arraybackslash}p{(\linewidth - 2\tabcolsep) * \real{0.6667}}@{}}
\toprule\noalign{}
\begin{minipage}[b]{\linewidth}\raggedright
શબ્દ
\end{minipage} & \begin{minipage}[b]{\linewidth}\raggedright
વ્યાખ્યા
\end{minipage} \\
\midrule\noalign{}
\endhead
\bottomrule\noalign{}
\endlastfoot
\textbf{ફિલ્ડ} & ડેટાબેઝ ટેબલમાં ચોક્કસ એટ્રિબ્યુટને રજૂ કરતી ડેટાની એક એકલ એકમ
(દા.ત. નામ, ઉંમર, ID) \\
\textbf{રેકોર્ડ} & સંબંધિત ફિલ્ડ્સનો સંપૂર્ણ સેટ જે એક એન્ટિટી ઇન્સ્ટન્સને રજૂ કરે છે
(ટેબલમાં એક રો) \\
\textbf{મેટાડેટા} & ડેટા જે અન્ય ડેટાની રચના, ગુણધર્મો અને સંબંધોનું વર્ણન કરે છે (``ડેટા
વિશે ડેટા'') \\
\end{longtable}
}

\end{solutionbox}
\begin{mnemonicbox}
``FRM: ફિલ્ડ્સ રો-અપ એઝ મેટાડેટા''

\end{mnemonicbox}
\subsection*{પ્રશ્ન 1(બ) [4
ગુણ]}\label{uxaaauxab0uxab6uxaa8-1uxaac-4-uxa97uxaa3}

\textbf{વ્યાખ્યા લખો (i) E-R મોડલ (ii) એન્ટિટી (iii) એન્ટિટી સેટ અને (iv)
એટ્રીબ્યુટ્સ}

\begin{solutionbox}

{\def\LTcaptype{none} % do not increment counter
\begin{longtable}[]{@{}
  >{\raggedright\arraybackslash}p{(\linewidth - 2\tabcolsep) * \real{0.3333}}
  >{\raggedright\arraybackslash}p{(\linewidth - 2\tabcolsep) * \real{0.6667}}@{}}
\toprule\noalign{}
\begin{minipage}[b]{\linewidth}\raggedright
શબ્દ
\end{minipage} & \begin{minipage}[b]{\linewidth}\raggedright
વ્યાખ્યા
\end{minipage} \\
\midrule\noalign{}
\endhead
\bottomrule\noalign{}
\endlastfoot
\textbf{E-R મોડલ} & ડેટાબેઝ ડિઝાઇનનો ગ્રાફિકલ અભિગમ જે એન્ટિટીઝ, તેમના
એટ્રિબ્યુટ્સ અને રિલેશનશીપને મોડેલ કરે છે \\
\textbf{એન્ટિટી} & એક વાસ્તવિક-વિશ્વ વસ્તુ, વિચાર અથવા ચીજ જેનું સ્વતંત્ર અસ્તિત્વ
છે \\
\textbf{એન્ટિટી સેટ} & સમાન એન્ટિટીઓનો સંગ્રહ જે સમાન એટ્રિબ્યુટ્સ ધરાવે છે (ટેબલ
તરીકે રજૂ કરાય છે) \\
\textbf{એટ્રિબ્યુટ્સ} & ગુણધર્મો અથવા લક્ષણો જે એન્ટિટીનું વર્ણન કરે છે (ટેબલના કોલમ
તરીકે રજૂ કરાય છે) \\
\end{longtable}
}

\begin{verbatim}
erDiagram
    ENTITY \{
        string attribute1
        number attribute2
    \}
    ENTITY\_SET ||{-{-}o\{ ENTITY : contains}
\end{verbatim}

\end{solutionbox}
\begin{mnemonicbox}
``EEAA: એન્ટિટીસ એક્ઝિસ્ટ એઝ એટ્રિબ્યુટ્સ''

\end{mnemonicbox}
\subsection*{પ્રશ્ન 1(ક) [7
ગુણ]}\label{uxaaauxab0uxab6uxaa8-1uxa95-7-uxa97uxaa3}

\textbf{DBMS નાં ફાયદા અને ગેરફાયદા જણાવો.}

\begin{solutionbox}

{\def\LTcaptype{none} % do not increment counter
\begin{longtable}[]{@{}
  >{\raggedright\arraybackslash}p{(\linewidth - 2\tabcolsep) * \real{0.4444}}
  >{\raggedright\arraybackslash}p{(\linewidth - 2\tabcolsep) * \real{0.5556}}@{}}
\toprule\noalign{}
\begin{minipage}[b]{\linewidth}\raggedright
ફાયદા
\end{minipage} & \begin{minipage}[b]{\linewidth}\raggedright
ગેરફાયદા
\end{minipage} \\
\midrule\noalign{}
\endhead
\bottomrule\noalign{}
\endlastfoot
\textbf{ડેટા શેરિંગ}: ઘણા વપરાશકર્તાઓ એક સાથે એક્સેસ કરી શકે છે & \textbf{ખર્ચ}:
મોંઘા હાર્ડવેર/સોફ્ટવેર જરૂરિયાતો \\
\textbf{ડેટા ઇન્ટિગ્રિટી}: કન્સ્ટ્રેન્ટ્સ દ્વારા ચોકસાઈ જાળવે છે & \textbf{જટિલતા}:
વિશિષ્ટ તાલીમની જરૂર પડે છે \\
\textbf{ડેટા સિક્યુરિટી}: પરમિશન દ્વારા એક્સેસ નિયંત્રિત કરે છે & \textbf{પ્રદર્શન}:
મોટા ડેટાબેઝ માટે ધીમું હોઈ શકે છે \\
\textbf{ડેટા ઇન્ડિપેન્ડન્સ}: સ્ટોરેજ બદલવાથી એપ્લિકેશન પર અસર થતી નથી &
\textbf{નબળાઈ}: કેન્દ્રીય નિષ્ફળતા બિંદુ ડેટા લોસનું જોખમ છે \\
\textbf{ઘટાડેલ રિડન્ડન્સી}: ડુપ્લીકેટ ડેટા દૂર કરે છે & \textbf{કન્વર્ઝન ખર્ચ}: ફાઇલ
સિસ્ટમથી માઇગ્રેટ કરવું ખર્ચાળ છે \\
\end{longtable}
}

\end{solutionbox}
\begin{mnemonicbox}
``SIDSR vs CCPVC'' (શેરિંગ, ઇન્ટિગ્રિટી, ડેટા ઇન્ડિપેન્ડન્સ,
સિક્યુરિટી, રિડન્ડન્સી vs કોસ્ટ, કોમ્પ્લેક્સિટી, પરફોર્મન્સ, વલ્નરેબિલિટી, કન્વર્ઝન)

\end{mnemonicbox}
\subsection*{પ્રશ્ન 1(ક) OR [7
ગુણ]}\label{uxaaauxab0uxab6uxaa8-1uxa95-or-7-uxa97uxaa3}

\textbf{DBA નું પુરુનામ લખો. DBAની ભૂમિકા અને જવાબદારીઓ સમજાવો.}

\begin{solutionbox}

\textbf{DBA}: Database Administrator (ડેટાબેઝ એડમિનિસ્ટ્રેટર)

{\def\LTcaptype{none} % do not increment counter
\begin{longtable}[]{@{}l@{}}
\toprule\noalign{}
DBAની જવાબદારીઓ \\
\midrule\noalign{}
\endhead
\bottomrule\noalign{}
\endlastfoot
\textbf{ડેટાબેઝ ડિઝાઇન}: કાર્યક્ષમ ડેટાબેઝ સ્કીમા બનાવે છે \\
\textbf{સિક્યુરિટી મેનેજમેન્ટ}: યુઝર એક્સેસ કંટ્રોલ સેટ કરે છે \\
\textbf{પ્રદર્શન ટ્યુનિંગ}: ક્વેરી અને ઇન્ડેક્સને ઓપ્ટિમાઇઝ કરે છે \\
\textbf{બેકઅપ અને રિકવરી}: ડેટા સુરક્ષા યોજનાઓ લાગુ કરે છે \\
\textbf{મેઇન્ટેનન્સ}: સોફ્ટવેર અપડેટ કરે છે અને પેચ લાગુ કરે છે \\
\textbf{ટ્રબલશૂટિંગ}: ડેટાબેઝ સમસ્યાઓનો ઉકેલ કરે છે \\
\textbf{યુઝર સપોર્ટ}: ડેટાબેઝ વપરાશકર્તાઓને તાલીમ આપે છે અને સહાય કરે છે \\
\end{longtable}
}

\begin{verbatim}
flowchart TD
    A[ડેટાબેઝ એડમિનિસ્ટ્રેટર] {-{-} B[ડેટાબેઝ ડિઝાઇન]}
    A {-{-} C[સિક્યુરિટી મેનેજમેન્ટ]}
    A {-{-} D[પ્રદર્શન ટ્યુનિંગ]}
    A {-{-} E[બેકઅપ અને રિકવરી]}
    A {-{-} F[મેઇન્ટેનન્સ]}
    A {-{-} G[ટ્રબલશૂટિંગ]}
    A {-{-} H[યુઝર સપોર્ટ]}
\end{verbatim}

\end{solutionbox}
\begin{mnemonicbox}
``SPBT-MUS'' (સિક્યુરિટી, પરફોર્મન્સ, બેકઅપ, ટ્રબલશૂટિંગ,
મેઇન્ટેનન્સ, યુઝર સપોર્ટ)

\end{mnemonicbox}
\subsection*{પ્રશ્ન 2(અ) [3
ગુણ]}\label{uxaaauxab0uxab6uxaa8-2uxa85-3-uxa97uxaa3}

\textbf{યોગ્ય ઉદાહરણ સાથે સિંગલ વેલ્યુડ અને મલ્ટી વેલ્યુડ એટ્રીબ્યુટ્સ વચ્ચેનો તફાવત
સમજાવો}

\begin{solutionbox}

{\def\LTcaptype{none} % do not increment counter
\begin{longtable}[]{@{}
  >{\raggedright\arraybackslash}p{(\linewidth - 4\tabcolsep) * \real{0.4103}}
  >{\raggedright\arraybackslash}p{(\linewidth - 4\tabcolsep) * \real{0.3333}}
  >{\raggedright\arraybackslash}p{(\linewidth - 4\tabcolsep) * \real{0.2564}}@{}}
\toprule\noalign{}
\begin{minipage}[b]{\linewidth}\raggedright
એટ્રિબ્યુટ પ્રકાર
\end{minipage} & \begin{minipage}[b]{\linewidth}\raggedright
વર્ણન
\end{minipage} & \begin{minipage}[b]{\linewidth}\raggedright
ઉદાહરણો
\end{minipage} \\
\midrule\noalign{}
\endhead
\bottomrule\noalign{}
\endlastfoot
\textbf{સિંગલ-વેલ્યુડ} & દરેક એન્ટિટી ઇન્સ્ટન્સ માટે માત્ર એક મૂલ્ય ધરાવે છે & Employee
ID, જન્મતારીખ, નામ \\
\textbf{મલ્ટી-વેલ્યુડ} & એક જ એન્ટિટી માટે ઘણા મૂલ્યો ધરાવી શકે છે & ફોન નંબર,
કૌશલ્યો, ઇમેઇલ એડ્રેસ \\
\end{longtable}
}

\begin{verbatim}
erDiagram
    EMPLOYEE \{
        string emp\_id
        string name
        date birth\_date
        string phone\_numbers
        string skills
    \}
\end{verbatim}

\end{solutionbox}
\begin{mnemonicbox}
``SIM: સિંગલ ઇઝ મિનિમલ, મલ્ટી ઇઝ મેની''

\end{mnemonicbox}
\subsection*{પ્રશ્ન 2(બ) [4
ગુણ]}\label{uxaaauxab0uxab6uxaa8-2uxaac-4-uxa97uxaa3}

\textbf{E-R ડાયાગ્રામ માટે કી કન્સ્ટ્રેન્ટ્સ સમજાવો}

\begin{solutionbox}

{\def\LTcaptype{none} % do not increment counter
\begin{longtable}[]{@{}ll@{}}
\toprule\noalign{}
કી કન્સ્ટ્રેન્ટ & વર્ણન \\
\midrule\noalign{}
\endhead
\bottomrule\noalign{}
\endlastfoot
\textbf{પ્રાઇમરી કી} & એન્ટિટી સેટમાં દરેક એન્ટિટીને અનન્ય રીતે ઓળખે છે \\
\textbf{કેન્ડિડેટ કી} & કોઈપણ એટ્રિબ્યુટ જે પ્રાઇમરી કી તરીકે કામ કરી શકે \\
\textbf{ફોરેન કી} & અન્ય એન્ટિટી સેટની પ્રાઇમરી કીનો સંદર્ભ આપે છે \\
\textbf{સુપર કી} & એટ્રિબ્યુટ્સનો કોઈપણ સેટ જે અનન્ય રીતે એન્ટિટીને ઓળખે છે \\
\end{longtable}
}

\begin{verbatim}
erDiagram
    STUDENT \{
        int student\_id PK
        string name
        string email
    \}
    COURSE \{
        int course\_id PK
        string title
    \}
    ENROLLMENT \{
        int student\_id FK
        int course\_id FK
        date enroll\_date
    \}
    STUDENT ||{-{-}o\{ ENROLLMENT : has}
    COURSE ||{-{-}o\{ ENROLLMENT : includes}
\end{verbatim}

\end{solutionbox}
\begin{mnemonicbox}
``PCFS: પ્રાઇમરી કેન્ડિડેટ્સ ફાઇન્ડ સુપરકીઝ''

\end{mnemonicbox}
\subsection*{પ્રશ્ન 2(ક) [7
ગુણ]}\label{uxaaauxab0uxab6uxaa8-2uxa95-7-uxa97uxaa3}

\textbf{બેંકિંગ મેનેજમેન્ટ સિસ્ટમ માટે E-R ડાયાગ્રામ બનાવો}

\begin{solutionbox}

\begin{verbatim}
erDiagram
    CUSTOMER \{
        int customer\_id PK
        string name
        string address
        string phone
    \}
    ACCOUNT \{
        int account\_no PK
        string type
        float balance
        date open\_date
    \}
    TRANSACTION \{
        int trans\_id PK
        float amount
        string type
        date trans\_date
    \}
    BRANCH \{
        int branch\_id PK
        string name
        string location
    \}
    CUSTOMER ||{-{-}o\{ ACCOUNT : has}
    ACCOUNT ||{-{-}o\{ TRANSACTION : includes}
    BRANCH ||{-{-}o\{ ACCOUNT : manages}
    ACCOUNT {-}{-}|| CUSTOMER : belongs\_to}
\end{verbatim}

\textbf{મુખ્ય એન્ટિટીઝ અને રિલેશનશિપ્સ}:

\begin{itemize}
\tightlist
\item
  \textbf{ગ્રાહક}: ગ્રાહક માહિતી સંગ્રહિત કરે છે
\item
  \textbf{એકાઉન્ટ}: વિવિધ એકાઉન્ટ પ્રકારો (સેવિંગ્સ, ચેકિંગ)
\item
  \textbf{ટ્રાન્ઝેક્શન}: ડિપોઝિટ, વિડ્રોઅલ રેકોર્ડ કરે છે
\item
  \textbf{બ્રાન્ચ}: વિવિધ બેંક સ્થાનો
\item
  \textbf{રિલેશનશિપ્સ}: ગ્રાહકો પાસે એકાઉન્ટ છે, એકાઉન્ટમાં ટ્રાન્ઝેક્શન છે, બ્રાન્ચ
  એકાઉન્ટ મેનેજ કરે છે
\end{itemize}

\end{solutionbox}
\begin{mnemonicbox}
``CATB: કસ્ટમર્સ એક્સેસ ટ્રાન્ઝેક્શન્સ એટ બ્રાન્ચીસ''

\end{mnemonicbox}
\subsection*{પ્રશ્ન 2(અ) OR [3
ગુણ]}\label{uxaaauxab0uxab6uxaa8-2uxa85-or-3-uxa97uxaa3}

\textbf{યોગ્ય ઉદાહરણ સાથે સ્પેશિયલાઈઝેશન અને જનરલાઈઝેશન વચ્ચેનો તફાવત સમજાવો}

\begin{solutionbox}

{\def\LTcaptype{none} % do not increment counter
\begin{longtable}[]{@{}
  >{\raggedright\arraybackslash}p{(\linewidth - 6\tabcolsep) * \real{0.2143}}
  >{\raggedright\arraybackslash}p{(\linewidth - 6\tabcolsep) * \real{0.2619}}
  >{\raggedright\arraybackslash}p{(\linewidth - 6\tabcolsep) * \real{0.3095}}
  >{\raggedright\arraybackslash}p{(\linewidth - 6\tabcolsep) * \real{0.2143}}@{}}
\toprule\noalign{}
\begin{minipage}[b]{\linewidth}\raggedright
વિચાર
\end{minipage} & \begin{minipage}[b]{\linewidth}\raggedright
દિશા
\end{minipage} & \begin{minipage}[b]{\linewidth}\raggedright
વર્ણન
\end{minipage} & \begin{minipage}[b]{\linewidth}\raggedright
ઉદાહરણ
\end{minipage} \\
\midrule\noalign{}
\endhead
\bottomrule\noalign{}
\endlastfoot
\textbf{સ્પેશિયલાઈઝેશન} & ટોપ-ડાઉન & સામાન્ય એન્ટિટીને વધુ ચોક્કસ સબ-એન્ટિટીઓમાં
વિભાજિત કરવું & વ્યક્તિ \rightarrow વિદ્યાર્થી, કર્મચારી \\
\textbf{જનરલાઈઝેશન} & બોટમ-અપ & સમાન એન્ટિટીઓને ઉચ્ચ-સ્તરીય એન્ટિટીમાં જોડવું &
કાર, ટ્રક \rightarrow વાહન \\
\end{longtable}
}

\begin{verbatim}
erDiagram
    PERSON \{
        int person\_id
        string name
        string address
    \}
    STUDENT \{
        string major
        float gpa
    \}
    EMPLOYEE \{
        string department
        float salary
    \}
    PERSON ||{-{-}|| STUDENT : specializes}
    PERSON ||{-{-}|| EMPLOYEE : specializes}
\end{verbatim}

\end{solutionbox}
\begin{mnemonicbox}
``SG-TD-BU: સ્પેશિયલાઈઝેશન ગોઝ ટોપ-ડાઉન, જનરલાઈઝેશન
બિલ્ડ્સ અપ''

\end{mnemonicbox}
\subsection*{પ્રશ્ન 2(બ) OR [4
ગુણ]}\label{uxaaauxab0uxab6uxaa8-2uxaac-or-4-uxa97uxaa3}

\textbf{ચાસ્પ ટ્રેપની વ્યાખ્યા લખો. તે ક્યારે ઉદ્ભવે છે તે સમજાવો. ચાસ્પ ટ્રેપ માટેનો
ઉપાય સમજાવો}

\begin{solutionbox}

\textbf{ચાસ્પ ટ્રેપ}: ER ડાયાગ્રામમાં ઉદ્ભવતી સમસ્યા જ્યારે એન્ટિટીઓ વચ્ચે મલ્ટિપલ
પાથ હોય છે, જેથી રિલેશનશિપના અર્થઘટનમાં અસ્પષ્ટતા આવે છે.

{\def\LTcaptype{none} % do not increment counter
\begin{longtable}[]{@{}
  >{\raggedright\arraybackslash}p{(\linewidth - 2\tabcolsep) * \real{0.3810}}
  >{\raggedright\arraybackslash}p{(\linewidth - 2\tabcolsep) * \real{0.6190}}@{}}
\toprule\noalign{}
\begin{minipage}[b]{\linewidth}\raggedright
પાસું
\end{minipage} & \begin{minipage}[b]{\linewidth}\raggedright
વર્ણન
\end{minipage} \\
\midrule\noalign{}
\endhead
\bottomrule\noalign{}
\endlastfoot
\textbf{ઉદ્ભવ} & જ્યારે એન્ટિટી પ્રકારો વચ્ચે બે અથવા વધુ અલગ પાથ હોય જે ચક્ર બનાવે
છે \\
\textbf{સમસ્યા} & અયોગ્ય અથવા અસ્પષ્ટ ક્વેરી પરિણામો તરફ દોરી જાય છે \\
\textbf{ઉકેલ} & એક રિલેશનશિપને તોડવું અથવા ઇચ્છિત પાથને સ્પષ્ટ કરવા માટે કન્સ્ટ્રેન્ટ્સ
ઉમેરવા \\
\end{longtable}
}

\begin{verbatim}
erDiagram
    STUDENT \|{-}{-}|| SECTION : enrolled\_in}
    SECTION \|{-}{-}|| COURSE : part\_of}
    STUDENT \|{-}{-}o\{ COURSE : studies}

    \%\% Solution:
    \%\% Remove direct STUDENT to COURSE relationship
    \%\% Or add clear constraints
\end{verbatim}

\end{solutionbox}
\begin{mnemonicbox}
``COP: સાયકલ્સ ઓફ પાથસ નીડ બ્રેકિંગ''

\end{mnemonicbox}
\subsection*{પ્રશ્ન 2(ક) OR [7
ગુણ]}\label{uxaaauxab0uxab6uxaa8-2uxa95-or-7-uxa97uxaa3}

\textbf{કોલેજ મેનેજમેન્ટ સિસ્ટમ માટે E-R ડાયાગ્રામ બનાવો}

\begin{solutionbox}

\begin{verbatim}
erDiagram
    STUDENT \{
        int student\_id PK
        string name
        string address
        date dob
        string phone
    \}
    DEPARTMENT \{
        int dept\_id PK
        string name
        string location
        string hod
    \}
    FACULTY \{
        int faculty\_id PK
        string name
        string qualification
        date join\_date
    \}
    COURSE \{
        int course\_id PK
        string title
        int credits
        string description
    \}
    EXAM \{
        int exam\_id PK
        date date
        string type
    \}
    STUDENT \|{-}{-}|| DEPARTMENT : belongs\_to}
    FACULTY \|{-}{-}|| DEPARTMENT : works\_in}
    DEPARTMENT ||{-{-}o\{ COURSE : offers}
    FACULTY ||{-{-}o\{ COURSE : teaches}
    STUDENT {-}{-}o\{ COURSE : enrolls}
    STUDENT {-}{-}o\{ EXAM : takes}
    COURSE ||{-{-}o\{ EXAM : has}
\end{verbatim}

\textbf{મુખ્ય એન્ટિટીઝ અને રિલેશનશિપ્સ}:

\begin{itemize}
\tightlist
\item
  \textbf{વિદ્યાર્થી}: વિદ્યાર્થી વિગતો સંગ્રહિત કરે છે
\item
  \textbf{વિભાગ}: શૈક્ષણિક વિભાગો
\item
  \textbf{ફેકલ્ટી}: શિક્ષકો અને પ્રોફેસરો
\item
  \textbf{કોર્સ}: ભણાવવામાં આવતા વિષયો
\item
  \textbf{પરીક્ષા}: મૂલ્યાંકન કાર્યક્રમો
\item
  \textbf{રિલેશનશિપ્સ}: વિદ્યાર્થીઓ કોર્સમાં એનરોલ થાય છે, ફેકલ્ટી કોર્સ શીખવે છે,
  વિભાગો કોર્સ ઓફર કરે છે
\end{itemize}

\end{solutionbox}
\begin{mnemonicbox}
``SDFCE: સ્ટુડન્ટ્સ ડિલાઇટ ફેકલ્ટી બાય કમ્પ્લીટિંગ એક્ઝામ્સ''

\end{mnemonicbox}
\subsection*{પ્રશ્ન 3(અ) [3
ગુણ]}\label{uxaaauxab0uxab6uxaa8-3uxa85-3-uxa97uxaa3}

\textbf{GROUP BY ક્લોઝ ઉદાહરણ સાથે સમજાવો.}

\begin{solutionbox}

\textbf{GROUP BY} ક્લોઝ સમાન મૂલ્યો ધરાવતી રો સારાંશ રોમાં જૂથ કરે છે.

{\def\LTcaptype{none} % do not increment counter
\begin{longtable}[]{@{}ll@{}}
\toprule\noalign{}
ફીચર & વર્ણન \\
\midrule\noalign{}
\endhead
\bottomrule\noalign{}
\endlastfoot
\textbf{હેતુ} & એકસરખા ડેટાને એગ્રીગેટ ફંક્શન માટે જૂથોમાં ગોઠવે છે \\
\textbf{ઉપયોગ} & એગ્રીગેટ ફંક્શન (COUNT, SUM, AVG, MAX, MIN) સાથે વપરાય છે \\
\textbf{સિન્ટેક્સ} & SELECT column1, COUNT(*) FROM table GROUP BY
column1; \\
\end{longtable}
}

\begin{verbatim}
SELECT department, AVG(salary) 
FROM employees
GROUP BY department;
\end{verbatim}

\end{solutionbox}
\begin{mnemonicbox}
``GAS: ગ્રુપ એન્ડ સમરાઈઝ''

\end{mnemonicbox}
\subsection*{પ્રશ્ન 3(બ) [4
ગુણ]}\label{uxaaauxab0uxab6uxaa8-3uxaac-4-uxa97uxaa3}

\textbf{Data Definition Language (DDL) કમાન્ડની યાદી બનાવો. કોઈ પણ ૨ DDL
કમાન્ડ ઉદાહરણ સાથે સમજાવો.}

\begin{solutionbox}

\textbf{DDL કમાન્ડ્સ}: CREATE, ALTER, DROP, TRUNCATE, RENAME

{\def\LTcaptype{none} % do not increment counter
\begin{longtable}[]{@{}
  >{\raggedright\arraybackslash}p{(\linewidth - 4\tabcolsep) * \real{0.2903}}
  >{\raggedright\arraybackslash}p{(\linewidth - 4\tabcolsep) * \real{0.4194}}
  >{\raggedright\arraybackslash}p{(\linewidth - 4\tabcolsep) * \real{0.2903}}@{}}
\toprule\noalign{}
\begin{minipage}[b]{\linewidth}\raggedright
કમાન્ડ
\end{minipage} & \begin{minipage}[b]{\linewidth}\raggedright
વર્ણન
\end{minipage} & \begin{minipage}[b]{\linewidth}\raggedright
ઉદાહરણ
\end{minipage} \\
\midrule\noalign{}
\endhead
\bottomrule\noalign{}
\endlastfoot
\textbf{CREATE} & ડેટાબેઝ ઓબ્જેક્ટ્સ જેમ કે ટેબલ, વ્યૂ, ઇન્ડેક્સ બનાવે છે &
\texttt{CREATE\ TABLE\ students\ (id\ INT\ PRIMARY\ KEY,\ name\ VARCHAR(50));} \\
\textbf{ALTER} & મૌજૂદા ડેટાબેઝ ઓબ્જેક્ટ્સ સુધારે છે &
\texttt{ALTER\ TABLE\ students\ ADD\ COLUMN\ email\ VARCHAR(100);} \\
\textbf{DROP} & ડેટાબેઝ ઓબ્જેક્ટ્સ દૂર કરે છે &
\texttt{DROP\ TABLE\ students;} \\
\textbf{TRUNCATE} & ટેબલમાંથી બધા રેકોર્ડ્સ દૂર કરે છે &
\texttt{TRUNCATE\ TABLE\ students;} \\
\end{longtable}
}

\end{solutionbox}
\begin{mnemonicbox}
``CADTR: ક્રિએટ, ઓલ્ટર, ડ્રોપ, ટ્રન્કેટ, રીનેમ''

\end{mnemonicbox}
\subsection*{પ્રશ્ન 3(ક) [7
ગુણ]}\label{uxaaauxab0uxab6uxaa8-3uxa95-7-uxa97uxaa3}

\textbf{enr\_no, name, percent, branch ફિલ્ડ ધરાવતા Students ટેબલ પર નીચેની
Query perform કરો.}

\begin{solutionbox}

\begin{verbatim}
{-{-} ૧. Students ટેબલના તમામ રેકોર્ડ ડિસ્પ્લે કરો.}
SELECT * FROM Students;

{-{-} ૨. ડુપ્લીકેટ વેલ્યુ સિવાય માત્ર branch ડિસ્પ્લે કરો.}
SELECT DISTINCT branch FROM Students;

{-{-} ૩. name નાં ઉતરતા ક્રમમાં તમામ રેકોર્ડ ડિસ્પ્લે કરો.}
SELECT * FROM Students ORDER BY name DESC;

{-{-} ૪. સરનામું સ્ટોર કરવા માટે "address" નામથી નવી કોલમ ઉમેરો.}
ALTER TABLE Students ADD address VARCHAR(100);

{-{-} ૫. "ICT" બ્રાંચ ધરાવતા બધા વિદ્યાર્થીને ડિસ્પ્લે કરો.}
SELECT * FROM Students WHERE branch = {ICT};

{-{-} ૬. ૬૦ કરતા ઓછા percent ધરાવતા વિદ્યાર્થીઓને ડીલીટ કરો.}
DELETE FROM Students WHERE percent {} 60;

{-{-} ૭. "S" થી શરૂ થતા તમામ વિદ્યાર્થીઓના નામ ડિસ્પ્લે કરો.}
SELECT * FROM Students WHERE name LIKE {S\%};
\end{verbatim}

{\def\LTcaptype{none} % do not increment counter
\begin{longtable}[]{@{}ll@{}}
\toprule\noalign{}
ક્વેરી & હેતુ \\
\midrule\noalign{}
\endhead
\bottomrule\noalign{}
\endlastfoot
\textbf{SELECT} & ટેબલમાંથી ડેટા મેળવે છે \\
\textbf{DISTINCT} & ડુપ્લિકેટ મૂલ્યો દૂર કરે છે \\
\textbf{ORDER BY} & પરિણામોને ચોક્કસ ક્રમમાં ગોઠવે છે \\
\textbf{ALTER TABLE} & ટેબલ સ્ટ્રક્ચર સુધારે છે \\
\textbf{WHERE} & શરતો પર આધારિત રેકોર્ડ્સ ફિલ્ટર કરે છે \\
\textbf{DELETE} & શરતો મેળવતા રેકોર્ડ્સ દૂર કરે છે \\
\textbf{LIKE} & સ્ટ્રિંગ તુલનામાં પેટર્ન મેચિંગ \\
\end{longtable}
}

\end{solutionbox}
\begin{mnemonicbox}
``SDOAWDL: સિલેક્ટ ડિસ્ટિંક્ટ ઓર્ડર ઓલ્ટર વ્હેર ડિલીટ લાઇક''

\end{mnemonicbox}
\subsection*{પ્રશ્ન 3(અ) OR [3
ગુણ]}\label{uxaaauxab0uxab6uxaa8-3uxa85-or-3-uxa97uxaa3}

\textbf{સિન્ટેક્સ અને ઉદાહરણ સાથે GRANT કમાન્ડ સમજાવો.}

\begin{solutionbox}

\textbf{GRANT} કમાન્ડ વપરાશકર્તાઓને ડેટાબેઝ ઓબ્જેક્ટ્સ પર ચોક્કસ અધિકારો આપે છે.

{\def\LTcaptype{none} % do not increment counter
\begin{longtable}[]{@{}
  >{\raggedright\arraybackslash}p{(\linewidth - 2\tabcolsep) * \real{0.4583}}
  >{\raggedright\arraybackslash}p{(\linewidth - 2\tabcolsep) * \real{0.5417}}@{}}
\toprule\noalign{}
\begin{minipage}[b]{\linewidth}\raggedright
ઘટક
\end{minipage} & \begin{minipage}[b]{\linewidth}\raggedright
વર્ણન
\end{minipage} \\
\midrule\noalign{}
\endhead
\bottomrule\noalign{}
\endlastfoot
\textbf{સિન્ટેક્સ} &
\texttt{GRANT\ privilege(s)\ ON\ object\ TO\ user\ [WITH\ GRANT\ OPTION];} \\
\textbf{પ્રિવિલેજીસ} & SELECT, INSERT, UPDATE, DELETE, ALL PRIVILEGES \\
\textbf{ઓબ્જેક્ટ્સ} & ટેબલ્સ, વ્યૂ, સિક્વેન્સિસ, વગેરે \\
\end{longtable}
}

\begin{verbatim}
GRANT SELECT, UPDATE ON employees TO user1;
GRANT ALL PRIVILEGES ON database\_name.* TO user2 WITH GRANT OPTION;
\end{verbatim}

\end{solutionbox}
\begin{mnemonicbox}
``GPO: ગ્રાન્ટ પ્રિવિલેજીસ ટુ અધર્સ''

\end{mnemonicbox}
\subsection*{પ્રશ્ન 3(બ) OR [4
ગુણ]}\label{uxaaauxab0uxab6uxaa8-3uxaac-or-4-uxa97uxaa3}

\textbf{Truncate અને Drop કમાન્ડનો તફાવત લખો.}

\begin{solutionbox}

{\def\LTcaptype{none} % do not increment counter
\begin{longtable}[]{@{}
  >{\raggedright\arraybackslash}p{(\linewidth - 4\tabcolsep) * \real{0.3600}}
  >{\raggedright\arraybackslash}p{(\linewidth - 4\tabcolsep) * \real{0.4000}}
  >{\raggedright\arraybackslash}p{(\linewidth - 4\tabcolsep) * \real{0.2400}}@{}}
\toprule\noalign{}
\begin{minipage}[b]{\linewidth}\raggedright
ફીચર
\end{minipage} & \begin{minipage}[b]{\linewidth}\raggedright
TRUNCATE
\end{minipage} & \begin{minipage}[b]{\linewidth}\raggedright
DROP
\end{minipage} \\
\midrule\noalign{}
\endhead
\bottomrule\noalign{}
\endlastfoot
\textbf{હેતુ} & ટેબલથી બધી પંક્તિઓ દૂર કરે છે & સંપૂર્ણ ટેબલ સ્ટ્રક્ચર દૂર કરે છે \\
\textbf{સ્ટ્રક્ચર} & ટેબલ સ્ટ્રક્ચર જાળવી રાખે છે & ટેબલની વ્યાખ્યા સંપૂર્ણપણે દૂર કરે
છે \\
\textbf{રિકવરી} & સરળતાથી રોલબેક નથી કરી શકાતું & કમિટ થાય ત્યાં સુધી પુનઃપ્રાપ્ત
કરી શકાય છે \\
\textbf{સ્પીડ} & DELETE કરતાં ઝડપી & ઝડપી ઓપરેશન \\
\textbf{ટ્રિગર્સ} & ટ્રિગર્સ સક્રિય કરતું નથી & ટ્રિગર્સ સક્રિય કરતું નથી \\
\end{longtable}
}

\begin{verbatim}
{-{-} Truncate ઉદાહરણ}
TRUNCATE TABLE students;

{-{-} Drop ઉદાહરણ}
DROP TABLE students;
\end{verbatim}

\end{solutionbox}
\begin{mnemonicbox}
``TRC-DST: ટ્રન્કેટ રિમૂવ્સ કન્ટેન્ટ્સ, ડ્રોપ ડિસ્ટ્રોય્સ સ્ટ્રક્ચર
ટોટલી''

\end{mnemonicbox}
\subsection*{પ્રશ્ન 3(ક) OR [7
ગુણ]}\label{uxaaauxab0uxab6uxaa8-3uxa95-or-7-uxa97uxaa3}

\textbf{નીચેની Query ના આઉટપુટ લખો.}

\begin{solutionbox}

{\def\LTcaptype{none} % do not increment counter
\begin{longtable}[]{@{}
  >{\raggedright\arraybackslash}p{(\linewidth - 4\tabcolsep) * \real{0.2500}}
  >{\raggedright\arraybackslash}p{(\linewidth - 4\tabcolsep) * \real{0.2857}}
  >{\raggedright\arraybackslash}p{(\linewidth - 4\tabcolsep) * \real{0.4643}}@{}}
\toprule\noalign{}
\begin{minipage}[b]{\linewidth}\raggedright
ક્વેરી
\end{minipage} & \begin{minipage}[b]{\linewidth}\raggedright
આઉટપુટ
\end{minipage} & \begin{minipage}[b]{\linewidth}\raggedright
સમજૂતી
\end{minipage} \\
\midrule\noalign{}
\endhead
\bottomrule\noalign{}
\endlastfoot
\textbf{ABS(-23), ABS(49)} & 23, 49 & નિરપેક્ષ મૂલ્ય પાછું આપે છે \\
\textbf{SQRT(25), SQRT(81)} & 5, 9 & વર્ગમૂળ પાછું આપે છે \\
\textbf{POWER(3,2), POWER(-2,3)} & 9, -8 & x\^{}y (પ્રથમ મૂલ્યને બીજા મૂલ્યની
પાવર સુધી ઉંચકે છે) \\
\textbf{MOD(15,4), MOD(21,3)} & 3, 0 & વિભાજન પછી શેષ પાછો આપે છે \\
\textbf{ROUND(123.446,1), ROUND(123.456,2)} & 123.4, 123.46 & ચોક્કસ દશાંશ
જગ્યાઓ પર રાઉન્ડ કરે છે \\
\textbf{CEIL(234.45), CEIL(-234.45)} & 235, -234 & નજીકના પૂર્ણાંક સુધી ઉપર
રાઉન્ડ કરે છે \\
\textbf{FLOOR(-12.7), FLOOR(12.7)} & -13, 12 & નજીકના પૂર્ણાંક સુધી નીચે
રાઉન્ડ કરે છે \\
\end{longtable}
}

\begin{verbatim}
SELECT ABS({-}23), ABS(49);          {-{-} 23, 49}
SELECT SQRT(25), SQRT(81);         {-{-} 5, 9}
SELECT POWER(3,2), POWER({-}2,3);    {-{-} 9, {-}8}
SELECT MOD(15,4), MOD(21,3);       {-{-} 3, 0}
SELECT ROUND(123.446,1), ROUND(123.456,2); {-{-} 123.4, 123.46}
SELECT CEIL(234.45), CEIL({-}234.45);  {-{-} 235, {-}234}
SELECT FLOOR({-}12.7), FLOOR(12.7);    {-{-} {-}13, 12}
\end{verbatim}

\end{solutionbox}
\begin{mnemonicbox}
``ASPMRCF: એબ્સોલ્યુટ સ્ક્વેર પાવર મોડ્યુલો રાઉન્ડ સીલિંગ
ફ્લોર''

\end{mnemonicbox}
\subsection*{પ્રશ્ન 4(અ) [3
ગુણ]}\label{uxaaauxab0uxab6uxaa8-4uxa85-3-uxa97uxaa3}

\textbf{SQLમાં ડેટા ટાઈપની યાદી બનાવો. કોઈ પણ ૨ ડેટા ટાઈપ ઉદાહરણ સાથે
સમજાવો.}

\begin{solutionbox}

\textbf{SQL ડેટા ટાઈપ}: INTEGER, FLOAT, VARCHAR, CHAR, DATE, DATETIME,
BOOLEAN, BLOB

{\def\LTcaptype{none} % do not increment counter
\begin{longtable}[]{@{}
  >{\raggedright\arraybackslash}p{(\linewidth - 4\tabcolsep) * \real{0.3333}}
  >{\raggedright\arraybackslash}p{(\linewidth - 4\tabcolsep) * \real{0.3939}}
  >{\raggedright\arraybackslash}p{(\linewidth - 4\tabcolsep) * \real{0.2727}}@{}}
\toprule\noalign{}
\begin{minipage}[b]{\linewidth}\raggedright
ડેટા ટાઈપ
\end{minipage} & \begin{minipage}[b]{\linewidth}\raggedright
વર્ણન
\end{minipage} & \begin{minipage}[b]{\linewidth}\raggedright
ઉદાહરણ
\end{minipage} \\
\midrule\noalign{}
\endhead
\bottomrule\noalign{}
\endlastfoot
\textbf{INTEGER} & દશાંશ પોઇન્ટ વિના પૂર્ણ સંખ્યાઓ &
\texttt{id\ INTEGER\ =\ 101} \\
\textbf{VARCHAR} & પરિવર્તનશીલ-લંબાઈ સ્ટ્રિંગ &
\texttt{name\ VARCHAR(50)\ =\ \textquotesingle{}John\textquotesingle{}} \\
\textbf{DATE} & તારીખ મૂલ્યો સંગ્રહિત કરે છે (YYYY-MM-DD) &
\texttt{birth\_date\ DATE\ =\ \textquotesingle{}2000-05-15\textquotesingle{}} \\
\textbf{FLOAT} & ફ્લોટિંગ પોઇન્ટ સાથે દશાંશ સંખ્યાઓ &
\texttt{salary\ FLOAT\ =\ 45000.50} \\
\end{longtable}
}

\begin{verbatim}
CREATE TABLE employees (
    id INTEGER,
    name VARCHAR(50),
    salary FLOAT
);
\end{verbatim}

\end{solutionbox}
\begin{mnemonicbox}
``IVDB: ઈન્ટિજર અને વારચાર આર ડેટાબેઝ બેસિક્સ''

\end{mnemonicbox}
\subsection*{પ્રશ્ન 4(બ) [4
ગુણ]}\label{uxaaauxab0uxab6uxaa8-4uxaac-4-uxa97uxaa3}

\textbf{Full function dependency ઉદાહરણ સાથે સમજાવો.}

\begin{solutionbox}

\textbf{Full Function Dependency}: જ્યારે Y, X પર ફંક્શનલી ડિપેન્ડન્ટ હોય, પરંતુ
X ના કોઈ સબસેટ પર નહીં.

{\def\LTcaptype{none} % do not increment counter
\begin{longtable}[]{@{}
  >{\raggedright\arraybackslash}p{(\linewidth - 4\tabcolsep) * \real{0.2903}}
  >{\raggedright\arraybackslash}p{(\linewidth - 4\tabcolsep) * \real{0.4194}}
  >{\raggedright\arraybackslash}p{(\linewidth - 4\tabcolsep) * \real{0.2903}}@{}}
\toprule\noalign{}
\begin{minipage}[b]{\linewidth}\raggedright
વિચાર
\end{minipage} & \begin{minipage}[b]{\linewidth}\raggedright
વર્ણન
\end{minipage} & \begin{minipage}[b]{\linewidth}\raggedright
ઉદાહરણ
\end{minipage} \\
\midrule\noalign{}
\endhead
\bottomrule\noalign{}
\endlastfoot
\textbf{વ્યાખ્યા} & એટ્રિબ્યુટ B, A પર પૂર્ણપણે ફંક્શનલી ડિપેન્ડન્ટ છે જો B સંપૂર્ણ A પર
આધાર રાખે છે & Student\_ID \rightarrow Name (પૂર્ણ ડિપેન્ડન્સી) \\
\textbf{નોન-ઉદાહરણ} & જ્યારે એટ્રિબ્યુટ કોમ્પોઝિટ કીના માત્ર ભાગ પર આધાર રાખે છે
& \{Student\_ID, Course\_ID\} \rightarrow Student\_Name (આંશિક) \\
\end{longtable}
}

\begin{verbatim}
flowchart TD
    A[Student\_ID] {-{-} B[Student\_Name]}
    subgraph Full Function Dependency
    C[Course\_ID] {-{-} D[Course\_Name]}
    end
    subgraph Partial Function Dependency
    E[Student\_ID, Course\_ID] {-{-} F[Student\_Name]}
    end
\end{verbatim}

\end{solutionbox}
\begin{mnemonicbox}
``FFD: ફુલ, નોટ ફ્રેક્શન ઓફ ડિપેન્ડન્સી''

\end{mnemonicbox}
\subsection*{પ્રશ્ન 4(ક) [7
ગુણ]}\label{uxaaauxab0uxab6uxaa8-4uxa95-7-uxa97uxaa3}

\textbf{નોર્મલાઇઝેશનની વ્યાખ્યા આપો. 2NF (સેકન્ડ નોર્મલ ફોર્મ) ઉદાહરણ અને ઉકેલ સાથે
સમજાવો.}

\begin{solutionbox}

\textbf{નોર્મલાઈઝેશન}: ડેટાબેઝની રચના કરવાની પ્રક્રિયા જેથી મોટા ટેબલને નાના
ટેબલોમાં વિભાજિત કરીને અને તેમની વચ્ચે સંબંધો વ્યાખ્યાયિત કરીને, રિડન્ડન્સી અને
ડિપેન્ડન્સી ઘટાડવામાં આવે.

\textbf{2NF (સેકન્ડ નોર્મલ ફોર્મ)}:

\begin{itemize}
\tightlist
\item
  ટેબલ 2NF માં છે જો તે 1NF માં હોય અને કોઈ નોન-પ્રાઇમ એટ્રિબ્યુટ કેન્ડિડેટ કીના કોઈ
  પણ યોગ્ય સબસેટ પર આધાર રાખતું ન હોય.
\end{itemize}

{\def\LTcaptype{none} % do not increment counter
\begin{longtable}[]{@{}
  >{\raggedright\arraybackslash}p{(\linewidth - 2\tabcolsep) * \real{0.5714}}
  >{\raggedright\arraybackslash}p{(\linewidth - 2\tabcolsep) * \real{0.4286}}@{}}
\toprule\noalign{}
\begin{minipage}[b]{\linewidth}\raggedright
2NF પહેલાં
\end{minipage} & \begin{minipage}[b]{\linewidth}\raggedright
સમસ્યા
\end{minipage} \\
\midrule\noalign{}
\endhead
\bottomrule\noalign{}
\endlastfoot
\textbf{Order(Order\_ID, Product\_ID, Product\_Name, Quantity, Price)} &
Product\_Name માત્ર Product\_ID પર આધાર રાખે છે, સંપૂર્ણ કી પર નહીં \\
\end{longtable}
}

{\def\LTcaptype{none} % do not increment counter
\begin{longtable}[]{@{}
  >{\raggedright\arraybackslash}p{(\linewidth - 2\tabcolsep) * \real{0.5238}}
  >{\raggedright\arraybackslash}p{(\linewidth - 2\tabcolsep) * \real{0.4762}}@{}}
\toprule\noalign{}
\begin{minipage}[b]{\linewidth}\raggedright
2NF પછી
\end{minipage} & \begin{minipage}[b]{\linewidth}\raggedright
ઉકેલ
\end{minipage} \\
\midrule\noalign{}
\endhead
\bottomrule\noalign{}
\endlastfoot
\textbf{Order(Order\_ID, Product\_ID, Quantity)} & માત્ર પૂર્ણ કી
ડિપેન્ડન્સી \\
\textbf{Product(Product\_ID, Product\_Name, Price)} & પ્રોડક્ટ વિગતો માત્ર
Product\_ID પર આધાર રાખે છે \\
\end{longtable}
}

\begin{verbatim}
erDiagram
    ORDER \{
        int order\_id
        int product\_id
        int quantity
    \}
    PRODUCT \{
        int product\_id
        string product\_name
        float price
    \}
    ORDER {-}{-}|| PRODUCT : contains}
\end{verbatim}

\end{solutionbox}
\begin{mnemonicbox}
``2NF-PPD: પાર્શિયલ ડિપેન્ડન્સી પ્રોબ્લેમ્સ ડિવાઇડેડ''

\end{mnemonicbox}
\subsection*{પ્રશ્ન 4(અ) OR [3
ગુણ]}\label{uxaaauxab0uxab6uxaa8-4uxa85-or-3-uxa97uxaa3}

\textbf{કમાન્ડ સમજાવવો. ૧) To\_Number (), ૨) To\_Char()}

\begin{solutionbox}

{\def\LTcaptype{none} % do not increment counter
\begin{longtable}[]{@{}
  >{\raggedright\arraybackslash}p{(\linewidth - 6\tabcolsep) * \real{0.2778}}
  >{\raggedright\arraybackslash}p{(\linewidth - 6\tabcolsep) * \real{0.2500}}
  >{\raggedright\arraybackslash}p{(\linewidth - 6\tabcolsep) * \real{0.2222}}
  >{\raggedright\arraybackslash}p{(\linewidth - 6\tabcolsep) * \real{0.2500}}@{}}
\toprule\noalign{}
\begin{minipage}[b]{\linewidth}\raggedright
ફંક્શન
\end{minipage} & \begin{minipage}[b]{\linewidth}\raggedright
હેતુ
\end{minipage} & \begin{minipage}[b]{\linewidth}\raggedright
સિન્ટેક્સ
\end{minipage} & \begin{minipage}[b]{\linewidth}\raggedright
ઉદાહરણ
\end{minipage} \\
\midrule\noalign{}
\endhead
\bottomrule\noalign{}
\endlastfoot
\textbf{TO\_NUMBER()} & સ્ટ્રિંગને નંબરમાં રૂપાંતરિત કરે છે &
\texttt{TO\_NUMBER(string,\ [format])} &
\texttt{TO\_NUMBER(\textquotesingle{}123.45\textquotesingle{})\ =\ 123.45} \\
\textbf{TO\_CHAR()} & નંબર/તારીખને સ્ટ્રિંગમાં રૂપાંતરિત કરે છે &
\texttt{TO\_CHAR(value,\ [format])} &
\texttt{TO\_CHAR(1234,\ \textquotesingle{}9999\textquotesingle{})\ =\ \textquotesingle{}1234\textquotesingle{}} \\
\end{longtable}
}

\begin{verbatim}
{-{-} સ્ટ્રિંગને નંબરમાં રૂપાંતરિત કરે છે}
SELECT TO\_NUMBER({123.45}) FROM dual;  {-{-} 123.45}

{-{-} તારીખને ફોર્મેટેડ સ્ટ્રિંગમાં રૂપાંતરિત કરે છે}
SELECT TO\_CHAR(SYSDATE, {DD{-}MON{-}YYYY}) FROM dual;  {-{-} 20{-}JAN{-}2024}

{-{-} નંબરને ફોર્મેટેડ સ્ટ્રિંગમાં રૂપાંતરિત કરે છે}
SELECT TO\_CHAR(1234.56, {$9,999.99}) FROM dual;  {-{-} $1,234.56}
\end{verbatim}

\end{solutionbox}
\begin{mnemonicbox}
``NC: નંબર્સ એન્ડ કેરેક્ટર્સ કન્વર્ઝન''

\end{mnemonicbox}
\subsection*{પ્રશ્ન 4(બ) OR [4
ગુણ]}\label{uxaaauxab0uxab6uxaa8-4uxaac-or-4-uxa97uxaa3}

\textbf{1NF (ફર્સ્ટ નોર્મલ ફોર્મ) ઉદાહરણ અને ઉકેલ સાથે સમજાવો.}

\begin{solutionbox}

\textbf{1NF (ફર્સ્ટ નોર્મલ ફોર્મ)}: એક રિલેશન 1NF માં છે જો તેમાં કોઈ રિપીટિંગ
ગ્રુપ્સ અથવા એરે ન હોય.

{\def\LTcaptype{none} % do not increment counter
\begin{longtable}[]{@{}ll@{}}
\toprule\noalign{}
1NF પહેલાં & સમસ્યા \\
\midrule\noalign{}
\endhead
\bottomrule\noalign{}
\endlastfoot
\textbf{Student(ID, Name, Courses)} & Courses કોલમમાં મલ્ટિપલ વેલ્યુ છે \\
\textbf{ઉદાહરણ}: (101, John, ``Math,Science,History'') & મલ્ટી-વેલ્યુડ
એટ્રિબ્યુટ \\
\end{longtable}
}

{\def\LTcaptype{none} % do not increment counter
\begin{longtable}[]{@{}
  >{\raggedright\arraybackslash}p{(\linewidth - 2\tabcolsep) * \real{0.5238}}
  >{\raggedright\arraybackslash}p{(\linewidth - 2\tabcolsep) * \real{0.4762}}@{}}
\toprule\noalign{}
\begin{minipage}[b]{\linewidth}\raggedright
1NF પછી
\end{minipage} & \begin{minipage}[b]{\linewidth}\raggedright
ઉકેલ
\end{minipage} \\
\midrule\noalign{}
\endhead
\bottomrule\noalign{}
\endlastfoot
\textbf{Student(ID, Name, Course)} & દરેક રોમાં એક કોર્સ \\
\textbf{ઉદાહરણો}: (101, John, Math), (101, John, Science), (101, John,
History) & એટોમિક વેલ્યુઝ \\
\end{longtable}
}

\begin{verbatim}
erDiagram
    STUDENT\_BEFORE \{
        int id
        string name
        string courses
    \}
    STUDENT\_AFTER \{
        int id
        string name
        string course
    \}
\end{verbatim}

\end{solutionbox}
\begin{mnemonicbox}
``1NF-ARM: એટોમિક વેલ્યુઝ રિમૂવ મલ્ટિવેલ્યુઝ''

\end{mnemonicbox}
\subsection*{પ્રશ્ન 4(ક) OR [7
ગુણ]}\label{uxaaauxab0uxab6uxaa8-4uxa95-or-7-uxa97uxaa3}

\textbf{SQL માં Function dependency સમજાવો. Partial function dependency
ઉદાહરણ સાથે સમજાવો.}

\begin{solutionbox}

\textbf{ફંક્શનલ ડિપેન્ડન્સી}: એક સંબંધ જ્યાં એક એટ્રિબ્યુટ બીજા એટ્રિબ્યુટનું મૂલ્ય નક્કી કરે
છે.

\textbf{નોટેશન}: X \rightarrow Y (X, Y ને નક્કી કરે છે)

\textbf{પાર્શિયલ ફંક્શનલ ડિપેન્ડન્સી}: જ્યારે એક એટ્રિબ્યુટ કમ્પોઝિટ પ્રાઇમરી કીના
માત્ર એક ભાગ પર આધાર રાખે છે.

{\def\LTcaptype{none} % do not increment counter
\begin{longtable}[]{@{}
  >{\raggedright\arraybackslash}p{(\linewidth - 4\tabcolsep) * \real{0.2903}}
  >{\raggedright\arraybackslash}p{(\linewidth - 4\tabcolsep) * \real{0.2903}}
  >{\raggedright\arraybackslash}p{(\linewidth - 4\tabcolsep) * \real{0.4194}}@{}}
\toprule\noalign{}
\begin{minipage}[b]{\linewidth}\raggedright
વિચાર
\end{minipage} & \begin{minipage}[b]{\linewidth}\raggedright
ઉદાહરણ
\end{minipage} & \begin{minipage}[b]{\linewidth}\raggedright
સમજૂતી
\end{minipage} \\
\midrule\noalign{}
\endhead
\bottomrule\noalign{}
\endlastfoot
\textbf{કમ્પોઝિટ કી} & \{Student\_ID, Course\_ID\} & સાથે મળીને પ્રાઇમરી કી
બનાવે છે \\
\textbf{પાર્શિયલ ડિપેન્ડન્સી} & \{Student\_ID, Course\_ID\} \rightarrow Student\_Name
& Student\_Name માત્ર Student\_ID પર આધાર રાખે છે \\
\textbf{સમસ્યા} & અપડેટ એનોમલીઝ, ડેટા રિડન્ડન્સી & એક જ વિદ્યાર્થીનું નામ ઘણા બધા
કોર્સ માટે પુનરાવર્તિત થાય છે \\
\end{longtable}
}

\begin{verbatim}
flowchart TD
    A[Student\_ID] {-{-} B[Student\_Name]}
    C[Course\_ID] {-{-} D[Course\_Name]}
    E["Student\_ID, Course\_ID"] {-{-} F[Grade]}
    subgraph "Partial Dependency"
    A {-{-} B}
    end
    subgraph "Full Dependency"
    E {-{-} F}
    end
\end{verbatim}

\textbf{ઉકેલ}: અલગ ટેબલોમાં વિભાજિત કરો જ્યાં દરેક નોન-કી એટ્રિબ્યુટ કી પર સંપૂર્ણપણે
આધારિત હોય.

\end{solutionbox}
\begin{mnemonicbox}
``PD-CPK: પાર્શિયલ ડિપેન્ડન્સી - કમ્પોનન્ટ ઓફ પ્રાઇમરી કી''

\end{mnemonicbox}
\subsection*{પ્રશ્ન 5(અ) [3
ગુણ]}\label{uxaaauxab0uxab6uxaa8-5uxa85-3-uxa97uxaa3}

\textbf{Transaction નાં ગુણધર્મો ઉદાહરણ સાથે સમજાવો.}

\begin{solutionbox}

\textbf{Transaction ગુણધર્મો} (ACID):

{\def\LTcaptype{none} % do not increment counter
\begin{longtable}[]{@{}
  >{\raggedright\arraybackslash}p{(\linewidth - 4\tabcolsep) * \real{0.3125}}
  >{\raggedright\arraybackslash}p{(\linewidth - 4\tabcolsep) * \real{0.4062}}
  >{\raggedright\arraybackslash}p{(\linewidth - 4\tabcolsep) * \real{0.2812}}@{}}
\toprule\noalign{}
\begin{minipage}[b]{\linewidth}\raggedright
ગુણધર્મ
\end{minipage} & \begin{minipage}[b]{\linewidth}\raggedright
વર્ણન
\end{minipage} & \begin{minipage}[b]{\linewidth}\raggedright
ઉદાહરણ
\end{minipage} \\
\midrule\noalign{}
\endhead
\bottomrule\noalign{}
\endlastfoot
\textbf{એટોમિસિટી} & બધા ઓપરેશનો સફળતાપૂર્વક પૂર્ણ થાય છે અથવા એક પણ થતું નથી &
બેંક ટ્રાન્સફર: ડેબિટ અને ક્રેડિટ બંને થાય અથવા બંને ન થાય \\
\textbf{કન્સિસ્ટન્સી} & ડેટાબેઝ પહેલા અને પછી માન્ય સ્થિતિમાં રહે છે & એકાઉન્ટ બેલેન્સ
કન્સ્ટ્રેન્ટ્સ માન્ય રહે છે \\
\textbf{આઇસોલેશન} & ટ્રાન્ઝેક્શન એવી રીતે એક્ઝિક્યુટ થાય છે જાણે તે એકમાત્ર હોય & બે
યુઝર એક જ રેકોર્ડ અપડેટ કરી રહ્યા હોય ત્યારે દખલ કરતા નથી \\
\textbf{ડ્યુરેબિલિટી} & કમિટ કરેલા ફેરફારો સિસ્ટમ નિષ્ફળતા પછી પણ ટકી રહે છે &
એકવાર પુષ્ટિ થઈ જાય, પછી વીજળી જતી રહે તો પણ ડિપોઝિટ યથાવત રહે છે \\
\end{longtable}
}

\begin{verbatim}
flowchart LR
    A[START TRANSACTION] {-{-} B[Debit Account A]}
    B {-{-} C[Credit Account B]}
    C {-{-} D\{Successful?\}}
    D {-{-}|Yes| E[COMMIT]}
    D {-{-}|No| F[ROLLBACK]}
\end{verbatim}

\end{solutionbox}
\begin{mnemonicbox}
``ACID: એટોમિસિટી, કન્સિસ્ટન્સી, આઇસોલેશન, ડ્યુરેબિલિટી''

\end{mnemonicbox}
\subsection*{પ્રશ્ન 5(બ) [4
ગુણ]}\label{uxaaauxab0uxab6uxaa8-5uxaac-4-uxa97uxaa3}

\textbf{ઉપર Q.5 (b) માં આપેલ ``Students'' અને ``CR'' ટેબલનો ઉપયોગ કરીને સેટ
ઓપરેટર દ્વારા નીચેની Query લખો.}

\begin{solutionbox}

\begin{verbatim}
{-{-} ૧. Students અથવા CR હોય તેવા વ્યક્તિઓની યાદી બનાવો.}
SELECT Stnd\_Name FROM Student
UNION
SELECT CR\_Name FROM CR;

{-{-} ૨. Students અને CR હોય તેવા વ્યક્તિઓની યાદી બનાવો.}
SELECT Stnd\_Name FROM Student
INTERSECT
SELECT CR\_Name FROM CR;

{-{-} ૩. Students હોય અને CR ન હોય માત્ર તેવા વ્યક્તિઓની યાદી બનાવો.}
SELECT Stnd\_Name FROM Student
MINUS
SELECT CR\_Name FROM CR;

{-{-} ૪. CR હોય અને Student ન હોય માત્ર તેવા વ્યક્તિઓની યાદી બનાવો.}
SELECT CR\_Name FROM CR
MINUS
SELECT Stnd\_Name FROM Student;
\end{verbatim}

{\def\LTcaptype{none} % do not increment counter
\begin{longtable}[]{@{}
  >{\raggedright\arraybackslash}p{(\linewidth - 4\tabcolsep) * \real{0.3333}}
  >{\raggedright\arraybackslash}p{(\linewidth - 4\tabcolsep) * \real{0.2143}}
  >{\raggedright\arraybackslash}p{(\linewidth - 4\tabcolsep) * \real{0.4524}}@{}}
\toprule\noalign{}
\begin{minipage}[b]{\linewidth}\raggedright
સેટ ઓપરેટર
\end{minipage} & \begin{minipage}[b]{\linewidth}\raggedright
હેતુ
\end{minipage} & \begin{minipage}[b]{\linewidth}\raggedright
ઉદાહરણ માટે પરિણામ
\end{minipage} \\
\midrule\noalign{}
\endhead
\bottomrule\noalign{}
\endlastfoot
\textbf{UNION} & બધી અલગ રો જોડે છે & Manoj, Rahil, Jiya, Rina, Jitesh,
Priya \\
\textbf{INTERSECT} & માત્ર સામાન્ય રો પરત કરે છે & Manoj, Rina \\
\textbf{MINUS} & પ્રથમ સેટમાં હોય પણ બીજા સેટમાં ન હોય તે રો & Rahil, Jiya \\
\textbf{MINUS (ઊલટું)} & બીજા સેટમાં હોય પણ પહેલા સેટમાં ન હોય તે રો & Jitesh,
Priya \\
\end{longtable}
}

\end{solutionbox}
\begin{mnemonicbox}
``UIMD: યુનિયન ઇન્ક્લૂડ્સ, માઈનસ ડિવાઈડ્સ''

\end{mnemonicbox}
\subsection*{પ્રશ્ન 5(ક) [7
ગુણ]}\label{uxaaauxab0uxab6uxaa8-5uxa95-7-uxa97uxaa3}

\textbf{Conflict serializability વિસ્તારથી સમજાવો.}

\begin{solutionbox}

\textbf{Conflict Serializability}: એક શેડ્યૂલ કન્ફ્લિક્ટ સીરિયલાઇઝેબલ છે જો તેને
નોન-કન્ફ્લિક્ટિંગ ઓપરેશન્સને સ્વેપ કરીને સીરિયલ શેડ્યૂલમાં રૂપાંતરિત કરી શકાય.

{\def\LTcaptype{none} % do not increment counter
\begin{longtable}[]{@{}
  >{\raggedright\arraybackslash}p{(\linewidth - 2\tabcolsep) * \real{0.5185}}
  >{\raggedright\arraybackslash}p{(\linewidth - 2\tabcolsep) * \real{0.4815}}@{}}
\toprule\noalign{}
\begin{minipage}[b]{\linewidth}\raggedright
મુખ્ય વિચારો
\end{minipage} & \begin{minipage}[b]{\linewidth}\raggedright
વર્ણન
\end{minipage} \\
\midrule\noalign{}
\endhead
\bottomrule\noalign{}
\endlastfoot
\textbf{કન્ફ્લિક્ટ ઓપરેશન્સ} & બે ઓપરેશન કન્ફ્લિક્ટ કરે છે જો તેઓ એક જ ડેટા આઇટમને એક્સેસ
કરે છે અને ઓછામાં ઓછું એક રાઇટ હોય \\
\textbf{પ્રીસિડન્સ ગ્રાફ} & ટ્રાન્ઝેક્શન વચ્ચેના કન્ફ્લિક્ટને દર્શાવતો ડાયરેક્ટેડ ગ્રાફ \\
\textbf{સીરિયલાઇઝેબલ} & જો પ્રીસિડન્સ ગ્રાફમાં કોઈ સાયકલ ન હોય, તો શેડ્યૂલ
કન્ફ્લિક્ટ સીરિયલાઇઝેબલ છે \\
\end{longtable}
}

\begin{verbatim}
flowchart LR
    subgraph "Transaction T1"
    A[Read X] {-{-} B[Write X]}
    end
    subgraph "Transaction T2"
    C[Read X] {-{-} D[Write X]}
    end
    subgraph "Conflicts"
    B {-{-} C}
    end
\end{verbatim}

\textbf{ઉદાહરણ}:

\begin{itemize}
\tightlist
\item
  T1: R(X), W(X)
\item
  T2: R(X), W(X)
\end{itemize}

\textbf{સીરિયલાઇઝેબલ શેડ્યૂલ્સ}:

\begin{itemize}
\tightlist
\item
  T1 બાદ T2: R1(X), W1(X), R2(X), W2(X)
\item
  T2 બાદ T1: R2(X), W2(X), R1(X), W1(X)
\end{itemize}

\textbf{નોન-સીરિયલાઇઝેબલ}: R1(X), R2(X), W1(X), W2(X) - પ્રીસિડન્સ ગ્રાફમાં
સાયકલ બનાવે છે

\end{solutionbox}
\begin{mnemonicbox}
``COPS: કન્ફ્લિક્ટ ઓપરેશન્સ પ્રોડ્યુસ સીરિયલાઇઝેબિલિટી''

\end{mnemonicbox}
\subsection*{પ્રશ્ન 5(અ) OR [3
ગુણ]}\label{uxaaauxab0uxab6uxaa8-5uxa85-or-3-uxa97uxaa3}

\textbf{Transaction નો concept ઉદાહરણ સાથે સમજાવો.}

\begin{solutionbox}

\textbf{ટ્રાન્ઝેક્શન}: કામની એક તાર્કિક એકમ જે સંપૂર્ણપણે કરવું અથવા સંપૂર્ણપણે અનડૂ કરવું
આવશ્યક છે.

{\def\LTcaptype{none} % do not increment counter
\begin{longtable}[]{@{}
  >{\raggedright\arraybackslash}p{(\linewidth - 4\tabcolsep) * \real{0.4634}}
  >{\raggedright\arraybackslash}p{(\linewidth - 4\tabcolsep) * \real{0.3171}}
  >{\raggedright\arraybackslash}p{(\linewidth - 4\tabcolsep) * \real{0.2195}}@{}}
\toprule\noalign{}
\begin{minipage}[b]{\linewidth}\raggedright
ટ્રાન્ઝેક્શન તબક્કાઓ
\end{minipage} & \begin{minipage}[b]{\linewidth}\raggedright
વર્ણન
\end{minipage} & \begin{minipage}[b]{\linewidth}\raggedright
ઉદાહરણ
\end{minipage} \\
\midrule\noalign{}
\endhead
\bottomrule\noalign{}
\endlastfoot
\textbf{BEGIN} & ટ્રાન્ઝેક્શનની શરૂઆત ચિહ્નિત કરે છે & START TRANSACTION \\
\textbf{ઓપરેશન્સ એક્ઝિક્યુટ} & ડેટાબેઝ ઓપરેશન્સ (રીડ/રાઇટ) & UPDATE account SET
balance = balance - 1000 WHERE id = 123 \\
\textbf{COMMIT/ROLLBACK} & સફળતા/નિષ્ફળતા સાથે ટ્રાન્ઝેક્શન સમાપ્ત કરે છે &
COMMIT અથવા ROLLBACK \\
\end{longtable}
}

\begin{verbatim}
flowchart LR
    A[BEGIN TRANSACTION] {-{-} B[Read account balance]}
    B {-{-} C[Check if sufficient funds]}
    C {-{-}|Yes| D[Update account balance]}
    D {-{-} E[Create transaction record]}
    E {-{-} F[COMMIT]}
    C {-{-}|No| G[ROLLBACK]}
\end{verbatim}

\textbf{ઉદાહરણ}:

\begin{verbatim}
BEGIN TRANSACTION;
UPDATE accounts SET balance = balance {-} 1000 WHERE acc\_no = 123;
UPDATE accounts SET balance = balance + 1000 WHERE acc\_no = 456;
COMMIT;
\end{verbatim}

\end{solutionbox}
\begin{mnemonicbox}
``BEC: બિગિન, એક્ઝિક્યુટ, કમિટ''

\end{mnemonicbox}
\subsection*{પ્રશ્ન 5(બ) OR [4
ગુણ]}\label{uxaaauxab0uxab6uxaa8-5uxaac-or-4-uxa97uxaa3}

\textbf{Equi-join સિન્ટેક્સ અને ઉદાહરણ સાથે સમજાવો.}

\begin{solutionbox}

\textbf{Equi-join}: એક જોઈન ઓપરેશન જે સમાનતા કમ્પેરિઝન ઓપરેટરનો ઉપયોગ કરે છે.

{\def\LTcaptype{none} % do not increment counter
\begin{longtable}[]{@{}
  >{\raggedright\arraybackslash}p{(\linewidth - 2\tabcolsep) * \real{0.4091}}
  >{\raggedright\arraybackslash}p{(\linewidth - 2\tabcolsep) * \real{0.5909}}@{}}
\toprule\noalign{}
\begin{minipage}[b]{\linewidth}\raggedright
ફીચર
\end{minipage} & \begin{minipage}[b]{\linewidth}\raggedright
વર્ણન
\end{minipage} \\
\midrule\noalign{}
\endhead
\bottomrule\noalign{}
\endlastfoot
\textbf{સિન્ટેક્સ} &
\texttt{SELECT\ columns\ FROM\ table1,\ table2\ WHERE\ table1.column\ =\ table2.column;} \\
\textbf{હેતુ} & મેચિંગ કોલમ વેલ્યુના આધારે બે ટેબલમાંથી રો જોડે છે \\
\textbf{વૈકલ્પિક} &
\texttt{SELECT\ columns\ FROM\ table1\ INNER\ JOIN\ table2\ ON\ table1.column\ =\ table2.column;} \\
\end{longtable}
}

\begin{verbatim}
{-{-} પરંપરાગત સિન્ટેક્સ}
SELECT s.name, d.dept\_name 
FROM students s, departments d 
WHERE s.dept\_id = d.dept\_id;

{-{-} INNER JOIN સિન્ટેક્સ}
SELECT s.name, d.dept\_name 
FROM students s INNER JOIN departments d 
ON s.dept\_id = d.dept\_id;
\end{verbatim}

\end{solutionbox}
\begin{mnemonicbox}
``EQ-ME: ઇક્વાલિટી મેચિસ એન્ટ્રીસ''

\end{mnemonicbox}
\subsection*{પ્રશ્ન 5(ક) OR [7
ગુણ]}\label{uxaaauxab0uxab6uxaa8-5uxa95-or-7-uxa97uxaa3}

\textbf{View serializability વિસ્તારથી સમજાવો.}

\begin{solutionbox}

\textbf{View Serializability}: એક શેડ્યૂલ વ્યૂ સીરિયલાઇઝેબલ છે જો તે કોઈ સીરિયલ
શેડ્યૂલ સાથે વ્યૂ ઇક્વિવેલન્ટ હોય.

{\def\LTcaptype{none} % do not increment counter
\begin{longtable}[]{@{}
  >{\raggedright\arraybackslash}p{(\linewidth - 2\tabcolsep) * \real{0.4583}}
  >{\raggedright\arraybackslash}p{(\linewidth - 2\tabcolsep) * \real{0.5417}}@{}}
\toprule\noalign{}
\begin{minipage}[b]{\linewidth}\raggedright
શરત
\end{minipage} & \begin{minipage}[b]{\linewidth}\raggedright
વર્ણન
\end{minipage} \\
\midrule\noalign{}
\endhead
\bottomrule\noalign{}
\endlastfoot
\textbf{ઇનિશિયલ રીડ} & જો T1 શેડ્યૂલ S માં ડેટા આઇટમ X ની પ્રારંભિક વેલ્યુ વાંચે છે,
તો તેણે S' શેડ્યૂલમાં પણ પ્રારંભિક વેલ્યુ વાંચવી જોઈએ \\
\textbf{ફાઇનલ રાઇટ} & જો T1, S માં ડેટા આઇટમ X નું અંતિમ લખાણ કરે છે, તો તેણે S'
માં પણ અંતિમ લખાણ કરવું જોઈએ \\
\textbf{ડિપેન્ડન્સી પ્રિઝર્વેશન} & જો T1, S માં T2 દ્વારા લખાયેલ X ની વેલ્યુ વાંચે છે,
તો તેણે S' માં પણ T2 પાસેથી વાંચવું જોઈએ \\
\end{longtable}
}

\begin{verbatim}
flowchart LR
    A[Schedule S] {-{-} B\{View Equivalent?\}}
    B {-{-}|Yes| C[View Serializable]}
    B {-{-}|No| D[Not View Serializable]}

    subgraph "Read{-Write Analysis"}
    E[Initial Read Check]
    F[Final Write Check]
    G[Read{-from{-}Write Check]}
    end
\end{verbatim}

\textbf{તુલના}:

\begin{itemize}
\tightlist
\item
  \textbf{કન્ફ્લિક્ટ સીરિયલાઇઝેબિલિટી}: વધુ પ્રતિબંધિત, પરીક્ષણ કરવું સરળ
  (પ્રીસિડન્સ ગ્રાફ)
\item
  \textbf{વ્યૂ સીરિયલાઇઝેબિલિટી}: વધુ સામાન્ય, પરીક્ષણ કરવું વધુ મુશ્કેલ (NP-કમ્પ્લીટ)
\end{itemize}

\textbf{વ્યૂ સીરિયલાઇઝેબલ પરંતુ કન્ફ્લિક્ટ સીરિયલાઇઝેબલ નહીં તેનું ઉદાહરણ}:

\begin{itemize}
\tightlist
\item
  T1: W(X)
\item
  T2: W(X)
\item
  T3: R(X)
\item
  શેડ્યૂલ: W1(X), W2(X), R3(X) - સીરિયલ શેડ્યૂલ T2,T1,T3 સાથે વ્યૂ ઇક્વિવેલન્ટ
\end{itemize}

\end{solutionbox}
\begin{mnemonicbox}
``VIR-FF: વ્યૂ પ્રિઝર્વ્સ ઇનિશિયલ રીડ્સ એન્ડ ફાઇનલ રાઇટ્સ''

\end{mnemonicbox}

\end{document}
