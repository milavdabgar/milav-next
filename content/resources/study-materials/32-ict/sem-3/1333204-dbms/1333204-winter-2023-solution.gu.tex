\documentclass{article}

% content/resources/templates/preamble.tex
\usepackage[margin=0.6in]{geometry}
\author{Milav Dabgar}
\usepackage{amsmath,amssymb,amsthm}
\usepackage{booktabs}
\usepackage{multirow}
\usepackage{xcolor}
\usepackage{tcolorbox}
\tcbuselibrary{breakable,skins}
\usepackage[colorlinks=true,linkcolor=blue]{hyperref}
\usepackage{titlesec}
\usepackage{enumitem}
\usepackage{tikz}
\usepackage{pgfplots}
\usepackage{circuitikz}
\usepackage[version=4]{mhchem}
\usepackage{longtable}
\usepackage{array}
\usepackage{float}
\usepackage{caption}
\usepackage{listings}

\lstset{
  basicstyle=\small\ttfamily,
  breaklines=true,
  breakatwhitespace=false,
  postbreak=\mbox{\textcolor{red}{$\hookrightarrow$}\space},
  float=false,
  numbers=left,
  numberstyle=\tiny\color{gray},
  numbersep=10pt,
  xleftmargin=2em,
  keywordstyle=\color{blue},
  commentstyle=\color{green!60!black},
  stringstyle=\color{purple},
  backgroundcolor=\color{gray!5},
  showstringspaces=false,
  tabsize=2,
  captionpos=b,
  keepspaces=true,
  columns=flexible
}

\pgfplotsset{compat=1.18}
\usetikzlibrary{shapes,arrows,positioning,calc,patterns,decorations.pathmorphing,decorations.markings,arrows.meta}

% Color scheme
\definecolor{headcolor}{RGB}{0,102,204}
\definecolor{keycolor}{RGB}{220,20,60}
\definecolor{solutioncolor}{RGB}{34,139,34}
\definecolor{mnemoniccolor}{RGB}{148,0,211}
\definecolor{codecolor}{RGB}{0,0,100}

% Spacing
\setlength{\parskip}{3pt}
\setlist[itemize]{nosep}
\setlist[enumerate]{nosep}

% Title formatting
\titleformat{\section}{\Large\bfseries\color{headcolor}}{\thesection}{1em}{}
\titleformat{\subsection}{\large\bfseries\color{headcolor}}{\thesubsection}{1em}{}

% Pandoc tightlist compatibility
\providecommand{\tightlist}{%
  \setlength{\itemsep}{0pt}\setlength{\parskip}{0pt}}

% Pandoc longtable compatibility
\newcounter{none}
\def\thenone{}


% content/resources/templates/gujarati-boxes.tex
\usepackage{fontspec}
\usepackage{polyglossia}

% Set Gujarati as main language (document is primarily in Gujarati)
% Note: gloss-gujarati.ldf doesn't exist in polyglossia, but it will use hyphenation patterns
\setdefaultlanguage{gujarati}
\setotherlanguage{english}

% Configure Gujarati font properly
% Use Language=Default to prevent polyglossia from trying to add language-specific features
% that don't exist for Gujarati, which causes "empty feature" warnings
\newfontfamily\gujaratifont[Script=Gujarati,AutoFakeBold=2.5,AutoFakeSlant=0.3]{Noto Sans Gujarati}
\setmainfont[Script=Gujarati,AutoFakeBold=2.5,AutoFakeSlant=0.3]{Noto Sans Gujarati}
% Use Noto Sans Gujarati for monospace to support Gujarati in text
\setmonofont[Scale=0.9]{Noto Sans Gujarati}

% Configure English to use the same font
\newfontfamily\englishfont[Script=Gujarati,AutoFakeBold=2.5,AutoFakeSlant=0.3]{Noto Sans Gujarati}

% Translations for polyglossia
\gappto\captionsgujarati{
  \renewcommand{\tablename}{કોષ્ટક}
  \renewcommand{\figurename}{આકૃતિ}
}

% Helper for TikZ nodes to ensure Gujarati font
\newcommand{\gu}[1]{{\gujaratifont #1}}

% Custom environments
\newtcolorbox{solutionbox}{
    breakable,
    enhanced,
    colback=solutioncolor!5!white,
    colframe=solutioncolor!75!black,
    fonttitle=\bfseries,
    title=જવાબ
}

\newtcolorbox{solutionboxnobreak}{
 colback=solutioncolor!5!white,
 colframe=solutioncolor!75!black,
 fonttitle=\bfseries,
 title=જવાબ
}

\newtcolorbox{keyformula}{
 breakable,
 enhanced,
 colback=keycolor!5!white,
 colframe=keycolor!75!black,
 fonttitle=\bfseries,
 title=રાસાયણિક સમીકરણ/સૂત્ર
}

\newtcolorbox{mnemonicbox}{
 breakable,
 enhanced,
 colback=mnemoniccolor!5!white,
 colframe=mnemoniccolor!75!black,
 fonttitle=\bfseries,
 title=મેમરી ટ્રીક
}


% Custom commands for GTU solutions
% This file defines semantic commands for consistent formatting

% Question command with automatic formatting
\newcommand{\question}[2]{%
  \section*{Question #1}%
  \textbf{#2}%
}

% OR question variant
\newcommand{\questionor}[2]{%
  \section*{Question #1 OR}%
  \textbf{#2}%
}

% Proper table environment with caption
\newenvironment{answertable}[1]{%
  \begin{table}[htbp]
  \centering
  \caption{#1}
}{%
  \end{table}
}

% Proper figure environment for diagrams
\newenvironment{answerdiagram}[1]{%
  \begin{figure}[htbp]
  \centering
  \caption{#1}
}{%
  \end{figure}
}

% Semantic markup for key terms
\newcommand{\keyword}[1]{\textbf{#1}}
\newcommand{\code}[1]{\texttt{#1}}
\newcommand{\classname}[1]{\texttt{#1}}
\newcommand{\methodname}[1]{\texttt{#1}}

% Proper quotation marks
\newcommand{\mnemonic}[1]{``#1''}


\usetikzlibrary{calc,positioning,shapes,arrows,automata,fit,shapes.multipart}
\tikzset{
    entity/.style={rectangle, draw, fill=white, align=center, minimum height=2em, font=\small, thick},
    relationship/.style={diamond, draw, fill=white, align=center, aspect=2, font=\small, thick},
    attribute/.style={ellipse, draw, fill=white, align=center, font=\small},
    multi attribute/.style={ellipse, draw, double, fill=white, align=center, font=\small},
    gtu line/.style={draw, thick},
    gtu arrow/.style={draw, -latex, thick}
}

\title{Database Management System (1333204) - Winter 2023 Solution}
\date{January 20, 2024}

\begin{document}
\maketitle

\questionmarks{1(અ)}{3}{વ્યાખ્યા આપો: ફિલ્ડ, રેકોર્ડ, મેટાડેટા}
\begin{solutionbox}
\begin{table}[H]
    \centering
    \caption{મૂળભૂત ડેટાબેઝ શરતો}
    \begin{tabulary}{\linewidth}{LCL}
        \toprule
        \textbf{શબ્દ} & \textbf{વ્યાખ્યા} \\
        \midrule
        \textbf{ફિલ્ડ} & ડેટાબેઝ ટેબલમાં ચોક્કસ એટ્રિબ્યુટને રજૂ કરતી ડેટાની એક એકલ એકમ (દા.ત. નામ, ઉંમર, ID) \\
        \textbf{રેકોર્ડ} & સંબંધિત ફિલ્ડ્સનો સંપૂર્ણ સેટ જે એક એન્ટિટી ઇન્સ્ટન્સને રજૂ કરે છે (ટેબલમાં એક રો) \\
        \textbf{મેટાડેટા} & ડેટા જે અન્ય ડેટાની રચના, ગુણધર્મો અને સંબંધોનું વર્ણન કરે છે ("ડેટા વિશે ડેટા") \\
        \bottomrule
    \end{tabulary}
\end{table}

\begin{mnemonicbox}
    \textbf{મેમરી ટ્રીક:} "FRM: ફિલ્ડ્સ રો-અપ એઝ મેટાડેટા"
\end{mnemonicbox}
\end{solutionbox}

\questionmarks{1(બ)}{4}{વ્યાખ્યા લખો (i) E-R મોડલ (ii) એન્ટિટી (iii) એન્ટિટી સેટ અને (iv) એટ્રીબ્યુટ્સ}
\begin{solutionbox}
\begin{table}[H]
    \centering
    \caption{E-R મોડલ પરિભાષા}
    \begin{tabulary}{\linewidth}{LCL}
        \toprule
        \textbf{શબ્દ} & \textbf{વ્યાખ્યા} \\
        \midrule
        \textbf{E-R મોડલ} & ડેટાબેઝ ડિઝાઇનનો ગ્રાફિકલ અભિગમ જે એન્ટિટીઝ, તેમના એટ્રિબ્યુટ્સ અને રિલેશનશીપને મોડેલ કરે છે \\
        \textbf{એન્ટિટી} & એક વાસ્તવિક-વિશ્વ વસ્તુ, વિચાર અથવા ચીજ જેનું સ્વતંત્ર અસ્તિત્વ છે \\
        \textbf{એન્ટિટી સેટ} & સમાન એન્ટિટીઓનો સંગ્રહ જે સમાન એટ્રિબ્યુટ્સ ધરાવે છે (ટેબલ તરીકે રજૂ કરાય છે) \\
        \textbf{એટ્રિબ્યુટ્સ} & ગુણધર્મો અથવા લક્ષણો જે એન્ટિટીનું વર્ણન કરે છે (ટેબલના કોલમ તરીકે રજૂ કરાય છે) \\
        \bottomrule
    \end{tabulary}
\end{table}

\begin{center}
\begin{tikzpicture}[gtu block, node distance=2.5cm]
    \node [gtu block, fill=blue!10] (es) {ENTITY SET};
    \node [gtu block, right of=es, xshift=2cm] (e) {ENTITY};
    \node [attribute, below of=e, xshift=-1cm] (a1) {attribute1};
    \node [attribute, below of=e, xshift=1cm] (a2) {attribute2};

    \draw [gtu line] (es) -- node[above] {contains} (e);
    \draw [gtu line] (e) -- (a1);
    \draw [gtu line] (e) -- (a2);
\end{tikzpicture}
\captionof{figure}{એન્ટિટી અને એન્ટિટી સેટ સંબંધ}
\end{center}

\begin{mnemonicbox}
    \textbf{મેમરી ટ્રીક:} "EEAA: એન્ટિટીસ એક્ઝિસ્ટ એઝ એટ્રિબ્યુટ્સ"
\end{mnemonicbox}
\end{solutionbox}

\questionmarks{1(ક)}{7}{DBMS નાં ફાયદા અને ગેરફાયદા જણાવો.}
\begin{solutionbox}
\begin{table}[H]
    \centering
    \caption{DBMS નાં ફાયદા અને ગેરફાયદા}
    \begin{tabulary}{\linewidth}{LCL}
        \toprule
        \textbf{ફાયદા} & \textbf{ગેરફાયદા} \\
        \midrule
        \textbf{ડેટા શેરિંગ}: ઘણા વપરાશકર્તાઓ એક સાથે એક્સેસ કરી શકે છે & \textbf{ખર્ચ}: મોંઘા હાર્ડવેર/સોફ્ટવેર જરૂરિયાતો \\
        \textbf{ડેટા ઇન્ટિગ્રિટી}: કન્સ્ટ્રેન્ટ્સ દ્વારા ચોકસાઈ જાળવે છે & \textbf{જટિલતા}: વિશિષ્ટ તાલીમની જરૂર પડે છે \\
        \textbf{ડેટા સિક્યુરિટી}: પરમિશન દ્વારા એક્સેસ નિયંત્રિત કરે છે & \textbf{પ્રદર્શન}: મોટા ડેટાબેઝ માટે ધીમું હોઈ શકે છે \\
        \textbf{ડેટા ઇન્ડિપેન્ડન્સ}: સ્ટોરેજ બદલવાથી એપ્લિકેશન પર અસ & \textbf{નબળાઈ}: કેન્દ્રીય નિષ્ફળતા બિંદુ ડેટા લોસનું જોખમ છે \\
        \textbf{ઘટાડેલ રિડન્ડન્સી}: ડુપ્લીકેટ ડેટા દૂર કરે છે & \textbf{કન્વર્ઝન ખર્ચ}: ફાઇલ સિસ્ટમથી માઇગ્રેટ કરવું ખર્ચાળ છે \\
        \bottomrule
    \end{tabulary}
\end{table}

\begin{mnemonicbox}
    \textbf{મેમરી ટ્રીક:} "SIDSR vs CCPVC" (શેરિંગ, ઇન્ટિગ્રિટી, ડેટા ઇન્ડિપેન્ડન્સ, સિક્યુરિટી, રિડન્ડન્સી vs કોસ્ટ, કોમ્પ્લેક્સિટી, પરફોર્મન્સ, વલ્નરેબિલિટી, કન્વર્ઝન)
\end{mnemonicbox}
\end{solutionbox}

\orquestionmarks{1(ક)}{7}{DBA નું પુરુનામ લખો. DBAની ભૂમિકા અને જવાબદારીઓ સમજાવો.}
\begin{solutionbox}
\textbf{DBA}: Database Administrator (ડેટાબેઝ એડમિનિસ્ટ્રેટર)

\begin{center}
\begin{tikzpicture}[gtu block, node distance=5cm]
    \node [gtu block, fill=orange!20, minimum height=3em, minimum width=3em] (dba) {\textbf{DBA}};
    
    \node [gtu block, above of=dba, node distance=2.5cm] (design) {ડેટાબેઝ ડિઝાઇન};
    \node [gtu block, above right of=dba, node distance=3cm] (security) {સિક્યુરિટી મેનેજમેન્ટ};
    \node [gtu block, right of=dba, node distance=3.5cm] (perf) {પ્રદર્શન ટ્યુનિંગ};
    \node [gtu block, below right of=dba, node distance=3cm] (backup) {બેકઅપ અને રિકવરી};
    \node [gtu block, below of=dba, node distance=2.5cm] (maint) {મેઇન્ટેનન્સ};
    \node [gtu block, below left of=dba, node distance=3cm] (trouble) {ટ્રબલશૂટિંગ};
    \node [gtu block, left of=dba, node distance=3.5cm] (support) {યુઝર સપોર્ટ};

    \draw [gtu arrow] (dba) -- (design);
    \draw [gtu arrow] (dba) -- (security);
    \draw [gtu arrow] (dba) -- (perf);
    \draw [gtu arrow] (dba) -- (backup);
    \draw [gtu arrow] (dba) -- (maint);
    \draw [gtu arrow] (dba) -- (trouble);
    \draw [gtu arrow] (dba) -- (support);
\end{tikzpicture}
\captionof{figure}{DBA ની ભૂમિકાઓ}
\end{center}

\begin{table}[H]
    \centering
    \caption{DBA ની જવાબદારીઓ}
    \begin{tabulary}{\linewidth}{LCL}
        \toprule
        \textbf{ભૂમિકા} & \textbf{વર્ણન} \\
        \midrule
        \textbf{ડેટાબેઝ ડિઝાઇન} & કાર્યક્ષમ ડેટાબેઝ સ્કીમા બનાવે છે \\
        \textbf{સિક્યુરિટી મેનેજમેન્ટ} & યુઝર એક્સેસ કંટ્રોલ સેટ કરે છે \\
        \textbf{પ્રદર્શન ટ્યુનિંગ} & ક્વેરી અને ઇન્ડેક્સને ઓપ્ટિમાઇઝ કરે છે \\
        \textbf{બેકઅપ અને રિકવરી} & ડેટા સુરક્ષા યોજનાઓ લાગુ કરે છે \\
        \textbf{મેઇન્ટેનન્સ} & સોફ્ટવેર અપડેટ કરે છે અને પેચ લાગુ કરે છે \\
        \textbf{ટ્રબલશૂટિંગ} & ડેટાબેઝ સમસ્યાઓનો ઉકેલ કરે છે \\
        \textbf{યુઝર સપોર્ટ} & ડેટાબેઝ વપરાશકર્તાઓને તાલીમ આપે છે અને સહાય કરે છે \\
        \bottomrule
    \end{tabulary}
\end{table}

\begin{mnemonicbox}
    \textbf{મેમરી ટ્રીક:} "SPBT-MUS" (સિક્યુરિટી, પરફોર્મન્સ, બેકઅપ, ટ્રબલશૂટિંગ, મેઇન્ટેનન્સ, યુઝર સપોર્ટ)
\end{mnemonicbox}
\end{solutionbox}

\questionmarks{2(અ)}{3}{યોગ્ય ઉદાહરણ સાથે સિંગલ વેલ્યુડ અને મલ્ટી વેલ્યુડ એટ્રીબ્યુટ્સ વચ્ચેનો તફાવત સમજાવો}
\begin{solutionbox}
\begin{table}[H]
    \centering
    \caption{સિંગલ અને મલ્ટી-વેલ્યુડ એટ્રિબ્યુટ્સ}
    \begin{tabulary}{\linewidth}{LCL}
        \toprule
        \textbf{એટ્રિબ્યુટ પ્રકાર} & \textbf{વર્ણન} & \textbf{ઉદાહરણો} \\
        \midrule
        \textbf{સિંગલ-વેલ્યુડ} & દરેક એન્ટિટી ઇન્સ્ટન્સ માટે માત્ર એક મૂલ્ય ધરાવે છે & Employee ID, જન્મતારીખ, નામ \\
        \textbf{મલ્ટી-વેલ્યુડ} & એક જ એન્ટિટી માટે ઘણા મૂલ્યો ધરાવી શકે છે & ફોન નંબર, કૌશલ્યો, ઇમેઇલ એડ્રેસ \\
        \bottomrule
    \end{tabulary}
\end{table}

\begin{center}
\begin{tikzpicture}[gtu block, node distance=3cm]
    \node [entity] (emp) {EMPLOYEE};
    
    \node [attribute, above left of=emp] (id) {emp\_id};
    \node [attribute, above of=emp] (name) {name};
    \node [attribute, above right of=emp] (dob) {birth\_date};
    
    \node [multi attribute, below left of=emp] (phone) {phone\_numbers};
    \node [multi attribute, below right of=emp] (skills) {skills};
    
    \draw [gtu line] (emp) -- (id);
    \draw [gtu line] (emp) -- (name);
    \draw [gtu line] (emp) -- (dob);
    \draw [gtu line] (emp) -- (phone);
    \draw [gtu line] (emp) -- (skills);
\end{tikzpicture}
\captionof{figure}{એટ્રિબ્યુટ પ્રકારોનું ઉદાહરણ}
\end{center}

\begin{mnemonicbox}
    \textbf{મેમરી ટ્રીક:} "SIM: સિંગલ ઇઝ મિનિમલ, મલ્ટી ઇઝ મેની"
\end{mnemonicbox}
\end{solutionbox}

\questionmarks{2(બ)}{4}{E-R ડાયાગ્રામ માટે કી કન્સ્ટ્રેન્ટ્સ સમજાવો}
\begin{solutionbox}
\begin{table}[H]
    \centering
    \caption{કી કન્સ્ટ્રેન્ટ્સ}
    \begin{tabulary}{\linewidth}{LCL}
        \toprule
        \textbf{કી કન્સ્ટ્રેન્ટ} & \textbf{વર્ણન} \\
        \midrule
        \textbf{પ્રાઇમરી કી} & એન્ટિટી સેટમાં દરેક એન્ટિટીને અનન્ય રીતે ઓળખે છે \\
        \textbf{કેન્ડિડેટ કી} & કોઈપણ એટ્રિબ્યુટ જે પ્રાઇમરી કી તરીકે કામ કરી શકે \\
        \textbf{ફોરેન કી} & અન્ય એન્ટિટી સેટની પ્રાઇમરી કીનો સંદર્ભ આપે છે \\
        \textbf{સુપર કી} & એટ્રિબ્યુટ્સનો કોઈપણ સેટ જે અનન્ય રીતે એન્ટિટીને ઓળખે છે \\
        \bottomrule
    \end{tabulary}
\end{table}

\begin{center}
\begin{tikzpicture}[gtu block, node distance=2.5cm]
    \node [entity] (student) {STUDENT};
    \node [attribute, above left of=student] (sid) {\underline{student\_id}};
    \node [attribute, above of=student] (sname) {name};
    
    \node [entity, right of=student, xshift=4cm] (course) {COURSE};
    \node [attribute, above right of=course] (cid) {\underline{course\_id}};
    \node [attribute, above of=course] (ctitle) {title};
    
    \node [relationship, below of=student, xshift=3cm] (enroll) {ENROLLMENT};
    \node [attribute, below of=enroll] (edate) {enroll\_date};
    
    \draw [gtu line] (student) -- (sid);
    \draw [gtu line] (student) -- (sname);
    \draw [gtu line] (course) -- (cid);
    \draw [gtu line] (course) -- (ctitle);
    
    \draw [gtu line] (student) -- node[above, sloped] {has} (enroll);
    \draw [gtu line] (course) -- node[above, sloped] {includes} (enroll);
    \draw [gtu line] (enroll) -- (edate);
\end{tikzpicture}
\captionof{figure}{કી કન્સ્ટ્રેન્ટ્સ ઉદાહરણ}
\end{center}

\begin{mnemonicbox}
    \textbf{મેમરી ટ્રીક:} "PCFS: પ્રાઇમરી કેન્ડિડેટ્સ ફાઇન્ડ સુપરકીઝ"
\end{mnemonicbox}
\end{solutionbox}

\questionmarks{2(ક)}{7}{બેંકિંગ મેનેજમેન્ટ સિસ્ટમ માટે E-R ડાયાગ્રામ બનાવો}
\begin{solutionbox}
\begin{center}
\begin{tikzpicture}[gtu block, node distance=3cm]
    \node [entity] (cust) {CUSTOMER};
    \node [relationship, right of=cust] (has) {has};
    \node [entity, right of=has] (acc) {ACCOUNT};
    \node [relationship, right of=acc] (includes) {includes};
    \node [entity, right of=includes] (trans) {TRANSACTION};
    
    \node [relationship, below of=acc] (manages) {manages};
    \node [entity, below of=manages] (branch) {BRANCH};
    
    % Customer attributes
    \node [attribute, above left of=cust] (cid) {\underline{customer\_id}};
    \node [attribute, below left of=cust] (cname) {name};
    \node [attribute, below of=cust] (caddr) {address};
    
    % Account attributes
    \node [attribute, above of=acc] (ano) {\underline{account\_no}};
    \node [attribute, above right of=acc] (bal) {balance};
    
    % Transaction attributes
    \node [attribute, above of=trans] (tid) {\underline{trans\_id}};
    \node [attribute, right of=trans] (amt) {amount};
    
    % Branch attributes
    \node [attribute, left of=branch] (bid) {\underline{branch\_id}};
    \node [attribute, right of=branch] (bname) {location};
    
    % Connections
    \draw [gtu line] (cust) -- (has);
    \draw [gtu line] (has) -- (acc);
    \draw [gtu line] (acc) -- (includes);
    \draw [gtu line] (includes) -- (trans);
    \draw [gtu line] (branch) -- (manages);
    \draw [gtu line] (manages) -- (acc);
    
    % Attribute connections
    \draw [gtu line] (cust) -- (cid);
    \draw [gtu line] (cust) -- (cname);
    \draw [gtu line] (cust) -- (caddr);
    
    \draw [gtu line] (acc) -- (ano);
    \draw [gtu line] (acc) -- (bal);
    
    \draw [gtu line] (trans) -- (tid);
    \draw [gtu line] (trans) -- (amt);
    
    \draw [gtu line] (branch) -- (bid);
    \draw [gtu line] (branch) -- (bname);
\end{tikzpicture}
\captionof{figure}{બેંકિંગ મેનેજમેન્ટ સિસ્ટમ ER ડાયાગ્રામ}
\end{center}

\textbf{મુખ્ય એન્ટિટીઝ અને રિલેશનશિપ્સ}:
\begin{itemize}
    \item \textbf{ગ્રાહક}: ગ્રાહક માહિતી સંગ્રહિત કરે છે
    \item \textbf{એકાઉન્ટ}: વિવિધ એકાઉન્ટ પ્રકારો (સેવિંગ્સ, ચેકિંગ)
    \item \textbf{ટ્રાન્ઝેક્શન}: ડિપોઝિટ, વિડ્રોઅલ રેકોર્ડ કરે છે
    \item \textbf{બ્રાન્ચ}: વિવિધ બેંક સ્થાનો
    \item \textbf{રિલેશનશિપ્સ}: ગ્રાહકો પાસે એકાઉન્ટ છે, એકાઉન્ટમાં ટ્રાન્ઝેક્શન છે, બ્રાન્ચ એકાઉન્ટ મેનેજ કરે છે
\end{itemize}

\begin{mnemonicbox}
    \textbf{મેમરી ટ્રીક:} "CATB: કસ્ટમર્સ એક્સેસ ટ્રાન્ઝેક્શન્સ એટ બ્રાન્ચીસ"
\end{mnemonicbox}
\end{solutionbox}

\orquestionmarks{2(અ)}{3}{યોગ્ય ઉદાહરણ સાથે સ્પેશિયલાઈઝેશન અને જનરલાઈઝેશન વચ્ચેનો તફાવત સમજાવો}
\begin{solutionbox}
\begin{table}[H]
    \centering
    \caption{સ્પેશિયલાઈઝેશન vs જનરલાઈઝેશન}
    \begin{tabulary}{\linewidth}{LCL}
        \toprule
        \textbf{વિચાર} & \textbf{દિશા} & \textbf{વર્ણન} \\
        \midrule
        \textbf{સ્પેશિયલાઈઝેશન} & ટોપ-ડાઉન & સામાન્ય એન્ટિટીને વધુ ચોક્કસ સબ-એન્ટિટીઓમાં વિભાજિત કરવું (વ્યક્તિ $\to$ વિદ્યાર્થી, કર્મચારી) \\
        \textbf{જનરલાઈઝેશન} & બોટમ-અપ & સમાન એન્ટિટીઓને ઉચ્ચ-સ્તરીય એન્ટિટીમાં જોડવું (કાર, ટ્રક $\to$ વાહન) \\
        \bottomrule
    \end{tabulary}
\end{table}

\begin{center}
\begin{tikzpicture}[gtu block, node distance=2cm]
    \node [entity] (person) {PERSON};
    \node [relationship, below of=person] (isa) {IS-A};
    \node [entity, below left of=isa, xshift=-1cm] (stud) {STUDENT};
    \node [entity, below right of=isa, xshift=1cm] (emp) {EMPLOYEE};
    
    \draw [gtu line] (person) -- (isa);
    \draw [gtu line] (isa) -- (stud);
    \draw [gtu line] (isa) -- (emp);
    
    \node [attribute, left of=person] {person\_id};
    \node [attribute, left of=stud] {major};
    \node [attribute, right of=emp] {salary};
    
    \draw [gtu line] (person) -- ++(-1.5,0);
\end{tikzpicture}
\captionof{figure}{સ્પેશિયલાઈઝેશન/જનરલાઈઝેશન ઉદાહરણ}
\end{center}

\begin{mnemonicbox}
    \textbf{મેમરી ટ્રીક:} "SG-TD-BU: સ્પેશિયલાઈઝેશન ગોઝ ટોપ-ડાઉન, જનરલાઈઝેશન બિલ્ડ્સ અપ"
\end{mnemonicbox}
\end{solutionbox}

\orquestionmarks{2(બ)}{4}{ચાસ્પ ટ્રેપની વ્યાખ્યા લખો. તે ક્યારે ઉદ્ભવે છે તે સમજાવો. ચાસ્પ ટ્રેપ માટેનો ઉપાય સમજાવો}
\begin{solutionbox}
\textbf{ચાસ્પ ટ્રેપ}: ER ડાયાગ્રામમાં ઉદ્ભવતી સમસ્યા જ્યારે એન્ટિટીઓ વચ્ચે મલ્ટિપલ પાથ હોય છે, જેથી રિલેશનશિપના અર્થઘટનમાં અસ્પષ્ટતા આવે છે.

\begin{table}[H]
    \centering
    \caption{ચાસ્પ ટ્રેપ વિગતો}
    \begin{tabulary}{\linewidth}{LCL}
        \toprule
        \textbf{પાસું} & \textbf{વર્ણન} \\
        \midrule
        \textbf{ઉદ્ભવ} & જ્યારે એન્ટિટી પ્રકારો વચ્ચે બે અથવા વધુ અલગ પાથ હોય જે ચક્ર બનાવે છે \\
        \textbf{સમસ્યા} & અયોગ્ય અથવા અસ્પષ્ટ ક્વેરી પરિણામો તરફ દોરી જાય છે \\
        \textbf{ઉકેલ} & એક રિલેશનશિપને તોડવું અથવા ઇચ્છિત પાથને સ્પષ્ટ કરવા માટે કન્સ્ટ્રેન્ટ્સ ઉમેરવા \\
        \bottomrule
    \end{tabulary}
\end{table}

\begin{center}
\begin{tikzpicture}[gtu block, node distance=2.5cm]
    \node [entity] (student) {STUDENT};
    \node [relationship, right of=student] (enroll) {enrolled\_in};
    \node [entity, right of=enroll] (section) {SECTION};
    \node [relationship, below of=section] (part) {part\_of};
    \node [entity, below of=part] (course) {COURSE};
    \node [relationship, left of=course] (studies) {studies};
    
    \draw [gtu line] (student) -- (enroll);
    \draw [gtu line] (enroll) -- (section);
    \draw [gtu line] (section) -- (part);
    \draw [gtu line] (part) -- (course);
    \draw [gtu line, dashed, red] (student) |- (studies);
    \draw [gtu line, dashed, red] (studies) -- (course);
\end{tikzpicture}
\captionof{figure}{ચાસ્પ ટ્રેપ (સાયકલ) ઉદાહરણ}
\end{center}

\begin{mnemonicbox}
    \textbf{મેમરી ટ્રીક:} "COP: સાયકલ્સ ઓફ પાથસ નીડ બ્રેકિંગ"
\end{mnemonicbox}
\end{solutionbox}

\orquestionmarks{2(ક)}{7}{કોલેજ મેનેજમેન્ટ સિસ્ટમ માટે E-R ડાયાગ્રામ બનાવો}
\begin{solutionbox}
\begin{center}
\begin{tikzpicture}[gtu block, node distance=2.5cm]
    \node [entity] (dept) {DEPARTMENT};
    \node [relationship, above of=dept] (belongs) {belongs\_to};
    \node [entity, above of=belongs] (stud) {STUDENT};
    
    \node [relationship, right of=dept, xshift=1cm] (works) {works\_in};
    \node [entity, right of=works, xshift=1cm] (faculty) {FACULTY};
    
    \node [relationship, below of=dept] (offers) {offers};
    \node [entity, below of=offers] (course) {COURSE};
    
    \node [relationship, right of=course, xshift=1cm] (teaches) {teaches};
    
    \node [relationship, left of=course, xshift=-1cm] (enrolls) {enrolls};
    
    \node [relationship, below of=course] (has) {has};
    \node [entity, below of=has] (exam) {EXAM};
    
    \node [relationship, left of=exam, xshift=-1cm] (takes) {takes};
    
    % Connections
    \draw [gtu line] (stud) -- (belongs);
    \draw [gtu line] (belongs) -- (dept);
    
    \draw [gtu line] (faculty) -- (works);
    \draw [gtu line] (works) -- (dept);
    
    \draw [gtu line] (dept) -- (offers);
    \draw [gtu line] (offers) -- (course);
    
    \draw [gtu line] (faculty) |- (teaches);
    \draw [gtu line] (teaches) -- (course);
    
    \draw [gtu line] (stud) -| (enrolls);
    \draw [gtu line] (enrolls) |- (course);
    
    \draw [gtu line] (course) -- (has);
    \draw [gtu line] (has) -- (exam);
    
    \draw [gtu line] (stud) |- (takes) -| (exam);
    
\end{tikzpicture}
\captionof{figure}{કોલેજ મેનેજમેન્ટ સિસ્ટમ}
\end{center}

\textbf{મુખ્ય એન્ટિટીઝ અને રિલેશનશિપ્સ}:
\begin{itemize}
    \item \textbf{વિદ્યાર્થી (Student)}: વિદ્યાર્થી વિગતો સંગ્રહિત કરે છે
    \item \textbf{વિભાગ (Department)}: શૈક્ષણિક વિભાગો
    \item \textbf{ફેકલ્ટી (Faculty)}: શિક્ષકો અને પ્રોફેસરો
    \item \textbf{કોર્સ (Course)}: ભણાવવામાં આવતા વિષયો
    \item \textbf{પરીક્ષા (Exam)}: મૂલ્યાંકન કાર્યક્રમો
    \item \textbf{રિલેશનશિપ્સ}: વિદ્યાર્થીઓ કોર્સમાં એનરોલ થાય છે, ફેકલ્ટી કોર્સ શીખવે છે, વિભાગો કોર્સ ઓફર કરે છે
\end{itemize}

\begin{mnemonicbox}
    \textbf{મેમરી ટ્રીક:} "SDFCE: સ્ટુડન્ટ્સ ડિલાઇટ ફેકલ્ટી બાય કમ્પ્લીટિંગ એક્ઝામ્સ"
\end{mnemonicbox}
\end{solutionbox}

\questionmarks{3(અ)}{3}{GROUP BY ક્લોઝ ઉદાહરણ સાથે સમજાવો.}
\begin{solutionbox}
\textbf{GROUP BY} ક્લોઝ સમાન મૂલ્યો ધરાવતી રો સારાંશ રોમાં જૂથ કરે છે.

\begin{table}[H]
    \centering
    \caption{GROUP BY ક્લોઝ}
    \begin{tabulary}{\linewidth}{LCL}
        \toprule
        \textbf{ફીચર} & \textbf{વર્ણન} \\
        \midrule
        \textbf{હેતુ} & એકસરખા ડેટાને એગ્રીગેટ ફંક્શન માટે જૂથોમાં ગોઠવે છે \\
        \textbf{ઉપયોગ} & એગ્રીગેટ ફંક્શન (COUNT, SUM, AVG, MAX, MIN) સાથે વપરાય છે \\
        \textbf{સિન્ટેક્સ} & \code{SELECT column1, COUNT(*) FROM table GROUP BY column1;} \\
        \bottomrule
    \end{tabulary}
\end{table}

\begin{lstlisting}[language=SQL]
SELECT department, AVG(salary) 
FROM employees
GROUP BY department;
\end{lstlisting}

\begin{mnemonicbox}
    \textbf{મેમરી ટ્રીક:} "GAS: ગ્રુપ એન્ડ સમરાઈઝ"
\end{mnemonicbox}
\end{solutionbox}

\questionmarks{3(બ)}{4}{Data Definition Language (DDL) કમાન્ડની યાદી બનાવો. કોઈ પણ ૨ DDL કમાન્ડ ઉદાહરણ સાથે સમજાવો.}
\begin{solutionbox}
\textbf{DDL કમાન્ડ્સ}: CREATE, ALTER, DROP, TRUNCATE, RENAME

\begin{table}[H]
    \centering
    \caption{DDL કમાન્ડ્સ}
    \begin{tabulary}{\linewidth}{LCL}
        \toprule
        \textbf{કમાન્ડ} & \textbf{વર્ણન} & \textbf{ઉદાહરણ} \\
        \midrule
        \textbf{CREATE} & ડેટાબેઝ ઓબ્જેક્ટ્સ જેમ કે ટેબલ, વ્યૂ, ઇન્ડેક્સ બનાવે છે & \code{CREATE TABLE s (id INT PK, n TEXT);} \\
        \textbf{ALTER} & મૌજૂદા ડેટાબેઝ ઓબ્જેક્ટ્સ સુધારે છે & \code{ALTER TABLE s ADD COLUMN e TEXT;} \\
        \textbf{DROP} & ડેટાબેઝ ઓબ્જેક્ટ્સ દૂર કરે છે & \code{DROP TABLE s;} \\
        \textbf{TRUNCATE} & ટેબલમાંથી બધા રેકોર્ડ્સ દૂર કરે છે & \code{TRUNCATE TABLE s;} \\
        \bottomrule
    \end{tabulary}
\end{table}

\begin{mnemonicbox}
    \textbf{મેમરી ટ્રીક:} "CADTR: ક્રિએટ, ઓલ્ટર, ડ્રોપ, ટ્રન્કેટ, રીનેમ"
\end{mnemonicbox}
\end{solutionbox}

\questionmarks{3(ક)}{7}{enr\_no, name, percent, branch ફિલ્ડ ધરાવતા Students ટેબલ પર નીચેની Query perform કરો.}
\begin{solutionbox}
\begin{lstlisting}[language=SQL]
-- ૧. Students ટેબલના તમામ રેકોર્ડ ડિસ્પ્લે કરો.
SELECT * FROM Students;

-- ૨. ડુપ્લીકેટ વેલ્યુ સિવાય માત્ર branch ડિસ્પ્લે કરો.
SELECT DISTINCT branch FROM Students;

-- ૩. name નાં ઉતરતા ક્રમમાં તમામ રેકોર્ડ ડિસ્પ્લે કરો.
SELECT * FROM Students ORDER BY name DESC;

-- ૪. સરનામું સ્ટોર કરવા માટે "address" નામથી નવી કોલમ ઉમેરો.
ALTER TABLE Students ADD address VARCHAR(100);

-- ૫. "ICT" બ્રાંચ ધરાવતા બધા વિદ્યાર્થીને ડિસ્પ્લે કરો.
SELECT * FROM Students WHERE branch = 'ICT';

-- ૬. ૬૦ કરતા ઓછા percent ધરાવતા વિદ્યાર્થીઓને ડીલીટ કરો.
DELETE FROM Students WHERE percent < 60;

-- ૭. "S" થી શરૂ થતા તમામ વિદ્યાર્થીઓના નામ ડિસ્પ્લે કરો.
SELECT * FROM Students WHERE name LIKE 'S%';
\end{lstlisting}

\begin{table}[H]
    \centering
    \caption{ક્વેરી સમજૂતી}
    \begin{tabulary}{\linewidth}{LCL}
        \toprule
        \textbf{ક્વેરી} & \textbf{હેતુ} \\
        \midrule
        \textbf{SELECT} & ટેબલમાંથી ડેટા મેળવે છે \\
        \textbf{DISTINCT} & ડુપ્લિકેટ મૂલ્યો દૂર કરે છે \\
        \textbf{ORDER BY} & પરિણામોને ચોક્કસ ક્રમમાં ગોઠવે છે \\
        \textbf{ALTER TABLE} & ટેબલ સ્ટ્રક્ચર સુધારે છે \\
        \textbf{WHERE} & શરતો પર આધારિત રેકોર્ડ્સ ફિલ્ટર કરે છે \\
        \textbf{DELETE} & શરતો મેળવતા રેકોર્ડ્સ દૂર કરે છે \\
        \textbf{LIKE} & સ્ટ્રિંગ તુલનામાં પેટર્ન મેચિંગ \\
        \bottomrule
    \end{tabulary}
\end{table}

\begin{mnemonicbox}
    \textbf{મેમરી ટ્રીક:} "SDOAWDL: સિલેક્ટ ડિસ્ટિંક્ટ ઓર્ડર ઓલ્ટર વ્હેર ડિલીટ લાઇક"
\end{mnemonicbox}
\end{solutionbox}

\orquestionmarks{3(અ)}{3}{સિન્ટેક્સ અને ઉદાહરણ સાથે GRANT કમાન્ડ સમજાવો.}
\begin{solutionbox}
\textbf{GRANT} કમાન્ડ વપરાશકર્તાઓને ડેટાબેઝ ઓબ્જેક્ટ્સ પર ચોક્કસ અધિકારો આપે છે.

\begin{table}[H]
    \centering
    \caption{GRANT કમાન્ડ}
    \begin{tabulary}{\linewidth}{LCL}
        \toprule
        \textbf{ઘટક} & \textbf{વર્ણન} \\
        \midrule
        \textbf{સિન્ટેક્સ} & \code{GRANT privilege ON object TO user;} \\
        \textbf{પ્રિવિલેજીસ} & SELECT, INSERT, UPDATE, DELETE \\
        \textbf{ઓબ્જેક્ટ્સ} & ટેબલ્સ, વ્યૂ, સિક્વેન્સિસ, વગેરે \\
        \bottomrule
    \end{tabulary}
\end{table}

\begin{lstlisting}[language=SQL]
GRANT SELECT, UPDATE ON employees TO user1;
GRANT ALL PRIVILEGES ON db.* TO user2 WITH GRANT OPTION;
\end{lstlisting}

\begin{mnemonicbox}
    \textbf{મેમરી ટ્રીક:} "GPO: ગ્રાન્ટ પ્રિવિલેજીસ ટુ અધર્સ"
\end{mnemonicbox}
\end{solutionbox}

\orquestionmarks{3(બ)}{4}{Truncate અને Drop કમાન્ડનો તફાવત લખો.}
\begin{solutionbox}
\begin{table}[H]
    \centering
    \caption{TRUNCATE vs DROP}
    \begin{tabulary}{\linewidth}{LCL}
        \toprule
        \textbf{ફીચર} & \textbf{TRUNCATE} & \textbf{DROP} \\
        \midrule
        \textbf{હેતુ} & ટેબલથી બધી પંક્તિઓ દૂર કરે છે & સંપૂર્ણ ટેબલ સ્ટ્રક્ચર દૂર કરે છે \\
        \textbf{સ્ટ્રક્ચર} & ટેબલ સ્ટ્રક્ચર જાળવી રાખે છે & ટેબલની વ્યાખ્યા સંપૂર્ણપણે દૂર કરે છે \\
        \textbf{રિકવરી} & સરળતાથી રોલબેક નથી કરી શકાતું & કમિટ થાય ત્યાં સુધી પુનઃપ્રાપ્ત કરી શકાય છે \\
        \textbf{સ્પીડ} & DELETE કરતાં ઝડપી & ઝડપી ઓપરેશન \\
        \textbf{ટ્રિગર્સ} & ટ્રિગર્સ સક્રિય કરતું નથી & ટ્રિગર્સ સક્રિય કરતું નથી \\
        \bottomrule
    \end{tabulary}
\end{table}

\begin{lstlisting}[language=SQL]
-- Truncate ઉદાહરણ
TRUNCATE TABLE students;

-- Drop ઉદાહરણ
DROP TABLE students;
\end{lstlisting}

\begin{mnemonicbox}
    \textbf{મેમરી ટ્રીક:} "TRC-DST: ટ્રન્કેટ રિમૂવ્સ કન્ટેન્ટ્સ, ડ્રોપ ડિસ્ટ્રોય્સ સ્ટ્રક્ચર ટોટલી"
\end{mnemonicbox}
\end{solutionbox}

\orquestionmarks{3(ક)}{7}{નીચેની Query ના આઉટપુટ લખો.}
\begin{solutionbox}
\begin{table}[H]
    \centering
    \caption{SQL ક્વેરી આઉટપુટ}
    \begin{tabulary}{\linewidth}{LCL}
        \toprule
        \textbf{ક્વેરી} & \textbf{આઉટપુટ} & \textbf{સમજૂતી} \\
        \midrule
        \textbf{ABS(-23), ABS(49)} & 23, 49 & નિરપેક્ષ મૂલ્ય પાછું આપે છે \\
        \textbf{SQRT(25), SQRT(81)} & 5, 9 & વર્ગમૂળ પાછું આપે છે \\
        \textbf{POWER(3,2), POWER(-2,3)} & 9, -8 & $x^y$ \\
        \textbf{MOD(15,4), MOD(21,3)} & 3, 0 & વિભાજન પછી શેષ પાછો આપે છે \\
        \textbf{ROUND(123.446,1), ROUND(123.456,2)} & 123.4, 123.46 & ચોક્કસ દશાંશ જગ્યાઓ પર રાઉન્ડ કરે છે \\
        \textbf{CEIL(234.45), CEIL(-234.45)} & 235, -234 & નજીકના પૂર્ણાંક સુધી ઉપર રાઉન્ડ કરે છે \\
        \textbf{FLOOR(-12.7), FLOOR(12.7)} & -13, 12 & નજીકના પૂર્ણાંક સુધી નીચે રાઉન્ડ કરે છે \\
        \bottomrule
    \end{tabulary}
\end{table}

\begin{lstlisting}[language=SQL]
SELECT ABS(-23), ABS(49);          -- 23, 49
SELECT SQRT(25), SQRT(81);         -- 5, 9
SELECT POWER(3,2), POWER(-2,3);    -- 9, -8
SELECT MOD(15,4), MOD(21,3);       -- 3, 0
SELECT ROUND(123.446,1), ROUND(123.456,2); -- 123.4, 123.46
SELECT CEIL(234.45), CEIL(-234.45);  -- 235, -234
SELECT FLOOR(-12.7), FLOOR(12.7);    -- -13, 12
\end{lstlisting}

\begin{mnemonicbox}
    \textbf{મેમરી ટ્રીક:} "ASPMRCF: એબ્સોલ્યુટ સ્ક્વેર પાવર મોડ્યુલો રાઉન્ડ સીલિંગ ફ્લોર"
\end{mnemonicbox}
\end{solutionbox}

\questionmarks{4(અ)}{3}{SQLમાં ડેટા ટાઈપની યાદી બનાવો. કોઈ પણ ૨ ડેટા ટાઈપ ઉદાહરણ સાથે સમજાવો.}
\begin{solutionbox}
\textbf{SQL ડેટા ટાઈપ}: INTEGER, FLOAT, VARCHAR, CHAR, DATE, DATETIME, BOOLEAN, BLOB

\begin{table}[H]
    \centering
    \caption{SQL ડેટા ટાઈપ}
    \begin{tabulary}{\linewidth}{LCL}
        \toprule
        \textbf{ડેટા ટાઈપ} & \textbf{વર્ણન} & \textbf{ઉદાહરણ} \\
        \midrule
        \textbf{INTEGER} & દશાંશ પોઇન્ટ વિના પૂર્ણ સંખ્યાઓ & \code{id INTEGER = 101} \\
        \textbf{VARCHAR} & પરિવર્તનશીલ-લંબાઈ સ્ટ્રિંગ & \code{name VARCHAR(50) = 'John'} \\
        \textbf{DATE} & તારીખ મૂલ્યો સંગ્રહિત કરે છે (YYYY-MM-DD) & \code{birth\_date DATE = '2000-05-15'} \\
        \textbf{FLOAT} & ફ્લોટિંગ પોઇન્ટ સાથે દશાંશ સંખ્યાઓ & \code{salary FLOAT = 45000.50} \\
        \bottomrule
    \end{tabulary}
\end{table}

\begin{lstlisting}[language=SQL]
CREATE TABLE employees (
    id INTEGER,
    name VARCHAR(50),
    salary FLOAT
);
\end{lstlisting}

\begin{mnemonicbox}
    \textbf{મેમરી ટ્રીક:} "IVDB: ઈન્ટિજર અને વારચાર આર ડેટાબેઝ બેસિક્સ"
\end{mnemonicbox}
\end{solutionbox}

\questionmarks{4(બ)}{4}{Full function dependency ઉદાહરણ સાથે સમજાવો.}
\begin{solutionbox}
\textbf{Full Function Dependency}: જ્યારે Y, X પર ફંક્શનલી ડિપેન્ડન્ટ હોય, પરંતુ X ના કોઈ સબસેટ પર નહીં.

\begin{table}[H]
    \centering
    \caption{પૂર્ણ ફંક્શન ડિપેન્ડન્સી}
    \begin{tabulary}{\linewidth}{LCL}
        \toprule
        \textbf{વિચાર} & \textbf{વર્ણન} & \textbf{ઉદાહરણ} \\
        \midrule
        \textbf{વ્યાખ્યા} & એટ્રિબ્યુટ B, A પર પૂર્ણપણે ફંક્શનલી ડિપેન્ડન્ટ છે જો B સંપૂર્ણ A પર આધાર રાખે છે & Student\_ID $\to$ Name (પૂર્ણ ડિપેન્ડન્સી) \\
        \textbf{નોન-ઉદાહરણ} & જ્યારે એટ્રિબ્યુટ કોમ્પોઝિટ કીના માત્ર ભાગ પર આધાર રાખે છે & \{Student\_ID, Course\_ID\} $\to$ Student\_Name (આંશિક) \\
        \bottomrule
    \end{tabulary}
\end{table}

\begin{center}
\begin{tikzpicture}[gtu block, node distance=2.5cm]
    \node [attribute] (sid) {Student\_ID};
    \node [attribute, right of=sid] (sname) {Student\_Name};
    \draw [gtu arrow] (sid) -- (sname);
    
    \node [gtu container, fit=(sid) (sname), label=below:Full Dependency] {};
    
    \node [attribute, below of=sid] (cid) {Course\_ID};
    \node [attribute, right of=cid] (cname) {Course\_Name};
    \draw [gtu arrow] (cid) -- (cname);
    
    \node [gtu container, fit=(cid) (cname), label=below:Full Dependency] {};
    
    \node [attribute, right of=sname, xshift=1cm] (sid2) {Student\_ID};
    \node [attribute, right of=cname, xshift=1cm] (cid2) {Course\_ID};
    \node [attribute, right of=sid2, yshift=-1.25cm] (grade) {Grade};
    
    \draw [gtu arrow] (sid2) -- (grade);
    \draw [gtu arrow] (cid2) -- (grade);
    
    \node [gtu container, fit=(sid2) (cid2) (grade), label=below:Full Dependency (Composite Key)] {};
\end{tikzpicture}
\captionof{figure}{Full Functional Dependency}
\end{center}

\begin{mnemonicbox}
    \textbf{મેમરી ટ્રીક:} "FFD: ફુલ, નોટ ફ્રેક્શન ઓફ ડિપેન્ડન્સી"
\end{mnemonicbox}
\end{solutionbox}

\questionmarks{4(ક)}{7}{નોર્મલાઇઝેશનની વ્યાખ્યા આપો. 2NF (સેકન્ડ નોર્મલ ફોર્મ) ઉદાહરણ અને ઉકેલ સાથે સમજાવો.}
\begin{solutionbox}
\textbf{નોર્મલાઈઝેશન}: ડેટાબેઝની રચના કરવાની પ્રક્રિયા જેથી મોટા ટેબલને નાના ટેબલોમાં વિભાજિત કરીને અને તેમની વચ્ચે સંબંધો વ્યાખ્યાયિત કરીને, રિડન્ડન્સી અને ડિપેન્ડન્સી ઘટાડવામાં આવે.

\textbf{2NF (સેકન્ડ નોર્મલ ફોર્મ)}:
\begin{itemize}
    \item ટેબલ 2NF માં છે જો તે 1NF માં હોય અને કોઈ નોન-પ્રાઇમ એટ્રિબ્યુટ કેન્ડિડેટ કીના કોઈ પણ યોગ્ય સબસેટ પર આધાર રાખતું ન હોય.
\end{itemize}

\begin{table}[H]
    \centering
    \caption{2NF પહેલાં (સમસ્યા)}
    \begin{tabulary}{\linewidth}{LCL}
        \toprule
        \textbf{ટેબલ સ્કીમા} & \textbf{સમસ્યા} \\
        \midrule
        \textbf{Order(Order\_ID, Product\_ID, Product\_Name, Quantity, Price)} & Product\_Name માત્ર Product\_ID પર આધાર રાખે છે, સંપૂર્ણ કી પર નહીં \\
        \bottomrule
    \end{tabulary}
\end{table}

\begin{table}[H]
    \centering
    \caption{2NF પછી (ઉકેલ)}
    \begin{tabulary}{\linewidth}{LCL}
        \toprule
        \textbf{નવી સ્કીમા} & \textbf{ઉકેલ} \\
        \midrule
        \textbf{Order(Order\_ID, Product\_ID, Quantity)} & માત્ર પૂર્ણ કી ડિપેન્ડન્સી \\
        \textbf{Product(Product\_ID, Product\_Name, Price)} & પ્રોડક્ટ વિગતો માત્ર Product\_ID પર આધાર રાખે છે \\
        \bottomrule
    \end{tabulary}
\end{table}

\begin{center}
\begin{tikzpicture}[gtu block, node distance=3cm]
    \node [entity] (order) {ORDER};
    \node [attribute, above left of=order] (oid) {\underline{order\_id}};
    \node [attribute, above of=order] (pid) {\underline{product\_id}};
    \node [attribute, above right of=order] (qty) {quantity};
    
    \node [entity, right of=order, xshift=2cm] (prod) {PRODUCT};
    \node [attribute, above left of=prod] (pid2) {\underline{product\_id}};
    \node [attribute, above of=prod] (pname) {product\_name};
    \node [attribute, above right of=prod] (price) {price};
    
    \node [relationship, between=order and prod] (contains) {contains};
    
    \draw [gtu line] (order) -- (oid);
    \draw [gtu line] (order) -- (pid);
    \draw [gtu line] (order) -- (qty);
    
    \draw [gtu line] (prod) -- (pid2);
    \draw [gtu line] (prod) -- (pname);
    \draw [gtu line] (prod) -- (price);
    
    \draw [gtu line] (order) -- (contains);
    \draw [gtu line] (contains) -- (prod);
\end{tikzpicture}
\captionof{figure}{2NF ઉકેલ ER ડાયાગ્રામ}
\end{center}

\begin{mnemonicbox}
    \textbf{મેમરી ટ્રીક:} "2NF-PPD: પાર્શિયલ ડિપેન્ડન્સી પ્રોબ્લેમ્સ ડિવાઇડેડ"
\end{mnemonicbox}
\end{solutionbox}

\orquestionmarks{4(અ)}{3}{કમાન્ડ સમજાવવો. ૧) To\_Number (), ૨) To\_Char()}
\begin{solutionbox}
\begin{table}[H]
    \centering
    \caption{કન્વર્ઝન ફંક્શન}
    \begin{tabulary}{\linewidth}{LCL}
        \toprule
        \textbf{ફંક્શન} & \textbf{હેતુ} & \textbf{સિન્ટેક્સ} & \textbf{ઉદાહરણ} \\
        \midrule
        \textbf{TO\_NUMBER()} & સ્ટ્રિંગને નંબરમાં રૂપાંતરિત કરે છે & \code{TO\_NUMBER(s, [fmt])} & \code{TO\_NUMBER('123.45')} \\
        \textbf{TO\_CHAR()} & નંબર/તારીખને સ્ટ્રિંગમાં રૂપાંતરિત કરે છે & \code{TO\_CHAR(v, [fmt])} & \code{TO\_CHAR(1234, '9999')} \\
        \bottomrule
    \end{tabulary}
\end{table}

\begin{lstlisting}[language=SQL]
-- સ્ટ્રિંગને નંબરમાં રૂપાંતરિત કરે છે
SELECT TO_NUMBER('123.45') FROM dual;  -- 123.45

-- તારીખને ફોર્મેટેડ સ્ટ્રિંગમાં રૂપાંતરિત કરે છે
SELECT TO_CHAR(SYSDATE, 'DD-MON-YYYY') FROM dual;  -- 20-JAN-2024

-- નંબરને ફોર્મેટેડ સ્ટ્રિંગમાં રૂપાંતરિત કરે છે
SELECT TO_CHAR(1234.56, '$9,999.99') FROM dual;  -- $1,234.56
\end{lstlisting}

\begin{mnemonicbox}
    \textbf{મેમરી ટ્રીક:} "NC: નંબર્સ એન્ડ કેરેક્ટર્સ કન્વર્ઝન"
\end{mnemonicbox}
\end{solutionbox}

\orquestionmarks{4(બ)}{4}{1NF (ફર્સ્ટ નોર્મલ ફોર્મ) ઉદાહરણ અને ઉકેલ સાથે સમજાવો.}
\begin{solutionbox}
\textbf{1NF (ફર્સ્ટ નોર્મલ ફોર્મ)}: એક રિલેશન 1NF માં છે જો તેમાં કોઈ રિપીટિંગ ગ્રુપ્સ અથવા એરે ન હોય.

\begin{table}[H]
    \centering
    \caption{1NF ઉદાહરણ}
    \begin{tabulary}{\linewidth}{LCL}
        \toprule
        \textbf{સ્ટેટ} & \textbf{સ્કીમા/ઉદાહરણ} & \textbf{રીમાર્ક} \\
        \midrule
        \textbf{1NF પહેલાં} & **Student(ID, Name, Courses)** & મલ્ટિપલ વેલ્યુ છે \\
         & (101, John, "Math,Science,History") & સમસ્યા \\
        \textbf{1NF પછી} & **Student(ID, Name, Course)** & એટોમિક વેલ્યુઝ \\
         & (101, John, Math), (101, John, Science)... & ઉકેલ \\
        \bottomrule
    \end{tabulary}
\end{table}

\begin{center}
\begin{tikzpicture}[gtu block, node distance=2.5cm]
    \node [entity, dashed] (before) {STUDENT (Unnormalized)};
    \node [multi attribute, below of=before] (courses) {courses};
    \draw [gtu line, dashed] (before) -- (courses);
    
    \node [right of=before, xshift=2cm] (arrow) {$\implies$};
    
    \node [entity, right of=arrow, xshift=2cm] (after) {STUDENT (1NF)};
    \node [attribute, below of=after] (course) {course};
    \draw [gtu line] (after) -- (course);
    
    \node [attribute, above of=after] (id) {id};
    \node [attribute, right of=after] (name) {name};
    \draw [gtu line] (after) -- (id);
    \draw [gtu line] (after) -- (name);
\end{tikzpicture}
\captionof{figure}{1NF Conversion}
\end{center}

\begin{mnemonicbox}
    \textbf{મેમરી ટ્રીક:} "1NF-ARM: એટોમિક વેલ્યુઝ રિમૂવ મલ્ટિવેલ્યુઝ"
\end{mnemonicbox}
\end{solutionbox}

\orquestionmarks{4(ક)}{7}{SQL માં Function dependency સમજાવો. Partial function dependency ઉદાહરણ સાથે સમજાવો.}
\begin{solutionbox}
\textbf{ફંક્શનલ ડિપેન્ડન્સી}: એક સંબંધ જ્યાં એક એટ્રિબ્યુટ બીજા એટ્રિબ્યુટનું મૂલ્ય નક્કી કરે છે.
\textbf{નોટેશન}: $X \to Y$ (X, Y ને નક્કી કરે છે)

\textbf{પાર્શિયલ ફંક્શનલ ડિપેન્ડન્સી}: જ્યારે એક એટ્રિબ્યુટ કમ્પોઝિટ પ્રાઇમરી કીના માત્ર એક ભાગ પર આધાર રાખે છે.

\begin{table}[H]
    \centering
    \caption{પાર્શિયલ ડિપેન્ડન્સી}
    \begin{tabulary}{\linewidth}{LCL}
        \toprule
        \textbf{વિચાર} & \textbf{ઉદાહરણ} & \textbf{સમજૂતી} \\
        \midrule
        \textbf{કમ્પોઝિટ કી} & \{Student\_ID, Course\_ID\} & સાથે મળીને પ્રાઇમરી કી બનાવે છે \\
        \textbf{પાર્શિયલ ડિપેન્ડન્સી} & \{Student\_ID, Course\_ID\} $\to$ Student\_Name & Student\_Name માત્ર Student\_ID પર આધાર રાખે છે \\
        \textbf{સમસ્યા} & અપડેટ એનોમલીઝ, ડેટા રિડન્ડન્સી & એક જ વિદ્યાર્થીનું નામ ઘણા બધા કોર્સ માટે પુનરાવર્તિત થાય છે \\
        \bottomrule
    \end{tabulary}
\end{table}

\begin{center}
\begin{tikzpicture}[gtu block, node distance=2.5cm]
    \node [attribute] (sid) {Student\_ID};
    \node [attribute, right of=sid, xshift=2cm] (cid) {Course\_ID};
    \node [attribute, below of=sid] (sname) {Student\_Name};
    \node [attribute, below of=cid] (cname) {Course\_Name};
    \node [attribute, below right of=sname, xshift=1cm] (grade) {Grade};
    
    \draw [gtu arrow, red] (sid) -- (sname);
    \node [below of=sname, node distance=0.8cm, font=\small, red] {Partial Dependency};
    
    \draw [gtu arrow] (cid) -- (cname);
    
    \draw [gtu arrow] (sid) -- (grade);
    \draw [gtu arrow] (cid) -- (grade);
    
    \node [gtu container, fit=(sid) (cid) (grade), label=above:Full Dependency (PK to Grade)] {};
\end{tikzpicture}
\captionof{figure}{Partial Functional Dependency}
\end{center}

\textbf{ઉકેલ}: અલગ ટેબલોમાં વિભાજિત કરો જ્યાં દરેક નોન-કી એટ્રિબ્યુટ કી પર સંપૂર્ણપણે આધારિત હોય.

\begin{mnemonicbox}
    \textbf{મેમરી ટ્રીક:} "PD-CPK: પાર્શિયલ ડિપેન્ડન્સી - કમ્પોનન્ટ ઓફ પ્રાઇમરી કી"
\end{mnemonicbox}
\end{solutionbox}

\questionmarks{5(અ)}{3}{Transaction નાં ગુણધર્મો ઉદાહરણ સાથે સમજાવો.}
\begin{solutionbox}
\textbf{Transaction ગુણધર્મો} (ACID):

\begin{table}[H]
    \centering
    \caption{ACID ગુણધર્મો}
    \begin{tabulary}{\linewidth}{LCL}
        \toprule
        \textbf{ગુણધર્મ} & \textbf{વર્ણન} & \textbf{ઉદાહરણ} \\
        \midrule
        \textbf{એટોમિસિટી} & બધા ઓપરેશનો સફળતાપૂર્વક પૂર્ણ થાય છે અથવા એક પણ થતું નથી & બેંક ટ્રાન્સફર: ડેબિટ અને ક્રેડિટ બંને થાય અથવા બંને ન થાય \\
        \textbf{કન્સિસ્ટન્સી} & ડેટાબેઝ પહેલા અને પછી માન્ય સ્થિતિમાં રહે છે & એકાઉન્ટ બેલેન્સ કન્સ્ટ્રેન્ટ્સ માન્ય રહે છે \\
        \textbf{આઇસોલેશન} & ટ્રાન્ઝેક્શન એવી રીતે એક્ઝિક્યુટ થાય છે જાણે તે એકમાત્ર હોય & બે યુઝર એક જ રેકોર્ડ અપડેટ કરી રહ્યા હોય ત્યારે દખલ કરતા નથી \\
        \textbf{ડ્યુરેબિલિટી} & કમિટ કરેલા ફેરફારો સિસ્ટમ નિષ્ફળતા પછી પણ ટકી રહે છે & એકવાર પુષ્ટિ થઈ જાય, પછી વીજળી જતી રહે તો પણ ડિપોઝિટ યથાવત રહે છે \\
        \bottomrule
    \end{tabulary}
\end{table}

\begin{center}
\begin{tikzpicture}[gtu flow]
    \node [gtu start] (start) {START};
    \node [gtu process, below of=start] (debit) {Debit Account A};
    \node [gtu process, below of=debit] (credit) {Credit Account B};
    \node [gtu decision, below of=credit] (check) {Success?};
    \node [gtu stop, below left of=check, xshift=-1cm] (commit) {COMMIT};
    \node [gtu stop, below right of=check, xshift=1cm] (rollback) {ROLLBACK};
    
    \draw [gtu arrow] (start) -- (debit);
    \draw [gtu arrow] (debit) -- (credit);
    \draw [gtu arrow] (credit) -- (check);
    \draw [gtu arrow] (check) -| node[near start] {Yes} (commit);
    \draw [gtu arrow] (check) -| node[near start] {No} (rollback);
\end{tikzpicture}
\captionof{figure}{એટોમિસિટી ફ્લોચાર્ટ}
\end{center}

\begin{mnemonicbox}
    \textbf{મેમરી ટ્રીક:} "ACID: એટોમિસિટી, કન્સિસ્ટન્સી, આઇસોલેશન, ડ્યુરેબિલિટી"
\end{mnemonicbox}
\end{solutionbox}

\questionmarks{5(બ)}{4}{ઉપર Q.5 (b) માં આપેલ "Students" અને "CR" ટેબલનો ઉપયોગ કરીને સેટ ઓપરેટર દ્વારા નીચેની Query લખો.}
\begin{solutionbox}
\begin{lstlisting}[language=SQL]
-- ૧. Students અથવા CR હોય તેવા વ્યક્તિઓની યાદી બનાવો.
SELECT Stnd_Name FROM Student
UNION
SELECT CR_Name FROM CR;

-- ૨. Students અને CR હોય તેવા વ્યક્તિઓની યાદી બનાવો.
SELECT Stnd_Name FROM Student
INTERSECT
SELECT CR_Name FROM CR;

-- ૩. Students હોય અને CR ન હોય માત્ર તેવા વ્યક્તિઓની યાદી બનાવો.
SELECT Stnd_Name FROM Student
MINUS
SELECT CR_Name FROM CR;

-- ૪. CR હોય અને Student ન હોય માત્ર તેવા વ્યક્તિઓની યાદી બનાવો.
SELECT CR_Name FROM CR
MINUS
SELECT Stnd_Name FROM Student;
\end{lstlisting}

\begin{table}[H]
    \centering
    \caption{સેટ ઓપરેટર}
    \begin{tabulary}{\linewidth}{LCL}
        \toprule
        \textbf{સેટ ઓપરેટર} & \textbf{હેતુ} & \textbf{ઉદાહરણ માટે પરિણામ} \\
        \midrule
        \textbf{UNION} & બધી અલગ રો જોડે છે & Manoj, Rahil, Jiya, Rina, Jitesh, Priya \\
        \textbf{INTERSECT} & માત્ર સામાન્ય રો પરત કરે છે & Manoj, Rina \\
        \textbf{MINUS} & પ્રથમ સેટમાં હોય પણ બીજા સેટમાં ન હોય તે રો & Rahil, Jiya \\
        \textbf{MINUS (ઊલટું)} & બીજા સેટમાં હોય પણ પહેલા સેટમાં ન હોય તે રો & Jitesh, Priya \\
        \bottomrule
    \end{tabulary}
\end{table}

\begin{mnemonicbox}
    \textbf{મેમરી ટ્રીક:} "UIMD: યુનિયન ઇન્ક્લૂડ્સ, માઈનસ ડિવાઈડ્સ"
\end{mnemonicbox}
\end{solutionbox}

\questionmarks{5(ક)}{7}{Conflict serializability વિસ્તારથી સમજાવો.}
\begin{solutionbox}
\textbf{Conflict Serializability}: એક શેડ્યૂલ કન્ફ્લિક્ટ સીરિયલાઇઝેબલ છે જો તેને નોન-કન્ફ્લિક્ટિંગ ઓપરેશન્સને સ્વેપ કરીને સીરિયલ શેડ્યૂલમાં રૂપાંતરિત કરી શકાય.

\begin{table}[H]
    \centering
    \caption{Conflict Serializability વિચારો}
    \begin{tabulary}{\linewidth}{LCL}
        \toprule
        \textbf{મુખ્ય વિચારો} & \textbf{વર્ણન} \\
        \midrule
        \textbf{કન્ફ્લિક્ટ ઓપરેશન્સ} & બે ઓપરેશન કન્ફ્લિક્ટ કરે છે જો તેઓ એક જ ડેટા આઇટમને એક્સેસ કરે છે અને ઓછામાં ઓછું એક રાઇટ હોય \\
        \textbf{પ્રીસિડન્સ ગ્રાફ} & ટ્રાન્ઝેક્શન વચ્ચેના કન્ફ્લિક્ટને દર્શાવતો ડાયરેક્ટેડ ગ્રાફ \\
        \textbf{સીરિયલાઇઝેબલ} & જો પ્રીસિડન્સ ગ્રાફમાં કોઈ સાયકલ ન હોય, તો શેડ્યૂલ કન્ફ્લિક્ટ સીરિયલાઇઝેબલ છે \\
        \bottomrule
    \end{tabulary}
\end{table}

\begin{center}
\begin{tikzpicture}[gtu block, node distance=3cm]
    \node [gtu state] (t1) {T1};
    \node [gtu state, right of=t1] (t2) {T2};
    
    \draw [gtu arrow] (t1) -- node[above] {Conflict: W(X) $\to$ R(X)} (t2);
    \draw [gtu arrow, bend right] (t2) to node[below] {Conflict (if cyclic)} (t1);
    
    \node [below of=t1, yshift=1cm] {No Cycle $\implies$ Serializable};
\end{tikzpicture}
\captionof{figure}{પ્રીસિડન્સ ગ્રાફ કોન્સેપ્ટ}
\end{center}

\textbf{ઉદાહરણ}:
\begin{itemize}
    \item T1: R(X), W(X)
    \item T2: R(X), W(X)
    \item \textbf{સીરિયલાઇઝેબલ શેડ્યૂલ્સ}: T1 $\to$ T2 અથવા T2 $\to$ T1
    \item \textbf{નોન-સીરિયલાઇઝેબલ}: R1(X), R2(X), W1(X), W2(X) (સાયકલ બનાવે છે)
\end{itemize}

\begin{mnemonicbox}
    \textbf{મેમરી ટ્રીક:} "COPS: કન્ફ્લિક્ટ ઓપરેશન્સ પ્રોડ્યુસ સીરિયલાઇઝેબિલિટી"
\end{mnemonicbox}
\end{solutionbox}

\orquestionmarks{5(અ)}{3}{Transaction નો concept ઉદાહરણ સાથે સમજાવો.}
\begin{solutionbox}
\textbf{ટ્રાન્ઝેક્શન}: કામની એક તાર્કિક એકમ જે સંપૂર્ણપણે કરવું અથવા સંપૂર્ણપણે અનડૂ કરવું આવશ્યક છે.

\begin{table}[H]
    \centering
    \caption{ટ્રાન્ઝેક્શન તબક્કાઓ}
    \begin{tabulary}{\linewidth}{LCL}
        \toprule
        \textbf{તબક્કાઓ} & \textbf{વર્ણન} & \textbf{ઉદાહરણ} \\
        \midrule
        \textbf{BEGIN} & ટ્રાન્ઝેક્શનની શરૂઆત ચિહ્નિત કરે છે & START TRANSACTION \\
        \textbf{Execute} & ડેટાબેઝ ઓપરેશન્સ (રીડ/રાઇટ) & UPDATE account... \\
        \textbf{COMMIT/ROLLBACK} & સફળતા/નિષ્ફળતા સાથે ટ્રાન્ઝેક્શન સમાપ્ત કરે છે & COMMIT / ROLLBACK \\
        \bottomrule
    \end{tabulary}
\end{table}

\begin{center}
\begin{tikzpicture}[gtu flow]
    \node [gtu start] (start) {BEGIN};
    \node [gtu process, below of=start] (read) {Read Balance};
    \node [gtu decision, below of=read] (check) {Sufficient?};
    \node [gtu process, below of=check] (update) {Update Balance};
    \node [gtu process, below of=update] (rec) {Log Record};
    \node [gtu stop, below of=rec] (commit) {COMMIT};
    \node [gtu stop, right of=check, xshift=2cm] (rollback) {ROLLBACK};
    
    \draw [gtu arrow] (start) -- (read);
    \draw [gtu arrow] (read) -- (check);
    \draw [gtu arrow] (check) -- node[left] {Yes} (update);
    \draw [gtu arrow] (check) -- node[above] {No} (rollback);
    \draw [gtu arrow] (update) -- (rec);
    \draw [gtu arrow] (rec) -- (commit);
\end{tikzpicture}
\captionof{figure}{ટ્રાન્ઝેક્શન જીવન ચક્ર}
\end{center}

\begin{lstlisting}[language=SQL]
BEGIN TRANSACTION;
UPDATE accounts SET balance = balance - 1000 WHERE acc_no = 123;
UPDATE accounts SET balance = balance + 1000 WHERE acc_no = 456;
COMMIT;
\end{lstlisting}

\begin{mnemonicbox}
    \textbf{મેમરી ટ્રીક:} "BEC: બિગિન, એક્ઝિક્યુટ, કમિટ"
\end{mnemonicbox}
\end{solutionbox}

\orquestionmarks{5(બ)}{4}{Equi-join સિન્ટેક્સ અને ઉદાહરણ સાથે સમજાવો.}
\begin{solutionbox}
\textbf{Equi-join}: એક જોઈન ઓપરેશન જે સમાનતા કમ્પેરિઝન ઓપરેટરનો ઉપયોગ કરે છે.

\begin{table}[H]
    \centering
    \caption{Equi-join}
    \begin{tabulary}{\linewidth}{LCL}
        \toprule
        \textbf{ફીચર} & \textbf{વર્ણન} \\
        \midrule
        \textbf{સિન્ટેક્સ} & \code{SELECT * FROM t1, t2 WHERE t1.c = t2.c;} \\
        \textbf{કન્ડિશન} & \code{=} ઓપરેટર વાપરે છે \\
        \textbf{કોલમ્સ} & બંને ટેબલમાંથી કોલમનો સમાવેશ કરે છે \\
        \bottomrule
    \end{tabulary}
\end{table}

\begin{lstlisting}[language=SQL]
SELECT name, course_name 
FROM students s, courses c 
WHERE s.course_id = c.course_id;
\end{lstlisting}

\begin{mnemonicbox}
    \textbf{મેમરી ટ્રીક:} "EJE: ઇક્વિ જોઈન ઇક્વેશન (=)"
\end{mnemonicbox}
\end{solutionbox}

\orquestionmarks{5(ક)}{7}{View serializability વિસ્તારથી સમજાવો.}
\begin{solutionbox}
\textbf{View Serializability}: એક શેડ્યૂલ વ્યૂ સીરિયલાઇઝેબલ છે જો તે કોઈ સીરિયલ શેડ્યૂલ સાથે વ્યૂ ઇક્વિવેલન્ટ હોય.

\begin{table}[H]
    \centering
    \caption{View Equivalency શરતો}
    \begin{tabulary}{\linewidth}{LCL}
        \toprule
        \textbf{શરત} & \textbf{વર્ણન} \\
        \midrule
        \textbf{ઇનિશિયલ રીડ} & જો T1 શેડ્યૂલ S માં ડેટા આઇટમ X ની પ્રારંભિક વેલ્યુ વાંચે છે, તો તેણે S' શેડ્યૂલમાં પણ પ્રારંભિક વેલ્યુ વાંચવી જોઈએ \\
        \textbf{ફાઇનલ રાઇટ} & જો T1, S માં ડેટા આઇટમ X નું અંતિમ લખાણ કરે છે, તો તેણે S' માં પણ અંતિમ લખાણ કરવું જોઈએ \\
        \textbf{ડિપેન્ડન્સી પ્રિઝર્વેશન} & જો T1, S માં T2 દ્વારા લખાયેલ X ની વેલ્યુ વાંચે છે, તો તેણે S' માં પણ T2 પાસેથી વાંચવું જોઈએ \\
        \bottomrule
    \end{tabulary}
\end{table}

\begin{center}
\begin{tikzpicture}[gtu flow]
    \node [gtu start] (start) {Schedule S};
    \node [gtu decision, below of=start] (check) {View Equivalent?};
    \node [gtu stop, below left of=check, xshift=-1cm] (yes) {View Serializable};
    \node [gtu stop, below right of=check, xshift=1cm] (no) {Not View Serializable};
    
    \draw [gtu arrow] (start) -- (check);
    \draw [gtu arrow] (check) -| node[near start] {Yes} (yes);
    \draw [gtu arrow] (check) -| node[near start] {No} (no);
    
    \node [gtu block, right of=start, xshift=4cm] (analysis) {Read-Write Analysis};
    \node [below of=analysis, yshift=0.5cm, align=left] {1. Initial Read Check\\2. Final Write Check\\3. Read-from-Write Check};
    
    \draw [gtu line, dashed] (start) -- (analysis);
\end{tikzpicture}
\captionof{figure}{View Serializability ચેક}
\end{center}

\textbf{તુલના}:
\begin{itemize}
    \item \textbf{કન્ફ્લિક્ટ સીરિયલાઇઝેબિલિટી}: વધુ પ્રતિબંધિત, પરીક્ષણ કરવું સરળ (પ્રીસિડન્સ ગ્રાફ)
    \item \textbf{વ્યૂ સીરિયલાઇઝેબિલિટી}: વધુ સામાન્ય, પરીક્ષણ કરવું વધુ મુશ્કેલ (NP-કમ્પ્લીટ)
\end{itemize}

\textbf{ઉદાહરણ}:
\begin{itemize}
    \item T1: W(X), T2: W(X), T3: R(X)
    \item શેડ્યૂલ: W1(X), W2(X), R3(X) - સીરિયલ શેડ્યૂલ T2,T1,T3 સાથે વ્યૂ ઇક્વિવેલન્ટ
\end{itemize}

\begin{mnemonicbox}
    \textbf{મેમરી ટ્રીક:} "VIR-FF: વ્યૂ પ્રિઝર્વ્સ ઇનિશિયલ રીડ્સ એન્ડ ફાઇનલ રાઇટ્સ"
\end{mnemonicbox}
\end{solutionbox}

\end{document}
