\documentclass[10pt,a4paper]{article}

% content/resources/templates/preamble.tex
\usepackage[margin=0.6in]{geometry}
\author{Milav Dabgar}
\usepackage{amsmath,amssymb,amsthm}
\usepackage{booktabs}
\usepackage{multirow}
\usepackage{xcolor}
\usepackage{tcolorbox}
\tcbuselibrary{breakable,skins}
\usepackage[colorlinks=true,linkcolor=blue]{hyperref}
\usepackage{titlesec}
\usepackage{enumitem}
\usepackage{tikz}
\usepackage{pgfplots}
\usepackage{circuitikz}
\usepackage[version=4]{mhchem}
\usepackage{longtable}
\usepackage{array}
\usepackage{float}
\usepackage{caption}
\usepackage{listings}

\lstset{
  basicstyle=\small\ttfamily,
  breaklines=true,
  breakatwhitespace=false,
  postbreak=\mbox{\textcolor{red}{$\hookrightarrow$}\space},
  float=false,
  numbers=left,
  numberstyle=\tiny\color{gray},
  numbersep=10pt,
  xleftmargin=2em,
  keywordstyle=\color{blue},
  commentstyle=\color{green!60!black},
  stringstyle=\color{purple},
  backgroundcolor=\color{gray!5},
  showstringspaces=false,
  tabsize=2,
  captionpos=b,
  keepspaces=true,
  columns=flexible
}

\pgfplotsset{compat=1.18}
\usetikzlibrary{shapes,arrows,positioning,calc,patterns,decorations.pathmorphing,decorations.markings,arrows.meta}

% Color scheme
\definecolor{headcolor}{RGB}{0,102,204}
\definecolor{keycolor}{RGB}{220,20,60}
\definecolor{solutioncolor}{RGB}{34,139,34}
\definecolor{mnemoniccolor}{RGB}{148,0,211}
\definecolor{codecolor}{RGB}{0,0,100}

% Spacing
\setlength{\parskip}{3pt}
\setlist[itemize]{nosep}
\setlist[enumerate]{nosep}

% Title formatting
\titleformat{\section}{\Large\bfseries\color{headcolor}}{\thesection}{1em}{}
\titleformat{\subsection}{\large\bfseries\color{headcolor}}{\thesubsection}{1em}{}

% Pandoc tightlist compatibility
\providecommand{\tightlist}{%
  \setlength{\itemsep}{0pt}\setlength{\parskip}{0pt}}

% Pandoc longtable compatibility
\newcounter{none}
\def\thenone{}


% content/resources/templates/english-boxes.tex
% This file is currently empty - it exists to maintain consistency with the import structure.
% Add custom environments here if needed in the future.


\begin{document}

\begin{center}
{\Huge\bfseries\color{headcolor} Subject Name Solutions}\\[5pt]
{\LARGE 1333204 -- Summer 2025}\\[3pt]
{\large Semester 1 Study Material}\\[3pt]
{\normalsize\textit{Detailed Solutions and Explanations}}
\end{center}

\vspace{10pt}

\subsection*{Question 1(a) [3 marks]}\label{q1a}

\textbf{Write a short note: Data Dictionary}

\begin{solutionbox}
A \textbf{Data Dictionary} is a centralized repository
that stores metadata about database structure, elements, and
relationships.


{\def\LTcaptype{none} % do not increment counter
\vspace{-5pt}
\captionof{table}{Data Dictionary Components}
\vspace{-10pt}
\begin{longtable}[]{@{}ll@{}}
\toprule\noalign{}
Component & Description \\
\midrule\noalign{}
\endhead
\bottomrule\noalign{}
\endlastfoot
\textbf{Table Names} & List of all tables in database \\
\textbf{Column Details} & Data types, constraints, lengths \\
\textbf{Relationships} & Foreign key connections \\
\textbf{Indexes} & Performance optimization structures \\
\end{longtable}
}

\textbf{Key Features:}

\begin{itemize}
\tightlist
\item
  \textbf{Metadata Storage}: Contains information about data structure
\item
  \textbf{Data Integrity}: Maintains consistency rules and constraints
\item
  \textbf{Documentation}: Provides comprehensive database documentation
\end{itemize}

\end{solutionbox}
\begin{mnemonicbox}
``Data Dictionary Delivers Details''

\end{mnemonicbox}
\subsection*{Question 1(b) [4 marks]}\label{q1b}

\textbf{Define (i) E-R model (ii) Entity (iii) Entity set and (iv)
attributes}

\begin{solutionbox}


{\def\LTcaptype{none} % do not increment counter
\vspace{-5pt}
\captionof{table}{ER Model Definitions}
\vspace{-10pt}
\begin{longtable}[]{@{}
  >{\raggedright\arraybackslash}p{(\linewidth - 2\tabcolsep) * \real{0.3333}}
  >{\raggedright\arraybackslash}p{(\linewidth - 2\tabcolsep) * \real{0.6667}}@{}}
\toprule\noalign{}
\begin{minipage}[b]{\linewidth}\raggedright
Term
\end{minipage} & \begin{minipage}[b]{\linewidth}\raggedright
Definition
\end{minipage} \\
\midrule\noalign{}
\endhead
\bottomrule\noalign{}
\endlastfoot
\textbf{E-R Model} & Conceptual data model using entities and
relationships \\
\textbf{Entity} & Real-world object with independent existence \\
\textbf{Entity Set} & Collection of similar entities of same type \\
\textbf{Attributes} & Properties that describe entity characteristics \\
\end{longtable}
}

\textbf{Diagram: ER Model Components}

\begin{lstlisting}
    +----------+     +-------------+     +----------+
    |  Entity  |-----| Relationship|-----|  Entity  |
    |    A     |     |             |     |    B     |
    +----------+     +-------------+     +----------+
         |                                    |
    Attributes                           Attributes
\end{lstlisting}

\textbf{Key Points:}

\begin{itemize}
\tightlist
\item
  \textbf{Conceptual Design}: High-level database design approach
\item
  \textbf{Visual Representation}: Uses diagrams for clear understanding
\end{itemize}

\end{solutionbox}
\begin{mnemonicbox}
``Entities Relate Meaningfully''

\end{mnemonicbox}
\subsection*{Question 1(c) [7 marks]}\label{q1c}

\textbf{Explain Advantages of DBMS}

\begin{solutionbox}


{\def\LTcaptype{none} % do not increment counter
\vspace{-5pt}
\captionof{table}{DBMS Advantages}
\vspace{-10pt}
\begin{longtable}[]{@{}
  >{\raggedright\arraybackslash}p{(\linewidth - 2\tabcolsep) * \real{0.5500}}
  >{\raggedright\arraybackslash}p{(\linewidth - 2\tabcolsep) * \real{0.4500}}@{}}
\toprule\noalign{}
\begin{minipage}[b]{\linewidth}\raggedright
Advantage
\end{minipage} & \begin{minipage}[b]{\linewidth}\raggedright
Benefit
\end{minipage} \\
\midrule\noalign{}
\endhead
\bottomrule\noalign{}
\endlastfoot
\textbf{Data Independence} & Applications isolated from data structure
changes \\
\textbf{Data Sharing} & Multiple users access same data
simultaneously \\
\textbf{Data Security} & Access control and authentication mechanisms \\
\textbf{Data Integrity} & Consistency maintained through constraints \\
\textbf{Backup \& Recovery} & Automatic data protection and
restoration \\
\textbf{Reduced Redundancy} & Eliminates duplicate data storage \\
\end{longtable}
}

\textbf{Key Benefits:}

\begin{itemize}
\tightlist
\item
  \textbf{Centralized Control}: Single point of data management
\item
  \textbf{Cost Effectiveness}: Reduces development and maintenance costs
\item
  \textbf{Data Consistency}: Ensures uniform data across applications
\item
  \textbf{Concurrent Access}: Multiple users can work simultaneously
\item
  \textbf{Query Optimization}: Efficient data retrieval mechanisms
\end{itemize}

\end{solutionbox}
\begin{mnemonicbox}
``Database Benefits Business Better''

\end{mnemonicbox}
\subsection*{Question 1(c) OR [7
marks]}\label{q1c}

\textbf{Explain Architecture of DBMS}

\begin{solutionbox}

\textbf{Diagram: Three-Level DBMS Architecture}

\includegraphics[width=1\linewidth,height=\textheight,keepaspectratio]{mermaid-bf2ca2c1.pdf}


{\def\LTcaptype{none} % do not increment counter
\vspace{-5pt}
\captionof{table}{Architecture Levels}
\vspace{-10pt}
\begin{longtable}[]{@{}lll@{}}
\toprule\noalign{}
Level & Purpose & Users \\
\midrule\noalign{}
\endhead
\bottomrule\noalign{}
\endlastfoot
\textbf{External} & Individual user views & End users, Applications \\
\textbf{Conceptual} & Complete logical structure & Database
Administrator \\
\textbf{Internal} & Physical storage details & System programmers \\
\end{longtable}
}

\textbf{Key Features:}

\begin{itemize}
\tightlist
\item
  \textbf{Data Independence}: Changes at one level don't affect others
\item
  \textbf{Security}: Different access levels for different users
\item
  \textbf{Abstraction}: Hides complexity from users
\end{itemize}

\end{solutionbox}
\begin{mnemonicbox}
``External Conceptual Internal Architecture''

\end{mnemonicbox}
\subsection*{Question 2(a) [3 marks]}\label{q2a}

\textbf{Explain UNIQUE KEY and PRIMARY KEY}

\begin{solutionbox}


{\def\LTcaptype{none} % do not increment counter
\vspace{-5pt}
\captionof{table}{Key Comparison}
\vspace{-10pt}
\begin{longtable}[]{@{}lll@{}}
\toprule\noalign{}
Feature & PRIMARY KEY & UNIQUE KEY \\
\midrule\noalign{}
\endhead
\bottomrule\noalign{}
\endlastfoot
\textbf{Null Values} & Not allowed & One null allowed \\
\textbf{Number per Table} & Only one & Multiple allowed \\
\textbf{Index Creation} & Automatic clustered & Automatic
non-clustered \\
\textbf{Purpose} & Entity identification & Data uniqueness \\
\end{longtable}
}

\textbf{Key Differences:}

\begin{itemize}
\tightlist
\item
  \textbf{Primary Key}: Uniquely identifies each record, cannot be null
\item
  \textbf{Unique Key}: Ensures uniqueness but allows one null value
\end{itemize}

\end{solutionbox}
\begin{mnemonicbox}
``Primary Prevents Nulls, Unique Understands Nulls''

\end{mnemonicbox}
\subsection*{Question 2(b) [4 marks]}\label{q2b}

\textbf{Write a short note on Participation of Entity in ER diagram}

\begin{solutionbox}


{\def\LTcaptype{none} % do not increment counter
\vspace{-5pt}
\captionof{table}{Participation Types}
\vspace{-10pt}
\begin{longtable}[]{@{}
  >{\raggedright\arraybackslash}p{(\linewidth - 4\tabcolsep) * \real{0.2222}}
  >{\raggedright\arraybackslash}p{(\linewidth - 4\tabcolsep) * \real{0.4815}}
  >{\raggedright\arraybackslash}p{(\linewidth - 4\tabcolsep) * \real{0.2963}}@{}}
\toprule\noalign{}
\begin{minipage}[b]{\linewidth}\raggedright
Type
\end{minipage} & \begin{minipage}[b]{\linewidth}\raggedright
Description
\end{minipage} & \begin{minipage}[b]{\linewidth}\raggedright
Symbol
\end{minipage} \\
\midrule\noalign{}
\endhead
\bottomrule\noalign{}
\endlastfoot
\textbf{Total Participation} & Every entity must participate & Double
line \\
\textbf{Partial Participation} & Some entities may not participate &
Single line \\
\end{longtable}
}

\textbf{Diagram: Participation Example}

\begin{lstlisting}
Employee ========== Works_for ---------- Department
  (Total)                                 (Partial)
\end{lstlisting}

\textbf{Key Concepts:}

\begin{itemize}
\tightlist
\item
  \textbf{Mandatory Participation}: Every instance must be involved
\item
  \textbf{Optional Participation}: Some instances may not be involved
\item
  \textbf{Business Rules}: Reflects real-world constraints
\end{itemize}

\end{solutionbox}
\begin{mnemonicbox}
``Total Participation Requires All''

\end{mnemonicbox}
\subsection*{Question 2(c) [7 marks]}\label{q2c}

\textbf{Describe Generalization concept in Detail for ER diagram}

\begin{solutionbox}

\textbf{Diagram: Generalization Example}

\includegraphics[width=1\linewidth,height=\textheight,keepaspectratio]{mermaid-558f6c13.pdf}


{\def\LTcaptype{none} % do not increment counter
\vspace{-5pt}
\captionof{table}{Generalization Characteristics}
\vspace{-10pt}
\begin{longtable}[]{@{}ll@{}}
\toprule\noalign{}
Aspect & Description \\
\midrule\noalign{}
\endhead
\bottomrule\noalign{}
\endlastfoot
\textbf{Bottom-up Process} & Combines similar entities into
superclass \\
\textbf{Inheritance} & Subclasses inherit superclass attributes \\
\textbf{Specialization} & Reverse process of generalization \\
\textbf{Overlap Constraints} & Disjoint or overlapping subclasses \\
\end{longtable}
}

\textbf{Key Features:}

\begin{itemize}
\tightlist
\item
  \textbf{Attribute Inheritance}: Common attributes moved to superclass
\item
  \textbf{Relationship Inheritance}: Relationships also inherited
\item
  \textbf{Constraint Types}: Total/partial, disjoint/overlapping
\item
  \textbf{ISA Relationship}: Represents ``is-a'' connection
\end{itemize}

\end{solutionbox}
\begin{mnemonicbox}
``Generalization Groups Similar Entities''

\end{mnemonicbox}
\subsection*{Question 2(a) OR [3
marks]}\label{q2a}

\textbf{Explain Mapping Cardinality in ER diagram}

\begin{solutionbox}


{\def\LTcaptype{none} % do not increment counter
\vspace{-5pt}
\captionof{table}{Cardinality Types}
\vspace{-10pt}
\begin{longtable}[]{@{}
  >{\raggedright\arraybackslash}p{(\linewidth - 4\tabcolsep) * \real{0.2143}}
  >{\raggedright\arraybackslash}p{(\linewidth - 4\tabcolsep) * \real{0.4643}}
  >{\raggedright\arraybackslash}p{(\linewidth - 4\tabcolsep) * \real{0.3214}}@{}}
\toprule\noalign{}
\begin{minipage}[b]{\linewidth}\raggedright
Type
\end{minipage} & \begin{minipage}[b]{\linewidth}\raggedright
Description
\end{minipage} & \begin{minipage}[b]{\linewidth}\raggedright
Example
\end{minipage} \\
\midrule\noalign{}
\endhead
\bottomrule\noalign{}
\endlastfoot
\textbf{One-to-One (1:1)} & One entity relates to one other &
Person-Passport \\
\textbf{One-to-Many (1:M)} & One entity relates to many others &
Department-Employee \\
\textbf{Many-to-One (M:1)} & Many entities relate to one &
Employee-Department \\
\textbf{Many-to-Many (M:N)} & Many entities relate to many &
Student-Course \\
\end{longtable}
}

\textbf{Key Concepts:}

\begin{itemize}
\tightlist
\item
  \textbf{Relationship Constraints}: Defines how entities can be related
\item
  \textbf{Business Rules}: Reflects real-world relationship limits
\end{itemize}

\end{solutionbox}
\begin{mnemonicbox}
``One Or Many Mappings Matter''

\end{mnemonicbox}
\subsection*{Question 2(b) OR [4
marks]}\label{q2b}

\textbf{Explain Aggregation in E-R diagram}

\begin{solutionbox}

\textbf{Diagram: Aggregation Example}

\begin{lstlisting}
    Employee ---- Works_on ---- Project
        |                         |
        +----------+----------+
                   |
               Manages
                   |
                Manager
\end{lstlisting}

\textbf{Key Features:}

\begin{itemize}
\tightlist
\item
  \textbf{Relationship as Entity}: Treats relationship set as entity
\item
  \textbf{Higher-level Relationships}: Allows relationships between
  relationships
\item
  \textbf{Complex Modeling}: Handles advanced business scenarios
\item
  \textbf{Abstraction Mechanism}: Simplifies complex relationships
\end{itemize}


{\def\LTcaptype{none} % do not increment counter
\vspace{-5pt}
\captionof{table}{Aggregation Benefits}
\vspace{-10pt}
\begin{longtable}[]{@{}ll@{}}
\toprule\noalign{}
Benefit & Description \\
\midrule\noalign{}
\endhead
\bottomrule\noalign{}
\endlastfoot
\textbf{Modeling Flexibility} & Handles complex relationships \\
\textbf{Semantic Clarity} & Clear representation of business rules \\
\textbf{Design Simplicity} & Reduces model complexity \\
\end{longtable}
}

\end{solutionbox}
\begin{mnemonicbox}
``Aggregation Abstracts Advanced Associations''

\end{mnemonicbox}
\subsection*{Question 2(c) OR [7
marks]}\label{q2c}

\textbf{Draw ER diagram of Library Management system using Enhanced ER
model}

\begin{solutionbox}

\textbf{Diagram: Library Management System}

\includegraphics[width=1\linewidth,height=\textheight,keepaspectratio]{mermaid-97fbd7d5.pdf}

\textbf{Enhanced ER Features Used:}

\begin{itemize}
\tightlist
\item
  \textbf{Generalization}: Person superclass with Member and Librarian
  subclasses
\item
  \textbf{Specialization}: Different attributes for different person
  types
\item
  \textbf{Aggregation}: Transaction relationship involving multiple
  entities
\item
  \textbf{Multiple Inheritance}: Complex relationship handling
\end{itemize}

\end{solutionbox}
\begin{mnemonicbox}
``Library Links Literature Logically''

\end{mnemonicbox}
\subsection*{Question 3(a) [3 marks]}\label{q3a}

\textbf{Explain SQL data types}

\begin{solutionbox}


{\def\LTcaptype{none} % do not increment counter
\vspace{-5pt}
\captionof{table}{Common SQL Data Types}
\vspace{-10pt}
\begin{longtable}[]{@{}lll@{}}
\toprule\noalign{}
Category & Data Type & Description \\
\midrule\noalign{}
\endhead
\bottomrule\noalign{}
\endlastfoot
\textbf{Numeric} & INT, DECIMAL, FLOAT & Store numbers \\
\textbf{Character} & CHAR, VARCHAR, TEXT & Store text \\
\textbf{Date/Time} & DATE, TIME, DATETIME & Store temporal data \\
\textbf{Boolean} & BOOLEAN & Store true/false \\
\end{longtable}
}

\textbf{Key Points:}

\begin{itemize}
\tightlist
\item
  \textbf{Data Integrity}: Ensures correct data storage
\item
  \textbf{Storage Optimization}: Appropriate size allocation
\item
  \textbf{Validation}: Automatic data type checking
\end{itemize}

\end{solutionbox}
\begin{mnemonicbox}
``Data Types Define Storage''

\end{mnemonicbox}
\subsection*{Question 3(b) [4 marks]}\label{q3b}

\textbf{Compare DROP and TRUNCATE commands}

\begin{solutionbox}


{\def\LTcaptype{none} % do not increment counter
\vspace{-5pt}
\captionof{table}{DROP vs TRUNCATE Comparison}
\vspace{-10pt}
\begin{longtable}[]{@{}lll@{}}
\toprule\noalign{}
Feature & DROP & TRUNCATE \\
\midrule\noalign{}
\endhead
\bottomrule\noalign{}
\endlastfoot
\textbf{Operation} & Removes table structure & Removes all data only \\
\textbf{Rollback} & Cannot rollback & Can rollback (in transaction) \\
\textbf{Speed} & Slower & Faster \\
\textbf{Triggers} & Fires triggers & Does not fire triggers \\
\textbf{Where Clause} & Not applicable & Not supported \\
\textbf{Auto-increment} & Resets & Resets to initial value \\
\end{longtable}
}

\textbf{Code Examples:}

\begin{lstlisting}[language=SQL]
-- DROP command
DROP TABLE student;

-- TRUNCATE command  
TRUNCATE TABLE student;
\end{lstlisting}

\textbf{Key Differences:}

\begin{itemize}
\tightlist
\item
  \textbf{Structure Impact}: DROP removes everything, TRUNCATE keeps
  structure
\item
  \textbf{Performance}: TRUNCATE is faster for large tables
\end{itemize}

\end{solutionbox}
\begin{mnemonicbox}
``DROP Destroys, TRUNCATE Trims''

\end{mnemonicbox}
\subsection*{Question 3(c) [7 marks]}\label{q3c}

\textbf{Consider a following Relational Schema and give Relational
Algebra Expression for the following Queries} \textbf{Students (Name,
SPI, DOB, Enrollment No)}

\begin{solutionbox}

\textbf{Relational Algebra Expressions:}

\textbf{i) List out all students whose SPI is lower than 6.0:}

\begin{lstlisting}
σ(SPI < 6.0)(Students)
\end{lstlisting}

\textbf{ii) List name of student whose enrollment number contains 006:}

\begin{lstlisting}
π(Name)(σ(Enrollment_No LIKE '%006%')(Students))
\end{lstlisting}

\textbf{iii) List all students with same DOB:}

\begin{lstlisting}
Students ⋈ (ρ(S2)(Students)) WHERE Students.DOB = S2.DOB AND Students.Enrollment_No \neq S2.Enrollment_No
\end{lstlisting}

\textbf{iv) Display students name starting from same letter:}

\begin{lstlisting}
π(Name)(Students ⋈ (ρ(S2)(Students)) WHERE SUBSTR(Students.Name,1,1) = SUBSTR(S2.Name,1,1) AND Students.Enrollment_No \neq S2.Enrollment_No)
\end{lstlisting}


{\def\LTcaptype{none} % do not increment counter
\vspace{-5pt}
\captionof{table}{Relational Algebra Operators Used}
\vspace{-10pt}
\begin{longtable}[]{@{}lll@{}}
\toprule\noalign{}
Operator & Symbol & Purpose \\
\midrule\noalign{}
\endhead
\bottomrule\noalign{}
\endlastfoot
\textbf{Selection} & σ & Filter rows based on condition \\
\textbf{Projection} & π & Select specific columns \\
\textbf{Join} & ⋈ & Combine related tuples \\
\textbf{Rename} & ρ & Rename relations/attributes \\
\end{longtable}
}

\end{solutionbox}
\begin{mnemonicbox}
``Select Project Join Rename''

\end{mnemonicbox}
\subsection*{Question 3(a) OR [3
marks]}\label{q3a}

\textbf{Explain use of Grant and Revoke command with example}

\begin{solutionbox}

\textbf{Code Examples:}

\begin{lstlisting}[language=SQL]
-- GRANT command
GRANT SELECT, INSERT ON student TO user1;
GRANT ALL PRIVILEGES ON database1 TO user2;

-- REVOKE command  
REVOKE INSERT ON student FROM user1;
REVOKE ALL PRIVILEGES ON database1 FROM user2;
\end{lstlisting}

\textbf{Key Features:}

\begin{itemize}
\tightlist
\item
  \textbf{Access Control}: Manages user permissions
\item
  \textbf{Security}: Prevents unauthorized access
\item
  \textbf{Granular Control}: Specific privilege assignment
\end{itemize}


{\def\LTcaptype{none} % do not increment counter
\vspace{-5pt}
\captionof{table}{Common Privileges}
\vspace{-10pt}
\begin{longtable}[]{@{}ll@{}}
\toprule\noalign{}
Privilege & Description \\
\midrule\noalign{}
\endhead
\bottomrule\noalign{}
\endlastfoot
\textbf{SELECT} & Read data \\
\textbf{INSERT} & Add new records \\
\textbf{UPDATE} & Modify existing data \\
\textbf{DELETE} & Remove records \\
\textbf{ALL} & Complete access \\
\end{longtable}
}

\end{solutionbox}
\begin{mnemonicbox}
``Grant Gives, Revoke Removes''

\end{mnemonicbox}
\subsection*{Question 3(b) OR [4
marks]}\label{q3b}

\textbf{Describe DML commands with Example}

\begin{solutionbox}


{\def\LTcaptype{none} % do not increment counter
\vspace{-5pt}
\captionof{table}{DML Commands}
\vspace{-10pt}
\begin{longtable}[]{@{}
  >{\raggedright\arraybackslash}p{(\linewidth - 4\tabcolsep) * \real{0.3333}}
  >{\raggedright\arraybackslash}p{(\linewidth - 4\tabcolsep) * \real{0.3333}}
  >{\raggedright\arraybackslash}p{(\linewidth - 4\tabcolsep) * \real{0.3333}}@{}}
\toprule\noalign{}
\begin{minipage}[b]{\linewidth}\raggedright
Command
\end{minipage} & \begin{minipage}[b]{\linewidth}\raggedright
Purpose
\end{minipage} & \begin{minipage}[b]{\linewidth}\raggedright
Example
\end{minipage} \\
\midrule\noalign{}
\endhead
\bottomrule\noalign{}
\endlastfoot
\textbf{INSERT} & Add new records &
\passthrough{\lstinline!INSERT INTO student VALUES (1,'John',8.5)!} \\
\textbf{UPDATE} & Modify existing data &
\passthrough{\lstinline!UPDATE student SET spi=9.0 WHERE id=1!} \\
\textbf{DELETE} & Remove records &
\passthrough{\lstinline!DELETE FROM student WHERE spi<6.0!} \\
\textbf{SELECT} & Retrieve data &
\passthrough{\lstinline!SELECT * FROM student WHERE spi>8.0!} \\
\end{longtable}
}

\textbf{Code Examples:}

\begin{lstlisting}[language=SQL]
-- INSERT command
INSERT INTO Students (name, spi, dob) 
VALUES ('Alice', 8.5, '2000-05-15');

-- UPDATE command
UPDATE Students SET spi = 9.0 
WHERE name = 'Alice';

-- DELETE command
DELETE FROM Students 
WHERE spi < 6.0;

-- SELECT command
SELECT name, spi FROM Students 
WHERE spi > 8.0;
\end{lstlisting}

\textbf{Key Features:}

\begin{itemize}
\tightlist
\item
  \textbf{Data Manipulation}: Core database operations
\item
  \textbf{Transaction Support}: Can be rolled back
\item
  \textbf{Conditional Operations}: WHERE clause support
\end{itemize}

\end{solutionbox}
\begin{mnemonicbox}
``Insert Update Delete Select''

\end{mnemonicbox}
\subsection*{Question 3(c) OR [7
marks]}\label{q3c}

\textbf{List all Conversion function of DBMS and explain any three of
them in detail}

\begin{solutionbox}


{\def\LTcaptype{none} % do not increment counter
\vspace{-5pt}
\captionof{table}{Conversion Functions}
\vspace{-10pt}
\begin{longtable}[]{@{}
  >{\raggedright\arraybackslash}p{(\linewidth - 4\tabcolsep) * \real{0.3571}}
  >{\raggedright\arraybackslash}p{(\linewidth - 4\tabcolsep) * \real{0.3214}}
  >{\raggedright\arraybackslash}p{(\linewidth - 4\tabcolsep) * \real{0.3214}}@{}}
\toprule\noalign{}
\begin{minipage}[b]{\linewidth}\raggedright
Function
\end{minipage} & \begin{minipage}[b]{\linewidth}\raggedright
Purpose
\end{minipage} & \begin{minipage}[b]{\linewidth}\raggedright
Example
\end{minipage} \\
\midrule\noalign{}
\endhead
\bottomrule\noalign{}
\endlastfoot
\textbf{TO\_CHAR} & Convert to character &
\passthrough{\lstinline!TO\_CHAR(sysdate, 'DD-MM-YYYY')!} \\
\textbf{TO\_DATE} & Convert to date &
\passthrough{\lstinline!TO\_DATE('15-05-2025', 'DD-MM-YYYY')!} \\
\textbf{TO\_NUMBER} & Convert to number &
\passthrough{\lstinline!TO\_NUMBER('123.45')!} \\
\textbf{CAST} & General conversion &
\passthrough{\lstinline!CAST('123' AS INTEGER)!} \\
\textbf{CONVERT} & Data type conversion &
\passthrough{\lstinline!CONVERT(varchar, 123)!} \\
\end{longtable}
}

\textbf{Detailed Explanation of Three Functions:}

\textbf{1. TO\_CHAR Function:}

\begin{itemize}
\tightlist
\item
  \textbf{Purpose}: Converts dates and numbers to character strings
\item
  \textbf{Syntax}: \passthrough{\lstinline!TO\_CHAR(value, format)!}
\item
  \textbf{Usage}: Date formatting, number formatting with specific
  patterns
\end{itemize}

\textbf{2. TO\_DATE Function:}

\begin{itemize}
\tightlist
\item
  \textbf{Purpose}: Converts character strings to date values
\item
  \textbf{Syntax}: \passthrough{\lstinline!TO\_DATE(string, format)!}\\
\item
  \textbf{Usage}: String to date conversion with specified format
\end{itemize}

\textbf{3. TO\_NUMBER Function:}

\begin{itemize}
\tightlist
\item
  \textbf{Purpose}: Converts character strings to numeric values
\item
  \textbf{Syntax}: \passthrough{\lstinline!TO\_NUMBER(string, format)!}
\item
  \textbf{Usage}: String to number conversion for calculations
\end{itemize}

\textbf{Key Benefits:}

\begin{itemize}
\tightlist
\item
  \textbf{Data Type Flexibility}: Seamless conversion between types
\item
  \textbf{Format Control}: Specific formatting options
\item
  \textbf{Error Handling}: Validation during conversion
\end{itemize}

\end{solutionbox}
\begin{mnemonicbox}
``Convert Characters Dates Numbers''

\end{mnemonicbox}
\subsection*{Question 4(a) [3 marks]}\label{q4a}

\textbf{Write short note: Domain Integrity Constraint}

\begin{solutionbox}

\textbf{Domain Integrity Constraints} ensure that data values fall
within acceptable ranges and formats for specific attributes.


{\def\LTcaptype{none} % do not increment counter
\vspace{-5pt}
\captionof{table}{Domain Constraint Types}
\vspace{-10pt}
\begin{longtable}[]{@{}
  >{\raggedright\arraybackslash}p{(\linewidth - 4\tabcolsep) * \real{0.4000}}
  >{\raggedright\arraybackslash}p{(\linewidth - 4\tabcolsep) * \real{0.3000}}
  >{\raggedright\arraybackslash}p{(\linewidth - 4\tabcolsep) * \real{0.3000}}@{}}
\toprule\noalign{}
\begin{minipage}[b]{\linewidth}\raggedright
Constraint
\end{minipage} & \begin{minipage}[b]{\linewidth}\raggedright
Purpose
\end{minipage} & \begin{minipage}[b]{\linewidth}\raggedright
Example
\end{minipage} \\
\midrule\noalign{}
\endhead
\bottomrule\noalign{}
\endlastfoot
\textbf{CHECK} & Value range validation &
\passthrough{\lstinline!CHECK (age >= 0 AND age <= 100)!} \\
\textbf{NOT NULL} & Prevents null values &
\passthrough{\lstinline!name VARCHAR(50) NOT NULL!} \\
\textbf{DEFAULT} & Sets default values &
\passthrough{\lstinline!status VARCHAR(10) DEFAULT 'Active'!} \\
\end{longtable}
}

\textbf{Key Features:}

\begin{itemize}
\tightlist
\item
  \textbf{Data Validation}: Ensures data quality at entry
\item
  \textbf{Business Rules}: Implements domain-specific rules
\item
  \textbf{Automatic Checking}: Validation occurs during DML operations
\end{itemize}

\end{solutionbox}
\begin{mnemonicbox}
``Domain Defines Data Boundaries''

\end{mnemonicbox}
\subsection*{Question 4(b) [4 marks]}\label{q4b}

\textbf{List all JOIN in DBMS and explain any two}

\begin{solutionbox}


{\def\LTcaptype{none} % do not increment counter
\vspace{-5pt}
\captionof{table}{Types of JOINs}
\vspace{-10pt}
\begin{longtable}[]{@{}ll@{}}
\toprule\noalign{}
JOIN Type & Description \\
\midrule\noalign{}
\endhead
\bottomrule\noalign{}
\endlastfoot
\textbf{INNER JOIN} & Returns matching records from both tables \\
\textbf{LEFT JOIN} & Returns all records from left table \\
\textbf{RIGHT JOIN} & Returns all records from right table \\
\textbf{FULL OUTER JOIN} & Returns all records from both tables \\
\textbf{CROSS JOIN} & Cartesian product of both tables \\
\textbf{SELF JOIN} & Table joined with itself \\
\end{longtable}
}

\textbf{Detailed Explanation:}

\textbf{1. INNER JOIN:}

\begin{lstlisting}[language=SQL]
SELECT s.name, c.course_name
FROM students s
INNER JOIN courses c ON s.course_id = c.course_id;
\end{lstlisting}

\begin{itemize}
\tightlist
\item
  Returns only matching records from both tables
\item
  Most commonly used join type
\end{itemize}

\textbf{2. LEFT JOIN:}

\begin{lstlisting}[language=SQL]
SELECT s.name, c.course_name
FROM students s
LEFT JOIN courses c ON s.course_id = c.course_id;
\end{lstlisting}

\begin{itemize}
\tightlist
\item
  Returns all students, even if no course assigned
\item
  NULL values for unmatched records
\end{itemize}

\end{solutionbox}
\begin{mnemonicbox}
``Join Tables Together Thoughtfully''

\end{mnemonicbox}
\subsection*{Question 4(c) [7 marks]}\label{q4c}

\textbf{Explain Concept of Functional Dependency in detail}

\begin{solutionbox}

\textbf{Functional Dependency} occurs when the value of one attribute
uniquely determines the value of another attribute.

\textbf{Notation:} A \rightarrow B (A functionally determines B)


{\def\LTcaptype{none} % do not increment counter
\vspace{-5pt}
\captionof{table}{Types of Functional Dependencies}
\vspace{-10pt}
\begin{longtable}[]{@{}
  >{\raggedright\arraybackslash}p{(\linewidth - 4\tabcolsep) * \real{0.2222}}
  >{\raggedright\arraybackslash}p{(\linewidth - 4\tabcolsep) * \real{0.4444}}
  >{\raggedright\arraybackslash}p{(\linewidth - 4\tabcolsep) * \real{0.3333}}@{}}
\toprule\noalign{}
\begin{minipage}[b]{\linewidth}\raggedright
Type
\end{minipage} & \begin{minipage}[b]{\linewidth}\raggedright
Definition
\end{minipage} & \begin{minipage}[b]{\linewidth}\raggedright
Example
\end{minipage} \\
\midrule\noalign{}
\endhead
\bottomrule\noalign{}
\endlastfoot
\textbf{Full FD} & All attributes in LHS needed & \{Student\_ID,
Course\_ID\} \rightarrow Grade \\
\textbf{Partial FD} & Some LHS attributes redundant & \{Student\_ID,
Course\_ID\} \rightarrow Student\_Name \\
\textbf{Transitive FD} & Indirect dependency through another attribute &
Student\_ID \rightarrow Dept\_ID \rightarrow Dept\_Name \\
\end{longtable}
}

\textbf{Diagram: Functional Dependency Example}

\begin{lstlisting}
Student_ID ---------> Student_Name
    |                      |
    |                      v
    |-----------------> Address
    |
    v
Course_ID ---------> Course_Name
\end{lstlisting}

\textbf{Key Properties:}

\begin{itemize}
\tightlist
\item
  \textbf{Reflexivity}: A \rightarrow A (trivial dependency)
\item
  \textbf{Augmentation}: If A \rightarrow B, then AC \rightarrow BC
\item
  \textbf{Transitivity}: If A \rightarrow B and B \rightarrow C, then A \rightarrow C
\item
  \textbf{Decomposition}: If A \rightarrow BC, then A \rightarrow B and A \rightarrow C
\end{itemize}

\textbf{Applications:}

\begin{itemize}
\tightlist
\item
  \textbf{Normalization}: Eliminates redundancy using FD
\item
  \textbf{Database Design}: Determines table structure
\item
  \textbf{Data Integrity}: Maintains consistency
\end{itemize}

\end{solutionbox}
\begin{mnemonicbox}
``Functions Determine Dependencies Directly''

\end{mnemonicbox}
\subsection*{Question 4(a) OR [3
marks]}\label{q4a}

\textbf{Write short note: Referential integrity Constraints}

\begin{solutionbox}

\textbf{Referential Integrity} ensures that foreign key values in one
table correspond to existing primary key values in referenced table.


{\def\LTcaptype{none} % do not increment counter
\vspace{-5pt}
\captionof{table}{Referential Integrity Rules}
\vspace{-10pt}
\begin{longtable}[]{@{}
  >{\raggedright\arraybackslash}p{(\linewidth - 4\tabcolsep) * \real{0.2222}}
  >{\raggedright\arraybackslash}p{(\linewidth - 4\tabcolsep) * \real{0.4815}}
  >{\raggedright\arraybackslash}p{(\linewidth - 4\tabcolsep) * \real{0.2963}}@{}}
\toprule\noalign{}
\begin{minipage}[b]{\linewidth}\raggedright
Rule
\end{minipage} & \begin{minipage}[b]{\linewidth}\raggedright
Description
\end{minipage} & \begin{minipage}[b]{\linewidth}\raggedright
Action
\end{minipage} \\
\midrule\noalign{}
\endhead
\bottomrule\noalign{}
\endlastfoot
\textbf{INSERT Rule} & Foreign key must exist in parent & Reject invalid
inserts \\
\textbf{DELETE Rule} & Handle parent record deletion & CASCADE,
RESTRICT, SET NULL \\
\textbf{UPDATE Rule} & Handle primary key updates & CASCADE, RESTRICT \\
\end{longtable}
}

\textbf{Key Features:}

\begin{itemize}
\tightlist
\item
  \textbf{Foreign Key Constraint}: Links related tables
\item
  \textbf{Data Consistency}: Prevents orphaned records
\item
  \textbf{Relationship Maintenance}: Preserves table relationships
\end{itemize}

\textbf{Code Example:}

\begin{lstlisting}[language=SQL]
ALTER TABLE Orders 
ADD CONSTRAINT FK_Customer 
FOREIGN KEY (customer_id) 
REFERENCES Customers(customer_id);
\end{lstlisting}

\end{solutionbox}
\begin{mnemonicbox}
``References Require Related Records''

\end{mnemonicbox}
\subsection*{Question 4(b) OR [4
marks]}\label{q4b}

\textbf{Explain union and intersection operations of relational algebra}

\begin{solutionbox}


{\def\LTcaptype{none} % do not increment counter
\vspace{-5pt}
\captionof{table}{Set Operations Comparison}
\vspace{-10pt}
\begin{longtable}[]{@{}
  >{\raggedright\arraybackslash}p{(\linewidth - 6\tabcolsep) * \real{0.2444}}
  >{\raggedright\arraybackslash}p{(\linewidth - 6\tabcolsep) * \real{0.1778}}
  >{\raggedright\arraybackslash}p{(\linewidth - 6\tabcolsep) * \real{0.2889}}
  >{\raggedright\arraybackslash}p{(\linewidth - 6\tabcolsep) * \real{0.2889}}@{}}
\toprule\noalign{}
\begin{minipage}[b]{\linewidth}\raggedright
Operation
\end{minipage} & \begin{minipage}[b]{\linewidth}\raggedright
Symbol
\end{minipage} & \begin{minipage}[b]{\linewidth}\raggedright
Description
\end{minipage} & \begin{minipage}[b]{\linewidth}\raggedright
Requirement
\end{minipage} \\
\midrule\noalign{}
\endhead
\bottomrule\noalign{}
\endlastfoot
\textbf{UNION} & \cup & Combines all tuples from both relations & Union
compatible \\
\textbf{INTERSECTION} & \cap & Common tuples in both relations & Union
compatible \\
\end{longtable}
}

\textbf{Union Operation:}

\begin{itemize}
\tightlist
\item
  \textbf{Syntax}: R \cup S
\item
  \textbf{Result}: All tuples from R and S (duplicates removed)
\item
  \textbf{Requirement}: Same number and types of attributes
\end{itemize}

\textbf{Intersection Operation:}

\begin{itemize}
\tightlist
\item
  \textbf{Syntax}: R \cap S\\
\item
  \textbf{Result}: Tuples that exist in both R and S
\item
  \textbf{Requirement}: Union compatible relations
\end{itemize}

\textbf{Example:}

\begin{lstlisting}
Students_CS \cup Students_IT = All students from both departments
Students_CS \cap Students_IT = Students in both departments
\end{lstlisting}

\textbf{Key Points:}

\begin{itemize}
\tightlist
\item
  \textbf{Union Compatibility}: Relations must have same structure
\item
  \textbf{Duplicate Elimination}: Results contain unique tuples only
\end{itemize}

\end{solutionbox}
\begin{mnemonicbox}
``Union Unites, Intersection Identifies Common''

\end{mnemonicbox}
\subsection*{Question 4(c) OR [7
marks]}\label{q4c}

\textbf{Explain Concept of Normalization in DBMS in detail}

\begin{solutionbox}

\textbf{Normalization} is the process of organizing database tables to
minimize data redundancy and improve data integrity.


{\def\LTcaptype{none} % do not increment counter
\vspace{-5pt}
\captionof{table}{Normal Forms}
\vspace{-10pt}
\begin{longtable}[]{@{}
  >{\raggedright\arraybackslash}p{(\linewidth - 4\tabcolsep) * \real{0.3421}}
  >{\raggedright\arraybackslash}p{(\linewidth - 4\tabcolsep) * \real{0.3421}}
  >{\raggedright\arraybackslash}p{(\linewidth - 4\tabcolsep) * \real{0.3158}}@{}}
\toprule\noalign{}
\begin{minipage}[b]{\linewidth}\raggedright
Normal Form
\end{minipage} & \begin{minipage}[b]{\linewidth}\raggedright
Requirements
\end{minipage} & \begin{minipage}[b]{\linewidth}\raggedright
Eliminates
\end{minipage} \\
\midrule\noalign{}
\endhead
\bottomrule\noalign{}
\endlastfoot
\textbf{1NF} & Atomic values, no repeating groups & Multivalued
attributes \\
\textbf{2NF} & 1NF + No partial dependencies & Partial functional
dependencies \\
\textbf{3NF} & 2NF + No transitive dependencies & Transitive
dependencies \\
\textbf{BCNF} & 3NF + Every determinant is candidate key & Remaining
anomalies \\
\end{longtable}
}

\textbf{Normalization Process:}

\textbf{Step 1 - First Normal Form (1NF):}

\begin{itemize}
\tightlist
\item
  Eliminate repeating groups
\item
  Each cell contains single value
\item
  Each record is unique
\end{itemize}

\textbf{Step 2 - Second Normal Form (2NF):}

\begin{itemize}
\tightlist
\item
  Must be in 1NF
\item
  Remove partial dependencies
\item
  Non-key attributes fully dependent on primary key
\end{itemize}

\textbf{Step 3 - Third Normal Form (3NF):}

\begin{itemize}
\tightlist
\item
  Must be in 2NF
\item
  Remove transitive dependencies
\item
  Non-key attributes not dependent on other non-key attributes
\end{itemize}

\textbf{Benefits of Normalization:}

\begin{itemize}
\tightlist
\item
  \textbf{Reduced Redundancy}: Eliminates duplicate data
\item
  \textbf{Data Integrity}: Maintains consistency
\item
  \textbf{Storage Efficiency}: Minimizes storage space
\item
  \textbf{Update Anomalies}: Prevents inconsistent updates
\end{itemize}

\textbf{Drawbacks:}

\begin{itemize}
\tightlist
\item
  \textbf{Complex Queries}: May require multiple joins
\item
  \textbf{Performance Impact}: Can slow down retrieval
\end{itemize}

\end{solutionbox}
\begin{mnemonicbox}
``Normalize to Neat, Non-redundant Tables''

\end{mnemonicbox}
\subsection*{Question 5(a) [3 marks]}\label{q5a}

\textbf{Describe Need of Normalization in DBMS}

\begin{solutionbox}


{\def\LTcaptype{none} % do not increment counter
\vspace{-5pt}
\captionof{table}{Problems Solved by Normalization}
\vspace{-10pt}
\begin{longtable}[]{@{}
  >{\raggedright\arraybackslash}p{(\linewidth - 4\tabcolsep) * \real{0.2812}}
  >{\raggedright\arraybackslash}p{(\linewidth - 4\tabcolsep) * \real{0.4062}}
  >{\raggedright\arraybackslash}p{(\linewidth - 4\tabcolsep) * \real{0.3125}}@{}}
\toprule\noalign{}
\begin{minipage}[b]{\linewidth}\raggedright
Problem
\end{minipage} & \begin{minipage}[b]{\linewidth}\raggedright
Description
\end{minipage} & \begin{minipage}[b]{\linewidth}\raggedright
Solution
\end{minipage} \\
\midrule\noalign{}
\endhead
\bottomrule\noalign{}
\endlastfoot
\textbf{Insertion Anomaly} & Cannot insert data without complete info &
Separate tables \\
\textbf{Update Anomaly} & Multiple updates for single change & Remove
redundancy \\
\textbf{Deletion Anomaly} & Loss of important data when deleting &
Preserve dependencies \\
\end{longtable}
}

\textbf{Key Needs:}

\begin{itemize}
\tightlist
\item
  \textbf{Data Consistency}: Ensures uniform data across database
\item
  \textbf{Storage Optimization}: Reduces redundant storage
\item
  \textbf{Maintenance Simplicity}: Easier database updates
\end{itemize}

\textbf{Benefits:}

\begin{itemize}
\tightlist
\item
  \textbf{Improved Data Quality}: Reduces errors and inconsistencies
\item
  \textbf{Flexible Design}: Easier to modify and extend
\item
  \textbf{Better Performance}: For update operations
\end{itemize}

\end{solutionbox}
\begin{mnemonicbox}
``Normalization Needs Neat Organization''

\end{mnemonicbox}
\subsection*{Question 5(b) [4 marks]}\label{q5b}

\textbf{Explain properties of Transaction in DBMS}

\begin{solutionbox}


{\def\LTcaptype{none} % do not increment counter
\vspace{-5pt}
\captionof{table}{ACID Properties}
\vspace{-10pt}
\begin{longtable}[]{@{}
  >{\raggedright\arraybackslash}p{(\linewidth - 4\tabcolsep) * \real{0.3125}}
  >{\raggedright\arraybackslash}p{(\linewidth - 4\tabcolsep) * \real{0.4062}}
  >{\raggedright\arraybackslash}p{(\linewidth - 4\tabcolsep) * \real{0.2812}}@{}}
\toprule\noalign{}
\begin{minipage}[b]{\linewidth}\raggedright
Property
\end{minipage} & \begin{minipage}[b]{\linewidth}\raggedright
Description
\end{minipage} & \begin{minipage}[b]{\linewidth}\raggedright
Purpose
\end{minipage} \\
\midrule\noalign{}
\endhead
\bottomrule\noalign{}
\endlastfoot
\textbf{Atomicity} & All operations succeed or all fail & Ensures
completeness \\
\textbf{Consistency} & Database remains in valid state & Maintains
integrity \\
\textbf{Isolation} & Concurrent transactions don't interfere & Prevents
conflicts \\
\textbf{Durability} & Committed changes are permanent & Ensures
persistence \\
\end{longtable}
}

\textbf{Detailed Explanation:}

\textbf{Atomicity:}

\begin{itemize}
\tightlist
\item
  Transaction is indivisible unit
\item
  Either all operations complete or none
\end{itemize}

\textbf{Consistency:}

\begin{itemize}
\tightlist
\item
  Database transitions from one valid state to another
\item
  All integrity constraints maintained
\end{itemize}

\textbf{Isolation:}

\begin{itemize}
\tightlist
\item
  Concurrent transactions appear to run sequentially
\item
  Intermediate states not visible to other transactions
\end{itemize}

\textbf{Durability:}

\begin{itemize}
\tightlist
\item
  Once committed, changes survive system failures
\item
  Data permanently stored
\end{itemize}

\end{solutionbox}
\begin{mnemonicbox}
``ACID Assures Correct Database''

\end{mnemonicbox}
\subsection*{Question 5(c) [7 marks]}\label{q5c}

\textbf{Explain View Serializability in detail}

\begin{solutionbox}

\textbf{View Serializability} determines if a concurrent schedule
produces the same result as some serial schedule by examining read and
write operations.


{\def\LTcaptype{none} % do not increment counter
\vspace{-5pt}
\captionof{table}{View Equivalence Conditions}
\vspace{-10pt}
\begin{longtable}[]{@{}ll@{}}
\toprule\noalign{}
Condition & Description \\
\midrule\noalign{}
\endhead
\bottomrule\noalign{}
\endlastfoot
\textbf{Initial Reads} & Same transactions read initial values \\
\textbf{Final Writes} & Same transactions perform final writes \\
\textbf{Intermediate Reads} & Read values from same writing
transactions \\
\end{longtable}
}

\textbf{Key Concepts:}

\textbf{View Equivalent Schedules:} Two schedules are view equivalent
if:

\begin{enumerate}
\tightlist
\item
  For each data item, if transaction T reads initial value in one
  schedule, it reads initial value in other
\item
  For each read operation, if T reads value written by T' in one
  schedule, same holds in other
\item
  For each data item, if T performs final write in one schedule, it
  performs final write in other
\end{enumerate}

\textbf{Testing View Serializability:}

\begin{enumerate}
\tightlist
\item
  \textbf{Precedence Graph}: Create directed graph
\item
  \textbf{Cycle Detection}: Check for cycles in graph
\item
  \textbf{Conflict Analysis}: Examine read-write conflicts
\end{enumerate}

\textbf{Example Analysis:}

\begin{lstlisting}
Schedule S1: R1(X) W1(X) R2(X) W2(X)
Schedule S2: R1(X) R2(X) W1(X) W2(X)
\end{lstlisting}

\textbf{Benefits:}

\begin{itemize}
\tightlist
\item
  \textbf{Concurrency Control}: Ensures correctness
\item
  \textbf{Performance}: Allows maximum concurrency
\item
  \textbf{Consistency}: Maintains database integrity
\end{itemize}

\textbf{Comparison with Conflict Serializability:}

\begin{itemize}
\tightlist
\item
  View serializability is less restrictive
\item
  Some view serializable schedules are not conflict serializable
\item
  More complex to test
\end{itemize}

\end{solutionbox}
\begin{mnemonicbox}
``View Verifies Valid Schedules''

\end{mnemonicbox}
\subsection*{Question 5(a) OR [3
marks]}\label{q5a}

\textbf{Perform 2NF on any Database}

\begin{solutionbox}

\textbf{Example: Student Course Database}

\textbf{Original Table (Not in 2NF):}

\begin{lstlisting}
Student_Course (Student_ID, Student_Name, Course_ID, Course_Name, Grade, Instructor)
Primary Key: {Student_ID, Course_ID}
\end{lstlisting}

\textbf{Functional Dependencies:}

\begin{itemize}
\tightlist
\item
  Student\_ID \rightarrow Student\_Name (Partial dependency)
\item
  Course\_ID \rightarrow Course\_Name, Instructor (Partial dependency)
\item
  \{Student\_ID, Course\_ID\} \rightarrow Grade
\end{itemize}

\textbf{2NF Decomposition:}

\textbf{Table 1: Students}

\begin{lstlisting}
Students (Student_ID, Student_Name)
Primary Key: Student_ID
\end{lstlisting}

\textbf{Table 2: Courses}

\begin{lstlisting}
Courses (Course_ID, Course_Name, Instructor)  
Primary Key: Course_ID
\end{lstlisting}

\textbf{Table 3: Enrollments}

\begin{lstlisting}
Enrollments (Student_ID, Course_ID, Grade)
Primary Key: {Student_ID, Course_ID}
Foreign Keys: Student_ID \rightarrow Students, Course_ID \rightarrow Courses
\end{lstlisting}

\textbf{Result:} All partial dependencies eliminated, now in 2NF.

\end{solutionbox}
\begin{mnemonicbox}
``Second Normal Form Separates Dependencies''

\end{mnemonicbox}
\subsection*{Question 5(b) OR [4
marks]}\label{q5b}

\textbf{Explain States of Transaction}

\begin{solutionbox}

\textbf{Diagram: Transaction State Diagram}

\includegraphics[width=1\linewidth,height=\textheight,keepaspectratio]{mermaid-8bd000c5.pdf}


{\def\LTcaptype{none} % do not increment counter
\vspace{-5pt}
\captionof{table}{Transaction States}
\vspace{-10pt}
\begin{longtable}[]{@{}
  >{\raggedright\arraybackslash}p{(\linewidth - 4\tabcolsep) * \real{0.2414}}
  >{\raggedright\arraybackslash}p{(\linewidth - 4\tabcolsep) * \real{0.4483}}
  >{\raggedright\arraybackslash}p{(\linewidth - 4\tabcolsep) * \real{0.3103}}@{}}
\toprule\noalign{}
\begin{minipage}[b]{\linewidth}\raggedright
State
\end{minipage} & \begin{minipage}[b]{\linewidth}\raggedright
Description
\end{minipage} & \begin{minipage}[b]{\linewidth}\raggedright
Actions
\end{minipage} \\
\midrule\noalign{}
\endhead
\bottomrule\noalign{}
\endlastfoot
\textbf{Active} & Transaction is executing & Read/Write operations \\
\textbf{Partially Committed} & Final statement executed & Waiting for
commit \\
\textbf{Committed} & Transaction completed successfully & Changes
permanent \\
\textbf{Failed} & Cannot proceed normally & Error occurred \\
\textbf{Aborted} & Transaction rolled back & All changes undone \\
\end{longtable}
}

\textbf{State Transitions:}

\begin{itemize}
\tightlist
\item
  \textbf{Active to Failed}: Due to errors or explicit abort
\item
  \textbf{Active to Partially Committed}: After final statement
\item
  \textbf{Partially Committed to Committed}: Successful completion
\item
  \textbf{Failed to Aborted}: After rollback operations
\end{itemize}

\textbf{Key Points:}

\begin{itemize}
\tightlist
\item
  \textbf{Recovery}: System can recover from failed states
\item
  \textbf{Durability}: Committed changes are permanent
\item
  \textbf{Atomicity}: Aborted transactions leave no trace
\end{itemize}

\end{solutionbox}
\begin{mnemonicbox}
``Transactions Travel Through States''

\end{mnemonicbox}
\subsection*{Question 5(c) OR [7
marks]}\label{q5c}

\textbf{Explain Conflict Serializability in detail}

\begin{solutionbox}

\textbf{Conflict Serializability} ensures that a concurrent schedule is
equivalent to some serial schedule by analyzing conflicting operations.


{\def\LTcaptype{none} % do not increment counter
\vspace{-5pt}
\captionof{table}{Conflicting Operations}
\vspace{-10pt}
\begin{longtable}[]{@{}lll@{}}
\toprule\noalign{}
Operation Pair & Conflict Type & Reason \\
\midrule\noalign{}
\endhead
\bottomrule\noalign{}
\endlastfoot
\textbf{Read-Write} & RW Conflict & Read before write \\
\textbf{Write-Read} & WR Conflict & Write before read \\
\textbf{Write-Write} & WW Conflict & Multiple writes \\
\end{longtable}
}

\textbf{Testing Conflict Serializability:}

\textbf{Step 1: Identify Conflicts}

\begin{itemize}
\tightlist
\item
  Find pairs of operations on same data item
\item
  Check if operations belong to different transactions
\item
  Determine if operations conflict
\end{itemize}

\textbf{Step 2: Create Precedence Graph}

\begin{itemize}
\tightlist
\item
  Nodes represent transactions
\item
  Directed edges represent conflicts
\item
  Edge from Ti to Tj if Ti conflicts with Tj
\end{itemize}

\textbf{Step 3: Check for Cycles}

\begin{itemize}
\tightlist
\item
  If graph has no cycles \rightarrow Conflict serializable
\item
  If graph has cycles \rightarrow Not conflict serializable
\end{itemize}

\textbf{Example Analysis:}

\begin{lstlisting}
Schedule: R1(A) W1(A) R2(A) W2(B) R1(B) W1(B)

Conflicts:
- W1(A) conflicts with R2(A) \rightarrow T1 before T2
- W2(B) conflicts with R1(B) \rightarrow T2 before T1
- W2(B) conflicts with W1(B) \rightarrow T2 before T1
\end{lstlisting}

\textbf{Precedence Graph:}

\begin{lstlisting}
    T1 \leftarrow----\rightarrow T2
       (cycle)
\end{lstlisting}

\textbf{Result:} Contains cycle, therefore NOT conflict serializable.


{\def\LTcaptype{none} % do not increment counter
\vspace{-5pt}
\captionof{table}{Serializability Testing Steps}
\vspace{-10pt}
\begin{longtable}[]{@{}lll@{}}
\toprule\noalign{}
Step & Action & Purpose \\
\midrule\noalign{}
\endhead
\bottomrule\noalign{}
\endlastfoot
\textbf{1} & List all operations & Identify transaction operations \\
\textbf{2} & Find conflicts & Determine operation dependencies \\
\textbf{3} & Build precedence graph & Visualize dependencies \\
\textbf{4} & Check for cycles & Test serializability \\
\end{longtable}
}

\textbf{Key Properties:}

\begin{itemize}
\tightlist
\item
  \textbf{Conflict Equivalent}: Same conflicts, same relative order
\item
  \textbf{Serial Schedule}: One transaction at a time
\item
  \textbf{Precedence Graph}: Directed graph showing dependencies
\item
  \textbf{Cycle Detection}: Determines conflict serializability
\end{itemize}

\textbf{Benefits:}

\begin{itemize}
\tightlist
\item
  \textbf{Concurrency Control}: Ensures correctness
\item
  \textbf{Performance}: Maximizes concurrent execution
\item
  \textbf{Consistency}: Maintains database integrity
\end{itemize}

\textbf{Comparison with View Serializability:}

\begin{itemize}
\tightlist
\item
  Conflict serializability is more restrictive
\item
  All conflict serializable schedules are view serializable
\item
  Easier to test than view serializability
\end{itemize}

\textbf{Algorithms for Testing:}

\begin{enumerate}
\tightlist
\item
  \textbf{Precedence Graph Method}: Build graph and check cycles
\item
  \textbf{Timestamp Ordering}: Use timestamps to order operations
\item
  \textbf{Two-Phase Locking}: Use locks to ensure serializability
\end{enumerate}

\end{solutionbox}
\begin{mnemonicbox}
``Conflicts Create Cycles, Check Carefully''

\end{mnemonicbox}

\end{document}
