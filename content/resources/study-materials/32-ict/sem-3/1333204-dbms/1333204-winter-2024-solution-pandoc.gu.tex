\documentclass[10pt,a4paper]{article}

% content/resources/templates/preamble.tex
\usepackage[margin=0.6in]{geometry}
\author{Milav Dabgar}
\usepackage{amsmath,amssymb,amsthm}
\usepackage{booktabs}
\usepackage{multirow}
\usepackage{xcolor}
\usepackage{tcolorbox}
\tcbuselibrary{breakable,skins}
\usepackage[colorlinks=true,linkcolor=blue]{hyperref}
\usepackage{titlesec}
\usepackage{enumitem}
\usepackage{tikz}
\usepackage{pgfplots}
\usepackage{circuitikz}
\usepackage[version=4]{mhchem}
\usepackage{longtable}
\usepackage{array}
\usepackage{float}
\usepackage{caption}
\usepackage{listings}

\lstset{
  basicstyle=\small\ttfamily,
  breaklines=true,
  breakatwhitespace=false,
  postbreak=\mbox{\textcolor{red}{$\hookrightarrow$}\space},
  float=false,
  numbers=left,
  numberstyle=\tiny\color{gray},
  numbersep=10pt,
  xleftmargin=2em,
  keywordstyle=\color{blue},
  commentstyle=\color{green!60!black},
  stringstyle=\color{purple},
  backgroundcolor=\color{gray!5},
  showstringspaces=false,
  tabsize=2,
  captionpos=b,
  keepspaces=true,
  columns=flexible
}

\pgfplotsset{compat=1.18}
\usetikzlibrary{shapes,arrows,positioning,calc,patterns,decorations.pathmorphing,decorations.markings,arrows.meta}

% Color scheme
\definecolor{headcolor}{RGB}{0,102,204}
\definecolor{keycolor}{RGB}{220,20,60}
\definecolor{solutioncolor}{RGB}{34,139,34}
\definecolor{mnemoniccolor}{RGB}{148,0,211}
\definecolor{codecolor}{RGB}{0,0,100}

% Spacing
\setlength{\parskip}{3pt}
\setlist[itemize]{nosep}
\setlist[enumerate]{nosep}

% Title formatting
\titleformat{\section}{\Large\bfseries\color{headcolor}}{\thesection}{1em}{}
\titleformat{\subsection}{\large\bfseries\color{headcolor}}{\thesubsection}{1em}{}

% Pandoc tightlist compatibility
\providecommand{\tightlist}{%
  \setlength{\itemsep}{0pt}\setlength{\parskip}{0pt}}

% Pandoc longtable compatibility
\newcounter{none}
\def\thenone{}


% content/resources/templates/gujarati-boxes.tex
\usepackage{fontspec}
\usepackage{polyglossia}

% Set Gujarati as main language (document is primarily in Gujarati)
% Note: gloss-gujarati.ldf doesn't exist in polyglossia, but it will use hyphenation patterns
\setdefaultlanguage{gujarati}
\setotherlanguage{english}

% Configure Gujarati font properly
% Use Language=Default to prevent polyglossia from trying to add language-specific features
% that don't exist for Gujarati, which causes "empty feature" warnings
\newfontfamily\gujaratifont[Script=Gujarati,AutoFakeBold=2.5,AutoFakeSlant=0.3]{Noto Sans Gujarati}
\setmainfont[Script=Gujarati,AutoFakeBold=2.5,AutoFakeSlant=0.3]{Noto Sans Gujarati}
% Use Noto Sans Gujarati for monospace to support Gujarati in text
\setmonofont[Scale=0.9]{Noto Sans Gujarati}

% Configure English to use the same font
\newfontfamily\englishfont[Script=Gujarati,AutoFakeBold=2.5,AutoFakeSlant=0.3]{Noto Sans Gujarati}

% Translations for polyglossia
\gappto\captionsgujarati{
  \renewcommand{\tablename}{કોષ્ટક}
  \renewcommand{\figurename}{આકૃતિ}
}

% Helper for TikZ nodes to ensure Gujarati font
\newcommand{\gu}[1]{{\gujaratifont #1}}

% Custom environments
\newtcolorbox{solutionbox}{
    breakable,
    enhanced,
    colback=solutioncolor!5!white,
    colframe=solutioncolor!75!black,
    fonttitle=\bfseries,
    title=જવાબ
}

\newtcolorbox{solutionboxnobreak}{
 colback=solutioncolor!5!white,
 colframe=solutioncolor!75!black,
 fonttitle=\bfseries,
 title=જવાબ
}

\newtcolorbox{keyformula}{
 breakable,
 enhanced,
 colback=keycolor!5!white,
 colframe=keycolor!75!black,
 fonttitle=\bfseries,
 title=રાસાયણિક સમીકરણ/સૂત્ર
}

\newtcolorbox{mnemonicbox}{
 breakable,
 enhanced,
 colback=mnemoniccolor!5!white,
 colframe=mnemoniccolor!75!black,
 fonttitle=\bfseries,
 title=મેમરી ટ્રીક
}


\begin{document}

\begin{center}
{\Huge\bfseries\color{headcolor} Subject Name (Gujarati)}\\[5pt]
{\LARGE 1333204 -- Winter 2024}\\[3pt]
{\large Semester 1 Study Material}\\[3pt]
{\normalsize\textit{Detailed Solutions and Explanations}}
\end{center}

\vspace{10pt}

\subsection*{પ્રશ્ન 1(અ) [3
માર્ક્સ]}\label{uxaaauxab0uxab6uxaa8-1uxa85-3-uxaaeuxab0uxa95uxab8}

\textbf{ફિલ્ડ, રેકોર્ડ, મેટાડેટા ને વ્યાખ્યાયિત કરો.}

\begin{solutionbox}

\begin{itemize}
\tightlist
\item
  \textbf{ફિલ્ડ}: એન્ટિટીના એક એટ્રિબ્યુટને રજૂ કરતો ડેટાનો એક એકમ
\item
  \textbf{રેકોર્ડ}: એન્ટિટી વિશે ડેટા સંગ્રહિત કરતા સંબંધિત ફિલ્ડ્સનો સમૂહ
\item
  \textbf{મેટાડેટા}: ડેટા વિશેની માહિતી જે ડેટાબેઝ ઓબ્જેક્ટ્સની સંરચના, ગુણધર્મો અને
  સંબંધોનું વર્ણન કરે છે
\end{itemize}

\end{solutionbox}
\begin{mnemonicbox}
``FRaMe'' (ફિલ્ડ, રેકોર્ડ, મેટાડેટા)

\end{mnemonicbox}
\subsection*{પ્રશ્ન 1(બ) [4
માર્ક્સ]}\label{uxaaauxab0uxab6uxaa8-1uxaac-4-uxaaeuxab0uxa95uxab8}

\textbf{સ્ટ્રોંગ અને વીક entity set ને વ્યાખ્યાયિત કરો.}

\begin{solutionbox}

{\def\LTcaptype{none} % do not increment counter
\begin{longtable}[]{@{}
  >{\raggedright\arraybackslash}p{(\linewidth - 6\tabcolsep) * \real{0.2549}}
  >{\raggedright\arraybackslash}p{(\linewidth - 6\tabcolsep) * \real{0.2549}}
  >{\raggedright\arraybackslash}p{(\linewidth - 6\tabcolsep) * \real{0.3137}}
  >{\raggedright\arraybackslash}p{(\linewidth - 6\tabcolsep) * \real{0.1765}}@{}}
\toprule\noalign{}
\begin{minipage}[b]{\linewidth}\raggedright
એન્ટિટી પ્રકાર
\end{minipage} & \begin{minipage}[b]{\linewidth}\raggedright
વર્ણન
\end{minipage} & \begin{minipage}[b]{\linewidth}\raggedright
ઓળખ
\end{minipage} & \begin{minipage}[b]{\linewidth}\raggedright
ઉદાહરણ
\end{minipage} \\
\midrule\noalign{}
\endhead
\bottomrule\noalign{}
\endlastfoot
\textbf{સ્ટ્રોંગ એન્ટિટી} & સ્વતંત્ર રીતે અસ્તિત્વમાં છે & તેની પોતાની પ્રાથમિક કી
ધરાવે છે & ગ્રાહક, કર્મચારી \\
\textbf{વીક એન્ટિટી} & સ્ટ્રોંગ એન્ટિટી પર આધાર રાખે છે & પેરેન્ટ એન્ટિટી કી જરૂરી છે
& બેંક એકાઉન્ટ, ઓર્ડર આઈટમ \\
\end{longtable}
}

\end{solutionbox}
\begin{mnemonicbox}
``SWing'' (Strong is With own identity, weak is Not
Getting own identity)

\end{mnemonicbox}
\subsection*{પ્રશ્ન 1(ક) [7
માર્ક્સ]}\label{uxaaauxab0uxab6uxaa8-1uxa95-7-uxaaeuxab0uxa95uxab8}

\textbf{ડેટા એબ્સ્ટ્રેક્શનના 3 સ્તરો સમજાવો.}

\begin{solutionbox}

{\def\LTcaptype{none} % do not increment counter
\begin{longtable}[]{@{}
  >{\raggedright\arraybackslash}p{(\linewidth - 4\tabcolsep) * \real{0.2414}}
  >{\raggedright\arraybackslash}p{(\linewidth - 4\tabcolsep) * \real{0.4483}}
  >{\raggedright\arraybackslash}p{(\linewidth - 4\tabcolsep) * \real{0.3103}}@{}}
\toprule\noalign{}
\begin{minipage}[b]{\linewidth}\raggedright
સ્તર
\end{minipage} & \begin{minipage}[b]{\linewidth}\raggedright
વર્ણન
\end{minipage} & \begin{minipage}[b]{\linewidth}\raggedright
વપરાશકર્તા
\end{minipage} \\
\midrule\noalign{}
\endhead
\bottomrule\noalign{}
\endlastfoot
\textbf{ફિઝિકલ લેવલ} & ડેટા ભૌતિક રીતે કેવી રીતે સંગ્રહિત થાય છે તે વર્ણવે છે &
સિસ્ટમ એડમિનિસ્ટ્રેટર્સ \\
\textbf{કન્સેપ્ચુઅલ લેવલ} & કયો ડેટા સંગ્રહિત થયેલો છે અને સંબંધોનું વર્ણન કરે છે & ડેટાબેઝ
ડિઝાઇનર્સ \\
\textbf{વ્યૂ લેવલ} & વપરાશકર્તાઓ માટે પ્રસ્તુત ડેટાબેઝનો ભાગ વર્ણવે છે & એન્ડ યુઝર્સ \\
\end{longtable}
}

\textbf{ડાયાગ્રામ:}

\includegraphics[width=1\linewidth,height=\textheight,keepaspectratio]{mermaid-688740fb.pdf}

\end{solutionbox}
\begin{mnemonicbox}
``PCV'' (Physical, Conceptual, View - નીચેથી ઉપર)

\end{mnemonicbox}
\subsection*{પ્રશ્ન 1(ક) OR [7
માર્ક્સ]}\label{uxaaauxab0uxab6uxaa8-1uxa95-or-7-uxaaeuxab0uxa95uxab8}

\textbf{DBMS ના ફાયદાઓ અને ગેરફાયદાઓ સમજાવો.}

\begin{solutionbox}

{\def\LTcaptype{none} % do not increment counter
\begin{longtable}[]{@{}ll@{}}
\toprule\noalign{}
ફાયદાઓ & ગેરફાયદાઓ \\
\midrule\noalign{}
\endhead
\bottomrule\noalign{}
\endlastfoot
\textbf{ડેટા રીડન્ડન્સી કંટ્રોલ} & સોફ્ટવેર અને હાર્ડવેરની \textbf{ઊંચી કિંમત} \\
\textbf{ડેટા કન્સિસ્ટન્સી} & ડિઝાઇન અને જાળવણીમાં \textbf{જટિલતા} \\
\textbf{બહેતર ડેટા સિક્યુરિટી} & ભારે ઉપયોગ સાથે \textbf{પર્ફોર્મન્સ પર અસર} \\
\textbf{ડેટા શેરિંગ} & સિસ્ટમ નિષ્ફળતાઓથી \textbf{સંવેદનશીલતા} \\
\textbf{ડેટા ઇન્ડિપેન્ડન્સ} & નિષ્ફળતા પછી \textbf{રિકવરી ચેલેન્જીસ} \\
\textbf{પ્રમાણભૂત એક્સેસ} & \textbf{વધારેલ તાલીમ આવશ્યકતાઓ} \\
\end{longtable}
}

\end{solutionbox}
\begin{mnemonicbox}
``BASIC-DV'' (Benefits: Access, Security,
Independence, Consistency - Drawbacks: Vulnerability)

\end{mnemonicbox}
\subsection*{પ્રશ્ન 2(અ) [3
માર્ક્સ]}\label{uxaaauxab0uxab6uxaa8-2uxa85-3-uxaaeuxab0uxa95uxab8}

\textbf{રિલેશનલ એલ્જેબ્રા નું સિલેક્ટ ઓપરેશન સમજાવો.}

\begin{solutionbox}

{\def\LTcaptype{none} % do not increment counter
\begin{longtable}[]{@{}ll@{}}
\toprule\noalign{}
સિલેક્ટ ઓપરેશન (σ) & વર્ણન \\
\midrule\noalign{}
\endhead
\bottomrule\noalign{}
\endlastfoot
\textbf{સિન્ટેક્સ} & σ(Relation) \\
\textbf{કાર્ય} & શરત સંતોષતા ટપલ્સ મેળવે છે \\
\textbf{ઉદાહરણ} & σsalary\textgreater30000(Employee) \\
\end{longtable}
}

\end{solutionbox}
\begin{mnemonicbox}
``SERVe'' (Select Exactly Required Values)

\end{mnemonicbox}
\subsection*{પ્રશ્ન 2(બ) [4
માર્ક્સ]}\label{uxaaauxab0uxab6uxaa8-2uxaac-4-uxaaeuxab0uxa95uxab8}

\textbf{DBMS માં પ્રાઇમરી, ફોરેઇન, સુપર, કેન્ડીડેટ કી ની વ્યાખ્યા આપો.}

\begin{solutionbox}

{\def\LTcaptype{none} % do not increment counter
\begin{longtable}[]{@{}ll@{}}
\toprule\noalign{}
કી પ્રકાર & વર્ણન \\
\midrule\noalign{}
\endhead
\bottomrule\noalign{}
\endlastfoot
\textbf{પ્રાઇમરી કી} & દરેક રેકોર્ડ માટે યુનિક ઓળખકર્તા \\
\textbf{ફોરેઇન કી} & અન્ય ટેબલમાં પ્રાઇમરી કી સાથે જોડાતું એટ્રિબ્યુટ \\
\textbf{સુપર કી} & એટ્રિબ્યુટ્સનો સેટ જે રેકોર્ડ્સને યુનિક રીતે ઓળખી શકે છે \\
\textbf{કેન્ડીડેટ કી} & મિનિમલ સુપર કી જે પ્રાઇમરી કી બની શકે છે \\
\end{longtable}
}

\end{solutionbox}
\begin{mnemonicbox}
``PFSC'' (Person First Shows Credentials)

\end{mnemonicbox}
\subsection*{પ્રશ્ન 2(ક) [7
માર્ક્સ]}\label{uxaaauxab0uxab6uxaa8-2uxa95-7-uxaaeuxab0uxa95uxab8}

\textbf{Library Management System નો E R Diagram દોરો.}

\begin{solutionbox}

\includegraphics[width=1\linewidth,height=\textheight,keepaspectratio]{mermaid-6cca00d6.pdf}

\end{solutionbox}
\begin{mnemonicbox}
``LIMB'' (Library Items, Members, Borrowing)

\end{mnemonicbox}
\subsection*{પ્રશ્ન 2(અ) OR [3
માર્ક્સ]}\label{uxaaauxab0uxab6uxaa8-2uxa85-or-3-uxaaeuxab0uxa95uxab8}

\textbf{રિલેશનલ એલ્જેબ્રા નું યુનિયન ઓપરેશન સમજાવો.}

\begin{solutionbox}

{\def\LTcaptype{none} % do not increment counter
\begin{longtable}[]{@{}ll@{}}
\toprule\noalign{}
યુનિયન ઓપરેશન (\cup) & વર્ણન \\
\midrule\noalign{}
\endhead
\bottomrule\noalign{}
\endlastfoot
\textbf{સિન્ટેક્સ} & Relation1 \cup Relation2 \\
\textbf{કાર્ય} & બંને સંબંધોમાંથી ટપલ્સ એકત્રિત કરે છે \\
\textbf{આવશ્યકતા} & બંને સંબંધો યુનિયન-સંગત હોવા જોઈએ \\
\end{longtable}
}

\textbf{ઉદાહરણ:} Students\_CS \cup Students\_IT

\end{solutionbox}
\begin{mnemonicbox}
``CUP'' (Combining Union of Parts)

\end{mnemonicbox}
\subsection*{પ્રશ્ન 2(બ) OR [4
માર્ક્સ]}\label{uxaaauxab0uxab6uxaa8-2uxaac-or-4-uxaaeuxab0uxa95uxab8}

\textbf{ઉદાહરણ સાથે કંપોઝિટ એટ્રિબ્યુટ અને મલ્ટીવેલ્યુડ એટ્રિબ્યુટ ને વ્યાખ્યાયિત કરો.}

\begin{solutionbox}

{\def\LTcaptype{none} % do not increment counter
\begin{longtable}[]{@{}
  >{\raggedright\arraybackslash}p{(\linewidth - 4\tabcolsep) * \real{0.4054}}
  >{\raggedright\arraybackslash}p{(\linewidth - 4\tabcolsep) * \real{0.3514}}
  >{\raggedright\arraybackslash}p{(\linewidth - 4\tabcolsep) * \real{0.2432}}@{}}
\toprule\noalign{}
\begin{minipage}[b]{\linewidth}\raggedright
એટ્રિબ્યુટ પ્રકાર
\end{minipage} & \begin{minipage}[b]{\linewidth}\raggedright
વર્ણન
\end{minipage} & \begin{minipage}[b]{\linewidth}\raggedright
ઉદાહરણ
\end{minipage} \\
\midrule\noalign{}
\endhead
\bottomrule\noalign{}
\endlastfoot
\textbf{કંપોઝિટ} & નાના સબપાર્ટ્સમાં વિભાજિત થઈ શકે છે & એડ્રેસ (સ્ટ્રીટ, શહેર,
રાજ્ય, પિન) \\
\textbf{મલ્ટીવેલ્યુડ} & એક કરતાં વધુ મૂલ્ય ધરાવી શકે છે & ફોન નંબર્સ, ઈમેલ એડ્રેસિસ \\
\end{longtable}
}

\textbf{ડાયાગ્રામ:}

\includegraphics[width=1\linewidth,height=\textheight,keepaspectratio]{mermaid-9bbb13c3.pdf}

\end{solutionbox}
\begin{mnemonicbox}
``CoMbo'' (Composite has Multiple components)

\end{mnemonicbox}
\subsection*{પ્રશ્ન 2(ક) OR [7
માર્ક્સ]}\label{uxaaauxab0uxab6uxaa8-2uxa95-or-7-uxaaeuxab0uxa95uxab8}

\textbf{College Management System નો E R Diagram દોરો.}

\begin{solutionbox}

\includegraphics[width=1\linewidth,height=\textheight,keepaspectratio]{mermaid-821c5a53.pdf}

\end{solutionbox}
\begin{mnemonicbox}
``DECFS'' (Departments, Enrollments, Courses,
Faculty, Students)

\end{mnemonicbox}
\subsection*{પ્રશ્ન 3(અ) [3
માર્ક્સ]}\label{uxaaauxab0uxab6uxaa8-3uxa85-3-uxaaeuxab0uxa95uxab8}

\textbf{SQL માં વિવિધ ડેટા ટાઈપ્સ ની યાદી બનાવો અને ટુંક માં સમજાવો}

\begin{solutionbox}

{\def\LTcaptype{none} % do not increment counter
\begin{longtable}[]{@{}lll@{}}
\toprule\noalign{}
ડેટા ટાઈપ કેટેગરી & ઉદાહરણો & ઉપયોગ \\
\midrule\noalign{}
\endhead
\bottomrule\noalign{}
\endlastfoot
\textbf{ન્યુમેરિક} & INT, FLOAT, DECIMAL & સંખ્યાઓ સંગ્રહ કરવા \\
\textbf{કેરેક્ટર} & CHAR, VARCHAR, TEXT & ટેક્સ્ટ સંગ્રહ કરવા \\
\textbf{ડેટ/ટાઈમ} & DATE, TIME, TIMESTAMP & સમય સંબંધિત ડેટા સંગ્રહ કરવા \\
\textbf{બૂલિયન} & BOOLEAN & સાચા/ખોટા મૂલ્યો સંગ્રહ કરવા \\
\textbf{બાઇનરી} & BLOB, BINARY & બાઇનરી ડેટા સંગ્રહ કરવા \\
\end{longtable}
}

\end{solutionbox}
\begin{mnemonicbox}
``NCDBB'' (Numbers, Characters, Dates, Booleans,
Binaries)

\end{mnemonicbox}
\subsection*{પ્રશ્ન 3(બ) [4
માર્ક્સ]}\label{uxaaauxab0uxab6uxaa8-3uxaac-4-uxaaeuxab0uxa95uxab8}

\textbf{કોઈ પણ બે DDL કમાન્ડસ સિંટેક્ષ અને ઉદાહરણ સાથે સમજાવો.}

\begin{solutionbox}

{\def\LTcaptype{none} % do not increment counter
\begin{longtable}[]{@{}
  >{\raggedright\arraybackslash}p{(\linewidth - 4\tabcolsep) * \real{0.3462}}
  >{\raggedright\arraybackslash}p{(\linewidth - 4\tabcolsep) * \real{0.3077}}
  >{\raggedright\arraybackslash}p{(\linewidth - 4\tabcolsep) * \real{0.3462}}@{}}
\toprule\noalign{}
\begin{minipage}[b]{\linewidth}\raggedright
કમાન્ડ
\end{minipage} & \begin{minipage}[b]{\linewidth}\raggedright
સિન્ટેક્સ
\end{minipage} & \begin{minipage}[b]{\linewidth}\raggedright
ઉદાહરણ
\end{minipage} \\
\midrule\noalign{}
\endhead
\bottomrule\noalign{}
\endlastfoot
\textbf{CREATE} & CREATE TABLE table\_name (column\_definitions); &
CREATE TABLE Student (id INT PRIMARY KEY, name VARCHAR(50)); \\
\textbf{ALTER} & ALTER TABLE table\_name ADD/DROP/MODIFY column\_name
data\_type; & ALTER TABLE Student ADD email VARCHAR(100); \\
\end{longtable}
}

\textbf{ડાયાગ્રામ:}

\includegraphics[width=1\linewidth,height=\textheight,keepaspectratio]{mermaid-fe1011e0.pdf}

\end{solutionbox}
\begin{mnemonicbox}
``CAD'' (Create And Define)

\end{mnemonicbox}
\subsection*{પ્રશ્ન 3(ક) [7
માર્ક્સ]}\label{uxaaauxab0uxab6uxaa8-3uxa95-7-uxaaeuxab0uxa95uxab8}

\textbf{નીચે ની ક્વેરી નું આઉટપુટ લખો.} \textbf{a. CEIL(123.57), CEIL(4.1)}
\textbf{b. MOD(12,4), MOD(10,4)} \textbf{c.~POWER(2,3), POWER(3,3)}
\textbf{d.~ROUND(121.413,1), ROUND(121.413,2)} \textbf{e.
FLOOR(25.3),FLOOR(25.7)} \textbf{f.~LENGTH(`AHMEDABAD')} \textbf{g.
ABS(-25),ABS(36)}

\begin{solutionbox}

{\def\LTcaptype{none} % do not increment counter
\begin{longtable}[]{@{}lll@{}}
\toprule\noalign{}
ફંક્શન & પરિણામ & સમજૂતી \\
\midrule\noalign{}
\endhead
\bottomrule\noalign{}
\endlastfoot
\textbf{CEIL(123.57)} & 124 & 123.57 થી મોટી કે સમાન સૌથી નાની પૂર્ણ
સંખ્યા \\
\textbf{CEIL(4.1)} & 5 & 4.1 થી મોટી કે સમાન સૌથી નાની પૂર્ણ સંખ્યા \\
\textbf{MOD(12,4)} & 0 & 12\div4 નો શેષ \\
\textbf{MOD(10,4)} & 2 & 10\div4 નો શેષ \\
\textbf{POWER(2,3)} & 8 & 2 ને 3 ની ઘાત \\
\textbf{POWER(3,3)} & 27 & 3 ને 3 ની ઘાત \\
\textbf{ROUND(121.413,1)} & 121.4 & 1 દશાંશ સ્થાન સુધી રાઉન્ડ \\
\textbf{ROUND(121.413,2)} & 121.41 & 2 દશાંશ સ્થાન સુધી રાઉન્ડ \\
\textbf{FLOOR(25.3)} & 25 & 25.3 થી નાની કે સમાન સૌથી મોટી પૂર્ણ સંખ્યા \\
\textbf{FLOOR(25.7)} & 25 & 25.7 થી નાની કે સમાન સૌથી મોટી પૂર્ણ સંખ્યા \\
\textbf{LENGTH(`AHMEDABAD')} & 9 & અક્ષરોની સંખ્યા \\
\textbf{ABS(-25)} & 25 & -25 ની નિરપેક્ષ કિંમત \\
\textbf{ABS(36)} & 36 & 36 ની નિરપેક્ષ કિંમત \\
\end{longtable}
}

\end{solutionbox}
\begin{mnemonicbox}
``CMPRFLA'' (Ceiling, Modulus, Power, Round, Floor,
Length, Absolute)

\end{mnemonicbox}
\subsection*{પ્રશ્ન 3(અ) OR [3
માર્ક્સ]}\label{uxaaauxab0uxab6uxaa8-3uxa85-or-3-uxaaeuxab0uxa95uxab8}

\textbf{કોઈ પણ ત્રણ ડેટ ફંક્શન સમજાવો.}

\begin{solutionbox}

{\def\LTcaptype{none} % do not increment counter
\begin{longtable}[]{@{}
  >{\raggedright\arraybackslash}p{(\linewidth - 6\tabcolsep) * \real{0.3500}}
  >{\raggedright\arraybackslash}p{(\linewidth - 6\tabcolsep) * \real{0.2250}}
  >{\raggedright\arraybackslash}p{(\linewidth - 6\tabcolsep) * \real{0.2250}}
  >{\raggedright\arraybackslash}p{(\linewidth - 6\tabcolsep) * \real{0.2000}}@{}}
\toprule\noalign{}
\begin{minipage}[b]{\linewidth}\raggedright
ડેટ ફંક્શન
\end{minipage} & \begin{minipage}[b]{\linewidth}\raggedright
હેતુ
\end{minipage} & \begin{minipage}[b]{\linewidth}\raggedright
ઉદાહરણ
\end{minipage} & \begin{minipage}[b]{\linewidth}\raggedright
પરિણામ
\end{minipage} \\
\midrule\noalign{}
\endhead
\bottomrule\noalign{}
\endlastfoot
\textbf{ADD\_MONTHS} & તારીખમાં મહિના ઉમેરે છે & ADD\_MONTHS(`01-JAN-2023',
3) & 01-APR-2023 \\
\textbf{MONTHS\_BETWEEN} & બે તારીખો વચ્ચેના મહિના ગણે છે &
MONTHS\_BETWEEN(`01-MAR-2023', `01-JAN-2023') & 2 \\
\textbf{SYSDATE} & વર્તમાન તારીખ અને સમય આપે છે & SYSDATE & વર્તમાન સિસ્ટમ
તારીખ/સમય \\
\end{longtable}
}

\end{solutionbox}
\begin{mnemonicbox}
``AMS'' (Add\_months, Months\_between, Sysdate)

\end{mnemonicbox}
\subsection*{પ્રશ્ન 3(બ) OR [4
માર્ક્સ]}\label{uxaaauxab0uxab6uxaa8-3uxaac-or-4-uxaaeuxab0uxa95uxab8}

\textbf{કોઈ પણ બે DML કમાન્ડ સિંટેક્ષ અને ઉદાહરણ સાથે સમજાવો.}

\begin{solutionbox}

{\def\LTcaptype{none} % do not increment counter
\begin{longtable}[]{@{}
  >{\raggedright\arraybackslash}p{(\linewidth - 4\tabcolsep) * \real{0.3462}}
  >{\raggedright\arraybackslash}p{(\linewidth - 4\tabcolsep) * \real{0.3077}}
  >{\raggedright\arraybackslash}p{(\linewidth - 4\tabcolsep) * \real{0.3462}}@{}}
\toprule\noalign{}
\begin{minipage}[b]{\linewidth}\raggedright
કમાન્ડ
\end{minipage} & \begin{minipage}[b]{\linewidth}\raggedright
સિન્ટેક્સ
\end{minipage} & \begin{minipage}[b]{\linewidth}\raggedright
ઉદાહરણ
\end{minipage} \\
\midrule\noalign{}
\endhead
\bottomrule\noalign{}
\endlastfoot
\textbf{INSERT} & INSERT INTO table\_name VALUES (value1,
value2,\ldots); & INSERT INTO Student VALUES (1, `Raj',
`raj@example.com'); \\
\textbf{UPDATE} & UPDATE table\_name SET column=value WHERE condition; &
UPDATE Student SET email=`new@example.com' WHERE id=1; \\
\end{longtable}
}

\textbf{ડાયાગ્રામ:}

\includegraphics[width=1\linewidth,height=\textheight,keepaspectratio]{mermaid-4feee73c.pdf}

\end{solutionbox}
\begin{mnemonicbox}
``IUM'' (Insert, Update, Manipulate)

\end{mnemonicbox}
\subsection*{પ્રશ્ન 3(ક) OR [7
માર્ક્સ]}\label{uxaaauxab0uxab6uxaa8-3uxa95-or-7-uxaaeuxab0uxa95uxab8}

\textbf{ટેબલ EMP(emp\_no, emp\_name, designation, salary, deptno) ને ધ્યાને
લઈ ને નીચે આપેલા operations માટે SQL commands લખો.}

\begin{solutionbox}

{\def\LTcaptype{none} % do not increment counter
\begin{longtable}[]{@{}
  >{\raggedright\arraybackslash}p{(\linewidth - 2\tabcolsep) * \real{0.4583}}
  >{\raggedright\arraybackslash}p{(\linewidth - 2\tabcolsep) * \real{0.5417}}@{}}
\toprule\noalign{}
\begin{minipage}[b]{\linewidth}\raggedright
ઓપરેશન
\end{minipage} & \begin{minipage}[b]{\linewidth}\raggedright
SQL કમાન્ડ
\end{minipage} \\
\midrule\noalign{}
\endhead
\bottomrule\noalign{}
\endlastfoot
\textbf{EMP ટેબલ ને ક્રિએટ કરો} & CREATE TABLE EMP (emp\_no INT PRIMARY
KEY, emp\_name VARCHAR(50), designation VARCHAR(30), salary
DECIMAL(10,2), deptno INT); \\
\textbf{emp\_no, emp\_name, designation, salary, deptno ને EMP ને આપો} &
SELECT emp\_no, emp\_name, designation, salary, deptno FROM EMP; \\
\textbf{જેમના નામ `p' થી શરૂ થતાં હોય તેવા બધા એમ્પ્લોયી ની માહિતી દશાર્વો} &
SELECT * FROM EMP WHERE emp\_name LIKE `p\%'; \\
\textbf{Department wise salary total દશાર્વો} & SELECT deptno,
SUM(salary) AS total\_salary FROM EMP GROUP BY deptno; \\
\textbf{EMP table માં નવી કૉલમ email\_id ઉમેરો} & ALTER TABLE EMP ADD
email\_id VARCHAR(100); \\
\textbf{કૉલમ નામ ``designation'' ને ``post'' તરીકે બદલાવો} & ALTER TABLE
EMP RENAME COLUMN designation TO post; \\
\textbf{ટેબલ person ના તમામ records delete કરો} & DELETE FROM person; \\
\end{longtable}
}

\end{solutionbox}
\begin{mnemonicbox}
``CSDAACD'' (Create, Select, Display, Aggregate,
Add, Change, Delete)

\end{mnemonicbox}
\subsection*{પ્રશ્ન 4(અ) [3
માર્ક્સ]}\label{uxaaauxab0uxab6uxaa8-4uxa85-3-uxaaeuxab0uxa95uxab8}

\textbf{વિવિધ aggregate functions ની યાદી બનાવો અને કોઈ એક ને syntax અને
ઉદાહરણ સાથે સમજાવો.}

\begin{solutionbox}

{\def\LTcaptype{none} % do not increment counter
\begin{longtable}[]{@{}ll@{}}
\toprule\noalign{}
એગ્રીગેટ ફંક્શન & હેતુ \\
\midrule\noalign{}
\endhead
\bottomrule\noalign{}
\endlastfoot
\textbf{SUM} & કુલ ગણતરી કરે છે \\
\textbf{AVG} & સરેરાશ ગણતરી કરે છે \\
\textbf{COUNT} & રો ની સંખ્યા ગણે છે \\
\textbf{MAX} & મહત્તમ મૂલ્ય શોધે છે \\
\textbf{MIN} & લઘુત્તમ મૂલ્ય શોધે છે \\
\end{longtable}
}

\textbf{AVG માટે ઉદાહરણ:}\\
\passthrough{\lstinline!AVG(column\_name)!} - કોલમમાં મૂલ્યોની સરેરાશ ગણતરી
કરે છે\\
\passthrough{\lstinline!SELECT AVG(salary) FROM Employee;!} - સરેરાશ પગાર
આપે છે

\end{solutionbox}
\begin{mnemonicbox}
``SCAMM'' (Sum, Count, Avg, Max, Min)

\end{mnemonicbox}
\subsection*{પ્રશ્ન 4(બ) [4
માર્ક્સ]}\label{uxaaauxab0uxab6uxaa8-4uxaac-4-uxaaeuxab0uxa95uxab8}

\textbf{ટ્રાન્સેક્શન ને ઉદાહરણ સાથે વ્યાખ્યાયિત કરો.}

\begin{solutionbox}

{\def\LTcaptype{none} % do not increment counter
\begin{longtable}[]{@{}
  >{\raggedright\arraybackslash}p{(\linewidth - 2\tabcolsep) * \real{0.6061}}
  >{\raggedright\arraybackslash}p{(\linewidth - 2\tabcolsep) * \real{0.3939}}@{}}
\toprule\noalign{}
\begin{minipage}[b]{\linewidth}\raggedright
ટ્રાન્સેક્શન કન્સેપ્ટ
\end{minipage} & \begin{minipage}[b]{\linewidth}\raggedright
વર્ણન
\end{minipage} \\
\midrule\noalign{}
\endhead
\bottomrule\noalign{}
\endlastfoot
\textbf{વ્યાખ્યા} & કાર્યનો તાર્કિક એકમ જે સંપૂર્ણપણે પ્રોસેસ થવો જોઈએ અથવા સંપૂર્ણપણે
નિષ્ફળ જવો જોઈએ \\
\textbf{ગુણધર્મો} & ACID (એટોમિસિટી, કન્સિસ્ટન્સી, આઈસોલેશન, ડ્યુરેબિલિટી) \\
\textbf{સ્થિતિઓ} & એક્ટિવ, પાર્શિયલી કમિટેડ, કમિટેડ, ફેઇલ્ડ, એબોર્ટેડ \\
\end{longtable}
}

\textbf{ઉદાહરણ:}

\begin{lstlisting}[language=SQL]
BEGIN TRANSACTION;
    UPDATE Accounts SET balance = balance - 5000 WHERE acc_no = 'A123';
    UPDATE Accounts SET balance = balance + 5000 WHERE acc_no = 'B456';
COMMIT;
\end{lstlisting}

\end{solutionbox}
\begin{mnemonicbox}
``TAPS'' (Transaction As Process Set)

\end{mnemonicbox}
\subsection*{પ્રશ્ન 4(ક) [7
માર્ક્સ]}\label{uxaaauxab0uxab6uxaa8-4uxa95-7-uxaaeuxab0uxa95uxab8}

\textbf{SQL માં ઓપરેટર શું છે? એરિથમેટિક અને લોજિકલ ઓપરેટર ઉદાહરણ સાથે સમજાવો.}

\begin{solutionbox}

{\def\LTcaptype{none} % do not increment counter
\begin{longtable}[]{@{}
  >{\raggedright\arraybackslash}p{(\linewidth - 6\tabcolsep) * \real{0.1765}}
  >{\raggedright\arraybackslash}p{(\linewidth - 6\tabcolsep) * \real{0.3235}}
  >{\raggedright\arraybackslash}p{(\linewidth - 6\tabcolsep) * \real{0.2647}}
  >{\raggedright\arraybackslash}p{(\linewidth - 6\tabcolsep) * \real{0.2353}}@{}}
\toprule\noalign{}
\begin{minipage}[b]{\linewidth}\raggedright
પ્રકાર
\end{minipage} & \begin{minipage}[b]{\linewidth}\raggedright
ઓપરેટર્સ
\end{minipage} & \begin{minipage}[b]{\linewidth}\raggedright
ઉદાહરણ
\end{minipage} & \begin{minipage}[b]{\linewidth}\raggedright
પરિણામ
\end{minipage} \\
\midrule\noalign{}
\endhead
\bottomrule\noalign{}
\endlastfoot
\textbf{એરિથમેટિક} & + (ઉમેરો) & 5 + 3 & 8 \\
& - (બાદબાકી) & 5 - 3 & 2 \\
& * (ગુણાકાર) & 5 * 3 & 15 \\
& / (ભાગાકાર) & 15 / 3 & 5 \\
& \% (મોડ્યુલસ) & 5 \% 2 & 1 \\
\textbf{લોજિકલ} & AND & salary \textgreater{} 30000 AND dept = `IT' & બંને
શરતો સાચી હોય તો સાચું \\
& OR & salary \textgreater{} 50000 OR dept = `HR' & કોઈપણ એક શરત સાચી
હોય તો સાચું \\
& NOT & NOT (salary \textless{} 20000) & જો પગાર 20000 થી ઓછો ન હોય તો
સાચું \\
\end{longtable}
}

\textbf{SQL ઉદાહરણો:}

\begin{lstlisting}[language=SQL]
-- એરિથમેટિક
SELECT product_name, price * 1.18 AS price_with_tax FROM Products;

-- લોજિકલ
SELECT * FROM Employees WHERE (salary > 30000 AND dept = 'IT') OR (experience > 5);
\end{lstlisting}

\end{solutionbox}
\begin{mnemonicbox}
``ASMDOLA'' (Add, Subtract, Multiply, Divide, OR,
AND, NOT)

\end{mnemonicbox}
\subsection*{પ્રશ્ન 4(અ) OR [3
માર્ક્સ]}\label{uxaaauxab0uxab6uxaa8-4uxa85-or-3-uxaaeuxab0uxa95uxab8}

\textbf{વિવિધ numeric functions ની યાદી બનાવો અને કોઈ એક ને syntax અને
ઉદાહરણ સાથે સમજાવો.}

\begin{solutionbox}

{\def\LTcaptype{none} % do not increment counter
\begin{longtable}[]{@{}ll@{}}
\toprule\noalign{}
ન્યુમેરિક ફંક્શન & હેતુ \\
\midrule\noalign{}
\endhead
\bottomrule\noalign{}
\endlastfoot
\textbf{ROUND} & સંખ્યાને નિર્દિષ્ટ દશાંશ સ્થાનો સુધી રાઉન્ડ કરે છે \\
\textbf{TRUNC} & સંખ્યાને નિર્દિષ્ટ દશાંશ સ્થાનો સુધી ટ્રંકેટ કરે છે \\
\textbf{CEIL} & સંખ્યાથી મોટી કે સમાન સૌથી નાની પૂર્ણ સંખ્યા આપે છે \\
\textbf{FLOOR} & સંખ્યાથી નાની કે સમાન સૌથી મોટી પૂર્ણ સંખ્યા આપે છે \\
\textbf{ABS} & નિરપેક્ષ મૂલ્ય આપે છે \\
\end{longtable}
}

\textbf{ROUND માટે ઉદાહરણ:}\\
\passthrough{\lstinline!ROUND(number, decimal\_places)!} - સંખ્યાને નિર્દિષ્ટ
દશાંશ સ્થાનો સુધી રાઉન્ડ કરે છે\\
\passthrough{\lstinline!SELECT ROUND(125.679, 2) FROM DUAL;!} - 125.68
આપે છે

\end{solutionbox}
\begin{mnemonicbox}
``RTCFA'' (Round, Truncate, Ceiling, Floor,
Absolute)

\end{mnemonicbox}
\subsection*{પ્રશ્ન 4(બ) OR [4
માર્ક્સ]}\label{uxaaauxab0uxab6uxaa8-4uxaac-or-4-uxaaeuxab0uxa95uxab8}

\textbf{ટ્રાન્સેક્શન માટે વિવિધ database operations ની યાદી બનાવો.}

\begin{solutionbox}

{\def\LTcaptype{none} % do not increment counter
\begin{longtable}[]{@{}ll@{}}
\toprule\noalign{}
ઓપરેશન & વર્ણન \\
\midrule\noalign{}
\endhead
\bottomrule\noalign{}
\endlastfoot
\textbf{BEGIN/START} & ટ્રાન્સેક્શન શરૂઆત બિંદુ ચિહ્નિત કરે છે \\
\textbf{READ} & ડેટાબેઝમાંથી ડેટા મેળવે છે \\
\textbf{WRITE} & ડેટાબેઝમાં ડેટા સુધારે છે \\
\textbf{COMMIT} & ફેરફારો કાયમી બનાવે છે \\
\textbf{ROLLBACK} & ફેરફારો રદ કરે છે અને પ્રારંભિક બિંદુ પર પાછા ફરે છે \\
\textbf{SAVEPOINT} & આંશિક રૂપે પાછા ફરવા માટે બિંદુઓ બનાવે છે \\
\end{longtable}
}

\textbf{ડાયાગ્રામ:}

\includegraphics[width=1\linewidth,height=\textheight,keepaspectratio]{mermaid-f5eaac78.pdf}

\end{solutionbox}
\begin{mnemonicbox}
``BRWCRS'' (Begin, Read, Write, Commit, Rollback,
Savepoint)

\end{mnemonicbox}
\subsection*{પ્રશ્ન 4(ક) OR [7
માર્ક્સ]}\label{uxaaauxab0uxab6uxaa8-4uxa95-or-7-uxaaeuxab0uxa95uxab8}

\textbf{જોઇન શું છે? વિવિધ પ્રકાર ના જોઇન ને syntax અને ઉદાહરણ સાથે સમજાવો.}

\begin{solutionbox}

{\def\LTcaptype{none} % do not increment counter
\begin{longtable}[]{@{}
  >{\raggedright\arraybackslash}p{(\linewidth - 4\tabcolsep) * \real{0.2750}}
  >{\raggedright\arraybackslash}p{(\linewidth - 4\tabcolsep) * \real{0.3250}}
  >{\raggedright\arraybackslash}p{(\linewidth - 4\tabcolsep) * \real{0.4000}}@{}}
\toprule\noalign{}
\begin{minipage}[b]{\linewidth}\raggedright
જોઇન પ્રકાર
\end{minipage} & \begin{minipage}[b]{\linewidth}\raggedright
વર્ણન
\end{minipage} & \begin{minipage}[b]{\linewidth}\raggedright
સિન્ટેક્સ ઉદાહરણ
\end{minipage} \\
\midrule\noalign{}
\endhead
\bottomrule\noalign{}
\endlastfoot
\textbf{INNER JOIN} & બંને ટેબલમાં મેચ હોય ત્યારે રો આપે છે & SELECT * FROM
TableA INNER JOIN TableB ON TableA.id = TableB.id; \\
\textbf{LEFT JOIN} & ડાબા ટેબલના બધા રો અને જમણા ટેબલના મેચ થતા રો આપે છે &
SELECT * FROM TableA LEFT JOIN TableB ON TableA.id = TableB.id; \\
\textbf{RIGHT JOIN} & જમણા ટેબલના બધા રો અને ડાબા ટેબલના મેચ થતા રો આપે છે &
SELECT * FROM TableA RIGHT JOIN TableB ON TableA.id = TableB.id; \\
\textbf{FULL JOIN} & કોઈપણ એક ટેબલમાં મેચ હોય ત્યારે રો આપે છે & SELECT * FROM
TableA FULL JOIN TableB ON TableA.id = TableB.id; \\
\textbf{SELF JOIN} & ટેબલને તેની જાત સાથે જોડે છે & SELECT * FROM Employee e1
JOIN Employee e2 ON e1.manager\_id = e2.emp\_id; \\
\end{longtable}
}

\textbf{ડાયાગ્રામ:}

\includegraphics[width=1\linewidth,height=\textheight,keepaspectratio]{mermaid-e741fe33.pdf}

\end{solutionbox}
\begin{mnemonicbox}
``ILRFS'' (Inner, Left, Right, Full, Self)

\end{mnemonicbox}
\subsection*{પ્રશ્ન 5(અ) [3
માર્ક્સ]}\label{uxaaauxab0uxab6uxaa8-5uxa85-3-uxaaeuxab0uxa95uxab8}

\textbf{નીચે આપેલા customer relation ને 1NF માં બદલાવો.}

\textbf{Customer}

{\def\LTcaptype{none} % do not increment counter
\begin{longtable}[]{@{}llll@{}}
\toprule\noalign{}
cid & name & address & Contact\_no \\
\midrule\noalign{}
\endhead
\bottomrule\noalign{}
\endlastfoot
CO1 & Riya & Amu aavas, Anand & \{5322332123\} \\
CO2 & Jiya & Sardar colony, Ahmedabad & \{5326521456, 5265232849\} \\
\end{longtable}
}

\begin{solutionbox}

\textbf{Customer Table (1NF):}

{\def\LTcaptype{none} % do not increment counter
\begin{longtable}[]{@{}lllll@{}}
\toprule\noalign{}
cid & name & society & city & Contact\_no \\
\midrule\noalign{}
\endhead
\bottomrule\noalign{}
\endlastfoot
CO1 & Riya & Amu aavas & Anand & 5322332123 \\
CO2 & Jiya & Sardar colony & Ahmedabad & 5326521456 \\
CO2 & Jiya & Sardar colony & Ahmedabad & 5265232849 \\
\end{longtable}
}

\end{solutionbox}
\begin{mnemonicbox}
``AFM'' (Atomic values, Flatten Multivalued
attributes)

\end{mnemonicbox}
\subsection*{પ્રશ્ન 5(બ) [4
માર્ક્સ]}\label{uxaaauxab0uxab6uxaa8-5uxaac-4-uxaaeuxab0uxa95uxab8}

\textbf{ટ્રાન્સેક્શન ની ACID properties ની યાદી બનાવો અને સમજાવો.}

\begin{solutionbox}

{\def\LTcaptype{none} % do not increment counter
\begin{longtable}[]{@{}ll@{}}
\toprule\noalign{}
ACID Property & વર્ણન \\
\midrule\noalign{}
\endhead
\bottomrule\noalign{}
\endlastfoot
\textbf{Atomicity} & ટ્રાન્સેક્શન સંપૂર્ણપણે ચાલે છે અથવા બિલકુલ નહીં \\
\textbf{Consistency} & ડેટાબેઝ ટ્રાન્સેક્શન પહેલાં અને પછી સુસંગત રહે છે \\
\textbf{Isolation} & સમાંતર ટ્રાન્સેક્શન એકબીજા સાથે દખલ કરતા નથી \\
\textbf{Durability} & કમિટેડ ફેરફારો સિસ્ટમ નિષ્ફળતા પછી પણ કાયમી રહે છે \\
\end{longtable}
}

\textbf{ડાયાગ્રામ:}

\includegraphics[width=1\linewidth,height=\textheight,keepaspectratio]{mermaid-34e3cf0a.pdf}

\end{solutionbox}
\begin{mnemonicbox}
``ACID'' (Atomicity, Consistency, Isolation,
Durability)

\end{mnemonicbox}
\subsection*{પ્રશ્ન 5(ક) [7
માર્ક્સ]}\label{uxaaauxab0uxab6uxaa8-5uxa95-7-uxaaeuxab0uxa95uxab8}

\textbf{વિવિધ functional dependencies ની યાદી બનાવો અને દરેક ને ઉદાહરણ સાથે
સમજાવો.}

\begin{solutionbox}

{\def\LTcaptype{none} % do not increment counter
\begin{longtable}[]{@{}
  >{\raggedright\arraybackslash}p{(\linewidth - 4\tabcolsep) * \real{0.5000}}
  >{\raggedright\arraybackslash}p{(\linewidth - 4\tabcolsep) * \real{0.2955}}
  >{\raggedright\arraybackslash}p{(\linewidth - 4\tabcolsep) * \real{0.2045}}@{}}
\toprule\noalign{}
\begin{minipage}[b]{\linewidth}\raggedright
Functional Dependency
\end{minipage} & \begin{minipage}[b]{\linewidth}\raggedright
વર્ણન
\end{minipage} & \begin{minipage}[b]{\linewidth}\raggedright
ઉદાહરણ
\end{minipage} \\
\midrule\noalign{}
\endhead
\bottomrule\noalign{}
\endlastfoot
\textbf{Trivial FD} & X \rightarrow Y જ્યાં Y એ X નો સબસેટ છે & \{StudentID, Name\} \rightarrow
\{Name\} \\
\textbf{Non-trivial FD} & X \rightarrow Y જ્યાં Y એ X નો સબસેટ નથી & \{StudentID\} \rightarrow
\{Name\} \\
\textbf{Partial FD} & કમ્પોઝિટ કી નો ભાગ નોન-કી એટ્રિબ્યુટ નક્કી કરે છે &
\{CourseID, StudentID\} \rightarrow \{CourseName\} \\
\textbf{Transitive FD} & X \rightarrow Y અને Y \rightarrow Z એટલે X \rightarrow Z & \{StudentID\} \rightarrow
\{DeptID\} અને \{DeptID\} \rightarrow \{DeptName\} \\
\textbf{Multivalued FD} & એક એટ્રિબ્યુટ બીજા એટ્રિબ્યુટના મૂલ્યોનો સેટ નક્કી કરે છે
& \{CourseID\} \rightarrow\rightarrow \{TextbookID\} \\
\end{longtable}
}

\textbf{ડાયાગ્રામ:}

\includegraphics[width=1\linewidth,height=\textheight,keepaspectratio]{mermaid-40533685.pdf}

\end{solutionbox}
\begin{mnemonicbox}
``TNPTMv'' (Trivial, Non-trivial, Partial,
Transitive, Multivalued)

\end{mnemonicbox}
\subsection*{પ્રશ્ન 5(અ) OR [3
માર્ક્સ]}\label{uxaaauxab0uxab6uxaa8-5uxa85-or-3-uxaaeuxab0uxa95uxab8}

\textbf{નીચે આપેલા Depositor\_Account relation ને 2NF માં બદલાવો.}
\textbf{જ્યાં functional dependencies(FD) નીચે મુજબ છે.} \textbf{FD1: \{cid,
ano\} \rightarrow \{access\_date, balance, bname\}} \textbf{FD2: ano \rightarrow \{balance,
bname\}}

\textbf{Depositor\_Account}

{\def\LTcaptype{none} % do not increment counter
\begin{longtable}[]{@{}lllll@{}}
\toprule\noalign{}
cid & ano & access\_date & balance & bname \\
\midrule\noalign{}
\endhead
\bottomrule\noalign{}
\endlastfoot
\end{longtable}
}

\begin{solutionbox}

\textbf{Account Table (2NF):}

{\def\LTcaptype{none} % do not increment counter
\begin{longtable}[]{@{}lll@{}}
\toprule\noalign{}
ano & balance & bname \\
\midrule\noalign{}
\endhead
\bottomrule\noalign{}
\endlastfoot
\end{longtable}
}

\textbf{Depositor Table (2NF):}

{\def\LTcaptype{none} % do not increment counter
\begin{longtable}[]{@{}lll@{}}
\toprule\noalign{}
cid & ano & access\_date \\
\midrule\noalign{}
\endhead
\bottomrule\noalign{}
\endlastfoot
\end{longtable}
}

\end{solutionbox}
\begin{mnemonicbox}
``RPKD'' (Remove Partial Key Dependencies)

\end{mnemonicbox}
\subsection*{પ્રશ્ન 5(બ) OR [4
માર્ક્સ]}\label{uxaaauxab0uxab6uxaa8-5uxaac-or-4-uxaaeuxab0uxa95uxab8}

\textbf{Conflict serializability સમજાવો.}

\begin{solutionbox}

{\def\LTcaptype{none} % do not increment counter
\begin{longtable}[]{@{}
  >{\raggedright\arraybackslash}p{(\linewidth - 2\tabcolsep) * \real{0.4091}}
  >{\raggedright\arraybackslash}p{(\linewidth - 2\tabcolsep) * \real{0.5909}}@{}}
\toprule\noalign{}
\begin{minipage}[b]{\linewidth}\raggedright
કન્સેપ્ટ
\end{minipage} & \begin{minipage}[b]{\linewidth}\raggedright
વર્ણન
\end{minipage} \\
\midrule\noalign{}
\endhead
\bottomrule\noalign{}
\endlastfoot
\textbf{વ્યાખ્યા} & સિરિયલ શેડ્યૂલ સાથે સમકક્ષ હોય તો શેડ્યૂલ કન્ફ્લિક્ટ સીરિયલાઇઝેબલ
છે \\
\textbf{કન્ફ્લિક્ટ ઓપરેશન્સ} & એક જ ડેટા આઇટમ પર રીડ-રાઇટ, રાઇટ-રીડ, રાઇટ-રાઇટ
ઓપરેશન્સ \\
\textbf{કન્ફ્લિક્ટ ગ્રાફ} & ટ્રાન્સેક્શન વચ્ચેના કન્ફ્લિક્ટ દર્શાવતો ડાયરેક્ટેડ ગ્રાફ \\
\textbf{ટેસ્ટિંગ} & જો કન્ફ્લિક્ટ ગ્રાફમાં ચક્ર ન હોય તો શેડ્યૂલ કન્ફ્લિક્ટ સીરિયલાઇઝેબલ
છે \\
\end{longtable}
}

\textbf{ડાયાગ્રામ:}

\includegraphics[width=1\linewidth,height=\textheight,keepaspectratio]{mermaid-94ceb202.pdf}

\end{solutionbox}
\begin{mnemonicbox}
``COGS'' (Conflict Operations Graph Serializable)

\end{mnemonicbox}
\subsection*{પ્રશ્ન 5(ક) OR [7
માર્ક્સ]}\label{uxaaauxab0uxab6uxaa8-5uxa95-or-7-uxaaeuxab0uxa95uxab8}

\textbf{ઉદાહરણ સાથે 3NF normalization સમજાવો.}

\begin{solutionbox}

{\def\LTcaptype{none} % do not increment counter
\begin{longtable}[]{@{}
  >{\raggedright\arraybackslash}p{(\linewidth - 4\tabcolsep) * \real{0.3824}}
  >{\raggedright\arraybackslash}p{(\linewidth - 4\tabcolsep) * \real{0.3529}}
  >{\raggedright\arraybackslash}p{(\linewidth - 4\tabcolsep) * \real{0.2647}}@{}}
\toprule\noalign{}
\begin{minipage}[b]{\linewidth}\raggedright
Normal Form
\end{minipage} & \begin{minipage}[b]{\linewidth}\raggedright
વ્યાખ્યા
\end{minipage} & \begin{minipage}[b]{\linewidth}\raggedright
ઉદાહરણ
\end{minipage} \\
\midrule\noalign{}
\endhead
\bottomrule\noalign{}
\endlastfoot
\textbf{1NF} & એટોમિક વેલ્યુ, કોઈ રિપીટિંગ ગ્રુપ નહીં & Student(ID, Name,
Phone1, Phone2) \rightarrow Student(ID, Name, Phone) \\
\textbf{2NF} & 1NF + કોઈ પાર્શિયલ ડિપેન્ડન્સી નહીં & Order(OrderID,
ProductID, CustomerID, ProductName) \rightarrow Order(OrderID, ProductID,
CustomerID) + Product(ProductID, ProductName) \\
\textbf{3NF} & 2NF + કોઈ ટ્રાન્ઝિટિવ ડિપેન્ડન્સી નહીં & Student(ID, DeptID,
DeptName) \rightarrow Student(ID, DeptID) + Department(DeptID, DeptName) \\
\end{longtable}
}

\textbf{ઉલ્લંઘન ઉદાહરણ:}

\begin{lstlisting}
Employee(EmpID, EmpName, DeptID, DeptName, Location)
\end{lstlisting}

\textbf{3NF રૂપાંતરણ:}

\begin{lstlisting}
Employee(EmpID, EmpName, DeptID)
Department(DeptID, DeptName, Location)
\end{lstlisting}

\textbf{ડાયાગ્રામ:}

\includegraphics[width=1\linewidth,height=\textheight,keepaspectratio]{mermaid-e23f2722.pdf}

\end{solutionbox}
\begin{mnemonicbox}
``APTN'' (Atomic values, Partial dependencies
removed, Transitive dependencies removed, Normalized)

\end{mnemonicbox}

\end{document}
