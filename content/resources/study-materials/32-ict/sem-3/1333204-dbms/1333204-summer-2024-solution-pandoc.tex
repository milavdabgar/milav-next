\documentclass[10pt,a4paper]{article}

% content/resources/templates/preamble.tex
\usepackage[margin=0.6in]{geometry}
\author{Milav Dabgar}
\usepackage{amsmath,amssymb,amsthm}
\usepackage{booktabs}
\usepackage{multirow}
\usepackage{xcolor}
\usepackage{tcolorbox}
\tcbuselibrary{breakable,skins}
\usepackage[colorlinks=true,linkcolor=blue]{hyperref}
\usepackage{titlesec}
\usepackage{enumitem}
\usepackage{tikz}
\usepackage{pgfplots}
\usepackage{circuitikz}
\usepackage[version=4]{mhchem}
\usepackage{longtable}
\usepackage{array}
\usepackage{float}
\usepackage{caption}
\usepackage{listings}

\lstset{
  basicstyle=\small\ttfamily,
  breaklines=true,
  breakatwhitespace=false,
  postbreak=\mbox{\textcolor{red}{$\hookrightarrow$}\space},
  float=false,
  numbers=left,
  numberstyle=\tiny\color{gray},
  numbersep=10pt,
  xleftmargin=2em,
  keywordstyle=\color{blue},
  commentstyle=\color{green!60!black},
  stringstyle=\color{purple},
  backgroundcolor=\color{gray!5},
  showstringspaces=false,
  tabsize=2,
  captionpos=b,
  keepspaces=true,
  columns=flexible
}

\pgfplotsset{compat=1.18}
\usetikzlibrary{shapes,arrows,positioning,calc,patterns,decorations.pathmorphing,decorations.markings,arrows.meta}

% Color scheme
\definecolor{headcolor}{RGB}{0,102,204}
\definecolor{keycolor}{RGB}{220,20,60}
\definecolor{solutioncolor}{RGB}{34,139,34}
\definecolor{mnemoniccolor}{RGB}{148,0,211}
\definecolor{codecolor}{RGB}{0,0,100}

% Spacing
\setlength{\parskip}{3pt}
\setlist[itemize]{nosep}
\setlist[enumerate]{nosep}

% Title formatting
\titleformat{\section}{\Large\bfseries\color{headcolor}}{\thesection}{1em}{}
\titleformat{\subsection}{\large\bfseries\color{headcolor}}{\thesubsection}{1em}{}

% Pandoc tightlist compatibility
\providecommand{\tightlist}{%
  \setlength{\itemsep}{0pt}\setlength{\parskip}{0pt}}

% Pandoc longtable compatibility
\newcounter{none}
\def\thenone{}


% content/resources/templates/english-boxes.tex
% This file is currently empty - it exists to maintain consistency with the import structure.
% Add custom environments here if needed in the future.


\begin{document}

\begin{center}
{\Huge\bfseries\color{headcolor} Subject Name Solutions}\\[5pt]
{\LARGE 1333204 -- Summer 2024}\\[3pt]
{\large Semester 1 Study Material}\\[3pt]
{\normalsize\textit{Detailed Solutions and Explanations}}
\end{center}

\vspace{10pt}

\subsection*{Question 1(a) [3 marks]}\label{q1a}

\textbf{Define: DBMS, Instance, Metadata}

\begin{solutionbox}

\begin{itemize}
\tightlist
\item
  \textbf{DBMS (Database Management System)}: Software that enables
  users to create, maintain, and access databases by controlling data
  organization, storage, retrieval, security, and integrity.
\item
  \textbf{Instance}: The actual data stored in a database at a
  particular moment in time. It's the current state or snapshot of a
  database.
\item
  \textbf{Metadata}: Data about data that describes database structure,
  including tables, fields, relationships, constraints, and indexes.
\end{itemize}

\end{solutionbox}
\begin{mnemonicbox}
``DIM view'' - Database system, Instance snapshot,
Metadata description

\end{mnemonicbox}
\subsection*{Question 1(b) [4 marks]}\label{q1b}

\textbf{Define and Explain with example: 1.Entity 2. Attribute}

\begin{solutionbox}


{\def\LTcaptype{none} % do not increment counter
\vspace{-5pt}
\captionof{table}{Entity vs Attribute}
\vspace{-10pt}
\begin{longtable}[]{@{}
  >{\raggedright\arraybackslash}p{(\linewidth - 4\tabcolsep) * \real{0.3000}}
  >{\raggedright\arraybackslash}p{(\linewidth - 4\tabcolsep) * \real{0.4000}}
  >{\raggedright\arraybackslash}p{(\linewidth - 4\tabcolsep) * \real{0.3000}}@{}}
\toprule\noalign{}
\begin{minipage}[b]{\linewidth}\raggedright
Concept
\end{minipage} & \begin{minipage}[b]{\linewidth}\raggedright
Definition
\end{minipage} & \begin{minipage}[b]{\linewidth}\raggedright
Example
\end{minipage} \\
\midrule\noalign{}
\endhead
\bottomrule\noalign{}
\endlastfoot
Entity & A real-world object or concept that can be distinctly
identified & Student (John), Book (Harry Potter), Car (Toyota Camry) \\
Attribute & Characteristic or property that describes an entity &
Student: roll\_no, name, addressBook: ISBN, title, author \\
\end{longtable}
}

\textbf{Diagram:}

\includegraphics[width=1\linewidth,height=\textheight,keepaspectratio]{mermaid-5505ae51.pdf}

\end{solutionbox}
\begin{mnemonicbox}
``EA-PC'' - Entities Are Physical/Conceptual,
Attributes Provide Characteristics

\end{mnemonicbox}
\subsection*{Question 1(c) [7 marks]}\label{q1c}

\textbf{Write the full form of DBA. Explain the roles and
responsibilities of DBA.}

\begin{solutionbox}

DBA stands for \textbf{Database Administrator}.


{\def\LTcaptype{none} % do not increment counter
\vspace{-5pt}
\captionof{table}{DBA Responsibilities}
\vspace{-10pt}
\begin{longtable}[]{@{}
  >{\raggedright\arraybackslash}p{(\linewidth - 2\tabcolsep) * \real{0.3158}}
  >{\raggedright\arraybackslash}p{(\linewidth - 2\tabcolsep) * \real{0.6842}}@{}}
\toprule\noalign{}
\begin{minipage}[b]{\linewidth}\raggedright
Role
\end{minipage} & \begin{minipage}[b]{\linewidth}\raggedright
Description
\end{minipage} \\
\midrule\noalign{}
\endhead
\bottomrule\noalign{}
\endlastfoot
Database Design & Creates logical/physical database structure and
schema \\
Security Management & Controls access through user accounts and
permissions \\
Performance Tuning & Optimizes queries, indexes for faster data
retrieval \\
Backup \& Recovery & Implements strategies to prevent data loss \\
Maintenance & Updates software, applies patches, monitors space \\
\end{longtable}
}

\textbf{Diagram:}

\includegraphics[width=1\linewidth,height=\textheight,keepaspectratio]{mermaid-d985bc2f.pdf}

\end{solutionbox}
\begin{mnemonicbox}
``SPMBU'' - Security, Performance, Maintenance,
Backup, Updates

\end{mnemonicbox}
\subsection*{Question 1(c) OR [7
marks]}\label{q1c}

\textbf{Explain relational and network data models in detail.}

\begin{solutionbox}


{\def\LTcaptype{none} % do not increment counter
\vspace{-5pt}
\captionof{table}{Relational vs Network Data Models}
\vspace{-10pt}
\begin{longtable}[]{@{}
  >{\raggedright\arraybackslash}p{(\linewidth - 4\tabcolsep) * \real{0.2143}}
  >{\raggedright\arraybackslash}p{(\linewidth - 4\tabcolsep) * \real{0.4286}}
  >{\raggedright\arraybackslash}p{(\linewidth - 4\tabcolsep) * \real{0.3571}}@{}}
\toprule\noalign{}
\begin{minipage}[b]{\linewidth}\raggedright
Feature
\end{minipage} & \begin{minipage}[b]{\linewidth}\raggedright
Relational Model
\end{minipage} & \begin{minipage}[b]{\linewidth}\raggedright
Network Model
\end{minipage} \\
\midrule\noalign{}
\endhead
\bottomrule\noalign{}
\endlastfoot
Structure & Tables (relations) with rows and columns & Records connected
by pointers forming complex networks \\
Relationship & Related through primary \& foreign keys & Direct links
between parent-child records \\
Flexibility & High - tables can be joined as needed & Limited -
predefined physical connections \\
Examples & MySQL, Oracle, SQL Server & IDS, IDMS \\
Query Language & SQL (structured query language) & Procedural
languages \\
\end{longtable}
}

\textbf{Diagram:}

\includegraphics[width=1\linewidth,height=\textheight,keepaspectratio]{mermaid-9ea2a7a5.pdf}

\end{solutionbox}
\begin{mnemonicbox}
``RSPEN'' - Relational uses Sets, Pointers Enable
Networks

\end{mnemonicbox}
\subsection*{Question 2(a) [3 marks]}\label{q2a}

\textbf{Draw figure and Explain Generalization.}

\begin{solutionbox}

\textbf{Generalization}: The process of extracting common
characteristics from two or more entities to create a new higher-level
entity.

\textbf{Diagram:}

\includegraphics[width=1\linewidth,height=\textheight,keepaspectratio]{mermaid-afff1363.pdf}

\end{solutionbox}
\begin{mnemonicbox}
``BUSH'' - Bottom-Up Shared Hierarchy

\end{mnemonicbox}
\subsection*{Question 2(b) [4 marks]}\label{q2b}

\textbf{Explain Primary Key and Foreign Key Constraints.}

\begin{solutionbox}


{\def\LTcaptype{none} % do not increment counter
\vspace{-5pt}
\captionof{table}{Primary Key vs Foreign Key}
\vspace{-10pt}
\begin{longtable}[]{@{}
  >{\raggedright\arraybackslash}p{(\linewidth - 6\tabcolsep) * \real{0.2667}}
  >{\raggedright\arraybackslash}p{(\linewidth - 6\tabcolsep) * \real{0.2667}}
  >{\raggedright\arraybackslash}p{(\linewidth - 6\tabcolsep) * \real{0.2667}}
  >{\raggedright\arraybackslash}p{(\linewidth - 6\tabcolsep) * \real{0.2000}}@{}}
\toprule\noalign{}
\begin{minipage}[b]{\linewidth}\raggedright
Constraint
\end{minipage} & \begin{minipage}[b]{\linewidth}\raggedright
Definition
\end{minipage} & \begin{minipage}[b]{\linewidth}\raggedright
Properties
\end{minipage} & \begin{minipage}[b]{\linewidth}\raggedright
Example
\end{minipage} \\
\midrule\noalign{}
\endhead
\bottomrule\noalign{}
\endlastfoot
Primary Key & Uniquely identifies each record in a table & Unique, Not
Null, Only one per table & StudentID in Students table \\
Foreign Key & Links data between tables, references a primary key in
another table & Can be NULL, Multiple allowed per table & DeptID in
Employees table referencing Departments table \\
\end{longtable}
}

\textbf{Diagram:}

\includegraphics[width=1\linewidth,height=\textheight,keepaspectratio]{mermaid-66b27e40.pdf}

\end{solutionbox}
\begin{mnemonicbox}
``PURE FIRE'' - Primary Uniquely References Entities,
Foreign Imports Referenced Entities

\end{mnemonicbox}
\subsection*{Question 2(c) [7 marks]}\label{q2c}

\textbf{Construct an E-R diagram for Hospital Management System.}

\begin{solutionbox}

\textbf{E-R Diagram for Hospital Management System:}

\includegraphics[width=1\linewidth,height=\textheight,keepaspectratio]{mermaid-a9a1b3c0.pdf}

\end{solutionbox}
\begin{mnemonicbox}
``PADRE'' - Patients Appointments Doctors Rooms
Entities

\end{mnemonicbox}
\subsection*{Question 2(a) OR [3
marks]}\label{q2a}

\textbf{Draw figure and Explain Specialization.}

\begin{solutionbox}

\textbf{Specialization}: The process of creating new entities from an
existing entity by adding unique attributes to distinguish them.

\textbf{Diagram:}

\includegraphics[width=1\linewidth,height=\textheight,keepaspectratio]{mermaid-0edd51bf.pdf}

\end{solutionbox}
\begin{mnemonicbox}
``TDSB'' - Top-Down Specialized Breakdown

\end{mnemonicbox}
\subsection*{Question 2(b) OR [4
marks]}\label{q2b}

\textbf{Explain single valued v/s multi-valued attributes with suitable
examples.}

\begin{solutionbox}


{\def\LTcaptype{none} % do not increment counter
\vspace{-5pt}
\captionof{table}{Single-valued vs Multi-valued Attributes}
\vspace{-10pt}
\begin{longtable}[]{@{}
  >{\raggedright\arraybackslash}p{(\linewidth - 6\tabcolsep) * \real{0.1395}}
  >{\raggedright\arraybackslash}p{(\linewidth - 6\tabcolsep) * \real{0.2791}}
  >{\raggedright\arraybackslash}p{(\linewidth - 6\tabcolsep) * \real{0.2093}}
  >{\raggedright\arraybackslash}p{(\linewidth - 6\tabcolsep) * \real{0.3721}}@{}}
\toprule\noalign{}
\begin{minipage}[b]{\linewidth}\raggedright
Type
\end{minipage} & \begin{minipage}[b]{\linewidth}\raggedright
Definition
\end{minipage} & \begin{minipage}[b]{\linewidth}\raggedright
Example
\end{minipage} & \begin{minipage}[b]{\linewidth}\raggedright
Implementation
\end{minipage} \\
\midrule\noalign{}
\endhead
\bottomrule\noalign{}
\endlastfoot
Single-valued & Contains only one value for each entity instance &
Person's birth date, SSN & Directly stored in table columns \\
Multi-valued & Can have multiple values for the same entity & Person's
skills, phone numbers & Separate table or specialized formats \\
\end{longtable}
}

\textbf{Diagram:}

\includegraphics[width=1\linewidth,height=\textheight,keepaspectratio]{mermaid-ca937045.pdf}

\end{solutionbox}
\begin{mnemonicbox}
``SOME'' - Single One, Multiple Entries

\end{mnemonicbox}
\subsection*{Question 2(c) OR [7
marks]}\label{q2c}

\textbf{Construct an E-R diagram for Banking Management System.}

\begin{solutionbox}

\textbf{E-R Diagram for Banking Management System:}

\includegraphics[width=1\linewidth,height=\textheight,keepaspectratio]{mermaid-706328e6.pdf}

\end{solutionbox}
\begin{mnemonicbox}
``CABLE'' - Customers Accounts Branches Loans
Employees

\end{mnemonicbox}
\subsection*{Question 3(a) [3 marks]}\label{q3a}

\textbf{Explain WHERE and DESC clause with example.}

\begin{solutionbox}


{\def\LTcaptype{none} % do not increment counter
\vspace{-5pt}
\captionof{table}{WHERE and DESC Clauses}
\vspace{-10pt}
\begin{longtable}[]{@{}
  >{\raggedright\arraybackslash}p{(\linewidth - 6\tabcolsep) * \real{0.2353}}
  >{\raggedright\arraybackslash}p{(\linewidth - 6\tabcolsep) * \real{0.2647}}
  >{\raggedright\arraybackslash}p{(\linewidth - 6\tabcolsep) * \real{0.2353}}
  >{\raggedright\arraybackslash}p{(\linewidth - 6\tabcolsep) * \real{0.2647}}@{}}
\toprule\noalign{}
\begin{minipage}[b]{\linewidth}\raggedright
Clause
\end{minipage} & \begin{minipage}[b]{\linewidth}\raggedright
Purpose
\end{minipage} & \begin{minipage}[b]{\linewidth}\raggedright
Syntax
\end{minipage} & \begin{minipage}[b]{\linewidth}\raggedright
Example
\end{minipage} \\
\midrule\noalign{}
\endhead
\bottomrule\noalign{}
\endlastfoot
WHERE & Filters rows based on specified condition & SELECT columns FROM
table WHERE condition & SELECT * FROM employees WHERE salary
\textgreater{} 50000 \\
DESC & Sorts results in descending order & SELECT columns FROM table
ORDER BY column DESC & SELECT * FROM products ORDER BY price DESC \\
\end{longtable}
}

\textbf{Diagram:}

\begin{lstlisting}[language=SQL]
-- Original data in Students table
| ID | Name   | Marks |
|----|--------|-------|
| 1  | Alice  | 85    |
| 2  | Bob    | 92    |
| 3  | Carol  | 78    |
| 4  | David  | 65    |

-- Using WHERE: SELECT * FROM Students WHERE Marks > 80
| ID | Name   | Marks |
|----|--------|-------|
| 1  | Alice  | 85    |
| 2  | Bob    | 92    |

-- Using DESC: SELECT * FROM Students ORDER BY Marks DESC
| ID | Name   | Marks |
|----|--------|-------|
| 2  | Bob    | 92    |
| 1  | Alice  | 85    |
| 3  | Carol  | 78    |
| 4  | David  | 65    |
\end{lstlisting}

\end{solutionbox}
\begin{mnemonicbox}
``WDF'' - Where filters Data, DESC orders
First-highest

\end{mnemonicbox}
\subsection*{Question 3(b) [4 marks]}\label{q3b}

\textbf{List DDL commands. Explain any two DDL commands with examples.}

\begin{solutionbox}

\textbf{DDL (Data Definition Language) Commands:}

\begin{enumerate}
\tightlist
\item
  CREATE
\item
  ALTER
\item
  DROP
\item
  TRUNCATE
\item
  RENAME
\end{enumerate}


{\def\LTcaptype{none} % do not increment counter
\vspace{-5pt}
\captionof{table}{CREATE and ALTER Commands}
\vspace{-10pt}
\begin{longtable}[]{@{}
  >{\raggedright\arraybackslash}p{(\linewidth - 6\tabcolsep) * \real{0.2571}}
  >{\raggedright\arraybackslash}p{(\linewidth - 6\tabcolsep) * \real{0.2571}}
  >{\raggedright\arraybackslash}p{(\linewidth - 6\tabcolsep) * \real{0.2286}}
  >{\raggedright\arraybackslash}p{(\linewidth - 6\tabcolsep) * \real{0.2571}}@{}}
\toprule\noalign{}
\begin{minipage}[b]{\linewidth}\raggedright
Command
\end{minipage} & \begin{minipage}[b]{\linewidth}\raggedright
Purpose
\end{minipage} & \begin{minipage}[b]{\linewidth}\raggedright
Syntax
\end{minipage} & \begin{minipage}[b]{\linewidth}\raggedright
Example
\end{minipage} \\
\midrule\noalign{}
\endhead
\bottomrule\noalign{}
\endlastfoot
CREATE & Creates database objects like tables, views, indexes & CREATE
TABLE table\_name (column definitions) & CREATE TABLE students (id INT
PRIMARY KEY, name VARCHAR(50)) \\
ALTER & Modifies structure of existing database objects & ALTER TABLE
table\_name action & ALTER TABLE students ADD COLUMN email
VARCHAR(100) \\
\end{longtable}
}

\textbf{CodeBlock:}

\begin{lstlisting}[language=SQL]
-- CREATE example
CREATE TABLE employees (
    emp_id INT PRIMARY KEY,
    name VARCHAR(50) NOT NULL,
    dept VARCHAR(30),
    salary DECIMAL(10,2)
);

-- ALTER example
ALTER TABLE employees 
ADD COLUMN hire_date DATE;
\end{lstlisting}

\end{solutionbox}
\begin{mnemonicbox}
``CADTR'' - Create Alter Drop Truncate Rename

\end{mnemonicbox}
\subsection*{Question 3(c) [7 marks]}\label{q3c}

\textbf{Perform the following Query on the table ``Company'' having the
field's eno, ename, salary, dept in SQL.} \textbf{1. Display all records
in Company table.} \textbf{2. Display only dept without duplicate
value.} \textbf{3. Display all records sorted in descending order of
ename.} \textbf{4. Add one new column ``cityname'' to store city.}
\textbf{5. Display name of all employees who do not stay in city
``Mumbai''.} \textbf{6. Delete all employees having salary less than
10,000.} \textbf{7. Display the employee names starts with ``A''.}

\begin{solutionbox}

\textbf{CodeBlock:}

\begin{lstlisting}[language=SQL]
-- 1. Display all records in Company table
SELECT * FROM Company;

-- 2. Display only dept without duplicate value
SELECT DISTINCT dept FROM Company;

-- 3. Display all records sorted in descending order of ename
SELECT * FROM Company ORDER BY ename DESC;

-- 4. Add one new column "cityname" to store city
ALTER TABLE Company ADD COLUMN cityname VARCHAR(50);

-- 5. Display name of all employees who do not stay in city "Mumbai"
SELECT ename FROM Company WHERE cityname != 'Mumbai';

-- 6. Delete all employees having salary less than 10,000
DELETE FROM Company WHERE salary < 10000;

-- 7. Display the employee names starts with "A"
SELECT ename FROM Company WHERE ename LIKE 'A%';
\end{lstlisting}


{\def\LTcaptype{none} % do not increment counter
\vspace{-5pt}
\captionof{table}{SQL Operations}
\vspace{-10pt}
\begin{longtable}[]{@{}lll@{}}
\toprule\noalign{}
Operation & SQL Command & Purpose \\
\midrule\noalign{}
\endhead
\bottomrule\noalign{}
\endlastfoot
SELECT & SELECT * FROM Company & Retrieve all data \\
DISTINCT & SELECT DISTINCT dept & Remove duplicates \\
ORDER BY & ORDER BY ename DESC & Sort in descending \\
ALTER & ALTER TABLE ADD COLUMN & Add new column \\
WHERE & WHERE cityname != `Mumbai' & Filter condition \\
DELETE & DELETE FROM WHERE & Remove records \\
LIKE & WHERE ename LIKE `A\%' & Pattern matching \\
\end{longtable}
}

\end{solutionbox}
\begin{mnemonicbox}
``SODA-WDL'' - Select Order Distinct Alter - Where
Delete Like

\end{mnemonicbox}
\subsection*{Question 3(a) OR [3
marks]}\label{q3a}

\textbf{Explain SELECT and DISTINCT clause with example.}

\begin{solutionbox}


{\def\LTcaptype{none} % do not increment counter
\vspace{-5pt}
\captionof{table}{SELECT and DISTINCT Clauses}
\vspace{-10pt}
\begin{longtable}[]{@{}
  >{\raggedright\arraybackslash}p{(\linewidth - 6\tabcolsep) * \real{0.2353}}
  >{\raggedright\arraybackslash}p{(\linewidth - 6\tabcolsep) * \real{0.2647}}
  >{\raggedright\arraybackslash}p{(\linewidth - 6\tabcolsep) * \real{0.2353}}
  >{\raggedright\arraybackslash}p{(\linewidth - 6\tabcolsep) * \real{0.2647}}@{}}
\toprule\noalign{}
\begin{minipage}[b]{\linewidth}\raggedright
Clause
\end{minipage} & \begin{minipage}[b]{\linewidth}\raggedright
Purpose
\end{minipage} & \begin{minipage}[b]{\linewidth}\raggedright
Syntax
\end{minipage} & \begin{minipage}[b]{\linewidth}\raggedright
Example
\end{minipage} \\
\midrule\noalign{}
\endhead
\bottomrule\noalign{}
\endlastfoot
SELECT & Retrieves data from database & SELECT columns FROM table &
SELECT name, age FROM students \\
DISTINCT & Eliminates duplicate values & SELECT DISTINCT columns FROM
table & SELECT DISTINCT department FROM employees \\
\end{longtable}
}

\textbf{Diagram:}

\begin{lstlisting}[language=SQL]
-- Original data in Departments table
| dept_id | dept_name |
|---------|-----------|
| 1       | Sales     |
| 2       | IT        |
| 3       | HR        |
| 4       | IT        |
| 5       | Sales     |

-- Using SELECT: SELECT dept_name FROM Departments
| dept_name |
|-----------|
| Sales     |
| IT        |
| HR        |
| IT        |
| Sales     |

-- Using DISTINCT: SELECT DISTINCT dept_name FROM Departments
| dept_name |
|-----------|
| Sales     |
| IT        |
| HR        |
\end{lstlisting}

\end{solutionbox}
\begin{mnemonicbox}
``SUD'' - Select Unique with Distinct

\end{mnemonicbox}
\subsection*{Question 3(b) OR [4
marks]}\label{q3b}

\textbf{List DML commands. Explain any two DML commands with examples.}

\begin{solutionbox}

\textbf{DML (Data Manipulation Language) Commands:}

\begin{enumerate}
\tightlist
\item
  INSERT
\item
  UPDATE
\item
  DELETE
\item
  SELECT
\end{enumerate}


{\def\LTcaptype{none} % do not increment counter
\vspace{-5pt}
\captionof{table}{INSERT and UPDATE Commands}
\vspace{-10pt}
\begin{longtable}[]{@{}
  >{\raggedright\arraybackslash}p{(\linewidth - 6\tabcolsep) * \real{0.2571}}
  >{\raggedright\arraybackslash}p{(\linewidth - 6\tabcolsep) * \real{0.2571}}
  >{\raggedright\arraybackslash}p{(\linewidth - 6\tabcolsep) * \real{0.2286}}
  >{\raggedright\arraybackslash}p{(\linewidth - 6\tabcolsep) * \real{0.2571}}@{}}
\toprule\noalign{}
\begin{minipage}[b]{\linewidth}\raggedright
Command
\end{minipage} & \begin{minipage}[b]{\linewidth}\raggedright
Purpose
\end{minipage} & \begin{minipage}[b]{\linewidth}\raggedright
Syntax
\end{minipage} & \begin{minipage}[b]{\linewidth}\raggedright
Example
\end{minipage} \\
\midrule\noalign{}
\endhead
\bottomrule\noalign{}
\endlastfoot
INSERT & Adds new records to a table & INSERT INTO table\_name VALUES
(values) & INSERT INTO students VALUES (1, `John', 85) \\
UPDATE & Modifies existing records & UPDATE table\_name SET column=value
WHERE condition & UPDATE students SET marks=90 WHERE id=1 \\
\end{longtable}
}

\textbf{CodeBlock:}

\begin{lstlisting}[language=SQL]
-- INSERT example
INSERT INTO employees (emp_id, name, dept, salary)
VALUES (101, 'John Smith', 'IT', 65000);

-- UPDATE example
UPDATE employees 
SET salary = 70000 
WHERE emp_id = 101;
\end{lstlisting}

\end{solutionbox}
\begin{mnemonicbox}
``IUDS'' - Insert Update Delete Select

\end{mnemonicbox}
\subsection*{Question 3(c) OR [7
marks]}\label{q3c}

\textbf{Write the Output of Following Query.} \textbf{1.
ABS(-34),ABS(16)} \textbf{2. SQRT(16),SQRT(64)} \textbf{3. POWER(5,2),
POWER(2,4)} \textbf{4. MOD(15,3), MOD(13,3)} \textbf{5.
ROUND(123.456,1), ROUND(123.456,2)} \textbf{6. CEIL(122.6),
CEIL(-122.6)} \textbf{7. FLOOR(-157.5),FLOOR(157.5)}

\begin{solutionbox}


{\def\LTcaptype{none} % do not increment counter
\vspace{-5pt}
\captionof{table}{SQL Function Outputs}
\vspace{-10pt}
\begin{longtable}[]{@{}
  >{\raggedright\arraybackslash}p{(\linewidth - 4\tabcolsep) * \real{0.3226}}
  >{\raggedright\arraybackslash}p{(\linewidth - 4\tabcolsep) * \real{0.4194}}
  >{\raggedright\arraybackslash}p{(\linewidth - 4\tabcolsep) * \real{0.2581}}@{}}
\toprule\noalign{}
\begin{minipage}[b]{\linewidth}\raggedright
Function
\end{minipage} & \begin{minipage}[b]{\linewidth}\raggedright
Description
\end{minipage} & \begin{minipage}[b]{\linewidth}\raggedright
Output
\end{minipage} \\
\midrule\noalign{}
\endhead
\bottomrule\noalign{}
\endlastfoot
ABS(-34),ABS(16) & Absolute value & 34, 16 \\
SQRT(16),SQRT(64) & Square root & 4, 8 \\
POWER(5,2), POWER(2,4) & Power function & 25, 16 \\
MOD(15,3), MOD(13,3) & Modulus (remainder) & 0, 1 \\
ROUND(123.456,1), ROUND(123.456,2) & Round to decimal places & 123.5,
123.46 \\
CEIL(122.6), CEIL(-122.6) & Round up to nearest integer & 123, -122 \\
FLOOR(-157.5),FLOOR(157.5) & Round down to nearest integer & -158,
157 \\
\end{longtable}
}

\textbf{Diagram:}

\includegraphics[width=1\linewidth,height=\textheight,keepaspectratio]{mermaid-56899f02.pdf}

\end{solutionbox}
\begin{mnemonicbox}
``ASPRCF'' - Absolute Square Power Remainder Ceiling
Floor

\end{mnemonicbox}
\subsection*{Question 4(a) [3 marks]}\label{q4a}

\textbf{List data types in SQL. Explain 1.VARCHAR() and 2.INT() data
types with example.}

\begin{solutionbox}

\textbf{SQL Data Types Categories:}

\begin{enumerate}
\tightlist
\item
  Numeric (INT, FLOAT, DECIMAL)
\item
  Character (CHAR, VARCHAR)
\item
  Date/Time (DATE, TIME, DATETIME)
\item
  Binary (BLOB, BINARY)
\item
  Boolean (BOOL)
\end{enumerate}


{\def\LTcaptype{none} % do not increment counter
\vspace{-5pt}
\captionof{table}{VARCHAR and INT Data Types}
\vspace{-10pt}
\begin{longtable}[]{@{}
  >{\raggedright\arraybackslash}p{(\linewidth - 6\tabcolsep) * \real{0.2821}}
  >{\raggedright\arraybackslash}p{(\linewidth - 6\tabcolsep) * \real{0.3333}}
  >{\raggedright\arraybackslash}p{(\linewidth - 6\tabcolsep) * \real{0.1538}}
  >{\raggedright\arraybackslash}p{(\linewidth - 6\tabcolsep) * \real{0.2308}}@{}}
\toprule\noalign{}
\begin{minipage}[b]{\linewidth}\raggedright
Data Type
\end{minipage} & \begin{minipage}[b]{\linewidth}\raggedright
Description
\end{minipage} & \begin{minipage}[b]{\linewidth}\raggedright
Size
\end{minipage} & \begin{minipage}[b]{\linewidth}\raggedright
Example
\end{minipage} \\
\midrule\noalign{}
\endhead
\bottomrule\noalign{}
\endlastfoot
VARCHAR(n) & Variable-length character string & Up to n characters, only
uses needed space & VARCHAR(50) for names, emails \\
INT & Integer numeric data & Usually 4 bytes, -2,147,483,648 to
2,147,483,647 & INT for IDs, counts, ages \\
\end{longtable}
}

\textbf{CodeBlock:}

\begin{lstlisting}[language=SQL]
CREATE TABLE students (
    student_id INT PRIMARY KEY,
    name VARCHAR(50) NOT NULL,
    age INT,
    email VARCHAR(100)
);
\end{lstlisting}

\end{solutionbox}
\begin{mnemonicbox}
``VIA'' - Variable strings, Integers for Ages

\end{mnemonicbox}
\subsection*{Question 4(b) [4 marks]}\label{q4b}

\textbf{Explain 2NF (Second Normal Form) with example and solution.}

\begin{solutionbox}

\textbf{2NF Definition}: A relation is in 2NF if it is in 1NF and no
non-prime attribute is dependent on any proper subset of any candidate
key.


{\def\LTcaptype{none} % do not increment counter
\vspace{-5pt}
\captionof{table}{Before 2NF}
\vspace{-10pt}
\begin{longtable}[]{@{}llll@{}}
\toprule\noalign{}
student\_id & course\_id & course\_name & instructor \\
\midrule\noalign{}
\endhead
\bottomrule\noalign{}
\endlastfoot
S1 & C1 & Database & Prof.~Smith \\
S1 & C2 & Networking & Prof.~Jones \\
S2 & C1 & Database & Prof.~Smith \\
S3 & C3 & Programming & Prof.~Wilson \\
\end{longtable}
}

\textbf{Problem}: Non-prime attributes (course\_name, instructor) depend
only on course\_id, not the entire key (student\_id, course\_id).

\textbf{Diagram: 2NF Solution}

\includegraphics[width=1\linewidth,height=\textheight,keepaspectratio]{mermaid-b97101c6.pdf}


{\def\LTcaptype{none} % do not increment counter
\vspace{-5pt}
\captionof{table}{After 2NF}
\vspace{-10pt}
\begin{longtable}[]{@{}ll@{}}
\toprule\noalign{}
student\_id & course\_id \\
\midrule\noalign{}
\endhead
\bottomrule\noalign{}
\endlastfoot
S1 & C1 \\
S1 & C2 \\
S2 & C1 \\
S3 & C3 \\
\end{longtable}
}

Course Table:

{\def\LTcaptype{none} % do not increment counter
\begin{longtable}[]{@{}lll@{}}
\toprule\noalign{}
course\_id & course\_name & instructor \\
\midrule\noalign{}
\endhead
\bottomrule\noalign{}
\endlastfoot
C1 & Database & Prof.~Smith \\
C2 & Networking & Prof.~Jones \\
C3 & Programming & Prof.~Wilson \\
\end{longtable}
}

\end{solutionbox}
\begin{mnemonicbox}
``PFPK'' - Partial Functional dependency on Primary
Key

\end{mnemonicbox}
\subsection*{Question 4(c) [7 marks]}\label{q4c}

\textbf{Explain function dependency. Explain Partial function dependency
with example.}

\begin{solutionbox}

\textbf{Functional Dependency}: Relationship between attributes where
one attribute's value determines another attribute's value.

\textbf{Notation}: X \rightarrow Y (X determines Y)

\textbf{Partial Functional Dependency}: When a non-prime attribute
depends on part of a composite key rather than the whole key.


{\def\LTcaptype{none} % do not increment counter
\vspace{-5pt}
\captionof{table}{Order Details (Before Normalization)}
\vspace{-10pt}
\begin{longtable}[]{@{}lllll@{}}
\toprule\noalign{}
order\_id & product\_id & quantity & product\_name & price \\
\midrule\noalign{}
\endhead
\bottomrule\noalign{}
\endlastfoot
O1 & P1 & 5 & Keyboard & 50 \\
O1 & P2 & 2 & Mouse & 25 \\
O2 & P1 & 1 & Keyboard & 50 \\
O3 & P3 & 3 & Monitor & 200 \\
\end{longtable}
}

\textbf{Functional Dependencies:}

\begin{itemize}
\tightlist
\item
  (order\_id, product\_id) \rightarrow quantity
\item
  product\_id \rightarrow product\_name
\item
  product\_id \rightarrow price
\end{itemize}

\textbf{Diagram:}

\includegraphics[width=1\linewidth,height=\textheight,keepaspectratio]{mermaid-9a463cc3.pdf}

\textbf{Solution (Normalized Tables):} Orders Table:

{\def\LTcaptype{none} % do not increment counter
\begin{longtable}[]{@{}lll@{}}
\toprule\noalign{}
order\_id & product\_id & quantity \\
\midrule\noalign{}
\endhead
\bottomrule\noalign{}
\endlastfoot
O1 & P1 & 5 \\
O1 & P2 & 2 \\
O2 & P1 & 1 \\
O3 & P3 & 3 \\
\end{longtable}
}

Products Table:

{\def\LTcaptype{none} % do not increment counter
\begin{longtable}[]{@{}lll@{}}
\toprule\noalign{}
product\_id & product\_name & price \\
\midrule\noalign{}
\endhead
\bottomrule\noalign{}
\endlastfoot
P1 & Keyboard & 50 \\
P2 & Mouse & 25 \\
P3 & Monitor & 200 \\
\end{longtable}
}

\end{solutionbox}
\begin{mnemonicbox}
``PDPK'' - Partial Dependency on Part of Key

\end{mnemonicbox}
\subsection*{Question 4(a) OR [3
marks]}\label{q4a}

\textbf{Explain commands: 1) To\_Char() 2) To\_Date()}

\begin{solutionbox}


{\def\LTcaptype{none} % do not increment counter
\vspace{-5pt}
\captionof{table}{Conversion Functions}
\vspace{-10pt}
\begin{longtable}[]{@{}
  >{\raggedright\arraybackslash}p{(\linewidth - 6\tabcolsep) * \real{0.2778}}
  >{\raggedright\arraybackslash}p{(\linewidth - 6\tabcolsep) * \real{0.2500}}
  >{\raggedright\arraybackslash}p{(\linewidth - 6\tabcolsep) * \real{0.2222}}
  >{\raggedright\arraybackslash}p{(\linewidth - 6\tabcolsep) * \real{0.2500}}@{}}
\toprule\noalign{}
\begin{minipage}[b]{\linewidth}\raggedright
Function
\end{minipage} & \begin{minipage}[b]{\linewidth}\raggedright
Purpose
\end{minipage} & \begin{minipage}[b]{\linewidth}\raggedright
Syntax
\end{minipage} & \begin{minipage}[b]{\linewidth}\raggedright
Example
\end{minipage} \\
\midrule\noalign{}
\endhead
\bottomrule\noalign{}
\endlastfoot
TO\_CHAR() & Converts date/number to character string using format model
& TO\_CHAR(value, [format]) & TO\_CHAR(SYSDATE, `DD-MON-YYYY') \rightarrow
`14-JUN-2024' \\
TO\_DATE() & Converts character string to date using format model &
TO\_DATE(string, [format]) & TO\_DATE(`14-JUN-2024', `DD-MON-YYYY')
\rightarrow date value \\
\end{longtable}
}

\textbf{CodeBlock:}

\begin{lstlisting}[language=SQL]
-- TO_CHAR examples
SELECT TO_CHAR(SYSDATE, 'DD-MON-YYYY') FROM DUAL;  -- '14-JUN-2024'
SELECT TO_CHAR(1234.56, '$9,999.99') FROM DUAL;    -- '$1,234.56'

-- TO_DATE examples
SELECT TO_DATE('2024-06-14', 'YYYY-MM-DD') FROM DUAL;
SELECT TO_DATE('14/06/24', 'DD/MM/YY') FROM DUAL;
\end{lstlisting}

\end{solutionbox}
\begin{mnemonicbox}
``DCS'' - Date Conversion Strings

\end{mnemonicbox}
\subsection*{Question 4(b) OR [4
marks]}\label{q4b}

\textbf{Explain Full function dependency with example.}

\begin{solutionbox}

\textbf{Full Functional Dependency}: When an attribute is functionally
dependent on a composite key, and dependent on the entire key, not just
part of it.


{\def\LTcaptype{none} % do not increment counter
\vspace{-5pt}
\captionof{table}{Exam Results}
\vspace{-10pt}
\begin{longtable}[]{@{}llll@{}}
\toprule\noalign{}
student\_id & course\_id & exam\_date & score \\
\midrule\noalign{}
\endhead
\bottomrule\noalign{}
\endlastfoot
S1 & C1 & 2024-05-10 & 85 \\
S1 & C2 & 2024-05-15 & 92 \\
S2 & C1 & 2024-05-10 & 78 \\
S2 & C2 & 2024-05-15 & 88 \\
\end{longtable}
}

\textbf{Full Functional Dependency:}

\begin{itemize}
\tightlist
\item
  (student\_id, course\_id) \rightarrow score (score depends on both student and
  course)
\end{itemize}

\textbf{Diagram:}

\includegraphics[width=1\linewidth,height=\textheight,keepaspectratio]{mermaid-1f4e5b38.pdf}

\textbf{Explanation}: The score attribute fully depends on the composite
key (student\_id, course\_id) because:

\begin{itemize}
\tightlist
\item
  Different students can have different scores for the same course
\item
  Same student can have different scores for different courses
\item
  We need both student\_id and course\_id to determine a specific score
\end{itemize}

\end{solutionbox}
\begin{mnemonicbox}
``FCEK'' - Fully dependent on Complete/Entire Key

\end{mnemonicbox}
\subsection*{Question 4(c) OR [7
marks]}\label{q4c}

\textbf{Define normalization. Explain 1NF (First Normal Form) with
example and solution.}

\begin{solutionbox}

\textbf{Normalization}: Process of organizing data to minimize
redundancy, improve data integrity, and eliminate anomalies by dividing
larger tables into smaller related tables.

\textbf{1NF Definition}: A relation is in 1NF if all attributes contain
atomic (indivisible) values only.


{\def\LTcaptype{none} % do not increment counter
\vspace{-5pt}
\captionof{table}{Before 1NF}
\vspace{-10pt}
\begin{longtable}[]{@{}lll@{}}
\toprule\noalign{}
student\_id & name & courses \\
\midrule\noalign{}
\endhead
\bottomrule\noalign{}
\endlastfoot
S1 & John & Math, Physics \\
S2 & Mary & Chemistry, Biology, Physics \\
S3 & Tim & Computer Science \\
\end{longtable}
}

\textbf{Problems}:

\begin{itemize}
\tightlist
\item
  Non-atomic values (multiple courses per cell)
\item
  Cannot easily query or update specific courses
\end{itemize}

\textbf{Diagram:}

\includegraphics[width=1\linewidth,height=\textheight,keepaspectratio]{mermaid-47b128a6.pdf}


{\def\LTcaptype{none} % do not increment counter
\vspace{-5pt}
\captionof{table}{After 1NF}
\vspace{-10pt}
\begin{longtable}[]{@{}lll@{}}
\toprule\noalign{}
student\_id & name & course \\
\midrule\noalign{}
\endhead
\bottomrule\noalign{}
\endlastfoot
S1 & John & Math \\
S1 & John & Physics \\
S2 & Mary & Chemistry \\
S2 & Mary & Biology \\
S2 & Mary & Physics \\
S3 & Tim & Computer Science \\
\end{longtable}
}

\end{solutionbox}
\begin{mnemonicbox}
``ASAV'' - Atomic Single-value Attributes only Valid

\end{mnemonicbox}
\subsection*{Question 5(a) [3 marks]}\label{q5a}

\textbf{Explain the concept of Transaction with example.}

\begin{solutionbox}

\textbf{Transaction}: A logical unit of work that must be either
completely executed or completely undone.


{\def\LTcaptype{none} % do not increment counter
\vspace{-5pt}
\captionof{table}{Transaction Properties}
\vspace{-10pt}
\begin{longtable}[]{@{}
  >{\raggedright\arraybackslash}p{(\linewidth - 2\tabcolsep) * \real{0.4348}}
  >{\raggedright\arraybackslash}p{(\linewidth - 2\tabcolsep) * \real{0.5652}}@{}}
\toprule\noalign{}
\begin{minipage}[b]{\linewidth}\raggedright
Property
\end{minipage} & \begin{minipage}[b]{\linewidth}\raggedright
Description
\end{minipage} \\
\midrule\noalign{}
\endhead
\bottomrule\noalign{}
\endlastfoot
Atomicity & All operations complete successfully or none do \\
Consistency & Database remains in consistent state before and after
transaction \\
Isolation & Concurrent transactions don't interfere with each other \\
Durability & Completed transactions persist even after system
failures \\
\end{longtable}
}

\textbf{Example:}

\begin{lstlisting}[language=SQL]
-- Bank Account Transfer Transaction
BEGIN TRANSACTION;
    -- Deduct $500 from Account A
    UPDATE accounts SET balance = balance - 500 WHERE account_id = 'A';
    
    -- Add $500 to Account B
    UPDATE accounts SET balance = balance + 500 WHERE account_id = 'B';
    
    -- If both operations successful
    COMMIT;
    -- If any operation fails
    -- ROLLBACK;
END TRANSACTION;
\end{lstlisting}

\end{solutionbox}
\begin{mnemonicbox}
``ACID'' - Atomicity Consistency Isolation Durability

\end{mnemonicbox}
\subsection*{Question 5(b) [4 marks]}\label{q5b}

\textbf{Explain equi join with syntax and example.}

\begin{solutionbox}

\textbf{Equi Join}: A join that uses equality comparison operator to
match records from two or more tables based on a common field.

\textbf{Syntax:}

\begin{lstlisting}[language=SQL]
SELECT columns
FROM table1, table2 
WHERE table1.column = table2.column;

-- Alternative syntax (explicit JOIN)
SELECT columns
FROM table1 JOIN table2
ON table1.column = table2.column;
\end{lstlisting}

\textbf{Table Example:} Employees Table:

{\def\LTcaptype{none} % do not increment counter
\begin{longtable}[]{@{}lll@{}}
\toprule\noalign{}
emp\_id & name & dept\_id \\
\midrule\noalign{}
\endhead
\bottomrule\noalign{}
\endlastfoot
101 & Alice & 1 \\
102 & Bob & 2 \\
103 & Carol & 1 \\
\end{longtable}
}

Departments Table:

{\def\LTcaptype{none} % do not increment counter
\begin{longtable}[]{@{}lll@{}}
\toprule\noalign{}
dept\_id & dept\_name & location \\
\midrule\noalign{}
\endhead
\bottomrule\noalign{}
\endlastfoot
1 & HR & New York \\
2 & IT & Chicago \\
3 & Finance & Boston \\
\end{longtable}
}

\textbf{CodeBlock:}

\begin{lstlisting}[language=SQL]
-- Equi Join Example
SELECT e.name, d.dept_name, d.location
FROM employees e, departments d
WHERE e.dept_id = d.dept_id;
\end{lstlisting}

\textbf{Result:}

{\def\LTcaptype{none} % do not increment counter
\begin{longtable}[]{@{}lll@{}}
\toprule\noalign{}
name & dept\_name & location \\
\midrule\noalign{}
\endhead
\bottomrule\noalign{}
\endlastfoot
Alice & HR & New York \\
Bob & IT & Chicago \\
Carol & HR & New York \\
\end{longtable}
}

\textbf{Diagram:}

\includegraphics[width=1\linewidth,height=\textheight,keepaspectratio]{mermaid-34d30bff.pdf}

\end{solutionbox}
\begin{mnemonicbox}
``MEET'' - Match Equal Elements Every Table

\end{mnemonicbox}
\subsection*{Question 5(c) [7 marks]}\label{q5c}

\textbf{Explain Conflict Serializability in detail.}

\begin{solutionbox}

\textbf{Conflict Serializability}: A way to ensure correctness of
concurrent transactions by guaranteeing that the execution schedule is
equivalent to some serial execution.


{\def\LTcaptype{none} % do not increment counter
\vspace{-5pt}
\captionof{table}{Key Concepts in Conflict Serializability}
\vspace{-10pt}
\begin{longtable}[]{@{}
  >{\raggedright\arraybackslash}p{(\linewidth - 2\tabcolsep) * \real{0.4091}}
  >{\raggedright\arraybackslash}p{(\linewidth - 2\tabcolsep) * \real{0.5909}}@{}}
\toprule\noalign{}
\begin{minipage}[b]{\linewidth}\raggedright
Concept
\end{minipage} & \begin{minipage}[b]{\linewidth}\raggedright
Description
\end{minipage} \\
\midrule\noalign{}
\endhead
\bottomrule\noalign{}
\endlastfoot
Conflicting Operations & Two operations conflict if they access same
data item and at least one is a write \\
Precedence Graph & Directed graph showing conflicts between
transactions \\
Conflict Serializable & Schedule is conflict serializable if its
precedence graph is acyclic \\
\end{longtable}
}

\textbf{Diagram:}

\includegraphics[width=1\linewidth,height=\textheight,keepaspectratio]{mermaid-30cacd38.pdf}

\textbf{Example:} Consider transactions T1 and T2:

\begin{itemize}
\tightlist
\item
  T1: Read(A), Write(A)
\item
  T2: Read(A), Write(A)
\end{itemize}

Schedule S1: R1(A), W1(A), R2(A), W2(A) - Serializable (equivalent to
T1\rightarrowT2) Schedule S2: R1(A), R2(A), W1(A), W2(A) - Not serializable
(contains cycle in precedence graph)

\textbf{Steps to Determine Conflict Serializability:}

\begin{enumerate}
\tightlist
\item
  Identify all pairs of conflicting operations
\item
  Construct the precedence graph
\item
  Check if the graph has cycles
\item
  If no cycles, the schedule is conflict serializable
\end{enumerate}

\end{solutionbox}
\begin{mnemonicbox}
``COPS'' - Conflicts, Operations, Precedence,
Serializability

\end{mnemonicbox}
\subsection*{Question 5(a) OR [3
marks]}\label{q5a}

\textbf{Explain the properties of Transaction with example.}

\begin{solutionbox}

\textbf{ACID Properties of Transactions:}


{\def\LTcaptype{none} % do not increment counter
\vspace{-5pt}
\captionof{table}{ACID Properties}
\vspace{-10pt}
\begin{longtable}[]{@{}
  >{\raggedright\arraybackslash}p{(\linewidth - 4\tabcolsep) * \real{0.3125}}
  >{\raggedright\arraybackslash}p{(\linewidth - 4\tabcolsep) * \real{0.4062}}
  >{\raggedright\arraybackslash}p{(\linewidth - 4\tabcolsep) * \real{0.2812}}@{}}
\toprule\noalign{}
\begin{minipage}[b]{\linewidth}\raggedright
Property
\end{minipage} & \begin{minipage}[b]{\linewidth}\raggedright
Description
\end{minipage} & \begin{minipage}[b]{\linewidth}\raggedright
Example
\end{minipage} \\
\midrule\noalign{}
\endhead
\bottomrule\noalign{}
\endlastfoot
Atomicity & All operations complete successfully or none do & Bank
transfer - both debit and credit must succeed or fail together \\
Consistency & Database must be in a consistent state before and after
transaction & After transferring \$100, total money in system remains
unchanged \\
Isolation & Concurrent transactions don't interfere with each other &
Transaction A doesn't see partial results of Transaction B \\
Durability & Once committed, changes are permanent & Power failure won't
cause committed transaction to be lost \\
\end{longtable}
}

\textbf{Diagram:}

\includegraphics[width=1\linewidth,height=\textheight,keepaspectratio]{mermaid-c9565f1e.pdf}

\textbf{Example:}

\begin{lstlisting}[language=SQL]
-- ATM Withdrawal Transaction
BEGIN TRANSACTION;
    -- Check balance
    SELECT balance FROM accounts WHERE account_id = 'A123';
    
    -- If sufficient, update balance
    UPDATE accounts SET balance = balance - 100 WHERE account_id = 'A123';
    
    -- Record the withdrawal
    INSERT INTO transactions (account_id, type, amount, date)
    VALUES ('A123', 'WITHDRAWAL', 100, SYSDATE);
    
    -- If all operations successful
    COMMIT;
    -- If any operation fails
    -- ROLLBACK;
END TRANSACTION;
\end{lstlisting}

\end{solutionbox}
\begin{mnemonicbox}
``ACID'' - Atomicity Consistency Isolation Durability

\end{mnemonicbox}
\subsection*{Question 5(b) OR [4
marks]}\label{q5b}

\textbf{Write the Queries using set operators to find following using
given ``Faculty'' and ``CT'' tables.} \textbf{1. List the name of the
persons who are either a Faculty or a CT.} \textbf{2. List the name of
the persons who are a Faculty as well as a CT.} \textbf{3. List the name
of the persons who are only a Faculty and not a CT.} \textbf{4. List the
name of the persons who are only a CT and not a Faculty.}

\begin{solutionbox}

\textbf{Table Data:} Faculty Table:

{\def\LTcaptype{none} % do not increment counter
\begin{longtable}[]{@{}lll@{}}
\toprule\noalign{}
FacultyName & ErNo & Dept \\
\midrule\noalign{}
\endhead
\bottomrule\noalign{}
\endlastfoot
Prakash & FC01 & ICT \\
Ronak & FC02 & IT \\
Rakesh & FC03 & EC \\
Kinjal & FC04 & ICT \\
\end{longtable}
}

CT (Class Teacher) Table:

{\def\LTcaptype{none} % do not increment counter
\begin{longtable}[]{@{}ll@{}}
\toprule\noalign{}
Dept & CTName \\
\midrule\noalign{}
\endhead
\bottomrule\noalign{}
\endlastfoot
EC & Rakesh \\
CE & Jigar \\
ICT & Prakash \\
IT & Gunjan \\
\end{longtable}
}

\textbf{CodeBlock:}

\begin{lstlisting}[language=SQL]
-- 1. List the name of the persons who are either a Faculty or a CT
SELECT FacultyName AS Name FROM Faculty
UNION
SELECT CTName AS Name FROM CT;

-- 2. List the name of the persons who are a Faculty as well as a CT
SELECT FacultyName AS Name FROM Faculty
INTERSECT
SELECT CTName AS Name FROM CT;

-- 3. List the name of the persons who are only a Faculty and not a CT
SELECT FacultyName AS Name FROM Faculty
MINUS
SELECT CTName AS Name FROM CT;

-- 4. List the name of the persons who are only a CT and not a Faculty
SELECT CTName AS Name FROM CT
MINUS
SELECT FacultyName AS Name FROM Faculty;
\end{lstlisting}

\textbf{Diagram:}

\includegraphics[width=1\linewidth,height=\textheight,keepaspectratio]{mermaid-f99540d1.pdf}

\textbf{Results:}

\begin{enumerate}
\tightlist
\item
  UNION: Prakash, Ronak, Rakesh, Kinjal, Jigar, Gunjan
\item
  INTERSECT: Prakash, Rakesh
\item
  MINUS (Faculty - CT): Ronak, Kinjal
\item
  MINUS (CT - Faculty): Jigar, Gunjan
\end{enumerate}

\end{solutionbox}
\begin{mnemonicbox}
``UIMM'' - Union Intersect Minus Minus

\end{mnemonicbox}
\subsection*{Question 5(c) OR [7
marks]}\label{q5c}

\textbf{Explain View Serializability in detail.}

\begin{solutionbox}

\textbf{View Serializability}: A schedule is view serializable if it is
view equivalent to some serial schedule, meaning it produces the same
``view'' (or final state) of the database.


{\def\LTcaptype{none} % do not increment counter
\vspace{-5pt}
\captionof{table}{Comparison with Conflict Serializability}
\vspace{-10pt}
\begin{longtable}[]{@{}
  >{\raggedright\arraybackslash}p{(\linewidth - 4\tabcolsep) * \real{0.1455}}
  >{\raggedright\arraybackslash}p{(\linewidth - 4\tabcolsep) * \real{0.3818}}
  >{\raggedright\arraybackslash}p{(\linewidth - 4\tabcolsep) * \real{0.4727}}@{}}
\toprule\noalign{}
\begin{minipage}[b]{\linewidth}\raggedright
Aspect
\end{minipage} & \begin{minipage}[b]{\linewidth}\raggedright
View Serializability
\end{minipage} & \begin{minipage}[b]{\linewidth}\raggedright
Conflict Serializability
\end{minipage} \\
\midrule\noalign{}
\endhead
\bottomrule\noalign{}
\endlastfoot
Definition & Based on the final results of reads and writes & Based on
conflicts between operations \\
Condition & Preserves initial read, final write, and read-write
dependency & Preserves all conflicts between operations \\
Scope & Broader class of schedules & Subset of view serializable
schedules \\
Testing & More complex to test & Can test with precedence graph \\
\end{longtable}
}

\textbf{Diagram:}

\includegraphics[width=1\linewidth,height=\textheight,keepaspectratio]{mermaid-61cc6cc6.pdf}

\textbf{View Equivalence Conditions:}

\begin{enumerate}
\tightlist
\item
  Initial Reads: If T1 reads an initial value of data item A in schedule
  S1, it must also read the initial value in S2.
\item
  Final Writes: If T1 performs the final write on data item A in S1, it
  must also perform the final write in S2.
\item
  Read-Write Dependency: If T1 reads a value of A written by T2 in S1,
  it must also read the value written by T2 in S2.
\end{enumerate}

\textbf{Example of View Serializable but not Conflict Serializable
Schedule:} Consider transactions with blind writes (writes without
reading):

\begin{itemize}
\tightlist
\item
  T1: W1(A)
\item
  T2: W2(A)
\end{itemize}

Schedule S: W1(A), W2(A) - View serializable to both T1\rightarrowT2 and T2\rightarrowT1
(final write is always T2) But W1(A) and W2(A) conflict, so a conflict
graph would have an edge in both directions.

\end{solutionbox}
\begin{mnemonicbox}
``IRF'' - Initial reads, Result writes, Final view

\end{mnemonicbox}

\end{document}
