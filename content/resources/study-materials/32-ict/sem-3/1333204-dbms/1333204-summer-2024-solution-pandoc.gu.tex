\documentclass[10pt,a4paper]{article}

% content/resources/templates/preamble.tex
\usepackage[margin=0.6in]{geometry}
\author{Milav Dabgar}
\usepackage{amsmath,amssymb,amsthm}
\usepackage{booktabs}
\usepackage{multirow}
\usepackage{xcolor}
\usepackage{tcolorbox}
\tcbuselibrary{breakable,skins}
\usepackage[colorlinks=true,linkcolor=blue]{hyperref}
\usepackage{titlesec}
\usepackage{enumitem}
\usepackage{tikz}
\usepackage{pgfplots}
\usepackage{circuitikz}
\usepackage[version=4]{mhchem}
\usepackage{longtable}
\usepackage{array}
\usepackage{float}
\usepackage{caption}
\usepackage{listings}

\lstset{
  basicstyle=\small\ttfamily,
  breaklines=true,
  breakatwhitespace=false,
  postbreak=\mbox{\textcolor{red}{$\hookrightarrow$}\space},
  float=false,
  numbers=left,
  numberstyle=\tiny\color{gray},
  numbersep=10pt,
  xleftmargin=2em,
  keywordstyle=\color{blue},
  commentstyle=\color{green!60!black},
  stringstyle=\color{purple},
  backgroundcolor=\color{gray!5},
  showstringspaces=false,
  tabsize=2,
  captionpos=b,
  keepspaces=true,
  columns=flexible
}

\pgfplotsset{compat=1.18}
\usetikzlibrary{shapes,arrows,positioning,calc,patterns,decorations.pathmorphing,decorations.markings,arrows.meta}

% Color scheme
\definecolor{headcolor}{RGB}{0,102,204}
\definecolor{keycolor}{RGB}{220,20,60}
\definecolor{solutioncolor}{RGB}{34,139,34}
\definecolor{mnemoniccolor}{RGB}{148,0,211}
\definecolor{codecolor}{RGB}{0,0,100}

% Spacing
\setlength{\parskip}{3pt}
\setlist[itemize]{nosep}
\setlist[enumerate]{nosep}

% Title formatting
\titleformat{\section}{\Large\bfseries\color{headcolor}}{\thesection}{1em}{}
\titleformat{\subsection}{\large\bfseries\color{headcolor}}{\thesubsection}{1em}{}

% Pandoc tightlist compatibility
\providecommand{\tightlist}{%
  \setlength{\itemsep}{0pt}\setlength{\parskip}{0pt}}

% Pandoc longtable compatibility
\newcounter{none}
\def\thenone{}


% content/resources/templates/gujarati-boxes.tex
\usepackage{fontspec}
\usepackage{polyglossia}

% Set Gujarati as main language (document is primarily in Gujarati)
% Note: gloss-gujarati.ldf doesn't exist in polyglossia, but it will use hyphenation patterns
\setdefaultlanguage{gujarati}
\setotherlanguage{english}

% Configure Gujarati font properly
% Use Language=Default to prevent polyglossia from trying to add language-specific features
% that don't exist for Gujarati, which causes "empty feature" warnings
\newfontfamily\gujaratifont[Script=Gujarati,AutoFakeBold=2.5,AutoFakeSlant=0.3]{Noto Sans Gujarati}
\setmainfont[Script=Gujarati,AutoFakeBold=2.5,AutoFakeSlant=0.3]{Noto Sans Gujarati}
% Use Noto Sans Gujarati for monospace to support Gujarati in text
\setmonofont[Scale=0.9]{Noto Sans Gujarati}

% Configure English to use the same font
\newfontfamily\englishfont[Script=Gujarati,AutoFakeBold=2.5,AutoFakeSlant=0.3]{Noto Sans Gujarati}

% Translations for polyglossia
\gappto\captionsgujarati{
  \renewcommand{\tablename}{કોષ્ટક}
  \renewcommand{\figurename}{આકૃતિ}
}

% Helper for TikZ nodes to ensure Gujarati font
\newcommand{\gu}[1]{{\gujaratifont #1}}

% Custom environments
\newtcolorbox{solutionbox}{
    breakable,
    enhanced,
    colback=solutioncolor!5!white,
    colframe=solutioncolor!75!black,
    fonttitle=\bfseries,
    title=જવાબ
}

\newtcolorbox{solutionboxnobreak}{
 colback=solutioncolor!5!white,
 colframe=solutioncolor!75!black,
 fonttitle=\bfseries,
 title=જવાબ
}

\newtcolorbox{keyformula}{
 breakable,
 enhanced,
 colback=keycolor!5!white,
 colframe=keycolor!75!black,
 fonttitle=\bfseries,
 title=રાસાયણિક સમીકરણ/સૂત્ર
}

\newtcolorbox{mnemonicbox}{
 breakable,
 enhanced,
 colback=mnemoniccolor!5!white,
 colframe=mnemoniccolor!75!black,
 fonttitle=\bfseries,
 title=મેમરી ટ્રીક
}


\begin{document}

\begin{center}
{\Huge\bfseries\color{headcolor} Subject Name (Gujarati)}\\[5pt]
{\LARGE 1333204 -- Summer 2024}\\[3pt]
{\large Semester 1 Study Material}\\[3pt]
{\normalsize\textit{Detailed Solutions and Explanations}}
\end{center}

\vspace{10pt}

\subsection*{પ્રશ્ન 1(અ) [3
માર્ક્સ]}\label{uxaaauxab0uxab6uxaa8-1uxa85-3-uxaaeuxab0uxa95uxab8}

\textbf{વ્યાખ્યા આપો: DBMS, ઈન્સટન્સ, મેટાડેટા}

\begin{solutionbox}

\begin{itemize}
\tightlist
\item
  \textbf{DBMS (ડેટાબેઝ મેનેજમેન્ટ સિસ્ટમ)}: એક સોફ્ટવેર જે વપરાશકર્તાઓને ડેટાબેઝ
  બનાવવા, જાળવવા, અને ઍક્સેસ કરવા સક્ષમ બનાવે છે. જે ડેટા ઑર્ગેનાઈઝેશન, સ્ટોરેજ,
  પુનઃપ્રાપ્તિ, સુરક્ષા, અને અખંડતાનું નિયંત્રણ કરે છે.
\item
  \textbf{ઈન્સટન્સ}: કોઈ ચોક્કસ સમયે ડેટાબેઝમાં સંગ્રહિત વાસ્તવિક ડેટા. તે ડેટાબેઝની
  વર્તમાન સ્થિતિ અથવા સ્નેપશોટ છે.
\item
  \textbf{મેટાડેટા}: ડેટા વિશેનો ડેટા, જે ડેટાબેઝ સ્ટ્રક્ચરનું વર્ણન કરે છે, જેમાં ટેબલ્સ,
  ફીલ્ડ્સ, સંબંધો, કન્સ્ટ્રેઈન્ટ્સ, અને ઇન્ડેક્સનો સમાવેશ થાય છે.
\end{itemize}

\end{solutionbox}
\begin{mnemonicbox}
``DIM દૃશ્ય'' - ડેટાબેઝ સિસ્ટમ, ઈન્સટન્સ સ્નેપશોટ, મેટાડેટા
વર્ણન

\end{mnemonicbox}
\subsection*{પ્રશ્ન 1(બ) [4
માર્ક્સ]}\label{uxaaauxab0uxab6uxaa8-1uxaac-4-uxaaeuxab0uxa95uxab8}

\textbf{વ્યાખ્યા આપો અને ઉદાહરણ સાથે સમજાવો: 1.Entity 2. Attribute}

\begin{solutionbox}


{\def\LTcaptype{none} % do not increment counter
\vspace{-5pt}
\captionof{table}{Entity અને Attribute વચ્ચેનો તફાવત}
\vspace{-10pt}
\begin{longtable}[]{@{}
  >{\raggedright\arraybackslash}p{(\linewidth - 4\tabcolsep) * \real{0.3000}}
  >{\raggedright\arraybackslash}p{(\linewidth - 4\tabcolsep) * \real{0.4000}}
  >{\raggedright\arraybackslash}p{(\linewidth - 4\tabcolsep) * \real{0.3000}}@{}}
\toprule\noalign{}
\begin{minipage}[b]{\linewidth}\raggedright
કોન્સેપ્ટ
\end{minipage} & \begin{minipage}[b]{\linewidth}\raggedright
વ્યાખ્યા
\end{minipage} & \begin{minipage}[b]{\linewidth}\raggedright
ઉદાહરણ
\end{minipage} \\
\midrule\noalign{}
\endhead
\bottomrule\noalign{}
\endlastfoot
એન્ટિટી & એક વાસ્તવિક દુનિયાની વસ્તુ અથવા ખ્યાલ જેને સ્પષ્ટપણે ઓળખી શકાય છે &
વિદ્યાર્થી (જોન), પુસ્તક (હેરી પોટર), કાર (ટોયોટા કેમરી) \\
એટ્રિબ્યુટ & એક લક્ષણ અથવા ગુણધર્મ જે એન્ટિટીનું વર્ણન કરે છે & વિદ્યાર્થી: રોલ\_નં,
નામ, સરનામુંપુસ્તક: ISBN, શીર્ષક, લેખક \\
\end{longtable}
}

\textbf{આકૃતિ:}

\begin{verbatim}
erDiagram
    STUDENT \{
        int student\_id
        string name
        string address
    \}
    BOOK \{
        string ISBN
        string title
        string author
    \}
\end{verbatim}

\end{solutionbox}
\begin{mnemonicbox}
``EA-PC'' - એન્ટિટીઝ આર ફિઝિકલ/કોન્સેપ્ચ્યુઅલ, એટ્રિબ્યુટ્સ
પ્રોવાઇડ કેરેક્ટરિસ્ટિક્સ

\end{mnemonicbox}
\subsection*{પ્રશ્ન 1(ક) [7
માર્ક્સ]}\label{uxaaauxab0uxab6uxaa8-1uxa95-7-uxaaeuxab0uxa95uxab8}

\textbf{DBA નું પૂર્ણ નામ લખો. DBAની ભૂમિકા અને જવાબદારીઓ સમજાવો.}

\begin{solutionbox}

DBA એટલે \textbf{ડેટાબેઝ એડમિનિસ્ટ્રેટર}.


{\def\LTcaptype{none} % do not increment counter
\vspace{-5pt}
\captionof{table}{DBA જવાબદારીઓ}
\vspace{-10pt}
\begin{longtable}[]{@{}ll@{}}
\toprule\noalign{}
ભૂમિકા & વર્ણન \\
\midrule\noalign{}
\endhead
\bottomrule\noalign{}
\endlastfoot
ડેટાબેઝ ડિઝાઇન & લોજિકલ/ફિઝિકલ ડેટાબેઝ સ્ટ્રક્ચર અને સ્કીમા બનાવે છે \\
સિક્યોરિટી મેનેજમેન્ટ & યુઝર એકાઉન્ટ્સ અને પરમિશન્સ દ્વારા ઍક્સેસ નિયંત્રિત કરે છે \\
પરફોર્મન્સ ટ્યુનિંગ & ઝડપી ડેટા પુનઃપ્રાપ્તિ માટે ક્વેરીઝ, ઇન્ડેક્સ ઓપ્ટિમાઇઝ કરે છે \\
બેકઅપ \& રિકવરી & ડેટા નુકસાન રોકવા માટેની વ્યૂહરચના અમલમાં મૂકે છે \\
મેઇન્ટેનન્સ & સોફ્ટવેર અપડેટ કરે છે, પેચિસ લાગુ કરે છે, સ્પેસનું મોનિટરિંગ કરે છે \\
\end{longtable}
}

\textbf{આકૃતિ:}

\begin{verbatim}
mindmap
  root((DBA))
    ડિઝાઇન
      સ્કીમા
      ટેબલ્સ
      રિલેશનશિપ્સ
    સિક્યોરિટી
      યુઝર્સ
      પરમિશન્સ
      એન્ક્રિપ્શન
    પરફોર્મન્સ
      ક્વેરી ઓપ્ટિમાઇઝેશન
      ઇન્ડેક્સિંગ
      મોનિટરિંગ
    બેકઅપ
      નિયમિત બેકઅપ્સ
      રિકવરી પ્લાન્સ
    મેઇન્ટેનન્સ
      અપડેટ્સ
      સ્પેસ મેનેજમેન્ટ
\end{verbatim}

\end{solutionbox}
\begin{mnemonicbox}
``SPMBU'' - સિક્યોરિટી, પરફોર્મન્સ, મેઇન્ટેનન્સ, બેકઅપ,
અપડેટ્સ

\end{mnemonicbox}
\subsection*{પ્રશ્ન 1(ક) OR [7
માર્ક્સ]}\label{uxaaauxab0uxab6uxaa8-1uxa95-or-7-uxaaeuxab0uxa95uxab8}

\textbf{રીલેશનલ અને નેટવર્ક ડેટા મોડેલ વિસ્તારથી સમજાવો.}

\begin{solutionbox}


{\def\LTcaptype{none} % do not increment counter
\vspace{-5pt}
\captionof{table}{રીલેશનલ અને નેટવર્ક ડેટા મોડેલની તુલના}
\vspace{-10pt}
\begin{longtable}[]{@{}
  >{\raggedright\arraybackslash}p{(\linewidth - 4\tabcolsep) * \real{0.2143}}
  >{\raggedright\arraybackslash}p{(\linewidth - 4\tabcolsep) * \real{0.4286}}
  >{\raggedright\arraybackslash}p{(\linewidth - 4\tabcolsep) * \real{0.3571}}@{}}
\toprule\noalign{}
\begin{minipage}[b]{\linewidth}\raggedright
લક્ષણ
\end{minipage} & \begin{minipage}[b]{\linewidth}\raggedright
રીલેશનલ મોડેલ
\end{minipage} & \begin{minipage}[b]{\linewidth}\raggedright
નેટવર્ક મોડેલ
\end{minipage} \\
\midrule\noalign{}
\endhead
\bottomrule\noalign{}
\endlastfoot
સ્ટ્રક્ચર & ટેબલ્સ (રીલેશન્સ) - રો અને કોલમ્સ સાથે & રેકોર્ડ્સ પોઇન્ટર્સ દ્વારા જોડાયેલા
જટિલ નેટવર્ક બનાવે છે \\
સંબંધ & પ્રાઇમરી અને ફોરેન કી દ્વારા જોડાયેલા & પેરન્ટ-ચાઇલ્ડ રેકોર્ડ્સ વચ્ચે ડાયરેક્ટ
લિંક્સ \\
ફ્લેક્સિબિલિટી & ઉચ્ચ - ટેબલ્સ જરૂરિયાત મુજબ જોઈન કરી શકાય છે & સીમિત -
પૂર્વનિર્ધારિત ફિઝિકલ કનેક્શન \\
ઉદાહરણો & MySQL, Oracle, SQL Server & IDS, IDMS \\
ક્વેરી લેંગ્વેજ & SQL (સ્ટ્રક્ચર્ડ ક્વેરી લેંગ્વેજ) & પ્રોસીજરલ લેંગ્વેજ \\
\end{longtable}
}

\textbf{આકૃતિ:}

\begin{center}
\textbf{Mermaid Diagram (Code)}
\begin{verbatim}
{Shaded}
{Highlighting}[]
graph TD
    subgraph "રીલેશનલ મોડેલ"
    A[ટેબલ: વિદ્યાર્થીઓ] {-{-}{-} B[ટેબલ: અભ્યાસક્રમો]}
    A {-{-}{-} C[ટેબલ: ગ્રેડ્સ]}
    end

    subgraph "નેટવર્ક મોડેલ"
    D[રેકોર્ડ: વિદ્યાર્થી] {-{-}{} E[રેકોર્ડ: અભ્યાસક્રમ1]}
    D {-{-}{} F[રેકોર્ડ: અભ્યાસક્રમ2]}
    F {-{-}{} G[રેકોર્ડ: ગ્રેડ]}
    end
{Highlighting}
{Shaded}
\end{verbatim}
\end{center}

\end{solutionbox}
\begin{mnemonicbox}
``RSPEN'' - રીલેશનલ યુઝિસ સેટ્સ, પોઇન્ટર્સ એનેબલ નેટવર્ક્સ

\end{mnemonicbox}
\subsection*{પ્રશ્ન 2(અ) [3
માર્ક્સ]}\label{uxaaauxab0uxab6uxaa8-2uxa85-3-uxaaeuxab0uxa95uxab8}

\textbf{Generalization આકૃતિ સાથે સમજાવો.}

\begin{solutionbox}

\textbf{Generalization}: બે કે વધુ એન્ટિટીઓમાંથી સામાન્ય લક્ષણો કાઢીને નવી ઉચ્ચ
સ્તરની એન્ટિટી બનાવવાની પ્રક્રિયા.

\textbf{આકૃતિ:}

\begin{verbatim}
classDiagram
    Vehicle {|{-}{-} Car}
    Vehicle {|{-}{-} Truck}
    Vehicle {|{-}{-} Motorcycle}

    class Vehicle\{
        +vehicle\_id
        +manufacturer
        +year
    \}
    class Car\{
        +num\_doors
        +fuel\_type
    \}
    class Truck\{
        +cargo\_capacity
        +towing\_capacity
    \}
    class Motorcycle\{
        +engine\_size
        +type
    \}
\end{verbatim}

\end{solutionbox}
\begin{mnemonicbox}
``BUSH'' - બોટમ-અપ શેર્ડ હાયરાર્કી

\end{mnemonicbox}
\subsection*{પ્રશ્ન 2(બ) [4
માર્ક્સ]}\label{uxaaauxab0uxab6uxaa8-2uxaac-4-uxaaeuxab0uxa95uxab8}

\textbf{Primary કી અને Foreign કી Constraints સમજાઓ.}

\begin{solutionbox}


{\def\LTcaptype{none} % do not increment counter
\vspace{-5pt}
\captionof{table}{પ્રાઇમરી કી વિ. ફોરેન કી}
\vspace{-10pt}
\begin{longtable}[]{@{}
  >{\raggedright\arraybackslash}p{(\linewidth - 6\tabcolsep) * \real{0.2667}}
  >{\raggedright\arraybackslash}p{(\linewidth - 6\tabcolsep) * \real{0.2667}}
  >{\raggedright\arraybackslash}p{(\linewidth - 6\tabcolsep) * \real{0.2667}}
  >{\raggedright\arraybackslash}p{(\linewidth - 6\tabcolsep) * \real{0.2000}}@{}}
\toprule\noalign{}
\begin{minipage}[b]{\linewidth}\raggedright
કન્સ્ટ્રેઇન્ટ
\end{minipage} & \begin{minipage}[b]{\linewidth}\raggedright
વ્યાખ્યા
\end{minipage} & \begin{minipage}[b]{\linewidth}\raggedright
ગુણધર્મો
\end{minipage} & \begin{minipage}[b]{\linewidth}\raggedright
ઉદાહરણ
\end{minipage} \\
\midrule\noalign{}
\endhead
\bottomrule\noalign{}
\endlastfoot
પ્રાઇમરી કી & ટેબલમાં દરેક રેકોર્ડને અનન્ય રીતે ઓળખે છે & અનન્ય, નોટ નલ, ટેબલ દીઠ
માત્ર એક & વિદ્યાર્થી ટેબલમાં StudentID \\
ફોરેન કી & ટેબલો વચ્ચે ડેટાને જોડે છે, બીજા ટેબલના પ્રાઇમરી કીનો સંદર્ભ આપે છે & NULL
હોઈ શકે, એક ટેબલમાં અનેક હોઈ શકે & એમ્પ્લોયી ટેબલમાં DeptID \\
\end{longtable}
}

\textbf{આકૃતિ:}

\begin{verbatim}
erDiagram
    DEPARTMENT \{
        int dept\_id PK
        string dept\_name
    \}
    EMPLOYEE \{
        int emp\_id PK
        string name
        int dept\_id FK
    \}
    DEPARTMENT ||{-{-}o\{ EMPLOYEE : "has"}
\end{verbatim}

\end{solutionbox}
\begin{mnemonicbox}
``PURE FIRE'' - પ્રાઇમરી યુનિકલી રેફરન્સિસ એન્ટિટીઝ, ફોરેન
ઇમ્પોર્ટ્સ રેફરન્સ્ડ એન્ટિટીઝ

\end{mnemonicbox}
\subsection*{પ્રશ્ન 2(ક) [7
માર્ક્સ]}\label{uxaaauxab0uxab6uxaa8-2uxa95-7-uxaaeuxab0uxa95uxab8}

\textbf{હોસ્પિટલ મેનેજમેન્ટ સિસ્ટમ માટે E-R ડાયાગ્રામ બનાવો}

\begin{solutionbox}

\textbf{હોસ્પિટલ મેનેજમેન્ટ સિસ્ટમ માટે E-R ડાયાગ્રામ:}

\begin{verbatim}
erDiagram
    PATIENT ||{-{-}o\{ APPOINTMENT : makes}
    DOCTOR ||{-{-}o\{ APPOINTMENT : conducts}
    APPOINTMENT ||{-{-}o\{ PRESCRIPTION : generates}
    DEPARTMENT ||{-{-}o\{ DOCTOR : employs}
    ROOM ||{-{-}o\{ PATIENT : admits}

    PATIENT \{
        int patient\_id PK
        string name
        string address
        date DOB
        string phone
    \}
    DOCTOR \{
        int doctor\_id PK
        string name
        string specialization
        int dept\_id FK
    \}
    DEPARTMENT \{
        int dept\_id PK
        string name
        string location
    \}
    APPOINTMENT \{
        int app\_id PK
        int patient\_id FK
        int doctor\_id FK
        datetime date\_time
        string status
    \}
    PRESCRIPTION \{
        int pres\_id PK
        int app\_id FK
        date date
        string medications
    \}
    ROOM \{
        int room\_id PK
        string type
        boolean availability
    \}
\end{verbatim}

\end{solutionbox}
\begin{mnemonicbox}
``PADRE'' - પેશન્ટ અપોઇન્ટમેન્ટ ડોક્ટર રૂમ એન્ટિટીઝ

\end{mnemonicbox}
\subsection*{પ્રશ્ન 2(અ) OR [3
માર્ક્સ]}\label{uxaaauxab0uxab6uxaa8-2uxa85-or-3-uxaaeuxab0uxa95uxab8}

\textbf{Specialization આકૃતિ સાથે સમજાવો.}

\begin{solutionbox}

\textbf{Specialization}: હાલની એન્ટિટીમાંથી તેમને અલગ ઓળખવા માટે અનન્ય લક્ષણો
ઉમેરીને નવી એન્ટિટીઓ બનાવવાની પ્રક્રિયા.

\textbf{આકૃતિ:}

\begin{verbatim}
classDiagram
    Employee {-{-} FullTime}
    Employee {-{-} PartTime}

    class Employee\{
        +emp\_id
        +name
        +address
        +phone
    \}
    class FullTime\{
        +salary
        +benefits
    \}
    class PartTime\{
        +hourly\_rate
        +hours\_worked
    \}
\end{verbatim}

\end{solutionbox}
\begin{mnemonicbox}
``TDSB'' - ટોપ-ડાઉન સ્પેશલાઇઝ્ડ બ્રેકડાઉન

\end{mnemonicbox}
\subsection*{પ્રશ્ન 2(બ) OR [4
માર્ક્સ]}\label{uxaaauxab0uxab6uxaa8-2uxaac-or-4-uxaaeuxab0uxa95uxab8}

\textbf{યોગ્ય ઉદાહરણ સાથે સિંગલ વેલ્યુડ અને મલ્ટીવેલ્યુડ એટ્રીબ્યુટ વચ્ચેનો તફાવત
સમજાવો.}

\begin{solutionbox}


{\def\LTcaptype{none} % do not increment counter
\vspace{-5pt}
\captionof{table}{સિંગલ-વેલ્યુડ અને મલ્ટી-વેલ્યુડ એટ્રીબ્યુટ્સ}
\vspace{-10pt}
\begin{longtable}[]{@{}
  >{\raggedright\arraybackslash}p{(\linewidth - 6\tabcolsep) * \real{0.1395}}
  >{\raggedright\arraybackslash}p{(\linewidth - 6\tabcolsep) * \real{0.2791}}
  >{\raggedright\arraybackslash}p{(\linewidth - 6\tabcolsep) * \real{0.2093}}
  >{\raggedright\arraybackslash}p{(\linewidth - 6\tabcolsep) * \real{0.3721}}@{}}
\toprule\noalign{}
\begin{minipage}[b]{\linewidth}\raggedright
પ્રકાર
\end{minipage} & \begin{minipage}[b]{\linewidth}\raggedright
વ્યાખ્યા
\end{minipage} & \begin{minipage}[b]{\linewidth}\raggedright
ઉદાહરણ
\end{minipage} & \begin{minipage}[b]{\linewidth}\raggedright
ઇમ્પ્લિમેન્ટેશન
\end{minipage} \\
\midrule\noalign{}
\endhead
\bottomrule\noalign{}
\endlastfoot
સિંગલ-વેલ્યુડ & દરેક એન્ટિટી ઇન્સ્ટન્સ માટે માત્ર એક જ મૂલ્ય ધરાવે છે & વ્યક્તિની
જન્મતારીખ, SSN & સીધા ટેબલ કોલમમાં સંગ્રહિત \\
મલ્ટી-વેલ્યુડ & એક જ એન્ટિટી માટે અનેક મૂલ્યો ધરાવી શકે છે & વ્યક્તિની કુશળતાઓ, ફોન
નંબરો & અલગ ટેબલ અથવા વિશિષ્ટ ફોર્મેટ \\
\end{longtable}
}

\textbf{આકૃતિ:}

\begin{verbatim}
erDiagram
    EMPLOYEE \{
        int emp\_id
        string name
        date birth\_date "સિંગલ{-વેલ્યુડ"}
    \}
    EMPLOYEE ||{-{-}o\{ PHONE\_NUMBERS : has}
    EMPLOYEE ||{-{-}o\{ SKILLS : possesses}

    PHONE\_NUMBERS \{
        int emp\_id
        string phone\_number "મલ્ટી{-વેલ્યુડ"}
    \}
    SKILLS \{
        int emp\_id
        string skill "મલ્ટી{-વેલ્યુડ"}
    \}
\end{verbatim}

\end{solutionbox}
\begin{mnemonicbox}
``SOME'' - સિંગલ વન, મલ્ટિપલ એન્ટ્રીઝ

\end{mnemonicbox}
\subsection*{પ્રશ્ન 2(ક) OR [7
માર્ક્સ]}\label{uxaaauxab0uxab6uxaa8-2uxa95-or-7-uxaaeuxab0uxa95uxab8}

\textbf{બેન્કિંગ મેનેજમેન્ટ સિસ્ટમ માટે E-R ડાયાગ્રામ બનાવો}

\begin{solutionbox}

\textbf{બેન્કિંગ મેનેજમેન્ટ સિસ્ટમ માટે E-R ડાયાગ્રામ:}

\begin{verbatim}
erDiagram
    CUSTOMER ||{-{-}o\{ ACCOUNT : owns}
    ACCOUNT ||{-{-}o\{ TRANSACTION : has}
    BRANCH ||{-{-}o\{ ACCOUNT : maintains}
    EMPLOYEE \|{-}{-}|| BRANCH : works\_at}
    LOAN {-}{-}|| CUSTOMER : takes}

    CUSTOMER \{
        int customer\_id PK
        string name
        string address
        string phone
        string email
    \}
    ACCOUNT \{
        int account\_no PK
        int customer\_id FK
        int branch\_id FK
        float balance
        string type
        date opening\_date
    \}
    TRANSACTION \{
        int trans\_id PK
        int account\_no FK
        date trans\_date
        float amount
        string type
        string description
    \}
    BRANCH \{
        int branch\_id PK
        string name
        string location
        string manager
    \}
    EMPLOYEE \{
        int emp\_id PK
        string name
        string position
        float salary
        int branch\_id FK
    \}
    LOAN \{
        int loan\_id PK
        int customer\_id FK
        float amount
        float interest\_rate
        date start\_date
        date end\_date
    \}
\end{verbatim}

\end{solutionbox}
\begin{mnemonicbox}
``CABLE'' - કસ્ટમર્સ અકાઉન્ટ્સ બ્રાન્ચિસ લોન્સ એમ્પ્લોયીઝ

\end{mnemonicbox}
\subsection*{પ્રશ્ન 3(અ) [3
માર્ક્સ]}\label{uxaaauxab0uxab6uxaa8-3uxa85-3-uxaaeuxab0uxa95uxab8}

\textbf{WHERE અને DESC ક્લોઝ ઉદાહરણ સાથે સમજાવો.}

\begin{solutionbox}


{\def\LTcaptype{none} % do not increment counter
\vspace{-5pt}
\captionof{table}{WHERE અને DESC ક્લોઝનો ઉપયોગ}
\vspace{-10pt}
\begin{longtable}[]{@{}
  >{\raggedright\arraybackslash}p{(\linewidth - 6\tabcolsep) * \real{0.2353}}
  >{\raggedright\arraybackslash}p{(\linewidth - 6\tabcolsep) * \real{0.2647}}
  >{\raggedright\arraybackslash}p{(\linewidth - 6\tabcolsep) * \real{0.2353}}
  >{\raggedright\arraybackslash}p{(\linewidth - 6\tabcolsep) * \real{0.2647}}@{}}
\toprule\noalign{}
\begin{minipage}[b]{\linewidth}\raggedright
ક્લોઝ
\end{minipage} & \begin{minipage}[b]{\linewidth}\raggedright
હેતુ
\end{minipage} & \begin{minipage}[b]{\linewidth}\raggedright
સિન્ટેક્સ
\end{minipage} & \begin{minipage}[b]{\linewidth}\raggedright
ઉદાહરણ
\end{minipage} \\
\midrule\noalign{}
\endhead
\bottomrule\noalign{}
\endlastfoot
WHERE & ચોક્કસ શરત પર આધારિત રો ફિલ્ટર કરે છે & SELECT columns FROM table
WHERE condition & SELECT * FROM employees WHERE salary \textgreater{}
50000 \\
DESC & પરિણામોને ઉતરતા ક્રમમાં ગોઠવે છે & SELECT columns FROM table ORDER BY
column DESC & SELECT * FROM products ORDER BY price DESC \\
\end{longtable}
}

\textbf{આકૃતિ:}

\begin{verbatim}
{-{-} Students ટેબલમાં મૂળ ડેટા}
| ID | Name   | Marks |
|{-{-}{-}{-}|{-}{-}{-}{-}{-}{-}{-}{-}|{-}{-}{-}{-}{-}{-}{-}|}
| 1  | Alice  | 85    |
| 2  | Bob    | 92    |
| 3  | Carol  | 78    |
| 4  | David  | 65    |

{-{-} WHERE વાપરીને: SELECT * FROM Students WHERE Marks  80}
| ID | Name   | Marks |
|{-{-}{-}{-}|{-}{-}{-}{-}{-}{-}{-}{-}|{-}{-}{-}{-}{-}{-}{-}|}
| 1  | Alice  | 85    |
| 2  | Bob    | 92    |

{-{-} DESC વાપરીને: SELECT * FROM Students ORDER BY Marks DESC}
| ID | Name   | Marks |
|{-{-}{-}{-}|{-}{-}{-}{-}{-}{-}{-}{-}|{-}{-}{-}{-}{-}{-}{-}|}
| 2  | Bob    | 92    |
| 1  | Alice  | 85    |
| 3  | Carol  | 78    |
| 4  | David  | 65    |
\end{verbatim}

\end{solutionbox}
\begin{mnemonicbox}
``WDF'' - Where ડેટા ફિલ્ટર કરે છે, DESC ઉચ્ચતમ પહેલા ક્રમ
આપે છે

\end{mnemonicbox}
\subsection*{પ્રશ્ન 3(બ) [4
માર્ક્સ]}\label{uxaaauxab0uxab6uxaa8-3uxaac-4-uxaaeuxab0uxa95uxab8}

\textbf{DDL કમાન્ડની યાદી બનાવો. કોઈ પણ ૨ DDL કમાન્ડ ઉદાહરણ સાથે સમજાવો.}

\begin{solutionbox}

\textbf{DDL (ડેટા ડેફિનિશન લેંગ્વેજ) કમાન્ડ્સ:}

\begin{enumerate}
\tightlist
\item
  CREATE
\item
  ALTER
\item
  DROP
\item
  TRUNCATE
\item
  RENAME
\end{enumerate}


{\def\LTcaptype{none} % do not increment counter
\vspace{-5pt}
\captionof{table}{CREATE અને ALTER કમાન્ડ્સ}
\vspace{-10pt}
\begin{longtable}[]{@{}
  >{\raggedright\arraybackslash}p{(\linewidth - 6\tabcolsep) * \real{0.2571}}
  >{\raggedright\arraybackslash}p{(\linewidth - 6\tabcolsep) * \real{0.2571}}
  >{\raggedright\arraybackslash}p{(\linewidth - 6\tabcolsep) * \real{0.2286}}
  >{\raggedright\arraybackslash}p{(\linewidth - 6\tabcolsep) * \real{0.2571}}@{}}
\toprule\noalign{}
\begin{minipage}[b]{\linewidth}\raggedright
કમાન્ડ
\end{minipage} & \begin{minipage}[b]{\linewidth}\raggedright
હેતુ
\end{minipage} & \begin{minipage}[b]{\linewidth}\raggedright
સિન્ટેક્સ
\end{minipage} & \begin{minipage}[b]{\linewidth}\raggedright
ઉદાહરણ
\end{minipage} \\
\midrule\noalign{}
\endhead
\bottomrule\noalign{}
\endlastfoot
CREATE & ટેબલ, વ્યૂ, ઇન્ડેક્સ જેવા ડેટાબેઝ ઑબ્જેક્ટ્સ બનાવે છે & CREATE TABLE
table\_name (column definitions) & CREATE TABLE students (id INT PRIMARY
KEY, name VARCHAR(50)) \\
ALTER & હાલના ડેટાબેઝ ઑબ્જેક્ટની સ્ટ્રક્ચર સુધારે છે & ALTER TABLE table\_name
action & ALTER TABLE students ADD COLUMN email VARCHAR(100) \\
\end{longtable}
}

\textbf{કોડબ્લોક:}

\begin{verbatim}
{-{-} CREATE ઉદાહરણ}
CREATE TABLE employees (
    emp\_id INT PRIMARY KEY,
    name VARCHAR(50) NOT NULL,
    dept VARCHAR(30),
    salary DECIMAL(10,2)
);

{-{-} ALTER ઉદાહરણ}
ALTER TABLE employees 
ADD COLUMN hire\_date DATE;
\end{verbatim}

\end{solutionbox}
\begin{mnemonicbox}
``CADTR'' - Create Alter Drop Truncate Rename

\end{mnemonicbox}
\subsection*{પ્રશ્ન 3(ક) [7
માર્ક્સ]}\label{uxaaauxab0uxab6uxaa8-3uxa95-7-uxaaeuxab0uxa95uxab8}

\textbf{eno, ename, salary, dept ફિલ્ડ ધરાવતા Company ટેબલ પર નીચેની Query
perform કરો.} \textbf{૧. Company ટેબલના તમામ રેકોર્ડ ડિસ્પ્લે કરો.} \textbf{૨.
ડુપ્લિકેટ વેલ્યુ સિવાય માત્ર dept ડિસ્પ્લે કરો.} \textbf{૩. ename ના ઉતરતા ક્રમમાં
તમામ રેકોર્ડ ડિસ્પ્લે કરો.} \textbf{૪. શહેરનું નામ સ્ટોર કરવા માટે ``cityname''
નામથી નવી કોલમ ઉમેરો.} \textbf{૫. ``Mumbai'' શહેરમાં ન રહેતા હોય તેવા તમામ
કર્મચારીઓનાં નામ ડિસ્પ્લે કરો.} \textbf{૬. ૧૦૦૦૦ કરતા ઓછું પગાર ધરાવતા તમામ
કર્મચારીઓને ડીલીટ કરો.} \textbf{૭. ``A'' થી શરુ થતા તમામ કર્મચારીઓના નામ
ડિસ્પ્લે કરો.}

\begin{solutionbox}

\textbf{કોડબ્લોક:}

\begin{verbatim}
{-{-} ૧. Company ટેબલના તમામ રેકોર્ડ ડિસ્પ્લે કરો}
SELECT * FROM Company;

{-{-} ૨. ડુપ્લિકેટ વેલ્યુ સિવાય માત્ર dept ડિસ્પ્લે કરો}
SELECT DISTINCT dept FROM Company;

{-{-} ૩. ename ના ઉતરતા ક્રમમાં તમામ રેકોર્ડ ડિસ્પ્લે કરો}
SELECT * FROM Company ORDER BY ename DESC;

{-{-} ૪. શહેરનું નામ સ્ટોર કરવા માટે "cityname" નામથી નવી કોલમ ઉમેરો}
ALTER TABLE Company ADD COLUMN cityname VARCHAR(50);

{-{-} ૫. "Mumbai" શહેરમાં ન રહેતા હોય તેવા તમામ કર્મચારીઓનાં નામ ડિસ્પ્લે કરો}
SELECT ename FROM Company WHERE cityname != {Mumbai};

{-{-} ૬. ૧૦૦૦૦ કરતા ઓછું પગાર ધરાવતા તમામ કર્મચારીઓને ડીલીટ કરો}
DELETE FROM Company WHERE salary {} 10000;

{-{-} ૭. "A" થી શરુ થતા તમામ કર્મચારીઓના નામ ડિસ્પ્લે કરો}
SELECT ename FROM Company WHERE ename LIKE {A\%};
\end{verbatim}


{\def\LTcaptype{none} % do not increment counter
\vspace{-5pt}
\captionof{table}{SQL ઓપરેશન્સ}
\vspace{-10pt}
\begin{longtable}[]{@{}lll@{}}
\toprule\noalign{}
ઓપરેશન & SQL કમાન્ડ & હેતુ \\
\midrule\noalign{}
\endhead
\bottomrule\noalign{}
\endlastfoot
SELECT & SELECT * FROM Company & બધો ડેટા મેળવે છે \\
DISTINCT & SELECT DISTINCT dept & ડુપ્લિકેટ દૂર કરે છે \\
ORDER BY & ORDER BY ename DESC & ઉતરતા ક્રમમાં ગોઠવે છે \\
ALTER & ALTER TABLE ADD COLUMN & નવી કોલમ ઉમેરે છે \\
WHERE & WHERE cityname != `Mumbai' & ફિલ્ટર શરત \\
DELETE & DELETE FROM WHERE & રેકોર્ડ દૂર કરે છે \\
LIKE & WHERE ename LIKE `A\%' & પેટર્ન મેચિંગ \\
\end{longtable}
}

\end{solutionbox}
\begin{mnemonicbox}
``SODA-WDL'' - Select Order Distinct Alter - Where
Delete Like

\end{mnemonicbox}
\subsection*{પ્રશ્ન 3(અ) OR [3
માર્ક્સ]}\label{uxaaauxab0uxab6uxaa8-3uxa85-or-3-uxaaeuxab0uxa95uxab8}

\textbf{SELECT અને DISTINCT ક્લોઝ ઉદાહરણ સાથે સમજાવો.}

\begin{solutionbox}


{\def\LTcaptype{none} % do not increment counter
\vspace{-5pt}
\captionof{table}{SELECT અને DISTINCT ક્લોઝનો ઉપયોગ}
\vspace{-10pt}
\begin{longtable}[]{@{}
  >{\raggedright\arraybackslash}p{(\linewidth - 6\tabcolsep) * \real{0.2353}}
  >{\raggedright\arraybackslash}p{(\linewidth - 6\tabcolsep) * \real{0.2647}}
  >{\raggedright\arraybackslash}p{(\linewidth - 6\tabcolsep) * \real{0.2353}}
  >{\raggedright\arraybackslash}p{(\linewidth - 6\tabcolsep) * \real{0.2647}}@{}}
\toprule\noalign{}
\begin{minipage}[b]{\linewidth}\raggedright
ક્લોઝ
\end{minipage} & \begin{minipage}[b]{\linewidth}\raggedright
હેતુ
\end{minipage} & \begin{minipage}[b]{\linewidth}\raggedright
સિન્ટેક્સ
\end{minipage} & \begin{minipage}[b]{\linewidth}\raggedright
ઉદાહરણ
\end{minipage} \\
\midrule\noalign{}
\endhead
\bottomrule\noalign{}
\endlastfoot
SELECT & ડેટાબેઝમાંથી ડેટા મેળવે છે & SELECT columns FROM table & SELECT name,
age FROM students \\
DISTINCT & ડુપ્લિકેટ મૂલ્યો દૂર કરે છે & SELECT DISTINCT columns FROM table &
SELECT DISTINCT department FROM employees \\
\end{longtable}
}

\textbf{આકૃતિ:}

\begin{verbatim}
{-{-} Departments ટેબલમાં મૂળ ડેટા}
| dept\_id | dept\_name |
|{-{-}{-}{-}{-}{-}{-}{-}{-}|{-}{-}{-}{-}{-}{-}{-}{-}{-}{-}{-}|}
| 1       | Sales     |
| 2       | IT        |
| 3       | HR        |
| 4       | IT        |
| 5       | Sales     |

{-{-} SELECT વાપરીને: SELECT dept\_name FROM Departments}
| dept\_name |
|{-{-}{-}{-}{-}{-}{-}{-}{-}{-}{-}|}
| Sales     |
| IT        |
| HR        |
| IT        |
| Sales     |

{-{-} DISTINCT વાપરીને: SELECT DISTINCT dept\_name FROM Departments}
| dept\_name |
|{-{-}{-}{-}{-}{-}{-}{-}{-}{-}{-}|}
| Sales     |
| IT        |
| HR        |
\end{verbatim}

\end{solutionbox}
\begin{mnemonicbox}
``SUD'' - Select Unique with Distinct

\end{mnemonicbox}
\subsection*{પ્રશ્ન 3(બ) OR [4
માર્ક્સ]}\label{uxaaauxab0uxab6uxaa8-3uxaac-or-4-uxaaeuxab0uxa95uxab8}

\textbf{DML કમાન્ડની યાદી બનાવો. કોઈ પણ ૨ DML કમાન્ડ ઉદાહરણ સાથે સમજાવો.}

\begin{solutionbox}

\textbf{DML (ડેટા મેનિપ્યુલેશન લેંગ્વેજ) કમાન્ડ્સ:}

\begin{enumerate}
\tightlist
\item
  INSERT
\item
  UPDATE
\item
  DELETE
\item
  SELECT
\end{enumerate}


{\def\LTcaptype{none} % do not increment counter
\vspace{-5pt}
\captionof{table}{INSERT અને UPDATE કમાન્ડ્સ}
\vspace{-10pt}
\begin{longtable}[]{@{}
  >{\raggedright\arraybackslash}p{(\linewidth - 6\tabcolsep) * \real{0.2571}}
  >{\raggedright\arraybackslash}p{(\linewidth - 6\tabcolsep) * \real{0.2571}}
  >{\raggedright\arraybackslash}p{(\linewidth - 6\tabcolsep) * \real{0.2286}}
  >{\raggedright\arraybackslash}p{(\linewidth - 6\tabcolsep) * \real{0.2571}}@{}}
\toprule\noalign{}
\begin{minipage}[b]{\linewidth}\raggedright
કમાન્ડ
\end{minipage} & \begin{minipage}[b]{\linewidth}\raggedright
હેતુ
\end{minipage} & \begin{minipage}[b]{\linewidth}\raggedright
સિન્ટેક્સ
\end{minipage} & \begin{minipage}[b]{\linewidth}\raggedright
ઉદાહરણ
\end{minipage} \\
\midrule\noalign{}
\endhead
\bottomrule\noalign{}
\endlastfoot
INSERT & ટેબલમાં નવા રેકોર્ડ ઉમેરે છે & INSERT INTO table\_name VALUES (values)
& INSERT INTO students VALUES (1, `John', 85) \\
UPDATE & હાલના રેકોર્ડમાં ફેરફાર કરે છે & UPDATE table\_name SET column=value
WHERE condition & UPDATE students SET marks=90 WHERE id=1 \\
\end{longtable}
}

\textbf{કોડબ્લોક:}

\begin{verbatim}
{-{-} INSERT ઉદાહરણ}
INSERT INTO employees (emp\_id, name, dept, salary)
VALUES (101, {John Smith}, {IT}, 65000);

{-{-} UPDATE ઉદાહરણ}
UPDATE employees 
SET salary = 70000 
WHERE emp\_id = 101;
\end{verbatim}

\end{solutionbox}
\begin{mnemonicbox}
``IUDS'' - Insert Update Delete Select

\end{mnemonicbox}
\subsection*{પ્રશ્ન 3(ક) OR [7
માર્ક્સ]}\label{uxaaauxab0uxab6uxaa8-3uxa95-or-7-uxaaeuxab0uxa95uxab8}

\textbf{નીચેની Query ના આઉટપુટ લખો.} \textbf{1. ABS(-34),ABS(16)}
\textbf{2. SQRT(16),SQRT(64)} \textbf{3. POWER(5,2), POWER(2,4)}
\textbf{4. MOD(15,3), MOD(13,3)} \textbf{5. ROUND(123.456,1),
ROUND(123.456,2)} \textbf{6. CEIL(122.6), CEIL(-122.6)} \textbf{7.
FLOOR(-157.5),FLOOR(157.5)}

\begin{solutionbox}


{\def\LTcaptype{none} % do not increment counter
\vspace{-5pt}
\captionof{table}{SQL ફંક્શન આઉટપુટ}
\vspace{-10pt}
\begin{longtable}[]{@{}
  >{\raggedright\arraybackslash}p{(\linewidth - 4\tabcolsep) * \real{0.3226}}
  >{\raggedright\arraybackslash}p{(\linewidth - 4\tabcolsep) * \real{0.4194}}
  >{\raggedright\arraybackslash}p{(\linewidth - 4\tabcolsep) * \real{0.2581}}@{}}
\toprule\noalign{}
\begin{minipage}[b]{\linewidth}\raggedright
ફંક્શન
\end{minipage} & \begin{minipage}[b]{\linewidth}\raggedright
વર્ણન
\end{minipage} & \begin{minipage}[b]{\linewidth}\raggedright
આઉટપુટ
\end{minipage} \\
\midrule\noalign{}
\endhead
\bottomrule\noalign{}
\endlastfoot
ABS(-34),ABS(16) & નિરપેક્ષ મૂલ્ય & 34, 16 \\
SQRT(16),SQRT(64) & વર્ગમૂળ & 4, 8 \\
POWER(5,2), POWER(2,4) & પાવર ફંક્શન & 25, 16 \\
MOD(15,3), MOD(13,3) & મોડ્યુલસ (બાકી) & 0, 1 \\
ROUND(123.456,1), ROUND(123.456,2) & દશાંશ સ્થાન સુધી રાઉન્ડ & 123.5,
123.46 \\
CEIL(122.6), CEIL(-122.6) & પૂર્ણાંક સુધી ઉપર રાઉન્ડ & 123, -122 \\
FLOOR(-157.5),FLOOR(157.5) & પૂર્ણાંક સુધી નીચે રાઉન્ડ & -158, 157 \\
\end{longtable}
}

\textbf{આકૃતિ:}

\begin{center}
\textbf{Mermaid Diagram (Code)}
\begin{verbatim}
{Shaded}
{Highlighting}[]
graph TD
    A[SQL ગણિત ફંક્શન્સ]
    A {-{-}{} B["ABS: નિરપેક્ષ મૂલ્ય{}br /{}ABS({-}34) = 34{}br /{}ABS(16) = 16"]}
    A {-{-}{} C["SQRT: વર્ગમૂળ{}br /{}SQRT(16) = 4{}br /{}SQRT(64) = 8"]}
    A {-{-}{} D["POWER: ઘાતાંક{}br /{}POWER(5,2) = 25{}br /{}POWER(2,4) = 16"]}
    A {-{-}{} E["MOD: બાકી{}br /{}MOD(15,3) = 0{}br /{}MOD(13,3) = 1"]}
    A {-{-}{} F["ROUND: દશાંશ રાઉન્ડ{}br /{}ROUND(123.456,1) = 123.5{}br /{}ROUND(123.456,2) = 123.46"]}
    A {-{-}{} G["CEIL: ઉપર રાઉન્ડ{}br /{}CEIL(122.6) = 123{}br /{}CEIL({-}122.6) = {-}122"]}
    A {-{-}{} H["FLOOR: નીચે રાઉન્ડ{}br /{}FLOOR({-}157.5) = {-}158{}br /{}FLOOR(157.5) = 157"]}
{Highlighting}
{Shaded}
\end{verbatim}
\end{center}

\end{solutionbox}
\begin{mnemonicbox}
``ASPRCF'' - Absolute Square Power Remainder Ceiling
Floor

\end{mnemonicbox}
\subsection*{પ્રશ્ન 4(અ) [3
માર્ક્સ]}\label{uxaaauxab0uxab6uxaa8-4uxa85-3-uxaaeuxab0uxa95uxab8}

\textbf{SQLમાં ડેટા ટાઈપની યાદી બનાવો. 1.VARCHAR() અને 2.INT() ડેટા ટાઈપ
ઉદાહરણ સાથે સમજાવો.}

\begin{solutionbox}

\textbf{SQL ડેટા ટાઈપ કેટેગરીઝ:}

\begin{enumerate}
\tightlist
\item
  ન્યુમેરિક (INT, FLOAT, DECIMAL)
\item
  કેરેક્ટર (CHAR, VARCHAR)
\item
  ડેટ/ટાઈમ (DATE, TIME, DATETIME)
\item
  બાઈનરી (BLOB, BINARY)
\item
  બૂલિયન (BOOL)
\end{enumerate}


{\def\LTcaptype{none} % do not increment counter
\vspace{-5pt}
\captionof{table}{VARCHAR અને INT ડેટા ટાઈપ્સ}
\vspace{-10pt}
\begin{longtable}[]{@{}
  >{\raggedright\arraybackslash}p{(\linewidth - 6\tabcolsep) * \real{0.2821}}
  >{\raggedright\arraybackslash}p{(\linewidth - 6\tabcolsep) * \real{0.3333}}
  >{\raggedright\arraybackslash}p{(\linewidth - 6\tabcolsep) * \real{0.1538}}
  >{\raggedright\arraybackslash}p{(\linewidth - 6\tabcolsep) * \real{0.2308}}@{}}
\toprule\noalign{}
\begin{minipage}[b]{\linewidth}\raggedright
ડેટા ટાઈપ
\end{minipage} & \begin{minipage}[b]{\linewidth}\raggedright
વર્ણન
\end{minipage} & \begin{minipage}[b]{\linewidth}\raggedright
સાઈઝ
\end{minipage} & \begin{minipage}[b]{\linewidth}\raggedright
ઉદાહરણ
\end{minipage} \\
\midrule\noalign{}
\endhead
\bottomrule\noalign{}
\endlastfoot
VARCHAR(n) & વેરિએબલ-લેન્થ કેરેક્ટર સ્ટ્રિંગ & n કેરેક્ટર સુધી, માત્ર જરૂરી જગ્યાનો
ઉપયોગ & નામ, ઈમેલ માટે VARCHAR(50) \\
INT & ઇન્ટિજર ન્યુમેરિક ડેટા & સામાન્ય રીતે 4 બાઈટ્સ, -2,147,483,648 થી
2,147,483,647 & ID, કાઉન્ટ, ઉંમર માટે INT \\
\end{longtable}
}

\textbf{કોડબ્લોક:}

\begin{verbatim}
CREATE TABLE students (
    student\_id INT PRIMARY KEY,
    name VARCHAR(50) NOT NULL,
    age INT,
    email VARCHAR(100)
);
\end{verbatim}

\end{solutionbox}
\begin{mnemonicbox}
``VIA'' - Variable strings, Integers for Ages

\end{mnemonicbox}
\subsection*{પ્રશ્ન 4(બ) [4
માર્ક્સ]}\label{uxaaauxab0uxab6uxaa8-4uxaac-4-uxaaeuxab0uxa95uxab8}

\textbf{2NF (સેકન્ડ નોર્મલ ફોર્મ) ઉદાહરણ અને ઉકેલ સાથે સમજાવો.}

\begin{solutionbox}

\textbf{2NF વ્યાખ્યા}: એક સંબંધ 2NF માં છે જો તે 1NF માં હોય અને કોઈપણ નોન-પ્રાઈમ
એટ્રિબ્યુટ કોઈપણ કેન્ડિડેટ કીના સબસેટ પર આધારિત ન હોય.


{\def\LTcaptype{none} % do not increment counter
\vspace{-5pt}
\captionof{table}{2NF પહેલાં}
\vspace{-10pt}
\begin{longtable}[]{@{}llll@{}}
\toprule\noalign{}
student\_id & course\_id & course\_name & instructor \\
\midrule\noalign{}
\endhead
\bottomrule\noalign{}
\endlastfoot
S1 & C1 & Database & Prof.~Smith \\
S1 & C2 & Networking & Prof.~Jones \\
S2 & C1 & Database & Prof.~Smith \\
S3 & C3 & Programming & Prof.~Wilson \\
\end{longtable}
}

\textbf{સમસ્યા}: નોન-પ્રાઈમ એટ્રિબ્યુટ્સ (course\_name, instructor) માત્ર
course\_id પર આધારિત છે, સંપૂર્ણ કી (student\_id, course\_id) પર નહીં.

\textbf{આકૃતિ: 2NF ઉકેલ}

\begin{verbatim}
erDiagram
    ENROLLMENT \{
        string student\_id PK
        string course\_id PK
    \}
    COURSE \{
        string course\_id PK
        string course\_name
        string instructor
    \}
    ENROLLMENT {-}{-}|| COURSE : references}
\end{verbatim}


{\def\LTcaptype{none} % do not increment counter
\vspace{-5pt}
\captionof{table}{2NF પછી}
\vspace{-10pt}
\begin{longtable}[]{@{}ll@{}}
\toprule\noalign{}
student\_id & course\_id \\
\midrule\noalign{}
\endhead
\bottomrule\noalign{}
\endlastfoot
S1 & C1 \\
S1 & C2 \\
S2 & C1 \\
S3 & C3 \\
\end{longtable}
}

Course ટેબલ:

{\def\LTcaptype{none} % do not increment counter
\begin{longtable}[]{@{}lll@{}}
\toprule\noalign{}
course\_id & course\_name & instructor \\
\midrule\noalign{}
\endhead
\bottomrule\noalign{}
\endlastfoot
C1 & Database & Prof.~Smith \\
C2 & Networking & Prof.~Jones \\
C3 & Programming & Prof.~Wilson \\
\end{longtable}
}

\end{solutionbox}
\begin{mnemonicbox}
``PFPK'' - Partial Functional dependency on Primary
Key

\end{mnemonicbox}
\subsection*{પ્રશ્ન 4(ક) [7
માર્ક્સ]}\label{uxaaauxab0uxab6uxaa8-4uxa95-7-uxaaeuxab0uxa95uxab8}

\textbf{Function dependency સમજાવો. Partial function dependency ઉદાહરણ
સાથે સમજાવો.}

\begin{solutionbox}

\textbf{Functional Dependency}: એટ્રિબ્યુટ્સ વચ્ચેનો સંબંધ જ્યાં એક એટ્રિબ્યુટનું મૂલ્ય
બીજા એટ્રિબ્યુટના મૂલ્યને નક્કી કરે છે.

\textbf{નોટેશન}: X \rightarrow Y (X Y ને નક્કી કરે છે)

\textbf{Partial Functional Dependency}: જ્યારે નોન-પ્રાઈમ એટ્રિબ્યુટ કંપોઝિટ
કીના સંપૂર્ણ કરતાં ભાગ પર આધારિત હોય.


{\def\LTcaptype{none} % do not increment counter
\vspace{-5pt}
\captionof{table}{Order Details (નોર્મલાઈઝેશન પહેલાં)}
\vspace{-10pt}
\begin{longtable}[]{@{}lllll@{}}
\toprule\noalign{}
order\_id & product\_id & quantity & product\_name & price \\
\midrule\noalign{}
\endhead
\bottomrule\noalign{}
\endlastfoot
O1 & P1 & 5 & Keyboard & 50 \\
O1 & P2 & 2 & Mouse & 25 \\
O2 & P1 & 1 & Keyboard & 50 \\
O3 & P3 & 3 & Monitor & 200 \\
\end{longtable}
}

\textbf{Functional Dependencies:}

\begin{itemize}
\tightlist
\item
  (order\_id, product\_id) \rightarrow quantity
\item
  product\_id \rightarrow product\_name
\item
  product\_id \rightarrow price
\end{itemize}

\textbf{આકૃતિ:}

\begin{verbatim}
flowchart TD
    A["(order\_id, product\_id)"] {-{-}|"પૂર્ણપણે નક્કી કરે છે"| B[quantity]}
    C[product\_id] {-{-}|"આંશિક રીતે નક્કી કરે છે"| D[product\_name]}
    C {-{-}|"આંશિક રીતે નક્કી કરે છે"| E[price]}

    style C fill:\#f9f,stroke:\#333,stroke{-width:2px}
    style D fill:\#bbf,stroke:\#333,stroke{-width:2px}
    style E fill:\#bbf,stroke:\#333,stroke{-width:2px}
\end{verbatim}

\textbf{ઉકેલ (નોર્મલાઈઝ્ડ ટેબલ્સ):} Orders ટેબલ:

{\def\LTcaptype{none} % do not increment counter
\begin{longtable}[]{@{}lll@{}}
\toprule\noalign{}
order\_id & product\_id & quantity \\
\midrule\noalign{}
\endhead
\bottomrule\noalign{}
\endlastfoot
O1 & P1 & 5 \\
O1 & P2 & 2 \\
O2 & P1 & 1 \\
O3 & P3 & 3 \\
\end{longtable}
}

Products ટેબલ:

{\def\LTcaptype{none} % do not increment counter
\begin{longtable}[]{@{}lll@{}}
\toprule\noalign{}
product\_id & product\_name & price \\
\midrule\noalign{}
\endhead
\bottomrule\noalign{}
\endlastfoot
P1 & Keyboard & 50 \\
P2 & Mouse & 25 \\
P3 & Monitor & 200 \\
\end{longtable}
}

\end{solutionbox}
\begin{mnemonicbox}
``PDPK'' - Partial Dependency on Part of Key

\end{mnemonicbox}
\subsection*{પ્રશ્ન 4(અ) OR [3
માર્ક્સ]}\label{uxaaauxab0uxab6uxaa8-4uxa85-or-3-uxaaeuxab0uxa95uxab8}

\textbf{કમાન્ડ સમજાવવો: 1) To\_Char() 2) To\_Date()}

\begin{solutionbox}


{\def\LTcaptype{none} % do not increment counter
\vspace{-5pt}
\captionof{table}{કન્વર્ઝન ફંક્શન્સ}
\vspace{-10pt}
\begin{longtable}[]{@{}
  >{\raggedright\arraybackslash}p{(\linewidth - 6\tabcolsep) * \real{0.2778}}
  >{\raggedright\arraybackslash}p{(\linewidth - 6\tabcolsep) * \real{0.2500}}
  >{\raggedright\arraybackslash}p{(\linewidth - 6\tabcolsep) * \real{0.2222}}
  >{\raggedright\arraybackslash}p{(\linewidth - 6\tabcolsep) * \real{0.2500}}@{}}
\toprule\noalign{}
\begin{minipage}[b]{\linewidth}\raggedright
ફંક્શન
\end{minipage} & \begin{minipage}[b]{\linewidth}\raggedright
હેતુ
\end{minipage} & \begin{minipage}[b]{\linewidth}\raggedright
સિન્ટેક્સ
\end{minipage} & \begin{minipage}[b]{\linewidth}\raggedright
ઉદાહરણ
\end{minipage} \\
\midrule\noalign{}
\endhead
\bottomrule\noalign{}
\endlastfoot
TO\_CHAR() & ડેટ/નંબરને ફોર્મેટ મોડેલનો ઉપયોગ કરીને કેરેક્ટર સ્ટ્રિંગમાં રૂપાંતરિત કરે છે
& TO\_CHAR(value, [format]) & TO\_CHAR(SYSDATE, `DD-MON-YYYY') \rightarrow
`14-JUN-2024' \\
TO\_DATE() & કેરેક્ટર સ્ટ્રિંગને ફોર્મેટ મોડેલનો ઉપયોગ કરીને ડેટમાં રૂપાંતરિત કરે છે &
TO\_DATE(string, [format]) & TO\_DATE(`14-JUN-2024', `DD-MON-YYYY')
\rightarrow ડેટ વેલ્યુ \\
\end{longtable}
}

\textbf{કોડબ્લોક:}

\begin{verbatim}
{-{-} TO\_CHAR ઉદાહરણો}
SELECT TO\_CHAR(SYSDATE, {DD{-}MON{-}YYYY}) FROM DUAL;  {-{-} 14{-}JUN{-}2024}
SELECT TO\_CHAR(1234.56, {$9,999.99}) FROM DUAL;    {-{-} $1,234.56}

{-{-} TO\_DATE ઉદાહરણો}
SELECT TO\_DATE({2024{-}06{-}14}, {YYYY{-}MM{-}DD}) FROM DUAL;
SELECT TO\_DATE({14/06/24}, {DD/MM/YY}) FROM DUAL;
\end{verbatim}

\end{solutionbox}
\begin{mnemonicbox}
``DCS'' - Date Conversion Strings

\end{mnemonicbox}
\subsection*{પ્રશ્ન 4(બ) OR [4
માર્ક્સ]}\label{uxaaauxab0uxab6uxaa8-4uxaac-or-4-uxaaeuxab0uxa95uxab8}

\textbf{Full function dependency ઉદાહરણ સાથે સમજાવો.}

\begin{solutionbox}

\textbf{Full Functional Dependency}: જ્યારે એક એટ્રિબ્યુટ કંપોઝિટ કી પર ફંક્શનલી
ડિપેન્ડન્ટ હોય, અને માત્ર ભાગ પર નહીં પણ સંપૂર્ણ કી પર આધારિત હોય.


{\def\LTcaptype{none} % do not increment counter
\vspace{-5pt}
\captionof{table}{Exam Results}
\vspace{-10pt}
\begin{longtable}[]{@{}llll@{}}
\toprule\noalign{}
student\_id & course\_id & exam\_date & score \\
\midrule\noalign{}
\endhead
\bottomrule\noalign{}
\endlastfoot
S1 & C1 & 2024-05-10 & 85 \\
S1 & C2 & 2024-05-15 & 92 \\
S2 & C1 & 2024-05-10 & 78 \\
S2 & C2 & 2024-05-15 & 88 \\
\end{longtable}
}

\textbf{Full Functional Dependency:}

\begin{itemize}
\tightlist
\item
  (student\_id, course\_id) \rightarrow score (સ્કોર વિદ્યાર્થી અને કોર્સ બંને પર આધારિત
  છે)
\end{itemize}

\textbf{આકૃતિ:}

\begin{verbatim}
flowchart LR
    A["(student\_id, course\_id)"] {-{-}|"પૂર્ણપણે નક્કી કરે છે"| B[score]}

    style A fill:\#f9f,stroke:\#333,stroke{-width:2px}
    style B fill:\#bbf,stroke:\#333,stroke{-width:2px}
\end{verbatim}

\textbf{સમજૂતી}: સ્કોર એટ્રિબ્યુટ સંપૂર્ણ રીતે કંપોઝિટ કી (student\_id,
course\_id) પર આધારિત છે કારણ કે:

\begin{itemize}
\tightlist
\item
  અલગ અલગ વિદ્યાર્થીઓના એક જ કોર્સ માટે અલગ અલગ સ્કોર હોઈ શકે છે
\item
  એક જ વિદ્યાર્થીના અલગ અલગ કોર્સ માટે અલગ અલગ સ્કોર હોઈ શકે છે
\item
  ચોક્કસ સ્કોર જાણવા માટે આપણને student\_id અને course\_id બંનેની જરૂર પડે છે
\end{itemize}

\end{solutionbox}
\begin{mnemonicbox}
``FCEK'' - Fully dependent on Complete/Entire Key

\end{mnemonicbox}
\subsection*{પ્રશ્ન 4(ક) OR [7
માર્ક્સ]}\label{uxaaauxab0uxab6uxaa8-4uxa95-or-7-uxaaeuxab0uxa95uxab8}

\textbf{નોર્મલાઇઝેશનની વ્યાખ્યા આપો. 1NF (ફર્સ્ટ નોર્મલ ફોર્મ) ઉદાહરણ અને ઉકેલ સાથે
સમજાવો.}

\begin{solutionbox}

\textbf{નોર્મલાઇઝેશન}: ડેટા રિડન્ડન્સી ઘટાડવા, ડેટા અખંડતા સુધારવા અને એનોમલીઓને
દૂર કરવા માટે મોટા ટેબલને નાના સંબંધિત ટેબલમાં વિભાજિત કરીને ડેટાને વ્યવસ્થિત કરવાની
પ્રક્રિયા.

\textbf{1NF વ્યાખ્યા}: એક સંબંધ 1NF માં છે જો તેના બધા એટ્રિબ્યુટ્સ માત્ર અવિભાજ્ય
(એટોમિક) મૂલ્યો ધરાવતા હોય.


{\def\LTcaptype{none} % do not increment counter
\vspace{-5pt}
\captionof{table}{1NF પહેલાં}
\vspace{-10pt}
\begin{longtable}[]{@{}lll@{}}
\toprule\noalign{}
student\_id & name & courses \\
\midrule\noalign{}
\endhead
\bottomrule\noalign{}
\endlastfoot
S1 & John & Math, Physics \\
S2 & Mary & Chemistry, Biology, Physics \\
S3 & Tim & Computer Science \\
\end{longtable}
}

\textbf{સમસ્યાઓ}:

\begin{itemize}
\tightlist
\item
  નોન-એટોમિક મૂલ્યો (એક સેલમાં અનેક કોર્સ)
\item
  ચોક્કસ કોર્સને ક્વેરી કે અપડેટ કરવું સરળ નથી
\end{itemize}

\textbf{આકૃતિ:}

\begin{verbatim}
flowchart LR
    A[Non{-1NF ટેબલ] {-}{-} B[સમસ્યા: એક કોલમમાં અનેક મૂલ્યો]}
    B {-{-} C[ઉકેલ: દરેક મૂલ્ય અલગ રોમાં]}
    C {-{-} D[1NF ટેબલ]}
\end{verbatim}


{\def\LTcaptype{none} % do not increment counter
\vspace{-5pt}
\captionof{table}{1NF પછી}
\vspace{-10pt}
\begin{longtable}[]{@{}lll@{}}
\toprule\noalign{}
student\_id & name & course \\
\midrule\noalign{}
\endhead
\bottomrule\noalign{}
\endlastfoot
S1 & John & Math \\
S1 & John & Physics \\
S2 & Mary & Chemistry \\
S2 & Mary & Biology \\
S2 & Mary & Physics \\
S3 & Tim & Computer Science \\
\end{longtable}
}

\end{solutionbox}
\begin{mnemonicbox}
``ASAV'' - Atomic Single-value Attributes only Valid

\end{mnemonicbox}
\subsection*{પ્રશ્ન 5(અ) [3
માર્ક્સ]}\label{uxaaauxab0uxab6uxaa8-5uxa85-3-uxaaeuxab0uxa95uxab8}

\textbf{Transaction નો concept ઉદાહરણ સાથે સમજાવો.}

\begin{solutionbox}

\textbf{Transaction}: એક લૉજિકલ કાર્ય એકમ જે સંપૂર્ણપણે અમલમાં મૂકવામાં આવે અથવા
સંપૂર્ણપણે રદ કરવામાં આવે.


{\def\LTcaptype{none} % do not increment counter
\vspace{-5pt}
\captionof{table}{Transaction ગુણધર્મો}
\vspace{-10pt}
\begin{longtable}[]{@{}ll@{}}
\toprule\noalign{}
ગુણધર્મ & વર્ણન \\
\midrule\noalign{}
\endhead
\bottomrule\noalign{}
\endlastfoot
Atomicity & બધા ઓપરેશન સફળતાપૂર્વક પૂર્ણ થાય અથવા કોઈ નહીં \\
Consistency & ટ્રાન્ઝેક્શન પહેલાં અને પછી ડેટાબેઝ સુસંગત સ્થિતિમાં રહે \\
Isolation & સમાંતર ટ્રાન્ઝેક્શન એકબીજામાં દખલ ન કરે \\
Durability & સફળ ટ્રાન્ઝેક્શન પછી પણ ફેરફાર ટકી રહે \\
\end{longtable}
}

\textbf{ઉદાહરણ:}

\begin{verbatim}
{-{-} બેંક અકાઉન્ટ ટ્રાન્સફર ટ્રાન્ઝેક્શન}
BEGIN TRANSACTION;
    {-{-} અકાઉન્ટ A માંથી $500 કાઢવા}
    UPDATE accounts SET balance = balance {-} 500 WHERE account\_id = {A};
    
    {-{-} અકાઉન્ટ B માં $500 ઉમેરવા}
    UPDATE accounts SET balance = balance + 500 WHERE account\_id = {B};
    
    {-{-} જો બંને ઓપરેશન સફળ હોય તો}
    COMMIT;
    {-{-} જો કોઈ ઓપરેશન નિષ્ફળ જાય તો}
    {-{-} ROLLBACK;}
END TRANSACTION;
\end{verbatim}

\end{solutionbox}
\begin{mnemonicbox}
``ACID'' - Atomicity Consistency Isolation
Durability

\end{mnemonicbox}
\subsection*{પ્રશ્ન 5(બ) [4
માર્ક્સ]}\label{uxaaauxab0uxab6uxaa8-5uxaac-4-uxaaeuxab0uxa95uxab8}

\textbf{equi join સિન્ટેક્સ અને ઉદાહરણ સાથે સમજાવો.}

\begin{solutionbox}

\textbf{Equi Join}: એક જોઈન જે સામાન્ય ફીલ્ડના આધારે બે કે વધુ ટેબલના રેકોર્ડને મેચ
કરવા માટે સમાનતા તુલના ઓપરેટરનો ઉપયોગ કરે છે.

\textbf{સિન્ટેક્સ:}

\begin{verbatim}
SELECT columns
FROM table1, table2 
WHERE table1.column = table2.column;

{-{-} વૈકલ્પિક સિન્ટેક્સ (સ્પષ્ટ JOIN)}
SELECT columns
FROM table1 JOIN table2
ON table1.column = table2.column;
\end{verbatim}

\textbf{ટેબલ ઉદાહરણ:} Employees ટેબલ:

{\def\LTcaptype{none} % do not increment counter
\begin{longtable}[]{@{}lll@{}}
\toprule\noalign{}
emp\_id & name & dept\_id \\
\midrule\noalign{}
\endhead
\bottomrule\noalign{}
\endlastfoot
101 & Alice & 1 \\
102 & Bob & 2 \\
103 & Carol & 1 \\
\end{longtable}
}

Departments ટેબલ:

{\def\LTcaptype{none} % do not increment counter
\begin{longtable}[]{@{}lll@{}}
\toprule\noalign{}
dept\_id & dept\_name & location \\
\midrule\noalign{}
\endhead
\bottomrule\noalign{}
\endlastfoot
1 & HR & New York \\
2 & IT & Chicago \\
3 & Finance & Boston \\
\end{longtable}
}

\textbf{કોડબ્લોક:}

\begin{verbatim}
{-{-} Equi Join ઉદાહરણ}
SELECT e.name, d.dept\_name, d.location
FROM employees e, departments d
WHERE e.dept\_id = d.dept\_id;
\end{verbatim}

\textbf{પરિણામ:}

{\def\LTcaptype{none} % do not increment counter
\begin{longtable}[]{@{}lll@{}}
\toprule\noalign{}
name & dept\_name & location \\
\midrule\noalign{}
\endhead
\bottomrule\noalign{}
\endlastfoot
Alice & HR & New York \\
Bob & IT & Chicago \\
Carol & HR & New York \\
\end{longtable}
}

\textbf{આકૃતિ:}

\begin{center}
\textbf{Mermaid Diagram (Code)}
\begin{verbatim}
{Shaded}
{Highlighting}[]
graph TD
    subgraph Employees
    E1[emp\_id: 101{br /{}name: Alice{}br /{}dept\_id: 1]}
    E2[emp\_id: 102{br /{}name: Bob{}br /{}dept\_id: 2]}
    E3[emp\_id: 103{br /{}name: Carol{}br /{}dept\_id: 1]}
    end

    subgraph Departments
    D1[dept\_id: 1{br /{}dept\_name: HR{}br /{}location: New York]}
    D2[dept\_id: 2{br /{}dept\_name: IT{}br /{}location: Chicago]}
    D3[dept\_id: 3{br /{}dept\_name: Finance{}br /{}location: Boston]}
    end
    
    E1{-{-}{}|સમાન|D1}
    E2{-{-}{}|સમાન|D2}
    E3{-{-}{}|સમાન|D1}
{Highlighting}
{Shaded}
\end{verbatim}
\end{center}

\end{solutionbox}
\begin{mnemonicbox}
``MEET'' - Match Equal Elements Every Table

\end{mnemonicbox}
\subsection*{પ્રશ્ન 5(ક) [7
માર્ક્સ]}\label{uxaaauxab0uxab6uxaa8-5uxa95-7-uxaaeuxab0uxa95uxab8}

\textbf{Conflict serializability વિસ્તારથી સમજાવો.}

\begin{solutionbox}

\textbf{Conflict Serializability}: સમાંતર ટ્રાન્ઝેક્શનની સાચી કાર્યપ્રણાલી
સુનિશ્ચિત કરવાની એક રીત, જે એ ગેરંટી આપે છે કે એક્ઝિક્યુશન શેડ્યૂલ કોઈ સીરિયલ
એક્ઝિક્યુશનના સમકક્ષ છે.


{\def\LTcaptype{none} % do not increment counter
\vspace{-5pt}
\captionof{table}{Conflict Serializability ના મુખ્ય ખ્યાલો}
\vspace{-10pt}
\begin{longtable}[]{@{}
  >{\raggedright\arraybackslash}p{(\linewidth - 2\tabcolsep) * \real{0.4091}}
  >{\raggedright\arraybackslash}p{(\linewidth - 2\tabcolsep) * \real{0.5909}}@{}}
\toprule\noalign{}
\begin{minipage}[b]{\linewidth}\raggedright
ખ્યાલ
\end{minipage} & \begin{minipage}[b]{\linewidth}\raggedright
વર્ણન
\end{minipage} \\
\midrule\noalign{}
\endhead
\bottomrule\noalign{}
\endlastfoot
Conflicting Operations & બે ઓપરેશન કોન્ફ્લિક્ટ કરે છે જો તેઓ એક જ ડેટા આઇટમ ઍક્સેસ
કરે અને ઓછામાં ઓછું એક રાઇટ હોય \\
Precedence Graph & સંઘર્ષો દર્શાવતો ડાયરેક્ટેડ ગ્રાફ \\
Conflict Serializable & શેડ્યૂલ conflict serializable છે જો તેનો precedence
graph એસાઇક્લિક હોય \\
\end{longtable}
}

\textbf{આકૃતિ:}

\begin{center}
\textbf{Mermaid Diagram (Code)}
\begin{verbatim}
{Shaded}
{Highlighting}[]
graph LR
    A[Conflict Serializable શેડ્યૂલ] {-{-}{} B\{શું precedence graph એસાઇક્લિક છે?\}}
    B {-{-}{}|હા| C[કોઈ સીરિયલ શેડ્યૂલના સમકક્ષ]}
    B {-{-}{}|ના| D[conflict serializable નથી]}

    subgraph "ઉદાહરણ Precedence Graph"
    direction LR
    T1 {-{-}{} T2}
    T2 {-{-}{} T3}
    end
    
    subgraph "સાયકલ ઉદાહરણ (Serializable નથી)"
    direction LR
    T4 {-{-}{} T5}
    T5 {-{-}{} T6}
    T6 {-{-}{} T4}
    end
{Highlighting}
{Shaded}
\end{verbatim}
\end{center}

\textbf{ઉદાહરણ:} ટ્રાન્ઝેક્શન T1 અને T2 ધ્યાનમાં લો:

\begin{itemize}
\tightlist
\item
  T1: Read(A), Write(A)
\item
  T2: Read(A), Write(A)
\end{itemize}

શેડ્યૂલ S1: R1(A), W1(A), R2(A), W2(A) - Serializable (T1\rightarrowT2 સમકક્ષ) શેડ્યૂલ
S2: R1(A), R2(A), W1(A), W2(A) - Not serializable (precedence ગ્રાફમાં
સાયકલ છે)

\textbf{Conflict Serializability નક્કી કરવાના પગલાં:}

\begin{enumerate}
\tightlist
\item
  બધા કોન્ફ્લિક્ટિંગ ઓપરેશન જોડીઓ ઓળખો
\item
  precedence ગ્રાફ બનાવો
\item
  ચેક કરો કે ગ્રાફમાં સાયકલ છે કે નહીં
\item
  જો સાયકલ ન હોય, તો શેડ્યૂલ conflict serializable છે
\end{enumerate}

\end{solutionbox}
\begin{mnemonicbox}
``COPS'' - Conflicts, Operations, Precedence,
Serializability

\end{mnemonicbox}
\subsection*{પ્રશ્ન 5(અ) OR [3
માર્ક્સ]}\label{uxaaauxab0uxab6uxaa8-5uxa85-or-3-uxaaeuxab0uxa95uxab8}

\textbf{Transaction નાં ગુણધર્મો ઉદાહરણ સાથે સમજાવો.}

\begin{solutionbox}

\textbf{ટ્રાન્ઝેક્શનના ACID ગુણધર્મો:}


{\def\LTcaptype{none} % do not increment counter
\vspace{-5pt}
\captionof{table}{ACID ગુણધર્મો}
\vspace{-10pt}
\begin{longtable}[]{@{}
  >{\raggedright\arraybackslash}p{(\linewidth - 4\tabcolsep) * \real{0.3125}}
  >{\raggedright\arraybackslash}p{(\linewidth - 4\tabcolsep) * \real{0.4062}}
  >{\raggedright\arraybackslash}p{(\linewidth - 4\tabcolsep) * \real{0.2812}}@{}}
\toprule\noalign{}
\begin{minipage}[b]{\linewidth}\raggedright
ગુણધર્મ
\end{minipage} & \begin{minipage}[b]{\linewidth}\raggedright
વર્ણન
\end{minipage} & \begin{minipage}[b]{\linewidth}\raggedright
ઉદાહરણ
\end{minipage} \\
\midrule\noalign{}
\endhead
\bottomrule\noalign{}
\endlastfoot
Atomicity & બધા ઓપરેશન સફળતાપૂર્વક પૂર્ણ થાય અથવા કોઈ નહીં & બેંક ટ્રાન્સફર - ડેબિટ
અને ક્રેડિટ બંને એકસાથે સફળ થવા જોઈએ અથવા નિષ્ફળ થવા જોઈએ \\
Consistency & ટ્રાન્ઝેક્શન પહેલાં અને પછી ડેટાબેઝ સુસંગત સ્થિતિમાં રહે & \$100 ટ્રાન્સફર
કર્યા પછી, સિસ્ટમમાં કુલ પૈસા અપરિવર્તિત રહે \\
Isolation & સમાંતર ટ્રાન્ઝેક્શન એકબીજામાં દખલ ન કરે & ટ્રાન્ઝેક્શન A ટ્રાન્ઝેક્શન B ના
આંશિક પરિણામો જોતું નથી \\
Durability & એકવાર કમિટ થયા પછી, ફેરફારો કાયમી છે & પાવર ફેલ્યોર પણ કમિટેડ
ટ્રાન્ઝેક્શનને ખોવાતું નથી \\
\end{longtable}
}

\textbf{આકૃતિ:}

\begin{center}
\textbf{Mermaid Diagram (Code)}
\begin{verbatim}
{Shaded}
{Highlighting}[]
graph TD
    A[ACID ગુણધર્મો] {-{-}{} B[Atomicity]}
    A {-{-}{} C[Consistency]}
    A {-{-}{} D[Isolation]}
    A {-{-}{} E[Durability]}

    B {-{-}{} B1[All or Nothing]}
    C {-{-}{} C1[Valid State Transition]}
    D {-{-}{} D1[Concurrent Execution]}
    E {-{-}{} E1[Permanent Changes]}
{Highlighting}
{Shaded}
\end{verbatim}
\end{center}

\textbf{ઉદાહરણ:}

\begin{verbatim}
{-{-} ATM Withdrawal ટ્રાન્ઝેક્શન}
BEGIN TRANSACTION;
    {-{-} બેલેન્સ ચેક કરો}
    SELECT balance FROM accounts WHERE account\_id = {A123};
    
    {-{-} જો પૂરતું હોય, તો બેલેન્સ અપડેટ કરો}
    UPDATE accounts SET balance = balance {-} 100 WHERE account\_id = {A123};
    
    {-{-} ઉપાડની નોંધ કરો}
    INSERT INTO transactions (account\_id, type, amount, date)
    VALUES ({A123}, {WITHDRAWAL}, 100, SYSDATE);
    
    {-{-} જો બધા ઓપરેશન સફળ હોય તો}
    COMMIT;
    {-{-} જો કોઈ ઓપરેશન નિષ્ફળ જાય તો}
    {-{-} ROLLBACK;}
END TRANSACTION;
\end{verbatim}

\end{solutionbox}
\begin{mnemonicbox}
``ACID'' - Atomicity Consistency Isolation
Durability

\end{mnemonicbox}
\subsection*{પ્રશ્ન 5(બ) OR [4
માર્ક્સ]}\label{uxaaauxab0uxab6uxaa8-5uxaac-or-4-uxaaeuxab0uxa95uxab8}

\textbf{ઉપર Q.5 (b) માં આપેલ ``Faculty'' અને ``CT'' ટેબલનો ઉપયોગ કરીને સેટ
ઓપરેટર દ્વારા નીચેની Query લખો.} \textbf{૧. Faculty અથવા CT હોય તેવા
વ્યક્તિઓની યાદી બનાવો.} \textbf{૨. Faculty અને CT હોય તેવા વ્યક્તિઓની યાદી
બનાવો.} \textbf{૩. માત્ર Faculty હોય તેવા વ્યક્તિઓની યાદી બનાવો.} \textbf{૪.
માત્ર CT હોય તેવા વ્યક્તિઓની યાદી બનાવો.}

\begin{solutionbox}

\textbf{ટેબલ ડેટા:} Faculty ટેબલ:

{\def\LTcaptype{none} % do not increment counter
\begin{longtable}[]{@{}lll@{}}
\toprule\noalign{}
FacultyName & ErNo & Dept \\
\midrule\noalign{}
\endhead
\bottomrule\noalign{}
\endlastfoot
Prakash & FC01 & ICT \\
Ronak & FC02 & IT \\
Rakesh & FC03 & EC \\
Kinjal & FC04 & ICT \\
\end{longtable}
}

CT (ક્લાસ ટીચર) ટેબલ:

{\def\LTcaptype{none} % do not increment counter
\begin{longtable}[]{@{}ll@{}}
\toprule\noalign{}
Dept & CTName \\
\midrule\noalign{}
\endhead
\bottomrule\noalign{}
\endlastfoot
EC & Rakesh \\
CE & Jigar \\
ICT & Prakash \\
IT & Gunjan \\
\end{longtable}
}

\textbf{કોડબ્લોક:}

\begin{verbatim}
{-{-} ૧. Faculty અથવા CT હોય તેવા વ્યક્તિઓની યાદી બનાવો}
SELECT FacultyName AS Name FROM Faculty
UNION
SELECT CTName AS Name FROM CT;

{-{-} ૨. Faculty અને CT હોય તેવા વ્યક્તિઓની યાદી બનાવો}
SELECT FacultyName AS Name FROM Faculty
INTERSECT
SELECT CTName AS Name FROM CT;

{-{-} ૩. માત્ર Faculty હોય તેવા વ્યક્તિઓની યાદી બનાવો}
SELECT FacultyName AS Name FROM Faculty
MINUS
SELECT CTName AS Name FROM CT;

{-{-} ૪. માત્ર CT હોય તેવા વ્યક્તિઓની યાદી બનાવો}
SELECT CTName AS Name FROM CT
MINUS
SELECT FacultyName AS Name FROM Faculty;
\end{verbatim}

\textbf{આકૃતિ:}

\begin{center}
\textbf{Mermaid Diagram (Code)}
\begin{verbatim}
{Shaded}
{Highlighting}[]
graph TD
    subgraph Faculty
        F1[Ronak]
        F2[Kinjal]
        Both1[Prakash]
        Both2[Rakesh]
    end

    subgraph CT
        C1[Jigar]
        C2[Gunjan]
        Both1
        Both2
    end
{Highlighting}
{Shaded}
\end{verbatim}
\end{center}

\textbf{પરિણામો:}

\begin{enumerate}
\tightlist
\item
  UNION: Prakash, Ronak, Rakesh, Kinjal, Jigar, Gunjan
\item
  INTERSECT: Prakash, Rakesh
\item
  MINUS (Faculty - CT): Ronak, Kinjal
\item
  MINUS (CT - Faculty): Jigar, Gunjan
\end{enumerate}

\end{solutionbox}
\begin{mnemonicbox}
``UIMM'' - Union Intersect Minus Minus

\end{mnemonicbox}
\subsection*{પ્રશ્ન 5(ક) OR [7
માર્ક્સ]}\label{uxaaauxab0uxab6uxaa8-5uxa95-or-7-uxaaeuxab0uxa95uxab8}

\textbf{View serializability વિસ્તારથી સમજાવો.}

\begin{solutionbox}

\textbf{View Serializability}: એક શેડ્યૂલ view serializable છે જો તે કોઈ
સીરિયલ શેડ્યૂલના view equivalent હોય, એટલે કે તે ડેટાબેઝની એક જ ``દૃશ્ય'' (અથવા
અંતિમ સ્થિતિ) ઉત્પન્ન કરે.


{\def\LTcaptype{none} % do not increment counter
\vspace{-5pt}
\captionof{table}{Conflict Serializability સાથે તુલના}
\vspace{-10pt}
\begin{longtable}[]{@{}
  >{\raggedright\arraybackslash}p{(\linewidth - 4\tabcolsep) * \real{0.1455}}
  >{\raggedright\arraybackslash}p{(\linewidth - 4\tabcolsep) * \real{0.3818}}
  >{\raggedright\arraybackslash}p{(\linewidth - 4\tabcolsep) * \real{0.4727}}@{}}
\toprule\noalign{}
\begin{minipage}[b]{\linewidth}\raggedright
પાસું
\end{minipage} & \begin{minipage}[b]{\linewidth}\raggedright
View Serializability
\end{minipage} & \begin{minipage}[b]{\linewidth}\raggedright
Conflict Serializability
\end{minipage} \\
\midrule\noalign{}
\endhead
\bottomrule\noalign{}
\endlastfoot
વ્યાખ્યા & રીડ અને રાઇટના અંતિમ પરિણામો પર આધારિત & ઓપરેશન વચ્ચેના કોન્ફ્લિક્ટ પર
આધારિત \\
શરત & પ્રારંભિક રીડ, અંતિમ લખાણ, અને રીડ-રાઇટ ડિપેન્ડન્સી જાળવે છે & ઓપરેશન વચ્ચેના
બધા કોન્ફ્લિક્ટ જાળવે છે \\
સ્કોપ & શેડ્યૂલનો વ્યાપક વર્ગ & view serializable શેડ્યૂલનો સબસેટ \\
ટેસ્ટિંગ & પરીક્ષણ વધુ જટિલ & precedence ગ્રાફ વડે ટેસ્ટ કરી શકાય \\
\end{longtable}
}

\textbf{આકૃતિ:}

\begin{center}
\textbf{Mermaid Diagram (Code)}
\begin{verbatim}
{Shaded}
{Highlighting}[]
graph LR
    A[View Serializable શેડ્યૂલ] {-{-}{}|subset of| B[તમામ શક્ય શેડ્યૂલ]}
    C[Conflict Serializable શેડ્યૂલ] {-{-}{}|subset of| A}

    subgraph "View Equivalence આવશ્યકતાઓ"
    D[પ્રારંભિક રીડ મેચ]
    E[અંતિમ રાઇટ મેચ]
    F[રીડ{-રાઇટ ડિપેન્ડન્સી મેચ]}
    end
{Highlighting}
{Shaded}
\end{verbatim}
\end{center}

\textbf{View Equivalence શરતો:}

\begin{enumerate}
\tightlist
\item
  પ્રારંભિક રીડ: જો T1 શેડ્યૂલ S1 માં ડેટા આઇટમ A ની પ્રારંભિક વેલ્યુ વાંચે છે, તો તેણે
  S2 માં પણ પ્રારંભિક વેલ્યુ વાંચવી જોઈએ.
\item
  અંતિમ રાઇટ: જો T1 શેડ્યૂલ S1 માં ડેટા આઇટમ A પર અંતિમ લખાણ કરે છે, તો તેણે S2 માં
  પણ અંતિમ લખાણ કરવું જોઈએ.
\item
  રીડ-રાઇટ ડિપેન્ડન્સી: જો T1 શેડ્યૂલ S1 માં T2 દ્વારા લખાયેલ A ની વેલ્યુ વાંચે છે, તો
  તેણે S2 માં પણ T2 દ્વારા લખાયેલ વેલ્યુ વાંચવી જોઈએ.
\end{enumerate}

\textbf{ઉદાહરણ - View Serializable પરંતુ Conflict Serializable નહીં:}
બ્લાઇન્ડ રાઇટ (વાંચ્યા વિના લખાણ) ધરાવતા ટ્રાન્ઝેક્શન ધ્યાનમાં લો:

\begin{itemize}
\tightlist
\item
  T1: W1(A)
\item
  T2: W2(A)
\end{itemize}

શેડ્યૂલ S: W1(A), W2(A) - T1\rightarrowT2 અને T2\rightarrowT1 બંને માટે view serializable છે (અંતિમ
લખાણ હંમેશા T2 દ્વારા થાય છે) પરંતુ W1(A) અને W2(A) કોન્ફ્લિક્ટ કરે છે, એટલે કોન્ફ્લિક્ટ
ગ્રાફમાં બંને દિશામાં એજ હશે.

\end{solutionbox}
\begin{mnemonicbox}
``IRF'' - Initial reads, Result writes, Final view

\end{mnemonicbox}

\end{document}
