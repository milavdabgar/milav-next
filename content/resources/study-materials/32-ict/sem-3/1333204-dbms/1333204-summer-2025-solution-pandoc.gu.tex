\documentclass[10pt,a4paper]{article}

% content/resources/templates/preamble.tex
\usepackage[margin=0.6in]{geometry}
\author{Milav Dabgar}
\usepackage{amsmath,amssymb,amsthm}
\usepackage{booktabs}
\usepackage{multirow}
\usepackage{xcolor}
\usepackage{tcolorbox}
\tcbuselibrary{breakable,skins}
\usepackage[colorlinks=true,linkcolor=blue]{hyperref}
\usepackage{titlesec}
\usepackage{enumitem}
\usepackage{tikz}
\usepackage{pgfplots}
\usepackage{circuitikz}
\usepackage[version=4]{mhchem}
\usepackage{longtable}
\usepackage{array}
\usepackage{float}
\usepackage{caption}
\usepackage{listings}

\lstset{
  basicstyle=\small\ttfamily,
  breaklines=true,
  breakatwhitespace=false,
  postbreak=\mbox{\textcolor{red}{$\hookrightarrow$}\space},
  float=false,
  numbers=left,
  numberstyle=\tiny\color{gray},
  numbersep=10pt,
  xleftmargin=2em,
  keywordstyle=\color{blue},
  commentstyle=\color{green!60!black},
  stringstyle=\color{purple},
  backgroundcolor=\color{gray!5},
  showstringspaces=false,
  tabsize=2,
  captionpos=b,
  keepspaces=true,
  columns=flexible
}

\pgfplotsset{compat=1.18}
\usetikzlibrary{shapes,arrows,positioning,calc,patterns,decorations.pathmorphing,decorations.markings,arrows.meta}

% Color scheme
\definecolor{headcolor}{RGB}{0,102,204}
\definecolor{keycolor}{RGB}{220,20,60}
\definecolor{solutioncolor}{RGB}{34,139,34}
\definecolor{mnemoniccolor}{RGB}{148,0,211}
\definecolor{codecolor}{RGB}{0,0,100}

% Spacing
\setlength{\parskip}{3pt}
\setlist[itemize]{nosep}
\setlist[enumerate]{nosep}

% Title formatting
\titleformat{\section}{\Large\bfseries\color{headcolor}}{\thesection}{1em}{}
\titleformat{\subsection}{\large\bfseries\color{headcolor}}{\thesubsection}{1em}{}

% Pandoc tightlist compatibility
\providecommand{\tightlist}{%
  \setlength{\itemsep}{0pt}\setlength{\parskip}{0pt}}

% Pandoc longtable compatibility
\newcounter{none}
\def\thenone{}


% content/resources/templates/gujarati-boxes.tex
\usepackage{fontspec}
\usepackage{polyglossia}

% Set Gujarati as main language (document is primarily in Gujarati)
% Note: gloss-gujarati.ldf doesn't exist in polyglossia, but it will use hyphenation patterns
\setdefaultlanguage{gujarati}
\setotherlanguage{english}

% Configure Gujarati font properly
% Use Language=Default to prevent polyglossia from trying to add language-specific features
% that don't exist for Gujarati, which causes "empty feature" warnings
\newfontfamily\gujaratifont[Script=Gujarati,AutoFakeBold=2.5,AutoFakeSlant=0.3]{Noto Sans Gujarati}
\setmainfont[Script=Gujarati,AutoFakeBold=2.5,AutoFakeSlant=0.3]{Noto Sans Gujarati}
% Use Noto Sans Gujarati for monospace to support Gujarati in text
\setmonofont[Scale=0.9]{Noto Sans Gujarati}

% Configure English to use the same font
\newfontfamily\englishfont[Script=Gujarati,AutoFakeBold=2.5,AutoFakeSlant=0.3]{Noto Sans Gujarati}

% Translations for polyglossia
\gappto\captionsgujarati{
  \renewcommand{\tablename}{કોષ્ટક}
  \renewcommand{\figurename}{આકૃતિ}
}

% Helper for TikZ nodes to ensure Gujarati font
\newcommand{\gu}[1]{{\gujaratifont #1}}

% Custom environments
\newtcolorbox{solutionbox}{
    breakable,
    enhanced,
    colback=solutioncolor!5!white,
    colframe=solutioncolor!75!black,
    fonttitle=\bfseries,
    title=જવાબ
}

\newtcolorbox{solutionboxnobreak}{
 colback=solutioncolor!5!white,
 colframe=solutioncolor!75!black,
 fonttitle=\bfseries,
 title=જવાબ
}

\newtcolorbox{keyformula}{
 breakable,
 enhanced,
 colback=keycolor!5!white,
 colframe=keycolor!75!black,
 fonttitle=\bfseries,
 title=રાસાયણિક સમીકરણ/સૂત્ર
}

\newtcolorbox{mnemonicbox}{
 breakable,
 enhanced,
 colback=mnemoniccolor!5!white,
 colframe=mnemoniccolor!75!black,
 fonttitle=\bfseries,
 title=મેમરી ટ્રીક
}


\begin{document}

\begin{center}
{\Huge\bfseries\color{headcolor} Subject Name (Gujarati)}\\[5pt]
{\LARGE 1333204 -- Summer 2025}\\[3pt]
{\large Semester 1 Study Material}\\[3pt]
{\normalsize\textit{Detailed Solutions and Explanations}}
\end{center}

\vspace{10pt}

\subsection*{પ્રશ્ન 1(અ) [3
ગુણ]}\label{uxaaauxab0uxab6uxaa8-1uxa85-3-uxa97uxaa3}

\textbf{ટૂંકી નોંધ લખો: ડેટા ડિક્શનરી}

\begin{solutionbox}
\textbf{ડેટા ડિક્શનરી} એ કેન્દ્રીય ભંડાર છે જે ડેટાબેઝ બંધારણ, તત્વો
અને સંબંધો વિશે મેટાડેટા સંગ્રહિત કરે છે.


{\def\LTcaptype{none} % do not increment counter
\vspace{-5pt}
\captionof{table}{ડેટા ડિક્શનરી ઘટકો}
\vspace{-10pt}
\begin{longtable}[]{@{}ll@{}}
\toprule\noalign{}
ઘટક & વર્ણન \\
\midrule\noalign{}
\endhead
\bottomrule\noalign{}
\endlastfoot
\textbf{ટેબલ નામો} & ડેટાબેઝમાં બધા ટેબલોની યાદી \\
\textbf{કૉલમ વિગતો} & ડેટા પ્રકારો, મર્યાદાઓ, લંબાઈ \\
\textbf{સંબંધો} & ફોરેન કી કનેક્શન્સ \\
\textbf{ઇન્ડેક્સ} & પ્રદર્શન ઑપ્ટિમાઇઝેશન બંધારણો \\
\end{longtable}
}

\textbf{મુખ્ય લક્ષણો:}

\begin{itemize}
\tightlist
\item
  \textbf{મેટાડેટા સ્ટોરેજ}: ડેટા બંધારણ વિશે માહિતી સમાવે છે
\item
  \textbf{ડેટા અખંડિતતા}: સુસંગતતા નિયમો અને મર્યાદાઓ જાળવે છે
\item
  \textbf{દસ્તાવેજીકરણ}: વ્યાપક ડેટાબેઝ દસ્તાવેજીકરણ પ્રદાન કરે છે
\end{itemize}

\end{solutionbox}
\begin{mnemonicbox}
``ડેટા ડિક્શનરી વિગતો આપે''

\end{mnemonicbox}
\subsection*{પ્રશ્ન 1(બ) [4
ગુણ]}\label{uxaaauxab0uxab6uxaa8-1uxaac-4-uxa97uxaa3}

\textbf{વ્યાખ્યા આપો (i) E-R મોડેલ (ii) એન્ટિટી (iii) એન્ટિટી સેટ અને (iv)
ગુણધર્મો}

\begin{solutionbox}


{\def\LTcaptype{none} % do not increment counter
\vspace{-5pt}
\captionof{table}{ER મોડેલ વ્યાખ્યાઓ}
\vspace{-10pt}
\begin{longtable}[]{@{}ll@{}}
\toprule\noalign{}
શબ્દ & વ્યાખ્યા \\
\midrule\noalign{}
\endhead
\bottomrule\noalign{}
\endlastfoot
\textbf{E-R મોડેલ} & એન્ટિટી અને સંબંધોનો ઉપયોગ કરતો કન્સેપ્ચ્યુઅલ ડેટા મોડેલ \\
\textbf{એન્ટિટી} & સ્વતંત્ર અસ્તિત્વ ધરાવતો વાસ્તવિક વિશ્વનો ઑબ્જેક્ટ \\
\textbf{એન્ટિટી સેટ} & સમાન પ્રકારની સમાન એન્ટિટીઓનો સંગ્રહ \\
\textbf{ગુણધર્મો} & એન્ટિટીની લાક્ષણિકતાઓનું વર્ણન કરતા ગુણધર્મો \\
\end{longtable}
}

\textbf{આકૃતિ: ER મોડેલ ઘટકો}

\begin{verbatim}
    +{-{-}{-}{-}{-}{-}{-}{-}{-}{-}+     +{-}{-}{-}{-}{-}{-}{-}{-}{-}{-}{-}{-}{-}+     +{-}{-}{-}{-}{-}{-}{-}{-}{-}{-}+}
    |  Entity  |{-{-}{-}{-}{-}| Relationship|{-}{-}{-}{-}{-}|  Entity  |}
    |    A     |     |             |     |    B     |
    +{-{-}{-}{-}{-}{-}{-}{-}{-}{-}+     +{-}{-}{-}{-}{-}{-}{-}{-}{-}{-}{-}{-}{-}+     +{-}{-}{-}{-}{-}{-}{-}{-}{-}{-}+}
         |                                    |
    Attributes                           Attributes
\end{verbatim}

\textbf{મુખ્ય મુદ્દાઓ:}

\begin{itemize}
\tightlist
\item
  \textbf{કન્સેપ્ચ્યુઅલ ડિઝાઇન}: ઉચ્ચ સ્તરનો ડેટાબેઝ ડિઝાઇન અભિગમ
\item
  \textbf{વિઝ્યુઅલ રજૂઆત}: સ્પષ્ટ સમજ માટે આકૃતિઓનો ઉપયોગ
\end{itemize}

\end{solutionbox}
\begin{mnemonicbox}
``એન્ટિટી સંબંધો અર્થપૂર્ણ રીતે''

\end{mnemonicbox}
\subsection*{પ્રશ્ન 1(ક) [7
ગુણ]}\label{uxaaauxab0uxab6uxaa8-1uxa95-7-uxa97uxaa3}

\textbf{DBMS ના ફાયદા સમજાવો}

\begin{solutionbox}


{\def\LTcaptype{none} % do not increment counter
\vspace{-5pt}
\captionof{table}{DBMS ફાયદા}
\vspace{-10pt}
\begin{longtable}[]{@{}ll@{}}
\toprule\noalign{}
ફાયદો & લાભ \\
\midrule\noalign{}
\endhead
\bottomrule\noalign{}
\endlastfoot
\textbf{ડેટા સ્વતંત્રતા} & એપ્લિકેશન ડેટા સ્ટ્રક્ચર ફેરફારોથી અલગ \\
\textbf{ડેટા શેરિંગ} & બહુવિધ વપરાશકર્તાઓ એકસાથે સમાન ડેટા એક્સેસ કરે \\
\textbf{ડેટા સુરક્ષા} & એક્સેસ કંટ્રોલ અને પ્રમાણીકરણ પદ્ધતિઓ \\
\textbf{ડેટા અખંડિતતા} & મર્યાદાઓ દ્વારા સુસંગતતા જાળવવામાં આવે છે \\
\textbf{બેકઅપ અને રિકવરી} & આપોઆપ ડેટા સંરક્ષણ અને પુનઃસ્થાપન \\
\textbf{ઘટાડેલી રીડન્ડન્સી} & ડુપ્લિકેટ ડેટા સ્ટોરેજ દૂર કરે છે \\
\end{longtable}
}

\textbf{મુખ્ય લાભો:}

\begin{itemize}
\tightlist
\item
  \textbf{કેન્દ્રીકૃત નિયંત્રણ}: ડેટા વ્યવસ્થાપનનો એક બિંદુ
\item
  \textbf{ખર્ચ અસરકારકતા}: વિકાસ અને જાળવણીનો ખર્ચ ઘટાડે છે
\item
  \textbf{ડેટા સુસંગતતા}: એપ્લિકેશન્સમાં એકસમાન ડેટા સુનિશ્ચિત કરે છે
\item
  \textbf{સંગામિત એક્સેસ}: બહુવિધ વપરાશકર્તાઓ એકસાથે કામ કરી શકે છે
\item
  \textbf{ક્વેરી ઑપ્ટિમાઇઝેશન}: કાર્યક્ષમ ડેટા પુનઃપ્રાપ્તિ પદ્ધતિઓ
\end{itemize}

\end{solutionbox}
\begin{mnemonicbox}
``ડેટાબેઝ બિઝનેસને બહેતર બનાવે''

\end{mnemonicbox}
\subsection*{પ્રશ્ન 1(ક) અથવા [7
ગુણ]}\label{uxaaauxab0uxab6uxaa8-1uxa95-uxa85uxaa5uxab5-7-uxa97uxaa3}

\textbf{DBMS નું આર્કિટેક્ચર સમજાવો}

\begin{solutionbox}

\textbf{આકૃતિ: ત્રણ-સ્તરીય DBMS આર્કિટેક્ચર}

\begin{center}
\textbf{Mermaid Diagram (Code)}
\begin{verbatim}
{Shaded}
{Highlighting}[]
graph LR
    A[બાહ્ય સ્તર{br/{}વપરાશકર્તા દૃશ્યો] {-}{-}{} B[કન્સેપ્ચ્યુઅલ સ્તર{}br/{}લોજિકલ સ્કીમા]}
    B {-{-}{} C[આંતરિક સ્તર{}br/{}ભૌતિક સ્ટોરેજ]}
    D[વપરાશકર્તા 1] {-{-}{} A}
    E[વપરાશકર્તા 2] {-{-}{} A}
    F[DBA] {-{-}{} B}
    G[સિસ્ટમ] {-{-}{} C}
{Highlighting}
{Shaded}
\end{verbatim}
\end{center}


{\def\LTcaptype{none} % do not increment counter
\vspace{-5pt}
\captionof{table}{આર્કિટેક્ચર સ્તરો}
\vspace{-10pt}
\begin{longtable}[]{@{}lll@{}}
\toprule\noalign{}
સ્તર & હેતુ & વપરાશકર્તાઓ \\
\midrule\noalign{}
\endhead
\bottomrule\noalign{}
\endlastfoot
\textbf{બાહ્ય} & વ્યક્તિગત વપરાશકર્તા દૃશ્યો & અંતિમ વપરાશકર્તાઓ, એપ્લિકેશન્સ \\
\textbf{કન્સેપ્ચ્યુઅલ} & સંપૂર્ણ લોજિકલ બંધારણ & ડેટાબેઝ એડમિનિસ્ટ્રેટર \\
\textbf{આંતરિક} & ભૌતિક સ્ટોરેજ વિગતો & સિસ્ટમ પ્રોગ્રામર્સ \\
\end{longtable}
}

\textbf{મુખ્ય લક્ષણો:}

\begin{itemize}
\tightlist
\item
  \textbf{ડેટા સ્વતંત્રતા}: એક સ્તરે ફેરફારો અન્યને અસર કરતા નથી
\item
  \textbf{સુરક્ષા}: વિવિધ વપરાશકર્તાઓ માટે વિવિધ એક્સેસ સ્તરો
\item
  \textbf{અમૂર્તતા}: વપરાશકર્તાઓથી જટિલતા છુપાવે છે
\end{itemize}

\end{solutionbox}
\begin{mnemonicbox}
``બાહ્ય કન્સેપ્ચ્યુઅલ આંતરિક આર્કિટેક્ચર''

\end{mnemonicbox}
\subsection*{પ્રશ્ન 2(અ) [3
ગુણ]}\label{uxaaauxab0uxab6uxaa8-2uxa85-3-uxa97uxaa3}

\textbf{UNIQUE KEY અને PRIMARY KEY સમજાવો}

\begin{solutionbox}


{\def\LTcaptype{none} % do not increment counter
\vspace{-5pt}
\captionof{table}{કી સરખામણી}
\vspace{-10pt}
\begin{longtable}[]{@{}lll@{}}
\toprule\noalign{}
લક્ષણ & PRIMARY KEY & UNIQUE KEY \\
\midrule\noalign{}
\endhead
\bottomrule\noalign{}
\endlastfoot
\textbf{Null મૂલ્યો} & મંજૂર નથી & એક null મંજૂર \\
\textbf{ટેબલ દીઠ સંખ્યા} & માત્ર એક & બહુવિધ મંજૂર \\
\textbf{ઇન્ડેક્સ બનાવટ} & આપોઆપ clustered & આપોઆપ non-clustered \\
\textbf{હેતુ} & એન્ટિટી ઓળખ & ડેટા વિશિષ્ટતા \\
\end{longtable}
}

\textbf{મુખ્ય તફાવતો:}

\begin{itemize}
\tightlist
\item
  \textbf{પ્રાથમિક કી}: દરેક રેકોર્ડને વિશિષ્ટ રીતે ઓળખે છે, null હોઈ શકતી નથી
\item
  \textbf{યુનિક કી}: વિશિષ્ટતા સુનિશ્ચિત કરે છે પણ એક null મૂલ્યની મંજૂરી આપે છે
\end{itemize}

\end{solutionbox}
\begin{mnemonicbox}
``પ્રાથમિક નલને અટકાવે, યુનિક નલને સમજે''

\end{mnemonicbox}
\subsection*{પ્રશ્ન 2(બ) [4
ગુણ]}\label{uxaaauxab0uxab6uxaa8-2uxaac-4-uxa97uxaa3}

\textbf{ER ડાયાગ્રામમાં એન્ટિટીની Participation પર ટૂંકી નોંધ લખો}

\begin{solutionbox}


{\def\LTcaptype{none} % do not increment counter
\vspace{-5pt}
\captionof{table}{Participation પ્રકારો}
\vspace{-10pt}
\begin{longtable}[]{@{}lll@{}}
\toprule\noalign{}
પ્રકાર & વર્ણન & પ્રતીક \\
\midrule\noalign{}
\endhead
\bottomrule\noalign{}
\endlastfoot
\textbf{કુલ Participation} & દરેક એન્ટિટી સહભાગી થવી જ જોઈએ & ડબલ લાઇન \\
\textbf{આંશિક Participation} & કેટલીક એન્ટિટી સહભાગી ન પણ થઈ શકે & સિંગલ
લાઇન \\
\end{longtable}
}

\textbf{આકૃતિ: Participation ઉદાહરણ}

\begin{verbatim}
કર્મચારી ========== કામ\_કરે {-{-}{-}{-}{-}{-}{-}{-}{-}{-} વિભાગ}
  (કુલ)                                 (આંશિક)
\end{verbatim}

\textbf{મુખ્ય સંકેતો:}

\begin{itemize}
\tightlist
\item
  \textbf{ફરજિયાત Participation}: દરેક ઇન્સ્ટન્સ સંકળાયેલું હોવું જ જોઈએ
\item
  \textbf{વૈકલ્પિક Participation}: કેટલાક ઇન્સ્ટન્સ સંકળાયેલા ન હોઈ શકે
\item
  \textbf{બિઝનેસ નિયમો}: વાસ્તવિક વિશ્વની મર્યાદાઓને પ્રતિબિંબિત કરે છે
\end{itemize}

\end{solutionbox}
\begin{mnemonicbox}
``કુલ Participation બધાની જરૂર''

\end{mnemonicbox}
\subsection*{પ્રશ્ન 2(ક) [7
ગુણ]}\label{uxaaauxab0uxab6uxaa8-2uxa95-7-uxa97uxaa3}

\textbf{ER ડાયાગ્રામ માટે Generalization concept વિગતવાર વર્ણન કરો}

\begin{solutionbox}

\textbf{આકૃતિ: Generalization ઉદાહરણ}

\begin{verbatim}
erDiagram
    PERSON \{
        int person\_id
        string name
        string address
    \}
    EMPLOYEE \{
        int employee\_id
        decimal salary
        string department
    \}
    STUDENT \{
        int student\_id
        string course
        decimal fees
    \}
    PERSON ||{-{-}|| EMPLOYEE : is{-}a}
    PERSON ||{-{-}|| STUDENT : is{-}a}
\end{verbatim}


{\def\LTcaptype{none} % do not increment counter
\vspace{-5pt}
\captionof{table}{Generalization લાક્ષણિકતાઓ}
\vspace{-10pt}
\begin{longtable}[]{@{}ll@{}}
\toprule\noalign{}
પાસું & વર્ણન \\
\midrule\noalign{}
\endhead
\bottomrule\noalign{}
\endlastfoot
\textbf{બોટમ-અપ પ્રક્રિયા} & સમાન એન્ટિટીઓને સુપરક્લાસમાં જોડે છે \\
\textbf{વારસાગતતા} & સબક્લાસ સુપરક્લાસ ગુણધર્મો વારસે મેળવે છે \\
\textbf{વિશેષીકરણ} & Generalization ની વિપરીત પ્રક્રિયા \\
\textbf{ઓવરલેપ મર્યાદાઓ} & અલગ અથવા ઓવરલેપિંગ સબક્લાસ \\
\end{longtable}
}

\textbf{મુખ્ય લક્ષણો:}

\begin{itemize}
\tightlist
\item
  \textbf{ગુણધર્મ વારસાગતતા}: સામાન્ય ગુણધર્મો સુપરક્લાસમાં ખસેડવામાં આવે છે
\item
  \textbf{સંબંધ વારસાગતતા}: સંબંધો પણ વારસામાં મળે છે
\item
  \textbf{મર્યાદા પ્રકારો}: કુલ/આંશિક, અલગ/ઓવરલેપિંગ
\item
  \textbf{ISA સંબંધ}: ``is-a'' કનેક્શનને રજૂ કરે છે
\end{itemize}

\end{solutionbox}
\begin{mnemonicbox}
``સામાન્યીકરણ સમાન એન્ટિટીઓને જૂથ બનાવે''

\end{mnemonicbox}
\subsection*{પ્રશ્ન 2(અ) અથવા [3
ગુણ]}\label{uxaaauxab0uxab6uxaa8-2uxa85-uxa85uxaa5uxab5-3-uxa97uxaa3}

\textbf{ER ડાયાગ્રામમાં મેપિંગ કાર્ડિનાલિટી સમજાવો}

\begin{solutionbox}


{\def\LTcaptype{none} % do not increment counter
\vspace{-5pt}
\captionof{table}{કાર્ડિનાલિટી પ્રકારો}
\vspace{-10pt}
\begin{longtable}[]{@{}lll@{}}
\toprule\noalign{}
પ્રકાર & વર્ણન & ઉદાહરણ \\
\midrule\noalign{}
\endhead
\bottomrule\noalign{}
\endlastfoot
\textbf{એક-થી-એક (1:1)} & એક એન્ટિટી અન્ય એક સાથે સંબંધિત & વ્યક્તિ-પાસપોર્ટ \\
\textbf{એક-થી-ઘણા (1:M)} & એક એન્ટિટી ઘણા અન્ય સાથે સંબંધિત &
વિભાગ-કર્મચારી \\
\textbf{ઘણા-થી-એક (M:1)} & ઘણી એન્ટિટી એક સાથે સંબંધિત & કર્મચારી-વિભાગ \\
\textbf{ઘણા-થી-ઘણા (M:N)} & ઘણી એન્ટિટી ઘણા સાથે સંબંધિત & વિદ્યાર્થી-કોર્સ \\
\end{longtable}
}

\textbf{મુખ્ય સંકેતો:}

\begin{itemize}
\tightlist
\item
  \textbf{સંબંધ મર્યાદાઓ}: એન્ટિટી કેવી રીતે સંબંધિત થઈ શકે છે તે વ્યાખ્યાયિત કરે છે
\item
  \textbf{બિઝનેસ નિયમો}: વાસ્તવિક વિશ્વ સંબંધ મર્યાદાઓને પ્રતિબિંબિત કરે છે
\end{itemize}

\end{solutionbox}
\begin{mnemonicbox}
``એક કે ઘણા મેપિંગ મહત્વપૂર્ણ''

\end{mnemonicbox}
\subsection*{પ્રશ્ન 2(બ) અથવા [4
ગુણ]}\label{uxaaauxab0uxab6uxaa8-2uxaac-uxa85uxaa5uxab5-4-uxa97uxaa3}

\textbf{E-R ડાયાગ્રામમાં Aggregation સમજાવો}

\begin{solutionbox}

\textbf{આકૃતિ: Aggregation ઉદાહરણ}

\begin{verbatim}
    કર્મચારી {-{-}{-}{-} કામ\_કરે {-}{-}{-}{-} પ્રોજેક્ટ}
        |                         |
        +{-{-}{-}{-}{-}{-}{-}{-}{-}{-}+{-}{-}{-}{-}{-}{-}{-}{-}{-}{-}+}
                   |
               વ્યવસ્થાપન
                   |
                મેનેજર
\end{verbatim}

\textbf{મુખ્ય લક્ષણો:}

\begin{itemize}
\tightlist
\item
  \textbf{સંબંધ એન્ટિટી તરીકે}: સંબંધ સેટને એન્ટિટી તરીકે ગણે છે
\item
  \textbf{ઉચ્ચ સ્તરના સંબંધો}: સંબંધો વચ્ચે સંબંધોની મંજૂરી આપે છે
\item
  \textbf{જટિલ મોડેલિંગ}: અદ્યતન બિઝનેસ દૃશ્યોને હેન્ડલ કરે છે
\item
  \textbf{અમૂર્ત પદ્ધતિ}: જટિલ સંબંધોને સરળ બનાવે છે
\end{itemize}


{\def\LTcaptype{none} % do not increment counter
\vspace{-5pt}
\captionof{table}{Aggregation લાભો}
\vspace{-10pt}
\begin{longtable}[]{@{}ll@{}}
\toprule\noalign{}
લાભ & વર્ણન \\
\midrule\noalign{}
\endhead
\bottomrule\noalign{}
\endlastfoot
\textbf{મોડેલિંગ લવચીકતા} & જટિલ સંબંધોને હેન્ડલ કરે છે \\
\textbf{અર્થપૂર્ણ સ્પષ્ટતા} & બિઝનેસ નિયમોની સ્પષ્ટ રજૂઆત \\
\textbf{ડિઝાઇન સરળતા} & મોડેલ જટિલતા ઘટાડે છે \\
\end{longtable}
}

\end{solutionbox}
\begin{mnemonicbox}
``એકીકરણ અદ્યતન સંગઠનોને અમૂર્ત બનાવે''

\end{mnemonicbox}
\subsection*{પ્રશ્ન 2(ક) અથવા [7
ગુણ]}\label{uxaaauxab0uxab6uxaa8-2uxa95-uxa85uxaa5uxab5-7-uxa97uxaa3}

\textbf{Enhanced ER મોડેલનો ઉપયોગ કરીને લાઇબ્રેરી મેનેજમેન્ટ સિસ્ટમનો ER ડાયાગ્રામ
દોરો}

\begin{solutionbox}

\textbf{આકૃતિ: લાઇબ્રેરી મેનેજમેન્ટ સિસ્ટમ}

\begin{verbatim}
erDiagram
    PERSON \{
        int person\_id
        string name
        string address
        string phone
    \}
    MEMBER \{
        int member\_id
        date join\_date
        string membership\_type
    \}
    LIBRARIAN \{
        int employee\_id
        decimal salary
        string department
    \}
    BOOK \{
        int book\_id
        string title
        string author
        string isbn
        int copies
    \}
    CATEGORY \{
        int category\_id
        string category\_name
        string description
    \}
    TRANSACTION \{
        int transaction\_id
        date issue\_date
        date return\_date
        string status
    \}
    
    PERSON ||{-{-}|| MEMBER : is{-}a}
    PERSON ||{-{-}|| LIBRARIAN : is{-}a}
    MEMBER ||{-{-}o\{ TRANSACTION : makes}
    BOOK ||{-{-}o\{ TRANSACTION : involved\_in}
    BOOK {-}{-}|| CATEGORY : belongs\_to}
    LIBRARIAN ||{-{-}o\{ TRANSACTION : processes}
\end{verbatim}

\textbf{વપરાયેલ Enhanced ER લક્ષણો:}

\begin{itemize}
\tightlist
\item
  \textbf{સામાન્યીકરણ}: મેમ્બર અને લાઇબ્રેરિયન સબક્લાસ સાથે વ્યક્તિ સુપરક્લાસ
\item
  \textbf{વિશેષીકરણ}: વિવિધ વ્યક્તિ પ્રકારો માટે વિવિધ ગુણધર્મો
\item
  \textbf{એકીકરણ}: બહુવિધ એન્ટિટી સાથે Transaction સંબંધ
\item
  \textbf{બહુવિધ વારસાગતતા}: જટિલ સંબંધ હેન્ડલિંગ
\end{itemize}

\end{solutionbox}
\begin{mnemonicbox}
``લાઇબ્રેરી સાહિત્યને તાર્કિક રીતે જોડે''

\end{mnemonicbox}
\begin{center}\rule{0.5\linewidth}{0.5pt}\end{center}

\subsection*{પ્રશ્ન 3(અ) [3
ગુણ]}\label{uxaaauxab0uxab6uxaa8-3uxa85-3-uxa97uxaa3}

\textbf{SQL ડેટા પ્રકાર સમજાવો}

\begin{solutionbox}


{\def\LTcaptype{none} % do not increment counter
\vspace{-5pt}
\captionof{table}{સામાન્ય SQL ડેટા પ્રકારો}
\vspace{-10pt}
\begin{longtable}[]{@{}lll@{}}
\toprule\noalign{}
કેટેગરી & ડેટા પ્રકાર & વર્ણન \\
\midrule\noalign{}
\endhead
\bottomrule\noalign{}
\endlastfoot
\textbf{સંખ્યાત્મક} & INT, DECIMAL, FLOAT & સંખ્યાઓ સંગ્રહિત કરે \\
\textbf{અક્ષર} & CHAR, VARCHAR, TEXT & ટેક્સ્ટ સંગ્રહિત કરે \\
\textbf{તારીખ/સમય} & DATE, TIME, DATETIME & સમયગત ડેટા સંગ્રહિત કરે \\
\textbf{બુલિયન} & BOOLEAN & સાચું/ખોટું સંગ્રહિત કરે \\
\end{longtable}
}

\textbf{મુખ્ય મુદ્દાઓ:}

\begin{itemize}
\tightlist
\item
  \textbf{ડેટા અખંડિતતા}: યોગ્ય ડેટા સ્ટોરેજ સુનિશ્ચિત કરે છે
\item
  \textbf{સ્ટોરેજ ઑપ્ટિમાઇઝેશન}: યોગ્ય કદ ફાળવણી
\item
  \textbf{માન્યતા}: આપોઆપ ડેટા પ્રકાર તપાસ
\end{itemize}

\end{solutionbox}
\begin{mnemonicbox}
``ડેટા પ્રકારો સ્ટોરેજ વ્યાખ્યાયિત કરે''

\end{mnemonicbox}
\subsection*{પ્રશ્ન 3(બ) [4
ગુણ]}\label{uxaaauxab0uxab6uxaa8-3uxaac-4-uxa97uxaa3}

\textbf{DROP અને TRUNCATE COMMAND સરખામણી કરો}

\begin{solutionbox}


{\def\LTcaptype{none} % do not increment counter
\vspace{-5pt}
\captionof{table}{DROP vs TRUNCATE સરખામણી}
\vspace{-10pt}
\begin{longtable}[]{@{}lll@{}}
\toprule\noalign{}
લક્ષણ & DROP & TRUNCATE \\
\midrule\noalign{}
\endhead
\bottomrule\noalign{}
\endlastfoot
\textbf{ઑપરેશન} & ટેબલ સ્ટ્રક્ચર દૂર કરે & માત્ર બધો ડેટા દૂર કરે \\
\textbf{રોલબેક} & રોલબેક કરી શકાતું નથી & રોલબેક કરી શકાય (ટ્રાન્ઝેક્શનમાં) \\
\textbf{ઝડપ} & ધીમું & ઝડપી \\
\textbf{ટ્રિગર્સ} & ટ્રિગર્સ ચલાવે છે & ટ્રિગર્સ ચલાવતું નથી \\
\textbf{વ્હેર ક્લોઝ} & લાગુ નથી & સપોર્ટ કરતું નથી \\
\textbf{ઓટો-ઇન્ક્રિમેન્ટ} & રીસેટ થાય છે & પ્રારંભિક વેલ્યુ પર રીસેટ થાય છે \\
\end{longtable}
}

\textbf{કોડ ઉદાહરણો:}

\begin{verbatim}
{-{-} DROP આદેશ}
DROP TABLE student;

{-{-} TRUNCATE આદેશ  }
TRUNCATE TABLE student;
\end{verbatim}

\textbf{મુખ્ય તફાવતો:}

\begin{itemize}
\tightlist
\item
  \textbf{સ્ટ્રક્ચર પ્રભાવ}: DROP બધું દૂર કરે છે, TRUNCATE સ્ટ્રક્ચર રાખે છે
\item
  \textbf{પ્રદર્શન}: TRUNCATE મોટા ટેબલો માટે ઝડપી છે
\end{itemize}

\end{solutionbox}
\begin{mnemonicbox}
``DROP નાશ કરે, TRUNCATE કાપે''

\end{mnemonicbox}
\subsection*{પ્રશ્ન 3(ક) [7
ગુણ]}\label{uxaaauxab0uxab6uxaa8-3uxa95-7-uxa97uxaa3}

\textbf{નીચેના Relational Schema અને નીચેના પ્રશ્નો માટે Relational Algebra
Expression આપો} \textbf{વિદ્યાર્થીઓ (નામ, SPI, DOB, નોંધણી નંબર)}

\begin{solutionbox}

\textbf{રિલેશનલ આલ્જિબ્રા એક્સપ્રેશન્સ:}

\textbf{i) એવા તમામ વિદ્યાર્થીઓની યાદી બનાવો કે જેમનું SPI 6.0 કરતાં ઓછું છે:}

\begin{verbatim}
σ(SPI < 6.0)(વિદ્યાર્થીઓ)
\end{verbatim}

\textbf{ii) વિદ્યાર્થીનું નામ જેની નોંધણી નંબર 006 ધરાવે છે:}

\begin{verbatim}
π(નામ)(σ(નોંધણી_નંબર LIKE '%006%')(વિદ્યાર્થીઓ))
\end{verbatim}

\textbf{iii) સમાન DOB ધરાવતા તમામ વિદ્યાર્થીઓની યાદી બનાવો:}

\begin{verbatim}
વિદ્યાર્થીઓ ⋈ (ρ(S2)(વિદ્યાર્થીઓ)) WHERE વિદ્યાર્થીઓ.DOB = S2.DOB AND વિદ્યાર્થીઓ.નોંધણી_નંબર \neq S2.નોંધણી_નંબર
\end{verbatim}

\textbf{iv) સમાન અક્ષરથી શરૂ થતા વિદ્યાર્થીઓનું નામ દર્શાવો:}

\begin{verbatim}
π(નામ)(વિદ્યાર્થીઓ ⋈ (ρ(S2)(વિદ્યાર્થીઓ)) WHERE SUBSTR(વિદ્યાર્થીઓ.નામ,1,1) = SUBSTR(S2.નામ,1,1) AND વિદ્યાર્થીઓ.નોંધણી_નંબર \neq S2.નોંધણી_નંબર)
\end{verbatim}


{\def\LTcaptype{none} % do not increment counter
\vspace{-5pt}
\captionof{table}{વપરાયેલ રિલેશનલ આલ્જિબ્રા ઓપરેટર્સ}
\vspace{-10pt}
\begin{longtable}[]{@{}lll@{}}
\toprule\noalign{}
ઓપરેટર & પ્રતીક & હેતુ \\
\midrule\noalign{}
\endhead
\bottomrule\noalign{}
\endlastfoot
\textbf{પસંદગી} & σ & શરત આધારિત પંક્તિઓ ફિલ્ટર કરે \\
\textbf{પ્રોજેક્શન} & π & ચોક્કસ કોલમ પસંદ કરે \\
\textbf{જોઇન} & ⋈ & સંબંધિત ટ્યુપલ્સ સંયોજિત કરે \\
\textbf{નામ બદલવું} & ρ & રિલેશન્સ/એટ્રિબ્યુટ્સનું નામ બદલે \\
\end{longtable}
}

\end{solutionbox}
\begin{mnemonicbox}
``પસંદ કરો પ્રોજેક્ટ કરો જોડો નામ બદલો''

\end{mnemonicbox}
\subsection*{પ્રશ્ન 3(અ) અથવા [3
ગુણ]}\label{uxaaauxab0uxab6uxaa8-3uxa85-uxa85uxaa5uxab5-3-uxa97uxaa3}

\textbf{ઉદાહરણ સાથે Grant અને Revoke આદેશનો ઉપયોગ સમજાવો}

\begin{solutionbox}

\textbf{કોડ ઉદાહરણો:}

\begin{verbatim}
{-{-} GRANT આદેશ}
GRANT SELECT, INSERT ON student TO user1;
GRANT ALL PRIVILEGES ON database1 TO user2;

{-{-} REVOKE આદેશ  }
REVOKE INSERT ON student FROM user1;
REVOKE ALL PRIVILEGES ON database1 FROM user2;
\end{verbatim}

\textbf{મુખ્ય લક્ષણો:}

\begin{itemize}
\tightlist
\item
  \textbf{એક્સેસ કંટ્રોલ}: વપરાશકર્તા અનુમતિઓ સંચાલિત કરે છે
\item
  \textbf{સુરક્ષા}: અનધિકૃત એક્સેસ અટકાવે છે
\item
  \textbf{ગ્રેન્યુલર કંટ્રોલ}: ચોક્કસ વિશેષાધિકાર અસાઇનમેન્ટ
\end{itemize}


{\def\LTcaptype{none} % do not increment counter
\vspace{-5pt}
\captionof{table}{સામાન્ય વિશેષાધિકારો}
\vspace{-10pt}
\begin{longtable}[]{@{}ll@{}}
\toprule\noalign{}
વિશેષાધિકાર & વર્ણન \\
\midrule\noalign{}
\endhead
\bottomrule\noalign{}
\endlastfoot
\textbf{SELECT} & ડેટા વાંચે \\
\textbf{INSERT} & નવા રેકોર્ડ ઉમેરે \\
\textbf{UPDATE} & હાલનો ડેટા બદલે \\
\textbf{DELETE} & રેકોર્ડ દૂર કરે \\
\textbf{ALL} & સંપૂર્ણ એક્સેસ \\
\end{longtable}
}

\end{solutionbox}
\begin{mnemonicbox}
``Grant આપે, Revoke દૂર કરે''

\end{mnemonicbox}
\subsection*{પ્રશ્ન 3(બ) અથવા [4
ગુણ]}\label{uxaaauxab0uxab6uxaa8-3uxaac-uxa85uxaa5uxab5-4-uxa97uxaa3}

\textbf{ઉદાહરણ સાથે DML આદેશોનું વર્ણન કરો}

\begin{solutionbox}


{\def\LTcaptype{none} % do not increment counter
\vspace{-5pt}
\captionof{table}{DML આદેશો}
\vspace{-10pt}
\begin{longtable}[]{@{}
  >{\raggedright\arraybackslash}p{(\linewidth - 4\tabcolsep) * \real{0.3182}}
  >{\raggedright\arraybackslash}p{(\linewidth - 4\tabcolsep) * \real{0.2727}}
  >{\raggedright\arraybackslash}p{(\linewidth - 4\tabcolsep) * \real{0.4091}}@{}}
\toprule\noalign{}
\begin{minipage}[b]{\linewidth}\raggedright
આદેશ
\end{minipage} & \begin{minipage}[b]{\linewidth}\raggedright
હેતુ
\end{minipage} & \begin{minipage}[b]{\linewidth}\raggedright
ઉદાહરણ
\end{minipage} \\
\midrule\noalign{}
\endhead
\bottomrule\noalign{}
\endlastfoot
\textbf{INSERT} & નવા રેકોર્ડ ઉમેરે &
\texttt{INSERT\ INTO\ student\ VALUES\ (1,\textquotesingle{}John\textquotesingle{},8.5)} \\
\textbf{UPDATE} & હાલનો ડેટા બદલે &
\texttt{UPDATE\ student\ SET\ spi=9.0\ WHERE\ id=1} \\
\textbf{DELETE} & રેકોર્ડ દૂર કરે &
\texttt{DELETE\ FROM\ student\ WHERE\ spi\textless{}6.0} \\
\textbf{SELECT} & ડેટા પુનઃપ્રાપ્ત કરે &
\texttt{SELECT\ *\ FROM\ student\ WHERE\ spi\textgreater{}8.0} \\
\end{longtable}
}

\textbf{કોડ ઉદાહરણો:}

\begin{verbatim}
{-{-} INSERT આદેશ}
INSERT INTO Students (name, spi, dob) 
VALUES ({Alice}, 8.5, {2000{-}05{-}15});

{-{-} UPDATE આદેશ}
UPDATE Students SET spi = 9.0 
WHERE name = {Alice};

{-{-} DELETE આદેશ}
DELETE FROM Students 
WHERE spi {} 6.0;

{-{-} SELECT આદેશ}
SELECT name, spi FROM Students 
WHERE spi {} 8.0;
\end{verbatim}

\textbf{મુખ્ય લક્ષણો:}

\begin{itemize}
\tightlist
\item
  \textbf{ડેટા મેનિપ્યુલેશન}: મુખ્ય ડેટાબેઝ ઓપરેશન્સ
\item
  \textbf{ટ્રાન્ઝેક્શન સપોર્ટ}: રોલબેક કરી શકાય છે
\item
  \textbf{શરતી ઓપરેશન્સ}: WHERE ક્લોઝ સપોર્ટ
\end{itemize}

\end{solutionbox}
\begin{mnemonicbox}
``Insert Update Delete Select''

\end{mnemonicbox}
\subsection*{પ્રશ્ન 3(ક) અથવા [7
ગુણ]}\label{uxaaauxab0uxab6uxaa8-3uxa95-uxa85uxaa5uxab5-7-uxa97uxaa3}

\textbf{DBMS ના તમામ કન્વર્ઝન ફંક્શનની યાદી બનાવો અને તેમાંથી કોઈપણ ત્રણને
વિગતવાર સમજાવો}

\begin{solutionbox}


{\def\LTcaptype{none} % do not increment counter
\vspace{-5pt}
\captionof{table}{કન્વર્ઝન ફંક્શન્સ}
\vspace{-10pt}
\begin{longtable}[]{@{}
  >{\raggedright\arraybackslash}p{(\linewidth - 4\tabcolsep) * \real{0.3478}}
  >{\raggedright\arraybackslash}p{(\linewidth - 4\tabcolsep) * \real{0.2609}}
  >{\raggedright\arraybackslash}p{(\linewidth - 4\tabcolsep) * \real{0.3913}}@{}}
\toprule\noalign{}
\begin{minipage}[b]{\linewidth}\raggedright
ફંક્શન
\end{minipage} & \begin{minipage}[b]{\linewidth}\raggedright
હેતુ
\end{minipage} & \begin{minipage}[b]{\linewidth}\raggedright
ઉદાહરણ
\end{minipage} \\
\midrule\noalign{}
\endhead
\bottomrule\noalign{}
\endlastfoot
\textbf{TO\_CHAR} & કેરેક્ટરમાં કન્વર્ટ કરે &
\texttt{TO\_CHAR(sysdate,\ \textquotesingle{}DD-MM-YYYY\textquotesingle{})} \\
\textbf{TO\_DATE} & તારીખમાં કન્વર્ટ કરે &
\texttt{TO\_DATE(\textquotesingle{}15-05-2025\textquotesingle{},\ \textquotesingle{}DD-MM-YYYY\textquotesingle{})} \\
\textbf{TO\_NUMBER} & નંબરમાં કન્વર્ટ કરે &
\texttt{TO\_NUMBER(\textquotesingle{}123.45\textquotesingle{})} \\
\textbf{CAST} & સામાન્ય કન્વર્ઝન &
\texttt{CAST(\textquotesingle{}123\textquotesingle{}\ AS\ INTEGER)} \\
\textbf{CONVERT} & ડેટા પ્રકાર કન્વર્ઝન & \texttt{CONVERT(varchar,\ 123)} \\
\end{longtable}
}

\textbf{ત્રણ ફંક્શન્સની વિગતવાર સમજૂતી:}

\textbf{1. TO\_CHAR ફંક્શન:}

\begin{itemize}
\tightlist
\item
  \textbf{હેતુ}: તારીખો અને નંબરોને કેરેક્ટર સ્ટ્રિંગમાં કન્વર્ટ કરે છે
\item
  \textbf{સિન્ટેક્સ}: \texttt{TO\_CHAR(value,\ format)}
\item
  \textbf{ઉપયોગ}: તારીખ ફોર્મેટિંગ, ચોક્કસ પેટર્ન સાથે નંબર ફોર્મેટિંગ
\end{itemize}

\textbf{2. TO\_DATE ફંક્શન:}

\begin{itemize}
\tightlist
\item
  \textbf{હેતુ}: કેરેક્ટર સ્ટ્રિંગને તારીખ વેલ્યુમાં કન્વર્ટ કરે છે
\item
  \textbf{સિન્ટેક્સ}: \texttt{TO\_DATE(string,\ format)}\\
\item
  \textbf{ઉપયોગ}: ચોક્કસ ફોર્મેટ સાથે સ્ટ્રિંગ થી તારીખ કન્વર્ઝન
\end{itemize}

\textbf{3. TO\_NUMBER ફંક્શન:}

\begin{itemize}
\tightlist
\item
  \textbf{હેતુ}: કેરેક્ટર સ્ટ્રિંગને સંખ્યાત્મક વેલ્યુમાં કન્વર્ટ કરે છે
\item
  \textbf{સિન્ટેક્સ}: \texttt{TO\_NUMBER(string,\ format)}
\item
  \textbf{ઉપયોગ}: ગણતરીઓ માટે સ્ટ્રિંગ થી નંબર કન્વર્ઝન
\end{itemize}

\textbf{મુખ્ય લાભો:}

\begin{itemize}
\tightlist
\item
  \textbf{ડેટા પ્રકાર લવચીકતા}: પ્રકારો વચ્ચે સહજ કન્વર્ઝન
\item
  \textbf{ફોર્મેટ કંટ્રોલ}: ચોક્કસ ફોર્મેટિંગ વિકલ્પો
\item
  \textbf{એરર હેન્ડલિંગ}: કન્વર્ઝન દરમિયાન માન્યતા
\end{itemize}

\end{solutionbox}
\begin{mnemonicbox}
``કેરેક્ટર્સ તારીખો નંબર્સ કન્વર્ટ કરો''

\end{mnemonicbox}
\subsection*{પ્રશ્ન 4(અ) [3
ગુણ]}\label{uxaaauxab0uxab6uxaa8-4uxa85-3-uxa97uxaa3}

\textbf{ટૂંકી નોંધ લખો: ડોમેઇન ઇન્ટેગ્રિટી કન્સ્ટ્રેઇન્ટ}

\begin{solutionbox}

\textbf{ડોમેઇન ઇન્ટેગ્રિટી કન્સ્ટ્રેઇન્ટ્સ} સુનિશ્ચિત કરે છે કે ડેટા વેલ્યુઝ ચોક્કસ એટ્રિબ્યુટ્સ
માટે સ્વીકાર્ય રેન્જ અને ફોર્મેટમાં આવે છે.


{\def\LTcaptype{none} % do not increment counter
\vspace{-5pt}
\captionof{table}{ડોમેઇન કન્સ્ટ્રેઇન્ટ પ્રકારો}
\vspace{-10pt}
\begin{longtable}[]{@{}
  >{\raggedright\arraybackslash}p{(\linewidth - 4\tabcolsep) * \real{0.4643}}
  >{\raggedright\arraybackslash}p{(\linewidth - 4\tabcolsep) * \real{0.2143}}
  >{\raggedright\arraybackslash}p{(\linewidth - 4\tabcolsep) * \real{0.3214}}@{}}
\toprule\noalign{}
\begin{minipage}[b]{\linewidth}\raggedright
કન્સ્ટ્રેઇન્ટ
\end{minipage} & \begin{minipage}[b]{\linewidth}\raggedright
હેતુ
\end{minipage} & \begin{minipage}[b]{\linewidth}\raggedright
ઉદાહરણ
\end{minipage} \\
\midrule\noalign{}
\endhead
\bottomrule\noalign{}
\endlastfoot
\textbf{CHECK} & વેલ્યુ રેન્જ માન્યતા &
\texttt{CHECK\ (age\ \textgreater{}=\ 0\ AND\ age\ \textless{}=\ 100)} \\
\textbf{NOT NULL} & null વેલ્યુઝ અટકાવે છે &
\texttt{name\ VARCHAR(50)\ NOT\ NULL} \\
\textbf{DEFAULT} & ડિફોલ્ટ વેલ્યુઝ સેટ કરે છે &
\texttt{status\ VARCHAR(10)\ DEFAULT\ \textquotesingle{}Active\textquotesingle{}} \\
\end{longtable}
}

\textbf{મુખ્ય લક્ષણો:}

\begin{itemize}
\tightlist
\item
  \textbf{ડેટા માન્યતા}: એન્ટ્રી વખતે ડેટા ગુણવત્તા સુનિશ્ચિત કરે છે
\item
  \textbf{બિઝનેસ રૂલ્સ}: ડોમેઇન-સ્પેસિફિક રૂલ્સ અમલમાં મૂકે છે
\item
  \textbf{આપોઆપ તપાસ}: DML ઓપરેશન્સ દરમિયાન માન્યતા થાય છે
\end{itemize}

\end{solutionbox}
\begin{mnemonicbox}
``ડોમેઇન ડેટા બાઉન્ડરીઝ વ્યાખ્યાયિત કરે''

\end{mnemonicbox}
\subsection*{પ્રશ્ન 4(બ) [4
ગુણ]}\label{uxaaauxab0uxab6uxaa8-4uxaac-4-uxa97uxaa3}

\textbf{DBMS માં બધા JOIN ની યાદી બનાવો અને કોઈપણ બે સમજાવો}

\begin{solutionbox}


{\def\LTcaptype{none} % do not increment counter
\vspace{-5pt}
\captionof{table}{JOIN પ્રકારો}
\vspace{-10pt}
\begin{longtable}[]{@{}ll@{}}
\toprule\noalign{}
JOIN પ્રકાર & વર્ણન \\
\midrule\noalign{}
\endhead
\bottomrule\noalign{}
\endlastfoot
\textbf{INNER JOIN} & બંને ટેબલમાંથી મેચિંગ રેકોર્ડ્સ પરત કરે \\
\textbf{LEFT JOIN} & ડાબા ટેબલના બધા રેકોર્ડ્સ પરત કરે \\
\textbf{RIGHT JOIN} & જમણા ટેબલના બધા રેકોર્ડ્સ પરત કરે \\
\textbf{FULL OUTER JOIN} & બંને ટેબલના બધા રેકોર્ડ્સ પરત કરે \\
\textbf{CROSS JOIN} & બંને ટેબલનું કાર્ટેસિયન પ્રોડક્ટ \\
\textbf{SELF JOIN} & ટેબલ પોતાની સાથે જોડાય છે \\
\end{longtable}
}

\textbf{વિગતવાર સમજૂતી:}

\textbf{1. INNER JOIN:}

\begin{verbatim}
SELECT s.name, c.course\_name
FROM students s
INNER JOIN courses c ON s.course\_id = c.course\_id;
\end{verbatim}

\begin{itemize}
\tightlist
\item
  બંને ટેબલમાંથી માત્ર મેચિંગ રેકોર્ડ્સ પરત કરે છે
\item
  સૌથી વધુ વપરાતો join પ્રકાર
\end{itemize}

\textbf{2. LEFT JOIN:}

\begin{verbatim}
SELECT s.name, c.course\_name
FROM students s
LEFT JOIN courses c ON s.course\_id = c.course\_id;
\end{verbatim}

\begin{itemize}
\tightlist
\item
  બધા વિદ્યાર્થીઓ પરત કરે છે, ભલે કોઈ કોર્સ અસાઇન ન હોય
\item
  અનમેચ્ડ રેકોર્ડ્સ માટે NULL વેલ્યુઝ
\end{itemize}

\end{solutionbox}
\begin{mnemonicbox}
``ટેબલોને વિચારપૂર્વક જોડો''

\end{mnemonicbox}
\subsection*{પ્રશ્ન 4(ક) [7
ગુણ]}\label{uxaaauxab0uxab6uxaa8-4uxa95-7-uxa97uxaa3}

\textbf{ફંક્શનલ ડિપેન્ડેન્સીનો કન્સેપ્ટ વિગતવાર સમજાવો}

\begin{solutionbox}

\textbf{ફંક્શનલ ડિપેન્ડેન્સી} ત્યારે થાય છે જ્યારે એક એટ્રિબ્યુટની વેલ્યુ અન્ય એટ્રિબ્યુટની
વેલ્યુને વિશિષ્ટ રીતે નિર્ધારિત કરે છે.

\textbf{નોટેશન:} A \rightarrow B (A ફંક્શનલી B નિર્ધારિત કરે છે)


{\def\LTcaptype{none} % do not increment counter
\vspace{-5pt}
\captionof{table}{ફંક્શનલ ડિપેન્ડેન્સીના પ્રકારો}
\vspace{-10pt}
\begin{longtable}[]{@{}
  >{\raggedright\arraybackslash}p{(\linewidth - 4\tabcolsep) * \real{0.2500}}
  >{\raggedright\arraybackslash}p{(\linewidth - 4\tabcolsep) * \real{0.3750}}
  >{\raggedright\arraybackslash}p{(\linewidth - 4\tabcolsep) * \real{0.3750}}@{}}
\toprule\noalign{}
\begin{minipage}[b]{\linewidth}\raggedright
પ્રકાર
\end{minipage} & \begin{minipage}[b]{\linewidth}\raggedright
વ્યાખ્યા
\end{minipage} & \begin{minipage}[b]{\linewidth}\raggedright
ઉદાહરણ
\end{minipage} \\
\midrule\noalign{}
\endhead
\bottomrule\noalign{}
\endlastfoot
\textbf{પૂર્ણ FD} & LHS માં બધા એટ્રિબ્યુટ્સ જરૂરી & \{Student\_ID, Course\_ID\}
\rightarrow Grade \\
\textbf{આંશિક FD} & કેટલાક LHS એટ્રિબ્યુટ્સ રીડન્ડન્ટ & \{Student\_ID,
Course\_ID\} \rightarrow Student\_Name \\
\textbf{ટ્રાન્ઝિટિવ FD} & અન્ય એટ્રિબ્યુટ દ્વારા અપ્રત્યક્ષ ડિપેન્ડેન્સી & Student\_ID
\rightarrow Dept\_ID \rightarrow Dept\_Name \\
\end{longtable}
}

\textbf{આકૃતિ: ફંક્શનલ ડિપેન્ડેન્સી ઉદાહરણ}

\begin{verbatim}
Student\_ID {-{-}{-}{-}{-}{-}{-}{-}{-} Student\_Name}
    |                      |
    |                      v
    |{-{-}{-}{-}{-}{-}{-}{-}{-}{-}{-}{-}{-}{-}{-}{-}{-} Address}
    |
    v
Course\_ID {-{-}{-}{-}{-}{-}{-}{-}{-} Course\_Name}
\end{verbatim}

\textbf{મુખ્ય ગુણધર્મો:}

\begin{itemize}
\tightlist
\item
  \textbf{રિફ્લેક્સિવિટી}: A \rightarrow A (ટ્રિવિયલ ડિપેન્ડેન્સી)
\item
  \textbf{ઓગમેન્ટેશન}: જો A \rightarrow B, તો AC \rightarrow BC
\item
  \textbf{ટ્રાન્ઝિટિવિટી}: જો A \rightarrow B અને B \rightarrow C, તો A \rightarrow C
\item
  \textbf{ડીકમ્પોઝિશન}: જો A \rightarrow BC, તો A \rightarrow B અને A \rightarrow C
\end{itemize}

\textbf{ઉપયોગો:}

\begin{itemize}
\tightlist
\item
  \textbf{નોર્મલાઇઝેશન}: FD નો ઉપયોગ કરીને રીડન્ડન્સી દૂર કરે છે
\item
  \textbf{ડેટાબેઝ ડિઝાઇન}: ટેબલ સ્ટ્રક્ચર નિર્ધારિત કરે છે
\item
  \textbf{ડેટા ઇન્ટેગ્રિટી}: સુસંગતતા જાળવે છે
\end{itemize}

\end{solutionbox}
\begin{mnemonicbox}
``ફંક્શન્સ ડિરેક્ટલી ડિપેન્ડેન્સીઝ નિર્ધારિત કરે''

\end{mnemonicbox}
\subsection*{પ્રશ્ન 4(અ) અથવા [3
ગુણ]}\label{uxaaauxab0uxab6uxaa8-4uxa85-uxa85uxaa5uxab5-3-uxa97uxaa3}

\textbf{ટૂંકી નોંધ લખો: રેફરેન્શિયલ ઇન્ટેગ્રિટી કન્સ્ટ્રેઇન્ટ્સ}

\begin{solutionbox}

\textbf{રેફરેન્શિયલ ઇન્ટેગ્રિટી} સુનિશ્ચિત કરે છે કે એક ટેબલમાં ફોરેન કી વેલ્યુઝ રેફરેન્સ
કરેલા ટેબલમાં હાલના પ્રાથમિક કી વેલ્યુઝને અનુરૂપ હોય.


{\def\LTcaptype{none} % do not increment counter
\vspace{-5pt}
\captionof{table}{રેફરેન્શિયલ ઇન્ટેગ્રિટી નિયમો}
\vspace{-10pt}
\begin{longtable}[]{@{}
  >{\raggedright\arraybackslash}p{(\linewidth - 4\tabcolsep) * \real{0.2857}}
  >{\raggedright\arraybackslash}p{(\linewidth - 4\tabcolsep) * \real{0.3810}}
  >{\raggedright\arraybackslash}p{(\linewidth - 4\tabcolsep) * \real{0.3333}}@{}}
\toprule\noalign{}
\begin{minipage}[b]{\linewidth}\raggedright
નિયમ
\end{minipage} & \begin{minipage}[b]{\linewidth}\raggedright
વર્ણન
\end{minipage} & \begin{minipage}[b]{\linewidth}\raggedright
ક્રિયા
\end{minipage} \\
\midrule\noalign{}
\endhead
\bottomrule\noalign{}
\endlastfoot
\textbf{INSERT નિયમ} & ફોરેન કી પેરેન્ટમાં હોવી જ જોઈએ & અમાન્ય inserts
નકારે \\
\textbf{DELETE નિયમ} & પેરેન્ટ રેકોર્ડ ડિલીશન હેન્ડલ કરે & CASCADE, RESTRICT,
SET NULL \\
\textbf{UPDATE નિયમ} & પ્રાથમિક કી અપડેટ્સ હેન્ડલ કરે & CASCADE, RESTRICT \\
\end{longtable}
}

\textbf{મુખ્ય લક્ષણો:}

\begin{itemize}
\tightlist
\item
  \textbf{ફોરેન કી કન્સ્ટ્રેઇન્ટ}: સંબંધિત ટેબલોને લિંક કરે છે
\item
  \textbf{ડેટા સુસંગતતા}: અનાથ રેકોર્ડ્સ અટકાવે છે
\item
  \textbf{સંબંધ જાળવણી}: ટેબલ સંબંધો જાળવે છે
\end{itemize}

\textbf{કોડ ઉદાહરણ:}

\begin{verbatim}
ALTER TABLE Orders 
ADD CONSTRAINT FK\_Customer 
FOREIGN KEY (customer\_id) 
REFERENCES Customers(customer\_id);
\end{verbatim}

\end{solutionbox}
\begin{mnemonicbox}
``રેફરેન્સને સંબંધિત રેકોર્ડ્સ જરૂરી''

\end{mnemonicbox}
\subsection*{પ્રશ્ન 4(બ) અથવા [4
ગુણ]}\label{uxaaauxab0uxab6uxaa8-4uxaac-uxa85uxaa5uxab5-4-uxa97uxaa3}

\textbf{રિલેશનલ આલ્જિબ્રાના યુનિયન અને ઇન્ટરસેક્શન ઓપરેશન્સ સમજાવો}

\begin{solutionbox}


{\def\LTcaptype{none} % do not increment counter
\vspace{-5pt}
\captionof{table}{સેટ ઓપરેશન્સ સરખામણી}
\vspace{-10pt}
\begin{longtable}[]{@{}llll@{}}
\toprule\noalign{}
ઓપરેશન & પ્રતીક & વર્ણન & જરૂરિયાત \\
\midrule\noalign{}
\endhead
\bottomrule\noalign{}
\endlastfoot
\textbf{યુનિયન} & \cup & બંને રિલેશન્સના બધા ટ્યુપલ્સ સંયોજિત કરે & યુનિયન કોમ્પેટિબલ \\
\textbf{ઇન્ટરસેક્શન} & \cap & બંને રિલેશન્સમાં સામાન્ય ટ્યુપલ્સ & યુનિયન કોમ્પેટિબલ \\
\end{longtable}
}

\textbf{યુનિયન ઓપરેશન:}

\begin{itemize}
\tightlist
\item
  \textbf{સિન્ટેક્સ}: R \cup S
\item
  \textbf{પરિણામ}: R અને S ના બધા ટ્યુપલ્સ (ડુપ્લિકેટ્સ દૂર કરવામાં આવે છે)
\item
  \textbf{જરૂરિયાત}: સમાન સંખ્યા અને પ્રકારના એટ્રિબ્યુટ્સ
\end{itemize}

\textbf{ઇન્ટરસેક્શન ઓપરેશન:}

\begin{itemize}
\tightlist
\item
  \textbf{સિન્ટેક્સ}: R \cap S\\
\item
  \textbf{પરિણામ}: R અને S બંનેમાં અસ્તિત્વ ધરાવતા ટ્યુપલ્સ
\item
  \textbf{જરૂરિયાત}: યુનિયન કોમ્પેટિબલ રિલેશન્સ
\end{itemize}

\textbf{ઉદાહરણ:}

\begin{verbatim}
Students_CS \cup Students_IT = બંને વિભાગના બધા વિદ્યાર્થીઓ
Students_CS \cap Students_IT = બંને વિભાગમાં વિદ્યાર્થીઓ
\end{verbatim}

\textbf{મુખ્ય મુદ્દાઓ:}

\begin{itemize}
\tightlist
\item
  \textbf{યુનિયન કોમ્પેટિબિલિટી}: રિલેશન્સનું સમાન સ્ટ્રક્ચર હોવું જ જોઈએ
\item
  \textbf{ડુપ્લિકેટ એલિમિનેશન}: પરિણામોમાં માત્ર યુનિક ટ્યુપલ્સ સમાવે છે
\end{itemize}

\end{solutionbox}
\begin{mnemonicbox}
``યુનિયન એકમ કરે, ઇન્ટરસેક્શન સામાન્ય ઓળખે''

\end{mnemonicbox}
\subsection*{પ્રશ્ન 4(ક) અથવા [7
ગુણ]}\label{uxaaauxab0uxab6uxaa8-4uxa95-uxa85uxaa5uxab5-7-uxa97uxaa3}

\textbf{DBMS માં નોર્મલાઇઝેશનનો કન્સેપ્ટ વિગતવાર સમજાવો}

\begin{solutionbox}

\textbf{નોર્મલાઇઝેશન} એ ડેટા રીડન્ડન્સી ઘટાડવા અને ડેટા ઇન્ટેગ્રિટી સુધારવા માટે
ડેટાબેઝ ટેબલોને સંગઠિત કરવાની પ્રક્રિયા છે.


{\def\LTcaptype{none} % do not increment counter
\vspace{-5pt}
\captionof{table}{નોર્મલ ફોર્મ્સ}
\vspace{-10pt}
\begin{longtable}[]{@{}lll@{}}
\toprule\noalign{}
નોર્મલ ફોર્મ & જરૂરિયાતો & દૂર કરે છે \\
\midrule\noalign{}
\endhead
\bottomrule\noalign{}
\endlastfoot
\textbf{1NF} & અણુ વેલ્યુઝ, પુનરાવર્તન જૂથો નહીં & બહુવિધ વેલ્યુ એટ્રિબ્યુટ્સ \\
\textbf{2NF} & 1NF + આંશિક ડિપેન્ડેન્સીઝ નહીં & આંશિક ફંક્શનલ ડિપેન્ડેન્સીઝ \\
\textbf{3NF} & 2NF + ટ્રાન્ઝિટિવ ડિપેન્ડેન્સીઝ નહીં & ટ્રાન્ઝિટિવ ડિપેન્ડેન્સીઝ \\
\textbf{BCNF} & 3NF + દરેક ડિટર્મિનન્ટ કેન્ડિડેટ કી & બાકીની વિસંગતતાઓ \\
\end{longtable}
}

\textbf{નોર્મલાઇઝેશન પ્રક્રિયા:}

\textbf{સ્ટેપ 1 - પ્રથમ નોર્મલ ફોર્મ (1NF):}

\begin{itemize}
\tightlist
\item
  પુનરાવર્તન જૂથો દૂર કરો
\item
  દરેક સેલમાં એક જ વેલ્યુ સમાવો
\item
  દરેક રેકોર્ડ વિશિષ્ટ હોય
\end{itemize}

\textbf{સ્ટેપ 2 - બીજું નોર્મલ ફોર્મ (2NF):}

\begin{itemize}
\tightlist
\item
  1NF માં હોવું જ જોઈએ
\item
  આંશિક ડિપેન્ડેન્સીઝ દૂર કરો
\item
  નોન-કી એટ્રિબ્યુટ્સ પ્રાથમિક કી પર સંપૂર્ણ આધારિત
\end{itemize}

\textbf{સ્ટેપ 3 - ત્રીજું નોર્મલ ફોર્મ (3NF):}

\begin{itemize}
\tightlist
\item
  2NF માં હોવું જ જોઈએ
\item
  ટ્રાન્ઝિટિવ ડિપેન્ડેન્સીઝ દૂર કરો
\item
  નોન-કી એટ્રિબ્યુટ્સ અન્ય નોન-કી એટ્રિબ્યુટ્સ પર આધારિત નહીં
\end{itemize}

\textbf{નોર્મલાઇઝેશનના ફાયદા:}

\begin{itemize}
\tightlist
\item
  \textbf{ઘટાડેલી રીડન્ડન્સી}: ડુપ્લિકેટ ડેટા દૂર કરે છે
\item
  \textbf{ડેટા ઇન્ટેગ્રિટી}: સુસંગતતા જાળવે છે
\item
  \textbf{સ્ટોરેજ એફિશિયન્સી}: સ્ટોરેજ સ્પેસ ઘટાડે છે
\item
  \textbf{અપડેટ એનોમેલીઝ}: અસંગત અપડેટ્સ અટકાવે છે
\end{itemize}

\textbf{ગેરફાયદા:}

\begin{itemize}
\tightlist
\item
  \textbf{જટિલ ક્વેરીઝ}: બહુવિધ join જરૂરી થઈ શકે છે
\item
  \textbf{પ્રદર્શન પ્રભાવ}: પુનઃપ્રાપ્તિ ધીમી કરી શકે છે
\end{itemize}

\end{solutionbox}
\begin{mnemonicbox}
``વ્યવસ્થિત, નોન-રીડન્ડન્ટ ટેબલો માટે નોર્મલાઇઝ કરો''

\end{mnemonicbox}
\subsection*{પ્રશ્ન 5(અ) [3
ગુણ]}\label{uxaaauxab0uxab6uxaa8-5uxa85-3-uxa97uxaa3}

\textbf{DBMS માં નોર્મલાઇઝેશનની જરૂરિયાતું વર્ણન કરો}

\begin{solutionbox}


{\def\LTcaptype{none} % do not increment counter
\vspace{-5pt}
\captionof{table}{નોર્મલાઇઝેશન દ્વારા હલ થતી સમસ્યાઓ}
\vspace{-10pt}
\begin{longtable}[]{@{}
  >{\raggedright\arraybackslash}p{(\linewidth - 4\tabcolsep) * \real{0.3636}}
  >{\raggedright\arraybackslash}p{(\linewidth - 4\tabcolsep) * \real{0.3636}}
  >{\raggedright\arraybackslash}p{(\linewidth - 4\tabcolsep) * \real{0.2727}}@{}}
\toprule\noalign{}
\begin{minipage}[b]{\linewidth}\raggedright
સમસ્યા
\end{minipage} & \begin{minipage}[b]{\linewidth}\raggedright
વર્ણન
\end{minipage} & \begin{minipage}[b]{\linewidth}\raggedright
ઉકેલ
\end{minipage} \\
\midrule\noalign{}
\endhead
\bottomrule\noalign{}
\endlastfoot
\textbf{ઇન્સર્શન એનોમેલી} & સંપૂર્ણ માહિતી વિના ડેટા ઇન્સર્ટ કરી શકાતો નથી & અલગ
ટેબલો \\
\textbf{અપડેટ એનોમેલી} & એક ફેરફાર માટે બહુવિધ અપડેટ્સ & રીડન્ડન્સી દૂર કરો \\
\textbf{ડિલીશન એનોમેલી} & ડિલીટ કરતી વખતે મહત્વપૂર્ણ ડેટાની ખોટ & ડિપેન્ડેન્સીઝ
સાચવો \\
\end{longtable}
}

\textbf{મુખ્ય જરૂરિયાતો:}

\begin{itemize}
\tightlist
\item
  \textbf{ડેટા સુસંગતતા}: ડેટાબેઝમાં એકસમાન ડેટા સુનિશ્ચિત કરે છે
\item
  \textbf{સ્ટોરેજ ઑપ્ટિમાઇઝેશન}: રીડન્ડન્ટ સ્ટોરેજ ઘટાડે છે
\item
  \textbf{જાળવણી સરળતા}: સરળ ડેટાબેઝ અપડેટ્સ
\end{itemize}

\textbf{ફાયદા:}

\begin{itemize}
\tightlist
\item
  \textbf{સુધારેલી ડેટા ગુણવત્તા}: એરર્સ અને અસંગતતાઓ ઘટાડે છે
\item
  \textbf{લવચીક ડિઝાઇન}: બદલવું અને વિસ્તારવું સરળ
\item
  \textbf{બહેતર પ્રદર્શન}: અપડેટ ઑપરેશન્સ માટે
\end{itemize}

\end{solutionbox}
\begin{mnemonicbox}
``નોર્મલાઇઝેશનને વ્યવસ્થિત સંગઠનની જરૂર''

\end{mnemonicbox}
\subsection*{પ્રશ્ન 5(બ) [4
ગુણ]}\label{uxaaauxab0uxab6uxaa8-5uxaac-4-uxa97uxaa3}

\textbf{DBMS માં ટ્રાન્ઝેક્શનના પ્રોપર્ટીઝ સમજાવો}

\begin{solutionbox}


{\def\LTcaptype{none} % do not increment counter
\vspace{-5pt}
\captionof{table}{ACID પ્રોપર્ટીઝ}
\vspace{-10pt}
\begin{longtable}[]{@{}
  >{\raggedright\arraybackslash}p{(\linewidth - 4\tabcolsep) * \real{0.3913}}
  >{\raggedright\arraybackslash}p{(\linewidth - 4\tabcolsep) * \real{0.3478}}
  >{\raggedright\arraybackslash}p{(\linewidth - 4\tabcolsep) * \real{0.2609}}@{}}
\toprule\noalign{}
\begin{minipage}[b]{\linewidth}\raggedright
પ્રોપર્ટી
\end{minipage} & \begin{minipage}[b]{\linewidth}\raggedright
વર્ણન
\end{minipage} & \begin{minipage}[b]{\linewidth}\raggedright
હેતુ
\end{minipage} \\
\midrule\noalign{}
\endhead
\bottomrule\noalign{}
\endlastfoot
\textbf{અટોમિસિટી} & બધા ઑપરેશન્સ સફળ થાય અથવા બધા નિષ્ફળ થાય & સંપૂર્ણતા
સુનિશ્ચિત કરે \\
\textbf{કન્સિસ્ટન્સી} & ડેટાબેઝ માન્ય સ્થિતિમાં રહે છે & ઇન્ટેગ્રિટી જાળવે છે \\
\textbf{આઇસોલેશન} & સંગામિત ટ્રાન્ઝેક્શન્સ દખલ કરતા નથી & સંઘર્ષ અટકાવે છે \\
\textbf{ડ્યુરેબિલિટી} & કમિટ થયેલા ફેરફારો કાયમી છે & પર્સિસ્ટન્સ સુનિશ્ચિત કરે \\
\end{longtable}
}

\textbf{વિગતવાર સમજૂતી:}

\textbf{અટોમિસિટી:}

\begin{itemize}
\tightlist
\item
  ટ્રાન્ઝેક્શન અવિભાજ્ય એકમ છે
\item
  કાં તો બધા ઑપરેશન્સ સંપૂર્ણ થાય અથવા કોઈ પણ નહીં
\end{itemize}

\textbf{કન્સિસ્ટન્સી:}

\begin{itemize}
\tightlist
\item
  ડેટાબેઝ એક માન્ય સ્થિતિથી બીજી માન્ય સ્થિતિમાં ટ્રાન્ઝિશન
\item
  બધી ઇન્ટેગ્રિટી કન્સ્ટ્રેઇન્ટ્સ જાળવાય છે
\end{itemize}

\textbf{આઇસોલેશન:}

\begin{itemize}
\tightlist
\item
  સંગામિત ટ્રાન્ઝેક્શન્સ અનુક્રમિક રીતે ચાલે છે એમ લાગે છે
\item
  ઇન્ટરમીડિયેટ સ્ટેટ્સ અન્ય ટ્રાન્ઝેક્શન્સને દેખાતા નથી
\end{itemize}

\textbf{ડ્યુરેબિલિટી:}

\begin{itemize}
\tightlist
\item
  એકવાર કમિટ થયા પછી, ફેરફારો સિસ્ટમ ફેલ્યોર્સથી બચે છે
\item
  ડેટા કાયમી ધોરણે સ્ટોર થાય છે
\end{itemize}

\end{solutionbox}
\begin{mnemonicbox}
``ACID યોગ્ય ડેટાબેઝની ખાતરી આપે''

\end{mnemonicbox}
\subsection*{પ્રશ્ન 5(ક) [7
ગુણ]}\label{uxaaauxab0uxab6uxaa8-5uxa95-7-uxa97uxaa3}

\textbf{વ્યુ સીરિયલાઇઝેબિલિટી વિગતવાર સમજાવો}

\begin{solutionbox}

\textbf{વ્યુ સીરિયલાઇઝેબિલિટી} રીડ અને રાઇટ ઑપરેશન્સની તપાસ કરીને સંગામિત શેડ્યુલ
કોઈ સીરિયલ શેડ્યુલ જેવો જ પરિણામ આપે છે કે કેમ તે નિર્ધારિત કરે છે.


{\def\LTcaptype{none} % do not increment counter
\vspace{-5pt}
\captionof{table}{વ્યુ સમકક્ષતાની શરતો}
\vspace{-10pt}
\begin{longtable}[]{@{}ll@{}}
\toprule\noalign{}
શરત & વર્ણન \\
\midrule\noalign{}
\endhead
\bottomrule\noalign{}
\endlastfoot
\textbf{પ્રારંભિક રીડ્સ} & સમાન ટ્રાન્ઝેક્શન્સ પ્રારંભિક વેલ્યુઝ વાંચે છે \\
\textbf{અંતિમ રાઇટ્સ} & સમાન ટ્રાન્ઝેક્શન્સ અંતિમ રાઇટ્સ કરે છે \\
\textbf{ઇન્ટરમીડિયેટ રીડ્સ} & સમાન રાઇટિંગ ટ્રાન્ઝેક્શન્સમાંથી વેલ્યુ વાંચે છે \\
\end{longtable}
}

\textbf{મુખ્ય સંકેતો:}

\textbf{વ્યુ સમકક્ષ શેડ્યુલ્સ:} બે શેડ્યુલ્સ વ્યુ સમકક્ષ છે જો:

\begin{enumerate}
\tightlist
\item
  દરેક ડેટા આઇટમ માટે, જો ટ્રાન્ઝેક્શન T એક શેડ્યુલમાં પ્રારંભિક વેલ્યુ વાંચે છે, તો બીજામાં
  પણ પ્રારંભિક વેલ્યુ વાંચે છે
\item
  દરેક રીડ ઑપરેશન માટે, જો T એક શેડ્યુલમાં T' દ્વારા લખાયેલી વેલ્યુ વાંચે છે, તો બીજામાં
  પણ તે જ થાય છે
\item
  દરેક ડેટા આઇટમ માટે, જો T એક શેડ્યુલમાં અંતિમ રાઇટ કરે છે, તો બીજામાં પણ અંતિમ
  રાઇટ કરે છે
\end{enumerate}

\textbf{વ્યુ સીરિયલાઇઝેબિલિટીની તપાસ:}

\begin{enumerate}
\tightlist
\item
  \textbf{પ્રીસીડન્સ ગ્રાફ}: ડાયરેક્ટેડ ગ્રાફ બનાવો
\item
  \textbf{સાયકલ ડિટેક્શન}: ગ્રાફમાં સાયકલ્સ તપાસો
\item
  \textbf{કોટલિક્ટ વિશ્લેષણ}: રીડ-રાઇટ કોટલિક્ટ્સની તપાસ કરો
\end{enumerate}

\textbf{ઉદાહરણ વિશ્લેષણ:}

\begin{verbatim}
શેડ્યુલ S1: R1(X) W1(X) R2(X) W2(X)
શેડ્યુલ S2: R1(X) R2(X) W1(X) W2(X)
\end{verbatim}

\textbf{ફાયદા:}

\begin{itemize}
\tightlist
\item
  \textbf{કન્કરન્સી કંટ્રોલ}: શુદ્ધતા સુનિશ્ચિત કરે છે
\item
  \textbf{પ્રદર્શન}: મહત્તમ કન્કરન્સીની મંજૂરી આપે છે
\item
  \textbf{સુસંગતતા}: ડેટાબેઝ ઇન્ટેગ્રિટી જાળવે છે
\end{itemize}

\textbf{કોટલિક્ટ સીરિયલાઇઝેબિલિટી સાથે સરખામણી:}

\begin{itemize}
\tightlist
\item
  વ્યુ સીરિયલાઇઝેબિલિટી ઓછી પ્રતિબંધક છે
\item
  કેટલાક વ્યુ સીરિયલાઇઝેબલ શેડ્યુલ્સ કોટલિક્ટ સીરિયલાઇઝેબલ નથી
\item
  તપાસવું વધુ જટિલ છે
\end{itemize}

\end{solutionbox}
\begin{mnemonicbox}
``વ્યુ માન્ય શેડ્યુલ્સ વેરિફાઇ કરે''

\end{mnemonicbox}
\subsection*{પ્રશ્ન 5(અ) અથવા [3
ગુણ]}\label{uxaaauxab0uxab6uxaa8-5uxa85-uxa85uxaa5uxab5-3-uxa97uxaa3}

\textbf{કોઈપણ ડેટાબેઝ પર 2NF પરફોર્મ કરો}

\begin{solutionbox}

\textbf{ઉદાહરણ: સ્ટુડન્ટ કોર્સ ડેટાબેઝ}

\textbf{મૂળ ટેબલ (2NF માં નથી):}

\begin{verbatim}
Student_Course (Student_ID, Student_Name, Course_ID, Course_Name, Grade, Instructor)
પ્રાથમિક કી: {Student_ID, Course_ID}
\end{verbatim}

\textbf{ફંક્શનલ ડિપેન્ડેન્સીઝ:}

\begin{itemize}
\tightlist
\item
  Student\_ID \rightarrow Student\_Name (આંશિક ડિપેન્ડેન્સી)
\item
  Course\_ID \rightarrow Course\_Name, Instructor (આંશિક ડિપેન્ડેન્સી)
\item
  \{Student\_ID, Course\_ID\} \rightarrow Grade
\end{itemize}

\textbf{2NF ડીકમ્પોઝિશન:}

\textbf{ટેબલ 1: વિદ્યાર્થીઓ}

\begin{verbatim}
Students (Student_ID, Student_Name)
પ્રાથમિક કી: Student_ID
\end{verbatim}

\textbf{ટેબલ 2: કોર્સેસ}

\begin{verbatim}
Courses (Course_ID, Course_Name, Instructor)  
પ્રાથમિક કી: Course_ID
\end{verbatim}

\textbf{ટેબલ 3: નોંધણીઓ}

\begin{verbatim}
Enrollments (Student_ID, Course_ID, Grade)
પ્રાથમિક કી: {Student_ID, Course_ID}
ફોરેન કીઝ: Student_ID \rightarrow Students, Course_ID \rightarrow Courses
\end{verbatim}

\textbf{પરિણામ:} બધી આંશિક ડિપેન્ડેન્સીઝ દૂર કરવામાં આવી, હવે 2NF માં છે.

\end{solutionbox}
\begin{mnemonicbox}
``બીજું નોર્મલ ફોર્મ ડિપેન્ડેન્સીઝ અલગ કરે''

\end{mnemonicbox}
\subsection*{પ્રશ્ન 5(બ) અથવા [4
ગુણ]}\label{uxaaauxab0uxab6uxaa8-5uxaac-uxa85uxaa5uxab5-4-uxa97uxaa3}

\textbf{ટ્રાન્ઝેક્શનની સ્ટેટ્સ સમજાવો}

\begin{solutionbox}

\textbf{આકૃતિ: ટ્રાન્ઝેક્શન સ્ટેટ ડાયાગ્રામ}

\begin{verbatim}
stateDiagram{-v2}
        direction LR
    [*] {-{-} સક્રિય}
    સક્રિય {-{-} આંશિક\_કમિટેડ : કમિટ}
    સક્રિય {-{-} નિષ્ફળ : નિષ્ફળતા/રદ}
    આંશિક\_કમિટેડ {-{-} કમિટેડ : સફળ પૂર્ણતા}
    આંશિક\_કમિટેડ {-{-} નિષ્ફળ : નિષ્ફળતા}
    નિષ્ફળ {-{-} રદ\_કરેલ : રોલબેક પૂર્ણ}
    કમિટેડ {-{-} [*]}
    રદ\_કરેલ {-{-} [*]}
\end{verbatim}


{\def\LTcaptype{none} % do not increment counter
\vspace{-5pt}
\captionof{table}{ટ્રાન્ઝેક્શન સ્ટેટ્સ}
\vspace{-10pt}
\begin{longtable}[]{@{}lll@{}}
\toprule\noalign{}
સ્ટેટ & વર્ણન & ક્રિયાઓ \\
\midrule\noalign{}
\endhead
\bottomrule\noalign{}
\endlastfoot
\textbf{સક્રિય} & ટ્રાન્ઝેક્શન ચાલી રહ્યું છે & રીડ/રાઇટ ઑપરેશન્સ \\
\textbf{આંશિક કમિટેડ} & અંતિમ સ્ટેટમેન્ટ એક્ઝિક્યુટ થયું & કમિટની રાહમાં \\
\textbf{કમિટેડ} & ટ્રાન્ઝેક્શન સફળતાપૂર્વક પૂર્ણ & ફેરફારો કાયમી \\
\textbf{નિષ્ફળ} & સામાન્ય રીતે આગળ વધી શકતું નથી & એરર આવી ગયો \\
\textbf{રદ કરેલ} & ટ્રાન્ઝેક્શન રોલબેક કરવામાં આવ્યું & બધા ફેરફારો પાછા ફેરવાયા \\
\end{longtable}
}

\textbf{સ્ટેટ ટ્રાન્ઝિશન્સ:}

\begin{itemize}
\tightlist
\item
  \textbf{સક્રિય થી નિષ્ફળ}: એરર્સ અથવા સ્પષ્ટ રદ કારણે
\item
  \textbf{સક્રિય થી આંશિક કમિટેડ}: અંતિમ સ્ટેટમેન્ટ પછી
\item
  \textbf{આંશિક કમિટેડ થી કમિટેડ}: સફળ પૂર્ણતા
\item
  \textbf{નિષ્ફળ થી રદ કરેલ}: રોલબેક ઑપરેશન્સ પછી
\end{itemize}

\textbf{મુખ્ય મુદ્દાઓ:}

\begin{itemize}
\tightlist
\item
  \textbf{રિકવરી}: સિસ્ટમ નિષ્ફળ સ્ટેટ્સમાંથી રિકવર કરી શકે છે
\item
  \textbf{ડ્યુરેબિલિટી}: કમિટેડ ફેરફારો કાયમી છે
\item
  \textbf{અટોમિસિટી}: રદ કરેલા ટ્રાન્ઝેક્શન્સ કોઈ ચિહ્ન છોડતા નથી
\end{itemize}

\end{solutionbox}
\begin{mnemonicbox}
``ટ્રાન્ઝેક્શન્સ સ્ટેટ્સ દ્વારા મુસાફરી કરે''

\end{mnemonicbox}
\subsection*{પ્રશ્ન 5(ક) અથવા [7
ગુણ]}\label{uxaaauxab0uxab6uxaa8-5uxa95-uxa85uxaa5uxab5-7-uxa97uxaa3}

\textbf{કોટલિક્ટ સીરિયલાઇઝેબિલિટી વિગતવાર સમજાવો}

\begin{solutionbox}

\textbf{કોટલિક્ટ સીરિયલાઇઝેબિલિટી} કોટલિક્ટિંગ ઑપરેશન્સના વિશ્લેષણ દ્વારા સંગામિત
શેડ્યુલ કોઈ સીરિયલ શેડ્યુલની સમકક્ષ છે કે કેમ તે સુનિશ્ચિત કરે છે.


{\def\LTcaptype{none} % do not increment counter
\vspace{-5pt}
\captionof{table}{કોટલિક્ટિંગ ઑપરેશન્સ}
\vspace{-10pt}
\begin{longtable}[]{@{}lll@{}}
\toprule\noalign{}
ઑપરેશન જોડી & કોટલિક્ટ પ્રકાર & કારણ \\
\midrule\noalign{}
\endhead
\bottomrule\noalign{}
\endlastfoot
\textbf{રીડ-રાઇટ} & RW કોટલિક્ટ & રાઇટ પહેલાં રીડ \\
\textbf{રાઇટ-રીડ} & WR કોટલિક્ટ & રીડ પહેલાં રાઇટ \\
\textbf{રાઇટ-રાઇટ} & WW કોટલિક્ટ & બહુવિધ રાઇટ્સ \\
\end{longtable}
}

\textbf{કોટલિક્ટ સીરિયલાઇઝેબિલિટીની તપાસ:}

\textbf{સ્ટેપ 1: કોટલિક્ટ્સ ઓળખો}

\begin{itemize}
\tightlist
\item
  સમાન ડેટા આઇટમ પર ઑપરેશન જોડીઓ શોધો
\item
  તપાસો કે ઑપરેશન્સ વિવિધ ટ્રાન્ઝેક્શન્સના છે કે કેમ
\item
  નિર્ધારિત કરો કે ઑપરેશન્સ કોટલિક્ટ કરે છે કે કેમ
\end{itemize}

\textbf{સ્ટેપ 2: પ્રીસીડન્સ ગ્રાફ બનાવો}

\begin{itemize}
\tightlist
\item
  નોડ્સ ટ્રાન્ઝેક્શન્સ દર્શાવે છે
\item
  ડાયરેક્ટેડ એજેસ કોટલિક્ટ્સ દર્શાવે છે
\item
  Ti થી Tj એજ જો Ti, Tj સાથે કોટલિક્ટ કરે છે
\end{itemize}

\textbf{સ્ટેપ 3: સાયકલ્સ તપાસો}

\begin{itemize}
\tightlist
\item
  જો ગ્રાફમાં સાયકલ્સ નથી \rightarrow કોટલિક્ટ સીરિયલાઇઝેબલ
\item
  જો ગ્રાફમાં સાયકલ્સ છે \rightarrow કોટલિક્ટ સીરિયલાઇઝેબલ નથી
\end{itemize}

\textbf{ઉદાહરણ વિશ્લેષણ:}

\begin{verbatim}
શેડ્યુલ: R1(A) W1(A) R2(A) W2(B) R1(B) W1(B)

કોટલિક્ટ્સ:
- W1(A) કોટલિક્ટ R2(A) સાથે \rightarrow T1 પહેલાં T2
- W2(B) કોટલિક્ટ R1(B) સાથે \rightarrow T2 પહેલાં T1
- W2(B) કોટલિક્ટ W1(B) સાથે \rightarrow T2 પહેલાં T1
\end{verbatim}

\textbf{પ્રીસીડન્સ ગ્રાફ:}

\begin{verbatim}
    T1 {-{-}{-}{-} T2}
       (સાયકલ)
\end{verbatim}

\textbf{પરિણામ:} સાયકલ સમાવે છે, તેથી કોટલિક્ટ સીરિયલાઇઝેબલ નથી.

\textbf{મુખ્ય ગુણધર્મો:}

\begin{itemize}
\tightlist
\item
  \textbf{કોટલિક્ટ સમકક્ષ}: સમાન કોટલિક્ટ્સ, સમાન સંબંધિત ક્રમ
\item
  \textbf{સીરિયલ શેડ્યુલ}: એક સમયે એક ટ્રાન્ઝેક્શન
\item
  \textbf{પ્રીસીડન્સ ગ્રાફ}: ડિપેન્ડેન્સીઝ દર્શાવતો ડાયરેક્ટેડ ગ્રાફ
\item
  \textbf{સાયકલ ડિટેક્શન}: કોટલિક્ટ સીરિયલાઇઝેબિલિટી નિર્ધારિત કરે છે
\end{itemize}

\textbf{ફાયદા:}

\begin{itemize}
\tightlist
\item
  \textbf{કન્કરન્સી કંટ્રોલ}: શુદ્ધતા સુનિશ્ચિત કરે છે
\item
  \textbf{પ્રદર્શન}: મહત્તમ સંગામિત એક્ઝિક્યુશન
\item
  \textbf{સુસંગતતા}: ડેટાબેઝ ઇન્ટેગ્રિટી જાળવે છે
\end{itemize}

\textbf{વ્યુ સીરિયલાઇઝેબિલિટી સાથે સરખામણી:}

\begin{itemize}
\tightlist
\item
  કોટલિક્ટ સીરિયલાઇઝેબિલિટી વધુ પ્રતિબંધક છે
\item
  બધા કોટલિક્ટ સીરિયલાઇઝેબલ શેડ્યુલ્સ વ્યુ સીરિયલાઇઝેબલ છે
\item
  વ્યુ સીરિયલાઇઝેબિલિટી કરતાં તપાસવું સરળ છે
\end{itemize}

\textbf{તપાસ માટેના અલ્ગોરિધમ્સ:}

\begin{enumerate}
\tightlist
\item
  \textbf{પ્રીસીડન્સ ગ્રાફ મેથડ}: ગ્રાફ બનાવો અને સાયકલ્સ તપાસો
\item
  \textbf{ટાઇમસ્ટેમ્પ ઓર્ડરિંગ}: ઑપરેશન્સને ઓર્ડર કરવા માટે ટાઇમસ્ટેમ્પનો ઉપયોગ
\item
  \textbf{ટુ-ફેઝ લોકિંગ}: સીરિયલાઇઝેબિલિટી સુનિશ્ચિત કરવા માટે લોકનો ઉપયોગ
\end{enumerate}

\end{solutionbox}
\begin{mnemonicbox}
``કોટલિક્ટ્સ સાયકલ્સ બનાવે, કાળજીપૂર્વક તપાસો''

\end{mnemonicbox}

\end{document}
