\documentclass[10pt,a4paper]{article}

% content/resources/templates/preamble.tex
\usepackage[margin=0.6in]{geometry}
\author{Milav Dabgar}
\usepackage{amsmath,amssymb,amsthm}
\usepackage{booktabs}
\usepackage{multirow}
\usepackage{xcolor}
\usepackage{tcolorbox}
\tcbuselibrary{breakable,skins}
\usepackage[colorlinks=true,linkcolor=blue]{hyperref}
\usepackage{titlesec}
\usepackage{enumitem}
\usepackage{tikz}
\usepackage{pgfplots}
\usepackage{circuitikz}
\usepackage[version=4]{mhchem}
\usepackage{longtable}
\usepackage{array}
\usepackage{float}
\usepackage{caption}
\usepackage{listings}

\lstset{
  basicstyle=\small\ttfamily,
  breaklines=true,
  breakatwhitespace=false,
  postbreak=\mbox{\textcolor{red}{$\hookrightarrow$}\space},
  float=false,
  numbers=left,
  numberstyle=\tiny\color{gray},
  numbersep=10pt,
  xleftmargin=2em,
  keywordstyle=\color{blue},
  commentstyle=\color{green!60!black},
  stringstyle=\color{purple},
  backgroundcolor=\color{gray!5},
  showstringspaces=false,
  tabsize=2,
  captionpos=b,
  keepspaces=true,
  columns=flexible
}

\pgfplotsset{compat=1.18}
\usetikzlibrary{shapes,arrows,positioning,calc,patterns,decorations.pathmorphing,decorations.markings,arrows.meta}

% Color scheme
\definecolor{headcolor}{RGB}{0,102,204}
\definecolor{keycolor}{RGB}{220,20,60}
\definecolor{solutioncolor}{RGB}{34,139,34}
\definecolor{mnemoniccolor}{RGB}{148,0,211}
\definecolor{codecolor}{RGB}{0,0,100}

% Spacing
\setlength{\parskip}{3pt}
\setlist[itemize]{nosep}
\setlist[enumerate]{nosep}

% Title formatting
\titleformat{\section}{\Large\bfseries\color{headcolor}}{\thesection}{1em}{}
\titleformat{\subsection}{\large\bfseries\color{headcolor}}{\thesubsection}{1em}{}

% Pandoc tightlist compatibility
\providecommand{\tightlist}{%
  \setlength{\itemsep}{0pt}\setlength{\parskip}{0pt}}

% Pandoc longtable compatibility
\newcounter{none}
\def\thenone{}


% content/resources/templates/english-boxes.tex
% This file is currently empty - it exists to maintain consistency with the import structure.
% Add custom environments here if needed in the future.


\begin{document}

\begin{center}
{\Huge\bfseries\color{headcolor} Subject Name Solutions}\\[5pt]
{\LARGE 1333204 -- Winter 2024}\\[3pt]
{\large Semester 1 Study Material}\\[3pt]
{\normalsize\textit{Detailed Solutions and Explanations}}
\end{center}

\vspace{10pt}

\subsection*{Question 1(a) [3 marks]}\label{q1a}

\textbf{Define: Field, Record, Metadata}

\begin{solutionbox}

\begin{itemize}
\tightlist
\item
  \textbf{Field}: A single unit of data representing one attribute of an
  entity
\item
  \textbf{Record}: Collection of related fields that store data about an
  entity
\item
  \textbf{Metadata}: Data about data that describes the structure,
  properties, and relationships of database objects
\end{itemize}

\end{solutionbox}
\begin{mnemonicbox}
``FRaMe'' (Field, Record, Metadata)

\end{mnemonicbox}
\subsection*{Question 1(b) [4 marks]}\label{q1b}

\textbf{Define: strong and weak entity set.}

\begin{solutionbox}

{\def\LTcaptype{none} % do not increment counter
\begin{longtable}[]{@{}
  >{\raggedright\arraybackslash}p{(\linewidth - 6\tabcolsep) * \real{0.2549}}
  >{\raggedright\arraybackslash}p{(\linewidth - 6\tabcolsep) * \real{0.2549}}
  >{\raggedright\arraybackslash}p{(\linewidth - 6\tabcolsep) * \real{0.3137}}
  >{\raggedright\arraybackslash}p{(\linewidth - 6\tabcolsep) * \real{0.1765}}@{}}
\toprule\noalign{}
\begin{minipage}[b]{\linewidth}\raggedright
Entity Type
\end{minipage} & \begin{minipage}[b]{\linewidth}\raggedright
Description
\end{minipage} & \begin{minipage}[b]{\linewidth}\raggedright
Identification
\end{minipage} & \begin{minipage}[b]{\linewidth}\raggedright
Example
\end{minipage} \\
\midrule\noalign{}
\endhead
\bottomrule\noalign{}
\endlastfoot
\textbf{Strong Entity} & Exists independently & Has its own primary key
& Customer, Employee \\
\textbf{Weak Entity} & Depends on strong entity & Requires parent entity
key & Bank Account, Order Item \\
\end{longtable}
}

\end{solutionbox}
\begin{mnemonicbox}
``SWing'' (Strong is With own identity, weak is Not
Getting own identity)

\end{mnemonicbox}
\subsection*{Question 1(c) [7 marks]}\label{q1c}

\textbf{Explain 3 Levels of Data Abstraction}

\begin{solutionbox}

{\def\LTcaptype{none} % do not increment counter
\begin{longtable}[]{@{}
  >{\raggedright\arraybackslash}p{(\linewidth - 4\tabcolsep) * \real{0.2414}}
  >{\raggedright\arraybackslash}p{(\linewidth - 4\tabcolsep) * \real{0.4483}}
  >{\raggedright\arraybackslash}p{(\linewidth - 4\tabcolsep) * \real{0.3103}}@{}}
\toprule\noalign{}
\begin{minipage}[b]{\linewidth}\raggedright
Level
\end{minipage} & \begin{minipage}[b]{\linewidth}\raggedright
Description
\end{minipage} & \begin{minipage}[b]{\linewidth}\raggedright
Used By
\end{minipage} \\
\midrule\noalign{}
\endhead
\bottomrule\noalign{}
\endlastfoot
\textbf{Physical Level} & Describes how data is stored physically &
System Administrators \\
\textbf{Conceptual Level} & Describes what data is stored and
relationships & Database Designers \\
\textbf{View Level} & Describes part of database relevant to users & End
Users \\
\end{longtable}
}

\textbf{Diagram:}

\includegraphics[width=1\linewidth,height=\textheight,keepaspectratio]{mermaid-60d02b2a.pdf}

\end{solutionbox}
\begin{mnemonicbox}
``PCV'' (Physical, Conceptual, View - bottom to top)

\end{mnemonicbox}
\subsection*{Question 1(c) OR [7
marks]}\label{q1c}

\textbf{Explain advantages and disadvantages of DBMS.}

\begin{solutionbox}

{\def\LTcaptype{none} % do not increment counter
\begin{longtable}[]{@{}ll@{}}
\toprule\noalign{}
Advantages & Disadvantages \\
\midrule\noalign{}
\endhead
\bottomrule\noalign{}
\endlastfoot
\textbf{Data Redundancy Control} & \textbf{High Cost} of software and
hardware \\
\textbf{Data Consistency} & \textbf{Complexity} in design and
maintenance \\
\textbf{Improved Data Security} & \textbf{Performance Impact} with heavy
usage \\
\textbf{Data Sharing} & \textbf{Vulnerability} to system failures \\
\textbf{Data Independence} & \textbf{Recovery Challenges} after
failure \\
\textbf{Standardized Access} & \textbf{Increased Training
Requirements} \\
\end{longtable}
}

\end{solutionbox}
\begin{mnemonicbox}
``BASIC-DV'' (Benefits: Access, Security,
Independence, Consistency - Drawbacks: Vulnerability)

\end{mnemonicbox}
\subsection*{Question 2(a) [3 marks]}\label{q2a}

\textbf{Explain select operation in relational algebra with example}

\begin{solutionbox}

{\def\LTcaptype{none} % do not increment counter
\begin{longtable}[]{@{}ll@{}}
\toprule\noalign{}
Select Operation (σ) & Description \\
\midrule\noalign{}
\endhead
\bottomrule\noalign{}
\endlastfoot
\textbf{Syntax} & σ(Relation) \\
\textbf{Function} & Retrieves tuples satisfying condition \\
\textbf{Example} & σsalary\textgreater30000(Employee) \\
\end{longtable}
}

\end{solutionbox}
\begin{mnemonicbox}
``SERVe'' (Select Exactly Required Values)

\end{mnemonicbox}
\subsection*{Question 2(b) [4 marks]}\label{q2b}

\textbf{Define Primary, Foreign, Super, Candidate Keys in DBMS.}

\begin{solutionbox}

{\def\LTcaptype{none} % do not increment counter
\begin{longtable}[]{@{}ll@{}}
\toprule\noalign{}
Key Type & Description \\
\midrule\noalign{}
\endhead
\bottomrule\noalign{}
\endlastfoot
\textbf{Primary Key} & Unique identifier for each record \\
\textbf{Foreign Key} & Attribute linking to primary key in another
table \\
\textbf{Super Key} & Set of attributes that can uniquely identify
records \\
\textbf{Candidate Key} & Minimal super key that can be primary key \\
\end{longtable}
}

\end{solutionbox}
\begin{mnemonicbox}
``PFSC'' (Person First Shows Credentials)

\end{mnemonicbox}
\subsection*{Question 2(c) [7 marks]}\label{q2c}

\textbf{Draw E R Diagram of Library Management System.}

\begin{solutionbox}

\includegraphics[width=1\linewidth,height=\textheight,keepaspectratio]{mermaid-6cca00d6.pdf}

\end{solutionbox}
\begin{mnemonicbox}
``LIMB'' (Library Items, Members, Borrowing)

\end{mnemonicbox}
\subsection*{Question 2(a) OR [3
marks]}\label{q2a}

\textbf{Explain union operation in relational algebra with example.}

\begin{solutionbox}

{\def\LTcaptype{none} % do not increment counter
\begin{longtable}[]{@{}ll@{}}
\toprule\noalign{}
Union Operation (\cup) & Description \\
\midrule\noalign{}
\endhead
\bottomrule\noalign{}
\endlastfoot
\textbf{Syntax} & Relation1 \cup Relation2 \\
\textbf{Function} & Combines tuples from both relations \\
\textbf{Requirement} & Both relations must be union-compatible \\
\end{longtable}
}

\textbf{Example:} Students\_CS \cup Students\_IT

\end{solutionbox}
\begin{mnemonicbox}
``CUP'' (Combining Union of Parts)

\end{mnemonicbox}
\subsection*{Question 2(b) OR [4
marks]}\label{q2b}

\textbf{Define Composite attribute and Multivalued attribute with
example}

\begin{solutionbox}

{\def\LTcaptype{none} % do not increment counter
\begin{longtable}[]{@{}
  >{\raggedright\arraybackslash}p{(\linewidth - 4\tabcolsep) * \real{0.4054}}
  >{\raggedright\arraybackslash}p{(\linewidth - 4\tabcolsep) * \real{0.3514}}
  >{\raggedright\arraybackslash}p{(\linewidth - 4\tabcolsep) * \real{0.2432}}@{}}
\toprule\noalign{}
\begin{minipage}[b]{\linewidth}\raggedright
Attribute Type
\end{minipage} & \begin{minipage}[b]{\linewidth}\raggedright
Description
\end{minipage} & \begin{minipage}[b]{\linewidth}\raggedright
Example
\end{minipage} \\
\midrule\noalign{}
\endhead
\bottomrule\noalign{}
\endlastfoot
\textbf{Composite} & Can be divided into smaller subparts & Address
(street, city, state, zip) \\
\textbf{Multivalued} & Can have more than one value & Phone numbers,
Email addresses \\
\end{longtable}
}

\textbf{Diagram:}

\includegraphics[width=1\linewidth,height=\textheight,keepaspectratio]{mermaid-7b1272a4.pdf}

\end{solutionbox}
\begin{mnemonicbox}
``CoMbo'' (Composite has Multiple components)

\end{mnemonicbox}
\subsection*{Question 2(c) OR [7
marks]}\label{q2c}

\textbf{Draw E R Diagram of College Management System.}

\begin{solutionbox}

\includegraphics[width=1\linewidth,height=\textheight,keepaspectratio]{mermaid-821c5a53.pdf}

\end{solutionbox}
\begin{mnemonicbox}
``DECFS'' (Departments, Enrollments, Courses,
Faculty, Students)

\end{mnemonicbox}
\subsection*{Question 3(a) [3 marks]}\label{q3a}

\textbf{List different data types in SQL and Explain in brief}

\begin{solutionbox}

{\def\LTcaptype{none} % do not increment counter
\begin{longtable}[]{@{}lll@{}}
\toprule\noalign{}
Data Type Category & Examples & Usage \\
\midrule\noalign{}
\endhead
\bottomrule\noalign{}
\endlastfoot
\textbf{Numeric} & INT, FLOAT, DECIMAL & Store numbers \\
\textbf{Character} & CHAR, VARCHAR, TEXT & Store text \\
\textbf{Date/Time} & DATE, TIME, TIMESTAMP & Store temporal data \\
\textbf{Boolean} & BOOLEAN & Store true/false values \\
\textbf{Binary} & BLOB, BINARY & Store binary data \\
\end{longtable}
}

\end{solutionbox}
\begin{mnemonicbox}
``NCDBB'' (Numbers, Characters, Dates, Booleans,
Binaries)

\end{mnemonicbox}
\subsection*{Question 3(b) [4 marks]}\label{q3b}

\textbf{Explain any two DDL Commands with Syntax and Example}

\begin{solutionbox}

{\def\LTcaptype{none} % do not increment counter
\begin{longtable}[]{@{}
  >{\raggedright\arraybackslash}p{(\linewidth - 4\tabcolsep) * \real{0.3462}}
  >{\raggedright\arraybackslash}p{(\linewidth - 4\tabcolsep) * \real{0.3077}}
  >{\raggedright\arraybackslash}p{(\linewidth - 4\tabcolsep) * \real{0.3462}}@{}}
\toprule\noalign{}
\begin{minipage}[b]{\linewidth}\raggedright
Command
\end{minipage} & \begin{minipage}[b]{\linewidth}\raggedright
Syntax
\end{minipage} & \begin{minipage}[b]{\linewidth}\raggedright
Example
\end{minipage} \\
\midrule\noalign{}
\endhead
\bottomrule\noalign{}
\endlastfoot
\textbf{CREATE} & CREATE TABLE table\_name (column\_definitions); &
CREATE TABLE Student (id INT PRIMARY KEY, name VARCHAR(50)); \\
\textbf{ALTER} & ALTER TABLE table\_name ADD/DROP/MODIFY column\_name
data\_type; & ALTER TABLE Student ADD email VARCHAR(100); \\
\end{longtable}
}

\textbf{Diagram:}

\includegraphics[width=1\linewidth,height=\textheight,keepaspectratio]{mermaid-32b695ba.pdf}

\end{solutionbox}
\begin{mnemonicbox}
``CAD'' (Create And Define)

\end{mnemonicbox}
\subsection*{Question 3(c) [7 marks]}\label{q3c}

\textbf{Write the Output of Following Query.} \textbf{a. CEIL(123.57),
CEIL(4.1)} \textbf{b. MOD(12,4), MOD(10,4)} \textbf{c.~POWER(2,3),
POWER(3,3)} \textbf{d.~ROUND(121.413,1), ROUND(121.413,2)} \textbf{e.
FLOOR(25.3),FLOOR(25.7)} \textbf{f.~LENGTH(`AHMEDABAD')} \textbf{g.
ABS(-25),ABS(36)}

\begin{solutionbox}

{\def\LTcaptype{none} % do not increment counter
\begin{longtable}[]{@{}lll@{}}
\toprule\noalign{}
Function & Result & Explanation \\
\midrule\noalign{}
\endhead
\bottomrule\noalign{}
\endlastfoot
\textbf{CEIL(123.57)} & 124 & Smallest integer \geq 123.57 \\
\textbf{CEIL(4.1)} & 5 & Smallest integer \geq 4.1 \\
\textbf{MOD(12,4)} & 0 & Remainder of 12\div4 \\
\textbf{MOD(10,4)} & 2 & Remainder of 10\div4 \\
\textbf{POWER(2,3)} & 8 & 2 raised to power 3 \\
\textbf{POWER(3,3)} & 27 & 3 raised to power 3 \\
\textbf{ROUND(121.413,1)} & 121.4 & Round to 1 decimal place \\
\textbf{ROUND(121.413,2)} & 121.41 & Round to 2 decimal places \\
\textbf{FLOOR(25.3)} & 25 & Largest integer \leq 25.3 \\
\textbf{FLOOR(25.7)} & 25 & Largest integer \leq 25.7 \\
\textbf{LENGTH(`AHMEDABAD')} & 9 & Number of characters \\
\textbf{ABS(-25)} & 25 & Absolute value of -25 \\
\textbf{ABS(36)} & 36 & Absolute value of 36 \\
\end{longtable}
}

\end{solutionbox}
\begin{mnemonicbox}
``CMPRFLA'' (Ceiling, Modulus, Power, Round, Floor,
Length, Absolute)

\end{mnemonicbox}
\subsection*{Question 3(a) OR [3
marks]}\label{q3a}

\textbf{Explain any three Date Functions.}

\begin{solutionbox}

{\def\LTcaptype{none} % do not increment counter
\begin{longtable}[]{@{}
  >{\raggedright\arraybackslash}p{(\linewidth - 6\tabcolsep) * \real{0.3500}}
  >{\raggedright\arraybackslash}p{(\linewidth - 6\tabcolsep) * \real{0.2250}}
  >{\raggedright\arraybackslash}p{(\linewidth - 6\tabcolsep) * \real{0.2250}}
  >{\raggedright\arraybackslash}p{(\linewidth - 6\tabcolsep) * \real{0.2000}}@{}}
\toprule\noalign{}
\begin{minipage}[b]{\linewidth}\raggedright
Date Function
\end{minipage} & \begin{minipage}[b]{\linewidth}\raggedright
Purpose
\end{minipage} & \begin{minipage}[b]{\linewidth}\raggedright
Example
\end{minipage} & \begin{minipage}[b]{\linewidth}\raggedright
Result
\end{minipage} \\
\midrule\noalign{}
\endhead
\bottomrule\noalign{}
\endlastfoot
\textbf{ADD\_MONTHS} & Adds months to date & ADD\_MONTHS(`01-JAN-2023',
3) & 01-APR-2023 \\
\textbf{MONTHS\_BETWEEN} & Calculates months between dates &
MONTHS\_BETWEEN(`01-MAR-2023', `01-JAN-2023') & 2 \\
\textbf{SYSDATE} & Returns current date and time & SYSDATE & Current
system date/time \\
\end{longtable}
}

\end{solutionbox}
\begin{mnemonicbox}
``AMS'' (Add\_months, Months\_between, Sysdate)

\end{mnemonicbox}
\subsection*{Question 3(b) OR [4
marks]}\label{q3b}

\textbf{Explain any two DML Commands with Syntax and Example}

\begin{solutionbox}

{\def\LTcaptype{none} % do not increment counter
\begin{longtable}[]{@{}
  >{\raggedright\arraybackslash}p{(\linewidth - 4\tabcolsep) * \real{0.3462}}
  >{\raggedright\arraybackslash}p{(\linewidth - 4\tabcolsep) * \real{0.3077}}
  >{\raggedright\arraybackslash}p{(\linewidth - 4\tabcolsep) * \real{0.3462}}@{}}
\toprule\noalign{}
\begin{minipage}[b]{\linewidth}\raggedright
Command
\end{minipage} & \begin{minipage}[b]{\linewidth}\raggedright
Syntax
\end{minipage} & \begin{minipage}[b]{\linewidth}\raggedright
Example
\end{minipage} \\
\midrule\noalign{}
\endhead
\bottomrule\noalign{}
\endlastfoot
\textbf{INSERT} & INSERT INTO table\_name VALUES (value1,
value2,\ldots); & INSERT INTO Student VALUES (1, `Raj',
`raj@example.com'); \\
\textbf{UPDATE} & UPDATE table\_name SET column=value WHERE condition; &
UPDATE Student SET email=`new@example.com' WHERE id=1; \\
\end{longtable}
}

\textbf{Diagram:}

\includegraphics[width=1\linewidth,height=\textheight,keepaspectratio]{mermaid-5e13599e.pdf}

\end{solutionbox}
\begin{mnemonicbox}
``IUM'' (Insert, Update, Manipulate)

\end{mnemonicbox}
\subsection*{Question 3(c) OR [7
marks]}\label{q3c}

\textbf{For the table: EMP(emp\_no, emp\_name, designation, salary,
deptno), Write SQL commands for following operations.}

\begin{solutionbox}

{\def\LTcaptype{none} % do not increment counter
\begin{longtable}[]{@{}
  >{\raggedright\arraybackslash}p{(\linewidth - 2\tabcolsep) * \real{0.4583}}
  >{\raggedright\arraybackslash}p{(\linewidth - 2\tabcolsep) * \real{0.5417}}@{}}
\toprule\noalign{}
\begin{minipage}[b]{\linewidth}\raggedright
Operation
\end{minipage} & \begin{minipage}[b]{\linewidth}\raggedright
SQL Command
\end{minipage} \\
\midrule\noalign{}
\endhead
\bottomrule\noalign{}
\endlastfoot
\textbf{Create table EMP} & CREATE TABLE EMP (emp\_no INT PRIMARY KEY,
emp\_name VARCHAR(50), designation VARCHAR(30), salary DECIMAL(10,2),
deptno INT); \\
\textbf{Give the emp\_no, emp\_name, designation, salary, deptno of EMP}
& SELECT emp\_no, emp\_name, designation, salary, deptno FROM EMP; \\
\textbf{Display information of all employees whose name starts with `p'}
& SELECT * FROM EMP WHERE emp\_name LIKE `p\%'; \\
\textbf{Display department wise salary total} & SELECT deptno,
SUM(salary) AS total\_salary FROM EMP GROUP BY deptno; \\
\textbf{Add new column email\_id in EMP table} & ALTER TABLE EMP ADD
email\_id VARCHAR(100); \\
\textbf{Change the column name ``designation'' to ``post''} & ALTER
TABLE EMP RENAME COLUMN designation TO post; \\
\textbf{Delete all the records from the table person} & DELETE FROM
person; \\
\end{longtable}
}

\end{solutionbox}
\begin{mnemonicbox}
``CSDAACD'' (Create, Select, Display, Aggregate, Add,
Change, Delete)

\end{mnemonicbox}
\subsection*{Question 4(a) [3 marks]}\label{q4a}

\textbf{List different aggregate functions and explain any one with
syntax and example.}

\begin{solutionbox}

{\def\LTcaptype{none} % do not increment counter
\begin{longtable}[]{@{}ll@{}}
\toprule\noalign{}
Aggregate Function & Purpose \\
\midrule\noalign{}
\endhead
\bottomrule\noalign{}
\endlastfoot
\textbf{SUM} & Calculates total \\
\textbf{AVG} & Calculates average \\
\textbf{COUNT} & Counts number of rows \\
\textbf{MAX} & Finds maximum value \\
\textbf{MIN} & Finds minimum value \\
\end{longtable}
}

\textbf{Example for AVG:}\\
\passthrough{\lstinline!AVG(column\_name)!} - Calculates average of
values in column\\
\passthrough{\lstinline!SELECT AVG(salary) FROM Employee;!} - Returns
average salary

\end{solutionbox}
\begin{mnemonicbox}
``SCAMM'' (Sum, Count, Avg, Max, Min)

\end{mnemonicbox}
\subsection*{Question 4(b) [4 marks]}\label{q4b}

\textbf{Define the transaction with example.}

\begin{solutionbox}

{\def\LTcaptype{none} % do not increment counter
\begin{longtable}[]{@{}
  >{\raggedright\arraybackslash}p{(\linewidth - 2\tabcolsep) * \real{0.6061}}
  >{\raggedright\arraybackslash}p{(\linewidth - 2\tabcolsep) * \real{0.3939}}@{}}
\toprule\noalign{}
\begin{minipage}[b]{\linewidth}\raggedright
Transaction Concept
\end{minipage} & \begin{minipage}[b]{\linewidth}\raggedright
Description
\end{minipage} \\
\midrule\noalign{}
\endhead
\bottomrule\noalign{}
\endlastfoot
\textbf{Definition} & Logical unit of work that must be completely
processed or completely fail \\
\textbf{Properties} & ACID (Atomicity, Consistency, Isolation,
Durability) \\
\textbf{States} & Active, Partially Committed, Committed, Failed,
Aborted \\
\end{longtable}
}

\textbf{Example:}

\begin{lstlisting}[language=SQL]
BEGIN TRANSACTION;
    UPDATE Accounts SET balance = balance - 5000 WHERE acc_no = 'A123';
    UPDATE Accounts SET balance = balance + 5000 WHERE acc_no = 'B456';
COMMIT;
\end{lstlisting}

\end{solutionbox}
\begin{mnemonicbox}
``TAPS'' (Transaction As Process Set)

\end{mnemonicbox}
\subsection*{Question 4(c) [7 marks]}\label{q4c}

\textbf{What is an Operator in SQL? Explain Arithmetic and Logical
operators with Syntax and Example}

\begin{solutionbox}

{\def\LTcaptype{none} % do not increment counter
\begin{longtable}[]{@{}
  >{\raggedright\arraybackslash}p{(\linewidth - 6\tabcolsep) * \real{0.1765}}
  >{\raggedright\arraybackslash}p{(\linewidth - 6\tabcolsep) * \real{0.3235}}
  >{\raggedright\arraybackslash}p{(\linewidth - 6\tabcolsep) * \real{0.2647}}
  >{\raggedright\arraybackslash}p{(\linewidth - 6\tabcolsep) * \real{0.2353}}@{}}
\toprule\noalign{}
\begin{minipage}[b]{\linewidth}\raggedright
Type
\end{minipage} & \begin{minipage}[b]{\linewidth}\raggedright
Operators
\end{minipage} & \begin{minipage}[b]{\linewidth}\raggedright
Example
\end{minipage} & \begin{minipage}[b]{\linewidth}\raggedright
Result
\end{minipage} \\
\midrule\noalign{}
\endhead
\bottomrule\noalign{}
\endlastfoot
\textbf{Arithmetic} & + (Addition) & 5 + 3 & 8 \\
& - (Subtraction) & 5 - 3 & 2 \\
& * (Multiplication) & 5 * 3 & 15 \\
& / (Division) & 15 / 3 & 5 \\
& \% (Modulus) & 5 \% 2 & 1 \\
\textbf{Logical} & AND & salary \textgreater{} 30000 AND dept = `IT' &
True if both conditions true \\
& OR & salary \textgreater{} 50000 OR dept = `HR' & True if either
condition true \\
& NOT & NOT (salary \textless{} 20000) & True if salary not less than
20000 \\
\end{longtable}
}

\textbf{SQL Examples:}

\begin{lstlisting}[language=SQL]
-- Arithmetic
SELECT product_name, price * 1.18 AS price_with_tax FROM Products;

-- Logical
SELECT * FROM Employees WHERE (salary > 30000 AND dept = 'IT') OR (experience > 5);
\end{lstlisting}

\end{solutionbox}
\begin{mnemonicbox}
``ASMDOLA'' (Add, Subtract, Multiply, Divide, OR,
AND, NOT)

\end{mnemonicbox}
\subsection*{Question 4(a) OR [3
marks]}\label{q4a}

\textbf{List different numeric functions and explain any one with syntax
and example.}

\begin{solutionbox}

{\def\LTcaptype{none} % do not increment counter
\begin{longtable}[]{@{}ll@{}}
\toprule\noalign{}
Numeric Function & Purpose \\
\midrule\noalign{}
\endhead
\bottomrule\noalign{}
\endlastfoot
\textbf{ROUND} & Rounds a number to specified decimal places \\
\textbf{TRUNC} & Truncates a number to specified decimal places \\
\textbf{CEIL} & Returns smallest integer greater than or equal to
number \\
\textbf{FLOOR} & Returns largest integer less than or equal to number \\
\textbf{ABS} & Returns absolute value \\
\end{longtable}
}

\textbf{Example for ROUND:}\\
\passthrough{\lstinline!ROUND(number, decimal\_places)!} - Rounds number
to specified decimal places\\
\passthrough{\lstinline!SELECT ROUND(125.679, 2) FROM DUAL;!} - Returns
125.68

\end{solutionbox}
\begin{mnemonicbox}
``RTCFA'' (Round, Truncate, Ceiling, Floor, Absolute)

\end{mnemonicbox}
\subsection*{Question 4(b) OR [4
marks]}\label{q4b}

\textbf{List various database operations of a transaction.}

\begin{solutionbox}

{\def\LTcaptype{none} % do not increment counter
\begin{longtable}[]{@{}ll@{}}
\toprule\noalign{}
Operation & Description \\
\midrule\noalign{}
\endhead
\bottomrule\noalign{}
\endlastfoot
\textbf{BEGIN/START} & Marks transaction start point \\
\textbf{READ} & Retrieves data from database \\
\textbf{WRITE} & Modifies data in database \\
\textbf{COMMIT} & Makes changes permanent \\
\textbf{ROLLBACK} & Undoes changes and returns to start point \\
\textbf{SAVEPOINT} & Creates points to rollback partially \\
\end{longtable}
}

\textbf{Diagram:}

\includegraphics[width=1\linewidth,height=\textheight,keepaspectratio]{mermaid-cc48873e.pdf}

\end{solutionbox}
\begin{mnemonicbox}
``BRWCRS'' (Begin, Read, Write, Commit, Rollback,
Savepoint)

\end{mnemonicbox}
\subsection*{Question 4(c) OR [7
marks]}\label{q4c}

\textbf{What is join? Explain different types of joins with syntax and
example.}

\begin{solutionbox}

{\def\LTcaptype{none} % do not increment counter
\begin{longtable}[]{@{}
  >{\raggedright\arraybackslash}p{(\linewidth - 4\tabcolsep) * \real{0.2750}}
  >{\raggedright\arraybackslash}p{(\linewidth - 4\tabcolsep) * \real{0.3250}}
  >{\raggedright\arraybackslash}p{(\linewidth - 4\tabcolsep) * \real{0.4000}}@{}}
\toprule\noalign{}
\begin{minipage}[b]{\linewidth}\raggedright
Join Type
\end{minipage} & \begin{minipage}[b]{\linewidth}\raggedright
Description
\end{minipage} & \begin{minipage}[b]{\linewidth}\raggedright
Syntax Example
\end{minipage} \\
\midrule\noalign{}
\endhead
\bottomrule\noalign{}
\endlastfoot
\textbf{INNER JOIN} & Returns rows when there is a match in both tables
& SELECT * FROM TableA INNER JOIN TableB ON TableA.id = TableB.id; \\
\textbf{LEFT JOIN} & Returns all rows from left table and matched rows
from right & SELECT * FROM TableA LEFT JOIN TableB ON TableA.id =
TableB.id; \\
\textbf{RIGHT JOIN} & Returns all rows from right table and matched rows
from left & SELECT * FROM TableA RIGHT JOIN TableB ON TableA.id =
TableB.id; \\
\textbf{FULL JOIN} & Returns rows when there is a match in one of the
tables & SELECT * FROM TableA FULL JOIN TableB ON TableA.id =
TableB.id; \\
\textbf{SELF JOIN} & Joins a table to itself & SELECT * FROM Employee e1
JOIN Employee e2 ON e1.manager\_id = e2.emp\_id; \\
\end{longtable}
}

\textbf{Diagram:}

\includegraphics[width=1\linewidth,height=\textheight,keepaspectratio]{mermaid-1cef0cdb.pdf}

\end{solutionbox}
\begin{mnemonicbox}
``ILRFS'' (Inner, Left, Right, Full, Self)

\end{mnemonicbox}
\subsection*{Question 5(a) [3 marks]}\label{q5a}

\textbf{Convert the customer relation into 1NF shown below.}
\textbf{Customer}

{\def\LTcaptype{none} % do not increment counter
\begin{longtable}[]{@{}llll@{}}
\toprule\noalign{}
cid & name & address & Contact\_no \\
\midrule\noalign{}
\endhead
\bottomrule\noalign{}
\endlastfoot
CO1 & Riya & Amu aavas, Anand & \{5322332123\} \\
CO2 & Jiya & Sardar colony, Ahmedabad & \{5326521456, 5265232849\} \\
\end{longtable}
}

\begin{solutionbox}

\textbf{Customer Table (1NF):}

{\def\LTcaptype{none} % do not increment counter
\begin{longtable}[]{@{}lllll@{}}
\toprule\noalign{}
cid & name & society & city & Contact\_no \\
\midrule\noalign{}
\endhead
\bottomrule\noalign{}
\endlastfoot
CO1 & Riya & Amu aavas & Anand & 5322332123 \\
CO2 & Jiya & Sardar colony & Ahmedabad & 5326521456 \\
CO2 & Jiya & Sardar colony & Ahmedabad & 5265232849 \\
\end{longtable}
}

\end{solutionbox}
\begin{mnemonicbox}
``AFM'' (Atomic values, Flatten Multivalued
attributes)

\end{mnemonicbox}
\subsection*{Question 5(b) [4 marks]}\label{q5b}

\textbf{List and Explain ACID properties of transaction.}

\begin{solutionbox}

{\def\LTcaptype{none} % do not increment counter
\begin{longtable}[]{@{}
  >{\raggedright\arraybackslash}p{(\linewidth - 2\tabcolsep) * \real{0.5357}}
  >{\raggedright\arraybackslash}p{(\linewidth - 2\tabcolsep) * \real{0.4643}}@{}}
\toprule\noalign{}
\begin{minipage}[b]{\linewidth}\raggedright
ACID Property
\end{minipage} & \begin{minipage}[b]{\linewidth}\raggedright
Description
\end{minipage} \\
\midrule\noalign{}
\endhead
\bottomrule\noalign{}
\endlastfoot
\textbf{Atomicity} & Transaction executes completely or not at all \\
\textbf{Consistency} & Database remains consistent before and after
transaction \\
\textbf{Isolation} & Concurrent transactions don't interfere with each
other \\
\textbf{Durability} & Committed changes are permanent even after system
failure \\
\end{longtable}
}

\textbf{Diagram:}

\includegraphics[width=1\linewidth,height=\textheight,keepaspectratio]{mermaid-96fa5529.pdf}

\end{solutionbox}
\begin{mnemonicbox}
``ACID'' (Atomicity, Consistency, Isolation,
Durability)

\end{mnemonicbox}
\subsection*{Question 5(c) [7 marks]}\label{q5c}

\textbf{List different types of functional dependencies and explain each
using example.}

\begin{solutionbox}

{\def\LTcaptype{none} % do not increment counter
\begin{longtable}[]{@{}
  >{\raggedright\arraybackslash}p{(\linewidth - 4\tabcolsep) * \real{0.5000}}
  >{\raggedright\arraybackslash}p{(\linewidth - 4\tabcolsep) * \real{0.2955}}
  >{\raggedright\arraybackslash}p{(\linewidth - 4\tabcolsep) * \real{0.2045}}@{}}
\toprule\noalign{}
\begin{minipage}[b]{\linewidth}\raggedright
Functional Dependency
\end{minipage} & \begin{minipage}[b]{\linewidth}\raggedright
Description
\end{minipage} & \begin{minipage}[b]{\linewidth}\raggedright
Example
\end{minipage} \\
\midrule\noalign{}
\endhead
\bottomrule\noalign{}
\endlastfoot
\textbf{Trivial FD} & X \rightarrow Y where Y is a subset of X & \{StudentID,
Name\} \rightarrow \{Name\} \\
\textbf{Non-trivial FD} & X \rightarrow Y where Y is not a subset of X &
\{StudentID\} \rightarrow \{Name\} \\
\textbf{Partial FD} & Part of composite key determines non-key attribute
& \{CourseID, StudentID\} \rightarrow \{CourseName\} \\
\textbf{Transitive FD} & X \rightarrow Y and Y \rightarrow Z implies X \rightarrow Z & \{StudentID\} \rightarrow
\{DeptID\} and \{DeptID\} \rightarrow \{DeptName\} \\
\textbf{Multivalued FD} & One attribute determines set of values for
another & \{CourseID\} \rightarrow\rightarrow \{TextbookID\} \\
\end{longtable}
}

\textbf{Diagram:}

\includegraphics[width=1\linewidth,height=\textheight,keepaspectratio]{mermaid-40533685.pdf}

\end{solutionbox}
\begin{mnemonicbox}
``TNPTMv'' (Trivial, Non-trivial, Partial,
Transitive, Multivalued)

\end{mnemonicbox}
\subsection*{Question 5(a) OR [3
marks]}\label{q5a}

\textbf{Convert the Depositor\_Account relation into 2NF shown below.}
\textbf{Where functional dependencies(FD) are as under,} \textbf{FD1:
\{cid, ano\} \rightarrow \{access\_date, balance, bname\}} \textbf{FD2: ano \rightarrow
\{balance, bname\}}

\textbf{Depositor\_Account}

{\def\LTcaptype{none} % do not increment counter
\begin{longtable}[]{@{}lllll@{}}
\toprule\noalign{}
cid & ano & access\_date & balance & bname \\
\midrule\noalign{}
\endhead
\bottomrule\noalign{}
\endlastfoot
\end{longtable}
}

\begin{solutionbox}

\textbf{Account Table (2NF):}

{\def\LTcaptype{none} % do not increment counter
\begin{longtable}[]{@{}lll@{}}
\toprule\noalign{}
ano & balance & bname \\
\midrule\noalign{}
\endhead
\bottomrule\noalign{}
\endlastfoot
\end{longtable}
}

\textbf{Depositor Table (2NF):}

{\def\LTcaptype{none} % do not increment counter
\begin{longtable}[]{@{}lll@{}}
\toprule\noalign{}
cid & ano & access\_date \\
\midrule\noalign{}
\endhead
\bottomrule\noalign{}
\endlastfoot
\end{longtable}
}

\end{solutionbox}
\begin{mnemonicbox}
``RPKD'' (Remove Partial Key Dependencies)

\end{mnemonicbox}
\subsection*{Question 5(b) OR [4
marks]}\label{q5b}

\textbf{Explain conflict serializability.}

\begin{solutionbox}

{\def\LTcaptype{none} % do not increment counter
\begin{longtable}[]{@{}
  >{\raggedright\arraybackslash}p{(\linewidth - 2\tabcolsep) * \real{0.4091}}
  >{\raggedright\arraybackslash}p{(\linewidth - 2\tabcolsep) * \real{0.5909}}@{}}
\toprule\noalign{}
\begin{minipage}[b]{\linewidth}\raggedright
Concept
\end{minipage} & \begin{minipage}[b]{\linewidth}\raggedright
Description
\end{minipage} \\
\midrule\noalign{}
\endhead
\bottomrule\noalign{}
\endlastfoot
\textbf{Definition} & Schedule is conflict serializable if equivalent to
some serial schedule \\
\textbf{Conflict Operations} & Read-Write, Write-Read, Write-Write
operations on same data item \\
\textbf{Conflict Graph} & Directed graph showing conflicts between
transactions \\
\textbf{Testing} & Schedule is conflict serializable if conflict graph
has no cycles \\
\end{longtable}
}

\textbf{Diagram:}

\includegraphics[width=1\linewidth,height=\textheight,keepaspectratio]{mermaid-8aef7997.pdf}

\end{solutionbox}
\begin{mnemonicbox}
``COGS'' (Conflict Operations Graph Serializable)

\end{mnemonicbox}
\subsection*{Question 5(c) OR [7
marks]}\label{q5c}

\textbf{Explain 3NF normalization with example}

\begin{solutionbox}

{\def\LTcaptype{none} % do not increment counter
\begin{longtable}[]{@{}
  >{\raggedright\arraybackslash}p{(\linewidth - 4\tabcolsep) * \real{0.3824}}
  >{\raggedright\arraybackslash}p{(\linewidth - 4\tabcolsep) * \real{0.3529}}
  >{\raggedright\arraybackslash}p{(\linewidth - 4\tabcolsep) * \real{0.2647}}@{}}
\toprule\noalign{}
\begin{minipage}[b]{\linewidth}\raggedright
Normal Form
\end{minipage} & \begin{minipage}[b]{\linewidth}\raggedright
Definition
\end{minipage} & \begin{minipage}[b]{\linewidth}\raggedright
Example
\end{minipage} \\
\midrule\noalign{}
\endhead
\bottomrule\noalign{}
\endlastfoot
\textbf{1NF} & Atomic values, no repeating groups & Student(ID, Name,
Phone1, Phone2) \rightarrow Student(ID, Name, Phone) \\
\textbf{2NF} & 1NF + No partial dependencies & Order(OrderID, ProductID,
CustomerID, ProductName) \rightarrow Order(OrderID, ProductID, CustomerID) +
Product(ProductID, ProductName) \\
\textbf{3NF} & 2NF + No transitive dependencies & Student(ID, DeptID,
DeptName) \rightarrow Student(ID, DeptID) + Department(DeptID, DeptName) \\
\end{longtable}
}

\textbf{Violation Example:}

\begin{lstlisting}
Employee(EmpID, EmpName, DeptID, DeptName, Location)
\end{lstlisting}

\textbf{3NF Conversion:}

\begin{lstlisting}
Employee(EmpID, EmpName, DeptID)
Department(DeptID, DeptName, Location)
\end{lstlisting}

\textbf{Diagram:}

\includegraphics[width=1\linewidth,height=\textheight,keepaspectratio]{mermaid-bd87e709.pdf}

\end{solutionbox}
\begin{mnemonicbox}
``APTN'' (Atomic values, Partial dependencies
removed, Transitive dependencies removed, Normalized)

\end{mnemonicbox}

\end{document}
