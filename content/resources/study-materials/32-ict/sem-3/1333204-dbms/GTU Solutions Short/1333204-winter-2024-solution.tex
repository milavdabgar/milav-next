\documentclass{article}

% content/resources/templates/preamble.tex
\usepackage[margin=0.6in]{geometry}
\author{Milav Dabgar}
\usepackage{amsmath,amssymb,amsthm}
\usepackage{booktabs}
\usepackage{multirow}
\usepackage{xcolor}
\usepackage{tcolorbox}
\tcbuselibrary{breakable,skins}
\usepackage[colorlinks=true,linkcolor=blue]{hyperref}
\usepackage{titlesec}
\usepackage{enumitem}
\usepackage{tikz}
\usepackage{pgfplots}
\usepackage{circuitikz}
\usepackage[version=4]{mhchem}
\usepackage{longtable}
\usepackage{array}
\usepackage{float}
\usepackage{caption}
\usepackage{listings}

\lstset{
  basicstyle=\small\ttfamily,
  breaklines=true,
  breakatwhitespace=false,
  postbreak=\mbox{\textcolor{red}{$\hookrightarrow$}\space},
  float=false,
  numbers=left,
  numberstyle=\tiny\color{gray},
  numbersep=10pt,
  xleftmargin=2em,
  keywordstyle=\color{blue},
  commentstyle=\color{green!60!black},
  stringstyle=\color{purple},
  backgroundcolor=\color{gray!5},
  showstringspaces=false,
  tabsize=2,
  captionpos=b,
  keepspaces=true,
  columns=flexible
}

\pgfplotsset{compat=1.18}
\usetikzlibrary{shapes,arrows,positioning,calc,patterns,decorations.pathmorphing,decorations.markings,arrows.meta}

% Color scheme
\definecolor{headcolor}{RGB}{0,102,204}
\definecolor{keycolor}{RGB}{220,20,60}
\definecolor{solutioncolor}{RGB}{34,139,34}
\definecolor{mnemoniccolor}{RGB}{148,0,211}
\definecolor{codecolor}{RGB}{0,0,100}

% Spacing
\setlength{\parskip}{3pt}
\setlist[itemize]{nosep}
\setlist[enumerate]{nosep}

% Title formatting
\titleformat{\section}{\Large\bfseries\color{headcolor}}{\thesection}{1em}{}
\titleformat{\subsection}{\large\bfseries\color{headcolor}}{\thesubsection}{1em}{}

% Pandoc tightlist compatibility
\providecommand{\tightlist}{%
  \setlength{\itemsep}{0pt}\setlength{\parskip}{0pt}}

% Pandoc longtable compatibility
\newcounter{none}
\def\thenone{}


% content/resources/templates/english-boxes.tex
% This file is currently empty - it exists to maintain consistency with the import structure.
% Add custom environments here if needed in the future.


% Custom commands for GTU solutions
% This file defines semantic commands for consistent formatting

% Question command with automatic formatting
\newcommand{\question}[2]{%
  \section*{Question #1}%
  \textbf{#2}%
}

% OR question variant
\newcommand{\questionor}[2]{%
  \section*{Question #1 OR}%
  \textbf{#2}%
}

% Proper table environment with caption
\newenvironment{answertable}[1]{%
  \begin{table}[htbp]
  \centering
  \caption{#1}
}{%
  \end{table}
}

% Proper figure environment for diagrams
\newenvironment{answerdiagram}[1]{%
  \begin{figure}[htbp]
  \centering
  \caption{#1}
}{%
  \end{figure}
}

% Semantic markup for key terms
\newcommand{\keyword}[1]{\textbf{#1}}
\newcommand{\code}[1]{\texttt{#1}}
\newcommand{\classname}[1]{\texttt{#1}}
\newcommand{\methodname}[1]{\texttt{#1}}

% Proper quotation marks
\newcommand{\mnemonic}[1]{``#1''}


\usetikzlibrary{calc,positioning,shapes,arrows,automata,fit,shapes.multipart,trees}
\tikzset{
    entity/.style={rectangle, draw, fill=white, align=center, minimum height=2em, font=\small, thick},
    relationship/.style={diamond, draw, fill=white, align=center, aspect=2, font=\small, thick},
    attribute/.style={ellipse, draw, fill=white, align=center, font=\small},
    multi attribute/.style={ellipse, draw, double, fill=white, align=center, font=\small},
    gtu line/.style={draw, thick},
    gtu arrow/.style={draw, -latex, thick}
}

\title{Database Management System (1333204) - Winter 2024 Solution}
\date{December 09, 2024}

\begin{document}
\maketitle

\questionmarks{1(a)}{3}{Define: Field, Record, Metadata}
\begin{solutionbox}
\begin{itemize}
    \item \textbf{Field}: A single unit of data representing one attribute of an entity.
    \item \textbf{Record}: Collection of related fields that store data about an entity.
    \item \textbf{Metadata}: Data about data that describes the structure, properties, and relationships of database objects.
\end{itemize}

\begin{mnemonicbox}
    \textbf{Mnemonic:} "FRaMe" (Field, Record, Metadata)
\end{mnemonicbox}
\end{solutionbox}

\questionmarks{1(b)}{4}{Define: strong and weak entity set.}
\begin{solutionbox}
\begin{table}[H]
    \centering
    \caption{Strong vs Weak Entity Set}
    \begin{tabulary}{\linewidth}{LCL}
        \toprule
        \textbf{Entity Type} & \textbf{Description} & \textbf{Identification} \\
        \midrule
        \textbf{Strong Entity} & Exists independently & Has its own primary key \\
        \textbf{Weak Entity} & Depends on strong entity & Requires parent entity key \\
        \bottomrule
    \end{tabulary}
\end{table}

\textbf{Examples}: Strong Entity - Customer, Employee; Weak Entity - Bank Account, Order Item.

\begin{mnemonicbox}
    \textbf{Mnemonic:} "SWing" (Strong is With own identity, weak is Not Getting own identity)
\end{mnemonicbox}
\end{solutionbox}

\questionmarks{1(c)}{7}{Explain 3 Levels of Data Abstraction}
\begin{solutionbox}
\begin{table}[H]
    \centering
    \caption{Levels of Data Abstraction}
    \begin{tabulary}{\linewidth}{LCL}
        \toprule
        \textbf{Level} & \textbf{Description} & \textbf{Used By} \\
        \midrule
        \textbf{Physical Level} & Describes how data is stored physically & System Administrators \\
        \textbf{Conceptual Level} & Describes what data is stored and relationships & Database Designers \\
        \textbf{View Level} & Describes part of database relevant to users & End Users \\
        \bottomrule
    \end{tabulary}
\end{table}

\begin{center}
\begin{tikzpicture}[gtu block, node distance=2cm]
    \node [gtu block, fill=blue!10] (view) {View Level};
    \node [gtu block, below of=view] (conc) {Conceptual Level};
    \node [gtu block, below of=conc] (phys) {Physical Level};
    
    \node [left of=view, xshift=-3cm] (users) {End Users};
    \node [left of=conc, xshift=-3cm] (dba) {Database Designers};
    \node [left of=phys, xshift=-3cm] (admin) {System Administrators};

    \draw [gtu arrow] (users) -- (view);
    \draw [gtu arrow] (dba) -- (conc);
    \draw [gtu arrow] (admin) -- (phys);
    
    \draw [gtu arrow] (view) -- (conc);
    \draw [gtu arrow] (conc) -- (phys);
\end{tikzpicture}
\captionof{figure}{Data Abstraction Levels}
\end{center}

\begin{mnemonicbox}
    \textbf{Mnemonic:} "PCV" (Physical, Conceptual, View - bottom to top)
\end{mnemonicbox}
\end{solutionbox}

\orquestionmarks{1(c)}{7}{Explain advantages and disadvantages of DBMS.}
\begin{solutionbox}
\begin{table}[H]
    \centering
    \caption{Advantages and Disadvantages of DBMS}
    \begin{tabulary}{\linewidth}{LCL}
        \toprule
        \textbf{Advantages} & \textbf{Disadvantages} \\
        \midrule
        \textbf{Data Redundancy Control} & \textbf{High Cost} of software and hardware \\
        \textbf{Data Consistency} & \textbf{Complexity} in design and maintenance \\
        \textbf{Improved Data Security} & \textbf{Performance Impact} with heavy usage \\
        \textbf{Data Sharing} & \textbf{Vulnerability} to system failures \\
        \textbf{Data Independence} & \textbf{Recovery Challenges} after failure \\
        \textbf{Standardized Access} & \textbf{Increased Training Requirements} \\
        \bottomrule
    \end{tabulary}
\end{table}

\begin{mnemonicbox}
    \textbf{Mnemonic:} "BASIC-DV" (Benefits: Access, Security, Independence, Consistency - Drawbacks: Vulnerability)
\end{mnemonicbox}
\end{solutionbox}

\questionmarks{2(a)}{3}{Explain select operation in relational algebra with example}
\begin{solutionbox}
\begin{table}[H]
    \centering
    \caption{Select Operation ($\sigma$)}
    \begin{tabulary}{\linewidth}{LCL}
        \toprule
        \textbf{Feature} & \textbf{Description} \\
        \midrule
        \textbf{Syntax} & $\sigma_{condition}(Relation)$ \\
        \textbf{Function} & Retrieves tuples satisfying condition \\
        \textbf{Example} & $\sigma_{salary>30000}(Employee)$ \\
        \bottomrule
    \end{tabulary}
\end{table}

\begin{mnemonicbox}
    \textbf{Mnemonic:} "SERVe" (Select Exactly Required Values)
\end{mnemonicbox}
\end{solutionbox}

\questionmarks{2(b)}{4}{Define Primary, Foreign, Super, Candidate Keys in DBMS.}
\begin{solutionbox}
\begin{table}[H]
    \centering
    \caption{Key Types}
    \begin{tabulary}{\linewidth}{LCL}
        \toprule
        \textbf{Key Type} & \textbf{Description} \\
        \midrule
        \textbf{Primary Key} & Unique identifier for each record \\
        \textbf{Foreign Key} & Attribute linking to primary key in another table \\
        \textbf{Super Key} & Set of attributes that can uniquely identify records \\
        \textbf{Candidate Key} & Minimal super key that can be primary key \\
        \bottomrule
    \end{tabulary}
\end{table}

\begin{mnemonicbox}
    \textbf{Mnemonic:} "PFSC" (Person First Shows Credentials)
\end{mnemonicbox}
\end{solutionbox}

\questionmarks{2(c)}{7}{Draw E R Diagram of Library Management System.}
\begin{solutionbox}
\begin{center}
\begin{tikzpicture}[gtu block, node distance=2.5cm]
    \node [entity] (book) {BOOK};
    \node [relationship, below of=book] (issued) {is\_issued};
    \node [entity, below of=issued] (issue) {ISSUE};
    
    \node [relationship, left of=issue, xshift=-1cm] (borrows) {borrows};
    \node [entity, left of=borrows, xshift=-1cm] (member) {MEMBER};
    
    \node [relationship, right of=issue, xshift=1cm] (processes) {processes};
    \node [entity, right of=processes, xshift=1cm] (lib) {LIBRARIAN};
    
    % Attributes Book
    \node [attribute, above left of=book] (bid) {\underline{book\_id}};
    \node [attribute, above of=book] (btitle) {title};
    \node [attribute, above right of=book] (bauth) {author};
    
    % Attributes Member
    \node [attribute, above left of=member] (mid) {\underline{member\_id}};
    \node [attribute, left of=member] (mname) {name};
    \node [attribute, below left of=member] (mphone) {phone};
    
    % Attributes Issue
    \node [attribute, below left of=issue] (iid) {\underline{issue\_id}};
    \node [attribute, below right of=issue] (idate) {issue\_date};
    
    % Attributes Librarian
    \node [attribute, above right of=lib] (lid) {\underline{staff\_id}};
    \node [attribute, right of=lib] (lname) {name};
    
    % Connections
    \draw [gtu line] (book) -- (issued);
    \draw [gtu line] (issued) -- (issue);
    \draw [gtu line] (member) -- (borrows);
    \draw [gtu line] (borrows) -- (issue);
    \draw [gtu line] (lib) -- (processes);
    \draw [gtu line] (processes) -- (issue);
    
    \draw [gtu line] (book) -- (bid);
    \draw [gtu line] (book) -- (btitle);
    \draw [gtu line] (book) -- (bauth);
    
    \draw [gtu line] (member) -- (mid);
    \draw [gtu line] (member) -- (mname);
    \draw [gtu line] (member) -- (mphone);
    
    \draw [gtu line] (issue) -- (iid);
    \draw [gtu line] (issue) -- (idate);
    
    \draw [gtu line] (lib) -- (lid);
    \draw [gtu line] (lib) -- (lname);
\end{tikzpicture}
\captionof{figure}{Library Management System ER Diagram}
\end{center}

\begin{mnemonicbox}
    \textbf{Mnemonic:} "LIMB" (Library Items, Members, Borrowing)
\end{mnemonicbox}
\end{solutionbox}

\orquestionmarks{2(a)}{3}{Explain union operation in relational algebra with example.}
\begin{solutionbox}
\begin{table}[H]
    \centering
    \caption{Union Operation ($\cup$)}
    \begin{tabulary}{\linewidth}{LCL}
        \toprule
        \textbf{Feature} & \textbf{Description} \\
        \midrule
        \textbf{Syntax} & $Relation1 \cup Relation2$ \\
        \textbf{Function} & Combines tuples from both relations \\
        \textbf{Requirement} & Both relations must be union-compatible \\
        \bottomrule
    \end{tabulary}
\end{table}

\textbf{Example:} $Students\_CS \cup Students\_IT$

\begin{mnemonicbox}
    \textbf{Mnemonic:} "CUP" (Combining Union of Parts)
\end{mnemonicbox}
\end{solutionbox}

\orquestionmarks{2(b)}{4}{Define Composite attribute and Multivalued attribute with example}
\begin{solutionbox}
\begin{table}[H]
    \centering
    \caption{Attribute Types}
    \begin{tabulary}{\linewidth}{LCL}
        \toprule
        \textbf{Attribute Type} & \textbf{Description} & \textbf{Example} \\
        \midrule
        \textbf{Composite} & Can be divided into smaller subparts & Address (street, city, state, zip) \\
        \textbf{Multivalued} & Can have more than one value & Phone numbers, Email addresses \\
        \bottomrule
    \end{tabulary}
\end{table}

\begin{center}
\begin{tikzpicture}[gtu block, node distance=2cm, level 1/.style={sibling distance=3cm}, level 2/.style={sibling distance=1.5cm}]
    \node [entity] {Person}
        child {node [attribute] {Address (Composite)}
            child {node [attribute] {Street}}
            child {node [attribute] {City}}
            child {node [attribute] {State}}
        }
        child {node [multi attribute] {Phone Numbers}
            child {node [attribute] {Num 1}}
            child {node [attribute] {Num 2}}
        };
\end{tikzpicture}
\captionof{figure}{Attribute Types Hierarchy}
\end{center}

\begin{mnemonicbox}
    \textbf{Mnemonic:} "CoMbo" (Composite has Multiple components)
\end{mnemonicbox}
\end{solutionbox}

\orquestionmarks{2(c)}{7}{Draw E R Diagram of College Management System.}
\begin{solutionbox}
\begin{center}
\begin{tikzpicture}[gtu block, node distance=3cm]
    \node [entity] (dept) {DEPARTMENT};
    \node [relationship, right of=dept] (offers) {offers};
    \node [entity, right of=offers] (course) {COURSE};
    
    \node [relationship, below of=dept] (employs) {employs};
    \node [entity, below of=employs] (faculty) {FACULTY};
    
    \node [relationship, below of=course] (teaches) {teaches};
    
    \node [relationship, above of=dept] (enrolls) {enrolls};
    \node [entity, above of=enrolls] (student) {STUDENT};
    
    \node [relationship, right of=student, xshift=1cm] (registers) {registers};
    \node [entity, right of=registers, xshift=1cm] (enrollment) {ENROLLMENT};
    
    % Connections
    \draw [gtu line] (dept) -- (offers);
    \draw [gtu line] (offers) -- (course);
    \draw [gtu line] (dept) -- (employs);
    \draw [gtu line] (employs) -- (faculty);
    \draw [gtu line] (faculty) -| (teaches);
    \draw [gtu line] (teaches) |- (course);
    \draw [gtu line] (dept) -- (enrolls);
    \draw [gtu line] (enrolls) -- (student);
    \draw [gtu line] (student) -- (registers);
    \draw [gtu line] (registers) -- (enrollment);
    \draw [gtu line] (course) -| (registers); % Course linked to enrollment too
    
    % Attributes
    \node [attribute, left of=dept] {\underline{dept\_id}};
    \node [attribute, left of=student] {\underline{student\_id}};
    \node [attribute, right of=course] {\underline{course\_id}};
    \node [attribute, left of=faculty] {\underline{faculty\_id}};
    \node [attribute, right of=enrollment] {\underline{enrollment\_id}};
\end{tikzpicture}
\captionof{figure}{College Management System ER Diagram}
\end{center}

\begin{mnemonicbox}
    \textbf{Mnemonic:} "DECFS" (Departments, Enrollments, Courses, Faculty, Students)
\end{mnemonicbox}
\end{solutionbox}

\questionmarks{3(a)}{3}{List different data types in SQL and Explain in brief}
\begin{solutionbox}
\begin{table}[H]
    \centering
    \caption{SQL Data Types}
    \begin{tabulary}{\linewidth}{LCL}
        \toprule
        \textbf{Category} & \textbf{Examples} & \textbf{Usage} \\
        \midrule
        \textbf{Numeric} & INT, FLOAT, DECIMAL & Store numbers \\
        \textbf{Character} & CHAR, VARCHAR, TEXT & Store text \\
        \textbf{Date/Time} & DATE, TIME, TIMESTAMP & Store temporal data \\
        \textbf{Boolean} & BOOLEAN & Store true/false values \\
        \textbf{Binary} & BLOB, BINARY & Store binary data \\
        \bottomrule
    \end{tabulary}
\end{table}

\begin{mnemonicbox}
    \textbf{Mnemonic:} "NCDBB" (Numbers, Characters, Dates, Booleans, Binaries)
\end{mnemonicbox}
\end{solutionbox}

\questionmarks{3(b)}{4}{Explain any two DDL Commands with Syntax and Example}
\begin{solutionbox}
\begin{table}[H]
    \centering
    \caption{DDL Commands}
    \begin{tabulary}{\linewidth}{LCL}
        \toprule
        \textbf{Command} & \textbf{Syntax} & \textbf{Example} \\
        \midrule
        \textbf{CREATE} & \code{CREATE TABLE t (cols);} & \code{CREATE TABLE Student (id INT, n TEXT);} \\
        \textbf{ALTER} & \code{ALTER TABLE t ADD c type;} & \code{ALTER TABLE Student ADD email TEXT;} \\
        \bottomrule
    \end{tabulary}
\end{table}

\begin{center}
\begin{tikzpicture}[gtu block, node distance=3cm]
    \node [gtu block, fill=orange!10] (ddl) {DDL Commands};
    \node [gtu process, below left of=ddl] (create) {CREATE: New Objects};
    \node [gtu process, below right of=ddl] (alter) {ALTER: Modify Objects};
    
    \draw [gtu arrow] (ddl) -- (create);
    \draw [gtu arrow] (ddl) -- (alter);
\end{tikzpicture}
\captionof{figure}{DDL Commands}
\end{center}

\begin{mnemonicbox}
    \textbf{Mnemonic:} "CAD" (Create And Define)
\end{mnemonicbox}
\end{solutionbox}

\questionmarks{3(c)}{7}{Write the Output of Following Query.}
\begin{solutionbox}
\begin{table}[H]
    \centering
    \caption{SQL Query Outputs}
    \begin{tabulary}{\linewidth}{LCL}
        \toprule
        \textbf{Function} & \textbf{Result} & \textbf{Explanation} \\
        \midrule
        \textbf{CEIL(123.57)} & 124 & Smallest integer $\ge$ 123.57 \\
        \textbf{CEIL(4.1)} & 5 & Smallest integer $\ge$ 4.1 \\
        \textbf{MOD(12,4)} & 0 & Remainder of 12 $\div$ 4 \\
        \textbf{MOD(10,4)} & 2 & Remainder of 10 $\div$ 4 \\
        \textbf{POWER(2,3)} & 8 & $2^3$ \\
        \textbf{POWER(3,3)} & 27 & $3^3$ \\
        \textbf{ROUND(121.413,1)} & 121.4 & Round to 1 decimal \\
        \textbf{ROUND(121.413,2)} & 121.41 & Round to 2 decimals \\
        \textbf{FLOOR(25.3)} & 25 & Largest integer $\le$ 25.3 \\
        \textbf{FLOOR(25.7)} & 25 & Largest integer $\le$ 25.7 \\
        \textbf{LENGTH('AHMEDABAD')} & 9 & Number of characters \\
        \textbf{ABS(-25)} & 25 & Absolute value \\
        \textbf{ABS(36)} & 36 & Absolute value \\
        \bottomrule
    \end{tabulary}
\end{table}

\begin{mnemonicbox}
    \textbf{Mnemonic:} "CMPRFLA" (Ceiling, Modulus, Power, Round, Floor, Length, Absolute)
\end{mnemonicbox}
\end{solutionbox}

\orquestionmarks{3(a)}{3}{Explain any three Date Functions.}
\begin{solutionbox}
\begin{table}[H]
    \centering
    \caption{Date Functions}
    \begin{tabulary}{\linewidth}{LCL}
        \toprule
        \textbf{Function} & \textbf{Purpose} & \textbf{Example Idea} \\
        \midrule
        \textbf{ADD\_MONTHS} & Adds months to date & Add 3 months to Jan $\to$ Apr \\
        \textbf{MONTHS\_BETWEEN} & Calculates months between dates & Months between Mar and Jan $\to$ 2 \\
        \textbf{SYSDATE} & Returns current date and time & Returns system timestamp \\
        \bottomrule
    \end{tabulary}
\end{table}

\begin{mnemonicbox}
    \textbf{Mnemonic:} "AMS" (Add\_months, Months\_between, Sysdate)
\end{mnemonicbox}
\end{solutionbox}

\orquestionmarks{3(b)}{4}{Explain any two DML Commands with Syntax and Example}
\begin{solutionbox}
\begin{table}[H]
    \centering
    \caption{DML Commands}
    \begin{tabulary}{\linewidth}{LCL}
        \toprule
        \textbf{Command} & \textbf{Syntax} & \textbf{Example} \\
        \midrule
        \textbf{INSERT} & \code{INSERT INTO t VALUES (v1...);} & \code{INSERT INTO S VALUES (1, 'Raj');} \\
        \textbf{UPDATE} & \code{UPDATE t SET c=v WHERE cond;} & \code{UPDATE S SET e='n' WHERE id=1;} \\
        \bottomrule
    \end{tabulary}
\end{table}

\begin{center}
\begin{tikzpicture}[gtu block, node distance=3cm]
    \node [gtu block, fill=green!10] (dml) {DML Commands};
    \node [gtu process, below left of=dml] (insert) {INSERT: Add Records};
    \node [gtu process, below right of=dml] (update) {UPDATE: Modify Records};
    
    \draw [gtu arrow] (dml) -- (insert);
    \draw [gtu arrow] (dml) -- (update);
\end{tikzpicture}
\captionof{figure}{DML Commands}
\end{center}

\begin{mnemonicbox}
    \textbf{Mnemonic:} "IUM" (Insert, Update, Manipulate)
\end{mnemonicbox}
\end{solutionbox}

\orquestionmarks{3(c)}{7}{For the table: EMP(emp\_no, emp\_name, designation, salary, deptno), Write SQL commands for following operations.}
\begin{solutionbox}
\begin{itemize}
    \item \textbf{Create table EMP}:
    \begin{lstlisting}[language=SQL]
CREATE TABLE EMP (
    emp_no INT PRIMARY KEY, 
    emp_name VARCHAR(50), 
    designation VARCHAR(30), 
    salary DECIMAL(10,2), 
    deptno INT
);
    \end{lstlisting}
    
    \item \textbf{Given Columns Select}:
    \begin{lstlisting}[language=SQL]
SELECT emp_no, emp_name, designation, salary, deptno FROM EMP;
    \end{lstlisting}
    
    \item \textbf{Name starts with 'p'}:
    \begin{lstlisting}[language=SQL]
SELECT * FROM EMP WHERE emp_name LIKE 'p%';
    \end{lstlisting}
    
    \item \textbf{Department wise salary}:
    \begin{lstlisting}[language=SQL]
SELECT deptno, SUM(salary) AS total_salary 
FROM EMP GROUP BY deptno;
    \end{lstlisting}
    
    \item \textbf{Add email\_id}:
    \begin{lstlisting}[language=SQL]
ALTER TABLE EMP ADD email_id VARCHAR(100);
    \end{lstlisting}
    
    \item \textbf{Rename column}:
    \begin{lstlisting}[language=SQL]
ALTER TABLE EMP RENAME COLUMN designation TO post;
    \end{lstlisting}
    
    \item \textbf{Delete all records}:
    \begin{lstlisting}[language=SQL]
DELETE FROM person;
    \end{lstlisting}
\end{itemize}

\begin{mnemonicbox}
    \textbf{Mnemonic:} "CSDAACD" (Create, Select, Display, Aggregate, Add, Change, Delete)
\end{mnemonicbox}
\end{solutionbox}

\questionmarks{4(a)}{3}{List different aggregate functions and explain any one with syntax and example.}
\begin{solutionbox}
\begin{table}[H]
    \centering
    \caption{Aggregate Functions}
    \begin{tabulary}{\linewidth}{LCL}
        \toprule
        \textbf{Function} & \textbf{Purpose} \\
        \midrule
        \textbf{SUM} & Calculates total \\
        \textbf{AVG} & Calculates average \\
        \textbf{COUNT} & Counts number of rows \\
        \textbf{MAX} & Finds maximum value \\
        \textbf{MIN} & Finds minimum value \\
        \bottomrule
    \end{tabulary}
\end{table}

\textbf{Example (AVG)}: \code{SELECT AVG(salary) FROM Employee;}

\begin{mnemonicbox}
    \textbf{Mnemonic:} "SCAMM" (Sum, Count, Avg, Max, Min)
\end{mnemonicbox}
\end{solutionbox}

\questionmarks{4(b)}{4}{Define the transaction with example.}
\begin{solutionbox}
\textbf{Transaction}: Logical unit of work that must be completely processed or completely fail.

\begin{table}[H]
    \centering
    \caption{Transaction Properties}
    \begin{tabulary}{\linewidth}{LCL}
        \toprule
        \textbf{Concept} & \textbf{Description} \\
        \midrule
        \textbf{ACID} & Atomicity, Consistency, Isolation, Durability \\
        \textbf{States} & Active, Partially Committed, Committed, Failed, Aborted \\
        \bottomrule
    \end{tabulary}
\end{table}

\begin{lstlisting}[language=SQL]
BEGIN TRANSACTION;
    UPDATE Accounts SET balance = balance - 5000 WHERE acc_no = 'A123';
    UPDATE Accounts SET balance = balance + 5000 WHERE acc_no = 'B456';
COMMIT;
\end{lstlisting}

\begin{mnemonicbox}
    \textbf{Mnemonic:} "TAPS" (Transaction As Process Set)
\end{mnemonicbox}
\end{solutionbox}

\questionmarks{4(c)}{7}{What is an Operator in SQL? Explain Arithmetic and Logical operators with Syntax and Example}
\begin{solutionbox}
\begin{table}[H]
    \centering
    \caption{SQL Operators}
    \begin{tabulary}{\linewidth}{LCL}
        \toprule
        \textbf{Type} & \textbf{Operators} & \textbf{Example} \\
        \midrule
        \textbf{Arithmetic} & +, -, *, /, \% & 5 + 3 = 8 \\
        \textbf{Logical} & AND & salary > 3k AND dept = 'IT' \\
        \textbf{Logical} & OR & salary > 5k OR dept = 'HR' \\
        \textbf{Logical} & NOT & NOT (condition) \\
        \bottomrule
    \end{tabulary}
\end{table}

\begin{mnemonicbox}
    \textbf{Mnemonic:} "ASMDOLA" (Add, Subtract, Multiply, Divide, OR, AND, NOT)
\end{mnemonicbox}
\end{solutionbox}

\orquestionmarks{4(a)}{3}{List different numeric functions and explain any one with syntax and example.}
\begin{solutionbox}
\begin{table}[H]
    \centering
    \caption{Numeric Functions}
    \begin{tabulary}{\linewidth}{LCL}
        \toprule
        \textbf{Function} & \textbf{Purpose} \\
        \midrule
        \textbf{ROUND} & Rounds to decimal places \\
        \textbf{TRUNC} & Truncates to decimal places \\
        \textbf{CEIL} & Smallest integer $\ge$ number \\
        \textbf{FLOOR} & Largest integer $\le$ number \\
        \textbf{ABS} & Absolute value \\
        \bottomrule
    \end{tabulary}
\end{table}

\textbf{Example (ROUND)}: \code{ROUND(125.679, 2)} $\to$ 125.68

\begin{mnemonicbox}
    \textbf{Mnemonic:} "RTCFA" (Round, Truncate, Ceiling, Floor, Absolute)
\end{mnemonicbox}
\end{solutionbox}

\orquestionmarks{4(b)}{4}{List various database operations of a transaction.}
\begin{solutionbox}
\begin{table}[H]
    \centering
    \caption{Transaction Operations}
    \begin{tabulary}{\linewidth}{LCL}
        \toprule
        \textbf{Operation} & \textbf{Description} \\
        \midrule
        \textbf{BEGIN} & Marks start \\
        \textbf{READ} & Retrieves data \\
        \textbf{WRITE} & Modifies data \\
        \textbf{COMMIT} & Saves changes \\
        \textbf{ROLLBACK} & Undoes changes \\
        \textbf{SAVEPOINT} & Partial rollback point \\
        \bottomrule
    \end{tabulary}
\end{table}

\begin{center}
\begin{tikzpicture}[gtu flow]
    \node [gtu start] (begin) {BEGIN};
    \node [gtu process, below of=begin] (rw) {READ / WRITE};
    \node [gtu decision, below of=rw] (check) {Success?};
    \node [gtu stop, below left of=check, xshift=-1cm] (commit) {COMMIT};
    \node [gtu stop, below right of=check, xshift=1cm] (rollback) {ROLLBACK};
    
    \draw [gtu arrow] (begin) -- (rw);
    \draw [gtu arrow] (rw) -- (check);
    \draw [gtu arrow] (check) -| node[near start] {Yes} (commit);
    \draw [gtu arrow] (check) -| node[near start] {No} (rollback);
\end{tikzpicture}
\captionof{figure}{Transaction Flow}
\end{center}

\begin{mnemonicbox}
    \textbf{Mnemonic:} "BRWCRS" (Begin, Read, Write, Commit, Rollback, Savepoint)
\end{mnemonicbox}
\end{solutionbox}

\orquestionmarks{4(c)}{7}{What is join? Explain different types of joins with syntax and example.}
\begin{solutionbox}
\begin{table}[H]
    \centering
    \caption{Join Types}
    \begin{tabulary}{\linewidth}{LCL}
        \toprule
        \textbf{Join Type} & \textbf{Description} \\
        \midrule
        \textbf{INNER JOIN} & Match in both tables \\
        \textbf{LEFT JOIN} & All from left, matched from right \\
        \textbf{RIGHT JOIN} & All from right, matched from left \\
        \textbf{FULL JOIN} & Match in either table \\
        \textbf{SELF JOIN} & Join table to itself \\
        \bottomrule
    \end{tabulary}
\end{table}

\begin{center}
\begin{tikzpicture}[gtu block, node distance=2cm]
    \node [gtu block, fill=purple!10] (joins) {JOIN TYPES};
    \node [gtu process, below of=joins] (inner) {INNER};
    \node [gtu process, left of=inner, xshift=-1cm] (left) {LEFT};
    \node [gtu process, right of=inner, xshift=1cm] (right) {RIGHT};
    \node [gtu process, below of=left] (full) {FULL};
    \node [gtu process, below of=right] (self) {SELF};
    
    \draw [gtu arrow] (joins) -- (inner);
    \draw [gtu arrow] (joins) -- (left);
    \draw [gtu arrow] (joins) -- (right);
    \draw [gtu arrow] (joins) -- (full);
    \draw [gtu arrow] (joins) -- (self);
\end{tikzpicture}
\captionof{figure}{Types of Joins}
\end{center}

\begin{mnemonicbox}
    \textbf{Mnemonic:} "ILRFS" (Inner, Left, Right, Full, Self)
\end{mnemonicbox}
\end{solutionbox}

\questionmarks{5(a)}{3}{Convert the customer relation into 1NF shown below.}
\begin{solutionbox}
\textbf{Customer Table (1NF)}:
\begin{table}[H]
    \centering
    \caption{Customer in 1NF}
    \begin{tabulary}{\linewidth}{LCLLL}
        \toprule
        \textbf{cid} & \textbf{name} & \textbf{society} & \textbf{city} & \textbf{Contact\_no} \\
        \midrule
        CO1 & Riya & Amu aavas & Anand & 5322332123 \\
        CO2 & Jiya & Sardar colony & Ahmedabad & 5326521456 \\
        CO2 & Jiya & Sardar colony & Ahmedabad & 5265232849 \\
        \bottomrule
    \end{tabulary}
\end{table}

\begin{mnemonicbox}
    \textbf{Mnemonic:} "AFM" (Atomic values, Flatten Multivalued attributes)
\end{mnemonicbox}
\end{solutionbox}

\questionmarks{5(b)}{4}{List and Explain ACID properties of transaction.}
\begin{solutionbox}
\begin{table}[H]
    \centering
    \caption{ACID Properties}
    \begin{tabulary}{\linewidth}{LCL}
        \toprule
        \textbf{Property} & \textbf{Description} \\
        \midrule
        \textbf{Atomicity} & All or nothing \\
        \textbf{Consistency} & Database remains valid \\
        \textbf{Isolation} & Transactions don't interfere \\
        \textbf{Durability} & Changes are permanent \\
        \bottomrule
    \end{tabulary}
\end{table}

\begin{center}
\begin{tikzpicture}[gtu block, node distance=2.5cm]
    \node [gtu block, fill=yellow!10] (acid) {ACID};
    \node [gtu process, above left of=acid] (a) {Atomicity};
    \node [gtu process, above right of=acid] (c) {Consistency};
    \node [gtu process, below right of=acid] (i) {Isolation};
    \node [gtu process, below left of=acid] (d) {Durability};
    
    \draw [gtu arrow] (acid) -- (a);
    \draw [gtu arrow] (acid) -- (c);
    \draw [gtu arrow] (acid) -- (i);
    \draw [gtu arrow] (acid) -- (d);
\end{tikzpicture}
\captionof{figure}{ACID Properties}
\end{center}

\begin{mnemonicbox}
    \textbf{Mnemonic:} "ACID" (Atomicity, Consistency, Isolation, Durability)
\end{mnemonicbox}
\end{solutionbox}

\questionmarks{5(c)}{7}{List different types of functional dependencies and explain each using example.}
\begin{solutionbox}
\begin{table}[H]
    \centering
    \caption{Functional Dependencies}
    \begin{tabulary}{\linewidth}{LCL}
        \toprule
        \textbf{Type} & \textbf{Description} & \textbf{Example} \\
        \midrule
        \textbf{Trivial FD} & $Y \subseteq X$ & $\{ID, Name\} \to \{Name\}$ \\
        \textbf{Non-trivial FD} & $Y \not\subseteq X$ & $\{ID\} \to \{Name\}$ \\
        \textbf{Partial FD} & Part of key determines attr & $\{Course, Student\} \to CourseName$ \\
        \textbf{Transitive FD} & $X \to Y \to Z$ & $Student \to Dept \to DeptName$ \\
        \textbf{Multivalued FD} & One determines set of another & $Course \to\to Textbook$ \\
        \bottomrule
    \end{tabulary}
\end{table}

\begin{center}
\begin{tikzpicture}[gtu block, node distance=2.5cm]
    \node [attribute] (a) {A};
    \node [attribute, right of=a] (b) {B};
    \node [attribute, right of=b] (c) {C};
    
    \draw [gtu arrow] (a) -- node[above] {determines} (b);
    \draw [gtu arrow] (b) -- node[above] {determines} (c);
    \draw [gtu arrow, bend right] (a) to node[below] {Transitive} (c);
\end{tikzpicture}
\captionof{figure}{Transitive Dependency}
\end{center}

\begin{mnemonicbox}
    \textbf{Mnemonic:} "TNPTMv" (Trivial, Non-trivial, Partial, Transitive, Multivalued)
\end{mnemonicbox}
\end{solutionbox}

\orquestionmarks{5(a)}{3}{Convert the Depositor\_Account relation into 2NF.}
\begin{solutionbox}
\textbf{Account Table (2NF)}:
\begin{table}[H]
    \centering
    \begin{tabulary}{\linewidth}{LCL}
        \toprule
        \textbf{ano} & \textbf{balance} & \textbf{bname} \\
        \midrule
         & & \\
        \bottomrule
    \end{tabulary}
\end{table}

\textbf{Depositor Table (2NF)}:
\begin{table}[H]
    \centering
    \begin{tabulary}{\linewidth}{LCL}
        \toprule
        \textbf{cid} & \textbf{ano} & \textbf{access\_date} \\
        \midrule
         & & \\
        \bottomrule
    \end{tabulary}
\end{table}

\begin{mnemonicbox}
    \textbf{Mnemonic:} "RPKD" (Remove Partial Key Dependencies)
\end{mnemonicbox}
\end{solutionbox}

\orquestionmarks{5(b)}{4}{Explain conflict serializability.}
\begin{solutionbox}
\begin{table}[H]
    \centering
    \caption{Conflict Serializability}
    \begin{tabulary}{\linewidth}{LCL}
        \toprule
        \textbf{Concept} & \textbf{Description} \\
        \midrule
        \textbf{Definition} & Equivalent to serial schedule \\
        \textbf{Conflict Ops} & RW, WR, WW on same item \\
        \textbf{Testing} & No cycles in precedence graph \\
        \bottomrule
    \end{tabulary}
\end{table}

\begin{center}
\begin{tikzpicture}[gtu state, node distance=3cm]
    \node [gtu state] (t1) {T1};
    \node [gtu state, right of=t1] (t2) {T2};
    \node [gtu state, right of=t2] (t3) {T3};
    
    \draw [gtu arrow] (t1) -- (t2);
    \draw [gtu arrow] (t2) -- (t3);
    \node [below of=t2, yshift=1cm] {No Cycle $\implies$ Serializable};
\end{tikzpicture}
\captionof{figure}{Precedence Graph}
\end{center}

\begin{mnemonicbox}
    \textbf{Mnemonic:} "COGS" (Conflict Operations Graph Serializable)
\end{mnemonicbox}
\end{solutionbox}

\orquestionmarks{5(c)}{7}{Explain 3NF normalization with example}
\begin{solutionbox}
\begin{table}[H]
    \centering
    \caption{Normal Forms}
    \begin{tabulary}{\linewidth}{LCL}
        \toprule
        \textbf{Form} & \textbf{Definition} & \textbf{Example} \\
        \midrule
        \textbf{1NF} & Atomic & Split phone numbers \\
        \textbf{2NF} & No Partial Dep & Split Order details \\
        \textbf{3NF} & No Transitive Dep & Split Student Dept \\
        \bottomrule
    \end{tabulary}
\end{table}

\textbf{Example}: \code{Emp(ID, Name, DeptID, DeptName)} $\to$ \\
\code{Emp(ID, Name, DeptID)} + \code{Dept(DeptID, DeptName)}

\begin{center}
\begin{tikzpicture}[gtu flow]
    \node [gtu start] (nf1) {1NF};
    \node [gtu process, right of=nf1, xshift=2cm] (nf2) {2NF};
    \node [gtu process, right of=nf2, xshift=2cm] (nf3) {3NF};
    
    \draw [gtu arrow] (nf1) -- node[above] {Remove Partial} (nf2);
    \draw [gtu arrow] (nf2) -- node[above] {Remove Transitive} (nf3);
\end{tikzpicture}
\captionof{figure}{Normalization Flow}
\end{center}

\begin{mnemonicbox}
    \textbf{Mnemonic:} "APTN" (Atomic, Partial, Transitive, Normalized)
\end{mnemonicbox}
\end{solutionbox}

\end{document}
