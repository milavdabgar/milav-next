\documentclass{article}

% content/resources/templates/preamble.tex
\usepackage[margin=0.6in]{geometry}
\author{Milav Dabgar}
\usepackage{amsmath,amssymb,amsthm}
\usepackage{booktabs}
\usepackage{multirow}
\usepackage{xcolor}
\usepackage{tcolorbox}
\tcbuselibrary{breakable,skins}
\usepackage[colorlinks=true,linkcolor=blue]{hyperref}
\usepackage{titlesec}
\usepackage{enumitem}
\usepackage{tikz}
\usepackage{pgfplots}
\usepackage{circuitikz}
\usepackage[version=4]{mhchem}
\usepackage{longtable}
\usepackage{array}
\usepackage{float}
\usepackage{caption}
\usepackage{listings}

\lstset{
  basicstyle=\small\ttfamily,
  breaklines=true,
  breakatwhitespace=false,
  postbreak=\mbox{\textcolor{red}{$\hookrightarrow$}\space},
  float=false,
  numbers=left,
  numberstyle=\tiny\color{gray},
  numbersep=10pt,
  xleftmargin=2em,
  keywordstyle=\color{blue},
  commentstyle=\color{green!60!black},
  stringstyle=\color{purple},
  backgroundcolor=\color{gray!5},
  showstringspaces=false,
  tabsize=2,
  captionpos=b,
  keepspaces=true,
  columns=flexible
}

\pgfplotsset{compat=1.18}
\usetikzlibrary{shapes,arrows,positioning,calc,patterns,decorations.pathmorphing,decorations.markings,arrows.meta}

% Color scheme
\definecolor{headcolor}{RGB}{0,102,204}
\definecolor{keycolor}{RGB}{220,20,60}
\definecolor{solutioncolor}{RGB}{34,139,34}
\definecolor{mnemoniccolor}{RGB}{148,0,211}
\definecolor{codecolor}{RGB}{0,0,100}

% Spacing
\setlength{\parskip}{3pt}
\setlist[itemize]{nosep}
\setlist[enumerate]{nosep}

% Title formatting
\titleformat{\section}{\Large\bfseries\color{headcolor}}{\thesection}{1em}{}
\titleformat{\subsection}{\large\bfseries\color{headcolor}}{\thesubsection}{1em}{}

% Pandoc tightlist compatibility
\providecommand{\tightlist}{%
  \setlength{\itemsep}{0pt}\setlength{\parskip}{0pt}}

% Pandoc longtable compatibility
\newcounter{none}
\def\thenone{}


% content/resources/templates/gujarati-boxes.tex
\usepackage{fontspec}
\usepackage{polyglossia}

% Set Gujarati as main language (document is primarily in Gujarati)
% Note: gloss-gujarati.ldf doesn't exist in polyglossia, but it will use hyphenation patterns
\setdefaultlanguage{gujarati}
\setotherlanguage{english}

% Configure Gujarati font properly
% Use Language=Default to prevent polyglossia from trying to add language-specific features
% that don't exist for Gujarati, which causes "empty feature" warnings
\newfontfamily\gujaratifont[Script=Gujarati,AutoFakeBold=2.5,AutoFakeSlant=0.3]{Noto Sans Gujarati}
\setmainfont[Script=Gujarati,AutoFakeBold=2.5,AutoFakeSlant=0.3]{Noto Sans Gujarati}
% Use Noto Sans Gujarati for monospace to support Gujarati in text
\setmonofont[Scale=0.9]{Noto Sans Gujarati}

% Configure English to use the same font
\newfontfamily\englishfont[Script=Gujarati,AutoFakeBold=2.5,AutoFakeSlant=0.3]{Noto Sans Gujarati}

% Translations for polyglossia
\gappto\captionsgujarati{
  \renewcommand{\tablename}{કોષ્ટક}
  \renewcommand{\figurename}{આકૃતિ}
}

% Helper for TikZ nodes to ensure Gujarati font
\newcommand{\gu}[1]{{\gujaratifont #1}}

% Custom environments
\newtcolorbox{solutionbox}{
    breakable,
    enhanced,
    colback=solutioncolor!5!white,
    colframe=solutioncolor!75!black,
    fonttitle=\bfseries,
    title=જવાબ
}

\newtcolorbox{solutionboxnobreak}{
 colback=solutioncolor!5!white,
 colframe=solutioncolor!75!black,
 fonttitle=\bfseries,
 title=જવાબ
}

\newtcolorbox{keyformula}{
 breakable,
 enhanced,
 colback=keycolor!5!white,
 colframe=keycolor!75!black,
 fonttitle=\bfseries,
 title=રાસાયણિક સમીકરણ/સૂત્ર
}

\newtcolorbox{mnemonicbox}{
 breakable,
 enhanced,
 colback=mnemoniccolor!5!white,
 colframe=mnemoniccolor!75!black,
 fonttitle=\bfseries,
 title=મેમરી ટ્રીક
}


% Custom commands for GTU solutions
% This file defines semantic commands for consistent formatting

% Question command with automatic formatting
\newcommand{\question}[2]{%
  \section*{Question #1}%
  \textbf{#2}%
}

% OR question variant
\newcommand{\questionor}[2]{%
  \section*{Question #1 OR}%
  \textbf{#2}%
}

% Proper table environment with caption
\newenvironment{answertable}[1]{%
  \begin{table}[htbp]
  \centering
  \caption{#1}
}{%
  \end{table}
}

% Proper figure environment for diagrams
\newenvironment{answerdiagram}[1]{%
  \begin{figure}[htbp]
  \centering
  \caption{#1}
}{%
  \end{figure}
}

% Semantic markup for key terms
\newcommand{\keyword}[1]{\textbf{#1}}
\newcommand{\code}[1]{\texttt{#1}}
\newcommand{\classname}[1]{\texttt{#1}}
\newcommand{\methodname}[1]{\texttt{#1}}

% Proper quotation marks
\newcommand{\mnemonic}[1]{``#1''}

\usetikzlibrary{calc,positioning,shapes,arrows,automata}
\tikzset{
    entity/.style={rectangle, draw, fill=white, align=center, minimum height=2em, font=\small, thick},
    relationship/.style={diamond, draw, fill=white, align=center, aspect=2, font=\small, thick},
    attribute/.style={ellipse, draw, fill=white, align=center, font=\small},
    gtu line/.style={draw, thick},
    gtu arrow/.style={draw, -latex, thick}
}

\title{Database Management System (1333204) - Summer 2025 Solution}
\date{May 17, 2025}

\begin{document}
\maketitle

\questionmarks{1(અ)}{3}{ટૂંકી નોંધ લખો: ડેટા ડિક્શનરી}
\begin{solutionbox}
\textbf{ડેટા ડિક્શનરી} એ કેન્દ્રીય ભંડાર છે જે ડેટાબેઝ બંધારણ, તત્વો અને સંબંધો વિશે મેટાડેટા સંગ્રહિત કરે છે.

\begin{table}[H]
    \centering
    \caption{ડેટા ડિક્શનરી ઘટકો}
    \begin{tabulary}{\linewidth}{LC}
        \toprule
        \textbf{ઘટક} & \textbf{વર્ણન} \\
        \midrule
        \textbf{ટેબલ નામો} & ડેટાબેઝમાં બધા ટેબલોની યાદી \\
        \textbf{કૉલમ વિગતો} & ડેટા પ્રકારો, મર્યાદાઓ, લંબાઈ \\
        \textbf{સંબંધો} & ફોરેન કી કનેક્શન્સ \\
        \textbf{ઇન્ડેક્સ} & પ્રદર્શન ઑપ્ટિમાઇઝેશન બંધારણો \\
        \bottomrule
    \end{tabulary}
\end{table}

\textbf{મુખ્ય લક્ષણો:}
\begin{itemize}
    \item \textbf{મેટાડેટા સ્ટોરેજ}: ડેટા બંધારણ વિશે માહિતી સમાવે છે
    \item \textbf{ડેટા અખંડિતતા}: સુસંગતતા નિયમો અને મર્યાદાઓ જાળવે છે
    \item \textbf{દસ્તાવેજીકરણ}: વ્યાપક ડેટાબેઝ દસ્તાવેજીકરણ પ્રદાન કરે છે
\end{itemize}

\begin{mnemonicbox}
    \textbf{મેમરી ટ્રીક:} "ડેટા ડિક્શનરી વિગતો આપે"
\end{mnemonicbox}
\end{solutionbox}

\questionmarks{1(બ)}{4}{વ્યાખ્યા આપો (i) E-R મોડેલ (ii) એન્ટિટી (iii) એન્ટિટી સેટ અને (iv) ગુણધર્મો}
\begin{solutionbox}
\begin{table}[H]
    \centering
    \caption{ER મોડેલ વ્યાખ્યાઓ}
    \begin{tabulary}{\linewidth}{LC}
        \toprule
        \textbf{શબ્દ} & \textbf{વ્યાખ્યા} \\
        \midrule
        \textbf{E-R મોડેલ} & એન્ટિટી અને સંબંધોનો ઉપયોગ કરતો કન્સેપ્ચ્યુઅલ ડેટા મોડેલ \\
        \textbf{એન્ટિટી} & સ્વતંત્ર અસ્તિત્વ ધરાવતો વાસ્તવિક વિશ્વનો ઑબ્જેક્ટ \\
        \textbf{એન્ટિટી સેટ} & સમાન પ્રકારની સમાન એન્ટિટીઓનો સંગ્રહ \\
        \textbf{ગુણધર્મો} & એન્ટિટીની લાક્ષણિકતાઓનું વર્ણન કરતા ગુણધર્મો \\
        \bottomrule
    \end{tabulary}
\end{table}

\begin{center}
\begin{tikzpicture}[gtu block, node distance=2cm]
    \node [entity] (e1) {એન્ટિટી A};
    \node [relationship, right of=e1, xshift=2cm] (rel) {સંબંધ};
    \node [entity, right of=rel, xshift=2cm] (e2) {એન્ટિટી B};
    
    \node [attribute, below of=e1, node distance=1.5cm] (a1) {ગુણધર્મો};
    \node [attribute, below of=e2, node distance=1.5cm] (a2) {ગુણધર્મો};

    \draw [gtu arrow] (e1) -- (rel);
    \draw [gtu arrow] (rel) -- (e2);
    \draw [gtu line] (e1) -- (a1);
    \draw [gtu line] (e2) -- (a2);
\end{tikzpicture}
\captionof{figure}{ER મોડેલ ઘટકો}
\end{center}

\textbf{મુખ્ય મુદ્દાઓ:}
\begin{itemize}
    \item \textbf{કન્સેપ્ચ્યુઅલ ડિઝાઇન}: ઉચ્ચ સ્તરનો ડેટાબેઝ ડિઝાઇન અભિગમ
    \item \textbf{વિઝ્યુઅલ રજૂઆત}: સ્પષ્ટ સમજ માટે આકૃતિઓનો ઉપયોગ
\end{itemize}

\begin{mnemonicbox}
    \textbf{મેમરી ટ્રીક:} "એન્ટિટી સંબંધો અર્થપૂર્ણ રીતે"
\end{mnemonicbox}
\end{solutionbox}

\questionmarks{1(ક)}{7}{DBMS ના ફાયદા સમજાવો}
\begin{solutionbox}
\begin{table}[H]
    \centering
    \caption{DBMS ફાયદા}
    \begin{tabulary}{\linewidth}{LC}
        \toprule
        \textbf{ફાયદો} & \textbf{લાભ} \\
        \midrule
        \textbf{ડેટા સ્વતંત્રતા} & એપ્લિકેશન ડેટા સ્ટ્રક્ચર ફેરફારોથી અલગ \\
        \textbf{ડેટા શેરિંગ} & બહુવિધ વપરાશકર્તાઓ એકસાથે સમાન ડેટા એક્સેસ કરે \\
        \textbf{ડેટા સુરક્ષા} & એક્સેસ કંટ્રોલ અને પ્રમાણીકરણ પદ્ધતિઓ \\
        \textbf{ડેટા અખંડિતતા} & મર્યાદાઓ દ્વારા સુસંગતતા જાળવવામાં આવે છે \\
        \textbf{બેકઅપ અને રિકવરી} & આપોઆપ ડેટા સંરક્ષણ અને પુનઃસ્થાપન \\
        \textbf{ઘટાડેલી રીડન્ડન્સી} & ડુપ્લિકેટ ડેટા સ્ટોરેજ દૂર કરે છે \\
        \bottomrule
    \end{tabulary}
\end{table}

\textbf{મુખ્ય લાભો:}
\begin{itemize}
    \item \textbf{કેન્દ્રીકૃત નિયંત્રણ}: ડેટા વ્યવસ્થાપનનો એક બિંદુ
    \item \textbf{ખર્ચ અસરકારકતા}: વિકાસ અને જાળવણીનો ખર્ચ ઘટાડે છે
    \item \textbf{ડેટા સુસંગતતા}: એપ્લિકેશન્સમાં એકસમાન ડેટા સુનિશ્ચિત કરે છે
    \item \textbf{સંગામિત એક્સેસ}: બહુવિધ વપરાશકર્તાઓ એકસાથે કામ કરી શકે છે
    \item \textbf{ક્વેરી ઑપ્ટિમાઇઝેશન}: કાર્યક્ષમ ડેટા પુનઃપ્રાપ્તિ પદ્ધતિઓ
\end{itemize}

\begin{mnemonicbox}
    \textbf{મેમરી ટ્રીક:} "ડેટાબેઝ બિઝનેસને બહેતર બનાવે"
\end{mnemonicbox}
\end{solutionbox}

\questionmarks{1(ક) અથવા}{7}{DBMS નું આર્કિટેક્ચર સમજાવો}
\begin{solutionbox}
\begin{center}
\begin{tikzpicture}[gtu block, node distance=2cm]
    \node [block] (ext) {બાહ્ય સ્તર\\વપરાશકર્તા દૃશ્યો};
    \node [block, below of=ext] (con) {કન્સેપ્ચ્યુઅલ સ્તર\\લોજિકલ સ્કીમા};
    \node [block, below of=con] (int) {આંતરિક સ્તર\\ભૌતિક સ્ટોરેજ};
    
    \node [above of=ext, xshift=-2cm, node distance=1.5cm] (u1) {વપરાશકર્તા 1};
    \node [above of=ext, xshift=2cm, node distance=1.5cm] (u2) {વપરાશકર્તા 2};
    \node [right of=con, xshift=3cm] (dba) {DBA};
    \node [right of=int, xshift=3cm] (sys) {સિસ્ટમ};

    \draw [gtu arrow] (u1) -- (ext);
    \draw [gtu arrow] (u2) -- (ext);
    \draw [gtu arrow] (ext) -- (con);
    \draw [gtu arrow] (con) -- (int);
    \draw [gtu arrow] (dba) -- (con);
    \draw [gtu arrow] (sys) -- (int);
\end{tikzpicture}
\captionof{figure}{ત્રણ-સ્તરીય DBMS આર્કિટેક્ચર}
\end{center}

\begin{table}[H]
    \centering
    \caption{આર્કિટેક્ચર સ્તરો}
    \begin{tabulary}{\linewidth}{LCL}
        \toprule
        \textbf{સ્તર} & \textbf{હેતુ} & \textbf{વપરાશકર્તાઓ} \\
        \midrule
        \textbf{બાહ્ય} & વ્યક્તિગત વપરાશકર્તા દૃશ્યો & અંતિમ વપરાશકર્તાઓ, એપ્લિકેશન્સ \\
        \textbf{કન્સેપ્ચ્યુઅલ} & સંપૂર્ણ લોજિકલ બંધારણ & ડેટાબેઝ એડમિનિસ્ટ્રેટર \\
        \textbf{આંતરિક} & ભૌતિક સ્ટોરેજ વિગતો & સિસ્ટમ પ્રોગ્રામર્સ \\
        \bottomrule
    \end{tabulary}
\end{table}

\textbf{મુખ્ય લક્ષણો:}
\begin{itemize}
    \item \textbf{ડેટા સ્વતંત્રતા}: એક સ્તરે ફેરફારો અન્યને અસર કરતા નથી
    \item \textbf{સુરક્ષા}: વિવિધ વપરાશકર્તાઓ માટે વિવિધ એક્સેસ સ્તરો
    \item \textbf{અમૂર્તતા}: વપરાશકર્તાઓથી જટિલતા છુપાવે છે
\end{itemize}

\begin{mnemonicbox}
    \textbf{મેમરી ટ્રીક:} "બાહ્ય કન્સેપ્ચ્યુઅલ આંતરિક આર્કિટેક્ચર"
\end{mnemonicbox}
\end{solutionbox}

\questionmarks{2(અ)}{3}{UNIQUE KEY અને PRIMARY KEY સમજાવો}
\begin{solutionbox}
\begin{table}[H]
    \centering
    \caption{કી સરખામણી}
    \begin{tabulary}{\linewidth}{LCC}
        \toprule
        \textbf{લક્ષણ} & \textbf{PRIMARY KEY} & \textbf{UNIQUE KEY} \\
        \midrule
        \textbf{Null મૂલ્યો} & મંજૂર નથી & એક null મંજૂર \\
        \textbf{ટેબલ દીઠ સંખ્યા} & માત્ર એક & બહુવિધ મંજૂર \\
        \textbf{ઇન્ડેક્સ બનાવટ} & આપોઆપ clustered & આપોઆપ non-clustered \\
        \textbf{હેતુ} & એન્ટિટી ઓળખ & ડેટા વિશિષ્ટતા \\
        \bottomrule
    \end{tabulary}
\end{table}

\textbf{મુખ્ય તફાવતો:}
\begin{itemize}
    \item \textbf{પ્રાથમિક કી}: દરેક રેકોર્ડને વિશિષ્ટ રીતે ઓળખે છે, null હોઈ શકતી નથી
    \item \textbf{યુનિક કી}: વિશિષ્ટતા સુનિશ્ચિત કરે છે પણ એક null મૂલ્યની મંજૂરી આપે છે
\end{itemize}

\begin{mnemonicbox}
    \textbf{મેમરી ટ્રીક:} "પ્રાથમિક નલને અટકાવે, યુનિક નલને સમજે"
\end{mnemonicbox}
\end{solutionbox}

\questionmarks{2(બ)}{4}{ER ડાયાગ્રામમાં એન્ટિટીની Participation પર ટૂંકી નોંધ લખો}
\begin{solutionbox}
\begin{table}[H]
    \centering
    \caption{Participation પ્રકારો}
    \begin{tabulary}{\linewidth}{LCL}
        \toprule
        \textbf{પ્રકાર} & \textbf{વર્ણન} & \textbf{પ્રતીક} \\
        \midrule
        \textbf{કુલ Participation} & દરેક એન્ટિટી સહભાગી થવી જ જોઈએ & ડબલ લાઇન \\
        \textbf{આંશિક Participation} & કેટલીક એન્ટિટી સહભાગી ન પણ થઈ શકે & સિંગલ લાઇન \\
        \bottomrule
    \end{tabulary}
\end{table}

\begin{center}
\begin{tikzpicture}[gtu block, node distance=2.5cm]
    \node [entity] (emp) {કર્મચારી};
    \node [relationship, right of=emp, xshift=2cm] (works) {કામ\_કરે};
    \node [entity, right of=works, xshift=2cm] (dept) {વિભાગ};
    
    \draw [double, thick] (emp) -- (works) node[midway, below] {(કુલ)};
    \draw [gtu line] (works) -- (dept) node[midway, below] {(આંશિક)};
\end{tikzpicture}
\captionof{figure}{Participation ઉદાહરણ}
\end{center}

\textbf{મુખ્ય સંકેતો:}
\begin{itemize}
    \item \textbf{ફરજિયાત Participation}: દરેક ઇન્સ્ટન્સ સંકળાયેલું હોવું જ જોઈએ
    \item \textbf{વૈકલ્પિક Participation}: કેટલાક ઇન્સ્ટન્સ સંકળાયેલા ન હોઈ શકે
    \item \textbf{બિઝનેસ નિયમો}: વાસ્તવિક વિશ્વની મર્યાદાઓને પ્રતિબિંબિત કરે છે
\end{itemize}

\begin{mnemonicbox}
    \textbf{મેમરી ટ્રીક:} "કુલ Participation બધાની જરૂર"
\end{mnemonicbox}
\end{solutionbox}

\questionmarks{2(ક)}{7}{ER ડાયાગ્રામ માટે Generalization concept વિગતવાર વર્ણન કરો}
\begin{solutionbox}
\begin{center}
\begin{tikzpicture}[gtu block, node distance=2cm]
    \node [entity] (person) {PERSON};
    \node [relationship, below of=person] (isa) {IS-A};
    \node [entity, below of=isa, xshift=-2cm] (emp) {EMPLOYEE};
    \node [entity, below of=isa, xshift=2cm] (std) {STUDENT};
    
    \node [attribute, left of=person] (pid) {person\_id};
    \node [attribute, above of=person] (name) {name};
    \node [attribute, right of=person] (addr) {address};
    
    \node [attribute, left of=emp] (eid) {emp\_id};
    \node [attribute, below of=emp] (sal) {salary};
    
    \node [attribute, right of=std] (sid) {std\_id};
    \node [attribute, below of=std] (course) {course};

    \draw [gtu line] (person) -- (isa);
    \draw [gtu line] (isa) -- (emp);
    \draw [gtu line] (isa) -- (std);
    
    \draw [gtu line] (person) -- (pid);
    \draw [gtu line] (person) -- (name);
    \draw [gtu line] (person) -- (addr);
    \draw [gtu line] (emp) -- (eid);
    \draw [gtu line] (emp) -- (sal);
    \draw [gtu line] (std) -- (sid);
    \draw [gtu line] (std) -- (course);
\end{tikzpicture}
\captionof{figure}{Generalization ઉદાહરણ}
\end{center}

\begin{table}[H]
    \centering
    \caption{Generalization લાક્ષણિકતાઓ}
    \begin{tabulary}{\linewidth}{LC}
        \toprule
        \textbf{પાસું} & \textbf{વર્ણન} \\
        \midrule
        \textbf{બોટમ-અપ પ્રક્રિયા} & સમાન એન્ટિટીઓને સુપરક્લાસમાં જોડે છે \\
        \textbf{વારસાગતતા} & સબક્લાસ સુપરક્લાસ ગુણધર્મો વારસે મેળવે છે \\
        \textbf{વિશેષીકરણ} & Generalization ની વિપરીત પ્રક્રિયા \\
        \textbf{ઓવરલેપ મર્યાદાઓ} & અલગ અથવા ઓવરલેપિંગ સબક્લાસ \\
        \bottomrule
    \end{tabulary}
\end{table}

\textbf{મુખ્ય લક્ષણો:}
\begin{itemize}
    \item \textbf{ગુણધર્મ વારસાગતતા}: સામાન્ય ગુણધર્મો સુપરક્લાસમાં ખસેડવામાં આવે છે
    \item \textbf{સંબંધ વારસાગતતા}: સંબંધો પણ વારસામાં મળે છે
    \item \textbf{મર્યાદા પ્રકારો}: કુલ/આંશિક, અલગ/ઓવરલેપિંગ
    \item \textbf{ISA સંબંધ}: "is-a" કનેક્શનને રજૂ કરે છે
\end{itemize}

\begin{mnemonicbox}
    \textbf{મેમરી ટ્રીક:} "સામાન્યીકરણ સમાન એન્ટિટીઓને જૂથ બનાવે"
\end{mnemonicbox}
\end{solutionbox}

\questionmarks{2(અ) અથવા}{3}{ER ડાયાગ્રામમાં મેપિંગ કાર્ડિનાલિટી સમજાવો}
\begin{solutionbox}
\begin{table}[H]
    \centering
    \caption{કાર્ડિનાલિટી પ્રકારો}
    \begin{tabulary}{\linewidth}{LCL}
        \toprule
        \textbf{પ્રકાર} & \textbf{વર્ણન} & \textbf{ઉદાહરણ} \\
        \midrule
        \textbf{એક-થી-એક (1:1)} & એક એન્ટિટી અન્ય એક સાથે સંબંધિત & વ્યક્તિ-પાસપોર્ટ \\
        \textbf{એક-થી-ઘણા (1:M)} & એક એન્ટિટી ઘણા અન્ય સાથે સંબંધિત & વિભાગ-કર્મચારી \\
        \textbf{ઘણા-થી-એક (M:1)} & ઘણી એન્ટિટી એક સાથે સંબંધિત & કર્મચારી-વિભાગ \\
        \textbf{ઘણા-થી-ઘણા (M:N)} & ઘણી એન્ટિટી ઘણા સાથે સંબંધિત & વિદ્યાર્થી-કોર્સ \\
        \bottomrule
    \end{tabulary}
\end{table}

\textbf{મુખ્ય સંકેતો:}
\begin{itemize}
    \item \textbf{સંબંધ મર્યાદાઓ}: એન્ટિટી કેવી રીતે સંબંધિત થઈ શકે છે તે વ્યાખ્યાયિત કરે છે
    \item \textbf{બિઝનેસ નિયમો}: વાસ્તવિક વિશ્વ સંબંધ મર્યાદાઓને પ્રતિબિંબિત કરે છે
\end{itemize}

\begin{mnemonicbox}
    \textbf{મેમરી ટ્રીક:} "એક કે ઘણા મેપિંગ મહત્વપૂર્ણ"
\end{mnemonicbox}
\end{solutionbox}

\questionmarks{2(બ) અથવા}{4}{E-R ડાયાગ્રામમાં Aggregation સમજાવો}
\begin{solutionbox}
\begin{center}
\begin{tikzpicture}[gtu block, node distance=2.5cm]
    \node [entity] (emp) {કર્મચારી};
    \node [relationship, right of=emp] (works) {કામ\_કરે};
    \node [entity, right of=works] (proj) {પ્રોજેક્ટ};
    
    % Aggregation box
    \draw [dashed, thick] ($(emp.north west)+(-0.5,0.5)$) rectangle ($(proj.south east)+(0.5,-0.5)$);
    
    \node [relationship, below of=works, yshift=-1cm] (manages) {વ્યવસ્થાપન};
    \node [entity, below of=manages] (mgr) {મેનેજર};
    
    \draw [gtu line] (emp) -- (works);
    \draw [gtu line] (works) -- (proj);
    \draw [gtu line] (manages) -- (mgr);
    \draw [gtu line] (works) -- (manages);
\end{tikzpicture}
\captionof{figure}{Aggregation ઉદાહરણ}
\end{center}

\textbf{મુખ્ય લક્ષણો:}
\begin{itemize}
    \item \textbf{સંબંધ એન્ટિટી તરીકે}: સંબંધ સેટને એન્ટિટી તરીકે ગણે છે
    \item \textbf{ઉચ્ચ સ્તરના સંબંધો}: સંબંધો વચ્ચે સંબંધોની મંજૂરી આપે છે
    \item \textbf{જટિલ મોડેલિંગ}: અદ્યતન બિઝનેસ દૃશ્યોને હેન્ડલ કરે છે
    \item \textbf{અમૂર્ત પદ્ધતિ}: જટિલ સંબંધોને સરળ બનાવે છે
\end{itemize}

\begin{table}[H]
    \centering
    \caption{Aggregation લાભો}
    \begin{tabulary}{\linewidth}{LC}
        \toprule
        \textbf{લાભ} & \textbf{વર્ણન} \\
        \midrule
        \textbf{મોડેલિંગ લવચીકતા} & જટિલ સંબંધોને હેન્ડલ કરે છે \\
        \textbf{અર્થપૂર્ણ સ્પષ્ટતા} & બિઝનેસ નિયમોની સ્પષ્ટ રજૂઆત \\
        \textbf{ડિઝાઇન સરળતા} & મોડેલ જટિલતા ઘટાડે છે \\
        \bottomrule
    \end{tabulary}
\end{table}

\begin{mnemonicbox}
    \textbf{મેમરી ટ્રીક:} "એકીકરણ અદ્યતન સંગઠનોને અમૂર્ત બનાવે"
\end{mnemonicbox}
\end{solutionbox}

\questionmarks{2(ક) અથવા}{7}{Enhanced ER મોડેલનો ઉપયોગ કરીને લાઇબ્રેરી મેનેજમેન્ટ સિસ્ટમનો ER ડાયાગ્રામ દોરો}
\begin{solutionbox}
\begin{center}
\begin{tikzpicture}[gtu block, node distance=2cm]
    \node [entity] (person) {PERSON};
    \node [relationship, below of=person] (isa) {IS-A};
    \node [entity, below of=isa, xshift=-3cm] (member) {MEMBER};
    \node [entity, below of=isa, xshift=3cm] (lib) {LIBRARIAN};
    
    \node [relationship, below of=member] (makes) {Makes};
    \node [relationship, below of=lib] (proc) {Processes};
    \node [entity, below of=makes, xshift=3cm] (trans) {TRANSACTION};
    
    \node [relationship, below of=trans] (invol) {Involved\_In};
    \node [entity, below of=invol] (book) {BOOK};
    \node [relationship, right of=book, xshift=2cm] (belongs) {Belongs\_To};
    \node [entity, right of=belongs, xshift=2cm] (cat) {CATEGORY};
    
    % Connections
    \draw [gtu line] (person) -- (isa);
    \draw [gtu line] (isa) -- (member);
    \draw [gtu line] (isa) -- (lib);
    \draw [gtu line] (member) -- (makes);
    \draw [gtu line] (lib) -- (proc);
    \draw [gtu arrow] (makes) -- (trans);
    \draw [gtu arrow] (proc) -- (trans);
    \draw [gtu arrow] (invol) -- (trans);
    \draw [gtu arrow] (invol) -- (book);
    \draw [gtu arrow] (belongs) -- (book);
    \draw [gtu line] (belongs) -- (cat);
    
    % Sample Attributes (simplified)
    \node [attribute, left of=person] {person\_id};
    \node [attribute, left of=member] {member\_id};
    \node [attribute, right of=lib] {emp\_id};
    \node [attribute, left of=book] {isbn};
\end{tikzpicture}
\captionof{figure}{લાઇબ્રેરી મેનેજમેન્ટ સિસ્ટમ}
\end{center}

\textbf{વપરાયેલ Enhanced ER લક્ષણો:}
\begin{itemize}
    \item \textbf{સામાન્યીકરણ}: મેમ્બર અને લાઇબ્રેરિયન સબક્લાસ સાથે વ્યક્તિ સુપરક્લાસ
    \item \textbf{વિશેષીકરણ}: વિવિધ વ્યક્તિ પ્રકારો માટે વિવિધ ગુણધર્મો
    \item \textbf{એકીકરણ}: બહુવિધ એન્ટિટી સાથે Transaction સંબંધ
    \item \textbf{બહુવિધ વારસાગતતા}: જટિલ સંબંધ હેન્ડલિંગ
\end{itemize}

\begin{mnemonicbox}
    \textbf{મેમરી ટ્રીક:} "લાઇબ્રેરી સાહિત્યને તાર્કિક રીતે જોડે"
\end{mnemonicbox}
\end{solutionbox}

\questionmarks{3(અ)}{3}{SQL ડેટા પ્રકાર સમજાવો}
\begin{solutionbox}
\begin{table}[H]
    \centering
    \caption{સામાન્ય SQL ડેટા પ્રકારો}
    \begin{tabulary}{\linewidth}{LCL}
        \toprule
        \textbf{કેટેગરી} & \textbf{ડેટા પ્રકાર} & \textbf{વર્ણન} \\
        \midrule
        \textbf{સંખ્યાત્મક} & INT, DECIMAL, FLOAT & સંખ્યાઓ સંગ્રહિત કરે \\
        \textbf{અક્ષર} & CHAR, VARCHAR, TEXT & ટેક્સ્ટ સંગ્રહિત કરે \\
        \textbf{તારીખ/સમય} & DATE, TIME, DATETIME & સમયગત ડેટા સંગ્રહિત કરે \\
        \textbf{બુલિયન} & BOOLEAN & સાચું/ખોટું સંગ્રહિત કરે \\
        \bottomrule
    \end{tabulary}
\end{table}

\textbf{મુખ્ય મુદ્દાઓ:}
\begin{itemize}
    \item \textbf{ડેટા અખંડિતતા}: યોગ્ય ડેટા સ્ટોરેજ સુનિશ્ચિત કરે છે
    \item \textbf{સ્ટોરેજ ઑપ્ટિમાઇઝેશન}: યોગ્ય કદ ફાળવણી
    \item \textbf{માન્યતા}: આપોઆપ ડેટા પ્રકાર તપાસ
\end{itemize}

\begin{mnemonicbox}
    \textbf{મેમરી ટ્રીક:} "ડેટા પ્રકારો સ્ટોરેજ વ્યાખ્યાયિત કરે"
\end{mnemonicbox}
\end{solutionbox}

\questionmarks{3(બ)}{4}{DROP અને TRUNCATE COMMAND સરખામણી કરો}
\begin{solutionbox}
\begin{table}[H]
    \centering
    \caption{DROP vs TRUNCATE સરખામણી}
    \begin{tabulary}{\linewidth}{LCC}
        \toprule
        \textbf{લક્ષણ} & \textbf{DROP} & \textbf{TRUNCATE} \\
        \midrule
        \textbf{ઑપરેશન} & ટેબલ સ્ટ્રક્ચર દૂર કરે & માત્ર બધો ડેટા દૂર કરે \\
        \textbf{રોલબેક} & રોલબેક કરી શકાતું નથી & રોલબેક કરી શકાય (ટ્રાન્ઝેક્શનમાં) \\
        \textbf{ઝડપ} & ધીમું & ઝડપી \\
        \textbf{ટ્રિગર્સ} & ટ્રિગર્સ ચલાવે છે & ટ્રિગર્સ ચલાવતું નથી \\
        \textbf{વ્હેર ક્લોઝ} & લાગુ નથી & સપોર્ટ કરતું નથી \\
        \textbf{ઓટો-ઇન્ક્રિમેન્ટ} & રીસેટ થાય છે & પ્રારંભિક વેલ્યુ પર રીસેટ થાય છે \\
        \bottomrule
    \end{tabulary}
\end{table}

\textbf{કોડ ઉદાહરણો:}
\begin{lstlisting}[language=SQL]
-- DROP command
DROP TABLE student;

-- TRUNCATE command  
TRUNCATE TABLE student;
\end{lstlisting}

\textbf{મુખ્ય તફાવતો:}
\begin{itemize}
    \item \textbf{સ્ટ્રક્ચર પ્રભાવ}: DROP બધું દૂર કરે છે, TRUNCATE સ્ટ્રક્ચર રાખે છે
    \item \textbf{પ્રદર્શન}: TRUNCATE મોટા ટેબલો માટે ઝડપી છે
\end{itemize}

\begin{mnemonicbox}
    \textbf{મેમરી ટ્રીક:} "DROP નાશ કરે, TRUNCATE કાપે"
\end{mnemonicbox}
\end{solutionbox}

\questionmarks{3(ક)}{7}{નીચેના Relational Schema અને નીચેના પ્રશ્નો માટે Relational Algebra Expression આપો\\ \textbf{વિદ્યાર્થીઓ (નામ, SPI, DOB, નોંધણી નંબર)}}
\begin{solutionbox}
\textbf{રિલેશનલ આલ્જિબ્રા એક્સપ્રેશન્સ:}

\textbf{i) એવા તમામ વિદ્યાર્થીઓની યાદી બનાવો કે જેમનું SPI 6.0 કરતાં ઓછું છે:}
\[ \sigma_{\text{SPI} < 6.0}(\text{વિદ્યાર્થીઓ}) \]

\textbf{ii) વિદ્યાર્થીનું નામ જેની નોંધણી નંબર 006 ધરાવે છે:}
\[ \pi_{\text{નામ}}(\sigma_{\text{નોંધણી\_નંબર LIKE } '\%006\%'}(\text{વિદ્યાર્થીઓ})) \]

\textbf{iii) સમાન DOB ધરાવતા તમામ વિદ્યાર્થીઓની યાદી બનાવો:}
\[ \text{વિદ્યાર્થીઓ} \bowtie_{\substack{\text{વિદ્યાર્થીઓ.DOB} = S2.\text{DOB} \\ \land \text{વિદ્યાર્થીઓ.નોંધણી\_નંબર} \neq S2.\text{નોંધણી\_નંબર}}} (\rho_{S2}(\text{વિદ્યાર્થીઓ})) \]

\textbf{iv) સમાન અક્ષરથી શરૂ થતા વિદ્યાર્થીઓનું નામ દર્શાવો:}
\[ \pi_{\text{નામ}}(\text{વિદ્યાર્થીઓ} \bowtie_{\substack{\text{SUBSTR(વિદ્યાર્થીઓ.નામ,1,1)} = \text{SUBSTR}(S2.\text{નામ,1,1)} \\ \land \text{વિદ્યાર્થીઓ.નોંધણી\_નંબર} \neq S2.\text{નોંધણી\_નંબર}}} (\rho_{S2}(\text{વિદ્યાર્થીઓ}))) \]

\begin{table}[H]
    \centering
    \caption{વપરાયેલ રિલેશનલ આલ્જિબ્રા ઓપરેટર્સ}
    \begin{tabulary}{\linewidth}{LCL}
        \toprule
        \textbf{ઓપરેટર} & \textbf{પ્રતીક} & \textbf{હેતુ} \\
        \midrule
        \textbf{પસંદગી} & $\sigma$ & શરત આધારિત પંક્તિઓ ફિલ્ટર કરે \\
        \textbf{પ્રોજેક્શન} & $\pi$ & ચોક્કસ કોલમ પસંદ કરે \\
        \textbf{જોઇન} & $\bowtie$ & સંબંધિત ટ્યુપલ્સ સંયોજિત કરે \\
        \textbf{નામ બદલવું} & $\rho$ & રિલેશન્સ/એટ્રિબ્યુટ્સનું નામ બદલે \\
        \bottomrule
    \end{tabulary}
\end{table}

\begin{mnemonicbox}
    \textbf{મેમરી ટ્રીક:} "પસંદ કરો પ્રોજેક્ટ કરો જોડો નામ બદલો"
\end{mnemonicbox}
\end{solutionbox}

\questionmarks{3(અ) અથવા}{3}{ઉદાહરણ સાથે Grant અને Revoke આદેશનો ઉપયોગ સમજાવો}
\begin{solutionbox}
\textbf{કોડ ઉદાહરણો:}
\begin{lstlisting}[language=SQL]
-- GRANT command
GRANT SELECT, INSERT ON student TO user1;
GRANT ALL PRIVILEGES ON database1 TO user2;

-- REVOKE command  
REVOKE INSERT ON student FROM user1;
REVOKE ALL PRIVILEGES ON database1 FROM user2;
\end{lstlisting}

\textbf{મુખ્ય લક્ષણો:}
\begin{itemize}
    \item \textbf{એક્સેસ કંટ્રોલ}: વપરાશકર્તા અનુમતિઓ સંચાલિત કરે છે
    \item \textbf{સુરક્ષા}: અનધિકૃત એક્સેસ અટકાવે છે
    \item \textbf{ગ્રેન્યુલર કંટ્રોલ}: ચોક્કસ વિશેષાધિકાર અસાઇનમેન્ટ
\end{itemize}

\begin{table}[H]
    \centering
    \caption{સામાન્ય વિશેષાધિકારો}
    \begin{tabulary}{\linewidth}{LC}
        \toprule
        \textbf{વિશેષાધિકાર} & \textbf{વર્ણન} \\
        \midrule
        \textbf{SELECT} & ડેટા વાંચે \\
        \textbf{INSERT} & નવા રેકોર્ડ ઉમેરે \\
        \textbf{UPDATE} & હાલનો ડેટા બદલે \\
        \textbf{DELETE} & રેકોર્ડ દૂર કરે \\
        \textbf{ALL} & સંપૂર્ણ એક્સેસ \\
        \bottomrule
    \end{tabulary}
\end{table}

\begin{mnemonicbox}
    \textbf{મેમરી ટ્રીક:} "Grant આપે, Revoke દૂર કરે"
\end{mnemonicbox}
\end{solutionbox}

\questionmarks{3(બ) અથવા}{4}{ઉદાહરણ સાથે DML આદેશોનું વર્ણન કરો}
\begin{solutionbox}
\begin{table}[H]
    \centering
    \caption{DML આદેશો}
    \begin{tabulary}{\linewidth}{LCL}
        \toprule
        \textbf{આદેશ} & \textbf{હેતુ} & \textbf{ઉદાહરણ} \\
        \midrule
        \textbf{INSERT} & નવા રેકોર્ડ ઉમેરે & \code{INSERT INTO student...} \\
        \textbf{UPDATE} & હાલનો ડેટા બદલે & \code{UPDATE student SET...} \\
        \textbf{DELETE} & રેકોર્ડ દૂર કરે & \code{DELETE FROM student...} \\
        \textbf{SELECT} & ડેટા પુનઃપ્રાપ્ત કરે & \code{SELECT * FROM student...} \\
        \bottomrule
    \end{tabulary}
\end{table}

\textbf{કોડ ઉદાહરણો:}
\begin{lstlisting}[language=SQL]
-- INSERT command
INSERT INTO Students (name, spi, dob) 
VALUES ('Alice', 8.5, '2000-05-15');

-- UPDATE command
UPDATE Students SET spi = 9.0 
WHERE name = 'Alice';

-- DELETE command
DELETE FROM Students 
WHERE spi < 6.0;

-- SELECT command
SELECT name, spi FROM Students 
WHERE spi > 8.0;
\end{lstlisting}

\textbf{મુખ્ય લક્ષણો:}
\begin{itemize}
    \item \textbf{ડેટા મેનિપ્યુલેશન}: મુખ્ય ડેટાબેઝ ઓપરેશન્સ
    \item \textbf{ટ્રાન્ઝેક્શન સપોર્ટ}: રોલબેક કરી શકાય છે
    \item \textbf{શરતી ઓપરેશન્સ}: WHERE ક્લોઝ સપોર્ટ
\end{itemize}

\begin{mnemonicbox}
    \textbf{મેમરી ટ્રીક:} "Insert Update Delete Select"
\end{mnemonicbox}
\end{solutionbox}

\questionmarks{3(ક) અથવા}{7}{DBMS ના તમામ કન્વર્ઝન ફંક્શનની યાદી બનાવો અને તેમાંથી કોઈપણ ત્રણને વિગતવાર સમજાવો}
\begin{solutionbox}
\begin{table}[H]
    \centering
    \caption{કન્વર્ઝન ફંક્શન્સ}
    \begin{tabulary}{\linewidth}{LCL}
        \toprule
        \textbf{ફંક્શન} & \textbf{હેતુ} & \textbf{ઉદાહરણ} \\
        \midrule
        \textbf{TO\_CHAR} & કેરેક્ટરમાં કન્વર્ટ કરે & \code{TO\_CHAR(sysdate)} \\
        \textbf{TO\_DATE} & તારીખમાં કન્વર્ટ કરે & \code{TO\_DATE('15-05-2025')} \\
        \textbf{TO\_NUMBER} & નંબરમાં કન્વર્ટ કરે & \code{TO\_NUMBER('123.45')} \\
        \textbf{CAST} & સામાન્ય કન્વર્ઝન & \code{CAST('123' AS INT)} \\
        \textbf{CONVERT} & ડેટા પ્રકાર કન્વર્ઝન & \code{CONVERT(varchar, 123)} \\
        \bottomrule
    \end{tabulary}
\end{table}

\textbf{ત્રણ ફંક્શન્સની વિગતવાર સમજૂતી:}

\textbf{1. TO\_CHAR ફંક્શન:}
\begin{itemize}
    \item \textbf{હેતુ}: તારીખો અને નંબરોને કેરેક્ટર સ્ટ્રિંગમાં કન્વર્ટ કરે છે
    \item \textbf{સિન્ટેક્સ}: \code{TO\_CHAR(value, format)}
    \item \textbf{ઉપયોગ}: તારીખ ફોર્મેટિંગ, ચોક્કસ પેટર્ન સાથે નંબર ફોર્મેટિંગ
\end{itemize}

\textbf{2. TO\_DATE ફંક્શન:}
\begin{itemize}
    \item \textbf{હેતુ}: કેરેક્ટર સ્ટ્રિંગને તારીખ વેલ્યુમાં કન્વર્ટ કરે છે
    \item \textbf{સિન્ટેક્સ}: \code{TO\_DATE(string, format)}
    \item \textbf{ઉપયોગ}: ચોક્કસ ફોર્મેટ સાથે સ્ટ્રિંગ થી તારીખ કન્વર્ઝન
\end{itemize}

\textbf{3. TO\_NUMBER ફંક્શન:}
\begin{itemize}
    \item \textbf{હેતુ}: કેરેક્ટર સ્ટ્રિંગને સંખ્યાત્મક વેલ્યુમાં કન્વર્ટ કરે છે
    \item \textbf{સિન્ટેક્સ}: \code{TO\_NUMBER(string, format)}
    \item \textbf{ઉપયોગ}: ગણતરીઓ માટે સ્ટ્રિંગ થી નંબર કન્વર્ઝન
\end{itemize}

\textbf{મુખ્ય લાભો:}
\begin{itemize}
    \item \textbf{ડેટા પ્રકાર લવચીકતા}: પ્રકારો વચ્ચે સહજ કન્વર્ઝન
    \item \textbf{ફોર્મેટ કંટ્રોલ}: ચોક્કસ ફોર્મેટિંગ વિકલ્પો
    \item \textbf{એરર હેન્ડલિંગ}: કન્વર્ઝન દરમિયાન માન્યતા
\end{itemize}

\begin{mnemonicbox}
    \textbf{મેમરી ટ્રીક:} "કેરેક્ટર્સ તારીખો નંબર્સ કન્વર્ટ કરો"
\end{mnemonicbox}
\end{solutionbox}

\questionmarks{4(અ)}{3}{ટૂંકી નોંધ લખો: Domain Integrity Constraint}
\begin{solutionbox}
\textbf{ડોમેન ઇન્ટીગ્રિટી કન્સ્ટ્રેઇન્ટ્સ} સુનિશ્ચિત કરે છે કે ડેટા મૂલ્યો ચોક્કસ એટ્રિબ્યુટ્સ માટે સ્વીકાર્ય રેન્જ અને ફોર્મેટમાં છે.

\begin{table}[H]
    \centering
    \caption{ડોમેન કન્સ્ટ્રેઇન્ટ પ્રકારો}
    \begin{tabulary}{\linewidth}{LCL}
        \toprule
        \textbf{કન્સ્ટ્રેઇન્ટ} & \textbf{હેતુ} & \textbf{ઉદાહરણ} \\
        \midrule
        \textbf{CHECK} & વેલ્યુ રેન્જ માન્યતા & \code{CHECK (age >= 0 AND age <= 100)} \\
        \textbf{NOT NULL} & null વેલ્યુ અટકાવે & \code{name VARCHAR(50) NOT NULL} \\
        \textbf{DEFAULT} & ડિફોલ્ટ વેલ્યુ સેટ કરે & \code{status VARCHAR(10) DEFAULT 'Active'} \\
        \bottomrule
    \end{tabulary}
\end{table}

\textbf{મુખ્ય લક્ષણો:}
\begin{itemize}
    \item \textbf{ડેટા માન્યતા}: એન્ટ્રી સમયે ડેટા ગુણવત્તા સુનિશ્ચિત કરે છે
    \item \textbf{બિઝનેસ નિયમો}: ડોમેન-વિશિષ્ટ નિયમો લાગુ કરે છે
    \item \textbf{આપોઆપ તપાસ}: DML ઓપરેશન્સ દરમિયાન માન્યતા થાય છે
\end{itemize}

\begin{mnemonicbox}
    \textbf{મેમરી ટ્રીક:} "ડોમેન ડેટા સીમાઓ વ્યાખ્યાયિત કરે"
\end{mnemonicbox}
\end{solutionbox}

\questionmarks{4(બ)}{4}{DBMS માં તમામ JOIN ની યાદી બનાવો અને કોઈપણ બે સમજાવો}
\begin{solutionbox}
\begin{table}[H]
    \centering
    \caption{JOIN ના પ્રકારો}
    \begin{tabulary}{\linewidth}{LC}
        \toprule
        \textbf{JOIN પ્રકાર} & \textbf{વર્ણન} \\
        \midrule
        \textbf{INNER JOIN} & બંને ટેબલમાંથી મેચ થતા રેકોર્ડ્સ આપે છે \\
        \textbf{LEFT JOIN} & ડાબા ટેબલમાંથી તમામ રેકોર્ડ્સ આપે છે \\
        \textbf{RIGHT JOIN} & જમણા ટેબલમાંથી તમામ રેકોર્ડ્સ આપે છે \\
        \textbf{FULL OUTER JOIN} & બંને ટેબલમાંથી તમામ રેકોર્ડ્સ આપે છે \\
        \textbf{CROSS JOIN} & બંને ટેબલનું કાર્ટેશિયન ગુણાકાર \\
        \textbf{SELF JOIN} & ટેબલ તેની પોતાની સાથે જોડાય છે \\
        \bottomrule
    \end{tabulary}
\end{table}

\textbf{વિગતવાર સમજૂતી:}

\textbf{1. INNER JOIN:}
\begin{lstlisting}[language=SQL]
SELECT s.name, c.course_name
FROM students s
INNER JOIN courses c ON s.course_id = c.course_id;
\end{lstlisting}
\begin{itemize}
    \item બંને ટેબલમાંથી માત્ર મેચ થતા રેકોર્ડ્સ આપે છે
    \item સૌથી સામાન્ય રીતે વપરાતો જોઈન પ્રકાર
\end{itemize}

\textbf{2. LEFT JOIN:}
\begin{lstlisting}[language=SQL]
SELECT s.name, c.course_name
FROM students s
LEFT JOIN courses c ON s.course_id = c.course_id;
\end{lstlisting}
\begin{itemize}
    \item તમામ વિદ્યાર્થીઓ આપે છે, ભલે કોર્સ અસાઇન ન હોય
    \item મેચ ન થતા રેકોર્ડ્સ માટે NULL વેલ્યુ
\end{itemize}

\begin{mnemonicbox}
    \textbf{મેમરી ટ્રીક:} "ટેબલોને વિચારપૂર્વક જોડો"
\end{mnemonicbox}
\end{solutionbox}

\questionmarks{4(ક)}{7}{Functional Dependency નો કોન્સેપ્ટ વિગતવાર સમજાવો}
\begin{solutionbox}
\textbf{ફંક્શનલ ડિપેન્ડન્સી} ત્યારે થાય છે જ્યારે એક એટ્રિબ્યુટનું મૂલ્ય અન્ય એટ્રિબ્યુટના મૂલ્યને વિશિષ્ટ રીતે નક્કી કરે છે.

\textbf{નોટેશન:} $A \to B$ (A ફંક્શનલી B ને નક્કી કરે છે)

\begin{table}[H]
    \centering
    \caption{ફંક્શનલ ડિપેન્ડન્સીના પ્રકારો}
    \begin{tabulary}{\linewidth}{LCL}
        \toprule
        \textbf{પ્રકાર} & \textbf{વ્યાખ્યા} & \textbf{ઉદાહરણ} \\
        \midrule
        \textbf{સંપૂર્ણ FD} & LHS માં તમામ એટ્રિબ્યુટ્સ જરૂરી & \{Student\_ID, Course\_ID\} $\to$ Grade \\
        \textbf{આંશિક FD} & કેટલાક LHS એટ્રિબ્યુટ્સ વધારાના & \{Student\_ID, Course\_ID\} $\to$ Student\_Name \\
        \textbf{ટ્રાન્ઝિટિવ FD} & અન્ય એટ્રિબ્યુટ દ્વારા પરોક્ષ નિર્ભરતા & Student\_ID $\to$ Dept\_ID $\to$ Dept\_Name \\
        \bottomrule
    \end{tabulary}
\end{table}

\begin{center}
\begin{tikzpicture}[gtu block, node distance=2.5cm]
    \node [attribute] (sid) {Student\_ID};
    \node [attribute, right of=sid, xshift=2cm] (sname) {Student\_Name};
    \node [attribute, below of=sname] (addr) {Address};
    \node [attribute, below of=sid] (cid) {Course\_ID};
    \node [attribute, right of=cid, xshift=2cm] (cname) {Course\_Name};

    \draw [gtu arrow] (sid) -- (sname);
    \draw [gtu arrow] (sid) |- (addr);
    \draw [gtu arrow] (cid) -- (cname);
\end{tikzpicture}
\captionof{figure}{ફંક્શનલ ડિપેન્ડન્સી ઉદાહરણ}
\end{center}

\textbf{મુખ્ય ગુણધર્મો:}
\begin{itemize}
    \item \textbf{રીફ્લેક્સિવિટી}: $A \to A$ (trivial dependency)
    \item \textbf{ઓગમેન્ટેશન}: જો $A \to B$, તો $AC \to BC$
    \item \textbf{ટ્રાન્ઝિટિવિટી}: જો $A \to B$ અને $B \to C$, તો $A \to C$
    \item \textbf{ડિકમ્પોઝિશન}: જો $A \to BC$, તો $A \to B$ અને $A \to C$
\end{itemize}

\begin{mnemonicbox}
    \textbf{મેમરી ટ્રીક:} "ફંક્શન્સ ડિપેન્ડન્સી નક્કી કરે છે"
\end{mnemonicbox}
\end{solutionbox}

\questionmarks{4(અ) અથવા}{3}{ટૂંકી નોંધ લખો: Referential integrity Constraints}
\begin{solutionbox}
\textbf{રેફરન્શિયલ ઇન્ટીગ્રિટી} સુનિશ્ચિત કરે છે કે એક ટેબલમાં ફોરેન કી મૂલ્યો સંદર્ભિત ટેબલમાં હાલના પ્રાથમિક કી મૂલ્યોને અનુરૂપ છે.

\begin{table}[H]
    \centering
    \caption{રેફરન્શિયલ ઇન્ટીગ્રિટી નિયમો}
    \begin{tabulary}{\linewidth}{LCL}
        \toprule
        \textbf{નિયમ} & \textbf{વર્ણન} & \textbf{ક્રિયા} \\
        \midrule
        \textbf{INSERT નિયમ} & ફોરેન કી પેરેન્ટમાં હોવી જોઈએ & અમાન્ય inserts નકારો \\
        \textbf{DELETE નિયમ} & પેરેન્ટ રેકોર્ડ ડિલીશન હેન્ડલ કરો & CASCADE, RESTRICT, SET NULL \\
        \textbf{UPDATE નિયમ} & પ્રાથમિક કી અપડેટ્સ હેન્ડલ કરો & CASCADE, RESTRICT \\
        \bottomrule
    \end{tabulary}
\end{table}

\textbf{કોડ ઉદાહરણ:}
\begin{lstlisting}[language=SQL]
ALTER TABLE Orders 
ADD CONSTRAINT FK_Customer 
FOREIGN KEY (customer_id) 
REFERENCES Customers(customer_id);
\end{lstlisting}

\textbf{મુખ્ય લક્ષણો:}
\begin{itemize}
    \item \textbf{ફોરેન કી કન્સ્ટ્રેઇન્ટ}: સંબંધિત ટેબલોને જોડે છે
    \item \textbf{ડેટા સુસંગતતા}: ઓર્ફન રેકોર્ડ્સ અટકાવે છે
    \item \textbf{સંબંધ જાળવણી}: ટેબલ સંબંધો સાચવે છે
\end{itemize}

\begin{mnemonicbox}
    \textbf{મેમરી ટ્રીક:} "રેફરન્સને રિલેટેડ રેકોર્ડ્સની જરૂર"
\end{mnemonicbox}
\end{solutionbox}

\questionmarks{4(બ) અથવા}{4}{રિલેશનલ આલ્જિબ્રાના union અને intersection ઓપરેશન્સ સમજાવો}
\begin{solutionbox}
\begin{table}[H]
    \centering
    \caption{સેટ ઓપરેશન્સ સરખામણી}
    \begin{tabulary}{\linewidth}{LCCL}
        \toprule
        \textbf{ઓપરેશન} & \textbf{પ્રતીક} & \textbf{વર્ણન} & \textbf{જરૂરિયાત} \\
        \midrule
        \textbf{UNION} & $\cup$ & બંને રિલેશન્સમાંથી તમામ ટ્યુપલ્સ જોડે છે & યુનિયન સુસંગત \\
        \textbf{INTERSECTION} & $\cap$ & બંને રિલેશન્સમાં સામાન્ય ટ્યુપલ્સ & યુનિયન સુસંગત \\
        \bottomrule
    \end{tabulary}
\end{table}

\textbf{Union ઓપરેશન:}
\begin{itemize}
    \item \textbf{સિન્ટેક્સ}: $R \cup S$
    \item \textbf{પરિણામ}: R અને S માંથી તમામ ટ્યુપલ્સ (ડુપ્લિકેટ્સ દૂર)
    \item \textbf{જરૂરિયાત}: એટ્રિબ્યુટ્સની સમાન સંખ્યા અને પ્રકારો
\end{itemize}

\textbf{Intersection ઓપરેશન:}
\begin{itemize}
    \item \textbf{સિન્ટેક્સ}: $R \cap S$
    \item \textbf{પરિણામ}: R અને S બંનેમાં અસ્તિત્વ ધરાવતા ટ્યુપલ્સ
    \item \textbf{જરૂરિયાત}: યુનિયન સુસંગત રિલેશન્સ
\end{itemize}

\textbf{ઉદાહરણ:}
\begin{lstlisting}
Students_CS U Students_IT = બંને વિભાગના તમામ વિદ્યાર્થીઓ
Students_CS n Students_IT = બંને વિભાગના વિદ્યાર્થીઓ
\end{lstlisting}

\begin{mnemonicbox}
    \textbf{મેમરી ટ્રીક:} "યુનિયન જોડે, ઇન્ટરસેક્શન સામાન્ય ઓળખે"
\end{mnemonicbox}
\end{solutionbox}

\questionmarks{4(ક) અથવા}{7}{DBMS માં Normalization નો કોન્સેપ્ટ વિગતવાર સમજાવો}
\begin{solutionbox}
\textbf{નોર્મલાઇઝેશન} એ ડેટા રિડન્ડન્સી ઘટાડવા અને ડેટા અખંડિતતા સુધારવા માટે ડેટાબેઝ ટેબલોને વ્યવસ્થિત કરવાની પ્રક્રિયા છે.

\begin{table}[H]
    \centering
    \caption{નોર્મલ ફોર્મ્સ}
    \begin{tabulary}{\linewidth}{LCC}
        \toprule
        \textbf{નોર્મલ ફોર્મ} & \textbf{જરૂરિયાતો} & \textbf{દૂર કરે છે} \\
        \midrule
        \textbf{1NF} & એટોમિક વેલ્યુ, રિપીટિંગ ગ્રુપ નહીં & મલ્ટીવેલ્યુડ એટ્રિબ્યુટ્સ \\
        \textbf{2NF} & 1NF + આંશિક નિર્ભરતા નહીં & આંશિક ફંક્શનલ ડિપેન્ડન્સી \\
        \textbf{3NF} & 2NF + ટ્રાન્ઝિટિવ નિર્ભરતા નહીં & ટ્રાન્ઝિટિવ ડિપેન્ડન્સી \\
        \textbf{BCNF} & 3NF + દરેક ડિટર્મિનન્ટ કેન્ડિડેટ કી & બાકીની વિસંગતતાઓ \\
        \bottomrule
    \end{tabulary}
\end{table}

\textbf{નોર્મલાઇઝેશન પ્રક્રિયા:}

\textbf{સ્ટેપ 1 - ફર્સ્ટ નોર્મલ ફોર્મ (1NF):}
\begin{itemize}
    \item રિપીટિંગ ગ્રુપ્સ દૂર કરો
    \item દરેક સેલ સિંગલ વેલ્યુ ધરાવે છે
    \item દરેક રેકોર્ડ વિશિષ્ટ છે
\end{itemize}

\textbf{સ્ટેપ 2 - સેકન્ડ નોર્મલ ફોર્મ (2NF):}
\begin{itemize}
    \item 1NF માં હોવું જોઈએ
    \item આંશિક નિર્ભરતા દૂર કરો
    \item નોન-કી એટ્રિબ્યુટ્સ સંપૂર્ણપણે પ્રાથમિક કી પર નિર્ભર
\end{itemize}

\textbf{સ્ટેપ 3 - થર્ડ નોર્મલ ફોર્મ (3NF):}
\begin{itemize}
    \item 2NF માં હોવું જોઈએ
    \item ટ્રાન્ઝિટિવ નિર્ભરતા દૂર કરો
    \item નોન-કી એટ્રિબ્યુટ્સ અન્ય નોન-કી એટ્રિબ્યુટ્સ પર નિર્ભર નથી
\end{itemize}

\textbf{નોર્મલાઇઝેશનના ફાયદા:}
\begin{itemize}
    \item \textbf{ઘટાડેલી રિડન્ડન્સી}: ડુપ્લિકેટ ડેટા દૂર કરે છે
    \item \textbf{ડેટા અખંડિતતા}: સુસંગતતા જાળવે છે
    \item \textbf{સ્ટોરેજ કાર્યક્ષમતા}: સ્ટોરેજ જગ્યા ઘટાડે છે
    \item \textbf{અપડેટ વિસંગતતાઓ}: અસંગત અપડેટ્સ અટકાવે છે
\end{itemize}

\begin{mnemonicbox}
    \textbf{મેમરી ટ્રીક:} "નોર્મલાઇઝેશન સુઘડ ટેબલો બનાવે"
\end{mnemonicbox}
\end{solutionbox}

\questionmarks{5(અ)}{3}{DBMS માં Normalization ની જરૂરિયાત વર્ણવો}
\begin{solutionbox}
\begin{table}[H]
    \centering
    \caption{નોર્મલાઇઝેશન દ્વારા ઉકેલાતી સમસ્યાઓ}
    \begin{tabulary}{\linewidth}{LCL}
        \toprule
        \textbf{સમસ્યા} & \textbf{વર્ણન} & \textbf{ઉકેલ} \\
        \midrule
        \textbf{ઇન્સર્શન વિસંગતતા} & સંપૂર્ણ માહિતી વિના ડેટા દાખલ કરી શકાતો નથી & અલગ ટેબલો \\
        \textbf{અપડેટ વિસંગતતા} & એક ફેરફાર માટે બહુવિધ અપડેટ્સ & રિડન્ડન્સી દૂર કરો \\
        \textbf{ડિલીશન વિસંગતતા} & ડિલીટ કરતી વખતે મહત્વપૂર્ણ ડેટા ગુમાવવો & નિર્ભરતા સાચવો \\
        \bottomrule
    \end{tabulary}
\end{table}

\textbf{મુખ્ય જરૂરિયાતો:}
\begin{itemize}
    \item \textbf{ડેટા સુસંગતતા}: ડેટાબેઝમાં એકસમાન ડેટા સુનિશ્ચિત કરે છે
    \item \textbf{સ્ટોરેજ ઑપ્ટિમાઇઝેશન}: બિનજરૂરી સ્ટોરેજ ઘટાડે છે
    \item \textbf{જાળવણી સરળતા}: ડેટાબેઝ અપડેટ્સ સરળ બનાવે છે
\end{itemize}

\textbf{ફાયદા:}
\begin{itemize}
    \item \textbf{સુધારેલી ડેટા ગુણવત્તા}: ભૂલો અને અસંગતતાઓ ઘટાડે છે
    \item \textbf{લવચીક ડિઝાઇન}: ફેરફાર અને વિસ્તરણ માટે સરળ
    \item \textbf{વધુ સારું પ્રદર્શન}: અપડેટ ઓપરેશન્સ માટે
\end{itemize}

\begin{mnemonicbox}
    \textbf{મેમરી ટ્રીક:} "નોર્મલાઇઝેશન વ્યવસ્થિત સંસ્થાની જરૂર"
\end{mnemonicbox}
\end{solutionbox}

\questionmarks{5(બ)}{4}{DBMS માં Transaction ના ગુણધર્મો સમજાવો}
\begin{solutionbox}
\begin{table}[H]
    \centering
    \caption{ACID ગુણધર્મો}
    \begin{tabulary}{\linewidth}{LCL}
        \toprule
        \textbf{ગુણધર્મ} & \textbf{વર્ણન} & \textbf{હેતુ} \\
        \midrule
        \textbf{Atomicity} & તમામ ઓપરેશન્સ સફળ થાય અથવા તમામ નિષ્ફળ & પૂર્ણતા સુનિશ્ચિત કરે \\
        \textbf{Consistency} & ડેટાબેઝ માન્ય સ્થિતિમાં રહે છે & અખંડિતતા જાળવે \\
        \textbf{Isolation} & સંગામી ટ્રાન્ઝેક્શન્સ દખલ કરતા નથી & સંઘર્ષ અટકાવે \\
        \textbf{Durability} & કમિટ થયેલા ફેરફારો કાયમી છે & સ્થાયીતા સુનિશ્ચિત કરે \\
        \bottomrule
    \end{tabulary}
\end{table}

\textbf{વિગતવાર સમજૂતી:}
\begin{itemize}
    \item \textbf{Atomicity}: ટ્રાન્ઝેક્શન અવિભાજ્ય એકમ છે. કાં તો બધું પૂર્ણ થાય અથવા કંઈ નહીં.
    \item \textbf{Consistency}: ડેટાબેઝ એક માન્ય સ્થિતિમાંથી બીજી સ્થિતિમાં જાય છે. તમામ અખંડિતતા મર્યાદાઓ જળવાય છે.
    \item \textbf{Isolation}: સંગામી ટ્રાન્ઝેક્શન્સ ક્રમિક રીતે ચાલતા હોય તેવું લાગે છે. મધ્યવર્તી સ્થિતિઓ અન્ય ટ્રાન્ઝેક્શન્સને દેખાતી નથી.
    \item \textbf{Durability}: એકવાર કમિટ થઈ ગયા પછી, ફેરફારો સિસ્ટમ નિષ્ફળતામાં પણ ટકી રહે છે. ડેટા કાયમી ધોરણે સંગ્રહિત થાય છે.
\end{itemize}

\begin{mnemonicbox}
    \textbf{મેમરી ટ્રીક:} "ACID સાચો ડેટાબેઝ સુનિશ્ચિત કરે"
\end{mnemonicbox}
\end{solutionbox}

\questionmarks{5(ક)}{7}{View Serializability વિગતવાર સમજાવો}
\begin{solutionbox}
\textbf{View Serializability} નક્કી કરે છે કે શું સમવર્તી શિડ્યુલ સીરીયલ શિડ્યુલ જેવું જ પરિણામ આપે છે કે કેમ, તે રીડ અને રાઈટ ઓપરેશન્સ તપાસીને.

\begin{table}[H]
    \centering
    \caption{વ્યુ ઇક્વિવેલન્સ શરતો}
    \begin{tabulary}{\linewidth}{LC}
        \toprule
        \textbf{શરત} & \textbf{વર્ણન} \\
        \midrule
        \textbf{પ્રારંભિક વાંચન} & સમાન ટ્રાન્ઝેક્શન્સ પ્રારંભિક મૂલ્યો વાંચે છે \\
        \textbf{અંતિમ લેખન} & સમાન ટ્રાન્ઝેક્શન્સ અંતિમ લેખન કરે છે \\
        \textbf{મધ્યવર્તી વાંચન} & સમાન લખનાર ટ્રાન્ઝેક્શન્સમાંથી મૂલ્યો વાંચો \\
        \bottomrule
    \end{tabulary}
\end{table}

\textbf{મુખ્ય ખ્યાલો:}

\textbf{વ્યુ ઇક્વિવેલન્ટ શિડ્યુલ્સ:}
બે શિડ્યુલ્સ વ્યુ ઇક્વિવેલન્ટ છે જો:
\begin{enumerate}
    \item દરેક ડેટા આઇટમ માટે, જો ટ્રાન્ઝેક્શન T એક શિડ્યુલમાં પ્રારંભિક મૂલ્ય વાંચે, તો તે બીજામાં પ્રારંભિક મૂલ્ય વાંચે
    \item દરેક રીડ ઓપરેશન માટે, જો T એક શિડ્યુલમાં T' દ્વારા લખાયેલું મૂલ્ય વાંચે, તો બીજામાં પણ તે જ હોવું જોઈએ
    \item દરેક ડેટા આઇટમ માટે, જો T એક શિડ્યુલમાં અંતિમ લેખન કરે, તો તે બીજામાં પણ અંતિમ લેખન કરે
\end{enumerate}

\textbf{View Serializability પરીક્ષણ:}
\begin{itemize}
    \item \textbf{પ્રેસિડન્સ ગ્રાફ}: ડાયરેક્ટેડ ગ્રાફ બનાવો
    \item \textbf{સાયકલ ડિટેક્શન}: ગ્રાફમાં સાયકલ તપાસો
    \item \textbf{સંઘર્ષ વિશ્લેષણ}: રીડ-રાઈટ સંઘર્ષો તપાસો
\end{itemize}

\textbf{ઉદાહરણ વિશ્લેષણ:}
\begin{lstlisting}
Schedule S1: R1(X) W1(X) R2(X) W2(X)
Schedule S2: R1(X) R2(X) W1(X) W2(X)
\end{lstlisting}

\textbf{Conflict Serializability સાથે સરખામણી:}
\begin{itemize}
    \item View serializability ઓછું પ્રતિબંધિત છે
    \item કેટલાક view serializable શિડ્યુલ્સ conflict serializable હોતા નથી
    \item પરીક્ષણ કરવું વધુ જટિલ
\end{itemize}

\begin{mnemonicbox}
    \textbf{મેમરી ટ્રીક:} "વ્યુ માન્ય શિડ્યુલ્સ ચકાસે"
\end{mnemonicbox}
\end{solutionbox}

\questionmarks{5(અ) અથવા}{3}{કોઈપણ ડેટાબેઝ પર 2NF કરો}
\begin{solutionbox}
\textbf{ઉદાહરણ: Student Course Database}

\textbf{મૂળ ટેબલ (2NF માં નથી):}
\begin{lstlisting}
Student_Course (Student_ID, Student_Name, Course_ID, Course_Name, Grade, Instructor)
Primary Key: {Student_ID, Course_ID}
\end{lstlisting}

\textbf{ફંક્શનલ ડિપેન્ડન્સી:}
\begin{itemize}
    \item Student\_ID $\to$ Student\_Name (આંશિક નિર્ભરતા)
    \item Course\_ID $\to$ Course\_Name, Instructor (આંશિક નિર્ભરતા)
    \item \{Student\_ID, Course\_ID\} $\to$ Grade
\end{itemize}

\textbf{2NF વિભાજન:}

\textbf{ટેબલ 1: Students}
\begin{lstlisting}
Students (Student_ID, Student_Name)
Primary Key: Student_ID
\end{lstlisting}

\textbf{ટેબલ 2: Courses}
\begin{lstlisting}
Courses (Course_ID, Course_Name, Instructor)  
Primary Key: Course_ID
\end{lstlisting}

\textbf{ટેબલ 3: Enrollments}
\begin{lstlisting}
Enrollments (Student_ID, Course_ID, Grade)
Primary Key: {Student_ID, Course_ID}
Foreign Keys: Student_ID -> Students, Course_ID -> Courses
\end{lstlisting}

\textbf{પરિણામ:} તમામ આંશિક નિર્ભરતા દૂર કરવામાં આવી, હવે 2NF માં છે.

\begin{mnemonicbox}
    \textbf{મેમરી ટ્રીક:} "બીજું નોર્મલ ફોર્મ નિર્ભરતા અલગ કરે"
\end{mnemonicbox}
\end{solutionbox}

\questionmarks{5(બ) અથવા}{4}{Transaction ના સ્ટેટ્સ સમજાવો}
\begin{solutionbox}
\begin{center}
\begin{tikzpicture}[gtu block, node distance=2.5cm]
    \node [state] (active) {સક્રિય};
    \node [state, right of=active, xshift=1cm] (part) {આંશિક\\કમિટેડ};
    \node [state, right of=part, xshift=1cm] (commit) {કમિટેડ};
    \node [state, below of=active] (failed) {નિષ્ફળ};
    \node [state, below of=part] (aborted) {એબોર્ટેડ};
    
    \draw [gtu arrow] (active) -- (part) node[midway, above] {\scriptsize અંતિમ સૂચના};
    \draw [gtu arrow] (part) -- (commit);
    \draw [gtu arrow] (active) -- (failed) node[midway, left] {\scriptsize ભૂલ};
    \draw [gtu arrow] (part) -- (failed);
    \draw [gtu arrow] (failed) -- (aborted) node[midway, below] {\scriptsize રોલબેક};
\end{tikzpicture}
\captionof{figure}{ટ્રાન્ઝેક્શન સ્ટેટ ડાયાગ્રામ}
\end{center}

\begin{table}[H]
    \centering
    \caption{ટ્રાન્ઝેક્શન સ્ટેટ્સ}
    \begin{tabulary}{\linewidth}{LCL}
        \toprule
        \textbf{સ્ટેટ} & \textbf{વર્ણન} & \textbf{ક્રિયાઓ} \\
        \midrule
        \textbf{સક્રિય} & ટ્રાન્ઝેક્શન ચાલી રહ્યું છે & રીડ/રાઈટ ઓપરેશન્સ \\
        \textbf{આંશિક કમિટેડ} & અંતિમ સૂચના અમલમાં & કમિટની રાહ જોઈ રહ્યું છે \\
        \textbf{કમિટેડ} & ટ્રાન્ઝેક્શન સફળતાપૂર્વક પૂર્ણ થયું & ફેરફારો કાયમી \\
        \textbf{નિષ્ફળ} & સામાન્ય રીતે આગળ વધી શકતું નથી & ભૂલ આવી \\
        \textbf{એબોર્ટેડ} & ટ્રાન્ઝેક્શન રોલબેક થયું & તમામ ફેરફારો રદ કર્યા \\
        \bottomrule
    \end{tabulary}
\end{table}

\textbf{મુખ્ય મુદ્દાઓ:}
\begin{itemize}
    \item \textbf{રિકવરી}: સિસ્ટમ નિષ્ફળ સ્ટેટ્સમાંથી પુનઃપ્રાપ્ત કરી શકે છે
    \item \textbf{ટકાઉપણું}: કમિટ થયેલા ફેરફારો કાયમી છે
    \item \textbf{એટોમિસિટી}: એબોર્ટેડ ટ્રાન્ઝેક્શન્સ કોઈ નિશાન છોડતા નથી
\end{itemize}

\begin{mnemonicbox}
    \textbf{મેમરી ટ્રીક:} "ટ્રાન્ઝેક્શન્સ સ્ટેટ્સમાંથી પસાર થાય"
\end{mnemonicbox}
\end{solutionbox}

\questionmarks{5(ક) અથવા}{7}{Conflict Serializability વિગતવાર સમજાવો}
\begin{solutionbox}
\textbf{Conflict Serializability} સુનિશ્ચિત કરે છે કે સમવર્તી શિડ્યુલ કોઈ સીરીયલ શિડ્યુલ સમાન છે, જે સંઘર્ષમય ઓપરેશન્સનું વિશ્લેષણ કરીને નક્કી થાય છે.

\begin{table}[H]
    \centering
    \caption{સંઘર્ષમય ઓપરેશન્સ}
    \begin{tabulary}{\linewidth}{LCL}
        \toprule
        \textbf{ઓપરેશન જોડી} & \textbf{સંઘર્ષ પ્રકાર} & \textbf{કારણ} \\
        \midrule
        \textbf{Read-Write} & RW Conflict & લખતા પહેલા વાંચો \\
        \textbf{Write-Read} & WR Conflict & વાંચતા પહેલા લખો \\
        \textbf{Write-Write} & WW Conflict & બહુવિધ લખવાનું \\
        \bottomrule
    \end{tabulary}
\end{table}

\textbf{Conflict Serializability પરીક્ષણ:}

\textbf{સ્ટેપ 1: સંઘર્ષો ઓળખો}
\begin{itemize}
    \item સમાન ડેટા આઇટમ પર ઓપરેશન્સની જોડી શોધો
    \item તપાસો કે ઓપરેશન્સ અલગ ટ્રાન્ઝેક્શન્સના છે કે નહીં
    \item નક્કી કરો કે ઓપરેશન્સ સંઘર્ષ કરે છે કે નહીં
\end{itemize}

\textbf{સ્ટેપ 2: પ્રેસિડન્સ ગ્રાફ બનાવો}
\begin{itemize}
    \item નોડ્સ ટ્રાન્ઝેક્શન્સ રજૂ કરે છે
    \item ડાયરેક્ટેડ એજિસ સંઘર્ષો રજૂ કરે છે
    \item Ti થી Tj સુધીની એજ જો Ti Tj સાથે સંઘર્ષ કરે છે
\end{itemize}

\textbf{સ્ટેપ 3: સાયકલ્સ માટે તપાસો}
\begin{itemize}
    \item જો ગ્રાફમાં કોઈ સાયકલ નથી $\to$ Conflict serializable
    \item જો ગ્રાફમાં સાયકલ છે $\to$ Conflict serializable નથી
\end{itemize}

\textbf{ઉદાહરણ વિશ્લેષણ:}
\begin{lstlisting}
Schedule: R1(A) W1(A) R2(A) W2(B) R1(B) W1(B)
Conflicts:
- W1(A) conflicts with R2(A) -> T1 before T2
- W2(B) conflicts with R1(B) -> T2 before T1
- W2(B) conflicts with W1(B) -> T2 before T1
\end{lstlisting}

\textbf{પ્રેસિડન્સ ગ્રાફ:}
\begin{center}
\begin{tikzpicture}[gtu block, node distance=2.5cm]
    \node [state] (t1) {T1};
    \node [state, right of=t1] (t2) {T2};
    
    \draw [gtu arrow] (t1) to[bend left] (t2);
    \draw [gtu arrow] (t2) to[bend left] (t1);
\end{tikzpicture}
\captionof{figure}{પ્રેસિડન્સ ગ્રાફ (સાયકલ)}
\end{center}

\textbf{પરિણામ:} સાયકલ ધરાવે છે, તેથી Conflict serializable નથી.

\textbf{View Serializability સાથે સરખામણી:}
\begin{itemize}
    \item Conflict serializability વધુ પ્રતિબંધિત છે
    \item તમામ conflict serializable શિડ્યુલ્સ view serializable છે
    \item View serializability કરતાં પરીક્ષણ કરવું સરળ છે
\end{itemize}

\begin{mnemonicbox}
    \textbf{મેમરી ટ્રીક:} "સંઘર્ષો સાયકલ બનાવે, કાળજીપૂર્વક તપાસો"
\end{mnemonicbox}
\end{solutionbox}

\end{document}
