\documentclass{article}

% content/resources/templates/preamble.tex
\usepackage[margin=0.6in]{geometry}
\author{Milav Dabgar}
\usepackage{amsmath,amssymb,amsthm}
\usepackage{booktabs}
\usepackage{multirow}
\usepackage{xcolor}
\usepackage{tcolorbox}
\tcbuselibrary{breakable,skins}
\usepackage[colorlinks=true,linkcolor=blue]{hyperref}
\usepackage{titlesec}
\usepackage{enumitem}
\usepackage{tikz}
\usepackage{pgfplots}
\usepackage{circuitikz}
\usepackage[version=4]{mhchem}
\usepackage{longtable}
\usepackage{array}
\usepackage{float}
\usepackage{caption}
\usepackage{listings}

\lstset{
  basicstyle=\small\ttfamily,
  breaklines=true,
  breakatwhitespace=false,
  postbreak=\mbox{\textcolor{red}{$\hookrightarrow$}\space},
  float=false,
  numbers=left,
  numberstyle=\tiny\color{gray},
  numbersep=10pt,
  xleftmargin=2em,
  keywordstyle=\color{blue},
  commentstyle=\color{green!60!black},
  stringstyle=\color{purple},
  backgroundcolor=\color{gray!5},
  showstringspaces=false,
  tabsize=2,
  captionpos=b,
  keepspaces=true,
  columns=flexible
}

\pgfplotsset{compat=1.18}
\usetikzlibrary{shapes,arrows,positioning,calc,patterns,decorations.pathmorphing,decorations.markings,arrows.meta}

% Color scheme
\definecolor{headcolor}{RGB}{0,102,204}
\definecolor{keycolor}{RGB}{220,20,60}
\definecolor{solutioncolor}{RGB}{34,139,34}
\definecolor{mnemoniccolor}{RGB}{148,0,211}
\definecolor{codecolor}{RGB}{0,0,100}

% Spacing
\setlength{\parskip}{3pt}
\setlist[itemize]{nosep}
\setlist[enumerate]{nosep}

% Title formatting
\titleformat{\section}{\Large\bfseries\color{headcolor}}{\thesection}{1em}{}
\titleformat{\subsection}{\large\bfseries\color{headcolor}}{\thesubsection}{1em}{}

% Pandoc tightlist compatibility
\providecommand{\tightlist}{%
  \setlength{\itemsep}{0pt}\setlength{\parskip}{0pt}}

% Pandoc longtable compatibility
\newcounter{none}
\def\thenone{}


% content/resources/templates/gujarati-boxes.tex
\usepackage{fontspec}
\usepackage{polyglossia}

% Set Gujarati as main language (document is primarily in Gujarati)
% Note: gloss-gujarati.ldf doesn't exist in polyglossia, but it will use hyphenation patterns
\setdefaultlanguage{gujarati}
\setotherlanguage{english}

% Configure Gujarati font properly
% Use Language=Default to prevent polyglossia from trying to add language-specific features
% that don't exist for Gujarati, which causes "empty feature" warnings
\newfontfamily\gujaratifont[Script=Gujarati,AutoFakeBold=2.5,AutoFakeSlant=0.3]{Noto Sans Gujarati}
\setmainfont[Script=Gujarati,AutoFakeBold=2.5,AutoFakeSlant=0.3]{Noto Sans Gujarati}
% Use Noto Sans Gujarati for monospace to support Gujarati in text
\setmonofont[Scale=0.9]{Noto Sans Gujarati}

% Configure English to use the same font
\newfontfamily\englishfont[Script=Gujarati,AutoFakeBold=2.5,AutoFakeSlant=0.3]{Noto Sans Gujarati}

% Translations for polyglossia
\gappto\captionsgujarati{
  \renewcommand{\tablename}{કોષ્ટક}
  \renewcommand{\figurename}{આકૃતિ}
}

% Helper for TikZ nodes to ensure Gujarati font
\newcommand{\gu}[1]{{\gujaratifont #1}}

% Custom environments
\newtcolorbox{solutionbox}{
    breakable,
    enhanced,
    colback=solutioncolor!5!white,
    colframe=solutioncolor!75!black,
    fonttitle=\bfseries,
    title=જવાબ
}

\newtcolorbox{solutionboxnobreak}{
 colback=solutioncolor!5!white,
 colframe=solutioncolor!75!black,
 fonttitle=\bfseries,
 title=જવાબ
}

\newtcolorbox{keyformula}{
 breakable,
 enhanced,
 colback=keycolor!5!white,
 colframe=keycolor!75!black,
 fonttitle=\bfseries,
 title=રાસાયણિક સમીકરણ/સૂત્ર
}

\newtcolorbox{mnemonicbox}{
 breakable,
 enhanced,
 colback=mnemoniccolor!5!white,
 colframe=mnemoniccolor!75!black,
 fonttitle=\bfseries,
 title=મેમરી ટ્રીક
}


% Custom commands for GTU solutions
% This file defines semantic commands for consistent formatting

% Question command with automatic formatting
\newcommand{\question}[2]{%
  \section*{Question #1}%
  \textbf{#2}%
}

% OR question variant
\newcommand{\questionor}[2]{%
  \section*{Question #1 OR}%
  \textbf{#2}%
}

% Proper table environment with caption
\newenvironment{answertable}[1]{%
  \begin{table}[htbp]
  \centering
  \caption{#1}
}{%
  \end{table}
}

% Proper figure environment for diagrams
\newenvironment{answerdiagram}[1]{%
  \begin{figure}[htbp]
  \centering
  \caption{#1}
}{%
  \end{figure}
}

% Semantic markup for key terms
\newcommand{\keyword}[1]{\textbf{#1}}
\newcommand{\code}[1]{\texttt{#1}}
\newcommand{\classname}[1]{\texttt{#1}}
\newcommand{\methodname}[1]{\texttt{#1}}

% Proper quotation marks
\newcommand{\mnemonic}[1]{``#1''}


\usetikzlibrary{calc,positioning,shapes,arrows,automata,fit,shapes.multipart,trees}
\tikzset{
    entity/.style={rectangle, draw, fill=white, align=center, minimum height=2em, font=\small, thick},
    relationship/.style={diamond, draw, fill=white, align=center, aspect=2, font=\small, thick},
    attribute/.style={ellipse, draw, fill=white, align=center, font=\small},
    multi attribute/.style={ellipse, draw, double, fill=white, align=center, font=\small},
    gtu line/.style={draw, thick},
    gtu arrow/.style={draw, -latex, thick}
}

\title{Database Management System (1333204) - Winter 2024 Solution}
\date{December 09, 2024}

\begin{document}
\maketitle

\questionmarks{1(અ)}{3}{ફિલ્ડ, રેકોર્ડ, મેટાડેટા ને વ્યાખ્યાયિત કરો.}
\begin{solutionbox}
\begin{itemize}
    \item \textbf{ફિલ્ડ}: એન્ટિટીના એક એટ્રિબ્યુટને રજૂ કરતો ડેટાનો એક એકમ (જેમ કે નામ, ઉંમર).
    \item \textbf{રેકોર્ડ}: એન્ટિટી વિશે ડેટા સંગ્રહિત કરતા સંબંધિત ફિલ્ડ્સનો સમૂહ.
    \item \textbf{મેટાડેટા}: ડેટા વિશેની માહિતી જે ડેટાબેઝ ઓબ્જેક્ટ્સની સંરચના, ગુણધર્મો અને સંબંધોનું વર્ણન કરે છે (ડેટા અબાઉટ ડેટા).
\end{itemize}

\begin{mnemonicbox}
    \textbf{મેમરી ટ્રીક:} "FRaMe" (ફિલ્ડ, રેકોર્ડ, મેટાડેટા)
\end{mnemonicbox}
\end{solutionbox}

\questionmarks{1(બ)}{4}{સ્ટ્રોંગ અને વીક entity set ને વ્યાખ્યાયિત કરો.}
\begin{solutionbox}
\begin{table}[H]
    \centering
    \caption{સ્ટ્રોંગ અને વીક એન્ટિટી સેટ}
    \begin{tabulary}{\linewidth}{LCL}
        \toprule
        \textbf{એન્ટિટી પ્રકાર} & \textbf{વર્ણન} & \textbf{ઓળખ} \\
        \midrule
        \textbf{સ્ટ્રોંગ એન્ટિટી} & સ્વતંત્ર રીતે અસ્તિત્વમાં છે & તેની પોતાની પ્રાઇમરી કી ધરાવે છે \\
        \textbf{વીક એન્ટિટી} & સ્ટ્રોંગ એન્ટિટી પર આધાર રાખે છે & પેરેન્ટ એન્ટિટી કી જરૂરી છે \\
        \bottomrule
    \end{tabulary}
\end{table}

\textbf{ઉદાહરણ}: સ્ટ્રોંગ - ગ્રાહક; વીક - બેંક એકાઉન્ટ.

\begin{mnemonicbox}
    \textbf{મેમરી ટ્રીક:} "SWing" (Strong is With own identity, weak is Not Getting own identity)
\end{mnemonicbox}
\end{solutionbox}

\questionmarks{1(ક)}{7}{ડેટા એબ્સ્ટ્રેક્શનના 3 સ્તરો સમજાવો.}
\begin{solutionbox}
\begin{table}[H]
    \centering
    \caption{ડેટા એબ્સ્ટ્રેક્શન સ્તરો}
    \begin{tabulary}{\linewidth}{LCL}
        \toprule
        \textbf{સ્તર} & \textbf{વર્ણન} & \textbf{વપરાશકર્તા} \\
        \midrule
        \textbf{ફિઝિકલ લેવલ} & ડેટા ભૌતિક રીતે કેવી રીતે સંગ્રહિત થાય છે તે વર્ણવે છે & સિસ્ટમ એડમિનિસ્ટ્રેટર્સ \\
        \textbf{કન્સેપ્ચુઅલ લેવલ} & કયો ડેટા સંગ્રહિત થયેલો છે અને સંબંધોનું વર્ણન કરે છે & ડેટાબેઝ ડિઝાઇનર્સ \\
        \textbf{વ્યૂ લેવલ} & વપરાશકર્તાઓ માટે પ્રસ્તુત ડેટાબેઝનો ભાગ વર્ણવે છે & એન્ડ યુઝર્સ \\
        \bottomrule
    \end{tabulary}
\end{table}

\begin{center}
\begin{tikzpicture}[gtu block, node distance=2cm]
    \node [gtu block, fill=blue!10] (view) {વ્યૂ લેવલ};
    \node [gtu block, below of=view] (conc) {કન્સેપ્ચુઅલ લેવલ};
    \node [gtu block, below of=conc] (phys) {ફિઝિકલ લેવલ};
    
    \node [left of=view, xshift=-3cm] (users) {એન્ડ યુઝર્સ};
    \node [left of=conc, xshift=-3cm] (dba) {ડેટાબેઝ ડિઝાઇનર્સ};
    \node [left of=phys, xshift=-3cm] (admin) {સિસ્ટમ એડમિનિસ્ટ્રેટર્સ};

    \draw [gtu arrow] (users) -- (view);
    \draw [gtu arrow] (dba) -- (conc);
    \draw [gtu arrow] (admin) -- (phys);
    
    \draw [gtu arrow] (view) -- (conc);
    \draw [gtu arrow] (conc) -- (phys);
\end{tikzpicture}
\captionof{figure}{ડેટા એબ્સ્ટ્રેક્શન સ્તરો}
\end{center}

\begin{mnemonicbox}
    \textbf{મેમરી ટ્રીક:} "PCV" (Physical, Conceptual, View - નીચેથી ઉપર)
\end{mnemonicbox}
\end{solutionbox}

\orquestionmarks{1(ક)}{7}{DBMS ના ફાયદાઓ અને ગેરફાયદાઓ સમજાવો.}
\begin{solutionbox}
\begin{table}[H]
    \centering
    \caption{DBMS ના ફાયદા અને ગેરફાયદા}
    \begin{tabulary}{\linewidth}{LCL}
        \toprule
        \textbf{ફાયદા} & \textbf{ગેરફાયદા} \\
        \midrule
        \textbf{ડેટા રિડન્ડન્સી કંટ્રોલ} & \textbf{ઊંચી કિંમત} (સોફ્ટવેર/હાર્ડવેર) \\
        \textbf{ડેટા કન્સિસ્ટન્સી} & \textbf{જટિલતા} (ડિઝાઇન/નિભાવ) \\
        \textbf{બહેતર ડેટા સિક્યુરિટી} & \textbf{પર્ફોર્મન્સ પર અસર} \\
        \textbf{ડેટા શેરિંગ} & \textbf{સંવેદનશીલતા} (નિષ્ફળતા માટે) \\
        \textbf{ડેટા ઇન્ડિપેન્ડન્સ} & \textbf{રિકવરી ચેલેન્જીસ} \\
        \textbf{પ્રમાણભૂત એક્સેસ} & \textbf{તાલીમ આવશ્યકતાઓ} \\
        \bottomrule
    \end{tabulary}
\end{table}

\begin{mnemonicbox}
    \textbf{મેમરી ટ્રીક:} "BASIC-DV" (Benefits: Access, Security, Independence, Consistency - Drawbacks: Vulnerability)
\end{mnemonicbox}
\end{solutionbox}

\questionmarks{2(અ)}{3}{રિલેશનલ એલ્જેબ્રા નું સિલેક્ટ ઓપરેશન સમજાવો.}
\begin{solutionbox}
\begin{table}[H]
    \centering
    \caption{સિલેક્ટ ઓપરેશન ($\sigma$)}
    \begin{tabulary}{\linewidth}{LCL}
        \toprule
        \textbf{ફીચર} & \textbf{વર્ણન} \\
        \midrule
        \textbf{સિન્ટેક્સ} & $\sigma_{condition}(Relation)$ \\
        \textbf{કાર્ય} & શરત સંતોષતા ટપલ્સ મેળવે છે \\
        \textbf{ઉદાહરણ} & $\sigma_{salary>30000}(Employee)$ \\
        \bottomrule
    \end{tabulary}
\end{table}

\begin{mnemonicbox}
    \textbf{મેમરી ટ્રીક:} "SERVe" (Select Exactly Required Values)
\end{mnemonicbox}
\end{solutionbox}

\questionmarks{2(બ)}{4}{DBMS માં પ્રાઇમરી, ફોરેઇન, સુપર, કેન્ડીડેટ કી ની વ્યાખ્યા આપો.}
\begin{solutionbox}
\begin{table}[H]
    \centering
    \caption{કી ના પ્રકારો}
    \begin{tabulary}{\linewidth}{LCL}
        \toprule
        \textbf{કી પ્રકાર} & \textbf{વર્ણન} \\
        \midrule
        \textbf{પ્રાઇમરી કી} & દરેક રેકોર્ડ માટે યુનિક ઓળખકર્તા \\
        \textbf{ફોરેઇન કી} & અન્ય ટેબલમાં પ્રાઇમરી કી સાથે જોડાતું એટ્રિબ્યુટ \\
        \textbf{સુપર કી} & એટ્રિબ્યુટ્સનો સેટ જે રેકોર્ડ્સને યુનિક રીતે ઓળખી શકે છે \\
        \textbf{કેન્ડીડેટ કી} & મિનિમલ સુપર કી જે પ્રાઇમરી કી બની શકે છે \\
        \bottomrule
    \end{tabulary}
\end{table}

\begin{mnemonicbox}
    \textbf{મેમરી ટ્રીક:} "PFSC" (Person First Shows Credentials)
\end{mnemonicbox}
\end{solutionbox}

\questionmarks{2(ક)}{7}{Library Management System નો E R Diagram દોરો.}
\begin{solutionbox}
\begin{center}
\begin{tikzpicture}[gtu block, node distance=2.5cm]
    \node [entity] (book) {BOOK};
    \node [relationship, below of=book] (issued) {is\_issued};
    \node [entity, below of=issued] (issue) {ISSUE};
    
    \node [relationship, left of=issue, xshift=-1cm] (borrows) {borrows};
    \node [entity, left of=borrows, xshift=-1cm] (member) {MEMBER};
    
    \node [relationship, right of=issue, xshift=1cm] (processes) {processes};
    \node [entity, right of=processes, xshift=1cm] (lib) {LIBRARIAN};
    
    % Attributes Book
    \node [attribute, above left of=book] (bid) {\underline{book\_id}};
    \node [attribute, above of=book] (btitle) {title};
    \node [attribute, above right of=book] (bauth) {author};
    
    % Attributes Member
    \node [attribute, above left of=member] (mid) {\underline{member\_id}};
    \node [attribute, left of=member] (mname) {name};
    \node [attribute, below left of=member] (mphone) {phone};
    
    % Attributes Issue
    \node [attribute, below left of=issue] (iid) {\underline{issue\_id}};
    \node [attribute, below right of=issue] (idate) {issue\_date};
    
    % Attributes Librarian
    \node [attribute, above right of=lib] (lid) {\underline{staff\_id}};
    \node [attribute, right of=lib] (lname) {name};
    
    % Connections
    \draw [gtu line] (book) -- (issued);
    \draw [gtu line] (issued) -- (issue);
    \draw [gtu line] (member) -- (borrows);
    \draw [gtu line] (borrows) -- (issue);
    \draw [gtu line] (lib) -- (processes);
    \draw [gtu line] (processes) -- (issue);
    
    \draw [gtu line] (book) -- (bid);
    \draw [gtu line] (book) -- (btitle);
    \draw [gtu line] (book) -- (bauth);
    
    \draw [gtu line] (member) -- (mid);
    \draw [gtu line] (member) -- (mname);
    \draw [gtu line] (member) -- (mphone);
    
    \draw [gtu line] (issue) -- (iid);
    \draw [gtu line] (issue) -- (idate);
    
    \draw [gtu line] (lib) -- (lid);
    \draw [gtu line] (lib) -- (lname);
\end{tikzpicture}
\captionof{figure}{લાઇબ્રેરી મેનેજમેન્ટ ER ડાયાગ્રામ}
\end{center}

\begin{mnemonicbox}
    \textbf{મેમરી ટ્રીક:} "LIMB" (Library Items, Members, Borrowing)
\end{mnemonicbox}
\end{solutionbox}

\orquestionmarks{2(અ)}{3}{રિલેશનલ એલ્જેબ્રા નું યુનિયન ઓપરેશન સમજાવો.}
\begin{solutionbox}
\begin{table}[H]
    \centering
    \caption{યુનિયન ઓપરેશન ($\cup$)}
    \begin{tabulary}{\linewidth}{LCL}
        \toprule
        \textbf{ફીચર} & \textbf{વર્ણન} \\
        \midrule
        \textbf{સિન્ટેક્સ} & $Relation1 \cup Relation2$ \\
        \textbf{કાર્ય} & બંને સંબંધોમાંથી ટપલ્સ એકત્રિત કરે છે \\
        \textbf{આવશ્યકતા} & બંને સંબંધો યુનિયન-સંગત હોવા જોઈએ \\
        \bottomrule
    \end{tabulary}
\end{table}

\begin{mnemonicbox}
    \textbf{મેમરી ટ્રીક:} "CUP" (Combining Union of Parts)
\end{mnemonicbox}
\end{solutionbox}

\orquestionmarks{2(બ)}{4}{ઉદાહરણ સાથે કંપોઝિટ એટ્રિબ્યુટ અને મલ્ટીવેલ્યુડ એટ્રિબ્યુટ ને વ્યાખ્યાયિત કરો.}
\begin{solutionbox}
\begin{table}[H]
    \centering
    \caption{એટ્રિબ્યુટ પ્રકારો}
    \begin{tabulary}{\linewidth}{LCL}
        \toprule
        \textbf{પ્રકાર} & \textbf{વર્ણન} & \textbf{ઉદાહરણ} \\
        \midrule
        \textbf{કંપોઝિટ} & નાના સબપાર્ટ્સમાં વિભાજિત થઈ શકે છે & એડ્રેસ (સ્ટ્રીટ, શહેર, રાજ્ય) \\
        \textbf{મલ્ટીવેલ્યુડ} & એક કરતાં વધુ મૂલ્ય ધરાવી શકે છે & ફોન નંબર્સ \\
        \bottomrule
    \end{tabulary}
\end{table}

\begin{center}
\begin{tikzpicture}[gtu block, node distance=2cm, level 1/.style={sibling distance=3cm}, level 2/.style={sibling distance=1.5cm}]
    \node [entity] {વ્યક્તિ (Person)}
        child {node [attribute] {એડ્રેસ}
            child {node [attribute] {સ્ટ્રીટ}}
            child {node [attribute] {શહેર}}
            child {node [attribute] {રાજ્ય}}
        }
        child {node [multi attribute] {ફોન નંબર્સ}
            child {node [attribute] {નંબર 1}}
            child {node [attribute] {નંબર 2}}
        };
\end{tikzpicture}
\captionof{figure}{એટ્રિબ્યુટ પ્રકારો}
\end{center}

\begin{mnemonicbox}
    \textbf{મેમરી ટ્રીક:} "CoMbo" (Composite has Multiple components)
\end{mnemonicbox}
\end{solutionbox}

\orquestionmarks{2(ક)}{7}{College Management System નો E R Diagram દોરો.}
\begin{solutionbox}
\begin{center}
\begin{tikzpicture}[gtu block, node distance=3cm]
    \node [entity] (dept) {DEPARTMENT};
    \node [relationship, right of=dept] (offers) {offers};
    \node [entity, right of=offers] (course) {COURSE};
    
    \node [relationship, below of=dept] (employs) {employs};
    \node [entity, below of=employs] (faculty) {FACULTY};
    
    \node [relationship, below of=course] (teaches) {teaches};
    
    \node [relationship, above of=dept] (enrolls) {enrolls};
    \node [entity, above of=enrolls] (student) {STUDENT};
    
    \node [relationship, right of=student, xshift=1cm] (registers) {registers};
    \node [entity, right of=registers, xshift=1cm] (enrollment) {ENROLLMENT};
    
    % Connections
    \draw [gtu line] (dept) -- (offers);
    \draw [gtu line] (offers) -- (course);
    \draw [gtu line] (dept) -- (employs);
    \draw [gtu line] (employs) -- (faculty);
    \draw [gtu line] (faculty) -| (teaches);
    \draw [gtu line] (teaches) |- (course);
    \draw [gtu line] (dept) -- (enrolls);
    \draw [gtu line] (enrolls) -- (student);
    \draw [gtu line] (student) -- (registers);
    \draw [gtu line] (registers) -- (enrollment);
    \draw [gtu line] (course) -| (registers);
    
    % Attributes
    \node [attribute, left of=dept] {\underline{dept\_id}};
    \node [attribute, left of=student] {\underline{student\_id}};
    \node [attribute, right of=course] {\underline{course\_id}};
    \node [attribute, left of=faculty] {\underline{faculty\_id}};
    \node [attribute, right of=enrollment] {\underline{enrollment\_id}};
\end{tikzpicture}
\captionof{figure}{કોલેજ મેનેજમેન્ટ સિસ્ટમ ER ડાયાગ્રામ}
\end{center}

\begin{mnemonicbox}
    \textbf{મેમરી ટ્રીક:} "DECFS" (Departments, Enrollments, Courses, Faculty, Students)
\end{mnemonicbox}
\end{solutionbox}

\questionmarks{3(અ)}{3}{SQL માં વિવિધ ડેટા ટાઈપ્સ ની યાદી બનાવો અને ટુંક માં સમજાવો}
\begin{solutionbox}
\begin{table}[H]
    \centering
    \caption{SQL ડેટા ટાઈપ્સ}
    \begin{tabulary}{\linewidth}{LCL}
        \toprule
        \textbf{કેટેગરી} & \textbf{ઉદાહરણો} & \textbf{ઉપયોગ} \\
        \midrule
        \textbf{ન્યુમેરિક} & INT, FLOAT, DECIMAL & સંખ્યાઓ સંગ્રહ કરવા \\
        \textbf{કેરેક્ટર} & CHAR, VARCHAR, TEXT & ટેક્સ્ટ સંગ્રહ કરવા \\
        \textbf{ડેટ/ટાઈમ} & DATE, TIME, TIMESTAMP & સમય સંબંધિત ડેટા સંગ્રહ કરવા \\
        \textbf{બૂલિયન} & BOOLEAN & સાચા/ખોટા મૂલ્યો સંગ્રહ કરવા \\
        \textbf{બાઇનરી} & BLOB, BINARY & બાઇનરી ડેટા સંગ્રહ કરવા \\
        \bottomrule
    \end{tabulary}
\end{table}

\begin{mnemonicbox}
    \textbf{મેમરી ટ્રીક:} "NCDBB" (Numbers, Characters, Dates, Booleans, Binaries)
\end{mnemonicbox}
\end{solutionbox}

\questionmarks{3(બ)}{4}{કોઈ પણ બે DDL કમાન્ડસ સિંટેક્ષ અને ઉદાહરણ સાથે સમજાવો.}
\begin{solutionbox}
\begin{table}[H]
    \centering
    \caption{DDL કમાન્ડ્સ}
    \begin{tabulary}{\linewidth}{LCL}
        \toprule
        \textbf{કમાન્ડ} & \textbf{સિન્ટેક્સ} & \textbf{ઉદાહરણ} \\
        \midrule
        \textbf{CREATE} & \code{CREATE TABLE t (cols);} & \code{CREATE TABLE Student (id INT, n TEXT);} \\
        \textbf{ALTER} & \code{ALTER TABLE t ADD c type;} & \code{ALTER TABLE Student ADD email TEXT;} \\
        \bottomrule
    \end{tabulary}
\end{table}

\begin{center}
\begin{tikzpicture}[gtu block, node distance=3cm]
    \node [gtu block, fill=orange!10] (ddl) {DDL કમાન્ડ્સ};
    \node [gtu process, below left of=ddl] (create) {CREATE: નવા ઓબ્જેક્ટ્સ};
    \node [gtu process, below right of=ddl] (alter) {ALTER: સુધારા કરવા};
    
    \draw [gtu arrow] (ddl) -- (create);
    \draw [gtu arrow] (ddl) -- (alter);
\end{tikzpicture}
\captionof{figure}{DDL કમાન્ડ્સ}
\end{center}

\begin{mnemonicbox}
    \textbf{મેમરી ટ્રીક:} "CAD" (Create And Define)
\end{mnemonicbox}
\end{solutionbox}

\questionmarks{3(ક)}{7}{નીચે ની ક્વેરી નું આઉટપુટ લખો.}
\begin{solutionbox}
\begin{table}[H]
    \centering
    \caption{SQL ક્વેરી પરિણામો}
    \begin{tabulary}{\linewidth}{LCL}
        \toprule
        \textbf{ફંક્શન} & \textbf{પરિણામ} & \textbf{સમજૂતી} \\
        \midrule
        \textbf{CEIL(123.57)} & 124 & 123.57 થી નાની નહીં તેવી પૂર્ણ સંખ્યા \\
        \textbf{CEIL(4.1)} & 5 & 4.1 થી નાની નહીં તેવી પૂર્ણ સંખ્યા \\
        \textbf{MOD(12,4)} & 0 & 12 $\div$ 4 નો શેષ \\
        \textbf{MOD(10,4)} & 2 & 10 $\div$ 4 નો શેષ \\
        \textbf{POWER(2,3)} & 8 & $2^3$ \\
        \textbf{POWER(3,3)} & 27 & $3^3$ \\
        \textbf{ROUND(121.413,1)} & 121.4 & 1 દશાંશ સુધી રાઉન્ડ \\
        \textbf{ROUND(121.413,2)} & 121.41 & 2 દશાંશ સુધી રાઉન્ડ \\
        \textbf{FLOOR(25.3)} & 25 & 25.3 થી મોટી નહીં તેવી પૂર્ણ સંખ્યા \\
        \textbf{FLOOR(25.7)} & 25 & 25.7 થી મોટી નહીં તેવી પૂર્ણ સંખ્યા \\
        \textbf{LENGTH('AHMEDABAD')} & 9 & અક્ષરોની સંખ્યા \\
        \textbf{ABS(-25)} & 25 & નિરપેક્ષ મૂલ્ય \\
        \textbf{ABS(36)} & 36 & નિરપેક્ષ મૂલ્ય \\
        \bottomrule
    \end{tabulary}
\end{table}

\begin{mnemonicbox}
    \textbf{મેમરી ટ્રીક:} "CMPRFLA" (Ceiling, Modulus, Power, Round, Floor, Length, Absolute)
\end{mnemonicbox}
\end{solutionbox}

\orquestionmarks{3(અ)}{3}{કોઈ પણ ત્રણ ડેટ ફંક્શન સમજાવો.}
\begin{solutionbox}
\begin{table}[H]
    \centering
    \caption{ડેટ ફંક્શન્સ}
    \begin{tabulary}{\linewidth}{LCL}
        \toprule
        \textbf{ફંક્શન} & \textbf{હેતુ} & \textbf{ઉદાહરણ} \\
        \midrule
        \textbf{ADD\_MONTHS} & તારીખમાં મહિના ઉમેરે છે & જાન્યુઆરીમાં 3 મહિના $\to$ એપ્રિલ \\
        \textbf{MONTHS\_BETWEEN} & બે તારીખો વચ્ચેના મહિના & માર્ચ અને જાન્યુઆરી વચ્ચે $\to$ 2 \\
        \textbf{SYSDATE} & વર્તમાન તારીખ અને સમય & સિસ્ટમ તારીખ આપે છે \\
        \bottomrule
    \end{tabulary}
\end{table}

\begin{mnemonicbox}
    \textbf{મેમરી ટ્રીક:} "AMS" (Add\_months, Months\_between, Sysdate)
\end{mnemonicbox}
\end{solutionbox}

\orquestionmarks{3(બ)}{4}{કોઈ પણ બે DML કમાન્ડ સિંટેક્ષ અને ઉદાહરણ સાથે સમજાવો.}
\begin{solutionbox}
\begin{table}[H]
    \centering
    \caption{DML કમાન્ડ્સ}
    \begin{tabulary}{\linewidth}{LCL}
        \toprule
        \textbf{કમાન્ડ} & \textbf{સિન્ટેક્સ} & \textbf{ઉદાહરણ} \\
        \midrule
        \textbf{INSERT} & \code{INSERT INTO t VALUES (v1...);} & \code{INSERT INTO S VALUES (1, 'Raj');} \\
        \textbf{UPDATE} & \code{UPDATE t SET c=v WHERE cond;} & \code{UPDATE S SET e='n' WHERE id=1;} \\
        \bottomrule
    \end{tabulary}
\end{table}

\begin{center}
\begin{tikzpicture}[gtu block, node distance=3cm]
    \node [gtu block, fill=green!10] (dml) {DML કમાન્ડ્સ};
    \node [gtu process, below left of=dml] (insert) {INSERT: રેકોર્ડ ઉમેરો};
    \node [gtu process, below right of=dml] (update) {UPDATE: રેકોર્ડ સુધારો};
    
    \draw [gtu arrow] (dml) -- (insert);
    \draw [gtu arrow] (dml) -- (update);
\end{tikzpicture}
\captionof{figure}{DML કમાન્ડ્સ}
\end{center}

\begin{mnemonicbox}
    \textbf{મેમરી ટ્રીક:} "IUM" (Insert, Update, Manipulate)
\end{mnemonicbox}
\end{solutionbox}

\orquestionmarks{3(ક)}{7}{ટેબલ EMP(emp\_no, emp\_name, designation, salary, deptno) ને ધ્યાને લઈ ને નીચે આપેલા operations માટે SQL commands લખો.}
\begin{solutionbox}
\begin{itemize}
    \item \textbf{Create table EMP}:
    \begin{lstlisting}[language=SQL]
CREATE TABLE EMP (
    emp_no INT PRIMARY KEY, 
    emp_name VARCHAR(50), 
    designation VARCHAR(30), 
    salary DECIMAL(10,2), 
    deptno INT
);
    \end{lstlisting}
    
    \item \textbf{Given Columns Select}:
    \begin{lstlisting}[language=SQL]
SELECT emp_no, emp_name, designation, salary, deptno FROM EMP;
    \end{lstlisting}
    
    \item \textbf{Name starts with 'p'}:
    \begin{lstlisting}[language=SQL]
SELECT * FROM EMP WHERE emp_name LIKE 'p%';
    \end{lstlisting}
    
    \item \textbf{Department wise salary}:
    \begin{lstlisting}[language=SQL]
SELECT deptno, SUM(salary) AS total_salary 
FROM EMP GROUP BY deptno;
    \end{lstlisting}
    
    \item \textbf{Add email\_id}:
    \begin{lstlisting}[language=SQL]
ALTER TABLE EMP ADD email_id VARCHAR(100);
    \end{lstlisting}
    
    \item \textbf{Rename column}:
    \begin{lstlisting}[language=SQL]
ALTER TABLE EMP RENAME COLUMN designation TO post;
    \end{lstlisting}
    
    \item \textbf{Delete all records}:
    \begin{lstlisting}[language=SQL]
DELETE FROM person;
    \end{lstlisting}
\end{itemize}

\begin{mnemonicbox}
    \textbf{મેમરી ટ્રીક:} "CSDAACD" (Create, Select, Display, Aggregate, Add, Change, Delete)
\end{mnemonicbox}
\end{solutionbox}

\questionmarks{4(અ)}{3}{વિવિધ aggregate functions ની યાદી બનાવો અને કોઈ એક ને syntax અને ઉદાહરણ સાથે સમજાવો.}
\begin{solutionbox}
\begin{table}[H]
    \centering
    \caption{એગ્રીગેટ ફંક્શન્સ}
    \begin{tabulary}{\linewidth}{LCL}
        \toprule
        \textbf{ફંક્શન} & \textbf{હેતુ} \\
        \midrule
        \textbf{SUM} & કુલ ગણતરી કરે છે \\
        \textbf{AVG} & સરેરાશ ગણતરી કરે છે \\
        \textbf{COUNT} & રો ની સંખ્યા ગણે છે \\
        \textbf{MAX} & મહત્તમ મૂલ્ય શોધે છે \\
        \textbf{MIN} & લઘુત્તમ મૂલ્ય શોધે છે \\
        \bottomrule
    \end{tabulary}
\end{table}

\textbf{ઉદાહરણ (AVG)}: \code{SELECT AVG(salary) FROM Employee;}

\begin{mnemonicbox}
    \textbf{મેમરી ટ્રીક:} "SCAMM" (Sum, Count, Avg, Max, Min)
\end{mnemonicbox}
\end{solutionbox}

\questionmarks{4(બ)}{4}{ટ્રાન્સેક્શન ને ઉદાહરણ સાથે વ્યાખ્યાયિત કરો.}
\begin{solutionbox}
\textbf{ટ્રાન્સેક્શન}: કાર્યનો તાર્કિક એકમ જે સંપૂર્ણપણે પ્રોસેસ થવો જોઈએ અથવા સંપૂર્ણપણે નિષ્ફળ જવો જોઈએ.

\begin{table}[H]
    \centering
    \caption{ટ્રાન્સેક્શન}
    \begin{tabulary}{\linewidth}{LCL}
        \toprule
        \textbf{વિચાર} & \textbf{વર્ણન} \\
        \midrule
        \textbf{ACID} & એટોમિસિટી, કન્સિસ્ટન્સી, આઈસોલેશન, ડ્યુરેબિલિટી \\
        \textbf{સ્થિતિઓ} & એક્ટિવ, પાર્શિયલી કમિટેડ, કમિટેડ, ફેઇલ્ડ, એબોર્ટેડ \\
        \bottomrule
    \end{tabulary}
\end{table}

\begin{lstlisting}[language=SQL]
BEGIN TRANSACTION;
    UPDATE Accounts SET balance = balance - 5000 WHERE acc_no = 'A123';
    UPDATE Accounts SET balance = balance + 5000 WHERE acc_no = 'B456';
COMMIT;
\end{lstlisting}

\begin{mnemonicbox}
    \textbf{મેમરી ટ્રીક:} "TAPS" (Transaction As Process Set)
\end{mnemonicbox}
\end{solutionbox}

\questionmarks{4(ક)}{7}{SQL માં ઓપરેટર શું છે? એરિથમેટિક અને લોજિકલ ઓપરેટર ઉદાહરણ સાથે સમજાવો.}
\begin{solutionbox}
\begin{table}[H]
    \centering
    \caption{SQL ઓપરેટર્સ}
    \begin{tabulary}{\linewidth}{LCL}
        \toprule
        \textbf{પ્રકાર} & \textbf{ઓપરેટર્સ} & \textbf{ઉદાહરણ} \\
        \midrule
        \textbf{એરિથમેટિક} & +, -, *, /, \% & 5 + 3 = 8 \\
        \textbf{લોજિકલ} & AND & salary > 3k AND dept = 'IT' \\
        \textbf{લોજિકલ} & OR & salary > 5k OR dept = 'HR' \\
        \textbf{લોજિકલ} & NOT & NOT (condition) \\
        \bottomrule
    \end{tabulary}
\end{table}

\begin{mnemonicbox}
    \textbf{મેમરી ટ્રીક:} "ASMDOLA" (Add, Subtract, Multiply, Divide, OR, AND, NOT)
\end{mnemonicbox}
\end{solutionbox}

\orquestionmarks{4(અ)}{3}{વિવિધ numeric functions ની યાદી બનાવો અને કોઈ એક ને syntax અને ઉદાહરણ સાથે સમજાવો.}
\begin{solutionbox}
\begin{table}[H]
    \centering
    \caption{ન્યુમેરિક ફંક્શન્સ}
    \begin{tabulary}{\linewidth}{LCL}
        \toprule
        \textbf{ફંક્શન} & \textbf{હેતુ} \\
        \midrule
        \textbf{ROUND} & નિર્દિષ્ટ દશાંશ સુધી રાઉન્ડ કરે છે \\
        \textbf{TRUNC} & નિર્દિષ્ટ દશાંશ સુધી ટ્રંકેટ કરે છે \\
        \textbf{CEIL} & મોટી કે સમાન સૌથી નાની પૂર્ણ સંખ્યા \\
        \textbf{FLOOR} & નાની કે સમાન સૌથી મોટી પૂર્ણ સંખ્યા \\
        \textbf{ABS} & નિરપેક્ષ મૂલ્ય \\
        \bottomrule
    \end{tabulary}
\end{table}

\textbf{ઉદાહરણ (ROUND)}: \code{ROUND(125.679, 2)} $\to$ 125.68

\begin{mnemonicbox}
    \textbf{મેમરી ટ્રીક:} "RTCFA" (Round, Truncate, Ceiling, Floor, Absolute)
\end{mnemonicbox}
\end{solutionbox}

\orquestionmarks{4(બ)}{4}{ટ્રાન્સેક્શન માટે વિવિધ database operations ની યાદી બનાવો.}
\begin{solutionbox}
\begin{table}[H]
    \centering
    \caption{ટ્રાન્સેક્શન ઓપરેશન્સ}
    \begin{tabulary}{\linewidth}{LCL}
        \toprule
        \textbf{ઓપરેશન} & \textbf{વર્ણન} \\
        \midrule
        \textbf{BEGIN} & શરૂઆત \\
        \textbf{READ} & ડેટા મેળવવો \\
        \textbf{WRITE} & સુધારા કરવા \\
        \textbf{COMMIT} & કાયમી સેવ કરવું \\
        \textbf{ROLLBACK} & ફેરફારો રદ કરવા \\
        \textbf{SAVEPOINT} & આંશિક રોલબેક \\
        \bottomrule
    \end{tabulary}
\end{table}

\begin{center}
\begin{tikzpicture}[gtu flow]
    \node [gtu start] (begin) {BEGIN};
    \node [gtu process, below of=begin] (rw) {READ / WRITE};
    \node [gtu decision, below of=rw] (check) {સફળ?};
    \node [gtu stop, below left of=check, xshift=-1cm] (commit) {COMMIT};
    \node [gtu stop, below right of=check, xshift=1cm] (rollback) {ROLLBACK};
    
    \draw [gtu arrow] (begin) -- (rw);
    \draw [gtu arrow] (rw) -- (check);
    \draw [gtu arrow] (check) -| node[near start] {હા} (commit);
    \draw [gtu arrow] (check) -| node[near start] {ના} (rollback);
\end{tikzpicture}
\captionof{figure}{ટ્રાન્સેક્શન ફ્લો}
\end{center}

\begin{mnemonicbox}
    \textbf{મેમરી ટ્રીક:} "BRWCRS" (Begin, Read, Write, Commit, Rollback, Savepoint)
\end{mnemonicbox}
\end{solutionbox}

\orquestionmarks{4(ક)}{7}{જોઇન શું છે? વિવિધ પ્રકાર ના જોઇન ને syntax અને ઉદાહરણ સાથે સમજાવો.}
\begin{solutionbox}
\begin{table}[H]
    \centering
    \caption{જોઇન પ્રકારો}
    \begin{tabulary}{\linewidth}{LCL}
        \toprule
        \textbf{પ્રકાર} & \textbf{વર્ણન} \\
        \midrule
        \textbf{INNER JOIN} & બંનેમાં મેચ હોય \\
        \textbf{LEFT JOIN} & ડાબું બધું, જમણું મેચિંગ \\
        \textbf{RIGHT JOIN} & જમણું બધું, ડાબું મેચિંગ \\
        \textbf{FULL JOIN} & કોઈપણ એકમાં મેચ હોય \\
        \textbf{SELF JOIN} & ટેબલનું પોતાની સાથે જોઇન \\
        \bottomrule
    \end{tabulary}
\end{table}

\begin{center}
\begin{tikzpicture}[gtu block, node distance=2cm]
    \node [gtu block, fill=purple!10] (joins) {JOIN પ્રકારો};
    \node [gtu process, below of=joins] (inner) {INNER};
    \node [gtu process, left of=inner, xshift=-1cm] (left) {LEFT};
    \node [gtu process, right of=inner, xshift=1cm] (right) {RIGHT};
    \node [gtu process, below of=left] (full) {FULL};
    \node [gtu process, below of=right] (self) {SELF};
    
    \draw [gtu arrow] (joins) -- (inner);
    \draw [gtu arrow] (joins) -- (left);
    \draw [gtu arrow] (joins) -- (right);
    \draw [gtu arrow] (joins) -- (full);
    \draw [gtu arrow] (joins) -- (self);
\end{tikzpicture}
\captionof{figure}{જોઇન પ્રકારો}
\end{center}

\begin{mnemonicbox}
    \textbf{મેમરી ટ્રીક:} "ILRFS" (Inner, Left, Right, Full, Self)
\end{mnemonicbox}
\end{solutionbox}

\questionmarks{5(અ)}{3}{નીચે આપેલા customer relation ને 1NF માં બદલાવો.}
\begin{solutionbox}
\textbf{Customer Table (1NF)}:
\begin{table}[H]
    \centering
    \caption{Customer (1NF)}
    \begin{tabulary}{\linewidth}{LCLLL}
        \toprule
        \textbf{cid} & \textbf{name} & \textbf{society} & \textbf{city} & \textbf{Contact\_no} \\
        \midrule
        CO1 & Riya & Amu aavas & Anand & 5322332123 \\
        CO2 & Jiya & Sardar colony & Ahmedabad & 5326521456 \\
        CO2 & Jiya & Sardar colony & Ahmedabad & 5265232849 \\
        \bottomrule
    \end{tabulary}
\end{table}

\begin{mnemonicbox}
    \textbf{મેમરી ટ્રીક:} "AFM" (Atomic values, Flatten Multivalued attributes)
\end{mnemonicbox}
\end{solutionbox}

\questionmarks{5(બ)}{4}{ટ્રાન્સેક્શન ની ACID properties ની યાદી બનાવો અને સમજાવો.}
\begin{solutionbox}
\begin{table}[H]
    \centering
    \caption{ACID ગુણધર્મો}
    \begin{tabulary}{\linewidth}{LCL}
        \toprule
        \textbf{ગુણધર્મ} & \textbf{વર્ણન} \\
        \midrule
        \textbf{Atomicity} & બધું અથવા કંઈ નહીં \\
        \textbf{Consistency} & ડેટાબેઝ સુસંગત રહે છે \\
        \textbf{Isolation} & સમાંતર સુરક્ષા \\
        \textbf{Durability} & કાયમી ફેરફારો \\
        \bottomrule
    \end{tabulary}
\end{table}

\begin{center}
\begin{tikzpicture}[gtu block, node distance=2.5cm]
    \node [gtu block, fill=yellow!10] (acid) {ACID};
    \node [gtu process, above left of=acid] (a) {Atomicity};
    \node [gtu process, above right of=acid] (c) {Consistency};
    \node [gtu process, below right of=acid] (i) {Isolation};
    \node [gtu process, below left of=acid] (d) {Durability};
    
    \draw [gtu arrow] (acid) -- (a);
    \draw [gtu arrow] (acid) -- (c);
    \draw [gtu arrow] (acid) -- (i);
    \draw [gtu arrow] (acid) -- (d);
\end{tikzpicture}
\captionof{figure}{ACID ગુણધર્મો}
\end{center}

\begin{mnemonicbox}
    \textbf{મેમરી ટ્રીક:} "ACID" (Atomicity, Consistency, Isolation, Durability)
\end{mnemonicbox}
\end{solutionbox}

\questionmarks{5(ક)}{7}{વિવિધ functional dependencies ની યાદી બનાવો અને દરેક ને ઉદાહરણ સાથે સમજાવો.}
\begin{solutionbox}
\begin{table}[H]
    \centering
    \caption{ફંક્શનલ ડિપેન્ડન્સી}
    \begin{tabulary}{\linewidth}{LCL}
        \toprule
        \textbf{પ્રકાર} & \textbf{વર્ણન} & \textbf{ઉદાહરણ} \\
        \midrule
        \textbf{Trivial FD} & $Y \subseteq X$ & $\{ID, Name\} \to \{Name\}$ \\
        \textbf{Non-trivial FD} & $Y \not\subseteq X$ & $\{ID\} \to \{Name\}$ \\
        \textbf{Partial FD} & કી નો ભાગ $\to$ નોન-કી & $\{Course, Student\} \to CourseName$ \\
        \textbf{Transitive FD} & $X \to Y \to Z$ & $Student \to Dept \to DeptName$ \\
        \textbf{Multivalued FD} & એક $\to$ સેટ & $Course \to\to Textbook$ \\
        \bottomrule
    \end{tabulary}
\end{table}

\begin{center}
\begin{tikzpicture}[gtu block, node distance=2.5cm]
    \node [attribute] (a) {A};
    \node [attribute, right of=a] (b) {B};
    \node [attribute, right of=b] (c) {C};
    
    \draw [gtu arrow] (a) -- node[above] {નક્કી કરે} (b);
    \draw [gtu arrow] (b) -- node[above] {નક્કી કરે} (c);
    \draw [gtu arrow, bend right] (a) to node[below] {ટ્રાન્ઝિટિવ} (c);
\end{tikzpicture}
\captionof{figure}{ટ્રાન્ઝિટિવ ડિપેન્ડન્સી}
\end{center}

\begin{mnemonicbox}
    \textbf{મેમરી ટ્રીક:} "TNPTMv" (Trivial, Non-trivial, Partial, Transitive, Multivalued)
\end{mnemonicbox}
\end{solutionbox}

\orquestionmarks{5(અ)}{3}{નીચે આપેલા Depositor\_Account relation ને 2NF માં બદલાવો.}
\begin{solutionbox}
\textbf{Account Table (2NF)}:
\begin{table}[H]
    \centering
    \begin{tabulary}{\linewidth}{LCL}
        \toprule
        \textbf{ano} & \textbf{balance} & \textbf{bname} \\
        \midrule
         & & \\
        \bottomrule
    \end{tabulary}
\end{table}

\textbf{Depositor Table (2NF)}:
\begin{table}[H]
    \centering
    \begin{tabulary}{\linewidth}{LCL}
        \toprule
        \textbf{cid} & \textbf{ano} & \textbf{access\_date} \\
        \midrule
         & & \\
        \bottomrule
    \end{tabulary}
\end{table}

\begin{mnemonicbox}
    \textbf{મેમરી ટ્રીક:} "RPKD" (Remove Partial Key Dependencies)
\end{mnemonicbox}
\end{solutionbox}

\orquestionmarks{5(બ)}{4}{Conflict serializability સમજાવો.}
\begin{solutionbox}
\begin{table}[H]
    \centering
    \caption{Conflict Serializability}
    \begin{tabulary}{\linewidth}{LCL}
        \toprule
        \textbf{કન્સેપ્ટ} & \textbf{વર્ણન} \\
        \midrule
        \textbf{વ્યાખ્યા} & સિરિયલ શેડ્યૂલ સાથે સમકક્ષ \\
        \textbf{કન્ફ્લિક્ટ Ops} & RW, WR, WW એક જ આઇટમ પર \\
        \textbf{ટેસ્ટિંગ} & ગ્રાફમાં કોઈ ચક્ર નહીં \\
        \bottomrule
    \end{tabulary}
\end{table}

\begin{center}
\begin{tikzpicture}[gtu state, node distance=3cm]
    \node [gtu state] (t1) {T1};
    \node [gtu state, right of=t1] (t2) {T2};
    \node [gtu state, right of=t2] (t3) {T3};
    
    \draw [gtu arrow] (t1) -- (t2);
    \draw [gtu arrow] (t2) -- (t3);
    \node [below of=t2, yshift=1cm] {કોઈ ચક્ર નહીં $\implies$ સીરિયલાઇઝેબલ};
\end{tikzpicture}
\captionof{figure}{પ્રીસિડન્સ ગ્રાફ}
\end{center}

\begin{mnemonicbox}
    \textbf{મેમરી ટ્રીક:} "COGS" (Conflict Operations Graph Serializable)
\end{mnemonicbox}
\end{solutionbox}

\orquestionmarks{5(ક)}{7}{ઉદાહરણ સાથે 3NF normalization સમજાવો.}
\begin{solutionbox}
\begin{table}[H]
    \centering
    \caption{Normal Forms}
    \begin{tabulary}{\linewidth}{LCL}
        \toprule
        \textbf{Form} & \textbf{વ્યાખ્યા} & \textbf{ઉદાહરણ} \\
        \midrule
        \textbf{1NF} & એટોમિક & ફોન નંબર્સ અલગ કરો \\
        \textbf{2NF} & પાર્શિયલ Dep નહીં & ઓર્ડર વિગતો અલગ કરો \\
        \textbf{3NF} & ટ્રાન્ઝિટિવ Dep નહીં & સ્ટુડન્ટ વિભાગ અલગ કરો \\
        \bottomrule
    \end{tabulary}
\end{table}

\textbf{ઉદાહરણ}: \code{Emp(ID, Name, DeptID, DeptName)} $\to$ \\
\code{Emp(ID, Name, DeptID)} + \code{Dept(DeptID, DeptName)}

\begin{center}
\begin{tikzpicture}[gtu flow]
    \node [gtu start] (nf1) {1NF};
    \node [gtu process, right of=nf1, xshift=2cm] (nf2) {2NF};
    \node [gtu process, right of=nf2, xshift=2cm] (nf3) {3NF};
    
    \draw [gtu arrow] (nf1) -- node[above] {પાર્શિયલ દૂર કરો} (nf2);
    \draw [gtu arrow] (nf2) -- node[above] {ટ્રાન્ઝિટિવ દૂર કરો} (nf3);
\end{tikzpicture}
\captionof{figure}{નોર્મલાઈઝેશન ફ્લો}
\end{center}

\begin{mnemonicbox}
    \textbf{મેમરી ટ્રીક:} "APTN" (Atomic, Partial, Transitive, Normalized)
\end{mnemonicbox}
\end{solutionbox}

\end{document}
