\documentclass[10pt,a4paper]{article}

% content/resources/templates/preamble.tex
\usepackage[margin=0.6in]{geometry}
\author{Milav Dabgar}
\usepackage{amsmath,amssymb,amsthm}
\usepackage{booktabs}
\usepackage{multirow}
\usepackage{xcolor}
\usepackage{tcolorbox}
\tcbuselibrary{breakable,skins}
\usepackage[colorlinks=true,linkcolor=blue]{hyperref}
\usepackage{titlesec}
\usepackage{enumitem}
\usepackage{tikz}
\usepackage{pgfplots}
\usepackage{circuitikz}
\usepackage[version=4]{mhchem}
\usepackage{longtable}
\usepackage{array}
\usepackage{float}
\usepackage{caption}
\usepackage{listings}

\lstset{
  basicstyle=\small\ttfamily,
  breaklines=true,
  breakatwhitespace=false,
  postbreak=\mbox{\textcolor{red}{$\hookrightarrow$}\space},
  float=false,
  numbers=left,
  numberstyle=\tiny\color{gray},
  numbersep=10pt,
  xleftmargin=2em,
  keywordstyle=\color{blue},
  commentstyle=\color{green!60!black},
  stringstyle=\color{purple},
  backgroundcolor=\color{gray!5},
  showstringspaces=false,
  tabsize=2,
  captionpos=b,
  keepspaces=true,
  columns=flexible
}

\pgfplotsset{compat=1.18}
\usetikzlibrary{shapes,arrows,positioning,calc,patterns,decorations.pathmorphing,decorations.markings,arrows.meta}

% Color scheme
\definecolor{headcolor}{RGB}{0,102,204}
\definecolor{keycolor}{RGB}{220,20,60}
\definecolor{solutioncolor}{RGB}{34,139,34}
\definecolor{mnemoniccolor}{RGB}{148,0,211}
\definecolor{codecolor}{RGB}{0,0,100}

% Spacing
\setlength{\parskip}{3pt}
\setlist[itemize]{nosep}
\setlist[enumerate]{nosep}

% Title formatting
\titleformat{\section}{\Large\bfseries\color{headcolor}}{\thesection}{1em}{}
\titleformat{\subsection}{\large\bfseries\color{headcolor}}{\thesubsection}{1em}{}

% Pandoc tightlist compatibility
\providecommand{\tightlist}{%
  \setlength{\itemsep}{0pt}\setlength{\parskip}{0pt}}

% Pandoc longtable compatibility
\newcounter{none}
\def\thenone{}


% content/resources/templates/english-boxes.tex
% This file is currently empty - it exists to maintain consistency with the import structure.
% Add custom environments here if needed in the future.


\begin{document}

\begin{center}
{\Huge\bfseries\color{headcolor} Subject Name Solutions}\\[5pt]
{\LARGE 1333204 -- Winter 2023}\\[3pt]
{\large Semester 1 Study Material}\\[3pt]
{\normalsize\textit{Detailed Solutions and Explanations}}
\end{center}

\vspace{10pt}

\subsection*{Question 1(a) [3 marks]}\label{q1a}

\textbf{Define: Field, Record, Metadata}

\begin{solutionbox}

{\def\LTcaptype{none} % do not increment counter
\begin{longtable}[]{@{}
  >{\raggedright\arraybackslash}p{(\linewidth - 2\tabcolsep) * \real{0.3333}}
  >{\raggedright\arraybackslash}p{(\linewidth - 2\tabcolsep) * \real{0.6667}}@{}}
\toprule\noalign{}
\begin{minipage}[b]{\linewidth}\raggedright
Term
\end{minipage} & \begin{minipage}[b]{\linewidth}\raggedright
Definition
\end{minipage} \\
\midrule\noalign{}
\endhead
\bottomrule\noalign{}
\endlastfoot
\textbf{Field} & A single unit of data representing a specific attribute
in a database table (e.g., name, age, ID) \\
\textbf{Record} & A complete set of related fields that represents one
entity instance (a row in a table) \\
\textbf{Metadata} & Data that describes the structure, properties, and
relationships of other data (``data about data'') \\
\end{longtable}
}

\end{solutionbox}
\begin{mnemonicbox}
``FRM: Fields Row-up as Metadata''

\end{mnemonicbox}
\subsection*{Question 1(b) [4 marks]}\label{q1b}

\textbf{Define (i) E-R model (ii) Entity (iii) Entity set and (iv)
attributes}

\begin{solutionbox}

{\def\LTcaptype{none} % do not increment counter
\begin{longtable}[]{@{}
  >{\raggedright\arraybackslash}p{(\linewidth - 2\tabcolsep) * \real{0.3333}}
  >{\raggedright\arraybackslash}p{(\linewidth - 2\tabcolsep) * \real{0.6667}}@{}}
\toprule\noalign{}
\begin{minipage}[b]{\linewidth}\raggedright
Term
\end{minipage} & \begin{minipage}[b]{\linewidth}\raggedright
Definition
\end{minipage} \\
\midrule\noalign{}
\endhead
\bottomrule\noalign{}
\endlastfoot
\textbf{E-R Model} & A graphical approach to database design that models
entities, their attributes, and relationships \\
\textbf{Entity} & A real-world object, concept, or thing that has an
independent existence \\
\textbf{Entity Set} & A collection of similar entities that share the
same attributes (represented as a table) \\
\textbf{Attributes} & Properties or characteristics that describe an
entity (represented as columns in tables) \\
\end{longtable}
}

\includegraphics[width=1\linewidth,height=\textheight,keepaspectratio]{mermaid-e4055608.pdf}

\end{solutionbox}
\begin{mnemonicbox}
``EEAA: Entities Exist As Attributes''

\end{mnemonicbox}
\subsection*{Question 1(c) [7 marks]}\label{q1c}

\textbf{List the advantages and disadvantages of DBMS.}

\begin{solutionbox}

{\def\LTcaptype{none} % do not increment counter
\begin{longtable}[]{@{}
  >{\raggedright\arraybackslash}p{(\linewidth - 2\tabcolsep) * \real{0.4444}}
  >{\raggedright\arraybackslash}p{(\linewidth - 2\tabcolsep) * \real{0.5556}}@{}}
\toprule\noalign{}
\begin{minipage}[b]{\linewidth}\raggedright
Advantages
\end{minipage} & \begin{minipage}[b]{\linewidth}\raggedright
Disadvantages
\end{minipage} \\
\midrule\noalign{}
\endhead
\bottomrule\noalign{}
\endlastfoot
\textbf{Data sharing}: Multiple users can access simultaneously &
\textbf{Cost}: Expensive hardware/software requirements \\
\textbf{Data integrity}: Maintains accuracy through constraints &
\textbf{Complexity}: Requires specialized training \\
\textbf{Data security}: Controls access through permissions &
\textbf{Performance}: Can be slow for large databases \\
\textbf{Data independence}: Changes to storage don't affect apps &
\textbf{Vulnerability}: Central failure point risks data loss \\
\textbf{Reduced redundancy}: Eliminates duplicate data &
\textbf{Conversion costs}: Migrating from file systems is expensive \\
\end{longtable}
}

\end{solutionbox}
\begin{mnemonicbox}
``SIDSR vs CCPVC'' (Sharing, Integrity, Data
independence, Security, Redundancy vs Cost, Complexity, Performance,
Vulnerability, Conversion)

\end{mnemonicbox}
\subsection*{Question 1(c) OR [7
marks]}\label{q1c}

\textbf{Write the full form of DBA. Explain the roles and
responsibilities of DBA.}

\begin{solutionbox}

\textbf{DBA}: Database Administrator

{\def\LTcaptype{none} % do not increment counter
\begin{longtable}[]{@{}l@{}}
\toprule\noalign{}
Responsibilities of DBA \\
\midrule\noalign{}
\endhead
\bottomrule\noalign{}
\endlastfoot
\textbf{Database design}: Creates efficient database schema \\
\textbf{Security management}: Sets up user access controls \\
\textbf{Performance tuning}: Optimizes queries and indexes \\
\textbf{Backup \& recovery}: Implements data protection plans \\
\textbf{Maintenance}: Updates software and applies patches \\
\textbf{Troubleshooting}: Resolves database issues \\
\textbf{User support}: Trains and assists database users \\
\end{longtable}
}

\includegraphics[width=1\linewidth,height=\textheight,keepaspectratio]{mermaid-1d78db7b.pdf}

\end{solutionbox}
\begin{mnemonicbox}
``SPBT-MUS'' (Security, Performance, Backup,
Troubleshooting, Maintenance, User support)

\end{mnemonicbox}
\subsection*{Question 2(a) [3 marks]}\label{q2a}

\textbf{Explain single valued v/s multi-valued attributes with suitable
examples}

\begin{solutionbox}

{\def\LTcaptype{none} % do not increment counter
\begin{longtable}[]{@{}
  >{\raggedright\arraybackslash}p{(\linewidth - 4\tabcolsep) * \real{0.4103}}
  >{\raggedright\arraybackslash}p{(\linewidth - 4\tabcolsep) * \real{0.3333}}
  >{\raggedright\arraybackslash}p{(\linewidth - 4\tabcolsep) * \real{0.2564}}@{}}
\toprule\noalign{}
\begin{minipage}[b]{\linewidth}\raggedright
Attribute Type
\end{minipage} & \begin{minipage}[b]{\linewidth}\raggedright
Description
\end{minipage} & \begin{minipage}[b]{\linewidth}\raggedright
Examples
\end{minipage} \\
\midrule\noalign{}
\endhead
\bottomrule\noalign{}
\endlastfoot
\textbf{Single-valued} & Holds only one value for each entity instance &
Employee ID, Birth Date, Name \\
\textbf{Multi-valued} & Can hold multiple values for the same entity &
Phone Numbers, Skills, Email Addresses \\
\end{longtable}
}

\includegraphics[width=1\linewidth,height=\textheight,keepaspectratio]{mermaid-149e1e75.pdf}

\end{solutionbox}
\begin{mnemonicbox}
``SIM: Single Is Minimal, Multi Is Many''

\end{mnemonicbox}
\subsection*{Question 2(b) [4 marks]}\label{q2b}

\textbf{Explain Key Constraints for E-R diagram}

\begin{solutionbox}

{\def\LTcaptype{none} % do not increment counter
\begin{longtable}[]{@{}
  >{\raggedright\arraybackslash}p{(\linewidth - 2\tabcolsep) * \real{0.5517}}
  >{\raggedright\arraybackslash}p{(\linewidth - 2\tabcolsep) * \real{0.4483}}@{}}
\toprule\noalign{}
\begin{minipage}[b]{\linewidth}\raggedright
Key Constraint
\end{minipage} & \begin{minipage}[b]{\linewidth}\raggedright
Description
\end{minipage} \\
\midrule\noalign{}
\endhead
\bottomrule\noalign{}
\endlastfoot
\textbf{Primary Key} & Uniquely identifies each entity in an entity
set \\
\textbf{Candidate Key} & Any attribute that could serve as a primary
key \\
\textbf{Foreign Key} & References primary key of another entity set \\
\textbf{Super Key} & Any set of attributes that uniquely identifies an
entity \\
\end{longtable}
}

\includegraphics[width=1\linewidth,height=\textheight,keepaspectratio]{mermaid-a5bfa4e6.pdf}

\end{solutionbox}
\begin{mnemonicbox}
``PCFS: Primary Candidates Find Superkeys''

\end{mnemonicbox}
\subsection*{Question 2(c) [7 marks]}\label{q2c}

\textbf{Construct an E-R diagram for banking management system.}

\begin{solutionbox}

\includegraphics[width=1\linewidth,height=\textheight,keepaspectratio]{mermaid-4dbcb4e7.pdf}

\textbf{Key Entities and Relationships}:

\begin{itemize}
\tightlist
\item
  \textbf{Customer}: Stores customer information
\item
  \textbf{Account}: Different account types (savings, checking)
\item
  \textbf{Transaction}: Records deposits, withdrawals
\item
  \textbf{Branch}: Different bank locations
\item
  \textbf{Relationships}: Customers have accounts, accounts have
  transactions, branches manage accounts
\end{itemize}

\end{solutionbox}
\begin{mnemonicbox}
``CATB: Customers Access Transactions at Branches''

\end{mnemonicbox}
\subsection*{Question 2(a) OR [3
marks]}\label{q2a}

\textbf{Explain specialization v/s generalization with suitable
examples}

\begin{solutionbox}

{\def\LTcaptype{none} % do not increment counter
\begin{longtable}[]{@{}
  >{\raggedright\arraybackslash}p{(\linewidth - 6\tabcolsep) * \real{0.2143}}
  >{\raggedright\arraybackslash}p{(\linewidth - 6\tabcolsep) * \real{0.2619}}
  >{\raggedright\arraybackslash}p{(\linewidth - 6\tabcolsep) * \real{0.3095}}
  >{\raggedright\arraybackslash}p{(\linewidth - 6\tabcolsep) * \real{0.2143}}@{}}
\toprule\noalign{}
\begin{minipage}[b]{\linewidth}\raggedright
Concept
\end{minipage} & \begin{minipage}[b]{\linewidth}\raggedright
Direction
\end{minipage} & \begin{minipage}[b]{\linewidth}\raggedright
Description
\end{minipage} & \begin{minipage}[b]{\linewidth}\raggedright
Example
\end{minipage} \\
\midrule\noalign{}
\endhead
\bottomrule\noalign{}
\endlastfoot
\textbf{Specialization} & Top-down & Breaking a general entity into more
specific sub-entities & Person \rightarrow Student, Employee \\
\textbf{Generalization} & Bottom-up & Combining similar entities into a
higher-level entity & Car, Truck \rightarrow Vehicle \\
\end{longtable}
}

\includegraphics[width=1\linewidth,height=\textheight,keepaspectratio]{mermaid-74ea43b3.pdf}

\end{solutionbox}
\begin{mnemonicbox}
``SG-TD-BU: Specialization Goes Top-Down,
Generalization Builds Up''

\end{mnemonicbox}
\subsection*{Question 2(b) OR [4
marks]}\label{q2b}

\textbf{Define Chasp trap. Explain when it occurs. Explain the solution
for Chasp trap}

\begin{solutionbox}

\textbf{Chasp trap}: A problem that occurs in ER diagrams when there are
multiple paths between entities, causing ambiguity in relationship
interpretations.

{\def\LTcaptype{none} % do not increment counter
\begin{longtable}[]{@{}
  >{\raggedright\arraybackslash}p{(\linewidth - 2\tabcolsep) * \real{0.3810}}
  >{\raggedright\arraybackslash}p{(\linewidth - 2\tabcolsep) * \real{0.6190}}@{}}
\toprule\noalign{}
\begin{minipage}[b]{\linewidth}\raggedright
Aspect
\end{minipage} & \begin{minipage}[b]{\linewidth}\raggedright
Description
\end{minipage} \\
\midrule\noalign{}
\endhead
\bottomrule\noalign{}
\endlastfoot
\textbf{Occurrence} & When there are two or more distinct paths between
entity types creating a cycle \\
\textbf{Problem} & Leads to incorrect or ambiguous query results \\
\textbf{Solution} & Break one of the relationships or add constraints to
clarify the intended path \\
\end{longtable}
}

\includegraphics[width=1\linewidth,height=\textheight,keepaspectratio]{mermaid-fd332452.pdf}

\end{solutionbox}
\begin{mnemonicbox}
``COP: Cycles Of Paths need breaking''

\end{mnemonicbox}
\subsection*{Question 2(c) OR [7
marks]}\label{q2c}

\textbf{Construct an E-R diagram for college management system.}

\begin{solutionbox}

\includegraphics[width=1\linewidth,height=\textheight,keepaspectratio]{mermaid-430713cc.pdf}

\textbf{Key Entities and Relationships}:

\begin{itemize}
\tightlist
\item
  \textbf{Student}: Stores student details
\item
  \textbf{Department}: Academic divisions
\item
  \textbf{Faculty}: Teachers and professors
\item
  \textbf{Course}: Subjects taught
\item
  \textbf{Exam}: Evaluation events
\item
  \textbf{Relationships}: Students enroll in courses, faculty teach
  courses, departments offer courses
\end{itemize}

\end{solutionbox}
\begin{mnemonicbox}
``SDFCE: Students Delight Faculty by Completing
Exams''

\end{mnemonicbox}
\subsection*{Question 3(a) [3 marks]}\label{q3a}

\textbf{Explain GROUP BY clause with example.}

\begin{solutionbox}

\textbf{GROUP BY} clause groups rows that have the same values into
summary rows.

{\def\LTcaptype{none} % do not increment counter
\begin{longtable}[]{@{}
  >{\raggedright\arraybackslash}p{(\linewidth - 2\tabcolsep) * \real{0.4091}}
  >{\raggedright\arraybackslash}p{(\linewidth - 2\tabcolsep) * \real{0.5909}}@{}}
\toprule\noalign{}
\begin{minipage}[b]{\linewidth}\raggedright
Feature
\end{minipage} & \begin{minipage}[b]{\linewidth}\raggedright
Description
\end{minipage} \\
\midrule\noalign{}
\endhead
\bottomrule\noalign{}
\endlastfoot
\textbf{Purpose} & Arranges identical data into groups for aggregate
functions \\
\textbf{Usage} & Used with aggregate functions (COUNT, SUM, AVG, MAX,
MIN) \\
\textbf{Syntax} & SELECT column1, COUNT(*) FROM table GROUP BY
column1; \\
\end{longtable}
}

\begin{lstlisting}[language=SQL]
SELECT department, AVG(salary) 
FROM employees
GROUP BY department;
\end{lstlisting}

\end{solutionbox}
\begin{mnemonicbox}
``GAS: Group And Summarize''

\end{mnemonicbox}
\subsection*{Question 3(b) [4 marks]}\label{q3b}

\textbf{List Data Definition Language (DDL) commands. Explain any two
DDL commands with examples.}

\begin{solutionbox}

\textbf{DDL Commands}: CREATE, ALTER, DROP, TRUNCATE, RENAME

{\def\LTcaptype{none} % do not increment counter
\begin{longtable}[]{@{}
  >{\raggedright\arraybackslash}p{(\linewidth - 4\tabcolsep) * \real{0.2903}}
  >{\raggedright\arraybackslash}p{(\linewidth - 4\tabcolsep) * \real{0.4194}}
  >{\raggedright\arraybackslash}p{(\linewidth - 4\tabcolsep) * \real{0.2903}}@{}}
\toprule\noalign{}
\begin{minipage}[b]{\linewidth}\raggedright
Command
\end{minipage} & \begin{minipage}[b]{\linewidth}\raggedright
Description
\end{minipage} & \begin{minipage}[b]{\linewidth}\raggedright
Example
\end{minipage} \\
\midrule\noalign{}
\endhead
\bottomrule\noalign{}
\endlastfoot
\textbf{CREATE} & Creates database objects like tables, views, indexes &
\passthrough{\lstinline!CREATE TABLE students (id INT PRIMARY KEY, name VARCHAR(50));!} \\
\textbf{ALTER} & Modifies existing database objects &
\passthrough{\lstinline!ALTER TABLE students ADD COLUMN email VARCHAR(100);!} \\
\textbf{DROP} & Removes database objects &
\passthrough{\lstinline!DROP TABLE students;!} \\
\textbf{TRUNCATE} & Removes all records from a table &
\passthrough{\lstinline!TRUNCATE TABLE students;!} \\
\end{longtable}
}

\end{solutionbox}
\begin{mnemonicbox}
``CADTR: Create, Alter, Drop, Truncate, Rename''

\end{mnemonicbox}
\subsection*{Question 3(c) [7 marks]}\label{q3c}

\textbf{Perform the following Query on the ``Students'' table having the
field's enr\_no, name, percent, branch in SQL.}

\begin{solutionbox}

\begin{lstlisting}[language=SQL]
-- 1. Display all records in Students table
SELECT * FROM Students;

-- 2. Display only branch without duplicate value
SELECT DISTINCT branch FROM Students;

-- 3. Display all records sorted in descending order of name
SELECT * FROM Students ORDER BY name DESC;

-- 4. Add one new column to store address, named "address"
ALTER TABLE Students ADD address VARCHAR(100);

-- 5. Display all students belongs to branch "ICT"
SELECT * FROM Students WHERE branch = 'ICT';

-- 6. Delete all students having percent less than 60
DELETE FROM Students WHERE percent < 60;

-- 7. Display the students names starts with "S"
SELECT * FROM Students WHERE name LIKE 'S%';
\end{lstlisting}

{\def\LTcaptype{none} % do not increment counter
\begin{longtable}[]{@{}ll@{}}
\toprule\noalign{}
Query & Purpose \\
\midrule\noalign{}
\endhead
\bottomrule\noalign{}
\endlastfoot
\textbf{SELECT} & Retrieves data from tables \\
\textbf{DISTINCT} & Eliminates duplicate values \\
\textbf{ORDER BY} & Sorts results in specified order \\
\textbf{ALTER TABLE} & Modifies table structure \\
\textbf{WHERE} & Filters records based on conditions \\
\textbf{DELETE} & Removes records matching conditions \\
\textbf{LIKE} & Pattern matching in string comparison \\
\end{longtable}
}

\end{solutionbox}
\begin{mnemonicbox}
``SDOAWDL: Select Distinct Order Alter Where Delete
Like''

\end{mnemonicbox}
\subsection*{Question 3(a) OR [3
marks]}\label{q3a}

\textbf{Explain GRANT command with syntax and example.}

\begin{solutionbox}

\textbf{GRANT} command gives specific privileges to users on database
objects.

{\def\LTcaptype{none} % do not increment counter
\begin{longtable}[]{@{}
  >{\raggedright\arraybackslash}p{(\linewidth - 2\tabcolsep) * \real{0.4583}}
  >{\raggedright\arraybackslash}p{(\linewidth - 2\tabcolsep) * \real{0.5417}}@{}}
\toprule\noalign{}
\begin{minipage}[b]{\linewidth}\raggedright
Component
\end{minipage} & \begin{minipage}[b]{\linewidth}\raggedright
Description
\end{minipage} \\
\midrule\noalign{}
\endhead
\bottomrule\noalign{}
\endlastfoot
\textbf{Syntax} &
\passthrough{\lstinline!GRANT privilege(s) ON object TO user [WITH GRANT OPTION];!} \\
\textbf{Privileges} & SELECT, INSERT, UPDATE, DELETE, ALL PRIVILEGES \\
\textbf{Objects} & Tables, views, sequences, etc. \\
\end{longtable}
}

\begin{lstlisting}[language=SQL]
GRANT SELECT, UPDATE ON employees TO user1;
GRANT ALL PRIVILEGES ON database_name.* TO user2 WITH GRANT OPTION;
\end{lstlisting}

\end{solutionbox}
\begin{mnemonicbox}
``GPO: Grant Privileges to Others''

\end{mnemonicbox}
\subsection*{Question 3(b) OR [4
marks]}\label{q3b}

\textbf{Compare Truncate command and Drop command.}

\begin{solutionbox}

{\def\LTcaptype{none} % do not increment counter
\begin{longtable}[]{@{}
  >{\raggedright\arraybackslash}p{(\linewidth - 4\tabcolsep) * \real{0.3600}}
  >{\raggedright\arraybackslash}p{(\linewidth - 4\tabcolsep) * \real{0.4000}}
  >{\raggedright\arraybackslash}p{(\linewidth - 4\tabcolsep) * \real{0.2400}}@{}}
\toprule\noalign{}
\begin{minipage}[b]{\linewidth}\raggedright
Feature
\end{minipage} & \begin{minipage}[b]{\linewidth}\raggedright
TRUNCATE
\end{minipage} & \begin{minipage}[b]{\linewidth}\raggedright
DROP
\end{minipage} \\
\midrule\noalign{}
\endhead
\bottomrule\noalign{}
\endlastfoot
\textbf{Purpose} & Removes all rows from table & Removes entire table
structure \\
\textbf{Structure} & Keeps table structure intact & Deletes table
definition completely \\
\textbf{Recovery} & Cannot be easily rolled back & Can be recovered
until committed \\
\textbf{Speed} & Faster than DELETE & Quick operation \\
\textbf{Triggers} & Does not activate triggers & Does not activate
triggers \\
\end{longtable}
}

\begin{lstlisting}[language=SQL]
-- Truncate example
TRUNCATE TABLE students;

-- Drop example
DROP TABLE students;
\end{lstlisting}

\end{solutionbox}
\begin{mnemonicbox}
``TRC-DST: Truncate Removes Contents, Drop Destroys
Structure Totally''

\end{mnemonicbox}
\subsection*{Question 3(c) OR [7
marks]}\label{q3c}

\textbf{Write the Output of Following Query.}

\begin{solutionbox}

{\def\LTcaptype{none} % do not increment counter
\begin{longtable}[]{@{}
  >{\raggedright\arraybackslash}p{(\linewidth - 4\tabcolsep) * \real{0.2500}}
  >{\raggedright\arraybackslash}p{(\linewidth - 4\tabcolsep) * \real{0.2857}}
  >{\raggedright\arraybackslash}p{(\linewidth - 4\tabcolsep) * \real{0.4643}}@{}}
\toprule\noalign{}
\begin{minipage}[b]{\linewidth}\raggedright
Query
\end{minipage} & \begin{minipage}[b]{\linewidth}\raggedright
Output
\end{minipage} & \begin{minipage}[b]{\linewidth}\raggedright
Explanation
\end{minipage} \\
\midrule\noalign{}
\endhead
\bottomrule\noalign{}
\endlastfoot
\textbf{ABS(-23), ABS(49)} & 23, 49 & Returns absolute value \\
\textbf{SQRT(25), SQRT(81)} & 5, 9 & Returns square root \\
\textbf{POWER(3,2), POWER(-2,3)} & 9, -8 & Returns x\^{}y (first value
raised to power of second) \\
\textbf{MOD(15,4), MOD(21,3)} & 3, 0 & Returns remainder after
division \\
\textbf{ROUND(123.446,1), ROUND(123.456,2)} & 123.4, 123.46 & Rounds to
specified decimal places \\
\textbf{CEIL(234.45), CEIL(-234.45)} & 235, -234 & Rounds up to nearest
integer \\
\textbf{FLOOR(-12.7), FLOOR(12.7)} & -13, 12 & Rounds down to nearest
integer \\
\end{longtable}
}

\begin{lstlisting}[language=SQL]
SELECT ABS(-23), ABS(49);          -- 23, 49
SELECT SQRT(25), SQRT(81);         -- 5, 9
SELECT POWER(3,2), POWER(-2,3);    -- 9, -8
SELECT MOD(15,4), MOD(21,3);       -- 3, 0
SELECT ROUND(123.446,1), ROUND(123.456,2); -- 123.4, 123.46
SELECT CEIL(234.45), CEIL(-234.45);  -- 235, -234
SELECT FLOOR(-12.7), FLOOR(12.7);    -- -13, 12
\end{lstlisting}

\end{solutionbox}
\begin{mnemonicbox}
``ASPMRCF: Absolute Square Power Modulo Round Ceiling
Floor''

\end{mnemonicbox}
\subsection*{Question 4(a) [3 marks]}\label{q4a}

\textbf{List data types in SQL. Explain any two data types with
example.}

\begin{solutionbox}

\textbf{SQL Data Types}: INTEGER, FLOAT, VARCHAR, CHAR, DATE, DATETIME,
BOOLEAN, BLOB

{\def\LTcaptype{none} % do not increment counter
\begin{longtable}[]{@{}
  >{\raggedright\arraybackslash}p{(\linewidth - 4\tabcolsep) * \real{0.3333}}
  >{\raggedright\arraybackslash}p{(\linewidth - 4\tabcolsep) * \real{0.3939}}
  >{\raggedright\arraybackslash}p{(\linewidth - 4\tabcolsep) * \real{0.2727}}@{}}
\toprule\noalign{}
\begin{minipage}[b]{\linewidth}\raggedright
Data Type
\end{minipage} & \begin{minipage}[b]{\linewidth}\raggedright
Description
\end{minipage} & \begin{minipage}[b]{\linewidth}\raggedright
Example
\end{minipage} \\
\midrule\noalign{}
\endhead
\bottomrule\noalign{}
\endlastfoot
\textbf{INTEGER} & Whole numbers without decimal points &
\passthrough{\lstinline!id INTEGER = 101!} \\
\textbf{VARCHAR} & Variable-length character string &
\passthrough{\lstinline!name VARCHAR(50) = 'John'!} \\
\textbf{DATE} & Stores date values (YYYY-MM-DD) &
\passthrough{\lstinline!birth\_date DATE = '2000-05-15'!} \\
\textbf{FLOAT} & Decimal numbers with floating point &
\passthrough{\lstinline!salary FLOAT = 45000.50!} \\
\end{longtable}
}

\begin{lstlisting}[language=SQL]
CREATE TABLE employees (
    id INTEGER,
    name VARCHAR(50),
    salary FLOAT
);
\end{lstlisting}

\end{solutionbox}
\begin{mnemonicbox}
``IVDB: Integers \& Varchars are Database Basics''

\end{mnemonicbox}
\subsection*{Question 4(b) [4 marks]}\label{q4b}

\textbf{Explain Full function dependency with example.}

\begin{solutionbox}

\textbf{Full Function Dependency}: When Y is functionally dependent on
X, but not on any subset of X.

{\def\LTcaptype{none} % do not increment counter
\begin{longtable}[]{@{}
  >{\raggedright\arraybackslash}p{(\linewidth - 4\tabcolsep) * \real{0.2903}}
  >{\raggedright\arraybackslash}p{(\linewidth - 4\tabcolsep) * \real{0.4194}}
  >{\raggedright\arraybackslash}p{(\linewidth - 4\tabcolsep) * \real{0.2903}}@{}}
\toprule\noalign{}
\begin{minipage}[b]{\linewidth}\raggedright
Concept
\end{minipage} & \begin{minipage}[b]{\linewidth}\raggedright
Description
\end{minipage} & \begin{minipage}[b]{\linewidth}\raggedright
Example
\end{minipage} \\
\midrule\noalign{}
\endhead
\bottomrule\noalign{}
\endlastfoot
\textbf{Definition} & Attribute B is fully functionally dependent on A
if B depends on all of A & Student\_ID \rightarrow Name (full dependency) \\
\textbf{Non-example} & When attribute depends only on part of composite
key & \{Student\_ID, Course\_ID\} \rightarrow Student\_Name (partial) \\
\end{longtable}
}

\includegraphics[width=1\linewidth,height=\textheight,keepaspectratio]{mermaid-33201882.pdf}

\end{solutionbox}
\begin{mnemonicbox}
``FFD: Full, not Fraction of Dependency''

\end{mnemonicbox}
\subsection*{Question 4(c) [7 marks]}\label{q4c}

\textbf{Define normalization. Explain 2NF (Second Normal Form) with
example and solution.}

\begin{solutionbox}

\textbf{Normalization}: Process of organizing database to minimize
redundancy and dependency by dividing large tables into smaller tables
and defining relationships between them.

\textbf{2NF (Second Normal Form)}:

\begin{itemize}
\tightlist
\item
  A table is in 2NF if it is in 1NF and no non-prime attribute is
  dependent on any proper subset of candidate key.
\end{itemize}

{\def\LTcaptype{none} % do not increment counter
\begin{longtable}[]{@{}
  >{\raggedright\arraybackslash}p{(\linewidth - 2\tabcolsep) * \real{0.5714}}
  >{\raggedright\arraybackslash}p{(\linewidth - 2\tabcolsep) * \real{0.4286}}@{}}
\toprule\noalign{}
\begin{minipage}[b]{\linewidth}\raggedright
Before 2NF
\end{minipage} & \begin{minipage}[b]{\linewidth}\raggedright
Problem
\end{minipage} \\
\midrule\noalign{}
\endhead
\bottomrule\noalign{}
\endlastfoot
\textbf{Order(Order\_ID, Product\_ID, Product\_Name, Quantity, Price)} &
Product\_Name depends on only Product\_ID, not full key \\
\end{longtable}
}

{\def\LTcaptype{none} % do not increment counter
\begin{longtable}[]{@{}
  >{\raggedright\arraybackslash}p{(\linewidth - 2\tabcolsep) * \real{0.5238}}
  >{\raggedright\arraybackslash}p{(\linewidth - 2\tabcolsep) * \real{0.4762}}@{}}
\toprule\noalign{}
\begin{minipage}[b]{\linewidth}\raggedright
After 2NF
\end{minipage} & \begin{minipage}[b]{\linewidth}\raggedright
Solution
\end{minipage} \\
\midrule\noalign{}
\endhead
\bottomrule\noalign{}
\endlastfoot
\textbf{Order(Order\_ID, Product\_ID, Quantity)} & Only full key
dependencies \\
\textbf{Product(Product\_ID, Product\_Name, Price)} & Product details
depend only on Product\_ID \\
\end{longtable}
}

\includegraphics[width=1\linewidth,height=\textheight,keepaspectratio]{mermaid-920503d7.pdf}

\end{solutionbox}
\begin{mnemonicbox}
``2NF-PPD: Partial dependency Problems Divided''

\end{mnemonicbox}
\subsection*{Question 4(a) OR [3
marks]}\label{q4a}

\textbf{Explain commands: 1) To\_Number() 2) To\_Char()}

\begin{solutionbox}

{\def\LTcaptype{none} % do not increment counter
\begin{longtable}[]{@{}
  >{\raggedright\arraybackslash}p{(\linewidth - 6\tabcolsep) * \real{0.2778}}
  >{\raggedright\arraybackslash}p{(\linewidth - 6\tabcolsep) * \real{0.2500}}
  >{\raggedright\arraybackslash}p{(\linewidth - 6\tabcolsep) * \real{0.2222}}
  >{\raggedright\arraybackslash}p{(\linewidth - 6\tabcolsep) * \real{0.2500}}@{}}
\toprule\noalign{}
\begin{minipage}[b]{\linewidth}\raggedright
Function
\end{minipage} & \begin{minipage}[b]{\linewidth}\raggedright
Purpose
\end{minipage} & \begin{minipage}[b]{\linewidth}\raggedright
Syntax
\end{minipage} & \begin{minipage}[b]{\linewidth}\raggedright
Example
\end{minipage} \\
\midrule\noalign{}
\endhead
\bottomrule\noalign{}
\endlastfoot
\textbf{TO\_NUMBER()} & Converts string to number &
\passthrough{\lstinline!TO\_NUMBER(string, [format])!} &
\passthrough{\lstinline!TO\_NUMBER('123.45') = 123.45!} \\
\textbf{TO\_CHAR()} & Converts number/date to string &
\passthrough{\lstinline!TO\_CHAR(value, [format])!} &
\passthrough{\lstinline!TO\_CHAR(1234, '9999') = '1234'!} \\
\end{longtable}
}

\begin{lstlisting}[language=SQL]
-- Convert string to number
SELECT TO_NUMBER('123.45') FROM dual;  -- 123.45

-- Convert date to formatted string
SELECT TO_CHAR(SYSDATE, 'DD-MON-YYYY') FROM dual;  -- 20-JAN-2024

-- Convert number to formatted string
SELECT TO_CHAR(1234.56, '$9,999.99') FROM dual;  -- $1,234.56
\end{lstlisting}

\end{solutionbox}
\begin{mnemonicbox}
``NC: Numbers and Characters conversion''

\end{mnemonicbox}
\subsection*{Question 4(b) OR [4
marks]}\label{q4b}

\textbf{Explain 1NF (First Normal Form) with example and solution.}

\begin{solutionbox}

\textbf{1NF (First Normal Form)}: A relation is in 1NF if it contains no
repeating groups or arrays.

{\def\LTcaptype{none} % do not increment counter
\begin{longtable}[]{@{}
  >{\raggedright\arraybackslash}p{(\linewidth - 2\tabcolsep) * \real{0.5714}}
  >{\raggedright\arraybackslash}p{(\linewidth - 2\tabcolsep) * \real{0.4286}}@{}}
\toprule\noalign{}
\begin{minipage}[b]{\linewidth}\raggedright
Before 1NF
\end{minipage} & \begin{minipage}[b]{\linewidth}\raggedright
Problem
\end{minipage} \\
\midrule\noalign{}
\endhead
\bottomrule\noalign{}
\endlastfoot
\textbf{Student(ID, Name, Courses)} & Courses column contains multiple
values \\
\textbf{Example}: (101, John, ``Math,Science,History'') & Multi-valued
attribute \\
\end{longtable}
}

{\def\LTcaptype{none} % do not increment counter
\begin{longtable}[]{@{}
  >{\raggedright\arraybackslash}p{(\linewidth - 2\tabcolsep) * \real{0.5238}}
  >{\raggedright\arraybackslash}p{(\linewidth - 2\tabcolsep) * \real{0.4762}}@{}}
\toprule\noalign{}
\begin{minipage}[b]{\linewidth}\raggedright
After 1NF
\end{minipage} & \begin{minipage}[b]{\linewidth}\raggedright
Solution
\end{minipage} \\
\midrule\noalign{}
\endhead
\bottomrule\noalign{}
\endlastfoot
\textbf{Student(ID, Name, Course)} & One course per row \\
\textbf{Examples}: (101, John, Math), (101, John, Science), (101, John,
History) & Atomic values \\
\end{longtable}
}

\includegraphics[width=1\linewidth,height=\textheight,keepaspectratio]{mermaid-13c28bd6.pdf}

\end{solutionbox}
\begin{mnemonicbox}
``1NF-ARM: Atomic values Remove Multivalues''

\end{mnemonicbox}
\subsection*{Question 4(c) OR [7
marks]}\label{q4c}

\textbf{Explain function dependency in SQL. Explain Partial function
dependency with example.}

\begin{solutionbox}

\textbf{Functional Dependency}: A relationship where one attribute
determines the value of another attribute.

\textbf{Notation}: X \rightarrow Y (X determines Y)

\textbf{Partial Functional Dependency}: When an attribute depends on
only part of a composite primary key.

{\def\LTcaptype{none} % do not increment counter
\begin{longtable}[]{@{}
  >{\raggedright\arraybackslash}p{(\linewidth - 4\tabcolsep) * \real{0.2903}}
  >{\raggedright\arraybackslash}p{(\linewidth - 4\tabcolsep) * \real{0.2903}}
  >{\raggedright\arraybackslash}p{(\linewidth - 4\tabcolsep) * \real{0.4194}}@{}}
\toprule\noalign{}
\begin{minipage}[b]{\linewidth}\raggedright
Concept
\end{minipage} & \begin{minipage}[b]{\linewidth}\raggedright
Example
\end{minipage} & \begin{minipage}[b]{\linewidth}\raggedright
Explanation
\end{minipage} \\
\midrule\noalign{}
\endhead
\bottomrule\noalign{}
\endlastfoot
\textbf{Composite Key} & \{Student\_ID, Course\_ID\} & Together forms
primary key \\
\textbf{Partial Dependency} & \{Student\_ID, Course\_ID\} \rightarrow
Student\_Name & Student\_Name depends only on Student\_ID \\
\textbf{Problem} & Update anomalies, data redundancy & Same student name
repeated for multiple courses \\
\end{longtable}
}

\includegraphics[width=1\linewidth,height=\textheight,keepaspectratio]{mermaid-2f363d62.pdf}

\textbf{Solution}: Decompose into separate tables where each non-key
attribute is fully dependent on the key.

\end{solutionbox}
\begin{mnemonicbox}
``PD-CPK: Partial Dependency - Component of Primary
Key''

\end{mnemonicbox}
\subsection*{Question 5(a) [3 marks]}\label{q5a}

\textbf{Explain the properties of Transaction with example.}

\begin{solutionbox}

\textbf{Transaction Properties} (ACID):

{\def\LTcaptype{none} % do not increment counter
\begin{longtable}[]{@{}
  >{\raggedright\arraybackslash}p{(\linewidth - 4\tabcolsep) * \real{0.3125}}
  >{\raggedright\arraybackslash}p{(\linewidth - 4\tabcolsep) * \real{0.4062}}
  >{\raggedright\arraybackslash}p{(\linewidth - 4\tabcolsep) * \real{0.2812}}@{}}
\toprule\noalign{}
\begin{minipage}[b]{\linewidth}\raggedright
Property
\end{minipage} & \begin{minipage}[b]{\linewidth}\raggedright
Description
\end{minipage} & \begin{minipage}[b]{\linewidth}\raggedright
Example
\end{minipage} \\
\midrule\noalign{}
\endhead
\bottomrule\noalign{}
\endlastfoot
\textbf{Atomicity} & All operations complete successfully or none does &
Bank transfer: debit and credit both happen or neither \\
\textbf{Consistency} & Database remains in valid state before and after
& Account balance constraints remain valid \\
\textbf{Isolation} & Transactions execute as if they were the only one &
Two users updating same record don't interfere \\
\textbf{Durability} & Committed changes survive system failure & Once
confirmed, a deposit remains even after power loss \\
\end{longtable}
}

\includegraphics[width=1\linewidth,height=\textheight,keepaspectratio]{mermaid-0dda1ae0.pdf}

\end{solutionbox}
\begin{mnemonicbox}
``ACID: Atomicity, Consistency, Isolation,
Durability''

\end{mnemonicbox}
\subsection*{Question 5(b) [4 marks]}\label{q5b}

\textbf{Write the Queries using set operators to find following using
given ``Student'' and ``CR'' (Class Representative) tables.}

\begin{solutionbox}

\begin{lstlisting}[language=SQL]
-- 1. List the name of the persons who are either a student or a CR
SELECT Stnd_Name FROM Student
UNION
SELECT CR_Name FROM CR;

-- 2. List the name of the persons who are a student as well as a CR
SELECT Stnd_Name FROM Student
INTERSECT
SELECT CR_Name FROM CR;

-- 3. List the name of the persons who are only a student and not a CR
SELECT Stnd_Name FROM Student
MINUS
SELECT CR_Name FROM CR;

-- 4. List the name of the persons who are only a CR and not a student
SELECT CR_Name FROM CR
MINUS
SELECT Stnd_Name FROM Student;
\end{lstlisting}

{\def\LTcaptype{none} % do not increment counter
\begin{longtable}[]{@{}
  >{\raggedright\arraybackslash}p{(\linewidth - 4\tabcolsep) * \real{0.3333}}
  >{\raggedright\arraybackslash}p{(\linewidth - 4\tabcolsep) * \real{0.2143}}
  >{\raggedright\arraybackslash}p{(\linewidth - 4\tabcolsep) * \real{0.4524}}@{}}
\toprule\noalign{}
\begin{minipage}[b]{\linewidth}\raggedright
Set Operator
\end{minipage} & \begin{minipage}[b]{\linewidth}\raggedright
Purpose
\end{minipage} & \begin{minipage}[b]{\linewidth}\raggedright
Result for Example
\end{minipage} \\
\midrule\noalign{}
\endhead
\bottomrule\noalign{}
\endlastfoot
\textbf{UNION} & Combines all distinct rows & Manoj, Rahil, Jiya, Rina,
Jitesh, Priya \\
\textbf{INTERSECT} & Returns only common rows & Manoj, Rina \\
\textbf{MINUS} & Returns rows in first set but not second & Rahil,
Jiya \\
\textbf{MINUS (reversed)} & Returns rows in second set but not first &
Jitesh, Priya \\
\end{longtable}
}

\end{solutionbox}
\begin{mnemonicbox}
``UIMD: Union Includes, Minus Divides''

\end{mnemonicbox}
\subsection*{Question 5(c) [7 marks]}\label{q5c}

\textbf{Explain Conflict Serializability in detail.}

\begin{solutionbox}

\textbf{Conflict Serializability}: A schedule is conflict serializable
if it can be transformed into a serial schedule by swapping
non-conflicting operations.

{\def\LTcaptype{none} % do not increment counter
\begin{longtable}[]{@{}
  >{\raggedright\arraybackslash}p{(\linewidth - 2\tabcolsep) * \real{0.5185}}
  >{\raggedright\arraybackslash}p{(\linewidth - 2\tabcolsep) * \real{0.4815}}@{}}
\toprule\noalign{}
\begin{minipage}[b]{\linewidth}\raggedright
Key Concepts
\end{minipage} & \begin{minipage}[b]{\linewidth}\raggedright
Description
\end{minipage} \\
\midrule\noalign{}
\endhead
\bottomrule\noalign{}
\endlastfoot
\textbf{Conflict operations} & Two operations conflict if they access
same data item and at least one is write \\
\textbf{Precedence graph} & Directed graph showing conflicts between
transactions \\
\textbf{Serializable} & If precedence graph has no cycles, schedule is
conflict serializable \\
\end{longtable}
}

\includegraphics[width=1\linewidth,height=\textheight,keepaspectratio]{mermaid-1d907cd1.pdf}

\textbf{Example}:

\begin{itemize}
\tightlist
\item
  T1: R(X), W(X)
\item
  T2: R(X), W(X)
\end{itemize}

\textbf{Serializable schedules}:

\begin{itemize}
\tightlist
\item
  T1 followed by T2: R1(X), W1(X), R2(X), W2(X)
\item
  T2 followed by T1: R2(X), W2(X), R1(X), W1(X)
\end{itemize}

\textbf{Non-serializable}: R1(X), R2(X), W1(X), W2(X) - Creates cycle in
precedence graph

\end{solutionbox}
\begin{mnemonicbox}
``COPS: Conflict Operations Produce Serializability''

\end{mnemonicbox}
\subsection*{Question 5(a) OR [3
marks]}\label{q5a}

\textbf{Explain the concept of Transaction with example.}

\begin{solutionbox}

\textbf{Transaction}: A logical unit of work that must be either
completely performed or completely undone.

{\def\LTcaptype{none} % do not increment counter
\begin{longtable}[]{@{}
  >{\raggedright\arraybackslash}p{(\linewidth - 4\tabcolsep) * \real{0.4634}}
  >{\raggedright\arraybackslash}p{(\linewidth - 4\tabcolsep) * \real{0.3171}}
  >{\raggedright\arraybackslash}p{(\linewidth - 4\tabcolsep) * \real{0.2195}}@{}}
\toprule\noalign{}
\begin{minipage}[b]{\linewidth}\raggedright
Transaction Phases
\end{minipage} & \begin{minipage}[b]{\linewidth}\raggedright
Description
\end{minipage} & \begin{minipage}[b]{\linewidth}\raggedright
Example
\end{minipage} \\
\midrule\noalign{}
\endhead
\bottomrule\noalign{}
\endlastfoot
\textbf{BEGIN} & Marks start of transaction & START TRANSACTION \\
\textbf{Execute operations} & Database operations (read/write) & UPDATE
account SET balance = balance - 1000 WHERE id = 123 \\
\textbf{COMMIT/ROLLBACK} & End transaction with success/failure & COMMIT
or ROLLBACK \\
\end{longtable}
}

\includegraphics[width=1\linewidth,height=\textheight,keepaspectratio]{mermaid-ccc5a2d0.pdf}

\textbf{Example}:

\begin{lstlisting}[language=SQL]
BEGIN TRANSACTION;
UPDATE accounts SET balance = balance - 1000 WHERE acc_no = 123;
UPDATE accounts SET balance = balance + 1000 WHERE acc_no = 456;
COMMIT;
\end{lstlisting}

\end{solutionbox}
\begin{mnemonicbox}
``BEC: Begin, Execute, Commit''

\end{mnemonicbox}
\subsection*{Question 5(b) OR [4
marks]}\label{q5b}

\textbf{Explain equi-join with syntax and example.}

\begin{solutionbox}

\textbf{Equi-join}: A join operation that uses equality comparison
operator.

{\def\LTcaptype{none} % do not increment counter
\begin{longtable}[]{@{}
  >{\raggedright\arraybackslash}p{(\linewidth - 2\tabcolsep) * \real{0.4091}}
  >{\raggedright\arraybackslash}p{(\linewidth - 2\tabcolsep) * \real{0.5909}}@{}}
\toprule\noalign{}
\begin{minipage}[b]{\linewidth}\raggedright
Feature
\end{minipage} & \begin{minipage}[b]{\linewidth}\raggedright
Description
\end{minipage} \\
\midrule\noalign{}
\endhead
\bottomrule\noalign{}
\endlastfoot
\textbf{Syntax} &
\passthrough{\lstinline!SELECT columns FROM table1, table2 WHERE table1.column = table2.column;!} \\
\textbf{Purpose} & Combines rows from two tables based on matching
column values \\
\textbf{Alternative} &
\passthrough{\lstinline!SELECT columns FROM table1 INNER JOIN table2 ON table1.column = table2.column;!} \\
\end{longtable}
}

\begin{lstlisting}[language=SQL]
-- Traditional syntax
SELECT s.name, d.dept_name 
FROM students s, departments d 
WHERE s.dept_id = d.dept_id;

-- INNER JOIN syntax
SELECT s.name, d.dept_name 
FROM students s INNER JOIN departments d 
ON s.dept_id = d.dept_id;
\end{lstlisting}

\end{solutionbox}
\begin{mnemonicbox}
``EQ-ME: Equality Matches Entries''

\end{mnemonicbox}
\subsection*{Question 5(c) OR [7
marks]}\label{q5c}

\textbf{Explain View Serializability in detail.}

\begin{solutionbox}

\textbf{View Serializability}: A schedule is view serializable if it is
view equivalent to some serial schedule.

{\def\LTcaptype{none} % do not increment counter
\begin{longtable}[]{@{}
  >{\raggedright\arraybackslash}p{(\linewidth - 2\tabcolsep) * \real{0.4583}}
  >{\raggedright\arraybackslash}p{(\linewidth - 2\tabcolsep) * \real{0.5417}}@{}}
\toprule\noalign{}
\begin{minipage}[b]{\linewidth}\raggedright
Condition
\end{minipage} & \begin{minipage}[b]{\linewidth}\raggedright
Description
\end{minipage} \\
\midrule\noalign{}
\endhead
\bottomrule\noalign{}
\endlastfoot
\textbf{Initial read} & If T1 reads initial value of data item X in
schedule S, it must also read initial value in schedule S' \\
\textbf{Final write} & If T1 performs final write of data item X in S,
it must also perform final write in S' \\
\textbf{Dependency preservation} & If T1 reads value of X written by T2
in S, it must also read from T2 in S' \\
\end{longtable}
}

\includegraphics[width=1\linewidth,height=\textheight,keepaspectratio]{mermaid-eb27c0a0.pdf}

\textbf{Comparison}:

\begin{itemize}
\tightlist
\item
  \textbf{Conflict serializability}: More restrictive, easier to test
  (precedence graph)
\item
  \textbf{View serializability}: More general, harder to test
  (NP-complete)
\end{itemize}

\textbf{Example of view serializable but not conflict serializable}:

\begin{itemize}
\tightlist
\item
  T1: W(X)
\item
  T2: W(X)
\item
  T3: R(X)
\item
  Schedule: W1(X), W2(X), R3(X) - View equivalent to serial schedule
  T2,T1,T3
\end{itemize}

\end{solutionbox}
\begin{mnemonicbox}
``VIR-FF: View preserves Initial Reads and Final
writes''

\end{mnemonicbox}

\end{document}
