\documentclass{article}

% content/resources/templates/preamble.tex
\usepackage[margin=0.6in]{geometry}
\author{Milav Dabgar}
\usepackage{amsmath,amssymb,amsthm}
\usepackage{booktabs}
\usepackage{multirow}
\usepackage{xcolor}
\usepackage{tcolorbox}
\tcbuselibrary{breakable,skins}
\usepackage[colorlinks=true,linkcolor=blue]{hyperref}
\usepackage{titlesec}
\usepackage{enumitem}
\usepackage{tikz}
\usepackage{pgfplots}
\usepackage{circuitikz}
\usepackage[version=4]{mhchem}
\usepackage{longtable}
\usepackage{array}
\usepackage{float}
\usepackage{caption}
\usepackage{listings}

\lstset{
  basicstyle=\small\ttfamily,
  breaklines=true,
  breakatwhitespace=false,
  postbreak=\mbox{\textcolor{red}{$\hookrightarrow$}\space},
  float=false,
  numbers=left,
  numberstyle=\tiny\color{gray},
  numbersep=10pt,
  xleftmargin=2em,
  keywordstyle=\color{blue},
  commentstyle=\color{green!60!black},
  stringstyle=\color{purple},
  backgroundcolor=\color{gray!5},
  showstringspaces=false,
  tabsize=2,
  captionpos=b,
  keepspaces=true,
  columns=flexible
}

\pgfplotsset{compat=1.18}
\usetikzlibrary{shapes,arrows,positioning,calc,patterns,decorations.pathmorphing,decorations.markings,arrows.meta}

% Color scheme
\definecolor{headcolor}{RGB}{0,102,204}
\definecolor{keycolor}{RGB}{220,20,60}
\definecolor{solutioncolor}{RGB}{34,139,34}
\definecolor{mnemoniccolor}{RGB}{148,0,211}
\definecolor{codecolor}{RGB}{0,0,100}

% Spacing
\setlength{\parskip}{3pt}
\setlist[itemize]{nosep}
\setlist[enumerate]{nosep}

% Title formatting
\titleformat{\section}{\Large\bfseries\color{headcolor}}{\thesection}{1em}{}
\titleformat{\subsection}{\large\bfseries\color{headcolor}}{\thesubsection}{1em}{}

% Pandoc tightlist compatibility
\providecommand{\tightlist}{%
  \setlength{\itemsep}{0pt}\setlength{\parskip}{0pt}}

% Pandoc longtable compatibility
\newcounter{none}
\def\thenone{}


% content/resources/templates/gujarati-boxes.tex
\usepackage{fontspec}
\usepackage{polyglossia}

% Set Gujarati as main language (document is primarily in Gujarati)
% Note: gloss-gujarati.ldf doesn't exist in polyglossia, but it will use hyphenation patterns
\setdefaultlanguage{gujarati}
\setotherlanguage{english}

% Configure Gujarati font properly
% Use Language=Default to prevent polyglossia from trying to add language-specific features
% that don't exist for Gujarati, which causes "empty feature" warnings
\newfontfamily\gujaratifont[Script=Gujarati,AutoFakeBold=2.5,AutoFakeSlant=0.3]{Noto Sans Gujarati}
\setmainfont[Script=Gujarati,AutoFakeBold=2.5,AutoFakeSlant=0.3]{Noto Sans Gujarati}
% Use Noto Sans Gujarati for monospace to support Gujarati in text
\setmonofont[Scale=0.9]{Noto Sans Gujarati}

% Configure English to use the same font
\newfontfamily\englishfont[Script=Gujarati,AutoFakeBold=2.5,AutoFakeSlant=0.3]{Noto Sans Gujarati}

% Translations for polyglossia
\gappto\captionsgujarati{
  \renewcommand{\tablename}{કોષ્ટક}
  \renewcommand{\figurename}{આકૃતિ}
}

% Helper for TikZ nodes to ensure Gujarati font
\newcommand{\gu}[1]{{\gujaratifont #1}}

% Custom environments
\newtcolorbox{solutionbox}{
    breakable,
    enhanced,
    colback=solutioncolor!5!white,
    colframe=solutioncolor!75!black,
    fonttitle=\bfseries,
    title=જવાબ
}

\newtcolorbox{solutionboxnobreak}{
 colback=solutioncolor!5!white,
 colframe=solutioncolor!75!black,
 fonttitle=\bfseries,
 title=જવાબ
}

\newtcolorbox{keyformula}{
 breakable,
 enhanced,
 colback=keycolor!5!white,
 colframe=keycolor!75!black,
 fonttitle=\bfseries,
 title=રાસાયણિક સમીકરણ/સૂત્ર
}

\newtcolorbox{mnemonicbox}{
 breakable,
 enhanced,
 colback=mnemoniccolor!5!white,
 colframe=mnemoniccolor!75!black,
 fonttitle=\bfseries,
 title=મેમરી ટ્રીક
}


% Custom commands for GTU solutions
% This file defines semantic commands for consistent formatting

% Question command with automatic formatting
\newcommand{\question}[2]{%
  \section*{Question #1}%
  \textbf{#2}%
}

% OR question variant
\newcommand{\questionor}[2]{%
  \section*{Question #1 OR}%
  \textbf{#2}%
}

% Proper table environment with caption
\newenvironment{answertable}[1]{%
  \begin{table}[htbp]
  \centering
  \caption{#1}
}{%
  \end{table}
}

% Proper figure environment for diagrams
\newenvironment{answerdiagram}[1]{%
  \begin{figure}[htbp]
  \centering
  \caption{#1}
}{%
  \end{figure}
}

% Semantic markup for key terms
\newcommand{\keyword}[1]{\textbf{#1}}
\newcommand{\code}[1]{\texttt{#1}}
\newcommand{\classname}[1]{\texttt{#1}}
\newcommand{\methodname}[1]{\texttt{#1}}

% Proper quotation marks
\newcommand{\mnemonic}[1]{``#1''}


\title{ડેટા સ્ટ્રક્ચર અને એપ્લિકેશન (1333203) - વિન્ટર 2023 સોલ્યુશન}
\date{જાન્યુઆરી 18, 2024}

\begin{document}
\maketitle

\questionmarks{1(a)}{3}{લીન્કડ લીસ્ટની વ્યાખ્યા આપો. વિવિધ પ્રકારના લિન્ક્ડ લીસ્ટ ની યાદી આપો.}

\begin{solutionbox}
\begin{center}
\captionof{table}{લિન્ક્ડ લિસ્ટ વ્યાખ્યા અને પ્રકાર}
\begin{tabulary}{\linewidth}{|L|L|}
\hline
\textbf{વ્યાખ્યા} & \textbf{લિન્ક્ડ લિસ્ટના પ્રકાર} \\ \hline
લિન્ક્ડ લિસ્ટ એ લીનિયર ડેટા સ્ટ્રક્ચર છે જેમાં એલિમેન્ટ્સ નોડ્સમાં સ્ટોર થાય છે, અને દરેક નોડ ક્રમમાં આગળના નોડને પોઇન્ટ કરે છે & 1. સિંગલી લિન્ક્ડ લિસ્ટ \\
& 2. ડબલી લિન્ક્ડ લિસ્ટ \\
& 3. સર્ક્યુલર લિન્ક્ડ લિસ્ટ \\
& 4. સર્ક્યુલર ડબલી લિન્ક્ડ લિસ્ટ \\ \hline
\end{tabulary}
\end{center}

\begin{center}
\begin{tikzpicture}[
    listnode/.style={rectangle, draw=black, minimum size=6mm},
    ptr/.style={->, >=stealth, thick}
]
    % Singly
    \node[anchor=west] at (-1, 1.5) {Singly:};
    \node[listnode] (s1) at (1, 1.5) {Data|Next};
    \node[listnode] (s2) at (3.5, 1.5) {Data|Next};
    \node[listnode] (s3) at (6, 1.5) {Data|Next};
    \node (snull) at (8, 1.5) {NULL};
    \draw[ptr] (s1) -- (s2);
    \draw[ptr] (s2) -- (s3);
    \draw[ptr] (s3) -- (snull);

    % Doubly
    \node[anchor=west] at (-1, 0) {Doubly:};
    \node[listnode] (d1) at (1.5, 0) {P|D|N};
    \node[listnode] (d2) at (4, 0) {P|D|N};
    \node[listnode] (d3) at (6.5, 0) {P|D|N};
    \node (dnull) at (8.5, 0) {NULL};
    \draw[ptr] (d1) -- (d2);
    \draw[ptr] (d2) -- (d1);
    \draw[ptr] (d2) -- (d3);
    \draw[ptr] (d3) -- (d2);
    \draw[ptr] (d3) -- (dnull);

    % Circular
    \node[anchor=west] at (-1, -1.5) {Circular:};
    \node[listnode] (c1) at (1, -1.5) {Data|Next};
    \node[listnode] (c2) at (3.5, -1.5) {Data|Next};
    \node[listnode] (c3) at (6, -1.5) {Data|Next};
    \draw[ptr] (c1) -- (c2);
    \draw[ptr] (c2) -- (c3);
    \draw[ptr] (c3.south) -- ++(0,-0.3) -| (c1.south);
\end{tikzpicture}
\captionof{figure}{લિન્ક્ડ લિસ્ટના પ્રકારો}
\end{center}
\end{solutionbox}

\begin{mnemonicbox}
\mnemonic{એક, બે, ગોળ, બે-ગોળ}
\end{mnemonicbox}

\questionmarks{1(b)}{4}{પાયથનમાં લીનીયર અને નોન-લીનીયર ડેટા સ્ટર્ચર ઉદાહરણ સાથે સમજાવો.}

\begin{solutionbox}
\begin{center}
\captionof{table}{લીનીયર વિરુદ્ધ નોન-લીનીયર ડેટા સ્ટ્રક્ચર}
\begin{tabulary}{\linewidth}{|L|L|L|}
\hline
\textbf{ડેટા સ્ટ્રક્ચર} & \textbf{વર્ણન} & \textbf{પાયથન ઉદાહરણો} \\ \hline
\textbf{લીનીયર} & એલિમેન્ટ્સ ક્રમિક રીતે ગોઠવાયેલા હોય છે જેમાં દરેક એલિમેન્ટને એકદમ એક અગાઉનું અને એક પછીનું એલિમેન્ટ હોય છે (પ્રથમ અને છેલ્લા સિવાય) & Lists: \code{[1, 2, 3]} \newline Tuples: \code{(1, 2, 3)} \newline Strings: \code{"abc"} \newline Queue: \code{queue.Queue()} \\ \hline
\textbf{નોન-લીનીયર} & એલિમેન્ટ્સ ક્રમિક રીતે ગોઠવાયેલા નથી; એક એલિમેન્ટ અનેક એલિમેન્ટ્સ સાથે જોડાઈ શકે છે & Dictionary: \code{\{"a": 1, "b": 2\}} \newline Set: \code{\{1, 2, 3\}} \newline Tree: કસ્ટમ ઇમ્પ્લીમેન્ટેશન \newline Graph: કસ્ટમ ઇમ્પ્લીમેન્ટેશન \\ \hline
\end{tabulary}
\end{center}

\begin{center}
\begin{tikzpicture}[
    node distance=1.5cm,
    level 1/.style={sibling distance=4cm},
    level 2/.style={sibling distance=2cm}
]
    \node[gtu block] {ડેટા સ્ટ્રક્ચર્સ}
        child {node[gtu block] {લીનીયર}
            child {node[gtu state] {એરે}}
            child {node[gtu state] {લિન્ક્ડ લિસ્ટ}}
            child {node[gtu state] {સ્ટેક}}
            child {node[gtu state] {ક્યુ}}
        }
        child {node[gtu block] {નોન-લીનીયર}
            child {node[gtu state] {ટ્રી}}
            child {node[gtu state] {ગ્રાફ}}
            child {node[gtu state] {હેશ ટેબલ}}
        };
\end{tikzpicture}
\captionof{figure}{ડેટા સ્ટ્રક્ચર્સનું વર્ગીકરણ}
\end{center}
\end{solutionbox}

\begin{mnemonicbox}
\mnemonic{લીનીયર લાઈનમાં, નોન-લીનીયર ચારે બાજુ}
\end{mnemonicbox}

\questionmarks{1(c)}{7}{પાયથનમાં ક્લાસ, એટ્રીબ્યુટ, ઓબ્જેક્ટ અને ક્લાસ મેથડ યોગ્ય ઉદાહરણ સાથે સમજાવો.}

\begin{solutionbox}
\begin{center}
\begin{tikzpicture}
    \node[draw, rectangle split, rectangle split parts=3, minimum width=4cm] (student) {
        \textbf{Student}
        \nodepart{second}
        - roll\_no \newline
        - name
        \nodepart{third}
        + \_\_init\_\_() \newline
        + display()
    };
\end{tikzpicture}
\captionof{figure}{ક્લાસ ડાયાગ્રામ ઉદાહરણ}
\end{center}

\begin{center}
\captionof{table}{OOP સંકલ્પનાઓ}
\begin{tabulary}{\linewidth}{|L|L|}
\hline
\textbf{શબ્દ} & \textbf{વર્ણન} \\ \hline
\textbf{ક્લાસ} & ઓબ્જેક્ટ્સ બનાવવા માટેનો બ્લૂપ્રિન્ટ, જેમાં શેર્ડ એટ્રિબ્યુટ્સ અને મેથડ્સ હોય છે \\ \hline
\textbf{એટ્રિબ્યુટ્સ} & ક્લાસની અંદર ડેટા સ્ટોર કરતા વેરિએબલ્સ \\ \hline
\textbf{ઓબ્જેક્ટ} & ક્લાસનું ઇન્સ્ટન્સ, જેમાં ચોક્કસ એટ્રિબ્યુટ વેલ્યુ હોય છે \\ \hline
\textbf{ક્લાસ મેથડ} & ક્લાસની અંદર ડિફાઇન થયેલા ફંક્શન્સ જે ક્લાસની સ્થિતિને એક્સેસ અને મોડિફાય કરી શકે છે \\ \hline
\end{tabulary}
\end{center}

\begin{lstlisting}[language=Python]
class Student:
    # ક્લાસ એટ્રિબ્યુટ
    school = "GTU"
    
    # કન્સ્ટ્રક્ટર
    def __init__(self, roll_no, name):
        # ઇન્સ્ટન્સ એટ્રિબ્યુટ્સ
        self.roll_no = roll_no
        self.name = name
    
    # ઇન્સ્ટન્સ મેથડ
    def display(self):
        print(f"Roll No: {self.roll_no}, Name: {self.name}")
    
    # ક્લાસ મેથડ
    @classmethod
    def change_school(cls, new_school):
        cls.school = new_school

# ઓબ્જેક્ટ બનાવવું
student1 = Student(101, "રાજ")
student1.display()  # આઉટપુટ: Roll No: 101, Name: રાજ
\end{lstlisting}
\end{solutionbox}

\begin{mnemonicbox}
\mnemonic{ક્લાસ બનાવે, એટ્રિબ્યુટ સંગ્રહે, ઓબ્જેક્ટ વાપરે, મેથડ ક્રિયા કરે}
\end{mnemonicbox}

\questionmarks{1(c) OR}{7}{ડેટા એન્કેપ્સુલેસન અને પોલી મોર્ફીસમની વ્યાખ્યા આપો. પોલી મોર્ફીસમ સમજાવવા માટેનો પાયથન કોડ વિકસાવો.}

\begin{solutionbox}
\begin{center}
\captionof{table}{વ્યાખ્યાઓ}
\begin{tabulary}{\linewidth}{|L|L|}
\hline
\textbf{કોન્સેપ્ટ} & \textbf{વ્યાખ્યા} \\ \hline
\textbf{ડેટા એન્કેપ્સુલેસન} & ડેટા અને મેથડ્સને એક એકમ (ક્લાસ)માં બંધ કરવા અને કેટલાક કોમ્પોનન્ટ્સને સીધી એક્સેસથી પ્રતિબંધિત કરવા \\ \hline
\textbf{પોલીમોર્ફિઝમ} & વિવિધ ક્લાસને એક જ નામના મેથડનો પોતાનો અમલ પૂરો પાડવાની ક્ષમતા \\ \hline
\end{tabulary}
\end{center}

\begin{center}
\begin{tikzpicture}[
    level 1/.style={sibling distance=4cm},
    level 2/.style={sibling distance=2.5cm}
]
    \node[gtu block] {પોલીમોર્ફિઝમ}
        child {node[gtu state] {મેથડ ઓવરરાઇડિંગ}}
        child {node[gtu state] {મેથડ ઓવરલોડિંગ}}
        child {node[gtu state] {ડક ટાઇપિંગ}};
\end{tikzpicture}
\captionof{figure}{પોલીમોર્ફિઝમના પ્રકાર}
\end{center}

\begin{lstlisting}[language=Python]
# પોલીમોર્ફિઝમ ઉદાહરણ
class Animal:
    def speak(self):
        pass

class Dog(Animal):
    def speak(self):
        return "ભૌં ભૌં!"

class Cat(Animal):
    def speak(self):
        return "મ્યાઉં!"

class Duck(Animal):
    def speak(self):
        return "ક્વેક!"

# પોલીમોર્ફિઝમ દર્શાવતું ફંક્શન
def animal_sound(animal):
    return animal.speak()

# ઓબ્જેક્ટ્સ બનાવવા
dog = Dog()
cat = Cat()
duck = Duck()

# એક જ ફંક્શન વિવિધ પ્રાણી ઓબ્જેક્ટ્સ માટે કામ કરે છે
print(animal_sound(dog))   # આઉટપુટ: ભૌં ભૌં!
print(animal_sound(cat))   # આઉટપુટ: મ્યાઉં!
print(animal_sound(duck))  # આઉટપુટ: ક્વેક!
\end{lstlisting}
\end{solutionbox}

\begin{mnemonicbox}
\mnemonic{એન્કેપ્સુલેશન છુપાવે છે, પોલીમોર્ફિઝમ બદલાય છે}
\end{mnemonicbox}

\questionmarks{2(a)}{3}{સ્ટેક અને ક્યુ નો તફાવત આપો.}

\begin{solutionbox}
\begin{center}
\captionof{table}{સ્ટેક વિરુદ્ધ ક્યુ}
\begin{tabulary}{\linewidth}{|L|L|L|}
\hline
\textbf{ફીચર} & \textbf{સ્ટેક} & \textbf{ક્યુ} \\ \hline
\textbf{સિદ્ધાંત} & LIFO (છેલ્લું આવે પહેલું જાય) & FIFO (પહેલું આવે પહેલું જાય) \\ \hline
\textbf{ઓપરેશન} & પુશ, પોપ & એનક્યુ, ડિક્યુ \\ \hline
\textbf{એક્સેસ} & એલિમેન્ટ્સ ફક્ત એક છેડેથી ઉમેરાય/દૂર થાય છે (ટોપ) & એલિમેન્ટ્સ છેલ્લે ઉમેરાય છે અને આગળથી દૂર થાય છે \\ \hline
\end{tabulary}
\end{center}

\begin{center}
\begin{tikzpicture}[
    node distance=2cm,
    stacknode/.style={rectangle, draw, minimum width=1cm, minimum height=0.6cm},
    queuenode/.style={rectangle, draw, minimum size=0.8cm}
]
    % Stack
    \node at (-3, 2) {\textbf{Stack}};
    \node[stacknode] (s3) at (-3, 1.2) {3};
    \node[stacknode, below=0cm of s3] (s2) {2};
    \node[stacknode, below=0cm of s2] (s1) {1};
    \draw[->] (-4, 1.2) -- (-3.6, 1.2) node[left, pos=0] {Push} node[midway, below] {In};
    \draw[->] (-3.6, 0.4) -- (-4, 0.4) node[left, pos=1] {Pop} node[midway, below] {Out};

    % Queue
    \node at (3, 2) {\textbf{Queue}};
    \node[queuenode] (q1) at (1.5, 0.5) {1};
    \node[queuenode, right=0cm of q1] (q2) {2};
    \node[queuenode, right=0cm of q2] (q3) {3};
    \node[below] at (q1.south) {Front};
    \node[below] at (q3.south) {Rear};
    \draw[<-] (q1.west) -- ++(-0.5, 0) node[left] {Out (Dequeue)};
    \draw[<-] (q3.east) -- ++(0.5, 0) node[right] {In (Enqueue)};
\end{tikzpicture}
\captionof{figure}{સ્ટેક અને ક્યુ}
\end{center}
\end{solutionbox}

\begin{mnemonicbox}
\mnemonic{સ્ટેક ઉપરનું પહેલા, ક્યુ આગળનું પહેલા}
\end{mnemonicbox}

\questionmarks{2(b)}{4}{પુશ અને પોપ ઓપરેશન માટેનો અલ્ગોરીધમ લખો.}

\begin{solutionbox}
\textbf{PUSH અલ્ગોરિધમ:}
\begin{lstlisting}
શરુઆત
  1. ચેક કરો કે સ્ટેક ભરેલો છે કે નહીં
  2. જો ભરેલો ન હોય, તો top ને 1 વધારો
  3. 'top' પોઝિશન પર એલિમેન્ટ ઉમેરો
સમાપ્ત
\end{lstlisting}

\textbf{POP અલ્ગોરિધમ:}
\begin{lstlisting}
શરુઆત
  1. ચેક કરો કે સ્ટેક ખાલી છે કે નહીં
  2. જો ખાલી ન હોય, તો 'top' પરના એલિમેન્ટને લો
  3. top ને 1 ઘટાડો
  4. મેળવેલ એલિમેન્ટ પાછો આપો
સમાપ્ત
\end{lstlisting}

\begin{lstlisting}[language=Python]
class Stack:
    def __init__(self, size):
        self.stack = []
        self.size = size
        self.top = -1
    
    def push(self, element):
        if self.top >= self.size - 1:
            return "Stack Overflow"
        else:
            self.top += 1
            self.stack.append(element)
            return "Pushed " + str(element)
    
    def pop(self):
        if self.top < 0:
            return "Stack Underflow"
        else:
            element = self.stack.pop()
            self.top -= 1
            return element
\end{lstlisting}
\end{solutionbox}

\begin{mnemonicbox}
\mnemonic{ટોપ પર પુશ, ટોપથી પોપ}
\end{mnemonicbox}

\questionmarks{2(c)}{7}{નીચે. આપેલ સમીકરણ ને ઇન્ફીક્સ માંથી પોસ્ટફિક્ષ માં બદલો. \newline
\textbf{A * (B + C) - D / (E + F)}}

\begin{solutionbox}
\begin{center}
\begin{tabulary}{\linewidth}{|L|L|}
\hline
\textbf{Infix} & $A * (B + C) - D / (E + F)$ \\ \hline
\textbf{Postfix} & $A B C + * D E F + / -$ \\ \hline
\end{tabulary}
\end{center}

\begin{center}
\captionof{table}{ઇન્ફીક્સ માંથી પોસ્ટફિક્ષ ટ્રેસ}
\begin{tabulary}{\linewidth}{|C|C|L|L|}
\hline
\textbf{સ્ટેપ} & \textbf{સિમ્બોલ} & \textbf{સ્ટેક} & \textbf{આઉટપુટ} \\ \hline
1 & A & & A \\ \hline
2 & * & * & A \\ \hline
3 & ( & * ( & A \\ \hline
4 & B & * ( & A B \\ \hline
5 & + & * ( + & A B \\ \hline
6 & C & * ( + & A B C \\ \hline
7 & ) & * & A B C + \\ \hline
8 & - & - & A B C + * \\ \hline
9 & D & - & A B C + * D \\ \hline
10 & / & - / & A B C + * D \\ \hline
11 & ( & - / ( & A B C + * D \\ \hline
12 & E & - / ( & A B C + * D E \\ \hline
13 & + & - / ( + & A B C + * D E \\ \hline
14 & F & - / ( + & A B C + * D E F \\ \hline
15 & ) & - / & A B C + * D E F + \\ \hline
16 & end & & A B C + * D E F + / - \\ \hline
\end{tabulary}
\end{center}

\textbf{જવાબ:} \code{A B C + * D E F + / -}
\end{solutionbox}

\begin{mnemonicbox}
\mnemonic{ઓપરેટર સ્ટેક પર, ઓપરન્ડ સીધા પ્રિન્ટ}
\end{mnemonicbox}

\questionmarks{2(a) OR}{3}{સિમ્પલ ક્યુ અને સર્ક્યુલર ક્યુ નો તફાવત આપો.}

\begin{solutionbox}
\begin{center}
\captionof{table}{સિમ્પલ વિરુદ્ધ સર્ક્યુલર ક્યુ}
\begin{tabulary}{\linewidth}{|L|L|L|}
\hline
\textbf{ફીચર} & \textbf{સિમ્પલ ક્યુ} & \textbf{સર્ક્યુલર ક્યુ} \\ \hline
\textbf{સ્ટ્રક્ચર} & લીનિયર ડેટા સ્ટ્રક્ચર & જોડાયેલા છેડાવાળો લીનિયર ડેટા સ્ટ્રક્ચર \\ \hline
\textbf{મેમરી} & ડિક્યુ પછી ખાલી જગ્યાઓને કારણે અકાર્યક્ષમ મેમરી વપરાશ & ખાલી જગ્યાઓનો ફરીથી ઉપયોગ કરીને કાર્યક્ષમ મેમરી વપરાશ \\ \hline
\textbf{ઇમ્પ્લિમેન્ટેશન} & ફ્રન્ટ હંમેશા ઇન્ડેક્સ 0 પર, રીયર વધે & ફ્રન્ટ અને રીયર મોડ્યુલો ઓપરેશન સાથે સર્ક્યુલર રીતે ફરે \\ \hline
\end{tabulary}
\end{center}

\begin{center}
\begin{tikzpicture}
    % Simple Queue
    \node[anchor=west] at (-4, 1) {Simple Queue:};
    \draw (-4, 0) grid (0, 0.6);
    \node[above] at (-3.5, 0.6) {Front};
    \node[above] at (-0.5, 0.6) {Rear};

    % Circular Queue
    \node[anchor=west] at (2, 1) {Circular Queue:};
    \draw (3.5,0) circle [radius=1.2];
    \draw (3.5,0) circle [radius=0.6];
    \foreach \angle in {0, 45, ..., 315}
        \draw (3.5,0) ++(\angle:0.6) -- ++(\angle:0.6);
    \node at (3.5, 0) {Cycle};
\end{tikzpicture}
\captionof{figure}{સિમ્પલ વિરુદ્ધ સર્ક્યુલર ક્યુ સ્ટ્રક્ચર}
\end{center}
\end{solutionbox}

\begin{mnemonicbox}
\mnemonic{સાદી વેડફે, ગોળ ફરીથી વાપરે}
\end{mnemonicbox}

\questionmarks{2(b) OR}{4}{રીકસીવ ફંક્શનનો કોન્સેપ્ટ યોગ્ય ઉદાહરણ સાથે સમજાવો.}

\begin{solutionbox}
\begin{center}
\captionof{table}{રિકર્ઝન કોન્સેપ્ટ}
\begin{tabulary}{\linewidth}{|L|L|}
\hline
\textbf{મુખ્ય પાસાઓ} & \textbf{વર્ણન} \\ \hline
\textbf{વ્યાખ્યા} & એવું ફંક્શન જે એક જ સમસ્યાના નાના ભાગને હલ કરવા માટે પોતાને જ કોલ કરે છે \\ \hline
\textbf{બેઝ કેસ} & એવી સ્થિતિ જ્યાં ફંક્શન પોતાને કોલ કરવાનું બંધ કરે છે \\ \hline
\textbf{રિકર્સિવ કેસ} & એવી સ્થિતિ જ્યાં ફંક્શન સમસ્યાના સરળ સ્વરૂપ સાથે પોતાને કોલ કરે છે \\ \hline
\end{tabulary}
\end{center}

\begin{center}
\begin{tikzpicture}[
    level distance=1.2cm,
    level 1/.style={sibling distance=4cm},
    every node/.style={rectangle, draw, rounded corners}
]
    \node {factorial(3)}
        child {node {3 * factorial(2)}
            child {node {3 * 2 * factorial(1)}
                child {node {3 * 2 * 1 * factorial(0)}
                    child {node {3 * 2 * 1 * 1 = 6}}
                }
            }
        };
\end{tikzpicture}
\captionof{figure}{રિકર્સિવ કોલ્સ ટ્રેસ}
\end{center}

\begin{lstlisting}[language=Python]
def factorial(n):
    # બેઝ કેસ
    if n == 0:
        return 1
    # રિકર્સિવ કેસ
    else:
        return n * factorial(n-1)

# ઉદાહરણ
result = factorial(5)  # 5! = 120
\end{lstlisting}
\end{solutionbox}

\begin{mnemonicbox}
\mnemonic{બેઝ તોડે, રિકર્શન પાછું આપે}
\end{mnemonicbox}

\questionmarks{2(c) OR}{7}{Enqueue અને Dequeue ઓપરેશન માટેનો પાયથન કોડ વિકસાવો.}

\begin{solutionbox}
\begin{center}
\begin{tikzpicture}[
    cell/.style={rectangle, draw, minimum size=0.8cm},
    scale=0.8, transform shape
]
    % Enqueue
    \node[anchor=west] at (-6, 2) {\textbf{Enqueue Operation:}};
    \node[cell] at (-5, 1) {1};
    \node[cell, right=0cm] at (-4.2, 1) {2};
    \node[cell, right=0cm] at (-3.4, 1) {3};
    \draw[->, thick] (-1.5, 1) -- (-2.5, 1) node[midway, above] {Add 4};
    \node[cell] at (0, 1) {1};
    \node[cell, right=0cm] at (0.8, 1) {2};
    \node[cell, right=0cm] at (1.6, 1) {3};
    \node[cell, right=0cm, fill=green!20] at (2.4, 1) {4};

    % Dequeue
    \node[anchor=west] at (-6, 0) {\textbf{Dequeue Operation:}};
    \node[cell, fill=red!20] at (-5, -1) {1};
    \node[cell, right=0cm] at (-4.2, -1) {2};
    \node[cell, right=0cm] at (-3.4, -1) {3};
    \node[cell, right=0cm] at (-2.6, -1) {4};
    \draw[->, thick] (-1.5, -1) -- (-0.5, -1) node[midway, above] {Remove Front};
    \node[cell] at (1, -1) {2};
    \node[cell, right=0cm] at (1.8, -1) {3};
    \node[cell, right=0cm] at (2.6, -1) {4};
\end{tikzpicture}
\captionof{figure}{Enqueue અને Dequeue વિઝ્યુઅલાઈઝેશન}
\end{center}

\begin{lstlisting}[language=Python]
class Queue:
    def __init__(self, size):
        self.queue = []
        self.size = size
        self.front = 0
        self.rear = -1
        self.count = 0
    
    def enqueue(self, item):
        if self.count >= self.size:
            return "ક્યુ ભરેલી છે"
        else:
            self.rear += 1
            self.queue.append(item)
            self.count += 1
            return "Enqueued " + str(item)
    
    def dequeue(self):
        if self.count <= 0:
            return "ક્યુ ખાલી છે"
        else:
            item = self.queue.pop(0)
            self.count -= 1
            return item
    
    def display(self):
        return self.queue

# ટેસ્ટ
q = Queue(5)
q.enqueue(10)
q.enqueue(20)
q.enqueue(30)
print(q.display())  # [10, 20, 30]
print(q.dequeue())  # 10
print(q.display())  # [20, 30]
\end{lstlisting}
\end{solutionbox}

\begin{mnemonicbox}
\mnemonic{છેડે ઉમેરો, શરૂઆતથી કાઢો}
\end{mnemonicbox}

\questionmarks{3(a)}{3}{સીન્ગલી લિન્ક્ડ લીસ્ટ અને સર્ક્યુલર લિન્ક્ડ લીસ્ટ નો તફાવત આપો.}

\begin{solutionbox}
\begin{center}
\captionof{table}{સિંગલી વિરુદ્ધ સર્ક્યુલર લિન્ક્ડ લિસ્ટ}
\begin{tabulary}{\linewidth}{|L|L|L|}
\hline
\textbf{ફીચર} & \textbf{સિંગલી લિન્ક્ડ લિસ્ટ} & \textbf{સર્ક્યુલર લિન્ક્ડ લિસ્ટ} \\ \hline
\textbf{છેલ્લો નોડ} & NULL તરફ પોઇન્ટ કરે છે & પહેલા નોડ તરફ પાછો પોઇન્ટ કરે છે \\ \hline
\textbf{ટ્રાવર્સલ} & ચોક્કસ અંત ધરાવે છે & સતત ટ્રાવર્સ કરી શકાય છે \\ \hline
\textbf{મેમરી} & દરેક નોડને એક પોઇન્ટર જોઈએ & દરેક નોડને એક પોઇન્ટર જોઈએ \\ \hline
\end{tabulary}
\end{center}

\begin{center}
\begin{tikzpicture}[
    listnode/.style={rectangle, draw, minimum width=0.8cm},
    ptr/.style={->, >=stealth}
]
    % Singly
    \node[anchor=west] at (-2, 1) {Singly:};
    \node[listnode] (s1) at (0, 1) {1};
    \node[listnode] (s2) at (1.5, 1) {2};
    \node[listnode] (s3) at (3, 1) {3};
    \node (null) at (4.5, 1) {NULL};
    \draw[ptr] (s1) -- (s2);
    \draw[ptr] (s2) -- (s3);
    \draw[ptr] (s3) -- (null);

    % Circular
    \node[anchor=west] at (-2, -0.5) {Circular:};
    \node[listnode] (c1) at (0, -0.5) {1};
    \node[listnode] (c2) at (1.5, -0.5) {2};
    \node[listnode] (c3) at (3, -0.5) {3};
    \draw[ptr] (c1) -- (c2);
    \draw[ptr] (c2) -- (c3);
    \draw[ptr] (c3.south) -- ++(0,-0.3) -| (c1.south);
\end{tikzpicture}
\captionof{figure}{સિંગલી વિરુદ્ધ સર્ક્યુલર સ્ટ્રક્ચર}
\end{center}
\end{solutionbox}

\begin{mnemonicbox}
\mnemonic{સિંગલી અટકે, સર્ક્યુલર ફરે}
\end{mnemonicbox}

\questionmarks{3(b)}{4}{ડબલી લિન્ક્ડ લીસ્ટ નો કોન્સેપ્ટ સમજાવો.}

\begin{solutionbox}
\begin{center}
\begin{tikzpicture}[
    dnode/.style={rectangle split, rectangle split parts=3, draw, rectangle split horizontal, minimum height=0.6cm},
    ptr/.style={->, >=stealth, thick}
]
    \node[dnode] (n1) at (0,0) {\nodepart{one}P \nodepart{two}1 \nodepart{three}N};
    \node[dnode] (n2) at (3,0) {\nodepart{one}P \nodepart{two}2 \nodepart{three}N};
    \node[dnode] (n3) at (6,0) {\nodepart{one}P \nodepart{two}3 \nodepart{three}N};
    
    \node (start) at (-2, 0) {NULL};
    \node (end) at (8, 0) {NULL};

    \draw[ptr] (start) -- (n1.west);
    \draw[ptr] (n1.three) -- (n2.one);
    \draw[ptr] (n2.one) to[bend left] (n1.three);
    \draw[ptr] (n2.three) -- (n3.one);
    \draw[ptr] (n3.one) to[bend left] (n2.three);
    \draw[ptr] (n3.three) -- (end);

    \draw[ptr] (n1.one) to[bend left] (start);
\end{tikzpicture}
\captionof{figure}{ડબલી લિન્ક્ડ લિસ્ટ સ્ટ્રક્ચર}
\end{center}

\begin{center}
\captionof{table}{ડબલી લિન્ક્ડ લિસ્ટ ફીચર્સ}
\begin{tabulary}{\linewidth}{|L|L|}
\hline
\textbf{ફીચર} & \textbf{વર્ણન} \\ \hline
\textbf{નોડ સ્ટ્રક્ચર} & દરેક નોડમાં ડેટા અને બે પોઇન્ટર્સ (previous અને next) હોય છે \\ \hline
\textbf{નેવિગેશન} & આગળ અને પાછળ એમ બંને દિશામાં ટ્રાવર્સ કરી શકાય છે \\ \hline
\textbf{ઓપરેશન્સ} & બંને છેડેથી ઇન્સર્શન અને ડિલીશન કરી શકાય છે \\ \hline
\textbf{મેમરી વપરાશ} & વધારાના પોઇન્ટરને કારણે સિંગલી લિન્ક્ડ લિસ્ટ કરતા વધુ મેમરી જોઈએ \\ \hline
\end{tabulary}
\end{center}

\begin{lstlisting}[language=Python]
class Node:
    def __init__(self, data):
        self.data = data
        self.prev = None
        self.next = None
\end{lstlisting}
\end{solutionbox}

\begin{mnemonicbox}
\mnemonic{બે પોઇન્ટર, બે દિશા}
\end{mnemonicbox}

\questionmarks{3(c)}{7}{નીચે આપેલ ઓપરેશન માટે અલગોરિધમ લખો: \newline
\textbf{૧. લીસ્ટ ની શરૂઆતમાં નોડ દાખલ કરવા} \newline
\textbf{૨. લીસ્ટ ના અંતમાં નોડ દાખલ કરવા}}

\begin{solutionbox}
\textbf{શરૂઆતમાં ઇન્સર્ટ:}

\begin{center}
\begin{tikzpicture}[node distance=1.5cm, auto]
    \node[gtu block] (start) {શરૂ};
    \node[gtu block, right=of start] (create) {નવો નોડ બનાવો};
    \node[gtu block, right=of create] (data) {ડેટા સેટ કરો};
    \node[gtu block, below=of data] (next) {next હેડ પર સેટ};
    \node[gtu block, left=of next] (head) {હેડ = નવો નોડ};
    \node[gtu state, left=of head] (end) {અંત};

    \path[gtu arrow] (start) -- (create);
    \path[gtu arrow] (create) -- (data);
    \path[gtu arrow] (data) -- (next);
    \path[gtu arrow] (next) -- (head);
    \path[gtu arrow] (head) -- (end);
\end{tikzpicture}
\end{center}

\textbf{અંતે ઇન્સર્ટ:}

\begin{center}
\begin{tikzpicture}[node distance=1.5cm, auto]
    \node[gtu block] (start) {શરૂ};
    \node[gtu block, right=of start] (create) {નવો નોડ બનાવો};
    \node[gtu block, right=of create] (data) {ડેટા સેટ કરો};
    \node[gtu block, below=of data] (next) {next=NULL સેટ};
    \node[gtu decision, left=of next] (check) {head NULL છે?};
    \node[gtu block, below=of check] (sethead) {head = નવો નોડ};
    \node[gtu block, left=of check] (traverse) {છેલ્લે સુધી જાઓ};
    \node[gtu block, below=of traverse] (link) {last.next = new node};
    \node[gtu state, left=of traverse] (end) {અંત};

    \path[gtu arrow] (start) -- (create);
    \path[gtu arrow] (create) -- (data);
    \path[gtu arrow] (data) -- (next);
    \path[gtu arrow] (next) -- (check);
    \path[gtu arrow] (check) -- node[right] {હા} (sethead);
    \path[gtu arrow] (check) -- node[above] {ના} (traverse);
    \path[gtu arrow] (traverse) -- (link);
    \path[gtu arrow] (sethead) -| (end);
    \path[gtu arrow] (link) -| (end);
\end{tikzpicture}
\end{center}

\begin{lstlisting}[language=Python]
def insert_at_beginning(head, data):
    new_node = Node(data)
    new_node.next = head
    return new_node  # નવો head

def insert_at_end(head, data):
    new_node = Node(data)
    new_node.next = None
    
    # જો લિન્ક્ડ લિસ્ટ ખાલી હોય
    if head is None:
        return new_node
    
    # છેલ્લા નોડ સુધી ટ્રાવર્સ કરો
    temp = head
    while temp.next:
        temp = temp.next
    
    # છેલ્લા નોડને નવા નોડ સાથે જોડો
    temp.next = new_node
    return head
\end{lstlisting}
\end{solutionbox}

\begin{mnemonicbox}
\mnemonic{શરૂઆત: નવો જૂનાને આગળ કરે, અંત: જૂનો નવાને આગળ કરે}
\end{mnemonicbox}

\questionmarks{3(a) OR}{3}{સીન્ગલી લિન્ક્ડ લીસ્ટ પરના વિવિધ ઓપરેશન ની યાદી આપો.}

\begin{solutionbox}
\begin{center}
\captionof{table}{સિંગલી લિન્ક્ડ લિસ્ટ પરના ઓપરેશન}
\begin{tabulary}{\linewidth}{|L|}
\hline
\textbf{સિંગલી લિન્ક્ડ લિસ્ટ પરના ઓપરેશન} \\ \hline
1. ઇન્સર્શન (શરૂઆતમાં, મધ્યમાં, અંતે) \\ \hline
2. ડિલીશન (શરૂઆતથી, મધ્યમાંથી, અંતથી) \\ \hline
3. ટ્રાવર્સલ (દરેક નોડની મુલાકાત) \\ \hline
4. શોધ (ચોક્કસ નોડ શોધવો) \\ \hline
5. અપડેટિંગ (નોડ ડેટા બદલવો) \\ \hline
\end{tabulary}
\end{center}

\begin{center}
\begin{tikzpicture}[
    level 1/.style={sibling distance=3.5cm},
    level 2/.style={sibling distance=2cm}
]
    \node[gtu block] {લિન્ક્ડ લિસ્ટ ઓપરેશન}
        child {node[gtu state] {ઇન્સર્શન}}
        child {node[gtu state] {ડિલિશન}}
        child {node[gtu state] {ટ્રાવર્સલ}}
        child {node[gtu state] {શોધ}}
        child {node[gtu state] {અપડેટિંગ}};
\end{tikzpicture}
\captionof{figure}{લિન્ક્ડ લિસ્ટ ઓપરેશન}
\end{center}
\end{solutionbox}

\begin{mnemonicbox}
\mnemonic{ઉમેરો કાઢો ફરો શોધો બદલો}
\end{mnemonicbox}

\questionmarks{3(b) OR}{4}{સર્ક્યુલર લિન્ક્ડ લીસ્ટ નો કોન્સેપ્ટ સમજાવો.}

\begin{solutionbox}
\begin{center}
\begin{tikzpicture}[
    listnode/.style={rectangle, draw, minimum size=0.8cm},
    ptr/.style={->, >=stealth, thick}
]
    \node[listnode] (n1) at (0,0) {1};
    \node[listnode] (n2) at (2,0) {2};
    \node[listnode] (n3) at (4,0) {3};
    \node[listnode] (n4) at (6,0) {4};

    \draw[ptr] (n1) -- (n2);
    \draw[ptr] (n2) -- (n3);
    \draw[ptr] (n3) -- (n4);
    
    \draw[ptr] (n4.south) -- ++(0,-0.5) -| (n1.south);
    \node at (3, -0.7) {Last points to First};
\end{tikzpicture}
\captionof{figure}{સર્ક્યુલર લિન્ક્ડ લિસ્ટ}
\end{center}

\begin{center}
\captionof{table}{સર્ક્યુલર લિન્ક્ડ લિસ્ટ ફીચર્સ}
\begin{tabulary}{\linewidth}{|L|L|}
\hline
\textbf{ફીચર} & \textbf{વર્ણન} \\ \hline
\textbf{સ્ટ્રક્ચર} & છેલ્લો નોડ NULL ને બદલે પહેલા નોડને પોઇન્ટ કરે છે \\ \hline
\textbf{ફાયદો} & બધા નોડમાં સતત ટ્રાવર્સલની અનુમતિ આપે છે \\ \hline
\textbf{એપ્લિકેશન} & રાઉન્ડ રોબિન શેડ્યુલિંગ, સર્ક્યુલર બફર ઇમ્પ્લિમેન્ટેશન \\ \hline
\textbf{ઓપરેશન} & છેલ્લા નોડ માટે ખાસ હેન્ડલિંગ સાથે સિંગલી લિન્ક્ડ લિસ્ટ જેવા ઇન્સર્શન અને ડિલીશન \\ \hline
\end{tabulary}
\end{center}

\begin{lstlisting}[language=Python]
class Node:
    def __init__(self, data):
        self.data = data
        self.next = None

# 3 નોડવાળી સર્ક્યુલર લિન્ક્ડ લિસ્ટ બનાવવી
head = Node(1)
node2 = Node(2)
node3 = Node(3)

head.next = node2
node2.next = node3
node3.next = head  # તેને સર્ક્યુલર બનાવે છે
\end{lstlisting}
\end{solutionbox}

\begin{mnemonicbox}
\mnemonic{છેલ્લો પહેલાને જોડે}
\end{mnemonicbox}

\questionmarks{3(c) OR}{7}{લિન્ક્ડ લીસ્ટની એપ્લીકેશનોની યાદી આપો. સીન્ગલી લિન્ક્ડ લીસ્ટમાં કુલ નોડ ગણવા માટેનો અલગોરિધમ લખો.}

\begin{solutionbox}
\begin{center}
\captionof{table}{એપ્લિકેશન}
\begin{tabulary}{\linewidth}{|L|}
\hline
\textbf{લિન્ક્ડ લિસ્ટની એપ્લિકેશન} \\ \hline
1. સ્ટેક અને ક્યુનો અમલીકરણ \\ \hline
2. ડાયનેમિક મેમરી એલોકેશન \\ \hline
3. એપ્લિકેશનમાં અન્ડુ ફંક્શનાલિટી \\ \hline
4. હેશ ટેબલ્સ (ચેઇનિંગ) \\ \hline
5. ગ્રાફ્સ માટે એડજસન્સી લિસ્ટ \\ \hline
\end{tabulary}
\end{center}

\textbf{નોડ ગણવા માટેનો અલ્ગોરિધમ:}
\begin{center}
\begin{tikzpicture}[node distance=1.5cm, auto]
    \node[gtu block] (start) {શરૂ};
    \node[gtu block, right=of start] (init) {count = 0};
    \node[gtu block, right=of init] (temp) {temp = head};
    \node[gtu decision, below=of temp] (check) {temp != NULL?};
    \node[gtu block, left=of check] (inc) {count++};
    \node[gtu block, below=of inc] (next) {temp = temp.next};
    \node[gtu state, right=of check] (end) {રિટર્ન count};

    \path[gtu arrow] (start) -- (init);
    \path[gtu arrow] (init) -- (temp);
    \path[gtu arrow] (temp) -- (check);
    \path[gtu arrow] (check) -- node[above] {હા} (inc);
    \path[gtu arrow] (inc) -- (next);
    \path[gtu arrow] (next) -- (check.south);
    \path[gtu arrow] (check) -- node[above] {ના} (end);
\end{tikzpicture}
\end{center}

\begin{lstlisting}[language=Python]
def count_nodes(head):
    count = 0
    temp = head
    
    while temp:
        count += 1
        temp = temp.next
    
    return count

# ઉદાહરણ
# ધારી લો કે head લિન્ક્ડ લિસ્ટના પ્રથમ નોડને પોઇન્ટ કરે છે
total_nodes = count_nodes(head)
print(f"કુલ નોડ: {total_nodes}")
\end{lstlisting}
\end{solutionbox}

\begin{mnemonicbox}
\mnemonic{ગણો ત્યારે ખસો}
\end{mnemonicbox}

\questionmarks{4(a)}{3}{લીનીયર સર્ચ અને બાયનરી સર્ચની સરખામણી કરો.}

\begin{solutionbox}
\begin{center}
\captionof{table}{લીનીયર વિરુદ્ધ બાયનરી સર્ચ}
\begin{tabulary}{\linewidth}{|L|L|L|}
\hline
\textbf{ફીચર} & \textbf{લીનીયર સર્ચ} & \textbf{બાયનરી સર્ચ} \\ \hline
\textbf{ડેટા ગોઠવણ} & સોર્ટેડ અને અનસોર્ટેડ બંને ડેટા પર કામ કરે છે & ફક્ત સોર્ટેડ ડેટા પર કામ કરે છે \\ \hline
\textbf{ટાઇમ કોમ્પ્લેક્સિટી} & O(n) & O(log n) \\ \hline
\textbf{ઇમ્પ્લિમેન્ટેશન} & સરળ & વધુ જટિલ \\ \hline
\textbf{શેના માટે શ્રેષ્ઠ} & નાના ડેટાસેટ અથવા અનસોર્ટેડ ડેટા & મોટા સોર્ટેડ ડેટાસેટ \\ \hline
\end{tabulary}
\end{center}

\begin{center}
\begin{tikzpicture}
    % Linear
    \node[anchor=west] at (-4, 2) {Linear:};
    \foreach \x/\val in {0/1, 1/2, 2/3, 3/4, 4/5} {
        \node[draw, rectangle, minimum size=0.6cm] at (-3+\x*0.7, 1.2) {\val};
        \draw[->] (-3+\x*0.7, 1.8) -- (-3+\x*0.7, 1.5);
    }
    \node at (-1.5, 0.8) {Sequential Check};

    % Binary
    \node[anchor=west] at (1, 2) {Binary:};
    \foreach \x/\val in {0/1, 1/2, 2/3, 3/4, 4/5, 5/6, 6/7} {
        \node[draw, rectangle, minimum size=0.6cm] at (2+\x*0.7, 1.2) {\val};
    }
    \draw[->, thick] (2+3*0.7, 1.8) -- (2+3*0.7, 1.5) node[midway, right] {Middle};
    \draw (2, 0.8) -- (4, 0.8) node[midway, below] {Halving};
\end{tikzpicture}
\captionof{figure}{સર્ચ સરખામણી}
\end{center}
\end{solutionbox}

\begin{mnemonicbox}
\mnemonic{લીનીયર બધું જુએ, બાઈનરી આધું કાપે}
\end{mnemonicbox}

\questionmarks{4(b)}{4}{સિલેકશન સોર્ટ માટેનો અલગોરિધમ લખો.}

\begin{solutionbox}
\textbf{વિઝ્યુઅલાઈઝેશન:}
\begin{center}
\begin{tikzpicture}[
    arrnode/.style={draw, rectangle, minimum size=0.6cm},
    scale=0.8, transform shape
]
    \node[arrnode] at (0, 3) {5}; \node[arrnode] at (0.7, 3) {3}; \node[arrnode] at (1.4, 3) {8}; \node[arrnode] at (2.1, 3) {1}; \node[arrnode] at (2.8, 3) {2};
    \node at (4.5, 3) {Initial};

    \node[arrnode, fill=green!20] at (0, 2) {1}; \node[arrnode] at (0.7, 2) {3}; \node[arrnode] at (1.4, 2) {8}; \node[arrnode] at (2.1, 2) {5}; \node[arrnode] at (2.8, 2) {2};
    \node at (5, 2) {Pass 1 (Swap 1 \& 5)};

    \node[arrnode, fill=green!20] at (0, 1) {1}; \node[arrnode, fill=green!20] at (0.7, 1) {2}; \node[arrnode] at (1.4, 1) {8}; \node[arrnode] at (2.1, 1) {5}; \node[arrnode] at (2.8, 1) {3};
    \node at (5, 1) {Pass 2 (Swap 2 \& 3)};
\end{tikzpicture}
\end{center}

\textbf{અલ્ગોરિધમ:}
\begin{center}
\begin{tikzpicture}[node distance=1.5cm]
    \node[gtu block] (start) {Start};
    \node[gtu block, right=of start] (loop) {For i = 0 to n-1};
    \node[gtu block, right=of loop] (find) {Find min};
    \node[gtu block, below=of find] (swap) {Swap min with arr[i]};
    \node[gtu state, left=of swap] (end) {End};

    \path[gtu arrow] (start) -- (loop);
    \path[gtu arrow] (loop) -- (find);
    \path[gtu arrow] (find) -- (swap);
    \path[gtu arrow] (swap) -- (end);
\end{tikzpicture}
\end{center}

\begin{lstlisting}[language=Python]
def selection_sort(arr):
    n = len(arr)
    
    for i in range(n):
        min_idx = i
        
        # અનસોર્ટેડ એરેમાં લઘુતમ એલિમેન્ટ શોધો
        for j in range(i+1, n):
            if arr[j] < arr[min_idx]:
                min_idx = j
        
        # શોધેલા લઘુતમ એલિમેન્ટને પ્રથમ એલિમેન્ટ સાથે સ્વેપ કરો
        arr[i], arr[min_idx] = arr[min_idx], arr[i]
\end{lstlisting}
\end{solutionbox}

\begin{mnemonicbox}
\mnemonic{લઘુતમ શોધો, પોઝિશન બદલો}
\end{mnemonicbox}

\questionmarks{4(c)}{7}{નીચે આપેલા લીસ્ટ ને બબલ સોર્ટ મેથડ વડે ચઢતા ક્રમમાં ગોઠવવા માટેનો પાયથન કોડ વિકસાવો. \newline
\textbf{list1=[5,4,3,2,1,0]}}

\begin{solutionbox}
\begin{center}
\begin{tikzpicture}[
    arrnode/.style={draw, rectangle, minimum size=0.6cm},
    scale=0.8, transform shape
]
    \node at (-2, 2.5) {Initial:};
    \foreach \x/\val in {0/5, 1/4, 2/3, 3/2, 4/1, 5/0} \node[arrnode] at (\x*0.7, 2.5) {\val};

    \node at (-2, 1.5) {Pass 1:};
    \foreach \x/\val in {0/4, 1/3, 2/2, 3/1, 4/0, 5/5} \node[arrnode] at (\x*0.7, 1.5) {\val};
    \node at (6, 1.5) {(5 bubbles to end)};

    \node at (-2, 0.5) {Pass 2:};
    \foreach \x/\val in {0/3, 1/2, 2/1, 3/0, 4/4, 5/5} \node[arrnode] at (\x*0.7, 0.5) {\val};
    \node at (6, 0.5) {(4 bubbles up)};
    
    \node at (2, -0.5) {... continues until sorted ...};
\end{tikzpicture}
\captionof{figure}{બબલ સોર્ટ ટ્રેસ}
\end{center}

\begin{lstlisting}[language=Python]
def bubble_sort(arr):
    n = len(arr)
    
    # બધા એરે એલિમેન્ટ્સ પર ટ્રાવર્સ કરો
    for i in range(n):
        # છેલ્લા i એલિમેન્ટ્સ પહેલેથી જ યોગ્ય જગ્યા પર છે
        for j in range(0, n-i-1):
            # જો વર્તમાન એલિમેન્ટ આગળના એલિમેન્ટ કરતાં મોટો હોય, તો સ્વેપ કરો
            if arr[j] > arr[j+1]:
                arr[j], arr[j+1] = arr[j+1], arr[j]
    
    return arr

# ઇનપુટ લિસ્ટ
list1 = [5, 4, 3, 2, 1, 0]

# લિસ્ટ સોર્ટ કરવી
sorted_list = bubble_sort(list1)

# રિઝલ્ટ ડિસ્પ્લે કરવું
print("સોર્ટેડ લિસ્ટ:", sorted_list)
# આઉટપુટ: સોર્ટેડ લિસ્ટ: [0, 1, 2, 3, 4, 5]
\end{lstlisting}
\end{solutionbox}

\begin{mnemonicbox}
\mnemonic{મોટા બબલ ઉપર જાય}
\end{mnemonicbox}

\questionmarks{4(a) OR}{3}{સોર્ટિંગ ની વ્યાખ્યા આપો. વિવિધ પ્રકારના સોર્ટિંગ ની યાદી આપો.}

\begin{solutionbox}
\begin{center}
\captionof{table}{સોર્ટિંગ વ્યાખ્યા}
\begin{tabulary}{\linewidth}{|L|L|}
\hline
\textbf{વ્યાખ્યા} & \textbf{સોર્ટિંગ મેથડ્સ} \\ \hline
સોર્ટિંગ એટલે ડેટાને ચોક્કસ ક્રમમાં (ચઢતા અથવા ઉતરતા) ગોઠવવાની પ્રક્રિયા & 1. બબલ સોર્ટ \newline 2. સિલેક્શન સોર્ટ \newline 3. ઇન્સર્શન સોર્ટ \newline 4. મર્જ સોર્ટ \newline 5. ક્વિક સોર્ટ \newline 6. હીપ સોર્ટ \newline 7. રેડિક્સ સોર્ટ \\ \hline
\end{tabulary}
\end{center}

\begin{center}
\begin{tikzpicture}[
    level 1/.style={sibling distance=4cm},
    level 2/.style={sibling distance=1.5cm}
]
    \node[gtu block] {સોર્ટિંગ અલ્ગોરિધમ}
        child {node[gtu block] {કમ્પેરિઝન-બેઝ્ડ}
            child {node[gtu state] {બબલ}}
            child {node[gtu state] {સિલેક્શન}}
            child {node[gtu state] {ઇન્સર્શન}}
            child {node[gtu state] {મર્જ}}
            child {node[gtu state] {ક્વિક}}
        }
        child {node[gtu block] {નોન-કમ્પેરિઝન}
            child {node[gtu state] {કાઉન્ટિંગ}}
            child {node[gtu state] {રેડિક્સ}}
            child {node[gtu state] {બકેટ}}
        };
\end{tikzpicture}
\captionof{figure}{સોર્ટિંગ અલ્ગોરિધમ્સ}
\end{center}
\end{solutionbox}

\begin{mnemonicbox}
\mnemonic{સારા સોર્ટથી શોધવાનું સરળ બને}
\end{mnemonicbox}

\questionmarks{4(b) OR}{4}{Insertion sort method નો અલગોરિધમ લખો.}

\begin{solutionbox}
\textbf{વિઝ્યુઅલાઈઝેશન:}
\begin{center}
\begin{tikzpicture}[
    arrnode/.style={draw, rectangle, minimum size=0.6cm},
    scale=0.8, transform shape
]
    \node at (0, 3) {Initial:}; \foreach \x/\val in {0/5, 1/2, 2/4} \node[arrnode] at (2+\x*0.7, 3) {\val};
    \node at (0, 2) {Pass 1:}; \foreach \x/\val in {0/2, 1/5, 2/4} \node[arrnode] at (2+\x*0.7, 2) {\val}; \node at (5, 2) {Insert 2};
    \node at (0, 1) {Pass 2:}; \foreach \x/\val in {0/2, 1/4, 2/5} \node[arrnode] at (2+\x*0.7, 1) {\val}; \node at (5, 1) {Insert 4};
\end{tikzpicture}
\end{center}

\textbf{અલ્ગોરિધમ:}
\begin{center}
\begin{tikzpicture}[node distance=1.2cm]
    \node[gtu block] (start) {Start};
    \node[gtu block, right=of start] (loop) {For i=1 to n-1};
    \node[gtu block, right=of loop] (key) {key = arr[i]};
    \node[gtu decision, below=of key] (check) {arr[j] > key?};
    \node[gtu block, left=of check] (move) {Move right};
    \node[gtu block, below=of check] (place) {Place key};
    \node[gtu state, left=of place] (end) {End};

    \path[gtu arrow] (start) -- (loop);
    \path[gtu arrow] (loop) -- (key);
    \path[gtu arrow] (key) -- (check);
    \path[gtu arrow] (check) -- node[above] {Yes} (move);
    \path[gtu arrow] (move) |- (check);
    \path[gtu arrow] (check) -- node[right] {No} (place);
    \path[gtu arrow] (place) -- (end);
\end{tikzpicture}
\end{center}

\begin{lstlisting}[language=Python]
def insertion_sort(arr):
    for i in range(1, len(arr)):
        key = arr[i]
        j = i - 1
        
        # key કરતાં મોટા એલિમેન્ટ્સને એક પોઝિશન આગળ ખસેડો
        while j >= 0 and arr[j] > key:
            arr[j + 1] = arr[j]
            j -= 1
        
        arr[j + 1] = key
\end{lstlisting}
\end{solutionbox}

\begin{mnemonicbox}
\mnemonic{કાર્ડ લો, યોગ્ય ક્રમમાં મૂકો}
\end{mnemonicbox}

\questionmarks{4(c) OR}{7}{નીચે આપેલા લીસ્ટ ને સિલેકશન સોર્ટ મેથડ વડે ચઢતા ક્રમમાં ગોઠવવા માટેનો પાયથન કોડ વિકસાવો. \newline
\textbf{list1=[6,3,25,8,-1,55,0]}}

\begin{solutionbox}
\begin{center}
\begin{tikzpicture}[
    arrnode/.style={draw, rectangle, minimum size=0.6cm},
    scale=0.8, transform shape
]
    \node at (-1, 3.5) {Initial:};
    \foreach \x/\val in {0/6, 1/3, 2/25, 3/8, 4/-1, 5/55, 6/0} \node[arrnode] at (2+\x*0.7, 3.5) {\val};

    \node at (-1, 2.5) {Pass 1:};
    \foreach \x/\val in {0/-1, 1/3, 2/25, 3/8, 4/6, 5/55, 6/0} \node[arrnode] at (2+\x*0.7, 2.5) {\val};
    \node at (8, 2.5) {Swap -1 \& 6};

    \node at (-1, 1.5) {Pass 2:};
    \foreach \x/\val in {0/-1, 1/0, 2/25, 3/8, 4/6, 5/55, 6/3} \node[arrnode] at (2+\x*0.7, 1.5) {\val};
    \node at (8, 1.5) {Swap 0 \& 3};

    \node at (3, 0.5) {... continues ...};
\end{tikzpicture}
\captionof{figure}{Selection Sort Trace}
\end{center}

\begin{lstlisting}[language=Python]
def selection_sort(arr):
    n = len(arr)
    
    for i in range(n):
        # બાકીના અનસોર્ટેડ એરેમાં લઘુતમ એલિમેન્ટ શોધો
        min_idx = i
        for j in range(i+1, n):
            if arr[j] < arr[min_idx]:
                min_idx = j
                
        # શોધેલા લઘુતમ એલિમેન્ટને પ્રથમ એલિમેન્ટ સાથે સ્વેપ કરો
        arr[i], arr[min_idx] = arr[min_idx], arr[i]
    
    return arr

# ઇનપુટ લિસ્ટ
list1 = [6, 3, 25, 8, -1, 55, 0]

# લિસ્ટ સોર્ટ કરવી
sorted_list = selection_sort(list1)

# રિઝલ્ટ ડિસ્પ્લે કરવું
print("સોર્ટેડ લિસ્ટ:", sorted_list)
# આઉટપુટ: સોર્ટેડ લિસ્ટ: [-1, 0, 3, 6, 8, 25, 55]
\end{lstlisting}
\end{solutionbox}

\begin{mnemonicbox}
\mnemonic{નાનામાં નાનું શોધો, આગળ મૂકો}
\end{mnemonicbox}

\questionmarks{5(a)}{3}{ટ્રી ડેટા સ્ટ્રક્ચરના સંદર્ભમાં નીચેના પદો વ્યાખ્યાયિત કરો: \newline
\textbf{૧. ફોરેસ્ટ} \newline
\textbf{૨. રૂટ નોડ} \newline
\textbf{૩. લીફ નોડ}}

\begin{solutionbox}
\begin{center}
\captionof{table}{ટ્રી પરિભાષા}
\begin{tabulary}{\linewidth}{|L|L|}
\hline
\textbf{શબ્દ} & \textbf{વ્યાખ્યા} \\ \hline
\textbf{ફોરેસ્ટ} & ડિસજોઇન્ટ ટ્રીનો સંગ્રહ (એકબીજા સાથે જોડાણ વિનાના અનેક ટ્રી) \\ \hline
\textbf{રૂટ નોડ} & પેરેન્ટ વગરનો ટ્રીનો સૌથી ઉપરનો નોડ, જ્યાંથી અન્ય તમામ નોડ્સ ઉતરી આવે છે \\ \hline
\textbf{લીફ નોડ} & જે નોડને કોઈ ચિલ્ડ્રન નથી (ટ્રીના તળિયે ટર્મિનલ નોડ) \\ \hline
\end{tabulary}
\end{center}

\begin{center}
\begin{tikzpicture}[
    treenode/.style={circle, draw, minimum size=0.6cm},
    scale=0.8, transform shape
]
    % Forest
    \node[anchor=west] at (-2, 2) {Forest:};
    \node[treenode] (t1) at (0, 2) {T1}; \node[treenode] (t1a) at (-0.5, 1.2) {}; \node[treenode] (t1b) at (0.5, 1.2) {}; \draw (t1)--(t1a); \draw (t1)--(t1b);
    \node[treenode] (t2) at (2, 2) {T2}; \node[treenode] (t2a) at (2, 1.2) {}; \draw (t2)--(t2a);
    \node[treenode] (t3) at (4, 2) {T3};

    % Root
    \node[anchor=west] at (-2, 0) {Root:};
    \node[treenode, fill=yellow!20] (r) at (0, 0) {R};
    \node[treenode] (c1) at (-0.8, -1) {A}; \node[treenode] (c2) at (0.8, -1) {B};
    \draw (r)--(c1); \draw (r)--(c2);
    \node[left] at (r) {Root};

    % Leaf
    \node[anchor=west] at (2, 0) {Leaf:};
    \node[treenode] (p) at (4, 0) {P};
    \node[treenode, fill=green!20] (l1) at (3.2, -1) {L1}; \node[treenode, fill=green!20] (l2) at (4.8, -1) {L2};
    \draw (p)--(l1); \draw (p)--(l2);
    \node[below] at (l1) {Leaf}; \node[below] at (l2) {Leaf};
\end{tikzpicture}
\captionof{figure}{ટ્રી ટર્મ્સ}
\end{center}
\end{solutionbox}

\begin{mnemonicbox}
\mnemonic{ફોરેસ્ટમાં ઘણા મૂળ, મૂળ બધાને દોરે}
\end{mnemonicbox}

\questionmarks{5(b)}{4}{78, 58, 82, 15, 66, 80, 99 માટે બાઈનરી સર્ચ ટ્રી દોરો અને ટ્રી માટે ઈન-ઓર્ડર ટ્રાવર્સલ લખો.}

\begin{solutionbox}
\begin{center}
\begin{tikzpicture}[
    level distance=1.5cm,
    level 1/.style={sibling distance=4cm},
    level 2/.style={sibling distance=2cm},
    every node/.style={circle, draw, minimum size=0.8cm}
]
    \node {78}
        child {node {58}
            child {node {15}}
            child {node {66}}
        }
        child {node {82}
            child {node {80}}
            child {node {99}}
        };
\end{tikzpicture}
\captionof{figure}{આપેલ ડેટા માટે બાઈનરી સર્ચ ટ્રી}
\end{center}

\textbf{ઈન-ઓર્ડર ટ્રાવર્સલ:}
\begin{center}
\begin{tabulary}{\linewidth}{|C|L|}
\hline
\textbf{સ્ટેપ} & \textbf{મુલાકાત ક્રમ} \\ \hline
1 & 78 નો ડાબો સબટ્રી મુલાકાત લો \\ \hline
2 & 58 નો ડાબો સબટ્રી મુલાકાત લો \\ \hline
3 & 15 ની મુલાકાત લો \\ \hline
4 & 58 ની મુલાકાત લો \\ \hline
5 & 66 ની મુલાકાત લો \\ \hline
6 & 78 ની મુલાકાત લો \\ \hline
7 & 82 નો ડાબો સબટ્રી મુલાકાત લો \\ \hline
8 & 80 ની મુલાકાત લો \\ \hline
9 & 82 ની મુલાકાત લો \\ \hline
10 & 99 ની મુલાકાત લો \\ \hline
\end{tabulary}
\end{center}

\textbf{ઈન-ઓર્ડર ટ્રાવર્સલ પરિણામ: 15, 58, 66, 78, 80, 82, 99}
\end{solutionbox}

\begin{mnemonicbox}
\mnemonic{ડાબે, મૂળ, જમણે}
\end{mnemonicbox}

\questionmarks{5(c)}{7}{નીચેના ઓપરેશન માટે અલગોરિધમ લખો: \newline
\textbf{૧. બાઈનરી ટ્રીમાં નોડનું ઇન્સર્શન} \newline
\textbf{૨. બાઈનરી ટ્રીમાં નોડનું ડિલીશન}}

\begin{solutionbox}
\textbf{ઇન્સર્શન અલ્ગોરિધમ:}
\begin{center}
\begin{tikzpicture}[node distance=1.5cm]
    \node[gtu block] (start) {શરૂ};
    \node[gtu block, right=of start] (create) {નોડ બનાવો};
    \node[gtu decision, right=of create] (check) {Root NULL?};
    \node[gtu block, below=of check] (setroot) {સેટ Root};
    \node[gtu block, right=of check] (find) {ખાલી શોધો (Bfs)};
    \node[gtu block, below=of find] (insert) {ઇન્સર્ટ કરો};
    \node[gtu state, below=of setroot] (end) {અંત};

    \path[gtu arrow] (start) -- (create);
    \path[gtu arrow] (create) -- (check);
    \path[gtu arrow] (check) -- node[right] {હા} (setroot);
    \path[gtu arrow] (check) -- node[above] {ના} (find);
    \path[gtu arrow] (find) -- (insert);
    \path[gtu arrow] (setroot) -- (end);
    \path[gtu arrow] (insert) |- (end);
\end{tikzpicture}
\end{center}

\textbf{ડિલીશન અલ્ગોરિધમ:}
\begin{center}
\begin{tikzpicture}[node distance=1.5cm]
    \node[gtu block] (start) {શરૂ};
    \node[gtu decision, right=of start] (empty) {ટ્રી ખાલી?};
    \node[gtu state, below=of empty] (ret) {રિટર્ન};
    \node[gtu block, right=of empty] (find) {નોડ શોધો};
    \node[gtu block, right=of find] (deep) {ડીપેસ્ટ રાઈટ શોધો};
    \node[gtu block, below=of deep] (repl) {ડેટા રિપ્લેસ કરો};
    \node[gtu block, left=of repl] (del) {ડીપેસ્ટ ડિલીટ કરો};
    \node[gtu state, left=of del] (end) {અંત};

    \path[gtu arrow] (start) -- (empty);
    \path[gtu arrow] (empty) -- node[right] {હા} (ret);
    \path[gtu arrow] (empty) -- node[above] {ના} (find);
    \path[gtu arrow] (find) -- (deep);
    \path[gtu arrow] (deep) -- (repl);
    \path[gtu arrow] (repl) -- (del);
    \path[gtu arrow] (del) -- (end);
\end{tikzpicture}
\end{center}

\begin{lstlisting}[language=Python]
class Node:
    def __init__(self, data):
        self.data = data
        self.left = None
        self.right = None

# બાઈનરી ટ્રીમાં ઇન્સર્શન
def insert(root, data):
    if root is None:
        return Node(data)
    
    # ખાલી જગ્યા શોધવા માટે લેવલ ઓર્ડર ટ્રાવર્સલ
    queue = []
    queue.append(root)
    
    while queue:
        temp = queue.pop(0)
        
        if temp.left is None:
            temp.left = Node(data)
            break
        else:
            queue.append(temp.left)
            
        if temp.right is None:
            temp.right = Node(data)
            break
        else:
            queue.append(temp.right)
    
    return root

# બાઈનરી ટ્રીમાં ડિલીશન
def delete_node(root, key):
    if root is None:
        return None
    
    if root.left is None and root.right is None:
        if root.data == key:
            return None
        else:
            return root
    
    # ડિલીટ કરવા માટેનો નોડ અને ડીપેસ્ટ નોડ શોધો
    key_node = None
    last = None
    parent = None
    queue = []
    queue.append(root)
    
    while queue:
        temp = queue.pop(0)
        if temp.data == key:
            key_node = temp
        if temp.left:
            parent = temp
            queue.append(temp.left)
            last = temp.left
        if temp.right:
            parent = temp
            queue.append(temp.right)
            last = temp.right
    
    if key_node:
        key_node.data = last.data
        if parent.right == last:
            parent.right = None
        else:
            parent.left = None
    
    return root
\end{lstlisting}
\end{solutionbox}

\begin{mnemonicbox}
\mnemonic{ખાલી જગ્યાએ ઉમેરો, સ્વેપ કરીને કાઢો}
\end{mnemonicbox}

\questionmarks{5(a) OR}{3}{ટ્રી ડેટા સ્ટ્રક્ચરના સંદર્ભમાં નીચેના પદો વ્યાખ્યાયિત કરો: \newline
\textbf{૧. ઇન-ડિગ્રી} \newline
\textbf{૨. આઉટ-ડિગ્રી} \newline
\textbf{૩. ડેપ્થ}}

\begin{solutionbox}
\begin{center}
\captionof{table}{વ્યાખ્યાઓ}
\begin{tabulary}{\linewidth}{|L|L|}
\hline
\textbf{શબ્દ} & \textbf{વ્યાખ્યા} \\ \hline
\textbf{ઇન-ડિગ્રી} & નોડમાં આવતી એજેસની સંખ્યા (ટ્રીમાં રૂટ નોડ સિવાય દરેક નોડ માટે હંમેશા 1 હોય છે) \\ \hline
\textbf{આઉટ-ડિગ્રી} & નોડમાંથી બહાર જતી એજેસની સંખ્યા (ચિલ્ડ્રનની સંખ્યા) \\ \hline
\textbf{ડેપ્થ} & રૂટથી નોડ સુધીના પાથની લંબાઈ (પાથમાં એજેસની સંખ્યા) \\ \hline
\end{tabulary}
\end{center}

\begin{center}
\begin{tikzpicture}[
    level distance=1.5cm,
    level 1/.style={sibling distance=4cm},
    level 2/.style={sibling distance=2cm},
    every node/.style={circle, draw}
]
    \node (a) {A}
        child {node (b) {B}
            child {node (d) {D}}
            child {node (e) {E}}
        }
        child {node (c) {C}
            child {node (f) {F}}
        };
    
    \node[right=0.5cm of a] {Root (Depth 0)};
    \node[right=0.5cm of c] {Depth 1};
    \node[right=0.5cm of f] {Depth 2};
\end{tikzpicture}
\captionof{figure}{ટ્રી ડેપ્થ અને ડિગ્રી}
\end{center}

\begin{center}
\captionof{table}{ડિગ્રી વિશ્લેષણ}
\begin{tabulary}{\linewidth}{|C|C|C|}
\hline
\textbf{નોડ} & \textbf{ઇન-ડિગ્રી} & \textbf{આઉટ-ડિગ્રી} \\ \hline
A & 0 & 2 \\ \hline
B & 1 & 2 \\ \hline
C & 1 & 1 \\ \hline
D & 1 & 0 \\ \hline
E & 1 & 0 \\ \hline
F & 1 & 0 \\ \hline
\end{tabulary}
\end{center}
\end{solutionbox}

\begin{mnemonicbox}
\mnemonic{ઇન પેરેન્ટ ગણે, આઉટ ચાઈલ્ડ ગણે, ડેપ્થ રૂટથી એજેસ ગણે}
\end{mnemonicbox}

\questionmarks{5(b) OR}{4}{નીચેના બાઈનરી ટ્રી માટે પ્રી-ઓર્ડર અને પોસ્ટ-ઓર્ડર ટ્રાવર્સલ લખો. \newline
\textbf{100 -> (20 -> (10, 30), 200 -> (150, 300))}}

\begin{solutionbox}
\begin{center}
\begin{tikzpicture}[
    level distance=1.2cm,
    level 1/.style={sibling distance=4cm},
    level 2/.style={sibling distance=2cm},
    every node/.style={circle, draw}
]
    \node {100}
        child {node {20}
            child {node {10}}
            child {node {30}}
        }
        child {node {200}
            child {node {150}}
            child {node {300}}
        };
\end{tikzpicture}
\captionof{figure}{આપેલ બાઈનરી ટ્રી}
\end{center}

\begin{center}
\captionof{table}{ટ્રાવર્સલ્સ}
\begin{tabulary}{\linewidth}{|L|L|L|}
\hline
\textbf{ટ્રાવર્સલ} & \textbf{ક્રમ} & \textbf{પરિણામ} \\ \hline
\textbf{પ્રી-ઓર્ડર} & રૂટ, લેફ્ટ, રાઈટ & 100, 20, 10, 30, 200, 150, 300 \\ \hline
\textbf{પોસ્ટ-ઓર્ડર} & લેફ્ટ, રાઈટ, રૂટ & 10, 30, 20, 150, 300, 200, 100 \\ \hline
\end{tabulary}
\end{center}
\end{solutionbox}

\begin{mnemonicbox}
\mnemonic{પ્રી: રૂટ પહેલા, પોસ્ટ: ચિલ્ડ્રન પહેલા}
\end{mnemonicbox}

\questionmarks{5(c) OR}{7}{બાઈનરી સર્ચ ટ્રીના કન્સ્ટ્રકશન માટે પ્રોગ્રામ ડેવલપ કરો.}

\begin{solutionbox}
\textbf{કન્સ્ટ્રકશન વિઝ્યુઅલાઈઝેશન:}
\begin{center}
\begin{tikzpicture}[
    every node/.style={circle, draw, minimum size=0.8cm},
    level distance=1.2cm,
    level 1/.style={sibling distance=2.5cm},
    level 2/.style={sibling distance=1.5cm}
]
    \node {50}
        child {node {30}
            child {node {20}}
            child {node {40}}
        }
        child {node {70}
            child {node {60}}
            child {node {80}}
        };
\end{tikzpicture}
\captionof{figure}{[50, 30, 20, 40, 70, 60, 80] માંથી બનાવેલ BST}
\end{center}

\begin{lstlisting}[language=Python]
class Node:
    def __init__(self, key):
        self.key = key
        self.left = None
        self.right = None

def insert(root, key):
    # જો ટ્રી ખાલી હોય, તો નવો નોડ પરત કરો
    if root is None:
        return Node(key)
    
    # નહીંતર, ટ્રીમાં નીચે જાઓ
    if key < root.key:
        root.left = insert(root.left, key)
    else:
        root.right = insert(root.right, key)
    
    # યથાવત નોડ પોઇન્ટર પરત કરો
    return root

def inorder(root):
    if root:
        inorder(root.left)
        print(root.key, end=" ")
        inorder(root.right)

def preorder(root):
    if root:
        print(root.key, end=" ")
        preorder(root.left)
        preorder(root.right)

def postorder(root):
    if root:
        postorder(root.left)
        postorder(root.right)
        print(root.key, end=" ")

# ડ્રાઈવર પ્રોગ્રામ
def main():
    # આ એલિમેન્ટ્સ સાથે BST બનાવો: 50, 30, 20, 40, 70, 60, 80
    root = None
    elements = [50, 30, 20, 40, 70, 60, 80]
    
    for element in elements:
        root = insert(root, element)
    
    # ટ્રાવર્સલ્સ પ્રિન્ટ કરો
    print("Inorder traversal: ", end="")
    inorder(root)
    print("\nPreorder traversal: ", end="")
    preorder(root)
    print("\nPostorder traversal: ", end="")
    postorder(root)

# પ્રોગ્રામ રન કરો
main()
\end{lstlisting}

\textbf{ઉદાહરણ આઉટપુટ:}
\begin{lstlisting}
Inorder traversal: 20 30 40 50 60 70 80
Preorder traversal: 50 30 20 40 70 60 80
Postorder traversal: 20 40 30 60 80 70 50
\end{lstlisting}
\end{solutionbox}

\begin{mnemonicbox}
\mnemonic{નાના ડાબે, મોટા જમણે}
\end{mnemonicbox}

\end{document}



