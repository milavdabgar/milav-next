\documentclass[10pt,a4paper]{article}

% content/resources/templates/preamble.tex
\usepackage[margin=0.6in]{geometry}
\author{Milav Dabgar}
\usepackage{amsmath,amssymb,amsthm}
\usepackage{booktabs}
\usepackage{multirow}
\usepackage{xcolor}
\usepackage{tcolorbox}
\tcbuselibrary{breakable,skins}
\usepackage[colorlinks=true,linkcolor=blue]{hyperref}
\usepackage{titlesec}
\usepackage{enumitem}
\usepackage{tikz}
\usepackage{pgfplots}
\usepackage{circuitikz}
\usepackage[version=4]{mhchem}
\usepackage{longtable}
\usepackage{array}
\usepackage{float}
\usepackage{caption}
\usepackage{listings}

\lstset{
  basicstyle=\small\ttfamily,
  breaklines=true,
  breakatwhitespace=false,
  postbreak=\mbox{\textcolor{red}{$\hookrightarrow$}\space},
  float=false,
  numbers=left,
  numberstyle=\tiny\color{gray},
  numbersep=10pt,
  xleftmargin=2em,
  keywordstyle=\color{blue},
  commentstyle=\color{green!60!black},
  stringstyle=\color{purple},
  backgroundcolor=\color{gray!5},
  showstringspaces=false,
  tabsize=2,
  captionpos=b,
  keepspaces=true,
  columns=flexible
}

\pgfplotsset{compat=1.18}
\usetikzlibrary{shapes,arrows,positioning,calc,patterns,decorations.pathmorphing,decorations.markings,arrows.meta}

% Color scheme
\definecolor{headcolor}{RGB}{0,102,204}
\definecolor{keycolor}{RGB}{220,20,60}
\definecolor{solutioncolor}{RGB}{34,139,34}
\definecolor{mnemoniccolor}{RGB}{148,0,211}
\definecolor{codecolor}{RGB}{0,0,100}

% Spacing
\setlength{\parskip}{3pt}
\setlist[itemize]{nosep}
\setlist[enumerate]{nosep}

% Title formatting
\titleformat{\section}{\Large\bfseries\color{headcolor}}{\thesection}{1em}{}
\titleformat{\subsection}{\large\bfseries\color{headcolor}}{\thesubsection}{1em}{}

% Pandoc tightlist compatibility
\providecommand{\tightlist}{%
  \setlength{\itemsep}{0pt}\setlength{\parskip}{0pt}}

% Pandoc longtable compatibility
\newcounter{none}
\def\thenone{}


% content/resources/templates/gujarati-boxes.tex
\usepackage{fontspec}
\usepackage{polyglossia}

% Set Gujarati as main language (document is primarily in Gujarati)
% Note: gloss-gujarati.ldf doesn't exist in polyglossia, but it will use hyphenation patterns
\setdefaultlanguage{gujarati}
\setotherlanguage{english}

% Configure Gujarati font properly
% Use Language=Default to prevent polyglossia from trying to add language-specific features
% that don't exist for Gujarati, which causes "empty feature" warnings
\newfontfamily\gujaratifont[Script=Gujarati,AutoFakeBold=2.5,AutoFakeSlant=0.3]{Noto Sans Gujarati}
\setmainfont[Script=Gujarati,AutoFakeBold=2.5,AutoFakeSlant=0.3]{Noto Sans Gujarati}
% Use Noto Sans Gujarati for monospace to support Gujarati in text
\setmonofont[Scale=0.9]{Noto Sans Gujarati}

% Configure English to use the same font
\newfontfamily\englishfont[Script=Gujarati,AutoFakeBold=2.5,AutoFakeSlant=0.3]{Noto Sans Gujarati}

% Translations for polyglossia
\gappto\captionsgujarati{
  \renewcommand{\tablename}{કોષ્ટક}
  \renewcommand{\figurename}{આકૃતિ}
}

% Helper for TikZ nodes to ensure Gujarati font
\newcommand{\gu}[1]{{\gujaratifont #1}}

% Custom environments
\newtcolorbox{solutionbox}{
    breakable,
    enhanced,
    colback=solutioncolor!5!white,
    colframe=solutioncolor!75!black,
    fonttitle=\bfseries,
    title=જવાબ
}

\newtcolorbox{solutionboxnobreak}{
 colback=solutioncolor!5!white,
 colframe=solutioncolor!75!black,
 fonttitle=\bfseries,
 title=જવાબ
}

\newtcolorbox{keyformula}{
 breakable,
 enhanced,
 colback=keycolor!5!white,
 colframe=keycolor!75!black,
 fonttitle=\bfseries,
 title=રાસાયણિક સમીકરણ/સૂત્ર
}

\newtcolorbox{mnemonicbox}{
 breakable,
 enhanced,
 colback=mnemoniccolor!5!white,
 colframe=mnemoniccolor!75!black,
 fonttitle=\bfseries,
 title=મેમરી ટ્રીક
}


\begin{document}

\begin{center}
{\Huge\bfseries\color{headcolor} Subject Name (Gujarati)}\\[5pt]
{\LARGE 1333203 -- Winter 2024}\\[3pt]
{\large Semester 1 Study Material}\\[3pt]
{\normalsize\textit{Detailed Solutions and Explanations}}
\end{center}

\vspace{10pt}

\subsection*{પ્રશ્ન 1(અ) [3
ગુણ]}\label{uxaaauxab0uxab6uxaa8-1uxa85-3-uxa97uxaa3}

\textbf{રેખીય ડેટા સ્ટ્રક્ચર્સના નામ લખો.}

\begin{solutionbox}

{\def\LTcaptype{none} % do not increment counter
\begin{longtable}[]{@{}l@{}}
\toprule\noalign{}
રેખીય ડેટા સ્ટ્રક્ચર્સ \\
\midrule\noalign{}
\endhead
\bottomrule\noalign{}
\endlastfoot
1. એરે (Array) \\
2. સ્ટેક (Stack) \\
3. ક્યુ (Queue) \\
4. લિંક્ડ લિસ્ટ (Linked List) \\
\end{longtable}
}

\end{solutionbox}
\begin{mnemonicbox}
``બધા વિદ્યાર્થીઓ લાઈનમાં ઊભા રહે છે''

\end{mnemonicbox}
\subsection*{પ્રશ્ન 1(બ) [4
ગુણ]}\label{uxaaauxab0uxab6uxaa8-1uxaac-4-uxa97uxaa3}

\textbf{ટાઇમ અને સ્પેસ કોમ્પલેક્ષીટી વ્યાખ્યાયીત કરો.}

\begin{solutionbox}

{\def\LTcaptype{none} % do not increment counter
\begin{longtable}[]{@{}
  >{\raggedright\arraybackslash}p{(\linewidth - 4\tabcolsep) * \real{0.4211}}
  >{\raggedright\arraybackslash}p{(\linewidth - 4\tabcolsep) * \real{0.3158}}
  >{\raggedright\arraybackslash}p{(\linewidth - 4\tabcolsep) * \real{0.2632}}@{}}
\toprule\noalign{}
\begin{minipage}[b]{\linewidth}\raggedright
કોમ્પ્લેક્સિટી પ્રકાર
\end{minipage} & \begin{minipage}[b]{\linewidth}\raggedright
વ્યાખ્યા
\end{minipage} & \begin{minipage}[b]{\linewidth}\raggedright
નોટેશન
\end{minipage} \\
\midrule\noalign{}
\endhead
\bottomrule\noalign{}
\endlastfoot
ટાઇમ કોમ્પ્લેક્સિટી & માપે છે કે ઇનપુટ સાઇઝ વધતાં એક્ઝિક્યુશન ટાઇમ કેવી રીતે વધે છે &
O(n), O(1), O(log n) \\
સ્પેસ કોમ્પ્લેક્સિટી & માપે છે કે ઇનપુટ સાઇઝ વધતાં મેમરી વપરાશ કેવી રીતે વધે છે & O(n),
O(1), O(log n) \\
\end{longtable}
}

\textbf{ડાયાગ્રામ:}

\begin{lstlisting}
+---------------+         +----------------+
| INPUT SIZE    |-------->| ALGORITHM      |
| (n)           |         |                |
+---------------+         +----------------+
                              |       |
                              v       v
                   +---------+       +----------+
                   | TIME    |       | SPACE    |
                   | O(n)    |       | O(n)     |
                   +---------+       +----------+
\end{lstlisting}

\end{solutionbox}
\begin{mnemonicbox}
``ટાઇમ સ્ટેપ્સ, સ્પેસ સ્ટોર્સ''

\end{mnemonicbox}
\subsection*{પ્રશ્ન 1(ક) [7
ગુણ]}\label{uxaaauxab0uxab6uxaa8-1uxa95-7-uxa97uxaa3}

\textbf{ઉદાહરણ સાથે ક્લાસ અને ઓબ્જેક્ટનો કોન્સેપ્ટ સમજાવો.}

\begin{solutionbox}

\textbf{ડાયાગ્રામ:}

\includegraphics[width=1\linewidth,height=\textheight,keepaspectratio]{mermaid-447442f9.pdf}

{\def\LTcaptype{none} % do not increment counter
\begin{longtable}[]{@{}
  >{\raggedright\arraybackslash}p{(\linewidth - 4\tabcolsep) * \real{0.3000}}
  >{\raggedright\arraybackslash}p{(\linewidth - 4\tabcolsep) * \real{0.4000}}
  >{\raggedright\arraybackslash}p{(\linewidth - 4\tabcolsep) * \real{0.3000}}@{}}
\toprule\noalign{}
\begin{minipage}[b]{\linewidth}\raggedright
કોન્સેપ્ટ
\end{minipage} & \begin{minipage}[b]{\linewidth}\raggedright
વ્યાખ્યા
\end{minipage} & \begin{minipage}[b]{\linewidth}\raggedright
ઉદાહરણ
\end{minipage} \\
\midrule\noalign{}
\endhead
\bottomrule\noalign{}
\endlastfoot
ક્લાસ & ઓબ્જેક્ટ બનાવવા માટેનો બ્લૂપ્રિન્ટ અથવા ટેમ્પલેટ & Student ક્લાસ જેમાં
properties (rollNo, name) અને methods (setData, displayData) છે \\
ઓબ્જેક્ટ & ક્લાસનું ચોક્કસ ડેટા ધરાવતું ઇન્સ્ટન્સ & student1 (rollNo=101,
name=``રાજ'') \\
\end{longtable}
}

\textbf{કોડ ઉદાહરણ:}

\begin{lstlisting}[language=Python]
class Student:
    def __init__(self):
        self.rollNo = 0
        self.name = ""
        
    def setData(self, r, n):
        self.rollNo = r
        self.name = n
        
    def displayData(self):
        print(self.rollNo, self.name)

# ઓબ્જેક્ટ બનાવવા
student1 = Student()
student1.setData(101, "રાજ")
\end{lstlisting}

\end{solutionbox}
\begin{mnemonicbox}
``ક્લાસ બનાવે, ઓબ્જેક્ટ વાપરે''

\end{mnemonicbox}
\subsection*{પ્રશ્ન 1(ક) OR [7
ગુણ]}\label{uxaaauxab0uxab6uxaa8-1uxa95-or-7-uxa97uxaa3}

\textbf{વિદ્યાર્થીઓના રેકોર્ડ્સ ને સંચાલિત કરવા માટેનો એક ક્લાસ બનાવો જેમા
વિદ્યાર્થીને ઉમેરવા તેમજ બાદ કરવા માટેની મેથડ હોય.}

\begin{solutionbox}

\textbf{ડાયાગ્રામ:}

\includegraphics[width=1\linewidth,height=\textheight,keepaspectratio]{mermaid-7911879e.pdf}

\textbf{કોડ:}

\begin{lstlisting}[language=Python]
class StudentManager:
    def __init__(self):
        self.students = []
        
    def addStudent(self, roll, name):
        student = Student()
        student.setData(roll, name)
        self.students.append(student)
        
    def removeStudent(self, roll):
        for i in range(len(self.students)):
            if self.students[i].rollNo == roll:
                self.students.pop(i)
                break
    
    def displayAll(self):
        for student in self.students:
            student.displayData()
\end{lstlisting}

\end{solutionbox}
\begin{mnemonicbox}
``ઉમેરો વધારે, કાઢો ઘટાડે''

\end{mnemonicbox}
\subsection*{પ્રશ્ન 2(અ) [3
ગુણ]}\label{uxaaauxab0uxab6uxaa8-2uxa85-3-uxa97uxaa3}

\textbf{ક્લાસમાં કન્સ્ટ્રક્ટરનું મહત્વ સમજાવો.}

\begin{solutionbox}

{\def\LTcaptype{none} % do not increment counter
\begin{longtable}[]{@{}l@{}}
\toprule\noalign{}
કન્સ્ટ્રક્ટરનું મહત્વ \\
\midrule\noalign{}
\endhead
\bottomrule\noalign{}
\endlastfoot
1. ઓબ્જેક્ટના ડેટા મેમ્બર્સને પ્રારંભિક મૂલ્ય આપે છે \\
2. ઓબ્જેક્ટ બનતી વખતે આપોઆપ કોલ થાય છે \\
3. અલગ અલગ પ્રકારના હોઈ શકે (ડિફોલ્ટ, પેરામીટરાઈઝ્ડ, કોપી) \\
\end{longtable}
}

\end{solutionbox}
\begin{mnemonicbox}
``શરૂઆત હંમેશા સારી''

\end{mnemonicbox}
\subsection*{પ્રશ્ન 2(બ) [4
ગુણ]}\label{uxaaauxab0uxab6uxaa8-2uxaac-4-uxa97uxaa3}

\textbf{સ્ટેક પર વિવિધ ઓપરેશન સમજાવો.}

\begin{solutionbox}

{\def\LTcaptype{none} % do not increment counter
\begin{longtable}[]{@{}lll@{}}
\toprule\noalign{}
ઓપરેશન & વર્ણન & ઉદાહરણ \\
\midrule\noalign{}
\endhead
\bottomrule\noalign{}
\endlastfoot
પુશ (Push) & ટોપ પર એલિમેન્ટ ઉમેરે છે & push(5) \\
પોપ (Pop) & ટોપ પરથી એલિમેન્ટ દૂર કરે છે & x = pop() \\
પીક/ટોપ (Peek/Top) & ટોપ એલિમેન્ટને દૂર કર્યા વગર જુએ છે & x = peek() \\
isEmpty & ચકાસે છે કે સ્ટેક ખાલી છે કે નહીં & if(isEmpty()) \\
\end{longtable}
}

\textbf{ડાયાગ્રામ:}

\begin{lstlisting}
     PUSH                     POP
      |                        ^
      v                        |
    +---+                    +---+
    | 5 |                    | 8 |
    +---+                    +---+
    | 7 |  PEEK/TOP ------>  | 7 |
    +---+                    +---+
    | 8 |                    | 2 |
    +---+                    +---+
\end{lstlisting}

\end{solutionbox}
\begin{mnemonicbox}
``નાખો કાઢો જુઓ''

\end{mnemonicbox}
\subsection*{પ્રશ્ન 2(ક) [7
ગુણ]}\label{uxaaauxab0uxab6uxaa8-2uxa95-7-uxa97uxaa3}

**પોસ્ટફિક્સ એક્સપ્રેશન ABC+*D/ નું મૂલ્યાંકન અલગોરિધમનું વર્ણન કરો.**

\begin{solutionbox}

\textbf{ડાયાગ્રામ:}

\begin{lstlisting}
Input: A B C + * D /

+---+---+---+---+---+---+---+---+
| A | B | C | + | * | D | / |   |
+---+---+---+---+---+---+---+---+
      Read left to right
\end{lstlisting}

{\def\LTcaptype{none} % do not increment counter
\begin{longtable}[]{@{}llll@{}}
\toprule\noalign{}
સ્ટેપ & સિમ્બોલ & એક્શન & સ્ટેક \\
\midrule\noalign{}
\endhead
\bottomrule\noalign{}
\endlastfoot
1 & A & સ્ટેક પર પુશ કરો & A \\
2 & B & સ્ટેક પર પુશ કરો & A,B \\
3 & C & સ્ટેક પર પુશ કરો & A,B,C \\
4 & + & B,C પોપ કરો; B+C પુશ કરો & A,B+C \\
5 & * & A,B+C પોપ કરો; A*(B+C) પુશ કરો & A*(B+C) \\
6 & D & સ્ટેક પર પુશ કરો & A*(B+C),D \\
7 & / & A\emph{(B+C),D પોપ કરો; A}(B+C)/D પુશ કરો & A*(B+C)/D \\
\end{longtable}
}

\end{solutionbox}
\begin{mnemonicbox}
``વાંચો, પુશ કરો, પોપ કરો, ગણતરી કરો''

\end{mnemonicbox}
\subsection*{પ્રશ્ન 2(અ) OR [3
ગુણ]}\label{uxaaauxab0uxab6uxaa8-2uxa85-or-3-uxa97uxaa3}

\textbf{સ્ટેક અને ક્યુ વચ્ચેનો તફાવત લખો.}

\begin{solutionbox}

{\def\LTcaptype{none} % do not increment counter
\begin{longtable}[]{@{}lll@{}}
\toprule\noalign{}
ફીચર & સ્ટેક & ક્યુ \\
\midrule\noalign{}
\endhead
\bottomrule\noalign{}
\endlastfoot
સિદ્ધાંત & LIFO (છેલ્લું આવે પહેલું જાય) & FIFO (પહેલું આવે પહેલું જાય) \\
ઓપરેશન & પુશ/પોપ & એનક્યુ/ડિક્યુ \\
એક્સેસ પોઈન્ટ્સ & એક છેડો (ટોપ) & બે છેડા (ફ્રન્ટ, રીઅર) \\
\end{longtable}
}

\end{solutionbox}
\begin{mnemonicbox}
``સ્ટેક છેલ્લું પહેલું, ક્યુ પહેલું પહેલું''

\end{mnemonicbox}
\subsection*{પ્રશ્ન 2(બ) OR [4
ગુણ]}\label{uxaaauxab0uxab6uxaa8-2uxaac-or-4-uxa97uxaa3}

\textbf{સર્ક્યુલર ક્યુ નો કોન્સેપ્ટ સમજાવો.}

\begin{solutionbox}

\textbf{ડાયાગ્રામ:}

\includegraphics[width=1\linewidth,height=\textheight,keepaspectratio]{mermaid-7e2eef04.pdf}

{\def\LTcaptype{none} % do not increment counter
\begin{longtable}[]{@{}ll@{}}
\toprule\noalign{}
ફીચર & વર્ણન \\
\midrule\noalign{}
\endhead
\bottomrule\noalign{}
\endlastfoot
સ્ટ્રક્ચર & છેડાઓ જોડાયેલ હોય તેવો લીનિયર ડેટા સ્ટ્રક્ચર \\
ફાયદો & ખાલી જગ્યાનો ફરીથી ઉપયોગ કરીને મેમરી કાર્યક્ષમ રીતે વાપરે છે \\
ઓપરેશન & એનક્યુ, ડિક્યુ (મોડ્યુલો ગણતરી સાથે) \\
\end{longtable}
}

\end{solutionbox}
\begin{mnemonicbox}
``સર્ક્યુલર ફ્રન્ટને રીઅર સાથે જોડે''

\end{mnemonicbox}
\subsection*{પ્રશ્ન 2(ક) OR [7
ગુણ]}\label{uxaaauxab0uxab6uxaa8-2uxa95-or-7-uxa97uxaa3}

\textbf{સિંગલી લિંક્ડ લિસ્ટમાં આપેલ નોડ પછી અને પહેલાં નવા નોડ દાખલ કરવાની
પ્રક્રિયાનું વર્ણન કરો.}

\begin{solutionbox}

\textbf{ડાયાગ્રામ:}

\begin{lstlisting}
Insert After Node X:
Before: A \rightarrow X \rightarrow B
After:  A \rightarrow X \rightarrow N \rightarrow B

Insert Before Node X:
Before: A \rightarrow X \rightarrow B
After:  A \rightarrow N \rightarrow X \rightarrow B
\end{lstlisting}

{\def\LTcaptype{none} % do not increment counter
\begin{longtable}[]{@{}
  >{\raggedright\arraybackslash}p{(\linewidth - 2\tabcolsep) * \real{0.6111}}
  >{\raggedright\arraybackslash}p{(\linewidth - 2\tabcolsep) * \real{0.3889}}@{}}
\toprule\noalign{}
\begin{minipage}[b]{\linewidth}\raggedright
ઇન્સર્શન
\end{minipage} & \begin{minipage}[b]{\linewidth}\raggedright
સ્ટેપ્સ
\end{minipage} \\
\midrule\noalign{}
\endhead
\bottomrule\noalign{}
\endlastfoot
નોડ X પછી & 1. નવો નોડ N બનાવો2. N નો next X ના next પર સેટ કરો3. X નો
next N પર સેટ કરો \\
નોડ X પહેલા & 1. નવો નોડ N બનાવો2. X પર પોઇન્ટ કરતો નોડ A શોધો3. N નો next
X પર સેટ કરો4. A નો next N પર સેટ કરો \\
\end{longtable}
}

\end{solutionbox}
\begin{mnemonicbox}
``પછી: લિંક બદલો, પહેલા: અગાઉનો શોધો''

\end{mnemonicbox}
\subsection*{પ્રશ્ન 3(અ) [3
ગુણ]}\label{uxaaauxab0uxab6uxaa8-3uxa85-3-uxa97uxaa3}

\textbf{લિંક્ડ લિસ્ટ મા એક છેડાથી બીજા છેડા સુધી પસાર થવાની પ્રક્રિયા સમજાવો.}

\begin{solutionbox}

\textbf{ડાયાગ્રામ:}

\begin{lstlisting}
start \rightarrow [10] \rightarrow [20] \rightarrow [30] \rightarrow NULL
         ^      ^      ^
         |      |      |
       Visit  Visit  Visit
\end{lstlisting}

{\def\LTcaptype{none} % do not increment counter
\begin{longtable}[]{@{}ll@{}}
\toprule\noalign{}
સ્ટેપ & એક્શન \\
\midrule\noalign{}
\endhead
\bottomrule\noalign{}
\endlastfoot
1 & હેડ નોડથી શરૂ કરો \\
2 & વર્તમાન નોડનો ડેટા એક્સેસ કરો \\
3 & પોઈન્ટરને આગળના નોડ પર ખસેડો \\
4 & NULL મળે ત્યાં સુધી દોહરાવો \\
\end{longtable}
}

\end{solutionbox}
\begin{mnemonicbox}
``શરૂ કરો, જુઓ, આગળ વધો, દોહરાવો''

\end{mnemonicbox}
\subsection*{પ્રશ્ન 3(બ) [4
ગુણ]}\label{uxaaauxab0uxab6uxaa8-3uxaac-4-uxa97uxaa3}

\textbf{ઇનફિક્સથી પોસ્ટફિક્સમાં એક્સપ્રેસનનું રૂપાંતર સમજાવો.}

\begin{solutionbox}

\textbf{ડાયાગ્રામ:}

\begin{lstlisting}
Infix:    A + B * C
Postfix:  A B C * +
\end{lstlisting}

{\def\LTcaptype{none} % do not increment counter
\begin{longtable}[]{@{}llll@{}}
\toprule\noalign{}
સ્ટેપ & એક્શન & સ્ટેક & આઉટપુટ \\
\midrule\noalign{}
\endhead
\bottomrule\noalign{}
\endlastfoot
1 & ડાબેથી જમણે સ્કેન કરો & & \\
2 & જો ઓપરેન્ડ હોય, તો આઉટપુટમાં ઉમેરો & & A \\
3 & જો ઓપરેટર હોય, તો ઉચ્ચ પ્રાધાન્યતા હોય તો પુશ કરો & + & A \\
4 & ઓછી પ્રાધાન્યતાવાળા ઓપરેટર પોપ કરો & + & A B \\
5 & વર્તમાન ઓપરેટર પુશ કરો & * & A B \\
6 & એક્સપ્રેશન પૂરું થાય ત્યાં સુધી ચાલુ રાખો & * & A B C \\
7 & બાકીના ઓપરેટર પોપ કરો & & A B C * + \\
\end{longtable}
}

\end{solutionbox}
\begin{mnemonicbox}
``ઓપરેટર પુશ-પોપ, ઓપરેન્ડ સીધા આઉટપુટમાં''

\end{mnemonicbox}
\subsection*{પ્રશ્ન 3(ક) [7
ગુણ]}\label{uxaaauxab0uxab6uxaa8-3uxa95-7-uxa97uxaa3}

\textbf{સિંગલી લિંક્ડ લિસ્ટની શરૂઆતનો અને અંતનો નોડ ડીલીટ કરવા માટેનો પ્રોગ્રામ
લખો.}

\begin{solutionbox}

\textbf{ડાયાગ્રામ:}

\begin{lstlisting}
Before:  Head \rightarrow [10] \rightarrow [20] \rightarrow [30] \rightarrow NULL
After:   Head \rightarrow [20] \rightarrow NULL
\end{lstlisting}

\textbf{કોડ:}

\begin{lstlisting}[language=Python]
class Node:
    def __init__(self, data):
        self.data = data
        self.next = None

class LinkedList:
    def __init__(self):
        self.head = None
    
    def deleteFirst(self):
        if self.head is None:
            return
        self.head = self.head.next
    
    def deleteLast(self):
        if self.head is None:
            return
        
        # જો માત્ર એક જ નોડ હોય
        if self.head.next is None:
            self.head = None
            return
            
        temp = self.head
        while temp.next.next:
            temp = temp.next
        
        temp.next = None
\end{lstlisting}

\end{solutionbox}
\begin{mnemonicbox}
``પહેલો: હેડ શિફ્ટ કરો, છેલ્લો: પાછલો શોધો''

\end{mnemonicbox}
\subsection*{પ્રશ્ન 3(અ) OR [3
ગુણ]}\label{uxaaauxab0uxab6uxaa8-3uxa85-or-3-uxa97uxaa3}

\textbf{લિંક્ડ લિસ્ટમાં કોઇ એલિમેન્ટ શોધવાની પ્રક્રિયા સમજાવો.}

\begin{solutionbox}

\textbf{ડાયાગ્રામ:}

\begin{lstlisting}
Head \rightarrow [10] \rightarrow [20] \rightarrow [30] \rightarrow NULL
         ^      ^      ^
       Check  Check  Check
\end{lstlisting}

{\def\LTcaptype{none} % do not increment counter
\begin{longtable}[]{@{}ll@{}}
\toprule\noalign{}
સ્ટેપ & વર્ણન \\
\midrule\noalign{}
\endhead
\bottomrule\noalign{}
\endlastfoot
1 & હેડ નોડથી શરૂ કરો \\
2 & વર્તમાન નોડના ડેટાને કી સાથે સરખાવો \\
3 & જો મેચ મળે, તો true રીટર્ન કરો \\
4 & નહીંતર, આગળના નોડ પર જાઓ અને રિપીટ કરો \\
\end{longtable}
}

\end{solutionbox}
\begin{mnemonicbox}
``શરૂ કરો, ચેક કરો, આગળ વધો, દોહરાવો''

\end{mnemonicbox}
\subsection*{પ્રશ્ન 3(બ) OR [4
ગુણ]}\label{uxaaauxab0uxab6uxaa8-3uxaac-or-4-uxa97uxaa3}

\textbf{સર્ક્યુલર લિંક્ડ લિસ્ટ નો કોન્સેપ્ટ સમજાવો.}

\begin{solutionbox}

\textbf{ડાયાગ્રામ:}

\includegraphics[width=1\linewidth,height=\textheight,keepaspectratio]{mermaid-8d2e3c9e.pdf}

{\def\LTcaptype{none} % do not increment counter
\begin{longtable}[]{@{}ll@{}}
\toprule\noalign{}
ફીચર & વર્ણન \\
\midrule\noalign{}
\endhead
\bottomrule\noalign{}
\endlastfoot
સ્ટ્રક્ચર & છેલ્લો નોડ પહેલા નોડને પોઇન્ટ કરે છે \\
ફાયદો & NULL પોઈન્ટર્સ નથી, સર્ક્યુલર ઓપરેશન માટે કાર્યક્ષમ \\
ટ્રાવર્સલ & અનંત લૂપ ટાળવા માટે વધારાની શરત જરૂરી \\
\end{longtable}
}

\end{solutionbox}
\begin{mnemonicbox}
``છેલ્લો પહેલાને જોડે''

\end{mnemonicbox}
\subsection*{પ્રશ્ન 3(ક) OR [7
ગુણ]}\label{uxaaauxab0uxab6uxaa8-3uxa95-or-7-uxa97uxaa3}

\textbf{લિસ્ટમાંથી બાઇનરી સર્ચનો ઉપયોગ કરીને કોઇ એક એલિમેન્ટ શોધવાનું અલગોરીધમ
સમજાવો.}

\begin{solutionbox}

\textbf{ડાયાગ્રામ:}

\includegraphics[width=1\linewidth,height=\textheight,keepaspectratio]{mermaid-2e9dfa01.pdf}

\textbf{કોડ:}

\begin{lstlisting}[language=Python]
def binarySearch(arr, key):
    low = 0
    high = len(arr) - 1
    
    while low <= high:
        mid = (low + high) // 2
        
        if arr[mid] == key:
            return mid
        elif arr[mid] < key:
            low = mid + 1
        else:
            high = mid - 1
            
    return -1
\end{lstlisting}

\end{solutionbox}
\begin{mnemonicbox}
``મધ્ય, તુલના, અડધું કાઢો''

\end{mnemonicbox}
\subsection*{પ્રશ્ન 4(અ) [3
ગુણ]}\label{uxaaauxab0uxab6uxaa8-4uxa85-3-uxa97uxaa3}

\textbf{લિંક્ડ લિસ્ટના ઉપયોગ લખો.}

\begin{solutionbox}

{\def\LTcaptype{none} % do not increment counter
\begin{longtable}[]{@{}l@{}}
\toprule\noalign{}
લિંક્ડ લિસ્ટના ઉપયોગ \\
\midrule\noalign{}
\endhead
\bottomrule\noalign{}
\endlastfoot
1. સ્ટેક અને ક્યુનો અમલીકરણ \\
2. ડાયનેમિક મેમરી એલોકેશન \\
3. ઇમેજ વ્યૂઅર (આગલી/પાછલી ઇમેજ) \\
\end{longtable}
}

\end{solutionbox}
\begin{mnemonicbox}
``ડેટા ડાયનેમિક સ્ટોર કરો''

\end{mnemonicbox}
\subsection*{પ્રશ્ન 4(બ) [4
ગુણ]}\label{uxaaauxab0uxab6uxaa8-4uxaac-4-uxa97uxaa3}

\textbf{સિંગલી અને ડબલી લિંક્ડ લિસ્ટ વચ્ચેનો તફાવત લખો.}

\begin{solutionbox}

{\def\LTcaptype{none} % do not increment counter
\begin{longtable}[]{@{}lll@{}}
\toprule\noalign{}
ફીચર & સિંગલી લિંક્ડ લિસ્ટ & ડબલી લિંક્ડ લિસ્ટ \\
\midrule\noalign{}
\endhead
\bottomrule\noalign{}
\endlastfoot
નોડ સ્ટ્રક્ચર & એક પોઈન્ટર (next) & બે પોઈન્ટર (next, prev) \\
ટ્રાવર્સલ & માત્ર ફોરવર્ડ & બંને દિશામાં \\
મેમરી & ઓછી મેમરી & વધુ મેમરી \\
ઓપરેશન & સરળ, ઓછો કોડ & જટિલ, વધુ ફ્લેક્સિબલ \\
\end{longtable}
}

\textbf{ડાયાગ્રામ:}

\begin{lstlisting}
Singly: [Data|Next] \rightarrow [Data|Next] \rightarrow [Data|Next]

Doubly: [Prev|Data|Next] ⟷ [Prev|Data|Next] ⟷ [Prev|Data|Next]
\end{lstlisting}

\end{solutionbox}
\begin{mnemonicbox}
``એક દિશા, બે દિશા''

\end{mnemonicbox}
\subsection*{પ્રશ્ન 4(ક) [7
ગુણ]}\label{uxaaauxab0uxab6uxaa8-4uxa95-7-uxa97uxaa3}

\textbf{સિલેક્શન સોર્ટ અલગોરીધમનો ઉપયોગ કરીને આંકડાઓને ચઢતા ક્રમમાં ગોઠવવાનો
પ્રોગ્રામ લખો.}

\begin{solutionbox}

\textbf{ડાયાગ્રામ:}

\begin{lstlisting}
Initial: [5, 3, 8, 1, 2]
Pass 1:  [1, 3, 8, 5, 2]  (Swap 5,1)
Pass 2:  [1, 2, 8, 5, 3]  (Swap 3,2)
Pass 3:  [1, 2, 3, 5, 8]  (Swap 8,3)
Pass 4:  [1, 2, 3, 5, 8]  (No swap)
\end{lstlisting}

\textbf{કોડ:}

\begin{lstlisting}[language=Python]
def selectionSort(arr):
    n = len(arr)
    
    for i in range(n):
        min_idx = i
        
        for j in range(i+1, n):
            if arr[j] < arr[min_idx]:
                min_idx = j
        
        # મિનિમમ એલિમેન્ટને પહેલા એલિમેન્ટ સાથે સ્વેપ કરો
        arr[i], arr[min_idx] = arr[min_idx], arr[i]
    
    return arr

# ઉદાહરણ
arr = [5, 3, 8, 1, 2]
sorted_arr = selectionSort(arr)
print(sorted_arr)  # આઉટપુટ: [1, 2, 3, 5, 8]
\end{lstlisting}

\end{solutionbox}
\begin{mnemonicbox}
``મિનિમમ શોધો, પોઝિશન બદલો''

\end{mnemonicbox}
\subsection*{પ્રશ્ન 4(અ) OR [3
ગુણ]}\label{uxaaauxab0uxab6uxaa8-4uxa85-or-3-uxa97uxaa3}

\textbf{બબલ સોર્ટ અલગોરીધમ સમજાવો.}

\begin{solutionbox}

\textbf{ડાયાગ્રામ:}

\includegraphics[width=1\linewidth,height=\textheight,keepaspectratio]{mermaid-e9d1cb7b.pdf}

{\def\LTcaptype{none} % do not increment counter
\begin{longtable}[]{@{}l@{}}
\toprule\noalign{}
મુખ્ય પોઈન્ટ્સ \\
\midrule\noalign{}
\endhead
\bottomrule\noalign{}
\endlastfoot
આસપાસના એલિમેન્ટની તુલના કરો \\
જો ખોટા ક્રમમાં હોય તો સ્વેપ કરો \\
દરેક પાસમાં મોટા એલિમેન્ટ છેવટે પહોંચે \\
\end{longtable}
}

\end{solutionbox}
\begin{mnemonicbox}
``મોટા બબલ ઉપર જાય''

\end{mnemonicbox}
\subsection*{પ્રશ્ન 4(બ) OR [4
ગુણ]}\label{uxaaauxab0uxab6uxaa8-4uxaac-or-4-uxa97uxaa3}

\textbf{લિનિયર અને બાઇનરી સર્ચ વચ્ચેનો તફાવત લખો.}

\begin{solutionbox}

{\def\LTcaptype{none} % do not increment counter
\begin{longtable}[]{@{}lll@{}}
\toprule\noalign{}
ફીચર & લિનિયર સર્ચ & બાઇનરી સર્ચ \\
\midrule\noalign{}
\endhead
\bottomrule\noalign{}
\endlastfoot
કાર્ય સિદ્ધાંત & ક્રમિક ચકાસણી & વિભાજન અને જીત \\
ટાઇમ કોમ્પ્લેક્સિટી & O(n) & O(log n) \\
ડેટા અરેન્જમેન્ટ & અનસોર્ટેડ અથવા સોર્ટેડ & સોર્ટેડ હોવું જરૂરી \\
શેના માટે સારું & નાના ડેટાસેટ & મોટા ડેટાસેટ \\
\end{longtable}
}

\end{solutionbox}
\begin{mnemonicbox}
``લિનિયર બધાને જુએ, બાઇનરી અડધું કાપે''

\end{mnemonicbox}
\subsection*{પ્રશ્ન 4(ક) OR [7
ગુણ]}\label{uxaaauxab0uxab6uxaa8-4uxa95-or-7-uxa97uxaa3}

\textbf{ક્વીક સોર્ટ અને મર્જ સોર્ટ સમજાવો.}

\begin{solutionbox}

\textbf{ક્વીક સોર્ટ:}

\includegraphics[width=1\linewidth,height=\textheight,keepaspectratio]{mermaid-fbc19991.pdf}

\textbf{મર્જ સોર્ટ:}

\includegraphics[width=1\linewidth,height=\textheight,keepaspectratio]{mermaid-16c29927.pdf}

{\def\LTcaptype{none} % do not increment counter
\begin{longtable}[]{@{}llll@{}}
\toprule\noalign{}
અલગોરિધમ & સિદ્ધાંત & સરેરાશ ટાઇમ & સ્પેસ કોમ્પ્લેક્સિટી \\
\midrule\noalign{}
\endhead
\bottomrule\noalign{}
\endlastfoot
ક્વીક સોર્ટ & પીવોટની આસપાસ પાર્ટિશનિંગ & O(n log n) & O(log n) \\
મર્જ સોર્ટ & વિભાજન, જીત, જોડાણ & O(n log n) & O(n) \\
\end{longtable}
}

\end{solutionbox}
\begin{mnemonicbox}
``ક્વીક વિભાજે, મર્જ જોડે''

\end{mnemonicbox}
\subsection*{પ્રશ્ન 5(અ) [3
ગુણ]}\label{uxaaauxab0uxab6uxaa8-5uxa85-3-uxa97uxaa3}

\textbf{પૂર્ણ બાઇનરી ટ્રી ની વ્યાખ્યા આપો.}

\begin{solutionbox}

\textbf{ડાયાગ્રામ:}

\begin{lstlisting}
    1
   / \
  2   3
 / \  /
4  5 6
\end{lstlisting}

{\def\LTcaptype{none} % do not increment counter
\begin{longtable}[]{@{}ll@{}}
\toprule\noalign{}
પ્રોપર્ટી & વર્ણન \\
\midrule\noalign{}
\endhead
\bottomrule\noalign{}
\endlastfoot
બધા લેવલ ભરેલા & છેલ્લા લેવલ સિવાય \\
છેલ્લુ લેવલ ડાબેથી ભરેલું & નોડ ડાબેથી જમણે એડ થાય \\
\end{longtable}
}

\end{solutionbox}
\begin{mnemonicbox}
``ડાબેથી જમણે, લેવલ દર લેવલ ભરો''

\end{mnemonicbox}
\subsection*{પ્રશ્ન 5(બ) [4
ગુણ]}\label{uxaaauxab0uxab6uxaa8-5uxaac-4-uxa97uxaa3}

\textbf{બાઇનરી ટ્રી મા ઇનઓર્ડર ટ્રાવર્સલ સમજાવો.}

\begin{solutionbox}

\textbf{ડાયાગ્રામ:}

\begin{lstlisting}
      A
     / \
    B   C
   / \
  D   E

Inorder: D \rightarrow B \rightarrow E \rightarrow A \rightarrow C
\end{lstlisting}

{\def\LTcaptype{none} % do not increment counter
\begin{longtable}[]{@{}ll@{}}
\toprule\noalign{}
સ્ટેપ & એક્શન \\
\midrule\noalign{}
\endhead
\bottomrule\noalign{}
\endlastfoot
1 & ડાબા સબટ્રી પર ટ્રાવર્સ કરો \\
2 & રૂટ નોડની મુલાકાત લો \\
3 & જમણા સબટ્રી પર ટ્રાવર્સ કરો \\
\end{longtable}
}

\textbf{કોડ:}

\begin{lstlisting}[language=Python]
def inorderTraversal(root):
    if root:
        inorderTraversal(root.left)
        print(root.data, end=" \rightarrow ")
        inorderTraversal(root.right)
\end{lstlisting}

\end{solutionbox}
\begin{mnemonicbox}
``ડાબું, રૂટ, જમણું''

\end{mnemonicbox}
\subsection*{પ્રશ્ન 5(ક) [7
ગુણ]}\label{uxaaauxab0uxab6uxaa8-5uxa95-7-uxa97uxaa3}

\textbf{બાઇનરી સર્ચ ટ્રી મા નોડ દાખલ કરવા માટેનો પ્રોગ્રામ લખો.}

\begin{solutionbox}

\textbf{ડાયાગ્રામ:}

\includegraphics[width=1\linewidth,height=\textheight,keepaspectratio]{mermaid-8a418ad2.pdf}

\textbf{કોડ:}

\begin{lstlisting}[language=Python]
class Node:
    def __init__(self, key):
        self.key = key
        self.left = None
        self.right = None

def insert(root, key):
    if root is None:
        return Node(key)
    
    if key < root.key:
        root.left = insert(root.left, key)
    else:
        root.right = insert(root.right, key)
        
    return root
\end{lstlisting}

\end{solutionbox}
\begin{mnemonicbox}
``તુલના કરો, મૂવ કરો, દાખલ કરો''

\end{mnemonicbox}
\subsection*{પ્રશ્ન 5(અ) OR [3
ગુણ]}\label{uxaaauxab0uxab6uxaa8-5uxa85-or-3-uxa97uxaa3}

\textbf{બાઇનરી સર્ચ ટ્રીની મૂળભૂત ખાસિયતો જણાવો.}

\begin{solutionbox}

{\def\LTcaptype{none} % do not increment counter
\begin{longtable}[]{@{}l@{}}
\toprule\noalign{}
બાઇનરી સર્ચ ટ્રીની ખાસિયતો \\
\midrule\noalign{}
\endhead
\bottomrule\noalign{}
\endlastfoot
1. ડાબા ચાઈલ્ડ નોડ \textless{} પેરેન્ટ નોડ \\
2. જમણા ચાઈલ્ડ નોડ \textgreater{} પેરેન્ટ નોડ \\
3. ડુપ્લિકેટ વેલ્યુ માન્ય નથી \\
\end{longtable}
}

\end{solutionbox}
\begin{mnemonicbox}
``ડાબે ઓછું, જમણે વધુ''

\end{mnemonicbox}
\subsection*{પ્રશ્ન 5(બ) OR [4
ગુણ]}\label{uxaaauxab0uxab6uxaa8-5uxaac-or-4-uxa97uxaa3}

\textbf{બાઇનરી ટ્રી મા પોસ્ટ ઓર્ડર ટ્રાવર્સલ સમજાવો.}

\begin{solutionbox}

\textbf{ડાયાગ્રામ:}

\begin{lstlisting}
      A
     / \
    B   C
   / \
  D   E

Postorder: D \rightarrow E \rightarrow B \rightarrow C \rightarrow A
\end{lstlisting}

{\def\LTcaptype{none} % do not increment counter
\begin{longtable}[]{@{}ll@{}}
\toprule\noalign{}
સ્ટેપ & એક્શન \\
\midrule\noalign{}
\endhead
\bottomrule\noalign{}
\endlastfoot
1 & ડાબા સબટ્રી પર ટ્રાવર્સ કરો \\
2 & જમણા સબટ્રી પર ટ્રાવર્સ કરો \\
3 & રૂટ નોડની મુલાકાત લો \\
\end{longtable}
}

\textbf{કોડ:}

\begin{lstlisting}[language=Python]
def postorderTraversal(root):
    if root:
        postorderTraversal(root.left)
        postorderTraversal(root.right)
        print(root.data, end=" \rightarrow ")
\end{lstlisting}

\end{solutionbox}
\begin{mnemonicbox}
``ડાબું, જમણું, રૂટ''

\end{mnemonicbox}
\subsection*{પ્રશ્ન 5(ક) OR [7
ગુણ]}\label{uxaaauxab0uxab6uxaa8-5uxa95-or-7-uxa97uxaa3}

\textbf{બાઇનરી સર્ચ ટ્રી માંથી નોડ ડીલીટ કરવા માટેનો પ્રોગ્રામ લખો.}

\begin{solutionbox}

\textbf{ડાયાગ્રામ:}

\includegraphics[width=1\linewidth,height=\textheight,keepaspectratio]{mermaid-9b3d628a.pdf}

\textbf{કોડ:}

\begin{lstlisting}[language=Python]
def minValueNode(node):
    current = node
    while current.left is not None:
        current = current.left
    return current

def deleteNode(root, key):
    if root is None:
        return root
        
    if key < root.key:
        root.left = deleteNode(root.left, key)
    elif key > root.key:
        root.right = deleteNode(root.right, key)
    else:
        # એક અથવા કોઈ ચાઈલ્ડ નથી
        if root.left is None:
            return root.right
        elif root.right is None:
            return root.left
            
        # બે ચાઈલ્ડ
        successor = minValueNode(root.right)
        root.key = successor.key
        root.right = deleteNode(root.right, successor.key)
        
    return root
\end{lstlisting}

\end{solutionbox}
\begin{mnemonicbox}
``શોધો, બદલો, ફરી જોડો''

\end{mnemonicbox}

\end{document}
