\documentclass[10pt,a4paper]{article}

% content/resources/templates/preamble.tex
\usepackage[margin=0.6in]{geometry}
\author{Milav Dabgar}
\usepackage{amsmath,amssymb,amsthm}
\usepackage{booktabs}
\usepackage{multirow}
\usepackage{xcolor}
\usepackage{tcolorbox}
\tcbuselibrary{breakable,skins}
\usepackage[colorlinks=true,linkcolor=blue]{hyperref}
\usepackage{titlesec}
\usepackage{enumitem}
\usepackage{tikz}
\usepackage{pgfplots}
\usepackage{circuitikz}
\usepackage[version=4]{mhchem}
\usepackage{longtable}
\usepackage{array}
\usepackage{float}
\usepackage{caption}
\usepackage{listings}

\lstset{
  basicstyle=\small\ttfamily,
  breaklines=true,
  breakatwhitespace=false,
  postbreak=\mbox{\textcolor{red}{$\hookrightarrow$}\space},
  float=false,
  numbers=left,
  numberstyle=\tiny\color{gray},
  numbersep=10pt,
  xleftmargin=2em,
  keywordstyle=\color{blue},
  commentstyle=\color{green!60!black},
  stringstyle=\color{purple},
  backgroundcolor=\color{gray!5},
  showstringspaces=false,
  tabsize=2,
  captionpos=b,
  keepspaces=true,
  columns=flexible
}

\pgfplotsset{compat=1.18}
\usetikzlibrary{shapes,arrows,positioning,calc,patterns,decorations.pathmorphing,decorations.markings,arrows.meta}

% Color scheme
\definecolor{headcolor}{RGB}{0,102,204}
\definecolor{keycolor}{RGB}{220,20,60}
\definecolor{solutioncolor}{RGB}{34,139,34}
\definecolor{mnemoniccolor}{RGB}{148,0,211}
\definecolor{codecolor}{RGB}{0,0,100}

% Spacing
\setlength{\parskip}{3pt}
\setlist[itemize]{nosep}
\setlist[enumerate]{nosep}

% Title formatting
\titleformat{\section}{\Large\bfseries\color{headcolor}}{\thesection}{1em}{}
\titleformat{\subsection}{\large\bfseries\color{headcolor}}{\thesubsection}{1em}{}

% Pandoc tightlist compatibility
\providecommand{\tightlist}{%
  \setlength{\itemsep}{0pt}\setlength{\parskip}{0pt}}

% Pandoc longtable compatibility
\newcounter{none}
\def\thenone{}


% content/resources/templates/english-boxes.tex
% This file is currently empty - it exists to maintain consistency with the import structure.
% Add custom environments here if needed in the future.


\begin{document}

\begin{center}
{\Huge\bfseries\color{headcolor} Subject Name Solutions}\\[5pt]
{\LARGE 1333203 -- Winter 2023}\\[3pt]
{\large Semester 1 Study Material}\\[3pt]
{\normalsize\textit{Detailed Solutions and Explanations}}
\end{center}

\vspace{10pt}

\subsection*{Question 1(a) [3 marks]}\label{q1a}

\textbf{Define linked list. List different types of linked list.}

\begin{solutionbox}

{\def\LTcaptype{none} % do not increment counter
\begin{longtable}[]{@{}
  >{\raggedright\arraybackslash}p{(\linewidth - 2\tabcolsep) * \real{0.3636}}
  >{\raggedright\arraybackslash}p{(\linewidth - 2\tabcolsep) * \real{0.6364}}@{}}
\toprule\noalign{}
\begin{minipage}[b]{\linewidth}\raggedright
Definition
\end{minipage} & \begin{minipage}[b]{\linewidth}\raggedright
Types of Linked List
\end{minipage} \\
\midrule\noalign{}
\endhead
\bottomrule\noalign{}
\endlastfoot
A linked list is a linear data structure where elements are stored in
nodes, and each node points to the next node in the sequence & 1. Singly
Linked List 2. Doubly Linked List 3. Circular Linked List 4. Circular
Doubly Linked List \\
\end{longtable}
}

\textbf{Diagram:}

\begin{lstlisting}
Singly:     [Data|Next] \rightarrow [Data|Next] \rightarrow [Data|Next] \rightarrow NULL
Doubly:     [Prev|Data|Next] ⟷ [Prev|Data|Next] ⟷ [Prev|Data|Next] \rightarrow NULL
Circular:   [Data|Next] \rightarrow [Data|Next] \rightarrow [Data|Next] ↩
\end{lstlisting}

\end{solutionbox}
\begin{mnemonicbox}
``Single, Double, Circle, Double-Circle''

\end{mnemonicbox}
\subsection*{Question 1(b) [4 marks]}\label{q1b}

\textbf{Explain Linear and Non Linear Data structure in Python with
examples.}

\begin{solutionbox}

{\def\LTcaptype{none} % do not increment counter
\begin{longtable}[]{@{}
  >{\raggedright\arraybackslash}p{(\linewidth - 4\tabcolsep) * \real{0.3556}}
  >{\raggedright\arraybackslash}p{(\linewidth - 4\tabcolsep) * \real{0.2889}}
  >{\raggedright\arraybackslash}p{(\linewidth - 4\tabcolsep) * \real{0.3556}}@{}}
\toprule\noalign{}
\begin{minipage}[b]{\linewidth}\raggedright
Data Structure
\end{minipage} & \begin{minipage}[b]{\linewidth}\raggedright
Description
\end{minipage} & \begin{minipage}[b]{\linewidth}\raggedright
Python Examples
\end{minipage} \\
\midrule\noalign{}
\endhead
\bottomrule\noalign{}
\endlastfoot
Linear & Elements arranged in sequential order where each element has
exactly one predecessor and successor (except first and last) & Lists:
\passthrough{\lstinline![1, 2, 3]!} Tuples:
\passthrough{\lstinline!(1, 2, 3)!} Strings:
\passthrough{\lstinline!"abc"!} Queue:
\passthrough{\lstinline!queue.Queue()!} \\
Non-Linear & Elements not arranged sequentially; an element can connect
to multiple elements & Dictionary:
\passthrough{\lstinline!\{"a": 1, "b": 2\}!} Set:
\passthrough{\lstinline!\{1, 2, 3\}!} Tree: Custom implementation Graph:
Custom implementation \\
\end{longtable}
}

\textbf{Diagram:}

\includegraphics[width=1\linewidth,height=\textheight,keepaspectratio]{mermaid-ca878d70.pdf}

\end{solutionbox}
\begin{mnemonicbox}
``Linear Listens In Sequence, Non-linear Navigates
Various Paths''

\end{mnemonicbox}
\subsection*{Question 1(c) [7 marks]}\label{q1c}

\textbf{Explain class, attributes, object and class method in python
with suitable example.}

\begin{solutionbox}

\textbf{Diagram:}

\includegraphics[width=1\linewidth,height=\textheight,keepaspectratio]{mermaid-4513c07f.pdf}

{\def\LTcaptype{none} % do not increment counter
\begin{longtable}[]{@{}
  >{\raggedright\arraybackslash}p{(\linewidth - 2\tabcolsep) * \real{0.3158}}
  >{\raggedright\arraybackslash}p{(\linewidth - 2\tabcolsep) * \real{0.6842}}@{}}
\toprule\noalign{}
\begin{minipage}[b]{\linewidth}\raggedright
Term
\end{minipage} & \begin{minipage}[b]{\linewidth}\raggedright
Description
\end{minipage} \\
\midrule\noalign{}
\endhead
\bottomrule\noalign{}
\endlastfoot
Class & Blueprint for creating objects with shared attributes and
methods \\
Attributes & Variables that store data inside a class \\
Object & Instance of a class with specific attribute values \\
Class Method & Functions defined within a class that can access and
modify class states \\
\end{longtable}
}

\textbf{Code:}

\begin{lstlisting}[language=Python]
class Student:
    # Class attribute
    school = "GTU"
    
    # Constructor
    def __init__(self, roll_no, name):
        # Instance attributes
        self.roll_no = roll_no
        self.name = name
    
    # Instance method
    def display(self):
        print(f"Roll No: {self.roll_no}, Name: {self.name}")
    
    # Class method
    @classmethod
    def change_school(cls, new_school):
        cls.school = new_school

# Creating object
student1 = Student(101, "Raj")
student1.display()  # Output: Roll No: 101, Name: Raj
\end{lstlisting}

\end{solutionbox}
\begin{mnemonicbox}
``Class Creates, Attributes Store, Objects Use,
Methods Operate''

\end{mnemonicbox}
\subsection*{Question 1(c) OR [7
marks]}\label{q1c}

\textbf{Define Data Encapsulation \& Polymorphism. Develop a Python code
to explain Polymorphism.}

\begin{solutionbox}

{\def\LTcaptype{none} % do not increment counter
\begin{longtable}[]{@{}
  >{\raggedright\arraybackslash}p{(\linewidth - 2\tabcolsep) * \real{0.4286}}
  >{\raggedright\arraybackslash}p{(\linewidth - 2\tabcolsep) * \real{0.5714}}@{}}
\toprule\noalign{}
\begin{minipage}[b]{\linewidth}\raggedright
Concept
\end{minipage} & \begin{minipage}[b]{\linewidth}\raggedright
Definition
\end{minipage} \\
\midrule\noalign{}
\endhead
\bottomrule\noalign{}
\endlastfoot
Data Encapsulation & Bundling data and methods into a single unit
(class) and restricting direct access to some components \\
Polymorphism & Ability of different classes to provide their own
implementation of methods with the same name \\
\end{longtable}
}

\textbf{Diagram:}

\includegraphics[width=1\linewidth,height=\textheight,keepaspectratio]{mermaid-1db3ebdf.pdf}

\textbf{Code:}

\begin{lstlisting}[language=Python]
# Polymorphism example
class Animal:
    def speak(self):
        pass

class Dog(Animal):
    def speak(self):
        return "Woof!"

class Cat(Animal):
    def speak(self):
        return "Meow!"

class Duck(Animal):
    def speak(self):
        return "Quack!"

# Function demonstrating polymorphism
def animal_sound(animal):
    return animal.speak()

# Creating objects
dog = Dog()
cat = Cat()
duck = Duck()

# Same function works for different animal objects
print(animal_sound(dog))   # Output: Woof!
print(animal_sound(cat))   # Output: Meow!
print(animal_sound(duck))  # Output: Quack!
\end{lstlisting}

\end{solutionbox}
\begin{mnemonicbox}
``Encapsulate to Protect, Polymorphism for
Flexibility''

\end{mnemonicbox}
\subsection*{Question 2(a) [3 marks]}\label{q2a}

\textbf{Differentiate between Stack and Queue.}

\begin{solutionbox}

{\def\LTcaptype{none} % do not increment counter
\begin{longtable}[]{@{}
  >{\raggedright\arraybackslash}p{(\linewidth - 4\tabcolsep) * \real{0.3913}}
  >{\raggedright\arraybackslash}p{(\linewidth - 4\tabcolsep) * \real{0.3043}}
  >{\raggedright\arraybackslash}p{(\linewidth - 4\tabcolsep) * \real{0.3043}}@{}}
\toprule\noalign{}
\begin{minipage}[b]{\linewidth}\raggedright
Feature
\end{minipage} & \begin{minipage}[b]{\linewidth}\raggedright
Stack
\end{minipage} & \begin{minipage}[b]{\linewidth}\raggedright
Queue
\end{minipage} \\
\midrule\noalign{}
\endhead
\bottomrule\noalign{}
\endlastfoot
Principle & LIFO (Last In First Out) & FIFO (First In First Out) \\
Operations & Push, Pop & Enqueue, Dequeue \\
Access & Elements can only be added/removed from one end (top) &
Elements are added at rear end and removed from front end \\
\end{longtable}
}

\textbf{Diagram:}

\begin{lstlisting}
Stack:       [3]      Queue:  [1] \rightarrow [2] \rightarrow [3]
             [2]              Front      Rear
             [1]      
             ---
\end{lstlisting}

\end{solutionbox}
\begin{mnemonicbox}
``Stack Piles Up, Queue Lines Up''

\end{mnemonicbox}
\subsection*{Question 2(b) [4 marks]}\label{q2b}

\textbf{Write an algorithm for PUSH and POP operation of stack in
python.}

\begin{solutionbox}

\textbf{PUSH Algorithm:}

\begin{lstlisting}
Start
  1. Check if stack is full
  2. If not full, increment top by 1
  3. Add element at position 'top'
End
\end{lstlisting}

\textbf{POP Algorithm:}

\begin{lstlisting}
Start
  1. Check if stack is empty
  2. If not empty, retrieve element at 'top'
  3. Decrement top by 1
  4. Return retrieved element
End
\end{lstlisting}

\textbf{Code:}

\begin{lstlisting}[language=Python]
class Stack:
    def __init__(self, size):
        self.stack = []
        self.size = size
        self.top = -1
    
    def push(self, element):
        if self.top >= self.size - 1:
            return "Stack Overflow"
        else:
            self.top += 1
            self.stack.append(element)
            return "Pushed " + str(element)
    
    def pop(self):
        if self.top < 0:
            return "Stack Underflow"
        else:
            element = self.stack.pop()
            self.top -= 1
            return element
\end{lstlisting}

\end{solutionbox}
\begin{mnemonicbox}
``Push to Top, Pop from Top''

\end{mnemonicbox}
\subsection*{Question 2(c) [7 marks]}\label{q2c}

\textbf{Convert following equation from infix to postfix using Stack.}
\textbf{A * (B + C) - D / (E + F)}

\begin{solutionbox}

\textbf{Diagram:}

\begin{lstlisting}
Infix:   A * (B + C) - D / (E + F)
Postfix: A B C + * D E F + / -
\end{lstlisting}

{\def\LTcaptype{none} % do not increment counter
\begin{longtable}[]{@{}llll@{}}
\toprule\noalign{}
Step & Symbol & Stack & Output \\
\midrule\noalign{}
\endhead
\bottomrule\noalign{}
\endlastfoot
1 & A & & A \\
2 & * & * & A \\
3 & ( & * ( & A \\
4 & B & * ( & A B \\
5 & + & * ( + & A B \\
6 & C & * ( + & A B C \\
7 & ) & * & A B C + \\
8 & - & - & A B C + * \\
9 & D & - & A B C + * D \\
10 & / & - / & A B C + * D \\
11 & ( & - / ( & A B C + * D \\
12 & E & - / ( & A B C + * D E \\
13 & + & - / ( + & A B C + * D E \\
14 & F & - / ( + & A B C + * D E F \\
15 & ) & - / & A B C + * D E F + \\
16 & end & & A B C + * D E F + / - \\
\end{longtable}
}

\end{solutionbox}
\begin{solutionbox}
\passthrough{\lstinline!A B C + * D E F + / -!}

\end{solutionbox}
\begin{mnemonicbox}
``Operators Stack, Operands Print''

\end{mnemonicbox}
\subsection*{Question 2(a) OR [3
marks]}\label{q2a}

\textbf{Differentiate between simple Queue and circular Queue.}

\begin{solutionbox}

{\def\LTcaptype{none} % do not increment counter
\begin{longtable}[]{@{}
  >{\raggedright\arraybackslash}p{(\linewidth - 4\tabcolsep) * \real{0.2308}}
  >{\raggedright\arraybackslash}p{(\linewidth - 4\tabcolsep) * \real{0.3590}}
  >{\raggedright\arraybackslash}p{(\linewidth - 4\tabcolsep) * \real{0.4103}}@{}}
\toprule\noalign{}
\begin{minipage}[b]{\linewidth}\raggedright
Feature
\end{minipage} & \begin{minipage}[b]{\linewidth}\raggedright
Simple Queue
\end{minipage} & \begin{minipage}[b]{\linewidth}\raggedright
Circular Queue
\end{minipage} \\
\midrule\noalign{}
\endhead
\bottomrule\noalign{}
\endlastfoot
Structure & Linear data structure & Linear data structure with connected
ends \\
Memory & Inefficient memory usage due to unused space after dequeue &
Efficient memory usage by reusing empty spaces \\
Implementation & Front always at index 0, rear increases & Front and
rear move in circular fashion using modulo \\
\end{longtable}
}

\textbf{Diagram:}

\includegraphics[width=1\linewidth,height=\textheight,keepaspectratio]{mermaid-91d8eea4.pdf}

\end{solutionbox}
\begin{mnemonicbox}
``Simple Wastes, Circular Reuses''

\end{mnemonicbox}
\subsection*{Question 2(b) OR [4
marks]}\label{q2b}

\textbf{Explain concept of recursive function with suitable example.}

\begin{solutionbox}

{\def\LTcaptype{none} % do not increment counter
\begin{longtable}[]{@{}
  >{\raggedright\arraybackslash}p{(\linewidth - 2\tabcolsep) * \real{0.5000}}
  >{\raggedright\arraybackslash}p{(\linewidth - 2\tabcolsep) * \real{0.5000}}@{}}
\toprule\noalign{}
\begin{minipage}[b]{\linewidth}\raggedright
Key Aspects
\end{minipage} & \begin{minipage}[b]{\linewidth}\raggedright
Description
\end{minipage} \\
\midrule\noalign{}
\endhead
\bottomrule\noalign{}
\endlastfoot
Definition & A function that calls itself to solve a smaller instance of
the same problem \\
Base Case & The condition where the function stops calling itself \\
Recursive Case & The condition where the function calls itself with a
simpler version of the problem \\
\end{longtable}
}

\textbf{Diagram:}

\includegraphics[width=1\linewidth,height=\textheight,keepaspectratio]{mermaid-01ca7293.pdf}

\textbf{Code:}

\begin{lstlisting}[language=Python]
def factorial(n):
    # Base case
if

n == 0:

        return 1
    # Recursive case
    else:
        return n * factorial(n-1)

# Example
result = factorial(5)  # 5! = 120
\end{lstlisting}

\end{solutionbox}
\begin{mnemonicbox}
``Base Breaks, Recursion Returns''

\end{mnemonicbox}
\subsection*{Question 2(c) OR [7
marks]}\label{q2c}

\textbf{Develop a python code to implement Enqueue and Dequeue operation
in Queue.}

\begin{solutionbox}

\textbf{Diagram:}

\begin{lstlisting}
Enqueue:
  [1][2][3] \rightarrow [1][2][3][4]
  
Dequeue:
  [1][2][3][4] \rightarrow [2][3][4]
\end{lstlisting}

\textbf{Code:}

\begin{lstlisting}[language=Python]
class Queue:
    def __init__(self, size):
        self.queue = []
        self.size = size
        self.front = 0
        self.rear = -1
        self.count = 0
    
    def enqueue(self, item):
        if self.count >= self.size:
            return "Queue is full"
        else:
            self.rear += 1
            self.queue.append(item)
            self.count += 1
            return "Enqueued " + str(item)
    
    def dequeue(self):
        if self.count <= 0:
            return "Queue is empty"
        else:
            item = self.queue.pop(0)
            self.count -= 1
            return item
    
    def display(self):
        return self.queue

# Test
q = Queue(5)
q.enqueue(10)
q.enqueue(20)
q.enqueue(30)
print(q.display())  # [10, 20, 30]
print(q.dequeue())  # 10
print(q.display())  # [20, 30]
\end{lstlisting}

\end{solutionbox}
\begin{mnemonicbox}
``Enqueue at End, Dequeue from Start''

\end{mnemonicbox}
\subsection*{Question 3(a) [3 marks]}\label{q3a}

\textbf{Give Difference between Singly linked list and Circular linked
list.}

\begin{solutionbox}

{\def\LTcaptype{none} % do not increment counter
\begin{longtable}[]{@{}lll@{}}
\toprule\noalign{}
Feature & Singly Linked List & Circular Linked List \\
\midrule\noalign{}
\endhead
\bottomrule\noalign{}
\endlastfoot
Last Node & Points to NULL & Points back to the first node \\
Traversal & Has a definite end & Can be traversed continuously \\
Memory & Each node needs one pointer & Each node needs one pointer \\
\end{longtable}
}

\textbf{Diagram:}

\begin{lstlisting}
Singly:   [1] \rightarrow [2] \rightarrow [3] \rightarrow NULL
Circular: [1] \rightarrow [2] \rightarrow [3] \rightarrow ↩
\end{lstlisting}

\end{solutionbox}
\begin{mnemonicbox}
``Singly Stops, Circular Cycles''

\end{mnemonicbox}
\subsection*{Question 3(b) [4 marks]}\label{q3b}

\textbf{Explain concept of Doubly linked list.}

\begin{solutionbox}

\textbf{Diagram:}

\begin{lstlisting}
NULL \leftarrow [Prev|1|Next] ⟷ [Prev|2|Next] ⟷ [Prev|3|Next] \rightarrow NULL
\end{lstlisting}

{\def\LTcaptype{none} % do not increment counter
\begin{longtable}[]{@{}
  >{\raggedright\arraybackslash}p{(\linewidth - 2\tabcolsep) * \real{0.4091}}
  >{\raggedright\arraybackslash}p{(\linewidth - 2\tabcolsep) * \real{0.5909}}@{}}
\toprule\noalign{}
\begin{minipage}[b]{\linewidth}\raggedright
Feature
\end{minipage} & \begin{minipage}[b]{\linewidth}\raggedright
Description
\end{minipage} \\
\midrule\noalign{}
\endhead
\bottomrule\noalign{}
\endlastfoot
Node Structure & Each node contains data and two pointers (previous and
next) \\
Navigation & Can traverse in both forward and backward directions \\
Operations & Insertion and deletion can be performed from both ends \\
Memory Usage & Requires more memory than singly linked list due to extra
pointer \\
\end{longtable}
}

\textbf{Code:}

\begin{lstlisting}[language=Python]
class Node:
    def __init__(self, data):
        self.data = data
        self.prev = None
        self.next = None
\end{lstlisting}

\end{solutionbox}
\begin{mnemonicbox}
``Double Pointers, Double Directions''

\end{mnemonicbox}
\subsection*{Question 3(c) [7 marks]}\label{q3c}

\textbf{Write an algorithm for following operation on singly linked
list:} \textbf{1. To insert a node at the beginning of the list.}
\textbf{2. To insert the node at the end of the list.}

\begin{solutionbox}

\textbf{Insert at Beginning:}

\includegraphics[width=1\linewidth,height=\textheight,keepaspectratio]{mermaid-b01938dd.pdf}

\textbf{Insert at End:}

\includegraphics[width=1\linewidth,height=\textheight,keepaspectratio]{mermaid-4872ced8.pdf}

\textbf{Code:}

\begin{lstlisting}[language=Python]
def insert_at_beginning(head, data):
    new_node = Node(data)
    new_node.next = head
    return new_node  # New head

def insert_at_end(head, data):
    new_node = Node(data)
    new_node.next = None
    
    # If linked list is empty
    if head is None:
        return new_node
    
    # Traverse to the last node
    temp = head
    while temp.next:
        temp = temp.next
    
    # Link the last node to new node
    temp.next = new_node
    return head
\end{lstlisting}

\end{solutionbox}
\begin{mnemonicbox}
``Begin: New Leads Old, End: Old Leads New''

\end{mnemonicbox}
\subsection*{Question 3(a) OR [3
marks]}\label{q3a}

\textbf{List different operations performed on singly linked list.}

\begin{solutionbox}

{\def\LTcaptype{none} % do not increment counter
\begin{longtable}[]{@{}l@{}}
\toprule\noalign{}
Operations on Singly Linked List \\
\midrule\noalign{}
\endhead
\bottomrule\noalign{}
\endlastfoot
1. Insertion (at beginning, middle, end) \\
2. Deletion (from beginning, middle, end) \\
3. Traversal (visiting each node) \\
4. Searching (finding a specific node) \\
5. Updating (modifying node data) \\
\end{longtable}
}

\textbf{Diagram:}

\includegraphics[width=1\linewidth,height=\textheight,keepaspectratio]{mermaid-5d51f3ee.pdf}

\end{solutionbox}
\begin{mnemonicbox}
``Insert Delete Traverse Search Update''

\end{mnemonicbox}
\subsection*{Question 3(b) OR [4
marks]}\label{q3b}

\textbf{Explain concept of Circular linked list.}

\begin{solutionbox}

\textbf{Diagram:}

\begin{lstlisting}
    ↗-----------↘
   /             \
  ↓               ↓
[1] \rightarrow [2] \rightarrow [3] \rightarrow [4]
\end{lstlisting}

{\def\LTcaptype{none} % do not increment counter
\begin{longtable}[]{@{}
  >{\raggedright\arraybackslash}p{(\linewidth - 2\tabcolsep) * \real{0.4091}}
  >{\raggedright\arraybackslash}p{(\linewidth - 2\tabcolsep) * \real{0.5909}}@{}}
\toprule\noalign{}
\begin{minipage}[b]{\linewidth}\raggedright
Feature
\end{minipage} & \begin{minipage}[b]{\linewidth}\raggedright
Description
\end{minipage} \\
\midrule\noalign{}
\endhead
\bottomrule\noalign{}
\endlastfoot
Structure & Last node points to the first node instead of NULL \\
Advantage & Allows continuous traversal through all nodes \\
Applications & Round robin scheduling, circular buffer implementation \\
Operations & Insertion and deletion similar to singly linked list with
special handling for the last node \\
\end{longtable}
}

\textbf{Code:}

\begin{lstlisting}[language=Python]
class Node:
    def __init__(self, data):
        self.data = data
        self.next = None

# Creating a circular linked list with 3 nodes
head = Node(1)
node2 = Node(2)
node3 = Node(3)

head.next = node2
node2.next = node3
node3.next = head  # Makes it circular
\end{lstlisting}

\end{solutionbox}
\begin{mnemonicbox}
``Last Links to First''

\end{mnemonicbox}
\subsection*{Question 3(c) OR [7
marks]}\label{q3c}

\textbf{List application of linked list. Write an algorithm to count the
number of nodes in singly linked list.}

\begin{solutionbox}

{\def\LTcaptype{none} % do not increment counter
\begin{longtable}[]{@{}l@{}}
\toprule\noalign{}
Applications of Linked List \\
\midrule\noalign{}
\endhead
\bottomrule\noalign{}
\endlastfoot
1. Implementation of stacks and queues \\
2. Dynamic memory allocation \\
3. Undo functionality in applications \\
4. Hash tables (chaining) \\
5. Adjacency lists for graphs \\
\end{longtable}
}

\textbf{Algorithm to Count Nodes:}

\includegraphics[width=1\linewidth,height=\textheight,keepaspectratio]{mermaid-cfbe8763.pdf}

\textbf{Code:}

\begin{lstlisting}[language=Python]
def count_nodes(head):
    count = 0
    temp = head
    
    while temp:
        count += 1
        temp = temp.next
    
    return count

# Example usage
# Assuming head points to the first node of a linked list
total_nodes = count_nodes(head)
print(f"Total nodes: {total_nodes}")
\end{lstlisting}

\end{solutionbox}
\begin{mnemonicbox}
``Count While Moving''

\end{mnemonicbox}
\subsection*{Question 4(a) [3 marks]}\label{q4a}

\textbf{Compare Linear search with Binary search.}

\begin{solutionbox}

{\def\LTcaptype{none} % do not increment counter
\begin{longtable}[]{@{}
  >{\raggedright\arraybackslash}p{(\linewidth - 4\tabcolsep) * \real{0.2368}}
  >{\raggedright\arraybackslash}p{(\linewidth - 4\tabcolsep) * \real{0.3684}}
  >{\raggedright\arraybackslash}p{(\linewidth - 4\tabcolsep) * \real{0.3947}}@{}}
\toprule\noalign{}
\begin{minipage}[b]{\linewidth}\raggedright
Feature
\end{minipage} & \begin{minipage}[b]{\linewidth}\raggedright
Linear Search
\end{minipage} & \begin{minipage}[b]{\linewidth}\raggedright
Binary Search
\end{minipage} \\
\midrule\noalign{}
\endhead
\bottomrule\noalign{}
\endlastfoot
Data Arrangement & Works on both sorted and unsorted data & Works only
on sorted data \\
Time Complexity & O(n) & O(log n) \\
Implementation & Simpler & More complex \\
Best For & Small datasets or unsorted data & Large sorted datasets \\
\end{longtable}
}

\textbf{Diagram:}

\begin{lstlisting}
Linear: [1] [2] [3] [4] [5] [6] [7] [8]
        ↓   ↓   ↓   ↓   ↓   ↓   ↓   ↓
        Sequential checking

Binary: [1] [2] [3] [4] [5] [6] [7] [8]
                    ↓
                Check middle
                /     \
               /       \
        Lower half    Upper half
\end{lstlisting}

\end{solutionbox}
\begin{mnemonicbox}
``Linear Looks at All, Binary Breaks in Half''

\end{mnemonicbox}
\subsection*{Question 4(b) [4 marks]}\label{q4b}

\textbf{Write an algorithm for selection sort method.}

\begin{solutionbox}

\textbf{Diagram:}

\begin{lstlisting}
Initial: [5, 3, 8, 1, 2]
Pass 1:  [1, 3, 8, 5, 2]  (Find min = 1, swap with 5)
Pass 2:  [1, 2, 8, 5, 3]  (Find min = 2, swap with 3)
Pass 3:  [1, 2, 3, 5, 8]  (Find min = 3, swap with 8)
Pass 4:  [1, 2, 3, 5, 8]  (Find min = 5, already in place)
\end{lstlisting}

\textbf{Algorithm:}

\includegraphics[width=1\linewidth,height=\textheight,keepaspectratio]{mermaid-80180359.pdf}

\textbf{Code Outline:}

\begin{lstlisting}[language=Python]
def selection_sort(arr):
    n = len(arr)
    
    for i in range(n):
        min_idx = i
        
        # Find the minimum element in unsorted array
        for j in range(i+1, n):
            if arr[j] < arr[min_idx]:
                min_idx = j
        
        # Swap the found minimum element with the first element
        arr[i], arr[min_idx] = arr[min_idx], arr[i]
\end{lstlisting}

\end{solutionbox}
\begin{mnemonicbox}
``Find Minimum, Swap Position''

\end{mnemonicbox}
\subsection*{Question 4(c) [7 marks]}\label{q4c}

\textbf{Develop a python code to sort following list in ascending order
using Bubble sort method.} \textbf{list1=[5,4,3,2,1,0]}

\begin{solutionbox}

\textbf{Diagram:}

\begin{lstlisting}
Initial: [5, 4, 3, 2, 1, 0]
Pass 1:  [4, 3, 2, 1, 0, 5]
Pass 2:  [3, 2, 1, 0, 4, 5]
Pass 3:  [2, 1, 0, 3, 4, 5]
Pass 4:  [1, 0, 2, 3, 4, 5]
Pass 5:  [0, 1, 2, 3, 4, 5]
\end{lstlisting}

\textbf{Code:}

\begin{lstlisting}[language=Python]
def bubble_sort(arr):
    n = len(arr)
    
    # Traverse through all array elements
    for i in range(n):
        # Last i elements are already in place
        for j in range(0, n-i-1):
            # Swap if current element is greater than next element
            if arr[j] > arr[j+1]:
                arr[j], arr[j+1] = arr[j+1], arr[j]
    
    return arr

# Input list
list1 = [5, 4, 3, 2, 1, 0]

# Sorting the list
sorted_list = bubble_sort(list1)

# Displaying the result
print("Sorted list:", sorted_list)
# Output: Sorted list: [0, 1, 2, 3, 4, 5]
\end{lstlisting}

\end{solutionbox}
\begin{mnemonicbox}
``Bubble Biggest Upward''

\end{mnemonicbox}
\subsection*{Question 4(a) OR [3
marks]}\label{q4a}

\textbf{Define sorting. List different sorting methods.}

\begin{solutionbox}

{\def\LTcaptype{none} % do not increment counter
\begin{longtable}[]{@{}
  >{\raggedright\arraybackslash}p{(\linewidth - 2\tabcolsep) * \real{0.4138}}
  >{\raggedright\arraybackslash}p{(\linewidth - 2\tabcolsep) * \real{0.5862}}@{}}
\toprule\noalign{}
\begin{minipage}[b]{\linewidth}\raggedright
Definition
\end{minipage} & \begin{minipage}[b]{\linewidth}\raggedright
Sorting Methods
\end{minipage} \\
\midrule\noalign{}
\endhead
\bottomrule\noalign{}
\endlastfoot
Sorting is the process of arranging data in a specified order (ascending
or descending) & 1. Bubble Sort 2. Selection Sort 3. Insertion Sort 4.
Merge Sort 5. Quick Sort 6. Heap Sort 7. Radix Sort \\
\end{longtable}
}

\textbf{Diagram:}

\includegraphics[width=1\linewidth,height=\textheight,keepaspectratio]{mermaid-1e027c1b.pdf}

\end{solutionbox}
\begin{mnemonicbox}
``Better Sort Improves Many Query Results''

\end{mnemonicbox}
\subsection*{Question 4(b) OR [4
marks]}\label{q4b}

\textbf{Write an algorithm for Insertion sort method.}

\begin{solutionbox}

\textbf{Diagram:}

\begin{lstlisting}
Initial: [5, 2, 4, 6, 1, 3]
Pass 1:  [2, 5, 4, 6, 1, 3]  (Insert 2 before 5)
Pass 2:  [2, 4, 5, 6, 1, 3]  (Insert 4 before 5)
Pass 3:  [2, 4, 5, 6, 1, 3]  (6 is already in place)
Pass 4:  [1, 2, 4, 5, 6, 3]  (Insert 1 at beginning)
Pass 5:  [1, 2, 3, 4, 5, 6]  (Insert 3 after 2)
\end{lstlisting}

\textbf{Algorithm:}

\includegraphics[width=1\linewidth,height=\textheight,keepaspectratio]{mermaid-c462dd6e.pdf}

\textbf{Code Outline:}

\begin{lstlisting}[language=Python]
def insertion_sort(arr):
    for i in range(1, len(arr)):
        key = arr[i]
        j = i - 1
        
        # Move elements that are greater than key
        # to one position ahead of their current position
        while j >= 0 and arr[j] > key:
            arr[j + 1] = arr[j]
            j -= 1
        
        arr[j + 1] = key
\end{lstlisting}

\end{solutionbox}
\begin{mnemonicbox}
``Take Card, Insert In Order''

\end{mnemonicbox}
\subsection*{Question 4(c) OR [7
marks]}\label{q4c}

\textbf{Develop a python code to sort following list in ascending order
using selection sort method.} \textbf{list1=[6,3,25,8,-1,55,0]}

\begin{solutionbox}

\textbf{Diagram:}

\begin{lstlisting}
Initial: [6, 3, 25, 8, -1, 55, 0]
Pass 1:  [-1, 3, 25, 8, 6, 55, 0]  (Find min = -1, swap with 6)
Pass 2:  [-1, 0, 25, 8, 6, 55, 3]  (Find min = 0, swap with 3)
Pass 3:  [-1, 0, 3, 8, 6, 55, 25]  (Find min = 3, swap with 25)
Pass 4:  [-1, 0, 3, 6, 8, 55, 25]  (Find min = 6, swap with 8)
Pass 5:  [-1, 0, 3, 6, 8, 55, 25]  (Find min = 8, already in place)
Pass 6:  [-1, 0, 3, 6, 8, 25, 55]  (Find min = 25, swap with 55)
\end{lstlisting}

\textbf{Code:}

\begin{lstlisting}[language=Python]
def selection_sort(arr):
    n = len(arr)
    
    for i in range(n):
        # Find the minimum element in remaining unsorted array
        min_idx = i
        for j in range(i+1, n):
            if arr[j] < arr[min_idx]:
                min_idx = j
                
        # Swap the found minimum element with the first element
        arr[i], arr[min_idx] = arr[min_idx], arr[i]
    
    return arr

# Input list
list1 = [6, 3, 25, 8, -1, 55, 0]

# Sorting the list
sorted_list = selection_sort(list1)

# Displaying the result
print("Sorted list:", sorted_list)
# Output: Sorted list: [-1, 0, 3, 6, 8, 25, 55]
\end{lstlisting}

\end{solutionbox}
\begin{mnemonicbox}
``Select Smallest, Shift to Start''

\end{mnemonicbox}
\subsection*{Question 5(a) [3 marks]}\label{q5a}

\textbf{Define following terms regarding Tree data structure:}
\textbf{1. Forest} \textbf{2. Root node} \textbf{3. Leaf node}

\begin{solutionbox}

{\def\LTcaptype{none} % do not increment counter
\begin{longtable}[]{@{}
  >{\raggedright\arraybackslash}p{(\linewidth - 2\tabcolsep) * \real{0.3333}}
  >{\raggedright\arraybackslash}p{(\linewidth - 2\tabcolsep) * \real{0.6667}}@{}}
\toprule\noalign{}
\begin{minipage}[b]{\linewidth}\raggedright
Term
\end{minipage} & \begin{minipage}[b]{\linewidth}\raggedright
Definition
\end{minipage} \\
\midrule\noalign{}
\endhead
\bottomrule\noalign{}
\endlastfoot
Forest & Collection of disjoint trees (multiple trees without
connections between them) \\
Root Node & Topmost node of a tree with no parent, from which all other
nodes are descended \\
Leaf Node & Node with no children (terminal node at the bottom of the
tree) \\
\end{longtable}
}

\textbf{Diagram:}

\begin{lstlisting}
Forest:    Tree1    Tree2    Tree3
           /  \      / \      |
          /    \    /   \     |
         
Root:     [R]
         /   \
        /     \
        
Leaf:  [A] \rightarrow [B] \rightarrow [L] \rightarrow [L]
                    No children
\end{lstlisting}

\end{solutionbox}
\begin{mnemonicbox}
``Forest has Many Roots, Roots Lead All, Leaves End
All''

\end{mnemonicbox}
\subsection*{Question 5(b) [4 marks]}\label{q5b}

\textbf{Draw Binary search tree for 78,58,82,15,66,80,99 and write
In-order traversal for the tree.}

\begin{solutionbox}

\textbf{Binary Search Tree:}

\begin{lstlisting}
          78
         /  \
        /    \
      58      82
     /  \    /  \
   15   66  80   99
\end{lstlisting}

\textbf{In-order Traversal:}

{\def\LTcaptype{none} % do not increment counter
\begin{longtable}[]{@{}ll@{}}
\toprule\noalign{}
Step & Visit Order \\
\midrule\noalign{}
\endhead
\bottomrule\noalign{}
\endlastfoot
1 & Visit left subtree of 78 \\
2 & Visit left subtree of 58 \\
3 & Visit 15 \\
4 & Visit 58 \\
5 & Visit 66 \\
6 & Visit 78 \\
7 & Visit left subtree of 82 \\
8 & Visit 80 \\
9 & Visit 82 \\
10 & Visit 99 \\
\end{longtable}
}

\textbf{In-order Traversal Result: 15, 58, 66, 78, 80, 82, 99}

\end{solutionbox}
\begin{mnemonicbox}
``Left, Root, Right''

\end{mnemonicbox}
\subsection*{Question 5(c) [7 marks]}\label{q5c}

\textbf{Write an algorithm for following operation:} \textbf{1.
Insertion of Node in Binary Tree} \textbf{2. Deletion of Node in Binary
Tree}

\begin{solutionbox}

\textbf{Insertion Algorithm:}

\includegraphics[width=1\linewidth,height=\textheight,keepaspectratio]{mermaid-024d4882.pdf}

\textbf{Deletion Algorithm:}

\includegraphics[width=1\linewidth,height=\textheight,keepaspectratio]{mermaid-df1006f7.pdf}

\textbf{Code:}

\begin{lstlisting}[language=Python]
class Node:
    def __init__(self, data):
        self.data = data
        self.left = None
        self.right = None

# Insertion in Binary Tree
def insert(root, data):
    if root is None:
        return Node(data)
    
    # Level order traversal to find vacant position
    queue = []
    queue.append(root)
    
    while queue:
        temp = queue.pop(0)
        
        if temp.left is None:
            temp.left = Node(data)
            break
        else:
            queue.append(temp.left)
            
        if temp.right is None:
            temp.right = Node(data)
            break
        else:
            queue.append(temp.right)
    
    return root

# Deletion in Binary Tree
def delete_node(root, key):
    if root is None:
        return None
    
    if root.left is None and root.right is None:
        if root.data == key:
            return None
        else:
            return root
    
    # Find the node to delete
    key_node = None
    # Find the deepest node
    last = None
    parent = None
    
    # Level order traversal
    queue = []
    queue.append(root)
    
    while queue:
        temp = queue.pop(0)
        
        if temp.data == key:
            key_node = temp
            
        if temp.left:
            parent = temp
            queue.append(temp.left)
            last = temp.left
            
        if temp.right:
            parent = temp
            queue.append(temp.right)
            last = temp.right
    
    if key_node:
        # Replace with deepest node's data
        key_node.data = last.data
        
        # Delete the deepest node
        if parent.right == last:
            parent.right = None
        else:
            parent.left = None
    
    return root
\end{lstlisting}

\end{solutionbox}
\begin{mnemonicbox}
``Insert at Empty, Delete by Swap and Remove''

\end{mnemonicbox}
\subsection*{Question 5(a) OR [3
marks]}\label{q5a}

\textbf{Define following terms regarding Tree data structure:}
\textbf{1. In-degree} \textbf{2. Out-degree} \textbf{3. Depth}

\begin{solutionbox}

{\def\LTcaptype{none} % do not increment counter
\begin{longtable}[]{@{}
  >{\raggedright\arraybackslash}p{(\linewidth - 2\tabcolsep) * \real{0.3333}}
  >{\raggedright\arraybackslash}p{(\linewidth - 2\tabcolsep) * \real{0.6667}}@{}}
\toprule\noalign{}
\begin{minipage}[b]{\linewidth}\raggedright
Term
\end{minipage} & \begin{minipage}[b]{\linewidth}\raggedright
Definition
\end{minipage} \\
\midrule\noalign{}
\endhead
\bottomrule\noalign{}
\endlastfoot
In-degree & Number of edges coming into a node (always 1 for each node
except root node in a tree) \\
Out-degree & Number of edges going out from a node (number of
children) \\
Depth & Length of the path from root to the node (number of edges in
path) \\
\end{longtable}
}

\textbf{Diagram:}

\begin{lstlisting}
        A (Root, Depth 0)
       / \
      /   \
     B     C (Depth 1)
    / \     \
   D   E     F (Depth 2)
\end{lstlisting}

{\def\LTcaptype{none} % do not increment counter
\begin{longtable}[]{@{}lll@{}}
\toprule\noalign{}
Node & In-degree & Out-degree \\
\midrule\noalign{}
\endhead
\bottomrule\noalign{}
\endlastfoot
A & 0 & 2 \\
B & 1 & 2 \\
C & 1 & 1 \\
D & 1 & 0 \\
E & 1 & 0 \\
F & 1 & 0 \\
\end{longtable}
}

\end{solutionbox}
\begin{mnemonicbox}
``In Counts Parents, Out Counts Children, Depth
Counts Edges from Root''

\end{mnemonicbox}
\subsection*{Question 5(b) OR [4
marks]}\label{q5b}

\textbf{Write Preorder and postorder traversal of following Binary
tree.}

\textbf{Binary Tree:}

\begin{lstlisting}
        100
       /   \
      /     \
    20      200
   /  \     /  \
  10   30  150  300
\end{lstlisting}

\begin{solutionbox}

{\def\LTcaptype{none} % do not increment counter
\begin{longtable}[]{@{}lll@{}}
\toprule\noalign{}
Traversal & Order & Result \\
\midrule\noalign{}
\endhead
\bottomrule\noalign{}
\endlastfoot
Preorder & Root, Left, Right & 100, 20, 10, 30, 200, 150, 300 \\
Postorder & Left, Right, Root & 10, 30, 20, 150, 300, 200, 100 \\
\end{longtable}
}

\textbf{Preorder Visualization:}

\includegraphics[width=1\linewidth,height=\textheight,keepaspectratio]{mermaid-ba35e6f8.pdf}

\textbf{Postorder Visualization:}

\includegraphics[width=1\linewidth,height=\textheight,keepaspectratio]{mermaid-785631dd.pdf}

\end{solutionbox}
\begin{mnemonicbox}

\begin{itemize}
\tightlist
\item
  Preorder: ``Root First, Then Children''
\item
  Postorder: ``Children First, Then Root''
\end{itemize}

\end{mnemonicbox}
\subsection*{Question 5(c) OR [7
marks]}\label{q5c}

\textbf{Develop a program to implement construction of Binary Search
Tree.}

\begin{solutionbox}

\textbf{Diagram:}

\includegraphics[width=1\linewidth,height=\textheight,keepaspectratio]{mermaid-800aeb39.pdf}

\textbf{Code:}

\begin{lstlisting}[language=Python]
class Node:
    def __init__(self, key):
        self.key = key
        self.left = None
        self.right = None

def insert(root, key):
    # If the tree is empty, return a new node
    if root is None:
        return Node(key)
    
    # Otherwise, recur down the tree
    if key < root.key:
        root.left = insert(root.left, key)
    else:
        root.right = insert(root.right, key)
    
    # Return the unchanged node pointer
    return root

def inorder(root):
    if root:
        inorder(root.left)
        print(root.key, end=" ")
        inorder(root.right)

def preorder(root):
    if root:
        print(root.key, end=" ")
        preorder(root.left)
        preorder(root.right)

def postorder(root):
    if root:
        postorder(root.left)
        postorder(root.right)
        print(root.key, end=" ")

# Driver program to test the above functions
def main():
    # Create BST with these elements: 50, 30, 20, 40, 70, 60, 80
    root = None
    elements = [50, 30, 20, 40, 70, 60, 80]
    
    for element in elements:
        root = insert(root, element)
    
    # Print traversals
    print("Inorder traversal: ", end="")
    inorder(root)
    print("\nPreorder traversal: ", end="")
    preorder(root)
    print("\nPostorder traversal: ", end="")
    postorder(root)

# Run the program
main()
\end{lstlisting}

\textbf{Example Output:}

\begin{lstlisting}
Inorder traversal: 20 30 40 50 60 70 80
Preorder traversal: 50 30 20 40 70 60 80
Postorder traversal: 20 40 30 60 80 70 50
\end{lstlisting}

\end{solutionbox}
\begin{mnemonicbox}
``Insert Smaller Left, Larger Right''

\end{mnemonicbox}

\end{document}
