\documentclass{article}

% content/resources/templates/preamble.tex
\usepackage[margin=0.6in]{geometry}
\author{Milav Dabgar}
\usepackage{amsmath,amssymb,amsthm}
\usepackage{booktabs}
\usepackage{multirow}
\usepackage{xcolor}
\usepackage{tcolorbox}
\tcbuselibrary{breakable,skins}
\usepackage[colorlinks=true,linkcolor=blue]{hyperref}
\usepackage{titlesec}
\usepackage{enumitem}
\usepackage{tikz}
\usepackage{pgfplots}
\usepackage{circuitikz}
\usepackage[version=4]{mhchem}
\usepackage{longtable}
\usepackage{array}
\usepackage{float}
\usepackage{caption}
\usepackage{listings}

\lstset{
  basicstyle=\small\ttfamily,
  breaklines=true,
  breakatwhitespace=false,
  postbreak=\mbox{\textcolor{red}{$\hookrightarrow$}\space},
  float=false,
  numbers=left,
  numberstyle=\tiny\color{gray},
  numbersep=10pt,
  xleftmargin=2em,
  keywordstyle=\color{blue},
  commentstyle=\color{green!60!black},
  stringstyle=\color{purple},
  backgroundcolor=\color{gray!5},
  showstringspaces=false,
  tabsize=2,
  captionpos=b,
  keepspaces=true,
  columns=flexible
}

\pgfplotsset{compat=1.18}
\usetikzlibrary{shapes,arrows,positioning,calc,patterns,decorations.pathmorphing,decorations.markings,arrows.meta}

% Color scheme
\definecolor{headcolor}{RGB}{0,102,204}
\definecolor{keycolor}{RGB}{220,20,60}
\definecolor{solutioncolor}{RGB}{34,139,34}
\definecolor{mnemoniccolor}{RGB}{148,0,211}
\definecolor{codecolor}{RGB}{0,0,100}

% Spacing
\setlength{\parskip}{3pt}
\setlist[itemize]{nosep}
\setlist[enumerate]{nosep}

% Title formatting
\titleformat{\section}{\Large\bfseries\color{headcolor}}{\thesection}{1em}{}
\titleformat{\subsection}{\large\bfseries\color{headcolor}}{\thesubsection}{1em}{}

% Pandoc tightlist compatibility
\providecommand{\tightlist}{%
  \setlength{\itemsep}{0pt}\setlength{\parskip}{0pt}}

% Pandoc longtable compatibility
\newcounter{none}
\def\thenone{}


% content/resources/templates/gujarati-boxes.tex
\usepackage{fontspec}
\usepackage{polyglossia}

% Set Gujarati as main language (document is primarily in Gujarati)
% Note: gloss-gujarati.ldf doesn't exist in polyglossia, but it will use hyphenation patterns
\setdefaultlanguage{gujarati}
\setotherlanguage{english}

% Configure Gujarati font properly
% Use Language=Default to prevent polyglossia from trying to add language-specific features
% that don't exist for Gujarati, which causes "empty feature" warnings
\newfontfamily\gujaratifont[Script=Gujarati,AutoFakeBold=2.5,AutoFakeSlant=0.3]{Noto Sans Gujarati}
\setmainfont[Script=Gujarati,AutoFakeBold=2.5,AutoFakeSlant=0.3]{Noto Sans Gujarati}
% Use Noto Sans Gujarati for monospace to support Gujarati in text
\setmonofont[Scale=0.9]{Noto Sans Gujarati}

% Configure English to use the same font
\newfontfamily\englishfont[Script=Gujarati,AutoFakeBold=2.5,AutoFakeSlant=0.3]{Noto Sans Gujarati}

% Translations for polyglossia
\gappto\captionsgujarati{
  \renewcommand{\tablename}{કોષ્ટક}
  \renewcommand{\figurename}{આકૃતિ}
}

% Helper for TikZ nodes to ensure Gujarati font
\newcommand{\gu}[1]{{\gujaratifont #1}}

% Custom environments
\newtcolorbox{solutionbox}{
    breakable,
    enhanced,
    colback=solutioncolor!5!white,
    colframe=solutioncolor!75!black,
    fonttitle=\bfseries,
    title=જવાબ
}

\newtcolorbox{solutionboxnobreak}{
 colback=solutioncolor!5!white,
 colframe=solutioncolor!75!black,
 fonttitle=\bfseries,
 title=જવાબ
}

\newtcolorbox{keyformula}{
 breakable,
 enhanced,
 colback=keycolor!5!white,
 colframe=keycolor!75!black,
 fonttitle=\bfseries,
 title=રાસાયણિક સમીકરણ/સૂત્ર
}

\newtcolorbox{mnemonicbox}{
 breakable,
 enhanced,
 colback=mnemoniccolor!5!white,
 colframe=mnemoniccolor!75!black,
 fonttitle=\bfseries,
 title=મેમરી ટ્રીક
}


% Custom commands for GTU solutions
% This file defines semantic commands for consistent formatting

% Question command with automatic formatting
\newcommand{\question}[2]{%
  \section*{Question #1}%
  \textbf{#2}%
}

% OR question variant
\newcommand{\questionor}[2]{%
  \section*{Question #1 OR}%
  \textbf{#2}%
}

% Proper table environment with caption
\newenvironment{answertable}[1]{%
  \begin{table}[htbp]
  \centering
  \caption{#1}
}{%
  \end{table}
}

% Proper figure environment for diagrams
\newenvironment{answerdiagram}[1]{%
  \begin{figure}[htbp]
  \centering
  \caption{#1}
}{%
  \end{figure}
}

% Semantic markup for key terms
\newcommand{\keyword}[1]{\textbf{#1}}
\newcommand{\code}[1]{\texttt{#1}}
\newcommand{\classname}[1]{\texttt{#1}}
\newcommand{\methodname}[1]{\texttt{#1}}

% Proper quotation marks
\newcommand{\mnemonic}[1]{``#1''}


\title{Communication Engineering (1333201) - Winter 2023 Solution}
\date{January 11, 2024}

\begin{document}
\maketitle

\questionmarks{1(a)}{3}{વ્યાખ્યા આપો: (અ) Amplitude Modulation, (બ) Frequency Modulation અને (ક) Phase Modulation}

\begin{solutionbox}
\textbf{જવાબ}:

\begin{center}
\captionof{table}{મોડ્યુલેશન પ્રકારો}
\begin{tabulary}{\linewidth}{|L|L|}
\hline
\textbf{મોડ્યુલેશન પ્રકાર} & \textbf{વ્યાખ્યા} \\
\hline
\textbf{Amplitude Modulation (AM)} & એક પ્રક્રિયા જેમાં carrier સિગ્નલનું amplitude, modulating સિગ્નલની ક્ષણિક કિંમત અનુસાર બદલાય છે જ્યારે frequency અચળ રહે છે \\
\hline
\textbf{Frequency Modulation (FM)} & એક પ્રક્રિયા જેમાં carrier સિગ્નલની frequency, modulating સિગ્નલની ક્ષણિક કિંમત અનુસાર બદલાય છે જ્યારે amplitude અચળ રહે છે \\
\hline
\textbf{Phase Modulation (PM)} & એક પ્રક્રિયા જેમાં carrier સિગ્નલનો phase, modulating સિગ્નલની ક્ષણિક કિંમત અનુસાર બદલાય છે જ્યારે amplitude અચળ રહે છે \\
\hline
\end{tabulary}
\end{center}
\end{solutionbox}

\begin{mnemonicbox}
"A-F-P: Amplitude બદલાય છે, Frequency ખસે છે, Phase સમાયોજિત થાય છે"
\end{mnemonicbox}

\questionmarks{1(b)}{4}{મોડ્યુલેશનની જરૂરિયાત સમજાવો.}

\begin{solutionbox}
\textbf{જવાબ}:

\begin{center}
\captionof{table}{મોડ્યુલેશનની જરૂરિયાત}
\begin{tabulary}{\linewidth}{|L|L|}
\hline
\textbf{જરૂરિયાત} & \textbf{સમજૂતી} \\
\hline
\textbf{પ્રેક્ટિકલ એન્ટેના સાઈઝ} & frequency વધારીને એન્ટેનાનું કદ ઘટાડે છે (એન્ટેના લંબાઈ = $\lambda/4$) \\
\hline
\textbf{ઇન્ટરફેરન્સ ઘટાડો} & અલગ-અલગ frequencies પર એક સાથે ઘણા સિગ્નલો પ્રસારિત કરવાની મંજૂરી આપે છે \\
\hline
\textbf{રેન્જ વિસ્તરણ} & ઉચ્ચ frequency સિગ્નલો વાતાવરણમાં વધુ દૂર સુધી જાય છે \\
\hline
\textbf{મલ્ટિપ્લેક્સિંગ} & ઘણા સિગ્નલોને કોમ્યુનિકેશન માધ્યમ શેર કરવા સક્ષમ બનાવે છે \\
\hline
\end{tabulary}
\end{center}

\begin{center}
\begin{tikzpicture}[auto, >=latex, thick]
    \node [gtu block, align=center] (need) {મોડ્યુલેશનની\\જરૂરિયાત};
    \node [gtu block, below of=need, node distance=2.5cm, xshift=-4.5cm, align=center] (size) {પ્રેક્ટિકલ\\એન્ટેના સાઈઝ};
    \node [gtu block, below of=need, node distance=2.5cm, xshift=-1.5cm, align=center] (inter) {ઇન્ટરફેરન્સ\\ઘટાડો};
    \node [gtu block, below of=need, node distance=2.5cm, xshift=1.5cm, align=center] (range) {રેન્જ\\વિસ્તરણ};
    \node [gtu block, below of=need, node distance=2.5cm, xshift=4.5cm, align=center] (mux) {મલ્ટિપ્લેક્સિંગ};
    
    \draw [gtu arrow] (need) -- (size);
    \draw [gtu arrow] (need) -- (inter);
    \draw [gtu arrow] (need) -- (range);
    \draw [gtu arrow] (need) -- (mux);
\end{tikzpicture}
\captionof{figure}{મોડ્યુલેશનની જરૂરિયાત}
\end{center}
\end{solutionbox}

\begin{mnemonicbox}
"PIRM: પ્રેક્ટિકલ એન્ટેના, ઇન્ટરફેરન્સ ઘટાડો, રેન્જ વિસ્તરણ, મલ્ટિપ્લેક્સિંગ"
\end{mnemonicbox}

\questionmarks{1(c)}{7}{અમ્પ્લિટુડ મોડ્યુલેશનમાં મોડ્યુલેટિંગ સિગ્નલને 3V નું અમ્પ્લિટુડ અને 1 KHz ની ફ્રિક્વન્સી છે જ્યારે કેરિયર સિગ્નલને 10 V નું અમ્પ્લિટુડ અને 30 KHz ની ફ્રિક્વન્સી છે. મોડ્યુલેશન ઇન્ડેક્સ, સાઇડબેન્ડ ફ્રીક્વન્સીઝ અને તેમના અમ્પ્લિટુડ શોધો તેમજ આ AM વેવનું સ્પેક્ટ્રમ દોરો.}

\begin{solutionbox}
\textbf{જવાબ}:

\textbf{આપેલ માહિતી:}
\begin{itemize}
    \item મોડ્યુલેટિંગ સિગ્નલ: $A_m = 3$ V, $f_m = 1$ kHz
    \item કેરિયર સિગ્નલ: $A_c = 10$ V, $f_c = 30$ kHz
\end{itemize}

\textbf{ગણતરી:}

\begin{enumerate}
    \item \textbf{મોડ્યુલેશન ઇન્ડેક્સ (m)}:
    \[ m = \frac{A_m}{A_c} = \frac{3}{10} = 0.3 \]
    
    \item \textbf{સાઇડબેન્ડ ફ્રિક્વન્સી}:
    \[ f_{LSB} = f_c - f_m = 30 - 1 = 29 \text{ kHz} \]
    \[ f_{USB} = f_c + f_m = 30 + 1 = 31 \text{ kHz} \]
    
    \item \textbf{સાઇડબેન્ડ અમ્પ્લિટુડ}:
    \[ A_{SB} = \frac{m \cdot A_c}{2} = \frac{0.3 \cdot 10}{2} = 1.5 \text{ V} \]
\end{enumerate}

\begin{center}
\begin{tikzpicture}[scale=0.8]
    \draw[->] (0,0) -- (10,0) node[right] {$f$};
    \draw[->] (0,0) -- (0,5) node[above] {Amplitude};
    
    % Carrier
    \draw[thick, blue] (5,0) -- (5,4);
    \node[above] at (5,4) {10 V};
    \node[below] at (5,0) {30 kHz ($f_c$)};
    
    % LSB
    \draw[thick, red] (3,0) -- (3,1.5);
    \node[above] at (3,1.5) {1.5 V};
    \node[below] at (3,0) {29 kHz (LSB)};
    
    % USB
    \draw[thick, red] (7,0) -- (7,1.5);
    \node[above] at (7,1.5) {1.5 V};
    \node[below] at (7,0) {31 kHz (USB)};
    
    % connecting lines (envelope shape hint)
    \draw[dashed, gray] (3,1.5) -- (5,4) -- (7,1.5);
\end{tikzpicture}
\captionof{figure}{AM સ્પેક્ટ્રમ}
\end{center}
\end{solutionbox}

\begin{mnemonicbox}
"LSB-C-USB: લોઅર સાઇડબેન્ડ, કેરિયર, અપર સાઇડબેન્ડ 29-30-31 પર"
\end{mnemonicbox}

\questionmarks{1(c) OR}{7}{કેરિયર પાવર અને મોડુલેટેડ સિગ્નલ પાવરના મેથેમેટિકલ ઇક્વેશન તારવો.}

\begin{solutionbox}
\textbf{જવાબ}:

\textbf{મેથેમેટિકલ રિલેશન:}

કેરિયર સિગ્નલ: $c(t) = A_c \cos(2\pi f_c t)$
મોડ્યુલેટિંગ સિગ્નલ: $m(t) = A_m \cos(2\pi f_m t)$
AM સિગ્નલનું સમીકરણ:
\[ s(t) = A_c [1 + m \cos(2\pi f_m t)] \cos(2\pi f_c t) \]
વિસ્તરણ:
\[ s(t) = A_c \cos(2\pi f_c t) + \frac{m A_c}{2} \cos[2\pi(f_c - f_m)t] + \frac{m A_c}{2} \cos[2\pi(f_c + f_m)t] \]

\textbf{AM માં પાવર વિતરણ:}

1. \textbf{કેરિયર પાવર ($P_c$)}:
\[ P_c = \frac{(A_c/\sqrt{2})^2}{R} = \frac{A_c^2}{2} \]

2. \textbf{કુલ સાઇડબેન્ડ પાવર ($P_s$)}:
\[ P_{LSB} = \frac{(m A_c/2\sqrt{2})^2}{R} = \frac{m^2 A_c^2}{8} \]
\[ P_{USB} = \frac{(m A_c/2\sqrt{2})^2}{R} = \frac{m^2 A_c^2}{8} \]
\[ P_s = P_{LSB} + P_{USB} = \frac{m^2 A_c^2}{4} = P_c \cdot \frac{m^2}{2} \]

3. \textbf{કુલ AM પાવર ($P_t$)}:
\[ P_t = P_c + P_s = P_c + P_c \frac{m^2}{2} \]
\[ P_t = P_c \left( 1 + \frac{m^2}{2} \right) \]

\begin{center}
\captionof{table}{AM માં પાવર વિતરણ}
\begin{tabulary}{\linewidth}{|L|L|L|}
\hline
\textbf{ઘટક} & \textbf{સૂત્ર} & \textbf{$P_c$ ના સંદર્ભમાં} \\
\hline
\textbf{કેરિયર પાવર ($P_c$)} & $A_c^2/2$ & $P_c$ \\
\hline
\textbf{કુલ સાઇડબેન્ડ પાવર ($P_s$)} & $m^2 A_c^2/4$ & $m^2 P_c/2$ \\
\hline
\textbf{કુલ AM પાવર ($P_t$)} & $P_c(1 + m^2/2)$ & $P_c(1 + m^2/2)$ \\
\hline
\end{tabulary}
\end{center}

\textbf{મોડ્યુલેશન કાર્યક્ષમતા ($\eta$)}: સાઇડબેન્ડ પાવર અને કુલ પાવરનો ગુણોત્તર.
\[ \eta = \frac{P_s}{P_t} = \frac{m^2/2}{1 + m^2/2} \times 100\% \]
\end{solutionbox}

\begin{mnemonicbox}
"કુલ પાવર = કેરિયર પાવર × $(1 + m^2/2)$"
\end{mnemonicbox}

\questionmarks{2(a)}{3}{AM અને FM ની સરખામણી કરો.}

\begin{solutionbox}
\textbf{જવાબ}:

\begin{center}
\captionof{table}{AM અને FM વચ્ચે તુલના}
\begin{tabulary}{\linewidth}{|L|L|L|}
\hline
\textbf{પરિમાણ} & \textbf{AM} & \textbf{FM} \\
\hline
\textbf{મોડ્યુલેશન પરિમાણ} & અમ્પ્લિટુડ બદલાય છે & ફ્રિક્વન્સી બદલાય છે \\
\hline
\textbf{બેન્ડવિડ્થ} & $2 \times f_m$ & $2 \times (\Delta f + f_m)$ \\
\hline
\textbf{નોઇઝ ઇમ્યુનિટી} & નબળી & ઉત્તમ \\
\hline
\textbf{પાવર કાર્યક્ષમતા} & નીચી & ઉંચી \\
\hline
\textbf{સર્કિટ જટિલતા} & સરળ & જટિલ \\
\hline
\end{tabulary}
\end{center}
\end{solutionbox}

\begin{mnemonicbox}
"ABNPC: અમ્પ્લિટુડ/બેન્ડવિડ્થ/નોઇઝ/પાવર/જટિલતા તફાવત"
\end{mnemonicbox}

\questionmarks{2(b)}{4}{સર્કિટ ડાયાગ્રામની મદદથી એન્વલેપ ડિટેક્ટરને સમજાવો.}

\begin{solutionbox}
\textbf{જવાબ}:

\begin{center}
\begin{tikzpicture}[auto, >=latex, thick]
   \node (in) at (0,2) {AM Input};
   \draw (in) -- (2,2);
   
   % Diode
   \draw[thick] (2,2) -- (2,2); % Wire
   \draw[thick] (2,2.3) -- (2,1.7) -- (2.6,2) -- cycle; % Triangle
   \draw[thick] (2.6,2.3) -- (2.6,1.7); % Bar
   \draw[thick] (2.6,2) -- (4,2);
   \node[above] at (2.3,2.4) {D};
   
   % Capacitor
   \draw[thick] (4,2) -- (4,1.5);
   \draw[thick] (3.7,1.5) -- (4.3,1.5);
   \draw[thick] (3.7,1.3) -- (4.3,1.3);
   \draw[thick] (4,1.3) -- (4,0);
   \node[right] at (4.3,1.4) {C};
   
   % Resistor
   \draw[thick] (6,2) -- (6,1.8);
   \draw[thick] (6,1.8) -- (5.8,1.7) -- (6.2,1.6) -- (5.8,1.5) -- (6.2,1.4) -- (6,1.3);
   \draw[thick] (6,1.3) -- (6,0);
   \node[right] at (6.2,1.6) {$R_L$};
   
   % Connections
   \draw[thick] (4,2) -- (7,2);
   \node[right] at (7,2) {Demodulated Output};
   \draw[thick] (4,0) -- (6,0); % Ground line
   \node[ground] at (5,0) {};
\end{tikzpicture}
\captionof{figure}{એન્વલેપ ડિટેક્ટર સર્કિટ}
\end{center}

\textbf{એન્વલેપ ડિટેક્ટર ઘટકો:}

\begin{center}
\captionof{table}{એન્વલેપ ડિટેક્ટર ઘટકો}
\begin{tabulary}{\linewidth}{|L|L|}
\hline
\textbf{ઘટક} & \textbf{કાર્ય} \\
\hline
\textbf{ડાયોડ (D)} & AM સિગ્નલને રેક્ટિફાય કરે છે અને પોઝિટિવ હાફ સાયકલ મેળવે છે \\
\hline
\textbf{કેપેસિટર (C)} & ઇનપુટના પીક સુધી ચાર્જ થાય છે, પીક વચ્ચે ચાર્જ જાળવી રાખે છે \\
\hline
\textbf{રેઝિસ્ટર ($R_L$)} & એન્વેલોપ એક્સટ્રેક્શન માટે યોગ્ય દરે કેપેસિટરને ડિસ્ચાર્જ કરે છે \\
\hline
\end{tabulary}
\end{center}

\textbf{ટાઈમ કોન્સ્ટન્ટ સિલેક્શન:}
\[ \frac{1}{f_c} \ll RC \ll \frac{1}{f_m} \]
(યોગ્ય એન્વેલોપ ડિટેક્શન માટે)
\end{solutionbox}

\begin{mnemonicbox}
"DCR: ડાયોડ રેક્ટિફાય કરે છે, કેપેસિટર ચાર્જ થાય છે, રેઝિસ્ટર ડિસ્ચાર્જ કરે છે"
\end{mnemonicbox}

\questionmarks{2(c)}{7}{સુપરહીટરોડાઈન રીસીવરનો બ્લોક ડાયાગ્રામ દોરો અને સમજાવો.}

\begin{solutionbox}
\textbf{જવાબ}:

\begin{center}
\begin{tikzpicture}[node distance=2.5cm, auto, >=latex, thick, scale=0.8, transform shape]
    \node [gtu block, align=center] (rf) {RF\\એમ્પ્લિફાયર};
    \node [left of=rf, node distance=2.5cm, align=center] (ant) {એન્ટેના};
    \node [gtu block, right of=rf, align=center] (mix) {મિક્સર};
    \node [gtu block, below of=mix, align=center] (lo) {લોકલ\\ઓસિલેટર};
    \node [gtu block, right of=mix, align=center] (if) {IF\\એમ્પ્લિફાયર};
    \node [gtu block, right of=if, align=center] (det) {ડિટેક્ટર};
    \node [gtu block, right of=det, align=center] (af) {AF\\એમ્પ્લિફાયર};
    \node [right of=af, node distance=2.5cm, align=center] (spk) {સ્પીકર};
    
    \draw [gtu arrow] (ant) -- (rf);
    \draw [gtu arrow] (rf) -- (mix);
    \draw [gtu arrow] (lo) -- (mix);
    \draw [gtu arrow] (mix) -- (if);
    \draw [gtu arrow] (if) -- (det);
    \draw [gtu arrow] (det) -- (af);
    \draw [gtu arrow] (af) -- (spk);
\end{tikzpicture}
\captionof{figure}{સુપરહીટરોડાઈન રીસીવર}
\end{center}

\textbf{સુપરહીટરોડાઈન રીસીવર બ્લોક્સના કાર્યો:}

\begin{center}
\captionof{table}{સુપરહીટરોડાઈન રીસીવર બ્લોક્સના કાર્યો}
\begin{tabulary}{\linewidth}{|L|L|}
\hline
\textbf{બ્લોક} & \textbf{કાર્ય} \\
\hline
\textbf{RF એમ્પ્લિફાયર} & નબળા RF સિગ્નલને એમ્પ્લિફાય કરે છે, સિલેક્ટિવિટી પ્રદાન કરે છે, ઇમેજ ફ્રિક્વન્સીને રદ કરે છે \\
\hline
\textbf{લોકલ ઓસિલેટર} & મિક્સિંગ માટે ફ્રિક્વન્સી $f_o = f_{RF} + f_{IF}$ ઉત્પન્ન કરે છે \\
\hline
\textbf{મિક્સર} & IF (ઇન્ટરમીડિયેટ ફ્રિક્વન્સી) બનાવવા માટે RF સિગ્નલને લોકલ ઓસિલેટર સાથે જોડે છે \\
\hline
\textbf{IF એમ્પ્લિફાયર} & ફિક્સ્ડ ફ્રિક્વન્સી પર મોટાભાગના રિસીવર ગેઇન અને સિલેક્ટિવિટી પ્રદાન કરે છે \\
\hline
\textbf{ડિટેક્ટર} & IF સિગ્નલમાંથી મોડ્યુલેટિંગ સિગ્નલ એક્સટ્રેક્ટ કરે છે \\
\hline
\textbf{AF એમ્પ્લિફાયર} & સ્પીકર ચલાવવા માટે રિકવર થયેલ ઓડિયોને એમ્પ્લિફાય કરે છે \\
\hline
\end{tabulary}
\end{center}
\end{solutionbox}

\begin{mnemonicbox}
"RLMIDS: RF, લોકલ ઓસિલેટર, મિક્સર, IF, ડિટેક્ટર, સ્પીકર"
\end{mnemonicbox}

\questionmarks{2(a) OR}{3}{નીચેના શબ્દો વ્યાખ્યાયિત કરો: (અ) Sensitivity અને (બ) Selectivity}

\begin{solutionbox}
\textbf{જવાબ}:

\begin{center}
\captionof{table}{રિસીવર લક્ષણો}
\begin{tabulary}{\linewidth}{|L|L|}
\hline
\textbf{શબ્દ} & \textbf{વ્યાખ્યા} \\
\hline
\textbf{Sensitivity} & નબળા સિગ્નલોને શોધવા અને એમ્પ્લિફાય કરવાની રિસીવરની ક્ષમતા; સ્ટાન્ડર્ડ આઉટપુટ માટે જરૂરી ન્યૂનતમ ઇનપુટ સિગ્નલ સ્ટ્રેન્થ ($\mu$V) તરીકે માપવામાં આવે છે \\
\hline
\textbf{Selectivity} & અડીન ચેનલોથી ઇચ્છિત સિગ્નલને અલગ કરવાની રિસીવરની ક્ષમતા; રેસોનન્ટ ફ્રિક્વન્સી પર રિસ્પોન્સના ઓફ-રેસોનન્ટ ફ્રિક્વન્સી પર રિસ્પોન્સના ગુણોત્તર તરીકે માપવામાં આવે છે \\
\hline
\end{tabulary}
\end{center}

\begin{center}
\begin{tikzpicture}[scale=0.8]
    \draw[->] (0,0) -- (6,0) node[right] {Frequency};
    \draw[->] (0,0) -- (0,4) node[above] {Response};
    
    % Selectivity Curve
    \draw[thick, blue] (1,0.5) .. controls (2,0.5) and (2.5,3.5) .. (3,3.5) 
                       .. controls (3.5,3.5) and (4,0.5) .. (5,0.5);
    
    \draw[dashed] (3,0) -- (3,3.5);
    \node[below] at (3,0) {$f_c$};
    
    \draw[dashed] (0, 2.5) -- (6, 2.5);
    \node[left] at (0, 2.5) {-3dB};
    
    \draw[<->] (2.5, 2.5) -- (3.5, 2.5);
    \node[above] at (3, 2.5) {BW};
\end{tikzpicture}
\captionof{figure}{સિલેક્ટિવિટી કર્વ}
\end{center}
\end{solutionbox}

\begin{mnemonicbox}
"SS: સિગ્નલ સ્ટ્રેન્થ ફોર સેન્સિટિવિટી, સિગ્નલ સેપરેશન ફોર સિલેક્ટિવિટી"
\end{mnemonicbox}

\questionmarks{2(b) OR}{4}{જનરલ કમ્યુનિકેશનના બ્લોક ડાયાગ્રામનું વર્ણન કરો}

\begin{solutionbox}
\textbf{જવાબ}:

\begin{center}
\begin{tikzpicture}[node distance=2.5cm, auto, >=latex, thick]
    \node [gtu block, align=center] (source) {ઇન્ફોર્મેશન\\સોર્સ};
    \node [gtu block, right of=source, node distance=3cm, align=center] (tx) {ટ્રાન્સમીટર};
    \node [gtu block, right of=tx, node distance=3cm, align=center] (ch) {ચેનલ};
    \node [gtu block, right of=ch, node distance=3cm, align=center] (rx) {રિસીવર};
    \node [gtu block, right of=rx, node distance=3cm, align=center] (dest) {ડેસ્ટિનેશન};
    \node [gtu block, below of=ch, node distance=2cm, align=center] (noise) {નોઇઝ\\સોર્સ};
    
    \draw [gtu arrow] (source) -- (tx);
    \draw [gtu arrow] (tx) -- (ch);
    \draw [gtu arrow] (ch) -- (rx);
    \draw [gtu arrow] (rx) -- (dest);
    \draw [gtu arrow] (noise) -- (ch);
\end{tikzpicture}
\captionof{figure}{જનરલ કમ્યુનિકેશન સિસ્ટમ}
\end{center}

\begin{center}
\captionof{table}{કમ્યુનિકેશન સિસ્ટમના ઘટકો}
\begin{tabulary}{\linewidth}{|L|L|}
\hline
\textbf{ઘટક} & \textbf{કાર્ય} \\
\hline
\textbf{ઇન્ફોર્મેશન સોર્સ} & કમ્યુનિકેટ કરવા માટેનો સંદેશ ઉત્પન્ન કરે છે (વૉઇસ, ડેટા, વિડિઓ) \\
\hline
\textbf{ટ્રાન્સમીટર} & સંદેશને ટ્રાન્સમિશન માટે યોગ્ય સિગ્નલમાં રૂપાંતરિત કરે છે \\
\hline
\textbf{ચેનલ} & જેના દ્વારા સિગ્નલ પસાર થાય છે તે માધ્યમ (વાયર, ફાઇબર, હવા) \\
\hline
\textbf{રિસીવર} & મળેલા સિગ્નલમાંથી મૂળ સંદેશ એક્સટ્રેક્ટ કરે છે \\
\hline
\textbf{ડેસ્ટિનેશન} & જેના માટે સંદેશ અભિપ્રેત છે તે એન્ટિટી \\
\hline
\textbf{નોઇઝ સોર્સ} & અવાંછિત સિગ્નલો જે સંદેશમાં દખલ કરે છે \\
\hline
\end{tabulary}
\end{center}
\end{solutionbox}

\begin{mnemonicbox}
"I-T-C-R-D: ઇન્ફોર્મેશન ટ્રાવેલ્સ કેરફુલી, રીચેસ ડેસ્ટિનેશન"
\end{mnemonicbox}

\questionmarks{2(c) OR}{7}{સુપરહીટરોડાઈન FM રીસીવરનો બ્લોક ડાયાગ્રામ દોરો અને સમજાવો.}

\begin{solutionbox}
\textbf{જવાબ}:

\begin{center}
\begin{tikzpicture}[node distance=2.2cm, auto, >=latex, thick, scale=0.75, transform shape]
    \node [gtu block, align=center] (rf) {RF\\એમ્પ્લિફાયર};
    \node [left of=rf, node distance=2cm, align=center] (ant) {Ant};
    \node [gtu block, right of=rf, align=center] (mix) {મિક્સર};
    \node [gtu block, below of=mix, align=center] (lo) {લોકલ\\ઓસિલેટર};
    \node [gtu block, right of=mix, align=center] (if) {IF\\એમ્પ્લિફાયર};
    \node [gtu block, right of=if, align=center] (lim) {લિમિટર};
    \node [gtu block, right of=lim, align=center] (disc) {FM\\ડિસ્ક્રિમિનેટર};
    \node [gtu block, right of=disc, align=center] (deemp) {ડી-\\એમ્ફેસિસ};
    \node [gtu block, right of=deemp, align=center] (af) {AF\\એમ્પ્લિફાયર};
    \node [right of=af, node distance=2cm, align=center] (spk) {Spk};
    
    \draw [gtu arrow] (ant) -- (rf);
    \draw [gtu arrow] (rf) -- (mix);
    \draw [gtu arrow] (lo) -- (mix);
    \draw [gtu arrow] (mix) -- (if);
    \draw [gtu arrow] (if) -- (lim);
    \draw [gtu arrow] (lim) -- (disc);
    \draw [gtu arrow] (disc) -- (deemp);
    \draw [gtu arrow] (deemp) -- (af);
    \draw [gtu arrow] (af) -- (spk);
\end{tikzpicture}
\captionof{figure}{સુપરહીટરોડાઈન FM રીસીવર}
\end{center}

\textbf{FM રિસીવરમાં વધારાના ઘટકો:}

\begin{center}
\captionof{table}{FM રિસીવરમાં વધારાના ઘટકો}
\begin{tabulary}{\linewidth}{|L|L|}
\hline
\textbf{ઘટક} & \textbf{કાર્ય} \\
\hline
\textbf{લિમિટર} & અમ્પ્લિટુડ વેરિએશન્સ દૂર કરે છે, સ્થિર અમ્પ્લિટુડ સિગ્નલ પ્રદાન કરે છે \\
\hline
\textbf{FM ડિસ્ક્રિમિનેટર} & ફ્રિક્વન્સી વેરિએશન્સને અમ્પ્લિટુડ વેરિએશન્સમાં રૂપાંતરિત કરે છે (ડિમોડ્યુલેશન) \\
\hline
\textbf{ડી-એમ્ફેસિસ} & ટ્રાન્સમીટર પર બૂસ્ટ થયેલ ઉચ્ચ ફ્રિક્વન્સીને ઘટાડે છે \\
\hline
\end{tabulary}
\end{center}

\textbf{FM રિસીવરની વિશિષ્ટ બાબતો:}
\begin{itemize}
    \item વધુ પહોળી બેન્ડવિડ્થ IF એમ્પ્લિફાયર (AM માટે 10 kHz ની સરખામણીમાં 200 kHz) વાપરે છે
    \item નોઇઝ ઘટાડવા માટે લિમિટર સ્ટેજની જરૂર પડે છે
    \item FM ડિમોડ્યુલેશન માટે વિશિષ્ટ ડિસ્ક્રિમિનેટર વાપરે છે
\end{itemize}
\end{solutionbox}

\begin{mnemonicbox}
"MILD: મિક્સર, IF, લિમિટર, ડિસ્ક્રિમિનેટર - FM રિસેપ્શનમાં મુખ્ય ઘટકો"
\end{mnemonicbox}

\questionmarks{3(a)}{3}{વેવફોર્મ ટાઈમ અને ફ્રિક્વન્સી ડોમેન માં દોરો (અ) Impulse અને (બ) Pulse}

\begin{solutionbox}
\textbf{જવાબ}:

\begin{center}
\captionof{table}{Impulse અને Pulse લક્ષણો}
\begin{tabulary}{\linewidth}{|L|L|L|}
\hline
\textbf{સિગ્નલ} & \textbf{ટાઈમ ડોમેન} & \textbf{ફ્રિક્વન્સી ડોમેન} \\
\hline
\textbf{Impulse} & અનંત સાંકડો સ્પાઇક અનંત અમ્પ્લિટુડ સાથે & ફ્લેટ સ્પેક્ટ્રમ જેમાં બધી ફ્રિક્વન્સી સમાન રીતે હાજર હોય \\
\hline
\textbf{Pulse} & આયતાકાર આકાર સાથે મર્યાદિત પહોળાઈ અને ઊંચાઈ & Sinc ફંક્શન (sin(x)/x) આકાર \\
\hline
\end{tabulary}
\end{center}

\begin{center}
\begin{tikzpicture}[scale=0.8]
    % Impulse Time
    \draw[->] (0,0) -- (3,0) node[right] {$t$};
    \draw[->] (0,0) -- (0,2);
    \draw[thick, ->] (1.5,0) -- (1.5,1.5);
    \node[below] at (1.5,0) {$t_0$};
    \node[above] at (1.5,1.5) {$\delta(t)$};
    \node at (1.5,-0.5) {Impulse (Time)};
    
    % Impulse Freq
    \draw[->] (4,0) -- (7,0) node[right] {$f$};
    \draw[->] (4,0) -- (4,2);
    \draw[thick] (4,1) -- (7,1);
    \node at (5.5,-0.5) {Impulse (Freq)};
    
    % Pulse Time
    \draw[->] (0,-3) -- (3,-3) node[right] {$t$};
    \draw[->] (0,-3) -- (0,-1);
    \draw[thick] (1,-3) -- (1,-1.5) -- (2,-1.5) -- (2,-3);
    \node at (1.5,-3.5) {Pulse (Time)};
    
    % Pulse Freq (Sinc)
    \draw[->] (4,-3) -- (7,-3) node[right] {$f$};
    \draw[->] (4,-3) -- (4,-1);
    \draw[thick, domain=4:7, samples=100] plot (\x, {-3 + 1*sin(((\x-5.5)*5)*180/3.14159)/((\x-5.5)*5+0.001)}); 
    % Approximation of Sinc centered at 5.5
    \node at (5.5,-3.5) {Pulse (Freq)};
\end{tikzpicture}
\captionof{figure}{Impulse અને Pulse}
\end{center}
\end{solutionbox}

\begin{mnemonicbox}
"I-P: Impulse એ Pinpoint સ્પાઇક છે, Pulse ને Persistent પહોળાઈ છે"
\end{mnemonicbox}

\questionmarks{3(b)}{4}{અંડર સેમ્પલિંગ અને ક્રિટિકલ સેમ્પલિંગનું વર્ણન કરો}

\begin{solutionbox}
\textbf{જવાબ}:

\begin{center}
\captionof{table}{સેમ્પલિંગના પ્રકારો}
\begin{tabulary}{\linewidth}{|L|L|L|}
\hline
\textbf{સેમ્પલિંગનો પ્રકાર} & \textbf{વર્ણન} & \textbf{અસર} \\
\hline
\textbf{અંડર સેમ્પલિંગ} & સેમ્પલિંગ ફ્રિક્વન્સી $f_s < 2f_m$ (નાયક્વિસ્ટ રેટ કરતાં ઓછી) & એલિયાસિંગ થાય છે; સિગ્નલ પુનઃપ્રાપ્ત કરી શકાતો નથી \\
\hline
\textbf{ક્રિટિકલ સેમ્પલિંગ} & સેમ્પલિંગ ફ્રિક્વન્સી $f_s = 2f_m$ (ચોક્કસ નાયક્વિસ્ટ રેટ) & સૈદ્ધાંતિક રીતે સંપૂર્ણ પુનર્નિર્માણ શક્ય છે \\
\hline
\textbf{ઓવર સેમ્પલિંગ} & સેમ્પલિંગ ફ્રિક્વન્સી $f_s > 2f_m$ (નાયક્વિસ્ટ રેટ કરતાં વધારે) & વધુ સારું પુનર્નિર્માણ, સરળ ફિલ્ટરિંગ \\
\hline
\end{tabulary}
\end{center}

\begin{center}
\begin{tikzpicture}[scale=0.7]
    % Undersampling
    \draw[->] (0,0) -- (5,0) node[right] {$f$};
    \draw[thick, blue] (0.5,0) -- (1.5,1.5) -- (2.5,0); % Baseband
    \draw[thick, red] (1.5,0) -- (2.5,1.5) -- (3.5,0); % Replica overlapping
    \node[below] at (2, -0.5) {Aliasing ($f_s < 2f_m$)};
    \node[above] at (2, 0.5) {Overlap};
    
    % Critical Sampling
    \draw[->] (6,0) -- (11,0) node[right] {$f$};
    \draw[thick, blue] (6.5,0) -- (7.5,1.5) -- (8.5,0);
    \draw[thick, red] (8.5,0) -- (9.5,1.5) -- (10.5,0); % Touching
    \node[below] at (8.5, -0.5) {Critical ($f_s = 2f_m$)};
\end{tikzpicture}
\captionof{figure}{અંડર સેમ્પલિંગ vs ક્રિટિકલ સેમ્પલિંગ}
\end{center}
\end{solutionbox}

\begin{mnemonicbox}
"UCO: અંડર (fs<2fm), ક્રિટિકલ (fs=2fm), ઓવર (fs>2fm)"
\end{mnemonicbox}

\questionmarks{3(c)}{7}{PAM, PWM અને PPM સિગ્નલોને વેવફોર્મ સાથે જણાવો.}

\begin{solutionbox}
\textbf{જવાબ}:

\begin{center}
\captionof{table}{પલ્સ મોડ્યુલેશન ટેકનિક્સ}
\begin{tabulary}{\linewidth}{|L|L|L|}
\hline
\textbf{ટેકનિક} & \textbf{વર્ણન} & \textbf{સિગ્નલનું બદલાતું પરિમાણ} \\
\hline
\textbf{PAM} & પલ્સનું અમ્પ્લિટુડ મોડ્યુલેટિંગ સિગ્નલ અનુસાર બદલાય છે & અમ્પ્લિટુડ \\
\hline
\textbf{PWM} & પલ્સની પહોળાઈ/અવધિ મોડ્યુલેટિંગ સિગ્નલ અનુસાર બદલાય છે & પલ્સ પહોળાઈ \\
\hline
\textbf{PPM} & પલ્સની સ્થિતિ/સમય મોડ્યુલેટિંગ સિગ્નલ અનુસાર બદલાય છે & પલ્સ સ્થિતિ \\
\hline
\end{tabulary}
\end{center}

\begin{center}
\begin{tikzpicture}[scale=0.8]
    \draw[->] (0,6) -- (6,6) node[right] {$t$};
    \node[left] at (0,6.5) {Signal};
    \draw[thick, blue] (0,6) sin (1.5,7) cos (3,6) sin (4.5,5) cos (6,6);
    
    \draw[->] (0,4) -- (6,4) node[right] {$t$};
    \node[left] at (0,4.5) {PAM};
    \foreach \x/\y in {0.5/0.5, 1.5/1, 2.5/0.5, 3.5/-0.5, 4.5/-1, 5.5/-0.5}
        \draw[thick, red] (\x,4) -- (\x,4+\y);

    \draw[->] (0,2) -- (6,2) node[right] {$t$};
    \node[left] at (0,2.5) {PWM};
    % Width varies
    \draw[thick, red] (0.4,2) -- (0.4,2.5) -- (0.6,2.5) -- (0.6,2);
    \draw[thick, red] (1.3,2) -- (1.3,2.5) -- (1.7,2.5) -- (1.7,2); % Wider at peak
    \draw[thick, red] (2.4,2) -- (2.4,2.5) -- (2.6,2.5) -- (2.6,2);
    
    \draw[->] (0,0) -- (6,0) node[right] {$t$};
    \node[left] at (0,0.5) {PPM};
    % Position varies
    \draw[thick, red] (0.5,0) -- (0.5,0.5) -- (0.6,0.5) -- (0.6,0);
    \draw[thick, red] (1.6,0) -- (1.6,0.5) -- (1.7,0.5) -- (1.7,0); % Delayed
    \draw[thick, red] (2.5,0) -- (2.5,0.5) -- (2.6,0.5) -- (2.6,0);
\end{tikzpicture}
\captionof{figure}{PAM, PWM, PPM વેવફોર્મ્સ}
\end{center}
\end{solutionbox}

\begin{mnemonicbox}
"APP: અમ્પ્લિટુડ, પોઝિશન, પલ્સ-વિડ્થ અનુક્રમે બદલાય છે"
\end{mnemonicbox}

\questionmarks{3(a) OR}{3}{સેમ્પલિંગ થીયરમ જણાવો અને સમજાવો.}

\begin{solutionbox}
\textbf{જવાબ}:

\textbf{સેમ્પલિંગ થીયરમ સ્ટેટમેન્ટ:}
"બેન્ડ-લિમિટેડ કન્ટિન્યુઅસ-ટાઈમ સિગ્નલને તેના સેમ્પલ્સ દ્વારા સંપૂર્ણપણે રજૂ કરી શકાય છે અને પુનઃપ્રાપ્ત કરી શકાય છે, જો સેમ્પલિંગ ફ્રિક્વન્સી સિગ્નલમાં ઉચ્ચતમ ફ્રિક્વન્સી ઘટકના ઓછામાં ઓછી બે ગણી હોય."

\begin{center}
\captionof{table}{સેમ્પલિંગ થીયરમના મુખ્ય તત્વો}
\begin{tabulary}{\linewidth}{|L|L|}
\hline
\textbf{શબ્દ} & \textbf{વર્ણન} \\
\hline
\textbf{નાયક્વિસ્ટ રેટ} & જરૂરી ન્યૂનતમ સેમ્પલિંગ ફ્રિક્વન્સી ($f_s$) = $2f_m$ \\
\hline
\textbf{નાયક્વિસ્ટ ઇન્ટરવલ} & સેમ્પલ્સ વચ્ચેનો મહત્તમ સમય = $1/(2f_m)$ \\
\hline
\textbf{બેન્ડ-લિમિટેડ સિગ્નલ} & મર્યાદિત ઉચ્ચતમ ફ્રિક્વન્સી ઘટક ધરાવતું સિગ્નલ \\
\hline
\end{tabulary}
\end{center}
\end{solutionbox}

\begin{mnemonicbox}
"2F: ફ્રિક્વન્સીને તેની ઉચ્ચતમ ફ્રિક્વન્સીના ઓછામાં ઓછા બે ગણા પર સેમ્પલ કરવી જોઈએ"
\end{mnemonicbox}

\questionmarks{3(b) OR}{4}{કોન્ટાઇજેશન સમજાવો.}

\begin{solutionbox}
\textbf{જવાબ}:

\begin{center}
\captionof{table}{ક્વોન્ટાઈઝેશન કોન્સેપ્ટ્સ}
\begin{tabulary}{\linewidth}{|L|L|}
\hline
\textbf{શબ્દ} & \textbf{વર્ણન} \\
\hline
\textbf{ક્વોન્ટાઈઝેશન} & સતત અમ્પ્લિટુડ મૂલ્યોને ડિસ્ક્રીટ લેવલ્સમાં રૂપાંતરિત કરવાની પ્રક્રિયા \\
\hline
\textbf{ક્વોન્ટાઈઝેશન લેવલ્સ} & ઉપયોગમાં લેવાતા ડિસ્ક્રીટ મૂલ્યોની કુલ સંખ્યા (સામાન્ય રીતે $2^n$) \\
\hline
\textbf{ક્વોન્ટાઈઝેશન સ્ટેપ સાઈઝ} & નજીકના લેવલ્સ વચ્ચેનો વોલ્ટેજ તફાવત ($\Delta = V_{max}/2^n$) \\
\hline
\textbf{ક્વોન્ટાઈઝેશન એરર} & વાસ્તવિક સિગ્નલ મૂલ્ય અને ક્વોન્ટાઈઝ્ડ મૂલ્ય વચ્ચેનો તફાવત \\
\hline
\end{tabulary}
\end{center}

\begin{center}
\begin{tikzpicture}[scale=0.8]
    \draw[->] (0,0) -- (4,0) node[right] {$t$};
    \draw[->] (0,0) -- (0,3) node[above] {$V$};
    
    % Levels
    \foreach \y in {0.5, 1.0, 1.5, 2.0, 2.5} \draw[dotted, gray] (0,\y) -- (4,\y);
    
    % Signal
    \draw[blue, thick] (0,0.5) .. controls (1,2.8) and (3,0.2) .. (4,1.5);
    
    % Staircase
    \draw[red, thick] (0,0.5) -- (0.2,0.5) -- (0.2,1.0) -- (0.4,1.0) -- (0.4,1.5) -- (0.6,1.5) -- (0.6,2.0) -- (0.8,2.0) -- (0.8,2.5) -- (1.2,2.5) -- (1.2,2.0);
    \node[right, red] at (2, 2.5) {Quantized};
    \node[right, blue] at (2, 0.5) {Analog};
\end{tikzpicture}
\captionof{figure}{ક્વોન્ટાઈઝેશન પ્રક્રિયા}
\end{center}
\end{solutionbox}

\begin{mnemonicbox}
"LSED: લેવલ્સ, સ્ટેપ સાઈઝ, એરર, ડિસ્ક્રીટ વેલ્યુ"
\end{mnemonicbox}

\questionmarks{3(c) OR}{7}{કમ્પાન્ડિંગને વિગતવાર સમજાવો.}

\begin{solutionbox}
\textbf{જવાબ}:

\begin{center}
\captionof{table}{કમ્પાન્ડિંગ કોન્સેપ્ટ્સ}
\begin{tabulary}{\linewidth}{|L|L|}
\hline
\textbf{શબ્દ} & \textbf{વર્ણન} \\
\hline
\textbf{કમ્પાન્ડિંગ} & COMપ્રેસિંગ + exPANDિંગ; નોન-લિનિયર ક્વોન્ટાઈઝેશન ટેકનિક \\
\hline
\textbf{કમ્પ્રેશન} & ટ્રાન્સમિશન પહેલા સિગ્નલની અમ્પ્લિટુડ રેન્જ ઘટાડે છે \\
\hline
\textbf{એક્સપાન્શન} & રિસીવર પર મૂળ અમ્પ્લિટુડ રેન્જ પુનઃસ્થાપિત કરે છે \\
\hline
\textbf{હેતુ} & ડાયનેમિક રેન્જ જાળવી રાખતી વખતે નબળા સિગ્નલ માટે SNR સુધારે છે \\
\hline
\end{tabulary}
\end{center}

\begin{center}
\begin{tikzpicture}[node distance=2.5cm, auto, >=latex, thick]
    \node [gtu block] (in) {ઇનપુટ};
    \node [gtu block, right of=in] (comp) {કમ્પ્રેસર};
    \node [gtu block, right of=comp] (quant) {ક્વોન્ટાઈઝર};
    \node [gtu block, right of=quant] (ch) {ચેનલ};
    \node [gtu block, right of=ch] (exp) {એક્સપાન્ડર};
    \node [gtu block, right of=exp] (out) {આઉટપુટ};
    
    \draw [gtu arrow] (in) -- (comp);
    \draw [gtu arrow] (comp) -- (quant);
    \draw [gtu arrow] (quant) -- (ch);
    \draw [gtu arrow] (ch) -- (exp);
    \draw [gtu arrow] (exp) -- (out);
\end{tikzpicture}
\captionof{figure}{કમ્પાન્ડિંગ પ્રક્રિયા}
\end{center}

\textbf{કમ્પાન્ડિંગ લો:}
\begin{itemize}
    \item \textbf{$\mu$-law}: $y = \text{sgn}(x) \times \ln(1+\mu|x|)/\ln(1+\mu)$ જ્યાં $\mu = 255$ USA માં
    \item \textbf{A-law}: યુરોપમાં વપરાય છે.
\end{itemize}
\end{solutionbox}

\begin{mnemonicbox}
"CEQS: કમ્પ્રેસ, એનકોડ, ક્વોન્ટાઈઝ, સેન્ડ; પછી ડિકોડ, એક્સપાન્ડ, રિકવર"
\end{mnemonicbox}

\questionmarks{4(a)}{3}{ડેલ્ટા મોડ્યુલેશન સમજાવો}

\begin{solutionbox}
\textbf{જવાબ}:

\begin{center}
\captionof{table}{ડેલ્ટા મોડ્યુલેશન કોન્સેપ્ટ્સ}
\begin{tabulary}{\linewidth}{|L|L|}
\hline
\textbf{કોન્સેપ્ટ} & \textbf{વર્ણન} \\
\hline
\textbf{ડેલ્ટા મોડ્યુલેશન} & DPCM નું સૌથી સરળ રૂપ જ્યાં ફક્ત 1-બિટ ક્વોન્ટાઈઝેશન વાપરવામાં આવે છે \\
\hline
\textbf{સ્ટેપ સાઈઝ} & સિગ્નલને અનુમાનિત કરવામાં ફિક્સ્ડ વધારો/ઘટાડો \\
\hline
\textbf{આઉટપુટ} & બાઇનરી સ્ટ્રીમ (વધારા માટે 1, ઘટાડા માટે 0) \\
\hline
\textbf{ફાયદા} & સરળ અમલીકરણ, ઓછી બેન્ડવિડ્થ \\
\hline
\end{tabulary}
\end{center}

\begin{center}
\begin{tikzpicture}[scale=0.8]
    \draw[->] (0,0) -- (6,0) node[right] {$t$};
    \draw[->] (0,0) -- (0,4) node[above] {$m(t)$};
    
    % Signal
    \draw[thick, blue] (0,1) .. controls (2,3.5) and (4,0.5) .. (6,2);
    
    % Staircase
    \draw[thick, red] (0,1) -- (0.2,1) -- (0.2,1.2) -- (0.4,1.2) -- (0.4,1.4) -- (0.6,1.4) -- (0.6,1.6) -- (0.8,1.6) -- (0.8,1.8) -- (1.0,1.8) -- (1.0,2.0);
    
    \node[right, red] at (2, 2) {Approximation};
    \node[right, blue] at (4, 1) {Original};
    
    % Binary output below
    \node at (3,-1) {Binary: 1 1 1 1 ...};
\end{tikzpicture}
\captionof{figure}{ડેલ્ટા મોડ્યુલેશન}
\end{center}
\end{solutionbox}

\begin{mnemonicbox}
"1B1S: 1-બિટ, 1-સ્ટેપ ટ્રેકિંગ"
\end{mnemonicbox}

\questionmarks{4(b)}{4}{PCM ના ફાયદા અને ગેરફાયદા લખો}

\begin{solutionbox}
\textbf{જવાબ}:

\begin{center}
\captionof{table}{PCM ના ફાયદા અને ગેરફાયદા}
\begin{tabulary}{\linewidth}{|L|L|}
\hline
\textbf{ફાયદા} & \textbf{ગેરફાયદા} \\
\hline
ઉચ્ચ નોઇઝ ઇમ્યુનિટી & વધારે બેન્ડવિડ્થની જરૂર પડે છે \\
\hline
વધુ સારી સિગ્નલ ક્વોલિટી & જટિલ સિસ્ટમ અમલીકરણ \\
\hline
ડિજિટલ સિસ્ટમ સાથે સુસંગત & ક્વોન્ટાઈઝેશન નોઇઝ હાજર હોય છે \\
\hline
સુરક્ષિત ટ્રાન્સમિશન શક્ય છે & સિન્ક્રનાઈઝેશનની જરૂર પડે છે \\
\hline
મલ્ટિપ્લેક્સિંગ ક્ષમતા & વધુ પાવરની જરૂરિયાત \\
\hline
\end{tabulary}
\end{center}

\begin{center}
\begin{tikzpicture}[node distance=2.2cm, auto, >=latex, thick, scale=0.8, transform shape]
    \node [gtu block] (ana) {એનાલોગ\\સિગ્નલ};
    \node [gtu block, right of=ana] (samp) {સેમ્પલિંગ};
    \node [gtu block, right of=samp] (quant) {ક્વોન્ટાઈઝેશન};
    \node [gtu block, right of=quant] (enc) {એનકોડિંગ};
    \node [gtu block, right of=enc] (tx) {ડિજિટલ ટ્રાન્સમિશન};
    
    \draw [gtu arrow] (ana) -- (samp);
    \draw [gtu arrow] (samp) -- (quant);
    \draw [gtu arrow] (quant) -- (enc);
    \draw [gtu arrow] (enc) -- (tx);
\end{tikzpicture}
\captionof{figure}{PCM સિસ્ટમ ઓવરવ્યુ}
\end{center}
\end{solutionbox}

\begin{mnemonicbox}
"NCSMP: નોઇઝ ઇમ્યુનિટી, કમ્પેટિબલ વિથ ડિજિટલ, સિક્યોર, મલ્ટિપ્લેક્સિંગ, પ્રોસેસિંગ બેનિફિટ્સ"
\end{mnemonicbox}

\questionmarks{4(c)}{7}{PCM-TDM સિસ્ટમનો બ્લોક ડાયાગ્રામ દોરો અને સમજાવો.}

\begin{solutionbox}
\textbf{જવાબ}:

\begin{center}
\begin{tikzpicture}[node distance=2cm, auto, >=latex, thick, scale=0.7, transform shape]
    % Transmitter
    \node [gtu block] (mux) {Multiplexer};
    \node [left of=mux, node distance=2.5cm, yshift=1cm] (in1) {LPF 1};
    \node [left of=mux, node distance=2.5cm] (in2) {LPF 2};
    \node [left of=mux, node distance=2.5cm, yshift=-1cm] (in3) {LPF 3};
    
    \draw[->] (-4,1) -- (in1);
    \draw[->] (-4,0) -- (in2);
    \draw[->] (-4,-1) -- (in3);
    
    \node [left of=in1, node distance=1.5cm] {In 1};
    \node [left of=in2, node distance=1.5cm] {In 2};
    \node [left of=in3, node distance=1.5cm] {In 3};
    
    \draw[->] (in1) -- (mux);
    \draw[->] (in2) -- (mux);
    \draw[->] (in3) -- (mux);
    
    \node [gtu block, right of=mux, node distance=2.5cm] (enc) {PCM\\Encoder};
    \draw[->] (mux) -- (enc);
    
    % Channel
    \node [gtu block, right of=enc, node distance=2.5cm] (ch) {Channel};
    \draw[->] (enc) -- (ch);
    
    % Receiver
    \node [gtu block, right of=ch, node distance=2.5cm] (reg) {Regen\\Repeater};
    \node [gtu block, right of=reg, node distance=2.5cm] (dec) {PCM\\Decoder};
    \node [gtu block, right of=dec, node distance=2.5cm] (demux) {Demux};
    
    \draw[->] (ch) -- (reg);
    \draw[->] (reg) -- (dec);
    \draw[->] (dec) -- (demux);
    
    \node [right of=demux, node distance=2.5cm, yshift=1cm] (out1) {LPF 1};
    \node [right of=demux, node distance=2.5cm] (out2) {LPF 2};
    \node [right of=demux, node distance=2.5cm, yshift=-1cm] (out3) {LPF 3};
    
    \draw[->] (demux) -- (out1);
    \draw[->] (demux) -- (out2);
    \draw[->] (demux) -- (out3);
    
    \node [right of=out1, node distance=1.5cm] {Out 1};
    \node [right of=out2, node distance=1.5cm] {Out 2};
    \node [right of=out3, node distance=1.5cm] {Out 3};

\end{tikzpicture}
\captionof{figure}{PCM-TDM સિસ્ટમ}
\end{center}

\begin{center}
\captionof{table}{PCM-TDM સિસ્ટમ ઘટકો}
\begin{tabulary}{\linewidth}{|L|L|}
\hline
\textbf{ઘટક} & \textbf{કાર્ય} \\
\hline
\textbf{એન્ટી-એલિયાસિંગ ફિલ્ટર} & એલિયાસિંગ ટાળવા માટે સિગ્નલ બેન્ડવિડ્થને મર્યાદિત કરે છે \\
\hline
\textbf{મલ્ટીપ્લેક્સર} & એકલ ટાઇમ ડિવિઝન મલ્ટિપ્લેક્સ્ડ સ્ટ્રીમમાં ઘણા ઇનપુટ ચેનલો જૉડે છે \\
\hline
\textbf{ક્વોન્ટાઈઝર/એનકોડર} & સતત સેમ્પલ્સને ડિજીટલ કોડમાં રૂપાંતરિત કરે છે \\
\hline
\textbf{ફ્રેમ જનરેટર} & સિન્ક્રોનાઈઝેશન અને કંટ્રોલ બિટ્સ ઉમેરે છે \\
\hline
\textbf{ડિમલ્ટીપ્લેક્સર} & જોડાયેલા સિગ્નલને પાછા અલગ-અલગ ચેનલમાં વિભાજિત કરે છે \\
\hline
\textbf{રિકન્સ્ટ્રક્શન ફિલ્ટર} & એનાલોગ વેવફોર્મ પુનઃપ્રાપ્ત કરવા માટે ડિકોડેડ સિગ્નલને સ્મૂધ કરે છે \\
\hline
\end{tabulary}
\end{center}
\end{solutionbox}

\begin{mnemonicbox}
"SAMPLER: સેમ્પલ, એમ્પ્લિફાય, મલ્ટિપ્લેક્સ, પ્રોસેસ, લિમિટ, એનકોડ, રિકન્સ્ટ્રક્ટ"
\end{mnemonicbox}

\questionmarks{4(a) OR}{3}{સ્લોપ ઓવરલોડ એરરનું વર્ણન કરો.}

\begin{solutionbox}
\textbf{જવાબ}:

\begin{center}
\captionof{table}{સ્લોપ ઓવરલોડ એરર}
\begin{tabulary}{\linewidth}{|L|L|}
\hline
\textbf{કોન્સેપ્ટ} & \textbf{વર્ણન} \\
\hline
\textbf{સ્લોપ ઓવરલોડ એરર} & ઇનપુટ સિગ્નલ DM સ્ટેપ સાઈઝ કરતાં ઝડપથી બદલાય ત્યારે થતી ભૂલ \\
\hline
\textbf{કારણ} & ડેલ્ટા મોડ્યુલેશનમાં ફિક્સ્ડ સ્ટેપ સાઈઝ ઇનપુટના ઊંચા ઢાળ માટે ખૂબ નાની હોય છે \\
\hline
\textbf{અસર} & રિકન્સ્ટ્રક્ટેડ સિગ્નલમાં ડિસ્ટોર્શન, ખાસ કરીને ઉચ્ચ ફ્રિક્વન્સી પર \\
\hline
\textbf{ઉકેલ} & એડેપ્ટિવ ડેલ્ટા મોડ્યુલેશન (વેરિએબલ સ્ટેપ સાઈઝ) \\
\hline
\end{tabulary}
\end{center}

\begin{center}
\begin{tikzpicture}[scale=0.8]
    \draw[->] (0,0) -- (6,0) node[right] {$t$};
    
    % Steep signal
    \draw[thick, blue] (0,0) -- (1,3) -- (2,0);
    \node[right, blue] at (1,3) {$m(t)$};
    
    % Lagging steps
    \draw[thick, red] (0,0) -- (0.2,0) -- (0.2,0.2) -- (0.4,0.2) -- (0.4,0.4) -- (0.6,0.4) -- (0.6,0.6) -- (0.8,0.6) -- (0.8,0.8) -- (1.0,0.8);
    \node[right, red] at (1.2,1) {Slope Overload};
    
    \node at (3,-1) {Step size $\Delta$ too small};
\end{tikzpicture}
\captionof{figure}{સ્લોપ ઓવરલોડ એરર}
\end{center}
\end{solutionbox}

\begin{mnemonicbox}
"SOS: સિગ્નલ ઓવરટેક્સ સ્ટેપ્સ જ્યારે સ્લોપ સ્ટીપ હોય"
\end{mnemonicbox}

\questionmarks{4(b) OR}{4}{ડિફરન્શિયલ PCM નું ટ્રાન્સમીટર સમજાવો}

\begin{solutionbox}
\textbf{જવાબ}:

\begin{center}
\begin{tikzpicture}[node distance=2.5cm, auto, >=latex, thick]
    \node [gtu block, align=center] (diff) {ડિફરન્સ\\કેલ્ક્યુલેટર};
    \node [left of=diff, node distance=2.5cm, align=center] (sh) {સેમ્પલ\\\& હોલ્ડ};
    \node [left of=sh, node distance=2cm] (in) {Input};
    \node [gtu block, right of=diff, align=center] (quant) {ક્વોન્ટાઈઝર};
    \node [gtu block, right of=quant, align=center] (enc) {એનકોડર};
    \node [right of=enc, node distance=2cm] (out) {Output};
    
    \node [gtu block, below of=quant, align=center] (pred) {પ્રેડિક્ટર};
    \node [gtu block, below of=enc, node distance=2cm, xshift=-1cm, align=center] (add) {Adder};
    
    \draw [->] (in) -- (sh);
    \draw [gtu arrow] (sh) -- (diff);
    \draw [gtu arrow] (diff) -- (quant);
    \draw [gtu arrow] (quant) -- (enc);
    \draw [->] (enc) -- (out);
    
    \draw [gtu arrow] (quant) -- (add);
    \draw [gtu arrow] (add) -- (pred);
    \draw [gtu arrow] (pred) -| (diff);
    \draw [gtu arrow] (pred) -| (add);

\end{tikzpicture}
\captionof{figure}{DPCM ટ્રાન્સમીટર}
\end{center}

\begin{center}
\captionof{table}{DPCM ટ્રાન્સમીટર ઘટકો}
\begin{tabulary}{\linewidth}{|L|L|}
\hline
\textbf{ઘટક} & \textbf{કાર્ય} \\
\hline
\textbf{સેમ્પલ \& હોલ્ડ} & નિયમિત અંતરે એનાલોગ સિગ્નલ પકડે છે \\
\hline
\textbf{ડિફરન્સ કેલ્ક્યુલેટર} & વર્તમાન સેમ્પલ અને અનુમાનિત મૂલ્ય વચ્ચે એરર ગણે છે \\
\hline
\textbf{ક્વોન્ટાઈઝર} & એરર સિગ્નલને ડિસ્ક્રીટ લેવલમાં રૂપાંતરિત કરે છે \\
\hline
\textbf{એનકોડર} & ક્વોન્ટાઈઝ્ડ મૂલ્યોને બાઇનરી કોડમાં રૂપાંતરિત કરે છે \\
\hline
\textbf{પ્રેડિક્ટર} & અગાઉના મૂલ્યોના આધારે આગામી સેમ્પલનો અંદાજ લગાવે છે \\
\hline
\end{tabulary}
\end{center}
\end{solutionbox}

\begin{mnemonicbox}
"SDQEP: સેમ્પલ, ડિફરન્સ, ક્વોન્ટાઈઝ, એનકોડ, પ્રેડિક્ટ"
\end{mnemonicbox}

\questionmarks{4(c) OR}{7}{વિગતવાર PCM ટ્રાન્સમીટર સમજાવો}

\begin{solutionbox}
\textbf{જવાબ}:

\begin{center}
\begin{tikzpicture}[node distance=2.5cm, auto, >=latex, thick]
    \node [gtu block, align=center] (lpf) {Anti-Aliasing\\Filter};
    \node [left of=lpf, node distance=2.5cm] (in) {Analog In};
    \node [gtu block, right of=lpf] (sh) {Sample\\\& Hold};
    \node [gtu block, right of=sh] (quant) {Quantizer};
    \node [gtu block, right of=quant] (enc) {Encoder};
    \node [right of=enc, node distance=2.5cm] (out) {Digital Out};
    
    \draw [->] (in) -- (lpf);
    \draw [gtu arrow] (lpf) -- (sh);
    \draw [gtu arrow] (sh) -- (quant);
    \draw [gtu arrow] (quant) -- (enc);
    \draw [->] (enc) -- (out);
\end{tikzpicture}
\captionof{figure}{PCM ટ્રાન્સમીટર}
\end{center}

\textbf{PCM ટ્રાન્સમીટર ઘટકોની વિગત:}

\begin{center}
\captionof{table}{PCM ઘટકો વિગત}
\begin{tabulary}{\linewidth}{|L|L|L|}
\hline
\textbf{ઘટક} & \textbf{કાર્ય} & \textbf{ડિઝાઇન કન્સિડરેશન્સ} \\
\hline
\textbf{એન્ટી-એલિયાસિંગ ફિલ્ટર} & ઇનપુટ બેન્ડવિડ્થને $f_s/2$ સુધી મર્યાદિત કરે છે & કટઓફ ફ્રિક્વન્સી $< f_s/2$, શાર્પ રોલ-ઓફ \\
\hline
\textbf{સેમ્પલ \& હોલ્ડ} & ક્ષણિક સિગ્નલ મૂલ્ય પકડે છે & સેમ્પલિંગ રેટ $\ge 2f_m$ \\
\hline
\textbf{ક્વોન્ટાઈઝર} & સેમ્પલ અમ્પ્લિટ્યુડને ડિસ્ક્રીટ લેવલમાં અંદાજિત કરે છે & લેવલ્સ = $2^n$ \\
\hline
\textbf{એનકોડર} & ક્વોન્ટાઈઝ્ડ મૂલ્યોને ડિજિટલ કોડમાં રૂપાંતરિત કરે છે & NRZ, RZ કોડિંગ વપરાય છે \\
\hline
\end{tabulary}
\end{center}
\end{solutionbox}

\begin{mnemonicbox}
"SAFE-Q: સેમ્પલ એન્ડ ફિલ્ટર, ધેન એનકોડ આફ્ટર ક્વોન્ટાઈઝિંગ"
\end{mnemonicbox}

\questionmarks{5(a)}{3}{PCM અને DM ની સરખામણી કરો}

\begin{solutionbox}
\textbf{જવાબ}:

\begin{center}
\captionof{table}{PCM અને DM વચ્ચે તુલના}
\begin{tabulary}{\linewidth}{|L|L|L|}
\hline
\textbf{પરિમાણ} & \textbf{PCM} & \textbf{DM (ડેલ્ટા મોડ્યુલેશન)} \\
\hline
\textbf{બિટ રેટ} & ઉચ્ચ (પ્રતિ સેમ્પલ ઘણા બિટ્સ) & નીચો (પ્રતિ સેમ્પલ 1 બિટ) \\
\hline
\textbf{સર્કિટ જટિલતા} & વધુ જટિલ & સરળ \\
\hline
\textbf{સિગ્નલ ક્વોલિટી} & સારી & નીચી, સ્લોપ ઓવરલોડ \& ગ્રેન્યુલર નોઇઝથી પ્રભાવિત \\
\hline
\textbf{બેન્ડવિડ્થ} & વધુ પહોળી & સાંકડી \\
\hline
\textbf{સેમ્પલિંગ રેટ} & ઓછામાં ઓછી $2f_m$ & $2f_m$ કરતાં ઘણી વધારે \\
\hline
\end{tabulary}
\end{center}
\end{solutionbox}

\begin{mnemonicbox}
"BCSBS: બિટ રેટ, કમ્પ્લેક્સિટી, સિગ્નલ ક્વોલિટી, બેન્ડવિડ્થ, સેમ્પલિંગ"
\end{mnemonicbox}

\questionmarks{5(b)}{4}{વ્યાખ્યા આપો: (અ) Antenna (બ) Radiation pattern (ક) Directivity અને (ડ) Polarization}

\begin{solutionbox}
\textbf{જવાબ}:

\begin{center}
\captionof{table}{એન્ટેના શબ્દાવલી}
\begin{tabulary}{\linewidth}{|L|L|}
\hline
\textbf{શબ્દ} & \textbf{વ્યાખ્યા} \\
\hline
\textbf{એન્ટેના} & ઇલેક્ટ્રિકલ સિગ્નલ્સને ઇલેક્ટ્રોમેગ્નેટિક વેવ્સમાં અને તેનાથી ઉલટું ફેરવતું ઉપકરણ \\
\hline
\textbf{રેડિએશન પેટર્ન} & અંતરિક્ષ કોઓર્ડિનેટ્સના ફંક્શન તરીકે એન્ટેનાની રેડિએશન પ્રોપર્ટીઝનું ગ્રાફિકલ રેપ્રેઝન્ટેશન \\
\hline
\textbf{ડિરેક્ટિવિટી} & આપેલી દિશામાં રેડિએશન ઇન્ટેન્સિટીનો સરેરાશ રેડિએશન ઇન્ટેન્સિટી સાથેનો ગુણોત્તર \\
\hline
\textbf{પોલરાઇઝેશન} & એન્ટેના દ્વારા રેડિએટ થયેલા ઇલેક્ટ્રોમેગ્નેટિક વેવના ઇલેક્ટ્રિક ફિલ્ડ વેક્ટરની ઓરિએન્ટેશન \\
\hline
\end{tabulary}
\end{center}

\begin{center}
\begin{tikzpicture}[scale=0.6]
    % Coordinates
    \draw[->] (-3,0) -- (3,0);
    \draw[->] (0,-3) -- (0,3);
    
    % Main Lobe
    \draw[fill=blue!20] (0,0) -- (2,0.5) .. controls (3,0) .. (2,-0.5) -- cycle;
    
    % Side Lobes
    \draw[fill=blue!10] (0,0) -- (0.5,1) .. controls (0.8,0.8) .. (1,0.5) -- cycle;
    \draw[fill=blue!10] (0,0) -- (0.5,-1) .. controls (0.8,-0.8) .. (1,-0.5) -- cycle;
    
    % Back Lobe
    \draw[fill=blue!5] (0,0) -- (-1,0.3) .. controls (-1.5,0) .. (-1,-0.3) -- cycle;
    
    \node at (2.5,1) {Main Lobe};
    \node at (1.5,1.5) {Side Lobe};
    \node at (-2,0.5) {Back Lobe};
\end{tikzpicture}
\captionof{figure}{રેડિએશન પેટર્ન}
\end{center}
\end{solutionbox}

\begin{mnemonicbox}
"ARDP: એન્ટેના રેડિએટ વિથ ડિરેક્ટિવિટી એન્ડ પોલરાઈઝેશન"
\end{mnemonicbox}

\questionmarks{5(c)}{7}{સંક્ષિપ્ત નોંધ લખો (અ) સ્માર્ટ એન્ટેના (બ) પેરાબોલિક રિફ્લેક્ટર એન્ટેના}

\begin{solutionbox}
\textbf{જવાબ}:

\subsubsection*{(અ) સ્માર્ટ એન્ટેના}
\begin{center}
\captionof{table}{સ્માર્ટ એન્ટેના લક્ષણો}
\begin{tabulary}{\linewidth}{|L|L|}
\hline
\textbf{વિશેષતા} & \textbf{વર્ણન} \\
\hline
\textbf{વ્યાખ્યા} & બદલાતી પરિસ્થિતિઓ સાથે અનુકૂલિત થવાની ક્ષમતા સાથે એન્ટેના એરે સિગ્નલ પ્રોસેસિંગ \\
\hline
\textbf{પ્રકારો} & સ્વિચ્ડ બીમ, એડેપ્ટિવ એરે \\
\hline
\textbf{ફાયદા} & વધારેલી રેન્જ/કવરેજ, ઇન્ટરફેરન્સ ઘટાડો, ક્ષમતા સુધારણા \\
\hline
\textbf{એપ્લિકેશન્સ} & મોબાઇલ કમ્યુનિકેશન, 5G નેટવર્ક્સ, WiMAX, મિલિટરી સિસ્ટમ્સ (Military Systems) \\
\hline
\end{tabulary}
\end{center}

\textbf{બ્લોક ડાયાગ્રામ:}
\begin{center}
\begin{tikzpicture}[node distance=2.5cm, auto, >=latex, thick]
    \node [gtu block] (sp) {Signal\\Processor};
    \node [gtu block, left of=sp, node distance=3cm] (fe) {RF Front\\End};
    \node [left of=fe, node distance=2.5cm] (ant) {Array};
    \node [gtu block, right of=sp, node distance=3cm] (bf) {Beam\\Forming};
    
    \draw [thick] (ant) -- (fe);
    \foreach \y in {0.3, 0.1, -0.1, -0.3} \draw (ant.east) -- (fe.west); % Multiple lines
    
    \draw [gtu arrow] (fe) -- (sp);
    \draw [gtu arrow] (sp) -- (bf);
    \draw [gtu arrow] (bf) -- (fe); % Feedback
\end{tikzpicture}
\captionof{figure}{સ્માર્ટ એન્ટેના સિસ્ટમ}
\end{center}

\subsubsection*{(બ) પેરાબોલિક રિફ્લેક્ટર એન્ટેના}
\begin{center}
\captionof{table}{પેરાબોલિક રિફ્લેક્ટર લક્ષણો}
\begin{tabulary}{\linewidth}{|L|L|}
\hline
\textbf{વિશેષતા} & \textbf{વર્ણન} \\
\hline
\textbf{સ્ટ્રક્ચર} & ફોકલ પોઈન્ટ પર ફીડ એન્ટેના સાથે પેરાબોલિક રિફ્લેક્ટિંગ સરફેસ \\
\hline
\textbf{ઓપરેશન} & સમાંતર આવતા તરંગોને ફોકલ પોઈન્ટ પર કેન્દ્રિત કરે છે અથવા ફોકલ પોઈન્ટથી સમાંતર બીમ્સમાં રેડિએટ કરે છે \\
\hline
\textbf{ગેઇન} & ખૂબ ઉચ્ચ દિશાત્મકતા અને ગેઇન \\
\hline
\textbf{એપ્લિકેશન્સ} & સેટેલાઇટ કમ્યુનિકેશન, રેડિયો એસ્ટ્રોનોમી, રડાર સિસ્ટમ્સ \\
\hline
\end{tabulary}
\end{center}

\begin{center}
\begin{tikzpicture}[scale=0.8]
    % Parabola
    \draw[thick, domain=-2:2] plot (\x, {0.25*\x*\x});
    
    % Focus
    \node[draw, circle, fill=black, inner sep=1pt] (focus) at (0,1) {};
    \node[left] at (focus) {Feed};
    
    % Rays
    \draw[->, red] (0,1) -- (1.5, 0.56) -- (1.5, 3);
    \draw[->, red] (0,1) -- (-1.5, 0.56) -- (-1.5, 3);
    \draw[->, red] (0,1) -- (0.8, 0.16) -- (0.8, 3);
    \draw[->, red] (0,1) -- (-0.8, 0.16) -- (-0.8, 3);
    
    \node at (0,-0.5) {Parabolic Reflector};
\end{tikzpicture}
\captionof{figure}{પેરાબોલિક રિફ્લેક્ટર}
\end{center}
\end{solutionbox}

\begin{mnemonicbox}
"PFHS: પેરાબોલિક ફોકસ ગિવ્સ હાઇ સિગ્નલ સ્ટ્રેન્થ"
\end{mnemonicbox}

\questionmarks{5(a) OR}{3}{માઇક્રોસ્ટ્રીપ એન્ટેના પર ટૂંકી નોંધ લખો}

\begin{solutionbox}
\textbf{જવાબ}:

\begin{center}
\captionof{table}{માઇક્રોસ્ટ્રીપ એન્ટેના લક્ષણો}
\begin{tabulary}{\linewidth}{|L|L|}
\hline
\textbf{વિશેષતા} & \textbf{વર્ણન} \\
\hline
\textbf{સ્ટ્રક્ચર} & ગ્રાઉન્ડ પ્લેન સાથે ડાયલેક્ટ્રિક સબસ્ટ્રેટ પર કન્ડક્ટિવ પેચ \\
\hline
\textbf{આકાર} & લંબચોરસ, ગોળ, ઈંડાકાર, ત્રિકોણાકાર પેચ \\
\hline
\textbf{સાઈઝ} & સામાન્ય રીતે $\lambda/2$ લંબાઈમાં, ખૂબ પાતળી ($h \ll \lambda$) \\
\hline
\textbf{ફાયદા} & લો પ્રોફાઇલ, હલકા વજન, ઓછી કિંમત, સરળ ફેબ્રિકેશન, PCB ટેકનોલોજી સાથે સુસંગત \\
\hline
\textbf{ગેરફાયદા} & ઓછી કાર્યક્ષમતા, સાંકડી બેન્ડવિડ્થ, ઓછી પાવર હેન્ડલિંગ \\
\hline
\end{tabulary}
\end{center}

\begin{center}
\begin{tikzpicture}[scale=1]
    % Ground Plane
    \draw[fill=gray!20] (0,0) rectangle (4,0.2);
    \node[right] at (4,0.1) {Ground Plane};
    
    % Substrate
    \draw[fill=yellow!20] (0,0.2) rectangle (4,0.8);
    \node[right] at (4,0.5) {Dielectric Substrate};
    
    % Patch
    \draw[fill=orange!40] (1,0.8) rectangle (3,1.0);
    \node[right] at (3,0.9) {Radiating Patch};
    
    % Feed line
    \draw[thick] (1.5, 0) -- (1.5, 0.8); % Probe feed representation
\end{tikzpicture}
\captionof{figure}{માઇક્રોસ્ટ્રીપ પેચ એન્ટેના (Side View)}
\end{center}
\end{solutionbox}

\begin{mnemonicbox}
"PDGF: પેચ ઓન ડાયલેક્ટ્રિક વિથ ગ્રાઉન્ડ પ્લેન ગિવ્સ ફ્લેટ પ્રોફાઇલ"
\end{mnemonicbox}

\questionmarks{5(b) OR}{4}{EM વેવ સ્પેક્ટ્રમ, તેની ફ્રીક્વન્સી રેન્જ અને તેની એપ્લિકેશન્સ સમજાવો.}

\begin{solutionbox}
\textbf{જવાબ}:

\begin{center}
\captionof{table}{EM વેવ સ્પેક્ટ્રમ અને એપ્લિકેશન્સ}
\begin{tabulary}{\linewidth}{|L|L|L|L|}
\hline
\textbf{બેન્ડ} & \textbf{ફ્રિક્વન્સી રેન્જ} & \textbf{વેવલેન્થ} & \textbf{એપ્લિકેશન્સ} \\
\hline
\textbf{ELF} & 3-30 Hz & 100 Mm & સબમરીન કમ્યુનિકેશન \\
\hline
\textbf{VLF} & 3-30 kHz & 10-100 km & નેવિગેશન \\
\hline
\textbf{LF} & 30-300 kHz & 1-10 km & AM રેડિઓ \\
\hline
\textbf{MF} & 300k-3 MHz & 100m-1 km & AM બ્રોડકાસ્ટિંગ \\
\hline
\textbf{HF} & 3-30 MHz & 10-100 m & શોર્ટવેવ રેડિઓ \\
\hline
\textbf{VHF} & 30-300 MHz & 1-10 m & FM, TV \\
\hline
\textbf{UHF} & 300M-3 GHz & 10cm-1m & મોબાઇલ, WiFi \\
\hline
\textbf{SHF} & 3-30 GHz & 1-10 cm & સેટેલાઇટ, રડાર \\
\hline
\textbf{EHF} & 30-300 GHz & 1-10 mm & રેડિઓ એસ્ટ્રોનોમી \\
\hline
\textbf{Visible} & 400-800 THz & 380-750 nm & ઓપ્ટિકલ કમ્યુનિકેશન \\
\hline
\end{tabulary}
\end{center}

\begin{center}
\begin{tikzpicture}[scale=0.9, transform shape]
    \node [gtu block, fill=blue!10] (radio) {Radio};
    \node [gtu block, right of=radio, node distance=2.5cm, fill=blue!20] (micro) {Microwave};
    \node [gtu block, right of=micro, node distance=2.5cm, fill=orange!20] (ir) {IR};
    \node [gtu block, right of=ir, node distance=2.5cm, fill=yellow!20] (vis) {Visible};
    \node [gtu block, right of=vis, node distance=2.5cm, fill=purple!20] (uv) {UV};
    
    \draw [gtu arrow] (radio) -- (micro);
    \draw [gtu arrow] (micro) -- (ir);
    \draw [gtu arrow] (ir) -- (vis);
    \draw [gtu arrow] (vis) -- (uv);
    
    \node [below of=radio, node distance=1cm] {Low $f$};
    \node [below of=uv, node distance=1cm] {High $f$};
\end{tikzpicture}
\captionof{figure}{EM સ્પેક્ટ્રમ}
\end{center}
\end{solutionbox}

\begin{mnemonicbox}
"RVMIXG: રેડિઓ, વિઝિબલ, માઇક્રોવેવ, ઇન્ફ્રારેડ, X-રે, ગામા"
\end{mnemonicbox}

\questionmarks{5(c) OR}{7}{સંક્ષિપ્ત નોંધ લખો (અ) Space Wave Propagation અને (બ) Ground Wave Propagation પર સંક્ષિપ્ત નોંધ લખો.}

\begin{solutionbox}
\textbf{જવાબ}:

\subsubsection*{(અ) Space Wave Propagation}
\begin{center}
\captionof{table}{Space Wave Propagation લક્ષણો}
\begin{tabulary}{\linewidth}{|L|L|}
\hline
\textbf{વિશેષતા} & \textbf{વર્ણન} \\
\hline
\textbf{વ્યાખ્યા} & સ્પેસ દ્વારા સીધું વેવ પ્રોપેગેશન, જેમાં લાઇન-ઓફ-સાઇટ અને રિફ્લેક્ટેડ વેવ્સ શામેલ છે \\
\hline
\textbf{ફ્રિક્વન્સી રેન્જ} & VHF અને ઉપર (>30 MHz) \\
\hline
\textbf{અંતર} & હોરિઝન દ્વારા મર્યાદિત, સામાન્ય રીતે 50-80 km \\
\hline
\textbf{પ્રકારો} & ડાયરેક્ટ વેવ, ગ્રાઉન્ડ રિફ્લેક્ટેડ વેવ \\
\hline
\textbf{એપ્લિકેશન્સ} & TV બ્રોડકાસ્ટિંગ, માઇક્રોવેવ લિંક્સ, સેટેલાઇટ કમ્યુનિકેશન \\
\hline
\end{tabulary}
\end{center}

\begin{center}
\begin{tikzpicture}[scale=1]
    % Ground
    \draw[thick] (-4,0) -- (4,0);
    \node at (0,-0.3) {Earth Surface};
    
    % Tx Tower
    \draw[thick] (-3,0) -- (-3,1.5);
    \node[above] at (-3,1.5) {Tx};
    
    % Rx Tower
    \draw[thick] (3,0) -- (3,1.5);
    \node[above] at (3,1.5) {Rx};
    
    % Direct Wave
    \draw[->, blue, thick] (-3,1.5) -- node[above] {Direct Wave} (3,1.5);
    
    % Reflected Wave
    \draw[dashed, red, thick] (-3,1.5) -- (0,0) -- (3,1.5);
    \node[below] at (0,0) {Reflection Point};
\end{tikzpicture}
\captionof{figure}{Space Wave Propagation}
\end{center}

\subsubsection*{(બ) Ground Wave Propagation}
\begin{center}
\captionof{table}{Ground Wave Characteristics}
\begin{tabulary}{\linewidth}{|L|L|}
\hline
\textbf{વિશેષતા} & \textbf{વર્ણન} \\
\hline
\textbf{વ્યાખ્યા} & પૃથ્વીની સપાટી સાથે વેવ પ્રોપેગેશન, પૃથ્વીની વક્રતાને અનુસરે છે \\
\hline
\textbf{ફ્રિક્વન્સી રેન્જ} & LF, MF (2 MHz સુધી) \\
\hline
\textbf{અંતર} & ફ્રિક્વન્સી અને પાવર પર આધારિત 1000 km સુધી \\
\hline
\textbf{મેકેનિઝમ} & વર્ટિકલી પોલરાઇઝ્ડ વેવ કન્ડક્ટિવ અર્થ સરફેસને જોડાય છે \\
\hline
\textbf{એપ્લિકેશન્સ} & AM રેડિઓ બ્રોડકાસ્ટિંગ, મેરિટાઈમ કમ્યુનિકેશન \\
\hline
\end{tabulary}
\end{center}

\begin{center}
\begin{tikzpicture}[scale=1]
    % Curved Earth
    \draw[thick] (-4,0) to[bend left=10] (4,0);
    \node at (0,-1) {Earth Curvature};
    
    % Tx
    \draw[thick] (-3.5, -0.2) -- (-3.5, 1);
    \node[above] at (-3.5, 1) {Tx};
    
    % Rx
    \draw[thick] (3.5, -0.2) -- (3.5, 1);
    \node[above] at (3.5, 1) {Rx};
    
    % Surface Wave
    \draw[->, blue, thick, snake=coil, segment aspect=0] (-3.5, 0.5) to[bend left=15] node[above] {Ground Wave} (3.5, 0.5);
    
\end{tikzpicture}
\captionof{figure}{Ground Wave Propagation}
\end{center}
\end{solutionbox}

\begin{mnemonicbox}
"SHGM: સ્પેસ વેવ્સ ગો હાઇ, ગ્રાઉન્ડ વેવ્સ હગ મીડિયમ સરફેસ"
\end{mnemonicbox}

\end{document}

