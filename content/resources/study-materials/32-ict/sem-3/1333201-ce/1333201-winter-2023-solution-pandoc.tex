\documentclass[10pt,a4paper]{article}

% content/resources/templates/preamble.tex
\usepackage[margin=0.6in]{geometry}
\author{Milav Dabgar}
\usepackage{amsmath,amssymb,amsthm}
\usepackage{booktabs}
\usepackage{multirow}
\usepackage{xcolor}
\usepackage{tcolorbox}
\tcbuselibrary{breakable,skins}
\usepackage[colorlinks=true,linkcolor=blue]{hyperref}
\usepackage{titlesec}
\usepackage{enumitem}
\usepackage{tikz}
\usepackage{pgfplots}
\usepackage{circuitikz}
\usepackage[version=4]{mhchem}
\usepackage{longtable}
\usepackage{array}
\usepackage{float}
\usepackage{caption}
\usepackage{listings}

\lstset{
  basicstyle=\small\ttfamily,
  breaklines=true,
  breakatwhitespace=false,
  postbreak=\mbox{\textcolor{red}{$\hookrightarrow$}\space},
  float=false,
  numbers=left,
  numberstyle=\tiny\color{gray},
  numbersep=10pt,
  xleftmargin=2em,
  keywordstyle=\color{blue},
  commentstyle=\color{green!60!black},
  stringstyle=\color{purple},
  backgroundcolor=\color{gray!5},
  showstringspaces=false,
  tabsize=2,
  captionpos=b,
  keepspaces=true,
  columns=flexible
}

\pgfplotsset{compat=1.18}
\usetikzlibrary{shapes,arrows,positioning,calc,patterns,decorations.pathmorphing,decorations.markings,arrows.meta}

% Color scheme
\definecolor{headcolor}{RGB}{0,102,204}
\definecolor{keycolor}{RGB}{220,20,60}
\definecolor{solutioncolor}{RGB}{34,139,34}
\definecolor{mnemoniccolor}{RGB}{148,0,211}
\definecolor{codecolor}{RGB}{0,0,100}

% Spacing
\setlength{\parskip}{3pt}
\setlist[itemize]{nosep}
\setlist[enumerate]{nosep}

% Title formatting
\titleformat{\section}{\Large\bfseries\color{headcolor}}{\thesection}{1em}{}
\titleformat{\subsection}{\large\bfseries\color{headcolor}}{\thesubsection}{1em}{}

% Pandoc tightlist compatibility
\providecommand{\tightlist}{%
  \setlength{\itemsep}{0pt}\setlength{\parskip}{0pt}}

% Pandoc longtable compatibility
\newcounter{none}
\def\thenone{}


% content/resources/templates/english-boxes.tex
% This file is currently empty - it exists to maintain consistency with the import structure.
% Add custom environments here if needed in the future.


\begin{document}

\begin{center}
{\Huge\bfseries\color{headcolor} Subject Name Solutions}\\[5pt]
{\LARGE 1333201 -- Winter 2023}\\[3pt]
{\large Semester 1 Study Material}\\[3pt]
{\normalsize\textit{Detailed Solutions and Explanations}}
\end{center}

\vspace{10pt}

\subsection*{Question 1(a) [3 marks]}\label{q1a}

\textbf{Define: (A) Amplitude Modulation, (B) Frequency Modulation, and
(C) Phase Modulation}

\begin{solutionbox}


{\def\LTcaptype{none} % do not increment counter
\vspace{-5pt}
\captionof{table}{Types of Modulation Techniques}
\vspace{-10pt}
\begin{longtable}[]{@{}
  >{\raggedright\arraybackslash}p{(\linewidth - 2\tabcolsep) * \real{0.5714}}
  >{\raggedright\arraybackslash}p{(\linewidth - 2\tabcolsep) * \real{0.4286}}@{}}
\toprule\noalign{}
\begin{minipage}[b]{\linewidth}\raggedright
Modulation Type
\end{minipage} & \begin{minipage}[b]{\linewidth}\raggedright
Definition
\end{minipage} \\
\midrule\noalign{}
\endhead
\bottomrule\noalign{}
\endlastfoot
\textbf{Amplitude Modulation (AM)} & Process where amplitude of carrier
signal is varied according to the instantaneous value of modulating
signal while frequency remains constant \\
\textbf{Frequency Modulation (FM)} & Process where frequency of carrier
signal is varied according to the instantaneous value of modulating
signal while amplitude remains constant \\
\textbf{Phase Modulation (PM)} & Process where phase of carrier signal
is varied according to the instantaneous value of modulating signal
while amplitude remains constant \\
\end{longtable}
}

\end{solutionbox}
\begin{mnemonicbox}
``A-F-P: Amplitude changes, Frequency shifts, Phase
adjusts''

\end{mnemonicbox}
\subsection*{Question 1(b) [4 marks]}\label{q1b}

\textbf{Explain the need for modulation.}

\begin{solutionbox}


{\def\LTcaptype{none} % do not increment counter
\vspace{-5pt}
\captionof{table}{Need for Modulation}
\vspace{-10pt}
\begin{longtable}[]{@{}
  >{\raggedright\arraybackslash}p{(\linewidth - 2\tabcolsep) * \real{0.3158}}
  >{\raggedright\arraybackslash}p{(\linewidth - 2\tabcolsep) * \real{0.6842}}@{}}
\toprule\noalign{}
\begin{minipage}[b]{\linewidth}\raggedright
Need
\end{minipage} & \begin{minipage}[b]{\linewidth}\raggedright
Explanation
\end{minipage} \\
\midrule\noalign{}
\endhead
\bottomrule\noalign{}
\endlastfoot
\textbf{Practical Antenna Size} & Reduces antenna size by increasing
frequency (Antenna length = λ/4) \\
\textbf{Interference Reduction} & Allows multiple signals to be
transmitted simultaneously on different frequencies \\
\textbf{Range Extension} & Higher frequency signals travel farther in
atmosphere \\
\textbf{Multiplexing} & Enables multiple signals to share communication
medium \\
\end{longtable}
}

\textbf{Diagram:}

\includegraphics[width=1\linewidth,height=\textheight,keepaspectratio]{mermaid-f292617d.pdf}

\end{solutionbox}
\begin{mnemonicbox}
``PIRM: Practical antennas, Interference reduction,
Range extension, Multiplexing''

\end{mnemonicbox}
\subsection*{Question 1(c) [7 marks]}\label{q1c}

\textbf{A modulating signal has amplitude of 3 V and frequency of 1 KHz
is amplitude modulated by a carrier of amplitude 10 V and frequency
30KHz. Find modulation index, frequencies of sideband components and
their amplitudes. Also draw the spectrum of AM wave.}

\begin{solutionbox}


{\def\LTcaptype{none} % do not increment counter
\vspace{-5pt}
\captionof{table}{Given Information}
\vspace{-10pt}
\begin{longtable}[]{@{}lll@{}}
\toprule\noalign{}
Parameter & Modulating Signal & Carrier Signal \\
\midrule\noalign{}
\endhead
\bottomrule\noalign{}
\endlastfoot
Amplitude & 3 V & 10 V \\
Frequency & 1 kHz & 30 kHz \\
\end{longtable}
}

\textbf{Calculations:}

\begin{itemize}
\tightlist
\item
  \textbf{Modulation Index (m)} = Am/Ac = 3/10 = 0.3
\item
  \textbf{Sideband Frequencies} = fc \pm fm = 30 \pm 1 = 29 kHz and 31 kHz
\item
  \textbf{Sideband Amplitudes} = m \times Ac/2 = 0.3 \times 10/2 = 1.5 V
\end{itemize}

\textbf{Diagram: AM Spectrum}

\begin{lstlisting}
                                ┌───┐
                                │   │
                                │   │ 10V
                                │   │
                                │   │
                                │   │
     ┌───┐                      │   │                      ┌───┐
     │   │                      │   │                      │   │
     │   │ 1.5V                 │   │                      │   │ 1.5V
     │   │                      │   │                      │   │
     │   │                      │   │                      │   │
─────┴───┴──────────────────────┴───┴──────────────────────┴───┴─────────▶ f
            29kHz                30kHz                31kHz
         (fc - fm)                 fc               (fc + fm)
\end{lstlisting}

\end{solutionbox}
\begin{mnemonicbox}
``LSB-C-USB: Lower sideband, Carrier, Upper sideband
at 29-30-31''

\end{mnemonicbox}
\subsection*{Question 1(c) OR [7
marks]}\label{q1c}

\textbf{Derive mathematical relation between carrier powers, and
modulated signal power for AM.}

\begin{solutionbox}

\textbf{Mathematical Relation:}

\begin{itemize}
\tightlist
\item
  Carrier signal: c(t) = Ac cos(2πfc·t)
\item
  Modulating signal: m(t) = Am cos(2πfm·t)
\item
  AM signal: s(t) = Ac[1 + m·cos(2πfm·t)]·cos(2πfc·t)
\end{itemize}


{\def\LTcaptype{none} % do not increment counter
\vspace{-5pt}
\captionof{table}{Power Distribution in AM}
\vspace{-10pt}
\begin{longtable}[]{@{}lll@{}}
\toprule\noalign{}
Component & Expression & In Terms of Pc \\
\midrule\noalign{}
\endhead
\bottomrule\noalign{}
\endlastfoot
Carrier Power (Pc) & Ac^{2}/2 & Pc \\
Total Sideband Power (Ps) & m^{2}·Ac^{2}/4 & m^{2}·Pc/2 \\
Total AM Power (Pt) & Pc(1 + m^{2}/2) & Pc(1 + m^{2}/2) \\
\end{longtable}
}

\textbf{Diagram: Power Distribution}

\includegraphics[width=1\linewidth,height=\textheight,keepaspectratio]{mermaid-88258b99.pdf}

\begin{itemize}
\tightlist
\item
  \textbf{Modulation Efficiency} = Ps/Pt = (m^{2}/2)/(1 + m^{2}/2) \times 100\%
\end{itemize}

\end{solutionbox}
\begin{mnemonicbox}
``Total Power = Carrier Power \times (1 + m^{2}/2)''

\end{mnemonicbox}
\subsection*{Question 2(a) [3 marks]}\label{q2a}

\textbf{Compare AM and FM.}

\begin{solutionbox}


{\def\LTcaptype{none} % do not increment counter
\vspace{-5pt}
\captionof{table}{Comparison between AM and FM}
\vspace{-10pt}
\begin{longtable}[]{@{}lll@{}}
\toprule\noalign{}
Parameter & AM & FM \\
\midrule\noalign{}
\endhead
\bottomrule\noalign{}
\endlastfoot
\textbf{Modulation Parameter} & Amplitude varies & Frequency varies \\
\textbf{Bandwidth} & 2 \times fm & 2 \times (Δf + fm) \\
\textbf{Noise Immunity} & Poor & Excellent \\
\textbf{Power Efficiency} & Low & High \\
\textbf{Circuit Complexity} & Simple & Complex \\
\end{longtable}
}

\end{solutionbox}
\begin{mnemonicbox}
``ABNPC: Amplitude/Bandwidth/Noise/Power/Complexity
differences''

\end{mnemonicbox}
\subsection*{Question 2(b) [4 marks]}\label{q2b}

\textbf{Explain envelope detector with the help of circuit diagram.}

\begin{solutionbox}

\textbf{Diagram: Envelope Detector Circuit}

\begin{lstlisting}
    ┌─────┐     D     ┌───┬───┐
    │     │     ▶|    │   │   │
AM  │     ├────────┬──┤   │   │  Demodulated
Inpt│     │        │  │   │  ┌┴┐ Output
    │     │        │  │   │  │R│
    └─────┘        │  │   │  │L│
                   │  │   │  └┬┘
                   │  │ C │   │
                   │  │   │   │
                   └──┴───┴───┘
\end{lstlisting}


{\def\LTcaptype{none} % do not increment counter
\vspace{-5pt}
\captionof{table}{Envelope Detector Components}
\vspace{-10pt}
\begin{longtable}[]{@{}
  >{\raggedright\arraybackslash}p{(\linewidth - 2\tabcolsep) * \real{0.5238}}
  >{\raggedright\arraybackslash}p{(\linewidth - 2\tabcolsep) * \real{0.4762}}@{}}
\toprule\noalign{}
\begin{minipage}[b]{\linewidth}\raggedright
Component
\end{minipage} & \begin{minipage}[b]{\linewidth}\raggedright
Function
\end{minipage} \\
\midrule\noalign{}
\endhead
\bottomrule\noalign{}
\endlastfoot
\textbf{Diode (D)} & Rectifies AM signal to extract positive half
cycles \\
\textbf{Capacitor (C)} & Charges to peak of input, holds charge between
peaks \\
\textbf{Resistor (RL)} & Discharges capacitor at rate suitable for
envelope extraction \\
\end{longtable}
}

\textbf{Time Constant Selection:}

\begin{itemize}
\tightlist
\item
  1/fm \textless\textless{} RC \textless\textless{} 1/fc (for proper
  envelope detection)
\end{itemize}

\end{solutionbox}
\begin{mnemonicbox}
``DCR: Diode rectifies, Capacitor charges, Resistor
discharges''

\end{mnemonicbox}
\subsection*{Question 2(c) [7 marks]}\label{q2c}

\textbf{Draw and explain the block diagram of Superheterodyne receiver.}

\begin{solutionbox}

\textbf{Diagram: Superheterodyne Receiver}

\includegraphics[width=1\linewidth,height=\textheight,keepaspectratio]{mermaid-282ccfdf.pdf}


{\def\LTcaptype{none} % do not increment counter
\vspace{-5pt}
\captionof{table}{Functions of Superheterodyne Receiver Blocks}
\vspace{-10pt}
\begin{longtable}[]{@{}
  >{\raggedright\arraybackslash}p{(\linewidth - 2\tabcolsep) * \real{0.4118}}
  >{\raggedright\arraybackslash}p{(\linewidth - 2\tabcolsep) * \real{0.5882}}@{}}
\toprule\noalign{}
\begin{minipage}[b]{\linewidth}\raggedright
Block
\end{minipage} & \begin{minipage}[b]{\linewidth}\raggedright
Function
\end{minipage} \\
\midrule\noalign{}
\endhead
\bottomrule\noalign{}
\endlastfoot
\textbf{RF Amplifier} & Amplifies weak RF signal, provides selectivity,
rejects image frequency \\
\textbf{Local Oscillator} & Generates frequency fo = fRF + fIF for
mixing \\
\textbf{Mixer} & Combines RF signal with local oscillator to produce IF
(Intermediate Frequency) \\
\textbf{IF Amplifier} & Provides most of the receiver gain and
selectivity at fixed frequency \\
\textbf{Detector} & Extracts the modulating signal from the IF signal \\
\textbf{AF Amplifier} & Amplifies recovered audio to drive speaker \\
\end{longtable}
}

\end{solutionbox}
\begin{mnemonicbox}
``RLMIDS: RF, Local oscillator, Mixer, IF, Detector,
Speaker''

\end{mnemonicbox}
\subsection*{Question 2(a) OR [3
marks]}\label{q2a}

\textbf{Define the followings terms: (A) Sensitivity, and (B)
Selectivity}

\begin{solutionbox}


{\def\LTcaptype{none} % do not increment counter
\vspace{-5pt}
\captionof{table}{Receiver Characteristics}
\vspace{-10pt}
\begin{longtable}[]{@{}
  >{\raggedright\arraybackslash}p{(\linewidth - 2\tabcolsep) * \real{0.3333}}
  >{\raggedright\arraybackslash}p{(\linewidth - 2\tabcolsep) * \real{0.6667}}@{}}
\toprule\noalign{}
\begin{minipage}[b]{\linewidth}\raggedright
Term
\end{minipage} & \begin{minipage}[b]{\linewidth}\raggedright
Definition
\end{minipage} \\
\midrule\noalign{}
\endhead
\bottomrule\noalign{}
\endlastfoot
\textbf{Sensitivity} & Ability of receiver to detect and amplify weak
signals; measured as minimum input signal strength (µV) needed for
standard output \\
\textbf{Selectivity} & Ability of receiver to separate desired signal
from adjacent channels; measured as ratio of response at resonant
frequency to off-resonant frequency \\
\end{longtable}
}

\textbf{Diagram: Selectivity Curve}

\begin{lstlisting}
    │     ▲
    │     │Response
    │     │
    │     │      ┌───┐
    │     │      │   │
    │     │      │   │
    │     │      │   │
    │     │   ┌──┘   └──┐
    │     │ ┌─┘         └─┐
    │     └─┘             └─┐
    └─────────────────────────▶
          f1   fc    f2    Frequency
\end{lstlisting}

\end{solutionbox}
\begin{mnemonicbox}
``SS: Signal Strength for Sensitivity, Signal
Separation for Selectivity''

\end{mnemonicbox}
\subsection*{Question 2(b) OR [4
marks]}\label{q2b}

\textbf{Describe the block diagram of general communication system.}

\begin{solutionbox}

\textbf{Diagram: General Communication System}

\includegraphics[width=1\linewidth,height=\textheight,keepaspectratio]{mermaid-1ed3b125.pdf}


{\def\LTcaptype{none} % do not increment counter
\vspace{-5pt}
\captionof{table}{Components of Communication System}
\vspace{-10pt}
\begin{longtable}[]{@{}
  >{\raggedright\arraybackslash}p{(\linewidth - 2\tabcolsep) * \real{0.5238}}
  >{\raggedright\arraybackslash}p{(\linewidth - 2\tabcolsep) * \real{0.4762}}@{}}
\toprule\noalign{}
\begin{minipage}[b]{\linewidth}\raggedright
Component
\end{minipage} & \begin{minipage}[b]{\linewidth}\raggedright
Function
\end{minipage} \\
\midrule\noalign{}
\endhead
\bottomrule\noalign{}
\endlastfoot
\textbf{Information Source} & Generates message to be communicated
(voice, data, video) \\
\textbf{Transmitter} & Converts message into signals suitable for
transmission \\
\textbf{Channel} & Medium through which signals travel (wire, fiber,
air) \\
\textbf{Receiver} & Extracts original message from received signals \\
\textbf{Destination} & Entity for which message is intended \\
\textbf{Noise Source} & Unwanted signals that interfere with the
message \\
\end{longtable}
}

\end{solutionbox}
\begin{mnemonicbox}
``I-T-C-R-D: Information Travels Carefully, Reaches
Destination''

\end{mnemonicbox}
\subsection*{Question 2(c) OR [7
marks]}\label{q2c}

\textbf{Draw and explain the block diagram of Superheterodyne FM
receiver.}

\begin{solutionbox}

\textbf{Diagram: Superheterodyne FM Receiver}

\includegraphics[width=1\linewidth,height=\textheight,keepaspectratio]{mermaid-ffddae11.pdf}


{\def\LTcaptype{none} % do not increment counter
\vspace{-5pt}
\captionof{table}{Additional Components in FM Receiver}
\vspace{-10pt}
\begin{longtable}[]{@{}
  >{\raggedright\arraybackslash}p{(\linewidth - 2\tabcolsep) * \real{0.5238}}
  >{\raggedright\arraybackslash}p{(\linewidth - 2\tabcolsep) * \real{0.4762}}@{}}
\toprule\noalign{}
\begin{minipage}[b]{\linewidth}\raggedright
Component
\end{minipage} & \begin{minipage}[b]{\linewidth}\raggedright
Function
\end{minipage} \\
\midrule\noalign{}
\endhead
\bottomrule\noalign{}
\endlastfoot
\textbf{Limiter} & Removes amplitude variations, provides constant
amplitude signal \\
\textbf{FM Discriminator} & Converts frequency variations to amplitude
variations (demodulation) \\
\textbf{De-emphasis} & Attenuates higher frequencies boosted at
transmitter \\
\end{longtable}
}

\textbf{Unique Aspects of FM Receiver:}

\begin{itemize}
\tightlist
\item
  Uses wider bandwidth IF amplifier (200 kHz vs 10 kHz for AM)
\item
  Requires limiter stage for noise reduction
\item
  Employs specialized discriminator for FM demodulation
\end{itemize}

\end{solutionbox}
\begin{mnemonicbox}
``MILD: Mixer, IF, Limiter, Discriminator - key
components in FM reception''

\end{mnemonicbox}
\subsection*{Question 3(a) [3 marks]}\label{q3a}

\textbf{Draw the waveform of (A) Impulse (B) Pulse in time and frequency
domain}

\begin{solutionbox}


{\def\LTcaptype{none} % do not increment counter
\vspace{-5pt}
\captionof{table}{Impulse and Pulse Characteristics}
\vspace{-10pt}
\begin{longtable}[]{@{}
  >{\raggedright\arraybackslash}p{(\linewidth - 4\tabcolsep) * \real{0.2051}}
  >{\raggedright\arraybackslash}p{(\linewidth - 4\tabcolsep) * \real{0.3333}}
  >{\raggedright\arraybackslash}p{(\linewidth - 4\tabcolsep) * \real{0.4615}}@{}}
\toprule\noalign{}
\begin{minipage}[b]{\linewidth}\raggedright
Signal
\end{minipage} & \begin{minipage}[b]{\linewidth}\raggedright
Time Domain
\end{minipage} & \begin{minipage}[b]{\linewidth}\raggedright
Frequency Domain
\end{minipage} \\
\midrule\noalign{}
\endhead
\bottomrule\noalign{}
\endlastfoot
\textbf{Impulse} & Infinitely narrow spike with infinite amplitude &
Flat spectrum with all frequencies equally present \\
\textbf{Pulse} & Rectangular shape with finite width and height & Sinc
function (sin(x)/x) shape \\
\end{longtable}
}

\textbf{Diagram: Impulse and Pulse}

\begin{lstlisting}
Time Domain                      Frequency Domain
     
Impulse                          Impulse
    │                                │
    │                                │
    │ ↑                              │───────────────
    │ │                              │
    └─┼─────────▶                    └────────────────▶
      t_{0}                               f

Pulse                            Pulse
    │                                │
    │  ┌───────┐                     │    ┌─┐
    │  │       │                     │    │ │
    │  │       │                     │  ┌─┘ └─┐  ┌─┐
    └──┴───────┴────▶                └──┴─────┴──┴─┴───▶
       t_{0}  t_{0}+T                         f_{0}  2f_{0}  3f_{0}
\end{lstlisting}

\end{solutionbox}
\begin{mnemonicbox}
``I-P: Impulse is a Pinpoint spike, Pulse has
Persistent width''

\end{mnemonicbox}
\subsection*{Question 3(b) [4 marks]}\label{q3b}

\textbf{Describe under sampling and critical sampling}

\begin{solutionbox}


{\def\LTcaptype{none} % do not increment counter
\vspace{-5pt}
\captionof{table}{Types of Sampling}
\vspace{-10pt}
\begin{longtable}[]{@{}
  >{\raggedright\arraybackslash}p{(\linewidth - 4\tabcolsep) * \real{0.4615}}
  >{\raggedright\arraybackslash}p{(\linewidth - 4\tabcolsep) * \real{0.3333}}
  >{\raggedright\arraybackslash}p{(\linewidth - 4\tabcolsep) * \real{0.2051}}@{}}
\toprule\noalign{}
\begin{minipage}[b]{\linewidth}\raggedright
Type of Sampling
\end{minipage} & \begin{minipage}[b]{\linewidth}\raggedright
Description
\end{minipage} & \begin{minipage}[b]{\linewidth}\raggedright
Effect
\end{minipage} \\
\midrule\noalign{}
\endhead
\bottomrule\noalign{}
\endlastfoot
\textbf{Under Sampling} & Sampling frequency fs \textless{} 2fm (less
than Nyquist rate) & Aliasing occurs; signal cannot be recovered \\
\textbf{Critical Sampling} & Sampling frequency fs = 2fm (exactly
Nyquist rate) & Theoretically perfect reconstruction possible \\
\textbf{Over Sampling} & Sampling frequency fs \textgreater{} 2fm
(exceeds Nyquist rate) & Better reconstruction, easier filtering \\
\end{longtable}
}

\textbf{Diagram: Under Sampling vs Critical Sampling}

\begin{lstlisting}
Under Sampling (fs < 2fm)
    │     ┌───┐     ┌───┐     ┌───┐     ┌───┐
    │     │   │     │   │     │   │     │   │
    │─────┘   └─────┘   └─────┘   └─────┘   └────▶
    ↑     ↑     ↑     ↑     ↑
    Aliasing occurs - samples too far apart

Critical Sampling (fs = 2fm)
    │     ┌───┐     ┌───┐     ┌───┐     ┌───┐
    │     │   │     │   │     │   │     │   │
    │─────┘   └─────┘   └─────┘   └─────┘   └────▶
    ↑   ↑   ↑   ↑   ↑   ↑   ↑   ↑
    Just enough samples to reconstruct
\end{lstlisting}

\end{solutionbox}
\begin{mnemonicbox}
``UCO: Under (fs\textless2fm), Critical (fs=2fm),
Over (fs\textgreater2fm)''

\end{mnemonicbox}
\subsection*{Question 3(c) [7 marks]}\label{q3c}

\textbf{State the PAM, PWM and PPM signals with waveform.}

\begin{solutionbox}


{\def\LTcaptype{none} % do not increment counter
\vspace{-5pt}
\captionof{table}{Pulse Modulation Techniques}
\vspace{-10pt}
\begin{longtable}[]{@{}
  >{\raggedright\arraybackslash}p{(\linewidth - 4\tabcolsep) * \real{0.2292}}
  >{\raggedright\arraybackslash}p{(\linewidth - 4\tabcolsep) * \real{0.2708}}
  >{\raggedright\arraybackslash}p{(\linewidth - 4\tabcolsep) * \real{0.5000}}@{}}
\toprule\noalign{}
\begin{minipage}[b]{\linewidth}\raggedright
Technique
\end{minipage} & \begin{minipage}[b]{\linewidth}\raggedright
Description
\end{minipage} & \begin{minipage}[b]{\linewidth}\raggedright
Signal Parameter Varied
\end{minipage} \\
\midrule\noalign{}
\endhead
\bottomrule\noalign{}
\endlastfoot
\textbf{PAM (Pulse Amplitude Modulation)} & Amplitude of pulses varies
according to modulating signal & Amplitude \\
\textbf{PWM (Pulse Width Modulation)} & Width/duration of pulses varies
according to modulating signal & Pulse width \\
\textbf{PPM (Pulse Position Modulation)} & Position/timing of pulses
varies according to modulating signal & Pulse position \\
\end{longtable}
}

\textbf{Diagram: PAM, PWM, PPM Waveforms}

\begin{lstlisting}
Modulating Signal
    │    ┌───┐
    │   /     \
    │  /       \
    │ /         \        /\
    │/           \      /  \
    │             \    /    \
    │              \  /      \
    └───────────────\/────────────▶

PAM
    │    ┌─┐   ┌┐  ┌┐   ┌─┐
    │    │ │   ││  ││   │ │
    │    │ │   ││  ││   │ │
    │    │ │   ││  ││   │ │
    └────┘ └───┘└──┘└───┘ └────▶

PWM
    │    ┌───┐ ┌─┐ ┌┐  ┌──┐
    │    │   │ │ │ ││  │  │
    │    │   │ │ │ ││  │  │
    │    │   │ │ │ ││  │  │
    └────┘   └─┘ └─┘└──┘  └────▶

PPM
    │    ┌┐    ┌┐   ┌┐    ┌┐
    │    ││    ││   ││    ││
    │    ││    ││   ││    ││
    │    ││    ││   ││    ││
    └────┘└────┘└───┘└────┘└────▶
\end{lstlisting}

\end{solutionbox}
\begin{mnemonicbox}
``APP: Amplitude, Position, Pulse-width change
respectively''

\end{mnemonicbox}
\subsection*{Question 3(a) OR [3
marks]}\label{q3a}

\textbf{State and explain sampling theorem.}

\begin{solutionbox}

\textbf{Sampling Theorem Statement:} ``A band-limited continuous-time
signal can be completely represented by and reconstructed from its
samples, if the sampling frequency is at least twice the highest
frequency component in the signal.''


{\def\LTcaptype{none} % do not increment counter
\vspace{-5pt}
\captionof{table}{Key Elements of Sampling Theorem}
\vspace{-10pt}
\begin{longtable}[]{@{}
  >{\raggedright\arraybackslash}p{(\linewidth - 2\tabcolsep) * \real{0.3158}}
  >{\raggedright\arraybackslash}p{(\linewidth - 2\tabcolsep) * \real{0.6842}}@{}}
\toprule\noalign{}
\begin{minipage}[b]{\linewidth}\raggedright
Term
\end{minipage} & \begin{minipage}[b]{\linewidth}\raggedright
Description
\end{minipage} \\
\midrule\noalign{}
\endhead
\bottomrule\noalign{}
\endlastfoot
\textbf{Nyquist Rate} & Minimum sampling frequency (fs) required =
2fm \\
\textbf{Nyquist Interval} & Maximum time between samples = 1/(2fm) \\
\textbf{Band-limited Signal} & Signal with finite highest frequency
component \\
\end{longtable}
}

\textbf{Diagram: Proper Sampling}

\begin{lstlisting}
Original Signal
    │   ┌───┐
    │  /     \
    │ /       \
    │/         \
    │           \
    │            \
    └─────────────────▶

Sampled at fs \geq 2fm
    │   *   *
    │  /|\  |\
    │ / | \ | \
    │/  |  \|  \
    │   |   *   *
    │   |       |
    └───*───────*───▶
\end{lstlisting}

\end{solutionbox}
\begin{mnemonicbox}
``2F: Frequency must be sampled at least Twice its
highest Frequency''

\end{mnemonicbox}
\subsection*{Question 3(b) OR [4
marks]}\label{q3b}

\textbf{Explain Concept of Quantization.}

\begin{solutionbox}


{\def\LTcaptype{none} % do not increment counter
\vspace{-5pt}
\captionof{table}{Quantization Concepts}
\vspace{-10pt}
\begin{longtable}[]{@{}
  >{\raggedright\arraybackslash}p{(\linewidth - 2\tabcolsep) * \real{0.3158}}
  >{\raggedright\arraybackslash}p{(\linewidth - 2\tabcolsep) * \real{0.6842}}@{}}
\toprule\noalign{}
\begin{minipage}[b]{\linewidth}\raggedright
Term
\end{minipage} & \begin{minipage}[b]{\linewidth}\raggedright
Description
\end{minipage} \\
\midrule\noalign{}
\endhead
\bottomrule\noalign{}
\endlastfoot
\textbf{Quantization} & Process of converting continuous amplitude
values into discrete levels \\
\textbf{Quantization Levels} & Total number of discrete values used
(usually 2^{n}) \\
\textbf{Quantization Step Size} & Voltage difference between adjacent
levels (Q = Vmax/2^{n}) \\
\textbf{Quantization Error} & Difference between actual signal value and
quantized value \\
\end{longtable}
}

\textbf{Diagram: Quantization Process}

\begin{lstlisting}
Continuous Signal           Quantized Signal
    │                           │       
    │   /\                      │   ┌─┐  
    │  /  \                     │   │ │  
    │ /    \      ───────▶      │┌──┘ └──┐
    │/      \                   ││       │
    │        \                  ││       └──┐
    │         \                 ││          │
    └──────────────▶            └───────────────▶
                               Quantization
                                  Levels
\end{lstlisting}

\end{solutionbox}
\begin{mnemonicbox}
``LSED: Levels, Step size, Error, Discrete values''

\end{mnemonicbox}
\subsection*{Question 3(c) OR [7
marks]}\label{q3c}

\textbf{Explain the Companding in detail.}

\begin{solutionbox}


{\def\LTcaptype{none} % do not increment counter
\vspace{-5pt}
\captionof{table}{Companding Concepts}
\vspace{-10pt}
\begin{longtable}[]{@{}
  >{\raggedright\arraybackslash}p{(\linewidth - 2\tabcolsep) * \real{0.3158}}
  >{\raggedright\arraybackslash}p{(\linewidth - 2\tabcolsep) * \real{0.6842}}@{}}
\toprule\noalign{}
\begin{minipage}[b]{\linewidth}\raggedright
Term
\end{minipage} & \begin{minipage}[b]{\linewidth}\raggedright
Description
\end{minipage} \\
\midrule\noalign{}
\endhead
\bottomrule\noalign{}
\endlastfoot
\textbf{Companding} & COMpressing + exPANDING; non-linear quantization
technique \\
\textbf{Compression} & Reduces amplitude range of signal before
transmission \\
\textbf{Expansion} & Restores original amplitude range at receiver \\
\textbf{Purpose} & Improves SNR for weak signals while maintaining
dynamic range \\
\textbf{Types} & μ-law (North America, Japan), A-law (Europe) \\
\end{longtable}
}

\textbf{Diagram: Companding Process}

\includegraphics[width=1\linewidth,height=\textheight,keepaspectratio]{mermaid-436c8998.pdf}

\textbf{Companding Laws:}

\begin{itemize}
\tightlist
\item
  \textbf{μ-law}: y = sgn(x) \times ln(1+μ\textbar x\textbar)/ln(1+μ) where μ
  = 255 in USA
\item
  \textbf{A-law}: y = sgn(x) \times A\textbar x\textbar/(1+ln(A)) for
  \textbar x\textbar{} \textless{} 1/A y = sgn(x) \times
  (1+ln(A\textbar x\textbar))/(1+ln(A)) for 1/A \leq \textbar x\textbar{} \leq
  1
\end{itemize}

\end{solutionbox}
\begin{mnemonicbox}
``CEQS: Compress, Encode, Quantize, Send; then
Decode, Expand, Recover''

\end{mnemonicbox}
\subsection*{Question 4(a) [3 marks]}\label{q4a}

\textbf{Explain delta modulation}

\begin{solutionbox}


{\def\LTcaptype{none} % do not increment counter
\vspace{-5pt}
\captionof{table}{Delta Modulation Concepts}
\vspace{-10pt}
\begin{longtable}[]{@{}
  >{\raggedright\arraybackslash}p{(\linewidth - 2\tabcolsep) * \real{0.4091}}
  >{\raggedright\arraybackslash}p{(\linewidth - 2\tabcolsep) * \real{0.5909}}@{}}
\toprule\noalign{}
\begin{minipage}[b]{\linewidth}\raggedright
Concept
\end{minipage} & \begin{minipage}[b]{\linewidth}\raggedright
Description
\end{minipage} \\
\midrule\noalign{}
\endhead
\bottomrule\noalign{}
\endlastfoot
\textbf{Delta Modulation} & Simplest form of DPCM where only 1-bit
quantization is used \\
\textbf{Step Size} & Fixed increment/decrement in approximating
signal \\
\textbf{Output} & Binary stream (1 for increase, 0 for decrease) \\
\textbf{Advantages} & Simple implementation, low bandwidth \\
\end{longtable}
}

\textbf{Diagram: Delta Modulation}

\begin{lstlisting}
Original Signal    Delta Modulated
                   Approximation
    │                   │
    │  /\               │    ┌┐┌┐
    │ /  \              │   ┌┘└┘└┐
    │/    \             │  ┌┘    └┐
    │      \            │ ┌┘      └┐
    │       \           │┌┘        └┐
    │        \          ││          │
    └─────────────▶     └───────────────▶
                       
Binary Output: 1 1 1 1 0 0 0 0 0 0
\end{lstlisting}

\end{solutionbox}
\begin{mnemonicbox}
``1B1S: 1-Bit, 1-Step tracking''

\end{mnemonicbox}
\subsection*{Question 4(b) [4 marks]}\label{q4b}

\textbf{List out of advantage and disadvantage of PCM.}

\begin{solutionbox}


{\def\LTcaptype{none} % do not increment counter
\vspace{-5pt}
\captionof{table}{Advantages and Disadvantages of PCM}
\vspace{-10pt}
\begin{longtable}[]{@{}ll@{}}
\toprule\noalign{}
Advantages & Disadvantages \\
\midrule\noalign{}
\endhead
\bottomrule\noalign{}
\endlastfoot
\textbf{High noise immunity} & \textbf{Requires higher bandwidth} \\
\textbf{Better signal quality} & \textbf{Complex system
implementation} \\
\textbf{Compatible with digital systems} & \textbf{Quantization noise
present} \\
\textbf{Secure transmission possible} & \textbf{Synchronization
required} \\
\textbf{Multiplexing capability} & \textbf{Higher power requirement} \\
\end{longtable}
}

\textbf{Diagram: PCM System Overview}

\includegraphics[width=1\linewidth,height=\textheight,keepaspectratio]{mermaid-671ce7f7.pdf}

\end{solutionbox}
\begin{mnemonicbox}
``NCSMP: Noise immunity, Compatible with digital,
Secure, Multiplexing, Processing benefits''

\end{mnemonicbox}
\subsection*{Question 4(c) [7 marks]}\label{q4c}

\textbf{Draw and explain block diagram of PCM-TDM system.}

\begin{solutionbox}

\textbf{Diagram: PCM-TDM System}

\includegraphics[width=1\linewidth,height=\textheight,keepaspectratio]{mermaid-77ee2109.pdf}


{\def\LTcaptype{none} % do not increment counter
\vspace{-5pt}
\captionof{table}{PCM-TDM System Components}
\vspace{-10pt}
\begin{longtable}[]{@{}
  >{\raggedright\arraybackslash}p{(\linewidth - 2\tabcolsep) * \real{0.5238}}
  >{\raggedright\arraybackslash}p{(\linewidth - 2\tabcolsep) * \real{0.4762}}@{}}
\toprule\noalign{}
\begin{minipage}[b]{\linewidth}\raggedright
Component
\end{minipage} & \begin{minipage}[b]{\linewidth}\raggedright
Function
\end{minipage} \\
\midrule\noalign{}
\endhead
\bottomrule\noalign{}
\endlastfoot
\textbf{Anti-aliasing Filter} & Limits signal bandwidth to avoid
aliasing \\
\textbf{Sample \& Hold} & Captures analog value and holds it for
processing \\
\textbf{Multiplexer} & Combines multiple input channels into single time
division multiplexed stream \\
\textbf{Quantizer} & Converts continuous samples to discrete values \\
\textbf{Encoder} & Converts quantized values to binary code \\
\textbf{Frame Generator} & Adds synchronization and control bits \\
\textbf{Demultiplexer} & Separates combined signal back into individual
channels \\
\textbf{Reconstruction Filter} & Smooths the decoded signal to recover
analog waveform \\
\end{longtable}
}

\end{solutionbox}
\begin{mnemonicbox}
``SAMPLER: Sample, Amplify, Multiplex, Process,
Limit, Encode, Reconstruct''

\end{mnemonicbox}
\subsection*{Question 4(a) OR [3
marks]}\label{q4a}

\textbf{Describe slop overload error.}

\begin{solutionbox}


{\def\LTcaptype{none} % do not increment counter
\vspace{-5pt}
\captionof{table}{Slope Overload Error}
\vspace{-10pt}
\begin{longtable}[]{@{}
  >{\raggedright\arraybackslash}p{(\linewidth - 2\tabcolsep) * \real{0.4091}}
  >{\raggedright\arraybackslash}p{(\linewidth - 2\tabcolsep) * \real{0.5909}}@{}}
\toprule\noalign{}
\begin{minipage}[b]{\linewidth}\raggedright
Concept
\end{minipage} & \begin{minipage}[b]{\linewidth}\raggedright
Description
\end{minipage} \\
\midrule\noalign{}
\endhead
\bottomrule\noalign{}
\endlastfoot
\textbf{Slope Overload Error} & Error occurring when input signal
changes faster than DM step size can track \\
\textbf{Cause} & Fixed step size in Delta Modulation too small for steep
input slopes \\
\textbf{Effect} & Distortion in reconstructed signal, particularly at
high frequencies \\
\textbf{Solution} & Adaptive Delta Modulation (variable step size) \\
\end{longtable}
}

\textbf{Diagram: Slope Overload Error}

\begin{lstlisting}
Original Signal vs DM Approximation
                
    │                  Slope Overload
    │                      │
    │    /│\              /│\
    │   / │ \            / │ \
    │  /  │  \    vs    /  │  \
    │ /   │   \        /┌─┐│   \
    │/    │    \      /┌┘ └┤    \
    │     │     \    /┌┘   │     \
    │     │      \  /┌┘    │      \
    └─────┴───────\/┴──────┴───────▶
          Original     DM Approximation
\end{lstlisting}

\end{solutionbox}
\begin{mnemonicbox}
``SOS: Signal Outpaces Steps when slope is steep''

\end{mnemonicbox}
\subsection*{Question 4(b) OR [4
marks]}\label{q4b}

\textbf{Explain transmitter of Differential PCM}

\begin{solutionbox}

\textbf{Diagram: DPCM Transmitter}

\includegraphics[width=1\linewidth,height=\textheight,keepaspectratio]{mermaid-e298a980.pdf}


{\def\LTcaptype{none} % do not increment counter
\vspace{-5pt}
\captionof{table}{DPCM Transmitter Components}
\vspace{-10pt}
\begin{longtable}[]{@{}
  >{\raggedright\arraybackslash}p{(\linewidth - 2\tabcolsep) * \real{0.5238}}
  >{\raggedright\arraybackslash}p{(\linewidth - 2\tabcolsep) * \real{0.4762}}@{}}
\toprule\noalign{}
\begin{minipage}[b]{\linewidth}\raggedright
Component
\end{minipage} & \begin{minipage}[b]{\linewidth}\raggedright
Function
\end{minipage} \\
\midrule\noalign{}
\endhead
\bottomrule\noalign{}
\endlastfoot
\textbf{Sample \& Hold} & Captures analog signal at regular intervals \\
\textbf{Difference Calculator} & Computes error between current sample
and predicted value \\
\textbf{Quantizer} & Converts error signal to discrete levels \\
\textbf{Encoder} & Converts quantized values to binary code \\
\textbf{Predictor} & Estimates next sample based on previous values \\
\textbf{Decoder} & Same as in receiver, used in feedback loop \\
\end{longtable}
}

\textbf{Key Advantage:}

\begin{itemize}
\tightlist
\item
  Transmits only the difference between successive samples
\item
  Reduces bit rate compared to standard PCM
\end{itemize}

\end{solutionbox}
\begin{mnemonicbox}
``SDQEP: Sample, Difference, Quantize, Encode,
Predict''

\end{mnemonicbox}
\subsection*{Question 4(c) OR [7
marks]}\label{q4c}

\textbf{Explain in detail PCM transmitter}

\begin{solutionbox}

\textbf{Diagram: PCM Transmitter}

\includegraphics[width=1\linewidth,height=\textheight,keepaspectratio]{mermaid-aeb60861.pdf}


{\def\LTcaptype{none} % do not increment counter
\vspace{-5pt}
\captionof{table}{PCM Transmitter Components in Detail}
\vspace{-10pt}
\begin{longtable}[]{@{}
  >{\raggedright\arraybackslash}p{(\linewidth - 4\tabcolsep) * \real{0.2444}}
  >{\raggedright\arraybackslash}p{(\linewidth - 4\tabcolsep) * \real{0.2222}}
  >{\raggedright\arraybackslash}p{(\linewidth - 4\tabcolsep) * \real{0.5333}}@{}}
\toprule\noalign{}
\begin{minipage}[b]{\linewidth}\raggedright
Component
\end{minipage} & \begin{minipage}[b]{\linewidth}\raggedright
Function
\end{minipage} & \begin{minipage}[b]{\linewidth}\raggedright
Design Considerations
\end{minipage} \\
\midrule\noalign{}
\endhead
\bottomrule\noalign{}
\endlastfoot
\textbf{Anti-aliasing Filter} & Limits input bandwidth to fs/2 & Cutoff
frequency \textless{} fs/2, sharp roll-off \\
\textbf{Sample \& Hold} & Captures instantaneous signal value & Sampling
rate \geq 2fm, aperture time \textless\textless{} sampling period \\
\textbf{Quantizer} & Approximates sample amplitudes to discrete levels &
Levels = 2^{n} where

n = bit depth, typically 8-16 bits \\

\textbf{Encoder} & Converts quantized values to digital codes & Uses
coding schemes like NRZ, RZ, Manchester \\
\textbf{Line Coder} & Prepares binary sequence for transmission & May
use regenerative repeaters for long distance \\
\end{longtable}
}

\textbf{Signal Processing Details:}

\begin{itemize}
\tightlist
\item
  \textbf{Time Domain}: Sampling at intervals Ts = 1/fs
\item
  \textbf{Amplitude Domain}: Quantizing continuous amplitudes into 2^{n}
  discrete levels
\item
  \textbf{Code Domain}: Converting levels to n-bit binary code
\end{itemize}

\end{solutionbox}
\begin{mnemonicbox}
``SAFE-Q: Sample And Filter, then Encode after
Quantizing''

\end{mnemonicbox}
\subsection*{Question 5(a) [3 marks]}\label{q5a}

\textbf{Compare PCM and DM}

\begin{solutionbox}


{\def\LTcaptype{none} % do not increment counter
\vspace{-5pt}
\captionof{table}{Comparison of PCM and DM}
\vspace{-10pt}
\begin{longtable}[]{@{}
  >{\raggedright\arraybackslash}p{(\linewidth - 4\tabcolsep) * \real{0.5238}}
  >{\raggedright\arraybackslash}p{(\linewidth - 4\tabcolsep) * \real{0.2381}}
  >{\raggedright\arraybackslash}p{(\linewidth - 4\tabcolsep) * \real{0.2381}}@{}}
\toprule\noalign{}
\begin{minipage}[b]{\linewidth}\raggedright
Parameter
\end{minipage} & \begin{minipage}[b]{\linewidth}\raggedright
PCM
\end{minipage} & \begin{minipage}[b]{\linewidth}\raggedright
DM
\end{minipage} \\
\midrule\noalign{}
\endhead
\bottomrule\noalign{}
\endlastfoot
\textbf{Bit Rate} & Higher (multiple bits per sample) & Lower (1 bit per
sample) \\
\textbf{Circuit Complexity} & More complex & Simpler \\
\textbf{Signal Quality} & Better & Lower, suffers from slope overload \&
granular noise \\
\textbf{Bandwidth} & Wider & Narrower \\
\textbf{Sampling Rate} & At least 2fm & Much higher than 2fm \\
\end{longtable}
}

\end{solutionbox}
\begin{mnemonicbox}
``BCSBS: Bit rate, Complexity, Signal quality,
Bandwidth, Sampling''

\end{mnemonicbox}
\subsection*{Question 5(b) [4 marks]}\label{q5b}

\textbf{Define: (A) Antenna (B) Radiation pattern (C) Directivity and
(D) Polarization}

\begin{solutionbox}


{\def\LTcaptype{none} % do not increment counter
\vspace{-5pt}
\captionof{table}{Antenna Terminology}
\vspace{-10pt}
\begin{longtable}[]{@{}
  >{\raggedright\arraybackslash}p{(\linewidth - 2\tabcolsep) * \real{0.3333}}
  >{\raggedright\arraybackslash}p{(\linewidth - 2\tabcolsep) * \real{0.6667}}@{}}
\toprule\noalign{}
\begin{minipage}[b]{\linewidth}\raggedright
Term
\end{minipage} & \begin{minipage}[b]{\linewidth}\raggedright
Definition
\end{minipage} \\
\midrule\noalign{}
\endhead
\bottomrule\noalign{}
\endlastfoot
\textbf{Antenna} & Device that converts electrical signals into
electromagnetic waves and vice versa \\
\textbf{Radiation Pattern} & Graphical representation of radiation
properties of antenna as function of space coordinates \\
\textbf{Directivity} & Ratio of radiation intensity in a given direction
to average radiation intensity \\
\textbf{Polarization} & Orientation of electric field vector of
electromagnetic wave radiated by antenna \\
\end{longtable}
}

\textbf{Diagram: Radiation Pattern}

\begin{lstlisting}
      │
      │          ┌───┐
      │        ╱       ╲
      │      ╱           ╲
      │    ╱               ╲
      │  ╱                   ╲
      │╱                       ╲
 ─────┼─────────────────────────────▶
      │╲                       ╱
      │  ╲                   ╱
      │    ╲               ╱
      │      ╲           ╱
      │        ╲       ╱
      │          └───┘
      │
\end{lstlisting}

\end{solutionbox}
\begin{mnemonicbox}
``ARDP: Antennas Radiate with Directivity and
Polarization''

\end{mnemonicbox}
\subsection*{Question 5(c) [7 marks]}\label{q5c}

\textbf{Write brief note on (A) smart antenna (B) parabolic reflector
antenna}

\begin{solutionbox}

\end{solutionbox}
\subsubsection{(A) Smart Antenna}\label{a-smart-antenna}


{\def\LTcaptype{none} % do not increment counter
\vspace{-5pt}
\captionof{table}{Smart Antenna Characteristics}
\vspace{-10pt}
\begin{longtable}[]{@{}
  >{\raggedright\arraybackslash}p{(\linewidth - 2\tabcolsep) * \real{0.4091}}
  >{\raggedright\arraybackslash}p{(\linewidth - 2\tabcolsep) * \real{0.5909}}@{}}
\toprule\noalign{}
\begin{minipage}[b]{\linewidth}\raggedright
Feature
\end{minipage} & \begin{minipage}[b]{\linewidth}\raggedright
Description
\end{minipage} \\
\midrule\noalign{}
\endhead
\bottomrule\noalign{}
\endlastfoot
\textbf{Definition} & Antenna array with signal processing capability to
adapt to changing conditions \\
\textbf{Types} & Switched beam, Adaptive array \\
\textbf{Benefits} & Increased range/coverage, interference reduction,
capacity improvement \\
\textbf{Applications} & Mobile communications, 5G networks, WiMAX,
military systems \\
\end{longtable}
}

\textbf{Diagram: Smart Antenna System}

\includegraphics[width=1\linewidth,height=\textheight,keepaspectratio]{mermaid-037b040c.pdf}

\subsubsection{(B) Parabolic Reflector
Antenna}\label{b-parabolic-reflector-antenna}


{\def\LTcaptype{none} % do not increment counter
\vspace{-5pt}
\captionof{table}{Parabolic Reflector Characteristics}
\vspace{-10pt}
\begin{longtable}[]{@{}
  >{\raggedright\arraybackslash}p{(\linewidth - 2\tabcolsep) * \real{0.4091}}
  >{\raggedright\arraybackslash}p{(\linewidth - 2\tabcolsep) * \real{0.5909}}@{}}
\toprule\noalign{}
\begin{minipage}[b]{\linewidth}\raggedright
Feature
\end{minipage} & \begin{minipage}[b]{\linewidth}\raggedright
Description
\end{minipage} \\
\midrule\noalign{}
\endhead
\bottomrule\noalign{}
\endlastfoot
\textbf{Structure} & Feed antenna at focal point with parabolic
reflecting surface \\
\textbf{Operation} & Focuses parallel incoming waves to focal point or
radiates from focal point into parallel beams \\
\textbf{Gain} & Very high directivity and gain \\
\textbf{Applications} & Satellite communication, radio astronomy, radar
systems \\
\end{longtable}
}

\textbf{Diagram: Parabolic Reflector}

\begin{lstlisting}
                 ╱│╲
             ╱    │    ╲
         ╱        │        ╲
     ╱            │            ╲
 ╱                │                ╲
 ╲                │                ╱
     ╲            │            ╱
         ╲        │        ╱
             ╲    │    ╱
                 ╲│╱
                  X
                  │
                  │
                  ▼
                Receiver
             (at focal point)
\end{lstlisting}

\begin{mnemonicbox}
``PFHS: Parabolic Focus gives High Signal strength''

\end{mnemonicbox}
\subsection*{Question 5(a) OR [3
marks]}\label{q5a}

\textbf{Write a short note on Microstrip antenna}

\begin{solutionbox}


{\def\LTcaptype{none} % do not increment counter
\vspace{-5pt}
\captionof{table}{Microstrip Antenna Characteristics}
\vspace{-10pt}
\begin{longtable}[]{@{}
  >{\raggedright\arraybackslash}p{(\linewidth - 2\tabcolsep) * \real{0.4091}}
  >{\raggedright\arraybackslash}p{(\linewidth - 2\tabcolsep) * \real{0.5909}}@{}}
\toprule\noalign{}
\begin{minipage}[b]{\linewidth}\raggedright
Feature
\end{minipage} & \begin{minipage}[b]{\linewidth}\raggedright
Description
\end{minipage} \\
\midrule\noalign{}
\endhead
\bottomrule\noalign{}
\endlastfoot
\textbf{Structure} & Conductive patch on dielectric substrate with
ground plane \\
\textbf{Shape} & Rectangular, circular, elliptical, triangular
patches \\
\textbf{Size} & Typically λ/2 in length, very thin (h
\textless\textless{} λ) \\
\textbf{Advantages} & Low profile, lightweight, low cost, easy
fabrication, compatible with PCB technology \\
\textbf{Disadvantages} & Low efficiency, narrow bandwidth, low power
handling \\
\end{longtable}
}

\textbf{Diagram: Microstrip Patch Antenna}

\begin{lstlisting}
    ┌─────────────────────┐  \leftarrow── Patch (Copper)
    │                     │
    │                     │
    │                     │
    └─────────────────────┘
    ┌─────────────────────┐  \leftarrow── Dielectric Substrate
    │                     │      (FR4, PTFE, etc.)
    └─────────────────────┘
    ┌─────────────────────┐  \leftarrow── Ground Plane (Copper)
    └─────────────────────┘
\end{lstlisting}

\end{solutionbox}
\begin{mnemonicbox}
``PDGF: Patch on Dielectric with Ground plane gives
Flat profile''

\end{mnemonicbox}
\subsection*{Question 5(b) OR [4
marks]}\label{q5b}

\textbf{Explain EM wave spectrum, its Frequency ranges and its
applications.}

\begin{solutionbox}


{\def\LTcaptype{none} % do not increment counter
\vspace{-5pt}
\captionof{table}{EM Wave Spectrum and Applications}
\vspace{-10pt}
\begin{longtable}[]{@{}
  >{\raggedright\arraybackslash}p{(\linewidth - 6\tabcolsep) * \real{0.1224}}
  >{\raggedright\arraybackslash}p{(\linewidth - 6\tabcolsep) * \real{0.3469}}
  >{\raggedright\arraybackslash}p{(\linewidth - 6\tabcolsep) * \real{0.2449}}
  >{\raggedright\arraybackslash}p{(\linewidth - 6\tabcolsep) * \real{0.2857}}@{}}
\toprule\noalign{}
\begin{minipage}[b]{\linewidth}\raggedright
Band
\end{minipage} & \begin{minipage}[b]{\linewidth}\raggedright
Frequency Range
\end{minipage} & \begin{minipage}[b]{\linewidth}\raggedright
Wavelength
\end{minipage} & \begin{minipage}[b]{\linewidth}\raggedright
Applications
\end{minipage} \\
\midrule\noalign{}
\endhead
\bottomrule\noalign{}
\endlastfoot
\textbf{ELF} & 3 Hz - 30 Hz & 10,000 - 100,000 km & Submarine
communication \\
\textbf{VLF} & 3 kHz - 30 kHz & 10 - 100 km & Navigation, time
signals \\
\textbf{LF} & 30 kHz - 300 kHz & 1 - 10 km & AM radio, maritime radio \\
\textbf{MF} & 300 kHz - 3 MHz & 100 m - 1 km & AM broadcasting \\
\textbf{HF} & 3 MHz - 30 MHz & 10 - 100 m & Shortwave radio, amateur
radio \\
\textbf{VHF} & 30 MHz - 300 MHz & 1 - 10 m & FM radio, TV
broadcasting \\
\textbf{UHF} & 300 MHz - 3 GHz & 10 cm - 1 m & TV, mobile phones,
WiFi \\
\textbf{SHF} & 3 GHz - 30 GHz & 1 - 10 cm & Satellite, radar, 5G \\
\textbf{EHF} & 30 GHz - 300 GHz & 1 mm - 1 cm & Radio astronomy,
security scanning \\
\textbf{IR} & 300 GHz - 400 THz & 750 nm - 1 mm & Thermal imaging,
remote control \\
\textbf{Visible} & 400 THz - 800 THz & 380 - 750 nm & Optical
communications \\
\end{longtable}
}

\textbf{Diagram: EM Wave Spectrum}

\includegraphics[width=1\linewidth,height=\textheight,keepaspectratio]{mermaid-97aa290c.pdf}

\end{solutionbox}
\begin{mnemonicbox}
``RVMIXG: Radio, Visible, Microwave, Infrared, X-ray,
Gamma''

\end{mnemonicbox}
\subsection*{Question 5(c) OR [7
marks]}\label{q5c}

\textbf{Write brief note on (A) Space Wave Propagation (B) Ground Wave
Propagation.}

\begin{solutionbox}

\end{solutionbox}
\subsubsection{(A) Space Wave
Propagation}\label{a-space-wave-propagation}


{\def\LTcaptype{none} % do not increment counter
\vspace{-5pt}
\captionof{table}{Space Wave Propagation Characteristics}
\vspace{-10pt}
\begin{longtable}[]{@{}
  >{\raggedright\arraybackslash}p{(\linewidth - 2\tabcolsep) * \real{0.4091}}
  >{\raggedright\arraybackslash}p{(\linewidth - 2\tabcolsep) * \real{0.5909}}@{}}
\toprule\noalign{}
\begin{minipage}[b]{\linewidth}\raggedright
Feature
\end{minipage} & \begin{minipage}[b]{\linewidth}\raggedright
Description
\end{minipage} \\
\midrule\noalign{}
\endhead
\bottomrule\noalign{}
\endlastfoot
\textbf{Definition} & Direct wave propagation through space, including
line-of-sight and reflected waves \\
\textbf{Frequency Range} & VHF and above (\textgreater30 MHz) \\
\textbf{Distance} & Limited by horizon, typically 50-80 km \\
\textbf{Types} & Direct wave, Ground reflected wave, Tropospheric
scatter, Duct propagation \\
\textbf{Applications} & TV broadcasting, microwave links, satellite
communication \\
\end{longtable}
}

\textbf{Diagram: Space Wave Propagation}

\begin{lstlisting}
                    /\/\/\/\/\/\/\/\  \leftarrow Troposphere
                   /                \
                  /                  \
                 /                    \
    Transmitter *                       * Receiver
                |                       |
                |                       |
     ___________|_______________________|__________
                     Ground Surface
\end{lstlisting}

\subsubsection{(B) Ground Wave
Propagation}\label{b-ground-wave-propagation}


{\def\LTcaptype{none} % do not increment counter
\vspace{-5pt}
\captionof{table}{Ground Wave Characteristics}
\vspace{-10pt}
\begin{longtable}[]{@{}
  >{\raggedright\arraybackslash}p{(\linewidth - 2\tabcolsep) * \real{0.4091}}
  >{\raggedright\arraybackslash}p{(\linewidth - 2\tabcolsep) * \real{0.5909}}@{}}
\toprule\noalign{}
\begin{minipage}[b]{\linewidth}\raggedright
Feature
\end{minipage} & \begin{minipage}[b]{\linewidth}\raggedright
Description
\end{minipage} \\
\midrule\noalign{}
\endhead
\bottomrule\noalign{}
\endlastfoot
\textbf{Definition} & Wave propagation along Earth's surface, follows
curvature of Earth \\
\textbf{Frequency Range} & LF, MF (up to 2 MHz) \\
\textbf{Distance} & Up to 1000 km depending on frequency and power \\
\textbf{Mechanism} & Vertically polarized wave attaches to conductive
Earth surface \\
\textbf{Applications} & AM radio broadcasting, maritime communication \\
\end{longtable}
}

\textbf{Diagram: Ground Wave Propagation}

\begin{lstlisting}
    Transmitter                           Receiver
         *                                   *
         |                                   |
         |       Ground Wave                 |
     ____|___________________________________|____
         \\\\\\\\\\\\\\\\\\\\\\\\\\\\\\\\\\\\\\\\
                       Earth
\end{lstlisting}

\begin{mnemonicbox}
``SHGM: Space waves go High, Ground waves hug Medium
surface''

\end{mnemonicbox}

\end{document}
