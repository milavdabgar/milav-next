\documentclass[10pt,a4paper]{article}

% content/resources/templates/preamble.tex
\usepackage[margin=0.6in]{geometry}
\author{Milav Dabgar}
\usepackage{amsmath,amssymb,amsthm}
\usepackage{booktabs}
\usepackage{multirow}
\usepackage{xcolor}
\usepackage{tcolorbox}
\tcbuselibrary{breakable,skins}
\usepackage[colorlinks=true,linkcolor=blue]{hyperref}
\usepackage{titlesec}
\usepackage{enumitem}
\usepackage{tikz}
\usepackage{pgfplots}
\usepackage{circuitikz}
\usepackage[version=4]{mhchem}
\usepackage{longtable}
\usepackage{array}
\usepackage{float}
\usepackage{caption}
\usepackage{listings}

\lstset{
  basicstyle=\small\ttfamily,
  breaklines=true,
  breakatwhitespace=false,
  postbreak=\mbox{\textcolor{red}{$\hookrightarrow$}\space},
  float=false,
  numbers=left,
  numberstyle=\tiny\color{gray},
  numbersep=10pt,
  xleftmargin=2em,
  keywordstyle=\color{blue},
  commentstyle=\color{green!60!black},
  stringstyle=\color{purple},
  backgroundcolor=\color{gray!5},
  showstringspaces=false,
  tabsize=2,
  captionpos=b,
  keepspaces=true,
  columns=flexible
}

\pgfplotsset{compat=1.18}
\usetikzlibrary{shapes,arrows,positioning,calc,patterns,decorations.pathmorphing,decorations.markings,arrows.meta}

% Color scheme
\definecolor{headcolor}{RGB}{0,102,204}
\definecolor{keycolor}{RGB}{220,20,60}
\definecolor{solutioncolor}{RGB}{34,139,34}
\definecolor{mnemoniccolor}{RGB}{148,0,211}
\definecolor{codecolor}{RGB}{0,0,100}

% Spacing
\setlength{\parskip}{3pt}
\setlist[itemize]{nosep}
\setlist[enumerate]{nosep}

% Title formatting
\titleformat{\section}{\Large\bfseries\color{headcolor}}{\thesection}{1em}{}
\titleformat{\subsection}{\large\bfseries\color{headcolor}}{\thesubsection}{1em}{}

% Pandoc tightlist compatibility
\providecommand{\tightlist}{%
  \setlength{\itemsep}{0pt}\setlength{\parskip}{0pt}}

% Pandoc longtable compatibility
\newcounter{none}
\def\thenone{}


% content/resources/templates/english-boxes.tex
% This file is currently empty - it exists to maintain consistency with the import structure.
% Add custom environments here if needed in the future.


\begin{document}

\begin{center}
{\Huge\bfseries\color{headcolor} Subject Name Solutions}\\[5pt]
{\LARGE 1333201 -- Summer 2024}\\[3pt]
{\large Semester 1 Study Material}\\[3pt]
{\normalsize\textit{Detailed Solutions and Explanations}}
\end{center}

\vspace{10pt}

\subsection*{Question 1(a) [3 marks]}\label{q1a}

\textbf{Define modulation and explain its need.}

\begin{solutionbox}
Modulation is the process of varying one or more
properties of a high-frequency carrier signal with a modulating signal
containing information.


{\def\LTcaptype{none} % do not increment counter
\vspace{-5pt}
\captionof{table}{Need for Modulation}
\vspace{-10pt}
\begin{longtable}[]{@{}
  >{\raggedright\arraybackslash}p{(\linewidth - 2\tabcolsep) * \real{0.3158}}
  >{\raggedright\arraybackslash}p{(\linewidth - 2\tabcolsep) * \real{0.6842}}@{}}
\toprule\noalign{}
\begin{minipage}[b]{\linewidth}\raggedright
Need
\end{minipage} & \begin{minipage}[b]{\linewidth}\raggedright
Explanation
\end{minipage} \\
\midrule\noalign{}
\endhead
\bottomrule\noalign{}
\endlastfoot
\textbf{Antenna Size Reduction} & Allows practical antenna size (λ/4) by
increasing frequency \\
\textbf{Signal Propagation} & Higher frequencies travel farther through
atmosphere \\
\textbf{Multiplexing} & Allows multiple signals to be transmitted
simultaneously \\
\textbf{Interference Reduction} & Shifts signal to band with less
noise/interference \\
\textbf{Bandwidth Allocation} & Enables efficient spectrum usage by
different services \\
\end{longtable}
}

\end{solutionbox}
\begin{mnemonicbox}
``ASPIM'' - Antenna size, Signal propagation, Proper
multiplexing, Interference reduction, Manage bandwidth

\end{mnemonicbox}
\subsection*{Question 1(b) [4 marks]}\label{q1b}

\textbf{Draw \& explain block diagram of Communication system}

\begin{solutionbox}
A communication system transfers information from
source to destination through a channel.

\begin{figure}
\centering
\pandocbounded{\includesvg[keepaspectratio]{diagrams/1333201-s2024-q1b.svg}}
\caption{Communication System Block Diagram}
\end{figure}

\includegraphics[width=1\linewidth,height=\textheight,keepaspectratio]{mermaid-1ed3b125.pdf}


{\def\LTcaptype{none} % do not increment counter
\vspace{-5pt}
\captionof{table}{Communication System Components}
\vspace{-10pt}
\begin{longtable}[]{@{}
  >{\raggedright\arraybackslash}p{(\linewidth - 2\tabcolsep) * \real{0.5238}}
  >{\raggedright\arraybackslash}p{(\linewidth - 2\tabcolsep) * \real{0.4762}}@{}}
\toprule\noalign{}
\begin{minipage}[b]{\linewidth}\raggedright
Component
\end{minipage} & \begin{minipage}[b]{\linewidth}\raggedright
Function
\end{minipage} \\
\midrule\noalign{}
\endhead
\bottomrule\noalign{}
\endlastfoot
\textbf{Information Source} & Produces message to be transmitted (voice,
video, data) \\
\textbf{Transmitter} & Converts message to suitable signals (modulation,
coding) \\
\textbf{Channel} & Medium through which signals travel (wire, fiber,
air) \\
\textbf{Noise Source} & Unwanted signals that corrupt the transmitted
signal \\
\textbf{Receiver} & Extracts original message from received signal
(demodulation) \\
\textbf{Destination} & Where the message is delivered (human,
machine) \\
\end{longtable}
}

\end{solutionbox}
\begin{mnemonicbox}
``I Try Communicating Neatly, Receive Data''
(I-T-C-N-R-D)

\end{mnemonicbox}
\subsection*{Question 1(c) [7 marks]}\label{q1c}

\textbf{Derive voltage equation for Amplitude modulation.}

\begin{solutionbox}
Amplitude modulation varies the amplitude of carrier
signal proportionally to the message signal.

\textbf{Mathematical Derivation:}

\begin{itemize}
\tightlist
\item
  Let carrier signal be: c(t) = Ac cos(ωct)
\item
  Message signal: m(t) = Am cos(ωmt)
\item
  AM signal: s(t) = Ac[1 + μ·m(t)/Am]cos(ωct)
\item
Where

μ = modulation index = Am/Ac

\item
  Substituting m(t): s(t) = Ac[1 + μ·cos(ωmt)]cos(ωct)
\item
  Expanding: s(t) = Ac·cos(ωct) + μ·Ac·cos(ωmt)·cos(ωct)
\item
  Using identity (cos A·cos B): s(t) = Ac·cos(ωct) +
  (μ·Ac/2)[cos(ωc+ωm)t + cos(ωc-ωm)t]
\end{itemize}

\textbf{Diagram: AM Signal in Time Domain}

\begin{figure}
\centering
\pandocbounded{\includesvg[keepaspectratio]{diagrams/1333201-s2024-q1c.svg}}
\caption{AM Signal in Time Domain}
\end{figure}

\includegraphics[width=1\linewidth,height=\textheight,keepaspectratio]{mermaid-eaeba8db.pdf}

\end{solutionbox}
\begin{mnemonicbox}
``CAMDS'' - Carrier Amplitude Modulated by Data
Signal

\end{mnemonicbox}
\subsection*{Question 1(c) OR [7
marks]}\label{q1c}

\textbf{Derive the equation for total power in AM, calculate percentage
of power savings in DSB and SSB.}

\begin{solutionbox}
For an AM signal with modulation index μ, the total
power consists of carrier power and sideband power.


{\def\LTcaptype{none} % do not increment counter
\vspace{-5pt}
\captionof{table}{Power Distribution in AM}
\vspace{-10pt}
\begin{longtable}[]{@{}lll@{}}
\toprule\noalign{}
Component & Power Formula & Percentage of Total Power \\
\midrule\noalign{}
\endhead
\bottomrule\noalign{}
\endlastfoot
Carrier & Pc = Ac^{2}/2 & 1/(1+μ^{2}/2) \times 100\% \\
Upper Sideband & PUSB = Pc·μ^{2}/4 & (μ^{2}/4)/(1+μ^{2}/2) \times 100\% \\
Lower Sideband & PLSB = Pc·μ^{2}/4 & (μ^{2}/4)/(1+μ^{2}/2) \times 100\% \\
Total & PT = Pc(1+μ^{2}/2) & 100\% \\
\end{longtable}
}

\textbf{Power Savings Calculation:}

\begin{itemize}
\tightlist
\item
  In DSB-SC: 100\% carrier suppression = (Pc/PT)\times100\% =
  1/(1+μ^{2}/2)\times100\%

  \begin{itemize}
  \tightlist
  \item
For

μ = 1: Saving = 2/3\times100\% = 66.67\%

  \end{itemize}
\item
  In SSB: One sideband + carrier suppression = (Pc+PLSB)/PT\times100\% =
  (1+μ^{2}/4)/(1+μ^{2}/2)\times100\%

  \begin{itemize}
  \tightlist
  \item
For

μ = 1: Saving = 5/6\times100\% = 83.33\%

  \end{itemize}
\end{itemize}

\end{solutionbox}
\begin{mnemonicbox}
``CAPS'' - Carrier And Power in Sidebands

\end{mnemonicbox}
\subsection*{Question 2(a) [3 marks]}\label{q2a}

\textbf{Define Image frequency in a radio receiver and explain it with
suitable example.}

\begin{solutionbox}
Image frequency is an unwanted frequency that can
produce the same IF (Intermediate Frequency) as the desired signal in a
superheterodyne receiver.


{\def\LTcaptype{none} % do not increment counter
\vspace{-5pt}
\captionof{table}{Image Frequency}
\vspace{-10pt}
\begin{longtable}[]{@{}lll@{}}
\toprule\noalign{}
Parameter & Formula & Example \\
\midrule\noalign{}
\endhead
\bottomrule\noalign{}
\endlastfoot
\textbf{Desired Signal} & fs & 100 MHz \\
\textbf{Local Oscillator} & fLO & 110 MHz \\
\textbf{IF} & fIF = fLO - fs & 10 MHz \\
\textbf{Image Frequency} & fimage = fLO + fIF & 120 MHz \\
\end{longtable}
}

If both 100 MHz and 120 MHz signals exist, both will produce 10 MHz IF,
causing interference.

\end{solutionbox}
\begin{mnemonicbox}
``LIDS'' - Local oscillator plus/minus IF gives
Desired signal and Signal image

\end{mnemonicbox}
\subsection*{Question 2(b) [4 marks]}\label{q2b}

\textbf{Draw and explain block diagram for envelope detector.}

\begin{solutionbox}
Envelope detector extracts the modulating signal from
AM wave by following the envelope.

\begin{figure}
\centering
\pandocbounded{\includesvg[keepaspectratio]{diagrams/1333201-s2024-q2a.svg}}
\caption{Envelope Detector}
\end{figure}

\includegraphics[width=1\linewidth,height=\textheight,keepaspectratio]{mermaid-59e5a063.pdf}


{\def\LTcaptype{none} % do not increment counter
\vspace{-5pt}
\captionof{table}{Envelope Detector Components}
\vspace{-10pt}
\begin{longtable}[]{@{}
  >{\raggedright\arraybackslash}p{(\linewidth - 2\tabcolsep) * \real{0.5238}}
  >{\raggedright\arraybackslash}p{(\linewidth - 2\tabcolsep) * \real{0.4762}}@{}}
\toprule\noalign{}
\begin{minipage}[b]{\linewidth}\raggedright
Component
\end{minipage} & \begin{minipage}[b]{\linewidth}\raggedright
Function
\end{minipage} \\
\midrule\noalign{}
\endhead
\bottomrule\noalign{}
\endlastfoot
\textbf{Diode} & Rectifies the AM signal (passes positive half) \\
\textbf{Capacitor} & Charges to peak value of rectified signal \\
\textbf{Resistor} & Discharges capacitor with time constant RC \\
\textbf{RC Value} & 1/ωm \textless{} RC \textless{} 1/ωc (where ωm is
message frequency, ωc is carrier) \\
\end{longtable}
}

\end{solutionbox}
\begin{mnemonicbox}
``DRCT'' - Diode Rectifies, Capacitor Tracks

\end{mnemonicbox}
\subsection*{Question 2(c) [7 marks]}\label{q2c}

\textbf{Draw block diagram of AM radio receiver and explain working of
each block.}

\begin{solutionbox}
AM receiver converts radio signal to audio output.

\includegraphics[width=1\linewidth,height=\textheight,keepaspectratio]{mermaid-282ccfdf.pdf}


{\def\LTcaptype{none} % do not increment counter
\vspace{-5pt}
\captionof{table}{AM Receiver Blocks}
\vspace{-10pt}
\begin{longtable}[]{@{}
  >{\raggedright\arraybackslash}p{(\linewidth - 2\tabcolsep) * \real{0.4118}}
  >{\raggedright\arraybackslash}p{(\linewidth - 2\tabcolsep) * \real{0.5882}}@{}}
\toprule\noalign{}
\begin{minipage}[b]{\linewidth}\raggedright
Block
\end{minipage} & \begin{minipage}[b]{\linewidth}\raggedright
Function
\end{minipage} \\
\midrule\noalign{}
\endhead
\bottomrule\noalign{}
\endlastfoot
\textbf{Antenna} & Captures electromagnetic signals from air \\
\textbf{RF Amplifier} & Amplifies weak RF signals, provides
selectivity \\
\textbf{Local Oscillator} & Generates frequency to mix with incoming
signal \\
\textbf{Mixer} & Combines RF and oscillator signals to produce IF \\
\textbf{IF Amplifier} & Amplifies fixed IF signal with high gain \\
\textbf{Detector} & Extracts audio signal from AM carrier \\
\textbf{AF Amplifier} & Boosts audio signal power to drive speaker \\
\textbf{Speaker} & Converts electrical signal to sound \\
\end{longtable}
}

\end{solutionbox}
\begin{mnemonicbox}
``ARMLIDAS'' - Antenna Receives, Mixer Links Input
and Detector, Audio to Speaker

\end{mnemonicbox}
\subsection*{Question 2(a) OR [3
marks]}\label{q2a}

\textbf{Define any FOUR characteristics of radio receiver.}

\begin{solutionbox}


{\def\LTcaptype{none} % do not increment counter
\vspace{-5pt}
\captionof{table}{Radio Receiver Characteristics}
\vspace{-10pt}
\begin{longtable}[]{@{}
  >{\raggedright\arraybackslash}p{(\linewidth - 2\tabcolsep) * \real{0.5714}}
  >{\raggedright\arraybackslash}p{(\linewidth - 2\tabcolsep) * \real{0.4286}}@{}}
\toprule\noalign{}
\begin{minipage}[b]{\linewidth}\raggedright
Characteristic
\end{minipage} & \begin{minipage}[b]{\linewidth}\raggedright
Definition
\end{minipage} \\
\midrule\noalign{}
\endhead
\bottomrule\noalign{}
\endlastfoot
\textbf{Sensitivity} & Minimum signal strength that produces standard
output \\
\textbf{Selectivity} & Ability to separate desired signal from adjacent
channels \\
\textbf{Fidelity} & Accuracy of reproducing original modulating
signal \\
\textbf{Image Rejection} & Ability to reject image frequency signals \\
\textbf{Signal-to-Noise Ratio} & Ratio of desired signal power to noise
power \\
\end{longtable}
}

\end{solutionbox}
\begin{mnemonicbox}
``SSFIS'' - Super Sensitive Fidelity with Image
Suppression

\end{mnemonicbox}
\subsection*{Question 2(b) OR [4
marks]}\label{q2b}

\textbf{Explain Ratio detector circuit for FM detection.}

\begin{solutionbox}
Ratio detector extracts audio from FM signals while
rejecting amplitude variations.

\begin{figure}
\centering
\pandocbounded{\includesvg[keepaspectratio]{diagrams/1333201-s2024-q2b.svg}}
\caption{Ratio Detector}
\end{figure}

\includegraphics[width=1\linewidth,height=\textheight,keepaspectratio]{mermaid-ec13ad4d.pdf}


{\def\LTcaptype{none} % do not increment counter
\vspace{-5pt}
\captionof{table}{Ratio Detector Components}
\vspace{-10pt}
\begin{longtable}[]{@{}
  >{\raggedright\arraybackslash}p{(\linewidth - 2\tabcolsep) * \real{0.5238}}
  >{\raggedright\arraybackslash}p{(\linewidth - 2\tabcolsep) * \real{0.4762}}@{}}
\toprule\noalign{}
\begin{minipage}[b]{\linewidth}\raggedright
Component
\end{minipage} & \begin{minipage}[b]{\linewidth}\raggedright
Function
\end{minipage} \\
\midrule\noalign{}
\endhead
\bottomrule\noalign{}
\endlastfoot
\textbf{Transformer} & Creates phase shifts proportional to frequency
deviation \\
\textbf{Diodes} & Arranged in opposite polarity to produce voltage
ratio \\
\textbf{Stabilizing Capacitor} & Large value (10μF) to suppress AM
variations \\
\textbf{RC Network} & Extracts the audio signal from ratio of
voltages \\
\end{longtable}
}

\end{solutionbox}
\begin{mnemonicbox}
``RADS'' - Ratio detector Avoids Disturbance from
Strength variations

\end{mnemonicbox}
\subsection*{Question 2(c) OR [7
marks]}\label{q2c}

\textbf{Draw and explain block diagram of super heterodyne receiver.}

\begin{solutionbox}
Superheterodyne receiver converts all incoming RF to
fixed IF for better amplification.

\includegraphics[width=1\linewidth,height=\textheight,keepaspectratio]{mermaid-5ae0c795.pdf}


{\def\LTcaptype{none} % do not increment counter
\vspace{-5pt}
\captionof{table}{Superheterodyne Receiver Components}
\vspace{-10pt}
\begin{longtable}[]{@{}
  >{\raggedright\arraybackslash}p{(\linewidth - 2\tabcolsep) * \real{0.4118}}
  >{\raggedright\arraybackslash}p{(\linewidth - 2\tabcolsep) * \real{0.5882}}@{}}
\toprule\noalign{}
\begin{minipage}[b]{\linewidth}\raggedright
Block
\end{minipage} & \begin{minipage}[b]{\linewidth}\raggedright
Function
\end{minipage} \\
\midrule\noalign{}
\endhead
\bottomrule\noalign{}
\endlastfoot
\textbf{Antenna} & Captures RF signals \\
\textbf{RF Amplifier} & Amplifies and selects desired frequency band \\
\textbf{Local Oscillator} & Generates frequency above/below signal by IF
value \\
\textbf{Mixer} & Heterodynes signal and oscillator to produce IF \\
\textbf{IF Amplifier} & Provides most gain and selectivity at fixed
frequency \\
\textbf{Detector} & Recovers original modulating signal \\
\textbf{AGC} & Automatic Gain Control - maintains constant output
level \\
\textbf{AF Amplifier} & Amplifies audio to drive speaker \\
\textbf{Speaker} & Converts electrical signal to sound \\
\end{longtable}
}

\end{solutionbox}
\begin{mnemonicbox}
``ARMLIADS'' - Antenna Receives, Mixer Links,
Intermediate Amplifies, Detector Separates

\end{mnemonicbox}
\subsection*{Question 3(a) [3 marks]}\label{q3a}

\textbf{Draw the Time and frequency domain representation of the below
signals. 1. Analog signal (sine) 2. Digital signal (square).}

\begin{solutionbox}


{\def\LTcaptype{none} % do not increment counter
\vspace{-5pt}
\captionof{table}{Signal Representations}
\vspace{-10pt}
\begin{longtable}[]{@{}
  >{\raggedright\arraybackslash}p{(\linewidth - 4\tabcolsep) * \real{0.2955}}
  >{\raggedright\arraybackslash}p{(\linewidth - 4\tabcolsep) * \real{0.2955}}
  >{\raggedright\arraybackslash}p{(\linewidth - 4\tabcolsep) * \real{0.4091}}@{}}
\toprule\noalign{}
\begin{minipage}[b]{\linewidth}\raggedright
Signal Type
\end{minipage} & \begin{minipage}[b]{\linewidth}\raggedright
Time Domain
\end{minipage} & \begin{minipage}[b]{\linewidth}\raggedright
Frequency Domain
\end{minipage} \\
\midrule\noalign{}
\endhead
\bottomrule\noalign{}
\endlastfoot
\textbf{Sine Wave} & Sinusoidal curve & Single spike at frequency f \\
\textbf{Square Wave} & Alternating levels & Fundamental and odd
harmonics (1/n pattern) \\
\end{longtable}
}

\textbf{Diagram: Signal Representations}

\begin{figure}
\centering
\pandocbounded{\includesvg[keepaspectratio]{diagrams/1333201-s2024-q3a.svg}}
\caption{Signal Representations}
\end{figure}

\begin{lstlisting}
                  Time Domain                     Frequency Domain
Sine     /\      /\      /\          |
Wave    /  \    /  \    /  \         |            ^
       /    \  /    \  /    \        |            |
      /      \/      \/      \       |            |
------------------------------       |     -------+-------+--------
                                     |            f_{0}      
                                     |
Square  ____      ____               |            ^
Wave   |    |    |    |              |            |
       |    |    |    |              |            |  ^     ^
_______|    |____|    |______        |     -------+--+-----+------
                                     |            f_{0} 3f_{0}   5f_{0}
\end{lstlisting}

\end{solutionbox}
\begin{mnemonicbox}
``SOFT'' - Sine has One Frequency, square has
Timeless harmonics

\end{mnemonicbox}
\subsection*{Question 3(b) [4 marks]}\label{q3b}

\textbf{Explain sampling theorem.}

\begin{solutionbox}
Sampling theorem states the conditions for accurate
signal reconstruction from samples.


{\def\LTcaptype{none} % do not increment counter
\vspace{-5pt}
\captionof{table}{Sampling Theorem}
\vspace{-10pt}
\begin{longtable}[]{@{}
  >{\raggedright\arraybackslash}p{(\linewidth - 2\tabcolsep) * \real{0.3810}}
  >{\raggedright\arraybackslash}p{(\linewidth - 2\tabcolsep) * \real{0.6190}}@{}}
\toprule\noalign{}
\begin{minipage}[b]{\linewidth}\raggedright
Aspect
\end{minipage} & \begin{minipage}[b]{\linewidth}\raggedright
Description
\end{minipage} \\
\midrule\noalign{}
\endhead
\bottomrule\noalign{}
\endlastfoot
\textbf{Statement} & To reconstruct a signal perfectly, sampling
frequency must be at least twice the highest frequency in signal \\
\textbf{Nyquist Rate} & fs \geq 2fmax (minimum sampling frequency) \\
\textbf{Aliasing} & Distortion that occurs when sampling below Nyquist
rate \\
\textbf{Example} & For voice (300-3400 Hz), fs \geq 6.8 kHz (typically 8
kHz) \\
\end{longtable}
}

\textbf{Diagram: Aliasing Effect}

\begin{figure}
\centering
\pandocbounded{\includesvg[keepaspectratio]{diagrams/1333201-s2024-q3b.svg}}
\caption{Aliasing Effect}
\end{figure}

\begin{lstlisting}
                        
  Original               Proper Sampling           Undersampling (Aliasing)
    /\      /\            *       *                  *           *  
   /  \    /  \                                                   
  /    \  /    \           *       *                 *           * 
 /      \/      \                                                 
----------------------    ----------------           ----------------
\end{lstlisting}

\end{solutionbox}
\begin{mnemonicbox}
``SNAP'' - Sample at Nyquist And Prevent aliasing

\end{mnemonicbox}
\subsection*{Question 3(c) [7 marks]}\label{q3c}

\textbf{Explain PAM, PPM and PWM.}

\begin{solutionbox}
These are pulse modulation techniques where a parameter
of pulse is varied.


{\def\LTcaptype{none} % do not increment counter
\vspace{-5pt}
\captionof{table}{Pulse Modulation Types}
\vspace{-10pt}
\begin{longtable}[]{@{}
  >{\raggedright\arraybackslash}p{(\linewidth - 6\tabcolsep) * \real{0.1176}}
  >{\raggedright\arraybackslash}p{(\linewidth - 6\tabcolsep) * \real{0.2157}}
  >{\raggedright\arraybackslash}p{(\linewidth - 6\tabcolsep) * \real{0.3529}}
  >{\raggedright\arraybackslash}p{(\linewidth - 6\tabcolsep) * \real{0.3137}}@{}}
\toprule\noalign{}
\begin{minipage}[b]{\linewidth}\raggedright
Type
\end{minipage} & \begin{minipage}[b]{\linewidth}\raggedright
Full Form
\end{minipage} & \begin{minipage}[b]{\linewidth}\raggedright
Parameter Varied
\end{minipage} & \begin{minipage}[b]{\linewidth}\raggedright
Characteristics
\end{minipage} \\
\midrule\noalign{}
\endhead
\bottomrule\noalign{}
\endlastfoot
\textbf{PAM} & Pulse Amplitude Modulation & Amplitude & Direct sampling
of analog signal \\
\textbf{PPM} & Pulse Position Modulation & Position/Time & Better noise
immunity than PAM \\
\textbf{PWM} & Pulse Width Modulation & Width/Duration & Superior noise
immunity, widely used in control systems \\
\end{longtable}
}

\textbf{Diagram: Pulse Modulation Techniques}

\begin{figure}
\centering
\pandocbounded{\includesvg[keepaspectratio]{diagrams/1333201-s2024-q3c.svg}}
\caption{Pulse Modulation Techniques}
\end{figure}

\begin{lstlisting}
Message:   /\/\/\

PAM:      _   _    __  _
         | | | |  |  || |
         |_| |_|  |__||_|

PPM:      _   _    _    _    _    _
         | | | |  | |  | |  | |  | |
         |_| |_|  |_|  |_|  |_|  |_|
         <-->     <----><->  <---->

PWM:      __      ___         _
         |  |    |   |       | |
         |__|    |___|       |_|
\end{lstlisting}

\end{solutionbox}
\begin{mnemonicbox}
``AAA-PPW'' - Amplitude, Position, Width are
modulated in PAM, PPM, PWM

\end{mnemonicbox}
\subsection*{Question 3(a) OR [3
marks]}\label{q3a}

\textbf{Define Nyquist rate and explain.}

\begin{solutionbox}
Nyquist rate is the minimum sampling frequency required
for accurate signal reconstruction.


{\def\LTcaptype{none} % do not increment counter
\vspace{-5pt}
\captionof{table}{Nyquist Rate}
\vspace{-10pt}
\begin{longtable}[]{@{}
  >{\raggedright\arraybackslash}p{(\linewidth - 2\tabcolsep) * \real{0.3810}}
  >{\raggedright\arraybackslash}p{(\linewidth - 2\tabcolsep) * \real{0.6190}}@{}}
\toprule\noalign{}
\begin{minipage}[b]{\linewidth}\raggedright
Aspect
\end{minipage} & \begin{minipage}[b]{\linewidth}\raggedright
Description
\end{minipage} \\
\midrule\noalign{}
\endhead
\bottomrule\noalign{}
\endlastfoot
\textbf{Definition} & Minimum sampling frequency needed to avoid
aliasing (fs = 2fmax) \\
\textbf{Implications} & Sampling below Nyquist rate causes irreversible
distortion \\
\textbf{Formula} & fs \geq 2fmax where fmax is highest frequency in
signal \\
\textbf{Application} & CD audio: 44.1 kHz sampling for 20 kHz audio \\
\end{longtable}
}

\end{solutionbox}
\begin{mnemonicbox}
``TANS'' - Twice As Needed for Sampling

\end{mnemonicbox}
\subsection*{Question 3(b) OR [4
marks]}\label{q3b}

\textbf{Explain quantization process.}

\begin{solutionbox}
Quantization assigns discrete amplitude levels to
sampled values in analog-to-digital conversion.


{\def\LTcaptype{none} % do not increment counter
\vspace{-5pt}
\captionof{table}{Quantization Process}
\vspace{-10pt}
\begin{longtable}[]{@{}
  >{\raggedright\arraybackslash}p{(\linewidth - 2\tabcolsep) * \real{0.3158}}
  >{\raggedright\arraybackslash}p{(\linewidth - 2\tabcolsep) * \real{0.6842}}@{}}
\toprule\noalign{}
\begin{minipage}[b]{\linewidth}\raggedright
Step
\end{minipage} & \begin{minipage}[b]{\linewidth}\raggedright
Description
\end{minipage} \\
\midrule\noalign{}
\endhead
\bottomrule\noalign{}
\endlastfoot
\textbf{Sampling} & Discrete-time samples taken from continuous
signal \\
\textbf{Level Assignment} & Each sample assigned to nearest quantization
level \\
\textbf{Quantization Error} & Difference between actual and quantized
value \\
\textbf{Quantization Noise} & Statistical effect of errors in signal \\
\textbf{Resolution} & Determined by number of bits (2^{n} levels for n
bits) \\
\end{longtable}
}

\textbf{Diagram: Quantization Process}

\begin{lstlisting}
Analog   /\      /\
Signal  /  \    /  \
       /    \  /    \

Sampled   *  *    *  *
Signal       *       *

          -------------
          |           |
Quantized  *--*    *--*
Signal        *       *
          |           |
          -------------
\end{lstlisting}

\end{solutionbox}
\begin{mnemonicbox}
``SLERN'' - Sample, Level assign, Error occurs,
Resolution determines Noise

\end{mnemonicbox}
\subsection*{Question 3(c) OR [7
marks]}\label{q3c}

\textbf{Explain Ideal, Natural and Flat top sampling.}

\begin{solutionbox}
These are different practical implementations of
sampling process.


{\def\LTcaptype{none} % do not increment counter
\vspace{-5pt}
\captionof{table}{Sampling Types Comparison}
\vspace{-10pt}
\begin{longtable}[]{@{}
  >{\raggedright\arraybackslash}p{(\linewidth - 6\tabcolsep) * \real{0.0938}}
  >{\raggedright\arraybackslash}p{(\linewidth - 6\tabcolsep) * \real{0.2031}}
  >{\raggedright\arraybackslash}p{(\linewidth - 6\tabcolsep) * \real{0.2656}}
  >{\raggedright\arraybackslash}p{(\linewidth - 6\tabcolsep) * \real{0.4375}}@{}}
\toprule\noalign{}
\begin{minipage}[b]{\linewidth}\raggedright
Type
\end{minipage} & \begin{minipage}[b]{\linewidth}\raggedright
Description
\end{minipage} & \begin{minipage}[b]{\linewidth}\raggedright
Characteristics
\end{minipage} & \begin{minipage}[b]{\linewidth}\raggedright
Mathematical Representation
\end{minipage} \\
\midrule\noalign{}
\endhead
\bottomrule\noalign{}
\endlastfoot
\textbf{Ideal} & Instantaneous samples at zero width & Theoretical
concept, not physically realizable & s(t) = m(t) \times \sumδ(t-nTs) \\
\textbf{Natural} & Samples modulate pulse train & Practical
implementation using analog switch & s(t) = m(t) \times p(t) \\
\textbf{Flat-top} & Holds sample value until next sample & Easiest to
implement, sample-and-hold circuit & s(t) =
\summ(nTs)[u(t-nTs)-u(t-(n+1)Ts)] \\
\end{longtable}
}

\textbf{Diagram: Sampling Types}

\begin{lstlisting}
Original:   /\/\/\

Ideal:      |  |  |  |  |

Natural:    _   _   _   _
           | | | | | | | |
           |_| |_| |_| |_|

Flat-top:   _____ _____ _____
           |     |     |     |
           |_____|_____|_____|
\end{lstlisting}

\end{solutionbox}
\begin{mnemonicbox}
``INF'' - Ideal is theoretical, Natural is practical,
Flat-top holds values

\end{mnemonicbox}
\subsection*{Question 4(a) [3 marks]}\label{q4a}

\textbf{List the advantages and disadvantages of PCM.}

\begin{solutionbox}


{\def\LTcaptype{none} % do not increment counter
\vspace{-5pt}
\captionof{table}{PCM Advantages and Disadvantages}
\vspace{-10pt}
\begin{longtable}[]{@{}
  >{\raggedright\arraybackslash}p{(\linewidth - 2\tabcolsep) * \real{0.4444}}
  >{\raggedright\arraybackslash}p{(\linewidth - 2\tabcolsep) * \real{0.5556}}@{}}
\toprule\noalign{}
\begin{minipage}[b]{\linewidth}\raggedright
Advantages
\end{minipage} & \begin{minipage}[b]{\linewidth}\raggedright
Disadvantages
\end{minipage} \\
\midrule\noalign{}
\endhead
\bottomrule\noalign{}
\endlastfoot
\textbf{High noise immunity} & Requires higher bandwidth \\
\textbf{Better signal quality} & Complex circuitry \\
\textbf{Compatible with digital systems} & Quantization noise \\
\textbf{Secure communication possible} & Higher power consumption \\
\textbf{Can be regenerated without degradation} & Synchronization
required \\
\end{longtable}
}

\end{solutionbox}
\begin{mnemonicbox}
``NICHE'' vs ``BCQPS'' - Noise immunity, Integration,
Complex circuitry, Higher bandwidth, Error correction vs Bandwidth,
Cost, Quantization, Power, Synchronization

\end{mnemonicbox}
\subsection*{Question 4(b) [4 marks]}\label{q4b}

\textbf{Draw and Explain Block Diagram of Delta Modulation.}

\begin{solutionbox}
Delta modulation transmits only changes in signal level
using 1-bit quantization.

\includegraphics[width=1\linewidth,height=\textheight,keepaspectratio]{mermaid-127ce72a.pdf}


{\def\LTcaptype{none} % do not increment counter
\vspace{-5pt}
\captionof{table}{Delta Modulation Components}
\vspace{-10pt}
\begin{longtable}[]{@{}
  >{\raggedright\arraybackslash}p{(\linewidth - 2\tabcolsep) * \real{0.4118}}
  >{\raggedright\arraybackslash}p{(\linewidth - 2\tabcolsep) * \real{0.5882}}@{}}
\toprule\noalign{}
\begin{minipage}[b]{\linewidth}\raggedright
Block
\end{minipage} & \begin{minipage}[b]{\linewidth}\raggedright
Function
\end{minipage} \\
\midrule\noalign{}
\endhead
\bottomrule\noalign{}
\endlastfoot
\textbf{Comparator} & Compares input with predicted value \\
\textbf{1-bit Quantizer} & Outputs 1 if difference positive, 0 if
negative \\
\textbf{Integrator} & Accumulates step values to track input \\
\textbf{Delay} & Provides previous output for comparison \\
\end{longtable}
}

\end{solutionbox}
\begin{mnemonicbox}
``CQID'' - Compare, Quantize with 1-bit, Integrate,
Delay

\end{mnemonicbox}
\subsection*{Question 4(c) [7 marks]}\label{q4c}

\textbf{Compare PCM, DM and DPCM.}

\begin{solutionbox}


{\def\LTcaptype{none} % do not increment counter
\vspace{-5pt}
\captionof{table}{Comparison of Digital Modulation Techniques}
\vspace{-10pt}
\begin{longtable}[]{@{}
  >{\raggedright\arraybackslash}p{(\linewidth - 6\tabcolsep) * \real{0.4583}}
  >{\raggedright\arraybackslash}p{(\linewidth - 6\tabcolsep) * \real{0.2083}}
  >{\raggedright\arraybackslash}p{(\linewidth - 6\tabcolsep) * \real{0.1667}}
  >{\raggedright\arraybackslash}p{(\linewidth - 6\tabcolsep) * \real{0.1667}}@{}}
\toprule\noalign{}
\begin{minipage}[b]{\linewidth}\raggedright
Parameter
\end{minipage} & \begin{minipage}[b]{\linewidth}\raggedright
PCM
\end{minipage} & \begin{minipage}[b]{\linewidth}\raggedright
DM
\end{minipage} & \begin{minipage}[b]{\linewidth}\raggedright
DPCM
\end{minipage} \\
\midrule\noalign{}
\endhead
\bottomrule\noalign{}
\endlastfoot
\textbf{Bits per sample} & 8-16 bits & 1 bit & 4-6 bits \\
\textbf{Bandwidth} & Highest & Lowest & Medium \\
\textbf{Signal-to-Noise Ratio} & Highest & Lowest & Medium \\
\textbf{Circuit Complexity} & High & Simple & Medium \\
\textbf{Sampling Rate} & Nyquist & Multiple of Nyquist & Nyquist \\
\textbf{Error Types} & Quantization error & Slope overload, granular
noise & Prediction error \\
\textbf{Applications} & CD audio, digital telephony & Low-quality voice
& Speech, video coding \\
\end{longtable}
}

\end{solutionbox}
\begin{mnemonicbox}
``PCM-DM-DPCM: More Bits Better Quality, More
Complexity Needed''

\end{mnemonicbox}
\subsection*{Question 4(a) OR [3
marks]}\label{q4a}

\textbf{Explain DPCM.}

\begin{solutionbox}
Differential Pulse Code Modulation encodes difference
between actual and predicted sample.


{\def\LTcaptype{none} % do not increment counter
\vspace{-5pt}
\captionof{table}{DPCM Characteristics}
\vspace{-10pt}
\begin{longtable}[]{@{}
  >{\raggedright\arraybackslash}p{(\linewidth - 2\tabcolsep) * \real{0.3810}}
  >{\raggedright\arraybackslash}p{(\linewidth - 2\tabcolsep) * \real{0.6190}}@{}}
\toprule\noalign{}
\begin{minipage}[b]{\linewidth}\raggedright
Aspect
\end{minipage} & \begin{minipage}[b]{\linewidth}\raggedright
Description
\end{minipage} \\
\midrule\noalign{}
\endhead
\bottomrule\noalign{}
\endlastfoot
\textbf{Basic Principle} & Encodes difference between actual and
predicted value \\
\textbf{Predictor} & Uses previous samples to predict current value \\
\textbf{Advantage} & Requires fewer bits than PCM (exploits
correlation) \\
\textbf{Bit Rate Reduction} & Typically 25-50\% compared to PCM \\
\textbf{Applications} & Speech coding, image compression \\
\end{longtable}
}

\end{solutionbox}
\begin{mnemonicbox}
``DPCM: Difference Predicted, Correlation Matters''

\end{mnemonicbox}
\subsection*{Question 4(b) OR [4
marks]}\label{q4b}

\textbf{List the advantages and disadvantages of Delta Modulation.}

\begin{solutionbox}


{\def\LTcaptype{none} % do not increment counter
\vspace{-5pt}
\captionof{table}{Delta Modulation - Pros and Cons}
\vspace{-10pt}
\begin{longtable}[]{@{}ll@{}}
\toprule\noalign{}
Advantages & Disadvantages \\
\midrule\noalign{}
\endhead
\bottomrule\noalign{}
\endlastfoot
\textbf{Simple implementation} & Slope overload distortion \\
\textbf{Low bit rate} & Granular noise at low amplitudes \\
\textbf{Single bit transmission} & Limited dynamic range \\
\textbf{Robust against channel errors} & Higher sampling rate
required \\
\textbf{Low complexity hardware} & Lower SNR than PCM \\
\end{longtable}
}

\end{solutionbox}
\begin{mnemonicbox}
``SLSRL'' vs ``SGLSH'' - Simple, Low bit-rate, Single
bit, Robust, Low cost vs Slope overload, Granular noise, Limited range,
Sampling high, SNR low

\end{mnemonicbox}
\subsection*{Question 4(c) OR [7
marks]}\label{q4c}

\textbf{Explain Block diagram of basic PCM-TDM system.}

\begin{solutionbox}
PCM-TDM combines multiple digitized signals into a
single high-speed channel.

\includegraphics[width=1\linewidth,height=\textheight,keepaspectratio]{mermaid-3ff5bf9b.pdf}


{\def\LTcaptype{none} % do not increment counter
\vspace{-5pt}
\captionof{table}{PCM-TDM System Components}
\vspace{-10pt}
\begin{longtable}[]{@{}
  >{\raggedright\arraybackslash}p{(\linewidth - 2\tabcolsep) * \real{0.4118}}
  >{\raggedright\arraybackslash}p{(\linewidth - 2\tabcolsep) * \real{0.5882}}@{}}
\toprule\noalign{}
\begin{minipage}[b]{\linewidth}\raggedright
Block
\end{minipage} & \begin{minipage}[b]{\linewidth}\raggedright
Function
\end{minipage} \\
\midrule\noalign{}
\endhead
\bottomrule\noalign{}
\endlastfoot
\textbf{PCM Encoder} & Converts analog signal to digital (sampling,
quantization, coding) \\
\textbf{TDM Multiplexer} & Combines multiple PCM streams into single
high-speed stream \\
\textbf{Transmission Channel} & Medium for signal transmission \\
\textbf{TDM Demultiplexer} & Separates time-multiplexed stream back into
individual channels \\
\textbf{PCM Decoder} & Converts digital back to analog (decoding,
filtering) \\
\textbf{Synchronization} & Clock and frame sync signals ensure proper
demultiplexing \\
\textbf{Frame Structure} & Contains samples from all channels plus sync
bits \\
\end{longtable}
}

\end{solutionbox}
\begin{mnemonicbox}
``PETDSF'' - PCM Encodes, TDM combines, Digital
transmits, Separation occurs, Frames synchronize

\end{mnemonicbox}
\subsection*{Question 5(a) [3 marks]}\label{q5a}

\textbf{Explain Adaptive Delta modulation.}

\begin{solutionbox}
Adaptive Delta Modulation adjusts step size based on
signal characteristics.


{\def\LTcaptype{none} % do not increment counter
\vspace{-5pt}
\captionof{table}{Adaptive Delta Modulation}
\vspace{-10pt}
\begin{longtable}[]{@{}
  >{\raggedright\arraybackslash}p{(\linewidth - 2\tabcolsep) * \real{0.4091}}
  >{\raggedright\arraybackslash}p{(\linewidth - 2\tabcolsep) * \real{0.5909}}@{}}
\toprule\noalign{}
\begin{minipage}[b]{\linewidth}\raggedright
Feature
\end{minipage} & \begin{minipage}[b]{\linewidth}\raggedright
Description
\end{minipage} \\
\midrule\noalign{}
\endhead
\bottomrule\noalign{}
\endlastfoot
\textbf{Basic Principle} & Varies step size according to signal slope \\
\textbf{Step Size Control} & Increases when same bit pattern repeats
(signal changing rapidly) \\
\textbf{Advantages} & Reduced slope overload and granular noise \\
\textbf{Implementation} & Uses shift register to detect bit patterns \\
\textbf{Performance} & Better SNR than standard DM \\
\end{longtable}
}

\textbf{Diagram: Step Size Adaptation}

\begin{figure}
\centering
\pandocbounded{\includesvg[keepaspectratio]{diagrams/1333201-s2024-q4c.svg}}
\caption{Step Size Adaptation}
\end{figure}

\begin{lstlisting}
Signal:   /\      /\
         /  \    /  \
        /    \  /    \

Steps:   _                  Larger steps
        / \                for steep slopes
       /   \___            
      /        \__         Smaller steps
     /            \___     for flat regions
\end{lstlisting}

\end{solutionbox}
\begin{mnemonicbox}
``ASSG'' - Adaptive Step Size Gives better
performance

\end{mnemonicbox}
\subsection*{Question 5(b) [4 marks]}\label{q5b}

\textbf{Define the terms 1. Radiation Pattern 2. Antenna gain.}

\begin{solutionbox}


{\def\LTcaptype{none} % do not increment counter
\vspace{-5pt}
\captionof{table}{Antenna Terms}
\vspace{-10pt}
\begin{longtable}[]{@{}
  >{\raggedright\arraybackslash}p{(\linewidth - 4\tabcolsep) * \real{0.1765}}
  >{\raggedright\arraybackslash}p{(\linewidth - 4\tabcolsep) * \real{0.3529}}
  >{\raggedright\arraybackslash}p{(\linewidth - 4\tabcolsep) * \real{0.4706}}@{}}
\toprule\noalign{}
\begin{minipage}[b]{\linewidth}\raggedright
Term
\end{minipage} & \begin{minipage}[b]{\linewidth}\raggedright
Definition
\end{minipage} & \begin{minipage}[b]{\linewidth}\raggedright
Characteristics
\end{minipage} \\
\midrule\noalign{}
\endhead
\bottomrule\noalign{}
\endlastfoot
\textbf{Radiation Pattern} & Graphical representation of radiation
properties of antenna in space & Shows directional dependencies of
radiated power \\
\textbf{Antenna Gain} & Measure of antenna's ability to direct or
concentrate radio energy in a particular direction & Expressed in dB,
compared to isotropic radiator (dBi) \\
\end{longtable}
}

\textbf{Diagram: Radiation Pattern Types}

\begin{figure}
\centering
\pandocbounded{\includesvg[keepaspectratio]{diagrams/1333201-s2024-q5a.svg}}
\caption{Radiation Pattern Types}
\end{figure}

\begin{lstlisting}
                  y                      y
                  ^                      ^
                  |                      |
         .........|.........    *********|*********
         *        |        *    *        |        *
         *        |        *    *        |        *
         *        +------->x    *        +------->x
         *                 *    *                 *
         *                 *    *                 *
         *******************    ...................
         
      Omnidirectional         Directional
\end{lstlisting}

\end{solutionbox}
\begin{mnemonicbox}
``RPGD'' - Radiation Pattern shows Gain Direction

\end{mnemonicbox}
\subsection*{Question 5(c) [7 marks]}\label{q5c}

\textbf{Explain Base station antenna and Mobile station antenna.}

\begin{solutionbox}
Different antenna designs serve different purposes in
wireless communication systems.


{\def\LTcaptype{none} % do not increment counter
\vspace{-5pt}
\captionof{table}{Comparison of Base Station and Mobile Station Antennas}
\vspace{-10pt}
\begin{longtable}[]{@{}lll@{}}
\toprule\noalign{}
Parameter & Base Station Antenna & Mobile Station Antenna \\
\midrule\noalign{}
\endhead
\bottomrule\noalign{}
\endlastfoot
\textbf{Height} & 15-50 meters & Less than 2 meters \\
\textbf{Gain} & Higher (10-20 dBi) & Lower (0-3 dBi) \\
\textbf{Pattern} & Sectoral (120^\circ sectors) & Omnidirectional \\
\textbf{Size} & Larger arrays & Compact, integrated \\
\textbf{Types} & Panel, Yagi, Collinear & Monopole, PIFA, chip \\
\textbf{Polarization} & Vertical, cross-polarized & Typically
vertical \\
\textbf{Beamforming} & Often used & Rarely used in basic devices \\
\textbf{Diversity} & Space/polarization diversity & Rarely
implemented \\
\end{longtable}
}

\textbf{Diagram: Antenna Types}

\begin{figure}
\centering
\pandocbounded{\includesvg[keepaspectratio]{diagrams/1333201-s2024-q5b.svg}}
\caption{Antenna Types}
\end{figure}

\begin{lstlisting}
Base Station:                  Mobile Station:
  
    |     |                         |
    |     |                         |
   /|     |\                       /|\
  / |     | \                      / \
 /__|_____|__\                    /___\
     |     |
\end{lstlisting}

\end{solutionbox}
\begin{mnemonicbox}
``BHPSTBD'' - Base stations Have Power, Size, Tower
mounting, Beamforming, Diversity

\end{mnemonicbox}
\subsection*{Question 5(a) OR [3
marks]}\label{q5a}

\textbf{Write down range of frequencies for HF, VHF and UHF.}

\begin{solutionbox}


{\def\LTcaptype{none} % do not increment counter
\vspace{-5pt}
\captionof{table}{Frequency Bands}
\vspace{-10pt}
\begin{longtable}[]{@{}
  >{\raggedright\arraybackslash}p{(\linewidth - 6\tabcolsep) * \real{0.1053}}
  >{\raggedright\arraybackslash}p{(\linewidth - 6\tabcolsep) * \real{0.2982}}
  >{\raggedright\arraybackslash}p{(\linewidth - 6\tabcolsep) * \real{0.2105}}
  >{\raggedright\arraybackslash}p{(\linewidth - 6\tabcolsep) * \real{0.3860}}@{}}
\toprule\noalign{}
\begin{minipage}[b]{\linewidth}\raggedright
Band
\end{minipage} & \begin{minipage}[b]{\linewidth}\raggedright
Frequency Range
\end{minipage} & \begin{minipage}[b]{\linewidth}\raggedright
Wavelength
\end{minipage} & \begin{minipage}[b]{\linewidth}\raggedright
Notable Applications
\end{minipage} \\
\midrule\noalign{}
\endhead
\bottomrule\noalign{}
\endlastfoot
\textbf{HF} & 3-30 MHz & 100-10 m & Shortwave radio, amateur radio,
aviation \\
\textbf{VHF} & 30-300 MHz & 10-1 m & FM radio, TV channels 2-13, air
traffic \\
\textbf{UHF} & 300-3000 MHz & 1-0.1 m & TV channels 14-83, mobile
phones, Wi-Fi \\
\end{longtable}
}

\end{solutionbox}
\begin{mnemonicbox}
``3-30-300-3000'' - Each band starts at 3 times a
power of 10 MHz

\end{mnemonicbox}
\subsection*{Question 5(b) OR [4
marks]}\label{q5b}

\textbf{Define the terms 1. Antenna Directivity 2. Polarization.}

\begin{solutionbox}


{\def\LTcaptype{none} % do not increment counter
\vspace{-5pt}
\captionof{table}{Antenna Properties}
\vspace{-10pt}
\begin{longtable}[]{@{}
  >{\raggedright\arraybackslash}p{(\linewidth - 4\tabcolsep) * \real{0.1765}}
  >{\raggedright\arraybackslash}p{(\linewidth - 4\tabcolsep) * \real{0.3529}}
  >{\raggedright\arraybackslash}p{(\linewidth - 4\tabcolsep) * \real{0.4706}}@{}}
\toprule\noalign{}
\begin{minipage}[b]{\linewidth}\raggedright
Term
\end{minipage} & \begin{minipage}[b]{\linewidth}\raggedright
Definition
\end{minipage} & \begin{minipage}[b]{\linewidth}\raggedright
Characteristics
\end{minipage} \\
\midrule\noalign{}
\endhead
\bottomrule\noalign{}
\endlastfoot
\textbf{Directivity} & Ratio of radiation intensity in a given direction
to average radiation intensity & Measured in dBi, indicates focus of
antenna \\
\textbf{Polarization} & Orientation of electric field vector of radiated
wave & Linear (vertical/horizontal), circular, elliptical \\
\end{longtable}
}

\textbf{Diagram: Polarization Types}

\begin{figure}
\centering
\pandocbounded{\includesvg[keepaspectratio]{diagrams/1333201-s2024-q5c.svg}}
\caption{Polarization Types}
\end{figure}

\begin{lstlisting}
Vertical:     Horizontal:     Circular:
   
    |             ----            /|\ 
    |             ----             |
    |             ----            \|/
    |             ----            /|\
\end{lstlisting}

\end{solutionbox}
\begin{mnemonicbox}
``DIVE POLE'' - DIrectivity shows Vector Excellence,
POLarization shows Electric field

\end{mnemonicbox}
\subsection*{Question 5(c) OR [7
marks]}\label{q5c}

\textbf{Explain Ground wave propagation and Space wave propagation in
detail.}

\begin{solutionbox}
These are two primary modes of radio wave propagation
in the lower atmosphere.


{\def\LTcaptype{none} % do not increment counter
\vspace{-5pt}
\captionof{table}{Wave Propagation Comparison}
\vspace{-10pt}
\begin{longtable}[]{@{}
  >{\raggedright\arraybackslash}p{(\linewidth - 4\tabcolsep) * \real{0.3056}}
  >{\raggedright\arraybackslash}p{(\linewidth - 4\tabcolsep) * \real{0.3611}}
  >{\raggedright\arraybackslash}p{(\linewidth - 4\tabcolsep) * \real{0.3333}}@{}}
\toprule\noalign{}
\begin{minipage}[b]{\linewidth}\raggedright
Parameter
\end{minipage} & \begin{minipage}[b]{\linewidth}\raggedright
Ground Wave
\end{minipage} & \begin{minipage}[b]{\linewidth}\raggedright
Space Wave
\end{minipage} \\
\midrule\noalign{}
\endhead
\bottomrule\noalign{}
\endlastfoot
\textbf{Frequency Range} & Below 2 MHz & Above 30 MHz \\
\textbf{Distance Coverage} & 100-300 km & Limited to line-of-sight +
diffraction \\
\textbf{Path} & Follows earth's curvature & Direct and ground-reflected
paths \\
\textbf{Mechanism} & Diffraction around earth's surface & Line-of-sight
propagation with reflection \\
\textbf{Attenuation} & Higher (increases with frequency) & Lower at
VHF/UHF ranges \\
\textbf{Polarization} & Vertical polarization preferred & Both vertical
and horizontal usable \\
\textbf{Applications} & AM broadcasting, navigation beacons & TV, FM
radio, microwave links \\
\textbf{Factors Affecting} & Ground conductivity, terrain & Antenna
height, terrain, obstacles \\
\end{longtable}
}

\textbf{Diagram: Ground Wave vs Space Wave Propagation}

\begin{lstlisting}
                                      /
                                     /  Direct
      Transmitter                   /      Wave         Receiver
          |                        /                       |
          |                       /                        |
          |                      /                         |
          |                     /                          |
          |                    /                           |
          |                   /                            |
         /|\ - - - - - - - - - - - - - - - - - - - - - - -/|\
          |        Ground Wave                             |
          |                                                |
    =====================================================
                  Earth Surface
\end{lstlisting}

\textbf{Ground Wave Propagation:}

\begin{itemize}
\tightlist
\item
  Travels along earth's surface
\item
  Signal strength decreases with distance
\item
  Better propagation over sea than land
\item
  Affected by ground conductivity and dielectric constant
\item
  Used for AM broadcasting, maritime communication
\end{itemize}

\textbf{Space Wave Propagation:}

\begin{itemize}
\tightlist
\item
  Consists of direct wave and ground-reflected wave
\item
  Range extended by atmospheric refraction
\item
  Range formula: d = \sqrt(2Rh) where R is earth's radius, h is antenna
  height
\item
  Affected by diffraction over obstacles
\item
  Used for line-of-sight communications like TV, FM, microwave links
\end{itemize}

\end{solutionbox}
\begin{mnemonicbox}
``GAFFS'' - Ground Adheres to earth, Follows surface,
Frequencies low, Short wavelengths

\end{mnemonicbox}

\end{document}
