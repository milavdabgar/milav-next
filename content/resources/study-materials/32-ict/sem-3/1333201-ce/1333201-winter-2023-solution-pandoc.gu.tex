\documentclass[10pt,a4paper]{article}

% content/resources/templates/preamble.tex
\usepackage[margin=0.6in]{geometry}
\author{Milav Dabgar}
\usepackage{amsmath,amssymb,amsthm}
\usepackage{booktabs}
\usepackage{multirow}
\usepackage{xcolor}
\usepackage{tcolorbox}
\tcbuselibrary{breakable,skins}
\usepackage[colorlinks=true,linkcolor=blue]{hyperref}
\usepackage{titlesec}
\usepackage{enumitem}
\usepackage{tikz}
\usepackage{pgfplots}
\usepackage{circuitikz}
\usepackage[version=4]{mhchem}
\usepackage{longtable}
\usepackage{array}
\usepackage{float}
\usepackage{caption}
\usepackage{listings}

\lstset{
  basicstyle=\small\ttfamily,
  breaklines=true,
  breakatwhitespace=false,
  postbreak=\mbox{\textcolor{red}{$\hookrightarrow$}\space},
  float=false,
  numbers=left,
  numberstyle=\tiny\color{gray},
  numbersep=10pt,
  xleftmargin=2em,
  keywordstyle=\color{blue},
  commentstyle=\color{green!60!black},
  stringstyle=\color{purple},
  backgroundcolor=\color{gray!5},
  showstringspaces=false,
  tabsize=2,
  captionpos=b,
  keepspaces=true,
  columns=flexible
}

\pgfplotsset{compat=1.18}
\usetikzlibrary{shapes,arrows,positioning,calc,patterns,decorations.pathmorphing,decorations.markings,arrows.meta}

% Color scheme
\definecolor{headcolor}{RGB}{0,102,204}
\definecolor{keycolor}{RGB}{220,20,60}
\definecolor{solutioncolor}{RGB}{34,139,34}
\definecolor{mnemoniccolor}{RGB}{148,0,211}
\definecolor{codecolor}{RGB}{0,0,100}

% Spacing
\setlength{\parskip}{3pt}
\setlist[itemize]{nosep}
\setlist[enumerate]{nosep}

% Title formatting
\titleformat{\section}{\Large\bfseries\color{headcolor}}{\thesection}{1em}{}
\titleformat{\subsection}{\large\bfseries\color{headcolor}}{\thesubsection}{1em}{}

% Pandoc tightlist compatibility
\providecommand{\tightlist}{%
  \setlength{\itemsep}{0pt}\setlength{\parskip}{0pt}}

% Pandoc longtable compatibility
\newcounter{none}
\def\thenone{}


% content/resources/templates/gujarati-boxes.tex
\usepackage{fontspec}
\usepackage{polyglossia}

% Set Gujarati as main language (document is primarily in Gujarati)
% Note: gloss-gujarati.ldf doesn't exist in polyglossia, but it will use hyphenation patterns
\setdefaultlanguage{gujarati}
\setotherlanguage{english}

% Configure Gujarati font properly
% Use Language=Default to prevent polyglossia from trying to add language-specific features
% that don't exist for Gujarati, which causes "empty feature" warnings
\newfontfamily\gujaratifont[Script=Gujarati,AutoFakeBold=2.5,AutoFakeSlant=0.3]{Noto Sans Gujarati}
\setmainfont[Script=Gujarati,AutoFakeBold=2.5,AutoFakeSlant=0.3]{Noto Sans Gujarati}
% Use Noto Sans Gujarati for monospace to support Gujarati in text
\setmonofont[Scale=0.9]{Noto Sans Gujarati}

% Configure English to use the same font
\newfontfamily\englishfont[Script=Gujarati,AutoFakeBold=2.5,AutoFakeSlant=0.3]{Noto Sans Gujarati}

% Translations for polyglossia
\gappto\captionsgujarati{
  \renewcommand{\tablename}{કોષ્ટક}
  \renewcommand{\figurename}{આકૃતિ}
}

% Helper for TikZ nodes to ensure Gujarati font
\newcommand{\gu}[1]{{\gujaratifont #1}}

% Custom environments
\newtcolorbox{solutionbox}{
    breakable,
    enhanced,
    colback=solutioncolor!5!white,
    colframe=solutioncolor!75!black,
    fonttitle=\bfseries,
    title=જવાબ
}

\newtcolorbox{solutionboxnobreak}{
 colback=solutioncolor!5!white,
 colframe=solutioncolor!75!black,
 fonttitle=\bfseries,
 title=જવાબ
}

\newtcolorbox{keyformula}{
 breakable,
 enhanced,
 colback=keycolor!5!white,
 colframe=keycolor!75!black,
 fonttitle=\bfseries,
 title=રાસાયણિક સમીકરણ/સૂત્ર
}

\newtcolorbox{mnemonicbox}{
 breakable,
 enhanced,
 colback=mnemoniccolor!5!white,
 colframe=mnemoniccolor!75!black,
 fonttitle=\bfseries,
 title=મેમરી ટ્રીક
}


\begin{document}

\begin{center}
{\Huge\bfseries\color{headcolor} Subject Name (Gujarati)}\\[5pt]
{\LARGE 1333201 -- Winter 2023}\\[3pt]
{\large Semester 1 Study Material}\\[3pt]
{\normalsize\textit{Detailed Solutions and Explanations}}
\end{center}

\vspace{10pt}

\subsection*{પ્રશ્ન 1(અ) [3
ગુણ]}\label{uxaaauxab0uxab6uxaa8-1uxa85-3-uxa97uxaa3}

\textbf{વ્યાખ્યા આપો: (અ) Amplitude Modulation, (બ) Frequency Modulation
અને (ક) Phase Modulation}

\begin{solutionbox}


{\def\LTcaptype{none} % do not increment counter
\vspace{-5pt}
\captionof{table}{મોડ્યુલેશન પ્રકારો}
\vspace{-10pt}
\begin{longtable}[]{@{}
  >{\raggedright\arraybackslash}p{(\linewidth - 2\tabcolsep) * \real{0.5714}}
  >{\raggedright\arraybackslash}p{(\linewidth - 2\tabcolsep) * \real{0.4286}}@{}}
\toprule\noalign{}
\begin{minipage}[b]{\linewidth}\raggedright
મોડ્યુલેશન પ્રકાર
\end{minipage} & \begin{minipage}[b]{\linewidth}\raggedright
વ્યાખ્યા
\end{minipage} \\
\midrule\noalign{}
\endhead
\bottomrule\noalign{}
\endlastfoot
\textbf{Amplitude Modulation (AM)} & એક પ્રક્રિયા જેમાં carrier સિગ્નલનું
amplitude, modulating સિગ્નલની ક્ષણિક કિંમત અનુસાર બદલાય છે જ્યારે frequency
અચળ રહે છે \\
\textbf{Frequency Modulation (FM)} & એક પ્રક્રિયા જેમાં carrier સિગ્નલની
frequency, modulating સિગ્નલની ક્ષણિક કિંમત અનુસાર બદલાય છે જ્યારે amplitude
અચળ રહે છે \\
\textbf{Phase Modulation (PM)} & એક પ્રક્રિયા જેમાં carrier સિગ્નલનો phase,
modulating સિગ્નલની ક્ષણિક કિંમત અનુસાર બદલાય છે જ્યારે amplitude અચળ રહે છે \\
\end{longtable}
}

\end{solutionbox}
\begin{mnemonicbox}
``A-F-P: Amplitude બદલાય છે, Frequency ખસે છે, Phase
સમાયોજિત થાય છે''

\end{mnemonicbox}
\subsection*{પ્રશ્ન 1(બ) [4
ગુણ]}\label{uxaaauxab0uxab6uxaa8-1uxaac-4-uxa97uxaa3}

\textbf{મોડ્યુલેશનની જરૂરિયાત સમજાવો.}

\begin{solutionbox}


{\def\LTcaptype{none} % do not increment counter
\vspace{-5pt}
\captionof{table}{મોડ્યુલેશનની જરૂરિયાત}
\vspace{-10pt}
\begin{longtable}[]{@{}
  >{\raggedright\arraybackslash}p{(\linewidth - 2\tabcolsep) * \real{0.3158}}
  >{\raggedright\arraybackslash}p{(\linewidth - 2\tabcolsep) * \real{0.6842}}@{}}
\toprule\noalign{}
\begin{minipage}[b]{\linewidth}\raggedright
જરૂરિયાત
\end{minipage} & \begin{minipage}[b]{\linewidth}\raggedright
સમજૂતી
\end{minipage} \\
\midrule\noalign{}
\endhead
\bottomrule\noalign{}
\endlastfoot
\textbf{પ્રેક્ટિકલ એન્ટેના સાઈઝ} & frequency વધારીને એન્ટેનાનું કદ ઘટાડે છે (એન્ટેના
લંબાઈ = λ/4) \\
\textbf{ઇન્ટરફેરન્સ ઘટાડો} & અલગ-અલગ frequencies પર એક સાથે ઘણા સિગ્નલો
પ્રસારિત કરવાની મંજૂરી આપે છે \\
\textbf{રેન્જ વિસ્તરણ} & ઉચ્ચ frequency સિગ્નલો વાતાવરણમાં વધુ દૂર સુધી જાય છે \\
\textbf{મલ્ટિપ્લેક્સિંગ} & ઘણા સિગ્નલોને કોમ્યુનિકેશન માધ્યમ શેર કરવા સક્ષમ બનાવે છે \\
\end{longtable}
}

\textbf{આકૃતિ:}

\includegraphics[width=1\linewidth,height=\textheight,keepaspectratio]{mermaid-22ca4ab0.pdf}

\end{solutionbox}
\begin{mnemonicbox}
``PIRM: પ્રેક્ટિકલ એન્ટેના, ઇન્ટરફેરન્સ ઘટાડો, રેન્જ વિસ્તરણ,
મલ્ટિપ્લેક્સિંગ''

\end{mnemonicbox}
\subsection*{પ્રશ્ન 1(ક) [7
ગુણ]}\label{uxaaauxab0uxab6uxaa8-1uxa95-7-uxa97uxaa3}

\textbf{અમ્પ્લિટુડ મોડ્યુલેશનમાં મોડ્યુલેટિંગ સિગ્નલને 3V નું અમ્પ્લિટુડ અને 1 KHz ની
ફ્રિક્વન્સી છે જ્યારે કેરિયર સિગ્નલને 10 V નું અમ્પ્લિટુડ અને 30 KHz ની ફ્રિક્વન્સી છે.
મોડ્યુલેશન ઇન્ડેક્સ, સાઇડબેન્ડ ફ્રીક્વન્સીઝ અને તેમના અમ્પ્લિટુડ શોધો તેમજ આ AM વેવનું
સ્પેક્ટ્રમ દોરો.}

\begin{solutionbox}


{\def\LTcaptype{none} % do not increment counter
\vspace{-5pt}
\captionof{table}{આપેલ માહિતી}
\vspace{-10pt}
\begin{longtable}[]{@{}lll@{}}
\toprule\noalign{}
પરિમાણ & મોડ્યુલેટિંગ સિગ્નલ & કેરિયર સિગ્નલ \\
\midrule\noalign{}
\endhead
\bottomrule\noalign{}
\endlastfoot
અમ્પ્લિટુડ & 3 V & 10 V \\
ફ્રિક્વન્સી & 1 kHz & 30 kHz \\
\end{longtable}
}

\textbf{ગણતરી:}

\begin{itemize}
\tightlist
\item
  \textbf{મોડ્યુલેશન ઇન્ડેક્સ (m)} = Am/Ac = 3/10 = 0.3
\item
  \textbf{સાઇડબેન્ડ ફ્રિક્વન્સી} = fc \pm fm = 30 \pm 1 = 29 kHz અને 31 kHz
\item
  \textbf{સાઇડબેન્ડ અમ્પ્લિટુડ} = m \times Ac/2 = 0.3 \times 10/2 = 1.5 V
\end{itemize}

\textbf{આકૃતિ: AM સ્પેક્ટ્રમ}

\begin{lstlisting}
                                ┌───┐
                                │   │
                                │   │ 10V
                                │   │
                                │   │
                                │   │
     ┌───┐                      │   │                      ┌───┐
     │   │                      │   │                      │   │
     │   │ 1.5V                 │   │                      │   │ 1.5V
     │   │                      │   │                      │   │
     │   │                      │   │                      │   │
─────┴───┴──────────────────────┴───┴──────────────────────┴───┴─────────▶ f
            29kHz                30kHz                31kHz
         (fc - fm)                 fc               (fc + fm)
\end{lstlisting}

\end{solutionbox}
\begin{mnemonicbox}
``LSB-C-USB: લોઅર સાઇડબેન્ડ, કેરિયર, અપર સાઇડબેન્ડ
29-30-31 પર''

\end{mnemonicbox}
\subsection*{પ્રશ્ન 1(ક) OR [7
ગુણ]}\label{uxaaauxab0uxab6uxaa8-1uxa95-or-7-uxa97uxaa3}

\textbf{કેરિયર પાવર અને મોડુલેટેડ સિગ્નલ પાવરના મેથેમેટિકલ ઇક્વેશન તારવો.}

\begin{solutionbox}

\textbf{મેથેમેટિકલ રિલેશન:}

\begin{itemize}
\tightlist
\item
  કેરિયર સિગ્નલ: c(t) = Ac cos(2πfc·t)
\item
  મોડ્યુલેટિંગ સિગ્નલ: m(t) = Am cos(2πfm·t)
\item
  AM સિગ્નલ: s(t) = Ac[1 + m·cos(2πfm·t)]·cos(2πfc·t)
\end{itemize}


{\def\LTcaptype{none} % do not increment counter
\vspace{-5pt}
\captionof{table}{AM માં પાવર વિતરણ}
\vspace{-10pt}
\begin{longtable}[]{@{}lll@{}}
\toprule\noalign{}
ઘટક & સૂત્ર & Pc ના સંદર્ભમાં \\
\midrule\noalign{}
\endhead
\bottomrule\noalign{}
\endlastfoot
કેરિયર પાવર (Pc) & Ac^{2}/2 & Pc \\
કુલ સાઇડબેન્ડ પાવર (Ps) & m^{2}·Ac^{2}/4 & m^{2}·Pc/2 \\
કુલ AM પાવર (Pt) & Pc(1 + m^{2}/2) & Pc(1 + m^{2}/2) \\
\end{longtable}
}

\textbf{આકૃતિ: પાવર વિતરણ}

\includegraphics[width=1\linewidth,height=\textheight,keepaspectratio]{mermaid-9fa60ebd.pdf}

\begin{itemize}
\tightlist
\item
  \textbf{મોડ્યુલેશન કાર્યક્ષમતા} = Ps/Pt = (m^{2}/2)/(1 + m^{2}/2) \times 100\%
\end{itemize}

\end{solutionbox}
\begin{mnemonicbox}
``કુલ પાવર = કેરિયર પાવર \times (1 + m^{2}/2)''

\end{mnemonicbox}
\subsection*{પ્રશ્ન 2(અ) [3
ગુણ]}\label{uxaaauxab0uxab6uxaa8-2uxa85-3-uxa97uxaa3}

\textbf{AM અને FM ની સરખામણી કરો.}

\begin{solutionbox}


{\def\LTcaptype{none} % do not increment counter
\vspace{-5pt}
\captionof{table}{AM અને FM વચ્ચે તુલના}
\vspace{-10pt}
\begin{longtable}[]{@{}lll@{}}
\toprule\noalign{}
પરિમાણ & AM & FM \\
\midrule\noalign{}
\endhead
\bottomrule\noalign{}
\endlastfoot
\textbf{મોડ્યુલેશન પરિમાણ} & અમ્પ્લિટુડ બદલાય છે & ફ્રિક્વન્સી બદલાય છે \\
\textbf{બેન્ડવિડ્થ} & 2 \times fm & 2 \times (Δf + fm) \\
\textbf{નોઇઝ ઇમ્યુનિટી} & નબળી & ઉત્તમ \\
\textbf{પાવર કાર્યક્ષમતા} & નીચી & ઉંચી \\
\textbf{સર્કિટ જટિલતા} & સરળ & જટિલ \\
\end{longtable}
}

\end{solutionbox}
\begin{mnemonicbox}
``ABNPC: અમ્પ્લિટુડ/બેન્ડવિડ્થ/નોઇઝ/પાવર/જટિલતા તફાવત''

\end{mnemonicbox}
\subsection*{પ્રશ્ન 2(બ) [4
ગુણ]}\label{uxaaauxab0uxab6uxaa8-2uxaac-4-uxa97uxaa3}

\textbf{સર્કિટ ડાયાગ્રામની મદદથી એન્વલેપ ડિટેક્ટરને સમજાવો.}

\begin{solutionbox}

\textbf{આકૃતિ: એન્વલેપ ડિટેક્ટર સર્કિટ}

\begin{lstlisting}
    ┌─────┐     D     ┌───┬───┐
    │     │     ▶|    │   │   │
AM  │     ├────────┬──┤   │   │  Demodulated
Inpt│     │        │  │   │  ┌┴┐ Output
    │     │        │  │   │  │R│
    └─────┘        │  │   │  │L│
                   │  │   │  └┬┘
                   │  │ C │   │
                   │  │   │   │
                   └──┴───┴───┘
\end{lstlisting}


{\def\LTcaptype{none} % do not increment counter
\vspace{-5pt}
\captionof{table}{એન્વલેપ ડિટેક્ટર ઘટકો}
\vspace{-10pt}
\begin{longtable}[]{@{}ll@{}}
\toprule\noalign{}
ઘટક & કાર્ય \\
\midrule\noalign{}
\endhead
\bottomrule\noalign{}
\endlastfoot
\textbf{ડાયોડ (D)} & AM સિગ્નલને રેક્ટિફાય કરે છે અને પોઝિટિવ હાફ સાયકલ મેળવે
છે \\
\textbf{કેપેસિટર (C)} & ઇનપુટના પીક સુધી ચાર્જ થાય છે, પીક વચ્ચે ચાર્જ જાળવી રાખે
છે \\
\textbf{રેઝિસ્ટર (RL)} & એન્વેલોપ એક્સટ્રેક્શન માટે યોગ્ય દરે કેપેસિટરને ડિસ્ચાર્જ કરે
છે \\
\end{longtable}
}

\textbf{ટાઈમ કોન્સ્ટન્ટ સિલેક્શન:}

\begin{itemize}
\tightlist
\item
  1/fm \textless\textless{} RC \textless\textless{} 1/fc (યોગ્ય એન્વેલોપ
  ડિટેક્શન માટે)
\end{itemize}

\end{solutionbox}
\begin{mnemonicbox}
``DCR: ડાયોડ રેક્ટિફાય કરે છે, કેપેસિટર ચાર્જ થાય છે, રેઝિસ્ટર
ડિસ્ચાર્જ કરે છે''

\end{mnemonicbox}
\subsection*{પ્રશ્ન 2(ક) [7
ગુણ]}\label{uxaaauxab0uxab6uxaa8-2uxa95-7-uxa97uxaa3}

\textbf{સુપરહીટરોડાઈન રીસીવરનો બ્લોક ડાયાગ્રામ દોરો અને સમજાવો.}

\begin{solutionbox}

\textbf{આકૃતિ: સુપરહીટરોડાઈન રીસીવર}

\includegraphics[width=1\linewidth,height=\textheight,keepaspectratio]{mermaid-c7df81f3.pdf}


{\def\LTcaptype{none} % do not increment counter
\vspace{-5pt}
\captionof{table}{સુપરહીટરોડાઈન રીસીવર બ્લોક્સના કાર્યો}
\vspace{-10pt}
\begin{longtable}[]{@{}
  >{\raggedright\arraybackslash}p{(\linewidth - 2\tabcolsep) * \real{0.4118}}
  >{\raggedright\arraybackslash}p{(\linewidth - 2\tabcolsep) * \real{0.5882}}@{}}
\toprule\noalign{}
\begin{minipage}[b]{\linewidth}\raggedright
બ્લોક
\end{minipage} & \begin{minipage}[b]{\linewidth}\raggedright
કાર્ય
\end{minipage} \\
\midrule\noalign{}
\endhead
\bottomrule\noalign{}
\endlastfoot
\textbf{RF એમ્પ્લિફાયર} & નબળા RF સિગ્નલને એમ્પ્લિફાય કરે છે, સિલેક્ટિવિટી પ્રદાન
કરે છે, ઇમેજ ફ્રિક્વન્સીને રદ કરે છે \\
\textbf{લોકલ ઓસિલેટર} & મિક્સિંગ માટે ફ્રિક્વન્સી fo = fRF + fIF ઉત્પન્ન કરે છે \\
\textbf{મિક્સર} & IF (ઇન્ટરમીડિયેટ ફ્રિક્વન્સી) બનાવવા માટે RF સિગ્નલને લોકલ
ઓસિલેટર સાથે જોડે છે \\
\textbf{IF એમ્પ્લિફાયર} & ફિક્સ્ડ ફ્રિક્વન્સી પર મોટાભાગના રિસીવર ગેઇન અને
સિલેક્ટિવિટી પ્રદાન કરે છે \\
\textbf{ડિટેક્ટર} & IF સિગ્નલમાંથી મોડ્યુલેટિંગ સિગ્નલ એક્સટ્રેક્ટ કરે છે \\
\textbf{AF એમ્પ્લિફાયર} & સ્પીકર ચલાવવા માટે રિકવર થયેલ ઓડિયોને એમ્પ્લિફાય કરે
છે \\
\end{longtable}
}

\end{solutionbox}
\begin{mnemonicbox}
``RLMIDS: RF, લોકલ ઓસિલેટર, મિક્સર, IF, ડિટેક્ટર,
સ્પીકર''

\end{mnemonicbox}
\subsection*{પ્રશ્ન 2(અ) OR [3
ગુણ]}\label{uxaaauxab0uxab6uxaa8-2uxa85-or-3-uxa97uxaa3}

\textbf{નીચેના શબ્દો વ્યાખ્યાયિત કરો: (અ) Sensitivity અને (બ) Selectivity}

\begin{solutionbox}


{\def\LTcaptype{none} % do not increment counter
\vspace{-5pt}
\captionof{table}{રિસીવર લક્ષણો}
\vspace{-10pt}
\begin{longtable}[]{@{}
  >{\raggedright\arraybackslash}p{(\linewidth - 2\tabcolsep) * \real{0.3333}}
  >{\raggedright\arraybackslash}p{(\linewidth - 2\tabcolsep) * \real{0.6667}}@{}}
\toprule\noalign{}
\begin{minipage}[b]{\linewidth}\raggedright
શબ્દ
\end{minipage} & \begin{minipage}[b]{\linewidth}\raggedright
વ્યાખ્યા
\end{minipage} \\
\midrule\noalign{}
\endhead
\bottomrule\noalign{}
\endlastfoot
\textbf{Sensitivity} & નબળા સિગ્નલોને શોધવા અને એમ્પ્લિફાય કરવાની રિસીવરની
ક્ષમતા; સ્ટાન્ડર્ડ આઉટપુટ માટે જરૂરી ન્યૂનતમ ઇનપુટ સિગ્નલ સ્ટ્રેન્થ (µV) તરીકે માપવામાં
આવે છે \\
\textbf{Selectivity} & અડીન ચેનલોથી ઇચ્છિત સિગ્નલને અલગ કરવાની રિસીવરની
ક્ષમતા; રેસોનન્ટ ફ્રિક્વન્સી પર રિસ્પોન્સના ઓફ-રેસોનન્ટ ફ્રિક્વન્સી પર રિસ્પોન્સના
ગુણોત્તર તરીકે માપવામાં આવે છે \\
\end{longtable}
}

\textbf{આકૃતિ: સિલેક્ટિવિટી કર્વ}

\begin{lstlisting}
    │     ▲
    │     │રિસ્પોન્સ
    │     │
    │     │      ┌───┐
    │     │      │   │
    │     │      │   │
    │     │      │   │
    │     │   ┌──┘   └──┐
    │     │ ┌─┘         └─┐
    │     └─┘             └─┐
    └─────────────────────────▶
          f1   fc    f2    ફ્રિક્વન્સી
\end{lstlisting}

\end{solutionbox}
\begin{mnemonicbox}
``SS: સિગ્નલ સ્ટ્રેન્થ ફોર સેન્સિટિવિટી, સિગ્નલ સેપરેશન ફોર
સિલેક્ટિવિટી''

\end{mnemonicbox}
\subsection*{પ્રશ્ન 2(બ) OR [4
ગુણ]}\label{uxaaauxab0uxab6uxaa8-2uxaac-or-4-uxa97uxaa3}

\textbf{જનરલ કમ્યુનિકેશનના બ્લોક ડાયાગ્રામનું વર્ણન કરો}

\begin{solutionbox}

\textbf{આકૃતિ: જનરલ કમ્યુનિકેશન સિસ્ટમ}

\includegraphics[width=1\linewidth,height=\textheight,keepaspectratio]{mermaid-f9f80b6e.pdf}


{\def\LTcaptype{none} % do not increment counter
\vspace{-5pt}
\captionof{table}{કમ્યુનિકેશન સિસ્ટમના ઘટકો}
\vspace{-10pt}
\begin{longtable}[]{@{}
  >{\raggedright\arraybackslash}p{(\linewidth - 2\tabcolsep) * \real{0.5238}}
  >{\raggedright\arraybackslash}p{(\linewidth - 2\tabcolsep) * \real{0.4762}}@{}}
\toprule\noalign{}
\begin{minipage}[b]{\linewidth}\raggedright
ઘટક
\end{minipage} & \begin{minipage}[b]{\linewidth}\raggedright
કાર્ય
\end{minipage} \\
\midrule\noalign{}
\endhead
\bottomrule\noalign{}
\endlastfoot
\textbf{ઇન્ફોર્મેશન સોર્સ} & કમ્યુનિકેટ કરવા માટેનો સંદેશ ઉત્પન્ન કરે છે (વૉઇસ, ડેટા,
વિડિઓ) \\
\textbf{ટ્રાન્સમીટર} & સંદેશને ટ્રાન્સમિશન માટે યોગ્ય સિગ્નલમાં રૂપાંતરિત કરે છે \\
\textbf{ચેનલ} & જેના દ્વારા સિગ્નલ પસાર થાય છે તે માધ્યમ (વાયર, ફાઇબર, હવા) \\
\textbf{રિસીવર} & મળેલા સિગ્નલમાંથી મૂળ સંદેશ એક્સટ્રેક્ટ કરે છે \\
\textbf{ડેસ્ટિનેશન} & જેના માટે સંદેશ અભિપ્રેત છે તે એન્ટિટી \\
\textbf{નોઇઝ સોર્સ} & અવાંછિત સિગ્નલો જે સંદેશમાં દખલ કરે છે \\
\end{longtable}
}

\end{solutionbox}
\begin{mnemonicbox}
``I-T-C-R-D: ઇન્ફોર્મેશન ટ્રાવેલ્સ કેરફુલી, રીચેસ ડેસ્ટિનેશન''

\end{mnemonicbox}
\subsection*{પ્રશ્ન 2(ક) OR [7
ગુણ]}\label{uxaaauxab0uxab6uxaa8-2uxa95-or-7-uxa97uxaa3}

\textbf{સુપરહીટરોડાઈન FM રીસીવરનો બ્લોક ડાયાગ્રામ દોરો અને સમજાવો.}

\begin{solutionbox}

\textbf{આકૃતિ: સુપરહીટરોડાઈન FM રીસીવર}

\includegraphics[width=1\linewidth,height=\textheight,keepaspectratio]{mermaid-4a5b82b8.pdf}


{\def\LTcaptype{none} % do not increment counter
\vspace{-5pt}
\captionof{table}{FM રિસીવરમાં વધારાના ઘટકો}
\vspace{-10pt}
\begin{longtable}[]{@{}
  >{\raggedright\arraybackslash}p{(\linewidth - 2\tabcolsep) * \real{0.5238}}
  >{\raggedright\arraybackslash}p{(\linewidth - 2\tabcolsep) * \real{0.4762}}@{}}
\toprule\noalign{}
\begin{minipage}[b]{\linewidth}\raggedright
ઘટક
\end{minipage} & \begin{minipage}[b]{\linewidth}\raggedright
કાર્ય
\end{minipage} \\
\midrule\noalign{}
\endhead
\bottomrule\noalign{}
\endlastfoot
\textbf{લિમિટર} & અમ્પ્લિટુડ વેરિએશન્સ દૂર કરે છે, સ્થિર અમ્પ્લિટુડ સિગ્નલ પ્રદાન કરે
છે \\
\textbf{FM ડિસ્ક્રિમિનેટર} & ફ્રિક્વન્સી વેરિએશન્સને અમ્પ્લિટુડ વેરિએશન્સમાં રૂપાંતરિત કરે
છે (ડિમોડ્યુલેશન) \\
\textbf{ડી-એમ્ફેસિસ} & ટ્રાન્સમીટર પર બૂસ્ટ થયેલ ઉચ્ચ ફ્રિક્વન્સીને ઘટાડે છે \\
\end{longtable}
}

\textbf{FM રિસીવરની વિશિષ્ટ બાબતો:}

\begin{itemize}
\tightlist
\item
  વધુ પહોળી બેન્ડવિડ્થ IF એમ્પ્લિફાયર (AM માટે 10 kHz ની સરખામણીમાં 200 kHz)
  વાપરે છે
\item
  નોઇઝ ઘટાડવા માટે લિમિટર સ્ટેજની જરૂર પડે છે
\item
  FM ડિમોડ્યુલેશન માટે વિશિષ્ટ ડિસ્ક્રિમિનેટર વાપરે છે
\end{itemize}

\end{solutionbox}
\begin{mnemonicbox}
``MILD: મિક્સર, IF, લિમિટર, ડિસ્ક્રિમિનેટર - FM રિસેપ્શનમાં
મુખ્ય ઘટકો''

\end{mnemonicbox}
\subsection*{પ્રશ્ન 3(અ) [3
ગુણ]}\label{uxaaauxab0uxab6uxaa8-3uxa85-3-uxa97uxaa3}

\textbf{વેવફોર્મ ટાઈમ અને ફ્રિક્વન્સી ડોમેન માં દોરો (અ) Impulse અને (બ) Pulse}

\begin{solutionbox}


{\def\LTcaptype{none} % do not increment counter
\vspace{-5pt}
\captionof{table}{Impulse અને Pulse લક્ષણો}
\vspace{-10pt}
\begin{longtable}[]{@{}
  >{\raggedright\arraybackslash}p{(\linewidth - 4\tabcolsep) * \real{0.2051}}
  >{\raggedright\arraybackslash}p{(\linewidth - 4\tabcolsep) * \real{0.3333}}
  >{\raggedright\arraybackslash}p{(\linewidth - 4\tabcolsep) * \real{0.4615}}@{}}
\toprule\noalign{}
\begin{minipage}[b]{\linewidth}\raggedright
સિગ્નલ
\end{minipage} & \begin{minipage}[b]{\linewidth}\raggedright
ટાઈમ ડોમેન
\end{minipage} & \begin{minipage}[b]{\linewidth}\raggedright
ફ્રિક્વન્સી ડોમેન
\end{minipage} \\
\midrule\noalign{}
\endhead
\bottomrule\noalign{}
\endlastfoot
\textbf{Impulse} & અનંત સાંકડો સ્પાઇક અનંત અમ્પ્લિટુડ સાથે & ફ્લેટ સ્પેક્ટ્રમ જેમાં બધી
ફ્રિક્વન્સી સમાન રીતે હાજર હોય \\
\textbf{Pulse} & આયતાકાર આકાર સાથે મર્યાદિત પહોળાઈ અને ઊંચાઈ & Sinc ફંક્શન
(sin(x)/x) આકાર \\
\end{longtable}
}

\textbf{આકૃતિ: Impulse અને Pulse}

\begin{lstlisting}
ટાઈમ ડોમેન                       ફ્રિક્વન્સી ડોમેન
     
Impulse                          Impulse
    │                                │
    │                                │
    │ ↑                              │───────────────
    │ │                              │
    └─┼─────────▶                    └────────────────▶
      t_{0}                               f

Pulse                            Pulse
    │                                │
    │  ┌───────┐                     │    ┌─┐
    │  │       │                     │    │ │
    │  │       │                     │  ┌─┘ └─┐  ┌─┐
    └──┴───────┴────▶                └──┴─────┴──┴─┴───▶
       t_{0}  t_{0}+T                         f_{0}  2f_{0}  3f_{0}
\end{lstlisting}

\end{solutionbox}
\begin{mnemonicbox}
``I-P: Impulse એ Pinpoint સ્પાઇક છે, Pulse ને
Persistent પહોળાઈ છે''

\end{mnemonicbox}
\subsection*{પ્રશ્ન 3(બ) [4
ગુણ]}\label{uxaaauxab0uxab6uxaa8-3uxaac-4-uxa97uxaa3}

\textbf{અંડર સેમ્પલિંગ અને ક્રિટિકલ સેમ્પલિંગનું વર્ણન કરો}

\begin{solutionbox}


{\def\LTcaptype{none} % do not increment counter
\vspace{-5pt}
\captionof{table}{સેમ્પલિંગના પ્રકારો}
\vspace{-10pt}
\begin{longtable}[]{@{}
  >{\raggedright\arraybackslash}p{(\linewidth - 4\tabcolsep) * \real{0.4615}}
  >{\raggedright\arraybackslash}p{(\linewidth - 4\tabcolsep) * \real{0.3333}}
  >{\raggedright\arraybackslash}p{(\linewidth - 4\tabcolsep) * \real{0.2051}}@{}}
\toprule\noalign{}
\begin{minipage}[b]{\linewidth}\raggedright
સેમ્પલિંગનો પ્રકાર
\end{minipage} & \begin{minipage}[b]{\linewidth}\raggedright
વર્ણન
\end{minipage} & \begin{minipage}[b]{\linewidth}\raggedright
અસર
\end{minipage} \\
\midrule\noalign{}
\endhead
\bottomrule\noalign{}
\endlastfoot
\textbf{અંડર સેમ્પલિંગ} & સેમ્પલિંગ ફ્રિક્વન્સી fs \textless{} 2fm (નાયક્વિસ્ટ રેટ
કરતાં ઓછી) & એલિયાસિંગ થાય છે; સિગ્નલ પુનઃપ્રાપ્ત કરી શકાતો નથી \\
\textbf{ક્રિટિકલ સેમ્પલિંગ} & સેમ્પલિંગ ફ્રિક્વન્સી fs = 2fm (ચોક્કસ નાયક્વિસ્ટ રેટ) &
સૈદ્ધાંતિક રીતે સંપૂર્ણ પુનર્નિર્માણ શક્ય છે \\
\textbf{ઓવર સેમ્પલિંગ} & સેમ્પલિંગ ફ્રિક્વન્સી fs \textgreater{} 2fm (નાયક્વિસ્ટ રેટ
કરતાં વધારે) & વધુ સારું પુનર્નિર્માણ, સરળ ફિલ્ટરિંગ \\
\end{longtable}
}

\textbf{આકૃતિ: અંડર સેમ્પલિંગ vs ક્રિટિકલ સેમ્પલિંગ}

\begin{lstlisting}
અંડર સેમ્પલિંગ (fs < 2fm)
    │     ┌───┐     ┌───┐     ┌───┐     ┌───┐
    │     │   │     │   │     │   │     │   │
    │─────┘   └─────┘   └─────┘   └─────┘   └────▶
    ↑     ↑     ↑     ↑     ↑
    એલિયાસિંગ થાય છે - સેમ્પલ્સ ખૂબ દૂર છે

ક્રિટિકલ સેમ્પલિંગ (fs = 2fm)
    │     ┌───┐     ┌───┐     ┌───┐     ┌───┐
    │     │   │     │   │     │   │     │   │
    │─────┘   └─────┘   └─────┘   └─────┘   └────▶
    ↑   ↑   ↑   ↑   ↑   ↑   ↑   ↑
    પુનર્નિર્માણ માટે પૂરતા સેમ્પલ્સ છે
\end{lstlisting}

\end{solutionbox}
\begin{mnemonicbox}
``UCO: અંડર (fs\textless2fm), ક્રિટિકલ (fs=2fm), ઓવર
(fs\textgreater2fm)''

\end{mnemonicbox}
\subsection*{પ્રશ્ન 3(ક) [7
ગુણ]}\label{uxaaauxab0uxab6uxaa8-3uxa95-7-uxa97uxaa3}

\textbf{PAM, PWM અને PPM સિગ્નલોને વેવફોર્મ સાથે જણાવો.}

\begin{solutionbox}


{\def\LTcaptype{none} % do not increment counter
\vspace{-5pt}
\captionof{table}{પલ્સ મોડ્યુલેશન ટેકનિક્સ}
\vspace{-10pt}
\begin{longtable}[]{@{}
  >{\raggedright\arraybackslash}p{(\linewidth - 4\tabcolsep) * \real{0.2292}}
  >{\raggedright\arraybackslash}p{(\linewidth - 4\tabcolsep) * \real{0.2708}}
  >{\raggedright\arraybackslash}p{(\linewidth - 4\tabcolsep) * \real{0.5000}}@{}}
\toprule\noalign{}
\begin{minipage}[b]{\linewidth}\raggedright
ટેકનિક
\end{minipage} & \begin{minipage}[b]{\linewidth}\raggedright
વર્ણન
\end{minipage} & \begin{minipage}[b]{\linewidth}\raggedright
સિગ્નલનું બદલાતું પરિમાણ
\end{minipage} \\
\midrule\noalign{}
\endhead
\bottomrule\noalign{}
\endlastfoot
\textbf{PAM (પલ્સ અમ્પ્લિટુડ મોડ્યુલેશન)} & પલ્સનું અમ્પ્લિટુડ મોડ્યુલેટિંગ સિગ્નલ અનુસાર
બદલાય છે & અમ્પ્લિટુડ \\
\textbf{PWM (પલ્સ વિડ્થ મોડ્યુલેશન)} & પલ્સની પહોળાઈ/અવધિ મોડ્યુલેટિંગ સિગ્નલ અનુસાર
બદલાય છે & પલ્સ પહોળાઈ \\
\textbf{PPM (પલ્સ પોઝિશન મોડ્યુલેશન)} & પલ્સની સ્થિતિ/સમય મોડ્યુલેટિંગ સિગ્નલ અનુસાર
બદલાય છે & પલ્સ સ્થિતિ \\
\end{longtable}
}

\textbf{આકૃતિ: PAM, PWM, PPM વેવફોર્મ્સ}

\begin{lstlisting}
મોડ્યુલેટિંગ સિગ્નલ
    │    ┌───┐
    │   /     \
    │  /       \
    │ /         \        /\
    │/           \      /  \
    │             \    /    \
    │              \  /      \
    └───────────────\/────────────▶

PAM
    │    ┌─┐   ┌┐  ┌┐   ┌─┐
    │    │ │   ││  ││   │ │
    │    │ │   ││  ││   │ │
    │    │ │   ││  ││   │ │
    └────┘ └───┘└──┘└───┘ └────▶

PWM
    │    ┌───┐ ┌─┐ ┌┐  ┌──┐
    │    │   │ │ │ ││  │  │
    │    │   │ │ │ ││  │  │
    │    │   │ │ │ ││  │  │
    └────┘   └─┘ └─┘└──┘  └────▶

PPM
    │    ┌┐    ┌┐   ┌┐    ┌┐
    │    ││    ││   ││    ││
    │    ││    ││   ││    ││
    │    ││    ││   ││    ││
    └────┘└────┘└───┘└────┘└────▶
\end{lstlisting}

\end{solutionbox}
\begin{mnemonicbox}
``APP: અમ્પ્લિટુડ, પોઝિશન, પલ્સ-વિડ્થ અનુક્રમે બદલાય છે''

\end{mnemonicbox}
\subsection*{પ્રશ્ન 3(અ) OR [3
ગુણ]}\label{uxaaauxab0uxab6uxaa8-3uxa85-or-3-uxa97uxaa3}

\textbf{સેમ્પલિંગ થીયરમ જણાવો અને સમજાવો.}

\begin{solutionbox}

\textbf{સેમ્પલિંગ થીયરમ સ્ટેટમેન્ટ:} ``બેન્ડ-લિમિટેડ કન્ટિન્યુઅસ-ટાઈમ સિગ્નલને તેના
સેમ્પલ્સ દ્વારા સંપૂર્ણપણે રજૂ કરી શકાય છે અને પુનઃપ્રાપ્ત કરી શકાય છે, જો સેમ્પલિંગ
ફ્રિક્વન્સી સિગ્નલમાં ઉચ્ચતમ ફ્રિક્વન્સી ઘટકના ઓછામાં ઓછી બે ગણી હોય.''


{\def\LTcaptype{none} % do not increment counter
\vspace{-5pt}
\captionof{table}{સેમ્પલિંગ થીયરમના મુખ્ય તત્વો}
\vspace{-10pt}
\begin{longtable}[]{@{}ll@{}}
\toprule\noalign{}
શબ્દ & વર્ણન \\
\midrule\noalign{}
\endhead
\bottomrule\noalign{}
\endlastfoot
\textbf{નાયક્વિસ્ટ રેટ} & જરૂરી ન્યૂનતમ સેમ્પલિંગ ફ્રિક્વન્સી (fs) = 2fm \\
\textbf{નાયક્વિસ્ટ ઇન્ટરવલ} & સેમ્પલ્સ વચ્ચેનો મહત્તમ સમય = 1/(2fm) \\
\textbf{બેન્ડ-લિમિટેડ સિગ્નલ} & મર્યાદિત ઉચ્ચતમ ફ્રિક્વન્સી ઘટક ધરાવતું સિગ્નલ \\
\end{longtable}
}

\textbf{આકૃતિ: યોગ્ય સેમ્પલિંગ}

\begin{lstlisting}
મૂળ સિગ્નલ
    │   ┌───┐
    │  /     \
    │ /       \
    │/         \
    │           \
    │            \
    └─────────────────▶

fs \geq 2fm પર સેમ્પલ કરેલ
    │   *   *
    │  /|\  |\
    │ / | \ | \
    │/  |  \|  \
    │   |   *   *
    │   |       |
    └───*───────*───▶
\end{lstlisting}

\end{solutionbox}
\begin{mnemonicbox}
``2F: ફ્રિક્વન્સીને તેની ઉચ્ચતમ ફ્રિક્વન્સીના ઓછામાં ઓછા બે ગણા
પર સેમ્પલ કરવી જોઈએ''

\end{mnemonicbox}
\subsection*{પ્રશ્ન 3(બ) OR [4
ગુણ]}\label{uxaaauxab0uxab6uxaa8-3uxaac-or-4-uxa97uxaa3}

\textbf{કોન્ટાઇજેશન સમજાવો.}

\begin{solutionbox}


{\def\LTcaptype{none} % do not increment counter
\vspace{-5pt}
\captionof{table}{ક્વોન્ટાઈઝેશન કોન્સેપ્ટ્સ}
\vspace{-10pt}
\begin{longtable}[]{@{}
  >{\raggedright\arraybackslash}p{(\linewidth - 2\tabcolsep) * \real{0.3158}}
  >{\raggedright\arraybackslash}p{(\linewidth - 2\tabcolsep) * \real{0.6842}}@{}}
\toprule\noalign{}
\begin{minipage}[b]{\linewidth}\raggedright
શબ્દ
\end{minipage} & \begin{minipage}[b]{\linewidth}\raggedright
વર્ણન
\end{minipage} \\
\midrule\noalign{}
\endhead
\bottomrule\noalign{}
\endlastfoot
\textbf{ક્વોન્ટાઈઝેશન} & સતત અમ્પ્લિટુડ મૂલ્યોને ડિસ્ક્રીટ લેવલ્સમાં રૂપાંતરિત કરવાની
પ્રક્રિયા \\
\textbf{ક્વોન્ટાઈઝેશન લેવલ્સ} & ઉપયોગમાં લેવાતા ડિસ્ક્રીટ મૂલ્યોની કુલ સંખ્યા (સામાન્ય
રીતે 2^{n}) \\
\textbf{ક્વોન્ટાઈઝેશન સ્ટેપ સાઈઝ} & નજીકના લેવલ્સ વચ્ચેનો વોલ્ટેજ તફાવત (Q =
Vmax/2^{n}) \\
\textbf{ક્વોન્ટાઈઝેશન એરર} & વાસ્તવિક સિગ્નલ મૂલ્ય અને ક્વોન્ટાઈઝ્ડ મૂલ્ય વચ્ચેનો
તફાવત \\
\end{longtable}
}

\textbf{આકૃતિ: ક્વોન્ટાઈઝેશન પ્રક્રિયા}

\begin{lstlisting}
કન્ટિન્યુઅસ સિગ્નલ           ક્વોન્ટાઈઝ્ડ સિગ્નલ
    │                           │       
    │   /\                      │   ┌─┐  
    │  /  \                     │   │ │  
    │ /    \      ───────▶      │┌──┘ └──┐
    │/      \                   ││       │
    │        \                  ││       └──┐
    │         \                 ││          │
    └──────────────▶            └───────────────▶
                               ક્વોન્ટાઈઝેશન
                                  લેવલ્સ
\end{lstlisting}

\end{solutionbox}
\begin{mnemonicbox}
``LSED: લેવલ્સ, સ્ટેપ સાઈઝ, એરર, ડિસ્ક્રીટ વેલ્યુ''

\end{mnemonicbox}
\subsection*{પ્રશ્ન 3(ક) OR [7
ગુણ]}\label{uxaaauxab0uxab6uxaa8-3uxa95-or-7-uxa97uxaa3}

\textbf{કમ્પાન્ડિંગને વિગતવાર સમજાવો.}

\begin{solutionbox}


{\def\LTcaptype{none} % do not increment counter
\vspace{-5pt}
\captionof{table}{કમ્પાન્ડિંગ કોન્સેપ્ટ્સ}
\vspace{-10pt}
\begin{longtable}[]{@{}ll@{}}
\toprule\noalign{}
શબ્દ & વર્ણન \\
\midrule\noalign{}
\endhead
\bottomrule\noalign{}
\endlastfoot
\textbf{કમ્પાન્ડિંગ} & COMપ્રેસિંગ + exPANDિંગ; નોન-લિનિયર ક્વોન્ટાઈઝેશન ટેકનિક \\
\textbf{કમ્પ્રેશન} & ટ્રાન્સમિશન પહેલા સિગ્નલની અમ્પ્લિટુડ રેન્જ ઘટાડે છે \\
\textbf{એક્સપાન્શન} & રિસીવર પર મૂળ અમ્પ્લિટુડ રેન્જ પુનઃસ્થાપિત કરે છે \\
\textbf{હેતુ} & ડાયનેમિક રેન્જ જાળવી રાખતી વખતે નબળા સિગ્નલ માટે SNR સુધારે છે \\
\textbf{પ્રકારો} & μ-law (ઉત્તર અમેરિકા, જાપાન), A-law (યુરોપ) \\
\end{longtable}
}

\textbf{આકૃતિ: કમ્પાન્ડિંગ પ્રક્રિયા}

\includegraphics[width=1\linewidth,height=\textheight,keepaspectratio]{mermaid-355817f5.pdf}

\textbf{કમ્પાન્ડિંગ લો:}

\begin{itemize}
\tightlist
\item
  \textbf{μ-law}: y = sgn(x) \times ln(1+μ\textbar x\textbar)/ln(1+μ) જ્યાં μ =
  255 USA માં
\item
  \textbf{A-law}: y = sgn(x) \times A\textbar x\textbar/(1+ln(A)) જ્યારે
  \textbar x\textbar{} \textless{} 1/A y = sgn(x) \times
  (1+ln(A\textbar x\textbar))/(1+ln(A)) જ્યારે 1/A \leq \textbar x\textbar{}
  \leq 1
\end{itemize}

\end{solutionbox}
\begin{mnemonicbox}
``CEQS: કમ્પ્રેસ, એનકોડ, ક્વોન્ટાઈઝ, સેન્ડ; પછી ડિકોડ,
એક્સપાન્ડ, રિકવર''

\end{mnemonicbox}
\subsection*{પ્રશ્ન 4(અ) [3
ગુણ]}\label{uxaaauxab0uxab6uxaa8-4uxa85-3-uxa97uxaa3}

\textbf{ડેલ્ટા મોડ્યુલેશન સમજાવો}

\begin{solutionbox}


{\def\LTcaptype{none} % do not increment counter
\vspace{-5pt}
\captionof{table}{ડેલ્ટા મોડ્યુલેશન કોન્સેપ્ટ્સ}
\vspace{-10pt}
\begin{longtable}[]{@{}
  >{\raggedright\arraybackslash}p{(\linewidth - 2\tabcolsep) * \real{0.4091}}
  >{\raggedright\arraybackslash}p{(\linewidth - 2\tabcolsep) * \real{0.5909}}@{}}
\toprule\noalign{}
\begin{minipage}[b]{\linewidth}\raggedright
કોન્સેપ્ટ
\end{minipage} & \begin{minipage}[b]{\linewidth}\raggedright
વર્ણન
\end{minipage} \\
\midrule\noalign{}
\endhead
\bottomrule\noalign{}
\endlastfoot
\textbf{ડેલ્ટા મોડ્યુલેશન} & DPCM નું સૌથી સરળ રૂપ જ્યાં ફક્ત 1-બિટ ક્વોન્ટાઈઝેશન
વાપરવામાં આવે છે \\
\textbf{સ્ટેપ સાઈઝ} & સિગ્નલને અનુમાનિત કરવામાં ફિક્સ્ડ વધારો/ઘટાડો \\
\textbf{આઉટપુટ} & બાઇનરી સ્ટ્રીમ (વધારા માટે 1, ઘટાડા માટે 0) \\
\textbf{ફાયદા} & સરળ અમલીકરણ, ઓછી બેન્ડવિડ્થ \\
\end{longtable}
}

\textbf{આકૃતિ: ડેલ્ટા મોડ્યુલેશન}

\begin{lstlisting}
ઓરિજિનલ સિગ્નલ    ડેલ્ટા મોડ્યુલેટેડ
                   એપ્રોક્સિમેશન
    │                   │
    │  /\               │    ┌┐┌┐
    │ /  \              │   ┌┘└┘└┐
    │/    \             │  ┌┘    └┐
    │      \            │ ┌┘      └┐
    │       \           │┌┘        └┐
    │        \          ││          │
    └─────────────▶     └───────────────▶
                       
બાઇનરી આઉટપુટ: 1 1 1 1 0 0 0 0 0 0
\end{lstlisting}

\end{solutionbox}
\begin{mnemonicbox}
``1B1S: 1-બિટ, 1-સ્ટેપ ટ્રેકિંગ''

\end{mnemonicbox}
\subsection*{પ્રશ્ન 4(બ) [4
ગુણ]}\label{uxaaauxab0uxab6uxaa8-4uxaac-4-uxa97uxaa3}

\textbf{PCM ના ફાયદા અને ગેરફાયદા લખો}

\begin{solutionbox}


{\def\LTcaptype{none} % do not increment counter
\vspace{-5pt}
\captionof{table}{PCM ના ફાયદા અને ગેરફાયદા}
\vspace{-10pt}
\begin{longtable}[]{@{}ll@{}}
\toprule\noalign{}
ફાયદા & ગેરફાયદા \\
\midrule\noalign{}
\endhead
\bottomrule\noalign{}
\endlastfoot
\textbf{ઉચ્ચ નોઇઝ ઇમ્યુનિટી} & \textbf{વધારે બેન્ડવિડ્થની જરૂર પડે છે} \\
\textbf{વધુ સારી સિગ્નલ ક્વોલિટી} & \textbf{જટિલ સિસ્ટમ અમલીકરણ} \\
\textbf{ડિજિટલ સિસ્ટમ સાથે સુસંગત} & \textbf{ક્વોન્ટાઈઝેશન નોઇઝ હાજર હોય છે} \\
\textbf{સુરક્ષિત ટ્રાન્સમિશન શક્ય છે} & \textbf{સિન્ક્રનાઈઝેશનની જરૂર પડે છે} \\
\textbf{મલ્ટિપ્લેક્સિંગ ક્ષમતા} & \textbf{વધુ પાવરની જરૂરિયાત} \\
\end{longtable}
}

\textbf{આકૃતિ: PCM સિસ્ટમ ઓવરવ્યુ}

\includegraphics[width=1\linewidth,height=\textheight,keepaspectratio]{mermaid-49814a33.pdf}

\end{solutionbox}
\begin{mnemonicbox}
``NCSMP: નોઇઝ ઇમ્યુનિટી, કમ્પેટિબલ વિથ ડિજિટલ, સિક્યોર,
મલ્ટિપ્લેક્સિંગ, પ્રોસેસિંગ બેનિફિટ્સ''

\end{mnemonicbox}
\subsection*{પ્રશ્ન 4(ક) [7
ગુણ]}\label{uxaaauxab0uxab6uxaa8-4uxa95-7-uxa97uxaa3}

\textbf{PCM-TDM સિસ્ટમનો બ્લોક ડાયાગ્રામ દોરો અને સમજાવો.}

\begin{solutionbox}

\textbf{આકૃતિ: PCM-TDM સિસ્ટમ}

\includegraphics[width=1\linewidth,height=\textheight,keepaspectratio]{mermaid-a2391b2f.pdf}


{\def\LTcaptype{none} % do not increment counter
\vspace{-5pt}
\captionof{table}{PCM-TDM સિસ્ટમ ઘટકો}
\vspace{-10pt}
\begin{longtable}[]{@{}
  >{\raggedright\arraybackslash}p{(\linewidth - 2\tabcolsep) * \real{0.5238}}
  >{\raggedright\arraybackslash}p{(\linewidth - 2\tabcolsep) * \real{0.4762}}@{}}
\toprule\noalign{}
\begin{minipage}[b]{\linewidth}\raggedright
ઘટક
\end{minipage} & \begin{minipage}[b]{\linewidth}\raggedright
કાર્ય
\end{minipage} \\
\midrule\noalign{}
\endhead
\bottomrule\noalign{}
\endlastfoot
\textbf{એન્ટી-એલિયાસિંગ ફિલ્ટર} & એલિયાસિંગ ટાળવા માટે સિગ્નલ બેન્ડવિડ્થને મર્યાદિત
કરે છે \\
\textbf{સેમ્પલ \& હોલ્ડ} & એનાલોગ મૂલ્ય પકડે છે અને પ્રોસેસિંગ માટે જાળવી રાખે છે \\
\textbf{મલ્ટીપ્લેક્સર} & એકલ ટાઇમ ડિવિઝન મલ્ટિપ્લેક્સ્ડ સ્ટ્રીમમાં ઘણા ઇનપુટ ચેનલો જોડે
છે \\
\textbf{ક્વોન્ટાઈઝર} & સતત સેમ્પલ્સને ડિસ્ક્રીટ મૂલ્યોમાં ફેરવે છે \\
\textbf{એનકોડર} & ક્વોન્ટાઈઝ્ડ મૂલ્યોને બાઇનરી કોડમાં રૂપાંતરિત કરે છે \\
\textbf{ફ્રેમ જનરેટર} & સિન્ક્રોનાઈઝેશન અને કંટ્રોલ બિટ્સ ઉમેરે છે \\
\textbf{ડિમલ્ટીપ્લેક્સર} & જોડાયેલા સિગ્નલને પાછા અલગ-અલગ ચેનલમાં વિભાજિત કરે છે \\
\textbf{રિકન્સ્ટ્રક્શન ફિલ્ટર} & એનાલોગ વેવફોર્મ પુનઃપ્રાપ્ત કરવા માટે ડિકોડેડ
સિગ્નલને સ્મૂધ કરે છે \\
\end{longtable}
}

\end{solutionbox}
\begin{mnemonicbox}
``SAMPLER: સેમ્પલ, એમ્પ્લિફાય, મલ્ટિપ્લેક્સ, પ્રોસેસ, લિમિટ,
એનકોડ, રિકન્સ્ટ્રક્ટ''

\end{mnemonicbox}
\subsection*{પ્રશ્ન 4(અ) OR [3
ગુણ]}\label{uxaaauxab0uxab6uxaa8-4uxa85-or-3-uxa97uxaa3}

\textbf{સ્લોપ ઓવરલોડ એરરનું વર્ણન કરો.}

\begin{solutionbox}


{\def\LTcaptype{none} % do not increment counter
\vspace{-5pt}
\captionof{table}{સ્લોપ ઓવરલોડ એરર}
\vspace{-10pt}
\begin{longtable}[]{@{}
  >{\raggedright\arraybackslash}p{(\linewidth - 2\tabcolsep) * \real{0.4091}}
  >{\raggedright\arraybackslash}p{(\linewidth - 2\tabcolsep) * \real{0.5909}}@{}}
\toprule\noalign{}
\begin{minipage}[b]{\linewidth}\raggedright
કોન્સેપ્ટ
\end{minipage} & \begin{minipage}[b]{\linewidth}\raggedright
વર્ણન
\end{minipage} \\
\midrule\noalign{}
\endhead
\bottomrule\noalign{}
\endlastfoot
\textbf{સ્લોપ ઓવરલોડ એરર} & ઇનપુટ સિગ્નલ DM સ્ટેપ સાઈઝ કરતાં ઝડપથી બદલાય ત્યારે
થતી ભૂલ \\
\textbf{કારણ} & ડેલ્ટા મોડ્યુલેશનમાં ફિક્સ્ડ સ્ટેપ સાઈઝ ઇનપુટના ઊંચા ઢાળ માટે ખૂબ નાની
હોય છે \\
\textbf{અસર} & રિકન્સ્ટ્રક્ટેડ સિગ્નલમાં ડિસ્ટોર્શન, ખાસ કરીને ઉચ્ચ ફ્રિક્વન્સી પર \\
\textbf{ઉકેલ} & એડેપ્ટિવ ડેલ્ટા મોડ્યુલેશન (વેરિએબલ સ્ટેપ સાઈઝ) \\
\end{longtable}
}

\textbf{આકૃતિ: સ્લોપ ઓવરલોડ એરર}

\begin{lstlisting}
ઓરિજિનલ સિગ્નલ vs DM એપ્રોક્સિમેશન
                
    │                  સ્લોપ ઓવરલોડ
    │                      │
    │    /│\              /│\
    │   / │ \            / │ \
    │  /  │  \    vs    /  │  \
    │ /   │   \        /┌─┐│   \
    │/    │    \      /┌┘ └┤    \
    │     │     \    /┌┘   │     \
    │     │      \  /┌┘    │      \
    └─────┴───────\/┴──────┴───────▶
          ઓરિજિનલ     DM એપ્રોક્સિમેશન
\end{lstlisting}

\end{solutionbox}
\begin{mnemonicbox}
``SOS: સિગ્નલ ઓવરટેક્સ સ્ટેપ્સ જ્યારે સ્લોપ સ્ટીપ હોય''

\end{mnemonicbox}
\subsection*{પ્રશ્ન 4(બ) OR [4
ગુણ]}\label{uxaaauxab0uxab6uxaa8-4uxaac-or-4-uxa97uxaa3}

\textbf{ડિફરન્શિયલ PCM નું ટ્રાન્સમીટર સમજાવો}

\begin{solutionbox}

\textbf{આકૃતિ: DPCM ટ્રાન્સમીટર}

\includegraphics[width=1\linewidth,height=\textheight,keepaspectratio]{mermaid-c3c1b2e0.pdf}


{\def\LTcaptype{none} % do not increment counter
\vspace{-5pt}
\captionof{table}{DPCM ટ્રાન્સમીટર ઘટકો}
\vspace{-10pt}
\begin{longtable}[]{@{}ll@{}}
\toprule\noalign{}
ઘટક & કાર્ય \\
\midrule\noalign{}
\endhead
\bottomrule\noalign{}
\endlastfoot
\textbf{સેમ્પલ \& હોલ્ડ} & નિયમિત અંતરે એનાલોગ સિગ્નલ પકડે છે \\
\textbf{ડિફરન્સ કેલ્ક્યુલેટર} & વર્તમાન સેમ્પલ અને અનુમાનિત મૂલ્ય વચ્ચે એરર ગણે છે \\
\textbf{ક્વોન્ટાઈઝર} & એરર સિગ્નલને ડિસ્ક્રીટ લેવલમાં રૂપાંતરિત કરે છે \\
\textbf{એનકોડર} & ક્વોન્ટાઈઝ્ડ મૂલ્યોને બાઇનરી કોડમાં રૂપાંતરિત કરે છે \\
\textbf{પ્રેડિક્ટર} & અગાઉના મૂલ્યોના આધારે આગામી સેમ્પલનો અંદાજ લગાવે છે \\
\textbf{ડિકોડર} & રિસીવરમાં જે હોય તે જ, ફીડબેક લૂપમાં ઉપયોગ થાય છે \\
\end{longtable}
}

\textbf{મુખ્ય ફાયદો:}

\begin{itemize}
\tightlist
\item
  ફક્ત સળંગ સેમ્પલ્સ વચ્ચેનો તફાવત ટ્રાન્સમિટ કરે છે
\item
  સ્ટાન્ડર્ડ PCM ની સરખામણીમાં બિટ રેટ ઘટાડે છે
\end{itemize}

\end{solutionbox}
\begin{mnemonicbox}
``SDQEP: સેમ્પલ, ડિફરન્સ, ક્વોન્ટાઈઝ, એનકોડ, પ્રેડિક્ટ''

\end{mnemonicbox}
\subsection*{પ્રશ્ન 4(ક) OR [7
ગુણ]}\label{uxaaauxab0uxab6uxaa8-4uxa95-or-7-uxa97uxaa3}

\textbf{વિગતવાર PCM ટ્રાન્સમીટર સમજાવો}

\begin{solutionbox}

\textbf{આકૃતિ: PCM ટ્રાન્સમીટર}

\includegraphics[width=1\linewidth,height=\textheight,keepaspectratio]{mermaid-5f846301.pdf}


{\def\LTcaptype{none} % do not increment counter
\vspace{-5pt}
\captionof{table}{PCM ટ્રાન્સમીટર ઘટકોની વિગત}
\vspace{-10pt}
\begin{longtable}[]{@{}
  >{\raggedright\arraybackslash}p{(\linewidth - 4\tabcolsep) * \real{0.2444}}
  >{\raggedright\arraybackslash}p{(\linewidth - 4\tabcolsep) * \real{0.2222}}
  >{\raggedright\arraybackslash}p{(\linewidth - 4\tabcolsep) * \real{0.5333}}@{}}
\toprule\noalign{}
\begin{minipage}[b]{\linewidth}\raggedright
ઘટક
\end{minipage} & \begin{minipage}[b]{\linewidth}\raggedright
કાર્ય
\end{minipage} & \begin{minipage}[b]{\linewidth}\raggedright
ડિઝાઇન કન્સિડરેશન્સ
\end{minipage} \\
\midrule\noalign{}
\endhead
\bottomrule\noalign{}
\endlastfoot
\textbf{એન્ટી-એલિયાસિંગ ફિલ્ટર} & ઇનપુટ બેન્ડવિડ્થને fs/2 સુધી મર્યાદિત કરે છે &
કટઓફ ફ્રિક્વન્સી \textless{} fs/2, શાર્પ રોલ-ઓફ \\
\textbf{સેમ્પલ \& હોલ્ડ} & ક્ષણિક સિગ્નલ મૂલ્ય પકડે છે & સેમ્પલિંગ રેટ \geq 2fm, અપર્ચર
ટાઈમ \textless\textless{} સેમ્પલિંગ પીરિયડ \\
\textbf{ક્વોન્ટાઈઝર} & સેમ્પલ અમ્પ્લિટ્યુડને ડિસ્ક્રીટ લેવલમાં અંદાજિત કરે છે & લેવલ્સ =
2^{n} જ્યાં n = બિટ ડેપ્થ, સામાન્ય રીતે 8-16 બિટ્સ \\
\textbf{એનકોડર} & ક્વોન્ટાઈઝ્ડ મૂલ્યોને ડિજિટલ કોડમાં રૂપાંતરિત કરે છે & NRZ, RZ,
મેનચેસ્ટર જેવા કોડિંગ સ્કીમ્સ વાપરે છે \\
\textbf{લાઈન કોડર} & ટ્રાન્સમિશન માટે બાઇનરી સિક્વન્સ તૈયાર કરે છે & લાંબા અંતર
માટે રિજનરેટિવ રિપીટર્સ વાપરી શકે છે \\
\end{longtable}
}

\textbf{સિગ્નલ પ્રોસેસિંગ વિગતો:}

\begin{itemize}
\tightlist
\item
  \textbf{ટાઈમ ડોમેન}: Ts = 1/fs અંતરે સેમ્પલિંગ
\item
  \textbf{અમ્પ્લિટુડ ડોમેન}: સતત અમ્પ્લિટ્યુડને 2^{n} ડિસ્ક્રીટ લેવલમાં ક્વોન્ટાઈઝિંગ
\item
  \textbf{કોડ ડોમેન}: લેવલ્સને n-બિટ બાઇનરી કોડમાં રૂપાંતરિત કરવું
\end{itemize}

\end{solutionbox}
\begin{mnemonicbox}
``SAFE-Q: સેમ્પલ એન્ડ ફિલ્ટર, ધેન એનકોડ આફ્ટર ક્વોન્ટાઈઝિંગ''

\end{mnemonicbox}
\subsection*{પ્રશ્ન 5(અ) [3
ગુણ]}\label{uxaaauxab0uxab6uxaa8-5uxa85-3-uxa97uxaa3}

\textbf{PCM અને DMની સરખામણી કરો}

\begin{solutionbox}


{\def\LTcaptype{none} % do not increment counter
\vspace{-5pt}
\captionof{table}{PCM અને DM વચ્ચે તુલના}
\vspace{-10pt}
\begin{longtable}[]{@{}lll@{}}
\toprule\noalign{}
પરિમાણ & PCM & DM \\
\midrule\noalign{}
\endhead
\bottomrule\noalign{}
\endlastfoot
\textbf{બિટ રેટ} & ઉચ્ચ (પ્રતિ સેમ્પલ ઘણા બિટ્સ) & નીચો (પ્રતિ સેમ્પલ 1 બિટ) \\
\textbf{સર્કિટ જટિલતા} & વધુ જટિલ & સરળ \\
\textbf{સિગ્નલ ક્વોલિટી} & સારી & નીચી, સ્લોપ ઓવરલોડ \& ગ્રેન્યુલર નોઇઝથી
પ્રભાવિત \\
\textbf{બેન્ડવિડ્થ} & વધુ પહોળી & સાંકડી \\
\textbf{સેમ્પલિંગ રેટ} & ઓછામાં ઓછી 2fm & 2fm કરતાં ઘણી વધારે \\
\end{longtable}
}

\end{solutionbox}
\begin{mnemonicbox}
``BCSBS: બિટ રેટ, કમ્પ્લેક્સિટી, સિગ્નલ ક્વોલિટી, બેન્ડવિડ્થ,
સેમ્પલિંગ''

\end{mnemonicbox}
\subsection*{પ્રશ્ન 5(બ) [4
ગુણ]}\label{uxaaauxab0uxab6uxaa8-5uxaac-4-uxa97uxaa3}

\textbf{વ્યાખ્યા આપો: (અ) Antenna (બ) Radiation pattern (ક) Directivity અને
(ડ) Polarization}

\begin{solutionbox}


{\def\LTcaptype{none} % do not increment counter
\vspace{-5pt}
\captionof{table}{એન્ટેના શબ્દાવલી}
\vspace{-10pt}
\begin{longtable}[]{@{}
  >{\raggedright\arraybackslash}p{(\linewidth - 2\tabcolsep) * \real{0.3333}}
  >{\raggedright\arraybackslash}p{(\linewidth - 2\tabcolsep) * \real{0.6667}}@{}}
\toprule\noalign{}
\begin{minipage}[b]{\linewidth}\raggedright
શબ્દ
\end{minipage} & \begin{minipage}[b]{\linewidth}\raggedright
વ્યાખ્યા
\end{minipage} \\
\midrule\noalign{}
\endhead
\bottomrule\noalign{}
\endlastfoot
\textbf{એન્ટેના} & ઇલેક્ટ્રિકલ સિગ્નલ્સને ઇલેક્ટ્રોમેગ્નેટિક વેવ્સમાં અને તેનાથી ઉલટું ફેરવતું
ઉપકરણ \\
\textbf{રેડિએશન પેટર્ન} & અંતરિક્ષ કોઓર્ડિનેટ્સના ફંક્શન તરીકે એન્ટેનાની રેડિએશન
પ્રોપર્ટીઝનું ગ્રાફિકલ રેપ્રેઝન્ટેશન \\
\textbf{ડિરેક્ટિવિટી} & આપેલી દિશામાં રેડિએશન ઇન્ટેન્સિટીનો સરેરાશ રેડિએશન
ઇન્ટેન્સિટી સાથેનો ગુણોત્તર \\
\textbf{પોલરાઇઝેશન} & એન્ટેના દ્વારા રેડિએટ થયેલા ઇલેક્ટ્રોમેગ્નેટિક વેવના ઇલેક્ટ્રિક
ફિલ્ડ વેક્ટરની ઓરિએન્ટેશન \\
\end{longtable}
}

\textbf{આકૃતિ: રેડિએશન પેટર્ન}

\begin{lstlisting}
      │
      │          ┌───┐
      │        ╱       ╲
      │      ╱           ╲
      │    ╱               ╲
      │  ╱                   ╲
      │╱                       ╲
 ─────┼─────────────────────────────▶
      │╲                       ╱
      │  ╲                   ╱
      │    ╲               ╱
      │      ╲           ╱
      │        ╲       ╱
      │          └───┘
      │
\end{lstlisting}

\end{solutionbox}
\begin{mnemonicbox}
``ARDP: એન્ટેના રેડિએટ વિથ ડિરેક્ટિવિટી એન્ડ પોલરાઈઝેશન''

\end{mnemonicbox}
\subsection*{પ્રશ્ન 5(ક) [7
ગુણ]}\label{uxaaauxab0uxab6uxaa8-5uxa95-7-uxa97uxaa3}

\textbf{સંક્ષિપ્ત નોંધ લખો (અ) સ્માર્ટ એન્ટેના (બ) પેરાબોલિક રિફ્લેક્ટર એન્ટેના}

\begin{solutionbox}

\end{solutionbox}
\subsubsection{(અ) સ્માર્ટ
એન્ટેના}\label{uxa85-uxab8uxaaeuxab0uxa9f-uxa8fuxaa8uxa9fuxaa8}


{\def\LTcaptype{none} % do not increment counter
\vspace{-5pt}
\captionof{table}{સ્માર્ટ એન્ટેના લક્ષણો}
\vspace{-10pt}
\begin{longtable}[]{@{}
  >{\raggedright\arraybackslash}p{(\linewidth - 2\tabcolsep) * \real{0.4091}}
  >{\raggedright\arraybackslash}p{(\linewidth - 2\tabcolsep) * \real{0.5909}}@{}}
\toprule\noalign{}
\begin{minipage}[b]{\linewidth}\raggedright
વિશેષતા
\end{minipage} & \begin{minipage}[b]{\linewidth}\raggedright
વર્ણન
\end{minipage} \\
\midrule\noalign{}
\endhead
\bottomrule\noalign{}
\endlastfoot
\textbf{વ્યાખ્યા} & બદલાતી પરિસ્થિતિઓ સાથે અનુકૂલિત થવાની ક્ષમતા સાથે એન્ટેના એરે
સિગ્નલ પ્રોસેસિંગ \\
\textbf{પ્રકારો} & સ્વિચ્ડ બીમ, એડેપ્ટિવ એરે \\
\textbf{ફાયદા} & વધારેલી રેન્જ/કવરેજ, ઇન્ટરફેરન્સ ઘટાડો, ક્ષમતા સુધારણા \\
\textbf{એપ્લિકેશન્સ} & મોબાઇલ કમ્યુનિકેશન, 5G નેટવર્ક્સ, WiMAX, મિલિટરી સિસ્ટમ્સ \\
\end{longtable}
}

\textbf{આકૃતિ: સ્માર્ટ એન્ટેના સિસ્ટમ}

\includegraphics[width=1\linewidth,height=\textheight,keepaspectratio]{mermaid-1ee581a3.pdf}

\subsubsection{(બ) પેરાબોલિક રિફ્લેક્ટર
એન્ટેના}\label{uxaac-uxaaauxab0uxaacuxab2uxa95-uxab0uxaabuxab2uxa95uxa9fuxab0-uxa8fuxaa8uxa9fuxaa8}


{\def\LTcaptype{none} % do not increment counter
\vspace{-5pt}
\captionof{table}{પેરાબોલિક રિફ્લેક્ટર લક્ષણો}
\vspace{-10pt}
\begin{longtable}[]{@{}
  >{\raggedright\arraybackslash}p{(\linewidth - 2\tabcolsep) * \real{0.4091}}
  >{\raggedright\arraybackslash}p{(\linewidth - 2\tabcolsep) * \real{0.5909}}@{}}
\toprule\noalign{}
\begin{minipage}[b]{\linewidth}\raggedright
વિશેષતા
\end{minipage} & \begin{minipage}[b]{\linewidth}\raggedright
વર્ણન
\end{minipage} \\
\midrule\noalign{}
\endhead
\bottomrule\noalign{}
\endlastfoot
\textbf{સ્ટ્રક્ચર} & ફોકલ પોઈન્ટ પર ફીડ એન્ટેના સાથે પેરાબોલિક રિફ્લેક્ટિંગ સરફેસ \\
\textbf{ઓપરેશન} & સમાંતર આવતા તરંગોને ફોકલ પોઈન્ટ પર કેન્દ્રિત કરે છે અથવા ફોકલ
પોઈન્ટથી સમાંતર બીમ્સમાં રેડિએટ કરે છે \\
\textbf{ગેઇન} & ખૂબ ઉચ્ચ દિશાત્મકતા અને ગેઇન \\
\textbf{એપ્લિકેશન્સ} & સેટેલાઇટ કમ્યુનિકેશન, રેડિયો એસ્ટ્રોનોમી, રડાર સિસ્ટમ્સ \\
\end{longtable}
}

\textbf{આકૃતિ: પેરાબોલિક રિફ્લેક્ટર}

\begin{lstlisting}
                 ╱│╲
             ╱    │    ╲
         ╱        │        ╲
     ╱            │            ╲
 ╱                │                ╲
 ╲                │                ╱
     ╲            │            ╱
         ╲        │        ╱
             ╲    │    ╱
                 ╲│╱
                  X
                  │
                  │
                  ▼
                રિસીવર
             (ફોકલ પોઈન્ટ પર)
\end{lstlisting}

\begin{mnemonicbox}
``PFHS: પેરાબોલિક ફોકસ ગિવ્સ હાઇ સિગ્નલ સ્ટ્રેન્થ''

\end{mnemonicbox}
\subsection*{પ્રશ્ન 5(અ) OR [3
ગુણ]}\label{uxaaauxab0uxab6uxaa8-5uxa85-or-3-uxa97uxaa3}

\textbf{માઇક્રોસ્ટ્રીપ એન્ટેના પર ટૂંકી નોંધ લખો}

\begin{solutionbox}


{\def\LTcaptype{none} % do not increment counter
\vspace{-5pt}
\captionof{table}{માઇક્રોસ્ટ્રીપ એન્ટેના લક્ષણો}
\vspace{-10pt}
\begin{longtable}[]{@{}
  >{\raggedright\arraybackslash}p{(\linewidth - 2\tabcolsep) * \real{0.4091}}
  >{\raggedright\arraybackslash}p{(\linewidth - 2\tabcolsep) * \real{0.5909}}@{}}
\toprule\noalign{}
\begin{minipage}[b]{\linewidth}\raggedright
વિશેષતા
\end{minipage} & \begin{minipage}[b]{\linewidth}\raggedright
વર્ણન
\end{minipage} \\
\midrule\noalign{}
\endhead
\bottomrule\noalign{}
\endlastfoot
\textbf{સ્ટ્રક્ચર} & ગ્રાઉન્ડ પ્લેન સાથે ડાયલેક્ટ્રિક સબસ્ટ્રેટ પર કન્ડક્ટિવ પેચ \\
\textbf{આકાર} & લંબચોરસ, ગોળ, ઈંડાકાર, ત્રિકોણાકાર પેચ \\
\textbf{સાઈઝ} & સામાન્ય રીતે λ/2 લંબાઈમાં, ખૂબ પાતળી (h \textless\textless{}
λ) \\
\textbf{ફાયદા} & લો પ્રોફાઇલ, હલકા વજન, ઓછી કિંમત, સરળ ફેબ્રિકેશન, PCB
ટેકનોલોજી સાથે સુસંગત \\
\textbf{ગેરફાયદા} & ઓછી કાર્યક્ષમતા, સાંકડી બેન્ડવિડ્થ, ઓછી પાવર હેન્ડલિંગ \\
\end{longtable}
}

\textbf{આકૃતિ: માઇક્રોસ્ટ્રીપ પેચ એન્ટેના}

\begin{lstlisting}
    ┌─────────────────────┐  \leftarrow── પેચ (કોપર)
    │                     │
    │                     │
    │                     │
    └─────────────────────┘
    ┌─────────────────────┐  \leftarrow── ડાયલેક્ટ્રિક સબસ્ટ્રેટ
    │                     │      (FR4, PTFE, વગેરે)
    └─────────────────────┘
    ┌─────────────────────┐  \leftarrow── ગ્રાઉન્ડ પ્લેન (કોપર)
    └─────────────────────┘
\end{lstlisting}

\end{solutionbox}
\begin{mnemonicbox}
``PDGF: પેચ ઓન ડાયલેક્ટ્રિક વિથ ગ્રાઉન્ડ પ્લેન ગિવ્સ ફ્લેટ
પ્રોફાઇલ''

\end{mnemonicbox}
\subsection*{પ્રશ્ન 5(બ) OR [4
ગુણ]}\label{uxaaauxab0uxab6uxaa8-5uxaac-or-4-uxa97uxaa3}

\textbf{EM વેવ સ્પેક્ટ્રમ, તેની ફ્રીક્વન્સી રેન્જ અને તેની એપ્લિકેશન્સ સમજાવો.}

\begin{solutionbox}


{\def\LTcaptype{none} % do not increment counter
\vspace{-5pt}
\captionof{table}{EM વેવ સ્પેક્ટ્રમ અને એપ્લિકેશન્સ}
\vspace{-10pt}
\begin{longtable}[]{@{}
  >{\raggedright\arraybackslash}p{(\linewidth - 6\tabcolsep) * \real{0.1224}}
  >{\raggedright\arraybackslash}p{(\linewidth - 6\tabcolsep) * \real{0.3469}}
  >{\raggedright\arraybackslash}p{(\linewidth - 6\tabcolsep) * \real{0.2449}}
  >{\raggedright\arraybackslash}p{(\linewidth - 6\tabcolsep) * \real{0.2857}}@{}}
\toprule\noalign{}
\begin{minipage}[b]{\linewidth}\raggedright
બેન્ડ
\end{minipage} & \begin{minipage}[b]{\linewidth}\raggedright
ફ્રિક્વન્સી રેન્જ
\end{minipage} & \begin{minipage}[b]{\linewidth}\raggedright
વેવલેન્થ
\end{minipage} & \begin{minipage}[b]{\linewidth}\raggedright
એપ્લિકેશન્સ
\end{minipage} \\
\midrule\noalign{}
\endhead
\bottomrule\noalign{}
\endlastfoot
\textbf{ELF} & 3 Hz - 30 Hz & 10,000 - 100,000 km & સબમરીન કમ્યુનિકેશન \\
\textbf{VLF} & 3 kHz - 30 kHz & 10 - 100 km & નેવિગેશન, ટાઈમ સિગ્નલ્સ \\
\textbf{LF} & 30 kHz - 300 kHz & 1 - 10 km & AM રેડિઓ, મેરિટાઈમ રેડિઓ \\
\textbf{MF} & 300 kHz - 3 MHz & 100 m - 1 km & AM બ્રોડકાસ્ટિંગ \\
\textbf{HF} & 3 MHz - 30 MHz & 10 - 100 m & શોર્ટવેવ રેડિઓ, એમેચ્યોર રેડિઓ \\
\textbf{VHF} & 30 MHz - 300 MHz & 1 - 10 m & FM રેડિઓ, TV બ્રોડકાસ્ટિંગ \\
\textbf{UHF} & 300 MHz - 3 GHz & 10 cm - 1 m & TV, મોબાઇલ ફોન, WiFi \\
\textbf{SHF} & 3 GHz - 30 GHz & 1 - 10 cm & સેટેલાઇટ, રડાર, 5G \\
\textbf{EHF} & 30 GHz - 300 GHz & 1 mm - 1 cm & રેડિઓ એસ્ટ્રોનોમી, સિક્યુરિટી
સ્કેનિંગ \\
\textbf{IR} & 300 GHz - 400 THz & 750 nm - 1 mm & થર્મલ ઇમેજિંગ, રિમોટ
કંટ્રોલ \\
\textbf{Visible} & 400 THz - 800 THz & 380 - 750 nm & ઓપ્ટિકલ
કમ્યુનિકેશન્સ \\
\end{longtable}
}

\textbf{આકૃતિ: EM વેવ સ્પેક્ટ્રમ}

\includegraphics[width=1\linewidth,height=\textheight,keepaspectratio]{mermaid-249684cb.pdf}

\end{solutionbox}
\begin{mnemonicbox}
``RVMIXG: રેડિઓ, વિઝિબલ, માઇક્રોવેવ, ઇન્ફ્રારેડ, X-રે,
ગામા''

\end{mnemonicbox}
\subsection*{પ્રશ્ન 5(ક) OR [7
ગુણ]}\label{uxaaauxab0uxab6uxaa8-5uxa95-or-7-uxa97uxaa3}

\textbf{સંક્ષિપ્ત નોંધ લખો (અ) Space Wave Propagation અને (બ) Ground Wave
Propagation પર સંક્ષિપ્ત નોંધ લખો.}

\begin{solutionbox}

\end{solutionbox}
\subsubsection{(અ) Space Wave
Propagation}\label{uxa85-space-wave-propagation}


{\def\LTcaptype{none} % do not increment counter
\vspace{-5pt}
\captionof{table}{Space Wave Propagation લક્ષણો}
\vspace{-10pt}
\begin{longtable}[]{@{}
  >{\raggedright\arraybackslash}p{(\linewidth - 2\tabcolsep) * \real{0.4091}}
  >{\raggedright\arraybackslash}p{(\linewidth - 2\tabcolsep) * \real{0.5909}}@{}}
\toprule\noalign{}
\begin{minipage}[b]{\linewidth}\raggedright
વિશેષતા
\end{minipage} & \begin{minipage}[b]{\linewidth}\raggedright
વર્ણન
\end{minipage} \\
\midrule\noalign{}
\endhead
\bottomrule\noalign{}
\endlastfoot
\textbf{વ્યાખ્યા} & સ્પેસ દ્વારા સીધું વેવ પ્રોપેગેશન, જેમાં લાઇન-ઓફ-સાઇટ અને રિફ્લેક્ટેડ
વેવ્સ શામેલ છે \\
\textbf{ફ્રિક્વન્સી રેન્જ} & VHF અને ઉપર (\textgreater30 MHz) \\
\textbf{અંતર} & હોરિઝન દ્વારા મર્યાદિત, સામાન્ય રીતે 50-80 km \\
\textbf{પ્રકારો} & ડાયરેક્ટ વેવ, ગ્રાઉન્ડ રિફ્લેક્ટેડ વેવ, ટ્રોપોસ્ફેરિક સ્કેટર, ડક્ટ
પ્રોપેગેશન \\
\textbf{એપ્લિકેશન્સ} & TV બ્રોડકાસ્ટિંગ, માઇક્રોવેવ લિંક્સ, સેટેલાઇટ કમ્યુનિકેશન \\
\end{longtable}
}

\textbf{આકૃતિ: Space Wave Propagation}

\begin{lstlisting}
                    /\/\/\/\/\/\/\/\  \leftarrow Troposphere
                   /                \
                  /                  \
                 /                    \
    Transmitter *                       * Receiver
                |                       |
                |                       |
     ___________|_______________________|__________
                     Ground Surface
\end{lstlisting}

\subsubsection{(બ) Ground Wave
Propagation}\label{uxaac-ground-wave-propagation}


{\def\LTcaptype{none} % do not increment counter
\vspace{-5pt}
\captionof{table}{Ground Wave Characteristics}
\vspace{-10pt}
\begin{longtable}[]{@{}ll@{}}
\toprule\noalign{}
વિશેષતા & વર્ણન \\
\midrule\noalign{}
\endhead
\bottomrule\noalign{}
\endlastfoot
\textbf{વ્યાખ્યા} & પૃથ્વીની સપાટી સાથે વેવ પ્રોપેગેશન, પૃથ્વીની વક્રતાને અનુસરે છે \\
\textbf{ફ્રિક્વન્સી રેન્જ} & LF, MF (2 MHz સુધી) \\
\textbf{અંતર} & ફ્રિક્વન્સી અને પાવર પર આધારિત 1000 km સુધી \\
\textbf{મેકેનિઝમ} & વર્ટિકલી પોલરાઇઝ્ડ વેવ કન્ડક્ટિવ અર્થ સરફેસને જોડાય છે \\
\textbf{એપ્લિકેશન્સ} & AM રેડિઓ બ્રોડકાસ્ટિંગ, મેરિટાઈમ કમ્યુનિકેશન \\
\end{longtable}
}

\textbf{આકૃતિ: Ground Wave Propagation}

\begin{lstlisting}
       ટ્રાન્સમીટર                              રિસીવર
         *                                   *
         |                                   |
         |       ગ્રાઉન્ડ વેવ                     |
     ____|___________________________________|____
         \\\\\\\\\\\\\\\\\\\\\\\\\\\\\\\\\\\\\\\\
                       અર્થ
\end{lstlisting}

\begin{mnemonicbox}
``SHGM: સ્પેસ વેવ્સ ગો હાઇ, ગ્રાઉન્ડ વેવ્સ હગ મીડિયમ સરફેસ''

\end{mnemonicbox}

\end{document}
