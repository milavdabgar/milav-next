\documentclass[10pt,a4paper]{article}

% content/resources/templates/preamble.tex
\usepackage[margin=0.6in]{geometry}
\author{Milav Dabgar}
\usepackage{amsmath,amssymb,amsthm}
\usepackage{booktabs}
\usepackage{multirow}
\usepackage{xcolor}
\usepackage{tcolorbox}
\tcbuselibrary{breakable,skins}
\usepackage[colorlinks=true,linkcolor=blue]{hyperref}
\usepackage{titlesec}
\usepackage{enumitem}
\usepackage{tikz}
\usepackage{pgfplots}
\usepackage{circuitikz}
\usepackage[version=4]{mhchem}
\usepackage{longtable}
\usepackage{array}
\usepackage{float}
\usepackage{caption}
\usepackage{listings}

\lstset{
  basicstyle=\small\ttfamily,
  breaklines=true,
  breakatwhitespace=false,
  postbreak=\mbox{\textcolor{red}{$\hookrightarrow$}\space},
  float=false,
  numbers=left,
  numberstyle=\tiny\color{gray},
  numbersep=10pt,
  xleftmargin=2em,
  keywordstyle=\color{blue},
  commentstyle=\color{green!60!black},
  stringstyle=\color{purple},
  backgroundcolor=\color{gray!5},
  showstringspaces=false,
  tabsize=2,
  captionpos=b,
  keepspaces=true,
  columns=flexible
}

\pgfplotsset{compat=1.18}
\usetikzlibrary{shapes,arrows,positioning,calc,patterns,decorations.pathmorphing,decorations.markings,arrows.meta}

% Color scheme
\definecolor{headcolor}{RGB}{0,102,204}
\definecolor{keycolor}{RGB}{220,20,60}
\definecolor{solutioncolor}{RGB}{34,139,34}
\definecolor{mnemoniccolor}{RGB}{148,0,211}
\definecolor{codecolor}{RGB}{0,0,100}

% Spacing
\setlength{\parskip}{3pt}
\setlist[itemize]{nosep}
\setlist[enumerate]{nosep}

% Title formatting
\titleformat{\section}{\Large\bfseries\color{headcolor}}{\thesection}{1em}{}
\titleformat{\subsection}{\large\bfseries\color{headcolor}}{\thesubsection}{1em}{}

% Pandoc tightlist compatibility
\providecommand{\tightlist}{%
  \setlength{\itemsep}{0pt}\setlength{\parskip}{0pt}}

% Pandoc longtable compatibility
\newcounter{none}
\def\thenone{}


% content/resources/templates/english-boxes.tex
% This file is currently empty - it exists to maintain consistency with the import structure.
% Add custom environments here if needed in the future.


\begin{document}

\begin{center}
{\Huge\bfseries\color{headcolor} Subject Name Solutions}\\[5pt]
{\LARGE 1333201 -- Summer 2025}\\[3pt]
{\large Semester 1 Study Material}\\[3pt]
{\normalsize\textit{Detailed Solutions and Explanations}}
\end{center}

\vspace{10pt}

\subsection*{Question 1(a) [3 marks]}\label{q1a}

\textbf{Define AM, FM and PM.}

\begin{solutionbox}

{\def\LTcaptype{none} % do not increment counter
\begin{longtable}[]{@{}
  >{\raggedright\arraybackslash}p{(\linewidth - 2\tabcolsep) * \real{0.5714}}
  >{\raggedright\arraybackslash}p{(\linewidth - 2\tabcolsep) * \real{0.4286}}@{}}
\toprule\noalign{}
\begin{minipage}[b]{\linewidth}\raggedright
Modulation Type
\end{minipage} & \begin{minipage}[b]{\linewidth}\raggedright
Definition
\end{minipage} \\
\midrule\noalign{}
\endhead
\bottomrule\noalign{}
\endlastfoot
\textbf{AM (Amplitude Modulation)} & Process where amplitude of carrier
signal varies in accordance with the instantaneous amplitude of the
message signal \\
\textbf{FM (Frequency Modulation)} & Process where frequency of carrier
signal varies in accordance with the instantaneous amplitude of the
message signal \\
\textbf{PM (Phase Modulation)} & Process where phase of carrier signal
varies in accordance with the instantaneous amplitude of the message
signal \\
\end{longtable}
}

\end{solutionbox}
\begin{mnemonicbox}
``AFaP'' - ``Amplitude, Frequency and Phase'' are the
three parameters changed during modulation.

\end{mnemonicbox}
\subsection*{Question 1(b) [4 marks]}\label{q1b}

\textbf{Explain block diagram of communication system.}

\begin{solutionbox}

\begin{figure}
\centering
\pandocbounded{\includesvg[keepaspectratio]{diagrams/1333201-s2025-q1b.svg}}
\caption{Communication System}
\end{figure}

\includegraphics[width=1\linewidth,height=\textheight,keepaspectratio]{mermaid-1ed3b125.pdf}

\textbf{Components of Communication System:}

\begin{itemize}
\tightlist
\item
  \textbf{Information Source}: Produces message to be communicated
\item
  \textbf{Transmitter}: Converts message to signals suitable for
  transmission
\item
  \textbf{Channel}: Medium through which signals travel
\item
  \textbf{Receiver}: Extracts original message from received signal
\item
  \textbf{Destination}: Person/device for whom message is intended
\item
  \textbf{Noise Source}: Unwanted signals that interfere with
  transmitted signal
\end{itemize}

\end{solutionbox}
\begin{mnemonicbox}
``I Transmit Communication Reliably Despite Noise''

\end{mnemonicbox}
\subsection*{Question 1(c) [7 marks]}\label{q1c}

\textbf{Explain Amplitude modulation with waveform and derive voltage
equation for modulated signal also Sketch the frequency spectrum of the
DSBFC AM.}

\begin{solutionbox}

Amplitude Modulation is the process where the amplitude of a
high-frequency carrier wave varies according to the instantaneous value
of the modulating signal.

\textbf{Waveform and Equation:}

\begin{figure}
\centering
\pandocbounded{\includesvg[keepaspectratio]{diagrams/1333201-s2025-q1c.svg}}
\caption{AM Waveform}
\end{figure}

\includegraphics[width=1\linewidth,height=\textheight,keepaspectratio]{mermaid-a9825ee2.pdf}

\textbf{Derivation of AM equation:}

\begin{itemize}
\tightlist
\item
  Carrier signal: c(t) = Ac cos(ωc·t)
\item
  Modulating signal: m(t) = Am cos(ωm·t)
\item
  Modulation Index: μ = Am/Ac
\item
  AM signal: s(t) = Ac[1 + μ·cos(ωm·t)]cos(ωc·t)
\item
  Expanding: s(t) = Ac·cos(ωc·t) + μ·Ac/2·cos[(ωc+ωm)t] +
  μ·Ac/2·cos[(ωc-ωm)t]
\end{itemize}

\textbf{DSBFC AM Frequency Spectrum:}

\begin{lstlisting}
    |
    |        Carrier
    |          |
    |          |
    |          |
    |    LSB   |   USB
    |     |    |    |
    |_____|____|____|_____
         fc-fm fc  fc+fm
\end{lstlisting}

\textbf{Key Points:}

\begin{itemize}
\tightlist
\item
  \textbf{LSB (Lower Sideband)}: Located at fc-fm
\item
  \textbf{USB (Upper Sideband)}: Located at fc+fm
\item
  \textbf{Bandwidth}: 2fm (twice the highest modulating frequency)
\end{itemize}

\end{solutionbox}
\begin{mnemonicbox}
``CARrying Two SideBands'' - DSBFC AM carries both
sidebands.

\end{mnemonicbox}
\subsection*{Question 1(c OR) [7
marks]}\label{question-1c-or-7-marks}

\textbf{Derive the equation for total power in AM, calculate percentage
of power savings in DSBFC And SSBSC.}

\begin{solutionbox}

\textbf{Total Power in AM:}

For AM signal s(t) = Ac[1 + μ·cos(ωm·t)]cos(ωc·t)

\includegraphics[width=1\linewidth,height=\textheight,keepaspectratio]{mermaid-5594433b.pdf}

\textbf{Power Calculation:}

\begin{itemize}
\tightlist
\item
  Carrier Power: Pc = Ac^{2}/2
\item
  Power in each sideband: PUSB = PLSB = Pc·μ^{2}/4
\item
  Total Sideband Power: PUSB + PLSB = Pc·μ^{2}/2
\item
  Total Power: Pt = Pc + PUSB + PLSB = Pc(1 + μ^{2}/2)
\end{itemize}

\textbf{Power Savings:}

{\def\LTcaptype{none} % do not increment counter
\begin{longtable}[]{@{}
  >{\raggedright\arraybackslash}p{(\linewidth - 4\tabcolsep) * \real{0.2609}}
  >{\raggedright\arraybackslash}p{(\linewidth - 4\tabcolsep) * \real{0.4130}}
  >{\raggedright\arraybackslash}p{(\linewidth - 4\tabcolsep) * \real{0.3261}}@{}}
\toprule\noalign{}
\begin{minipage}[b]{\linewidth}\raggedright
Modulation
\end{minipage} & \begin{minipage}[b]{\linewidth}\raggedright
Power Distribution
\end{minipage} & \begin{minipage}[b]{\linewidth}\raggedright
Power Savings
\end{minipage} \\
\midrule\noalign{}
\endhead
\bottomrule\noalign{}
\endlastfoot
DSBFC AM & Uses carrier + both sidebands & 0\% (reference) \\
SSBSC AM & Uses only one sideband, no carrier & (2 - μ^{2}/2)/(1 + μ^{2}/2) \times
100\% \\
\end{longtable}
}

For μ = 1, SSBSC saves approximately 85\% power compared to DSBFC.

\end{solutionbox}
\begin{mnemonicbox}
``SSB Saves Power By Cutting Carrier''

\end{mnemonicbox}
\subsection*{Question 2(a) [3 marks]}\label{q2a}

\textbf{Compare AM and FM.}

\begin{solutionbox}

{\def\LTcaptype{none} % do not increment counter
\begin{longtable}[]{@{}
  >{\raggedright\arraybackslash}p{(\linewidth - 4\tabcolsep) * \real{0.5789}}
  >{\raggedright\arraybackslash}p{(\linewidth - 4\tabcolsep) * \real{0.2105}}
  >{\raggedright\arraybackslash}p{(\linewidth - 4\tabcolsep) * \real{0.2105}}@{}}
\toprule\noalign{}
\begin{minipage}[b]{\linewidth}\raggedright
Parameter
\end{minipage} & \begin{minipage}[b]{\linewidth}\raggedright
AM
\end{minipage} & \begin{minipage}[b]{\linewidth}\raggedright
FM
\end{minipage} \\
\midrule\noalign{}
\endhead
\bottomrule\noalign{}
\endlastfoot
\textbf{Definition} & Amplitude of carrier varies with message signal &
Frequency of carrier varies with message signal \\
\textbf{Bandwidth} & 2 \times message frequency & 2 \times (Δf + fm) \\
\textbf{Noise Immunity} & Poor (noise affects amplitude) & Excellent
(noise mainly affects amplitude) \\
\textbf{Power Efficiency} & Low (carrier contains most power) & High
(all transmitted power contains information) \\
\textbf{Circuit Complexity} & Simple, inexpensive & Complex,
expensive \\
\end{longtable}
}

\end{solutionbox}
\begin{mnemonicbox}
``AM Needs Power, FM Fights Noise''

\end{mnemonicbox}
\subsection*{Question 2(b) [4 marks]}\label{q2b}

\textbf{Draw and explain block diagram for envelope detector.}

\begin{solutionbox}

\includegraphics[width=1\linewidth,height=\textheight,keepaspectratio]{mermaid-72f59575.pdf}

\textbf{Components of Envelope Detector:}

\begin{itemize}
\tightlist
\item
  \textbf{Diode}: Rectifies the AM signal (allows current flow in one
  direction)
\item
  \textbf{RC Circuit}: R and C values chosen such that:

  \begin{itemize}
  \tightlist
  \item
    RC \textgreater\textgreater{} 1/fc (to filter carrier frequency)
  \item
    RC \textless\textless{} 1/fm (to follow the envelope)
  \end{itemize}
\end{itemize}

\textbf{Working:}

\begin{enumerate}
\tightlist
\item
  Diode conducts during positive half-cycles of carrier
\item
  Capacitor charges to peak value
\item
  When input falls, capacitor discharges through resistor
\item
  Output follows envelope of AM signal
\end{enumerate}

\end{solutionbox}
\begin{mnemonicbox}
``Detect, Rect, and Connect'' - Detection through
Rectification and RC connection.

\end{mnemonicbox}
\subsection*{Question 2(c) [7 marks]}\label{q2c}

\textbf{Draw block diagram of FM radio receiver and explain working of
each block.}

\begin{solutionbox}

\includegraphics[width=1\linewidth,height=\textheight,keepaspectratio]{mermaid-35a029a8.pdf}

\textbf{Working of Each Block:}

\begin{itemize}
\tightlist
\item
  \textbf{Antenna}: Receives FM broadcast signals (88-108 MHz)
\item
  \textbf{RF Amplifier}: Amplifies weak RF signals, provides selectivity
\item
  \textbf{Mixer \& Local Oscillator}: Converts RF to fixed IF (10.7 MHz)
  using heterodyning
\item
  \textbf{IF Amplifier}: Provides most of receiver's gain and
  selectivity
\item
  \textbf{Limiter}: Removes amplitude variations from FM signal
\item
  \textbf{FM Detector}: Converts frequency variations to audio (uses
  ratio detector/PLL)
\item
  \textbf{Audio Amplifier}: Amplifies recovered audio signal
\item
  \textbf{Speaker}: Converts electrical signals to sound
\end{itemize}

\end{solutionbox}
\begin{mnemonicbox}
``Really Mighty Instruments Limit Frequency And Make
Sound''

\end{mnemonicbox}
\subsection*{Question 2(a OR) [3
marks]}\label{question-2a-or-3-marks}

\textbf{Define Sensitivity, Selectivity, Fidelity for radio receiver.}

\begin{solutionbox}

{\def\LTcaptype{none} % do not increment counter
\begin{longtable}[]{@{}
  >{\raggedright\arraybackslash}p{(\linewidth - 2\tabcolsep) * \real{0.4783}}
  >{\raggedright\arraybackslash}p{(\linewidth - 2\tabcolsep) * \real{0.5217}}@{}}
\toprule\noalign{}
\begin{minipage}[b]{\linewidth}\raggedright
Parameter
\end{minipage} & \begin{minipage}[b]{\linewidth}\raggedright
Definition
\end{minipage} \\
\midrule\noalign{}
\endhead
\bottomrule\noalign{}
\endlastfoot
\textbf{Sensitivity} & Ability of receiver to amplify weak signals
(measured in μV) \\
\textbf{Selectivity} & Ability to separate desired signal from adjacent
signals \\
\textbf{Fidelity} & Ability to reproduce the original signal without
distortion \\
\end{longtable}
}

\end{solutionbox}
\begin{mnemonicbox}
``SSF'' - ``Select Signals Faithfully''

\end{mnemonicbox}
\subsection*{Question 2(b OR) [4
marks]}\label{question-2b-or-4-marks}

\textbf{Explain ratio detector for FM.}

\begin{solutionbox}

\begin{figure}
\centering
\pandocbounded{\includesvg[keepaspectratio]{diagrams/1333201-s2025-q2b.svg}}
\caption{Ratio Detector}
\end{figure}

\includegraphics[width=1\linewidth,height=\textheight,keepaspectratio]{mermaid-52778ba7.pdf}

\textbf{Working of Ratio Detector:}

\begin{itemize}
\tightlist
\item
  Uses balanced circuit with two diodes in series
\item
  Large stabilizing capacitor keeps sum of voltages constant
\item
  Output voltage is proportional to frequency deviation
\item
  Inherently insensitive to amplitude variations (no limiter needed)
\item
  Less susceptible to impulse noise than discriminator
\end{itemize}

\end{solutionbox}
\begin{mnemonicbox}
``RADS'' - ``Ratio And Diodes Stabilize''

\end{mnemonicbox}
\subsection*{Question 2(c OR) [7
marks]}\label{question-2c-or-7-marks}

\textbf{Draw block diagram of AM radio receiver and explain working of
each block.}

\begin{solutionbox}

\begin{figure}
\centering
\pandocbounded{\includesvg[keepaspectratio]{diagrams/1333201-s2025-q2c.svg}}
\caption{AM Radio Receiver}
\end{figure}

\includegraphics[width=1\linewidth,height=\textheight,keepaspectratio]{mermaid-34e83ad9.pdf}

\textbf{Working of Each Block:}

\begin{itemize}
\tightlist
\item
  \textbf{Antenna}: Intercepts AM broadcast signals (535-1605 kHz)
\item
  \textbf{RF Amplifier}: Amplifies weak RF signals with good SNR
\item
  \textbf{Mixer \& Local Oscillator}: Converts RF to fixed IF (455 kHz)
\item
  \textbf{IF Amplifier}: Provides most gain and selectivity at 455 kHz
\item
  \textbf{Detector}: Extracts audio from AM signal (envelope detector)
\item
  \textbf{AGC (Automatic Gain Control)}: Maintains constant output level
\item
  \textbf{Audio Amplifier}: Boosts detected audio to drive speaker
\item
  \textbf{Speaker}: Converts electrical signals to sound waves
\end{itemize}

\end{solutionbox}
\begin{mnemonicbox}
``ARMIDAS'' - ``Amplify, Mix, IF, Detect, Audio,
Speak''

\end{mnemonicbox}
\subsection*{Question 3(a) [3 marks]}\label{q3a}

\textbf{Describe the Nyquist criteria.}

\begin{solutionbox}

\textbf{Nyquist Criteria}: To accurately reconstruct a signal from its
samples, the sampling frequency (fs) must be at least twice the highest
frequency (fmax) present in the signal.

{\def\LTcaptype{none} % do not increment counter
\begin{longtable}[]{@{}lll@{}}
\toprule\noalign{}
Parameter & Formula & Description \\
\midrule\noalign{}
\endhead
\bottomrule\noalign{}
\endlastfoot
\textbf{Nyquist Rate} & fs \geq 2fmax & Minimum sampling rate required \\
\textbf{Nyquist Interval} & Ts \leq 1/2fmax & Maximum time between
samples \\
\end{longtable}
}

\textbf{Consequence if violated}: Aliasing occurs - higher frequencies
appear as lower frequencies in sampled signal.

\end{solutionbox}
\begin{mnemonicbox}
``Sample Double to Dodge Aliasing''

\end{mnemonicbox}
\subsection*{Question 3(b) [4 marks]}\label{q3b}

\textbf{Explain Sample and hold Circuit with Waveform.}

\begin{solutionbox}

\begin{figure}
\centering
\pandocbounded{\includesvg[keepaspectratio]{diagrams/1333201-s2025-q3a.svg}}
\caption{Sample and Hold Circuit}
\end{figure}

\includegraphics[width=1\linewidth,height=\textheight,keepaspectratio]{mermaid-6762491d.pdf}

\textbf{Sample and Hold Circuit Operation:}

\begin{itemize}
\tightlist
\item
  \textbf{Electronic Switch}: Closes briefly during sampling
\item
  \textbf{Capacitor}: Stores sampled voltage
\item
  \textbf{Buffer Amplifier}: Provides high input impedance and low
  output impedance
\end{itemize}

\textbf{Waveform:}

\begin{figure}
\centering
\pandocbounded{\includesvg[keepaspectratio]{diagrams/1333201-s2025-q3b.svg}}
\caption{Sample and Hold Waveform}
\end{figure}

\textbf{Applications:}

\begin{itemize}
\tightlist
\item
  Analog-to-Digital Conversion
\item
  Data Acquisition Systems
\item
  Pulse Amplitude Modulation
\end{itemize}

\end{solutionbox}
\begin{mnemonicbox}
``SCAB'' - ``Switch, Capacitor And Buffer''

\end{mnemonicbox}
\subsection*{Question 3(c) [7 marks]}\label{q3c}

\textbf{Define quantization explain uniform and non-uniform quantization
in details.}

\begin{solutionbox}

\textbf{Quantization}: Process of mapping a large set of input values to
a smaller set of discrete output values.

\begin{figure}
\centering
\pandocbounded{\includesvg[keepaspectratio]{diagrams/1333201-s2025-q3c.svg}}
\caption{Quantization Process}
\end{figure}

\includegraphics[width=1\linewidth,height=\textheight,keepaspectratio]{mermaid-6f836e5c.pdf}

\textbf{Uniform Quantization vs Non-uniform Quantization:}

{\def\LTcaptype{none} % do not increment counter
\begin{longtable}[]{@{}
  >{\raggedright\arraybackslash}p{(\linewidth - 4\tabcolsep) * \real{0.1897}}
  >{\raggedright\arraybackslash}p{(\linewidth - 4\tabcolsep) * \real{0.3621}}
  >{\raggedright\arraybackslash}p{(\linewidth - 4\tabcolsep) * \real{0.4483}}@{}}
\toprule\noalign{}
\begin{minipage}[b]{\linewidth}\raggedright
Parameter
\end{minipage} & \begin{minipage}[b]{\linewidth}\raggedright
Uniform Quantization
\end{minipage} & \begin{minipage}[b]{\linewidth}\raggedright
Non-uniform Quantization
\end{minipage} \\
\midrule\noalign{}
\endhead
\bottomrule\noalign{}
\endlastfoot
\textbf{Step Size} & Equal throughout range & Varies (smaller for small
signals) \\
\textbf{Characteristic} & Linear & Non-linear
(logarithmic/exponential) \\
\textbf{SNR} & Poor for small signals & Better for small signals \\
\textbf{Implementation} & Simple & Complex (companding required) \\
\textbf{Applications} & Simple signals, images & Speech, audio (μ-law,
A-law) \\
\end{longtable}
}

\textbf{Quantization Error:}

\begin{itemize}
\tightlist
\item
  Difference between original and quantized signal
\item
  Maximum error = \pmQ/2 (where Q is quantization step size)
\item
  Appears as quantization noise in reconstructed signal
\end{itemize}

\end{solutionbox}
\begin{mnemonicbox}
``UNIQ'' - ``UNIform has equal steps, non-uniform
Quiets noise''

\end{mnemonicbox}
\subsection*{Question 3(a OR) [3
marks]}\label{question-3a-or-3-marks}

\textbf{Explain aliasing error and how to overcome it.}

\begin{solutionbox}

\textbf{Aliasing Error}: Distortion that occurs when a signal is sampled
at a rate lower than twice its highest frequency component.

\includegraphics[width=1\linewidth,height=\textheight,keepaspectratio]{mermaid-e0efe6ac.pdf}

\textbf{How to Overcome Aliasing:}

\begin{itemize}
\tightlist
\item
  Use anti-aliasing filter (low-pass) before sampling
\item
  Increase sampling rate above Nyquist rate (fs \textgreater{} 2fmax)
\item
  Bandlimit the input signal before sampling
\end{itemize}

\end{solutionbox}
\begin{mnemonicbox}
``ALIAS'' - ``Avoid Low sampling by Increasing And
Screening''

\end{mnemonicbox}
\subsection*{Question 3(b OR) [4
marks]}\label{question-3b-or-4-marks}

\textbf{Draw following signal in time domain and frequency domain:}
\textbf{1) Sawtooth signal} \textbf{2) Pulse signal}

\begin{solutionbox}

\textbf{Sawtooth Signal:}

Time Domain:

\begin{lstlisting}
    /|  /|  /|  /|
   / | / | / | / |
  /  |/  |/  |/  |
     T   2T  3T
\end{lstlisting}

Frequency Domain:

\begin{lstlisting}
    |
    |
    |\
    | \
    |  \
    |   \
    |____\____________
    0  f0 2f0 3f0 4f0
\end{lstlisting}

\textbf{Pulse Signal:}

Time Domain:

\begin{lstlisting}
    |‾|     |‾|     |‾|
    | |     | |     | |
____|_|_____|_|_____|_|____
    T       2T      3T
\end{lstlisting}

Frequency Domain:

\begin{lstlisting}
    |
    |    sinc function
    |\       /\
    | \     /  \
    |  \___/    \____
    |
    |___________________
    0   f0    2f0    3f0
\end{lstlisting}

\end{solutionbox}
\begin{mnemonicbox}
``STPF'' - ``SawTooth slopes down, Pulse has sinc
Function''

\end{mnemonicbox}
\subsection*{Question 3(c OR) [7
marks]}\label{question-3c-or-7-marks}

\textbf{Compare PAM, PWM and PPM with waveform.}

\begin{solutionbox}

{\def\LTcaptype{none} % do not increment counter
\begin{longtable}[]{@{}
  >{\raggedright\arraybackslash}p{(\linewidth - 6\tabcolsep) * \real{0.4231}}
  >{\raggedright\arraybackslash}p{(\linewidth - 6\tabcolsep) * \real{0.1923}}
  >{\raggedright\arraybackslash}p{(\linewidth - 6\tabcolsep) * \real{0.1923}}
  >{\raggedright\arraybackslash}p{(\linewidth - 6\tabcolsep) * \real{0.1923}}@{}}
\toprule\noalign{}
\begin{minipage}[b]{\linewidth}\raggedright
Parameter
\end{minipage} & \begin{minipage}[b]{\linewidth}\raggedright
PAM
\end{minipage} & \begin{minipage}[b]{\linewidth}\raggedright
PWM
\end{minipage} & \begin{minipage}[b]{\linewidth}\raggedright
PPM
\end{minipage} \\
\midrule\noalign{}
\endhead
\bottomrule\noalign{}
\endlastfoot
\textbf{Full Form} & Pulse Amplitude Modulation & Pulse Width Modulation
& Pulse Position Modulation \\
\textbf{Parameter Varied} & Amplitude of pulses & Width/duration of
pulses & Position/timing of pulses \\
\textbf{Noise Immunity} & Poor & Good & Excellent \\
\textbf{Bandwidth} & Lower & Higher & Highest \\
\textbf{Power Efficiency} & Low & Medium & High \\
\textbf{Demodulation} & Simple & Moderate & Complex \\
\end{longtable}
}

\textbf{Waveforms:}

\begin{lstlisting}
Message:    /\/\/\

PAM:        ‖  ‖   ‖ ‖  ‖   ‖
            ‖  ‖   ‖ ‖  ‖   ‖

PWM:        ‖‖‖ ‖‖  ‖ ‖‖‖ ‖‖  ‖
                    
PPM:        ‖  ‖   ‖ ‖  ‖   ‖
            |--|---||-|--|---||
\end{lstlisting}

\end{solutionbox}
\begin{mnemonicbox}
``APP'' - ``Amplitude, Pulse-width, Position''

\end{mnemonicbox}
\subsection*{Question 4(a) [3 marks]}\label{q4a}

\textbf{Explain Space wave propagation.}

\begin{solutionbox}

\textbf{Space Wave Propagation}: Mode where radio waves travel through
lower atmosphere (troposphere) directly or via ground reflection.

\begin{figure}
\centering
\pandocbounded{\includesvg[keepaspectratio]{diagrams/1333201-s2025-q4a.svg}}
\caption{Space Wave Propagation}
\end{figure}

\includegraphics[width=1\linewidth,height=\textheight,keepaspectratio]{mermaid-75e2f1a7.pdf}

\textbf{Characteristics:}

\begin{itemize}
\tightlist
\item
  Frequency range: VHF, UHF (30 MHz - 3 GHz)
\item
  Limited to line-of-sight distance
\item
  Range = 4.12(\sqrth_{1} + \sqrth_{2}) km (where h_{1}, h_{2} = heights in meters)
\item
  Affected by terrain, buildings, and atmospheric conditions
\end{itemize}

\end{solutionbox}
\begin{mnemonicbox}
``SLOT'' - ``Straight Line Over Terrain''

\end{mnemonicbox}
\subsection*{Question 4(b) [4 marks]}\label{q4b}

\textbf{Explain working of Differential PCM (DPCM) transmitter.}

\begin{solutionbox}

\begin{figure}
\centering
\pandocbounded{\includesvg[keepaspectratio]{diagrams/1333201-s2025-q4b.svg}}
\caption{DPCM Transmitter}
\end{figure}

\includegraphics[width=1\linewidth,height=\textheight,keepaspectratio]{mermaid-ae89d37d.pdf}

\textbf{Working of DPCM Transmitter:}

\begin{itemize}
\tightlist
\item
  \textbf{Predictor}: Estimates current sample based on previous samples
\item
  \textbf{Subtractor}: Computes difference between actual and predicted
  value
\item
  \textbf{Quantizer}: Converts difference signal to discrete levels
\item
  \textbf{Encoder}: Converts quantized values to binary code
\item
  \textbf{Feedback Loop}: Reconstructs signal as receiver would see it
\end{itemize}

\textbf{Advantage}: Only difference signal is transmitted, which
requires fewer bits

\end{solutionbox}
\begin{mnemonicbox}
``SPEQIF'' - ``Subtract, Predict, Encode, Quantize In
Feedback''

\end{mnemonicbox}
\subsection*{Question 4(c) [7 marks]}\label{q4c}

\textbf{Explain delta modulator in details also explain slop overload
noise and granular noise.}

\begin{solutionbox}

\textbf{Delta Modulation (DM)}: Simplest form of differential PCM where
the difference signal is encoded with just 1 bit.

\begin{figure}
\centering
\pandocbounded{\includesvg[keepaspectratio]{diagrams/1333201-s2025-q4c.svg}}
\caption{Delta Modulator}
\end{figure}

\includegraphics[width=1\linewidth,height=\textheight,keepaspectratio]{mermaid-79b6cdce.pdf}

\textbf{Working Principle:}

\begin{itemize}
\tightlist
\item
  Compares input signal with integrated version of previous output
\item
  If input \textgreater{} integrated value: transmit 1
\item
  If input \textless{} integrated value: transmit 0
\item
  Step size (δ) is fixed
\end{itemize}

\textbf{Noise in Delta Modulation:}

{\def\LTcaptype{none} % do not increment counter
\begin{longtable}[]{@{}
  >{\raggedright\arraybackslash}p{(\linewidth - 4\tabcolsep) * \real{0.4688}}
  >{\raggedright\arraybackslash}p{(\linewidth - 4\tabcolsep) * \real{0.2188}}
  >{\raggedright\arraybackslash}p{(\linewidth - 4\tabcolsep) * \real{0.3125}}@{}}
\toprule\noalign{}
\begin{minipage}[b]{\linewidth}\raggedright
Type of Noise
\end{minipage} & \begin{minipage}[b]{\linewidth}\raggedright
Cause
\end{minipage} & \begin{minipage}[b]{\linewidth}\raggedright
Solution
\end{minipage} \\
\midrule\noalign{}
\endhead
\bottomrule\noalign{}
\endlastfoot
\textbf{Slope Overload Noise} & Input signal changes faster than δ can
track & Increase step size or sampling frequency \\
\textbf{Granular Noise} & Step size too large for slowly varying signals
& Decrease step size \\
\end{longtable}
}

\end{solutionbox}
\begin{mnemonicbox}
``DOGS'' - ``Delta modulation has Overload and
Granular noiseS''

\end{mnemonicbox}
\subsection*{Question 4(a OR) [3
marks]}\label{question-4a-or-3-marks}

\textbf{Explain Ground wave propagation.}

\begin{solutionbox}

\textbf{Ground Wave Propagation}: Radio wave propagation that follows
the curvature of the Earth.

\includegraphics[width=1\linewidth,height=\textheight,keepaspectratio]{mermaid-c4179f01.pdf}

\textbf{Characteristics:}

\begin{itemize}
\tightlist
\item
  Frequency range: LF, MF (30 kHz - 3 MHz)
\item
  Propagates along Earth's surface (vertically polarized)
\item
  Range depends on transmitter power, ground conductivity, frequency
\item
  Signal strength decreases with distance and frequency
\item
  Used for AM broadcasting, marine communication
\end{itemize}

\end{solutionbox}
\begin{mnemonicbox}
``GEL'' - ``Ground waves follow Earth at Low
frequencies''

\end{mnemonicbox}
\subsection*{Question 4(b OR) [4
marks]}\label{question-4b-or-4-marks}

\textbf{Explain ADM transmitter.}

\begin{solutionbox}

\textbf{Adaptive Delta Modulation (ADM)}: Improved version of DM where
step size varies according to signal characteristics.

\includegraphics[width=1\linewidth,height=\textheight,keepaspectratio]{mermaid-514a2561.pdf}

\textbf{Working of ADM Transmitter:}

\begin{itemize}
\tightlist
\item
  \textbf{Basic Operation}: Similar to standard DM
\item
  \textbf{Step Size Control}: Analyzes recent output bits
\item
  \textbf{Adaptation Logic}:

  \begin{itemize}
  \tightlist
  \item
    If consecutive bits are same: Increase step size
  \item
    If consecutive bits alternate: Decrease step size
  \end{itemize}
\end{itemize}

\textbf{Advantages over DM:}

\begin{itemize}
\tightlist
\item
  Reduces both slope overload and granular noise
\item
  Better signal tracking
\item
  Improved SNR
\end{itemize}

\end{solutionbox}
\begin{mnemonicbox}
``ASIC'' - ``Adapt Step-size, Improve Coding''

\end{mnemonicbox}
\subsection*{Question 4(c OR) [7
marks]}\label{question-4c-or-7-marks}

\textbf{Explain Block diagram of basic PCM-TDM system.}

\begin{solutionbox}

\textbf{PCM-TDM System}: Combines Pulse Code Modulation with Time
Division Multiplexing to transmit multiple digital signals over single
channel.

\includegraphics[width=1\linewidth,height=\textheight,keepaspectratio]{mermaid-6e667d28.pdf}

\textbf{Working of PCM-TDM System:}

\begin{itemize}
\tightlist
\item
  \textbf{Transmitter}:

  \begin{itemize}
  \tightlist
  \item
    Multiple analog signals sampled simultaneously
  \item
    Samples time-multiplexed into single stream
  \item
    Stream quantized and encoded into PCM format
  \item
    Framing bits added for synchronization
  \end{itemize}
\item
  \textbf{Receiver}:

  \begin{itemize}
  \tightlist
  \item
    Frame sync detected for alignment
  \item
    PCM stream decoded to recover samples
  \item
    Demultiplexer separates individual channel samples
  \item
    Low-pass filters reconstruct original analog signals
  \end{itemize}
\end{itemize}

\end{solutionbox}
\begin{mnemonicbox}
``SAMPLE-CODE-MUX'' - Sampling, Coding, and
Multiplexing

\end{mnemonicbox}
\subsection*{Question 5(a) [3 marks]}\label{q5a}

\textbf{Define radiation pattern, Directivity and Gain for antenna.}

\begin{solutionbox}

\begin{figure}
\centering
\pandocbounded{\includesvg[keepaspectratio]{diagrams/1333201-s2025-q5a.svg}}
\caption{Antenna Parameters}
\end{figure}

{\def\LTcaptype{none} % do not increment counter
\begin{longtable}[]{@{}
  >{\raggedright\arraybackslash}p{(\linewidth - 2\tabcolsep) * \real{0.4783}}
  >{\raggedright\arraybackslash}p{(\linewidth - 2\tabcolsep) * \real{0.5217}}@{}}
\toprule\noalign{}
\begin{minipage}[b]{\linewidth}\raggedright
Parameter
\end{minipage} & \begin{minipage}[b]{\linewidth}\raggedright
Definition
\end{minipage} \\
\midrule\noalign{}
\endhead
\bottomrule\noalign{}
\endlastfoot
\textbf{Radiation Pattern} & Graphical representation of radiation
properties (field strength or power) as function of space coordinates \\
\textbf{Directivity} & Ratio of maximum radiation intensity to average
radiation intensity \\
\textbf{Gain} & Product of directivity and efficiency (practical measure
of antenna performance) \\
\end{longtable}
}

\textbf{Relationship}: Gain = Directivity \times Efficiency

\end{solutionbox}
\begin{mnemonicbox}
``RDG'' - ``Radiation Directs with Gain''

\end{mnemonicbox}
\subsection*{Question 5(b) [4 marks]}\label{q5b}

\textbf{Explain Microstrip Antenna with sketch.}

\begin{solutionbox}

\textbf{Microstrip (Patch) Antenna}: Low-profile antenna consisting of a
metal patch on a substrate with ground plane.

\begin{figure}
\centering
\pandocbounded{\includesvg[keepaspectratio]{diagrams/1333201-s2025-q5b.svg}}
\caption{Microstrip Antenna}
\end{figure}

\includegraphics[width=1\linewidth,height=\textheight,keepaspectratio]{mermaid-1926b02e.pdf}

\textbf{Key Features:}

\begin{itemize}
\tightlist
\item
  \textbf{Patch}: Typically rectangular or circular (λ/2 in length)
\item
  \textbf{Substrate}: Low-loss dielectric material (εr = 2.2 to 12)
\item
  \textbf{Feeding Methods}: Microstrip line, coaxial probe, aperture
  coupling
\item
  \textbf{Radiation}: Primarily from fringing fields at patch edges
\end{itemize}

\textbf{Applications}: Mobile devices, GPS, RFID, satellite
communications

\end{solutionbox}
\begin{mnemonicbox}
``PSDG'' - ``Patch on Substrate with Dielectric over
Ground''

\end{mnemonicbox}
\subsection*{Question 5(c) [7 marks]}\label{q5c}

\textbf{Explain PCM transmitter and receiver in details.}

\begin{solutionbox}

\textbf{PCM (Pulse Code Modulation)} Transmitter:

\includegraphics[width=1\linewidth,height=\textheight,keepaspectratio]{mermaid-9eeecf53.pdf}

\textbf{PCM Receiver:}

\begin{figure}
\centering
\pandocbounded{\includesvg[keepaspectratio]{diagrams/1333201-s2025-q5c.svg}}
\caption{PCM System}
\end{figure}

\includegraphics[width=1\linewidth,height=\textheight,keepaspectratio]{mermaid-7101fd3e.pdf}

\textbf{Working Details:}

{\def\LTcaptype{none} % do not increment counter
\begin{longtable}[]{@{}ll@{}}
\toprule\noalign{}
Block & Function \\
\midrule\noalign{}
\endhead
\bottomrule\noalign{}
\endlastfoot
\textbf{Anti-aliasing Filter} & Limits bandwidth to prevent aliasing \\
\textbf{Sample \& Hold} & Takes samples at regular intervals \\
\textbf{Quantizer} & Assigns discrete amplitude levels \\
\textbf{Encoder} & Converts levels to binary codes \\
\textbf{Line Coder} & Converts digital data to transmission format \\
\textbf{Regenerative Repeater} & Restores signal quality \\
\textbf{Decoder} & Converts binary to amplitude levels \\
\textbf{Reconstruction Filter} & Smoothens staircase output to analog \\
\end{longtable}
}

\end{solutionbox}
\begin{mnemonicbox}
``SAFE PCR'' - ``Sample, Amplify, Filter, Encode,
Pulse Code Receiver''

\end{mnemonicbox}
\subsection*{Question 5(a OR) [3
marks]}\label{question-5a-or-3-marks}

\textbf{Explain dipole antenna with sketch.}

\begin{solutionbox}

\textbf{Dipole Antenna}: Simplest and most widely used antenna
consisting of two conducting elements.

\includegraphics[width=1\linewidth,height=\textheight,keepaspectratio]{mermaid-1b6822ab.pdf}

\textbf{Key Characteristics:}

\begin{itemize}
\tightlist
\item
  \textbf{Length}: Typically λ/2 (half-wavelength dipole)
\item
  \textbf{Radiation Pattern}: Figure-8 pattern perpendicular to antenna
  axis
\item
  \textbf{Impedance}: \textasciitilde73 Ω for half-wave dipole
\item
  \textbf{Polarization}: Same as the orientation of the antenna
\end{itemize}

\textbf{Applications}: Radio broadcasting, TV reception, amateur radio

\end{solutionbox}
\begin{mnemonicbox}
``HALF'' - ``Half-wavelength Antenna Leads Field''

\end{mnemonicbox}
\subsection*{Question 5(b OR) [4
marks]}\label{question-5b-or-4-marks}

\textbf{Explain parabolic reflector antenna With Sketch.}

\begin{solutionbox}

\textbf{Parabolic Reflector Antenna}: High-gain antenna using parabolic
dish to focus electromagnetic waves.

\includegraphics[width=1\linewidth,height=\textheight,keepaspectratio]{mermaid-f9a758c5.pdf}

\textbf{Working Principle:}

\begin{itemize}
\tightlist
\item
  \textbf{Feed}: Located at focal point of parabola
\item
  \textbf{Reflector}: Parabolic surface reflects waves in parallel
  direction
\item
  \textbf{Reflection Property}: All paths from focal point to reflector
  to parallel line are equal
\end{itemize}

\textbf{Applications}:

\begin{itemize}
\tightlist
\item
  Satellite communications
\item
  Radio astronomy
\item
  Radar systems
\item
  Microwave links
\end{itemize}

\end{solutionbox}
\begin{mnemonicbox}
``PROF'' - ``Parabola Reflects On Focus''

\end{mnemonicbox}
\subsection*{Question 5(c OR) [7
marks]}\label{question-5c-or-7-marks}

\textbf{Compare PCM, DM, ADM and DPCM.}

\begin{solutionbox}

{\def\LTcaptype{none} % do not increment counter
\begin{longtable}[]{@{}
  >{\raggedright\arraybackslash}p{(\linewidth - 8\tabcolsep) * \real{0.3667}}
  >{\raggedright\arraybackslash}p{(\linewidth - 8\tabcolsep) * \real{0.1667}}
  >{\raggedright\arraybackslash}p{(\linewidth - 8\tabcolsep) * \real{0.1333}}
  >{\raggedright\arraybackslash}p{(\linewidth - 8\tabcolsep) * \real{0.1333}}
  >{\raggedright\arraybackslash}p{(\linewidth - 8\tabcolsep) * \real{0.2000}}@{}}
\toprule\noalign{}
\begin{minipage}[b]{\linewidth}\raggedright
Parameter
\end{minipage} & \begin{minipage}[b]{\linewidth}\raggedright
PCM
\end{minipage} & \begin{minipage}[b]{\linewidth}\raggedright
DM
\end{minipage} & \begin{minipage}[b]{\linewidth}\raggedright
ADM
\end{minipage} & \begin{minipage}[b]{\linewidth}\raggedright
DPCM
\end{minipage} \\
\midrule\noalign{}
\endhead
\bottomrule\noalign{}
\endlastfoot
\textbf{Full Form} & Pulse Code Modulation & Delta Modulation & Adaptive
Delta Modulation & Differential PCM \\
\textbf{Bits per Sample} & 8-16 bits & 1 bit & 1 bit & 3-4 bits \\
\textbf{Step Size} & Fixed quantization levels & Fixed step size &
Variable step size & Fixed quantization of difference \\
\textbf{Bandwidth Requirement} & Highest & Lowest & Low & Medium \\
\textbf{Signal Quality} & Excellent & Poor to moderate & Moderate &
Good \\
\textbf{Implementation Complexity} & Moderate & Very simple & Moderate &
Complex \\
\textbf{Applications} & Digital audio, telephony & Simple telemetry &
Voice communication & Video, speech \\
\end{longtable}
}

\textbf{Key Differences:}

\begin{itemize}
\tightlist
\item
  \textbf{PCM}: Encodes absolute amplitude values
\item
  \textbf{DM}: Encodes only 1-bit difference with fixed step
\item
  \textbf{ADM}: Improves DM by adapting step size
\item
  \textbf{DPCM}: Encodes multi-bit difference signal
\end{itemize}

\end{solutionbox}
\begin{mnemonicbox}
``PAID'' - ``PCM, ADM, Integrate in DPCM''

\end{mnemonicbox}

\end{document}
