\documentclass{article}

% content/resources/templates/preamble.tex
\usepackage[margin=0.6in]{geometry}
\author{Milav Dabgar}
\usepackage{amsmath,amssymb,amsthm}
\usepackage{booktabs}
\usepackage{multirow}
\usepackage{xcolor}
\usepackage{tcolorbox}
\tcbuselibrary{breakable,skins}
\usepackage[colorlinks=true,linkcolor=blue]{hyperref}
\usepackage{titlesec}
\usepackage{enumitem}
\usepackage{tikz}
\usepackage{pgfplots}
\usepackage{circuitikz}
\usepackage[version=4]{mhchem}
\usepackage{longtable}
\usepackage{array}
\usepackage{float}
\usepackage{caption}
\usepackage{listings}

\lstset{
  basicstyle=\small\ttfamily,
  breaklines=true,
  breakatwhitespace=false,
  postbreak=\mbox{\textcolor{red}{$\hookrightarrow$}\space},
  float=false,
  numbers=left,
  numberstyle=\tiny\color{gray},
  numbersep=10pt,
  xleftmargin=2em,
  keywordstyle=\color{blue},
  commentstyle=\color{green!60!black},
  stringstyle=\color{purple},
  backgroundcolor=\color{gray!5},
  showstringspaces=false,
  tabsize=2,
  captionpos=b,
  keepspaces=true,
  columns=flexible
}

\pgfplotsset{compat=1.18}
\usetikzlibrary{shapes,arrows,positioning,calc,patterns,decorations.pathmorphing,decorations.markings,arrows.meta}

% Color scheme
\definecolor{headcolor}{RGB}{0,102,204}
\definecolor{keycolor}{RGB}{220,20,60}
\definecolor{solutioncolor}{RGB}{34,139,34}
\definecolor{mnemoniccolor}{RGB}{148,0,211}
\definecolor{codecolor}{RGB}{0,0,100}

% Spacing
\setlength{\parskip}{3pt}
\setlist[itemize]{nosep}
\setlist[enumerate]{nosep}

% Title formatting
\titleformat{\section}{\Large\bfseries\color{headcolor}}{\thesection}{1em}{}
\titleformat{\subsection}{\large\bfseries\color{headcolor}}{\thesubsection}{1em}{}

% Pandoc tightlist compatibility
\providecommand{\tightlist}{%
  \setlength{\itemsep}{0pt}\setlength{\parskip}{0pt}}

% Pandoc longtable compatibility
\newcounter{none}
\def\thenone{}


% content/resources/templates/gujarati-boxes.tex
\usepackage{fontspec}
\usepackage{polyglossia}

% Set Gujarati as main language (document is primarily in Gujarati)
% Note: gloss-gujarati.ldf doesn't exist in polyglossia, but it will use hyphenation patterns
\setdefaultlanguage{gujarati}
\setotherlanguage{english}

% Configure Gujarati font properly
% Use Language=Default to prevent polyglossia from trying to add language-specific features
% that don't exist for Gujarati, which causes "empty feature" warnings
\newfontfamily\gujaratifont[Script=Gujarati,AutoFakeBold=2.5,AutoFakeSlant=0.3]{Noto Sans Gujarati}
\setmainfont[Script=Gujarati,AutoFakeBold=2.5,AutoFakeSlant=0.3]{Noto Sans Gujarati}
% Use Noto Sans Gujarati for monospace to support Gujarati in text
\setmonofont[Scale=0.9]{Noto Sans Gujarati}

% Configure English to use the same font
\newfontfamily\englishfont[Script=Gujarati,AutoFakeBold=2.5,AutoFakeSlant=0.3]{Noto Sans Gujarati}

% Translations for polyglossia
\gappto\captionsgujarati{
  \renewcommand{\tablename}{કોષ્ટક}
  \renewcommand{\figurename}{આકૃતિ}
}

% Helper for TikZ nodes to ensure Gujarati font
\newcommand{\gu}[1]{{\gujaratifont #1}}

% Custom environments
\newtcolorbox{solutionbox}{
    breakable,
    enhanced,
    colback=solutioncolor!5!white,
    colframe=solutioncolor!75!black,
    fonttitle=\bfseries,
    title=જવાબ
}

\newtcolorbox{solutionboxnobreak}{
 colback=solutioncolor!5!white,
 colframe=solutioncolor!75!black,
 fonttitle=\bfseries,
 title=જવાબ
}

\newtcolorbox{keyformula}{
 breakable,
 enhanced,
 colback=keycolor!5!white,
 colframe=keycolor!75!black,
 fonttitle=\bfseries,
 title=રાસાયણિક સમીકરણ/સૂત્ર
}

\newtcolorbox{mnemonicbox}{
 breakable,
 enhanced,
 colback=mnemoniccolor!5!white,
 colframe=mnemoniccolor!75!black,
 fonttitle=\bfseries,
 title=મેમરી ટ્રીક
}


% Custom commands for GTU solutions
% This file defines semantic commands for consistent formatting

% Question command with automatic formatting
\newcommand{\question}[2]{%
  \section*{Question #1}%
  \textbf{#2}%
}

% OR question variant
\newcommand{\questionor}[2]{%
  \section*{Question #1 OR}%
  \textbf{#2}%
}

% Proper table environment with caption
\newenvironment{answertable}[1]{%
  \begin{table}[htbp]
  \centering
  \caption{#1}
}{%
  \end{table}
}

% Proper figure environment for diagrams
\newenvironment{answerdiagram}[1]{%
  \begin{figure}[htbp]
  \centering
  \caption{#1}
}{%
  \end{figure}
}

% Semantic markup for key terms
\newcommand{\keyword}[1]{\textbf{#1}}
\newcommand{\code}[1]{\texttt{#1}}
\newcommand{\classname}[1]{\texttt{#1}}
\newcommand{\methodname}[1]{\texttt{#1}}

% Proper quotation marks
\newcommand{\mnemonic}[1]{``#1''}


\title{Communication Engineering (1333201) - Summer 2025 Solution}
\date{May 09, 2025}

\begin{document}
\maketitle

\questionmarks{1(a)}{3}{AM, FM અને PM ને વ્યાખ્યાયિત કરો.}

\begin{solutionbox}
\textbf{જવાબ}:

\begin{center}
\captionof{table}{મોડ્યુલેશન પ્રકારો વ્યાખ્યા}
\begin{tabulary}{\linewidth}{|L|L|}
\hline
\textbf{મોડ્યુલેશન પ્રકાર} & \textbf{વ્યાખ્યા} \\
\hline
\textbf{AM (Amplitude Modulation)} & એવી પ્રક્રિયા જેમાં કેરિઅર સિગ્નલનું amplitude, મેસેજ સિગ્નલના તાત્કાલિક amplitude અનુસાર બદલાય છે \\
\hline
\textbf{FM (Frequency Modulation)} & એવી પ્રક્રિયા જેમાં કેરિઅર સિગ્નલની frequency, મેસેજ સિગ્નલના તાત્કાલિક amplitude અનુસાર બદલાય છે \\
\hline
\textbf{PM (Phase Modulation)} & એવી પ્રક્રિયા જેમાં કેરિઅર સિગ્નલનો phase, મેસેજ સિગ્નલના તાત્કાલિક amplitude અનુસાર બદલાય છે \\
\hline
\end{tabulary}
\end{center}
\end{solutionbox}

\begin{mnemonicbox}
"AFaP" - "Amplitude, Frequency અને Phase" એ ત્રણ પરામિતિઓ છે જે મોડ્યુલેશન દરમિયાન બદલાય છે.
\end{mnemonicbox}

\questionmarks{1(b)}{4}{કોમ્યુનિકેશન સિસ્ટમનો બ્લોક ડાયાગ્રામ સમજાવો.}

\begin{solutionbox}
\textbf{જવાબ}:

\begin{center}
\begin{tikzpicture}[node distance=2.5cm, auto, >=latex, thick]
    % Nodes
    \node [gtu block] (source) {માહિતી\\સ્ત્રોત};
    \node [gtu block, right of=source] (tx) {ટ્રાન્સમીટર};
    \node [gtu block, right of=tx] (channel) {ચેનલ};
    \node [gtu block, right of=channel] (rx) {રિસીવર};
    \node [gtu block, right of=rx] (dest) {ગંતવ્ય};
    \node [gtu block, below of=channel, node distance=2cm] (noise) {નોઇઝ સ્ત્રોત};

    % Arrows
    \draw [gtu arrow] (source) -- (tx);
    \draw [gtu arrow] (tx) -- (channel);
    \draw [gtu arrow] (channel) -- (rx);
    \draw [gtu arrow] (rx) -- (dest);
    \draw [gtu arrow] (noise) -- (channel);
\end{tikzpicture}
\captionof{figure}{કોમ્યુનિકેશન સિસ્ટમ}
\end{center}

\textbf{કોમ્યુનિકેશન સિસ્ટમના ઘટકો:}
\begin{itemize}
    \item \textbf{માહિતી સ્ત્રોત}: સંદેશાનું ઉત્પાદન કરે છે
    \item \textbf{ટ્રાન્સમીટર}: સંદેશને પ્રસારણ માટે યોગ્ય સિગ્નલમાં રૂપાંતરિત કરે છે
    \item \textbf{ચેનલ}: માધ્યમ જેના દ્વારા સિગ્નલ્સ પ્રવાસ કરે છે
    \item \textbf{રિસીવર}: પ્રાપ્ત સિગ્નલમાંથી મૂળ સંદેશ કાઢે છે
    \item \textbf{ગંતવ્ય}: વ્યક્તિ/ઉપકરણ જેના માટે સંદેશ છે
    \item \textbf{નોઇઝ સ્ત્રોત}: અવાંછિત સિગ્નલ્સ જે પ્રસારિત સિગ્નલમાં દખલ કરે છે
\end{itemize}
\end{solutionbox}

\begin{mnemonicbox}
"માદરચગ" - "માહિતી, ટ્રાન્સમીટર, દાખલ, રિસીવર, ચેનલ, ગંતવ્ય"
\end{mnemonicbox}

\questionmarks{1(c)}{7}{AM મોડ્યુલેશન વેવફોર્મ સાથે સમજાવો અને મોડ્યુલેટેડ સિગ્નલ માટે વોલ્ટેજ સમીકરણ મેળવો. DSBFC AM ફ્રીક્વન્સી સ્પેક્ટ્રમ દોરો.}

\begin{solutionbox}
\textbf{જવાબ}:
Amplitude Modulation એ એવી પ્રક્રિયા છે જેમાં ઉચ્ચ આવૃત્તિવાળા કેરિયર વેવનું amplitude મોડ્યુલેટિંગ સિગ્નલના તાત્કાલિક મૂલ્ય અનુસાર બદલાય છે.

\textbf{વેવફોર્મ અને સમીકરણ:}

\begin{center}
\begin{tikzpicture}[domain=0:12, samples=200, scale=0.8]
    \draw[->] (-0.5,0) -- (12.5,0) node[right] {$t$};
    \draw[->] (0,-2.5) -- (0,2.5) node[above] {$s(t)$};
    
    % AM Signal
    \draw[blue, thick] plot (\x, {(1 + 0.5*cos(\x r)) * cos(10*\x r)});
    \draw[red, dashed] plot (\x, {1 + 0.5*cos(\x r)});
    \draw[red, dashed] plot (\x, {-1 - 0.5*cos(\x r)});
    
    \node[right] at (12.5, 1.5) {Envelope};
\end{tikzpicture}
\captionof{figure}{AM વેવફોર્મ}
\end{center}

\textbf{AM સમીકરણનું તારણ:}
\begin{itemize}
    \item કેરિયર સિગ્નલ: $c(t) = A_c \cos(\omega_c t)$
    \item મોડ્યુલેટિંગ સિગ્નલ: $m(t) = A_m \cos(\omega_m t)$
    \item મોડ્યુલેશન ઇન્ડેક્સ: $\mu = A_m/A_c$
    \item AM સિગ્નલ: $s(t) = A_c[1 + \mu \cdot \cos(\omega_m t)]\cos(\omega_c t)$
    \item વિસ્તરણ: $s(t) = A_c \cdot \cos(\omega_c t) + \frac{\mu A_c}{2} \cos[(\omega_c+\omega_m)t] + \frac{\mu A_c}{2} \cos[(\omega_c-\omega_m)t]$
\end{itemize}

\textbf{DSBFC AM ફ્રીકવન્સી સ્પેક્ટ્રમ:}

\begin{center}
\begin{tikzpicture}[scale=1]
    \draw[->] (0,0) -- (6,0) node[below] {$f$};
    \draw[->] (0,0) -- (0,3) node[left] {Amplitude};
    
    % Carrier
    \draw[thick, blue] (3,0) -- (3,2.5);
    \node[above] at (3,2.5) {$A_c$};
    \node[below] at (3,0) {$f_c$};
    
    % LSB
    \draw[thick, blue] (1.5,0) -- (1.5,1.5);
    \node[above] at (1.5,1.5) {$\frac{\mu A_c}{2}$};
    \node[below] at (1.5,0) {$f_c - f_m$};
    \node[above] at (1.5, 0.5) {LSB};

    % USB
    \draw[thick, blue] (4.5,0) -- (4.5,1.5);
    \node[above] at (4.5,1.5) {$\frac{\mu A_c}{2}$};
    \node[below] at (4.5,0) {$f_c + f_m$};
    \node[above] at (4.5, 0.5) {USB};
\end{tikzpicture}
\captionof{figure}{DSBFC AM ફ્રીકવન્સી સ્પેક્ટ્રમ}
\end{center}

\textbf{મુખ્ય બિંદુઓ:}
\begin{itemize}
    \item \textbf{LSB (લોઅર સાઇડબેન્ડ)}: $f_c-f_m$ પર સ્થિત
    \item \textbf{USB (અપર સાઇડબેન્ડ)}: $f_c+f_m$ પર સ્થિત
    \item \textbf{બેન્ડવિડ્થ}: $2f_m$ (ઉચ્ચતમ મોડ્યુલેટિંગ આવૃત્તિનો બે ગણો)
\end{itemize}
\end{solutionbox}

\begin{mnemonicbox}
"બે ઓળ સાથે" - DSBFC AM બંને સાઇડબેન્ડ્સ વહન કરે છે.
\end{mnemonicbox}

\questionmarks{1(c) OR}{7}{AM માં કુલ પાવર માટે સમીકરણ મેળવો, DSB અને SSB માં પાવર બચતની ટકાવારીની ગણતરી કરો.}

\begin{solutionbox}
\textbf{જવાબ}:

\textbf{AM માં કુલ પાવર:}
AM સિગ્નલ $s(t) = A_c[1 + \mu \cdot \cos(\omega_m t)]\cos(\omega_c t)$ માટે:

\textbf{પાવર ગણતરી:}
\begin{itemize}
    \item કેરિયર પાવર: $P_c = A_c^2/2$
    \item દરેક સાઇડબેન્ડમાં પાવર: $P_{USB} = P_{LSB} = P_c \cdot \mu^2/4$
    \item કુલ સાઇડબેન્ડ પાવર: $P_{USB} + P_{LSB} = P_c \cdot \mu^2/2$
    \item કુલ પાવર: $P_t = P_c + P_{USB} + P_{LSB} = P_c(1 + \mu^2/2)$
\end{itemize}

\textbf{પાવર બચત:}

\begin{center}
\captionof{table}{પાવર બચત}
\begin{tabulary}{\linewidth}{|L|L|L|}
\hline
\textbf{મોડ્યુલેશન} & \textbf{પાવર વિતરણ} & \textbf{પાવર બચત} \\
\hline
\textbf{DSBFC AM} & કેરિયર + બંને સાઇડબેન્ડ્સ વાપરે છે & 0\% (સંદર્ભ) \\
\hline
\textbf{SSBSC AM} & ફક્ત એક સાઇડબેન્ડ, કેરિયર નહીં & $\frac{2 - \mu^2/2}{1 + \mu^2/2} \times 100\%$ \\
\hline
\end{tabulary}
\end{center}

$\mu = 1$ માટે, SSBSC લગભગ 85\% પાવર બચાવે છે, DSBFC ની તુલનામાં.
\end{solutionbox}

\begin{mnemonicbox}
"SSB કેરિયર કાપી પાવર બચાવે"
\end{mnemonicbox}

\questionmarks{2(a)}{3}{AM અને FM ની સરખામણી કરો.}

\begin{solutionbox}
\textbf{જવાબ}:

\begin{center}
\captionof{table}{AM અને FM ની સરખામણી}
\begin{tabulary}{\linewidth}{|L|L|L|}
\hline
\textbf{પેરામીટર} & \textbf{AM} & \textbf{FM} \\
\hline
\textbf{વ્યાખ્યા} & કેરિયરનું એમ્પ્લિટ્યુડ મેસેજ સિગ્નલ મુજબ બદલાય છે & કેરિયરની ફ્રીક્વન્સી મેસેજ સિગ્નલ મુજબ બદલાય છે \\
\hline
\textbf{બેન્ડવિડ્થ} & 2 $\times$ મેસેજ ફ્રીક્વન્સી & 2 $\times$ ($\Delta f + f_m$) \\
\hline
\textbf{નોઇઝ ઇમ્યુનિટી} & નબળી (નોઇઝ એમ્પ્લિટ્યુડને અસર કરે છે) & ઉત્તમ (નોઇઝ મુખ્યત્વે એમ્પ્લિટ્યુડને અસર કરે છે) \\
\hline
\textbf{પાવર કાર્યક્ષમતા} & ઓછી (કેરિયરમાં મોટાભાગનો પાવર હોય છે) & ઉંચી (બધો ટ્રાન્સમિટેડ પાવર માહિતી ધરાવે છે) \\
\hline
\textbf{સર્કિટ જટિલતા} & સરળ, સસ્તું & જટિલ, મોંઘું \\
\hline
\end{tabulary}
\end{center}
\end{solutionbox}

\begin{mnemonicbox}
"AM ને પાવર જોઈએ, FM નોઇઝ સામે લડે"
\end{mnemonicbox}

\questionmarks{2(b)}{4}{Envelope detector નો બ્લોક ડાયાગ્રામ દોરો અને સમજાવો.}

\begin{solutionbox}
\textbf{જવાબ}:

\begin{center}
\begin{tikzpicture}[auto, >=latex, thick]
    \node (input) {AM ઇનપુટ};
    \node [right of=input, node distance=2cm] (diode) {};
    % Diode symbol
    \draw (diode) -- ++(1,0) coordinate (d1);
    \draw (d1) -- ++(0.5,0.5) -- ++(0,-1) -- ++(-0.5,0.5); 
    \draw (d1) ++(0.5,-0.5) -- ++(0,1);
    \draw (d1) ++(0.5,0) -- ++(1,0) coordinate (node1);
    
    % RC parallel
    \draw (node1) -- ++(0,-1.5) coordinate (c_top);
    \draw (c_top) to[C, l=C] ++(0,-1.5) coordinate (gnd);
    \draw (node1) -- ++(2,0) coordinate (node2);
    \draw (node2) -- ++(0,-1.5) coordinate (r_top);
    \draw (r_top) to[R, l=R] ++(0,-1.5) coordinate (gnd2);
    
    % Ground
    \node [ground] at (gnd) {};
    \node [ground] at (gnd2) {};
    
    % Output
    \draw (node2) -- ++(1.5,0) node[right] (output) {આઉટપુટ સિગ્નલ};
    
    \node [above of=diode] {ડાયોડ};
\end{tikzpicture}
\captionof{figure}{એન્વેલપ ડિટેક્ટર}
\end{center}

\textbf{એન્વેલપ ડિટેક્ટરના ઘટકો:}
\begin{itemize}
    \item \textbf{ડાયોડ}: AM સિગ્નલને રેક્ટિફાય કરે છે (એક દિશામાં પ્રવાહ વહેવા દે છે)
    \item \textbf{RC સર્કિટ}: R અને C કિંમતો એવી રીતે પસંદ કરવામાં આવે છે કે:
    \begin{itemize}
        \item $RC \gg 1/f_c$ (કેરિયર ફ્રીક્વન્સી ફિલ્ટર કરવા માટે)
        \item $RC \ll 1/f_m$ (એન્વેલપને ફોલો કરવા માટે)
    \end{itemize}
\end{itemize}

\textbf{કાર્યપદ્ધતિ:}
\begin{enumerate}
    \item કેરિયરના પોઝિટિવ હાફ-સાયકલ દરમિયાન ડાયોડ કન્ડક્ટ થાય છે
    \item કેપેસિટર પીક વેલ્યુ સુધી ચાર્જ થાય છે
    \item જ્યારે ઇનપુટ ઘટે છે, ત્યારે કેપેસિટર રઝિસ્ટર દ્વારા ડિસ્ચાર્જ થાય છે
    \item આઉટપુટ AM સિગ્નલના એન્વેલપને ફોલો કરે છે
\end{enumerate}
\end{solutionbox}

\begin{mnemonicbox}
"ડીટેક્ટ, રેક્ટ, અને કનેક્ટ"
\end{mnemonicbox}

\questionmarks{2(c)}{7}{FM રેડિયો રીસીવરનો બ્લોક ડાયાગ્રામ દોરો અને દરેક બ્લોકનું કાર્ય સમજાવો.}

\begin{solutionbox}
\textbf{જવાબ}:

\begin{center}
\begin{tikzpicture}[node distance=2cm, auto, >=latex, thick, scale=0.8, transform shape]
    \node [gtu block] (ant) {એન્ટેના};
    \node [gtu block, right of=ant, node distance=2.2cm] (rf) {RF એમ્પ};
    \node [gtu block, right of=rf, node distance=2.2cm] (mix) {મિક્સર};
    \node [gtu block, below of=mix] (lo) {લોકલ ઓસ્સિલેટર};
    \node [gtu block, right of=mix, node distance=2.2cm] (if) {IF એમ્પ};
    \node [gtu block, right of=if, node distance=2.2cm] (lim) {લિમિટર};
    \node [gtu block, right of=lim, node distance=2.2cm] (det) {FM ડિટેક્ટર};
    \node [gtu block, right of=det, node distance=2.2cm] (af) {ઓડિયો એમ્પ};
    \node [gtu block, right of=af, node distance=2.2cm] (spk) {સ્પીકર};

    \draw [gtu arrow] (ant) -- (rf);
    \draw [gtu arrow] (rf) -- (mix);
    \draw [gtu arrow] (lo) -- (mix);
    \draw [gtu arrow] (mix) -- (if);
    \draw [gtu arrow] (if) -- (lim);
    \draw [gtu arrow] (lim) -- (det);
    \draw [gtu arrow] (det) -- (af);
    \draw [gtu arrow] (af) -- (spk);
\end{tikzpicture}
\captionof{figure}{FM રેડિયો રીસીવર}
\end{center}

\textbf{દરેક બ્લોકનું કાર્ય:}
\begin{itemize}
    \item \textbf{એન્ટેના}: FM બ્રોડકાસ્ટ સિગ્નલ્સ (88-108 MHz) પ્રાપ્ત કરે છે
    \item \textbf{RF એમ્પ્લિફાયર}: નબળા RF સિગ્નલ્સને એમ્પ્લીફાય કરે છે, સિલેક્ટિવિટી પ્રદાન કરે છે
    \item \textbf{મિક્સર અને લોકલ ઓસ્સિલેટર}: RF ને ફિક્સ્ડ IF (10.7 MHz) માં રૂપાંતરિત કરે છે
    \item \textbf{IF એમ્પ્લિફાયર}: રીસીવરનો મોટાભાગનો ગેઇન અને સિલેક્ટિવિટી પ્રદાન કરે છે
    \item \textbf{લિમિટર}: FM સિગ્નલમાંથી એમ્પ્લિટ્યુડ ભિન્નતા દૂર કરે છે
    \item \textbf{FM ડિટેક્ટર}: ફ્રીક્વન્સી ભિન્નતાને ઓડિયોમાં રૂપાંતરિત કરે છે
    \item \textbf{ઓડિયો એમ્પ્લિફાયર}: રિકવર થયેલ ઓડિયો સિગ્નલને એમ્પ્લીફાય કરે છે
    \item \textbf{સ્પીકર}: ઇલેક્ટ્રિકલ સિગ્નલ્સને અવાજમાં ફેરવે છે
\end{itemize}
\end{solutionbox}

\begin{mnemonicbox}
"ખરેખર મોટા સાધનો ફ્રીક્વન્સી લિમિટ કરે અને અવાજ કરે"
\end{mnemonicbox}

\questionmarks{2(a) OR}{3}{રેડિયો રીસીવર માટે Sensitivity, Selectivity, Fidelity વ્યાખ્યાયિત કરો.}

\begin{solutionbox}
\textbf{જવાબ}:

\begin{center}
\captionof{table}{રિસીવર લાક્ષણિકતાઓ}
\begin{tabulary}{\linewidth}{|L|L|}
\hline
\textbf{પેરામીટર} & \textbf{વ્યાખ્યા} \\
\hline
\textbf{Sensitivity (સંવેદનશીલતા)} & નબળા સિગ્નલને એમ્પ્લીફાય કરવાની રીસીવરની ક્ષમતા ($\mu V$ માં મપાય છે) \\
\hline
\textbf{Selectivity (પસંદગી)} & ઇચ્છિત સિગ્નલને નજીકના અનિચ્છનીય સિગ્નલોથી અલગ કરવાની ક્ષમતા \\
\hline
\textbf{Fidelity (ફિડેલિટી)} & મૂળ સિગ્નલને વિકૃતિ (distortion) વગર પુનઃઉત્પાદિત કરવાની ક્ષમતા \\
\hline
\end{tabulary}
\end{center}
\end{solutionbox}

\begin{mnemonicbox}
"SSF" - "Select Signals Faithfully"
\end{mnemonicbox}

\questionmarks{2(b) OR}{4}{FM માટે રેડિયો ડિટેક્ટર સમજાવો.}

\begin{solutionbox}
\textbf{જવાબ}:

\begin{center}
\begin{tikzpicture}[auto, >=latex, thick]
    % Simplified representation
    \node [gtu block] (input) {FM ઇનપુટ};
    \node [gtu block, right of=input, node distance=3.5cm] (diodes) {ડાયોડ બ્રિજ};
    \node [gtu block, right of=diodes, node distance=3.5cm] (cap) {મોટો કેપેસિટર ($10\mu F$)};
    \node [gtu block, right of=cap, node distance=3.5cm] (out) {ઓડિયો આઉટપુટ};

    \draw [gtu arrow] (input) -- (diodes);
    \draw [gtu arrow] (diodes) -- (cap);
    \draw [gtu arrow] (cap) -- (out);
\end{tikzpicture}
\captionof{figure}{રેશિયો ડિટેક્ટર}
\end{center}

\textbf{રેશિયો ડિટેક્ટરનું કાર્ય:}
\begin{itemize}
    \item સીરીઝમાં બે ડાયોડ સાથે બેલેન્સ સર્કિટ વાપરે છે
    \item મોટો સ્ટેબિલાઇઝિંગ કેપેસિટર વોલ્ટેજનો સરવાળો અચળ રાખે છે
    \item આઉટપુટ વોલ્ટેજ ફ્રીક્વન્સી ડેવિએશનના પ્રમાણમાં હોય છે
    \item સ્વાભાવિક રીતે એમ્પ્લિટ્યુડ ભિન્નતા પ્રત્યે અસંવેદનશીલ (લિમિટરની જરૂર નથી)
    \item ડિસ્ક્રિમિનેટર કરતાં ઇમ્પલ્સ નોઇઝ માટે ઓછું સંવેદનશીલ છે
\end{itemize}
\end{solutionbox}

\begin{mnemonicbox}
"RADS" - "Ratio And Diodes Stabilize"
\end{mnemonicbox}

\questionmarks{2(c) OR}{7}{AM રેડિયો રીસીવરનો બ્લોક ડાયાગ્રામ દોરો અને દરેક બ્લોકનું કાર્ય સમજાવો.}

\begin{solutionbox}
\textbf{જવાબ}:

\begin{center}
\begin{tikzpicture}[node distance=2cm, auto, >=latex, thick, scale=0.8, transform shape]
    \node [gtu block] (ant) {એન્ટેના};
    \node [gtu block, right of=ant, node distance=2.2cm] (rf) {RF એમ્પ};
    \node [gtu block, right of=rf, node distance=2.2cm] (mix) {મિક્સર};
    \node [gtu block, below of=mix] (lo) {લોકલ ઓસ્સિલેટર};
    \node [gtu block, right of=mix, node distance=2.2cm] (if) {IF એમ્પ};
    \node [gtu block, right of=if, node distance=2.2cm] (det) {ડિટેક્ટર};
    \node [gtu block, below of=det] (agc) {AGC};
    \node [gtu block, right of=det, node distance=2.2cm] (af) {ઓડિયો એમ્પ};
    \node [gtu block, right of=af, node distance=2.2cm] (spk) {સ્પીકર};

    \draw [gtu arrow] (ant) -- (rf);
    \draw [gtu arrow] (rf) -- (mix);
    \draw [gtu arrow] (lo) -- (mix);
    \draw [gtu arrow] (mix) -- (if);
    \draw [gtu arrow] (if) -- (det);
    \draw [gtu arrow] (det) -- (af);
    \draw [gtu arrow] (af) -- (spk);
    
    % AGC feedback
    \draw [gtu arrow] (det) -- (agc);
    \draw [gtu arrow] (agc) -| (if);
    \draw [gtu arrow] (agc) -| (rf);
\end{tikzpicture}
\captionof{figure}{AM રેડિયો રીસીવર}
\end{center}

\textbf{દરેક બ્લોકનું કાર્ય:}
\begin{itemize}
    \item \textbf{એન્ટેના}: AM બ્રોડકાસ્ટ સિગ્નલ્સ (535-1605 kHz) ઇન્ટરસેપ્ટ કરે છે
    \item \textbf{RF એમ્પ્લિફાયર}: સારા SNR સાથે નબળા RF સિગ્નલ્સને એમ્પ્લીફાય કરે છે
    \item \textbf{મિક્સર અને લોકલ ઓસ્સિલેટર}: RF ને ફિક્સ્ડ IF (455 kHz) માં રૂપાંતરિત કરે છે
    \item \textbf{IF એમ્પ્લિફાયર}: 455 kHz પર સૌથી વધુ ગેઇન અને સિલેક્ટિવિટી પ્રદાન કરે છે
    \item \textbf{ડિટેક્ટર}: AM સિગ્નલમાંથી ઓડિયો એક્સટ્રેક્ટ કરે છે (એન્વેલપ ડિટેક્ટર)
    \item \textbf{AGC (ઓટોમેટિક ગેઇન કંટ્રોલ)}: આઉટપુટ લેવલ અચળ જાળવી રાખે છે
    \item \textbf{ઓડિયો એમ્પ્લિફાયર}: સ્પીકર ચલાવવા માટે ડિટેક્ટ થયેલ ઓડિયોને બૂસ્ટ કરે છે
    \item \textbf{સ્પીકર}: ઇલેક્ટ્રિકલ સિગ્નલ્સને સાઉન્ડ વેવ્સમાં ફેરવે છે
\end{itemize}
\end{solutionbox}

\begin{mnemonicbox}
"ARMIDAS" - "Amplify, Mix, IF, Detect, Audio, Speak"
\end{mnemonicbox}

\questionmarks{3(a)}{3}{Nyquist criteria વર્ણન કરો.}

\begin{solutionbox}
\textbf{જવાબ}:

\textbf{નાઇક્વીસ્ટ ક્રાયટેરિયા}: સિગ્નલને તેના સેમ્પલ્સમાંથી સચોટપણે રીકન્સ્ટ્રક્ટ કરવા માટે, સેમ્પલિંગ આવૃત્તિ ($f_s$) સિગ્નલમાં હાજર ઉચ્ચતમ આવૃત્તિ ($f_{max}$) કરતાં ઓછામાં ઓછી બમણી હોવી જોઇએ.

\begin{center}
\captionof{table}{નાઇક્વીસ્ટ ક્રાયટેરિયા}
\begin{tabulary}{\linewidth}{|L|L|L|}
\hline
\textbf{પેરામીટર} & \textbf{ફોર્મ્યુલા} & \textbf{વિવરણ} \\
\hline
\textbf{નાઇક્વીસ્ટ રેટ} & $f_s \ge 2f_{max}$ & જરૂરી ન્યૂનતમ સેમ્પલિંગ રેટ \\
\hline
\textbf{નાઇક્વીસ્ટ ઇન્ટરવલ} & $T_s \le 1/2f_{max}$ & સેમ્પલ્સ વચ્ચેનો મહત્તમ સમય \\
\hline
\end{tabulary}
\end{center}

\textbf{જો ઉલ્લંઘન થાય તો પરિણામ}: એલિયાસિંગ થાય છે - ઉચ્ચ આવૃત્તિઓ સેમ્પલ્ડ સિગ્નલમાં નીચી આવૃત્તિઓ તરીકે દેખાય છે.
\end{solutionbox}

\begin{mnemonicbox}
"બે ગણી લો એલિયાસિંગ ટાળવા"
\end{mnemonicbox}

\questionmarks{3(b)}{4}{Sample and Hold સર્કિટ વેવફોર્મ સાથે સમજાવો.}

\begin{solutionbox}
\textbf{જવાબ}:

\begin{center}
\begin{tikzpicture}[auto, >=latex, thick]
    \node (input) {એનાલોગ ઇનપુટ};
    \node [draw, rectangle, right of=input, node distance=2.5cm] (switch) {સ્વિચ};
    \node [above of=switch] (clock) {ક્લોક};
    \node [right of=switch, node distance=2cm] (node1) {};
    \node [below of=node1] (cap) {કેપેસિટર};
    \node [gtu block, right of=node1, node distance=2cm] (buffer) {બફર};
    \node [right of=buffer, node distance=2cm] (output) {આઉટપુટ};

    \draw [gtu arrow] (input) -- (switch);
    \draw [gtu arrow] (clock) -- (switch);
    \draw [gtu arrow] (switch) -- (buffer);
    \draw [gtu arrow] (buffer) -- (output);
    \draw (node1) -- (cap);
    \node [ground, below of=cap, node distance=0.8cm] (gnd) {};
    \draw (cap) -- (gnd);
\end{tikzpicture}
\captionof{figure}{સેમ્પલ એન્ડ હોલ્ડ સર્કિટ}
\end{center}

\textbf{સેમ્પલ એન્ડ હોલ્ડ સર્કિટ ઓપરેશન:}
\begin{itemize}
    \item \textbf{ઇલેક્ટ્રોનિક સ્વિચ}: સેમ્પલિંગ દરમિયાન થોડો સમય બંધ થાય છે
    \item \textbf{કેપેસિટર}: સેમ્પલ કરેલા વોલ્ટેજને સ્ટોર કરે છે
    \item \textbf{બફર એમ્પ્લિફાયર}: ઉચ્ચ ઇનપુટ અવરોધ અને નીચો આઉટપુટ અવરોધ પ્રદાન કરે છે
\end{itemize}

\textbf{વેવફોર્મ:}
\begin{center}
\begin{tikzpicture}[scale=0.8]
    \draw[->] (0,0) -- (8,0) node[right] {t};
    % Analog Input
    \draw[gray, dashed] plot[domain=0:8, samples=100] (\x, {1.5 + 1*sin(\x*50)});
    \node[left] at (0, 1.5) {ઇનપુટ};
    
    % Clock (Switch closing)
    \foreach \x in {1,2,3,4,5,6,7} {
        \draw[thick, red] (\x, 0) -- (\x, 0.5);
    }
    
    % Output (Step)
    \draw[blue, thick] (0, 1.6) -- (1, 1.6); % Start
    \foreach \x in {1,2,3,4,5,6} {
        \draw[blue, thick] (\x, {1.5 + 1*sin(\x*50)}) -- (\x+1, {1.5 + 1*sin(\x*50)});
        \draw[blue, dotted] (\x, {1.5 + 1*sin(\x*50)}) -- (\x, {1.5 + 1*sin((\x-1)*50)}); 
    }
    \draw[blue, thick] (7, {1.5 + 1*sin(350)}) -- (8, {1.5 + 1*sin(350)});
    
    \node[right] at (8, 1.5) {S/H આઉટપુટ};
\end{tikzpicture}
\captionof{figure}{સેમ્પલ એન્ડ હોલ્ડ વેવફોર્મ્સ}
\end{center}

\textbf{એપ્લિકેશન્સ:}
\begin{itemize}
    \item એનાલોગ-ટુ-ડિજિટલ કન્વર્ઝન
    \item ડેટા એક્વિઝિશન સિસ્ટમ્સ
    \item પલ્સ એમ્પ્લિટ્યુડ મોડ્યુલેશન
\end{itemize}
\end{solutionbox}

\begin{mnemonicbox}
"સ્કેબ" - "સ્વિચ, કેપેસિટર અને બફર"
\end{mnemonicbox}

\questionmarks{3(c)}{7}{વ્યાખ્યાયિત કરો quantization અને uniform and non-uniform quantization સમજાવો.}

\begin{solutionbox}
\textbf{જવાબ}:

\textbf{ક્વોન્ટિઝેશન}: ઇનપુટના મોટા સેટને નાના સેટના ડિસ્ક્રીટ આઉટપુટ વેલ્યુમાં મેપિંગ કરવાની પ્રક્રિયા.

\begin{center}
\begin{tikzpicture}[auto, >=latex, thick]
    \node [gtu block] (cont) {સતત\\એમ્પ્લિટ્યુડ};
    \node [gtu block, right of=cont, node distance=3.5cm] (disc) {ડિસ્ક્રીટ\\એમ્પ્લિટ્યુડ};
    \node [gtu block, right of=disc, node distance=3.5cm] (dig) {ડિજિટલ\\કોડ};
    \draw [gtu arrow] (cont) -- (disc);
    \draw [gtu arrow] (disc) -- (dig);
\end{tikzpicture}
\captionof{figure}{ક્વોન્ટિઝેશન પ્રક્રિયા}
\end{center}

\textbf{યુનિફોર્મ ક્વોન્ટિઝેશન વિરુદ્ધ નોન-યુનિફોર્મ ક્વોન્ટિઝેશન:}

\begin{center}
\captionof{table}{ક્વોન્ટિઝેશન પ્રકારોની સરખામણી}
\begin{tabulary}{\linewidth}{|L|L|L|}
\hline
\textbf{પેરામીટર} & \textbf{યુનિફોર્મ ક્વોન્ટિઝેશન} & \textbf{નોન-યુનિફોર્મ ક્વોન્ટિઝેશન} \\
\hline
\textbf{સ્ટેપ સાઇઝ} & સમગ્ર રેન્જમાં સરખી & બદલાતી રહે છે (નાના સિગ્નલ્સ માટે નાની) \\
\hline
\textbf{લક્ષણ} & લિનિયર & નોન-લિનિયર (લોગેરિધમિક/એક્સપોનેન્શિયલ) \\
\hline
\textbf{SNR} & નાના સિગ્નલ્સ માટે ખરાબ & નાના સિગ્નલ્સ માટે સારા \\
\hline
\textbf{ઇમ્પ્લિમેન્ટેશન} & સરળ & જટિલ (કોમ્પાન્ડિંગ જરૂરી) \\
\hline
\textbf{એપ્લિકેશન} & સરળ સિગ્નલ્સ, ઇમેજિસ & સ્પીચ, ઓડિયો ($\mu$-law, A-law) \\
\hline
\end{tabulary}
\end{center}

\textbf{ક્વોન્ટિઝેશન એરર:}
\begin{itemize}
    \item મૂળ અને ક્વોન્ટાઇઝ્ડ સિગ્નલ વચ્ચેનો તફાવત
    \item મહત્તમ એરર = $\pm Q/2$ (જ્યાં Q ક્વોન્ટિઝેશન સ્ટેપ સાઇઝ છે)
    \item રીકન્સ્ટ્રક્ટેડ સિગ્નલમાં ક્વોન્ટિઝેશન નોઇઝ તરીકે દેખાય છે
\end{itemize}
\end{solutionbox}

\begin{mnemonicbox}
"સરન" - "સરખા સ્ટેપ્સ, નાની સ્ટેપ્સ નાના સિગ્નલ્સ માટે"
\end{mnemonicbox}

\questionmarks{3(a) OR}{3}{સમજાવો aliasing error અને તેને કેવી રીતે દૂર કરવું.}

\begin{solutionbox}
\textbf{જવાબ}:

\textbf{એલિયાસિંગ એરર}: વિકૃતિ જે ત્યારે થાય છે જ્યારે સિગ્નલને તેના ઉચ્ચતમ આવૃત્તિ ઘટકના બે ગણા કરતાં ઓછા દરે સેમ્પલ કરવામાં આવે છે.

\begin{center}
\begin{tikzpicture}[node distance=2cm, auto, >=latex]
    \node [gtu block] (alias) {એલિયાસિંગ એરર};
    \node [right of=alias, node distance=4cm, text width=4cm, align=left] (effects) {1. મૂળ ઉચ્ચ આવૃત્તિ ખોટી નીચી આવૃત્તિ તરીકે દેખાય છે\\2. બદલી ન શકાય તેવી વિકૃતિ};
    \draw [->, thick] (alias) -- (effects);
\end{tikzpicture}
\end{center}

\textbf{એલિયાસિંગ દૂર કરવાના ઉપાય:}
\begin{itemize}
    \item સેમ્પલિંગ પહેલાં એન્ટી-એલિયાસિંગ ફિલ્ટર (લો-પાસ) વાપરવું
    \item નાઇક્વીસ્ટ રેટ કરતાં સેમ્પલિંગ રેટ વધારવી ($f_s > 2f_{max}$)
    \item સેમ્પલિંગ પહેલાં ઇનપુટ સિગ્નલની બેન્ડવિડ્થ મર્યાદિત કરવી
\end{itemize}
\end{solutionbox}

\begin{mnemonicbox}
"વધવ" - "વધારો, ધીમા, વિલ્ટર"
\end{mnemonicbox}

\questionmarks{3(b) OR}{4}{ટાઇમ ડોમેન અને ફ્રીક્વન્સી ડોમેનમાં નીચેના સિગ્નલ દોરો: 1) Sawtooth signal 2) Pulse signal.}

\begin{solutionbox}
\textbf{જવાબ}:

\textbf{1. Sawtooth Signal:}
\begin{center}
\begin{tikzpicture}[scale=0.7]
    % Time
    \begin{scope}
        \draw[->] (0,0) -- (4,0) node[right] {t};
        \draw[->] (0,0) -- (0,1.5) node[above] {A};
        \draw[thick] (0,0) -- (1,1) -- (1,0) -- (2,1) -- (2,0) -- (3,1) -- (3,0);
        \node[below] at (1,0) {T};
        \node[below] at (2,0) {2T};
        \node[below] at (2,-0.5) {ટાઇમ ડોમેન};
    \end{scope}
    
    % Freq
    \begin{scope}[xshift=5cm]
        \draw[->] (0,0) -- (4,0) node[right] {f};
        \draw[->] (0,0) -- (0,1.5) node[above] {Amp};
        \draw[thick] (0.5,0) -- (0.5,1.2);
        \draw[thick] (1.5,0) -- (1.5,0.6);
        \draw[thick] (2.5,0) -- (2.5,0.4);
        \draw[thick] (3.5,0) -- (3.5,0.3);
        
        \draw[dashed, gray] (0.5,1.2) -- (3.5,0.3);
        
        \node[below] at (0.5,0) {$f_0$};
        \node[below] at (1.5,0) {$2f_0$};
        \node[below] at (2,-0.5) {ફ્રીક્વન્સી ડોમેન};
    \end{scope}
\end{tikzpicture}
\captionof{figure}{Sawtooth Signal}
\end{center}

\textbf{2. Pulse Signal:}
\begin{center}
\begin{tikzpicture}[scale=0.7]
    % Time
    \begin{scope}
        \draw[->] (0,0) -- (4,0) node[right] {t};
        \draw[->] (0,0) -- (0,1.5) node[above] {A};
        \draw[thick] (0,0) -- (0.5,0) -- (0.5,1) -- (1,1) -- (1,0) 
                     -- (1.5,0) -- (1.5,1) -- (2,1) -- (2,0)
                     -- (2.5,0) -- (2.5,1) -- (3,1) -- (3,0) -- (3.5,0);
        \node[below] at (0.75,0) {T};
        \node[below] at (1.75,0) {2T};
        \node[below] at (2,-0.5) {ટાઇમ ડોમેન};
    \end{scope}
    
    % Freq
    \begin{scope}[xshift=5cm]
        \draw[->] (0,0) -- (4,0) node[right] {f};
        \draw[->] (0,0) -- (0,1.5) node[above] {Amp};
        
        \draw[domain=0.1:3.8, samples=100, blue] plot (\x, {abs(sin(300*\x)/(5*\x))});
        
        \node[below] at (2,-0.5) {ફ્રીક્વન્સી ડોમેન (Sinc)};
    \end{scope}
\end{tikzpicture}
\captionof{figure}{Pulse Signal}
\end{center}
\end{solutionbox}

\begin{mnemonicbox}
"સોડા" - "સોટૂથનો ડાઉન સ્લોપ, sinc function"
\end{mnemonicbox}

\questionmarks{3(c) OR}{7}{વેવફોર્મ સાથે PAM, PWM અને PPM ની સરખામણી કરો.}

\begin{solutionbox}
\textbf{જવાબ}:

\begin{center}
\captionof{table}{પલ્સ મોડ્યુલેશન પ્રકારોની સરખામણી}
\begin{tabulary}{\linewidth}{|L|L|L|L|}
\hline
\textbf{પેરામીટર} & \textbf{PAM} & \textbf{PWM} & \textbf{PPM} \\
\hline
\textbf{પૂરું નામ} & Pulse Amplitude Modulation & Pulse Width Modulation & Pulse Position Modulation \\
\hline
\textbf{બદલાતો પેરામીટર} & પલ્સની એમ્પ્લિટ્યુડ & પલ્સની પહોળાઈ/અવધિ & પલ્સની સ્થિતિ/સમય \\
\hline
\textbf{નોઇઝ ઇમ્યુનિટી} & નબળી & સારી & ઉત્તમ \\
\hline
\textbf{બેન્ડવિડ્થ} & ઓછી & વધારે & સૌથી વધારે \\
\hline
\textbf{પાવર કાર્યક્ષમતા} & નીચી & મધ્યમ & ઉંચી \\
\hline
\textbf{ડીમોડ્યુલેશન} & સરળ & મધ્યમ & જટિલ \\
\hline
\end{tabulary}
\end{center}

\textbf{વેવફોર્મ્સ:}
\begin{center}
\begin{tikzpicture}[scale=0.8]
    % Message
    \draw[gray, dashed] plot[domain=0:8, samples=100] (\x, {1 + 0.5*sin(180*\x/4)});
    \node[left] at (0,1) {મેસેજ};

    % PAM
    \begin{scope}[yshift=-2cm]
        \foreach \x in {0.5, 1.5, ..., 7.5} {
            \draw[thick] (\x,0) -- (\x, {1 + 0.5*sin(180*\x/4)});
        }
        \draw (0,0) -- (8,0);
        \node[left] at (0,0.5) {PAM};
    \end{scope}

    % PWM
    \begin{scope}[yshift=-4cm]
        \foreach \x in {0.5, 1.5, ..., 7.5} {
            \pgfmathsetmacro{\w}{0.2 + 0.2*sin(180*\x/4)}
            \draw[thick, fill=blue!20] (\x-\w, 0) rectangle (\x+\w, 1);
        }
        \draw (0,0) -- (8,0);
        \node[left] at (0,0.5) {PWM};
    \end{scope}

    % PPM
    \begin{scope}[yshift=-6cm]
        \foreach \x in {0.5, 1.5, ..., 7.5} {
             \pgfmathsetmacro{\s}{0.3*sin(180*\x/4)}
            \draw[thick] (\x+\s, 0) -- (\x+\s, 1);
        }
        \draw (0,0) -- (8,0);
        \node[left] at (0,0.5) {PPM};
    \end{scope}
\end{tikzpicture}
\captionof{figure}{પલ્સ મોડ્યુલેશન વેવફોર્મ્સ}
\end{center}
\end{solutionbox}

\begin{mnemonicbox}
"ઊપસ" - "ઊંચાઈ, પહોળાઈ, સ્થિતિ"
\end{mnemonicbox}

\questionmarks{4(a)}{3}{સમજાવો Space wave propagation.}

\begin{solutionbox}
\textbf{જવાબ}:

\textbf{સ્પેસ વેવ પ્રોપેગેશન}: એવું મોડ જ્યાં રેડિયો તરંગો નીચલા વાતાવરણ (ટ્રોપોસ્ફિયર) મારફતે સીધા અથવા જમીન પરાવર્તન દ્વારા પ્રવાસ કરે છે.

\begin{center}
\begin{tikzpicture}[scale=1]
    % Ground
    \draw[thick, brown] (-4,0) to[out=10, in=170] (4,0);
    \node at (0,-0.5) {પૃથ્વીની સપાટી};
    
    % TX Tower
    \draw[thick] (-3,0.5) -- (-3,2.5);
    \draw (-3.2, 0.5) -- (-2.8, 0.5);
    \node[above] at (-3,2.5) {TX};
    
    % RX Tower
    \draw[thick] (3,0.8) -- (3,2);
    \draw (2.8, 0.8) -- (3.2, 0.8);
    \node[above] at (3,2) {RX};
    
    % Direct Wave
    \draw[blue, thick, ->] (-3,2.5) -- (3,2);
    \node[above, blue] at (0,2.3) {ડાયરેક્ટ વેવ};
    
    % Reflected Wave
    \draw[red, dashed, ->] (-3,2.5) -- (0,0.2) -- (3,2);
    \node[below, red] at (0,1.5) {ગ્રાઉન્ડ રિફ્લેક્ટેડ વેવ};
\end{tikzpicture}
\captionof{figure}{સ્પેસ વેવ પ્રોપેગેશન}
\end{center}

\textbf{લક્ષણો:}
\begin{itemize}
    \item આવૃત્તિ રેન્જ: VHF, UHF (30 MHz - 3 GHz)
    \item સીધી લાઇન-ઓફ-સાઇટ અંતર સુધી મર્યાદિત
    \item રેન્જ = $4.12(\sqrt{h_1} + \sqrt{h_2})$ km (જ્યાં $h_1, h_2$ = મીટરમાં ઊંચાઈઓ)
    \item ભૂમિ, ઇમારતો અને વાતાવરણીય પરિસ્થિતિઓથી પ્રભાવિત
\end{itemize}
\end{solutionbox}

\begin{mnemonicbox}
"સીધે સીધા" - "સીધી લાઇન જમીન ઉપર"
\end{mnemonicbox}

\questionmarks{4(b)}{4}{ડિફરન્શિયલ પીસીએમ (ડીપીસીએમ) ટ્રાન્સમીટરનું કાર્ય સમજાવો.}

\begin{solutionbox}
\textbf{જવાબ}:

\begin{center}
\begin{tikzpicture}[auto, >=latex, thick]
    \node (input) {$x(nT)$};
    \node [draw, circle, right of=input, node distance=1.5cm] (sub) {$\Sigma$};
    \node [gtu block, right of=sub, node distance=2.5cm] (quant) {ક્વોન્ટાઇઝર};
    \node [gtu block, right of=quant, node distance=3cm] (enc) {એન્કોડર};
    \node [right of=enc, node distance=2cm] (out) {DPCM};
    
    \node [draw, circle, below of=quant, node distance=2.5cm] (adder) {$\Sigma$};
    \node [gtu block, left of=adder, node distance=3cm] (pred) {પ્રેડિક્ટર};
    
    \draw [gtu arrow] (input) -- node[pos=0.9] {$+$} (sub);
    \draw [gtu arrow] (sub) -- node {$e(nT)$} (quant);
    \draw [gtu arrow] (quant) -- node {$e_q(nT)$} (enc);
    \draw [gtu arrow] (enc) -- (out);
    
    % Feedback path
    \draw [gtu arrow] (quant) -- (adder);
    \draw [gtu arrow] (adder) -- node {$x_q(nT)$} (pred);
    \draw [gtu arrow] (pred) -| node[near end] {$-$} node[midway, above] {$\hat{x}(nT)$} (sub);
    \draw [gtu arrow] (pred) -- (adder);
    
\end{tikzpicture}
\captionof{figure}{DPCM ટ્રાન્સમીટર}
\end{center}

\textbf{DPCM ટ્રાન્સમીટરની કાર્યપદ્ધતિ:}
\begin{itemize}
    \item \textbf{પ્રેડિક્ટર}: અગાઉના સેમ્પલ્સના આધારે વર્તમાન સેમ્પલનો અંદાજ લગાવે છે
    \item \textbf{સબટ્રેક્ટર}: વાસ્તવિક અને અનુમાનિત મૂલ્ય વચ્ચેનો તફાવત ગણે છે
    \item \textbf{ક્વોન્ટાઇઝર}: તફાવત સિગ્નલને ડિસ્ક્રીટ લેવલમાં રૂપાંતરિત કરે છે
    \item \textbf{એન્કોડર}: ક્વોન્ટાઇઝ્ડ મૂલ્યોને બાઇનરી કોડમાં રૂપાંતરિત કરે છે
    \item \textbf{ફીડબેક લૂપ}: રિસીવર તેને જોશે તે રીતે સિગ્નલ પુનઃનિર્માણ કરે છે
\end{itemize}

\textbf{ફાયદો}: ફક્ત તફાવત સિગ્નલ પ્રસારિત થાય છે, જેને ઓછા બિટ્સની જરૂર પડે છે.
\end{solutionbox}

\begin{mnemonicbox}
"પતાએ" - "પ્રેડિક્ટર, તફાવત, એન્કોડ"
\end{mnemonicbox}

\questionmarks{4(c)}{7}{વિગતોમાં ડેલ્ટા મોડ્યુલેટર સમજાવો, slop overload noise અને granular noise પણ સમજાવો.}

\begin{solutionbox}
\textbf{જવાબ}:

\textbf{ડેલ્ટા મોડ્યુલેશન (DM)}: ડિફરન્શિયલ PCM નું સૌથી સરળ સ્વરૂપ જ્યાં તફાવત સિગ્નલને માત્ર 1 બિટ સાથે એન્કોડ કરવામાં આવે છે.

\begin{center}
\begin{tikzpicture}[auto, >=latex, thick]
    \node (input) {$x(t)$};
    \node [draw, circle, right of=input, node distance=1.5cm] (sub) {$\Sigma$};
    \node [gtu block, right of=sub, node distance=2.5cm] (comp) {કમ્પેરેટર};
    \node [gtu block, right of=comp, node distance=2.5cm] (sample) {સેમ્પલર};
    \node [right of=sample, node distance=2cm] (out) {DM Out};
    
    \node [gtu block, below of=comp, node distance=2cm] (int) {ઇન્ટીગ્રેટર};
    
    \draw [gtu arrow] (input) -- (sub);
    \draw [gtu arrow] (sub) -- (comp);
    \draw [gtu arrow] (comp) -- (sample);
    \draw [gtu arrow] (sample) -- (out);
    
    \draw [gtu arrow] (sample) |- (int);
    \draw [gtu arrow] (int) -| node[near end] {$-$} (sub);
\end{tikzpicture}
\captionof{figure}{ડેલ્ટા મોડ્યુલેટર}
\end{center}

\textbf{કાર્ય સિદ્ધાંત:}
\begin{itemize}
    \item ઇનપુટ સિગ્નલની અગાઉના આઉટપુટના ઇન્ટીગ્રેટેડ વર્ઝન સાથે તુલના કરે છે
    \item જો ઇનપુટ > ઇન્ટીગ્રેટેડ વેલ્યુ: 1 પ્રસારિત કરે
    \item જો ઇનપુટ < ઇન્ટીગ્રેટેડ વેલ્યુ: 0 પ્રસારિત કરે
    \item સ્ટેપ સાઇઝ ($\delta$) ફિક્સ્ડ હોય છે
\end{itemize}

\textbf{ડેલ્ટા મોડ્યુલેશનમાં નોઇઝ:}

\begin{center}
\captionof{table}{ડેલ્ટા મોડ્યુલેશનમાં નોઇઝ}
\begin{tabulary}{\linewidth}{|L|L|L|}
\hline
\textbf{નોઇઝનો પ્રકાર} & \textbf{કારણ} & \textbf{ઉપાય} \\
\hline
\textbf{સ્લોપ ઓવરલોડ નોઇઝ} & ઇનપુટ સિગ્નલ $\delta$ ટ્રેક કરી શકે તેના કરતાં ઝડપથી બદલાય છે & સ્ટેપ સાઇઝ અથવા સેમ્પલિંગ ફ્રીક્વન્સી વધારો \\
\hline
\textbf{ગ્રેન્યુલર નોઇઝ} & ધીમે ધીમે બદલાતા સિગ્નલ્સ માટે સ્ટેપ સાઇઝ ખૂબ મોટી છે & સ્ટેપ સાઇઝ ઘટાડો \\
\hline
\end{tabulary}
\end{center}
\end{solutionbox}

\begin{mnemonicbox}
"સ્લોગ્રે" - "સ્લોપ અને ગ્રેન્યુલર ડેલ્ટામાં"
\end{mnemonicbox}

\questionmarks{4(a) OR}{3}{સમજાવો Ground wave propagation.}

\begin{solutionbox}
\textbf{જવાબ}:

\textbf{ગ્રાઉન્ડ વેવ પ્રોપેગેશન}: રેડિયો તરંગ પ્રસારણ જે પૃથ્વીની વક્રતાને અનુસરે છે.

\begin{center}
\begin{tikzpicture}[scale=1]
    % Ground
    \draw[thick, brown] (-4,0) to[out=10, in=170] (4,0);
    \node at (0,-0.5) {પૃથ્વીની સપાટી};
    
    % TX Tower
    \draw[thick] (-3,0.5) -- (-3,2.5);
    \node[above] at (-3,2.5) {TX};
    
    % RX Tower
    \draw[thick] (3,0.8) -- (3,2);
    \node[above] at (3,2) {RX};
    
    % Ground Wave
    \draw[blue, thick, ->] (-3,0.5) to[out=20, in=160] (3,0.8);
    \node[above, blue] at (0,1.2) {ગ્રાઉન્ડ વેવ};
    
    % Vertical Polarization lines
    \foreach \x in {-2, -1, 0, 1, 2} {
        \draw[blue, ->] (\x, {0.6 + 0.1*\x}) -- (\x, {1.0 + 0.1*\x});
    }
\end{tikzpicture}
\captionof{figure}{ગ્રાઉન્ડ વેવ પ્રોપેગેશન}
\end{center}

\textbf{લક્ષણો:}
\begin{itemize}
    \item આવૃત્તિ રેન્જ: LF, MF (30 kHz - 3 MHz)
    \item પૃથ્વીની સપાટી સાથે પ્રસરે છે (ઊભી રીતે ધ્રુવીકરણ)
    \item રેન્જ ટ્રાન્સમીટર પાવર, જમીન વાહકતા, આવૃત્તિ પર આધાર રાખે છે
    \item સિગ્નલની તાકાત અંતર અને આવૃત્તિ સાથે ઘટે છે
    \item AM બ્રોડકાસ્ટિંગ, મરીન કોમ્યુનિકેશન માટે વપરાય છે
\end{itemize}
\end{solutionbox}

\begin{mnemonicbox}
"જઅઆ" - "જમીન આગળ આવે અને અનુસરે"
\end{mnemonicbox}

\questionmarks{4(b) OR}{4}{ADM ટ્રાન્સમીટર સમજાવો.}

\begin{solutionbox}
\textbf{જવાબ}:

\textbf{એડેપ્ટિવ ડેલ્ટા મોડ્યુલેશન (ADM)}: ડીએમનું સુધારેલું સંસ્કરણ જ્યાં સ્ટેપ સાઇઝ સિગ્નલ લક્ષણો અનુસાર બદલાય છે.

\begin{center}
\begin{tikzpicture}[auto, >=latex, thick]
    \node (input) {$x(t)$};
    \node [draw, circle, right of=input, node distance=1.5cm] (sub) {$\Sigma$};
    \node [gtu block, right of=sub, node distance=2.5cm] (quant) {ક્વોન્ટાઇઝર};
    \node [right of=quant, node distance=2.5cm] (out) {ADM Out};
    
    \node [gtu block, below of=quant, node distance=2cm] (ctrl) {સ્ટેપ સાઇઝ\\કંટ્રોલ};
    \node [gtu block, left of=ctrl, node distance=2.5cm] (int) {ઇન્ટીગ્રેટર};
    
    \draw [gtu arrow] (input) -- (sub);
    \draw [gtu arrow] (sub) -- (quant);
    \draw [gtu arrow] (quant) -- (out);
    
    \draw [gtu arrow] (quant) -- (ctrl);
    \draw [gtu arrow] (ctrl) -- (int);
    \draw [gtu arrow] (int) -| node[near end] {$-$} (sub);
\end{tikzpicture}
\captionof{figure}{ADM ટ્રાન્સમીટર}
\end{center}

\textbf{ADM ટ્રાન્સમીટરની કાર્યપદ્ધતિ:}
\begin{itemize}
    \item \textbf{મૂળભૂત ઓપરેશન}: સ્ટાન્ડર્ડ DM જેવું જ
    \item \textbf{સ્ટેપ સાઇઝ કંટ્રોલ}: તાજેતરના આઉટપુટ બિટ્સનું વિશ્લેષણ કરે છે
    \item \textbf{એડેપ્ટેશન લોજિક}:
    \begin{itemize}
        \item જો સળંગ બિટ્સ સમાન હોય: સ્ટેપ સાઇઝ વધારો
        \item જો સળંગ બિટ્સ વૈકલ્પિક હોય: સ્ટેપ સાઇઝ ઘટાડો
    \end{itemize}
\end{itemize}

\textbf{DM કરતા ફાયદાઓ:}
\begin{itemize}
    \item સ્લોપ ઓવરલોડ અને ગ્રેન્યુલર નોઇઝ બંને ઘટાડે છે
    \item સિગ્નલ ટ્રેકિંગ વધુ સારું
    \item SNR માં સુધારો
\end{itemize}
\end{solutionbox}

\begin{mnemonicbox}
"સચક" - "સ્ટેપ, ચેક, કોડિંગ"
\end{mnemonicbox}

\questionmarks{4(c) OR}{7}{મૂળભૂત PCM-TDM સિસ્ટમનો બ્લોક ડાયાગ્રામ સમજાવો.}

\begin{solutionbox}
\textbf{જવાબ}:

\textbf{PCM-TDM સિસ્ટમ}: એક જ ચેનલ પર મલ્ટિપલ ડિજિટલ સિગ્નલ્સ પ્રસારિત કરવા માટે પલ્સ કોડ મોડ્યુલેશનને ટાઇમ ડિવિઝન મલ્ટિપ્લેક્સિંગ સાથે જોડે છે.

\begin{center}
\begin{tikzpicture}[node distance=2.5cm, auto, >=latex, thick, scale=0.8, transform shape]
    % Transmitter
    \node [gtu block] (in1) {ઇનપુટ 1};
    \node [gtu block, below of=in1, node distance=1.5cm] (in2) {ઇનપુટ 2};
    \node [gtu block, below of=in2, node distance=1.5cm] (in3) {ઇનપુટ n};
    
    \node [gtu block, right of=in2, node distance=3cm] (mux) {મલ્ટિપ્લેક્સર};
    \node [gtu block, right of=mux] (pcm) {PCM\\એન્કોડર};
    \node [gtu block, right of=pcm] (tx) {TX};
    
    \draw [gtu arrow] (in1) -- (mux);
    \draw [gtu arrow] (in2) -- (mux);
    \draw [gtu arrow] (in3) -- (mux);
    \draw [gtu arrow] (mux) -- (pcm);
    \draw [gtu arrow] (pcm) -- (tx);
    
    % Channel
    \node [right of=tx, node distance=2cm] (ch) {};
    \draw [dashed] (tx) -- (ch);
    
    % Receiver
    \node [gtu block, right of=ch, node distance=2cm] (rx) {RX};
    \node [gtu block, right of=rx] (dec) {PCM\\ડિકોડર};
    \node [gtu block, right of=dec] (demux) {ડિમલ્ટિ-\\પ્લેક્સર};
    
    \node [gtu block, right of=demux, node distance=3cm] (out2) {આઉટપુટ 2};
    \node [gtu block, above of=out2, node distance=1.5cm] (out1) {આઉટપુટ 1};
    \node [gtu block, below of=out2, node distance=1.5cm] (out3) {આઉટપુટ n};

    \draw [dashed] (ch) -- (rx);
    \draw [gtu arrow] (rx) -- (dec);
    \draw [gtu arrow] (dec) -- (demux);
    \draw [gtu arrow] (demux) -- (out1);
    \draw [gtu arrow] (demux) -- (out2);
    \draw [gtu arrow] (demux) -- (out3);
\end{tikzpicture}
\captionof{figure}{PCM-TDM સિસ્ટમ બ્લોક ડાયાગ્રામ}
\end{center}

\textbf{PCM-TDM સિસ્ટમની કાર્યપદ્ધતિ:}
\begin{itemize}
    \item \textbf{ટ્રાન્સમીટર}:
    \begin{itemize}
        \item મલ્ટિપલ એનાલોગ સિગ્નલ્સ એક સાથે સેમ્પલ થાય છે
        \item સેમ્પલ્સ ટાઇમ-મલ્ટિપ્લેક્સ્ડ થઈને સિંગલ સ્ટ્રીમમાં બદલાય છે
        \item સ્ટ્રીમ ક્વોન્ટાઇઝ્ડ અને PCM ફોર્મેટમાં એન્કોડેડ થાય છે
        \item સિન્ક્રોનાઇઝેશન માટે ફ્રેમિંગ બિટ્સ ઉમેરાય છે
    \end{itemize}
    \item \textbf{રિસીવર}:
    \begin{itemize}
        \item અલાઇનમેન્ટ માટે ફ્રેમ સિન્ક શોધાય છે
        \item PCM સ્ટ્રીમ ડિકોડ થઈને સેમ્પલ્સ રિકવર થાય છે
        \item ડિમલ્ટિપ્લેક્સર વ્યક્તિગત ચેનલના સેમ્પલ્સને અલગ કરે છે
        \item લો-પાસ ફિલ્ટર્સ મૂળ એનાલોગ સિગ્નલ્સનું પુનઃનિર્માણ કરે છે
    \end{itemize}
\end{itemize}
\end{solutionbox}

\begin{mnemonicbox}
"સેકોમલ" - "સેમ્પલિંગ, કોડિંગ, અને મલ્ટિપ્લેક્સિંગ"
\end{mnemonicbox}

\questionmarks{5(a)}{3}{TDM અને FDM ની સરખામણી કરો.}

\begin{solutionbox}
\textbf{જવાબ}:

\begin{center}
\captionof{table}{TDM અને FDM ની સરખામણી}
\begin{tabulary}{\linewidth}{|L|L|L|}
\hline
\textbf{લક્ષણ} & \textbf{TDM (Time Division Multiplexing)} & \textbf{FDM (Frequency Division Multiplexing)} \\
\hline
\textbf{વ્યાખ્યા} & સિગ્નલ્સ જુદા જુદા સમયે એક જ ફ્રીક્વન્સી પર મોકલાય છે & સિગ્નલ્સ એક જ સમયે જુદી જુદી ફ્રીક્વન્સી પર મોકલાય છે \\
\hline
\textbf{સિગ્નલ પ્રકાર} & ડિજિટલ સિગ્નલ્સ માટે વધુ યોગ્ય & એનાલોગ સિગ્નલ્સ માટે વધુ યોગ્ય \\
\hline
\textbf{સિન્ક્રોનાઇઝેશન} & ઘણું જરૂરી (પલ્સ સિન્ક) & જરૂરી નથી (ફક્ત કેરિયર સિન્ક) \\
\hline
\textbf{જટિલતા} & ઓછી & વધુ (ફિલ્ટર્સની જરૂર છે) \\
\hline
\textbf{નોઇઝ} & ઓછો પ્રભાવ & વધુ પ્રભાવ (ક્રોસ ટોક) \\
\hline
\end{tabulary}
\end{center}
\end{solutionbox}

\begin{mnemonicbox}
"T-Time, F-Freq" - "TDM માં સમય વહેંચાય, FDM માં ફ્રીક્વન્સી"
\end{mnemonicbox}

\questionmarks{5(b)}{4}{Fiber optic cable નું કન્સ્ટ્રક્શન સમજાવો.}

\begin{solutionbox}
\textbf{જવાબ}:

\begin{center}
\begin{tikzpicture}[scale=0.8]
    % Core
    \draw[fill=cyan!30] (0,0) ellipse (0.5 and 0.5);
    \draw[top color=cyan!30, bottom color=cyan!30] (0,0.5) -- (6,0.5) arc(90:-90:0.5 and 0.5) -- (0,-0.5) arc(-90:-270:0.5 and 0.5);
    \node at (7,0) {કોર (Core)};
    \draw[->] (6.2,0) -- (5.5,0);

    % Cladding
    \draw[fill=gray!30] (0,0) ellipse (1 and 1);
    \draw[top color=gray!30, bottom color=gray!30] (0,1) -- (5,1) arc(90:-90:1 and 1) -- (0,-1) arc(-90:-270:1 and 1);
    \node at (7,0.8) {ક્લેડિંગ (Cladding)};
    \draw[->] (6.2,0.8) -- (4.5,0.8);

    % Coating/Buffer
    \draw[fill=orange!30] (0,0) ellipse (1.5 and 1.5);
    \draw[top color=orange!30, bottom color=orange!30] (0,1.5) -- (4,1.5) arc(90:-90:1.5 and 1.5) -- (0,-1.5) arc(-90:-270:1.5 and 1.5);
    \node at (7,1.5) {બફર કોટિંગ (Buffer)};
    \draw[->] (6.2,1.5) -- (3.5,1.5);
    
    % Core visible at front
    \draw[fill=cyan!50] (0,0) ellipse (0.5 and 0.5);
\end{tikzpicture}
\captionof{figure}{ફાઈબર ઓપ્ટિક કેબલ રચના}
\end{center}

\textbf{મુખ્ય ભાગો:}
\begin{itemize}
    \item \textbf{કોર (Core)}:
    \begin{itemize}
        \item મધ્ય ભાગ જ્યાંથી પ્રકાશ પ્રવાસ કરે છે
        \item શુદ્ધ સિલિકા/ગ્લાસથી બનેલું
        \item ઉચ્ચ રિફ્રેક્ટિવ ઇન્ડેક્સ ($n_1$)
    \end{itemize}
    \item \textbf{ક્લેડિંગ (Cladding)}:
    \begin{itemize}
        \item કોરને ઘેરે છે
        \item કોર કરતા ઓછો રિફ્રેક્ટિવ ઇન્ડેક્સ ($n_2 < n_1$)
        \item પ્રકાશને કોરમાં પાછો પરાવર્તિત કરે છે (TIR)
    \end{itemize}
    \item \textbf{બફર/જેકેટ}:
    \begin{itemize}
        \item પ્લાસ્ટિકનું રક્ષણાત્મક આવરણ
        \item ભૌતિક નુકસાન અને ભેજથી બચાવે છે
    \end{itemize}
\end{itemize}
\end{solutionbox}

\begin{mnemonicbox}
"CCB" - "Core, Cladding, Buffer"
\end{mnemonicbox}

\questionmarks{5(c)}{7}{ઓપ્ટિકલ ફાઈબર કોમ્યુનિકેશનનો બ્લોક ડાયાગ્રામ દોરો અને સમજાવો.}

\begin{solutionbox}
\textbf{જવાબ}:

\begin{center}
\begin{tikzpicture}[node distance=2.2cm, auto, >=latex, thick, scale=0.9, transform shape]
    \node [gtu block] (info) {માહિતી\\સ્ત્રોત};
    \node [gtu block, right of=info] (tx) {ઇલેક્ટ્રિકલ\\ટ્રાન્સમીટર};
    \node [gtu block, right of=tx] (source) {ઓપ્ટિકલ\\સોર્સ (LED/Laser)};
    \node [gtu block, right of=source, node distance=3cm] (cable) {ઓપ્ટિકલ\\ફાઈબર કેબલ};
    \node [gtu block, right of=cable, node distance=3cm] (det) {ઓપ્ટિકલ\\ડિટેક્ટર (Diode)};
    \node [gtu block, right of=det] (rx) {ઇલેક્ટ્રિકલ\\રીસીવર};
    
    \draw [gtu arrow] (info) -- (tx);
    \draw [gtu arrow] (tx) -- (source);
    \draw [gtu arrow] (source) -- node {Light} (cable);
    \draw [gtu arrow] (cable) -- node {Light} (det);
    \draw [gtu arrow] (det) -- (rx);
\end{tikzpicture}
\captionof{figure}{ઓપ્ટિકલ ફાઈબર કોમ્યુનિકેશન}
\end{center}

\textbf{કાર્યપદ્ધતિ:}
\begin{itemize}
    \item \textbf{ઇલેક્ટ્રિકલ ટ્રાન્સમીટર}: ઇનપુટ સિગ્નલને ડ્રાઇવ કરન્ટમાં ફેરવે છે
    \item \textbf{ઓપ્ટિકલ સોર્સ}: ઇલેક્ટ્રિકલ સિગ્નલને લાઈટ પલ્સમાં ફેરવે છે (E/O રૂપાંતર)
    \item \textbf{ઓપ્ટિકલ ફાઈબર}: ટોટલ ઇન્ટરનલ રિફ્લેક્શન દ્વારા પ્રકાશ વહન કરે છે
    \item \textbf{ઓપ્ટિકલ ડિટેક્ટર}: પ્રકાશને પાછું ઇલેક્ટ્રિકલ સિગ્નલમાં ફેરવે છે (O/E રૂપાંતર) (દા.ત. Photodiode)
    \item \textbf{ઇલેક્ટ્રિકલ રીસીવર}: સિગ્નલને એમ્પ્લીફાય અને આકાર આપે છે
\end{itemize}

\textbf{ફાયદા}: ઉચ્ચ બેન્ડવિડ્થ, ઓછું લોસ, ઇલેક્ટ્રોમેગ્નેટિક દખલગીરી (EMI) થી મુક્ત.
\end{solutionbox}

\begin{mnemonicbox}
"ET-OS-Cable-OD-ER" - "Electrical TX, Optical Source, Cable, Optical Detector, Electrical RX"
\end{mnemonicbox}

\questionmarks{5(a) OR}{3}{Numerical Aperture સમજાવો.}

\begin{solutionbox}
\textbf{જવાબ}:

\textbf{ન્યુમેરિકલ એપરચર (NA)}: ઓપ્ટિકલ ફાઈબરની પ્રકાશ એકત્ર કરવાની ક્ષમતાનું માપ.

\begin{itemize}
    \item તે ફાઈબરના એક્સેપ્ટન્સ એન્ગલ ($\theta_a$) નું સાઈન છે
    \item સૂત્ર: $NA = \sin \theta_a = \sqrt{n_1^2 - n_2^2}$
    \item જ્યાં $n_1$ = કોર રિફ્રેક્ટિવ ઇન્ડેક્સ, $n_2$ = ક્લેડિંગ રિફ્રેક્ટિવ ઇન્ડેક્સ
    \item NA જેટલું વધારે, તેટલી વધારે પ્રકાશ એકત્ર કરવાની ક્ષમતા
\end{itemize}

\begin{center}
\begin{tikzpicture}[scale=1]
    \draw[thick] (0,0) -- (4,0); % Axis
    \draw[thick, cyan] (0,-1) rectangle (4,1); % Fiber
    \node at (3.5,0.5) {$n_1$};
    
    \draw[->, thick, red] (-1, -0.5) -- (0,0); % Ray
    \draw[dashed] (-1,0) -- (0,0);
    \draw (-0.5,0) arc(180:206:0.5);
    \node at (-0.8, -0.2) {$\theta_a$};
    
    \node[below] at (2,-1.2) {Acceptance Cone};
\end{tikzpicture}
\end{center}
\end{solutionbox}

\begin{mnemonicbox}
"ના. સાઈન થીટા" - "NA = sin(theta)"
\end{mnemonicbox}

\questionmarks{5(b) OR}{4}{PWM જનરેશન અને ડિમોડ્યુલેશન સમજાવો.}

\begin{solutionbox}
\textbf{જવાબ}:

\textbf{PWM જનરેશન:}
\begin{itemize}
    \item કમ્પેરેટરનો ઉપયોગ કરીને જનરેટ થાય છે
    \item સાઈન વેવ (મેસેજ) ની આરી જેવા (Sawtooth) વેવ સાથે તુલના કરવામાં આવે છે
    \item જ્યારે મેસેજ > Sawtooth $\rightarrow$ Output High
    \item જ્યારે મેસેજ < Sawtooth $\rightarrow$ Output Low
\end{itemize}

\textbf{PWM ડિમોડ્યુલેશન:}
\begin{itemize}
    \item સાદી રીત: PWM સિગ્નલને ઇન્ટીગ્રેટર સર્કિટમાં આપવું
    \item ઇન્ટીગ્રેટર પલ્સની પહોળાઈના પ્રમાણમાં વોલ્ટેજ ઉત્પન્ન કરે છે
    \item ત્યારબાદ લો-પાસ ફિલ્ટર સ્મૂથ સિગ્નલ મેળવવા માટે વપરાય છે
\end{itemize}

\begin{center}
\begin{tikzpicture}[auto, >=latex, thick]
    \node (pwm) {PWM};
    \node [gtu block, right of=pwm, node distance=2.5cm] (int) {ઇન્ટીગ્રેટર};
    \node [gtu block, right of=int, node distance=2.5cm] (lpf) {LPF};
    \node [right of=lpf, node distance=2cm] (out) {Output};
    
    \draw [gtu arrow] (pwm) -- (int);
    \draw [gtu arrow] (int) -- (lpf);
    \draw [gtu arrow] (lpf) -- (out);
\end{tikzpicture}
\captionof{figure}{PWM ડિમોડ્યુલેટર}
\end{center}
\end{solutionbox}

\begin{mnemonicbox}
"જનરેટ કમ્પેર, ડિમોડ ઇન્ટીગ્રેટ"
\end{mnemonicbox}

\questionmarks{5(c) OR}{7}{સેટેલાઇટ કોમ્યુનિકેશનનો બ્લોક ડાયાગ્રામ દોરો અને સમજાવો.}

\begin{solutionbox}
\textbf{જવાબ}:

\begin{center}
\begin{tikzpicture}[scale=0.9]
    % Earth Station TX
    \node[draw, rectangle] (tx) at (0,0) {અર્થ સ્ટેશન TX};
    \draw[thick] (0,0.5) -- (0,1.5);
    \draw (-0.5,1.5) arc(180:360:0.5); % Dish
    
    % Satellite
    % Replaced missing image with a drawing
    \node at (4,4) (sat) {}; 
    \draw[fill=gray!20] (3.5,3.5) rectangle (4.5,4.5);
    \node at (4,4) {Sat};
    
    % Earth Station RX
    \node[draw, rectangle] (rx) at (8,0) {અર્થ સ્ટેશન RX};
    \draw[thick] (8,0.5) -- (8,1.5);
    \draw (7.5,1.5) arc(180:360:0.5); % Dish
    
    % Up-link
    \draw[->, blue, decorate, decoration={snake}] (0,1.6) -- node[left,sloped] {Up-link (6 GHz)} (3.4,3.6);
    
    % Down-link
    \draw[->, red, decorate, decoration={snake}] (4.6,3.6) -- node[right,sloped] {Down-link (4 GHz)} (8,1.6);
    
\end{tikzpicture}
\captionof{figure}{સેટેલાઇટ કોમ્યુનિકેશન}
\end{center}

\textbf{ઘટકો:}
\begin{enumerate}
    \item \textbf{અર્થ સ્ટેશન (TX)}: ઉચ્ચ શક્તિશાળી સિગ્નલ આકાશમાં મોકલે છે (Up-link)
    \item \textbf{સેટેલાઇટ (Transponder)}:
    \begin{itemize}
        \item સિગ્નલ પ્રાપ્ત કરે છે
        \item ફ્રીક્વન્સી બદલે છે (Up-link $\rightarrow$ Down-link)
        \item સિગ્નલ એમ્પ્લીફાય કરે છે
        \item પૃથ્વી પર પાછું મોકલે છે
    \end{itemize}
    \item \textbf{અર્થ સ્ટેશન (RX)}: નબળા સિગ્નલને પ્રાપ્ત કરે છે અને પ્રોસેસ કરે છે (Down-link)
\end{enumerate}

\textbf{ફ્રીક્વન્સી:} Up-link ફ્રીક્વન્સી હંમેશા Down-link કરતા વધુ હોય છે (દા.ત. 6/4 GHz).
\end{solutionbox}

\begin{mnemonicbox}
"ઉપર મોંઘું, નીચે સસ્તું" - "Up freq ઊંચી, Down freq નીચી"
\end{mnemonicbox}

\end{document}



