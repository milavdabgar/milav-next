\documentclass[10pt,a4paper]{article}

% content/resources/templates/preamble.tex
\usepackage[margin=0.6in]{geometry}
\author{Milav Dabgar}
\usepackage{amsmath,amssymb,amsthm}
\usepackage{booktabs}
\usepackage{multirow}
\usepackage{xcolor}
\usepackage{tcolorbox}
\tcbuselibrary{breakable,skins}
\usepackage[colorlinks=true,linkcolor=blue]{hyperref}
\usepackage{titlesec}
\usepackage{enumitem}
\usepackage{tikz}
\usepackage{pgfplots}
\usepackage{circuitikz}
\usepackage[version=4]{mhchem}
\usepackage{longtable}
\usepackage{array}
\usepackage{float}
\usepackage{caption}
\usepackage{listings}

\lstset{
  basicstyle=\small\ttfamily,
  breaklines=true,
  breakatwhitespace=false,
  postbreak=\mbox{\textcolor{red}{$\hookrightarrow$}\space},
  float=false,
  numbers=left,
  numberstyle=\tiny\color{gray},
  numbersep=10pt,
  xleftmargin=2em,
  keywordstyle=\color{blue},
  commentstyle=\color{green!60!black},
  stringstyle=\color{purple},
  backgroundcolor=\color{gray!5},
  showstringspaces=false,
  tabsize=2,
  captionpos=b,
  keepspaces=true,
  columns=flexible
}

\pgfplotsset{compat=1.18}
\usetikzlibrary{shapes,arrows,positioning,calc,patterns,decorations.pathmorphing,decorations.markings,arrows.meta}

% Color scheme
\definecolor{headcolor}{RGB}{0,102,204}
\definecolor{keycolor}{RGB}{220,20,60}
\definecolor{solutioncolor}{RGB}{34,139,34}
\definecolor{mnemoniccolor}{RGB}{148,0,211}
\definecolor{codecolor}{RGB}{0,0,100}

% Spacing
\setlength{\parskip}{3pt}
\setlist[itemize]{nosep}
\setlist[enumerate]{nosep}

% Title formatting
\titleformat{\section}{\Large\bfseries\color{headcolor}}{\thesection}{1em}{}
\titleformat{\subsection}{\large\bfseries\color{headcolor}}{\thesubsection}{1em}{}

% Pandoc tightlist compatibility
\providecommand{\tightlist}{%
  \setlength{\itemsep}{0pt}\setlength{\parskip}{0pt}}

% Pandoc longtable compatibility
\newcounter{none}
\def\thenone{}


% content/resources/templates/gujarati-boxes.tex
\usepackage{fontspec}
\usepackage{polyglossia}

% Set Gujarati as main language (document is primarily in Gujarati)
% Note: gloss-gujarati.ldf doesn't exist in polyglossia, but it will use hyphenation patterns
\setdefaultlanguage{gujarati}
\setotherlanguage{english}

% Configure Gujarati font properly
% Use Language=Default to prevent polyglossia from trying to add language-specific features
% that don't exist for Gujarati, which causes "empty feature" warnings
\newfontfamily\gujaratifont[Script=Gujarati,AutoFakeBold=2.5,AutoFakeSlant=0.3]{Noto Sans Gujarati}
\setmainfont[Script=Gujarati,AutoFakeBold=2.5,AutoFakeSlant=0.3]{Noto Sans Gujarati}
% Use Noto Sans Gujarati for monospace to support Gujarati in text
\setmonofont[Scale=0.9]{Noto Sans Gujarati}

% Configure English to use the same font
\newfontfamily\englishfont[Script=Gujarati,AutoFakeBold=2.5,AutoFakeSlant=0.3]{Noto Sans Gujarati}

% Translations for polyglossia
\gappto\captionsgujarati{
  \renewcommand{\tablename}{કોષ્ટક}
  \renewcommand{\figurename}{આકૃતિ}
}

% Helper for TikZ nodes to ensure Gujarati font
\newcommand{\gu}[1]{{\gujaratifont #1}}

% Custom environments
\newtcolorbox{solutionbox}{
    breakable,
    enhanced,
    colback=solutioncolor!5!white,
    colframe=solutioncolor!75!black,
    fonttitle=\bfseries,
    title=જવાબ
}

\newtcolorbox{solutionboxnobreak}{
 colback=solutioncolor!5!white,
 colframe=solutioncolor!75!black,
 fonttitle=\bfseries,
 title=જવાબ
}

\newtcolorbox{keyformula}{
 breakable,
 enhanced,
 colback=keycolor!5!white,
 colframe=keycolor!75!black,
 fonttitle=\bfseries,
 title=રાસાયણિક સમીકરણ/સૂત્ર
}

\newtcolorbox{mnemonicbox}{
 breakable,
 enhanced,
 colback=mnemoniccolor!5!white,
 colframe=mnemoniccolor!75!black,
 fonttitle=\bfseries,
 title=મેમરી ટ્રીક
}


\begin{document}

\begin{center}
{\Huge\bfseries\color{headcolor} Subject Name (Gujarati)}\\[5pt]
{\LARGE 1333201 -- Winter 2024}\\[3pt]
{\large Semester 1 Study Material}\\[3pt]
{\normalsize\textit{Detailed Solutions and Explanations}}
\end{center}

\vspace{10pt}

\subsection*{પ્રશ્ન 1(અ) [3
ગુણ]}\label{uxaaauxab0uxab6uxaa8-1uxa85-3-uxa97uxaa3}

\textbf{મોડ્યુલેશનશું છે? તેની શું જરૂર છે?}

\begin{solutionbox}
મોડ્યુલેશન એ એક પ્રક્રિયા છે જેમાં માહિતી ધરાવતા મોડ્યુલેટિંગ સિગ્નલ
દ્વારા ઉચ્ચ આવૃત્તિના કેરિયર સિગ્નલના એક અથવા વધુ ગુણધર્મોમાં ફેરફાર કરવામાં આવે છે.


{\def\LTcaptype{none} % do not increment counter
\vspace{-5pt}
\captionof{table}{મોડ્યુલેશનની જરૂરિયાત}
\vspace{-10pt}
\begin{longtable}[]{@{}ll@{}}
\toprule\noalign{}
કારણ & સમજૂતી \\
\midrule\noalign{}
\endhead
\bottomrule\noalign{}
\endlastfoot
એન્ટેના સાઇઝ & એન્ટેનાના કદની જરૂરિયાતો ઘટાડે છે (λ = c/f) \\
મલ્ટિપ્લેક્સિંગ & ઘણા સિગ્નલોને સ્પેક્ટ્રમ શેર કરવાની મંજૂરી આપે છે \\
રેન્જ & ટ્રાન્સમિશન અંતર વધારે છે \\
ઇન્ટરફેરન્સ & નોઇઝ ઇન્ટરફેરન્સ ઘટાડે છે \\
\end{longtable}
}

\begin{itemize}
\tightlist
\item
  \textbf{વ્યવહારુ ટ્રાન્સમિશન}: ઓછી આવૃત્તિના માહિતી સિગ્નલને વાયરલેસ ટ્રાન્સમિશન
  માટે યોગ્ય બનાવે છે
\item
  \textbf{સિગ્નલ અલગીકરણ}: વિવિધ સિગ્નલોને એકસાથે ટ્રાન્સમિટ કરવા સક્ષમ બનાવે છે
\end{itemize}

\end{solutionbox}
\begin{mnemonicbox}
``RARE Messages'' (Range, Antenna, Reduce
interference, Enable multiplexing)

\end{mnemonicbox}
\subsection*{પ્રશ્ન 1(બ) [4
ગુણ]}\label{uxaaauxab0uxab6uxaa8-1uxaac-4-uxa97uxaa3}

\textbf{AM અને FM ની સરખામણી કરો.}

\begin{solutionbox}


{\def\LTcaptype{none} % do not increment counter
\vspace{-5pt}
\captionof{table}{AM અને FM વચ્ચેનો તફાવત}
\vspace{-10pt}
\begin{longtable}[]{@{}lll@{}}
\toprule\noalign{}
પરિમાણ & AM (એમ્પલિટ્યૂડ મોડ્યુલેશન) & FM (ફ્રિક્વન્સી મોડ્યુલેશન) \\
\midrule\noalign{}
\endhead
\bottomrule\noalign{}
\endlastfoot
બદલાતો પરિમાણ & કેરિયરની એમ્પલિટ્યૂડ & કેરિયરની આવૃત્તિ \\
બેન્ડવિડ્થ & સાંકડી (2 \times fm) & વિશાળ (2 \times mf \times fm) \\
નોઇઝ પ્રતિરક્ષા & નબળી & ઉત્તમ \\
પાવર કાર્યક્ષમતા & ઓછી કાર્યક્ષમ & વધુ કાર્યક્ષમ \\
સર્કિટ જટિલતા & સરળ & જટિલ \\
ગુણવત્તા & મધ્યમ & ઉચ્ચ \\
ઉપયોગો & મધ્યમ વેવ બ્રોડકાસ્ટિંગ & હાઈ-ફિડેલિટી બ્રોડકાસ્ટિંગ \\
\end{longtable}
}

\end{solutionbox}
\begin{mnemonicbox}
``BANC-QA'' (Bandwidth, Amplitude/frequency, Noise,
Complexity, Quality, Applications)

\end{mnemonicbox}
\subsection*{પ્રશ્ન 1(ક) [7
ગુણ]}\label{uxaaauxab0uxab6uxaa8-1uxa95-7-uxa97uxaa3}

\textbf{AM મોડ્યુલેશન વેવફોર્મ સાથે સમજાવો અને મોડ્યુલેટેડ સિગ્નલ માટે વોલ્ટેજ સમીકરણ
મેળવો. DSBFC AM ફ્રીક્વન્સી સ્પેક્ટ્રમ દોરો.}

\begin{solutionbox}

એમ્પલિટ્યૂડ મોડ્યુલેશન (AM) એ એક તકનીક છે જેમાં કેરિયર વેવની એમ્પલિટ્યૂડ મોડ્યુલેટિંગ
સિગ્નલની તત્કાલીન એમ્પલિટ્યૂડના પ્રમાણમાં બદલાય છે.

\textbf{વોલ્ટેજ સમીકરણ:}

\begin{itemize}
\tightlist
\item
  કેરિયર સિગ્નલ: v_{1}(t) = A_{1} sin(ωct)
\item
  મોડ્યુલેટિંગ સિગ્નલ: v_{2}(t) = A_{2} sin(ωmt)
\item
  મોડ્યુલેટેડ સિગ્નલ: v(t) = A_{1}[1 + m sin(ωmt)] sin(ωct)
\item
  જ્યાં m = A_{2}/A_{1} (મોડ્યુલેશન ઇન્ડેક્સ)
\end{itemize}

\textbf{આકૃતિ: AM વેવફોર્મ}

\begin{figure}
\centering
\pandocbounded{\includesvg[keepaspectratio]{diagrams/1333201-w2024-q1c.svg}}
\caption{AM વેવફોર્મ}
\end{figure}

\textbf{DSBFC AM નું ફ્રિક્વન્સી સ્પેક્ટ્રમ}

\begin{figure}
\centering
\pandocbounded{\includesvg[keepaspectratio]{diagrams/1333201-w2024-q1cb.svg}}
\caption{DSBFC AM સ્પેક્ટ્રમ}
\end{figure}

\begin{itemize}
\tightlist
\item
  \textbf{બેન્ડવિડ્થ}: AM સિગ્નલની બેન્ડવિડ્થ 2 \times fm છે
\item
  \textbf{સાઇડબેન્ડ્સ}: અપર સાઇડબેન્ડ (USB) fc+fm પર અને લોઅર સાઇડબેન્ડ (LSB)
  fc-fm પર
\item
  \textbf{પાવર વિતરણ}: કેરિયર અને બે સાઇડબેન્ડસમાં
\end{itemize}

\end{solutionbox}
\begin{mnemonicbox}
``CAM-SIP'' (Carrier Amplitude Modified, Sidebands
In Pair)

\end{mnemonicbox}
\subsection*{પ્રશ્ન 1(ક) OR [7
ગુણ]}\label{uxaaauxab0uxab6uxaa8-1uxa95-or-7-uxa97uxaa3}

\textbf{AM માં કુલ પાવર માટે સમીકરણ મેળવો, DSB અને SSB માં પાવર બચતની
ટકાવારીની ગણતરી કરો.}

\begin{solutionbox}

\textbf{AM માં કુલ પાવરનું વ્યુત્પાદન:}

\begin{itemize}
\tightlist
\item
  AM સિગ્નલ: v(t) = A_{1}[1 + m sin(ωmt)] sin(ωct)
\item
  કુલ પાવર: P = P_{(}carrier_{)} + P_{(}sidebands_{)}
\item
  P_{(}carrier_{)} = A_{1}^{2}/2
\item
  P_{(}sidebands_{)} = A_{1}^{2}m^{2}/4
\end{itemize}


{\def\LTcaptype{none} % do not increment counter
\vspace{-5pt}
\captionof{table}{AM માં પાવર વિતરણ}
\vspace{-10pt}
\begin{longtable}[]{@{}lll@{}}
\toprule\noalign{}
ઘટક & પાવર સમીકરણ & કુલ પાવરની \% (m=1) \\
\midrule\noalign{}
\endhead
\bottomrule\noalign{}
\endlastfoot
કેરિયર & P_{(}c_{)} = A_{1}^{2}/2 & 66.67\% \\
સાઇડબેન્ડ્સ & P_{(}s_{)} = A_{1}^{2}m^{2}/4 & 33.33\% \\
કુલ & P_{(}t_{)} = A_{1}^{2}(1+m^{2}/2)/2 & 100\% \\
\end{longtable}
}

\textbf{પાવર બચત:}

\begin{itemize}
\tightlist
\item
  \textbf{DSB-SC}: 100\% કેરિયર પાવર બચે (કુલ પાવરનો 66.67\%)

  \begin{itemize}
  \tightlist
  \item
    માત્ર સાઇડબેન્ડ્સ ટ્રાન્સમિટ થાય છે
  \item
    ટકાવારી બચત = (P_{(}c_{)}/P_{(}t_{)}) \times 100 = 66.67\%
  \end{itemize}
\item
  \textbf{SSB}: 50\% સાઇડબેન્ડ પાવર + 100\% કેરિયર પાવર બચે

  \begin{itemize}
  \tightlist
  \item
    એક સાઇડબેન્ડ + કેરિયર દૂર કરેલ છે
  \item
    ટકાવારી બચત = (P_{(}c_{)} + P_{(}s_{)}/2)/P_{(}t_{)} \times 100 = 83.33\%
  \end{itemize}
\end{itemize}

\textbf{આકૃતિ: પાવર વિતરણ}

\begin{figure}
\centering
\pandocbounded{\includesvg[keepaspectratio]{diagrams/1333201-w2024-q1cor.svg}}
\caption{પાવર વિતરણ}
\end{figure}

\end{solutionbox}
\begin{mnemonicbox}
``CAST-83'' (Carrier And Sideband Transmission, 83\%
saved in SSB)

\end{mnemonicbox}
\subsection*{પ્રશ્ન 2(અ) [3
ગુણ]}\label{uxaaauxab0uxab6uxaa8-2uxa85-3-uxa97uxaa3}

\textbf{વ્યાખ્યાયિત કરો (1) AM માટે મોડ્યુલેશન ઇન્ડેક્સ (2) FM માટે મોડ્યુલેશન
ઇન્ડેક્સ.}

\begin{solutionbox}


{\def\LTcaptype{none} % do not increment counter
\vspace{-5pt}
\captionof{table}{મોડ્યુલેશન ઇન્ડેક્સની વ્યાખ્યાઓ}
\vspace{-10pt}
\begin{longtable}[]{@{}
  >{\raggedright\arraybackslash}p{(\linewidth - 4\tabcolsep) * \real{0.2075}}
  >{\raggedright\arraybackslash}p{(\linewidth - 4\tabcolsep) * \real{0.3962}}
  >{\raggedright\arraybackslash}p{(\linewidth - 4\tabcolsep) * \real{0.3962}}@{}}
\toprule\noalign{}
\begin{minipage}[b]{\linewidth}\raggedright
પરિમાણ
\end{minipage} & \begin{minipage}[b]{\linewidth}\raggedright
AM મોડ્યુલેશન ઇન્ડેક્સ
\end{minipage} & \begin{minipage}[b]{\linewidth}\raggedright
FM મોડ્યુલેશન ઇન્ડેક્સ
\end{minipage} \\
\midrule\noalign{}
\endhead
\bottomrule\noalign{}
\endlastfoot
વ્યાખ્યા & મોડ્યુલેટિંગ સિગ્નલની મહત્તમ એમ્પલિટ્યૂડનો કેરિયરની મહત્તમ એમ્પલિટ્યૂડ સાથેનો
ગુણોત્તર & ફ્રિક્વન્સી વિચલનનો મોડ્યુલેટિંગ ફ્રિક્વન્સી સાથેનો ગુણોત્તર \\
સૂત્ર &

m = Am/Ac & mf = Δf/fm \\

મર્યાદા & 0 \leq m \leq 1 (વિકૃતિ ટાળવા માટે) & કોઈ ચોક્કસ ઉપરી મર્યાદા નથી \\
અસર & એમ્પલિટ્યૂડ વેરિએશન અને પાવર વિતરણ નિયંત્રિત કરે છે & બેન્ડવિડ્થ અને સિગ્નલ
ગુણવત્તા નક્કી કરે છે \\
\end{longtable}
}

\begin{itemize}
\tightlist
\item
  \textbf{AM મોડ્યુલેશન ઇન્ડેક્સ}: એમ્પલિટ્યૂડ વેરિએશન અને પાવર વિતરણ નિયંત્રિત કરે છે
\item
  \textbf{FM મોડ્યુલેશન ઇન્ડેક્સ}: બેન્ડવિડ્થ અને સિગ્નલ ગુણવત્તા નિર્ધારિત કરે છે
\end{itemize}

\end{solutionbox}
\begin{mnemonicbox}
``ARM-FDM'' (Amplitude Ratio for Modulation,
Frequency Deviation for Modulation)

\end{mnemonicbox}
\subsection*{પ્રશ્ન 2(બ) [4
ગુણ]}\label{uxaaauxab0uxab6uxaa8-2uxaac-4-uxa97uxaa3}

\textbf{એન્વેલપ ડિટેક્ટર માટે બ્લોક ડાયાગ્રામ દોરો અને સમજાવો.}

\begin{solutionbox}

\textbf{આકૃતિ: એન્વેલપ ડિટેક્ટર}

\begin{figure}
\centering
\pandocbounded{\includesvg[keepaspectratio]{diagrams/1333201-w2024-q2a.svg}}
\caption{એન્વેલપ ડિટેક્ટર}
\end{figure}


{\def\LTcaptype{none} % do not increment counter
\vspace{-5pt}
\captionof{table}{ઘટકો અને તેમના કાર્યો}
\vspace{-10pt}
\begin{longtable}[]{@{}ll@{}}
\toprule\noalign{}
ઘટક & કાર્ય \\
\midrule\noalign{}
\endhead
\bottomrule\noalign{}
\endlastfoot
ડાયોડ & AM સિગ્નલનું રેક્ટિફિકેશન કરે છે (નકારાત્મક અર્ધ-ચક્રો દૂર કરે છે) \\
RC ફિલ્ટર & રેક્ટિફાઇડ સિગ્નલને સ્મૂધ કરીને એન્વેલપ રિકવર કરે છે \\
લોડ & આઉટપુટ સર્કિટ અને ઇમ્પિડન્સ મેચિંગ પ્રદાન કરે છે \\
\end{longtable}
}

\begin{itemize}
\tightlist
\item
  \textbf{કાર્યપ્રણાલી}: ડાયોડ માત્ર પોઝિટિવ અર્ધ-ચક્રો દરમિયાન કન્ડક્ટ કરે છે
\item
  \textbf{સમય અચળાંક}: RC એટલું મોટું હોવું જોઈએ કે રિપલ ન આવે પરંતુ મોડ્યુલેશનને
  અનુસરવા માટે પૂરતું નાનું હોવું જોઈએ
\item
  \textbf{શરત}: RC \textgreater\textgreater{} 1/fc પરંતુ RC
  \textless\textless{} 1/fm
\end{itemize}

\end{solutionbox}
\begin{mnemonicbox}
``DEER'' (Diode Extracts Envelope Representation)

\end{mnemonicbox}
\subsection*{પ્રશ્ન 2(ક) [7
ગુણ]}\label{uxaaauxab0uxab6uxaa8-2uxa95-7-uxa97uxaa3}

\textbf{FM રેડિયો રીસીવરનો બ્લોક ડાયાગ્રામ દોરો અને દરેક બ્લોકની કામગીરી
સમજાવો.}

\begin{solutionbox}

\textbf{આકૃતિ: FM રેડિયો રીસીવર}

\begin{figure}
\centering
\pandocbounded{\includesvg[keepaspectratio]{diagrams/1333201-w2024-q2b.svg}}
\caption{FM રેડિયો રીસીવર}
\end{figure}


{\def\LTcaptype{none} % do not increment counter
\vspace{-5pt}
\captionof{table}{દરેક બ્લોકનાં કાર્યો}
\vspace{-10pt}
\begin{longtable}[]{@{}ll@{}}
\toprule\noalign{}
બ્લોક & કાર્ય \\
\midrule\noalign{}
\endhead
\bottomrule\noalign{}
\endlastfoot
એન્ટેના & ઇલેક્ટ્રોમેગ્નેટિક તરંગો મેળવે છે \\
RF એમ્પ્લિફાયર & નબળા RF સિગ્નલ્સ (88-108 MHz) એમ્પ્લિફાય કરે છે \\
મિક્સર & RF ને IF ફ્રિક્વન્સી (10.7 MHz) માં કન્વર્ટ કરે છે \\
લોકલ ઓસિલેટર & મિક્સિંગ માટે ફ્રિક્વન્સી જનરેટ કરે છે (RF+10.7 MHz) \\
IF એમ્પ્લિફાયર & IF સિગ્નલને ફિક્સ્ડ ગેઈન સાથે એમ્પ્લિફાય કરે છે \\
લિમિટર & એમ્પલિટ્યૂડ વેરિએશન્સ દૂર કરે છે \\
FM ડિસ્ક્રિમિનેટર & ફ્રિક્વન્સી વેરિએશન્સને વોલ્ટેજમાં કન્વર્ટ કરે છે \\
ઓડિયો એમ્પ્લિફાયર & રિકવર્ડ ઓડિયો એમ્પ્લિફાય કરે છે \\
સ્પીકર & ઇલેક્ટ્રિકલ થી સાઉન્ડ વેવ્સમાં કન્વર્ટ કરે છે \\
\end{longtable}
}

\begin{itemize}
\tightlist
\item
  \textbf{સુપરહેટરોડાઈન પ્રિન્સિપલ}: ફિક્સ્ડ IF પર સિગ્નલ પ્રોસેસ કરવા ફ્રિક્વન્સી
  કન્વર્ઝન વાપરે છે
\item
  \textbf{વિશિષ્ટ FM ફીચર}: લિમિટર ડિમોડ્યુલેશન પહેલા એમ્પલિટ્યૂડમાં નોઈઝ દૂર કરે છે
\end{itemize}

\end{solutionbox}
\begin{mnemonicbox}
``RAMLIDASS'' (RF, Amplifier, Mixer, Local
oscillator, IF, Discriminator, Audio, Speaker System)

\end{mnemonicbox}
\subsection*{પ્રશ્ન 2(અ) OR [3
ગુણ]}\label{uxaaauxab0uxab6uxaa8-2uxa85-or-3-uxa97uxaa3}

\textbf{ફ્રીક્વન્સી મોડ્યુલેશન અને ફેઝ મોડ્યુલેશન માટે માત્ર વેવફોર્મ દોરો.}

\begin{solutionbox}

\textbf{આકૃતિ: FM અને PM વેવફોર્મ્સ}

\begin{figure}
\centering
\pandocbounded{\includesvg[keepaspectratio]{diagrams/1333201-w2024-q2c.svg}}
\caption{FM અને PM વેવફોર્મ્સ}
\end{figure}

\textbf{મુખ્ય લક્ષણો:}

\begin{itemize}
\tightlist
\item
  \textbf{FM}: જ્યારે મોડ્યુલેટિંગ સિગ્નલ પોઝિટિવ હોય ત્યારે ફ્રિક્વન્સી વધે છે
\item
  \textbf{PM}: ફેઝ એમ્પલિટ્યૂડ પરિવર્તન સાથે તરત જ શિફ્ટ થાય છે
\end{itemize}

\end{solutionbox}
\begin{mnemonicbox}
``FIP-PAF'' (Frequency Increases with Positive
signal, Phase Advances with Faster changes)

\end{mnemonicbox}
\subsection*{પ્રશ્ન 2(બ) OR [4
ગુણ]}\label{uxaaauxab0uxab6uxaa8-2uxaac-or-4-uxa97uxaa3}

\textbf{રેડિયો રીસીવરની કોઈ પણ ચાર લાક્ષણિકતાઓને વ્યાખ્યાયિત કરો.}

\begin{solutionbox}


{\def\LTcaptype{none} % do not increment counter
\vspace{-5pt}
\captionof{table}{રેડિયો રીસીવરની લાક્ષણિકતાઓ}
\vspace{-10pt}
\begin{longtable}[]{@{}ll@{}}
\toprule\noalign{}
લાક્ષણિકતા & વ્યાખ્યા \\
\midrule\noalign{}
\endhead
\bottomrule\noalign{}
\endlastfoot
સેન્સિટિવિટી & નબળા સિગ્નલ્સ મેળવવાની ક્ષમતા (μV અથવા dBm માં માપવામાં આવે છે) \\
સિલેક્ટિવિટી & ઇચ્છિત સિગ્નલને આસપાસના ચેનલોથી અલગ કરવાની ક્ષમતા \\
ફિડેલિટી & મૂળ મોડ્યુલેટિંગ સિગ્નલને સચોટતાથી પુનઃઉત્પન્ન કરવાની ક્ષમતા \\
ઈમેજ રિજેક્શન & ઈમેજ ફ્રિક્વન્સી ઇન્ટરફેરન્સને અસ્વીકાર કરવાની ક્ષમતા \\
\end{longtable}
}

\textbf{વધારાની લાક્ષણિકતાઓ:}

\begin{itemize}
\tightlist
\item
  \textbf{સિગ્નલ-ટુ-નોઈઝ રેશિયો}: સિગ્નલ પાવરનો નોઈઝ પાવર સાથેનો ગુણોત્તર
\item
  \textbf{બેન્ડવિડ્થ}: મેળવી શકાય તેવી ફ્રિક્વન્સીઓની રેન્જ
\item
  \textbf{સ્ટેબિલિટી}: ટ્યૂન કરેલી ફ્રિક્વન્સી જાળવી રાખવાની ક્ષમતા
\end{itemize}

\end{solutionbox}
\begin{mnemonicbox}
``SFIS-BSS'' (Sensitivity, Fidelity, Image
rejection, Selectivity - Better Signal Stability)

\end{mnemonicbox}
\subsection*{પ્રશ્ન 2(ક) OR [7
ગુણ]}\label{uxaaauxab0uxab6uxaa8-2uxa95-or-7-uxa97uxaa3}

\textbf{AM રેડિયો રીસીવરનો બ્લોક ડાયાગ્રામ દોરો અને દરેક બ્લોકની કામગીરી
સમજાવો.}

\begin{solutionbox}

\textbf{આકૃતિ: AM રેડિયો રીસીવર}

\begin{figure}
\centering
\pandocbounded{\includesvg[keepaspectratio]{diagrams/1333201-w2024-q2d.svg}}
\caption{AM રેડિયો રીસીવર}
\end{figure}


{\def\LTcaptype{none} % do not increment counter
\vspace{-5pt}
\captionof{table}{દરેક બ્લોકનાં કાર્યો}
\vspace{-10pt}
\begin{longtable}[]{@{}ll@{}}
\toprule\noalign{}
બ્લોક & કાર્ય \\
\midrule\noalign{}
\endhead
\bottomrule\noalign{}
\endlastfoot
એન્ટેના & AM રેડિયો તરંગો પકડે છે \\
RF ટ્યૂનર \& એમ્પ્લિફાયર & ઇચ્છિત ફ્રિક્વન્સી પસંદ કરે અને એમ્પ્લિફાય કરે છે \\
મિક્સર & RF સિગ્નલને IF (455 kHz) માં કન્વર્ટ કરે છે \\
લોકલ ઓસિલેટર & મિક્સિંગ માટે ફ્રિક્વન્સી જનરેટ કરે છે (RF+455 kHz) \\
IF એમ્પ્લિફાયર & ફિક્સ્ડ સિલેક્ટિવિટી સાથે IF સિગ્નલ એમ્પ્લિફાય કરે છે \\
ડિટેક્ટર & AM એન્વેલપમાંથી ઓડિયો રિકવર કરે છે \\
AGC & ઓટોમેટિક ગેઈન કંટ્રોલ પ્રદાન કરે છે \\
ઓડિયો એમ્પ્લિફાયર & ઓડિયો સિગ્નલ એમ્પ્લિફાય કરે છે \\
સ્પીકર & ઇલેક્ટ્રિકલ થી સાઉન્ડ વેવ્સમાં કન્વર્ટ કરે છે \\
\end{longtable}
}

\begin{itemize}
\tightlist
\item
  \textbf{સુપરહેટરોડાઈન પ્રિન્સિપલ}: બેટર સિલેક્ટિવિટી માટે ફ્રિક્વન્સી કન્વર્ઝન વાપરે
  છે
\item
  \textbf{AGC ફીડબેક લૂપ}: સિગ્નલ સ્ટ્રેન્થના ફેરફાર છતાં કોન્સ્ટન્ટ આઉટપુટ જાળવે છે
\end{itemize}

\end{solutionbox}
\begin{mnemonicbox}
``ARMLESS'' (Antenna, RF, Mixer, Local oscillator,
Envelope detector, Sound System)

\end{mnemonicbox}
\subsection*{પ્રશ્ન 3(અ) [3
ગુણ]}\label{uxaaauxab0uxab6uxaa8-3uxa85-3-uxa97uxaa3}

\textbf{Quantization વ્યાખ્યાયિત કરો. Non uniform quantization સંક્ષિપ્તમાં
સમજાવો.}

\begin{solutionbox}

\textbf{ક્વોન્ટાઇઝેશન} એ સતત એમ્પલિટ્યૂડ મૂલ્યોને ડિજિટલ રજૂઆત માટે ડિસ્ક્રીટ લેવલમાં
કન્વર્ટ કરવાની પ્રક્રિયા છે.


{\def\LTcaptype{none} % do not increment counter
\vspace{-5pt}
\captionof{table}{નોન-યુનિફોર્મ ક્વોન્ટાઇઝેશન}
\vspace{-10pt}
\begin{longtable}[]{@{}ll@{}}
\toprule\noalign{}
પાસું & વર્ણન \\
\midrule\noalign{}
\endhead
\bottomrule\noalign{}
\endlastfoot
વ્યાખ્યા & વિવિધ એમ્પલિટ્યૂડ રેન્જ માટે વિવિધ સ્ટેપ સાઇઝ ફાળવવી \\
ફાયદો & નાના એમ્પલિટ્યૂડ સિગ્નલ્સ માટે ક્વોન્ટાઇઝેશન નોઇઝ ઘટાડે છે \\
અમલીકરણ & કોમ્પેન્ડિંગ (કોમ્પ્રેશન-એક્સપાન્શન) તકનીકોનો ઉપયોગ \\
ઉદાહરણ & ટેલિફોનીમાં વપરાતા μ-law અને A-law કોમ્પેન્ડિંગ \\
\end{longtable}
}

\begin{itemize}
\tightlist
\item
  \textbf{કાર્યસિદ્ધાંત}: ઓછા એમ્પલિટ્યૂડ માટે નાના સ્ટેપ સાઇઝ, ઉચ્ચ એમ્પલિટ્યૂડ માટે
  મોટા સ્ટેપ
\item
  \textbf{અસર}: મજબૂત સિગ્નલ્સના ખર્ચે નબળા સિગ્નલ્સ માટે SNR સુધારે છે
\end{itemize}

\end{solutionbox}
\begin{mnemonicbox}
``QUEST-CS'' (QUantization with Enhanced Steps -
Compressing Small signals)

\end{mnemonicbox}
\subsection*{પ્રશ્ન 3(બ) [4
ગુણ]}\label{uxaaauxab0uxab6uxaa8-3uxaac-4-uxa97uxaa3}

\textbf{Sample and Hold સર્કિટ વેવફોર્મ સાથે સમજાવો.}

\begin{solutionbox}

\textbf{આકૃતિ: સેમ્પલ અને હોલ્ડ સર્કિટ}

\begin{lstlisting}
    Analog       ┌────────┐      Sampled
    Input ───────│Sample &│─────\rightarrowOutput
                 │ Hold   │
                 └───┬────┘
                     │
    Clock ───────────┘
\end{lstlisting}

\textbf{આકૃતિ: સેમ્પલ અને હોલ્ડ વેવફોર્મ}

\begin{lstlisting}
Analog Signal
     /\      /\
    /  \    /  \
   /    \  /    \
  /      \/      \

Clock Pulses
  _   _   _   _   _
 | | | | | | | | | |
 | | | | | | | | | |
 |_| |_| |_| |_| |_|

Sampled Output
     __      __
    |  |    |  |
   _|  |____/  |___
  /                \
\end{lstlisting}

\textbf{સેમ્પલ અને હોલ્ડ ઓપરેશન:}

\begin{itemize}
\tightlist
\item
  \textbf{સેમ્પલિંગ મોડ}: સ્વિચ બંધ થાય છે, કેપેસિટર ઇનપુટ વોલ્ટેજ પર ચાર્જ થાય છે
\item
  \textbf{હોલ્ડ મોડ}: સ્વિચ ખુલે છે, કેપેસિટર વોલ્ટેજ જાળવે છે
\item
  \textbf{પરિમાણો}: એક્વિઝિશન ટાઇમ, એપર્ચર ટાઇમ, હોલ્ડ ટાઇમ, ડ્રૂપ રેટ
\end{itemize}

\end{solutionbox}
\begin{mnemonicbox}
``CHASED'' (Capacitor Holds Amplitude Samples for
Extended Duration)

\end{mnemonicbox}
\subsection*{પ્રશ્ન 3(ક) [7
ગુણ]}\label{uxaaauxab0uxab6uxaa8-3uxa95-7-uxa97uxaa3}

\textbf{સેમ્પલિંગ શું છે? સેમ્પલિંગ પ્રકારો સમજાવો.}

\begin{solutionbox}

\textbf{સેમ્પલિંગ} એ કન્ટિન્યુઅસ-ટાઇમ સિગ્નલને નિયમિત અંતરાલે માપ લઈને ડિસ્ક્રીટ-ટાઇમ
સિગ્નલમાં રૂપાંતરિત કરવાની પ્રક્રિયા છે.


{\def\LTcaptype{none} % do not increment counter
\vspace{-5pt}
\captionof{table}{સેમ્પલિંગના પ્રકારો}
\vspace{-10pt}
\begin{longtable}[]{@{}
  >{\raggedright\arraybackslash}p{(\linewidth - 4\tabcolsep) * \real{0.1714}}
  >{\raggedright\arraybackslash}p{(\linewidth - 4\tabcolsep) * \real{0.3714}}
  >{\raggedright\arraybackslash}p{(\linewidth - 4\tabcolsep) * \real{0.4571}}@{}}
\toprule\noalign{}
\begin{minipage}[b]{\linewidth}\raggedright
પ્રકાર
\end{minipage} & \begin{minipage}[b]{\linewidth}\raggedright
વર્ણન
\end{minipage} & \begin{minipage}[b]{\linewidth}\raggedright
લક્ષણો
\end{minipage} \\
\midrule\noalign{}
\endhead
\bottomrule\noalign{}
\endlastfoot
નેચરલ સેમ્પલિંગ & સિગ્નલને રેક્ટેન્ગ્યુલર પલ્સ સાથે ગુણાકાર કરવામાં આવે છે & પલ્સ દરમિયાન
મૂળ સિગ્નલની આકૃતિ જાળવે છે \\
ફ્લેટ-ટોપ સેમ્પલિંગ & સેમ્પલ મૂલ્ય સેમ્પલિંગ અંતરાલ દરમિયાન અચળ રહે છે & સ્ટેરકેસ જેવો
આઉટપુટ બનાવે છે \\
આદર્શ સેમ્પલિંગ & તાત્કાલિક નમૂનાઓ ઇમ્પલ્સ તરીકે રજૂ થાય છે & શૂન્ય પહોળાઈવાળા પલ્સ
સાથે સૈદ્ધાંતિક ખ્યાલ \\
યુનિફોર્મ સેમ્પલિંગ & સમાન સમય અંતરાલે લેવાતા નમૂનાઓ & વ્યવહારમાં સૌથી સામાન્ય \\
નોન-યુનિફોર્મ સેમ્પલિંગ & બદલાતા અંતરાલે લેવાતા નમૂનાઓ & વિશેષ ઉપયોગો માટે વપરાય
છે \\
\end{longtable}
}

\textbf{આકૃતિ: સેમ્પલિંગ પ્રકારો}

\begin{lstlisting}
Original Signal
     /\      /\
    /  \    /  \
   /    \  /    \
  /      \/      \

Natural Sampling
   _     _     _ 
  | |   | |   | |
  | |/\ | |   | |/\
  |/  \| |   |/  \|

Flat-top Sampling
   ___    ___    
  |   |  |   |   
  |   |__|   |___
\end{lstlisting}

\begin{itemize}
\tightlist
\item
  \textbf{નાયક્વિસ્ટ ક્રાઇટેરિયા}: સેમ્પલિંગ ફ્રિક્વન્સી સિગ્નલમાં સર્વોચ્ચ ફ્રિક્વન્સીના
  ઓછામાં ઓછી બે ગણી હોવી જોઈએ
\end{itemize}

\end{solutionbox}
\begin{mnemonicbox}
``INFUN'' (Ideal, Natural, Flat-top, Uniform,
Non-uniform)

\end{mnemonicbox}
\subsection*{પ્રશ્ન 3(અ) OR [3
ગુણ]}\label{uxaaauxab0uxab6uxaa8-3uxa85-or-3-uxa97uxaa3}

\textbf{Quantization પ્રક્રિયા અને તેની આવશ્યકતા સમજાવો.}

\begin{solutionbox}

\textbf{ક્વોન્ટાઇઝેશન પ્રક્રિયા} સતત એમ્પલિટ્યૂડ મૂલ્યોને ડિજિટલ રજૂઆત માટે મર્યાદિત
ડિસ્ક્રીટ લેવલમાં મેપ કરે છે.


{\def\LTcaptype{none} % do not increment counter
\vspace{-5pt}
\captionof{table}{ક્વોન્ટાઇઝેશન પ્રક્રિયા અને આવશ્યકતા}
\vspace{-10pt}
\begin{longtable}[]{@{}ll@{}}
\toprule\noalign{}
પાસું & વર્ણન \\
\midrule\noalign{}
\endhead
\bottomrule\noalign{}
\endlastfoot
પ્રક્રિયા & એમ્પલિટ્યૂડ રેન્જને ડિસ્ક્રીટ લેવલમાં વિભાજીત કરવી \\
આવશ્યકતા & એનાલોગ-ટુ-ડિજિટલ કન્વર્ઝન માટે જરૂરી \\
અસર & ક્વોન્ટાઇઝેશન એરર/નોઇઝ દાખલ કરે છે \\
પરિમાણો & સ્ટેપ સાઇઝ, લેવલની સંખ્યા (n-બિટ માટે 2^{n}) \\
\end{longtable}
}

\begin{itemize}
\tightlist
\item
  \textbf{સ્ટેપ સાઇઝ ગણતરી}: સ્ટેપ સાઇઝ = (Vmax - Vmin)/2^{n}
\item
  \textbf{ક્વોન્ટાઇઝેશન એરર}: મહત્તમ એરર \pmQ/2 છે જ્યાં Q સ્ટેપ સાઇઝ છે
\item
  \textbf{ઉપયોગો}: ડિજિટલ કોમ્યુનિકેશન, ઓડિયો/વિડિઓ પ્રોસેસિંગ, ડેટા સ્ટોરેજ
\end{itemize}

\end{solutionbox}
\begin{mnemonicbox}
``SEND'' (Step-size Establishes Noise in
Digitization)

\end{mnemonicbox}
\subsection*{પ્રશ્ન 3(બ) OR [4
ગુણ]}\label{uxaaauxab0uxab6uxaa8-3uxaac-or-4-uxa97uxaa3}

\textbf{સિગ્નલના નમૂના લેવા માટે Nyquist માપદંડ જણાવો અને સમજાવો.}

\begin{solutionbox}

\textbf{નાયક્વિસ્ટ સેમ્પલિંગ થિયરમ} જણાવે છે કે બેન્ડલિમિટેડ સિગ્નલને સંપૂર્ણ રીતે
પુનઃનિર્માણ કરવા માટે, સેમ્પલિંગ ફ્રિક્વન્સી સિગ્નલમાં સર્વોચ્ચ ફ્રિક્વન્સી ઘટકના ઓછામાં
ઓછી બે ગણી હોવી જોઈએ.


{\def\LTcaptype{none} % do not increment counter
\vspace{-5pt}
\captionof{table}{નાયક્વિસ્ટ માપદંડ}
\vspace{-10pt}
\begin{longtable}[]{@{}ll@{}}
\toprule\noalign{}
પરિમાણ & વર્ણન \\
\midrule\noalign{}
\endhead
\bottomrule\noalign{}
\endlastfoot
માપદંડ & fs \geq 2fmax \\
નાયક્વિસ્ટ રેટ & 2fmax (લઘુત્તમ સેમ્પલિંગ ફ્રિક્વન્સી) \\
નાયક્વિસ્ટ ઇન્ટરવલ & 1/(2fmax) (મહત્તમ સેમ્પલિંગ પીરિયડ) \\
એલિયાસિંગ & જ્યારે fs \textless{} 2fmax થાય ત્યારે ઉદ્ભવે છે \\
\end{longtable}
}

\textbf{આકૃતિ: સેમ્પલિંગની અસરો}

\begin{lstlisting}
    Proper Sampling (fs > 2fmax)
    Original: /\/\/\/\
    Samples:  * * * * * * * *
    Result:   /\/\/\/\

    Aliasing (fs < 2fmax)
    Original: /\/\/\/\/\/\/\
    Samples:  *   *   *   *
    Result:   /\/\    (lower frequency)
\end{lstlisting}

\begin{itemize}
\tightlist
\item
  \textbf{અન્ડરસેમ્પલિંગના પરિણામો}: એલિયાસિંગ (ફ્રિક્વન્સી ફોલ્ડિંગ)
\item
  \textbf{વ્યવહારિક ઉપયોગ}: સેમ્પલિંગ પહેલા એન્ટી-એલિયાસિંગ ફિલ્ટર્સનો ઉપયોગ
\end{itemize}

\end{solutionbox}
\begin{mnemonicbox}
``TRAP-A'' (Twice Rate Avoids Problematic Aliasing)

\end{mnemonicbox}
\subsection*{પ્રશ્ન 3(ક) OR [7
ગુણ]}\label{uxaaauxab0uxab6uxaa8-3uxa95-or-7-uxa97uxaa3}

\textbf{PAM, PWM અને PPM વેવફોર્મ સાથે સમજાવો.}

\begin{solutionbox}


{\def\LTcaptype{none} % do not increment counter
\vspace{-5pt}
\captionof{table}{પલ્સ મોડ્યુલેશન તકનીકો}
\vspace{-10pt}
\begin{longtable}[]{@{}
  >{\raggedright\arraybackslash}p{(\linewidth - 6\tabcolsep) * \real{0.2000}}
  >{\raggedright\arraybackslash}p{(\linewidth - 6\tabcolsep) * \real{0.2364}}
  >{\raggedright\arraybackslash}p{(\linewidth - 6\tabcolsep) * \real{0.3273}}
  >{\raggedright\arraybackslash}p{(\linewidth - 6\tabcolsep) * \real{0.2364}}@{}}
\toprule\noalign{}
\begin{minipage}[b]{\linewidth}\raggedright
તકનીક
\end{minipage} & \begin{minipage}[b]{\linewidth}\raggedright
વર્ણન
\end{minipage} & \begin{minipage}[b]{\linewidth}\raggedright
બદલાતો પરિમાણ
\end{minipage} & \begin{minipage}[b]{\linewidth}\raggedright
ઉપયોગ
\end{minipage} \\
\midrule\noalign{}
\endhead
\bottomrule\noalign{}
\endlastfoot
PAM & પલ્સ એમ્પલિટ્યૂડ મોડ્યુલેશન & પલ્સની એમ્પલિટ્યૂડ & સિમ્પલ ADC સિસ્ટમ્સ \\
PWM & પલ્સ વિડ્થ મોડ્યુલેશન & પલ્સની પહોળાઈ/સમયગાળો & મોટર કંટ્રોલ, પાવર
રેગ્યુલેશન \\
PPM & પલ્સ પોઝિશન મોડ્યુલેશન & પલ્સની સ્થિતિ/ટાઇમિંગ & હાઈ નોઇઝ ઇમ્યુનિટી
સિસ્ટમ્સ \\
\end{longtable}
}

\textbf{આકૃતિ: પલ્સ મોડ્યુલેશન વેવફોર્મ્સ}

\begin{lstlisting}
Modulating Signal
    /\        /\
   /  \      /  \
  /    \    /    \
 /      \  /      \

PAM
  |  |    |  |    |  |
  |  |    |  |    |  |
  |  |    |  |    |  |
  █  █    █  █    █  █

PWM
  █████    ███    █████
  |    |   | |    |    |
  |    |   | |    |    |
  |    |   | |    |    |

PPM
  █ █ █ █ █ █ █ █
  | | | | | | | |
  | | | | | | | |
  | | | | | | | |
\end{lstlisting}

\begin{itemize}
\tightlist
\item
  \textbf{PAM}: સૌથી સરળ સ્વરૂપ, નોઇઝના સૌથી વધુ સંવેદનશીલ
\item
  \textbf{PWM}: બેહતર નોઇઝ ઇમ્યુનિટી, સરળ જનરેશન
\item
  \textbf{PPM}: શ્રેષ્ઠ નોઇઝ ઇમ્યુનિટી, ચોક્કસ ટાઇમિંગની જરૂર છે
\end{itemize}

\end{solutionbox}
\begin{mnemonicbox}
``AWP-PAW'' (Amplitude, Width, Position - Pulse
Alteration Ways)

\end{mnemonicbox}
\subsection*{પ્રશ્ન 4(અ) [3
ગુણ]}\label{uxaaauxab0uxab6uxaa8-4uxa85-3-uxa97uxaa3}

\textbf{ડેલ્ટા મોડયુલેશન માટે સ્લોપ ઓવરલોડ અને ગ્રૅનુલરનોઈઝ એટલે શું?}

\begin{solutionbox}


{\def\LTcaptype{none} % do not increment counter
\vspace{-5pt}
\captionof{table}{ડેલ્ટા મોડ્યુલેશનમાં નોઇઝના પ્રકારો}
\vspace{-10pt}
\begin{longtable}[]{@{}
  >{\raggedright\arraybackslash}p{(\linewidth - 6\tabcolsep) * \real{0.2927}}
  >{\raggedright\arraybackslash}p{(\linewidth - 6\tabcolsep) * \real{0.2927}}
  >{\raggedright\arraybackslash}p{(\linewidth - 6\tabcolsep) * \real{0.1707}}
  >{\raggedright\arraybackslash}p{(\linewidth - 6\tabcolsep) * \real{0.2439}}@{}}
\toprule\noalign{}
\begin{minipage}[b]{\linewidth}\raggedright
નોઇઝ પ્રકાર
\end{minipage} & \begin{minipage}[b]{\linewidth}\raggedright
વ્યાખ્યા
\end{minipage} & \begin{minipage}[b]{\linewidth}\raggedright
કારણ
\end{minipage} & \begin{minipage}[b]{\linewidth}\raggedright
ઉપાય
\end{minipage} \\
\midrule\noalign{}
\endhead
\bottomrule\noalign{}
\endlastfoot
સ્લોપ ઓવરલોડ નોઇઝ & જ્યારે સિગ્નલ સ્લોપ સ્ટેપ સાઇઝ ક્ષમતાને ઓળંગી જાય ત્યારે થતી ભૂલ
& ઝડપી બદલાતા સિગ્નલ માટે સ્ટેપ સાઇઝ ખૂબ નાની & સ્ટેપ સાઇઝ અથવા સેમ્પલિંગ ફ્રિક્વન્સી
વધારવી \\
ગ્રેન્યુલર નોઇઝ & ધીમી ગતિએ બદલાતા સિગ્નલોની આસપાસ સતત હંટિંગને કારણે થતી ભૂલ &
ધીમી ગતિએ બદલાતા સિગ્નલો માટે સ્ટેપ સાઇઝ ખૂબ મોટી & સ્ટેપ સાઇઝ ઘટાડવી \\
\end{longtable}
}

\textbf{આકૃતિ: DM નોઇઝ પ્રકારો}

\begin{lstlisting}
Slope Overload:
  Actual  /‾‾‾‾
         /
        /
       /      
   ___/       
  /
 /  DM Output (steps can't keep up)

Granular Noise:
  Actual  _________
         
   /‾\/‾\/‾\/‾\/‾\  DM Output (continuous zigzag)
\end{lstlisting}

\end{solutionbox}
\begin{mnemonicbox}
``FAST-SLOW'' (Fast signals cause Slope overload,
SLOW signals cause Granular noise)

\end{mnemonicbox}
\subsection*{પ્રશ્ન 4(બ) [4
ગુણ]}\label{uxaaauxab0uxab6uxaa8-4uxaac-4-uxa97uxaa3}

\textbf{TDM ફ્રેમ દોરો અને સમજાવો.}

\begin{solutionbox}

\textbf{આકૃતિ: TDM ફ્રેમ સ્ટ્રક્ચર}

\begin{lstlisting}
    ┌───────────────────────────────────┐
    │ FS │ CH1 │ CH2 │ CH3 │...│ CHn │ FS │
    └───────────────────────────────────┘
       |    |     |     |        |     |
       |    |     |     |        |     └── Frame Sync for next frame
       |    |     |     |        └──────── Last channel sample
       |    |     |     └──────────────── Channel 3 sample
       |    |     └───────────────────── Channel 2 sample
       |    └─────────────────────────── Channel 1 sample
       └────────────────────────────────── Frame Synchronization
\end{lstlisting}


{\def\LTcaptype{none} % do not increment counter
\vspace{-5pt}
\captionof{table}{TDM ફ્રેમ ઘટકો}
\vspace{-10pt}
\begin{longtable}[]{@{}ll@{}}
\toprule\noalign{}
ઘટક & વર્ણન \\
\midrule\noalign{}
\endhead
\bottomrule\noalign{}
\endlastfoot
ફ્રેમ સિન્ક (FS) & ફ્રેમની શરૂઆતને ચિહ્નિત કરતો પેટર્ન \\
ટાઇમ સ્લોટ & એક ચેનલને ફાળવેલો ભાગ \\
ચેનલ સેમ્પલ & ચોક્કસ ચેનલના ડેટા \\
ફ્રેમ લંબાઈ & કુલ સમયગાળો (FS + બધી ચેનલો) \\
\end{longtable}
}

\begin{itemize}
\tightlist
\item
  \textbf{કાર્યસિદ્ધાંત}: વિવિધ ચેનલોને વિવિધ ટાઇમ સ્લોટ ફાળવે છે
\item
  \textbf{સિન્ક્રોનાઇઝેશન}: યોગ્ય ડિમલ્ટિપ્લેક્સિંગ માટે આવશ્યક છે
\item
  \textbf{પ્રકારો}: સિન્ક્રોનસ TDM (ફિક્સ્ડ સ્લોટ્સ) અને સ્ટેટિસ્ટિકલ TDM (ડાયનેમિક
  એલોકેશન)
\end{itemize}

\end{solutionbox}
\begin{mnemonicbox}
``FAST-Ch'' (Frame And Slots for Transmitting
Channels)

\end{mnemonicbox}
\subsection*{પ્રશ્ન 4(ક) [7
ગુણ]}\label{uxaaauxab0uxab6uxaa8-4uxa95-7-uxa97uxaa3}

\textbf{PCM ટ્રાન્સમીટર અને રીસીવરના દરેક બ્લોકના કાર્યનું વર્ણન કરો, PCM સિસ્ટમનો
ઉપયોગીતા, ફાયદા અને નુકસાન આપો.}

\begin{solutionbox}

\textbf{આકૃતિ: PCM સિસ્ટમ}

\includegraphics[width=1\linewidth,height=\textheight,keepaspectratio]{mermaid-e8258a78.pdf}


{\def\LTcaptype{none} % do not increment counter
\vspace{-5pt}
\captionof{table}{PCM બ્લોક કાર્યો}
\vspace{-10pt}
\begin{longtable}[]{@{}ll@{}}
\toprule\noalign{}
બ્લોક & કાર્ય \\
\midrule\noalign{}
\endhead
\bottomrule\noalign{}
\endlastfoot
સેમ્પલર & એનાલોગ સિગ્નલને PAM સિગ્નલમાં રૂપાંતરિત કરે છે \\
ક્વોન્ટાઇઝર & સેમ્પલ્સને ડિસ્ક્રીટ લેવલ ફાળવે છે \\
એન્કોડર & ક્વોન્ટાઇઝ્ડ લેવલને બાઇનરી કોડમાં રૂપાંતરિત કરે છે \\
લાઇન કોડર & બાઇનરીને ટ્રાન્સમિશન ફોર્મેટમાં કન્વર્ટ કરે છે \\
લાઇન ડિકોડર & મળેલા સિગ્નલમાંથી બાઇનરી પુનઃપ્રાપ્ત કરે છે \\
ડિકોડર & બાઇનરીને ક્વોન્ટાઇઝ્ડ લેવલમાં પાછું કન્વર્ટ કરે છે \\
રિકન્સ્ટ્રક્શન ફિલ્ટર & ડિકોડેડ આઉટપુટને એનાલોગ સિગ્નલમાં સ્મૂધ કરે છે \\
\end{longtable}
}

\textbf{એપ્લિકેશન્સ, ફાયદા અને ગેરફાયદા:}


{\def\LTcaptype{none} % do not increment counter
\vspace{-5pt}
\captionof{table}{PCM સિસ્ટમની લાક્ષણિકતાઓ}
\vspace{-10pt}
\begin{longtable}[]{@{}ll@{}}
\toprule\noalign{}
શ્રેણી & વર્ણન \\
\midrule\noalign{}
\endhead
\bottomrule\noalign{}
\endlastfoot
એપ્લિકેશન્સ & ટેલિફોન સિસ્ટમ, CD ઓડિયો, ડિજિટલ TV, મોબાઇલ કોમ્યુનિકેશન \\
ફાયદા & નોઇઝથી સુરક્ષિત, સિગ્નલ રિજનરેશન શક્ય, ડિજિટલ સિસ્ટમ સાથે સુસંગત \\
ગેરફાયદા & વધુ બેન્ડવિડ્થની જરૂર, વધુ જટિલતા, ક્વોન્ટાઇઝેશન નોઇઝ \\
\end{longtable}
}

\end{solutionbox}
\begin{mnemonicbox}
``SEQUEL-DR'' (Sample, Quantize, Encode - Line code,
Decode, Reconstruct)

\end{mnemonicbox}
\subsection*{પ્રશ્ન 4(અ) OR [3
ગુણ]}\label{uxaaauxab0uxab6uxaa8-4uxa85-or-3-uxa97uxaa3}

\textbf{DM અને ADM મોડ્યુલેશન વચ્ચે તફાવત આપો.}

\begin{solutionbox}


{\def\LTcaptype{none} % do not increment counter
\vspace{-5pt}
\captionof{table}{DM અને ADM વચ્ચેની તુલના}
\vspace{-10pt}
\begin{longtable}[]{@{}
  >{\raggedright\arraybackslash}p{(\linewidth - 4\tabcolsep) * \real{0.1692}}
  >{\raggedright\arraybackslash}p{(\linewidth - 4\tabcolsep) * \real{0.3385}}
  >{\raggedright\arraybackslash}p{(\linewidth - 4\tabcolsep) * \real{0.4923}}@{}}
\toprule\noalign{}
\begin{minipage}[b]{\linewidth}\raggedright
પરિમાણ
\end{minipage} & \begin{minipage}[b]{\linewidth}\raggedright
ડેલ્ટા મોડ્યુલેશન (DM)
\end{minipage} & \begin{minipage}[b]{\linewidth}\raggedright
એડેપ્ટિવ ડેલ્ટા મોડ્યુલેશન (ADM)
\end{minipage} \\
\midrule\noalign{}
\endhead
\bottomrule\noalign{}
\endlastfoot
સ્ટેપ સાઇઝ & ફિક્સ્ડ & વેરિએબલ (સિગ્નલ સ્લોપને અનુકૂળ) \\
ટ્રેકિંગ ક્ષમતા & મર્યાદિત & બેહતર સિગ્નલ ટ્રેકિંગ \\
નોઇઝ પરફોર્મન્સ & સ્લોપ ઓવરલોડ અને ગ્રેન્યુલર નોઇઝથી પીડાય છે & ઓછી નોઇઝ
સમસ્યાઓ \\
જટિલતા & સરળ & વધુ જટિલ \\
\end{longtable}
}

\textbf{આકૃતિ: DM વિરુદ્ધ ADM ટ્રેકિંગ}

\begin{lstlisting}
Input Signal:   /‾‾‾‾\
               /      \
              /        \
             /          \

DM Output:   /‾\/‾\/‾\
            /         \/‾\/‾\

ADM Output: /‾‾\/‾‾‾\
           /         \‾‾\/‾‾\
           (larger steps for steep slopes)
\end{lstlisting}

\end{solutionbox}
\begin{mnemonicbox}
``FAST-VAR'' (Fixed And Simple Tracking vs Variable
Adaptive Response)

\end{mnemonicbox}
\subsection*{પ્રશ્ન 4(બ) OR [4
ગુણ]}\label{uxaaauxab0uxab6uxaa8-4uxaac-or-4-uxa97uxaa3}

\textbf{મૂળભૂત PCM-TDM સિસ્ટમનો બ્લોક ડાયાગ્રામ સમજાવો.}

\begin{solutionbox}

\textbf{આકૃતિ: PCM-TDM સિસ્ટમ}

\includegraphics[width=1\linewidth,height=\textheight,keepaspectratio]{mermaid-39f881c0.pdf}


{\def\LTcaptype{none} % do not increment counter
\vspace{-5pt}
\captionof{table}{PCM-TDM સિસ્ટમ ઘટકો}
\vspace{-10pt}
\begin{longtable}[]{@{}ll@{}}
\toprule\noalign{}
ઘટક & કાર્ય \\
\midrule\noalign{}
\endhead
\bottomrule\noalign{}
\endlastfoot
લો-પાસ ફિલ્ટર & ઇનપુટ સિગ્નલોની બેન્ડવિડ્થ મર્યાદિત કરે છે \\
મલ્ટિપ્લેક્સર & ટાઇમ સ્લોટમાં ઘણા સિગ્નલો જોડે છે \\
PCM એન્કોડર & ડિજિટલમાં રૂપાંતરિત કરે છે (સેમ્પલ, ક્વોન્ટાઇઝ, એન્કોડ) \\
ટ્રાન્સમિશન ચેનલ & ડિજિટાઇઝ્ડ, મલ્ટિપ્લેક્સ્ડ સિગ્નલ વહન કરે છે \\
PCM ડિકોડર & ક્વોન્ટાઇઝ્ડ સેમ્પલ્સ પુનઃનિર્માણ કરે છે \\
ડિમલ્ટિપ્લેક્સર & ટાઇમ સ્લોટમાંથી ચેનલો અલગ કરે છે \\
\end{longtable}
}

\begin{itemize}
\tightlist
\item
  \textbf{કાર્યસિદ્ધાંત}: ટાઇમ ડિવિઝન મલ્ટિપ્લેક્સિંગને પલ્સ કોડ મોડ્યુલેશન સાથે જોડે છે
\item
  \textbf{એપ્લિકેશન્સ}: ડિજિટલ ટેલિફોની, ડિજિટલ ઓડિયો બ્રોડકાસ્ટિંગ, કોમ્યુનિકેશન
  નેટવર્ક્સ
\end{itemize}

\end{solutionbox}
\begin{mnemonicbox}
``FLIMPED'' (Filter, Limit, Multiplex, PCM Encode,
Decode)

\end{mnemonicbox}
\subsection*{પ્રશ્ન 4(ક) OR [7
ગુણ]}\label{uxaaauxab0uxab6uxaa8-4uxa95-or-7-uxa97uxaa3}

\textbf{DPCM મોડ્યુલેટરને સમીકરણ અને વેવફોર્મ સાથે સમજાવો.}

\begin{solutionbox}

\textbf{ડિફરેન્શિયલ પલ્સ કોડ મોડ્યુલેશન (DPCM)} વર્તમાન સેમ્પલ અને અગાઉના સેમ્પલ્સના
આધારે અનુમાનિત મૂલ્ય વચ્ચેના તફાવતને એન્કોડ કરે છે.

\textbf{સમીકરણ:}

\begin{itemize}
\tightlist
\item
  એરર સિગ્નલ: e(n) = x(n) - x̂(n)
\item
  જ્યાં x(n) વર્તમાન સેમ્પલ છે, x̂(n) અનુમાનિત સેમ્પલ છે
\item
  અનુમાન: x̂(n) = Σ(aᵢ \times x(n-i))
\item
  ટ્રાન્સમિટેડ સિગ્નલ: DPCM આઉટપુટ = Q[e(n)]
\end{itemize}

\textbf{આકૃતિ: DPCM મોડ્યુલેટર}

\includegraphics[width=1\linewidth,height=\textheight,keepaspectratio]{mermaid-b20a1485.pdf}

\textbf{આકૃતિ: DPCM વેવફોર્મ}

\begin{lstlisting}
Original Samples:
  *   *   *   *   *
  |   |   |   |   |
  |   |   |   |   |
  |   |   |   |   |
  
Predicted Samples:
    o   o   o   o
    |   |   |   |
    |   |   |   |
    |   |   |   |
    
Difference (DPCM):
  ↕   ↕   ↕   ↕   ↕  (smaller values)
\end{lstlisting}


{\def\LTcaptype{none} % do not increment counter
\vspace{-5pt}
\captionof{table}{DPCM લાક્ષણિકતાઓ}
\vspace{-10pt}
\begin{longtable}[]{@{}ll@{}}
\toprule\noalign{}
ફીચર & વર્ણન \\
\midrule\noalign{}
\endhead
\bottomrule\noalign{}
\endlastfoot
ફાયદો & ઘટાડેલો બિટ રેટ (PCMની તુલનામાં 30-50\%) \\
અનુમાન & વર્તમાન અનુમાન માટે અગાઉના સેમ્પલ(સ)નો ઉપયોગ \\
જટિલતા & PCM કરતાં વધુ પરંતુ ADPCM કરતાં ઓછી \\
એપ્લિકેશન & સ્પીચ કોડિંગ, ઇમેજ કોમ્પ્રેશન \\
\end{longtable}
}

\end{solutionbox}
\begin{mnemonicbox}
``PQED'' (Predict, Quantize Error, Encode
Difference)

\end{mnemonicbox}
\subsection*{પ્રશ્ન 5(અ) [3
ગુણ]}\label{uxaaauxab0uxab6uxaa8-5uxa85-3-uxa97uxaa3}

\textbf{એન્ટેના, રેડિયેશનપેટર્ન અને ધ્રુવીકરણ વ્યાખ્યાયિત કરો.}

\begin{solutionbox}


{\def\LTcaptype{none} % do not increment counter
\vspace{-5pt}
\captionof{table}{એન્ટેનાની વ્યાખ્યાઓ}
\vspace{-10pt}
\begin{longtable}[]{@{}
  >{\raggedright\arraybackslash}p{(\linewidth - 2\tabcolsep) * \real{0.3333}}
  >{\raggedright\arraybackslash}p{(\linewidth - 2\tabcolsep) * \real{0.6667}}@{}}
\toprule\noalign{}
\begin{minipage}[b]{\linewidth}\raggedright
શબ્દ
\end{minipage} & \begin{minipage}[b]{\linewidth}\raggedright
વ્યાખ્યા
\end{minipage} \\
\midrule\noalign{}
\endhead
\bottomrule\noalign{}
\endlastfoot
એન્ટેના & એક ઉપકરણ જે ઇલેક્ટ્રિકલ ઊર્જાને ઇલેક્ટ્રોમેગ્નેટિક તરંગમાં અને તેનાથી વિપરીત
રૂપાંતરિત કરે છે \\
રેડિયેશન પેટર્ન & અવકાશ કોઓર્ડિનેટ્સના ફંકશન તરીકે એન્ટેનાના રેડિયેશન ગુણધર્મોનું ગ્રાફિકલ
રજૂઆત \\
ધ્રુવીકરણ & એન્ટેના દ્વારા રેડિયેટ કરાયેલા ઇલેક્ટ્રોમેગ્નેટિક તરંગના ઇલેક્ટ્રિક ફીલ્ડ
વેક્ટરની ઓરિએન્ટેશન \\
\end{longtable}
}

\textbf{ધ્રુવીકરણના પ્રકારો:}

\begin{itemize}
\tightlist
\item
  \textbf{લિનિયર}: ઇલેક્ટ્રિક ફીલ્ડ એક દિશામાં આંદોલિત થાય છે (વર્ટિકલ,
  હોરિઝોન્ટલ)
\item
  \textbf{સર્ક્યુલર}: ઇલેક્ટ્રિક ફીલ્ડ અચળ એમ્પલિટ્યૂડ સાથે ફરે છે (RHCP, LHCP)
\item
  \textbf{ઇલિપ્ટિકલ}: ઇલેક્ટ્રિક ફીલ્ડ બદલાતી એમ્પલિટ્યૂડ સાથે ફરે છે
\end{itemize}

\end{solutionbox}
\begin{mnemonicbox}
``WAVE-PRO'' (Wireless Antenna Validates
Electromagnetic Propagation, Radiation, Orientation)

\end{mnemonicbox}
\subsection*{પ્રશ્ન 5(બ) [4
ગુણ]}\label{uxaaauxab0uxab6uxaa8-5uxaac-4-uxa97uxaa3}

\textbf{માઇક્રોસ્ટ્રીપ એન્ટેના સ્કેચ સાથે સમજાવો.}

\begin{solutionbox}

\textbf{આકૃતિ: માઇક્રોસ્ટ્રીપ પેચ એન્ટેના}

\begin{lstlisting}
    ┌───────────────────┐  \leftarrowPatch (radiating element)
    │                   │
    │                   │
    └───────────────────┘
    ┌───────────────────────────────┐
    │                               │  \leftarrowDielectric substrate
    └───────────────────────────────┘
    ┌───────────────────────────────┐
    │                               │  \leftarrowGround plane
    └───────────────────────────────┘
              │
              │ Feed point
              ▼
\end{lstlisting}


{\def\LTcaptype{none} % do not increment counter
\vspace{-5pt}
\captionof{table}{માઇક્રોસ્ટ્રીપ એન્ટેના ઘટકો}
\vspace{-10pt}
\begin{longtable}[]{@{}ll@{}}
\toprule\noalign{}
ઘટક & કાર્ય \\
\midrule\noalign{}
\endhead
\bottomrule\noalign{}
\endlastfoot
પેચ & રેડિયેટિંગ એલિમેન્ટ (સામાન્ય રીતે કોપર) \\
સબસ્ટ્રેટ & પેચ અને ગ્રાઉન્ડ વચ્ચેનું ડાઇલેક્ટ્રિક મટિરિયલ \\
ગ્રાઉન્ડ પ્લેન & તળિયે મેટલ લેયર \\
ફીડ પોઇન્ટ & સિગ્નલ માટે કનેક્શન પોઇન્ટ \\
\end{longtable}
}

\begin{itemize}
\tightlist
\item
  \textbf{કાર્યસિદ્ધાંત}: ધારો પર ફ્રિન્જિંગ ફીલ્ડ્સ રેડિએશન ઉત્પન્ન કરે છે
\item
  \textbf{ફાયદા}: લો પ્રોફાઇલ, હળવું વજન, સરળ ફેબ્રિકેશન, PCB સાથે સુસંગત
\item
  \textbf{એપ્લિકેશન્સ}: મોબાઇલ ડિવાઇસ, સેટેલાઇટ, એરક્રાફ્ટ, RFID ટેગ્સ
\end{itemize}

\end{solutionbox}
\begin{mnemonicbox}
``SPGF'' (Substrate, Patch, Ground, Feed)

\end{mnemonicbox}
\subsection*{પ્રશ્ન 5(ક) [7
ગુણ]}\label{uxaaauxab0uxab6uxaa8-5uxa95-7-uxa97uxaa3}

\textbf{ડેલ્ટા મોડ્યુલેશન જરૂરી સ્કેચ અને વેવફોર્મ સાથે સમજાવો.}

\begin{solutionbox}

ડેલ્ટા મોડ્યુલેશન (DM) એ ડિફરેન્શિયલ પલ્સ કોડ મોડ્યુલેશનનું સૌથી સરળ સ્વરૂપ છે જ્યાં ક્રમિક
સેમ્પલ્સ વચ્ચેનો તફાવત એક બિટમાં એન્કોડ થાય છે.

\textbf{આકૃતિ: ડેલ્ટા મોડ્યુલેટર}

\includegraphics[width=1\linewidth,height=\textheight,keepaspectratio]{mermaid-2a251cf7.pdf}

\textbf{આકૃતિ: ડેલ્ટા મોડ્યુલેશન વેવફોર્મ}

\begin{lstlisting}
Input Signal:
        /‾‾‾‾‾\
       /       \
      /         \
     /           \
    /             \

Clock Pulses:
    ˉ|ˉ|ˉ|ˉ|ˉ|ˉ|ˉ|ˉ|ˉ|ˉ|ˉ|ˉ|ˉ|ˉ|ˉ|ˉ

DM Output (bits):
    1 1 1 1 0 0 0 0 0 1 1 1 0 0 0 0

Step Approximation:
       /‾\/‾\
      /     \
     /       \/‾\
    /           \
\end{lstlisting}


{\def\LTcaptype{none} % do not increment counter
\vspace{-5pt}
\captionof{table}{ડેલ્ટા મોડ્યુલેશન લાક્ષણિકતાઓ}
\vspace{-10pt}
\begin{longtable}[]{@{}
  >{\raggedright\arraybackslash}p{(\linewidth - 2\tabcolsep) * \real{0.5517}}
  >{\raggedright\arraybackslash}p{(\linewidth - 2\tabcolsep) * \real{0.4483}}@{}}
\toprule\noalign{}
\begin{minipage}[b]{\linewidth}\raggedright
લાક્ષણિકતા
\end{minipage} & \begin{minipage}[b]{\linewidth}\raggedright
વર્ણન
\end{minipage} \\
\midrule\noalign{}
\endhead
\bottomrule\noalign{}
\endlastfoot
બિટ રેટ & પ્રતિ સેમ્પલ 1 બિટ \\
સ્ટેપ સાઇઝ & ફિક્સ્ડ (મુખ્ય મર્યાદા) \\
સ્લોપ ઓવરલોડ & જ્યારે સિગ્નલ સ્ટેપ સાઇઝ ટ્રેક કરી શકે તેના કરતાં ઝડપથી બદલાય
ત્યારે \\
ગ્રેન્યુલર નોઇઝ & ધીમી ગતિએ બદલાતા સિગ્નલમાં (સતત હંટિંગ) \\
ફાયદા & સરળતા, ઓછો બિટ રેટ \\
ગેરફાયદા & મર્યાદિત ડાયનેમિક રેન્જ, નોઇઝ સમસ્યાઓ \\
\end{longtable}
}

\end{solutionbox}
\begin{mnemonicbox}
``SIGN-UP'' (SInGle bit, Next step Up or down,
Predict)

\end{mnemonicbox}
\subsection*{પ્રશ્ન 5(અ) OR [3
ગુણ]}\label{uxaaauxab0uxab6uxaa8-5uxa85-or-3-uxa97uxaa3}

\textbf{સ્માર્ટ એન્ટેના શું છે? સ્માર્ટ એન્ટેના એપ્લિકેશન આપો.}

\begin{solutionbox}

\textbf{સ્માર્ટ એન્ટેના} એ એક એડેપ્ટિવ એરે સિસ્ટમ છે જે કોમ્યુનિકેશન પરફોર્મન્સ વધારવા
માટે ડિજિટલ સિગ્નલ પ્રોસેસિંગ એલ્ગોરિધમનો ઉપયોગ કરીને ડાયનેમિક રીતે તેની રેડિએશન
પેટર્ન એડજસ્ટ કરે છે.


{\def\LTcaptype{none} % do not increment counter
\vspace{-5pt}
\captionof{table}{સ્માર્ટ એન્ટેના એપ્લિકેશન્સ}
\vspace{-10pt}
\begin{longtable}[]{@{}ll@{}}
\toprule\noalign{}
એપ્લિકેશન & ફાયદો \\
\midrule\noalign{}
\endhead
\bottomrule\noalign{}
\endlastfoot
સેલ્યુલર બેઝ સ્ટેશન્સ & વધેલી ક્ષમતા અને કવરેજ \\
વાયરલેસ LAN & સુધારેલું થ્રૂપુટ અને ઘટેલું ઇન્ટરફેરન્સ \\
સેટેલાઇટ કોમ્યુનિકેશન્સ & બેહતર સિગ્નલ ક્વોલિટી અને પાવર કાર્યક્ષમતા \\
મિલિટરી કોમ્યુનિકેશન્સ & વધેલી સુરક્ષા અને જામ રેસિસ્ટન્સ \\
IoT નેટવર્ક્સ & વિસ્તારિત બેટરી લાઇફ, સુધારેલી કનેક્ટિવિટી \\
\end{longtable}
}

\begin{itemize}
\tightlist
\item
  \textbf{કાર્યસિદ્ધાંત}: ઇચ્છિત યુઝર્સ તરફ સિગ્નલ એનર્જી ફોકસ કરવા બીમફોર્મિંગનો
  ઉપયોગ કરે છે
\item
  \textbf{પ્રકારો}: સ્વિચ્ડ બીમ સિસ્ટમ્સ અને એડેપ્ટિવ એરે સિસ્ટમ્સ
\end{itemize}

\end{solutionbox}
\begin{mnemonicbox}
``SWIM-CM'' (Smart Wireless In
Mobile-Cellular-Military)

\end{mnemonicbox}
\subsection*{પ્રશ્ન 5(બ) OR [4
ગુણ]}\label{uxaaauxab0uxab6uxaa8-5uxaac-or-4-uxa97uxaa3}

\textbf{પેરાબોલિક રિફ્લેક્ટર એન્ટેના સ્કેચ સાથે સમજાવો.}

\begin{solutionbox}

\textbf{આકૃતિ: પેરાબોલિક રિફ્લેક્ટર એન્ટેના}

\begin{lstlisting}
                  ╱│╲
               ╱   │  ╲
            ╱      │     ╲
         ╱         │        ╲
      ╱            │           ╲
   ╱               │              ╲
 ╱─────────────────┼─────────────────╲
                   │
                   │
                   ▼
                 Feed
                 Point
\end{lstlisting}


{\def\LTcaptype{none} % do not increment counter
\vspace{-5pt}
\captionof{table}{પેરાબોલિક રિફ્લેક્ટર ઘટકો}
\vspace{-10pt}
\begin{longtable}[]{@{}ll@{}}
\toprule\noalign{}
ઘટક & કાર્ય \\
\midrule\noalign{}
\endhead
\bottomrule\noalign{}
\endlastfoot
પેરાબોલિક ડિશ & સિગ્નલ્સને પરાવર્તિત અને કેન્દ્રિત કરે છે \\
ફીડ હોર્ન & ફોકલ પોઇન્ટ પર સિગ્નલ્સને રેડિયેટ/રિસીવ કરે છે \\
સપોર્ટિંગ સ્ટ્રક્ચર & જ્યોમેટ્રી અને સ્થિરતા જાળવે છે \\
વેવગાઇડ & ફીડ હોર્નને ટ્રાન્સમિટર/રિસીવર સાથે જોડે છે \\
\end{longtable}
}

\begin{itemize}
\tightlist
\item
  \textbf{કાર્યસિદ્ધાંત}: આવતા સમાંતર કિરણો ફોકલ પોઇન્ટ પર પરાવર્તિત થાય છે
\item
  \textbf{લાક્ષણિકતાઓ}: ઉચ્ચ ગેઇન, દિશાત્મકતા, સાંકડી બીમવિડ્થ
\item
  \textbf{એપ્લિકેશન્સ}: સેટેલાઇટ કોમ્યુનિકેશન, રેડિયો એસ્ટ્રોનોમી, રડાર, માઇક્રોવેવ
  લિંક્સ
\end{itemize}

\end{solutionbox}
\begin{mnemonicbox}
``PFGH'' (Parabolic Focus Gives High-gain)

\end{mnemonicbox}
\subsection*{પ્રશ્ન 5(ક) OR [7
ગુણ]}\label{uxaaauxab0uxab6uxaa8-5uxa95-or-7-uxa97uxaa3}

\textbf{એડેપ્ટિવ ડેલ્ટા મોડ્યુલેશન જરૂરી સ્કેચ અને વેવફોર્મ સાથે સમજાવો.}

\begin{solutionbox}

એડેપ્ટિવ ડેલ્ટા મોડ્યુલેશન (ADM) ઇનપુટ સિગ્નલની લાક્ષણિકતાઓ અનુસાર સ્ટેપ સાઇઝને
ડાયનેમિક રીતે એડજસ્ટ કરીને સ્ટાન્ડર્ડ DMમાં સુધારો કરે છે.

\textbf{આકૃતિ: એડેપ્ટિવ ડેલ્ટા મોડ્યુલેટર}

\includegraphics[width=1\linewidth,height=\textheight,keepaspectratio]{mermaid-d158d701.pdf}

\textbf{આકૃતિ: ADM વેવફોર્મ}

\begin{lstlisting}
Input Signal:
        /‾‾‾‾‾\
       /       \
      /         \
     /           \
    /             \

ADM Output (variable step):
       /‾‾‾\
      /     \
     /       \
    /         \
   /           \
  (larger steps for steep slopes)
\end{lstlisting}


{\def\LTcaptype{none} % do not increment counter
\vspace{-5pt}
\captionof{table}{ADM લાક્ષણિકતાઓ}
\vspace{-10pt}
\begin{longtable}[]{@{}ll@{}}
\toprule\noalign{}
પાસું & વર્ણન \\
\midrule\noalign{}
\endhead
\bottomrule\noalign{}
\endlastfoot
સ્ટેપ સાઇઝ & વેરિએબલ (સિગ્નલ સ્લોપને અનુકૂળ) \\
કંટ્રોલ લોજિક & ક્રમિક સમાન બિટ્સ માટે સ્ટેપ સાઇઝ વધારે છે \\
ફાયદા & ઘટાડેલ સ્લોપ ઓવરલોડ અને ગ્રેન્યુલર નોઇઝ \\
ગેરફાયદા & DM કરતાં વધુ જટિલ \\
એપ્લિકેશન્સ & સ્પીચ કોડિંગ, ટેલિમેટ્રી, ડિજિટલ ટેલિફોની \\
પરફોર્મન્સ & સમાન બિટ રેટ પર DM કરતાં વધુ સારું SNR \\
\end{longtable}
}

\begin{itemize}
\tightlist
\item
  \textbf{સ્ટેપ સાઇઝ એડજસ્ટમેન્ટ}: μ(n) = μ(n-1) \times K જો ક્રમિક બિટ્સ સમાન હોય
\item
  \textbf{સ્ટેપ સાઇઝ એડજસ્ટમેન્ટ}: μ(n) = μ(n-1) / K જો ક્રમિક બિટ્સ બદલાય
\end{itemize}

\end{solutionbox}
\begin{mnemonicbox}
``ADVISED'' (ADaptive Variable Increment Step for
Enhanced Delta modulation)

\end{mnemonicbox}

\end{document}
