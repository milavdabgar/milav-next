\documentclass[10pt,a4paper]{article}

% content/resources/templates/preamble.tex
\usepackage[margin=0.6in]{geometry}
\author{Milav Dabgar}
\usepackage{amsmath,amssymb,amsthm}
\usepackage{booktabs}
\usepackage{multirow}
\usepackage{xcolor}
\usepackage{tcolorbox}
\tcbuselibrary{breakable,skins}
\usepackage[colorlinks=true,linkcolor=blue]{hyperref}
\usepackage{titlesec}
\usepackage{enumitem}
\usepackage{tikz}
\usepackage{pgfplots}
\usepackage{circuitikz}
\usepackage[version=4]{mhchem}
\usepackage{longtable}
\usepackage{array}
\usepackage{float}
\usepackage{caption}
\usepackage{listings}

\lstset{
  basicstyle=\small\ttfamily,
  breaklines=true,
  breakatwhitespace=false,
  postbreak=\mbox{\textcolor{red}{$\hookrightarrow$}\space},
  float=false,
  numbers=left,
  numberstyle=\tiny\color{gray},
  numbersep=10pt,
  xleftmargin=2em,
  keywordstyle=\color{blue},
  commentstyle=\color{green!60!black},
  stringstyle=\color{purple},
  backgroundcolor=\color{gray!5},
  showstringspaces=false,
  tabsize=2,
  captionpos=b,
  keepspaces=true,
  columns=flexible
}

\pgfplotsset{compat=1.18}
\usetikzlibrary{shapes,arrows,positioning,calc,patterns,decorations.pathmorphing,decorations.markings,arrows.meta}

% Color scheme
\definecolor{headcolor}{RGB}{0,102,204}
\definecolor{keycolor}{RGB}{220,20,60}
\definecolor{solutioncolor}{RGB}{34,139,34}
\definecolor{mnemoniccolor}{RGB}{148,0,211}
\definecolor{codecolor}{RGB}{0,0,100}

% Spacing
\setlength{\parskip}{3pt}
\setlist[itemize]{nosep}
\setlist[enumerate]{nosep}

% Title formatting
\titleformat{\section}{\Large\bfseries\color{headcolor}}{\thesection}{1em}{}
\titleformat{\subsection}{\large\bfseries\color{headcolor}}{\thesubsection}{1em}{}

% Pandoc tightlist compatibility
\providecommand{\tightlist}{%
  \setlength{\itemsep}{0pt}\setlength{\parskip}{0pt}}

% Pandoc longtable compatibility
\newcounter{none}
\def\thenone{}


% content/resources/templates/english-boxes.tex
% This file is currently empty - it exists to maintain consistency with the import structure.
% Add custom environments here if needed in the future.


\begin{document}

\begin{center}
{\Huge\bfseries\color{headcolor} Subject Name Solutions}\\[5pt]
{\LARGE 1333201 -- Winter 2024}\\[3pt]
{\large Semester 1 Study Material}\\[3pt]
{\normalsize\textit{Detailed Solutions and Explanations}}
\end{center}

\vspace{10pt}

\subsection*{Question 1(a) [3 marks]}\label{q1a}

\textbf{What is modulation? What is the need of it?}

\begin{solutionbox}
Modulation is the process of varying one or more
properties of a high-frequency carrier signal with a modulating signal
containing information.


{\def\LTcaptype{none} % do not increment counter
\vspace{-5pt}
\captionof{table}{Need for Modulation}
\vspace{-10pt}
\begin{longtable}[]{@{}ll@{}}
\toprule\noalign{}
Reason & Explanation \\
\midrule\noalign{}
\endhead
\bottomrule\noalign{}
\endlastfoot
Antenna Size & Reduces antenna size requirements (λ = c/f) \\
Multiplexing & Allows multiple signals to share the spectrum \\
Range & Increases transmission distance \\
Interference & Reduces noise interference \\
\end{longtable}
}

\begin{itemize}
\tightlist
\item
  \textbf{Practical transmission}: Makes low-frequency information
  signals suitable for wireless transmission
\item
  \textbf{Signal separation}: Enables different signals to be
  transmitted simultaneously
\end{itemize}

\end{solutionbox}
\begin{mnemonicbox}
``RARE Messages'' (Range, Antenna, Reduce
interference, Enable multiplexing)

\textbf{Diagram: Communication System}

\begin{figure}
\centering
\pandocbounded{\includesvg[keepaspectratio]{diagrams/1333201-w2024-q1ab.svg}}
\caption{Communication System Block Diagram}
\end{figure}

\end{mnemonicbox}
\subsection*{Question 1(b) [4 marks]}\label{q1b}

\textbf{Compare AM and FM.}

\begin{solutionbox}


{\def\LTcaptype{none} % do not increment counter
\vspace{-5pt}
\captionof{table}{Comparison between AM and FM}
\vspace{-10pt}
\begin{longtable}[]{@{}lll@{}}
\toprule\noalign{}
Parameter & AM (Amplitude Modulation) & FM (Frequency Modulation) \\
\midrule\noalign{}
\endhead
\bottomrule\noalign{}
\endlastfoot
Parameter varied & Amplitude of carrier & Frequency of carrier \\
Bandwidth & Narrow (2 \times fm) & Wide (2 \times mf \times fm) \\
Noise immunity & Poor & Excellent \\
Power efficiency & Less efficient & More efficient \\
Circuit complexity & Simple & Complex \\
Quality & Moderate & High \\
Applications & Medium wave broadcasting & High-fidelity broadcasting \\
\end{longtable}
}

\end{solutionbox}
\begin{mnemonicbox}
``BANC-QA'' (Bandwidth, Amplitude/frequency, Noise,
Complexity, Quality, Applications)

\end{mnemonicbox}
\subsection*{Question 1(c) [7 marks]}\label{q1c}

\textbf{Explain Amplitude modulation with waveform and derive voltage
equation for modulated signal also Sketch the frequency spectrum of the
DSBFC AM.}

\begin{solutionbox}

Amplitude Modulation (AM) is a technique where the amplitude of a
carrier wave is varied in proportion to the instantaneous amplitude of
the modulating signal.

\textbf{Voltage Equation:}

\begin{itemize}
\tightlist
\item
  Carrier signal: v_{1}(t) = A_{1} sin(ωct)
\item
  Modulating signal: v_{2}(t) = A_{2} sin(ωmt)
\item
  Modulated signal: v(t) = A_{1}[1 + m sin(ωmt)] sin(ωct)
\item
  Where m = A_{2}/A_{1} (modulation index)
\end{itemize}

\textbf{Diagram: AM Waveform}

\begin{figure}
\centering
\pandocbounded{\includesvg[keepaspectratio]{diagrams/1333201-w2024-q1c.svg}}
\caption{AM Waveform}
\end{figure}

\includegraphics[width=1\linewidth,height=\textheight,keepaspectratio]{mermaid-9b5f853a.pdf}

\begin{lstlisting}
     Carrier
      /\  /\  /\  /\  /\  /\  /\
     /  \/  \/  \/  \/  \/  \/  \
    
    Modulating
     /\              /\
    /  \            /  \
         \        /
          \/    \/
           
    AM Signal
      /\    /\          /\    /\
     /  \  /  \        /  \  /  \
    /    \/    \      /    \/    \
                \    /
                 \  /
                  \/
\end{lstlisting}

\textbf{Frequency Spectrum of DSBFC AM}

\begin{figure}
\centering
\pandocbounded{\includesvg[keepaspectratio]{diagrams/1333201-am-frequency-spectrum.svg}}
\caption{AM Frequency Spectrum}
\end{figure}

\begin{lstlisting}
    Amplitude
        |
        |     A_{1}
        |      ↓
        |      █
        |      
        |m·A_{1}/2 ↓     ↓
        |      █      █
        |______|______|______|______▶ Frequency
               |      |      |
             fc-fm    fc    fc+fm
\end{lstlisting}

\begin{itemize}
\tightlist
\item
  \textbf{Bandwidth}: The bandwidth of AM signal is 2 \times fm
\item
  \textbf{Sidebands}: Upper sideband (USB) at fc+fm and Lower sideband
  (LSB) at fc-fm
\item
  \textbf{Power distribution}: In carrier and two sidebands
\end{itemize}

\end{solutionbox}
\begin{mnemonicbox}
``CAM-SIP'' (Carrier Amplitude Modified, Sidebands In
Pair)

\end{mnemonicbox}
\subsection*{Question 1(c) OR [7
marks]}\label{q1c}

\textbf{Derive the equation for total power in AM, calculate percentage
of power savings in DSB and SSB.}

\begin{solutionbox}

\textbf{Derivation of Total Power in AM:}

\begin{itemize}
\tightlist
\item
  AM signal: v(t) = A_{1}[1 + m sin(ωmt)] sin(ωct)
\item
  Total power: P = P_{(}carrier_{)} + P_{(}sidebands_{)}
\item
  P_{(}carrier_{)} = A_{1}^{2}/2
\item
  P_{(}sidebands_{)} = A_{1}^{2}m^{2}/4
\end{itemize}


{\def\LTcaptype{none} % do not increment counter
\vspace{-5pt}
\captionof{table}{Power Distribution in AM}
\vspace{-10pt}
\begin{longtable}[]{@{}lll@{}}
\toprule\noalign{}
Component & Power Expression & \% of Total Power (m=1) \\
\midrule\noalign{}
\endhead
\bottomrule\noalign{}
\endlastfoot
Carrier & P_{(}c_{)} = A_{1}^{2}/2 & 66.67\% \\
Sidebands & P_{(}s_{)} = A_{1}^{2}m^{2}/4 & 33.33\% \\
Total & P_{(}t_{)} = A_{1}^{2}(1+m^{2}/2)/2 & 100\% \\
\end{longtable}
}

\textbf{Power Savings:}

\begin{itemize}
\tightlist
\item
  \textbf{DSB-SC}: 100\% carrier power saved (66.67\% of total power)

  \begin{itemize}
  \tightlist
  \item
    Only sidebands are transmitted
  \item
    Percentage savings = (P_{(}c_{)}/P_{(}t_{)}) \times 100 = 66.67\%
  \end{itemize}
\item
  \textbf{SSB}: 50\% of sideband power + 100\% carrier power saved

  \begin{itemize}
  \tightlist
  \item
    One sideband + carrier removed
  \item
    Percentage savings = (P_{(}c_{)} + P_{(}s_{)}/2)/P_{(}t_{)} \times 100 = 83.33\%
  \end{itemize}
\end{itemize}

\textbf{Diagram: Power Distribution}

\begin{lstlisting}
    Power
    ^
    |                       
    |  66.67%               
    |   ┌───┐                
    |   │   │                
    |   │   │   16.67%  16.67%    
    |   │   │   ┌───┐  ┌───┐     
    |   │   │   │   │  │   │     
    └───┴───┴───┴───┴──┴───┴────► Freq
        Carrier   LSB    USB
\end{lstlisting}

\end{solutionbox}
\begin{mnemonicbox}
``CAST-83'' (Carrier And Sideband Transmission, 83\%
saved in SSB)

\end{mnemonicbox}
\subsection*{Question 2(a) [3 marks]}\label{q2a}

\textbf{Define (1) Modulation index for AM (2) Modulation index For FM.}

\begin{solutionbox}


{\def\LTcaptype{none} % do not increment counter
\vspace{-5pt}
\captionof{table}{Modulation Index Definitions}
\vspace{-10pt}
\begin{longtable}[]{@{}
  >{\raggedright\arraybackslash}p{(\linewidth - 4\tabcolsep) * \real{0.2075}}
  >{\raggedright\arraybackslash}p{(\linewidth - 4\tabcolsep) * \real{0.3962}}
  >{\raggedright\arraybackslash}p{(\linewidth - 4\tabcolsep) * \real{0.3962}}@{}}
\toprule\noalign{}
\begin{minipage}[b]{\linewidth}\raggedright
Parameter
\end{minipage} & \begin{minipage}[b]{\linewidth}\raggedright
AM Modulation Index
\end{minipage} & \begin{minipage}[b]{\linewidth}\raggedright
FM Modulation Index
\end{minipage} \\
\midrule\noalign{}
\endhead
\bottomrule\noalign{}
\endlastfoot
Definition & Ratio of peak amplitude of modulating signal to peak
amplitude of carrier & Ratio of frequency deviation to modulating
frequency \\
Formula &

m = Am/Ac & mf = Δf/fm \\

Range & 0 \leq m \leq 1 for no distortion & No specific upper limit \\
Effect & Determines \% modulation & Determines bandwidth \\
\end{longtable}
}

\begin{itemize}
\tightlist
\item
  \textbf{AM Modulation Index}: Controls the amplitude variation and
  power distribution
\item
  \textbf{FM Modulation Index}: Determines bandwidth and signal quality
\end{itemize}

\end{solutionbox}
\begin{mnemonicbox}
``ARM-FDM'' (Amplitude Ratio for Modulation,
Frequency Deviation for Modulation)

\end{mnemonicbox}
\subsection*{Question 2(b) [4 marks]}\label{q2b}

\textbf{Draw and explain block diagram for envelope detector.}

\begin{solutionbox}

\textbf{Diagram: Envelope Detector}

\begin{figure}
\centering
\pandocbounded{\includesvg[keepaspectratio]{diagrams/1333201-w2024-q2b.svg}}
\caption{Envelope Detector}
\end{figure}

\begin{lstlisting}
    AM Signal                   
     ───────\rightarrow┌──────┐    ┌──────┐     ┌─────┐     Demodulated
             │      │    │      │     │     │     Output
             │ Diode│───\rightarrow│ RC   │────\rightarrow│ Load│───\rightarrow
             │      │    │Filter│     │     │
             └──────┘    └──────┘     └─────┘
\end{lstlisting}


{\def\LTcaptype{none} % do not increment counter
\vspace{-5pt}
\captionof{table}{Components and Their Functions}
\vspace{-10pt}
\begin{longtable}[]{@{}ll@{}}
\toprule\noalign{}
Component & Function \\
\midrule\noalign{}
\endhead
\bottomrule\noalign{}
\endlastfoot
Diode & Rectifies the AM signal (removes negative half-cycles) \\
RC Filter & Smooths the rectified signal to recover the envelope \\
Load & Provides output circuit and impedance matching \\
\end{longtable}
}

\begin{itemize}
\tightlist
\item
  \textbf{Working principle}: The diode conducts only during positive
  half-cycles
\item
  \textbf{Time constant}: RC must be large enough to prevent ripple but
  small enough to follow modulation
\item
  \textbf{Condition}: RC \textgreater\textgreater{} 1/fc but RC
  \textless\textless{} 1/fm
\end{itemize}

\end{solutionbox}
\begin{mnemonicbox}
``DEER'' (Diode Extracts Envelope Representation)

\end{mnemonicbox}
\subsection*{Question 2(c) [7 marks]}\label{q2c}

\textbf{Draw block diagram of FM radio receiver and explain working of
each block.}

\begin{solutionbox}

\textbf{Diagram: FM Radio Receiver}

\begin{figure}
\centering
\pandocbounded{\includesvg[keepaspectratio]{diagrams/1333201-w2024-q2c.svg}}
\caption{FM Radio Receiver}
\end{figure}

\includegraphics[width=1\linewidth,height=\textheight,keepaspectratio]{mermaid-60a46926.pdf}


{\def\LTcaptype{none} % do not increment counter
\vspace{-5pt}
\captionof{table}{Functions of Each Block}
\vspace{-10pt}
\begin{longtable}[]{@{}ll@{}}
\toprule\noalign{}
Block & Function \\
\midrule\noalign{}
\endhead
\bottomrule\noalign{}
\endlastfoot
Antenna & Receives electromagnetic waves \\
RF Amplifier & Amplifies weak RF signals (88-108 MHz) \\
Mixer & Converts RF to IF frequency (10.7 MHz) \\
Local Oscillator & Generates frequency for mixing (RF+10.7 MHz) \\
IF Amplifier & Amplifies IF signal with fixed gain \\
Limiter & Removes amplitude variations \\
FM Discriminator & Converts frequency variations to voltage \\
Audio Amplifier & Amplifies recovered audio \\
Speaker & Converts electrical to sound waves \\
\end{longtable}
}

\begin{itemize}
\tightlist
\item
  \textbf{Superheterodyne principle}: Uses frequency conversion to
  process signals at fixed IF
\item
  \textbf{Distinctive FM feature}: Limiter removes noise in amplitude
  before demodulation
\end{itemize}

\end{solutionbox}
\begin{mnemonicbox}
``RAMLIDASS'' (RF, Amplifier, Mixer, Local
oscillator, IF, Discriminator, Audio, Speaker System)

\end{mnemonicbox}
\subsection*{Question 2(a) OR [3
marks]}\label{q2a}

\textbf{Draw only Waveform For frequency modulation and Phase
modulation.}

\begin{solutionbox}

\textbf{Diagram: FM and PM Waveforms}

\begin{lstlisting}
Modulating Signal
    ────┐     ┌─────
        │     │     
        │     │     
    ────┘     └─────
        
FM Signal
    /\/\/\     /\/\/\/\/\/\     /\/\/\
   /      \   /            \   /      \
  /        \ /              \ /        \
            
PM Signal
    /\/\/\/\/\  /\/\  /\/\/\/\/\
   /          \/    \/          \
  /                              \
\end{lstlisting}

\textbf{Key Characteristics:}

\begin{itemize}
\tightlist
\item
  \textbf{FM}: Frequency increases when modulating signal is positive
\item
  \textbf{PM}: Phase shifts immediately with amplitude changes
\end{itemize}

\end{solutionbox}
\begin{mnemonicbox}
``FIP-PAF'' (Frequency Increases with Positive
signal, Phase Advances with Faster changes)

\end{mnemonicbox}
\subsection*{Question 2(b) OR [4
marks]}\label{q2b}

\textbf{Define any FOUR characteristics of radio receiver.}

\begin{solutionbox}


{\def\LTcaptype{none} % do not increment counter
\vspace{-5pt}
\captionof{table}{Characteristics of Radio Receiver}
\vspace{-10pt}
\begin{longtable}[]{@{}
  >{\raggedright\arraybackslash}p{(\linewidth - 2\tabcolsep) * \real{0.5714}}
  >{\raggedright\arraybackslash}p{(\linewidth - 2\tabcolsep) * \real{0.4286}}@{}}
\toprule\noalign{}
\begin{minipage}[b]{\linewidth}\raggedright
Characteristic
\end{minipage} & \begin{minipage}[b]{\linewidth}\raggedright
Definition
\end{minipage} \\
\midrule\noalign{}
\endhead
\bottomrule\noalign{}
\endlastfoot
Sensitivity & Ability to receive weak signals (measured in μV or dBm) \\
Selectivity & Ability to separate desired signal from adjacent
channels \\
Fidelity & Accuracy of reproducing the original modulating signal \\
Image Rejection & Ability to reject image frequency interference \\
\end{longtable}
}

\textbf{Additional characteristics:}

\begin{itemize}
\tightlist
\item
  \textbf{Signal-to-Noise Ratio}: Ratio of signal power to noise power
\item
  \textbf{Bandwidth}: Range of frequencies that can be received
\item
  \textbf{Stability}: Ability to maintain tuned frequency
\end{itemize}

\end{solutionbox}
\begin{mnemonicbox}
``SFIS-BSS'' (Sensitivity, Fidelity, Image rejection,
Selectivity - Better Signal Stability)

\end{mnemonicbox}
\subsection*{Question 2(c) OR [7
marks]}\label{q2c}

\textbf{Draw block diagram of AM radio receiver and explain working of
each block.}

\begin{solutionbox}

\textbf{Diagram: AM Radio Receiver}

\includegraphics[width=1\linewidth,height=\textheight,keepaspectratio]{mermaid-c6afcc9b.pdf}


{\def\LTcaptype{none} % do not increment counter
\vspace{-5pt}
\captionof{table}{Functions of Each Block}
\vspace{-10pt}
\begin{longtable}[]{@{}ll@{}}
\toprule\noalign{}
Block & Function \\
\midrule\noalign{}
\endhead
\bottomrule\noalign{}
\endlastfoot
Antenna & Captures AM radio waves \\
RF Tuner \& Amplifier & Selects and amplifies desired frequency \\
Mixer & Converts RF signal to IF (455 kHz) \\
Local Oscillator & Generates frequency for mixing (RF+455 kHz) \\
IF Amplifier & Amplifies IF signal with fixed selectivity \\
Detector & Recovers audio from AM envelope \\
AGC & Provides automatic gain control \\
Audio Amplifier & Amplifies audio signal \\
Speaker & Converts electrical to sound waves \\
\end{longtable}
}

\begin{itemize}
\tightlist
\item
  \textbf{Superheterodyne principle}: Uses frequency conversion for
  better selectivity
\item
  \textbf{AGC feedback loop}: Maintains constant output despite signal
  strength variations
\end{itemize}

\end{solutionbox}
\begin{mnemonicbox}
``ARMLESS'' (Antenna, RF, Mixer, Local oscillator,
Envelope detector, Sound System)

\end{mnemonicbox}
\subsection*{Question 3(a) [3 marks]}\label{q3a}

\textbf{Define quantization. Explain non uniform quantization in brief.}

\begin{solutionbox}

\textbf{Quantization} is the process of converting continuous amplitude
values into discrete levels for digital representation.


{\def\LTcaptype{none} % do not increment counter
\vspace{-5pt}
\captionof{table}{Non-uniform Quantization}
\vspace{-10pt}
\begin{longtable}[]{@{}
  >{\raggedright\arraybackslash}p{(\linewidth - 2\tabcolsep) * \real{0.3810}}
  >{\raggedright\arraybackslash}p{(\linewidth - 2\tabcolsep) * \real{0.6190}}@{}}
\toprule\noalign{}
\begin{minipage}[b]{\linewidth}\raggedright
Aspect
\end{minipage} & \begin{minipage}[b]{\linewidth}\raggedright
Description
\end{minipage} \\
\midrule\noalign{}
\endhead
\bottomrule\noalign{}
\endlastfoot
Definition & Assigning different step sizes for different amplitude
ranges \\
Advantage & Reduces quantization noise for small amplitude signals \\
Implementation & Using companding (compression-expansion) techniques \\
Example & μ-law and A-law companding used in telephony \\
\end{longtable}
}

\begin{itemize}
\tightlist
\item
  \textbf{Working principle}: Smaller step sizes for lower amplitudes,
  larger steps for higher amplitudes
\item
  \textbf{Effect}: Improves SNR for weak signals at the expense of
  strong signals
\end{itemize}

\end{solutionbox}
\begin{mnemonicbox}
``QUEST-CS'' (QUantization with Enhanced Steps -
Compressing Small signals)

\end{mnemonicbox}
\subsection*{Question 3(b) [4 marks]}\label{q3b}

\textbf{Explain Sample and hold Circuit with Waveform.}

\begin{solutionbox}

\textbf{Diagram: Sample and Hold Circuit}

\begin{figure}
\centering
\pandocbounded{\includesvg[keepaspectratio]{diagrams/1333201-w2024-q3b.svg}}
\caption{Sample and Hold Circuit}
\end{figure}

\begin{lstlisting}
    Analog       ┌───────┐      Sampled
    Input ───────│Sample &│─────\rightarrowOutput
                 │ Hold   │
                 └───┬───┘
                     │
    Clock ───────────┘
\end{lstlisting}

\textbf{Diagram: Sample and Hold Waveform}

\begin{lstlisting}
Analog Signal
     /\      /\
    /  \    /  \
   /    \  /    \
  /      \/      \

Clock Pulses
  _   _   _   _   _
 | | | | | | | | | |
 | | | | | | | | | |
 |_| |_| |_| |_| |_|

Sampled Output
     __      __
    |  |    |  |
   _|  |____/  |___
  /                \
\end{lstlisting}

\textbf{Sample and Hold Operation:}

\begin{itemize}
\tightlist
\item
  \textbf{Sampling mode}: Switch closes, capacitor charges to input
  voltage
\item
  \textbf{Hold mode}: Switch opens, capacitor maintains voltage
\item
  \textbf{Parameters}: Acquisition time, aperture time, hold time, droop
  rate
\end{itemize}

\end{solutionbox}
\begin{mnemonicbox}
``CHASED'' (Capacitor Holds Amplitude Samples for
Extended Duration)

\end{mnemonicbox}
\subsection*{Question 3(c) [7 marks]}\label{q3c}

\textbf{What is sampling? Explain types of sampling in brief.}

\begin{solutionbox}

\textbf{Sampling} is the process of converting a continuous-time signal
into a discrete-time signal by taking measurements at regular intervals.


{\def\LTcaptype{none} % do not increment counter
\vspace{-5pt}
\captionof{table}{Types of Sampling}
\vspace{-10pt}
\begin{longtable}[]{@{}
  >{\raggedright\arraybackslash}p{(\linewidth - 4\tabcolsep) * \real{0.1714}}
  >{\raggedright\arraybackslash}p{(\linewidth - 4\tabcolsep) * \real{0.3714}}
  >{\raggedright\arraybackslash}p{(\linewidth - 4\tabcolsep) * \real{0.4571}}@{}}
\toprule\noalign{}
\begin{minipage}[b]{\linewidth}\raggedright
Type
\end{minipage} & \begin{minipage}[b]{\linewidth}\raggedright
Description
\end{minipage} & \begin{minipage}[b]{\linewidth}\raggedright
Characteristics
\end{minipage} \\
\midrule\noalign{}
\endhead
\bottomrule\noalign{}
\endlastfoot
Natural Sampling & Signal is multiplied with rectangular pulses &
Retains original signal shape during pulse \\
Flat-top Sampling & Sample value is held constant during sampling
interval & Creates a staircase-like output \\
Ideal Sampling & Instantaneous samples represented as impulses &
Theoretical concept with zero width pulses \\
Uniform Sampling & Samples taken at equal time intervals & Most common
in practice \\
Non-uniform Sampling & Samples taken at varying intervals & Used for
specialized applications \\
\end{longtable}
}

\textbf{Diagram: Sampling Types}

\begin{figure}
\centering
\pandocbounded{\includesvg[keepaspectratio]{diagrams/1333201-w2024-q3c.svg}}
\caption{Sampling Types}
\end{figure}

\begin{lstlisting}
Original Signal
     /\      /\
    /  \    /  \
   /    \  /    \
  /      \/      \

Natural Sampling
   _     _     _ 
  | |   | |   | |
  | |/\ | |   | |/\
  |/  \| |   |/  \|

Flat-top Sampling
   ___    ___    
  |   |  |   |   
  |   |__|   |___
\end{lstlisting}

\begin{itemize}
\tightlist
\item
  \textbf{Nyquist criterion}: Sampling frequency must be at least twice
  the highest frequency in the signal
\end{itemize}

\end{solutionbox}
\begin{mnemonicbox}
``INFUN'' (Ideal, Natural, Flat-top, Uniform,
Non-uniform)

\end{mnemonicbox}
\subsection*{Question 3(a) OR [3
marks]}\label{q3a}

\textbf{Explain quantization process and its necessity.}

\begin{solutionbox}

\textbf{Quantization Process} maps continuous amplitude values to finite
discrete levels for digital representation.


{\def\LTcaptype{none} % do not increment counter
\vspace{-5pt}
\captionof{table}{Quantization Process and Necessity}
\vspace{-10pt}
\begin{longtable}[]{@{}ll@{}}
\toprule\noalign{}
Aspect & Description \\
\midrule\noalign{}
\endhead
\bottomrule\noalign{}
\endlastfoot
Process & Dividing amplitude range into discrete levels \\
Necessity & Required for analog-to-digital conversion \\
Effect & Introduces quantization error/noise \\
Parameters & Step size, number of levels (2^{n} for n-bit) \\
\end{longtable}
}

\begin{itemize}
\tightlist
\item
  \textbf{Step size calculation}: Step size = (Vmax - Vmin)/2^{n}
\item
  \textbf{Quantization error}: Maximum error is \pmQ/2 where Q is step
  size
\item
  \textbf{Applications}: Digital communication, audio/video processing,
  data storage
\end{itemize}

\end{solutionbox}
\begin{mnemonicbox}
``SEND'' (Step-size Establishes Noise in
Digitization)

\end{mnemonicbox}
\subsection*{Question 3(b) OR [4
marks]}\label{q3b}

\textbf{State and explain Nyquist Criteria for sampling of signal.}

\begin{solutionbox}

\textbf{Nyquist Sampling Theorem} states that to perfectly reconstruct a
bandlimited signal, the sampling frequency must be at least twice the
highest frequency component in the signal.


{\def\LTcaptype{none} % do not increment counter
\vspace{-5pt}
\captionof{table}{Nyquist Criteria}
\vspace{-10pt}
\begin{longtable}[]{@{}ll@{}}
\toprule\noalign{}
Parameter & Description \\
\midrule\noalign{}
\endhead
\bottomrule\noalign{}
\endlastfoot
Criterion & fs \geq 2fmax \\
Nyquist Rate & 2fmax (minimum sampling frequency) \\
Nyquist Interval & 1/(2fmax) (maximum sampling period) \\
Aliasing & Occurs when fs \textless{} 2fmax \\
\end{longtable}
}

\textbf{Diagram: Sampling Effects}

\begin{lstlisting}
    Proper Sampling (fs > 2fmax)
    Original: /\/\/\/\
    Samples:  * * * * * * * *
    Result:   /\/\/\/\

    Aliasing (fs < 2fmax)
    Original: /\/\/\/\/\/\/\
    Samples:  *   *   *   *
    Result:   /\/\    (lower frequency)
\end{lstlisting}

\begin{itemize}
\tightlist
\item
  \textbf{Consequences of undersampling}: Aliasing (frequency folding)
\item
  \textbf{Practical application}: Anti-aliasing filters used before
  sampling
\end{itemize}

\end{solutionbox}
\begin{mnemonicbox}
``TRAP-A'' (Twice Rate Avoids Problematic Aliasing)

\end{mnemonicbox}
\subsection*{Question 3(c) OR [7
marks]}\label{q3c}

\textbf{Explain PAM, PWM and PPM with waveform.}

\begin{solutionbox}


{\def\LTcaptype{none} % do not increment counter
\vspace{-5pt}
\captionof{table}{Pulse Modulation Techniques}
\vspace{-10pt}
\begin{longtable}[]{@{}
  >{\raggedright\arraybackslash}p{(\linewidth - 6\tabcolsep) * \real{0.2000}}
  >{\raggedright\arraybackslash}p{(\linewidth - 6\tabcolsep) * \real{0.2364}}
  >{\raggedright\arraybackslash}p{(\linewidth - 6\tabcolsep) * \real{0.3273}}
  >{\raggedright\arraybackslash}p{(\linewidth - 6\tabcolsep) * \real{0.2364}}@{}}
\toprule\noalign{}
\begin{minipage}[b]{\linewidth}\raggedright
Technique
\end{minipage} & \begin{minipage}[b]{\linewidth}\raggedright
Description
\end{minipage} & \begin{minipage}[b]{\linewidth}\raggedright
Parameter Varied
\end{minipage} & \begin{minipage}[b]{\linewidth}\raggedright
Application
\end{minipage} \\
\midrule\noalign{}
\endhead
\bottomrule\noalign{}
\endlastfoot
PAM & Pulse Amplitude Modulation & Amplitude of pulses & Simple ADC
systems \\
PWM & Pulse Width Modulation & Width/duration of pulses & Motor control,
power regulation \\
PPM & Pulse Position Modulation & Position/timing of pulses & High noise
immunity systems \\
\end{longtable}
}

\textbf{Diagram: Pulse Modulation Waveforms}

\begin{lstlisting}
Modulating Signal
    /\        /\
   /  \      /  \
  /    \    /    \
 /      \  /      \

PAM
  |  |    |  |    |  |
  |  |    |  |    |  |
  |  |    |  |    |  |
  █  █    █  █    █  █

PWM
  █████    ███    █████
  |    |   | |    |    |
  |    |   | |    |    |
  |    |   | |    |    |

PPM
  █ █ █ █ █ █ █ █
  | | | | | | | |
  | | | | | | | |
  | | | | | | | |
\end{lstlisting}

\begin{itemize}
\tightlist
\item
  \textbf{PAM}: Simplest form, most susceptible to noise
\item
  \textbf{PWM}: Better noise immunity, easy generation
\item
  \textbf{PPM}: Best noise immunity, requires precise timing
\end{itemize}

\end{solutionbox}
\begin{mnemonicbox}
``AWP-PAW'' (Amplitude, Width, Position - Pulse
Alteration Ways)

\end{mnemonicbox}
\subsection*{Question 4(a) [3 marks]}\label{q4a}

\textbf{What is slop overload noise and granular noise in DM?}

\begin{solutionbox}


{\def\LTcaptype{none} % do not increment counter
\vspace{-5pt}
\captionof{table}{Noise Types in Delta Modulation}
\vspace{-10pt}
\begin{longtable}[]{@{}
  >{\raggedright\arraybackslash}p{(\linewidth - 6\tabcolsep) * \real{0.2927}}
  >{\raggedright\arraybackslash}p{(\linewidth - 6\tabcolsep) * \real{0.2927}}
  >{\raggedright\arraybackslash}p{(\linewidth - 6\tabcolsep) * \real{0.1707}}
  >{\raggedright\arraybackslash}p{(\linewidth - 6\tabcolsep) * \real{0.2439}}@{}}
\toprule\noalign{}
\begin{minipage}[b]{\linewidth}\raggedright
Noise Type
\end{minipage} & \begin{minipage}[b]{\linewidth}\raggedright
Definition
\end{minipage} & \begin{minipage}[b]{\linewidth}\raggedright
Cause
\end{minipage} & \begin{minipage}[b]{\linewidth}\raggedright
Solution
\end{minipage} \\
\midrule\noalign{}
\endhead
\bottomrule\noalign{}
\endlastfoot
Slope Overload Noise & Error when signal slope exceeds step size
capability & Step size too small for rapidly changing signals & Increase
step size or sampling frequency \\
Granular Noise & Error due to continuous hunting around slowly varying
signals & Step size too large for slowly changing signals & Decrease
step size \\
\end{longtable}
}

\textbf{Diagram: DM Noise Types}

\begin{lstlisting}
Slope Overload:
  Actual  /‾‾‾‾
         /
        /
       /      
   ___/       
  /
 /  DM Output (steps can't keep up)

Granular Noise:
  Actual  _________
         
   /‾\/‾\/‾\/‾\/‾\  DM Output (continuous zigzag)
\end{lstlisting}

\end{solutionbox}
\begin{mnemonicbox}
``FAST-SLOW'' (Fast signals cause Slope overload,
SLOW signals cause Granular noise)

\end{mnemonicbox}
\subsection*{Question 4(b) [4 marks]}\label{q4b}

\textbf{Draw and explain TDM frame.}

\begin{solutionbox}

\textbf{Diagram: TDM Frame Structure}

\begin{figure}
\centering
\pandocbounded{\includesvg[keepaspectratio]{diagrams/1333201-w2024-q4b.svg}}
\caption{TDM Frame Structure}
\end{figure}

\begin{lstlisting}
    ┌───────────────────────────────────┐
    │ FS │ CH1 │ CH2 │ CH3 │...│ CHn │ FS │
    └───────────────────────────────────┘
       |    |     |     |        |     |
       |    |     |     |        |     └── Frame Sync for next frame
       |    |     |     |        └──────── Last channel sample
       |    |     |     └──────────────── Channel 3 sample
       |    |     └───────────────────── Channel 2 sample
       |    └─────────────────────────── Channel 1 sample
       └────────────────────────────────── Frame Synchronization
\end{lstlisting}


{\def\LTcaptype{none} % do not increment counter
\vspace{-5pt}
\captionof{table}{TDM Frame Components}
\vspace{-10pt}
\begin{longtable}[]{@{}ll@{}}
\toprule\noalign{}
Component & Description \\
\midrule\noalign{}
\endhead
\bottomrule\noalign{}
\endlastfoot
Frame Sync (FS) & Pattern that marks the start of frame \\
Time Slot & Portion allocated to one channel \\
Channel Sample & Data from a specific channel \\
Frame Length & Total duration (FS + all channels) \\
\end{longtable}
}

\begin{itemize}
\tightlist
\item
  \textbf{Working principle}: Allocates different time slots to
  different channels
\item
  \textbf{Synchronization}: Essential for proper demultiplexing
\item
  \textbf{Types}: Synchronous TDM (fixed slots) and Statistical TDM
  (dynamic allocation)
\end{itemize}

\end{solutionbox}
\begin{mnemonicbox}
``FAST-Ch'' (Frame And Slots for Transmitting
Channels)

\end{mnemonicbox}
\subsection*{Question 4(c) [7 marks]}\label{q4c}

\textbf{Describe the function of each block of PCM transmitter and
Receiver. Give application, advantage and disadvantage of PCM system.}

\begin{solutionbox}

\textbf{Diagram: PCM System}

\begin{figure}
\centering
\pandocbounded{\includesvg[keepaspectratio]{diagrams/1333201-w2024-q4c.svg}}
\caption{PCM System}
\end{figure}

\includegraphics[width=1\linewidth,height=\textheight,keepaspectratio]{mermaid-e8258a78.pdf}


{\def\LTcaptype{none} % do not increment counter
\vspace{-5pt}
\captionof{table}{PCM Block Functions}
\vspace{-10pt}
\begin{longtable}[]{@{}ll@{}}
\toprule\noalign{}
Block & Function \\
\midrule\noalign{}
\endhead
\bottomrule\noalign{}
\endlastfoot
Sampler & Converts analog signal to PAM signal \\
Quantizer & Assigns discrete levels to samples \\
Encoder & Converts quantized levels to binary code \\
Line Coder & Converts binary to transmission format \\
Line Decoder & Recovers binary from received signal \\
Decoder & Converts binary back to quantized levels \\
Reconstruction Filter & Smooths decoded output into analog signal \\
\end{longtable}
}

\textbf{Applications, Advantages and Disadvantages:}


{\def\LTcaptype{none} % do not increment counter
\vspace{-5pt}
\captionof{table}{PCM System Characteristics}
\vspace{-10pt}
\begin{longtable}[]{@{}
  >{\raggedright\arraybackslash}p{(\linewidth - 2\tabcolsep) * \real{0.4348}}
  >{\raggedright\arraybackslash}p{(\linewidth - 2\tabcolsep) * \real{0.5652}}@{}}
\toprule\noalign{}
\begin{minipage}[b]{\linewidth}\raggedright
Category
\end{minipage} & \begin{minipage}[b]{\linewidth}\raggedright
Description
\end{minipage} \\
\midrule\noalign{}
\endhead
\bottomrule\noalign{}
\endlastfoot
Applications & Telephone systems, CD audio, Digital TV, Mobile
communications \\
Advantages & Immune to noise, Signal regeneration possible, Compatible
with digital systems \\
Disadvantages & Requires higher bandwidth, Higher complexity,
Quantization noise \\
\end{longtable}
}

\end{solutionbox}
\begin{mnemonicbox}
``SEQUEL-DR'' (Sample, Quantize, Encode - Line code,
Decode, Reconstruct)

\end{mnemonicbox}
\subsection*{Question 4(a) OR [3
marks]}\label{q4a}

\textbf{Give difference between DM and ADM modulation.}

\begin{solutionbox}


{\def\LTcaptype{none} % do not increment counter
\vspace{-5pt}
\captionof{table}{Comparison between DM and ADM}
\vspace{-10pt}
\begin{longtable}[]{@{}
  >{\raggedright\arraybackslash}p{(\linewidth - 4\tabcolsep) * \real{0.1692}}
  >{\raggedright\arraybackslash}p{(\linewidth - 4\tabcolsep) * \real{0.3385}}
  >{\raggedright\arraybackslash}p{(\linewidth - 4\tabcolsep) * \real{0.4923}}@{}}
\toprule\noalign{}
\begin{minipage}[b]{\linewidth}\raggedright
Parameter
\end{minipage} & \begin{minipage}[b]{\linewidth}\raggedright
Delta Modulation (DM)
\end{minipage} & \begin{minipage}[b]{\linewidth}\raggedright
Adaptive Delta Modulation (ADM)
\end{minipage} \\
\midrule\noalign{}
\endhead
\bottomrule\noalign{}
\endlastfoot
Step Size & Fixed & Variable (adapts to signal slope) \\
Tracking Ability & Limited & Better signal tracking \\
Noise Performance & Suffers from slope overload and granular noise &
Reduced noise problems \\
Complexity & Simpler & More complex \\
\end{longtable}
}

\textbf{Diagram: DM vs ADM Tracking}

\begin{lstlisting}
Input Signal:   /‾‾‾‾\
               /      \
              /        \
             /          \

DM Output:   /‾\/‾\/‾\
            /         \/‾\/‾\

ADM Output: /‾‾\/‾‾‾\
           /         \‾‾\/‾‾\
           (larger steps for steep slopes)
\end{lstlisting}

\end{solutionbox}
\begin{mnemonicbox}
``FAST-VAR'' (Fixed And Simple Tracking vs Variable
Adaptive Response)

\end{mnemonicbox}
\subsection*{Question 4(b) OR [4
marks]}\label{q4b}

\textbf{Explain Block diagram of basic PCM-TDM system.}

\begin{solutionbox}

\textbf{Diagram: PCM-TDM System}

\includegraphics[width=1\linewidth,height=\textheight,keepaspectratio]{mermaid-39f881c0.pdf}


{\def\LTcaptype{none} % do not increment counter
\vspace{-5pt}
\captionof{table}{PCM-TDM System Components}
\vspace{-10pt}
\begin{longtable}[]{@{}ll@{}}
\toprule\noalign{}
Component & Function \\
\midrule\noalign{}
\endhead
\bottomrule\noalign{}
\endlastfoot
Low-pass Filters & Limit bandwidth of input signals \\
Multiplexer & Combines multiple signals into time slots \\
PCM Encoder & Converts to digital (sample, quantize, encode) \\
Transmission Channel & Carries digitized, multiplexed signal \\
PCM Decoder & Reconstructs quantized samples \\
Demultiplexer & Separates channels from time slots \\
\end{longtable}
}

\begin{itemize}
\tightlist
\item
  \textbf{Working principle}: Combines time division multiplexing with
  pulse code modulation
\item
  \textbf{Applications}: Digital telephony, digital audio broadcasting,
  communication networks
\end{itemize}

\end{solutionbox}
\begin{mnemonicbox}
``FLIMPED'' (Filter, Limit, Multiplex, PCM Encode,
Decode)

\end{mnemonicbox}
\subsection*{Question 4(c) OR [7
marks]}\label{q4c}

\textbf{Explain DPCM modulator with equation and waveform.}

\begin{solutionbox}

\textbf{Differential Pulse Code Modulation (DPCM)} encodes the
difference between the current sample and a predicted value based on
previous samples.

\textbf{Equation:}

\begin{itemize}
\tightlist
\item
  Error signal: e(n) = x(n) - x̂(n)
\item
  Where x(n) is current sample, x̂(n) is predicted sample
\item
  Prediction: x̂(n) = Σ(aᵢ \times x(n-i))
\item
  Transmitted signal: DPCM output = Q[e(n)]
\end{itemize}

\textbf{Diagram: DPCM Modulator}

\includegraphics[width=1\linewidth,height=\textheight,keepaspectratio]{mermaid-b20a1485.pdf}

\textbf{Diagram: DPCM Waveform}

\begin{lstlisting}
Original Samples:
  *   *   *   *   *
  |   |   |   |   |
  |   |   |   |   |
  |   |   |   |   |
  
Predicted Samples:
    o   o   o   o
    |   |   |   |
    |   |   |   |
    |   |   |   |
    
Difference (DPCM):
  ↕   ↕   ↕   ↕   ↕  (smaller values)
\end{lstlisting}


{\def\LTcaptype{none} % do not increment counter
\vspace{-5pt}
\captionof{table}{DPCM Characteristics}
\vspace{-10pt}
\begin{longtable}[]{@{}ll@{}}
\toprule\noalign{}
Feature & Description \\
\midrule\noalign{}
\endhead
\bottomrule\noalign{}
\endlastfoot
Advantage & Reduced bit rate (30-50\% compared to PCM) \\
Prediction & Uses previous sample(s) for current prediction \\
Complexity & Higher than PCM but lower than ADPCM \\
Application & Speech coding, image compression \\
\end{longtable}
}

\end{solutionbox}
\begin{mnemonicbox}
``PQED'' (Predict, Quantize Error, Encode Difference)

\end{mnemonicbox}
\subsection*{Question 5(a) [3 marks]}\label{q5a}

\textbf{Define Antenna and radiation pattern and polarization.}

\begin{solutionbox}


{\def\LTcaptype{none} % do not increment counter
\vspace{-5pt}
\captionof{table}{Antenna Definitions}
\vspace{-10pt}
\begin{longtable}[]{@{}
  >{\raggedright\arraybackslash}p{(\linewidth - 2\tabcolsep) * \real{0.3333}}
  >{\raggedright\arraybackslash}p{(\linewidth - 2\tabcolsep) * \real{0.6667}}@{}}
\toprule\noalign{}
\begin{minipage}[b]{\linewidth}\raggedright
Term
\end{minipage} & \begin{minipage}[b]{\linewidth}\raggedright
Definition
\end{minipage} \\
\midrule\noalign{}
\endhead
\bottomrule\noalign{}
\endlastfoot
Antenna & A device that converts electrical energy into electromagnetic
waves and vice versa \\
Radiation Pattern & Graphical representation of radiation properties of
an antenna as a function of space coordinates \\
Polarization & Orientation of the electric field vector of the
electromagnetic wave radiated by the antenna \\
\end{longtable}
}

\textbf{Types of Polarization:}

\begin{itemize}
\tightlist
\item
  \textbf{Linear}: Electric field oscillates in one direction (vertical,
  horizontal)
\item
  \textbf{Circular}: Electric field rotates with constant amplitude
  (RHCP, LHCP)
\item
  \textbf{Elliptical}: Electric field rotates with varying amplitude
\end{itemize}

\end{solutionbox}
\begin{mnemonicbox}
``WAVE-PRO'' (Wireless Antenna Validates
Electromagnetic Propagation, Radiation, Orientation)

\end{mnemonicbox}
\subsection*{Question 5(b) [4 marks]}\label{q5b}

\textbf{Explain Microstrip Antenna with sketch.}

\begin{solutionbox}

\textbf{Diagram: Microstrip Patch Antenna}

\begin{figure}
\centering
\pandocbounded{\includesvg[keepaspectratio]{diagrams/1333201-w2024-q5b.svg}}
\caption{Microstrip Patch Antenna}
\end{figure}

\begin{lstlisting}
    ┌───────────────────┐  \leftarrowPatch (radiating element)
    │                   │
    │                   │
    └───────────────────┘
    ┌───────────────────────────────┐
    │                               │  \leftarrowDielectric substrate
    └───────────────────────────────┘
    ┌───────────────────────────────┐
    │                               │  \leftarrowGround plane
    └───────────────────────────────┘
              │
              │ Feed point
              ▼
\end{lstlisting}


{\def\LTcaptype{none} % do not increment counter
\vspace{-5pt}
\captionof{table}{Microstrip Antenna Components}
\vspace{-10pt}
\begin{longtable}[]{@{}ll@{}}
\toprule\noalign{}
Component & Function \\
\midrule\noalign{}
\endhead
\bottomrule\noalign{}
\endlastfoot
Patch & Radiating element (usually copper) \\
Substrate & Dielectric material between patch and ground \\
Ground Plane & Metal layer at bottom \\
Feed Point & Connection point for signal \\
\end{longtable}
}

\begin{itemize}
\tightlist
\item
  \textbf{Working principle}: Fringing fields at edges cause radiation
\item
  \textbf{Advantages}: Low profile, lightweight, easy fabrication,
  compatible with PCB
\item
  \textbf{Applications}: Mobile devices, satellites, aircraft, RFID tags
\end{itemize}

\end{solutionbox}
\begin{mnemonicbox}
``SPGF'' (Substrate, Patch, Ground, Feed)

\end{mnemonicbox}
\subsection*{Question 5(c) [7 marks]}\label{q5c}

\textbf{Explain delta modulation with necessary sketch and waveform.}

\begin{solutionbox}

Delta Modulation (DM) is the simplest form of differential pulse code
modulation where the difference between successive samples is encoded
into a single bit.

\textbf{Diagram: Delta Modulator}

\begin{figure}
\centering
\pandocbounded{\includesvg[keepaspectratio]{diagrams/1333201-w2024-q5c.svg}}
\caption{Delta Modulator}
\end{figure}

\includegraphics[width=1\linewidth,height=\textheight,keepaspectratio]{mermaid-2a251cf7.pdf}

\textbf{Diagram: Delta Modulation Waveform}

\begin{lstlisting}
Input Signal:
        /‾‾‾‾‾\
       /       \
      /         \
     /           \
    /             \

Clock Pulses:
    ˉ|ˉ|ˉ|ˉ|ˉ|ˉ|ˉ|ˉ|ˉ|ˉ|ˉ|ˉ|ˉ|ˉ|ˉ|ˉ

DM Output (bits):
    1 1 1 1 0 0 0 0 0 1 1 1 0 0 0 0

Step Approximation:
       /‾\/‾\
      /     \
     /       \/‾\
    /           \
\end{lstlisting}


{\def\LTcaptype{none} % do not increment counter
\vspace{-5pt}
\captionof{table}{Delta Modulation Characteristics}
\vspace{-10pt}
\begin{longtable}[]{@{}
  >{\raggedright\arraybackslash}p{(\linewidth - 2\tabcolsep) * \real{0.5517}}
  >{\raggedright\arraybackslash}p{(\linewidth - 2\tabcolsep) * \real{0.4483}}@{}}
\toprule\noalign{}
\begin{minipage}[b]{\linewidth}\raggedright
Characteristic
\end{minipage} & \begin{minipage}[b]{\linewidth}\raggedright
Description
\end{minipage} \\
\midrule\noalign{}
\endhead
\bottomrule\noalign{}
\endlastfoot
Bit Rate & 1 bit per sample \\
Step Size & Fixed (major limitation) \\
Slope Overload & Occurs when signal changes faster than step size can
track \\
Granular Noise & Occurs in slowly changing signals (continuous
hunting) \\
Advantages & Simplicity, low bit rate \\
Disadvantages & Limited dynamic range, noise problems \\
\end{longtable}
}

\end{solutionbox}
\begin{mnemonicbox}
``SIGN-UP'' (SInGle bit, Next step Up or down,
Predict)

\end{mnemonicbox}
\subsection*{Question 5(a) OR [3
marks]}\label{q5a}

\textbf{What is smart antenna? list application of it.}

\begin{solutionbox}

A \textbf{Smart Antenna} is an adaptive array system that uses digital
signal processing algorithms to dynamically adjust its radiation pattern
to enhance communication performance.


{\def\LTcaptype{none} % do not increment counter
\vspace{-5pt}
\captionof{table}{Smart Antenna Applications}
\vspace{-10pt}
\begin{longtable}[]{@{}
  >{\raggedright\arraybackslash}p{(\linewidth - 2\tabcolsep) * \real{0.5909}}
  >{\raggedright\arraybackslash}p{(\linewidth - 2\tabcolsep) * \real{0.4091}}@{}}
\toprule\noalign{}
\begin{minipage}[b]{\linewidth}\raggedright
Application
\end{minipage} & \begin{minipage}[b]{\linewidth}\raggedright
Benefit
\end{minipage} \\
\midrule\noalign{}
\endhead
\bottomrule\noalign{}
\endlastfoot
Cellular Base Stations & Increased capacity and coverage \\
Wireless LAN & Improved throughput and reduced interference \\
Satellite Communications & Better signal quality and power efficiency \\
Military Communications & Enhanced security and jam resistance \\
IoT Networks & Extended battery life, improved connectivity \\
\end{longtable}
}

\begin{itemize}
\tightlist
\item
  \textbf{Working principle}: Uses beamforming to focus signal energy
  toward desired users
\item
  \textbf{Types}: Switched beam systems and adaptive array systems
\end{itemize}

\end{solutionbox}
\begin{mnemonicbox}
``SWIM-CM'' (Smart Wireless In
Mobile-Cellular-Military)

\end{mnemonicbox}
\subsection*{Question 5(b) OR [4
marks]}\label{q5b}

\textbf{Explain parabolic reflector antenna With Sketch.}

\begin{solutionbox}

\textbf{Diagram: Parabolic Reflector Antenna}

\begin{lstlisting}
                  ╱│╲
               ╱   │  ╲
            ╱      │     ╲
         ╱         │        ╲
      ╱            │           ╲
   ╱               │              ╲
 ╱─────────────────┼─────────────────╲
                   │
                   │
                   ▼
                 Feed
                 Point
\end{lstlisting}


{\def\LTcaptype{none} % do not increment counter
\vspace{-5pt}
\captionof{table}{Parabolic Reflector Components}
\vspace{-10pt}
\begin{longtable}[]{@{}ll@{}}
\toprule\noalign{}
Component & Function \\
\midrule\noalign{}
\endhead
\bottomrule\noalign{}
\endlastfoot
Parabolic Dish & Reflects and focuses signals \\
Feed Horn & Radiates/receives signals at focal point \\
Supporting Structure & Maintains geometry and stability \\
Waveguide & Connects feed horn to transmitter/receiver \\
\end{longtable}
}

\begin{itemize}
\tightlist
\item
  \textbf{Working principle}: Incoming parallel rays are reflected to
  focus at focal point
\item
  \textbf{Characteristics}: High gain, directivity, narrow beamwidth
\item
  \textbf{Applications}: Satellite communication, radio astronomy,
  radar, microwave links
\end{itemize}

\end{solutionbox}
\begin{mnemonicbox}
``PFGH'' (Parabolic Focus Gives High-gain)

\end{mnemonicbox}
\subsection*{Question 5(c) OR [7
marks]}\label{q5c}

\textbf{Explain Adaptive Delta modulation with necessary sketch and
waveform.}

\begin{solutionbox}

Adaptive Delta Modulation (ADM) improves on standard DM by dynamically
adjusting the step size according to the input signal characteristics.

\textbf{Diagram: Adaptive Delta Modulator}

\includegraphics[width=1\linewidth,height=\textheight,keepaspectratio]{mermaid-d158d701.pdf}

\textbf{Diagram: ADM Waveform}

\begin{lstlisting}
Input Signal:
        /‾‾‾‾‾\
       /       \
      /         \
     /           \
    /             \

ADM Output (variable step):
       /‾‾‾\
      /     \
     /       \
    /         \
   /           \
  (larger steps for steep slopes)
\end{lstlisting}


{\def\LTcaptype{none} % do not increment counter
\vspace{-5pt}
\captionof{table}{ADM Characteristics}
\vspace{-10pt}
\begin{longtable}[]{@{}ll@{}}
\toprule\noalign{}
Aspect & Description \\
\midrule\noalign{}
\endhead
\bottomrule\noalign{}
\endlastfoot
Step Size & Variable (adapts to signal slope) \\
Control Logic & Increases step size for consecutive same bits \\
Advantages & Reduced slope overload and granular noise \\
Disadvantages & More complex than DM \\
Applications & Speech coding, telemetry, digital telephony \\
Performance & Better SNR than DM at same bit rate \\
\end{longtable}
}

\begin{itemize}
\tightlist
\item
  \textbf{Step size adaptation}: μ(n) = μ(n-1) \times K if consecutive bits
  are same
\item
  \textbf{Step size adaptation}: μ(n) = μ(n-1) / K if consecutive bits
  change
\end{itemize}

\end{solutionbox}
\begin{mnemonicbox}
``ADVISED'' (ADaptive Variable Increment Step for
Enhanced Delta modulation)

\end{mnemonicbox}

\end{document}
