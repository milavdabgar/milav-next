\documentclass[10pt,a4paper]{article}

% content/resources/templates/preamble.tex
\usepackage[margin=0.6in]{geometry}
\author{Milav Dabgar}
\usepackage{amsmath,amssymb,amsthm}
\usepackage{booktabs}
\usepackage{multirow}
\usepackage{xcolor}
\usepackage{tcolorbox}
\tcbuselibrary{breakable,skins}
\usepackage[colorlinks=true,linkcolor=blue]{hyperref}
\usepackage{titlesec}
\usepackage{enumitem}
\usepackage{tikz}
\usepackage{pgfplots}
\usepackage{circuitikz}
\usepackage[version=4]{mhchem}
\usepackage{longtable}
\usepackage{array}
\usepackage{float}
\usepackage{caption}
\usepackage{listings}

\lstset{
  basicstyle=\small\ttfamily,
  breaklines=true,
  breakatwhitespace=false,
  postbreak=\mbox{\textcolor{red}{$\hookrightarrow$}\space},
  float=false,
  numbers=left,
  numberstyle=\tiny\color{gray},
  numbersep=10pt,
  xleftmargin=2em,
  keywordstyle=\color{blue},
  commentstyle=\color{green!60!black},
  stringstyle=\color{purple},
  backgroundcolor=\color{gray!5},
  showstringspaces=false,
  tabsize=2,
  captionpos=b,
  keepspaces=true,
  columns=flexible
}

\pgfplotsset{compat=1.18}
\usetikzlibrary{shapes,arrows,positioning,calc,patterns,decorations.pathmorphing,decorations.markings,arrows.meta}

% Color scheme
\definecolor{headcolor}{RGB}{0,102,204}
\definecolor{keycolor}{RGB}{220,20,60}
\definecolor{solutioncolor}{RGB}{34,139,34}
\definecolor{mnemoniccolor}{RGB}{148,0,211}
\definecolor{codecolor}{RGB}{0,0,100}

% Spacing
\setlength{\parskip}{3pt}
\setlist[itemize]{nosep}
\setlist[enumerate]{nosep}

% Title formatting
\titleformat{\section}{\Large\bfseries\color{headcolor}}{\thesection}{1em}{}
\titleformat{\subsection}{\large\bfseries\color{headcolor}}{\thesubsection}{1em}{}

% Pandoc tightlist compatibility
\providecommand{\tightlist}{%
  \setlength{\itemsep}{0pt}\setlength{\parskip}{0pt}}

% Pandoc longtable compatibility
\newcounter{none}
\def\thenone{}


% content/resources/templates/gujarati-boxes.tex
\usepackage{fontspec}
\usepackage{polyglossia}

% Set Gujarati as main language (document is primarily in Gujarati)
% Note: gloss-gujarati.ldf doesn't exist in polyglossia, but it will use hyphenation patterns
\setdefaultlanguage{gujarati}
\setotherlanguage{english}

% Configure Gujarati font properly
% Use Language=Default to prevent polyglossia from trying to add language-specific features
% that don't exist for Gujarati, which causes "empty feature" warnings
\newfontfamily\gujaratifont[Script=Gujarati,AutoFakeBold=2.5,AutoFakeSlant=0.3]{Noto Sans Gujarati}
\setmainfont[Script=Gujarati,AutoFakeBold=2.5,AutoFakeSlant=0.3]{Noto Sans Gujarati}
% Use Noto Sans Gujarati for monospace to support Gujarati in text
\setmonofont[Scale=0.9]{Noto Sans Gujarati}

% Configure English to use the same font
\newfontfamily\englishfont[Script=Gujarati,AutoFakeBold=2.5,AutoFakeSlant=0.3]{Noto Sans Gujarati}

% Translations for polyglossia
\gappto\captionsgujarati{
  \renewcommand{\tablename}{કોષ્ટક}
  \renewcommand{\figurename}{આકૃતિ}
}

% Helper for TikZ nodes to ensure Gujarati font
\newcommand{\gu}[1]{{\gujaratifont #1}}

% Custom environments
\newtcolorbox{solutionbox}{
    breakable,
    enhanced,
    colback=solutioncolor!5!white,
    colframe=solutioncolor!75!black,
    fonttitle=\bfseries,
    title=જવાબ
}

\newtcolorbox{solutionboxnobreak}{
 colback=solutioncolor!5!white,
 colframe=solutioncolor!75!black,
 fonttitle=\bfseries,
 title=જવાબ
}

\newtcolorbox{keyformula}{
 breakable,
 enhanced,
 colback=keycolor!5!white,
 colframe=keycolor!75!black,
 fonttitle=\bfseries,
 title=રાસાયણિક સમીકરણ/સૂત્ર
}

\newtcolorbox{mnemonicbox}{
 breakable,
 enhanced,
 colback=mnemoniccolor!5!white,
 colframe=mnemoniccolor!75!black,
 fonttitle=\bfseries,
 title=મેમરી ટ્રીક
}


\begin{document}

\begin{center}
{\Huge\bfseries\color{headcolor} Subject Name (Gujarati)}\\[5pt]
{\LARGE 1333201 -- Summer 2024}\\[3pt]
{\large Semester 1 Study Material}\\[3pt]
{\normalsize\textit{Detailed Solutions and Explanations}}
\end{center}

\vspace{10pt}

\subsection*{Question 1(a) [3 marks]}\label{q1a}

\textbf{મોડ્યુલેશનની વ્યાખ્યા આપો અને તેની જરૂરિયત સમજાવો.}

\begin{solutionbox}
મોડ્યુલેશન એ ઉચ્ચ આવૃત્તિની કેરિયર સિગ્નલના એક અથવા વધુ ગુણધર્મોને
માહિતી ધરાવતા મોડ્યુલેટિંગ સિગ્નલ સાથે બદલવાની પ્રક્રિયા છે.


{\def\LTcaptype{none} % do not increment counter
\vspace{-5pt}
\captionof{table}{મોડ્યુલેશનની જરૂરિયત}
\vspace{-10pt}
\begin{longtable}[]{@{}
  >{\raggedright\arraybackslash}p{(\linewidth - 2\tabcolsep) * \real{0.5263}}
  >{\raggedright\arraybackslash}p{(\linewidth - 2\tabcolsep) * \real{0.4737}}@{}}
\toprule\noalign{}
\begin{minipage}[b]{\linewidth}\raggedright
જરૂરિયાત
\end{minipage} & \begin{minipage}[b]{\linewidth}\raggedright
સમજૂતી
\end{minipage} \\
\midrule\noalign{}
\endhead
\bottomrule\noalign{}
\endlastfoot
\textbf{એન્ટેના સાઈઝ ઘટાડવા} & આવૃત્તિ વધારીને વ્યવહારિક એન્ટેના સાઈઝ (λ/4)
મેળવવા \\
\textbf{સિગ્નલ પ્રસારણ} & ઉચ્ચ આવૃત્તિઓ વાતાવરણમાં વધુ દૂર સુધી પ્રવાસ કરે છે \\
\textbf{મલ્ટિપ્લેક્સિંગ} & એક સાથે ઘણા સિગ્નલ્સને ટ્રાન્સમિટ કરવાની મંજૂરી આપે છે \\
\textbf{દખલગીરી ઘટાડવી} & સિગ્નલને ઓછા નોઈઝ/ઇન્ટરફેરન્સવાળા બેન્ડમાં શિફ્ટ કરે
છે \\
\textbf{બેન્ડવિડ્થ ફાળવણી} & વિવિધ સેવાઓ દ્વારા સ્પેક્ટ્રમના કાર્યક્ષમ ઉપયોગને સક્ષમ
બનાવે છે \\
\end{longtable}
}

\end{solutionbox}
\begin{mnemonicbox}
``ASPIM'' - Antenna size, Signal propagation, Proper
multiplexing, Interference reduction, Manage bandwidth

\end{mnemonicbox}
\subsection*{Question 1(b) [4 marks]}\label{q1b}

\textbf{કોમ્યુનીકેશન સિસ્ટમનો બ્લોક ડાયાગ્રામ દોરો અને સમજાવો.}

\begin{solutionbox}
કોમ્યુનિકેશન સિસ્ટમ માહિતીને સ્ત્રોતથી ચેનલ મારફતે ગંતવ્ય સુધી
પહોંચાડે છે.

\begin{figure}
\centering
\pandocbounded{\includesvg[keepaspectratio]{diagrams/1333201-s2024-q1b.svg}}
\caption{કોમ્યુનિકેશન સિસ્ટમ}
\end{figure}


{\def\LTcaptype{none} % do not increment counter
\vspace{-5pt}
\captionof{table}{કોમ્યુનિકેશન સિસ્ટમના ઘટકો}
\vspace{-10pt}
\begin{longtable}[]{@{}
  >{\raggedright\arraybackslash}p{(\linewidth - 2\tabcolsep) * \real{0.5000}}
  >{\raggedright\arraybackslash}p{(\linewidth - 2\tabcolsep) * \real{0.5000}}@{}}
\toprule\noalign{}
\begin{minipage}[b]{\linewidth}\raggedright
ઘટક
\end{minipage} & \begin{minipage}[b]{\linewidth}\raggedright
કાર્ય
\end{minipage} \\
\midrule\noalign{}
\endhead
\bottomrule\noalign{}
\endlastfoot
\textbf{માહિતી સ્ત્રોત} & ટ્રાન્સમિટ કરવા માટેનો સંદેશ ઉત્પન્ન કરે છે (અવાજ, વિડિઓ,
ડેટા) \\
\textbf{ટ્રાન્સમીટર} & સંદેશને યોગ્ય સિગ્નલમાં રૂપાંતરિત કરે છે (મોડ્યુલેશન, કોડિંગ) \\
\textbf{ચેનલ} & માધ્યમ જેમાં સિગ્નલ પ્રવાસ કરે છે (તાર, ફાઇબર, હવા) \\
\textbf{નોઈઝ સ્ત્રોત} & અવાંછિત સિગ્નલ જે ટ્રાન્સમિટ કરેલા સિગ્નલને બગાડે છે \\
\textbf{રીસીવર} & પ્રાપ્ત સિગ્નલમાંથી મૂળ સંદેશ કાઢે છે (ડીમોડ્યુલેશન) \\
\textbf{ગંતવ્ય} & જ્યાં સંદેશ પહોંચાડવામાં આવે છે (માનવ, મશીન) \\
\end{longtable}
}

\end{solutionbox}
\begin{mnemonicbox}
``I Try Communicating Neatly, Receive Data''
(I-T-C-N-R-D)

\end{mnemonicbox}
\subsection*{Question 1(c) [7 marks]}\label{q1c}

\textbf{એમ્પ્લિટ્યુડ મોડ્યુલેશન માટેનાં વોલ્ટેજનુ સુત્ર તારવો.}

\begin{solutionbox}
એમ્પ્લિટ્યુડ મોડ્યુલેશન કેરિયર સિગ્નલની એમ્પ્લિટ્યુડને મેસેજ સિગ્નલના
પ્રમાણમાં બદલે છે.

\textbf{ગાણિતિક ડેરિવેશન:}

\begin{itemize}
\tightlist
\item
  ધારો કે કેરિયર સિગ્નલ: c(t) = Ac cos(ωct)
\item
  મેસેજ સિગ્નલ: m(t) = Am cos(ωmt)
\item
  AM સિગ્નલ: s(t) = Ac[1 + μ·m(t)/Am]cos(ωct)
\item
જ્યાં

μ = મોડ્યુલેશન ઇન્ડેક્સ = Am/Ac

\item
  m(t) ને સબ્સ્ટિટ્યુટ કરતા: s(t) = Ac[1 + μ·cos(ωmt)]cos(ωct)
\item
  વિસ્તારીને: s(t) = Ac·cos(ωct) + μ·Ac·cos(ωmt)·cos(ωct)
\item
  આઇડેન્ટિટી (cos A·cos B) વાપરીને: s(t) = Ac·cos(ωct) +
  (μ·Ac/2)[cos(ωc+ωm)t + cos(ωc-ωm)t]
\end{itemize}

\textbf{Diagram: ટાઈમ ડોમેનમાં AM સિગ્નલ}

\begin{figure}
\centering
\pandocbounded{\includesvg[keepaspectratio]{diagrams/1333201-s2024-q1c.svg}}
\caption{AM વેવફોર્મ}
\end{figure}

\end{solutionbox}
\begin{mnemonicbox}
``CAMDS'' - Carrier Amplitude Modulated by Data
Signal

\end{mnemonicbox}
\subsection*{Question 1(c) OR [7
marks]}\label{q1c}

\textbf{AM માં ટોટલ પાવરનુ સુત્ર તારવો તથા DSB અને SSBમાં થતા પાવર સેવિંગની
ગણતરી કરો.}

\begin{solutionbox}
મોડ્યુલેશન ઇન્ડેક્સ μ વાળા AM સિગ્નલ માટે, કુલ પાવર કેરિયર પાવર
અને સાઇડબેન્ડ પાવરનો સમાવેશ કરે છે.


{\def\LTcaptype{none} % do not increment counter
\vspace{-5pt}
\captionof{table}{AM માં પાવર ડિસ્ટ્રિબ્યુશન}
\vspace{-10pt}
\begin{longtable}[]{@{}lll@{}}
\toprule\noalign{}
ઘટક & પાવર ફોર્મ્યુલા & કુલ પાવરની ટકાવારી \\
\midrule\noalign{}
\endhead
\bottomrule\noalign{}
\endlastfoot
કેરિયર & Pc = Ac^{2}/2 & 1/(1+μ^{2}/2) \times 100\% \\
અપર સાઇડબેન્ડ & PUSB = Pc·μ^{2}/4 & (μ^{2}/4)/(1+μ^{2}/2) \times 100\% \\
લોઅર સાઇડબેન્ડ & PLSB = Pc·μ^{2}/4 & (μ^{2}/4)/(1+μ^{2}/2) \times 100\% \\
કુલ & PT = Pc(1+μ^{2}/2) & 100\% \\
\end{longtable}
}

\textbf{પાવર સેવિંગ્સ ગણતરી:}

\begin{itemize}
\tightlist
\item
  DSB-SC માં: 100\% કેરિયર દબાવવાથી = (Pc/PT)\times100\% = 1/(1+μ^{2}/2)\times100\%

  \begin{itemize}
  \tightlist
  \item
    μ = 1 માટે: સેવિંગ = 2/3\times100\% = 66.67\%
  \end{itemize}
\item
  SSB માં: એક સાઇડબેન્ડ + કેરિયર દબાવવાથી = (Pc+PLSB)/PT\times100\% =
  (1+μ^{2}/4)/(1+μ^{2}/2)\times100\%

  \begin{itemize}
  \tightlist
  \item
    μ = 1 માટે: સેવિંગ = 5/6\times100\% = 83.33\%
  \end{itemize}
\end{itemize}

\end{solutionbox}
\begin{mnemonicbox}
``CAPS'' - Carrier And Power in Sidebands

\end{mnemonicbox}
\subsection*{Question 2(a) [3 marks]}\label{q2a}

\textbf{રેડિયો રીસીવરમાં ઇમેજ ફ્રીક્વન્સીને વ્યાખ્યાયિત કરો અને તેને યોગ્ય ઉદાહરણ સાથે
સમજાવો.}

\begin{solutionbox}
ઇમેજ ફ્રીક્વન્સી એ અનચાહતી આવૃત્તિ છે જે સુપરહેટેરોડાઇન રિસીવરમાં
ઇચ્છિત સિગ્નલની જેમ જ IF (ઇન્ટરમીડિયેટ ફ્રીક્વન્સી) ઉત્પન્ન કરી શકે છે.


{\def\LTcaptype{none} % do not increment counter
\vspace{-5pt}
\captionof{table}{ઇમેજ ફ્રીક્વન્સી}
\vspace{-10pt}
\begin{longtable}[]{@{}lll@{}}
\toprule\noalign{}
પેરામીટર & ફોર્મ્યુલા & ઉદાહરણ \\
\midrule\noalign{}
\endhead
\bottomrule\noalign{}
\endlastfoot
\textbf{ઇચ્છિત સિગ્નલ} & fs & 100 MHz \\
\textbf{લોકલ ઓસિલેટર} & fLO & 110 MHz \\
\textbf{IF} & fIF = fLO - fs & 10 MHz \\
\textbf{ઇમેજ ફ્રીક્વન્સી} & fimage = fLO + fIF & 120 MHz \\
\end{longtable}
}

\begin{figure}
\centering
\pandocbounded{\includesvg[keepaspectratio]{diagrams/1333201-s2024-q2a.svg}}
\caption{ઇમેજ ફ્રીક્વન્સી}
\end{figure}

જો 100 MHz અને 120 MHz બંને સિગ્નલ મોજૂદ હોય, તો બંને 10 MHz IF ઉત્પન્ન કરશે,
જેનાથી દખલ થશે.

\end{solutionbox}
\begin{mnemonicbox}
``LIDS'' - Local oscillator plus/minus IF gives
Desired signal and Signal image

\end{mnemonicbox}
\subsection*{Question 2(b) [4 marks]}\label{q2b}

\textbf{એન્વેલપ ડિટેક્ટરનો બ્લોક ડાયાગ્રામ દોરો અને સમજાવો.}

\begin{solutionbox}
એન્વેલપ ડિટેક્ટર AM વેવમાંથી એન્વેલપને અનુસરીને મોડ્યુલેટિંગ સિગ્નલ કાઢે
છે.

\begin{figure}
\centering
\pandocbounded{\includesvg[keepaspectratio]{diagrams/1333201-s2024-q2b.svg}}
\caption{એન્વેલપ ડિટેક્ટર}
\end{figure}


{\def\LTcaptype{none} % do not increment counter
\vspace{-5pt}
\captionof{table}{એન્વેલપ ડિટેક્ટર ઘટકો}
\vspace{-10pt}
\begin{longtable}[]{@{}ll@{}}
\toprule\noalign{}
ઘટક & કાર્ય \\
\midrule\noalign{}
\endhead
\bottomrule\noalign{}
\endlastfoot
\textbf{ડાયોડ} & AM સિગ્નલને રેક્ટિફાય કરે છે (પોઝિટિવ હાફ પસાર કરે છે) \\
\textbf{કેપેસિટર} & રેક્ટિફાઇડ સિગ્નલની પીક વેલ્યુ સુધી ચાર્જ થાય છે \\
\textbf{રેસિસ્ટર} & RC ટાઇમ કોન્સ્ટન્ટ સાથે કેપેસિટરને ડિસ્ચાર્જ કરે છે \\
\textbf{RC વેલ્યુ} & 1/ωm \textless{} RC \textless{} 1/ωc (જ્યાં ωm મેસેજ
ફ્રીક્વન્સી છે, ωc કેરિયર છે) \\
\end{longtable}
}

\end{solutionbox}
\begin{mnemonicbox}
``DRCT'' - Diode Rectifies, Capacitor Tracks

\end{mnemonicbox}
\subsection*{Question 2(c) [7 marks]}\label{q2c}

\textbf{AM રેડીયો રિસિવરનો બ્લોક ડાયાગ્રામ દોરો અને દરેક બ્લોકનુ કાર્ય વિગતવાર
સમજાવો.}

\begin{solutionbox}
AM રિસીવર રેડિયો સિગ્નલને ઓડિયો આઉટપુટમાં રૂપાંતરિત કરે છે.

\begin{figure}
\centering
\pandocbounded{\includesvg[keepaspectratio]{diagrams/1333201-s2024-q2c.svg}}
\caption{AM રેડિયો રિસીવર}
\end{figure}


{\def\LTcaptype{none} % do not increment counter
\vspace{-5pt}
\captionof{table}{AM રિસીવરના બ્લોક્સ}
\vspace{-10pt}
\begin{longtable}[]{@{}
  >{\raggedright\arraybackslash}p{(\linewidth - 2\tabcolsep) * \real{0.5385}}
  >{\raggedright\arraybackslash}p{(\linewidth - 2\tabcolsep) * \real{0.4615}}@{}}
\toprule\noalign{}
\begin{minipage}[b]{\linewidth}\raggedright
બ્લોક
\end{minipage} & \begin{minipage}[b]{\linewidth}\raggedright
કાર્ય
\end{minipage} \\
\midrule\noalign{}
\endhead
\bottomrule\noalign{}
\endlastfoot
\textbf{એન્ટેના} & હવામાંથી ઇલેક્ટ્રોમેગ્નેટિક સિગ્નલ પકડે છે \\
\textbf{RF એમ્પ્લિફાયર} & નબળા RF સિગ્નલને એમ્પ્લિફાય કરે છે, સિલેક્ટિવિટી પ્રદાન
કરે છે \\
\textbf{લોકલ ઓસિલેટર} & ઇનકમિંગ સિગ્નલ સાથે મિક્સ કરવા માટે ફ્રીક્વન્સી ઉત્પન્ન કરે
છે \\
\textbf{મિક્સર} & RF અને ઓસિલેટર સિગ્નલને જોડીને IF ઉત્પન્ન કરે છે \\
\textbf{IF એમ્પ્લિફાયર} & ફિક્સ્ડ IF સિગ્નલને ઉચ્ચ ગેઇન સાથે એમ્પ્લિફાય કરે છે \\
\textbf{ડિટેક્ટર} & AM કેરિયરમાંથી ઓડિયો સિગ્નલ કાઢે છે \\
\textbf{AF એમ્પ્લિફાયર} & સ્પીકર ચલાવવા માટે ઓડિયો સિગ્નલ પાવર વધારે છે \\
\textbf{સ્પીકર} & ઇલેક્ટ્રિકલ સિગ્નલને અવાજમાં રૂપાંતરિત કરે છે \\
\end{longtable}
}

\end{solutionbox}
\begin{mnemonicbox}
``ARMLIDAS'' - Antenna Receives, Mixer Links Input
and Detector, Audio to Speaker

\end{mnemonicbox}
\subsection*{Question 2(a) OR [3
marks]}\label{q2a}

\textbf{રેડીયો રીસિવર ની કોઈ પણ ચાર લાક્ષણીકતાઓ વ્યાખ્યાયીત કરો.}

\begin{solutionbox}


{\def\LTcaptype{none} % do not increment counter
\vspace{-5pt}
\captionof{table}{રેડિયો રિસીવરની લાક્ષણિકતાઓ}
\vspace{-10pt}
\begin{longtable}[]{@{}ll@{}}
\toprule\noalign{}
લાક્ષણિકતા & વ્યાખ્યા \\
\midrule\noalign{}
\endhead
\bottomrule\noalign{}
\endlastfoot
\textbf{સેન્સિટિવિટી} & માનક આઉટપુટ ઉત્પન્ન કરતી ન્યૂનતમ સિગ્નલ સ્ટ્રેન્થ \\
\textbf{સિલેક્ટિવિટી} & ઇચ્છિત સિગ્નલને અડજાસન્ટ ચેનલોથી અલગ કરવાની ક્ષમતા \\
\textbf{ફિડેલિટી} & મૂળ મોડ્યુલેટિંગ સિગ્નલને ચોકસાઈથી પુનઃઉત્પાદિત કરવાની
ક્ષમતા \\
\textbf{ઇમેજ રિજેક્શન} & ઇમેજ ફ્રીક્વન્સી સિગ્નલને નકારવાની ક્ષમતા \\
\textbf{સિગ્નલ-ટુ-નોઇઝ રેશિયો} & ઇચ્છિત સિગ્નલ પાવરનો નોઇઝ પાવર સાથેનો
ગુણોત્તર \\
\end{longtable}
}

\end{solutionbox}
\begin{mnemonicbox}
``SSFIS'' - Super Sensitive Fidelity with Image
Suppression

\end{mnemonicbox}
\subsection*{Question 2(b) OR [4
marks]}\label{q2b}

\textbf{FM ડીટેક્શન માટેની રેશિયો ડીટેક્ટર સર્કિટ સમજાવો.}

\begin{solutionbox}
રેશિયો ડિટેક્ટર FM સિગ્નલમાંથી એમ્પ્લિટ્યુડ વેરિએશન્સને અવગણીને
ઓડિયો કાઢે છે.

\begin{figure}
\centering
\pandocbounded{\includesvg[keepaspectratio]{diagrams/1333201-s2024-q2bor.svg}}
\caption{રેશિયો ડિટેક્ટર}
\end{figure}


{\def\LTcaptype{none} % do not increment counter
\vspace{-5pt}
\captionof{table}{રેશિયો ડિટેક્ટર ઘટકો}
\vspace{-10pt}
\begin{longtable}[]{@{}ll@{}}
\toprule\noalign{}
ઘટક & કાર્ય \\
\midrule\noalign{}
\endhead
\bottomrule\noalign{}
\endlastfoot
\textbf{ટ્રાન્સફોર્મર} & ફ્રીક્વન્સી ડેવિએશનના પ્રમાણમાં ફેઝ શિફ્ટ ઉત્પન્ન કરે છે \\
\textbf{ડાયોડ્સ} & વોલ્ટેજ રેશિયો ઉત્પન્ન કરવા માટે વિરુદ્ધ ધ્રુવતા સાથે ગોઠવાયેલા
છે \\
\textbf{સ્ટેબિલાઇઝિંગ કેપેસિટર} & AM વેરિએશન્સને દબાવવા માટે મોટી વેલ્યુ (10μF) \\
\textbf{RC નેટવર્ક} & વોલ્ટેજના રેશિયોમાંથી ઓડિયો સિગ્નલ કાઢે છે \\
\end{longtable}
}

\end{solutionbox}
\begin{mnemonicbox}
``RADS'' - Ratio detector Avoids Disturbance from
Strength variations

\end{mnemonicbox}
\subsection*{Question 2(c) OR [7
marks]}\label{q2c}

\textbf{સુપર હેટરોડાઈન રીસિવર નો બ્લોક ડાયાગ્રામ દોરો અને વિગતવાર સમજુતિ આપો.}

\begin{solutionbox}
સુપરહેટરોડાઇન રિસીવર બધા ઇનકમિંગ RF સિગ્નલને બેટર એમ્પ્લિફિકેશન
માટે ફિક્સ્ડ IF માં રૂપાંતરિત કરે છે.


{\def\LTcaptype{none} % do not increment counter
\vspace{-5pt}
\captionof{table}{સુપરહેટરોડાઇન રિસીવર ઘટકો}
\vspace{-10pt}
\begin{longtable}[]{@{}
  >{\raggedright\arraybackslash}p{(\linewidth - 2\tabcolsep) * \real{0.5385}}
  >{\raggedright\arraybackslash}p{(\linewidth - 2\tabcolsep) * \real{0.4615}}@{}}
\toprule\noalign{}
\begin{minipage}[b]{\linewidth}\raggedright
બ્લોક
\end{minipage} & \begin{minipage}[b]{\linewidth}\raggedright
કાર્ય
\end{minipage} \\
\midrule\noalign{}
\endhead
\bottomrule\noalign{}
\endlastfoot
\textbf{એન્ટેના} & RF સિગ્નલ પકડે છે \\
\textbf{RF એમ્પ્લિફાયર} & ઇચ્છિત ફ્રીક્વન્સી બેન્ડને એમ્પ્લિફાય અને પસંદ કરે છે \\
\textbf{લોકલ ઓસિલેટર} & IF વેલ્યુ દ્વારા સિગ્નલની ઉપર/નીચે ફ્રીક્વન્સી ઉત્પન્ન કરે
છે \\
\textbf{મિક્સર} & IF ઉત્પન્ન કરવા માટે સિગ્નલ અને ઓસિલેટરને હેટરોડાઇન કરે છે \\
\textbf{IF એમ્પ્લિફાયર} & ફિક્સ્ડ ફ્રીક્વન્સી પર મોટાભાગનો ગેઇન અને સિલેક્ટિવિટી
પ્રદાન કરે છે \\
\textbf{ડિટેક્ટર} & મૂળ મોડ્યુલેટિંગ સિગ્નલ પુનઃપ્રાપ્ત કરે છે \\
\textbf{AGC} & ઓટોમેટિક ગેઇન કંટ્રોલ - સ્થિર આઉટપુટ લેવલ જાળવે છે \\
\textbf{AF એમ્પ્લિફાયર} & સ્પીકર ચલાવવા માટે ઓડિયો એમ્પ્લિફાય કરે છે \\
\textbf{સ્પીકર} & ઇલેક્ટ્રિકલ સિગ્નલને અવાજમાં રૂપાંતરિત કરે છે \\
\end{longtable}
}

\end{solutionbox}
\begin{mnemonicbox}
``ARMLIADS'' - Antenna Receives, Mixer Links,
Intermediate Amplifies, Detector Separates

\end{mnemonicbox}
\subsection*{Question 3(a) [3 marks]}\label{q3a}

\textbf{નિચે આપેલા સિગ્નલનુ ટાઈમ અને ફ્રીક્વંસી ડોમેઈનમાં દોરો ૧.એનાલોગ સિગ્નલ
(સાઈન) ૨.ડિજિટલ સિગ્નલ (સ્ક્વેર)}

\begin{solutionbox}


{\def\LTcaptype{none} % do not increment counter
\vspace{-5pt}
\captionof{table}{સિગ્નલ રેપ્રેઝન્ટેશન}
\vspace{-10pt}
\begin{longtable}[]{@{}lll@{}}
\toprule\noalign{}
સિગ્નલ ટાઇપ & ટાઇમ ડોમેઇન & ફ્રીક્વન્સી ડોમેઇન \\
\midrule\noalign{}
\endhead
\bottomrule\noalign{}
\endlastfoot
\textbf{સાઇન વેવ} & સાઇન્યુસોઇડલ કર્વ & ફ્રીક્વન્સી f પર સિંગલ સ્પાઇક \\
\textbf{સ્ક્વેર વેવ} & અલ્ટરનેટિંગ લેવલ્સ & ફંડામેન્ટલ અને ઓડ હાર્મોનિક્સ (1/n પેટર્ન) \\
\end{longtable}
}

\textbf{Diagram: સિગ્નલ રેપ્રેઝન્ટેશન}

\begin{figure}
\centering
\pandocbounded{\includesvg[keepaspectratio]{diagrams/1333201-s2024-q3a.svg}}
\caption{સિગ્નલ રેપ્રેઝન્ટેશન}
\end{figure}

\end{solutionbox}
\begin{mnemonicbox}
``SOFT'' - Sine has One Frequency, square has
Timeless harmonics

\end{mnemonicbox}
\subsection*{Question 3(b) [4 marks]}\label{q3b}

\textbf{સેમ્પલિંગ થિયોરમ સમજાવો.}

\begin{solutionbox}
સેમ્પલિંગ થિયરમ સેમ્પલમાંથી અચૂક સિગ્નલ પુનઃનિર્માણ માટેની શરતો
જણાવે છે.


{\def\LTcaptype{none} % do not increment counter
\vspace{-5pt}
\captionof{table}{સેમ્પલિંગ થિયોરમ}
\vspace{-10pt}
\begin{longtable}[]{@{}
  >{\raggedright\arraybackslash}p{(\linewidth - 2\tabcolsep) * \real{0.5000}}
  >{\raggedright\arraybackslash}p{(\linewidth - 2\tabcolsep) * \real{0.5000}}@{}}
\toprule\noalign{}
\begin{minipage}[b]{\linewidth}\raggedright
પાસું
\end{minipage} & \begin{minipage}[b]{\linewidth}\raggedright
વર્ણન
\end{minipage} \\
\midrule\noalign{}
\endhead
\bottomrule\noalign{}
\endlastfoot
\textbf{સ્ટેટમેન્ટ} & સિગ્નલને સંપૂર્ણપણે પુનઃનિર્માણ કરવા માટે, સેમ્પલિંગ ફ્રીક્વન્સી
સિગ્નલમાં સૌથી ઉંચી ફ્રીક્વન્સીની ઓછામાં ઓછી બે ગણી હોવી જોઈએ \\
\textbf{નાઇક્વિસ્ટ રેટ} & fs \geq 2fmax (ન્યૂનતમ સેમ્પલિંગ ફ્રીક્વન્સી) \\
\textbf{અલાયસિંગ} & વિકૃતિ જે નાઇક્વિસ્ટ રેટથી નીચે સેમ્પલિંગ કરવાથી થાય છે \\
\textbf{ઉદાહરણ} & અવાજ (300-3400 Hz) માટે, fs \geq 6.8 kHz (સામાન્ય રીતે 8
kHz) \\
\end{longtable}
}

\textbf{Diagram: અલાયસિંગ ઇફેક્ટ}

\begin{figure}
\centering
\pandocbounded{\includesvg[keepaspectratio]{diagrams/1333201-s2024-q3b.svg}}
\caption{અલાયસિંગ ઇફેક્ટ}
\end{figure}

\end{solutionbox}
\begin{mnemonicbox}
``SNAP'' - Sample at Nyquist And Prevent aliasing

\end{mnemonicbox}
\subsection*{Question 3(c) [7 marks]}\label{q3c}

\textbf{PAM, PPM અને PWM સમજાવો.}

\begin{solutionbox}
આ પલ્સ મોડ્યુલેશન ટેકનિક્સ છે જ્યાં પલ્સના પેરામિટરને બદલવામાં આવે છે.


{\def\LTcaptype{none} % do not increment counter
\vspace{-5pt}
\captionof{table}{પલ્સ મોડ્યુલેશન પ્રકારો}
\vspace{-10pt}
\begin{longtable}[]{@{}
  >{\raggedright\arraybackslash}p{(\linewidth - 6\tabcolsep) * \real{0.1458}}
  >{\raggedright\arraybackslash}p{(\linewidth - 6\tabcolsep) * \real{0.2292}}
  >{\raggedright\arraybackslash}p{(\linewidth - 6\tabcolsep) * \real{0.3750}}
  >{\raggedright\arraybackslash}p{(\linewidth - 6\tabcolsep) * \real{0.2500}}@{}}
\toprule\noalign{}
\begin{minipage}[b]{\linewidth}\raggedright
પ્રકાર
\end{minipage} & \begin{minipage}[b]{\linewidth}\raggedright
ફુલ ફોર્મ
\end{minipage} & \begin{minipage}[b]{\linewidth}\raggedright
બદલાયેલ પેરામિટર
\end{minipage} & \begin{minipage}[b]{\linewidth}\raggedright
લાક્ષણિકતાઓ
\end{minipage} \\
\midrule\noalign{}
\endhead
\bottomrule\noalign{}
\endlastfoot
\textbf{PAM} & પલ્સ એમ્પ્લિટ્યુડ મોડ્યુલેશન & એમ્પ્લિટ્યુડ & એનાલોગ સિગ્નલનું સીધું
સેમ્પલિંગ \\
\textbf{PPM} & પલ્સ પોઝિશન મોડ્યુલેશન & પોઝિશન/ટાઇમ & PAM કરતાં બેટર નોઇઝ
ઇમ્યુનિટી \\
\textbf{PWM} & પલ્સ વિડ્થ મોડ્યુલેશન & વિડ્થ/અવધિ & શ્રેષ્ઠ નોઇઝ ઇમ્યુનિટી, કંટ્રોલ
સિસ્ટમ્સમાં વ્યાપકપણે વપરાય છે \\
\end{longtable}
}

\textbf{Diagram: પલ્સ મોડ્યુલેશન ટેકનિક્સ}

\begin{figure}
\centering
\pandocbounded{\includesvg[keepaspectratio]{diagrams/1333201-s2024-q3c.svg}}
\caption{પલ્સ મોડ્યુલેશન}
\end{figure}

\end{solutionbox}
\begin{mnemonicbox}
``AAA-PPW'' - Amplitude, Position, Width are
modulated in PAM, PPM, PWM

\end{mnemonicbox}
\subsection*{Question 3(a) OR [3
marks]}\label{q3a}

\textbf{નાઈક્વિસ્ટ રેટની વ્યાખ્યા આપી સમજાવો.}

\begin{solutionbox}
નાઇક્વિસ્ટ રેટ એ અચૂક સિગ્નલ પુનઃનિર્માણ માટે જરૂરી ન્યૂનતમ સેમ્પલિંગ
ફ્રીક્વન્સી છે.


{\def\LTcaptype{none} % do not increment counter
\vspace{-5pt}
\captionof{table}{નાઇક્વિસ્ટ રેટ}
\vspace{-10pt}
\begin{longtable}[]{@{}
  >{\raggedright\arraybackslash}p{(\linewidth - 2\tabcolsep) * \real{0.5000}}
  >{\raggedright\arraybackslash}p{(\linewidth - 2\tabcolsep) * \real{0.5000}}@{}}
\toprule\noalign{}
\begin{minipage}[b]{\linewidth}\raggedright
પાસું
\end{minipage} & \begin{minipage}[b]{\linewidth}\raggedright
વર્ણન
\end{minipage} \\
\midrule\noalign{}
\endhead
\bottomrule\noalign{}
\endlastfoot
\textbf{વ્યાખ્યા} & અલાયસિંગ ટાળવા માટે જરૂરી ન્યૂનતમ સેમ્પલિંગ ફ્રીક્વન્સી (fs =
2fmax) \\
\textbf{અસરો} & નાઇક્વિસ્ટ રેટથી નીચે સેમ્પલિંગ કરવાથી અપરિવર્તનીય વિકૃતિ થાય
છે \\
\textbf{ફોર્મ્યુલા} & fs \geq 2fmax જ્યાં fmax સિગ્નલમાં સૌથી ઉંચી ફ્રીક્વન્સી છે \\
\textbf{એપ્લિકેશન} & CD ઓડિયો: 20 kHz ઓડિયો માટે 44.1 kHz સેમ્પલિંગ \\
\end{longtable}
}

\end{solutionbox}
\begin{mnemonicbox}
``TANS'' - Twice As Needed for Sampling

\end{mnemonicbox}
\subsection*{Question 3(b) OR [4
marks]}\label{q3b}

\textbf{ક્વોન્ટાઈઝેશન પ્રોસેસ વિગતવાર સમજાવો.}

\begin{solutionbox}
ક્વોન્ટાઇઝેશન એનાલોગ-ટુ-ડિજિટલ કન્વર્ઝનમાં સેમ્પલ કરેલા મૂલ્યોને
ડિસ્ક્રીટ એમ્પ્લિટ્યુડ લેવલ્સ આપે છે.


{\def\LTcaptype{none} % do not increment counter
\vspace{-5pt}
\captionof{table}{ક્વોન્ટાઇઝેશન પ્રોસેસ}
\vspace{-10pt}
\begin{longtable}[]{@{}
  >{\raggedright\arraybackslash}p{(\linewidth - 2\tabcolsep) * \real{0.5000}}
  >{\raggedright\arraybackslash}p{(\linewidth - 2\tabcolsep) * \real{0.5000}}@{}}
\toprule\noalign{}
\begin{minipage}[b]{\linewidth}\raggedright
સ્ટેપ
\end{minipage} & \begin{minipage}[b]{\linewidth}\raggedright
વર્ણન
\end{minipage} \\
\midrule\noalign{}
\endhead
\bottomrule\noalign{}
\endlastfoot
\textbf{સેમ્પલિંગ} & કન્ટિન્યુઅસ સિગ્નલમાંથી ડિસ્ક્રીટ-ટાઇમ સેમ્પલ લેવાય છે \\
\textbf{લેવલ એસાઇનમેન્ટ} & દરેક સેમ્પલને નજીકના ક્વોન્ટાઇઝેશન લેવલમાં એસાઇન કરવામાં
આવે છે \\
\textbf{ક્વોન્ટાઇઝેશન એરર} & વાસ્તવિક અને ક્વોન્ટાઇઝ કરેલા મૂલ્ય વચ્ચેનો તફાવત \\
\textbf{ક્વોન્ટાઇઝેશન નોઇઝ} & સિગ્નલમાં ત્રુટિઓની આંકડાકીય અસર \\
\textbf{રિઝોલ્યુશન} & બિટ્સની સંખ્યા દ્વારા નક્કી થાય છે (n બિટ્સ માટે 2^{n} લેવલ્સ) \\
\end{longtable}
}

\textbf{Diagram: ક્વોન્ટાઇઝેશન પ્રોસેસ}

\begin{figure}
\centering
\pandocbounded{\includesvg[keepaspectratio]{diagrams/1333201-s2024-q3bor.svg}}
\caption{ક્વોન્ટાઇઝેશન પ્રોસેસ}
\end{figure}

\end{solutionbox}
\begin{mnemonicbox}
``SLERN'' - Sample, Level assign, Error occurs,
Resolution determines Noise

\end{mnemonicbox}
\subsection*{Question 3(c) OR [7
marks]}\label{q3c}

\textbf{આઈડિયલ, નેચરલ અને ફ્લેટ ટોપ સેમ્પલિંગ સમજાવો.}

\begin{solutionbox}
આ સેમ્પલિંગ પ્રક્રિયાના વિવિધ વ્યવહારિક અમલીકરણો છે.


{\def\LTcaptype{none} % do not increment counter
\vspace{-5pt}
\captionof{table}{સેમ્પલિંગ પ્રકારોની તુલના}
\vspace{-10pt}
\begin{longtable}[]{@{}
  >{\raggedright\arraybackslash}p{(\linewidth - 6\tabcolsep) * \real{0.1538}}
  >{\raggedright\arraybackslash}p{(\linewidth - 6\tabcolsep) * \real{0.1538}}
  >{\raggedright\arraybackslash}p{(\linewidth - 6\tabcolsep) * \real{0.3333}}
  >{\raggedright\arraybackslash}p{(\linewidth - 6\tabcolsep) * \real{0.3590}}@{}}
\toprule\noalign{}
\begin{minipage}[b]{\linewidth}\raggedright
પ્રકાર
\end{minipage} & \begin{minipage}[b]{\linewidth}\raggedright
વર્ણન
\end{minipage} & \begin{minipage}[b]{\linewidth}\raggedright
લાક્ષણિકતાઓ
\end{minipage} & \begin{minipage}[b]{\linewidth}\raggedright
ગાણિતિક રજૂઆત
\end{minipage} \\
\midrule\noalign{}
\endhead
\bottomrule\noalign{}
\endlastfoot
\textbf{આઇડિયલ} & શૂન્ય વિડ્થ પર તત્કાલિક સેમ્પલ્સ & સૈદ્ધાંતિક કન્સેપ્ટ, ભૌતિક રીતે
વાસ્તવિક નથી & s(t) = m(t) \times \sumδ(t-nTs) \\
\textbf{નેચરલ} & સેમ્પલ્સ પલ્સ ટ્રેનને મોડ્યુલેટ કરે છે & એનાલોગ સ્વિચનો ઉપયોગ કરીને
વ્યવહારિક અમલીકરણ & s(t) = m(t) \times p(t) \\
\textbf{ફ્લેટ-ટોપ} & આગલા સેમ્પલ સુધી સેમ્પલનું મૂલ્ય જાળવે છે & અમલીકરણ માટે સૌથી
સરળ, સેમ્પલ-એન્ડ-હોલ્ડ સર્કિટ & s(t) = \summ(nTs)[u(t-nTs)-u(t-(n+1)Ts)] \\
\end{longtable}
}

\textbf{Diagram: સેમ્પલિંગ પ્રકારો}

\begin{figure}
\centering
\pandocbounded{\includesvg[keepaspectratio]{diagrams/1333201-s2024-q3cor.svg}}
\caption{સેમ્પલિંગ પ્રકારો}
\end{figure}

\end{solutionbox}
\begin{mnemonicbox}
``INF'' - Ideal is theoretical, Natural is
practical, Flat-top holds values

\end{mnemonicbox}
\subsection*{Question 4(a) [3 marks]}\label{q4a}

\textbf{PCMનાં ફાયદાઓ અને ગેરફાયદફાઓ લખો.}

\begin{solutionbox}


{\def\LTcaptype{none} % do not increment counter
\vspace{-5pt}
\captionof{table}{PCM ફાયદા અને ગેરફાયદા}
\vspace{-10pt}
\begin{longtable}[]{@{}ll@{}}
\toprule\noalign{}
ફાયદા & ગેરફાયદા \\
\midrule\noalign{}
\endhead
\bottomrule\noalign{}
\endlastfoot
\textbf{ઉચ્ચ નોઇઝ ઇમ્યુનિટી} & વધારે બેન્ડવિડ્થની જરૂર પડે છે \\
\textbf{બેટર સિગ્નલ ક્વોલિટી} & જટિલ સર્કિટરી \\
\textbf{ડિજિટલ સિસ્ટમ્સ સાથે સુસંગત} & ક્વોન્ટાઇઝેશન નોઇઝ \\
\textbf{સુરક્ષિત કોમ્યુનિકેશન શક્ય} & ઉચ્ચ પાવર વપરાશ \\
\textbf{ડિગ્રેડેશન વિના રીજનરેટ થઈ શકે છે} & સિન્ક્રોનાઇઝેશનની જરૂર પડે છે \\
\end{longtable}
}

\textbf{Diagram: PCM ફાયદા અને ગેરફાયદા}

\begin{figure}
\centering
\pandocbounded{\includesvg[keepaspectratio]{diagrams/1333201-s2024-q4a.svg}}
\caption{PCM ફાયદા અને ગેરફાયદા}
\end{figure}

\end{solutionbox}
\begin{mnemonicbox}
``NICHE'' vs ``BCQPS'' - Noise immunity,
Integration, Complex circuitry, Higher bandwidth, Error correction vs
Bandwidth, Cost, Quantization, Power, Synchronization

\end{mnemonicbox}
\subsection*{Question 4(b) [4 marks]}\label{q4b}

\textbf{ડેલ્ટા મોડ્યુલેશનનો બ્લોક ડાયાગ્રામ દોરો અને સમજાવો.}

\begin{solutionbox}
ડેલ્ટા મોડ્યુલેશન 1-બિટ ક્વોન્ટાઇઝેશનનો ઉપયોગ કરીને માત્ર સિગ્નલ
લેવલમાં ફેરફારને ટ્રાન્સમિટ કરે છે.

\begin{figure}
\centering
\pandocbounded{\includesvg[keepaspectratio]{diagrams/1333201-s2024-q4b.svg}}
\caption{ડેલ્ટા મોડ્યુલેશન}
\end{figure}


{\def\LTcaptype{none} % do not increment counter
\vspace{-5pt}
\captionof{table}{ડેલ્ટા મોડ્યુલેશન ઘટકો}
\vspace{-10pt}
\begin{longtable}[]{@{}
  >{\raggedright\arraybackslash}p{(\linewidth - 2\tabcolsep) * \real{0.5385}}
  >{\raggedright\arraybackslash}p{(\linewidth - 2\tabcolsep) * \real{0.4615}}@{}}
\toprule\noalign{}
\begin{minipage}[b]{\linewidth}\raggedright
બ્લોક
\end{minipage} & \begin{minipage}[b]{\linewidth}\raggedright
કાર્ય
\end{minipage} \\
\midrule\noalign{}
\endhead
\bottomrule\noalign{}
\endlastfoot
\textbf{કમ્પેરેટર} & ઇનપુટને પ્રેડિક્ટેડ વેલ્યુ સાથે સરખાવે છે \\
\textbf{1-બિટ ક્વોન્ટાઇઝર} & જો તફાવત પોઝિટિવ હોય તો 1, નેગેટિવ હોય તો 0
આઉટપુટ કરે છે \\
\textbf{ઇન્ટિગ્રેટર} & ઇનપુટને ટ્રેક કરવા માટે સ્ટેપ વેલ્યુઓને એકત્રિત કરે છે \\
\textbf{ડિલે} & તુલના માટે અગાઉનો આઉટપુટ પ્રદાન કરે છે \\
\end{longtable}
}

\end{solutionbox}
\begin{mnemonicbox}
``CQID'' - Compare, Quantize with 1-bit, Integrate,
Delay

\end{mnemonicbox}
\subsection*{Question 4(c) [7 marks]}\label{q4c}

\textbf{PCM, DM અને DPCM ને સરખાવો.}

\begin{solutionbox}


{\def\LTcaptype{none} % do not increment counter
\vspace{-5pt}
\captionof{table}{ડિજિટલ મોડ્યુલેશન ટેકનિક્સની તુલના}
\vspace{-10pt}
\begin{longtable}[]{@{}
  >{\raggedright\arraybackslash}p{(\linewidth - 6\tabcolsep) * \real{0.4348}}
  >{\raggedright\arraybackslash}p{(\linewidth - 6\tabcolsep) * \real{0.2174}}
  >{\raggedright\arraybackslash}p{(\linewidth - 6\tabcolsep) * \real{0.1739}}
  >{\raggedright\arraybackslash}p{(\linewidth - 6\tabcolsep) * \real{0.1739}}@{}}
\toprule\noalign{}
\begin{minipage}[b]{\linewidth}\raggedright
પેરામિટર
\end{minipage} & \begin{minipage}[b]{\linewidth}\raggedright
PCM
\end{minipage} & \begin{minipage}[b]{\linewidth}\raggedright
DM
\end{minipage} & \begin{minipage}[b]{\linewidth}\raggedright
DPCM
\end{minipage} \\
\midrule\noalign{}
\endhead
\bottomrule\noalign{}
\endlastfoot
\textbf{સેમ્પલ દીઠ બિટ્સ} & 8-16 બિટ્સ & 1 બિટ & 4-6 બિટ્સ \\
\textbf{બેન્ડવિડ્થ} & સૌથી વધુ & સૌથી ઓછી & મધ્યમ \\
\textbf{સિગ્નલ-ટુ-નોઇઝ રેશિયો} & સૌથી વધુ & સૌથી ઓછો & મધ્યમ \\
\textbf{સર્કિટ જટિલતા} & ઉચ્ચ & સરળ & મધ્યમ \\
\textbf{સેમ્પલિંગ રેટ} & નાઇક્વિસ્ટ & નાઇક્વિસ્ટનો ગુણક & નાઇક્વિસ્ટ \\
\textbf{એરર ટાઇપ્સ} & ક્વોન્ટાઇઝેશન એરર & સ્લોપ ઓવરલોડ, ગ્રેન્યુલર નોઇઝ & પ્રેડિક્શન
એરર \\
\textbf{એપ્લિકેશન્સ} & CD ઓડિયો, ડિજિટલ ટેલિફોની & ઓછી-ક્વોલિટી વૉઇસ & સ્પીચ,
વિડિયો કોડિંગ \\
\end{longtable}
}

\end{solutionbox}
\begin{mnemonicbox}
``PCM-DM-DPCM: More Bits Better Quality, More
Complexity Needed''

\end{mnemonicbox}
\subsection*{Question 4(a) OR [3
marks]}\label{q4a}

\textbf{DPCM સમજાવો.}

\begin{solutionbox}
ડિફરેન્શિયલ પલ્સ કોડ મોડ્યુલેશન વાસ્તવિક અને પ્રિડિક્ટેડ સેમ્પલ
વચ્ચેના તફાવતને એન્કોડ કરે છે.


{\def\LTcaptype{none} % do not increment counter
\vspace{-5pt}
\captionof{table}{DPCM લાક્ષણિકતાઓ}
\vspace{-10pt}
\begin{longtable}[]{@{}ll@{}}
\toprule\noalign{}
પાસું & વર્ણન \\
\midrule\noalign{}
\endhead
\bottomrule\noalign{}
\endlastfoot
\textbf{મૂળભૂત સિદ્ધાંત} & વાસ્તવિક અને પ્રિડિક્ટેડ મૂલ્ય વચ્ચેના તફાવતને એન્કોડ કરે
છે \\
\textbf{પ્રિડિક્ટર} & વર્તમાન મૂલ્યની આગાહી કરવા માટે અગાઉના સેમ્પલ્સનો ઉપયોગ કરે
છે \\
\textbf{ફાયદો} & PCM કરતાં ઓછા બિટ્સની જરૂર પડે છે (કોરિલેશનનો ઉપયોગ કરે છે) \\
\textbf{બિટ રેટ ઘટાડો} & PCM ની તુલનામાં સામાન્ય રીતે 25-50\% \\
\textbf{એપ્લિકેશન્સ} & સ્પીચ કોડિંગ, ઇમેજ કમ્પ્રેશન \\
\end{longtable}
}

\begin{figure}
\centering
\pandocbounded{\includesvg[keepaspectratio]{diagrams/1333201-s2024-q4aor.svg}}
\caption{DPCM સિસ્ટમ}
\end{figure}

\end{solutionbox}
\begin{mnemonicbox}
``DPCM: Difference Predicted, Correlation Matters''

\end{mnemonicbox}
\subsection*{Question 4(b) OR [4
marks]}\label{q4b}

\textbf{ડેલ્ટા મોડ્યુલેશનનાં ફાયદાઓ અને ગેરફાયદાઓ લખો.}

\begin{solutionbox}


{\def\LTcaptype{none} % do not increment counter
\vspace{-5pt}
\captionof{table}{ડેલ્ટા મોડ્યુલેશન - ફાયદા અને ગેરફાયદા}
\vspace{-10pt}
\begin{longtable}[]{@{}ll@{}}
\toprule\noalign{}
ફાયદા & ગેરફાયદા \\
\midrule\noalign{}
\endhead
\bottomrule\noalign{}
\endlastfoot
\textbf{સરળ અમલીકરણ} & સ્લોપ ઓવરલોડ ડિસ્ટોર્શન \\
\textbf{નીચો બિટ રેટ} & ઓછી એમ્પ્લિટ્યુડ પર ગ્રેન્યુલર નોઇઝ \\
\textbf{સિંગલ બિટ ટ્રાન્સમિશન} & મર્યાદિત ડાયનેમિક રેન્જ \\
\textbf{ચેનલ એરર સામે મજબૂત} & ઉચ્ચ સેમ્પલિંગ રેટની જરૂર પડે છે \\
\textbf{ઓછી જટિલતા વાળું હાર્ડવેર} & PCM કરતાં નીચો SNR \\
\end{longtable}
}

\end{solutionbox}
\begin{mnemonicbox}
``SLSRL'' vs ``SGLSH'' - Simple, Low bit-rate,
Single bit, Robust, Low cost vs Slope overload, Granular noise, Limited
range, Sampling high, SNR low

\end{mnemonicbox}
\subsection*{Question 4(c) OR [7
marks]}\label{q4c}

\textbf{બેઝિક PCM-TDM સિસ્ટમનો બ્લોક ડાયાગ્રામ સમજાવો.}

\begin{solutionbox}
PCM-TDM મલ્ટિપલ ડિજિટાઇઝ્ડ સિગ્નલ્સને એક સિંગલ હાઇ-સ્પીડ ચેનલમાં
જોડે છે.

\begin{figure}
\centering
\pandocbounded{\includesvg[keepaspectratio]{diagrams/1333201-s2024-q4c.svg}}
\caption{PCM-TDM સિસ્ટમ}
\end{figure}


{\def\LTcaptype{none} % do not increment counter
\vspace{-5pt}
\captionof{table}{PCM-TDM સિસ્ટમ ઘટકો}
\vspace{-10pt}
\begin{longtable}[]{@{}
  >{\raggedright\arraybackslash}p{(\linewidth - 2\tabcolsep) * \real{0.5385}}
  >{\raggedright\arraybackslash}p{(\linewidth - 2\tabcolsep) * \real{0.4615}}@{}}
\toprule\noalign{}
\begin{minipage}[b]{\linewidth}\raggedright
બ્લોક
\end{minipage} & \begin{minipage}[b]{\linewidth}\raggedright
કાર્ય
\end{minipage} \\
\midrule\noalign{}
\endhead
\bottomrule\noalign{}
\endlastfoot
\textbf{PCM એન્કોડર} & એનાલોગ સિગ્નલને ડિજિટલમાં રૂપાંતરિત કરે છે (સેમ્પલિંગ,
ક્વોન્ટાઇઝેશન, કોડિંગ) \\
\textbf{TDM મલ્ટિપ્લેક્સર} & મલ્ટિપલ PCM સ્ટ્રીમ્સને સિંગલ હાઇ-સ્પીડ સ્ટ્રીમમાં જોડે
છે \\
\textbf{ટ્રાન્સમિશન ચેનલ} & સિગ્નલ ટ્રાન્સમિશન માટેનું માધ્યમ \\
\textbf{TDM ડીમલ્ટિપ્લેક્સર} & ટાઇમ-મલ્ટિપ્લેક્સ્ડ સ્ટ્રીમને પાછા વ્યક્તિગત ચેનલ્સમાં અલગ
કરે છે \\
\textbf{PCM ડિકોડર} & ડિજિટલને પાછું એનાલોગમાં રૂપાંતરિત કરે છે (ડિકોડિંગ,
ફિલ્ટરિંગ) \\
\textbf{સિન્ક્રોનાઇઝેશન} & ક્લોક અને ફ્રેમ સિન્ક સિગ્નલ્સ યોગ્ય ડીમલ્ટિપ્લેક્સિંગ
સુનિશ્ચિત કરે છે \\
\textbf{ફ્રેમ સ્ટ્રક્ચર} & બધા ચેનલ્સના સેમ્પલ્સ અને સિન્ક બિટ્સ ધરાવે છે \\
\end{longtable}
}

\end{solutionbox}
\begin{mnemonicbox}
``PETDSF'' - PCM Encodes, TDM combines, Digital
transmits, Separation occurs, Frames synchronize

\end{mnemonicbox}
\subsection*{Question 5(a) [3 marks]}\label{q5a}

\textbf{અડેપ્ટિવ ડેલ્ટા મોડ્યુલેશન સમજાવો.}

\begin{solutionbox}
અડેપ્ટિવ ડેલ્ટા મોડ્યુલેશન સિગ્નલની લાક્ષણિકતાઓના આધારે સ્ટેપ સાઇઝને
એડજસ્ટ કરે છે.


{\def\LTcaptype{none} % do not increment counter
\vspace{-5pt}
\captionof{table}{અડેપ્ટિવ ડેલ્ટા મોડ્યુલેશન}
\vspace{-10pt}
\begin{longtable}[]{@{}
  >{\raggedright\arraybackslash}p{(\linewidth - 2\tabcolsep) * \real{0.5385}}
  >{\raggedright\arraybackslash}p{(\linewidth - 2\tabcolsep) * \real{0.4615}}@{}}
\toprule\noalign{}
\begin{minipage}[b]{\linewidth}\raggedright
ફીચર
\end{minipage} & \begin{minipage}[b]{\linewidth}\raggedright
વર્ણન
\end{minipage} \\
\midrule\noalign{}
\endhead
\bottomrule\noalign{}
\endlastfoot
\textbf{મૂળભૂત સિદ્ધાંત} & સિગ્નલના સ્લોપ અનુસાર સ્ટેપ સાઇઝ બદલે છે \\
\textbf{સ્ટેપ સાઇઝ કંટ્રોલ} & જ્યારે સમાન બિટ પેટર્ન રિપીટ થાય (સિગ્નલ ઝડપથી
બદલાઈ રહ્યો હોય) ત્યારે વધારો કરે છે \\
\textbf{ફાયદા} & ઘટાડેલ સ્લોપ ઓવરલોડ અને ગ્રેન્યુલર નોઇઝ \\
\textbf{અમલીકરણ} & બિટ પેટર્ન શોધવા માટે શિફ્ટ રજિસ્ટરનો ઉપયોગ કરે છે \\
\textbf{પરફોર્મન્સ} & સ્ટાન્ડર્ડ DM કરતાં બેટર SNR \\
\end{longtable}
}

\textbf{Diagram: સ્ટેપ સાઇઝ એડેપ્ટેશન}

\begin{figure}
\centering
\pandocbounded{\includesvg[keepaspectratio]{diagrams/1333201-s2024-q5a.svg}}
\caption{અડેપ્ટિવ ડેલ્ટા મોડ્યુલેશન}
\end{figure}

\end{solutionbox}
\begin{mnemonicbox}
``ASSG'' - Adaptive Step Size Gives better
performance

\end{mnemonicbox}
\subsection*{Question 5(b) [4 marks]}\label{q5b}

\textbf{ટર્મ વ્યાખ્યાયિત કરો ૧.રેડિએશન પેટર્ન ૨.એન્ટેના ગેઈન}

\begin{solutionbox}


{\def\LTcaptype{none} % do not increment counter
\vspace{-5pt}
\captionof{table}{એન્ટેના ટર્મ્સ}
\vspace{-10pt}
\begin{longtable}[]{@{}
  >{\raggedright\arraybackslash}p{(\linewidth - 4\tabcolsep) * \real{0.2222}}
  >{\raggedright\arraybackslash}p{(\linewidth - 4\tabcolsep) * \real{0.3333}}
  >{\raggedright\arraybackslash}p{(\linewidth - 4\tabcolsep) * \real{0.4444}}@{}}
\toprule\noalign{}
\begin{minipage}[b]{\linewidth}\raggedright
ટર્મ
\end{minipage} & \begin{minipage}[b]{\linewidth}\raggedright
વ્યાખ્યા
\end{minipage} & \begin{minipage}[b]{\linewidth}\raggedright
લાક્ષણિકતાઓ
\end{minipage} \\
\midrule\noalign{}
\endhead
\bottomrule\noalign{}
\endlastfoot
\textbf{રેડિએશન પેટર્ન} & સ્પેસમાં એન્ટેનાના રેડિએશન પ્રોપર્ટીઝની ગ્રાફિકલ રજૂઆત &
રેડિએટેડ પાવરની દિશાત્મક નિર્ભરતા દર્શાવે છે \\
\textbf{એન્ટેના ગેઇન} & ચોક્કસ દિશામાં રેડિયો એનર્જીને નિર્દેશિત કરવા અથવા કેન્દ્રિત
કરવાની એન્ટેનાની ક્ષમતાનું માપ & dB માં વ્યક્ત, આઇસોટ્રોપિક રેડિએટરની (dBi)
સરખામણી \\
\end{longtable}
}

\textbf{Diagram: રેડિએશન પેટર્ન ટાઇપ્સ}

\begin{figure}
\centering
\pandocbounded{\includesvg[keepaspectratio]{diagrams/1333201-s2024-q5b.svg}}
\caption{રેડિએશન પેટર્ન}
\end{figure}

\end{solutionbox}
\begin{mnemonicbox}
``RPGD'' - Radiation Pattern shows Gain Direction

\end{mnemonicbox}
\subsection*{Question 5(c) [7 marks]}\label{q5c}

\textbf{બેઝ સ્ટેશન અને મોબાઈલ સ્ટેશન એન્ટેના સમજાવો.}

\begin{solutionbox}
વાયરલેસ કોમ્યુનિકેશન સિસ્ટમ્સમાં વિવિધ એન્ટેના ડિઝાઇન વિવિધ હેતુઓ
માટે સેવા આપે છે.


{\def\LTcaptype{none} % do not increment counter
\vspace{-5pt}
\captionof{table}{બેઝ સ્ટેશન અને મોબાઇલ સ્ટેશન એન્ટેનાની તુલના}
\vspace{-10pt}
\begin{longtable}[]{@{}lll@{}}
\toprule\noalign{}
પેરામિટર & બેઝ સ્ટેશન એન્ટેના & મોબાઇલ સ્ટેશન એન્ટેના \\
\midrule\noalign{}
\endhead
\bottomrule\noalign{}
\endlastfoot
\textbf{ઊંચાઈ} & 15-50 મીટર & 2 મીટરથી ઓછી \\
\textbf{ગેઇન} & ઉચ્ચ (10-20 dBi) & નીચો (0-3 dBi) \\
\textbf{પેટર્ન} & સેક્ટોરલ (120^\circ સેક્ટર્સ) & ઓમ્નિડાયરેક્શનલ \\
\textbf{સાઇઝ} & મોટા એરે & કોમ્પેક્ટ, ઇન્ટિગ્રેટેડ \\
\textbf{પ્રકારો} & પેનલ, યાગી, કોલિનિયર & મોનોપોલ, PIFA, ચિપ \\
\textbf{પોલરાઇઝેશન} & વર્ટિકલ, ક્રોસ-પોલરાઇઝ્ડ & સામાન્ય રીતે વર્ટિકલ \\
\textbf{બીમફોર્મિંગ} & વારંવાર વપરાય છે & મૂળભૂત ડિવાઇસમાં ભાગ્યે જ \\
\textbf{ડાયવર્સિટી} & સ્પેસ/પોલરાઇઝેશન ડાયવર્સિટી & ભાગ્યે જ અમલીકરણ \\
\end{longtable}
}

\textbf{Diagram: એન્ટેના ટાઇપ્સ}

\begin{figure}
\centering
\pandocbounded{\includesvg[keepaspectratio]{diagrams/1333201-s2024-q5c.svg}}
\caption{એન્ટેના ટાઇપ્સ}
\end{figure}

\end{solutionbox}
\begin{mnemonicbox}
``BHPSTBD'' - Base stations Have Power, Size, Tower
mounting, Beamforming, Diversity

\end{mnemonicbox}
\subsection*{Question 5(a) OR [3
marks]}\label{q5a}

\textbf{HF, VHF and UHF માટેની ફ્રીક્વન્સી રેન્જ લખો.}

\begin{solutionbox}


{\def\LTcaptype{none} % do not increment counter
\vspace{-5pt}
\captionof{table}{ફ્રીક્વન્સી બેન્ડ્સ}
\vspace{-10pt}
\begin{longtable}[]{@{}llll@{}}
\toprule\noalign{}
બેન્ડ & ફ્રીક્વન્સી રેન્જ & વેવલેન્થ & નોંધપાત્ર એપ્લિકેશન્સ \\
\midrule\noalign{}
\endhead
\bottomrule\noalign{}
\endlastfoot
\textbf{HF} & 3-30 MHz & 100-10 m & શોર્ટવેવ રેડિયો, એમેચ્યોર રેડિયો,
એવિએશન \\
\textbf{VHF} & 30-300 MHz & 10-1 m & FM રેડિયો, TV ચેનલ્સ 2-13, એર
ટ્રાફિક \\
\textbf{UHF} & 300-3000 MHz & 1-0.1 m & TV ચેનલ્સ 14-83, મોબાઇલ ફોન્સ,
Wi-Fi \\
\end{longtable}
}

\textbf{Diagram: ફ્રીક્વન્સી બેન્ડ્સ}

\begin{figure}
\centering
\pandocbounded{\includesvg[keepaspectratio]{diagrams/1333201-s2024-q5aor.svg}}
\caption{ફ્રીક્વન્સી બેન્ડ્સ}
\end{figure}

\end{solutionbox}
\begin{mnemonicbox}
``3-30-300-3000'' - દરેક બેન્ડ 10 MHz ની પાવરના 3 ગણાથી
શરૂ થાય છે

\end{mnemonicbox}
\subsection*{Question 5(b) OR [4
marks]}\label{q5b}

\textbf{ટર્મ વ્યાખ્યાયિત કરો ૧.એન્ટેના ડાઈરેક્ટીવીટી ૨.પોલરાઈઝેશન.}

\begin{solutionbox}


{\def\LTcaptype{none} % do not increment counter
\vspace{-5pt}
\captionof{table}{એન્ટેના પ્રોપર્ટીઝ}
\vspace{-10pt}
\begin{longtable}[]{@{}
  >{\raggedright\arraybackslash}p{(\linewidth - 4\tabcolsep) * \real{0.2222}}
  >{\raggedright\arraybackslash}p{(\linewidth - 4\tabcolsep) * \real{0.3333}}
  >{\raggedright\arraybackslash}p{(\linewidth - 4\tabcolsep) * \real{0.4444}}@{}}
\toprule\noalign{}
\begin{minipage}[b]{\linewidth}\raggedright
ટર્મ
\end{minipage} & \begin{minipage}[b]{\linewidth}\raggedright
વ્યાખ્યા
\end{minipage} & \begin{minipage}[b]{\linewidth}\raggedright
લાક્ષણિકતાઓ
\end{minipage} \\
\midrule\noalign{}
\endhead
\bottomrule\noalign{}
\endlastfoot
\textbf{ડાયરેક્ટિવિટી} & આપેલી દિશામાં રેડિઆશન ઇન્ટેન્સિટીનો સરેરાશ રેડિઆશન
ઇન્ટેન્સિટી સાથેનો ગુણોત્તર & dBi માં માપવામાં આવે છે, એન્ટેનાના ફોકસને દર્શાવે છે \\
\textbf{પોલરાઇઝેશન} & રેડિએટેડ વેવના ઇલેક્ટ્રિક ફિલ્ડ વેક્ટરનું ઓરિએન્ટેશન & લિનિયર
(વર્ટિકલ/હોરિઝોન્ટલ), સર્ક્યુલર, ઇલિપ્ટિકલ \\
\end{longtable}
}

\textbf{Diagram: એન્ટેના ડાયરેક્ટિવિટી અને પોલરાઇઝેશન}

\begin{figure}
\centering
\pandocbounded{\includesvg[keepaspectratio]{diagrams/1333201-s2024-q5bor.svg}}
\caption{એન્ટેના ડાયરેક્ટિવિટી અને પોલરાઇઝેશન}
\end{figure}

\end{solutionbox}
\begin{mnemonicbox}
``DIVE POLE'' - DIrectivity shows Vector Excellence,
POLarization shows Electric field

\end{mnemonicbox}
\subsection*{Question 5(c) OR [7
marks]}\label{q5c}

\textbf{ગ્રાઉન્ડ વેવ અને સ્કાય વેવ પ્રોપોગેશન વિગતવાર સમજાવો.}

\begin{solutionbox}
આ નીચલા વાતાવરણમાં રેડિયો વેવ પ્રોપોગેશનના બે પ્રાથમિક મોડ છે.


{\def\LTcaptype{none} % do not increment counter
\vspace{-5pt}
\captionof{table}{વેવ પ્રોપોગેશન તુલના}
\vspace{-10pt}
\begin{longtable}[]{@{}
  >{\raggedright\arraybackslash}p{(\linewidth - 4\tabcolsep) * \real{0.3226}}
  >{\raggedright\arraybackslash}p{(\linewidth - 4\tabcolsep) * \real{0.3548}}
  >{\raggedright\arraybackslash}p{(\linewidth - 4\tabcolsep) * \real{0.3226}}@{}}
\toprule\noalign{}
\begin{minipage}[b]{\linewidth}\raggedright
પેરામિટર
\end{minipage} & \begin{minipage}[b]{\linewidth}\raggedright
ગ્રાઉન્ડ વેવ
\end{minipage} & \begin{minipage}[b]{\linewidth}\raggedright
સ્પેસ વેવ
\end{minipage} \\
\midrule\noalign{}
\endhead
\bottomrule\noalign{}
\endlastfoot
\textbf{ફ્રીક્વન્સી રેન્જ} & 2 MHz થી નીચે & 30 MHz થી ઉપર \\
\textbf{ડિસ્ટન્સ કવરેજ} & 100-300 km & લાઇન-ઓફ-સાઇટ + ડિફ્રેક્શન સુધી
મર્યાદિત \\
\textbf{પાથ} & પૃથ્વીના વક્રતાને અનુસરે છે & ડાયરેક્ટ અને ગ્રાઉન્ડ-રિફ્લેક્ટેડ પાથ \\
\textbf{મેકેનિઝમ} & પૃથ્વીની સપાટીની આસપાસ ડિફ્રેક્શન & લાઇન-ઓફ-સાઇટ પ્રોપોગેશન
વિથ રિફ્લેક્શન \\
\textbf{એટેન્યુએશન} & ઉચ્ચ (ફ્રીક્વન્સી સાથે વધે છે) & VHF/UHF રેન્જમાં ઓછું \\
\textbf{પોલરાઇઝેશન} & વર્ટિકલ પોલરાઇઝેશન પસંદગીયુક્ત & વર્ટિકલ અને હોરિઝોન્ટલ બંને
વાપરી શકાય \\
\textbf{એપ્લિકેશન્સ} & AM બ્રોડકાસ્ટિંગ, નેવિગેશન બીકન્સ & TV, FM રેડિયો, માઇક્રોવેવ
લિંક્સ \\
\textbf{અસર કરતા પરિબળો} & ગ્રાઉન્ડ કન્ડક્ટિવિટી, ટેરેન & એન્ટેના ઊંચાઈ, ટેરેન,
અવરોધો \\
\end{longtable}
}

\textbf{Diagram: ગ્રાઉન્ડ વેવ vs સ્પેસ વેવ પ્રોપોગેશન}

\begin{figure}
\centering
\pandocbounded{\includesvg[keepaspectratio]{diagrams/1333201-s2024-q5cor.svg}}
\caption{વેવ પ્રોપોગેશન}
\end{figure}

\textbf{ગ્રાઉન્ડ વેવ પ્રોપોગેશન:}

\begin{itemize}
\tightlist
\item
  પૃથ્વીની સપાટી સાથે પ્રવાસ કરે છે
\item
  અંતર સાથે સિગ્નલ સ્ટ્રેન્થ ઘટે છે
\item
  જમીન કરતાં સમુદ્ર પર બેટર પ્રોપોગેશન
\item
  ગ્રાઉન્ડ કન્ડક્ટિવિટી અને ડાયલેક્ટ્રિક કોન્સ્ટન્ટથી અસર થાય છે
\item
  AM બ્રોડકાસ્ટિંગ, મેરિટાઇમ કોમ્યુનિકેશન માટે ઉપયોગ થાય છે
\end{itemize}

\textbf{સ્પેસ વેવ પ્રોપોગેશન:}

\begin{itemize}
\tightlist
\item
  ડાયરેક્ટ વેવ અને ગ્રાઉન્ડ-રિફ્લેક્ટેડ વેવનો સમાવેશ કરે છે
\item
  એટ્મોસ્ફેરિક રિફ્રેક્શન દ્વારા રેન્જ વિસ્તારિત થાય છે
\item
  રેન્જ ફોર્મ્યુલા: d = \sqrt(2Rh) જ્યાં R પૃથ્વીની ત્રિજ્યા છે, h એન્ટેનાની ઊંચાઈ છે
\item
  અવરોધો ઉપર ડિફ્રેક્શનથી અસર થાય છે
\item
  લાઇન-ઓફ-સાઇટ કોમ્યુનિકેશન જેમ કે TV, FM, માઇક્રોવેવ લિંક્સ માટે ઉપયોગ થાય છે
\end{itemize}

\end{solutionbox}
\begin{mnemonicbox}
``GAFFS'' - Ground Adheres to earth, Follows
surface, Frequencies low, Short wavelengths

\end{mnemonicbox}

\end{document}
