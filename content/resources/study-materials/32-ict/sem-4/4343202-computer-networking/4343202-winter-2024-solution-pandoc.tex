\documentclass[10pt,a4paper]{article}

% content/resources/templates/preamble.tex
\usepackage[margin=0.6in]{geometry}
\author{Milav Dabgar}
\usepackage{amsmath,amssymb,amsthm}
\usepackage{booktabs}
\usepackage{multirow}
\usepackage{xcolor}
\usepackage{tcolorbox}
\tcbuselibrary{breakable,skins}
\usepackage[colorlinks=true,linkcolor=blue]{hyperref}
\usepackage{titlesec}
\usepackage{enumitem}
\usepackage{tikz}
\usepackage{pgfplots}
\usepackage{circuitikz}
\usepackage[version=4]{mhchem}
\usepackage{longtable}
\usepackage{array}
\usepackage{float}
\usepackage{caption}
\usepackage{listings}

\lstset{
  basicstyle=\small\ttfamily,
  breaklines=true,
  breakatwhitespace=false,
  postbreak=\mbox{\textcolor{red}{$\hookrightarrow$}\space},
  float=false,
  numbers=left,
  numberstyle=\tiny\color{gray},
  numbersep=10pt,
  xleftmargin=2em,
  keywordstyle=\color{blue},
  commentstyle=\color{green!60!black},
  stringstyle=\color{purple},
  backgroundcolor=\color{gray!5},
  showstringspaces=false,
  tabsize=2,
  captionpos=b,
  keepspaces=true,
  columns=flexible
}

\pgfplotsset{compat=1.18}
\usetikzlibrary{shapes,arrows,positioning,calc,patterns,decorations.pathmorphing,decorations.markings,arrows.meta}

% Color scheme
\definecolor{headcolor}{RGB}{0,102,204}
\definecolor{keycolor}{RGB}{220,20,60}
\definecolor{solutioncolor}{RGB}{34,139,34}
\definecolor{mnemoniccolor}{RGB}{148,0,211}
\definecolor{codecolor}{RGB}{0,0,100}

% Spacing
\setlength{\parskip}{3pt}
\setlist[itemize]{nosep}
\setlist[enumerate]{nosep}

% Title formatting
\titleformat{\section}{\Large\bfseries\color{headcolor}}{\thesection}{1em}{}
\titleformat{\subsection}{\large\bfseries\color{headcolor}}{\thesubsection}{1em}{}

% Pandoc tightlist compatibility
\providecommand{\tightlist}{%
  \setlength{\itemsep}{0pt}\setlength{\parskip}{0pt}}

% Pandoc longtable compatibility
\newcounter{none}
\def\thenone{}


% content/resources/templates/english-boxes.tex
% This file is currently empty - it exists to maintain consistency with the import structure.
% Add custom environments here if needed in the future.


\begin{document}

\begin{center}
{\Huge\bfseries\color{headcolor} Subject Name Solutions}\\[5pt]
{\LARGE 4343202 -- Winter 2024}\\[3pt]
{\large Semester 1 Study Material}\\[3pt]
{\normalsize\textit{Detailed Solutions and Explanations}}
\end{center}

\vspace{10pt}

\subsection*{Question 1(a) [3 marks]}\label{q1a}

\textbf{What is the Computer Network? Why it is important?}

\begin{solutionbox}
A computer network is a collection of interconnected
computing devices that can exchange data and share resources.

\textbf{Diagram:}

\begin{lstlisting}
     +--------+           +--------+
     |Computer|-----------|Computer|
     +--------+    |      +--------+
                   |
     +--------+    |      +--------+
     |Computer|----+------|Computer|
     +--------+           +--------+
\end{lstlisting}

\begin{itemize}
\tightlist
\item
  \textbf{Resource sharing}: Enables sharing of printers, files,
  applications
\item
  \textbf{Communication}: Facilitates information exchange between users
\item
  \textbf{Scalability}: Allows networks to grow as needs increase
\end{itemize}

\end{solutionbox}
\begin{mnemonicbox}
``CSI'' - ``Connect, Share, Interact''

\end{mnemonicbox}
\subsection*{Question 1(b) [4 marks]}\label{q1b}

\textbf{Define terms: 1) Web Server, 2)Encrypted data, 3)Hacking,
4)Client-server}

\begin{solutionbox}

{\def\LTcaptype{none} % do not increment counter
\begin{longtable}[]{@{}
  >{\raggedright\arraybackslash}p{(\linewidth - 2\tabcolsep) * \real{0.3333}}
  >{\raggedright\arraybackslash}p{(\linewidth - 2\tabcolsep) * \real{0.6667}}@{}}
\toprule\noalign{}
\begin{minipage}[b]{\linewidth}\raggedright
Term
\end{minipage} & \begin{minipage}[b]{\linewidth}\raggedright
Definition
\end{minipage} \\
\midrule\noalign{}
\endhead
\bottomrule\noalign{}
\endlastfoot
Web Server & Software/hardware that serves web content to clients using
HTTP/HTTPS \\
Encrypted Data & Information converted to code to prevent unauthorized
access \\
Hacking & Unauthorized access to computer systems through security
vulnerabilities \\
Client-Server & Network model where centralized servers provide services
to client computers \\
\end{longtable}
}

\textbf{Diagram:}

\begin{lstlisting}
CLIENT-SERVER MODEL:
  +--------+   REQUEST    +--------+
  | CLIENT |------------->| SERVER |
  |        |<-------------|        |
  +--------+   RESPONSE   +--------+
\end{lstlisting}

\end{solutionbox}
\begin{mnemonicbox}
``WECHS'' - ``Web servers Encrypt data, Clients and
Hackers use Servers''

\end{mnemonicbox}
\subsection*{Question 1(c) [7 marks]}\label{q1c}

\textbf{Classify and explain the transmission media in detail.}

\begin{solutionbox}
Transmission media are physical pathways that carry
data in a network.

{\def\LTcaptype{none} % do not increment counter
\begin{longtable}[]{@{}
  >{\raggedright\arraybackslash}p{(\linewidth - 6\tabcolsep) * \real{0.2083}}
  >{\raggedright\arraybackslash}p{(\linewidth - 6\tabcolsep) * \real{0.1458}}
  >{\raggedright\arraybackslash}p{(\linewidth - 6\tabcolsep) * \real{0.3542}}
  >{\raggedright\arraybackslash}p{(\linewidth - 6\tabcolsep) * \real{0.2917}}@{}}
\toprule\noalign{}
\begin{minipage}[b]{\linewidth}\raggedright
Category
\end{minipage} & \begin{minipage}[b]{\linewidth}\raggedright
Types
\end{minipage} & \begin{minipage}[b]{\linewidth}\raggedright
Characteristics
\end{minipage} & \begin{minipage}[b]{\linewidth}\raggedright
Applications
\end{minipage} \\
\midrule\noalign{}
\endhead
\bottomrule\noalign{}
\endlastfoot
\textbf{Guided Media} & & & \\
Twisted Pair & UTP, STP & 100m range, 10Mbps-10Gbps & Office LANs \\
Coaxial Cable & Baseband, Broadband & 500m range, 10-100Mbps & Cable TV,
Internet \\
Fiber Optic & Single-mode, Multi-mode & Long distance, 100Mbps-100Gbps &
Backbone, WAN \\
\textbf{Unguided Media} & & & \\
Radio Waves & WiFi, Cellular & Omnidirectional, 1-100Mbps & Wireless
networks \\
Microwaves & Terrestrial, Satellite & Line-of-sight, 1-10Gbps &
Point-to-point links \\
Infrared & IrDA & Short-range, 4-16Mbps & Remote controls \\
\end{longtable}
}

\textbf{Diagram:}

\begin{lstlisting}
GUIDED MEDIA:
  Twisted Pair: =~=~=~=~=~=~=~
  Coaxial:      =====|=====|=====
  Fiber Optic:  ======================>

UNGUIDED MEDIA:
  Radio:        ((( o )))
  Microwave:    <---> <--->
  Infrared:     * * * >
\end{lstlisting}

\begin{itemize}
\tightlist
\item
  \textbf{Guided media}: Physical paths for signal confinement
\item
  \textbf{Unguided media}: Wireless transmission through air/space
\item
  \textbf{Selection factors}: Cost, bandwidth, distance, environment
\end{itemize}

\end{solutionbox}
\begin{mnemonicbox}
``TCFRIM'' - ``Twisted pair, Coaxial, Fiber, Radio,
Infrared, Microwave''

\end{mnemonicbox}
\subsection*{Question 1(c) OR [7
marks]}\label{q1c}

\textbf{Explain WAN and MAN type of network.}

\begin{solutionbox}
Wide Area Networks (WAN) and Metropolitan Area Networks
(MAN) are network types classified by geographic scope.

{\def\LTcaptype{none} % do not increment counter
\begin{longtable}[]{@{}
  >{\raggedright\arraybackslash}p{(\linewidth - 4\tabcolsep) * \real{0.1324}}
  >{\raggedright\arraybackslash}p{(\linewidth - 4\tabcolsep) * \real{0.5000}}
  >{\raggedright\arraybackslash}p{(\linewidth - 4\tabcolsep) * \real{0.3676}}@{}}
\toprule\noalign{}
\begin{minipage}[b]{\linewidth}\raggedright
Feature
\end{minipage} & \begin{minipage}[b]{\linewidth}\raggedright
MAN (Metropolitan Area Network)
\end{minipage} & \begin{minipage}[b]{\linewidth}\raggedright
WAN (Wide Area Network)
\end{minipage} \\
\midrule\noalign{}
\endhead
\bottomrule\noalign{}
\endlastfoot
Coverage & City-wide (5-50 km) & Country/Global (\textgreater50 km) \\
Speed & 10 Mbps - 10 Gbps & 1.5 Mbps - 1 Gbps \\
Ownership & Municipal/Telecom & Multiple organizations \\
Technologies & Ethernet, SONET, WiMAX & Frame Relay, ATM, MPLS \\
Examples & City networks, Campus networks & Internet, Corporate
networks \\
\end{longtable}
}

\textbf{Diagram:}

\begin{lstlisting}
     WAN                       MAN
  +--------+               +--------+
  | Global |               |  City  |
  +--------+               +--------+
      |                        |
      v                        v
  +---------+           +-----------+
  | Multiple |           | Connected |
  | Countries|           |Campuses/  |
  +---------+           |City areas  |
                        +-----------+
\end{lstlisting}

\begin{itemize}
\tightlist
\item
  \textbf{MAN}: Connects LANs within a city/metropolitan area
\item
  \textbf{WAN}: Spans large geographical areas across cities/countries
\item
  \textbf{Management}: WAN typically requires service providers
\item
  \textbf{Infrastructure}: Different transmission media and technologies
\end{itemize}

\end{solutionbox}
\begin{mnemonicbox}
``SWIM'' - ``Size: WAN Is Massive compared to MAN''

\end{mnemonicbox}
\subsection*{Question 2(a) [3 marks]}\label{q2a}

\textbf{Explain in detail: Transmission technology.}

\begin{solutionbox}
Transmission technology refers to methods used to
transfer data between network devices.

{\def\LTcaptype{none} % do not increment counter
\begin{longtable}[]{@{}
  >{\raggedright\arraybackslash}p{(\linewidth - 4\tabcolsep) * \real{0.4211}}
  >{\raggedright\arraybackslash}p{(\linewidth - 4\tabcolsep) * \real{0.3421}}
  >{\raggedright\arraybackslash}p{(\linewidth - 4\tabcolsep) * \real{0.2368}}@{}}
\toprule\noalign{}
\begin{minipage}[b]{\linewidth}\raggedright
Technology Type
\end{minipage} & \begin{minipage}[b]{\linewidth}\raggedright
Description
\end{minipage} & \begin{minipage}[b]{\linewidth}\raggedright
Example
\end{minipage} \\
\midrule\noalign{}
\endhead
\bottomrule\noalign{}
\endlastfoot
Point-to-Point & Direct connection between two nodes & Leased line \\
Broadcast & Single communication channel shared by all nodes & Wireless
LAN \\
Multipoint & Multiple devices share single link & Cable networks \\
\end{longtable}
}

\begin{itemize}
\tightlist
\item
  \textbf{Analog transmission}: Continuous signal, susceptible to noise
\item
  \textbf{Digital transmission}: Discrete signal, more reliable
\item
  \textbf{Baseband}: Single signal uses entire bandwidth (Ethernet)
\item
  \textbf{Broadband}: Multiple signals share bandwidth (Cable TV)
\end{itemize}

\end{solutionbox}
\begin{mnemonicbox}
``ABP-DMB'' - ``Analog or Baseband, Point-to-point;
Digital or Multipoint, Broadcast''

\end{mnemonicbox}
\subsection*{Question 2(b) [4 marks]}\label{q2b}

\textbf{Draw and explain Star topology in detail.}

\begin{solutionbox}
Star topology is a network configuration where all
devices connect to a central hub/switch.

\textbf{Diagram:}

\begin{lstlisting}
              STAR TOPOLOGY
                 +-----+
                 | HUB/|
                 |SWITCH|
                 +-----+
                    |
         +----------+-----------+
         |          |           |
      +-----+    +-----+     +-----+
      |Node1|    |Node2|     |Node3|
      +-----+    +-----+     +-----+
         |          |           |
      +-----+    +-----+     +-----+
      |Node4|    |Node5|     |Node6|
      +-----+    +-----+     +-----+
\end{lstlisting}

{\def\LTcaptype{none} % do not increment counter
\begin{longtable}[]{@{}ll@{}}
\toprule\noalign{}
Advantages & Disadvantages \\
\midrule\noalign{}
\endhead
\bottomrule\noalign{}
\endlastfoot
Easy installation & Single point of failure (hub/switch) \\
Simple troubleshooting & Requires more cable than bus topology \\
Scalable & Higher cost due to central device \\
Better performance & Hub/switch limits determine network size \\
\end{longtable}
}

\begin{itemize}
\tightlist
\item
  \textbf{Operation}: All data passes through central device
\item
  \textbf{Installation}: Easier to manage and expand
\item
  \textbf{Fault isolation}: Node failure doesn't affect others
\end{itemize}

\end{solutionbox}
\begin{mnemonicbox}
``CASE'' - ``Centralized, All connected, Simple
expansion, Easy troubleshooting''

\end{mnemonicbox}
\subsection*{Question 2(c) [7 marks]}\label{q2c}

\textbf{Draw and explain TCP/IP model.}

\begin{solutionbox}
TCP/IP model is a conceptual framework used for network
communications, consisting of four layers.

\textbf{Diagram:}

\begin{lstlisting}
+-----------------------------+
|       APPLICATION LAYER     |
| (HTTP, FTP, SMTP, DNS, etc.)|
+-----------------------------+
|        TRANSPORT LAYER      |
|          (TCP, UDP)         |
+-----------------------------+
|        INTERNET LAYER       |
|      (IP, ICMP, ARP)        |
+-----------------------------+
|    NETWORK ACCESS LAYER     |
| (Ethernet, Wi-Fi, PPP, etc.)|
+-----------------------------+
        PHYSICAL MEDIA
\end{lstlisting}

{\def\LTcaptype{none} % do not increment counter
\begin{longtable}[]{@{}
  >{\raggedright\arraybackslash}p{(\linewidth - 4\tabcolsep) * \real{0.2059}}
  >{\raggedright\arraybackslash}p{(\linewidth - 4\tabcolsep) * \real{0.4706}}
  >{\raggedright\arraybackslash}p{(\linewidth - 4\tabcolsep) * \real{0.3235}}@{}}
\toprule\noalign{}
\begin{minipage}[b]{\linewidth}\raggedright
Layer
\end{minipage} & \begin{minipage}[b]{\linewidth}\raggedright
Main Functions
\end{minipage} & \begin{minipage}[b]{\linewidth}\raggedright
Protocols
\end{minipage} \\
\midrule\noalign{}
\endhead
\bottomrule\noalign{}
\endlastfoot
Application & User interfaces, data formatting & HTTP, FTP, SMTP, DNS \\
Transport & End-to-end communication, reliability & TCP, UDP \\
Internet & Logical addressing, routing & IP, ICMP, ARP, IGMP \\
Network Access & Physical addressing, media access & Ethernet, WiFi,
PPP \\
\end{longtable}
}

\begin{itemize}
\tightlist
\item
  \textbf{Application Layer}: Interface between applications and network
\item
  \textbf{Transport Layer}: Reliable data transfer between end systems
\item
  \textbf{Internet Layer}: Routing packets across networks
\item
  \textbf{Network Access Layer}: Physical connection to network media
\end{itemize}

\end{solutionbox}
\begin{mnemonicbox}
``ATNI'' - ``Application Talks, Network Internet
Interfaces''

\end{mnemonicbox}
\subsection*{Question 2(a) OR [3
marks]}\label{q2a}

\textbf{Draw and explain Bus topology in detail}

\begin{solutionbox}
Bus topology is a network configuration where all
devices connect to a single communication line.

\textbf{Diagram:}

\begin{lstlisting}
        BUS TOPOLOGY
+-----+    +-----+    +-----+    +-----+
|Node1|====|Node2|====|Node3|====|Node4|
+-----+    +-----+    +-----+    +-----+
                |
             +-----+
             |Node5|
             +-----+
\end{lstlisting}

{\def\LTcaptype{none} % do not increment counter
\begin{longtable}[]{@{}ll@{}}
\toprule\noalign{}
Advantages & Disadvantages \\
\midrule\noalign{}
\endhead
\bottomrule\noalign{}
\endlastfoot
Simple layout & Single point of failure (main cable) \\
Less cabling & Limited cable length \\
Low cost & Performance degrades with more nodes \\
Easy to extend & Difficult to troubleshoot \\
\end{longtable}
}

\begin{itemize}
\tightlist
\item
  \textbf{Operation}: Data travels along the bus in both directions
\item
  \textbf{Terminator}: Required at both ends to prevent signal
  reflection
\item
  \textbf{Usage}: Primarily in older networks, small setups
\end{itemize}

\end{solutionbox}
\begin{mnemonicbox}
``SLUE'' - ``Simple Layout, Uses less cable, Easy
installation''

\end{mnemonicbox}
\subsection*{Question 2(b) OR [4
marks]}\label{q2b}

\textbf{Explain Network Classification based on its architecture.}

\begin{solutionbox}
Networks can be classified based on their architectural
models that define how devices interact.

{\def\LTcaptype{none} % do not increment counter
\begin{longtable}[]{@{}
  >{\raggedright\arraybackslash}p{(\linewidth - 4\tabcolsep) * \real{0.3500}}
  >{\raggedright\arraybackslash}p{(\linewidth - 4\tabcolsep) * \real{0.4250}}
  >{\raggedright\arraybackslash}p{(\linewidth - 4\tabcolsep) * \real{0.2250}}@{}}
\toprule\noalign{}
\begin{minipage}[b]{\linewidth}\raggedright
Architecture
\end{minipage} & \begin{minipage}[b]{\linewidth}\raggedright
Characteristics
\end{minipage} & \begin{minipage}[b]{\linewidth}\raggedright
Example
\end{minipage} \\
\midrule\noalign{}
\endhead
\bottomrule\noalign{}
\endlastfoot
Peer-to-Peer & Equal privileges, no dedicated servers & Home networks,
small workgroups \\
Client-Server & Centralized services, dedicated servers & Enterprise
networks, web services \\
Three-Tier & Presentation, application, and data tiers & Modern web
applications \\
N-Tier & Multiple specialized tiers & Large distributed systems \\
\end{longtable}
}

\textbf{Diagram:}

\begin{lstlisting}
PEER-TO-PEER:               CLIENT-SERVER:
  +----+     +----+           +------+
  |Node|-----|Node|           |Client|
  +----+\   /+----+           +------+
         \ /                      |
          X                   +------+
         / \                  |Server|
  +----+/   \+----+           +------+
  |Node|-----|Node|
  +----+     +----+
\end{lstlisting}

\begin{itemize}
\tightlist
\item
  \textbf{Peer-to-Peer}: Direct device communication, distributed
  resources
\item
  \textbf{Client-Server}: Centralized resource management, better
  security
\item
  \textbf{Hybrid}: Combines elements of both architectures
\end{itemize}

\end{solutionbox}
\begin{mnemonicbox}
``PCAN'' - ``Peer-to-peer, Client-server,
Architecture Networks''

\end{mnemonicbox}
\subsection*{Question 2(c) OR [7
marks]}\label{q2c}

\textbf{Explain classification of IP address.}

\begin{solutionbox}
IP addresses are classified into different categories
based on their structure and purpose.

{\def\LTcaptype{none} % do not increment counter
\begin{longtable}[]{@{}
  >{\raggedright\arraybackslash}p{(\linewidth - 8\tabcolsep) * \real{0.2568}}
  >{\raggedright\arraybackslash}p{(\linewidth - 8\tabcolsep) * \real{0.0946}}
  >{\raggedright\arraybackslash}p{(\linewidth - 8\tabcolsep) * \real{0.1892}}
  >{\raggedright\arraybackslash}p{(\linewidth - 8\tabcolsep) * \real{0.2568}}
  >{\raggedright\arraybackslash}p{(\linewidth - 8\tabcolsep) * \real{0.2027}}@{}}
\toprule\noalign{}
\begin{minipage}[b]{\linewidth}\raggedright
IP Classification
\end{minipage} & \begin{minipage}[b]{\linewidth}\raggedright
Range
\end{minipage} & \begin{minipage}[b]{\linewidth}\raggedright
Default Mask
\end{minipage} & \begin{minipage}[b]{\linewidth}\raggedright
Available Networks
\end{minipage} & \begin{minipage}[b]{\linewidth}\raggedright
Hosts/Network
\end{minipage} \\
\midrule\noalign{}
\endhead
\bottomrule\noalign{}
\endlastfoot
Class A & 1.0.0.0 - 127.255.255.255 & 255.0.0.0 (/8) & 126 &
16,777,214 \\
Class B & 128.0.0.0 - 191.255.255.255 & 255.255.0.0 (/16) & 16,384 &
65,534 \\
Class C & 192.0.0.0 - 223.255.255.255 & 255.255.255.0 (/24) & 2,097,152
& 254 \\
Class D (Multicast) & 224.0.0.0 - 239.255.255.255 & N/A & N/A & N/A \\
Class E (Reserved) & 240.0.0.0 - 255.255.255.255 & N/A & N/A & N/A \\
\end{longtable}
}

\textbf{Special IP Ranges:}

\begin{itemize}
\tightlist
\item
  \textbf{Private IPs}: 10.0.0.0/8, 172.16.0.0/12, 192.168.0.0/16
\item
  \textbf{Loopback}: 127.0.0.0/8 (typically 127.0.0.1)
\item
  \textbf{Link-local}: 169.254.0.0/16
\end{itemize}

\textbf{Diagram:}

\begin{lstlisting}
CLASS A: |0|NETWORK(7 bits)|      HOST(24 bits)       |
CLASS B: |10|  NETWORK(14 bits)   |    HOST(16 bits)   |
CLASS C: |110| NETWORK(21 bits)        |  HOST(8 bits) |
CLASS D: |1110|       MULTICAST ADDRESS(28 bits)       |
CLASS E: |1111|       RESERVED ADDRESS(28 bits)        |
\end{lstlisting}

\begin{itemize}
\tightlist
\item
  \textbf{Classful addressing}: Original IP address classification
  scheme
\item
  \textbf{CIDR (Classless)}: Modern approach that allows flexible subnet
  masks
\item
  \textbf{IPv4 vs IPv6}: IPv4 uses 32-bit addresses, IPv6 uses 128-bit
  addresses
\end{itemize}

\end{solutionbox}
\begin{mnemonicbox}
``ABCDE'' - ``Address Blocks Categorized by
Decreasing End-host counts''

\end{mnemonicbox}
\subsection*{Question 3(a) [3 marks]}\label{q3a}

\textbf{What is full name of LAN? Explain it in detail.}

\begin{solutionbox}
LAN stands for Local Area Network, a network confined
to a limited geographic area.

\textbf{Diagram:}

\begin{lstlisting}
              LOCAL AREA NETWORK
   +--------+     +--------+     +--------+
   |Computer|-----|  Switch|-----|Computer|
   +--------+     +--------+     +--------+
                      |
                 +--------+     +--------+
                 |Printer |-----|Computer|
                 +--------+     +--------+
\end{lstlisting}

{\def\LTcaptype{none} % do not increment counter
\begin{longtable}[]{@{}ll@{}}
\toprule\noalign{}
LAN Characteristics & Description \\
\midrule\noalign{}
\endhead
\bottomrule\noalign{}
\endlastfoot
Geographic Scope & Building, campus, or small area (1-2 km) \\
Data Rate & High (10 Mbps to 10 Gbps) \\
Ownership & Single organization or individual \\
Technology & Ethernet, WiFi, Token Ring \\
Media & Twisted pair, fiber optic, wireless \\
\end{longtable}
}

\begin{itemize}
\tightlist
\item
  \textbf{Purpose}: Connect nearby devices for resource sharing
\item
  \textbf{Administration}: Easier management than larger networks
\item
  \textbf{Applications}: Office networking, home networking
\end{itemize}

\end{solutionbox}
\begin{mnemonicbox}
``LOCAL'' - ``Limited in range, Owned by one entity,
Connected devices, Access control, Low latency''

\end{mnemonicbox}
\subsection*{Question 3(b) [4 marks]}\label{q3b}

\textbf{Write a short-note of Repeater.}

\begin{solutionbox}
A repeater is a network device that amplifies and
regenerates signals to extend network range.

\textbf{Diagram:}

\begin{lstlisting}
 Signal              Signal
 weakens             restored
    |                   |
    v                   v
+-------+  Weak   +----------+  Strong  +-------+
|Network|-------->| Repeater |--------->|Network|
|Segment|  Signal +----------+  Signal  |Segment|
+-------+                             +-------+
\end{lstlisting}

{\def\LTcaptype{none} % do not increment counter
\begin{longtable}[]{@{}ll@{}}
\toprule\noalign{}
Feature & Description \\
\midrule\noalign{}
\endhead
\bottomrule\noalign{}
\endlastfoot
OSI Layer & Physical Layer (Layer 1) \\
Function & Signal regeneration and amplification \\
Purpose & Extend network transmission distance \\
Limitation & Cannot filter traffic or connect different networks \\
\end{longtable}
}

\begin{itemize}
\tightlist
\item
  \textbf{Operation}: Receives, regenerates, and retransmits signals
\item
  \textbf{Usage}: Extending cable length beyond normal limits
\item
  \textbf{Types}: Traditional repeaters, hubs (multiport repeaters)
\end{itemize}

\end{solutionbox}
\begin{mnemonicbox}
``RARE'' - ``Repeaters Amplify and Regenerate
Electrical signals''

\end{mnemonicbox}
\subsection*{Question 3(c) [7 marks]}\label{q3c}

\textbf{Write short note on FTP.}

\begin{solutionbox}
File Transfer Protocol (FTP) is a standard network
protocol for transferring files between clients and servers.

\textbf{Diagram:}

\begin{lstlisting}
                     Control Connection (Port 21)
               +--------------------------------+
               |                                |
     +--------+|                                |+--------+
     |        ||                                ||        |
     | CLIENT |+--------------------------------+| SERVER |
     |        |                                  |        |
     |        |                                  |        |
     +--------+                                  +--------+
               +--------------------------------+
                     Data Connection (Port 20)
\end{lstlisting}

{\def\LTcaptype{none} % do not increment counter
\begin{longtable}[]{@{}ll@{}}
\toprule\noalign{}
Feature & Description \\
\midrule\noalign{}
\endhead
\bottomrule\noalign{}
\endlastfoot
Port & Control: 21, Data: 20 \\
Mode & Active or Passive \\
Authentication & Username/password (or anonymous) \\
Transfer Types & ASCII (text) or Binary (raw data) \\
Security & Basic FTP (unsecured), FTPS, SFTP (secure variants) \\
\end{longtable}
}

\begin{itemize}
\tightlist
\item
  \textbf{Dual Channel}: Separate control and data connections
\item
  \textbf{Commands}: GET, PUT, LIST, DELETE, RENAME, etc.
\item
  \textbf{User Authentication}: Requires login credentials
\end{itemize}

\end{solutionbox}
\begin{mnemonicbox}
``CDATA'' - ``Control channel, Data channel,
Active/passive modes, Transfer types, Authentication''

\end{mnemonicbox}
\subsection*{Question 3(a) OR [3
marks]}\label{q3a}

\textbf{What is full name of PAN? Explain in detail.}

\begin{solutionbox}
PAN stands for Personal Area Network, a network for
connecting devices centered around an individual.

\textbf{Diagram:}

\begin{lstlisting}
              PERSONAL AREA NETWORK
                    +------+
                    |Person|
                    +------+
                       |
          +------------+------------+
          |            |            |
      +---------+   +----------+   +--------+
      |Smartphone|  |Smartwatch|   | Laptop |
      +---------+   +----------+   +--------+
          |
      +--------+
      |Earbuds |
      +--------+
\end{lstlisting}

{\def\LTcaptype{none} % do not increment counter
\begin{longtable}[]{@{}ll@{}}
\toprule\noalign{}
PAN Characteristics & Description \\
\midrule\noalign{}
\endhead
\bottomrule\noalign{}
\endlastfoot
Geographic Scope & Very small (1-10 meters) \\
Data Rate & Low to medium (100 Kbps - 100 Mbps) \\
Ownership & Individual person \\
Technology & Bluetooth, Zigbee, NFC, Infrared \\
Devices & Personal devices (phones, wearables, laptops) \\
\end{longtable}
}

\begin{itemize}
\tightlist
\item
  \textbf{Purpose}: Connect personal devices for communication/data
  sharing
\item
  \textbf{Types}: Wired PAN (USB) and Wireless PAN (Bluetooth)
\item
  \textbf{Applications}: Data synchronization, audio streaming, health
  monitoring
\end{itemize}

\end{solutionbox}
\begin{mnemonicbox}
``PIPER'' - ``Personal, Individual, Proximity, Easy
setup, Reduced range''

\end{mnemonicbox}
\subsection*{Question 3(b) OR [4
marks]}\label{q3b}

\textbf{What is the importance of a Bridge? Write short-note on it.}

\begin{solutionbox}
A bridge is a network device that connects and filters
traffic between network segments.

\textbf{Diagram:}

\begin{lstlisting}
   SEGMENT A                SEGMENT B
+-------------+          +-------------+
|             |          |             |
|  +------+   |          |  +------+   |
|  |Device|   |          |  |Device|   |
|  +------+   |          |  +------+   |
|      |      |          |      |      |
|      |      |   +---------+   |      |
|      +------|---| BRIDGE  |---+      |
|             |   +---------+          |
+-------------+          +-------------+
\end{lstlisting}

{\def\LTcaptype{none} % do not increment counter
\begin{longtable}[]{@{}ll@{}}
\toprule\noalign{}
Feature & Description \\
\midrule\noalign{}
\endhead
\bottomrule\noalign{}
\endlastfoot
OSI Layer & Data Link Layer (Layer 2) \\
Function & Connect similar network segments \\
Intelligence & Uses MAC addresses to filter traffic \\
Advantage & Reduces unnecessary traffic between segments \\
\end{longtable}
}

\begin{itemize}
\tightlist
\item
  \textbf{Importance}: Extends network, reduces collision domains
\item
  \textbf{Operation}: Learns MAC addresses, forwards frames selectively
\item
  \textbf{Types}: Transparent, translational, source-route bridges
\end{itemize}

\end{solutionbox}
\begin{mnemonicbox}
``SELF'' - ``Segmentation, Extension, Learning
addresses, Filtering traffic''

\end{mnemonicbox}
\subsection*{Question 3(c) OR [7
marks]}\label{q3c}

\textbf{What is DSL? Explain its different types.}

\begin{solutionbox}
Digital Subscriber Line (DSL) is a family of
technologies that provides digital data transmission over telephone
lines.

\textbf{Diagram:}

\begin{lstlisting}
                           +-------+
        +--------+         |       |
HOME----|  DSL   |---------| DSLAM |-------INTERNET
        | MODEM  |  Copper |       |
        +--------+   Line  +-------+
                    (POTS)    ISP
\end{lstlisting}

{\def\LTcaptype{none} % do not increment counter
\begin{longtable}[]{@{}
  >{\raggedright\arraybackslash}p{(\linewidth - 8\tabcolsep) * \real{0.1639}}
  >{\raggedright\arraybackslash}p{(\linewidth - 8\tabcolsep) * \real{0.1803}}
  >{\raggedright\arraybackslash}p{(\linewidth - 8\tabcolsep) * \real{0.2787}}
  >{\raggedright\arraybackslash}p{(\linewidth - 8\tabcolsep) * \real{0.1639}}
  >{\raggedright\arraybackslash}p{(\linewidth - 8\tabcolsep) * \real{0.2131}}@{}}
\toprule\noalign{}
\begin{minipage}[b]{\linewidth}\raggedright
DSL Type
\end{minipage} & \begin{minipage}[b]{\linewidth}\raggedright
Full Name
\end{minipage} & \begin{minipage}[b]{\linewidth}\raggedright
Speed (Down/Up)
\end{minipage} & \begin{minipage}[b]{\linewidth}\raggedright
Distance
\end{minipage} & \begin{minipage}[b]{\linewidth}\raggedright
Application
\end{minipage} \\
\midrule\noalign{}
\endhead
\bottomrule\noalign{}
\endlastfoot
ADSL & Asymmetric DSL & 8 Mbps/1 Mbps & Up to 5.5 km & Residential
internet \\
SDSL & Symmetric DSL & 2 Mbps/2 Mbps & Up to 3 km & Small business \\
VDSL & Very high-bit-rate DSL & 52-85 Mbps/16-85 Mbps & Up to 1.2 km &
Video streaming, businesses \\
HDSL & High-bit-rate DSL & 2 Mbps/2 Mbps & Up to 3.6 km & T1/E1
replacement \\
IDSL & ISDN DSL & 144 Kbps/144 Kbps & Up to 5.5 km & ISDN alternative \\
\end{longtable}
}

\begin{itemize}
\tightlist
\item
  \textbf{Working Principle}: Uses unused frequency spectrum on phone
  lines
\item
  \textbf{Advantage}: Uses existing telephone infrastructure
\item
  \textbf{Always-on}: Continuous connection without dial-up
\end{itemize}

\end{solutionbox}
\begin{mnemonicbox}
``SAVHI'' - ``Symmetric, Asymmetric, Very
high-bit-rate, High-bit-rate, ISDN DSL''

\end{mnemonicbox}
\subsection*{Question 4(a) [3 marks]}\label{q4a}

\textbf{Explain an error control and flow control at data link layer.}

\begin{solutionbox}
Error and flow control are essential data link layer
functions that ensure reliable data transmission.

{\def\LTcaptype{none} % do not increment counter
\begin{longtable}[]{@{}
  >{\raggedright\arraybackslash}p{(\linewidth - 4\tabcolsep) * \real{0.3438}}
  >{\raggedright\arraybackslash}p{(\linewidth - 4\tabcolsep) * \real{0.2812}}
  >{\raggedright\arraybackslash}p{(\linewidth - 4\tabcolsep) * \real{0.3750}}@{}}
\toprule\noalign{}
\begin{minipage}[b]{\linewidth}\raggedright
Mechanism
\end{minipage} & \begin{minipage}[b]{\linewidth}\raggedright
Purpose
\end{minipage} & \begin{minipage}[b]{\linewidth}\raggedright
Techniques
\end{minipage} \\
\midrule\noalign{}
\endhead
\bottomrule\noalign{}
\endlastfoot
Error Control & Detect/correct transmission errors & CRC, Checksums,
Parity bits \\
Flow Control & Prevent sender overwhelming receiver & Stop-and-wait,
Sliding window \\
\end{longtable}
}

\textbf{Diagram:}

\begin{lstlisting}
ERROR CONTROL:
  +------+  DATA   +-------+  ACK/NAK +--------+
  |Sender|-------->|Channel|--------->|Receiver|
  +------+         +-------+          +--------+

FLOW CONTROL:
  +------+  DATA   +--------+
  |Sender|-------->|Receiver|
  +------+  STOP   +--------+
         <---------
\end{lstlisting}

\begin{itemize}
\tightlist
\item
  \textbf{Error Detection}: CRC, checksum identify corrupted frames
\item
  \textbf{Error Correction}: Forward Error Correction (FEC),
  retransmission
\item
  \textbf{Flow Control}: Prevents buffer overflow at receiver
\end{itemize}

\end{solutionbox}
\begin{mnemonicbox}
``SAFE'' - ``Stop-and-wait, Acknowledgment, Flow
control, Error detection''

\end{mnemonicbox}
\subsection*{Question 4(b) [4 marks]}\label{q4b}

\textbf{What is Firewall? Explain it in detail.}

\begin{solutionbox}
A firewall is a network security device that monitors
and filters incoming and outgoing network traffic.

\textbf{Diagram:}

\begin{lstlisting}
   INTERNAL NETWORK                  INTERNET
+-------------------+             +--------------+
|                   |  FIREWALL   |              |
|  +-----+  +-----+ |  +------+   |  +--------+  |
|  |Host1|  |Host2| |  |      |   |  |External|  |
|  +-----+  +-----+ |--|FILTER|---|  |Server  |  |
|                   |  |      |   |  +--------+  |
|  +-----+  +-----+ |  +------+   |              |
|  |Host3|  |Host4| |             |              |
|  +-----+  +-----+ |             |              |
+-------------------+             +--------------+
\end{lstlisting}

{\def\LTcaptype{none} % do not increment counter
\begin{longtable}[]{@{}
  >{\raggedright\arraybackslash}p{(\linewidth - 4\tabcolsep) * \real{0.3947}}
  >{\raggedright\arraybackslash}p{(\linewidth - 4\tabcolsep) * \real{0.3684}}
  >{\raggedright\arraybackslash}p{(\linewidth - 4\tabcolsep) * \real{0.2368}}@{}}
\toprule\noalign{}
\begin{minipage}[b]{\linewidth}\raggedright
Firewall Type
\end{minipage} & \begin{minipage}[b]{\linewidth}\raggedright
Functionality
\end{minipage} & \begin{minipage}[b]{\linewidth}\raggedright
Example
\end{minipage} \\
\midrule\noalign{}
\endhead
\bottomrule\noalign{}
\endlastfoot
Packet Filtering & Examines packet headers & Router ACLs \\
Stateful Inspection & Tracks connection state & Most hardware
firewalls \\
Application Layer & Inspects content & Web application firewalls \\
Next-Generation & Combines multiple technologies & Palo Alto,
Fortinet \\
\end{longtable}
}

\begin{itemize}
\tightlist
\item
  \textbf{Purpose}: Protects networks from unauthorized access
\item
  \textbf{Implementation}: Hardware, software, or cloud-based
\item
  \textbf{Security Policy}: Rules defining allowed/blocked traffic
\end{itemize}

\end{solutionbox}
\begin{mnemonicbox}
``PAPSI'' - ``Packet filtering, Application layer,
Policies, Stateful inspection''

\end{mnemonicbox}
\subsection*{Question 4(c) [7 marks]}\label{q4c}

\textbf{Compare IPV4 and IPV6.}

\begin{solutionbox}
IPv4 and IPv6 are Internet Protocol versions with
significant differences in addressing and capabilities.

{\def\LTcaptype{none} % do not increment counter
\begin{longtable}[]{@{}
  >{\raggedright\arraybackslash}p{(\linewidth - 4\tabcolsep) * \real{0.4286}}
  >{\raggedright\arraybackslash}p{(\linewidth - 4\tabcolsep) * \real{0.2857}}
  >{\raggedright\arraybackslash}p{(\linewidth - 4\tabcolsep) * \real{0.2857}}@{}}
\toprule\noalign{}
\begin{minipage}[b]{\linewidth}\raggedright
Feature
\end{minipage} & \begin{minipage}[b]{\linewidth}\raggedright
IPv4
\end{minipage} & \begin{minipage}[b]{\linewidth}\raggedright
IPv6
\end{minipage} \\
\midrule\noalign{}
\endhead
\bottomrule\noalign{}
\endlastfoot
Address Size & 32-bit (4 bytes) & 128-bit (16 bytes) \\
Format & Dotted decimal (192.168.1.1) & Hexadecimal with colons
(2001:0db8:85a3::8a2e:0370:7334) \\
Address Space & \textasciitilde4.3 billion addresses & 340 undecillion
addresses \\
Header & Variable length (20-60 bytes) & Fixed length (40 bytes) \\
Fragmentation & Routers and sending hosts & Only sending hosts \\
Checksum & Included in header & Removed from header \\
Security & Not built-in (IPsec optional) & Built-in IPsec support \\
\end{longtable}
}

\textbf{Diagram:}

\begin{lstlisting}
IPv4: |VER|IHL|DSCP|ECN|  TOTAL LENGTH   |
      |  IDENTIFICATION   |FLAGS|FRAGMENT|
      |TTL |PROTOCOL|  HEADER CHECKSUM   |
      |        SOURCE ADDRESS            |
      |      DESTINATION ADDRESS         |
      |          OPTIONS...              |

IPv6: |VER|TRAFFIC CLASS|     FLOW LABEL      |
      |   PAYLOAD LENGTH   |NEXT HDR|HOP LIMIT|
      |                                       |
      |           SOURCE ADDRESS              |
      |                                       |
      |                                       |
      |                                       |
      |          DESTINATION ADDRESS          |
      |                                       |
\end{lstlisting}

\begin{itemize}
\tightlist
\item
  \textbf{Auto-configuration}: IPv6 has stateless address
  auto-configuration
\item
  \textbf{NAT}: Not required in IPv6 due to larger address space
\item
  \textbf{Transition}: Dual-stack, tunneling, translation mechanisms
\item
  \textbf{Header efficiency}: IPv6 has streamlined header for better
  performance
\end{itemize}

\end{solutionbox}
\begin{mnemonicbox}
``SHAPE'' - ``Size, Header, Addressing, Performance,
Extensibility''

\end{mnemonicbox}
\subsection*{Question 4(a) OR [3
marks]}\label{q4a}

\textbf{What is an IP address? How it is used in network?}

\begin{solutionbox}
An IP address is a numerical identifier assigned to
each device connected to a network that uses Internet Protocol.

\textbf{Diagram:}

\begin{lstlisting}
IP ADDRESS: 192.168.1.100
 +---+---+---+---+
 |192|168| 1 |100|  <-- Dotted decimal notation
 +---+---+---+---+
  |   |   |   |
  |   |   |   +---- Host identifier
  |   |   +-------- Subnet identifier
  +---+------------- Network identifier
\end{lstlisting}

{\def\LTcaptype{none} % do not increment counter
\begin{longtable}[]{@{}ll@{}}
\toprule\noalign{}
IP Address Usage & Description \\
\midrule\noalign{}
\endhead
\bottomrule\noalign{}
\endlastfoot
Identification & Uniquely identifies devices on a network \\
Routing & Determines path for data packets \\
Addressing & Enables sending data to specific destinations \\
Network Division & Allows subdivision into subnets \\
\end{longtable}
}

\begin{itemize}
\tightlist
\item
  \textbf{Structure}: Network portion and host portion
\item
  \textbf{Assignment}: Static (manual) or dynamic (DHCP)
\item
  \textbf{Versions}: IPv4 (32-bit) and IPv6 (128-bit)
\end{itemize}

\end{solutionbox}
\begin{mnemonicbox}
``IRAN'' - ``Identification, Routing, Addressing,
Network division''

\end{mnemonicbox}
\subsection*{Question 4(b) OR [4
marks]}\label{q4b}

\textbf{Compare FDDI and CDDI.}

\begin{solutionbox}
FDDI (Fiber Distributed Data Interface) and CDDI
(Copper Distributed Data Interface) are high-speed network technologies.

{\def\LTcaptype{none} % do not increment counter
\begin{longtable}[]{@{}
  >{\raggedright\arraybackslash}p{(\linewidth - 4\tabcolsep) * \real{0.4286}}
  >{\raggedright\arraybackslash}p{(\linewidth - 4\tabcolsep) * \real{0.2857}}
  >{\raggedright\arraybackslash}p{(\linewidth - 4\tabcolsep) * \real{0.2857}}@{}}
\toprule\noalign{}
\begin{minipage}[b]{\linewidth}\raggedright
Feature
\end{minipage} & \begin{minipage}[b]{\linewidth}\raggedright
FDDI
\end{minipage} & \begin{minipage}[b]{\linewidth}\raggedright
CDDI
\end{minipage} \\
\midrule\noalign{}
\endhead
\bottomrule\noalign{}
\endlastfoot
Medium & Fiber optic cable & Copper twisted pair \\
Speed & 100 Mbps & 100 Mbps \\
Distance & Up to 200 km total, 2 km between stations & Up to 100 m
between stations \\
Topology & Dual counter-rotating rings & Dual counter-rotating rings \\
Cost & Higher & Lower \\
Reliability & Very high & Moderate \\
Standard & ANSI X3T9.5 & Same as FDDI (adapted for copper) \\
\end{longtable}
}

\textbf{Diagram:}

\begin{lstlisting}
FDDI/CDDI DUAL RING TOPOLOGY:
      +-----+         +-----+
      |     |         |     |
  +-->|Node1|-------->|Node2|----+
  |   |     |         |     |    |
  |   +-----+         +-----+    |
  |                              |
  |   +-----+         +-----+    |
  +---|Node4|<--------|Node3|<---+
      |     |         |     |
      +-----+         +-----+
\end{lstlisting}

\begin{itemize}
\tightlist
\item
  \textbf{Redundancy}: Secondary ring for fault tolerance
\item
  \textbf{Access Method}: Token passing with timed token rotation
\item
  \textbf{Applications}: FDDI for backbones, CDDI for workstations
\end{itemize}

\end{solutionbox}
\begin{mnemonicbox}
``FDDI Flies, CDDI Crawls'' - Fiber for long
distance, Copper for shorter runs

\end{mnemonicbox}
\subsection*{Question 4(c) OR [7
marks]}\label{q4c}

\textbf{Draw and explain OSI reference model in detail.}

\begin{solutionbox}
The OSI (Open Systems Interconnection) model is a
conceptual framework that standardizes network functions into seven
layers.

\textbf{Diagram:}

\begin{lstlisting}
+-----------------------------+
|         APPLICATION (7)     |
|      User interface, apps   |
+-----------------------------+
|        PRESENTATION (6)     |
|    Data format, encryption  |
+-----------------------------+
|          SESSION (5)        |
|    Connection management    |
+-----------------------------+
|         TRANSPORT (4)       |
|   End-to-end reliability    |
+-----------------------------+
|          NETWORK (3)        |
|   Routing between networks  |
+-----------------------------+
|         DATA LINK (2)       |
|  Node-to-node reliability   |
+-----------------------------+
|          PHYSICAL (1)       |
|   Physical transmission     |
+-----------------------------+
\end{lstlisting}

{\def\LTcaptype{none} % do not increment counter
\begin{longtable}[]{@{}
  >{\raggedright\arraybackslash}p{(\linewidth - 6\tabcolsep) * \real{0.1228}}
  >{\raggedright\arraybackslash}p{(\linewidth - 6\tabcolsep) * \real{0.3158}}
  >{\raggedright\arraybackslash}p{(\linewidth - 6\tabcolsep) * \real{0.3684}}
  >{\raggedright\arraybackslash}p{(\linewidth - 6\tabcolsep) * \real{0.1930}}@{}}
\toprule\noalign{}
\begin{minipage}[b]{\linewidth}\raggedright
Layer
\end{minipage} & \begin{minipage}[b]{\linewidth}\raggedright
Primary Function
\end{minipage} & \begin{minipage}[b]{\linewidth}\raggedright
Protocols/Standards
\end{minipage} & \begin{minipage}[b]{\linewidth}\raggedright
Data Unit
\end{minipage} \\
\midrule\noalign{}
\endhead
\bottomrule\noalign{}
\endlastfoot
Application & User interface, network services & HTTP, FTP, SMTP &
Data \\
Presentation & Data formatting, encryption & SSL/TLS, JPEG, MIME &
Data \\
Session & Connection establishment, management & NetBIOS, RPC & Data \\
Transport & End-to-end delivery, flow control & TCP, UDP & Segments \\
Network & Logical addressing, routing & IP, ICMP, OSPF & Packets \\
Data Link & Physical addressing, media access & Ethernet, PPP, HDLC &
Frames \\
Physical & Bit transmission, cabling, signaling & USB, Ethernet,
Bluetooth & Bits \\
\end{longtable}
}

\begin{itemize}
\tightlist
\item
  \textbf{Layer Independence}: Each layer performs specific functions
\item
  \textbf{Encapsulation}: Data wrapped with headers at each layer
\item
  \textbf{Standardization}: Promotes interoperability between systems
\item
  \textbf{Troubleshooting}: Isolates problems to specific layers
\end{itemize}

\end{solutionbox}
\begin{mnemonicbox}
``All People Seem To Need Data Processing'' (Layers 7
to 1)

\end{mnemonicbox}
\subsection*{Question 5(a) [3 marks]}\label{q5a}

\textbf{What is ISO? How it works in information security?}

\begin{solutionbox}
ISO (International Organization for Standardization)
develops and publishes standards including those for information
security.

{\def\LTcaptype{none} % do not increment counter
\begin{longtable}[]{@{}ll@{}}
\toprule\noalign{}
ISO Security Standards & Purpose \\
\midrule\noalign{}
\endhead
\bottomrule\noalign{}
\endlastfoot
ISO/IEC 27001 & Information security management systems \\
ISO/IEC 27002 & Code of practice for security controls \\
ISO/IEC 27005 & Information security risk management \\
ISO/IEC 27017 & Cloud security \\
ISO/IEC 27018 & Protection of personally identifiable information \\
\end{longtable}
}

\textbf{Working in Information Security:}

\begin{itemize}
\tightlist
\item
  \textbf{Framework-based}: Provides structured approach to security
\item
  \textbf{Risk-based}: Focuses on identification and mitigation of risks
\item
  \textbf{Process-oriented}: Establishes continuous improvement cycle
\item
  \textbf{Certification}: Organizations can be certified for compliance
\end{itemize}

\end{solutionbox}
\begin{mnemonicbox}
``PRIMP'' - ``Policies, Risk assessment,
Implementation, Monitoring, Process improvement''

\end{mnemonicbox}
\subsection*{Question 5(b) [4 marks]}\label{q5b}

\textbf{Explain terms in detail for cryptography: 1) Encryption 2)
Decryption}

\begin{solutionbox}
Encryption and decryption are fundamental processes in
cryptography that secure information.

{\def\LTcaptype{none} % do not increment counter
\begin{longtable}[]{@{}
  >{\raggedright\arraybackslash}p{(\linewidth - 6\tabcolsep) * \real{0.1364}}
  >{\raggedright\arraybackslash}p{(\linewidth - 6\tabcolsep) * \real{0.2727}}
  >{\raggedright\arraybackslash}p{(\linewidth - 6\tabcolsep) * \real{0.1591}}
  >{\raggedright\arraybackslash}p{(\linewidth - 6\tabcolsep) * \real{0.4318}}@{}}
\toprule\noalign{}
\begin{minipage}[b]{\linewidth}\raggedright
Term
\end{minipage} & \begin{minipage}[b]{\linewidth}\raggedright
Definition
\end{minipage} & \begin{minipage}[b]{\linewidth}\raggedright
Types
\end{minipage} & \begin{minipage}[b]{\linewidth}\raggedright
Example Algorithms
\end{minipage} \\
\midrule\noalign{}
\endhead
\bottomrule\noalign{}
\endlastfoot
Encryption & Process of converting plaintext to ciphertext using an
algorithm and key & Symmetric, Asymmetric, Hybrid & AES, RSA, ECC \\
Decryption & Process of converting ciphertext back to plaintext using an
algorithm and key & Symmetric, Asymmetric, Hybrid & AES, RSA, ECC \\
\end{longtable}
}

\textbf{Diagram:}

\begin{lstlisting}
ENCRYPTION:
  +-----------+    ENCRYPTION    +------------+
  | PLAINTEXT |----------------->| CIPHERTEXT |
  +-----------+   ALGORITHM &    +------------+
                  KEY

DECRYPTION:
  +------------+    DECRYPTION    +-----------+
  | CIPHERTEXT |----------------->| PLAINTEXT |
  +------------+   ALGORITHM &    +-----------+
                   KEY
\end{lstlisting}

\textbf{Encryption:}

\begin{itemize}
\tightlist
\item
  \textbf{Purpose}: Protects confidentiality of information
\item
  \textbf{Methods}: Substitution, transposition, block cipher, stream
  cipher
\item
  \textbf{Key Management}: Critical aspect of secure encryption
\end{itemize}

\textbf{Decryption:}

\begin{itemize}
\tightlist
\item
  \textbf{Purpose}: Retrieves original information from encrypted form
\item
  \textbf{Requirements}: Correct algorithm and key
\item
  \textbf{Implementation}: Hardware or software-based
\end{itemize}

\end{solutionbox}
\begin{mnemonicbox}
``PACK-DUKE'' - ``Plaintext Algorithm Cipher Key -
Decoding Using Key for Extraction''

\end{mnemonicbox}
\subsection*{Question 5(c) [7 marks]}\label{q5c}

\textbf{Write a short-note on 1) E-mail and 2) DNS}

\begin{solutionbox}
\textbf{1) E-mail (Electronic Mail):}

E-mail is a method of exchanging digital messages over a communication
network.

\textbf{Diagram:}

\begin{lstlisting}
E-MAIL SYSTEM:
   +--------+    SMTP     +---------+    POP3/IMAP   +--------+
   | SENDER |------------>|  MAIL   |--------------->|RECEIVER|
   | CLIENT |             | SERVER  |                | CLIENT |
   +--------+             +---------+                +--------+
                               |
                          +---------+
                          |   DNS   |
                          | SERVER  |
                          +---------+
\end{lstlisting}

{\def\LTcaptype{none} % do not increment counter
\begin{longtable}[]{@{}ll@{}}
\toprule\noalign{}
Component & Function \\
\midrule\noalign{}
\endhead
\bottomrule\noalign{}
\endlastfoot
Mail User Agent (MUA) & Email client software used by end-users \\
Mail Transfer Agent (MTA) & Server software that transfers emails \\
Mail Delivery Agent (MDA) & Delivers email to recipient's mailbox \\
Protocols & SMTP (sending), POP3/IMAP (receiving) \\
\end{longtable}
}

\begin{itemize}
\tightlist
\item
  \textbf{Structure}: Headers (To, From, Subject) and Body
\item
  \textbf{Security}: Features like encryption (TLS), authentication
  (SPF, DKIM)
\item
  \textbf{Attachments}: Binary files encoded for text transmission
\item
  \textbf{Features}: Forwarding, filtering, organizing, searching
\end{itemize}

\textbf{2) DNS (Domain Name System):}

DNS is a hierarchical and decentralized naming system for translating
domain names to IP addresses.

\textbf{Diagram:}

\begin{lstlisting}
DNS HIERARCHY:
              +---------+
              |   Root  |
              |   "."   |
              +---------+
                   |
       +-----------+-----------+
       |           |           |
  +---------+ +---------+ +---------+
  |   com   | |   org   | |   net   | ... (TLDs)
  +---------+ +---------+ +---------+
       |           |           |
  +-----------+ +-----------+ +-----------+
  |example.com| |example.org| |example.net| ... (Domains)
  +-----------+ +-----------+ +-----------+
       |
  +---------------+
  |www.example.com| ... (Subdomains)
  +---------------+
\end{lstlisting}

{\def\LTcaptype{none} % do not increment counter
\begin{longtable}[]{@{}ll@{}}
\toprule\noalign{}
DNS Component & Function \\
\midrule\noalign{}
\endhead
\bottomrule\noalign{}
\endlastfoot
Root Servers & Top of DNS hierarchy \\
TLD Servers & Manage top-level domains (.com, .org) \\
Authoritative Servers & Store DNS records for specific domains \\
Recursive Resolvers & Query other servers to resolve domain names \\
DNS Records & Resource records (A, AAAA, MX, CNAME, etc.) \\
\end{longtable}
}

\begin{itemize}
\tightlist
\item
  \textbf{Purpose}: Map human-readable names to machine-readable
  addresses
\item
  \textbf{Resolution Process}: Recursive or iterative queries through
  hierarchy
\item
  \textbf{Caching}: Temporary storage of results to improve performance
\item
  \textbf{Security}: DNSSEC provides authentication and integrity
\end{itemize}

\end{solutionbox}
\begin{mnemonicbox}
``MAPS'' - ``Mail needs Addresses, Protocols, and
\end{mnemonicbox}
\begin{mnemonicbox}
Servers''  ``HARD'' - ``Hierarchy, Addressing,
Resolution, Distributed system''

\end{mnemonicbox}
\subsection*{Question 5(a) OR [3
marks]}\label{q5a}

\textbf{What do you mean by security topology and security zone?}

\begin{solutionbox}
Security topology and security zones are network
security concepts that organize and protect network resources.

{\def\LTcaptype{none} % do not increment counter
\begin{longtable}[]{@{}
  >{\raggedright\arraybackslash}p{(\linewidth - 4\tabcolsep) * \real{0.2903}}
  >{\raggedright\arraybackslash}p{(\linewidth - 4\tabcolsep) * \real{0.3871}}
  >{\raggedright\arraybackslash}p{(\linewidth - 4\tabcolsep) * \real{0.3226}}@{}}
\toprule\noalign{}
\begin{minipage}[b]{\linewidth}\raggedright
Concept
\end{minipage} & \begin{minipage}[b]{\linewidth}\raggedright
Definition
\end{minipage} & \begin{minipage}[b]{\linewidth}\raggedright
Examples
\end{minipage} \\
\midrule\noalign{}
\endhead
\bottomrule\noalign{}
\endlastfoot
Security Topology & Physical and logical arrangement of security
controls & DMZ, Defense-in-depth \\
Security Zone & Segment of network with specific security requirements &
DMZ, Intranet, Extranet \\
\end{longtable}
}

\textbf{Diagram:}

\begin{lstlisting}
SECURITY TOPOLOGY WITH ZONES:
                  +----------+
                  | INTERNET |
                  +----+-----+
                       |
                       | Firewall
                       |
                  +----+-----+
                  |   DMZ    |  Web, Email, DNS servers
                  +----+-----+
                       |
                       | Firewall
                       |
         +-------------+-------------+
         |                           |
    +----+----+                 +----+----+
    | INTRANET |                 | SECURED |
    | ZONE     |                 | ZONE    |  Sensitive data
    +----+----+                 +----+----+
         |
    +----+----+
    |   USER   |  Workstations
    |   ZONE   |
    +---------+
\end{lstlisting}

\begin{itemize}
\tightlist
\item
  \textbf{Security Topology}: Overall security architecture design
\item
  \textbf{Security Zones}: Logical boundaries with consistent security
  policies
\item
  \textbf{Defense-in-depth}: Multiple layers of security controls
\end{itemize}

\end{solutionbox}
\begin{mnemonicbox}
``TIPS'' - ``Topology Isolates and Protects Systems''

\end{mnemonicbox}
\subsection*{Question 5(b) OR [4
marks]}\label{q5b}

\textbf{Write short-note on Voice and Video IP.}

\begin{solutionbox}
Voice and Video over IP (VoIP/Video IP) refers to
technologies for transmitting voice and video communications over IP
networks.

\textbf{Diagram:}

\begin{lstlisting}
  +--------+                      +--------+
  |        |      INTERNET        |        |
  | CALLER |----------------------|RECEIVER|
  |        |   RTP/UDP/IP         |        |
  +--------+                      +--------+
      |                               |
      |                               |
   +-----+                         +-----+
   |Codec|                         |Codec|
   +-----+                         +-----+
   Digital                         Digital
   encoding                        decoding
\end{lstlisting}

{\def\LTcaptype{none} % do not increment counter
\begin{longtable}[]{@{}ll@{}}
\toprule\noalign{}
Component & Function \\
\midrule\noalign{}
\endhead
\bottomrule\noalign{}
\endlastfoot
Codecs & Encode/decode audio and video (G.711, H.264) \\
Signaling Protocols & Call setup/teardown (SIP, H.323) \\
Transport Protocol & Real-time media transport (RTP/RTCP) \\
QoS Mechanisms & Prioritize voice/video traffic \\
\end{longtable}
}

\textbf{Voice over IP (VoIP):}

\begin{itemize}
\tightlist
\item
  \textbf{Benefits}: Cost savings, flexibility, integration with apps
\item
  \textbf{Challenges}: Latency, jitter, packet loss
\item
  \textbf{Applications}: IP phones, softphones, conferencing
\end{itemize}

\textbf{Video over IP:}

\begin{itemize}
\tightlist
\item
  \textbf{Types}: Video conferencing, streaming, surveillance
\item
  \textbf{Requirements}: Higher bandwidth, low latency
\item
  \textbf{Technologies}: WebRTC, SIP video, RTSP streaming
\end{itemize}

\end{solutionbox}
\begin{mnemonicbox}
``CLEAR'' - ``Codecs compress, Latency matters,
Encodes audio/video, Applications integrate, Real-time transport''

\end{mnemonicbox}
\subsection*{Question 5(c) OR [7
marks]}\label{q5c}

\textbf{What is IP security? Explain in detail.}

\begin{solutionbox}
IP Security (IPsec) is a suite of protocols designed to
secure IP communications by authenticating and encrypting each IP
packet.

\textbf{Diagram:}

\begin{lstlisting}
IPSEC PROTOCOL SUITE:
+--------------------------------------+
|            APPLICATIONS              |
+--------------------------------------+
|      TRANSPORT LAYER (TCP/UDP)       |
+--------------------------------------+
|                                      |
|              IP LAYER                |
|                                      |
|  +------------+    +-------------+   |
|  |    AH      |    |     ESP     |   |
|  | (Auth Hdr) |    | (Enc Sec Pay)|  |
|  +------------+    +-------------+   |
|                                      |
|         +-------------------+        |
|         |   IKE/ISAKMP      |        |
|         | (Key Management)  |        |
|         +-------------------+        |
+--------------------------------------+
|           NETWORK ACCESS             |
+--------------------------------------+
\end{lstlisting}

{\def\LTcaptype{none} % do not increment counter
\begin{longtable}[]{@{}
  >{\raggedright\arraybackslash}p{(\linewidth - 4\tabcolsep) * \real{0.4211}}
  >{\raggedright\arraybackslash}p{(\linewidth - 4\tabcolsep) * \real{0.2632}}
  >{\raggedright\arraybackslash}p{(\linewidth - 4\tabcolsep) * \real{0.3158}}@{}}
\toprule\noalign{}
\begin{minipage}[b]{\linewidth}\raggedright
IPsec Protocol
\end{minipage} & \begin{minipage}[b]{\linewidth}\raggedright
Function
\end{minipage} & \begin{minipage}[b]{\linewidth}\raggedright
Protection
\end{minipage} \\
\midrule\noalign{}
\endhead
\bottomrule\noalign{}
\endlastfoot
Authentication Header (AH) & Data integrity, authentication & No
encryption \\
Encapsulating Security Payload (ESP) & Confidentiality, integrity,
authentication & Encrypts data \\
Internet Key Exchange (IKE) & Key exchange, SA negotiation & Secure key
management \\
\end{longtable}
}

\textbf{IPsec Modes:}

{\def\LTcaptype{none} % do not increment counter
\begin{longtable}[]{@{}
  >{\raggedright\arraybackslash}p{(\linewidth - 4\tabcolsep) * \real{0.2069}}
  >{\raggedright\arraybackslash}p{(\linewidth - 4\tabcolsep) * \real{0.4483}}
  >{\raggedright\arraybackslash}p{(\linewidth - 4\tabcolsep) * \real{0.3448}}@{}}
\toprule\noalign{}
\begin{minipage}[b]{\linewidth}\raggedright
Mode
\end{minipage} & \begin{minipage}[b]{\linewidth}\raggedright
Description
\end{minipage} & \begin{minipage}[b]{\linewidth}\raggedright
Use Case
\end{minipage} \\
\midrule\noalign{}
\endhead
\bottomrule\noalign{}
\endlastfoot
Transport Mode & Protects payload only & Host-to-host communications \\
Tunnel Mode & Protects entire packet & Site-to-site VPNs, remote
access \\
\end{longtable}
}

\textbf{Security Services:}

\begin{itemize}
\tightlist
\item
  \textbf{Authentication}: Verifies identity of communicating entities
\item
  \textbf{Confidentiality}: Protects data from unauthorized disclosure
\item
  \textbf{Data Integrity}: Ensures data hasn't been altered in transit
\item
  \textbf{Replay Protection}: Prevents packet replay attacks
\item
  \textbf{Access Control}: Limits access to network resources
\end{itemize}

\textbf{Applications:}

\begin{itemize}
\tightlist
\item
  \textbf{VPNs}: Remote access and site-to-site connections
\item
  \textbf{Secure Routing}: Protects routing protocols
\item
  \textbf{Secure Host-to-Host}: End-to-end security
\end{itemize}

\end{solutionbox}
\begin{mnemonicbox}
``AVID TC'' - ``Authentication, Verification,
Integrity, Datagram protection, Transport mode, Confidentiality''

\end{mnemonicbox}

\end{document}
