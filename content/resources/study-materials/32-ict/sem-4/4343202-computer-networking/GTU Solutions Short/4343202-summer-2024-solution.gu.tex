\documentclass{article}

% content/resources/templates/preamble.tex
\usepackage[margin=0.6in]{geometry}
\author{Milav Dabgar}
\usepackage{amsmath,amssymb,amsthm}
\usepackage{booktabs}
\usepackage{multirow}
\usepackage{xcolor}
\usepackage{tcolorbox}
\tcbuselibrary{breakable,skins}
\usepackage[colorlinks=true,linkcolor=blue]{hyperref}
\usepackage{titlesec}
\usepackage{enumitem}
\usepackage{tikz}
\usepackage{pgfplots}
\usepackage{circuitikz}
\usepackage[version=4]{mhchem}
\usepackage{longtable}
\usepackage{array}
\usepackage{float}
\usepackage{caption}
\usepackage{listings}

\lstset{
  basicstyle=\small\ttfamily,
  breaklines=true,
  breakatwhitespace=false,
  postbreak=\mbox{\textcolor{red}{$\hookrightarrow$}\space},
  float=false,
  numbers=left,
  numberstyle=\tiny\color{gray},
  numbersep=10pt,
  xleftmargin=2em,
  keywordstyle=\color{blue},
  commentstyle=\color{green!60!black},
  stringstyle=\color{purple},
  backgroundcolor=\color{gray!5},
  showstringspaces=false,
  tabsize=2,
  captionpos=b,
  keepspaces=true,
  columns=flexible
}

\pgfplotsset{compat=1.18}
\usetikzlibrary{shapes,arrows,positioning,calc,patterns,decorations.pathmorphing,decorations.markings,arrows.meta}

% Color scheme
\definecolor{headcolor}{RGB}{0,102,204}
\definecolor{keycolor}{RGB}{220,20,60}
\definecolor{solutioncolor}{RGB}{34,139,34}
\definecolor{mnemoniccolor}{RGB}{148,0,211}
\definecolor{codecolor}{RGB}{0,0,100}

% Spacing
\setlength{\parskip}{3pt}
\setlist[itemize]{nosep}
\setlist[enumerate]{nosep}

% Title formatting
\titleformat{\section}{\Large\bfseries\color{headcolor}}{\thesection}{1em}{}
\titleformat{\subsection}{\large\bfseries\color{headcolor}}{\thesubsection}{1em}{}

% Pandoc tightlist compatibility
\providecommand{\tightlist}{%
  \setlength{\itemsep}{0pt}\setlength{\parskip}{0pt}}

% Pandoc longtable compatibility
\newcounter{none}
\def\thenone{}


% content/resources/templates/gujarati-boxes.tex
\usepackage{fontspec}
\usepackage{polyglossia}

% Set Gujarati as main language (document is primarily in Gujarati)
% Note: gloss-gujarati.ldf doesn't exist in polyglossia, but it will use hyphenation patterns
\setdefaultlanguage{gujarati}
\setotherlanguage{english}

% Configure Gujarati font properly
% Use Language=Default to prevent polyglossia from trying to add language-specific features
% that don't exist for Gujarati, which causes "empty feature" warnings
\newfontfamily\gujaratifont[Script=Gujarati,AutoFakeBold=2.5,AutoFakeSlant=0.3]{Noto Sans Gujarati}
\setmainfont[Script=Gujarati,AutoFakeBold=2.5,AutoFakeSlant=0.3]{Noto Sans Gujarati}
% Use Noto Sans Gujarati for monospace to support Gujarati in text
\setmonofont[Scale=0.9]{Noto Sans Gujarati}

% Configure English to use the same font
\newfontfamily\englishfont[Script=Gujarati,AutoFakeBold=2.5,AutoFakeSlant=0.3]{Noto Sans Gujarati}

% Translations for polyglossia
\gappto\captionsgujarati{
  \renewcommand{\tablename}{કોષ્ટક}
  \renewcommand{\figurename}{આકૃતિ}
}

% Helper for TikZ nodes to ensure Gujarati font
\newcommand{\gu}[1]{{\gujaratifont #1}}

% Custom environments
\newtcolorbox{solutionbox}{
    breakable,
    enhanced,
    colback=solutioncolor!5!white,
    colframe=solutioncolor!75!black,
    fonttitle=\bfseries,
    title=જવાબ
}

\newtcolorbox{solutionboxnobreak}{
 colback=solutioncolor!5!white,
 colframe=solutioncolor!75!black,
 fonttitle=\bfseries,
 title=જવાબ
}

\newtcolorbox{keyformula}{
 breakable,
 enhanced,
 colback=keycolor!5!white,
 colframe=keycolor!75!black,
 fonttitle=\bfseries,
 title=રાસાયણિક સમીકરણ/સૂત્ર
}

\newtcolorbox{mnemonicbox}{
 breakable,
 enhanced,
 colback=mnemoniccolor!5!white,
 colframe=mnemoniccolor!75!black,
 fonttitle=\bfseries,
 title=મેમરી ટ્રીક
}


% Custom commands for GTU solutions
% This file defines semantic commands for consistent formatting

% Question command with automatic formatting
\newcommand{\question}[2]{%
  \section*{Question #1}%
  \textbf{#2}%
}

% OR question variant
\newcommand{\questionor}[2]{%
  \section*{Question #1 OR}%
  \textbf{#2}%
}

% Proper table environment with caption
\newenvironment{answertable}[1]{%
  \begin{table}[htbp]
  \centering
  \caption{#1}
}{%
  \end{table}
}

% Proper figure environment for diagrams
\newenvironment{answerdiagram}[1]{%
  \begin{figure}[htbp]
  \centering
  \caption{#1}
}{%
  \end{figure}
}

% Semantic markup for key terms
\newcommand{\keyword}[1]{\textbf{#1}}
\newcommand{\code}[1]{\texttt{#1}}
\newcommand{\classname}[1]{\texttt{#1}}
\newcommand{\methodname}[1]{\texttt{#1}}

% Proper quotation marks
\newcommand{\mnemonic}[1]{``#1''}


\title{કમ્પ્યુટર નેટવર્કિંગ (4343202) – સમર 2024 સોલ્યુશન}
\date{June 13, 2024}

\begin{document}
\maketitle

\questionmarks{1(અ)}{3}{પેકેટ સ્વીચીંગ નેટવર્ક સમજાવો.}

\begin{solutionbox}
પેકેટ સ્વીચીંગ એ નેટવર્ક કમ્યુનિકેશન પદ્ધતિ છે જેમાં ડેટા ટ્રાન્સમિશન પહેલા નાના પેકેટ્સમાં વિભાજિત કરવામાં આવે છે.

\textbf{Packet Switching Process:}

\begin{center}
\begin{tikzpicture}[node distance=1.3cm, >=stealth, every node/.style={font=\small}]
    \node [draw, rectangle, rounded corners, minimum width=2cm, minimum height=0.7cm] (source) {સોર્સ};
    \node [draw, rectangle, rounded corners, minimum width=2.5cm, minimum height=0.7cm, right=of source, align=center] (split) {પેકેટ્સમાં\\વિભાજન};
    \node [draw, rectangle, rounded corners, minimum width=2.5cm, minimum height=0.7cm, right=of split, align=center] (route) {નેટવર્ક દ્વારા\\રાઉટિંગ};
    \node [draw, rectangle, rounded corners, minimum width=2.5cm, minimum height=0.7cm, right=of route, align=center] (reassemble) {પેકેટ્સ\\પુનઃએસેમ્બલ};
    \node [draw, rectangle, rounded corners, minimum width=2cm, minimum height=0.7cm, right=of reassemble] (dest) {ડેસ્ટિનેશન};
    
    \draw [->, thick] (source) -- (split);
    \draw [->, thick] (split) -- (route);
    \draw [->, thick] (route) -- (reassemble);
    \draw [->, thick] (reassemble) -- (dest);
\end{tikzpicture}
\captionof{figure}{Packet Switching Network}
\end{center}

\begin{itemize}
    \item \keyword{સ્વતંત્ર રાઉટિંગ}: દરેક પેકેટ નેટવર્કમાં સ્વતંત્ર રીતે પ્રવાસ કરે છે
    \item \keyword{લવચીક માર્ગો}: પેકેટ્સ ડેસ્ટિનેશન સુધી પહોંચવા માટે અલગ-અલગ રૂટ્સ લઈ શકે છે
    \item \keyword{કાર્યક્ષમતા}: નેટવર્ક બેન્ડવિડ્થનો વધુ સારો ઉપયોગ
\end{itemize}
\end{solutionbox}

\begin{mnemonicbox}
\mnemonic{DIVE: Data Into Various Elements}
\end{mnemonicbox}

\questionmarks{1(બ)}{4}{OSI રેફરન્સ મોડેલનાં કોઈ પણ 4 સ્તરોનું કાર્ય સમજાવો.}

\begin{solutionbox}
OSI મોડેલ નેટવર્ક કમ્યુનિકેશનને સાત અલગ-અલગ સ્તરોમાં વિભાજિત કરે છે, દરેક સ્તરની ચોક્કસ કાર્યો છે.

\begin{center}
\captionof{table}{OSI Layer Functions}
\begin{tabulary}{\linewidth}{|L|L|L|}
\hline
\textbf{સ્તર} & \textbf{કાર્ય} & \textbf{મુખ્ય પ્રોટોકોલ્સ} \\ \hline
એપ્લિકેશન & યુઝર એપ્લિકેશનને સીધી નેટવર્ક સેવાઓ પ્રદાન કરે છે & HTTP, FTP, SMTP \\ \hline
પ્રેઝન્ટેશન & ડેટાનું અનુવાદ, એન્ક્રિપ્શન અને કમ્પ્રેશન કરે છે & SSL, TLS, JPEG \\ \hline
સેશન & કનેક્શન સ્થાપિત, સંચાલિત અને સમાપ્ત કરે છે & NetBIOS, RPC \\ \hline
ટ્રાન્સપોર્ટ & એન્ડ-ટુ-એન્ડ ડેટા ટ્રાન્સફર સુનિશ્ચિત કરે છે & TCP, UDP \\ \hline
\end{tabulary}
\end{center}

\begin{itemize}
    \item \keyword{એપ્લિકેશન લેયર}: નેટવર્ક અને એપ્લિકેશન વચ્ચે ઇન્ટરફેસ
    \item \keyword{પ્રેઝન્ટેશન લેયર}: ડેટા ફોર્મેટિંગ અને એન્ક્રિપ્શન
    \item \keyword{સેશન લેયર}: ડાયલોગ કંટ્રોલ અને સિંક્રોનાઇઝેશન
    \item \keyword{ટ્રાન્સપોર્ટ લેયર}: એન્ડ-ટુ-એન્ડ કનેક્શન અને વિશ્વસનીયતા
\end{itemize}
\end{solutionbox}

\begin{mnemonicbox}
\mnemonic{All People Seem To Need Data Processing}
\end{mnemonicbox}

\questionmarks{1(ક)}{7}{નેટવર્ક ટોપોલોજી આકૃતિ સાથે સમજાવો.}

\begin{solutionbox}
નેટવર્ક ટોપોલોજી નેટવર્કમાં ડિવાઇસની ભૌતિક અથવા તાર્કિક ગોઠવણને દર્શાવે છે.

\begin{center}
\captionof{table}{Network Topology Comparison}
\begin{tabulary}{\linewidth}{|L|L|L|}
\hline
\textbf{ટોપોલોજી} & \textbf{ફાયદાઓ} & \textbf{ગેરફાયદાઓ} \\ \hline
બસ & સરળ, સસ્તી & એક પોઇન્ટ ફેલ્યોર \\ \hline
સ્ટાર & સહેલાઈથી ટ્રબલશૂટિંગ, કેન્દ્રીય & હબ/સ્વિચ ફેલ્યોરથી બધા પ્રભાવિત \\ \hline
રિંગ & બધા નોડ્સને સમાન એક્સેસ & એક કેબલ ફેલ્યોર નેટવર્કને અસર કરે \\ \hline
મેશ & ઉચ્ચ વિશ્વસનીયતા, ટ્રાફિક સમસ્યાઓ નહીં & ખર્ચાળ, જટિલ \\ \hline
ટ્રી & સરળતાથી વિસ્તરણીય, સંરચિત & રૂટ પર આધારિત, જટિલ \\ \hline
\end{tabulary}
\end{center}

\textbf{આકૃતિ:}

\begin{center}
\begin{tikzpicture}[node distance=1.5cm]
    % Bus Topology
    \node at (0,3) {\textbf{BUS TOPOLOGY}};
    \node [gtu block, minimum width=0.8cm] (B1) at (-2,2) {N1};
    \node [gtu block, minimum width=0.8cm] (B2) at (0,2) {N2};
    \node [gtu block, minimum width=0.8cm] (B3) at (2,2) {N3};
    \node [gtu block, minimum width=0.8cm] (B4) at (4,2) {N4};
    \draw [very thick] (-2.5,2) -- (4.5,2);
    
    % Star Topology
    \node at (0,-1) {\textbf{STAR TOPOLOGY}};
    \node [gtu block] (Hub) at (1,-2.5) {Hub/Switch};
    \node [gtu block, minimum width=0.8cm] (S1) at (-1,-2.5) {N1};
    \node [gtu block, minimum width=0.8cm] (S2) at (1,-4) {N2};
    \node [gtu block, minimum width=0.8cm] (S3) at (3,-2.5) {N3};
    \path [gtu arrow] (Hub) -- (S1);
    \path [gtu arrow] (Hub) -- (S2);
    \path [gtu arrow] (Hub) -- (S3);
\end{tikzpicture}
\captionof{figure}{Network Topologies}
\end{center}

\begin{itemize}
    \item \keyword{બસ ટોપોલોજી}: બધા ડિવાઇસ સિંગલ કેબલ સાથે જોડાયેલા
    \item \keyword{સ્ટાર ટોપોલોજી}: બધા ડિવાઇસ સેન્ટ્રલ હબ/સ્વિચ સાથે જોડાયેલા
    \item \keyword{રિંગ ટોપોલોજી}: ડિવાઇસ બંધ લૂપમાં જોડાયેલા
    \item \keyword{મેશ ટોપોલોજી}: દરેક ડિવાઇસ દરેક અન્ય ડિવાઇસ સાથે જોડાયેલું
    \item \keyword{ટ્રી ટોપોલોજી}: હાયરાર્કિકલ સ્ટાર નેટવર્ક્સ બસ વાયા કનેક્ટેડ
\end{itemize}
\end{solutionbox}

\begin{mnemonicbox}
\mnemonic{BSRMT: Better Solutions Require Multiple Topologies}
\end{mnemonicbox}

\questionmarks{1(ક OR)}{7}{TCP/IP પ્રોટોકોલ સ્યુટનો ડાયાગ્રામ દોરો અને એપ્લીકેશન લેયર, ટ્રાન્સપોર્ટ લેયર અને નેટવર્ક લેયરનું કાર્યપધ્ધતી સમજાવો.}

\begin{solutionbox}
TCP/IP પ્રોટોકોલ સ્યુટ નેટવર્ક કોમ્યુનિકેશનને ચાર કાર્યાત્મક સ્તરોમાં વ્યવસ્થિત કરે છે.

\textbf{આકૃતિ:}

\begin{center}
\begin{tikzpicture}[node distance=0cm]
    \node [gtu block, minimum width=8cm, minimum height=1.2cm] (App) {APPLICATION LAYER};
    \node [below=0.1cm of App, font=\small] {(HTTP, FTP, SMTP, DNS, TELNET)};
    \node [gtu block, minimum width=8cm, minimum height=1.2cm, below=0.5cm of App] (Trans) {TRANSPORT LAYER};
    \node [below=0.1cm of Trans, font=\small] {(TCP, UDP)};
    \node [gtu block, minimum width=8cm, minimum height=1.2cm, below=0.5cm of Trans] (Net) {INTERNET LAYER};
    \node [below=0.1cm of Net, font=\small] {(IP, ICMP, ARP, RARP)};
    \node [gtu block, minimum width=8cm, minimum height=1.2cm, below=0.5cm of Net] (Link) {NETWORK ACCESS LAYER};
    \node [below=0.1cm of Link, font=\small] {(Ethernet, Wi-Fi, Token Ring)};
\end{tikzpicture}
\captionof{figure}{TCP/IP Protocol Suite}
\end{center}

\begin{center}
\captionof{table}{TCP/IP Layer Functions}
\begin{tabulary}{\linewidth}{|L|L|L|}
\hline
\textbf{સ્તર} & \textbf{મુખ્ય કાર્ય} & \textbf{મુખ્ય પ્રોટોકોલ્સ} \\ \hline
એપ્લિકેશન & એપ્લિકેશન્સને નેટવર્ક સેવાઓ પ્રદાન કરે & HTTP, FTP, SMTP \\ \hline
ટ્રાન્સપોર્ટ & એન્ડ-ટુ-એન્ડ કોમ્યુનિકેશન, ડેટા ફ્લો કંટ્રોલ & TCP, UDP \\ \hline
ઈન્ટરનેટ (નેટવર્ક) & લોજિકલ એડ્રેસિંગ અને રાઉટિંગ & IP, ICMP, ARP \\ \hline
\end{tabulary}
\end{center}

\begin{itemize}
    \item \keyword{એપ્લિકેશન લેયર}: નેટવર્ક માટે યુઝર ઇન્ટરફેસ, એપ્લિકેશન-સ્પેસિફિક પ્રોટોકોલ્સ
    \item \keyword{ટ્રાન્સપોર્ટ લેયર}: વિશ્વસનીય ડેટા ટ્રાન્સમિશન, એરર રિકવરી, ફ્લો કંટ્રોલ
    \item \keyword{નેટવર્ક લેયર}: નેટવર્ક્સ વચ્ચે પેકેટ્સ રાઉટિંગ, IP એડ્રેસિંગ
\end{itemize}
\end{solutionbox}

\begin{mnemonicbox}
\mnemonic{ATN works: Application, Transport, Network સાથે મળીને કામ કરે છે}
\end{mnemonicbox}

\questionmarks{2(અ)}{3}{કનેક્શન ઓરિએન્ટેડ પ્રોટોકોલ અને કનેક્શન લેસ પ્રોટોકોલની સરખામણી કરો.}

\begin{solutionbox}
કનેક્શન-ઓરિએન્ટેડ અને કનેક્શનલેસ પ્રોટોકોલ્સ ડેટા ટ્રાન્સમિશનના હેન્ડલિંગમાં અલગ પડે છે.

\begin{center}
\captionof{table}{Connection-oriented vs Connectionless}
\begin{tabulary}{\linewidth}{|L|L|L|}
\hline
\textbf{ફીચર} & \textbf{કનેક્શન-ઓરિએન્ટેડ} & \textbf{કનેક્શનલેસ} \\ \hline
કનેક્શન & ટ્રાન્સમિશન પહેલા સ્થાપિત & કોઈ કનેક્શન સેટઅપ નહીં \\ \hline
વિશ્વસનીયતા & ગેરંટેડ ડિલિવરી & કોઈ ડિલિવરી ગેરંટી નહીં \\ \hline
એરર ચેકિંગ & વિસ્તૃત & મર્યાદિત અથવા કોઈ નહીં \\ \hline
ઉદાહરણ & TCP & UDP \\ \hline
ઉપયોગ & ફાઈલ ટ્રાન્સફર, વેબ બ્રાઉઝિંગ & સ્ટ્રીમિંગ, DNS લુકઅપ્સ \\ \hline
\end{tabulary}
\end{center}
\end{solutionbox}

\begin{mnemonicbox}
\mnemonic{REACH: Reliability Exists in All Connection Handshakes}
\end{mnemonicbox}

\questionmarks{2(બ)}{4}{ફાસ્ટ ઇથરનેટ અને ગીગાબાઈટ ઈથરનેટ સમજાવો.}

\begin{solutionbox}
ફાસ્ટ ઇથરનેટ અને ગીગાબિટ ઇથરનેટ મૂળ ઇથરનેટ સ્ટાન્ડર્ડના ઉચ્ચ-સ્પીડ વર્ઝન છે.

\begin{center}
\captionof{table}{Fast vs Gigabit Ethernet}
\begin{tabulary}{\linewidth}{|L|L|L|}
\hline
\textbf{ફીચર} & \textbf{ફાસ્ટ ઇથરનેટ} & \textbf{ગીગાબિટ ઇથરનેટ} \\ \hline
સ્પીડ & 100 Mbps & 1000 Mbps (1 Gbps) \\ \hline
IEEE સ્ટાન્ડર્ડ & 802.3u & 802.3z/802.3ab \\ \hline
કેબલ ટાઇપ & Cat5 UTP & Cat5e/Cat6 UTP, ફાઇબર \\ \hline
મેક્સ ડિસ્ટન્સ & 100m (કોપર) & 100m (કોપર), 5km (ફાઇબર) \\ \hline
\end{tabulary}
\end{center}

\begin{itemize}
    \item \keyword{ફાસ્ટ ઇથરનેટ}: ઓરિજિનલ 10Base-T ઇથરનેટથી 10x ઝડપી
    \item \keyword{ગીગાબિટ ઇથરનેટ}: ફાસ્ટ ઇથરનેટથી 10x ઝડપી, બેકવર્ડ કમ્પેટિબલ
    \item \keyword{કેબલિંગ}: વધુ સ્પીડ માટે ઉચ્ચ ગુણવત્તાવાળા કેબલિંગનો ઉપયોગ
    \item \keyword{એપ્લિકેશન્સ}: હાઈ-બેન્ડવિડ્થ નેટવર્ક બેકબોન્સ, સર્વર કનેક્શન્સ
\end{itemize}
\end{solutionbox}

\begin{mnemonicbox}
\mnemonic{Fast Gets Going: 100થી 1000 Mbps સુધીની પ્રગતિ}
\end{mnemonicbox}

\questionmarks{2(ક)}{7}{રાઉટર, હબ અને સ્વીચ વચ્ચેનો તફાવત આપો.}

\begin{solutionbox}
રાઉટર, હબ અને સ્વિચ અલગ-અલગ ક્ષમતાઓ અને કાર્યો ધરાવતા નેટવર્ક ડિવાઇસ છે.

\begin{center}
\captionof{table}{Router vs Hub vs Switch}
\begin{tabulary}{\linewidth}{|L|L|L|L|}
\hline
\textbf{ફીચર} & \textbf{રાઉટર} & \textbf{હબ} & \textbf{સ્વિચ} \\ \hline
OSI લેયર & નેટવર્ક (3) & ફિઝિકલ (1) & ડેટા લિંક (2) \\ \hline
કાર્ય & નેટવર્ક્સ કનેક્ટ કરે & ડિવાઇસ કનેક્ટ કરે & ડિવાઇસ કનેક્ટ કરે \\ \hline
ડેટા હેન્ડલિંગ & ઇન્ટેલિજન્ટ રાઉટિંગ & બધાને બ્રોડકાસ્ટ & ચોક્કસ ડિવાઇસને મોકલે \\ \hline
સિક્યોરિટી & ફાયરવોલ પ્રદાન કરે & કોઈ સિક્યોરિટી નહીં & બેઝિક ફિલ્ટરિંગ \\ \hline
એડ્રેસિંગ & IP એડ્રેસનો ઉપયોગ & કોઈ એડ્રેસિંગ નહીં & MAC એડ્રેસનો ઉપયોગ \\ \hline
\end{tabulary}
\end{center}

\textbf{આકૃતિ:}

\begin{center}
\begin{tikzpicture}[node distance=3cm]
    \node [gtu block, minimum width=2cm, minimum height=2cm, align=center] (R) {ROUTER\\[0.3cm]Routes\\between\\networks};
    \node [gtu block, minimum width=2cm, minimum height=2cm, align=center, right=of R] (H) {HUB\\[0.3cm]Shares\\signal\\to all\\ports};
    \node [gtu block, minimum width=2cm, minimum height=2cm, align=center, right=of H] (S) {SWITCH\\[0.3cm]Forwards\\to MAC\\address};
\end{tikzpicture}
\captionof{figure}{Network Devices Comparison}
\end{center}
\end{solutionbox}

\begin{mnemonicbox}
\mnemonic{RHS order: Router Has Smarts, Hub Shares Signal, Switch Sends Specifically}
\end{mnemonicbox}

\questionmarks{2(અ OR)}{3}{ઈ-મેઈલ સીસ્ટમની વ્યાખ્યા આપો અને ઈ-મેઈલનાં ઉપયોગો જણાવો.}

\begin{solutionbox}
ઈમેલ સિસ્ટમ એ નેટવર્ક સેવા છે જે યુઝર્સ વચ્ચે ડિજિટલ મેસેજનું આદાન-પ્રદાન કરવાની મંજૂરી આપે છે.

\begin{center}
\captionof{table}{Email System Components}
\begin{tabulary}{\linewidth}{|L|L|}
\hline
\textbf{કોમ્પોનન્ટ} & \textbf{કાર્ય} \\ \hline
મેઇલ યુઝર એજન્ટ (MUA) & એન્ડ-યુઝર્સ દ્વારા ઉપયોગમાં લેવાતા ઈમેઇલ ક્લાયન્ટ સોફ્ટવેર \\ \hline
મેઇલ ટ્રાન્સફર એજન્ટ (MTA) & ઈમેઇલ્સ ટ્રાન્સફર કરતું સર્વર સોફ્ટવેર \\ \hline
મેઇલ ડિલિવરી એજન્ટ (MDA) & પ્રાપ્તકર્તાના મેઇલબોક્સમાં ઈમેઇલ ડિલિવર કરે છે \\ \hline
પ્રોટોકોલ્સ & SMTP, POP3, IMAP \\ \hline
\end{tabulary}
\end{center}

\textbf{ઈમેઇલના ઉપયોગો:}
\begin{itemize}
    \item બિઝનેસ કોમ્યુનિકેશન
    \item પર્સનલ મેસેજિંગ
    \item ફાઇલ શેરિંગ
    \item માર્કેટિંગ અને ન્યૂઝલેટર્સ
    \item નોટિફિકેશન્સ અને એલર્ટ્સ
\end{itemize}
\end{solutionbox}

\begin{mnemonicbox}
\mnemonic{BCPFN: Business Communication, Personal, Files, Newsletters}
\end{mnemonicbox}

\questionmarks{2(બ OR)}{4}{IPv4 અને IPv6નો તફાવત આપો.}

\begin{solutionbox}
IPv4 અને IPv6 ઇન્ટરનેટ પ્રોટોકોલ વર્ઝન્સ છે જેમાં નોંધપાત્ર તફાવતો છે.

\begin{center}
\captionof{table}{IPv4 vs IPv6}
\begin{tabulary}{\linewidth}{|L|L|L|}
\hline
\textbf{ફીચર} & \textbf{IPv4} & \textbf{IPv6} \\ \hline
એડ્રેસ લંબાઈ & 32-બિટ (4 બાઇટ્સ) & 128-બિટ (16 બાઇટ્સ) \\ \hline
ફોર્મેટ & ડોટેડ ડેસિમલ (192.168.1.1) & હેક્સાડેસિમલ વિથ કોલન્સ \\ \hline
એડ્રેસ સ્પેસ & \textasciitilde4.3 બિલિયન એડ્રેસ & 340 અન્ડેસિલિયન એડ્રેસ \\ \hline
સિક્યોરિટી & સિક્યોરિટી પછીથી ઉમેરાયેલી & બિલ્ટ-ઇન IPSec \\ \hline
કોન્ફિગરેશન & મેન્યુઅલ અથવા DHCP & સ્ટેટલેસ ઓટો-કોન્ફિગરેશન \\ \hline
\end{tabulary}
\end{center}

\begin{itemize}
    \item \keyword{IPv4}: મર્યાદિત સ્પેસ સાથે પરંપરાગત એડ્રેસિંગ
    \item \keyword{IPv6}: વિશાળ ક્ષમતા સાથે આગામી-પેઢી એડ્રેસિંગ
    \item \keyword{ટ્રાન્ઝિશન}: ડ્યુઅલ-સ્ટેક, ટનલિંગ અને ટ્રાન્સલેશન મેકેનિઝમ્સ
\end{itemize}
\end{solutionbox}

\begin{mnemonicbox}
\mnemonic{4 SMALL, 6 HUGE: IPv4 નાનો એડ્રેસ સ્પેસ, IPv6 વિશાળ એડ્રેસ સ્પેસ}
\end{mnemonicbox}

\questionmarks{2(ક OR)}{7}{નેટવર્કમાં ફાયરવોલ સાથે કોન્સેપ્ટ, પ્રિન્સીપલ, લીમીટેશન, trusted system, Kerberos-conceptની ચર્ચા કરો.}

\begin{solutionbox}
ફાયરવોલ્સ ક્રિટિકલ નેટવર્ક સિક્યોરિટી સિસ્ટમ્સ છે જે ઇનકમિંગ અને આઉટગોઇંગ ટ્રાફિકને મોનિટર અને કંટ્રોલ કરે છે.

\textbf{ફાયરવોલના સિદ્ધાંતો:}
\begin{itemize}
    \item \keyword{ડિફોલ્ટ ડિનાય}: સ્પષ્ટપણે મંજૂર ન હોય ત્યાં સુધી બધું બ્લોક કરો
    \item \keyword{ડિફેન્સ ઇન ડેપ્થ}: મલ્ટિપલ સિક્યોરિટી લેયર્સ
    \item \keyword{લીસ્ટ પ્રિવિલેજ}: ન્યૂનતમ જરૂરી એક્સેસ
\end{itemize}

\textbf{મર્યાદાઓ:}
\begin{itemize}
    \item અધિકૃત યુઝર્સ સામે રક્ષણ આપી શકતું નથી
    \item એન્ક્રિપ્ટેડ મેલિશિયસ ટ્રાફિક સામે મર્યાદિત
    \item નેટવર્ક પરફોર્મન્સ પર અસર
\end{itemize}

\textbf{કર્બેરોસ કોન્સેપ્ટ:}

\begin{center}
\begin{tikzpicture}[node distance=5cm]
    \node [gtu block] (client) {CLIENT};
    \node [gtu block, right=of client] (kdc) {KDC};
    \node [gtu block, right=of kdc] (server) {SERVER};
    
    \path [gtu arrow, bend left=20] (client) -- node[above, font=\tiny] {1. Request TGT} (kdc);
    \path [gtu arrow, bend left=20] (kdc) -- node[below, font=\tiny] {2. TGT} (client);
    \path [gtu arrow, bend left=20] (client) -- node[above, font=\tiny] {3. Request Service Ticket} (kdc);
    \path [gtu arrow, bend left=20] (kdc) -- node[below, font=\tiny] {4. Service Ticket} (client);
    \path [gtu arrow] (client) -- node[above, font=\tiny] {5. Service Request + Ticket} (server);
    \path [gtu arrow] (server) -- node[below, font=\tiny] {6. Session Key} (client);
\end{tikzpicture}
\captionof{figure}{Kerberos Authentication}
\end{center}

\begin{itemize}
    \item ટ્રસ્ટેડ થર્ડ પાર્ટીનો ઉપયોગ કરતો \keyword{ઓથેન્ટિકેશન પ્રોટોકોલ}
    \item \keyword{ટિકિટ-આધારિત} એક્સેસ કંટ્રોલ સિસ્ટમ
    \item ક્લાયન્ટ અને સર્વર વચ્ચે \keyword{મ્યુચ્યુઅલ ઓથેન્ટિકેશન}
    \item રિપ્લે એટેક્સને રોકવા માટે \keyword{સમય-સંવેદનશીલ} ટિકિટ્સ
\end{itemize}
\end{solutionbox}

\begin{mnemonicbox}
\mnemonic{FLASK: Firewalls Lock Access, Secure with Kerberos}
\end{mnemonicbox}

\questionmarks{3(અ)}{3}{ડેટા લિંક લેયરના સબ લેયર્સ સમજાવો.}

\begin{solutionbox}
OSI મોડેલમાં ડેટા લિંક લેયર બે અલગ-અલગ કાર્યો સાથે બે સબલેયર્સમાં વિભાજિત છે.

\begin{center}
\captionof{table}{Data Link Sublayers}
\begin{tabulary}{\linewidth}{|L|L|L|}
\hline
\textbf{સબલેયર} & \textbf{કાર્ય} & \textbf{સ્ટાન્ડર્ડ્સ} \\ \hline
લોજિકલ લિંક કંટ્રોલ (LLC) & ફ્લો કંટ્રોલ, એરર ચેકિંગ & IEEE 802.2 \\ \hline
મીડિયા એક્સેસ કંટ્રોલ (MAC) & ચેનલ એક્સેસ, એડ્રેસિંગ & IEEE 802.3, 802.11 \\ \hline
\end{tabulary}
\end{center}

\textbf{આકૃતિ:}

\begin{center}
\begin{tikzpicture}[node distance=0cm]
    \node [gtu block, minimum width=8cm, minimum height=1cm] (net) {NETWORK LAYER};
    \node [gtu block, minimum width=8cm, minimum height=1.2cm, below=0.2cm of net] (llc) {LOGICAL LINK CONTROL (LLC - 802.2)};
    \node [below=0.1cm of llc, font=\small] {Flow control, Error handling};
    \node [gtu block, minimum width=8cm, minimum height=1.2cm, below=0.5cm of llc] (mac) {MEDIA ACCESS CONTROL (MAC - 802.3, 802.11)};
    \node [below=0.1cm of mac, font=\small] {MAC addressing, Channel access};
    \node [gtu block, minimum width=8cm, minimum height=1cm, below=0.5cm of mac] (phy) {PHYSICAL LAYER};
\end{tikzpicture}
\captionof{figure}{Data Link Layer Sublayers}
\end{center}

\begin{itemize}
    \item \keyword{LLC}: નેટવર્ક લેયર માટે ઇન્ટરફેસ પ્રદાન કરે છે, એરર/ફ્લો કંટ્રોલ
    \item \keyword{MAC}: ફિઝિકલ એડ્રેસિંગ અને મીડિયા એક્સેસનું સંચાલન કરે છે
\end{itemize}
\end{solutionbox}

\begin{mnemonicbox}
\mnemonic{MAC LLCs order: MAC Lower Layer handles, LLC Higher Coordinates}
\end{mnemonicbox}

\questionmarks{3(બ)}{4}{IP layer protocols વિસ્તૃતમાં સમજાવો.}

\begin{solutionbox}
IP લેયરમાં કેટલાક મહત્વપૂર્ણ પ્રોટોકોલ્સ છે જે ઇન્ટરનેટવર્ક કોમ્યુનિકેશનમાં સાથે મળીને કામ કરે છે.

\begin{center}
\captionof{table}{IP Layer Protocols}
\begin{tabulary}{\linewidth}{|L|L|L|}
\hline
\textbf{પ્રોટોકોલ} & \textbf{કાર્ય} & \textbf{મુખ્ય ફીચર્સ} \\ \hline
IP & બેઝિક ડેટાગ્રામ ડિલિવરી & એડ્રેસિંગ, ફ્રેગમેન્ટેશન, TTL \\ \hline
ICMP & નેટવર્ક ડાયગ્નોસ્ટિક્સ & એરર રિપોર્ટિંગ, પિંગ, ટ્રેસરાઉટ \\ \hline
ARP & એડ્રેસ રિઝોલ્યુશન & IP થી MAC એડ્રેસ મેપિંગ \\ \hline
RARP & રિવર્સ એડ્રેસ રિઝોલ્યુશન & MAC થી IP એડ્રેસ મેપિંગ \\ \hline
\end{tabulary}
\end{center}

\begin{itemize}
    \item \keyword{IP}: એડ્રેસિંગ અને પેકેટ્સ રાઉટિંગ માટે કોર પ્રોટોકોલ
    \item \keyword{ICMP}: એરર મેસેજ અને ઓપરેશનલ ઇન્ફોર્મેશન
    \item \keyword{ARP/RARP}: લેયર્સ વચ્ચે એડ્રેસ ટ્રાન્સલેશન
\end{itemize}
\end{solutionbox}

\begin{mnemonicbox}
\mnemonic{I PAIR-up: IP, ICMP, ARP, RARP એક ટીમ તરીકે કામ કરે છે}
\end{mnemonicbox}

\questionmarks{3(ક)}{7}{વિવિધ પ્રકારની IP એડ્રેસિંગ સ્કીમનું વર્ણન કરો અને ક્લાસફુલ IP એડ્રેસિંગમાં વિવિધ વર્ગોને ઉદાહરણ સાથે સમજાવો.}

\begin{solutionbox}
IP એડ્રેસિંગ સ્કીમ્સ IP એડ્રેસના ફાળવણી અને સ્ટ્રક્ચરને વ્યાખ્યાયિત કરે છે.

\textbf{ક્લાસફુલ IP એડ્રેસિંગ:}

\begin{center}
\captionof{table}{Classful IP Addressing}
\begin{tabulary}{\linewidth}{|L|L|L|L|L|}
\hline
\textbf{ક્લાસ} & \textbf{પ્રથમ બાઇટ રેન્જ} & \textbf{ડિફોલ્ટ સબનેટ માસ્ક} & \textbf{ઉદાહરણ} & \textbf{હોસ્ટ્સ/નેટવર્ક} \\ \hline
A & 1-127 & 255.0.0.0 (/8) & 10.52.36.12 & 16,777,214 \\ \hline
B & 128-191 & 255.255.0.0 (/16) & 172.16.52.63 & 65,534 \\ \hline
C & 192-223 & 255.255.255.0 (/24) & 192.168.10.15 & 254 \\ \hline
D & 224-239 & N/A (મલ્ટિકાસ્ટ) & 224.0.0.5 & N/A \\ \hline
E & 240-255 & N/A (એક્સપેરિમેન્ટલ) & 240.0.0.1 & N/A \\ \hline
\end{tabulary}
\end{center}

\begin{itemize}
    \item \keyword{ક્લાસ A}: મોટી સંસ્થાઓ, હોસ્ટ્સની વિશાળ સંખ્યા
    \item \keyword{ક્લાસ B}: મધ્યમ કદની સંસ્થાઓ
    \item \keyword{ક્લાસ C}: ઓછા હોસ્ટ્સ સાથેના નાના નેટવર્ક્સ
    \item \keyword{ક્લાસ D}: મલ્ટિકાસ્ટ ગ્રુપ્સ
    \item \keyword{ક્લાસ E}: પ્રાયોગિક ઉપયોગ માટે અનામત
\end{itemize}
\end{solutionbox}

\begin{mnemonicbox}
\mnemonic{All Businesses Care During Exams: ક્લાસ A, B, C, D, E}
\end{mnemonicbox}

\questionmarks{3(અ OR)}{3}{ડીજીટલ સબસ્કાઈબર લાઈન ટેકનોલોજી સમજાવો.}

\begin{solutionbox}
ડિજિટલ સબસ્ક્રાઇબર લાઇન (DSL) એ ટેલિફોન લાઇન્સ પર ડિજિટલ ડેટા ટ્રાન્સમિશન પ્રદાન કરતી ટેકનોલોજી છે.

\begin{center}
\captionof{table}{DSL Types}
\begin{tabulary}{\linewidth}{|L|L|L|L|}
\hline
\textbf{DSL ટાઇપ} & \textbf{સ્પીડ (ડાઉન/અપ)} & \textbf{ડિસ્ટન્સ} & \textbf{એપ્લિકેશન} \\ \hline
ADSL & 8 Mbps/1 Mbps & 5.5 km સુધી & હોમ ઇન્ટરનેટ \\ \hline
SDSL & 2 Mbps/2 Mbps & 3 km સુધી & બિઝનેસ \\ \hline
VDSL & 52 Mbps/16 Mbps & 1.2 km સુધી & વિડીયો સ્ટ્રીમિંગ \\ \hline
HDSL & 2 Mbps/2 Mbps & 3.6 km સુધી & T1/E1 રિપ્લેસમેન્ટ \\ \hline
\end{tabulary}
\end{center}

\textbf{આકૃતિ:}

\begin{center}
\begin{tikzpicture}[node distance=3cm]
    \node [gtu block] (home) {HOME};
    \node [gtu block, right=of home] (modem) {DSL MODEM};
    \node [gtu block, right=of modem] (dslam) {DSLAM};
    \node [gtu block, right=of dslam] (internet) {INTERNET};
    
    \path [gtu arrow] (home) -- (modem);
    \path [gtu arrow] (modem) -- node[above, font=\small] {Copper Line (POTS)} (dslam);
    \path [gtu arrow] (dslam) -- node[above, font=\small] {ISP} (internet);
\end{tikzpicture}
\captionof{figure}{DSL System}
\end{center}

\begin{itemize}
    \item \keyword{સ્પેક્ટ્રમ ઉપયોગ}: અવાજ કરતાં ઉચ્ચ ફ્રિક્વન્સીનો ઉપયોગ
    \item \keyword{ઓલવેઝ-ઓન}: સતત કનેક્શન, ડાયલ-અપ નહીં
    \item \keyword{xDSL}: અલગ-અલગ ક્ષમતાઓ સાથે ટેકનોલોજીનો પરિવાર
\end{itemize}
\end{solutionbox}

\begin{mnemonicbox}
\mnemonic{SAVE Bandwidth: SDSL, ADSL, VDSL, HDSL બેન્ડવિડ્થ ઓપ્શન્સ}
\end{mnemonicbox}

\questionmarks{3(બ OR)}{4}{કેબલ મોડેમ સીસ્ટમને ચર્ચા કરો.}

\begin{solutionbox}
કેબલ મોડેમ સિસ્ટમ કેબલ ટીવી માટે વપરાતા એજ કોએક્સિયલ કેબલ દ્વારા ઇન્ટરનેટ એક્સેસ પ્રદાન કરે છે.

\begin{center}
\captionof{table}{Cable Modem Components}
\begin{tabulary}{\linewidth}{|L|L|}
\hline
\textbf{કોમ્પોનન્ટ} & \textbf{કાર્ય} \\ \hline
કેબલ મોડેમ & ડિજિટલ સિગ્નલ્સ કન્વર્ટ કરતું યુઝર-એન્ડ ડિવાઇસ \\ \hline
CMTS & પ્રોવાઇડર એન્ડ પર કેબલ મોડેમ ટર્મિનેશન સિસ્ટમ \\ \hline
HFC & હાઇબ્રિડ ફાઇબર-કોએક્સિયલ નેટવર્ક ઇન્ફ્રાસ્ટ્રક્ચર \\ \hline
DOCSIS & ડેટા ઓવર કેબલ સર્વિસ ઇન્ટરફેસ સ્પેસિફિકેશન \\ \hline
\end{tabulary}
\end{center}

\textbf{આકૃતિ:}

\begin{center}
\begin{tikzpicture}[node distance=3cm]
    \node [gtu block] (home) {HOME MODEM};
    \node [gtu block, right=of home] (node) {NODE};
    \node [gtu block, right=of node] (cmts) {ISP CMTS};
    \node [gtu block, right=of cmts] (internet) {INTERNET};
    
    \path [gtu arrow] (home) -- node[above, font=\small] {COAX} (node);
    \path [gtu arrow] (node) -- node[above, font=\small] {FIBER} (cmts);
    \path [gtu arrow] (cmts) -- (internet);
    
    \node [below=0.3cm of home, font=\small] {NEIGHBORHOOD};
    \node [below=0.3cm of cmts, font=\small] {HEAD-END};
\end{tikzpicture}
\captionof{figure}{Cable Modem System}
\end{center}

\begin{itemize}
    \item \keyword{શેર્ડ મીડિયમ}: નેબરહુડ બેન્ડવિડ્થ શેર કરે છે
    \item \keyword{એસિમેટ્રિક}: સામાન્ય રીતે અપલોડ કરતાં ડાઉનલોડ ઝડપી
    \item \keyword{DOCSIS સ્ટાન્ડર્ડ્સ}: સ્પીડ/ફીચર્સ માટે વિકસિત થતાં સ્પેસિફિકેશન્સ
\end{itemize}
\end{solutionbox}

\begin{mnemonicbox}
\mnemonic{CHAMPS: Cable, HFC, Access, Modem, Provider, Shared}
\end{mnemonicbox}

\questionmarks{3(ક OR)}{7}{સંક્ષિપ્તમાં તમામ ટ્રાન્સમિશન મીડિયાનું વર્ણન કરો.}

\begin{solutionbox}
ટ્રાન્સમિશન મીડિયા એ ભૌતિક પાથ છે જેના દ્વારા નેટવર્કમાં ડેટા પ્રવાસ કરે છે.

\begin{center}
\captionof{table}{Transmission Media}
\begin{tabulary}{\linewidth}{|L|L|L|L|}
\hline
\textbf{મીડિયમ ટાઇપ} & \textbf{મેક્સ ડિસ્ટન્સ} & \textbf{મેક્સ બેન્ડવિડ્થ} & \textbf{એપ્લિકેશન} \\ \hline
\multicolumn{4}{|c|}{\textbf{ગાઇડેડ (વાયર્ડ)}} \\ \hline
ટ્વિસ્ટેડ પેર & 100m & 10 Gbps & ઓફિસ LANs \\ \hline
કોએક્સિયલ કેબલ & 500m & 10 Gbps & કેબલ TV, ઇન્ટરનેટ \\ \hline
ફાઇબર ઓપ્ટિક & 100km+ & 100+ Tbps & બેકબોન્સ, લોંગ-ડિસ્ટન્સ \\ \hline
\multicolumn{4}{|c|}{\textbf{અનગાઇડેડ (વાયરલેસ)}} \\ \hline
રેડિયો વેવ્સ & 100m-50km & 600 Mbps & વાયરલેસ નેટવર્ક્સ \\ \hline
માઇક્રોવેવ્સ & લાઇન ઓફ સાઇટ & 10 Gbps & પોઇન્ટ-ટુ-પોઇન્ટ લિંક્સ \\ \hline
ઇન્ફ્રારેડ & 1m & 16 Mbps & રિમોટ કંટ્રોલ્સ \\ \hline
\end{tabulary}
\end{center}

\begin{itemize}
    \item \keyword{ગાઇડેડ મીડિયા}: સિગ્નલ્સને સીમિત કરતા ભૌતિક પાથ
    \item \keyword{અનગાઇડેડ મીડિયા}: હવા/શૂન્યાવકાશ દ્વારા વાયરલેસ ટ્રાન્સમિશન
\end{itemize}
\end{solutionbox}

\begin{mnemonicbox}
\mnemonic{TRIM-CWF: Twisted, Radio, Infrared, Microwave, Coaxial, Wireless, Fiber}
\end{mnemonicbox}

\questionmarks{4(અ)}{3}{DNS પર નોંધ લખો.}

\begin{solutionbox}
ડોમેન નેમ સિસ્ટમ (DNS) માનવ-મૈત્રીપૂર્ણ ડોમેન નેમ્સને IP એડ્રેસમાં અનુવાદિત કરે છે.

\begin{center}
\captionof{table}{DNS Components}
\begin{tabulary}{\linewidth}{|L|L|}
\hline
\textbf{કોમ્પોનન્ટ} & \textbf{કાર્ય} \\ \hline
ડોમેન નેમ & હાયરાર્કિકલ, વાંચી શકાય તેવું એડ્રેસ (www.example.com) \\ \hline
DNS સર્વર & ડોમેન નેમ્સને IP એડ્રેસમાં રિઝોલ્વ કરે છે \\ \hline
રૂટ સર્વર & DNS હાયરાર્કીનો ટોપ, TLDs તરફ પોઇન્ટ કરે છે \\ \hline
TLD સર્વર & ટોપ-લેવલ ડોમેન્સ (.com, .org) મેનેજ કરે છે \\ \hline
\end{tabulary}
\end{center}

\textbf{આકૃતિ:}

\begin{center}
\begin{tikzpicture}[node distance=2.5cm]
    \node [gtu block] (client) {CLIENT};
    \node [gtu block, right=of client] (local) {LOCAL DNS};
    \node [gtu block, above right=1cm and 2cm of local] (root) {ROOT DNS};
    \node [gtu block, below right=1cm and 2cm of local] (tld) {TLD DNS};
    \node [gtu block, right=of tld] (auth) {AUTHORITATIVE DNS};
    
    \path [gtu arrow] (client) -- node[above, font=\tiny] {1. Query} (local);
    \path [gtu arrow] (local) -- node[above, font=\tiny] {2} (root);
    \path [gtu arrow] (root) -- node[above, font=\tiny] {3} (local);
    \path [gtu arrow] (local) -- node[above, font=\tiny] {4} (tld);
    \path [gtu arrow] (tld) -- node[above, font=\tiny] {5} (local);
    \path [gtu arrow] (local) -- node[below, font=\tiny] {6} (auth);
    \path [gtu arrow] (auth) -- node[below, font=\tiny] {7. IP} (local);
    \path [gtu arrow] (local) -- node[below, font=\tiny] {8. Response} (client);
\end{tikzpicture}
\captionof{figure}{DNS Resolution Process}
\end{center}

\begin{itemize}
    \item \keyword{ડિસ્ટ્રિબ્યુટેડ ડેટાબેઝ}: હાયરાર્કિકલ, ગ્લોબલી ડિસ્ટ્રિબ્યુટેડ
    \item \keyword{કેશિંગ}: પરફોર્મન્સ સુધારે છે, લોડ ઘટાડે છે
    \item \keyword{ક્રિટિકલ ઇન્ફ્રાસ્ટ્રક્ચર}: ઇન્ટરનેટ ફંક્શનાલિટી માટે આવશ્યક
\end{itemize}
\end{solutionbox}

\begin{mnemonicbox}
\mnemonic{DIRT: Domain names Into Routable TCP/IP}
\end{mnemonicbox}

\questionmarks{4(બ)}{4}{ફાઇલ ટ્રાન્સફર પ્રોટોકોલ સમજાવો.}

\begin{solutionbox}
ફાઇલ ટ્રાન્સફર પ્રોટોકોલ (FTP) નેટવર્ક પર ક્લાયન્ટ અને સર્વર વચ્ચે ફાઇલ્સના ટ્રાન્સફરને સક્ષમ બનાવે છે.

\begin{center}
\captionof{table}{FTP Features}
\begin{tabulary}{\linewidth}{|L|L|}
\hline
\textbf{ફીચર} & \textbf{વર્ણન} \\ \hline
પોર્ટ & કંટ્રોલ: 21, ડેટા: 20 \\ \hline
મોડ & એક્ટિવ અને પેસિવ \\ \hline
સિક્યોરિટી & બેઝિક (ક્લિયર ટેક્સ્ટ), અથવા એન્ક્રિપ્શન માટે FTPS/SFTP \\ \hline
કમાન્ડ્સ & GET, PUT, LIST, DELETE, વગેરે \\ \hline
\end{tabulary}
\end{center}

\textbf{આકૃતિ:}

\begin{center}
\begin{tikzpicture}[node distance=4cm]
    \node [gtu block] (client) {CLIENT};
    \node [gtu block, right=of client] (server) {SERVER};
    
    \path [gtu arrow] (client) -- node[above, font=\small] {Control Connection (Port 21)} (server);
    \path [gtu arrow, bend right=30] (client) -- node[below, font=\small] {Data Connection (Port 20)} (server);
\end{tikzpicture}
\captionof{figure}{FTP Dual Channel}
\end{center}

\begin{itemize}
    \item \keyword{ડ્યુઅલ ચેનલ}: કંટ્રોલ ચેનલ અને ડેટા ચેનલ
    \item \keyword{ઓથેન્ટિકેશન}: યુઝરનેમ/પાસવર્ડ જરૂરી
    \item \keyword{મોડ્સ}: ASCII (ટેક્સ્ટ) અથવા બાઇનરી (રો ડેટા)
\end{itemize}
\end{solutionbox}

\begin{mnemonicbox}
\mnemonic{CAPS: Control And Port Separation}
\end{mnemonicbox}

\questionmarks{4(ક)}{7}{વિવિધ ઇન્ટરનેટ સેવાઓનું વર્ગીકરણ કરો અને વિગતવાર સમજાવો.}

\begin{solutionbox}
ઇન્ટરનેટ સેવાઓ નેટવર્ક પર વિવિધ કાર્યક્ષમતા પ્રદાન કરે છે.

\begin{center}
\captionof{table}{Internet Services}
\begin{tabulary}{\linewidth}{|L|L|L|}
\hline
\textbf{સેવા કેટેગરી} & \textbf{સામાન્ય પ્રોટોકોલ્સ} & \textbf{એપ્લિકેશન ઉદાહરણો} \\ \hline
કોમ્યુનિકેશન & SMTP, POP3, IMAP & ઇમેઇલ, ઇન્સ્ટન્ટ મેસેજિંગ \\ \hline
ઇન્ફોર્મેશન એક્સેસ & HTTP, HTTPS & વર્લ્ડ વાઇડ વેબ, પોર્ટલ્સ \\ \hline
ફાઇલ શેરિંગ & FTP, BitTorrent, SMB & ફાઇલ હોસ્ટિંગ, P2P શેરિંગ \\ \hline
રિમોટ એક્સેસ & SSH, Telnet, RDP & રિમોટ એડમિનિસ્ટ્રેશન \\ \hline
રિયલ-ટાઇમ સર્વિસિસ & VoIP, WebRTC & વિડિયો કોન્ફરન્સિંગ, VoIP \\ \hline
\end{tabulary}
\end{center}
\end{solutionbox}

\begin{mnemonicbox}
\mnemonic{CIFRR: Communication, Information, File, Remote, Real-time}
\end{mnemonicbox}

\questionmarks{4(અ OR)}{3}{મેઇલ પ્રોટોકોલ્સ સમજાવો.}

\begin{solutionbox}
મેઇલ પ્રોટોકોલ્સ વપરાશકર્તાઓ વચ્ચે ઇલેક્ટ્રોનિક મેસેજિંગ સરળ બનાવે છે.

\begin{center}
\captionof{table}{Mail Protocols}
\begin{tabulary}{\linewidth}{|L|L|L|L|}
\hline
\textbf{પ્રોટોકોલ} & \textbf{કાર્ય} & \textbf{પોર્ટ} & \textbf{દિશા} \\ \hline
SMTP & સિમ્પલ મેઇલ ટ્રાન્સફર પ્રોટોકોલ & 25, 587 & મેઇલ મોકલવું \\ \hline
POP3 & પોસ્ટ ઓફિસ પ્રોટોકોલ v3 & 110 & મેઇલ પ્રાપ્ત કરવું \\ \hline
IMAP & ઇન્ટરનેટ મેસેજ એક્સેસ પ્રોટોકોલ & 143 & એડવાન્સ્ડ મેઇલ રિટ્રિવલ \\ \hline
\end{tabulary}
\end{center}

\begin{itemize}
    \item \keyword{SMTP}: આઉટગોઇંગ મેઇલ ડિલિવરી, પુશ પ્રોટોકોલ
    \item \keyword{POP3}: સરળ મેઇલ રિટ્રિવલ, ડાઉનલોડ અને ડિલીટ કરે છે
    \item \keyword{IMAP}: એડવાન્સ્ડ રિટ્રિવલ, સર્વર-સાઇડ સ્ટોરેજ, ફોલ્ડર્સ
\end{itemize}
\end{solutionbox}

\begin{mnemonicbox}
\mnemonic{SIM-P: SMTP Sends, IMAP Manages, POP3 Pulls}
\end{mnemonicbox}

\questionmarks{4(બ OR)}{4}{સંક્ષિપ્તમાં VOIP નું વર્ણન કરો.}

\begin{solutionbox}
વોઇસ ઓવર ઇન્ટરનેટ પ્રોટોકોલ (VoIP) IP નેટવર્ક્સ પર વોઇસ કોમ્યુનિકેશન ટ્રાન્સમિટ કરે છે.

\begin{center}
\captionof{table}{VoIP Components}
\begin{tabulary}{\linewidth}{|L|L|}
\hline
\textbf{કોમ્પોનન્ટ} & \textbf{કાર્ય} \\ \hline
કોડેક & વોઇસ સિગ્નલ્સ એન્કોડ/ડિકોડ કરે છે \\ \hline
સિગ્નલિંગ પ્રોટોકોલ & કોલ સેટઅપ/ટિયરડાઉન (SIP, H.323) \\ \hline
ટ્રાન્સપોર્ટ પ્રોટોકોલ & વોઇસ પેકેટ ડિલિવરી (RTP) \\ \hline
QoS મેકેનિઝમ & વોઇસ ક્વોલિટી સુનિશ્ચિત કરે છે \\ \hline
\end{tabulary}
\end{center}

\textbf{આકૃતિ:}

\begin{center}
\begin{tikzpicture}[node distance=4cm]
    \node [gtu block, align=center] (caller) {CALLER\\ENDPOINT};
    \node [gtu block, align=center, right=of caller] (callee) {CALLEE\\ENDPOINT};
    
    \path [gtu arrow] (caller) -- node[above, font=\small] {Internet/IP Network} (callee);
    \path [gtu arrow, bend right=20] (caller) -- node[below, font=\small] {RTP (Voice Packets)} (callee);
    
    \node [below=0.5cm of caller, font=\small] {[Analog] $\rightarrow$ [Digital] $\rightarrow$ [Packets]};
    \node [below=0.5cm of callee, font=\small] {[Packets] $\rightarrow$ [Digital] $\rightarrow$ [Analog]};
\end{tikzpicture}
\captionof{figure}{VoIP Communication}
\end{center}

\begin{itemize}
    \item \keyword{પેકેટાઇઝેશન}: એનાલોગ વોઇસને ડિજિટલ પેકેટ્સમાં કન્વર્ટ કરે છે
    \item \keyword{લાભો}: કોસ્ટ સેવિંગ્સ, ફ્લેક્સિબિલિટી, એપ્સ સાથે ઇન્ટિગ્રેશન
    \item \keyword{ચેલેન્જીસ}: ક્વોલિટી ઓફ સર્વિસ, લેટન્સી, જિટર, પેકેટ લોસ
\end{itemize}
\end{solutionbox}

\begin{mnemonicbox}
\mnemonic{PALS: Packets Allowing Live Speech}
\end{mnemonicbox}

\questionmarks{4(ક OR)}{7}{TCP અને UDP પ્રોટોકોલ્સનું વર્ણન કરો.}

\begin{solutionbox}
TCP અને UDP TCP/IP સ્યુટમાં પ્રાથમિક ટ્રાન્સપોર્ટ લેયર પ્રોટોકોલ્સ છે.

\begin{center}
\captionof{table}{TCP vs UDP}
\begin{tabulary}{\linewidth}{|L|L|L|}
\hline
\textbf{ફીચર} & \textbf{TCP} & \textbf{UDP} \\ \hline
કનેક્શન & કનેક્શન-ઓરિએન્ટેડ & કનેક્શનલેસ \\ \hline
વિશ્વસનીયતા & ગેરંટેડ ડિલિવરી & બેસ્ટ-એફર્ટ ડિલિવરી \\ \hline
હેડર સાઇઝ & 20-60 બાઇટ્સ & 8 બાઇટ્સ \\ \hline
સ્પીડ & ઓવરહેડને કારણે ધીમું & મિનિમલ ઓવરહેડ સાથે ઝડપી \\ \hline
ઉપયોગ & વેબ, ઇમેઇલ, ફાઇલ ટ્રાન્સફર & સ્ટ્રીમિંગ, DNS, VoIP \\ \hline
\end{tabulary}
\end{center}

\textbf{TCP ફીચર્સ:}
\begin{itemize}
    \item \keyword{વિશ્વસનીયતા}: એક્નોલેજમેન્ટ્સ, રિટ્રાન્સમિશન
    \item \keyword{ફ્લો કંટ્રોલ}: વિન્ડો-બેઝ્ડ, ઓવરવ્હેલ્મિંગને રોકે છે
    \item \keyword{કન્જેશન કંટ્રોલ}: સ્લો સ્ટાર્ટ, કન્જેશન અવોઇડન્સ
\end{itemize}

\textbf{TCP થ્રી-વે હેન્ડશેક:}

\begin{center}
\begin{tikzpicture}[node distance=5cm]
    \node [gtu block] (client) {CLIENT};
    \node [gtu block, right=of client] (server) {SERVER};
    
    \path [gtu arrow] (client) -- node[above, font=\small, pos=0.3] {1. SYN} (server);
    \path [gtu arrow] (server) -- node[above, font=\small, pos=0.7] {2. SYN-ACK} (client);
    \path [gtu arrow] (client) -- node[below, font=\small, pos=0.3] {3. ACK} (server);
    
    \node [below=1.5cm of client, font=\small] {Connection Established};
    \node [below=1.5cm of server, font=\small] {Data Transfer Begins};
\end{tikzpicture}
\captionof{figure}{TCP Three-Way Handshake}
\end{center}

\textbf{UDP ફીચર્સ:}
\begin{itemize}
    \item \keyword{લાઇટવેઇટ}: મિનિમલ હેડર્સ, કોઈ કનેક્શન સ્ટેટ નહીં
    \item \keyword{લો લેટન્સી}: કોઈ હેન્ડશેકિંગ કે એક્નોલેજમેન્ટ્સ નહીં
    \item \keyword{બ્રોડકાસ્ટ/મલ્ટિકાસ્ટ}: વન-ટુ-મેની ટ્રાન્સમિશનને સપોર્ટ કરે છે
\end{itemize}
\end{solutionbox}

\begin{mnemonicbox}
\mnemonic{CRUFS: Connection, Reliability, UDP Fast, Simple}
\end{mnemonicbox}

\questionmarks{5(અ)}{3}{ક્રિપ્ટોગ્રાફીનું વર્ણન કરો.}

\begin{solutionbox}
ક્રિપ્ટોગ્રાફી એ માહિતીનું રક્ષણ કરતી સુરક્ષિત કોમ્યુનિકેશન ટેકનિક્સનું વિજ્ઞાન છે.

\begin{center}
\captionof{table}{Cryptography Types}
\begin{tabulary}{\linewidth}{|L|L|L|}
\hline
\textbf{ટાઇપ} & \textbf{વર્ણન} & \textbf{ઉદાહરણ} \\ \hline
સિમેટ્રિક & એન્ક્રિપ્શન અને ડિક્રિપ્શન માટે એક જ કી & AES, DES \\ \hline
એસિમેટ્રિક & એન્ક્રિપ્શન અને ડિક્રિપ્શન માટે અલગ કી & RSA, ECC \\ \hline
હેશ ફંક્શન્સ & વન-વે ફંક્શન્સ, ફિક્સ્ડ આઉટપુટ સાઇઝ & SHA-256, MD5 \\ \hline
ડિજિટલ સિગ્નેચર & ઓથેન્ટિકેશન અને ઇન્ટિગ્રિટી વેરિફિકેશન & RSA સિગ્નેચર \\ \hline
\end{tabulary}
\end{center}

\begin{itemize}
    \item \keyword{કોન્ફિડેન્શિયાલિટી}: અનધિકૃત એક્સેસથી માહિતીનું રક્ષણ
    \item \keyword{ઇન્ટિગ્રિટી}: માહિતી બદલાઈ નથી તે સુનિશ્ચિત કરવું
    \item \keyword{ઓથેન્ટિકેશન}: કોમ્યુનિકેટિંગ પક્ષોની ઓળખ ચકાસવી
\end{itemize}
\end{solutionbox}

\begin{mnemonicbox}
\mnemonic{SHAPE: Symmetric, Hashing, Asymmetric, Protect, Encrypt}
\end{mnemonicbox}

\questionmarks{5(બ)}{4}{સામાજિક મુદ્દાઓ સમજાવો અને હેકિંગ તેની સાવચેતીઓની પણ ચર્ચા કરો.}

\begin{solutionbox}
સાયબર સિક્યોરિટીમાં સામાજિક મુદ્દાઓમાં માનવ મેનિપ્યુલેશન અને સાયબર ખતરાઓની સામાજિક અસરો શામેલ છે.

\textbf{હેકિંગ ટાઇપ્સ:}
\begin{itemize}
    \item \keyword{વ્હાઇટ હેટ}: એથિકલ હેકિંગ, સિક્યોરિટી સુધારણા
    \item \keyword{બ્લેક હેટ}: મેલિશિયસ હેકિંગ, ગેરકાયદેસર પ્રવૃત્તિઓ
    \item \keyword{ગ્રે હેટ}: એથિકલ અને શંકાસ્પદ ક્રિયાઓનું મિશ્રણ
\end{itemize}

\textbf{સાવચેતીઓ:}
\begin{itemize}
    \item \keyword{એજ્યુકેશન}: નિયમિત સિક્યોરિટી અવેરનેસ ટ્રેનિંગ
    \item \keyword{સ્ટ્રોંગ પોલિસીઝ}: સ્પષ્ટ સિક્યોરિટી પ્રક્રિયાઓ અને નીતિઓ
    \item \keyword{ટેકનિકલ કંટ્રોલ્સ}: ફાયરવોલ્સ, એન્ટિવાઇરસ, એન્ક્રિપ્શન
    \item \keyword{રેગ્યુલર અપડેટ્સ}: વલ્નરેબિલિટી સામે સિસ્ટમ્સ પેચિંગ
    \item \keyword{મોનિટરિંગ}: એક્ટિવિટી લોગ્સ, ઇન્ટ્રુઝન ડિટેક્શન
\end{itemize}
\end{solutionbox}

\begin{mnemonicbox}
\mnemonic{STEPS: Social Engineering, Training, Encryption, Patches, Strong passwords}
\end{mnemonicbox}

\questionmarks{5(ક)}{7}{IP સુરક્ષાને વિગતવાર સમજાવો.}

\begin{solutionbox}
IP સિક્યોરિટી (IPsec) એ IP લેયર પર કોમ્યુનિકેશન સુરક્ષિત કરતો પ્રોટોકોલ સ્યુટ છે.

\begin{center}
\captionof{table}{IPsec Components}
\begin{tabulary}{\linewidth}{|L|L|L|}
\hline
\textbf{કોમ્પોનન્ટ} & \textbf{કાર્ય} & \textbf{વર્ણન} \\ \hline
AH & ઓથેન્ટિકેશન હેડર & ઇન્ટિગ્રિટી અને ઓથેન્ટિકેશન પ્રદાન કરે છે \\ \hline
ESP & એન્કેપ્સુલેટિંગ સિક્યોરિટી પેલોડ & કોન્ફિડેન્શિયાલિટી, ઇન્ટિગ્રિટી, ઓથેન્ટિકેશન \\ \hline
IKE & ઇન્ટરનેટ કી એક્સચેન્જ & સિક્યોરિટી એસોસિએશન સ્થાપિત અને સંચાલિત કરે છે \\ \hline
\end{tabulary}
\end{center}

\textbf{IPsec મોડ્સ:}

\begin{center}
\begin{tikzpicture}[node distance=0cm]
    \node [font=\small] at (-3,1.5) {TRANSPORT MODE:};
    \node [gtu block, minimum width=2cm, minimum height=0.8cm] (tip) at (0,1.5) {IP Header};
    \node [gtu block, minimum width=2.5cm, minimum height=0.8cm, right=0cm of tip] (tipsec) {IPsec Header};
    \node [gtu block, minimum width=3cm, minimum height=0.8cm, right=0cm of tipsec] (tpay) {Payload};
    
    \node [font=\small] at (-3,0) {TUNNEL MODE:};
    \node [gtu block, minimum width=2cm, minimum height=0.8cm] (nip) at (0,0) {New IP};
    \node [gtu block, minimum width=2cm, minimum height=0.8cm, right=0cm of nip] (nipsec) {IPsec};
    \node [gtu block, minimum width=2cm, minimum height=0.8cm, right=0cm of nipsec] (oip) {Orig IP};
    \node [gtu block, minimum width=2.5cm, minimum height=0.8cm, right=0cm of oip] (npay) {Payload};
\end{tikzpicture}
\captionof{figure}{IPsec Modes}
\end{center}

\begin{itemize}
    \item \keyword{ટ્રાન્સપોર્ટ}: માત્ર પેલોડને સુરક્ષિત કરે છે - હોસ્ટ-ટુ-હોસ્ટ કોમ્યુનિકેશન
    \item \keyword{ટનલ}: સંપૂર્ણ પેકેટને સુરક્ષિત કરે છે - ગેટવે-ટુ-ગેટવે (VPN)
\end{itemize}

\textbf{IPsec સર્વિસિસ:}
\begin{itemize}
    \item \keyword{ઓથેન્ટિકેશન}: સેન્ડરની ઓળખ ચકાસે છે
    \item \keyword{કોન્ફિડેન્શિયાલિટી}: ઇવ્સડ્રોપિંગ રોકવા માટે ડેટા એન્ક્રિપ્ટ કરે છે
    \item \keyword{ઇન્ટિગ્રિટી}: ડેટા મોડિફાઈ નથી થયો તે સુનિશ્ચિત કરે છે
    \item \keyword{એન્ટી-રિપ્લે}: પેકેટ રિપ્લે એટેક રોકે છે
\end{itemize}
\end{solutionbox}

\begin{mnemonicbox}
\mnemonic{ACCEPT: Authentication, Confidentiality, Cryptography, Encapsulation, Protocols, Tunnel}
\end{mnemonicbox}

\questionmarks{5(અ OR)}{3}{નેટવર્ક સુરક્ષા વ્યાખ્યાયિત કરો અને તેના ઘટકો સમજાવો.}

\begin{solutionbox}
નેટવર્ક સિક્યોરિટી એ નેટવર્ક અને તેના ડેટાને અનધિકૃત એક્સેસ, દુરુપયોગ અને ખતરાઓથી સુરક્ષિત કરવાની પ્રેક્ટિસ છે.

\begin{center}
\captionof{table}{Network Security Components}
\begin{tabulary}{\linewidth}{|L|L|L|}
\hline
\textbf{ઘટક} & \textbf{વર્ણન} & \textbf{ઉદાહરણો} \\ \hline
એક્સેસ કંટ્રોલ & નેટવર્ક એક્સેસને મર્યાદિત કરવું & પાસવર્ડ, મલ્ટી-ફેક્ટર ઓથ \\ \hline
થ્રેટ પ્રિવેન્શન & એટેક બ્લોક કરવા & ફાયરવોલ્સ, IDS/IPS \\ \hline
એન્ક્રિપ્શન & ટ્રાન્ઝિટમાં ડેટા સુરક્ષિત કરવો & SSL/TLS, IPsec \\ \hline
વલ્નરેબિલિટી મેનેજમેન્ટ & નબળાઈઓ ઓળખવી & સ્કેનિંગ, પેચિંગ \\ \hline
મોનિટરિંગ & નેટવર્ક એક્ટિવિટી નિરીક્ષણ & SIEM, લોગ એનાલિસિસ \\ \hline
\end{tabulary}
\end{center}

\textbf{આકૃતિ:}

\begin{center}
\begin{tikzpicture}[node distance=0cm]
    \node [gtu block, minimum width=6cm, minimum height=1cm] (ns) {NETWORK SECURITY};
    
    \node [gtu block, minimum width=2cm, minimum height=0.8cm, below left=0.5cm and -1cm of ns] (ac) {ACCESS CONTROL};
    \node [gtu block, minimum width=2cm, minimum height=0.8cm, below=0.5cm of ns] (tp) {THREAT PREVENT};
    \node [gtu block, minimum width=2cm, minimum height=0.8cm, below right=0.5cm and -1cm of ns] (enc) {ENCRYPTION};
    
    \path [gtu arrow] (ns) -- (ac);
    \path [gtu arrow] (ns) -- (tp);
    \path [gtu arrow] (ns) -- (enc);
\end{tikzpicture}
\captionof{figure}{Network Security Components}
\end{center}

\begin{itemize}
    \item \keyword{કોન્ફિડેન્શિયાલિટી}: અનધિકૃત એક્સેસથી માહિતીનું રક્ષણ
    \item \keyword{ઇન્ટિગ્રિટી}: માહિતીની ચોકસાઈ અને વિશ્વસનીયતા સુનિશ્ચિત કરવી
    \item \keyword{અવેલેબિલિટી}: જરૂર પડે ત્યારે સિસ્ટમ્સ એક્સેસિબલ રાખવા
\end{itemize}
\end{solutionbox}

\begin{mnemonicbox}
\mnemonic{CIMA TV: Confidentiality, Integrity, Monitoring, Access control, Threats, Vulnerabilities}
\end{mnemonicbox}

\questionmarks{5(બ OR)}{4}{સંક્ષિપ્તમાં માહિતી ટેકનોલોજી (સુધારા) અધિનિયમ, 2008 અને ભારતમાં સાયબર કાયદાઓ પર તેની અસરનું વર્ણન કરો.}

\begin{solutionbox}
IT (સુધારા) એક્ટ, 2008 ઉભરતા સાયબર સિક્યોરિટી પડકારોને સંબોધવા માટે ભારતના સાયબર કાયદાઓ અપડેટ કર્યા.

\textbf{મુખ્ય સેક્શન્સ:}
\begin{itemize}
    \item \keyword{સેક્શન 43}: અનધિકૃત એક્સેસ, ડેટા થેફ્ટ માટે પેનલ્ટી
    \item \keyword{સેક્શન 66}: કમ્પ્યુટર સંબંધિત ગુનાઓ અને સજાઓ
    \item \keyword{સેક્શન 69}: ઇન્ટરસેપ્શન અને મોનિટરિંગ માટે અધિકારો
    \item \keyword{સેક્શન 72A}: વ્યક્તિગત ડેટા ગોપનીયતાનું રક્ષણ
\end{itemize}

\textbf{સાયબર કાયદાઓ પર અસર:}
\begin{itemize}
    \item \keyword{વધુ મજબૂત અમલ}: સાયબર ક્રાઇમ માટે વધારેલી જોગવાઈઓ
    \item \keyword{વિસ્તૃત અવકાશ}: નવા ટેકનોલોજિકલ વિકાસને આવરી લીધા
    \item \keyword{કોર્પોરેટ જવાબદારી}: ડેટા માટે સિક્યોરિટી પ્રેક્ટિસની આવશ્યકતા
    \item \keyword{ગ્લોબલ એલાઇન્મેન્ટ}: આંતરરાષ્ટ્રીય ધોરણો સાથે સંકલન
\end{itemize}
\end{solutionbox}

\begin{mnemonicbox}
\mnemonic{SPEC: Security, Privacy, Evidence, Cyber crimes}
\end{mnemonicbox}

\questionmarks{5(ક OR)}{7}{SMTP, PEM, PGP, S/MINE, સ્પામના સંદર્ભમાં ઇમેઇલ સુરક્ષા સમજાવો.}

\begin{solutionbox}
ઇમેઇલ સિક્યોરિટી ઇમેઇલ કન્ટેન્ટ અને એકાઉન્ટ્સને અનધિકૃત એક્સેસ અને એટેક્સથી સુરક્ષિત કરે છે.

\begin{center}
\captionof{table}{Email Security Technologies}
\begin{tabulary}{\linewidth}{|L|L|L|}
\hline
\textbf{ટેકનોલોજી} & \textbf{કાર્ય} & \textbf{ફીચર્સ} \\ \hline
SMTP & સિમ્પલ મેઇલ ટ્રાન્સફર પ્રોટોકોલ & બેઝિક ઇમેઇલ ટ્રાન્સમિશન \\ \hline
PEM & પ્રાઇવસી એન્હાન્સ્ડ મેઇલ & અર્લી ઇમેઇલ એન્ક્રિપ્શન સ્ટાન્ડર્ડ \\ \hline
PGP & પ્રિટી ગુડ પ્રાઇવસી & એન્ડ-ટુ-એન્ડ એન્ક્રિપ્શન, ડિજિટલ સિગ્નેચર \\ \hline
S/MIME & સિક્યોર/મલ્ટિપરપઝ ઇન્ટરનેટ મેઇલ & સર્ટિફિકેટ-બેઝ્ડ એન્ક્રિપ્શન \\ \hline
\end{tabulary}
\end{center}

\textbf{PGP ઇમેઇલ સિક્યોરિટી:}

\begin{center}
\begin{tikzpicture}[node distance=5cm]
    \node [gtu block, align=center] (sender) {SENDER};
    \node [gtu block, align=center, right=of sender] (receiver) {RECEIVER};
    
    \node [below=0.3cm of sender, font=\tiny, align=left] {
        1. Create message\\
        2. Sign with private key\\
        3. Encrypt with receiver's public key
    };
    
    \node [below=0.3cm of receiver, font=\tiny, align=left] {
        4. Decrypt with private key\\
        5. Verify with sender's public key
    };
    
    \path [gtu arrow] (sender) -- node[above, font=\small] {Encrypted Email} (receiver);
\end{tikzpicture}
\captionof{figure}{PGP Email Security}
\end{center}

\begin{itemize}
    \item સેન્ડરની પ્રાઇવેટ કી સાથે મેસેજ સાઇન કરો
    \item રિસીવરની પબ્લિક કી સાથે એન્ક્રિપ્ટ કરો
    \item રિસીવર પ્રાઇવેટ કી સાથે ડિક્રિપ્ટ કરે છે
    \item સેન્ડરની પબ્લિક કી સાથે વેરિફાઇ કરે છે
\end{itemize}

\textbf{સ્પામ પ્રોટેક્શન:}
\begin{itemize}
    \item \keyword{કન્ટેન્ટ ફિલ્ટરિંગ}: મેસેજ કન્ટેન્ટનું એનાલિસિસ
    \item \keyword{સેન્ડર વેરિફિકેશન}: SPF, DKIM, DMARC
    \item \keyword{બિહેવિયરલ એનાલિસિસ}: પેટર્ન રિકગ્નિશન
    \item \keyword{બ્લેકલિસ્ટ/વ્હાઇટલિસ્ટ}: ચોક્કસ સેન્ડર્સને બ્લોકિંગ/એલાઉ કરવા
\end{itemize}
\end{solutionbox}

\begin{mnemonicbox}
\mnemonic{SPEED: S/MIME, PGP, Encryption, Email Security, DMARC}
\end{mnemonicbox}

\end{document}
