\documentclass{article}

% content/resources/templates/preamble.tex
\usepackage[margin=0.6in]{geometry}
\author{Milav Dabgar}
\usepackage{amsmath,amssymb,amsthm}
\usepackage{booktabs}
\usepackage{multirow}
\usepackage{xcolor}
\usepackage{tcolorbox}
\tcbuselibrary{breakable,skins}
\usepackage[colorlinks=true,linkcolor=blue]{hyperref}
\usepackage{titlesec}
\usepackage{enumitem}
\usepackage{tikz}
\usepackage{pgfplots}
\usepackage{circuitikz}
\usepackage[version=4]{mhchem}
\usepackage{longtable}
\usepackage{array}
\usepackage{float}
\usepackage{caption}
\usepackage{listings}

\lstset{
  basicstyle=\small\ttfamily,
  breaklines=true,
  breakatwhitespace=false,
  postbreak=\mbox{\textcolor{red}{$\hookrightarrow$}\space},
  float=false,
  numbers=left,
  numberstyle=\tiny\color{gray},
  numbersep=10pt,
  xleftmargin=2em,
  keywordstyle=\color{blue},
  commentstyle=\color{green!60!black},
  stringstyle=\color{purple},
  backgroundcolor=\color{gray!5},
  showstringspaces=false,
  tabsize=2,
  captionpos=b,
  keepspaces=true,
  columns=flexible
}

\pgfplotsset{compat=1.18}
\usetikzlibrary{shapes,arrows,positioning,calc,patterns,decorations.pathmorphing,decorations.markings,arrows.meta}

% Color scheme
\definecolor{headcolor}{RGB}{0,102,204}
\definecolor{keycolor}{RGB}{220,20,60}
\definecolor{solutioncolor}{RGB}{34,139,34}
\definecolor{mnemoniccolor}{RGB}{148,0,211}
\definecolor{codecolor}{RGB}{0,0,100}

% Spacing
\setlength{\parskip}{3pt}
\setlist[itemize]{nosep}
\setlist[enumerate]{nosep}

% Title formatting
\titleformat{\section}{\Large\bfseries\color{headcolor}}{\thesection}{1em}{}
\titleformat{\subsection}{\large\bfseries\color{headcolor}}{\thesubsection}{1em}{}

% Pandoc tightlist compatibility
\providecommand{\tightlist}{%
  \setlength{\itemsep}{0pt}\setlength{\parskip}{0pt}}

% Pandoc longtable compatibility
\newcounter{none}
\def\thenone{}


% content/resources/templates/english-boxes.tex

% Custom environments
\newtcolorbox{solutionbox}{
 breakable,
 enhanced,
 colback=solutioncolor!5!white,
 colframe=solutioncolor!75!black,
 fonttitle=\bfseries,
 title=Solution
}

\newtcolorbox{solutionboxnobreak}{
 colback=solutioncolor!5!white,
 colframe=solutioncolor!75!black,
 fonttitle=\bfseries,
 title=Solution
}

\newtcolorbox{keyformula}{
 breakable,
 enhanced,
 colback=keycolor!5!white,
 colframe=keycolor!75!black,
 fonttitle=\bfseries,
 title=Key Formula
}

\newtcolorbox{mnemonicboxenv}{
 breakable,
 enhanced,
 colback=mnemoniccolor!5!white,
 colframe=mnemoniccolor!75!black,
 fonttitle=\bfseries,
 title=Mnemonic
}

\newcommand{\mnemonicbox}[1]{%
  \begin{mnemonicboxenv}
    #1
  \end{mnemonicboxenv}
}


% Custom commands for GTU solutions
% This file defines semantic commands for consistent formatting

% Question command with automatic formatting
\newcommand{\question}[2]{%
  \section*{Question #1}%
  \textbf{#2}%
}

% OR question variant
\newcommand{\questionor}[2]{%
  \section*{Question #1 OR}%
  \textbf{#2}%
}

% Proper table environment with caption
\newenvironment{answertable}[1]{%
  \begin{table}[htbp]
  \centering
  \caption{#1}
}{%
  \end{table}
}

% Proper figure environment for diagrams
\newenvironment{answerdiagram}[1]{%
  \begin{figure}[htbp]
  \centering
  \caption{#1}
}{%
  \end{figure}
}

% Semantic markup for key terms
\newcommand{\keyword}[1]{\textbf{#1}}
\newcommand{\code}[1]{\texttt{#1}}
\newcommand{\classname}[1]{\texttt{#1}}
\newcommand{\methodname}[1]{\texttt{#1}}

% Proper quotation marks
\newcommand{\mnemonic}[1]{``#1''}


\title{Computer Networking (4343202) - Summer 2025 Solution}
\date{May 17, 2025}

\begin{document}
\maketitle

% Define missing style locally
\tikzset{
  gtu decision/.style={diamond, aspect=1.5, draw, fill=blue!10, align=center, font=\small}
}

\questionmarks{1(a)}{3}{List Various network Topologies of computer network and explain any one.}

\begin{solutionbox}
\begin{center}
\captionof{table}{Network Topologies}
\begin{tabulary}{\linewidth}{|L|L|}
\hline
\textbf{Topology} & \textbf{Description} \\ \hline
Star & Central hub connects all devices \\ \hline
Ring & Devices connected in circular chain \\ \hline
Bus & Single cable backbone connection \\ \hline
Mesh & Every device connects to every other \\ \hline
Tree & Hierarchical branching structure \\ \hline
Hybrid & Combination of multiple topologies \\ \hline
\end{tabulary}
\end{center}

\textbf{Star Topology Explanation:}

\begin{itemize}
    \item \keyword{Central Hub}: All devices connect to one central point
    \item \keyword{Easy Installation}: Simple to add/remove devices
    \item \keyword{Single Point Failure}: Hub failure affects entire network
\end{itemize}
\end{solutionbox}

\begin{mnemonicbox}
\mnemonic{SRBMTH - Star Ring Bus Mesh Tree Hybrid}
\end{mnemonicbox}

\questionmarks{1(b)}{4}{Compare LAN, WAN and MAN.}

\begin{solutionbox}
\begin{center}
\captionof{table}{LAN vs MAN vs WAN}
\begin{tabulary}{\linewidth}{|L|L|L|L|}
\hline
\textbf{Parameter} & \textbf{LAN} & \textbf{MAN} & \textbf{WAN} \\ \hline
Coverage & Building/Campus & City/Metropolitan & Country/Global \\ \hline
Speed & Very High (1-100 Gbps) & High (10-100 Mbps) & Medium (1-100 Mbps) \\ \hline
Cost & Low & Medium & High \\ \hline
Ownership & Private & Public/Private & Public \\ \hline
\end{tabulary}
\end{center}

\textbf{Key Points:}

\begin{itemize}
    \item \keyword{LAN}: Local Area Network for small areas
    \item \keyword{MAN}: Metropolitan Area Network for cities
    \item \keyword{WAN}: Wide Area Network for large distances
\end{itemize}
\end{solutionbox}

\begin{mnemonicbox}
\mnemonic{LMW - Local Metropolitan Wide}
\end{mnemonicbox}

\questionmarks{1(c)}{7}{Draw the layered architecture of OSI reference model and write at least two services provided by each layer of the model.}

\begin{solutionbox}
\textbf{OSI Model Architecture:}

\begin{center}
\begin{tikzpicture}[node distance=0cm]
    \node [gtu block, minimum width=8cm, minimum height=1cm] (app) {Application Layer - 7};
    \node [gtu block, minimum width=8cm, minimum height=1cm, below=0.1cm of app] (pres) {Presentation Layer - 6};
    \node [gtu block, minimum width=8cm, minimum height=1cm, below=0.1cm of pres] (sess) {Session Layer - 5};
    \node [gtu block, minimum width=8cm, minimum height=1cm, below=0.1cm of sess] (trans) {Transport Layer - 4};
    \node [gtu block, minimum width=8cm, minimum height=1cm, below=0.1cm of trans] (net) {Network Layer - 3};
    \node [gtu block, minimum width=8cm, minimum height=1cm, below=0.1cm of net] (data) {Data Link Layer - 2};
    \node [gtu block, minimum width=8cm, minimum height=1cm, below=0.1cm of data] (phy) {Physical Layer - 1};
\end{tikzpicture}
\captionof{figure}{OSI Reference Model}
\end{center}

\begin{center}
\captionof{table}{OSI Layer Services}
\begin{tabulary}{\linewidth}{|L|L|}
\hline
\textbf{Layer} & \textbf{Services} \\ \hline
Application (7) & Email services, File transfer \\ \hline
Presentation (6) & Data encryption, Data compression \\ \hline
Session (5) & Session establishment, Session termination \\ \hline
Transport (4) & Flow control, Error correction \\ \hline
Network (3) & Routing, Path determination \\ \hline
Data Link (2) & Frame synchronization, Error detection \\ \hline
Physical (1) & Bit transmission, Signal conversion \\ \hline
\end{tabulary}
\end{center}
\end{solutionbox}

\begin{mnemonicbox}
\mnemonic{All People Seem To Need Data Processing}
\end{mnemonicbox}

\questionmarks{1(c OR)}{7}{Explain Each layer of TCP/IP Model with its protocol.}

\begin{solutionbox}
\textbf{TCP/IP Model Layers:}

\begin{center}
\begin{tikzpicture}[node distance=0cm]
    \node [gtu block, minimum width=8cm, minimum height=1.2cm] (app) {Application Layer};
    \node [below=0.1cm of app, font=\small] {HTTP, FTP, SMTP, DNS};
    \node [gtu block, minimum width=8cm, minimum height=1.2cm, below=0.5cm of app] (trans) {Transport Layer};
    \node [below=0.1cm of trans, font=\small] {TCP, UDP};
    \node [gtu block, minimum width=8cm, minimum height=1.2cm, below=0.5cm of trans] (net) {Internet Layer};
    \node [below=0.1cm of net, font=\small] {IP, ICMP, ARP};
    \node [gtu block, minimum width=8cm, minimum height=1.2cm, below=0.5cm of net] (link) {Network Access Layer};
    \node [below=0.1cm of link, font=\small] {Ethernet, Wi-Fi};
\end{tikzpicture}
\captionof{figure}{TCP/IP Model}
\end{center}

\begin{center}
\captionof{table}{TCP/IP Layer Details}
\begin{tabulary}{\linewidth}{|L|L|L|}
\hline
\textbf{Layer} & \textbf{Protocols} & \textbf{Function} \\ \hline
Application & HTTP, FTP, SMTP, DNS & User applications \\ \hline
Transport & TCP, UDP & End-to-end delivery \\ \hline
Internet & IP, ICMP, ARP & Routing packets \\ \hline
Network Access & Ethernet, Wi-Fi & Physical transmission \\ \hline
\end{tabulary}
\end{center}

\textbf{Key Features:}
\begin{itemize}
    \item \keyword{Simplified Model}: Only 4 layers vs OSI's 7
    \item \keyword{Protocol Suite}: Complete networking solution
    \item \keyword{Internet Standard}: Basis of modern internet
\end{itemize}
\end{solutionbox}

\begin{mnemonicbox}
\mnemonic{ATIN - Application Transport Internet Network}
\end{mnemonicbox}

\questionmarks{2(a)}{3}{Explain functions of following network devices: Repeater, Hub}

\begin{solutionbox}
\begin{center}
\captionof{table}{Network Device Functions}
\begin{tabulary}{\linewidth}{|L|L|L|}
\hline
\textbf{Device} & \textbf{Function} & \textbf{Layer} \\ \hline
Repeater & Signal amplification, Range extension & Physical (1) \\ \hline
Hub & Signal broadcasting, Collision domain sharing & Physical (1) \\ \hline
\end{tabulary}
\end{center}

\textbf{Details:}
\begin{itemize}
    \item \keyword{Repeater}: Regenerates weak signals over long distances
    \item \keyword{Hub}: Connects multiple devices in star topology
    \item \keyword{Shared Medium}: Both create single collision domain
\end{itemize}
\end{solutionbox}

\begin{mnemonicbox}
\mnemonic{RH - Repeat Hub signals}
\end{mnemonicbox}

\questionmarks{2(b)}{4}{Explain the following term 1) FDDI 2) ARP, RARP}

\begin{solutionbox}
\textbf{FDDI (Fiber Distributed Data Interface):}

\begin{itemize}
    \item \keyword{Technology}: 100 Mbps fiber optic network
    \item \keyword{Topology}: Dual ring for fault tolerance
    \item \keyword{Application}: Backbone networks, high reliability
\end{itemize}

\textbf{ARP (Address Resolution Protocol):}

\begin{itemize}
    \item \keyword{Function}: Maps IP address to MAC address
    \item \keyword{Process}: Broadcasts request, receives reply
\end{itemize}

\textbf{RARP (Reverse ARP):}

\begin{itemize}
    \item \keyword{Function}: Maps MAC address to IP address
    \item \keyword{Usage}: Diskless workstations, boot process
\end{itemize}
\end{solutionbox}

\begin{mnemonicbox}
\mnemonic{FAR - FDDI ARP RARP}
\end{mnemonicbox}

\questionmarks{2(c)}{7}{Explain the Function of firewall in network security with principles and Kerberos-concept.}

\begin{solutionbox}
\textbf{Firewall Principles:}

\begin{itemize}
    \item \keyword{Packet Filtering}: Examines packet headers
    \item \keyword{Stateful Inspection}: Tracks connection states
    \item \keyword{Application Gateway}: Deep packet inspection
\end{itemize}

\begin{center}
\begin{tikzpicture}[node distance=3cm]
    \node [gtu block] (internet) {Internet};
    \node [gtu block, right=of internet] (firewall) {Firewall};
    \node [gtu block, right=of firewall] (internal) {Internal Network};
    
    \path [gtu arrow] (internet) -- (firewall);
    \path [gtu arrow] (firewall) -- (internal);
    
    \node [below=0.5cm of firewall, align=center, font=\small] {Blocks Threats\\Allows Safe Traffic};
\end{tikzpicture}
\captionof{figure}{Firewall Function}
\end{center}

\textbf{Kerberos Concept:}

\begin{center}
\begin{tikzpicture}[node distance=5cm]
    \node [gtu block] (client) {CLIENT};
    \node [gtu block, right=of client] (kdc) {KDC};
    \node [gtu block, right=of kdc] (server) {SERVER};
    
    \path [gtu arrow, bend left=20] (client) -- node[above, font=\tiny] {1. Request TGT} (kdc);
    \path [gtu arrow, bend left=20] (kdc) -- node[below, font=\tiny] {2. TGT} (client);
    \path [gtu arrow, bend left=20] (client) -- node[above, font=\tiny] {3. Request Service Ticket} (kdc);
    \path [gtu arrow, bend left=20] (kdc) -- node[below, font=\tiny] {4. Service Ticket} (client);
    \path [gtu arrow] (client) -- node[above, font=\tiny] {5. Service Request} (server);
    \path [gtu arrow] (server) -- node[below, font=\tiny] {6. Session Key} (client);
\end{tikzpicture}
\captionof{figure}{Kerberos Authentication}
\end{center}

\begin{itemize}
    \item \keyword{Authentication Service}: Secure user verification
    \item \keyword{Ticket System}: Time-limited access tokens
    \item \keyword{Three-party Protocol}: Client, Server, Key Distribution Center
\end{itemize}

\textbf{Security Benefits:}
\begin{itemize}
    \item \keyword{Access Control}: Prevents unauthorized access
    \item \keyword{Network Protection}: Shields internal resources
\end{itemize}
\end{solutionbox}

\begin{mnemonicbox}
\mnemonic{FPK - Firewall Protects with Kerberos}
\end{mnemonicbox}

% Question 3
\questionmarks{3(a)}{3}{Find the class of following IP address.\\ 1) 01111000 00001111 10101010 11000000\\ 2) 11101000 01010101 11111111 11000011}

\begin{solutionbox}
\textbf{IP Address Classification:}

\begin{center}
\captionof{table}{IP Class Analysis}
\begin{tabulary}{\linewidth}{|L|L|L|L|}
\hline
\textbf{Binary Address} & \textbf{Decimal} & \textbf{First Octet} & \textbf{Class} \\ \hline
01111000... & 120.15.170.192 & 120 (64-127) & \textbf{Class A} \\ \hline
11101000... & 232.85.255.195 & 232 (224-239) & \textbf{Class D} \\ \hline
\end{tabulary}
\end{center}

\textbf{Class Ranges:}

\begin{itemize}
    \item \keyword{Class A}: 1-126 (0xxxxxxx)
    \item \keyword{Class B}: 128-191 (10xxxxxx)
    \item \keyword{Class C}: 192-223 (110xxxxx)
    \item \keyword{Class D}: 224-239 (1110xxxx)
\end{itemize}

\textbf{Results:}

\begin{itemize}
    \item \keyword{First IP}: Class A (Unicast)
    \item \keyword{Second IP}: Class D (Multicast)
\end{itemize}
\end{solutionbox}

\begin{mnemonicbox}
\mnemonic{ABCD - A(1-126) B(128-191) C(192-223) D(224-239)}
\end{mnemonicbox}

\questionmarks{3(b)}{4}{Differentiate IPv4 and IPv6.}

\begin{solutionbox}
\textbf{IPv4 vs IPv6 Comparison:}

\begin{center}
\captionof{table}{IPv4 vs IPv6}
\begin{tabulary}{\linewidth}{|L|L|L|}
\hline
\textbf{Feature} & \textbf{IPv4} & \textbf{IPv6} \\ \hline
\textbf{Address Length} & 32 bits & 128 bits \\ \hline
\textbf{Address Format} & Dotted decimal & Hexadecimal \\ \hline
\textbf{Address Space} & 4.3 billion & 340 undecillion \\ \hline
\textbf{Header Size} & Variable (20-60 bytes) & Fixed (40 bytes) \\ \hline
\textbf{Security} & Optional (IPSec) & Built-in (IPSec) \\ \hline
\textbf{Configuration} & Manual/DHCP & Auto-configuration \\ \hline
\end{tabulary}
\end{center}

\textbf{Key Differences:}

\begin{itemize}
    \item \keyword{Addressing}: IPv6 provides vastly more addresses
    \item \keyword{Security}: IPv6 has mandatory security features
    \item \keyword{Performance}: IPv6 has simplified header structure
\end{itemize}
\end{solutionbox}

\begin{mnemonicbox}
\mnemonic{IPv4 to IPv6 = More addresses, Better security}
\end{mnemonicbox}

\questionmarks{3(c)}{7}{Explain Static and Dynamic Routing Algorithms.}

\begin{solutionbox}
\textbf{Static Routing:}

\begin{center}
\begin{tikzpicture}[node distance=1.5cm]
    \node [gtu block] (admin) {Administrator};
    \node [gtu block, right=of admin] (entry) {Manual Route Entry};
    \node [gtu block, right=of entry] (table) {Routing Table};
    
    \draw [gtu arrow] (admin) -- (entry);
    \draw [gtu arrow] (entry) -- (table);
\end{tikzpicture}
\end{center}

\textbf{Dynamic Routing:}

\begin{center}
\begin{tikzpicture}[node distance=1.5cm]
    \node [gtu block] (proto) {Routing Protocol};
    \node [gtu block, right=of proto] (disc) {Route Discovery};
    \node [gtu block, right=of disc] (update) {Auto Updates};
    \node [gtu block, right=of update] (table) {Adaptive Table};
    
    \draw [gtu arrow] (proto) -- (disc);
    \draw [gtu arrow] (disc) -- (update);
    \draw [gtu arrow] (update) -- (table);
\end{tikzpicture}
\captionof{figure}{Static vs Dynamic Routing}
\end{center}

\textbf{Comparison Table:}

\begin{center}
\captionof{table}{Static vs Dynamic Routing}
\begin{tabulary}{\linewidth}{|L|L|L|}
\hline
\textbf{Aspect} & \textbf{Static Routing} & \textbf{Dynamic Routing} \\ \hline
\textbf{Configuration} & Manual setup & Automatic discovery \\ \hline
\textbf{Adaptability} & No adaptation & Adapts to changes \\ \hline
\textbf{Resource Usage} & Low CPU/Memory & Higher CPU/Memory \\ \hline
\textbf{Scalability} & Poor for large networks & Good for large networks \\ \hline
\textbf{Protocols} & None required & RIP, OSPF, BGP \\ \hline
\end{tabulary}
\end{center}

\textbf{Applications:}

\begin{itemize}
    \item \keyword{Static}: Small networks, specific paths
    \item \keyword{Dynamic}: Large networks, fault tolerance
\end{itemize}
\end{solutionbox}

\begin{mnemonicbox}
\mnemonic{SD - Static=Simple, Dynamic=Automatic}
\end{mnemonicbox}

\questionmarks{3(a OR)}{3}{Explain CIDR. How does it differ from traditional IP address allocation methods?}

\begin{solutionbox}
\textbf{CIDR (Classless Inter-Domain Routing):}

\begin{itemize}
    \item \keyword{Concept}: Variable length subnet masking
    \item \keyword{Notation}: IP address/prefix length (e.g., 192.168.1.0/24)
    \item \keyword{Flexibility}: Subnets of any size
\end{itemize}

\textbf{Traditional vs CIDR:}

\begin{center}
\captionof{table}{Traditional vs CIDR}
\begin{tabulary}{\linewidth}{|L|L|L|}
\hline
\textbf{Method} & \textbf{Allocation} & \textbf{Efficiency} \\ \hline
\textbf{Traditional} & Fixed class boundaries & Wasteful (Class B = 65,536 IPs) \\ \hline
\textbf{CIDR} & Variable subnet sizes & Efficient allocation \\ \hline
\end{tabulary}
\end{center}

\textbf{Benefits:}

\begin{itemize}
    \item \keyword{Address Conservation}: Reduces IP address waste
    \item \keyword{Route Aggregation}: Summarizes multiple routes
\end{itemize}
\end{solutionbox}

\begin{mnemonicbox}
\mnemonic{CIDR = Classless Intelligent Address Routing}
\end{mnemonicbox}

\questionmarks{3(b OR)}{4}{Describe DSL technology with its types, advantages and limitations.}

\begin{solutionbox}
\textbf{DSL (Digital Subscriber Line):}

\begin{itemize}
    \item \keyword{Technology}: High-speed internet over telephone lines
    \item \keyword{Frequency}: Uses higher frequencies than voice
\end{itemize}

\textbf{DSL Types:}

\begin{center}
\captionof{table}{DSL Types}
\begin{tabulary}{\linewidth}{|L|L|L|}
\hline
\textbf{Type} & \textbf{Speed} & \textbf{Application} \\ \hline
\textbf{ADSL} & Asymmetric (faster download) & Home users \\ \hline
\textbf{SDSL} & Symmetric (equal up/down) & Business \\ \hline
\textbf{VDSL} & Very high speed & Short distances \\ \hline
\end{tabulary}
\end{center}

\textbf{Advantages:}

\begin{itemize}
    \item \keyword{Always-on Connection}: No dial-up required
    \item \keyword{Existing Infrastructure}: Uses phone lines
    \item \keyword{Cost-effective}: Affordable high-speed access
\end{itemize}

\textbf{Limitations:}

\begin{itemize}
    \item \keyword{Distance Dependent}: Speed decreases with distance
    \item \keyword{Line Quality}: Requires good copper lines
    \item \keyword{Availability}: Not available everywhere
\end{itemize}
\end{solutionbox}

\begin{mnemonicbox}
\mnemonic{DSL = Digital Speed Limited by distance}
\end{mnemonicbox}

\questionmarks{3(c OR)}{7}{Explain error control and flow control at data link layer in detail.}

\begin{solutionbox}
\textbf{Error Control:}

\begin{center}
\begin{tikzpicture}[node distance=1.5cm, auto]
    \node [gtu block] (data) {Data Transmit};
    \node [gtu decision, right=of data, align=center] (check) {Error\\Found?};
    \node [gtu block, below=of check] (req) {Retransmit};
    \node [gtu block, right=of check] (accept) {Accept Data};
    
    \draw [gtu arrow] (data) -- (check);
    \draw [gtu arrow] (check) -- node {Yes} (req);
    \draw [gtu arrow] (check) -- node {No} (accept);
    \draw [gtu arrow] (req) -| (data);
\end{tikzpicture}
\captionof{figure}{Error Control Logic}
\end{center}

\textbf{Methods:}
\begin{center}
\captionof{table}{Error Control Methods}
\begin{tabulary}{\linewidth}{|L|L|L|}
\hline
\textbf{Method} & \textbf{Technique} & \textbf{Application} \\ \hline
\textbf{Parity Check} & Single bit error detection & Simple systems \\ \hline
\textbf{Checksum} & Mathematical sum verification & TCP/UDP \\ \hline
\textbf{CRC} & Polynomial division & Ethernet, Wi-Fi \\ \hline
\textbf{ARQ} & Automatic Repeat Request & Reliable protocols \\ \hline
\end{tabulary}
\end{center}

\textbf{Flow Control:}

\begin{center}
\begin{tikzpicture}[node distance=1.5cm, auto]
    \node [gtu block] (check) {Check Buffer};
    \node [gtu decision, right=of check, align=center] (full) {Buffer\\Full?};
    \node [gtu block, below=of full] (wait) {Wait};
    \node [gtu block, right=of full] (send) {Send Data};
    
    \draw [gtu arrow] (check) -- (full);
    \draw [gtu arrow] (full) -- node {Yes} (wait);
    \draw [gtu arrow] (full) -- node {No} (send);
    \draw [gtu arrow] (wait) -| (check);
\end{tikzpicture}
\captionof{figure}{Flow Control Logic}
\end{center}

\textbf{Techniques:}

\begin{itemize}
    \item \keyword{Stop-and-Wait}: Send one frame, wait for ACK
    \item \keyword{Sliding Window}: Multiple frames in transit
    \item \keyword{Buffer Management}: Prevents overflow
\end{itemize}
\end{solutionbox}

\begin{mnemonicbox}
\mnemonic{EF - Error detection, Flow regulation}
\end{mnemonicbox}

% Question 4
\questionmarks{4(a)}{3}{Explain video over IP.}

\begin{solutionbox}
\textbf{Video over IP (VoIP):}

\begin{itemize}
    \item \keyword{Technology}: Transmits video signals over IP networks
    \item \keyword{Digitization}: Converts analog video to digital packets
    \item \keyword{Real-time}: Requires low latency transmission
\end{itemize}

\textbf{Components:}

\begin{itemize}
    \item \keyword{Encoder}: Compresses video data
    \item \keyword{Network}: IP infrastructure for transport
    \item \keyword{Decoder}: Decompresses at destination
\end{itemize}

\textbf{Applications:}

\begin{itemize}
    \item \keyword{Video Conferencing}: Business communications
    \item \keyword{Streaming}: Entertainment services
    \item \keyword{Surveillance}: Security systems
\end{itemize}
\end{solutionbox}

\begin{mnemonicbox}
\mnemonic{VIP = Video Internet Protocol}
\end{mnemonicbox}

\questionmarks{4(b)}{4}{Explain Electronic-Mail with its protocol.}

\begin{solutionbox}
\textbf{Email System Components:}

\begin{center}
\begin{tikzpicture}[node distance=1.5cm]
    \node [gtu block] (ua) {User Agent};
    \node [gtu block, right=of ua] (smtp) {SMTP Server};
    \node [gtu state, right=of smtp] (inet) {Internet};
    \node [gtu block, right=of inet] (pop) {POP3/IMAP};
    \node [gtu block, below=of pop] (recip) {Recipient};
    
    \draw [gtu arrow] (ua) -- (smtp);
    \draw [gtu arrow] (smtp) -- (inet);
    \draw [gtu arrow] (inet) -- (pop);
    \draw [gtu arrow] (pop) -- (recip);
\end{tikzpicture}
\captionof{figure}{Email Architecture}
\end{center}

\textbf{Email Protocols:}

\begin{center}
\captionof{table}{Email Protocols}
\begin{tabulary}{\linewidth}{|L|L|L|}
\hline
\textbf{Protocol} & \textbf{Function} & \textbf{Port} \\ \hline
\textbf{SMTP} & Send/relay messages & 25, 587 \\ \hline
\textbf{POP3} & Download messages & 110 \\ \hline
\textbf{IMAP} & Server-based access & 143 \\ \hline
\end{tabulary}
\end{center}

\textbf{Message Flow:}

\begin{itemize}
    \item \keyword{Composition}: User creates message
    \item \keyword{Submission}: SMTP sends to server
    \item \keyword{Delivery}: Server forwards to recipient
    \item \keyword{Retrieval}: POP3/IMAP downloads message
\end{itemize}
\end{solutionbox}

\begin{mnemonicbox}
\mnemonic{SPI - SMTP sends, POP3/IMAP receives}
\end{mnemonicbox}

\questionmarks{4(c)}{7}{Explain Role of DNS- Domain Name System Describe the process of DNS resolution.}

\begin{solutionbox}
\textbf{DNS Role:}

\begin{itemize}
    \item \keyword{Name Resolution}: Converts domain names to IP addresses
    \item \keyword{Hierarchical System}: Distributed database structure
    \item \keyword{Internet Navigation}: Makes web browsing user-friendly
\end{itemize}

\textbf{DNS Resolution Process:}

\begin{center}
\begin{tikzpicture}[node distance=3cm, auto]
    \node [gtu block] (client) {Client};
    \node [gtu block, right=of client] (local) {Local DNS};
    \node [gtu block, below=of client] (root) {Root Server};
    \node [gtu block, right=of root] (tld) {TLD (.com)};
    \node [gtu block, right=of tld] (auth) {Auth (ex.com)};
    
    \draw [->, dashed] (client) -- node[above] {1. Query} (local);
    \draw [->] (local) -- node[left] {2} (root);
    \draw [->] (root) -- node[right] {3. Ref} (local);
    \draw [->] (local) -- node[left] {4} (tld);
    \draw [->] (tld) -- node[right] {5. Ref} (local);
    \draw [->] (local) -- node[right] {6} (auth);
    \draw [->] (auth) -- node[right] {7. IP} (local);
    \draw [->, dashed] (local) -- node[below] {8. IP} (client);
\end{tikzpicture}
\captionof{figure}{Iterative DNS Resolution}
\end{center}

\textbf{Resolution Steps:}

\begin{enumerate}
    \item \textbf{Local Cache Check}: Check local DNS cache
    \item \textbf{Recursive Query}: Contact local DNS server
    \item \textbf{Root Server}: Get TLD server reference
    \item \textbf{TLD Server}: Get authoritative server reference
    \item \textbf{Authoritative Server}: Get final IP address
    \item \textbf{Response Return}: IP address returned to client
\end{enumerate}

\textbf{DNS Record Types:}

\begin{itemize}
    \item \keyword{A Record}: Maps name to IPv4 address
    \item \keyword{AAAA Record}: Maps name to IPv6 address
    \item \keyword{CNAME}: Canonical name alias
    \item \keyword{MX}: Mail exchange server
\end{itemize}
\end{solutionbox}

\begin{mnemonicbox}
\mnemonic{DNS = Directory Name Service}
\end{mnemonicbox}

\questionmarks{4(a OR)}{3}{Explain WWW, HTML.}

\begin{solutionbox}
\textbf{WWW (World Wide Web):}

\begin{itemize}
    \item \keyword{Definition}: Information system of interlinked documents
    \item \keyword{Access}: Through web browsers using HTTP
    \item \keyword{Components}: Web pages, links, URLs
\end{itemize}

\textbf{HTML (HyperText Markup Language):}

\begin{itemize}
    \item \keyword{Purpose}: Standard markup language for web pages
    \item \keyword{Structure}: Tags define document elements
    \item \keyword{Hyperlinks}: Connect different web resources
\end{itemize}

\textbf{Relationship:}

\begin{itemize}
    \item \keyword{WWW}: The system/platform
    \item \keyword{HTML}: The content format
    \item \keyword{Integration}: HTML creates WWW content
\end{itemize}
\end{solutionbox}

\begin{mnemonicbox}
\mnemonic{WWW uses HTML for content}
\end{mnemonicbox}

\questionmarks{4(b OR)}{4}{Explain HTTP and FTP.}

\begin{solutionbox}
\textbf{Protocol Comparison:}

\begin{center}
\captionof{table}{HTTP vs FTP}
\begin{tabulary}{\linewidth}{|L|L|L|}
\hline
\textbf{Feature} & \textbf{HTTP} & \textbf{FTP} \\ \hline
\textbf{Purpose} & Web page transfer & File transfer \\ \hline
\textbf{Port} & 80 (HTTP), 443 (HTTPS) & 21 (control), 20 (data) \\ \hline
\textbf{Connection} & Stateless & Stateful \\ \hline
\textbf{Security} & HTTPS for security & FTPS for security \\ \hline
\end{tabulary}
\end{center}

\textbf{HTTP (HyperText Transfer Protocol):}

\begin{itemize}
    \item \keyword{Function}: Request-response protocol for web
    \item \keyword{Methods}: GET, POST, PUT, DELETE
    \item \keyword{Stateless}: Each request independent
\end{itemize}

\textbf{FTP (File Transfer Protocol):}

\begin{itemize}
    \item \keyword{Function}: Upload/download files between systems
    \item \keyword{Modes}: Active and Passive
    \item \keyword{Authentication}: Username/password required
\end{itemize}
\end{solutionbox}

\begin{mnemonicbox}
\mnemonic{HF - HTTP for Hypertext, FTP for Files}
\end{mnemonicbox}

\questionmarks{4(c OR)}{7}{Explain TCP and UDP protocol in transport layer in relation to connection oriented and connection less network.}

\begin{solutionbox}
\textbf{Transport Layer Protocols:}

\begin{center}
\begin{tikzpicture}[node distance=1.5cm]
    \node [gtu block] (trans) {Transport Layer};
    \node [gtu block, below left=of trans] (tcp) {TCP\\(Connection Oriented)};
    \node [gtu block, below right=of trans] (udp) {UDP\\(Connectionless)};
    \node [gtu block, below=of tcp] (rel) {Reliable Delivery};
    \node [gtu block, below=of udp] (fast) {Fast Delivery};
    
    \draw [gtu arrow] (trans) -- (tcp);
    \draw [gtu arrow] (trans) -- (udp);
    \draw [gtu arrow] (tcp) -- (rel);
    \draw [gtu arrow] (udp) -- (fast);
\end{tikzpicture}
\captionof{figure}{TCP vs UDP Overview}
\end{center}

\textbf{Protocol Comparison:}

\begin{center}
\captionof{table}{TCP vs UDP}
\begin{tabulary}{\linewidth}{|L|L|L|}
\hline
\textbf{Feature} & \textbf{TCP} & \textbf{UDP} \\ \hline
\textbf{Connection} & Connection-oriented & Connectionless \\ \hline
\textbf{Reliability} & Guaranteed delivery & Best effort \\ \hline
\textbf{Speed} & Slower (overhead) & Faster (minimal overhead) \\ \hline
\textbf{Header Size} & 20 bytes & 8 bytes \\ \hline
\textbf{Flow Control} & Yes & No \\ \hline
\textbf{Error Control} & Yes & Limited \\ \hline
\end{tabulary}
\end{center}

\textbf{Details:}

\begin{itemize}
    \item \keyword{TCP}: Three-way Handshake (SYN, SYN-ACK, ACK), Reliable, Flow Control. Used for Web, Email.
    \item \keyword{UDP}: No Connection Setup, Lightweight, No Guarantees. Used for Video, Gaming, DNS.
\end{itemize}
\end{solutionbox}

\begin{mnemonicbox}
\mnemonic{TCP = Thorough, UDP = Ultra-fast}
\end{mnemonicbox}

\questionmarks{2(a OR)}{3}{Explain functions of following network devices: Switch, Router}

\begin{solutionbox}
\begin{center}
\captionof{table}{Switch vs Router}
\begin{tabulary}{\linewidth}{|L|L|L|}
\hline
\textbf{Device} & \textbf{Function} & \textbf{Layer} \\ \hline
Switch & MAC address learning, Frame forwarding & Data Link (2) \\ \hline
Router & IP routing, Path selection & Network (3) \\ \hline
\end{tabulary}
\end{center}

\textbf{Details:}
\begin{itemize}
    \item \keyword{Switch}: Creates separate collision domains per port
    \item \keyword{Router}: Connects different networks, makes routing decisions
    \item \keyword{Intelligence}: Switch learns MAC, Router maintains routing table
\end{itemize}
\end{solutionbox}

\begin{mnemonicbox}
\mnemonic{SR - Switch Routes intelligently}
\end{mnemonicbox}

\questionmarks{2(b OR)}{4}{Explain the following term 1) CDDI 2) DHCP and BOOTP}

\begin{solutionbox}
\textbf{CDDI (Copper Distributed Data Interface):}

\begin{itemize}
    \item \keyword{Technology}: FDDI over copper cables
    \item \keyword{Speed}: 100 Mbps over twisted pair
    \item \keyword{Cost}: Cheaper alternative to fiber FDDI
\end{itemize}

\textbf{DHCP (Dynamic Host Configuration Protocol):}

\begin{itemize}
    \item \keyword{Function}: Automatic IP address assignment
    \item \keyword{Process}: Discover, Offer, Request, Acknowledge
    \item \keyword{Benefits}: Centralized IP management
\end{itemize}

\textbf{BOOTP (Bootstrap Protocol):}

\begin{itemize}
    \item \keyword{Function}: Network bootstrap for diskless clients
    \item \keyword{Static}: Fixed IP address assignment
    \item \keyword{Predecessor}: Earlier version of DHCP
\end{itemize}
\end{solutionbox}

\begin{mnemonicbox}
\mnemonic{CDB - CDDI DHCP BOOTP}
\end{mnemonicbox}

\questionmarks{2(c OR)}{7}{Explain Software define network(SDN) with its Architecture, Application, Advantage and limitation.}

\begin{solutionbox}
\textbf{SDN Architecture:}

\begin{center}
\begin{tikzpicture}[node distance=1.5cm]
    \node [gtu block, minimum width=6cm] (app) {Application Plane\\(Network Apps)};
    \node [gtu block, minimum width=6cm, below=of app] (ctrl) {Control Plane\\(SDN Controller)};
    \node [gtu block, minimum width=6cm, below=of ctrl] (data) {Data Plane\\(Switches/Routers)};
    
    \path [gtu arrow] (app) -- node[right, font=\tiny] {Northbound API} (ctrl);
    \path [gtu arrow] (ctrl) -- node[right, font=\tiny] {Southbound API (OpenFlow)} (data);
\end{tikzpicture}
\captionof{figure}{SDN Architecture}
\end{center}

\begin{itemize}
    \item \keyword{Control Plane}: Centralized network intelligence
    \item \keyword{Data Plane}: Packet forwarding devices
    \item \keyword{Application Plane}: Network applications and services
\end{itemize}

\textbf{Applications:}

\begin{itemize}
    \item \keyword{Cloud Computing}: Dynamic resource allocation
    \item \keyword{Network Virtualization}: Multiple virtual networks
    \item \keyword{Traffic Engineering}: Optimized path selection
\end{itemize}

\textbf{Advantages:}

\begin{itemize}
    \item \keyword{Centralized Control}: Simplified network management
    \item \keyword{Programmability}: Custom network behaviors
    \item \keyword{Flexibility}: Rapid service deployment
\end{itemize}

\textbf{Limitations:}

\begin{itemize}
    \item \keyword{Single Point Failure}: Controller dependency
    \item \keyword{Scalability}: Performance bottlenecks
    \item \keyword{Security}: New attack vectors
\end{itemize}
\end{solutionbox}

\begin{mnemonicbox}
\mnemonic{SCAP - Software Control Application Programmable}
\end{mnemonicbox}

\questionmarks{3(a)}{3}{Find the class of following IP address. 1) 01111000 00001111 10101010 11000000 2) 11101000 01010101 11111111 11000011}

\begin{solutionbox}
\begin{center}
\captionof{table}{IP Address Classification}
\begin{tabulary}{\linewidth}{|L|L|L|L|}
\hline
\textbf{Binary Address} & \textbf{Decimal} & \textbf{First Octet} & \textbf{Class} \\ \hline
01111000... & 120.15.170.192 & 120 (64-127) & Class A \\ \hline
11101000... & 232.85.255.195 & 232 (224-239) & Class D \\ \hline
\end{tabulary}
\end{center}

\textbf{Class Ranges:}

\begin{itemize}
    \item \keyword{Class A}: 1-126 (0xxxxxxx)
    \item \keyword{Class B}: 128-191 (10xxxxxx)
    \item \keyword{Class C}: 192-223 (110xxxxx)
    \item \keyword{Class D}: 224-239 (1110xxxx)
\end{itemize}

\textbf{Results:}

\begin{itemize}
    \item \keyword{First IP}: Class A (Unicast)
    \item \keyword{Second IP}: Class D (Multicast)
\end{itemize}
\end{solutionbox}

\begin{mnemonicbox}
\mnemonic{ABCD - A(1-126) B(128-191) C(192-223) D(224-239)}
\end{mnemonicbox}

\questionmarks{3(b)}{4}{Differentiate IPv4 and IPv6.}

\begin{solutionbox}
\begin{center}
\captionof{table}{IPv4 vs IPv6}
\begin{tabulary}{\linewidth}{|L|L|L|}
\hline
\textbf{Feature} & \textbf{IPv4} & \textbf{IPv6} \\ \hline
Address Length & 32 bits & 128 bits \\ \hline
Address Format & Dotted decimal & Hexadecimal \\ \hline
Address Space & 4.3 billion & 340 undecillion \\ \hline
Header Size & Variable (20-60 bytes) & Fixed (40 bytes) \\ \hline
Security & Optional (IPSec) & Built-in (IPSec) \\ \hline
Configuration & Manual/DHCP & Auto-configuration \\ \hline
\end{tabulary}
\end{center}

\textbf{Key Differences:}

\begin{itemize}
    \item \keyword{Addressing}: IPv6 provides vastly more addresses
    \item \keyword{Security}: IPv6 has mandatory security features
    \item \keyword{Performance}: IPv6 has simplified header structure
\end{itemize}
\end{solutionbox}

\begin{mnemonicbox}
\mnemonic{IPv4 to IPv6 = More addresses, Better security}
\end{mnemonicbox}

\questionmarks{3(c)}{7}{Explain Static and Dynamic Routing Algorithms.}

\begin{solutionbox}
\textbf{Static Routing:}

\begin{center}
\begin{tikzpicture}[node distance=2cm]
    \node [gtu state] (admin) {Administrator};
    \node [gtu block, right=of admin] (table) {Routing Table};
    \node [gtu block, right=of table] (path) {Fixed Paths};
    
    \path [gtu arrow] (admin) -- node[above, font=\tiny] {Configures} (table);
    \path [gtu arrow] (table) -- (path);
\end{tikzpicture}
\end{center}

\textbf{Dynamic Routing:}

\begin{center}
\begin{tikzpicture}[node distance=2cm]
    \node [gtu state] (proto) {Routing Protocol};
    \node [gtu block, right=of proto] (disc) {Route Discovery};
    \node [gtu block, right=of disc] (update) {Auto Updates};
    
    \path [gtu arrow] (proto) -- (disc);
    \path [gtu arrow] (disc) -- (update);
\end{tikzpicture}
\captionof{figure}{Routing Methods}
\end{center}

\begin{center}
\captionof{table}{Static vs Dynamic Routing}
\begin{tabulary}{\linewidth}{|L|L|L|}
\hline
\textbf{Aspect} & \textbf{Static Routing} & \textbf{Dynamic Routing} \\ \hline
Configuration & Manual setup & Automatic discovery \\ \hline
Adaptability & No adaptation & Adapts to changes \\ \hline
Resource Usage & Low CPU/Memory & Higher CPU/Memory \\ \hline
Scalability & Poor for large networks & Good for large networks \\ \hline
Protocols & None required & RIP, OSPF, BGP \\ \hline
\end{tabulary}
\end{center}

\textbf{Applications:}

\begin{itemize}
    \item \keyword{Static}: Small networks, specific paths
    \item \keyword{Dynamic}: Large networks, fault tolerance
\end{itemize}
\end{solutionbox}

\begin{mnemonicbox}
\mnemonic{SD - Static=Simple, Dynamic=Automatic}
\end{mnemonicbox}

\questionmarks{3(a OR)}{3}{Explain CIDR. How does it differ from traditional IP address allocation methods?}

\begin{solutionbox}
\textbf{CIDR (Classless Inter-Domain Routing):}

\begin{itemize}
    \item \keyword{Concept}: Variable length subnet masking
    \item \keyword{Notation}: IP address/prefix length (e.g., 192.168.1.0/24)
    \item \keyword{Flexibility}: Subnets of any size
\end{itemize}

\begin{center}
\captionof{table}{Traditional vs CIDR}
\begin{tabulary}{\linewidth}{|L|L|L|}
\hline
\textbf{Method} & \textbf{Allocation} & \textbf{Efficiency} \\ \hline
Traditional & Fixed class boundaries & Wasteful (Class B = 65,536 IPs) \\ \hline
CIDR & Variable subnet sizes & Efficient allocation \\ \hline
\end{tabulary}
\end{center}

\textbf{Benefits:}
\begin{itemize}
    \item \keyword{Address Conservation}: Reduces IP address waste
    \item \keyword{Route Aggregation}: Summarizes multiple routes
\end{itemize}
\end{solutionbox}

\begin{mnemonicbox}
\mnemonic{CIDR = Classless Intelligent Address Routing}
\end{mnemonicbox}

\questionmarks{3(b OR)}{4}{Describe DSL technology with its types, advantages and limitations.}

\begin{solutionbox}
\textbf{DSL (Digital Subscriber Line):}

\begin{center}
\captionof{table}{DSL Types}
\begin{tabulary}{\linewidth}{|L|L|L|}
\hline
\textbf{Type} & \textbf{Speed} & \textbf{Application} \\ \hline
ADSL & Asymmetric (faster download) & Home users \\ \hline
SDSL & Symmetric (equal up/down) & Business \\ \hline
VDSL & Very high speed & Short distances \\ \hline
\end{tabulary}
\end{center}

\textbf{Advantages:}
\begin{itemize}
    \item \keyword{Always-on Connection}: No dial-up required
    \item \keyword{Existing Infrastructure}: Uses phone lines
    \item \keyword{Cost-effective}: Affordable high-speed access
\end{itemize}

\textbf{Limitations:}
\begin{itemize}
    \item \keyword{Distance Dependent}: Speed decreases with distance
    \item \keyword{Line Quality}: Requires good copper lines
    \item \keyword{Availability}: Not available everywhere
\end{itemize}
\end{solutionbox}

\begin{mnemonicbox}
\mnemonic{DSL = Digital Speed Limited by distance}
\end{mnemonicbox}

\questionmarks{3(c OR)}{7}{Explain error control and flow control at data link layer in detail.}

\begin{solutionbox}
\textbf{Error Control:}

\begin{center}
\begin{tikzpicture}[node distance=2cm]
    \node [gtu block] (data) {Data Transmission};
    \node [gtu block, right=of data] (detect) {Error Detection};
    \node [gtu state, right=of detect] (check) {Error?};
    \node [gtu block, above=of check] (retrans) {Retransmit};
    \node [gtu block, right=of check] (accept) {Accept Data};
    
    \path [gtu arrow] (data) -- (detect);
    \path [gtu arrow] (detect) -- (check);
    \path [gtu arrow] (check) -- node[left, font=\tiny] {Yes} (retrans);
    \path [gtu arrow] (check) -- node[above, font=\tiny] {No} (accept);
    \path [gtu arrow, bend right] (retrans) to (data);
\end{tikzpicture}
\captionof{figure}{Error Control Logic}
\end{center}

\textbf{Error Control Methods:}

\begin{center}
\captionof{table}{Error Control Methods}
\begin{tabulary}{\linewidth}{|L|L|L|}
\hline
\textbf{Method} & \textbf{Technique} & \textbf{Application} \\ \hline
Parity Check & Single bit error detection & Simple systems \\ \hline
Checksum & Mathematical sum verification & TCP/UDP \\ \hline
CRC & Polynomial division & Ethernet, Wi-Fi \\ \hline
ARQ & Automatic Repeat Request & Reliable protocols \\ \hline
\end{tabulary}
\end{center}

\textbf{Flow Control:}

\begin{center}
\begin{tikzpicture}[node distance=2cm]
    \node [gtu state] (sender) {Sender};
    \node [gtu block, right=of sender] (check) {Buffer Check};
    \node [gtu state, right=of check] (full) {Full?};
    \node [gtu block, right=of full] (send) {Send Data};
    \node [gtu block, below=of full] (wait) {Wait};
    
    \path [gtu arrow] (sender) -- (check);
    \path [gtu arrow] (check) -- (full);
    \path [gtu arrow] (full) -- node[above, font=\tiny] {No} (send);
    \path [gtu arrow] (full) -- node[left, font=\tiny] {Yes} (wait);
    \path [gtu arrow, bend left] (wait) to (check);
\end{tikzpicture}
\captionof{figure}{Flow Control Logic}
\end{center}

\textbf{Techniques:}
\begin{itemize}
    \item \keyword{Stop-and-Wait}: Send one frame, wait for ACK
    \item \keyword{Sliding Window}: Multiple frames in transit
    \item \keyword{Buffer Management}: Prevents overflow
\end{itemize}
\end{solutionbox}

\begin{mnemonicbox}
\mnemonic{EF - Error detection, Flow regulation}
\end{mnemonicbox}

\questionmarks{4(a)}{3}{Explain video over IP.}

\begin{solutionbox}
\textbf{Video over IP (VoIP):}

\begin{itemize}
    \item \keyword{Technology}: Transmits video signals over IP networks
    \item \keyword{Digitization}: Converts analog video to digital packets
    \item \keyword{Real-time}: Requires low latency transmission
\end{itemize}

\textbf{Components:}
\begin{itemize}
    \item \keyword{Encoder}: Compresses video data
    \item \keyword{Network}: IP infrastructure for transport
    \item \keyword{Decoder}: Decompresses at destination
\end{itemize}

\textbf{Applications:}
\begin{itemize}
    \item Video Conferencing, Streaming, Surveillance
\end{itemize}
\end{solutionbox}

\begin{mnemonicbox}
\mnemonic{VIP = Video Internet Protocol}
\end{mnemonicbox}

\questionmarks{4(b)}{4}{Explain Electronic-Mail with its protocol.}

\begin{solutionbox}
\textbf{Email System:}

\begin{center}
\begin{tikzpicture}[node distance=1cm]
    \node [gtu state] (user) {User Agent};
    \node [gtu block, right=of user] (smtp) {SMTP Server};
    \node [gtu block, right=1.5cm of smtp] (pop) {POP3/IMAP};
    \node [gtu state, right=of pop] (recipient) {Recipient};
    
    \path [gtu arrow] (user) -- (smtp);
    \path [gtu arrow] (smtp) -- node[above, font=\tiny] {Internet} (pop);
    \path [gtu arrow] (pop) -- (recipient);
\end{tikzpicture}
\captionof{figure}{Email Architecture}
\end{center}

\begin{center}
\captionof{table}{Email Protocols}
\begin{tabulary}{\linewidth}{|L|L|L|L|}
\hline
\textbf{Protocol} & \textbf{Function} & \textbf{Port} & \textbf{Direction} \\ \hline
SMTP & Send/relay messages & 25, 587 & Sending \\ \hline
POP3 & Download messages & 110 & Receiving \\ \hline
IMAP & Server-based access & 143 & Advanced Ret. \\ \hline
\end{tabulary}
\end{center}

\textbf{Message Flow:}
\begin{itemize}
    \item Composition $\rightarrow$ Submission $\rightarrow$ Delivery $\rightarrow$ Retrieval
\end{itemize}
\end{solutionbox}

\begin{mnemonicbox}
\mnemonic{SPI - SMTP sends, POP3/IMAP receives}
\end{mnemonicbox}

\questionmarks{4(c)}{7}{Explain Role of DNS- Domain Name System Describe the process of DNS resolution.}

\begin{solutionbox}
\textbf{DNS Role:}

\begin{itemize}
    \item \keyword{Name Resolution}: Converts domain names to IP addresses
    \item \keyword{Hierarchical System}: Distributed database structure
    \item \keyword{Internet Navigation}: Makes web browsing user-friendly
\end{itemize}

\textbf{DNS Resolution Process:}

\begin{center}
\begin{tikzpicture}[node distance=0.5cm]
    \node [gtu state] (client) {Client};
    \node [gtu block, right=2cm of client] (local) {Local DNS};
    \node [gtu block, right=2cm of local] (root) {Prop. Servers};
    
    \path [gtu arrow] (client) -- node[above, font=\tiny] {1. Query} (local);
    \path [gtu arrow, bend left] (local) -- node[above, font=\tiny] {2. Recursive Query} (root);
    \path [gtu arrow, bend left] (root) -- node[below, font=\tiny] {3. IP Found} (local);
    \path [gtu arrow, bend left] (local) -- node[below, font=\tiny] {4. IP Response} (client);
\end{tikzpicture}
\captionof{figure}{Simplified DNS Resolution}
\end{center}

\textbf{Resolution Steps:}

\begin{enumerate}
    \item \keyword{Local Cache Check}: Check local DNS cache
    \item \keyword{Recursive Query}: Contact local DNS server
    \item \keyword{Root Server}: Get TLD server reference
    \item \keyword{TLD/Auth Server}: Get final IP address
    \item \keyword{Response Return}: IP address returned to client
\end{enumerate}

\textbf{DNS Record Types:}
\begin{itemize}
    \item \keyword{A Record}: IPv4 address
    \item \keyword{AAAA Record}: IPv6 address
    \item \keyword{CNAME}: Alias
    \item \keyword{MX}: Mail Exchange
\end{itemize}
\end{solutionbox}

\begin{mnemonicbox}
\mnemonic{DNS = Directory Name Service}
\end{mnemonicbox}

\questionmarks{4(a OR)}{3}{Explain WWW, HTML.}

\begin{solutionbox}
\textbf{WWW (World Wide Web):}

\begin{itemize}
    \item \keyword{Definition}: Information system of interlinked documents
    \item \keyword{Access}: Through web browsers using HTTP
    \item \keyword{Components}: Web pages, links, URLs
\end{itemize}

\textbf{HTML (HyperText Markup Language):}

\begin{itemize}
    \item \keyword{Purpose}: Standard markup language for web pages
    \item \keyword{Structure}: Tags define document elements
    \item \keyword{Hyperlinks}: Connect different web resources
\end{itemize}

\textbf{Relationship:}
\begin{itemize}
    \item WWW: System/Platform
    \item HTML: Content Format
\end{itemize}
\end{solutionbox}

\begin{mnemonicbox}
\mnemonic{WWW uses HTML for content}
\end{mnemonicbox}

\questionmarks{4(b OR)}{4}{Explain HTTP and FTP.}

\begin{solutionbox}
\begin{center}
\captionof{table}{HTTP vs FTP}
\begin{tabulary}{\linewidth}{|L|L|L|}
\hline
\textbf{Feature} & \textbf{HTTP} & \textbf{FTP} \\ \hline
Purpose & Web page transfer & File transfer \\ \hline
Port & 80 (HTTP), 443 (HTTPS) & 21 (control), 20 (data) \\ \hline
Connection & Stateless & Stateful \\ \hline
Security & HTTPS for security & FTPS for security \\ \hline
\end{tabulary}
\end{center}

\textbf{HTTP (HyperText Transfer Protocol):}
\begin{itemize}
    \item \keyword{Function}: Request-response protocol for web
    \item \keyword{Methods}: GET, POST, PUT, DELETE
    \item \keyword{Stateless}: Each request independent
\end{itemize}

\textbf{FTP (File Transfer Protocol):}
\begin{itemize}
    \item \keyword{Function}: Upload/download files between systems
    \item \keyword{Modes}: Active and Passive
    \item \keyword{Authentication}: Username/password required
\end{itemize}
\end{solutionbox}

\begin{mnemonicbox}
\mnemonic{HF - HTTP for Hypertext, FTP for Files}
\end{mnemonicbox}

\questionmarks{4(c OR)}{7}{Explain TCP and UDP protocol in transport layer in relation to connection oriented and connection less network.}

\begin{solutionbox}
\begin{center}
\captionof{table}{TCP vs UDP}
\begin{tabulary}{\linewidth}{|L|L|L|}
\hline
\textbf{Feature} & \textbf{TCP} & \textbf{UDP} \\ \hline
Connection & Connection-oriented & Connectionless \\ \hline
Reliability & Guaranteed delivery & Best effort \\ \hline
Speed & Slower (overhead) & Faster (minimal overhead) \\ \hline
Header & 20 bytes & 8 bytes \\ \hline
Flow Control & Yes & No \\ \hline
Error Control & Yes & Limited \\ \hline
\end{tabulary}
\end{center}

\textbf{Protocol Diagram:}

\begin{center}
\begin{tikzpicture}[node distance=2cm]
    \node [gtu block] (trans) {Transport Layer};
    \node [gtu block, below left=of trans] (tcp) {TCP};
    \node [gtu block, below right=of trans] (udp) {UDP};
    \node [below=0.1cm of tcp, align=center, font=\footnotesize] {Reliable\\Ordered\\Heavy};
    \node [below=0.1cm of udp, align=center, font=\footnotesize] {Fast\\Unreliable\\Light};
    
    \path [gtu arrow] (trans) -- (tcp);
    \path [gtu arrow] (trans) -- (udp);
\end{tikzpicture}
\captionof{figure}{Transport Protocols}
\end{center}

\textbf{Description:}

\begin{itemize}
    \item \keyword{TCP}: Three-way handshake (SYN, SYN-ACK, ACK), reliable delivery, flow control used for web/email.
    \item \keyword{UDP}: Fire-and-forget, no connection setup, used for streaming/gaming where speed matters.
\end{itemize}
\end{solutionbox}

\begin{mnemonicbox}
\mnemonic{TCP = Thorough, UDP = Ultra-fast}
\end{mnemonicbox}

\questionmarks{5(a)}{3}{Describe Hacking and its related precautions.}

\begin{solutionbox}
\textbf{Hacking Definition:}

\begin{itemize}
    \item \keyword{Unauthorized Access}: Breaking into computer systems
    \item \keyword{Malicious Intent}: Steal, modify, or destroy data
    \item \keyword{Security Breach}: Exploit system vulnerabilities
\end{itemize}

\begin{center}
\captionof{table}{Security Precautions}
\begin{tabulary}{\linewidth}{|L|L|}
\hline
\textbf{Measure} & \textbf{Implementation} \\ \hline
Strong Passwords & Complex, unique passwords \\ \hline
Software Updates & Regular patches and updates \\ \hline
Firewalls & Network access control \\ \hline
Antivirus & Malware detection and removal \\ \hline
Backup & Regular data backups \\ \hline
User Training & Security awareness programs \\ \hline
\end{tabulary}
\end{center}
\end{solutionbox}

\begin{mnemonicbox}
\mnemonic{HSPFAB - Hacking Stopped by Passwords, Firewalls, Antivirus, Backups}
\end{mnemonicbox}

\questionmarks{5(b)}{4}{Explain IPSec architecture.}

\begin{solutionbox}
\textbf{IPSec Architecture:}

\begin{center}
\begin{tikzpicture}[node distance=1.5cm]
    \node [gtu block] (ipsec) {IPSec Suite};
    \node [gtu block, below left=of ipsec] (ah) {AH\\Auth Header};
    \node [gtu block, below right=of ipsec] (esp) {ESP\\Encapsulating Security Payload};
    \node [gtu block, below=of ipsec] (ike) {IKE\\Key Exchange};
    
    \path [gtu arrow] (ipsec) -- (ah);
    \path [gtu arrow] (ipsec) -- (esp);
    \path [gtu arrow] (ipsec) -- (ike);
\end{tikzpicture}
\captionof{figure}{IPSec Components}
\end{center}

\begin{center}
\captionof{table}{IPSec Components}
\begin{tabulary}{\linewidth}{|L|L|}
\hline
\textbf{Component} & \textbf{Function} \\ \hline
AH & Authentication and integrity (No encryption) \\ \hline
ESP & Confidentiality and authentication (Encryption) \\ \hline
SA & Security parameter agreement \\ \hline
IKE & Key management protocol \\ \hline
\end{tabulary}
\end{center}

\textbf{Modes:}
\begin{itemize}
    \item \keyword{Transport Mode}: Protects payload only
    \item \keyword{Tunnel Mode}: Protects transmission (entire packet)
\end{itemize}
\end{solutionbox}

\begin{mnemonicbox}
\mnemonic{AISE - AH, IPSec, SA, ESP}
\end{mnemonicbox}

\questionmarks{5(c)}{7}{Explain network Security topologies.}

\begin{solutionbox}
\textbf{Security Topology:}

\begin{center}
\begin{tikzpicture}[node distance=2cm]
    \node [gtu block] (internet) {Internet};
    \node [gtu block, right=of internet] (firewall) {Firewall};
    \node [gtu block, right=of firewall] (dmz) {DMZ};
    \node [below=0.1cm of dmz, font=\tiny] {Public Servers};
    \node [gtu block, right=of dmz] (internal) {Internal};
    \node [below=0.1cm of internal, font=\tiny] {Private Workstations};
    
    \path [gtu arrow] (internet) -- (firewall);
    \path [gtu arrow] (firewall) -- (dmz);
    \path [gtu arrow] (firewall) -- (internal);
\end{tikzpicture}
\captionof{figure}{Network Security Zones}
\end{center}

\begin{center}
\captionof{table}{Security Zones}
\begin{tabulary}{\linewidth}{|L|L|L|}
\hline
\textbf{Zone} & \textbf{Purpose} & \textbf{Level} \\ \hline
Internet & External untrusted network & Lowest \\ \hline
DMZ & Semi-trusted public services & Medium \\ \hline
Internal & Private trusted network & Highest \\ \hline
\end{tabulary}
\end{center}

\textbf{Principles:}
\begin{itemize}
    \item \keyword{Defense in Depth}: Multiple layers
    \item \keyword{Least Privilege}: Minimum access
    \item \keyword{Segmentation}: Isolate critical systems
\end{itemize}
\end{solutionbox}

\begin{mnemonicbox}
\mnemonic{NST = Network Security Through topology design}
\end{mnemonicbox}

\questionmarks{5(a OR)}{3}{Explain ISO and how it contributes to information security?}

\begin{solutionbox}
\textbf{ISO (International Organization for Standardization):}

\begin{itemize}
    \item \keyword{Global Standards}: Develops international standards
    \item \keyword{ISO 27001}: Standard for Information Security
    \item \keyword{ISMS}: Information Security Management System framework
\end{itemize}

\textbf{Contribution:}
\begin{itemize}
    \item Provides risk management framework, ensures compliance, and establishes best practices for data protection.
\end{itemize}
\end{solutionbox}

\begin{mnemonicbox}
\mnemonic{ISO = International Security Organization}
\end{mnemonicbox}

\questionmarks{5(b OR)}{4}{Give Difference between symmetric and asymmetric encryption algorithms.}

\begin{solutionbox}
\begin{center}
\captionof{table}{Symmetric vs Asymmetric Encryption}
\begin{tabulary}{\linewidth}{|L|L|L|}
\hline
\textbf{Feature} & \textbf{Symmetric} & \textbf{Asymmetric} \\ \hline
Keys & Single shared key & Key pair (Public/Private) \\ \hline
Speed & Fast & Slower \\ \hline
Key Dist. & Difficult & Easier \\ \hline
Scalability & Poor ($n^2$ keys) & Better \\ \hline
Examples & AES, DES & RSA, ECC \\ \hline
\end{tabulary}
\end{center}

\textbf{Hybrid Approach:}
\begin{itemize}
    \item Use Asymmetric for secure key exchange, then Symmetric for data transfer (Best of both).
\end{itemize}
\end{solutionbox}

\begin{mnemonicbox}
\mnemonic{SA = Symmetric Shared, Asymmetric Apart}
\end{mnemonicbox}

\questionmarks{5(c OR)}{7}{Explain Email security with its standards.}

\begin{solutionbox}
\textbf{Email Threats:}

\begin{center}
\begin{tikzpicture}[node distance=1.5cm]
    \node [gtu state] (email) {Email System};
    \node [gtu block, above right=of email] (phish) {Phishing};
    \node [gtu block, right=of email] (spam) {Spam/Malware};
    \node [gtu block, below right=of email] (spoof) {Spoofing};
    
    \path [gtu arrow] (phish) -- (email);
    \path [gtu arrow] (spam) -- (email);
    \path [gtu arrow] (spoof) -- (email);
\end{tikzpicture}
\captionof{figure}{Email Threats}
\end{center}

\begin{center}
\captionof{table}{Email Security Standards}
\begin{tabulary}{\linewidth}{|L|L|L|}
\hline
\textbf{Standard} & \textbf{Purpose} & \textbf{Function} \\ \hline
S/MIME & Content Security & Encryption/Signatures \\ \hline
PGP & Privacy & End-to-end encryption \\ \hline
TLS & Transport Security & Secure transmission \\ \hline
SPF & Sender Auth & Prevent spoofing \\ \hline
DKIM & Integrity & Signature verification \\ \hline
DMARC & Policy & Auth policy enforcement \\ \hline
\end{tabulary}
\end{center}

\textbf{Mechanisms:}
\begin{itemize}
    \item \keyword{Encryption}: Confidentiality
    \item \keyword{Digital Signatures}: Integrity \& Non-repudiation
\end{itemize}
\end{solutionbox}

\begin{mnemonicbox}
\mnemonic{SPTSD = S/MIME, PGP, TLS, SPF, DKIM protect email}
\end{mnemonicbox}

% Question 5
\questionmarks{5(a)}{3}{Describe Hacking and its related precautions.}

\begin{solutionbox}
\textbf{Hacking Definition:}

\begin{itemize}
    \item \keyword{Unauthorized Access}: Breaking into computer systems
    \item \keyword{Malicious Intent}: Steal, modify, or destroy data
    \item \keyword{Security Breach}: Exploit system vulnerabilities
\end{itemize}

\textbf{Types of Hacking:}

\begin{itemize}
    \item \keyword{Ethical Hacking}: Authorized security testing
    \item \keyword{Malicious Hacking}: Criminal activities
    \item \keyword{Social Engineering}: Manipulate human behavior
\end{itemize}

\textbf{Precautions:}

\begin{center}
\captionof{table}{Security Measures}
\begin{tabulary}{\linewidth}{|L|L|}
\hline
\textbf{Security Measure} & \textbf{Implementation} \\ \hline
\textbf{Strong Passwords} & Complex, unique passwords \\ \hline
\textbf{Software Updates} & Regular patches and updates \\ \hline
\textbf{Firewalls} & Network access control \\ \hline
\textbf{Antivirus} & Malware detection and removal \\ \hline
\textbf{Backup} & Regular data backups \\ \hline
\textbf{User Training} & Security awareness programs \\ \hline
\end{tabulary}
\end{center}
\end{solutionbox}

\begin{mnemonicbox}
\mnemonic{HSPFAB - Hacking Stopped by Passwords, Firewalls, Antivirus, Backups}
\end{mnemonicbox}

\questionmarks{5(b)}{4}{Explain IPSec architecture.}

\begin{solutionbox}
\textbf{IPSec (Internet Protocol Security):}

\begin{center}
\begin{tikzpicture}[node distance=1.5cm]
    \node [gtu block] (ipsec) {IPSec Architecture};
    \node [gtu block, below left=of ipsec] (ah) {AH\\(Auth Header)};
    \node [gtu block, below=of ipsec] (esp) {ESP\\(Encapsulating Security)};
    \node [gtu block, below right=of ipsec] (ike) {IKE/SA\\(Key Mgmt)};
    
    \draw [gtu arrow] (ipsec) -- (ah);
    \draw [gtu arrow] (ipsec) -- (esp);
    \draw [gtu arrow] (ipsec) -- (ike);
\end{tikzpicture}
\captionof{figure}{IPSec Architecture Components}
\end{center}

\textbf{IPSec Components:}

\begin{itemize}
    \item \keyword{AH}: Authentication and integrity
    \item \keyword{ESP}: Confidentiality and authentication
    \item \keyword{SA}: Security parameter agreement
    \item \keyword{IKE}: Key management protocol
\end{itemize}

\textbf{Operating Modes:}

\begin{itemize}
    \item \keyword{Transport Mode}: Protects payload only
    \item \keyword{Tunnel Mode}: Protects entire IP packet
\end{itemize}

\textbf{Security Services:}

\begin{itemize}
    \item \keyword{Authentication}: Verify sender identity
    \item \keyword{Integrity}: Ensure data unchanged
    \item \keyword{Confidentiality}: Encrypt data content
    \item \keyword{Anti-replay}: Prevent packet replay attacks
\end{itemize}
\end{solutionbox}

\begin{mnemonicbox}
\mnemonic{AISE - AH, IPSec, SA, ESP}
\end{mnemonicbox}

\questionmarks{5(c)}{7}{Explain network Security topologies.}

\begin{solutionbox}
\textbf{Network Security Topologies:}

\begin{center}
\begin{tikzpicture}[node distance=1.5cm]
    \node [gtu state] (inet) {Internet};
    \node [gtu block, right=of inet] (fw) {Firewall};
    \node [gtu block, above right=of fw, fill=yellow!10] (dmz) {DMZ};
    \node [gtu block, right=of dmz] (web) {Web/Mail Server};
    \node [gtu block, below right=of fw, fill=green!10] (int) {Internal Network};
    \node [gtu block, right=of int] (db) {Database/Workstations};
    
    \draw [gtu arrow, <->] (inet) -- (fw);
    \draw [gtu arrow, <->] (fw) -- (dmz);
    \draw [gtu arrow, <->] (dmz) -- (web);
    \draw [gtu arrow, <->] (fw) -- (int);
    \draw [gtu arrow, <->] (int) -- (db);
\end{tikzpicture}
\captionof{figure}{DMZ Network Topology}
\end{center}

\textbf{Security Zones:}

\begin{center}
\captionof{table}{Security Zones}
\begin{tabulary}{\linewidth}{|L|L|L|}
\hline
\textbf{Zone} & \textbf{Purpose} & \textbf{Security Level} \\ \hline
\textbf{Internet} & External untrusted network & Lowest \\ \hline
\textbf{DMZ} & Semi-trusted public services & Medium \\ \hline
\textbf{Internal} & Private trusted network & Highest \\ \hline
\end{tabulary}
\end{center}

\textbf{Topology Components:}

\begin{itemize}
    \item \keyword{Perimeter Security}: Firewalls, IDS/IPS
    \item \keyword{Network Segmentation}: VLANs, subnets
    \item \keyword{Access Control}: Authentication, authorization
\end{itemize}

\textbf{Security Principles:}

\begin{itemize}
    \item \keyword{Defense in Depth}: Multiple security layers
    \item \keyword{Least Privilege}: Minimum required access
    \item \keyword{Network Isolation}: Separate critical systems
\end{itemize}
\end{solutionbox}

\begin{mnemonicbox}
\mnemonic{NST = Network Security Through topology design}
\end{mnemonicbox}

\questionmarks{5(a OR)}{3}{Explain ISO and how it contributes to information security?}

\begin{solutionbox}
\textbf{ISO (International Organization for Standardization):}

\begin{itemize}
    \item \keyword{Global Standards}: Develops international standards
    \item \keyword{Quality Assurance}: Ensures consistent practices
    \item \keyword{Best Practices}: Provides framework for implementation
\end{itemize}

\textbf{ISO 27001 - Information Security:}

\begin{itemize}
    \item \keyword{ISMS}: Information Security Management System
    \item \keyword{Risk Management}: Systematic approach to security
    \item \keyword{Continuous Improvement}: Regular review and updates
\end{itemize}

\textbf{Benefits:}

\begin{itemize}
    \item \keyword{Standardization}: Common security language
    \item \keyword{Credibility}: International recognition
    \item \keyword{Improvement}: Ongoing security enhancement
\end{itemize}
\end{solutionbox}

\begin{mnemonicbox}
\mnemonic{ISO = International Security Organization}
\end{mnemonicbox}

\questionmarks{5(b OR)}{4}{Give Difference between symmetric and asymmetric encryption algorithms.}

\begin{solutionbox}
\textbf{Encryption Algorithm Comparison:}

\begin{center}
\captionof{table}{Symmetric vs Asymmetric Encryption}
\begin{tabulary}{\linewidth}{|L|L|L|}
\hline
\textbf{Feature} & \textbf{Symmetric} & \textbf{Asymmetric} \\ \hline
\textbf{Keys} & Single shared key & Key pair (public/private) \\ \hline
\textbf{Speed} & Fast & Slower \\ \hline
\textbf{Key Distribution} & Difficult & Easier \\ \hline
\textbf{Scalability} & Poor ($n^2-1$ keys) & Better \\ \hline
\textbf{Security} & Depends on key secrecy & Mathematical complexity \\ \hline
\end{tabulary}
\end{center}

\textbf{Symmetric Encryption:}

\begin{itemize}
    \item \keyword{Process}: Same key encrypts and decrypts
    \item \keyword{Challenge}: Secure key distribution
    \item \keyword{Examples}: AES, DES, 3DES
\end{itemize}

\textbf{Asymmetric Encryption:}

\begin{itemize}
    \item \keyword{Process}: Public key encrypts, private key decrypts
    \item \keyword{Advantage}: No key distribution problem
    \item \keyword{Examples}: RSA, ECC, Diffie-Hellman
\end{itemize}
\end{solutionbox}

\begin{mnemonicbox}
\mnemonic{SA = Symmetric Shared, Asymmetric Apart}
\end{mnemonicbox}

\questionmarks{5(c OR)}{7}{Explain Email security with its standards.}

\begin{solutionbox}
\textbf{Email Security Challenges:}

\begin{center}
\begin{tikzpicture}[node distance=1.5cm]
    \node [gtu block] (threats) {Email Threats};
    \node [gtu block, right=of threats] (phish) {Phishing};
    \node [gtu block, above=of phish] (mal) {Malware};
    \node [gtu block, below=of phish] (spam) {Spam};
    \node [gtu block, right=of phish] (spoof) {Spoofing/Data Loss};
    
    \draw [gtu arrow] (threats) -- (phish);
    \draw [gtu arrow] (threats) |- (mal);
    \draw [gtu arrow] (threats) |- (spam);
    \draw [gtu arrow] (phish) -- (spoof);
\end{tikzpicture}
\captionof{figure}{Email Security Threats}
\end{center}

\textbf{Email Security Standards:}

\begin{center}
\captionof{table}{Security Standards}
\begin{tabulary}{\linewidth}{|L|L|L|}
\hline
\textbf{Standard} & \textbf{Purpose} & \textbf{Function} \\ \hline
\textbf{S/MIME} & Secure email content & Encryption and digital signatures \\ \hline
\textbf{PGP} & Pretty Good Privacy & End-to-end encryption \\ \hline
\textbf{TLS} & Transport security & Secure email transmission \\ \hline
\textbf{SPF} & Sender authentication & Prevent email spoofing \\ \hline
\textbf{DKIM} & Message integrity & Digital signature verification \\ \hline
\textbf{DMARC} & Policy enforcement & Email authentication policy \\ \hline
\end{tabulary}
\end{center}

\textbf{Security Mechanisms:}

\begin{itemize}
    \item \keyword{Encryption}: Protect message content
    \item \keyword{Digital Signatures}: Verify sender identity (Authentication)
    \item \keyword{Integrity}: Ensure message unchanged
\end{itemize}

\textbf{Best Practices:}

\begin{itemize}
    \item \keyword{User Education}: Recognize phishing attempts
    \item \keyword{Gateway Filtering}: Block malicious emails
    \item \keyword{Regular Updates}: Keep security software current
\end{itemize}
\end{solutionbox}

\begin{mnemonicbox}
\mnemonic{SPTSD - S/MIME, PGP, TLS, SPF, DKIM protect email}
\end{mnemonicbox}

\end{document}
