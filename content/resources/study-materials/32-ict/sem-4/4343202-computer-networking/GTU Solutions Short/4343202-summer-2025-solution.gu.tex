\documentclass{article}

% content/resources/templates/preamble.tex
\usepackage[margin=0.6in]{geometry}
\author{Milav Dabgar}
\usepackage{amsmath,amssymb,amsthm}
\usepackage{booktabs}
\usepackage{multirow}
\usepackage{xcolor}
\usepackage{tcolorbox}
\tcbuselibrary{breakable,skins}
\usepackage[colorlinks=true,linkcolor=blue]{hyperref}
\usepackage{titlesec}
\usepackage{enumitem}
\usepackage{tikz}
\usepackage{pgfplots}
\usepackage{circuitikz}
\usepackage[version=4]{mhchem}
\usepackage{longtable}
\usepackage{array}
\usepackage{float}
\usepackage{caption}
\usepackage{listings}

\lstset{
  basicstyle=\small\ttfamily,
  breaklines=true,
  breakatwhitespace=false,
  postbreak=\mbox{\textcolor{red}{$\hookrightarrow$}\space},
  float=false,
  numbers=left,
  numberstyle=\tiny\color{gray},
  numbersep=10pt,
  xleftmargin=2em,
  keywordstyle=\color{blue},
  commentstyle=\color{green!60!black},
  stringstyle=\color{purple},
  backgroundcolor=\color{gray!5},
  showstringspaces=false,
  tabsize=2,
  captionpos=b,
  keepspaces=true,
  columns=flexible
}

\pgfplotsset{compat=1.18}
\usetikzlibrary{shapes,arrows,positioning,calc,patterns,decorations.pathmorphing,decorations.markings,arrows.meta}

% Color scheme
\definecolor{headcolor}{RGB}{0,102,204}
\definecolor{keycolor}{RGB}{220,20,60}
\definecolor{solutioncolor}{RGB}{34,139,34}
\definecolor{mnemoniccolor}{RGB}{148,0,211}
\definecolor{codecolor}{RGB}{0,0,100}

% Spacing
\setlength{\parskip}{3pt}
\setlist[itemize]{nosep}
\setlist[enumerate]{nosep}

% Title formatting
\titleformat{\section}{\Large\bfseries\color{headcolor}}{\thesection}{1em}{}
\titleformat{\subsection}{\large\bfseries\color{headcolor}}{\thesubsection}{1em}{}

% Pandoc tightlist compatibility
\providecommand{\tightlist}{%
  \setlength{\itemsep}{0pt}\setlength{\parskip}{0pt}}

% Pandoc longtable compatibility
\newcounter{none}
\def\thenone{}


% content/resources/templates/gujarati-boxes.tex
\usepackage{fontspec}
\usepackage{polyglossia}

% Set Gujarati as main language (document is primarily in Gujarati)
% Note: gloss-gujarati.ldf doesn't exist in polyglossia, but it will use hyphenation patterns
\setdefaultlanguage{gujarati}
\setotherlanguage{english}

% Configure Gujarati font properly
% Use Language=Default to prevent polyglossia from trying to add language-specific features
% that don't exist for Gujarati, which causes "empty feature" warnings
\newfontfamily\gujaratifont[Script=Gujarati,AutoFakeBold=2.5,AutoFakeSlant=0.3]{Noto Sans Gujarati}
\setmainfont[Script=Gujarati,AutoFakeBold=2.5,AutoFakeSlant=0.3]{Noto Sans Gujarati}
% Use Noto Sans Gujarati for monospace to support Gujarati in text
\setmonofont[Scale=0.9]{Noto Sans Gujarati}

% Configure English to use the same font
\newfontfamily\englishfont[Script=Gujarati,AutoFakeBold=2.5,AutoFakeSlant=0.3]{Noto Sans Gujarati}

% Translations for polyglossia
\gappto\captionsgujarati{
  \renewcommand{\tablename}{કોષ્ટક}
  \renewcommand{\figurename}{આકૃતિ}
}

% Helper for TikZ nodes to ensure Gujarati font
\newcommand{\gu}[1]{{\gujaratifont #1}}

% Custom environments
\newtcolorbox{solutionbox}{
    breakable,
    enhanced,
    colback=solutioncolor!5!white,
    colframe=solutioncolor!75!black,
    fonttitle=\bfseries,
    title=જવાબ
}

\newtcolorbox{solutionboxnobreak}{
 colback=solutioncolor!5!white,
 colframe=solutioncolor!75!black,
 fonttitle=\bfseries,
 title=જવાબ
}

\newtcolorbox{keyformula}{
 breakable,
 enhanced,
 colback=keycolor!5!white,
 colframe=keycolor!75!black,
 fonttitle=\bfseries,
 title=રાસાયણિક સમીકરણ/સૂત્ર
}

\newtcolorbox{mnemonicbox}{
 breakable,
 enhanced,
 colback=mnemoniccolor!5!white,
 colframe=mnemoniccolor!75!black,
 fonttitle=\bfseries,
 title=મેમરી ટ્રીક
}


% Custom commands for GTU solutions
% This file defines semantic commands for consistent formatting

% Question command with automatic formatting
\newcommand{\question}[2]{%
  \section*{Question #1}%
  \textbf{#2}%
}

% OR question variant
\newcommand{\questionor}[2]{%
  \section*{Question #1 OR}%
  \textbf{#2}%
}

% Proper table environment with caption
\newenvironment{answertable}[1]{%
  \begin{table}[htbp]
  \centering
  \caption{#1}
}{%
  \end{table}
}

% Proper figure environment for diagrams
\newenvironment{answerdiagram}[1]{%
  \begin{figure}[htbp]
  \centering
  \caption{#1}
}{%
  \end{figure}
}

% Semantic markup for key terms
\newcommand{\keyword}[1]{\textbf{#1}}
\newcommand{\code}[1]{\texttt{#1}}
\newcommand{\classname}[1]{\texttt{#1}}
\newcommand{\methodname}[1]{\texttt{#1}}

% Proper quotation marks
\newcommand{\mnemonic}[1]{``#1''}


\title{કમ્પ્યુટર નેટવર્કિંગ (4343202) - ઉનાળો 2025 સોલ્યુશન}
\date{મે 17, 2025}

\begin{document}
\maketitle

\questionmarks{1(અ)}{3}{કમ્પ્યુટર નેટવર્કની વિવિધ નેટવર્ક ટોપોલોજીઓની યાદી બનાવો અને કોઈપણ એક સમજાવો.}

\begin{solutionbox}
\textbf{કોષ્ટક: નેટવર્ક ટોપોલોજી}

\begin{center}
\captionof{table}{નેટવર્ક ટોપોલોજી}
\begin{tabulary}{\linewidth}{|L|L|}
\hline
\textbf{ટોપોલોજી} & \textbf{વર્ણન} \\ \hline
\textbf{Star} & સેન્ટ્રલ હબ તમામ ઉપકરણોને જોડે છે \\ \hline
\textbf{Ring} & ઉપકરણો ગોળાકાર ચેઇનમાં જોડાયેલા હોય છે \\ \hline
\textbf{Bus} & સિંગલ કેબલ બેકબોન કનેક્શન \\ \hline
\textbf{Mesh} & દરેક ઉપકરણ દરેક અન્ય સાથે જોડાય છે \\ \hline
\textbf{Tree} & હાયરાર્કિકલ બ્રાન્ચિંગ સ્ટ્રક્ચર \\ \hline
\textbf{Hybrid} & બહુવિધ ટોપોલોજીનું સંયોજન \\ \hline
\end{tabulary}
\end{center}

\textbf{સ્ટાર ટોપોલોજી સમજૂતી:}

\begin{itemize}
    \item \keyword{Central Hub}: તમામ ઉપકરણો એક કેન્દ્રબિંદુ સાથે જોડાય છે
    \item \keyword{Easy Installation}: ઉપકરણો ઉમેરવા/દૂર કરવા સરળ
    \item \keyword{Single Point Failure}: હબ નિષ્ફળતા સમગ્ર નેટવર્કને અસર કરે છે
\end{itemize}

\textbf{ડાયાગ્રામ:}

\begin{center}
\begin{tikzpicture}[node distance=2cm]
    \node [gtu block] (hub) {Hub/Switch};
    \node [gtu block, above=of hub] (pc1) {PC1};
    \node [gtu block, below=of hub] (pc2) {PC2};
    \node [gtu block, left=of hub] (pc3) {PC3};
    \node [gtu block, right=of hub] (pc4) {PC4};
    
    \draw [gtu arrow, <->] (hub) -- (pc1);
    \draw [gtu arrow, <->] (hub) -- (pc2);
    \draw [gtu arrow, <->] (hub) -- (pc3);
    \draw [gtu arrow, <->] (hub) -- (pc4);
\end{tikzpicture}
\captionof{figure}{સ્ટાર ટોપોલોજી}
\end{center}
\end{solutionbox}

\begin{mnemonicbox}
\mnemonic{SRBMTH - Star Ring Bus Mesh Tree Hybrid}
\end{mnemonicbox}

\questionmarks{1(બ)}{4}{LAN, WAN અને MAN ની સરખામણી કરો.}

\begin{solutionbox}
\textbf{સરખામણી કોષ્ટક:}

\begin{center}
\captionof{table}{LAN vs MAN vs WAN}
\begin{tabulary}{\linewidth}{|L|L|L|L|}
\hline
\textbf{પેરામીટર} & \textbf{LAN} & \textbf{MAN} & \textbf{WAN} \\ \hline
\textbf{કવરેજ} & બિલ્ડિંગ/કેમ્પસ & શહેર/મેટ્રોપોલિટન & દેશ/વૈશ્વિક \\ \hline
\textbf{ઝડપ} & ખૂબ વધારે (1-100 Gbps) & ઊંચી (1-100 Mbps) & મધ્યમ (1-100 Mbps) \\ \hline
\textbf{ખર્ચ} & ઓછો & મધ્યમ & વધારે \\ \hline
\textbf{માલિકી} & ખાનગી & જાહેર/ખાનગી & જાહેર \\ \hline
\end{tabulary}
\end{center}

\textbf{મુખ્ય મુદ્દાઓ:}

\begin{itemize}
    \item \keyword{LAN}: નાના વિસ્તારો માટે લોકલ એરિયા નેટવર્ક
    \item \keyword{MAN}: શહેરો માટે મેટ્રોપોલિટન એરિયા નેટવર્ક
    \item \keyword{WAN}: મોટા અંતર માટે વાઈડ એરિયા નેટવર્ક
\end{itemize}
\end{solutionbox}

\begin{mnemonicbox}
\mnemonic{LMW - Local Metropolitan Wide}
\end{mnemonicbox}

\questionmarks{1(ક)}{7}{OSI રેફરન્સ મોડેલનું લેયર્ડ આર્કિટેક્ચર દોરો અને મોડેલના દરેક લેયર દ્વારા પૂરી પાડવામાં આવતી ઓછામાં ઓછી બે સેવાઓ લખો.}

\begin{solutionbox}
\textbf{OSI લેયર્ડ આર્કિટેક્ચર:}

\begin{center}
\begin{tikzpicture}[node distance=0cm, outer sep=0pt]
    \node [draw, rectangle, minimum width=4cm, minimum height=0.8cm, fill=blue!10] (l7) {7. Application Layer};
    \node [draw, rectangle, minimum width=4cm, minimum height=0.8cm, below=of l7] (l6) {6. Presentation Layer};
    \node [draw, rectangle, minimum width=4cm, minimum height=0.8cm, below=of l6] (l5) {5. Session Layer};
    \node [draw, rectangle, minimum width=4cm, minimum height=0.8cm, below=of l5, fill=yellow!10] (l4) {4. Transport Layer};
    \node [draw, rectangle, minimum width=4cm, minimum height=0.8cm, below=of l4, fill=yellow!10] (l3) {3. Network Layer};
    \node [draw, rectangle, minimum width=4cm, minimum height=0.8cm, below=of l3, fill=green!10] (l2) {2. Data Link Layer};
    \node [draw, rectangle, minimum width=4cm, minimum height=0.8cm, below=of l2, fill=green!10] (l1) {1. Physical Layer};
    
    \draw [->, thick] (l7.east) to[out=0, in=0] node[right, font=\tiny] {Data} (l6.east);
    \draw [->, thick] (l6.east) to[out=0, in=0] (l5.east);
    \draw [->, thick] (l5.east) to[out=0, in=0] (l4.east);
    \draw [->, thick] (l4.east) to[out=0, in=0] (l3.east);
    \draw [->, thick] (l3.east) to[out=0, in=0] (l2.east);
    \draw [->, thick] (l2.east) to[out=0, in=0] (l1.east);
\end{tikzpicture}
\captionof{figure}{OSI રેફરન્સ મોડેલ}
\end{center}

\textbf{દરેક લેયર દ્વારા સેવાઓ:}

\begin{center}
\captionof{table}{OSI લેયર્સ અને સેવાઓ}
\begin{tabulary}{\linewidth}{|L|L|}
\hline
\textbf{લેયર} & \textbf{સેવાઓ} \\ \hline
\textbf{Application (7)} & ઇમેઇલ સેવાઓ, ફાઇલ ટ્રાન્સફર \\ \hline
\textbf{Presentation (6)} & ડેટા એન્ક્રિપ્શન, ડેટા કોમ્પ્રેશન \\ \hline
\textbf{Session (5)} & સેશન સ્થાપના, સેશન સમાપ્તિ \\ \hline
\textbf{Transport (4)} & ફ્લો કંટ્રોલ, એરર કરેક્શન \\ \hline
\textbf{Network (3)} & રાઉટિંગ, પાથ નિર્ધારણ \\ \hline
\textbf{Data Link (2)} & ફ્રેમ સિંક્રનાઇઝેશન, એરર ડિટેક્શન \\ \hline
\textbf{Physical (1)} & બીટ ટ્રાન્સમિશન, સિગ્નલ કન્વર્ઝન \\ \hline
\end{tabulary}
\end{center}
\end{solutionbox}

\begin{mnemonicbox}
\mnemonic{All People Seem To Need Data Processing}
\end{mnemonicbox}

\questionmarks{1(ક OR)}{7}{TCP/IP મોડેલના દરેક લેયરને તેના પ્રોટોકોલ સાથે સમજાવો.}

\begin{solutionbox}
\textbf{TCP/IP મોડેલ લેયર્સ:}

\begin{center}
\begin{tikzpicture}[node distance=0cm, outer sep=0pt]
    \node [draw, rectangle, minimum width=4cm, minimum height=0.8cm, fill=blue!10] (l4) {Application Layer};
    \node [draw, rectangle, minimum width=4cm, minimum height=0.8cm, below=of l4, fill=yellow!10] (l3) {Transport Layer};
    \node [draw, rectangle, minimum width=4cm, minimum height=0.8cm, below=of l3, fill=green!10] (l2) {Internet Layer};
    \node [draw, rectangle, minimum width=4cm, minimum height=0.8cm, below=of l2, fill=gray!10] (l1) {Network Access Layer};
    
    \node [right=1cm of l4, align=left, font=\footnotesize] {HTTP, FTP, SMTP, DNS};
    \node [right=1cm of l3, align=left, font=\footnotesize] {TCP, UDP};
    \node [right=1cm of l2, align=left, font=\footnotesize] {IP, ICMP, ARP};
    \node [right=1cm of l1, align=left, font=\footnotesize] {Ethernet, Wi-Fi};
\end{tikzpicture}
\captionof{figure}{TCP/IP મોડેલ}
\end{center}

\begin{center}
\captionof{table}{TCP/IP લેયર્સ અને પ્રોટોકોલ્સ}
\begin{tabulary}{\linewidth}{|L|L|L|}
\hline
\textbf{લેયર} & \textbf{પ્રોટોકોલ્સ} & \textbf{કાર્ય} \\ \hline
\textbf{Application} & HTTP, FTP, SMTP, DNS & યુઝર એપ્લિકેશન્સ \\ \hline
\textbf{Transport} & TCP, UDP & એન્ડ-ટુ-એન્ડ ડિલિવરી \\ \hline
\textbf{Internet} & IP, ICMP, ARP & પેકેટ્સ રાઉટિંગ \\ \hline
\textbf{Network Access} & Ethernet, Wi-Fi & ફિઝિકલ ટ્રાન્સમિશન \\ \hline
\end{tabulary}
\end{center}

\textbf{મુખ્ય વિશેષતાઓ:}

\begin{itemize}
    \item \keyword{સરળ મોડેલ}: OSI ના 7 વિરુદ્ધ માત્ર 4 લેયર્સ
    \item \keyword{પ્રોટોકોલ સ્યુટ}: સંપૂર્ણ નેટવર્કિંગ સોલ્યુશન
    \item \keyword{ઇન્ટરનેટ સ્ટાન્ડર્ડ}: આધુનિક ઇન્ટરનેટનો આધાર
\end{itemize}
\end{solutionbox}

\begin{mnemonicbox}
\mnemonic{ATIN - Application Transport Internet Network}
\end{mnemonicbox}

\questionmarks{2(અ)}{3}{નીચેના નેટવર્ક ઉપકરણોના કાર્યો સમજાવો: Repeater, Hub}

\begin{solutionbox}
\textbf{ઉપકરણ કાર્યો:}

\begin{center}
\captionof{table}{Repeater અન Hub}
\begin{tabulary}{\linewidth}{|L|L|L|}
\hline
\textbf{ઉપકરણ} & \textbf{કાર્ય} & \textbf{લેયર} \\ \hline
\textbf{Repeater} & સિગ્નલ એમ્પ્લીફિકેશન, રેન્જ એક્સટેન્શન & Physical (1) \\ \hline
\textbf{Hub} & સિગ્નલ બ્રોડકાસ્ટિંગ, કોલિઝન ડોમેન શેરિંગ & Physical (1) \\ \hline
\end{tabulary}
\end{center}

\textbf{વિગતો:}

\begin{itemize}
    \item \keyword{Repeater}: લાંબા અંતર પર નબળા સિગ્નલોને ફરીથી ઉત્પન્ન કરે છે
    \item \keyword{Hub}: સ્ટાર ટોપોલોજીમાં બહુવિધ ઉપકરણોને જોડે છે
    \item \keyword{Shared Medium}: બંને સિંગલ કોલિઝન ડોમેન બનાવે છે
\end{itemize}
\end{solutionbox}

\begin{mnemonicbox}
\mnemonic{RH - Repeat Hub signals}
\end{mnemonicbox}

\questionmarks{2(બ)}{4}{નીચેના પદો સમજાવો: 1) FDDI 2) ARP, RARP}

\begin{solutionbox}
\textbf{FDDI (Fiber Distributed Data Interface):}

\begin{itemize}
    \item \keyword{ટેકનોલોજી}: 100 Mbps ફાઈબર ઓપ્ટિક નેટવર્ક
    \item \keyword{ટોપોલોજી}: ફોલ્ટ ટોલરન્સ માટે ડ્યુઅલ રિંગ
    \item \keyword{એપ્લિકેશન}: બેકબોન નેટવર્ક્સ, ઉચ્ચ વિશ્વસનીયતા
\end{itemize}

\textbf{ARP (Address Resolution Protocol):}

\begin{itemize}
    \item \keyword{કાર્ય}: IP એડ્રેસને MAC એડ્રેસ સાથે મેપ કરે છે
    \item \keyword{પ્રક્રિયા}: રિક્વેસ્ટ બ્રોડકાસ્ટ કરે છે, રિપ્લાય મેળવે છે
\end{itemize}

\textbf{RARP (Reverse ARP):}

\begin{itemize}
    \item \keyword{કાર્ય}: MAC એડ્રેસને IP એડ્રેસ સાથે મેપ કરે છે
    \item \keyword{ઉપયોગ}: ડિસ્કલેસ વર્કસ્ટેશન્સ, બૂટ પ્રક્રિયા
\end{itemize}
\end{solutionbox}

\begin{mnemonicbox}
\mnemonic{FAR - FDDI ARP RARP}
\end{mnemonicbox}

\questionmarks{2(ક)}{7}{નેટવર્ક સુરક્ષામાં ફાયરવોલનું કાર્ય સિદ્ધાંતો અને Kerberos-કોન્સેપ્ટ સાથે સમજાવો.}

\begin{solutionbox}
\textbf{ફાયરવોલ કાર્યો:}

\begin{center}
\begin{tikzpicture}[node distance=1.5cm]
    \node [gtu block] (fw) {Firewall};
    \node [gtu state, left=of fw] (inet) {Internet};
    \node [gtu block, right=of fw] (int) {Internal Network};
    
    \node [below=0.5cm of fw, align=center, font=\footnotesize] (actions) {Block Threats\\Allow Traffic\\Log Activity};
    
    \draw [gtu arrow, <->] (inet) -- (fw);
    \draw [gtu arrow, <->] (fw) -- (int);
    \draw [dashed] (fw) -- (actions);
\end{tikzpicture}
\captionof{figure}{ફાયરવોલ ઓપરેશન}
\end{center}

\textbf{ફાયરવોલ સિદ્ધાંતો:}

\begin{itemize}
    \item \keyword{Packet Filtering}: પેકેટ હેડરો તપાસે છે
    \item \keyword{Stateful Inspection}: કનેક્શન સ્ટેટ્સ ટ્રેક કરે છે
    \item \keyword{Application Gateway}: ડીપ પેકેટ ઇન્સ્પેક્શન
\end{itemize}

\textbf{Kerberos કોન્સેપ્ટ:}

\begin{itemize}
    \item \keyword{Authentication Service}: સુરક્ષિત યુઝર વેરિફિકેશન
    \item \keyword{Ticket System}: સમય-મર્યાદિત એક્સેસ ટોકન્સ
    \item \keyword{Three-party Protocol}: ક્લાયંટ, સર્વર, કી ડિસ્ટ્રિબ્યુશન સેન્ટર
\end{itemize}

\textbf{સુરક્ષા લાભો:}

\begin{itemize}
    \item \keyword{Access Control}: અનધિકૃત એક્સેસ અટકાવે છે
    \item \keyword{Network Protection}: આંતરિક સંસાધનોનું રક્ષણ કરે છે
\end{itemize}
\end{solutionbox}

\begin{mnemonicbox}
\mnemonic{FPK - Firewall Protects with Kerberos}
\end{mnemonicbox}

\questionmarks{2(અ OR)}{3}{નીચેના નેટવર્ક ઉપકરણોના કાર્યો સમજાવો: Switch, Router}

\begin{solutionbox}
\textbf{ઉપકરણ કાર્યો:}

\begin{center}
\captionof{table}{Switch vs Router}
\begin{tabulary}{\linewidth}{|L|L|L|}
\hline
\textbf{ઉપકરણ} & \textbf{કાર્ય} & \textbf{લેયર} \\ \hline
\textbf{Switch} & MAC એડ્રેસ લર્નિંગ, ફ્રેમ ફોરવર્ડિંગ & Data Link (2) \\ \hline
\textbf{Router} & IP રાઉટિંગ, પાથ સિલેક્શન & Network (3) \\ \hline
\end{tabulary}
\end{center}

\textbf{વિગતો:}

\begin{itemize}
    \item \keyword{Switch}: પોર્ટ દીઠ અલગ કોલિઝન ડોમેન્સ બનાવે છે
    \item \keyword{Router}: વિવિધ નેટવર્ક્સ જોડે છે, રાઉટિંગ નિર્ણયો લે છે
    \item \keyword{Intelligence}: સ્વિચ MAC શીખે છે, રાઉટર રાઉટિંગ ટેબલ જાળવે છે
\end{itemize}
\end{solutionbox}

\begin{mnemonicbox}
\mnemonic{SR - Switch Routes intelligently}
\end{mnemonicbox}

\questionmarks{2(બ OR)}{4}{નીચેના પદો સમજાવો: 1) CDDI 2) DHCP અને BOOTP}

\begin{solutionbox}
\textbf{CDDI (Copper Distributed Data Interface):}

\begin{itemize}
    \item \keyword{ટેકનોલોજી}: કોપર કેબલ્સ પર FDDI
    \item \keyword{ઝડપ}: ટ્વિસ્ટેડ પેર પર 100 Mbps
    \item \keyword{ખર્ચ}: ફાઈબર FDDI નો સસ્તો વિકલ્પ
\end{itemize}

\textbf{DHCP (Dynamic Host Configuration Protocol):}

\begin{itemize}
    \item \keyword{કાર્ય}: ઓટોમેટિક IP એડ્રેસ અસાઇનમેન્ટ
    \item \keyword{પ્રક્રિયા}: ડિસ્કવર, ઓફર, રિક્વેસ્ટ, એકનોલેજ
    \item \keyword{લાભો}: સેન્ટ્રલાઇઝડ IP મેનેજમેન્ટ
\end{itemize}

\textbf{BOOTP (Bootstrap Protocol):}

\begin{itemize}
    \item \keyword{કાર્ય}: ડિસ્કલેસ ક્લાયંટ્સ માટે નેટવર્ક બૂટસ્ટ્રેપ
    \item \keyword{Static}: ફિક્સ IP એડ્રેસ અસાઇનમેન્ટ
    \item \keyword{Predecessor}: DHCP નું જૂનું વર્ઝન
\end{itemize}
\end{solutionbox}

\begin{mnemonicbox}
\mnemonic{CDB - CDDI DHCP BOOTP}
\end{mnemonicbox}

\questionmarks{2(ક OR)}{7}{Software defined network (SDN) તેના આર્કિટેક્ચર, એપ્લિકેશન, ફાયદા અને મર્યાદા સાથે સમજાવો.}

\begin{solutionbox}
\textbf{SDN આર્કિટેક્ચર:}

\begin{center}
\begin{tikzpicture}[node distance=1.5cm]
    \node [gtu block, minimum width=4cm, fill=blue!10] (app) {Application Plane\\(Network Services)};
    \node [gtu block, minimum width=4cm, below=of app, fill=yellow!10] (ctrl) {Control Plane\\(SDN Controller)};
    \node [gtu block, minimum width=4cm, below=of ctrl, fill=green!10] (data) {Data Plane\\(Switches/Routers)};
    
    \draw [gtu arrow, <->] (app) -- node[right, font=\tiny] {Northbound API} (ctrl);
    \draw [gtu arrow, <->] (ctrl) -- node[right, font=\tiny] {Southbound API (OpenFlow)} (data);
    
    \node [right=0.5cm of ctrl, align=left, font=\footnotesize] (funcs) {Traffic Engineering\\Security Policies};
    \draw [dashed] (ctrl) -- (funcs);
\end{tikzpicture}
\captionof{figure}{SDN આર્કિટેક્ચર}
\end{center}

\begin{itemize}
    \item \keyword{Control Plane}: સેન્ટ્રલાઇઝડ નેટવર્ક ઇન્ટેલિજન્સ
    \item \keyword{Data Plane}: પેકેટ ફોરવર્ડિંગ ઉપકરણો
    \item \keyword{Application Plane}: નેટવર્ક એપ્લિકેશન્સ અને સેવાઓ
\end{itemize}

\textbf{એપ્લિકેશન્સ:}

\begin{itemize}
    \item \keyword{Cloud Computing}: ડાયનેમિક રિસોર્સ ફાળવણી
    \item \keyword{Network Virtualization}: બહુવિધ વર્ચ્યુઅલ નેટવર્ક્સ
    \item \keyword{Traffic Engineering}: ઓપ્ટિમાઇઝડ પેથ સિલેક્શન
\end{itemize}

\textbf{ફાયદા:}

\begin{itemize}
    \item \keyword{Centralized Control}: સરળ નેટવર્ક મેનેજમેન્ટ
    \item \keyword{Programmability}: કસ્ટમ નેટવર્ક બિહેવિયર્સ
    \item \keyword{Flexibility}: ઝડપી સર્વિસ ડિપ્લોયમેન્ટ
\end{itemize}

\textbf{મર્યાદાઓ:}

\begin{itemize}
    \item \keyword{Single Point Failure}: કંટ્રોલર પર નિર્ભરતા
    \item \keyword{Scalability}: પરફોર્મન્સ બોટલનેક્સ
    \item \keyword{Security}: નવા હુમલા વેક્ટર્સ
\end{itemize}
\end{solutionbox}

\begin{mnemonicbox}
\mnemonic{SCAP - Software Control Application Programmable}
\end{mnemonicbox}

% Question 3
\questionmarks{3(a)}{3}{નીચેના IP એડ્રેસનો ક્લાસ શોધો.\\ 1) 01111000 00001111 10101010 11000000\\ 2) 11101000 01010101 11111111 11000011}

\begin{solutionbox}
\textbf{IP એડ્રેસ વર્ગીકરણ:}

\begin{center}
\captionof{table}{IP ક્લાસ વિશ્લેષણ}
\begin{tabulary}{\linewidth}{|L|L|L|L|}
\hline
\textbf{બાઈનરી એડ્રેસ} & \textbf{ડેસિમલ} & \textbf{પ્રથમ ઓક્ટેટ} & \textbf{ક્લાસ} \\ \hline
01111000... & 120.15.170.192 & 120 (64-127) & \textbf{Class A} \\ \hline
11101000... & 232.85.255.195 & 232 (224-239) & \textbf{Class D} \\ \hline
\end{tabulary}
\end{center}

\textbf{ક્લાસ રેન્જ:}

\begin{itemize}
    \item \keyword{Class A}: 1-126 (0xxxxxxx)
    \item \keyword{Class B}: 128-191 (10xxxxxx)
    \item \keyword{Class C}: 192-223 (110xxxxx)
    \item \keyword{Class D}: 224-239 (1110xxxx)
\end{itemize}

\textbf{પરિણામો:}

\begin{itemize}
    \item \keyword{First IP}: Class A (Unicast)
    \item \keyword{Second IP}: Class D (Multicast)
\end{itemize}
\end{solutionbox}

\begin{mnemonicbox}
\mnemonic{ABCD - A(1-126) B(128-191) C(192-223) D(224-239)}
\end{mnemonicbox}

\questionmarks{3(બ)}{4}{IPv4 અને IPv6 વચ્ચે તફાવત આપો.}

\begin{solutionbox}
\textbf{IPv4 vs IPv6 સરખામણી:}

\begin{center}
\captionof{table}{IPv4 vs IPv6}
\begin{tabulary}{\linewidth}{|L|L|L|}
\hline
\textbf{વિશેષતા} & \textbf{IPv4} & \textbf{IPv6} \\ \hline
\textbf{એડ્રેસ લંબાઈ} & 32 bits & 128 bits \\ \hline
\textbf{એડ્રેસ ફોર્મેટ} & ડોટેડ ડેસિમલ & હેક્સાડેસિમલ \\ \hline
\textbf{એડ્રેસ સ્પેસ} & 4.3 બિલિયન & 340 અનડેસિલિયન \\ \hline
\textbf{હેડર સાઇઝ} & વેરિએબલ (20-60 બાઇટ્સ) & ફિક્સ્ડ (40 બાઇટ્સ) \\ \hline
\textbf{સુરક્ષા} & વૈકલ્પિક (IPSec) & બિલ્ટ-ઇન (IPSec) \\ \hline
\textbf{કન્ફિગરેશન} & મેન્યુઅલ/DHCP & ઓટો-કન્ફિગરેશન \\ \hline
\end{tabulary}
\end{center}

\textbf{મુખ્ય તફાવતો:}

\begin{itemize}
    \item \keyword{Addressing}: IPv6 વિશાળ માત્રામાં એડ્રેસ પૂરા પાડે છે
    \item \keyword{Security}: IPv6 માં ફરજિયાત સુરક્ષા સુવિધાઓ છે
    \item \keyword{Performance}: IPv6 માં સરળ હેડર સ્ટ્રક્ચર છે
\end{itemize}
\end{solutionbox}

\begin{mnemonicbox}
\mnemonic{IPv4 to IPv6 = More addresses, Better security}
\end{mnemonicbox}

\questionmarks{3(ક)}{7}{Static અને Dynamic Routing એલ્ગોરિધમ્સ સમજાવો.}

\begin{solutionbox}
\textbf{Static Routing:}

\begin{center}
\begin{tikzpicture}[node distance=1.5cm]
    \node [gtu block] (admin) {Administrator};
    \node [gtu block, right=of admin] (entry) {Manual Route Entry};
    \node [gtu block, right=of entry] (table) {Routing Table};
    
    \draw [gtu arrow] (admin) -- (entry);
    \draw [gtu arrow] (entry) -- (table);
\end{tikzpicture}
\end{center}

\textbf{Dynamic Routing:}

\begin{center}
\begin{tikzpicture}[node distance=1.5cm]
    \node [gtu block] (proto) {Routing Protocol};
    \node [gtu block, right=of proto] (disc) {Route Discovery};
    \node [gtu block, right=of disc] (update) {Auto Updates};
    \node [gtu block, right=of update] (table) {Adaptive Table};
    
    \draw [gtu arrow] (proto) -- (disc);
    \draw [gtu arrow] (disc) -- (update);
    \draw [gtu arrow] (update) -- (table);
\end{tikzpicture}
\captionof{figure}{Static vs Dynamic Routing}
\end{center}

\textbf{સરખામણી કોષ્ટક:}

\begin{center}
\captionof{table}{Static vs Dynamic Routing}
\begin{tabulary}{\linewidth}{|L|L|L|}
\hline
\textbf{પાસું} & \textbf{Static Routing} & \textbf{Dynamic Routing} \\ \hline
\textbf{કન્ફિગરેશન} & મેન્યુઅલ સેટઅપ & ઓટોમેટિક ડિસ્કવરી \\ \hline
\textbf{અનુકૂલનક્ષમતા} & કોઈ અનુકૂલન નથી & ફેરફારોને અનુકૂળ થાય છે \\ \hline
\textbf{સંસાધન વપરાશ} & ઓછો CPU/મેમરી & વધુ CPU/મેમરી \\ \hline
\textbf{સ્કેલેબિલિટી} & મોટા નેટવર્ક્સ માટે નબળું & મોટા નેટવર્ક્સ માટે સારું \\ \hline
\textbf{પ્રોટોકોલ્સ} & કોઈ જરૂર નથી & RIP, OSPF, BGP \\ \hline
\end{tabulary}
\end{center}

\textbf{એપ્લિકેશન્સ:}

\begin{itemize}
    \item \keyword{Static}: નાના નેટવર્ક્સ, ચોક્કસ પાથ
    \item \keyword{Dynamic}: મોટા નેટવર્ક્સ, ફોલ્ટ ટોલરન્સ
\end{itemize}
\end{solutionbox}

\begin{mnemonicbox}
\mnemonic{SD - Static=Simple, Dynamic=Automatic}
\end{mnemonicbox}

\questionmarks{3(અ OR)}{3}{CIDR સમજાવો. તે પરંપરાગત IP એડ્રેસ એલોકેશન પદ્ધતિઓથી કેવી રીતે અલગ પડે છે?}

\begin{solutionbox}
\textbf{CIDR (Classless Inter-Domain Routing):}

\begin{itemize}
    \item \keyword{Concept}: વેરિએબલ લેન્થ સબનેટ માસ્કિંગ
    \item \keyword{Notation}: IP એડ્રેસ/પ્રિફિક્સ લંબાઈ (દા.ત., 192.168.1.0/24)
    \item \keyword{Flexibility}: કોઈપણ કદના સબનેટ
\end{itemize}

\textbf{Traditional vs CIDR:}

\begin{center}
\captionof{table}{Traditional vs CIDR}
\begin{tabulary}{\linewidth}{|L|L|L|}
\hline
\textbf{પદ્ધતિ} & \textbf{એલોકેશન} & \textbf{કાર્યક્ષમતા} \\ \hline
\textbf{Traditional} & ફિક્સ્ડ ક્લાસ બાઉન્ડ્રીઝ & બગાડ (Class B = 65,536 IPs) \\ \hline
\textbf{CIDR} & વેરિએબલ સબનેટ સાઈઝ & કાર્યક્ષમ એલોકેશન \\ \hline
\end{tabulary}
\end{center}

\textbf{લાભો:}

\begin{itemize}
    \item \keyword{Address Conservation}: IP એડ્રેસનો બગાડ ઘટાડે છે
    \item \keyword{Route Aggregation}: બહુવિધ રૂટ્સનો સારાંશ આપે છે
\end{itemize}
\end{solutionbox}

\begin{mnemonicbox}
\mnemonic{CIDR = Classless Intelligent Address Routing}
\end{mnemonicbox}

\questionmarks{3(બ OR)}{4}{DSL ટેકનોલોજી તેના પ્રકારો, ફાયદા અને મર્યાદાઓ સાથે વર્ણવો.}

\begin{solutionbox}
\textbf{DSL (Digital Subscriber Line):}

\begin{itemize}
    \item \keyword{Technology}: ટેલિફોન લાઇન પર હાઇ-સ્પીડ ઇન્ટરનેટ
    \item \keyword{Frequency}: વોઇસ કરતા ઉચ્ચ ફ્રીક્વન્સીનો ઉપયોગ કરે છે
\end{itemize}

\textbf{DSL પ્રકારો:}

\begin{center}
\captionof{table}{DSL પ્રકારો}
\begin{tabulary}{\linewidth}{|L|L|L|}
\hline
\textbf{પ્રકાર} & \textbf{ઝડપ} & \textbf{એપ્લિકેશન} \\ \hline
\textbf{ADSL} & અસમપ્રમાણ (ઝડપી ડાઉનલોડ) & હોમ યુઝર્સ \\ \hline
\textbf{SDSL} & સમપ્રમાણ (સમાન અપ/ડાઉન) & બિઝનેસ \\ \hline
\textbf{VDSL} & ખૂબ ઊંચી ઝડપ & ટૂંકા અંતર \\ \hline
\end{tabulary}
\end{center}

\textbf{ફાયદા:}

\begin{itemize}
    \item \keyword{Always-on Connection}: ડાયલ-અપની જરૂર નથી
    \item \keyword{Existing Infrastructure}: ફોન લાઇનનો ઉપયોગ કરે છે
    \item \keyword{Cost-effective}: સસ્તું હાઇ-સ્પીડ એક્સેસ
\end{itemize}

\textbf{મર્યાદાઓ:}

\begin{itemize}
    \item \keyword{Distance Dependent}: અંતર સાથે ઝડપ ઘટે છે
    \item \keyword{Line Quality}: સારી કોપર લાઇન જરૂરી છે
    \item \keyword{Availability}: દરેક જગ્યાએ ઉપલબ્ધ નથી
\end{itemize}
\end{solutionbox}

\begin{mnemonicbox}
\mnemonic{DSL = Digital Speed Limited by distance}
\end{mnemonicbox}

\questionmarks{3(ક OR)}{7}{ડેટા લિંક લેયર પર એરર કંટ્રોલ અને ફ્લો કંટ્રોલ વિગતવાર સમજાવો.}

\begin{solutionbox}
\textbf{Error Control:}

\begin{center}
\begin{tikzpicture}[node distance=1.5cm, auto]
    \node [gtu block] (data) {Data Transmit};
    \node [gtu decision, right=of data, align=center] (check) {Error\\Found?};
    \node [gtu block, below=of check] (req) {Retransmit};
    \node [gtu block, right=of check] (accept) {Accept Data};
    
    \draw [gtu arrow] (data) -- (check);
    \draw [gtu arrow] (check) -- node {હા} (req);
    \draw [gtu arrow] (check) -- node {ના} (accept);
    \draw [gtu arrow] (req) -| (data);
\end{tikzpicture}
\captionof{figure}{એરર કંટ્રોલ લોજિક}
\end{center}

\textbf{પદ્ધતિઓ:}
\begin{center}
\captionof{table}{એરર કંટ્રોલ પદ્ધતિઓ}
\begin{tabulary}{\linewidth}{|L|L|L|}
\hline
\textbf{પદ્ધતિ} & \textbf{ટેકનિક} & \textbf{એપ્લિકેશન} \\ \hline
\textbf{Parity Check} & સિંગલ બીટ એરર ડિટેક્શન & સરળ સિસ્ટમો \\ \hline
\textbf{Checksum} & મેથેમેટિકલ સમ વેરિફિકેશન & TCP/UDP \\ \hline
\textbf{CRC} & પોલીનોમિયલ ડિવિઝન & Ethernet, Wi-Fi \\ \hline
\textbf{ARQ} & ઓટોમેટિક રિપીટ રિક્વેસ્ટ & વિશ્વસનીય પ્રોટોકોલ્સ \\ \hline
\end{tabulary}
\end{center}

\textbf{Flow Control:}

\begin{center}
\begin{tikzpicture}[node distance=1.5cm, auto]
    \node [gtu block] (check) {Check Buffer};
    \node [gtu decision, right=of check, align=center] (full) {Buffer\\Full?};
    \node [gtu block, below=of full] (wait) {Wait};
    \node [gtu block, right=of full] (send) {Send Data};
    
    \draw [gtu arrow] (check) -- (full);
    \draw [gtu arrow] (full) -- node {હા} (wait);
    \draw [gtu arrow] (full) -- node {ના} (send);
    \draw [gtu arrow] (wait) -| (check);
\end{tikzpicture}
\captionof{figure}{ફ્લો કંટ્રોલ લોજિક}
\end{center}

\textbf{ટેકનિકસ:}

\begin{itemize}
    \item \keyword{Stop-and-Wait}: એક ફ્રેમ મોકલો, ACK માટે રાહ જુઓ
    \item \keyword{Sliding Window}: ટ્રાન્ઝિટમાં બહુવિધ ફ્રેમ્સ
    \item \keyword{Buffer Management}: ઓવરફ્લો અટકાવે છે
\end{itemize}
\end{solutionbox}

\begin{mnemonicbox}
\mnemonic{EF - Error detection, Flow regulation}
\end{mnemonicbox}

% Question 4
\questionmarks{4(અ)}{3}{Video over IP સમજાવો.}

\begin{solutionbox}
\textbf{Video over IP (VoIP):}

\begin{itemize}
    \item \keyword{Technology}: IP નેટવર્ક્સ પર વિડિઓ સિગ્નલો ટ્રાન્સમિટ કરે છે
    \item \keyword{Digitization}: એનાલોગ વિડિઓને ડિજિટલ પેકેટ્સમાં કન્વર્ટ કરે છે
    \item \keyword{Real-time}: લો લેટન્સી ટ્રાન્સમિશનની જરૂર છે
\end{itemize}

\textbf{ઘટકો:}

\begin{itemize}
    \item \keyword{Encoder}: વિડિઓ ડેટા કોમ્પ્રેસ કરે છે
    \item \keyword{Network}: ટ્રાન્સપોર્ટ માટે IP ઇન્ફ્રાસ્ટ્રક્ચર
    \item \keyword{Decoder}: ડેસ્ટિનેશન પર ડિકોમ્પ્રેસ કરે છે
\end{itemize}

\textbf{એપ્લિકેશન્સ:}

\begin{itemize}
    \item \keyword{Video Conferencing}: બિઝનેસ કોમ્યુનિકેશન્સ
    \item \keyword{Streaming}: મનોરંજન સેવાઓ
    \item \keyword{Surveillance}: સુરક્ષા સિસ્ટમો
\end{itemize}
\end{solutionbox}

\begin{mnemonicbox}
\mnemonic{VIP = Video Internet Protocol}
\end{mnemonicbox}

\questionmarks{4(બ)}{4}{Electronic-Mail તેના પ્રોટોકોલ સાથે સમજાવો.}

\begin{solutionbox}
\textbf{ઇમેઇલ સિસ્ટમ ઘટકો:}

\begin{center}
\begin{tikzpicture}[node distance=1.5cm]
    \node [gtu block] (ua) {User Agent};
    \node [gtu block, right=of ua] (smtp) {SMTP Server};
    \node [gtu state, right=of smtp] (inet) {Internet};
    \node [gtu block, right=of inet] (pop) {POP3/IMAP};
    \node [gtu block, below=of pop] (recip) {Recipient};
    
    \draw [gtu arrow] (ua) -- (smtp);
    \draw [gtu arrow] (smtp) -- (inet);
    \draw [gtu arrow] (inet) -- (pop);
    \draw [gtu arrow] (pop) -- (recip);
\end{tikzpicture}
\captionof{figure}{ઇમેઇલ આર્કિટેક્ચર}
\end{center}

\textbf{ઇમેઇલ પ્રોટોકોલ્સ:}

\begin{center}
\captionof{table}{ઇમેઇલ પ્રોટોકોલ્સ}
\begin{tabulary}{\linewidth}{|L|L|L|}
\hline
\textbf{પ્રોટોકોલ} & \textbf{કાર્ય} & \textbf{પોર્ટ} \\ \hline
\textbf{SMTP} & મેસેજ મોકલે/રીલે કરે છે & 25, 587 \\ \hline
\textbf{POP3} & મેસેજ ડાઉનલોડ કરે છે & 110 \\ \hline
\textbf{IMAP} & સર્વર-આધારિત એક્સેસ & 143 \\ \hline
\end{tabulary}
\end{center}

\textbf{મેસેજ ફ્લો:}

\begin{itemize}
    \item \keyword{Composition}: યુઝર મેસેજ બનાવે છે
    \item \keyword{Submission}: SMTP સર્વર પર મોકલે છે
    \item \keyword{Delivery}: સર્વર રેસિપિઅન્ટને ફોરવર્ડ કરે છે
    \item \keyword{Retrieval}: POP3/IMAP મેસેજ ડાઉનલોડ કરે છે
\end{itemize}
\end{solutionbox}

\begin{mnemonicbox}
\mnemonic{SPI - SMTP sends, POP3/IMAP receives}
\end{mnemonicbox}

\questionmarks{4(ક)}{7}{DNS - Domain Name System નો રોલ સમજાવો અને DNS રિઝોલ્યુશન પ્રક્રિયા વર્ણવો.}

\begin{solutionbox}
\textbf{DNS રોલ:}

\begin{itemize}
    \item \keyword{Name Resolution}: ડોમેન નામોને IP એડ્રેસમાં કન્વર્ટ કરે છે
    \item \keyword{Hierarchical System}: ડિસ્ટ્રિબ્યુટેડ ડેટાબેઝ સ્ટ્રક્ચર
    \item \keyword{Internet Navigation}: વેબ બ્રાઉઝિંગને યુઝર-ફ્રેન્ડલી બનાવે છે
\end{itemize}

\textbf{DNS રિઝોલ્યુશન પ્રક્રિયા:}

\begin{center}
\begin{tikzpicture}[node distance=3cm, auto]
    \node [gtu block] (client) {Client};
    \node [gtu block, right=of client] (local) {Local DNS};
    \node [gtu block, below=of client] (root) {Root Server};
    \node [gtu block, right=of root] (tld) {TLD (.com)};
    \node [gtu block, right=of tld] (auth) {Auth (ex.com)};
    
    \draw [->, dashed] (client) -- node[above] {1. Query} (local);
    \draw [->] (local) -- node[left] {2} (root);
    \draw [->] (root) -- node[right] {3. Ref} (local);
    \draw [->] (local) -- node[left] {4} (tld);
    \draw [->] (tld) -- node[right] {5. Ref} (local);
    \draw [->] (local) -- node[right] {6} (auth);
    \draw [->] (auth) -- node[right] {7. IP} (local);
    \draw [->, dashed] (local) -- node[below] {8. IP} (client);
\end{tikzpicture}
\captionof{figure}{ઇટરેટિવ DNS રિઝોલ્યુશન}
\end{center}

\textbf{રિઝોલ્યુશન સ્ટેપ્સ:}

\begin{enumerate}
    \item \textbf{Local Cache Check}: લોકલ DNS કેશ તપાસો
    \item \textbf{Recursive Query}: લોકલ DNS સર્વરનો સંપર્ક કરો
    \item \textbf{Root Server}: TLD સર્વર રેફરન્સ મેળવો
    \item \textbf{TLD Server}: ઓથોરિટેટિવ સર્વર રેફરન્સ મેળવો
    \item \textbf{Authoritative Server}: ફાઇનલ IP એડ્રેસ મેળવો
    \item \textbf{Response Return}: ક્લાયંટને IP એડ્રેસ પરત કરવામાં આવે છે
\end{enumerate}

\textbf{DNS રેકોર્ડ પ્રકારો:}

\begin{itemize}
    \item \keyword{A Record}: નામ ને IPv4 એડ્રેસ સાથે મેપ કરે છે
    \item \keyword{AAAA Record}: નામ ને IPv6 એડ્રેસ સાથે મેપ કરે છે
    \item \keyword{CNAME}: કેનોનિકલ નામ ઉપનામ
    \item \keyword{MX}: મેઇલ એક્સચેન્જ સર્વર
\end{itemize}
\end{solutionbox}

\begin{mnemonicbox}
\mnemonic{DNS = Directory Name Service}
\end{mnemonicbox}

\questionmarks{4(અ OR)}{3}{WWW અને HTML સમજાવો.}

\begin{solutionbox}
\textbf{WWW (World Wide Web):}

\begin{itemize}
    \item \keyword{Definition}: ઇન્ટરલિંક થયેલા દસ્તાવેજોની માહિતી સિસ્ટમ
    \item \keyword{Access}: HTTP નો ઉપયોગ કરીને વેબ બ્રાઉઝર્સ દ્વારા
    \item \keyword{Components}: વેબ પેજીસ, લિંક્સ, URLs
\end{itemize}

\textbf{HTML (HyperText Markup Language):}

\begin{itemize}
    \item \keyword{Purpose}: વેબ પેજીસ માટે સ્ટાન્ડર્ડ માર્કઅપ લેંગ્વેજ
    \item \keyword{Structure}: ટૅગ્સ દસ્તાવેજ તત્વોને વ્યાખ્યાયિત કરે છે
    \item \keyword{Hyperlinks}: વિવિધ વેબ સંસાધનોને જોડે છે
\end{itemize}

\textbf{Relationship:}

\begin{itemize}
    \item \keyword{WWW}: સિસ્ટમ/પ્લેટફોર્મ
    \item \keyword{HTML}: કન્ટેન્ટ ફોર્મેટ
    \item \keyword{Integration}: HTML WWW કન્ટેન્ટ બનાવે છે
\end{itemize}
\end{solutionbox}

\begin{mnemonicbox}
\mnemonic{WWW uses HTML for content}
\end{mnemonicbox}

\questionmarks{4(બ OR)}{4}{HTTP અને FTP સમજાવો.}

\begin{solutionbox}
\textbf{પ્રોટોકોલ સરખામણી:}

\begin{center}
\captionof{table}{HTTP vs FTP}
\begin{tabulary}{\linewidth}{|L|L|L|}
\hline
\textbf{વિશેષતા} & \textbf{HTTP} & \textbf{FTP} \\ \hline
\textbf{હેતુ} & વેબ પેજ ટ્રાન્સફર & ફાઇલ ટ્રાન્સફર \\ \hline
\textbf{પોર્ટ} & 80 (HTTP), 443 (HTTPS) & 21 (control), 20 (data) \\ \hline
\textbf{કનેક્શન} & સ્ટેટલેસ & સ્ટેટફુલ \\ \hline
\textbf{સુરક્ષા} & સુરક્ષા માટે HTTPS & સુરક્ષા માટે FTPS \\ \hline
\end{tabulary}
\end{center}

\textbf{HTTP (HyperText Transfer Protocol):}

\begin{itemize}
    \item \keyword{Function}: વેબ માટે રિક્વેસ્ટ-રિસ્પોન્સ પ્રોટોકોલ
    \item \keyword{Methods}: GET, POST, PUT, DELETE
    \item \keyword{Stateless}: દરેક રિક્વેસ્ટ સ્વતંત્ર
\end{itemize}

\textbf{FTP (File Transfer Protocol):}

\begin{itemize}
    \item \keyword{Function}: સિસ્ટમો વચ્ચે ફાઇલો અપલોડ/ડાઉનલોડ
    \item \keyword{Modes}: એક્ટિવ અને પેસિવ
    \item \keyword{Authentication}: યુઝરનેમ/પાસવર્ડ જરૂરી
\end{itemize}
\end{solutionbox}

\begin{mnemonicbox}
\mnemonic{HF - HTTP for Hypertext, FTP for Files}
\end{mnemonicbox}

\questionmarks{4(ક OR)}{7}{કનેક્શન ઓરિએન્ટેડ અને કનેક્શન લેસ નેટવર્કના સંદર્ભમાં ટ્રાન્સપોર્ટ લેયરમાં TCP અને UDP પ્રોટોકોલ સમજાવો.}

\begin{solutionbox}
\textbf{ટ્રાન્સપોર્ટ લેયર પ્રોટોકોલ્સ:}

\begin{center}
\begin{tikzpicture}[node distance=1.5cm]
    \node [gtu block] (trans) {Transport Layer};
    \node [gtu block, below left=of trans] (tcp) {TCP\\(Connection Oriented)};
    \node [gtu block, below right=of trans] (udp) {UDP\\(Connectionless)};
    \node [gtu block, below=of tcp] (rel) {Reliable Delivery};
    \node [gtu block, below=of udp] (fast) {Fast Delivery};
    
    \draw [gtu arrow] (trans) -- (tcp);
    \draw [gtu arrow] (trans) -- (udp);
    \draw [gtu arrow] (tcp) -- (rel);
    \draw [gtu arrow] (udp) -- (fast);
\end{tikzpicture}
\captionof{figure}{TCP vs UDP ઝાંખી}
\end{center}

\textbf{પ્રોટોકોલ સરખામણી:}

\begin{center}
\captionof{table}{TCP vs UDP}
\begin{tabulary}{\linewidth}{|L|L|L|}
\hline
\textbf{વિશેષતા} & \textbf{TCP} & \textbf{UDP} \\ \hline
\textbf{કનેક્શન} & કનેક્શન-ઓરિએન્ટેડ & કનેક્શનલેસ \\ \hline
\textbf{વિશ્વસનીયતા} & ગેરંટીડ ડિલિવરી & બેસ્ટ પ્રયાસ \\ \hline
\textbf{ઝડપ} & ધીમી (ઓવરહેડ) & ઝડપી (ન્યૂનતમ ઓવરહેડ) \\ \hline
\textbf{હેડર સાઇઝ} & 20 બાઇટ્સ & 8 બાઇટ્સ \\ \hline
\textbf{ફ્લો કંટ્રોલ} & હા & ના \\ \hline
\textbf{એરર કંટ્રોલ} & હા & મર્યાદિત \\ \hline
\end{tabulary}
\end{center}

\textbf{વિગતો:}

\begin{itemize}
    \item \keyword{TCP}: થ્રી-વે હેન્ડશેક (SYN, SYN-ACK, ACK), વિશ્વસનીય, ફ્લો કંટ્રોલ. વેબ, ઇમેઇલ માટે વપરાય છે.
    \item \keyword{UDP}: કોઈ કનેક્શન સેટઅપ નથી, લાઇટવેઇટ, કોઈ ગેરંટી નથી. વિડિઓ, ગેમિંગ, DNS માટે વપરાય છે.
\end{itemize}
\end{solutionbox}

\begin{mnemonicbox}
\mnemonic{TCP = Thorough, UDP = Ultra-fast}
\end{mnemonicbox}

% Question 5
\questionmarks{5(અ)}{3}{Hacking અને તેની સાવચેતીઓનું વર્ણન કરો.}

\begin{solutionbox}
\textbf{Hacking વ્યાખ્યા:}

\begin{itemize}
    \item \keyword{Unauthorized Access}: કમ્પ્યુટર સિસ્ટમમાં ઘૂસણખોરી
    \item \keyword{Malicious Intent}: ડેટા ચોરી, ફેરફાર અથવા નાશ કરવો
    \item \keyword{Security Breach}: સિસ્ટમની નબળાઈઓનો લાભ લેવો
\end{itemize}

\textbf{Hacking ના પ્રકારો:}

\begin{itemize}
    \item \keyword{Ethical Hacking}: અધિકૃત સુરક્ષા પરીક્ષણ
    \item \keyword{Malicious Hacking}: ગુનાહિત પ્રવૃત્તિઓ
    \item \keyword{Social Engineering}: માનવ વર્તણૂક સાથે ચેડાં
\end{itemize}

\textbf{સાવચેતીઓ:}

\begin{center}
\captionof{table}{સુરક્ષા પગલાં}
\begin{tabulary}{\linewidth}{|L|L|}
\hline
\textbf{સુરક્ષા પગલાં} & \textbf{અમલીકરણ} \\ \hline
\textbf{Strong Passwords} & જટિલ, અનન્ય પાસવર્ડ્સ \\ \hline
\textbf{Software Updates} & નિયમિત પેચ અને અપડેટ્સ \\ \hline
\textbf{Firewalls} & નેટવર્ક એક્સેસ કંટ્રોલ \\ \hline
\textbf{Antivirus} & મૉલવેર તપાસ અને દૂર કરવું \\ \hline
\textbf{Backup} & નિયમિત ડેટા બેકઅપ \\ \hline
\textbf{User Training} & સુરક્ષા જાગૃતિ કાર્યક્રમો \\ \hline
\end{tabulary}
\end{center}
\end{solutionbox}

\begin{mnemonicbox}
\mnemonic{HSPFAB - Hacking Stopped by Passwords, Firewalls, Antivirus, Backups}
\end{mnemonicbox}

\questionmarks{5(બ)}{4}{IPSec આર્કિટેક્ચર સમજાવો.}

\begin{solutionbox}
\textbf{IPSec (Internet Protocol Security):}

\begin{center}
\begin{tikzpicture}[node distance=1.5cm]
    \node [gtu block] (ipsec) {IPSec Architecture};
    \node [gtu block, below left=of ipsec] (ah) {AH\\(Auth Header)};
    \node [gtu block, below=of ipsec] (esp) {ESP\\(Encapsulating Security)};
    \node [gtu block, below right=of ipsec] (ike) {IKE/SA\\(Key Mgmt)};
    
    \draw [gtu arrow] (ipsec) -- (ah);
    \draw [gtu arrow] (ipsec) -- (esp);
    \draw [gtu arrow] (ipsec) -- (ike);
\end{tikzpicture}
\captionof{figure}{IPSec આર્કિટેક્ચર ઘટકો}
\end{center}

\textbf{IPSec ઘટકો:}

\begin{itemize}
    \item \keyword{AH}: ઓથેન્ટિકેશન અને ઇન્ટિગ્રિટી
    \item \keyword{ESP}: ગુપ્તતા અને ઓથેન્ટિકેશન
    \item \keyword{SA}: સુરક્ષા પરિમાણ કરાર
    \item \keyword{IKE}: કી મેનેજમેન્ટ પ્રોટોકોલ
\end{itemize}

\textbf{ઓપરેટિંગ મોડ્સ:}

\begin{itemize}
    \item \keyword{Transport Mode}: માત્ર પેલોડનું રક્ષણ કરે છે
    \item \keyword{Tunnel Mode}: સંપૂર્ણ IP પેકેટનું રક્ષણ કરે છે
\end{itemize}

\textbf{સુરક્ષા સેવાઓ:}

\begin{itemize}
    \item \keyword{Authentication}: સેન્ડરની ઓળખ ચકાસણી
    \item \keyword{Integrity}: ડેટા બદલાયેલ નથી તેની ખાતરી
    \item \keyword{Confidentiality}: ડેટા કન્ટેન્ટ એન્ક્રિપ્ટ
    \item \keyword{Anti-replay}: પેકેટ રિપ્લે હુમલા અટકાવે છે
\end{itemize}
\end{solutionbox}

\begin{mnemonicbox}
\mnemonic{AISE - AH, IPSec, SA, ESP}
\end{mnemonicbox}

\questionmarks{5(ક)}{7}{નેટવર્ક સુરક્ષા ટોપોલોજી સમજાવો.}

\begin{solutionbox}
\textbf{નેટવર્ક સુરક્ષા ટોપોલોજી:}

\begin{center}
\begin{tikzpicture}[node distance=1.5cm]
    \node [gtu state] (inet) {Internet};
    \node [gtu block, right=of inet] (fw) {Firewall};
    \node [gtu block, above right=of fw, fill=yellow!10] (dmz) {DMZ};
    \node [gtu block, right=of dmz] (web) {Web/Mail Server};
    \node [gtu block, below right=of fw, fill=green!10] (int) {Internal Network};
    \node [gtu block, right=of int] (db) {Database/Workstations};
    
    \draw [gtu arrow, <->] (inet) -- (fw);
    \draw [gtu arrow, <->] (fw) -- (dmz);
    \draw [gtu arrow, <->] (dmz) -- (web);
    \draw [gtu arrow, <->] (fw) -- (int);
    \draw [gtu arrow, <->] (int) -- (db);
\end{tikzpicture}
\captionof{figure}{DMZ નેટવર્ક ટોપોલોજી}
\end{center}

\textbf{સુરક્ષા ઝોન:}

\begin{center}
\captionof{table}{સુરક્ષા ઝોન}
\begin{tabulary}{\linewidth}{|L|L|L|}
\hline
\textbf{ઝોન} & \textbf{હેતુ} & \textbf{સુરક્ષા સ્તર} \\ \hline
\textbf{Internet} & બાહ્ય અવિશ્વસનીય નેટવર્ક & સૌથી ઓછું \\ \hline
\textbf{DMZ} & અર્ધ-વિશ્વસનીય જાહેર સેવાઓ & મધ્યમ \\ \hline
\textbf{Internal} & ખાનગી વિશ્વસનીય નેટવર્ક & સૌથી વધુ \\ \hline
\end{tabulary}
\end{center}

\textbf{ટોપોલોજી ઘટકો:}

\begin{itemize}
    \item \keyword{Perimeter Security}: Firewalls, IDS/IPS
    \item \keyword{Network Segmentation}: VLANs, સબનેટ
    \item \keyword{Access Control}: ઓથેન્ટિકેશન, ઓથોરાઇઝેશન
\end{itemize}

\textbf{સુરક્ષા સિદ્ધાંતો:}

\begin{itemize}
    \item \keyword{Defense in Depth}: બહુવિધ સુરક્ષા સ્તરો
    \item \keyword{Least Privilege}: ન્યૂનતમ જરૂરી એક્સેસ
    \item \keyword{Network Isolation}: જટિલ સિસ્ટમોને અલગ કરવી
\end{itemize}
\end{solutionbox}

\begin{mnemonicbox}
\mnemonic{NST = Network Security Through topology design}
\end{mnemonicbox}

\questionmarks{5(અ OR)}{3}{ISO સમજાવો અને તે માહિતી સુરક્ષામાં કેવી રીતે ફાળો આપે છે?}

\begin{solutionbox}
\textbf{ISO (International Organization for Standardization):}

\begin{itemize}
    \item \keyword{Global Standards}: આંતરરાષ્ટ્રીય ધોરણો વિકસાવે છે
    \item \keyword{Quality Assurance}: સાતત્યપૂર્ણ પ્રથાઓ સુનિશ્ચિત કરે છે
    \item \keyword{Best Practices}: અમલીકરણ માટે ફ્રેમવર્ક પૂરું પાડે છે
\end{itemize}

\textbf{ISO 27001 - માહિતી સુરક્ષા:}

\begin{itemize}
    \item \keyword{ISMS}: ઇન્ફોર્મેશન સિક્યુરિટી મેનેજમેન્ટ સિસ્ટમ
    \item \keyword{Risk Management}: સુરક્ષા માટે વ્યવસ્થિત અભિગમ
    \item \keyword{Continuous Improvement}: નિયમિત સમીક્ષા અને અપડેટ્સ
\end{itemize}

\textbf{લાભો:}

\begin{itemize}
    \item \keyword{Standardization}: સામાન્ય સુરક્ષા ભાષા
    \item \keyword{Credibility}: આંતરરાષ્ટ્રીય માન્યતા
    \item \keyword{Improvement}: સતત સુરક્ષા વૃદ્ધિ
\end{itemize}
\end{solutionbox}

\begin{mnemonicbox}
\mnemonic{ISO = International Security Organization}
\end{mnemonicbox}

\questionmarks{5(બ OR)}{4}{Symmetric અને Asymmetric એન્ક્રિપ્શન એલ્ગોરિધમ્સ વચ્ચે તફાવત આપો.}

\begin{solutionbox}
\textbf{એન્ક્રિપ્શન એલ્ગોરિધમ સરખામણી:}

\begin{center}
\captionof{table}{Symmetric vs Asymmetric Encryption}
\begin{tabulary}{\linewidth}{|L|L|L|}
\hline
\textbf{વિશેષતા} & \textbf{Symmetric} & \textbf{Asymmetric} \\ \hline
\textbf{કી} & સિંગલ શેયર્ડ કી & કી પેર (જાહેર/ખાનગી) \\ \hline
\textbf{ઝડપ} & ઝડપી & ધીમી \\ \hline
\textbf{કી વિતરણ} & મુશ્કેલ & સરળ \\ \hline
\textbf{સ્કેલેબિલિટી} & નબળું ($n^2-1$ કી) & સારું \\ \hline
\textbf{સુરક્ષા} & કી ગુપ્તતા પર આધારિત & ગાણિતિક જટિલતા \\ \hline
\end{tabulary}
\end{center}

\textbf{Symmetric Encryption:}

\begin{itemize}
    \item \keyword{Process}: સમાન કી એન્ક્રિપ્ટ અને ડિક્રિપ્ટ કરે છે
    \item \keyword{Challenge}: સુરક્ષિત કી વિતરણ
    \item \keyword{Examples}: AES, DES, 3DES
\end{itemize}

\textbf{Asymmetric Encryption:}

\begin{itemize}
    \item \keyword{Process}: પબ્લિક કી એન્ક્રિપ્ટ, પ્રાઇવેટ કી ડિક્રિપ્ટ કરે છે
    \item \keyword{Advantage}: કોઈ કી વિતરણ સમસ્યા નથી
    \item \keyword{Examples}: RSA, ECC, Diffie-Hellman
\end{itemize}
\end{solutionbox}

\begin{mnemonicbox}
\mnemonic{SA = Symmetric Shared, Asymmetric Apart}
\end{mnemonicbox}

\questionmarks{5(ક OR)}{7}{Email સુરક્ષા તેના ધોરણો સાથે સમજાવો.}

\begin{solutionbox}
\textbf{ઇમેઇલ સુરક્ષા પડકારો:}

\begin{center}
\begin{tikzpicture}[node distance=1.5cm]
    \node [gtu block] (threats) {Email Threats};
    \node [gtu block, right=of threats] (phish) {Phishing};
    \node [gtu block, above=of phish] (mal) {Malware};
    \node [gtu block, below=of phish] (spam) {Spam};
    \node [gtu block, right=of phish] (spoof) {Spoofing/Data Loss};
    
    \draw [gtu arrow] (threats) -- (phish);
    \draw [gtu arrow] (threats) |- (mal);
    \draw [gtu arrow] (threats) |- (spam);
    \draw [gtu arrow] (phish) -- (spoof);
\end{tikzpicture}
\captionof{figure}{ઇમેઇલ સુરક્ષા જોખમો}
\end{center}

\textbf{ઇમેઇલ સુરક્ષા ધોરણો:}

\begin{center}
\captionof{table}{સુરક્ષા ધોરણો}
\begin{tabulary}{\linewidth}{|L|L|L|}
\hline
\textbf{ધોરણ} & \textbf{હેતુ} & \textbf{કાર્ય} \\ \hline
\textbf{S/MIME} & સુરક્ષિત ઇમેઇલ કન્ટેન્ટ & એન્ક્રિપ્શન અને ડિજિટલ સહી \\ \hline
\textbf{PGP} & Pretty Good Privacy & એન્ડ-ટુ-એન્ડ એન્ક્રિપ્શન \\ \hline
\textbf{TLS} & ટ્રાન્સપોર્ટ સુરક્ષા & સુરક્ષિત ઇમેઇલ ટ્રાન્સમિશન \\ \hline
\textbf{SPF} & સેન્ડર ઓથેન્ટિકેશન & ઇમેઇલ સ્પૂફિંગ અટકાવો \\ \hline
\textbf{DKIM} & મેસેજ ઇન્ટિગ્રિટી & ડિજિટલ સહી ચકાસણી \\ \hline
\textbf{DMARC} & નીતિ અમલીકરણ & ઇમેઇલ ઓથેન્ટિકેશન નીતિ \\ \hline
\end{tabulary}
\end{center}

\textbf{સુરક્ષા મિકેનિઝમ્સ:}

\begin{itemize}
    \item \keyword{Encryption}: મેસેજ કન્ટેન્ટનું રક્ષણ કરો
    \item \keyword{Digital Signatures}: સેન્ડરની ઓળખ ચકાસો (Authentication)
    \item \keyword{Integrity}: મેસેજ બદલાયેલ નથી તેની ખાતરી કરો
\end{itemize}

\textbf{શ્રેષ્ઠ પ્રયાસો:}

\begin{itemize}
    \item \keyword{User Education}: ફિશિંગ પ્રયાસો ઓળખો
    \item \keyword{Gateway Filtering}: દૂષિત ઇમેઇલ બ્લોક કરો
    \item \keyword{Regular Updates}: સુરક્ષા સોફ્ટવેર અપડેટ રાખો
\end{itemize}
\end{solutionbox}

\begin{mnemonicbox}
\mnemonic{SPTSD - S/MIME, PGP, TLS, SPF, DKIM protect email}
\end{mnemonicbox}

\end{document}
