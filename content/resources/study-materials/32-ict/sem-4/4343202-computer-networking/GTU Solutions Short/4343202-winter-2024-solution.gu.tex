\documentclass{article}

% content/resources/templates/preamble.tex
\usepackage[margin=0.6in]{geometry}
\author{Milav Dabgar}
\usepackage{amsmath,amssymb,amsthm}
\usepackage{booktabs}
\usepackage{multirow}
\usepackage{xcolor}
\usepackage{tcolorbox}
\tcbuselibrary{breakable,skins}
\usepackage[colorlinks=true,linkcolor=blue]{hyperref}
\usepackage{titlesec}
\usepackage{enumitem}
\usepackage{tikz}
\usepackage{pgfplots}
\usepackage{circuitikz}
\usepackage[version=4]{mhchem}
\usepackage{longtable}
\usepackage{array}
\usepackage{float}
\usepackage{caption}
\usepackage{listings}

\lstset{
  basicstyle=\small\ttfamily,
  breaklines=true,
  breakatwhitespace=false,
  postbreak=\mbox{\textcolor{red}{$\hookrightarrow$}\space},
  float=false,
  numbers=left,
  numberstyle=\tiny\color{gray},
  numbersep=10pt,
  xleftmargin=2em,
  keywordstyle=\color{blue},
  commentstyle=\color{green!60!black},
  stringstyle=\color{purple},
  backgroundcolor=\color{gray!5},
  showstringspaces=false,
  tabsize=2,
  captionpos=b,
  keepspaces=true,
  columns=flexible
}

\pgfplotsset{compat=1.18}
\usetikzlibrary{shapes,arrows,positioning,calc,patterns,decorations.pathmorphing,decorations.markings,arrows.meta}

% Color scheme
\definecolor{headcolor}{RGB}{0,102,204}
\definecolor{keycolor}{RGB}{220,20,60}
\definecolor{solutioncolor}{RGB}{34,139,34}
\definecolor{mnemoniccolor}{RGB}{148,0,211}
\definecolor{codecolor}{RGB}{0,0,100}

% Spacing
\setlength{\parskip}{3pt}
\setlist[itemize]{nosep}
\setlist[enumerate]{nosep}

% Title formatting
\titleformat{\section}{\Large\bfseries\color{headcolor}}{\thesection}{1em}{}
\titleformat{\subsection}{\large\bfseries\color{headcolor}}{\thesubsection}{1em}{}

% Pandoc tightlist compatibility
\providecommand{\tightlist}{%
  \setlength{\itemsep}{0pt}\setlength{\parskip}{0pt}}

% Pandoc longtable compatibility
\newcounter{none}
\def\thenone{}


% content/resources/templates/gujarati-boxes.tex
\usepackage{fontspec}
\usepackage{polyglossia}

% Set Gujarati as main language (document is primarily in Gujarati)
% Note: gloss-gujarati.ldf doesn't exist in polyglossia, but it will use hyphenation patterns
\setdefaultlanguage{gujarati}
\setotherlanguage{english}

% Configure Gujarati font properly
% Use Language=Default to prevent polyglossia from trying to add language-specific features
% that don't exist for Gujarati, which causes "empty feature" warnings
\newfontfamily\gujaratifont[Script=Gujarati,AutoFakeBold=2.5,AutoFakeSlant=0.3]{Noto Sans Gujarati}
\setmainfont[Script=Gujarati,AutoFakeBold=2.5,AutoFakeSlant=0.3]{Noto Sans Gujarati}
% Use Noto Sans Gujarati for monospace to support Gujarati in text
\setmonofont[Scale=0.9]{Noto Sans Gujarati}

% Configure English to use the same font
\newfontfamily\englishfont[Script=Gujarati,AutoFakeBold=2.5,AutoFakeSlant=0.3]{Noto Sans Gujarati}

% Translations for polyglossia
\gappto\captionsgujarati{
  \renewcommand{\tablename}{કોષ્ટક}
  \renewcommand{\figurename}{આકૃતિ}
}

% Helper for TikZ nodes to ensure Gujarati font
\newcommand{\gu}[1]{{\gujaratifont #1}}

% Custom environments
\newtcolorbox{solutionbox}{
    breakable,
    enhanced,
    colback=solutioncolor!5!white,
    colframe=solutioncolor!75!black,
    fonttitle=\bfseries,
    title=જવાબ
}

\newtcolorbox{solutionboxnobreak}{
 colback=solutioncolor!5!white,
 colframe=solutioncolor!75!black,
 fonttitle=\bfseries,
 title=જવાબ
}

\newtcolorbox{keyformula}{
 breakable,
 enhanced,
 colback=keycolor!5!white,
 colframe=keycolor!75!black,
 fonttitle=\bfseries,
 title=રાસાયણિક સમીકરણ/સૂત્ર
}

\newtcolorbox{mnemonicbox}{
 breakable,
 enhanced,
 colback=mnemoniccolor!5!white,
 colframe=mnemoniccolor!75!black,
 fonttitle=\bfseries,
 title=મેમરી ટ્રીક
}


% Custom commands for GTU solutions
% This file defines semantic commands for consistent formatting

% Question command with automatic formatting
\newcommand{\question}[2]{%
  \section*{Question #1}%
  \textbf{#2}%
}

% OR question variant
\newcommand{\questionor}[2]{%
  \section*{Question #1 OR}%
  \textbf{#2}%
}

% Proper table environment with caption
\newenvironment{answertable}[1]{%
  \begin{table}[htbp]
  \centering
  \caption{#1}
}{%
  \end{table}
}

% Proper figure environment for diagrams
\newenvironment{answerdiagram}[1]{%
  \begin{figure}[htbp]
  \centering
  \caption{#1}
}{%
  \end{figure}
}

% Semantic markup for key terms
\newcommand{\keyword}[1]{\textbf{#1}}
\newcommand{\code}[1]{\texttt{#1}}
\newcommand{\classname}[1]{\texttt{#1}}
\newcommand{\methodname}[1]{\texttt{#1}}

% Proper quotation marks
\newcommand{\mnemonic}[1]{``#1''}


\title{કમ્પ્યુટર નેટવર્કિંગ (4343202) - વિન્ટર 2024 સોલ્યુશન}
\date{નવેમ્બર 28, 2024}

\begin{document}
\maketitle

\questionmarks{1(અ)}{3}{કોમ્પ્યુટર નેટવર્ક શું છે? તે શા માટે મહત્વનું છે?}

\begin{solutionbox}
\textbf{જવાબ}:
કમ્પ્યુટર નેટવર્ક એ ઇન્ટરકનેક્ટેડ કમ્પ્યુટિંગ ડિવાઇસનો સમૂહ છે જે ડેટા એક્સચેન્જ અને રિસોર્સ શેરિંગ કરી શકે છે.

\textbf{આકૃતિ:}

\begin{center}
\begin{tikzpicture}[node distance=2cm]
    \node [gtu block] (pc1) {Computer};
    \node [gtu block, right=of pc1] (pc2) {Computer};
    \node [gtu block, below=of pc1] (pc3) {Computer};
    \node [gtu block, right=of pc3] (pc4) {Computer};
    \draw [gtu arrow, <->] (pc1) -- (pc2);
    \draw [gtu arrow, <->] (pc1) -- (pc3);
    \draw [gtu arrow, <->] (pc3) -- (pc4);
    \draw [gtu arrow, <->] (pc2) -- (pc4);
\end{tikzpicture}
\captionof{figure}{સરળ કમ્પ્યુટર નેટવર્ક}
\end{center}

\begin{itemize}
    \item \keyword{રિસોર્સ શેરિંગ}: પ્રિન્ટર, ફાઇલ, એપ્લિકેશન શેર કરવાની સુવિધા
    \item \keyword{કોમ્યુનિકેશન}: વપરાશકર્તાઓ વચ્ચે માહિતીનું આદાન-પ્રદાન સરળ બનાવે
    \item \keyword{સ્કેલેબિલિટી}: નેટવર્કને જરૂરિયાત મુજબ વિસ્તારી શકાય છે
\end{itemize}
\end{solutionbox}

\begin{mnemonicbox}
\mnemonic{CSI - કનેક્ટ, શેર, ઇન્ટરેક્ટ}
\end{mnemonicbox}

\questionmarks{1(બ)}{4}{વ્યાખ્યા આપો: ૧)વેબ સર્વર, ૨)એનક્રિપ્તેડ ડેટા, ૩) હેકિંગ, ૪) ક્લાયન્ટ-સર્વર}

\begin{solutionbox}
\textbf{વ્યાખ્યાઓ:}

\begin{center}
\captionof{table}{નેટવર્ક ટર્મ્સ}
\begin{tabulary}{\linewidth}{|L|L|}
\hline
\textbf{શબ્દ} & \textbf{વ્યાખ્યા} \\ \hline
\textbf{વેબ સર્વર} & HTTP/HTTPS નો ઉપયોગ કરી ક્લાયન્ટને વેબ કન્ટેન્ટ પ્રદાન કરતું સોફ્ટવેર/હાર્ડવેર \\ \hline
\textbf{એનક્રિપ્ટેડ ડેટા} & અનધિકૃત એક્સેસને રોકવા માટે કોડમાં રૂપાંતરિત કરેલી માહિતી \\ \hline
\textbf{હેકિંગ} & સિક્યોરિટી વલ્નરેબિલિટીઝ દ્વારા કમ્પ્યુટર સિસ્ટમમાં અનધિકૃત એક્સેસ \\ \hline
\textbf{ક્લાયન્ટ-સર્વર} & સેન્ટ્રલાઈઝ્ડ સર્વર ક્લાયન્ટ કમ્પ્યુટરને સેવાઓ પ્રદાન કરે તે નેટવર્ક મોડેલ \\ \hline
\end{tabulary}
\end{center}

\textbf{ક્લાયન્ટ-સર્વર મોડેલ:}

\begin{center}
\begin{tikzpicture}[node distance=2.5cm, auto]
    \node [gtu block] (client) {Client};
    \node [gtu block, right=of client] (server) {Server};
    
    \draw [gtu arrow, bend left] (client) to node {Request} (server);
    \draw [gtu arrow, bend left] (server) to node {Response} (client);
\end{tikzpicture}
\captionof{figure}{ક્લાયન્ટ-સર્વર ઇન્ટરેક્શન}
\end{center}
\end{solutionbox}

\begin{mnemonicbox}
\mnemonic{WECHS - વેબ સર્વર એનક્રિપ્ટ ડેટા, ક્લાયન્ટ અને હેકર્સ સર્વરનો ઉપયોગ કરે છે}
\end{mnemonicbox}

\questionmarks{1(ક)}{7}{ટ્રાન્સમિશન મીડીયાનું ક્લાસીફીકેશન આપો અને સમજાવો.}

\begin{solutionbox}
\textbf{ટ્રાન્સમિશન મીડિયાના પ્રકારો:}

\begin{center}
\captionof{table}{ટ્રાન્સમિશન મીડિયા વર્ગીકરણ}
\begin{tabulary}{\linewidth}{|L|L|L|L|}
\hline
\textbf{કેટેગરી} & \textbf{પ્રકાર} & \textbf{લાક્ષણિકતાઓ} & \textbf{ઉપયોગો} \\ \hline
\multicolumn{4}{|l|}{\textbf{ગાઇડેડ મીડિયા}} \\ \hline
ટ્વિસ્ટેડ પેર & UTP, STP & 100m રેન્જ, 10Mbps-10Gbps & ઓફિસ LANs \\ \hline
કોએક્સિયલ કેબલ & બેસબેન્ડ, બ્રોડબેન્ડ & 500m રેન્જ, 10-100Mbps & કેબલ TV, ઇન્ટરનેટ \\ \hline
ફાયબર ઓપ્ટિક & સિંગલ/મલ્ટી-મોડ & લાંબું અંતર, હાઇ સ્પીડ & બેકબોન, WAN \\ \hline
\multicolumn{4}{|l|}{\textbf{અનગાઇડેડ મીડિયા}} \\ \hline
રેડિયો વેવ્સ & WiFi, સેલ્યુલર & ઓમ્નિડિરેક્શનલ & વાયરલેસ નેટવર્ક \\ \hline
માઇક્રોવેવ્સ & ટેરેસ્ટ્રિયલ/સેટેલાઇટ & લાઇન-ઓફ-સાઇટ & પોઇન્ટ-ટુ-પોઇન્ટ \\ \hline
ઇન્ફ્રારેડ & IrDA & શોર્ટ-રેન્જ & રિમોટ કંટ્રોલ \\ \hline
\end{tabulary}
\end{center}

\textbf{આકૃતિ:}

\begin{center}
\begin{tikzpicture}[node distance=1.5cm]
    \node [gtu block, font=\bfseries] (media) {Transmission Media};
    \node [gtu block, below left=of media] (guided) {Guided (Wired)};
    \node [gtu block, below right=of media] (unguided) {Unguided (Wireless)};
    
    \node [below=0.2cm of guided, align=center, font=\footnotesize] {Twisted Pair\\Coaxial\\Fiber Optic};
    \node [below=0.2cm of unguided, align=center, font=\footnotesize] {Radio\\Microwave\\Infrared};
    
    \draw [gtu arrow] (media) -- (guided);
    \draw [gtu arrow] (media) -- (unguided);
\end{tikzpicture}
\captionof{figure}{ટ્રાન્સમિશન મીડિયા હાયરાર્કી}
\end{center}

\begin{itemize}
    \item \keyword{ગાઇડેડ મીડિયા}: સિગ્નલને મર્યાદિત કરતા ભૌતિક માર્ગો
    \item \keyword{અનગાઇડેડ મીડિયા}: હવા/અવકાશ દ્વારા વાયરલેસ ટ્રાન્સમિશન
    \item \keyword{પસંદગીના પરિબળો}: ખર્ચ, બેન્ડવિડ્થ, અંતર, પર્યાવરણ
\end{itemize}
\end{solutionbox}

\begin{mnemonicbox}
\mnemonic{TCFRIM - ટ્વિસ્ટેડ પેર, કોએક્સિયલ, ફાયબર, રેડિયો, ઇન્ફ્રારેડ, માઇક્રોવેવ}
\end{mnemonicbox}

\questionmarks{1(ક) અથવા}{7}{WAN અને MAN ને સમજાવો.}

\begin{solutionbox}
\textbf{સરખામણી:}

\begin{center}
\captionof{table}{MAN vs WAN}
\begin{tabulary}{\linewidth}{|L|L|L|}
\hline
\textbf{ફીચર} & \textbf{MAN (મેટ્રોપોલિટન)} & \textbf{WAN (વાઇડ)} \\ \hline
\textbf{કવરેજ} & શહેર-વ્યાપી (5-50 km) & દેશ/વૈશ્વિક (>50 km) \\ \hline
\textbf{સ્પીડ} & 10 Mbps - 10 Gbps & 1.5 Mbps - 1 Gbps \\ \hline
\textbf{માલિકી} & મ્યુનિસિપલ/ટેલિકોમ & મલ્ટિપલ ઓર્ગેનાઇઝેશન \\ \hline
\textbf{ઉદાહરણો} & સિટી નેટવર્ક & ઇન્ટરનેટ, 4G/5G \\ \hline
\end{tabulary}
\end{center}

\textbf{નેટવર્ક સ્કોપ આકૃતિ:}

\begin{center}
\begin{tikzpicture}[node distance=1cm]
    \node [gtu container, label=above:WAN (Global)] (wan) {
        \begin{tikzpicture}
             \node [gtu container, fill=white, label=center:MAN (City)] (man) {};
             \node [gtu container, fill=blue!10, minimum width=1.5cm, below=0.5cm of man, label=center:LAN] (lan) {};
        \end{tikzpicture}
    };
\end{tikzpicture}
\captionof{figure}{નેટવર્ક સ્કોપ હાયરાર્કી}
\end{center}

\begin{itemize}
    \item \keyword{MAN}: શહેર/મેટ્રોપોલિટન એરિયામાં LANsને જોડે છે
    \item \keyword{WAN}: શહેરો/દેશો વચ્ચે મોટા ભૌગોલિક વિસ્તારોને આવરે છે
    \item \keyword{મેનેજમેન્ટ}: WAN સામાન્ય રીતે સર્વિસ પ્રોવાઇડર્સની જરૂર પડે છે
\end{itemize}
\end{solutionbox}

\begin{mnemonicbox}
\mnemonic{SWIM - સાઇઝ: WAN ઇઝ મેસિવ કમ્પેર્ડ ટુ MAN}
\end{mnemonicbox}

\questionmarks{2(અ)}{3}{વિગતવાર સમજાવો: ટ્રાન્સમિશન ટેકનોલોજી.}

\begin{solutionbox}
\textbf{ટ્રાન્સમિશન ટેકનોલોજી:}

\begin{center}
\captionof{table}{ટ્રાન્સમિશન પ્રકારો}
\begin{tabulary}{\linewidth}{|L|L|L|}
\hline
\textbf{ટેકનોલોજી} & \textbf{વર્ણન} & \textbf{ઉદાહરણ} \\ \hline
\textbf{પોઇન્ટ-ટુ-પોઇન્ટ} & બે નોડ્સ વચ્ચે સીધું કનેક્શન & લીઝ્ડ લાઇન \\ \hline
\textbf{બ્રોડકાસ્ટ} & બધા નોડ્સ દ્વારા શેર કરાતું સિંગલ ચેનલ & વાયરલેસ LAN \\ \hline
\textbf{મલ્ટિપોઇન્ટ} & મલ્ટિપલ ડિવાઇસ એક લિંક શેર કરે & કેબલ TV \\ \hline
\end{tabulary}
\end{center}

\begin{itemize}
    \item \keyword{એનાલોગ}: કન્ટિન્યુઅસ સિગ્નલ, નોઇઝને લગતું
    \item \keyword{ડિજિટલ}: ડિસ્ક્રીટ સિગ્નલ, વધુ વિશ્વસનીય
    \item \keyword{બેસબેન્ડ}: સિંગલ સિગ્નલ સમગ્ર બેન્ડવિડ્થનો ઉપયોગ કરે છે (Ethernet)
    \item \keyword{બ્રોડબેન્ડ}: મલ્ટિપલ સિગ્નલ્સ બેન્ડવિડ્થ શેર કરે છે (કેબલ TV)
\end{itemize}
\end{solutionbox}

\begin{mnemonicbox}
\mnemonic{ABP-DMB - એનાલોગ ઓર બેસબેન્ડ, પોઇન્ટ-ટુ-પોઇન્ટ; ડિજિટલ ઓર મલ્ટિપોઇન્ટ, બ્રોડકાસ્ટ}
\end{mnemonicbox}

\questionmarks{2(બ)}{4}{સ્ટાર ટોપોલોજી દોરો અને સમજાવો.}

\begin{solutionbox}
\textbf{સ્ટાર ટોપોલોજી આકૃતિ:}

\begin{center}
\begin{tikzpicture}[node distance=2cm]
    \node [gtu block, fill=yellow!20] (hub) {Hub/Switch};
    \node [gtu state, above=of hub] (n1) {Node 1};
    \node [gtu state, below left=of hub] (n2) {Node 2};
    \node [gtu state, below right=of hub] (n3) {Node 3};
    
    \draw [gtu arrow, <->] (hub) -- (n1);
    \draw [gtu arrow, <->] (hub) -- (n2);
    \draw [gtu arrow, <->] (hub) -- (n3);
\end{tikzpicture}
\captionof{figure}{સ્ટાર ટોપોલોજી}
\end{center}

\textbf{વિશ્લેષણ:}

\begin{center}
\captionof{table}{સ્ટાર ટોપોલોજી ફાયદા/ગેરફાયદા}
\begin{tabulary}{\linewidth}{|L|L|}
\hline
\textbf{ફાયદા} & \textbf{ગેરફાયદા} \\ \hline
સરળ ઇન્સ્ટોલેશન & સિંગલ પોઇન્ટ ઓફ ફેલ્યોર (હબ) \\ \hline
સરળ ટ્રબલશૂટિંગ & વધુ કેબલની જરૂર \\ \hline
સ્કેલેબલ & સેન્ટ્રલ ડિવાઇસને કારણે ઉંચી કિંમત \\ \hline
\end{tabulary}
\end{center}
\end{solutionbox}

\begin{mnemonicbox}
\mnemonic{CASE - સેન્ટ્રલાઇઝ્ડ, ઓલ કનેક્ટેડ, સિમ્પલ એક્સપાન્શન, ઇઝી ટ્રબલશૂટિંગ}
\end{mnemonicbox}

\questionmarks{2(ક)}{7}{TCP/IP મોડેલ દોરો અને સમજાવો.}

\begin{solutionbox}
\textbf{TCP/IP મોડેલ લેયર્સ:}

\begin{center}
\begin{tikzpicture}[node distance=0cm, outer sep=0pt]
    \node [draw, rectangle, minimum width=5cm, minimum height=0.8cm, fill=blue!10] (app) {Application Layer (HTTP, FTP, DNS)};
    \node [draw, rectangle, minimum width=5cm, minimum height=0.8cm, below=of app, fill=yellow!10] (trans) {Transport Layer (TCP, UDP)};
    \node [draw, rectangle, minimum width=5cm, minimum height=0.8cm, below=of trans, fill=green!10] (inet) {Internet Layer (IP, ICMP, ARP)};
    \node [draw, rectangle, minimum width=5cm, minimum height=0.8cm, below=of inet, fill=gray!10] (net) {Network Access Layer (Ethernet, WiFi)};
    
    \node [right=0.5cm of app, align=left, font=\footnotesize] {User Interfaces};
    \node [right=0.5cm of trans, align=left, font=\footnotesize] {End-to-end reliability};
    \node [right=0.5cm of inet, align=left, font=\footnotesize] {Packets \& Routing};
    \node [right=0.5cm of net, align=left, font=\footnotesize] {Physical Media};
\end{tikzpicture}
\captionof{figure}{TCP/IP સ્ટેક}
\end{center}

\textbf{લેયર ફંક્શન્સ:}

\begin{itemize}
    \item \keyword{એપ્લિકેશન}: એપ્લિકેશન અને નેટવર્ક વચ્ચે ઇન્ટરફેસ
    \item \keyword{ટ્રાન્સપોર્ટ}: એન્ડ સિસ્ટમ્સ વચ્ચે વિશ્વસનીય ડેટા ટ્રાન્સફર
    \item \keyword{ઇન્ટરનેટ}: પેકેટનું એડ્રેસિંગ અને રાઉટિંગ
    \item \keyword{નેટવર્ક એક્સેસ}: ફિઝિકલ હાર્ડવેર ઇન્ટરફેસ
\end{itemize}
\end{solutionbox}

\begin{mnemonicbox}
\mnemonic{ATNI - એપ્લિકેશન ટોક્સ, નેટવર્ક ઇન્ટરનેટ ઇન્ટરફેસીસ}
\end{mnemonicbox}

\questionmarks{2(અ) અથવા}{3}{બસ ટોપોલોજી દોરો અને સમજાવો.}

\begin{solutionbox}
\textbf{બસ ટોપોલોજી આકૃતિ:}

\begin{center}
\begin{tikzpicture}[node distance=1.5cm]
    \draw [thick, double] (-3,0) -- (3,0) node[right] {Backbone Cable};
    \node [gtu state, below=0.5cm] (n1) at (-2,0) {1};
    \draw [thick] (-2,0) -- (n1);
    \node [gtu state, below=0.5cm] (n2) at (-0.5,0) {2};
    \draw [thick] (-0.5,0) -- (n2);
    \node [gtu state, below=0.5cm] (n3) at (1,0) {3};
    \draw [thick] (1,0) -- (n3);
    \node [gtu state, below=0.5cm] (n4) at (2.5,0) {4};
    \draw [thick] (2.5,0) -- (n4);
    \node [fill=black, minimum size=0.2cm] at (-3,0) {}; % Terminator
    \node [fill=black, minimum size=0.2cm] at (3,0) {}; % Terminator
\end{tikzpicture}
\captionof{figure}{બસ ટોપોલોજી}
\end{center}

\textbf{વિશ્લેષણ:}

\begin{itemize}
    \item \keyword{ફાયદા}: સરળ લેઆઉટ, ઓછું કેબલિંગ, ઓછી કિંમત
    \item \keyword{ગેરફાયદા}: બેકબોન ફેઈલ થાય તો બધું બંધ, ટ્રબલશૂટિંગ મુશ્કેલ
    \item \keyword{ટર્મિનેટર}: સિગ્નલ રિફ્લેક્શન રોકવા માટે છેડે જરૂરી
\end{itemize}
\end{solutionbox}

\begin{mnemonicbox}
\mnemonic{SLUE - સિમ્પલ લેઆઉટ, યુઝીસ લેસ કેબલ, ઇઝી ઇન્સ્ટોલેશન}
\end{mnemonicbox}

\questionmarks{2(બ) અથવા}{4}{આર્કિટેક્ચર અન્વયે નેટવર્ક ક્લાસીફીકેશન સમજાવો.}

\begin{solutionbox}
\textbf{નેટવર્ક આર્કિટેક્ચર:}

\begin{center}
\captionof{table}{આર્કિટેક્ચર સરખામણી}
\begin{tabulary}{\linewidth}{|L|L|L|}
\hline
\textbf{આર્કિટેક્ચર} & \textbf{લાક્ષણિકતાઓ} & \textbf{ઉદાહરણ} \\ \hline
\textbf{પીઅર-ટુ-પીઅર} & સમાન અધિકારો, ડીસેન્ટ્રલાઇઝ્ડ & ટોરેન્ટ્સ, હોમ LAN \\ \hline
\textbf{ક્લાયન્ટ-સર્વર} & સેન્ટ્રલાઇઝ્ડ સર્વિસીસ & એન્ટરપ્રાઇઝ નેટવર્ક \\ \hline
\textbf{થ્રી-ટાયર} & પ્રેઝન્ટેશન, લોજિક, ડેટા ટાયર્સ & વેબ એપ્સ \\ \hline
\end{tabulary}
\end{center}

\textbf{આકૃતિઓ:}

\begin{center}
\begin{tikzpicture}[node distance=1.5cm]
    % P2P
    \node [gtu state, font=\tiny] (p1) {Peer};
    \node [gtu state, font=\tiny, right=of p1] (p2) {Peer};
    \draw [gtu arrow, <->] (p1) -- (p2);
    
    % Client-Server
    \node [gtu block, right=3cm of p2] (server) {Server};
    \node [gtu state, font=\tiny, below left=of server] (c1) {Client};
    \node [gtu state, font=\tiny, below right=of server] (c2) {Client};
    \draw [gtu arrow, <->] (c1) -- (server);
    \draw [gtu arrow, <->] (c2) -- (server);
    
    \node [below=0.2cm of p1] {P2P};
    \node [below=0.2cm of server] {Client-Server};
\end{tikzpicture}
\captionof{figure}{આર્કિટેક્ચર મોડેલ્સ}
\end{center}
\end{solutionbox}

\begin{mnemonicbox}
\mnemonic{PCAN - પીઅર-ટુ-પીઅર, ક્લાયન્ટ-સર્વર, આર્કિટેક્ચર નેટવર્ક્સ}
\end{mnemonicbox}

\questionmarks{2(ક) અથવા}{7}{IP એડ્રેસનું ક્લાસીફીકેશન સમજાવો.}

\begin{solutionbox}
\textbf{IP ક્લાસિફિકેશન:}

\begin{center}
\captionof{table}{IP એડ્રેસિંગ ક્લાસીસ}
\begin{tabulary}{\linewidth}{|L|L|L|L|}
\hline
\textbf{ક્લાસ} & \textbf{રેન્જ (1st Octet)} & \textbf{માસ્ક} & \textbf{હોસ્ટ્સ} \\ \hline
\textbf{A} & 1 - 126 & 255.0.0.0 & 16M+ \\ \hline
\textbf{B} & 128 - 191 & 255.255.0.0 & 65,534 \\ \hline
\textbf{C} & 192 - 223 & 255.255.255.0 & 254 \\ \hline
\textbf{D} & 224 - 239 & N/A & મલ્ટિકાસ્ટ \\ \hline
\textbf{E} & 240 - 255 & N/A & રિઝર્વ્ડ \\ \hline
\end{tabulary}
\end{center}

\textbf{સ્ટ્રક્ચર આકૃતિ:}

\begin{center}
\begin{tikzpicture}[node distance=0cm, outer sep=0pt, font=\scriptsize]
    \node [draw, rectangle, minimum width=2cm] (a1) {Net (8)};
    \node [draw, rectangle, minimum width=4cm, right=of a1] (a2) {Host (24)};
    
    \node [draw, rectangle, minimum width=3cm, below=0.5cm of a1] (b1) {Net (16)};
    \node [draw, rectangle, minimum width=3cm, right=of b1] (b2) {Host (16)};
    
    \node [draw, rectangle, minimum width=4cm, below=0.5cm of b1] (c1) {Net (24)};
    \node [draw, rectangle, minimum width=2cm, right=of c1] (c2) {Host (8)};
    
    \node [left=0.2cm of a1] {Class A};
    \node [left=0.2cm of b1] {Class B};
    \node [left=0.2cm of c1] {Class C};
\end{tikzpicture}
\captionof{figure}{ક્લાસફુલ એડ્રેસિંગ સ્ટ્રક્ચર}
\end{center}

\begin{itemize}
    \item \keyword{સ્પેશ્યલ રેન્જ}: પ્રાઇવેટ IPs (10.x, 192.168.x), લૂપબેક (127.0.0.1)
    \item \keyword{CIDR}: આ જૂની સિસ્ટમને બદલે નવું ક્લાસલેસ રાઉટિંગ વપરાય છે
\end{itemize}
\end{solutionbox}

\begin{mnemonicbox}
\mnemonic{ABCDE - એડ્રેસ બ્લોક્સ કેટેગરાઇઝ્ડ બાય ડિક્રીઝિંગ એન્ડ-હોસ્ટ કાઉન્ટ્સ}
\end{mnemonicbox}

\questionmarks{3(અ)}{3}{LANનું આખું નામ શું છે? LAN વિગતવાર સમજાવો.}

\begin{solutionbox}
\textbf{વ્યાખ્યા:}
LAN એટલે Local Area Network, એક મર્યાદિત ભૌગોલિક વિસ્તારમાં સીમિત નેટવર્ક.

\textbf{આકૃતિ:}

\begin{center}
\begin{tikzpicture}[node distance=2cm]
    \node [gtu block] (switch) {Switch};
    \node [gtu block, left=of switch] (pc1) {Computer};
    \node [gtu block, right=of switch] (pc2) {Computer};
    \node [gtu block, above=of switch] (prn) {Printer};
    \node [gtu block, below=of switch] (pc3) {Computer};
    
    \draw [gtu arrow, <->] (switch) -- (pc1);
    \draw [gtu arrow, <->] (switch) -- (pc2);
    \draw [gtu arrow, <->] (switch) -- (prn);
    \draw [gtu arrow, <->] (switch) -- (pc3);
\end{tikzpicture}
\captionof{figure}{લોકલ એરિયા નેટવર્ક}
\end{center}

\textbf{લાક્ષણિકતાઓ:}

\begin{center}
\captionof{table}{LAN ફીચર્સ}
\begin{tabulary}{\linewidth}{|L|L|}
\hline
\textbf{લાક્ષણિકતા} & \textbf{વર્ણન} \\ \hline
\textbf{સ્કોપ} & બિલ્ડિંગ/કેમ્પસ (1-2 km) \\ \hline
\textbf{સ્પીડ} & ઉચ્ચ (10 Mbps - 10 Gbps) \\ \hline
\textbf{માલિકી} & એક સંસ્થા/વ્યક્તિ \\ \hline
\textbf{મીડિયા} & ટ્વિસ્ટેડ પેર, ફાયબર, WiFi \\ \hline
\end{tabulary}
\end{center}
\end{solutionbox}

\begin{mnemonicbox}
\mnemonic{LOCAL - લિમિટેડ ઇન રેન્જ, ઓન્ડ બાય વન એન્ટિટી, કનેક્ટેડ ડિવાઇસિસ, એક્સેસ કંટ્રોલ, લો લેટન્સી}
\end{mnemonicbox}

\questionmarks{3(બ)}{4}{રીપીટર પર ટૂંકનોંધ લખો.}

\begin{solutionbox}
\textbf{રિપીટર ફંક્શન:}

\begin{center}
\begin{tikzpicture}[node distance=2cm]
    \node [gtu state, align=center] (weak) {Weak\\Signal};
    \node [gtu block, right=of weak] (repeater) {Repeater};
    \node [gtu state, right=of repeater, align=center] (strong) {Strong\\Signal};
    
    \draw [gtu arrow] (weak) -- (repeater);
    \draw [gtu arrow] (repeater) -- (strong);
    
    \node [below=0.5cm of repeater, font=\footnotesize] {Layer 1 Device};
\end{tikzpicture}
\captionof{figure}{રિપીટર ઓપરેશન}
\end{center}

\begin{itemize}
    \item \keyword{લેયર}: ફિઝિકલ લેયર (OSI લેયર 1)
    \item \keyword{ફંક્શન}: સિગ્નલ રિજનરેટ અને એમ્પ્લિફાય કરે છે
    \item \keyword{હેતુ}: કેબલ લિમિટથી વધુ નેટવર્ક અંતર વધારવું
    \item \keyword{મર્યાદા}: ટ્રાફિક ફિલ્ટર કરી શકતા નથી અથવા કોલિઝન ડોમેન અલગ કરી શકતા નથી
\end{itemize}
\end{solutionbox}

\begin{mnemonicbox}
\mnemonic{RARE - રિપીટર્સ એમ્પ્લિફાઇ એન્ડ રિજનરેટ ઇલેક્ટ્રિકલ સિગ્નલ્સ}
\end{mnemonicbox}

\questionmarks{3(ક)}{7}{ટૂંકનોંધ લખો: FTP}

\begin{solutionbox}
\textbf{ફાઇલ ટ્રાન્સફર પ્રોટોકોલ (FTP):}

\begin{center}
\begin{tikzpicture}[node distance=3cm]
    \node [gtu block] (client) {FTP Client};
    \node [gtu block, right=of client] (server) {FTP Server};
    
    \draw [gtu arrow, bend left] (client) to node[above] {Control (Port 21)} (server);
    \draw [gtu arrow, bend left] (server) to node[below] {Data (Port 20)} (client);
\end{tikzpicture}
\captionof{figure}{FTP ડ્યુઅલ કનેક્શન્સ}
\end{center}

\textbf{મુખ્ય ફીચર્સ:}

\begin{center}
\captionof{table}{FTP વિગતો}
\begin{tabulary}{\linewidth}{|L|L|}
\hline
\textbf{ફીચર} & \textbf{વર્ણન} \\ \hline
\textbf{પોર્ટ્સ} & 21 (કંટ્રોલ) અને 20 (ડેટા) \\ \hline
\textbf{મોડ્સ} & એક્ટિવ અને પેસિવ \\ \hline
\textbf{ઓથેન્ટિકેશન} & યુઝરનેમ/પાસવર્ડ અથવા એનોનિમસ \\ \hline
\textbf{ડેટા ટાઇપ્સ} & ASCII (ટેક્સ્ટ) અને બાઇનરી \\ \hline
\end{tabulary}
\end{center}

\begin{itemize}
    \item \keyword{ડ્યુઅલ ચેનલ}: ડેટા ટ્રાન્સફરથી કમાન્ડ્સ અલગ કરે છે
    \item \keyword{કમાન્ડ્સ}: GET, PUT, LIST, DELETE, RENAME
    \item \keyword{સિક્યોરિટી}: બેઝિક FTP અનસિક્યોર છે; FTPS/SFTP વાપરો
\end{itemize}
\end{solutionbox}

\begin{mnemonicbox}
\mnemonic{CDATA - કંટ્રોલ ચેનલ, ડેટા ચેનલ, એક્ટિવ/પેસિવ મોડ્સ, ટ્રાન્સફર ટાઇપ્સ, ઓથેન્ટિકેશન}
\end{mnemonicbox}

\questionmarks{3(અ) અથવા}{3}{PANનું આખું નામ શું છે? PAN વિગતવાર સમજાવો.}

\begin{solutionbox}
\textbf{પર્સનલ એરિયા નેટવર્ક (PAN):}

\begin{center}
\begin{tikzpicture}[node distance=1.5cm]
    \node [gtu state] (user) {User};
    \node [gtu block, above=of user] (phone) {Phone};
    \node [gtu block, right=of user] (laptop) {Laptop};
    \node [gtu block, left=of user] (watch) {Watch};
    \node [gtu block, below=of user] (buds) {Earbuds};
    
    \draw [dashed] (user) -- (phone);
    \draw [dashed] (user) -- (laptop);
    \draw [dashed] (user) -- (watch);
    \draw [dashed] (user) -- (buds);
\end{tikzpicture}
\captionof{figure}{PAN ઇકોસિસ્ટમ}
\end{center}

\begin{itemize}
    \item \keyword{સ્કોપ}: ખૂબ નાનો (1-10 મીટર), વ્યક્તિ પર કેન્દ્રિત
    \item \keyword{ટેક}: Bluetooth, Zigbee, NFC (વાયરલેસ); USB (વાયર્ડ)
    \item \keyword{ઉપયોગ}: ડેટા સિન્ક, ઓડિયો સ્ટ્રીમિંગ, વેરેબલ્સ
\end{itemize}
\end{solutionbox}

\begin{mnemonicbox}
\mnemonic{PIPER - પર્સનલ, ઇન્ડિવિજ્યુઅલ, પ્રોક્સિમિટી, ઇઝી સેટઅપ, રિડ્યુસ્ડ રેન્જ}
\end{mnemonicbox}

\questionmarks{3(બ) અથવા}{4}{બ્રિજનું મહત્વ શું છે? બ્રિજ પર ટૂંકનોંધ લખો.}

\begin{solutionbox}
\textbf{બ્રિજ ઓપરેશન:}

\begin{center}
\begin{tikzpicture}[node distance=2cm]
    \node [gtu block] (bridge) {Bridge};
    \node [gtu container, left=of bridge, label=below:Segment A] (segA) {Nodes};
    \node [gtu container, right=of bridge, label=below:Segment B] (segB) {Nodes};
    
    \draw [gtu arrow, <->] (bridge) -- (segA);
    \draw [gtu arrow, <->] (bridge) -- (segB);
    
    \node [below=0.5cm of bridge, font=\footnotesize] {Filters by MAC};
\end{tikzpicture}
\captionof{figure}{નેટવર્ક બ્રિજ}
\end{center}

\begin{itemize}
    \item \keyword{લેયર}: ડેટા લિંક લેયર (લેયર 2)
    \item \keyword{ફંક્શન}: સેગમેન્ટ્સ જોડે છે, MAC એડ્રેસ દ્વારા ટ્રાફિક ફિલ્ટર કરે છે
    \item \keyword{ફાયદો}: કોલિઝન ડોમેન ઘટાડે છે, ટ્રાફિક ઘટાડે છે
    \item \keyword{પ્રકારો}: ટ્રાન્સપેરન્ટ, સોર્સ-રૂટ
\end{itemize}
\end{solutionbox}

\begin{mnemonicbox}
\mnemonic{SELF - સેગમેન્ટેશન, એક્સટેન્શન, લર્નિંગ એડ્રેસિસ, ફિલ્ટરિંગ ટ્રાફિક}
\end{mnemonicbox}

\questionmarks{3(ક) અથવા}{7}{DSL શું છે? તેનાં જુદા-જુદા પ્રકાર સમજાવો.}

\begin{solutionbox}
\textbf{ડિજિટલ સબસ્ક્રાઇબર લાઇન (DSL):}

\begin{center}
\begin{tikzpicture}[node distance=2cm]
    \node [gtu block] (home) {Home Modem};
    \node [gtu block, right=3cm of home] (dslam) {ISP DSLAM};
    
    \draw [thick] (home) -- node[above] {Phone Line} (dslam);
    \node [below=0.5cm of home] {Data + Voice};
\end{tikzpicture}
\captionof{figure}{DSL કનેક્શન}
\end{center}

\textbf{DSL પ્રકારો:}

\begin{center}
\captionof{table}{DSL વેરિઅન્ટ્સ}
\begin{tabulary}{\linewidth}{|L|L|L|}
\hline
\textbf{પ્રકાર} & \textbf{નામ} & \textbf{લાક્ષણિકતાઓ} \\ \hline
\textbf{ADSL} & અસિમેટ્રિક & અપલોડ કરતાં ડાઉનલોડ ફાસ્ટ (ઘર વપરાશ) \\ \hline
\textbf{SDSL} & સિમેટ્રિક & સમાન સ્પીડ (બિઝનેસ વપરાશ) \\ \hline
\textbf{VDSL} & વેરી હાઇ-બિટ-રેટ & ખૂબ ફાસ્ટ, ટૂંકું અંતર \\ \hline
\textbf{HDSL} & હાઇ-બિટ-રેટ & T1/E1 રિપ્લેસમેન્ટ \\ \hline
\end{tabulary}
\end{center}

\begin{itemize}
    \item \keyword{મિકેનિઝમ}: કોપર ફોન લાઇન્સ પર ઉચ્ચ ફ્રિક્વન્સી વાપરે છે
    \item \keyword{ફાયદો}: વોઇસ અને ડેટા એકસાથે, ઓલવેઝ-ઓન
\end{itemize}
\end{solutionbox}

\begin{mnemonicbox}
\mnemonic{SAVHI - સિમેટ્રિક, અસિમેટ્રિક, વેરી હાઇ-બિટ-રેટ, હાઇ-બિટ-રેટ, ISDN DSL}
\end{mnemonicbox}

\questionmarks{4(અ)}{3}{ડેટા લિંક લેયર માટે એરર કન્ટ્રોલ અને ફ્લો કન્ટ્રોલ સમજાવો.}

\begin{solutionbox}
\textbf{ડેટા લિંક કંટ્રોલ્સ:}

\begin{center}
\captionof{table}{કંટ્રોલ મિકેનિઝમ્સ}
\begin{tabulary}{\linewidth}{|L|L|L|}
\hline
\textbf{મિકેનિઝમ} & \textbf{હેતુ} & \textbf{ટેકનિક્સ} \\ \hline
\textbf{એરર કંટ્રોલ} & એરર ડિટેક્ટ/ફિક્સ કરવી & CRC, ચેકસમ, રિટ્રાન્સમિશન (ARQ) \\ \hline
\textbf{ફ્લો કંટ્રોલ} & ઓવરફ્લો અટકાવવો & સ્ટોપ-એન્ડ-વેઇટ, સ્લાઇડિંગ વિન્ડો \\ \hline
\end{tabulary}
\end{center}

\textbf{ફ્લો કંટ્રોલ આકૃતિ:}

\begin{center}
\begin{tikzpicture}[node distance=2cm]
    \node [gtu state] (s) {Sender};
    \node [gtu state, right=of s] (r) {Receiver};
    
    \draw [gtu arrow, bend left] (s) to node[above] {Data} (r);
    \draw [gtu arrow, bend left, dashed] (r) to node[below] {Stop / ACK} (s);
\end{tikzpicture}
\captionof{figure}{ફ્લો કંટ્રોલ કન્સેપ્ટ}
\end{center}
\end{solutionbox}

\begin{mnemonicbox}
\mnemonic{SAFE - સ્ટોપ-એન્ડ-વેઇટ, એકનોલેજમેન્ટ, ફ્લો કંટ્રોલ, એરર ડિટેક્શન}
\end{mnemonicbox}

\questionmarks{4(બ)}{4}{ફાયરવોલ શું છે? વિગતવાર સમજાવો.}

\begin{solutionbox}
\textbf{ફાયરવોલ ઓપરેશન:}

\begin{center}
\begin{tikzpicture}[node distance=2cm]
    \node [gtu state] (inet) {Internet};
    \node [gtu block, fill=red!20, right=of inet] (fw) {Firewall};
    \node [gtu state, right=of fw] (lan) {Intranet};
    
    \draw [gtu arrow, <->] (inet) -- (fw);
    \draw [gtu arrow, <->] (fw) -- (lan);
    
    \node [below=0.5cm of fw, font=\footnotesize] {Allow/Block Rules};
\end{tikzpicture}
\captionof{figure}{નેટવર્ક ફાયરવોલ}
\end{center}

\textbf{પ્રકારો:}

\begin{itemize}
    \item \keyword{પેકેટ ફિલ્ટરિંગ}: હેડર્સ તપાસે છે (IP/Port)
    \item \keyword{સ્ટેટફુલ}: કનેક્શન સ્ટેટ ટ્રેક કરે છે
    \item \keyword{એપ્લિકેશન}: પેલોડ ડેટા તપાસે છે
    \item \keyword{નેક્સ્ટ-જનરેશન}: ઇન્ટિગ્રેટેડ સિક્યોરિટી ફીચર્સ
\end{itemize}
\end{solutionbox}

\begin{mnemonicbox}
\mnemonic{PAPSI - પેકેટ ફિલ્ટરિંગ, એપ્લિકેશન લેયર, પોલિસીઝ, સ્ટેટફુલ ઇન્સ્પેક્શન}
\end{mnemonicbox}

\questionmarks{4(ક)}{7}{IPV4 અને IPV6ને સરખાવો.}

\begin{solutionbox}
\textbf{સરખામણી:}

\begin{center}
\captionof{table}{IPv4 vs IPv6}
\begin{tabulary}{\linewidth}{|L|L|L|}
\hline
\textbf{ફીચર} & \textbf{IPv4} & \textbf{IPv6} \\ \hline
\textbf{સાઇઝ} & 32-બિટ (4.3B એડ્રેસ) & 128-બિટ (Undecillion) \\ \hline
\textbf{ફોર્મેટ} & ડોટેડ ડેસિમલ & હેક્સાડેસિમલ વિથ કોલન \\ \hline
\textbf{હેડર} & વેરિએબલ (20-60B) & ફિક્સ્ડ (40B) \\ \hline
\textbf{સિક્યોરિટી} & ઓપ્શનલ (IPSec) & બિલ્ટ-ઇન (IPSec) \\ \hline
\textbf{ચેકસમ} & હેડરમાં છે & દૂર કર્યું છે \\ \hline
\end{tabulary}
\end{center}

\textbf{હેડર સ્ટ્રક્ચર્સ:}

\begin{center}
\begin{tikzpicture}[node distance=0.5cm, font=\tiny]
    \node [draw, rectangle, minimum width=3cm] (h4) {IPv4: Ver | Len | ... | Source | Dest};
    \node [draw, rectangle, minimum width=4cm, right=of h4] (h6) {IPv6: Ver | Class | Flow | Source | Dest};
\end{tikzpicture}
\captionof{figure}{સરળ હેડર સ્ટ્રક્ચર}
\end{center}

\begin{itemize}
    \item \keyword{ઓટો-કોન્ફિગ}: IPv6 સ્ટેટલેસ ઓટો-કોન્ફિગરેશન (SLAAC) સપોર્ટ કરે છે
    \item \keyword{નો NAT}: IPv6 માં NAT ની જરૂર નથી, એન્ડ-ટુ-એન્ડ કનેક્ટિવિટી
\end{itemize}
\end{solutionbox}

\begin{mnemonicbox}
\mnemonic{SHAPE - સાઇઝ, હેડર, એડ્રેસિંગ, પરફોર્મન્સ, એક્સટેન્સિબિલિટી}
\end{mnemonicbox}

\questionmarks{4(અ) અથવા}{3}{IP એડ્રેસ શું છે? તે નેટવર્કમાં કઈ રીતે ઉપયોગી છે?}

\begin{solutionbox}
\textbf{IP એડ્રેસ વ્યાખ્યા:}
નેટવર્ક પર કોમ્યુનિકેશન માટે દરેક ડિવાઇસને અસાઇન કરેલ ન્યુમેરિકલ આઈડેન્ટિફાયર.

\textbf{આકૃતિ:}

\begin{center}
\begin{tikzpicture}[node distance=0cm, outer sep=0pt]
    \node [draw, rectangle, minimum width=3cm] (ip) {\large 192.168.1.100};
    \node [below left=0.5cm of ip] (net) {Network ID};
    \node [below right=0.5cm of ip] (host) {Host ID};
    
    \draw [->] (net) -- (ip.south west);
    \draw [->] (host) -- (ip.south east);
\end{tikzpicture}
\captionof{figure}{IPv4 સ્ટ્રક્ચર}
\end{center}

\begin{itemize}
    \item \keyword{આઈડેન્ટિફિકેશન}: ડિવાઇસને અનન્ય રીતે ઓળખે છે
    \item \keyword{એડ્રેસિંગ}: ડિવાઇસનું સ્થાન નક્કી કરે છે (પોસ્ટલ એડ્રેસની જેમ)
    \item \keyword{રાઉટિંગ}: નેટવર્કમાં પાથ શોધવા માટે સક્ષમ કરે છે
\end{itemize}
\end{solutionbox}

\begin{mnemonicbox}
\mnemonic{IRAN - આઈડેન્ટિફિકેશન, રાઉટિંગ, એડ્રેસિંગ, નેટવર્ક ડિવિઝન}
\end{mnemonicbox}

\questionmarks{4(બ) અથવા}{4}{FDDI અને CDDIને સરખાવો.}

\begin{solutionbox}
\textbf{FDDI vs CDDI:}

\begin{center}
\captionof{table}{ટેકનોલોજી સરખામણી}
\begin{tabulary}{\linewidth}{|L|L|L|}
\hline
\textbf{ફીચર} & \textbf{FDDI (ફાયબર)} & \textbf{CDDI (કોપર)} \\ \hline
\textbf{મીડિયા} & ફાયબર ઓપ્ટિક & ટ્વિસ્ટેડ પેર (કોપર) \\ \hline
\textbf{સ્પીડ} & 100 Mbps & 100 Mbps \\ \hline
\textbf{રેન્જ} & 200 km સુધી & ~100 મીટર \\ \hline
\textbf{કિંમત} & ઉચ્ચ & ઓછી \\ \hline
\textbf{ટોપોલોજી} & ડ્યુઅલ રિંગ & ડ્યુઅલ રિંગ \\ \hline
\end{tabulary}
\end{center}

\textbf{ડ્યુઅલ રિંગ ટોપોલોજી:}

\begin{center}
\begin{tikzpicture}[node distance=1.5cm]
    \node [gtu state] (n1) {1};
    \node [gtu state, right=of n1] (n2) {2};
    \node [gtu state, below=of n2] (n3) {3};
    \node [gtu state, left=of n3] (n4) {4};
    
    \draw [gtu arrow] (n1) -- (n2);
    \draw [gtu arrow] (n2) -- (n3);
    \draw [gtu arrow] (n3) -- (n4);
    \draw [gtu arrow] (n4) -- (n1);
    
    \draw [gtu arrow, dashed] (n1) -- (n4);
    \draw [gtu arrow, dashed] (n4) -- (n3);
    \draw [gtu arrow, dashed] (n3) -- (n2);
    \draw [gtu arrow, dashed] (n2) -- (n1);
\end{tikzpicture}
\captionof{figure}{ડ્યુઅલ કાઉન્ટર-રોટેટિંગ રિંગ્સ}
\end{center}
\end{solutionbox}

\begin{mnemonicbox}
\mnemonic{FDDI ફ્લાઇઝ, CDDI ક્રોલ્સ (અંતર મુજબ)}
\end{mnemonicbox}

\questionmarks{4(ક) અથવા}{7}{OSI રેફરન્સ મોડેલ દોરો અને વિગતવાર સમજાવો.}

\begin{solutionbox}
\textbf{OSI લેયર્ડ મોડેલ:}

\begin{center}
\begin{tikzpicture}[node distance=0cm, outer sep=0pt]
    \node [draw, rectangle, minimum width=5cm, minimum height=0.7cm, fill=blue!10] (l7) {7. Application (User)};
    \node [draw, rectangle, minimum width=5cm, minimum height=0.7cm, below=of l7] (l6) {6. Presentation (Format)};
    \node [draw, rectangle, minimum width=5cm, minimum height=0.7cm, below=of l6] (l5) {5. Session (Dialog)};
    \node [draw, rectangle, minimum width=5cm, minimum height=0.7cm, below=of l5, fill=yellow!10] (l4) {4. Transport (Segments)};
    \node [draw, rectangle, minimum width=5cm, minimum height=0.7cm, below=of l4, fill=yellow!10] (l3) {3. Network (Packets)};
    \node [draw, rectangle, minimum width=5cm, minimum height=0.7cm, below=of l3, fill=green!10] (l2) {2. Data Link (Frames)};
    \node [draw, rectangle, minimum width=5cm, minimum height=0.7cm, below=of l2, fill=green!10] (l1) {1. Physical (Bits)};
\end{tikzpicture}
\captionof{figure}{OSI 7-લેયર મોડેલ}
\end{center}

\textbf{વિશ્લેષણ:}
\begin{itemize}
    \item \keyword{એન્કેપ્સુલેશન}: ડેટા લેયર્સમાં નીચે જાય છે, હેડરો ઉમેરાય છે.
    \item \keyword{લેયર્સ 1-3}: મીડિયા લેયર્સ (નેટવર્ક સ્પેસિફિક).
    \item \keyword{લેયર્સ 4-7}: હોસ્ટ લેયર્સ (એપ્લિકેશન સ્પેસિફિક).
\end{itemize}
\end{solutionbox}

\begin{mnemonicbox}
\mnemonic{All People Seem To Need Data Processing}
\end{mnemonicbox}

\questionmarks{5(અ)}{3}{ISO શું છે? ઇન્ફોમેશન સિક્યોરિટીમાં કઈ રીતે કામ કરે છે?}

\begin{solutionbox}
\textbf{વ્યાખ્યા:}
ISO (ઇન્ટરનેશનલ ઓર્ગેનાઇઝેશન ફોર સ્ટાન્ડર્ડાઇઝેશન) ગ્લોબલ સ્ટાન્ડર્ડ્સ બનાવે છે, જેમાં સિક્યોરિટી માટે ISO 27000 સિરીઝનો સમાવેશ થાય છે.

\textbf{કાર્યક્ષમતા:}
\begin{itemize}
    \item \keyword{સ્ટાન્ડર્ડ્સ}: ISO 27001 (ISMS), 27002 (કંટ્રોલ્સ).
    \item \keyword{ફ્રેમવર્ક}: સ્ટ્રક્ચર્ડ રિસ્ક મેનેજમેન્ટ (ISMS) પૂરું પાડે છે.
    \item \keyword{કમ્પ્લાયન્સ}: સંસ્થાઓ વિશ્વાસ માટે પ્રમાણિત થાય છે.
\end{itemize}
\end{solutionbox}

\begin{mnemonicbox}
\mnemonic{PRIMP - પોલિસીઝ, રિસ્ક અસેસમેન્ટ, ઇમ્પ્લિમેન્ટેશન, મોનિટરિંગ, પ્રોસેસ ઇમ્પ્રુવમેન્ટ}
\end{mnemonicbox}

\questionmarks{5(બ)}{4}{ક્રિપ્ટોગ્રાફીની ટર્મ વિગતવાર સમજાવો: ૧) એનક્રિપ્શન ૨) ડીક્રિપ્શન}

\begin{solutionbox}
\textbf{ક્રિપ્ટોગ્રાફી કન્સેપ્ટ્સ:}

\begin{center}
\begin{tikzpicture}[node distance=2.5cm]
    \node [gtu block] (plain) {Plaintext};
    \node [gtu block, right=of plain] (cipher) {Ciphertext};
    \node [gtu block, right=of cipher] (plain2) {Plaintext};
    
    \draw [gtu arrow] (plain) -- node[above] {Encrypt} node[below] {Key} (cipher);
    \draw [gtu arrow] (cipher) -- node[above] {Decrypt} node[below] {Key} (plain2);
\end{tikzpicture}
\captionof{figure}{ક્રિપ્ટો પ્રક્રિયા}
\end{center}

\begin{itemize}
    \item \keyword{એન્ક્રિપ્શન}: ગોપનીયતા સુનિશ્ચિત કરવા પ્લેનટેક્સ્ટને અનરીડેબલ સાયફરટેક્સ્ટમાં કન્વર્ટ કરવું. (AES, RSA).
    \item \keyword{ડિક્રિપ્શન}: સાચી કી વાપરીને સાયફરટેક્સ્ટને પ્લેનટેક્સ્ટમાં પાછું ફેરવવું.
\end{itemize}
\end{solutionbox}

\begin{mnemonicbox}
\mnemonic{PACK-DUKE - પ્લેનટેક્સ્ટ એલ્ગોરિધમ સાયફર કી - ડિકોડિંગ યુઝિંગ કી એક્સટ્રેક્શન}
\end{mnemonicbox}

\questionmarks{5(ક)}{7}{ટૂંકનોંધ લખો ૧) ઈ-મેઈલ 2) DNS}

\begin{solutionbox}
\textbf{1) ઈ-મેઈલ સિસ્ટમ:}

\begin{center}
\begin{tikzpicture}[node distance=2cm]
    \node [gtu state] (sender) {Sender};
    \node [gtu block, right=of sender] (smtp) {SMTP Server};
    \node [gtu block, right=of smtp] (imap) {Mail Server};
    \node [gtu state, right=of imap] (rec) {Receiver};
    
    \draw [gtu arrow] (sender) -- (smtp);
    \draw [gtu arrow] (smtp) -- (imap);
    \draw [gtu arrow] (imap) -- (rec);
    
    \node [below=0.5cm of sender] {User Agent};
    \node [below=0.5cm of rec] {POP3/IMAP};
\end{tikzpicture}
\captionof{figure}{ઇમેઇલ ફ્લો}
\end{center}

\begin{itemize}
    \item \keyword{પ્રોટોકોલ્સ}: SMTP (મોકલવા), POP3/IMAP (મેળવવા).
    \item \keyword{કોમ્પોનન્ટસ}: MUA (ક્લાયન્ટ), MTA (સર્વર).
\end{itemize}

\textbf{2) DNS (ડોમેન નેમ સિસ્ટમ):}

\begin{center}
\begin{tikzpicture}[node distance=1.5cm]
    \node [gtu block] (root) {Root (.)};
    \node [gtu block, below left=of root] (com) {.com TLD};
    \node [gtu block, below right=of root] (org) {.org TLD};
    \node [gtu block, below=of com] (ex) {example.com};
    
    \draw [gtu arrow] (root) -- (com);
    \draw [gtu arrow] (root) -- (org);
    \draw [gtu arrow] (com) -- (ex);
\end{tikzpicture}
\captionof{figure}{DNS હાયરાર્કી}
\end{center}

\begin{itemize}
    \item \keyword{ફંક્શન}: ડોમેન નેમ્સ -> IP એડ્રેસ.
    \item \keyword{હાયરાર્કી}: રૂટ -> TLD -> ઓથોરિટેટિવ.
    \item \keyword{રેકોર્ડ્સ}: A (IPv4), AAAA (IPv6), MX (Mail).
\end{itemize}
\end{solutionbox}

\begin{mnemonicbox}
\mnemonic{MAPS - મેઇલ નીડ્સ એડ્રેસિસ, પ્રોટોકોલ્સ, એન્ડ સર્વર્સ. HARD - હાયરાર્કી, એડ્રેસિંગ, રિઝોલ્યુશન, ડિસ્ટ્રિબ્યુટેડ}
\end{mnemonicbox}

\questionmarks{5(અ) અથવા}{3}{સિક્યોરીટી ટોપોલોજી અને સિક્યોરીટી ઝોન શું છે?}

\begin{solutionbox}
\textbf{કન્સેપ્ટ્સ:}

\begin{center}
\begin{tikzpicture}[node distance=1.5cm]
    \node [gtu state] (net) {Internet};
    \node [gtu block, right=of net] (fw) {FW};
    \node [gtu class, right=of fw, fill=yellow!10] (dmz) {DMZ Zone};
    \node [gtu class, below=of fw, fill=green!10] (lan) {Private Zone};
    
    \draw [gtu arrow] (net) -- (fw);
    \draw [gtu arrow] (fw) -- (dmz);
    \draw [gtu arrow] (fw) -- (lan);
\end{tikzpicture}
\captionof{figure}{સિક્યોરિટી ઝોન્સ}
\end{center}

\begin{itemize}
    \item \keyword{સિક્યોરિટી ટોપોલોજી}: સિક્યોરિટી કંટ્રોલ્સ (ફાયરવોલ, IDS) ની ફિઝિકલ/લોજિકલ રચના.
    \item \keyword{સિક્યોરિટી ઝોન}: ચોક્કસ ટ્રસ્ટ લેવલ સાથે નેટવર્ક સેગમેન્ટ (દા.ત., DMZ vs ઇન્ટરનલ).
\end{itemize}
\end{solutionbox}

\begin{mnemonicbox}
\mnemonic{TIPS - ટોપોલોજી આઇસોલેટ્સ એન્ડ પ્રોટેક્ટ્સ સિસ્ટમ્સ}
\end{mnemonicbox}

\questionmarks{5(બ) અથવા}{4}{વોઇસ અને વિડીયો IP પર ટૂંકનોંધ લખો.}

\begin{solutionbox}
\textbf{VoIP / Video over IP:}

\begin{center}
\begin{tikzpicture}[node distance=2.5cm]
    \node [gtu state] (user1) {User};
    \node [gtu state, right=of user1] (user2) {User};
    
    \draw [gtu arrow, <->, dashed] (user1) -- node[above] {IP Network} node[below] {Data Packets} (user2);
    
    \node [below=0.5cm of user1, font=\tiny] {Digitize -> Packetize};
    \node [below=0.5cm of user2, font=\tiny] {Depacketize -> Analog};
\end{tikzpicture}
\captionof{figure}{પેકેટ વોઇસ/વિડિયો}
\end{center}

\begin{center}
\captionof{table}{મુખ્ય કોમ્પોનન્ટસ}
\begin{tabulary}{\linewidth}{|L|L|}
\hline
\textbf{કોમ્પોનન્ટ} & \textbf{ઉદાહરણો} \\ \hline
\textbf{કોડેક્સ} & G.711 (Voice), H.264 (Video) \\ \hline
\textbf{પ્રોટોકોલ્સ} & SIP (Setup), RTP (Transport) \\ \hline
\textbf{આવશ્યકતાઓ} & લો લેટન્સી (QoS), હાઈ બેન્ડવિડ્થ \\ \hline
\end{tabulary}
\end{center}
\end{solutionbox}

\begin{mnemonicbox}
\mnemonic{CLEAR - કોડેક્સ કમ્પ્રેસ, લેટન્સી મેટર્સ, એન્કોડ્સ AV, રિયલ-ટાઇમ ટ્રાન્સપોર્ટ}
\end{mnemonicbox}

\questionmarks{5(ક) અથવા}{7}{IP સિક્યોરીટી શું છે? વિગતવાર સમજાવો.}

\begin{solutionbox}
\textbf{IPsec (ઇન્ટરનેટ પ્રોટોકોલ સિક્યોરિટી):}
ઓથેન્ટિકેશન અને એન્ક્રિપ્શન દ્વારા IP કોમ્યુનિકેશન સુરક્ષિત કરવા માટે પ્રોટોકોલ સ્યુટ.

\begin{center}
\begin{tikzpicture}[node distance=2cm]
    \node [gtu block, fill=blue!10] (ipsec) {IPsec Suite};
    \node [gtu block, below left=of ipsec] (ah) {AH (Auth)};
    \node [gtu block, below=of ipsec] (esp) {ESP (Encrypt)};
    \node [gtu block, below right=of ipsec] (ike) {IKE (Keys)};
    
    \draw [gtu arrow] (ipsec) -- (ah);
    \draw [gtu arrow] (ipsec) -- (esp);
    \draw [gtu arrow] (ipsec) -- (ike);
\end{tikzpicture}
\captionof{figure}{IPsec કોમ્પોનન્ટસ}
\end{center}

\textbf{કોમ્પોનન્ટસ:}
\begin{itemize}
    \item \keyword{AH (ઓથેન્ટિકેશન હેડર)}: ઇન્ટિગ્રિટી અને ઓથેન્ટિકેશન. એન્ક્રિપ્શન નહીં.
    \item \keyword{ESP (એન્કેપ્સુલેટિંગ સિક્યોરિટી પેલોડ)}: એન્ક્રિપ્શન + ઇન્ટિગ્રિટી + ઓથેન્ટિકેશન.
    \item \keyword{IKE (ઇન્ટરનેટ કી એક્સચેન્જ)}: કી નેગોશિએશન (SA).
\end{itemize}

\textbf{મોડ્સ:}
\begin{itemize}
    \item \keyword{ટ્રાન્સપોર્ટ મોડ}: માત્ર પેલોડ એન્ક્રિપ્ટ કરે છે (હોસ્ટ-ટુ-હોસ્ટ).
    \item \keyword{ટનલ મોડ}: આખું પેકેટ એન્ક્રિપ્ટ કરે છે (VPNs).
\end{itemize}
\end{solutionbox}

\begin{mnemonicbox}
\mnemonic{AVID TC - ઓથેન્ટિકેશન, વેરિફિકેશન, ઇન્ટિગ્રિટી, ડેટાગ્રામ પ્રોટેક્શન, ટ્રાન્સપોર્ટ/ટનલ મોડ્સ, કોન્ફિડેન્શિયાલિટી}
\end{mnemonicbox}

\end{document}
