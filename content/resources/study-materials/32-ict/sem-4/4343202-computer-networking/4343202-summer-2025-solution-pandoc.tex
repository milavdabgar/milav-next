\documentclass[10pt,a4paper]{article}

% content/resources/templates/preamble.tex
\usepackage[margin=0.6in]{geometry}
\author{Milav Dabgar}
\usepackage{amsmath,amssymb,amsthm}
\usepackage{booktabs}
\usepackage{multirow}
\usepackage{xcolor}
\usepackage{tcolorbox}
\tcbuselibrary{breakable,skins}
\usepackage[colorlinks=true,linkcolor=blue]{hyperref}
\usepackage{titlesec}
\usepackage{enumitem}
\usepackage{tikz}
\usepackage{pgfplots}
\usepackage{circuitikz}
\usepackage[version=4]{mhchem}
\usepackage{longtable}
\usepackage{array}
\usepackage{float}
\usepackage{caption}
\usepackage{listings}

\lstset{
  basicstyle=\small\ttfamily,
  breaklines=true,
  breakatwhitespace=false,
  postbreak=\mbox{\textcolor{red}{$\hookrightarrow$}\space},
  float=false,
  numbers=left,
  numberstyle=\tiny\color{gray},
  numbersep=10pt,
  xleftmargin=2em,
  keywordstyle=\color{blue},
  commentstyle=\color{green!60!black},
  stringstyle=\color{purple},
  backgroundcolor=\color{gray!5},
  showstringspaces=false,
  tabsize=2,
  captionpos=b,
  keepspaces=true,
  columns=flexible
}

\pgfplotsset{compat=1.18}
\usetikzlibrary{shapes,arrows,positioning,calc,patterns,decorations.pathmorphing,decorations.markings,arrows.meta}

% Color scheme
\definecolor{headcolor}{RGB}{0,102,204}
\definecolor{keycolor}{RGB}{220,20,60}
\definecolor{solutioncolor}{RGB}{34,139,34}
\definecolor{mnemoniccolor}{RGB}{148,0,211}
\definecolor{codecolor}{RGB}{0,0,100}

% Spacing
\setlength{\parskip}{3pt}
\setlist[itemize]{nosep}
\setlist[enumerate]{nosep}

% Title formatting
\titleformat{\section}{\Large\bfseries\color{headcolor}}{\thesection}{1em}{}
\titleformat{\subsection}{\large\bfseries\color{headcolor}}{\thesubsection}{1em}{}

% Pandoc tightlist compatibility
\providecommand{\tightlist}{%
  \setlength{\itemsep}{0pt}\setlength{\parskip}{0pt}}

% Pandoc longtable compatibility
\newcounter{none}
\def\thenone{}


% content/resources/templates/english-boxes.tex
% This file is currently empty - it exists to maintain consistency with the import structure.
% Add custom environments here if needed in the future.


\begin{document}

\begin{center}
{\Huge\bfseries\color{headcolor} Subject Name Solutions}\\[5pt]
{\LARGE 4343202 -- Summer 2025}\\[3pt]
{\large Semester 1 Study Material}\\[3pt]
{\normalsize\textit{Detailed Solutions and Explanations}}
\end{center}

\vspace{10pt}

\subsection*{Question 1(a) [3 marks]}\label{q1a}

\textbf{List Various network Topologies of computer network and explain
any one.}

\begin{solutionbox}


{\def\LTcaptype{none} % do not increment counter
\vspace{-5pt}
\captionof{table}{Network Topologies}
\vspace{-10pt}
\begin{longtable}[]{@{}ll@{}}
\toprule\noalign{}
Topology & Description \\
\midrule\noalign{}
\endhead
\bottomrule\noalign{}
\endlastfoot
\textbf{Star} & Central hub connects all devices \\
\textbf{Ring} & Devices connected in circular chain \\
\textbf{Bus} & Single cable backbone connection \\
\textbf{Mesh} & Every device connects to every other \\
\textbf{Tree} & Hierarchical branching structure \\
\textbf{Hybrid} & Combination of multiple topologies \\
\end{longtable}
}

\textbf{Star Topology Explanation:}

\begin{itemize}
\tightlist
\item
  \textbf{Central Hub}: All devices connect to one central point
\item
  \textbf{Easy Installation}: Simple to add/remove devices
\item
  \textbf{Single Point Failure}: Hub failure affects entire network
\end{itemize}

\end{solutionbox}
\begin{mnemonicbox}
``SRBMTH - Star Ring Bus Mesh Tree Hybrid''

\end{mnemonicbox}
\subsection*{Question 1(b) [4 marks]}\label{q1b}

\textbf{Compare LAN, WAN and MAN.}

\begin{solutionbox}

\textbf{Comparison Table:}

{\def\LTcaptype{none} % do not increment counter
\begin{longtable}[]{@{}
  >{\raggedright\arraybackslash}p{(\linewidth - 6\tabcolsep) * \real{0.2895}}
  >{\raggedright\arraybackslash}p{(\linewidth - 6\tabcolsep) * \real{0.2368}}
  >{\raggedright\arraybackslash}p{(\linewidth - 6\tabcolsep) * \real{0.2368}}
  >{\raggedright\arraybackslash}p{(\linewidth - 6\tabcolsep) * \real{0.2368}}@{}}
\toprule\noalign{}
\begin{minipage}[b]{\linewidth}\raggedright
Parameter
\end{minipage} & \begin{minipage}[b]{\linewidth}\raggedright
\textbf{LAN}
\end{minipage} & \begin{minipage}[b]{\linewidth}\raggedright
\textbf{MAN}
\end{minipage} & \begin{minipage}[b]{\linewidth}\raggedright
\textbf{WAN}
\end{minipage} \\
\midrule\noalign{}
\endhead
\bottomrule\noalign{}
\endlastfoot
\textbf{Coverage} & Building/Campus & City/Metropolitan &
Country/Global \\
\textbf{Speed} & Very High (1-100 Gbps) & High (10-100 Mbps) & Medium
(1-100 Mbps) \\
\textbf{Cost} & Low & Medium & High \\
\textbf{Ownership} & Private & Public/Private & Public \\
\end{longtable}
}

\textbf{Key Points:}

\begin{itemize}
\tightlist
\item
  \textbf{LAN}: Local Area Network for small areas
\item
  \textbf{MAN}: Metropolitan Area Network for cities
\item
  \textbf{WAN}: Wide Area Network for large distances
\end{itemize}

\end{solutionbox}
\begin{mnemonicbox}
``LMW - Local Metropolitan Wide''

\end{mnemonicbox}
\subsection*{Question 1(c) [7 marks]}\label{q1c}

\textbf{Draw the layered architecture of OSI reference model and write
at least two services provided by each layer of the model.}

\begin{solutionbox}

\includegraphics[width=1\linewidth,height=\textheight,keepaspectratio]{mermaid-8738f82a.pdf}

\textbf{Services by Each Layer:}

{\def\LTcaptype{none} % do not increment counter
\begin{longtable}[]{@{}ll@{}}
\toprule\noalign{}
Layer & \textbf{Services} \\
\midrule\noalign{}
\endhead
\bottomrule\noalign{}
\endlastfoot
\textbf{Application (7)} & Email services, File transfer \\
\textbf{Presentation (6)} & Data encryption, Data compression \\
\textbf{Session (5)} & Session establishment, Session termination \\
\textbf{Transport (4)} & Flow control, Error correction \\
\textbf{Network (3)} & Routing, Path determination \\
\textbf{Data Link (2)} & Frame synchronization, Error detection \\
\textbf{Physical (1)} & Bit transmission, Signal conversion \\
\end{longtable}
}

\end{solutionbox}
\begin{mnemonicbox}
``All People Seem To Need Data Processing''

\end{mnemonicbox}
\subsection*{Question 1(c OR) [7
marks]}\label{question-1c-or-7-marks}

\textbf{Explain Each layer of TCP/IP Model with its protocol.}

\begin{solutionbox}

\includegraphics[width=1\linewidth,height=\textheight,keepaspectratio]{mermaid-7eb79811.pdf}

\textbf{TCP/IP Model Layers:}

{\def\LTcaptype{none} % do not increment counter
\begin{longtable}[]{@{}lll@{}}
\toprule\noalign{}
Layer & \textbf{Protocols} & \textbf{Function} \\
\midrule\noalign{}
\endhead
\bottomrule\noalign{}
\endlastfoot
\textbf{Application} & HTTP, FTP, SMTP, DNS & User applications \\
\textbf{Transport} & TCP, UDP & End-to-end delivery \\
\textbf{Internet} & IP, ICMP, ARP & Routing packets \\
\textbf{Network Access} & Ethernet, Wi-Fi & Physical transmission \\
\end{longtable}
}

\textbf{Key Features:}

\begin{itemize}
\tightlist
\item
  \textbf{Simplified Model}: Only 4 layers vs OSI's 7
\item
  \textbf{Protocol Suite}: Complete networking solution
\item
  \textbf{Internet Standard}: Basis of modern internet
\end{itemize}

\end{solutionbox}
\begin{mnemonicbox}
``ATIN - Application Transport Internet Network''

\end{mnemonicbox}
\subsection*{Question 2(a) [3 marks]}\label{q2a}

\textbf{Explain functions of following network devices: Repeater, Hub}

\begin{solutionbox}

\textbf{Device Functions:}

{\def\LTcaptype{none} % do not increment counter
\begin{longtable}[]{@{}
  >{\raggedright\arraybackslash}p{(\linewidth - 4\tabcolsep) * \real{0.2424}}
  >{\raggedright\arraybackslash}p{(\linewidth - 4\tabcolsep) * \real{0.4242}}
  >{\raggedright\arraybackslash}p{(\linewidth - 4\tabcolsep) * \real{0.3333}}@{}}
\toprule\noalign{}
\begin{minipage}[b]{\linewidth}\raggedright
Device
\end{minipage} & \begin{minipage}[b]{\linewidth}\raggedright
\textbf{Function}
\end{minipage} & \begin{minipage}[b]{\linewidth}\raggedright
\textbf{Layer}
\end{minipage} \\
\midrule\noalign{}
\endhead
\bottomrule\noalign{}
\endlastfoot
\textbf{Repeater} & Signal amplification, Range extension & Physical
(1) \\
\textbf{Hub} & Signal broadcasting, Collision domain sharing & Physical
(1) \\
\end{longtable}
}

\textbf{Details:}

\begin{itemize}
\tightlist
\item
  \textbf{Repeater}: Regenerates weak signals over long distances
\item
  \textbf{Hub}: Connects multiple devices in star topology
\item
  \textbf{Shared Medium}: Both create single collision domain
\end{itemize}

\end{solutionbox}
\begin{mnemonicbox}
``RH - Repeat Hub signals''

\end{mnemonicbox}
\subsection*{Question 2(b) [4 marks]}\label{q2b}

\textbf{Explain the following term 1) FDDI 2) ARP, RARP}

\begin{solutionbox}

\textbf{FDDI (Fiber Distributed Data Interface):}

\begin{itemize}
\tightlist
\item
  \textbf{Technology}: 100 Mbps fiber optic network
\item
  \textbf{Topology}: Dual ring for fault tolerance
\item
  \textbf{Application}: Backbone networks, high reliability
\end{itemize}

\textbf{ARP (Address Resolution Protocol):}

\begin{itemize}
\tightlist
\item
  \textbf{Function}: Maps IP address to MAC address
\item
  \textbf{Process}: Broadcasts request, receives reply
\end{itemize}

\textbf{RARP (Reverse ARP):}

\begin{itemize}
\tightlist
\item
  \textbf{Function}: Maps MAC address to IP address
\item
  \textbf{Usage}: Diskless workstations, boot process
\end{itemize}

\end{solutionbox}
\begin{mnemonicbox}
``FAR - FDDI ARP RARP''

\end{mnemonicbox}
\subsection*{Question 2(c) [7 marks]}\label{q2c}

\textbf{Explain the Function of firewall in network security with
principles and Kerberos-concept.}

\begin{solutionbox}

\textbf{Firewall Functions:}

\includegraphics[width=1\linewidth,height=\textheight,keepaspectratio]{mermaid-fd7e3c9e.pdf}

\textbf{Firewall Principles:}

\begin{itemize}
\tightlist
\item
  \textbf{Packet Filtering}: Examines packet headers
\item
  \textbf{Stateful Inspection}: Tracks connection states
\item
  \textbf{Application Gateway}: Deep packet inspection
\end{itemize}

\textbf{Kerberos Concept:}

\begin{itemize}
\tightlist
\item
  \textbf{Authentication Service}: Secure user verification
\item
  \textbf{Ticket System}: Time-limited access tokens
\item
  \textbf{Three-party Protocol}: Client, Server, Key Distribution Center
\end{itemize}

\textbf{Security Benefits:}

\begin{itemize}
\tightlist
\item
  \textbf{Access Control}: Prevents unauthorized access
\item
  \textbf{Network Protection}: Shields internal resources
\end{itemize}

\end{solutionbox}
\begin{mnemonicbox}
``FPK - Firewall Protects with Kerberos''

\end{mnemonicbox}
\subsection*{Question 2(a OR) [3
marks]}\label{question-2a-or-3-marks}

\textbf{Explain functions of following network devices: Switch, Router}

\begin{solutionbox}

\textbf{Device Functions:}

{\def\LTcaptype{none} % do not increment counter
\begin{longtable}[]{@{}lll@{}}
\toprule\noalign{}
Device & \textbf{Function} & \textbf{Layer} \\
\midrule\noalign{}
\endhead
\bottomrule\noalign{}
\endlastfoot
\textbf{Switch} & MAC address learning, Frame forwarding & Data Link
(2) \\
\textbf{Router} & IP routing, Path selection & Network (3) \\
\end{longtable}
}

\textbf{Details:}

\begin{itemize}
\tightlist
\item
  \textbf{Switch}: Creates separate collision domains per port
\item
  \textbf{Router}: Connects different networks, makes routing decisions
\item
  \textbf{Intelligence}: Switch learns MAC, Router maintains routing
  table
\end{itemize}

\end{solutionbox}
\begin{mnemonicbox}
``SR - Switch Routes intelligently''

\end{mnemonicbox}
\subsection*{Question 2(b OR) [4
marks]}\label{question-2b-or-4-marks}

\textbf{Explain the following term 1) CDDI 2) DHCP and BOOTP}

\begin{solutionbox}

\textbf{CDDI (Copper Distributed Data Interface):}

\begin{itemize}
\tightlist
\item
  \textbf{Technology}: FDDI over copper cables
\item
  \textbf{Speed}: 100 Mbps over twisted pair
\item
  \textbf{Cost}: Cheaper alternative to fiber FDDI
\end{itemize}

\textbf{DHCP (Dynamic Host Configuration Protocol):}

\begin{itemize}
\tightlist
\item
  \textbf{Function}: Automatic IP address assignment
\item
  \textbf{Process}: Discover, Offer, Request, Acknowledge
\item
  \textbf{Benefits}: Centralized IP management
\end{itemize}

\textbf{BOOTP (Bootstrap Protocol):}

\begin{itemize}
\tightlist
\item
  \textbf{Function}: Network bootstrap for diskless clients
\item
  \textbf{Static}: Fixed IP address assignment
\item
  \textbf{Predecessor}: Earlier version of DHCP
\end{itemize}

\end{solutionbox}
\begin{mnemonicbox}
``CDB - CDDI DHCP BOOTP''

\end{mnemonicbox}
\subsection*{Question 2(c OR) [7
marks]}\label{question-2c-or-7-marks}

\textbf{Explain Software define network(SDN) with its Architecture,
Application, Advantage and limitation.}

\begin{solutionbox}

\includegraphics[width=1\linewidth,height=\textheight,keepaspectratio]{mermaid-1233e9a4.pdf}

\textbf{SDN Architecture:}

\begin{itemize}
\tightlist
\item
  \textbf{Control Plane}: Centralized network intelligence
\item
  \textbf{Data Plane}: Packet forwarding devices
\item
  \textbf{Application Plane}: Network applications and services
\end{itemize}

\textbf{Applications:}

\begin{itemize}
\tightlist
\item
  \textbf{Cloud Computing}: Dynamic resource allocation
\item
  \textbf{Network Virtualization}: Multiple virtual networks
\item
  \textbf{Traffic Engineering}: Optimized path selection
\end{itemize}

\textbf{Advantages:}

\begin{itemize}
\tightlist
\item
  \textbf{Centralized Control}: Simplified network management
\item
  \textbf{Programmability}: Custom network behaviors
\item
  \textbf{Flexibility}: Rapid service deployment
\end{itemize}

\textbf{Limitations:}

\begin{itemize}
\tightlist
\item
  \textbf{Single Point Failure}: Controller dependency
\item
  \textbf{Scalability}: Performance bottlenecks
\item
  \textbf{Security}: New attack vectors
\end{itemize}

\end{solutionbox}
\begin{mnemonicbox}
``SCAP - Software Control Application Programmable''

\end{mnemonicbox}
\subsection*{Question 3(a) [3 marks]}\label{q3a}

\textbf{Find the class of following IP address.} \textbf{1) 01111000
00001111 10101010 11000000} \textbf{2) 11101000 01010101 11111111
11000011}

\begin{solutionbox}

\textbf{IP Address Classification:}

{\def\LTcaptype{none} % do not increment counter
\begin{longtable}[]{@{}llll@{}}
\toprule\noalign{}
Binary Address & \textbf{Decimal} & \textbf{First Octet} &
\textbf{Class} \\
\midrule\noalign{}
\endhead
\bottomrule\noalign{}
\endlastfoot
01111000\ldots{} & 120.15.170.192 & 120 (64-127) & \textbf{Class A} \\
11101000\ldots{} & 232.85.255.195 & 232 (224-239) & \textbf{Class D} \\
\end{longtable}
}

\textbf{Class Ranges:}

\begin{itemize}
\tightlist
\item
  \textbf{Class A}: 1-126 (0xxxxxxx)
\item
  \textbf{Class B}: 128-191 (10xxxxxx)
\item
  \textbf{Class C}: 192-223 (110xxxxx)
\item
  \textbf{Class D}: 224-239 (1110xxxx)
\end{itemize}

\textbf{Results:}

\begin{itemize}
\tightlist
\item
  \textbf{First IP}: Class A (Unicast)
\item
  \textbf{Second IP}: Class D (Multicast)
\end{itemize}

\end{solutionbox}
\begin{mnemonicbox}
``ABCD - A(1-126) B(128-191) C(192-223) D(224-239)''

\end{mnemonicbox}
\subsection*{Question 3(b) [4 marks]}\label{q3b}

\textbf{Differentiate IPv4 and IPv6.}

\begin{solutionbox}

\textbf{IPv4 vs IPv6 Comparison:}

{\def\LTcaptype{none} % do not increment counter
\begin{longtable}[]{@{}lll@{}}
\toprule\noalign{}
Feature & \textbf{IPv4} & \textbf{IPv6} \\
\midrule\noalign{}
\endhead
\bottomrule\noalign{}
\endlastfoot
\textbf{Address Length} & 32 bits & 128 bits \\
\textbf{Address Format} & Dotted decimal & Hexadecimal \\
\textbf{Address Space} & 4.3 billion & 340 undecillion \\
\textbf{Header Size} & Variable (20-60 bytes) & Fixed (40 bytes) \\
\textbf{Security} & Optional (IPSec) & Built-in (IPSec) \\
\textbf{Configuration} & Manual/DHCP & Auto-configuration \\
\end{longtable}
}

\textbf{Key Differences:}

\begin{itemize}
\tightlist
\item
  \textbf{Addressing}: IPv6 provides vastly more addresses
\item
  \textbf{Security}: IPv6 has mandatory security features
\item
  \textbf{Performance}: IPv6 has simplified header structure
\end{itemize}

\end{solutionbox}
\begin{mnemonicbox}
``IPv4 to IPv6 = More addresses, Better security''

\end{mnemonicbox}
\subsection*{Question 3(c) [7 marks]}\label{q3c}

\textbf{Explain Static and Dynamic Routing Algorithms.}

\begin{solutionbox}

\textbf{Static Routing:}

\includegraphics[width=1\linewidth,height=\textheight,keepaspectratio]{mermaid-d01ceea4.pdf}

\textbf{Dynamic Routing:}

\includegraphics[width=1\linewidth,height=\textheight,keepaspectratio]{mermaid-e8d69b61.pdf}

\textbf{Comparison Table:}

{\def\LTcaptype{none} % do not increment counter
\begin{longtable}[]{@{}lll@{}}
\toprule\noalign{}
Aspect & \textbf{Static Routing} & \textbf{Dynamic Routing} \\
\midrule\noalign{}
\endhead
\bottomrule\noalign{}
\endlastfoot
\textbf{Configuration} & Manual setup & Automatic discovery \\
\textbf{Adaptability} & No adaptation & Adapts to changes \\
\textbf{Resource Usage} & Low CPU/Memory & Higher CPU/Memory \\
\textbf{Scalability} & Poor for large networks & Good for large
networks \\
\textbf{Protocols} & None required & RIP, OSPF, BGP \\
\end{longtable}
}

\textbf{Applications:}

\begin{itemize}
\tightlist
\item
  \textbf{Static}: Small networks, specific paths
\item
  \textbf{Dynamic}: Large networks, fault tolerance
\end{itemize}

\end{solutionbox}
\begin{mnemonicbox}
``SD - Static=Simple, Dynamic=Automatic''

\end{mnemonicbox}
\subsection*{Question 3(a OR) [3
marks]}\label{question-3a-or-3-marks}

\textbf{Explain CIDR. How does it differ from traditional IP address
allocation methods?}

\begin{solutionbox}

\textbf{CIDR (Classless Inter-Domain Routing):}

\begin{itemize}
\tightlist
\item
  \textbf{Concept}: Variable length subnet masking
\item
  \textbf{Notation}: IP address/prefix length (e.g., 192.168.1.0/24)
\item
  \textbf{Flexibility}: Subnets of any size
\end{itemize}

\textbf{Traditional vs CIDR:}

{\def\LTcaptype{none} % do not increment counter
\begin{longtable}[]{@{}
  >{\raggedright\arraybackslash}p{(\linewidth - 4\tabcolsep) * \real{0.2000}}
  >{\raggedright\arraybackslash}p{(\linewidth - 4\tabcolsep) * \real{0.4000}}
  >{\raggedright\arraybackslash}p{(\linewidth - 4\tabcolsep) * \real{0.4000}}@{}}
\toprule\noalign{}
\begin{minipage}[b]{\linewidth}\raggedright
Method
\end{minipage} & \begin{minipage}[b]{\linewidth}\raggedright
\textbf{Allocation}
\end{minipage} & \begin{minipage}[b]{\linewidth}\raggedright
\textbf{Efficiency}
\end{minipage} \\
\midrule\noalign{}
\endhead
\bottomrule\noalign{}
\endlastfoot
\textbf{Traditional} & Fixed class boundaries & Wasteful (Class B =
65,536 IPs) \\
\textbf{CIDR} & Variable subnet sizes & Efficient allocation \\
\end{longtable}
}

\textbf{Benefits:}

\begin{itemize}
\tightlist
\item
  \textbf{Address Conservation}: Reduces IP address waste
\item
  \textbf{Route Aggregation}: Summarizes multiple routes
\end{itemize}

\end{solutionbox}
\begin{mnemonicbox}
``CIDR = Classless Intelligent Address Routing''

\end{mnemonicbox}
\subsection*{Question 3(b OR) [4
marks]}\label{question-3b-or-4-marks}

\textbf{Describe DSL technology with its types, advantages and
limitations.}

\begin{solutionbox}

\textbf{DSL (Digital Subscriber Line):}

\begin{itemize}
\tightlist
\item
  \textbf{Technology}: High-speed internet over telephone lines
\item
  \textbf{Frequency}: Uses higher frequencies than voice
\end{itemize}

\textbf{DSL Types:}

{\def\LTcaptype{none} % do not increment counter
\begin{longtable}[]{@{}lll@{}}
\toprule\noalign{}
Type & \textbf{Speed} & \textbf{Application} \\
\midrule\noalign{}
\endhead
\bottomrule\noalign{}
\endlastfoot
\textbf{ADSL} & Asymmetric (faster download) & Home users \\
\textbf{SDSL} & Symmetric (equal up/down) & Business \\
\textbf{VDSL} & Very high speed & Short distances \\
\end{longtable}
}

\textbf{Advantages:}

\begin{itemize}
\tightlist
\item
  \textbf{Always-on Connection}: No dial-up required
\item
  \textbf{Existing Infrastructure}: Uses phone lines
\item
  \textbf{Cost-effective}: Affordable high-speed access
\end{itemize}

\textbf{Limitations:}

\begin{itemize}
\tightlist
\item
  \textbf{Distance Dependent}: Speed decreases with distance
\item
  \textbf{Line Quality}: Requires good copper lines
\item
  \textbf{Availability}: Not available everywhere
\end{itemize}

\end{solutionbox}
\begin{mnemonicbox}
``DSL = Digital Speed Limited by distance''

\end{mnemonicbox}
\subsection*{Question 3(c OR) [7
marks]}\label{question-3c-or-7-marks}

\textbf{Explain error control and flow control at data link layer in
detail.}

\begin{solutionbox}

\textbf{Error Control:}

\includegraphics[width=1\linewidth,height=\textheight,keepaspectratio]{mermaid-46f55fe9.pdf}

\textbf{Error Control Methods:}

{\def\LTcaptype{none} % do not increment counter
\begin{longtable}[]{@{}lll@{}}
\toprule\noalign{}
Method & \textbf{Technique} & \textbf{Application} \\
\midrule\noalign{}
\endhead
\bottomrule\noalign{}
\endlastfoot
\textbf{Parity Check} & Single bit error detection & Simple systems \\
\textbf{Checksum} & Mathematical sum verification & TCP/UDP \\
\textbf{CRC} & Polynomial division & Ethernet, Wi-Fi \\
\textbf{ARQ} & Automatic Repeat Request & Reliable protocols \\
\end{longtable}
}

\textbf{Flow Control:}

\includegraphics[width=1\linewidth,height=\textheight,keepaspectratio]{mermaid-a726f25c.pdf}

\textbf{Flow Control Techniques:}

\begin{itemize}
\tightlist
\item
  \textbf{Stop-and-Wait}: Send one frame, wait for ACK
\item
  \textbf{Sliding Window}: Multiple frames in transit
\item
  \textbf{Buffer Management}: Prevents overflow
\end{itemize}

\textbf{Implementation:}

\begin{itemize}
\tightlist
\item
  \textbf{Hardware Level}: Buffer status signals
\item
  \textbf{Software Level}: Protocol acknowledgments
\end{itemize}

\end{solutionbox}
\begin{mnemonicbox}
``EF - Error detection, Flow regulation''

\end{mnemonicbox}
\subsection*{Question 4(a) [3 marks]}\label{q4a}

\textbf{Explain video over IP.}

\begin{solutionbox}

\textbf{Video over IP (VoIP):}

\begin{itemize}
\tightlist
\item
  \textbf{Technology}: Transmits video signals over IP networks
\item
  \textbf{Digitization}: Converts analog video to digital packets
\item
  \textbf{Real-time}: Requires low latency transmission
\end{itemize}

\textbf{Components:}

\begin{itemize}
\tightlist
\item
  \textbf{Encoder}: Compresses video data
\item
  \textbf{Network}: IP infrastructure for transport
\item
  \textbf{Decoder}: Decompresses at destination
\end{itemize}

\textbf{Applications:}

\begin{itemize}
\tightlist
\item
  \textbf{Video Conferencing}: Business communications
\item
  \textbf{Streaming}: Entertainment services
\item
  \textbf{Surveillance}: Security systems
\end{itemize}

\textbf{Requirements:}

\begin{itemize}
\tightlist
\item
  \textbf{Bandwidth}: High data rate needs
\item
  \textbf{QoS}: Quality of Service guarantees
\end{itemize}

\end{solutionbox}
\begin{mnemonicbox}
``VIP = Video Internet Protocol''

\end{mnemonicbox}
\subsection*{Question 4(b) [4 marks]}\label{q4b}

\textbf{Explain Electronic-Mail with its protocol.}

\begin{solutionbox}

\textbf{Email System Components:}

\includegraphics[width=1\linewidth,height=\textheight,keepaspectratio]{mermaid-960e08d4.pdf}

\textbf{Email Protocols:}

{\def\LTcaptype{none} % do not increment counter
\begin{longtable}[]{@{}lll@{}}
\toprule\noalign{}
Protocol & \textbf{Function} & \textbf{Port} \\
\midrule\noalign{}
\endhead
\bottomrule\noalign{}
\endlastfoot
\textbf{SMTP} & Send/relay messages & 25, 587 \\
\textbf{POP3} & Download messages & 110 \\
\textbf{IMAP} & Server-based access & 143 \\
\end{longtable}
}

\textbf{Protocol Details:}

\begin{itemize}
\tightlist
\item
  \textbf{SMTP}: Simple Mail Transfer Protocol for sending
\item
  \textbf{POP3}: Downloads mail to local device
\item
  \textbf{IMAP}: Keeps mail on server, multi-device access
\end{itemize}

\textbf{Message Flow:}

\begin{itemize}
\tightlist
\item
  \textbf{Composition}: User creates message
\item
  \textbf{Submission}: SMTP sends to server
\item
  \textbf{Delivery}: Server forwards to recipient
\item
  \textbf{Retrieval}: POP3/IMAP downloads message
\end{itemize}

\end{solutionbox}
\begin{mnemonicbox}
``SPI - SMTP sends, POP3/IMAP receives''

\end{mnemonicbox}
\subsection*{Question 4(c) [7 marks]}\label{q4c}

\textbf{Explain Role of DNS- Domain Name System Describe the process of
DNS resolution.}

\begin{solutionbox}

\textbf{DNS Role:}

\begin{itemize}
\tightlist
\item
  \textbf{Name Resolution}: Converts domain names to IP addresses
\item
  \textbf{Hierarchical System}: Distributed database structure
\item
  \textbf{Internet Navigation}: Makes web browsing user-friendly
\end{itemize}

\textbf{DNS Resolution Process:}

\includegraphics[width=1\linewidth,height=\textheight,keepaspectratio]{mermaid-124799bf.pdf}

\textbf{Resolution Steps:}

\begin{enumerate}
\tightlist
\item
  \textbf{Local Cache Check}: Check local DNS cache
\item
  \textbf{Recursive Query}: Contact local DNS server
\item
  \textbf{Root Server}: Get TLD server reference
\item
  \textbf{TLD Server}: Get authoritative server reference
\item
  \textbf{Authoritative Server}: Get final IP address
\item
  \textbf{Response Return}: IP address returned to client
\end{enumerate}

\textbf{DNS Record Types:}

\begin{itemize}
\tightlist
\item
  \textbf{A Record}: Maps name to IPv4 address
\item
  \textbf{AAAA Record}: Maps name to IPv6 address
\item
  \textbf{CNAME}: Canonical name alias
\item
  \textbf{MX}: Mail exchange server
\end{itemize}

\textbf{Benefits:}

\begin{itemize}
\tightlist
\item
  \textbf{User Friendly}: Remember names, not numbers
\item
  \textbf{Load Distribution}: Multiple IP addresses
\item
  \textbf{Service Location}: Find specific services
\end{itemize}

\end{solutionbox}
\begin{mnemonicbox}
``DNS = Directory Name Service''

\end{mnemonicbox}
\subsection*{Question 4(a OR) [3
marks]}\label{question-4a-or-3-marks}

\textbf{Explain WWW, HTML.}

\begin{solutionbox}

\textbf{WWW (World Wide Web):}

\begin{itemize}
\tightlist
\item
  \textbf{Definition}: Information system of interlinked documents
\item
  \textbf{Access}: Through web browsers using HTTP
\item
  \textbf{Components}: Web pages, links, URLs
\end{itemize}

\textbf{HTML (HyperText Markup Language):}

\begin{itemize}
\tightlist
\item
  \textbf{Purpose}: Standard markup language for web pages
\item
  \textbf{Structure}: Tags define document elements
\item
  \textbf{Hyperlinks}: Connect different web resources
\end{itemize}

\textbf{Relationship:}

\begin{itemize}
\tightlist
\item
  \textbf{WWW}: The system/platform
\item
  \textbf{HTML}: The content format
\item
  \textbf{Integration}: HTML creates WWW content
\end{itemize}

\end{solutionbox}
\begin{mnemonicbox}
``WWW uses HTML for content''

\end{mnemonicbox}
\subsection*{Question 4(b OR) [4
marks]}\label{question-4b-or-4-marks}

\textbf{Explain HTTP and FTP.}

\begin{solutionbox}

\textbf{Protocol Comparison:}

{\def\LTcaptype{none} % do not increment counter
\begin{longtable}[]{@{}lll@{}}
\toprule\noalign{}
Feature & \textbf{HTTP} & \textbf{FTP} \\
\midrule\noalign{}
\endhead
\bottomrule\noalign{}
\endlastfoot
\textbf{Purpose} & Web page transfer & File transfer \\
\textbf{Port} & 80 (HTTP), 443 (HTTPS) & 21 (control), 20 (data) \\
\textbf{Connection} & Stateless & Stateful \\
\textbf{Security} & HTTPS for security & FTPS for security \\
\end{longtable}
}

\textbf{HTTP (HyperText Transfer Protocol):}

\begin{itemize}
\tightlist
\item
  \textbf{Function}: Request-response protocol for web
\item
  \textbf{Methods}: GET, POST, PUT, DELETE
\item
  \textbf{Stateless}: Each request independent
\end{itemize}

\textbf{FTP (File Transfer Protocol):}

\begin{itemize}
\tightlist
\item
  \textbf{Function}: Upload/download files between systems
\item
  \textbf{Modes}: Active and Passive
\item
  \textbf{Authentication}: Username/password required
\end{itemize}

\textbf{Applications:}

\begin{itemize}
\tightlist
\item
  \textbf{HTTP}: Web browsing, API calls
\item
  \textbf{FTP}: File sharing, website maintenance
\end{itemize}

\end{solutionbox}
\begin{mnemonicbox}
``HF - HTTP for Hypertext, FTP for Files''

\end{mnemonicbox}
\subsection*{Question 4(c OR) [7
marks]}\label{question-4c-or-7-marks}

\textbf{Explain TCP and UDP protocol in transport layer in relation to
connection oriented and connection less network.}

\begin{solutionbox}

\textbf{Transport Layer Protocols:}

\includegraphics[width=1\linewidth,height=\textheight,keepaspectratio]{mermaid-fc6c7d11.pdf}

\textbf{Protocol Comparison:}

{\def\LTcaptype{none} % do not increment counter
\begin{longtable}[]{@{}lll@{}}
\toprule\noalign{}
Feature & \textbf{TCP} & \textbf{UDP} \\
\midrule\noalign{}
\endhead
\bottomrule\noalign{}
\endlastfoot
\textbf{Connection} & Connection-oriented & Connectionless \\
\textbf{Reliability} & Guaranteed delivery & Best effort \\
\textbf{Speed} & Slower (overhead) & Faster (minimal overhead) \\
\textbf{Header Size} & 20 bytes & 8 bytes \\
\textbf{Flow Control} & Yes & No \\
\textbf{Error Control} & Yes & Limited \\
\end{longtable}
}

\textbf{TCP (Transmission Control Protocol):}

\begin{itemize}
\tightlist
\item
  \textbf{Three-way Handshake}: SYN, SYN-ACK, ACK
\item
  \textbf{Reliable}: Acknowledgment and retransmission
\item
  \textbf{Flow Control}: Prevents buffer overflow
\item
  \textbf{Applications}: Web browsing, email, file transfer
\end{itemize}

\textbf{UDP (User Datagram Protocol):}

\begin{itemize}
\tightlist
\item
  \textbf{No Connection Setup}: Direct data transmission
\item
  \textbf{Lightweight}: Minimal protocol overhead
\item
  \textbf{No Guarantees}: Fire-and-forget approach
\item
  \textbf{Applications}: Video streaming, DNS, gaming
\end{itemize}

\textbf{Connection Models:}

\begin{itemize}
\tightlist
\item
  \textbf{Connection-Oriented}: Establish, transfer, terminate
\item
  \textbf{Connectionless}: Direct transmission without setup
\end{itemize}

\textbf{Selection Criteria:}

\begin{itemize}
\tightlist
\item
  \textbf{Use TCP}: When reliability is critical
\item
  \textbf{Use UDP}: When speed is more important
\end{itemize}

\end{solutionbox}
\begin{mnemonicbox}
``TCP = Thorough, UDP = Ultra-fast''

\end{mnemonicbox}
\subsection*{Question 5(a) [3 marks]}\label{q5a}

\textbf{Describe Hacking and its related precautions.}

\begin{solutionbox}

\textbf{Hacking Definition:}

\begin{itemize}
\tightlist
\item
  \textbf{Unauthorized Access}: Breaking into computer systems
\item
  \textbf{Malicious Intent}: Steal, modify, or destroy data
\item
  \textbf{Security Breach}: Exploit system vulnerabilities
\end{itemize}

\textbf{Types of Hacking:}

\begin{itemize}
\tightlist
\item
  \textbf{Ethical Hacking}: Authorized security testing
\item
  \textbf{Malicious Hacking}: Criminal activities
\item
  \textbf{Social Engineering}: Manipulate human behavior
\end{itemize}

\textbf{Precautions:}

{\def\LTcaptype{none} % do not increment counter
\begin{longtable}[]{@{}ll@{}}
\toprule\noalign{}
Security Measure & \textbf{Implementation} \\
\midrule\noalign{}
\endhead
\bottomrule\noalign{}
\endlastfoot
\textbf{Strong Passwords} & Complex, unique passwords \\
\textbf{Software Updates} & Regular patches and updates \\
\textbf{Firewalls} & Network access control \\
\textbf{Antivirus} & Malware detection and removal \\
\textbf{Backup} & Regular data backups \\
\textbf{User Training} & Security awareness programs \\
\end{longtable}
}

\end{solutionbox}
\begin{mnemonicbox}
``HSPFAB - Hacking Stopped by Passwords, Firewalls,
Antivirus, Backups''

\end{mnemonicbox}
\subsection*{Question 5(b) [4 marks]}\label{q5b}

\textbf{Explain IPSec architecture.}

\begin{solutionbox}

\textbf{IPSec (Internet Protocol Security):}

\includegraphics[width=1\linewidth,height=\textheight,keepaspectratio]{mermaid-0d496917.pdf}

\textbf{IPSec Components:}

{\def\LTcaptype{none} % do not increment counter
\begin{longtable}[]{@{}ll@{}}
\toprule\noalign{}
Component & \textbf{Function} \\
\midrule\noalign{}
\endhead
\bottomrule\noalign{}
\endlastfoot
\textbf{AH} & Authentication and integrity \\
\textbf{ESP} & Confidentiality and authentication \\
\textbf{SA} & Security parameter agreement \\
\textbf{IKE} & Key management protocol \\
\end{longtable}
}

\textbf{Operating Modes:}

\begin{itemize}
\tightlist
\item
  \textbf{Transport Mode}: Protects payload only
\item
  \textbf{Tunnel Mode}: Protects entire IP packet
\end{itemize}

\textbf{Security Services:}

\begin{itemize}
\tightlist
\item
  \textbf{Authentication}: Verify sender identity
\item
  \textbf{Integrity}: Ensure data unchanged
\item
  \textbf{Confidentiality}: Encrypt data content
\item
  \textbf{Anti-replay}: Prevent packet replay attacks
\end{itemize}

\end{solutionbox}
\begin{mnemonicbox}
``AISE - AH, IPSec, SA, ESP''

\end{mnemonicbox}
\subsection*{Question 5(c) [7 marks]}\label{q5c}

\textbf{Explain network Security topologies.}

\begin{solutionbox}

\textbf{Network Security Topologies:}

\includegraphics[width=1\linewidth,height=\textheight,keepaspectratio]{mermaid-7b499969.pdf}

\textbf{Security Zones:}

{\def\LTcaptype{none} % do not increment counter
\begin{longtable}[]{@{}lll@{}}
\toprule\noalign{}
Zone & \textbf{Purpose} & \textbf{Security Level} \\
\midrule\noalign{}
\endhead
\bottomrule\noalign{}
\endlastfoot
\textbf{Internet} & External untrusted network & Lowest \\
\textbf{DMZ} & Semi-trusted public services & Medium \\
\textbf{Internal} & Private trusted network & Highest \\
\end{longtable}
}

\textbf{Topology Components:}

\begin{itemize}
\tightlist
\item
  \textbf{Perimeter Security}: Firewalls, IDS/IPS
\item
  \textbf{Network Segmentation}: VLANs, subnets
\item
  \textbf{Access Control}: Authentication, authorization
\item
  \textbf{Monitoring}: Logging, SIEM systems
\end{itemize}

\textbf{Security Principles:}

\begin{itemize}
\tightlist
\item
  \textbf{Defense in Depth}: Multiple security layers
\item
  \textbf{Least Privilege}: Minimum required access
\item
  \textbf{Network Isolation}: Separate critical systems
\end{itemize}

\textbf{Implementation Strategies:}

\begin{itemize}
\tightlist
\item
  \textbf{Firewall Rules}: Control traffic flow
\item
  \textbf{VPN Access}: Secure remote connections
\item
  \textbf{Network Monitoring}: Detect threats
\item
  \textbf{Incident Response}: Handle security events
\end{itemize}

\textbf{Benefits:}

\begin{itemize}
\tightlist
\item
  \textbf{Risk Reduction}: Minimize attack surface
\item
  \textbf{Compliance}: Meet regulatory requirements
\item
  \textbf{Business Continuity}: Protect operations
\end{itemize}

\end{solutionbox}
\begin{mnemonicbox}
``NST = Network Security Through topology design''

\end{mnemonicbox}
\subsection*{Question 5(a OR) [3
marks]}\label{question-5a-or-3-marks}

\textbf{Explain ISO and how it contributes to information security?}

\begin{solutionbox}

\textbf{ISO (International Organization for Standardization):}

\begin{itemize}
\tightlist
\item
  \textbf{Global Standards}: Develops international standards
\item
  \textbf{Quality Assurance}: Ensures consistent practices
\item
  \textbf{Best Practices}: Provides framework for implementation
\end{itemize}

\textbf{ISO 27001 - Information Security:}

\begin{itemize}
\tightlist
\item
  \textbf{ISMS}: Information Security Management System
\item
  \textbf{Risk Management}: Systematic approach to security
\item
  \textbf{Continuous Improvement}: Regular review and updates
\end{itemize}

\textbf{Contributions to Information Security:}

\begin{itemize}
\tightlist
\item
  \textbf{Framework}: Structured approach to security
\item
  \textbf{Compliance}: Meet regulatory requirements
\item
  \textbf{Risk Assessment}: Identify and mitigate threats
\end{itemize}

\textbf{Benefits:}

\begin{itemize}
\tightlist
\item
  \textbf{Standardization}: Common security language
\item
  \textbf{Credibility}: International recognition
\item
  \textbf{Improvement}: Ongoing security enhancement
\end{itemize}

\end{solutionbox}
\begin{mnemonicbox}
``ISO = International Security Organization''

\end{mnemonicbox}
\subsection*{Question 5(b OR) [4
marks]}\label{question-5b-or-4-marks}

\textbf{Give Difference between symmetric and asymmetric encryption
algorithms.}

\begin{solutionbox}

\textbf{Encryption Algorithm Comparison:}

{\def\LTcaptype{none} % do not increment counter
\begin{longtable}[]{@{}lll@{}}
\toprule\noalign{}
Feature & \textbf{Symmetric} & \textbf{Asymmetric} \\
\midrule\noalign{}
\endhead
\bottomrule\noalign{}
\endlastfoot
\textbf{Keys} & Single shared key & Key pair (public/private) \\
\textbf{Speed} & Fast & Slower \\
\textbf{Key Distribution} & Difficult & Easier \\
\textbf{Scalability} & Poor (n^{2}-1 keys) & Better \\
\textbf{Security} & Depends on key secrecy & Mathematical complexity \\
\end{longtable}
}

\textbf{Symmetric Encryption:}

\begin{itemize}
\tightlist
\item
  \textbf{Examples}: AES, DES, 3DES
\item
  \textbf{Process}: Same key encrypts and decrypts
\item
  \textbf{Challenge}: Secure key distribution
\end{itemize}

\textbf{Asymmetric Encryption:}

\begin{itemize}
\tightlist
\item
  \textbf{Examples}: RSA, ECC, Diffie-Hellman
\item
  \textbf{Process}: Public key encrypts, private key decrypts
\item
  \textbf{Advantage}: No key distribution problem
\end{itemize}

\textbf{Hybrid Approach:}

\begin{itemize}
\tightlist
\item
  \textbf{Combination}: Use both types together
\item
  \textbf{Method}: Asymmetric for key exchange, symmetric for data
\end{itemize}

\textbf{Applications:}

\begin{itemize}
\tightlist
\item
  \textbf{Symmetric}: Bulk data encryption
\item
  \textbf{Asymmetric}: Digital signatures, key exchange
\end{itemize}

\end{solutionbox}
\begin{mnemonicbox}
``SA = Symmetric Shared, Asymmetric Apart''

\end{mnemonicbox}
\subsection*{Question 5(c OR) [7
marks]}\label{question-5c-or-7-marks}

\textbf{Explain Email security with its standards.}

\begin{solutionbox}

\textbf{Email Security Challenges:}

\includegraphics[width=1\linewidth,height=\textheight,keepaspectratio]{mermaid-589a8c68.pdf}

\textbf{Email Security Standards:}

{\def\LTcaptype{none} % do not increment counter
\begin{longtable}[]{@{}
  >{\raggedright\arraybackslash}p{(\linewidth - 4\tabcolsep) * \real{0.2703}}
  >{\raggedright\arraybackslash}p{(\linewidth - 4\tabcolsep) * \real{0.3514}}
  >{\raggedright\arraybackslash}p{(\linewidth - 4\tabcolsep) * \real{0.3784}}@{}}
\toprule\noalign{}
\begin{minipage}[b]{\linewidth}\raggedright
Standard
\end{minipage} & \begin{minipage}[b]{\linewidth}\raggedright
\textbf{Purpose}
\end{minipage} & \begin{minipage}[b]{\linewidth}\raggedright
\textbf{Function}
\end{minipage} \\
\midrule\noalign{}
\endhead
\bottomrule\noalign{}
\endlastfoot
\textbf{S/MIME} & Secure email content & Encryption and digital
signatures \\
\textbf{PGP} & Pretty Good Privacy & End-to-end encryption \\
\textbf{TLS} & Transport security & Secure email transmission \\
\textbf{SPF} & Sender authentication & Prevent email spoofing \\
\textbf{DKIM} & Message integrity & Digital signature verification \\
\textbf{DMARC} & Policy enforcement & Email authentication policy \\
\end{longtable}
}

\textbf{Security Mechanisms:}

\begin{itemize}
\tightlist
\item
  \textbf{Encryption}: Protect message content
\item
  \textbf{Digital Signatures}: Verify sender identity
\item
  \textbf{Authentication}: Confirm message origin
\item
  \textbf{Integrity}: Ensure message unchanged
\end{itemize}

\textbf{Implementation Layers:}

\begin{itemize}
\tightlist
\item
  \textbf{Transport Layer}: TLS/SSL encryption
\item
  \textbf{Message Layer}: S/MIME, PGP encryption
\item
  \textbf{Policy Layer}: SPF, DKIM, DMARC
\end{itemize}

\textbf{Best Practices:}

\begin{itemize}
\tightlist
\item
  \textbf{User Education}: Recognize phishing attempts
\item
  \textbf{Gateway Filtering}: Block malicious emails
\item
  \textbf{Regular Updates}: Keep security software current
\item
  \textbf{Backup Systems}: Protect against data loss
\end{itemize}

\textbf{Benefits:}

\begin{itemize}
\tightlist
\item
  \textbf{Confidentiality}: Private communications
\item
  \textbf{Authentication}: Verified senders
\item
  \textbf{Compliance}: Meet regulatory requirements
\item
  \textbf{Trust}: Secure business communications
\end{itemize}

\end{solutionbox}
\begin{mnemonicbox}
``SPTSD = S/MIME, PGP, TLS, SPF, DKIM protect email''

\end{mnemonicbox}

\end{document}
