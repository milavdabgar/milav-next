\documentclass[10pt,a4paper]{article}

% content/resources/templates/preamble.tex
\usepackage[margin=0.6in]{geometry}
\author{Milav Dabgar}
\usepackage{amsmath,amssymb,amsthm}
\usepackage{booktabs}
\usepackage{multirow}
\usepackage{xcolor}
\usepackage{tcolorbox}
\tcbuselibrary{breakable,skins}
\usepackage[colorlinks=true,linkcolor=blue]{hyperref}
\usepackage{titlesec}
\usepackage{enumitem}
\usepackage{tikz}
\usepackage{pgfplots}
\usepackage{circuitikz}
\usepackage[version=4]{mhchem}
\usepackage{longtable}
\usepackage{array}
\usepackage{float}
\usepackage{caption}
\usepackage{listings}

\lstset{
  basicstyle=\small\ttfamily,
  breaklines=true,
  breakatwhitespace=false,
  postbreak=\mbox{\textcolor{red}{$\hookrightarrow$}\space},
  float=false,
  numbers=left,
  numberstyle=\tiny\color{gray},
  numbersep=10pt,
  xleftmargin=2em,
  keywordstyle=\color{blue},
  commentstyle=\color{green!60!black},
  stringstyle=\color{purple},
  backgroundcolor=\color{gray!5},
  showstringspaces=false,
  tabsize=2,
  captionpos=b,
  keepspaces=true,
  columns=flexible
}

\pgfplotsset{compat=1.18}
\usetikzlibrary{shapes,arrows,positioning,calc,patterns,decorations.pathmorphing,decorations.markings,arrows.meta}

% Color scheme
\definecolor{headcolor}{RGB}{0,102,204}
\definecolor{keycolor}{RGB}{220,20,60}
\definecolor{solutioncolor}{RGB}{34,139,34}
\definecolor{mnemoniccolor}{RGB}{148,0,211}
\definecolor{codecolor}{RGB}{0,0,100}

% Spacing
\setlength{\parskip}{3pt}
\setlist[itemize]{nosep}
\setlist[enumerate]{nosep}

% Title formatting
\titleformat{\section}{\Large\bfseries\color{headcolor}}{\thesection}{1em}{}
\titleformat{\subsection}{\large\bfseries\color{headcolor}}{\thesubsection}{1em}{}

% Pandoc tightlist compatibility
\providecommand{\tightlist}{%
  \setlength{\itemsep}{0pt}\setlength{\parskip}{0pt}}

% Pandoc longtable compatibility
\newcounter{none}
\def\thenone{}


% content/resources/templates/english-boxes.tex
% This file is currently empty - it exists to maintain consistency with the import structure.
% Add custom environments here if needed in the future.


\begin{document}

\begin{center}
{\Huge\bfseries\color{headcolor} Subject Name Solutions}\\[5pt]
{\LARGE 4343202 -- Summer 2024}\\[3pt]
{\large Semester 1 Study Material}\\[3pt]
{\normalsize\textit{Detailed Solutions and Explanations}}
\end{center}

\vspace{10pt}

\subsection*{Question 1(a) [3 marks]}\label{q1a}

\textbf{Explain packet switching network.}

\begin{solutionbox}
Packet switching is a network communication method
where data is divided into small packets before transmission.

\textbf{Diagram:}

\includegraphics[width=1\linewidth,height=\textheight,keepaspectratio]{mermaid-888417c1.pdf}

\begin{itemize}
\tightlist
\item
  \textbf{Independent routing}: Each packet travels independently
  through network
\item
  \textbf{Flexible paths}: Packets can take different routes to reach
  destination
\item
  \textbf{Efficiency}: Better utilization of network bandwidth
\end{itemize}

\end{solutionbox}
\begin{mnemonicbox}
``DIVE'' - Data Into Various Elements

\end{mnemonicbox}
\subsection*{Question 1(b) [4 marks]}\label{q1b}

\textbf{Write functional description of any four layers of OSI reference
model.}

\begin{solutionbox}
The OSI model divides network communication into seven
distinct layers, each with specific functions.

{\def\LTcaptype{none} % do not increment counter
\begin{longtable}[]{@{}
  >{\raggedright\arraybackslash}p{(\linewidth - 4\tabcolsep) * \real{0.2258}}
  >{\raggedright\arraybackslash}p{(\linewidth - 4\tabcolsep) * \real{0.3226}}
  >{\raggedright\arraybackslash}p{(\linewidth - 4\tabcolsep) * \real{0.4516}}@{}}
\toprule\noalign{}
\begin{minipage}[b]{\linewidth}\raggedright
Layer
\end{minipage} & \begin{minipage}[b]{\linewidth}\raggedright
Function
\end{minipage} & \begin{minipage}[b]{\linewidth}\raggedright
Key Protocols
\end{minipage} \\
\midrule\noalign{}
\endhead
\bottomrule\noalign{}
\endlastfoot
Application & Provides network services directly to user applications &
HTTP, FTP, SMTP \\
Presentation & Translates, encrypts, and compresses data & SSL, TLS,
JPEG \\
Session & Establishes, manages, and terminates connections & NetBIOS,
RPC \\
Transport & Ensures reliable end-to-end data transfer & TCP, UDP \\
\end{longtable}
}

\begin{itemize}
\tightlist
\item
  \textbf{Application layer}: Interface between network and applications
\item
  \textbf{Presentation layer}: Data formatting and encryption
\item
  \textbf{Session layer}: Dialog control and synchronization
\item
  \textbf{Transport layer}: End-to-end connection and reliability
\end{itemize}

\end{solutionbox}
\begin{mnemonicbox}
``All People Seem To Need Data Processing''

\end{mnemonicbox}
\subsection*{Question 1(c) [7 marks]}\label{q1c}

\textbf{Explain Network topologies and with diagram.}

\begin{solutionbox}
Network topology refers to the physical or logical
arrangement of devices in a network.

{\def\LTcaptype{none} % do not increment counter
\begin{longtable}[]{@{}
  >{\raggedright\arraybackslash}p{(\linewidth - 4\tabcolsep) * \real{0.2703}}
  >{\raggedright\arraybackslash}p{(\linewidth - 4\tabcolsep) * \real{0.3243}}
  >{\raggedright\arraybackslash}p{(\linewidth - 4\tabcolsep) * \real{0.4054}}@{}}
\toprule\noalign{}
\begin{minipage}[b]{\linewidth}\raggedright
Topology
\end{minipage} & \begin{minipage}[b]{\linewidth}\raggedright
Advantages
\end{minipage} & \begin{minipage}[b]{\linewidth}\raggedright
Disadvantages
\end{minipage} \\
\midrule\noalign{}
\endhead
\bottomrule\noalign{}
\endlastfoot
Bus & Simple, inexpensive & Single point of failure \\
Star & Easy troubleshooting, centralized & Hub/switch failure affects
all \\
Ring & Equal access for all nodes & Single cable failure affects
network \\
Mesh & High reliability, no traffic problems & Expensive, complex \\
Tree & Easily expandable, structured & Dependent on root, complex \\
\end{longtable}
}

\textbf{Diagram:}

\begin{lstlisting}
              BUS TOPOLOGY              
+-----+    +-----+    +-----+    +-----+
|Node1|====|Node2|====|Node3|====|Node4|
+-----+    +-----+    +-----+    +-----+
                                        
              STAR TOPOLOGY             
                 +-----+                
                 |Hub/ |                
                 |Switch|                
                 +-----+                
                     |                  
           +---------+---------+         
           |         |         |         
        +-----+   +-----+   +-----+      
        |Node1|   |Node2|   |Node3|      
        +-----+   +-----+   +-----+     
\end{lstlisting}

\begin{itemize}
\tightlist
\item
  \textbf{Bus topology}: All devices connected to single cable
\item
  \textbf{Star topology}: All devices connected to central hub/switch
\item
  \textbf{Ring topology}: Devices connected in closed loop
\item
  \textbf{Mesh topology}: Each device connected to every other device
\item
  \textbf{Tree topology}: Hierarchical star networks connected via bus
\end{itemize}

\end{solutionbox}
\begin{mnemonicbox}
``BSRMT'' - ``Better Solutions Require Multiple
Topologies''

\end{mnemonicbox}
\subsection*{Question 1(c) OR [7
marks]}\label{q1c}

\textbf{Draw the diagram of TCP/IP protocol suite and explain the
functions of Application Layer, Transport Layer and Network Layer in
detail.}

\begin{solutionbox}
The TCP/IP protocol suite organizes network
communication into four functional layers.

\textbf{Diagram:}

\begin{lstlisting}
+-------------------------------+
|       APPLICATION LAYER       |
| (HTTP, FTP, SMTP, DNS, TELNET)|
+-------------------------------+
|        TRANSPORT LAYER        |
|           (TCP, UDP)          |
+-------------------------------+
|        INTERNET LAYER         |
|      (IP, ICMP, ARP, RARP)    |
+-------------------------------+
|     NETWORK ACCESS LAYER      |
| (Ethernet, Wi-Fi, Token Ring) |
+-------------------------------+
\end{lstlisting}

{\def\LTcaptype{none} % do not increment counter
\begin{longtable}[]{@{}
  >{\raggedright\arraybackslash}p{(\linewidth - 4\tabcolsep) * \real{0.2000}}
  >{\raggedright\arraybackslash}p{(\linewidth - 4\tabcolsep) * \real{0.4000}}
  >{\raggedright\arraybackslash}p{(\linewidth - 4\tabcolsep) * \real{0.4000}}@{}}
\toprule\noalign{}
\begin{minipage}[b]{\linewidth}\raggedright
Layer
\end{minipage} & \begin{minipage}[b]{\linewidth}\raggedright
Main Function
\end{minipage} & \begin{minipage}[b]{\linewidth}\raggedright
Key Protocols
\end{minipage} \\
\midrule\noalign{}
\endhead
\bottomrule\noalign{}
\endlastfoot
Application & Provides network services to applications & HTTP, FTP,
SMTP \\
Transport & End-to-end communication, data flow control & TCP, UDP \\
Internet (Network) & Logical addressing and routing & IP, ICMP, ARP \\
\end{longtable}
}

\begin{itemize}
\tightlist
\item
  \textbf{Application Layer}: User interface to network,
  application-specific protocols
\item
  \textbf{Transport Layer}: Reliable data transmission, error recovery,
  flow control
\item
  \textbf{Network Layer}: Routing packets between networks, IP
  addressing
\end{itemize}

\end{solutionbox}
\begin{mnemonicbox}
``ATN works'' - Application, Transport, Network works
together

\end{mnemonicbox}
\subsection*{Question 2(a) [3 marks]}\label{q2a}

\textbf{Compare connection-oriented protocol and connection less
protocol.}

\begin{solutionbox}
Connection-oriented and connectionless protocols differ
in how they handle data transmission.

{\def\LTcaptype{none} % do not increment counter
\begin{longtable}[]{@{}lll@{}}
\toprule\noalign{}
Feature & Connection-oriented & Connectionless \\
\midrule\noalign{}
\endhead
\bottomrule\noalign{}
\endlastfoot
Connection & Establishes before transmission & No connection setup \\
Reliability & Guaranteed delivery & No delivery guarantee \\
Error checking & Extensive & Limited or none \\
Example & TCP & UDP \\
Usage & File transfer, web browsing & Streaming, DNS lookups \\
\end{longtable}
}

\end{solutionbox}
\begin{mnemonicbox}
``REACH'' - Reliability Exists in All Connection
Handshakes

\end{mnemonicbox}
\subsection*{Question 2(b) [4 marks]}\label{q2b}

\textbf{Explain Fast Ethernet \& Gigabit Ethernet.}

\begin{solutionbox}
Fast Ethernet and Gigabit Ethernet are higher-speed
versions of the original Ethernet standard.

{\def\LTcaptype{none} % do not increment counter
\begin{longtable}[]{@{}lll@{}}
\toprule\noalign{}
Feature & Fast Ethernet & Gigabit Ethernet \\
\midrule\noalign{}
\endhead
\bottomrule\noalign{}
\endlastfoot
Speed & 100 Mbps & 1000 Mbps (1 Gbps) \\
IEEE Standard & 802.3u & 802.3z/802.3ab \\
Cable Type & Cat5 UTP & Cat5e/Cat6 UTP, Fiber \\
Max Distance & 100m (copper) & 100m (copper), 5km (fiber) \\
\end{longtable}
}

\begin{itemize}
\tightlist
\item
  \textbf{Fast Ethernet}: 10x faster than original 10Base-T Ethernet
\item
  \textbf{Gigabit Ethernet}: 10x faster than Fast Ethernet, backward
  compatible
\item
  \textbf{Cabling}: Uses higher quality cabling to achieve greater
  speeds
\item
  \textbf{Applications}: High-bandwidth network backbones, server
  connections
\end{itemize}

\end{solutionbox}
\begin{mnemonicbox}
``Fast Gets Going'' - 100 to 1000 Mbps progression

\end{mnemonicbox}
\subsection*{Question 2(c) [7 marks]}\label{q2c}

\textbf{Differentiate between Router, Hub and Switch.}

\begin{solutionbox}
Routers, hubs, and switches are network devices with
different capabilities and functions.

{\def\LTcaptype{none} % do not increment counter
\begin{longtable}[]{@{}
  >{\raggedright\arraybackslash}p{(\linewidth - 6\tabcolsep) * \real{0.3000}}
  >{\raggedright\arraybackslash}p{(\linewidth - 6\tabcolsep) * \real{0.2667}}
  >{\raggedright\arraybackslash}p{(\linewidth - 6\tabcolsep) * \real{0.1667}}
  >{\raggedright\arraybackslash}p{(\linewidth - 6\tabcolsep) * \real{0.2667}}@{}}
\toprule\noalign{}
\begin{minipage}[b]{\linewidth}\raggedright
Feature
\end{minipage} & \begin{minipage}[b]{\linewidth}\raggedright
Router
\end{minipage} & \begin{minipage}[b]{\linewidth}\raggedright
Hub
\end{minipage} & \begin{minipage}[b]{\linewidth}\raggedright
Switch
\end{minipage} \\
\midrule\noalign{}
\endhead
\bottomrule\noalign{}
\endlastfoot
OSI Layer & Network (3) & Physical (1) & Data Link (2) \\
Function & Connects networks & Connects devices & Connects devices \\
Data handling & Intelligent routing & Broadcasts to all & Sends to
specific device \\
Security & Provides firewall & No security & Basic filtering \\
Addressing & Uses IP addresses & No addressing & Uses MAC addresses \\
Efficiency & High & Low & High \\
Intelligence & Smart & Dumb & Moderately smart \\
\end{longtable}
}

\textbf{Diagram:}

\begin{lstlisting}
    ROUTER                HUB                  SWITCH
  +--------+           +------+              +--------+
  |        |           |      |              |        |
  | Routes |           |Shares|              |Forwards|
  |between |           |signal|              | to MAC |
  |networks|           |to all|              |address |
  |        |           |ports |              |        |
  +--------+           +------+              +--------+
\end{lstlisting}

\end{solutionbox}
\begin{mnemonicbox}
``RHS order'' - ``Router Has Smarts, Hub Shares
Signal, Switch Sends Specifically''

\end{mnemonicbox}
\subsection*{Question 2(a) OR [3
marks]}\label{q2a}

\textbf{Define E-mail system and list application of E-Mail.}

\begin{solutionbox}
An email system is a network service that allows
exchange of digital messages between users.

{\def\LTcaptype{none} % do not increment counter
\begin{longtable}[]{@{}ll@{}}
\toprule\noalign{}
Component & Function \\
\midrule\noalign{}
\endhead
\bottomrule\noalign{}
\endlastfoot
Mail User Agent (MUA) & Email client software used by end-users \\
Mail Transfer Agent (MTA) & Server software that transfers emails \\
Mail Delivery Agent (MDA) & Delivers email to recipient's mailbox \\
Protocols & SMTP, POP3, IMAP \\
\end{longtable}
}

\textbf{Applications of Email:}

\begin{itemize}
\tightlist
\item
  Business communication
\item
  Personal messaging
\item
  File sharing
\item
  Marketing and newsletters
\item
  Notifications and alerts
\end{itemize}

\end{solutionbox}
\begin{mnemonicbox}
``BCPFN'' - ``Business Communication, Personal,
Files, Newsletters''

\end{mnemonicbox}
\subsection*{Question 2(b) OR [4
marks]}\label{q2b}

\textbf{Differentiate between IPv4 and IPv6.}

\begin{solutionbox}
IPv4 and IPv6 are Internet Protocol versions with
significant differences.

{\def\LTcaptype{none} % do not increment counter
\begin{longtable}[]{@{}
  >{\raggedright\arraybackslash}p{(\linewidth - 4\tabcolsep) * \real{0.4286}}
  >{\raggedright\arraybackslash}p{(\linewidth - 4\tabcolsep) * \real{0.2857}}
  >{\raggedright\arraybackslash}p{(\linewidth - 4\tabcolsep) * \real{0.2857}}@{}}
\toprule\noalign{}
\begin{minipage}[b]{\linewidth}\raggedright
Feature
\end{minipage} & \begin{minipage}[b]{\linewidth}\raggedright
IPv4
\end{minipage} & \begin{minipage}[b]{\linewidth}\raggedright
IPv6
\end{minipage} \\
\midrule\noalign{}
\endhead
\bottomrule\noalign{}
\endlastfoot
Address length & 32-bit (4 bytes) & 128-bit (16 bytes) \\
Format & Dotted decimal (192.168.1.1) & Hexadecimal with colons
(2001:0db8:85a3:0000:0000:8a2e:0370:7334) \\
Address space & \textasciitilde4.3 billion addresses & 340 undecillion
addresses \\
Security & Security added later & Built-in IPSec \\
Configuration & Manual or DHCP & Stateless auto-configuration \\
Header & Complex, variable & Simplified, fixed \\
\end{longtable}
}

\begin{itemize}
\tightlist
\item
  \textbf{IPv4}: Traditional addressing with limited space
\item
  \textbf{IPv6}: Next-generation addressing with massive capacity
\item
  \textbf{Transition}: Dual-stack, tunneling and translation mechanisms
\end{itemize}

\end{solutionbox}
\begin{mnemonicbox}
``4 SMALL, 6 HUGE'' - IPv4 Small address space, IPv6
Huge address space

\end{mnemonicbox}
\subsection*{Question 2(c) OR [7
marks]}\label{q2c}

\textbf{Discuss on Firewall with concept, principles, limitations,
trusted system, Kerberos- concept in network security.}

\begin{solutionbox}
Firewalls are critical network security systems that
monitor and control incoming and outgoing traffic.

{\def\LTcaptype{none} % do not increment counter
\begin{longtable}[]{@{}
  >{\raggedright\arraybackslash}p{(\linewidth - 4\tabcolsep) * \real{0.4242}}
  >{\raggedright\arraybackslash}p{(\linewidth - 4\tabcolsep) * \real{0.3030}}
  >{\raggedright\arraybackslash}p{(\linewidth - 4\tabcolsep) * \real{0.2727}}@{}}
\toprule\noalign{}
\begin{minipage}[b]{\linewidth}\raggedright
Firewall Type
\end{minipage} & \begin{minipage}[b]{\linewidth}\raggedright
Function
\end{minipage} & \begin{minipage}[b]{\linewidth}\raggedright
Example
\end{minipage} \\
\midrule\noalign{}
\endhead
\bottomrule\noalign{}
\endlastfoot
Packet filtering & Examines packet headers & Router ACLs \\
Stateful inspection & Tracks connection state & Most hardware
firewalls \\
Application layer & Inspects data contents & Web application
firewalls \\
Next-generation & Combines multiple techniques & Palo Alto, Fortinet \\
\end{longtable}
}

\textbf{Principles of Firewall:}

\begin{itemize}
\tightlist
\item
  \textbf{Default deny}: Block everything unless explicitly allowed
\item
  \textbf{Defense in depth}: Multiple security layers
\item
  \textbf{Least privilege}: Minimal necessary access
\end{itemize}

\textbf{Limitations:}

\begin{itemize}
\tightlist
\item
  Cannot protect against authorized users
\item
  Limited against encrypted malicious traffic
\item
  Performance impact on network
\end{itemize}

\textbf{Trusted Systems:}

\begin{itemize}
\tightlist
\item
  Systems meeting specific security requirements
\item
  Formal security policy enforcement
\item
  Access control and authentication mechanisms
\end{itemize}

\textbf{Kerberos Concept:}

\begin{lstlisting}
    +----------+       +----------+       +----------+
    |  Client  |<----->|   KDC    |<----->|  Server  |
    +----------+       +----------+       +----------+
         |                  |                  |
         |<--Ticket-granting ticket--|         |
         |-------Service request ticket------->|
         |<----------Session key-------------->|
\end{lstlisting}

\begin{itemize}
\tightlist
\item
  \textbf{Authentication protocol} using trusted third party
\item
  \textbf{Ticket-based} access control system
\item
  \textbf{Mutual authentication} between client and server
\item
  \textbf{Time-sensitive} tickets prevent replay attacks
\end{itemize}

\end{solutionbox}
\begin{mnemonicbox}
``FLASK'' - ``Firewalls Lock Access, Secure with
Kerberos''

\end{mnemonicbox}
\subsection*{Question 3(a) [3 marks]}\label{q3a}

\textbf{Describe Sub-layers of Data link Layers.}

\begin{solutionbox}
The Data Link Layer in the OSI model is divided into
two sublayers with distinct functions.

{\def\LTcaptype{none} % do not increment counter
\begin{longtable}[]{@{}
  >{\raggedright\arraybackslash}p{(\linewidth - 4\tabcolsep) * \real{0.3226}}
  >{\raggedright\arraybackslash}p{(\linewidth - 4\tabcolsep) * \real{0.3226}}
  >{\raggedright\arraybackslash}p{(\linewidth - 4\tabcolsep) * \real{0.3548}}@{}}
\toprule\noalign{}
\begin{minipage}[b]{\linewidth}\raggedright
Sublayer
\end{minipage} & \begin{minipage}[b]{\linewidth}\raggedright
Function
\end{minipage} & \begin{minipage}[b]{\linewidth}\raggedright
Standards
\end{minipage} \\
\midrule\noalign{}
\endhead
\bottomrule\noalign{}
\endlastfoot
Logical Link Control (LLC) & Flow control, error checking & IEEE
802.2 \\
Media Access Control (MAC) & Channel access, addressing & IEEE 802.3,
802.11 \\
\end{longtable}
}

\textbf{Diagram:}

\begin{lstlisting}
+-----------------------------+
|        NETWORK LAYER        |
+-----------------------------+
|     LOGICAL LINK CONTROL    |  <-- Flow control, Error handling
|        (LLC - 802.2)        |      Multiplexing, Connection mgmt
+-----------------------------+
|     MEDIA ACCESS CONTROL    |  <-- MAC addressing, Channel access 
|   (MAC - 802.3, 802.11)     |      Frame delimiting, Error detection
+-----------------------------+
|       PHYSICAL LAYER        |
+-----------------------------+
\end{lstlisting}

\begin{itemize}
\tightlist
\item
  \textbf{LLC}: Provides interface to network layer, error/flow control
\item
  \textbf{MAC}: Handles physical addressing and media access
\end{itemize}

\end{solutionbox}
\begin{mnemonicbox}
``MAC LLCs order'' - ``MAC handles Lower Layer, LLC
coordinates higher''

\end{mnemonicbox}
\subsection*{Question 3(b) [4 marks]}\label{q3b}

\textbf{Explain IP layer protocols in detail.}

\begin{solutionbox}
The IP layer contains several key protocols that work
together to facilitate internetwork communication.

{\def\LTcaptype{none} % do not increment counter
\begin{longtable}[]{@{}lll@{}}
\toprule\noalign{}
Protocol & Function & Key Features \\
\midrule\noalign{}
\endhead
\bottomrule\noalign{}
\endlastfoot
IP & Basic datagram delivery & Addressing, fragmentation, TTL \\
ICMP & Network diagnostics & Error reporting, ping, traceroute \\
ARP & Address resolution & Maps IP to MAC addresses \\
RARP & Reverse address resolution & Maps MAC to IP addresses \\
IGMP & Multicast group management & Manages host groups \\
\end{longtable}
}

\begin{itemize}
\tightlist
\item
  \textbf{IP}: Core protocol for addressing and routing packets
\item
  \textbf{ICMP}: Error messages and operational information
\item
  \textbf{ARP/RARP}: Address translation between layers
\item
  \textbf{IGMP}: Manages multicast group memberships
\end{itemize}

\end{solutionbox}
\begin{mnemonicbox}
``I PAIR-up'' - IP, ICMP, ARP, RARP work as a team

\end{mnemonicbox}
\subsection*{Question 3(c) [7 marks]}\label{q3c}

\textbf{Describe different types of IP addressing schemes and explain
various classes in classful IP addressing with example.}

\begin{solutionbox}
IP addressing schemes define how IP addresses are
allocated and structured.

{\def\LTcaptype{none} % do not increment counter
\begin{longtable}[]{@{}
  >{\raggedright\arraybackslash}p{(\linewidth - 4\tabcolsep) * \real{0.4359}}
  >{\raggedright\arraybackslash}p{(\linewidth - 4\tabcolsep) * \real{0.3333}}
  >{\raggedright\arraybackslash}p{(\linewidth - 4\tabcolsep) * \real{0.2308}}@{}}
\toprule\noalign{}
\begin{minipage}[b]{\linewidth}\raggedright
IP Addressing Scheme
\end{minipage} & \begin{minipage}[b]{\linewidth}\raggedright
Description
\end{minipage} & \begin{minipage}[b]{\linewidth}\raggedright
Example
\end{minipage} \\
\midrule\noalign{}
\endhead
\bottomrule\noalign{}
\endlastfoot
Classful & Traditional division into 5 classes & Class A: 10.0.0.0 \\
Classless (CIDR) & Flexible prefixes, more efficient & 192.168.1.0/24 \\
Private & Non-routable addresses for internal use & 192.168.0.0/16 \\
Special Purpose & Reserved for specific functions & 127.0.0.1
(localhost) \\
\end{longtable}
}

\textbf{Classful IP Addressing:}

{\def\LTcaptype{none} % do not increment counter
\begin{longtable}[]{@{}
  >{\raggedright\arraybackslash}p{(\linewidth - 12\tabcolsep) * \real{0.0769}}
  >{\raggedright\arraybackslash}p{(\linewidth - 12\tabcolsep) * \real{0.1209}}
  >{\raggedright\arraybackslash}p{(\linewidth - 12\tabcolsep) * \real{0.1978}}
  >{\raggedright\arraybackslash}p{(\linewidth - 12\tabcolsep) * \real{0.2308}}
  >{\raggedright\arraybackslash}p{(\linewidth - 12\tabcolsep) * \real{0.0989}}
  >{\raggedright\arraybackslash}p{(\linewidth - 12\tabcolsep) * \real{0.1099}}
  >{\raggedright\arraybackslash}p{(\linewidth - 12\tabcolsep) * \real{0.1648}}@{}}
\toprule\noalign{}
\begin{minipage}[b]{\linewidth}\raggedright
Class
\end{minipage} & \begin{minipage}[b]{\linewidth}\raggedright
First Bits
\end{minipage} & \begin{minipage}[b]{\linewidth}\raggedright
First Byte Range
\end{minipage} & \begin{minipage}[b]{\linewidth}\raggedright
Default Subnet Mask
\end{minipage} & \begin{minipage}[b]{\linewidth}\raggedright
Example
\end{minipage} & \begin{minipage}[b]{\linewidth}\raggedright
Networks
\end{minipage} & \begin{minipage}[b]{\linewidth}\raggedright
Hosts/Network
\end{minipage} \\
\midrule\noalign{}
\endhead
\bottomrule\noalign{}
\endlastfoot
A & 0 & 1-127 & 255.0.0.0 (/8) & 10.52.36.12 & 126 & 16,777,214 \\
B & 10 & 128-191 & 255.255.0.0 (/16) & 172.16.52.63 & 16,384 & 65,534 \\
C & 110 & 192-223 & 255.255.255.0 (/24) & 192.168.10.15 & 2,097,152 &
254 \\
D & 1110 & 224-239 & N/A (Multicast) & 224.0.0.5 & N/A & N/A \\
E & 1111 & 240-255 & N/A (Experimental) & 240.0.0.1 & N/A & N/A \\
\end{longtable}
}

\begin{itemize}
\tightlist
\item
  \textbf{Class A}: Large organizations, huge number of hosts
\item
  \textbf{Class B}: Medium-sized organizations
\item
  \textbf{Class C}: Small networks with few hosts
\item
  \textbf{Class D}: Multicast groups
\item
  \textbf{Class E}: Reserved for experimental use
\end{itemize}

\end{solutionbox}
\begin{mnemonicbox}
``All Businesses Care During Exams'' - Classes A, B,
C, D, E

\end{mnemonicbox}
\subsection*{Question 3(a) OR [3
marks]}\label{q3a}

\textbf{Describe Digital Subscriber Line technology.}

\begin{solutionbox}
Digital Subscriber Line (DSL) is a technology that
provides digital data transmission over telephone lines.

{\def\LTcaptype{none} % do not increment counter
\begin{longtable}[]{@{}llll@{}}
\toprule\noalign{}
DSL Type & Speed (Down/Up) & Distance & Application \\
\midrule\noalign{}
\endhead
\bottomrule\noalign{}
\endlastfoot
ADSL & 8 Mbps/1 Mbps & Up to 5.5 km & Home internet \\
SDSL & 2 Mbps/2 Mbps & Up to 3 km & Business \\
VDSL & 52 Mbps/16 Mbps & Up to 1.2 km & Video streaming \\
HDSL & 2 Mbps/2 Mbps & Up to 3.6 km & T1/E1 replacement \\
\end{longtable}
}

\textbf{Diagram:}

\begin{lstlisting}
                           +-------+
        +--------+         |       |
HOME----|  DSL   |---------| DSLAM |-------INTERNET
        | MODEM  |  Copper |       |
        +--------+   Line  +-------+
                    (POTS)    ISP
\end{lstlisting}

\begin{itemize}
\tightlist
\item
  \textbf{Spectrum usage}: Uses higher frequencies than voice
\item
  \textbf{Always-on}: Continuous connection, no dial-up
\item
  \textbf{xDSL}: Family of technologies with different capabilities
\end{itemize}

\end{solutionbox}
\begin{mnemonicbox}
``SAVE Bandwidth'' - SDSL, ADSL, VDSL, HDSL Bandwidth
options

\end{mnemonicbox}
\subsection*{Question 3(b) OR [4
marks]}\label{q3b}

\textbf{Discuss Cable Modem System.}

\begin{solutionbox}
Cable modem system provides internet access through the
same coaxial cable used for cable TV.

{\def\LTcaptype{none} % do not increment counter
\begin{longtable}[]{@{}ll@{}}
\toprule\noalign{}
Component & Function \\
\midrule\noalign{}
\endhead
\bottomrule\noalign{}
\endlastfoot
Cable modem & User-end device converting digital signals \\
CMTS & Cable Modem Termination System at provider end \\
HFC & Hybrid Fiber-Coaxial network infrastructure \\
DOCSIS & Data Over Cable Service Interface Specification \\
\end{longtable}
}

\textbf{Diagram:}

\begin{lstlisting}
                     FIBER
+--------+        +--------+        +---------+
|  HOME  |  COAX  |  NODE  |        |   ISP   |
| MODEM  |--------|        |--------|  CMTS   |-----INTERNET
+--------+        +--------+        +---------+
                 NEIGHBORHOOD        HEAD-END
\end{lstlisting}

\begin{itemize}
\tightlist
\item
  \textbf{Shared medium}: Neighborhood shares bandwidth
\item
  \textbf{Asymmetric}: Typically faster download than upload
\item
  \textbf{DOCSIS standards}: Evolving specifications for speed/features
\end{itemize}

\end{solutionbox}
\begin{mnemonicbox}
``CHAMPS'' - ``Cable, HFC, Access, Modem, Provider,
Shared''

\end{mnemonicbox}
\subsection*{Question 3(c) OR [7
marks]}\label{q3c}

\textbf{Describe in brief all Transmission Media.}

\begin{solutionbox}
Transmission media are the physical paths through which
data travels in a network.

{\def\LTcaptype{none} % do not increment counter
\begin{longtable}[]{@{}
  >{\raggedright\arraybackslash}p{(\linewidth - 8\tabcolsep) * \real{0.2000}}
  >{\raggedright\arraybackslash}p{(\linewidth - 8\tabcolsep) * \real{0.1538}}
  >{\raggedright\arraybackslash}p{(\linewidth - 8\tabcolsep) * \real{0.2154}}
  >{\raggedright\arraybackslash}p{(\linewidth - 8\tabcolsep) * \real{0.2308}}
  >{\raggedright\arraybackslash}p{(\linewidth - 8\tabcolsep) * \real{0.2000}}@{}}
\toprule\noalign{}
\begin{minipage}[b]{\linewidth}\raggedright
Medium Type
\end{minipage} & \begin{minipage}[b]{\linewidth}\raggedright
Examples
\end{minipage} & \begin{minipage}[b]{\linewidth}\raggedright
Max Distance
\end{minipage} & \begin{minipage}[b]{\linewidth}\raggedright
Max Bandwidth
\end{minipage} & \begin{minipage}[b]{\linewidth}\raggedright
Application
\end{minipage} \\
\midrule\noalign{}
\endhead
\bottomrule\noalign{}
\endlastfoot
\textbf{Guided (Wired)} & & & & \\
Twisted Pair & UTP, STP & 100m & 10 Gbps & Office LANs \\
Coaxial Cable & RG-6, RG-59 & 500m & 10 Gbps & Cable TV, Internet \\
Fiber Optic & Single-mode, Multi-mode & 100km+ & 100+ Tbps & Backbones,
Long-distance \\
\textbf{Unguided (Wireless)} & & & & \\
Radio Waves & WiFi, Cellular & 100m-50km & 600 Mbps & Wireless
networks \\
Microwaves & Terrestrial, Satellite & Line of sight & 10 Gbps &
Point-to-point links \\
Infrared & IrDA & 1m & 16 Mbps & Remote controls \\
\end{longtable}
}

\textbf{Diagram:}

\begin{lstlisting}
GUIDED MEDIA:
  Twisted Pair: =~=~=~=~=~=~=~
  Coaxial:      =====|=====|=====
  Fiber Optic:  ======================>

UNGUIDED MEDIA:
  Radio:        ((( o )))
  Microwave:    <---> <--->
  Infrared:     * * * >
\end{lstlisting}

\begin{itemize}
\tightlist
\item
  \textbf{Guided media}: Physical paths confining signals
\item
  \textbf{Unguided media}: Wireless transmission through air/vacuum
\item
  \textbf{Characteristics}: Bandwidth, attenuation, noise immunity, cost
\end{itemize}

\end{solutionbox}
\begin{mnemonicbox}
``TRIM-CWF'' - ``Twisted, Radio, Infrared, Microwave,
Coaxial, Wireless, Fiber''

\end{mnemonicbox}
\subsection*{Question 4(a) [3 marks]}\label{q4a}

\textbf{Write note on DNS.}

\begin{solutionbox}
Domain Name System (DNS) translates human-friendly
domain names to IP addresses.

{\def\LTcaptype{none} % do not increment counter
\begin{longtable}[]{@{}ll@{}}
\toprule\noalign{}
Component & Function \\
\midrule\noalign{}
\endhead
\bottomrule\noalign{}
\endlastfoot
Domain Name & Hierarchical, readable address (www.example.com) \\
DNS Server & Resolves domain names to IP addresses \\
Root Server & Top of DNS hierarchy, points to TLDs \\
TLD Server & Manages top-level domains (.com, .org) \\
Record Types & A, AAAA, MX, CNAME, NS, PTR, etc. \\
\end{longtable}
}

\textbf{Diagram:}

\begin{lstlisting}
  CLIENT                                      ROOT DNS
+--------+   1. Query                        +--------+
|        |---"www.example.com?"------------->|        |
|        |   8. Response                     |        |
|        |<--"192.0.2.1"-------------------- |        |
+--------+      |                            +--------+
                |                                ^
                |                                |
                v                                |
              +--------+  2            +--------+ 7
              |  LOCAL |--TLD Server?->|   TLD  |
              |  DNS   |<-".com"-------|        |
              +--------+  3            +--------+
                  |                        ^
                  v 4                      | 6
              +--------+                +--------+
              |EXAMPLE |<---------------| DOMAIN |
              |  DNS   |--------------->|SERVER  |
              +--------+ 5              +--------+
\end{lstlisting}

\begin{itemize}
\tightlist
\item
  \textbf{Distributed database}: Hierarchical, globally distributed
\item
  \textbf{Caching}: Improves performance, reduces load
\item
  \textbf{Critical infrastructure}: Essential for Internet functionality
\end{itemize}

\end{solutionbox}
\begin{mnemonicbox}
``DIRT'' - ``Domain names Into Routable TCP/IP''

\end{mnemonicbox}
\subsection*{Question 4(b) [4 marks]}\label{q4b}

\textbf{Explain File Transfer Protocol.}

\begin{solutionbox}
File Transfer Protocol (FTP) enables transfer of files
between client and server over a network.

{\def\LTcaptype{none} % do not increment counter
\begin{longtable}[]{@{}ll@{}}
\toprule\noalign{}
Feature & Description \\
\midrule\noalign{}
\endhead
\bottomrule\noalign{}
\endlastfoot
Port & Control: 21, Data: 20 \\
Mode & Active and Passive \\
Security & Basic (clear text), or FTPS/SFTP for encryption \\
Commands & GET, PUT, LIST, DELETE, etc. \\
Connection & Uses separate control and data connections \\
\end{longtable}
}

\textbf{Diagram:}

\begin{lstlisting}
                     Control Connection (Port 21)
               +--------------------------------+
               |                                |
     +--------+|                                |+--------+
     |        ||                                ||        |
     | CLIENT |+--------------------------------+| SERVER |
     |        |                                  |        |
     |        |                                  |        |
     +--------+                                  +--------+
               +--------------------------------+
                     Data Connection (Port 20)
\end{lstlisting}

\begin{itemize}
\tightlist
\item
  \textbf{Dual channel}: Control channel and data channel
\item
  \textbf{Authentication}: Username/password required
\item
  \textbf{Modes}: ASCII (text) or Binary (raw data)
\item
  \textbf{Active vs Passive}: Different connection establishment methods
\end{itemize}

\end{solutionbox}
\begin{mnemonicbox}
``CAPS'' - ``Control And Port Separation''

\end{mnemonicbox}
\subsection*{Question 4(c) [7 marks]}\label{q4c}

\textbf{Classify different Internet Services and explain in detail.}

\begin{solutionbox}
Internet services provide various functionality over
the network.

{\def\LTcaptype{none} % do not increment counter
\begin{longtable}[]{@{}
  >{\raggedright\arraybackslash}p{(\linewidth - 6\tabcolsep) * \real{0.2429}}
  >{\raggedright\arraybackslash}p{(\linewidth - 6\tabcolsep) * \real{0.2571}}
  >{\raggedright\arraybackslash}p{(\linewidth - 6\tabcolsep) * \real{0.1857}}
  >{\raggedright\arraybackslash}p{(\linewidth - 6\tabcolsep) * \real{0.3143}}@{}}
\toprule\noalign{}
\begin{minipage}[b]{\linewidth}\raggedright
Service Category
\end{minipage} & \begin{minipage}[b]{\linewidth}\raggedright
Common Protocols
\end{minipage} & \begin{minipage}[b]{\linewidth}\raggedright
Description
\end{minipage} & \begin{minipage}[b]{\linewidth}\raggedright
Example Applications
\end{minipage} \\
\midrule\noalign{}
\endhead
\bottomrule\noalign{}
\endlastfoot
Communication & SMTP, POP3, IMAP & Exchange of messages & Email, Instant
Messaging \\
Information Access & HTTP, HTTPS & Access to information resources &
World Wide Web, Portals \\
File Sharing & FTP, BitTorrent, SMB & Transfer and sharing of files &
File hosting, P2P sharing \\
Remote Access & SSH, Telnet, RDP & Access remote computers & Remote
administration \\
Real-time Services & VoIP, WebRTC & Live communication & Video
conferencing, VoIP \\
Domain Services & DNS, DHCP & Network infrastructure & Address
resolution \\
\end{longtable}
}

\textbf{Information Access Services (Web):}

\begin{itemize}
\tightlist
\item
  \textbf{HTTP/HTTPS}: HyperText Transfer Protocol, foundation of web
\item
  \textbf{HTML}: Document format for displaying content
\item
  \textbf{Web browsers}: Client software to access and render web
  content
\item
  \textbf{Web servers}: Hosts websites and applications
\end{itemize}

\textbf{Communication Services (Email):}

\begin{itemize}
\tightlist
\item
  \textbf{SMTP}: For sending email
\item
  \textbf{POP3/IMAP}: For receiving email
\item
  \textbf{Components}: Mail user agents, transfer agents, delivery
  agents
\end{itemize}

\textbf{File Sharing Services:}

\begin{itemize}
\tightlist
\item
  \textbf{FTP}: Traditional file transfer protocol
\item
  \textbf{P2P}: Distributed file sharing without central server
\item
  \textbf{Cloud storage}: Remote file storage and synchronization
\end{itemize}

\end{solutionbox}
\begin{mnemonicbox}
``CIFRRD'' - ``Communication, Information, File,
Remote, Real-time, Domain''

\end{mnemonicbox}
\subsection*{Question 4(a) OR [3
marks]}\label{q4a}

\textbf{Explain Mail Protocols.}

\begin{solutionbox}
Mail protocols facilitate electronic messaging between
users.

{\def\LTcaptype{none} % do not increment counter
\begin{longtable}[]{@{}
  >{\raggedright\arraybackslash}p{(\linewidth - 6\tabcolsep) * \real{0.2703}}
  >{\raggedright\arraybackslash}p{(\linewidth - 6\tabcolsep) * \real{0.2703}}
  >{\raggedright\arraybackslash}p{(\linewidth - 6\tabcolsep) * \real{0.1622}}
  >{\raggedright\arraybackslash}p{(\linewidth - 6\tabcolsep) * \real{0.2973}}@{}}
\toprule\noalign{}
\begin{minipage}[b]{\linewidth}\raggedright
Protocol
\end{minipage} & \begin{minipage}[b]{\linewidth}\raggedright
Function
\end{minipage} & \begin{minipage}[b]{\linewidth}\raggedright
Port
\end{minipage} & \begin{minipage}[b]{\linewidth}\raggedright
Direction
\end{minipage} \\
\midrule\noalign{}
\endhead
\bottomrule\noalign{}
\endlastfoot
SMTP & Simple Mail Transfer Protocol & 25, 587 & Sending mail \\
POP3 & Post Office Protocol v3 & 110 & Retrieving mail \\
IMAP & Internet Message Access Protocol & 143 & Advanced mail
retrieval \\
MIME & Multipurpose Internet Mail Extensions & N/A & Encoding
attachments \\
\end{longtable}
}

\textbf{Diagram:}

\begin{lstlisting}
+---------+   SMTP    +---------+   POP3/IMAP  +---------+
|  Sender |---------->|  Mail   |------------->| Receiver|
|  Client |           |  Server |              |  Client |
+---------+           +---------+              +---------+
\end{lstlisting}

\begin{itemize}
\tightlist
\item
  \textbf{SMTP}: Outgoing mail delivery, push protocol
\item
  \textbf{POP3}: Simple mail retrieval, downloads and deletes
\item
  \textbf{IMAP}: Advanced retrieval, server-side storage, folders
\item
  \textbf{MIME}: Extends email capability for non-text content
\end{itemize}

\end{solutionbox}
\begin{mnemonicbox}
``SIM-P'' - ``SMTP sends, IMAP manages, POP3 pulls''

\end{mnemonicbox}
\subsection*{Question 4(b) OR [4
marks]}\label{q4b}

\textbf{Describe VOIP in brief.}

\begin{solutionbox}
Voice over Internet Protocol (VoIP) transmits voice
communications over IP networks.

{\def\LTcaptype{none} % do not increment counter
\begin{longtable}[]{@{}ll@{}}
\toprule\noalign{}
Component & Function \\
\midrule\noalign{}
\endhead
\bottomrule\noalign{}
\endlastfoot
Codec & Encodes/decodes voice signals \\
Signaling Protocol & Call setup/tear down (SIP, H.323) \\
Transport Protocol & Voice packet delivery (RTP) \\
QoS mechanism & Ensures voice quality \\
\end{longtable}
}

\textbf{Diagram:}

\begin{lstlisting}
+--------+   Internet/   +--------+
| CALLER |---IP Network---| CALLEE |
|ENDPOINT|               |ENDPOINT|
+--------+               +--------+
    |                        |
 [Analog]                 [Analog]
    |                        |
 [Digital]                [Digital]
    |                        |
 [Packets]  <---RTP--->  [Packets]
\end{lstlisting}

\begin{itemize}
\tightlist
\item
  \textbf{Packetization}: Converts analog voice to digital packets
\item
  \textbf{Benefits}: Cost savings, flexibility, integration with apps
\item
  \textbf{Challenges}: Quality of service, latency, jitter, packet loss
\end{itemize}

\end{solutionbox}
\begin{mnemonicbox}
``PALS'' - ``Packets Allowing Live Speech''

\end{mnemonicbox}
\subsection*{Question 4(c) OR [7
marks]}\label{q4c}

\textbf{Describe TCP and UDP protocols.}

\begin{solutionbox}
TCP and UDP are the primary transport layer protocols
in the TCP/IP suite.

{\def\LTcaptype{none} % do not increment counter
\begin{longtable}[]{@{}lll@{}}
\toprule\noalign{}
Feature & TCP & UDP \\
\midrule\noalign{}
\endhead
\bottomrule\noalign{}
\endlastfoot
Connection & Connection-oriented & Connectionless \\
Reliability & Guaranteed delivery & Best-effort delivery \\
Header size & 20-60 bytes & 8 bytes \\
Speed & Slower due to overhead & Faster with minimal overhead \\
Order & Maintains sequence & No sequence preservation \\
Flow control & Yes & No \\
Error recovery & Retransmission & None \\
Usage & Web, email, file transfer & Streaming, DNS, VoIP \\
\end{longtable}
}

\textbf{TCP Three-Way Handshake:}

\begin{lstlisting}
  CLIENT                SERVER
    |                     |
    |       SYN           |
    |-------------------->|
    |                     |
    |     SYN-ACK         |
    |<--------------------|
    |                     |
    |       ACK           |
    |-------------------->|
    |                     |
    |    DATA TRANSFER    |
    |<------------------->|
\end{lstlisting}

\textbf{TCP Features:}

\begin{itemize}
\tightlist
\item
  \textbf{Reliability}: Acknowledgments, retransmission
\item
  \textbf{Flow control}: Window-based, prevents overwhelming
\item
  \textbf{Congestion control}: Slow start, congestion avoidance
\item
  \textbf{Connection management}: Establishment, maintenance,
  termination
\end{itemize}

\textbf{UDP Features:}

\begin{itemize}
\tightlist
\item
  \textbf{Lightweight}: Minimal headers, no connection state
\item
  \textbf{Low latency}: No handshaking or acknowledgments
\item
  \textbf{No guarantees}: Data may arrive out of order, duplicated, or
  not at all
\item
  \textbf{Broadcast/multicast}: Supports one-to-many transmission
\end{itemize}

\end{solutionbox}
\begin{mnemonicbox}
``CRUFS'' - ``Connection, Reliability, UDP Fast,
Simple''

\end{mnemonicbox}
\subsection*{Question 5(a) [3 marks]}\label{q5a}

\textbf{Describe Cryptography.}

\begin{solutionbox}
Cryptography is the science of secure communication
techniques that protect information.

{\def\LTcaptype{none} % do not increment counter
\begin{longtable}[]{@{}
  >{\raggedright\arraybackslash}p{(\linewidth - 4\tabcolsep) * \real{0.2143}}
  >{\raggedright\arraybackslash}p{(\linewidth - 4\tabcolsep) * \real{0.4643}}
  >{\raggedright\arraybackslash}p{(\linewidth - 4\tabcolsep) * \real{0.3214}}@{}}
\toprule\noalign{}
\begin{minipage}[b]{\linewidth}\raggedright
Type
\end{minipage} & \begin{minipage}[b]{\linewidth}\raggedright
Description
\end{minipage} & \begin{minipage}[b]{\linewidth}\raggedright
Example
\end{minipage} \\
\midrule\noalign{}
\endhead
\bottomrule\noalign{}
\endlastfoot
Symmetric & Same key for encryption and decryption & AES, DES \\
Asymmetric & Different keys for encryption and decryption & RSA, ECC \\
Hash Functions & One-way functions, fixed output size & SHA-256, MD5 \\
Digital Signatures & Authentication and integrity verification & RSA
signatures \\
\end{longtable}
}

\textbf{Diagram:}

\begin{lstlisting}
SYMMETRIC:
  Sender --(Encrypt with Key K)--> [Ciphertext] --(Decrypt with Key K)--> Receiver

ASYMMETRIC:
  Sender --(Encrypt with Public Key)--> [Ciphertext] --(Decrypt with Private Key)--> Receiver
\end{lstlisting}

\begin{itemize}
\tightlist
\item
  \textbf{Confidentiality}: Protect information from unauthorized access
\item
  \textbf{Integrity}: Ensure information hasn't been altered
\item
  \textbf{Authentication}: Verify identity of communicating parties
\end{itemize}

\end{solutionbox}
\begin{mnemonicbox}
``SHAPE'' - ``Symmetric, Hashing, Asymmetric,
Protect, Encrypt''

\end{mnemonicbox}
\subsection*{Question 5(b) [4 marks]}\label{q5b}

\textbf{Explain Social issues and Hacking also discuss its precautions.}

\begin{solutionbox}
Social issues in cybersecurity involve human
manipulation and societal impacts of cyber threats.

{\def\LTcaptype{none} % do not increment counter
\begin{longtable}[]{@{}
  >{\raggedright\arraybackslash}p{(\linewidth - 4\tabcolsep) * \real{0.4000}}
  >{\raggedright\arraybackslash}p{(\linewidth - 4\tabcolsep) * \real{0.3429}}
  >{\raggedright\arraybackslash}p{(\linewidth - 4\tabcolsep) * \real{0.2571}}@{}}
\toprule\noalign{}
\begin{minipage}[b]{\linewidth}\raggedright
Social Issue
\end{minipage} & \begin{minipage}[b]{\linewidth}\raggedright
Description
\end{minipage} & \begin{minipage}[b]{\linewidth}\raggedright
Example
\end{minipage} \\
\midrule\noalign{}
\endhead
\bottomrule\noalign{}
\endlastfoot
Social Engineering & Manipulating people to reveal information &
Phishing, pretexting \\
Privacy Concerns & Unauthorized data collection and use & Data breaches,
surveillance \\
Digital Divide & Inequality in technology access & Limited Internet in
rural areas \\
Cyberbullying & Using technology to harass others & Online harassment,
threats \\
\end{longtable}
}

\textbf{Hacking Types:}

\begin{itemize}
\tightlist
\item
  \textbf{White Hat}: Ethical hacking, security improvement
\item
  \textbf{Black Hat}: Malicious hacking, illegal activities
\item
  \textbf{Grey Hat}: Mix of ethical and questionable actions
\end{itemize}

\textbf{Precautions:}

\begin{itemize}
\tightlist
\item
  \textbf{Education}: Regular security awareness training
\item
  \textbf{Strong Policies}: Clear security procedures and policies
\item
  \textbf{Technical Controls}: Firewalls, antivirus, encryption
\item
  \textbf{Regular Updates}: Patching systems against vulnerabilities
\item
  \textbf{Monitoring}: Activity logs, intrusion detection
\end{itemize}

\end{solutionbox}
\begin{mnemonicbox}
``STEPS'' - ``Social engineering, Training,
Encryption, Patches, Strong passwords''

\end{mnemonicbox}
\subsection*{Question 5(c) [7 marks]}\label{q5c}

\textbf{Explain IP Security in detail.}

\begin{solutionbox}
IP Security (IPsec) is a protocol suite that secures
communications at the IP layer.

{\def\LTcaptype{none} % do not increment counter
\begin{longtable}[]{@{}
  >{\raggedright\arraybackslash}p{(\linewidth - 4\tabcolsep) * \real{0.3235}}
  >{\raggedright\arraybackslash}p{(\linewidth - 4\tabcolsep) * \real{0.2941}}
  >{\raggedright\arraybackslash}p{(\linewidth - 4\tabcolsep) * \real{0.3824}}@{}}
\toprule\noalign{}
\begin{minipage}[b]{\linewidth}\raggedright
Component
\end{minipage} & \begin{minipage}[b]{\linewidth}\raggedright
Function
\end{minipage} & \begin{minipage}[b]{\linewidth}\raggedright
Description
\end{minipage} \\
\midrule\noalign{}
\endhead
\bottomrule\noalign{}
\endlastfoot
AH & Authentication Header & Provides integrity and authentication \\
ESP & Encapsulating Security Payload & Provides confidentiality,
integrity, authentication \\
IKE & Internet Key Exchange & Establishes and manages security
associations \\
SA & Security Association & Security parameters for a connection \\
\end{longtable}
}

\textbf{IPsec Modes:}

{\def\LTcaptype{none} % do not increment counter
\begin{longtable}[]{@{}lll@{}}
\toprule\noalign{}
Mode & Description & Application \\
\midrule\noalign{}
\endhead
\bottomrule\noalign{}
\endlastfoot
Transport & Protects payload only & Host-to-host communications \\
Tunnel & Protects entire packet & Gateway-to-gateway (VPN) \\
\end{longtable}
}

\textbf{Diagram:}

\begin{lstlisting}
TRANSPORT MODE:
  +------+-------+----------------+
  |  IP  | IPsec |    Payload     |
  |Header|Header |                |
  +------+-------+----------------+

TUNNEL MODE:
  +------+-------+------+-------+----------------+
  | New  | IPsec | Orig |  TCP  |    Payload     |
  | IP   |Header | IP   |Header |                |
  +------+-------+------+-------+----------------+
\end{lstlisting}

\textbf{IPsec Services:}

\begin{itemize}
\tightlist
\item
  \textbf{Authentication}: Verifies sender identity
\item
  \textbf{Confidentiality}: Encrypts data to prevent eavesdropping
\item
  \textbf{Integrity}: Ensures data hasn't been modified
\item
  \textbf{Anti-replay}: Prevents packet replay attacks
\end{itemize}

\textbf{IPsec Implementation:}

\begin{itemize}
\tightlist
\item
  \textbf{VPNs}: Secure remote access and site-to-site connections
\item
  \textbf{L2TP/IPsec}: Combines tunneling with security
\item
  \textbf{Authentication methods}: Pre-shared keys, certificates,
  Kerberos
\end{itemize}

\end{solutionbox}
\begin{mnemonicbox}
``ACCEPT'' - ``Authentication, Confidentiality,
Cryptography, Encapsulation, Protocols, Tunnel''

\end{mnemonicbox}
\subsection*{Question 5(a) OR [3
marks]}\label{q5a}

\textbf{Define Network Security and explain its elements.}

\begin{solutionbox}
Network security is the protection of network
infrastructure, data, and access against unauthorized use, malfunction,
modification, or destruction.

{\def\LTcaptype{none} % do not increment counter
\begin{longtable}[]{@{}
  >{\raggedright\arraybackslash}p{(\linewidth - 4\tabcolsep) * \real{0.2812}}
  >{\raggedright\arraybackslash}p{(\linewidth - 4\tabcolsep) * \real{0.4062}}
  >{\raggedright\arraybackslash}p{(\linewidth - 4\tabcolsep) * \real{0.3125}}@{}}
\toprule\noalign{}
\begin{minipage}[b]{\linewidth}\raggedright
Element
\end{minipage} & \begin{minipage}[b]{\linewidth}\raggedright
Description
\end{minipage} & \begin{minipage}[b]{\linewidth}\raggedright
Examples
\end{minipage} \\
\midrule\noalign{}
\endhead
\bottomrule\noalign{}
\endlastfoot
Access Control & Limiting network access & Passwords, multi-factor
auth \\
Threat Prevention & Blocking attacks & Firewalls, IDS/IPS \\
Encryption & Securing data in transit & SSL/TLS, IPsec \\
Vulnerability Management & Identifying weaknesses & Scanning,
patching \\
Monitoring & Observing network activity & SIEM, log analysis \\
\end{longtable}
}

\textbf{Diagram:}

\begin{lstlisting}
                +------------------+
                | NETWORK SECURITY |
                +------------------+
                         |
       +--------+-----------+--------+--------+
       |        |           |        |        |
 +-----------+ +-------+ +------+ +------+ +--------+
 |   ACCESS  | |THREAT | |ENCRYP| |VULNER| |MONITOR |
 |  CONTROL  | |PREVENT| |TION  | |MGMT  | |ING     |
 +-----------+ +-------+ +------+ +------+ +--------+
\end{lstlisting}

\begin{itemize}
\tightlist
\item
  \textbf{Confidentiality}: Protecting information from unauthorized
  access
\item
  \textbf{Integrity}: Ensuring information accuracy and reliability
\item
  \textbf{Availability}: Maintaining systems accessible when needed
\end{itemize}

\end{solutionbox}
\begin{mnemonicbox}
``CIMA TV'' - ``Confidentiality, Integrity,
Monitoring, Access control, Threats, Vulnerabilities''

\end{mnemonicbox}
\subsection*{Question 5(b) OR [4
marks]}\label{q5b}

\textbf{Briefly describe the Information Technology (Amendment) Act,
2008, and its impact on cyber laws in India.}

\begin{solutionbox}
The IT (Amendment) Act, 2008 updated India's cyber laws
to address emerging cybersecurity challenges.

{\def\LTcaptype{none} % do not increment counter
\begin{longtable}[]{@{}
  >{\raggedright\arraybackslash}p{(\linewidth - 2\tabcolsep) * \real{0.4800}}
  >{\raggedright\arraybackslash}p{(\linewidth - 2\tabcolsep) * \real{0.5200}}@{}}
\toprule\noalign{}
\begin{minipage}[b]{\linewidth}\raggedright
Key Aspect
\end{minipage} & \begin{minipage}[b]{\linewidth}\raggedright
Description
\end{minipage} \\
\midrule\noalign{}
\endhead
\bottomrule\noalign{}
\endlastfoot
Cyber Crimes & Added new offenses, strengthened penalties \\
Electronic Evidence & Recognized digital evidence in court \\
Data Protection & Imposed obligations for sensitive data \\
Intermediary Liability & Defined responsibilities for service
providers \\
\end{longtable}
}

\textbf{Key Sections:}

\begin{itemize}
\tightlist
\item
  \textbf{Section 43}: Penalties for unauthorized access, data theft
\item
  \textbf{Section 66}: Computer-related offenses and punishments
\item
  \textbf{Section 69}: Powers for interception and monitoring
\item
  \textbf{Section 72A}: Protection of personal data privacy
\end{itemize}

\textbf{Impact on Cyber Laws:}

\begin{itemize}
\tightlist
\item
  \textbf{Stronger enforcement}: Enhanced penalties for cyber crimes
\item
  \textbf{Expanded scope}: Covered new technological developments
\item
  \textbf{Corporate responsibility}: Required security practices for
  data
\item
  \textbf{Global alignment}: Harmonized with international standards
\end{itemize}

\end{solutionbox}
\begin{mnemonicbox}
``SPEC'' - ``Security, Privacy, Evidence, Cyber
crimes''

\end{mnemonicbox}
\subsection*{Question 5(c) OR [7
marks]}\label{q5c}

\textbf{Explain Email security in terms of SMTP, PEM, PGP, S/MINE,
spam.}

\begin{solutionbox}
Email security protects email content and accounts from
unauthorized access and attacks.

{\def\LTcaptype{none} % do not increment counter
\begin{longtable}[]{@{}
  >{\raggedright\arraybackslash}p{(\linewidth - 4\tabcolsep) * \real{0.3750}}
  >{\raggedright\arraybackslash}p{(\linewidth - 4\tabcolsep) * \real{0.3125}}
  >{\raggedright\arraybackslash}p{(\linewidth - 4\tabcolsep) * \real{0.3125}}@{}}
\toprule\noalign{}
\begin{minipage}[b]{\linewidth}\raggedright
Technology
\end{minipage} & \begin{minipage}[b]{\linewidth}\raggedright
Function
\end{minipage} & \begin{minipage}[b]{\linewidth}\raggedright
Features
\end{minipage} \\
\midrule\noalign{}
\endhead
\bottomrule\noalign{}
\endlastfoot
SMTP & Simple Mail Transfer Protocol & Basic email transmission, limited
security \\
PEM & Privacy Enhanced Mail & Early email encryption standard \\
PGP & Pretty Good Privacy & End-to-end encryption, digital signatures \\
S/MIME & Secure/Multipurpose Internet Mail Extensions &
Certificate-based encryption and signing \\
Anti-spam & Unwanted email filtering & Content filtering, blacklists,
authentication \\
\end{longtable}
}

\textbf{SMTP Security Issues:}

\begin{itemize}
\tightlist
\item
  Originally designed without security
\item
  Authentication extensions (AUTH) added later
\item
  Vulnerable to eavesdropping without encryption
\item
  Supports STARTTLS for encrypted transmission
\end{itemize}

\textbf{PGP Email Security:}

\begin{lstlisting}
SENDER                                 RECEIVER
  |                                      |
  |-- Create message                     |
  |-- Sign with private key              |
  |-- Encrypt with recipients public key |
  |                                      |
  |      Encrypted Email                 |
  |------------------------------------->|
  |                                      |
  |                                      |-- Decrypt with private key
  |                                      |-- Verify with sender's public key
\end{lstlisting}

\textbf{S/MIME Features:}

\begin{itemize}
\tightlist
\item
  Uses X.509 certificates for authentication
\item
  Provides encryption and digital signatures
\item
  Integrated into many email clients
\item
  Requires certificate infrastructure
\end{itemize}

\textbf{Spam Protection:}

\begin{itemize}
\tightlist
\item
  \textbf{Content filtering}: Analyzing message content
\item
  \textbf{Sender verification}: SPF, DKIM, DMARC
\item
  \textbf{Behavioral analysis}: Pattern recognition
\item
  \textbf{Blacklists/whitelists}: Blocking/allowing specific senders
\end{itemize}

\textbf{Email Security Best Practices:}

\begin{itemize}
\tightlist
\item
  \textbf{Encryption}: Ensure privacy of message content
\item
  \textbf{Authentication}: Verify sender identity
\item
  \textbf{Access controls}: Protect email accounts
\item
  \textbf{Filtering}: Block malicious and unwanted messages
\item
  \textbf{User education}: Recognize phishing attempts
\end{itemize}

\end{solutionbox}
\begin{mnemonicbox}
``SPEED'' - ``S/MIME, PGP, Encryption, Email
security, DMARC''

\end{mnemonicbox}

\end{document}
