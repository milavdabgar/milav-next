\documentclass[10pt,a4paper]{article}

% content/resources/templates/preamble.tex
\usepackage[margin=0.6in]{geometry}
\author{Milav Dabgar}
\usepackage{amsmath,amssymb,amsthm}
\usepackage{booktabs}
\usepackage{multirow}
\usepackage{xcolor}
\usepackage{tcolorbox}
\tcbuselibrary{breakable,skins}
\usepackage[colorlinks=true,linkcolor=blue]{hyperref}
\usepackage{titlesec}
\usepackage{enumitem}
\usepackage{tikz}
\usepackage{pgfplots}
\usepackage{circuitikz}
\usepackage[version=4]{mhchem}
\usepackage{longtable}
\usepackage{array}
\usepackage{float}
\usepackage{caption}
\usepackage{listings}

\lstset{
  basicstyle=\small\ttfamily,
  breaklines=true,
  breakatwhitespace=false,
  postbreak=\mbox{\textcolor{red}{$\hookrightarrow$}\space},
  float=false,
  numbers=left,
  numberstyle=\tiny\color{gray},
  numbersep=10pt,
  xleftmargin=2em,
  keywordstyle=\color{blue},
  commentstyle=\color{green!60!black},
  stringstyle=\color{purple},
  backgroundcolor=\color{gray!5},
  showstringspaces=false,
  tabsize=2,
  captionpos=b,
  keepspaces=true,
  columns=flexible
}

\pgfplotsset{compat=1.18}
\usetikzlibrary{shapes,arrows,positioning,calc,patterns,decorations.pathmorphing,decorations.markings,arrows.meta}

% Color scheme
\definecolor{headcolor}{RGB}{0,102,204}
\definecolor{keycolor}{RGB}{220,20,60}
\definecolor{solutioncolor}{RGB}{34,139,34}
\definecolor{mnemoniccolor}{RGB}{148,0,211}
\definecolor{codecolor}{RGB}{0,0,100}

% Spacing
\setlength{\parskip}{3pt}
\setlist[itemize]{nosep}
\setlist[enumerate]{nosep}

% Title formatting
\titleformat{\section}{\Large\bfseries\color{headcolor}}{\thesection}{1em}{}
\titleformat{\subsection}{\large\bfseries\color{headcolor}}{\thesubsection}{1em}{}

% Pandoc tightlist compatibility
\providecommand{\tightlist}{%
  \setlength{\itemsep}{0pt}\setlength{\parskip}{0pt}}

% Pandoc longtable compatibility
\newcounter{none}
\def\thenone{}


% content/resources/templates/gujarati-boxes.tex
\usepackage{fontspec}
\usepackage{polyglossia}

% Set Gujarati as main language (document is primarily in Gujarati)
% Note: gloss-gujarati.ldf doesn't exist in polyglossia, but it will use hyphenation patterns
\setdefaultlanguage{gujarati}
\setotherlanguage{english}

% Configure Gujarati font properly
% Use Language=Default to prevent polyglossia from trying to add language-specific features
% that don't exist for Gujarati, which causes "empty feature" warnings
\newfontfamily\gujaratifont[Script=Gujarati,AutoFakeBold=2.5,AutoFakeSlant=0.3]{Noto Sans Gujarati}
\setmainfont[Script=Gujarati,AutoFakeBold=2.5,AutoFakeSlant=0.3]{Noto Sans Gujarati}
% Use Noto Sans Gujarati for monospace to support Gujarati in text
\setmonofont[Scale=0.9]{Noto Sans Gujarati}

% Configure English to use the same font
\newfontfamily\englishfont[Script=Gujarati,AutoFakeBold=2.5,AutoFakeSlant=0.3]{Noto Sans Gujarati}

% Translations for polyglossia
\gappto\captionsgujarati{
  \renewcommand{\tablename}{કોષ્ટક}
  \renewcommand{\figurename}{આકૃતિ}
}

% Helper for TikZ nodes to ensure Gujarati font
\newcommand{\gu}[1]{{\gujaratifont #1}}

% Custom environments
\newtcolorbox{solutionbox}{
    breakable,
    enhanced,
    colback=solutioncolor!5!white,
    colframe=solutioncolor!75!black,
    fonttitle=\bfseries,
    title=જવાબ
}

\newtcolorbox{solutionboxnobreak}{
 colback=solutioncolor!5!white,
 colframe=solutioncolor!75!black,
 fonttitle=\bfseries,
 title=જવાબ
}

\newtcolorbox{keyformula}{
 breakable,
 enhanced,
 colback=keycolor!5!white,
 colframe=keycolor!75!black,
 fonttitle=\bfseries,
 title=રાસાયણિક સમીકરણ/સૂત્ર
}

\newtcolorbox{mnemonicbox}{
 breakable,
 enhanced,
 colback=mnemoniccolor!5!white,
 colframe=mnemoniccolor!75!black,
 fonttitle=\bfseries,
 title=મેમરી ટ્રીક
}


\begin{document}

\begin{center}
{\Huge\bfseries\color{headcolor} Subject Name (Gujarati)}\\[5pt]
{\LARGE 4343202 -- Winter 2024}\\[3pt]
{\large Semester 1 Study Material}\\[3pt]
{\normalsize\textit{Detailed Solutions and Explanations}}
\end{center}

\vspace{10pt}

\subsection*{પ્રશ્ન 1(અ) [3
ગુણ]}\label{uxaaauxab0uxab6uxaa8-1uxa85-3-uxa97uxaa3}

\textbf{કોમ્પ્યુટર નેટવર્ક શું છે? તે શા માટે મહત્વનું છે?}

\begin{solutionbox}
કમ્પ્યુટર નેટવર્ક એ ઇન્ટરકનેક્ટેડ કમ્પ્યુટિંગ ડિવાઇસનો સમૂહ છે જે ડેટા
એક્સચેન્જ અને રિસોર્સ શેરિંગ કરી શકે છે.

\textbf{આકૃતિ:}

\begin{lstlisting}
     +--------+           +--------+
     |Computer|-----------|Computer|
     +--------+    |      +--------+
                   |
     +--------+    |      +--------+
     |Computer|----+------|Computer|
     +--------+           +--------+
\end{lstlisting}

\begin{itemize}
\tightlist
\item
  \textbf{રિસોર્સ શેરિંગ}: પ્રિન્ટર, ફાઇલ, એપ્લિકેશન શેર કરવાની સુવિધા
\item
  \textbf{કોમ્યુનિકેશન}: વપરાશકર્તાઓ વચ્ચે માહિતીનું આદાન-પ્રદાન સરળ બનાવે
\item
  \textbf{સ્કેલેબિલિટી}: નેટવર્કને જરૂરિયાત મુજબ વિસ્તારી શકાય છે
\end{itemize}

\end{solutionbox}
\begin{mnemonicbox}
``CSI'' - ``કનેક્ટ, શેર, ઇન્ટરેક્ટ''

\end{mnemonicbox}
\subsection*{પ્રશ્ન 1(બ) [4
ગુણ]}\label{uxaaauxab0uxab6uxaa8-1uxaac-4-uxa97uxaa3}

\textbf{વ્યાખ્યા આપો: ૧)વેબ સર્વર, ૨)એનક્રિપ્તેડ ડેટા, ૩) હેકિંગ, ૪) ક્લાયન્ટ-સર્વર}

\begin{solutionbox}

{\def\LTcaptype{none} % do not increment counter
\begin{longtable}[]{@{}
  >{\raggedright\arraybackslash}p{(\linewidth - 2\tabcolsep) * \real{0.4000}}
  >{\raggedright\arraybackslash}p{(\linewidth - 2\tabcolsep) * \real{0.6000}}@{}}
\toprule\noalign{}
\begin{minipage}[b]{\linewidth}\raggedright
શબ્દ
\end{minipage} & \begin{minipage}[b]{\linewidth}\raggedright
વ્યાખ્યા
\end{minipage} \\
\midrule\noalign{}
\endhead
\bottomrule\noalign{}
\endlastfoot
વેબ સર્વર & HTTP/HTTPS નો ઉપયોગ કરી ક્લાયન્ટને વેબ કન્ટેન્ટ પ્રદાન કરતું
સોફ્ટવેર/હાર્ડવેર \\
એનક્રિપ્ટેડ ડેટા & અનધિકૃત એક્સેસને રોકવા માટે કોડમાં રૂપાંતરિત કરેલી માહિતી \\
હેકિંગ & સિક્યોરિટી વલ્નરેબિલિટીઝ દ્વારા કમ્પ્યુટર સિસ્ટમમાં અનધિકૃત એક્સેસ \\
ક્લાયન્ટ-સર્વર & સેન્ટ્રલાઈઝ્ડ સર્વર ક્લાયન્ટ કમ્પ્યુટરને સેવાઓ પ્રદાન કરે તે નેટવર્ક
મોડેલ \\
\end{longtable}
}

\textbf{આકૃતિ:}

\begin{lstlisting}
CLIENT-SERVER MODEL:
  +--------+   REQUEST    +--------+
  | CLIENT |------------->| SERVER |
  |        |<-------------|        |
  +--------+   RESPONSE   +--------+
\end{lstlisting}

\end{solutionbox}
\begin{mnemonicbox}
``WECHS'' - ``વેબ સર્વર એનક્રિપ્ટ ડેટા, ક્લાયન્ટ અને હેકર્સ
સર્વરનો ઉપયોગ કરે છે''

\end{mnemonicbox}
\subsection*{પ્રશ્ન 1(ક) [7
ગુણ]}\label{uxaaauxab0uxab6uxaa8-1uxa95-7-uxa97uxaa3}

\textbf{ટ્રાન્સમિશન મીડીયાનું ક્લાસીફીકેશન આપો અને સમજાવો.}

\begin{solutionbox}
ટ્રાન્સમિશન મીડિયા એ ભૌતિક માધ્યમો છે જે નેટવર્કમાં ડેટાનું વહન કરે
છે.

{\def\LTcaptype{none} % do not increment counter
\begin{longtable}[]{@{}
  >{\raggedright\arraybackslash}p{(\linewidth - 6\tabcolsep) * \real{0.2083}}
  >{\raggedright\arraybackslash}p{(\linewidth - 6\tabcolsep) * \real{0.1458}}
  >{\raggedright\arraybackslash}p{(\linewidth - 6\tabcolsep) * \real{0.3542}}
  >{\raggedright\arraybackslash}p{(\linewidth - 6\tabcolsep) * \real{0.2917}}@{}}
\toprule\noalign{}
\begin{minipage}[b]{\linewidth}\raggedright
કેટેગરી
\end{minipage} & \begin{minipage}[b]{\linewidth}\raggedright
પ્રકાર
\end{minipage} & \begin{minipage}[b]{\linewidth}\raggedright
લાક્ષણિકતાઓ
\end{minipage} & \begin{minipage}[b]{\linewidth}\raggedright
ઉપયોગો
\end{minipage} \\
\midrule\noalign{}
\endhead
\bottomrule\noalign{}
\endlastfoot
\textbf{ગાઇડેડ મીડિયા} & & & \\
ટ્વિસ્ટેડ પેર & UTP, STP & 100m રેન્જ, 10Mbps-10Gbps & ઓફિસ LANs \\
કોએક્સિયલ કેબલ & બેસબેન્ડ, બ્રોડબેન્ડ & 500m રેન્જ, 10-100Mbps & કેબલ TV, ઇન્ટરનેટ \\
ફાયબર ઓપ્ટિક & સિંગલ-મોડ, મલ્ટી-મોડ & લાંબું અંતર, 100Mbps-100Gbps & બેકબોન,
WAN \\
\textbf{અનગાઇડેડ મીડિયા} & & & \\
રેડિયો વેવ્સ & WiFi, સેલ્યુલર & ઓમ્નિડિરેક્શનલ, 1-100Mbps & વાયરલેસ નેટવર્ક \\
માઇક્રોવેવ્સ & ટેરેસ્ટ્રિયલ, સેટેલાઇટ & લાઇન-ઓફ-સાઇટ, 1-10Gbps & પોઇન્ટ-ટુ-પોઇન્ટ
લિંક \\
ઇન્ફ્રારેડ & IrDA & શોર્ટ-રેન્જ, 4-16Mbps & રિમોટ કંટ્રોલ \\
\end{longtable}
}

\textbf{આકૃતિ:}

\begin{lstlisting}
GUIDED MEDIA:
  Twisted Pair: =~=~=~=~=~=~=~
  Coaxial:      =====|=====|=====
  Fiber Optic:  ======================>

UNGUIDED MEDIA:
  Radio:        ((( o )))
  Microwave:    <---> <--->
  Infrared:     * * * >
\end{lstlisting}

\begin{itemize}
\tightlist
\item
  \textbf{ગાઇડેડ મીડિયા}: સિગ્નલને મર્યાદિત કરતા ભૌતિક માર્ગો
\item
  \textbf{અનગાઇડેડ મીડિયા}: હવા/અવકાશ દ્વારા વાયરલેસ ટ્રાન્સમિશન
\item
  \textbf{પસંદગીના પરિબળો}: ખર્ચ, બેન્ડવિડ્થ, અંતર, પર્યાવરણ
\end{itemize}

\end{solutionbox}
\begin{mnemonicbox}
``TCFRIM'' - ``ટ્વિસ્ટેડ પેર, કોએક્સિયલ, ફાયબર, રેડિયો,
ઇન્ફ્રારેડ, માઇક્રોવેવ''

\end{mnemonicbox}
\subsection*{પ્રશ્ન 1(ક) અથવા [7
ગુણ]}\label{uxaaauxab0uxab6uxaa8-1uxa95-uxa85uxaa5uxab5-7-uxa97uxaa3}

\textbf{WAN અને MAN ને સમજાવો.}

\begin{solutionbox}
વાઇડ એરિયા નેટવર્ક (WAN) અને મેટ્રોપોલિટન એરિયા નેટવર્ક (MAN) એ
ભૌગોલિક વિસ્તારના આધારે વર્ગીકૃત થયેલા નેટવર્ક પ્રકારો છે.

{\def\LTcaptype{none} % do not increment counter
\begin{longtable}[]{@{}lll@{}}
\toprule\noalign{}
ફીચર & MAN (મેટ્રોપોલિટન એરિયા નેટવર્ક) & WAN (વાઇડ એરિયા નેટવર્ક) \\
\midrule\noalign{}
\endhead
\bottomrule\noalign{}
\endlastfoot
કવરેજ & શહેર-વ્યાપી (5-50 km) & દેશ/વૈશ્વિક (\textgreater50 km) \\
સ્પીડ & 10 Mbps - 10 Gbps & 1.5 Mbps - 1 Gbps \\
માલિકી & મ્યુનિસિપલ/ટેલિકોમ & મલ્ટિપલ ઓર્ગેનાઇઝેશન \\
ટેકનોલોજી & Ethernet, SONET, WiMAX & Frame Relay, ATM, MPLS \\
ઉદાહરણો & સિટી નેટવર્ક, કેમ્પસ નેટવર્ક & ઇન્ટરનેટ, કોર્પોરેટ નેટવર્ક \\
\end{longtable}
}

\textbf{આકૃતિ:}

\begin{lstlisting}
     WAN                       MAN
  +--------+               +--------+
  | Global |               |  City  |
  +--------+               +--------+
      |                        |
      v                        v
  +---------+           +-----------+
  | Multiple |           | Connected |
  | Countries|           |Campuses/  |
  +---------+           |City areas  |
                        +-----------+
\end{lstlisting}

\begin{itemize}
\tightlist
\item
  \textbf{MAN}: શહેર/મેટ્રોપોલિટન એરિયામાં LANsને જોડે છે
\item
  \textbf{WAN}: શહેરો/દેશો વચ્ચે મોટા ભૌગોલિક વિસ્તારોને આવરે છે
\item
  \textbf{મેનેજમેન્ટ}: WAN સામાન્ય રીતે સર્વિસ પ્રોવાઇડર્સની જરૂર પડે છે
\item
  \textbf{ઇન્ફ્રાસ્ટ્રક્ચર}: અલગ-અલગ ટ્રાન્સમિશન મીડિયા અને ટેકનોલોજીઓ
\end{itemize}

\end{solutionbox}
\begin{mnemonicbox}
``SWIM'' - ``સાઇઝ: WAN ઇઝ મેસિવ કમ્પેર્ડ ટુ MAN''

\end{mnemonicbox}
\subsection*{પ્રશ્ન 2(અ) [3
ગુણ]}\label{uxaaauxab0uxab6uxaa8-2uxa85-3-uxa97uxaa3}

\textbf{વિગતવાર સમજાવો: ટ્રાન્સમિશન ટેકનોલોજી.}

\begin{solutionbox}
ટ્રાન્સમિશન ટેકનોલોજી એ નેટવર્ક ડિવાઇસ વચ્ચે ડેટા ટ્રાન્સફર કરવા
માટે વપરાતી પદ્ધતિઓને કહે છે.

{\def\LTcaptype{none} % do not increment counter
\begin{longtable}[]{@{}lll@{}}
\toprule\noalign{}
ટેકનોલોજી ટાઇપ & વર્ણન & ઉદાહરણ \\
\midrule\noalign{}
\endhead
\bottomrule\noalign{}
\endlastfoot
પોઇન્ટ-ટુ-પોઇન્ટ & બે નોડ્સ વચ્ચે સીધું કનેક્શન & લીઝ્ડ લાઇન \\
બ્રોડકાસ્ટ & બધા નોડ્સ દ્વારા શેર કરાતું સિંગલ કોમ્યુનિકેશન ચેનલ & વાયરલેસ LAN \\
મલ્ટિપોઇન્ટ & મલ્ટિપલ ડિવાઇસ એક લિંક શેર કરે & કેબલ નેટવર્ક \\
\end{longtable}
}

\begin{itemize}
\tightlist
\item
  \textbf{એનાલોગ ટ્રાન્સમિશન}: કન્ટિન્યુઅસ સિગ્નલ, નોઇઝને લગતું
\item
  \textbf{ડિજિટલ ટ્રાન્સમિશન}: ડિસ્ક્રીટ સિગ્નલ, વધુ વિશ્વસનીય
\item
  \textbf{બેસબેન્ડ}: સિંગલ સિગ્નલ સમગ્ર બેન્ડવિડ્થનો ઉપયોગ કરે છે (Ethernet)
\item
  \textbf{બ્રોડબેન્ડ}: મલ્ટિપલ સિગ્નલ્સ બેન્ડવિડ્થ શેર કરે છે (કેબલ TV)
\end{itemize}

\end{solutionbox}
\begin{mnemonicbox}
``ABP-DMB'' - ``એનાલોગ ઓર બેસબેન્ડ, પોઇન્ટ-ટુ-પોઇન્ટ;
ડિજિટલ ઓર મલ્ટિપોઇન્ટ, બ્રોડકાસ્ટ''

\end{mnemonicbox}
\subsection*{પ્રશ્ન 2(બ) [4
ગુણ]}\label{uxaaauxab0uxab6uxaa8-2uxaac-4-uxa97uxaa3}

\textbf{સ્ટાર ટોપોલોજી દોરો અને સમજાવો.}

\begin{solutionbox}
સ્ટાર ટોપોલોજી એ નેટવર્ક કોન્ફિગરેશન છે જ્યાં બધા ડિવાઇસ સેન્ટ્રલ
હબ/સ્વિચ સાથે જોડાયેલા હોય છે.

\textbf{આકૃતિ:}

\begin{lstlisting}
              STAR TOPOLOGY
                 +-----+
                 | HUB/|
                 |SWITCH|
                 +-----+
                    |
         +----------+-----------+
         |          |           |
      +-----+    +-----+     +-----+
      |Node1|    |Node2|     |Node3|
      +-----+    +-----+     +-----+
         |          |           |
      +-----+    +-----+     +-----+
      |Node4|    |Node5|     |Node6|
      +-----+    +-----+     +-----+
\end{lstlisting}

{\def\LTcaptype{none} % do not increment counter
\begin{longtable}[]{@{}ll@{}}
\toprule\noalign{}
ફાયદા & ગેરફાયદા \\
\midrule\noalign{}
\endhead
\bottomrule\noalign{}
\endlastfoot
સરળ ઇન્સ્ટોલેશન & સિંગલ પોઇન્ટ ઓફ ફેલ્યોર (હબ/સ્વિચ) \\
સરળ ટ્રબલશૂટિંગ & બસ ટોપોલોજી કરતાં વધુ કેબલની જરૂર \\
સ્કેલેબલ & સેન્ટ્રલ ડિવાઇસને કારણે ઉંચી કિંમત \\
બેટર પરફોર્મન્સ & હબ/સ્વિચ લિમિટ નેટવર્ક સાઇઝ નક્કી કરે છે \\
\end{longtable}
}

\begin{itemize}
\tightlist
\item
  \textbf{ઓપરેશન}: બધો ડેટા સેન્ટ્રલ ડિવાઇસમાંથી પસાર થાય છે
\item
  \textbf{ઇન્સ્ટોલેશન}: મેનેજ અને એક્સપાન્ડ કરવામાં સરળ
\item
  \textbf{ફોલ્ટ આઇસોલેશન}: નોડ ફેલ્યોર અન્યને અસર કરતું નથી
\end{itemize}

\end{solutionbox}
\begin{mnemonicbox}
``CASE'' - ``સેન્ટ્રલાઇઝ્ડ, ઓલ કનેક્ટેડ, સિમ્પલ એક્સપાન્શન,
ઇઝી ટ્રબલશૂટિંગ''

\end{mnemonicbox}
\subsection*{પ્રશ્ન 2(ક) [7
ગુણ]}\label{uxaaauxab0uxab6uxaa8-2uxa95-7-uxa97uxaa3}

\textbf{TCP/IP મોડેલ દોરો અને સમજાવો.}

\begin{solutionbox}
TCP/IP મોડેલ એ નેટવર્ક કોમ્યુનિકેશન માટે વપરાતું કન્સેપ્ચ્યુઅલ ફ્રેમવર્ક
છે, જેમાં ચાર લેયર સમાવિષ્ટ છે.

\textbf{આકૃતિ:}

\begin{lstlisting}
+-----------------------------+
|       APPLICATION LAYER     |
| (HTTP, FTP, SMTP, DNS, etc.)|
+-----------------------------+
|        TRANSPORT LAYER      |
|          (TCP, UDP)         |
+-----------------------------+
|        INTERNET LAYER       |
|      (IP, ICMP, ARP)        |
+-----------------------------+
|    NETWORK ACCESS LAYER     |
| (Ethernet, Wi-Fi, PPP, etc.)|
+-----------------------------+
        PHYSICAL MEDIA
\end{lstlisting}

{\def\LTcaptype{none} % do not increment counter
\begin{longtable}[]{@{}lll@{}}
\toprule\noalign{}
લેયર & મુખ્ય ફંકશન & પ્રોટોકોલ્સ \\
\midrule\noalign{}
\endhead
\bottomrule\noalign{}
\endlastfoot
એપ્લિકેશન & યુઝર ઇન્ટરફેસ, ડેટા ફોર્મેટિંગ & HTTP, FTP, SMTP, DNS \\
ટ્રાન્સપોર્ટ & એન્ડ-ટુ-એન્ડ કોમ્યુનિકેશન, રિલાયબિલિટી & TCP, UDP \\
ઇન્ટરનેટ & લોજિકલ એડ્રેસિંગ, રાઉટિંગ & IP, ICMP, ARP, IGMP \\
નેટવર્ક એક્સેસ & ફિઝિકલ એડ્રેસિંગ, મીડિયા એક્સેસ & Ethernet, WiFi, PPP \\
\end{longtable}
}

\begin{itemize}
\tightlist
\item
  \textbf{એપ્લિકેશન લેયર}: એપ્લિકેશન અને નેટવર્ક વચ્ચે ઇન્ટરફેસ
\item
  \textbf{ટ્રાન્સપોર્ટ લેયર}: એન્ડ સિસ્ટમ્સ વચ્ચે વિશ્વસનીય ડેટા ટ્રાન્સફર
\item
  \textbf{ઇન્ટરનેટ લેયર}: નેટવર્ક વચ્ચે પેકેટ રાઉટિંગ
\item
  \textbf{નેટવર્ક એક્સેસ લેયર}: નેટવર્ક મીડિયા સાથે ફિઝિકલ કનેક્શન
\end{itemize}

\end{solutionbox}
\begin{mnemonicbox}
``ATNI'' - ``એપ્લિકેશન ટોક્સ, નેટવર્ક ઇન્ટરનેટ ઇન્ટરફેસીસ''

\end{mnemonicbox}
\subsection*{પ્રશ્ન 2(અ) અથવા [3
ગુણ]}\label{uxaaauxab0uxab6uxaa8-2uxa85-uxa85uxaa5uxab5-3-uxa97uxaa3}

\textbf{બસ ટોપોલોજી દોરો અને સમજાવો.}

\begin{solutionbox}
બસ ટોપોલોજી એ નેટવર્ક કોન્ફિગરેશન છે જ્યાં બધા ડિવાઇસ એક સિંગલ
કોમ્યુનિકેશન લાઇન સાથે જોડાયેલા હોય છે.

\textbf{આકૃતિ:}

\begin{lstlisting}
        BUS TOPOLOGY
+-----+    +-----+    +-----+    +-----+
|Node1|====|Node2|====|Node3|====|Node4|
+-----+    +-----+    +-----+    +-----+
                |
             +-----+
             |Node5|
             +-----+
\end{lstlisting}

{\def\LTcaptype{none} % do not increment counter
\begin{longtable}[]{@{}ll@{}}
\toprule\noalign{}
ફાયદા & ગેરફાયદા \\
\midrule\noalign{}
\endhead
\bottomrule\noalign{}
\endlastfoot
સરળ લેઆઉટ & સિંગલ પોઇન્ટ ઓફ ફેલ્યોર (મુખ્ય કેબલ) \\
ઓછું કેબલિંગ & મર્યાદિત કેબલ લંબાઈ \\
ઓછી કિંમત & વધુ નોડ્સ સાથે પરફોર્મન્સ ઘટે છે \\
સરળતાથી વિસ્તારી શકાય & ટ્રબલશૂટિંગ મુશ્કેલ \\
\end{longtable}
}

\begin{itemize}
\tightlist
\item
  \textbf{ઓપરેશન}: ડેટા બંને દિશામાં બસ પર પ્રવાસ કરે છે
\item
  \textbf{ટર્મિનેટર}: સિગ્નલ રિફ્લેક્શન રોકવા માટે બંને છેડે જરૂરી
\item
  \textbf{ઉપયોગ}: મુખ્યત્વે જૂના નેટવર્ક, નાના સેટઅપમાં
\end{itemize}

\end{solutionbox}
\begin{mnemonicbox}
``SLUE'' - ``સિમ્પલ લેઆઉટ, યુઝીસ લેસ કેબલ, ઇઝી ઇન્સ્ટોલેશન''

\end{mnemonicbox}
\subsection*{પ્રશ્ન 2(બ) અથવા [4
ગુણ]}\label{uxaaauxab0uxab6uxaa8-2uxaac-uxa85uxaa5uxab5-4-uxa97uxaa3}

\textbf{આર્કિટેક્ચર અન્વયે નેટવર્ક ક્લાસીફીકેશન સમજાવો.}

\begin{solutionbox}
આર્કિટેક્ચરના આધારે નેટવર્ક્સને વર્ગીકૃત કરી શકાય છે જે ડિવાઇસના
ઇન્ટરેક્શનની રીત વ્યાખ્યાયિત કરે છે.

{\def\LTcaptype{none} % do not increment counter
\begin{longtable}[]{@{}
  >{\raggedright\arraybackslash}p{(\linewidth - 4\tabcolsep) * \real{0.3500}}
  >{\raggedright\arraybackslash}p{(\linewidth - 4\tabcolsep) * \real{0.4250}}
  >{\raggedright\arraybackslash}p{(\linewidth - 4\tabcolsep) * \real{0.2250}}@{}}
\toprule\noalign{}
\begin{minipage}[b]{\linewidth}\raggedright
આર્કિટેક્ચર
\end{minipage} & \begin{minipage}[b]{\linewidth}\raggedright
લાક્ષણિકતાઓ
\end{minipage} & \begin{minipage}[b]{\linewidth}\raggedright
ઉદાહરણ
\end{minipage} \\
\midrule\noalign{}
\endhead
\bottomrule\noalign{}
\endlastfoot
પીઅર-ટુ-પીઅર & સમાન અધિકારો, કોઈ ડેડિકેટેડ સર્વર નહીં & હોમ નેટવર્ક, નાના
વર્કગ્રુપ \\
ક્લાયન્ટ-સર્વર & સેન્ટ્રલાઇઝ્ડ સર્વિસીસ, ડેડિકેટેડ સર્વર & એન્ટરપ્રાઇઝ નેટવર્ક, વેબ
સર્વિસીસ \\
થ્રી-ટાયર & પ્રેઝન્ટેશન, એપ્લિકેશન, અને ડેટા ટાયર્સ & મોડર્ન વેબ એપ્લિકેશન \\
N-ટાયર & મલ્ટિપલ સ્પેશિયલાઇઝ્ડ ટાયર્સ & લાર્જ ડિસ્ટ્રિબ્યુટેડ સિસ્ટમ \\
\end{longtable}
}

\textbf{આકૃતિ:}

\begin{lstlisting}
PEER-TO-PEER:               CLIENT-SERVER:
  +----+     +----+           +------+
  |Node|-----|Node|           |Client|
  +----+\   /+----+           +------+
         \ /                      |
          X                   +------+
         / \                  |Server|
  +----+/   \+----+           +------+
  |Node|-----|Node|
  +----+     +----+
\end{lstlisting}

\begin{itemize}
\tightlist
\item
  \textbf{પીઅર-ટુ-પીઅર}: ડાયરેક્ટ ડિવાઇસ કોમ્યુનિકેશન, ડિસ્ટ્રિબ્યુટેડ રિસોર્સિસ
\item
  \textbf{ક્લાયન્ટ-સર્વર}: સેન્ટ્રલાઇઝ્ડ રિસોર્સ મેનેજમેન્ટ, બેટર સિક્યોરિટી
\item
  \textbf{હાઇબ્રિડ}: બંને આર્કિટેક્ચરના તત્વોનું સંયોજન
\end{itemize}

\end{solutionbox}
\begin{mnemonicbox}
``PCAN'' - ``પીઅર-ટુ-પીઅર, ક્લાયન્ટ-સર્વર, આર્કિટેક્ચર
નેટવર્ક્સ''

\end{mnemonicbox}
\subsection*{પ્રશ્ન 2(ક) અથવા [7
ગુણ]}\label{uxaaauxab0uxab6uxaa8-2uxa95-uxa85uxaa5uxab5-7-uxa97uxaa3}

\textbf{IP એડ્રેસનું ક્લાસીફીકેશન સમજાવો.}

\begin{solutionbox}
IP એડ્રેસને તેમની સ્ટ્રક્ચર અને હેતુના આધારે વિવિધ કેટેગરીમાં વર્ગીકૃત
કરવામાં આવે છે.

{\def\LTcaptype{none} % do not increment counter
\begin{longtable}[]{@{}
  >{\raggedright\arraybackslash}p{(\linewidth - 8\tabcolsep) * \real{0.2568}}
  >{\raggedright\arraybackslash}p{(\linewidth - 8\tabcolsep) * \real{0.0946}}
  >{\raggedright\arraybackslash}p{(\linewidth - 8\tabcolsep) * \real{0.1892}}
  >{\raggedright\arraybackslash}p{(\linewidth - 8\tabcolsep) * \real{0.2568}}
  >{\raggedright\arraybackslash}p{(\linewidth - 8\tabcolsep) * \real{0.2027}}@{}}
\toprule\noalign{}
\begin{minipage}[b]{\linewidth}\raggedright
IP ક્લાસિફિકેશન
\end{minipage} & \begin{minipage}[b]{\linewidth}\raggedright
રેન્જ
\end{minipage} & \begin{minipage}[b]{\linewidth}\raggedright
ડિફોલ્ટ માસ્ક
\end{minipage} & \begin{minipage}[b]{\linewidth}\raggedright
ઉપલબ્ધ નેટવર્ક્સ
\end{minipage} & \begin{minipage}[b]{\linewidth}\raggedright
હોસ્ટ્સ/નેટવર્ક
\end{minipage} \\
\midrule\noalign{}
\endhead
\bottomrule\noalign{}
\endlastfoot
ક્લાસ A & 1.0.0.0 - 127.255.255.255 & 255.0.0.0 (/8) & 126 &
16,777,214 \\
ક્લાસ B & 128.0.0.0 - 191.255.255.255 & 255.255.0.0 (/16) & 16,384 &
65,534 \\
ક્લાસ C & 192.0.0.0 - 223.255.255.255 & 255.255.255.0 (/24) & 2,097,152 &
254 \\
ક્લાસ D (મલ્ટિકાસ્ટ) & 224.0.0.0 - 239.255.255.255 & N/A & N/A & N/A \\
ક્લાસ E (રિઝર્વ્ડ) & 240.0.0.0 - 255.255.255.255 & N/A & N/A & N/A \\
\end{longtable}
}

\textbf{સ્પેશ્યલ IP રેન્જીસ:}

\begin{itemize}
\tightlist
\item
  \textbf{પ્રાઇવેટ IPs}: 10.0.0.0/8, 172.16.0.0/12, 192.168.0.0/16
\item
  \textbf{લૂપબેક}: 127.0.0.0/8 (સામાન્ય રીતે 127.0.0.1)
\item
  \textbf{લિંક-લોકલ}: 169.254.0.0/16
\end{itemize}

\textbf{આકૃતિ:}

\begin{lstlisting}
CLASS A: |0|NETWORK(7 bits)|      HOST(24 bits)       |
CLASS B: |10|  NETWORK(14 bits)   |    HOST(16 bits)   |
CLASS C: |110| NETWORK(21 bits)        |  HOST(8 bits) |
CLASS D: |1110|       MULTICAST ADDRESS(28 bits)       |
CLASS E: |1111|       RESERVED ADDRESS(28 bits)        |
\end{lstlisting}

\begin{itemize}
\tightlist
\item
  \textbf{ક્લાસફુલ એડ્રેસિંગ}: મૂળ IP એડ્રેસ ક્લાસિફિકેશન સ્કીમ
\item
  \textbf{CIDR (ક્લાસલેસ)}: ફ્લેક્સિબલ સબનેટ માસ્ક આપતી આધુનિક અભિગમ
\item
  \textbf{IPv4 vs IPv6}: IPv4 32-બિટ એડ્રેસ વાપરે છે, IPv6 128-બિટ એડ્રેસ વાપરે
  છે
\end{itemize}

\end{solutionbox}
\begin{mnemonicbox}
``ABCDE'' - ``એડ્રેસ બ્લોક્સ કેટેગરાઇઝ્ડ બાય ડિક્રીઝિંગ
એન્ડ-હોસ્ટ કાઉન્ટ્સ''

\end{mnemonicbox}
\subsection*{પ્રશ્ન 3(અ) [3
ગુણ]}\label{uxaaauxab0uxab6uxaa8-3uxa85-3-uxa97uxaa3}

\textbf{LANનું આખું નામ શું છે? LAN વિગતવાર સમજાવો.}

\begin{solutionbox}
LAN એટલે Local Area Network, એક મર્યાદિત ભૌગોલિક વિસ્તારમાં
સીમિત નેટવર્ક.

\textbf{આકૃતિ:}

\begin{lstlisting}
              LOCAL AREA NETWORK
   +--------+     +--------+     +--------+
   |Computer|-----|  Switch|-----|Computer|
   +--------+     +--------+     +--------+
                      |
                 +--------+     +--------+
                 |Printer |-----|Computer|
                 +--------+     +--------+
\end{lstlisting}

{\def\LTcaptype{none} % do not increment counter
\begin{longtable}[]{@{}ll@{}}
\toprule\noalign{}
LAN લાક્ષણિકતાઓ & વર્ણન \\
\midrule\noalign{}
\endhead
\bottomrule\noalign{}
\endlastfoot
ભૌગોલિક સ્કોપ & બિલ્ડિંગ, કેમ્પસ, અથવા નાનો વિસ્તાર (1-2 km) \\
ડેટા રેટ & ઉચ્ચ (10 Mbps થી 10 Gbps) \\
માલિકી & એક સંસ્થા અથવા વ્યક્તિ \\
ટેકનોલોજી & Ethernet, WiFi, Token Ring \\
મીડિયા & ટ્વિસ્ટેડ પેર, ફાયબર ઓપ્ટિક, વાયરલેસ \\
\end{longtable}
}

\begin{itemize}
\tightlist
\item
  \textbf{હેતુ}: રિસોર્સ શેરિંગ માટે નજીકના ડિવાઇસ કનેક્ટ કરવા
\item
  \textbf{વહીવટ}: મોટા નેટવર્ક કરતાં સરળ મેનેજમેન્ટ
\item
  \textbf{અનુપ્રયોગો}: ઓફિસ નેટવર્કિંગ, હોમ નેટવર્કિંગ
\end{itemize}

\end{solutionbox}
\begin{mnemonicbox}
``LOCAL'' - ``લિમિટેડ ઇન રેન્જ, ઓન્ડ બાય વન એન્ટિટી,
કનેક્ટેડ ડિવાઇસિસ, એક્સેસ કંટ્રોલ, લો લેટન્સી''

\end{mnemonicbox}
\subsection*{પ્રશ્ન 3(બ) [4
ગુણ]}\label{uxaaauxab0uxab6uxaa8-3uxaac-4-uxa97uxaa3}

\textbf{રીપીટર પર ટૂંકનોંધ લખો.}

\begin{solutionbox}
રિપીટર એ નેટવર્ક ડિવાઇસ છે જે નેટવર્ક રેન્જ વધારવા માટે સિગ્નલ્સને
એમ્પ્લિફાય અને રિજનરેટ કરે છે.

\textbf{આકૃતિ:}

\begin{lstlisting}
 Signal              Signal
 weakens             restored
    |                   |
    v                   v
+-------+  Weak   +----------+  Strong  +-------+
|Network|-------->| Repeater |--------->|Network|
|Segment|  Signal +----------+  Signal  |Segment|
+-------+                               +-------+
\end{lstlisting}

{\def\LTcaptype{none} % do not increment counter
\begin{longtable}[]{@{}ll@{}}
\toprule\noalign{}
ફીચર & વર્ણન \\
\midrule\noalign{}
\endhead
\bottomrule\noalign{}
\endlastfoot
OSI લેયર & ફિઝિકલ લેયર (લેયર 1) \\
ફંક્શન & સિગ્નલ રિજનરેશન અને એમ્પ્લિફિકેશન \\
હેતુ & નેટવર્ક ટ્રાન્સમિશન અંતર વધારવું \\
મર્યાદા & ટ્રાફિક ફિલ્ટર કરી શકતા નથી અથવા અલગ નેટવર્ક જોડી શકતા નથી \\
\end{longtable}
}

\begin{itemize}
\tightlist
\item
  \textbf{ઓપરેશન}: સિગ્નલ્સ રિસીવ, રિજનરેટ, અને રિટ્રાન્સમિટ કરે છે
\item
  \textbf{ઉપયોગ}: સામાન્ય મર્યાદાઓથી વધુ કેબલ લંબાઈ વધારવા
\item
  \textbf{પ્રકારો}: ટ્રેડિશનલ રિપીટર્સ, હબ્સ (મલ્ટિપોર્ટ રિપીટર્સ)
\end{itemize}

\end{solutionbox}
\begin{mnemonicbox}
``RARE'' - ``રિપીટર્સ એમ્પ્લિફાઇ એન્ડ રિજનરેટ ઇલેક્ટ્રિકલ
સિગ્નલ્સ''

\end{mnemonicbox}
\subsection*{પ્રશ્ન 3(ક) [7
ગુણ]}\label{uxaaauxab0uxab6uxaa8-3uxa95-7-uxa97uxaa3}

\textbf{ટૂંકનોંધ લખો: FTP}

\begin{solutionbox}
ફાઇલ ટ્રાન્સફર પ્રોટોકોલ (FTP) એ ક્લાયન્ટ અને સર્વર વચ્ચે ફાઇલ
ટ્રાન્સફર માટેનો સ્ટાન્ડર્ડ નેટવર્ક પ્રોટોકોલ છે.

\textbf{આકૃતિ:}

\begin{lstlisting}
                     Control Connection (Port 21)
               +--------------------------------+
               |                                |
     +--------+|                                |+--------+
     |        ||                                ||        |
     | CLIENT |+--------------------------------+| SERVER |
     |        |                                  |        |
     |        |                                  |        |
     +--------+                                  +--------+
               +--------------------------------+
                     Data Connection (Port 20)
\end{lstlisting}

{\def\LTcaptype{none} % do not increment counter
\begin{longtable}[]{@{}ll@{}}
\toprule\noalign{}
ફીચર & વર્ણન \\
\midrule\noalign{}
\endhead
\bottomrule\noalign{}
\endlastfoot
પોર્ટ & કંટ્રોલ: 21, ડેટા: 20 \\
મોડ & એક્ટિવ અથવા પેસિવ \\
ઓથેન્ટિકેશન & યુઝરનેમ/પાસવર્ડ (અથવા એનોનિમસ) \\
ટ્રાન્સફર ટાઇપ્સ & ASCII (ટેક્સ્ટ) અથવા બાઇનરી (રૉ ડેટા) \\
સિક્યુરિટી & બેઝિક FTP (અનસિક્યોર્ડ), FTPS, SFTP (સિક્યોર વેરિઅન્ટ્સ) \\
\end{longtable}
}

\begin{itemize}
\tightlist
\item
  \textbf{ડ્યુઅલ ચેનલ}: અલગ કંટ્રોલ અને ડેટા કનેક્શન
\item
  \textbf{કમાન્ડ્સ}: GET, PUT, LIST, DELETE, RENAME, વગેરે
\item
  \textbf{યુઝર ઓથેન્ટિકેશન}: લોગિન ક્રેડેન્શિયલ્સની આવશ્યકતા
\end{itemize}

\end{solutionbox}
\begin{mnemonicbox}
``CDATA'' - ``કંટ્રોલ ચેનલ, ડેટા ચેનલ, એક્ટિવ/પેસિવ મોડ્સ,
ટ્રાન્સફર ટાઇપ્સ, ઓથેન્ટિકેશન''

\end{mnemonicbox}
\subsection*{પ્રશ્ન 3(અ) અથવા [3
ગુણ]}\label{uxaaauxab0uxab6uxaa8-3uxa85-uxa85uxaa5uxab5-3-uxa97uxaa3}

\textbf{PANનું આખું નામ શું છે? PAN વિગતવાર સમજાવો.}

\begin{solutionbox}
PAN એટલે Personal Area Network, વ્યક્તિની આસપાસ કેન્દ્રિત
ડિવાઇસ કનેક્ટ કરવા માટેનું નેટવર્ક.

\textbf{આકૃતિ:}

\begin{lstlisting}
              PERSONAL AREA NETWORK
                    +------+
                    |Person|
                    +------+
                       |
          +------------+------------+
          |            |            |
      +---------+   +----------+   +--------+
      |Smartphone|  |Smartwatch|   |Laptop  |
      +---------+   +----------+   +--------+
          |
      +--------+
      |Earbuds |
      +--------+
\end{lstlisting}

{\def\LTcaptype{none} % do not increment counter
\begin{longtable}[]{@{}ll@{}}
\toprule\noalign{}
PAN લાક્ષણિકતાઓ & વર્ણન \\
\midrule\noalign{}
\endhead
\bottomrule\noalign{}
\endlastfoot
ભૌગોલિક સ્કોપ & ખૂબ નાનો (1-10 મીટર) \\
ડેટા રેટ & લો થી મિડિયમ (100 Kbps - 100 Mbps) \\
માલિકી & વ્યક્તિગત વ્યક્તિ \\
ટેકનોલોજી & Bluetooth, Zigbee, NFC, Infrared \\
ડિવાઇસિસ & વ્યક્તિગત ડિવાઇસ (ફોન, વેરેબલ્સ, લેપટોપ) \\
\end{longtable}
}

\begin{itemize}
\tightlist
\item
  \textbf{હેતુ}: કોમ્યુનિકેશન/ડેટા શેરિંગ માટે વ્યક્તિગત ડિવાઇસ કનેક્ટ કરવા
\item
  \textbf{પ્રકારો}: વાયર્ડ PAN (USB) અને વાયરલેસ PAN (Bluetooth)
\item
  \textbf{અનુપ્રયોગો}: ડેટા સિન્ક્રોનાઇઝેશન, ઓડિયો સ્ટ્રીમિંગ, હેલ્થ મોનિટરિંગ
\end{itemize}

\end{solutionbox}
\begin{mnemonicbox}
``PIPER'' - ``પર્સનલ, ઇન્ડિવિજ્યુઅલ, પ્રોક્સિમિટી, ઇઝી
સેટઅપ, રિડ્યુસ્ડ રેન્જ''

\end{mnemonicbox}
\subsection*{પ્રશ્ન 3(બ) અથવા [4
ગુણ]}\label{uxaaauxab0uxab6uxaa8-3uxaac-uxa85uxaa5uxab5-4-uxa97uxaa3}

\textbf{બ્રિજનું મહત્વ શું છે? બ્રિજ પર ટૂંકનોંધ લખો.}

\begin{solutionbox}
બ્રિજ એ નેટવર્ક ડિવાઇસ છે જે નેટવર્ક સેગમેન્ટ્સને કનેક્ટ અને ફિલ્ટર કરે છે.

\textbf{આકૃતિ:}

\begin{lstlisting}
   SEGMENT A                SEGMENT B
+-------------+          +-------------+
|             |          |             |
|  +------+   |          |  +------+   |
|  |Device|   |          |  |Device|   |
|  +------+   |          |  +------+   |
|      |      |          |      |      |
|      |      |   +---------+   |      |
|      +------|---| BRIDGE |----+      |
|             |   +---------+          |
+-------------+          +-------------+
\end{lstlisting}

{\def\LTcaptype{none} % do not increment counter
\begin{longtable}[]{@{}ll@{}}
\toprule\noalign{}
ફીચર & વર્ણન \\
\midrule\noalign{}
\endhead
\bottomrule\noalign{}
\endlastfoot
OSI લેયર & ડેટા લિંક લેયર (લેયર 2) \\
ફંક્શન & સમાન નેટવર્ક સેગમેન્ટ્સ કનેક્ટ કરવા \\
ઇન્ટેલિજન્સ & MAC એડ્રેસનો ઉપયોગ કરીને ટ્રાફિક ફિલ્ટર કરે છે \\
ફાયદો & સેગમેન્ટ્સ વચ્ચે બિનજરૂરી ટ્રાફિક ઘટાડે છે \\
\end{longtable}
}

\begin{itemize}
\tightlist
\item
  \textbf{મહત્વ}: નેટવર્ક વિસ્તારે છે, કોલિઝન ડોમેન ઘટાડે છે
\item
  \textbf{ઓપરેશન}: MAC એડ્રેસ શીખે છે, ફ્રેમ્સ સિલેક્ટિવલી ફોરવર્ડ કરે છે
\item
  \textbf{પ્રકારો}: ટ્રાન્સપેરન્ટ, ટ્રાન્સલેશનલ, સોર્સ-રૂટ બ્રિજીસ
\end{itemize}

\end{solutionbox}
\begin{mnemonicbox}
``SELF'' - ``સેગમેન્ટેશન, એક્સટેન્શન, લર્નિંગ એડ્રેસિસ, ફિલ્ટરિંગ
ટ્રાફિક''

\end{mnemonicbox}
\subsection*{પ્રશ્ન 3(ક) અથવા [7
ગુણ]}\label{uxaaauxab0uxab6uxaa8-3uxa95-uxa85uxaa5uxab5-7-uxa97uxaa3}

\textbf{DSL શું છે? તેનાં જુદા-જુદા પ્રકાર સમજાવો.}

\begin{solutionbox}
ડિજિટલ સબસ્ક્રાઇબર લાઇન (DSL) એ ટેલિફોન લાઇન્સ પર ડિજિટલ ડેટા
ટ્રાન્સમિશન પ્રદાન કરતી ટેકનોલોજીઓનો પરિવાર છે.

\textbf{આકૃતિ:}

\begin{lstlisting}
                           +-------+
        +--------+         |       |
HOME----|  DSL   |---------| DSLAM |-------INTERNET
        | MODEM  |  Copper |       |
        +--------+   Line  +-------+
                    (POTS)    ISP
\end{lstlisting}

{\def\LTcaptype{none} % do not increment counter
\begin{longtable}[]{@{}
  >{\raggedright\arraybackslash}p{(\linewidth - 8\tabcolsep) * \real{0.1724}}
  >{\raggedright\arraybackslash}p{(\linewidth - 8\tabcolsep) * \real{0.1379}}
  >{\raggedright\arraybackslash}p{(\linewidth - 8\tabcolsep) * \real{0.2931}}
  >{\raggedright\arraybackslash}p{(\linewidth - 8\tabcolsep) * \real{0.1724}}
  >{\raggedright\arraybackslash}p{(\linewidth - 8\tabcolsep) * \real{0.2241}}@{}}
\toprule\noalign{}
\begin{minipage}[b]{\linewidth}\raggedright
DSL ટાઇપ
\end{minipage} & \begin{minipage}[b]{\linewidth}\raggedright
પૂરું નામ
\end{minipage} & \begin{minipage}[b]{\linewidth}\raggedright
સ્પીડ (ડાઉન/અપ)
\end{minipage} & \begin{minipage}[b]{\linewidth}\raggedright
ડિસ્ટન્સ
\end{minipage} & \begin{minipage}[b]{\linewidth}\raggedright
અનુપ્રયોગ
\end{minipage} \\
\midrule\noalign{}
\endhead
\bottomrule\noalign{}
\endlastfoot
ADSL & અસિમેટ્રિક DSL & 8 Mbps/1 Mbps & 5.5 km સુધી & રેસિડેન્શિયલ ઇન્ટરનેટ \\
SDSL & સિમેટ્રિક DSL & 2 Mbps/2 Mbps & 3 km સુધી & સ્મોલ બિઝનેસ \\
VDSL & વેરી હાઇ-બિટ-રેટ DSL & 52-85 Mbps/16-85 Mbps & 1.2 km સુધી & વિડિયો
સ્ટ્રીમિંગ, બિઝનેસ \\
HDSL & હાઇ-બિટ-રેટ DSL & 2 Mbps/2 Mbps & 3.6 km સુધી & T1/E1 રિપ્લેસમેન્ટ \\
IDSL & ISDN DSL & 144 Kbps/144 Kbps & 5.5 km સુધી & ISDN ઓલ્ટરનેટિવ \\
\end{longtable}
}

\begin{itemize}
\tightlist
\item
  \textbf{કાર્યપ્રણાલી}: ફોન લાઇન્સ પર વપરાયેલા ફ્રિક્વન્સી સ્પેક્ટ્રમનો ઉપયોગ કરે છે
\item
  \textbf{ફાયદો}: અસ્તિત્વમાં રહેલા ટેલિફોન ઇન્ફ્રાસ્ટ્રક્ચરનો ઉપયોગ કરે છે
\item
  \textbf{ઓલવેઝ-ઓન}: ડાયલ-અપ વગર સતત કનેક્શન
\end{itemize}

\end{solutionbox}
\begin{mnemonicbox}
``SAVHI'' - ``સિમેટ્રિક, અસિમેટ્રિક, વેરી હાઇ-બિટ-રેટ,
હાઇ-બિટ-રેટ, ISDN DSL''

\end{mnemonicbox}
\subsection*{પ્રશ્ન 4(અ) [3
ગુણ]}\label{uxaaauxab0uxab6uxaa8-4uxa85-3-uxa97uxaa3}

\textbf{ડેટા લિંક લેયર માટે એરર કન્ટ્રોલ અને ફ્લો કન્ટ્રોલ સમજાવો.}

\begin{solutionbox}
એરર અને ફ્લો કંટ્રોલ એ ડેટા લિંક લેયરના આવશ્યક કાર્યો છે જે વિશ્વસનીય
ડેટા ટ્રાન્સમિશન સુનિશ્ચિત કરે છે.

{\def\LTcaptype{none} % do not increment counter
\begin{longtable}[]{@{}
  >{\raggedright\arraybackslash}p{(\linewidth - 4\tabcolsep) * \real{0.3571}}
  >{\raggedright\arraybackslash}p{(\linewidth - 4\tabcolsep) * \real{0.2143}}
  >{\raggedright\arraybackslash}p{(\linewidth - 4\tabcolsep) * \real{0.4286}}@{}}
\toprule\noalign{}
\begin{minipage}[b]{\linewidth}\raggedright
મેકેનિઝમ
\end{minipage} & \begin{minipage}[b]{\linewidth}\raggedright
હેતુ
\end{minipage} & \begin{minipage}[b]{\linewidth}\raggedright
ટેકનિક્સ
\end{minipage} \\
\midrule\noalign{}
\endhead
\bottomrule\noalign{}
\endlastfoot
એરર કંટ્રોલ & ટ્રાન્સમિશન એરર ડિટેક્ટ/કરેક્ટ કરવા & CRC, ચેકસમ, પેરિટી બિટ્સ \\
ફ્લો કંટ્રોલ & સેન્ડર દ્વારા રિસીવરને ઓવરવ્હેલમ થતું રોકવા & સ્ટોપ-એન્ડ-વેઇટ, સ્લાઇડિંગ
વિન્ડો \\
\end{longtable}
}

\textbf{આકૃતિ:}

\begin{lstlisting}
ERROR CONTROL:
  +------+  DATA   +-------+  ACK/NAK +--------+
  |Sender|-------->|Channel|--------->|Receiver|
  +------+         +-------+          +--------+

FLOW CONTROL:
  +------+  DATA   +--------+
  |Sender|-------->|Receiver|
  +------+  STOP   +--------+
         <---------
\end{lstlisting}

\begin{itemize}
\tightlist
\item
  \textbf{એરર ડિટેક્શન}: CRC, ચેકસમ દ્વારા કરપ્ટેડ ફ્રેમ્સ ઓળખવા
\item
  \textbf{એરર કરેક્શન}: ફોરવર્ડ એરર કરેક્શન (FEC), રિટ્રાન્સમિશન
\item
  \textbf{ફ્લો કંટ્રોલ}: રિસીવરમાં બફર ઓવરફ્લો રોકે છે
\end{itemize}

\end{solutionbox}
\begin{mnemonicbox}
``SAFE'' - ``સ્ટોપ-એન્ડ-વેઇટ, એકનોલેજમેન્ટ, ફ્લો કંટ્રોલ, એરર
ડિટેક્શન''

\end{mnemonicbox}
\subsection*{પ્રશ્ન 4(બ) [4
ગુણ]}\label{uxaaauxab0uxab6uxaa8-4uxaac-4-uxa97uxaa3}

\textbf{ફાયરવોલ શું છે? વિગતવાર સમજાવો.}

\begin{solutionbox}
ફાયરવોલ એ નેટવર્ક સિક્યોરિટી ડિવાઇસ છે જે ઇનકમિંગ અને આઉટગોઇંગ
નેટવર્ક ટ્રાફિકનું મોનિટરિંગ અને ફિલ્ટરિંગ કરે છે.

\textbf{આકૃતિ:}

\begin{lstlisting}
   INTERNAL NETWORK                  INTERNET
+-------------------+             +--------------+
|                   |  FIREWALL   |              |
|  +-----+  +-----+ |  +------+   |  +--------+  |
|  |Host1|  |Host2| |  |      |   |  |External|  |
|  +-----+  +-----+ |--|FILTER|---|  |Server  |  |
|                   |  |      |   |  +--------+  |
|  +-----+  +-----+ |  +------+   |              |
|  |Host3|  |Host4| |             |              |
|  +-----+  +-----+ |             |              |
+-------------------+             +--------------+
\end{lstlisting}

{\def\LTcaptype{none} % do not increment counter
\begin{longtable}[]{@{}lll@{}}
\toprule\noalign{}
ફાયરવોલ ટાઇપ & ફંક્શનાલિટી & ઉદાહરણ \\
\midrule\noalign{}
\endhead
\bottomrule\noalign{}
\endlastfoot
પેકેટ ફિલ્ટરિંગ & પેકેટ હેડર્સ તપાસે છે & રાઉટર ACLs \\
સ્ટેટફુલ ઇન્સ્પેક્શન & કનેક્શન સ્ટેટ ટ્રેક કરે છે & મોટાભાગના હાર્ડવેર ફાયરવોલ \\
એપ્લિકેશન લેયર & કન્ટેન્ટ ઇન્સ્પેક્ટ કરે છે & વેબ એપ્લિકેશન ફાયરવોલ \\
નેક્સ્ટ-જનરેશન & મલ્ટિપલ ટેકનોલોજીનું સંયોજન & પાલો આલ્ટો, ફોર્ટિનેટ \\
\end{longtable}
}

\begin{itemize}
\tightlist
\item
  \textbf{હેતુ}: અનધિકૃત એક્સેસથી નેટવર્ક સુરક્ષિત કરે છે
\item
  \textbf{ઇમ્પ્લિમેન્ટેશન}: હાર્ડવેર, સોફ્ટવેર, અથવા ક્લાઉડ-બેઝ્ડ
\item
  \textbf{સિક્યોરિટી પોલિસી}: મંજૂર/બ્લોક્ડ ટ્રાફિક નિર્ધારિત કરતા નિયમો
\end{itemize}

\end{solutionbox}
\begin{mnemonicbox}
``PAPSI'' - ``પેકેટ ફિલ્ટરિંગ, એપ્લિકેશન લેયર, પોલિસીઝ,
સ્ટેટફુલ ઇન્સ્પેક્શન''

\end{mnemonicbox}
\subsection*{પ્રશ્ન 4(ક) [7
ગુણ]}\label{uxaaauxab0uxab6uxaa8-4uxa95-7-uxa97uxaa3}

\textbf{IPV4 અને IPV6ને સરખાવો.}

\begin{solutionbox}
IPv4 અને IPv6 એ ઇન્ટરનેટ પ્રોટોકોલ વર્ઝન્સ છે જેમાં એડ્રેસિંગ અને
કેપેબિલિટીમાં નોંધપાત્ર તફાવત છે.

{\def\LTcaptype{none} % do not increment counter
\begin{longtable}[]{@{}
  >{\raggedright\arraybackslash}p{(\linewidth - 4\tabcolsep) * \real{0.4286}}
  >{\raggedright\arraybackslash}p{(\linewidth - 4\tabcolsep) * \real{0.2857}}
  >{\raggedright\arraybackslash}p{(\linewidth - 4\tabcolsep) * \real{0.2857}}@{}}
\toprule\noalign{}
\begin{minipage}[b]{\linewidth}\raggedright
ફીચર
\end{minipage} & \begin{minipage}[b]{\linewidth}\raggedright
IPv4
\end{minipage} & \begin{minipage}[b]{\linewidth}\raggedright
IPv6
\end{minipage} \\
\midrule\noalign{}
\endhead
\bottomrule\noalign{}
\endlastfoot
એડ્રેસ સાઇઝ & 32-બિટ (4 બાઇટ્સ) & 128-બિટ (16 બાઇટ્સ) \\
ફોર્મેટ & ડોટેડ ડેસિમલ (192.168.1.1) & હેક્સાડેસિમલ વિથ કોલન
(2001:0db8:85a3::8a2e:0370:7334) \\
એડ્રેસ સ્પેસ & \textasciitilde4.3 બિલિયન એડ્રેસ & 340 અંડેસિલિયન એડ્રેસ \\
હેડર & વેરિએબલ લેન્થ (20-60 બાઇટ્સ) & ફિક્સ્ડ લેન્થ (40 બાઇટ્સ) \\
ફ્રેગમેન્ટેશન & રાઉટર્સ અને સેન્ડિંગ હોસ્ટ્સ & માત્ર સેન્ડિંગ હોસ્ટ્સ \\
ચેકસમ & હેડરમાં સમાવિષ્ટ & હેડરમાંથી દૂર કરાયું \\
સિક્યોરિટી & બિલ્ટ-ઇન નથી (IPsec ઓપ્શનલ) & બિલ્ટ-ઇન IPsec સપોર્ટ \\
\end{longtable}
}

\textbf{આકૃતિ:}

\begin{lstlisting}
IPv4: |VER|IHL|DSCP|ECN|  TOTAL LENGTH   |
      |  IDENTIFICATION   |FLAGS|FRAGMENT|
      |TTL |PROTOCOL|  HEADER CHECKSUM   |
      |        SOURCE ADDRESS            |
      |      DESTINATION ADDRESS         |
      |          OPTIONS...              |

IPv6: |VER|TRAFFIC CLASS|     FLOW LABEL      |
      |   PAYLOAD LENGTH   |NEXT HDR|HOP LIMIT|
      |                                       |
      |           SOURCE ADDRESS              |
      |                                       |
      |                                       |
      |                                       |
      |          DESTINATION ADDRESS          |
      |                                       |
\end{lstlisting}

\begin{itemize}
\tightlist
\item
  \textbf{ઓટો-કોન્ફિગરેશન}: IPv6માં સ્ટેટલેસ એડ્રેસ ઓટો-કોન્ફિગરેશન છે
\item
  \textbf{NAT}: મોટા એડ્રેસ સ્પેસને કારણે IPv6માં જરૂરી નથી
\item
  \textbf{ટ્રાન્ઝિશન}: ડ્યુઅલ-સ્ટેક, ટનલિંગ, ટ્રાન્સલેશન મેકેનિઝમ્સ
\item
  \textbf{હેડર એફિશિયન્સી}: IPv6માં બેટર પરફોર્મન્સ માટે સ્ટ્રીમલાઇન્ડ હેડર છે
\end{itemize}

\end{solutionbox}
\begin{mnemonicbox}
``SHAPE'' - ``સાઇઝ, હેડર, એડ્રેસિંગ, પરફોર્મન્સ,
એક્સટેન્સિબિલિટી''

\end{mnemonicbox}
\subsection*{પ્રશ્ન 4(અ) અથવા [3
ગુણ]}\label{uxaaauxab0uxab6uxaa8-4uxa85-uxa85uxaa5uxab5-3-uxa97uxaa3}

\textbf{IP એડ્રેસ શું છે? તે નેટવર્કમાં કઈ રીતે ઉપયોગી છે?}

\begin{solutionbox}
IP એડ્રેસ એ ન્યુમેરિકલ આઈડેન્ટિફાયર છે જે ઇન્ટરનેટ પ્રોટોકોલનો ઉપયોગ
કરતા નેટવર્કમાં કનેક્ટેડ દરેક ડિવાઇસને અસાઇન કરવામાં આવે છે.

\textbf{આકૃતિ:}

\begin{lstlisting}
IP ADDRESS: 192.168.1.100
 +---+---+---+---+
 |192|168| 1 |100|  <-- Dotted decimal notation
 +---+---+---+---+
  |   |   |   |
  |   |   |   +---- Host identifier
  |   |   +-------- Subnet identifier
  +---+------------- Network identifier
\end{lstlisting}

{\def\LTcaptype{none} % do not increment counter
\begin{longtable}[]{@{}ll@{}}
\toprule\noalign{}
IP એડ્રેસ ઉપયોગ & વર્ણન \\
\midrule\noalign{}
\endhead
\bottomrule\noalign{}
\endlastfoot
આઈડેન્ટિફિકેશન & નેટવર્ક પર ડિવાઇસને અનન્ય રીતે ઓળખે છે \\
રાઉટિંગ & ડેટા પેકેટ્સ માટે પાથ નક્કી કરે છે \\
એડ્રેસિંગ & ચોક્કસ ડેસ્ટિનેશન પર ડેટા મોકલવાની સુવિધા આપે છે \\
નેટવર્ક ડિવિઝન & સબનેટ્સમાં વિભાજન કરવાની મંજૂરી આપે છે \\
\end{longtable}
}

\begin{itemize}
\tightlist
\item
  \textbf{સ્ટ્રક્ચર}: નેટવર્ક પોર્શન અને હોસ્ટ પોર્શન
\item
  \textbf{અસાઇનમેન્ટ}: સ્ટેટિક (મેન્યુઅલ) અથવા ડાયનેમિક (DHCP)
\item
  \textbf{વર્ઝન્સ}: IPv4 (32-બિટ) અને IPv6 (128-બિટ)
\end{itemize}

\end{solutionbox}
\begin{mnemonicbox}
``IRAN'' - ``આઈડેન્ટિફિકેશન, રાઉટિંગ, એડ્રેસિંગ, નેટવર્ક
ડિવિઝન''

\end{mnemonicbox}
\subsection*{પ્રશ્ન 4(બ) અથવા [4
ગુણ]}\label{uxaaauxab0uxab6uxaa8-4uxaac-uxa85uxaa5uxab5-4-uxa97uxaa3}

\textbf{FDDI અને CDDIને સરખાવો.}

\begin{solutionbox}
FDDI (ફાયબર ડિસ્ટ્રિબ્યુટેડ ડેટા ઇન્ટરફેસ) અને CDDI (કોપર
ડિસ્ટ્રિબ્યુટેડ ડેટા ઇન્ટરફેસ) એ હાઈ-સ્પીડ નેટવર્ક ટેકનોલોજીઓ છે.

{\def\LTcaptype{none} % do not increment counter
\begin{longtable}[]{@{}lll@{}}
\toprule\noalign{}
ફીચર & FDDI & CDDI \\
\midrule\noalign{}
\endhead
\bottomrule\noalign{}
\endlastfoot
મીડિયમ & ફાયબર ઓપ્ટિક કેબલ & કોપર ટ્વિસ્ટેડ પેર \\
સ્પીડ & 100 Mbps & 100 Mbps \\
ડિસ્ટન્સ & કુલ 200 km સુધી, સ્ટેશન વચ્ચે 2 km & સ્ટેશન વચ્ચે 100 m સુધી \\
ટોપોલોજી & ડ્યુઅલ કાઉન્ટર-રોટેટિંગ રિંગ્સ & ડ્યુઅલ કાઉન્ટર-રોટેટિંગ રિંગ્સ \\
કોસ્ટ & ઉચ્ચ & ઓછી \\
રિલાયબિલિટી & ખૂબ ઉચ્ચ & મધ્યમ \\
સ્ટાન્ડર્ડ & ANSI X3T9.5 & FDDI જેવું જ (કોપર માટે અડાપ્ટેડ) \\
\end{longtable}
}

\textbf{આકૃતિ:}

\begin{lstlisting}
FDDI/CDDI DUAL RING TOPOLOGY:
      +-----+         +-----+
      |     |         |     |
  +-->|Node1|-------->|Node2|----+
  |   |     |         |     |    |
  |   +-----+         +-----+    |
  |                              |
  |   +-----+         +-----+    |
  +---|Node4|<--------|Node3|<---+
      |     |         |     |
      +-----+         +-----+
\end{lstlisting}

\begin{itemize}
\tightlist
\item
  \textbf{રિડન્ડન્સી}: ફોલ્ટ ટોલરન્સ માટે સેકન્ડરી રિંગ
\item
  \textbf{એક્સેસ મેથડ}: ટાઇમ્ડ ટોકન રોટેશન સાથે ટોકન પાસિંગ
\item
  \textbf{અનુપ્રયોગો}: FDDI બેકબોન્સ માટે, CDDI વર્કસ્ટેશન્સ માટે
\end{itemize}

\end{solutionbox}
\begin{mnemonicbox}
``FDDI ફ્લાઇઝ, CDDI ક્રોલ્સ'' - લાંબા અંતર માટે ફાયબર, ટૂંકા
રન માટે કોપર

\end{mnemonicbox}
\subsection*{પ્રશ્ન 4(ક) અથવા [7
ગુણ]}\label{uxaaauxab0uxab6uxaa8-4uxa95-uxa85uxaa5uxab5-7-uxa97uxaa3}

\textbf{OSI રેફરન્સ મોડેલ દોરો અને વિગતવાર સમજાવો.}

\begin{solutionbox}
OSI (ઓપન સિસ્ટમ્સ ઇન્ટરકનેક્શન) મોડેલ એ નેટવર્ક ફંક્શન્સને સાત લેયરમાં
સ્ટાન્ડર્ડાઇઝ કરતું કન્સેપ્ચ્યુઅલ ફ્રેમવર્ક છે.

\textbf{આકૃતિ:}

\begin{lstlisting}
+-----------------------------+
|         APPLICATION (7)     |
|      User interface, apps   |
+-----------------------------+
|        PRESENTATION (6)     |
|    Data format, encryption  |
+-----------------------------+
|          SESSION (5)        |
|    Connection management    |
+-----------------------------+
|         TRANSPORT (4)       |
|   End-to-end reliability    |
+-----------------------------+
|          NETWORK (3)        |
|   Routing between networks  |
+-----------------------------+
|         DATA LINK (2)       |
|  Node-to-node reliability   |
+-----------------------------+
|          PHYSICAL (1)       |
|   Physical transmission     |
+-----------------------------+
\end{lstlisting}

{\def\LTcaptype{none} % do not increment counter
\begin{longtable}[]{@{}
  >{\raggedright\arraybackslash}p{(\linewidth - 6\tabcolsep) * \real{0.1228}}
  >{\raggedright\arraybackslash}p{(\linewidth - 6\tabcolsep) * \real{0.3158}}
  >{\raggedright\arraybackslash}p{(\linewidth - 6\tabcolsep) * \real{0.3684}}
  >{\raggedright\arraybackslash}p{(\linewidth - 6\tabcolsep) * \real{0.1930}}@{}}
\toprule\noalign{}
\begin{minipage}[b]{\linewidth}\raggedright
લેયર
\end{minipage} & \begin{minipage}[b]{\linewidth}\raggedright
પ્રાથમિક ફંક્શન
\end{minipage} & \begin{minipage}[b]{\linewidth}\raggedright
પ્રોટોકોલ્સ/સ્ટાન્ડર્ડ્સ
\end{minipage} & \begin{minipage}[b]{\linewidth}\raggedright
ડેટા યુનિટ
\end{minipage} \\
\midrule\noalign{}
\endhead
\bottomrule\noalign{}
\endlastfoot
એપ્લિકેશન & યુઝર ઇન્ટરફેસ, નેટવર્ક સર્વિસિસ & HTTP, FTP, SMTP & ડેટા \\
પ્રેઝન્ટેશન & ડેટા ફોર્મેટિંગ, એન્ક્રિપ્શન & SSL/TLS, JPEG, MIME & ડેટા \\
સેશન & કનેક્શન સ્થાપના, મેનેજમેન્ટ & NetBIOS, RPC & ડેટા \\
ટ્રાન્સપોર્ટ & એન્ડ-ટુ-એન્ડ ડિલિવરી, ફ્લો કંટ્રોલ & TCP, UDP & સેગમેન્ટ્સ \\
નેટવર્ક & લોજિકલ એડ્રેસિંગ, રાઉટિંગ & IP, ICMP, OSPF & પેકેટ્સ \\
ડેટા લિંક & ફિઝિકલ એડ્રેસિંગ, મીડિયા એક્સેસ & Ethernet, PPP, HDLC & ફ્રેમ્સ \\
ફિઝિકલ & બિટ ટ્રાન્સમિશન, કેબલિંગ, સિગ્નલિંગ & USB, Ethernet, Bluetooth &
બિટ્સ \\
\end{longtable}
}

\begin{itemize}
\tightlist
\item
  \textbf{લેયર ઇન્ડિપેન્ડન્સ}: દરેક લેયર ચોક્કસ ફંક્શન્સ પરફોર્મ કરે છે
\item
  \textbf{એન્કેપ્સુલેશન}: ડેટા દરેક લેયરમાં હેડર સાથે રેપ થાય છે
\item
  \textbf{સ્ટાન્ડર્ડાઇઝેશન}: સિસ્ટમ્સ વચ્ચે ઇન્ટરઓપરેબિલિટી પ્રમોટ કરે છે
\item
  \textbf{ટ્રબલશૂટિંગ}: પ્રોબ્લેમ્સને ચોક્કસ લેયર્સમાં આઇસોલેટ કરે છે
\end{itemize}

\end{solutionbox}
\begin{mnemonicbox}
``All People Seem To Need Data Processing'' (લેયર 7
થી 1)

\end{mnemonicbox}
\subsection*{પ્રશ્ન 5(અ) [3
ગુણ]}\label{uxaaauxab0uxab6uxaa8-5uxa85-3-uxa97uxaa3}

\textbf{ISO શું છે? ઇન્ફોમેશન સિક્યોરિટીમાં કઈ રીતે કામ કરે છે?}

\begin{solutionbox}
ISO (ઇન્ટરનેશનલ ઓર્ગેનાઇઝેશન ફોર સ્ટાન્ડર્ડાઇઝેશન) ઇન્ફોર્મેશન
સિક્યોરિટી સહિતના સ્ટાન્ડર્ડ્સ વિકસાવે અને પ્રકાશિત કરે છે.

{\def\LTcaptype{none} % do not increment counter
\begin{longtable}[]{@{}ll@{}}
\toprule\noalign{}
ISO સિક્યોરિટી સ્ટાન્ડર્ડ્સ & હેતુ \\
\midrule\noalign{}
\endhead
\bottomrule\noalign{}
\endlastfoot
ISO/IEC 27001 & ઇન્ફોર્મેશન સિક્યોરિટી મેનેજમેન્ટ સિસ્ટમ્સ \\
ISO/IEC 27002 & સિક્યોરિટી કંટ્રોલ્સ માટે કોડ ઓફ પ્રેક્ટિસ \\
ISO/IEC 27005 & ઇન્ફોર્મેશન સિક્યોરિટી રિસ્ક મેનેજમેન્ટ \\
ISO/IEC 27017 & ક્લાઉડ સિક્યોરિટી \\
ISO/IEC 27018 & પર્સનલી આઈડેન્ટિફાયેબલ ઇન્ફોર્મેશનનું પ્રોટેક્શન \\
\end{longtable}
}

\textbf{ઇન્ફોર્મેશન સિક્યોરિટીમાં કાર્ય:}

\begin{itemize}
\tightlist
\item
  \textbf{ફ્રેમવર્ક-બેઝ્ડ}: સિક્યોરિટીના સ્ટ્રક્ચર્ડ અભિગમ પ્રદાન કરે છે
\item
  \textbf{રિસ્ક-બેઝ્ડ}: જોખમોની ઓળખ અને શમન પર ધ્યાન કેન્દ્રિત કરે છે
\item
  \textbf{પ્રોસેસ-ઓરિએન્ટેડ}: સતત સુધારણા ચક્ર સ્થાપિત કરે છે
\item
  \textbf{સર્ટિફિકેશન}: સંસ્થાઓને કમ્પ્લાયન્સ માટે સર્ટિફાઇડ કરી શકાય છે
\end{itemize}

\end{solutionbox}
\begin{mnemonicbox}
``PRIMP'' - ``પોલિસીઝ, રિસ્ક અસેસમેન્ટ, ઇમ્પ્લિમેન્ટેશન,
મોનિટરિંગ, પ્રોસેસ ઇમ્પ્રુવમેન્ટ''

\end{mnemonicbox}
\subsection*{પ્રશ્ન 5(બ) [4
ગુણ]}\label{uxaaauxab0uxab6uxaa8-5uxaac-4-uxa97uxaa3}

\textbf{ક્રિપ્ટોગ્રાફીની ટર્મ વિગતવાર સમજાવો: ૧) એનક્રિપ્શન ૨) ડીક્રિપ્શન}

\begin{solutionbox}
એન્ક્રિપ્શન અને ડિક્રિપ્શન માહિતીને સુરક્ષિત કરતી ક્રિપ્ટોગ્રાફીની
મૂળભૂત પ્રક્રિયાઓ છે.

{\def\LTcaptype{none} % do not increment counter
\begin{longtable}[]{@{}
  >{\raggedright\arraybackslash}p{(\linewidth - 6\tabcolsep) * \real{0.1463}}
  >{\raggedright\arraybackslash}p{(\linewidth - 6\tabcolsep) * \real{0.2195}}
  >{\raggedright\arraybackslash}p{(\linewidth - 6\tabcolsep) * \real{0.1707}}
  >{\raggedright\arraybackslash}p{(\linewidth - 6\tabcolsep) * \real{0.4634}}@{}}
\toprule\noalign{}
\begin{minipage}[b]{\linewidth}\raggedright
ટર્મ
\end{minipage} & \begin{minipage}[b]{\linewidth}\raggedright
વ્યાખ્યા
\end{minipage} & \begin{minipage}[b]{\linewidth}\raggedright
પ્રકારો
\end{minipage} & \begin{minipage}[b]{\linewidth}\raggedright
એલ્ગોરિધમ ઉદાહરણો
\end{minipage} \\
\midrule\noalign{}
\endhead
\bottomrule\noalign{}
\endlastfoot
એન્ક્રિપ્શન & એલ્ગોરિધમ અને કી વાપરીને પ્લેનટેક્સ્ટને સાયફરટેક્સ્ટમાં કન્વર્ટ કરવાની
પ્રક્રિયા & સિમેટ્રિક, એસિમેટ્રિક, હાઇબ્રિડ & AES, RSA, ECC \\
ડિક્રિપ્શન & એલ્ગોરિધમ અને કી વાપરીને સાયફરટેક્સ્ટને પાછા પ્લેનટેક્સ્ટમાં કન્વર્ટ કરવાની
પ્રક્રિયા & સિમેટ્રિક, એસિમેટ્રિક, હાઇબ્રિડ & AES, RSA, ECC \\
\end{longtable}
}

\textbf{આકૃતિ:}

\begin{lstlisting}
ENCRYPTION:
  +-----------+    ENCRYPTION    +------------+
  | PLAINTEXT |----------------->| CIPHERTEXT |
  +-----------+   ALGORITHM &    +------------+
                  KEY

DECRYPTION:
  +------------+    DECRYPTION    +-----------+
  | CIPHERTEXT |----------------->| PLAINTEXT |
  +------------+   ALGORITHM &    +-----------+
                   KEY
\end{lstlisting}

\textbf{એન્ક્રિપ્શન:}

\begin{itemize}
\tightlist
\item
  \textbf{હેતુ}: માહિતીની ગોપનીયતાનું રક્ષણ કરે છે
\item
  \textbf{પદ્ધતિઓ}: સબ્સ્ટિટ્યુશન, ટ્રાન્સપોઝિશન, બ્લોક સાયફર, સ્ટ્રીમ સાયફર
\item
  \textbf{કી મેનેજમેન્ટ}: સિક્યોર એન્ક્રિપ્શનનો ક્રિટિકલ પાસો
\end{itemize}

\textbf{ડિક્રિપ્શન:}

\begin{itemize}
\tightlist
\item
  \textbf{હેતુ}: એન્ક્રિપ્ટેડ ફોર્મમાંથી ઓરિજિનલ ઇન્ફોર્મેશન રિટ્રીવ કરે છે
\item
  \textbf{આવશ્યકતાઓ}: સાચો એલ્ગોરિધમ અને કી
\item
  \textbf{ઇમ્પ્લિમેન્ટેશન}: હાર્ડવેર અથવા સોફ્ટવેર-બેઝ્ડ
\end{itemize}

\end{solutionbox}
\begin{mnemonicbox}
``PACK-DUKE'' - ``પ્લેનટેક્સ્ટ એલ્ગોરિધમ સાયફર કી -
ડિકોડિંગ યુઝિંગ કી ફોર એક્સટ્રેક્શન''

\end{mnemonicbox}
\subsection*{પ્રશ્ન 5(ક) [7
ગુણ]}\label{uxaaauxab0uxab6uxaa8-5uxa95-7-uxa97uxaa3}

\textbf{ટૂંકનોંધ લખો ૧) ઈ-મેઈલ 2) DNS}

\begin{solutionbox}
\textbf{1) ઈ-મેઈલ (ઇલેક્ટ્રોનિક મેઇલ):}

ઈ-મેઇલ એ કોમ્યુનિકેશન નેટવર્ક પર ડિજિટલ મેસેજ એક્સચેન્જ કરવાની પદ્ધતિ છે.

\textbf{આકૃતિ:}

\begin{lstlisting}
E-MAIL SYSTEM:
   +--------+    SMTP     +---------+    POP3/IMAP   +--------+
   | SENDER |------------>|  MAIL   |--------------->|RECEIVER|
   | CLIENT |             | SERVER  |                | CLIENT |
   +--------+             +---------+                +--------+
                               |
                          +---------+
                          |   DNS   |
                          | SERVER  |
                          +---------+
\end{lstlisting}

{\def\LTcaptype{none} % do not increment counter
\begin{longtable}[]{@{}ll@{}}
\toprule\noalign{}
કોમ્પોનન્ટ & ફંક્શન \\
\midrule\noalign{}
\endhead
\bottomrule\noalign{}
\endlastfoot
મેઇલ યુઝર એજન્ટ (MUA) & એન્ડ-યુઝર્સ દ્વારા વપરાતું ઇમેઇલ ક્લાયન્ટ સોફ્ટવેર \\
મેઇલ ટ્રાન્સફર એજન્ટ (MTA) & ઇમેઇલ ટ્રાન્સફર કરતું સર્વર સોફ્ટવેર \\
મેઇલ ડિલિવરી એજન્ટ (MDA) & રિસિપિયન્ટના મેઇલબોક્સમાં ઇમેઇલ ડિલિવર કરે છે \\
પ્રોટોકોલ્સ & SMTP (સેન્ડિંગ), POP3/IMAP (રિસીવિંગ) \\
\end{longtable}
}

\begin{itemize}
\tightlist
\item
  \textbf{સ્ટ્રક્ચર}: હેડર્સ (To, From, Subject) અને બોડી
\item
  \textbf{સિક્યોરિટી}: એન્ક્રિપ્શન (TLS), ઓથેન્ટિકેશન (SPF, DKIM) જેવા ફીચર્સ
\item
  \textbf{એટેચમેન્ટ્સ}: ટેક્સ્ટ ટ્રાન્સમિશન માટે એન્કોડેડ બાઇનરી ફાઇલ્સ
\item
  \textbf{ફીચર્સ}: ફોરવર્ડિંગ, ફિલ્ટરિંગ, ઓર્ગેનાઇઝિંગ, સર્ચિંગ
\end{itemize}

\textbf{2) DNS (ડોમેન નેમ સિસ્ટમ):}

DNS એ ડોમેન નેમ્સને IP એડ્રેસમાં ટ્રાન્સલેટ કરવા માટેની હાયરાર્કિકલ અને ડિસેન્ટ્રલાઇઝ્ડ
નેમિંગ સિસ્ટમ છે.

\textbf{આકૃતિ:}

\begin{lstlisting}
DNS HIERARCHY:
              +---------+
              |   Root  |
              |   "."   |
              +---------+
                   |
       +-----------+-----------+
       |           |           |
  +---------+ +---------+ +---------+
  |   com   | |   org   | |   net   | ... (TLDs)
  +---------+ +---------+ +---------+
       |           |           |
  +-----------+ +-----------+ +-----------+
  |example.com| |example.org| |example.net| ... (Domains)
  +-----------+ +-----------+ +-----------+
       |
  +---------------+
  |www.example.com| ... (Subdomains)
  +---------------+
\end{lstlisting}

{\def\LTcaptype{none} % do not increment counter
\begin{longtable}[]{@{}ll@{}}
\toprule\noalign{}
DNS કોમ્પોનન્ટ & ફંક્શન \\
\midrule\noalign{}
\endhead
\bottomrule\noalign{}
\endlastfoot
રૂટ સર્વર્સ & DNS હાયરાર્કીનું ટોપ \\
TLD સર્વર્સ & ટોપ-લેવલ ડોમેન મેનેજ કરે છે (.com, .org) \\
ઓથોરિટેટિવ સર્વર્સ & ચોક્કસ ડોમેન માટે DNS રેકોર્ડ્સ સ્ટોર કરે છે \\
રિકર્સિવ રિઝોલ્વર્સ & ડોમેન નેમ્સ રિઝોલ્વ કરવા અન્ય સર્વર્સને ક્વેરી કરે છે \\
DNS રેકોર્ડ્સ & રિસોર્સ રેકોર્ડ્સ (A, AAAA, MX, CNAME, વગેરે) \\
\end{longtable}
}

\begin{itemize}
\tightlist
\item
  \textbf{હેતુ}: હ્યુમન-રીડેબલ નેમ્સને મશીન-રીડેબલ એડ્રેસમાં મેપ કરવા
\item
  \textbf{રિઝોલ્યુશન પ્રોસેસ}: હાયરાર્કી દ્વારા રિકર્સિવ અથવા ઇટરેટિવ ક્વેરીઝ
\item
  \textbf{કેશિંગ}: પરફોર્મન્સ સુધારવા માટે રિઝલ્ટ્સનો ટેમ્પરરી સ્ટોરેજ
\item
  \textbf{સિક્યોરિટી}: DNSSEC ઓથેન્ટિકેશન અને ઇન્ટિગ્રિટી પ્રદાન કરે છે
\end{itemize}

\end{solutionbox}
\begin{mnemonicbox}
``MAPS'' - ``મેઇલ નીડ્સ એડ્રેસિસ, પ્રોટોકોલ્સ, એન્ડ સર્વર્સ''
\end{mnemonicbox}
\begin{mnemonicbox}
``HARD'' - ``હાયરાર્કી, એડ્રેસિંગ, રિઝોલ્યુશન, ડિસ્ટ્રિબ્યુટેડ
સિસ્ટમ''

\end{mnemonicbox}
\subsection*{પ્રશ્ન 5(અ) અથવા [3
ગુણ]}\label{uxaaauxab0uxab6uxaa8-5uxa85-uxa85uxaa5uxab5-3-uxa97uxaa3}

\textbf{સિક્યોરીટી ટોપોલોજી અને સિક્યોરીટી ઝોન શું છે?}

\begin{solutionbox}
સિક્યોરિટી ટોપોલોજી અને સિક્યોરિટી ઝોન એ નેટવર્ક સિક્યોરિટી
કન્સેપ્ટ્સ છે જે નેટવર્ક રિસોર્સિસનું આયોજન અને રક્ષણ કરે છે.

{\def\LTcaptype{none} % do not increment counter
\begin{longtable}[]{@{}
  >{\raggedright\arraybackslash}p{(\linewidth - 4\tabcolsep) * \real{0.3333}}
  >{\raggedright\arraybackslash}p{(\linewidth - 4\tabcolsep) * \real{0.3333}}
  >{\raggedright\arraybackslash}p{(\linewidth - 4\tabcolsep) * \real{0.3333}}@{}}
\toprule\noalign{}
\begin{minipage}[b]{\linewidth}\raggedright
કન્સેપ્ટ
\end{minipage} & \begin{minipage}[b]{\linewidth}\raggedright
વ્યાખ્યા
\end{minipage} & \begin{minipage}[b]{\linewidth}\raggedright
ઉદાહરણો
\end{minipage} \\
\midrule\noalign{}
\endhead
\bottomrule\noalign{}
\endlastfoot
સિક્યોરિટી ટોપોલોજી & સિક્યોરિટી કંટ્રોલ્સની ફિઝિકલ અને લોજિકલ ગોઠવણી & DMZ,
ડિફેન્સ-ઇન-ડેપ્થ \\
સિક્યોરિટી ઝોન & ચોક્કસ સિક્યોરિટી આવશ્યકતાઓ સાથે નેટવર્કનો ભાગ & DMZ, ઇન્ટ્રાનેટ,
એક્સટ્રાનેટ \\
\end{longtable}
}

\textbf{આકૃતિ:}

\begin{lstlisting}
SECURITY TOPOLOGY WITH ZONES:
                  +----------+
                  | INTERNET |
                  +----+-----+
                       |
                       | Firewall
                       |
                  +----+-----+
                  |   DMZ    |  Web, Email, DNS servers
                  +----+-----+
                       |
                       | Firewall
                       |
         +-------------+--------------+
         |                            |
    +----+-----+                 +----+----+
    | INTRANET |                 | SECURED |
    | ZONE     |                 | ZONE    |  Sensitive data
    +----+-----+                 +----+----+
         |
    +----+-----+
    |   USER   |  Workstations
    |   ZONE   |
    +----------+
\end{lstlisting}

\begin{itemize}
\tightlist
\item
  \textbf{સિક્યોરિટી ટોપોલોજી}: સમગ્ર સિક્યોરિટી આર્કિટેક્ચર ડિઝાઇન
\item
  \textbf{સિક્યોરિટી ઝોન્સ}: કન્સિસ્ટન્ટ સિક્યોરિટી પોલિસીઓ સાથેની લોજિકલ
  બાઉન્ડરીઝ
\item
  \textbf{ડિફેન્સ-ઇન-ડેપ્થ}: સિક્યોરિટી કંટ્રોલ્સના મલ્ટિપલ લેયર્સ
\end{itemize}

\end{solutionbox}
\begin{mnemonicbox}
``TIPS'' - ``ટોપોલોજી આઇસોલેટ્સ એન્ડ પ્રોટેક્ટ્સ સિસ્ટમ્સ''

\end{mnemonicbox}
\subsection*{પ્રશ્ન 5(બ) અથવા [4
ગુણ]}\label{uxaaauxab0uxab6uxaa8-5uxaac-uxa85uxaa5uxab5-4-uxa97uxaa3}

\textbf{વોઇસ અને વિડીયો IP પર ટૂંકનોંધ લખો.}

\begin{solutionbox}
વોઇસ અને વિડિયો ઓવર IP (VoIP/Video IP) એ IP નેટવર્ક પર વોઇસ
અને વિડિયો કોમ્યુનિકેશન ટ્રાન્સમિટ કરવાની ટેકનોલોજી છે.

\textbf{આકૃતિ:}

\begin{lstlisting}
  +--------+                      +--------+
  |        |      INTERNET        |        |
  | CALLER |----------------------|RECEIVER|
  |        |   RTP/UDP/IP         |        |
  +--------+                      +--------+
      |                               |
      |                               |
   +-----+                         +-----+
   |Codec|                         |Codec|
   +-----+                         +-----+
   Digital                         Digital
   encoding                        decoding
\end{lstlisting}

{\def\LTcaptype{none} % do not increment counter
\begin{longtable}[]{@{}ll@{}}
\toprule\noalign{}
કોમ્પોનન્ટ & ફંક્શન \\
\midrule\noalign{}
\endhead
\bottomrule\noalign{}
\endlastfoot
કોડેક્સ & ઓડિયો અને વિડિયો એન્કોડ/ડિકોડ કરે છે (G.711, H.264) \\
સિગ્નલિંગ પ્રોટોકોલ્સ & કોલ સેટઅપ/ટિયરડાઉન (SIP, H.323) \\
ટ્રાન્સપોર્ટ પ્રોટોકોલ & રિયલ-ટાઇમ મીડિયા ટ્રાન્સપોર્ટ (RTP/RTCP) \\
QoS મેકેનિઝમ્સ & વોઇસ/વિડિયો ટ્રાફિકને પ્રાયોરિટાઇઝ કરે છે \\
\end{longtable}
}

\textbf{વોઇસ ઓવર IP (VoIP):}

\begin{itemize}
\tightlist
\item
  \textbf{ફાયદા}: કોસ્ટ સેવિંગ, ફ્લેક્સિબિલિટી, એપ્સ સાથે ઇન્ટિગ્રેશન
\item
  \textbf{ચેલેન્જીસ}: લેટન્સી, જિટર, પેકેટ લોસ
\item
  \textbf{અનુપ્રયોગો}: IP ફોન, સોફ્ટફોન, કોન્ફરન્સિંગ
\end{itemize}

\textbf{વિડિયો ઓવર IP:}

\begin{itemize}
\tightlist
\item
  \textbf{પ્રકારો}: વિડિયો કોન્ફરન્સિંગ, સ્ટ્રીમિંગ, સર્વેલન્સ
\item
  \textbf{આવશ્યકતાઓ}: ઉચ્ચ બેન્ડવિડ્થ, લો લેટન્સી
\item
  \textbf{ટેકનોલોજીઓ}: WebRTC, SIP વિડિયો, RTSP સ્ટ્રીમિંગ
\end{itemize}

\end{solutionbox}
\begin{mnemonicbox}
``CLEAR'' - ``કોડેક્સ કમ્પ્રેસ, લેટન્સી મેટર્સ, એન્કોડ્સ
ઓડિયો/વિડિયો, એપ્લિકેશન્સ ઇન્ટિગ્રેટ, રિયલ-ટાઇમ ટ્રાન્સપોર્ટ''

\end{mnemonicbox}
\subsection*{પ્રશ્ન 5(ક) અથવા [7
ગુણ]}\label{uxaaauxab0uxab6uxaa8-5uxa95-uxa85uxaa5uxab5-7-uxa97uxaa3}

\textbf{IP સિક્યોરીટી શું છે? વિગતવાર સમજાવો.}

\begin{solutionbox}
IP સિક્યોરિટી (IPsec) એ દરેક IP પેકેટને ઓથેન્ટિકેટ અને એન્ક્રિપ્ટ
કરીને IP કોમ્યુનિકેશન સિક્યોર કરવા માટે ડિઝાઇન કરાયેલ પ્રોટોકોલ્સનો સમૂહ છે.

\textbf{આકૃતિ:}

\begin{lstlisting}
IPSEC PROTOCOL SUITE:
+--------------------------------------+
|            APPLICATIONS              |
+--------------------------------------+
|      TRANSPORT LAYER (TCP/UDP)       |
+--------------------------------------+
|                                      |
|              IP LAYER                |
|                                      |
|  +------------+    +-------------+   |
|  |    AH      |    |     ESP     |   |
|  | (Auth Hdr) |    | (Enc Sec Pay)|  |
|  +------------+    +-------------+   |
|                                      |
|         +-------------------+        |
|         |   IKE/ISAKMP      |        |
|         | (Key Management)  |        |
|         +-------------------+        |
+--------------------------------------+
|           NETWORK ACCESS             |
+--------------------------------------+
\end{lstlisting}

{\def\LTcaptype{none} % do not increment counter
\begin{longtable}[]{@{}
  >{\raggedright\arraybackslash}p{(\linewidth - 4\tabcolsep) * \real{0.4571}}
  >{\raggedright\arraybackslash}p{(\linewidth - 4\tabcolsep) * \real{0.2286}}
  >{\raggedright\arraybackslash}p{(\linewidth - 4\tabcolsep) * \real{0.3143}}@{}}
\toprule\noalign{}
\begin{minipage}[b]{\linewidth}\raggedright
IPsec પ્રોટોકોલ
\end{minipage} & \begin{minipage}[b]{\linewidth}\raggedright
ફંક્શન
\end{minipage} & \begin{minipage}[b]{\linewidth}\raggedright
પ્રોટેક્શન
\end{minipage} \\
\midrule\noalign{}
\endhead
\bottomrule\noalign{}
\endlastfoot
ઓથેન્ટિકેશન હેડર (AH) & ડેટા ઇન્ટિગ્રિટી, ઓથેન્ટિકેશન & એન્ક્રિપ્શન નહીં \\
એન્કેપ્સુલેટિંગ સિક્યોરિટી પેલોડ (ESP) & કોન્ફિડેન્શિયાલિટી, ઇન્ટિગ્રિટી, ઓથેન્ટિકેશન &
ડેટા એન્ક્રિપ્ટ કરે છે \\
ઇન્ટરનેટ કી એક્સચેન્જ (IKE) & કી એક્સચેન્જ, SA નેગોશિએશન & સિક્યોર કી મેનેજમેન્ટ \\
\end{longtable}
}

\textbf{IPsec મોડ્સ:}

{\def\LTcaptype{none} % do not increment counter
\begin{longtable}[]{@{}lll@{}}
\toprule\noalign{}
મોડ & વર્ણન & યુઝ કેસ \\
\midrule\noalign{}
\endhead
\bottomrule\noalign{}
\endlastfoot
ટ્રાન્સપોર્ટ મોડ & માત્ર પેલોડનું રક્ષણ કરે છે & હોસ્ટ-ટુ-હોસ્ટ કોમ્યુનિકેશન \\
ટનલ મોડ & સમગ્ર પેકેટનું રક્ષણ કરે છે & સાઇટ-ટુ-સાઇટ VPNs, રિમોટ એક્સેસ \\
\end{longtable}
}

\textbf{સિક્યોરિટી સર્વિસિસ:}

\begin{itemize}
\tightlist
\item
  \textbf{ઓથેન્ટિકેશન}: કોમ્યુનિકેટિંગ એન્ટિટીઓની ઓળખ ચકાસે છે
\item
  \textbf{કોન્ફિડેન્શિયાલિટી}: ડેટાને અનધિકૃત જાહેરાતથી રક્ષણ આપે છે
\item
  \textbf{ડેટા ઇન્ટિગ્રિટી}: ડેટા ટ્રાન્ઝિટમાં બદલાયો નથી તે સુનિશ્ચિત કરે છે
\item
  \textbf{રિપ્લે પ્રોટેક્શન}: પેકેટ રિપ્લે એટેક્સને રોકે છે
\item
  \textbf{એક્સેસ કંટ્રોલ}: નેટવર્ક રિસોર્સિસની એક્સેસને મર્યાદિત કરે છે
\end{itemize}

\textbf{અનુપ્રયોગો:}

\begin{itemize}
\tightlist
\item
  \textbf{VPNs}: રિમોટ એક્સેસ અને સાઇટ-ટુ-સાઇટ કનેક્શન
\item
  \textbf{સિક્યોર રાઉટિંગ}: રાઉટિંગ પ્રોટોકોલ્સનું રક્ષણ કરે છે
\item
  \textbf{સિક્યોર હોસ્ટ-ટુ-હોસ્ટ}: એન્ડ-ટુ-એન્ડ સિક્યોરિટી
\end{itemize}

\end{solutionbox}
\begin{mnemonicbox}
``AVID TC'' - ``ઓથેન્ટિકેશન, વેરિફિકેશન, ઇન્ટિગ્રિટી,
ડેટાગ્રામ પ્રોટેક્શન, ટ્રાન્સપોર્ટ મોડ, કોન્ફિડેન્શિયાલિટી''

\end{mnemonicbox}

\end{document}
