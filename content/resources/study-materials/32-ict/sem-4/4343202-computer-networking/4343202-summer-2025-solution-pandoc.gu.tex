\documentclass[10pt,a4paper]{article}

% content/resources/templates/preamble.tex
\usepackage[margin=0.6in]{geometry}
\author{Milav Dabgar}
\usepackage{amsmath,amssymb,amsthm}
\usepackage{booktabs}
\usepackage{multirow}
\usepackage{xcolor}
\usepackage{tcolorbox}
\tcbuselibrary{breakable,skins}
\usepackage[colorlinks=true,linkcolor=blue]{hyperref}
\usepackage{titlesec}
\usepackage{enumitem}
\usepackage{tikz}
\usepackage{pgfplots}
\usepackage{circuitikz}
\usepackage[version=4]{mhchem}
\usepackage{longtable}
\usepackage{array}
\usepackage{float}
\usepackage{caption}
\usepackage{listings}

\lstset{
  basicstyle=\small\ttfamily,
  breaklines=true,
  breakatwhitespace=false,
  postbreak=\mbox{\textcolor{red}{$\hookrightarrow$}\space},
  float=false,
  numbers=left,
  numberstyle=\tiny\color{gray},
  numbersep=10pt,
  xleftmargin=2em,
  keywordstyle=\color{blue},
  commentstyle=\color{green!60!black},
  stringstyle=\color{purple},
  backgroundcolor=\color{gray!5},
  showstringspaces=false,
  tabsize=2,
  captionpos=b,
  keepspaces=true,
  columns=flexible
}

\pgfplotsset{compat=1.18}
\usetikzlibrary{shapes,arrows,positioning,calc,patterns,decorations.pathmorphing,decorations.markings,arrows.meta}

% Color scheme
\definecolor{headcolor}{RGB}{0,102,204}
\definecolor{keycolor}{RGB}{220,20,60}
\definecolor{solutioncolor}{RGB}{34,139,34}
\definecolor{mnemoniccolor}{RGB}{148,0,211}
\definecolor{codecolor}{RGB}{0,0,100}

% Spacing
\setlength{\parskip}{3pt}
\setlist[itemize]{nosep}
\setlist[enumerate]{nosep}

% Title formatting
\titleformat{\section}{\Large\bfseries\color{headcolor}}{\thesection}{1em}{}
\titleformat{\subsection}{\large\bfseries\color{headcolor}}{\thesubsection}{1em}{}

% Pandoc tightlist compatibility
\providecommand{\tightlist}{%
  \setlength{\itemsep}{0pt}\setlength{\parskip}{0pt}}

% Pandoc longtable compatibility
\newcounter{none}
\def\thenone{}


% content/resources/templates/gujarati-boxes.tex
\usepackage{fontspec}
\usepackage{polyglossia}

% Set Gujarati as main language (document is primarily in Gujarati)
% Note: gloss-gujarati.ldf doesn't exist in polyglossia, but it will use hyphenation patterns
\setdefaultlanguage{gujarati}
\setotherlanguage{english}

% Configure Gujarati font properly
% Use Language=Default to prevent polyglossia from trying to add language-specific features
% that don't exist for Gujarati, which causes "empty feature" warnings
\newfontfamily\gujaratifont[Script=Gujarati,AutoFakeBold=2.5,AutoFakeSlant=0.3]{Noto Sans Gujarati}
\setmainfont[Script=Gujarati,AutoFakeBold=2.5,AutoFakeSlant=0.3]{Noto Sans Gujarati}
% Use Noto Sans Gujarati for monospace to support Gujarati in text
\setmonofont[Scale=0.9]{Noto Sans Gujarati}

% Configure English to use the same font
\newfontfamily\englishfont[Script=Gujarati,AutoFakeBold=2.5,AutoFakeSlant=0.3]{Noto Sans Gujarati}

% Translations for polyglossia
\gappto\captionsgujarati{
  \renewcommand{\tablename}{કોષ્ટક}
  \renewcommand{\figurename}{આકૃતિ}
}

% Helper for TikZ nodes to ensure Gujarati font
\newcommand{\gu}[1]{{\gujaratifont #1}}

% Custom environments
\newtcolorbox{solutionbox}{
    breakable,
    enhanced,
    colback=solutioncolor!5!white,
    colframe=solutioncolor!75!black,
    fonttitle=\bfseries,
    title=જવાબ
}

\newtcolorbox{solutionboxnobreak}{
 colback=solutioncolor!5!white,
 colframe=solutioncolor!75!black,
 fonttitle=\bfseries,
 title=જવાબ
}

\newtcolorbox{keyformula}{
 breakable,
 enhanced,
 colback=keycolor!5!white,
 colframe=keycolor!75!black,
 fonttitle=\bfseries,
 title=રાસાયણિક સમીકરણ/સૂત્ર
}

\newtcolorbox{mnemonicbox}{
 breakable,
 enhanced,
 colback=mnemoniccolor!5!white,
 colframe=mnemoniccolor!75!black,
 fonttitle=\bfseries,
 title=મેમરી ટ્રીક
}


\begin{document}

\begin{center}
{\Huge\bfseries\color{headcolor} Subject Name (Gujarati)}\\[5pt]
{\LARGE 4343202 -- Summer 2025}\\[3pt]
{\large Semester 1 Study Material}\\[3pt]
{\normalsize\textit{Detailed Solutions and Explanations}}
\end{center}

\vspace{10pt}

\subsection*{પ્રશ્ન 1(અ) [3
ગુણ]}\label{uxaaauxab0uxab6uxaa8-1uxa85-3-uxa97uxaa3}

\textbf{કમ્પ્યુટર નેટવર્કની વિવિધ નેટવર્ક ટોપોલોજીની યાદી બનાવો અને કોઈપણ એકને
સમજાવો.}

\begin{solutionbox}

\textbf{નેટવર્ક ટોપોલોજીઓનું ટેબલ:}

{\def\LTcaptype{none} % do not increment counter
\begin{longtable}[]{@{}ll@{}}
\toprule\noalign{}
ટોપોલોજી & વર્ણન \\
\midrule\noalign{}
\endhead
\bottomrule\noalign{}
\endlastfoot
\textbf{સ્ટાર} & કેન્દ્રીય હબ બધા ઉપકરણોને જોડે છે \\
\textbf{રિંગ} & ઉપકરણો વર્તુળાકાર શૃંખલામાં જોડાયેલા \\
\textbf{બસ} & સિંગલ કેબલ બેકબોન કનેક્શન \\
\textbf{મેશ} & દરેક ઉપકરણ બીજા બધા સાથે જોડાય છે \\
\textbf{ટ્રી} & લવાલવ શાખાઓનું માળખું \\
\textbf{હાઇબ્રિડ} & અનેક ટોપોલોજીનું મિશ્રણ \\
\end{longtable}
}

\textbf{સ્ટાર ટોપોલોજી સમજૂતી:}

\begin{itemize}
\tightlist
\item
  \textbf{કેન્દ્રીય હબ}: બધા ઉપકરણો એક કેન્દ્રીય બિંદુ સાથે જોડાય
\item
  \textbf{સરળ ઇન્સ્ટોલેશન}: ઉપકરણો ઉમેરવા/દૂર કરવા સરળ
\item
  \textbf{સિંગલ પોઇન્ટ ફેલ્યર}: હબ નિષ્ફળતા આખા નેટવર્કને અસર કરે
\end{itemize}

\end{solutionbox}
\begin{mnemonicbox}
``SRBMTH - સ્ટાર રિંગ બસ મેશ ટ્રી હાઇબ્રિડ''

\end{mnemonicbox}
\subsection*{પ્રશ્ન 1(બ) [4
ગુણ]}\label{uxaaauxab0uxab6uxaa8-1uxaac-4-uxa97uxaa3}

\textbf{LAN, WAN અને MAN ની સરખામણી કરો.}

\begin{solutionbox}

\textbf{સરખામણી ટેબલ:}

{\def\LTcaptype{none} % do not increment counter
\begin{longtable}[]{@{}
  >{\raggedright\arraybackslash}p{(\linewidth - 6\tabcolsep) * \real{0.2895}}
  >{\raggedright\arraybackslash}p{(\linewidth - 6\tabcolsep) * \real{0.2368}}
  >{\raggedright\arraybackslash}p{(\linewidth - 6\tabcolsep) * \real{0.2368}}
  >{\raggedright\arraybackslash}p{(\linewidth - 6\tabcolsep) * \real{0.2368}}@{}}
\toprule\noalign{}
\begin{minipage}[b]{\linewidth}\raggedright
પેરામીટર
\end{minipage} & \begin{minipage}[b]{\linewidth}\raggedright
\textbf{LAN}
\end{minipage} & \begin{minipage}[b]{\linewidth}\raggedright
\textbf{MAN}
\end{minipage} & \begin{minipage}[b]{\linewidth}\raggedright
\textbf{WAN}
\end{minipage} \\
\midrule\noalign{}
\endhead
\bottomrule\noalign{}
\endlastfoot
\textbf{કવરેજ} & બિલ્ડિંગ/કેમ્પસ & શહેર/મેટ્રોપોલિટન & દેશ/વૈશ્વિક \\
\textbf{સ્પીડ} & અત્યંત વધુ (1-100 Gbps) & વધુ (10-100 Mbps) & મધ્યમ (1-100
Mbps) \\
\textbf{કિંમત} & ઓછી & મધ્યમ & વધુ \\
\textbf{માલિકી} & ખાનગી & સાર્વજનિક/ખાનગી & સાર્વજનિક \\
\end{longtable}
}

\textbf{મુખ્ય મુદ્દાઓ:}

\begin{itemize}
\tightlist
\item
  \textbf{LAN}: નાના વિસ્તારો માટે લોકલ એરિયા નેટવર્ક
\item
  \textbf{MAN}: શહેરો માટે મેટ્રોપોલિટન એરિયા નેટવર્ક
\item
  \textbf{WAN}: મોટા અંતર માટે વાઇડ એરિયા નેટવર્ક
\end{itemize}

\end{solutionbox}
\begin{mnemonicbox}
``LMW - લોકલ મેટ્રોપોલિટન વાઇડ''

\end{mnemonicbox}
\subsection*{પ્રશ્ન 1(ક) [7
ગુણ]}\label{uxaaauxab0uxab6uxaa8-1uxa95-7-uxa97uxaa3}

\textbf{OSI સંદર્ભ મોડેલનું સ્તરીય આર્કિટેક્ચર દોરો અને મોડેલના દરેક સ્તર દ્વારા પૂરી
પાડવામાં આવતી ઓછામાં ઓછી બે સેવાઓ લખો.}

\begin{solutionbox}

\includegraphics[width=1\linewidth,height=\textheight,keepaspectratio]{mermaid-dc8e3e20.pdf}

\textbf{દરેક લેયરની સેવાઓ:}

{\def\LTcaptype{none} % do not increment counter
\begin{longtable}[]{@{}ll@{}}
\toprule\noalign{}
લેયર & \textbf{સેવાઓ} \\
\midrule\noalign{}
\endhead
\bottomrule\noalign{}
\endlastfoot
\textbf{એપ્લિકેશન (7)} & ઇમેઇલ સેવાઓ, ફાઇલ ટ્રાન્સફર \\
\textbf{પ્રેઝન્ટેશન (6)} & ડેટા એન્ક્રિપ્શન, ડેટા કમ્પ્રેશન \\
\textbf{સેશન (5)} & સેશન સ્થાપના, સેશન સમાપ્તિ \\
\textbf{ટ્રાન્સપોર્ટ (4)} & ફ્લો કંટ્રોલ, એરર કરેક્શન \\
\textbf{નેટવર્ક (3)} & રૂટિંગ, પાથ નિર્ધારણ \\
\textbf{ડેટા લિંક (2)} & ફ્રેમ સિંક્રોનાઇઝેશન, એરર ડિટેક્શન \\
\textbf{ફિઝિકલ (1)} & બિટ ટ્રાન્સમિશન, સિગ્નલ કન્વર્ઝન \\
\end{longtable}
}

\end{solutionbox}
\begin{mnemonicbox}
``All People Seem To Need Data Processing''

\end{mnemonicbox}
\subsection*{પ્રશ્ન 1(ક OR) [7
ગુણ]}\label{uxaaauxab0uxab6uxaa8-1uxa95-or-7-uxa97uxaa3}

\textbf{TCP/IP મોડેલના દરેક સ્તરને તેના પ્રોટોકોલ સાથે સમજાવો.}

\begin{solutionbox}

\includegraphics[width=1\linewidth,height=\textheight,keepaspectratio]{mermaid-91583839.pdf}

\textbf{TCP/IP મોડેલ લેયર્સ:}

{\def\LTcaptype{none} % do not increment counter
\begin{longtable}[]{@{}lll@{}}
\toprule\noalign{}
લેયર & \textbf{પ્રોટોકોલ} & \textbf{કાર્ય} \\
\midrule\noalign{}
\endhead
\bottomrule\noalign{}
\endlastfoot
\textbf{એપ્લિકેશન} & HTTP, FTP, SMTP, DNS & યુઝર એપ્લિકેશન્સ \\
\textbf{ટ્રાન્સપોર્ટ} & TCP, UDP & અંત-થી-અંત ડિલિવરી \\
\textbf{ઇન્ટરનેટ} & IP, ICMP, ARP & પેકેટ રૂટિંગ \\
\textbf{નેટવર્ક એક્સેસ} & Ethernet, Wi-Fi & ફિઝિકલ ટ્રાન્સમિશન \\
\end{longtable}
}

\textbf{મુખ્ય લક્ષણો:}

\begin{itemize}
\tightlist
\item
  \textbf{સરળ મોડેલ}: OSI ના 7 ની સામે માત્ર 4 લેયર
\item
  \textbf{પ્રોટોકોલ સ્યૂટ}: સંપૂર્ણ નેટવર્કિંગ સોલ્યુશન
\item
  \textbf{ઇન્ટરનેટ સ્ટાન્ડર્ડ}: આધુનિક ઇન્ટરનેટનો આધાર
\end{itemize}

\end{solutionbox}
\begin{mnemonicbox}
``ATIN - એપ્લિકેશન ટ્રાન્સપોર્ટ ઇન્ટરનેટ નેટવર્ક''

\end{mnemonicbox}
\subsection*{પ્રશ્ન 2(અ) [3
ગુણ]}\label{uxaaauxab0uxab6uxaa8-2uxa85-3-uxa97uxaa3}

\textbf{નીચેના નેટવર્ક ઉપકરણોના કાર્યો સમજાવો: રીપીટર, હબ}

\begin{solutionbox}

\textbf{ઉપકરણ કાર્યો:}

{\def\LTcaptype{none} % do not increment counter
\begin{longtable}[]{@{}lll@{}}
\toprule\noalign{}
ઉપકરણ & \textbf{કાર્ય} & \textbf{લેયર} \\
\midrule\noalign{}
\endhead
\bottomrule\noalign{}
\endlastfoot
\textbf{રીપીટર} & સિગ્નલ એમ્પ્લિફિકેશન, રેન્જ વિસ્તરણ & ફિઝિકલ (1) \\
\textbf{હબ} & સિગ્નલ બ્રોડકાસ્ટિંગ, કોલિઝન ડોમેન શેરિંગ & ફિઝિકલ (1) \\
\end{longtable}
}

\textbf{વિગતો:}

\begin{itemize}
\tightlist
\item
  \textbf{રીપીટર}: લાંબા અંતર પર નબળા સિગ્નલને પુનર્જનરેટ કરે છે
\item
  \textbf{હબ}: સ્ટાર ટોપોલોજીમાં અનેક ઉપકરણોને જોડે છે
\item
  \textbf{શેર્ડ મીડિયમ}: બંને સિંગલ કોલિઝન ડોમેન બનાવે છે
\end{itemize}

\end{solutionbox}
\begin{mnemonicbox}
``RH - રીપીટ હબ સિગ્નલ્સ''

\end{mnemonicbox}
\subsection*{પ્રશ્ન 2(બ) [4
ગુણ]}\label{uxaaauxab0uxab6uxaa8-2uxaac-4-uxa97uxaa3}

\textbf{નીચેના શબ્દને સમજાવો 1) FDDI 2) ARP, RARP}

\begin{solutionbox}

\textbf{FDDI (ફાઇબર ડિસ્ટ્રિબ્યુટેડ ડેટા ઇન્ટરફેસ):}

\begin{itemize}
\tightlist
\item
  \textbf{ટેકનોલોજી}: 100 Mbps ફાઇબર ઓપ્ટિક નેટવર્ક
\item
  \textbf{ટોપોલોજી}: ફોલ્ટ ટોલરન્સ માટે ડ્યુઅલ રિંગ
\item
  \textbf{એપ્લિકેશન}: બેકબોન નેટવર્ક્સ, ઉચ્ચ વિશ્વસનીયતા
\end{itemize}

\textbf{ARP (એડ્રેસ રિઝોલ્યુશન પ્રોટોકોલ):}

\begin{itemize}
\tightlist
\item
  \textbf{કાર્ય}: IP એડ્રેસને MAC એડ્રેસ સાથે મેપ કરે છે
\item
  \textbf{પ્રક્રિયા}: રિક્વેસ્ટ બ્રોડકાસ્ટ કરે, રિપ્લાય મેળવે
\end{itemize}

\textbf{RARP (રિવર્સ ARP):}

\begin{itemize}
\tightlist
\item
  \textbf{કાર્ય}: MAC એડ્રેસને IP એડ્રેસ સાથે મેપ કરે છે
\item
  \textbf{ઉપયોગ}: ડિસ્કલેસ વર્કસ્ટેશન્સ, બૂટ પ્રક્રિયા
\end{itemize}

\end{solutionbox}
\begin{mnemonicbox}
``FAR - FDDI ARP RARP''

\end{mnemonicbox}
\subsection*{પ્રશ્ન 2(ક) [7
ગુણ]}\label{uxaaauxab0uxab6uxaa8-2uxa95-7-uxa97uxaa3}

\textbf{સિદ્ધાન્તો અને કર્બેરોસ-કન્સેપ્ટ સાથે નેટવર્ક સુરક્ષામાં ફાયરવોલનું કાર્ય સમજાવો}

\begin{solutionbox}

\textbf{ફાયરવોલ કાર્યો:}

\includegraphics[width=1\linewidth,height=\textheight,keepaspectratio]{mermaid-69756ea8.pdf}

\textbf{ફાયરવોલ સિદ્ધાન્તો:}

\begin{itemize}
\tightlist
\item
  \textbf{પેકેટ ફિલ્ટરિંગ}: પેકેટ હેડર્સની તપાસ કરે છે
\item
  \textbf{સ્ટેટફુલ ઇન્સ્પેક્શન}: કનેક્શન સ્ટેટ્સને ટ્રેક કરે છે
\item
  \textbf{એપ્લિકેશન ગેટવે}: ડીપ પેકેટ ઇન્સ્પેક્શન
\end{itemize}

\textbf{કર્બેરોસ કન્સેપ્ટ:}

\begin{itemize}
\tightlist
\item
  \textbf{ઓથેન્ટિકેશન સર્વિસ}: સુરક્ષિત યુઝર વેરિફિકેશન
\item
  \textbf{ટિકિટ સિસ્ટમ}: સમય-મર્યાદિત એક્સેસ ટોકન્સ
\item
  \textbf{થ્રી-પાર્ટી પ્રોટોકોલ}: ક્લાયંટ, સર્વર, key ડિસ્ટ્રિબ્યુશન સેન્ટર
\end{itemize}

\textbf{સુરક્ષા લાભો:}

\begin{itemize}
\tightlist
\item
  \textbf{એક્સેસ કંટ્રોલ}: અનધિકૃત પ્રવેશ અટકાવે છે
\item
  \textbf{નેટવર્ક પ્રોટેક્શન}: આંતરિક સંસાધનોને સુરક્ષા આપે છે
\end{itemize}

\end{solutionbox}
\begin{mnemonicbox}
``FPK - ફાયરવોલ કર્બેરોસ સાથે પ્રોટેક્ટ કરે''

\end{mnemonicbox}
\subsection*{પ્રશ્ન 2(અ OR) [3
ગુણ]}\label{uxaaauxab0uxab6uxaa8-2uxa85-or-3-uxa97uxaa3}

\textbf{નીચેના નેટવર્ક ઉપકરણોના કાર્યો સમજાવો: સ્વિચ , રાઉટર}

\begin{solutionbox}

\textbf{ઉપકરણ કાર્યો:}

{\def\LTcaptype{none} % do not increment counter
\begin{longtable}[]{@{}lll@{}}
\toprule\noalign{}
ઉપકરણ & \textbf{કાર્ય} & \textbf{લેયર} \\
\midrule\noalign{}
\endhead
\bottomrule\noalign{}
\endlastfoot
\textbf{સ્વિચ} & MAC એડ્રેસ લર્નિંગ, ફ્રેમ ફોરવર્ડિંગ & ડેટા લિંક (2) \\
\textbf{રાઉટર} & IP રૂટિંગ, પાથ સિલેક્શન & નેટવર્ક (3) \\
\end{longtable}
}

\textbf{વિગતો:}

\begin{itemize}
\tightlist
\item
  \textbf{સ્વિચ}: દરેક પોર્ટ માટે અલગ કોલિઝન ડોમેન બનાવે છે
\item
  \textbf{રાઉટર}: વિવિધ નેટવર્ક્સને જોડે છે, રૂટિંગ નિર્ણયો લે છે
\item
  \textbf{ઇન્ટેલિજન્સ}: સ્વિચ MAC શીખે છે, રાઉટર રૂટિંગ ટેબલ રાખે છે
\end{itemize}

\end{solutionbox}
\begin{mnemonicbox}
``SR - સ્વિચ રૂટ્સ ઇન્ટેલિજન્ટલી''

\end{mnemonicbox}
\subsection*{પ્રશ્ન 2(બ OR) [4
ગુણ]}\label{uxaaauxab0uxab6uxaa8-2uxaac-or-4-uxa97uxaa3}

\textbf{નીચેના શબ્દ સમજાવો 1) CDDI 2) DHCP અને BOOTP}

\begin{solutionbox}

\textbf{CDDI (કોપર ડિસ્ટ્રિબ્યુટેડ ડેટા ઇન્ટરફેસ):}

\begin{itemize}
\tightlist
\item
  \textbf{ટેકનોલોજી}: કોપર કેબલ પર FDDI
\item
  \textbf{સ્પીડ}: ટ્વિસ્ટેડ પેર પર 100 Mbps
\item
  \textbf{કિંમત}: ફાઇબર FDDI કરતાં સસ્તું વિકલ્પ
\end{itemize}

\textbf{DHCP (ડાયનેમિક હોસ્ટ કન્ફિગરેશન પ્રોટોકોલ):}

\begin{itemize}
\tightlist
\item
  \textbf{કાર્ય}: ઓટોમેટિક IP એડ્રેસ અસાઇનમેન્ટ
\item
  \textbf{પ્રક્રિયા}: ડિસ્કવર, ઓફર, રિક્વેસ્ટ, એકનોલેજ
\item
  \textbf{લાભો}: કેન્દ્રીકૃત IP મેનેજમેન્ટ
\end{itemize}

\textbf{BOOTP (બૂટસ્ટ્રેપ પ્રોટોકોલ):}

\begin{itemize}
\tightlist
\item
  \textbf{કાર્ય}: ડિસ્કલેસ ક્લાયંટ્સ માટે નેટવર્ક બૂટસ્ટ્રેપ
\item
  \textbf{સ્ટેટિક}: ફિક્સ્ડ IP એડ્રેસ અસાઇનમેન્ટ
\item
  \textbf{પૂર્વવર્તી}: DHCP નું અગાઉનું વર્ઝન
\end{itemize}

\end{solutionbox}
\begin{mnemonicbox}
``CDB - CDDI DHCP BOOTP''

\end{mnemonicbox}
\subsection*{પ્રશ્ન 2(ક OR) [7
ગુણ]}\label{uxaaauxab0uxab6uxaa8-2uxa95-or-7-uxa97uxaa3}

\textbf{સોફ્ટવેર ડિફાઇન નેટવર્ક(SDN) ને તેના આર્કિટેક્ચર, એપ્લિકેશન, એડવાન્ટેજ અને
મર્યાદા સાથે સમજાવો.}

\begin{solutionbox}

\includegraphics[width=1\linewidth,height=\textheight,keepaspectratio]{mermaid-24023825.pdf}

\textbf{SDN આર્કિટેક્ચર:}

\begin{itemize}
\tightlist
\item
  \textbf{કંટ્રોલ પ્લેન}: કેન્દ્રીકૃત નેટવર્ક ઇન્ટેલિજન્સ
\item
  \textbf{ડેટા પ્લેન}: પેકેટ ફોરવર્ડિંગ ઉપકરણો
\item
  \textbf{એપ્લિકેશન પ્લેન}: નેટવર્ક એપ્લિકેશન્સ અને સેવાઓ
\end{itemize}

\textbf{એપ્લિકેશન્સ:}

\begin{itemize}
\tightlist
\item
  \textbf{ક્લાઉડ કમ્પ્યુટિંગ}: ડાયનેમિક રિસોર્સ એલોકેશન
\item
  \textbf{નેટવર્ક વર્ચ્યુઅલાઇઝેશન}: મલ્ટિપલ વર્ચ્યુઅલ નેટવર્ક્સ
\item
  \textbf{ટ્રાફિક એન્જિનિયરિંગ}: ઓપ્ટિમાઇઝ્ડ પાથ સિલેક્શન
\end{itemize}

\textbf{ફાયદાઓ:}

\begin{itemize}
\tightlist
\item
  \textbf{કેન્દ્રીકૃત કંટ્રોલ}: સરળ નેટવર્ક મેનેજમેન્ટ
\item
  \textbf{પ્રોગ્રામેબિલિટી}: કસ્ટમ નેટવર્ક બિહેવિયર
\item
  \textbf{લવચીકતા}: ઝડપી સેવા ડિપ્લોયમેન્ટ
\end{itemize}

\textbf{મર્યાદાઓ:}

\begin{itemize}
\tightlist
\item
  \textbf{સિંગલ પોઇન્ટ ફેલ્યર}: કંટ્રોલર ડિપેન્ડન્સી
\item
  \textbf{સ્કેલેબિલિટી}: પર્ફોર્મન્સ બોટલનેક્સ
\item
  \textbf{સુરક્ષા}: નવા એટેક વેક્ટર્સ
\end{itemize}

\end{solutionbox}
\begin{mnemonicbox}
``SCAP - સોફ્ટવેર કંટ્રોલ એપ્લિકેશન પ્રોગ્રામેબલ''

\end{mnemonicbox}
\subsection*{પ્રશ્ન 3(અ) [3
ગુણ]}\label{uxaaauxab0uxab6uxaa8-3uxa85-3-uxa97uxaa3}

\textbf{નીચેના IP સરનામાનો વર્ગ શોધો.} \textbf{1) 01111000 00001111
10101010 11000000} \textbf{2) 11101000 01010101 11111111 11000011}

\begin{solutionbox}

\textbf{IP એડ્રેસ વર્ગીકરણ:}

{\def\LTcaptype{none} % do not increment counter
\begin{longtable}[]{@{}llll@{}}
\toprule\noalign{}
બાઇનરી એડ્રેસ & \textbf{ડેસિમલ} & \textbf{પ્રથમ ઓક્ટેટ} & \textbf{વર્ગ} \\
\midrule\noalign{}
\endhead
\bottomrule\noalign{}
\endlastfoot
01111000\ldots{} & 120.15.170.192 & 120 (64-127) & \textbf{વર્ગ A} \\
11101000\ldots{} & 232.85.255.195 & 232 (224-239) & \textbf{વર્ગ D} \\
\end{longtable}
}

\textbf{વર્ગ રેન્જ:}

\begin{itemize}
\tightlist
\item
  \textbf{વર્ગ A}: 1-126 (0xxxxxxx)
\item
  \textbf{વર્ગ B}: 128-191 (10xxxxxx)
\item
  \textbf{વર્ગ C}: 192-223 (110xxxxx)
\item
  \textbf{વર્ગ D}: 224-239 (1110xxxx)
\end{itemize}

\textbf{પરિણામો:}

\begin{itemize}
\tightlist
\item
  \textbf{પ્રથમ IP}: વર્ગ A (યુનિકાસ્ટ)
\item
  \textbf{બીજું IP}: વર્ગ D (મલ્ટિકાસ્ટ)
\end{itemize}

\end{solutionbox}
\begin{mnemonicbox}
``ABCD - A(1-126) B(128-191) C(192-223) D(224-239)''

\end{mnemonicbox}
\subsection*{પ્રશ્ન 3(બ) [4
ગુણ]}\label{uxaaauxab0uxab6uxaa8-3uxaac-4-uxa97uxaa3}

\textbf{IPv4 અને IPv6 વચ્ચે તફાવત આપો.}

\begin{solutionbox}

\textbf{IPv4 vs IPv6 સરખામણી:}

{\def\LTcaptype{none} % do not increment counter
\begin{longtable}[]{@{}lll@{}}
\toprule\noalign{}
લક્ષણ & \textbf{IPv4} & \textbf{IPv6} \\
\midrule\noalign{}
\endhead
\bottomrule\noalign{}
\endlastfoot
\textbf{એડ્રેસ લેન્થ} & 32 બિટ્સ & 128 બિટ્સ \\
\textbf{એડ્રેસ ફોર્મેટ} & ડોટેડ ડેસિમલ & હેક્સાડેસિમલ \\
\textbf{એડ્રેસ સ્પેસ} & 4.3 બિલિયન & 340 અન્ડેસિલિયન \\
\textbf{હેડર સાઇઝ} & વેરિયેબલ (20-60 બાઇટ્સ) & ફિક્સ્ડ (40 બાઇટ્સ) \\
\textbf{સુરક્ષા} & વૈકલ્પિક (IPSec) & બિલ્ટ-ઇન (IPSec) \\
\textbf{કન્ફિગરેશન} & મેન્યુઅલ/DHCP & ઓટો-કન્ફિગરેશન \\
\end{longtable}
}

\textbf{મુખ્ય તફાવતો:}

\begin{itemize}
\tightlist
\item
  \textbf{એડ્રેસિંગ}: IPv6 વધુ વિશાળ એડ્રેસ પ્રદાન કરે છે
\item
  \textbf{સુરક્ષા}: IPv6 માં ફરજિયાત સુરક્ષા લક્ષણો છે
\item
  \textbf{પર્ફોર્મન્સ}: IPv6 માં સરળ હેડર સ્ટ્રક્ચર છે
\end{itemize}

\end{solutionbox}
\begin{mnemonicbox}
``IPv4 થી IPv6 = વધુ એડ્રેસ, બેહતર સુરક્ષા''

\end{mnemonicbox}
\subsection*{પ્રશ્ન 3(ક) [7
ગુણ]}\label{uxaaauxab0uxab6uxaa8-3uxa95-7-uxa97uxaa3}

\textbf{સ્ટેટિક અને ડાયનેમિક રૂટિંગ અલ્ગોરિધમ્સ સમજાવો.}

\begin{solutionbox}

\textbf{સ્ટેટિક રૂટિંગ:}

\includegraphics[width=1\linewidth,height=\textheight,keepaspectratio]{mermaid-eaec18ee.pdf}

\textbf{ડાયનેમિક રૂટિંગ:}

\includegraphics[width=1\linewidth,height=\textheight,keepaspectratio]{mermaid-b8767c43.pdf}

\textbf{સરખામણી ટેબલ:}

{\def\LTcaptype{none} % do not increment counter
\begin{longtable}[]{@{}lll@{}}
\toprule\noalign{}
પાસાં & \textbf{સ્ટેટિક રૂટિંગ} & \textbf{ડાયનેમિક રૂટિંગ} \\
\midrule\noalign{}
\endhead
\bottomrule\noalign{}
\endlastfoot
\textbf{કન્ફિગરેશન} & મેન્યુઅલ સેટઅપ & ઓટોમેટિક ડિસ્કવરી \\
\textbf{અડેપ્ટેબિલિટી} & કોઈ અડેપ્ટેશન નહીં & ફેરફારોને અડેપ્ટ કરે \\
\textbf{રિસોર્સ યુસેજ} & ઓછું CPU/મેમરી & વધારે CPU/મેમરી \\
\textbf{સ્કેલેબિલિટી} & મોટા નેટવર્ક માટે નબળું & મોટા નેટવર્ક માટે સારું \\
\textbf{પ્રોટોકોલ} & કોઈ જરૂરી નહીં & RIP, OSPF, BGP \\
\end{longtable}
}

\textbf{એપ્લિકેશન્સ:}

\begin{itemize}
\tightlist
\item
  \textbf{સ્ટેટિક}: નાના નેટવર્ક્સ, વિશિષ્ટ પાથ્સ
\item
  \textbf{ડાયનેમિક}: મોટા નેટવર્ક્સ, ફોલ્ટ ટોલરન્સ
\end{itemize}

\end{solutionbox}
\begin{mnemonicbox}
``SD - સ્ટેટિક=સિમ્પલ, ડાયનેમિક=ઓટોમેટિક''

\end{mnemonicbox}
\subsection*{પ્રશ્ન 3(અ OR) [3
ગુણ]}\label{uxaaauxab0uxab6uxaa8-3uxa85-or-3-uxa97uxaa3}

\textbf{CIDR સમજાવો.તે પરંપરાગત IP સરનામું ફાળવણી પદ્ધતિઓથી કેવી રીતે અલગ છે?}

\begin{solutionbox}

\textbf{CIDR (ક્લાસલેસ ઇન્ટર-ડોમેન રૂટિંગ):}

\begin{itemize}
\tightlist
\item
  \textbf{કન્સેપ્ટ}: વેરિયેબલ લેન્થ સબનેટ માસ્કિંગ
\item
  \textbf{નોટેશન}: IP એડ્રેસ/પ્રીફિક્સ લેન્થ (દા.ત., 192.168.1.0/24)
\item
  \textbf{લવચીકતા}: કોઈપણ સાઇઝના સબનેટ્સ
\end{itemize}

\textbf{પરંપરાગત vs CIDR:}

{\def\LTcaptype{none} % do not increment counter
\begin{longtable}[]{@{}lll@{}}
\toprule\noalign{}
પદ્ધતિ & \textbf{ફાળવણી} & \textbf{કાર્યક્ષમતા} \\
\midrule\noalign{}
\endhead
\bottomrule\noalign{}
\endlastfoot
\textbf{પરંપરાગત} & ફિક્સ્ડ વર્ગ બાઉન્ડરીઝ & વેસ્ટફુલ (વર્ગ B = 65,536 IPs) \\
\textbf{CIDR} & વેરિયેબલ સબનેટ સાઇઝ & કાર્યક્ષમ ફાળવણી \\
\end{longtable}
}

\textbf{લાભો:}

\begin{itemize}
\tightlist
\item
  \textbf{એડ્રેસ કન્ઝર્વેશન}: IP એડ્રેસ વેસ્ટેજ ઘટાડે છે
\item
  \textbf{રૂટ એગ્રીગેશન}: મલ્ટિપલ રૂટ્સનો સારાંશ આપે છે
\end{itemize}

\end{solutionbox}
\begin{mnemonicbox}
``CIDR = ક્લાસલેસ ઇન્ટેલિજન્ટ એડ્રેસ રૂટિંગ''

\end{mnemonicbox}
\subsection*{પ્રશ્ન 3(બ OR) [4
ગુણ]}\label{uxaaauxab0uxab6uxaa8-3uxaac-or-4-uxa97uxaa3}

\textbf{DSL ટેકનોલોજીના પ્રકારો, ફાયદા અને મર્યાદાઓ નું વર્ણન કરો.}

\begin{solutionbox}

\textbf{DSL (ડિજિટલ સબસ્ક્રાઇબર લાઇન):}

\begin{itemize}
\tightlist
\item
  \textbf{ટેકનોલોજી}: ટેલિફોન લાઇન્સ પર હાઇ-સ્પીડ ઇન્ટરનેટ
\item
  \textbf{ફ્રીક્વન્સી}: વોઇસ કરતાં વધારે ફ્રીક્વન્સીનો ઉપયોગ
\end{itemize}

\textbf{DSL પ્રકારો:}

{\def\LTcaptype{none} % do not increment counter
\begin{longtable}[]{@{}lll@{}}
\toprule\noalign{}
પ્રકાર & \textbf{સ્પીડ} & \textbf{એપ્લિકેશન} \\
\midrule\noalign{}
\endhead
\bottomrule\noalign{}
\endlastfoot
\textbf{ADSL} & એસિમેટ્રિક (ઝડપી ડાઉનલોડ) & ઘર વપરાશકર્તાઓ \\
\textbf{SDSL} & સિમેટ્રિક (સમાન અપ/ડાઉન) & બિઝનેસ \\
\textbf{VDSL} & અત્યંત ઉચ્ચ ગતિ & ટૂંકા અંતર \\
\end{longtable}
}

\textbf{ફાયદાઓ:}

\begin{itemize}
\tightlist
\item
  \textbf{હંમેશા-ઓન કનેક્શન}: ડાયલ-અપની જરૂર નહીં
\item
  \textbf{હાલનું ઇન્ફ્રાસ્ટ્રક્ચર}: ફોન લાઇન્સનો ઉપયોગ
\item
  \textbf{કિફાયતી}: પોસાય તેવી હાઇ-સ્પીડ એક્સેસ
\end{itemize}

\textbf{મર્યાદાઓ:}

\begin{itemize}
\tightlist
\item
  \textbf{અંતર આધારિત}: અંતર વધે તો સ્પીડ ઘટે
\item
  \textbf{લાઇન ક્વોલિટી}: સારી કોપર લાઇન્સની જરૂર
\item
  \textbf{ઉપલબ્ધતા}: બધે ઉપલબ્ધ નથી
\end{itemize}

\end{solutionbox}
\begin{mnemonicbox}
``DSL = ડિજિટલ સ્પીડ અંતરથી મર્યાદિત''

\end{mnemonicbox}
\subsection*{પ્રશ્ન 3(ક OR) [7
ગુણ]}\label{uxaaauxab0uxab6uxaa8-3uxa95-or-7-uxa97uxaa3}

\textbf{ડેટા લિંક લેયર પર error control અને flow control વિસ્તરવાર સમજાવો.}

\begin{solutionbox}

\textbf{એરર કંટ્રોલ:}

\includegraphics[width=1\linewidth,height=\textheight,keepaspectratio]{mermaid-5d7af214.pdf}

\textbf{એરર કંટ્રોલ પદ્ધતિઓ:}

{\def\LTcaptype{none} % do not increment counter
\begin{longtable}[]{@{}lll@{}}
\toprule\noalign{}
પદ્ધતિ & \textbf{ટેકનીક} & \textbf{એપ્લિકેશન} \\
\midrule\noalign{}
\endhead
\bottomrule\noalign{}
\endlastfoot
\textbf{પેરિટી ચેક} & સિંગલ બિટ એરર ડિટેક્શન & સિમ્પલ સિસ્ટમ્સ \\
\textbf{ચેકસમ} & ગાણિતિક સરવાળો વેરિફિકેશન & TCP/UDP \\
\textbf{CRC} & પોલિનોમિયલ ડિવિઝન & Ethernet, Wi-Fi \\
\textbf{ARQ} & ઓટોમેટિક રિપીટ રિક્વેસ્ટ & વિશ્વસનીય પ્રોટોકોલ્સ \\
\end{longtable}
}

\textbf{ફ્લો કંટ્રોલ:}

\includegraphics[width=1\linewidth,height=\textheight,keepaspectratio]{mermaid-58128a72.pdf}

\textbf{ફ્લો કંટ્રોલ ટેકનીક્સ:}

\begin{itemize}
\tightlist
\item
  \textbf{સ્ટોપ-એન્ડ-વેઇટ}: એક ફ્રેમ મોકલો, ACK ની રાહ જુઓ
\item
  \textbf{સ્લાઇડિંગ વિન્ડો}: મલ્ટિપલ ફ્રેમ્સ ટ્રાન્ઝિટમાં
\item
  \textbf{બફર મેનેજમેન્ટ}: ઓવરફ્લો અટકાવે છે
\end{itemize}

\textbf{ઇમ્પ્લિમેન્ટેશન:}

\begin{itemize}
\tightlist
\item
  \textbf{હાર્ડવેર લેવલ}: બફર સ્ટેટસ સિગ્નલ્સ
\item
  \textbf{સોફ્ટવેર લેવલ}: પ્રોટોકોલ એકનોલેજમેન્ટ્સ
\end{itemize}

\end{solutionbox}
\begin{mnemonicbox}
``EF - એરર ડિટેક્શન, ફ્લો રેગ્યુલેશન''

\end{mnemonicbox}
\subsection*{પ્રશ્ન 4(અ) [3
ગુણ]}\label{uxaaauxab0uxab6uxaa8-4uxa85-3-uxa97uxaa3}

\textbf{Video over IP સમજાવો.}

\begin{solutionbox}

\textbf{વિડિયો ઓવર IP (VoIP):}

\begin{itemize}
\tightlist
\item
  \textbf{ટેકનોલોજી}: IP નેટવર્ક્સ પર વિડિયો સિગ્નલ્સ ટ્રાન્સમિટ કરે છે
\item
  \textbf{ડિજિટાઇઝેશન}: એનાલોગ વિડિયોને ડિજિટલ પેકેટ્સમાં કન્વર્ટ કરે છે
\item
  \textbf{રિયલ-ટાઇમ}: ઓછી લેટન્સી ટ્રાન્સમિશનની જરૂર
\end{itemize}

\textbf{કમ્પોનન્ટ્સ:}

\begin{itemize}
\tightlist
\item
  \textbf{એનકોડર}: વિડિયો ડેટાને કમ્પ્રેસ કરે છે
\item
  \textbf{નેટવર્ક}: ટ્રાન્સપોર્ટ માટે IP ઇન્ફ્રાસ્ટ્રક્ચર
\item
  \textbf{ડીકોડર}: ડેસ્ટિનેશન પર ડીકમ્પ્રેસ કરે છે
\end{itemize}

\textbf{એપ્લિકેશન્સ:}

\begin{itemize}
\tightlist
\item
  \textbf{વિડિયો કોન્ફરન્સિંગ}: બિઝનેસ કમ્યુનિકેશન્સ
\item
  \textbf{સ્ટ્રીમિંગ}: મનોરંજન સેવાઓ
\item
  \textbf{સર્વેલન્સ}: સુરક્ષા સિસ્ટમ્સ
\end{itemize}

\textbf{જરૂરિયાતો:}

\begin{itemize}
\tightlist
\item
  \textbf{બેન્ડવિડ્થ}: ઉચ્ચ ડેટા રેટની જરૂર
\item
  \textbf{QoS}: કોવાલિટી ઓફ સર્વિસ ગેરંટીઝ
\end{itemize}

\end{solutionbox}
\begin{mnemonicbox}
``VIP = વિડિયો ઇન્ટરનેટ પ્રોટોકોલ''

\end{mnemonicbox}
\subsection*{પ્રશ્ન 4(બ) [4
ગુણ]}\label{uxaaauxab0uxab6uxaa8-4uxaac-4-uxa97uxaa3}

\textbf{ઇલેક્ટ્રોનિક-મેઇલ તેના પ્રોટોકોલ સાથે સમજાવો.}

\begin{solutionbox}

\textbf{ઇમેઇલ સિસ્ટમ કમ્પોનન્ટ્સ:}

\includegraphics[width=1\linewidth,height=\textheight,keepaspectratio]{mermaid-523d5d31.pdf}

\textbf{ઇમેઇલ પ્રોટોકોલ્સ:}

{\def\LTcaptype{none} % do not increment counter
\begin{longtable}[]{@{}lll@{}}
\toprule\noalign{}
પ્રોટોકોલ & \textbf{કાર્ય} & \textbf{પોર્ટ} \\
\midrule\noalign{}
\endhead
\bottomrule\noalign{}
\endlastfoot
\textbf{SMTP} & મેસેજ મોકલો/રિલે કરો & 25, 587 \\
\textbf{POP3} & મેસેજ ડાઉનલોડ કરો & 110 \\
\textbf{IMAP} & સર્વર-બેસ્ડ એક્સેસ & 143 \\
\end{longtable}
}

\textbf{પ્રોટોકોલ વિગતો:}

\begin{itemize}
\tightlist
\item
  \textbf{SMTP}: મોકલવા માટે સિમ્પલ મેઇલ ટ્રાન્સફર પ્રોટોકોલ
\item
  \textbf{POP3}: લોકલ ડિવાઇસ પર મેઇલ ડાઉનલોડ કરે છે
\item
  \textbf{IMAP}: મેઇલ સર્વર પર રાખે છે, મલ્ટિ-ડિવાઇસ એક્સેસ
\end{itemize}

\textbf{મેસેજ ફ્લો:}

\begin{itemize}
\tightlist
\item
  \textbf{કમ્પોઝિશન}: યુઝર મેસેજ બનાવે છે
\item
  \textbf{સબમિશન}: SMTP સર્વર પર મોકલે છે
\item
  \textbf{ડિલિવરી}: સર્વર પ્રાપ્તકર્તા પર ફોરવર્ડ કરે છે
\item
  \textbf{રિટ્રીવલ}: POP3/IMAP મેસેજ ડાઉનલોડ કરે છે
\end{itemize}

\end{solutionbox}
\begin{mnemonicbox}
``SPI - SMTP મોકલે, POP3/IMAP મેળવે''

\end{mnemonicbox}
\subsection*{પ્રશ્ન 4(ક) [7
ગુણ]}\label{uxaaauxab0uxab6uxaa8-4uxa95-7-uxa97uxaa3}

\textbf{DNS- ડોમેન-નેમ સિસ્ટમની ભૂમિકા સમજાવો DNS રિઝોલ્યુશનની પ્રક્રિયાનું વર્ણન
કરો.}

\begin{solutionbox}

\textbf{DNS ભૂમિકા:}

\begin{itemize}
\tightlist
\item
  \textbf{નેમ રિઝોલ્યુશન}: ડોમેન નેમ્સને IP એડ્રેસમાં કન્વર્ટ કરે છે
\item
  \textbf{હાયરાર્કિકલ સિસ્ટમ}: વિતરિત ડેટાબેસ સ્ટ્રક્ચર
\item
  \textbf{ઇન્ટરનેટ નેવિગેશન}: વેબ બ્રાઉઝિંગને યુઝર-ફ્રેન્ડલી બનાવે છે
\end{itemize}

\textbf{DNS રિઝોલ્યુશન પ્રક્રિયા:}

\includegraphics[width=1\linewidth,height=\textheight,keepaspectratio]{mermaid-6538d2ca.pdf}

\textbf{રિઝોલ્યુશન સ્ટેપ્સ:}

\begin{enumerate}
\tightlist
\item
  \textbf{લોકલ કેશ ચેક}: લોકલ DNS કેશ ચેક કરો
\item
  \textbf{રિકર્સિવ ક્વેરી}: લોકલ DNS સર્વરનો સંપર્ક કરો
\item
  \textbf{રૂટ સર્વર}: TLD સર્વર રેફરન્સ મેળવો
\item
  \textbf{TLD સર્વર}: ઓથોરિટેટિવ સર્વર રેફરન્સ મેળવો
\item
  \textbf{ઓથોરિટેટિવ સર્વર}: અંતિમ IP એડ્રેસ મેળવો
\item
  \textbf{રિસ્પોન્સ રિટર્ન}: ક્લાયંટને IP એડ્રેસ પરત કરો
\end{enumerate}

\textbf{DNS રેકોર્ડ પ્રકારો:}

\begin{itemize}
\tightlist
\item
  \textbf{A રેકોર્ડ}: નેમને IPv4 એડ્રેસ સાથે મેપ કરે છે
\item
  \textbf{AAAA રેકોર્ડ}: નેમને IPv6 એડ્રેસ સાથે મેપ કરે છે
\item
  \textbf{CNAME}: કેનોનિકલ નેમ એલિયાસ
\item
  \textbf{MX}: મેઇલ એક્સચેન્જ સર્વર
\end{itemize}

\textbf{લાભો:}

\begin{itemize}
\tightlist
\item
  \textbf{યુઝર ફ્રેન્ડલી}: નંબર્સ નહીં, નેમ્સ યાદ રાખો
\item
  \textbf{લોડ ડિસ્ટ્રિબ્યુશન}: મલ્ટિપલ IP એડ્રેસ
\item
  \textbf{સર્વિસ લોકેશન}: વિશિષ્ટ સેવાઓ શોધો
\end{itemize}

\end{solutionbox}
\begin{mnemonicbox}
``DNS = ડાયરેક્ટરી નેમ સર્વિસ''

\end{mnemonicbox}
\subsection*{પ્રશ્ન 4(અ OR) [3
ગુણ]}\label{uxaaauxab0uxab6uxaa8-4uxa85-or-3-uxa97uxaa3}

\textbf{WWW, HTML સમજાવો.}

\begin{solutionbox}

\textbf{WWW (વર્લ્ડ વાઇડ વેબ):}

\begin{itemize}
\tightlist
\item
  \textbf{વ્યાખ્યા}: ઇન્ટરલિંક્ડ ડોક્યુમેન્ટ્સની માહિતી સિસ્ટમ
\item
  \textbf{એક્સેસ}: HTTP ઉપયોગ કરીને વેબ બ્રાઉઝર દ્વારા
\item
  \textbf{કમ્પોનન્ટ્સ}: વેબ પેજ, લિંક્સ, URLs
\end{itemize}

\textbf{HTML (હાઇપરટેક્સ્ટ માર્કઅપ લેંગ્વેજ):}

\begin{itemize}
\tightlist
\item
  \textbf{હેતુ}: વેબ પેજ માટે સ્ટાન્ડર્ડ માર્કઅપ લેંગ્વેજ
\item
  \textbf{સ્ટ્રક્ચર}: ટેગ્સ ડોક્યુમેન્ટ એલિમેન્ટ્સ વ્યાખ્યાયિત કરે છે
\item
  \textbf{હાઇપરલિંક્સ}: વિવિધ વેબ રિસોર્સ કનેક્ટ કરે છે
\end{itemize}

\textbf{સંબંધ:}

\begin{itemize}
\tightlist
\item
  \textbf{WWW}: સિસ્ટમ/પ્લેટફોર્મ
\item
  \textbf{HTML}: કન્ટેન્ટ ફોર્મેટ
\item
  \textbf{ઇન્ટિગ્રેશન}: HTML WWW કન્ટેન્ટ બનાવે છે
\end{itemize}

\end{solutionbox}
\begin{mnemonicbox}
``WWW કન્ટેન્ટ માટે HTML નો ઉપયોગ કરે છે''

\end{mnemonicbox}
\subsection*{પ્રશ્ન 4(બ OR) [4
ગુણ]}\label{uxaaauxab0uxab6uxaa8-4uxaac-or-4-uxa97uxaa3}

\textbf{HTTP અને FTP સમજાવો.}

\begin{solutionbox}

\textbf{પ્રોટોકોલ સરખામણી:}

{\def\LTcaptype{none} % do not increment counter
\begin{longtable}[]{@{}lll@{}}
\toprule\noalign{}
લક્ષણ & \textbf{HTTP} & \textbf{FTP} \\
\midrule\noalign{}
\endhead
\bottomrule\noalign{}
\endlastfoot
\textbf{હેતુ} & વેબ પેજ ટ્રાન્સફર & ફાઇલ ટ્રાન્સફર \\
\textbf{પોર્ટ} & 80 (HTTP), 443 (HTTPS) & 21 (કંટ્રોલ), 20 (ડેટા) \\
\textbf{કનેક્શન} & સ્ટેટલેસ & સ્ટેટફુલ \\
\textbf{સુરક્ષા} & સુરક્ષા માટે HTTPS & સુરક્ષા માટે FTPS \\
\end{longtable}
}

\textbf{HTTP (હાઇપરટેક્સ્ટ ટ્રાન્સફર પ્રોટોકોલ):}

\begin{itemize}
\tightlist
\item
  \textbf{કાર્ય}: વેબ માટે રિક્વેસ્ટ-રિસ્પોન્સ પ્રોટોકોલ
\item
  \textbf{મેથડ્સ}: GET, POST, PUT, DELETE
\item
  \textbf{સ્ટેટલેસ}: દરેક રિક્વેસ્ટ સ્વતંત્ર
\end{itemize}

\textbf{FTP (ફાઇલ ટ્રાન્સફર પ્રોટોકોલ):}

\begin{itemize}
\tightlist
\item
  \textbf{કાર્ય}: સિસ્ટમ્સ વચ્ચે ફાઇલો અપલોડ/ડાઉનલોડ
\item
  \textbf{મોડ્સ}: એક્ટિવ અને પેસિવ
\item
  \textbf{ઓથેન્ટિકેશન}: યુઝરનેમ/પાસવર્ડ જરૂરી
\end{itemize}

\textbf{એપ્લિકેશન્સ:}

\begin{itemize}
\tightlist
\item
  \textbf{HTTP}: વેબ બ્રાઉઝિંગ, API કોલ્સ
\item
  \textbf{FTP}: ફાઇલ શેરિંગ, વેબસાઇટ મેઇન્ટેનન્સ
\end{itemize}

\end{solutionbox}
\begin{mnemonicbox}
``HF - HTTP હાઇપરટેક્સ્ટ માટે, FTP ફાઇલો માટે''

\end{mnemonicbox}
\subsection*{પ્રશ્ન 4(ક OR) [7
ગુણ]}\label{uxaaauxab0uxab6uxaa8-4uxa95-or-7-uxa97uxaa3}

\textbf{કનેક્શન ઓરિએન્ટેડ અને કનેક્શન લેસ નેટવર્કના સંબંધમાં ટ્રાન્સપોર્ટ લેયરમાં TCP અને
UDP પ્રોટોકોલ સમજાવો.}

\begin{solutionbox}

\textbf{ટ્રાન્સપોર્ટ લેયર પ્રોટોકોલ્સ:}

\includegraphics[width=1\linewidth,height=\textheight,keepaspectratio]{mermaid-9aedcb98.pdf}

\textbf{પ્રોટોકોલ સરખામણી:}

{\def\LTcaptype{none} % do not increment counter
\begin{longtable}[]{@{}lll@{}}
\toprule\noalign{}
લક્ષણ & \textbf{TCP} & \textbf{UDP} \\
\midrule\noalign{}
\endhead
\bottomrule\noalign{}
\endlastfoot
\textbf{કનેક્શન} & કનેક્શન-ઓરિએન્ટેડ & કનેક્શનલેસ \\
\textbf{વિશ્વસનીયતા} & ગેરંટીડ ડિલિવરી & બેસ્ટ એફર્ટ \\
\textbf{સ્પીડ} & ધીમું (ઓવરહેડ) & ઝડપી (મિનિમલ ઓવરહેડ) \\
\textbf{હેડર સાઇઝ} & 20 બાઇટ્સ & 8 બાઇટ્સ \\
\textbf{ફ્લો કંટ્રોલ} & હા & ના \\
\textbf{એરર કંટ્રોલ} & હા & મર્યાદિત \\
\end{longtable}
}

\textbf{TCP (ટ્રાન્સમિશન કંટ્રોલ પ્રોટોકોલ):}

\begin{itemize}
\tightlist
\item
  \textbf{થ્રી-વે હેન્ડશેક}: SYN, SYN-ACK, ACK
\item
  \textbf{વિશ્વસનીય}: એકનોલેજમેન્ટ અને રીટ્રાન્સમિશન
\item
  \textbf{ફ્લો કંટ્રોલ}: બફર ઓવરફ્લો અટકાવે છે
\item
  \textbf{એપ્લિકેશન્સ}: વેબ બ્રાઉઝિંગ, ઇમેઇલ, ફાઇલ ટ્રાન્સફર
\end{itemize}

\textbf{UDP (યુઝર ડેટાગ્રામ પ્રોટોકોલ):}

\begin{itemize}
\tightlist
\item
  \textbf{કોઈ કનેક્શન સેટઅપ નહીં}: સીધું ડેટા ટ્રાન્સમિશન
\item
  \textbf{લાઇટવેઇટ}: મિનિમલ પ્રોટોકોલ ઓવરહેડ
\item
  \textbf{કોઈ ગેરંટી નહીં}: ફાયર-એન્ડ-ફોરગેટ એપ્રોચ
\item
  \textbf{એપ્લિકેશન્સ}: વિડિયો સ્ટ્રીમિંગ, DNS, ગેમિંગ
\end{itemize}

\textbf{કનેક્શન મોડલ્સ:}

\begin{itemize}
\tightlist
\item
  \textbf{કનેક્શન-ઓરિએન્ટેડ}: સ્થાપિત, ટ્રાન્સફર, સમાપ્ત
\item
  \textbf{કનેક્શનલેસ}: સેટઅપ વિના સીધું ટ્રાન્સમિશન
\end{itemize}

\textbf{સિલેક્શન માપદંડ:}

\begin{itemize}
\tightlist
\item
  \textbf{TCP ઉપયોગ કરો}: જ્યારે વિશ્વસનીયતા મહત્વપૂર્ણ હોય
\item
  \textbf{UDP ઉપયોગ કરો}: જ્યારે સ્પીડ વધુ મહત્વપૂર્ણ હોય
\end{itemize}

\end{solutionbox}
\begin{mnemonicbox}
``TCP = સંપૂર્ણ, UDP = અલ્ટ્રા-ફાસ્ટ''

\end{mnemonicbox}
\subsection*{પ્રશ્ન 5(અ) [3
ગુણ]}\label{uxaaauxab0uxab6uxaa8-5uxa85-3-uxa97uxaa3}

\textbf{હેકિંગ અને તેની સંબંધિત સાવચેતીઓનું વર્ણન કરો.}

\begin{solutionbox}

\textbf{હેકિંગ વ્યાખ્યા:}

\begin{itemize}
\tightlist
\item
  \textbf{અનધિકૃત પ્રવેશ}: કમ્પ્યુટર સિસ્ટમમાં પ્રવેશ
\item
  \textbf{દુષ્ટ હેતુ}: ડેટા ચોરી, સુધારો અથવા નાશ
\item
  \textbf{સુરક્ષા ભંગ}: સિસ્ટમ નબળાઈઓનો ગેરફાયદો
\end{itemize}

\textbf{હેકિંગના પ્રકારો:}

\begin{itemize}
\tightlist
\item
  \textbf{એથિકલ હેકિંગ}: અધિકૃત સુરક્ષા પરીક્ષણ
\item
  \textbf{મેલિશિયસ હેકિંગ}: ગુનાહિત પ્રવૃત્તિઓ
\item
  \textbf{સોશિયલ એન્જિનિયરિંગ}: માનવીય વર્તણૂકની હેરાફેરી
\end{itemize}

\textbf{સાવચેતીઓ:}

{\def\LTcaptype{none} % do not increment counter
\begin{longtable}[]{@{}ll@{}}
\toprule\noalign{}
સુરક્ષા માપ & \textbf{અમલીકરણ} \\
\midrule\noalign{}
\endhead
\bottomrule\noalign{}
\endlastfoot
\textbf{મજબૂત પાસવર્ડ} & જટિલ, અનન્ય પાસવર્ડ \\
\textbf{સોફ્ટવેર અપડેટ્સ} & નિયમિત પેચ અને અપડેટ્સ \\
\textbf{ફાયરવોલ} & નેટવર્ક એક્સેસ કંટ્રોલ \\
\textbf{એન્ટિવાયરસ} & મેલવેર ડિટેક્શન અને દૂર કરવું \\
\textbf{બેકઅપ} & નિયમિત ડેટા બેકઅપ \\
\textbf{યુઝર ટ્રેનિંગ} & સુરક્ષા જાગરૂકતા કાર્યક્રમો \\
\end{longtable}
}

\end{solutionbox}
\begin{mnemonicbox}
``HSPFAB - હેકિંગ પાસવર્ડ, ફાયરવોલ, એન્ટિવાયરસ, બેકઅપથી
અટકાવાય''

\end{mnemonicbox}
\subsection*{પ્રશ્ન 5(બ) [4
ગુણ]}\label{uxaaauxab0uxab6uxaa8-5uxaac-4-uxa97uxaa3}

\textbf{IPSec આર્કિટેક્ચર સમજાવો.}

\begin{solutionbox}

\textbf{IPSec (ઇન્ટરનેટ પ્રોટોકોલ સિક્યુરિટી):}

\includegraphics[width=1\linewidth,height=\textheight,keepaspectratio]{mermaid-99e88af9.pdf}

\textbf{IPSec કમ્પોનન્ટ્સ:}

{\def\LTcaptype{none} % do not increment counter
\begin{longtable}[]{@{}ll@{}}
\toprule\noalign{}
કમ્પોનન્ટ & \textbf{કાર્ય} \\
\midrule\noalign{}
\endhead
\bottomrule\noalign{}
\endlastfoot
\textbf{AH} & ઓથેન્ટિકેશન અને ઇન્ટેગ્રિટી \\
\textbf{ESP} & ગુપ્તતા અને ઓથેન્ટિકેશન \\
\textbf{SA} & સુરક્ષા પેરામીટર એગ્રીમેન્ટ \\
\textbf{IKE} & કી મેનેજમેન્ટ પ્રોટોકોલ \\
\end{longtable}
}

\textbf{ઓપરેટિંગ મોડ્સ:}

\begin{itemize}
\tightlist
\item
  \textbf{ટ્રાન્સપોર્ટ મોડ}: માત્ર પેલોડને સુરક્ષા આપે છે
\item
  \textbf{ટનલ મોડ}: સંપૂર્ણ IP પેકેટને સુરક્ષા આપે છે
\end{itemize}

\textbf{સુરક્ષા સેવાઓ:}

\begin{itemize}
\tightlist
\item
  \textbf{ઓથેન્ટિકેશન}: મોકલનારની ઓળખ ચકાસો
\item
  \textbf{ઇન્ટેગ્રિટી}: ડેટા અપરિવર્તિત છે તેની ખાતરી
\item
  \textbf{ગુપ્તતા}: ડેટા કન્ટેન્ટ એન્ક્રિપ્ટ કરો
\item
  \textbf{એન્ટિ-રિપ્લે}: પેકેટ રિપ્લે એટેક અટકાવો
\end{itemize}

\end{solutionbox}
\begin{mnemonicbox}
``AISE - AH, IPSec, SA, ESP''

\end{mnemonicbox}
\subsection*{પ્રશ્ન 5(ક) [7
ગુણ]}\label{uxaaauxab0uxab6uxaa8-5uxa95-7-uxa97uxaa3}

\textbf{નેટવર્ક સુરક્ષા ટોપોલોજી સમજાવો.}

\begin{solutionbox}

\textbf{નેટવર્ક સુરક્ષા ટોપોલોજીઓ:}

\includegraphics[width=1\linewidth,height=\textheight,keepaspectratio]{mermaid-c0d3a186.pdf}

\textbf{સુરક્ષા ઝોન્સ:}

{\def\LTcaptype{none} % do not increment counter
\begin{longtable}[]{@{}lll@{}}
\toprule\noalign{}
ઝોન & \textbf{હેતુ} & \textbf{સુરક્ષા સ્તર} \\
\midrule\noalign{}
\endhead
\bottomrule\noalign{}
\endlastfoot
\textbf{ઇન્ટરનેટ} & બાહ્ય અવિશ્વસનીય નેટવર્ક & સૌથી ઓછું \\
\textbf{DMZ} & સેમિ-ટ્રસ્ટેડ પબ્લિક સેવાઓ & મધ્યમ \\
\textbf{આંતરિક} & ખાનગી વિશ્વસનીય નેટવર્ક & સૌથી વધુ \\
\end{longtable}
}

\textbf{ટોપોલોજી કમ્પોનન્ટ્સ:}

\begin{itemize}
\tightlist
\item
  \textbf{પેરિમીટર સિક્યુરિટી}: ફાયરવોલ, IDS/IPS
\item
  \textbf{નેટવર્ક સેગમેન્ટેશન}: VLANs, સબનેટ્સ
\item
  \textbf{એક્સેસ કંટ્રોલ}: ઓથેન્ટિકેશન, ઓથોરાઇઝેશન
\item
  \textbf{મોનિટરિંગ}: લોગિંગ, SIEM સિસ્ટમ્સ
\end{itemize}

\textbf{સુરક્ષા સિદ્ધાન્તો:}

\begin{itemize}
\tightlist
\item
  \textbf{ડિફેન્સ ઇન ડેપ્થ}: મલ્ટિપલ સુરક્ષા સ્તરો
\item
  \textbf{લીસ્ટ પ્રિવિલેજ}: મિનિમમ જરૂરી એક્સેસ
\item
  \textbf{નેટવર્ક આઇસોલેશન}: ક્રિટિકલ સિસ્ટમ્સ અલગ કરો
\end{itemize}

\textbf{અમલીકરણ વ્યૂહરચનાઓ:}

\begin{itemize}
\tightlist
\item
  \textbf{ફાયરવોલ નિયમો}: ટ્રાફિક ફ્લો કંટ્રોલ કરો
\item
  \textbf{VPN એક્સેસ}: સુરક્ષિત રિમોટ કનેક્શન્સ
\item
  \textbf{નેટવર્ક મોનિટરિંગ}: ધમકીઓ શોધો
\item
  \textbf{ઇન્સિડન્ટ રિસ્પોન્સ}: સુરક્ષા ઘટનાઓ હેન્ડલ કરો
\end{itemize}

\textbf{લાભો:}

\begin{itemize}
\tightlist
\item
  \textbf{રિસ્ક રિડક્શન}: એટેક સર્ફેસ મિનિમાઇઝ કરો
\item
  \textbf{કમ્પ્લાયન્સ}: નિયમન જરૂરિયાતો પૂરી કરો
\item
  \textbf{બિઝનેસ કન્ટિન્યુટી}: ઓપરેશન્સને સુરક્ષા આપો
\end{itemize}

\end{solutionbox}
\begin{mnemonicbox}
``NST = નેટવર્ક સિક્યુરિટી ટોપોલોજી ડિઝાઇન દ્વારા''

\end{mnemonicbox}
\subsection*{પ્રશ્ન 5(અ OR) [3
ગુણ]}\label{uxaaauxab0uxab6uxaa8-5uxa85-or-3-uxa97uxaa3}

\textbf{ISO સમજાવો અને તે માહિતી સુરક્ષામાં કેવી રીતે ફાળો આપે છે?}

\begin{solutionbox}

\textbf{ISO (ઇન્ટરનેશનલ ઓર્ગેનાઇઝેશન ફોર સ્ટેન્ડર્ડાઇઝેશન):}

\begin{itemize}
\tightlist
\item
  \textbf{ગ્લોબલ સ્ટેન્ડર્ડ્સ}: આંતરરાષ્ટ્રીય ધોરણો વિકસાવે છે
\item
  \textbf{ક્વોલિટી એશ્યુરન્સ}: સતત પ્રથાઓની ખાતરી કરે છે
\item
  \textbf{બેસ્ટ પ્રેક્ટિસિસ}: અમલીકરણ માટે ફ્રેમવર્ક પ્રદાન કરે છે
\end{itemize}

\textbf{ISO 27001 - ઇન્ફર્મેશન સિક્યુરિટી:}

\begin{itemize}
\tightlist
\item
  \textbf{ISMS}: ઇન્ફર્મેશન સિક્યુરિટી મેનેજમેન્ટ સિસ્ટમ
\item
  \textbf{રિસ્ક મેનેજમેન્ટ}: સુરક્ષા માટે વ્યવસ્થિત અભિગમ
\item
  \textbf{સતત સુધારણા}: નિયમિત સમીક્ષા અને અપડેટ્સ
\end{itemize}

\textbf{ઇન્ફર્મેશન સિક્યુરિટીમાં યોગદાન:}

\begin{itemize}
\tightlist
\item
  \textbf{ફ્રેમવર્ક}: સુરક્ષા માટે સંરચિત અભિગમ
\item
  \textbf{કમ્પ્લાયન્સ}: નિયમન જરૂરિયાતો પૂરી કરો
\item
  \textbf{રિસ્ક એસેસમેન્ટ}: ધમકીઓ ઓળખો અને ઘટાડો
\end{itemize}

\textbf{લાભો:}

\begin{itemize}
\tightlist
\item
  \textbf{સ્ટેન્ડર્ડાઇઝેશન}: સામાન્ય સુરક્ષા ભાષા
\item
  \textbf{વિશ્વસનીયતા}: આંતરરાષ્ટ્રીય માન્યતા
\item
  \textbf{સુધારણા}: ચાલુ સુરક્ષા વૃદ્ધિ
\end{itemize}

\end{solutionbox}
\begin{mnemonicbox}
``ISO = ઇન્ટરનેશનલ સિક્યુરિટી ઓર્ગેનાઇઝેશન''

\end{mnemonicbox}
\subsection*{પ્રશ્ન 5(બ OR) [4
ગુણ]}\label{uxaaauxab0uxab6uxaa8-5uxaac-or-4-uxa97uxaa3}

\textbf{સમપ્રમાણ અને અસમપ્રમાણ એન્ક્રિપ્શન અલ્ગોરિધમ્સ વચ્ચે તફાવત આપો.}

\begin{solutionbox}

\textbf{એન્ક્રિપ્શન અલ્ગોરિધમ સરખામણી:}

{\def\LTcaptype{none} % do not increment counter
\begin{longtable}[]{@{}lll@{}}
\toprule\noalign{}
લક્ષણ & \textbf{સમપ્રમાણ} & \textbf{અસમપ્રમાણ} \\
\midrule\noalign{}
\endhead
\bottomrule\noalign{}
\endlastfoot
\textbf{કીઓ} & સિંગલ શેર્ડ કી & કી પેર (પબ્લિક/પ્રાઇવેટ) \\
\textbf{સ્પીડ} & ઝડપી & ધીમું \\
\textbf{કી ડિસ્ટ્રિબ્યુશન} & મુશ્કેલ & સરળ \\
\textbf{સ્કેલેબિલિટી} & નબળું (n^{2}-1 કીઓ) & બેહતર \\
\textbf{સુરક્ષા} & કી ગુપ્તતા પર આધાર & ગાણિતિક જટિલતા \\
\end{longtable}
}

\textbf{સમપ્રમાણ એન્ક્રિપ્શન:}

\begin{itemize}
\tightlist
\item
  \textbf{ઉદાહરણો}: AES, DES, 3DES
\item
  \textbf{પ્રક્રિયા}: સમાન કી એન્ક્રિપ્ટ અને ડિક્રિપ્ટ કરે છે
\item
  \textbf{પડકાર}: સુરક્ષિત કી ડિસ્ટ્રિબ્યુશન
\end{itemize}

\textbf{અસમપ્રમાણ એન્ક્રિપ્શન:}

\begin{itemize}
\tightlist
\item
  \textbf{ઉદાહરણો}: RSA, ECC, Diffie-Hellman
\item
  \textbf{પ્રક્રિયા}: પબ્લિક કી એન્ક્રિપ્ટ કરે, પ્રાઇવેટ કી ડિક્રિપ્ટ કરે
\item
  \textbf{ફાયદો}: કી ડિસ્ટ્રિબ્યુશન સમસ્યા નથી
\end{itemize}

\textbf{હાઇબ્રિડ અભિગમ:}

\begin{itemize}
\tightlist
\item
  \textbf{કોમ્બિનેશન}: બંને પ્રકારનો સાથે ઉપયોગ
\item
  \textbf{પદ્ધતિ}: કી એક્સચેન્જ માટે અસમપ્રમાણ, ડેટા માટે સમપ્રમાણ
\end{itemize}

\textbf{એપ્લિકેશન્સ:}

\begin{itemize}
\tightlist
\item
  \textbf{સમપ્રમાણ}: બલ્ક ડેટા એન્ક્રિપ્શન
\item
  \textbf{અસમપ્રમાણ}: ડિજિટલ સિગ્નેચર, કી એક્સચેન્જ
\end{itemize}

\end{solutionbox}
\begin{mnemonicbox}
``SA = સમપ્રમાણ શેર્ડ, અસમપ્રમાણ અલગ''

\end{mnemonicbox}
\subsection*{પ્રશ્ન 5(ક OR) [7
ગુણ]}\label{uxaaauxab0uxab6uxaa8-5uxa95-or-7-uxa97uxaa3}

\textbf{IEmail સુરક્ષાને તેના standards સાથે સમજાવો.}

\begin{solutionbox}

\textbf{ઇમેઇલ સુરક્ષા પડકારો:}

\includegraphics[width=1\linewidth,height=\textheight,keepaspectratio]{mermaid-edea92ee.pdf}

\textbf{ઇમેઇલ સુરક્ષા સ્ટેન્ડર્ડ્સ:}

{\def\LTcaptype{none} % do not increment counter
\begin{longtable}[]{@{}lll@{}}
\toprule\noalign{}
સ્ટેન્ડર્ડ & \textbf{હેતુ} & \textbf{કાર્ય} \\
\midrule\noalign{}
\endhead
\bottomrule\noalign{}
\endlastfoot
\textbf{S/MIME} & સુરક્ષિત ઇમેઇલ કન્ટેન્ટ & એન્ક્રિપ્શન અને ડિજિટલ સિગ્નેચર \\
\textbf{PGP} & પ્રિટી ગુડ પ્રાઇવસી & એન્ડ-ટુ-એન્ડ એન્ક્રિપ્શન \\
\textbf{TLS} & ટ્રાન્સપોર્ટ સુરક્ષા & સુરક્ષિત ઇમેઇલ ટ્રાન્સમિશન \\
\textbf{SPF} & સેન્ડર ઓથેન્ટિકેશન & ઇમેઇલ સ્પૂફિંગ અટકાવો \\
\textbf{DKIM} & મેસેજ ઇન્ટેગ્રિટી & ડિજિટલ સિગ્નેચર વેરિફિકેશન \\
\textbf{DMARC} & પોલિસી એન્ફોર્સમેન્ટ & ઇમેઇલ ઓથેન્ટિકેશન પોલિસી \\
\end{longtable}
}

\textbf{સુરક્ષા મેકેનિઝમ્સ:}

\begin{itemize}
\tightlist
\item
  \textbf{એન્ક્રિપ્શન}: મેસેજ કન્ટેન્ટ સુરક્ષા
\item
  \textbf{ડિજિટલ સિગ્નેચર}: સેન્ડર આઇડેન્ટિટી વેરિફાય કરો
\item
  \textbf{ઓથેન્ટિકેશન}: મેસેજ ઓરિજિન કન્ફર્મ કરો
\item
  \textbf{ઇન્ટેગ્રિટી}: મેસેજ અપરિવર્તિત છે તેની ખાતરી
\end{itemize}

\textbf{અમલીકરણ સ્તરો:}

\begin{itemize}
\tightlist
\item
  \textbf{ટ્રાન્સપોર્ટ લેયર}: TLS/SSL એન્ક્રિપ્શન
\item
  \textbf{મેસેજ લેયર}: S/MIME, PGP એન્ક્રિપ્શન
\item
  \textbf{પોલિસી લેયર}: SPF, DKIM, DMARC
\end{itemize}

\textbf{બેસ્ટ પ્રેક્ટિસિસ:}

\begin{itemize}
\tightlist
\item
  \textbf{યુઝર એજ્યુકેશન}: ફિશિંગ પ્રયાસો ઓળખો
\item
  \textbf{ગેટવે ફિલ્ટરિંગ}: દુષ્ટ ઇમેઇલ્સ બ્લોક કરો
\item
  \textbf{રેગ્યુલર અપડેટ્સ}: સુરક્ષા સોફ્ટવેર અપ-ટુ-ડેટ રાખો
\item
  \textbf{બેકઅપ સિસ્ટમ્સ}: ડેટા લોસ સામે સુરક્ષા
\end{itemize}

\textbf{લાભો:}

\begin{itemize}
\tightlist
\item
  \textbf{ગુપ્તતા}: ખાનગી સંવાદ
\item
  \textbf{ઓથેન્ટિકેશન}: વેરિફાઇડ સેન્ડર્સ
\item
  \textbf{કમ્પ્લાયન્સ}: નિયમન જરૂરિયાતો પૂરી કરો
\item
  \textbf{ટ્રસ્ટ}: સુરક્ષિત બિઝનેસ કમ્યુનિકેશન્સ
\end{itemize}

\end{solutionbox}
\begin{mnemonicbox}
``SPTSD = S/MIME, PGP, TLS, SPF, DKIM ઇમેઇલ સુરક્ષા આપે
છે''

\end{mnemonicbox}

\end{document}
