\documentclass[10pt,a4paper]{article}

% content/resources/templates/preamble.tex
\usepackage[margin=0.6in]{geometry}
\author{Milav Dabgar}
\usepackage{amsmath,amssymb,amsthm}
\usepackage{booktabs}
\usepackage{multirow}
\usepackage{xcolor}
\usepackage{tcolorbox}
\tcbuselibrary{breakable,skins}
\usepackage[colorlinks=true,linkcolor=blue]{hyperref}
\usepackage{titlesec}
\usepackage{enumitem}
\usepackage{tikz}
\usepackage{pgfplots}
\usepackage{circuitikz}
\usepackage[version=4]{mhchem}
\usepackage{longtable}
\usepackage{array}
\usepackage{float}
\usepackage{caption}
\usepackage{listings}

\lstset{
  basicstyle=\small\ttfamily,
  breaklines=true,
  breakatwhitespace=false,
  postbreak=\mbox{\textcolor{red}{$\hookrightarrow$}\space},
  float=false,
  numbers=left,
  numberstyle=\tiny\color{gray},
  numbersep=10pt,
  xleftmargin=2em,
  keywordstyle=\color{blue},
  commentstyle=\color{green!60!black},
  stringstyle=\color{purple},
  backgroundcolor=\color{gray!5},
  showstringspaces=false,
  tabsize=2,
  captionpos=b,
  keepspaces=true,
  columns=flexible
}

\pgfplotsset{compat=1.18}
\usetikzlibrary{shapes,arrows,positioning,calc,patterns,decorations.pathmorphing,decorations.markings,arrows.meta}

% Color scheme
\definecolor{headcolor}{RGB}{0,102,204}
\definecolor{keycolor}{RGB}{220,20,60}
\definecolor{solutioncolor}{RGB}{34,139,34}
\definecolor{mnemoniccolor}{RGB}{148,0,211}
\definecolor{codecolor}{RGB}{0,0,100}

% Spacing
\setlength{\parskip}{3pt}
\setlist[itemize]{nosep}
\setlist[enumerate]{nosep}

% Title formatting
\titleformat{\section}{\Large\bfseries\color{headcolor}}{\thesection}{1em}{}
\titleformat{\subsection}{\large\bfseries\color{headcolor}}{\thesubsection}{1em}{}

% Pandoc tightlist compatibility
\providecommand{\tightlist}{%
  \setlength{\itemsep}{0pt}\setlength{\parskip}{0pt}}

% Pandoc longtable compatibility
\newcounter{none}
\def\thenone{}


% content/resources/templates/gujarati-boxes.tex
\usepackage{fontspec}
\usepackage{polyglossia}

% Set Gujarati as main language (document is primarily in Gujarati)
% Note: gloss-gujarati.ldf doesn't exist in polyglossia, but it will use hyphenation patterns
\setdefaultlanguage{gujarati}
\setotherlanguage{english}

% Configure Gujarati font properly
% Use Language=Default to prevent polyglossia from trying to add language-specific features
% that don't exist for Gujarati, which causes "empty feature" warnings
\newfontfamily\gujaratifont[Script=Gujarati,AutoFakeBold=2.5,AutoFakeSlant=0.3]{Noto Sans Gujarati}
\setmainfont[Script=Gujarati,AutoFakeBold=2.5,AutoFakeSlant=0.3]{Noto Sans Gujarati}
% Use Noto Sans Gujarati for monospace to support Gujarati in text
\setmonofont[Scale=0.9]{Noto Sans Gujarati}

% Configure English to use the same font
\newfontfamily\englishfont[Script=Gujarati,AutoFakeBold=2.5,AutoFakeSlant=0.3]{Noto Sans Gujarati}

% Translations for polyglossia
\gappto\captionsgujarati{
  \renewcommand{\tablename}{કોષ્ટક}
  \renewcommand{\figurename}{આકૃતિ}
}

% Helper for TikZ nodes to ensure Gujarati font
\newcommand{\gu}[1]{{\gujaratifont #1}}

% Custom environments
\newtcolorbox{solutionbox}{
    breakable,
    enhanced,
    colback=solutioncolor!5!white,
    colframe=solutioncolor!75!black,
    fonttitle=\bfseries,
    title=જવાબ
}

\newtcolorbox{solutionboxnobreak}{
 colback=solutioncolor!5!white,
 colframe=solutioncolor!75!black,
 fonttitle=\bfseries,
 title=જવાબ
}

\newtcolorbox{keyformula}{
 breakable,
 enhanced,
 colback=keycolor!5!white,
 colframe=keycolor!75!black,
 fonttitle=\bfseries,
 title=રાસાયણિક સમીકરણ/સૂત્ર
}

\newtcolorbox{mnemonicbox}{
 breakable,
 enhanced,
 colback=mnemoniccolor!5!white,
 colframe=mnemoniccolor!75!black,
 fonttitle=\bfseries,
 title=મેમરી ટ્રીક
}


\begin{document}

\begin{center}
{\Huge\bfseries\color{headcolor} Subject Name (Gujarati)}\\[5pt]
{\LARGE 4343202 -- Summer 2024}\\[3pt]
{\large Semester 1 Study Material}\\[3pt]
{\normalsize\textit{Detailed Solutions and Explanations}}
\end{center}

\vspace{10pt}

\subsection*{પ્રશ્ન 1(અ) [3
ગુણ]}\label{uxaaauxab0uxab6uxaa8-1uxa85-3-uxa97uxaa3}

\textbf{પેકેટ સ્વીચીંગ નેટવર્ક સમજાવો.}

\begin{solutionbox}
પેકેટ સ્વીચીંગ એ નેટવર્ક કમ્યુનિકેશન પદ્ધતિ છે જેમાં ડેટા ટ્રાન્સમિશન
પહેલા નાના પેકેટ્સમાં વિભાજિત કરવામાં આવે છે.

\textbf{આકૃતિ:}

\begin{center}
\textbf{Mermaid Diagram (Code)}
\begin{verbatim}
{Shaded}
{Highlighting}[]
graph LR
    A[સોર્સ] {-{-}{} B[પેકેટ્સ બનાવવા]}
    B {-{-}{} C[પેકેટ 1]}
    B {-{-}{} D[પેકેટ 2]}
    B {-{-}{} E[પેકેટ 3]}
    C {-{-}{} F[રાઉટર]}
    D {-{-}{} F}
    E {-{-}{} F}
    F {-{-}{} G[અલગ{-}અલગ માર્ગો]}
    G {-{-}{} H[ડેસ્ટિનેશન]}
{Highlighting}
{Shaded}
\end{verbatim}
\end{center}

\begin{itemize}
\tightlist
\item
  \textbf{સ્વતંત્ર રાઉટિંગ}: દરેક પેકેટ નેટવર્કમાં સ્વતંત્ર રીતે પ્રવાસ કરે છે
\item
  \textbf{લવચીક માર્ગો}: પેકેટ્સ ડેસ્ટિનેશન સુધી પહોંચવા માટે અલગ-અલગ રૂટ્સ લઈ શકે છે
\item
  \textbf{કાર્યક્ષમતા}: નેટવર્ક બેન્ડવિડ્થનો વધુ સારો ઉપયોગ
\end{itemize}

\end{solutionbox}
\begin{mnemonicbox}
``DIVE'' - ડેટા ઇન્ટુ વેરિયસ એલિમેન્ટ્સ

\end{mnemonicbox}
\subsection*{પ્રશ્ન 1(બ) [4
ગુણ]}\label{uxaaauxab0uxab6uxaa8-1uxaac-4-uxa97uxaa3}

\textbf{OSI રેફરન્સ મોડેલનાં કોઈ પણ 4 સ્તરોનું કાર્ય સમજાવો.}

\begin{solutionbox}
OSI મોડેલ નેટવર્ક કમ્યુનિકેશનને સાત અલગ-અલગ સ્તરોમાં વિભાજિત કરે છે,
દરેક સ્તરની ચોક્કસ કાર્યો છે.

{\def\LTcaptype{none} % do not increment counter
\begin{longtable}[]{@{}lll@{}}
\toprule\noalign{}
સ્તર & કાર્ય & મુખ્ય પ્રોટોકોલ્સ \\
\midrule\noalign{}
\endhead
\bottomrule\noalign{}
\endlastfoot
એપ્લિકેશન & યુઝર એપ્લિકેશનને સીધી નેટવર્ક સેવાઓ પ્રદાન કરે છે & HTTP, FTP, SMTP \\
પ્રેઝન્ટેશન & ડેટાનું અનુવાદ, એન્ક્રિપ્શન અને કમ્પ્રેશન કરે છે & SSL, TLS, JPEG \\
સેશન & કનેક્શન સ્થાપિત, સંચાલિત અને સમાપ્ત કરે છે & NetBIOS, RPC \\
ટ્રાન્સપોર્ટ & એન્ડ-ટુ-એન્ડ ડેટા ટ્રાન્સફર સુનિશ્ચિત કરે છે & TCP, UDP \\
\end{longtable}
}

\begin{itemize}
\tightlist
\item
  \textbf{એપ્લિકેશન લેયર}: નેટવર્ક અને એપ્લિકેશન વચ્ચે ઇન્ટરફેસ
\item
  \textbf{પ્રેઝન્ટેશન લેયર}: ડેટા ફોર્મેટિંગ અને એન્ક્રિપ્શન
\item
  \textbf{સેશન લેયર}: ડાયલોગ કંટ્રોલ અને સિંક્રોનાઇઝેશન
\item
  \textbf{ટ્રાન્સપોર્ટ લેયર}: એન્ડ-ટુ-એન્ડ કનેક્શન અને વિશ્વસનીયતા
\end{itemize}

\end{solutionbox}
\begin{mnemonicbox}
``All People Seem To Need Data Processing'' (બધા
લોકોને ડેટા પ્રોસેસિંગની જરૂર લાગે છે)

\end{mnemonicbox}
\subsection*{પ્રશ્ન 1(ક) [7
ગુણ]}\label{uxaaauxab0uxab6uxaa8-1uxa95-7-uxa97uxaa3}

\textbf{નેટવર્ક ટોપોલોજી આકૃતિ સાથે સમજાવો.}

\begin{solutionbox}
નેટવર્ક ટોપોલોજી નેટવર્કમાં ડિવાઇસની ભૌતિક અથવા તાર્કિક ગોઠવણને
દર્શાવે છે.

{\def\LTcaptype{none} % do not increment counter
\begin{longtable}[]{@{}lll@{}}
\toprule\noalign{}
ટોપોલોજી & ફાયદાઓ & ગેરફાયદાઓ \\
\midrule\noalign{}
\endhead
\bottomrule\noalign{}
\endlastfoot
બસ & સરળ, સસ્તી & એક પોઇન્ટ ફેલ્યોર \\
સ્ટાર & સહેલાઈથી ટ્રબલશૂટિંગ, કેન્દ્રીય & હબ/સ્વિચ ફેલ્યોરથી બધા પ્રભાવિત \\
રિંગ & બધા નોડ્સને સમાન એક્સેસ & એક કેબલ ફેલ્યોર નેટવર્કને અસર કરે \\
મેશ & ઉચ્ચ વિશ્વસનીયતા, ટ્રાફિક સમસ્યાઓ નહીં & ખર્ચાળ, જટિલ \\
ટ્રી & સરળતાથી વિસ્તરણીય, સંરચિત & રૂટ પર આધારિત, જટિલ \\
\end{longtable}
}

\textbf{આકૃતિ:}

\begin{verbatim}
              BUS TOPOLOGY              
+{-{-}{-}{-}{-}+    +{-}{-}{-}{-}{-}+    +{-}{-}{-}{-}{-}+    +{-}{-}{-}{-}{-}+}
|Node1|====|Node2|====|Node3|====|Node4|
+{-{-}{-}{-}{-}+    +{-}{-}{-}{-}{-}+    +{-}{-}{-}{-}{-}+    +{-}{-}{-}{-}{-}+}
                                        
              STAR TOPOLOGY             
                 +{-{-}{-}{-}{-}{-}+                }
                 |Hub/  |                
                 |Switch|                
                 +{-{-}{-}{-}{-}{-}+                }
                     |                  
           +{-{-}{-}{-}{-}{-}{-}{-}{-}+{-}{-}{-}{-}{-}{-}{-}{-}{-}+         }
           |         |         |         
        +{-{-}{-}{-}{-}+   +{-}{-}{-}{-}{-}+   +{-}{-}{-}{-}{-}+      }
        |Node1|   |Node2|   |Node3|      
        +{-{-}{-}{-}{-}+   +{-}{-}{-}{-}{-}+   +{-}{-}{-}{-}{-}+     }
\end{verbatim}

\begin{itemize}
\tightlist
\item
  \textbf{બસ ટોપોલોજી}: બધા ડિવાઇસ સિંગલ કેબલ સાથે જોડાયેલા
\item
  \textbf{સ્ટાર ટોપોલોજી}: બધા ડિવાઇસ સેન્ટ્રલ હબ/સ્વિચ સાથે જોડાયેલા
\item
  \textbf{રિંગ ટોપોલોજી}: ડિવાઇસ બંધ લૂપમાં જોડાયેલા
\item
  \textbf{મેશ ટોપોલોજી}: દરેક ડિવાઇસ દરેક અન્ય ડિવાઇસ સાથે જોડાયેલું
\item
  \textbf{ટ્રી ટોપોલોજી}: હાયરાર્કિકલ સ્ટાર નેટવર્ક્સ બસ વાયા કનેક્ટેડ
\end{itemize}

\end{solutionbox}
\begin{mnemonicbox}
``BSRMT'' - ``બેટર સોલ્યુશન્સ રિક્વાયર મલ્ટિપલ ટોપોલોજીસ''

\end{mnemonicbox}
\subsection*{પ્રશ્ન 1(ક) અથવા [7
ગુણ]}\label{uxaaauxab0uxab6uxaa8-1uxa95-uxa85uxaa5uxab5-7-uxa97uxaa3}

\textbf{TCP/IP પ્રોટોકોલ સ્યુટનો ડાયાગ્રામ દોરો અને એપ્લીકેશન લેયર, ટ્રાન્સપોર્ટ
લેયર અને નેટવર્ક લેયરનું કાર્યપધ્ધતી સમજાવો.}

\begin{solutionbox}
TCP/IP પ્રોટોકોલ સ્યુટ નેટવર્ક કોમ્યુનિકેશનને ચાર કાર્યાત્મક સ્તરોમાં
વ્યવસ્થિત કરે છે.

\textbf{આકૃતિ:}

\begin{verbatim}
+{-{-}{-}{-}{-}{-}{-}{-}{-}{-}{-}{-}{-}{-}{-}{-}{-}{-}{-}{-}{-}{-}{-}{-}{-}{-}{-}{-}{-}{-}{-}+}
|       APPLICATION LAYER       |
| (HTTP, FTP, SMTP, DNS, TELNET)|
+{-{-}{-}{-}{-}{-}{-}{-}{-}{-}{-}{-}{-}{-}{-}{-}{-}{-}{-}{-}{-}{-}{-}{-}{-}{-}{-}{-}{-}{-}{-}+}
|        TRANSPORT LAYER        |
|           (TCP, UDP)          |
+{-{-}{-}{-}{-}{-}{-}{-}{-}{-}{-}{-}{-}{-}{-}{-}{-}{-}{-}{-}{-}{-}{-}{-}{-}{-}{-}{-}{-}{-}{-}+}
|        INTERNET LAYER         |
|      (IP, ICMP, ARP, RARP)    |
+{-{-}{-}{-}{-}{-}{-}{-}{-}{-}{-}{-}{-}{-}{-}{-}{-}{-}{-}{-}{-}{-}{-}{-}{-}{-}{-}{-}{-}{-}{-}+}
|     NETWORK ACCESS LAYER      |
| (Ethernet, Wi{-Fi, Token Ring) |}
+{-{-}{-}{-}{-}{-}{-}{-}{-}{-}{-}{-}{-}{-}{-}{-}{-}{-}{-}{-}{-}{-}{-}{-}{-}{-}{-}{-}{-}{-}{-}+}
\end{verbatim}

{\def\LTcaptype{none} % do not increment counter
\begin{longtable}[]{@{}lll@{}}
\toprule\noalign{}
સ્તર & મુખ્ય કાર્ય & મુખ્ય પ્રોટોકોલ્સ \\
\midrule\noalign{}
\endhead
\bottomrule\noalign{}
\endlastfoot
એપ્લિકેશન & એપ્લિકેશન્સને નેટવર્ક સેવાઓ પ્રદાન કરે & HTTP, FTP, SMTP \\
ટ્રાન્સપોર્ટ & એન્ડ-ટુ-એન્ડ કોમ્યુનિકેશન, ડેટા ફ્લો કંટ્રોલ & TCP, UDP \\
ઈન્ટરનેટ (નેટવર્ક) & લોજિકલ એડ્રેસિંગ અને રાઉટિંગ & IP, ICMP, ARP \\
\end{longtable}
}

\begin{itemize}
\tightlist
\item
  \textbf{એપ્લિકેશન લેયર}: નેટવર્ક માટે યુઝર ઇન્ટરફેસ, એપ્લિકેશન-સ્પેસિફિક પ્રોટોકોલ્સ
\item
  \textbf{ટ્રાન્સપોર્ટ લેયર}: વિશ્વસનીય ડેટા ટ્રાન્સમિશન, એરર રિકવરી, ફ્લો કંટ્રોલ
\item
  \textbf{નેટવર્ક લેયર}: નેટવર્ક્સ વચ્ચે પેકેટ્સ રાઉટિંગ, IP એડ્રેસિંગ
\end{itemize}

\end{solutionbox}
\begin{mnemonicbox}
``ATN works'' - એપ્લિકેશન, ટ્રાન્સપોર્ટ, નેટવર્ક સાથે મળીને
કામ કરે છે

\end{mnemonicbox}
\subsection*{પ્રશ્ન 2(અ) [3
ગુણ]}\label{uxaaauxab0uxab6uxaa8-2uxa85-3-uxa97uxaa3}

\textbf{કનેક્શન ઓરિએન્ટેડ પ્રોટોકોલ અને કનેક્શન લેસ પ્રોટોકોલની સરખામણી કરો.}

\begin{solutionbox}
કનેક્શન-ઓરિએન્ટેડ અને કનેક્શનલેસ પ્રોટોકોલ્સ ડેટા ટ્રાન્સમિશનના
હેન્ડલિંગમાં અલગ પડે છે.

{\def\LTcaptype{none} % do not increment counter
\begin{longtable}[]{@{}lll@{}}
\toprule\noalign{}
ફીચર & કનેક્શન-ઓરિએન્ટેડ & કનેક્શનલેસ \\
\midrule\noalign{}
\endhead
\bottomrule\noalign{}
\endlastfoot
કનેક્શન & ટ્રાન્સમિશન પહેલા સ્થાપિત & કોઈ કનેક્શન સેટઅપ નહીં \\
વિશ્વસનીયતા & ગેરંટેડ ડિલિવરી & કોઈ ડિલિવરી ગેરંટી નહીં \\
એરર ચેકિંગ & વિસ્તૃત & મર્યાદિત અથવા કોઈ નહીં \\
ઉદાહરણ & TCP & UDP \\
ઉપયોગ & ફાઈલ ટ્રાન્સફર, વેબ બ્રાઉઝિંગ & સ્ટ્રીમિંગ, DNS લુકઅપ્સ \\
\end{longtable}
}

\end{solutionbox}
\begin{mnemonicbox}
``REACH'' - રિલાયબિલિટી એક્ઝિસ્ટ્સ ઇન ઓલ કનેક્શન હેન્ડશેક્સ

\end{mnemonicbox}
\subsection*{પ્રશ્ન 2(બ) [4
ગુણ]}\label{uxaaauxab0uxab6uxaa8-2uxaac-4-uxa97uxaa3}

\textbf{ફાસ્ટ ઇથરનેટ અને ગીગાબાઈટ ઈથરનેટ સમજાવો.}

\begin{solutionbox}
ફાસ્ટ ઇથરનેટ અને ગીગાબિટ ઇથરનેટ મૂળ ઇથરનેટ સ્ટાન્ડર્ડના ઉચ્ચ-સ્પીડ
વર્ઝન છે.

{\def\LTcaptype{none} % do not increment counter
\begin{longtable}[]{@{}lll@{}}
\toprule\noalign{}
ફીચર & ફાસ્ટ ઇથરનેટ & ગીગાબિટ ઇથરનેટ \\
\midrule\noalign{}
\endhead
\bottomrule\noalign{}
\endlastfoot
સ્પીડ & 100 Mbps & 1000 Mbps (1 Gbps) \\
IEEE સ્ટાન્ડર્ડ & 802.3u & 802.3z/802.3ab \\
કેબલ ટાઇપ & Cat5 UTP & Cat5e/Cat6 UTP, ફાઇબર \\
મેક્સ ડિસ્ટન્સ & 100m (કોપર) & 100m (કોપર), 5km (ફાઇબર) \\
\end{longtable}
}

\begin{itemize}
\tightlist
\item
  \textbf{ફાસ્ટ ઇથરનેટ}: ઓરિજિનલ 10Base-T ઇથરનેટથી 10x ઝડપી
\item
  \textbf{ગીગાબિટ ઇથરનેટ}: ફાસ્ટ ઇથરનેટથી 10x ઝડપી, બેકવર્ડ કમ્પેટિબલ
\item
  \textbf{કેબલિંગ}: વધુ સ્પીડ માટે ઉચ્ચ ગુણવત્તાવાળા કેબલિંગનો ઉપયોગ
\item
  \textbf{એપ્લિકેશન્સ}: હાઈ-બેન્ડવિડ્થ નેટવર્ક બેકબોન્સ, સર્વર કનેક્શન્સ
\end{itemize}

\end{solutionbox}
\begin{mnemonicbox}
``Fast Gets Going'' - 100થી 1000 Mbps સુધીની પ્રગતિ

\end{mnemonicbox}
\subsection*{પ્રશ્ન 2(ક) [7
ગુણ]}\label{uxaaauxab0uxab6uxaa8-2uxa95-7-uxa97uxaa3}

\textbf{રાઉટર, હબ અને સ્વીચ વચ્ચેનો તફાવત આપો.}

\begin{solutionbox}
રાઉટર, હબ અને સ્વિચ અલગ-અલગ ક્ષમતાઓ અને કાર્યો ધરાવતા નેટવર્ક
ડિવાઇસ છે.

{\def\LTcaptype{none} % do not increment counter
\begin{longtable}[]{@{}llll@{}}
\toprule\noalign{}
ફીચર & રાઉટર & હબ & સ્વિચ \\
\midrule\noalign{}
\endhead
\bottomrule\noalign{}
\endlastfoot
OSI લેયર & નેટવર્ક (3) & ફિઝિકલ (1) & ડેટા લિંક (2) \\
કાર્ય & નેટવર્ક્સ કનેક્ટ કરે & ડિવાઇસ કનેક્ટ કરે & ડિવાઇસ કનેક્ટ કરે \\
ડેટા હેન્ડલિંગ & ઇન્ટેલિજન્ટ રાઉટિંગ & બધાને બ્રોડકાસ્ટ & ચોક્કસ ડિવાઇસને મોકલે \\
સિક્યોરિટી & ફાયરવોલ પ્રદાન કરે & કોઈ સિક્યોરિટી નહીં & બેઝિક ફિલ્ટરિંગ \\
એડ્રેસિંગ & IP એડ્રેસનો ઉપયોગ & કોઈ એડ્રેસિંગ નહીં & MAC એડ્રેસનો ઉપયોગ \\
કાર્યક્ષમતા & ઉચ્ચ & નીચી & ઉચ્ચ \\
બુદ્ધિમત્તા & સ્માર્ટ & ડંબ & મધ્યમ સ્માર્ટ \\
\end{longtable}
}

\textbf{આકૃતિ:}

\begin{verbatim}
    ROUTER                HUB                  SWITCH
  +{-{-}{-}{-}{-}{-}{-}{-}+           +{-}{-}{-}{-}{-}{-}+              +{-}{-}{-}{-}{-}{-}{-}{-}+}
  |        |           |      |              |        |
  | Routes |           |Shares|              |Forwards|
  |between |           |signal|              | to MAC |
  |networks|           |to all|              |address |
  |        |           |ports |              |        |
  +{-{-}{-}{-}{-}{-}{-}{-}+           +{-}{-}{-}{-}{-}{-}+              +{-}{-}{-}{-}{-}{-}{-}{-}+}
\end{verbatim}

\end{solutionbox}
\begin{mnemonicbox}
``RHS order'' - ``રાઉટર હેઝ સ્માર્ટ્સ, હબ શેર્સ સિગ્નલ,
સ્વિચ સેન્ડ્સ સ્પેસિફિકલી''

\end{mnemonicbox}
\subsection*{પ્રશ્ન 2(અ) અથવા [3
ગુણ]}\label{uxaaauxab0uxab6uxaa8-2uxa85-uxa85uxaa5uxab5-3-uxa97uxaa3}

\textbf{ઈ-મેઈલ સીસ્ટમની વ્યાખ્યા આપો અને ઈ-મેઈલનાં ઉપયોગો જણાવો.}

\begin{solutionbox}
ઈમેલ સિસ્ટમ એ નેટવર્ક સેવા છે જે યુઝર્સ વચ્ચે ડિજિટલ મેસેજનું આદાન-પ્રદાન
કરવાની મંજૂરી આપે છે.

{\def\LTcaptype{none} % do not increment counter
\begin{longtable}[]{@{}ll@{}}
\toprule\noalign{}
કોમ્પોનન્ટ & કાર્ય \\
\midrule\noalign{}
\endhead
\bottomrule\noalign{}
\endlastfoot
મેઇલ યુઝર એજન્ટ (MUA) & એન્ડ-યુઝર્સ દ્વારા ઉપયોગમાં લેવાતા ઈમેઇલ ક્લાયન્ટ સોફ્ટવેર \\
મેઇલ ટ્રાન્સફર એજન્ટ (MTA) & ઈમેઇલ્સ ટ્રાન્સફર કરતું સર્વર સોફ્ટવેર \\
મેઇલ ડિલિવરી એજન્ટ (MDA) & પ્રાપ્તકર્તાના મેઇલબોક્સમાં ઈમેઇલ ડિલિવર કરે છે \\
પ્રોટોકોલ્સ & SMTP, POP3, IMAP \\
\end{longtable}
}

\textbf{ઈમેઇલના ઉપયોગો:}

\begin{itemize}
\tightlist
\item
  બિઝનેસ કોમ્યુનિકેશન
\item
  પર્સનલ મેસેજિંગ
\item
  ફાઇલ શેરિંગ
\item
  માર્કેટિંગ અને ન્યૂઝલેટર્સ
\item
  નોટિફિકેશન્સ અને એલર્ટ્સ
\end{itemize}

\end{solutionbox}
\begin{mnemonicbox}
``BCPFN'' - ``બિઝનેસ કોમ્યુનિકેશન, પર્સનલ, ફાઇલ્સ,
ન્યૂઝલેટર્સ''

\end{mnemonicbox}
\subsection*{પ્રશ્ન 2(બ) અથવા [4
ગુણ]}\label{uxaaauxab0uxab6uxaa8-2uxaac-uxa85uxaa5uxab5-4-uxa97uxaa3}

\textbf{IPv4 અને IPv6નો તફાવત આપો.}

\begin{solutionbox}
IPv4 અને IPv6 ઇન્ટરનેટ પ્રોટોકોલ વર્ઝન્સ છે જેમાં નોંધપાત્ર તફાવતો
છે.

{\def\LTcaptype{none} % do not increment counter
\begin{longtable}[]{@{}
  >{\raggedright\arraybackslash}p{(\linewidth - 4\tabcolsep) * \real{0.3684}}
  >{\raggedright\arraybackslash}p{(\linewidth - 4\tabcolsep) * \real{0.3158}}
  >{\raggedright\arraybackslash}p{(\linewidth - 4\tabcolsep) * \real{0.3158}}@{}}
\toprule\noalign{}
\begin{minipage}[b]{\linewidth}\raggedright
ફીચર
\end{minipage} & \begin{minipage}[b]{\linewidth}\raggedright
IPv4
\end{minipage} & \begin{minipage}[b]{\linewidth}\raggedright
IPv6
\end{minipage} \\
\midrule\noalign{}
\endhead
\bottomrule\noalign{}
\endlastfoot
એડ્રેસ લંબાઈ & 32-બિટ (4 બાઇટ્સ) & 128-બિટ (16 બાઇટ્સ) \\
ફોર્મેટ & ડોટેડ ડેસિમલ (192.168.1.1) & હેક્સાડેસિમલ વિથ કોલન્સ
(2001:0db8:85a3:0000:0000:8a2e:0370:7334) \\
એડ્રેસ સ્પેસ & \textasciitilde4.3 બિલિયન એડ્રેસ & 340 અન્ડેસિલિયન એડ્રેસ \\
સિક્યોરિટી & સિક્યોરિટી પછીથી ઉમેરાયેલી & બિલ્ટ-ઇન IPSec \\
કોન્ફિગરેશન & મેન્યુઅલ અથવા DHCP & સ્ટેટલેસ ઓટો-કોન્ફિગરેશન \\
હેડર & જટિલ, ચલ & સરળ, ફિક્સ્ડ \\
\end{longtable}
}

\begin{itemize}
\tightlist
\item
  \textbf{IPv4}: મર્યાદિત સ્પેસ સાથે પરંપરાગત એડ્રેસિંગ
\item
  \textbf{IPv6}: વિશાળ ક્ષમતા સાથે આગામી-પેઢી એડ્રેસિંગ
\item
  \textbf{ટ્રાન્ઝિશન}: ડ્યુઅલ-સ્ટેક, ટનલિંગ અને ટ્રાન્સલેશન મેકેનિઝમ્સ
\end{itemize}

\end{solutionbox}
\begin{mnemonicbox}
``4 SMALL, 6 HUGE'' - IPv4 નાનો એડ્રેસ સ્પેસ, IPv6 વિશાળ
એડ્રેસ સ્પેસ

\end{mnemonicbox}
\subsection*{પ્રશ્ન 2(ક) અથવા [7
ગુણ]}\label{uxaaauxab0uxab6uxaa8-2uxa95-uxa85uxaa5uxab5-7-uxa97uxaa3}

\textbf{નેટવર્કમાં ફાયરવોલ સાથે કોન્સેપ્ટ, પ્રિન્સીપલ, લીમીટેશન, trusted system,
Kerberos-conceptની ચર્ચા કરો.}

\begin{solutionbox}
ફાયરવોલ્સ ક્રિટિકલ નેટવર્ક સિક્યોરિટી સિસ્ટમ્સ છે જે ઇનકમિંગ અને
આઉટગોઇંગ ટ્રાફિકને મોનિટર અને કંટ્રોલ કરે છે.

{\def\LTcaptype{none} % do not increment counter
\begin{longtable}[]{@{}lll@{}}
\toprule\noalign{}
ફાયરવોલ ટાઇપ & કાર્ય & ઉદાહરણ \\
\midrule\noalign{}
\endhead
\bottomrule\noalign{}
\endlastfoot
પેકેટ ફિલ્ટરિંગ & પેકેટ હેડર તપાસે & રાઉટર ACLs \\
સ્ટેટફુલ ઇન્સ્પેક્શન & કનેક્શન સ્ટેટ ટ્રેક કરે & મોટાભાગના હાર્ડવેર ફાયરવોલ્સ \\
એપ્લિકેશન લેયર & ડેટા કન્ટેન્ટ ઇન્સ્પેક્ટ કરે & વેબ એપ્લિકેશન ફાયરવોલ્સ \\
નેક્સ્ટ-જનરેશન & એકાધિક ટેકનિક્સ જોડે & પાલો આલ્ટો, ફોર્ટિનેટ \\
\end{longtable}
}

\textbf{ફાયરવોલના સિદ્ધાંતો:}

\begin{itemize}
\tightlist
\item
  \textbf{ડિફોલ્ટ ડિનાય}: સ્પષ્ટપણે મંજૂર ન હોય ત્યાં સુધી બધું બ્લોક કરો
\item
  \textbf{ડિફેન્સ ઇન ડેપ્થ}: મલ્ટિપલ સિક્યોરિટી લેયર્સ
\item
  \textbf{લીસ્ટ પ્રિવિલેજ}: ન્યૂનતમ જરૂરી એક્સેસ
\end{itemize}

\textbf{મર્યાદાઓ:}

\begin{itemize}
\tightlist
\item
  અધિકૃત યુઝર્સ સામે રક્ષણ આપી શકતું નથી
\item
  એન્ક્રિપ્ટેડ મેલિશિયસ ટ્રાફિક સામે મર્યાદિત
\item
  નેટવર્ક પરફોર્મન્સ પર અસર
\end{itemize}

\textbf{ટ્રસ્ટેડ સિસ્ટમ્સ:}

\begin{itemize}
\tightlist
\item
  ચોક્કસ સિક્યોરિટી આવશ્યકતાઓને પૂર્ણ કરતી સિસ્ટમ્સ
\item
  ફોર્મલ સિક્યોરિટી પોલિસી એન્ફોર્સમેન્ટ
\item
  એક્સેસ કંટ્રોલ અને ઓથેન્ટિકેશન મેકેનિઝમ્સ
\end{itemize}

\textbf{કર્બેરોસ કોન્સેપ્ટ:}

\begin{verbatim}
    +{-{-}{-}{-}{-}{-}{-}{-}{-}{-}+       +{-}{-}{-}{-}{-}{-}{-}{-}{-}{-}+       +{-}{-}{-}{-}{-}{-}{-}{-}{-}{-}+}
    |  Client  |{{-}{-}{-}{-}{-}|   KDC    |{-}{-}{-}{-}{-}|  Server  |}
    +{-{-}{-}{-}{-}{-}{-}{-}{-}{-}+       +{-}{-}{-}{-}{-}{-}{-}{-}{-}{-}+       +{-}{-}{-}{-}{-}{-}{-}{-}{-}{-}+}
         |                  |                  |
         |{{-}{-}Ticket{-}granting ticket{-}{-}|         |}
         |{-{-}{-}{-}{-}{-}{-}Service request ticket{-}{-}{-}{-}{-}{-}{-}|}
         |{{-}{-}{-}{-}{-}{-}{-}{-}{-}{-}Session key{-}{-}{-}{-}{-}{-}{-}{-}{-}{-}{-}{-}{-}{-}|}
\end{verbatim}

\begin{itemize}
\tightlist
\item
  ટ્રસ્ટેડ થર્ડ પાર્ટીનો ઉપયોગ કરતો \textbf{ઓથેન્ટિકેશન પ્રોટોકોલ}
\item
  \textbf{ટિકિટ-આધારિત} એક્સેસ કંટ્રોલ સિસ્ટમ
\item
  ક્લાયન્ટ અને સર્વર વચ્ચે \textbf{મ્યુચ્યુઅલ ઓથેન્ટિકેશન}
\item
  રિપ્લે એટેક્સને રોકવા માટે \textbf{સમય-સંવેદનશીલ} ટિકિટ્સ
\end{itemize}

\end{solutionbox}
\begin{mnemonicbox}
``FLASK'' - ``ફાયરવોલ્સ લોક એક્સેસ, સિક્યોર વિથ કર્બેરોસ''

\end{mnemonicbox}
\subsection*{પ્રશ્ન 3(અ) [3
ગુણ]}\label{uxaaauxab0uxab6uxaa8-3uxa85-3-uxa97uxaa3}

\textbf{ડેટા લિંક લેયરના સબ લેયર્સ સમજાવો.}

\begin{solutionbox}
OSI મોડેલમાં ડેટા લિંક લેયર બે અલગ-અલગ કાર્યો સાથે બે સબલેયર્સમાં
વિભાજિત છે.

{\def\LTcaptype{none} % do not increment counter
\begin{longtable}[]{@{}lll@{}}
\toprule\noalign{}
સબલેયર & કાર્ય & સ્ટાન્ડર્ડ્સ \\
\midrule\noalign{}
\endhead
\bottomrule\noalign{}
\endlastfoot
લોજિકલ લિંક કંટ્રોલ (LLC) & ફ્લો કંટ્રોલ, એરર ચેકિંગ & IEEE 802.2 \\
મીડિયા એક્સેસ કંટ્રોલ (MAC) & ચેનલ એક્સેસ, એડ્રેસિંગ & IEEE 802.3, 802.11 \\
\end{longtable}
}

\textbf{આકૃતિ:}

\begin{verbatim}
+{-{-}{-}{-}{-}{-}{-}{-}{-}{-}{-}{-}{-}{-}{-}{-}{-}{-}{-}{-}{-}{-}{-}{-}{-}{-}{-}{-}{-}+}
|        NETWORK LAYER        |
+{-{-}{-}{-}{-}{-}{-}{-}{-}{-}{-}{-}{-}{-}{-}{-}{-}{-}{-}{-}{-}{-}{-}{-}{-}{-}{-}{-}{-}+}
|     LOGICAL LINK CONTROL    |  {{-}{-} Flow control, Error handling}
|        (LLC {- 802.2)        |      Multiplexing, Connection mgmt}
+{-{-}{-}{-}{-}{-}{-}{-}{-}{-}{-}{-}{-}{-}{-}{-}{-}{-}{-}{-}{-}{-}{-}{-}{-}{-}{-}{-}{-}+}
|     MEDIA ACCESS CONTROL    |  {{-}{-} MAC addressing, Channel access }
|   (MAC {- 802.3, 802.11)     |      Frame delimiting, Error detection}
+{-{-}{-}{-}{-}{-}{-}{-}{-}{-}{-}{-}{-}{-}{-}{-}{-}{-}{-}{-}{-}{-}{-}{-}{-}{-}{-}{-}{-}+}
|       PHYSICAL LAYER        |
+{-{-}{-}{-}{-}{-}{-}{-}{-}{-}{-}{-}{-}{-}{-}{-}{-}{-}{-}{-}{-}{-}{-}{-}{-}{-}{-}{-}{-}+}
\end{verbatim}

\begin{itemize}
\tightlist
\item
  \textbf{LLC}: નેટવર્ક લેયર માટે ઇન્ટરફેસ પ્રદાન કરે છે, એરર/ફ્લો કંટ્રોલ
\item
  \textbf{MAC}: ફિઝિકલ એડ્રેસિંગ અને મીડિયા એક્સેસનું સંચાલન કરે છે
\end{itemize}

\end{solutionbox}
\begin{mnemonicbox}
``MAC LLCs order'' - ``MAC લોઅર લેયર હેન્ડલ કરે છે, LLC
હાયર કોઓર્ડિનેટ કરે છે''

\end{mnemonicbox}
\subsection*{પ્રશ્ન 3(બ) [4
ગુણ]}\label{uxaaauxab0uxab6uxaa8-3uxaac-4-uxa97uxaa3}

\textbf{IP layer protocols વિસ્તૃતમાં સમજાવો.}

\begin{solutionbox}
IP લેયરમાં કેટલાક મહત્વપૂર્ણ પ્રોટોકોલ્સ છે જે ઇન્ટરનેટવર્ક
કોમ્યુનિકેશનમાં સાથે મળીને કામ કરે છે.

{\def\LTcaptype{none} % do not increment counter
\begin{longtable}[]{@{}lll@{}}
\toprule\noalign{}
પ્રોટોકોલ & કાર્ય & મુખ્ય ફીચર્સ \\
\midrule\noalign{}
\endhead
\bottomrule\noalign{}
\endlastfoot
IP & બેઝિક ડેટાગ્રામ ડિલિવરી & એડ્રેસિંગ, ફ્રેગમેન્ટેશન, TTL \\
ICMP & નેટવર્ક ડાયગ્નોસ્ટિક્સ & એરર રિપોર્ટિંગ, પિંગ, ટ્રેસરાઉટ \\
ARP & એડ્રેસ રિઝોલ્યુશન & IP થી MAC એડ્રેસ મેપિંગ \\
RARP & રિવર્સ એડ્રેસ રિઝોલ્યુશન & MAC થી IP એડ્રેસ મેપિંગ \\
IGMP & મલ્ટિકાસ્ટ ગ્રુપ મેનેજમેન્ટ & હોસ્ટ ગ્રુપ્સનું મેનેજમેન્ટ \\
\end{longtable}
}

\begin{itemize}
\tightlist
\item
  \textbf{IP}: એડ્રેસિંગ અને પેકેટ્સ રાઉટિંગ માટે કોર પ્રોટોકોલ
\item
  \textbf{ICMP}: એરર મેસેજ અને ઓપરેશનલ ઇન્ફોર્મેશન
\item
  \textbf{ARP/RARP}: લેયર્સ વચ્ચે એડ્રેસ ટ્રાન્સલેશન
\item
  \textbf{IGMP}: મલ્ટિકાસ્ટ ગ્રુપ મેમ્બરશિપનું મેનેજમેન્ટ
\end{itemize}

\end{solutionbox}
\begin{mnemonicbox}
``I PAIR-up'' - IP, ICMP, ARP, RARP એક ટીમ તરીકે કામ
કરે છે

\end{mnemonicbox}
\subsection*{પ્રશ્ન 3(ક) [7
ગુણ]}\label{uxaaauxab0uxab6uxaa8-3uxa95-7-uxa97uxaa3}

\textbf{વિવિધ પ્રકારની IP એડ્રેસિંગ સ્કીમનું વર્ણન કરો અને ક્લાસફુલ IP એડ્રેસિંગમાં
વિવિધ વર્ગોને ઉદાહરણ સાથે સમજાવો.}

\begin{solutionbox}
IP એડ્રેસિંગ સ્કીમ્સ IP એડ્રેસના ફાળવણી અને સ્ટ્રક્ચરને વ્યાખ્યાયિત કરે
છે.

{\def\LTcaptype{none} % do not increment counter
\begin{longtable}[]{@{}lll@{}}
\toprule\noalign{}
IP એડ્રેસિંગ સ્કીમ & વર્ણન & ઉદાહરણ \\
\midrule\noalign{}
\endhead
\bottomrule\noalign{}
\endlastfoot
ક્લાસફુલ & 5 ક્લાસમાં પરંપરાગત વિભાજન & ક્લાસ A: 10.0.0.0 \\
ક્લાસલેસ (CIDR) & ફ્લેક્સિબલ પ્રિફિક્સ, વધુ કાર્યક્ષમ & 192.168.1.0/24 \\
પ્રાઇવેટ & આંતરિક ઉપયોગ માટે નોન-રાઉટેબલ એડ્રેસ & 192.168.0.0/16 \\
સ્પેશિયલ પર્પઝ & ચોક્કસ કાર્યો માટે અનામત & 127.0.0.1 (લોકલહોસ્ટ) \\
\end{longtable}
}

\textbf{ક્લાસફુલ IP એડ્રેસિંગ:}

{\def\LTcaptype{none} % do not increment counter
\begin{longtable}[]{@{}
  >{\raggedright\arraybackslash}p{(\linewidth - 12\tabcolsep) * \real{0.0769}}
  >{\raggedright\arraybackslash}p{(\linewidth - 12\tabcolsep) * \real{0.1209}}
  >{\raggedright\arraybackslash}p{(\linewidth - 12\tabcolsep) * \real{0.1978}}
  >{\raggedright\arraybackslash}p{(\linewidth - 12\tabcolsep) * \real{0.2308}}
  >{\raggedright\arraybackslash}p{(\linewidth - 12\tabcolsep) * \real{0.0989}}
  >{\raggedright\arraybackslash}p{(\linewidth - 12\tabcolsep) * \real{0.1099}}
  >{\raggedright\arraybackslash}p{(\linewidth - 12\tabcolsep) * \real{0.1648}}@{}}
\toprule\noalign{}
\begin{minipage}[b]{\linewidth}\raggedright
ક્લાસ
\end{minipage} & \begin{minipage}[b]{\linewidth}\raggedright
પ્રથમ બિટ્સ
\end{minipage} & \begin{minipage}[b]{\linewidth}\raggedright
પ્રથમ બાઇટ રેન્જ
\end{minipage} & \begin{minipage}[b]{\linewidth}\raggedright
ડિફોલ્ટ સબનેટ માસ્ક
\end{minipage} & \begin{minipage}[b]{\linewidth}\raggedright
ઉદાહરણ
\end{minipage} & \begin{minipage}[b]{\linewidth}\raggedright
નેટવર્ક્સ
\end{minipage} & \begin{minipage}[b]{\linewidth}\raggedright
હોસ્ટ્સ/નેટવર્ક
\end{minipage} \\
\midrule\noalign{}
\endhead
\bottomrule\noalign{}
\endlastfoot
A & 0 & 1-127 & 255.0.0.0 (/8) & 10.52.36.12 & 126 & 16,777,214 \\
B & 10 & 128-191 & 255.255.0.0 (/16) & 172.16.52.63 & 16,384 & 65,534 \\
C & 110 & 192-223 & 255.255.255.0 (/24) & 192.168.10.15 & 2,097,152 &
254 \\
D & 1110 & 224-239 & N/A (મલ્ટિકાસ્ટ) & 224.0.0.5 & N/A & N/A \\
E & 1111 & 240-255 & N/A (એક્સપેરિમેન્ટલ) & 240.0.0.1 & N/A & N/A \\
\end{longtable}
}

\begin{itemize}
\tightlist
\item
  \textbf{ક્લાસ A}: મોટી સંસ્થાઓ, હોસ્ટ્સની વિશાળ સંખ્યા
\item
  \textbf{ક્લાસ B}: મધ્યમ કદની સંસ્થાઓ
\item
  \textbf{ક્લાસ C}: ઓછા હોસ્ટ્સ સાથેના નાના નેટવર્ક્સ
\item
  \textbf{ક્લાસ D}: મલ્ટિકાસ્ટ ગ્રુપ્સ
\item
  \textbf{ક્લાસ E}: પ્રાયોગિક ઉપયોગ માટે અનામત
\end{itemize}

\end{solutionbox}
\begin{mnemonicbox}
``All Businesses Care During Exams'' - ક્લાસ A, B, C,
D, E

\end{mnemonicbox}
\subsection*{પ્રશ્ન 3(અ) અથવા [3
ગુણ]}\label{uxaaauxab0uxab6uxaa8-3uxa85-uxa85uxaa5uxab5-3-uxa97uxaa3}

\textbf{ડીજીટલ સબસ્કાઈબર લાઈન ટેકનોલોજી સમજાવો.}

\begin{solutionbox}
ડિજિટલ સબસ્ક્રાઇબર લાઇન (DSL) એ ટેલિફોન લાઇન્સ પર ડિજિટલ ડેટા
ટ્રાન્સમિશન પ્રદાન કરતી ટેકનોલોજી છે.

{\def\LTcaptype{none} % do not increment counter
\begin{longtable}[]{@{}llll@{}}
\toprule\noalign{}
DSL ટાઇપ & સ્પીડ (ડાઉન/અપ) & ડિસ્ટન્સ & એપ્લિકેશન \\
\midrule\noalign{}
\endhead
\bottomrule\noalign{}
\endlastfoot
ADSL & 8 Mbps/1 Mbps & 5.5 km સુધી & હોમ ઇન્ટરનેટ \\
SDSL & 2 Mbps/2 Mbps & 3 km સુધી & બિઝનેસ \\
VDSL & 52 Mbps/16 Mbps & 1.2 km સુધી & વિડીયો સ્ટ્રીમિંગ \\
HDSL & 2 Mbps/2 Mbps & 3.6 km સુધી & T1/E1 રિપ્લેસમેન્ટ \\
\end{longtable}
}

\textbf{આકૃતિ:}

\begin{verbatim}
                           +{-{-}{-}{-}{-}{-}{-}+}
        +{-{-}{-}{-}{-}{-}{-}{-}+         |       |}
HOME{-{-}{-}{-}|  DSL   |{-}{-}{-}{-}{-}{-}{-}{-}{-}| DSLAM |{-}{-}{-}{-}{-}{-}{-}INTERNET}
        | MODEM  |  Copper |       |
        +{-{-}{-}{-}{-}{-}{-}{-}+   Line  +{-}{-}{-}{-}{-}{-}{-}+}
                    (POTS)    ISP
\end{verbatim}

\begin{itemize}
\tightlist
\item
  \textbf{સ્પેક્ટ્રમ ઉપયોગ}: અવાજ કરતાં ઉચ્ચ ફ્રિક્વન્સીનો ઉપયોગ
\item
  \textbf{ઓલવેઝ-ઓન}: સતત કનેક્શન, ડાયલ-અપ નહીં
\item
  \textbf{xDSL}: અલગ-અલગ ક્ષમતાઓ સાથે ટેકનોલોજીનો પરિવાર
\end{itemize}

\end{solutionbox}
\begin{mnemonicbox}
``SAVE Bandwidth'' - SDSL, ADSL, VDSL, HDSL બેન્ડવિડ્થ
ઓપ્શન્સ

\end{mnemonicbox}
\subsection*{પ્રશ્ન 3(બ) અથવા [4
ગુણ]}\label{uxaaauxab0uxab6uxaa8-3uxaac-uxa85uxaa5uxab5-4-uxa97uxaa3}

\textbf{કેબલ મોડેમ સીસ્ટમને ચર્ચા કરો.}

\begin{solutionbox}
કેબલ મોડેમ સિસ્ટમ કેબલ ટીવી માટે વપરાતા એજ કોએક્સિયલ કેબલ દ્વારા
ઇન્ટરનેટ એક્સેસ પ્રદાન કરે છે.

{\def\LTcaptype{none} % do not increment counter
\begin{longtable}[]{@{}ll@{}}
\toprule\noalign{}
કોમ્પોનન્ટ & કાર્ય \\
\midrule\noalign{}
\endhead
\bottomrule\noalign{}
\endlastfoot
કેબલ મોડેમ & ડિજિટલ સિગ્નલ્સ કન્વર્ટ કરતું યુઝર-એન્ડ ડિવાઇસ \\
CMTS & પ્રોવાઇડર એન્ડ પર કેબલ મોડેમ ટર્મિનેશન સિસ્ટમ \\
HFC & હાઇબ્રિડ ફાઇબર-કોએક્સિયલ નેટવર્ક ઇન્ફ્રાસ્ટ્રક્ચર \\
DOCSIS & ડેટા ઓવર કેબલ સર્વિસ ઇન્ટરફેસ સ્પેસિફિકેશન \\
\end{longtable}
}

\textbf{આકૃતિ:}

\begin{verbatim}
                     FIBER
+{-{-}{-}{-}{-}{-}{-}{-}+        +{-}{-}{-}{-}{-}{-}{-}{-}+        +{-}{-}{-}{-}{-}{-}{-}{-}{-}+}
|  HOME  |  COAX  |  NODE  |        |   ISP   |
| MODEM  |{-{-}{-}{-}{-}{-}{-}{-}|        |{-}{-}{-}{-}{-}{-}{-}{-}|  CMTS   |{-}{-}{-}{-}{-}INTERNET}
+{-{-}{-}{-}{-}{-}{-}{-}+        +{-}{-}{-}{-}{-}{-}{-}{-}+        +{-}{-}{-}{-}{-}{-}{-}{-}{-}+}
                 NEIGHBORHOOD        HEAD{-END}
\end{verbatim}

\begin{itemize}
\tightlist
\item
  \textbf{શેર્ડ મીડિયમ}: નેબરહુડ બેન્ડવિડ્થ શેર કરે છે
\item
  \textbf{એસિમેટ્રિક}: સામાન્ય રીતે અપલોડ કરતાં ડાઉનલોડ ઝડપી
\item
  \textbf{DOCSIS સ્ટાન્ડર્ડ્સ}: સ્પીડ/ફીચર્સ માટે વિકસિત થતાં સ્પેસિફિકેશન્સ
\end{itemize}

\end{solutionbox}
\begin{mnemonicbox}
``CHAMPS'' - ``કેબલ, HFC, એક્સેસ, મોડેમ, પ્રોવાઇડર, શેર્ડ''

\end{mnemonicbox}
\subsection*{પ્રશ્ન 3(ક) અથવા [7
ગુણ]}\label{uxaaauxab0uxab6uxaa8-3uxa95-uxa85uxaa5uxab5-7-uxa97uxaa3}

\textbf{સંક્ષિપ્તમાં તમામ ટ્રાન્સમિશન મીડિયાનું વર્ણન કરો.}

\begin{solutionbox}
ટ્રાન્સમિશન મીડિયા એ ભૌતિક પાથ છે જેના દ્વારા નેટવર્કમાં ડેટા
પ્રવાસ કરે છે.

{\def\LTcaptype{none} % do not increment counter
\begin{longtable}[]{@{}
  >{\raggedright\arraybackslash}p{(\linewidth - 8\tabcolsep) * \real{0.2000}}
  >{\raggedright\arraybackslash}p{(\linewidth - 8\tabcolsep) * \real{0.1538}}
  >{\raggedright\arraybackslash}p{(\linewidth - 8\tabcolsep) * \real{0.2154}}
  >{\raggedright\arraybackslash}p{(\linewidth - 8\tabcolsep) * \real{0.2308}}
  >{\raggedright\arraybackslash}p{(\linewidth - 8\tabcolsep) * \real{0.2000}}@{}}
\toprule\noalign{}
\begin{minipage}[b]{\linewidth}\raggedright
મીડિયમ ટાઇપ
\end{minipage} & \begin{minipage}[b]{\linewidth}\raggedright
ઉદાહરણો
\end{minipage} & \begin{minipage}[b]{\linewidth}\raggedright
મેક્સ ડિસ્ટન્સ
\end{minipage} & \begin{minipage}[b]{\linewidth}\raggedright
મેક્સ બેન્ડવિડ્થ
\end{minipage} & \begin{minipage}[b]{\linewidth}\raggedright
એપ્લિકેશન
\end{minipage} \\
\midrule\noalign{}
\endhead
\bottomrule\noalign{}
\endlastfoot
\textbf{ગાઇડેડ (વાયર્ડ)} & & & & \\
ટ્વિસ્ટેડ પેર & UTP, STP & 100m & 10 Gbps & ઓફિસ LANs \\
કોએક્સિયલ કેબલ & RG-6, RG-59 & 500m & 10 Gbps & કેબલ TV, ઇન્ટરનેટ \\
ફાઇબર ઓપ્ટિક & સિંગલ-મોડ, મલ્ટી-મોડ & 100km+ & 100+ Tbps & બેકબોન્સ,
લોંગ-ડિસ્ટન્સ \\
\textbf{અનગાઇડેડ (વાયરલેસ)} & & & & \\
રેડિયો વેવ્સ & WiFi, સેલ્યુલર & 100m-50km & 600 Mbps & વાયરલેસ નેટવર્ક્સ \\
માઇક્રોવેવ્સ & ટેરેસ્ટ્રિયલ, સેટેલાઇટ & લાઇન ઓફ સાઇટ & 10 Gbps & પોઇન્ટ-ટુ-પોઇન્ટ
લિંક્સ \\
ઇન્ફ્રારેડ & IrDA & 1m & 16 Mbps & રિમોટ કંટ્રોલ્સ \\
\end{longtable}
}

\textbf{આકૃતિ:}

\begin{verbatim}
GUIDED MEDIA:
  Twisted Pair: ={======}
  Coaxial:      =====|=====|=====
  Fiber Optic:  ======================{}

UNGUIDED MEDIA:
  Radio:        ((( o )))
  Microwave:    {{-}{-}{-} {-}{-}{-}}
  Infrared:     * * * {}
\end{verbatim}

\begin{itemize}
\tightlist
\item
  \textbf{ગાઇડેડ મીડિયા}: સિગ્નલ્સને સીમિત કરતા ભૌતિક પાથ
\item
  \textbf{અનગાઇડેડ મીડિયા}: હવા/શૂન્યાવકાશ દ્વારા વાયરલેસ ટ્રાન્સમિશન
\item
  \textbf{લાક્ષણિકતાઓ}: બેન્ડવિડ્થ, એટેન્યુએશન, નોઇઝ ઇમ્યુનિટી, કોસ્ટ
\end{itemize}

\end{solutionbox}
\begin{mnemonicbox}
``TRIM-CWF'' - ``ટ્વિસ્ટેડ, રેડિયો, ઇન્ફ્રારેડ, માઇક્રોવેવ,
કોએક્સિયલ, વાયરલેસ, ફાઇબર''

\end{mnemonicbox}
\subsection*{પ્રશ્ન 4(અ) [3
ગુણ]}\label{uxaaauxab0uxab6uxaa8-4uxa85-3-uxa97uxaa3}

\textbf{DNS પર નોંધ લખો.}

\begin{solutionbox}
ડોમેન નેમ સિસ્ટમ (DNS) માનવ-મૈત્રીપૂર્ણ ડોમેન નેમ્સને IP એડ્રેસમાં
અનુવાદિત કરે છે.

{\def\LTcaptype{none} % do not increment counter
\begin{longtable}[]{@{}ll@{}}
\toprule\noalign{}
કોમ્પોનન્ટ & કાર્ય \\
\midrule\noalign{}
\endhead
\bottomrule\noalign{}
\endlastfoot
ડોમેન નેમ & હાયરાર્કિકલ, વાંચી શકાય તેવું એડ્રેસ (www.example.com) \\
DNS સર્વર & ડોમેન નેમ્સને IP એડ્રેસમાં રિઝોલ્વ કરે છે \\
રૂટ સર્વર & DNS હાયરાર્કીનો ટોપ, TLDs તરફ પોઇન્ટ કરે છે \\
TLD સર્વર & ટોપ-લેવલ ડોમેન્સ (.com, .org) મેનેજ કરે છે \\
રેકોર્ડ ટાઇપ્સ & A, AAAA, MX, CNAME, NS, PTR, વગેરે \\
\end{longtable}
}

\textbf{આકૃતિ:}

\begin{verbatim}
  CLIENT                                      ROOT DNS
+{-{-}{-}{-}{-}{-}{-}{-}+   1. Query                        +{-}{-}{-}{-}{-}{-}{-}{-}+}
|        |{-{-}{-}"www.example.com?"{-}{-}{-}{-}{-}{-}{-}{-}{-}{-}{-}{-}{-}|        |}
|        |   8. Response                     |        |
|        |{{-}{-}"192.0.2.1"{-}{-}{-}{-}{-}{-}{-}{-}{-}{-}{-}{-}{-}{-}{-}{-}{-}{-}{-}{-} |        |}
+{-{-}{-}{-}{-}{-}{-}{-}+      |                            +{-}{-}{-}{-}{-}{-}{-}{-}+}
                |                                \^{}
                |                                |
                v                                |
              +{-{-}{-}{-}{-}{-}{-}{-}+  2            +{-}{-}{-}{-}{-}{-}{-}{-}+ 7}
              |  LOCAL |{-{-}TLD Server?{-}|   TLD  |}
              |  DNS   |{{-}".com"{-}{-}{-}{-}{-}{-}{-}|        |}
              +{-{-}{-}{-}{-}{-}{-}{-}+  3            +{-}{-}{-}{-}{-}{-}{-}{-}+}
                  |                        \^{}
                  v 4                      | 6
              +{-{-}{-}{-}{-}{-}{-}{-}+                +{-}{-}{-}{-}{-}{-}{-}{-}+}
              |EXAMPLE |{{-}{-}{-}{-}{-}{-}{-}{-}{-}{-}{-}{-}{-}{-}{-}| DOMAIN |}
              |  DNS   |{-{-}{-}{-}{-}{-}{-}{-}{-}{-}{-}{-}{-}{-}{-}|SERVER  |}
              +{-{-}{-}{-}{-}{-}{-}{-}+ 5              +{-}{-}{-}{-}{-}{-}{-}{-}+}
\end{verbatim}

\begin{itemize}
\tightlist
\item
  \textbf{ડિસ્ટ્રિબ્યુટેડ ડેટાબેઝ}: હાયરાર્કિકલ, ગ્લોબલી ડિસ્ટ્રિબ્યુટેડ
\item
  \textbf{કેશિંગ}: પરફોર્મન્સ સુધારે છે, લોડ ઘટાડે છે
\item
  \textbf{ક્રિટિકલ ઇન્ફ્રાસ્ટ્રક્ચર}: ઇન્ટરનેટ ફંક્શનાલિટી માટે આવશ્યક
\end{itemize}

\end{solutionbox}
\begin{mnemonicbox}
``DIRT'' - ``ડોમેન નેમ્સ ઇન્ટુ રાઉટેબલ TCP/IP''

\end{mnemonicbox}
\subsection*{પ્રશ્ન 4(બ) [4
ગુણ]}\label{uxaaauxab0uxab6uxaa8-4uxaac-4-uxa97uxaa3}

\textbf{ફાઇલ ટ્રાન્સફર પ્રોટોકોલ સમજાવો.}

\begin{solutionbox}
ફાઇલ ટ્રાન્સફર પ્રોટોકોલ (FTP) નેટવર્ક પર ક્લાયન્ટ અને સર્વર વચ્ચે
ફાઇલ્સના ટ્રાન્સફરને સક્ષમ બનાવે છે.

{\def\LTcaptype{none} % do not increment counter
\begin{longtable}[]{@{}ll@{}}
\toprule\noalign{}
ફીચર & વર્ણન \\
\midrule\noalign{}
\endhead
\bottomrule\noalign{}
\endlastfoot
પોર્ટ & કંટ્રોલ: 21, ડેટા: 20 \\
મોડ & એક્ટિવ અને પેસિવ \\
સિક્યોરિટી & બેઝિક (ક્લિયર ટેક્સ્ટ), અથવા એન્ક્રિપ્શન માટે FTPS/SFTP \\
કમાન્ડ્સ & GET, PUT, LIST, DELETE, વગેરે \\
કનેક્શન & અલગ કંટ્રોલ અને ડેટા કનેક્શન્સનો ઉપયોગ કરે છે \\
\end{longtable}
}

\textbf{આકૃતિ:}

\begin{verbatim}
                     Control Connection (Port 21)
               +{-{-}{-}{-}{-}{-}{-}{-}{-}{-}{-}{-}{-}{-}{-}{-}{-}{-}{-}{-}{-}{-}{-}{-}{-}{-}{-}{-}{-}{-}{-}{-}+}
               |                                |
     +{-{-}{-}{-}{-}{-}{-}{-}+|                                |+{-}{-}{-}{-}{-}{-}{-}{-}+}
     |        ||                                ||        |
     | CLIENT |+{-{-}{-}{-}{-}{-}{-}{-}{-}{-}{-}{-}{-}{-}{-}{-}{-}{-}{-}{-}{-}{-}{-}{-}{-}{-}{-}{-}{-}{-}{-}{-}+| SERVER |}
     |        |                                  |        |
     |        |                                  |        |
     +{-{-}{-}{-}{-}{-}{-}{-}+                                  +{-}{-}{-}{-}{-}{-}{-}{-}+}
               +{-{-}{-}{-}{-}{-}{-}{-}{-}{-}{-}{-}{-}{-}{-}{-}{-}{-}{-}{-}{-}{-}{-}{-}{-}{-}{-}{-}{-}{-}{-}{-}+}
                     Data Connection (Port 20)
\end{verbatim}

\begin{itemize}
\tightlist
\item
  \textbf{ડ્યુઅલ ચેનલ}: કંટ્રોલ ચેનલ અને ડેટા ચેનલ
\item
  \textbf{ઓથેન્ટિકેશન}: યુઝરનેમ/પાસવર્ડ જરૂરી
\item
  \textbf{મોડ્સ}: ASCII (ટેક્સ્ટ) અથવા બાઇનરી (રો ડેટા)
\item
  \textbf{એક્ટિવ vs પેસિવ}: અલગ કનેક્શન સ્થાપના પદ્ધતિઓ
\end{itemize}

\end{solutionbox}
\begin{mnemonicbox}
``CAPS'' - ``કંટ્રોલ એન્ડ પોર્ટ સેપરેશન''

\end{mnemonicbox}
\subsection*{પ્રશ્ન 4(ક) [7
ગુણ]}\label{uxaaauxab0uxab6uxaa8-4uxa95-7-uxa97uxaa3}

\textbf{વિવિધ ઇન્ટરનેટ સેવાઓનું વર્ગીકરણ કરો અને વિગતવાર સમજાવો.}

\begin{solutionbox}
ઇન્ટરનેટ સેવાઓ નેટવર્ક પર વિવિધ કાર્યક્ષમતા પ્રદાન કરે છે.

{\def\LTcaptype{none} % do not increment counter
\begin{longtable}[]{@{}
  >{\raggedright\arraybackslash}p{(\linewidth - 6\tabcolsep) * \real{0.2373}}
  >{\raggedright\arraybackslash}p{(\linewidth - 6\tabcolsep) * \real{0.3051}}
  >{\raggedright\arraybackslash}p{(\linewidth - 6\tabcolsep) * \real{0.1356}}
  >{\raggedright\arraybackslash}p{(\linewidth - 6\tabcolsep) * \real{0.3220}}@{}}
\toprule\noalign{}
\begin{minipage}[b]{\linewidth}\raggedright
સેવા કેટેગરી
\end{minipage} & \begin{minipage}[b]{\linewidth}\raggedright
સામાન્ય પ્રોટોકોલ્સ
\end{minipage} & \begin{minipage}[b]{\linewidth}\raggedright
વર્ણન
\end{minipage} & \begin{minipage}[b]{\linewidth}\raggedright
એપ્લિકેશન ઉદાહરણો
\end{minipage} \\
\midrule\noalign{}
\endhead
\bottomrule\noalign{}
\endlastfoot
કોમ્યુનિકેશન & SMTP, POP3, IMAP & મેસેજનું આદાન-પ્રદાન & ઇમેઇલ, ઇન્સ્ટન્ટ મેસેજિંગ \\
ઇન્ફોર્મેશન એક્સેસ & HTTP, HTTPS & માહિતી સ્રોતોનો એક્સેસ & વર્લ્ડ વાઇડ વેબ,
પોર્ટલ્સ \\
ફાઇલ શેરિંગ & FTP, BitTorrent, SMB & ફાઇલ્સનું ટ્રાન્સફર અને શેરિંગ & ફાઇલ હોસ્ટિંગ,
P2P શેરિંગ \\
રિમોટ એક્સેસ & SSH, Telnet, RDP & રિમોટ કમ્પ્યુટર્સનો એક્સેસ & રિમોટ
એડમિનિસ્ટ્રેશન \\
રિયલ-ટાઇમ સર્વિસિસ & VoIP, WebRTC & લાઇવ કોમ્યુનિકેશન & વિડિયો કોન્ફરન્સિંગ,
VoIP \\
ડોમેન સર્વિસિસ & DNS, DHCP & નેટવર્ક ઇન્ફ્રાસ્ટ્રક્ચર & એડ્રેસ રિઝોલ્યુશન \\
\end{longtable}
}

\textbf{ઇન્ફોર્મેશન એક્સેસ સર્વિસિસ (વેબ):}

\begin{itemize}
\tightlist
\item
  \textbf{HTTP/HTTPS}: હાયપરટેક્સ્ટ ટ્રાન્સફર પ્રોટોકોલ, વેબનો પાયો
\item
  \textbf{HTML}: કન્ટેન્ટ ડિસ્પ્લે કરવા માટેનું ડોક્યુમેન્ટ ફોર્મેટ
\item
  \textbf{વેબ બ્રાઉઝર્સ}: વેબ કન્ટેન્ટ એક્સેસ અને રેન્ડર કરવા માટે ક્લાયન્ટ સોફ્ટવેર
\item
  \textbf{વેબ સર્વર્સ}: વેબસાઇટ્સ અને એપ્લિકેશન્સ હોસ્ટ કરે છે
\end{itemize}

\textbf{કોમ્યુનિકેશન સર્વિસિસ (ઇમેઇલ):}

\begin{itemize}
\tightlist
\item
  \textbf{SMTP}: ઇમેઇલ મોકલવા માટે
\item
  \textbf{POP3/IMAP}: ઇમેઇલ પ્રાપ્ત કરવા માટે
\item
  \textbf{કોમ્પોનન્ટ્સ}: મેઇલ યુઝર એજન્ટ્સ, ટ્રાન્સફર એજન્ટ્સ, ડિલિવરી એજન્ટ્સ
\end{itemize}

\textbf{ફાઇલ શેરિંગ સર્વિસિસ:}

\begin{itemize}
\tightlist
\item
  \textbf{FTP}: પરંપરાગત ફાઇલ ટ્રાન્સફર પ્રોટોકોલ
\item
  \textbf{P2P}: સેન્ટ્રલ સર્વર વગર ડિસ્ટ્રિબ્યુટેડ ફાઇલ શેરિંગ
\item
  \textbf{ક્લાઉડ સ્ટોરેજ}: રિમોટ ફાઇલ સ્ટોરેજ અને સિંક્રોનાઇઝેશન
\end{itemize}

\end{solutionbox}
\begin{mnemonicbox}
``CIFRRD'' - ``કોમ્યુનિકેશન, ઇન્ફોર્મેશન, ફાઇલ, રિમોટ,
રિયલ-ટાઇમ, ડોમેન''

\end{mnemonicbox}
\subsection*{પ્રશ્ન 4(અ) અથવા [3
ગુણ]}\label{uxaaauxab0uxab6uxaa8-4uxa85-uxa85uxaa5uxab5-3-uxa97uxaa3}

\textbf{મેઇલ પ્રોટોકોલ્સ સમજાવો.}

\begin{solutionbox}
મેઇલ પ્રોટોકોલ્સ વપરાશકર્તાઓ વચ્ચે ઇલેક્ટ્રોનિક મેસેજિંગ સરળ બનાવે છે.

{\def\LTcaptype{none} % do not increment counter
\begin{longtable}[]{@{}llll@{}}
\toprule\noalign{}
પ્રોટોકોલ & કાર્ય & પોર્ટ & દિશા \\
\midrule\noalign{}
\endhead
\bottomrule\noalign{}
\endlastfoot
SMTP & સિમ્પલ મેઇલ ટ્રાન્સફર પ્રોટોકોલ & 25, 587 & મેઇલ મોકલવું \\
POP3 & પોસ્ટ ઓફિસ પ્રોટોકોલ v3 & 110 & મેઇલ પ્રાપ્ત કરવું \\
IMAP & ઇન્ટરનેટ મેસેજ એક્સેસ પ્રોટોકોલ & 143 & એડવાન્સ્ડ મેઇલ રિટ્રિવલ \\
MIME & મલ્ટિપરપઝ ઇન્ટરનેટ મેઇલ એક્સટેન્શન & N/A & એટેચમેન્ટ એન્કોડિંગ \\
\end{longtable}
}

\textbf{આકૃતિ:}

\begin{verbatim}
+{-{-}{-}{-}{-}{-}{-}{-}{-}+   SMTP    +{-}{-}{-}{-}{-}{-}{-}{-}{-}+   POP3/IMAP  +{-}{-}{-}{-}{-}{-}{-}{-}{-}+}
|  Sender |{-{-}{-}{-}{-}{-}{-}{-}{-}{-}|  Mail   |{-}{-}{-}{-}{-}{-}{-}{-}{-}{-}{-}{-}{-}| Receiver|}
|  Client |           |  Server |              |  Client |
+{-{-}{-}{-}{-}{-}{-}{-}{-}+           +{-}{-}{-}{-}{-}{-}{-}{-}{-}+              +{-}{-}{-}{-}{-}{-}{-}{-}{-}+}
\end{verbatim}

\begin{itemize}
\tightlist
\item
  \textbf{SMTP}: આઉટગોઇંગ મેઇલ ડિલિવરી, પુશ પ્રોટોકોલ
\item
  \textbf{POP3}: સરળ મેઇલ રિટ્રિવલ, ડાઉનલોડ અને ડિલીટ કરે છે
\item
  \textbf{IMAP}: એડવાન્સ્ડ રિટ્રિવલ, સર્વર-સાઇડ સ્ટોરેજ, ફોલ્ડર્સ
\item
  \textbf{MIME}: નોન-ટેક્સ્ટ કન્ટેન્ટ માટે ઇમેઇલ ક્ષમતા વિસ્તારે છે
\end{itemize}

\end{solutionbox}
\begin{mnemonicbox}
``SIM-P'' - ``SMTP સેન્ડ્સ, IMAP મેનેજીસ, POP3 પુલ્સ''

\end{mnemonicbox}
\subsection*{પ્રશ્ન 4(બ) અથવા [4
ગુણ]}\label{uxaaauxab0uxab6uxaa8-4uxaac-uxa85uxaa5uxab5-4-uxa97uxaa3}

\textbf{સંક્ષિપ્તમાં VOIP નું વર્ણન કરો.}

\begin{solutionbox}
વોઇસ ઓવર ઇન્ટરનેટ પ્રોટોકોલ (VoIP) IP નેટવર્ક્સ પર વોઇસ
કોમ્યુનિકેશન ટ્રાન્સમિટ કરે છે.

{\def\LTcaptype{none} % do not increment counter
\begin{longtable}[]{@{}ll@{}}
\toprule\noalign{}
કોમ્પોનન્ટ & કાર્ય \\
\midrule\noalign{}
\endhead
\bottomrule\noalign{}
\endlastfoot
કોડેક & વોઇસ સિગ્નલ્સ એન્કોડ/ડિકોડ કરે છે \\
સિગ્નલિંગ પ્રોટોકોલ & કોલ સેટઅપ/ટિયરડાઉન (SIP, H.323) \\
ટ્રાન્સપોર્ટ પ્રોટોકોલ & વોઇસ પેકેટ ડિલિવરી (RTP) \\
QoS મેકેનિઝમ & વોઇસ ક્વોલિટી સુનિશ્ચિત કરે છે \\
\end{longtable}
}

\textbf{આકૃતિ:}

\begin{verbatim}
+{-{-}{-}{-}{-}{-}{-}{-}+   Internet/   +{-}{-}{-}{-}{-}{-}{-}{-}+}
| CALLER |{-{-}{-}IP Network{-}{-}{-}| CALLEE |}
|ENDPOINT|               |ENDPOINT|
+{-{-}{-}{-}{-}{-}{-}{-}+               +{-}{-}{-}{-}{-}{-}{-}{-}+}
    |                        |
 [Analog]                 [Analog]
    |                        |
 [Digital]                [Digital]
    |                        |
 [Packets]  {{-}{-}{-}RTP{-}{-}{-}  [Packets]}
\end{verbatim}

\begin{itemize}
\tightlist
\item
  \textbf{પેકેટાઇઝેશન}: એનાલોગ વોઇસને ડિજિટલ પેકેટ્સમાં કન્વર્ટ કરે છે
\item
  \textbf{લાભો}: કોસ્ટ સેવિંગ્સ, ફ્લેક્સિબિલિટી, એપ્સ સાથે ઇન્ટિગ્રેશન
\item
  \textbf{ચેલેન્જીસ}: ક્વોલિટી ઓફ સર્વિસ, લેટન્સી, જિટર, પેકેટ લોસ
\end{itemize}

\end{solutionbox}
\begin{mnemonicbox}
``PALS'' - ``પેકેટ્સ એલાઉઇંગ લાઇવ સ્પીચ''

\end{mnemonicbox}
\subsection*{પ્રશ્ન 4(ક) અથવા [7
ગુણ]}\label{uxaaauxab0uxab6uxaa8-4uxa95-uxa85uxaa5uxab5-7-uxa97uxaa3}

\textbf{TCP અને UDP પ્રોટોકોલ્સનું વર્ણન કરો.}

\begin{solutionbox}
TCP અને UDP TCP/IP સ્યુટમાં પ્રાથમિક ટ્રાન્સપોર્ટ લેયર પ્રોટોકોલ્સ
છે.

{\def\LTcaptype{none} % do not increment counter
\begin{longtable}[]{@{}lll@{}}
\toprule\noalign{}
ફીચર & TCP & UDP \\
\midrule\noalign{}
\endhead
\bottomrule\noalign{}
\endlastfoot
કનેક્શન & કનેક્શન-ઓરિએન્ટેડ & કનેક્શનલેસ \\
વિશ્વસનીયતા & ગેરંટેડ ડિલિવરી & બેસ્ટ-એફર્ટ ડિલિવરી \\
હેડર સાઇઝ & 20-60 બાઇટ્સ & 8 બાઇટ્સ \\
સ્પીડ & ઓવરહેડને કારણે ધીમું & મિનિમલ ઓવરહેડ સાથે ઝડપી \\
ઓર્ડર & સિક્વન્સ જાળવે છે & કોઈ સિક્વન્સ પ્રિઝર્વેશન નહીં \\
ફ્લો કંટ્રોલ & હા & ના \\
એરર રિકવરી & રિટ્રાન્સમિશન & કોઈ નહીં \\
ઉપયોગ & વેબ, ઇમેઇલ, ફાઇલ ટ્રાન્સફર & સ્ટ્રીમિંગ, DNS, VoIP \\
\end{longtable}
}

\textbf{TCP થ્રી-વે હેન્ડશેક:}

\begin{verbatim}
  CLIENT                SERVER
    |                     |
    |       SYN           |
    |{-{-}{-}{-}{-}{-}{-}{-}{-}{-}{-}{-}{-}{-}{-}{-}{-}{-}{-}{-}|}
    |                     |
    |     SYN{-ACK         |}
    |{{-}{-}{-}{-}{-}{-}{-}{-}{-}{-}{-}{-}{-}{-}{-}{-}{-}{-}{-}{-}|}
    |                     |
    |       ACK           |
    |{-{-}{-}{-}{-}{-}{-}{-}{-}{-}{-}{-}{-}{-}{-}{-}{-}{-}{-}{-}|}
    |                     |
    |    DATA TRANSFER    |
    |{{-}{-}{-}{-}{-}{-}{-}{-}{-}{-}{-}{-}{-}{-}{-}{-}{-}{-}{-}|}
\end{verbatim}

\textbf{TCP ફીચર્સ:}

\begin{itemize}
\tightlist
\item
  \textbf{વિશ્વસનીયતા}: એક્નોલેજમેન્ટ્સ, રિટ્રાન્સમિશન
\item
  \textbf{ફ્લો કંટ્રોલ}: વિન્ડો-બેઝ્ડ, ઓવરવ્હેલ્મિંગને રોકે છે
\item
  \textbf{કન્જેશન કંટ્રોલ}: સ્લો સ્ટાર્ટ, કન્જેશન અવોઇડન્સ
\item
  \textbf{કનેક્શન મેનેજમેન્ટ}: સ્થાપના, મેઇન્ટેનન્સ, ટર્મિનેશન
\end{itemize}

\textbf{UDP ફીચર્સ:}

\begin{itemize}
\tightlist
\item
  \textbf{લાઇટવેઇટ}: મિનિમલ હેડર્સ, કોઈ કનેક્શન સ્ટેટ નહીં
\item
  \textbf{લો લેટન્સી}: કોઈ હેન્ડશેકિંગ કે એક્નોલેજમેન્ટ્સ નહીં
\item
  \textbf{કોઈ ગેરંટી નહીં}: ડેટા આઉટ ઓફ ઓર્ડર, ડુપ્લિકેટેડ, અથવા બિલકુલ ન આવે
\item
  \textbf{બ્રોડકાસ્ટ/મલ્ટિકાસ્ટ}: વન-ટુ-મેની ટ્રાન્સમિશનને સપોર્ટ કરે છે
\end{itemize}

\end{solutionbox}
\begin{mnemonicbox}
``CRUFS'' - ``કનેક્શન, રિલાયબિલિટી, UDP ફાસ્ટ, સિમ્પલ''

\end{mnemonicbox}
\subsection*{પ્રશ્ન 5(અ) [3
ગુણ]}\label{uxaaauxab0uxab6uxaa8-5uxa85-3-uxa97uxaa3}

\textbf{ક્રિપ્ટોગ્રાફીનું વર્ણન કરો.}

\begin{solutionbox}
ક્રિપ્ટોગ્રાફી એ માહિતીનું રક્ષણ કરતી સુરક્ષિત કોમ્યુનિકેશન ટેકનિક્સનું
વિજ્ઞાન છે.

{\def\LTcaptype{none} % do not increment counter
\begin{longtable}[]{@{}lll@{}}
\toprule\noalign{}
ટાઇપ & વર્ણન & ઉદાહરણ \\
\midrule\noalign{}
\endhead
\bottomrule\noalign{}
\endlastfoot
સિમેટ્રિક & એન્ક્રિપ્શન અને ડિક્રિપ્શન માટે એક જ કી & AES, DES \\
એસિમેટ્રિક & એન્ક્રિપ્શન અને ડિક્રિપ્શન માટે અલગ કી & RSA, ECC \\
હેશ ફંક્શન્સ & વન-વે ફંક્શન્સ, ફિક્સ્ડ આઉટપુટ સાઇઝ & SHA-256, MD5 \\
ડિજિટલ સિગ્નેચર & ઓથેન્ટિકેશન અને ઇન્ટિગ્રિટી વેરિફિકેશન & RSA સિગ્નેચર \\
\end{longtable}
}

\textbf{આકૃતિ:}

\begin{verbatim}
SYMMETRIC:
  Sender {-{-}(Encrypt with Key K){-}{-} [Ciphertext] {-}{-}(Decrypt with Key K){-}{-} Receiver}

ASYMMETRIC:
  Sender {-{-}(Encrypt with Public Key){-}{-} [Ciphertext] {-}{-}(Decrypt with Private Key){-}{-} Receiver}
\end{verbatim}

\begin{itemize}
\tightlist
\item
  \textbf{કોન્ફિડેન્શિયાલિટી}: અનધિકૃત એક્સેસથી માહિતીનું રક્ષણ
\item
  \textbf{ઇન્ટિગ્રિટી}: માહિતી બદલાઈ નથી તે સુનિશ્ચિત કરવું
\item
  \textbf{ઓથેન્ટિકેશન}: કોમ્યુનિકેટિંગ પક્ષોની ઓળખ ચકાસવી
\end{itemize}

\end{solutionbox}
\begin{mnemonicbox}
``SHAPE'' - ``સિમેટ્રિક, હેશિંગ, એસિમેટ્રિક, પ્રોટેક્ટ,
એન્ક્રિપ્ટ''

\end{mnemonicbox}
\subsection*{પ્રશ્ન 5(બ) [4
ગુણ]}\label{uxaaauxab0uxab6uxaa8-5uxaac-4-uxa97uxaa3}

\textbf{સામાજિક મુદ્દાઓ સમજાવો અને હેકિંગ તેની સાવચેતીઓની પણ ચર્ચા કરો.}

\begin{solutionbox}
સાયબર સિક્યોરિટીમાં સામાજિક મુદ્દાઓમાં માનવ મેનિપ્યુલેશન અને સાયબર
ખતરાઓની સામાજિક અસરો શામેલ છે.

{\def\LTcaptype{none} % do not increment counter
\begin{longtable}[]{@{}
  >{\raggedright\arraybackslash}p{(\linewidth - 4\tabcolsep) * \real{0.5172}}
  >{\raggedright\arraybackslash}p{(\linewidth - 4\tabcolsep) * \real{0.2069}}
  >{\raggedright\arraybackslash}p{(\linewidth - 4\tabcolsep) * \real{0.2759}}@{}}
\toprule\noalign{}
\begin{minipage}[b]{\linewidth}\raggedright
સામાજિક મુદ્દો
\end{minipage} & \begin{minipage}[b]{\linewidth}\raggedright
વર્ણન
\end{minipage} & \begin{minipage}[b]{\linewidth}\raggedright
ઉદાહરણ
\end{minipage} \\
\midrule\noalign{}
\endhead
\bottomrule\noalign{}
\endlastfoot
સોશિયલ એન્જિનિયરિંગ & માહિતી જાહેર કરવા માટે લોકોને મેનિપ્યુલેટ કરવા & ફિશિંગ,
પ્રિટેક્સ્ટિંગ \\
પ્રાઇવસી કન્સર્ન & અનધિકૃત ડેટા કલેક્શન અને ઉપયોગ & ડેટા બ્રીચ, સર્વેલન્સ \\
ડિજિટલ ડિવાઇડ & ટેકનોલોજી એક્સેસમાં અસમાનતા & ગ્રામીણ વિસ્તારોમાં મર્યાદિત
ઇન્ટરનેટ \\
સાયબરબુલિંગ & અન્યને હેરાન કરવા માટે ટેકનોલોજીનો ઉપયોગ & ઓનલાઇન હેરાસમેન્ટ,
ધમકીઓ \\
\end{longtable}
}

\textbf{હેકિંગ ટાઇપ્સ:}

\begin{itemize}
\tightlist
\item
  \textbf{વ્હાઇટ હેટ}: એથિકલ હેકિંગ, સિક્યોરિટી સુધારણા
\item
  \textbf{બ્લેક હેટ}: મેલિશિયસ હેકિંગ, ગેરકાયદેસર પ્રવૃત્તિઓ
\item
  \textbf{ગ્રે હેટ}: એથિકલ અને શંકાસ્પદ ક્રિયાઓનું મિશ્રણ
\end{itemize}

\textbf{સાવચેતીઓ:}

\begin{itemize}
\tightlist
\item
  \textbf{એજ્યુકેશન}: નિયમિત સિક્યોરિટી અવેરનેસ ટ્રેનિંગ
\item
  \textbf{સ્ટ્રોંગ પોલિસીઝ}: સ્પષ્ટ સિક્યોરિટી પ્રક્રિયાઓ અને નીતિઓ
\item
  \textbf{ટેકનિકલ કંટ્રોલ્સ}: ફાયરવોલ્સ, એન્ટિવાઇરસ, એન્ક્રિપ્શન
\item
  \textbf{રેગ્યુલર અપડેટ્સ}: વલ્નરેબિલિટી સામે સિસ્ટમ્સ પેચિંગ
\item
  \textbf{મોનિટરિંગ}: એક્ટિવિટી લોગ્સ, ઇન્ટ્રુઝન ડિટેક્શન
\end{itemize}

\end{solutionbox}
\begin{mnemonicbox}
``STEPS'' - ``સોશિયલ એન્જિનિયરિંગ, ટ્રેનિંગ, એન્ક્રિપ્શન,
પેચિસ, સ્ટ્રોંગ પાસવર્ડ્સ''

\end{mnemonicbox}
\subsection*{પ્રશ્ન 5(ક) [7
ગુણ]}\label{uxaaauxab0uxab6uxaa8-5uxa95-7-uxa97uxaa3}

\textbf{IP સુરક્ષાને વિગતવાર સમજાવો.}

\begin{solutionbox}
IP સિક્યોરિટી (IPsec) એ IP લેયર પર કોમ્યુનિકેશન સુરક્ષિત કરતો
પ્રોટોકોલ સ્યુટ છે.

{\def\LTcaptype{none} % do not increment counter
\begin{longtable}[]{@{}
  >{\raggedright\arraybackslash}p{(\linewidth - 4\tabcolsep) * \real{0.4400}}
  >{\raggedright\arraybackslash}p{(\linewidth - 4\tabcolsep) * \real{0.2400}}
  >{\raggedright\arraybackslash}p{(\linewidth - 4\tabcolsep) * \real{0.3200}}@{}}
\toprule\noalign{}
\begin{minipage}[b]{\linewidth}\raggedright
કોમ્પોનન્ટ
\end{minipage} & \begin{minipage}[b]{\linewidth}\raggedright
કાર્ય
\end{minipage} & \begin{minipage}[b]{\linewidth}\raggedright
વર્ણન
\end{minipage} \\
\midrule\noalign{}
\endhead
\bottomrule\noalign{}
\endlastfoot
AH & ઓથેન્ટિકેશન હેડર & ઇન્ટિગ્રિટી અને ઓથેન્ટિકેશન પ્રદાન કરે છે \\
ESP & એન્કેપ્સુલેટિંગ સિક્યોરિટી પેલોડ & કોન્ફિડેન્શિયાલિટી, ઇન્ટિગ્રિટી, ઓથેન્ટિકેશન
પ્રદાન કરે છે \\
IKE & ઇન્ટરનેટ કી એક્સચેન્જ & સિક્યોરિટી એસોસિએશન સ્થાપિત અને સંચાલિત કરે છે \\
SA & સિક્યોરિટી એસોસિએશન & કનેક્શન માટે સિક્યોરિટી પેરામીટર્સ \\
\end{longtable}
}

\textbf{IPsec મોડ્સ:}

{\def\LTcaptype{none} % do not increment counter
\begin{longtable}[]{@{}lll@{}}
\toprule\noalign{}
મોડ & વર્ણન & એપ્લિકેશન \\
\midrule\noalign{}
\endhead
\bottomrule\noalign{}
\endlastfoot
ટ્રાન્સપોર્ટ & માત્ર પેલોડને સુરક્ષિત કરે છે & હોસ્ટ-ટુ-હોસ્ટ કોમ્યુનિકેશન \\
ટનલ & સંપૂર્ણ પેકેટને સુરક્ષિત કરે છે & ગેટવે-ટુ-ગેટવે (VPN) \\
\end{longtable}
}

\textbf{આકૃતિ:}

\begin{verbatim}
TRANSPORT MODE:
  +{-{-}{-}{-}{-}{-}+{-}{-}{-}{-}{-}{-}{-}+{-}{-}{-}{-}{-}{-}{-}{-}{-}{-}{-}{-}{-}{-}{-}{-}+}
  |  IP  | IPsec |    Payload     |
  |Header|Header |                |
  +{-{-}{-}{-}{-}{-}+{-}{-}{-}{-}{-}{-}{-}+{-}{-}{-}{-}{-}{-}{-}{-}{-}{-}{-}{-}{-}{-}{-}{-}+}

TUNNEL MODE:
  +{-{-}{-}{-}{-}{-}+{-}{-}{-}{-}{-}{-}{-}+{-}{-}{-}{-}{-}{-}+{-}{-}{-}{-}{-}{-}{-}+{-}{-}{-}{-}{-}{-}{-}{-}{-}{-}{-}{-}{-}{-}{-}{-}+}
  | New  | IPsec | Orig |  TCP  |    Payload     |
  | IP   |Header | IP   |Header |                |
  +{-{-}{-}{-}{-}{-}+{-}{-}{-}{-}{-}{-}{-}+{-}{-}{-}{-}{-}{-}+{-}{-}{-}{-}{-}{-}{-}+{-}{-}{-}{-}{-}{-}{-}{-}{-}{-}{-}{-}{-}{-}{-}{-}+}
\end{verbatim}

\textbf{IPsec સર્વિસિસ:}

\begin{itemize}
\tightlist
\item
  \textbf{ઓથેન્ટિકેશન}: સેન્ડરની ઓળખ ચકાસે છે
\item
  \textbf{કોન્ફિડેન્શિયાલિટી}: ઇવ્સડ્રોપિંગ રોકવા માટે ડેટા એન્ક્રિપ્ટ કરે છે
\item
  \textbf{ઇન્ટિગ્રિટી}: ડેટા મોડિફાઈ નથી થયો તે સુનિશ્ચિત કરે છે
\item
  \textbf{એન્ટી-રિપ્લે}: પેકેટ રિપ્લે એટેક રોકે છે
\end{itemize}

\textbf{IPsec ઇમ્પ્લિમેન્ટેશન:}

\begin{itemize}
\tightlist
\item
  \textbf{VPNs}: સિક્યોર રિમોટ એક્સેસ અને સાઇટ-ટુ-સાઇટ કનેક્શન
\item
  \textbf{L2TP/IPsec}: ટનલિંગને સિક્યોરિટી સાથે જોડે છે
\item
  \textbf{ઓથેન્ટિકેશન મેથડ્સ}: પ્રી-શેર્ડ કી, સર્ટિફિકેટ્સ, કર્બેરોસ
\end{itemize}

\end{solutionbox}
\begin{mnemonicbox}
``ACCEPT'' - ``ઓથેન્ટિકેશન, કોન્ફિડેન્શિયાલિટી,
ક્રિપ્ટોગ્રાફી, એન્કેપ્સુલેશન, પ્રોટોકોલ્સ, ટનલ''

\end{mnemonicbox}
\subsection*{પ્રશ્ન 5(અ) અથવા [3
ગુણ]}\label{uxaaauxab0uxab6uxaa8-5uxa85-uxa85uxaa5uxab5-3-uxa97uxaa3}

\textbf{નેટવર્ક સુરક્ષા વ્યાખ્યાયિત કરો અને તેના ઘટકો સમજાવો.}

{\def\LTcaptype{none} % do not increment counter
\begin{longtable}[]{@{}lll@{}}
\toprule\noalign{}
ઘટક & વર્ણન & ઉદાહરણો \\
\midrule\noalign{}
\endhead
\bottomrule\noalign{}
\endlastfoot
એક્સેસ કંટ્રોલ & નેટવર્ક એક્સેસને મર્યાદિત કરવું & પાસવર્ડ, મલ્ટી-ફેક્ટર ઓથ \\
થ્રેટ પ્રિવેન્શન & એટેક બ્લોક કરવા & ફાયરવોલ્સ, IDS/IPS \\
એન્ક્રિપ્શન & ટ્રાન્ઝિટમાં ડેટા સુરક્ષિત કરવો & SSL/TLS, IPsec \\
વલ્નરેબિલિટી મેનેજમેન્ટ & નબળાઈઓ ઓળખવી & સ્કેનિંગ, પેચિંગ \\
મોનિટરિંગ & નેટવર્ક એક્ટિવિટી નિરીક્ષણ & SIEM, લોગ એનાલિસિસ \\
\end{longtable}
}

\textbf{આકૃતિ:}

\begin{verbatim}
                +{-{-}{-}{-}{-}{-}{-}{-}{-}{-}{-}{-}{-}{-}{-}{-}{-}{-}+}
                | NETWORK SECURITY |
                +{-{-}{-}{-}{-}{-}{-}{-}{-}{-}{-}{-}{-}{-}{-}{-}{-}{-}+}
                         |
       +{-{-}{-}{-}{-}{-}{-}{-}+{-}{-}{-}{-}{-}{-}{-}{-}{-}{-}{-}+{-}{-}{-}{-}{-}{-}{-}{-}+{-}{-}{-}{-}{-}{-}{-}{-}+}
       |        |           |        |        |
 +{-{-}{-}{-}{-}{-}{-}{-}{-}{-}{-}+ +{-}{-}{-}{-}{-}{-}{-}+ +{-}{-}{-}{-}{-}{-}+ +{-}{-}{-}{-}{-}{-}+ +{-}{-}{-}{-}{-}{-}{-}{-}+}
 |   ACCESS  | |THREAT | |ENCRYP| |VULNER| |MONITOR |
 |  CONTROL  | |PREVENT| |TION  | |MGMT  | |ING     |
 +{-{-}{-}{-}{-}{-}{-}{-}{-}{-}{-}+ +{-}{-}{-}{-}{-}{-}{-}+ +{-}{-}{-}{-}{-}{-}+ +{-}{-}{-}{-}{-}{-}+ +{-}{-}{-}{-}{-}{-}{-}{-}+}
\end{verbatim}

\begin{itemize}
\tightlist
\item
  \textbf{કોન્ફિડેન્શિયાલિટી}: અનધિકૃત એક્સેસથી માહિતીનું રક્ષણ
\item
  \textbf{ઇન્ટિગ્રિટી}: માહિતીની ચોકસાઈ અને વિશ્વસનીયતા સુનિશ્ચિત કરવી
\item
  \textbf{અવેલેબિલિટી}: જરૂર પડે ત્યારે સિસ્ટમ્સ એક્સેસિબલ રાખવા
\end{itemize}

\begin{mnemonicbox}
``CIMA TV'' - ``કોન્ફિડેન્શિયાલિટી, ઇન્ટિગ્રિટી,
મોનિટરિંગ, એક્સેસ કંટ્રોલ, થ્રેટ્સ, વલ્નરેબિલિટીસ''

\end{mnemonicbox}
\subsection*{પ્રશ્ન 5(બ) અથવા [4
ગુણ]}\label{uxaaauxab0uxab6uxaa8-5uxaac-uxa85uxaa5uxab5-4-uxa97uxaa3}

\textbf{સંક્ષિપ્તમાં માહિતી ટેકનોલોજી (સુધારા) અધિનિયમ, 2008 અને ભારતમાં સાયબર
કાયદાઓ પર તેની અસરનું વર્ણન કરો.}

\begin{solutionbox}
IT (સુધારા) એક્ટ, 2008 ઉભરતા સાયબર સિક્યોરિટી પડકારોને સંબોધવા
માટે ભારતના સાયબર કાયદાઓ અપડેટ કર્યા.

{\def\LTcaptype{none} % do not increment counter
\begin{longtable}[]{@{}ll@{}}
\toprule\noalign{}
મુખ્ય પાસાં & વર્ણન \\
\midrule\noalign{}
\endhead
\bottomrule\noalign{}
\endlastfoot
સાયબર ક્રાઇમ & નવા ગુના ઉમેર્યા, પેનલ્ટી મજબૂત કરી \\
ઇલેક્ટ્રોનિક એવિડન્સ & કોર્ટમાં ડિજિટલ પુરાવાને માન્યતા આપી \\
ડેટા પ્રોટેક્શન & સંવેદનશીલ ડેટા માટે ફરજો લાદી \\
ઇન્ટરમીડિયરી લાયબિલિટી & સર્વિસ પ્રોવાઇડર્સ માટે જવાબદારીઓ વ્યાખ્યાયિત કરી \\
\end{longtable}
}

\textbf{મુખ્ય સેક્શન્સ:}

\begin{itemize}
\tightlist
\item
  \textbf{સેક્શન 43}: અનધિકૃત એક્સેસ, ડેટા થેફ્ટ માટે પેનલ્ટી
\item
  \textbf{સેક્શન 66}: કમ્પ્યુટર સંબંધિત ગુનાઓ અને સજાઓ
\item
  \textbf{સેક્શન 69}: ઇન્ટરસેપ્શન અને મોનિટરિંગ માટે અધિકારો
\item
  \textbf{સેક્શન 72A}: વ્યક્તિગત ડેટા ગોપનીયતાનું રક્ષણ
\end{itemize}

\textbf{સાયબર કાયદાઓ પર અસર:}

\begin{itemize}
\tightlist
\item
  \textbf{વધુ મજબૂત અમલ}: સાયબર ક્રાઇમ માટે વધારેલી જોગવાઈઓ
\item
  \textbf{વિસ્તૃત અવકાશ}: નવા ટેકનોલોજિકલ વિકાસને આવરી લીધા
\item
  \textbf{કોર્પોરેટ જવાબદારી}: ડેટા માટે સિક્યોરિટી પ્રેક્ટિસની આવશ્યકતા
\item
  \textbf{ગ્લોબલ એલાઇન્મેન્ટ}: આંતરરાષ્ટ્રીય ધોરણો સાથે સંકલન
\end{itemize}

\end{solutionbox}
\begin{mnemonicbox}
``SPEC'' - ``સિક્યોરિટી, પ્રાઇવસી, એવિડન્સ, સાયબર
ક્રાઇમ્સ''

\end{mnemonicbox}
\subsection*{પ્રશ્ન 5(ક) અથવા [7
ગુણ]}\label{uxaaauxab0uxab6uxaa8-5uxa95-uxa85uxaa5uxab5-7-uxa97uxaa3}

\textbf{SMTP, PEM, PGP, S/MINE, સ્પામના સંદર્ભમાં ઇમેઇલ સુરક્ષા સમજાવો.}

\begin{solutionbox}
ઇમેઇલ સિક્યોરિટી ઇમેઇલ કન્ટેન્ટ અને એકાઉન્ટ્સને અનધિકૃત એક્સેસ અને
એટેક્સથી સુરક્ષિત કરે છે.

{\def\LTcaptype{none} % do not increment counter
\begin{longtable}[]{@{}
  >{\raggedright\arraybackslash}p{(\linewidth - 4\tabcolsep) * \real{0.4231}}
  >{\raggedright\arraybackslash}p{(\linewidth - 4\tabcolsep) * \real{0.2308}}
  >{\raggedright\arraybackslash}p{(\linewidth - 4\tabcolsep) * \real{0.3462}}@{}}
\toprule\noalign{}
\begin{minipage}[b]{\linewidth}\raggedright
ટેકનોલોજી
\end{minipage} & \begin{minipage}[b]{\linewidth}\raggedright
કાર્ય
\end{minipage} & \begin{minipage}[b]{\linewidth}\raggedright
ફીચર્સ
\end{minipage} \\
\midrule\noalign{}
\endhead
\bottomrule\noalign{}
\endlastfoot
SMTP & સિમ્પલ મેઇલ ટ્રાન્સફર પ્રોટોકોલ & બેઝિક ઇમેઇલ ટ્રાન્સમિશન, મર્યાદિત
સિક્યોરિટી \\
PEM & પ્રાઇવસી એન્હાન્સ્ડ મેઇલ & અર્લી ઇમેઇલ એન્ક્રિપ્શન સ્ટાન્ડર્ડ \\
PGP & પ્રિટી ગુડ પ્રાઇવસી & એન્ડ-ટુ-એન્ડ એન્ક્રિપ્શન, ડિજિટલ સિગ્નેચર \\
S/MIME & સિક્યોર/મલ્ટિપરપઝ ઇન્ટરનેટ મેઇલ એક્સટેન્શન & સર્ટિફિકેટ-બેઝ્ડ એન્ક્રિપ્શન અને
સાઇનિંગ \\
એન્ટી-સ્પામ & અવાંછિત ઇમેઇલ ફિલ્ટરિંગ & કન્ટેન્ટ ફિલ્ટરિંગ, બ્લેકલિસ્ટ, ઓથેન્ટિકેશન \\
\end{longtable}
}

\textbf{SMTP સિક્યોરિટી ઇશ્યુ:}

\begin{itemize}
\tightlist
\item
  મૂળ રૂપે સિક્યોરિટી વગર ડિઝાઇન કરાયેલ
\item
  પછીથી ઓથેન્ટિકેશન એક્સટેન્શન (AUTH) ઉમેરાયા
\item
  એન્ક્રિપ્શન વગર ઇવ્સડ્રોપિંગ માટે વલ્નરેબલ
\item
  એન્ક્રિપ્ટેડ ટ્રાન્સમિશન માટે STARTTLS સપોર્ટ
\end{itemize}

\textbf{PGP ઇમેઇલ સિક્યોરિટી:}

\begin{verbatim}
SENDER                                 RECEIVER
  |                                      |
  |{-{-} Create message                     |}
  |{-{-} Sign with private key              |}
  |{-{-} Encrypt with recipients public key |}
  |                                      |
  |      Encrypted Email                 |
  |{-{-}{-}{-}{-}{-}{-}{-}{-}{-}{-}{-}{-}{-}{-}{-}{-}{-}{-}{-}{-}{-}{-}{-}{-}{-}{-}{-}{-}{-}{-}{-}{-}{-}{-}{-}{-}|}
  |                                      |
  |                                      |{-{-} Decrypt with private key}
  |                                      |{-{-} Verify with senders public key}
\end{verbatim}

\textbf{S/MIME ફીચર્સ:}

\begin{itemize}
\tightlist
\item
  ઓથેન્ટિકેશન માટે X.509 સર્ટિફિકેટ્સનો ઉપયોગ
\item
  એન્ક્રિપ્શન અને ડિજિટલ સિગ્નેચર પ્રદાન કરે છે
\item
  ઘણા ઇમેઇલ ક્લાયન્ટ્સમાં ઇન્ટિગ્રેટેડ
\item
  સર્ટિફિકેટ ઇન્ફ્રાસ્ટ્રક્ચરની જરૂર
\end{itemize}

\textbf{સ્પામ પ્રોટેક્શન:}

\begin{itemize}
\tightlist
\item
  \textbf{કન્ટેન્ટ ફિલ્ટરિંગ}: મેસેજ કન્ટેન્ટનું એનાલિસિસ
\item
  \textbf{સેન્ડર વેરિફિકેશન}: SPF, DKIM, DMARC
\item
  \textbf{બિહેવિયરલ એનાલિસિસ}: પેટર્ન રિકગ્નિશન
\item
  \textbf{બ્લેકલિસ્ટ/વ્હાઇટલિસ્ટ}: ચોક્કસ સેન્ડર્સને બ્લોકિંગ/એલાઉ કરવા
\end{itemize}

\textbf{ઇમેઇલ સિક્યોરિટી બેસ્ટ પ્રેક્ટિસિસ:}

\begin{itemize}
\tightlist
\item
  \textbf{એન્ક્રિપ્શન}: મેસેજ કન્ટેન્ટની ગોપનીયતા સુનિશ્ચિત કરવી
\item
  \textbf{ઓથેન્ટિકેશન}: સેન્ડરની ઓળખ ચકાસવી
\item
  \textbf{એક્સેસ કંટ્રોલ્સ}: ઇમેઇલ એકાઉન્ટ્સનું રક્ષણ કરવું
\item
  \textbf{ફિલ્ટરિંગ}: મેલિશિયસ અને અવાંછિત મેસેજ બ્લોક કરવા
\item
  \textbf{યુઝર એજ્યુકેશન}: ફિશિંગ પ્રયાસો ઓળખવા
\end{itemize}

\end{solutionbox}
\begin{mnemonicbox}
``SPEED'' - ``S/MIME, PGP, એન્ક્રિપ્શન, ઇમેઇલ સિક્યોરિટી,
DMARC''

\end{mnemonicbox}

\end{document}
