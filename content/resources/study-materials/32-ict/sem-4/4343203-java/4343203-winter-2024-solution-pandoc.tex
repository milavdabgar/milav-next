\documentclass[10pt,a4paper]{article}

% content/resources/templates/preamble.tex
\usepackage[margin=0.6in]{geometry}
\author{Milav Dabgar}
\usepackage{amsmath,amssymb,amsthm}
\usepackage{booktabs}
\usepackage{multirow}
\usepackage{xcolor}
\usepackage{tcolorbox}
\tcbuselibrary{breakable,skins}
\usepackage[colorlinks=true,linkcolor=blue]{hyperref}
\usepackage{titlesec}
\usepackage{enumitem}
\usepackage{tikz}
\usepackage{pgfplots}
\usepackage{circuitikz}
\usepackage[version=4]{mhchem}
\usepackage{longtable}
\usepackage{array}
\usepackage{float}
\usepackage{caption}
\usepackage{listings}

\lstset{
  basicstyle=\small\ttfamily,
  breaklines=true,
  breakatwhitespace=false,
  postbreak=\mbox{\textcolor{red}{$\hookrightarrow$}\space},
  float=false,
  numbers=left,
  numberstyle=\tiny\color{gray},
  numbersep=10pt,
  xleftmargin=2em,
  keywordstyle=\color{blue},
  commentstyle=\color{green!60!black},
  stringstyle=\color{purple},
  backgroundcolor=\color{gray!5},
  showstringspaces=false,
  tabsize=2,
  captionpos=b,
  keepspaces=true,
  columns=flexible
}

\pgfplotsset{compat=1.18}
\usetikzlibrary{shapes,arrows,positioning,calc,patterns,decorations.pathmorphing,decorations.markings,arrows.meta}

% Color scheme
\definecolor{headcolor}{RGB}{0,102,204}
\definecolor{keycolor}{RGB}{220,20,60}
\definecolor{solutioncolor}{RGB}{34,139,34}
\definecolor{mnemoniccolor}{RGB}{148,0,211}
\definecolor{codecolor}{RGB}{0,0,100}

% Spacing
\setlength{\parskip}{3pt}
\setlist[itemize]{nosep}
\setlist[enumerate]{nosep}

% Title formatting
\titleformat{\section}{\Large\bfseries\color{headcolor}}{\thesection}{1em}{}
\titleformat{\subsection}{\large\bfseries\color{headcolor}}{\thesubsection}{1em}{}

% Pandoc tightlist compatibility
\providecommand{\tightlist}{%
  \setlength{\itemsep}{0pt}\setlength{\parskip}{0pt}}

% Pandoc longtable compatibility
\newcounter{none}
\def\thenone{}


% content/resources/templates/english-boxes.tex
% This file is currently empty - it exists to maintain consistency with the import structure.
% Add custom environments here if needed in the future.


\begin{document}

\begin{center}
{\Huge\bfseries\color{headcolor} Subject Name Solutions}\\[5pt]
{\LARGE 4343203 -- Winter 2024}\\[3pt]
{\large Semester 1 Study Material}\\[3pt]
{\normalsize\textit{Detailed Solutions and Explanations}}
\end{center}

\vspace{10pt}

\subsection*{Question 1(a) [3 marks]}\label{q1a}

\textbf{List out various Primitive data types in Java.}

\begin{solutionbox}
Java offers eight primitive data types for storing
simple values directly in memory.


{\def\LTcaptype{none} % do not increment counter
\vspace{-5pt}
\captionof{table}{Java Primitive Data Types}
\vspace{-10pt}
\begin{longtable}[]{@{}llll@{}}
\toprule\noalign{}
Data Type & Size & Description & Range \\
\midrule\noalign{}
\endhead
\bottomrule\noalign{}
\endlastfoot
byte & 8 bits & Integer type & -128 to 127 \\
short & 16 bits & Integer type & -32,768 to 32,767 \\
int & 32 bits & Integer type & -2\^{}31 to 2\^{}31-1 \\
long & 64 bits & Integer type & -2\^{}63 to 2\^{}63-1 \\
float & 32 bits & Floating-point & Single precision \\
double & 64 bits & Floating-point & Double precision \\
char & 16 bits & Character & Unicode characters \\
boolean & 1 bit & Logical & true or false \\
\end{longtable}
}

\end{solutionbox}
\begin{mnemonicbox}
``BILFDC-B: Byte Int Long Float Double Char Boolean
types''

\end{mnemonicbox}
\subsection*{Question 1(b) [4 marks]}\label{q1b}

\textbf{Explain Structure of Java Program with suitable example.}

\begin{solutionbox}
Java program structure follows a specific organization
with package declarations, imports, class definitions, and methods.

\textbf{Diagram: Java Program Structure}

\begin{lstlisting}
+-----------------------+
| Documentation Comments|
+-----------------------+
| Package Declaration   |
+-----------------------+
| Import Statements     |
+-----------------------+
| Class Declaration     |
|  +------------------+ |
|  | Variables        | |
|  | Constructors     | |
|  | Methods          | |
|  +------------------+ |
+-----------------------+
\end{lstlisting}

\textbf{Code Block:}

\begin{lstlisting}[language=Java]
// Documentation comment
/**
 * Simple program to demonstrate Java structure
 * @author GTU Student
 */

// Package declaration
package com.example;

// Import statements
import java.util.Scanner;

// Class declaration
public class HelloWorld {
    // Variable declaration
    private String message;
    
    // Constructor
    public HelloWorld() {
        message = "Hello, World!";
    }
    
    // Method
    public void displayMessage() {
        System.out.println(message);
    }
    
    // Main method
    public static void main(String[] args) {
        HelloWorld obj = new HelloWorld();
        obj.displayMessage();
    }
}
\end{lstlisting}

\end{solutionbox}
\begin{mnemonicbox}
``PICOM: Package Import Class Objects Methods in
order''

\end{mnemonicbox}
\subsection*{Question 1(c) [7 marks]}\label{q1c}

\textbf{List arithmetic operators in Java. Develop a Java program using
any three arithmetic operators and show the output of program.}

\begin{solutionbox}
Arithmetic operators in Java perform mathematical
operations on numeric values.


{\def\LTcaptype{none} % do not increment counter
\vspace{-5pt}
\captionof{table}{Java Arithmetic Operators}
\vspace{-10pt}
\begin{longtable}[]{@{}lll@{}}
\toprule\noalign{}
Operator & Description & Example \\
\midrule\noalign{}
\endhead
\bottomrule\noalign{}
\endlastfoot
+ & Addition & a + b \\
- & Subtraction & a - b \\
* & Multiplication & a * b \\
/ & Division & a / b \\
\% & Modulus (Remainder) & a \% b \\
++ & Increment & a++ or ++a \\
-- & Decrement & a-- or --a \\
\end{longtable}
}

\textbf{Code Block:}

\begin{lstlisting}[language=Java]
public class ArithmeticDemo {
    public static void main(String[] args) {
        int a = 10;
        int b = 3;
        
        // Addition
        int sum = a + b;
        
        // Multiplication
        int product = a * b;
        
        // Modulus
        int remainder = a % b;
        
        // Display results
System.out.println("Values:

a = " + a + ",

b = " + b);

        System.out.println("Addition (a + b): " + sum);
        System.out.println("Multiplication (a * b): " + product);
        System.out.println("Modulus (a % b): " + remainder);
    }
}
\end{lstlisting}

\textbf{Output:}

\begin{lstlisting}
Values:

a = 10,

b = 3

Addition (a + b): 13
Multiplication (a * b): 30
Modulus (a % b): 1
\end{lstlisting}

\end{solutionbox}
\begin{mnemonicbox}
``SAME: Sum Addition Multiply Exponentiation basic
operations''

\end{mnemonicbox}
\subsection*{Question 1(c OR) [7
marks]}\label{question-1c-or-7-marks}

\textbf{Write syntax of Java for loop statement. Develop a Java program
to find out prime number between 1 to 10.}

\begin{solutionbox}
The for loop in Java provides a compact way to iterate
over a range of values.

\textbf{Syntax of Java for loop:}

\begin{lstlisting}
for (initialization; condition; increment/decrement) {
    // statements to be executed
}
\end{lstlisting}

\textbf{Code Block:}

\begin{lstlisting}[language=Java]
public class PrimeNumbers {
    public static void main(String[] args) {
        System.out.println("Prime numbers between 1 and 10:");
        
        // Check each number from 1 to 10
        for (int num = 1; num <= 10; num++) {
            boolean isPrime = true;
            
            // Check if num is divisible by any number from 2 to num-1
            if (num > 1) {
                for (int i = 2; i < num; i++) {
if (num %

i == 0) {

                        isPrime = false;
                        break;
                    }
                }
                
                // Print if prime
                if (isPrime) {
                    System.out.print(num + " ");
                }
            }
        }
    }
}
\end{lstlisting}

\textbf{Output:}

\begin{lstlisting}
Prime numbers between 1 and 10:
2 3 5 7
\end{lstlisting}

\end{solutionbox}
\begin{mnemonicbox}
``ICE: Initialize, Check, Execute steps of for loop''

\end{mnemonicbox}
\subsection*{Question 2(a) [3 marks]}\label{q2a}

\textbf{List the differences between Procedure-Oriented Programming
(POP) and Object-Oriented Programming (OOP).}

\begin{solutionbox}
Procedure-Oriented and Object-Oriented Programming
represent fundamentally different programming paradigms.


{\def\LTcaptype{none} % do not increment counter
\vspace{-5pt}
\captionof{table}{POP vs OOP}
\vspace{-10pt}
\begin{longtable}[]{@{}lll@{}}
\toprule\noalign{}
Feature & Procedure-Oriented & Object-Oriented \\
\midrule\noalign{}
\endhead
\bottomrule\noalign{}
\endlastfoot
Focus & Functions/Procedures & Objects \\
Data & Separate from functions & Encapsulated in objects \\
Security & Less secure & More secure with access control \\
Inheritance & Not supported & Supported \\
Reusability & Less reusable & Highly reusable \\
Complexity & Simpler for small programs & Better for complex systems \\
\end{longtable}
}

\begin{itemize}
\tightlist
\item
  \textbf{Organization}: POP divides into functions; OOP groups into
  objects
\item
  \textbf{Approach}: POP follows top-down; OOP follows bottom-up
\end{itemize}

\end{solutionbox}
\begin{mnemonicbox}
``FIOS: Functions In Objects Structure key
difference''

\end{mnemonicbox}
\subsection*{Question 2(b) [4 marks]}\label{q2b}

\textbf{Explain static keyword with example.}

\begin{solutionbox}
The static keyword in Java creates class-level members
shared across all objects of that class.


{\def\LTcaptype{none} % do not increment counter
\vspace{-5pt}
\captionof{table}{Uses of static Keyword}
\vspace{-10pt}
\begin{longtable}[]{@{}
  >{\raggedright\arraybackslash}p{(\linewidth - 4\tabcolsep) * \real{0.2174}}
  >{\raggedright\arraybackslash}p{(\linewidth - 4\tabcolsep) * \real{0.3913}}
  >{\raggedright\arraybackslash}p{(\linewidth - 4\tabcolsep) * \real{0.3913}}@{}}
\toprule\noalign{}
\begin{minipage}[b]{\linewidth}\raggedright
Use
\end{minipage} & \begin{minipage}[b]{\linewidth}\raggedright
Purpose
\end{minipage} & \begin{minipage}[b]{\linewidth}\raggedright
Example
\end{minipage} \\
\midrule\noalign{}
\endhead
\bottomrule\noalign{}
\endlastfoot
static variable & Shared across all objects &
\passthrough{\lstinline!static int count;!} \\
static method & Can be called without object &
\passthrough{\lstinline!static void display()!} \\
static block & Executed when class loads &
\passthrough{\lstinline!static \{ // code \}!} \\
static nested class & Associated with outer class &
\passthrough{\lstinline!static class Inner \{\}!} \\
\end{longtable}
}

\textbf{Code Block:}

\begin{lstlisting}[language=Java]
public class Counter {
    // Static variable shared by all objects
    static int count = 0;
    
    // Instance variable unique to each object
    int instanceCount = 0;
    
    // Constructor
    Counter() {
        count++;         // Increments the shared count
        instanceCount++; // Increments this object's count
    }
    
    public static void main(String[] args) {
        Counter c1 = new Counter();
        Counter c2 = new Counter();
        Counter c3 = new Counter();
        
        System.out.println("Static count: " + Counter.count);
        System.out.println("c1's instance count: " + c1.instanceCount);
        System.out.println("c2's instance count: " + c2.instanceCount);
        System.out.println("c3's instance count: " + c3.instanceCount);
    }
}
\end{lstlisting}

\textbf{Output:}

\begin{lstlisting}
Static count: 3
c1's instance count: 1
c2's instance count: 1
c3's instance count: 1
\end{lstlisting}

\end{solutionbox}
\begin{mnemonicbox}
``CBMS: Class-level, Before objects, Memory single,
Shared by all''

\end{mnemonicbox}
\subsection*{Question 2(c) [7 marks]}\label{q2c}

\textbf{Define Constructor. List types of Constructors. Develop a java
code to explain Parameterized constructor.}

\begin{solutionbox}
A constructor is a special method with the same name as
its class, used to initialize objects when created.

\textbf{Types of Constructors:}


{\def\LTcaptype{none} % do not increment counter
\vspace{-5pt}
\captionof{table}{Constructor Types in Java}
\vspace{-10pt}
\begin{longtable}[]{@{}
  >{\raggedright\arraybackslash}p{(\linewidth - 4\tabcolsep) * \real{0.2143}}
  >{\raggedright\arraybackslash}p{(\linewidth - 4\tabcolsep) * \real{0.4643}}
  >{\raggedright\arraybackslash}p{(\linewidth - 4\tabcolsep) * \real{0.3214}}@{}}
\toprule\noalign{}
\begin{minipage}[b]{\linewidth}\raggedright
Type
\end{minipage} & \begin{minipage}[b]{\linewidth}\raggedright
Description
\end{minipage} & \begin{minipage}[b]{\linewidth}\raggedright
Example
\end{minipage} \\
\midrule\noalign{}
\endhead
\bottomrule\noalign{}
\endlastfoot
Default & No parameters, created by compiler &
\passthrough{\lstinline!Student() \{\}!} \\
No-arg & Explicitly defined, no parameters &
\passthrough{\lstinline!Student() \{ name = "Unknown"; \}!} \\
Parameterized & Accepts parameters &
\passthrough{\lstinline!Student(String n) \{ name = n; \}!} \\
Copy & Creates object from another object &
\passthrough{\lstinline!Student(Student s) \{ name = s.name; \}!} \\
\end{longtable}
}

\textbf{Code Block:}

\begin{lstlisting}[language=Java]
public class Student {
    // Instance variables
    private String name;
    private int age;
    private String course;
    
    // Parameterized constructor
    public Student(String name, int age, String course) {
        this.name = name;
        this.age = age;
        this.course = course;
    }
    
    // Method to display student details
    public void displayDetails() {
        System.out.println("Student Details:");
        System.out.println("Name: " + name);
        System.out.println("Age: " + age);
        System.out.println("Course: " + course);
    }
    
    // Main method for demonstration
    public static void main(String[] args) {
        // Creating object using parameterized constructor
        Student student1 = new Student("John", 20, "Computer Science");
        student1.displayDetails();
        
        // Another student
        Student student2 = new Student("Lisa", 22, "Engineering");
        student2.displayDetails();
    }
}
\end{lstlisting}

\textbf{Output:}

\begin{lstlisting}
Student Details:
Name: John
Age: 20
Course: Computer Science
Student Details:
Name: Lisa
Age: 22
Course: Engineering
\end{lstlisting}

\end{solutionbox}
\begin{mnemonicbox}
``IDCR: Initialize Data Create Ready objects''

\end{mnemonicbox}
\subsection*{Question 2(a OR) [3
marks]}\label{question-2a-or-3-marks}

\textbf{List the basic OOP concepts in Java and explain any one.}

\begin{solutionbox}
Java implements Object-Oriented Programming through
several fundamental concepts.


{\def\LTcaptype{none} % do not increment counter
\vspace{-5pt}
\captionof{table}{Basic OOP Concepts in Java}
\vspace{-10pt}
\begin{longtable}[]{@{}ll@{}}
\toprule\noalign{}
Concept & Description \\
\midrule\noalign{}
\endhead
\bottomrule\noalign{}
\endlastfoot
Encapsulation & Binding data and methods together \\
Inheritance & Creating new classes from existing ones \\
Polymorphism & One interface, multiple implementations \\
Abstraction & Hiding implementation details \\
Association & Relationship between objects \\
\end{longtable}
}

\textbf{Encapsulation Example:}

\begin{lstlisting}[language=Java]
public class Person {
    // Private data - hidden from outside
    private String name;
    private int age;
    
    // Public methods - interface to access data
    public void setName(String name) {
        this.name = name;
    }
    
    public String getName() {
        return name;
    }
    
    public void setAge(int age) {
        // Validation ensures data integrity
        if (age > 0 && age < 120) {
            this.age = age;
        } else {
            System.out.println("Invalid age");
        }
    }
    
    public int getAge() {
        return age;
    }
}
\end{lstlisting}

\begin{itemize}
\tightlist
\item
  \textbf{Data Hiding}: Private variables inaccessible from outside
\item
  \textbf{Controlled Access}: Through public methods (getters/setters)
\item
  \textbf{Integrity}: Data validation ensures correct values
\end{itemize}

\end{solutionbox}
\begin{mnemonicbox}
``EIPA: Encapsulate Inherit Polymorphize Abstract''

\end{mnemonicbox}
\subsection*{Question 2(b OR) [4
marks]}\label{question-2b-or-4-marks}

\textbf{Explain final keyword with example.}

\begin{solutionbox}
The final keyword in Java restricts changes to
entities, creating constants, unchangeable methods, and non-inheritable
classes.


{\def\LTcaptype{none} % do not increment counter
\vspace{-5pt}
\captionof{table}{Uses of final Keyword}
\vspace{-10pt}
\begin{longtable}[]{@{}
  >{\raggedright\arraybackslash}p{(\linewidth - 4\tabcolsep) * \real{0.2273}}
  >{\raggedright\arraybackslash}p{(\linewidth - 4\tabcolsep) * \real{0.3636}}
  >{\raggedright\arraybackslash}p{(\linewidth - 4\tabcolsep) * \real{0.4091}}@{}}
\toprule\noalign{}
\begin{minipage}[b]{\linewidth}\raggedright
Use
\end{minipage} & \begin{minipage}[b]{\linewidth}\raggedright
Effect
\end{minipage} & \begin{minipage}[b]{\linewidth}\raggedright
Example
\end{minipage} \\
\midrule\noalign{}
\endhead
\bottomrule\noalign{}
\endlastfoot
final variable & Cannot be modified &
\passthrough{\lstinline!final int MAX = 100;!} \\
final method & Cannot be overridden &
\passthrough{\lstinline!final void display() \{\}!} \\
final class & Cannot be extended &
\passthrough{\lstinline!final class Math \{\}!} \\
final parameter & Cannot be changed in method &
\passthrough{\lstinline!void method(final int x) \{\}!} \\
\end{longtable}
}

\textbf{Code Block:}

\begin{lstlisting}[language=Java]
public class FinalDemo {
    // Final variable (constant)
    final int MAX_SPEED = 120;
    
    // Final method cannot be overridden
    final void showLimit() {
        System.out.println("Speed limit: " + MAX_SPEED);
    }
    
    public static void main(String[] args) {
        FinalDemo car = new FinalDemo();
        car.showLimit();
        
        // This would cause compile error:
        // car.MAX_SPEED = 150;
    }
}

// Final class cannot be extended
final class MathUtil {
    public int square(int num) {
        return num * num;
    }
}

// This would cause compile error:
// class AdvancedMath extends MathUtil { }
\end{lstlisting}

\textbf{Output:}

\begin{lstlisting}
Speed limit: 120
\end{lstlisting}

\end{solutionbox}
\begin{mnemonicbox}
``VMP: Variables Methods Permanence with final''

\end{mnemonicbox}
\subsection*{Question 2(c OR) [7
marks]}\label{question-2c-or-7-marks}

\textbf{Write scope of java access modifier. Develop a java code to
explain public modifier.}

\begin{solutionbox}
Access modifiers in Java control visibility and
accessibility of classes, methods, and variables.


{\def\LTcaptype{none} % do not increment counter
\vspace{-5pt}
\captionof{table}{Java Access Modifier Scope}
\vspace{-10pt}
\begin{longtable}[]{@{}lllll@{}}
\toprule\noalign{}
Modifier & Class & Package & Subclass & World \\
\midrule\noalign{}
\endhead
\bottomrule\noalign{}
\endlastfoot
private & ✓ & ✗ & ✗ & ✗ \\
default (no modifier) & ✓ & ✓ & ✗ & ✗ \\
protected & ✓ & ✓ & ✓ & ✗ \\
public & ✓ & ✓ & ✓ & ✓ \\
\end{longtable}
}

\textbf{Code Block:}

\begin{lstlisting}[language=Java]
// File: PublicDemo.java
package com.example;

// Public class accessible from anywhere
public class PublicDemo {
    // Public variable accessible from anywhere
    public String message = "Hello, World!";
    
    // Public method accessible from anywhere
    public void displayMessage() {
        System.out.println(message);
    }
}

// File: Main.java
package com.test;

// Importing from different package
import com.example.PublicDemo;

public class Main {
    public static void main(String[] args) {
        // Creating object of class from different package
        PublicDemo demo = new PublicDemo();
        
        // Accessing public variable from different package
        System.out.println("Message: " + demo.message);
        
        // Calling public method from different package
        demo.displayMessage();
        
        // Modifying public variable from different package
        demo.message = "Modified message";
        demo.displayMessage();
    }
}
\end{lstlisting}

\textbf{Output:}

\begin{lstlisting}
Message: Hello, World!
Hello, World!
Modified message
\end{lstlisting}

\end{solutionbox}
\begin{mnemonicbox}
``CEPM: Class Everywhere Public Most accessible''

\end{mnemonicbox}
\subsection*{Question 3(a) [3 marks]}\label{q3a}

\textbf{List out different types of inheritance and explain any one with
example.}

\begin{solutionbox}
Inheritance enables a class to inherit attributes and
behaviors from another class.


{\def\LTcaptype{none} % do not increment counter
\vspace{-5pt}
\captionof{table}{Types of Inheritance in Java}
\vspace{-10pt}
\begin{longtable}[]{@{}
  >{\raggedright\arraybackslash}p{(\linewidth - 2\tabcolsep) * \real{0.3158}}
  >{\raggedright\arraybackslash}p{(\linewidth - 2\tabcolsep) * \real{0.6842}}@{}}
\toprule\noalign{}
\begin{minipage}[b]{\linewidth}\raggedright
Type
\end{minipage} & \begin{minipage}[b]{\linewidth}\raggedright
Description
\end{minipage} \\
\midrule\noalign{}
\endhead
\bottomrule\noalign{}
\endlastfoot
Single & One class extends one class \\
Multilevel & Chain of inheritance (A\rightarrowB\rightarrowC) \\
Hierarchical & Multiple classes extend one class \\
Multiple & One class inherits from multiple classes (through
interfaces) \\
Hybrid & Combination of multiple inheritance types \\
\end{longtable}
}

\textbf{Single Inheritance Example:}

\begin{lstlisting}[language=Java]
// Parent class
class Animal {
    protected String name;
    
    public Animal(String name) {
        this.name = name;
    }
    
    public void eat() {
        System.out.println(name + " is eating");
    }
}

// Child class inheriting from Animal
class Dog extends Animal {
    private String breed;
    
    public Dog(String name, String breed) {
        super(name);  // Call parent constructor
        this.breed = breed;
    }
    
    public void bark() {
        System.out.println(name + " is barking");
    }
    
    public void displayInfo() {
        System.out.println("Name: " + name);
        System.out.println("Breed: " + breed);
    }
}

// Main class
public class InheritanceDemo {
    public static void main(String[] args) {
        Dog dog = new Dog("Max", "Labrador");
        dog.displayInfo();
        dog.eat();     // Inherited method
        dog.bark();    // Own method
    }
}
\end{lstlisting}

\textbf{Output:}

\begin{lstlisting}
Name: Max
Breed: Labrador
Max is eating
Max is barking
\end{lstlisting}

\end{solutionbox}
\begin{mnemonicbox}
``SMHMH: Single Multilevel Hierarchical Multiple
Hybrid types''

\end{mnemonicbox}
\subsection*{Question 3(b) [4 marks]}\label{q3b}

\textbf{Explain any two String buffer class methods with suitable
example.}

\begin{solutionbox}
StringBuffer is a mutable sequence of characters used
for modifying strings, offering various manipulation methods.


{\def\LTcaptype{none} % do not increment counter
\vspace{-5pt}
\captionof{table}{Two StringBuffer Methods}
\vspace{-10pt}
\begin{longtable}[]{@{}
  >{\raggedright\arraybackslash}p{(\linewidth - 4\tabcolsep) * \real{0.3200}}
  >{\raggedright\arraybackslash}p{(\linewidth - 4\tabcolsep) * \real{0.3600}}
  >{\raggedright\arraybackslash}p{(\linewidth - 4\tabcolsep) * \real{0.3200}}@{}}
\toprule\noalign{}
\begin{minipage}[b]{\linewidth}\raggedright
Method
\end{minipage} & \begin{minipage}[b]{\linewidth}\raggedright
Purpose
\end{minipage} & \begin{minipage}[b]{\linewidth}\raggedright
Syntax
\end{minipage} \\
\midrule\noalign{}
\endhead
\bottomrule\noalign{}
\endlastfoot
append() & Adds string at the end &
\passthrough{\lstinline!sb.append(String str)!} \\
insert() & Adds string at specified position &
\passthrough{\lstinline!sb.insert(int offset, String str)!} \\
\end{longtable}
}

\textbf{Code Block:}

\begin{lstlisting}[language=Java]
public class StringBufferMethodsDemo {
    public static void main(String[] args) {
        // Create StringBuffer
        StringBuffer sb = new StringBuffer("Hello");
        System.out.println("Original: " + sb);
        
        // append() method - adds text at the end
        sb.append(" World");
        System.out.println("After append(): " + sb);
        
        // Can append different data types
        sb.append('!');
        sb.append(2024);
        System.out.println("After appending more: " + sb);
        
        // Reset for demonstration
        sb = new StringBuffer("Java");
        System.out.println("\nNew Original: " + sb);
        
        // insert() method - adds text at specified position
        sb.insert(0, "Learn ");
        System.out.println("After insert() at beginning: " + sb);
        
        sb.insert(10, " Programming");
        System.out.println("After insert() in middle: " + sb);
    }
}
\end{lstlisting}

\textbf{Output:}

\begin{lstlisting}
Original: Hello
After append(): Hello World
After appending more: Hello World!2024

New Original: Java
After insert() at beginning: Learn Java
After insert() in middle: Learn Java Programming
\end{lstlisting}

\end{solutionbox}
\begin{mnemonicbox}
``AIMS: Append Insert Modify StringBuffer''

\end{mnemonicbox}
\subsection*{Question 3(c) [7 marks]}\label{q3c}

\textbf{Define Interface. Write a java program to demonstrate multiple
inheritance using interface.}

\begin{solutionbox}
An interface is a contract that declares methods a
class must implement, enabling multiple inheritance in Java.

\textbf{Definition:} An interface is a reference type containing only
constants, method signatures, default methods, static methods, and
nested types with no implementation for abstract methods.

\textbf{Diagram: Multiple Inheritance using Interfaces}

\includegraphics[width=1\linewidth,height=\textheight,keepaspectratio]{mermaid-cc7e9890.pdf}

\textbf{Code Block:}

\begin{lstlisting}[language=Java]
// First interface
interface Printable {
    void print();
}

// Second interface
interface Scannable {
    void scan();
}

// Class implementing multiple interfaces
class Device implements Printable, Scannable {
    private String model;
    
    public Device(String model) {
        this.model = model;
    }
    
    // Implementation of print() method from Printable
    @Override
    public void print() {
        System.out.println(model + " is printing a document");
    }
    
    // Implementation of scan() method from Scannable
    @Override
    public void scan() {
        System.out.println(model + " is scanning a document");
    }
    
    // Class's own method
    public void getModel() {
        System.out.println("Device Model: " + model);
    }
}

// Main class
public class MultipleInheritanceDemo {
    public static void main(String[] args) {
        Device device = new Device("HP LaserJet");
        
        // Display model
        device.getModel();
        
        // Using methods from multiple interfaces
        device.print();
        device.scan();
        
        // Checking if device is an instance of interfaces
        System.out.println("Is device Printable? " + (device instanceof Printable));
        System.out.println("Is device Scannable? " + (device instanceof Scannable));
    }
}
\end{lstlisting}

\textbf{Output:}

\begin{lstlisting}
Device Model: HP LaserJet
HP LaserJet is printing a document
HP LaserJet is scanning a document
Is device Printable? true
Is device Scannable? true
\end{lstlisting}

\end{solutionbox}
\begin{mnemonicbox}
``IMAC: Interface Multiple Abstract Contract''

\end{mnemonicbox}
\subsection*{Question 3(a OR) [3
marks]}\label{question-3a-or-3-marks}

\textbf{Give differences between Abstract class and Interface.}

\begin{solutionbox}
Abstract classes and interfaces are both used for
abstraction but differ in several key aspects.


{\def\LTcaptype{none} % do not increment counter
\vspace{-5pt}
\captionof{table}{Abstract Class vs Interface}
\vspace{-10pt}
\begin{longtable}[]{@{}
  >{\raggedright\arraybackslash}p{(\linewidth - 4\tabcolsep) * \real{0.2571}}
  >{\raggedright\arraybackslash}p{(\linewidth - 4\tabcolsep) * \real{0.4286}}
  >{\raggedright\arraybackslash}p{(\linewidth - 4\tabcolsep) * \real{0.3143}}@{}}
\toprule\noalign{}
\begin{minipage}[b]{\linewidth}\raggedright
Feature
\end{minipage} & \begin{minipage}[b]{\linewidth}\raggedright
Abstract Class
\end{minipage} & \begin{minipage}[b]{\linewidth}\raggedright
Interface
\end{minipage} \\
\midrule\noalign{}
\endhead
\bottomrule\noalign{}
\endlastfoot
Keyword & abstract & interface \\
Methods & Both abstract and concrete & Abstract (and default since Java
8) \\
Variables & Any type & Only public static final \\
Constructor & Has & Doesn't have \\
Inheritance & Single & Multiple \\
Access Modifiers & Any & Only public \\
Purpose & Partial implementation & Complete abstraction \\
\end{longtable}
}

\begin{itemize}
\tightlist
\item
  \textbf{Implementation}: Abstract classes can provide partial
  implementation; interfaces traditionally provide none
\item
  \textbf{Relationship}: Abstract class says ``is-a''; interface says
  ``can-do-this''
\end{itemize}

\end{solutionbox}
\begin{mnemonicbox}
``MAPS: Methods Access Purpose Single vs multiple''

\end{mnemonicbox}
\subsection*{Question 3(b OR) [4
marks]}\label{question-3b-or-4-marks}

\textbf{Explain any two String class methods with suitable example.}

\begin{solutionbox}
The String class offers various methods for string
manipulation, comparison, and transformation.


{\def\LTcaptype{none} % do not increment counter
\vspace{-5pt}
\captionof{table}{Two String Methods}
\vspace{-10pt}
\begin{longtable}[]{@{}
  >{\raggedright\arraybackslash}p{(\linewidth - 4\tabcolsep) * \real{0.3200}}
  >{\raggedright\arraybackslash}p{(\linewidth - 4\tabcolsep) * \real{0.3600}}
  >{\raggedright\arraybackslash}p{(\linewidth - 4\tabcolsep) * \real{0.3200}}@{}}
\toprule\noalign{}
\begin{minipage}[b]{\linewidth}\raggedright
Method
\end{minipage} & \begin{minipage}[b]{\linewidth}\raggedright
Purpose
\end{minipage} & \begin{minipage}[b]{\linewidth}\raggedright
Syntax
\end{minipage} \\
\midrule\noalign{}
\endhead
\bottomrule\noalign{}
\endlastfoot
substring() & Extracts portion of string &
\passthrough{\lstinline!str.substring(int beginIndex, int endIndex)!} \\
equals() & Compares string content &
\passthrough{\lstinline!str1.equals(str2)!} \\
\end{longtable}
}

\textbf{Code Block:}

\begin{lstlisting}[language=Java]
public class StringMethodsDemo {
    public static void main(String[] args) {
        String message = "Java Programming";
        
        // substring() method
        // Extract "Java" (index 0 to 3)
        String sub1 = message.substring(0, 4);
        System.out.println("substring(0, 4): " + sub1);
        
        // Extract "Programming" (index 5 to end)
        String sub2 = message.substring(5);
        System.out.println("substring(5): " + sub2);
        
        // equals() method
        String str1 = "Hello";
        String str2 = "Hello";
        String str3 = "hello";
        String str4 = new String("Hello");
        
        System.out.println("\nComparing strings with equals():");
        System.out.println("str1.equals(str2): " + str1.equals(str2));  // true
        System.out.println("str1.equals(str3): " + str1.equals(str3));  // false
        System.out.println("str1.equals(str4): " + str1.equals(str4));  // true
        
        System.out.println("\nComparing strings with ==:");
        System.out.println("str1 == str2: " + (str1 == str2));  // true
        System.out.println("str1 == str4: " + (str1 == str4));  // false
    }
}
\end{lstlisting}

\textbf{Output:}

\begin{lstlisting}
substring(0, 4): Java
substring(5): Programming

Comparing strings with equals():
str1.equals(str2): true
str1.equals(str3): false
str1.equals(str4): true

Comparing strings with ==:
str1 == str2: true
str1 == str4: false
\end{lstlisting}

\end{solutionbox}
\begin{mnemonicbox}
``SEC: Substring Equals Compare string content''

\end{mnemonicbox}
\subsection*{Question 3(c OR) [7
marks]}\label{question-3c-or-7-marks}

\textbf{Explain package and list out steps to create package with
suitable example.}

\begin{solutionbox}
A package in Java is a namespace that organizes related
classes and interfaces, preventing naming conflicts.

\textbf{Steps to Create a Package:}


{\def\LTcaptype{none} % do not increment counter
\vspace{-5pt}
\captionof{table}{Package Creation Steps}
\vspace{-10pt}
\begin{longtable}[]{@{}ll@{}}
\toprule\noalign{}
Step & Action \\
\midrule\noalign{}
\endhead
\bottomrule\noalign{}
\endlastfoot
1 & Declare package name at the top of source files \\
2 & Create proper directory structure matching package name \\
3 & Save Java file in the appropriate directory \\
4 & Compile with javac -d option to create package directory \\
5 & Run the program with fully qualified name \\
\end{longtable}
}

\textbf{Code Block:}

\begin{lstlisting}[language=Java]
// Step 1: Declare package at the top (save as Calculator.java)
package com.example.math;

// The Calculator class
public class Calculator {
    public int add(int a, int b) {
        return a + b;
    }
    
    public int subtract(int a, int b) {
        return a - b;
    }
    
    public int multiply(int a, int b) {
        return a * b;
    }
    
    public double divide(int a, int b) {
        if (b == 0) {
            throw new ArithmeticException("Cannot divide by zero");
        }
        return (double) a / b;
    }
}

// Step 1: Declare package (save as CalculatorApp.java)
package com.example.app;

// Import the package
import com.example.math.Calculator;

public class CalculatorApp {
    public static void main(String[] args) {
        // Using the Calculator class from the package
        Calculator calc = new Calculator();
        
        System.out.println("Addition: " + calc.add(10, 5));
        System.out.println("Subtraction: " + calc.subtract(10, 5));
        System.out.println("Multiplication: " + calc.multiply(10, 5));
        System.out.println("Division: " + calc.divide(10, 5));
    }
}
\end{lstlisting}

\textbf{Terminal Commands:}

\begin{lstlisting}
// Step 2: Create directory structure
mkdir -p com/example/math
mkdir -p com/example/app

// Step 3: Place files in appropriate directories
mv Calculator.java com/example/math/
mv CalculatorApp.java com/example/app/

// Step 4: Compile with -d option
javac -d . com/example/math/Calculator.java
javac -d . -cp . com/example/app/CalculatorApp.java

// Step 5: Run with fully qualified name
java com.example.app.CalculatorApp
\end{lstlisting}

\textbf{Output:}

\begin{lstlisting}
Addition: 15
Subtraction: 5
Multiplication: 50
Division: 2.0
\end{lstlisting}

\end{solutionbox}
\begin{mnemonicbox}
``DISCO: Declare Import Save Compile Organize''

\end{mnemonicbox}
\subsection*{Question 4(a) [3 marks]}\label{q4a}

\textbf{List types of errors in Java.}

\begin{solutionbox}
Java programs can encounter various errors during
development and execution.


{\def\LTcaptype{none} % do not increment counter
\vspace{-5pt}
\captionof{table}{Types of Errors in Java}
\vspace{-10pt}
\begin{longtable}[]{@{}
  >{\raggedright\arraybackslash}p{(\linewidth - 4\tabcolsep) * \real{0.3529}}
  >{\raggedright\arraybackslash}p{(\linewidth - 4\tabcolsep) * \real{0.3824}}
  >{\raggedright\arraybackslash}p{(\linewidth - 4\tabcolsep) * \real{0.2647}}@{}}
\toprule\noalign{}
\begin{minipage}[b]{\linewidth}\raggedright
Error Type
\end{minipage} & \begin{minipage}[b]{\linewidth}\raggedright
When Occurs
\end{minipage} & \begin{minipage}[b]{\linewidth}\raggedright
Example
\end{minipage} \\
\midrule\noalign{}
\endhead
\bottomrule\noalign{}
\endlastfoot
Compile-time Errors & During compilation & Syntax errors, type errors \\
Runtime Errors & During execution & NullPointerException,
ArrayIndexOutOfBoundsException \\
Logical Errors & During execution with wrong output & Incorrect
calculation, infinite loop \\
Linkage Errors & During class loading & NoClassDefFoundError \\
Thread Death & When thread terminates & ThreadDeath \\
\end{longtable}
}

\begin{itemize}
\tightlist
\item
  \textbf{Syntax Errors}: Missing semicolons, brackets, or typos
\item
  \textbf{Semantic Errors}: Type mismatches, incompatible operations
\item
  \textbf{Exceptions}: Runtime issues requiring handling
\end{itemize}

\end{solutionbox}
\begin{mnemonicbox}
``CRLLT: Compile Runtime Logical Linkage Thread
errors''

\end{mnemonicbox}
\subsection*{Question 4(b) [4 marks]}\label{q4b}

\textbf{Explain try catch block with example.}

\begin{solutionbox}
The try-catch block in Java handles exceptions,
allowing programs to continue executing despite errors.

\textbf{Diagram: Try-Catch Flow}

\includegraphics[width=1\linewidth,height=\textheight,keepaspectratio]{mermaid-242b4db2.pdf}

\textbf{Code Block:}

\begin{lstlisting}[language=Java]
public class TryCatchDemo {
    public static void main(String[] args) {
        int[] numbers = {10, 20, 30};
        
        try {
            // Try to access an element outside array bounds
            System.out.println("Trying to access element 5: " + numbers[4]);
            
            // This code will not be executed if exception occurs
            System.out.println("This won't be printed");
        } 
        catch (ArrayIndexOutOfBoundsException e) {
            // Handle the specific exception
            System.out.println("Exception caught: " + e.getMessage());
            System.out.println("Array index out of bounds");
        }
        catch (Exception e) {
            // Handle any other exceptions
            System.out.println("General exception caught: " + e.getMessage());
        }
        finally {
            // This block always executes
            System.out.println("Finally block executed");
        }
        
        // Program continues execution
        System.out.println("Program continues after exception handling");
    }
}
\end{lstlisting}

\textbf{Output:}

\begin{lstlisting}
Exception caught: Index 4 out of bounds for length 3
Array index out of bounds
Finally block executed
Program continues after exception handling
\end{lstlisting}

\end{solutionbox}
\begin{mnemonicbox}
``TCFE: Try Catch Finally Execute despite errors''

\end{mnemonicbox}
\subsection*{Question 4(c) [7 marks]}\label{q4c}

\textbf{List out any four differences between method overloading and
overriding. Write a java code to explain method overriding.}

\begin{solutionbox}
Method overloading and overriding are both forms of
polymorphism but differ in functionality and implementation.


{\def\LTcaptype{none} % do not increment counter
\vspace{-5pt}
\captionof{table}{Method Overloading vs Overriding}
\vspace{-10pt}
\begin{longtable}[]{@{}
  >{\raggedright\arraybackslash}p{(\linewidth - 4\tabcolsep) * \real{0.1915}}
  >{\raggedright\arraybackslash}p{(\linewidth - 4\tabcolsep) * \real{0.4043}}
  >{\raggedright\arraybackslash}p{(\linewidth - 4\tabcolsep) * \real{0.4043}}@{}}
\toprule\noalign{}
\begin{minipage}[b]{\linewidth}\raggedright
Feature
\end{minipage} & \begin{minipage}[b]{\linewidth}\raggedright
Method Overloading
\end{minipage} & \begin{minipage}[b]{\linewidth}\raggedright
Method Overriding
\end{minipage} \\
\midrule\noalign{}
\endhead
\bottomrule\noalign{}
\endlastfoot
Occurrence & Same class & Parent and child classes \\
Parameters & Different parameters & Same parameters \\
Return Type & Can be different & Must be same or covariant \\
Access Modifier & Can be different & Can't be more restrictive \\
Binding & Compile-time (static) & Runtime (dynamic) \\
Purpose & Multiple behaviors of same method & Specialized
implementation \\
Inheritance & Not required & Required \\
@Override & Not used & Recommended \\
\end{longtable}
}

\textbf{Code Block:}

\begin{lstlisting}[language=Java]
// Parent class
class Animal {
    public void makeSound() {
        System.out.println("Animal makes a sound");
    }
    
    public void eat() {
        System.out.println("Animal eats food");
    }
}

// Child class overriding methods
class Dog extends Animal {
    // Method overriding
    @Override
    public void makeSound() {
        System.out.println("Dog barks");
    }
    
    @Override
    public void eat() {
        System.out.println("Dog eats meat");
    }
}

// Another child class with different overrides
class Cat extends Animal {
    // Method overriding
    @Override
    public void makeSound() {
        System.out.println("Cat meows");
    }
}

// Main class to demonstrate method overriding
public class MethodOverridingDemo {
    public static void main(String[] args) {
        // Parent class reference and object
        Animal animal = new Animal();
        
        // Child class references and objects
        Animal dog = new Dog();
        Animal cat = new Cat();
        
        // Demonstrating method overriding behavior
        System.out.println("Animal behavior:");
        animal.makeSound();
        animal.eat();
        
        System.out.println("\nDog behavior:");
        dog.makeSound();  // Calls overridden method
        dog.eat();        // Calls overridden method
        
        System.out.println("\nCat behavior:");
        cat.makeSound();  // Calls overridden method
        cat.eat();        // Calls parent method (not overridden)
    }
}
\end{lstlisting}

\textbf{Output:}

\begin{lstlisting}
Animal behavior:
Animal makes a sound
Animal eats food

Dog behavior:
Dog barks
Dog eats meat

Cat behavior:
Cat meows
Animal eats food
\end{lstlisting}

\end{solutionbox}
\begin{mnemonicbox}
``SBRE: Same-name, Base-derived, Runtime-resolution,
Extend functionality''

\end{mnemonicbox}
\subsection*{Question 4(a OR) [3
marks]}\label{question-4a-or-3-marks}

\textbf{List any four inbuilt exceptions.}

\begin{solutionbox}
Java provides many built-in exception classes that
represent various error conditions.


{\def\LTcaptype{none} % do not increment counter
\vspace{-5pt}
\captionof{table}{Four Common Inbuilt Exceptions}
\vspace{-10pt}
\begin{longtable}[]{@{}
  >{\raggedright\arraybackslash}p{(\linewidth - 4\tabcolsep) * \real{0.4074}}
  >{\raggedright\arraybackslash}p{(\linewidth - 4\tabcolsep) * \real{0.2593}}
  >{\raggedright\arraybackslash}p{(\linewidth - 4\tabcolsep) * \real{0.3333}}@{}}
\toprule\noalign{}
\begin{minipage}[b]{\linewidth}\raggedright
Exception
\end{minipage} & \begin{minipage}[b]{\linewidth}\raggedright
Cause
\end{minipage} & \begin{minipage}[b]{\linewidth}\raggedright
Package
\end{minipage} \\
\midrule\noalign{}
\endhead
\bottomrule\noalign{}
\endlastfoot
NullPointerException & Access/modify null reference & java.lang \\
ArrayIndexOutOfBoundsException & Invalid array index & java.lang \\
ArithmeticException & Invalid arithmetic operation (division by zero) &
java.lang \\
ClassCastException & Invalid class casting & java.lang \\
\end{longtable}
}

\begin{itemize}
\tightlist
\item
  \textbf{Unchecked}: Runtime exceptions (don't require explicit
  handling)
\item
  \textbf{Hierarchy}: All extend from Exception class
\item
  \textbf{Handling}: Can be caught with try-catch blocks
\end{itemize}

\end{solutionbox}
\begin{mnemonicbox}
``NAAC: Null Array Arithmetic Cast common
exceptions''

\end{mnemonicbox}
\subsection*{Question 4(b OR) [4
marks]}\label{question-4b-or-4-marks}

\textbf{Explain ``throw'' keyword with suitable example.}

\begin{solutionbox}
The throw keyword in Java manually generates exceptions
for exceptional conditions in programs.


{\def\LTcaptype{none} % do not increment counter
\vspace{-5pt}
\captionof{table}{throw Keyword Usage}
\vspace{-10pt}
\begin{longtable}[]{@{}
  >{\raggedright\arraybackslash}p{(\linewidth - 2\tabcolsep) * \real{0.4375}}
  >{\raggedright\arraybackslash}p{(\linewidth - 2\tabcolsep) * \real{0.5625}}@{}}
\toprule\noalign{}
\begin{minipage}[b]{\linewidth}\raggedright
Usage
\end{minipage} & \begin{minipage}[b]{\linewidth}\raggedright
Purpose
\end{minipage} \\
\midrule\noalign{}
\endhead
\bottomrule\noalign{}
\endlastfoot
throw new ExceptionType() & Create and throw exception \\
throw new ExceptionType(message) & Create with custom message \\
throws in method signature & Declare exceptions method might throw \\
Can throw checked/unchecked & Requires try-catch for checked
exceptions \\
\end{longtable}
}

\textbf{Code Block:}

\begin{lstlisting}[language=Java]
public class ThrowDemo {
    // Method that uses throw to generate exception
    public static void validateAge(int age) {
        // Checking for invalid age
        if (age < 0) {
            throw new IllegalArgumentException("Age cannot be negative");
        }
        
        // Checking for age restriction
        if (age < 18) {
            throw new ArithmeticException("Not eligible to vote");
        } else {
            System.out.println("Eligible to vote");
        }
    }
    
    public static void main(String[] args) {
        try {
            // Valid age
            System.out.println("Validating age 20:");
            validateAge(20);
            
            // Underage
            System.out.println("\nValidating age 15:");
            validateAge(15);
        } catch (ArithmeticException e) {
            System.out.println("ArithmeticException: " + e.getMessage());
        } catch (IllegalArgumentException e) {
            System.out.println("IllegalArgumentException: " + e.getMessage());
        }
        
        try {
            // Negative age
            System.out.println("\nValidating age -5:");
            validateAge(-5);
        } catch (Exception e) {
            System.out.println("Exception: " + e.getMessage());
        }
    }
}
\end{lstlisting}

\textbf{Output:}

\begin{lstlisting}
Validating age 20:
Eligible to vote

Validating age 15:
ArithmeticException: Not eligible to vote

Validating age -5:
Exception: Age cannot be negative
\end{lstlisting}

\end{solutionbox}
\begin{mnemonicbox}
``CET: Create Exception Throw for error handling''

\end{mnemonicbox}
\subsection*{Question 4(c OR) [7
marks]}\label{question-4c-or-7-marks}

\textbf{Compare `this' keyword Vs `Super' keyword. Explain super keyword
with suitable Example.}

\begin{solutionbox}
The `this' and `super' keywords are used for
referencing in Java, with distinct purposes and behaviors.


{\def\LTcaptype{none} % do not increment counter
\vspace{-5pt}
\captionof{table}{this vs super Keyword Comparison}
\vspace{-10pt}
\begin{longtable}[]{@{}
  >{\raggedright\arraybackslash}p{(\linewidth - 4\tabcolsep) * \real{0.2432}}
  >{\raggedright\arraybackslash}p{(\linewidth - 4\tabcolsep) * \real{0.3514}}
  >{\raggedright\arraybackslash}p{(\linewidth - 4\tabcolsep) * \real{0.4054}}@{}}
\toprule\noalign{}
\begin{minipage}[b]{\linewidth}\raggedright
Feature
\end{minipage} & \begin{minipage}[b]{\linewidth}\raggedright
this Keyword
\end{minipage} & \begin{minipage}[b]{\linewidth}\raggedright
super Keyword
\end{minipage} \\
\midrule\noalign{}
\endhead
\bottomrule\noalign{}
\endlastfoot
Reference & Current class & Parent class \\
Usage & Access current class members & Access parent class members \\
Constructor call & this() & super() \\
Variable resolution & this.var (current class) & super.var (parent
class) \\
Method invocation & this.method() (current class) & super.method()
(parent class) \\
Position & First statement in constructor & First statement in
constructor \\
Inheritance & Not related to inheritance & Used with inheritance \\
\end{longtable}
}

\textbf{Code Block:}

\begin{lstlisting}[language=Java]
// Parent class
class Vehicle {
    // Parent class variables
    protected String brand = "Ford";
    protected String color = "Red";
    
    // Parent class constructor
    Vehicle() {
        System.out.println("Vehicle constructor called");
    }
    
    // Parent class method
    void displayInfo() {
        System.out.println("Brand: " + brand);
        System.out.println("Color: " + color);
    }
}

// Child class
class Car extends Vehicle {
    // Child class variables (same names as parent)
    private String brand = "Toyota";
    private String color = "Blue";
    
    // Child class constructor
    Car() {
        super(); // Call parent constructor
        System.out.println("Car constructor called");
    }
    
    // Method using super with variables
    void printDetails() {
        // Access child class variables using this
        System.out.println("Car brand (this): " + this.brand);
        System.out.println("Car color (this): " + this.color);
        
        // Access parent class variables using super
        System.out.println("Vehicle brand (super): " + super.brand);
        System.out.println("Vehicle color (super): " + super.color);
    }
    
    // Method using super with methods
    @Override
    void displayInfo() {
        System.out.println("Car information:");
        // Call parent method
        super.displayInfo();
        System.out.println("Model: Corolla");
    }
}

// Main class
public class SuperKeywordDemo {
    public static void main(String[] args) {
        // Create Car object
        Car myCar = new Car();
        
        System.out.println("\nVariable access with this and super:");
        myCar.printDetails();
        
        System.out.println("\nMethod call with super:");
        myCar.displayInfo();
    }
}
\end{lstlisting}

\textbf{Output:}

\begin{lstlisting}
Vehicle constructor called
Car constructor called

Variable access with this and super:
Car brand (this): Toyota
Car color (this): Blue
Vehicle brand (super): Ford
Vehicle color (super): Red

Method call with super:
Car information:
Brand: Ford
Color: Red
Model: Corolla
\end{lstlisting}

\end{solutionbox}
\begin{mnemonicbox}
``PCIM: Parent Class Inheritance Members with super''

\end{mnemonicbox}
\subsection*{Question 5(a) [3 marks]}\label{q5a}

\textbf{List Different Stream Classes.}

\begin{solutionbox}
Java I/O provides various stream classes for handling
input and output operations.


{\def\LTcaptype{none} % do not increment counter
\vspace{-5pt}
\captionof{table}{Java Stream Classes}
\vspace{-10pt}
\begin{longtable}[]{@{}
  >{\raggedright\arraybackslash}p{(\linewidth - 2\tabcolsep) * \real{0.4000}}
  >{\raggedright\arraybackslash}p{(\linewidth - 2\tabcolsep) * \real{0.6000}}@{}}
\toprule\noalign{}
\begin{minipage}[b]{\linewidth}\raggedright
Category
\end{minipage} & \begin{minipage}[b]{\linewidth}\raggedright
Stream Classes
\end{minipage} \\
\midrule\noalign{}
\endhead
\bottomrule\noalign{}
\endlastfoot
Byte Streams & FileInputStream, FileOutputStream, BufferedInputStream,
BufferedOutputStream \\
Character Streams & FileReader, FileWriter, BufferedReader,
BufferedWriter \\
Data Streams & DataInputStream, DataOutputStream \\
Object Streams & ObjectInputStream, ObjectOutputStream \\
Print Streams & PrintStream, PrintWriter \\
\end{longtable}
}

\begin{itemize}
\tightlist
\item
  \textbf{Byte Streams}: Work with binary data (8-bit bytes)
\item
  \textbf{Character Streams}: Work with characters (16-bit Unicode)
\item
  \textbf{Buffered Streams}: Improve performance through buffering
\end{itemize}

\end{solutionbox}
\begin{mnemonicbox}
``BCDOP: Byte Character Data Object Print streams''

\end{mnemonicbox}
\subsection*{Question 5(b) [4 marks]}\label{q5b}

\textbf{Write a java program to develop user defined exception for
``Divide by zero'' error.}

\begin{solutionbox}
User-defined exceptions allow creating custom exception
types for application-specific error conditions.

\textbf{Code Block:}

\begin{lstlisting}[language=Java]
// Custom exception for divide by zero
class DivideByZeroException extends Exception {
    // Constructor without parameters
    public DivideByZeroException() {
        super("Cannot divide by zero");
    }
    
    // Constructor with custom message
    public DivideByZeroException(String message) {
        super(message);
    }
}

// Main class demonstrating custom exception
public class CustomExceptionDemo {
    // Method that might throw our custom exception
    public static double divide(int numerator, int denominator) throws DivideByZeroException {
        if (denominator == 0) {
            throw new DivideByZeroException("Division by zero not allowed");
        }
        return (double) numerator / denominator;
    }
    
    public static void main(String[] args) {
        try {
            // Test with valid input
            System.out.println("10 / 2 = " + divide(10, 2));
            
            // Test with zero as denominator
            System.out.println("10 / 0 = " + divide(10, 0));
        } catch (DivideByZeroException e) {
            System.out.println("Error: " + e.getMessage());
            System.out.println("Custom exception stack trace:");
            e.printStackTrace();
        }
        
        System.out.println("Program continues execution...");
    }
}
\end{lstlisting}

\textbf{Output:}

\begin{lstlisting}
10 / 2 = 5.0
Error: Division by zero not allowed
Custom exception stack trace:
DivideByZeroException: Division by zero not allowed
    at CustomExceptionDemo.divide(CustomExceptionDemo.java:19)
    at CustomExceptionDemo.main(CustomExceptionDemo.java:29)
Program continues execution...
\end{lstlisting}

\end{solutionbox}
\begin{mnemonicbox}
``ETC: Extend Throw Catch custom exceptions''

\end{mnemonicbox}
\subsection*{Question 5(c) [7 marks]}\label{q5c}

\textbf{Write a program in Java that reads the content of a file byte by
byte and copy it into another file.}

\begin{solutionbox}
File I/O operations in Java allow reading from and
writing to files, with byte streams handling binary data.

\textbf{Code Block:}

\begin{lstlisting}[language=Java]
import java.io.FileInputStream;
import java.io.FileOutputStream;
import java.io.IOException;

public class FileCopyByteByByte {
    public static void main(String[] args) {
        // Define source and destination file paths
        String sourceFile = "source.txt";
        String destinationFile = "destination.txt";
        
        // Variables for file streams
        FileInputStream inputStream = null;
        FileOutputStream outputStream = null;
        
        try {
            // Initialize input and output streams
            inputStream = new FileInputStream(sourceFile);
            outputStream = new FileOutputStream(destinationFile);
            
            System.out.println("Copying file " + sourceFile + " to " + destinationFile);
            
            // Variables to track copy process
            int byteData;
            int byteCount = 0;
            
            // Read file byte by byte until end of file (-1)
            while ((byteData = inputStream.read()) != -1) {
                // Write the byte to destination file
                outputStream.write(byteData);
                byteCount++;
            }
            
            System.out.println("File copied successfully!");
            System.out.println("Total bytes copied: " + byteCount);
            
        } catch (IOException e) {
            System.out.println("Error during file copy: " + e.getMessage());
            e.printStackTrace();
        } finally {
            // Close resources in finally block
            try {
                if (inputStream != null) {
                    inputStream.close();
                }
                if (outputStream != null) {
                    outputStream.close();
                }
                System.out.println("File streams closed successfully");
            } catch (IOException e) {
                System.out.println("Error closing streams: " + e.getMessage());
            }
        }
    }
}
\end{lstlisting}

\textbf{Creating source.txt file first:}

\begin{lstlisting}[language=Java]
import java.io.FileWriter;
import java.io.IOException;

public class CreateSourceFile {
    public static void main(String[] args) {
        try {
            FileWriter writer = new FileWriter("source.txt");
            writer.write("This is a sample file.\n");
            writer.write("It will be copied byte by byte.\n");
            writer.write("Java I/O operations demo.");
            writer.close();
            System.out.println("Source file created successfully!");
        } catch (IOException e) {
            System.out.println("Error creating source file: " + e.getMessage());
        }
    }
}
\end{lstlisting}

\textbf{Output:}

\begin{lstlisting}
Source file created successfully!
Copying file source.txt to destination.txt
File copied successfully!
Total bytes copied: 82
File streams closed successfully
\end{lstlisting}

\end{solutionbox}
\begin{mnemonicbox}
``CROW: Create Read Open Write file operations''

\end{mnemonicbox}
\subsection*{Question 5(a OR) [3
marks]}\label{question-5a-or-3-marks}

\textbf{List different file operations in Java.}

\begin{solutionbox}
Java provides comprehensive file handling capabilities
through various file operations.


{\def\LTcaptype{none} % do not increment counter
\vspace{-5pt}
\captionof{table}{File Operations in Java}
\vspace{-10pt}
\begin{longtable}[]{@{}
  >{\raggedright\arraybackslash}p{(\linewidth - 4\tabcolsep) * \real{0.2895}}
  >{\raggedright\arraybackslash}p{(\linewidth - 4\tabcolsep) * \real{0.3421}}
  >{\raggedright\arraybackslash}p{(\linewidth - 4\tabcolsep) * \real{0.3684}}@{}}
\toprule\noalign{}
\begin{minipage}[b]{\linewidth}\raggedright
Operation
\end{minipage} & \begin{minipage}[b]{\linewidth}\raggedright
Description
\end{minipage} & \begin{minipage}[b]{\linewidth}\raggedright
Classes Used
\end{minipage} \\
\midrule\noalign{}
\endhead
\bottomrule\noalign{}
\endlastfoot
File Creation & Create new files & File, FileOutputStream, FileWriter \\
File Reading & Read from files & FileInputStream, FileReader, Scanner \\
File Writing & Write to files & FileOutputStream, FileWriter,
PrintWriter \\
File Deletion & Delete files & File.delete() \\
File Information & Get file metadata & File methods (length, isFile,
etc.) \\
Directory Operations & Create/list directories & File methods (mkdir,
list, etc.) \\
File Copy & Copy file contents & FileInputStream with
FileOutputStream \\
File Renaming & Rename or move files & File.renameTo() \\
\end{longtable}
}

\begin{itemize}
\tightlist
\item
  \textbf{Stream-based}: Low-level byte or character streams
\item
  \textbf{Reader/Writer}: Character-oriented file operations
\item
  \textbf{NIO Package}: Enhanced file operations (since Java 7)
\end{itemize}

\end{solutionbox}
\begin{mnemonicbox}
``CRWD: Create Read Write Delete basic operations''

\end{mnemonicbox}
\subsection*{Question 5(b OR) [4
marks]}\label{question-5b-or-4-marks}

\textbf{Write a java program to explain finally block in exception
handling.}

\begin{solutionbox}
The finally block in exception handling ensures code
execution regardless of whether an exception occurs.

\textbf{Diagram: try-catch-finally Flow}

\includegraphics[width=1\linewidth,height=\textheight,keepaspectratio]{mermaid-a9515e2b.pdf}

\textbf{Code Block:}

\begin{lstlisting}[language=Java]
import java.io.FileInputStream;
import java.io.FileNotFoundException;
import java.io.IOException;

public class FinallyBlockDemo {
    public static void main(String[] args) {
        // Example 1: finally with no exception
        System.out.println("Example 1: No exception");
        try {
            int result = 10 / 5;
            System.out.println("Result: " + result);
        } catch (ArithmeticException e) {
            System.out.println("Arithmetic exception caught: " + e.getMessage());
        } finally {
            System.out.println("Finally block executed - Example 1");
        }
        
        // Example 2: finally with exception caught
        System.out.println("\nExample 2: Exception caught");
        try {
            int result = 10 / 0; // This will throw exception
            System.out.println("This won't be printed");
        } catch (ArithmeticException e) {
            System.out.println("Arithmetic exception caught: " + e.getMessage());
        } finally {
            System.out.println("Finally block executed - Example 2");
        }
        
        // Example 3: finally with resource management
        System.out.println("\nExample 3: Resource management");
        FileInputStream file = null;
        try {
            file = new FileInputStream("nonexistent.txt"); // This will throw exception
            System.out.println("File opened successfully");
        } catch (FileNotFoundException e) {
            System.out.println("File not found: " + e.getMessage());
        } finally {
            // Close resources even if exception occurs
            try {
                if (file != null) {
                    file.close();
                }
                System.out.println("File resource closed in finally block");
            } catch (IOException e) {
                System.out.println("Error closing file: " + e.getMessage());
            }
        }
        
        System.out.println("\nProgram continues execution...");
    }
}
\end{lstlisting}

\textbf{Output:}

\begin{lstlisting}
Example 1: No exception
Result: 2
Finally block executed - Example 1

Example 2: Exception caught
Arithmetic exception caught: / by zero
Finally block executed - Example 2

Example 3: Resource management
File not found: nonexistent.txt (No such file or directory)
File resource closed in finally block

Program continues execution...
\end{lstlisting}

\end{solutionbox}
\begin{mnemonicbox}
``ACRE: Always Cleanup Resources Executes''

\end{mnemonicbox}
\subsection*{Question 5(c OR) [7
marks]}\label{question-5c-or-7-marks}

\textbf{Write a java program to create a file and perform write
operation on this file.}

\begin{solutionbox}
Java provides several ways to create files and write
data to them using character or byte streams.

\textbf{Code Block:}

\begin{lstlisting}[language=Java]
import java.io.File;
import java.io.FileWriter;
import java.io.IOException;
import java.io.BufferedWriter;
import java.text.SimpleDateFormat;
import java.util.Date;
import java.util.Scanner;

public class FileWriteDemo {
    public static void main(String[] args) {
        Scanner scanner = null;
        FileWriter fileWriter = null;
        BufferedWriter bufferedWriter = null;
        
        try {
            // Create a File object
            File myFile = new File("sample_data.txt");
            
            // Check if file already exists
            if (myFile.exists()) {
                System.out.println("File already exists: " + myFile.getName());
                System.out.println("File path: " + myFile.getAbsolutePath());
                System.out.println("File size: " + myFile.length() + " bytes");
            } else {
                // Create a new file
                if (myFile.createNewFile()) {
                    System.out.println("File created successfully: " + myFile.getName());
                } else {
                    System.out.println("Failed to create file");
                    return;
                }
            }
            
            // Initialize FileWriter (true parameter appends to file)
            fileWriter = new FileWriter(myFile);
            
            // Use BufferedWriter for efficient writing
            bufferedWriter = new BufferedWriter(fileWriter);
            
            // Get current date and time
            SimpleDateFormat formatter = new SimpleDateFormat("dd/MM/yyyy HH:mm:ss");
            Date date = new Date();
            
            // Write to file
            bufferedWriter.write("==== File Write Demonstration ====");
            bufferedWriter.newLine();
            bufferedWriter.write("Created on: " + formatter.format(date));
            bufferedWriter.newLine();
            
            // Get user input to write to file
            scanner = new Scanner(System.in);
            System.out.println("\nEnter text to write to file (type 'exit' to finish):");
            
            String line;
            while (true) {
                line = scanner.nextLine();
                if (line.equalsIgnoreCase("exit")) {
                    break;
                }
                bufferedWriter.write(line);
                bufferedWriter.newLine();
            }
            
            System.out.println("\nFile write operation completed successfully!");
            
        } catch (IOException e) {
            System.out.println("Error occurred: " + e.getMessage());
            e.printStackTrace();
        } finally {
            // Close resources
            try {
                if (bufferedWriter != null) {
                    bufferedWriter.close();
                }
                if (fileWriter != null) {
                    fileWriter.close();
                }
                if (scanner != null) {
                    scanner.close();
                }
            } catch (IOException e) {
                System.out.println("Error closing resources: " + e.getMessage());
            }
        }
    }
}
\end{lstlisting}

\textbf{Example output:}

\begin{lstlisting}
File created successfully: sample_data.txt

Enter text to write to file (type 'exit' to finish):
This is line 1 of my file.
This is line 2 with some Java content.
Here is line 3 with more text.
exit

File write operation completed successfully!
\end{lstlisting}

\end{solutionbox}
\begin{mnemonicbox}
``COWS: Create Open Write Save file operations''

\end{mnemonicbox}

\end{document}
