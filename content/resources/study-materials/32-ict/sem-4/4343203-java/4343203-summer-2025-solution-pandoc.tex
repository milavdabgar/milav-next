\documentclass[10pt,a4paper]{article}

% content/resources/templates/preamble.tex
\usepackage[margin=0.6in]{geometry}
\author{Milav Dabgar}
\usepackage{amsmath,amssymb,amsthm}
\usepackage{booktabs}
\usepackage{multirow}
\usepackage{xcolor}
\usepackage{tcolorbox}
\tcbuselibrary{breakable,skins}
\usepackage[colorlinks=true,linkcolor=blue]{hyperref}
\usepackage{titlesec}
\usepackage{enumitem}
\usepackage{tikz}
\usepackage{pgfplots}
\usepackage{circuitikz}
\usepackage[version=4]{mhchem}
\usepackage{longtable}
\usepackage{array}
\usepackage{float}
\usepackage{caption}
\usepackage{listings}

\lstset{
  basicstyle=\small\ttfamily,
  breaklines=true,
  breakatwhitespace=false,
  postbreak=\mbox{\textcolor{red}{$\hookrightarrow$}\space},
  float=false,
  numbers=left,
  numberstyle=\tiny\color{gray},
  numbersep=10pt,
  xleftmargin=2em,
  keywordstyle=\color{blue},
  commentstyle=\color{green!60!black},
  stringstyle=\color{purple},
  backgroundcolor=\color{gray!5},
  showstringspaces=false,
  tabsize=2,
  captionpos=b,
  keepspaces=true,
  columns=flexible
}

\pgfplotsset{compat=1.18}
\usetikzlibrary{shapes,arrows,positioning,calc,patterns,decorations.pathmorphing,decorations.markings,arrows.meta}

% Color scheme
\definecolor{headcolor}{RGB}{0,102,204}
\definecolor{keycolor}{RGB}{220,20,60}
\definecolor{solutioncolor}{RGB}{34,139,34}
\definecolor{mnemoniccolor}{RGB}{148,0,211}
\definecolor{codecolor}{RGB}{0,0,100}

% Spacing
\setlength{\parskip}{3pt}
\setlist[itemize]{nosep}
\setlist[enumerate]{nosep}

% Title formatting
\titleformat{\section}{\Large\bfseries\color{headcolor}}{\thesection}{1em}{}
\titleformat{\subsection}{\large\bfseries\color{headcolor}}{\thesubsection}{1em}{}

% Pandoc tightlist compatibility
\providecommand{\tightlist}{%
  \setlength{\itemsep}{0pt}\setlength{\parskip}{0pt}}

% Pandoc longtable compatibility
\newcounter{none}
\def\thenone{}


% content/resources/templates/english-boxes.tex
% This file is currently empty - it exists to maintain consistency with the import structure.
% Add custom environments here if needed in the future.


\begin{document}

\begin{center}
{\Huge\bfseries\color{headcolor} Java Programming (4343203) - Summer 2025 Solution}\\[5pt]
{\LARGE 4343203 -- Summer 2025}\\[3pt]
{\large Semester 1 Study Material}\\[3pt]
{\normalsize\textit{Detailed Solutions and Explanations}}
\end{center}

\vspace{10pt}

\subsection*{Question 1(a) [3 marks]}\label{q1a}

\textbf{List out the rules to name an identifier in Java with valid and
invalid examples.}

\begin{solutionbox}

\textbf{Rules for Java Identifiers:}

{\def\LTcaptype{none} % do not increment counter
\begin{longtable}[]{@{}
  >{\raggedright\arraybackslash}p{(\linewidth - 6\tabcolsep) * \real{0.1176}}
  >{\raggedright\arraybackslash}p{(\linewidth - 6\tabcolsep) * \real{0.2549}}
  >{\raggedright\arraybackslash}p{(\linewidth - 6\tabcolsep) * \real{0.2941}}
  >{\raggedright\arraybackslash}p{(\linewidth - 6\tabcolsep) * \real{0.3333}}@{}}
\toprule\noalign{}
\begin{minipage}[b]{\linewidth}\raggedright
Rule
\end{minipage} & \begin{minipage}[b]{\linewidth}\raggedright
Description
\end{minipage} & \begin{minipage}[b]{\linewidth}\raggedright
Valid Example
\end{minipage} & \begin{minipage}[b]{\linewidth}\raggedright
Invalid Example
\end{minipage} \\
\midrule\noalign{}
\endhead
\bottomrule\noalign{}
\endlastfoot
\textbf{Start Character} & Must begin with letter, underscore, or dollar
sign & \passthrough{\lstinline!name!},
\passthrough{\lstinline!_value!}, \passthrough{\lstinline!$cost!} &
\passthrough{\lstinline!2name!}, \passthrough{\lstinline!#id!} \\
\textbf{Following Characters} & Can contain letters, digits, underscore,
dollar & \passthrough{\lstinline!student123!},
\passthrough{\lstinline!user_name!} & \passthrough{\lstinline!my-var!},
\passthrough{\lstinline!class@!} \\
\textbf{Keywords Restriction} & Cannot use Java reserved words &
\passthrough{\lstinline!myClass!}, \passthrough{\lstinline!userName!} &
\passthrough{\lstinline!class!}, \passthrough{\lstinline!int!} \\
\textbf{Case Sensitivity} & Identifiers are case-sensitive &
\passthrough{\lstinline!Name!} \neq \passthrough{\lstinline!name!} & - \\
\textbf{Length} & No length limit (practically reasonable) &
\passthrough{\lstinline!verylongvariablename!} & - \\
\end{longtable}
}

\end{solutionbox}
\begin{mnemonicbox}
``Letters First, Keywords Never, Case Counts''

\end{mnemonicbox}
\begin{center}\rule{0.5\linewidth}{0.5pt}\end{center}

\subsection*{Question 1(b) [4 marks]}\label{q1b}

\textbf{List out different types of operators in Java. Explain
Arithmetic and Logical Operators in detail.}

\begin{solutionbox}

\textbf{Java Operator Types:}

{\def\LTcaptype{none} % do not increment counter
\begin{longtable}[]{@{}ll@{}}
\toprule\noalign{}
Operator Type & Examples \\
\midrule\noalign{}
\endhead
\bottomrule\noalign{}
\endlastfoot
\textbf{Arithmetic} & \passthrough{\lstinline!+!},
\passthrough{\lstinline!-!}, \passthrough{\lstinline!*!},
\passthrough{\lstinline!/!}, \passthrough{\lstinline!%!} \\
\textbf{Relational} & \passthrough{\lstinline!==!},
\passthrough{\lstinline"!="}, \passthrough{\lstinline!<!},
\passthrough{\lstinline!>!}, \passthrough{\lstinline!<=!},
\passthrough{\lstinline!>=!} \\
\textbf{Logical} & \passthrough{\lstinline!&&!},
\passthrough{\lstinline!\|\|!}, \passthrough{\lstinline"!"} \\
\textbf{Assignment} & \passthrough{\lstinline!=!},
\passthrough{\lstinline!+=!}, \passthrough{\lstinline!-=!},
\passthrough{\lstinline!*=!}, \passthrough{\lstinline!/=!} \\
\textbf{Unary} & \passthrough{\lstinline!++!},
\passthrough{\lstinline!--!}, \passthrough{\lstinline!+!},
\passthrough{\lstinline!-!}, \passthrough{\lstinline"!"} \\
\textbf{Bitwise} & \passthrough{\lstinline!&!},
\passthrough{\lstinline!\|!}, \passthrough{\lstinline!^!},
\passthrough{\lstinline!~!}, \passthrough{\lstinline!<<!},
\passthrough{\lstinline!>>!} \\
\textbf{Ternary} &
\passthrough{\lstinline!condition ? value1 : value2!} \\
\end{longtable}
}

\textbf{Arithmetic Operators:}

\begin{itemize}
\tightlist
\item
  \textbf{Addition (+)}: Adds two operands
\item
  \textbf{Subtraction (-)}: Subtracts second from first
\item
  **Multiplication (*)**: Multiplies two operands
\item
  \textbf{Division (/)}: Divides first by second
\item
  \textbf{Modulus (\%)}: Returns remainder of division
\end{itemize}

\textbf{Logical Operators:}

\begin{itemize}
\tightlist
\item
  \textbf{AND (\&\&)}: Returns true if both conditions are true
\item
  \textbf{OR (\textbar\textbar)}: Returns true if at least one condition
  is true
\item
  \textbf{NOT (!)}: Reverses the logical state
\end{itemize}

\end{solutionbox}
\begin{mnemonicbox}
``Add Subtract Multiply Divide Remainder, And Or
Not''

\end{mnemonicbox}
\begin{center}\rule{0.5\linewidth}{0.5pt}\end{center}

\subsection*{Question 1(c) [7 marks]}\label{q1c}

\textbf{Write a program in Java to reverse the digits of a number for
number having three digits. Like reverse of 653 is 356.}

\begin{solutionbox}

\begin{lstlisting}[language=Java]
import java.util.Scanner;

public class ReverseNumber {
    public static void main(String[] args) {
        Scanner sc = new Scanner(System.in);
        
        System.out.print("Enter 3-digit number: ");
        int num = sc.nextInt();
        
        int reverse = 0;
        int temp = num;
        
        while (temp > 0) {
            reverse = reverse * 10 + temp % 10;
            temp = temp / 10;
        }
        
        System.out.println("Original: " + num);
        System.out.println("Reversed: " + reverse);
    }
}
\end{lstlisting}

\textbf{Algorithm:}

\begin{itemize}
\tightlist
\item
  \textbf{Extract last digit}: Use modulus operator (\%)
\item
  \textbf{Build reversed number}: Multiply by 10 and add digit
\item
  \textbf{Remove last digit}: Use integer division (/)
\item
  \textbf{Repeat}: Until original number becomes 0
\end{itemize}

\end{solutionbox}
\begin{mnemonicbox}
``Extract, Build, Remove, Repeat''

\end{mnemonicbox}
\begin{center}\rule{0.5\linewidth}{0.5pt}\end{center}

\subsection*{Question 1(c OR) [7
marks]}\label{question-1c-or-7-marks}

**Write a program in Java to add two 3*3 matrices.**

\begin{solutionbox}

\begin{lstlisting}[language=Java]
import java.util.Scanner;

public class MatrixAddition {
    public static void main(String[] args) {
        Scanner sc = new Scanner(System.in);
        int[][] matrix1 = new int[3][3];
        int[][] matrix2 = new int[3][3];
        int[][] result = new int[3][3];
        
        // Input first matrix
        System.out.println("Enter first matrix:");
        for (int i = 0; i < 3; i++) {
            for (int j = 0; j < 3; j++) {
                matrix1[i][j] = sc.nextInt();
            }
        }
        
        // Input second matrix
        System.out.println("Enter second matrix:");
        for (int i = 0; i < 3; i++) {
            for (int j = 0; j < 3; j++) {
                matrix2[i][j] = sc.nextInt();
            }
        }
        
        // Add matrices
        for (int i = 0; i < 3; i++) {
            for (int j = 0; j < 3; j++) {
                result[i][j] = matrix1[i][j] + matrix2[i][j];
            }
        }
        
        // Display result
        System.out.println("Sum of matrices:");
        for (int i = 0; i < 3; i++) {
            for (int j = 0; j < 3; j++) {
                System.out.print(result[i][j] + " ");
            }
            System.out.println();
        }
    }
}
\end{lstlisting}

\textbf{Matrix Addition Steps:}

\begin{itemize}
\tightlist
\item
  \textbf{Create arrays}: Three 3x3 integer arrays
\item
  \textbf{Input matrices}: Read values for both matrices
\item
  \textbf{Add corresponding elements}: result[i][j] =
  matrix1[i][j] + matrix2[i][j]
\item
  \textbf{Display result}: Print the sum matrix
\end{itemize}

\end{solutionbox}
\begin{mnemonicbox}
``Create, Input, Add, Display''

\end{mnemonicbox}
\begin{center}\rule{0.5\linewidth}{0.5pt}\end{center}

\subsection*{Question 2(a) [3 marks]}\label{q2a}

\textbf{Write a program in Java that shows the use of parameterized
Constructor.}

\begin{solutionbox}

\begin{lstlisting}[language=Java]
class Student {
    private String name;
    private int rollNo;
    
    // Parameterized Constructor
    public Student(String name, int rollNo) {
        this.name = name;
        this.rollNo = rollNo;
    }
    
    public void display() {
        System.out.println("Name: " + name);
        System.out.println("Roll No: " + rollNo);
    }
}

public class ParameterizedConstructor {
    public static void main(String[] args) {
        Student s1 = new Student("John", 101);
        s1.display();
    }
}
\end{lstlisting}

\textbf{Parameterized Constructor Features:}

\begin{itemize}
\tightlist
\item
  \textbf{Takes parameters}: Accepts values during object creation
\item
  \textbf{Initializes instance variables}: Sets object state
\item
  \textbf{Same name as class}: Constructor name matches class name
\item
  \textbf{No return type}: Constructors don't have return type
\end{itemize}

\end{solutionbox}
\begin{mnemonicbox}
``Parameters Initialize Same-name No-return''

\end{mnemonicbox}
\begin{center}\rule{0.5\linewidth}{0.5pt}\end{center}

\subsection*{Question 2(b) [4 marks]}\label{q2b}

\textbf{Give the basic syntax of the following terms with an example:
(1) To create a Class, (2) To create an Object, (3) To define a Method,
(4) To declare a Variable.}

\begin{solutionbox}

\textbf{Java Basic Syntax:}

{\def\LTcaptype{none} % do not increment counter
\begin{longtable}[]{@{}
  >{\raggedright\arraybackslash}p{(\linewidth - 4\tabcolsep) * \real{0.3929}}
  >{\raggedright\arraybackslash}p{(\linewidth - 4\tabcolsep) * \real{0.2857}}
  >{\raggedright\arraybackslash}p{(\linewidth - 4\tabcolsep) * \real{0.3214}}@{}}
\toprule\noalign{}
\begin{minipage}[b]{\linewidth}\raggedright
Component
\end{minipage} & \begin{minipage}[b]{\linewidth}\raggedright
Syntax
\end{minipage} & \begin{minipage}[b]{\linewidth}\raggedright
Example
\end{minipage} \\
\midrule\noalign{}
\endhead
\bottomrule\noalign{}
\endlastfoot
\textbf{Class Creation} &
\passthrough{\lstinline!class ClassName \{ \}!} &
\passthrough{\lstinline!class Car \{ \}!} \\
\textbf{Object Creation} &
\passthrough{\lstinline!ClassName objectName = new ClassName();!} &
\passthrough{\lstinline!Car myCar = new Car();!} \\
\textbf{Method Definition} &
\passthrough{\lstinline!returnType methodName(parameters) \{ \}!} &
\passthrough{\lstinline!public void start() \{ \}!} \\
\textbf{Variable Declaration} &
\passthrough{\lstinline!dataType variableName;!} &
\passthrough{\lstinline!int age;!} \\
\end{longtable}
}

\textbf{Complete Example:}

\begin{lstlisting}[language=Java]
class Car {                           // Class Creation
    int speed;                        // Variable Declaration
    
    public void accelerate() {        // Method Definition
        speed += 10;
    }
}

public class Main {
    public static void main(String[] args) {
        Car myCar = new Car();        // Object Creation
    }
}
\end{lstlisting}

\end{solutionbox}
\begin{mnemonicbox}
``Class Object Method Variable - COMV''

\end{mnemonicbox}
\begin{center}\rule{0.5\linewidth}{0.5pt}\end{center}

\subsection*{Question 2(c) [7 marks]}\label{q2c}

\textbf{Write a program in Java which has a class Student having two
instance variables enrollmentNo and name. Create 3 objects of Student
class in main method and display student's name.}

\begin{solutionbox}

\begin{lstlisting}[language=Java]
class Student {
    String enrollmentNo;
    String name;
    
    // Constructor to initialize student data
    public Student(String enrollmentNo, String name) {
        this.enrollmentNo = enrollmentNo;
        this.name = name;
    }
    
    // Method to display student name
    public void displayName() {
        System.out.println("Student Name: " + name);
    }
}

public class StudentDemo {
    public static void main(String[] args) {
        // Creating 3 objects of Student class
        Student s1 = new Student("CS001", "Alice");
        Student s2 = new Student("CS002", "Bob");
        Student s3 = new Student("CS003", "Charlie");
        
        // Displaying student names
        s1.displayName();
        s2.displayName();
        s3.displayName();
    }
}
\end{lstlisting}

\textbf{Program Structure:}

\begin{itemize}
\tightlist
\item
  \textbf{Class definition}: Student class with instance variables
\item
  \textbf{Constructor}: Initialize enrollmentNo and name
\item
  \textbf{Method}: displayName() to show student name
\item
  \textbf{Object creation}: Three Student objects in main method
\item
  \textbf{Method calling}: Display names using displayName()
\end{itemize}

\end{solutionbox}
\begin{mnemonicbox}
``Define Initialize Display Create Call''

\end{mnemonicbox}
\begin{center}\rule{0.5\linewidth}{0.5pt}\end{center}

\subsection*{Question 2(a OR) [3
marks]}\label{question-2a-or-3-marks}

\textbf{Write a program in Java that shows the use of Default
Constructor.}

\begin{solutionbox}

\begin{lstlisting}[language=Java]
class Rectangle {
    int length;
    int width;
    
    // Default Constructor
    public Rectangle() {
        length = 5;
        width = 3;
        System.out.println("Default constructor called");
    }
    
    public void displayArea() {
        System.out.println("Area: " + (length * width));
    }
}

public class DefaultConstructor {
    public static void main(String[] args) {
        Rectangle r1 = new Rectangle();
        r1.displayArea();
    }
}
\end{lstlisting}

\textbf{Default Constructor Features:}

\begin{itemize}
\tightlist
\item
  \textbf{No parameters}: Takes no arguments
\item
  \textbf{Default values}: Sets default values for instance variables
\item
  \textbf{Automatic call}: Called when object is created
\item
  \textbf{Same name as class}: Constructor name matches class name
\end{itemize}

\end{solutionbox}
\begin{mnemonicbox}
``No-parameters Default Automatic Same-name''

\end{mnemonicbox}
\begin{center}\rule{0.5\linewidth}{0.5pt}\end{center}

\subsection*{Question 2(b OR) [4
marks]}\label{question-2b-or-4-marks}

\textbf{Give four Difference between Procedure Oriented Programming and
Object-Oriented Programming.}

\begin{solutionbox}

\textbf{POP vs OOP Comparison:}

{\def\LTcaptype{none} % do not increment counter
\begin{longtable}[]{@{}
  >{\raggedright\arraybackslash}p{(\linewidth - 4\tabcolsep) * \real{0.1194}}
  >{\raggedright\arraybackslash}p{(\linewidth - 4\tabcolsep) * \real{0.4627}}
  >{\raggedright\arraybackslash}p{(\linewidth - 4\tabcolsep) * \real{0.4179}}@{}}
\toprule\noalign{}
\begin{minipage}[b]{\linewidth}\raggedright
Aspect
\end{minipage} & \begin{minipage}[b]{\linewidth}\raggedright
Procedure Oriented Programming
\end{minipage} & \begin{minipage}[b]{\linewidth}\raggedright
Object-Oriented Programming
\end{minipage} \\
\midrule\noalign{}
\endhead
\bottomrule\noalign{}
\endlastfoot
\textbf{Approach} & Top-down approach & Bottom-up approach \\
\textbf{Focus} & Functions and procedures & Objects and classes \\
\textbf{Data Security} & No data hiding, global access & Data
encapsulation and hiding \\
\textbf{Problem Solving} & Divide into functions & Divide into
objects \\
\textbf{Code Reusability} & Limited reusability & High reusability
through inheritance \\
\textbf{Maintenance} & Difficult to maintain & Easy to maintain and
modify \\
\end{longtable}
}

\textbf{Key Differences:}

\begin{itemize}
\tightlist
\item
  \textbf{Structure}: POP uses functions, OOP uses classes
\item
  \textbf{Security}: OOP provides better data protection
\item
  \textbf{Reusability}: OOP supports inheritance and polymorphism
\item
  \textbf{Maintenance}: OOP code is easier to maintain
\end{itemize}

\end{solutionbox}
\begin{mnemonicbox}
``Structure Security Reusability Maintenance''

\end{mnemonicbox}
\begin{center}\rule{0.5\linewidth}{0.5pt}\end{center}

\subsection*{Question 2(c OR) [7
marks]}\label{question-2c-or-7-marks}

\textbf{Write a program in Java which has a class Shape having 2
overloaded methods area (float radius) and area (float length, float
width). Display the area of circle and rectangle using overloaded
methods.}

\begin{solutionbox}

\begin{lstlisting}[language=Java]
class Shape {
    // Method to calculate area of circle
    public void area(float radius) {
        float circleArea = 3.14f * radius * radius;
        System.out.println("Area of Circle: " + circleArea);
    }
    
    // Overloaded method to calculate area of rectangle
    public void area(float length, float width) {
        float rectangleArea = length * width;
        System.out.println("Area of Rectangle: " + rectangleArea);
    }
}

public class MethodOverloading {
    public static void main(String[] args) {
        Shape shape = new Shape();
        
        // Calculate area of circle with radius 5
        shape.area(5.0f);
        
        // Calculate area of rectangle with length 4 and width 6
        shape.area(4.0f, 6.0f);
    }
}
\end{lstlisting}

\textbf{Method Overloading Concepts:}

\begin{itemize}
\tightlist
\item
  \textbf{Same method name}: Both methods named ``area''
\item
  \textbf{Different parameters}: One takes radius, other takes length
  and width
\item
  \textbf{Compile-time polymorphism}: Method selected at compile time
\item
  \textbf{Parameter differentiation}: Different number or type of
  parameters
\end{itemize}

\end{solutionbox}
\begin{mnemonicbox}
``Same-name Different-parameters Compile-time
Parameter-differentiation''

\end{mnemonicbox}
\begin{center}\rule{0.5\linewidth}{0.5pt}\end{center}

\subsection*{Question 3(a) [3 marks]}\label{q3a}

\textbf{Write a program in Java to demonstrate single inheritance.}

\begin{solutionbox}

\begin{lstlisting}[language=Java]
// Parent class
class Animal {
    public void eat() {
        System.out.println("Animal is eating");
    }
    
    public void sleep() {
        System.out.println("Animal is sleeping");
    }
}

// Child class inheriting from Animal
class Dog extends Animal {
    public void bark() {
        System.out.println("Dog is barking");
    }
}

public class SingleInheritance {
    public static void main(String[] args) {
        Dog dog = new Dog();
        
        // Inherited methods from Animal class
        dog.eat();
        dog.sleep();
        
        // Own method of Dog class
        dog.bark();
    }
}
\end{lstlisting}

\textbf{Single Inheritance Features:}

\begin{itemize}
\tightlist
\item
  \textbf{One parent}: Child class inherits from one parent class
\item
  \textbf{extends keyword}: Used to establish inheritance relationship
\item
  \textbf{Method inheritance}: Child class inherits parent methods
\item
  \textbf{IS-A relationship}: Dog IS-A Animal
\end{itemize}

\end{solutionbox}
\begin{mnemonicbox}
``One-parent Extends Method IS-A''

\end{mnemonicbox}
\begin{center}\rule{0.5\linewidth}{0.5pt}\end{center}

\subsection*{Question 3(b) [4 marks]}\label{q3b}

\textbf{Define abstract class in JAVA with example.}

\begin{solutionbox}

\textbf{Abstract Class Definition:} An abstract class is a class that
cannot be instantiated and may contain abstract methods (methods without
implementation).

\begin{lstlisting}[language=Java]
// Abstract class
abstract class Vehicle {
    String brand;
    
    // Regular method
    public void displayBrand() {
        System.out.println("Brand: " + brand);
    }
    
    // Abstract method (no implementation)
    public abstract void start();
    public abstract void stop();
}

// Concrete class extending abstract class
class Car extends Vehicle {
    public Car(String brand) {
        this.brand = brand;
    }
    
    // Must implement abstract methods
    public void start() {
        System.out.println("Car started with key");
    }
    
    public void stop() {
        System.out.println("Car stopped with brake");
    }
}

public class AbstractDemo {
    public static void main(String[] args) {
        Car car = new Car("Toyota");
        car.displayBrand();
        car.start();
        car.stop();
    }
}
\end{lstlisting}

\textbf{Abstract Class Features:}

\begin{itemize}
\tightlist
\item
  \textbf{Cannot instantiate}: Cannot create objects directly
\item
  \textbf{Abstract methods}: Methods without body
\item
  \textbf{Concrete methods}: Regular methods with implementation
\item
  \textbf{Must extend}: Child classes must implement abstract methods
\end{itemize}

\end{solutionbox}
\begin{mnemonicbox}
``Cannot-instantiate Abstract-methods
Concrete-methods Must-extend''

\end{mnemonicbox}
\begin{center}\rule{0.5\linewidth}{0.5pt}\end{center}

\subsection*{Question 3(c) [7 marks]}\label{q3c}

\textbf{Write a program in Java to implement multiple inheritance using
interfaces.}

\begin{solutionbox}

\begin{lstlisting}[language=Java]
// First interface
interface Flyable {
    void fly();
}

// Second interface  
interface Swimmable {
    void swim();
}

// Class implementing multiple interfaces
class Duck implements Flyable, Swimmable {
    private String name;
    
    public Duck(String name) {
        this.name = name;
    }
    
    // Implementing fly method from Flyable interface
    public void fly() {
        System.out.println(name + " is flying in the sky");
    }
    
    // Implementing swim method from Swimmable interface
    public void swim() {
        System.out.println(name + " is swimming in water");
    }
    
    public void walk() {
        System.out.println(name + " is walking on land");
    }
}

public class MultipleInheritance {
    public static void main(String[] args) {
        Duck duck = new Duck("Donald");
        
        // Methods from interfaces
        duck.fly();
        duck.swim();
        
        // Own method
        duck.walk();
    }
}
\end{lstlisting}

\textbf{Multiple Inheritance via Interfaces:}

\begin{itemize}
\tightlist
\item
  \textbf{Multiple interfaces}: Class can implement multiple interfaces
\item
  \textbf{implements keyword}: Used to implement interfaces
\item
  \textbf{Must implement all methods}: All interface methods must be
  implemented
\item
  \textbf{Solves diamond problem}: Avoids ambiguity of multiple
  inheritance
\end{itemize}

\end{solutionbox}
\begin{mnemonicbox}
``Multiple-interfaces Implements Must-implement
Solves-diamond''

\end{mnemonicbox}
\begin{center}\rule{0.5\linewidth}{0.5pt}\end{center}

\subsection*{Question 3(a OR) [3
marks]}\label{question-3a-or-3-marks}

\textbf{Write a program in Java to demonstrate multilevel inheritance.}

\begin{solutionbox}

\begin{lstlisting}[language=Java]
// Grandparent class
class Animal {
    public void breathe() {
        System.out.println("Animal is breathing");
    }
}

// Parent class inheriting from Animal
class Mammal extends Animal {
    public void giveBirth() {
        System.out.println("Mammal gives birth to babies");
    }
}

// Child class inheriting from Mammal
class Dog extends Mammal {
    public void bark() {
        System.out.println("Dog is barking");
    }
}

public class MultilevelInheritance {
    public static void main(String[] args) {
        Dog dog = new Dog();
        
        // Method from Animal class (grandparent)
        dog.breathe();
        
        // Method from Mammal class (parent)
        dog.giveBirth();
        
        // Own method of Dog class
        dog.bark();
    }
}
\end{lstlisting}

\textbf{Multilevel Inheritance Features:}

\begin{itemize}
\tightlist
\item
  \textbf{Chain of inheritance}: Child \rightarrow Parent \rightarrow Grandparent
\item
  \textbf{Multiple levels}: More than two levels of inheritance
\item
  \textbf{Transitive inheritance}: Properties passed through levels
\item
  \textbf{extends keyword}: Each level uses extends
\end{itemize}

\end{solutionbox}
\begin{mnemonicbox}
``Chain Multiple Transitive Extends''

\end{mnemonicbox}
\begin{center}\rule{0.5\linewidth}{0.5pt}\end{center}

\subsection*{Question 3(b OR) [4
marks]}\label{question-3b-or-4-marks}

\textbf{Define package and write the syntax to create a package with
example.}

\begin{solutionbox}

\textbf{Package Definition:} A package is a namespace that organizes
related classes and interfaces. It provides access protection and
namespace management.

\textbf{Package Syntax:}

\begin{lstlisting}[language=Java]
package packageName;
\end{lstlisting}

\textbf{Example:}

\textbf{File: mypackage/Calculator.java}

\begin{lstlisting}[language=Java]
package mypackage;

public class Calculator {
    public int add(int a, int b) {
        return a + b;
    }
    
    public int subtract(int a, int b) {
        return a - b;
    }
}
\end{lstlisting}

\textbf{File: TestCalculator.java}

\begin{lstlisting}[language=Java]
import mypackage.Calculator;

public class TestCalculator {
    public static void main(String[] args) {
        Calculator calc = new Calculator();
        
        System.out.println("Addition: " + calc.add(10, 5));
        System.out.println("Subtraction: " + calc.subtract(10, 5));
    }
}
\end{lstlisting}

\textbf{Package Benefits:}

\begin{itemize}
\tightlist
\item
  \textbf{Namespace management}: Avoids naming conflicts
\item
  \textbf{Access control}: Controls class visibility
\item
  \textbf{Code organization}: Groups related classes
\item
  \textbf{Reusability}: Easy to reuse packaged classes
\end{itemize}

\end{solutionbox}
\begin{mnemonicbox}
``Namespace Access Organization Reusability''

\end{mnemonicbox}
\begin{center}\rule{0.5\linewidth}{0.5pt}\end{center}

\subsection*{Question 3(c OR) [7
marks]}\label{question-3c-or-7-marks}

\textbf{Write a program in Java to demonstrate method overriding.}

\begin{solutionbox}

\begin{lstlisting}[language=Java]
// Parent class
class Animal {
    public void makeSound() {
        System.out.println("Animal makes a sound");
    }
    
    public void move() {
        System.out.println("Animal moves");
    }
}

// Child class overriding parent methods
class Dog extends Animal {
    // Method overriding
    @Override
    public void makeSound() {
        System.out.println("Dog barks: Woof! Woof!");
    }
    
    @Override
    public void move() {
        System.out.println("Dog runs on four legs");
    }
}

class Cat extends Animal {
    // Method overriding
    @Override
    public void makeSound() {
        System.out.println("Cat meows: Meow! Meow!");
    }
    
    @Override
    public void move() {
        System.out.println("Cat walks silently");
    }
}

public class MethodOverriding {
    public static void main(String[] args) {
        Animal animal;
        
        // Dog object
        animal = new Dog();
        animal.makeSound();  // Calls Dog's makeSound()
        animal.move();       // Calls Dog's move()
        
        System.out.println();
        
        // Cat object
        animal = new Cat();
        animal.makeSound();  // Calls Cat's makeSound()
        animal.move();       // Calls Cat's move()
    }
}
\end{lstlisting}

\textbf{Method Overriding Features:}

\begin{itemize}
\tightlist
\item
  \textbf{Same method signature}: Same name, parameters, and return type
\item
  \textbf{Runtime polymorphism}: Method decided at runtime
\item
  \textbf{@Override annotation}: Optional but recommended
\item
  \textbf{IS-A relationship}: Child class overrides parent method
\end{itemize}

\end{solutionbox}
\begin{mnemonicbox}
``Same-signature Runtime Override IS-A''

\end{mnemonicbox}
\begin{center}\rule{0.5\linewidth}{0.5pt}\end{center}

\subsection*{Question 4(a) [3 marks]}\label{q4a}

\textbf{List and explain different types of errors in Java.}

\begin{solutionbox}

\textbf{Java Error Types:}

{\def\LTcaptype{none} % do not increment counter
\begin{longtable}[]{@{}
  >{\raggedright\arraybackslash}p{(\linewidth - 4\tabcolsep) * \real{0.3529}}
  >{\raggedright\arraybackslash}p{(\linewidth - 4\tabcolsep) * \real{0.3824}}
  >{\raggedright\arraybackslash}p{(\linewidth - 4\tabcolsep) * \real{0.2647}}@{}}
\toprule\noalign{}
\begin{minipage}[b]{\linewidth}\raggedright
Error Type
\end{minipage} & \begin{minipage}[b]{\linewidth}\raggedright
Description
\end{minipage} & \begin{minipage}[b]{\linewidth}\raggedright
Example
\end{minipage} \\
\midrule\noalign{}
\endhead
\bottomrule\noalign{}
\endlastfoot
\textbf{Compile-time Errors} & Detected during compilation & Syntax
errors, missing semicolons \\
\textbf{Runtime Errors} & Occur during program execution & Division by
zero, null pointer \\
\textbf{Logical Errors} & Program runs but gives wrong output &
Incorrect algorithm logic \\
\end{longtable}
}

\textbf{Detailed Explanation:}

\begin{itemize}
\tightlist
\item
  \textbf{Compile-time}: Prevented by compiler, must fix before running
\item
  \textbf{Runtime}: Program crashes during execution, handled by
  exceptions
\item
  \textbf{Logical}: Hardest to find, program works but results are
  incorrect
\end{itemize}

\textbf{Common Examples:}

\begin{itemize}
\tightlist
\item
  \textbf{Syntax Error}: Missing semicolon, wrong brackets
\item
  \textbf{RuntimeException}: ArrayIndexOutOfBounds, NullPointer
\item
  \textbf{Logic Error}: Wrong formula, incorrect condition
\end{itemize}

\end{solutionbox}
\begin{mnemonicbox}
``Compile Runtime Logic - CRL''

\end{mnemonicbox}
\begin{center}\rule{0.5\linewidth}{0.5pt}\end{center}

\subsection*{Question 4(b) [4 marks]}\label{q4b}

\textbf{What is wrapper class? Explain use of any two wrapper class.}

\begin{solutionbox}

\textbf{Wrapper Class Definition:} Wrapper classes provide object
representation of primitive data types. They convert primitives into
objects.

\textbf{Wrapper Class Table:}

{\def\LTcaptype{none} % do not increment counter
\begin{longtable}[]{@{}ll@{}}
\toprule\noalign{}
Primitive & Wrapper Class \\
\midrule\noalign{}
\endhead
\bottomrule\noalign{}
\endlastfoot
int & Integer \\
double & Double \\
boolean & Boolean \\
char & Character \\
\end{longtable}
}

\textbf{Example - Integer Wrapper:}

\begin{lstlisting}[language=Java]
// Primitive to Object (Boxing)
int num = 100;
Integer intObj = Integer.valueOf(num);

// Object to Primitive (Unboxing)  
int value = intObj.intValue();

// Utility methods
String str = "123";
int parsed = Integer.parseInt(str);
\end{lstlisting}

\textbf{Example - Double Wrapper:}

\begin{lstlisting}[language=Java]
// Creating Double object
Double doubleObj = Double.valueOf(45.67);

// Converting string to double
String str = "123.45";
double value = Double.parseDouble(str);

// Checking special values
boolean isNaN = Double.isNaN(doubleObj);
\end{lstlisting}

\textbf{Wrapper Class Uses:}

\begin{itemize}
\tightlist
\item
  \textbf{Collections}: Store primitives in collections
\item
  \textbf{Null values}: Can store null unlike primitives
\item
  \textbf{Utility methods}: Parsing, conversion methods
\item
  \textbf{Generics}: Use with generic types
\end{itemize}

\end{solutionbox}
\begin{mnemonicbox}
``Collections Null Utility Generics''

\end{mnemonicbox}
\begin{center}\rule{0.5\linewidth}{0.5pt}\end{center}

\subsection*{Question 4(c) [7 marks]}\label{q4c}

\textbf{Write a program in Java to develop Banking Application in which
user deposits the amount Rs 25000 and then start withdrawing of Rs
20000, Rs 4000 and it throws exception ``Not Sufficient Fund'' when user
withdraws Rs. 2000 thereafter.}

\begin{solutionbox}

\begin{lstlisting}[language=Java]
// Custom Exception class
class InsufficientFundException extends Exception {
    public InsufficientFundException(String message) {
        super(message);
    }
}

// Bank Account class
class BankAccount {
    private double balance;
    
    public BankAccount(double initialBalance) {
        this.balance = initialBalance;
    }
    
    public void deposit(double amount) {
        balance += amount;
        System.out.println("Deposited: Rs." + amount);
        System.out.println("Current Balance: Rs." + balance);
    }
    
    public void withdraw(double amount) throws InsufficientFundException {
        if (amount > balance) {
            throw new InsufficientFundException("Not Sufficient Fund");
        }
        balance -= amount;
        System.out.println("Withdrawn: Rs." + amount);
        System.out.println("Remaining Balance: Rs." + balance);
    }
    
    public double getBalance() {
        return balance;
    }
}

public class BankingApplication {
    public static void main(String[] args) {
        BankAccount account = new BankAccount(0);
        
        try {
            // Deposit Rs. 25000
            account.deposit(25000);
            System.out.println();
            
            // Withdraw Rs. 20000
            account.withdraw(20000);
            System.out.println();
            
            // Withdraw Rs. 4000
            account.withdraw(4000);
            System.out.println();
            
            // Try to withdraw Rs. 2000 (will throw exception)
            account.withdraw(2000);
            
        } catch (InsufficientFundException e) {
            System.out.println("Exception: " + e.getMessage());
            System.out.println("Available Balance: Rs." + account.getBalance());
        }
    }
}
\end{lstlisting}

\textbf{Exception Handling Components:}

\begin{itemize}
\tightlist
\item
  \textbf{Custom exception}: InsufficientFundException extends Exception
\item
  \textbf{throw keyword}: Throws exception when balance insufficient
\item
  \textbf{try-catch block}: Handles the exception
\item
  \textbf{Exception message}: Displays ``Not Sufficient Fund''
\end{itemize}

\textbf{Banking Operations:}

\begin{itemize}
\tightlist
\item
  \textbf{Deposit}: Adds money to balance
\item
  \textbf{Withdraw}: Subtracts money if sufficient balance
\item
  \textbf{Balance check}: Validates before withdrawal
\item
  \textbf{Exception handling}: Prevents program crash
\end{itemize}

\end{solutionbox}
\begin{mnemonicbox}
``Custom Throw Try-catch Message, Deposit Withdraw
Check Handle''

\end{mnemonicbox}
\begin{center}\rule{0.5\linewidth}{0.5pt}\end{center}

\subsection*{Question 4(a OR) [3
marks]}\label{question-4a-or-3-marks}

\textbf{Describe the complete lifecycle of a thread.}

\begin{solutionbox}

\textbf{Thread Lifecycle States:}

\includegraphics[width=1\linewidth,height=\textheight,keepaspectratio]{mermaid-cf7c3ace.pdf}

\textbf{Thread States:}

\begin{itemize}
\tightlist
\item
  \textbf{New}: Thread object created but not started
\item
  \textbf{Runnable}: Thread ready to run, waiting for CPU
\item
  \textbf{Running}: Thread currently executing
\item
  \textbf{Blocked}: Thread waiting for resource or condition
\item
  \textbf{Dead}: Thread execution completed
\end{itemize}

\textbf{State Transitions:}

\begin{itemize}
\tightlist
\item
  \textbf{start()}: New \rightarrow Runnable
\item
  \textbf{CPU allocation}: Runnable \rightarrow Running\\
\item
  \textbf{wait()/sleep()}: Running \rightarrow Blocked
\item
  \textbf{notify()/timeout}: Blocked \rightarrow Runnable
\item
  \textbf{completion}: Running \rightarrow Dead
\end{itemize}

\end{solutionbox}
\begin{mnemonicbox}
``New Runnable Running Blocked Dead''

\end{mnemonicbox}
\begin{center}\rule{0.5\linewidth}{0.5pt}\end{center}

\subsection*{Question 4(b OR) [4
marks]}\label{question-4b-or-4-marks}

\textbf{List access specifiers and describe their purpose in JAVA.}

\begin{solutionbox}

\textbf{Java Access Specifiers:}

{\def\LTcaptype{none} % do not increment counter
\begin{longtable}[]{@{}
  >{\raggedright\arraybackslash}p{(\linewidth - 8\tabcolsep) * \real{0.2609}}
  >{\raggedright\arraybackslash}p{(\linewidth - 8\tabcolsep) * \real{0.1739}}
  >{\raggedright\arraybackslash}p{(\linewidth - 8\tabcolsep) * \real{0.2029}}
  >{\raggedright\arraybackslash}p{(\linewidth - 8\tabcolsep) * \real{0.1449}}
  >{\raggedright\arraybackslash}p{(\linewidth - 8\tabcolsep) * \real{0.2174}}@{}}
\toprule\noalign{}
\begin{minipage}[b]{\linewidth}\raggedright
Access Specifier
\end{minipage} & \begin{minipage}[b]{\linewidth}\raggedright
Same Class
\end{minipage} & \begin{minipage}[b]{\linewidth}\raggedright
Same Package
\end{minipage} & \begin{minipage}[b]{\linewidth}\raggedright
Subclass
\end{minipage} & \begin{minipage}[b]{\linewidth}\raggedright
Other Package
\end{minipage} \\
\midrule\noalign{}
\endhead
\bottomrule\noalign{}
\endlastfoot
\textbf{private} & ✓ & ✗ & ✗ & ✗ \\
\textbf{default} & ✓ & ✓ & ✗ & ✗ \\
\textbf{protected} & ✓ & ✓ & ✓ & ✗ \\
\textbf{public} & ✓ & ✓ & ✓ & ✓ \\
\end{longtable}
}

\textbf{Access Specifier Purposes:}

\textbf{Private:}

\begin{itemize}
\tightlist
\item
  \textbf{Encapsulation}: Hides implementation details
\item
  \textbf{Data security}: Protects sensitive data
\item
  \textbf{Class-only access}: Accessible within same class only
\end{itemize}

\textbf{Default (Package-private):}

\begin{itemize}
\tightlist
\item
  \textbf{Package access}: Accessible within same package
\item
  \textbf{Module organization}: Groups related classes
\item
  \textbf{No keyword needed}: Default when no specifier mentioned
\end{itemize}

\textbf{Protected:}

\begin{itemize}
\tightlist
\item
  \textbf{Inheritance support}: Accessible to subclasses
\item
  \textbf{Package + inheritance}: Same package + subclasses
\item
  \textbf{Controlled access}: More access than private, less than public
\end{itemize}

\textbf{Public:}

\begin{itemize}
\tightlist
\item
  \textbf{Universal access}: Accessible from anywhere
\item
  \textbf{Interface methods}: Used for public APIs
\item
  \textbf{Maximum visibility}: No access restrictions
\end{itemize}

\end{solutionbox}
\begin{mnemonicbox}
``Private Encapsulates, Default Packages, Protected
inherits, Public Universal''

\end{mnemonicbox}
\begin{center}\rule{0.5\linewidth}{0.5pt}\end{center}

\subsection*{Question 4(c OR) [7
marks]}\label{question-4c-or-7-marks}

\textbf{Write a program that executes two threads. One thread displays
``Thread1'' every 1000 milliseconds, and the other displays ``Thread2''
every 2000 milliseconds. Create the threads by extending the Thread
class.}

\begin{solutionbox}

\begin{lstlisting}[language=Java]
// First thread class
class Thread1 extends Thread {
    public void run() {
        try {
for (int

i = 1; i <= 10; i++) {

                System.out.println("Thread1 - Count: " + i);
                Thread.sleep(1000); // Sleep for 1000 milliseconds
            }
        } catch (InterruptedException e) {
            System.out.println("Thread1 interrupted: " + e.getMessage());
        }
    }
}

// Second thread class
class Thread2 extends Thread {
    public void run() {
        try {
for (int

i = 1; i <= 5; i++) {

                System.out.println("Thread2 - Count: " + i);
                Thread.sleep(2000); // Sleep for 2000 milliseconds
            }
        } catch (InterruptedException e) {
            System.out.println("Thread2 interrupted: " + e.getMessage());
        }
    }
}

public class MultiThreadDemo {
    public static void main(String[] args) {
        // Create thread objects
        Thread1 t1 = new Thread1();
        Thread2 t2 = new Thread2();
        
        System.out.println("Starting both threads...");
        
        // Start both threads
        t1.start();
        t2.start();
        
        try {
            // Wait for both threads to complete
            t1.join();
            t2.join();
        } catch (InterruptedException e) {
            System.out.println("Main thread interrupted: " + e.getMessage());
        }
        
        System.out.println("Both threads completed execution");
    }
}
\end{lstlisting}

\textbf{Multithreading Concepts:}

\begin{itemize}
\tightlist
\item
  \textbf{Thread class extension}: Both classes extend Thread
\item
  \textbf{run() method}: Contains thread execution code
\item
  \textbf{sleep() method}: Pauses thread for specified time
\item
  \textbf{start() method}: Begins thread execution
\item
  \textbf{join() method}: Waits for thread completion
\end{itemize}

\textbf{Thread Synchronization:}

\begin{itemize}
\tightlist
\item
  \textbf{Concurrent execution}: Both threads run simultaneously
\item
  \textbf{Independent timing}: Each thread has its own sleep interval
\item
  \textbf{Exception handling}: InterruptedException caught and handled
\item
  \textbf{Thread coordination}: join() ensures main waits for completion
\end{itemize}

\end{solutionbox}
\begin{mnemonicbox}
``Extend Run Sleep Start Join''

\end{mnemonicbox}
\begin{center}\rule{0.5\linewidth}{0.5pt}\end{center}

\subsection*{Question 5(a) [3 marks]}\label{q5a}

\textbf{What is stream class? How are the stream classes classified?}

\begin{solutionbox}

\textbf{Stream Class Definition:} Stream classes in Java provide a way
to handle input and output operations. They represent a flow of data
from source to destination.

\textbf{Stream Classification:}

\begin{lstlisting}
                    Stream Classes
                          |
        ┌─────────────────┼─────────────────┐
        |                                   |
   Input Streams                      Output Streams
        |                                   |
┌───────┼───────┐                   ┌───────┼───────┐
|               |                   |               |
Byte Streams  Character Streams    Byte Streams  Character Streams
|               |                   |               |
InputStream   Reader              OutputStream    Writer
\end{lstlisting}

\textbf{Stream Types:}

{\def\LTcaptype{none} % do not increment counter
\begin{longtable}[]{@{}
  >{\raggedright\arraybackslash}p{(\linewidth - 4\tabcolsep) * \real{0.3333}}
  >{\raggedright\arraybackslash}p{(\linewidth - 4\tabcolsep) * \real{0.2308}}
  >{\raggedright\arraybackslash}p{(\linewidth - 4\tabcolsep) * \real{0.4359}}@{}}
\toprule\noalign{}
\begin{minipage}[b]{\linewidth}\raggedright
Stream Type
\end{minipage} & \begin{minipage}[b]{\linewidth}\raggedright
Purpose
\end{minipage} & \begin{minipage}[b]{\linewidth}\raggedright
Example Classes
\end{minipage} \\
\midrule\noalign{}
\endhead
\bottomrule\noalign{}
\endlastfoot
\textbf{InputStream} & Read bytes & FileInputStream,
BufferedInputStream \\
\textbf{OutputStream} & Write bytes & FileOutputStream,
BufferedOutputStream \\
\textbf{Reader} & Read characters & FileReader, BufferedReader \\
\textbf{Writer} & Write characters & FileWriter, BufferedWriter \\
\end{longtable}
}

\textbf{Classification Features:}

\begin{itemize}
\tightlist
\item
  \textbf{Direction}: Input (read) vs Output (write)
\item
  \textbf{Data type}: Byte streams vs Character streams
\item
  \textbf{Functionality}: Basic vs Buffered streams
\end{itemize}

\end{solutionbox}
\begin{mnemonicbox}
``Direction Data-type Functionality''

\end{mnemonicbox}
\begin{center}\rule{0.5\linewidth}{0.5pt}\end{center}

\subsection*{Question 5(b) [4 marks]}\label{q5b}

\textbf{Illustrate purpose of method overriding with example.}

\begin{solutionbox}

\textbf{Method Overriding Purpose:} Method overriding allows a subclass
to provide specific implementation of a method that is already defined
in its parent class.

\begin{lstlisting}[language=Java]
// Parent class
class Shape {
    public void draw() {
        System.out.println("Drawing a generic shape");
    }
    
    public double area() {
        return 0.0;
    }
}

// Child class overriding parent methods
class Circle extends Shape {
    private double radius;
    
    public Circle(double radius) {
        this.radius = radius;
    }
    
    // Overriding draw method
    @Override
    public void draw() {
        System.out.println("Drawing a circle with radius " + radius);
    }
    
    // Overriding area method
    @Override
    public double area() {
        return 3.14 * radius * radius;
    }
}

class Rectangle extends Shape {
    private double length, width;
    
    public Rectangle(double length, double width) {
        this.length = length;
        this.width = width;
    }
    
    @Override
    public void draw() {
        System.out.println("Drawing rectangle " + length + "x" + width);
    }
    
    @Override
    public double area() {
        return length * width;
    }
}

public class OverridingDemo {
    public static void main(String[] args) {
        Shape[] shapes = {
            new Circle(5.0),
            new Rectangle(4.0, 6.0)
        };
        
        for (Shape shape : shapes) {
            shape.draw();           // Calls overridden method
            System.out.println("Area: " + shape.area());
            System.out.println();
        }
    }
}
\end{lstlisting}

\textbf{Method Overriding Benefits:}

\begin{itemize}
\tightlist
\item
  \textbf{Runtime polymorphism}: Method selection at runtime
\item
  \textbf{Specific implementation}: Child class provides specific
  behavior
\item
  \textbf{Code flexibility}: Same interface, different implementations
\item
  \textbf{Dynamic method dispatch}: Correct method called based on
  object type
\end{itemize}

\end{solutionbox}
\begin{mnemonicbox}
``Runtime Specific Flexibility Dynamic''

\end{mnemonicbox}
\begin{center}\rule{0.5\linewidth}{0.5pt}\end{center}

\subsection*{Question 5(c) [7 marks]}\label{q5c}

\textbf{Write a program in Java to perform read and write operations on
a Text file named Abc.txt.}

\begin{solutionbox}

\begin{lstlisting}[language=Java]
import java.io.*;

public class FileOperations {
    public static void main(String[] args) {
        String fileName = "Abc.txt";
        
        // Write operation
        writeToFile(fileName);
        
        // Read operation
        readFromFile(fileName);
    }
    
    // Method to write data to file
    public static void writeToFile(String fileName) {
        try {
            FileWriter writer = new FileWriter(fileName);
            
            // Writing data to file
            writer.write("Hello, this is Java file handling.\n");
            writer.write("This is line 2 of the file.\n");
            writer.write("File operations in Java are easy.\n");
            writer.write("End of file content.");
            
            writer.close();
            System.out.println("Data written to file successfully.");
            
        } catch (IOException e) {
            System.out.println("Error writing to file: " + e.getMessage());
        }
    }
    
    // Method to read data from file
    public static void readFromFile(String fileName) {
        try {
            FileReader reader = new FileReader(fileName);
            BufferedReader bufferedReader = new BufferedReader(reader);
            
            System.out.println("\nReading from file:");
            System.out.println("-------------------");
            
            String line;
            int lineNumber = 1;
            
            // Reading file line by line
            while ((line = bufferedReader.readLine()) != null) {
                System.out.println("Line " + lineNumber + ": " + line);
                lineNumber++;
            }
            
            bufferedReader.close();
            reader.close();
            
        } catch (FileNotFoundException e) {
            System.out.println("File not found: " + e.getMessage());
        } catch (IOException e) {
            System.out.println("Error reading file: " + e.getMessage());
        }
    }
}
\end{lstlisting}

\textbf{Alternative using try-with-resources:}

\begin{lstlisting}[language=Java]
// More efficient way using try-with-resources
public static void writeToFileImproved(String fileName) {
    try (FileWriter writer = new FileWriter(fileName)) {
        writer.write("Hello from improved method!\n");
        writer.write("Using try-with-resources.\n");
        System.out.println("Data written successfully.");
    } catch (IOException e) {
        System.out.println("Error: " + e.getMessage());
    }
}

public static void readFromFileImproved(String fileName) {
    try (BufferedReader reader = new BufferedReader(new FileReader(fileName))) {
        String line;
        while ((line = reader.readLine()) != null) {
            System.out.println(line);
        }
    } catch (IOException e) {
        System.out.println("Error: " + e.getMessage());
    }
}
\end{lstlisting}

\textbf{File Operation Components:}

\begin{itemize}
\tightlist
\item
  \textbf{FileWriter}: Used for writing character data to file
\item
  \textbf{FileReader}: Used for reading character data from file
\item
  \textbf{BufferedReader}: More efficient reading with buffering
\item
  \textbf{Exception handling}: IOException and FileNotFoundException
\item
  \textbf{Resource management}: Close streams to prevent memory leaks
\end{itemize}

\textbf{File Handling Steps:}

\begin{itemize}
\tightlist
\item
  \textbf{Create stream}: FileWriter/FileReader object
\item
  \textbf{Perform operation}: write()/readLine() methods
\item
  \textbf{Handle exceptions}: try-catch blocks
\item
  \textbf{Close resources}: close() method or try-with-resources
\end{itemize}

\end{solutionbox}
\begin{mnemonicbox}
``Create Perform Handle Close''

\end{mnemonicbox}
\begin{center}\rule{0.5\linewidth}{0.5pt}\end{center}

\subsection*{Question 5(a OR) [3
marks]}\label{question-5a-or-3-marks}

\textbf{Explain InputStream.}

\begin{solutionbox}

\textbf{InputStream Definition:} InputStream is an abstract class that
represents an input stream of bytes. It is the superclass of all classes
representing an input stream of bytes.

\textbf{InputStream Hierarchy:}

\begin{lstlisting}
                    InputStream
                         |
        ┌────────────────┼────────────────┐
        |                |                |
FileInputStream   ByteArrayInputStream  FilterInputStream
                                             |
                                    ┌────────┼────────┐
                                    |                 |
                            BufferedInputStream  DataInputStream
\end{lstlisting}

\textbf{Common InputStream Methods:}

{\def\LTcaptype{none} % do not increment counter
\begin{longtable}[]{@{}
  >{\raggedright\arraybackslash}p{(\linewidth - 4\tabcolsep) * \real{0.2667}}
  >{\raggedright\arraybackslash}p{(\linewidth - 4\tabcolsep) * \real{0.4333}}
  >{\raggedright\arraybackslash}p{(\linewidth - 4\tabcolsep) * \real{0.3000}}@{}}
\toprule\noalign{}
\begin{minipage}[b]{\linewidth}\raggedright
Method
\end{minipage} & \begin{minipage}[b]{\linewidth}\raggedright
Description
\end{minipage} & \begin{minipage}[b]{\linewidth}\raggedright
Example
\end{minipage} \\
\midrule\noalign{}
\endhead
\bottomrule\noalign{}
\endlastfoot
\textbf{read()} & Reads single byte &
\passthrough{\lstinline!int b = in.read();!} \\
\textbf{read(byte[])} & Reads bytes into array &
\passthrough{\lstinline!in.read(buffer);!} \\
\textbf{available()} & Returns available bytes &
\passthrough{\lstinline!int count = in.available();!} \\
\textbf{close()} & Closes the stream &
\passthrough{\lstinline!in.close();!} \\
\textbf{skip(long)} & Skips specified bytes &
\passthrough{\lstinline!in.skip(10);!} \\
\end{longtable}
}

\textbf{Example Usage:}

\begin{lstlisting}[language=Java]
try (FileInputStream fis = new FileInputStream("data.txt")) {
    int data;
    while ((data = fis.read()) != -1) {
        System.out.print((char) data);
    }
} catch (IOException e) {
    e.printStackTrace();
}
\end{lstlisting}

\textbf{InputStream Features:}

\begin{itemize}
\tightlist
\item
  \textbf{Abstract class}: Cannot be instantiated directly
\item
  \textbf{Byte-oriented}: Handles byte data
\item
  \textbf{Input operations}: Reading data from various sources
\item
  \textbf{Resource management}: Must be closed after use
\end{itemize}

\end{solutionbox}
\begin{mnemonicbox}
``Abstract Byte Input Resource''

\end{mnemonicbox}
\begin{center}\rule{0.5\linewidth}{0.5pt}\end{center}

\subsection*{Question 5(b OR) [4
marks]}\label{question-5b-or-4-marks}

\textbf{Define package in JAVA. Write how package can be implemented in
Java with proper syntax and one example.}

\begin{solutionbox}

\textbf{Package Definition:} A package in Java is a namespace that
organizes related classes and interfaces together. It provides access
protection and namespace management.

\textbf{Package Implementation Syntax:}

\begin{lstlisting}[language=Java]
// 1. Package declaration (must be first line)
package packageName;

// 2. Import statements
import java.util.*;
import anotherPackage.ClassName;

// 3. Class definition
public class ClassName {
    // class body
}
\end{lstlisting}

\textbf{Package Creation Steps:}

{\def\LTcaptype{none} % do not increment counter
\begin{longtable}[]{@{}
  >{\raggedright\arraybackslash}p{(\linewidth - 4\tabcolsep) * \real{0.2609}}
  >{\raggedright\arraybackslash}p{(\linewidth - 4\tabcolsep) * \real{0.3478}}
  >{\raggedright\arraybackslash}p{(\linewidth - 4\tabcolsep) * \real{0.3913}}@{}}
\toprule\noalign{}
\begin{minipage}[b]{\linewidth}\raggedright
Step
\end{minipage} & \begin{minipage}[b]{\linewidth}\raggedright
Action
\end{minipage} & \begin{minipage}[b]{\linewidth}\raggedright
Example
\end{minipage} \\
\midrule\noalign{}
\endhead
\bottomrule\noalign{}
\endlastfoot
\textbf{1. Create directory} & Make folder with package name &
\passthrough{\lstinline!mkdir mypackage!} \\
\textbf{2. Package declaration} & Add package statement &
\passthrough{\lstinline!package mypackage;!} \\
\textbf{3. Compile} & Compile with proper classpath &
\passthrough{\lstinline!javac -d . ClassName.java!} \\
\textbf{4. Run} & Run with fully qualified name &
\passthrough{\lstinline!java mypackage.ClassName!} \\
\end{longtable}
}

\textbf{Complete Example:}

\textbf{File: utilities/MathOperations.java}

\begin{lstlisting}[language=Java]
package utilities;

public class MathOperations {
    public static int add(int a, int b) {
        return a + b;
    }
    
    public static int multiply(int a, int b) {
        return a * b;
    }
    
    public static double calculateArea(double radius) {
        return 3.14 * radius * radius;
    }
}
\end{lstlisting}

\textbf{File: utilities/StringOperations.java}

\begin{lstlisting}[language=Java]
package utilities;

public class StringOperations {
    public static String reverse(String str) {
        return new StringBuilder(str).reverse().toString();
    }
    
    public static boolean isPalindrome(String str) {
        String reversed = reverse(str);
        return str.equals(reversed);
    }
}
\end{lstlisting}

\textbf{File: TestPackage.java}

\begin{lstlisting}[language=Java]
import utilities.MathOperations;
import utilities.StringOperations;

public class TestPackage {
    public static void main(String[] args) {
        // Using MathOperations
        int sum = MathOperations.add(10, 20);
        int product = MathOperations.multiply(5, 4);
        double area = MathOperations.calculateArea(7.0);
        
        System.out.println("Sum: " + sum);
        System.out.println("Product: " + product);
        System.out.println("Area: " + area);
        
        // Using StringOperations
        String original = "hello";
        String reversed = StringOperations.reverse(original);
        boolean isPalindrome = StringOperations.isPalindrome("madam");
        
        System.out.println("Original: " + original);
        System.out.println("Reversed: " + reversed);
        System.out.println("Is 'madam' palindrome? " + isPalindrome);
    }
}
\end{lstlisting}

\textbf{Package Benefits:}

\begin{itemize}
\tightlist
\item
  \textbf{Namespace management}: Avoids naming conflicts
\item
  \textbf{Access control}: Package-private access level
\item
  \textbf{Code organization}: Groups related functionality
\item
  \textbf{Reusability}: Easy to import and use
\end{itemize}

\end{solutionbox}
\begin{mnemonicbox}
``Namespace Access Organization Reusability''

\end{mnemonicbox}
\begin{center}\rule{0.5\linewidth}{0.5pt}\end{center}

\subsection*{Question 5(c OR) [7
marks]}\label{question-5c-or-7-marks}

\textbf{Write a program in Java to demonstrate use of List. 1) Create
ArrayList and add weekdays (in string form), 2) Create LinkedList and
add months (in string form). Display both List.}

\begin{solutionbox}

\begin{lstlisting}[language=Java]
import java.util.*;

public class ListDemo {
    public static void main(String[] args) {
        // Demonstrate ArrayList with weekdays
        demonstrateArrayList();
        
        System.out.println("\n" + "=".repeat(50) + "\n");
        
        // Demonstrate LinkedList with months
        demonstrateLinkedList();
        
        System.out.println("\n" + "=".repeat(50) + "\n");
        
        // Compare both lists
        compareListOperations();
    }
    
    // Method to demonstrate ArrayList
    public static void demonstrateArrayList() {
        System.out.println("ARRAYLIST DEMONSTRATION");
        System.out.println("========================");
        
        // Create ArrayList and add weekdays
        ArrayList< String > weekdays = new ArrayList< String >();
        
        // Adding weekdays
        weekdays.add("Monday");
        weekdays.add("Tuesday");
        weekdays.add("Wednesday");
        weekdays.add("Thursday");
        weekdays.add("Friday");
        weekdays.add("Saturday");
        weekdays.add("Sunday");
        
        // Display ArrayList
        System.out.println("Weekdays in ArrayList:");
        for (int i = 0; i < weekdays.size(); i++) {
            System.out.println((i + 1) + ". " + weekdays.get(i));
        }
        
        // ArrayList specific operations
        System.out.println("\nArrayList Operations:");
        System.out.println("Size: " + weekdays.size());
        System.out.println("First day: " + weekdays.get(0));
        System.out.println("Last day: " + weekdays.get(weekdays.size() - 1));
        System.out.println("Contains 'Friday': " + weekdays.contains("Friday"));
        
        // Enhanced for loop
        System.out.println("\nUsing Enhanced For Loop:");
        for (String day : weekdays) {
            System.out.print(day + " ");
        }
        System.out.println();
    }
    
    // Method to demonstrate LinkedList
    public static void demonstrateLinkedList() {
        System.out.println("LINKEDLIST DEMONSTRATION");
        System.out.println("=========================");
        
        // Create LinkedList and add months
        LinkedList< String > months = new LinkedList< String >();
        
        // Adding months
        months.add("January");
        months.add("February");
        months.add("March");
        months.add("April");
        months.add("May");
        months.add("June");
        months.add("July");
        months.add("August");
        months.add("September");
        months.add("October");
        months.add("November");
        months.add("December");
        
        // Display LinkedList
        System.out.println("Months in LinkedList:");
        for (int i = 0; i < months.size(); i++) {
            System.out.println((i + 1) + ". " + months.get(i));
        }
        
        // LinkedList specific operations
        System.out.println("\nLinkedList Operations:");
        System.out.println("Size: " + months.size());
        System.out.println("First month: " + months.getFirst());
        System.out.println("Last month: " + months.getLast());
        
        // Add at specific positions
        months.addFirst("START");
        months.addLast("END");
        
        System.out.println("\nAfter adding at first and last:");
        System.out.println("First element: " + months.getFirst());
        System.out.println("Last element: " + months.getLast());
        System.out.println("Total size: " + months.size());
        
        // Remove the added elements
        months.removeFirst();
        months.removeLast();
        
        // Using Iterator
        System.out.println("\nUsing Iterator:");
        Iterator< String > iterator = months.iterator();
        while (iterator.hasNext()) {
            System.out.print(iterator.next() + " ");
        }
        System.out.println();
    }
    
    // Method to compare list operations
    public static void compareListOperations() {
        System.out.println("LIST COMPARISON");
        System.out.println("================");
        
        // Create both lists with sample data
        ArrayList< String > arrayList = new ArrayList< String >();
        LinkedList< String > linkedList = new LinkedList< String >();
        
        // Add sample data
        String[] data = {"A", "B", "C", "D", "E"};
        
        for (String item : data) {
            arrayList.add(item);
            linkedList.add(item);
        }
        
        System.out.println("ArrayList: " + arrayList);
        System.out.println("LinkedList: " + linkedList);
        
        // Performance comparison info
        System.out.println("\nPerformance Characteristics:");
        System.out.println("ArrayList - Better for: Random access, Memory efficient");
        System.out.println("LinkedList - Better for: Insertion/Deletion, Dynamic size");
    }
}
\end{lstlisting}

\textbf{List Interface Features:}

{\def\LTcaptype{none} % do not increment counter
\begin{longtable}[]{@{}lll@{}}
\toprule\noalign{}
Feature & ArrayList & LinkedList \\
\midrule\noalign{}
\endhead
\bottomrule\noalign{}
\endlastfoot
\textbf{Internal Structure} & Dynamic array & Doubly linked list \\
\textbf{Access Time} & O(1) random access & O(n) sequential access \\
\textbf{Insertion/Deletion} & O(n) at middle & O(1) at ends \\
\textbf{Memory} & Less memory overhead & More memory for pointers \\
\end{longtable}
}

\textbf{Common List Methods:}

\begin{itemize}
\tightlist
\item
  \textbf{add()}: Adds element to list
\item
  \textbf{get()}: Retrieves element by index
\item
  \textbf{size()}: Returns number of elements
\item
  \textbf{contains()}: Checks if element exists
\item
  \textbf{remove()}: Removes element
\item
  \textbf{iterator()}: Returns iterator for traversal
\end{itemize}

\textbf{List Benefits:}

\begin{itemize}
\tightlist
\item
  \textbf{Dynamic size}: Grows and shrinks automatically
\item
  \textbf{Ordered collection}: Maintains insertion order
\item
  \textbf{Duplicate elements}: Allows duplicate values
\item
  \textbf{Index-based access}: Access elements by position
\end{itemize}

\end{solutionbox}
\begin{mnemonicbox}
``Dynamic Ordered Duplicate Index-based''

\end{mnemonicbox}

\end{document}
