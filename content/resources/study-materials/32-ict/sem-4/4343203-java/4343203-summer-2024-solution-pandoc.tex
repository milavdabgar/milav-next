\documentclass[10pt,a4paper]{article}

% content/resources/templates/preamble.tex
\usepackage[margin=0.6in]{geometry}
\author{Milav Dabgar}
\usepackage{amsmath,amssymb,amsthm}
\usepackage{booktabs}
\usepackage{multirow}
\usepackage{xcolor}
\usepackage{tcolorbox}
\tcbuselibrary{breakable,skins}
\usepackage[colorlinks=true,linkcolor=blue]{hyperref}
\usepackage{titlesec}
\usepackage{enumitem}
\usepackage{tikz}
\usepackage{pgfplots}
\usepackage{circuitikz}
\usepackage[version=4]{mhchem}
\usepackage{longtable}
\usepackage{array}
\usepackage{float}
\usepackage{caption}
\usepackage{listings}

\lstset{
  basicstyle=\small\ttfamily,
  breaklines=true,
  breakatwhitespace=false,
  postbreak=\mbox{\textcolor{red}{$\hookrightarrow$}\space},
  float=false,
  numbers=left,
  numberstyle=\tiny\color{gray},
  numbersep=10pt,
  xleftmargin=2em,
  keywordstyle=\color{blue},
  commentstyle=\color{green!60!black},
  stringstyle=\color{purple},
  backgroundcolor=\color{gray!5},
  showstringspaces=false,
  tabsize=2,
  captionpos=b,
  keepspaces=true,
  columns=flexible
}

\pgfplotsset{compat=1.18}
\usetikzlibrary{shapes,arrows,positioning,calc,patterns,decorations.pathmorphing,decorations.markings,arrows.meta}

% Color scheme
\definecolor{headcolor}{RGB}{0,102,204}
\definecolor{keycolor}{RGB}{220,20,60}
\definecolor{solutioncolor}{RGB}{34,139,34}
\definecolor{mnemoniccolor}{RGB}{148,0,211}
\definecolor{codecolor}{RGB}{0,0,100}

% Spacing
\setlength{\parskip}{3pt}
\setlist[itemize]{nosep}
\setlist[enumerate]{nosep}

% Title formatting
\titleformat{\section}{\Large\bfseries\color{headcolor}}{\thesection}{1em}{}
\titleformat{\subsection}{\large\bfseries\color{headcolor}}{\thesubsection}{1em}{}

% Pandoc tightlist compatibility
\providecommand{\tightlist}{%
  \setlength{\itemsep}{0pt}\setlength{\parskip}{0pt}}

% Pandoc longtable compatibility
\newcounter{none}
\def\thenone{}


% content/resources/templates/english-boxes.tex
% This file is currently empty - it exists to maintain consistency with the import structure.
% Add custom environments here if needed in the future.


\begin{document}

\begin{center}
{\Huge\bfseries\color{headcolor} Subject Name Solutions}\\[5pt]
{\LARGE 4343203 -- Summer 2024}\\[3pt]
{\large Semester 1 Study Material}\\[3pt]
{\normalsize\textit{Detailed Solutions and Explanations}}
\end{center}

\vspace{10pt}

\subsection*{Question 1(a) [3 marks]}\label{q1a}

\textbf{Explain Garbage collection in java.}

\begin{solutionbox}
Garbage collection in Java automatically reclaims
memory by removing unused objects.


{\def\LTcaptype{none} % do not increment counter
\vspace{-5pt}
\captionof{table}{Garbage Collection Process}
\vspace{-10pt}
\begin{longtable}[]{@{}ll@{}}
\toprule\noalign{}
Phase & Description \\
\midrule\noalign{}
\endhead
\bottomrule\noalign{}
\endlastfoot
Mark & JVM identifies all live objects in memory \\
Sweep & Unused objects are removed \\
Compact & Remaining objects are reorganized to free up space \\
\end{longtable}
}

\begin{itemize}
\tightlist
\item
  \textbf{Automatic}: No manual memory management required
\item
  \textbf{Background}: Runs in separate low-priority thread
\end{itemize}

\end{solutionbox}
\begin{mnemonicbox}
``MSC: Mark-Sweep-Compact frees memory
automatically''

\end{mnemonicbox}
\subsection*{Question 1(b) [4 marks]}\label{q1b}

\textbf{Explain JVM in detail.}

\begin{solutionbox}
JVM (Java Virtual Machine) is a virtual machine that
enables Java's platform independence by converting bytecode to machine
code.

\textbf{Diagram: JVM Architecture}

\includegraphics[width=1\linewidth,height=\textheight,keepaspectratio]{mermaid-ed876782.pdf}

\begin{itemize}
\tightlist
\item
  \textbf{Platform Independence}: Write once, run anywhere
\item
  \textbf{Security}: Bytecode verification prevents dangerous operations
\item
  \textbf{Optimization}: Just-in-time compilation improves performance
\end{itemize}

\end{solutionbox}
\begin{mnemonicbox}
``CLASS: Class Loader Leads All System Security''

\end{mnemonicbox}
\subsection*{Question 1(c) [7 marks]}\label{q1c}

\textbf{Write a program in java to print Fibonacci series for N terms.}

\begin{solutionbox}
Fibonacci series generates numbers where each is the
sum of the two preceding ones.

\textbf{Code Block:}

\begin{lstlisting}[language=Java]
import java.util.Scanner;

public class FibonacciSeries {
    public static void main(String[] args) {
        Scanner input = new Scanner(System.in);
        
        System.out.print("Enter number of terms: ");
        int n = input.nextInt();
        
        int first = 0, second = 1;
        
        System.out.print("Fibonacci Series: ");
        
for (int

i = 1; i <= n; i++) {

            System.out.print(first + " ");
            
            int next = first + second;
            first = second;
            second = next;
        }
        
        input.close();
    }
}
\end{lstlisting}

\begin{itemize}
\tightlist
\item
  \textbf{Initialize}: Start with 0 and 1
\item
  \textbf{Loop}: Iterate N times to generate sequence
\item
  \textbf{Calculation}: Each number is sum of previous two
\end{itemize}

\end{solutionbox}
\begin{mnemonicbox}
``FSN: First + Second = Next number in sequence''

\end{mnemonicbox}
\subsection*{Question 1(c OR) [7
marks]}\label{question-1c-or-7-marks}

\textbf{Write a program in java to find out minimum from any ten numbers
using command line argument.}

\begin{solutionbox}
Command line arguments allow passing input values
directly when executing a Java program.

\textbf{Code Block:}

\begin{lstlisting}[language=Java]
public class FindMinimum {
    public static void main(String[] args) {
        if (args.length < 10) {
            System.out.println("Please provide 10 numbers");
            return;
        }
        
        int min = Integer.parseInt(args[0]);
        
        for (int i = 1; i < 10; i++) {
            int current = Integer.parseInt(args[i]);
            if (current < min) {
                min = current;
            }
        }
        
        System.out.println("Minimum number is: " + min);
    }
}
\end{lstlisting}

\begin{itemize}
\tightlist
\item
  \textbf{Parse Arguments}: Convert string arguments to integers
\item
  \textbf{Initialize}: Set first number as minimum
\item
  \textbf{Compare}: Check each number against current minimum
\end{itemize}

\end{solutionbox}
\begin{mnemonicbox}
``ICU: Initialize, Compare, Update the minimum''

\end{mnemonicbox}
\subsection*{Question 2(a) [3 marks]}\label{q2a}

\textbf{List out basic concepts of Java OOP. Explain any one in
details.}

\begin{solutionbox}
Java Object-Oriented Programming is built on
fundamental concepts for modeling real-world entities.


{\def\LTcaptype{none} % do not increment counter
\vspace{-5pt}
\captionof{table}{OOP Concepts in Java}
\vspace{-10pt}
\begin{longtable}[]{@{}ll@{}}
\toprule\noalign{}
Concept & Description \\
\midrule\noalign{}
\endhead
\bottomrule\noalign{}
\endlastfoot
Encapsulation & Binding data and methods together as a single unit \\
Inheritance & Creating new classes from existing ones \\
Polymorphism & One interface, multiple implementations \\
Abstraction & Hiding implementation details, showing functionality \\
\end{longtable}
}

\begin{itemize}
\tightlist
\item
  \textbf{Encapsulation}: Protects data through access control
\item
  \textbf{Data Hiding}: Private variables accessible through methods
\end{itemize}

\end{solutionbox}
\begin{mnemonicbox}
``PEAI: Programming Encapsulates Abstracts Inherits''

\end{mnemonicbox}
\subsection*{Question 2(b) [4 marks]}\label{q2b}

\textbf{Explain final keyword with example.}

\begin{solutionbox}
The final keyword in Java restricts modification and
creates constants, unchangeable methods, and non-inheritable classes.


{\def\LTcaptype{none} % do not increment counter
\vspace{-5pt}
\captionof{table}{Uses of final Keyword}
\vspace{-10pt}
\begin{longtable}[]{@{}lll@{}}
\toprule\noalign{}
Usage & Effect & Example \\
\midrule\noalign{}
\endhead
\bottomrule\noalign{}
\endlastfoot
final variable & Cannot be changed &
\passthrough{\lstinline!final int MAX = 100;!} \\
final method & Cannot be overridden &
\passthrough{\lstinline!final void display() \{\}!} \\
final class & Cannot be extended &
\passthrough{\lstinline!final class Math \{\}!} \\
\end{longtable}
}

\textbf{Code Block:}

\begin{lstlisting}[language=Java]
public class FinalDemo {
    final int MAX_VALUE = 100;  // constant
    
    final void display() {
        System.out.println("This method cannot be overridden");
    }
}

final class MathOperations {
    // This class cannot be inherited
}
\end{lstlisting}

\end{solutionbox}
\begin{mnemonicbox}
``VCM: Variables Constants Methods can't change''

\end{mnemonicbox}
\subsection*{Question 2(c) [7 marks]}\label{q2c}

\textbf{What is constructor? Explain parameterized constructor with
example.}

\begin{solutionbox}
A constructor initializes objects when created, with
the same name as its class and no return type.

\textbf{Diagram: Constructor Types}

\includegraphics[width=1\linewidth,height=\textheight,keepaspectratio]{mermaid-f17d38c4.pdf}

\textbf{Code Block:}

\begin{lstlisting}[language=Java]
public class Student {
    String name;
    int age;
    
    // Parameterized constructor
    Student(String n, int a) {
        name = n;
        age = a;
    }
    
    void display() {
        System.out.println("Name: " + name + ", Age: " + age);
    }
    
    public static void main(String[] args) {
        // Object creation using parameterized constructor
        Student s1 = new Student("John", 20);
        s1.display();
    }
}
\end{lstlisting}

\begin{itemize}
\tightlist
\item
  \textbf{Parameters}: Accept values during object creation
\item
  \textbf{Initialization}: Set object properties with passed values
\item
  \textbf{Overloading}: Multiple constructors with different parameters
\end{itemize}

\end{solutionbox}
\begin{mnemonicbox}
``SPO: Student Parameters Object initializes
properties''

\end{mnemonicbox}
\subsection*{Question 2(a OR) [3
marks]}\label{question-2a-or-3-marks}

\textbf{Explain the Java Program Structure with example.}

\begin{solutionbox}
Java program structure follows a specific hierarchy of
elements organized logically.

\textbf{Diagram: Java Program Structure}

\begin{lstlisting}
+--------------------+
| Documentation      |
| package statement  |
| import statements  |
+--------------------+
| Class declaration  |
|  +----------------+|
|  | Variables      ||
|  | Constructors   ||
|  | Methods        ||
|  +----------------+|
+--------------------+
\end{lstlisting}

\begin{itemize}
\tightlist
\item
  \textbf{Package}: Groups related classes
\item
  \textbf{Import}: Includes external classes
\item
  \textbf{Class}: Contains variables and methods
\end{itemize}

\end{solutionbox}
\begin{mnemonicbox}
``PIC: Package Imports Class in every program''

\end{mnemonicbox}
\subsection*{Question 2(b OR) [4
marks]}\label{question-2b-or-4-marks}

\textbf{Explain static keyword with suitable example.}

\begin{solutionbox}
Static keyword creates class-level variables and
methods shared by all objects, accessible without creating instances.


{\def\LTcaptype{none} % do not increment counter
\vspace{-5pt}
\captionof{table}{Static vs Non-Static}
\vspace{-10pt}
\begin{longtable}[]{@{}lll@{}}
\toprule\noalign{}
Feature & Static & Non-Static \\
\midrule\noalign{}
\endhead
\bottomrule\noalign{}
\endlastfoot
Memory & Single copy & Multiple copies \\
Access & Without object & Through object \\
Reference & Class name & Object name \\
When loaded & Class loading & Object creation \\
\end{longtable}
}

\textbf{Code Block:}

\begin{lstlisting}[language=Java]
public class Counter {
    static int count = 0;  // Shared by all objects
    int instanceCount = 0; // Unique to each object
    
    Counter() {
        count++;
        instanceCount++;
    }
    
    public static void main(String[] args) {
        Counter c1 = new Counter();
        Counter c2 = new Counter();
        
        System.out.println("Static count: " + Counter.count);
        System.out.println("c1's instance count: " + c1.instanceCount);
        System.out.println("c2's instance count: " + c2.instanceCount);
    }
}
\end{lstlisting}

\end{solutionbox}
\begin{mnemonicbox}
``SCM: Static Creates Memory once for all objects''

\end{mnemonicbox}
\subsection*{Question 2(c OR) [7
marks]}\label{question-2c-or-7-marks}

\textbf{Define Inheritance. List out types of it. Explain multilevel and
hierarchical inheritance with suitable example.}

\begin{solutionbox}
Inheritance is an OOP principle where a new class
acquires properties and behaviors from an existing class.


{\def\LTcaptype{none} % do not increment counter
\vspace{-5pt}
\captionof{table}{Types of Inheritance in Java}
\vspace{-10pt}
\begin{longtable}[]{@{}ll@{}}
\toprule\noalign{}
Type & Description \\
\midrule\noalign{}
\endhead
\bottomrule\noalign{}
\endlastfoot
Single & One subclass extends one superclass \\
Multilevel & Chain of inheritance (A\rightarrowB\rightarrowC) \\
Hierarchical & Multiple subclasses extend one superclass \\
Multiple & One class extends multiple classes (via interfaces) \\
\end{longtable}
}

\textbf{Diagram: Multilevel vs Hierarchical Inheritance}

\includegraphics[width=1\linewidth,height=\textheight,keepaspectratio]{mermaid-593f65e9.pdf}

\textbf{Code Block:}

\begin{lstlisting}[language=Java]
// Multilevel inheritance
class Animal {
    void eat() { System.out.println("eating"); }
}

class Dog extends Animal {
    void bark() { System.out.println("barking"); }
}

class Labrador extends Dog {
    void color() { System.out.println("golden"); }
}

// Hierarchical inheritance
class Vehicle {
    void move() { System.out.println("moving"); }
}

class Car extends Vehicle {
    void wheels() { System.out.println("4 wheels"); }
}

class Bike extends Vehicle {
    void wheels() { System.out.println("2 wheels"); }
}
\end{lstlisting}

\end{solutionbox}
\begin{mnemonicbox}
``SMHM: Single Multilevel Hierarchical Makes
inheritance types''

\end{mnemonicbox}
\subsection*{Question 3(a) [3 marks]}\label{q3a}

\textbf{Explain this keyword with suitable example.}

\begin{solutionbox}
The `this' keyword in Java refers to the current
object, used to differentiate between instance variables and parameters.


{\def\LTcaptype{none} % do not increment counter
\vspace{-5pt}
\captionof{table}{Uses of `this' Keyword}
\vspace{-10pt}
\begin{longtable}[]{@{}ll@{}}
\toprule\noalign{}
Use & Purpose \\
\midrule\noalign{}
\endhead
\bottomrule\noalign{}
\endlastfoot
this.variable & Access instance variables \\
this() & Call current class constructor \\
return this & Return current object \\
\end{longtable}
}

\textbf{Code Block:}

\begin{lstlisting}[language=Java]
public class Student {
    String name;
    
    Student(String name) {
        this.name = name;  // Refers to instance variable
    }
    
    void display() {
        System.out.println("Name: " + this.name);
    }
}
\end{lstlisting}

\end{solutionbox}
\begin{mnemonicbox}
``VAR: Variables Access Resolution using this''

\end{mnemonicbox}
\subsection*{Question 3(b) [4 marks]}\label{q3b}

\textbf{Explain different access controls in Java.}

\begin{solutionbox}
Access controls in Java regulate visibility and
accessibility of classes, methods, and variables.


{\def\LTcaptype{none} % do not increment counter
\vspace{-5pt}
\captionof{table}{Java Access Modifiers}
\vspace{-10pt}
\begin{longtable}[]{@{}lllll@{}}
\toprule\noalign{}
Modifier & Class & Package & Subclass & World \\
\midrule\noalign{}
\endhead
\bottomrule\noalign{}
\endlastfoot
private & ✓ & ✗ & ✗ & ✗ \\
default & ✓ & ✓ & ✗ & ✗ \\
protected & ✓ & ✓ & ✓ & ✗ \\
public & ✓ & ✓ & ✓ & ✓ \\
\end{longtable}
}

\begin{itemize}
\tightlist
\item
  \textbf{Private}: Only within the same class
\item
  \textbf{Default}: Within the same package
\item
  \textbf{Protected}: Within package and subclasses
\item
  \textbf{Public}: Accessible everywhere
\end{itemize}

\end{solutionbox}
\begin{mnemonicbox}
``PDPP: Private Default Protected Public from narrow
to wide''

\end{mnemonicbox}
\subsection*{Question 3(c) [7 marks]}\label{q3c}

\textbf{What is interface? Explain multiple inheritance using interface
with example.}

\begin{solutionbox}
An interface is a contract that specifies what a class
must do, containing abstract methods, constants, and (since Java 8)
default methods.

\textbf{Diagram: Multiple Inheritance with Interfaces}

\includegraphics[width=1\linewidth,height=\textheight,keepaspectratio]{mermaid-2b8980a4.pdf}

\textbf{Code Block:}

\begin{lstlisting}[language=Java]
interface Printable {
    void print();
}

interface Scannable {
    void scan();
}

// Multiple inheritance using interfaces
class Printer implements Printable, Scannable {
    public void print() {
        System.out.println("Printing...");
    }
    
    public void scan() {
        System.out.println("Scanning...");
    }
    
    public static void main(String[] args) {
        Printer p = new Printer();
        p.print();
        p.scan();
    }
}
\end{lstlisting}

\begin{itemize}
\tightlist
\item
  \textbf{Contract}: Defines behavior without implementation
\item
  \textbf{Implements}: Classes fulfill the contract
\item
  \textbf{Multiple}: Can implement many interfaces
\end{itemize}

\end{solutionbox}
\begin{mnemonicbox}
``CIM: Contract Implements Multiple interfaces''

\end{mnemonicbox}
\subsection*{Question 3(a OR) [3
marks]}\label{question-3a-or-3-marks}

\textbf{Explain super keyword with example.}

\begin{solutionbox}
The super keyword refers to the parent class, used to
access parent methods, constructors, and variables.


{\def\LTcaptype{none} % do not increment counter
\vspace{-5pt}
\captionof{table}{Uses of super Keyword}
\vspace{-10pt}
\begin{longtable}[]{@{}ll@{}}
\toprule\noalign{}
Use & Purpose \\
\midrule\noalign{}
\endhead
\bottomrule\noalign{}
\endlastfoot
super.variable & Access parent variable \\
super.method() & Call parent method \\
super() & Call parent constructor \\
\end{longtable}
}

\textbf{Code Block:}

\begin{lstlisting}[language=Java]
class Vehicle {
    String color = "white";
    
    void display() {
        System.out.println("Vehicle class");
    }
}

class Car extends Vehicle {
    String color = "black";
    
    void display() {
        super.display();  // Calls parent method
        System.out.println("Car color: " + color);
        System.out.println("Vehicle color: " + super.color);
    }
}
\end{lstlisting}

\end{solutionbox}
\begin{mnemonicbox}
``VMC: Variables Methods Constructors accessed by
super''

\end{mnemonicbox}
\subsection*{Question 3(b OR) [4
marks]}\label{question-3b-or-4-marks}

\textbf{What is package? Write steps to create a package and give
example of it.}

\begin{solutionbox}
A package in Java is a namespace that organizes related
classes and interfaces, preventing naming conflicts.


{\def\LTcaptype{none} % do not increment counter
\vspace{-5pt}
\captionof{table}{Steps to Create a Package}
\vspace{-10pt}
\begin{longtable}[]{@{}ll@{}}
\toprule\noalign{}
Step & Action \\
\midrule\noalign{}
\endhead
\bottomrule\noalign{}
\endlastfoot
1 & Declare package name at top of file \\
2 & Create directory structure matching package name \\
3 & Save Java file in the directory \\
4 & Compile with -d option \\
5 & Import package to use it \\
\end{longtable}
}

\textbf{Code Block:}

\begin{lstlisting}[language=Java]
// Step 1: Declare package (save as Calculator.java)
package mathematics;

public class Calculator {
    public int add(int a, int b) {
        return a + b;
    }
}

// In another file (UseCalculator.java)
import mathematics.Calculator;

class UseCalculator {
    public static void main(String[] args) {
        Calculator calc = new Calculator();
        System.out.println(calc.add(10, 20));
    }
}
\end{lstlisting}

\end{solutionbox}
\begin{mnemonicbox}
``DISCO: Declare Import Save Compile Organize''

\end{mnemonicbox}
\subsection*{Question 3(c OR) [7
marks]}\label{question-3c-or-7-marks}

\textbf{Define: Method Overriding. List out Rules for method overriding.
Write a java program that implements method overriding.}

\begin{solutionbox}
Method overriding occurs when a subclass provides a
specific implementation for a method already defined in its parent
class.


{\def\LTcaptype{none} % do not increment counter
\vspace{-5pt}
\captionof{table}{Rules for Method Overriding}
\vspace{-10pt}
\begin{longtable}[]{@{}ll@{}}
\toprule\noalign{}
Rule & Description \\
\midrule\noalign{}
\endhead
\bottomrule\noalign{}
\endlastfoot
Same name & Method must have same name \\
Same parameters & Parameter count and type must match \\
Same return type & Return type must be same or subtype (covariant) \\
Access modifier & Can't be more restrictive \\
Exceptions & Can't throw broader checked exceptions \\
\end{longtable}
}

\textbf{Code Block:}

\begin{lstlisting}[language=Java]
class Animal {
    void makeSound() {
        System.out.println("Animal makes a sound");
    }
}

class Dog extends Animal {
    // Method overriding
    @Override
    void makeSound() {
        System.out.println("Dog barks");
    }
}

class Cat extends Animal {
    // Method overriding
    @Override
    void makeSound() {
        System.out.println("Cat meows");
    }
}

public class MethodOverridingDemo {
    public static void main(String[] args) {
        Animal animal = new Animal();
        Animal dog = new Dog();
        Animal cat = new Cat();
        
        animal.makeSound();  // Output: Animal makes a sound
        dog.makeSound();     // Output: Dog barks
        cat.makeSound();     // Output: Cat meows
    }
}
\end{lstlisting}

\begin{itemize}
\tightlist
\item
  \textbf{Runtime Polymorphism}: Method resolution at runtime
\item
  \textbf{@Override}: Annotation ensures method is overriding
\item
  \textbf{Inheritance}: Requires IS-A relationship
\end{itemize}

\end{solutionbox}
\begin{mnemonicbox}
``SPARE: Same Parameters Access Return Exceptions''

\end{mnemonicbox}
\subsection*{Question 4(a) [3 marks]}\label{q4a}

\textbf{Explain abstract class with suitable example.}

\begin{solutionbox}
An abstract class cannot be instantiated and may
contain abstract methods that must be implemented by subclasses.


{\def\LTcaptype{none} % do not increment counter
\vspace{-5pt}
\captionof{table}{Abstract Class vs Interface}
\vspace{-10pt}
\begin{longtable}[]{@{}lll@{}}
\toprule\noalign{}
Feature & Abstract Class & Interface \\
\midrule\noalign{}
\endhead
\bottomrule\noalign{}
\endlastfoot
Instantiation & Cannot & Cannot \\
Methods & Concrete and abstract & Abstract (+ default since Java 8) \\
Variables & Any type & Only constants \\
Constructor & Has & Doesn't have \\
\end{longtable}
}

\textbf{Code Block:}

\begin{lstlisting}[language=Java]
abstract class Shape {
    // Abstract method - no implementation
    abstract double area();
    
    // Concrete method
    void display() {
        System.out.println("This is a shape");
    }
}

class Circle extends Shape {
    double radius;
    
    Circle(double r) {
        radius = r;
    }
    
    // Implementation of abstract method
    double area() {
        return 3.14 * radius * radius;
    }
}
\end{lstlisting}

\end{solutionbox}
\begin{mnemonicbox}
``PAI: Partial Abstract Implementation is key''

\end{mnemonicbox}
\subsection*{Question 4(b) [4 marks]}\label{q4b}

\textbf{What is Thread? Explain Thread life cycle.}

\begin{solutionbox}
A thread is a lightweight subprocess, the smallest unit
of processing that allows concurrent execution.

\textbf{Diagram: Thread Life Cycle}

\includegraphics[width=1\linewidth,height=\textheight,keepaspectratio]{mermaid-20370008.pdf}

\begin{itemize}
\tightlist
\item
  \textbf{New}: Thread created but not started
\item
  \textbf{Runnable}: Ready to run when CPU time is given
\item
  \textbf{Running}: Currently executing
\item
  \textbf{Blocked/Waiting}: Temporarily inactive
\item
  \textbf{Terminated}: Completed execution
\end{itemize}

\end{solutionbox}
\begin{mnemonicbox}
``NRRBT: New Runnable Running Blocked Terminated''

\end{mnemonicbox}
\subsection*{Question 4(c) [7 marks]}\label{q4c}

\textbf{Write a program in java that creates the multiple threads by
implementing the Thread class.}

\begin{solutionbox}
Creating threads by implementing Thread class allows
multiple tasks to execute concurrently.

\textbf{Code Block:}

\begin{lstlisting}[language=Java]
class MyThread extends Thread {
    private String threadName;
    
    MyThread(String name) {
        this.threadName = name;
    }
    
    @Override
    public void run() {
        try {
for (int

i = 1; i <= 5; i++) {

                System.out.println(threadName + ": " + i);
                Thread.sleep(500);
            }
        } catch (InterruptedException e) {
            System.out.println(threadName + " interrupted");
        }
        System.out.println(threadName + " completed");
    }
}

public class MultiThreadDemo {
    public static void main(String[] args) {
        MyThread thread1 = new MyThread("Thread-1");
        MyThread thread2 = new MyThread("Thread-2");
        MyThread thread3 = new MyThread("Thread-3");
        
        thread1.start();
        thread2.start();
        thread3.start();
    }
}
\end{lstlisting}

\begin{itemize}
\tightlist
\item
  \textbf{Extend Thread}: Create thread by extending Thread class
\item
  \textbf{Override run()}: Define task in run method
\item
  \textbf{start()}: Begin thread execution
\end{itemize}

\end{solutionbox}
\begin{mnemonicbox}
``ERS: Extend Run Start to create threads''

\end{mnemonicbox}
\subsection*{Question 4(a OR) [3
marks]}\label{question-4a-or-3-marks}

\textbf{Explain final class with suitable example.}

\begin{solutionbox}
A final class cannot be extended, preventing
inheritance and modification of its design.


{\def\LTcaptype{none} % do not increment counter
\vspace{-5pt}
\captionof{table}{Final Class Characteristics}
\vspace{-10pt}
\begin{longtable}[]{@{}ll@{}}
\toprule\noalign{}
Feature & Description \\
\midrule\noalign{}
\endhead
\bottomrule\noalign{}
\endlastfoot
Inheritance & Cannot be subclassed \\
Methods & Implicitly final \\
Security & Prevents design alteration \\
Example & String, Math classes \\
\end{longtable}
}

\textbf{Code Block:}

\begin{lstlisting}[language=Java]
final class Security {
    void secureMethod() {
        System.out.println("Secure implementation");
    }
}

// Error: Cannot extend final class
// class HackAttempt extends Security { }
\end{lstlisting}

\begin{itemize}
\tightlist
\item
  \textbf{Security}: Protects sensitive implementations
\item
  \textbf{Immutability}: Helps create immutable classes
\item
  \textbf{Optimization}: JVM can optimize final classes
\end{itemize}

\end{solutionbox}
\begin{mnemonicbox}
``SIO: Security Immutability Optimization''

\end{mnemonicbox}
\subsection*{Question 4(b OR) [4
marks]}\label{question-4b-or-4-marks}

\textbf{Explain thread priorities with suitable example.}

\begin{solutionbox}
Thread priorities determine the order in which threads
are scheduled for execution, from 1 (lowest) to 10 (highest).


{\def\LTcaptype{none} % do not increment counter
\vspace{-5pt}
\captionof{table}{Thread Priority Constants}
\vspace{-10pt}
\begin{longtable}[]{@{}lll@{}}
\toprule\noalign{}
Constant & Value & Description \\
\midrule\noalign{}
\endhead
\bottomrule\noalign{}
\endlastfoot
MIN\_PRIORITY & 1 & Lowest priority \\
NORM\_PRIORITY & 5 & Default priority \\
MAX\_PRIORITY & 10 & Highest priority \\
\end{longtable}
}

\textbf{Code Block:}

\begin{lstlisting}[language=Java]
class PriorityThread extends Thread {
    PriorityThread(String name) {
        super(name);
    }
    
    public void run() {
        System.out.println("Running: " + getName() + 
                          " with priority: " + getPriority());
    }
}

public class ThreadPriorityDemo {
    public static void main(String[] args) {
        PriorityThread low = new PriorityThread("Low Priority");
        PriorityThread norm = new PriorityThread("Normal Priority");
        PriorityThread high = new PriorityThread("High Priority");
        
        low.setPriority(Thread.MIN_PRIORITY);
        high.setPriority(Thread.MAX_PRIORITY);
        
        low.start();
        norm.start();
        high.start();
    }
}
\end{lstlisting}

\end{solutionbox}
\begin{mnemonicbox}
``HNL: High Normal Low priorities in threads''

\end{mnemonicbox}
\subsection*{Question 4(c OR) [7
marks]}\label{question-4c-or-7-marks}

\textbf{What is Exception? Write a program that shows the use of
Arithmetic Exception.}

\begin{solutionbox}
An exception is an abnormal condition that disrupts the
normal flow of program execution.

\textbf{Diagram: Exception Hierarchy}

\includegraphics[width=1\linewidth,height=\textheight,keepaspectratio]{mermaid-2e8b0f36.pdf}

\textbf{Code Block:}

\begin{lstlisting}[language=Java]
public class ArithmeticExceptionDemo {
    public static void main(String[] args) {
        try {
            // This will cause ArithmeticException
            int result = 100 / 0;
            System.out.println("Result: " + result);
        } 
        catch (ArithmeticException e) {
            System.out.println("ArithmeticException caught: " + e.getMessage());
            System.out.println("Cannot divide by zero");
        }
        finally {
            System.out.println("This block always executes");
        }
        
        System.out.println("Program continues after exception handling");
    }
}
\end{lstlisting}

\begin{itemize}
\tightlist
\item
  \textbf{Try Block}: Contains code that might throw exceptions
\item
  \textbf{Catch Block}: Handles the specific exception
\item
  \textbf{Finally Block}: Always executes regardless of exception
\end{itemize}

\end{solutionbox}
\begin{mnemonicbox}
``TCF: Try Catch Finally handles exceptions''

\end{mnemonicbox}
\subsection*{Question 5(a) [3 marks]}\label{q5a}

\textbf{Write a Java Program to find sum and average of 10 numbers of an
array.}

\begin{solutionbox}
Arrays store multiple values of the same type, enabling
sequential processing of elements.

\textbf{Code Block:}

\begin{lstlisting}[language=Java]
public class ArraySumAverage {
    public static void main(String[] args) {
        int[] numbers = {10, 20, 30, 40, 50, 60, 70, 80, 90, 100};
        
        int sum = 0;
        
        // Calculate sum
        for (int i = 0; i < numbers.length; i++) {
            sum += numbers[i];
        }
        
        // Calculate average
        double average = (double) sum / numbers.length;
        
        System.out.println("Sum = " + sum);
        System.out.println("Average = " + average);
    }
}
\end{lstlisting}

\begin{itemize}
\tightlist
\item
  \textbf{Declaration}: Creates fixed-size collection
\item
  \textbf{Iteration}: Sequential access to elements
\item
  \textbf{Calculation}: Process values for results
\end{itemize}

\end{solutionbox}
\begin{mnemonicbox}
``DIC: Declare Iterate Calculate for array
processing''

\end{mnemonicbox}
\subsection*{Question 5(b) [4 marks]}\label{q5b}

\textbf{Write a Java program to handle user defined exception for
`Divide by Zero' error.}

\begin{solutionbox}
User-defined exceptions allow creating custom exception
types for specific application requirements.

\textbf{Code Block:}

\begin{lstlisting}[language=Java]
// Custom exception class
class DivideByZeroException extends Exception {
    public DivideByZeroException(String message) {
        super(message);
    }
}

public class CustomExceptionDemo {
    // Method that throws custom exception
    static double divide(int numerator, int denominator) throws DivideByZeroException {
        if (denominator == 0) {
            throw new DivideByZeroException("Cannot divide by zero!");
        }
        return (double) numerator / denominator;
    }
    
    public static void main(String[] args) {
        try {
            System.out.println(divide(10, 2));
            System.out.println(divide(20, 0));
        } catch (DivideByZeroException e) {
            System.out.println("Custom exception caught: " + e.getMessage());
        }
    }
}
\end{lstlisting}

\begin{itemize}
\tightlist
\item
  \textbf{Custom Class}: Extends Exception class
\item
  \textbf{Throwing}: Use throw keyword with new instance
\item
  \textbf{Handling}: Catch specific exception type
\end{itemize}

\end{solutionbox}
\begin{mnemonicbox}
``CTE: Create Throw Exception when needed''

\end{mnemonicbox}
\subsection*{Question 5(c) [7 marks]}\label{q5c}

\textbf{Write a java program to create a text file and perform read
operation on the text file.}

\begin{solutionbox}
Java provides I/O classes to work with files, allowing
creation, writing, and reading operations.

\textbf{Code Block:}

\begin{lstlisting}[language=Java]
import java.io.FileWriter;
import java.io.FileReader;
import java.io.IOException;
import java.io.BufferedReader;

public class FileOperationsDemo {
    public static void main(String[] args) {
        try {
            // Create and write to file
            FileWriter writer = new FileWriter("sample.txt");
            writer.write("Hello World!\n");
            writer.write("Welcome to Java File Handling.\n");
            writer.write("This is the third line.");
            writer.close();
            System.out.println("Successfully wrote to the file.");
            
            // Read from file
            FileReader reader = new FileReader("sample.txt");
            BufferedReader buffReader = new BufferedReader(reader);
            
            String line;
            System.out.println("\nFile contents:");
            while ((line = buffReader.readLine()) != null) {
                System.out.println(line);
            }
            
            reader.close();
            
        } catch (IOException e) {
            System.out.println("An error occurred: " + e.getMessage());
        }
    }
}
\end{lstlisting}

\begin{itemize}
\tightlist
\item
  \textbf{FileWriter}: Creates and writes to files
\item
  \textbf{FileReader}: Reads character data from files
\item
  \textbf{BufferedReader}: Efficiently reads text by lines
\end{itemize}

\end{solutionbox}
\begin{mnemonicbox}
``WRC: Write Read Close for file operations''

\end{mnemonicbox}
\subsection*{Question 5(a OR) [3
marks]}\label{question-5a-or-3-marks}

\textbf{Explain java I/O process.}

\begin{solutionbox}
Java I/O process involves transferring data to and from
various sources using streams.


{\def\LTcaptype{none} % do not increment counter
\vspace{-5pt}
\captionof{table}{Java I/O Stream Types}
\vspace{-10pt}
\begin{longtable}[]{@{}ll@{}}
\toprule\noalign{}
Classification & Types \\
\midrule\noalign{}
\endhead
\bottomrule\noalign{}
\endlastfoot
Direction & Input, Output \\
Data Type & Byte Streams, Character Streams \\
Functionality & Basic, Buffered, Data, Object \\
\end{longtable}
}

\textbf{Diagram: Java I/O Hierarchy}

\begin{lstlisting}
   +-----------+
   |  Stream   |
   +-----------+
        |
   +----+----+
   |         |
+-----+   +------+
|Input|   |Output|
+-----+   +------+
   |         |
+------+  +------+
| Byte |  | Char |
+------+  +------+
\end{lstlisting}

\begin{itemize}
\tightlist
\item
  \textbf{Stream}: Sequence of data flowing between source and
  destination
\item
  \textbf{Buffering}: Improves performance by reducing disk access
\end{itemize}

\end{solutionbox}
\begin{mnemonicbox}
``SBI: Stream Buffered Input/Output''

\end{mnemonicbox}
\subsection*{Question 5(b OR) [4
marks]}\label{question-5b-or-4-marks}

\textbf{Explain throw and finally in Exception Handling with example.}

\begin{solutionbox}
Exception handling mechanisms control program flow
during errors, ensuring graceful execution.


{\def\LTcaptype{none} % do not increment counter
\vspace{-5pt}
\captionof{table}{throw vs finally}
\vspace{-10pt}
\begin{longtable}[]{@{}lll@{}}
\toprule\noalign{}
Feature & throw & finally \\
\midrule\noalign{}
\endhead
\bottomrule\noalign{}
\endlastfoot
Purpose & Explicitly throws exception & Ensures code execution \\
Placement & Inside method & After try-catch blocks \\
Execution & When condition met & Always, even with return \\
Usage & Control flow & Resource cleanup \\
\end{longtable}
}

\textbf{Code Block:}

\begin{lstlisting}[language=Java]
public class ThrowFinallyDemo {
    public static void validateAge(int age) {
        try {
            if (age < 18) {
                throw new ArithmeticException("Not eligible to vote");
            } else {
                System.out.println("Welcome to vote");
            }
        } catch (ArithmeticException e) {
            System.out.println("Exception caught: " + e.getMessage());
        } finally {
            System.out.println("Validation process completed");
        }
    }
    
    public static void main(String[] args) {
        validateAge(15);
        System.out.println("---------");
        validateAge(20);
    }
}
\end{lstlisting}

\end{solutionbox}
\begin{mnemonicbox}
``TERA: Throw Exception Regardless Always finally
executes''

\end{mnemonicbox}
\subsection*{Question 5(c OR) [7
marks]}\label{question-5c-or-7-marks}

\textbf{Write a java program to display the content of a text file and
perform append operation on the text file.}

\textbf{Code Block:}

\begin{lstlisting}[language=Java]
import java.io.*;

public class FileAppendDemo {
    public static void main(String[] args) {
        try {
            // Create initial file
            FileWriter writer = new FileWriter("example.txt");
            writer.write("Original content line 1\n");
            writer.write("Original content line 2\n");
            writer.close();
            
            // Display file content
            System.out.println("Original file content:");
            readFile("example.txt");
            
            // Append to file
            FileWriter appendWriter = new FileWriter("example.txt", true);
            appendWriter.write("Appended content line 1\n");
            appendWriter.write("Appended content line 2\n");
            appendWriter.close();
            
            // Display updated content
            System.out.println("\nFile content after append:");
            readFile("example.txt");
            
        } catch (IOException e) {
            System.out.println("An error occurred: " + e.getMessage());
        }
    }
    
    // Method to read and display file content
    public static void readFile(String fileName) {
        try {
            BufferedReader reader = new BufferedReader(new FileReader(fileName));
            String line;
            while ((line = reader.readLine()) != null) {
                System.out.println(line);
            }
            reader.close();
        } catch (IOException e) {
            System.out.println("Error reading file: " + e.getMessage());
        }
    }
}
\end{lstlisting}

\begin{itemize}
\tightlist
\item
  \textbf{FileWriter(file, true)}: Second parameter enables append mode
\item
  \textbf{BufferedReader}: Efficiently reads text by lines
\item
  \textbf{Reusable Method}: Encapsulates reading functionality
\end{itemize}

\begin{mnemonicbox}
``CAD: Create Append Display file operations''

\end{mnemonicbox}

\end{document}
