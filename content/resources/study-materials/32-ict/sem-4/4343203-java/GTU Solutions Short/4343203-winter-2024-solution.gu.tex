\documentclass{article}

% content/resources/templates/preamble.tex
\usepackage[margin=0.6in]{geometry}
\author{Milav Dabgar}
\usepackage{amsmath,amssymb,amsthm}
\usepackage{booktabs}
\usepackage{multirow}
\usepackage{xcolor}
\usepackage{tcolorbox}
\tcbuselibrary{breakable,skins}
\usepackage[colorlinks=true,linkcolor=blue]{hyperref}
\usepackage{titlesec}
\usepackage{enumitem}
\usepackage{tikz}
\usepackage{pgfplots}
\usepackage{circuitikz}
\usepackage[version=4]{mhchem}
\usepackage{longtable}
\usepackage{array}
\usepackage{float}
\usepackage{caption}
\usepackage{listings}

\lstset{
  basicstyle=\small\ttfamily,
  breaklines=true,
  breakatwhitespace=false,
  postbreak=\mbox{\textcolor{red}{$\hookrightarrow$}\space},
  float=false,
  numbers=left,
  numberstyle=\tiny\color{gray},
  numbersep=10pt,
  xleftmargin=2em,
  keywordstyle=\color{blue},
  commentstyle=\color{green!60!black},
  stringstyle=\color{purple},
  backgroundcolor=\color{gray!5},
  showstringspaces=false,
  tabsize=2,
  captionpos=b,
  keepspaces=true,
  columns=flexible
}

\pgfplotsset{compat=1.18}
\usetikzlibrary{shapes,arrows,positioning,calc,patterns,decorations.pathmorphing,decorations.markings,arrows.meta}

% Color scheme
\definecolor{headcolor}{RGB}{0,102,204}
\definecolor{keycolor}{RGB}{220,20,60}
\definecolor{solutioncolor}{RGB}{34,139,34}
\definecolor{mnemoniccolor}{RGB}{148,0,211}
\definecolor{codecolor}{RGB}{0,0,100}

% Spacing
\setlength{\parskip}{3pt}
\setlist[itemize]{nosep}
\setlist[enumerate]{nosep}

% Title formatting
\titleformat{\section}{\Large\bfseries\color{headcolor}}{\thesection}{1em}{}
\titleformat{\subsection}{\large\bfseries\color{headcolor}}{\thesubsection}{1em}{}

% Pandoc tightlist compatibility
\providecommand{\tightlist}{%
  \setlength{\itemsep}{0pt}\setlength{\parskip}{0pt}}

% Pandoc longtable compatibility
\newcounter{none}
\def\thenone{}


% content/resources/templates/gujarati-boxes.tex
\usepackage{fontspec}
\usepackage{polyglossia}

% Set Gujarati as main language (document is primarily in Gujarati)
% Note: gloss-gujarati.ldf doesn't exist in polyglossia, but it will use hyphenation patterns
\setdefaultlanguage{gujarati}
\setotherlanguage{english}

% Configure Gujarati font properly
% Use Language=Default to prevent polyglossia from trying to add language-specific features
% that don't exist for Gujarati, which causes "empty feature" warnings
\newfontfamily\gujaratifont[Script=Gujarati,AutoFakeBold=2.5,AutoFakeSlant=0.3]{Noto Sans Gujarati}
\setmainfont[Script=Gujarati,AutoFakeBold=2.5,AutoFakeSlant=0.3]{Noto Sans Gujarati}
% Use Noto Sans Gujarati for monospace to support Gujarati in text
\setmonofont[Scale=0.9]{Noto Sans Gujarati}

% Configure English to use the same font
\newfontfamily\englishfont[Script=Gujarati,AutoFakeBold=2.5,AutoFakeSlant=0.3]{Noto Sans Gujarati}

% Translations for polyglossia
\gappto\captionsgujarati{
  \renewcommand{\tablename}{કોષ્ટક}
  \renewcommand{\figurename}{આકૃતિ}
}

% Helper for TikZ nodes to ensure Gujarati font
\newcommand{\gu}[1]{{\gujaratifont #1}}

% Custom environments
\newtcolorbox{solutionbox}{
    breakable,
    enhanced,
    colback=solutioncolor!5!white,
    colframe=solutioncolor!75!black,
    fonttitle=\bfseries,
    title=જવાબ
}

\newtcolorbox{solutionboxnobreak}{
 colback=solutioncolor!5!white,
 colframe=solutioncolor!75!black,
 fonttitle=\bfseries,
 title=જવાબ
}

\newtcolorbox{keyformula}{
 breakable,
 enhanced,
 colback=keycolor!5!white,
 colframe=keycolor!75!black,
 fonttitle=\bfseries,
 title=રાસાયણિક સમીકરણ/સૂત્ર
}

\newtcolorbox{mnemonicbox}{
 breakable,
 enhanced,
 colback=mnemoniccolor!5!white,
 colframe=mnemoniccolor!75!black,
 fonttitle=\bfseries,
 title=મેમરી ટ્રીક
}


% Custom commands for GTU solutions
% This file defines semantic commands for consistent formatting

% Question command with automatic formatting
\newcommand{\question}[2]{%
  \section*{Question #1}%
  \textbf{#2}%
}

% OR question variant
\newcommand{\questionor}[2]{%
  \section*{Question #1 OR}%
  \textbf{#2}%
}

% Proper table environment with caption
\newenvironment{answertable}[1]{%
  \begin{table}[htbp]
  \centering
  \caption{#1}
}{%
  \end{table}
}

% Proper figure environment for diagrams
\newenvironment{answerdiagram}[1]{%
  \begin{figure}[htbp]
  \centering
  \caption{#1}
}{%
  \end{figure}
}

% Semantic markup for key terms
\newcommand{\keyword}[1]{\textbf{#1}}
\newcommand{\code}[1]{\texttt{#1}}
\newcommand{\classname}[1]{\texttt{#1}}
\newcommand{\methodname}[1]{\texttt{#1}}

% Proper quotation marks
\newcommand{\mnemonic}[1]{``#1''}


\title{જાવા પ્રોગ્રામિંગ (4343203) – વિન્ટર 2024 સોલ્યુશન}
\date{November 22, 2024}

\begin{document}
\maketitle

\questionmarks{1(અ)}{3}{Java ના વિવિધ પ્રકારના Primitive data typeની યાદી આપો.}

\begin{solutionbox}
Java મેમરીમાં સીધા સાદા મૂલ્યો સંગ્રહિત કરવા માટે આઠ primitive data types આપે છે.

\begin{center}
\captionof{table}{Java Primitive Data Types}
\begin{tabulary}{\linewidth}{|L|L|L|L|}
\hline
\textbf{ડેટા પ્રકાર} & \textbf{સાઈઝ} & \textbf{વર્ણન} & \textbf{રેન્જ} \\ \hline
byte & 8 બિટ્સ & પૂર્ણાંક પ્રકાર & -128 થી 127 \\ \hline
short & 16 બિટ્સ & પૂર્ણાંક પ્રકાર & -32,768 થી 32,767 \\ \hline
int & 32 બિટ્સ & પૂર્ણાંક પ્રકાર & -$2^{31}$ થી $2^{31}-1$ \\ \hline
long & 64 બિટ્સ & પૂર્ણાંક પ્રકાર & -$2^{63}$ થી $2^{63}-1$ \\ \hline
float & 32 બિટ્સ & ફ્લોટિંગ-પોઇન્ટ & સિંગલ પ્રિસિઝન \\ \hline
double & 64 બિટ્સ & ફ્લોટિંગ-પોઇન્ટ & ડબલ પ્રિસિઝન \\ \hline
char & 16 બિટ્સ & અક્ષર & યુનિકોડ અક્ષરો \\ \hline
boolean & 1 બિટ & લોજિકલ & true અથવા false \\ \hline
\end{tabulary}
\end{center}
\end{solutionbox}

\begin{mnemonicbox}
\mnemonic{BILFDC-B: Byte Int Long Float Double Char Boolean પ્રકારો}
\end{mnemonicbox}

\questionmarks{1(બ)}{4}{યોગ્ય ઉદાહરણ સાથે Java Programનું સ્ટ્રક્ચર સમજાવો.}

\begin{solutionbox}
Java પ્રોગ્રામનું સ્ટ્રક્ચર package ડેક્લેરેશન, imports, ક્લાસ ડેફિનિશન, અને મેથોડ્સ સાથે ચોક્કસ સંગઠનને અનુસરે છે.

\begin{center}
\begin{tikzpicture}[
    box/.style={draw, rectangle, minimum width=4cm, minimum height=1cm, align=left, fill=white}
]
    \node[gtu container, minimum height=4cm] (outer) {};
    \node[anchor=north, font=\bfseries] at (outer.north) {Source File};
    
    \node[box, below=0.5cm of outer.north] (top) {
        Documentation\\
        \code{package} statement\\
        \code{import} statements
    };
    
    \node[box, below=0.2cm of top, minimum height=2cm] (class) {
        \textbf{Class declaration}\\
        - Variables\\
        - Constructors\\
        - Methods
    };
\end{tikzpicture}
\captionof{figure}{જાવા પ્રોગ્રામ સ્ટ્રક્ચર}
\end{center}

\begin{lstlisting}[language=Java, caption={Java Program Structure Example}]
// Documentation comment
/**
 * Simple program to demonstrate Java structure
 * @author GTU Student
 */

// Package declaration
package com.example;

// Import statements
import java.util.Scanner;

// Class declaration
public class HelloWorld {
    // Variable declaration
    private String message;
    
    // Constructor
    public HelloWorld() {
        message = "Hello, World!";
    }
    
    // Method
    public void displayMessage() {
        System.out.println(message);
    }
    
    // Main method
    public static void main(String[] args) {
        HelloWorld obj = new HelloWorld();
        obj.displayMessage();
    }
}
\end{lstlisting}
\end{solutionbox}

\begin{mnemonicbox}
\mnemonic{PICOM: Package Import Class Objects Methods ક્રમમાં}
\end{mnemonicbox}

\questionmarks{1(ક)}{7}{Java ના arithmetic operatorsની યાદી આપો. કોઈ પણ ત્રણ arithmetic operatorsનો ઉપયોગ કરીને Java Program વિકસાવો અને તેનું output બતાવો.}

\begin{solutionbox}
Java માં arithmetic operators સંખ્યાત્મક મૂલ્યો પર ગાણિતિક કાર્યો કરે છે.

\begin{center}
\captionof{table}{Java Arithmetic Operators}
\begin{tabulary}{\linewidth}{|L|L|L|}
\hline
\textbf{ઓપરેટર} & \textbf{વર્ણન} & \textbf{ઉદાહરણ} \\ \hline
+ & સરવાળો & a + b \\ \hline
- & બાદબાકી & a - b \\ \hline
* & ગુણાકાર & a * b \\ \hline
/ & ભાગાકાર & a / b \\ \hline
\% & મોડ્યુલસ (શેષ) & a \% b \\ \hline
++ & ઇન્ક્રિમેન્ટ & a++ અથવા ++a \\ \hline
-- & ડિક્રિમેન્ટ & a-- અથવા --a \\ \hline
\end{tabulary}
\end{center}

\begin{lstlisting}[language=Java, caption={Arithmetic Operations}]
public class ArithmeticDemo {
    public static void main(String[] args) {
        int a = 10;
        int b = 3;
        
        // સરવાળો
        int sum = a + b;
        
        // ગુણાકાર
        int product = a * b;
        
        // મોડ્યુલસ
        int remainder = a % b;
        
        // પરિણામો દર્શાવો
        System.out.println("Values: a = " + a + ", b = " + b);
        System.out.println("Addition (a + b): " + sum);
        System.out.println("Multiplication (a * b): " + product);
        System.out.println("Modulus (a % b): " + remainder);
    }
}
\end{lstlisting}

\textbf{આઉટપુટ:}
\begin{lstlisting}
Values: a = 10, b = 3
Addition (a + b): 13
Multiplication (a * b): 30
Modulus (a % b): 1
\end{lstlisting}
\end{solutionbox}

\begin{mnemonicbox}
\mnemonic{SAME: સરવાળો, Addition, Multiply, Exponentiation મૂળભૂત ઓપરેશન્સ}
\end{mnemonicbox}

\questionmarks{1(ક OR)}{7}{Javaમાં for લૂપ માટેની સિન્ટેક્ષ લખો. ૧ થી ૧૦ વચ્ચે આવતા પ્રાઈમ નંબર શોધવા માટેનો java કોડ વિકસાવો.}

\begin{solutionbox}
Java માં for લૂપ મૂલ્યોની શ્રેણી પર પુનરાવર્તન માટે કૉમ્પેક્ટ રીત પ્રદાન કરે છે.

\textbf{Java for લૂપની સિન્ટેક્ષ:}
\begin{lstlisting}[language=Java]
for (initialization; condition; increment/decrement) {
    // statements to be executed
}
\end{lstlisting}

\begin{lstlisting}[language=Java, caption={Prime Numbers}]
public class PrimeNumbers {
    public static void main(String[] args) {
        System.out.println("Prime numbers between 1 and 10:");
        
        // 1 થી 10 સુધીની દરેક સંખ્યા તપાસો
        for (int num = 1; num <= 10; num++) {
            boolean isPrime = true;
            
            // num આ 2 થી num-1 સુધીની કોઈપણ સંખ્યાથી વિભાજ્ય છે કે નહીં તપાસો
            if (num > 1) {
                for (int i = 2; i < num; i++) {
                    if (num % i == 0) {
                        isPrime = false;
                        break;
                    }
                }
                
                // પ્રાઇમ હોય તો પ્રિન્ટ કરો
                if (isPrime) {
                    System.out.print(num + " ");
                }
            }
        }
    }
}
\end{lstlisting}

\textbf{આઉટપુટ:}
\begin{lstlisting}
Prime numbers between 1 and 10:
2 3 5 7
\end{lstlisting}
\end{solutionbox}

\begin{mnemonicbox}
\mnemonic{ICE: Initialize, Check, Execute for લૂપના પગલાઓ}
\end{mnemonicbox}

\questionmarks{2(અ)}{3}{Procedure-Oriented Programming (POP) અને Object-Oriented Programming (OOP) ના તફાવતોની યાદી આપો.}

\begin{solutionbox}
Procedure-Oriented અને Object-Oriented Programming મૂળભૂત રીતે અલગ પ્રોગ્રામિંગ પેરાડાઇમ્સનું પ્રતિનિધિત્વ કરે છે.

\begin{center}
\captionof{table}{POP vs OOP}
\begin{tabulary}{\linewidth}{|L|L|L|}
\hline
\textbf{ફીચર} & \textbf{Procedure-Oriented} & \textbf{Object-Oriented} \\ \hline
ફોકસ & ફંક્શન્સ/પ્રોસીજર્સ & ઓબ્જેક્ટ્સ \\ \hline
ડેટા & ફંક્શન્સથી અલગ & ઓબ્જેક્ટ્સમાં એન્કેપ્સ્યુલેટેડ \\ \hline
સુરક્ષા & ઓછી સુરક્ષિત & એક્સેસ કંટ્રોલ સાથે વધુ સુરક્ષિત \\ \hline
વારસો & સપોર્ટ નથી & સપોર્ટ કરે છે \\ \hline
રીયુઝેબિલિટી & ઓછી રીયુઝેબલ & ખૂબ રીયુઝેબલ \\ \hline
જટિલતા & નાના પ્રોગ્રામ માટે સરળ & જટિલ સિસ્ટમ માટે વધુ સારું \\ \hline
\end{tabulary}
\end{center}

\begin{itemize}
    \item \textbf{સંગઠન}: POP ફંક્શન્સમાં વિભાજિત કરે છે; OOP ઓબ્જેક્ટ્સમાં જૂથ બનાવે છે
    \item \textbf{અભિગમ}: POP ટોપ-ડાઉન અનુસરે છે; OOP બોટમ-અપ અનુસરે છે
\end{itemize}
\end{solutionbox}

\begin{mnemonicbox}
\mnemonic{FIOS: Functions In Objects Structure મુખ્ય તફાવત}
\end{mnemonicbox}

\questionmarks{2(બ)}{4}{યોગ્ય ઉદાહરણ સાથે static કીવર્ડ સમજાવો.}

\begin{solutionbox}
Java માં static કીવર્ડ તે ક્લાસના બધા ઓબ્જેક્ટ્સ વચ્ચે શેર થતા ક્લાસ-લેવલ મેમ્બર્સ બનાવે છે.

\begin{center}
\captionof{table}{static કીવર્ડના ઉપયોગો}
\begin{tabulary}{\linewidth}{|L|L|L|}
\hline
\textbf{ઉપયોગ} & \textbf{હેતુ} & \textbf{ઉદાહરણ} \\ \hline
static variable & બધા ઓબ્જેક્ટ્સ વચ્ચે શેર થાય છે & \code{static int count;} \\ \hline
static method & ઓબ્જેક્ટ વગર કૉલ કરી શકાય છે & \code{static void display()} \\ \hline
static block & ક્લાસ લોડ થાય ત્યારે એક્ઝિક્યુટ થાય છે & \code{static \{ // code \}} \\ \hline
static nested class & આઉટર ક્લાસ સાથે જોડાયેલ & \code{static class Inner \{\}} \\ \hline
\end{tabulary}
\end{center}

\begin{lstlisting}[language=Java, caption={Static Keyword Demo}]
public class Counter {
    // બધા ઓબ્જેક્ટ્સ દ્વારા શેર કરેલ Static variable
    static int count = 0;
    
    // દરેક ઓબ્જેક્ટ માટે અનન્ય Instance variable
    int instanceCount = 0;
    
    // Constructor
    Counter() {
        count++;         // શેર કરેલ કાઉન્ટને વધારે છે
        instanceCount++; // આ ઓબ્જેક્ટના કાઉન્ટને વધારે છે
    }
    
    public static void main(String[] args) {
        Counter c1 = new Counter();
        Counter c2 = new Counter();
        Counter c3 = new Counter();
        
        System.out.println("Static count: " + Counter.count);
        System.out.println("c1's instance count: " + c1.instanceCount);
        System.out.println("c2's instance count: " + c2.instanceCount);
        System.out.println("c3's instance count: " + c3.instanceCount);
    }
}
\end{lstlisting}

\textbf{આઉટપુટ:}
\begin{lstlisting}
Static count: 3
c1's instance count: 1
c2's instance count: 1
c3's instance count: 1
\end{lstlisting}
\end{solutionbox}

\begin{mnemonicbox}
\mnemonic{CBMS: Class-level, Before objects, Memory single, Shared by all}
\end{mnemonicbox}

\questionmarks{2(ક)}{7}{Constructorની વ્યાખ્યા આપો. Constructorના વિવિધ પ્રકારોની યાદી આપો. Parameterized constructor સમજાવવા માટેનો java code વિકસાવો.}

\begin{solutionbox}
Constructor એ વિશેષ મેથડ છે જેનું નામ તેના ક્લાસ સાથે સમાન હોય છે, જેનો ઉપયોગ ઓબ્જેક્ટ્સ બનાવતી વખતે તેમને પ્રારંભિક મૂલ્ય આપવા માટે થાય છે.

\textbf{Constructor ના પ્રકારો:}

\begin{center}
\captionof{table}{Java માં Constructor ના પ્રકારો}
\begin{tabulary}{\linewidth}{|L|L|L|}
\hline
\textbf{પ્રકાર} & \textbf{વર્ણન} & \textbf{ઉદાહરણ} \\ \hline
Default & કોઈ પેરામીટર નહીં, કમ્પાઇલર દ્વારા બનાવાયેલ & \code{Student() \{\}} \\ \hline
No-arg & સ્પષ્ટપણે વ્યાખ્યાયિત, પેરામીટર નહીં & \code{Student() \{ name = "Unknown"; \}} \\ \hline
Parameterized & પેરામીટર સ્વીકારે છે & \code{Student(String n) \{ name = n; \}} \\ \hline
Copy & બીજા ઓબ્જેક્ટથી ઓબ્જેક્ટ બનાવે & \code{Student(Student s) \{ name = s.name; \}} \\ \hline
\end{tabulary}
\end{center}

\begin{lstlisting}[language=Java, caption={Parameterized Constructor Example}]
public class Student {
    // Instance variables
    private String name;
    private int age;
    private String course;
    
    // Parameterized constructor
    public Student(String name, int age, String course) {
        this.name = name;
        this.age = age;
        this.course = course;
    }
    
    // વિદ્યાર્થીની વિગતો દર્શાવવા માટેની મેથડ
    public void displayDetails() {
        System.out.println("Student Details:");
        System.out.println("Name: " + name);
        System.out.println("Age: " + age);
        System.out.println("Course: " + course);
    }
    
    // ડેમોન્સ્ટ્રેશન માટે main મેથડ
    public static void main(String[] args) {
        // Parameterized constructor નો ઉપયોગ કરીને ઓબ્જેક્ટ બનાવવું
        Student student1 = new Student("John", 20, "Computer Science");
        student1.displayDetails();
        
        // બીજો વિદ્યાર્થી
        Student student2 = new Student("Lisa", 22, "Engineering");
        student2.displayDetails();
    }
}
\end{lstlisting}

\textbf{આઉટપુટ:}
\begin{lstlisting}
Student Details:
Name: John
Age: 20
Course: Computer Science
Student Details:
Name: Lisa
Age: 22
Course: Engineering
\end{lstlisting}
\end{solutionbox}

\begin{mnemonicbox}
\mnemonic{IDCR: Initialize Data Create Ready ઓબ્જેક્ટ્સ}
\end{mnemonicbox}

\questionmarks{2(અ OR)}{3}{java મા મૂળભૂત OOP conceptsની યાદી આપો અને કોઈ પણ એક સમજાવો.}

\begin{solutionbox}
Java વિવિધ મૂળભૂત કન્સેપ્ટ્સ દ્વારા Object-Oriented Programming નો અમલ કરે છે.

\begin{center}
\captionof{table}{Java માં મૂળભૂત OOP Concepts}
\begin{tabulary}{\linewidth}{|L|L|}
\hline
\textbf{Concept} & \textbf{વર્ણન} \\ \hline
Encapsulation & ડેટા અને મેથડ્સને એક સાથે બાંધવા \\ \hline
Inheritance & હાલના ક્લાસથી નવા ક્લાસ બનાવવા \\ \hline
Polymorphism & એક ઈન્ટરફેસ, વિવિધ અમલીકરણો \\ \hline
Abstraction & અમલીકરણની વિગતો છુપાવવી \\ \hline
Association & ઓબ્જેક્ટ્સ વચ્ચે સંબંધ \\ \hline
\end{tabulary}
\end{center}

\textbf{Encapsulation ઉદાહરણ:}

\begin{lstlisting}[language=Java, caption={Encapsulation Demo}]
public class Person {
    // Private data - બહારથી છુપાયેલ
    private String name;
    private int age;
    
    // Public methods - ડેટા ઍક્સેસ કરવા માટેનો ઇન્ટરફેસ
    public void setName(String name) {
        this.name = name;
    }
    
    public String getName() {
        return name;
    }
    
    public void setAge(int age) {
        // માન્યતા ડેટા અખંડિતતા સુનિશ્ચિત કરે છે
        if (age > 0 && age < 120) {
            this.age = age;
        } else {
            System.out.println("Invalid age");
        }
    }
    
    public int getAge() {
        return age;
    }
}
\end{lstlisting}

\begin{itemize}
    \item \textbf{ડેટા છુપાવવું}: Private variables બહારથી અપ્રાપ્ય
    \item \textbf{નિયંત્રિત ઍક્સેસ}: Public methods (getters/setters) દ્વારા
    \item \textbf{અખંડિતતા}: ડેટા માન્યતા યોગ્ય મૂલ્યો સુનિશ્ચિત કરે છે
\end{itemize}
\end{solutionbox}

\begin{mnemonicbox}
\mnemonic{EIPA: Encapsulate Inherit Polymorphize Abstract}
\end{mnemonicbox}

\questionmarks{2(બ OR)}{4}{યોગ્ય ઉદાહરણ સાથે final કીવર્ડ સમજાવો.}

\begin{solutionbox}
Java માં final કીવર્ડ એન્ટિટીઓમાં ફેરફારોને મર્યાદિત કરે છે, કોન્સ્ટન્ટ્સ, અપરિવર્તનીય મેથડ્સ, અને નોન-ઇન્હેરિટેબલ ક્લાસ બનાવે છે.

\begin{center}
\captionof{table}{final કીવર્ડના ઉપયોગો}
\begin{tabulary}{\linewidth}{|L|L|L|}
\hline
\textbf{ઉપયોગ} & \textbf{અસર} & \textbf{ઉદાહરણ} \\ \hline
final variable & સુધારી શકાતું નથી & \code{final int MAX = 100;} \\ \hline
final method & ઓવરરાઇડ કરી શકાતી નથી & \code{final void display() \{\}} \\ \hline
final class & વિસ્તૃત કરી શકાતો નથી & \code{final class Math \{\}} \\ \hline
final parameter & મેથડમાં બદલી શકાતા નથી & \code{void method(final int x) \{\}} \\ \hline
\end{tabulary}
\end{center}

\begin{lstlisting}[language=Java, caption={Final Keyword Demo}]
public class FinalDemo {
    // Final variable (constant)
    final int MAX_SPEED = 120;
    
    // Final method ઓવરરાઇડ કરી શકાતી નથી
    final void showLimit() {
        System.out.println("Speed limit: " + MAX_SPEED);
    }
    
    public static void main(String[] args) {
        FinalDemo car = new FinalDemo();
        car.showLimit();
        
        // આ કમ્પાઇલ એરર કરશે:
        // car.MAX_SPEED = 150;
    }
}

// Final class વિસ્તૃત કરી શકાતો નથી
final class MathUtil {
    public int square(int num) {
        return num * num;
    }
}

// આ કમ્પાઇલ એરર કરશે:
// class AdvancedMath extends MathUtil { }
\end{lstlisting}

\textbf{આઉટપુટ:}
\begin{lstlisting}
Speed limit: 120
\end{lstlisting}
\end{solutionbox}

\begin{mnemonicbox}
\mnemonic{VMP: Variables Methods Permanence with final}
\end{mnemonicbox}

\questionmarks{2(ક OR)}{7}{java access modifierમાટેનો scope લખો. public modifier સમજાવવા માટેનો java code વિકસાવો.}

\begin{solutionbox}
Java માં access modifiers ક્લાસ, મેથડ્સ, અને વેરિએબલ્સની દૃશ્યતા અને ઍક્સેસિબિલિટીને નિયંત્રિત કરે છે.

\begin{center}
\captionof{table}{Java Access Modifier Scope}
\begin{tabulary}{\linewidth}{|L|C|C|C|C|}
\hline
\textbf{Modifier} & \textbf{Class} & \textbf{Package} & \textbf{Subclass} & \textbf{World} \\ \hline
private & \checkmark{} & $\times$ & $\times$ & $\times$ \\ \hline
default (no modifier) & \checkmark{} & \checkmark{} & $\times$ & $\times$ \\ \hline
protected & \checkmark{} & \checkmark{} & \checkmark{} & $\times$ \\ \hline
public & \checkmark{} & \checkmark{} & \checkmark{} & \checkmark{} \\ \hline
\end{tabulary}
\end{center}

\begin{lstlisting}[language=Java, caption={Public Modifier Demo}]
// ફાઇલ: PublicDemo.java
package com.example;

// Public class બધે ઍક્સેસિબલ છે
public class PublicDemo {
    // Public variable બધે ઍક્સેસિબલ છે
    public String message = "Hello, World!";
    
    // Public method બધે ઍક્સેસિબલ છે
    public void displayMessage() {
        System.out.println(message);
    }
}

// ફાઇલ: Main.java
package com.test;

// અલગ પેકેજમાંથી import કરવું
import com.example.PublicDemo;

public class Main {
    public static void main(String[] args) {
        // અલગ પેકેજના ક્લાસનો ઓબ્જેક્ટ બનાવવો
        PublicDemo demo = new PublicDemo();
        
        // અલગ પેકેજમાંથી public variable ઍક્સેસ કરવો
        System.out.println("Message: " + demo.message);
        
        // અલગ પેકેજમાંથી public method કૉલ કરવી
        demo.displayMessage();
        
        // અલગ પેકેજમાંથી public variable સુધારવો
        demo.message = "Modified message";
        demo.displayMessage();
    }
}
\end{lstlisting}

\textbf{આઉટપુટ:}
\begin{lstlisting}
Message: Hello, World!
Hello, World!
Modified message
\end{lstlisting}
\end{solutionbox}

\begin{mnemonicbox}
\mnemonic{CEPM: Class Everywhere Public Most accessible}
\end{mnemonicbox}

\questionmarks{3(અ)}{3}{વિવિધ પ્રકારના inheritance ની યાદી આપો અને કોઈ પણ એક ઉદાહરણ સાથે સમજાવો.}

\begin{solutionbox}
Inheritance એક ક્લાસને બીજા ક્લાસમાંથી attributes અને behaviors વારસામાં લેવાની ક્ષમતા આપે છે.

\begin{center}
\captionof{table}{Java માં Inheritance ના પ્રકારો}
\begin{tabulary}{\linewidth}{|L|L|}
\hline
\textbf{પ્રકાર} & \textbf{વર્ણન} \\ \hline
Single & એક ક્લાસ એક ક્લાસને extends કરે છે \\ \hline
Multilevel & Inheritance ની સાંકળ (A→B→C) \\ \hline
Hierarchical & ઘણા ક્લાસ એક ક્લાસને extends કરે છે \\ \hline
Multiple & એક ક્લાસ ઘણા ક્લાસમાંથી વારસો મેળવે છે (ઇન્ટરફેસ દ્વારા) \\ \hline
Hybrid & ઘણા inheritance પ્રકારોનું સંયોજન \\ \hline
\end{tabulary}
\end{center}

\textbf{Single Inheritance ઉદાહરણ:}

\begin{lstlisting}[language=Java, caption={Single Inheritance Demo}]
// પેરેન્ટ ક્લાસ
class Animal {
    protected String name;
    
    public Animal(String name) {
        this.name = name;
    }
    
    public void eat() {
        System.out.println(name + " is eating");
    }
}

// Animal માંથી વારસો મેળવતો ચાઇલ્ડ ક્લાસ
class Dog extends Animal {
    private String breed;
    
    public Dog(String name, String breed) {
        super(name);  // પેરેન્ટ કન્સ્ટ્રક્ટર કૉલ કરો
        this.breed = breed;
    }
    
    public void bark() {
        System.out.println(name + " is barking");
    }
    
    public void displayInfo() {
        System.out.println("Name: " + name);
        System.out.println("Breed: " + breed);
    }
}

// મુખ્ય ક્લાસ
public class InheritanceDemo {
    public static void main(String[] args) {
        Dog dog = new Dog("Max", "Labrador");
        dog.displayInfo();
        dog.eat();     // વારસામાં મળેલી મેથડ
        dog.bark();    // પોતાની મેથડ
    }
}
\end{lstlisting}

\textbf{આઉટપુટ:}
\begin{lstlisting}
Name: Max
Breed: Labrador
Max is eating
Max is barking
\end{lstlisting}
\end{solutionbox}

\begin{mnemonicbox}
\mnemonic{SMHMH: Single Multilevel Hierarchical Multiple Hybrid પ્રકારો}
\end{mnemonicbox}

\questionmarks{3(બ)}{4}{કોઈ પણ બે String buffer class methods યોગ્ય ઉદાહરણ સાથે સમજાવો.}

\begin{solutionbox}
StringBuffer અક્ષરોનો બદલી શકાય તેવો ક્રમ છે જેનો ઉપયોગ સ્ટ્રિંગ્સને મોડિફાય કરવા માટે થાય છે, વિવિધ હેરફેર મેથડ્સ ઓફર કરે છે.

\begin{center}
\captionof{table}{બે StringBuffer મેથડ્સ}
\begin{tabulary}{\linewidth}{|L|L|L|}
\hline
\textbf{મેથડ} & \textbf{હેતુ} & \textbf{સિન્ટેક્સ} \\ \hline
append() & અંતે સ્ટ્રિંગ ઉમેરે છે & \code{sb.append(String str)} \\ \hline
insert() & નિર્દિષ્ટ સ્થાને સ્ટ્રિંગ ઉમેરે છે & \code{sb.insert(int offset, String str)} \\ \hline
\end{tabulary}
\end{center}

\begin{lstlisting}[language=Java, caption={StringBuffer Methods Demo}]
public class StringBufferMethodsDemo {
    public static void main(String[] args) {
        // StringBuffer બનાવો
        StringBuffer sb = new StringBuffer("Hello");
        System.out.println("Original: " + sb);
        
        // append() મેથડ - અંતે ટેક્સ્ટ ઉમેરે છે
        sb.append(" World");
        System.out.println("After append(): " + sb);
        
        // વિવિધ ડેટા પ્રકારો append કરી શકે છે
        sb.append('!');
        sb.append(2024);
        System.out.println("After appending more: " + sb);
        
        // ડેમોન્સ્ટ્રેશન માટે રીસેટ
        sb = new StringBuffer("Java");
        System.out.println("\nNew Original: " + sb);
        
        // insert() મેથડ - નિર્દિષ્ટ સ્થાને ટેક્સ્ટ ઉમેરે છે
        sb.insert(0, "Learn ");
        System.out.println("After insert() at beginning: " + sb);
        
        sb.insert(10, " Programming");
        System.out.println("After insert() in middle: " + sb);
    }
}
\end{lstlisting}

\textbf{આઉટપુટ:}
\begin{lstlisting}
Original: Hello
After append(): Hello World
After appending more: Hello World!2024

New Original: Java
After insert() at beginning: Learn Java
After insert() in middle: Learn Java Programming
\end{lstlisting}
\end{solutionbox}

\begin{mnemonicbox}
\mnemonic{AIMS: Append Insert Modify StringBuffer}
\end{mnemonicbox}

\questionmarks{3(ક)}{7}{Interfaceની વ્યાખ્યા આપો. Interfaceની મદદથી multiple inheritance નો java program લખો.}

\begin{solutionbox}
Interface એક કરાર છે જે એવી મેથડ્સ ઘોષિત કરે છે જે એક ક્લાસ અમલ કરવા માટે જરૂરી છે, જે Java માં multiple inheritance શક્ય બનાવે છે.

\textbf{વ્યાખ્યા:} Interface એ એક રેફરન્સ પ્રકાર છે જેમાં માત્ર કોન્સ્ટન્ટ્સ, મેથડ સિગ્નેચર્સ, ડિફોલ્ટ મેથડ્સ, સ્ટેટિક મેથડ્સ, અને નેસ્ટેડ પ્રકારો સમાવિષ્ટ છે, જેમાં abstract મેથડ્સ માટે કોઈ અમલીકરણ નથી.

\begin{center}
\begin{tikzpicture}[node distance=2cm]
    \node [gtu interface] (P) {Printable};
    \node [gtu interface, right=of P] (S) {Scannable};
    \node [gtu class, below=1.5cm of $(P)!0.5!(S)$] (D) {Device};

    \path [gtu dashed arrow] (D) -- (P);
    \path [gtu dashed arrow] (D) -- (S);
\end{tikzpicture}
\captionof{figure}{Interfaces નો ઉપયોગ કરીને Multiple Inheritance}
\end{center}

\begin{lstlisting}[language=Java, caption={Interface Multiple Inheritance}]
// પ્રથમ ઇન્ટરફેસ
interface Printable {
    void print();
}

// બીજું ઇન્ટરફેસ
interface Scannable {
    void scan();
}

// બંને ઇન્ટરફેસને અમલ કરતો ક્લાસ
class Device implements Printable, Scannable {
    private String model;
    
    public Device(String model) {
        this.model = model;
    }
    
    // Printable માંથી print() મેથડનું અમલીકરણ
    @Override
    public void print() {
        System.out.println(model + " is printing a document");
    }
    
    // Scannable માંથી scan() મેથડનું અમલીકરણ
    @Override
    public void scan() {
        System.out.println(model + " is scanning a document");
    }
    
    // ક્લાસની પોતાની મેથડ
    public void getModel() {
        System.out.println("Device Model: " + model);
    }
}

// મુખ્ય ક્લાસ
public class MultipleInheritanceDemo {
    public static void main(String[] args) {
        Device device = new Device("HP LaserJet");
        
        // મોડેલ દર્શાવો
        device.getModel();
        
        // ઘણા ઇન્ટરફેસની મેથડ્સનો ઉપયોગ
        device.print();
        device.scan();
        
        // ડિવાઇસ ઇન્ટરફેસનો ઇન્સ્ટન્સ છે કે નહીં તપાસો
        System.out.println("Is device Printable? " + (device instanceof Printable));
        System.out.println("Is device Scannable? " + (device instanceof Scannable));
    }
}
\end{lstlisting}

\textbf{આઉટપુટ:}
\begin{lstlisting}
Device Model: HP LaserJet
HP LaserJet is printing a document
HP LaserJet is scanning a document
Is device Printable? true
Is device Scannable? true
\end{lstlisting}
\end{solutionbox}

\begin{mnemonicbox}
\mnemonic{IMAC: Interface Multiple Abstract Contract}
\end{mnemonicbox}

\questionmarks{3(અ OR)}{3}{Abstract class અને Interface નો તફાવત આપો.}

\begin{solutionbox}
Abstract class અને interface બંને abstraction માટે વપરાય છે પરંતુ ઘણા મહત્વપૂર્ણ પાસાઓમાં અલગ પડે છે.

\begin{center}
\captionof{table}{Abstract Class vs Interface}
\begin{tabulary}{\linewidth}{|L|L|L|}
\hline
\textbf{ફીચર} & \textbf{Abstract Class} & \textbf{Interface} \\ \hline
કીવર્ડ & abstract & interface \\ \hline
મેથડ્સ & abstract અને concrete બંને & Abstract (અને Java 8થી default) \\ \hline
વેરિએબલ્સ & કોઈપણ પ્રકાર & માત્ર public static final \\ \hline
કન્સ્ટ્રક્ટર & ધરાવે છે & ધરાવતું નથી \\ \hline
વારસો & સિંગલ & મલ્ટિપલ \\ \hline
એક્સેસ મોડિફાયર્સ & કોઈપણ & માત્ર public \\ \hline
હેતુ & આંશિક અમલીકરણ & સંપૂર્ણ abstraction \\ \hline
\end{tabulary}
\end{center}

\begin{itemize}
    \item \textbf{અમલીકરણ}: Abstract class આંશિક અમલીકરણ પ્રદાન કરી શકે છે; interface પરંપરાગત રીતે કોઈ નહીં
    \item \textbf{સંબંધ}: Abstract class કહે છે "is-a"; interface કહે છે "can-do-this"
\end{itemize}
\end{solutionbox}

\begin{mnemonicbox}
\mnemonic{MAPS: Methods Access Purpose Single vs multiple}
\end{mnemonicbox}

\questionmarks{3(બ OR)}{4}{કોઈ પણ બે String class methods યોગ્ય ઉદાહરણ સાથે સમજાવો.}

\begin{solutionbox}
String ક્લાસ સ્ટ્રિંગ મેનિપ્યુલેશન, કમ્પેરિઝન અને ટ્રાન્સફોર્મેશન માટે વિવિધ મેથડ્સ આપે છે.

\begin{center}
\captionof{table}{બે String મેથડ્સ}
\begin{tabulary}{\linewidth}{|L|L|L|}
\hline
\textbf{મેથડ} & \textbf{હેતુ} & \textbf{સિન્ટેક્સ} \\ \hline
substring() & સ્ટ્રિંગનો ભાગ કાઢે છે & \code{str.substring(int beginIndex, int endIndex)} \\ \hline
equals() & સ્ટ્રિંગ કન્ટેન્ટની તુલના કરે છે & \code{str1.equals(str2)} \\ \hline
\end{tabulary}
\end{center}

\begin{lstlisting}[language=Java, caption={String Methods Demo}]
public class StringMethodsDemo {
    public static void main(String[] args) {
        String message = "Java Programming";
        
        // substring() મેથડ
        String sub1 = message.substring(0, 4);
        System.out.println("substring(0, 4): " + sub1);
        
        String sub2 = message.substring(5);
        System.out.println("substring(5): " + sub2);
        
        // equals() મેથડ
        String str1 = "Hello";
        String str2 = "Hello";
        String str3 = "hello";
        String str4 = new String("Hello");
        
        System.out.println("\nComparing strings with equals():");
        System.out.println("str1.equals(str2): " + str1.equals(str2));
        System.out.println("str1.equals(str3): " + str1.equals(str3));
        System.out.println("str1.equals(str4): " + str1.equals(str4));
        
        System.out.println("\nComparing strings with ==:");
        System.out.println("str1 == str2: " + (str1 == str2));
        System.out.println("str1 == str4: " + (str1 == str4));
    }
}
\end{lstlisting}

\textbf{આઉટપુટ:}
\begin{lstlisting}
substring(0, 4): Java
substring(5): Programming

Comparing strings with equals():
str1.equals(str2): true
str1.equals(str3): false
str1.equals(str4): true

Comparing strings with ==:
str1 == str2: true
str1 == str4: false
\end{lstlisting}
\end{solutionbox}

\begin{mnemonicbox}
\mnemonic{SEC: Substring Equals Compare સ્ટ્રિંગ કન્ટેન્ટ}
\end{mnemonicbox}

\questionmarks{3(ક OR)}{7}{Package સમજાવો અને package create કરવા માટેના સ્ટેપ્સની યાદી બનાવો.}

\begin{solutionbox}
Java માં package એ નેમસ્પેસ છે જે સંબંધિત ક્લાસ અને ઇન્ટરફેસને સંગઠિત કરે છે, નામકરણ સંઘર્ષોને અટકાવે છે.

\textbf{Package બનાવવાના પગલાં:}

\begin{center}
\captionof{table}{Package બનાવવાના પગલાં}
\begin{tabulary}{\linewidth}{|C|L|}
\hline
\textbf{પગલું} & \textbf{ક્રિયા} \\ \hline
1 & સોર્સ ફાઇલોની ટોચે package નામ ઘોષિત કરો \\ \hline
2 & package નામને મેચ કરતું ડિરેક્ટરી સ્ટ્રક્ચર બનાવો \\ \hline
3 & Java ફાઇલને યોગ્ય ડિરેક્ટરીમાં સેવ કરો \\ \hline
4 & javac -d વિકલ્પ સાથે package ડિરેક્ટરી બનાવવા માટે કમ્પાઇલ કરો \\ \hline
5 & ફુલી ક્વોલિફાઇડ નામથી પ્રોગ્રામ ચલાવો \\ \hline
\end{tabulary}
\end{center}

\begin{lstlisting}[language=Java, caption={Package Creation Example}]
// પગલું 1: ટોચે package ઘોષિત કરો (Calculator.java તરીકે સેવ કરો)
package com.example.math;

public class Calculator {
    public int add(int a, int b) {
        return a + b;
    }
    
    public int subtract(int a, int b) {
        return a - b;
    }
}

// પગલું 1: package ઘોષિત કરો (CalculatorApp.java તરીકે સેવ કરો)
package com.example.app;

import com.example.math.Calculator;

public class CalculatorApp {
    public static void main(String[] args) {
        Calculator calc = new Calculator();
        System.out.println("Addition: " + calc.add(10, 5));
        System.out.println("Subtraction: " + calc.subtract(10, 5));
    }
}
\end{lstlisting}

\textbf{ટર્મિનલ કમાન્ડ્સ:}
\begin{lstlisting}
// પગલું 4: -d વિકલ્પ સાથે કમ્પાઇલ કરો
javac -d . com/example/math/Calculator.java
javac -d . -cp . com/example/app/CalculatorApp.java

// પગલું 5: ફુલી ક્વોલિફાઇડ નામથી ચલાવો
java com.example.app.CalculatorApp
\end{lstlisting}

\textbf{આઉટપુટ:}
\begin{lstlisting}
Addition: 15
Subtraction: 5
\end{lstlisting}
\end{solutionbox}

\begin{mnemonicbox}
\mnemonic{DISCO: Declare Import Save Compile Organize}
\end{mnemonicbox}

\questionmarks{4(અ)}{3}{java માં errorના પ્રકારોની યાદી આપો.}

\begin{solutionbox}
Java પ્રોગ્રામ્સ ડેવલપમેન્ટ અને એક્ઝિક્યુશન દરમિયાન વિવિધ errors નો સામનો કરી શકે છે.

\begin{center}
\captionof{table}{Java માં Errors ના પ્રકારો}
\begin{tabulary}{\linewidth}{|L|L|L|}
\hline
\textbf{Error પ્રકાર} & \textbf{ક્યારે થાય છે} & \textbf{ઉદાહરણ} \\ \hline
Compile-time Errors & કમ્પાઇલેશન દરમિયાન & Syntax errors, type errors \\ \hline
Runtime Errors & એક્ઝિક્યુશન દરમિયાન & NullPointerException, ArrayIndexOutOfBoundsException \\ \hline
Logical Errors & ખોટા આઉટપુટ સાથે એક્ઝિક્યુશન દરમિયાન & ખોટી ગણતરી, અનંત લૂપ \\ \hline
Linkage Errors & ક્લાસ લોડિંગ દરમિયાન & NoClassDefFoundError \\ \hline
Thread Death & જ્યારે થ્રેડ સમાપ્ત થાય & ThreadDeath \\ \hline
\end{tabulary}
\end{center}

\begin{itemize}
    \item \textbf{Syntax Errors}: સેમિકોલોન, બ્રેકેટ્સની ગેરહાજરી, અથવા ટાઇપો
    \item \textbf{Semantic Errors}: ટાઇપ મિસમેચિસ, અસંગત ઓપરેશન્સ
    \item \textbf{Exceptions}: હેન્ડલિંગની જરૂર પડતી રનટાઇમ સમસ્યાઓ
\end{itemize}
\end{solutionbox}

\begin{mnemonicbox}
\mnemonic{CRLLT: Compile Runtime Logical Linkage Thread errors}
\end{mnemonicbox}

\questionmarks{4(બ)}{4}{try catch block યોગ્ય ઉદાહરણ સાથે સમજાવો.}

\begin{solutionbox}
Java માં try-catch બ્લોકએ exceptions ને હેન્ડલ કરે છે, જેનાથી ભૂલો હોવા છતાં પ્રોગ્રામ્સને ચાલુ રાખવાની મંજૂરી મળે છે.

\begin{lstlisting}[language=Java, caption={Try-Catch Demo}]
public class TryCatchDemo {
    public static void main(String[] args) {
        int[] numbers = {10, 20, 30};
        
        try {
            // એરે બાઉન્ડ્સની બહાર એલિમેન્ટને ઍક્સેસ કરવાનો પ્રયાસ
            System.out.println("Trying to access element 5: " + numbers[4]);
            
            // જો exception થાય તો આ કોડ એક્ઝિક્યુટ નહીં થાય
            System.out.println("This won't be printed");
        } 
        catch (ArrayIndexOutOfBoundsException e) {
            // ચોક્કસ exception ને હેન્ડલ કરો
            System.out.println("Exception caught: " + e.getMessage());
            System.out.println("Array index out of bounds");
        }
        catch (Exception e) {
            // અન્ય exceptions ને હેન્ડલ કરો
            System.out.println("General exception caught: " + e.getMessage());
        }
        finally {
            // આ બ્લોક હંમેશા એક્ઝિક્યુટ થાય છે
            System.out.println("Finally block executed");
        }
        
        // પ્રોગ્રામ એક્ઝિક્યુશન ચાલુ રાખે છે
        System.out.println("Program continues after exception handling");
    }
}
\end{lstlisting}

\textbf{આઉટપુટ:}
\begin{lstlisting}
Exception caught: Index 4 out of bounds for length 3
Array index out of bounds
Finally block executed
Program continues after exception handling
\end{lstlisting}
\end{solutionbox}

\begin{mnemonicbox}
\mnemonic{TCFE: Try Catch Finally Execute ભૂલો હોવા છતાં}
\end{mnemonicbox}

\questionmarks{4(ક)}{7}{method overloading અને overriding વચ્ચેના ચાર તફાવત આપો. method overriding સમજાવવા માટેનો java program લખો.}

\begin{solutionbox}
Method overloading અને overriding બંને polymorphism ના પ્રકારો છે પરંતુ ફંક્શનાલિટી અને અમલીકરણમાં અલગ પડે છે.

\begin{center}
\captionof{table}{Method Overloading vs Overriding}
\begin{tabulary}{\linewidth}{|L|L|L|}
\hline
\textbf{ફીચર} & \textbf{Method Overloading} & \textbf{Method Overriding} \\ \hline
ઉદ્ભવ & એક જ ક્લાસમાં & પેરન્ટ અને ચાઇલ્ડ ક્લાસમાં \\ \hline
પેરામીટર્સ & અલગ પેરામીટર્સ & સમાન પેરામીટર્સ \\ \hline
રિટર્ન ટાઇપ & અલગ હોઈ શકે & સમાન અથવા સબટાઇપ હોવી જોઈએ \\ \hline
Access Modifier & અલગ હોઈ શકે & વધુ પ્રતિબંધિત ન હોઈ શકે \\ \hline
બાઇન્ડિંગ & કમ્પાઇલ-ટાઇમ (સ્ટેટિક) & રનટાઇમ (ડાયનેમિક) \\ \hline
હેતુ & એક મેથડના ઘણા વર્તન & વિશેષ અમલીકરણ \\ \hline
ઇન્હેરિટન્સ & જરૂરી નથી & જરૂરી છે \\ \hline
@Override & વપરાતું નથી & ભલામણ કરાય છે \\ \hline
\end{tabulary}
\end{center}

\begin{lstlisting}[language=Java, caption={Method Overriding Demo}]
// પેરન્ટ ક્લાસ
class Animal {
    public void makeSound() {
        System.out.println("Animal makes a sound");
    }
    
    public void eat() {
        System.out.println("Animal eats food");
    }
}

// મેથડ્સ ઓવરરાઇડ કરતો ચાઇલ્ડ ક્લાસ
class Dog extends Animal {
    @Override
    public void makeSound() {
        System.out.println("Dog barks");
    }
    
    @Override
    public void eat() {
        System.out.println("Dog eats meat");
    }
}

// બીજો ચાઇલ્ડ ક્લાસ
class Cat extends Animal {
    @Override
    public void makeSound() {
        System.out.println("Cat meows");
    }
}

public class MethodOverridingDemo {
    public static void main(String[] args) {
        Animal animal = new Animal();
        Animal dog = new Dog();
        Animal cat = new Cat();
        
        System.out.println("Animal behavior:");
        animal.makeSound();
        animal.eat();
        
        System.out.println("\nDog behavior:");
        dog.makeSound();
        dog.eat();
        
        System.out.println("\nCat behavior:");
        cat.makeSound();
        cat.eat();
    }
}
\end{lstlisting}

\textbf{આઉટપુટ:}
\begin{lstlisting}
Animal behavior:
Animal makes a sound
Animal eats food

Dog behavior:
Dog barks
Dog eats meat

Cat behavior:
Cat meows
Animal eats food
\end{lstlisting}
\end{solutionbox}

\begin{mnemonicbox}
\mnemonic{SBRE: Same-name, Base-derived, Runtime-resolution, Extend functionality}
\end{mnemonicbox}

\questionmarks{4(અ OR)}{3}{કોઈ પણ ચાર inbuilt exceptions ની યાદી આપો.}

\begin{solutionbox}
Java ઘણા બિલ્ટ-ઇન exception ક્લાસ પ્રદાન કરે છે જે વિવિધ ભૂલની સ્થિતિઓનું પ્રતિનિધિત્વ કરે છે.

\begin{center}
\captionof{table}{ચાર સામાન્ય Inbuilt Exceptions}
\begin{tabulary}{\linewidth}{|L|L|L|}
\hline
\textbf{Exception} & \textbf{કારણ} & \textbf{Package} \\ \hline
NullPointerException & null રેફરન્સને ઍક્સેસ/મોડિફાય & java.lang \\ \hline
ArrayIndexOutOfBoundsException & અમાન્ય એરે ઇન્ડેક્સ & java.lang \\ \hline
ArithmeticException & અમાન્ય ગાણિતિક ઓપરેશન (શૂન્ય વડે ભાગાકાર) & java.lang \\ \hline
ClassCastException & અમાન્ય ક્લાસ કાસ્ટિંગ & java.lang \\ \hline
\end{tabulary}
\end{center}

\begin{itemize}
    \item \textbf{Unchecked}: Runtime exceptions (સ્પષ્ટ હેન્ડલિંગની જરૂર નથી)
    \item \textbf{Hierarchy}: બધા Exception ક્લાસમાંથી extends થાય છે
    \item \textbf{Handling}: try-catch બ્લોક્સથી પકડી શકાય છે
\end{itemize}
\end{solutionbox}

\begin{mnemonicbox}
\mnemonic{NAAC: Null Array Arithmetic Cast સામાન્ય exceptions}
\end{mnemonicbox}

\questionmarks{4(બ OR)}{4}{યોગ્ય ઉદાહરણ સાથે "throw" કીવર્ડ સમજાવો.}

\begin{solutionbox}
Java માં throw કીવર્ડ પ્રોગ્રામ્સમાં અસાધારણ સ્થિતિઓ માટે મેન્યુઅલી exceptions જનરેટ કરે છે.

\begin{center}
\captionof{table}{throw કીવર્ડના ઉપયોગો}
\begin{tabulary}{\linewidth}{|L|L|}
\hline
\textbf{ઉપયોગ} & \textbf{હેતુ} \\ \hline
throw new ExceptionType() & Exception બનાવવી અને ફેંકવી \\ \hline
throw new ExceptionType(message) & કસ્ટમ મેસેજ સાથે બનાવવી \\ \hline
throws in method signature & મેથડ કઈ exception ફેંકી શકે છે તે ઘોષિત કરવું \\ \hline
checked/unchecked ફેંકી શકે & checked exceptions માટે try-catch જરૂરી \\ \hline
\end{tabulary}
\end{center}

\begin{lstlisting}[language=Java, caption={Throw Keyword Demo}]
public class ThrowDemo {
    public static void validateAge(int age) {
        if (age < 0) {
            throw new IllegalArgumentException("Age cannot be negative");
        }
        
        if (age < 18) {
            throw new ArithmeticException("Not eligible to vote");
        } else {
            System.out.println("Eligible to vote");
        }
    }
    
    public static void main(String[] args) {
        try {
            System.out.println("Validating age 20:");
            validateAge(20);
            
            System.out.println("\nValidating age 15:");
            validateAge(15);
        } catch (ArithmeticException e) {
            System.out.println("ArithmeticException: " + e.getMessage());
        } catch (IllegalArgumentException e) {
            System.out.println("IllegalArgumentException: " + e.getMessage());
        }
        
        try {
            System.out.println("\nValidating age -5:");
            validateAge(-5);
        } catch (Exception e) {
            System.out.println("Exception: " + e.getMessage());
        }
    }
}
\end{lstlisting}

\textbf{આઉટપુટ:}
\begin{lstlisting}
Validating age 20:
Eligible to vote

Validating age 15:
ArithmeticException: Not eligible to vote

Validating age -5:
Exception: Age cannot be negative
\end{lstlisting}
\end{solutionbox}

\begin{mnemonicbox}
\mnemonic{CET: Create Exception Throw error handling માટે}
\end{mnemonicbox}

\questionmarks{4(ક OR)}{7}{'this' કીવર્ડ 'Super' કીવર્ડ સાથે સરખાવો. યોગ્ય ઉદાહરણ સાથે super કીવર્ડ સમજાવો.}

\begin{solutionbox}
'this' અને 'super' કીવર્ડ Java માં રેફરન્સિંગ માટે વપરાય છે, અલગ-અલગ હેતુઓ અને વર્તન સાથે.

\begin{center}
\captionof{table}{this vs super કીવર્ડ સરખામણી}
\begin{tabulary}{\linewidth}{|L|L|L|}
\hline
\textbf{ફીચર} & \textbf{this કીવર્ડ} & \textbf{super કીવર્ડ} \\ \hline
રેફરન્સ & વર્તમાન ક્લાસ & પેરન્ટ ક્લાસ \\ \hline
ઉપયોગ & વર્તમાન ક્લાસ મેમ્બર્સ ઍક્સેસ કરવા & પેરન્ટ ક્લાસ મેમ્બર્સ ઍક્સેસ કરવા \\ \hline
કન્સ્ટ્રક્ટર કૉલ & this() & super() \\ \hline
વેરિએબલ રેઝોલ્યુશન & this.var (વર્તમાન ક્લાસ) & super.var (પેરન્ટ ક્લાસ) \\ \hline
મેથડ ઇન્વોકેશન & this.method() (વર્તમાન ક્લાસ) & super.method() (પેરન્ટ ક્લાસ) \\ \hline
પોઝિશન & કન્સ્ટ્રક્ટરમાં પ્રથમ સ્ટેટમેન્ટ & કન્સ્ટ્રક્ટરમાં પ્રથમ સ્ટેટમેન્ટ \\ \hline
ઇન્હેરિટન્સ & ઇન્હેરિટન્સ સાથે સંબંધિત નથી & ઇન્હેરિટન્સ સાથે વપરાય છે \\ \hline
\end{tabulary}
\end{center}

\begin{lstlisting}[language=Java, caption={Super Keyword Demo}]
class Vehicle {
    protected String brand = "Ford";
    protected String color = "Red";
    
    Vehicle() {
        System.out.println("Vehicle constructor called");
    }
    
    void displayInfo() {
        System.out.println("Brand: " + brand);
        System.out.println("Color: " + color);
    }
}

class Car extends Vehicle {
    private String brand = "Toyota";
    private String color = "Blue";
    
    Car() {
        super();
        System.out.println("Car constructor called");
    }
    
    void printDetails() {
        System.out.println("Car brand (this): " + this.brand);
        System.out.println("Car color (this): " + this.color);
        System.out.println("Vehicle brand (super): " + super.brand);
        System.out.println("Vehicle color (super): " + super.color);
    }
    
    @Override
    void displayInfo() {
        System.out.println("Car information:");
        super.displayInfo();
        System.out.println("Model: Corolla");
    }
}

public class SuperKeywordDemo {
    public static void main(String[] args) {
        Car myCar = new Car();
        
        System.out.println("\nVariable access with this and super:");
        myCar.printDetails();
        
        System.out.println("\nMethod call with super:");
        myCar.displayInfo();
    }
}
\end{lstlisting}

\textbf{આઉટપુટ:}
\begin{lstlisting}
Vehicle constructor called
Car constructor called

Variable access with this and super:
Car brand (this): Toyota
Car color (this): Blue
Vehicle brand (super): Ford
Vehicle color (super): Red

Method call with super:
Car information:
Brand: Ford
Color: Red
Model: Corolla
\end{lstlisting}
\end{solutionbox}

\begin{mnemonicbox}
\mnemonic{PCIM: Parent Class Inheritance Members with super}
\end{mnemonicbox}

\questionmarks{5(અ)}{3}{વિવિધ Stream Classes ની યાદી આપો.}

\begin{solutionbox}
Java I/O ઇનપુટ અને આઉટપુટ ઓપરેશન્સ માટે વિવિધ સ્ટ્રીમ ક્લાસ પ્રદાન કરે છે.

\begin{center}
\captionof{table}{Java Stream Classes}
\begin{tabulary}{\linewidth}{|L|L|}
\hline
\textbf{કેટેગરી} & \textbf{સ્ટ્રીમ ક્લાસ} \\ \hline
Byte Streams & FileInputStream, FileOutputStream, BufferedInputStream, BufferedOutputStream \\ \hline
Character Streams & FileReader, FileWriter, BufferedReader, BufferedWriter \\ \hline
Data Streams & DataInputStream, DataOutputStream \\ \hline
Object Streams & ObjectInputStream, ObjectOutputStream \\ \hline
Print Streams & PrintStream, PrintWriter \\ \hline
\end{tabulary}
\end{center}

\begin{itemize}
    \item \textbf{Byte Streams}: બાઇનરી ડેટા (8-બિટ બાઇટ્સ) સાથે કામ કરે છે
    \item \textbf{Character Streams}: અક્ષરો (16-બિટ યુનિકોડ) સાથે કામ કરે છે
    \item \textbf{Buffered Streams}: બફરિંગ દ્વારા પરફોર્મન્સ સુધારે છે
\end{itemize}
\end{solutionbox}

\begin{mnemonicbox}
\mnemonic{BCDOP: Byte Character Data Object Print streams}
\end{mnemonicbox}

\questionmarks{5(બ)}{4}{'Divide by Zero' એરર માટે યુઝર ડીફાઇન એક્સેપ્સન હેન્ડલ કરવા માટે જાવા પ્રોગ્રામ લખો.}

\begin{solutionbox}
યુઝર-ડિફાઈન્ડ exceptions એપ્લિકેશન-સ્પેસિફિક ભૂલની સ્થિતિઓ માટે કસ્ટમ exception પ્રકારો બનાવવાની મંજૂરી આપે છે.

\begin{lstlisting}[language=Java, caption={Custom Exception Demo}]
// ડિવાઇડ બાય ઝીરો માટે કસ્ટમ exception
class DivideByZeroException extends Exception {
    public DivideByZeroException() {
        super("Cannot divide by zero");
    }
    
    public DivideByZeroException(String message) {
        super(message);
    }
}

public class CustomExceptionDemo {
    public static double divide(int numerator, int denominator) throws DivideByZeroException {
        if (denominator == 0) {
            throw new DivideByZeroException("Division by zero not allowed");
        }
        return (double) numerator / denominator;
    }
    
    public static void main(String[] args) {
        try {
            System.out.println("10 / 2 = " + divide(10, 2));
            System.out.println("10 / 0 = " + divide(10, 0));
        } catch (DivideByZeroException e) {
            System.out.println("Error: " + e.getMessage());
            System.out.println("Custom exception stack trace:");
            e.printStackTrace();
        }
        
        System.out.println("Program continues execution...");
    }
}
\end{lstlisting}

\textbf{આઉટપુટ:}
\begin{lstlisting}
10 / 2 = 5.0
Error: Division by zero not allowed
Custom exception stack trace:
DivideByZeroException: Division by zero not allowed
    at CustomExceptionDemo.divide(CustomExceptionDemo.java:19)
    at CustomExceptionDemo.main(CustomExceptionDemo.java:29)
Program continues execution...
\end{lstlisting}
\end{solutionbox}

\begin{mnemonicbox}
\mnemonic{ETC: Extend Throw Catch custom exceptions}
\end{mnemonicbox}

\questionmarks{5(ક)}{7}{જાવામાં એક પ્રોગ્રામ લખો જે બાઈટ બાય બાઈટ ફાઈલના કન્ટેન્ટ વાંચે અને તેને બીજી ફાઈલ માં કોપી કરે.}

\begin{solutionbox}
Java માં ફાઇલ I/O ઓપરેશન્સ ફાઇલ્સ માંથી વાંચવા અને લખવાની મંજૂરી આપે છે, બાઇટ સ્ટ્રીમ્સ બાઇનરી ડેટાને હેન્ડલ કરે છે.

\begin{lstlisting}[language=Java, caption={File Copy Byte by Byte}]
import java.io.FileInputStream;
import java.io.FileOutputStream;
import java.io.IOException;

public class FileCopyByteByByte {
    public static void main(String[] args) {
        String sourceFile = "source.txt";
        String destinationFile = "destination.txt";
        
        FileInputStream inputStream = null;
        FileOutputStream outputStream = null;
        
        try {
            inputStream = new FileInputStream(sourceFile);
            outputStream = new FileOutputStream(destinationFile);
            
            System.out.println("Copying file " + sourceFile + " to " + destinationFile);
            
            int byteData;
            int byteCount = 0;
            
            while ((byteData = inputStream.read()) != -1) {
                outputStream.write(byteData);
                byteCount++;
            }
            
            System.out.println("File copied successfully!");
            System.out.println("Total bytes copied: " + byteCount);
            
        } catch (IOException e) {
            System.out.println("Error during file copy: " + e.getMessage());
            e.printStackTrace();
        } finally {
            try {
                if (inputStream != null) {
                    inputStream.close();
                }
                if (outputStream != null) {
                    outputStream.close();
                }
                System.out.println("File streams closed successfully");
            } catch (IOException e) {
                System.out.println("Error closing streams: " + e.getMessage());
            }
        }
    }
}
\end{lstlisting}

\textbf{આઉટપુટ:}
\begin{lstlisting}
Copying file source.txt to destination.txt
File copied successfully!
Total bytes copied: 82
File streams closed successfully
\end{lstlisting}
\end{solutionbox}

\begin{mnemonicbox}
\mnemonic{CROW: Create Read Open Write file operations}
\end{mnemonicbox}

\questionmarks{5(અ OR)}{3}{javaના વિવિધ file operationsની યાદી આપો.}

\begin{solutionbox}
Java વિવિધ ફાઇલ ઓપરેશન્સ દ્વારા વ્યાપક ફાઇલ હેન્ડલિંગ ક્ષમતાઓ પ્રદાન કરે છે.

\begin{center}
\captionof{table}{Java માં File Operations}
\begin{tabulary}{\linewidth}{|L|L|L|}
\hline
\textbf{ઓપરેશન} & \textbf{વર્ણન} & \textbf{વપરાતા ક્લાસ} \\ \hline
File Creation & નવી ફાઇલ્સ બનાવવી & File, FileOutputStream, FileWriter \\ \hline
File Reading & ફાઇલ્સમાંથી વાંચવું & FileInputStream, FileReader, Scanner \\ \hline
File Writing & ફાઇલ્સમાં લખવું & FileOutputStream, FileWriter, PrintWriter \\ \hline
File Deletion & ફાઇલ્સ ડિલીટ કરવી & File.delete() \\ \hline
File Information & ફાઇલ મેટાડેટા મેળવવા & File methods (length, isFile, વગેરે) \\ \hline
Directory Operations & ડિરેક્ટરીઓ બનાવવી/લિસ્ટ કરવી & File methods (mkdir, list, વગેરે) \\ \hline
File Copy & ફાઇલ કન્ટેન્ટ કોપી કરવા & FileInputStream with FileOutputStream \\ \hline
File Renaming & ફાઇલ્સનું નામ બદલવું અથવા ખસેડવી & File.renameTo() \\ \hline
\end{tabulary}
\end{center}

\begin{itemize}
    \item \textbf{Stream-based}: લો-લેવલ બાઇટ અથવા કેરેક્ટર સ્ટ્રીમ્સ
    \item \textbf{Reader/Writer}: કેરેક્ટર-ઓરિએન્ટેડ ફાઇલ ઓપરેશન્સ
    \item \textbf{NIO Package}: એન્હાન્સ્ડ ફાઇલ ઓપરેશન્સ (Java 7થી)
\end{itemize}
\end{solutionbox}

\begin{mnemonicbox}
\mnemonic{CRWD: Create Read Write Delete મૂળભૂત ઓપરેશન્સ}
\end{mnemonicbox}

\questionmarks{5(બ OR)}{4}{એક્સેપ્સન હેન્ડલિંગ માં finally block સમજાવતો જાવા પ્રોગ્રામ લખો.}

\begin{solutionbox}
Exception હેન્ડલિંગમાં finally બ્લોક છે કે exception થાય કે ન થાય, કોડ એક્ઝિક્યુશન સુનિશ્ચિત કરે છે.

\begin{lstlisting}[language=Java, caption={Finally Block Demo}]
import java.io.FileInputStream;
import java.io.FileNotFoundException;
import java.io.IOException;

public class FinallyBlockDemo {
    public static void main(String[] args) {
        // ઉદાહરણ 1: કોઈ exception વગર finally
        System.out.println("Example 1: No exception");
        try {
            int result = 10 / 5;
            System.out.println("Result: " + result);
        } catch (ArithmeticException e) {
            System.out.println("Arithmetic exception caught: " + e.getMessage());
        } finally {
            System.out.println("Finally block executed - Example 1");
        }
        
        // ઉદાહરણ 2: catch થયેલા exception સાથે finally
        System.out.println("\nExample 2: Exception caught");
        try {
            int result = 10 / 0;
            System.out.println("This won't be printed");
        } catch (ArithmeticException e) {
            System.out.println("Arithmetic exception caught: " + e.getMessage());
        } finally {
            System.out.println("Finally block executed - Example 2");
        }
        
        // ઉદાહરણ 3: રિસોર્સ મેનેજમેન્ટ સાથે finally
        System.out.println("\nExample 3: Resource management");
        FileInputStream file = null;
        try {
            file = new FileInputStream("nonexistent.txt");
            System.out.println("File opened successfully");
        } catch (FileNotFoundException e) {
            System.out.println("File not found: " + e.getMessage());
        } finally {
            try {
                if (file != null) {
                    file.close();
                }
                System.out.println("File resource closed in finally block");
            } catch (IOException e) {
                System.out.println("Error closing file: " + e.getMessage());
            }
        }
        
        System.out.println("\nProgram continues execution...");
    }
}
\end{lstlisting}

\textbf{આઉટપુટ:}
\begin{lstlisting}
Example 1: No exception
Result: 2
Finally block executed - Example 1

Example 2: Exception caught
Arithmetic exception caught: / by zero
Finally block executed - Example 2

Example 3: Resource management
File not found: nonexistent.txt (No such file or directory)
File resource closed in finally block

Program continues execution...
\end{lstlisting}
\end{solutionbox}

\begin{mnemonicbox}
\mnemonic{ACRE: Always Cleanup Resources Executes}
\end{mnemonicbox}

\questionmarks{5(ક OR)}{7}{ફાઈલ ક્રિએટ કરવા અને તેમાં લખવા માટેનો જાવા પ્રોગ્રામ લખો.}

\begin{solutionbox}
Java કેરેક્ટર અથવા બાઇટ સ્ટ્રીમ્સનો ઉપયોગ કરીને ફાઇલ્સ બનાવવા અને તેમાં ડેટા લખવા માટે ઘણી રીતો પ્રદાન કરે છે.

\begin{lstlisting}[language=Java, caption={File Write Demo}]
import java.io.File;
import java.io.FileWriter;
import java.io.IOException;
import java.io.BufferedWriter;
import java.text.SimpleDateFormat;
import java.util.Date;
import java.util.Scanner;

public class FileWriteDemo {
    public static void main(String[] args) {
        Scanner scanner = null;
        FileWriter fileWriter = null;
        BufferedWriter bufferedWriter = null;
        
        try {
            File myFile = new File("sample_data.txt");
            
            if (myFile.exists()) {
                System.out.println("File already exists: " + myFile.getName());
                System.out.println("File path: " + myFile.getAbsolutePath());
                System.out.println("File size: " + myFile.length() + " bytes");
            } else {
                if (myFile.createNewFile()) {
                    System.out.println("File created successfully: " + myFile.getName());
                } else {
                    System.out.println("Failed to create file");
                    return;
                }
            }
            
            fileWriter = new FileWriter(myFile);
            bufferedWriter = new BufferedWriter(fileWriter);
            
            SimpleDateFormat formatter = new SimpleDateFormat("dd/MM/yyyy HH:mm:ss");
            Date date = new Date();
            
            bufferedWriter.write("==== File Write Demonstration ====");
            bufferedWriter.newLine();
            bufferedWriter.write("Created on: " + formatter.format(date));
            bufferedWriter.newLine();
            
            scanner = new Scanner(System.in);
            System.out.println("\nEnter text to write to file (type 'exit' to finish):");
            
            String line;
            while (true) {
                line = scanner.nextLine();
                if (line.equalsIgnoreCase("exit")) {
                    break;
                }
                bufferedWriter.write(line);
                bufferedWriter.newLine();
            }
            
            System.out.println("\nFile write operation completed successfully!");
            
        } catch (IOException e) {
            System.out.println("Error occurred: " + e.getMessage());
            e.printStackTrace();
        } finally {
            try {
                if (bufferedWriter != null) {
                    bufferedWriter.close();
                }
                if (fileWriter != null) {
                    fileWriter.close();
                }
                if (scanner != null) {
                    scanner.close();
                }
            } catch (IOException e) {
                System.out.println("Error closing resources: " + e.getMessage());
            }
        }
    }
}
\end{lstlisting}

\textbf{ઉદાહરણ આઉટપુટ:}
\begin{lstlisting}
File created successfully: sample_data.txt

Enter text to write to file (type 'exit' to finish):
This is line 1 of my file.
This is line 2 with some Java content.
Here is line 3 with more text.
exit

File write operation completed successfully!
\end{lstlisting}
\end{solutionbox}

\begin{mnemonicbox}
\mnemonic{COWS: Create Open Write Save file operations}
\end{mnemonicbox}

\end{document}

