\documentclass{article}
% Adjust the relative path to point to the latex-templates directory from:
% content/resources/study-materials/32-ict/sem-4/4343203-java/

% content/resources/templates/preamble.tex
\usepackage[margin=0.6in]{geometry}
\author{Milav Dabgar}
\usepackage{amsmath,amssymb,amsthm}
\usepackage{booktabs}
\usepackage{multirow}
\usepackage{xcolor}
\usepackage{tcolorbox}
\tcbuselibrary{breakable,skins}
\usepackage[colorlinks=true,linkcolor=blue]{hyperref}
\usepackage{titlesec}
\usepackage{enumitem}
\usepackage{tikz}
\usepackage{pgfplots}
\usepackage{circuitikz}
\usepackage[version=4]{mhchem}
\usepackage{longtable}
\usepackage{array}
\usepackage{float}
\usepackage{caption}
\usepackage{listings}

\lstset{
  basicstyle=\small\ttfamily,
  breaklines=true,
  breakatwhitespace=false,
  postbreak=\mbox{\textcolor{red}{$\hookrightarrow$}\space},
  float=false,
  numbers=left,
  numberstyle=\tiny\color{gray},
  numbersep=10pt,
  xleftmargin=2em,
  keywordstyle=\color{blue},
  commentstyle=\color{green!60!black},
  stringstyle=\color{purple},
  backgroundcolor=\color{gray!5},
  showstringspaces=false,
  tabsize=2,
  captionpos=b,
  keepspaces=true,
  columns=flexible
}

\pgfplotsset{compat=1.18}
\usetikzlibrary{shapes,arrows,positioning,calc,patterns,decorations.pathmorphing,decorations.markings,arrows.meta}

% Color scheme
\definecolor{headcolor}{RGB}{0,102,204}
\definecolor{keycolor}{RGB}{220,20,60}
\definecolor{solutioncolor}{RGB}{34,139,34}
\definecolor{mnemoniccolor}{RGB}{148,0,211}
\definecolor{codecolor}{RGB}{0,0,100}

% Spacing
\setlength{\parskip}{3pt}
\setlist[itemize]{nosep}
\setlist[enumerate]{nosep}

% Title formatting
\titleformat{\section}{\Large\bfseries\color{headcolor}}{\thesection}{1em}{}
\titleformat{\subsection}{\large\bfseries\color{headcolor}}{\thesubsection}{1em}{}

% Pandoc tightlist compatibility
\providecommand{\tightlist}{%
  \setlength{\itemsep}{0pt}\setlength{\parskip}{0pt}}

% Pandoc longtable compatibility
\newcounter{none}
\def\thenone{}


% content/resources/templates/english-boxes.tex

% Custom environments
\newtcolorbox{solutionbox}{
 breakable,
 enhanced,
 colback=solutioncolor!5!white,
 colframe=solutioncolor!75!black,
 fonttitle=\bfseries,
 title=Solution
}

\newtcolorbox{solutionboxnobreak}{
 colback=solutioncolor!5!white,
 colframe=solutioncolor!75!black,
 fonttitle=\bfseries,
 title=Solution
}

\newtcolorbox{keyformula}{
 breakable,
 enhanced,
 colback=keycolor!5!white,
 colframe=keycolor!75!black,
 fonttitle=\bfseries,
 title=Key Formula
}

\newtcolorbox{mnemonicboxenv}{
 breakable,
 enhanced,
 colback=mnemoniccolor!5!white,
 colframe=mnemoniccolor!75!black,
 fonttitle=\bfseries,
 title=Mnemonic
}

\newcommand{\mnemonicbox}[1]{%
  \begin{mnemonicboxenv}
    #1
  \end{mnemonicboxenv}
}


% Custom commands for GTU solutions
% This file defines semantic commands for consistent formatting

% Question command with automatic formatting
\newcommand{\question}[2]{%
  \section*{Question #1}%
  \textbf{#2}%
}

% OR question variant
\newcommand{\questionor}[2]{%
  \section*{Question #1 OR}%
  \textbf{#2}%
}

% Proper table environment with caption
\newenvironment{answertable}[1]{%
  \begin{table}[htbp]
  \centering
  \caption{#1}
}{%
  \end{table}
}

% Proper figure environment for diagrams
\newenvironment{answerdiagram}[1]{%
  \begin{figure}[htbp]
  \centering
  \caption{#1}
}{%
  \end{figure}
}

% Semantic markup for key terms
\newcommand{\keyword}[1]{\textbf{#1}}
\newcommand{\code}[1]{\texttt{#1}}
\newcommand{\classname}[1]{\texttt{#1}}
\newcommand{\methodname}[1]{\texttt{#1}}

% Proper quotation marks
\newcommand{\mnemonic}[1]{``#1''}


\title{Java Programming (4343203) -- Summer 2024 Solution}
\date{June 15, 2024}

\begin{document}
\maketitle

\questionmarks{1(a)}{3}{Explain Garbage collection in Java.}

\begin{solutionbox}
Garbage collection in Java automatically reclaims memory by removing unused objects.

\begin{center}
\captionof{table}{Garbage Collection Process}

\begin{tabulary}{\linewidth}{|L|L|}
\hline
\textbf{Phase} & \textbf{Description} \\ \hline
Mark & JVM identifies all live objects in memory \\ \hline
Sweep & Unused objects are removed \\ \hline
Compact & Remaining objects are reorganized to free up space \\ \hline
\end{tabulary}
\end{center}

\begin{itemize}
    \item \keyword{Automatic}: No manual memory management required
    \item \keyword{Background}: Runs in separate low-priority thread
\end{itemize}
\end{solutionbox}

\begin{mnemonicbox}
\mnemonic{MSC: Mark-Sweep-Compact frees memory automatically}
\end{mnemonicbox}

\questionmarks{1(b)}{4}{Explain JVM in detail.}

\begin{solutionbox}
JVM (Java Virtual Machine) is a virtual machine that enables Java's platform independence by converting bytecode to machine code.

\begin{center}
\begin{tikzpicture}[node distance=1.5cm, auto]

    \node [gtu block] (java) {Java Code};
    \node [gtu block, right=0.8cm of java] (compiler) {Compiler};
    \node [gtu block, right=0.8cm of compiler] (bytecode) {Bytecode};
    \node [gtu block, right=0.8cm of bytecode] (jvm) {JVM};
    \node [gtu block, right=0.8cm of jvm] (machine) {Machine Code};

    \path [gtu arrow] (java) -- (compiler);
    \path [gtu arrow] (compiler) -- (bytecode);
    \path [gtu arrow] (bytecode) -- (jvm);
    \path [gtu arrow] (jvm) -- (machine);

    \node [draw, dashed, inner sep=0.2cm, below=0.5cm of jvm, text width=3cm, align=center, font=\footnotesize] (components) {\textbf{JVM Components}\\Class Loader\\Runtime Data Areas\\Execution Engine};
    \path [gtu arrow, dashed] (jvm) -- (components);

\end{tikzpicture}
\captionof{figure}{JVM Architecture}
\end{center}

\begin{itemize}
    \item \keyword{Platform Independence}: Write once, run anywhere
    \item \keyword{Security}: Bytecode verification prevents dangerous operations
    \item \keyword{Optimization}: Just-in-time compilation improves performance
\end{itemize}
\end{solutionbox}

\begin{mnemonicbox}
\mnemonic{CLASS: Class Loader Leads All System Security}
\end{mnemonicbox}

\questionmarks{1(c)}{7}{Write a program in java to print Fibonacci series for N terms.}

\begin{solutionbox}
Fibonacci series generates numbers where each is the sum of the two preceding ones.

\begin{lstlisting}[language=Java,caption={Fibonacci Series Program}]
import java.util.Scanner;

public class FibonacciSeries {
    public static void main(String[] args) {
        Scanner input = new Scanner(System.in);
        
        System.out.print("Enter number of terms: ");
        int n = input.nextInt();
        
        int first = 0, second = 1;
        
        System.out.print("Fibonacci Series: ");
        
        for (int i = 1; i <= n; i++) {
            System.out.print(first + " ");
            
            int next = first + second;
            first = second;
            second = next;
        }
        
        input.close();
    }
}
\end{lstlisting}

\begin{itemize}
    \item \keyword{Initialize}: Start with 0 and 1
    \item \keyword{Loop}: Iterate N times to generate sequence
    \item \keyword{Calculation}: Each number is sum of previous two
\end{itemize}
\end{solutionbox}

\begin{mnemonicbox}
\mnemonic{FSN: First + Second = Next number in sequence}
\end{mnemonicbox}

\questionmarks{1(c OR)}{7}{Write a program in java to find out minimum from any ten numbers using command line argument.}

\begin{solutionbox}
Command line arguments allow passing input values directly when executing a Java program.

\begin{lstlisting}[language=Java,caption={Finding Minimum from Command Line Arguments}]
public class FindMinimum {
    public static void main(String[] args) {
        if (args.length < 10) {
            System.out.println("Please provide 10 numbers");
            return;
        }
        
        int min = Integer.parseInt(args[0]);
        
        for (int i = 1; i < 10; i++) {
            int current = Integer.parseInt(args[i]);
            if (current < min) {
                min = current;
            }
        }
        
        System.out.println("Minimum number is: " + min);
    }
}
\end{lstlisting}

\begin{itemize}
    \item \keyword{Parse Arguments}: Convert string arguments to integers
    \item \keyword{Initialize}: Set first number as minimum
    \item \keyword{Compare}: Check each number against current minimum
\end{itemize}
\end{solutionbox}

\begin{mnemonicbox}
\mnemonic{ICU: Initialize, Compare, Update the minimum}
\end{mnemonicbox}

\questionmarks{2(a)}{3}{List out basic concepts of Java OOP. Explain any one in detail.}

\begin{solutionbox}
Java Object-Oriented Programming is built on fundamental concepts for modeling real-world entities.

\begin{center}
\captionof{table}{OOP Concepts in Java}

\begin{tabulary}{\linewidth}{|L|L|}
\hline
\textbf{Concept} & \textbf{Description} \\ \hline
Encapsulation & Binding data and methods together as a single unit \\ \hline
Inheritance & Creating new classes from existing ones \\ \hline
Polymorphism & One interface, multiple implementations \\ \hline
Abstraction & Hiding implementation details, showing functionality \\ \hline
\end{tabulary}
\end{center}

\begin{itemize}
    \item \keyword{Encapsulation}: Protects data through access control
    \item \keyword{Data Hiding}: Private variables accessible through methods
\end{itemize}
\end{solutionbox}

\begin{mnemonicbox}
\mnemonic{PEAI: Programming Encapsulates Abstracts Inherits}
\end{mnemonicbox}

\questionmarks{2(b)}{4}{Explain final keyword with example.}

\begin{solutionbox}
The \code{final} keyword in Java restricts modification and creates constants, unchangeable methods, and non-inheritable classes.

\begin{center}
\captionof{table}{Uses of final Keyword}

\begin{tabulary}{\linewidth}{|L|L|L|}
\hline
\textbf{Usage} & \textbf{Effect} & \textbf{Example} \\ \hline
final variable & Cannot be changed & \code{final int MAX = 100;} \\ \hline
final method & Cannot be overridden & \code{final void display() \{\}} \\ \hline
final class & Cannot be extended & \code{final class Math \{\}} \\ \hline
\end{tabulary}
\end{center}

\begin{lstlisting}[language=Java,caption={Final Keyword Demonstration}]
public class FinalDemo {
    final int MAX_VALUE = 100;  // constant
    
    final void display() {
        System.out.println("This method cannot be overridden");
    }
}

final class MathOperations {
    // This class cannot be inherited
}
\end{lstlisting}
\end{solutionbox}

\begin{mnemonicbox}
\mnemonic{VCM: Variables Constants Methods can't change}
\end{mnemonicbox}

\questionmarks{2(c)}{7}{What is constructor? Explain parameterized constructor with example.}

\begin{solutionbox}
A constructor initializes objects when created, with the same name as its class and no return type.

\begin{center}
\begin{tikzpicture}[node distance=1.5cm, auto]

    \node [gtu block] (root) {Constructors};
    \node [gtu block, below left=1cm of root] (default) {Default Constructor};
    \node [gtu block, below=1cm of root] (param) {Parameterized Constructor};
    \node [gtu block, below right=1cm of root] (copy) {Copy Constructor};

    \path [gtu arrow] (root) -- (default);
    \path [gtu arrow] (root) -- (param);
    \path [gtu arrow] (root) -- (copy);
\end{tikzpicture}
\captionof{figure}{Constructor Types}
\end{center}

\begin{lstlisting}[language=Java,caption={Parameterized Constructor Example}]
public class Student {
    String name;
    int age;
    
    // Parameterized constructor
    Student(String n, int a) {
        name = n;
        age = a;
    }
    
    void display() {
        System.out.println("Name: " + name + ", Age: " + age);
    }
    
    public static void main(String[] args) {
        // Object creation using parameterized constructor
        Student s1 = new Student("John", 20);
        s1.display();
    }
}
\end{lstlisting}

\begin{itemize}
    \item \keyword{Parameters}: Accept values during object creation
    \item \keyword{Initialization}: Set object properties with passed values
    \item \keyword{Overloading}: Multiple constructors with different parameters
\end{itemize}
\end{solutionbox}

\begin{mnemonicbox}
\mnemonic{SPO: Student Parameters Object initializes properties}
\end{mnemonicbox}

\questionmarks{2(a OR)}{3}{Explain the Java Program Structure with example.}

\begin{solutionbox}
Java program structure follows a specific hierarchy of elements organized logically.

\begin{center}
\begin{tikzpicture}[
    box/.style={draw, rectangle, minimum width=4cm, minimum height=1cm, align=left, fill=white}
]
    \node[gtu container, minimum height=4cm] (outer) {};
    \node[anchor=north, font=\bfseries] at (outer.north) {Source File};
    
    \node[box, below=0.5cm of outer.north] (top) {
        Documentation\\
        \code{package} statement\\
        \code{import} statements
    };
    
    \node[box, below=0.2cm of top, minimum height=2cm] (class) {
        \textbf{Class declaration}\\
        - Variables\\
        - Constructors\\
        - Methods
    };
\end{tikzpicture}
\captionof{figure}{Java Program Structure}
\end{center}

\begin{itemize}
    \item \keyword{Package}: Groups related classes
    \item \keyword{Import}: Includes external classes
    \item \keyword{Class}: Contains variables and methods
\end{itemize}
\end{solutionbox}

\begin{mnemonicbox}
\mnemonic{PIC: Package Imports Class in every program}
\end{mnemonicbox}

\questionmarks{2(b OR)}{4}{Explain static keyword with suitable example.}

\begin{solutionbox}
The \code{static} keyword creates class-level variables and methods shared by all objects, accessible without creating instances.

\begin{center}
\captionof{table}{Static vs Non-Static}

\begin{tabulary}{\linewidth}{|L|L|L|}
\hline
\textbf{Feature} & \textbf{Static} & \textbf{Non-Static} \\ \hline
Memory & Single copy & Multiple copies \\ \hline
Access & Without object & Through object \\ \hline
Reference & Class name & Object name \\ \hline
When loaded & Class loading & Object creation \\ \hline
\end{tabulary}
\end{center}

\begin{lstlisting}[language=Java,caption={Static Keyword Demonstration}]
public class Counter {
    static int count = 0;  // Shared by all objects
    int instanceCount = 0; // Unique to each object
    
    Counter() {
        count++;
        instanceCount++;
    }
    
    public static void main(String[] args) {
        Counter c1 = new Counter();
        Counter c2 = new Counter();
        
        System.out.println("Static count: " + Counter.count);
        System.out.println("c1's instance count: " + c1.instanceCount);
        System.out.println("c2's instance count: " + c2.instanceCount);
    }
}
\end{lstlisting}
\end{solutionbox}

\begin{mnemonicbox}
\mnemonic{SCM: Static Creates Memory once for all objects}
\end{mnemonicbox}

\questionmarks{2(c OR)}{7}{Define Inheritance. List out types of it. Explain multilevel and hierarchical inheritance with suitable example.}

\begin{solutionbox}
Inheritance is an OOP principle where a new class acquires properties and behaviors from an existing class.

\begin{center}
\captionof{table}{Types of Inheritance in Java}

\begin{tabulary}{\linewidth}{|L|L|}
\hline
\textbf{Type} & \textbf{Description} \\ \hline
Single & One subclass extends one superclass \\ \hline
Multilevel & Chain of inheritance (\code{A}$\rightarrow$\code{B}$\rightarrow$\code{C}) \\ \hline
Hierarchical & Multiple subclasses extend one superclass \\ \hline
Multiple & One class extends multiple classes (via interfaces) \\ \hline
\end{tabulary}
\end{center}

\begin{center}
\begin{tikzpicture}[node distance=1.2cm, auto]

    % Multilevel
    \node [gtu class] (A) {Animal};
    \node [gtu class, below=0.8cm of A] (B) {Dog};
    \node [gtu class, below=0.8cm of B] (C) {Labrador};
    \node [above=0.2cm of A] {\textbf{Multilevel}};

    \path [gtu arrow] (B) -- (A);
    \path [gtu arrow] (C) -- (B);

    % Hierarchical
    \node [gtu class, right=3cm of A] (V) {Vehicle};
    \node [gtu class, below left=1cm and -0.5cm of V] (C1) {Car};
    \node [gtu class, below=1cm of V] (C2) {Bike};
    \node [gtu class, below right=1cm and -0.5cm of V] (C3) {Truck};
    \node [above=0.2cm of V] {\textbf{Hierarchical}};

    \path [gtu arrow] (C1) -- (V);
    \path [gtu arrow] (C2) -- (V);
    \path [gtu arrow] (C3) -- (V);
\end{tikzpicture}
\captionof{figure}{Multilevel vs Hierarchical Inheritance}
\end{center}

\begin{lstlisting}[language=Java,caption={Multilevel and Hierarchical Inheritance}]
// Multilevel inheritance
class Animal {
    void eat() { System.out.println("eating"); }
}

class Dog extends Animal {
    void bark() { System.out.println("barking"); }
}

class Labrador extends Dog {
    void color() { System.out.println("golden"); }
}

// Hierarchical inheritance
class Vehicle {
    void move() { System.out.println("moving"); }
}

class Car extends Vehicle {
    void wheels() { System.out.println("4 wheels"); }
}

class Bike extends Vehicle {
    void wheels() { System.out.println("2 wheels"); }
}
\end{lstlisting}
\end{solutionbox}

\begin{mnemonicbox}
\mnemonic{SMHM: Single Multilevel Hierarchical Makes inheritance types}
\end{mnemonicbox}

\questionmarks{3(a)}{3}{Explain this keyword with suitable example.}

\begin{solutionbox}
The \code{this} keyword in Java refers to the current object, used to differentiate between instance variables and parameters.

\begin{center}
\captionof{table}{Uses of \code{this} Keyword}

\begin{tabulary}{\linewidth}{|L|L|}
\hline
\textbf{Use} & \textbf{Purpose} \\ \hline
this.variable & Access instance variables \\ \hline
this() & Call current class constructor \\ \hline
return this & Return current object \\ \hline
\end{tabulary}
\end{center}

\begin{lstlisting}[language=Java,caption={This Keyword Example}]
public class Student {
    String name;
    
    Student(String name) {
        this.name = name;  // Refers to instance variable
    }
    
    void display() {
        System.out.println("Name: " + this.name);
    }
}
\end{lstlisting}
\end{solutionbox}

\begin{mnemonicbox}
\mnemonic{VAR: Variables Access Resolution using this}
\end{mnemonicbox}

\questionmarks{3(b)}{4}{Explain different access controls in Java.}

\begin{solutionbox}
Access controls in Java regulate visibility and accessibility of classes, methods, and variables.

\begin{center}
\captionof{table}{Java Access Modifiers}

\begin{tabulary}{\linewidth}{|L|C|C|C|C|}
\hline
\textbf{Modifier} & \textbf{Class} & \textbf{Package} & \textbf{Subclass} & \textbf{World} \\ \hline
private & $\checkmark$ & $\times$ & $\times$ & $\times$ \\ \hline
default & $\checkmark$ & $\checkmark$ & $\times$ & $\times$ \\ \hline
protected & $\checkmark$ & $\checkmark$ & $\checkmark$ & $\times$ \\ \hline
public & $\checkmark$ & $\checkmark$ & $\checkmark$ & $\checkmark$ \\ \hline
\end{tabulary}
\end{center}

\begin{itemize}
    \item \keyword{Private}: Only within the same class
    \item \keyword{Default}: Within the same package
    \item \keyword{Protected}: Within package and subclasses
    \item \keyword{Public}: Accessible everywhere
\end{itemize}
\end{solutionbox}

\begin{mnemonicbox}
\mnemonic{PDPP: Private Default Protected Public from narrow to wide}
\end{mnemonicbox}

\questionmarks{3(c)}{7}{What is interface? Explain multiple inheritance using interface with example.}

\begin{solutionbox}
An interface is a contract that specifies what a class must do, containing abstract methods, constants, and (since Java 8) default methods.

\begin{center}
\begin{tikzpicture}[node distance=1.5cm, auto]

    \node [gtu interface] (P) {\textit{<<interface>>}\\Printable};
    \node [gtu interface, right=2cm of P] (S) {\textit{<<interface>>}\\Scannable};
    \node [gtu class, below left=1.5cm and -1cm of S] (I) {Printer};

    \path [gtu arrow] (I) -- (P);
    \path [gtu arrow] (I) -- (S);
\end{tikzpicture}
\captionof{figure}{Multiple Inheritance with Interfaces}
\end{center}

\begin{lstlisting}[language=Java,caption={Multiple Inheritance with Interfaces}]
interface Printable {
    void print();
}

interface Scannable {
    void scan();
}

// Multiple inheritance using interfaces
class Printer implements Printable, Scannable {
    public void print() {
        System.out.println("Printing...");
    }
    
    public void scan() {
        System.out.println("Scanning...");
    }
    
    public static void main(String[] args) {
        Printer p = new Printer();
        p.print();
        p.scan();
    }
}
\end{lstlisting}

\begin{itemize}
    \item \keyword{Contract}: Defines behavior without implementation
    \item \keyword{Implements}: Classes fulfill the contract
    \item \keyword{Multiple}: Can implement many interfaces
\end{itemize}
\end{solutionbox}

\begin{mnemonicbox}
\mnemonic{CIM: Contract Implements Multiple interfaces}
\end{mnemonicbox}

\questionmarks{3(a OR)}{3}{Explain super keyword with example.}

\begin{solutionbox}
The \code{super} keyword refers to the parent class, used to access parent methods, constructors, and variables.

\begin{center}
\captionof{table}{Uses of super Keyword}

\begin{tabulary}{\linewidth}{|L|L|}
\hline
\textbf{Use} & \textbf{Purpose} \\ \hline
super.variable & Access parent variable \\ \hline
super.method() & Call parent method \\ \hline
super() & Call parent constructor \\ \hline
\end{tabulary}
\end{center}

\begin{lstlisting}[language=Java,caption={Super Keyword Example}]
class Vehicle {
    String color = "white";
    
    void display() {
        System.out.println("Vehicle class");
    }
}

class Car extends Vehicle {
    String color = "black";
    
    void display() {
        super.display();  // Calls parent method
        System.out.println("Car color: " + color);
        System.out.println("Vehicle color: " + super.color);
    }
}
\end{lstlisting}
\end{solutionbox}

\begin{mnemonicbox}
\mnemonic{VMC: Variables Methods Constructors accessed by super}
\end{mnemonicbox}

\questionmarks{3(b OR)}{4}{What is package? Write steps to create a package and give example of it.}

\begin{solutionbox}
A package in Java is a namespace that organizes related classes and interfaces, preventing naming conflicts.

\begin{center}
\captionof{table}{Steps to Create a Package}

\begin{tabulary}{\linewidth}{|C|L|}
\hline
\textbf{Step} & \textbf{Action} \\ \hline
1 & Declare package name at top of file \\ \hline
2 & Create directory structure matching package name \\ \hline
3 & Save Java file in the directory \\ \hline
4 & Compile with -d option \\ \hline
5 & Import package to use it \\ \hline
\end{tabulary}
\end{center}

\begin{lstlisting}[language=Java,caption={Package Creation and Usage}]
// Step 1: Declare package (save as Calculator.java)
package mathematics;

public class Calculator {
    public int add(int a, int b) {
        return a + b;
    }
}

// In another file (UseCalculator.java)
import mathematics.Calculator;

class UseCalculator {
    public static void main(String[] args) {
        Calculator calc = new Calculator();
        System.out.println(calc.add(10, 20));
    }
}
\end{lstlisting}
\end{solutionbox}

\begin{mnemonicbox}
\mnemonic{DISCO: Declare Import Save Compile Organize}
\end{mnemonicbox}

\questionmarks{3(c OR)}{7}{Define: Method Overriding. List out Rules for method overriding. Write a java program that implements method overriding.}

\begin{solutionbox}
Method overriding occurs when a subclass provides a specific implementation for a method already defined in its parent class.

\begin{center}
\captionof{table}{Rules for Method Overriding}

\begin{tabulary}{\linewidth}{|L|L|}
\hline
\textbf{Rule} & \textbf{Description} \\ \hline
Same name & Method must have same name \\ \hline
Same parameters & Parameter count and type must match \\ \hline
Same return type & Return type must be same or subtype (covariant) \\ \hline
Access modifier & Can't be more restrictive \\ \hline
Exceptions & Can't throw broader checked exceptions \\ \hline
\end{tabulary}
\end{center}

\begin{lstlisting}[language=Java,caption={Method Overriding Demonstration}]
class Animal {
    void makeSound() {
        System.out.println("Animal makes a sound");
    }
}

class Dog extends Animal {
    // Method overriding
    @Override
    void makeSound() {
        System.out.println("Dog barks");
    }
}

class Cat extends Animal {
    // Method overriding
    @Override
    void makeSound() {
        System.out.println("Cat meows");
    }
}

public class MethodOverridingDemo {
    public static void main(String[] args) {
        Animal animal = new Animal();
        Animal dog = new Dog();
        Animal cat = new Cat();
        
        animal.makeSound();  // Output: Animal makes a sound
        dog.makeSound();     // Output: Dog barks
        cat.makeSound();     // Output: Cat meows
    }
}
\end{lstlisting}

\begin{itemize}
    \item \keyword{Runtime Polymorphism}: Method resolution at runtime
    \item \keyword{@Override}: Annotation ensures method is overriding
    \item \keyword{Inheritance}: Requires IS-A relationship
\end{itemize}
\end{solutionbox}

\begin{mnemonicbox}
\mnemonic{SPARE: Same Parameters Access Return Exceptions}
\end{mnemonicbox}

\questionmarks{4(a)}{3}{Explain abstract class with suitable example.}

\begin{solutionbox}
An abstract class cannot be instantiated and may contain abstract methods that must be implemented by subclasses.

\begin{center}
\captionof{table}{Abstract Class vs Interface}

\begin{tabulary}{\linewidth}{|L|L|L|}
\hline
\textbf{Feature} & \textbf{Abstract Class} & \textbf{Interface} \\ \hline
Instantiation & Cannot & Cannot \\ \hline
Methods & Concrete and abstract & Abstract (+ default since Java 8) \\ \hline
Variables & Any type & Only constants \\ \hline
Constructor & Has & Doesn't have \\ \hline
\end{tabulary}
\end{center}

\begin{lstlisting}[language=Java,caption={Abstract Class Implementation}]
abstract class Shape {
    // Abstract method - no implementation
    abstract double area();
    
    // Concrete method
    void display() {
        System.out.println("This is a shape");
    }
}

class Circle extends Shape {
    double radius;
    
    Circle(double r) {
        radius = r;
    }
    
    // Implementation of abstract method
    double area() {
        return 3.14 * radius * radius;
    }
}
\end{lstlisting}
\end{solutionbox}

\begin{mnemonicbox}
\mnemonic{PAI: Partial Abstract Implementation is key}
\end{mnemonicbox}

\questionmarks{4(b)}{4}{What is Thread? Explain Thread life cycle.}

\begin{solutionbox}
A thread is a lightweight subprocess, the smallest unit of processing that allows concurrent execution.

\begin{center}
\begin{tikzpicture}[node distance=1.5cm, auto]

    \node [gtu state] (N) {New};
    \node [gtu state, right=1.5cm of N] (R) {Runnable};
    \node [gtu state, right=1.5cm of R] (Ru) {Running};
    \node [gtu state, right=1.5cm of Ru] (T) {Terminated};
    \node [gtu state, below=1.5cm of Ru] (B) {Blocked};

    \path [gtu arrow] (N) -- (R);
    \path [gtu arrow] (R) -- node[font=\tiny]{Scheduler} (Ru);
    \path [gtu arrow] (Ru) -- (T);
    \path [gtu arrow] (Ru) -- node[right, font=\tiny]{Wait} (B);
    \path [gtu arrow] (B) -| node[pos=0.2, above, font=\tiny]{Notify} (R);
\end{tikzpicture}
\captionof{figure}{Thread Life Cycle}
\end{center}

\begin{itemize}
    \item \keyword{New}: Thread created but not started
    \item \keyword{Runnable}: Ready to run when CPU time is given
    \item \keyword{Running}: Currently executing
    \item \keyword{Blocked/Waiting}: Temporarily inactive
    \item \keyword{Terminated}: Completed execution
\end{itemize}
\end{solutionbox}

\begin{mnemonicbox}
\mnemonic{NRRBT: New Runnable Running Blocked Terminated}
\end{mnemonicbox}

\questionmarks{4(c)}{7}{Write a program in java that creates the multiple threads by implementing the Thread class.}

\begin{solutionbox}
Creating threads by implementing Thread class allows multiple tasks to execute concurrently.

\begin{lstlisting}[language=Java,caption={Multiple Thread Creation}]
class MyThread extends Thread {
    private String threadName;
    
    MyThread(String name) {
        this.threadName = name;
    }
    
    @Override
    public void run() {
        try {
            for (int i = 1; i <= 5; i++) {
                System.out.println(threadName + ": " + i);
                Thread.sleep(500);
            }
        } catch (InterruptedException e) {
            System.out.println(threadName + " interrupted");
        }
        System.out.println(threadName + " completed");
    }
}

public class MultiThreadDemo {
    public static void main(String[] args) {
        MyThread thread1 = new MyThread("Thread-1");
        MyThread thread2 = new MyThread("Thread-2");
        MyThread thread3 = new MyThread("Thread-3");
        
        thread1.start();
        thread2.start();
        thread3.start();
    }
}
\end{lstlisting}

\begin{itemize}
    \item \keyword{Extend Thread}: Create thread by extending Thread class
    \item \keyword{Override run()}: Define task in run method
    \item \keyword{start()}: Begin thread execution
\end{itemize}
\end{solutionbox}

\begin{mnemonicbox}
\mnemonic{ERS: Extend Run Start to create threads}
\end{mnemonicbox}

\questionmarks{4(a OR)}{3}{Explain final class with suitable example.}

\begin{solutionbox}
A final class cannot be extended, preventing inheritance and modification of its design.

\begin{center}
\captionof{table}{Final Class Characteristics}

\begin{tabulary}{\linewidth}{|L|L|}
\hline
\textbf{Feature} & \textbf{Description} \\ \hline
Inheritance & Cannot be subclassed \\ \hline
Methods & Implicitly final \\ \hline
Security & Prevents design alteration \\ \hline
Example & String, Math classes \\ \hline
\end{tabulary}
\end{center}

\begin{lstlisting}[language=Java,caption={Final Class Example}]
final class Security {
    void secureMethod() {
        System.out.println("Secure implementation");
    }
}

// Error: Cannot extend final class
// class HackAttempt extends Security { }
\end{lstlisting}

\begin{itemize}
    \item \keyword{Security}: Protects sensitive implementations
    \item \keyword{Immutability}: Helps create immutable classes
    \item \keyword{Optimization}: JVM can optimize final classes
\end{itemize}
\end{solutionbox}

\begin{mnemonicbox}
\mnemonic{SIO: Security Immutability Optimization}
\end{mnemonicbox}

\questionmarks{4(b OR)}{4}{Explain thread priorities with suitable example.}

\begin{solutionbox}
Thread priorities determine the order in which threads are scheduled for execution, from 1 (lowest) to 10 (highest).

\begin{center}
\captionof{table}{Thread Priority Constants}

\begin{tabulary}{\linewidth}{|L|C|L|}
\hline
\textbf{Constant} & \textbf{Value} & \textbf{Description} \\ \hline
MIN\_PRIORITY & 1 & Lowest priority \\ \hline
NORM\_PRIORITY & 5 & Default priority \\ \hline
MAX\_PRIORITY & 10 & Highest priority \\ \hline
\end{tabulary}
\end{center}

\begin{lstlisting}[language=Java,caption={Thread Priority Demonstration}]
class PriorityThread extends Thread {
    PriorityThread(String name) {
        super(name);
    }
    
    public void run() {
        System.out.println("Running: " + getName() + 
                          " with priority: " + getPriority());
    }
}

public class ThreadPriorityDemo {
    public static void main(String[] args) {
        PriorityThread low = new PriorityThread("Low Priority");
        PriorityThread norm = new PriorityThread("Normal Priority");
        PriorityThread high = new PriorityThread("High Priority");
        
        low.setPriority(Thread.MIN_PRIORITY);
        high.setPriority(Thread.MAX_PRIORITY);
        
        low.start();
        norm.start();
        high.start();
    }
}
\end{lstlisting}
\end{solutionbox}

\begin{mnemonicbox}
\mnemonic{HNL: High Normal Low priorities in threads}
\end{mnemonicbox}

\questionmarks{4(c OR)}{7}{What is Exception? Write a program that shows the use of Arithmetic Exception.}

\begin{solutionbox}
An exception is an abnormal condition that disrupts the normal flow of program execution.

\begin{center}
\begin{tikzpicture}[node distance=1cm, auto]

    \node [gtu class] (T) {Throwable};
    \node [gtu class, below left=1cm of T] (E) {Exception};
    \node [gtu class, below right=1cm of T] (Err) {Error};
    \node [gtu class, below=1cm of E] (RE) {RuntimeException};
    \node [gtu class, below left=1cm and -1.5cm of RE] (AE) {ArithmeticException};
    \node [gtu class, below right=1cm and -1.5cm of RE] (NPE) {NullPointerException};

    \path [gtu arrow] (E) -- (T);
    \path [gtu arrow] (Err) -- (T);
    \path [gtu arrow] (RE) -- (E);
    \path [gtu arrow] (AE) -- (RE);
    \path [gtu arrow] (NPE) -- (RE);
\end{tikzpicture}
\captionof{figure}{Exception Hierarchy}
\end{center}

\begin{lstlisting}[language=Java,caption={Arithmetic Exception Handling}]
public class ArithmeticExceptionDemo {
    public static void main(String[] args) {
        try {
            // This will cause ArithmeticException
            int result = 100 / 0;
            System.out.println("Result: " + result);
        } 
        catch (ArithmeticException e) {
            System.out.println("ArithmeticException caught: " + e.getMessage());
            System.out.println("Cannot divide by zero");
        }
        finally {
            System.out.println("This block always executes");
        }
        
        System.out.println("Program continues after exception handling");
    }
}
\end{lstlisting}

\begin{itemize}
    \item \keyword{Try Block}: Contains code that might throw exceptions
    \item \keyword{Catch Block}: Handles the specific exception
    \item \keyword{Finally Block}: Always executes regardless of exception
\end{itemize}
\end{solutionbox}

\begin{mnemonicbox}
\mnemonic{TCF: Try Catch Finally handles exceptions}
\end{mnemonicbox}

\questionmarks{5(a)}{3}{Write a Java Program to find sum and average of 10 numbers of an array.}

\begin{solutionbox}
Arrays store multiple values of the same type, enabling sequential processing of elements.

\begin{lstlisting}[language=Java,caption={Array Sum and Average Calculation}]
public class ArraySumAverage {
    public static void main(String[] args) {
        int[] numbers = {10, 20, 30, 40, 50, 60, 70, 80, 90, 100};
        
        int sum = 0;
        
        // Calculate sum
        for (int i = 0; i < numbers.length; i++) {
            sum += numbers[i];
        }
        
        // Calculate average
        double average = (double) sum / numbers.length;
        
        System.out.println("Sum = " + sum);
        System.out.println("Average = " + average);
    }
}
\end{lstlisting}

\begin{itemize}
    \item \keyword{Declaration}: Creates fixed-size collection
    \item \keyword{Iteration}: Sequential access to elements
    \item \keyword{Calculation}: Process values for results
\end{itemize}
\end{solutionbox}

\begin{mnemonicbox}
\mnemonic{DIC: Declare Iterate Calculate for array processing}
\end{mnemonicbox}

\questionmarks{5(b)}{4}{Write a Java program to handle user defined exception for ``Divide by Zero'' error.}

\begin{solutionbox}
User-defined exceptions allow creating custom exception types for specific application requirements.

\begin{lstlisting}[language=Java,caption={User-Defined Exception for Division by Zero}]
// Custom exception class
class DivideByZeroException extends Exception {
    public DivideByZeroException(String message) {
        super(message);
    }
}

public class CustomExceptionDemo {
    // Method that throws custom exception
    static double divide(int numerator, int denominator) throws DivideByZeroException {
        if (denominator == 0) {
            throw new DivideByZeroException("Cannot divide by zero!");
        }
        return (double) numerator / denominator;
    }
    
    public static void main(String[] args) {
        try {
            System.out.println(divide(10, 2));
            System.out.println(divide(20, 0));
        } catch (DivideByZeroException e) {
            System.out.println("Custom exception caught: " + e.getMessage());
        }
    }
}
\end{lstlisting}

\begin{itemize}
    \item \keyword{Custom Class}: Extends Exception class
    \item \keyword{Throwing}: Use throw keyword with new instance
    \item \keyword{Handling}: Catch specific exception type
\end{itemize}
\end{solutionbox}

\begin{mnemonicbox}
\mnemonic{CTE: Create Throw Exception when needed}
\end{mnemonicbox}

\questionmarks{5(c)}{7}{Write a java program to create a text file and perform read operation on the text file.}

\begin{solutionbox}
Java provides I/O classes to work with files, allowing creation, writing, and reading operations.

\begin{lstlisting}[language=Java,caption={File Creation and Reading Operations}]
import java.io.FileWriter;
import java.io.FileReader;
import java.io.IOException;
import java.io.BufferedReader;

public class FileOperationsDemo {
    public static void main(String[] args) {
        try {
            // Create and write to file
            FileWriter writer = new FileWriter("sample.txt");
            writer.write("Hello World!\n");
            writer.write("Welcome to Java File Handling.\n");
            writer.write("This is the third line.");
            writer.close();
            System.out.println("Successfully wrote to the file.");
            
            // Read from file
            FileReader reader = new FileReader("sample.txt");
            BufferedReader buffReader = new BufferedReader(reader);
            
            String line;
            System.out.println("\nFile contents:");
            while ((line = buffReader.readLine()) != null) {
                System.out.println(line);
            }
            
            reader.close();
            
        } catch (IOException e) {
            System.out.println("An error occurred: " + e.getMessage());
        }
    }
}
\end{lstlisting}

\begin{itemize}
    \item \keyword{FileWriter}: Creates and writes to files
    \item \keyword{FileReader}: Reads character data from files
    \item \keyword{BufferedReader}: Efficiently reads text by lines
\end{itemize}
\end{solutionbox}

\begin{mnemonicbox}
\mnemonic{WRC: Write Read Close for file operations}
\end{mnemonicbox}

\questionmarks{5(a OR)}{3}{Explain java I/O process.}

\begin{solutionbox}
Java I/O process involves transferring data to and from various sources using streams.

\begin{center}
\captionof{table}{Java I/O Stream Types}

\begin{tabulary}{\linewidth}{|L|L|}
\hline
\textbf{Classification} & \textbf{Types} \\ \hline
Direction & Input, Output \\ \hline
Data Type & Byte Streams, Character Streams \\ \hline
Functionality & Basic, Buffered, Data, Object \\ \hline
\end{tabulary}
\end{center}

\begin{center}
\begin{tikzpicture}[node distance=1cm, auto]

    \node [gtu class] (S) {Stream};
    \node [gtu class, below left=1cm of S] (I) {Input};
    \node [gtu class, below right=1cm of S] (O) {Output};
    \node [gtu class, below=1cm of I] (B1) {Byte};
    \node [gtu class, below=1cm of O] (B2) {Char};

    \path [gtu arrow] (S) -- (I);
    \path [gtu arrow] (S) -- (O);
    \path [gtu arrow] (I) -- (B1);
    \path [gtu arrow] (O) -- (B2);
\end{tikzpicture}
\captionof{figure}{Java I/O Hierarchy}
\end{center}

\begin{itemize}
    \item \keyword{Stream}: Sequence of data flowing between source and destination
    \item \keyword{Buffering}: Improves performance by reducing disk access
\end{itemize}
\end{solutionbox}

\begin{mnemonicbox}
\mnemonic{SBI: Stream Buffered Input/Output}
\end{mnemonicbox}

\questionmarks{5(b OR)}{4}{Explain \code{throw} and \code{finally} in Exception Handling with example.}

\begin{solutionbox}
The \code{throw} and \code{finally} mechanisms control program flow during errors, ensuring graceful execution and resource cleanup.

\begin{center}
\captionof{table}{throw vs finally}

\begin{tabulary}{\linewidth}{|L|L|L|}
\hline
\textbf{Feature} & \textbf{throw} & \textbf{finally} \\ \hline
Purpose & Explicitly throws exception & Ensures code execution \\ \hline
Placement & Inside method & After try-catch blocks \\ \hline
Execution & When condition met & Always, even with return \\ \hline
Usage & Control flow & Resource cleanup \\ \hline
\end{tabulary}
\end{center}

\begin{lstlisting}[language=Java,caption={Throw and Finally in Exception Handling}]
public class ThrowFinallyDemo {
    public static void validateAge(int age) {
        try {
            if (age < 18) {
                throw new ArithmeticException("Not eligible to vote");
            } else {
                System.out.println("Welcome to vote");
            }
        } catch (ArithmeticException e) {
            System.out.println("Exception caught: " + e.getMessage());
        } finally {
            System.out.println("Validation process completed");
        }
    }
    
    public static void main(String[] args) {
        validateAge(15);
        System.out.println("---------");
        validateAge(20);
    }
}
\end{lstlisting}
\end{solutionbox}

\begin{mnemonicbox}
\mnemonic{TERA: Throw Exception Regardless Always finally executes}
\end{mnemonicbox}

\questionmarks{5(c OR)}{7}{Write a java program to display the content of a text file and perform append operation on the text file.}

\begin{solutionbox}
File append operations allow adding new data to the end of an existing file without overwriting its contents.

\begin{lstlisting}[language=Java,caption={File Display and Append Operations}]
import java.io.*;

public class FileAppendDemo {
    public static void main(String[] args) {
        try {
            // Create initial file
            FileWriter writer = new FileWriter("example.txt");
            writer.write("Original content line 1\n");
            writer.write("Original content line 2\n");
            writer.close();
            
            // Display file content
            System.out.println("Original file content:");
            readFile("example.txt");
            
            // Append to file
            FileWriter appendWriter = new FileWriter("example.txt", true);
            appendWriter.write("Appended content line 1\n");
            appendWriter.write("Appended content line 2\n");
            appendWriter.close();
            
            // Display updated content
            System.out.println("\nFile content after append:");
            readFile("example.txt");
            
        } catch (IOException e) {
            System.out.println("An error occurred: " + e.getMessage());
        }
    }
    
    // Method to read and display file content
    public static void readFile(String fileName) {
        try {
            BufferedReader reader = new BufferedReader(new FileReader(fileName));
            String line;
            while ((line = reader.readLine()) != null) {
                System.out.println(line);
            }
            reader.close();
        } catch (IOException e) {
            System.out.println("Error reading file: " + e.getMessage());
        }
    }
}
\end{lstlisting}

\begin{itemize}
    \item \keyword{FileWriter(file, true)}: Second parameter enables append mode
    \item \keyword{BufferedReader}: Efficiently reads text by lines
    \item \keyword{Reusable Method}: Encapsulates reading functionality
\end{itemize}
\end{solutionbox}

\begin{mnemonicbox}
\mnemonic{FTA: Filewriter True Append mode}
\end{mnemonicbox}

\end{document}
