\documentclass[10pt,a4paper]{article}

% content/resources/templates/preamble.tex
\usepackage[margin=0.6in]{geometry}
\author{Milav Dabgar}
\usepackage{amsmath,amssymb,amsthm}
\usepackage{booktabs}
\usepackage{multirow}
\usepackage{xcolor}
\usepackage{tcolorbox}
\tcbuselibrary{breakable,skins}
\usepackage[colorlinks=true,linkcolor=blue]{hyperref}
\usepackage{titlesec}
\usepackage{enumitem}
\usepackage{tikz}
\usepackage{pgfplots}
\usepackage{circuitikz}
\usepackage[version=4]{mhchem}
\usepackage{longtable}
\usepackage{array}
\usepackage{float}
\usepackage{caption}
\usepackage{listings}

\lstset{
  basicstyle=\small\ttfamily,
  breaklines=true,
  breakatwhitespace=false,
  postbreak=\mbox{\textcolor{red}{$\hookrightarrow$}\space},
  float=false,
  numbers=left,
  numberstyle=\tiny\color{gray},
  numbersep=10pt,
  xleftmargin=2em,
  keywordstyle=\color{blue},
  commentstyle=\color{green!60!black},
  stringstyle=\color{purple},
  backgroundcolor=\color{gray!5},
  showstringspaces=false,
  tabsize=2,
  captionpos=b,
  keepspaces=true,
  columns=flexible
}

\pgfplotsset{compat=1.18}
\usetikzlibrary{shapes,arrows,positioning,calc,patterns,decorations.pathmorphing,decorations.markings,arrows.meta}

% Color scheme
\definecolor{headcolor}{RGB}{0,102,204}
\definecolor{keycolor}{RGB}{220,20,60}
\definecolor{solutioncolor}{RGB}{34,139,34}
\definecolor{mnemoniccolor}{RGB}{148,0,211}
\definecolor{codecolor}{RGB}{0,0,100}

% Spacing
\setlength{\parskip}{3pt}
\setlist[itemize]{nosep}
\setlist[enumerate]{nosep}

% Title formatting
\titleformat{\section}{\Large\bfseries\color{headcolor}}{\thesection}{1em}{}
\titleformat{\subsection}{\large\bfseries\color{headcolor}}{\thesubsection}{1em}{}

% Pandoc tightlist compatibility
\providecommand{\tightlist}{%
  \setlength{\itemsep}{0pt}\setlength{\parskip}{0pt}}

% Pandoc longtable compatibility
\newcounter{none}
\def\thenone{}


% content/resources/templates/english-boxes.tex
% This file is currently empty - it exists to maintain consistency with the import structure.
% Add custom environments here if needed in the future.


\begin{document}

\begin{center}
{\Huge\bfseries\color{headcolor} Subject Name Solutions}\\[5pt]
{\LARGE 4343201 -- Summer 2025}\\[3pt]
{\large Semester 1 Study Material}\\[3pt]
{\normalsize\textit{Detailed Solutions and Explanations}}
\end{center}

\vspace{10pt}

\subsection*{Question 1(a) [3 marks]}\label{q1a}

\textbf{Define bit rate, baud rate and bandwidth}

\begin{solutionbox}

{\def\LTcaptype{none} % do not increment counter
\begin{longtable}[]{@{}
  >{\raggedright\arraybackslash}p{(\linewidth - 4\tabcolsep) * \real{0.3793}}
  >{\raggedright\arraybackslash}p{(\linewidth - 4\tabcolsep) * \real{0.4138}}
  >{\raggedright\arraybackslash}p{(\linewidth - 4\tabcolsep) * \real{0.2069}}@{}}
\toprule\noalign{}
\begin{minipage}[b]{\linewidth}\raggedright
Parameter
\end{minipage} & \begin{minipage}[b]{\linewidth}\raggedright
Definition
\end{minipage} & \begin{minipage}[b]{\linewidth}\raggedright
Unit
\end{minipage} \\
\midrule\noalign{}
\endhead
\bottomrule\noalign{}
\endlastfoot
\textbf{Bit Rate} & Number of bits transmitted per second & bps (bits
per second) \\
\textbf{Baud Rate} & Number of signal changes per second & Baud \\
\textbf{Bandwidth} & Range of frequencies in communication channel & Hz
(Hertz) \\
\end{longtable}
}

\begin{itemize}
\tightlist
\item
  \textbf{Bit rate}: Actual data transmission speed
\item
  \textbf{Baud rate}: Modulation rate or symbol rate\\
\item
  \textbf{Bandwidth}: Channel capacity for frequency range
\end{itemize}

\end{solutionbox}
\begin{mnemonicbox}
``Bits Baud Bandwidth - BBB for communication''

\end{mnemonicbox}
\subsection*{Question 1(b) [4 marks]}\label{q1b}

\textbf{Explain TDM with block diagram}

\begin{solutionbox}

\includegraphics[width=1\linewidth,height=\textheight,keepaspectratio]{mermaid-aab4cea3.pdf}

\begin{itemize}
\tightlist
\item
  \textbf{TDM principle}: Multiple signals share single channel by time
  slots
\item
  \textbf{Time slots}: Each input gets dedicated time period
\item
  \textbf{Synchronization}: Transmitter and receiver must be
  synchronized
\item
  \textbf{Applications}: Digital telephone systems, computer networks
\end{itemize}

\end{solutionbox}
\begin{mnemonicbox}
``Time Divided Multiple - TDM shares time''

\end{mnemonicbox}
\subsection*{Question 1(c) [7 marks]}\label{q1c}

\textbf{Explain block diagram of digital communication system}

\begin{solutionbox}

\includegraphics[width=1\linewidth,height=\textheight,keepaspectratio]{mermaid-eee3fe8d.pdf}


{\def\LTcaptype{none} % do not increment counter
\vspace{-5pt}
\captionof{table}{System Components}
\vspace{-10pt}
\begin{longtable}[]{@{}ll@{}}
\toprule\noalign{}
Component & Function \\
\midrule\noalign{}
\endhead
\bottomrule\noalign{}
\endlastfoot
\textbf{Source Encoder} & Converts analog to digital \\
\textbf{Channel Encoder} & Adds error correction codes \\
\textbf{Digital Modulator} & Converts digital to analog signal \\
\textbf{Channel} & Transmission medium \\
\textbf{Digital Demodulator} & Recovers digital signal \\
\textbf{Channel Decoder} & Detects and corrects errors \\
\textbf{Source Decoder} & Reconstructs original signal \\
\end{longtable}
}

\begin{itemize}
\tightlist
\item
  \textbf{Advantages}: Noise immunity, error correction capability
\item
  \textbf{Processing}: Digital signal processing techniques
\item
  \textbf{Reliability}: Better performance over long distances
\end{itemize}

\end{solutionbox}
\begin{mnemonicbox}
``Source Channel Modulate Transmit Demodulate Decode
- SCMTDD''

\end{mnemonicbox}
\subsection*{Question 1(c OR) [7
marks]}\label{question-1c-or-7-marks}

\textbf{Explain different types of Communication channel}

\begin{solutionbox}

\textbf{Channel Types Table:}

{\def\LTcaptype{none} % do not increment counter
\begin{longtable}[]{@{}
  >{\raggedright\arraybackslash}p{(\linewidth - 4\tabcolsep) * \real{0.3111}}
  >{\raggedright\arraybackslash}p{(\linewidth - 4\tabcolsep) * \real{0.3778}}
  >{\raggedright\arraybackslash}p{(\linewidth - 4\tabcolsep) * \real{0.3111}}@{}}
\toprule\noalign{}
\begin{minipage}[b]{\linewidth}\raggedright
Channel Type
\end{minipage} & \begin{minipage}[b]{\linewidth}\raggedright
Characteristics
\end{minipage} & \begin{minipage}[b]{\linewidth}\raggedright
Applications
\end{minipage} \\
\midrule\noalign{}
\endhead
\bottomrule\noalign{}
\endlastfoot
\textbf{Telephone Channel} & 300-3400 Hz bandwidth & Voice
communication \\
\textbf{Coaxial Cable} & High bandwidth, shielded & Cable TV,
Internet \\
\textbf{Optical Fiber} & Very high bandwidth, light signals & Long
distance, high speed \\
\textbf{Wireless Channel} & Radio frequency transmission & Mobile,
satellite \\
\textbf{Satellite Channel} & Long distance, space communication & Global
communication \\
\end{longtable}
}

\begin{itemize}
\tightlist
\item
  \textbf{Bandwidth}: Different channels offer varying frequency ranges
\item
  \textbf{Noise characteristics}: Each channel has specific noise
  properties
\item
  \textbf{Distance capability}: Varies from local to global coverage
\item
  \textbf{Cost factors}: Installation and maintenance costs differ
\end{itemize}

\end{solutionbox}
\begin{mnemonicbox}
``Telephone Coax Optical Wireless Satellite - TCOWS
channels''

\end{mnemonicbox}
\subsection*{Question 2(a) [3 marks]}\label{q2a}

\textbf{Draw the modulation waveform for ASK, FSK and BPSK for the
digital sequence 11100110}

\begin{solutionbox}

\begin{lstlisting}
Digital Data: 1  1  1  0  0  1  1  0
             +--+--+--+  +  +--+--+  +
             |  |  |  |  |  |  |  |  |
             |  |  |  |  |  |  |  |  |
             +  +  +  +--+--+  +  +--+

ASK:         +--+--+--+     +--+--+   
             |  |  |  |     |  |  |   
             |  |  |  |     |  |  |   
             +  +  +  +-----+  +  +---

FSK:         \cap\cap\cap\cap\cap\cap\cap\cap\cap    \cap\cap\cap\cap\cap\cap\cap   
             \cup\cup\cup\cup     \cup\cup\cup\cup     \cup\cup\cup\cup
             High freq   Low   High  Low

BPSK:        +--+--+--+     +--+--+   
             |  |  |  |     |  |  |   
             +  +  +  +-----+  +  +---
             -  -  -  -----   -  ----
\end{lstlisting}

\end{solutionbox}
\begin{mnemonicbox}
``ASK Amplitude, FSK Frequency, BPSK Phase - AFP
modulation''

\end{mnemonicbox}
\subsection*{Question 2(b) [4 marks]}\label{q2b}

\textbf{Explain the basic principle and generation of frequency shift
keying (FSK) signal}

\begin{solutionbox}

\textbf{FSK Generation Table:}

{\def\LTcaptype{none} % do not increment counter
\begin{longtable}[]{@{}lll@{}}
\toprule\noalign{}
Binary Data & Frequency & Output \\
\midrule\noalign{}
\endhead
\bottomrule\noalign{}
\endlastfoot
Logic `1' & f_{1} (High frequency) & High freq carrier \\
Logic `0' & f_{0} (Low frequency) & Low freq carrier \\
\end{longtable}
}

\includegraphics[width=1\linewidth,height=\textheight,keepaspectratio]{mermaid-ef4f9ab6.pdf}

\begin{itemize}
\tightlist
\item
  \textbf{Principle}: Binary data controls carrier frequency
\item
  \textbf{Two frequencies}: f_{1} for `1' and f_{0} for `0'
\item
  \textbf{Constant amplitude}: Only frequency changes
\item
  \textbf{Detection}: Frequency discrimination at receiver
\end{itemize}

\end{solutionbox}
\begin{mnemonicbox}
``Frequency Shifts Key - FSK frequency control''

\end{mnemonicbox}
\subsection*{Question 2(c) [7 marks]}\label{q2c}

\textbf{Explain the working of QPSK modulator and Demodulator with block
diagram and constellation diagram}

\begin{solutionbox}

\textbf{QPSK Modulator Block Diagram:}

\includegraphics[width=1\linewidth,height=\textheight,keepaspectratio]{mermaid-d3ec0247.pdf}

\textbf{Constellation Diagram:}

\begin{lstlisting}
     Q
     |
  01 * * 00
     |
-----*-----  I
     |
  11 * * 10
     |
\end{lstlisting}

\textbf{QPSK Truth Table:}

{\def\LTcaptype{none} % do not increment counter
\begin{longtable}[]{@{}llll@{}}
\toprule\noalign{}
I & Q & Phase & Symbol \\
\midrule\noalign{}
\endhead
\bottomrule\noalign{}
\endlastfoot
0 & 0 & 45^\circ & 00 \\
0 & 1 & 135^\circ & 01 \\
1 & 1 & 225^\circ & 11 \\
1 & 0 & 315^\circ & 10 \\
\end{longtable}
}

\begin{itemize}
\tightlist
\item
  \textbf{Four phases}: 45^\circ, 135^\circ, 225^\circ, 315^\circ
\item
  \textbf{Two bits per symbol}: Higher data rate
\item
  \textbf{Constant envelope}: Amplitude remains constant
\item
  \textbf{Demodulation}: Phase detection and parallel to serial
  conversion
\end{itemize}

\end{solutionbox}
\begin{mnemonicbox}
``Quadrature Phase Shift Key - QPSK four phases''

\end{mnemonicbox}
\subsection*{Question 2(a OR) [3
marks]}\label{question-2a-or-3-marks}

\textbf{Draw the block diagram of ASK modulator and describe working of
it}

\begin{solutionbox}

\includegraphics[width=1\linewidth,height=\textheight,keepaspectratio]{mermaid-f3524abb.pdf}

\begin{itemize}
\tightlist
\item
  \textbf{Working principle}: Digital data controls carrier amplitude
\item
  \textbf{Logic `1'}: Carrier transmitted with full amplitude
\item
  \textbf{Logic `0'}: No carrier transmitted (zero amplitude)
\item
  \textbf{Simple implementation}: Uses analog switch or multiplier
\end{itemize}

\end{solutionbox}
\begin{mnemonicbox}
``Amplitude Shift Key - ASK amplitude control''

\end{mnemonicbox}
\subsection*{Question 2(b OR) [4
marks]}\label{question-2b-or-4-marks}

\textbf{Explain the principal of 16-QAM and draw the constellation
diagram}

\begin{solutionbox}

\textbf{16-QAM Constellation:}

\begin{lstlisting}
     Q
     |
  *  *  *  *
     |
  *  *  *  *
-----*-----  I
     |
  *  *  *  *
     |
  *  *  *  *
\end{lstlisting}

\textbf{16-QAM Characteristics Table:}

{\def\LTcaptype{none} % do not increment counter
\begin{longtable}[]{@{}ll@{}}
\toprule\noalign{}
Parameter & Value \\
\midrule\noalign{}
\endhead
\bottomrule\noalign{}
\endlastfoot
\textbf{Bits per symbol} & 4 bits \\
\textbf{Number of states} & 16 \\
\textbf{Amplitude levels} & 4 levels \\
\textbf{Phase levels} & 4 phases \\
\end{longtable}
}

\begin{itemize}
\tightlist
\item
  \textbf{Principle}: Combines amplitude and phase modulation
\item
  \textbf{Higher data rate}: 4 bits per symbol
\item
  \textbf{Complex modulation}: Requires precise amplitude and phase
  control
\item
  \textbf{Applications}: High-speed digital communication
\end{itemize}

\end{solutionbox}
\begin{mnemonicbox}
``16 Quadrature Amplitude Modulation - 16QAM complex
signals''

\end{mnemonicbox}
\subsection*{Question 2(c OR) [7
marks]}\label{question-2c-or-7-marks}

\textbf{Explain working of BPSK modulator and demodulator with block
diagram and waveform}

\begin{solutionbox}

\textbf{BPSK Modulator:}

\includegraphics[width=1\linewidth,height=\textheight,keepaspectratio]{mermaid-6a68de3b.pdf}

\textbf{BPSK Demodulator:}

\includegraphics[width=1\linewidth,height=\textheight,keepaspectratio]{mermaid-2a025131.pdf}

\textbf{BPSK Waveforms:}

\begin{lstlisting}
Data:    1    0    1    0
        +----+    +----+
        |    |    |    |
        +    +----+    +----

Carrier: \cap\cap\cap\cap\cap\cap\cap\cap\cap\cap\cap\cap\cap\cap\cap\cap
         \cup\cup\cup\cup\cup\cup\cup\cup\cup\cup\cup\cup\cup\cup\cup\cup

BPSK:    \cap\cap\cap\cap     \cap\cap\cap\cap    
         \cup\cup\cup\cup\cup\cup\cup\cup\cup\cup\cup\cup\cup\cup\cup\cup
\end{lstlisting}

\begin{itemize}
\tightlist
\item
  \textbf{Phase shift}: 180^\circ between `1' and `0'
\item
  \textbf{Coherent detection}: Requires synchronized carrier
\item
  \textbf{Best performance}: Lowest bit error rate
\item
  \textbf{Constant envelope}: Amplitude remains constant
\end{itemize}

\end{solutionbox}
\begin{mnemonicbox}
``Binary Phase Shift Key - BPSK two phases''

\end{mnemonicbox}
\subsection*{Question 3(a) [3 marks]}\label{q3a}

\textbf{Define Channel Capacity in terms of SNR and explain importance
of it}

\begin{solutionbox}

\textbf{Shannon's Channel Capacity Formula:}

{\def\LTcaptype{none} % do not increment counter
\begin{longtable}[]{@{}ll@{}}
\toprule\noalign{}
Formula & C = B log_{2}(1 + S/N) \\
\midrule\noalign{}
\endhead
\bottomrule\noalign{}
\endlastfoot
\textbf{C} & Channel capacity (bps) \\
\textbf{B} & Bandwidth (Hz) \\
\textbf{S/N} & Signal-to-Noise ratio \\
\end{longtable}
}

\begin{itemize}
\tightlist
\item
  \textbf{Importance}: Maximum theoretical data rate
\item
  \textbf{SNR effect}: Higher SNR allows higher capacity
\item
  \textbf{Bandwidth trade-off}: Can exchange bandwidth for SNR
\item
  \textbf{Design limit}: Sets upper bound for system design
\end{itemize}

\end{solutionbox}
\begin{mnemonicbox}
``Channel Capacity Shannon's Limit - CCSL''

\end{mnemonicbox}
\subsection*{Question 3(b) [4 marks]}\label{q3b}

\textbf{Describe Asynchronous and synchronous serial data communication
techniques}

\begin{solutionbox}

\textbf{Comparison Table:}

{\def\LTcaptype{none} % do not increment counter
\begin{longtable}[]{@{}lll@{}}
\toprule\noalign{}
Parameter & Synchronous & Asynchronous \\
\midrule\noalign{}
\endhead
\bottomrule\noalign{}
\endlastfoot
\textbf{Clock} & Separate clock signal & No separate clock \\
\textbf{Start/Stop bits} & Not required & Start and stop bits \\
\textbf{Speed} & Higher & Lower \\
\textbf{Cost} & Higher & Lower \\
\end{longtable}
}

\begin{itemize}
\tightlist
\item
  \textbf{Synchronous}: Clock synchronization required
\item
  \textbf{Asynchronous}: Self-synchronizing with start/stop bits
\item
  \textbf{Applications}: Synchronous for high-speed, Asynchronous for
  simple systems
\item
  \textbf{Efficiency}: Synchronous more efficient, Asynchronous more
  flexible
\end{itemize}

\end{solutionbox}
\begin{mnemonicbox}
``Sync Clock, Async Start-Stop - SCSS''

\end{mnemonicbox}
\subsection*{Question 3(c) [7 marks]}\label{q3c}

\textbf{Explain Huffman coding with help of suitable example}

\begin{solutionbox}

\textbf{Example: Characters A, B, C, D with probabilities 0.4, 0.3, 0.2,
0.1}

\textbf{Step-by-step Huffman Tree Construction:}

\begin{lstlisting}
Step 1: List probabilities
A: 0.4, B: 0.3, C: 0.2, D: 0.1

Step 2: Combine lowest
       0.3
      /   \
   C:0.2  D:0.1

Step 3: Continue combining
       0.6
      /   \
   B:0.3   0.3
          /   \
       C:0.2  D:0.1

Step 4: Final tree
        1.0
       /   \
    A:0.4   0.6
           /   \
        B:0.3   0.3
               /   \
            C:0.2  D:0.1
\end{lstlisting}

\textbf{Huffman Codes Table:}

{\def\LTcaptype{none} % do not increment counter
\begin{longtable}[]{@{}lll@{}}
\toprule\noalign{}
Character & Probability & Code \\
\midrule\noalign{}
\endhead
\bottomrule\noalign{}
\endlastfoot
A & 0.4 & 0 \\
B & 0.3 & 10 \\
C & 0.2 & 110 \\
D & 0.1 & 111 \\
\end{longtable}
}

\begin{itemize}
\tightlist
\item
  \textbf{Average code length}: 0.4\times1 + 0.3\times2 + 0.2\times3 + 0.1\times3 = 1.9 bits
\item
  \textbf{Compression achieved}: Reduces average bits per character
\item
  \textbf{Prefix property}: No code is prefix of another
\end{itemize}

\end{solutionbox}
\begin{mnemonicbox}
``Huffman Minimum Average Length - HMAL''

\end{mnemonicbox}
\subsection*{Question 3(a OR) [3
marks]}\label{question-3a-or-3-marks}

\textbf{State the significance of probability and entropy in
communication}

\begin{solutionbox}

\textbf{Significance Table:}

{\def\LTcaptype{none} % do not increment counter
\begin{longtable}[]{@{}ll@{}}
\toprule\noalign{}
Concept & Significance \\
\midrule\noalign{}
\endhead
\bottomrule\noalign{}
\endlastfoot
\textbf{Probability} & Measures likelihood of information occurrence \\
\textbf{Entropy} & Measures average information content \\
\textbf{Maximum Entropy} & Occurs with equal probability events \\
\end{longtable}
}

\begin{itemize}
\tightlist
\item
  \textbf{Information content}: I = log_{2}(1/P) bits
\item
  \textbf{Entropy formula}: H = -Σ P(x) log_{2} P(x)
\item
  \textbf{Channel design}: Helps optimize communication systems
\item
  \textbf{Coding efficiency}: Guides source coding design
\end{itemize}

\end{solutionbox}
\begin{mnemonicbox}
``Probability Entropy Information - PEI
communication''

\end{mnemonicbox}
\subsection*{Question 3(b OR) [4
marks]}\label{question-3b-or-4-marks}

\textbf{Explain simplex, half duplex and full duplex data transmission
mode}

\begin{solutionbox}

\textbf{Transmission Modes Table:}

{\def\LTcaptype{none} % do not increment counter
\begin{longtable}[]{@{}
  >{\raggedright\arraybackslash}p{(\linewidth - 6\tabcolsep) * \real{0.1714}}
  >{\raggedright\arraybackslash}p{(\linewidth - 6\tabcolsep) * \real{0.3143}}
  >{\raggedright\arraybackslash}p{(\linewidth - 6\tabcolsep) * \real{0.2571}}
  >{\raggedright\arraybackslash}p{(\linewidth - 6\tabcolsep) * \real{0.2571}}@{}}
\toprule\noalign{}
\begin{minipage}[b]{\linewidth}\raggedright
Mode
\end{minipage} & \begin{minipage}[b]{\linewidth}\raggedright
Direction
\end{minipage} & \begin{minipage}[b]{\linewidth}\raggedright
Example
\end{minipage} & \begin{minipage}[b]{\linewidth}\raggedright
Diagram
\end{minipage} \\
\midrule\noalign{}
\endhead
\bottomrule\noalign{}
\endlastfoot
\textbf{Simplex} & One-way only & Radio broadcast & A \rightarrow B \\
\textbf{Half Duplex} & Both ways, not simultaneous & Walkie-talkie & A ⇄
B \\
\textbf{Full Duplex} & Both ways, simultaneous & Telephone & A ⇌ B \\
\end{longtable}
}

\begin{itemize}
\tightlist
\item
  \textbf{Simplex}: Unidirectional communication
\item
  \textbf{Half duplex}: Bidirectional but alternate
\item
  \textbf{Full duplex}: Simultaneous bidirectional
\item
  \textbf{Bandwidth requirement}: Full duplex needs twice the bandwidth
\end{itemize}

\end{solutionbox}
\begin{mnemonicbox}
``Simple Half Full - SHF transmission modes''

\end{mnemonicbox}
\subsection*{Question 3(c OR) [7
marks]}\label{question-3c-or-7-marks}

\textbf{Explain Shannon Fano coding with help of suitable example}

\begin{solutionbox}

\textbf{Example: Characters A, B, C, D with probabilities 0.4, 0.3, 0.2,
0.1}

\textbf{Shannon-Fano Algorithm Steps:}

\begin{lstlisting}
Step 1: Arrange in descending order
A: 0.4, B: 0.3, C: 0.2, D: 0.1

Step 2: Divide into two groups
Group 1: A(0.4) \rightarrow Code starts with 0
Group 2: B(0.3), C(0.2), D(0.1) \rightarrow Code starts with 1

Step 3: Subdivide Group 2
B(0.3) \rightarrow Code: 10
C(0.2), D(0.1) \rightarrow Code starts with 11

Step 4: Final subdivision
C(0.2) \rightarrow Code: 110
D(0.1) \rightarrow Code: 111
\end{lstlisting}

\textbf{Shannon-Fano Codes Table:}

{\def\LTcaptype{none} % do not increment counter
\begin{longtable}[]{@{}lll@{}}
\toprule\noalign{}
Character & Probability & Code \\
\midrule\noalign{}
\endhead
\bottomrule\noalign{}
\endlastfoot
A & 0.4 & 0 \\
B & 0.3 & 10 \\
C & 0.2 & 110 \\
D & 0.1 & 111 \\
\end{longtable}
}

\begin{itemize}
\tightlist
\item
  \textbf{Average length}: Same as Huffman (1.9 bits)
\item
  \textbf{Top-down approach}: Divides from root to leaves
\item
  \textbf{Not always optimal}: Huffman is generally better
\end{itemize}

\end{solutionbox}
\begin{mnemonicbox}
``Shannon Fano Top-Down - SFTD coding''

\end{mnemonicbox}
\subsection*{Question 4(a) [3 marks]}\label{q4a}

\textbf{Describe Ethical and Privacy Considerations in Data
Communication}

\begin{solutionbox}

\textbf{Ethics and Privacy Table:}

{\def\LTcaptype{none} % do not increment counter
\begin{longtable}[]{@{}ll@{}}
\toprule\noalign{}
Aspect & Consideration \\
\midrule\noalign{}
\endhead
\bottomrule\noalign{}
\endlastfoot
\textbf{Data Privacy} & User consent, data protection \\
\textbf{Security} & Encryption, access control \\
\textbf{Transparency} & Clear data usage policies \\
\end{longtable}
}

\begin{itemize}
\tightlist
\item
  \textbf{Privacy rights}: Users control over personal data
\item
  \textbf{Ethical use}: Responsible data handling practices
\item
  \textbf{Legal compliance}: Following data protection laws
\item
  \textbf{Security measures}: Protecting against unauthorized access
\end{itemize}

\end{solutionbox}
\begin{mnemonicbox}
``Privacy Security Transparency - PST ethics''

\end{mnemonicbox}
\subsection*{Question 4(b) [4 marks]}\label{q4b}

\textbf{Explain RS 232 standard with pin diagram}

\begin{solutionbox}

\textbf{RS-232 Pin Configuration (DB-9):}

{\def\LTcaptype{none} % do not increment counter
\begin{longtable}[]{@{}lll@{}}
\toprule\noalign{}
Pin & Signal & Function \\
\midrule\noalign{}
\endhead
\bottomrule\noalign{}
\endlastfoot
1 & DCD & Data Carrier Detect \\
2 & RXD & Receive Data \\
3 & TXD & Transmit Data \\
4 & DTR & Data Terminal Ready \\
5 & GND & Ground \\
6 & DSR & Data Set Ready \\
7 & RTS & Request To Send \\
8 & CTS & Clear To Send \\
9 & RI & Ring Indicator \\
\end{longtable}
}

\begin{itemize}
\tightlist
\item
  \textbf{Voltage levels}: +3V to +25V for `0', -3V to -25V for `1'
\item
  \textbf{Maximum distance}: 50 feet at 19.2 kbps
\item
  \textbf{Applications}: Serial communication between computers and
  modems
\end{itemize}

\end{solutionbox}
\begin{mnemonicbox}
``RS-232 Nine pins Serial - RNS communication''

\end{mnemonicbox}
\subsection*{Question 4(c) [7 marks]}\label{q4c}

\textbf{Explain Hamming code with help of suitable example}

\begin{solutionbox}

\textbf{Example: 4-bit data 1011}

\textbf{Hamming Code Construction:}

{\def\LTcaptype{none} % do not increment counter
\begin{longtable}[]{@{}llllllll@{}}
\toprule\noalign{}
Position & 1 & 2 & 3 & 4 & 5 & 6 & 7 \\
\midrule\noalign{}
\endhead
\bottomrule\noalign{}
\endlastfoot
\textbf{Type} & P1 & P2 & D1 & P4 & D2 & D3 & D4 \\
\textbf{Value} & ? & ? & 1 & ? & 0 & 1 & 1 \\
\end{longtable}
}

\textbf{Parity Calculations:}

\begin{itemize}
\tightlist
\item
  \textbf{P1} (positions 1,3,5,7): P1 \oplus 1 \oplus 0 \oplus 1 = 0, so P1 = 0
\item
  \textbf{P2} (positions 2,3,6,7): P2 \oplus 1 \oplus 1 \oplus 1 = 1, so P2 = 1\\
\item
  \textbf{P4} (positions 4,5,6,7): P4 \oplus 0 \oplus 1 \oplus 1 = 0, so P4 = 0
\end{itemize}

\textbf{Final Hamming Code: 0110111}

\textbf{Error Detection Process:}

\begin{itemize}
\item
  Calculate syndrome S = S4S2S1
\item
  If S = 000, no error
\item
  If S \neq 000, error at position indicated by S
\item
  \textbf{Single error correction}: Can correct one-bit errors
\item
  \textbf{Double error detection}: Can detect two-bit errors
\item
  \textbf{Systematic approach}: Organized parity bit placement
\end{itemize}

\end{solutionbox}
\begin{mnemonicbox}
``Hamming Single Error Correction - HSEC''

\end{mnemonicbox}
\subsection*{Question 4(a OR) [3
marks]}\label{question-4a-or-3-marks}

\textbf{Define Edge Computing and explain feature of it}

\begin{solutionbox}

\textbf{Edge Computing Features:}

{\def\LTcaptype{none} % do not increment counter
\begin{longtable}[]{@{}ll@{}}
\toprule\noalign{}
Feature & Description \\
\midrule\noalign{}
\endhead
\bottomrule\noalign{}
\endlastfoot
\textbf{Low Latency} & Processing near data source \\
\textbf{Bandwidth Saving} & Reduces network traffic \\
\textbf{Real-time Processing} & Immediate data analysis \\
\end{longtable}
}

\begin{itemize}
\tightlist
\item
  \textbf{Definition}: Computing at network edge, close to data sources
\item
  \textbf{Reduced latency}: Faster response times
\item
  \textbf{Distributed processing}: Reduces central server load
\item
  \textbf{Applications}: IoT, autonomous vehicles, smart cities
\end{itemize}

\end{solutionbox}
\begin{mnemonicbox}
``Edge Low-latency Real-time - ELR computing''

\end{mnemonicbox}
\subsection*{Question 4(b OR) [4
marks]}\label{question-4b-or-4-marks}

\textbf{Explain needs of multimedia processing for communication and
various file formats of different data}

\begin{solutionbox}

\textbf{Multimedia File Formats Table:}

{\def\LTcaptype{none} % do not increment counter
\begin{longtable}[]{@{}lll@{}}
\toprule\noalign{}
Data Type & Formats & Characteristics \\
\midrule\noalign{}
\endhead
\bottomrule\noalign{}
\endlastfoot
\textbf{Audio} & MP3, WAV, AAC & Compressed/Uncompressed \\
\textbf{Video} & MP4, AVI, MOV & Different codecs \\
\textbf{Image} & JPEG, PNG, GIF & Lossy/Lossless compression \\
\textbf{Text} & TXT, PDF, DOC & Various encodings \\
\end{longtable}
}

\begin{itemize}
\tightlist
\item
  \textbf{Processing needs}: Compression, format conversion, quality
  optimization
\item
  \textbf{Bandwidth optimization}: Reducing file sizes for transmission
\item
  \textbf{Quality preservation}: Maintaining acceptable quality levels
\item
  \textbf{Compatibility}: Supporting multiple devices and platforms
\end{itemize}

\end{solutionbox}
\begin{mnemonicbox}
``Audio Video Image Text - AVIT multimedia''

\end{mnemonicbox}
\subsection*{Question 4(c OR) [7
marks]}\label{question-4c-or-7-marks}

\textbf{Explain different Line coding with help of waveform}

\begin{solutionbox}

\textbf{Line Coding Waveforms for data 1011:}

\begin{lstlisting}
Data:        1    0    1    1
            +----+    +----+----
            |    |    |    |
            +    +----+    +

NRZ-L:      +----+    +----+----
            |    |    |    |
            +    +----+    +

NRZ-I:      +----+----+    +
            |    |    |    |
            +    +    +----+----

RZ:         +--+ +    +--+ +--+
            |  | |    |  | |  |
            +  +-+----+  +-+  +

Manchester: +--+    --+ +--+    +
            |  |   |  | |  |   |
            +  +---+  +-+  +---+
\end{lstlisting}

\textbf{Line Coding Comparison:}

{\def\LTcaptype{none} % do not increment counter
\begin{longtable}[]{@{}llll@{}}
\toprule\noalign{}
Code Type & Bandwidth & DC Component & Synchronization \\
\midrule\noalign{}
\endhead
\bottomrule\noalign{}
\endlastfoot
\textbf{NRZ-L} & Low & Present & Poor \\
\textbf{NRZ-I} & Low & Present & Poor \\
\textbf{RZ} & High & Present & Good \\
\textbf{Manchester} & High & Absent & Excellent \\
\end{longtable}
}

\begin{itemize}
\tightlist
\item
  \textbf{NRZ}: Non-Return-to-Zero, simple but has DC component
\item
  \textbf{RZ}: Return-to-Zero, better synchronization
\item
  \textbf{Manchester}: Self-synchronizing, no DC component
\item
  \textbf{Selection criteria}: Bandwidth, synchronization, complexity
\end{itemize}

\end{solutionbox}
\begin{mnemonicbox}
``NRZ RZ Manchester - NRM line codes''

\end{mnemonicbox}
\subsection*{Question 5(a) [3 marks]}\label{q5a}

\textbf{Explain concept of spread spectrum technology}

\begin{solutionbox}

\textbf{Spread Spectrum Characteristics:}

{\def\LTcaptype{none} % do not increment counter
\begin{longtable}[]{@{}ll@{}}
\toprule\noalign{}
Parameter & Description \\
\midrule\noalign{}
\endhead
\bottomrule\noalign{}
\endlastfoot
\textbf{Bandwidth Spreading} & Signal spread over wide frequency \\
\textbf{Low Power Density} & Power distributed across spectrum \\
\textbf{Interference Resistance} & Resistant to jamming \\
\end{longtable}
}

\begin{itemize}
\tightlist
\item
  \textbf{Principle}: Spreads signal over much wider bandwidth than
  required
\item
  \textbf{Techniques}: Direct Sequence (DS-SS), Frequency Hopping
  (FH-SS)
\item
  \textbf{Advantages}: Security, interference resistance, multiple
  access
\item
  \textbf{Applications}: GPS, CDMA, WiFi, Bluetooth
\end{itemize}

\end{solutionbox}
\begin{mnemonicbox}
``Spread Spectrum Security - SSS technology''

\end{mnemonicbox}
\subsection*{Question 5(b) [4 marks]}\label{q5b}

\textbf{Explain block diagram of satellite communication}

\begin{solutionbox}

\includegraphics[width=1\linewidth,height=\textheight,keepaspectratio]{mermaid-301cf134.pdf}

\textbf{Satellite Communication Components:}

{\def\LTcaptype{none} % do not increment counter
\begin{longtable}[]{@{}ll@{}}
\toprule\noalign{}
Component & Function \\
\midrule\noalign{}
\endhead
\bottomrule\noalign{}
\endlastfoot
\textbf{Earth Station} & Ground-based transmit/receive \\
\textbf{Uplink} & Earth to satellite transmission \\
\textbf{Transponder} & Satellite receiver-transmitter \\
\textbf{Downlink} & Satellite to earth transmission \\
\end{longtable}
}

\begin{itemize}
\tightlist
\item
  \textbf{Frequency bands}: C-band, Ku-band, Ka-band
\item
  \textbf{Coverage area}: Large geographical coverage
\item
  \textbf{Applications}: Broadcasting, telephony, internet
\item
  \textbf{Advantages}: Wide coverage, long-distance communication
\end{itemize}

\end{solutionbox}
\begin{mnemonicbox}
``Earth Uplink Transponder Downlink - EUTD
satellite''

\end{mnemonicbox}
\subsection*{Question 5(c) [7 marks]}\label{q5c}

\textbf{Demonstrate model of Multimedia Communications and elements of
Multimedia system}

\begin{solutionbox}

\textbf{Multimedia Communication Model:}

\includegraphics[width=1\linewidth,height=\textheight,keepaspectratio]{mermaid-06acb870.pdf}

\textbf{Multimedia System Elements:}

{\def\LTcaptype{none} % do not increment counter
\begin{longtable}[]{@{}lll@{}}
\toprule\noalign{}
Element & Function & Examples \\
\midrule\noalign{}
\endhead
\bottomrule\noalign{}
\endlastfoot
\textbf{Capture} & Input multimedia data & Camera, microphone \\
\textbf{Storage} & Store multimedia files & Hard disk, memory \\
\textbf{Processing} & Edit and manipulate & Video editing software \\
\textbf{Communication} & Transmit multimedia & Networks, internet \\
\textbf{Presentation} & Display multimedia & Monitor, speakers \\
\end{longtable}
}

\begin{itemize}
\tightlist
\item
  \textbf{Synchronization}: Audio-video synchronization critical
\item
  \textbf{Compression}: Reduces bandwidth requirements
\item
  \textbf{Quality of Service}: Maintains acceptable quality
\item
  \textbf{Real-time constraints}: Time-sensitive data delivery
\end{itemize}

\end{solutionbox}
\begin{mnemonicbox}
``Capture Store Process Communicate Present - CSPCP
multimedia''

\end{mnemonicbox}
\subsection*{Question 5(a OR) [3
marks]}\label{question-5a-or-3-marks}

\textbf{Explain importance of Block chain in Communication Security}

\begin{solutionbox}

\textbf{Blockchain Security Features:}

{\def\LTcaptype{none} % do not increment counter
\begin{longtable}[]{@{}ll@{}}
\toprule\noalign{}
Feature & Benefit \\
\midrule\noalign{}
\endhead
\bottomrule\noalign{}
\endlastfoot
\textbf{Decentralization} & No single point of failure \\
\textbf{Immutability} & Cannot alter past records \\
\textbf{Transparency} & All transactions visible \\
\end{longtable}
}

\begin{itemize}
\tightlist
\item
  \textbf{Cryptographic security}: Hash functions and digital signatures
\item
  \textbf{Distributed ledger}: Multiple copies prevent tampering
\item
  \textbf{Smart contracts}: Automated security protocols
\item
  \textbf{Applications}: Secure messaging, identity verification
\end{itemize}

\end{solutionbox}
\begin{mnemonicbox}
``Blockchain Distributed Immutable - BDI security''

\end{mnemonicbox}
\subsection*{Question 5(b OR) [4
marks]}\label{question-5b-or-4-marks}

\textbf{Explain important elements, features and advantages of 5G
technology}

\begin{solutionbox}

\textbf{5G Technology Elements:}

{\def\LTcaptype{none} % do not increment counter
\begin{longtable}[]{@{}ll@{}}
\toprule\noalign{}
Element & Specification \\
\midrule\noalign{}
\endhead
\bottomrule\noalign{}
\endlastfoot
\textbf{Speed} & Up to 10 Gbps \\
\textbf{Latency} & Less than 1 ms \\
\textbf{Connections} & 1 million devices per km^{2} \\
\textbf{Reliability} & 99.999\% availability \\
\end{longtable}
}

\textbf{Key Features:}

\begin{itemize}
\tightlist
\item
  \textbf{Enhanced Mobile Broadband}: Ultra-high-speed internet
\item
  \textbf{Ultra-Reliable Low Latency}: Critical applications
\item
  \textbf{Massive Machine Communication}: IoT connectivity
\item
  \textbf{Network Slicing}: Customized network services
\end{itemize}

\textbf{Advantages:}

\begin{itemize}
\tightlist
\item
  \textbf{Higher capacity}: More simultaneous users
\item
  \textbf{Energy efficiency}: Better battery life for devices
\item
  \textbf{New applications}: AR/VR, autonomous vehicles
\end{itemize}

\end{solutionbox}
\begin{mnemonicbox}
``5G Speed Latency Connections - SLC features''

\end{mnemonicbox}
\subsection*{Question 5(c OR) [7
marks]}\label{question-5c-or-7-marks}

\textbf{Compare RS 232, RS 422 and RS 485 standard}

\begin{solutionbox}

\textbf{RS Standards Comparison Table:}

{\def\LTcaptype{none} % do not increment counter
\begin{longtable}[]{@{}llll@{}}
\toprule\noalign{}
Parameter & RS-232 & RS-422 & RS-485 \\
\midrule\noalign{}
\endhead
\bottomrule\noalign{}
\endlastfoot
\textbf{Mode} & Single-ended & Differential & Differential \\
\textbf{Max Distance} & 50 feet & 4000 feet & 4000 feet \\
\textbf{Max Speed} & 20 kbps & 10 Mbps & 10 Mbps \\
\textbf{Drivers} & 1 & 1 & 32 \\
\textbf{Receivers} & 1 & 10 & 32 \\
\textbf{Topology} & Point-to-Point & Point-to-Multipoint & Multipoint \\
\end{longtable}
}

\textbf{Voltage Levels:}

{\def\LTcaptype{none} % do not increment counter
\begin{longtable}[]{@{}lll@{}}
\toprule\noalign{}
Standard & Logic 1 & Logic 0 \\
\midrule\noalign{}
\endhead
\bottomrule\noalign{}
\endlastfoot
\textbf{RS-232} & -3V to -25V & +3V to +25V \\
\textbf{RS-422} & Differential \textgreater{} +200mV & Differential
\textless{} -200mV \\
\textbf{RS-485} & Differential \textgreater{} +200mV & Differential
\textless{} -200mV \\
\end{longtable}
}

\textbf{Applications:}

\begin{itemize}
\tightlist
\item
  \textbf{RS-232}: Computer serial ports, modems
\item
  \textbf{RS-422}: Industrial automation, long-distance
\item
  \textbf{RS-485}: Building automation, industrial networks
\end{itemize}

\textbf{Key Differences:}

\begin{itemize}
\tightlist
\item
  \textbf{Noise immunity}: Differential signaling in RS-422/485 better
  than RS-232
\item
  \textbf{Distance capability}: RS-422/485 much longer than RS-232
\item
  \textbf{Multi-drop capability}: RS-485 supports multiple devices
\item
  \textbf{Cost}: RS-232 cheapest, RS-485 most complex
\end{itemize}

\end{solutionbox}
\begin{mnemonicbox}
``RS-232 Simple, RS-422 Long, RS-485 Multi - SLM
standards''

\end{mnemonicbox}

\end{document}
