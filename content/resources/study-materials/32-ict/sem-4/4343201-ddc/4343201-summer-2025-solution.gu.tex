\documentclass{article}

% content/resources/templates/preamble.tex
\usepackage[margin=0.6in]{geometry}
\author{Milav Dabgar}
\usepackage{amsmath,amssymb,amsthm}
\usepackage{booktabs}
\usepackage{multirow}
\usepackage{xcolor}
\usepackage{tcolorbox}
\tcbuselibrary{breakable,skins}
\usepackage[colorlinks=true,linkcolor=blue]{hyperref}
\usepackage{titlesec}
\usepackage{enumitem}
\usepackage{tikz}
\usepackage{pgfplots}
\usepackage{circuitikz}
\usepackage[version=4]{mhchem}
\usepackage{longtable}
\usepackage{array}
\usepackage{float}
\usepackage{caption}
\usepackage{listings}

\lstset{
  basicstyle=\small\ttfamily,
  breaklines=true,
  breakatwhitespace=false,
  postbreak=\mbox{\textcolor{red}{$\hookrightarrow$}\space},
  float=false,
  numbers=left,
  numberstyle=\tiny\color{gray},
  numbersep=10pt,
  xleftmargin=2em,
  keywordstyle=\color{blue},
  commentstyle=\color{green!60!black},
  stringstyle=\color{purple},
  backgroundcolor=\color{gray!5},
  showstringspaces=false,
  tabsize=2,
  captionpos=b,
  keepspaces=true,
  columns=flexible
}

\pgfplotsset{compat=1.18}
\usetikzlibrary{shapes,arrows,positioning,calc,patterns,decorations.pathmorphing,decorations.markings,arrows.meta}

% Color scheme
\definecolor{headcolor}{RGB}{0,102,204}
\definecolor{keycolor}{RGB}{220,20,60}
\definecolor{solutioncolor}{RGB}{34,139,34}
\definecolor{mnemoniccolor}{RGB}{148,0,211}
\definecolor{codecolor}{RGB}{0,0,100}

% Spacing
\setlength{\parskip}{3pt}
\setlist[itemize]{nosep}
\setlist[enumerate]{nosep}

% Title formatting
\titleformat{\section}{\Large\bfseries\color{headcolor}}{\thesection}{1em}{}
\titleformat{\subsection}{\large\bfseries\color{headcolor}}{\thesubsection}{1em}{}

% Pandoc tightlist compatibility
\providecommand{\tightlist}{%
  \setlength{\itemsep}{0pt}\setlength{\parskip}{0pt}}

% Pandoc longtable compatibility
\newcounter{none}
\def\thenone{}


% content/resources/templates/gujarati-boxes.tex
\usepackage{fontspec}
\usepackage{polyglossia}

% Set Gujarati as main language (document is primarily in Gujarati)
% Note: gloss-gujarati.ldf doesn't exist in polyglossia, but it will use hyphenation patterns
\setdefaultlanguage{gujarati}
\setotherlanguage{english}

% Configure Gujarati font properly
% Use Language=Default to prevent polyglossia from trying to add language-specific features
% that don't exist for Gujarati, which causes "empty feature" warnings
\newfontfamily\gujaratifont[Script=Gujarati,AutoFakeBold=2.5,AutoFakeSlant=0.3]{Noto Sans Gujarati}
\setmainfont[Script=Gujarati,AutoFakeBold=2.5,AutoFakeSlant=0.3]{Noto Sans Gujarati}
% Use Noto Sans Gujarati for monospace to support Gujarati in text
\setmonofont[Scale=0.9]{Noto Sans Gujarati}

% Configure English to use the same font
\newfontfamily\englishfont[Script=Gujarati,AutoFakeBold=2.5,AutoFakeSlant=0.3]{Noto Sans Gujarati}

% Translations for polyglossia
\gappto\captionsgujarati{
  \renewcommand{\tablename}{કોષ્ટક}
  \renewcommand{\figurename}{આકૃતિ}
}

% Helper for TikZ nodes to ensure Gujarati font
\newcommand{\gu}[1]{{\gujaratifont #1}}

% Custom environments
\newtcolorbox{solutionbox}{
    breakable,
    enhanced,
    colback=solutioncolor!5!white,
    colframe=solutioncolor!75!black,
    fonttitle=\bfseries,
    title=જવાબ
}

\newtcolorbox{solutionboxnobreak}{
 colback=solutioncolor!5!white,
 colframe=solutioncolor!75!black,
 fonttitle=\bfseries,
 title=જવાબ
}

\newtcolorbox{keyformula}{
 breakable,
 enhanced,
 colback=keycolor!5!white,
 colframe=keycolor!75!black,
 fonttitle=\bfseries,
 title=રાસાયણિક સમીકરણ/સૂત્ર
}

\newtcolorbox{mnemonicbox}{
 breakable,
 enhanced,
 colback=mnemoniccolor!5!white,
 colframe=mnemoniccolor!75!black,
 fonttitle=\bfseries,
 title=મેમરી ટ્રીક
}


% Custom commands for GTU solutions
% This file defines semantic commands for consistent formatting

% Question command with automatic formatting
\newcommand{\question}[2]{%
  \section*{Question #1}%
  \textbf{#2}%
}

% OR question variant
\newcommand{\questionor}[2]{%
  \section*{Question #1 OR}%
  \textbf{#2}%
}

% Proper table environment with caption
\newenvironment{answertable}[1]{%
  \begin{table}[htbp]
  \centering
  \caption{#1}
}{%
  \end{table}
}

% Proper figure environment for diagrams
\newenvironment{answerdiagram}[1]{%
  \begin{figure}[htbp]
  \centering
  \caption{#1}
}{%
  \end{figure}
}

% Semantic markup for key terms
\newcommand{\keyword}[1]{\textbf{#1}}
\newcommand{\code}[1]{\texttt{#1}}
\newcommand{\classname}[1]{\texttt{#1}}
\newcommand{\methodname}[1]{\texttt{#1}}

% Proper quotation marks
\newcommand{\mnemonic}[1]{``#1''}


\title{Digital \& Data Communication (4343201) - Summer 2025 Solution}
\date{May 15, 2025}

\begin{document}
\maketitle

\questionmarks{1(અ)}{3}{બિટ રેટ, બાઉડ રેટ અને બેન્ડવિડ્થ વ્યાખ્યાયિત કરો}

\begin{solutionbox}
\begin{center}
\captionof{table}{પેરામીટર વ્યાખ્યા}
\begin{tabulary}{\linewidth}{|L|L|L|}
\hline
\textbf{પેરામીટર} & \textbf{વ્યાખ્યા} & \textbf{એકમ} \\ \hline
\textbf{બિટ રેટ} & પ્રતિ સેકન્ડ ટ્રાન્સમિટ થતા બિટ્સની સંખ્યા & bps (બિટ્સ પર સેકન્ડ) \\ \hline
\textbf{બાઉડ રેટ} & પ્રતિ સેકન્ડ સિગ્નલ ફેરફારની સંખ્યા & બાઉડ \\ \hline
\textbf{બેન્ડવિડ્થ} & કોમ્યુનિકેશન ચેનલમાં ફ્રીક્વન્સીની રેંજ & Hz (હર્ટ્ઝ) \\ \hline
\end{tabulary}
\end{center}

\begin{itemize}
    \item \keyword{બિટ રેટ}: વાસ્તવિક ડેટા ટ્રાન્સમિશન સ્પીડ
    \item \keyword{બાઉડ રેટ}: મોડ્યુલેશન રેટ અથવા સિમ્બોલ રેટ
    \item \keyword{બેન્ડવિડ્થ}: ફ્રીક્વન્સી રેંજ માટે ચેનલ કેપેસિટી
\end{itemize}
\end{solutionbox}

\begin{mnemonicbox}
\mnemonic{બિટ્સ બાઉડ બેન્ડવિડ્થ - કોમ્યુનિકેશન માટે BBB}
\end{mnemonicbox}

\questionmarks{1(બ)}{4}{બ્લોક ડાયાગ્રામ સાથે TDM સમજાવો}

\begin{solutionbox}
\begin{center}
\begin{tikzpicture}[node distance=1.5cm, auto]
    \node [gtu block] (mux) {ટાઇમ ડિવિઝન મલ્ટિપ્લેક્સર};
    \node [gtu block, right=3cm of mux] (demux) {ટાઇમ ડિવિઝન ડીમલ્ટિપ્લેક્સર};
    
    \node [left=1cm of mux] (in2) {ઇનપુટ 2};
    \node [above=0.5cm of in2] (in1) {ઇનપુટ 1};
    \node [below=0.5cm of in2] (in3) {ઇનપુટ 3};
    \node [below=0.5cm of in2] (in4) {ઇનપુટ 4};
    
    \node [right=1cm of demux] (out2) {આઉટપુટ 2};
    \node [above=0.5cm of out2] (out1) {આઉટપુટ 1};
    \node [below=0.5cm of out2] (out3) {આઉટપુટ 3};
    \node [below=0.5cm of out2] (out4) {આઉટપુટ 4};
    
    \draw [gtu arrow] (in1) -- (mux.west |- in1);
    \draw [gtu arrow] (in2) -- (mux.west |- in2);
    \draw [gtu arrow] (in3) -- (mux.west |- in3);
    \draw [gtu arrow] (in4) -- (mux.west |- in4);
    
    \draw [gtu arrow] (mux) -- node {ટ્રાન્સમિશન ચેનલ} (demux);
    
    \draw [gtu arrow] (demux.east |- out1) -- (out1);
    \draw [gtu arrow] (demux.east |- out2) -- (out2);
    \draw [gtu arrow] (demux.east |- out3) -- (out3);
    \draw [gtu arrow] (demux.east |- out4) -- (out4);
\end{tikzpicture}
\captionof{figure}{ટાઇમ ડિવિઝન મલ્ટિપ્લેક્સિંગ (TDM)}
\end{center}

\begin{itemize}
    \item \keyword{TDM સિદ્ધાંત}: બહુવિધ સિગ્નલ્સ ટાઇમ સ્લોટ્સ દ્વારા સિંગલ ચેનલ શેર કરે છે
    \item \keyword{ટાઇમ સ્લોટ્સ}: દરેક ઇનપુટને સમર્પિત સમય અવધિ મળે છે
    \item \keyword{સિંક્રોનાઇઝેશન}: ટ્રાન્સમિટર અને રિસીવર સિંક્રોનાઇઝ હોવા જોઇએ
    \item \keyword{ઉપયોગ}: ડિજિટલ ટેલિફોન સિસ્ટમ્સ, કમ્પ્યુટર નેટવર્ક્સ
\end{itemize}
\end{solutionbox}

\begin{mnemonicbox}
\mnemonic{ટાઇમ ડિવાઇડેડ મલ્ટિપલ - TDM સમય શેર કરે છે}
\end{mnemonicbox}

\questionmarks{1(ક)}{7}{ડિજિટલ કોમ્યુનિકેશન સિસ્ટમનો બ્લોક ડાયાગ્રામ સમજાવો}

\begin{solutionbox}
\begin{center}
\begin{tikzpicture}[node distance=1.2cm, auto, font=\small]
    \node [gtu block] (source) {માહિતી સ્રોત};
    \node [gtu block, right=0.8cm of source] (senc) {સોર્સ એન્કોડર};
    \node [gtu block, right=0.8cm of senc] (cenc) {ચેનલ એન્કોડર};
    \node [gtu block, right=0.8cm of cenc] (mod) {ડિજિટલ મોડ્યુલેટર};
    
    \node [gtu block, below=1.5cm of mod] (channel) {ચેનલ};
    \node [gtu block, below=1.5cm of channel] (demod) {ડિજિટલ ડીમોડ્યુલેટર};
    \node [gtu block, left=0.8cm of demod] (cdec) {ચેનલ ડીકોડર};
    \node [gtu block, left=0.8cm of cdec] (sdec) {સોર્સ ડીકોડર};
    \node [gtu block, left=0.8cm of sdec] (dest) {ગંતવ્ય};
    
    \node [above=0.8cm of channel] (noise) {નોઇઝ};

    \draw [gtu arrow] (source) -- (senc);
    \draw [gtu arrow] (senc) -- (cenc);
    \draw [gtu arrow] (cenc) -- (mod);
    \draw [gtu arrow] (mod) -- (channel);
    \draw [gtu arrow] (channel) -- (demod);
    \draw [gtu arrow] (demod) -- (cdec);
    \draw [gtu arrow] (cdec) -- (sdec);
    \draw [gtu arrow] (sdec) -- (dest);
    \draw [gtu arrow] (noise) -- (channel);
\end{tikzpicture}
\captionof{figure}{ડિજિટલ કોમ્યુનિકેશન સિસ્ટમ}
\end{center}

\begin{center}
\captionof{table}{સિસ્ટમ કોમ્પોનન્ટ્સ}
\begin{tabulary}{\linewidth}{|L|L|}
\hline
\textbf{કોમ્પોનન્ટ} & \textbf{કાર્ય} \\ \hline
\textbf{સોર્સ એન્કોડર} & એનાલોગને ડિજિટલમાં કન્વર્ટ કરે છે \\ \hline
\textbf{ચેનલ એન્કોડર} & એરર કરેક્શન કોડ્સ ઉમેરે છે \\ \hline
\textbf{ડિજિટલ મોડ્યુલેટર} & ડિજિટલને એનાલોગ સિગ્નલમાં કન્વર્ટ કરે છે \\ \hline
\textbf{ચેનલ} & ટ્રાન્સમિશન મીડિયમ \\ \hline
\textbf{ડિજિટલ ડીમોડ્યુલેટર} & ડિજિટલ સિગ્નલ પુનઃપ્રાપ્ત કરે છે \\ \hline
\textbf{ચેનલ ડીકોડર} & એરર શોધે અને સુધારે છે \\ \hline
\textbf{સોર્સ ડીકોડર} & મૂળ સિગ્નલ પુનર્નિર્માણ કરે છે \\ \hline
\end{tabulary}
\end{center}

\begin{itemize}
    \item \keyword{ફાયદાઓ}: નોઇઝ પ્રતિરોધકતા, એરર કરેક્શન ક્ષમતા
    \item \keyword{પ્રોસેસિંગ}: ડિજિટલ સિગ્નલ પ્રોસેસિંગ તકનીકો
    \item \keyword{વિશ્વસનીયતા}: લાંબા અંતર પર વધુ સારી કામગીરી
\end{itemize}
\end{solutionbox}

\begin{mnemonicbox}
\mnemonic{સોર્સ ચેનલ મોડ્યુલેટ ટ્રાન્સમિટ ડીમોડ્યુલેટ ડીકોડ - SCMTDD}
\end{mnemonicbox}

\questionmarks{1(ક OR)}{7}{કોમ્યુનિકેશન ચેનલના વિવિધ પ્રકારો સમજાવો}

\begin{solutionbox}
\begin{center}
\captionof{table}{ચેનલ પ્રકારો}
\begin{tabulary}{\linewidth}{|L|L|L|}
\hline
\textbf{ચેનલ પ્રકાર} & \textbf{લાક્ષણિકતાઓ} & \textbf{ઉપયોગ} \\ \hline
\textbf{ટેલિફોન ચેનલ} & 300-3400 Hz બેન્ડવિડ્થ & વૉઇસ કોમ્યુનિકેશન \\ \hline
\textbf{કોએક્સિયલ કેબલ} & હાઇ બેન્ડવિડ્થ, શિલ્ડેડ & કેબલ TV, ઇન્ટરનેટ \\ \hline
\textbf{ઓપ્ટિકલ ફાઇબર} & ખૂબ હાઇ બેન્ડવિડ્થ, લાઇટ સિગ્નલ્સ & લાંબા અંતર, હાઇ સ્પીડ \\ \hline
\textbf{વાયરલેસ ચેનલ} & રેડિયો ફ્રીક્વન્સી ટ્રાન્સમિશન & મોબાઇલ, સેટેલાઇટ \\ \hline
\textbf{સેટેલાઇટ ચેનલ} & લાંબા અંતર, સ્પેસ કોમ્યુનિકેશન & ગ્લોબલ કોમ્યુનિકેશન \\ \hline
\end{tabulary}
\end{center}

\begin{itemize}
    \item \keyword{બેન્ડવિડ્થ}: વિવિધ ચેનલ્સ અલગ-અલગ ફ્રીક્વન્સી રેંજ આપે છે
    \item \keyword{નોઇઝ લાક્ષણિકતાઓ}: દરેક ચેનલની વિશિષ્ટ નોઇઝ પ્રોપર્ટીઝ છે
    \item \keyword{અંતર ક્ષમતા}: લોકલથી ગ્લોબલ કવરેજ સુધી બદલાય છે
    \item \keyword{કોસ્ટ ફેક્ટર્સ}: ઇન્સ્ટોલેશન અને મેઇન્ટેનન્સ કોસ્ટ અલગ છે
\end{itemize}
\end{solutionbox}

\begin{mnemonicbox}
\mnemonic{ટેલિફોન કોએક્સ ઓપ્ટિકલ વાયરલેસ સેટેલાઇટ - TCOWS ચેનલ્સ}
\end{mnemonicbox}

\questionmarks{2(અ)}{3}{ડિજિટલ સિક્વન્સ 11100110 માટે ASK, FSK અને BPSK માટે મોડ્યુલેશન વેવફોર્મ દોરો}

\begin{solutionbox}
\begin{center}
\begin{tikzpicture}[x=0.8cm,y=0.6cm]
    % Digital Data
    \node[anchor=east] at (-0.5, 1) {ડિજિટલ ડેટા:};
    \foreach \x/\val in {0/1, 1/1, 2/1, 3/0, 4/0, 5/1, 6/1, 7/0} {
        \draw (\x,0) -- (\x,\val) -- (\x+1,\val) -- (\x+1,0); 
        \node at (\x+0.5, 1.5) {\val};
    }
    
    % ASK
    \node[anchor=east] at (-0.5, -2) {ASK:};
    \foreach \x/\val in {0/1, 1/1, 2/1, 3/0, 4/0, 5/1, 6/1, 7/0} {
        \ifnum\val=1
            \draw[domain=\x:\x+1, samples=20] plot (\x, {sin(360*(\x)*2) * 0.8 - 2});
        \else
            \draw (\x,-2) -- (\x+1,-2);
        \fi
    }
    
    % FSK
    \node[anchor=east] at (-0.5, -5) {FSK:};
    \foreach \x/\val in {0/1, 1/1, 2/1, 3/0, 4/0, 5/1, 6/1, 7/0} {
        \ifnum\val=1
            \draw[domain=\x:\x+1, samples=40] plot (\x, {sin(360*(\x)*4) * 0.8 - 5});
        \else
            \draw[domain=\x:\x+1, samples=20] plot (\x, {sin(360*(\x)*2) * 0.8 - 5});
        \fi
    }
    
    % BPSK
    \node[anchor=east] at (-0.5, -8) {BPSK:};
    \foreach \x/\val in {0/1, 1/1, 2/1, 3/0, 4/0, 5/1, 6/1, 7/0} {
        \ifnum\val=1
            \draw[domain=\x:\x+1, samples=20] plot (\x, {sin(360*(\x)*2) * 0.8 - 8});
        \else
            \draw[domain=\x:\x+1, samples=20] plot (\x, {-sin(360*(\x)*2) * 0.8 - 8});
        \fi
    }
\end{tikzpicture}
\captionof{figure}{મોડ્યુલેશન વેવફોર્મ્સ}
\end{center}

\begin{itemize}
    \item \keyword{ASK}: એમ્પ્લિટ્યુડ શિફ્ટ કીઇંગ
    \item \keyword{FSK}: ફ્રીક્વન્સી શિફ્ટ કીઇંગ
    \item \keyword{BPSK}: બાઇનરી ફેઝ શિફ્ટ કીઇંગ
\end{itemize}
\end{solutionbox}

\begin{mnemonicbox}
\mnemonic{ASK એમ્પ્લિટ્યુડ, FSK ફ્રીક્વન્સી, BPSK ફેઝ - AFP મોડ્યુલેશન}
\end{mnemonicbox}

\questionmarks{2(બ)}{4}{ફ્રીક્વન્સી શિફ્ટ કીઇંગ (FSK) સિગ્નલના મૂળભૂત સિદ્ધાંત અને જનરેશનને સમજાવો}

\begin{solutionbox}
\begin{center}
\captionof{table}{FSK જનરેશન}
\begin{tabulary}{\linewidth}{|L|L|L|}
\hline
\textbf{બાઇનરી ડેટા} & \textbf{ફ્રીક્વન્સી} & \textbf{આઉટપુટ} \\ \hline
લોજિક `1' & $f_1$ (હાઇ ફ્રીક્વન્સી) & હાઇ ફ્રીક્વ કેરિયર \\ \hline
લોજિક `0' & $f_0$ (લો ફ્રીક્વન્સી) & લો ફ્રીક્વ કેરિયર \\ \hline
\end{tabulary}
\end{center}

\begin{center}
\begin{tikzpicture}[node distance=1.5cm, auto]
    \node [gtu block] (sel) {ફ્રીક્વન્સી સિલેક્ટર};
    \node [gtu block, left=1.5cm of sel, yshift=1cm] (osc1) {ઓસિલેટર 1 ($f_1$)};
    \node [gtu block, left=1.5cm of sel, yshift=-1cm] (osc2) {ઓસિલેટર 2 ($f_0$)};
    \node [left=1.5cm of sel] (data) {ડિજિટલ ડેટા};
    \node [right=1.5cm of sel] (out) {FSK આઉટપુટ};
    
    \draw [gtu arrow] (osc1) -| (sel);
    \draw [gtu arrow] (osc2) -| (sel);
    \draw [gtu arrow] (data) -- (sel);
    \draw [gtu arrow] (sel) -- (out);
\end{tikzpicture}
\captionof{figure}{FSK જનરેશન}
\end{center}

\begin{itemize}
    \item \keyword{સિદ્ધાંત}: બાઇનરી ડેટા કેરિયર ફ્રીક્વન્સી કંટ્રોલ કરે છે
    \item \keyword{બે ફ્રીક્વન્સીઝ}: `1' માટે $f_1$ અને `0' માટે $f_0$
    \item \keyword{કોન્સ્ટન્ટ એમ્પ્લિટ્યુડ}: માત્ર ફ્રીક્વન્સી બદલાય છે
    \item \keyword{ડિટેક્શન}: રિસીવર પર ફ્રીક્વન્સી ડિસ્ક્રિમિનેશન
\end{itemize}
\end{solutionbox}

\begin{mnemonicbox}
\mnemonic{ફ્રીક્વન્સી શિફ્ટ્સ કી - FSK ફ્રીક્વન્સી કંટ્રોલ}
\end{mnemonicbox}

\questionmarks{2(ક)}{7}{બ્લોક ડાયાગ્રામ અને કોન્સ્ટેલેશન ડાયાગ્રામ સાથે QPSK મોડ્યુલેટર અને ડીમોડ્યુલેટરની કામગીરી સમજાવો}

\begin{solutionbox}
\textbf{QPSK મોડ્યુલેટર:}
\begin{center}
\begin{tikzpicture}[node distance=1.5cm, auto]
    \node [gtu block] (sp) {સીરિયલ ટુ પેરેલલ};
    \node [gtu block, right=2cm of sp, yshift=1cm] (mult1) {મલ્ટિપ્લાયર 1};
    \node [gtu block, right=2cm of sp, yshift=-1cm] (mult2) {મલ્ટિપ્લાયર 2};
    \node [gtu block, right=2cm of mult1, yshift=-1cm] (adder) {એડર};
    
    \node [left=1cm of sp] (input) {સીરિયલ ડેટા};
    \node [above=1cm of mult1] (car1) {કેરિયર $\cos(\omega t)$};
    \node [below=1cm of mult2] (car2) {કેરિયર $\sin(\omega t)$};
    \node [right=1cm of adder] (out) {QPSK આઉટપુટ};
    
    \draw [gtu arrow] (input) -- (sp);
    \draw [gtu arrow] (sp) -- node[above, sloped] {I ચેનલ} (mult1);
    \draw [gtu arrow] (sp) -- node[below, sloped] {Q ચેનલ} (mult2);
    \draw [gtu arrow] (car1) -- (mult1);
    \draw [gtu arrow] (car2) -- (mult2);
    \draw [gtu arrow] (mult1) -| (adder);
    \draw [gtu arrow] (mult2) -| (adder);
    \draw [gtu arrow] (adder) -- (out);
\end{tikzpicture}
\captionof{figure}{QPSK મોડ્યુલેટર}
\end{center}

\textbf{કોન્સ્ટેલેશન ડાયાગ્રામ:}
\begin{center}
\begin{tikzpicture}[scale=1.5]
    \draw[->] (-2,0) -- (2,0) node[right] {I};
    \draw[->] (0,-2) -- (0,2) node[above] {Q};
    
    \foreach \x/\y/\l in {1/1/00 (45$^\circ$), -1/1/01 (135$^\circ$), -1/-1/11 (225$^\circ$), 1/-1/10 (315$^\circ$)} {
        \draw[fill] (\x,\y) circle (2pt);
        \node at (\x*1.4, \y*1.4) {\l};
        \draw[dashed] (0,0) -- (\x,\y);
    }
\end{tikzpicture}
\captionof{figure}{QPSK કોન્સ્ટેલેશન}
\end{center}

\begin{center}
\captionof{table}{QPSK ટ્રુથ ટેબલ}
\begin{tabulary}{\linewidth}{|C|C|C|C|}
\hline
\textbf{I} & \textbf{Q} & \textbf{ફેઝ} & \textbf{સિમ્બોલ} \\ \hline
0 & 0 & 45$^\circ$ & 00 \\ \hline
0 & 1 & 135$^\circ$ & 01 \\ \hline
1 & 1 & 225$^\circ$ & 11 \\ \hline
1 & 0 & 315$^\circ$ & 10 \\ \hline
\end{tabulary}
\end{center}

\begin{itemize}
    \item \keyword{ચાર ફેઝ}: 45$^\circ$, 135$^\circ$, 225$^\circ$, 315$^\circ$
    \item \keyword{બે બિટ્સ પર સિમ્બોલ}: હાયર ડેટા રેટ
    \item \keyword{કોન્સ્ટન્ટ એન્વેલોપ}: એમ્પ્લિટ્યુડ કોન્સ્ટન્ટ રહે છે
    \item \keyword{ડીમોડ્યુલેશન}: ફેઝ ડિટેક્શન અને પેરેલલ ટુ સીરિયલ કન્વર્શન
\end{itemize}
\end{solutionbox}

\begin{mnemonicbox}
\mnemonic{ક્વાડરેચર ફેઝ શિફ્ટ કી - QPSK ચાર ફેઝ}
\end{mnemonicbox}

\questionmarks{2(અ OR)}{3}{ASK મોડ્યુલેટરનો બ્લોક ડાયાગ્રામ દોરો અને તેના કામનું વર્ણન કરો}

\begin{solutionbox}
\begin{center}
\begin{tikzpicture}[node distance=1.5cm, auto]
    \node [gtu block] (switch) {સ્વિચ/મલ્ટિપ્લાયર};
    \node [left=1.5cm of switch] (data) {ડિજિટલ ડેટા};
    \node [above=1cm of switch] (carrier) {કેરિયર ઓસિલેટર};
    \node [right=1.5cm of switch] (out) {ASK આઉટપુટ};
    
    \draw [gtu arrow] (data) -- (switch);
    \draw [gtu arrow] (carrier) -- (switch);
    \draw [gtu arrow] (switch) -- (out);
\end{tikzpicture}
\captionof{figure}{ASK મોડ્યુલેટર}
\end{center}

\begin{itemize}
    \item \keyword{કામનો સિદ્ધાંત}: ડિજિટલ ડેટા કેરિયર એમ્પ્લિટ્યુડ કંટ્રોલ કરે છે
    \item \keyword{લોજિક `1'}: પૂર્ણ એમ્પ્લિટ્યુડ સાથે કેરિયર ટ્રાન્સમિટ થાય છે
    \item \keyword{લોજિક `0'}: કોઇ કેરિયર ટ્રાન્સમિટ થતું નથી (ઝીરો એમ્પ્લિટ્યુડ)
    \item \keyword{સિમ્પલ ઇમ્પ્લિમેન્ટેશન}: એનાલોગ સ્વિચ અથવા મલ્ટિપ્લાયર વાપરે છે
\end{itemize}
\end{solutionbox}

\begin{mnemonicbox}
\mnemonic{એમ્પ્લિટ્યુડ શિફ્ટ કી - ASK એમ્પ્લિટ્યુડ કંટ્રોલ}
\end{mnemonicbox}

\questionmarks{2(બ OR)}{4}{16-QAM ના પ્રિન્સિપલને સમજાવો અને કોન્સ્ટેલેશન ડાયાગ્રામ દોરો}

\begin{solutionbox}
\textbf{16-QAM કોન્સ્ટેલેશન:}
\begin{center}
\begin{tikzpicture}[scale=1.2]
    \draw[->] (-3,0) -- (3,0) node[right] {I};
    \draw[->] (0,-3) -- (0,3) node[above] {Q};
    
    \foreach \x in {-1.5, -0.5, 0.5, 1.5} {
        \foreach \y in {-1.5, -0.5, 0.5, 1.5} {
            \draw[fill] (\x,\y) circle (2pt);
        }
    }
\end{tikzpicture}
\captionof{figure}{16-QAM કોન્સ્ટેલેશન}
\end{center}

\begin{center}
\captionof{table}{16-QAM લાક્ષણિકતાઓ}
\begin{tabulary}{\linewidth}{|L|L|}
\hline
\textbf{પેરામીટર} & \textbf{વેલ્યુ} \\ \hline
\textbf{બિટ્સ પર સિમ્બોલ} & 4 બિટ્સ \\ \hline
\textbf{સ્ટેટ્સની સંખ્યા} & 16 \\ \hline
\textbf{એમ્પ્લિટ્યુડ લેવલ્સ} & 4 લેવલ્સ \\ \hline
\textbf{ફેઝ લેવલ્સ} & 4 ફેઝ \\ \hline
\end{tabulary}
\end{center}

\begin{itemize}
    \item \keyword{સિદ્ધાંત}: એમ્પ્લિટ્યુડ અને ફેઝ મોડ્યુલેશન કોમ્બાઇન કરે છે
    \item \keyword{હાયર ડેટા રેટ}: 4 બિટ્સ પર સિમ્બોલ
    \item \keyword{કોમ્પ્લેક્સ મોડ્યુલેશન}: પ્રિસાઇસ એમ્પ્લિટ્યુડ અને ફેઝ કંટ્રોલ જરૂરી
    \item \keyword{ઉપયોગ}: હાઇ-સ્પીડ ડિજિટલ કોમ્યુનિકેશન
\end{itemize}
\end{solutionbox}

\begin{mnemonicbox}
\mnemonic{16 ક્વાડરેચર એમ્પ્લિટ્યુડ મોડ્યુલેશન - 16QAM કોમ્પ્લેક્સ સિગ્નલ્સ}
\end{mnemonicbox}

\questionmarks{2(ક OR)}{7}{બ્લોક ડાયાગ્રામ અને વેવફોર્મ સાથે BPSK મોડ્યુલેટર અને ડીમોડ્યુલેટરનું કામ સમજાવો}

\begin{solutionbox}
\textbf{BPSK મોડ્યુલેટર:}
\begin{center}
\begin{tikzpicture}[node distance=1.5cm, auto]
    \node [gtu block] (nrz) {NRZ એન્કોડર};
    \node [gtu block, right=1.5cm of nrz] (mod) {બેલેન્સ્ડ મોડ્યુલેટર};
    \node [above=1cm of mod] (osc) {કેરિયર ઓસિલેટર};
    \node [left=1.5cm of nrz] (in) {ડિજિટલ ડેટા};
    \node [right=1.5cm of mod] (out) {BPSK આઉટપુટ};
    
    \draw [gtu arrow] (in) -- (nrz);
    \draw [gtu arrow] (nrz) -- (mod);
    \draw [gtu arrow] (osc) -- (mod);
    \draw [gtu arrow] (mod) -- (out);
\end{tikzpicture}
\captionof{figure}{BPSK મોડ્યુલેટર}
\end{center}

\textbf{BPSK ડીમોડ્યુલેટર:}
\begin{center}
\begin{tikzpicture}[node distance=1.5cm, auto]
    \node [gtu block] (demod) {બેલેન્સ્ડ ડીમોડ્યુલેટર};
    \node [gtu block, right=1.5cm of demod] (lpf) {લો પાસ ફિલ્ટર};
    \node [gtu block, right=1.5cm of lpf] (dec) {ડિસિઝન સર્કિટ};
    \node [above=1cm of demod] (local) {લોકલ કેરિયર};
    \node [left=1.5cm of demod] (in) {BPSK ઇનપુટ};
    \node [right=1.5cm of dec] (out) {ડિજિટલ આઉટપુટ};
    
    \draw [gtu arrow] (in) -- (demod);
    \draw [gtu arrow] (local) -- (demod);
    \draw [gtu arrow] (demod) -- (lpf);
    \draw [gtu arrow] (lpf) -- (dec);
    \draw [gtu arrow] (dec) -- (out);
\end{tikzpicture}
\captionof{figure}{BPSK ડીમોડ્યુલેટર}
\end{center}

\textbf{BPSK વેવફોર્મ્સ:}
\begin{center}
\begin{tikzpicture}[x=1cm,y=0.8cm]
    % Data
    \node[anchor=east] at (-0.5, 1) {ડેટા:};
    \foreach \x/\val in {0/1, 1/0, 2/1, 3/0} {
        \draw (\x,0) -- (\x,\val) -- (\x+1,\val) -- (\x+1,0);
        \node at (\x+0.5, 1.5) {\val};
    }
    
    % Carrier
    \node[anchor=east] at (-0.5, -1.5) {કેરિયર:};
    \draw[domain=0:4, samples=100] plot (\x, {sin(360*\x*4) * 0.8 - 1.5});
    
    % BPSK
    \node[anchor=east] at (-0.5, -4) {BPSK:};
    \foreach \x/\val in {0/1, 1/0, 2/1, 3/0} {
        \ifnum\val=1
            \draw[domain=\x:\x+1, samples=25] plot (\x, {sin(360*\x*4) * 0.8 - 4});
        \else
            \draw[domain=\x:\x+1, samples=25] plot (\x, {-sin(360*\x*4) * 0.8 - 4});
        \fi
    }
\end{tikzpicture}
\captionof{figure}{BPSK વેવફોર્મ્સ}
\end{center}

\begin{itemize}
    \item \keyword{ફેઝ શિફ્ટ}: `1' અને `0' વચ્ચે 180$^\circ$
    \item \keyword{કોહેરન્ટ ડિટેક્શન}: સિંક્રોનાઇઝ્ડ કેરિયર જરૂરી
    \item \keyword{બેસ્ટ પરફોર્મન્સ}: સૌથી ઓછી બિટ એરર રેટ
    \item \keyword{કોન્સ્ટન્ટ એન્વેલોપ}: એમ્પ્લિટ્યુડ કોન્સ્ટન્ટ રહે છે
\end{itemize}
\end{solutionbox}

\begin{mnemonicbox}
\mnemonic{બાઇનરી ફેઝ શિફ્ટ કી - BPSK બે ફેઝ}
\end{mnemonicbox}


\questionmarks{3(અ)}{3}{SNR ના સંદર્ભમાં ચેનલ ક્ષમતાને વ્યાખ્યાયિત કરો અને તેનું મહત્વ સમજાવો}

\begin{solutionbox}
\textbf{શેનોનના ચેનલ કેપેસિટી ફોર્મ્યુલા:}
\begin{center}
\captionof{table}{ચેનલ કેપેસિટી ફોર્મ્યુલા}
\begin{tabulary}{\linewidth}{|L|L|}
\hline
\textbf{ફોર્મ્યુલા} & $C = B \log_2(1 + S/N)$ \\ \hline
\textbf{C} & ચેનલ કેપેસિટી (bps) \\ \hline
\textbf{B} & બેન્ડવિડ્થ (Hz) \\ \hline
\textbf{S/N} & સિગ્નલ-ટુ-નોઇઝ રેશિયો \\ \hline
\end{tabulary}
\end{center}

\begin{itemize}
    \item \keyword{મહત્વ}: મહત્તમ થિયોરેટિકલ ડેટા રેટ
    \item \keyword{SNR અસર}: વધુ SNR વધુ કેપેસિટીને મંજૂરી આપે છે
    \item \keyword{બેન્ડવિડ્થ ટ્રેડ-ઓફ}: SNR માટે બેન્ડવિડ્થ બદલી શકાય છે
    \item \keyword{ડિઝાઇન લિમિટ}: સિસ્ટમ ડિઝાઇન માટે ઉપરની સીમા સેટ કરે છે
\end{itemize}
\end{solutionbox}

\begin{mnemonicbox}
\mnemonic{ચેનલ કેપેસિટી શેનોનની લિમિટ - CCSL}
\end{mnemonicbox}

\questionmarks{3(બ)}{4}{અસિંક્રોનસ અને સિંક્રોનસ સીરિયલ ડેટા કોમ્યુનિકેશન તકનીકોનું વર્ણન કરો}

\begin{solutionbox}
\begin{center}
\captionof{table}{સિંક્રોનસ વિ અસિંક્રોનસ}
\begin{tabulary}{\linewidth}{|L|L|L|}
\hline
\textbf{પેરામીટર} & \textbf{સિંક્રોનસ} & \textbf{અસિંક્રોનસ} \\ \hline
\textbf{ક્લોક} & અલગ ક્લોક સિગ્નલ & કોઇ અલગ ક્લોક નથી \\ \hline
\textbf{સ્ટાર્ટ/સ્ટોપ બિટ્સ} & જરૂરી નથી & સ્ટાર્ટ અને સ્ટોપ બિટ્સ \\ \hline
\textbf{સ્પીડ} & વધારે & ઓછી \\ \hline
\textbf{કોસ્ટ} & વધારે & ઓછી \\ \hline
\end{tabulary}
\end{center}

\begin{itemize}
    \item \keyword{સિંક્રોનસ}: ક્લોક સિંક્રોનાઇઝેશન જરૂરી
    \item \keyword{અસિંક્રોનસ}: સ્ટાર્ટ/સ્ટોપ બિટ્સ સાથે સેલ્ફ-સિંક્રોનાઇઝિંગ
    \item \keyword{ઉપયોગ}: સિંક્રોનસ હાઇ-સ્પીડ માટે, અસિંક્રોનસ સિમ્પલ સિસ્ટમ્સ માટે
    \item \keyword{કાર્યક્ષમતા}: સિંક્રોનસ વધુ કાર્યક્ષમ, અસિંક્રોનસ વધુ લવચીક
\end{itemize}
\end{solutionbox}

\begin{mnemonicbox}
\mnemonic{સિંક ક્લોક, અસિંક સ્ટાર્ટ-સ્ટોપ - SCSS}
\end{mnemonicbox}

\questionmarks{3(ક)}{7}{યોગ્ય ઉદાહરણની મદદથી હફમેન કોડિંગ સમજાવો}

\begin{solutionbox}
\textbf{ઉદાહરણ: અક્ષરો A, B, C, D સંભાવનાઓ 0.4, 0.3, 0.2, 0.1 સાથે}

\textbf{હફમેન ટ્રી કન્સ્ટ્રક્શન:}
\begin{center}
\begin{tikzpicture}[level distance=1.5cm, sibling distance=2.5cm, every node/.style={circle, draw, minimum size=0.8cm}]
    \node {1.0}
        child {node {A:0.4} edge from parent node[left, draw=none] {0}}
        child {node {0.6}
            child {node {B:0.3} edge from parent node[left, draw=none] {1}}
            child {node {0.3}
                child {node {C:0.2} edge from parent node[left, draw=none] {1}}
                child {node {D:0.1} edge from parent node[right, draw=none] {1}}
            edge from parent node[right, draw=none] {0}}
        edge from parent node[right, draw=none] {1}};
\end{tikzpicture}
\captionof{figure}{હફમેન ટ્રી}
\end{center}

\begin{center}
\captionof{table}{હફમેન કોડ્સ}
\begin{tabulary}{\linewidth}{|C|C|C|}
\hline
\textbf{અક્ષર} & \textbf{સંભાવના} & \textbf{કોડ} \\ \hline
A & 0.4 & 0 \\ \hline
B & 0.3 & 10 \\ \hline
C & 0.2 & 110 \\ \hline
D & 0.1 & 111 \\ \hline
\end{tabulary}
\end{center}

\begin{itemize}
    \item \keyword{એવરેજ કોડ લેન્થ}: $0.4 \times 1 + 0.3 \times 2 + 0.2 \times 3 + 0.1 \times 3 = 1.9$ બિટ્સ
    \item \keyword{કમ્પ્રેશન પ્રાપ્ત}: પ્રતિ અક્ષર એવરેજ બિટ્સ ઘટાડે છે
    \item \keyword{પ્રીફિક્સ પ્રોપર્ટી}: કોઇ કોડ બીજાનો પ્રીફિક્સ નથી
\end{itemize}
\end{solutionbox}

\begin{mnemonicbox}
\mnemonic{હફમેન મિનિમમ એવરેજ લેન્થ - HMAL}
\end{mnemonicbox}

\questionmarks{3(અ OR)}{3}{સંચારમાં સંભાવના અને એન્ટ્રોપીનું મહત્વ જણાવો}

\begin{solutionbox}
\begin{center}
\captionof{table}{મહત્વ}
\begin{tabulary}{\linewidth}{|L|L|}
\hline
\textbf{કન્સેપ્ટ} & \textbf{મહત્વ} \\ \hline
\textbf{સંભાવના} & માહિતીની ઘટનાની સંભાવના માપે છે \\ \hline
\textbf{એન્ટ્રોપી} & એવરેજ માહિતી સામગ્રી માપે છે \\ \hline
\textbf{મહત્તમ એન્ટ્રોપી} & સમાન સંભાવના ઘટનાઓ સાથે થાય છે \\ \hline
\end{tabulary}
\end{center}

\begin{itemize}
    \item \keyword{માહિતી સામગ્રી}: $I = \log_2(1/P)$ બિટ્સ
    \item \keyword{એન્ટ્રોપી ફોર્મ્યુલા}: $H = -\sum P(x) \log_2 P(x)$
    \item \keyword{ચેનલ ડિઝાઇન}: કોમ્યુનિકેશન સિસ્ટમ્સ ઑપ્ટિમાઇઝ કરવામાં મદદ કરે છે
    \item \keyword{કોડિંગ કાર્યક્ષમતા}: સોર્સ કોડિંગ ડિઝાઇનને માર્ગદર્શન આપે છે
\end{itemize}
\end{solutionbox}

\begin{mnemonicbox}
\mnemonic{પ્રોબેબિલિટી એન્ટ્રોપી ઇન્ફોર્મેશન - PEI કોમ્યુનિકેશન}
\end{mnemonicbox}

\questionmarks{3(બ OR)}{4}{સિમ્પ્લેક્સ, હાફ ડુપ્લેક્સ અને ફુલ ડુપ્લેક્સ ડેટા ટ્રાન્સમિશન મોડ સમજાવો}

\begin{solutionbox}
\begin{center}
\captionof{table}{ટ્રાન્સમિશન મોડ્સ}
\begin{tabulary}{\linewidth}{|L|L|L|L|}
\hline
\textbf{મોડ} & \textbf{દિશા} & \textbf{ઉદાહરણ} & \textbf{ડાયાગ્રામ} \\ \hline
\textbf{સિમ્પ્લેક્સ} & માત્ર એક દિશા & રેડિયો બ્રોડકાસ્ટ & A $\rightarrow$ B \\ \hline
\textbf{હાફ ડુપ્લેક્સ} & બંને દિશા, એકસાથે નહીં & વોકી-ટોકી & A $\leftrightarrow$ B \\ \hline
\textbf{ફુલ ડુપ્લેક્સ} & બંને દિશા, એકસાથે & ટેલિફોન & A $\rightleftarrows$ B \\ \hline
\end{tabulary}
\end{center}

\begin{itemize}
    \item \keyword{સિમ્પ્લેક્સ}: એકદિશીય કોમ્યુનિકેશન
    \item \keyword{હાફ ડુપ્લેક્સ}: દ્વિદિશીય પરંતુ વૈકલ્પિક
    \item \keyword{ફુલ ડુપ્લેક્સ}: એકસાથે દ્વિદિશીય
    \item \keyword{બેન્ડવિડ્થ આવશ્યકતા}: ફુલ ડુપ્લેક્સને બમણી બેન્ડવિડ્થ જોઇએ
\end{itemize}
\end{solutionbox}

\begin{mnemonicbox}
\mnemonic{સિમ્પલ હાફ ફુલ - SHF ટ્રાન્સમિશન મોડ્સ}
\end{mnemonicbox}

\questionmarks{3(ક OR)}{7}{યોગ્ય ઉદાહરણની મદદથી શેનોન ફાડો કોડિંગ સમજાવો}

\begin{solutionbox}
\textbf{ઉદાહરણ: અક્ષરો A, B, C, D સંભાવનાઓ 0.4, 0.3, 0.2, 0.1 સાથે}

\textbf{શેનોન-ફાડો અલ્ગોરિધમ સ્ટેપ્સ:}
\begin{enumerate}
    \item \textbf{સ્ટેપ 1}: ઘટતા ક્રમમાં ગોઠવો (A: 0.4, B: 0.3, C: 0.2, D: 0.1)
    \item \textbf{સ્ટેપ 2}: બે ગ્રુપમાં વિભાજિત કરો
    \begin{itemize}
        \item ગ્રુપ 1: A(0.4) $\rightarrow$ કોડ 0 થી શરૂ થાય છે
        \item ગ્રુપ 2: B(0.3), C(0.2), D(0.1) $\rightarrow$ કોડ 1 થી શરૂ થાય છે
    \end{itemize}
    \item \textbf{સ્ટેપ 3}: ગ્રુપ 2નું પેટાવિભાજન
    \begin{itemize}
        \item B(0.3) $\rightarrow$ કોડ: 10
        \item C(0.2), D(0.1) $\rightarrow$ કોડ 11 થી શરૂ થાય છે
    \end{itemize}
    \item \textbf{સ્ટેપ 4}: અંતિમ પેટાવિભાજન
    \begin{itemize}
        \item C(0.2) $\rightarrow$ કોડ: 110
        \item D(0.1) $\rightarrow$ કોડ: 111
    \end{itemize}
\end{enumerate}

\begin{center}
\captionof{table}{શેનોન-ફાડો કોડ્સ}
\begin{tabulary}{\linewidth}{|C|C|C|}
\hline
\textbf{અક્ષર} & \textbf{સંભાવના} & \textbf{કોડ} \\ \hline
A & 0.4 & 0 \\ \hline
B & 0.3 & 10 \\ \hline
C & 0.2 & 110 \\ \hline
D & 0.1 & 111 \\ \hline
\end{tabulary}
\end{center}

\begin{itemize}
    \item \keyword{એવરેજ લેન્થ}: હફમેન સમાન (1.9 બિટ્સ)
    \item \keyword{ટોપ-ડાઉન એપ્રોચ}: રૂટથી પાંદડાઓ સુધી વિભાજિત કરે છે
    \item \keyword{હંમેશા ઑપ્ટિમલ નથી}: હફમેન સામાન્ય રીતે વધુ સારું છે
\end{itemize}
\end{solutionbox}

\begin{mnemonicbox}
\mnemonic{શેનોન ફાડો ટોપ-ડાઉન - SFTD કોડિંગ}
\end{mnemonicbox}

\questionmarks{4(અ)}{3}{ડેટા કોમ્યુનિકેશનમાં નૈતિક અને ગોપનીયતાની બાબતોનું વર્ણન કરો}

\begin{solutionbox}
\begin{center}
\captionof{table}{નીતિશાસ્ત્ર અને ગોપનીયતા}
\begin{tabulary}{\linewidth}{|L|L|}
\hline
\textbf{પાસા} & \textbf{વિચારણા} \\ \hline
\textbf{ડેટા ગોપનીયતા} & વપરાશકર્તાની સંમતિ, ડેટા સુરક્ષા \\ \hline
\textbf{સિક્યુરિટી} & એન્ક્રિપ્શન, એક્સેસ કંટ્રોલ \\ \hline
\textbf{પારદર્શિતા} & સ્પષ્ટ ડેટા વપરાશ નીતિઓ \\ \hline
\end{tabulary}
\end{center}

\begin{itemize}
    \item \keyword{ગોપનીયતાના અધિકારો}: વ્યક્તિગત ડેટા પર વપરાશકર્તાનું નિયંત્રણ
    \item \keyword{નૈતિક ઉપયોગ}: જવાબદાર ડેટા હેન્ડલિંગ પ્રથાઓ
    \item \keyword{કાનૂની પાલન}: ડેટા સુરક્ષા કાયદાઓનું પાલન કરવું
    \item \keyword{સિક્યુરિટી પગલાં}: અનધિકૃત પ્રવેશ સામે સુરક્ષા
\end{itemize}
\end{solutionbox}

\begin{mnemonicbox}
\mnemonic{ગોપનીયતા સિક્યુરિટી પારદર્શિતા - PST નીતિશાસ્ત્ર}
\end{mnemonicbox}

\questionmarks{4(બ)}{4}{RS 232 સ્ટાન્ડર્ડને પિન ડાયાગ્રામ સાથે સમજાવો}

\begin{solutionbox}
\begin{center}
\captionof{table}{RS-232 પિન કન્ફિગરેશન (DB-9)}
\begin{tabulary}{\linewidth}{|C|C|L|}
\hline
\textbf{પિન} & \textbf{સિગ્નલ} & \textbf{કાર્ય} \\ \hline
1 & DCD & ડેટા કેરિયર ડિટેક્ટ \\ \hline
2 & RXD & રિસીવ ડેટા \\ \hline
3 & TXD & ટ્રાન્સમિટ ડેટા \\ \hline
4 & DTR & ડેટા ટર્મિનલ રેડી \\ \hline
5 & GND & ગ્રાઉન્ડ \\ \hline
6 & DSR & ડેટા સેટ રેડી \\ \hline
7 & RTS & રિક્વેસ્ટ ટુ સેન્ડ \\ \hline
8 & CTS & ક્લિયર ટુ સેન્ડ \\ \hline
9 & RI & રિંગ ઇન્ડિકેટર \\ \hline
\end{tabulary}
\end{center}

\begin{itemize}
    \item \keyword{વોલ્ટેજ લેવલ્સ}: `0' માટે +3V થી +25V, `1' માટે -3V થી -25V
    \item \keyword{મહત્તમ અંતર}: 19.2 kbps પર 50 ફુટ
    \item \keyword{ઉપયોગ}: કમ્પ્યુટર અને મોડેમ વચ્ચે સીરિયલ કોમ્યુનિકેશન
\end{itemize}
\end{solutionbox}

\begin{mnemonicbox}
\mnemonic{RS-232 નવ પિન્સ સીરિયલ - RNS કોમ્યુનિકેશન}
\end{mnemonicbox}

\questionmarks{4(ક)}{7}{યોગ્ય ઉદાહરણની મદદથી હેમિંગ કોડ સમજાવો}

\begin{solutionbox}
\textbf{ઉદાહરણ: 4-બિટ ડેટા 1011}

\textbf{હેમિંગ કોડ કન્સ્ટ્રક્શન:}
\begin{center}
\captionof{table}{હેમિંગ કોડ બિટ્સ}
\begin{tabulary}{\linewidth}{|L|C|C|C|C|C|C|C|}
\hline
\textbf{સ્થિતિ} & 1 & 2 & 3 & 4 & 5 & 6 & 7 \\ \hline
\textbf{પ્રકાર} & P1 & P2 & D1 & P4 & D2 & D3 & D4 \\ \hline
\textbf{વેલ્યુ} & 0 & 1 & 1 & 0 & 0 & 1 & 1 \\ \hline
\end{tabulary}
\end{center}

\begin{itemize}
    \item \keyword{P1} (સ્થિતિઓ 1,3,5,7): P1 $\oplus$ 1 $\oplus$ 0 $\oplus$ 1 = 0, તેથી P1 = 0
    \item \keyword{P2} (સ્થિતિઓ 2,3,6,7): P2 $\oplus$ 1 $\oplus$ 1 $\oplus$ 1 = 1, તેથી P2 = 1
    \item \keyword{P4} (સ્થિતિઓ 4,5,6,7): P4 $\oplus$ 0 $\oplus$ 1 $\oplus$ 1 = 0, તેથી P4 = 0
\end{itemize}

\textbf{અંતિમ હેમિંગ કોડ: 0110111}

\textbf{એરર ડિટેક્શન પ્રોસેસ:}
\begin{itemize}
    \item સિન્ડ્રોમ $S = S_4S_2S_1$ કેલ્ક્યુલેટ કરો
    \item જો $S = 000$, કોઇ એરર નથી
    \item જો $S \neq 000$, S દ્વારા દર્શાવેલ સ્થિતિએ એરર છે
\end{itemize}

\begin{itemize}
    \item \keyword{સિંગલ એરર કરેક્શન}: એક-બિટ એરર સુધારી શકે છે
    \item \keyword{ડબલ એરર ડિટેક્શન}: બે-બિટ એરર શોધી શકે છે
    \item \keyword{સિસ્ટેમેટિક એપ્રોચ}: વ્યવસ્થિત પેરિટી બિટ પ્લેસમેન્ટ
\end{itemize}
\end{solutionbox}

\begin{mnemonicbox}
\mnemonic{હેમિંગ સિંગલ એરર કરેક્શન - HSEC}
\end{mnemonicbox}

\questionmarks{4(અ OR)}{3}{એજ કમ્પ્યુટિંગને વ્યાખ્યાયિત કરો અને તેની વિશેષતા સમજાવો}

\begin{solutionbox}
\begin{center}
\captionof{table}{એજ કમ્પ્યુટિંગ વિશેષતાઓ}
\begin{tabulary}{\linewidth}{|L|L|}
\hline
\textbf{વિશેષતા} & \textbf{વર્ણન} \\ \hline
\textbf{લો લેટન્સી} & ડેટા સોર્સની નજીક પ્રોસેસિંગ \\ \hline
\textbf{બેન્ડવિડ્થ સેવિંગ} & નેટવર્ક્સ ટ્રાફિક ઘટાડે છે \\ \hline
\textbf{રિયલ-ટાઇમ પ્રોસેસિંગ} & તાત્કાલિક ડેટા એનાલિસિસ \\ \hline
\end{tabulary}
\end{center}

\begin{itemize}
    \item \keyword{વ્યાખ્યા}: નેટવર્ક એજ પર, ડેટા સોર્સની નજીક કમ્પ્યુટિંગ
    \item \keyword{ઘટાડેલી લેટન્સી}: ઝડપી રિસ્પોન્સ ટાઇમ
    \item \keyword{ડિસ્ટ્રિબ્યુટેડ પ્રોસેસિંગ}: સેન્ટ્રલ સર્વર લોડ ઘટાડે છે
    \item \keyword{ઉપયોગ}: IoT, ઓટોનોમસ વાહનો, સ્માર્ટ સિટીઓ
\end{itemize}
\end{solutionbox}

\begin{mnemonicbox}
\mnemonic{એજ લો-લેટન્સી રિયલ-ટાઇમ - ELR કમ્પ્યુટિંગ}
\end{mnemonicbox}

\questionmarks{4(બ OR)}{4}{સંદેશાવ્યવહાર માટે મલ્ટીમીડિયા પ્રોસેસિંગની જરૂરિયાતો અને વિવિધ ડેટાના વિવિધ ફાઇલ ફોર્મેટ સમજાવો}

\begin{solutionbox}
\begin{center}
\captionof{table}{મલ્ટીમીડિયા ફાઇલ ફોર્મેટ્સ}
\begin{tabulary}{\linewidth}{|L|L|L|}
\hline
\textbf{ડેટા પ્રકાર} & \textbf{ફોર્મેટ્સ} & \textbf{લાક્ષણિકતાઓ} \\ \hline
\textbf{ઓડિયો} & MP3, WAV, AAC & કમ્પ્રેસ્ડ/અનકમ્પ્રેસ્ડ \\ \hline
\textbf{વિડિયો} & MP4, AVI, MOV & વિવિધ કોડેક્સ \\ \hline
\textbf{ઇમેજ} & JPEG, PNG, GIF & લોસી/લૉસલેસ કમ્પ્રેશન \\ \hline
\textbf{ટેક્સ્ટ} & TXT, PDF, DOC & વિવિધ એન્કોડિંગ્સ \\ \hline
\end{tabulary}
\end{center}

\begin{itemize}
    \item \keyword{પ્રોસેસિંગ જરૂરિયાતો}: કમ્પ્રેશન, ફોર્મેટ કન્વર્શન, ક્વોલિટી ઑપ્ટિમાઇઝેશન
    \item \keyword{બેન્ડવિડ્થ ઑપ્ટિમાઇઝેશન}: ટ્રાન્સમિશન માટે ફાઇલ સાઇઝ ઘટાડવું
    \item \keyword{ક્વોલિટી પ્રિઝર્વેશન}: સ્વીકાર્ય ક્વોલિટી લેવલ રાખવું
    \item \keyword{કમ્પેટિબિલિટી}: મલ્ટિપલ ડિવાઇસ અને પ્લેટફોર્મ્સને સપોર્ટ કરવું
\end{itemize}
\end{solutionbox}

\begin{mnemonicbox}
\mnemonic{ઓડિયો વિડિયો ઇમેજ ટેક્સ્ટ - AVIT મલ્ટીમીડિયા}
\end{mnemonicbox}

\questionmarks{4(ક OR)}{7}{વેવફોર્મની મદદથી વિવિધ લાઇન કોડિંગ સમજાવો}

\begin{solutionbox}
\textbf{ડેટા 1011 માટે લાઇન કોડિંગ વેવફોર્મ્સ:}
\begin{center}
\begin{tikzpicture}[x=1cm,y=0.6cm]
    % Data
    \node[anchor=east] at (-1, 1) {ડેટા:};
    \foreach \x/\val in {0/1, 1/0, 2/1, 3/1} {
        \draw (\x,0) -- (\x,\val) -- (\x+1,\val) -- (\x+1,0); 
        \node at (\x+0.5, 1.5) {\val};
    }
    
    % NRZ-L
    \node[anchor=east] at (-1, -1.5) {NRZ-L:};
    \foreach \x/\val in {0/1, 1/0, 2/1, 3/1} {
        \ifnum\val=1
            \draw (\x,-1) -- (\x+1,-1); % High
        \else
            \draw (\x,-2) -- (\x+1,-2); % Low
        \fi
        \ifnum\x>0 \draw (\x,-1) -- (\x,-2); \fi % Transition
        \ifnum\x=0 \ifnum\val=1 \draw (0,-2) -- (0,-1); \fi \fi
    }
    
    % NRZ-I
    \node[anchor=east] at (-1, -4) {NRZ-I:};
    \draw (0,-3) -- (1,-3); % 1 -> toggle to low
    \draw (1,-3) -- (2,-3); % 0 -> same
    \draw (2,-2) -- (3,-2); % 1 -> toggle to high
    \draw (3,-3) -- (4,-3); % 1 -> toggle to low
    \draw (1,-3) -- (1,-3); % Vertical lines
    \draw (2,-3) -- (2,-2);
    \draw (3,-2) -- (3,-3);
    
    % RZ
    \node[anchor=east] at (-1, -6.5) {RZ:};
    \foreach \x/\val in {0/1, 1/0, 2/1, 3/1} {
        \ifnum\val=1
            \draw (\x,-5.5) -- (\x+0.5,-5.5) -- (\x+0.5,-6.5) -- (\x+1,-6.5);
            \draw (\x,-6.5) -- (\x,-5.5);
        \else
            \draw (\x,-6.5) -- (\x+1,-6.5);
        \fi
    }
    
    % Manchester
    \node[anchor=east] at (-1, -9) {Manch.:};
    \foreach \x/\val in {0/1, 1/0, 2/1, 3/1} {
        \ifnum\val=1
            \draw (\x,-8) -- (\x+0.5,-8) -- (\x+0.5,-9) -- (\x+1,-9);
            \ifnum\x>0 \draw (\x,-9) -- (\x,-8); \fi
        \else
            \draw (\x,-9) -- (\x+0.5,-9) -- (\x+0.5,-8) -- (\x+1,-8);
             \ifnum\x>0 \draw (\x,-9) -- (\x,-8); \fi
        \fi
    }
\end{tikzpicture}
\captionof{figure}{લાઇન કોડિંગ વેવફોર્મ્સ}
\end{center}

\begin{center}
\captionof{table}{લાઇન કોડિંગ સરખામણી}
\begin{tabulary}{\linewidth}{|L|L|L|L|}
\hline
\textbf{કોડ પ્રકાર} & \textbf{બેન્ડવિડ્થ} & \textbf{DC કોમ્પોનન્ટ} & \textbf{સિંક્રોનાઇઝેશન} \\ \hline
\textbf{NRZ-L} & લો & હાજર & ખરાબ \\ \hline
\textbf{NRZ-I} & લો & હાજર & ખરાબ \\ \hline
\textbf{RZ} & હાઇ & હાજર & સારું \\ \hline
\textbf{Manchester} & હાઇ & ગેરહાજર & ઉત્કૃષ્ટ \\ \hline
\end{tabulary}
\end{center}

\begin{itemize}
    \item \keyword{NRZ}: નોન-રિટર્ન-ટુ-ઝીરો, સિમ્પલ પરંતુ DC કોમ્પોનન્ટ છે
    \item \keyword{RZ}: રિટર્ન-ટુ-ઝીરો, વધુ સારું સિંક્રોનાઇઝેશન
    \item \keyword{Manchester}: સેલ્ફ-સિંક્રોનાઇઝિંગ, કોઇ DC કોમ્પોનન્ટ નથી
    \item \keyword{સિલેક્શન ક્રાઇટેરિયા}: બેન્ડવિડ્થ, સિંક્રોનાઇઝેશન, જટિલતા
\end{itemize}
\end{solutionbox}

\begin{mnemonicbox}
\mnemonic{NRZ RZ Manchester - NRM લાઇન કોડ્સ}
\end{mnemonicbox}

\questionmarks{5(અ)}{3}{સ્પ્રેડ સ્પેક્ટ્રમ ટેકનોલોજીનો ખ્યાલ સમજાવો}

\begin{solutionbox}
\begin{center}
\captionof{table}{સ્પ્રેડ સ્પેક્ટ્રમ લાક્ષણિકતાઓ}
\begin{tabulary}{\linewidth}{|L|L|}
\hline
\textbf{પેરામીટર} & \textbf{વર્ણન} \\ \hline
\textbf{બેન્ડવિડ્થ સ્પ્રેડિંગ} & વાઇડ ફ્રીક્વન્સી પર સિગ્નલ સ્પ્રેડ \\ \hline
\textbf{લો પાવર ડેન્સિટી} & સ્પેક્ટ્રમમાં પાવર વિતરિત \\ \hline
\textbf{ઇન્ટરફેરન્સ રેઝિસ્ટન્સ} & જેમિંગ સામે પ્રતિરોધક \\ \hline
\end{tabulary}
\end{center}

\begin{itemize}
    \item \keyword{સિદ્ધાંત}: જરૂરી કરતાં વધુ વાઇડ બેન્ડવિડ્થ પર સિગ્નલ ફેલાવે છે
    \item \keyword{તકનીકો}: ડાઇરેક્ટ સિક્વન્સ (DS-SS), ફ્રીક્વન્સી હોપિંગ (FH-SS)
    \item \keyword{ફાયદાઓ}: સિક્યુરિટી, ઇન્ટરફેરન્સ પ્રતિરોધ, મલ્ટિપલ એક્સેસ
    \item \keyword{ઉપયોગ}: GPS, CDMA, WiFi, Bluetooth
\end{itemize}
\end{solutionbox}

\begin{mnemonicbox}
\mnemonic{સ્પ્રેડ સ્પેક્ટ્રમ સિક્યુરિટી - SSS ટેકનોલોજી}
\end{mnemonicbox}

\questionmarks{5(બ)}{4}{સેટેલાઇટ કોમ્યુનિકેશનના બ્લોક ડાયાગ્રામને સમજાવો}

\begin{solutionbox}
\begin{center}
\begin{tikzpicture}[node distance=1.5cm, auto]
    \node [gtu block] (sat) {સેટેલાઇટ ટ્રાન્સપોન્ડર};
    \node [gtu block, below left=2cm of sat] (es1) {અર્થ સ્ટેશન 1};
    \node [gtu block, below right=2cm of sat] (es2) {અર્થ સ્ટેશન 2};
    \node [left=0.5cm of sat] (ant1) {એન્ટેના};
    \node [right=0.5cm of sat] (ant2) {એન્ટેના};
    
    \draw [gtu arrow] (es1) -- node[sloped, above] {અપલિંક} (sat);
    \draw [gtu arrow] (sat) -- node[sloped, above] {ડાઉનલિંક} (es2);
    \draw [dotted] (ant1) -- (sat);
    \draw [dotted] (ant2) -- (sat);
\end{tikzpicture}
\captionof{figure}{સેટેલાઇટ કોમ્યુનિકેશન}
\end{center}

\begin{center}
\captionof{table}{સેટેલાઇટ કોમ્યુનિકેશન કોમ્પોનન્ટ્સ}
\begin{tabulary}{\linewidth}{|L|L|}
\hline
\textbf{કોમ્પોનન્ટ} & \textbf{કાર્ય} \\ \hline
\textbf{અર્થ સ્ટેશન} & ગ્રાઉન્ડ-બેસ્ડ ટ્રાન્સમિટ/રિસીવ \\ \hline
\textbf{અપલિંક} & પૃથ્વીથી સેટેલાઇટ ટ્રાન્સમિશન \\ \hline
\textbf{ટ્રાન્સપોન્ડર} & સેટેલાઇટ રિસીવર-ટ્રાન્સમિટર \\ \hline
\textbf{ડાઉનલિંક} & સેટેલાઇટથી પૃથ્વી ટ્રાન્સમિશન \\ \hline
\end{tabulary}
\end{center}

\begin{itemize}
    \item \keyword{ફ્રીક્વન્સી બેન્ડ્સ}: C-બેન્ડ, Ku-બેન્ડ, Ka-બેન્ડ
    \item \keyword{કવરેજ એરિયા}: મોટા ભૌગોલિક કવરેજ
    \item \keyword{ઉપયોગ}: બ્રોડકાસ્ટિંગ, ટેલિફોની, ઇન્ટરનેટ
    \item \keyword{ફાયદાઓ}: વાઇડ કવરેજ, લાંબા-અંતરની કોમ્યુનિકેશન
\end{itemize}
\end{solutionbox}

\begin{mnemonicbox}
\mnemonic{અર્થ અપલિંક ટ્રાન્સપોન્ડર ડાઉનલિંક - EUTD સેટેલાઇટ}
\end{mnemonicbox}

\questionmarks{5(ક)}{7}{મલ્ટીમીડિયા કોમ્યુનિકેશન્સનું મોડેલ અને મલ્ટીમીડિયા સિસ્ટમના તત્વોનું પ્રદર્શન કરો}

\begin{solutionbox}
\textbf{મલ્ટીમીડિયા કોમ્યુનિકેશન મોડેલ:}
\begin{center}
\begin{tikzpicture}[node distance=1.2cm, auto]
    \node [gtu block] (enc) {એન્કોડર};
    \node [gtu block, right=1cm of enc] (mux) {મલ્ટિપ્લેક્સર};
    \node [gtu block, right=1cm of mux] (net) {નેટવર્ક};
    \node [gtu block, right=1cm of net] (demux) {ડીમલ્ટિપ્લેક્સર};
    \node [gtu block, right=1cm of demux] (dec) {ડીકોડર};
    \node [gtu block, right=1cm of dec] (dest) {ગંતવ્ય};
    
    \node [left=1cm of enc] (source) {સ્રોત ઇનપુટ્સ};
    \node [above=0.2cm of source] (aud) {ઓડિયો};
    \node [below=0.2cm of source] (vid) {વિડિયો};
    \node [above=0.6cm of source] (txt) {ટેક્સ્ટ};
    \node [below=0.6cm of source] (gfx) {ગ્રાફિક્સ};
    
    \draw [gtu arrow] (aud) -- (enc);
    \draw [gtu arrow] (vid) -- (enc);
    \draw [gtu arrow] (txt) -- (enc);
    \draw [gtu arrow] (gfx) -- (enc);
    
    \draw [gtu arrow] (enc) -- (mux);
    \draw [gtu arrow] (mux) -- (net);
    \draw [gtu arrow] (net) -- (demux);
    \draw [gtu arrow] (demux) -- (dec);
    \draw [gtu arrow] (dec) -- (dest);
\end{tikzpicture}
\captionof{figure}{મલ્ટીમીડિયા કોમ્યુનિકેશન મોડેલ}
\end{center}

\begin{center}
\captionof{table}{મલ્ટીમીડિયા સિસ્ટમ તત્વો}
\begin{tabulary}{\linewidth}{|L|L|L|}
\hline
\textbf{તત્વ} & \textbf{કાર્ય} & \textbf{ઉદાહરણો} \\ \hline
\textbf{કેપ્ચર} & મલ્ટીમીડિયા ડેટા ઇનપુટ & કેમેરા, માઇક્રોફોન \\ \hline
\textbf{સ્ટોરેજ} & મલ્ટીમીડિયા ફાઇલ્સ સ્ટોર કરવું & હાર્ડ ડિસ્ક, મેમોરી \\ \hline
\textbf{પ્રોસેસિંગ} & એડિટ અને મેનિપ્યુલેટ કરવું & વિડિયો એડિટિંગ સોફ્ટવેર \\ \hline
\textbf{કોમ્યુનિકેશન} & મલ્ટીમીડિયા ટ્રાન્સમિટ કરવું & નેટવર્ક્સ, ઇન્ટરનેટ \\ \hline
\textbf{પ્રેઝન્ટેશન} & મલ્ટીમીડિયા ડિસ્પ્લે કરવું & મોનિટર, સ્પીકર્સ \\ \hline
\end{tabulary}
\end{center}

\begin{itemize}
    \item \keyword{સિંક્રોનાઇઝેશન}: ઓડિયો-વિડિયો સિંક્રોનાઇઝેશન મહત્વપૂર્ણ
    \item \keyword{કમ્પ્રેશન}: બેન્ડવિડ્થ આવશ્યકતાઓ ઘટાડે છે
    \item \keyword{ક્વોલિટી ઓફ સર્વિસ}: સ્વીકાર્ય ક્વોલિટી જાળવે છે
    \item \keyword{રિયલ-ટાઇમ કન્સ્ટ્રેઇન્ટ્સ}: સમય-સંવેદનશીલ ડેટા વિતરણ
\end{itemize}
\end{solutionbox}

\begin{mnemonicbox}
\mnemonic{કેપ્ચર સ્ટોર પ્રોસેસ કોમ્યુનિકેટ પ્રેઝન્ટ - CSPCP મલ્ટીમીડિયા}
\end{mnemonicbox}

\questionmarks{5(અ OR)}{3}{કોમ્યુનિકેશન સિક્યુરિટીમાં બ્લોક ચેઇનનું મહત્વ સમજાવો}

\begin{solutionbox}
\begin{center}
\captionof{table}{બ્લોકચેઇન સિક્યુરિટી વિશેષતાઓ}
\begin{tabulary}{\linewidth}{|L|L|}
\hline
\textbf{વિશેષતા} & \textbf{લાભ} \\ \hline
\textbf{ડીસેન્ટ્રલાઇઝેશન} & કોઇ સિંગલ પોઇન્ટ ઓફ ફેઇલ્યુર નથી \\ \hline
\textbf{ઇમ્યુટેબિલિટી} & ભૂતકાળના રેકોર્ડ્સ બદલી શકાતા નથી \\ \hline
\textbf{ટ્રાન્સપેરન્સી} & બધા ટ્રાન્ઝેક્શન્સ દૃશ્યમાન \\ \hline
\end{tabulary}
\end{center}

\begin{itemize}
    \item \keyword{ક્રિપ્ટોગ્રાફિક સિક્યુરિટી}: હેશ ફંક્શન્સ અને ડિજિટલ સિગ્નેચર્સ
    \item \keyword{ડિસ્ટ્રિબ્યુટેડ લેજર}: બહુવિધ કોપીઓ ટેમ્પરિંગ અટકાવે છે
    \item \keyword{સ્માર્ટ કોન્ટ્રેક્ટ્સ}: ઓટોમેટેડ સિક્યુરિટી પ્રોટોકોલ્સ
    \item \keyword{ઉપયોગ}: સિક્યુર મેસેજિંગ, આઇડેન્ટિટી વેરિફિકેશન
\end{itemize}
\end{solutionbox}

\begin{mnemonicbox}
\mnemonic{બ્લોકચેઇન ડિસ્ટ્રિબ્યુટેડ ઇમ્યુટેબલ - BDI સિક્યુરિટી}
\end{mnemonicbox}

\questionmarks{5(બ OR)}{4}{5G ટેકનોલોજીના મહત્વના તત્વો, વિશેષતાઓ અને ફાયદાઓ સમજાવો}

\begin{solutionbox}
\begin{center}
\captionof{table}{5G સ્પેસિફિકેશન્સ}
\begin{tabulary}{\linewidth}{|L|L|}
\hline
\textbf{તત્વ} & \textbf{સ્પેસિફિકેશન} \\ \hline
\textbf{સ્પીડ} & 10 Gbps સુધી \\ \hline
\textbf{લેટન્સી} & 1 ms કરતાં ઓછી \\ \hline
\textbf{કનેક્શન્સ} & 1 મિલિયન ડિવાઇસ/km\textsuperscript{2} \\ \hline
\textbf{વિશ્વસનીયતા} & 99.999\% ઉપલબ્ધતા \\ \hline
\end{tabulary}
\end{center}

\textbf{મુખ્ય વિશેષતાઓ:}
\begin{itemize}
    \item \keyword{Enhanced Mobile Broadband}: અલ્ટ્રા-હાઇ-સ્પીડ ઇન્ટરનેટ
    \item \keyword{Ultra-Reliable Low Latency}: ક્રિટિકલ એપ્લિકેશન્સ
    \item \keyword{Massive Machine Communication}: IoT કનેક્ટિવિટી
    \item \keyword{Network Slicing}: કસ્ટમાઇઝ્ડ નેટવર્ક સર્વિસિસ
\end{itemize}

\textbf{ફાયદાઓ:}
\begin{itemize}
    \item \keyword{હાયર કેપેસિટી}: વધુ એકસાથે વપરાશકર્તાઓ
    \item \keyword{એનર્જી કાર્યક્ષમતા}: ડિવાઇસીસ માટે સારી બેટરી લાઇફ
    \item \keyword{નવી એપ્લિકેશન્સ}: AR/VR, ઓટોનોમસ વાહનો
\end{itemize}
\end{solutionbox}

\begin{mnemonicbox}
\mnemonic{5G સ્પીડ લેટન્સી કનેક્શન્સ - SLC વિશેષતાઓ}
\end{mnemonicbox}

\questionmarks{5(ક OR)}{7}{RS 232, RS 422 અને RS 485 સ્ટાન્ડર્ડની સરખામણી કરો}

\begin{solutionbox}
\begin{center}
\captionof{table}{RS સ્ટાન્ડર્ડ્સ સરખામણી}
\begin{tabulary}{\linewidth}{|L|L|L|L|}
\hline
\textbf{પેરામીટર} & \textbf{RS-232} & \textbf{RS-422} & \textbf{RS-485} \\ \hline
\textbf{મોડ} & સિંગલ-એન્ડેડ & ડિફરન્શિયલ & ડિફરન્શિયલ \\ \hline
\textbf{મહત્તમ અંતર} & 50 ફુટ & 4000 ફુટ & 4000 ફુટ \\ \hline
\textbf{મહત્તમ સ્પીડ} & 20 kbps & 10 Mbps & 10 Mbps \\ \hline
\textbf{ડ્રાઇવર્સ} & 1 & 1 & 32 \\ \hline
\textbf{રિસીવર્સ} & 1 & 10 & 32 \\ \hline
\textbf{ટોપોલોજી} & Pt-to-Pt & Pt-to-Multi & મલ્ટિપોઇન્ટ \\ \hline
\end{tabulary}
\end{center}

\begin{center}
\captionof{table}{વોલ્ટેજ લેવલ્સ}
\begin{tabulary}{\linewidth}{|L|L|L|}
\hline
\textbf{સ્ટાન્ડર્ડ} & \textbf{લોજિક 1} & \textbf{લોજિક 0} \\ \hline
\textbf{RS-232} & -3V થી -25V & +3V થી +25V \\ \hline
\textbf{RS-422} & Diff $< -200$mV & Diff $> +200$mV \\ \hline
\textbf{RS-485} & Diff $< -200$mV & Diff $> +200$mV \\ \hline
\end{tabulary}
\end{center}

\begin{itemize}
    \item \keyword{ઉપયોગ}: RS-232 (PC સીરિયલ), RS-422 (ઔદ્યોગિક), RS-485 (બિલ્ડિંગ ઓટોમેશન)
    \item \keyword{નોઇઝ ઇમ્યુનિટી}: RS-232 કરતાં RS-422/485 માં ડિફરન્શિયલ સિગ્નલિંગ વધુ સારું
    \item \keyword{અંતર ક્ષમતા}: RS-232 કરતાં RS-422/485 વધુ લાંબુ
    \item \keyword{કોસ્ટ}: RS-232 સસ્તું, RS-485 સૌથી જટિલ
\end{itemize}
\end{solutionbox}

\begin{mnemonicbox}
\mnemonic{RS-232 સિમ્પલ, RS-422 લોંગ, RS-485 મલ્ટી - SLM સ્ટાન્ડર્ડ્સ}
\end{mnemonicbox}

\end{document}

