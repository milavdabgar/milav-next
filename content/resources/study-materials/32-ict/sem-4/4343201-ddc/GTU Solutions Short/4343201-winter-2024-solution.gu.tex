\documentclass{article}

% content/resources/templates/preamble.tex
\usepackage[margin=0.6in]{geometry}
\author{Milav Dabgar}
\usepackage{amsmath,amssymb,amsthm}
\usepackage{booktabs}
\usepackage{multirow}
\usepackage{xcolor}
\usepackage{tcolorbox}
\tcbuselibrary{breakable,skins}
\usepackage[colorlinks=true,linkcolor=blue]{hyperref}
\usepackage{titlesec}
\usepackage{enumitem}
\usepackage{tikz}
\usepackage{pgfplots}
\usepackage{circuitikz}
\usepackage[version=4]{mhchem}
\usepackage{longtable}
\usepackage{array}
\usepackage{float}
\usepackage{caption}
\usepackage{listings}

\lstset{
  basicstyle=\small\ttfamily,
  breaklines=true,
  breakatwhitespace=false,
  postbreak=\mbox{\textcolor{red}{$\hookrightarrow$}\space},
  float=false,
  numbers=left,
  numberstyle=\tiny\color{gray},
  numbersep=10pt,
  xleftmargin=2em,
  keywordstyle=\color{blue},
  commentstyle=\color{green!60!black},
  stringstyle=\color{purple},
  backgroundcolor=\color{gray!5},
  showstringspaces=false,
  tabsize=2,
  captionpos=b,
  keepspaces=true,
  columns=flexible
}

\pgfplotsset{compat=1.18}
\usetikzlibrary{shapes,arrows,positioning,calc,patterns,decorations.pathmorphing,decorations.markings,arrows.meta}

% Color scheme
\definecolor{headcolor}{RGB}{0,102,204}
\definecolor{keycolor}{RGB}{220,20,60}
\definecolor{solutioncolor}{RGB}{34,139,34}
\definecolor{mnemoniccolor}{RGB}{148,0,211}
\definecolor{codecolor}{RGB}{0,0,100}

% Spacing
\setlength{\parskip}{3pt}
\setlist[itemize]{nosep}
\setlist[enumerate]{nosep}

% Title formatting
\titleformat{\section}{\Large\bfseries\color{headcolor}}{\thesection}{1em}{}
\titleformat{\subsection}{\large\bfseries\color{headcolor}}{\thesubsection}{1em}{}

% Pandoc tightlist compatibility
\providecommand{\tightlist}{%
  \setlength{\itemsep}{0pt}\setlength{\parskip}{0pt}}

% Pandoc longtable compatibility
\newcounter{none}
\def\thenone{}


% content/resources/templates/gujarati-boxes.tex
\usepackage{fontspec}
\usepackage{polyglossia}

% Set Gujarati as main language (document is primarily in Gujarati)
% Note: gloss-gujarati.ldf doesn't exist in polyglossia, but it will use hyphenation patterns
\setdefaultlanguage{gujarati}
\setotherlanguage{english}

% Configure Gujarati font properly
% Use Language=Default to prevent polyglossia from trying to add language-specific features
% that don't exist for Gujarati, which causes "empty feature" warnings
\newfontfamily\gujaratifont[Script=Gujarati,AutoFakeBold=2.5,AutoFakeSlant=0.3]{Noto Sans Gujarati}
\setmainfont[Script=Gujarati,AutoFakeBold=2.5,AutoFakeSlant=0.3]{Noto Sans Gujarati}
% Use Noto Sans Gujarati for monospace to support Gujarati in text
\setmonofont[Scale=0.9]{Noto Sans Gujarati}

% Configure English to use the same font
\newfontfamily\englishfont[Script=Gujarati,AutoFakeBold=2.5,AutoFakeSlant=0.3]{Noto Sans Gujarati}

% Translations for polyglossia
\gappto\captionsgujarati{
  \renewcommand{\tablename}{કોષ્ટક}
  \renewcommand{\figurename}{આકૃતિ}
}

% Helper for TikZ nodes to ensure Gujarati font
\newcommand{\gu}[1]{{\gujaratifont #1}}

% Custom environments
\newtcolorbox{solutionbox}{
    breakable,
    enhanced,
    colback=solutioncolor!5!white,
    colframe=solutioncolor!75!black,
    fonttitle=\bfseries,
    title=જવાબ
}

\newtcolorbox{solutionboxnobreak}{
 colback=solutioncolor!5!white,
 colframe=solutioncolor!75!black,
 fonttitle=\bfseries,
 title=જવાબ
}

\newtcolorbox{keyformula}{
 breakable,
 enhanced,
 colback=keycolor!5!white,
 colframe=keycolor!75!black,
 fonttitle=\bfseries,
 title=રાસાયણિક સમીકરણ/સૂત્ર
}

\newtcolorbox{mnemonicbox}{
 breakable,
 enhanced,
 colback=mnemoniccolor!5!white,
 colframe=mnemoniccolor!75!black,
 fonttitle=\bfseries,
 title=મેમરી ટ્રીક
}


% Custom commands for GTU solutions
% This file defines semantic commands for consistent formatting

% Question command with automatic formatting
\newcommand{\question}[2]{%
  \section*{Question #1}%
  \textbf{#2}%
}

% OR question variant
\newcommand{\questionor}[2]{%
  \section*{Question #1 OR}%
  \textbf{#2}%
}

% Proper table environment with caption
\newenvironment{answertable}[1]{%
  \begin{table}[htbp]
  \centering
  \caption{#1}
}{%
  \end{table}
}

% Proper figure environment for diagrams
\newenvironment{answerdiagram}[1]{%
  \begin{figure}[htbp]
  \centering
  \caption{#1}
}{%
  \end{figure}
}

% Semantic markup for key terms
\newcommand{\keyword}[1]{\textbf{#1}}
\newcommand{\code}[1]{\texttt{#1}}
\newcommand{\classname}[1]{\texttt{#1}}
\newcommand{\methodname}[1]{\texttt{#1}}

% Proper quotation marks
\newcommand{\mnemonic}[1]{``#1''}


\title{ડિજિટલ અને ડેટા કોમ્યુનિકેશન (4343201) - વિન્ટર 2024 સોલ્યુશન}
\date{November 26, 2024}

\begin{document}
\maketitle

\questionmarks{1(અ)}{3}{કોમ્યુનિકેશનની મૂળભૂત રીતોનો તફાવત આપો: બ્રોડ કાસ્ટિંગ કમ્યુનિકેશન અને પોઈન્ટ ટુ પોઈન્ટ કોમ્યુનિકેશન.}

\begin{solutionbox}
\begin{center}
\captionof{table}{બ્રોડકાસ્ટિંગ vs પોઈન્ટ ટુ પોઈન્ટ}
\begin{tabulary}{\linewidth}{|L|L|L|}
\hline
\textbf{પેરામીટર} & \textbf{બ્રોડકાસ્ટિંગ કમ્યુનિકેશન} & \textbf{પોઈન્ટ ટુ પોઈન્ટ કોમ્યુનિકેશન} \\ \hline
\textbf{વ્યાખ્યા} & એક ટ્રાન્સમીટર એક સાથે અનેક રિસીવર્સને સિગ્નલ મોકલે છે & એક ટ્રાન્સમીટર એક જ ચોક્કસ રિસીવર સાથે કમ્યુનિકેશન કરે છે \\ \hline
\textbf{દિશા} & એકદિશામાં (એકમાર્ગી) & દ્વિદિશામાં (દ્વિમાર્ગી) \\ \hline
\textbf{ઉદાહરણ} & ટીવી, રેડિયો, એફએમ & ટેલિફોન, મોબાઈલ કૉલ, પ્રાઈવેટ નેટવર્ક \\ \hline
\textbf{ગોપનીયતા} & ઓછી (મર્યાદામાં આવતા બધાને સિગ્નલ મળે છે) & વધારે (એન્ડપોઈન્ટ વચ્ચે ડેડિકેટેડ કનેક્શન) \\ \hline
\textbf{કાર્યક્ષમતા} & સામૂહિક કમ્યુનિકેશન માટે ઉત્તમ & વ્યક્તિગત/ખાનગી કમ્યુનિકેશન માટે વધુ સારું \\ \hline
\end{tabulary}
\end{center}
\end{solutionbox}

\begin{mnemonicbox}
\mnemonic{BDPEC - બ્રોડકાસ્ટિંગ ડિસ્ટ્રિબ્યુટ્સ ટુ પબ્લિક, એન્ડપોઈન્ટ્સ કનેક્ટ ઈન પોઈન્ટ-ટુ-પોઈન્ટ}
\end{mnemonicbox}

\questionmarks{1(બ)}{4}{વ્યાખ્યા આપો: બિટ રેટ, બોડ રેટ, બેન્ડવીડ્થ અને રીપીટર અંતર.}

\begin{solutionbox}
\begin{center}
\captionof{table}{વ્યાખ્યાઓ}
\begin{tabulary}{\linewidth}{|L|L|}
\hline
\textbf{પદ} & \textbf{વ્યાખ્યા} \\ \hline
\textbf{બિટ રેટ} & એક સેકન્ડમાં ટ્રાન્સમિટ થતા બાઈનરી બિટ્સની સંખ્યા (bps). વાસ્તવિક ડેટા ટ્રાન્સફર સ્પીડ માપે છે. \\ \hline
\textbf{બોડ રેટ} & એક સેકન્ડમાં ટ્રાન્સમિટ થતા સિગ્નલ યુનિટ્સ કે સિમ્બોલ્સની સંખ્યા. એક સિમ્બોલમાં એકથી વધુ બિટ હોઈ શકે. \\ \hline
\textbf{બેન્ડવીડ્થ} & સિગ્નલ દ્વારા ઉપયોગમાં લેવાતી ફ્રિક્વન્સીઓની રેન્જ, હર્ટ્ઝ (Hz)માં માપવામાં આવે છે. ચેનલની મહત્તમ ડેટા ક્ષમતા નક્કી કરે છે. \\ \hline
\textbf{રીપીટર અંતર} & કમ્યુનિકેશન સિસ્ટમમાં રીપીટર્સ વચ્ચેનું મહત્તમ અંતર જ્યાં સુધી સિગ્નલ ડિગ્રેડેશન પહેલાં રીજનરેશનની જરૂર પડે છે. \\ \hline
\end{tabulary}
\end{center}

\begin{center}
\begin{tikzpicture}[node distance=1.5cm, auto]
    \node [gtu block] (sig) {સિગ્નલ};
    \node [gtu block, right=2cm of sig] (bw) {બેન્ડવીડ્થ};
    \node [gtu block, below=1cm of sig] (bits) {બિટ્સ};
    \node [gtu block, right=2cm of bits] (br) {બિટ રેટ};
    \node [gtu block, below=1cm of bits] (sym) {સિમ્બોલ્સ};
    \node [gtu block, right=2cm of sym] (baurd) {બોડ રેટ};
    
    \draw [gtu arrow] (sig) -- node {ફ્રિક્વન્સી રેન્જ} (bw);
    \draw [gtu arrow] (bits) -- node {પ્રતિ સેકન્ડ} (br);
    \draw [gtu arrow] (sym) -- node {પ્રતિ સેકન્ડ} (baurd);
\end{tikzpicture}
\captionof{figure}{કોમ્યુનિકેશન રેટ કોન્સેપ્ટ્સ}
\end{center}
\end{solutionbox}

\begin{mnemonicbox}
\mnemonic{BBRR - બેટર બેન્ડવીડ્થ રિક્વાયર્સ રીપીટર્સ}
\end{mnemonicbox}

\questionmarks{1(ક)}{7}{ડિજિટલ કોમ્યુનિકેશન સિસ્ટમનો બ્લોક ડાયાગ્રામ દોરો. દરેક બ્લોકના કાર્યોને સંક્ષિપ્તમાં સમજાવો. તેના ફાયદા અને ગેરફાયદા જણાવો.}

\begin{solutionbox}
\begin{center}
\begin{tikzpicture}[node distance=1.5cm, auto, font=\small]
    \node [gtu block] (source) {ઇનપુટ સોર્સ};
    \node [gtu block, right=0.8cm of source] (senc) {સોર્સ એન્કોડર};
    \node [gtu block, right=0.8cm of senc] (cenc) {ચેનલ એન્કોડર};
    \node [gtu block, right=0.8cm of cenc] (mod) {ડિજિટલ મોડ્યુલેટર};
    
    \node [gtu block, below=1.5cm of mod] (channel) {ચેનલ};
    \node [gtu block, below=1.5cm of channel] (demod) {ડિજિટલ ડિમોડ્યુલેટર};
    \node [gtu block, left=0.8cm of demod] (cdec) {ચેનલ ડિકોડર};
    \node [gtu block, left=0.8cm of cdec] (sdec) {સોર્સ ડિકોડર};
    \node [gtu block, left=0.8cm of sdec] (dest) {આઉટપુટ};
    
    \draw [gtu arrow] (source) -- (senc);
    \draw [gtu arrow] (senc) -- (cenc);
    \draw [gtu arrow] (cenc) -- (mod);
    \draw [gtu arrow] (mod) -- (channel);
    \draw [gtu arrow] (channel) -- (demod);
    \draw [gtu arrow] (demod) -- (cdec);
    \draw [gtu arrow] (cdec) -- (sdec);
    \draw [gtu arrow] (sdec) -- (dest);
\end{tikzpicture}
\captionof{figure}{ડિજિટલ કોમ્યુનિકેશન સિસ્ટમ}
\end{center}

\begin{itemize}
    \item \keyword{સોર્સ એન્કોડર}: એનાલોગ સિગ્નલને ડિજિટલમાં કન્વર્ટ કરે છે, રિડન્ડન્સી દૂર કરે છે.
    \item \keyword{ચેનલ એન્કોડર}: ભૂલ શોધવા અને સુધારવા માટે રિડન્ડન્સી ઉમેરે છે.
    \item \keyword{ડિજિટલ મોડ્યુલેટર}: ડિજિટલ ડેટાને ટ્રાન્સમિશન માટે યોગ્ય ફોર્મમાં કન્વર્ટ કરે છે.
    \item \keyword{ચેનલ}: માધ્યમ જેના દ્વારા સિગ્નલ પ્રવાસ કરે છે.
    \item \keyword{ડિજિટલ ડિમોડ્યુલેટર}: મોડ્યુલેટેડ સિગ્નલમાંથી મૂળ ડિજિટલ ડેટા એક્સટ્રેક્ટ કરે છે.
    \item \keyword{ચેનલ ડિકોડર}: ભૂલો શોધે અને સુધારે છે.
    \item \keyword{સોર્સ ડિકોડર}: ડેટાને મૂળ સ્વરૂપમાં કન્વર્ટ કરે છે.
\end{itemize}

\begin{center}
\captionof{table}{ફાયદા અને ગેરફાયદા}
\begin{tabulary}{\linewidth}{|L|L|}
\hline
\textbf{ફાયદા} & \textbf{ગેરફાયદા} \\ \hline
નોઇઝ સામે સારી રક્ષા & વધુ બેન્ડવીડ્થની જરૂર પડે છે \\ \hline
સિગ્નલ રીજનરેશન સરળ & જટિલ અમલીકરણ \\ \hline
સુરક્ષિત ટ્રાન્સમિશન શક્ય & સિન્ક્રોનાઇઝેશનની જરૂર છે \\ \hline
કમ્પ્યુટર સાથે સરળ એકીકરણ & ક્વોન્ટાઇઝેશન ભૂલો \\ \hline
લાંબા અંતર માટે સારી ગુણવત્તા & સરળ એપ્લિકેશન માટે વધુ ખર્ચ \\ \hline
\end{tabulary}
\end{center}
\end{solutionbox}

\begin{mnemonicbox}
\mnemonic{SECDCSO - સિક્યોર એન્કોડિંગ ક્રિએટ્સ ડિજિટલ કમ્યુનિકેશન સિસ્ટમ આઉટપુટ}
\end{mnemonicbox}

\questionmarks{1(ક) OR}{7}{ડિજિટલ કોમ્યુનિકેશન માટે મલ્ટિપ્લેક્સિંગ તકનીકોની જરૂરિયાતોને ન્યાયી ઠેરવો. ટાઇમ ડિવિઝન મલ્ટિપ્લેક્સિંગ ટેકનિક દોરો અને સંક્ષિપ્તમાં સમજાવો. તેના ફાયદા અને ગેરફાયદાની ચર્ચા કરો.}

\begin{solutionbox}
\textbf{મલ્ટિપ્લેક્સિંગની જરૂરિયાત:}
\begin{center}
\captionof{table}{જરૂરિયાત}
\begin{tabulary}{\linewidth}{|L|L|}
\hline
\textbf{જરૂરિયાત} & \textbf{સમજૂતી} \\ \hline
\textbf{ચેનલ કાર્યક્ષમતા} & એક ચેનલ પર અનેક સિગ્નલ્સ, બેન્ડવીડ્થ બચાવે છે \\ \hline
\textbf{ખર્ચ ઘટાડો} & ટ્રાન્સમિશન માધ્યમોની જરૂરિયાત ઘટાડે છે \\ \hline
\textbf{ઇન્ફ્રાસ્ટ્રક્ચર ઉપયોગ} & ઇન્ફ્રાસ્ટ્રક્ચરનો મહત્તમ ઉપયોગ કરે છે \\ \hline
\textbf{સ્પેક્ટ્રમ સંરક્ષણ} & ફ્રિક્વન્સી સ્પેક્ટ્રમનું સંરક્ષણ કરે છે \\ \hline
\end{tabulary}
\end{center}

\textbf{ટાઇમ ડિવિઝન મલ્ટિપ્લેક્સિંગ (TDM):}
\begin{center}
\begin{tikzpicture}[node distance=1.5cm, auto]
    \node [gtu block] (mux) {મલ્ટિપ્લેક્સર};
    \node [gtu block, right=3cm of mux] (demux) {ડિમલ્ટિપ્લેક્સર};
    
    \node [left=1cm of mux] (in2) {ઇનપુટ 2};
    \node [above=0.5cm of in2] (in1) {ઇનપુટ 1};
    \node [below=0.5cm of in2] (in3) {ઇનપુટ 3};
    \node [below=0.5cm of in2] (in4) {ઇનપુટ 4};
    
    \node [right=1cm of demux] (out2) {આઉટપુટ 2};
    \node [above=0.5cm of out2] (out1) {આઉટપુટ 1};
    \node [below=0.5cm of out2] (out3) {આઉટપુટ 3};
    \node [below=0.5cm of out2] (out4) {આઉટપુટ 4};
    
    \draw [gtu arrow] (in1) -- (mux.west |- in1);
    \draw [gtu arrow] (in2) -- (mux.west |- in2);
    \draw [gtu arrow] (in3) -- (mux.west |- in3);
    \draw [gtu arrow] (in4) -- (mux.west |- in4);
    
    \draw [gtu arrow] (mux) -- node {ટ્રાન્સમિશન ચેનલ} (demux);
    
    \draw [gtu arrow] (demux.east |- out1) -- (out1);
    \draw [gtu arrow] (demux.east |- out2) -- (out2);
    \draw [gtu arrow] (demux.east |- out3) -- (out3);
    \draw [gtu arrow] (demux.east |- out4) -- (out4);
\end{tikzpicture}
\captionof{figure}{ટાઇમ ડિવિઝન મલ્ટિપ્લેક્સિંગ (TDM)}
\end{center}

\begin{itemize}
    \item \keyword{કાર્યપદ્ધતિ}: TDMમાં, દરેક ઇનપુટ સિગ્નલને એક ચોક્કસ ટાઇમ સ્લોટ મળે છે. રિસીવર પર, ડિમલ્ટિપ્લેક્સર ટાઇમિંગના આધારે સ્ટ્રીમને મૂળ સિગ્નલ્સમાં અલગ કરે છે.
\end{itemize}

\begin{center}
\captionof{table}{ફાયદા અને ગેરફાયદા}
\begin{tabulary}{\linewidth}{|L|L|}
\hline
\textbf{ફાયદા} & \textbf{ગેરફાયદા} \\ \hline
કાર્યક્ષમ બેન્ડવીડ્થ ઉપયોગ & સિન્ક્રોનાઇઝેશન જરૂરી છે \\ \hline
ગાર્ડ બેન્ડની જરૂર નથી & જટિલ બફરિંગની જરૂર પડે છે \\ \hline
ક્રોસ-ટોક નથી & ટાઇમિંગ સમસ્યાઓ ભૂલો પેદા કરી શકે છે \\ \hline
ફ્લેક્સિબલ એલોકેશન & વણવપરાયેલા સ્લોટ્સ ક્ષમતા બગાડે છે \\ \hline
ડિજિટલ અમલીકરણ & વ્યક્તિગત ચેનલો કરતાં વધુ ડેટા રેટ \\ \hline
\end{tabulary}
\end{center}
\end{solutionbox}

\begin{mnemonicbox}
\mnemonic{TIME - ટ્રાન્સમિશન ઇન્ટરલીવ્સ મલ્ટિપલ એન્ડપોઇન્ટ્સ}
\end{mnemonicbox}

\questionmarks{2(અ)}{3}{તફાવત કરો: કોહેરેંટ અને નોન-કોહેરેન્ટ ડીટેક્શન ટેક્નીક}

\begin{solutionbox}
\begin{center}
\captionof{table}{કોહેરેંટ vs નોન-કોહેરેંટ ડિટેક્શન}
\begin{tabulary}{\linewidth}{|L|L|L|}
\hline
\textbf{પેરામીટર} & \textbf{કોહેરેંટ ડિટેક્શન} & \textbf{નોન-કોહેરેંટ ડિટેક્શન} \\ \hline
\textbf{ફેઝ ઇન્ફોર્મેશન} & ફેઝ ઇન્ફોર્મેશનનો ઉપયોગ કરે છે & ફેઝ ઇન્ફોર્મેશનને અવગણે છે \\ \hline
\textbf{લોકલ ઓસિલેટર} & જરૂરી છે & જરૂરી નથી \\ \hline
\textbf{જટિલતા} & વધુ જટિલ & સરળ \\ \hline
\textbf{પરફોર્મન્સ} & નોઇઝમાં વધુ સારું & નોઇઝમાં ઓછું કાર્યક્ષમ \\ \hline
\textbf{અમલીકરણ} & મુશ્કેલ & સરળ \\ \hline
\textbf{એપ્લિકેશન્સ} & ઉચ્ચ-ગુણવત્તા સિસ્ટમો & ઓછી-કિંમતની સિસ્ટમો \\ \hline
\end{tabulary}
\end{center}
\end{solutionbox}

\begin{mnemonicbox}
\mnemonic{PLCPIA - ફેઝ લોકલ કોમ્પ્લેક્સ પરફોર્મન્સ ઇમ્પ્લિમેન્ટેશન એપ્લિકેશન્સ}
\end{mnemonicbox}

\questionmarks{2(બ)}{4}{ડેટા સિક્વન્સ 101100110110 માટે ASK, FSK, PSK અને QPSK વેવફોર્મ દોરો.}

\begin{solutionbox}
\begin{center}
\begin{tikzpicture}[x=0.5cm,y=0.6cm]
    % Digital Data
    \node[anchor=east] at (-1, 1) {Data:};
    \foreach \x/\val in {0/1, 1/0, 2/1, 3/1, 4/0, 5/0, 6/1, 7/1, 8/0, 9/1, 10/1, 11/0} {
        \draw (\x,0) -- (\x,\val) -- (\x+1,\val) -- (\x+1,0); 
        \node at (\x+0.5, 1.5) {\val};
    }
    
    % ASK
    \node[anchor=east] at (-1, -1.5) {ASK:};
    \foreach \x/\val in {0/1, 1/0, 2/1, 3/1, 4/0, 5/0, 6/1, 7/1, 8/0, 9/1, 10/1, 11/0} {
        \ifnum\val=1
            \draw[domain=\x:\x+1, samples=20] plot (\x, {sin(360*(\x)*2) * 0.8 - 1.5});
        \else
            \draw (\x,-1.5) -- (\x+1,-1.5);
        \fi
    }
    
    % FSK
    \node[anchor=east] at (-1, -4) {FSK:};
    \foreach \x/\val in {0/1, 1/0, 2/1, 3/1, 4/0, 5/0, 6/1, 7/1, 8/0, 9/1, 10/1, 11/0} {
        \ifnum\val=1
            \draw[domain=\x:\x+1, samples=40] plot (\x, {sin(360*(\x)*4) * 0.8 - 4});
        \else
            \draw[domain=\x:\x+1, samples=20] plot (\x, {sin(360*(\x)*2) * 0.8 - 4});
        \fi
    }
    
    % PSK
    \node[anchor=east] at (-1, -6.5) {PSK:};
    \foreach \x/\val in {0/1, 1/0, 2/1, 3/1, 4/0, 5/0, 6/1, 7/1, 8/0, 9/1, 10/1, 11/0} {
        \ifnum\val=1
            \draw[domain=\x:\x+1, samples=20] plot (\x, {sin(360*(\x)*2) * 0.8 - 6.5});
        \else
            \draw[domain=\x:\x+1, samples=20] plot (\x, {-sin(360*(\x)*2) * 0.8 - 6.5});
        \fi
    }
    
\end{tikzpicture}
\captionof{figure}{મોડ્યુલેશન વેવફોર્મ્સ}
\end{center}
\end{solutionbox}

\begin{mnemonicbox}
\mnemonic{AFPQ - એમ્પ્લિટ્યુડ ફ્રિક્વન્સી ફેઝ ક્વોડ્રેચર}
\end{mnemonicbox}

\questionmarks{2(ક)}{7}{16-QAMનો સિદ્ધાંત સમજાવો. 16-QAM માટે નક્ષત્ર આકૃતિ અને વેવફોર્મ પણ સમજાવો. તેના ફાયદા અને ગેરફાયદા લખો.}

\begin{solutionbox}
\textbf{16-QAMનો સિદ્ધાંત:}
16-QAM એમ્પ્લિટ્યુડ અને ફેઝ મોડ્યુલેશનને જોડે છે. 16 જુદા જુદા સંયોજનો વાપરે છે, જે સમાન બેન્ડવીડ્થમાં ઉચ્ચ ડેટા રેટ આપે છે.

\textbf{નક્ષત્ર આકૃતિ (Constellation Diagram):}
\begin{center}
\begin{tikzpicture}[scale=1.2]
    \draw[->] (-3,0) -- (3,0) node[right] {I};
    \draw[->] (0,-3) -- (0,3) node[above] {Q};
    
    \foreach \x in {-1.5, -0.5, 0.5, 1.5} {
        \foreach \y in {-1.5, -0.5, 0.5, 1.5} {
            \draw[fill] (\x,\y) circle (2pt);
        }
    }
    \node at (2, -2.5) {16 Points};
\end{tikzpicture}
\captionof{figure}{16-QAM નક્ષત્ર આકૃતિ}
\end{center}

\begin{center}
\captionof{table}{ફાયદા અને ગેરફાયદા}
\begin{tabulary}{\linewidth}{|L|L|}
\hline
\textbf{ફાયદા} & \textbf{ગેરફાયદા} \\ \hline
ઉચ્ચ સ્પેક્ટ્રલ કાર્યક્ષમતા & નોઇઝ અને ઇન્ટરફેરન્સ પ્રત્યે સંવેદનશીલ \\ \hline
ઉચ્ચ ડેટા રેટ & ઉચ્ચ SNRની જરૂર પડે છે \\ \hline
બેન્ડવીડ્થ કાર્યક્ષમ & જટિલ અમલીકરણ \\ \hline
ચેનલ ક્ષમતાનો સારો ઉપયોગ & એમ્પ્લિટ્યુડ વિકૃતિ પ્રત્યે સંવેદનશીલ \\ \hline
\end{tabulary}
\end{center}
\end{solutionbox}

\begin{mnemonicbox}
\mnemonic{SCHAP - સિક્સટીન કોમ્બિનેશન્સ હેવ એમ્પ્લિટ્યુડ એન્ડ ફેઝ}
\end{mnemonicbox}

\questionmarks{2(અ) OR}{3}{સરખામણી કરો: ASK અને PSK}

\begin{solutionbox}
\begin{center}
\captionof{table}{ASK vs PSK}
\begin{tabulary}{\linewidth}{|L|L|L|}
\hline
\textbf{પેરામીટર} & \textbf{ASK (એમ્પ્લિટ્યુડ શિફ્ટ કીઇંગ)} & \textbf{PSK (ફેઝ શિફ્ટ કીઇંગ)} \\ \hline
\textbf{મોડ્યુલેશન પેરામીટર} & એમ્પ્લિટ્યુડ & ફેઝ \\ \hline
\textbf{નોઇઝ ઇમ્યુનિટી} & નબળી & સારી \\ \hline
\textbf{પાવર એફિશિયન્સી} & ઓછી કાર્યક્ષમ & વધુ કાર્યક્ષમ \\ \hline
\textbf{બેન્ડવીડ્થ એફિશિયન્સી} & નીચી & ઉંચી \\ \hline
\textbf{અમલીકરણ} & સરળ & વધુ જટિલ \\ \hline
\textbf{BER પર્ફોર્મન્સ} & ઉચ્ચ ભૂલ દર & નીચો ભૂલ દર \\ \hline
\end{tabulary}
\end{center}
\end{solutionbox}

\begin{mnemonicbox}
\mnemonic{ANPBIP - એમ્પ્લિટ્યુડ નોઇઝ પાવર બેન્ડવીડ્થ ઇમ્પ્લિમેન્ટેશન પર્ફોર્મન્સ}
\end{mnemonicbox}

\questionmarks{2(બ) OR}{4}{BPSK મોડ્યુલેટર અને ડિમોડ્યુલેટરનો બ્લોક ડાયાગ્રામ દોરો.}

\begin{solutionbox}
\textbf{BPSK મોડ્યુલેટર:}
\begin{center}
\begin{tikzpicture}[node distance=1.5cm, auto]
    \node [gtu block] (nrz) {NRZ એન્કોડર};
    \node [gtu block, right=1.5cm of nrz] (mod) {મલ્ટિપ્લાયર};
    \node [above=1cm of mod] (osc) {કેરિયર જનરેટર};
    \node [left=1.5cm of nrz] (in) {બાઇનરી ઇનપુટ};
    \node [right=1.5cm of mod] (out) {BPSK આઉટપુટ};
    
    \draw [gtu arrow] (in) -- (nrz);
    \draw [gtu arrow] (nrz) -- (mod);
    \draw [gtu arrow] (osc) -- (mod);
    \draw [gtu arrow] (mod) -- (out);
\end{tikzpicture}
\captionof{figure}{BPSK મોડ્યુલેટર}
\end{center}

\textbf{BPSK ડિમોડ્યુલેટર:}
\begin{center}
\begin{tikzpicture}[node distance=1.5cm, auto]
    \node [gtu block] (mult) {મલ્ટિપ્લાયર};
    \node [gtu block, right=1.5cm of mult] (lpf) {લો પાસ ફિલ્ટર};
    \node [gtu block, right=1.5cm of lpf] (dec) {ડિસીઝન ડિવાઇસ};
    
    \node [above=1cm of mult] (local) {ફેઝ સિન્ક્રોનાઇઝર};
    \node [left=1cm of local] (osc) {લોકલ ઓસીલેટર};
    
    \node [left=1.5cm of mult] (in) {BPSK ઇનપુટ};
    \node [right=1.5cm of dec] (out) {બાઇનરી આઉટપુટ};
    
    \draw [gtu arrow] (in) -- (mult);
    \draw [gtu arrow] (osc) -- (local);
    \draw [gtu arrow] (local) -| (mult);
    \draw [gtu arrow] (mult) -- (lpf);
    \draw [gtu arrow] (lpf) -- (dec);
    \draw [gtu arrow] (dec) -- (out);
\end{tikzpicture}
\captionof{figure}{BPSK ડિમોડ્યુલેટર}
\end{center}
\end{solutionbox}

\begin{mnemonicbox}
\mnemonic{MNECO - મોડ્યુલેશન નીડ્સ એન્કોડિંગ, કેરિયર્સ, ઓસીલેટર્સ}
\end{mnemonicbox}

\questionmarks{2(ક) OR}{7}{બ્લોક ડાયાગ્રામ અને વેવફોર્મની મદદથી QPSK જનરેશન અને ડિટેક્શન સમજાવો. તેના ફાયદા અને ગેરફાયદાની ચર્ચા કરો.}

\begin{solutionbox}
\textbf{QPSK જનરેશન બ્લોક ડાયાગ્રામ:}
\begin{center}
\begin{tikzpicture}[node distance=1.5cm, auto]
    \node [gtu block] (sp) {સીરીયલ ટુ પેરેલલ};
    \node [gtu block, right=2cm of sp, yshift=1cm] (mult1) {મલ્ટિપ્લાયર I};
    \node [gtu block, right=2cm of sp, yshift=-1cm] (mult2) {મલ્ટિપ્લાયર Q};
    \node [gtu block, right=2cm of mult1, yshift=-1cm] (adder) {એડર};
    \node [right=1cm of adder] (out) {QPSK આઉટપુટ};
    
    \node [left=1cm of sp] (input) {બાઇનરી ઇનપુટ};
    \node [above=1.5cm of sp] (osc) {કેરિયર જનરેટર};
    \node [right=1cm of osc] (shift) {90° ફેઝ શિફ્ટર};
    
    \draw [gtu arrow] (input) -- (sp);
    \draw [gtu arrow] (sp) -- node[above, sloped] {I-ચેનલ} (mult1);
    \draw [gtu arrow] (sp) -- node[below, sloped] {Q-ચેનલ} (mult2);
    
    \draw [gtu arrow] (osc) -| (mult1);
    \draw [gtu arrow] (osc) -- (shift);
    \draw [gtu arrow] (shift) -| (mult2);
    
    \draw [gtu arrow] (mult1) -| (adder);
    \draw [gtu arrow] (mult2) -| (adder);
    \draw [gtu arrow] (adder) -- (out);
\end{tikzpicture}
\captionof{figure}{QPSK જનરેશન}
\end{center}

\textbf{QPSK ડિટેક્શન બ્લોક ડાયાગ્રામ:}
\begin{center}
\begin{tikzpicture}[node distance=1.5cm, auto]
    \node [gtu block] (mult1) {મલ્ટિપ્લાયર I};
    \node [gtu block] (mult2) [below=2cm of mult1] {મલ્ટિપ્લાયર Q};
    \node [left=1.5cm of mult1] (in) {QPSK ઇનપુટ};
    
    \node [above=1cm of mult1] (osc) {લોકલ ઓસીલેટર};
    \node [right=1cm of osc] (shift) {90° ફેઝ શિફ્ટર};
    
    \node [gtu block, right=1cm of mult1] (lpf1) {LPF I};
    \node [gtu block, right=1cm of mult2] (lpf2) {LPF Q};
    
    \node [gtu block, right=1cm of lpf1] (dec1) {ડિસીઝન I};
    \node [gtu block, right=1cm of lpf2] (dec2) {ડિસીઝન Q};
    
    \node [gtu block, right=1cm of dec1, yshift=-1cm] (ps) {પેરેલલ ટુ સીરીયલ};
    \node [right=1cm of ps] (out) {બાઇનરી આઉટપુટ};
    
    \draw [gtu arrow] (in) -- (mult1);
    \draw [gtu arrow] (in) |- (mult2);
    
    \draw [gtu arrow] (osc) -| (mult1);
    \draw [gtu arrow] (osc) -- (shift);
    \draw [gtu arrow] (shift) -| (mult2);
    
    \draw [gtu arrow] (mult1) -- (lpf1);
    \draw [gtu arrow] (mult2) -- (lpf2);
    \draw [gtu arrow] (lpf1) -- (dec1);
    \draw [gtu arrow] (lpf2) -- (dec2);
    \draw [gtu arrow] (dec1) -| (ps);
    \draw [gtu arrow] (dec2) -| (ps);
    \draw [gtu arrow] (ps) -- (out);
\end{tikzpicture}
\captionof{figure}{QPSK ડિટેક્શન}
\end{center}

\begin{center}
\captionof{table}{ફાયદા અને ગેરફાયદા}
\begin{tabulary}{\linewidth}{|L|L|}
\hline
\textbf{ફાયદા} & \textbf{ગેરફાયદા} \\ \hline
BPSKની તુલનામાં બમણો ડેટા રેટ & વધુ જટિલ અમલીકરણ \\ \hline
BPSK જેટલું જ બેન્ડવીડ્થ & ફેઝ ભૂલો પ્રત્યે સંવેદનશીલ \\ \hline
સારી નોઇઝ ઇમ્યુનિટી & કેરિયર રિકવરીની જરૂર પડે છે \\ \hline
સ્પેક્ટ્રલ કાર્યક્ષમતા & વધુ જટિલ સિન્ક્રોનાઇઝેશન \\ \hline
\end{tabulary}
\end{center}
\end{solutionbox}

\begin{mnemonicbox}
\mnemonic{PACE - ફેઝ અલ્ટરેશન કેરીસ એક્સ્ટ્રા ડેટા}
\end{mnemonicbox}

\questionmarks{3(અ)}{3}{RS-422ની લાક્ષણિકતાઓ જણાવો.}

\begin{solutionbox}
\begin{center}
\captionof{table}{RS-422 ની વિશેષતાઓ}
\begin{tabulary}{\linewidth}{|L|}
\hline
\textbf{RS-422 ની વિશેષતાઓ} \\ \hline
\textbf{ડિફરન્શિયલ સિગ્નલિંગ} નોઇઝ ઇમ્યુનિટી માટે \\ \hline
\textbf{મહત્તમ ડેટા રેટ} 10 Mbps \\ \hline
\textbf{મહત્તમ કેબલ લંબાઈ} 1200 મીટર \\ \hline
\textbf{મલ્ટિ-ડ્રોપ ક્ષમતા} (1 ડ્રાઈવર, 10 રિસીવર્સ સુધી) \\ \hline
\textbf{બેલેન્સ્ડ ટ્રાન્સમિશન લાઇન} \\ \hline
\textbf{તુલનાત્મક રીતે ઉચ્ચ નોઇઝ ઇમ્યુનિટી} (RS-232 કરતા) \\ \hline
\end{tabulary}
\end{center}
\end{solutionbox}

\begin{mnemonicbox}
\mnemonic{DMMBHN - ડિફરન્શિયલ મેક્સિમમ મલ્ટિ-ડ્રોપ બેલેન્સ્ડ હાયર નોઇઝ-ઇમ્યુનિટી}
\end{mnemonicbox}

\questionmarks{3(બ)}{4}{વ્યાખ્યા આપો: એન્ટ્રોપી, ઇન્ફોર્મેશન, મ્યુચ્યુઅલ ઇન્ફોર્મેશન અને પ્રોબેબિલિટી.}

\begin{solutionbox}
\begin{center}
\captionof{table}{વ્યાખ્યાઓ}
\begin{tabulary}{\linewidth}{|L|L|}
\hline
\textbf{પદ} & \textbf{વ્યાખ્યા} \\ \hline
\textbf{એન્ટ્રોપી} & મેસેજ સોર્સમાં રહેલી અનિશ્ચિતતા અથવા રૅન્ડમનેસનું માપ, સૂત્ર: $H(X) = -\sum p(x)\log_2 p(x)$ \\ \hline
\textbf{ઇન્ફોર્મેશન} & મેસેજ પ્રાપ્ત થાય ત્યારે અનિશ્ચિતતામાં થતો ઘટાડો, બિટ્સમાં માપવામાં આવે છે \\ \hline
\textbf{મ્યુચ્યુઅલ ઇન્ફોર્મેશન} & બે રૅન્ડમ વેરિયેબલ્સ વચ્ચેની અવલંબનનું માપ \\ \hline
\textbf{પ્રોબેબિલિટી} & કોઈ ઘટના બનવાની શક્યતાનું ગાણિતિક માપ, 0 (અશક્ય) થી 1 (ચોક્કસ) સુધી \\ \hline
\end{tabulary}
\end{center}

\begin{center}
\begin{tikzpicture}[node distance=2cm, auto]
    \node [gtu block] (hX) {એન્ટ્રોપી H(X)};
    \node [gtu block, right=2cm of hX] (hY) {એન્ટ્રોપી H(Y)};
    \node [gtu block, below=1.5cm of hX, xshift=2cm] (mi) {મ્યુચ્યુઅલ ઇન્ફો I(X;Y)};
    
    \draw [gtu arrow] (hX) -- (mi);
    \draw [gtu arrow] (hY) -- (mi);
    \node [below=0.5cm of mi] {સહિયારી માહિતી માપે છે};
\end{tikzpicture}
\captionof{figure}{ઇન્ફોર્મેશન થિયરી કોન્સેપ્ટ્સ}
\end{center}
\end{solutionbox}

\begin{mnemonicbox}
\mnemonic{EIMP - એન્ટ્રોપી ઇન્ફોર્મેશન મેઝર્સ પ્રોબેબિલિટી}
\end{mnemonicbox}

\questionmarks{3(ક)}{7}{યોગ્ય ઉદાહરણ સાથે હફમેન કોડ અને શેનોન-ફેનો કોડ સમજાવો.}

\begin{solutionbox}
\textbf{હફમેન કોડ:}
હફમેન કોડિંગ ફ્રિક્વન્સીના આધારે સિમ્બોલ્સને વેરિયેબલ-લેન્થ કોડ આપે છે, વધુ વારંવાર આવતા સિમ્બોલ્સ માટે ટૂંકા કોડ વપરાય છે.

\textbf{ઉદાહરણ:}
\begin{center}
\captionof{table}{હફમેન ઉદાહરણ}
\begin{tabulary}{\linewidth}{|C|C|C|}
\hline
\textbf{સિમ્બોલ} & \textbf{ફ્રિક્વન્સી} & \textbf{હફમેન કોડ} \\ \hline
A & 45\% & 0 \\ \hline
B & 25\% & 10 \\ \hline
C & 15\% & 110 \\ \hline
D & 10\% & 1110 \\ \hline
E & 5\% & 1111 \\ \hline
\end{tabulary}
\end{center}

\textbf{હફમેન ટ્રી:}
\begin{center}
\begin{tikzpicture}[level distance=1.5cm, level 1/.style={sibling distance=3cm}, level 2/.style={sibling distance=1.5cm}]
    \node {100\%}
        child {node {A (45\%) \textbf{0}}}
        child {node {55\%}
            child {node {B (25\%) \textbf{10}}}
            child {node {30\%}
                child {node {C (15\%) \textbf{110}}}
                child {node {15\%}
                    child {node {D (10\%) \textbf{1110}}}
                    child {node {E (5\%) \textbf{1111}}}
                }
            }
        };
\end{tikzpicture}
\captionof{figure}{હફમેન ટ્રી}
\end{center}

\textbf{શેનોન-ફેનો કોડ:}
શેનોન-ફેનો અલ્ગોરિધમ સિમ્બોલ્સને સમાન ફ્રિક્વન્સી ધરાવતા બે જૂથોમાં વિભાજીત કરે છે, એક જૂથને 0 અને બીજાને 1 આપે છે.

\textbf{શેનોન-ફેનો ટ્રી:}
\begin{center}
\begin{tikzpicture}[level distance=1.5cm, level 1/.style={sibling distance=4cm}, level 2/.style={sibling distance=2cm}]
    \node {ABCDE}
        child {node {A (45\%) \textbf{0}}}
        child {node {BCDE}
            child {node {B (25\%) \textbf{10}}}
            child {node {CDE}
                child {node {C (15\%) \textbf{110}}}
                child {node {DE}
                    child {node {D (10\%) \textbf{1110}}}
                    child {node {E (5\%) \textbf{1111}}}
                }
            }
        };
\end{tikzpicture}
\captionof{figure}{શેનોન-ફેનો ટ્રી}
\end{center}
\end{solutionbox}

\begin{mnemonicbox}
\mnemonic{FREDS - ફ્રિક્વન્સી રિડ્યુસિસ એન્કોડિંગ ડિજિટ સાઇઝ}
\end{mnemonicbox}

\questionmarks{3(અ) OR}{3}{RS-232 ની લાક્ષણિકતાઓ જણાવો.}

\begin{solutionbox}
\begin{center}
\captionof{table}{RS-232 ની વિશેષતાઓ}
\begin{tabulary}{\linewidth}{|L|}
\hline
\textbf{RS-232 ની વિશેષતાઓ} \\ \hline
\textbf{સિંગલ-એન્ડેડ સિગ્નલિંગ} \\ \hline
\textbf{મહત્તમ ડેટા રેટ} 20 kbps \\ \hline
\textbf{મહત્તમ કેબલ લંબાઈ} 15 મીટર \\ \hline
\textbf{પોઈન્ટ-ટુ-પોઈન્ટ કોમ્યુનિકેશન} (1 ડ્રાઈવર, 1 રિસીવર) \\ \hline
\textbf{વોલ્ટેજ લેવલ્સ}: -15V થી +15V \\ \hline
\textbf{25-પિન અથવા 9-પિન} DB કનેક્ટર સ્ટાન્ડર્ડ \\ \hline
\end{tabulary}
\end{center}
\end{solutionbox}

\begin{mnemonicbox}
\mnemonic{SMPVD - સિંગલ મેક્સિમમ પોઈન્ટ-ટુ-પોઈન્ટ વોલ્ટેજ DB-કનેક્ટર}
\end{mnemonicbox}

\questionmarks{3(બ) OR}{4}{SNR ના સંદર્ભમાં ચેનલ કેપેસીટી શું છે? તેનું મહત્વ સમજાવો.}

\begin{solutionbox}
\textbf{ચેનલ કેપેસીટી:}
કોમ્યુનિકેશન ચેનલ પર માહિતી ટ્રાન્સમિટ કરી શકાય તે મહત્તમ દર, જેમાં ભૂલની સંભાવના નહિવત્ હોય.

\textbf{સૂત્ર:} $C = B \times \log_2(1 + SNR)$
જ્યાં: C = ચેનલ કેપેસીટી (bps), B = બેન્ડવીડ્થ (Hz), SNR = સિગ્નલ-ટુ-નોઇઝ રેશિયો.

\begin{center}
\captionof{table}{મહત્વ}
\begin{tabulary}{\linewidth}{|L|}
\hline
\textbf{ચેનલ કેપેસીટીનું મહત્વ} \\ \hline
\textbf{સૈદ્ધાંતિક મર્યાદાઓ નક્કી કરે છે} \\ \hline
\textbf{સિસ્ટમ ડિઝાઇન} અને ઓપ્ટિમાઇઝેશનમાં મદદ કરે છે \\ \hline
\textbf{પર્ફોર્મન્સ મૂલ્યાંકન} માટે ઉપયોગી \\ \hline
\textbf{જરૂરી બેન્ડવીડ્થ} નક્કી કરે છે \\ \hline
\textbf{કોડિંગ તકનીકો} માટે માર્ગદર્શન આપે છે \\ \hline
\end{tabulary}
\end{center}

\begin{center}
\begin{tikzpicture}[node distance=2cm, auto]
    \node [gtu block] (bw) {બેન્ડવીડ્થ};
    \node [gtu block, below=1cm of bw] (snr) {SNR};
    \node [gtu block, right=2cm of bw, yshift=-0.5cm] (cap) {ચેનલ કેપેસીટી};
    \node [gtu block, right=1.5cm of cap] (rate) {મેક્સ ડેટા રેટ};
    
    \draw [gtu arrow] (bw) -- (cap);
    \draw [gtu arrow] (snr) -- (cap);
    \draw [gtu arrow] (cap) -- (rate);
\end{tikzpicture}
\captionof{figure}{ચેનલ કેપેસીટી પરિબળો}
\end{center}
\end{solutionbox}

\begin{mnemonicbox}
\mnemonic{BSNR - બેન્ડવીડ્થ એન્ડ SNR નીડ રિલેશનશિપ}
\end{mnemonicbox}

\questionmarks{3(ક) OR}{7}{ડિજિટલ કોમ્યુનિકેશનમાં કોઈપણ એક એરર ડિટેક્શન અને એરર કરેક્શન તકનીક વિગતવાર સમજાવો.}

\begin{solutionbox}
\textbf{હેમિંગ કોડ (Hamming Code):}
હેમિંગ કોડ એ લિનિયર એરર-કરેક્ટિંગ કોડ છે જે ડેટામાં રહેલી સિંગલ-બિટ ભૂલને શોધી અને સુધારી શકે છે.

\textbf{ઉદાહરણ: 7-બિટ હેમિંગ કોડ (4 ડેટા, 3 પેરિટી)}
\begin{center}
\captionof{table}{હેમિંગ કોડ સ્ટ્રક્ચર}
\begin{tabulary}{\linewidth}{|C|C|C|C|C|C|C|C|}
\hline
\textbf{સ્થાન} & 1 & 2 & 3 & 4 & 5 & 6 & 7 \\ \hline
\textbf{બિટ પ્રકાર} & P1 & P2 & D1 & P4 & D2 & D3 & D4 \\ \hline
\end{tabulary}
\end{center}

\textbf{ભૂલ સુધારણા:}
પેરિટી ચેક્સ ભૂલનું સ્થાન સૂચવે છે (P4 P2 P1 નું બાઇનરી મૂલ્ય સ્થાન આપે છે).

\begin{center}
\captionof{table}{ભૂલ સ્થાન}
\begin{tabulary}{\linewidth}{|C|C|C|L|}
\hline
\textbf{P4} & \textbf{P2} & \textbf{P1} & \textbf{ભૂલ સ્થાન} \\ \hline
0 & 0 & 0 & કોઈ ભૂલ નથી \\ \hline
0 & 0 & 1 & સ્થાન 1 \\ \hline
1 & 0 & 1 & સ્થાન 5 \\ \hline
1 & 1 & 1 & સ્થાન 7 \\ \hline
\end{tabulary}
\end{center}
\end{solutionbox}

\begin{mnemonicbox}
\mnemonic{PECD - પેરિટી એનેબલ્સ કરેક્શન ઓફ ડેટા}
\end{mnemonicbox}

\questionmarks{4(અ)}{3}{સેટેલાઇટ કોમ્યુનિકેશનનો બ્લોક ડાયાગ્રામ દોરો અને ટૂંકમાં સમજાવો.}

\begin{solutionbox}
\begin{center}
\begin{tikzpicture}[node distance=2.5cm, auto]
    \node [gtu block] (gs1) {ગ્રાઉન્ડ સ્ટેશન 1};
    \node [gtu block, right=4cm of gs1] (gs2) {ગ્રાઉન્ડ સ્ટેશન 2};
    \node [gtu block, above=2cm of $(gs1)!0.5!(gs2)$] (sat) {સેટેલાઇટ};
    
    \draw [gtu arrow, bend left] (gs1) to node {અપલિંક} (sat);
    \draw [gtu arrow, bend left] (sat) to node {ડાઉનલિંક} (gs2);
    
    \node [left=0.5cm of gs1] (tx) {ટ્રાન્સમીટર};
    \node [right=0.5cm of gs2] (rx) {રીસીવર};
    
    \draw [gtu arrow] (tx) -- (gs1);
    \draw [gtu arrow] (gs2) -- (rx);
\end{tikzpicture}
\captionof{figure}{સેટેલાઇટ કોમ્યુનિકેશન}
\end{center}

\textbf{સમજૂતી:}
સેટેલાઇટ કોમ્યુનિકેશનમાં અર્થ સ્ટેશનથી સેટેલાઇટ પર સિગ્નલ મોકલવામાં આવે છે (અપલિંક), જે તેને એમ્પ્લીફાય કરીને પૃથ્વી પર પાછું મોકલે છે (ડાઉનલિંક).
\end{solutionbox}

\begin{mnemonicbox}
\mnemonic{STAR - સેટેલાઇટ ટ્રાન્સમિટ્સ એન્ડ રિસીવ્સ}
\end{mnemonicbox}

\questionmarks{4(બ)}{4}{ડેટા સિક્વન્સ 10101101 માટે યુનિપોલર NRZ, પોલર RZ, પોલર NRZ અને AMI વેવફોર્મ દોરો.}

\begin{solutionbox}
\begin{center}
\begin{tikzpicture}[x=0.8cm,y=0.6cm]
    % Data: 1 0 1 0 1 1 0 1
    \node[anchor=east] at (-1, 1) {Data:};
    \foreach \x/\val in {0/1, 1/0, 2/1, 3/0, 4/1, 5/1, 6/0, 7/1} {
        \draw (\x,0) -- (\x,\val) -- (\x+1,\val) -- (\x+1,0); 
        \node at (\x+0.5, 1.5) {\val};
    }
    
    % Unipolar NRZ
    \node[anchor=east] at (-1, -1.5) {Unipolar NRZ:};
    \draw (0,-1.5) -- (0, -0.5) -- (1,-0.5) -- (1,-1.5) -- (2,-1.5) -- (2,-0.5) -- (3,-0.5) -- (3,-1.5) -- (4,-1.5) -- (4,-0.5) -- (6,-0.5) -- (6,-1.5) -- (7,-1.5) -- (7,-0.5) -- (8,-0.5);
    
    % Polar RZ
    \node[anchor=east] at (-1, -4) {Polar RZ:};
    \foreach \x/\val in {0/1, 1/-1, 2/1, 3/-1, 4/1, 5/1, 6/-1, 7/1} {
        \draw (\x,-4) -- (\x, -4+\val) -- (\x+0.5,-4+\val) -- (\x+0.5,-4) -- (\x+1,-4);
    }
    
    % Polar NRZ
    \node[anchor=east] at (-1, -6.5) {Polar NRZ:};
    \draw (0,-6.5) -- (0, -5.5) -- (1,-5.5) -- (1,-7.5) -- (2,-7.5) -- (2,-5.5) -- (3,-5.5) -- (3,-7.5) -- (4,-7.5) -- (4,-5.5) -- (6,-5.5) -- (6,-7.5) -- (7,-7.5) -- (7,-5.5) -- (8,-5.5);

    % AMI
    \node[anchor=east] at (-1, -9) {AMI:};
    \draw (0,-9) -- (0, -8) -- (1,-8) -- (1,-9) -- (2,-9) -- (2,-10) -- (3,-10) -- (3,-9) -- (4,-9) -- (4,-8) -- (5,-8) -- (5,-9) -- (5,-10) -- (6,-10) -- (6,-9) -- (7,-9) -- (7,-8) -- (8,-8);
    
\end{tikzpicture}
\captionof{figure}{લાઇન કોડિંગ વેવફોર્મ્સ}
\end{center}
\end{solutionbox}

\begin{mnemonicbox}
\mnemonic{UPPA - યુનિપોલર પોલર પોલર AMI}
\end{mnemonicbox}

\questionmarks{4(ક)}{7}{ડિજિટલ કોમ્યુનિકેશન માટે યોગ્ય ઉદાહરણ સાથે ડેટા ટ્રાન્સમિશન તકનીકો વિગતવાર સમજાવો.}

\begin{solutionbox}
\textbf{ડેટા ટ્રાન્સમિશન તકનીકો:}
\begin{center}
\captionof{table}{તકનીકો}
\begin{tabulary}{\linewidth}{|L|L|L|}
\hline
\textbf{તકનીક} & \textbf{વર્ણન} & \textbf{ઉદાહરણ} \\ \hline
\textbf{સીરીયલ} & એક જ ચેનલ પર એક પછી એક બિટ્સ મોકલાય છે & USB, UART \\ \hline
\textbf{પેરેલલ} & અનેક ચેનલો પર એક સાથે અનેક બિટ્સ મોકલાય છે & પ્રિન્ટર, SCSI \\ \hline
\textbf{સિન્ક્રોનસ} & ટાઇમિંગ સિગ્નલો સાથે સતત સ્ટ્રીમ & Ethernet \\ \hline
\textbf{એસિન્ક્રોનસ} & સ્ટાર્ટ/સ્ટોપ બિટ્સનો ઉપયોગ & RS-232 \\ \hline
\end{tabulary}
\end{center}

\textbf{સીરીયલ ટ્રાન્સમિશન (UART ઉદાહરણ):}
\begin{center}
\begin{tikzpicture}[x=0.5cm, y=0.5cm]
    \node[anchor=east] at (-1, 0.5) {Signal:};
    \draw (-2,1) -- (0,1) -- (0,0) -- (1,0) -- (1,1) -- (2,1) -- (2,0) -- (3,0) -- (3,1) -- (4,1) -- (4,0) -- (5,0) -- (5,1) -- (6,1) -- (7,1) -- (8,1) -- (8,0) -- (9,0) -- (9,1) -- (10,1) -- (11,1);
    \node at (0.5, 2) {Start};
    \node at (9.5, 2) {Stop};
    \foreach \x/\v in {1/1, 2/0, 3/1, 4/0, 5/1, 6/1, 7/0, 8/1} {
        \node at (\x+0.5, 1.5) {\v};
    }
\end{tikzpicture}
\captionof{figure}{સીરીયલ ટ્રાન્સમિશન}
\end{center}

\textbf{પેરેલલ ટ્રાન્સમિશન:}
\begin{center}
\begin{tikzpicture}
    \foreach \y in {0,1,2,3,4,5,6,7} {
        \draw[->] (0,\y*0.3) -- (4,\y*0.3);
        \node[left] at (0,\y*0.3) {Bit \y};
    }
    \node at (2, -0.5) {8 bits simultaneously};
\end{tikzpicture}
\captionof{figure}{પેરેલલ ટ્રાન્સમિશન}
\end{center}
\end{solutionbox}

\begin{mnemonicbox}
\mnemonic{SPASH - સીરીયલ પેરેલલ એસિન્ક્રોનસ સિન્ક્રોનસ હાફ-ડુપ્લેક્સ}
\end{mnemonicbox}

\questionmarks{4(અ) OR}{3}{સ્પ્રેડ સ્પેક્ટ્રમ ટેક્નિક્સના પાસાઓનું અર્થઘટન કરો.}

\begin{solutionbox}
\begin{center}
\captionof{table}{સ્પ્રેડ સ્પેક્ટ્રમ પાસાઓ}
\begin{tabulary}{\linewidth}{|L|L|}
\hline
\textbf{પાસું} & \textbf{અર્થઘટન} \\ \hline
\textbf{બેન્ડવીડ્થ સ્પ્રેડિંગ} & સિગ્નલ બહોળી બેન્ડવીડ્થ પર ફેલાય છે \\ \hline
\textbf{સુરક્ષા} & ઇન્ટરસેપ્ટ/જામ કરવું મુશ્કેલ \\ \hline
\textbf{નોઇઝ ઇમ્યુનિટી} & નેરોબેન્ડ ઇન્ટરફેરન્સ સામે પ્રતિરોધક \\ \hline
\textbf{મલ્ટિપલ એક્સેસ} & ફ્રિક્વન્સી શેર કરવાની મંજૂરી આપે છે \\ \hline
\textbf{લો પાવર ડેન્સિટી} & સિગ્નલ નોઇઝ જેવું લાગે છે \\ \hline
\end{tabulary}
\end{center}

\begin{center}
\begin{tikzpicture}[node distance=2cm, auto]
    \node [gtu block] (nb) {નેરોલ બેન્ડ};
    \node [gtu block, right=2cm of nb] (spread) {સ્પ્રેડિંગ};
    \node [gtu block, right=2cm of spread] (wb) {વાઇડબેન્ડ};
    \node [gtu block, below=1cm of spread] (code) {કોડ};
    
    \draw [gtu arrow] (nb) -- (spread);
    \draw [gtu arrow] (spread) -- (wb);
    \draw [gtu arrow] (code) -- (spread);
\end{tikzpicture}
\captionof{figure}{સ્પ્રેડ સ્પેક્ટ્રમ કોન્સેપ્ટ}
\end{center}
\end{solutionbox}

\begin{mnemonicbox}
\mnemonic{BSNML - બેન્ડવીડ્થ સિક્યોરિટી નોઇઝ મલ્ટિપલ લો-પાવર}
\end{mnemonicbox}

\questionmarks{4(બ) OR}{4}{પ્રોબેબિલિટી પર ટૂંક નોંધ લખો અને ડિજિટલ કોમ્યુનિકેશન માટે તેના ગુણધર્મોની ચર્ચા કરો.}

\begin{solutionbox}
\textbf{પ્રોબેબિલિટી:} એરર રેટ અને વિશ્વસનીયતાના વિશ્લેષણ માટેનો પાયો.

\begin{center}
\captionof{table}{ગુણધર્મો}
\begin{tabulary}{\linewidth}{|L|L|L|}
\hline
\textbf{ગુણધર્મ} & \textbf{વર્ણન} & \textbf{સંગતતા} \\ \hline
\textbf{રેન્જ} & $0 \le P(E) \le 1$ & એરર સંભાવનાની સીમા \\ \hline
\textbf{ચોકસાઈ} & $P(S) = 1$ & કુલ સંભાવના \\ \hline
\textbf{સરવાળો} & $P(A \cup B) = P(A) + P(B)$ & કુલ એરર રેટ \\ \hline
\textbf{શરતી સંભાવના} & $P(A|B)$ & ચેનલ મોડેલિંગ \\ \hline
\textbf{સ્વતંત્રતા} & $P(A \cap B) = P(A)P(B)$ & અસંબંધિત નોઇઝ \\ \hline
\end{tabulary}
\end{center}
\end{solutionbox}

\begin{mnemonicbox}
\mnemonic{RACIC - રેન્જ એડિટિવિટી સર્ટેઇન્ટી ઇન્ડિપેન્ડન્સ કન્ડીશનલ}
\end{mnemonicbox}

\questionmarks{4(ક) OR}{7}{ઉદાહરણ સાથે ડેટા ટ્રાન્સમિશન મોડ વિગતવાર સમજાવો.}

\begin{solutionbox}
\textbf{ડેટા ટ્રાન્સમિશન મોડ્સ:}

\begin{center}
\captionof{table}{મોડ્સ}
\begin{tabulary}{\linewidth}{|L|L|L|}
\hline
\textbf{મોડ} & \textbf{વર્ણન} & \textbf{ઉદાહરણ} \\ \hline
\textbf{સિમ્પ્લેક્સ} & માત્ર એકમાર્ગીય & ટીવી, રેડિયો \\ \hline
\textbf{હાફ-ડુપ્લેક્સ} & દ્વિમાર્ગીય, પણ એક સમયે એક જ & વોકી-ટોકી \\ \hline
\textbf{ફુલ-ડુપ્લેક્સ} & દ્વિમાર્ગીય, એક કાળે (simultaneous) & ટેલિફોન \\ \hline
\end{tabulary}
\end{center}

\begin{center}
\begin{tikzpicture}[node distance=2cm, auto]
    % Simplex
    \node [gtu block] (tx1) {Tx};
    \node [gtu block, right=1.5cm of tx1] (rx1) {Rx};
    \draw [gtu arrow] (tx1) -- node {સિમ્પ્લેક્સ} (rx1);
    
    % Half Duplex
    \node [gtu block, below=1.5cm of tx1] (devA) {Dev A};
    \node [gtu block, right=1.5cm of devA] (devB) {Dev B};
    \draw [gtu arrow, bend left] (devA) to node {Time 1} (devB);
    \draw [gtu arrow, bend left] (devB) to node {Time 2} (devA);
    \node at ($(devA)!0.5!(devB) + (0,-1)$) {હાફ-ડુપ્લેક્સ};
    
    % Full Duplex
    \node [gtu block, below=3cm of tx1] (devA2) {Dev A};
    \node [gtu block, right=1.5cm of devA2] (devB2) {Dev B};
    \draw [gtu arrow, transform canvas={yshift=0.1cm}] (devA2) -- node {Ch 1} (devB2);
    \draw [gtu arrow, transform canvas={yshift=-0.1cm}] (devB2) -- node {Ch 2} (devA2);
    \node at ($(devA2)!0.5!(devB2) + (0,-0.8)$) {ફુલ-ડુપ્લેક્સ};
\end{tikzpicture}
\captionof{figure}{ટ્રાન્સમિશન મોડ્સ}
\end{center}
\end{solutionbox}

\begin{mnemonicbox}
\mnemonic{SHF - સિમ્પ્લેક્સ હાફ ફુલ}
\end{mnemonicbox}

\questionmarks{5(અ)}{3}{એજ કમ્પ્યુટિંગ વિગતવાર સમજાવો.}

\begin{solutionbox}
\textbf{એજ કમ્પ્યુટિંગ:} ડિસ્ટ્રિબ્યુટેડ કમ્પ્યુટિંગ જે કમ્પ્યુટેશનને ડેટા સ્રોચની નજીક લાવે છે.

\begin{center}
\captionof{table}{મુખ્ય પાસાઓ}
\begin{tabulary}{\linewidth}{|L|L|}
\hline
\textbf{પાસું} & \textbf{વર્ણન} \\ \hline
\textbf{વિકેન્દ્રીકરણ} & નેટવર્ક કિનારી પર પ્રોસેસિંગ \\ \hline
\textbf{લેટન્સી ઘટાડો} & ઝડપી પ્રતિભાવ \\ \hline
\textbf{બેન્ડવીડ્થ કાર્યક્ષમતા} & ક્લાઉડ પર ઓછો ડેટા મોકલાય છે \\ \hline
\textbf{સુરક્ષા} & સંવેદનશીલ ડેટા સ્થાનિક રહે છે \\ \hline
\end{tabulary}
\end{center}

\begin{center}
\begin{tikzpicture}[node distance=2cm, auto]
    \node [gtu block] (iot) {IoT ડિવાઇસ};
    \node [gtu block, right=2cm of iot] (edge) {એજ કમ્પ્યુટિંગ};
    \node [gtu block, right=2cm of edge, yshift=1cm] (local) {લોકલ પ્રોસેસિંગ};
    \node [gtu block, right=2cm of edge, yshift=-1cm] (cloud) {ક્લાઉડ};
    
    \draw [gtu arrow] (iot) -- (edge);
    \draw [gtu arrow] (edge) -- (local);
    \draw [gtu arrow] (edge) -- (cloud);
\end{tikzpicture}
\captionof{figure}{એજ કમ્પ્યુટિંગ આર્કિટેક્ચર}
\end{center}
\end{solutionbox}

\begin{mnemonicbox}
\mnemonic{DRBLES - ડિસેન્ટ્રલાઇઝ્ડ રિડ્યુસિસ બેન્ડવીડ્થ, લેટન્સી, એક્સપોઝર}
\end{mnemonicbox}

\questionmarks{5(બ)}{4}{ડેટા કોમ્યુનિકેશનમાં 5G ટેકનોલોજીની વિશેષતાઓ જણાવો.}

\begin{solutionbox}
\begin{center}
\captionof{table}{5G ની વિશેષતાઓ}
\begin{tabulary}{\linewidth}{|L|}
\hline
\textbf{5G ટેકનોલોજીની વિશેષતાઓ} \\ \hline
\textbf{ઉચ્ચ ડેટા રેટ} (20 Gbps સુધી) \\ \hline
\textbf{અલ્ટ્રા-લો લેટન્સી} (1 ms કે ઓછું) \\ \hline
\textbf{મેસિવ ડિવાઇસ કનેક્ટિવિટી} (1M ડિવાઇસ પ્રતિ km$^2$) \\ \hline
\textbf{નેટવર્ક સ્લાઇસિંગ} (વર્ચ્યુઅલ નેટવર્ક્સ) \\ \hline
\textbf{બીમફોર્મિંગ} (દિશાકીય સિગ્નલ) \\ \hline
\textbf{મિલિમીટર વેવ સ્પેક્ટ્રમ} (24-100 GHz) \\ \hline
\end{tabulary}
\end{center}

\begin{center}
\begin{tikzpicture}[node distance=2cm, auto]
    \node [gtu block] (5g) {5G ટેકનોલોજી};
    \node [gtu block, right=2cm of 5g, yshift=1.5cm] (embb) {eMBB};
    \node [gtu block, right=2cm of 5g] (urllc) {URLLC};
    \node [gtu block, right=2cm of 5g, yshift=-1.5cm] (mmtc) {mMTC};
    
    \draw [gtu arrow] (5g) -- (embb);
    \draw [gtu arrow] (5g) -- (urllc);
    \draw [gtu arrow] (5g) -- (mmtc);
\end{tikzpicture}
\captionof{figure}{5G ના ઉપયોગો}
\end{center}
\end{solutionbox}

\begin{mnemonicbox}
\mnemonic{HUMBLE-MN - હાઇ-સ્પીડ અલ્ટ્રા-લો-લેટન્સી મેસિવ બીમફોર્મિંગ}
\end{mnemonicbox}

\questionmarks{5(ક)}{7}{ડેટા કોમ્યુનિકેશન પર વિગતવાર નોંધ લખો, તેની લાક્ષણિકતાઓ અને ઘટકો સહિત.}

\begin{solutionbox}
\textbf{ડેટા કોમ્યુનિકેશન:} ડિજિટલ માહિતીના સ્થાનાંતરણની પ્રક્રિયા.

\textbf{લાક્ષણિકતાઓ:}
\begin{itemize}
    \item \keyword{ડિલિવરી}: યોગ્ય ગંતવ્ય.
    \item \keyword{ચોકસાઈ}: કોઈ ભૂલ નહીં.
    \item \keyword{સમયસરતા}: સમયસર વિતરણ.
    \item \keyword{જીટર}: સમયમાં સાતત્ય.
    \item \keyword{સુરક્ષા}: સુરક્ષિત એક્સેસ.
\end{itemize}

\textbf{ઘટકો:}
\begin{center}
\captionof{table}{ઘટકો}
\begin{tabulary}{\linewidth}{|L|L|}
\hline
\textbf{ઘટક} & \textbf{વર્ણન} \\ \hline
\textbf{મેસેજ} & માહિતી જે મોકલવાની છે \\ \hline
\textbf{સેન્ડર} & ડેટા મોકલતું સાધન \\ \hline
\textbf{રિસીવર} & ડેટા પ્રાપ્ત કરતું સાધન \\ \hline
\textbf{માધ્યમ} & ભૌતિક માર્ગ \\ \hline
\textbf{પ્રોટોકોલ} & નિયમો \\ \hline
\end{tabulary}
\end{center}

\begin{center}
\begin{tikzpicture}[node distance=1.5cm, auto]
    \node [gtu block] (sender) {સેન્ડર};
    \node [gtu block, right=1cm of sender] (enc) {એન્કોડર};
    \node [gtu block, right=1cm of enc] (med) {માધ્યમ};
    \node [gtu block, right=1cm of med] (dec) {ડિકોડર};
    \node [gtu block, right=1cm of dec] (rx) {રિસીવર};
    
    \node [gtu block, above=1cm of med] (proto) {પ્રોટોકોલ};
    
    \draw [gtu arrow] (sender) -- (enc);
    \draw [gtu arrow] (enc) -- (med);
    \draw [gtu arrow] (med) -- (dec);
    \draw [gtu arrow] (dec) -- (rx);
    
    \draw [dashed] (proto) -- (sender);
    \draw [dashed] (proto) -- (enc);
    \draw [dashed] (proto) -- (med);
    \draw [dashed] (proto) -- (dec);
    \draw [dashed] (proto) -- (rx);
\end{tikzpicture}
\captionof{figure}{ડેટા કોમ્યુનિકેશન મોડેલ}
\end{center}
\end{solutionbox}

\begin{mnemonicbox}
\mnemonic{DATJS-MSRTP - ડિલિવરી એક્યુરસી ટાઇમલીનેસ જીટર સિક્યોરિટી}
\end{mnemonicbox}

\questionmarks{5(અ) OR}{3}{ડેટા કોમ્યુનિકેશનમાં પ્રાઇવસીની વિચારણાઓને ઓળખો અને લખો.}

\begin{solutionbox}
\begin{itemize}
    \item \keyword{ડેટા એન્ક્રિપ્શન}: ડેટા સુરક્ષા.
    \item \keyword{એક્સેસ કંટ્રોલ}: માત્ર અધિકૃત વપરાશકર્તાઓ.
    \item \keyword{ઓથેન્ટિકેશન}: ઓળખ ચકાસણી.
    \item \keyword{ડેટા મિનિમાઇઝેશન}: માત્ર જરૂરી ડેટા એકત્ર કરવો.
    \item \keyword{એન્ડ-ટુ-એન્ડ સિક્યોરિટી}: સંપૂર્ણ માર્ગ સુરક્ષા.
\end{itemize}

\begin{center}
\begin{tikzpicture}[node distance=2cm, auto]
    \node [gtu block] (priv) {પ્રાઇવસી};
    \node [gtu block, right=2cm of priv, yshift=1cm] (enc) {એન્ક્રિપ્શન};
    \node [gtu block, right=2cm of priv] (auth) {ઓથેન્ટિકેશન};
    \node [gtu block, right=2cm of priv, yshift=-1cm] (access) {એક્સેસ કંટ્રોલ};
    
    \draw [gtu arrow] (priv) -- (enc);
    \draw [gtu arrow] (priv) -- (auth);
    \draw [gtu arrow] (priv) -- (access);
\end{tikzpicture}
\captionof{figure}{પ્રાઇવસી વિચારણાઓ}
\end{center}
\end{solutionbox}

\begin{mnemonicbox}
\mnemonic{DAAESE - ડેટા ઓથેન્ટિકેટેડ, એક્સેસ્ડ, એન્ક્રિપ્ટેડ સિક્યોરલી}
\end{mnemonicbox}

\questionmarks{5(બ) OR}{4}{કોમ્યુનિકેશન સિક્યોરિટીમાં બ્લોક ચેઇન શું છે? તેની વિશેષતાઓ જણાવો.}

\begin{solutionbox}
\textbf{બ્લોકચેન:} ડિસ્ટ્રિબ્યુટેડ લેજર ટેકનોલોજી જે સુરક્ષિત, ટેમ્પર-પ્રૂફ રેકોર્ડ્સ પ્રદાન કરે છે.

\begin{center}
\captionof{table}{વિશેષતાઓ}
\begin{tabulary}{\linewidth}{|L|L|}
\hline
\textbf{વિશેષતા} & \textbf{વર્ણન} \\ \hline
\textbf{વિકેન્દ્રીકરણ} & કોઈ કેન્દ્રીય સત્તા નથી \\ \hline
\textbf{અપરિવર્તનક્ષમતા} & બદલી શકાતું નથી \\ \hline
\textbf{પારદર્શિતા} & બધા માટે દૃશ્યમાન \\ \hline
\textbf{ક્રિપ્ટોગ્રાફિક સુરક્ષા} & ક્રિપ્ટો વડે સુરક્ષિત \\ \hline
\textbf{સહમતિ} & નેટવર્કની મંજૂરી \\ \hline
\end{tabulary}
\end{center}

\begin{center}
\begin{tikzpicture}[node distance=1.5cm, auto]
    \node [gtu block] (b1) {બ્લોક 1};
    \node [gtu block, right=1cm of b1] (b2) {બ્લોક 2};
    \node [gtu block, right=1cm of b2] (b3) {બ્લોક 3};
    
    \draw [gtu arrow] (b1) -- node {હેશ} (b2);
    \draw [gtu arrow] (b2) -- node {હેશ} (b3);
\end{tikzpicture}
\captionof{figure}{બ્લોકચેન સ્ટ્રક્ચર}
\end{center}
\end{solutionbox}

\begin{mnemonicbox}
\mnemonic{DITCSD - ડિસેન્ટ્રલાઇઝ્ડ ઇમ્યુટેબલ ટ્રાન્સપરન્ટ ક્રિપ્ટોગ્રાફિક સિક્યોર}
\end{mnemonicbox}

\questionmarks{5(ક) OR}{7}{વિવિધ કોમ્યુનિકેશન પોર્ટ્સ: USB, HDMI, RCA અને Ethernet લખો અને ચિત્રિત કરો.}

\begin{solutionbox}
\textbf{કોમ્યુનિકેશન પોર્ટ્સ:}

\begin{enumerate}
    \item \textbf{USB (યુનિવર્સલ સીરીયલ બસ)}: ડેટા/પાવર, 40 Gbps, હોટ-સ્વેપેબલ.
    \item \textbf{HDMI}: ઓડિયો/વિડિયો, 48 Gbps, HDCP.
    \item \textbf{RCA}: એનાલોગ ઓડિયો/વિડિયો, રંગીન (લાલ/સફેદ/પીળો).
    \item \textbf{Ethernet (RJ-45)}: નેટવર્ક, 10 Gbps, ટ્વિસ્ટેડ પેર.
\end{enumerate}

\begin{center}
\begin{tikzpicture}
    % USB
    \node at (0, 2) {\textbf{USB}};
    \draw (-1,0) rectangle (1,1);
    \draw (-0.8, 0.2) rectangle (-0.6, 0.8);
    \draw (-0.4, 0.2) rectangle (-0.2, 0.8);
    
    % HDMI
    \node at (4, 2) {\textbf{HDMI}};
    \draw (3,0) -- (5,0) -- (5,0.7) -- (4.8, 1) -- (3.2, 1) -- (3, 0.7) -- cycle;
    \draw (3.2, 0.2) rectangle (4.8, 0.5);
    
    % RJ45
    \node at (0, -1) {\textbf{Ethernet}};
    \draw (-1,-3) rectangle (1,-2);
    \foreach \x in {-0.8, -0.6, -0.4, -0.2, 0, 0.2, 0.4, 0.6}
        \draw (\x, -2.8) -- (\x, -2.2);
        
    % RCA
    \node at (4, -1) {\textbf{RCA}};
    \draw (4,-2.5) circle (0.5);
    \draw (4,-2.5) circle (0.2);
    \fill (4,-2.5) circle (0.1);
\end{tikzpicture}
\captionof{figure}{પોર્ટ ચિત્રણ}
\end{center}

\begin{center}
\captionof{table}{સરખામણી}
\begin{tabulary}{\linewidth}{|L|L|L|L|}
\hline
\textbf{પોર્ટ} & \textbf{પ્રકાર} & \textbf{મહત્તમ સ્પીડ} & \textbf{ઉપયોગ} \\ \hline
\textbf{USB} & ડિજિટલ & 40 Gbps & ડેટા/પાવર \\ \hline
\textbf{HDMI} & ડિજિટલ & 48 Gbps & ઓડિયો/વિડિયો \\ \hline
\textbf{RCA} & એનાલોગ & ઓછી & ઓડિયો/વિડિયો \\ \hline
\textbf{Ethernet} & ડિજિટલ & 10 Gbps & નેટવર્ક \\ \hline
\end{tabulary}
\end{center}
\end{solutionbox}

\begin{mnemonicbox}
\mnemonic{UHRE - USB હેન્ડલ્સ રેપિડ ઈથરનેટ, HDMI ડિલિવર્સ રિચ એન્ટરટેઈનમેન્ટ}
\end{mnemonicbox}

\end{document}

