\documentclass[10pt,a4paper]{article}

% content/resources/templates/preamble.tex
\usepackage[margin=0.6in]{geometry}
\author{Milav Dabgar}
\usepackage{amsmath,amssymb,amsthm}
\usepackage{booktabs}
\usepackage{multirow}
\usepackage{xcolor}
\usepackage{tcolorbox}
\tcbuselibrary{breakable,skins}
\usepackage[colorlinks=true,linkcolor=blue]{hyperref}
\usepackage{titlesec}
\usepackage{enumitem}
\usepackage{tikz}
\usepackage{pgfplots}
\usepackage{circuitikz}
\usepackage[version=4]{mhchem}
\usepackage{longtable}
\usepackage{array}
\usepackage{float}
\usepackage{caption}
\usepackage{listings}

\lstset{
  basicstyle=\small\ttfamily,
  breaklines=true,
  breakatwhitespace=false,
  postbreak=\mbox{\textcolor{red}{$\hookrightarrow$}\space},
  float=false,
  numbers=left,
  numberstyle=\tiny\color{gray},
  numbersep=10pt,
  xleftmargin=2em,
  keywordstyle=\color{blue},
  commentstyle=\color{green!60!black},
  stringstyle=\color{purple},
  backgroundcolor=\color{gray!5},
  showstringspaces=false,
  tabsize=2,
  captionpos=b,
  keepspaces=true,
  columns=flexible
}

\pgfplotsset{compat=1.18}
\usetikzlibrary{shapes,arrows,positioning,calc,patterns,decorations.pathmorphing,decorations.markings,arrows.meta}

% Color scheme
\definecolor{headcolor}{RGB}{0,102,204}
\definecolor{keycolor}{RGB}{220,20,60}
\definecolor{solutioncolor}{RGB}{34,139,34}
\definecolor{mnemoniccolor}{RGB}{148,0,211}
\definecolor{codecolor}{RGB}{0,0,100}

% Spacing
\setlength{\parskip}{3pt}
\setlist[itemize]{nosep}
\setlist[enumerate]{nosep}

% Title formatting
\titleformat{\section}{\Large\bfseries\color{headcolor}}{\thesection}{1em}{}
\titleformat{\subsection}{\large\bfseries\color{headcolor}}{\thesubsection}{1em}{}

% Pandoc tightlist compatibility
\providecommand{\tightlist}{%
  \setlength{\itemsep}{0pt}\setlength{\parskip}{0pt}}

% Pandoc longtable compatibility
\newcounter{none}
\def\thenone{}


% content/resources/templates/english-boxes.tex
% This file is currently empty - it exists to maintain consistency with the import structure.
% Add custom environments here if needed in the future.


\begin{document}

\begin{center}
{\Huge\bfseries\color{headcolor} Subject Name Solutions}\\[5pt]
{\LARGE 4343201 -- Summer 2024}\\[3pt]
{\large Semester 1 Study Material}\\[3pt]
{\normalsize\textit{Detailed Solutions and Explanations}}
\end{center}

\vspace{10pt}

\subsection*{Question 1(a) [3 marks]}\label{q1a}

\textbf{Define: (1) Bit rate, (2) Baud rate, and (3) Bandwidth}

\begin{solutionbox}

{\def\LTcaptype{none} % do not increment counter
\begin{longtable}[]{@{}
  >{\raggedright\arraybackslash}p{(\linewidth - 2\tabcolsep) * \real{0.3333}}
  >{\raggedright\arraybackslash}p{(\linewidth - 2\tabcolsep) * \real{0.6667}}@{}}
\toprule\noalign{}
\begin{minipage}[b]{\linewidth}\raggedright
Term
\end{minipage} & \begin{minipage}[b]{\linewidth}\raggedright
Definition
\end{minipage} \\
\midrule\noalign{}
\endhead
\bottomrule\noalign{}
\endlastfoot
\textbf{Bit Rate} & Number of bits transmitted per second (bps) \\
\textbf{Baud Rate} & Number of signal elements or symbols transmitted
per second \\
\textbf{Bandwidth} & Range of frequencies required to transmit a signal,
measured in Hertz (Hz) \\
\end{longtable}
}

\end{solutionbox}
\begin{mnemonicbox}
``BBB - Bits move By Bands''

\end{mnemonicbox}
\subsection*{Question 1(b) [4 marks]}\label{q1b}

\textbf{A signal has a bit rate of 8000bps and baud rate of 1000 baud.
How many data element is carry by each signal? How many signals element
do we need?}

\begin{solutionbox}


{\def\LTcaptype{none} % do not increment counter
\vspace{-5pt}
\captionof{table}{Signal Calculation}
\vspace{-10pt}
\begin{longtable}[]{@{}
  >{\raggedright\arraybackslash}p{(\linewidth - 4\tabcolsep) * \real{0.3548}}
  >{\raggedright\arraybackslash}p{(\linewidth - 4\tabcolsep) * \real{0.2258}}
  >{\raggedright\arraybackslash}p{(\linewidth - 4\tabcolsep) * \real{0.4194}}@{}}
\toprule\noalign{}
\begin{minipage}[b]{\linewidth}\raggedright
Parameter
\end{minipage} & \begin{minipage}[b]{\linewidth}\raggedright
Value
\end{minipage} & \begin{minipage}[b]{\linewidth}\raggedright
Calculation
\end{minipage} \\
\midrule\noalign{}
\endhead
\bottomrule\noalign{}
\endlastfoot
Bit rate & 8000 bps & Given \\
Baud rate & 1000 baud & Given \\
Data elements per signal & 8 bits & Bit rate \div Baud rate = 8000 \div 1000 =
8 \\
Signal elements needed & 2\^{}8 = 256 & 2\^{}(bits per signal) \\
\end{longtable}
}

\textbf{Diagram: Signal Element Representation}

\includegraphics[width=1\linewidth,height=\textheight,keepaspectratio]{mermaid-805a4aa6.pdf}

\end{solutionbox}
\begin{mnemonicbox}
``Divide to Decide'' - Divide bit rate by baud rate
to decide how many bits per signal.

\end{mnemonicbox}
\subsection*{Question 1(c) [7 marks]}\label{q1c}

\textbf{Describe Elements of digital communication system with its block
diagram}

\begin{solutionbox}

\textbf{Diagram: Digital Communication System}

\includegraphics[width=1\linewidth,height=\textheight,keepaspectratio]{mermaid-b0000a6e.pdf}

\textbf{Key Elements:}

{\def\LTcaptype{none} % do not increment counter
\begin{longtable}[]{@{}
  >{\raggedright\arraybackslash}p{(\linewidth - 2\tabcolsep) * \real{0.4737}}
  >{\raggedright\arraybackslash}p{(\linewidth - 2\tabcolsep) * \real{0.5263}}@{}}
\toprule\noalign{}
\begin{minipage}[b]{\linewidth}\raggedright
Element
\end{minipage} & \begin{minipage}[b]{\linewidth}\raggedright
Function
\end{minipage} \\
\midrule\noalign{}
\endhead
\bottomrule\noalign{}
\endlastfoot
\textbf{Source} & Generates message to be transmitted \\
\textbf{Source Encoder} & Converts message to digital format, removes
redundancy \\
\textbf{Channel Encoder} & Adds redundancy for error
detection/correction \\
\textbf{Digital Modulator} & Converts digital data to signals suitable
for channel \\
\textbf{Channel} & Physical medium that carries the signal \\
\textbf{Digital Demodulator} & Extracts digital information from
received signals \\
\textbf{Channel Decoder} & Detects/corrects errors using added
redundancy \\
\textbf{Source Decoder} & Reconstructs original message from digital
data \\
\textbf{Destination} & Receives the final message \\
\end{longtable}
}

\end{solutionbox}
\begin{mnemonicbox}
``Send Messages Carefully; Destination Must
Comprehend Signals Deeply''

\end{mnemonicbox}
\subsection*{Question 1(c OR) [7
marks]}\label{question-1c-or-7-marks}

\textbf{What is fundamental limitation of digital communication system?
What are the advantages and disadvantages of digital communication
system?}

\begin{solutionbox}

\textbf{Fundamental Limitations:}

{\def\LTcaptype{none} % do not increment counter
\begin{longtable}[]{@{}ll@{}}
\toprule\noalign{}
Limitation & Description \\
\midrule\noalign{}
\endhead
\bottomrule\noalign{}
\endlastfoot
\textbf{Bandwidth} & Digital signals require more bandwidth than
analog \\
\textbf{Noise} & Limits maximum achievable data rate \\
\textbf{Equipment} & Digital systems need complex hardware and
processing \\
\end{longtable}
}

\textbf{Advantages vs Disadvantages:}

{\def\LTcaptype{none} % do not increment counter
\begin{longtable}[]{@{}ll@{}}
\toprule\noalign{}
Advantages & Disadvantages \\
\midrule\noalign{}
\endhead
\bottomrule\noalign{}
\endlastfoot
\textbf{Noise Immunity} & Higher bandwidth requirements \\
\textbf{Easy Multiplexing} & Complex equipment \\
\textbf{Error Detection \& Correction} & Quantization errors \\
\textbf{Enhanced Security} & Synchronization problems \\
\textbf{Signal Regeneration} & Higher initial cost \\
\textbf{Integration with Computers} & Sampling rate limitations \\
\end{longtable}
}

\end{solutionbox}
\begin{mnemonicbox}
``NEEDS'' - Noise, Equipment, and Environment
Determine Success

\end{mnemonicbox}
\subsection*{Question 2(a) [3 marks]}\label{q2a}

\textbf{Describe QPSK Modulator with block diagram}

\begin{solutionbox}

\textbf{Diagram: QPSK Modulator}

\begin{lstlisting}
              +------------+
              |  2-bit     |          +-----------+
input         | Serial-to- |--bit 1-->| Cos       |
data -------->| Parallel   |          | Carrier   |----+
              | Converter  |          +-----------+    |   QPSK
              |            |                           +-->Signal
              |            |          +-----------+    |   Output
              |            |--bit 2-->| Sin       |----+
              +------------+          | Carrier   |
                                      +-----------+
\end{lstlisting}

\textbf{Key Components:}

\begin{itemize}
\tightlist
\item
  \textbf{Serial-to-Parallel Converter}: Splits data into 2-bit groups
\item
  \textbf{Cosine Carrier}: Modulates first bit (I-channel)
\item
  \textbf{Sine Carrier}: Modulates second bit (Q-channel)
\end{itemize}

\end{solutionbox}
\begin{mnemonicbox}
``Split Pair, Carrier Square'' - data split into
pairs, carried by squared signals

\end{mnemonicbox}
\subsection*{Question 2(b) [4 marks]}\label{q2b}

\textbf{Describe ASK Modulator with block diagram}

\begin{solutionbox}

\textbf{Diagram: ASK Modulator}

\begin{lstlisting}
              +------------+     +---------+
Digital       | Product    |     |         |
Input ------->| Modulator  |---->| Filter  |---> ASK Signal
              | (Mixer)    |     |         |
              +------------+     +---------+
                     ^
                     |
              +------------+
              | Carrier    |
              | Oscillator |
              +------------+
\end{lstlisting}

\textbf{ASK Modulation Process:}

{\def\LTcaptype{none} % do not increment counter
\begin{longtable}[]{@{}ll@{}}
\toprule\noalign{}
Component & Function \\
\midrule\noalign{}
\endhead
\bottomrule\noalign{}
\endlastfoot
\textbf{Digital Input} & Binary data (0s and 1s) to be transmitted \\
\textbf{Carrier Oscillator} & Generates high-frequency sine wave \\
\textbf{Product Modulator} & Multiplies input with carrier (ON/OFF) \\
\textbf{Filter} & Removes unwanted frequency components \\
\end{longtable}
}

\end{solutionbox}
\begin{mnemonicbox}
``Amplify Signal when Keen'' - carrier amplitude
changes when signal is high

\end{mnemonicbox}
\subsection*{Question 2(c) [7 marks]}\label{q2c}

\textbf{Compare ASK, FSK and PSK and Draw the wave form of ASK, FSK and
PSK for the input digital signal 100101000101}

\begin{solutionbox}

\textbf{Comparison Table:}

{\def\LTcaptype{none} % do not increment counter
\begin{longtable}[]{@{}llll@{}}
\toprule\noalign{}
Parameter & ASK & FSK & PSK \\
\midrule\noalign{}
\endhead
\bottomrule\noalign{}
\endlastfoot
\textbf{Modulation Parameter} & Amplitude & Frequency & Phase \\
\textbf{Noise Immunity} & Poor & Moderate & Good \\
\textbf{Bandwidth} & Narrow & Wide & Moderate \\
\textbf{Power Efficiency} & Poor & Moderate & Good \\
\textbf{Implementation} & Simple & Moderate & Complex \\
\textbf{BER Performance} & Poor & Moderate & Good \\
\end{longtable}
}

\textbf{Waveforms for input 100101000101:}

\begin{lstlisting}
Digital: ‾‾‾_‾_‾‾___‾_‾  (1 0 0 1 0 1 0 0 0 1 0 1)
         
ASK:     ✓✓✓___✓___✓✓✓___✓___✓✓✓
         high low low high low high low low low high low high

FSK:     ✓✓✓~~~✓✓✓~~~✓✓✓~~~✓✓✓~~~✓✓✓
         f1  f2  f2  f1  f2  f1  f2  f2  f2  f1  f2  f1

PSK:     ✓✓✓˜˜˜✓✓✓˜˜˜✓✓✓˜˜˜✓✓✓˜˜˜✓✓✓
         0^\circ  180^\circ 180^\circ 0^\circ  180^\circ 0^\circ  180^\circ 180^\circ 180^\circ 0^\circ  180^\circ 0^\circ
\end{lstlisting}

\end{solutionbox}
\begin{mnemonicbox}
``AFP - Alter Frequencies or Phases'' to remember
modulation types

\end{mnemonicbox}
\subsection*{Question 2(a OR) [3
marks]}\label{question-2a-or-3-marks}

\textbf{Describe QPSK Demodulator with block diagram}

\begin{solutionbox}

\textbf{Diagram: QPSK Demodulator}

\begin{lstlisting}
                +-----------+
                | Cos       |
                | Carrier   |--+
                +-----------+  |
                               v
QPSK      +-----+      +------------+     +---------+
Signal--->| BPF |----->| Product    |---->| LPF     |----> Bit 1
          +-----+      | Detect     |     +---------+
                       +------------+
                             ^
                             |      +------------+    +---------+
                             +----->| Product    |--->| LPF     |----> Bit 2
                                    | Detect     |    +---------+
                             +      +------------+
                             |
                        +-----------+
                        | Sin       |
                        | Carrier   |
                        +-----------+
\end{lstlisting}

\textbf{Key Components:}

\begin{itemize}
\tightlist
\item
  \textbf{BPF (Bandpass Filter)}: Removes noise outside signal bandwidth
\item
  \textbf{Product Detectors}: Multiply with carrier signals (cos \& sin)
\item
  \textbf{LPF (Lowpass Filters)}: Extract original data bits
\end{itemize}

\end{solutionbox}
\begin{mnemonicbox}
``Filtered Pairs Deliver Data'' - filters and paired
carriers recover data

\end{mnemonicbox}
\subsection*{Question 2(b OR) [4
marks]}\label{question-2b-or-4-marks}

\textbf{Draw the Constellation diagram of ASK, BPSK and QPSK}

\begin{solutionbox}

\textbf{Constellation Diagrams:}

\begin{lstlisting}
ASK Constellation:      BPSK Constellation:     QPSK Constellation:
       
       |                      |                      |
       |                      |                      |      * 01
       |                      |                      |
       |                      |                      |
-------+-------       -------+-------       -------+-------
       |                      |                      |
       |                      |                      |
   *   |               *      |      *         *     |     *
 (0)   |  * (1)         (1)   |   (0)        10     |    00
       |                      |                      |
Q axis |                Q axis|                Q axis|
       I axis                 I axis                 I axis
\end{lstlisting}


{\def\LTcaptype{none} % do not increment counter
\vspace{-5pt}
\captionof{table}{Constellation Characteristics}
\vspace{-10pt}
\begin{longtable}[]{@{}llll@{}}
\toprule\noalign{}
Modulation & Points & Phase States & Amplitude States \\
\midrule\noalign{}
\endhead
\bottomrule\noalign{}
\endlastfoot
\textbf{ASK} & 2 & 1 (0^\circ) & 2 (0, A) \\
\textbf{BPSK} & 2 & 2 (0^\circ, 180^\circ) & 1 (A) \\
\textbf{QPSK} & 4 & 4 (45^\circ, 135^\circ, 225^\circ, 315^\circ) & 1 (A) \\
\end{longtable}
}

\end{solutionbox}
\begin{mnemonicbox}
``Points Double When Phases Double'' - BPSK has 2
points, QPSK has 4 points

\end{mnemonicbox}
\subsection*{Question 2(c OR) [7
marks]}\label{question-2c-or-7-marks}

\textbf{Describe FSK Modulator and demodulator with block diagram and
output wave form}

\begin{solutionbox}

\textbf{FSK Modulator Diagram:}

\begin{lstlisting}
                         +---------+
             +--------+  |         |
     '1' --->| Switch |->| Osc f1  |--+
             |        |  |         |  |
Digital      +--------+  +---------+  |     +---------+
Input ---+                            +---->|         |
          |                                 | Adder   |---> FSK Signal
          |   +--------+  +---------+  +--->|         |
     '0' -+-->| Switch |->| Osc f2  |--+    +---------+
              |        |  |         |
              +--------+  +---------+
\end{lstlisting}

\textbf{FSK Demodulator Diagram:}

\begin{lstlisting}
                +---------+  +---------+  +---------+
                |         |  |         |  |         |
                | BPF f1  |->| Env     |->| Thresh  |--+
                |         |  | Detect  |  | Detect  |  |
                +---------+  +---------+  +---------+  |
                                                       |  +---------+
FSK Signal --+                                         +->|         |
             |                                            | Logic   |---> Digital
             |  +---------+  +---------+  +---------+  +->| Circuit |     Output
             |  |         |  |         |  |         |  |  |         |
             +->| BPF f2  |->| Env     |->| Thresh  |--+  +---------+
                |         |  | Detect  |  | Detect  |
                +---------+  +---------+  +---------+
\end{lstlisting}

\textbf{FSK Waveform:}

\begin{lstlisting}
Digital:  ___‾‾‾___
          0  1  0

FSK:      ~~~~~~~
          f2 f1 f2
          Low freq when 0
          High freq when 1
\end{lstlisting}

\textbf{Key Components:}

{\def\LTcaptype{none} % do not increment counter
\begin{longtable}[]{@{}ll@{}}
\toprule\noalign{}
Component & Function \\
\midrule\noalign{}
\endhead
\bottomrule\noalign{}
\endlastfoot
\textbf{Oscillators} & Generate different frequencies for 0 and 1 \\
\textbf{Bandpass Filters} & Separate the two frequencies \\
\textbf{Envelope Detectors} & Extract amplitude variations \\
\textbf{Threshold Detectors} & Convert analog to digital \\
\end{longtable}
}

\end{solutionbox}
\begin{mnemonicbox}
``Frequency Shift Key - Two Tones Tell Truth''

\end{mnemonicbox}
\subsection*{Question 3(a) [3 marks]}\label{q3a}

\textbf{State the significance of probability in communication}

\begin{solutionbox}

{\def\LTcaptype{none} % do not increment counter
\begin{longtable}[]{@{}
  >{\raggedright\arraybackslash}p{(\linewidth - 2\tabcolsep) * \real{0.5185}}
  >{\raggedright\arraybackslash}p{(\linewidth - 2\tabcolsep) * \real{0.4815}}@{}}
\toprule\noalign{}
\begin{minipage}[b]{\linewidth}\raggedright
Significance
\end{minipage} & \begin{minipage}[b]{\linewidth}\raggedright
Description
\end{minipage} \\
\midrule\noalign{}
\endhead
\bottomrule\noalign{}
\endlastfoot
\textbf{Information Measurement} & Quantifies uncertainty/surprise in
messages \\
\textbf{Channel Capacity} & Determines maximum possible data rate \\
\textbf{Error Analysis} & Predicts and minimizes communication errors \\
\end{longtable}
}

\end{solutionbox}
\begin{mnemonicbox}
``ICE - Information, Capacity, Errors'' need
probability

\end{mnemonicbox}
\subsection*{Question 3(b) [4 marks]}\label{q3b}

\textbf{State channel capacity in terms of SNR and explain its
importance}

\begin{solutionbox}

\textbf{Shannon's Channel Capacity Formula:}

\begin{lstlisting}
C = B \times log_{2}(1 + SNR)
\end{lstlisting}

\textbf{Where:}

\begin{itemize}
\tightlist
\item
  C = Channel capacity (bits/second)
\item
  B = Bandwidth (Hz)
\item
  SNR = Signal-to-Noise Ratio
\end{itemize}

\textbf{Importance:}

{\def\LTcaptype{none} % do not increment counter
\begin{longtable}[]{@{}
  >{\raggedright\arraybackslash}p{(\linewidth - 2\tabcolsep) * \real{0.4000}}
  >{\raggedright\arraybackslash}p{(\linewidth - 2\tabcolsep) * \real{0.6000}}@{}}
\toprule\noalign{}
\begin{minipage}[b]{\linewidth}\raggedright
Aspect
\end{minipage} & \begin{minipage}[b]{\linewidth}\raggedright
Importance
\end{minipage} \\
\midrule\noalign{}
\endhead
\bottomrule\noalign{}
\endlastfoot
\textbf{Theoretical Limit} & Defines maximum possible error-free data
rate \\
\textbf{System Design} & Guides bandwidth and power requirements \\
\textbf{Performance Evaluation} & Benchmark for actual system
performance \\
\textbf{Coding Efficiency} & Indicates how close a system is to optimal
performance \\
\end{longtable}
}

\end{solutionbox}
\begin{mnemonicbox}
``BEST'' - Bandwidth and Error-free Signal
Transmission

\end{mnemonicbox}
\subsection*{Question 3(c) [7 marks]}\label{q3c}

\textbf{Discuss classification of line codes with suitable example}

\begin{solutionbox}

\textbf{Diagram: Line Code Classification}

\includegraphics[width=1\linewidth,height=\textheight,keepaspectratio]{mermaid-046a603d.pdf}

\textbf{Line Code Examples:}

\includegraphics[width=1\linewidth,height=\textheight,keepaspectratio]{mermaid-0e97d331.pdf}

\textbf{Waveform Visualization:}

\begin{lstlisting}
Data:       1    0    1    1    0    1    0    0
           _|_   |   _|_  _|_   |   _|_   |    |

Unipolar   ‾‾‾‾‾     ‾‾‾‾‾‾‾‾‾     ‾‾‾‾‾
NRZ:       _____‾‾‾‾‾_____‾‾‾‾‾‾‾‾‾_____‾‾‾‾‾‾‾‾‾‾‾‾‾

Polar      ‾‾‾‾‾_____‾‾‾‾‾‾‾‾‾_____‾‾‾‾‾_______________
NRZ:       

Bipolar    ‾‾‾‾‾     _____‾‾‾‾‾     
AMI:       _____‾‾‾‾‾     _____‾‾‾‾‾‾‾‾‾‾‾‾‾‾‾‾‾‾‾‾‾‾‾
           (+ for first 1, - for second 1, etc.)
\end{lstlisting}

\textbf{Comparison Table:}

{\def\LTcaptype{none} % do not increment counter
\begin{longtable}[]{@{}
  >{\raggedright\arraybackslash}p{(\linewidth - 8\tabcolsep) * \real{0.2222}}
  >{\raggedright\arraybackslash}p{(\linewidth - 8\tabcolsep) * \real{0.2083}}
  >{\raggedright\arraybackslash}p{(\linewidth - 8\tabcolsep) * \real{0.1944}}
  >{\raggedright\arraybackslash}p{(\linewidth - 8\tabcolsep) * \real{0.2222}}
  >{\raggedright\arraybackslash}p{(\linewidth - 8\tabcolsep) * \real{0.1528}}@{}}
\toprule\noalign{}
\begin{minipage}[b]{\linewidth}\raggedright
Line Code Type
\end{minipage} & \begin{minipage}[b]{\linewidth}\raggedright
Signal Levels
\end{minipage} & \begin{minipage}[b]{\linewidth}\raggedright
DC Component
\end{minipage} & \begin{minipage}[b]{\linewidth}\raggedright
Clock Recovery
\end{minipage} & \begin{minipage}[b]{\linewidth}\raggedright
Bandwidth
\end{minipage} \\
\midrule\noalign{}
\endhead
\bottomrule\noalign{}
\endlastfoot
\textbf{Unipolar NRZ} & 0, +A & Yes & Poor & Narrow \\
\textbf{Polar NRZ} & -A, +A & Maybe & Poor & Moderate \\
\textbf{Bipolar AMI} & -A, 0, +A & No & Good & Wide \\
\end{longtable}
}

\end{solutionbox}
\begin{mnemonicbox}
``UPB - Use Proper Bits'' for Unipolar, Polar,
Bipolar

\end{mnemonicbox}
\subsection*{Question 3(a OR) [3
marks]}\label{question-3a-or-3-marks}

\textbf{Discuss conditional probability}

\begin{solutionbox}

\textbf{Conditional Probability Definition:}

\begin{lstlisting}
P(A|B) = P(A\capB) / P(B)
\end{lstlisting}


{\def\LTcaptype{none} % do not increment counter
\vspace{-5pt}
\captionof{table}{Conditional Probability in Communication}
\vspace{-10pt}
\begin{longtable}[]{@{}
  >{\raggedright\arraybackslash}p{(\linewidth - 2\tabcolsep) * \real{0.5000}}
  >{\raggedright\arraybackslash}p{(\linewidth - 2\tabcolsep) * \real{0.5000}}@{}}
\toprule\noalign{}
\begin{minipage}[b]{\linewidth}\raggedright
Application
\end{minipage} & \begin{minipage}[b]{\linewidth}\raggedright
Description
\end{minipage} \\
\midrule\noalign{}
\endhead
\bottomrule\noalign{}
\endlastfoot
\textbf{Channel Modeling} & Probability of receiving Y given X was
sent \\
\textbf{Error Detection} & Probability of error given specific
patterns \\
\textbf{Decision Making} & Optimizing receiver decisions based on
observations \\
\end{longtable}
}

\end{solutionbox}
\begin{mnemonicbox}
``CEaD'' - Calculate Events after Data

\end{mnemonicbox}
\subsection*{Question 3(b OR) [4
marks]}\label{question-3b-or-4-marks}

\textbf{Define Entropy and Information. Discuss its physical
significance}

\begin{solutionbox}

\textbf{Definitions:}

{\def\LTcaptype{none} % do not increment counter
\begin{longtable}[]{@{}
  >{\raggedright\arraybackslash}p{(\linewidth - 4\tabcolsep) * \real{0.2222}}
  >{\raggedright\arraybackslash}p{(\linewidth - 4\tabcolsep) * \real{0.4444}}
  >{\raggedright\arraybackslash}p{(\linewidth - 4\tabcolsep) * \real{0.3333}}@{}}
\toprule\noalign{}
\begin{minipage}[b]{\linewidth}\raggedright
Term
\end{minipage} & \begin{minipage}[b]{\linewidth}\raggedright
Definition
\end{minipage} & \begin{minipage}[b]{\linewidth}\raggedright
Formula
\end{minipage} \\
\midrule\noalign{}
\endhead
\bottomrule\noalign{}
\endlastfoot
\textbf{Entropy} & Average information content of a source & H(X) =
-\sumP(x)log_{2}P(x) \\
\textbf{Information} & Measure of uncertainty reduction & I(x) =
log_{2}(1/P(x)) \\
\end{longtable}
}

\textbf{Physical Significance:}

{\def\LTcaptype{none} % do not increment counter
\begin{longtable}[]{@{}ll@{}}
\toprule\noalign{}
Aspect & Significance \\
\midrule\noalign{}
\endhead
\bottomrule\noalign{}
\endlastfoot
\textbf{Unpredictability} & Higher entropy means less predictable
source \\
\textbf{Compression Limit} & Minimum bits needed to represent a
source \\
\textbf{Optimal Coding} & Guides efficient source coding design \\
\textbf{Resource Allocation} & Determines bandwidth/power
requirements \\
\end{longtable}
}

\end{solutionbox}
\begin{mnemonicbox}
``UCOR'' - Uncertainty Correlates with Optimal
Resources

\end{mnemonicbox}
\subsection*{Question 3(c OR) [7
marks]}\label{question-3c-or-7-marks}

\textbf{Describe Huffman code with suitable example}

\begin{solutionbox}

\textbf{Huffman Coding: Variable-length prefix code for lossless data
compression}

\textbf{Example: Encoding symbols \{A, B, C, D, E\}}

\textbf{Step 1: Calculate probabilities}

{\def\LTcaptype{none} % do not increment counter
\begin{longtable}[]{@{}ll@{}}
\toprule\noalign{}
Symbol & Probability \\
\midrule\noalign{}
\endhead
\bottomrule\noalign{}
\endlastfoot
A & 0.4 \\
B & 0.2 \\
C & 0.2 \\
D & 0.1 \\
E & 0.1 \\
\end{longtable}
}

\textbf{Step 2: Build Huffman Tree}

\includegraphics[width=1\linewidth,height=\textheight,keepaspectratio]{mermaid-c12ee017.pdf}

\textbf{Step 3: Assign codes}

{\def\LTcaptype{none} % do not increment counter
\begin{longtable}[]{@{}lll@{}}
\toprule\noalign{}
Symbol & Probability & Huffman Code \\
\midrule\noalign{}
\endhead
\bottomrule\noalign{}
\endlastfoot
A & 0.4 & 0 \\
B & 0.2 & 10 \\
C & 0.2 & 11 \\
D & 0.1 & 100 \\
E & 0.1 & 101 \\
\end{longtable}
}

\textbf{Average code length:} (0.4\times1) + (0.2\times2) + (0.2\times2) + (0.1\times3) +
(0.1\times3) = 1.8 bits/symbol

\end{solutionbox}
\begin{mnemonicbox}
``HIGH PROB, LOW BITS'' - Higher probability symbols
get shorter codes

\end{mnemonicbox}
\subsection*{Question 4(a) [3 marks]}\label{q4a}

\textbf{List Data transmission techniques}

\begin{solutionbox}


{\def\LTcaptype{none} % do not increment counter
\vspace{-5pt}
\captionof{table}{Data Transmission Techniques}
\vspace{-10pt}
\begin{longtable}[]{@{}
  >{\raggedright\arraybackslash}p{(\linewidth - 2\tabcolsep) * \real{0.4583}}
  >{\raggedright\arraybackslash}p{(\linewidth - 2\tabcolsep) * \real{0.5417}}@{}}
\toprule\noalign{}
\begin{minipage}[b]{\linewidth}\raggedright
Technique
\end{minipage} & \begin{minipage}[b]{\linewidth}\raggedright
Description
\end{minipage} \\
\midrule\noalign{}
\endhead
\bottomrule\noalign{}
\endlastfoot
\textbf{Serial Transmission} & Bits sent one after another over single
channel \\
\textbf{Parallel Transmission} & Multiple bits sent simultaneously over
multiple channels \\
\textbf{Synchronous Transmission} & Data sent in blocks with timing
controlled by clock \\
\textbf{Asynchronous Transmission} & Data sent with start/stop bits, no
common clock \\
\textbf{Half-Duplex} & Data flows in both directions but not
simultaneously \\
\textbf{Full-Duplex} & Data flows in both directions simultaneously \\
\end{longtable}
}

\end{solutionbox}
\begin{mnemonicbox}
``SPASH-F'' - Serial, Parallel, Asynchronous,
Synchronous, Half/Full

\end{mnemonicbox}
\subsection*{Question 4(b) [4 marks]}\label{q4b}

\textbf{Explain needs of multimedia processing for communication}

\begin{solutionbox}

\textbf{Multimedia Processing Needs:}

{\def\LTcaptype{none} % do not increment counter
\begin{longtable}[]{@{}
  >{\raggedright\arraybackslash}p{(\linewidth - 2\tabcolsep) * \real{0.3158}}
  >{\raggedright\arraybackslash}p{(\linewidth - 2\tabcolsep) * \real{0.6842}}@{}}
\toprule\noalign{}
\begin{minipage}[b]{\linewidth}\raggedright
Need
\end{minipage} & \begin{minipage}[b]{\linewidth}\raggedright
Description
\end{minipage} \\
\midrule\noalign{}
\endhead
\bottomrule\noalign{}
\endlastfoot
\textbf{Compression} & Reduces bandwidth requirements for large media
files \\
\textbf{Format Standardization} & Ensures compatibility across different
systems \\
\textbf{Quality Control} & Maintains acceptable audio/video quality
levels \\
\textbf{Synchronization} & Coordinates different media types (audio,
video, text) \\
\textbf{Error Resilience} & Protects against data loss during
transmission \\
\end{longtable}
}

\textbf{Diagram: Multimedia Processing Flow}

\includegraphics[width=1\linewidth,height=\textheight,keepaspectratio]{mermaid-96b0a38f.pdf}

\end{solutionbox}
\begin{mnemonicbox}
``CQSEF'' - Compress Quality, Standardize and Ensure
Fidelity

\end{mnemonicbox}
\subsection*{Question 4(c) [7 marks]}\label{q4c}

\textbf{Explain data transmission mode}

\begin{solutionbox}


{\def\LTcaptype{none} % do not increment counter
\vspace{-5pt}
\captionof{table}{Data Transmission Modes}
\vspace{-10pt}
\begin{longtable}[]{@{}
  >{\raggedright\arraybackslash}p{(\linewidth - 6\tabcolsep) * \real{0.1622}}
  >{\raggedright\arraybackslash}p{(\linewidth - 6\tabcolsep) * \real{0.2973}}
  >{\raggedright\arraybackslash}p{(\linewidth - 6\tabcolsep) * \real{0.2973}}
  >{\raggedright\arraybackslash}p{(\linewidth - 6\tabcolsep) * \real{0.2432}}@{}}
\toprule\noalign{}
\begin{minipage}[b]{\linewidth}\raggedright
Mode
\end{minipage} & \begin{minipage}[b]{\linewidth}\raggedright
Direction
\end{minipage} & \begin{minipage}[b]{\linewidth}\raggedright
Operation
\end{minipage} & \begin{minipage}[b]{\linewidth}\raggedright
Example
\end{minipage} \\
\midrule\noalign{}
\endhead
\bottomrule\noalign{}
\endlastfoot
\textbf{Simplex} & One-way only & Sender can't receive & Radio
broadcast \\
\textbf{Half-Duplex} & Two-way, alternating & Only one device transmits
at a time & Walkie-talkie \\
\textbf{Full-Duplex} & Two-way, simultaneous & Both devices transmit at
same time & Telephone call \\
\end{longtable}
}

\textbf{Diagram: Data Transmission Modes}

\begin{lstlisting}
Simplex:
  A -----------------> B
     Data flows one way

Half-Duplex:
  A <----------------> B
     Data flows in both directions,
     but only one direction at a time

Full-Duplex:
  A <=================> B
     Data flows in both directions
     simultaneously
\end{lstlisting}

\textbf{Comparison:}

{\def\LTcaptype{none} % do not increment counter
\begin{longtable}[]{@{}llll@{}}
\toprule\noalign{}
Parameter & Simplex & Half-Duplex & Full-Duplex \\
\midrule\noalign{}
\endhead
\bottomrule\noalign{}
\endlastfoot
\textbf{Channel Usage} & 100\% one way & 100\% alternating & 100\% both
ways \\
\textbf{Efficiency} & Low & Medium & High \\
\textbf{Implementation} & Simple & Moderate & Complex \\
\textbf{Cost} & Low & Medium & High \\
\end{longtable}
}

\end{solutionbox}
\begin{mnemonicbox}
``SHF - Speed and Handling Factors'' for Simplex,
Half-duplex, Full-duplex

\end{mnemonicbox}
\subsection*{Question 4(a OR) [3
marks]}\label{question-4a-or-3-marks}

\textbf{List Important characteristics of data communication}

\begin{solutionbox}

\textbf{Key Data Communication Characteristics:}

{\def\LTcaptype{none} % do not increment counter
\begin{longtable}[]{@{}ll@{}}
\toprule\noalign{}
Characteristic & Description \\
\midrule\noalign{}
\endhead
\bottomrule\noalign{}
\endlastfoot
\textbf{Delivery} & System must deliver data to correct destination \\
\textbf{Accuracy} & Data must arrive without alteration \\
\textbf{Timeliness} & Data must arrive within useful timeframe \\
\textbf{Jitter} & Variation in packet arrival times \\
\textbf{Security} & Protection from unauthorized access \\
\textbf{Reliability} & System resilience against failures \\
\end{longtable}
}

\end{solutionbox}
\begin{mnemonicbox}
``DATJSR'' - Delivery, Accuracy, Timeliness, Jitter,
Security, Reliability

\end{mnemonicbox}
\subsection*{Question 4(b OR) [4
marks]}\label{question-4b-or-4-marks}

\textbf{Discuss the standards for data communication}

\begin{solutionbox}


{\def\LTcaptype{none} % do not increment counter
\vspace{-5pt}
\captionof{table}{Key Data Communication Standards}
\vspace{-10pt}
\begin{longtable}[]{@{}lll@{}}
\toprule\noalign{}
Standard & Organization & Purpose \\
\midrule\noalign{}
\endhead
\bottomrule\noalign{}
\endlastfoot
\textbf{IEEE 802.x} & IEEE & LAN/MAN networking protocols \\
\textbf{X.25, X.400} & ITU-T & Packet switching, messaging \\
\textbf{TCP/IP} & IETF & Internet protocols \\
\textbf{RS-232/422/485} & EIA/TIA & Physical interfaces \\
\textbf{USB, HDMI} & USB-IF, HDMI Forum & Device connections \\
\end{longtable}
}

\textbf{Standards Organizations:}

{\def\LTcaptype{none} % do not increment counter
\begin{longtable}[]{@{}ll@{}}
\toprule\noalign{}
Organization & Role \\
\midrule\noalign{}
\endhead
\bottomrule\noalign{}
\endlastfoot
\textbf{IEEE} & Technical standards for networks \\
\textbf{ITU-T} & Telecommunications standards \\
\textbf{IETF} & Internet protocols \\
\textbf{ISO} & Overall standardization \\
\end{longtable}
}

\end{solutionbox}
\begin{mnemonicbox}
``PITS'' - Protocols, Interfaces, Transmission and
Standards

\end{mnemonicbox}
\subsection*{Question 4(c OR) [7
marks]}\label{question-4c-or-7-marks}

\textbf{Explain model of Multimedia communications and elements of
Multimedia system}

\begin{solutionbox}

\textbf{Multimedia Communication Model:}

\includegraphics[width=1\linewidth,height=\textheight,keepaspectratio]{mermaid-9f799fb6.pdf}

\textbf{Multimedia System Elements:}

{\def\LTcaptype{none} % do not increment counter
\begin{longtable}[]{@{}
  >{\raggedright\arraybackslash}p{(\linewidth - 2\tabcolsep) * \real{0.4737}}
  >{\raggedright\arraybackslash}p{(\linewidth - 2\tabcolsep) * \real{0.5263}}@{}}
\toprule\noalign{}
\begin{minipage}[b]{\linewidth}\raggedright
Element
\end{minipage} & \begin{minipage}[b]{\linewidth}\raggedright
Function
\end{minipage} \\
\midrule\noalign{}
\endhead
\bottomrule\noalign{}
\endlastfoot
\textbf{Input Devices} & Capture multimedia content (camera,
microphone) \\
\textbf{Processing Hardware} & CPU, GPU for handling multimedia data \\
\textbf{Storage} & Hard drives, SSDs, cloud storage \\
\textbf{Communication Network} & Transmits multimedia data between
systems \\
\textbf{Output Devices} & Display, speakers for content presentation \\
\textbf{Software} & Codecs, players, editors for content manipulation \\
\end{longtable}
}

\textbf{Media Types:}

{\def\LTcaptype{none} % do not increment counter
\begin{longtable}[]{@{}lll@{}}
\toprule\noalign{}
Media Type & Characteristics & Common Formats \\
\midrule\noalign{}
\endhead
\bottomrule\noalign{}
\endlastfoot
\textbf{Audio} & Temporal, streaming & MP3, WAV, AAC \\
\textbf{Video} & Temporal, spatial, high bandwidth & MP4, AVI, HEVC \\
\textbf{Image} & Spatial, static & JPEG, PNG, GIF \\
\textbf{Text} & Structured, low bandwidth & TXT, HTML, XML \\
\end{longtable}
}

\end{solutionbox}
\begin{mnemonicbox}
``CNIS-OS'' - Capture, Network, Input-output,
Storage, Output, Software

\end{mnemonicbox}
\subsection*{Question 5(a) [3 marks]}\label{q5a}

\textbf{Explain important elements of 5G technology}

\begin{solutionbox}

\textbf{Key 5G Elements:}

{\def\LTcaptype{none} % do not increment counter
\begin{longtable}[]{@{}
  >{\raggedright\arraybackslash}p{(\linewidth - 2\tabcolsep) * \real{0.4091}}
  >{\raggedright\arraybackslash}p{(\linewidth - 2\tabcolsep) * \real{0.5909}}@{}}
\toprule\noalign{}
\begin{minipage}[b]{\linewidth}\raggedright
Element
\end{minipage} & \begin{minipage}[b]{\linewidth}\raggedright
Description
\end{minipage} \\
\midrule\noalign{}
\endhead
\bottomrule\noalign{}
\endlastfoot
\textbf{Millimeter Waves} & Higher frequency (24-100 GHz) for more
bandwidth \\
\textbf{Massive MIMO} & Multiple-input multiple-output antennas for
improved capacity \\
\textbf{Beamforming} & Focused signal transmission for better
efficiency \\
\textbf{Network Slicing} & Virtual networks on shared infrastructure \\
\textbf{Edge Computing} & Processing closer to data source for lower
latency \\
\end{longtable}
}

\end{solutionbox}
\begin{mnemonicbox}
``MMBN-E'' - Millimeter, MIMO, Beamforming, Network,
Edge

\end{mnemonicbox}
\subsection*{Question 5(b) [4 marks]}\label{q5b}

\textbf{Describe Spread spectrum communication}

\begin{solutionbox}

\textbf{Spread Spectrum Definition:} Technique where signal is spread
over a wide frequency band, much wider than the minimum bandwidth
required.

\textbf{Types of Spread Spectrum:}

{\def\LTcaptype{none} % do not increment counter
\begin{longtable}[]{@{}
  >{\raggedright\arraybackslash}p{(\linewidth - 4\tabcolsep) * \real{0.2308}}
  >{\raggedright\arraybackslash}p{(\linewidth - 4\tabcolsep) * \real{0.3077}}
  >{\raggedright\arraybackslash}p{(\linewidth - 4\tabcolsep) * \real{0.4615}}@{}}
\toprule\noalign{}
\begin{minipage}[b]{\linewidth}\raggedright
Type
\end{minipage} & \begin{minipage}[b]{\linewidth}\raggedright
Method
\end{minipage} & \begin{minipage}[b]{\linewidth}\raggedright
Advantages
\end{minipage} \\
\midrule\noalign{}
\endhead
\bottomrule\noalign{}
\endlastfoot
\textbf{DSSS} (Direct Sequence) & XOR data with higher-rate pseudorandom
code & Good noise immunity \\
\textbf{FHSS} (Frequency Hopping) & Rapidly switches carrier among many
frequencies & Resists jamming \\
\textbf{THSS} (Time Hopping) & Transmits in short bursts at different
time slots & Low probability of intercept \\
\end{longtable}
}

\textbf{Diagram: DSSS Process}

\begin{lstlisting}
Data:       |___|‾‾‾|___|
            
PN Code:    |_|‾|_|‾|_|‾|_|‾|

Spread
Signal:     |_|‾|‾|_|‾|_|_|‾|
\end{lstlisting}

\end{solutionbox}
\begin{mnemonicbox}
``DFT - Difficult For Trackers'' - Direct, Frequency,
Time Hopping

\end{mnemonicbox}
\subsection*{Question 5(c) [7 marks]}\label{q5c}

\textbf{Explain block diagram of satellite communication}

\begin{solutionbox}

\textbf{Satellite Communication Block Diagram:}

\includegraphics[width=1\linewidth,height=\textheight,keepaspectratio]{mermaid-9dc7e40d.pdf}

\textbf{Key Components:}

{\def\LTcaptype{none} % do not increment counter
\begin{longtable}[]{@{}
  >{\raggedright\arraybackslash}p{(\linewidth - 2\tabcolsep) * \real{0.5238}}
  >{\raggedright\arraybackslash}p{(\linewidth - 2\tabcolsep) * \real{0.4762}}@{}}
\toprule\noalign{}
\begin{minipage}[b]{\linewidth}\raggedright
Component
\end{minipage} & \begin{minipage}[b]{\linewidth}\raggedright
Function
\end{minipage} \\
\midrule\noalign{}
\endhead
\bottomrule\noalign{}
\endlastfoot
\textbf{Earth Station (Tx)} & Source of signals, performs uplink
functions \\
\textbf{Uplink} & Transmission from earth to satellite (higher
frequency) \\
\textbf{Satellite Transponder} & Receives, amplifies, and retransmits
signals \\
\textbf{Downlink} & Transmission from satellite to earth (lower
frequency) \\
\textbf{Earth Station (Rx)} & Receives and processes downlink signals \\
\end{longtable}
}

\textbf{Frequency Bands:}

{\def\LTcaptype{none} % do not increment counter
\begin{longtable}[]{@{}lll@{}}
\toprule\noalign{}
Band & Frequency Range & Applications \\
\midrule\noalign{}
\endhead
\bottomrule\noalign{}
\endlastfoot
\textbf{C-band} & 4-8 GHz & Television, voice, data \\
\textbf{Ku-band} & 12-18 GHz & Direct broadcast, VSAT \\
\textbf{Ka-band} & 26-40 GHz & High-speed data, internet \\
\end{longtable}
}

\end{solutionbox}
\begin{mnemonicbox}
``STUDER'' - Station Transmits Uplink, Downlink to
Earth Receiver

\end{mnemonicbox}
\subsection*{Question 5(a OR) [3
marks]}\label{question-5a-or-3-marks}

\textbf{Explain features and advantages of 5G technology}

\begin{solutionbox}

\textbf{5G Features and Advantages:}

{\def\LTcaptype{none} % do not increment counter
\begin{longtable}[]{@{}
  >{\raggedright\arraybackslash}p{(\linewidth - 2\tabcolsep) * \real{0.4500}}
  >{\raggedright\arraybackslash}p{(\linewidth - 2\tabcolsep) * \real{0.5500}}@{}}
\toprule\noalign{}
\begin{minipage}[b]{\linewidth}\raggedright
Feature
\end{minipage} & \begin{minipage}[b]{\linewidth}\raggedright
Advantage
\end{minipage} \\
\midrule\noalign{}
\endhead
\bottomrule\noalign{}
\endlastfoot
\textbf{High Speed} & Up to 10 Gbps data rates for faster downloads \\
\textbf{Ultra-Low Latency} & \textless1ms response time for real-time
applications \\
\textbf{Massive Connectivity} & Up to 1 million devices per sq. km \\
\textbf{Network Slicing} & Customized virtual networks for specific
applications \\
\textbf{Improved Reliability} & 99.999\% availability for critical
services \\
\textbf{Energy Efficiency} & Lower power consumption per bit of data \\
\end{longtable}
}

\end{solutionbox}
\begin{mnemonicbox}
``HUMNER'' - High-speed, Ultra-low latency, Massive
connectivity, Network slicing, Enhanced reliability

\end{mnemonicbox}
\subsection*{Question 5(b OR) [4
marks]}\label{question-5b-or-4-marks}

\textbf{Describe Edge Computing}

\begin{solutionbox}

\textbf{Edge Computing Definition:} Computing paradigm that brings data
processing closer to the source of data generation.

\textbf{Diagram: Edge Computing Architecture}

\includegraphics[width=1\linewidth,height=\textheight,keepaspectratio]{mermaid-b9c6355b.pdf}

\textbf{Key Characteristics:}

{\def\LTcaptype{none} % do not increment counter
\begin{longtable}[]{@{}
  >{\raggedright\arraybackslash}p{(\linewidth - 2\tabcolsep) * \real{0.5517}}
  >{\raggedright\arraybackslash}p{(\linewidth - 2\tabcolsep) * \real{0.4483}}@{}}
\toprule\noalign{}
\begin{minipage}[b]{\linewidth}\raggedright
Characteristic
\end{minipage} & \begin{minipage}[b]{\linewidth}\raggedright
Description
\end{minipage} \\
\midrule\noalign{}
\endhead
\bottomrule\noalign{}
\endlastfoot
\textbf{Proximity} & Processing near data source reduces latency \\
\textbf{Distributed} & Computing resources spread across network edge \\
\textbf{Real-time Processing} & Fast response for time-critical
applications \\
\textbf{Bandwidth Optimization} & Reduces data sent to central cloud \\
\textbf{Data Privacy} & Sensitive data processed locally \\
\end{longtable}
}

\end{solutionbox}
\begin{mnemonicbox}
``PDRBD'' - Process Data Rapidly By Distributing

\end{mnemonicbox}
\subsection*{Question 5(c OR) [7
marks]}\label{question-5c-or-7-marks}

\textbf{Explain importance of block chain in Communication Security}

\begin{solutionbox}

\textbf{Blockchain in Communication Security:}

\includegraphics[width=1\linewidth,height=\textheight,keepaspectratio]{mermaid-3c178fe5.pdf}

\textbf{Security Benefits:}

{\def\LTcaptype{none} % do not increment counter
\begin{longtable}[]{@{}
  >{\raggedright\arraybackslash}p{(\linewidth - 2\tabcolsep) * \real{0.4091}}
  >{\raggedright\arraybackslash}p{(\linewidth - 2\tabcolsep) * \real{0.5909}}@{}}
\toprule\noalign{}
\begin{minipage}[b]{\linewidth}\raggedright
Benefit
\end{minipage} & \begin{minipage}[b]{\linewidth}\raggedright
Description
\end{minipage} \\
\midrule\noalign{}
\endhead
\bottomrule\noalign{}
\endlastfoot
\textbf{Immutability} & Once recorded, data cannot be altered \\
\textbf{Decentralization} & No single point of failure or control \\
\textbf{Transparency} & All transactions visible to network
participants \\
\textbf{Cryptographic Security} & Strong encryption protects data
integrity \\
\textbf{Smart Contracts} & Self-executing agreements with built-in
security \\
\textbf{Consensus Mechanisms} & Multiple validators ensure transaction
legitimacy \\
\end{longtable}
}

\textbf{Communication Applications:}

{\def\LTcaptype{none} % do not increment counter
\begin{longtable}[]{@{}
  >{\raggedright\arraybackslash}p{(\linewidth - 2\tabcolsep) * \real{0.4194}}
  >{\raggedright\arraybackslash}p{(\linewidth - 2\tabcolsep) * \real{0.5806}}@{}}
\toprule\noalign{}
\begin{minipage}[b]{\linewidth}\raggedright
Application
\end{minipage} & \begin{minipage}[b]{\linewidth}\raggedright
Security Benefit
\end{minipage} \\
\midrule\noalign{}
\endhead
\bottomrule\noalign{}
\endlastfoot
\textbf{Secure Messaging} & End-to-end encryption with tamper-proof
records \\
\textbf{Identity Management} & Self-sovereign identity verification \\
\textbf{IoT Security} & Secure device authentication and data
integrity \\
\textbf{Network Infrastructure} & Secure routing and DNS systems \\
\end{longtable}
}

\end{solutionbox}
\begin{mnemonicbox}
``DTCSCI'' - Decentralized Transparent Cryptographic
System Creates Immutability

\end{mnemonicbox}

\end{document}
