\documentclass{article}

% content/resources/templates/preamble.tex
\usepackage[margin=0.6in]{geometry}
\author{Milav Dabgar}
\usepackage{amsmath,amssymb,amsthm}
\usepackage{booktabs}
\usepackage{multirow}
\usepackage{xcolor}
\usepackage{tcolorbox}
\tcbuselibrary{breakable,skins}
\usepackage[colorlinks=true,linkcolor=blue]{hyperref}
\usepackage{titlesec}
\usepackage{enumitem}
\usepackage{tikz}
\usepackage{pgfplots}
\usepackage{circuitikz}
\usepackage[version=4]{mhchem}
\usepackage{longtable}
\usepackage{array}
\usepackage{float}
\usepackage{caption}
\usepackage{listings}

\lstset{
  basicstyle=\small\ttfamily,
  breaklines=true,
  breakatwhitespace=false,
  postbreak=\mbox{\textcolor{red}{$\hookrightarrow$}\space},
  float=false,
  numbers=left,
  numberstyle=\tiny\color{gray},
  numbersep=10pt,
  xleftmargin=2em,
  keywordstyle=\color{blue},
  commentstyle=\color{green!60!black},
  stringstyle=\color{purple},
  backgroundcolor=\color{gray!5},
  showstringspaces=false,
  tabsize=2,
  captionpos=b,
  keepspaces=true,
  columns=flexible
}

\pgfplotsset{compat=1.18}
\usetikzlibrary{shapes,arrows,positioning,calc,patterns,decorations.pathmorphing,decorations.markings,arrows.meta}

% Color scheme
\definecolor{headcolor}{RGB}{0,102,204}
\definecolor{keycolor}{RGB}{220,20,60}
\definecolor{solutioncolor}{RGB}{34,139,34}
\definecolor{mnemoniccolor}{RGB}{148,0,211}
\definecolor{codecolor}{RGB}{0,0,100}

% Spacing
\setlength{\parskip}{3pt}
\setlist[itemize]{nosep}
\setlist[enumerate]{nosep}

% Title formatting
\titleformat{\section}{\Large\bfseries\color{headcolor}}{\thesection}{1em}{}
\titleformat{\subsection}{\large\bfseries\color{headcolor}}{\thesubsection}{1em}{}

% Pandoc tightlist compatibility
\providecommand{\tightlist}{%
  \setlength{\itemsep}{0pt}\setlength{\parskip}{0pt}}

% Pandoc longtable compatibility
\newcounter{none}
\def\thenone{}


% content/resources/templates/gujarati-boxes.tex
\usepackage{fontspec}
\usepackage{polyglossia}

% Set Gujarati as main language (document is primarily in Gujarati)
% Note: gloss-gujarati.ldf doesn't exist in polyglossia, but it will use hyphenation patterns
\setdefaultlanguage{gujarati}
\setotherlanguage{english}

% Configure Gujarati font properly
% Use Language=Default to prevent polyglossia from trying to add language-specific features
% that don't exist for Gujarati, which causes "empty feature" warnings
\newfontfamily\gujaratifont[Script=Gujarati,AutoFakeBold=2.5,AutoFakeSlant=0.3]{Noto Sans Gujarati}
\setmainfont[Script=Gujarati,AutoFakeBold=2.5,AutoFakeSlant=0.3]{Noto Sans Gujarati}
% Use Noto Sans Gujarati for monospace to support Gujarati in text
\setmonofont[Scale=0.9]{Noto Sans Gujarati}

% Configure English to use the same font
\newfontfamily\englishfont[Script=Gujarati,AutoFakeBold=2.5,AutoFakeSlant=0.3]{Noto Sans Gujarati}

% Translations for polyglossia
\gappto\captionsgujarati{
  \renewcommand{\tablename}{કોષ્ટક}
  \renewcommand{\figurename}{આકૃતિ}
}

% Helper for TikZ nodes to ensure Gujarati font
\newcommand{\gu}[1]{{\gujaratifont #1}}

% Custom environments
\newtcolorbox{solutionbox}{
    breakable,
    enhanced,
    colback=solutioncolor!5!white,
    colframe=solutioncolor!75!black,
    fonttitle=\bfseries,
    title=જવાબ
}

\newtcolorbox{solutionboxnobreak}{
 colback=solutioncolor!5!white,
 colframe=solutioncolor!75!black,
 fonttitle=\bfseries,
 title=જવાબ
}

\newtcolorbox{keyformula}{
 breakable,
 enhanced,
 colback=keycolor!5!white,
 colframe=keycolor!75!black,
 fonttitle=\bfseries,
 title=રાસાયણિક સમીકરણ/સૂત્ર
}

\newtcolorbox{mnemonicbox}{
 breakable,
 enhanced,
 colback=mnemoniccolor!5!white,
 colframe=mnemoniccolor!75!black,
 fonttitle=\bfseries,
 title=મેમરી ટ્રીક
}


% Custom commands for GTU solutions
% This file defines semantic commands for consistent formatting

% Question command with automatic formatting
\newcommand{\question}[2]{%
  \section*{Question #1}%
  \textbf{#2}%
}

% OR question variant
\newcommand{\questionor}[2]{%
  \section*{Question #1 OR}%
  \textbf{#2}%
}

% Proper table environment with caption
\newenvironment{answertable}[1]{%
  \begin{table}[htbp]
  \centering
  \caption{#1}
}{%
  \end{table}
}

% Proper figure environment for diagrams
\newenvironment{answerdiagram}[1]{%
  \begin{figure}[htbp]
  \centering
  \caption{#1}
}{%
  \end{figure}
}

% Semantic markup for key terms
\newcommand{\keyword}[1]{\textbf{#1}}
\newcommand{\code}[1]{\texttt{#1}}
\newcommand{\classname}[1]{\texttt{#1}}
\newcommand{\methodname}[1]{\texttt{#1}}

% Proper quotation marks
\newcommand{\mnemonic}[1]{``#1''}


\title{ડિજિટલ એન્ડ ડેટા કોમ્યુનિકેશન (4343201) - સમર 2024 સોલ્યુશન}
\date{જૂન 11, 2024}

\begin{document}

\questionmarks{1(અ)}{3}{વ્યાખ્યાયિત કરો: (1) બીટ રેટ, (2) બાઉન્ડ રેટ અને (3) બેન્ડવિડ્થ}

\begin{solutionbox}
\textbf{જવાબ}:

\begin{center}
\captionof{table}{વ્યાખ્યાઓ}
\begin{tabulary}{\linewidth}{|L|L|}
\hline
\textbf{શબ્દ} & \textbf{વ્યાખ્યા} \\ \hline
\textbf{બીટ રેટ} & દર સેકન્ડે ટ્રાન્સમિટ થતા બિટ્સની સંખ્યા (bps) \\ \hline
\textbf{બાઉન્ડ રેટ} & દર સેકન્ડે ટ્રાન્સમિટ થતા સિગ્નલ એલિમેન્ટ્સ અથવા સિમ્બોલ્સની સંખ્યા \\ \hline
\textbf{બેન્ડવિડ્થ} & સિગ્નલ ટ્રાન્સમિટ કરવા માટે જરૂરી ફ્રીક્વન્સીઓની રેન્જ, હર્ટ્ઝ (Hz)માં માપવામાં આવે છે \\ \hline
\end{tabulary}
\end{center}
\end{solutionbox}

\begin{mnemonicbox}
\mnemonic{BBB - બિટ્સ મૂવ બાય બેન્ડ્સ}
\end{mnemonicbox}

\questionmarks{1(બ)}{4}{સિગ્નલનો બીટ રેટ 8000bps અને બાઉન્ડ રેટ 1000 બાઉન્ડ છે. દરેક સિગ્નલ દ્વારા કેટલા ડેટા એલિમેન્ટ વહન કરવામાં આવે છે? આપણને કેટલા સિગ્નલ તત્વોની જરૂર છે?}

\begin{solutionbox}
\textbf{જવાબ}:

\begin{center}
\captionof{table}{સિગ્નલ ગણતરી}
\begin{tabulary}{\linewidth}{|L|L|L|}
\hline
\textbf{પેરામીટર} & \textbf{મૂલ્ય} & \textbf{ગણતરી} \\ \hline
બીટ રેટ & 8000 bps & આપેલ છે \\ \hline
બાઉન્ડ રેટ & 1000 બાઉન્ડ & આપેલ છે \\ \hline
દરેક સિગ્નલમાં ડેટા એલિમેન્ટ્સ & 8 બિટ્સ & બીટ રેટ $\div$ બાઉન્ડ રેટ = 8000 $\div$ 1000 = 8 \\ \hline
જરૂરી સિગ્નલ એલિમેન્ટ્સ & $2^8 = 256$ & $2^{\text{દરેક સિગ્નલના બિટ્સ}}$ \\ \hline
\end{tabulary}
\end{center}

\textbf{આકૃતિ:}

\begin{center}
\begin{tikzpicture}[node distance=2cm]
    \node [gtu block] (sig) {1000 સિગ્નલ્સ/સેકન્ડ};
    \node [gtu block, right=of sig] (bits) {8 બિટ્સ ડેટા};
    \node [gtu block, right=of bits] (elem) {256 સિગ્નલ એલિમેન્ટ્સ};
    
    \draw [gtu arrow] (sig) -- node[above, font=\footnotesize] {વહન કરે છે} (bits);
    \draw [gtu arrow] (bits) -- node[above, font=\footnotesize] {જરૂર છે} (elem);
\end{tikzpicture}
\captionof{figure}{સિગ્નલ એલિમેન્ટ રેપ્રેઝન્ટેશન}
\end{center}
\end{solutionbox}

\begin{mnemonicbox}
\mnemonic{ડિવાઇડ ટુ ડિસાઇડ - દરેક સિગ્નલમાં કેટલા બિટ્સ છે તે નક્કી કરવા માટે બીટ રેટને બાઉન્ડ રેટથી ભાગો.}
\end{mnemonicbox}

\questionmarks{1(ક)}{7}{ડિજીટલ કોમ્યુનિકેશન સિસ્ટમના તત્વોનું તેના બ્લોક ડાયાગ્રામ સાથે વર્ણન કરો}

\begin{solutionbox}
\textbf{બ્લોક ડાયાગ્રામ:}

\begin{center}
\begin{tikzpicture}[node distance=1.5cm, auto, font=\footnotesize]
    % Top row
    \node [gtu block] (src) {સોર્સ};
    \node [gtu block, right=of src] (senc) {સોર્સ\\એન્કોડર};
    \node [gtu block, right=of senc] (cenc) {ચેનલ\\એન્કોડર};
    \node [gtu block, right=of cenc] (mod) {ડિજિટલ\\મોડ્યુલેટર};
    
    % Channel
    \node [gtu container, fit=(src) (mod), inner sep=0.2cm] (tx) {};
    \node [gtu block, below=1cm of mod, minimum width=3cm] (chan) {ચેનલ};
    
    % Bottom row
    \node [gtu block, below=1cm of chan] (demod) {ડિજિટલ\\ડિમોડ્યુલેટર};
    \node [gtu block, left=of demod] (cdec) {ચેનલ\\ડિકોડર};
    \node [gtu block, left=of cdec] (sdec) {સોર્સ\\ડિકોડર};
    \node [gtu block, left=of sdec] (dest) {ડેસ્ટિનેશન};

    % Connections
    \draw [gtu arrow] (src) -- (senc);
    \draw [gtu arrow] (senc) -- (cenc);
    \draw [gtu arrow] (cenc) -- (mod);
    \draw [gtu arrow] (mod) -- (chan);
    \draw [gtu arrow] (chan) -- (demod);
    \draw [gtu arrow] (demod) -- (cdec);
    \draw [gtu arrow] (cdec) -- (sdec);
    \draw [gtu arrow] (sdec) -- (dest);
\end{tikzpicture}
\captionof{figure}{ડિજિટલ કોમ્યુનિકેશન સિસ્ટમ}
\end{center}

\textbf{મુખ્ય તત્વો:}

\begin{center}
\captionof{table}{સિસ્ટમ ઘટકો}
\begin{tabulary}{\linewidth}{|L|L|}
\hline
\textbf{તત્વ} & \textbf{કાર્ય} \\ \hline
\textbf{સોર્સ} & ટ્રાન્સમિટ કરવા માટેના મેસેજ જનરેટ કરે છે \\ \hline
\textbf{સોર્સ એન્કોડર} & મેસેજને ડિજિટલ ફોર્મેટમાં કન્વર્ટ કરે છે, રિડન્ડન્સી દૂર કરે છે \\ \hline
\textbf{ચેનલ એન્કોડર} & એરર ડિટેક્શન/કરેક્શન માટે રિડન્ડન્સી ઉમેરે છે \\ \hline
\textbf{ડિજિટલ મોડ્યુલેટર} & ડિજિટલ ડેટાને ચેનલ માટે યોગ્ય સિગ્નલમાં રૂપાંતરિત કરે છે \\ \hline
\textbf{ચેનલ} & ભૌતિક માધ્યમ જે સિગ્નલને વહન કરે છે \\ \hline
\textbf{ડિજિટલ ડિમોડ્યુલેટર} & પ્રાપ્ત સિગ્નલમાંથી ડિજિટલ માહિતી અલગ કરે છે \\ \hline
\textbf{ચેનલ ડિકોડર} & ઉમેરેલી રિડન્ડન્સીનો ઉપયોગ કરીને ભૂલો શોધે/સુધારે છે \\ \hline
\textbf{સોર્સ ડિકોડર} & ડિજિટલ ડેટામાંથી ઓરિજિનલ મેસેજને ફરીથી બનાવે છે \\ \hline
\textbf{ડેસ્ટિનેશન} & અંતિમ મેસેજ પ્રાપ્ત કરે છે \\ \hline
\end{tabulary}
\end{center}
\end{solutionbox}

\begin{mnemonicbox}
\mnemonic{સેન્ડ મેસેજિસ કેરફુલી; ડેસ્ટિનેશન મસ્ટ કોમ્પ્રિહેન્ડ સિગ્નલ્સ ડીપલી}
\end{mnemonicbox}

\questionmarks{1(ક) OR}{7}{ડિજિટલ કોમ્યુનિકેશન સિસ્ટમની મૂળભૂત મર્યાદા શું છે? ડિજિટલ કોમ્યુનિકેશન સિસ્ટમના ફાયદા અને ગેરફાયદા શું છે?}

\begin{solutionbox}
\textbf{મૂળભૂત મર્યાદાઓ:}

\begin{center}
\captionof{table}{મર્યાદાઓ}
\begin{tabulary}{\linewidth}{|L|L|}
\hline
\textbf{મર્યાદા} & \textbf{વર્ણન} \\ \hline
\textbf{બેન્ડવિડ્થ} & ડિજિટલ સિગ્નલને એનાલોગ કરતાં વધુ બેન્ડવિડ્થની જરૂર પડે છે \\ \hline
\textbf{નોઇઝ} & મહત્તમ પ્રાપ્ય ડેટા રેટને મર્યાદિત કરે છે \\ \hline
\textbf{ઇક્વિપમેન્ટ} & ડિજિટલ સિસ્ટમને જટિલ હાર્ડવેર અને પ્રોસેસિંગની જરૂર પડે છે \\ \hline
\end{tabulary}
\end{center}

\textbf{ફાયદા vs ગેરફાયદા:}

\begin{center}
\captionof{table}{ફાયદા અને ગેરફાયદા}
\begin{tabulary}{\linewidth}{|L|L|}
\hline
\textbf{ફાયદા} & \textbf{ગેરફાયદા} \\ \hline
નોઇઝ ઇમ્યુનિટી & ઊંચી બેન્ડવિડ્થની જરૂરિયાતો \\ \hline
સરળ મલ્ટિપ્લેક્સિંગ & જટિલ ઉપકરણો \\ \hline
એરર ડિટેક્શન \& કરેક્શન & ક્વોન્ટાઇઝેશન એરર \\ \hline
વધુ સુરક્ષા & સિંક્રોનાઇઝેશન સમસ્યાઓ \\ \hline
સિગ્નલ રિજનરેશન & ઊંચી પ્રારંભિક કિંમત \\ \hline
કોમ્પ્યુટર સાથે ઇન્ટિગ્રેશન & સેમ્પલિંગ રેટની મર્યાદાઓ \\ \hline
\end{tabulary}
\end{center}
\end{solutionbox}

\begin{mnemonicbox}
\mnemonic{NEEDS - નોઇઝ, ઇક્વિપમેન્ટ, એન્ડ એન્વાયરન્મેન્ટ ડિટરમાઇન સક્સેસ}
\end{mnemonicbox}

\questionmarks{2(અ)}{3}{બ્લોક ડાયાગ્રામ સાથે QPSK મોડ્યુલેટરનું વર્ણન કરો}

\begin{solutionbox}
\textbf{બ્લોક ડાયાગ્રામ:}

\begin{center}
\begin{tikzpicture}[node distance=2.5cm, auto]
    \node [gtu block] (input) {સીરિયલ-ટુ\\પેરેલલ};
    \node [left=of input] (data) {ઇનપુટ ડેટા};
    
    \node [gtu decision, right=of input, yshift=1.5cm] (mult1) {$\times$};
    \node [gtu decision, right=of input, yshift=-1.5cm] (mult2) {$\times$};
    
    \node [above=of mult1] (osc1) {Cos Carrier};
    \node [below=of mult2] (osc2) {Sin Carrier};
    
    \node [gtu decision, right=3cm of input] (add) {+};
    \node [right=of add] (out) {QPSK આઉટપુટ};
    
    \draw [gtu arrow] (data) -- (input);
    \draw [gtu arrow] (input) |- node[near start] {Bit 1} (mult1);
    \draw [gtu arrow] (input) |- node[near start] {Bit 2} (mult2);
    
    \draw [gtu arrow] (osc1) -- (mult1);
    \draw [gtu arrow] (osc2) -- (mult2);
    
    \draw [gtu arrow] (mult1) -| (add);
    \draw [gtu arrow] (mult2) -| (add);
    \draw [gtu arrow] (add) -- (out);
\end{tikzpicture}
\captionof{figure}{QPSK મોડ્યુલેટર}
\end{center}

\textbf{મુખ્ય ઘટકો:}
\begin{itemize}
    \item \keyword{સીરિયલ-ટુ-પેરેલલ કન્વર્ટર}: ડેટાને 2-બિટ ગ્રુપ્સમાં વિભાજિત કરે છે
    \item \keyword{કોસાઇન કેરિયર}: પ્રથમ બિટને મોડ્યુલેટ કરે છે (I-ચેનલ)
    \item \keyword{સાઇન કેરિયર}: બીજા બિટને મોડ્યુલેટ કરે છે (Q-ચેનલ)
\end{itemize}
\end{solutionbox}

\begin{mnemonicbox}
\mnemonic{સ્પ્લિટ પેર, કેરિયર સ્ક્વેર - ડેટા જોડી (પેર)માં વહેંચાય છે, ચોરસ સિગ્નલ્સ દ્વારા વહન થાય છે}
\end{mnemonicbox}

\questionmarks{2(બ)}{4}{બ્લોક ડાયાગ્રામ સાથે ASK મોડ્યુલેટરનું વર્ણન કરો}

\begin{solutionbox}
\textbf{બ્લોક ડાયાગ્રામ:}

\begin{center}
\begin{tikzpicture}[node distance=2cm, auto]
    \node [gtu block] (mod) {પ્રોડક્ટ\\મોડ્યુલેટર};
    \node [left=of mod] (input) {ડિજિટલ\\ઇનપુટ};
    \node [below=of mod] (osc) {કેરિયર\\ઓસિલેટર};
    \node [gtu block, right=of mod] (filter) {બેન્ડપાસ\\ફિલ્ટર};
    \node [right=of filter] (out) {ASK સિગ્નલ};
    
    \draw [gtu arrow] (input) -- (mod);
    \draw [gtu arrow] (osc) -- (mod);
    \draw [gtu arrow] (mod) -- (filter);
    \draw [gtu arrow] (filter) -- (out);
\end{tikzpicture}
\captionof{figure}{ASK મોડ્યુલેટર}
\end{center}

\textbf{ASK મોડ્યુલેશન પ્રક્રિયા:}
\begin{center}
\captionof{table}{ઘટકો}
\begin{tabulary}{\linewidth}{|L|L|}
\hline
\textbf{ઘટક} & \textbf{કાર્ય} \\ \hline
\textbf{ડિજિટલ ઇનપુટ} & ટ્રાન્સમિટ કરવાના બાઇનરી ડેટા (0 અને 1) \\ \hline
\textbf{કેરિયર ઓસિલેટર} & ઉચ્ચ ફ્રીક્વન્સી સાઇન વેવ જનરેટ કરે છે \\ \hline
\textbf{પ્રોડક્ટ મોડ્યુલેટર} & ઇનપુટને કેરિયર સાથે ગુણે છે (ON/OFF) \\ \hline
\textbf{ફિલ્ટર} & અનિચ્છનીય ફ્રીક્વન્સી ઘટકોને દૂર કરે છે \\ \hline
\end{tabulary}
\end{center}
\end{solutionbox}

\begin{mnemonicbox}
\mnemonic{એમ્પ્લિફાય સિગ્નલ વેન કીન - સિગ્નલ હાઈ હોય ત્યારે કેરિયર એમ્પ્લિટ્યુડ બદલાય છે}
\end{mnemonicbox}

\questionmarks{2(ક)}{7}{ASK, FSK અને PSK ની સરખામણી કરો અને ઇનપુટ ડિજિટલ સિગ્નલ 100101000101 માટે ASK, FSK અને PSK ના વેવ ફોર્મ દોરો}

\begin{solutionbox}
\textbf{સરખામણી:}

\begin{center}
\captionof{table}{મોડ્યુલેશન સરખામણી}
\begin{tabulary}{\linewidth}{|L|L|L|L|}
\hline
\textbf{પેરામીટર} & \textbf{ASK} & \textbf{FSK} & \textbf{PSK} \\ \hline
મોડ્યુલેશન & એમ્પ્લિટ્યુડ & ફ્રીક્વન્સી & ફેઝ \\ \hline
નોઇઝ ઇમ્યુનિટી & ખરાબ & મધ્યમ & સારું \\ \hline
બેન્ડવિડ્થ & સાંકડું & વિશાળ & મધ્યમ \\ \hline
પાવર એફિશિયન્સી & ખરાબ & મધ્યમ & સારું \\ \hline
અમલીકરણ & સરળ & મધ્યમ & જટિલ \\ \hline
BER & ખરાબ & મધ્યમ & સારું \\ \hline
\end{tabulary}
\end{center}

\textbf{વેવફોર્મ્સ (ઈનપુટ: 1 0 0 1 0 1 0 0 0 1 0 1):}

\begin{center}
\begin{tikzpicture}[x=0.5cm, y=0.5cm]
    % Digital Data
    \node at (-2, 1) {Digital:};
    \draw [thick] (0,0) -- (1,0) -- (1,2) -- (2,2) -- (2,0) -- (4,0) -- (4,2) -- (5,2) -- (5,0) -- (6,0) -- (6,2) -- (7,2) -- (7,0) -- (10,0) -- (10,2) -- (11,2) -- (11,0) -- (12,0) -- (12,2) -- (13,2);
    \foreach \x/\b in {0.5/1, 1.5/0, 2.5/0, 3.5/1, 4.5/0, 5.5/1, 6.5/0, 7.5/0, 8.5/0, 9.5/1, 10.5/0, 11.5/1} {
        \node at (\x, 2.5) {\b};
    }
    
    % ASK
    \node at (-2, -3) {ASK:};
    \draw [thick] plot[domain=0:12, samples=200] (\x, {-3 + (
        (\x>=0 && \x<1) ? sin(360*\x*2) : 
        (\x>=1 && \x<3) ? 0 : 
        (\x>=3 && \x<4) ? sin(360*\x*2) :
        (\x>=4 && \x<5) ? 0 :
        (\x>=5 && \x<6) ? sin(360*\x*2) :
        (\x>=6 && \x<9) ? 0 :
        (\x>=9 && \x<10) ? sin(360*\x*2) :
        (\x>=10 && \x<11) ? 0 :
        (\x>=11 && \x<12) ? sin(360*\x*2) : 0
    )});

    % FSK (Simulated with distinct spacing)
    \node at (-2, -6) {FSK:};
    \foreach \x in {0,1,...,11} {
        \ifnum \x=0 \draw[thick] plot[domain=\x:\x+1, samples=20, variable=\t] (\t, {-6 + 0.8*sin(360*(\t-\x)*4)}); \fi % High freq
        \ifnum \x=1 \draw[thick] plot[domain=\x:\x+1, samples=20, variable=\t] (\t, {-6 + 0.8*sin(360*(\t-\x)*2)}); \fi % Low freq
        \ifnum \x=2 \draw[thick] plot[domain=\x:\x+1, samples=20, variable=\t] (\t, {-6 + 0.8*sin(360*(\t-\x)*2)}); \fi 
        \ifnum \x=3 \draw[thick] plot[domain=\x:\x+1, samples=20, variable=\t] (\t, {-6 + 0.8*sin(360*(\t-\x)*4)}); \fi
        \ifnum \x=4 \draw[thick] plot[domain=\x:\x+1, samples=20, variable=\t] (\t, {-6 + 0.8*sin(360*(\t-\x)*2)}); \fi
        \ifnum \x=5 \draw[thick] plot[domain=\x:\x+1, samples=20, variable=\t] (\t, {-6 + 0.8*sin(360*(\t-\x)*4)}); \fi
        \ifnum \x=6 \draw[thick] plot[domain=\x:\x+1, samples=20, variable=\t] (\t, {-6 + 0.8*sin(360*(\t-\x)*2)}); \fi
        \ifnum \x=7 \draw[thick] plot[domain=\x:\x+1, samples=20, variable=\t] (\t, {-6 + 0.8*sin(360*(\t-\x)*2)}); \fi
        \ifnum \x=8 \draw[thick] plot[domain=\x:\x+1, samples=20, variable=\t] (\t, {-6 + 0.8*sin(360*(\t-\x)*2)}); \fi
        \ifnum \x=9 \draw[thick] plot[domain=\x:\x+1, samples=20, variable=\t] (\t, {-6 + 0.8*sin(360*(\t-\x)*4)}); \fi
        \ifnum \x=10 \draw[thick] plot[domain=\x:\x+1, samples=20, variable=\t] (\t, {-6 + 0.8*sin(360*(\t-\x)*2)}); \fi
        \ifnum \x=11 \draw[thick] plot[domain=\x:\x+1, samples=20, variable=\t] (\t, {-6 + 0.8*sin(360*(\t-\x)*4)}); \fi
    }

    % PSK
    \node at (-2, -9) {PSK:};
     \foreach \x in {0,1,...,11} {
        \ifnum \x=0 \draw[thick] plot[domain=\x:\x+1, samples=20, variable=\t] (\t, {-9 + 0.8*sin(360*(\t-\x)*2)}); \fi % 0 deg
        \ifnum \x=1 \draw[thick] plot[domain=\x:\x+1, samples=20, variable=\t] (\t, {-9 - 0.8*sin(360*(\t-\x)*2)}); \fi % 180 deg
        \ifnum \x=2 \draw[thick] plot[domain=\x:\x+1, samples=20, variable=\t] (\t, {-9 - 0.8*sin(360*(\t-\x)*2)}); \fi
        \ifnum \x=3 \draw[thick] plot[domain=\x:\x+1, samples=20, variable=\t] (\t, {-9 + 0.8*sin(360*(\t-\x)*2)}); \fi
        \ifnum \x=4 \draw[thick] plot[domain=\x:\x+1, samples=20, variable=\t] (\t, {-9 - 0.8*sin(360*(\t-\x)*2)}); \fi
        \ifnum \x=5 \draw[thick] plot[domain=\x:\x+1, samples=20, variable=\t] (\t, {-9 + 0.8*sin(360*(\t-\x)*2)}); \fi
        \ifnum \x=6 \draw[thick] plot[domain=\x:\x+1, samples=20, variable=\t] (\t, {-9 - 0.8*sin(360*(\t-\x)*2)}); \fi
        \ifnum \x=7 \draw[thick] plot[domain=\x:\x+1, samples=20, variable=\t] (\t, {-9 - 0.8*sin(360*(\t-\x)*2)}); \fi
        \ifnum \x=8 \draw[thick] plot[domain=\x:\x+1, samples=20, variable=\t] (\t, {-9 - 0.8*sin(360*(\t-\x)*2)}); \fi
        \ifnum \x=9 \draw[thick] plot[domain=\x:\x+1, samples=20, variable=\t] (\t, {-9 + 0.8*sin(360*(\t-\x)*2)}); \fi
        \ifnum \x=10 \draw[thick] plot[domain=\x:\x+1, samples=20, variable=\t] (\t, {-9 - 0.8*sin(360*(\t-\x)*2)}); \fi
        \ifnum \x=11 \draw[thick] plot[domain=\x:\x+1, samples=20, variable=\t] (\t, {-9 + 0.8*sin(360*(\t-\x)*2)}); \fi
    }
\end{tikzpicture}
\captionof{figure}{મોડ્યુલેશન વેવફોર્મ્સ}
\end{center}
\end{solutionbox}

\begin{mnemonicbox}
\mnemonic{AFP - ઓલ્ટર ફ્રીક્વન્સીઝ ઓર ફેઝિસ - મોડ્યુલેશન પ્રકારો યાદ રાખવા માટે}
\end{mnemonicbox}

\questionmarks{2(અ) OR}{3}{બ્લોક ડાયાગ્રામ સાથે QPSK ડિમોડ્યુલેટરનું વર્ણન કરો}

\begin{solutionbox}
\textbf{બ્લોક ડાયાગ્રામ:}

\begin{center}
\begin{tikzpicture}[node distance=1.5cm, auto]
    \node (input) {QPSK સિગ્નલ};
    \node [gtu block, right=of input] (bpf) {BPF};
    
    % Upper arm
    \node [gtu decision, right=of bpf, yshift=1cm] (prod1) {$\times$};
    \node [gtu block, right=of prod1] (lpf1) {LPF};
    \node [right=of lpf1] (out1) {Bit 1};
    
    % Lower arm
    \node [gtu decision, right=of bpf, yshift=-1cm] (prod2) {$\times$};
    \node [gtu block, right=of prod2] (lpf2) {LPF};
    \node [right=of lpf2] (out2) {Bit 2};
    
    % Carriers
    \node [above=0.5cm of prod1] (cos) {Cos Carrier};
    \node [below=0.5cm of prod2] (sin) {Sin Carrier};
    
    % Connections
    \draw [gtu arrow] (input) -- (bpf);
    \draw [gtu arrow] (bpf) -- (prod1);
    \draw [gtu arrow] (bpf) -- (prod2);
    \draw [gtu arrow] (cos) -- (prod1);
    \draw [gtu arrow] (sin) -- (prod2);
    \draw [gtu arrow] (prod1) -- (lpf1);
    \draw [gtu arrow] (lpf1) -- (out1);
    \draw [gtu arrow] (prod2) -- (lpf2);
    \draw [gtu arrow] (lpf2) -- (out2);
\end{tikzpicture}
\captionof{figure}{QPSK ડિમોડ્યુલેટર}
\end{center}

\textbf{મુખ્ય ઘટકો:}
\begin{itemize}
    \item \keyword{BPF}: બેન્ડપાસ ફિલ્ટર સિગ્નલ બેન્ડવિડ્થ બહારના નોઇઝને દૂર કરે છે
    \item \keyword{પ્રોડક્ટ ડિટેક્ટર્સ}: કેરિયર સિગ્નલ્સ (cos \& sin) સાથે ગુણાકાર કરે છે
    \item \keyword{LPF}: લોપાસ ફિલ્ટર્સ મૂળ ડેટા બિટ્સને અલગ કરે છે
\end{itemize}
\end{solutionbox}

\begin{mnemonicbox}
\mnemonic{ફિલ્ટર્ડ પેર્સ ડિલિવર ડેટા - ફિલ્ટર્સ અને જોડી કેરિયર્સ ડેટા પુનઃપ્રાપ્ત કરે છે}
\end{mnemonicbox}

\questionmarks{2(બ) OR}{4}{ASK, BPSK અને QPSK ના નક્ષત્ર રેખાકૃતિ દોરો}

\begin{solutionbox}
\textbf{નક્ષત્ર આકૃતિઓ:}

\begin{center}
\begin{tikzpicture}[scale=0.8]
    % ASK
    \begin{scope}[xshift=0cm]
        \draw[<->] (-2,0) -- (2,0) node[right] {I};
        \draw[<->] (0,-2) -- (0,2) node[above] {Q};
        \node [below] at (0,-2.2) {ASK};
        \filldraw (1,0) circle (2pt) node[above right] {(1)};
        \filldraw (-1,0) circle (2pt) node[above left] {(0)};
    \end{scope}
    
    % BPSK
    \begin{scope}[xshift=5cm]
        \draw[<->] (-2,0) -- (2,0) node[right] {I};
        \draw[<->] (0,-2) -- (0,2) node[above] {Q};
        \node [below] at (0,-2.2) {BPSK};
         \filldraw (1.5,0) circle (2pt) node[above right] {$0^\circ$};
        \filldraw (-1.5,0) circle (2pt) node[above left] {$180^\circ$};
    \end{scope}
    
    % QPSK
    \begin{scope}[xshift=10cm]
        \draw[<->] (-2,0) -- (2,0) node[right] {I};
        \draw[<->] (0,-2) -- (0,2) node[above] {Q};
        \node [below] at (0,-2.2) {QPSK};
        \filldraw (1,1) circle (2pt) node[above right] {00};
        \filldraw (-1,1) circle (2pt) node[above left] {10};
        \filldraw (-1,-1) circle (2pt) node[below left] {11};
        \filldraw (1,-1) circle (2pt) node[below right] {01};
    \end{scope}
\end{tikzpicture}
\captionof{figure}{નક્ષત્ર આકૃતિઓ}
\end{center}

\textbf{લક્ષણો:}

\begin{center}
\captionof{table}{નક્ષત્ર આકૃતિઓની લક્ષણો}
\begin{tabulary}{\linewidth}{|L|L|L|}
\hline
\textbf{મોડ્યુલેશન} & \textbf{પોઇન્ટ્સ} & \textbf{ફેઝ} \\ \hline
ASK & 2 & 1 ($0^\circ$) \\ \hline
BPSK & 2 & 2 ($0^\circ, 180^\circ$) \\ \hline
QPSK & 4 & 4 ($45^\circ, 135^\circ, 225^\circ, 315^\circ$) \\ \hline
\end{tabulary}
\end{center}
\end{solutionbox}

\begin{mnemonicbox}
\mnemonic{પોઇન્ટ્સ ડબલ વેન ફેઝિસ ડબલ - BPSK માં 2 પોઇન્ટ્સ છે, QPSK માં 4 પોઇન્ટ્સ છે}
\end{mnemonicbox}

\questionmarks{2(ક) OR}{7}{બ્લોક ડાયાગ્રામ અને આઉટપુટ વેવ ફોર્મ સાથે FSK મોડ્યુલેટર અને ડિમોડ્યુલેટરનું વર્ણન કરો}

\begin{solutionbox}
\textbf{FSK મોડ્યુલેટર:}

\begin{center}
\begin{tikzpicture}[node distance=1.5cm, auto]
    \node (in1) {'1'};
    \node [gtu block, right=of in1] (sw1) {સ્વિચ};
    \node [gtu block, right=of sw1] (osc1) {Osc $f_1$};
    
    \node [below=1cm of in1] (in2) {'0'};
    \node [gtu block, right=of in2] (sw2) {સ્વિચ};
    \node [gtu block, right=of sw2] (osc2) {Osc $f_2$};
    
    \node [gtu decision, right=of osc1, yshift=-0.75cm] (add) {+};
    \node [right=of add] (out) {FSK સિગ્નલ};
    
    \node [left=of sw1, xshift=-0.5cm, yshift=-0.75cm] (data) {ડેટા ઇનપુટ};
    
    \draw [gtu arrow] (data) |- (sw1);
    \draw [gtu arrow] (data) |- (sw2);
    \draw [gtu arrow] (in1) -- (sw1);
    \draw [gtu arrow] (in2) -- (sw2);
    \draw [gtu arrow] (sw1) -- (osc1);
    \draw [gtu arrow] (sw2) -- (osc2);
    \draw [gtu arrow] (osc1) -| (add);
    \draw [gtu arrow] (osc2) -| (add);
    \draw [gtu arrow] (add) -- (out);
\end{tikzpicture}
\captionof{figure}{FSK મોડ્યુલેટર}
\end{center}

\textbf{FSK ડિમોડ્યુલેટર:}

\begin{center}
\begin{tikzpicture}[node distance=1.5cm, auto]
    \node (in) {FSK In};
    
    % Upper
    \node [gtu block, right=of in, yshift=1cm] (bpf1) {BPF $f_1$};
    \node [gtu block, right=of bpf1] (env1) {Env Det};
    \node [gtu block, right=of env1] (th1) {Threshold};
    
    % Lower
    \node [gtu block, right=of in, yshift=-1cm] (bpf2) {BPF $f_2$};
    \node [gtu block, right=of bpf2] (env2) {Env Det};
    \node [gtu block, right=of env2] (th2) {Threshold};
    
    \node [gtu block, right=of th1, yshift=-1cm] (logic) {Logic};
    \node [right=of logic] (out) {Data};
    
    \draw [gtu arrow] (in) |- (bpf1);
    \draw [gtu arrow] (in) |- (bpf2);
    
    \draw [gtu arrow] (bpf1) -- (env1);
    \draw [gtu arrow] (env1) -- (th1);
    \draw [gtu arrow] (th1) -| (logic);
    
    \draw [gtu arrow] (bpf2) -- (env2);
    \draw [gtu arrow] (env2) -- (th2);
    \draw [gtu arrow] (th2) -| (logic);
    
    \draw [gtu arrow] (logic) -- (out);
\end{tikzpicture}
\captionof{figure}{FSK ડિમોડ્યુલેટર}
\end{center}

\textbf{વેવફોર્મ:} પ્રશ્ન 2(ક) જુઓ.
\end{solutionbox}

\begin{mnemonicbox}
\mnemonic{ફ્રીક્વન્સી શિફ્ટ કી - ટુ ટોન્સ ટેલ ટ્રુથ}
\end{mnemonicbox}


% Q3 and Q4 Content
\questionmarks{3(અ)}{3}{સંચારમાં સંભાવનાનું મહત્વ જણાવો}

\begin{solutionbox}
\textbf{જવાબ}:

\begin{center}
\captionof{table}{સંભાવનાનું મહત્વ}
\begin{tabulary}{\linewidth}{|L|L|}
\hline
\textbf{મહત્વ} & \textbf{વર્ણન} \\ \hline
\textbf{ઇન્ફોર્મેશન મેઝરમેન્ટ} & મેસેજમાં અનિશ્ચિતતા/આશ્ચર્યને ક્વાન્ટિફાય કરે છે \\ \hline
\textbf{ચેનલ કેપેસિટી} & શક્ય મહત્તમ ડેટા રેટ નિર્ધારિત કરે છે \\ \hline
\textbf{એરર એનાલિસિસ} & કોમ્યુનિકેશન એરર્સની આગાહી કરે છે અને ન્યૂનતમ કરે છે \\ \hline
\end{tabulary}
\end{center}
\end{solutionbox}

\begin{mnemonicbox}
\mnemonic{ICE - ઇન્ફોર્મેશન, કેપેસિટી, એરર્સ - ને સંભાવનાની જરૂર પડે છે}
\end{mnemonicbox}

\questionmarks{3(બ)}{4}{SNR ના સંદર્ભમાં રાજ્ય ચેનલ ક્ષમતા અને તેનું મહત્વ સમજાવો}

\begin{solutionbox}
\textbf{શેનન ચેનલ કેપેસિટી ફોર્મ્યુલા:}
$$C = B \times \log_2(1 + \text{SNR})$$

\textbf{જ્યાં:}
\begin{itemize}
    \item $C$ = ચેનલ કેપેસિટી (બિટ્સ/સેકન્ડ)
    \item $B$ = બેન્ડવિડ્થ (Hz)
    \item $\text{SNR}$ = સિગ્નલ-ટુ-નોઇઝ રેશિયો
\end{itemize}

\textbf{મહત્વ:}
\begin{center}
\captionof{table}{મહત્વ}
\begin{tabulary}{\linewidth}{|L|L|}
\hline
\textbf{પાસું} & \textbf{મહત્વ} \\ \hline
\textbf{થિયોરેટિકલ લિમિટ} & એરર-ફ્રી ડેટા રેટની મહત્તમ શક્ય સીમા નિર્ધારિત કરે છે \\ \hline
\textbf{સિસ્ટમ ડિઝાઇન} & બેન્ડવિડ્થ અને પાવર જરૂરિયાતોનું માર્ગદર્શન આપે છે \\ \hline
\textbf{પરફોર્મન્સ ઇવેલ્યુએશન} & વાસ્તવિક સિસ્ટમ પરફોર્મન્સ માટે બેન્ચમાર્ક \\ \hline
\textbf{કોડિંગ એફિશિયન્સી} & દર્શાવે છે કે સિસ્ટમ ઓપ્ટિમલ પરફોર્મન્સથી કેટલી નજીક છે \\ \hline
\end{tabulary}
\end{center}
\end{solutionbox}

\begin{mnemonicbox}
\mnemonic{BEST - બેન્ડવિડ્થ એન્ડ એરર-ફ્રી સિગ્નલ ટ્રાન્સમિશન}
\end{mnemonicbox}

\questionmarks{3(ક)}{7}{યોગ્ય ઉદાહરણ સાથે લાઇન કોડના વર્ગીકરણની ચર્ચા કરો}

\begin{solutionbox}
\textbf{લાઇન કોડ વર્ગીકરણ:}

\begin{center}
\begin{tikzpicture}[node distance=1.5cm, auto, font=\small]
    \node [gtu block] (root) {લાઇન કોડ્સ};
    \node [gtu block, below left=of root, xshift=-1cm] (uni) {યુનિપોલર};
    \node [gtu block, below=of root] (pol) {પોલર};
    \node [gtu block, below right=of root, xshift=1cm] (bi) {બાયપોલર};
    
    \node [below=0.5cm of uni] (u1) {NRZ};
    \node [below=0.5cm of u1] (u2) {RZ};
    
    \node [below=0.5cm of pol] (p1) {NRZ};
    \node [below=0.5cm of p1] (p2) {RZ};
    
    \node [below=0.5cm of bi] (b1) {AMI};
    \node [below=0.5cm of b1] (b2) {સ્યુડોટર્નરી};
    
    \draw [gtu arrow] (root) -- (uni);
    \draw [gtu arrow] (root) -- (pol);
    \draw [gtu arrow] (root) -- (bi);
    \draw [gtu arrow] (uni) -- (u1);
    \draw [gtu arrow] (u1) -- (u2);
    \draw [gtu arrow] (pol) -- (p1);
    \draw [gtu arrow] (p1) -- (p2);
    \draw [gtu arrow] (bi) -- (b1);
    \draw [gtu arrow] (b1) -- (b2);
\end{tikzpicture}
\captionof{figure}{લાઇન કોડ વર્ગીકરણ}
\end{center}

\textbf{વેવફોર્મ વિઝ્યુલાઇઝેશન (ડેટા: 1 0 1 1 0 1 0 0):}

\begin{center}
\begin{tikzpicture}[x=0.7cm, y=0.5cm]
    % Data bits
    \foreach \x/\b in {0.5/1, 1.5/0, 2.5/1, 3.5/1, 4.5/0, 5.5/1, 6.5/0, 7.5/0} {
        \node at (\x, 1.5) {\textbf{\b}};
        \draw [dashed, gray] (\x+0.5, -5) -- (\x+0.5, 2);
    }
    \node at (-1.5, 1.5) {ડેટા:};

    % Unipolar NRZ
    \node at (-1.5, 0) {યુનિપોલર NRZ:};
    \draw [thick, blue] (0,0) -- (0,1) -- (1,1) -- (1,0) -- (2,0) -- (2,1) -- (4,1) -- (4,0) -- (5,0) -- (5,1) -- (6,1) -- (6,0) -- (8,0);

    % Polar NRZ
    \node at (-1.5, -2.5) {પોલર NRZ:};
    \draw [thick, red] (0,-3.5) -- (0,-1.5) -- (1,-1.5) -- (1,-3.5) -- (2,-3.5) -- (2,-1.5) -- (4,-1.5) -- (4,-3.5) -- (5,-3.5) -- (5,-1.5) -- (6,-1.5) -- (6,-3.5) -- (8,-3.5);

    % Bipolar AMI
    \node at (-1.5, -5) {બાયપોલર AMI:};
    \draw [white, fill=white] (-1,-7) rectangle (9, -3.8); % clear previous
    \draw [thick, green!50!black] (0, -5) -- (0,-4) -- (1,-4) -- (1,-5) -- (2,-5) -- (2,-6) -- (3,-6) -- (3,-4) -- (4,-4) -- (4,-5) -- (5,-5) -- (5,-6) -- (6,-6) -- (6,-5) -- (8,-5);
\end{tikzpicture}
\captionof{figure}{લાઇન કોડ વેવફોર્મ્સ}
\end{center}
\end{solutionbox}

\begin{mnemonicbox}
\mnemonic{UPB - યુઝ પ્રોપર બિટ્સ}
\end{mnemonicbox}

\questionmarks{3(અ) OR}{3}{શરતી સંભાવનાની ચર્ચા કરો}

\begin{solutionbox}
\textbf{વ્યાખ્યા:}
$$P(A|B) = \frac{P(A \cap B)}{P(B)}$$

\textbf{કોમ્યુનિકેશનમાં:}
\begin{center}
\captionof{table}{એપ્લિકેશન્સ}
\begin{tabulary}{\linewidth}{|L|L|}
\hline
\textbf{એપ્લિકેશન} & \textbf{વર્ણન} \\ \hline
\textbf{ચેનલ મોડેલિંગ} & X મોકલવામાં આવ્યું હોય તો Y પ્રાપ્ત થવાની સંભાવના \\ \hline
\textbf{એરર ડિટેક્શન} & ચોક્કસ પેટર્ન આપેલી હોય તે સંજોગોમાં એરર થવાની સંભાવના \\ \hline
\textbf{નિર્ણય લેવો} & અવલોકનોના આધારે રિસીવર નિર્ણયને ઓપ્ટિમાઇઝ કરવું \\ \hline
\end{tabulary}
\end{center}
\end{solutionbox}

\begin{mnemonicbox}
\mnemonic{CEaD - કેલ્ક્યુલેટ ઇવેન્ટ્સ આફ્ટર ડેટા}
\end{mnemonicbox}

\questionmarks{3(બ) OR}{4}{એન્ટ્રોપી અને માહિતી વ્યાખ્યાયિત કરો. તેના ભૌતિક મહત્વની ચર્ચા કરો}

\begin{solutionbox}
\textbf{વ્યાખ્યાઓ:}
\begin{center}
\captionof{table}{વ્યાખ્યાઓ}
\begin{tabulary}{\linewidth}{|L|L|L|}
\hline
\textbf{શબ્દ} & \textbf{વ્યાખ્યા} & \textbf{ફોર્મ્યુલા} \\ \hline
\textbf{એન્ટ્રોપી} & સોર્સમાં સરેરાશ માહિતી સામગ્રી & $H(X) = -\sum P(x)\log_2 P(x)$ \\ \hline
\textbf{માહિતી} & અનિશ્ચિતતા ઘટાડાનું માપ & $I(x) = \log_2(1/P(x))$ \\ \hline
\end{tabulary}
\end{center}

\textbf{ભૌતિક મહત્વ:}
\begin{itemize}
    \item \textbf{અનપ્રેડિક્ટેબિલિટી}: ઊંચી એન્ટ્રોપીનો અર્થ છે ઓછો પ્રેડિક્ટેબલ સોર્સ
    \item \textbf{કોમ્પ્રેશન લિમિટ}: સોર્સને રજૂ કરવા માટે જરૂરી ન્યૂનતમ બિટ્સ
    \item \textbf{ઓપ્ટિમલ કોડિંગ}: કાર્યક્ષમ સોર્સ કોડિંગ ડિઝાઇનનું માર્ગદર્શન આપે છે
    \item \textbf{રિસોર્સ એલોકેશન}: બેન્ડવિડ્થ/પાવર જરૂરિયાતો નક્કી કરે છે
\end{itemize}
\end{solutionbox}

\begin{mnemonicbox}
\mnemonic{UCOR - અનસર્ટેનીટી કોરિલેટ્સ વિથ ઓપ્ટિમલ રિસોર્સિસ}
\end{mnemonicbox}

\questionmarks{3(ક) OR}{7}{યોગ્ય ઉદાહરણ સાથે હફમેન કોડનું વર્ણન કરો}

\begin{solutionbox}
\textbf{હફમેન કોડિંગ}: લોસલેસ ડેટા કોમ્પ્રેશન માટે વેરિએબલ-લેન્થ પ્રીફિક્સ કોડ.

\textbf{ઉદાહરણ}:
\begin{center}
\begin{tabulary}{\linewidth}{|L|L|L|}
\hline
\textbf{સિમ્બોલ} & \textbf{સંભાવના} & \textbf{કોડ} \\ \hline
A & 0.4 & 0 \\ \hline
B & 0.2 & 10 \\ \hline
C & 0.2 & 11 \\ \hline
D & 0.1 & 100 \\ \hline
E & 0.1 & 101 \\ \hline
\end{tabulary}
\end{center}

\textbf{હફમેન ટ્રી:}
\begin{center}
\begin{tikzpicture}[level distance=1.5cm, sibling distance=2.5cm, edge from parent/.style={draw,-latex,font=\scriptsize}]
    \node [gtu state] (root) {1.0}
        child {node [gtu state] (n0) {0.6}
            child {node [gtu state] (n00) {0.4 (A)} edge from parent node[left] {0}}
            child {node [gtu state] (n01) {0.2}
                 child {node [gtu state, fill=yellow!20] (n010) {0.1 (D)} edge from parent node[left] {0}}
                 child {node [gtu state, fill=yellow!20] (n011) {0.1 (E)} edge from parent node[right] {1}}
            edge from parent node[right] {1}}
        edge from parent node[left] {0}}
        child {node [gtu state] (n1) {0.4}
            child {node [gtu state] (n10) {0.2 (B)} edge from parent node[left] {0}}
            child {node [gtu state] (n11) {0.2 (C)} edge from parent node[right] {1}}
        edge from parent node[right] {1}};
\end{tikzpicture}

\textbf{કોડ્સ:}
\begin{itemize}
    \item A: 0
    \item D: 100
    \item E: 101
    \item B: 110
    \item C: 111
\end{itemize}
\end{center}
\end{solutionbox}

\begin{mnemonicbox}
\mnemonic{હાઈ પ્રોબ, લો બિટ્સ}
\end{mnemonicbox}

\questionmarks{4(અ)}{3}{ડેટા ટ્રાન્સમિશન તકનીકોની સૂચિ બનાવો}

\begin{solutionbox}
\textbf{ડેટા ટ્રાન્સમિશન તકનીકો:}
\begin{center}
\begin{tabulary}{\linewidth}{|L|L|}
\hline
\textbf{તકનીક} & \textbf{વર્ણન} \\ \hline
સીરિયલ & સિંગલ ચેનલ પર એક પછી એક બિટ્સ મોકલવામાં આવે છે \\ \hline
પેરેલલ & મલ્ટિપલ ચેનલ્સ પર એકસાથે મલ્ટિપલ બિટ્સ મોકલવામાં આવે છે \\ \hline
સિંક્રોનસ & ક્લોક દ્વારા નિયંત્રિત ટાઈમિંગ સાથે ડેટા બ્લોક્સમાં મોકલવામાં આવે છે \\ \hline
એસિંક્રોનસ & સ્ટાર્ટ/સ્ટોપ બિટ્સ સાથે ડેટા મોકલવામાં આવે છે, કોમન ક્લોક નથી \\ \hline
હાફ-ડુપ્લેક્સ & ડેટા બંને દિશામાં વહે છે, પરંતુ એક સાથે નહીં \\ \hline
ફુલ-ડુપ્લેક્સ & ડેટા બંને દિશામાં એક સાથે વહે છે \\ \hline
\end{tabulary}
\end{center}
\end{solutionbox}

\begin{mnemonicbox}
\mnemonic{SPASH-F - સીરિયલ, પેરેલલ, એસિંક્રોનસ, સિંક્રોનસ, હાફ/ફુલ}
\end{mnemonicbox}

\questionmarks{4(બ)}{4}{સંચાર માટે મલ્ટીમીડિયા પ્રોસેસિંગની જરૂરિયાતો સમજાવો}

\begin{solutionbox}
\textbf{જરૂરિયાતો:}
\begin{itemize}
    \item \textbf{કોમ્પ્રેશન}: મોટી મીડિયા ફાઇલો માટે બેન્ડવિડ્થ જરૂરિયાતો ઘટાડે છે
    \item \textbf{ફોર્મેટ સ્ટાન્ડર્ડાઇઝેશન}: જુદા જુદા સિસ્ટમો વચ્ચે સુસંગતતા સુનિશ્ચિત કરે છે
    \item \textbf{ક્વોલિટી કંટ્રોલ}: સ્વીકાર્ય ઓડિયો/વિડિયો ક્વોલિટી સ્તર જાળવે છે
    \item \textbf{સિંક્રોનાઇઝેશન}: જુદા જુદા મીડિયા પ્રકારો (ઓડિયો, વિડિયો, ટેક્સ્ટ) સંકલિત કરે છે
    \item \textbf{એરર રેસિસ્ટન્સ}: ટ્રાન્સમિશન દરમિયાન ડેટા લોસથી રક્ષણ કરે છે
\end{itemize}

\textbf{મલ્ટીમીડિયા પ્રોસેસિંગ ફ્લો:}
\begin{center}
\begin{tikzpicture}[node distance=1.5cm, auto, font=\small]
    \node [gtu block] (raw) {રો મીડિયા};
    \node [gtu block, right=of raw] (compress) {કોમ્પ્રેશન};
    \node [gtu block, right=of compress] (format) {ફોર્મેટિંગ};
    \node [gtu block, below=of format] (tx) {ટ્રાન્સમિશન};
    \node [gtu block, left=of tx] (decompress) {ડીકોમ્પ્રેશન};
    \node [gtu block, left=of decompress] (play) {પ્લેબેક};
    
    \draw [gtu arrow] (raw) -- (compress);
    \draw [gtu arrow] (compress) -- (format);
    \draw [gtu arrow] (format) -- (tx);
    \draw [gtu arrow] (tx) -- (decompress);
    \draw [gtu arrow] (decompress) -- (play);
\end{tikzpicture}
\captionof{figure}{મલ્ટીમીડિયા પ્રોસેસિંગ ફ્લો}
\end{center}
\end{solutionbox}

\begin{mnemonicbox}
\mnemonic{CQSEF - કોમ્પ્રેસ ક્વોલિટી, સ્ટાન્ડર્ડાઇઝ એન્ડ એન્શ્યોર ફિડિલિટી}
\end{mnemonicbox}

\questionmarks{4(ક)}{7}{ડેટા ટ્રાન્સમિશન મોડ સમજાવો}

\begin{solutionbox}
\textbf{મોડ:}

\begin{center}
\begin{tikzpicture}[node distance=2cm, auto]
    % Simplex
    \node (s1) {Tx};
    \node [right=2cm of s1] (r1) {Rx};
    \draw [gtu arrow] (s1) -- node[above] {સિમ્પ્લેક્સ (એક દિશામાં)} (r1);
    
    % Half Duplex
    \node [below=1.5cm of s1] (s2) {A};
    \node [right=2cm of s2] (r2) {B};
    \draw [gtu arrow] (s2.10) -- (r2.170);
    \draw [gtu arrow] (r2.190) -- (s2.350);
    \node at ($(s2)!0.5!(r2)$) [below=0.2cm] {હાફ-ડુપ્લેક્સ (વારાફરતી)};
    
    % Full Duplex
    \node [below=1.5cm of s2] (s3) {A};
    \node [right=2cm of s3] (r3) {B};
    \draw [gtu arrow] (s3) -- node[above] {ફુલ-ડુપ્લેક્સ (એકસાથે)} (r3);
    \draw [gtu arrow] (r3) -- (s3);
\end{tikzpicture}
\captionof{figure}{ટ્રાન્સમિશન મોડ}
\end{center}

\textbf{તુલના:}
\begin{center}
\begin{tabulary}{\linewidth}{|L|L|L|L|}
\hline
\textbf{પેરામીટર} & \textbf{સિમ્પ્લેક્સ} & \textbf{હાફ-ડુપ્લેક્સ} & \textbf{ફુલ-ડુપ્લેક્સ} \\ \hline
દિશા & એક દિશામાં & બે-દિશામાં (Alt) & બે-દિશામાં (Simul) \\ \hline
કાર્યક્ષમતા & નીચી & મધ્યમ & ઊંચી \\ \hline
ખર્ચ & ઓછો & મધ્યમ & ઊંચો \\ \hline
ઉદાહરણ & રેડિયો & વોકી-ટોકી & ટેલિફોન \\ \hline
\end{tabulary}
\end{center}
\end{solutionbox}

\begin{mnemonicbox}
\mnemonic{SHF - સ્પીડ એન્ડ હેન્ડલિંગ ફેક્ટર્સ}
\end{mnemonicbox}

\questionmarks{4(અ) OR}{3}{ડેટા કમ્યુનિકેશનની મહત્વપૂર્ણ લાક્ષણિકતાઓની સૂચિ બનાવો}

\begin{solutionbox}
\textbf{લાક્ષણિકતાઓ:}
\begin{itemize}
    \item \textbf{ડિલિવરી}: યોગ્ય ડેસ્ટિનેશન
    \item \textbf{એક્યુરસી}: ફેરફાર વિના
    \item \textbf{ટાઇમલીનેસ}: ઉપયોગી સમયમર્યાદામાં
    \item \textbf{જિટર}: પેકેટ આગમન વેરિએશન
    \item \textbf{સિક્યોરિટી}: અનધિકૃત એક્સેસથી સુરક્ષા
    \item \textbf{રિલાયબિલિટી}: રેસિલિયન્સ
\end{itemize}
\end{solutionbox}

\begin{mnemonicbox}
\mnemonic{DATJSR}
\end{mnemonicbox}

\questionmarks{4(બ) OR}{4}{ડેટા કમ્યુનિકેશન માટેના ધોરણોની ચર્ચા કરો}

\begin{solutionbox}
\textbf{ધોરણો:}
\begin{center}
\begin{tabulary}{\linewidth}{|L|L|L|}
\hline
\textbf{ધોરણ} & \textbf{સંસ્થા} & \textbf{હેતુ} \\ \hline
IEEE 802.x & IEEE & LAN/MAN \\ \hline
TCP/IP & IETF & ઇન્ટરનેટ \\ \hline
X.25 & ITU-T & પેકેટ સ્વિચિંગ \\ \hline
RS-232 & EIA & ફિઝિકલ ઇન્ટરફેસ \\ \hline
USB & USB-IF & ડિવાઇસ કનેક્શન \\ \hline
\end{tabulary}
\end{center}
\end{solutionbox}

\begin{mnemonicbox}
\mnemonic{PITS - પ્રોટોકોલ્સ, ઇન્ટરફેસિસ, ટ્રાન્સમિશન એન્ડ સ્ટાન્ડર્ડ્સ}
\end{mnemonicbox}

\questionmarks{4(ક) OR}{7}{મલ્ટીમીડિયા કોમ્યુનિકેશન્સનું મોડેલ અને મલ્ટીમીડિયા સિસ્ટમના તત્વો સમજાવો}

\begin{solutionbox}
\textbf{મલ્ટીમીડિયા કોમ્યુનિકેશન મોડેલ:}

\begin{center}
\begin{tikzpicture}[node distance=1.2cm, auto, font=\footnotesize]
    \node [gtu block] (create) {ક્રિએશન};
    \node [gtu block, right=of create] (comp) {કોમ્પ્રેશન};
    \node [gtu block, right=of comp] (store) {સ્ટોરેજ};
    \node [gtu block, right=of store] (dist) {ડિસ્ટ્રિબ્યુશન};
    \node [gtu block, below=of dist] (decomp) {ડીકોમ્પ્રેશન};
    \node [gtu block, left=of decomp] (pres) {પ્રેઝન્ટેશન};
    
    \draw [gtu arrow] (create) -- (comp);
    \draw [gtu arrow] (comp) -- (store);
    \draw [gtu arrow] (store) -- (dist);
    \draw [gtu arrow] (dist) -- (decomp);
    \draw [gtu arrow] (decomp) -- (pres);
\end{tikzpicture}
\captionof{figure}{મલ્ટીમીડિયા કોમ્યુનિકેશન મોડેલ}
\end{center}

\textbf{તત્વો:}
\begin{itemize}
    \item \textbf{ઇનપુટ ડિવાઇસિસ}: કેમેરા, માઇક
    \item \textbf{પ્રોસેસિંગ}: CPU, GPU
    \item \textbf{સ્ટોરેજ}: HDD, ક્લાઉડ
    \item \textbf{નેટવર્ક}: ટ્રાન્સમિશન માધ્યમ
    \item \textbf{આઉટપુટ}: ડિસ્પ્લે, સ્પીકર્સ
    \item \textbf{સોફ્ટવેર}: કોડેક્સ, પ્લેયર્સ
\end{itemize}
\end{solutionbox}

\begin{mnemonicbox}
\mnemonic{CNIS-OS - કેપ્ચર, નેટવર્ક, ઇનપુટ-આઉટપુટ, સ્ટોરેજ, આઉટપુટ, સોફ્ટવેર}
\end{mnemonicbox}

\questionmarks{5(અ)}{3}{5G ટેક્નોલોજીના મહત્વના ઘટકો સમજાવો}

\begin{solutionbox}
\textbf{5G ના મુખ્ય ઘટકો:}
\begin{center}
\begin{tabulary}{\linewidth}{|L|L|}
\hline
\textbf{ઘટક} & \textbf{વર્ણન} \\ \hline
મિલિમીટર વેવ્સ & વધુ બેન્ડવિડ્થ માટે ઊંચી ફ્રીક્વન્સી (24-100 GHz) \\ \hline
મેસિવ MIMO & સુધારેલી ક્ષમતા માટે મલ્ટિપલ-ઇનપુટ મલ્ટિપલ-આઉટપુટ એન્ટેનાઓ \\ \hline
બીમફોર્મિંગ & વધુ કાર્યક્ષમતા માટે કેન્દ્રિત સિગ્નલ ટ્રાન્સમિશન \\ \hline
નેટવર્ક સ્લાઇસિંગ & શેર્ડ ઇન્ફ્રાસ્ટ્રક્ચર પર વર્ચ્યુઅલ નેટવર્ક્સ \\ \hline
એજ કમ્પ્યુટિંગ & ઓછા લેટન્સી માટે ડેટા સોર્સની નજીક પ્રોસેસિંગ \\ \hline
\end{tabulary}
\end{center}
\end{solutionbox}

\begin{mnemonicbox}
\mnemonic{MMBN-E - મિલિમીટર, MIMO, બીમફોર્મિંગ, નેટવર્ક, એજ}
\end{mnemonicbox}

\questionmarks{5(બ)}{4}{સ્પ્રેડ સ્પેક્ટ્રમ કમ્યુનિકેશનનું વર્ણન કરો}

\begin{solutionbox}
\textbf{વ્યાખ્યા:} એવી તકનીક જેમાં સિગ્નલને પહોળા ફ્રીક્વન્સી બેન્ડ પર ફેલાવવામાં આવે છે, જે જરૂરી મિનિમમ બેન્ડવિડ્થ કરતાં ઘણું વધારે છે.

\textbf{પ્રકારો:}
\begin{itemize}
    \item \textbf{DSSS}: ઊંચા-રેટવાળા સ્યુડોરેન્ડમ કોડ સાથે ડેટાને XOR કરવામાં આવે છે
    \item \textbf{FHSS}: કેરિયર ફ્રીક્વન્સી ઝડપથી બદલાય છે
    \item \textbf{THSS}: અલગ-અલગ ટાઇમ સ્લોટ્સમાં ટૂંકા બર્સ્ટ ટ્રાન્સમિટ કરે છે
\end{itemize}

\textbf{DSSS પ્રક્રિયા:}
\begin{center}
\begin{tikzpicture}[x=0.5cm, y=0.5cm]
    \node at (-2, 1) {ડેટા:};
    \draw [thick] (0,0) -- (1,0) -- (1,2) -- (2,2) -- (2,0) -- (3,0);
    \node at (0.5, 2.5) {0}; \node at (1.5, 2.5) {1}; \node at (2.5, 2.5) {0};
    
    \node at (-2, -1.5) {PN કોડ:};
    \draw [thick] (0,-2.5) -- (0.5,-2.5) -- (0.5,-0.5) -- (1,-0.5) -- (1,-2.5) -- (1.5,-2.5) -- (1.5,-0.5) -- (2,-0.5) -- (2,-2.5) -- (2.5,-2.5) -- (2.5,-0.5) -- (3,-0.5);
    
    \node at (-2, -4) {સ્પ્રેડ:};
    \draw [thick] (0,-5) -- (0.5,-5) -- (0.5,-3) -- (1,-3) -- (1,-5) -- (3,-5);
    \draw [white, fill=white] (0, -6) rectangle (4, -2.8);
    \draw [thick] (0,-4.5) -- (0.5,-4.5) -- (0.5,-2.5) -- (1,-2.5) -- (1,-4.5); % Match PN (0)
    \draw [thick] (1,-2.5) -- (1.5,-2.5) -- (1.5,-4.5) -- (2,-4.5) -- (2,-2.5); % Invert PN (1)
    \draw [thick] (2,-4.5) -- (2.5,-4.5) -- (2.5,-2.5) -- (3,-2.5) -- (3,-4.5); % Match PN (0)
\end{tikzpicture}
\captionof{figure}{DSSS સ્પ્રેડ સિગ્નલ}
\end{center}
\end{solutionbox}

\begin{mnemonicbox}
\mnemonic{DFT - ડિફિકલ્ટ ફોર ટ્રેકર્સ - ડાયરેક્ટ, ફ્રીક્વન્સી, ટાઇમ હોપિંગ}
\end{mnemonicbox}

\questionmarks{5(ક)}{7}{સેટેલાઇટ કોમ્યુનિકેશનના બ્લોક ડાયાગ્રામને સમજાવો}

\begin{solutionbox}
\textbf{બ્લોક ડાયાગ્રામ:}

\begin{center}
\begin{tikzpicture}[node distance=2.5cm, auto, font=\small]
    \node [gtu block, dashed] (sat) {સેટેલાઇટ\\ટ્રાન્સપોન્ડર};
    \node [gtu container, fit=(sat), label=above:સ્પેસ સેગમેન્ટ] {};
    
    \node [gtu block, below left=of sat, yshift=-1cm] (tx) {અર્થ સ્ટેશન\\(Tx)};
    \node [gtu block, below right=of sat, yshift=-1cm] (rx) {અર્થ સ્ટેશન\\(Rx)};
    \node [gtu container, fit=(tx) (rx), label=below:ગ્રાઉન્ડ સેગમેન્ટ] {};
    
    \draw [gtu arrow, dashed] (tx) -- node[left, align=right] {અપલિંક\\(High Freq)} (sat);
    \draw [gtu arrow, dashed] (sat) -- node[right, align=left] {ડાઉનલિંક\\(Low Freq)} (rx);
\end{tikzpicture}
\captionof{figure}{સેટેલાઇટ કોમ્યુનિકેશન}
\end{center}

\textbf{ઘટકો:}
\begin{center}
\begin{tabulary}{\linewidth}{|L|L|}
\hline
\textbf{ઘટક} & \textbf{કાર્ય} \\ \hline
અર્થ સ્ટેશન (Tx) & સિગ્નલ્સનો સ્ત્રોત, અપલિંક ફંક્શન્સ કરે છે \\ \hline
અપલિંક & પૃથ્વીથી સેટેલાઇટ સુધીનું ટ્રાન્સમિશન (ઊંચી ફ્રીક્વન્સી) \\ \hline
સેટેલાઇટ ટ્રાન્સપોન્ડર & સિગ્નલ્સ પ્રાપ્ત કરે છે, એમ્પ્લિફાય કરે છે, અને ફરીથી ટ્રાન્સમિટ કરે છે \\ \hline
ડાઉનલિંક & સેટેલાઇટથી પૃથ્વી સુધીનું ટ્રાન્સમિશન (નીચી ફ્રીક્વન્સી) \\ \hline
અર્થ સ્ટેશન (Rx) & ડાઉનલિંક સિગ્નલ્સ પ્રાપ્ત કરે છે અને પ્રોસેસ કરે છે \\ \hline
\end{tabulary}
\end{center}

\textbf{ફ્રીક્વન્સી બેન્ડ્સ:} C-બેન્ડ (4/6 GHz), Ku-બેન્ડ (12/14 GHz), Ka-બેન્ડ (20/30 GHz).
\end{solutionbox}

\begin{mnemonicbox}
\mnemonic{STUDER - સ્ટેશન ટ્રાન્સમિટ્સ અપલિંક, ડાઉનલિંક ટુ અર્થ રિસીવર}
\end{mnemonicbox}

\questionmarks{5(અ) OR}{3}{5G ટેકનોલોજીની વિશેષતાઓ અને ફાયદાઓ સમજાવો}

\begin{solutionbox}
\textbf{ફાયદાઓ:}
\begin{center}
\begin{tabulary}{\linewidth}{|L|L|}
\hline
\textbf{વિશેષતા} & \textbf{ફાયદો} \\ \hline
હાઈ સ્પીડ & ઝડપી ડાઉનલોડ્સ માટે 10 Gbps સુધીના ડેટા રેટ્સ \\ \hline
અલ્ટ્રા-લો લેટન્સી & રિયલ-ટાઇમ એપ્લિકેશન્સ માટે <1ms રિસ્પોન્સ ટાઇમ \\ \hline
મેસિવ કનેક્ટિવિટી & દર ચોરસ કિમી દીઠ 1 મિલિયન ઉપકરણો સુધી \\ \hline
નેટવર્ક સ્લાઇસિંગ & ચોક્કસ એપ્લિકેશન્સ માટે કસ્ટમાઇઝ્ડ વર્ચ્યુઅલ નેટવર્ક્સ \\ \hline
સુધારેલી વિશ્વસનીયતા & ક્રિટિકલ સર્વિસિસ માટે 99.999\% ઉપલબ્ધતા \\ \hline
એનર્જી એફિશિયન્સી & ડેટાના દરેક બિટ દીઠ ઓછી પાવર વપરાશ \\ \hline
\end{tabulary}
\end{center}
\end{solutionbox}

\begin{mnemonicbox}
\mnemonic{HUMNER - હાઈ-સ્પીડ, અલ્ટ્રા-લો લેટન્સી, મેસિવ કનેક્ટિવિટી, નેટવર્ક સ્લાઇસિંગ, એન્હાન્સ્ડ રિલાયબિલિટી}
\end{mnemonicbox}

\questionmarks{5(બ) OR}{4}{એજ કમ્પ્યુટિંગનું વર્ણન કરો}

\begin{solutionbox}
\textbf{વ્યાખ્યા:} કમ્પ્યુટિંગ પેરાડાઇમ જે ડેટા પ્રોસેસિંગને ડેટા જનરેશનના સ્ત્રોતની નજીક લાવે છે.

\textbf{આર્કિટેક્ચર:}
\begin{center}
\begin{tikzpicture}[node distance=1.5cm, auto, font=\small]
    \node [gtu block] (iot) {IoT ડિવાઇસિસ};
    \node [gtu block, right=of iot] (edge) {એજ ડિવાઇસિસ};
    \node [gtu block, right=of edge] (server) {એજ સર્વર્સ};
    \node [gtu block, right=of server] (cloud) {ક્લાઉડ DC};
    
    \draw [gtu arrow] (iot) -- (edge);
    \draw [gtu arrow] (edge) -- (server);
    \draw [gtu arrow] (server) -- (cloud);
\end{tikzpicture}
\captionof{figure}{એજ કમ્પ્યુટિંગ}
\end{center}

\textbf{મુખ્ય લક્ષણો:} પ્રોક્સિમિટી, ડિસ્ટ્રિબ્યુટેડ, રિયલ-ટાઇમ પ્રોસેસિંગ, બેન્ડવિડ્થ ઓપ્ટિમાઇઝેશન, ડેટા પ્રાઇવસી.
\end{solutionbox}

\begin{mnemonicbox}
\mnemonic{PDRBD - પ્રોસેસ ડેટા રેપિડલી બાય ડિસ્ટ્રિબ્યુટિંગ}
\end{mnemonicbox}

\questionmarks{5(ક) OR}{7}{કોમ્યુનિકેશન સિક્યોરિટીમાં બ્લોક ચેઈનનું મહત્વ સમજાવો}

\begin{solutionbox}
\textbf{પ્રક્રિયા:}
\begin{center}
\begin{tikzpicture}[node distance=1.2cm, auto, font=\footnotesize]
    \node [gtu block] (req) {રિક્વેસ્ટ};
    \node [gtu decision, right=of req] (ver) {વેરિફાય};
    \node [gtu block, right=of ver] (block) {બ્લોક ક્રિએશન};
    \node [gtu block, right=of block] (chain) {ચેઇનમાં એડિશન};
    \node [gtu block, below=of chain] (dist) {ડિસ્ટ્રિબ્યુશન};
    
    \draw [gtu arrow] (req) -- (ver);
    \draw [gtu arrow] (ver) -- (block);
    \draw [gtu arrow] (block) -- (chain);
    \draw [gtu arrow] (chain) -- (dist);
\end{tikzpicture}
\captionof{figure}{બ્લોકચેઇન પ્રક્રિયા}
\end{center}

\textbf{સિક્યોરિટી બેનિફિટ્સ:}
\begin{itemize}
    \item \textbf{ઇમ્યુટેબિલિટી}: એકવાર રેકોર્ડ થયેલો ડેટા બદલી શકાતો નથી
    \item \textbf{ડિસેન્ટ્રલાઇઝેશન}: નિયંત્રણ કે નિષ્ફળતાનો કોઈ એકલ પોઇન્ટ નથી
    \item \textbf{ટ્રાન્સપેરન્સી}: બધા ટ્રાન્ઝેક્શન્સ નેટવર્ક પાર્ટિસિપન્ટ્સને દેખાય છે
    \item \textbf{ક્રિપ્ટોગ્રાફિક સિક્યોરિટી}: મજબૂત એન્ક્રિપ્શન ડેટા ઇન્ટેગ્રિટીનું રક્ષણ કરે છે
    \item \textbf{સ્માર્ટ કોન્ટ્રાક્ટ્સ}: બિલ્ટ-ઇન સિક્યોરિટી સાથે સેલ્ફ-એક્ઝિક્યુટિંગ એગ્રીમેન્ટ્સ
\end{itemize}

\textbf{એપ્લિકેશન્સ}: સિક્યોર મેસેજિંગ, આઇડેન્ટિટી મેનેજમેન્ટ, IoT સિક્યોરિટી.
\end{solutionbox}

\begin{mnemonicbox}
\mnemonic{DTCSCI}
\end{mnemonicbox}

\end{document}
