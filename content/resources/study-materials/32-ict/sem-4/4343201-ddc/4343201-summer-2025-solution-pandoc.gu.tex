\documentclass[10pt,a4paper]{article}

% content/resources/templates/preamble.tex
\usepackage[margin=0.6in]{geometry}
\author{Milav Dabgar}
\usepackage{amsmath,amssymb,amsthm}
\usepackage{booktabs}
\usepackage{multirow}
\usepackage{xcolor}
\usepackage{tcolorbox}
\tcbuselibrary{breakable,skins}
\usepackage[colorlinks=true,linkcolor=blue]{hyperref}
\usepackage{titlesec}
\usepackage{enumitem}
\usepackage{tikz}
\usepackage{pgfplots}
\usepackage{circuitikz}
\usepackage[version=4]{mhchem}
\usepackage{longtable}
\usepackage{array}
\usepackage{float}
\usepackage{caption}
\usepackage{listings}

\lstset{
  basicstyle=\small\ttfamily,
  breaklines=true,
  breakatwhitespace=false,
  postbreak=\mbox{\textcolor{red}{$\hookrightarrow$}\space},
  float=false,
  numbers=left,
  numberstyle=\tiny\color{gray},
  numbersep=10pt,
  xleftmargin=2em,
  keywordstyle=\color{blue},
  commentstyle=\color{green!60!black},
  stringstyle=\color{purple},
  backgroundcolor=\color{gray!5},
  showstringspaces=false,
  tabsize=2,
  captionpos=b,
  keepspaces=true,
  columns=flexible
}

\pgfplotsset{compat=1.18}
\usetikzlibrary{shapes,arrows,positioning,calc,patterns,decorations.pathmorphing,decorations.markings,arrows.meta}

% Color scheme
\definecolor{headcolor}{RGB}{0,102,204}
\definecolor{keycolor}{RGB}{220,20,60}
\definecolor{solutioncolor}{RGB}{34,139,34}
\definecolor{mnemoniccolor}{RGB}{148,0,211}
\definecolor{codecolor}{RGB}{0,0,100}

% Spacing
\setlength{\parskip}{3pt}
\setlist[itemize]{nosep}
\setlist[enumerate]{nosep}

% Title formatting
\titleformat{\section}{\Large\bfseries\color{headcolor}}{\thesection}{1em}{}
\titleformat{\subsection}{\large\bfseries\color{headcolor}}{\thesubsection}{1em}{}

% Pandoc tightlist compatibility
\providecommand{\tightlist}{%
  \setlength{\itemsep}{0pt}\setlength{\parskip}{0pt}}

% Pandoc longtable compatibility
\newcounter{none}
\def\thenone{}


% content/resources/templates/gujarati-boxes.tex
\usepackage{fontspec}
\usepackage{polyglossia}

% Set Gujarati as main language (document is primarily in Gujarati)
% Note: gloss-gujarati.ldf doesn't exist in polyglossia, but it will use hyphenation patterns
\setdefaultlanguage{gujarati}
\setotherlanguage{english}

% Configure Gujarati font properly
% Use Language=Default to prevent polyglossia from trying to add language-specific features
% that don't exist for Gujarati, which causes "empty feature" warnings
\newfontfamily\gujaratifont[Script=Gujarati,AutoFakeBold=2.5,AutoFakeSlant=0.3]{Noto Sans Gujarati}
\setmainfont[Script=Gujarati,AutoFakeBold=2.5,AutoFakeSlant=0.3]{Noto Sans Gujarati}
% Use Noto Sans Gujarati for monospace to support Gujarati in text
\setmonofont[Scale=0.9]{Noto Sans Gujarati}

% Configure English to use the same font
\newfontfamily\englishfont[Script=Gujarati,AutoFakeBold=2.5,AutoFakeSlant=0.3]{Noto Sans Gujarati}

% Translations for polyglossia
\gappto\captionsgujarati{
  \renewcommand{\tablename}{કોષ્ટક}
  \renewcommand{\figurename}{આકૃતિ}
}

% Helper for TikZ nodes to ensure Gujarati font
\newcommand{\gu}[1]{{\gujaratifont #1}}

% Custom environments
\newtcolorbox{solutionbox}{
    breakable,
    enhanced,
    colback=solutioncolor!5!white,
    colframe=solutioncolor!75!black,
    fonttitle=\bfseries,
    title=જવાબ
}

\newtcolorbox{solutionboxnobreak}{
 colback=solutioncolor!5!white,
 colframe=solutioncolor!75!black,
 fonttitle=\bfseries,
 title=જવાબ
}

\newtcolorbox{keyformula}{
 breakable,
 enhanced,
 colback=keycolor!5!white,
 colframe=keycolor!75!black,
 fonttitle=\bfseries,
 title=રાસાયણિક સમીકરણ/સૂત્ર
}

\newtcolorbox{mnemonicbox}{
 breakable,
 enhanced,
 colback=mnemoniccolor!5!white,
 colframe=mnemoniccolor!75!black,
 fonttitle=\bfseries,
 title=મેમરી ટ્રીક
}


\begin{document}

\begin{center}
{\Huge\bfseries\color{headcolor} Subject Name (Gujarati)}\\[5pt]
{\LARGE 4343201 -- Summer 2025}\\[3pt]
{\large Semester 1 Study Material}\\[3pt]
{\normalsize\textit{Detailed Solutions and Explanations}}
\end{center}

\vspace{10pt}

\subsection*{પ્રશ્ન 1(અ) [3
ગુણ]}\label{uxaaauxab0uxab6uxaa8-1uxa85-3-uxa97uxaa3}

\textbf{બિટ રેટ, બાઉડ રેટ અને બેન્ડવિડ્થ વ્યાખ્યાયિત કરો}

\begin{solutionbox}

{\def\LTcaptype{none} % do not increment counter
\begin{longtable}[]{@{}lll@{}}
\toprule\noalign{}
પેરામીટર & વ્યાખ્યા & એકમ \\
\midrule\noalign{}
\endhead
\bottomrule\noalign{}
\endlastfoot
\textbf{બિટ રેટ} & પ્રતિ સેકન્ડ ટ્રાન્સમિટ થતા બિટ્સની સંખ્યા & bps (બિટ્સ પર
સેકન્ડ) \\
\textbf{બાઉડ રેટ} & પ્રતિ સેકન્ડ સિગ્નલ ફેરફારની સંખ્યા & બાઉડ \\
\textbf{બેન્ડવિડ્થ} & કોમ્યુનિકેશન ચેનલમાં ફ્રીક્વન્સીની રેંજ & Hz (હર્ટ્ઝ) \\
\end{longtable}
}

\begin{itemize}
\tightlist
\item
  \textbf{બિટ રેટ}: વાસ્તવિક ડેટા ટ્રાન્સમિશન સ્પીડ
\item
  \textbf{બાઉડ રેટ}: મોડ્યુલેશન રેટ અથવા સિમ્બોલ રેટ\\
\item
  \textbf{બેન્ડવિડ્થ}: ફ્રીક્વન્સી રેંજ માટે ચેનલ કેપેસિટી
\end{itemize}

\end{solutionbox}
\begin{mnemonicbox}
``બિટ્સ બાઉડ બેન્ડવિડ્થ - કોમ્યુનિકેશન માટે BBB''

\end{mnemonicbox}
\subsection*{પ્રશ્ન 1(બ) [4
ગુણ]}\label{uxaaauxab0uxab6uxaa8-1uxaac-4-uxa97uxaa3}

\textbf{બ્લોક ડાયાગ્રામ સાથે TDM સમજાવો}

\begin{solutionbox}

\begin{center}
\textbf{Mermaid Diagram (Code)}
\begin{verbatim}
{Shaded}
{Highlighting}[]
graph LR
    A[ઇનપુટ 1] {-{-}{} MUX[ટાઇમ ડિવિઝન મલ્ટિપ્લેક્સર]}
    B[ઇનપુટ 2] {-{-}{} MUX}
    C[ઇનપુટ 3] {-{-}{} MUX}
    D[ઇનપુટ 4] {-{-}{} MUX}
    MUX {-{-}{} E[ટ્રાન્સમિશન ચેનલ]}
    E {-{-}{} DEMUX[ટાઇમ ડિવિઝન ડીમલ્ટિપ્લેક્સર]}
    DEMUX {-{-}{} F[આઉટપુટ 1]}
    DEMUX {-{-}{} G[આઉટપુટ 2]}
    DEMUX {-{-}{} H[આઉટપુટ 3]}
    DEMUX {-{-}{} I[આઉટપુટ 4]}
{Highlighting}
{Shaded}
\end{verbatim}
\end{center}

\begin{itemize}
\tightlist
\item
  \textbf{TDM સિદ્ધાંત}: બહુવિધ સિગ્નલ્સ ટાઇમ સ્લોટ્સ દ્વારા સિંગલ ચેનલ શેર કરે છે
\item
  \textbf{ટાઇમ સ્લોટ્સ}: દરેક ઇનપુટને સમર્પિત સમય અવધિ મળે છે
\item
  \textbf{સિંક્રોનાઇઝેશન}: ટ્રાન્સમિટર અને રિસીવર સિંક્રોનાઇઝ હોવા જોઇએ
\item
  \textbf{ઉપયોગ}: ડિજિટલ ટેલિફોન સિસ્ટમ્સ, કમ્પ્યુટર નેટવર્ક્સ
\end{itemize}

\end{solutionbox}
\begin{mnemonicbox}
``ટાઇમ ડિવાઇડેડ મલ્ટિપલ - TDM સમય શેર કરે છે''

\end{mnemonicbox}
\subsection*{પ્રશ્ન 1(ક) [7
ગુણ]}\label{uxaaauxab0uxab6uxaa8-1uxa95-7-uxa97uxaa3}

\textbf{ડિજિટલ કોમ્યુનિકેશન સિસ્ટમનો બ્લોક ડાયાગ્રામ સમજાવો}

\begin{solutionbox}

\begin{center}
\textbf{Mermaid Diagram (Code)}
\begin{verbatim}
{Shaded}
{Highlighting}[]
graph LR
    A[માહિતી સ્રોત] {-{-}{} B[સોર્સ એન્કોડર]}
    B {-{-}{} C[ચેનલ એન્કોડર]}
    C {-{-}{} D[ડિજિટલ મોડ્યુલેટર]}
    D {-{-}{} E[ચેનલ]}
    E {-{-}{} F[ડિજિટલ ડીમોડ્યુલેટર]}
    F {-{-}{} G[ચેનલ ડીકોડર]}
    G {-{-}{} H[સોર્સ ડીકોડર]}
    H {-{-}{} I[ગંતવ્ય]}
    J[નોઇઝ] {-{-}{} E}
{Highlighting}
{Shaded}
\end{verbatim}
\end{center}

\textbf{ટેબલ: સિસ્ટમ કોમ્પોનન્ટ્સ}

{\def\LTcaptype{none} % do not increment counter
\begin{longtable}[]{@{}ll@{}}
\toprule\noalign{}
કોમ્પોનન્ટ & કાર્ય \\
\midrule\noalign{}
\endhead
\bottomrule\noalign{}
\endlastfoot
\textbf{સોર્સ એન્કોડર} & એનાલોગને ડિજિટલમાં કન્વર્ટ કરે છે \\
\textbf{ચેનલ એન્કોડર} & એરર કરેક્શન કોડ્સ ઉમેરે છે \\
\textbf{ડિજિટલ મોડ્યુલેટર} & ડિજિટલને એનાલોગ સિગ્નલમાં કન્વર્ટ કરે છે \\
\textbf{ચેનલ} & ટ્રાન્સમિશન મીડિયમ \\
\textbf{ડિજિટલ ડીમોડ્યુલેટર} & ડિજિટલ સિગ્નલ પુનઃપ્રાપ્ત કરે છે \\
\textbf{ચેનલ ડીકોડર} & એરર શોધે અને સુધારે છે \\
\textbf{સોર્સ ડીકોડર} & મૂળ સિગ્નલ પુનર્નિર્માણ કરે છે \\
\end{longtable}
}

\begin{itemize}
\tightlist
\item
  \textbf{ફાયદાઓ}: નોઇઝ પ્રતિરોધકતા, એરર કરેક્શન ક્ષમતા
\item
  \textbf{પ્રોસેસિંગ}: ડિજિટલ સિગ્નલ પ્રોસેસિંગ તકનીકો
\item
  \textbf{વિશ્વસનીયતા}: લાંબા અંતર પર વધુ સારી કામગીરી
\end{itemize}

\end{solutionbox}
\begin{mnemonicbox}
``સોર્સ ચેનલ મોડ્યુલેટ ટ્રાન્સમિટ ડીમોડ્યુલેટ ડીકોડ - SCMTDD''

\end{mnemonicbox}
\subsection*{પ્રશ્ન 1(ક OR) [7
ગુણ]}\label{uxaaauxab0uxab6uxaa8-1uxa95-or-7-uxa97uxaa3}

\textbf{કોમ્યુનિકેશન ચેનલના વિવિધ પ્રકારો સમજાવો}

\begin{solutionbox}

\textbf{ચેનલ પ્રકારો ટેબલ:}

{\def\LTcaptype{none} % do not increment counter
\begin{longtable}[]{@{}lll@{}}
\toprule\noalign{}
ચેનલ પ્રકાર & લાક્ષણિકતાઓ & ઉપયોગ \\
\midrule\noalign{}
\endhead
\bottomrule\noalign{}
\endlastfoot
\textbf{ટેલિફોન ચેનલ} & 300-3400 Hz બેન્ડવિડ્થ & વૉઇસ કોમ્યુનિકેશન \\
\textbf{કોએક્સિયલ કેબલ} & હાઇ બેન્ડવિડ્થ, શિલ્ડેડ & કેબલ TV, ઇન્ટરનેટ \\
\textbf{ઓપ્ટિકલ ફાઇબર} & ખૂબ હાઇ બેન્ડવિડ્થ, લાઇટ સિગ્નલ્સ & લાંબા અંતર, હાઇ
સ્પીડ \\
\textbf{વાયરલેસ ચેનલ} & રેડિયો ફ્રીક્વન્સી ટ્રાન્સમિશન & મોબાઇલ, સેટેલાઇટ \\
\textbf{સેટેલાઇટ ચેનલ} & લાંબા અંતર, સ્પેસ કોમ્યુનિકેશન & ગ્લોબલ કોમ્યુનિકેશન \\
\end{longtable}
}

\begin{itemize}
\tightlist
\item
  \textbf{બેન્ડવિડ્થ}: વિવિધ ચેનલ્સ અલગ-અલગ ફ્રીક્વન્સી રેંજ આપે છે
\item
  \textbf{નોઇઝ લાક્ષણિકતાઓ}: દરેક ચેનલની વિશિષ્ટ નોઇઝ પ્રોપર્ટીઝ છે
\item
  \textbf{અંતર ક્ષમતા}: લોકલથી ગ્લોબલ કવરેજ સુધી બદલાય છે
\item
  \textbf{કોસ્ટ ફેક્ટર્સ}: ઇન્સ્ટોલેશન અને મેઇન્ટેનન્સ કોસ્ટ અલગ છે
\end{itemize}

\end{solutionbox}
\begin{mnemonicbox}
``ટેલિફોન કોએક્સ ઓપ્ટિકલ વાયરલેસ સેટેલાઇટ - TCOWS ચેનલ્સ''

\end{mnemonicbox}
\subsection*{પ્રશ્ન 2(અ) [3
ગુણ]}\label{uxaaauxab0uxab6uxaa8-2uxa85-3-uxa97uxaa3}

\textbf{ડિજિટલ સિક્વન્સ 11100110 માટે ASK, FSK અને BPSK માટે મોડ્યુલેશન વેવફોર્મ
દોરો}

\begin{solutionbox}

\begin{verbatim}
ડિજિટલ ડેટા: 1  1  1  0  0  1  1  0
             +{-{-}+{-}{-}+{-}{-}+  +  +{-}{-}+{-}{-}+  +}
             |  |  |  |  |  |  |  |  |
             |  |  |  |  |  |  |  |  |
             +  +  +  +{-{-}+{-}{-}+  +  +{-}{-}+}

ASK:         +{-{-}+{-}{-}+{-}{-}+     +{-}{-}+{-}{-}+   }
             |  |  |  |     |  |  |   
             |  |  |  |     |  |  |   
             +  +  +  +{-{-}{-}{-}{-}+  +  +{-}{-}{-}}

FSK:                
                       
             હાઇ ફ્રીક્વ   લો   હાઇ  લો

BPSK:        +{-{-}+{-}{-}+{-}{-}+     +{-}{-}+{-}{-}+   }
             |  |  |  |     |  |  |   
             +  +  +  +{-{-}{-}{-}{-}+  +  +{-}{-}{-}}
             {-  {-}  {-}  {-}{-}{-}{-}{-}   {-}  {-}{-}{-}{-}}
\end{verbatim}

\end{solutionbox}
\begin{mnemonicbox}
``ASK એમ્પ્લિટ્યુડ, FSK ફ્રીક્વન્સી, BPSK ફેઝ - AFP
મોડ્યુલેશન''

\end{mnemonicbox}
\subsection*{પ્રશ્ન 2(બ) [4
ગુણ]}\label{uxaaauxab0uxab6uxaa8-2uxaac-4-uxa97uxaa3}

\textbf{ફ્રીક્વન્સી શિફ્ટ કીઇંગ (FSK) સિગ્નલના મૂળભૂત સિદ્ધાંત અને જનરેશનને સમજાવો}

\begin{solutionbox}

\textbf{FSK જનરેશન ટેબલ:}

{\def\LTcaptype{none} % do not increment counter
\begin{longtable}[]{@{}lll@{}}
\toprule\noalign{}
બાઇનરી ડેટા & ફ્રીક્વન્સી & આઉટપુટ \\
\midrule\noalign{}
\endhead
\bottomrule\noalign{}
\endlastfoot
લોજિક `1' & f_{1} (હાઇ ફ્રીક્વન્સી) & હાઇ ફ્રીક્વ કેરિયર \\
લોજિક `0' & f_{0} (લો ફ્રીક્વન્સી) & લો ફ્રીક્વ કેરિયર \\
\end{longtable}
}

\begin{center}
\textbf{Mermaid Diagram (Code)}
\begin{verbatim}
{Shaded}
{Highlighting}[]
graph LR
    A[ડિજિટલ ડેટા] {-{-}{} B[ફ્રીક્વન્સી સિલેક્ટર]}
    C[ઓસિલેટર 1 {- f1] {-}{-}{} B}
    D[ઓસિલેટર 2 {- f0] {-}{-}{} B}
    B {-{-}{} E[FSK આઉટપુટ]}
{Highlighting}
{Shaded}
\end{verbatim}
\end{center}

\begin{itemize}
\tightlist
\item
  \textbf{સિદ્ધાંત}: બાઇનરી ડેટા કેરિયર ફ્રીક્વન્સી કંટ્રોલ કરે છે
\item
  \textbf{બે ફ્રીક્વન્સીઝ}: `1' માટે f_{1} અને `0' માટે f_{0}
\item
  \textbf{કોન્સ્ટન્ટ એમ્પ્લિટ્યુડ}: માત્ર ફ્રીક્વન્સી બદલાય છે
\item
  \textbf{ડિટેક્શન}: રિસીવર પર ફ્રીક્વન્સી ડિસ્ક્રિમિનેશન
\end{itemize}

\end{solutionbox}
\begin{mnemonicbox}
``ફ્રીક્વન્સી શિફ્ટ્સ કી - FSK ફ્રીક્વન્સી કંટ્રોલ''

\end{mnemonicbox}
\subsection*{પ્રશ્ન 2(ક) [7
ગુણ]}\label{uxaaauxab0uxab6uxaa8-2uxa95-7-uxa97uxaa3}

\textbf{બ્લોક ડાયાગ્રામ અને કોન્સ્ટેલેશન ડાયાગ્રામ સાથે QPSK મોડ્યુલેટર અને
ડીમોડ્યુલેટરની કામગીરી સમજાવો}

\begin{solutionbox}

\textbf{QPSK મોડ્યુલેટર બ્લોક ડાયાગ્રામ:}

\begin{center}
\textbf{Mermaid Diagram (Code)}
\begin{verbatim}
{Shaded}
{Highlighting}[]
graph LR
    A[સીરિયલ ડેટા] {-{-}{} B[સીરિયલ ટુ પેરેલલ]}
    B {-{-}{} C[I ચેનલ]}
    B {-{-}{} D[Q ચેનલ]}
    E["કેરિયર cos(ωt)"] {-{-}{} F[મલ્ટિપ્લાયર 1]}
    G["કેરિયર sin(ωt)"] {-{-}{} H[મલ્ટિપ્લાયર 2]}
    C {-{-}{} F}
    D {-{-}{} H}
    F {-{-}{} I[એડર]}
    H {-{-}{} I}
    I {-{-}{} J[QPSK આઉટપુટ]}
{Highlighting}
{Shaded}
\end{verbatim}
\end{center}

\textbf{કોન્સ્ટેલેશન ડાયાગ્રામ:}

\begin{verbatim}
     Q
     |
  01 * * 00
     |
{-{-}{-}{-}{-}*{-}{-}{-}{-}{-}  I}
     |
  11 * * 10
     |
\end{verbatim}

\textbf{QPSK ટ્રુથ ટેબલ:}

{\def\LTcaptype{none} % do not increment counter
\begin{longtable}[]{@{}llll@{}}
\toprule\noalign{}
I & Q & ફેઝ & સિમ્બોલ \\
\midrule\noalign{}
\endhead
\bottomrule\noalign{}
\endlastfoot
0 & 0 & 45^\circ & 00 \\
0 & 1 & 135^\circ & 01 \\
1 & 1 & 225^\circ & 11 \\
1 & 0 & 315^\circ & 10 \\
\end{longtable}
}

\begin{itemize}
\tightlist
\item
  \textbf{ચાર ફેઝ}: 45^\circ, 135^\circ, 225^\circ, 315^\circ
\item
  \textbf{બે બિટ્સ પર સિમ્બોલ}: હાયર ડેટા રેટ
\item
  \textbf{કોન્સ્ટન્ટ એન્વેલોપ}: એમ્પ્લિટ્યુડ કોન્સ્ટન્ટ રહે છે
\item
  \textbf{ડીમોડ્યુલેશન}: ફેઝ ડિટેક્શન અને પેરેલલ ટુ સીરિયલ કન્વર્શન
\end{itemize}

\end{solutionbox}
\begin{mnemonicbox}
``ક્વાડરેચર ફેઝ શિફ્ટ કી - QPSK ચાર ફેઝ''

\end{mnemonicbox}
\subsection*{પ્રશ્ન 2(અ OR) [3
ગુણ]}\label{uxaaauxab0uxab6uxaa8-2uxa85-or-3-uxa97uxaa3}

\textbf{ASK મોડ્યુલેટરનો બ્લોક ડાયાગ્રામ દોરો અને તેના કામનું વર્ણન કરો}

\begin{solutionbox}

\begin{center}
\textbf{Mermaid Diagram (Code)}
\begin{verbatim}
{Shaded}
{Highlighting}[]
graph LR
    A[ડિજિટલ ડેટા] {-{-}{} B[સ્વિચ/મલ્ટિપ્લાયર]}
    C[કેરિયર ઓસિલેટર] {-{-}{} B}
    B {-{-}{} D[ASK આઉટપુટ]}
{Highlighting}
{Shaded}
\end{verbatim}
\end{center}

\begin{itemize}
\tightlist
\item
  \textbf{કામનો સિદ્ધાંત}: ડિજિટલ ડેટા કેરિયર એમ્પ્લિટ્યુડ કંટ્રોલ કરે છે
\item
  \textbf{લોજિક `1'}: પૂર્ણ એમ્પ્લિટ્યુડ સાથે કેરિયર ટ્રાન્સમિટ થાય છે
\item
  \textbf{લોજિક `0'}: કોઇ કેરિયર ટ્રાન્સમિટ થતું નથી (ઝીરો એમ્પ્લિટ્યુડ)
\item
  \textbf{સિમ્પલ ઇમ્પ્લિમેન્ટેશન}: એનાલોગ સ્વિચ અથવા મલ્ટિપ્લાયર વાપરે છે
\end{itemize}

\end{solutionbox}
\begin{mnemonicbox}
``એમ્પ્લિટ્યુડ શિફ્ટ કી - ASK એમ્પ્લિટ્યુડ કંટ્રોલ''

\end{mnemonicbox}
\subsection*{પ્રશ્ન 2(બ OR) [4
ગુણ]}\label{uxaaauxab0uxab6uxaa8-2uxaac-or-4-uxa97uxaa3}

\textbf{16-QAM ના પ્રિન્સિપલને સમજાવો અને કોન્સ્ટેલેશન ડાયાગ્રામ દોરો}

\begin{solutionbox}

\textbf{16-QAM કોન્સ્ટેલેશન:}

\begin{verbatim}
     Q
     |
  *  *  *  *
     |
  *  *  *  *
{-{-}{-}{-}{-}*{-}{-}{-}{-}{-}  I}
     |
  *  *  *  *
     |
  *  *  *  *
\end{verbatim}

\textbf{16-QAM લાક્ષણિકતાઓ ટેબલ:}

{\def\LTcaptype{none} % do not increment counter
\begin{longtable}[]{@{}ll@{}}
\toprule\noalign{}
પેરામીટર & વેલ્યુ \\
\midrule\noalign{}
\endhead
\bottomrule\noalign{}
\endlastfoot
\textbf{બિટ્સ પર સિમ્બોલ} & 4 બિટ્સ \\
\textbf{સ્ટેટ્સની સંખ્યા} & 16 \\
\textbf{એમ્પ્લિટ્યુડ લેવલ્સ} & 4 લેવલ્સ \\
\textbf{ફેઝ લેવલ્સ} & 4 ફેઝ \\
\end{longtable}
}

\begin{itemize}
\tightlist
\item
  \textbf{સિદ્ધાંત}: એમ્પ્લિટ્યુડ અને ફેઝ મોડ્યુલેશન કોમ્બાઇન કરે છે
\item
  \textbf{હાયર ડેટા રેટ}: 4 બિટ્સ પર સિમ્બોલ
\item
  \textbf{કોમ્પ્લેક્સ મોડ્યુલેશન}: પ્રિસાઇસ એમ્પ્લિટ્યુડ અને ફેઝ કંટ્રોલ જરૂરી
\item
  \textbf{ઉપયોગ}: હાઇ-સ્પીડ ડિજિટલ કોમ્યુનિકેશન
\end{itemize}

\end{solutionbox}
\begin{mnemonicbox}
``16 ક્વાડરેચર એમ્પ્લિટ્યુડ મોડ્યુલેશન - 16QAM કોમ્પ્લેક્સ
સિગ્નલ્સ''

\end{mnemonicbox}
\subsection*{પ્રશ્ન 2(ક OR) [7
ગુણ]}\label{uxaaauxab0uxab6uxaa8-2uxa95-or-7-uxa97uxaa3}

\textbf{બ્લોક ડાયાગ્રામ અને વેવફોર્મ સાથે BPSK મોડ્યુલેટર અને ડીમોડ્યુલેટરનું કામ
સમજાવો}

\begin{solutionbox}

\textbf{BPSK મોડ્યુલેટર:}

\begin{center}
\textbf{Mermaid Diagram (Code)}
\begin{verbatim}
{Shaded}
{Highlighting}[]
graph LR
    A[ડિજિટલ ડેટા] {-{-}{} B[NRZ એન્કોડર]}
    B {-{-}{} C[બેલેન્સ્ડ મોડ્યુલેટર]}
    D[કેરિયર ઓસિલેટર] {-{-}{} C}
    C {-{-}{} E[BPSK આઉટપુટ]}
{Highlighting}
{Shaded}
\end{verbatim}
\end{center}

\textbf{BPSK ડીમોડ્યુલેટર:}

\begin{center}
\textbf{Mermaid Diagram (Code)}
\begin{verbatim}
{Shaded}
{Highlighting}[]
graph LR
    A[BPSK ઇનપુટ] {-{-}{} B[બેલેન્સ્ડ ડીમોડ્યુલેટર]}
    C[લોકલ કેરિયર] {-{-}{} B}
    B {-{-}{} D[લો પાસ ફિલ્ટર]}
    D {-{-}{} E[ડિસિઝન સર્કિટ]}
    E {-{-}{} F[ડિજિટલ આઉટપુટ]}
{Highlighting}
{Shaded}
\end{verbatim}
\end{center}

\textbf{BPSK વેવફોર્મ્સ:}

\begin{verbatim}
ડેટા:    1    0    1    0
        +{-{-}{-}{-}+    +{-}{-}{-}{-}+{-}{-}{-}{-}}
        |    |    |    |
        +    +{-{-}{-}{-}+    +{-}{-}{-}{-}}

કેરિયર: 
         

BPSK:             
         
\end{verbatim}

\begin{itemize}
\tightlist
\item
  \textbf{ફેઝ શિફ્ટ}: `1' અને `0' વચ્ચે 180^\circ
\item
  \textbf{કોહેરન્ટ ડિટેક્શન}: સિંક્રોનાઇઝ્ડ કેરિયર જરૂરી
\item
  \textbf{બેસ્ટ પરફોર્મન્સ}: સૌથી ઓછી બિટ એરર રેટ
\item
  \textbf{કોન્સ્ટન્ટ એન્વેલોપ}: એમ્પ્લિટ્યુડ કોન્સ્ટન્ટ રહે છે
\end{itemize}

\end{solutionbox}
\begin{mnemonicbox}
``બાઇનરી ફેઝ શિફ્ટ કી - BPSK બે ફેઝ''

\end{mnemonicbox}
\subsection*{પ્રશ્ન 3(અ) [3
ગુણ]}\label{uxaaauxab0uxab6uxaa8-3uxa85-3-uxa97uxaa3}

\textbf{SNR ના સંદર્ભમાં ચેનલ ક્ષમતાને વ્યાખ્યાયિત કરો અને તેનું મહત્વ સમજાવો}

\begin{solutionbox}

\textbf{શેનોનના ચેનલ કેપેસિટી ફોર્મ્યુલા:}

{\def\LTcaptype{none} % do not increment counter
\begin{longtable}[]{@{}ll@{}}
\toprule\noalign{}
ફોર્મ્યુલા & C = B log_{2}(1 + S/N) \\
\midrule\noalign{}
\endhead
\bottomrule\noalign{}
\endlastfoot
\textbf{C} & ચેનલ કેપેસિટી (bps) \\
\textbf{B} & બેન્ડવિડ્થ (Hz) \\
\textbf{S/N} & સિગ્નલ-ટુ-નોઇઝ રેશિયો \\
\end{longtable}
}

\begin{itemize}
\tightlist
\item
  \textbf{મહત્વ}: મહત્તમ થિયોરેટિકલ ડેટા રેટ
\item
  \textbf{SNR અસર}: વધુ SNR વધુ કેપેસિટીને મંજૂરી આપે છે
\item
  \textbf{બેન્ડવિડ્થ ટ્રેડ-ઓફ}: SNR માટે બેન્ડવિડ્થ બદલી શકાય છે
\item
  \textbf{ડિઝાઇન લિમિટ}: સિસ્ટમ ડિઝાઇન માટે ઉપરની સીમા સેટ કરે છે
\end{itemize}

\end{solutionbox}
\begin{mnemonicbox}
``ચેનલ કેપેસિટી શેનોનની લિમિટ - CCSL''

\end{mnemonicbox}
\subsection*{પ્રશ્ન 3(બ) [4
ગુણ]}\label{uxaaauxab0uxab6uxaa8-3uxaac-4-uxa97uxaa3}

\textbf{અસિંક્રોનસ અને સિંક્રોનસ સીરિયલ ડેટા કોમ્યુનિકેશન તકનીકોનું વર્ણન કરો}

\begin{solutionbox}

\textbf{સરખામણી ટેબલ:}

{\def\LTcaptype{none} % do not increment counter
\begin{longtable}[]{@{}lll@{}}
\toprule\noalign{}
પેરામીટર & સિંક્રોનસ & અસિંક્રોનસ \\
\midrule\noalign{}
\endhead
\bottomrule\noalign{}
\endlastfoot
\textbf{ક્લોક} & અલગ ક્લોક સિગ્નલ & કોઇ અલગ ક્લોક નથી \\
\textbf{સ્ટાર્ટ/સ્ટોપ બિટ્સ} & જરૂરી નથી & સ્ટાર્ટ અને સ્ટોપ બિટ્સ \\
\textbf{સ્પીડ} & વધારે & ઓછી \\
\textbf{કોસ્ટ} & વધારે & ઓછી \\
\end{longtable}
}

\begin{itemize}
\tightlist
\item
  \textbf{સિંક્રોનસ}: ક્લોક સિંક્રોનાઇઝેશન જરૂરી
\item
  \textbf{અસિંક્રોનસ}: સ્ટાર્ટ/સ્ટોપ બિટ્સ સાથે સેલ્ફ-સિંક્રોના
\item
  \textbf{ઉપયોગ}: સિંક્રોનસ હાઇ-સ્પીડ માટે, અસિંક્રોનસ સિમ્પલ સિસ્ટમ્સ માટે
\item
  \textbf{કાર્યક્ષમતા}: સિંક્રોનસ વધુ કાર્યક્ષમ, અસિંક્રોનસ વધુ લવચીક
\end{itemize}

\end{solutionbox}
\begin{mnemonicbox}
``સિંક ક્લોક, અસિંક સ્ટાર્ટ-સ્ટોપ - SCSS''

\end{mnemonicbox}
\subsection*{પ્રશ્ન 3(ક) [7
ગુણ]}\label{uxaaauxab0uxab6uxaa8-3uxa95-7-uxa97uxaa3}

\textbf{યોગ્ય ઉદાહરણની મદદથી હફમેન કોડિંગ સમજાવો}

\begin{solutionbox}

\textbf{ઉદાહરણ: અક્ષરો A, B, C, D સંભાવનાઓ 0.4, 0.3, 0.2, 0.1 સાથે}

\textbf{સ્ટેપ-બાય-સ્ટેપ હફમેન ટ્રી કન્સ્ટ્રક્શન:}

\begin{verbatim}
સ્ટેપ 1: સંભાવનાઓની યાદી
A: 0.4, B: 0.3, C: 0.2, D: 0.1

સ્ટેપ 2: સૌથી નીચી કોમ્બાઇન કરો
       0.3
      /   {}
   C:0.2  D:0.1

સ્ટેપ 3: કોમ્બાઇનિંગ ચાલુ રાખો
       0.6
      /   {}
   B:0.3   0.3
          /   {}
       C:0.2  D:0.1

સ્ટેપ 4: અંતિમ ટ્રી
        1.0
       /   {}
    A:0.4   0.6
           /   {}
        B:0.3   0.3
               /   {}
            C:0.2  D:0.1
\end{verbatim}

\textbf{હફમેન કોડ્સ ટેબલ:}

{\def\LTcaptype{none} % do not increment counter
\begin{longtable}[]{@{}lll@{}}
\toprule\noalign{}
અક્ષર & સંભાવના & કોડ \\
\midrule\noalign{}
\endhead
\bottomrule\noalign{}
\endlastfoot
A & 0.4 & 0 \\
B & 0.3 & 10 \\
C & 0.2 & 110 \\
D & 0.1 & 111 \\
\end{longtable}
}

\begin{itemize}
\tightlist
\item
  \textbf{એવરેજ કોડ લેન્થ}: 0.4\times1 + 0.3\times2 + 0.2\times3 + 0.1\times3 = 1.9 બિટ્સ
\item
  \textbf{કમ્પ્રેશન પ્રાપ્ત}: પ્રતિ અક્ષર એવરેજ બિટ્સ ઘટાડે છે
\item
  \textbf{પ્રીફિક્સ પ્રોપર્ટી}: કોઇ કોડ બીજાનો પ્રીફિક્સ નથી
\end{itemize}

\end{solutionbox}
\begin{mnemonicbox}
``હફમેન મિનિમમ એવરેજ લેન્થ - HMAL''

\end{mnemonicbox}
\subsection*{પ્રશ્ન 3(અ OR) [3
ગુણ]}\label{uxaaauxab0uxab6uxaa8-3uxa85-or-3-uxa97uxaa3}

\textbf{સંચારમાં સંભાવના અને એન્ટ્રોપીનું મહત્વ જણાવો}

\begin{solutionbox}

\textbf{મહત્વ ટેબલ:}

{\def\LTcaptype{none} % do not increment counter
\begin{longtable}[]{@{}ll@{}}
\toprule\noalign{}
કન્સેપ્ટ & મહત્વ \\
\midrule\noalign{}
\endhead
\bottomrule\noalign{}
\endlastfoot
\textbf{સંભાવના} & માહિતીની ઘટનાની સંભાવના માપે છે \\
\textbf{એન્ટ્રોપી} & એવરેજ માહિતી સામગ્રી માપે છે \\
\textbf{મહત્તમ એન્ટ્રોપી} & સમાન સંભાવના ઘટનાઓ સાથે થાય છે \\
\end{longtable}
}

\begin{itemize}
\tightlist
\item
  \textbf{માહિતી સામગ્રી}: I = log_{2}(1/P) બિટ્સ
\item
  \textbf{એન્ટ્રોપી ફોર્મ્યુલા}: H = -Σ P(x) log_{2} P(x)
\item
  \textbf{ચેનલ ડિઝાઇન}: કોમ્યુનિકેશન સિસ્ટમ્સ ઑપ્ટિમાઇઝ કરવામાં મદદ કરે છે
\item
  \textbf{કોડિંગ કાર્યક્ષમતા}: સોર્સ કોડિંગ ડિઝાઇનને માર્ગદર્શન આપે છે
\end{itemize}

\end{solutionbox}
\begin{mnemonicbox}
``પ્રોબેબિલિટી એન્ટ્રોપી ઇન્ફોર્મેશન - PEI કોમ્યુનિકેશન''

\end{mnemonicbox}
\subsection*{પ્રશ્ન 3(બ OR) [4
ગુણ]}\label{uxaaauxab0uxab6uxaa8-3uxaac-or-4-uxa97uxaa3}

\textbf{સિમ્પ્લેક્સ, હાફ ડુપ્લેક્સ અને ફુલ ડુપ્લેક્સ ડેટા ટ્રાન્સમિશન મોડ સમજાવો}

\begin{solutionbox}

\textbf{ટ્રાન્સમિશન મોડ્સ ટેબલ:}

{\def\LTcaptype{none} % do not increment counter
\begin{longtable}[]{@{}llll@{}}
\toprule\noalign{}
મોડ & દિશા & ઉદાહરણ & ડાયાગ્રામ \\
\midrule\noalign{}
\endhead
\bottomrule\noalign{}
\endlastfoot
\textbf{સિમ્પ્લેક્સ} & માત્ર એક દિશા & રેડિયો બ્રોડકાસ્ટ & A \rightarrow B \\
\textbf{હાફ ડુપ્લેક્સ} & બંને દિશા, એકસાથે નહીં & વોકી-ટોકી & A ⇄ B \\
\textbf{ફુલ ડુપ્લેક્સ} & બંને દિશા, એકસાથે & ટેલિફોન & A ⇌ B \\
\end{longtable}
}

\begin{itemize}
\tightlist
\item
  \textbf{સિમ્પ્લેક્સ}: એકદિશીય કોમ્યુનિકેશન
\item
  \textbf{હાફ ડુપ્લેક્સ}: દ્વિદિશીય પરંતુ વૈકલ્પિક
\item
  \textbf{ફુલ ડુપ્લેક્સ}: એકસાથે દ્વિદિશીય
\item
  \textbf{બેન્ડવિડ્થ આવશ્યકતા}: ફુલ ડુપ્લેક્સને બમણી બેન્ડવિડ્થ જોઇએ
\end{itemize}

\end{solutionbox}
\begin{mnemonicbox}
``સિમ્પલ હાફ ફુલ - SHF ટ્રાન્સમિશન મોડ્સ''

\end{mnemonicbox}
\subsection*{પ્રશ્ન 3(ક OR) [7
ગુણ]}\label{uxaaauxab0uxab6uxaa8-3uxa95-or-7-uxa97uxaa3}

\textbf{યોગ્ય ઉદાહરણની મદદથી શેનોન ફાડો કોડિંગ સમજાવો}

\begin{solutionbox}

\textbf{ઉદાહરણ: અક્ષરો A, B, C, D સંભાવનાઓ 0.4, 0.3, 0.2, 0.1 સાથે}

\textbf{શેનોન-ફાડો અલ્ગોરિધમ સ્ટેપ્સ:}

\begin{verbatim}
સ્ટેપ 1: ઘટતા ક્રમમાં ગોઠવો
A: 0.4, B: 0.3, C: 0.2, D: 0.1

સ્ટેપ 2: બે ગ્રુપમાં વિભાજિત કરો
ગ્રુપ 1: A(0.4)  કોડ 0 થી શરૂ થાય છે
ગ્રુપ 2: B(0.3), C(0.2), D(0.1)  કોડ 1 થી શરૂ થાય છે

સ્ટેપ 3: ગ્રુપ 2નું પેટાવિભાજન
B(0.3)  કોડ: 10
C(0.2), D(0.1)  કોડ 11 થી શરૂ થાય છે

સ્ટેપ 4: અંતિમ પેટાવિભાજન
C(0.2)  કોડ: 110
D(0.1)  કોડ: 111
\end{verbatim}

\textbf{શેનોન-ફાડો કોડ્સ ટેબલ:}

{\def\LTcaptype{none} % do not increment counter
\begin{longtable}[]{@{}lll@{}}
\toprule\noalign{}
અક્ષર & સંભાવના & કોડ \\
\midrule\noalign{}
\endhead
\bottomrule\noalign{}
\endlastfoot
A & 0.4 & 0 \\
B & 0.3 & 10 \\
C & 0.2 & 110 \\
D & 0.1 & 111 \\
\end{longtable}
}

\begin{itemize}
\tightlist
\item
  \textbf{એવરેજ લેન્થ}: હફમેન સમાન (1.9 બિટ્સ)
\item
  \textbf{ટોપ-ડાઉન એપ્રોચ}: રૂટથી પાંદડાઓ સુધી વિભાજિત કરે છે
\item
  \textbf{હંમેશા ઑપ્ટિમલ નથી}: હફમેન સામાન્ય રીતે વધુ સારું છે
\end{itemize}

\end{solutionbox}
\begin{mnemonicbox}
``શેનોન ફાડો ટોપ-ડાઉન - SFTD કોડિંગ''

\end{mnemonicbox}
\subsection*{પ્રશ્ન 4(અ) [3
ગુણ]}\label{uxaaauxab0uxab6uxaa8-4uxa85-3-uxa97uxaa3}

\textbf{ડેટા કોમ્યુનિકેશનમાં નૈતિક અને ગોપનીયતાની બાબતોનું વર્ણન કરો}

\begin{solutionbox}

\textbf{નીતિશાસ્ત્ર અને ગોપનીયતા ટેબલ:}

{\def\LTcaptype{none} % do not increment counter
\begin{longtable}[]{@{}ll@{}}
\toprule\noalign{}
પાસા & વિચારણા \\
\midrule\noalign{}
\endhead
\bottomrule\noalign{}
\endlastfoot
\textbf{ડેટા ગોપનીયતા} & વપરાશકર્તાની સંમતિ, ડેટા સુરક્ષા \\
\textbf{સિક્યુરિટી} & એન્ક્રિપ્શન, એક્સેસ કંટ્રોલ \\
\textbf{પારદર્શિતા} & સ્પષ્ટ ડેટા વપરાશ નીતિઓ \\
\end{longtable}
}

\begin{itemize}
\tightlist
\item
  \textbf{ગોપનીયતાના અધિકારો}: વ્યક્તિગત ડેટા પર વપરાશકર્તાનું નિયંત્રણ
\item
  \textbf{નૈતિક ઉપયોગ}: જવાબદાર ડેટા હેન્ડલિંગ પ્રથાઓ
\item
  \textbf{કાનૂની પાલન}: ડેટા સુરક્ષા કાયદાઓનું પાલન કરવું
\item
  \textbf{સિક્યુરિટી પગલાં}: અનધિકૃત પ્રવેશ સામે સુરક્ષા
\end{itemize}

\end{solutionbox}
\begin{mnemonicbox}
``ગોપનીયતા સિક્યુરિટી પારદર્શિતા - PST નીતિશાસ્ત્ર''

\end{mnemonicbox}
\subsection*{પ્રશ્ન 4(બ) [4
ગુણ]}\label{uxaaauxab0uxab6uxaa8-4uxaac-4-uxa97uxaa3}

\textbf{RS 232 સ્ટાન્ડર્ડને પિન ડાયાગ્રામ સાથે સમજાવો}

\begin{solutionbox}

\textbf{RS-232 પિન કન્ફિગરેશન (DB-9):}

{\def\LTcaptype{none} % do not increment counter
\begin{longtable}[]{@{}lll@{}}
\toprule\noalign{}
પિન & સિગ્નલ & કાર્ય \\
\midrule\noalign{}
\endhead
\bottomrule\noalign{}
\endlastfoot
1 & DCD & ડેટા કેરિયર ડિટેક્ટ \\
2 & RXD & રિસીવ ડેટા \\
3 & TXD & ટ્રાન્સમિટ ડેટા \\
4 & DTR & ડેટા ટર્મિનલ રેડી \\
5 & GND & ગ્રાઉન્ડ \\
6 & DSR & ડેટા સેટ રેડી \\
7 & RTS & રિક્વેસ્ટ ટુ સેન્ડ \\
8 & CTS & ક્લિયર ટુ સેન્ડ \\
9 & RI & રિંગ ઇન્ડિકેટર \\
\end{longtable}
}

\begin{itemize}
\tightlist
\item
  \textbf{વોલ્ટેજ લેવલ્સ}: `0' માટે +3V થી +25V, `1' માટે -3V થી -25V
\item
  \textbf{મહત્તમ અંતર}: 19.2 kbps પર 50 ફુટ
\item
  \textbf{ઉપયોગ}: કમ્પ્યુટર અને મોડેમ વચ્ચે સીરિયલ કોમ્યુનિકેશન
\end{itemize}

\end{solutionbox}
\begin{mnemonicbox}
``RS-232 નવ પિન્સ સીરિયલ - RNS કોમ્યુનિકેશન''

\end{mnemonicbox}
\subsection*{પ્રશ્ન 4(ક) [7
ગુણ]}\label{uxaaauxab0uxab6uxaa8-4uxa95-7-uxa97uxaa3}

\textbf{યોગ્ય ઉદાહરણની મદદથી હેમિંગ કોડ સમજાવો}

\begin{solutionbox}

\textbf{ઉદાહરણ: 4-બિટ ડેટા 1011}

\textbf{હેમિંગ કોડ કન્સ્ટ્રક્શન:}

{\def\LTcaptype{none} % do not increment counter
\begin{longtable}[]{@{}llllllll@{}}
\toprule\noalign{}
સ્થિતિ & 1 & 2 & 3 & 4 & 5 & 6 & 7 \\
\midrule\noalign{}
\endhead
\bottomrule\noalign{}
\endlastfoot
\textbf{પ્રકાર} & P1 & P2 & D1 & P4 & D2 & D3 & D4 \\
\textbf{વેલ્યુ} & ? & ? & 1 & ? & 0 & 1 & 1 \\
\end{longtable}
}

\textbf{પેરિટી કેલ્ક્યુલેશન્સ:}

\begin{itemize}
\tightlist
\item
  \textbf{P1} (સ્થિતિઓ 1,3,5,7): P1 \oplus 1 \oplus 0 \oplus 1 = 0, તેથી P1 = 0
\item
  \textbf{P2} (સ્થિતિઓ 2,3,6,7): P2 \oplus 1 \oplus 1 \oplus 1 = 1, તેથી P2 = 1\\
\item
  \textbf{P4} (સ્થિતિઓ 4,5,6,7): P4 \oplus 0 \oplus 1 \oplus 1 = 0, તેથી P4 = 0
\end{itemize}

\textbf{અંતિમ હેમિંગ કોડ: 0110111}

\textbf{એરર ડિટેક્શન પ્રોસેસ:}

\begin{itemize}
\item
  સિન્ડ્રોમ S = S4S2S1 કેલ્ક્યુલેટ કરો
\item
  જો S = 000, કોઇ એરર નથી
\item
  જો S \neq 000, S દ્વારા દર્શાવેલ સ્થિતિએ એરર છે
\item
  \textbf{સિંગલ એરર કરેક્શન}: એક-બિટ એરર સુધારી શકે છે
\item
  \textbf{ડબલ એરર ડિટેક્શન}: બે-બિટ એરર શોધી શકે છે
\item
  \textbf{સિસ્ટેમેટિક એપ્રોચ}: વ્યવસ્થિત પેરિટી બિટ પ્લેસમેન્ટ
\end{itemize}

\end{solutionbox}
\begin{mnemonicbox}
``હેમિંગ સિંગલ એરર કરેક્શન - HSEC''

\end{mnemonicbox}
\subsection*{પ્રશ્ન 4(અ OR) [3
ગુણ]}\label{uxaaauxab0uxab6uxaa8-4uxa85-or-3-uxa97uxaa3}

\textbf{એજ કમ્પ્યુટિંગને વ્યાખ્યાયિત કરો અને તેની વિશેષતા સમજાવો}

\begin{solutionbox}

\textbf{એજ કમ્પ્યુટિંગ વિશેષતાઓ:}

{\def\LTcaptype{none} % do not increment counter
\begin{longtable}[]{@{}ll@{}}
\toprule\noalign{}
વિશેષતા & વર્ણન \\
\midrule\noalign{}
\endhead
\bottomrule\noalign{}
\endlastfoot
\textbf{લો લેટન્સી} & ડેટા સોર્સની નજીક પ્રોસેસિંગ \\
\textbf{બેન્ડવિડ્થ સેવિંગ} & નેટવર્ક ટ્રાફિક ઘટાડે છે \\
\textbf{રિયલ-ટાઇમ પ્રોસેસિંગ} & તાત્કાલિક ડેટા એનાલિસિસ \\
\end{longtable}
}

\begin{itemize}
\tightlist
\item
  \textbf{વ્યાખ્યા}: નેટવર્ક એજ પર, ડેટા સોર્સની નજીક કમ્પ્યુટિંગ
\item
  \textbf{ઘટાડેલી લેટન્સી}: ઝડપી રિસ્પોન્સ ટાઇમ
\item
  \textbf{ડિસ્ટ્રિબ્યુટેડ પ્રોસેસિંગ}: સેન્ટ્રલ સર્વર લોડ ઘટાડે છે
\item
  \textbf{ઉપયોગ}: IoT, ઓટોનોમસ વાહનો, સ્માર્ટ સિટીઓ
\end{itemize}

\end{solutionbox}
\begin{mnemonicbox}
``એજ લો-લેટન્સી રિયલ-ટાઇમ - ELR કમ્પ્યુટિંગ''

\end{mnemonicbox}
\subsection*{પ્રશ્ન 4(બ OR) [4
ગુણ]}\label{uxaaauxab0uxab6uxaa8-4uxaac-or-4-uxa97uxaa3}

\textbf{સંદેશાવ્યવહાર માટે મલ્ટીમીડિયા પ્રોસેસિંગની જરૂરિયાતો અને વિવિધ ડેટાના
વિવિધ ફાઇલ ફોર્મેટ સમજાવો}

\begin{solutionbox}

\textbf{મલ્ટીમીડિયા ફાઇલ ફોર્મેટ્સ ટેબલ:}

{\def\LTcaptype{none} % do not increment counter
\begin{longtable}[]{@{}lll@{}}
\toprule\noalign{}
ડેટા પ્રકાર & ફોર્મેટ્સ & લાક્ષણિકતાઓ \\
\midrule\noalign{}
\endhead
\bottomrule\noalign{}
\endlastfoot
\textbf{ઓડિયો} & MP3, WAV, AAC & કમ્પ્રેસ્ડ/અનકમ્પ્રેસ્ડ \\
\textbf{વિડિયો} & MP4, AVI, MOV & વિવિધ કોડેક્સ \\
\textbf{ઇમેજ} & JPEG, PNG, GIF & લોસી/લૉસલેસ કમ્પ્રેશન \\
\textbf{ટેક્સ્ટ} & TXT, PDF, DOC & વિવિધ એન્કોડિંગ્સ \\
\end{longtable}
}

\begin{itemize}
\tightlist
\item
  \textbf{પ્રોસેસિંગ જરૂરિયાતો}: કમ્પ્રેશન, ફોર્મેટ કન્વર્શન, ક્વોલિટી ઑપ્ટિમાઇઝેશન
\item
  \textbf{બેન્ડવિડ્થ ઑપ્ટિમાઇઝેશન}: ટ્રાન્સમિશન માટે ફાઇલ સાઇઝ ઘટાડવું
\item
  \textbf{ક્વોલિટી પ્રિઝર્વેશન}: સ્વીકાર્ય ક્વોલિટી લેવલ રાખવું
\item
  \textbf{કમ્પેટિબિલિટી}: મલ્ટિપલ ડિવાઇસ અને પ્લેટફોર્મ્સને સપોર્ટ કરવું
\end{itemize}

\end{solutionbox}
\begin{mnemonicbox}
``ઓડિયો વિડિયો ઇમેજ ટેક્સ્ટ - AVIT મલ્ટીમીડિયા''

\end{mnemonicbox}
\subsection*{પ્રશ્ન 4(ક OR) [7
ગુણ]}\label{uxaaauxab0uxab6uxaa8-4uxa95-or-7-uxa97uxaa3}

\textbf{વેવફોર્મની મદદથી વિવિધ લાઇન કોડિંગ સમજાવો}

\begin{solutionbox}

\textbf{ડેટા 1011 માટે લાઇન કોડિંગ વેવફોર્મ્સ:}

\begin{verbatim}
ડેટા:        1    0    1    1
            +{-{-}{-}{-}+    +{-}{-}{-}{-}+{-}{-}{-}{-}}
            |    |    |    |
            +    +{-{-}{-}{-}+    +}

NRZ{-L:      +{-}{-}{-}{-}+    +{-}{-}{-}{-}+{-}{-}{-}{-}}
            |    |    |    |
            +    +{-{-}{-}{-}+    +}

NRZ{-I:      +{-}{-}{-}{-}+{-}{-}{-}{-}+    +}
            |    |    |    |
            +    +    +{-{-}{-}{-}+{-}{-}{-}{-}}

RZ:         +{-{-}+ +    +{-}{-}+ +{-}{-}+}
            |  | |    |  | |  |
            +  +{-+{-}{-}{-}{-}+  +{-}+  +}

Manchester: +{-{-}+    {-}{-}+ +{-}{-}+    +}
            |  |   |  | |  |   |
            +  +{-{-}{-}+  +{-}+  +{-}{-}{-}+}
\end{verbatim}

\textbf{લાઇન કોડિંગ સરખામણી:}

{\def\LTcaptype{none} % do not increment counter
\begin{longtable}[]{@{}llll@{}}
\toprule\noalign{}
કોડ પ્રકાર & બેન્ડવિડ્થ & DC કોમ્પોનન્ટ & સિંક્રોનાઇઝેશન \\
\midrule\noalign{}
\endhead
\bottomrule\noalign{}
\endlastfoot
\textbf{NRZ-L} & લો & હાજર & ખરાબ \\
\textbf{NRZ-I} & લો & હાજર & ખરાબ \\
\textbf{RZ} & હાઇ & હાજર & સારું \\
\textbf{Manchester} & હાઇ & ગેરહાજર & ઉત્કૃષ્ટ \\
\end{longtable}
}

\begin{itemize}
\tightlist
\item
  \textbf{NRZ}: નોન-રિટર્ન-ટુ-ઝીરો, સિમ્પલ પરંતુ DC કોમ્પોનન્ટ છે
\item
  \textbf{RZ}: રિટર્ન-ટુ-ઝીરો, વધુ સારું સિંક્રોનાઇઝેશન
\item
  \textbf{Manchester}: સેલ્ફ-સિંક્રોનાઇઝિંગ, કોઇ DC કોમ્પોનન્ટ નથી
\item
  \textbf{સિલેક્શન ક્રાઇટેરિયા}: બેન્ડવિડ્થ, સિંક્રોનાઇઝેશન, જટિલતા
\end{itemize}

\end{solutionbox}
\begin{mnemonicbox}
``NRZ RZ Manchester - NRM લાઇન કોડ્સ''

\end{mnemonicbox}
\subsection*{પ્રશ્ન 5(અ) [3
ગુણ]}\label{uxaaauxab0uxab6uxaa8-5uxa85-3-uxa97uxaa3}

\textbf{સ્પ્રેડ સ્પેક્ટ્રમ ટેકનોલોજીનો ખ્યાલ સમજાવો}

\begin{solutionbox}

\textbf{સ્પ્રેડ સ્પેક્ટ્રમ લાક્ષણિકતાઓ:}

{\def\LTcaptype{none} % do not increment counter
\begin{longtable}[]{@{}ll@{}}
\toprule\noalign{}
પેરામીટર & વર્ણન \\
\midrule\noalign{}
\endhead
\bottomrule\noalign{}
\endlastfoot
\textbf{બેન્ડવિડ્થ સ્પ્રેડિંગ} & વાઇડ ફ્રીક્વન્સી પર સિગ્નલ સ્પ્રેડ \\
\textbf{લો પાવર ડેન્સિટી} & સ્પેક્ટ્રમમાં પાવર વિતરિત \\
\textbf{ઇન્ટરફેરન્સ રેઝિસ્ટન્સ} & જેમિંગ સામે પ્રતિરોધક \\
\end{longtable}
}

\begin{itemize}
\tightlist
\item
  \textbf{સિદ્ધાંત}: જરૂરી કરતાં વધુ વાઇડ બેન્ડવિડ્થ પર સિગ્નલ ફેલાવે છે
\item
  \textbf{તકનીકો}: ડાઇરેક્ટ સિક્વન્સ (DS-SS), ફ્રીક્વન્સી હોપિંગ (FH-SS)
\item
  \textbf{ફાયદાઓ}: સિક્યુરિટી, ઇન્ટરફેરન્સ પ્રતિરોધ, મલ્ટિપલ એક્સેસ
\item
  \textbf{ઉપયોગ}: GPS, CDMA, WiFi, Bluetooth
\end{itemize}

\end{solutionbox}
\begin{mnemonicbox}
``સ્પ્રેડ સ્પેક્ટ્રમ સિક્યુરિટી - SSS ટેકનોલોજી''

\end{mnemonicbox}
\subsection*{પ્રશ્ન 5(બ) [4
ગુણ]}\label{uxaaauxab0uxab6uxaa8-5uxaac-4-uxa97uxaa3}

\textbf{સેટેલાઇટ કોમ્યુનિકેશનના બ્લોક ડાયાગ્રામને સમજાવો}

\begin{solutionbox}

\begin{center}
\textbf{Mermaid Diagram (Code)}
\begin{verbatim}
{Shaded}
{Highlighting}[]
graph LR
    A[અર્થ સ્ટેશન 1] {-{-}{} B[અપલિંક]}
    B {-{-}{} C[સેટેલાઇટ ટ્રાન્સપોન્ડર]}
    C {-{-}{} D[ડાઉનલિંક]}
    D {-{-}{} E[અર્થ સ્ટેશન 2]}
    F[એન્ટેના] {-{-}{} C}
    C {-{-}{} G[એન્ટેના]}
{Highlighting}
{Shaded}
\end{verbatim}
\end{center}

\textbf{સેટેલાઇટ કોમ્યુનિકેશન કોમ્પોનન્ટ્સ:}

{\def\LTcaptype{none} % do not increment counter
\begin{longtable}[]{@{}ll@{}}
\toprule\noalign{}
કોમ્પોનન્ટ & કાર્ય \\
\midrule\noalign{}
\endhead
\bottomrule\noalign{}
\endlastfoot
\textbf{અર્થ સ્ટેશન} & ગ્રાઉન્ડ-બેસ્ડ ટ્રાન્સમિટ/રિસીવ \\
\textbf{અપલિંક} & પૃથ્વીથી સેટેલાઇટ ટ્રાન્સમિશન \\
\textbf{ટ્રાન્સપોન્ડર} & સેટેલાઇટ રિસીવર-ટ્રાન્સમિટર \\
\textbf{ડાઉનલિંક} & સેટેલાઇટથી પૃથ્વી ટ્રાન્સમિશન \\
\end{longtable}
}

\begin{itemize}
\tightlist
\item
  \textbf{ફ્રીક્વન્સી બેન્ડ્સ}: C-બેન્ડ, Ku-બેન્ડ, Ka-બેન્ડ
\item
  \textbf{કવરેજ એરિયા}: મોટા ભૌગોલિક કવરેજ
\item
  \textbf{ઉપયોગ}: બ્રોડકાસ્ટિંગ, ટેલિફોની, ઇન્ટરનેટ
\item
  \textbf{ફાયદાઓ}: વાઇડ કવરેજ, લાંબા-અંતરની કોમ્યુનિકેશન
\end{itemize}

\end{solutionbox}
\begin{mnemonicbox}
``અર્થ અપલિંક ટ્રાન્સપોન્ડર ડાઉનલિંક - EUTD સેટેલાઇટ''

\end{mnemonicbox}
\subsection*{પ્રશ્ન 5(ક) [7
ગુણ]}\label{uxaaauxab0uxab6uxaa8-5uxa95-7-uxa97uxaa3}

\textbf{મલ્ટીમીડિયા કોમ્યુનિકેશન્સનું મોડેલ અને મલ્ટીમીડિયા સિસ્ટમના તત્વોનું પ્રદર્શન
કરો}

\begin{solutionbox}

\textbf{મલ્ટીમીડિયા કોમ્યુનિકેશન મોડેલ:}

\begin{center}
\textbf{Mermaid Diagram (Code)}
\begin{verbatim}
{Shaded}
{Highlighting}[]
graph LR
    A[સ્રોત] {-{-}{} B[એન્કોડર]}
    B {-{-}{} C[મલ્ટિપ્લેક્સર]}
    C {-{-}{} D[નેટવર્ક]}
    D {-{-}{} E[ડીમલ્ટિપ્લેક્સર]}
    E {-{-}{} F[ડીકોડર]}
    F {-{-}{} G[ગંતવ્ય]}
    H[ઓડિયો] {-{-}{} B}
    I[વિડિયો] {-{-}{} B}
    J[ટેક્સ્ટ] {-{-}{} B}
    K[ગ્રાફિક્સ] {-{-}{} B}
{Highlighting}
{Shaded}
\end{verbatim}
\end{center}

\textbf{મલ્ટીમીડિયા સિસ્ટમ તત્વો:}

{\def\LTcaptype{none} % do not increment counter
\begin{longtable}[]{@{}lll@{}}
\toprule\noalign{}
તત્વ & કાર્ય & ઉદાહરણો \\
\midrule\noalign{}
\endhead
\bottomrule\noalign{}
\endlastfoot
\textbf{કેપ્ચર} & મલ્ટીમીડિયા ડેટા ઇનપુટ & કેમેરા, માઇક્રોફોન \\
\textbf{સ્ટોરેજ} & મલ્ટીમીડિયા ફાઇલ્સ સ્ટોર કરવું & હાર્ડ ડિસ્ક, મેમોરી \\
\textbf{પ્રોસેસિંગ} & એડિટ અને મેનિપ્યુલેટ કરવું & વિડિયો એડિટિંગ સોફ્ટવેર \\
\textbf{કોમ્યુનિકેશન} & મલ્ટીમીડિયા ટ્રાન્સમિટ કરવું & નેટવર્ક્સ, ઇન્ટરનેટ \\
\textbf{પ્રેઝન્ટેશન} & મલ્ટીમીડિયા ડિસ્પ્લે કરવું & મોનિટર, સ્પીકર્સ \\
\end{longtable}
}

\begin{itemize}
\tightlist
\item
  \textbf{સિંક્રોનાઇઝેશન}: ઓડિયો-વિડિયો સિંક્રોનાઇઝેશન મહત્વપૂર્ણ
\item
  \textbf{કમ્પ્રેશન}: બેન્ડવિડ્થ આવશ્યકતાઓ ઘટાડે છે
\item
  \textbf{ક્વોલિટી ઓફ સર્વિસ}: સ્વીકાર્ય ક્વોલિટી જાળવે છે
\item
  \textbf{રિયલ-ટાઇમ કન્સ્ટ્રેઇન્ટ્સ}: સમય-સંવેદનશીલ ડેટા ડિલિવરી
\end{itemize}

\end{solutionbox}
\begin{mnemonicbox}
``કેપ્ચર સ્ટોર પ્રોસેસ કોમ્યુનિકેટ પ્રેઝન્ટ - CSPCP
મલ્ટીમીડિયા''

\end{mnemonicbox}
\subsection*{પ્રશ્ન 5(અ OR) [3
ગુણ]}\label{uxaaauxab0uxab6uxaa8-5uxa85-or-3-uxa97uxaa3}

\textbf{કોમ્યુનિકેશન સિક્યુરિટીમાં બ્લોક ચેઇનનું મહત્વ સમજાવો}

\begin{solutionbox}

\textbf{બ્લોકચેઇન સિક્યુરિટી વિશેષતાઓ:}

{\def\LTcaptype{none} % do not increment counter
\begin{longtable}[]{@{}ll@{}}
\toprule\noalign{}
વિશેષતા & લાભ \\
\midrule\noalign{}
\endhead
\bottomrule\noalign{}
\endlastfoot
\textbf{ડીસેન્ટ્રલાઇઝેશન} & કોઇ સિંગલ પોઇન્ટ ઓફ ફેઇલ્યુર નથી \\
\textbf{ઇમ્યુટેબિલિટી} & ભૂતકાળના રેકોર્ડ્સ બદલી શકાતા નથી \\
\textbf{ટ્રાન્સપેરન્સી} & બધા ટ્રાન્ઝેક્શન્સ દૃશ્યમાન \\
\end{longtable}
}

\begin{itemize}
\tightlist
\item
  \textbf{ક્રિપ્ટોગ્રાફિક સિક્યુરિટી}: હેશ ફંક્શન્સ અને ડિજિટલ સિગ્નેચર્સ
\item
  \textbf{ડિસ્ટ્રિબ્યુટેડ લેજર}: બહુવિધ કોપીઓ ટેમ્પરિંગ અટકાવે છે
\item
  \textbf{સ્માર્ટ કોન્ટ્રેક્ટ્સ}: ઓટોમેટેડ સિક્યુરિટી પ્રોટોકોલ્સ
\item
  \textbf{ઉપયોગ}: સિક્યુર મેસેજિંગ, આઇડેન્ટિટી વેરિફિકેશન
\end{itemize}

\end{solutionbox}
\begin{mnemonicbox}
``બ્લોકચેઇન ડિસ્ટ્રિબ્યુટેડ ઇમ્યુટેબલ - BDI સિક્યુરિટી''

\end{mnemonicbox}
\subsection*{પ્રશ્ન 5(બ OR) [4
ગુણ]}\label{uxaaauxab0uxab6uxaa8-5uxaac-or-4-uxa97uxaa3}

\textbf{5G ટેકનોલોજીના મહત્વના તત્વો, વિશેષતાઓ અને ફાયદાઓ સમજાવો}

\begin{solutionbox}

\textbf{5G ટેકનોલોજી તત્વો:}

{\def\LTcaptype{none} % do not increment counter
\begin{longtable}[]{@{}ll@{}}
\toprule\noalign{}
તત્વ & સ્પેસિફિકેશન \\
\midrule\noalign{}
\endhead
\bottomrule\noalign{}
\endlastfoot
\textbf{સ્પીડ} & 10 Gbps સુધી \\
\textbf{લેટન્સી} & 1 ms કરતાં ઓછી \\
\textbf{કનેક્શન્સ} & 1 મિલિયન ડિવાઇસ પર km^{2} \\
\textbf{રિલાયબિલિટી} & 99.999\% ઉપલબ્ધતા \\
\end{longtable}
}

\textbf{મુખ્ય વિશેષતાઓ:}

\begin{itemize}
\tightlist
\item
  \textbf{એન્હાન્સ્ડ મોબાઇલ બ્રોડબેન્ડ}: અતિ-હાઇ-સ્પીડ ઇન્ટરનેટ
\item
  \textbf{અલ્ટ્રા-રિલાયબલ લો લેટન્સી}: ક્રિટિકલ એપ્લિકેશન્સ
\item
  \textbf{મેસિવ મશીન કોમ્યુનિકેશન}: IoT કનેક્ટિવિટી
\item
  \textbf{નેટવર્ક સ્લાઇસિંગ}: કસ્ટમાઇઝ્ડ નેટવર્ક સર્વિસીસ
\end{itemize}

\textbf{ફાયદાઓ:}

\begin{itemize}
\tightlist
\item
  \textbf{હાયર કેપેસિટી}: વધુ સિમલ્ટેનિયસ યુઝર્સ
\item
  \textbf{એનર્જી એફિશિયન્સી}: ડિવાઇસ માટે વધુ સારી બેટરી લાઇફ
\item
  \textbf{નવા એપ્લિકેશન્સ}: AR/VR, ઓટોનોમસ વાહનો
\end{itemize}

\end{solutionbox}
\begin{mnemonicbox}
``5G સ્પીડ લેટન્સી કનેક્શન્સ - SLC વિશેષતાઓ''

\end{mnemonicbox}
\subsection*{પ્રશ્ન 5(ક OR) [7
ગુણ]}\label{uxaaauxab0uxab6uxaa8-5uxa95-or-7-uxa97uxaa3}

\textbf{RS 232, RS 422 અને RS 485 સ્ટાન્ડર્ડની સરખામણી કરો}

\begin{solutionbox}

\textbf{RS સ્ટાન્ડર્ડ્સ સરખામણી ટેબલ:}

{\def\LTcaptype{none} % do not increment counter
\begin{longtable}[]{@{}llll@{}}
\toprule\noalign{}
પેરામીટર & RS-232 & RS-422 & RS-485 \\
\midrule\noalign{}
\endhead
\bottomrule\noalign{}
\endlastfoot
\textbf{મોડ} & સિંગલ-એન્ડેડ & ડિફરન્શિયલ & ડિફરન્શિયલ \\
\textbf{મહત્તમ અંતર} & 50 ફુટ & 4000 ફુટ & 4000 ફુટ \\
\textbf{મહત્તમ સ્પીડ} & 20 kbps & 10 Mbps & 10 Mbps \\
\textbf{ડ્રાઇવર્સ} & 1 & 1 & 32 \\
\textbf{રિસીવર્સ} & 1 & 10 & 32 \\
\textbf{ટોપોલોજી} & પોઇન્ટ-ટુ-પોઇન્ટ & પોઇન્ટ-ટુ-મલ્ટિપોઇન્ટ & મલ્ટિપોઇન્ટ \\
\end{longtable}
}

\textbf{વોલ્ટેજ લેવલ્સ:}

{\def\LTcaptype{none} % do not increment counter
\begin{longtable}[]{@{}lll@{}}
\toprule\noalign{}
સ્ટાન્ડર્ડ & લોજિક 1 & લોજિક 0 \\
\midrule\noalign{}
\endhead
\bottomrule\noalign{}
\endlastfoot
\textbf{RS-232} & -3V થી -25V & +3V થી +25V \\
\textbf{RS-422} & ડિફરન્શિયલ \textgreater{} +200mV & ડિફરન્શિયલ
\textless{} -200mV \\
\textbf{RS-485} & ડિફરન્શિયલ \textgreater{} +200mV & ડિફરન્શિયલ
\textless{} -200mV \\
\end{longtable}
}

\textbf{ઉપયોગ:}

\begin{itemize}
\tightlist
\item
  \textbf{RS-232}: કમ્પ્યુટર સીરિયલ પોર્ટ્સ, મોડેમ્સ
\item
  \textbf{RS-422}: ઇન્ડસ્ટ્રિયલ ઓટોમેશન, લાંબા-અંતર
\item
  \textbf{RS-485}: બિલ્ડિંગ ઓટોમેશન, ઇન્ડસ્ટ્રિયલ નેટવર્ક્સ
\end{itemize}

\textbf{મુખ્ય તફાવતો:}

\begin{itemize}
\tightlist
\item
  \textbf{નોઇઝ ઇમ્યુનિટી}: RS-422/485માં ડિફરન્શિયલ સિગ્નલિંગ RS-232 કરતાં વધુ
  સારું
\item
  \textbf{અંતર ક્ષમતા}: RS-422/485 RS-232 કરતાં ઘણું લાંબું
\item
  \textbf{મલ્ટિ-ડ્રોપ ક્ષમતા}: RS-485 બહુવિધ ડિવાઇસને સપોર્ટ કરે છે
\item
  \textbf{કોસ્ટ}: RS-232 સૌથી સસ્તું, RS-485 સૌથી જટિલ
\end{itemize}

\end{solutionbox}
\begin{mnemonicbox}
``RS-232 સિમ્પલ, RS-422 લાંબું, RS-485 મલ્ટિ - SLM
સ્ટાન્ડર્ડ્સ''

\end{mnemonicbox}

\end{document}
