\documentclass[10pt,a4paper]{article}

% content/resources/templates/preamble.tex
\usepackage[margin=0.6in]{geometry}
\author{Milav Dabgar}
\usepackage{amsmath,amssymb,amsthm}
\usepackage{booktabs}
\usepackage{multirow}
\usepackage{xcolor}
\usepackage{tcolorbox}
\tcbuselibrary{breakable,skins}
\usepackage[colorlinks=true,linkcolor=blue]{hyperref}
\usepackage{titlesec}
\usepackage{enumitem}
\usepackage{tikz}
\usepackage{pgfplots}
\usepackage{circuitikz}
\usepackage[version=4]{mhchem}
\usepackage{longtable}
\usepackage{array}
\usepackage{float}
\usepackage{caption}
\usepackage{listings}

\lstset{
  basicstyle=\small\ttfamily,
  breaklines=true,
  breakatwhitespace=false,
  postbreak=\mbox{\textcolor{red}{$\hookrightarrow$}\space},
  float=false,
  numbers=left,
  numberstyle=\tiny\color{gray},
  numbersep=10pt,
  xleftmargin=2em,
  keywordstyle=\color{blue},
  commentstyle=\color{green!60!black},
  stringstyle=\color{purple},
  backgroundcolor=\color{gray!5},
  showstringspaces=false,
  tabsize=2,
  captionpos=b,
  keepspaces=true,
  columns=flexible
}

\pgfplotsset{compat=1.18}
\usetikzlibrary{shapes,arrows,positioning,calc,patterns,decorations.pathmorphing,decorations.markings,arrows.meta}

% Color scheme
\definecolor{headcolor}{RGB}{0,102,204}
\definecolor{keycolor}{RGB}{220,20,60}
\definecolor{solutioncolor}{RGB}{34,139,34}
\definecolor{mnemoniccolor}{RGB}{148,0,211}
\definecolor{codecolor}{RGB}{0,0,100}

% Spacing
\setlength{\parskip}{3pt}
\setlist[itemize]{nosep}
\setlist[enumerate]{nosep}

% Title formatting
\titleformat{\section}{\Large\bfseries\color{headcolor}}{\thesection}{1em}{}
\titleformat{\subsection}{\large\bfseries\color{headcolor}}{\thesubsection}{1em}{}

% Pandoc tightlist compatibility
\providecommand{\tightlist}{%
  \setlength{\itemsep}{0pt}\setlength{\parskip}{0pt}}

% Pandoc longtable compatibility
\newcounter{none}
\def\thenone{}


% content/resources/templates/english-boxes.tex
% This file is currently empty - it exists to maintain consistency with the import structure.
% Add custom environments here if needed in the future.


\begin{document}

\begin{center}
{\Huge\bfseries\color{headcolor} Subject Name Solutions}\\[5pt]
{\LARGE 4343201 -- Winter 2024}\\[3pt]
{\large Semester 1 Study Material}\\[3pt]
{\normalsize\textit{Detailed Solutions and Explanations}}
\end{center}

\vspace{10pt}

\subsection*{Question 1(a) [3 marks]}\label{q1a}

\textbf{Differentiate Basic modes of Communication: Broad casting
communication and Point to Point Communication.}

\begin{solutionbox}

{\def\LTcaptype{none} % do not increment counter
\begin{longtable}[]{@{}
  >{\raggedright\arraybackslash}p{(\linewidth - 4\tabcolsep) * \real{0.1594}}
  >{\raggedright\arraybackslash}p{(\linewidth - 4\tabcolsep) * \real{0.4058}}
  >{\raggedright\arraybackslash}p{(\linewidth - 4\tabcolsep) * \real{0.4348}}@{}}
\toprule\noalign{}
\begin{minipage}[b]{\linewidth}\raggedright
Parameter
\end{minipage} & \begin{minipage}[b]{\linewidth}\raggedright
Broadcasting Communication
\end{minipage} & \begin{minipage}[b]{\linewidth}\raggedright
Point to Point Communication
\end{minipage} \\
\midrule\noalign{}
\endhead
\bottomrule\noalign{}
\endlastfoot
\textbf{Definition} & One transmitter sends signals to multiple
receivers simultaneously & One transmitter communicates with one
specific receiver \\
\textbf{Direction} & Unidirectional (one-way) & Bidirectional
(two-way) \\
\textbf{Examples} & TV, Radio, FM & Telephone, Mobile calls, Private
networks \\
\textbf{Privacy} & Low (signal available to everyone in range) & High
(dedicated connection between endpoints) \\
\textbf{Efficiency} & High for mass communication & Better for
personal/private communication \\
\end{longtable}
}

\end{solutionbox}
\begin{mnemonicbox}
``BDPEC'' - Broadcasting Distributes to Public,
Endpoints Connect in point-to-point

\end{mnemonicbox}
\subsection*{Question 1(b) [4 marks]}\label{q1b}

\textbf{Define: Bit Rate, Baud Rate, Bandwidth and Repeater Distance.}

\begin{solutionbox}

{\def\LTcaptype{none} % do not increment counter
\begin{longtable}[]{@{}
  >{\raggedright\arraybackslash}p{(\linewidth - 2\tabcolsep) * \real{0.3333}}
  >{\raggedright\arraybackslash}p{(\linewidth - 2\tabcolsep) * \real{0.6667}}@{}}
\toprule\noalign{}
\begin{minipage}[b]{\linewidth}\raggedright
Term
\end{minipage} & \begin{minipage}[b]{\linewidth}\raggedright
Definition
\end{minipage} \\
\midrule\noalign{}
\endhead
\bottomrule\noalign{}
\endlastfoot
\textbf{Bit Rate} & Number of binary bits transmitted per second (bps).
Measures actual data transfer speed. \\
\textbf{Baud Rate} & Number of signal units or symbols transmitted per
second. One symbol may contain multiple bits. \\
\textbf{Bandwidth} & Range of frequencies used by a signal, measured in
Hertz (Hz). Determines maximum data capacity of a channel. \\
\textbf{Repeater Distance} & Maximum distance between repeaters in a
communication system before signal degradation requires regeneration. \\
\end{longtable}
}

\textbf{Diagram:}

\includegraphics[width=1\linewidth,height=\textheight,keepaspectratio]{mermaid-6667bdaf.pdf}

\end{solutionbox}
\begin{mnemonicbox}
``BBRR'' - ``Better Bandwidth Requires Repeaters''

\end{mnemonicbox}
\subsection*{Question 1(c) [7 marks]}\label{q1c}

\textbf{Draw the block diagram of digital communication system. Explain
the functions of each block in brief. State advantages and disadvantages
of it.}

\begin{solutionbox}

\textbf{Block Diagram:}

\includegraphics[width=1\linewidth,height=\textheight,keepaspectratio]{mermaid-1b9809a5.pdf}

\textbf{Functions:}

{\def\LTcaptype{none} % do not increment counter
\begin{longtable}[]{@{}
  >{\raggedright\arraybackslash}p{(\linewidth - 2\tabcolsep) * \real{0.4118}}
  >{\raggedright\arraybackslash}p{(\linewidth - 2\tabcolsep) * \real{0.5882}}@{}}
\toprule\noalign{}
\begin{minipage}[b]{\linewidth}\raggedright
Block
\end{minipage} & \begin{minipage}[b]{\linewidth}\raggedright
Function
\end{minipage} \\
\midrule\noalign{}
\endhead
\bottomrule\noalign{}
\endlastfoot
\textbf{Source Encoder} & Converts analog signal to digital, removes
redundancy, compresses data \\
\textbf{Channel Encoder} & Adds redundancy for error detection and
correction \\
\textbf{Digital Modulator} & Converts digital data to suitable form for
transmission (ASK, FSK, PSK, etc.) \\
\textbf{Channel} & Medium through which signal travels
(wired/wireless) \\
\textbf{Digital Demodulator} & Extracts original digital data from
received modulated signal \\
\textbf{Channel Decoder} & Detects and corrects errors using added
redundancy \\
\textbf{Source Decoder} & Decompresses data and converts to original
form \\
\end{longtable}
}

\textbf{Advantages and Disadvantages:}

{\def\LTcaptype{none} % do not increment counter
\begin{longtable}[]{@{}
  >{\raggedright\arraybackslash}p{(\linewidth - 2\tabcolsep) * \real{0.4444}}
  >{\raggedright\arraybackslash}p{(\linewidth - 2\tabcolsep) * \real{0.5556}}@{}}
\toprule\noalign{}
\begin{minipage}[b]{\linewidth}\raggedright
Advantages
\end{minipage} & \begin{minipage}[b]{\linewidth}\raggedright
Disadvantages
\end{minipage} \\
\midrule\noalign{}
\endhead
\bottomrule\noalign{}
\endlastfoot
Better noise immunity & Requires more bandwidth \\
Easier signal regeneration & Complex implementation \\
Secure transmission possible & Synchronization required \\
Integration with computers & Quantization errors \\
Better quality for long distance & Higher cost for simple
applications \\
\end{longtable}
}

\end{solutionbox}
\begin{mnemonicbox}
``SECDCSO'' - ``Secure Encoding Creates Digital
Communication System Output''

\end{mnemonicbox}
\subsection*{Question 1(c) OR [7
marks]}\label{q1c}

\textbf{Justify the needs of multiplexing techniques for digital
communication. Draw and explain Time Division multiplexing technique in
brief. Discuss its merits and demerits.}

\begin{solutionbox}

\textbf{Need for Multiplexing:}

{\def\LTcaptype{none} % do not increment counter
\begin{longtable}[]{@{}
  >{\raggedright\arraybackslash}p{(\linewidth - 2\tabcolsep) * \real{0.3158}}
  >{\raggedright\arraybackslash}p{(\linewidth - 2\tabcolsep) * \real{0.6842}}@{}}
\toprule\noalign{}
\begin{minipage}[b]{\linewidth}\raggedright
Need
\end{minipage} & \begin{minipage}[b]{\linewidth}\raggedright
Explanation
\end{minipage} \\
\midrule\noalign{}
\endhead
\bottomrule\noalign{}
\endlastfoot
\textbf{Channel Efficiency} & Allows multiple signals on one channel,
saving bandwidth \\
\textbf{Cost Reduction} & Reduces need for multiple transmission
media \\
\textbf{Infrastructure Utilization} & Maximizes use of expensive
infrastructure \\
\textbf{Spectrum Conservation} & Conserves limited frequency spectrum \\
\end{longtable}
}

\textbf{Time Division Multiplexing (TDM):}

\includegraphics[width=1\linewidth,height=\textheight,keepaspectratio]{mermaid-b39917e1.pdf}

\textbf{Working:} In TDM, each input signal gets a specific time slot.
The multiplexer samples each input sequentially, combining them into a
single high-speed data stream. At the receiver, the demultiplexer
separates the stream back into original signals based on timing.

\textbf{Merits and Demerits:}

{\def\LTcaptype{none} % do not increment counter
\begin{longtable}[]{@{}
  >{\raggedright\arraybackslash}p{(\linewidth - 2\tabcolsep) * \real{0.4444}}
  >{\raggedright\arraybackslash}p{(\linewidth - 2\tabcolsep) * \real{0.5556}}@{}}
\toprule\noalign{}
\begin{minipage}[b]{\linewidth}\raggedright
Merits
\end{minipage} & \begin{minipage}[b]{\linewidth}\raggedright
Demerits
\end{minipage} \\
\midrule\noalign{}
\endhead
\bottomrule\noalign{}
\endlastfoot
\textbf{Efficient bandwidth usage} & \textbf{Requires
synchronization} \\
\textbf{No guard bands needed} & \textbf{Complex buffering required} \\
\textbf{No cross-talk} & \textbf{Timing issues can cause errors} \\
\textbf{Flexible allocation} & \textbf{Unused slots waste capacity} \\
\textbf{Digital implementation} & \textbf{Higher data rate than
individual channels} \\
\end{longtable}
}

\end{solutionbox}
\begin{mnemonicbox}
``TIME'' - ``Transmission Interleaves Multiple
Endpoints''

\end{mnemonicbox}
\subsection*{Question 2(a) [3 marks]}\label{q2a}

\textbf{Differentiate: Coherent and Non-Coherent Detection Technique.}

\begin{solutionbox}

{\def\LTcaptype{none} % do not increment counter
\begin{longtable}[]{@{}
  >{\raggedright\arraybackslash}p{(\linewidth - 4\tabcolsep) * \real{0.2037}}
  >{\raggedright\arraybackslash}p{(\linewidth - 4\tabcolsep) * \real{0.3519}}
  >{\raggedright\arraybackslash}p{(\linewidth - 4\tabcolsep) * \real{0.4444}}@{}}
\toprule\noalign{}
\begin{minipage}[b]{\linewidth}\raggedright
Parameter
\end{minipage} & \begin{minipage}[b]{\linewidth}\raggedright
Coherent Detection
\end{minipage} & \begin{minipage}[b]{\linewidth}\raggedright
Non-Coherent Detection
\end{minipage} \\
\midrule\noalign{}
\endhead
\bottomrule\noalign{}
\endlastfoot
\textbf{Phase Information} & Uses phase information & Ignores phase
information \\
\textbf{Local Oscillator} & Required & Not required \\
\textbf{Complexity} & More complex & Simpler \\
\textbf{Performance} & Better noise immunity & Less efficient in
noise \\
\textbf{Implementation} & Difficult & Easier \\
\textbf{Applications} & High-quality systems & Low-cost systems \\
\end{longtable}
}

\end{solutionbox}
\begin{mnemonicbox}
``PLCPIA'' - ``Phase Local Complex Performance
Implementation Applications''

\end{mnemonicbox}
\subsection*{Question 2(b) [4 marks]}\label{q2b}

\textbf{Sketch the ASK, FSK, PSK and QPSK waveform for the data sequence
101100110110.}

\begin{solutionbox}

\begin{lstlisting}
Input Data:  1  0  1  1  0  0  1  1  0  1  1  0
            ▄▄    ▄▄▄▄       ▄▄▄▄    ▄▄▄▄    
            │ │   │  │       │  │    │  │    
Data:       │ │   │  │       │  │    │  │    
            │ └───┘  └───────┘  └────┘  └────
            
            ▄▄    ▄▄▄▄       ▄▄▄▄    ▄▄▄▄    
            │ │   │  │       │  │    │  │    
ASK:        │ │   │  │       │  │    │  │    
            └─┴───┴──┴───────┴──┴────┴──┴────
            
            ▄▄▄▄  ████▄▄▄▄▄  ████▄▄  ████▄▄  
FSK High:   │  │  │  │    │  │  │ │  │  │ │  
FSK Low:   ─┘  └──┘  └────┘──┘  └─┘──┘  └─┘──
            
            ▄▄    ▄▄▄▄       ▄▄▄▄    ▄▄▄▄    
            │ │   │  │       │  │    │  │    
PSK 0^\circ:     │ │   │  │       │  │    │  │    
            │ └───┘  └───────┘  └────┘  └────
PSK 180^\circ:  ─┘     ▄▄     ▄▄▄▄▄▄     ▄▄     ▄▄
                  │ │    │    │     │ │    │ 
                  │ │    │    │     │ │    │ 
                  └─┘    └────┘     └─┘    └─

QPSK:     ┌─┐   ┌─┐ ┌─┐   ┌─┐ ┌─┐   ┌─┐ ┌─┐ 
90^\circ 00:  _│ │___│ │_│ │___│ │_│ │___│ │_│ │__
180^\circ 10: _┘ └───┘ └─┘ └───┘ └─┘ └───┘ └─┘ └__
270^\circ 11: ───┐ ┌───┐   ┌───┐   ┌───┐   ┌──────
0^\circ 01:   ───┘ └───┘   └───┘   └───┘   └──────
\end{lstlisting}

\end{solutionbox}
\begin{mnemonicbox}
``AFPQ'' - ``Amplitude Frequency Phase Quadrature''

\end{mnemonicbox}
\subsection*{Question 2(c) [7 marks]}\label{q2c}

\textbf{Explain the principle of 16-QAM. Also explain constellation
diagram and waveform for 16-QAM. Write its advantages and
disadvantages.}

\begin{solutionbox}

\textbf{Principle of 16-QAM:} 16-QAM (Quadrature Amplitude Modulation)
combines amplitude and phase modulation to transmit 4 bits per symbol.
It uses 16 different combinations of amplitude and phase, allowing
higher data rates in the same bandwidth.

\textbf{Constellation Diagram:}

\begin{lstlisting}
           Q
           ▲
           │
   ●   ●   │   ●   ●
           │
   ●   ●   │   ●   ●
-----------+-----------> I
   ●   ●   │   ●   ●
           │
   ●   ●   │   ●   ●
           │
           
Each point represents 4 bits (0000 to 1111)
\end{lstlisting}

\textbf{Waveform:} The 16-QAM waveform varies in both amplitude (4
levels) and phase (4 phases), creating 16 unique symbols.

\textbf{Advantages and Disadvantages:}

{\def\LTcaptype{none} % do not increment counter
\begin{longtable}[]{@{}
  >{\raggedright\arraybackslash}p{(\linewidth - 2\tabcolsep) * \real{0.4444}}
  >{\raggedright\arraybackslash}p{(\linewidth - 2\tabcolsep) * \real{0.5556}}@{}}
\toprule\noalign{}
\begin{minipage}[b]{\linewidth}\raggedright
Advantages
\end{minipage} & \begin{minipage}[b]{\linewidth}\raggedright
Disadvantages
\end{minipage} \\
\midrule\noalign{}
\endhead
\bottomrule\noalign{}
\endlastfoot
\textbf{High spectral efficiency} & \textbf{Sensitive to noise and
interference} \\
\textbf{Higher data rate} & \textbf{Requires higher SNR} \\
\textbf{Bandwidth efficient} & \textbf{Complex implementation} \\
\textbf{Better use of channel capacity} & \textbf{Susceptible to
amplitude distortion} \\
\end{longtable}
}

\end{solutionbox}
\begin{mnemonicbox}
``SCHAP'' - ``Sixteen Combinations Have Amplitude and
Phase''

\end{mnemonicbox}
\subsection*{Question 2(a) OR [3
marks]}\label{q2a}

\textbf{Compare: ASK and PSK}

\begin{solutionbox}

{\def\LTcaptype{none} % do not increment counter
\begin{longtable}[]{@{}lll@{}}
\toprule\noalign{}
Parameter & ASK (Amplitude Shift Keying) & PSK (Phase Shift Keying) \\
\midrule\noalign{}
\endhead
\bottomrule\noalign{}
\endlastfoot
\textbf{Modulation Parameter} & Amplitude & Phase \\
\textbf{Noise Immunity} & Poor & Good \\
\textbf{Power Efficiency} & Less efficient & More efficient \\
\textbf{Bandwidth Efficiency} & Lower & Higher \\
\textbf{Implementation} & Simple & More complex \\
\textbf{BER Performance} & Higher error rate & Lower error rate \\
\end{longtable}
}

\end{solutionbox}
\begin{mnemonicbox}
``ANPBIP'' - ``Amplitude Noise Power Bandwidth
Implementation Performance''

\end{mnemonicbox}
\subsection*{Question 2(b) OR [4
marks]}\label{q2b}

\textbf{Draw the block diagram of BPSK modulator and demodulator.}

\begin{solutionbox}

\textbf{BPSK Modulator:}

\includegraphics[width=1\linewidth,height=\textheight,keepaspectratio]{mermaid-1c573f19.pdf}

\textbf{BPSK Demodulator:}

\includegraphics[width=1\linewidth,height=\textheight,keepaspectratio]{mermaid-932f387a.pdf}

\end{solutionbox}
\begin{mnemonicbox}
``MNECO'' - ``Modulation Needs Encoding, Carriers,
Oscillators''

\end{mnemonicbox}
\subsection*{Question 2(c) OR [7
marks]}\label{q2c}

\textbf{Explain QPSK generation and detection with the help of block
diagram and waveform. Discuss its advantages and disadvantages.}

\begin{solutionbox}

\textbf{QPSK Generation Block Diagram:}

\includegraphics[width=1\linewidth,height=\textheight,keepaspectratio]{mermaid-24e5aaef.pdf}

\textbf{QPSK Detection Block Diagram:}

\includegraphics[width=1\linewidth,height=\textheight,keepaspectratio]{mermaid-da06a40b.pdf}

\textbf{QPSK Waveform:} Each symbol in QPSK represents 2 bits, with 4
possible phase states (0^\circ, 90^\circ, 180^\circ, 270^\circ).

\textbf{Advantages and Disadvantages:}

{\def\LTcaptype{none} % do not increment counter
\begin{longtable}[]{@{}ll@{}}
\toprule\noalign{}
Advantages & Disadvantages \\
\midrule\noalign{}
\endhead
\bottomrule\noalign{}
\endlastfoot
\textbf{Twice the data rate of BPSK} & \textbf{More complex
implementation} \\
\textbf{Same bandwidth as BPSK} & \textbf{Sensitive to phase errors} \\
\textbf{Good noise immunity} & \textbf{Requires carrier recovery} \\
\textbf{Spectral efficiency} & \textbf{More complex synchronization} \\
\end{longtable}
}

\end{solutionbox}
\begin{mnemonicbox}
``PACE'' - ``Phase Alteration Carries Extra data''

\end{mnemonicbox}
\subsection*{Question 3(a) [3 marks]}\label{q3a}

\textbf{State the features of RS-422.}

\begin{solutionbox}

{\def\LTcaptype{none} % do not increment counter
\begin{longtable}[]{@{}l@{}}
\toprule\noalign{}
Features of RS-422 \\
\midrule\noalign{}
\endhead
\bottomrule\noalign{}
\endlastfoot
\textbf{Differential signaling} for noise immunity \\
\textbf{Maximum data rate} of 10 Mbps \\
\textbf{Maximum cable length} of 1200 meters \\
\textbf{Multi-drop capability} (1 driver, up to 10 receivers) \\
\textbf{Balanced transmission line} \\
\textbf{Higher noise immunity} than RS-232 \\
\end{longtable}
}

\end{solutionbox}
\begin{mnemonicbox}
``DMMBHN'' - ``Differential Maximum Multi-drop
Balanced Higher Noise-immunity''

\end{mnemonicbox}
\subsection*{Question 3(b) [4 marks]}\label{q3b}

\textbf{Define: Entropy, Information, Mutual Information and
Probability.}

\begin{solutionbox}

{\def\LTcaptype{none} % do not increment counter
\begin{longtable}[]{@{}
  >{\raggedright\arraybackslash}p{(\linewidth - 2\tabcolsep) * \real{0.3333}}
  >{\raggedright\arraybackslash}p{(\linewidth - 2\tabcolsep) * \real{0.6667}}@{}}
\toprule\noalign{}
\begin{minipage}[b]{\linewidth}\raggedright
Term
\end{minipage} & \begin{minipage}[b]{\linewidth}\raggedright
Definition
\end{minipage} \\
\midrule\noalign{}
\endhead
\bottomrule\noalign{}
\endlastfoot
\textbf{Entropy} & Measure of uncertainty or randomness in a message
source, calculated as H(X) = -\sump(x)log_{2}p(x) \\
\textbf{Information} & Reduction in uncertainty when a message is
received, measured in bits \\
\textbf{Mutual Information} & Measure of dependency between two random
variables, indicating how much information one variable contains about
the other \\
\textbf{Probability} & Mathematical measure of likelihood that an event
will occur, ranging from 0 (impossible) to 1 (certain) \\
\end{longtable}
}

\textbf{Diagram:}

\includegraphics[width=1\linewidth,height=\textheight,keepaspectratio]{mermaid-c9650871.pdf}

\end{solutionbox}
\begin{mnemonicbox}
``EIMP'' - ``Entropy Information Measures
Probability''

\end{mnemonicbox}
\subsection*{Question 3(c) [7 marks]}\label{q3c}

\textbf{Explain Huffman Code and Shannon-Fano code with suitable
example.}

\begin{solutionbox}

\textbf{Huffman Code:} Huffman coding assigns variable-length codes to
symbols based on their frequencies, with shorter codes for more frequent
symbols.

\textbf{Example:}

{\def\LTcaptype{none} % do not increment counter
\begin{longtable}[]{@{}lll@{}}
\toprule\noalign{}
Symbol & Frequency & Huffman Code \\
\midrule\noalign{}
\endhead
\bottomrule\noalign{}
\endlastfoot
A & 45\% & 0 \\
B & 25\% & 10 \\
C & 15\% & 110 \\
D & 10\% & 1110 \\
E & 5\% & 1111 \\
\end{longtable}
}

\textbf{Huffman Tree:}

\includegraphics[width=1\linewidth,height=\textheight,keepaspectratio]{mermaid-76607361.pdf}

\textbf{Shannon-Fano Code:} Shannon-Fano algorithm recursively divides
symbols into two groups of similar frequency, then assigns 0 to one
group and 1 to the other.

\textbf{Example:}

{\def\LTcaptype{none} % do not increment counter
\begin{longtable}[]{@{}lll@{}}
\toprule\noalign{}
Symbol & Frequency & Shannon-Fano Code \\
\midrule\noalign{}
\endhead
\bottomrule\noalign{}
\endlastfoot
A & 45\% & 0 \\
B & 25\% & 10 \\
C & 15\% & 110 \\
D & 10\% & 1110 \\
E & 5\% & 1111 \\
\end{longtable}
}

\textbf{Shannon-Fano Tree:}

\includegraphics[width=1\linewidth,height=\textheight,keepaspectratio]{mermaid-5c68b29d.pdf}

\end{solutionbox}
\begin{mnemonicbox}
``FREDS'' - ``Frequency Reduces Encoding Digit Size''

\end{mnemonicbox}
\subsection*{Question 3(a) OR [3
marks]}\label{q3a}

\textbf{State the features of RS-232.}

\begin{solutionbox}

{\def\LTcaptype{none} % do not increment counter
\begin{longtable}[]{@{}l@{}}
\toprule\noalign{}
Features of RS-232 \\
\midrule\noalign{}
\endhead
\bottomrule\noalign{}
\endlastfoot
\textbf{Single-ended signaling} \\
\textbf{Maximum data rate} of 20 kbps \\
\textbf{Maximum cable length} of 15 meters \\
\textbf{Point-to-point communication} (1 driver, 1 receiver) \\
\textbf{Voltage levels}: -15V to +15V \\
\textbf{25-pin or 9-pin} DB connector standard \\
\end{longtable}
}

\end{solutionbox}
\begin{mnemonicbox}
``SMPVD'' - ``Single Maximum Point-to-point Voltage
DB-connector''

\end{mnemonicbox}
\subsection*{Question 3(b) OR [4
marks]}\label{q3b}

\textbf{What is channel capacity in terms of SNR? Explain its
importance.}

\begin{solutionbox}

\textbf{Channel Capacity:} The maximum rate at which information can be
transmitted over a communication channel with an arbitrarily small
probability of error.

\textbf{Formula:} C = B \times log_{2}(1 + SNR)

Where:

\begin{itemize}
\tightlist
\item
  C = Channel capacity in bits per second
\item
  B = Bandwidth in Hertz
\item
  SNR = Signal-to-Noise Ratio
\end{itemize}

\textbf{Importance:}

{\def\LTcaptype{none} % do not increment counter
\begin{longtable}[]{@{}l@{}}
\toprule\noalign{}
Importance of Channel Capacity \\
\midrule\noalign{}
\endhead
\bottomrule\noalign{}
\endlastfoot
\textbf{Sets theoretical limits} for data transmission \\
\textbf{Guides system design} and optimization \\
\textbf{Helps evaluate performance} of communication systems \\
\textbf{Determines required bandwidth} for a given data rate \\
\textbf{Informs coding techniques} to approach capacity \\
\end{longtable}
}

\textbf{Diagram:}

\includegraphics[width=1\linewidth,height=\textheight,keepaspectratio]{mermaid-4a15bedd.pdf}

\end{solutionbox}
\begin{mnemonicbox}
``BSNR'' - ``Bandwidth and SNR Need Relationship''

\end{mnemonicbox}
\subsection*{Question 3(c) OR [7
marks]}\label{q3c}

\textbf{Explain in detail any one error detection and error correction
technique in digital communication.}

\begin{solutionbox}

\textbf{Hamming Code Error Detection and Correction}

Hamming code is a linear error-correcting code that can detect and
correct single-bit errors in data transmission.

\textbf{Working Principle:}

\begin{enumerate}
\tightlist
\item
  Data bits are positioned at locations that are powers of 2 (1, 2, 4,
  8, etc.)
\item
  Parity bits are added at positions 1, 2, 4, 8, etc.
\item
  Each parity bit checks specific data bits according to its position
\item
  On receiving, parity checks identify error position
\end{enumerate}

\textbf{Example: 7-bit Hamming code (4 data bits, 3 parity bits)}

{\def\LTcaptype{none} % do not increment counter
\begin{longtable}[]{@{}llllllll@{}}
\toprule\noalign{}
Position & 1 & 2 & 3 & 4 & 5 & 6 & 7 \\
\midrule\noalign{}
\endhead
\bottomrule\noalign{}
\endlastfoot
Bit type & P_{1} & P_{2} & D_{1} & P_{4} & D_{2} & D_{3} & D_{4} \\
\end{longtable}
}

\textbf{Parity Bit Calculation:}

\begin{itemize}
\tightlist
\item
  P_{1} checks bits 1, 3, 5, 7 (positions 1, 3, 5, 7)
\item
  P_{2} checks bits 2, 3, 6, 7 (positions 2, 3, 6, 7)
\item
  P_{4} checks bits 4, 5, 6, 7 (positions 4, 5, 6, 7)
\end{itemize}

\textbf{Error Correction:} If an error occurs, the parity checks will
indicate the error position, which can then be flipped to correct the
error.


{\def\LTcaptype{none} % do not increment counter
\vspace{-5pt}
\captionof{table}{Error Position from Parity Check Results}
\vspace{-10pt}
\begin{longtable}[]{@{}llll@{}}
\toprule\noalign{}
P_{4} & P_{2} & P_{1} & Error Position \\
\midrule\noalign{}
\endhead
\bottomrule\noalign{}
\endlastfoot
0 & 0 & 0 & No error \\
0 & 0 & 1 & Position 1 \\
0 & 1 & 0 & Position 2 \\
0 & 1 & 1 & Position 3 \\
1 & 0 & 0 & Position 4 \\
1 & 0 & 1 & Position 5 \\
1 & 1 & 0 & Position 6 \\
1 & 1 & 1 & Position 7 \\
\end{longtable}
}

\end{solutionbox}
\begin{mnemonicbox}
``PECD'' - ``Parity Enables Correction of Data''

\end{mnemonicbox}
\subsection*{Question 4(a) [3 marks]}\label{q4a}

\textbf{Draw the block diagram of satellite communication and explain in
brief.}

\begin{solutionbox}

\textbf{Satellite Communication Block Diagram:}

\includegraphics[width=1\linewidth,height=\textheight,keepaspectratio]{mermaid-b2b2bd21.pdf}

\textbf{Brief Explanation:} Satellite communication involves
transmitting signals from an Earth station to a satellite (uplink),
which then amplifies and retransmits the signals back to Earth
(downlink). The satellite acts as a repeater in space, enabling
long-distance communication.

\textbf{Key Components:}

\begin{itemize}
\tightlist
\item
  \textbf{Earth Stations}: Transmit and receive signals
\item
  \textbf{Transponders}: Receive, amplify, and retransmit signals
\item
  \textbf{Antennas}: Transmit and receive electromagnetic waves
\item
  \textbf{Modems}: Convert digital data to analog signals and vice versa
\end{itemize}

\end{solutionbox}
\begin{mnemonicbox}
``STAR'' - ``Satellite Transmits And Receives''

\end{mnemonicbox}
\subsection*{Question 4(b) [4 marks]}\label{q4b}

\textbf{Sketch the Unipolar NRZ, Polar RZ, Polar NRZ and AMI waveform
for 10101101 data sequence.}

\begin{solutionbox}

\begin{lstlisting}
Input Data:  1  0  1  0  1  1  0  1
            ▄▄    ▄▄    ▄▄▄▄    ▄▄  
            │ │   │ │   │  │    │ │ 
Data:       │ │   │ │   │  │    │ │ 
            │ └───┘ └───┘  └────┘ └─
            
            ▄▄    ▄▄    ▄▄▄▄    ▄▄  
            │ │   │ │   │  │    │ │ 
Unipolar    │ │   │ │   │  │    │ │ 
NRZ:        │ └───┘ └───┘  └────┘ └─
            
            ┌┐    ┌┐    ┌┐┌┐    ┌┐
Polar       ││    ││    │││││   ││
RZ:         ││    ││    │││││   ││
           ─┘└────┘└────┘┘┘┘└───┘└─
            ▄▄    ▄▄    ▄▄▄▄    ▄▄
            │ │   │ │   │  │    │ │
Polar       │ │   │ │   │  │    │ │
NRZ:       ─┘ └───┐ └───┘  └────┘ └
                 │                  
                 └──────────────────
                 
            ▄▄         ▄▄         ▄▄
            │ │        │ │        │ │
AMI:        │ │        │ │        │ │
           ─┘ └────────┘ └────────┘ └
                ▄▄         ▄▄        
                │ │        │ │       
                │ │        │ │       
           ─────┘ └────────┘ └───────
\end{lstlisting}

\end{solutionbox}
\begin{mnemonicbox}
``UPPA'' - ``Unipolar Polar Polar AMI''

\end{mnemonicbox}
\subsection*{Question 4(c) [7 marks]}\label{q4c}

\textbf{Explain data transmission techniques in details with suitable
example for digital communication.}

\begin{solutionbox}

\textbf{Data Transmission Techniques:}

{\def\LTcaptype{none} % do not increment counter
\begin{longtable}[]{@{}
  >{\raggedright\arraybackslash}p{(\linewidth - 4\tabcolsep) * \real{0.3333}}
  >{\raggedright\arraybackslash}p{(\linewidth - 4\tabcolsep) * \real{0.3939}}
  >{\raggedright\arraybackslash}p{(\linewidth - 4\tabcolsep) * \real{0.2727}}@{}}
\toprule\noalign{}
\begin{minipage}[b]{\linewidth}\raggedright
Technique
\end{minipage} & \begin{minipage}[b]{\linewidth}\raggedright
Description
\end{minipage} & \begin{minipage}[b]{\linewidth}\raggedright
Example
\end{minipage} \\
\midrule\noalign{}
\endhead
\bottomrule\noalign{}
\endlastfoot
\textbf{Serial Transmission} & Data bits sent one after another over a
single channel & USB, UART communication \\
\textbf{Parallel Transmission} & Multiple bits sent simultaneously over
multiple channels & Printer ports, SCSI \\
\textbf{Synchronous Transmission} & Data sent in continuous stream with
timing signals & Ethernet, HDLC \\
\textbf{Asynchronous Transmission} & Data sent with start/stop bits as
timing references & RS-232, UART \\
\textbf{Simplex} & One-way communication & TV broadcasting \\
\textbf{Half-Duplex} & Two-way communication, one direction at a time &
Walkie-talkie \\
\textbf{Full-Duplex} & Two-way simultaneous communication & Telephone
calls \\
\end{longtable}
}

\textbf{Serial Transmission Example:}

\begin{lstlisting}
            Start   1  0  1  0  1  1  0  1  Stop
             bit                          bit
            ┌───┐  ┌┐   ┌┐   ┌┐┌┐   ┌┐  ┌───┐
            │   │  ││   ││   │││││  ││  │   │
UART:       │   │  ││   ││   │││││  ││  │   │
          ──┘   └──┘└───┘└───┘┘┘┘└──┘└──┘   └──
\end{lstlisting}

\textbf{Parallel Transmission Example:}

\begin{lstlisting}
Data: 10101101

      Bit 7: ──────┐    ┌────────
                   │    │          
      Bit 6: ──────┘    └────────
                   
      Bit 5: ───────────────────
                   
      Bit 4: ──────┐    ┌────────
                   │    │          
      Bit 3: ──────┘    └────────
                   
      Bit 2: ──────┐          ┌───
                   │          │   
      Bit 1: ──────┘          └───
                   
      Bit 0: ──────┐    ┌────┐    ┌
                   │    │    │    │
           ────────┘    └────┘    └
                   
Clock:      ┌─┐  ┌─┐  ┌─┐  ┌─┐  ┌─┐
            │ │  │ │  │ │  │ │  │ │
            │ │  │ │  │ │  │ │  │ │
          ──┘ └──┘ └──┘ └──┘ └──┘ └─
\end{lstlisting}

\end{solutionbox}
\begin{mnemonicbox}
``SPASH'' - ``Serial Parallel Asynchronous
Synchronous Half-duplex''

\end{mnemonicbox}
\subsection*{Question 4(a) OR [3
marks]}\label{q4a}

\textbf{Interpret the aspects of spread spectrum techniques.}

\begin{solutionbox}

\textbf{Spread Spectrum Techniques:}

{\def\LTcaptype{none} % do not increment counter
\begin{longtable}[]{@{}
  >{\raggedright\arraybackslash}p{(\linewidth - 2\tabcolsep) * \real{0.3478}}
  >{\raggedright\arraybackslash}p{(\linewidth - 2\tabcolsep) * \real{0.6522}}@{}}
\toprule\noalign{}
\begin{minipage}[b]{\linewidth}\raggedright
Aspect
\end{minipage} & \begin{minipage}[b]{\linewidth}\raggedright
Interpretation
\end{minipage} \\
\midrule\noalign{}
\endhead
\bottomrule\noalign{}
\endlastfoot
\textbf{Bandwidth Spreading} & Signal spread over a wider bandwidth than
required \\
\textbf{Security} & Difficult to intercept or jam due to spreading \\
\textbf{Noise Immunity} & Resistant to narrowband interference \\
\textbf{Multiple Access} & Allows multiple users to share same frequency
band \\
\textbf{Low Power Density} & Signal power spread across wide band,
appearing as noise \\
\end{longtable}
}

\textbf{Diagram:}

\includegraphics[width=1\linewidth,height=\textheight,keepaspectratio]{mermaid-6a35c20b.pdf}

\end{solutionbox}
\begin{mnemonicbox}
``BSNML'' - ``Bandwidth Security Noise Multiple
Low-power''

\end{mnemonicbox}
\subsection*{Question 4(b) OR [4
marks]}\label{q4b}

\textbf{Write a short note on probability and discuss its properties for
digital communication.}

\begin{solutionbox}

\textbf{Probability in Digital Communication:} Probability theory
provides the mathematical foundation for analyzing performance, error
rates, and reliability of digital communication systems.

\textbf{Properties of Probability:}

{\def\LTcaptype{none} % do not increment counter
\begin{longtable}[]{@{}
  >{\raggedright\arraybackslash}p{(\linewidth - 4\tabcolsep) * \real{0.1724}}
  >{\raggedright\arraybackslash}p{(\linewidth - 4\tabcolsep) * \real{0.2241}}
  >{\raggedright\arraybackslash}p{(\linewidth - 4\tabcolsep) * \real{0.6034}}@{}}
\toprule\noalign{}
\begin{minipage}[b]{\linewidth}\raggedright
Property
\end{minipage} & \begin{minipage}[b]{\linewidth}\raggedright
Description
\end{minipage} & \begin{minipage}[b]{\linewidth}\raggedright
Relevance in Digital Communication
\end{minipage} \\
\midrule\noalign{}
\endhead
\bottomrule\noalign{}
\endlastfoot
\textbf{Range} & 0 \leq P(E) \leq 1 & Sets bounds for error probability \\
\textbf{Certainty} & P(S) = 1 for sample space S & Total probability of
all possible outcomes \\
\textbf{Additivity} & P(A\cupB) = P(A) + P(B) for disjoint events &
Calculating overall system error rates \\
\textbf{Conditional Probability} & P(A\textbar B) = P(A\capB)/P(B) & Useful
for channel modeling \\
\textbf{Independence} & P(A\capB) = P(A)\timesP(B) & Analyzing uncorrelated
noise sources \\
\end{longtable}
}

\textbf{Applications in Digital Communication:}

\begin{itemize}
\tightlist
\item
  Bit Error Rate calculation
\item
  Signal detection theory
\item
  Channel capacity estimation
\item
  Coding efficiency analysis
\end{itemize}

\end{solutionbox}
\begin{mnemonicbox}
``RACIC'' - ``Range Additivity Certainty Independence
Conditional''

\end{mnemonicbox}
\subsection*{Question 4(c) OR [7
marks]}\label{q4c}

\textbf{Explain Data transmission mode in details with example.}

\begin{solutionbox}

\textbf{Data Transmission Modes:}

{\def\LTcaptype{none} % do not increment counter
\begin{longtable}[]{@{}
  >{\raggedright\arraybackslash}p{(\linewidth - 6\tabcolsep) * \real{0.1622}}
  >{\raggedright\arraybackslash}p{(\linewidth - 6\tabcolsep) * \real{0.3514}}
  >{\raggedright\arraybackslash}p{(\linewidth - 6\tabcolsep) * \real{0.2432}}
  >{\raggedright\arraybackslash}p{(\linewidth - 6\tabcolsep) * \real{0.2432}}@{}}
\toprule\noalign{}
\begin{minipage}[b]{\linewidth}\raggedright
Mode
\end{minipage} & \begin{minipage}[b]{\linewidth}\raggedright
Description
\end{minipage} & \begin{minipage}[b]{\linewidth}\raggedright
Diagram
\end{minipage} & \begin{minipage}[b]{\linewidth}\raggedright
Example
\end{minipage} \\
\midrule\noalign{}
\endhead
\bottomrule\noalign{}
\endlastfoot
\textbf{Simplex} & One-way communication only. Transmitter can only
send, receiver can only receive. &
\passthrough{\lstinline!mermaid graph LR; A[Transmitter] -->|One-way| B[Receiver]!}
& TV broadcasting, Radio \\
\textbf{Half-Duplex} & Two-way communication, but only one direction at
a time. &
\passthrough{\lstinline!mermaid graph LR; A[Device A] -->|Time 1| B[Device B]; B -->|Time 2| A!}
& Walkie-talkie, CB radio \\
\textbf{Full-Duplex} & Two-way simultaneous communication. &
\passthrough{\lstinline!mermaid graph LR; A[Device A] -->|Channel 1| B[Device B]; B -->|Channel 2| A!}
& Telephone, Mobile calls \\
\end{longtable}
}

\textbf{Example of Half-Duplex Communication:}

\begin{lstlisting}
    Device A                     Device B
       |                            |
       |        REQUEST DATA        |
       |--------------------------->|
       |                            |
       |                            |
       |        SENDING DATA        |
       |<---------------------------|
       |                            |
       |    ACKNOWLEDGMENT (ACK)    |
       |--------------------------->|
       |                            |
\end{lstlisting}

\textbf{Example of Full-Duplex Communication:}

\begin{lstlisting}
    Device A                     Device B
       |                            |
       |        SENDING DATA        |
       |--------------------------->|
       |                            |
       |        SENDING DATA        |
       |<---------------------------|
       |                            |
       |      CONTINUOUS DATA       |
       |<-------------------------->|
       |                            |
\end{lstlisting}

\end{solutionbox}
\begin{mnemonicbox}
``SHF'' - ``Simplex Half Full'' or ``Stop, Halt,
Flow''

\end{mnemonicbox}
\subsection*{Question 5(a) [3 marks]}\label{q5a}

\textbf{Explain Edge Computing in detail.}

\begin{solutionbox}

\textbf{Edge Computing:} Edge computing is a distributed computing
paradigm that brings computation and data storage closer to the location
where it is needed to improve response times and save bandwidth.

\textbf{Key Aspects:}

{\def\LTcaptype{none} % do not increment counter
\begin{longtable}[]{@{}
  >{\raggedright\arraybackslash}p{(\linewidth - 2\tabcolsep) * \real{0.3810}}
  >{\raggedright\arraybackslash}p{(\linewidth - 2\tabcolsep) * \real{0.6190}}@{}}
\toprule\noalign{}
\begin{minipage}[b]{\linewidth}\raggedright
Aspect
\end{minipage} & \begin{minipage}[b]{\linewidth}\raggedright
Description
\end{minipage} \\
\midrule\noalign{}
\endhead
\bottomrule\noalign{}
\endlastfoot
\textbf{Decentralization} & Processing at network edge instead of
central cloud \\
\textbf{Reduced Latency} & Faster response due to proximity to data
source \\
\textbf{Bandwidth Efficiency} & Less data sent to cloud, reducing
network congestion \\
\textbf{Local Data Processing} & Data processed near collection point \\
\textbf{Improved Security} & Sensitive data remains local, reducing
exposure \\
\textbf{Reliability} & Continues to function during cloud connectivity
issues \\
\end{longtable}
}

\textbf{Diagram:}

\includegraphics[width=1\linewidth,height=\textheight,keepaspectratio]{mermaid-04f985b1.pdf}

\end{solutionbox}
\begin{mnemonicbox}
``DRBLES'' - ``Decentralized Reduces Bandwidth,
Latency, Exposure, Strengthens reliability''

\end{mnemonicbox}
\subsection*{Question 5(b) [4 marks]}\label{q5b}

\textbf{Enlist the features of 5G Technology in data communication.}

\begin{solutionbox}

{\def\LTcaptype{none} % do not increment counter
\begin{longtable}[]{@{}l@{}}
\toprule\noalign{}
Features of 5G Technology \\
\midrule\noalign{}
\endhead
\bottomrule\noalign{}
\endlastfoot
\textbf{High Data Rates} (up to 20 Gbps peak) \\
\textbf{Ultra-Low Latency} (1 ms or less) \\
\textbf{Massive Device Connectivity} (1 million devices per km^{2}) \\
\textbf{Network Slicing} (customized virtual networks) \\
\textbf{Beamforming} (directed signal transmission) \\
\textbf{Millimeter Wave Spectrum} (24-100 GHz) \\
\textbf{Enhanced Mobile Broadband} (eMBB) \\
\textbf{Ultra-Reliable Low-Latency Communication} (URLLC) \\
\end{longtable}
}

\textbf{Diagram:}

\includegraphics[width=1\linewidth,height=\textheight,keepaspectratio]{mermaid-07f5ca14.pdf}

\end{solutionbox}
\begin{mnemonicbox}
``HUMBLE-MN'' - ``High-speed Ultra-low-latency
Massive Beamforming Low-latency Enhanced Millimeter Network''

\end{mnemonicbox}
\subsection*{Question 5(c) [7 marks]}\label{q5c}

\textbf{Write a details note on Data communication including its
characteristics and components.}

\begin{solutionbox}

\textbf{Data Communication:} Data communication is the process of
transferring digital information between two or more points.

\textbf{Characteristics of Data Communication:}

{\def\LTcaptype{none} % do not increment counter
\begin{longtable}[]{@{}
  >{\raggedright\arraybackslash}p{(\linewidth - 2\tabcolsep) * \real{0.5517}}
  >{\raggedright\arraybackslash}p{(\linewidth - 2\tabcolsep) * \real{0.4483}}@{}}
\toprule\noalign{}
\begin{minipage}[b]{\linewidth}\raggedright
Characteristic
\end{minipage} & \begin{minipage}[b]{\linewidth}\raggedright
Description
\end{minipage} \\
\midrule\noalign{}
\endhead
\bottomrule\noalign{}
\endlastfoot
\textbf{Delivery} & System must deliver data to correct destination \\
\textbf{Accuracy} & System must deliver data accurately, without
errors \\
\textbf{Timeliness} & System must deliver data in a timely manner \\
\textbf{Jitter} & System must maintain consistent timing between data
arrivals \\
\textbf{Security} & System must protect data from unauthorized access \\
\end{longtable}
}

\textbf{Components of Data Communication:}

{\def\LTcaptype{none} % do not increment counter
\begin{longtable}[]{@{}ll@{}}
\toprule\noalign{}
Component & Description \\
\midrule\noalign{}
\endhead
\bottomrule\noalign{}
\endlastfoot
\textbf{Message} & The information (data) to be communicated \\
\textbf{Sender} & Device that sends the data message \\
\textbf{Receiver} & Device that receives the message \\
\textbf{Transmission Medium} & Physical path by which message travels \\
\textbf{Protocol} & Set of rules governing data communication \\
\end{longtable}
}

\textbf{Data Communication Model:}

\includegraphics[width=1\linewidth,height=\textheight,keepaspectratio]{mermaid-8c40f3e1.pdf}

\textbf{Data Communication Types:}

{\def\LTcaptype{none} % do not increment counter
\begin{longtable}[]{@{}
  >{\raggedright\arraybackslash}p{(\linewidth - 2\tabcolsep) * \real{0.3158}}
  >{\raggedright\arraybackslash}p{(\linewidth - 2\tabcolsep) * \real{0.6842}}@{}}
\toprule\noalign{}
\begin{minipage}[b]{\linewidth}\raggedright
Type
\end{minipage} & \begin{minipage}[b]{\linewidth}\raggedright
Description
\end{minipage} \\
\midrule\noalign{}
\endhead
\bottomrule\noalign{}
\endlastfoot
\textbf{Analog} & Continuous signal that varies in amplitude or
frequency \\
\textbf{Digital} & Discrete signal represented by binary digits (0s and
1s) \\
\textbf{Parallel} & Multiple bits transmitted simultaneously on separate
channels \\
\textbf{Serial} & Bits transmitted sequentially on a single channel \\
\end{longtable}
}

\end{solutionbox}
\begin{mnemonicbox}
``DATJS-MSRTP'' - ``Delivery Accuracy Timeliness
Jitter Security - Message Sender Receiver Transmission Protocol''

\end{mnemonicbox}
\subsection*{Question 5(a) OR [3
marks]}\label{q5a}

\textbf{Identify and write privacy consideration in Data communication.}

\begin{solutionbox}

\textbf{Privacy Considerations in Data Communication:}

{\def\LTcaptype{none} % do not increment counter
\begin{longtable}[]{@{}
  >{\raggedright\arraybackslash}p{(\linewidth - 2\tabcolsep) * \real{0.6286}}
  >{\raggedright\arraybackslash}p{(\linewidth - 2\tabcolsep) * \real{0.3714}}@{}}
\toprule\noalign{}
\begin{minipage}[b]{\linewidth}\raggedright
Privacy Consideration
\end{minipage} & \begin{minipage}[b]{\linewidth}\raggedright
Description
\end{minipage} \\
\midrule\noalign{}
\endhead
\bottomrule\noalign{}
\endlastfoot
\textbf{Data Encryption} & Protecting data during transmission using
encryption algorithms \\
\textbf{Access Control} & Ensuring only authorized users can access
communication systems \\
\textbf{Authentication} & Verifying the identity of users and devices \\
\textbf{Data Minimization} & Collecting only necessary data to minimize
privacy risks \\
\textbf{Secure Protocols} & Using communication protocols with built-in
security features \\
\textbf{End-to-End Security} & Ensuring data is protected throughout the
entire communication path \\
\end{longtable}
}

\textbf{Diagram:}

\includegraphics[width=1\linewidth,height=\textheight,keepaspectratio]{mermaid-d18629d1.pdf}

\end{solutionbox}
\begin{mnemonicbox}
``DAAESE'' - ``Data is Authenticated, Accessed,
Encrypted Securely End-to-end''

\end{mnemonicbox}
\subsection*{Question 5(b) OR [4
marks]}\label{q5b}

\textbf{What is block chain in communication security? Enlist its
features.}

\begin{solutionbox}

\textbf{Blockchain in Communication Security:} Blockchain is a
distributed ledger technology that provides secure, tamper-proof
record-keeping for data communication through cryptographic linking of
data blocks.

\textbf{Features of Blockchain:}

{\def\LTcaptype{none} % do not increment counter
\begin{longtable}[]{@{}
  >{\raggedright\arraybackslash}p{(\linewidth - 2\tabcolsep) * \real{0.4091}}
  >{\raggedright\arraybackslash}p{(\linewidth - 2\tabcolsep) * \real{0.5909}}@{}}
\toprule\noalign{}
\begin{minipage}[b]{\linewidth}\raggedright
Feature
\end{minipage} & \begin{minipage}[b]{\linewidth}\raggedright
Description
\end{minipage} \\
\midrule\noalign{}
\endhead
\bottomrule\noalign{}
\endlastfoot
\textbf{Decentralization} & No central authority; distributed across
network nodes \\
\textbf{Immutability} & Once recorded, data cannot be altered without
consensus \\
\textbf{Transparency} & All transactions visible to authorized
participants \\
\textbf{Cryptographic Security} & Data secured using advanced
cryptographic techniques \\
\textbf{Consensus Mechanism} & Network agrees on validity of
transactions \\
\textbf{Smart Contracts} & Self-executing contracts with terms directly
written in code \\
\textbf{Distributed Storage} & Data stored across multiple nodes,
preventing single point of failure \\
\end{longtable}
}

\textbf{Diagram:}

\includegraphics[width=1\linewidth,height=\textheight,keepaspectratio]{mermaid-29f5f072.pdf}

\end{solutionbox}
\begin{mnemonicbox}
``DITCSD'' - ``Decentralized Immutable Transparent
Cryptographic Secure Distributed''

\end{mnemonicbox}
\subsection*{Question 5(c) OR [7
marks]}\label{q5c}

\textbf{Write and illustrate different communication ports: USB, HDMI,
RCA and Ethernet.}

\begin{solutionbox}

\textbf{Communication Ports:}

\begin{enumerate}
\tightlist
\item
  \textbf{USB (Universal Serial Bus):}
\end{enumerate}

\begin{lstlisting}
    ┌───────────┐
    │           │
    │   USB-A   │
    │   ┌───┐   │
    │   │   │   │
    │   └───┘   │
    └───────────┘

    ┌───────────┐
    │           │
    │   USB-C   │
    │ ┌───────┐ │
    │ │       │ │
    │ └───────┘ │
    └───────────┘
\end{lstlisting}

\textbf{Features:}

\begin{itemize}
\tightlist
\item
  Data transfer, power delivery, and device connection
\item
  Versions: USB 1.0 to USB 4.0
\item
  Speed: Up to 40 Gbps (USB4)
\item
  Hot-swappable
\item
  Supports up to 127 devices in cascade
\end{itemize}

\begin{enumerate}
\tightlist
\item
  \textbf{HDMI (High-Definition Multimedia Interface):}
\end{enumerate}

\begin{lstlisting}
    ┌─────────────────┐
    │                 │
    │      HDMI       │
    │  ┌───────────┐  │
    │  │           │  │
    │  └───────────┘  │
    └─────────────────┘
\end{lstlisting}

\textbf{Features:}

\begin{itemize}
\tightlist
\item
  Digital audio/video transmission
\item
  Versions: HDMI 1.0 to HDMI 2.1
\item
  Resolution support: Up to 10K
\item
  Bandwidth: Up to 48 Gbps (HDMI 2.1)
\item
  HDCP (High-bandwidth Digital Content Protection)
\item
  CEC (Consumer Electronics Control) for device control
\end{itemize}

\begin{enumerate}
\tightlist
\item
  \textbf{RCA (Radio Corporation of America):}
\end{enumerate}

\begin{lstlisting}
    ┌───┐  ┌───┐  ┌───┐
    │   │  │   │  │   │
    │ R │  │ G │  │ B │
    │   │  │   │  │   │
    └───┘  └───┘  └───┘
    Red    Green  Blue
    
    ┌───┐  ┌───┐
    │   │  │   │
    │ W │  │ R │
    │   │  │   │
    └───┘  └───┘
    White  Red
    Video  Audio
\end{lstlisting}

\textbf{Features:}

\begin{itemize}
\tightlist
\item
  Analog audio/video transmission
\item
  Color-coded connectors (Red, White, Yellow)
\item
  Used for composite video and stereo audio
\item
  Simple connection but limited quality
\item
  No digital content protection
\item
  Being phased out by digital standards
\end{itemize}

\begin{enumerate}
\tightlist
\item
  \textbf{Ethernet (RJ-45):}
\end{enumerate}

\begin{lstlisting}
    ┌───────────────┐
    │               │
    │    RJ-45      │
    │ ┌───────────┐ │
    │ │|||||||||  │ │
    │ └───────────┘ │
    └───────────────┘
\end{lstlisting}

\textbf{Features:}

\begin{itemize}
\tightlist
\item
  Network connectivity
\item
  Standards: 10BASE-T to 10GBASE-T
\item
  Speed: 10 Mbps to 10 Gbps
\item
  Uses twisted-pair cabling (Cat5e, Cat6, Cat6a)
\item
  Supports Power over Ethernet (PoE)
\item
  Base communication for TCP/IP networks
\item
  Maximum cable length: 100 meters
\end{itemize}

\textbf{Comparison Table:}

{\def\LTcaptype{none} % do not increment counter
\begin{longtable}[]{@{}llllll@{}}
\toprule\noalign{}
Port & Type & Data Type & Max Speed & Power Delivery & Max Length \\
\midrule\noalign{}
\endhead
\bottomrule\noalign{}
\endlastfoot
USB & Digital & Data/Power & 40 Gbps & Yes (100W) & 5m \\
HDMI & Digital & Audio/Video & 48 Gbps & Limited & 15m \\
RCA & Analog & Audio/Video & Low & No & 10m \\
Ethernet & Digital & Network Data & 10 Gbps & Yes (PoE) & 100m \\
\end{longtable}
}

\end{solutionbox}
\begin{mnemonicbox}
``UHRE'' - ``USB Handles Rapid Ethernet, HDMI
Delivers Rich Entertainment''

\end{mnemonicbox}

\end{document}
