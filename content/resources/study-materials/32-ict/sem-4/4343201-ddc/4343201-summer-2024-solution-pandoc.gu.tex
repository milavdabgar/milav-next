\documentclass[10pt,a4paper]{article}

% content/resources/templates/preamble.tex
\usepackage[margin=0.6in]{geometry}
\author{Milav Dabgar}
\usepackage{amsmath,amssymb,amsthm}
\usepackage{booktabs}
\usepackage{multirow}
\usepackage{xcolor}
\usepackage{tcolorbox}
\tcbuselibrary{breakable,skins}
\usepackage[colorlinks=true,linkcolor=blue]{hyperref}
\usepackage{titlesec}
\usepackage{enumitem}
\usepackage{tikz}
\usepackage{pgfplots}
\usepackage{circuitikz}
\usepackage[version=4]{mhchem}
\usepackage{longtable}
\usepackage{array}
\usepackage{float}
\usepackage{caption}
\usepackage{listings}

\lstset{
  basicstyle=\small\ttfamily,
  breaklines=true,
  breakatwhitespace=false,
  postbreak=\mbox{\textcolor{red}{$\hookrightarrow$}\space},
  float=false,
  numbers=left,
  numberstyle=\tiny\color{gray},
  numbersep=10pt,
  xleftmargin=2em,
  keywordstyle=\color{blue},
  commentstyle=\color{green!60!black},
  stringstyle=\color{purple},
  backgroundcolor=\color{gray!5},
  showstringspaces=false,
  tabsize=2,
  captionpos=b,
  keepspaces=true,
  columns=flexible
}

\pgfplotsset{compat=1.18}
\usetikzlibrary{shapes,arrows,positioning,calc,patterns,decorations.pathmorphing,decorations.markings,arrows.meta}

% Color scheme
\definecolor{headcolor}{RGB}{0,102,204}
\definecolor{keycolor}{RGB}{220,20,60}
\definecolor{solutioncolor}{RGB}{34,139,34}
\definecolor{mnemoniccolor}{RGB}{148,0,211}
\definecolor{codecolor}{RGB}{0,0,100}

% Spacing
\setlength{\parskip}{3pt}
\setlist[itemize]{nosep}
\setlist[enumerate]{nosep}

% Title formatting
\titleformat{\section}{\Large\bfseries\color{headcolor}}{\thesection}{1em}{}
\titleformat{\subsection}{\large\bfseries\color{headcolor}}{\thesubsection}{1em}{}

% Pandoc tightlist compatibility
\providecommand{\tightlist}{%
  \setlength{\itemsep}{0pt}\setlength{\parskip}{0pt}}

% Pandoc longtable compatibility
\newcounter{none}
\def\thenone{}


% content/resources/templates/gujarati-boxes.tex
\usepackage{fontspec}
\usepackage{polyglossia}

% Set Gujarati as main language (document is primarily in Gujarati)
% Note: gloss-gujarati.ldf doesn't exist in polyglossia, but it will use hyphenation patterns
\setdefaultlanguage{gujarati}
\setotherlanguage{english}

% Configure Gujarati font properly
% Use Language=Default to prevent polyglossia from trying to add language-specific features
% that don't exist for Gujarati, which causes "empty feature" warnings
\newfontfamily\gujaratifont[Script=Gujarati,AutoFakeBold=2.5,AutoFakeSlant=0.3]{Noto Sans Gujarati}
\setmainfont[Script=Gujarati,AutoFakeBold=2.5,AutoFakeSlant=0.3]{Noto Sans Gujarati}
% Use Noto Sans Gujarati for monospace to support Gujarati in text
\setmonofont[Scale=0.9]{Noto Sans Gujarati}

% Configure English to use the same font
\newfontfamily\englishfont[Script=Gujarati,AutoFakeBold=2.5,AutoFakeSlant=0.3]{Noto Sans Gujarati}

% Translations for polyglossia
\gappto\captionsgujarati{
  \renewcommand{\tablename}{કોષ્ટક}
  \renewcommand{\figurename}{આકૃતિ}
}

% Helper for TikZ nodes to ensure Gujarati font
\newcommand{\gu}[1]{{\gujaratifont #1}}

% Custom environments
\newtcolorbox{solutionbox}{
    breakable,
    enhanced,
    colback=solutioncolor!5!white,
    colframe=solutioncolor!75!black,
    fonttitle=\bfseries,
    title=જવાબ
}

\newtcolorbox{solutionboxnobreak}{
 colback=solutioncolor!5!white,
 colframe=solutioncolor!75!black,
 fonttitle=\bfseries,
 title=જવાબ
}

\newtcolorbox{keyformula}{
 breakable,
 enhanced,
 colback=keycolor!5!white,
 colframe=keycolor!75!black,
 fonttitle=\bfseries,
 title=રાસાયણિક સમીકરણ/સૂત્ર
}

\newtcolorbox{mnemonicbox}{
 breakable,
 enhanced,
 colback=mnemoniccolor!5!white,
 colframe=mnemoniccolor!75!black,
 fonttitle=\bfseries,
 title=મેમરી ટ્રીક
}


\begin{document}

\begin{center}
{\Huge\bfseries\color{headcolor} Subject Name (Gujarati)}\\[5pt]
{\LARGE 4343201 -- Summer 2024}\\[3pt]
{\large Semester 1 Study Material}\\[3pt]
{\normalsize\textit{Detailed Solutions and Explanations}}
\end{center}

\vspace{10pt}

\subsection*{પ્રશ્ન 1(અ) [3
ગુણ]}\label{uxaaauxab0uxab6uxaa8-1uxa85-3-uxa97uxaa3}

\textbf{વ્યાખ્યાયિત કરો: (1) બીટ રેટ, (2) બાઉન્ડ રેટ અને (3) બેન્ડવિડ્થ}

\begin{solutionbox}

{\def\LTcaptype{none} % do not increment counter
\begin{longtable}[]{@{}
  >{\raggedright\arraybackslash}p{(\linewidth - 2\tabcolsep) * \real{0.3333}}
  >{\raggedright\arraybackslash}p{(\linewidth - 2\tabcolsep) * \real{0.6667}}@{}}
\toprule\noalign{}
\begin{minipage}[b]{\linewidth}\raggedright
શબ્દ
\end{minipage} & \begin{minipage}[b]{\linewidth}\raggedright
વ્યાખ્યા
\end{minipage} \\
\midrule\noalign{}
\endhead
\bottomrule\noalign{}
\endlastfoot
\textbf{બીટ રેટ} & દર સેકન્ડે ટ્રાન્સમિટ થતા બિટ્સની સંખ્યા (bps) \\
\textbf{બાઉન્ડ રેટ} & દર સેકન્ડે ટ્રાન્સમિટ થતા સિગ્નલ એલિમેન્ટ્સ અથવા સિમ્બોલ્સની
સંખ્યા \\
\textbf{બેન્ડવિડ્થ} & સિગ્નલ ટ્રાન્સમિટ કરવા માટે જરૂરી ફ્રીક્વન્સીઓની રેન્જ, હર્ટ્ઝ
(Hz)માં માપવામાં આવે છે \\
\end{longtable}
}

\end{solutionbox}
\begin{mnemonicbox}
``BBB - બિટ્સ મૂવ બાય બેન્ડ્સ''

\end{mnemonicbox}
\subsection*{પ્રશ્ન 1(બ) [4
ગુણ]}\label{uxaaauxab0uxab6uxaa8-1uxaac-4-uxa97uxaa3}

\textbf{સિગ્નલનો બીટ રેટ 8000bps અને બાઉન્ડ રેટ 1000 બાઉન્ડ છે. દરેક સિગ્નલ દ્વારા
કેટલા ડેટા એલિમેન્ટ વહન કરવામાં આવે છે? આપણને કેટલા સિગ્નલ તત્વોની જરૂર છે?}

\begin{solutionbox}


{\def\LTcaptype{none} % do not increment counter
\vspace{-5pt}
\captionof{table}{સિગ્નલ ગણતરી}
\vspace{-10pt}
\begin{longtable}[]{@{}
  >{\raggedright\arraybackslash}p{(\linewidth - 4\tabcolsep) * \real{0.3548}}
  >{\raggedright\arraybackslash}p{(\linewidth - 4\tabcolsep) * \real{0.2258}}
  >{\raggedright\arraybackslash}p{(\linewidth - 4\tabcolsep) * \real{0.4194}}@{}}
\toprule\noalign{}
\begin{minipage}[b]{\linewidth}\raggedright
પેરામીટર
\end{minipage} & \begin{minipage}[b]{\linewidth}\raggedright
મૂલ્ય
\end{minipage} & \begin{minipage}[b]{\linewidth}\raggedright
ગણતરી
\end{minipage} \\
\midrule\noalign{}
\endhead
\bottomrule\noalign{}
\endlastfoot
બીટ રેટ & 8000 bps & આપેલ છે \\
બાઉન્ડ રેટ & 1000 બાઉન્ડ & આપેલ છે \\
દરેક સિગ્નલમાં ડેટા એલિમેન્ટ્સ & 8 બિટ્સ & બીટ રેટ \div બાઉન્ડ રેટ = 8000 \div 1000 =
8 \\
જરૂરી સિગ્નલ એલિમેન્ટ્સ & 2\^{}8 = 256 & 2\^{}(દરેક સિગ્નલના બિટ્સ) \\
\end{longtable}
}

\textbf{આકૃતિ: સિગ્નલ એલિમેન્ટ રેપ્રેઝન્ટેશન}

\begin{center}
\textbf{Mermaid Diagram (Code)}
\begin{verbatim}
{Shaded}
{Highlighting}[]
graph LR
    A[1000 સિગ્નલ્સ પ્રતિ સેકન્ડ] {-{-}{}|દરેક સિગ્નલ વહન કરે છે| B[8 બિટ્સ ડેટા]}
    B {-{-}{}|જરૂર છે| C[256 અલગ{-}અલગ સિગ્નલ એલિમેન્ટ્સ]}
{Highlighting}
{Shaded}
\end{verbatim}
\end{center}

\end{solutionbox}
\begin{mnemonicbox}
``ડિવાઇડ ટુ ડિસાઇડ'' - દરેક સિગ્નલમાં કેટલા બિટ્સ છે તે નક્કી
કરવા માટે બીટ રેટને બાઉન્ડ રેટથી ભાગો.

\end{mnemonicbox}
\subsection*{પ્રશ્ન 1(ક) [7
ગુણ]}\label{uxaaauxab0uxab6uxaa8-1uxa95-7-uxa97uxaa3}

\textbf{ડિજીટલ કોમ્યુનિકેશન સિસ્ટમના તત્વોનું તેના બ્લોક ડાયાગ્રામ સાથે વર્ણન કરો}

\begin{solutionbox}

\textbf{આકૃતિ: ડિજિટલ કોમ્યુનિકેશન સિસ્ટમ}

\begin{center}
\textbf{Mermaid Diagram (Code)}
\begin{verbatim}
{Shaded}
{Highlighting}[]
graph LR
    A[સોર્સ] {-{-}{} B[સોર્સ એન્કોડર]}
    B {-{-}{} C[ચેનલ એન્કોડર]}
    C {-{-}{} D[ડિજિટલ મોડ્યુલેટર]}
    D {-{-}{} E[ચેનલ]}
    E {-{-}{} F[ડિજિટલ ડિમોડ્યુલેટર]}
    F {-{-}{} G[ચેનલ ડિકોડર]}
    G {-{-}{} H[સોર્સ ડિકોડર]}
    H {-{-}{} I[ડેસ્ટિનેશન]}
{Highlighting}
{Shaded}
\end{verbatim}
\end{center}

\textbf{મુખ્ય તત્વો:}

{\def\LTcaptype{none} % do not increment counter
\begin{longtable}[]{@{}ll@{}}
\toprule\noalign{}
તત્વ & કાર્ય \\
\midrule\noalign{}
\endhead
\bottomrule\noalign{}
\endlastfoot
\textbf{સોર્સ} & ટ્રાન્સમિટ કરવા માટેના મેસેજ જનરેટ કરે છે \\
\textbf{સોર્સ એન્કોડર} & મેસેજને ડિજિટલ ફોર્મેટમાં કન્વર્ટ કરે છે, રિડન્ડન્સી દૂર કરે
છે \\
\textbf{ચેનલ એન્કોડર} & એરર ડિટેક્શન/કરેક્શન માટે રિડન્ડન્સી ઉમેરે છે \\
\textbf{ડિજિટલ મોડ્યુલેટર} & ડિજિટલ ડેટાને ચેનલ માટે યોગ્ય સિગ્નલમાં રૂપાંતરિત કરે
છે \\
\textbf{ચેનલ} & ભૌતિક માધ્યમ જે સિગ્નલને વહન કરે છે \\
\textbf{ડિજિટલ ડિમોડ્યુલેટર} & પ્રાપ્ત સિગ્નલમાંથી ડિજિટલ માહિતી અલગ કરે છે \\
\textbf{ચેનલ ડિકોડર} & ઉમેરેલી રિડન્ડન્સીનો ઉપયોગ કરીને ભૂલો શોધે/સુધારે છે \\
\textbf{સોર્સ ડિકોડર} & ડિજિટલ ડેટામાંથી ઓરિજિનલ મેસેજને ફરીથી બનાવે છે \\
\textbf{ડેસ્ટિનેશન} & અંતિમ મેસેજ પ્રાપ્ત કરે છે \\
\end{longtable}
}

\end{solutionbox}
\begin{mnemonicbox}
``સેન્ડ મેસેજિસ કેરફુલી; ડેસ્ટિનેશન મસ્ટ કોમ્પ્રિહેન્ડ સિગ્નલ્સ
ડીપલી''

\end{mnemonicbox}
\subsection*{પ્રશ્ન 1(ક OR) [7
ગુણ]}\label{uxaaauxab0uxab6uxaa8-1uxa95-or-7-uxa97uxaa3}

\textbf{ડિજિટલ કોમ્યુનિકેશન સિસ્ટમની મૂળભૂત મર્યાદા શું છે? ડિજિટલ કોમ્યુનિકેશન
સિસ્ટમના ફાયદા અને ગેરફાયદા શું છે?}

\begin{solutionbox}

\textbf{મૂળભૂત મર્યાદાઓ:}

{\def\LTcaptype{none} % do not increment counter
\begin{longtable}[]{@{}ll@{}}
\toprule\noalign{}
મર્યાદા & વર્ણન \\
\midrule\noalign{}
\endhead
\bottomrule\noalign{}
\endlastfoot
\textbf{બેન્ડવિડ્થ} & ડિજિટલ સિગ્નલને એનાલોગ કરતાં વધુ બેન્ડવિડ્થની જરૂર પડે છે \\
\textbf{નોઇઝ} & મહત્તમ પ્રાપ્ય ડેટા રેટને મર્યાદિત કરે છે \\
\textbf{ઇક્વિપમેન્ટ} & ડિજિટલ સિસ્ટમને જટિલ હાર્ડવેર અને પ્રોસેસિંગની જરૂર પડે છે \\
\end{longtable}
}

\textbf{ફાયદા vs ગેરફાયદા:}

{\def\LTcaptype{none} % do not increment counter
\begin{longtable}[]{@{}ll@{}}
\toprule\noalign{}
ફાયદા & ગેરફાયદા \\
\midrule\noalign{}
\endhead
\bottomrule\noalign{}
\endlastfoot
\textbf{નોઇઝ ઇમ્યુનિટી} & ઊંચી બેન્ડવિડ્થની જરૂરિયાતો \\
\textbf{સરળ મલ્ટિપ્લેક્સિંગ} & જટિલ ઉપકરણો \\
\textbf{એરર ડિટેક્શન \& કરેક્શન} & ક્વોન્ટાઇઝેશન એરર \\
\textbf{વધુ સુરક્ષા} & સિંક્રોનાઇઝેશન સમસ્યાઓ \\
\textbf{સિગ્નલ રિજનરેશન} & ઊંચી પ્રારંભિક કિંમત \\
\textbf{કોમ્પ્યુટર સાથે ઇન્ટિગ્રેશન} & સેમ્પલિંગ રેટની મર્યાદાઓ \\
\end{longtable}
}

\end{solutionbox}
\begin{mnemonicbox}
``NEEDS'' - નોઇઝ, ઇક્વિપમેન્ટ, એન્ડ એન્વાયરન્મેન્ટ ડિટરમાઇન
સક્સેસ

\end{mnemonicbox}
\subsection*{પ્રશ્ન 2(અ) [3
ગુણ]}\label{uxaaauxab0uxab6uxaa8-2uxa85-3-uxa97uxaa3}

\textbf{બ્લોક ડાયાગ્રામ સાથે QPSK મોડ્યુલેટરનું વર્ણન કરો}

\begin{solutionbox}

\textbf{આકૃતિ: QPSK મોડ્યુલેટર}

\begin{verbatim}
              +{-{-}{-}{-}{-}{-}{-}{-}{-}{-}{-}{-}+}
              |  2{-bit     |          +{-}{-}{-}{-}{-}{-}{-}{-}{-}{-}{-}+}
input         | Serial{-to{-} |{-}{-}bit 1{-}{-}| Cos       |}
data {-{-}{-}{-}{-}{-}{-}{-}| Parallel   |          | Carrier   |{-}{-}{-}{-}+}
              | Converter  |          +{-{-}{-}{-}{-}{-}{-}{-}{-}{-}{-}+    |   QPSK}
              |            |                           +{-{-}Signal}
              |            |          +{-{-}{-}{-}{-}{-}{-}{-}{-}{-}{-}+    |   Output}
              |            |{-{-}bit 2{-}{-}| Sin       |{-}{-}{-}{-}+}
              +{-{-}{-}{-}{-}{-}{-}{-}{-}{-}{-}{-}+          | Carrier   |}
                                      +{-{-}{-}{-}{-}{-}{-}{-}{-}{-}{-}+}
\end{verbatim}

\textbf{મુખ્ય ઘટકો:}

\begin{itemize}
\tightlist
\item
  \textbf{સીરિયલ-ટુ-પેરેલલ કન્વર્ટર}: ડેટાને 2-બિટ ગ્રુપ્સમાં વિભાજિત કરે છે
\item
  \textbf{કોસાઇન કેરિયર}: પ્રથમ બિટને મોડ્યુલેટ કરે છે (I-ચેનલ)
\item
  \textbf{સાઇન કેરિયર}: બીજા બિટને મોડ્યુલેટ કરે છે (Q-ચેનલ)
\end{itemize}

\end{solutionbox}
\begin{mnemonicbox}
``સ્પ્લિટ પેર, કેરિયર સ્ક્વેર'' - ડેટા જોડી (પેર)માં વહેંચાય છે,
ચોરસ સિગ્નલ્સ દ્વારા વહન થાય છે

\end{mnemonicbox}
\subsection*{પ્રશ્ન 2(બ) [4
ગુણ]}\label{uxaaauxab0uxab6uxaa8-2uxaac-4-uxa97uxaa3}

\textbf{બ્લોક ડાયાગ્રામ સાથે ASK મોડ્યુલેટરનું વર્ણન કરો}

\begin{solutionbox}

\textbf{આકૃતિ: ASK મોડ્યુલેટર}

\begin{verbatim}
              +{-{-}{-}{-}{-}{-}{-}{-}{-}{-}{-}{-}+     +{-}{-}{-}{-}{-}{-}{-}{-}{-}+}
Digital       | Product    |     |         |
Input {-{-}{-}{-}{-}{-}{-}| Modulator  |{-}{-}{-}{-}| Filter  |{-}{-}{-} ASK Signal}
              | (Mixer)    |     |         |
              +{-{-}{-}{-}{-}{-}{-}{-}{-}{-}{-}{-}+     +{-}{-}{-}{-}{-}{-}{-}{-}{-}+}
                     \^{}
                     |
              +{-{-}{-}{-}{-}{-}{-}{-}{-}{-}{-}{-}+}
              | Carrier    |
              | Oscillator |
              +{-{-}{-}{-}{-}{-}{-}{-}{-}{-}{-}{-}+}
\end{verbatim}

\textbf{ASK મોડ્યુલેશન પ્રક્રિયા:}

{\def\LTcaptype{none} % do not increment counter
\begin{longtable}[]{@{}ll@{}}
\toprule\noalign{}
ઘટક & કાર્ય \\
\midrule\noalign{}
\endhead
\bottomrule\noalign{}
\endlastfoot
\textbf{ડિજિટલ ઇનપુટ} & ટ્રાન્સમિટ કરવાના બાઇનરી ડેટા (0 અને 1) \\
\textbf{કેરિયર ઓસિલેટર} & ઉચ્ચ ફ્રીક્વન્સી સાઇન વેવ જનરેટ કરે છે \\
\textbf{પ્રોડક્ટ મોડ્યુલેટર} & ઇનપુટને કેરિયર સાથે ગુણે છે (ON/OFF) \\
\textbf{ફિલ્ટર} & અનિચ્છનીય ફ્રીક્વન્સી ઘટકોને દૂર કરે છે \\
\end{longtable}
}

\end{solutionbox}
\begin{mnemonicbox}
``એમ્પ્લિફાય સિગ્નલ વેન કીન'' - સિગ્નલ હાઈ હોય ત્યારે કેરિયર
એમ્પ્લિટ્યુડ બદલાય છે

\end{mnemonicbox}
\subsection*{પ્રશ્ન 2(ક) [7
ગુણ]}\label{uxaaauxab0uxab6uxaa8-2uxa95-7-uxa97uxaa3}

\textbf{ASK, FSK અને PSK ની સરખામણી કરો અને ઇનપુટ ડિજિટલ સિગ્નલ 100101000101
માટે ASK, FSK અને PSK ના વેવ ફોર્મ દોરો}

\begin{solutionbox}

\textbf{તુલનાત્મક કોષ્ટક:}

{\def\LTcaptype{none} % do not increment counter
\begin{longtable}[]{@{}llll@{}}
\toprule\noalign{}
પેરામીટર & ASK & FSK & PSK \\
\midrule\noalign{}
\endhead
\bottomrule\noalign{}
\endlastfoot
\textbf{મોડ્યુલેશન પેરામીટર} & એમ્પ્લિટ્યુડ & ફ્રીક્વન્સી & ફેઝ \\
\textbf{નોઇઝ ઇમ્યુનિટી} & ખરાબ & મધ્યમ & સારું \\
\textbf{બેન્ડવિડ્થ} & સાંકડું & વિશાળ & મધ્યમ \\
\textbf{પાવર એફિશિયન્સી} & ખરાબ & મધ્યમ & સારું \\
\textbf{ઇમ્પ્લિમેન્ટેશન} & સરળ & મધ્યમ & જટિલ \\
\textbf{BER પરફોર્મન્સ} & ખરાબ & મધ્યમ & સારું \\
\end{longtable}
}

\textbf{ઇનપુટ 100101000101 માટે વેવફોર્મ્સ:}

\begin{verbatim}
Digital: ‾‾‾\_‾\_‾‾\_\_\_‾\_‾  (1 0 0 1 0 1 0 0 0 1 0 1)
         
ASK:     ✓✓✓\_\_\_✓\_\_\_✓✓✓\_\_\_✓\_\_\_✓✓✓
         high low low high low high low low low high low high

FSK:     ✓✓✓{✓✓✓✓✓✓✓✓✓✓✓✓}
         f1  f2  f2  f1  f2  f1  f2  f2  f2  f1  f2  f1

PSK:     ✓✓✓˜˜˜✓✓✓˜˜˜✓✓✓˜˜˜✓✓✓˜˜˜✓✓✓
         0^  180^ 180^ 0^  180^ 0^  180^ 180^ 180^ 0^  180^ 0^
\end{verbatim}

\end{solutionbox}
\begin{mnemonicbox}
``AFP - ઓલ્ટર ફ્રીક્વન્સીઝ ઓર ફેઝિસ'' - મોડ્યુલેશન પ્રકારો
યાદ રાખવા માટે

\end{mnemonicbox}
\subsection*{પ્રશ્ન 2(અ OR) [3
ગુણ]}\label{uxaaauxab0uxab6uxaa8-2uxa85-or-3-uxa97uxaa3}

\textbf{બ્લોક ડાયાગ્રામ સાથે QPSK ડિમોડ્યુલેટરનું વર્ણન કરો}

\begin{solutionbox}

\textbf{આકૃતિ: QPSK ડિમોડ્યુલેટર}

\begin{verbatim}
                +{-{-}{-}{-}{-}{-}{-}{-}{-}{-}{-}+}
                | Cos       |
                | Carrier   |{-{-}+}
                +{-{-}{-}{-}{-}{-}{-}{-}{-}{-}{-}+  |}
                               v
QPSK      +{-{-}{-}{-}{-}+      +{-}{-}{-}{-}{-}{-}{-}{-}{-}{-}{-}{-}+     +{-}{-}{-}{-}{-}{-}{-}{-}{-}+}
Signal{-{-}{-}| BPF |{-}{-}{-}{-}{-}| Product    |{-}{-}{-}{-}| LPF     |{-}{-}{-}{-} Bit 1}
          +{-{-}{-}{-}{-}+      | Detect     |     +{-}{-}{-}{-}{-}{-}{-}{-}{-}+}
                       +{-{-}{-}{-}{-}{-}{-}{-}{-}{-}{-}{-}+}
                             \^{}
                             |      +{-{-}{-}{-}{-}{-}{-}{-}{-}{-}{-}{-}+    +{-}{-}{-}{-}{-}{-}{-}{-}{-}+}
                             +{-{-}{-}{-}{-}| Product    |{-}{-}{-}| LPF     |{-}{-}{-}{-} Bit 2}
                                    | Detect     |    +{-{-}{-}{-}{-}{-}{-}{-}{-}+}
                             +      +{-{-}{-}{-}{-}{-}{-}{-}{-}{-}{-}{-}+}
                             |
                        +{-{-}{-}{-}{-}{-}{-}{-}{-}{-}{-}+}
                        | Sin       |
                        | Carrier   |
                        +{-{-}{-}{-}{-}{-}{-}{-}{-}{-}{-}+}
\end{verbatim}

\textbf{મુખ્ય ઘટકો:}

\begin{itemize}
\tightlist
\item
  \textbf{BPF (બેન્ડપાસ ફિલ્ટર)}: સિગ્નલ બેન્ડવિડ્થ બહારના નોઇઝને દૂર કરે છે
\item
  \textbf{પ્રોડક્ટ ડિટેક્ટર્સ}: કેરિયર સિગ્નલ્સ (cos \& sin) સાથે ગુણાકાર કરે છે
\item
  \textbf{LPF (લોપાસ ફિલ્ટર્સ)}: મૂળ ડેટા બિટ્સને અલગ કરે છે
\end{itemize}

\end{solutionbox}
\begin{mnemonicbox}
``ફિલ્ટર્ડ પેર્સ ડિલિવર ડેટા'' - ફિલ્ટર્સ અને જોડી કેરિયર્સ
ડેટા પુનઃપ્રાપ્ત કરે છે

\end{mnemonicbox}
\subsection*{પ્રશ્ન 2(બ) [4
ગુણ]}\label{uxaaauxab0uxab6uxaa8-2uxaac-4-uxa97uxaa3-1}

\textbf{ASK, BPSK અને QPSK ના નક્ષત્ર રેખાકૃતિ દોરો}

\begin{solutionbox}

\textbf{નક્ષત્ર આકૃતિઓ:}

\begin{verbatim}
ASK Constellation:      BPSK Constellation:     QPSK Constellation:
       
       |                      |                      |
       |                      |                      |      * 01
       |                      |                      |
       |                      |                      |
{-{-}{-}{-}{-}{-}{-}+{-}{-}{-}{-}{-}{-}{-}       {-}{-}{-}{-}{-}{-}{-}+{-}{-}{-}{-}{-}{-}{-}       {-}{-}{-}{-}{-}{-}{-}+{-}{-}{-}{-}{-}{-}{-}}
       |                      |                      |
       |                      |                      |
   *   |               *      |      *         *     |     *
 (0)   |  * (1)         (1)   |    (0)        10     |    00
       |                      |                      |
Q axis |                Q axis|                Q axis|
       I axis                 I axis                 I axis
\end{verbatim}


{\def\LTcaptype{none} % do not increment counter
\vspace{-5pt}
\captionof{table}{નક્ષત્ર આકૃતિઓની લક્ષણો}
\vspace{-10pt}
\begin{longtable}[]{@{}llll@{}}
\toprule\noalign{}
મોડ્યુલેશન & પોઇન્ટ્સ & ફેઝ સ્ટેટ્સ & એમ્પ્લિટ્યુડ સ્ટેટ્સ \\
\midrule\noalign{}
\endhead
\bottomrule\noalign{}
\endlastfoot
\textbf{ASK} & 2 & 1 (0^\circ) & 2 (0, A) \\
\textbf{BPSK} & 2 & 2 (0^\circ, 180^\circ) & 1 (A) \\
\textbf{QPSK} & 4 & 4 (45^\circ, 135^\circ, 225^\circ, 315^\circ) & 1 (A) \\
\end{longtable}
}

\end{solutionbox}
\begin{mnemonicbox}
``પોઇન્ટ્સ ડબલ વેન ફેઝિસ ડબલ'' - BPSK માં 2 પોઇન્ટ્સ છે,
QPSK માં 4 પોઇન્ટ્સ છે

\end{mnemonicbox}
\subsection*{પ્રશ્ન 2(ક) [7
ગુણ]}\label{uxaaauxab0uxab6uxaa8-2uxa95-7-uxa97uxaa3-1}

\textbf{બ્લોક ડાયાગ્રામ અને આઉટપુટ વેવ ફોર્મ સાથે FSK મોડ્યુલેટર અને ડિમોડ્યુલેટરનું
વર્ણન કરો}

\begin{solutionbox}

\textbf{FSK મોડ્યુલેટર આકૃતિ:}

\begin{verbatim}
                         +{-{-}{-}{-}{-}{-}{-}{-}{-}+}
             +{-{-}{-}{-}{-}{-}{-}{-}+  |         |}
     {1 {-}{-}{-}| Switch |{-}| Osc f1  |{-}{-}+}
             |        |  |         |  |
Digital      +{-{-}{-}{-}{-}{-}{-}{-}+  +{-}{-}{-}{-}{-}{-}{-}{-}{-}+  |     +{-}{-}{-}{-}{-}{-}{-}{-}{-}+}
Input {-{-}{-}+                            +{-}{-}{-}{-}|         |}
          |                                 | Adder   |{-{-}{-} FSK Signal}
          |   +{-{-}{-}{-}{-}{-}{-}{-}+  +{-}{-}{-}{-}{-}{-}{-}{-}{-}+  +{-}{-}{-}|         |}
     {0 {-}+{-}{-}| Switch |{-}| Osc f2  |{-}{-}+    +{-}{-}{-}{-}{-}{-}{-}{-}{-}+}
              |        |  |         |
              +{-{-}{-}{-}{-}{-}{-}{-}+  +{-}{-}{-}{-}{-}{-}{-}{-}{-}+}
\end{verbatim}

\textbf{FSK ડિમોડ્યુલેટર આકૃતિ:}

\begin{verbatim}
                +{-{-}{-}{-}{-}{-}{-}{-}{-}+  +{-}{-}{-}{-}{-}{-}{-}{-}{-}+  +{-}{-}{-}{-}{-}{-}{-}{-}{-}+}
                |         |  |         |  |         |
                | BPF f1  |{-| Env     |{-}| Thresh  |{-}{-}+}
                |         |  | Detect  |  | Detect  |  |
                +{-{-}{-}{-}{-}{-}{-}{-}{-}+  +{-}{-}{-}{-}{-}{-}{-}{-}{-}+  +{-}{-}{-}{-}{-}{-}{-}{-}{-}+  |}
                                                       |  +{-{-}{-}{-}{-}{-}{-}{-}{-}+}
FSK Signal {-{-}+                                         +{-}|         |}
             |                                            | Logic   |{-{-}{-} Digital}
             |  +{-{-}{-}{-}{-}{-}{-}{-}{-}+  +{-}{-}{-}{-}{-}{-}{-}{-}{-}+  +{-}{-}{-}{-}{-}{-}{-}{-}{-}+  +{-}| Circuit |     Output}
             |  |         |  |         |  |         |  |  |         |
             +{-| BPF f2  |{-}| Env     |{-}| Thresh  |{-}{-}+  +{-}{-}{-}{-}{-}{-}{-}{-}{-}+}
                |         |  | Detect  |  | Detect  |
                +{-{-}{-}{-}{-}{-}{-}{-}{-}+  +{-}{-}{-}{-}{-}{-}{-}{-}{-}+  +{-}{-}{-}{-}{-}{-}{-}{-}{-}+}
\end{verbatim}

\textbf{FSK વેવફોર્મ:}

\begin{verbatim}
Digital:  \_\_\_‾‾‾\_\_\_
          0  1  0

FSK:      {}
          f2 f1 f2
          Low freq when 0
          High freq when 1
\end{verbatim}

\textbf{મુખ્ય ઘટકો:}

{\def\LTcaptype{none} % do not increment counter
\begin{longtable}[]{@{}ll@{}}
\toprule\noalign{}
ઘટક & કાર્ય \\
\midrule\noalign{}
\endhead
\bottomrule\noalign{}
\endlastfoot
\textbf{ઓસિલેટર્સ} & 0 અને 1 માટે અલગ ફ્રીક્વન્સી જનરેટ કરે છે \\
\textbf{બેન્ડપાસ ફિલ્ટર્સ} & બે ફ્રીક્વન્સીઓને અલગ કરે છે \\
\textbf{એન્વેલોપ ડિટેક્ટર્સ} & એમ્પ્લિટ્યુડ વેરિએશન્સ અલગ કરે છે \\
\textbf{થ્રેશોલ્ડ ડિટેક્ટર્સ} & એનાલોગને ડિજિટલમાં કન્વર્ટ કરે છે \\
\end{longtable}
}

\end{solutionbox}
\begin{mnemonicbox}
``ફ્રીક્વન્સી શિફ્ટ કી - ટુ ટોન્સ ટેલ ટ્રુથ''

\end{mnemonicbox}
\subsection*{પ્રશ્ન 3(અ) [3
ગુણ]}\label{uxaaauxab0uxab6uxaa8-3uxa85-3-uxa97uxaa3}

\textbf{સંચારમાં સંભાવનાનું મહત્વ જણાવો}

\begin{solutionbox}

{\def\LTcaptype{none} % do not increment counter
\begin{longtable}[]{@{}ll@{}}
\toprule\noalign{}
મહત્વ & વર્ણન \\
\midrule\noalign{}
\endhead
\bottomrule\noalign{}
\endlastfoot
\textbf{ઇન્ફોર્મેશન મેઝરમેન્ટ} & મેસેજમાં અનિશ્ચિતતા/આશ્ચર્યને ક્વાન્ટિફાય કરે છે \\
\textbf{ચેનલ કેપેસિટી} & શક્ય મહત્તમ ડેટા રેટ નિર્ધારિત કરે છે \\
\textbf{એરર એનાલિસિસ} & કોમ્યુનિકેશન એરર્સની આગાહી કરે છે અને ન્યૂનતમ કરે છે \\
\end{longtable}
}

\end{solutionbox}
\begin{mnemonicbox}
``ICE - ઇન્ફોર્મેશન, કેપેસિટી, એરર્સ'' ને સંભાવનાની જરૂર પડે છે

\end{mnemonicbox}
\subsection*{પ્રશ્ન 3(બ) [4
ગુણ]}\label{uxaaauxab0uxab6uxaa8-3uxaac-4-uxa97uxaa3}

\textbf{SNR ના સંદર્ભમાં રાજ્ય ચેનલ ક્ષમતા અને તેનું મહત્વ સમજાવો}

\begin{solutionbox}

\textbf{શેનન ચેનલ કેપેસિટી ફોર્મ્યુલા:}

\begin{verbatim}
C = B \times log_{2}(1 + SNR)
\end{verbatim}

\textbf{જ્યાં:}

\begin{itemize}
\tightlist
\item
  C = ચેનલ કેપેસિટી (બિટ્સ/સેકન્ડ)
\item
  B = બેન્ડવિડ્થ (Hz)
\item
  SNR = સિગ્નલ-ટુ-નોઇઝ રેશિયો
\end{itemize}

\textbf{મહત્વ:}

{\def\LTcaptype{none} % do not increment counter
\begin{longtable}[]{@{}ll@{}}
\toprule\noalign{}
પાસું & મહત્વ \\
\midrule\noalign{}
\endhead
\bottomrule\noalign{}
\endlastfoot
\textbf{થિયોરેટિકલ લિમિટ} & એરર-ફ્રી ડેટા રેટની મહત્તમ શક્ય સીમા નિર્ધારિત કરે
છે \\
\textbf{સિસ્ટમ ડિઝાઇન} & બેન્ડવિડ્થ અને પાવર જરૂરિયાતોનું માર્ગદર્શન આપે છે \\
\textbf{પરફોર્મન્સ ઇવેલ્યુએશન} & વાસ્તવિક સિસ્ટમ પરફોર્મન્સ માટે બેન્ચમાર્ક \\
\textbf{કોડિંગ એફિશિયન્સી} & દર્શાવે છે કે સિસ્ટમ ઓપ્ટિમલ પરફોર્મન્સથી કેટલી નજીક
છે \\
\end{longtable}
}

\end{solutionbox}
\begin{mnemonicbox}
``BEST'' - બેન્ડવિડ્થ એન્ડ એરર-ફ્રી સિગ્નલ ટ્રાન્સમિશન

\end{mnemonicbox}
\subsection*{પ્રશ્ન 3(ક) [7
ગુણ]}\label{uxaaauxab0uxab6uxaa8-3uxa95-7-uxa97uxaa3}

\textbf{યોગ્ય ઉદાહરણ સાથે લાઇન કોડના વર્ગીકરણની ચર્ચા કરો}

\begin{solutionbox}

\textbf{આકૃતિ: લાઇન કોડ વર્ગીકરણ}

\begin{center}
\textbf{Mermaid Diagram (Code)}
\begin{verbatim}
{Shaded}
{Highlighting}[]
graph TD
    A[લાઇન કોડ્સ] {-{-}{} B[યુનિપોલર]}
    A {-{-}{} C[પોલર]}
    A {-{-}{} D[બાયપોલર]}
    B {-{-}{} B1[NRZ]}
    B {-{-}{} B2[RZ]}
    C {-{-}{} C1[NRZ]}
    C {-{-}{} C2[RZ]}
    D {-{-}{} D1[AMI]}
    D {-{-}{} D2[સ્યુડોટર્નરી]}
{Highlighting}
{Shaded}
\end{verbatim}
\end{center}

\textbf{લાઇન કોડ ઉદાહરણો:}

\begin{center}
\textbf{Mermaid Diagram (Code)}
\begin{verbatim}
{Shaded}
{Highlighting}[]
graph TD
    subgraph "ડિજિટલ ડેટા"
    D["1  0  1  1  0  1  0  0"]
    end

    subgraph "યુનિપોલર NRZ"
    U["હાઈ  લો  હાઈ  હાઈ  લો  હાઈ  લો  લો"]
    end
    
    subgraph "પોલર NRZ"
    P["+V   {-V   +V   +V   {-}V   +V   {-}V   {-}V"]}
    end
    
    subgraph "બાયપોલર AMI"
    B["+V   0   {-V   +V   0   {-}V   0    0"]}
    end
{Highlighting}
{Shaded}
\end{verbatim}
\end{center}

\textbf{વેવફોર્મ વિઝ્યુલાઇઝેશન:}

\begin{verbatim}
Data:       1    0    1    1    0    1    0    0
           \_|\_   |   \_|\_  \_|\_   |   \_|\_   |    |

Unipolar   ‾‾‾‾‾     ‾‾‾‾‾‾‾‾‾     ‾‾‾‾‾
NRZ:       \_\_\_\_\_‾‾‾‾‾\_\_\_\_\_‾‾‾‾‾‾‾‾‾\_\_\_\_\_‾‾‾‾‾‾‾‾‾‾‾‾‾

Polar      ‾‾‾‾‾\_\_\_\_\_‾‾‾‾‾‾‾‾‾\_\_\_\_\_‾‾‾‾‾\_\_\_\_\_\_\_\_\_\_\_\_\_\_\_
NRZ:       

Bipolar    ‾‾‾‾‾     \_\_\_\_\_‾‾‾‾‾     
AMI:       \_\_\_\_\_‾‾‾‾‾     \_\_\_\_\_‾‾‾‾‾‾‾‾‾‾‾‾‾‾‾‾‾‾‾‾‾‾‾
           (+ for first 1, {- for second 1, etc.)}
\end{verbatim}

\textbf{તુલનાત્મક કોષ્ટક:}

{\def\LTcaptype{none} % do not increment counter
\begin{longtable}[]{@{}
  >{\raggedright\arraybackslash}p{(\linewidth - 8\tabcolsep) * \real{0.2222}}
  >{\raggedright\arraybackslash}p{(\linewidth - 8\tabcolsep) * \real{0.2083}}
  >{\raggedright\arraybackslash}p{(\linewidth - 8\tabcolsep) * \real{0.1944}}
  >{\raggedright\arraybackslash}p{(\linewidth - 8\tabcolsep) * \real{0.2222}}
  >{\raggedright\arraybackslash}p{(\linewidth - 8\tabcolsep) * \real{0.1528}}@{}}
\toprule\noalign{}
\begin{minipage}[b]{\linewidth}\raggedright
લાઇન કોડ પ્રકાર
\end{minipage} & \begin{minipage}[b]{\linewidth}\raggedright
સિગ્નલ લેવલ્સ
\end{minipage} & \begin{minipage}[b]{\linewidth}\raggedright
DC કોમ્પોનેન્ટ
\end{minipage} & \begin{minipage}[b]{\linewidth}\raggedright
ક્લોક રિકવરી
\end{minipage} & \begin{minipage}[b]{\linewidth}\raggedright
બેન્ડવિડ્થ
\end{minipage} \\
\midrule\noalign{}
\endhead
\bottomrule\noalign{}
\endlastfoot
\textbf{યુનિપોલર NRZ} & 0, +A & હા & ખરાબ & સાંકડું \\
\textbf{પોલર NRZ} & -A, +A & કદાચ & ખરાબ & મધ્યમ \\
\textbf{બાયપોલર AMI} & -A, 0, +A & ના & સારું & વિશાળ \\
\end{longtable}
}

\end{solutionbox}
\begin{mnemonicbox}
``UPB - યુઝ પ્રોપર બિટ્સ'' - યુનિપોલર, પોલર, બાયપોલર માટે

\end{mnemonicbox}
\subsection*{પ્રશ્ન 3(અ OR) [3
ગુણ]}\label{uxaaauxab0uxab6uxaa8-3uxa85-or-3-uxa97uxaa3}

\textbf{શરતી સંભાવનાની ચર્ચા કરો}

\begin{solutionbox}

\textbf{શરતી સંભાવના વ્યાખ્યા:}

\begin{verbatim}
P(A|B) = P(A\capB) / P(B)
\end{verbatim}


{\def\LTcaptype{none} % do not increment counter
\vspace{-5pt}
\captionof{table}{કોમ્યુનિકેશનમાં શરતી સંભાવના}
\vspace{-10pt}
\begin{longtable}[]{@{}ll@{}}
\toprule\noalign{}
એપ્લિકેશન & વર્ણન \\
\midrule\noalign{}
\endhead
\bottomrule\noalign{}
\endlastfoot
\textbf{ચેનલ મોડેલિંગ} & X મોકલવામાં આવ્યું હોય તો Y પ્રાપ્ત થવાની સંભાવના \\
\textbf{એરર ડિટેક્શન} & ચોક્કસ પેટર્ન આપેલી હોય તે સંજોગોમાં એરર થવાની સંભાવના \\
\textbf{નિર્ણય લેવો} & અવલોકનોના આધારે રિસીવર નિર્ણયને ઓપ્ટિમાઇઝ કરવું \\
\end{longtable}
}

\end{solutionbox}
\begin{mnemonicbox}
``CEaD'' - કેલ્ક્યુલેટ ઇવેન્ટ્સ આફ્ટર ડેટા

\end{mnemonicbox}
\subsection*{પ્રશ્ન 3(બ) [4
ગુણ]}\label{uxaaauxab0uxab6uxaa8-3uxaac-4-uxa97uxaa3-1}

\textbf{એન્ટ્રોપી અને માહિતી વ્યાખ્યાયિત કરો. તેના ભૌતિક મહત્વની ચર્ચા કરો}

\begin{solutionbox}

\textbf{વ્યાખ્યાઓ:}

{\def\LTcaptype{none} % do not increment counter
\begin{longtable}[]{@{}lll@{}}
\toprule\noalign{}
શબ્દ & વ્યાખ્યા & ફોર્મ્યુલા \\
\midrule\noalign{}
\endhead
\bottomrule\noalign{}
\endlastfoot
\textbf{એન્ટ્રોપી} & સોર્સમાં સરેરાશ માહિતી સામગ્રી & H(X) = -\sumP(x)log_{2}P(x) \\
\textbf{માહિતી} & અનિશ્ચિતતા ઘટાડાનું માપ & I(x) = log_{2}(1/P(x)) \\
\end{longtable}
}

\textbf{ભૌતિક મહત્વ:}

{\def\LTcaptype{none} % do not increment counter
\begin{longtable}[]{@{}ll@{}}
\toprule\noalign{}
પાસું & મહત્વ \\
\midrule\noalign{}
\endhead
\bottomrule\noalign{}
\endlastfoot
\textbf{અનપ્રેડિક્ટેબિલિટી} & ઊંચી એન્ટ્રોપીનો અર્થ છે ઓછો પ્રેડિક્ટેબલ સોર્સ \\
\textbf{કોમ્પ્રેશન લિમિટ} & સોર્સને રજૂ કરવા માટે જરૂરી ન્યૂનતમ બિટ્સ \\
\textbf{ઓપ્ટિમલ કોડિંગ} & કાર્યક્ષમ સોર્સ કોડિંગ ડિઝાઇનનું માર્ગદર્શન આપે છે \\
\textbf{રિસોર્સ એલોકેશન} & બેન્ડવિડ્થ/પાવર જરૂરિયાતો નક્કી કરે છે \\
\end{longtable}
}

\end{solutionbox}
\begin{mnemonicbox}
``UCOR'' - અનસર્ટેનીટી કોરિલેટ્સ વિથ ઓપ્ટિમલ રિસોર્સિસ

\end{mnemonicbox}
\subsection*{પ્રશ્ન 3(ક) [7
ગુણ]}\label{uxaaauxab0uxab6uxaa8-3uxa95-7-uxa97uxaa3-1}

\textbf{યોગ્ય ઉદાહરણ સાથે હફમેન કોડનું વર્ણન કરો}

\begin{solutionbox}

\textbf{હફમેન કોડિંગ: લોસલેસ ડેટા કોમ્પ્રેશન માટે વેરિએબલ-લેન્થ પ્રીફિક્સ કોડ}

\textbf{ઉદાહરણ: સિમ્બોલ્સ \{A, B, C, D, E\} એન્કોડિંગ}

\textbf{સ્ટેપ 1: સંભાવના ગણતરી}

{\def\LTcaptype{none} % do not increment counter
\begin{longtable}[]{@{}ll@{}}
\toprule\noalign{}
સિમ્બોલ & સંભાવના \\
\midrule\noalign{}
\endhead
\bottomrule\noalign{}
\endlastfoot
A & 0.4 \\
B & 0.2 \\
C & 0.2 \\
D & 0.1 \\
E & 0.1 \\
\end{longtable}
}

\textbf{સ્ટેપ 2: હફમેન ટ્રી બનાવો}

\begin{center}
\textbf{Mermaid Diagram (Code)}
\begin{verbatim}
{Shaded}
{Highlighting}[]
graph LR
    A["1.0"] {-{-}{} B["0.6"]}
    A {-{-}{} C["0.4 (A)"]}
    C {-{-}{} C1["0"]}
    B {-{-}{} D["0.3"]}
    B {-{-}{} E["0.3"]}
    D {-{-}{} F["0.2 (B)"]}
    D {-{-}{} G["0.1 (E)"]}
    F {-{-}{} F1["0"]}
    G {-{-}{} G1["1"]}
    E {-{-}{} H["0.1 (D)"]}
    E {-{-}{} I["0.2 (C)"]}
    H {-{-}{} H1["1"]}
    I {-{-}{} I1["0"]}
{Highlighting}
{Shaded}
\end{verbatim}
\end{center}

\textbf{સ્ટેપ 3: કોડ્સ અસાઇન કરો}

{\def\LTcaptype{none} % do not increment counter
\begin{longtable}[]{@{}lll@{}}
\toprule\noalign{}
સિમ્બોલ & સંભાવના & હફમેન કોડ \\
\midrule\noalign{}
\endhead
\bottomrule\noalign{}
\endlastfoot
A & 0.4 & 0 \\
B & 0.2 & 10 \\
C & 0.2 & 11 \\
D & 0.1 & 100 \\
E & 0.1 & 101 \\
\end{longtable}
}

\textbf{સરેરાશ કોડ લંબાઈ:} (0.4\times1) + (0.2\times2) + (0.2\times2) + (0.1\times3) + (0.1\times3)
= 1.8 બિટ્સ/સિમ્બોલ

\end{solutionbox}
\begin{mnemonicbox}
``હાઈ પ્રોબ, લો બિટ્સ'' - ઊંચી સંભાવના ધરાવતા સિમ્બોલ્સને
ટૂંકા કોડ મળે છે

\end{mnemonicbox}
\subsection*{પ્રશ્ન 4(અ) [3
ગુણ]}\label{uxaaauxab0uxab6uxaa8-4uxa85-3-uxa97uxaa3}

\textbf{ડેટા ટ્રાન્સમિશન તકનીકોની સૂચિ બનાવો}

\begin{solutionbox}


{\def\LTcaptype{none} % do not increment counter
\vspace{-5pt}
\captionof{table}{ડેટા ટ્રાન્સમિશન તકનીકો}
\vspace{-10pt}
\begin{longtable}[]{@{}
  >{\raggedright\arraybackslash}p{(\linewidth - 2\tabcolsep) * \real{0.4583}}
  >{\raggedright\arraybackslash}p{(\linewidth - 2\tabcolsep) * \real{0.5417}}@{}}
\toprule\noalign{}
\begin{minipage}[b]{\linewidth}\raggedright
તકનીક
\end{minipage} & \begin{minipage}[b]{\linewidth}\raggedright
વર્ણન
\end{minipage} \\
\midrule\noalign{}
\endhead
\bottomrule\noalign{}
\endlastfoot
\textbf{સીરિયલ ટ્રાન્સમિશન} & સિંગલ ચેનલ પર એક પછી એક બિટ્સ મોકલવામાં આવે છે \\
\textbf{પેરેલલ ટ્રાન્સમિશન} & મલ્ટિપલ ચેનલ્સ પર એકસાથે મલ્ટિપલ બિટ્સ મોકલવામાં આવે
છે \\
\textbf{સિંક્રોનસ ટ્રાન્સમિશન} & ક્લોક દ્વારા નિયંત્રિત ટાઈમિંગ સાથે ડેટા બ્લોક્સમાં
મોકલવામાં આવે છે \\
\textbf{એસિંક્રોનસ ટ્રાન્સમિશન} & સ્ટાર્ટ/સ્ટોપ બિટ્સ સાથે ડેટા મોકલવામાં આવે છે,
કોમન ક્લોક નથી \\
\textbf{હાફ-ડુપ્લેક્સ} & ડેટા બંને દિશામાં વહે છે, પરંતુ એક સાથે નહીં \\
\textbf{ફુલ-ડુપ્લેક્સ} & ડેટા બંને દિશામાં એક સાથે વહે છે \\
\end{longtable}
}

\end{solutionbox}
\begin{mnemonicbox}
``SPASH-F'' - સીરિયલ, પેરેલલ, એસિંક્રોનસ, સિંક્રોનસ,
હાફ/ફુલ

\end{mnemonicbox}
\subsection*{પ્રશ્ન 4(બ) [4
ગુણ]}\label{uxaaauxab0uxab6uxaa8-4uxaac-4-uxa97uxaa3}

\textbf{સંચાર માટે મલ્ટીમીડિયા પ્રોસેસિંગની જરૂરિયાતો સમજાવો}

\begin{solutionbox}

\textbf{મલ્ટીમીડિયા પ્રોસેસિંગ જરૂરિયાતો:}

{\def\LTcaptype{none} % do not increment counter
\begin{longtable}[]{@{}
  >{\raggedright\arraybackslash}p{(\linewidth - 2\tabcolsep) * \real{0.3158}}
  >{\raggedright\arraybackslash}p{(\linewidth - 2\tabcolsep) * \real{0.6842}}@{}}
\toprule\noalign{}
\begin{minipage}[b]{\linewidth}\raggedright
જરૂરિયાત
\end{minipage} & \begin{minipage}[b]{\linewidth}\raggedright
વર્ણન
\end{minipage} \\
\midrule\noalign{}
\endhead
\bottomrule\noalign{}
\endlastfoot
\textbf{કોમ્પ્રેશન} & મોટી મીડિયા ફાઇલો માટે બેન્ડવિડ્થ જરૂરિયાતો ઘટાડે છે \\
\textbf{ફોર્મેટ સ્ટાન્ડર્ડાઇઝેશન} & જુદા જુદા સિસ્ટમો વચ્ચે સુસંગતતા સુનિશ્ચિત કરે છે \\
\textbf{ક્વોલિટી કંટ્રોલ} & સ્વીકાર્ય ઓડિયો/વિડિયો ક્વોલિટી સ્તર જાળવે છે \\
\textbf{સિંક્રોનાઇઝેશન} & જુદા જુદા મીડિયા પ્રકારો (ઓડિયો, વિડિયો, ટેક્સ્ટ) સંકલિત
કરે છે \\
\textbf{એરર રેસિસ્ટન્સ} & ટ્રાન્સમિશન દરમિયાન ડેટા લોસથી રક્ષણ કરે છે \\
\end{longtable}
}

\textbf{આકૃતિ: મલ્ટીમીડિયા પ્રોસેસિંગ ફ્લો}

\begin{center}
\textbf{Mermaid Diagram (Code)}
\begin{verbatim}
{Shaded}
{Highlighting}[]
graph LR
    A[રો મીડિયા] {-{-}{} B[કોમ્પ્રેશન]}
    B {-{-}{} C[ફોર્મેટ કન્વર્ઝન]}
    C {-{-}{} D[એરર પ્રોટેક્શન]}
    D {-{-}{} E[ટ્રાન્સમિશન]}
    E {-{-}{} F[એરર કરેક્શન]}
    F {-{-}{} G[ડીકોમ્પ્રેશન]}
    G {-{-}{} H[પ્લેબેક]}
{Highlighting}
{Shaded}
\end{verbatim}
\end{center}

\end{solutionbox}
\begin{mnemonicbox}
``CQSEF'' - કોમ્પ્રેસ ક્વોલિટી, સ્ટાન્ડર્ડાઇઝ એન્ડ એન્શ્યોર
ફિડિલિટી

\end{mnemonicbox}
\subsection*{પ્રશ્ન 4(ક) [7
ગુણ]}\label{uxaaauxab0uxab6uxaa8-4uxa95-7-uxa97uxaa3}

\textbf{ડેટા ટ્રાન્સમિશન મોડ સમજાવો}

\begin{solutionbox}


{\def\LTcaptype{none} % do not increment counter
\vspace{-5pt}
\captionof{table}{ડેટા ટ્રાન્સમિશન મોડ}
\vspace{-10pt}
\begin{longtable}[]{@{}
  >{\raggedright\arraybackslash}p{(\linewidth - 6\tabcolsep) * \real{0.1622}}
  >{\raggedright\arraybackslash}p{(\linewidth - 6\tabcolsep) * \real{0.2973}}
  >{\raggedright\arraybackslash}p{(\linewidth - 6\tabcolsep) * \real{0.2973}}
  >{\raggedright\arraybackslash}p{(\linewidth - 6\tabcolsep) * \real{0.2432}}@{}}
\toprule\noalign{}
\begin{minipage}[b]{\linewidth}\raggedright
મોડ
\end{minipage} & \begin{minipage}[b]{\linewidth}\raggedright
દિશા
\end{minipage} & \begin{minipage}[b]{\linewidth}\raggedright
ઓપરેશન
\end{minipage} & \begin{minipage}[b]{\linewidth}\raggedright
ઉદાહરણ
\end{minipage} \\
\midrule\noalign{}
\endhead
\bottomrule\noalign{}
\endlastfoot
\textbf{સિમ્પ્લેક્સ} & ફક્ત એક દિશામાં & સેન્ડર રિસીવ કરી શકતો નથી & રેડિયો
બ્રોડકાસ્ટ \\
\textbf{હાફ-ડુપ્લેક્સ} & બે-દિશામાં, વારાફરતી & એક સમયે ફક્ત એક ડિવાઇસ ટ્રાન્સમિટ
કરે છે & વોકી-ટોકી \\
\textbf{ફુલ-ડુપ્લેક્સ} & બે-દિશામાં, એકસાથે & બંને ડિવાઇસિસ એક સાથે ટ્રાન્સમિટ કરે છે &
ટેલિફોન કોલ \\
\end{longtable}
}

\textbf{આકૃતિ: ડેટા ટ્રાન્સમિશન મોડ}

\begin{verbatim}
Simplex:
  A {-{-}{-}{-}{-}{-}{-}{-}{-}{-}{-}{-}{-}{-}{-}{-}{-} B}
     Data flows one way

Half{-Duplex:}
  A {{-}{-}{-}{-}{-}{-}{-}{-}{-}{-}{-}{-}{-}{-}{-}{-} B}
     Data flows in both directions,
     but only one direction at a time

Full{-Duplex:}
  A {================= B}
     Data flows in both directions
     simultaneously
\end{verbatim}

\textbf{તુલના:}

{\def\LTcaptype{none} % do not increment counter
\begin{longtable}[]{@{}llll@{}}
\toprule\noalign{}
પેરામીટર & સિમ્પ્લેક્સ & હાફ-ડુપ્લેક્સ & ફુલ-ડુપ્લેક્સ \\
\midrule\noalign{}
\endhead
\bottomrule\noalign{}
\endlastfoot
\textbf{ચેનલ ઉપયોગ} & 100\% એક દિશામાં & 100\% વારાફરતી & 100\% બંને
દિશામાં \\
\textbf{કાર્યક્ષમતા} & નીચી & મધ્યમ & ઊંચી \\
\textbf{ઇમ્પ્લિમેન્ટેશન} & સરળ & મધ્યમ & જટિલ \\
\textbf{ખર્ચ} & ઓછો & મધ્યમ & ઊંચો \\
\end{longtable}
}

\end{solutionbox}
\begin{mnemonicbox}
``SHF - સ્પીડ એન્ડ હેન્ડલિંગ ફેક્ટર્સ'' - સિમ્પ્લેક્સ,
હાફ-ડુપ્લેક્સ, ફુલ-ડુપ્લેક્સ માટે

\end{mnemonicbox}
\subsection*{પ્રશ્ન 4(અ OR) [3
ગુણ]}\label{uxaaauxab0uxab6uxaa8-4uxa85-or-3-uxa97uxaa3}

\textbf{ડેટા કમ્યુનિકેશનની મહત્વપૂર્ણ લાક્ષણિકતાઓની સૂચિ બનાવો}

\begin{solutionbox}

\textbf{ડેટા કોમ્યુનિકેશનની મુખ્ય લાક્ષણિકતાઓ:}

{\def\LTcaptype{none} % do not increment counter
\begin{longtable}[]{@{}ll@{}}
\toprule\noalign{}
લાક્ષણિકતા & વર્ણન \\
\midrule\noalign{}
\endhead
\bottomrule\noalign{}
\endlastfoot
\textbf{ડિલિવરી} & સિસ્ટમે ડેટાને યોગ્ય ડેસ્ટિનેશન પર પહોંચાડવો જોઈએ \\
\textbf{એક્યુરસી} & ડેટા ફેરફાર વિના પહોંચવો જોઈએ \\
\textbf{ટાઇમલીનેસ} & ડેટા ઉપયોગી સમય ફ્રેમની અંદર પહોંચવો જોઈએ \\
\textbf{જિટર} & પેકેટ આગમન સમયમાં વેરિએશન \\
\textbf{સિક્યોરિટી} & અનધિકૃત એક્સેસથી સુરક્ષા \\
\textbf{રિલાયબિલિટી} & નિષ્ફળતાઓ સામે સિસ્ટમ રેસિલિયન્સ \\
\end{longtable}
}

\end{solutionbox}
\begin{mnemonicbox}
``DATJSR'' - ડિલિવરી, એક્યુરસી, ટાઇમલીનેસ, જિટર,
સિક્યોરિટી, રિલાયબિલિટી

\end{mnemonicbox}
\subsection*{પ્રશ્ન 4(બ) [4
ગુણ]}\label{uxaaauxab0uxab6uxaa8-4uxaac-4-uxa97uxaa3-1}

\textbf{ડેટા કમ્યુનિકેશન માટેના ધોરણોની ચર્ચા કરો}

\begin{solutionbox}


{\def\LTcaptype{none} % do not increment counter
\vspace{-5pt}
\captionof{table}{ડેટા કોમ્યુનિકેશનના મુખ્ય ધોરણો}
\vspace{-10pt}
\begin{longtable}[]{@{}lll@{}}
\toprule\noalign{}
ધોરણ & સંસ્થા & હેતુ \\
\midrule\noalign{}
\endhead
\bottomrule\noalign{}
\endlastfoot
\textbf{IEEE 802.x} & IEEE & LAN/MAN નેટવર્કિંગ પ્રોટોકોલ્સ \\
\textbf{X.25, X.400} & ITU-T & પેકેટ સ્વિચિંગ, મેસેજિંગ \\
\textbf{TCP/IP} & IETF & ઇન્ટરનેટ પ્રોટોકોલ્સ \\
\textbf{RS-232/422/485} & EIA/TIA & ફિઝિકલ ઇન્ટરફેસિસ \\
\textbf{USB, HDMI} & USB-IF, HDMI Forum & ડિવાઇસ કનેક્શન્સ \\
\end{longtable}
}

\textbf{સ્ટાન્ડર્ડ્સ ઓર્ગેનાઇઝેશન્સ:}

{\def\LTcaptype{none} % do not increment counter
\begin{longtable}[]{@{}ll@{}}
\toprule\noalign{}
સંસ્થા & ભૂમિકા \\
\midrule\noalign{}
\endhead
\bottomrule\noalign{}
\endlastfoot
\textbf{IEEE} & નેટવર્ક્સ માટે ટેક્નિકલ સ્ટાન્ડર્ડ્સ \\
\textbf{ITU-T} & ટેલિકોમ્યુનિકેશન સ્ટાન્ડર્ડ્સ \\
\textbf{IETF} & ઇન્ટરનેટ પ્રોટોકોલ્સ \\
\textbf{ISO} & સમગ્ર સ્ટાન્ડર્ડાઇઝેશન \\
\end{longtable}
}

\end{solutionbox}
\begin{mnemonicbox}
``PITS'' - પ્રોટોકોલ્સ, ઇન્ટરફેસિસ, ટ્રાન્સમિશન એન્ડ
સ્ટાન્ડર્ડ્સ

\end{mnemonicbox}
\subsection*{પ્રશ્ન 4(ક) [7
ગુણ]}\label{uxaaauxab0uxab6uxaa8-4uxa95-7-uxa97uxaa3-1}

\textbf{મલ્ટીમીડિયા કોમ્યુનિકેશન્સનું મોડેલ અને મલ્ટીમીડિયા સિસ્ટમના તત્વો સમજાવો}

\begin{solutionbox}

\textbf{મલ્ટીમીડિયા કોમ્યુનિકેશન મોડેલ:}

\begin{center}
\textbf{Mermaid Diagram (Code)}
\begin{verbatim}
{Shaded}
{Highlighting}[]
graph LR
    A[કન્ટેન્ટ ક્રિએશન] {-{-}{} B[કોમ્પ્રેશન]}
    B {-{-}{} C[સ્ટોરેજ]}
    C {-{-}{} D[ડિસ્ટ્રિબ્યુશન]}
    D {-{-}{} E[ડીકોમ્પ્રેશન]}
    E {-{-}{} F[પ્રેઝન્ટેશન]}
{Highlighting}
{Shaded}
\end{verbatim}
\end{center}

\textbf{મલ્ટીમીડિયા સિસ્ટમ તત્વો:}

{\def\LTcaptype{none} % do not increment counter
\begin{longtable}[]{@{}ll@{}}
\toprule\noalign{}
તત્વ & કાર્ય \\
\midrule\noalign{}
\endhead
\bottomrule\noalign{}
\endlastfoot
\textbf{ઇનપુટ ડિવાઇસિસ} & મલ્ટીમીડિયા કન્ટેન્ટ કેપ્ચર કરે છે (કેમેરા, માઇક્રોફોન) \\
\textbf{પ્રોસેસિંગ હાર્ડવેર} & મલ્ટીમીડિયા ડેટા હેન્ડલિંગ માટે CPU, GPU \\
\textbf{સ્ટોરેજ} & હાર્ડ ડ્રાઇવ, SSD, ક્લાઉડ સ્ટોરેજ \\
\textbf{કોમ્યુનિકેશન નેટવર્ક} & સિસ્ટમો વચ્ચે મલ્ટીમીડિયા ડેટા ટ્રાન્સમિટ કરે છે \\
\textbf{આઉટપુટ ડિવાઇસિસ} & કન્ટેન્ટ પ્રેઝન્ટેશન માટે ડિસ્પ્લે, સ્પીકર્સ \\
\textbf{સોફ્ટવેર} & કન્ટેન્ટ મેનિપ્યુલેશન માટે કોડેક્સ, પ્લેયર્સ, એડિટર્સ \\
\end{longtable}
}

\textbf{મીડિયા ટાઇપ્સ:}

{\def\LTcaptype{none} % do not increment counter
\begin{longtable}[]{@{}lll@{}}
\toprule\noalign{}
મીડિયા ટાઇપ & લક્ષણો & સામાન્ય ફોર્મેટ્સ \\
\midrule\noalign{}
\endhead
\bottomrule\noalign{}
\endlastfoot
\textbf{ઓડિયો} & ટેમ્પોરલ, સ્ટ્રીમિંગ & MP3, WAV, AAC \\
\textbf{વિડિયો} & ટેમ્પોરલ, સ્પેશિયલ, હાઈ બેન્ડવિડ્થ & MP4, AVI, HEVC \\
\textbf{ઇમેજ} & સ્પેશિયલ, સ્ટેટિક & JPEG, PNG, GIF \\
\textbf{ટેક્સ્ટ} & સ્ટ્રકચર્ડ, લો બેન્ડવિડ્થ & TXT, HTML, XML \\
\end{longtable}
}

\end{solutionbox}
\begin{mnemonicbox}
``CNIS-OS'' - કેપ્ચર, નેટવર્ક, ઇનપુટ-આઉટપુટ, સ્ટોરેજ, આઉટપુટ,
સોફ્ટવેર

\end{mnemonicbox}
\subsection*{પ્રશ્ન 5(અ) [3
ગુણ]}\label{uxaaauxab0uxab6uxaa8-5uxa85-3-uxa97uxaa3}

\textbf{5G ટેક્નોલોજીના મહત્વના ઘટકો સમજાવો}

\begin{solutionbox}

\textbf{5G ના મુખ્ય ઘટકો:}

{\def\LTcaptype{none} % do not increment counter
\begin{longtable}[]{@{}ll@{}}
\toprule\noalign{}
ઘટક & વર્ણન \\
\midrule\noalign{}
\endhead
\bottomrule\noalign{}
\endlastfoot
\textbf{મિલિમીટર વેવ્સ} & વધુ બેન્ડવિડ્થ માટે ઊંચી ફ્રીક્વન્સી (24-100 GHz) \\
\textbf{મેસિવ MIMO} & સુધારેલી ક્ષમતા માટે મલ્ટિપલ-ઇનપુટ મલ્ટિપલ-આઉટપુટ એન્ટેનાઓ \\
\textbf{બીમફોર્મિંગ} & વધુ કાર્યક્ષમતા માટે કેન્દ્રિત સિગ્નલ ટ્રાન્સમિશન \\
\textbf{નેટવર્ક સ્લાઇસિંગ} & શેર્ડ ઇન્ફ્રાસ્ટ્રક્ચર પર વર્ચ્યુઅલ નેટવર્ક્સ \\
\textbf{એજ કમ્પ્યુટિંગ} & ઓછા લેટન્સી માટે ડેટા સોર્સની નજીક પ્રોસેસિંગ \\
\end{longtable}
}

\end{solutionbox}
\begin{mnemonicbox}
``MMBN-E'' - મિલિમીટર, MIMO, બીમફોર્મિંગ, નેટવર્ક, એજ

\end{mnemonicbox}
\subsection*{પ્રશ્ન 5(બ) [4
ગુણ]}\label{uxaaauxab0uxab6uxaa8-5uxaac-4-uxa97uxaa3}

\textbf{સ્પ્રેડ સ્પેક્ટ્રમ કમ્યુનિકેશનનું વર્ણન કરો}

\begin{solutionbox}

\textbf{સ્પ્રેડ સ્પેક્ટ્રમ વ્યાખ્યા:} એવી તકનીક જેમાં સિગ્નલને પહોળા ફ્રીક્વન્સી બેન્ડ પર
ફેલાવવામાં આવે છે, જે જરૂરી મિનિમમ બેન્ડવિડ્થ કરતાં ઘણું વધારે છે.

\textbf{સ્પ્રેડ સ્પેક્ટ્રમના પ્રકારો:}

{\def\LTcaptype{none} % do not increment counter
\begin{longtable}[]{@{}
  >{\raggedright\arraybackslash}p{(\linewidth - 4\tabcolsep) * \real{0.2308}}
  >{\raggedright\arraybackslash}p{(\linewidth - 4\tabcolsep) * \real{0.3077}}
  >{\raggedright\arraybackslash}p{(\linewidth - 4\tabcolsep) * \real{0.4615}}@{}}
\toprule\noalign{}
\begin{minipage}[b]{\linewidth}\raggedright
પ્રકાર
\end{minipage} & \begin{minipage}[b]{\linewidth}\raggedright
પદ્ધતિ
\end{minipage} & \begin{minipage}[b]{\linewidth}\raggedright
ફાયદા
\end{minipage} \\
\midrule\noalign{}
\endhead
\bottomrule\noalign{}
\endlastfoot
\textbf{DSSS} (ડાયરેક્ટ સિક્વન્સ) & ઊંચા-રેટવાળા સ્યુડોરેન્ડમ કોડ સાથે ડેટાને XOR &
સારી નોઇઝ ઇમ્યુનિટી \\
\textbf{FHSS} (ફ્રીક્વન્સી હોપિંગ) & કેરિયરને ઝડપથી ઘણી ફ્રીક્વન્સીઓ પર બદલાય છે &
જેમિંગનો પ્રતિકાર કરે છે \\
\textbf{THSS} (ટાઇમ હોપિંગ) & અલગ-અલગ ટાઇમ સ્લોટ્સમાં ટૂંકા બર્સ્ટ ટ્રાન્સમિટ કરે છે
& ઇન્ટરસેપ્ટની ઓછી સંભાવના \\
\end{longtable}
}

\textbf{આકૃતિ: DSSS પ્રક્રિયા}

\begin{verbatim}
Data:       |\_\_\_|‾‾‾|\_\_\_|
            
PN Code:    |\_|‾|\_|‾|\_|‾|\_|‾|

Spread
Signal:     |\_|‾|‾|\_|‾|\_|\_|‾|
\end{verbatim}

\end{solutionbox}
\begin{mnemonicbox}
``DFT - ડિફિકલ્ટ ફોર ટ્રેકર્સ'' - ડાયરેક્ટ, ફ્રીક્વન્સી, ટાઇમ
હોપિંગ

\end{mnemonicbox}
\subsection*{પ્રશ્ન 5(ક) [7
ગુણ]}\label{uxaaauxab0uxab6uxaa8-5uxa95-7-uxa97uxaa3}

\textbf{સેટેલાઇટ કોમ્યુનિકેશનના બ્લોક ડાયાગ્રામને સમજાવો}

\begin{solutionbox}

\textbf{સેટેલાઇટ કોમ્યુનિકેશન બ્લોક ડાયાગ્રામ:}

\begin{center}
\textbf{Mermaid Diagram (Code)}
\begin{verbatim}
{Shaded}
{Highlighting}[]
graph LR
    A["સેટેલાઇટ ટ્રાન્સપોન્ડર"] {-{-}{-} B["અપલિંક"]}
    A {-{-}{-} C["ડાઉનલિંક"]}
    B {-{-}{-} D["અર્થ સ્ટેશન Tx"]}
    C {-{-}{-} E["અર્થ સ્ટેશન Rx"]}

    classDef satellite fill:\#f9f,stroke:\#333,stroke{-width:2px;}
    classDef earth fill:\#9cf,stroke:\#333,stroke{-width:2px;}
    classDef link stroke{-dasharray: 5 5;}
    
    class A satellite;
    class D,E earth;
    class B,C link;
{Highlighting}
{Shaded}
\end{verbatim}
\end{center}

\textbf{મુખ્ય ઘટકો:}

{\def\LTcaptype{none} % do not increment counter
\begin{longtable}[]{@{}
  >{\raggedright\arraybackslash}p{(\linewidth - 2\tabcolsep) * \real{0.5238}}
  >{\raggedright\arraybackslash}p{(\linewidth - 2\tabcolsep) * \real{0.4762}}@{}}
\toprule\noalign{}
\begin{minipage}[b]{\linewidth}\raggedright
ઘટક
\end{minipage} & \begin{minipage}[b]{\linewidth}\raggedright
કાર્ય
\end{minipage} \\
\midrule\noalign{}
\endhead
\bottomrule\noalign{}
\endlastfoot
\textbf{અર્થ સ્ટેશન (Tx)} & સિગ્નલ્સનો સ્ત્રોત, અપલિંક ફંક્શન્સ કરે છે \\
\textbf{અપલિંક} & પૃથ્વીથી સેટેલાઇટ સુધીનું ટ્રાન્સમિશન (ઊંચી ફ્રીક્વન્સી) \\
\textbf{સેટેલાઇટ ટ્રાન્સપોન્ડર} & સિગ્નલ્સ પ્રાપ્ત કરે છે, એમ્પ્લિફાય કરે છે, અને ફરીથી
ટ્રાન્સમિટ કરે છે \\
\textbf{ડાઉનલિંક} & સેટેલાઇટથી પૃથ્વી સુધીનું ટ્રાન્સમિશન (નીચી ફ્રીક્વન્સી) \\
\textbf{અર્થ સ્ટેશન (Rx)} & ડાઉનલિંક સિગ્નલ્સ પ્રાપ્ત કરે છે અને પ્રોસેસ કરે છે \\
\end{longtable}
}

\textbf{ફ્રીક્વન્સી બેન્ડ્સ:}

{\def\LTcaptype{none} % do not increment counter
\begin{longtable}[]{@{}lll@{}}
\toprule\noalign{}
બેન્ડ & ફ્રીક્વન્સી રેન્જ & એપ્લિકેશન્સ \\
\midrule\noalign{}
\endhead
\bottomrule\noalign{}
\endlastfoot
\textbf{C-બેન્ડ} & 4-8 GHz & ટેલિવિઝન, વોઇસ, ડેટા \\
\textbf{Ku-બેન્ડ} & 12-18 GHz & ડાયરેક્ટ બ્રોડકાસ્ટ, VSAT \\
\textbf{Ka-બેન્ડ} & 26-40 GHz & હાઈ-સ્પીડ ડેટા, ઇન્ટરનેટ \\
\end{longtable}
}

\end{solutionbox}
\begin{mnemonicbox}
``STUDER'' - સ્ટેશન ટ્રાન્સમિટ્સ અપલિંક, ડાઉનલિંક ટુ અર્થ
રિસીવર

\end{mnemonicbox}
\subsection*{પ્રશ્ન 5(અ OR) [3
ગુણ]}\label{uxaaauxab0uxab6uxaa8-5uxa85-or-3-uxa97uxaa3}

\textbf{5G ટેકનોલોજીની વિશેષતાઓ અને ફાયદાઓ સમજાવો}

\begin{solutionbox}

\textbf{5G વિશેષતાઓ અને ફાયદાઓ:}

{\def\LTcaptype{none} % do not increment counter
\begin{longtable}[]{@{}ll@{}}
\toprule\noalign{}
વિશેષતા & ફાયદો \\
\midrule\noalign{}
\endhead
\bottomrule\noalign{}
\endlastfoot
\textbf{હાઈ સ્પીડ} & ઝડપી ડાઉનલોડ્સ માટે 10 Gbps સુધીના ડેટા રેટ્સ \\
\textbf{અલ્ટ્રા-લો લેટન્સી} & રિયલ-ટાઇમ એપ્લિકેશન્સ માટે \textless1ms રિસ્પોન્સ
ટાઇમ \\
\textbf{મેસિવ કનેક્ટિવિટી} & દર ચોરસ કિમી દીઠ 1 મિલિયન ઉપકરણો સુધી \\
\textbf{નેટવર્ક સ્લાઇસિંગ} & ચોક્કસ એપ્લિકેશન્સ માટે કસ્ટમાઇઝ્ડ વર્ચ્યુઅલ નેટવર્ક્સ \\
\textbf{સુધારેલી વિશ્વસનીયતા} & ક્રિટિકલ સર્વિસિસ માટે 99.999\% ઉપલબ્ધતા \\
\textbf{એનર્જી એફિશિયન્સી} & ડેટાના દરેક બિટ દીઠ ઓછી પાવર વપરાશ \\
\end{longtable}
}

\end{solutionbox}
\begin{mnemonicbox}
``HUMNER'' - હાઈ-સ્પીડ, અલ્ટ્રા-લો લેટન્સી, મેસિવ
કનેક્ટિવિટી, નેટવર્ક સ્લાઇસિંગ, એન્હાન્સ્ડ રિલાયબિલિટી

\end{mnemonicbox}
\subsection*{પ્રશ્ન 5(બ) [4
ગુણ]}\label{uxaaauxab0uxab6uxaa8-5uxaac-4-uxa97uxaa3-1}

\textbf{એજ કમ્પ્યુટિંગનું વર્ણન કરો}

\begin{solutionbox}

\textbf{એજ કમ્પ્યુટિંગ વ્યાખ્યા:} કમ્પ્યુટિંગ પેરાડાઇમ જે ડેટા પ્રોસેસિંગને ડેટા જનરેશનના
સ્ત્રોતની નજીક લાવે છે.

\textbf{આકૃતિ: એજ કમ્પ્યુટિંગ આર્કિટેક્ચર}

\begin{center}
\textbf{Mermaid Diagram (Code)}
\begin{verbatim}
{Shaded}
{Highlighting}[]
graph LR
    A[IoT ડિવાઇસિસ] {-{-}{} B[એજ ડિવાઇસિસ]}
    B {-{-}{} C[એજ સર્વર્સ]}
    C {-{-}{} D[ક્લાઉડ ડેટા સેન્ટર્સ]}
{Highlighting}
{Shaded}
\end{verbatim}
\end{center}

\textbf{મુખ્ય લક્ષણો:}

{\def\LTcaptype{none} % do not increment counter
\begin{longtable}[]{@{}ll@{}}
\toprule\noalign{}
લક્ષણ & વર્ણન \\
\midrule\noalign{}
\endhead
\bottomrule\noalign{}
\endlastfoot
\textbf{પ્રોક્સિમિટી} & ડેટા સોર્સની નજીક પ્રોસેસિંગ લેટન્સી ઘટાડે છે \\
\textbf{ડિસ્ટ્રિબ્યુટેડ} & નેટવર્ક એજ પર ફેલાયેલા કમ્પ્યુટિંગ રિસોર્સિસ \\
\textbf{રિયલ-ટાઇમ પ્રોસેસિંગ} & સમય-મહત્વપૂર્ણ એપ્લિકેશન્સ માટે ઝડપી પ્રતિસાદ \\
\textbf{બેન્ડવિડ્થ ઓપ્ટિમાઇઝેશન} & સેન્ટ્રલ ક્લાઉડને મોકલવામાં આવતો ડેટા ઘટાડે છે \\
\textbf{ડેટા પ્રાઇવસી} & સંવેદનશીલ ડેટા સ્થાનિક રીતે પ્રોસેસ થાય છે \\
\end{longtable}
}

\end{solutionbox}
\begin{mnemonicbox}
``PDRBD'' - પ્રોસેસ ડેટા રેપિડલી બાય ડિસ્ટ્રિબ્યુટિંગ

\end{mnemonicbox}
\subsection*{પ્રશ્ન 5(ક) [7
ગુણ]}\label{uxaaauxab0uxab6uxaa8-5uxa95-7-uxa97uxaa3-1}

\textbf{કોમ્યુનિકેશન સિક્યોરિટીમાં બ્લોક ચેઈનનું મહત્વ સમજાવો}

\begin{solutionbox}

\textbf{કોમ્યુનિકેશન સિક્યોરિટીમાં બ્લોકચેઇન:}

\begin{center}
\textbf{Mermaid Diagram (Code)}
\begin{verbatim}
{Shaded}
{Highlighting}[]
graph LR
    A[ટ્રાન્ઝેક્શન રિક્વેસ્ટ] {-{-}{} B[બ્લોક ક્રિએશન]}
    B {-{-}{} C[બ્લોક વેરિફિકેશન]}
    C {-{-}{} D[ચેઇનમાં બ્લોક એડિશન]}
    D {-{-}{} E[ચેઇન ડિસ્ટ્રિબ્યુશન]}
{Highlighting}
{Shaded}
\end{verbatim}
\end{center}

\textbf{સિક્યોરિટી બેનિફિટ્સ:}

{\def\LTcaptype{none} % do not increment counter
\begin{longtable}[]{@{}ll@{}}
\toprule\noalign{}
બેનિફિટ & વર્ણન \\
\midrule\noalign{}
\endhead
\bottomrule\noalign{}
\endlastfoot
\textbf{ઇમ્યુટેબિલિટી} & એકવાર રેકોર્ડ થયેલો ડેટા બદલી શકાતો નથી \\
\textbf{ડિસેન્ટ્રલાઇઝેશન} & નિયંત્રણ કે નિષ્ફળતાનો કોઈ એકલ પોઇન્ટ નથી \\
\textbf{ટ્રાન્સપેરન્સી} & બધા ટ્રાન્ઝેક્શન્સ નેટવર્ક પાર્ટિસિપન્ટ્સને દેખાય છે \\
\textbf{ક્રિપ્ટોગ્રાફિક સિક્યોરિટી} & મજબૂત એન્ક્રિપ્શન ડેટા ઇન્ટેગ્રિટીનું રક્ષણ કરે
છે \\
\textbf{સ્માર્ટ કોન્ટ્રાક્ટ્સ} & બિલ્ટ-ઇન સિક્યોરિટી સાથે સેલ્ફ-એક્ઝિક્યુટિંગ
એગ્રીમેન્ટ્સ \\
\textbf{કન્સેન્સસ મેકેનિઝમ્સ} & મલ્ટિપલ વેલિડેટર્સ ટ્રાન્ઝેક્શન લેજિટિમસી સુનિશ્ચિત કરે
છે \\
\end{longtable}
}

\textbf{કોમ્યુનિકેશન એપ્લિકેશન્સ:}

{\def\LTcaptype{none} % do not increment counter
\begin{longtable}[]{@{}ll@{}}
\toprule\noalign{}
એપ્લિકેશન & સિક્યોરિટી બેનિફિટ \\
\midrule\noalign{}
\endhead
\bottomrule\noalign{}
\endlastfoot
\textbf{સિક્યોર મેસેજિંગ} & ટેમ્પર-પ્રૂફ રેકોર્ડ્સ સાથે એન્ડ-ટુ-એન્ડ એન્ક્રિપ્શન \\
\textbf{આઇડેન્ટિટી મેનેજમેન્ટ} & સેલ્ફ-સોવરેન આઇડેન્ટિટી વેરિફિકેશન \\
\textbf{IoT સિક્યોરિટી} & સિક્યોર ડિવાઇસ ઓથેન્ટિકેશન અને ડેટા ઇન્ટેગ્રિટી \\
\textbf{નેટવર્ક ઇન્ફ્રાસ્ટ્રક્ચર} & સિક્યોર રાઉટિંગ અને DNS સિસ્ટમ્સ \\
\end{longtable}
}

\end{solutionbox}
\begin{mnemonicbox}
``DTCSCI'' - ડિસેન્ટ્રલાઇઝ્ડ ટ્રાન્સપેરન્ટ ક્રિપ્ટોગ્રાફિક
સિસ્ટમ ક્રિએટ્સ ઇમ્યુટેબિલિટી

\end{mnemonicbox}

\end{document}
