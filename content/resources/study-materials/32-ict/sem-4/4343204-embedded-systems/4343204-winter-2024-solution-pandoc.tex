\documentclass[10pt,a4paper]{article}

% content/resources/templates/preamble.tex
\usepackage[margin=0.6in]{geometry}
\author{Milav Dabgar}
\usepackage{amsmath,amssymb,amsthm}
\usepackage{booktabs}
\usepackage{multirow}
\usepackage{xcolor}
\usepackage{tcolorbox}
\tcbuselibrary{breakable,skins}
\usepackage[colorlinks=true,linkcolor=blue]{hyperref}
\usepackage{titlesec}
\usepackage{enumitem}
\usepackage{tikz}
\usepackage{pgfplots}
\usepackage{circuitikz}
\usepackage[version=4]{mhchem}
\usepackage{longtable}
\usepackage{array}
\usepackage{float}
\usepackage{caption}
\usepackage{listings}

\lstset{
  basicstyle=\small\ttfamily,
  breaklines=true,
  breakatwhitespace=false,
  postbreak=\mbox{\textcolor{red}{$\hookrightarrow$}\space},
  float=false,
  numbers=left,
  numberstyle=\tiny\color{gray},
  numbersep=10pt,
  xleftmargin=2em,
  keywordstyle=\color{blue},
  commentstyle=\color{green!60!black},
  stringstyle=\color{purple},
  backgroundcolor=\color{gray!5},
  showstringspaces=false,
  tabsize=2,
  captionpos=b,
  keepspaces=true,
  columns=flexible
}

\pgfplotsset{compat=1.18}
\usetikzlibrary{shapes,arrows,positioning,calc,patterns,decorations.pathmorphing,decorations.markings,arrows.meta}

% Color scheme
\definecolor{headcolor}{RGB}{0,102,204}
\definecolor{keycolor}{RGB}{220,20,60}
\definecolor{solutioncolor}{RGB}{34,139,34}
\definecolor{mnemoniccolor}{RGB}{148,0,211}
\definecolor{codecolor}{RGB}{0,0,100}

% Spacing
\setlength{\parskip}{3pt}
\setlist[itemize]{nosep}
\setlist[enumerate]{nosep}

% Title formatting
\titleformat{\section}{\Large\bfseries\color{headcolor}}{\thesection}{1em}{}
\titleformat{\subsection}{\large\bfseries\color{headcolor}}{\thesubsection}{1em}{}

% Pandoc tightlist compatibility
\providecommand{\tightlist}{%
  \setlength{\itemsep}{0pt}\setlength{\parskip}{0pt}}

% Pandoc longtable compatibility
\newcounter{none}
\def\thenone{}


% content/resources/templates/english-boxes.tex
% This file is currently empty - it exists to maintain consistency with the import structure.
% Add custom environments here if needed in the future.


\begin{document}

\begin{center}
{\Huge\bfseries\color{headcolor} Subject Name Solutions}\\[5pt]
{\LARGE 4343204 -- Winter 2024}\\[3pt]
{\large Semester 1 Study Material}\\[3pt]
{\normalsize\textit{Detailed Solutions and Explanations}}
\end{center}

\vspace{10pt}

\subsection*{Question 1(a) [3 marks]}\label{q1a}

\textbf{Write the size of RAM, Flash and EEPROM memory in ATmega32 and
explain its need in microcontroller.}

\begin{solutionbox}

ATmega32 memory specifications and their importance in microcontroller
operation:


{\def\LTcaptype{none} % do not increment counter
\vspace{-5pt}
\captionof{table}{Memory Sizes in ATmega32}
\vspace{-10pt}
\begin{longtable}[]{@{}lll@{}}
\toprule\noalign{}
Memory Type & Size & Purpose \\
\midrule\noalign{}
\endhead
\bottomrule\noalign{}
\endlastfoot
SRAM (RAM) & 2 KB & Variables and stack storage \\
Flash & 32 KB & Program storage \\
EEPROM & 1 KB & Non-volatile data storage \\
\end{longtable}
}

\begin{itemize}
\tightlist
\item
  \textbf{RAM}: Temporary storage for variables during program execution
\item
  \textbf{Flash}: Permanent storage for program instructions and
  constants
\item
  \textbf{EEPROM}: Long-term storage for data that must survive power
  cycles
\end{itemize}

\end{solutionbox}
\begin{mnemonicbox}
``RAM for Run, Flash for Function, EEPROM for
Eternity''

\end{mnemonicbox}
\subsection*{Question 1(b) [4 marks]}\label{q1b}

\textbf{Discuss RAM memory of ATmega32.}

\begin{solutionbox}

ATmega32's RAM (SRAM) is organized into different sections for specific
purposes.

\textbf{Diagram:}

\begin{lstlisting}
    ATmega32 RAM (2KB)
+-------------------------+ 0x0000
| 32 General Registers    |
+-------------------------+ 0x0020
| 64 I/O Registers        |
+-------------------------+ 0x0060
| 160 Extended I/O Regs   |
+-------------------------+ 0x0100
|                         |
| Internal SRAM           |
| (1.85 KB)               |
|                         |
+-------------------------+ 0x085F
\end{lstlisting}

\begin{itemize}
\tightlist
\item
  \textbf{Register File}: First 32 locations (0x0000-0x001F)
\item
  \textbf{I/O Registers}: Standard I/O space (0x0020-0x005F)
\item
  \textbf{Extended I/O}: Additional peripheral registers (0x0060-0x00FF)
\item
  \textbf{Data Memory}: General purpose SRAM (0x0100-0x085F)
\end{itemize}

\end{solutionbox}
\begin{mnemonicbox}
``Registers, I/O, Extended, Data - RAM's Efficient
Design''

\end{mnemonicbox}
\subsection*{Question 1(c) [7 marks]}\label{q1c}

\textbf{Define Real Time Operating System and Explain Characteristics of
it.}

\begin{solutionbox}

A Real-Time Operating System (RTOS) is a specialized operating system
designed to process data and events with precise timing constraints.


{\def\LTcaptype{none} % do not increment counter
\vspace{-5pt}
\captionof{table}{Key Characteristics of RTOS}
\vspace{-10pt}
\begin{longtable}[]{@{}
  >{\raggedright\arraybackslash}p{(\linewidth - 2\tabcolsep) * \real{0.5517}}
  >{\raggedright\arraybackslash}p{(\linewidth - 2\tabcolsep) * \real{0.4483}}@{}}
\toprule\noalign{}
\begin{minipage}[b]{\linewidth}\raggedright
Characteristic
\end{minipage} & \begin{minipage}[b]{\linewidth}\raggedright
Description
\end{minipage} \\
\midrule\noalign{}
\endhead
\bottomrule\noalign{}
\endlastfoot
Determinism & Guaranteed response times for tasks \\
Preemptive Scheduling & Higher priority tasks can interrupt lower
ones \\
Low Latency & Minimal delay between event and response \\
Priority-Based & Tasks are assigned priorities for execution \\
Task Management & Provides mechanisms for task creation, deletion, and
synchronization \\
Resource Management & Prevents resource conflicts and deadlocks \\
Reliability & Robust operation even under peak loads \\
\end{longtable}
}

\begin{itemize}
\tightlist
\item
  \textbf{Multitasking}: Supports concurrent execution of multiple tasks
\item
  \textbf{Small Footprint}: Optimized for embedded systems with limited
  resources
\item
  \textbf{Time Management}: Precise timing services with microsecond
  resolution
\item
  \textbf{Kernel Services}: IPC, mutex, semaphores for task coordination
\end{itemize}

\end{solutionbox}
\begin{mnemonicbox}
``Deterministic Preemptive Tasks Run On Strict
Timelines''

\end{mnemonicbox}
\subsection*{Question 1(c OR) [7
marks]}\label{question-1c-or-7-marks}

\textbf{What is Embedded System? Draw and Explain General block diagram
of Embedded system.}

\begin{solutionbox}

An Embedded System is a dedicated computer system designed to perform
specific functions within a larger mechanical or electrical system,
often with real-time constraints.

\textbf{Diagram:}

\begin{lstlisting}
                           +----------------+
                           |   Power Supply |
                           +----------------+
                                   |
                                   v
+-----------+    +-------------+    +-----------+    +-----------+
|           |    |             |    |           |    |           |
|  Input    |--->| Processing  |--->|  Output   |    |  Memory   |
|  Devices  |    |    Unit     |    |  Devices  |    |           |
|           |    |             |    |           |    |           |
+-----------+    +-------------+    +-----------+    +-----------+
     ^                  ^                                  ^
     |                  |                                  |
     v                  v                                  v
+-----------+    +--------------+                    +-----------+
|           |    |              |                    |           |
|  Sensors  |    | Communication|                    |  Storage  |
|           |    |  Interface   |                    |           |
+-----------+    +--------------+                    +-----------+
\end{lstlisting}


{\def\LTcaptype{none} % do not increment counter
\vspace{-5pt}
\captionof{table}{Embedded System Components}
\vspace{-10pt}
\begin{longtable}[]{@{}
  >{\raggedright\arraybackslash}p{(\linewidth - 2\tabcolsep) * \real{0.5238}}
  >{\raggedright\arraybackslash}p{(\linewidth - 2\tabcolsep) * \real{0.4762}}@{}}
\toprule\noalign{}
\begin{minipage}[b]{\linewidth}\raggedright
Component
\end{minipage} & \begin{minipage}[b]{\linewidth}\raggedright
Function
\end{minipage} \\
\midrule\noalign{}
\endhead
\bottomrule\noalign{}
\endlastfoot
Processing Unit & Executes program instructions
(microcontroller/microprocessor) \\
Memory & Stores program and data (RAM, ROM, Flash) \\
Input/Output & Interfaces with external devices \\
Communication & Connects to other systems or networks \\
Power Supply & Provides regulated power \\
Sensors & Gather environmental data \\
\end{longtable}
}

\begin{itemize}
\tightlist
\item
  \textbf{Application-Specific}: Designed for dedicated tasks
\item
  \textbf{Resource-Constrained}: Limited processing power and memory
\item
  \textbf{Real-Time}: Responds to events within timing constraints
\item
  \textbf{High Reliability}: Must operate continuously without failure
\end{itemize}

\end{solutionbox}
\begin{mnemonicbox}
``Process, Memory, I/O - Every System Must Include''

\end{mnemonicbox}
\subsection*{Question 2(a) [3 marks]}\label{q2a}

\textbf{Write different Criteria for choosing microcontroller for any
application design in embedded system.}

\begin{solutionbox}

Selecting the right microcontroller requires evaluating multiple
criteria based on application requirements.


{\def\LTcaptype{none} % do not increment counter
\vspace{-5pt}
\captionof{table}{Microcontroller Selection Criteria}
\vspace{-10pt}
\begin{longtable}[]{@{}ll@{}}
\toprule\noalign{}
Criterion & Considerations \\
\midrule\noalign{}
\endhead
\bottomrule\noalign{}
\endlastfoot
Performance & CPU speed, MIPS, bit width (8/16/32) \\
Memory & Flash, RAM, EEPROM capacity \\
Power Consumption & Operating voltage, sleep modes \\
I/O Capabilities & Number of ports, special functions \\
Peripherals & ADC, timers, communication interfaces \\
Cost & Unit price, development tools \\
Form Factor & Size, package type, pin count \\
\end{longtable}
}

\begin{itemize}
\tightlist
\item
  \textbf{Application Requirements}: Specific features needed for the
  application
\item
  \textbf{Development Environment}: Available compilers, debuggers,
  libraries
\item
  \textbf{Future Expansion}: Scalability for future enhancements
\end{itemize}

\end{solutionbox}
\begin{mnemonicbox}
``Performance Memory Power I/O Cost''

\end{mnemonicbox}
\subsection*{Question 2(b) [4 marks]}\label{q2b}

\textbf{Draw and Explain TCCR0 register.}

\begin{solutionbox}

Timer/Counter Control Register 0 (TCCR0) controls the operation of
Timer/Counter0 in ATmega32.

\textbf{Diagram:}

\begin{lstlisting}
+-----+------+------+-----+-----+-----+-----+-----+
| FOC0| WGM00| COM01|COM00|WGM01| CS02| CS01| CS00|
+-----+------+------+-----+-----+-----+-----+-----+
   7     6       5     4     3     2     1     0
\end{lstlisting}


{\def\LTcaptype{none} % do not increment counter
\vspace{-5pt}
\captionof{table}{TCCR0 Bit Functions}
\vspace{-10pt}
\begin{longtable}[]{@{}lll@{}}
\toprule\noalign{}
Bits & Name & Function \\
\midrule\noalign{}
\endhead
\bottomrule\noalign{}
\endlastfoot
7 & FOC0 & Force Output Compare \\
6,3 & WGM01:0 & Waveform Generation Mode \\
5,4 & COM01:0 & Compare Match Output Mode \\
2,1,0 & CS02:0 & Clock Select (Prescaler) \\
\end{longtable}
}

\begin{itemize}
\tightlist
\item
  \textbf{WGM01:0}: Determines timer operating mode (Normal, CTC, PWM)
\item
  \textbf{COM01:0}: Controls OC0 pin output behavior
\item
  \textbf{CS02:0}: Selects clock source and prescaler value
\end{itemize}

\end{solutionbox}
\begin{mnemonicbox}
``Force Waveform Compare Clock Select''

\end{mnemonicbox}
\subsection*{Question 2(c) [7 marks]}\label{q2c}

\textbf{List timers of ATmega32 and Explain working modes of any one
timer in detail.}

\begin{solutionbox}

ATmega32 features multiple timers with various capabilities and
operating modes.


{\def\LTcaptype{none} % do not increment counter
\vspace{-5pt}
\captionof{table}{Timers in ATmega32}
\vspace{-10pt}
\begin{longtable}[]{@{}llll@{}}
\toprule\noalign{}
Timer & Type & Size & Features \\
\midrule\noalign{}
\endhead
\bottomrule\noalign{}
\endlastfoot
Timer0 & General Purpose & 8-bit & Simple timing, PWM \\
Timer1 & Advanced & 16-bit & Input capture, dual PWM \\
Timer2 & General Purpose & 8-bit & Asynchronous operation \\
\end{longtable}
}

\textbf{Timer0 Operating Modes:}

\begin{enumerate}
\tightlist
\item
  \textbf{Normal Mode}:

  \begin{itemize}
  \tightlist
  \item
    Counter increments from 0 to 255 then overflows back to 0
  \item
    Overflow interrupt can be generated
  \item
    Used for simple timing and delay generation
  \end{itemize}
\item
  \textbf{CTC (Clear Timer on Compare) Mode}:

  \begin{itemize}
  \tightlist
  \item
    Counter resets when it reaches OCR0 value
  \item
    Allows precise frequency generation
  \item
    Compare match interrupt can be generated
  \end{itemize}
\item
  \textbf{Fast PWM Mode}:

  \begin{itemize}
  \tightlist
  \item
    Counter counts from 0 to 255
  \item
    Output toggles at overflow and compare match
  \item
    High frequency PWM generation
  \end{itemize}
\item
  \textbf{Phase Correct PWM Mode}:

  \begin{itemize}
  \tightlist
  \item
    Counter counts up then down (0\rightarrow255\rightarrow0)
  \item
    Symmetric PWM waveform generation
  \item
    Lower frequency but better resolution than Fast PWM
  \end{itemize}
\end{enumerate}

\end{solutionbox}
\begin{mnemonicbox}
``Normal Compares Fast Phase - Timer Modes Matter''

\end{mnemonicbox}
\subsection*{Question 2(a OR) [3
marks]}\label{question-2a-or-3-marks}

\textbf{List various embedded system applications. Explain any one in
brief.}

\begin{solutionbox}

Embedded systems are found in numerous applications across various
domains.


{\def\LTcaptype{none} % do not increment counter
\vspace{-5pt}
\captionof{table}{Embedded System Applications}
\vspace{-10pt}
\begin{longtable}[]{@{}ll@{}}
\toprule\noalign{}
Domain & Applications \\
\midrule\noalign{}
\endhead
\bottomrule\noalign{}
\endlastfoot
Consumer & Smart appliances, entertainment systems \\
Automotive & Engine control, safety systems, infotainment \\
Industrial & Process control, automation, robotics \\
Medical & Patient monitoring, imaging, implantable devices \\
Communications & Routers, modems, network switches \\
Aerospace & Flight control, navigation, life support \\
\end{longtable}
}

\textbf{Smart Home Automation System:} A smart home system uses embedded
controllers to monitor and control household devices. Sensors detect
environmental conditions like temperature and motion, while
microcontrollers process this data and control actuators such as HVAC
systems, lighting, and security devices. The system can be programmed
for autonomous operation or user control via smartphone apps, providing
convenience, energy efficiency, and enhanced security.

\end{solutionbox}
\begin{mnemonicbox}
``Consumers Automate Industry Medical Communications
Aerospace''

\end{mnemonicbox}
\subsection*{Question 2(b OR) [4
marks]}\label{question-2b-or-4-marks}

\textbf{Explain the function of DDRA, PINA and PORTA registers in
ATmega32 microcontroller.}

\begin{solutionbox}

The three registers control the operation of Port A in ATmega32, each
serving a distinct purpose.


{\def\LTcaptype{none} % do not increment counter
\vspace{-5pt}
\captionof{table}{Port A Registers}
\vspace{-10pt}
\begin{longtable}[]{@{}lll@{}}
\toprule\noalign{}
Register & Function & Operation \\
\midrule\noalign{}
\endhead
\bottomrule\noalign{}
\endlastfoot
DDRA & Data Direction & Configures pins as input (0) or output (1) \\
PORTA & Data Register & Sets output values or enables pull-ups \\
PINA & Port Input Pins & Reads actual pin states \\
\end{longtable}
}

\textbf{Example Configurations:}

\begin{lstlisting}
DDRA = 0xFF;  // All pins as output
PORTA = 0xA5; // Set alternating pattern (10100101)

DDRA = 0x00;  // All pins as input
PORTA = 0xFF; // Enable internal pull-ups on all pins
data = PINA;  // Read current pin states
\end{lstlisting}

\begin{itemize}
\tightlist
\item
  \textbf{Bit-Level Control}: Each bit controls corresponding pin
\item
  \textbf{Atomic Operations}: Individual bits can be modified
\item
  \textbf{Read-Modify-Write}: Common operation pattern
\end{itemize}

\end{solutionbox}
\begin{mnemonicbox}
``Direction Determines, Port Provides, PIN
Perceives''

\end{mnemonicbox}
\subsection*{Question 2(c OR) [7
marks]}\label{question-2c-or-7-marks}

\textbf{Draw Status Register of ATmega32 and explain it in detail.}

\begin{solutionbox}

The Status Register (SREG) in ATmega32 contains processor status flags
affected by arithmetic operations and controls interrupts.

\textbf{Diagram:}

\begin{lstlisting}
+---+---+---+---+---+---+---+---+
| I | T | H | S | V | N | Z | C |
+---+---+---+---+---+---+---+---+
  7   6   5   4   3   2   1   0
\end{lstlisting}


{\def\LTcaptype{none} % do not increment counter
\vspace{-5pt}
\captionof{table}{SREG Bit Functions}
\vspace{-10pt}
\begin{longtable}[]{@{}llll@{}}
\toprule\noalign{}
Bit & Name & Function & Set When \\
\midrule\noalign{}
\endhead
\bottomrule\noalign{}
\endlastfoot
7 & I & Global Interrupt Enable & Programmatically enabled \\
6 & T & Bit Copy Storage & Used for bit copy instructions \\
5 & H & Half Carry Flag & Half-carry in BCD operations \\
4 & S & Sign Flag & N\oplusV (used for signed operations) \\
3 & V & Two's Complement Overflow & Arithmetic overflow occurs \\
2 & N & Negative Flag & Result is negative (MSB=1) \\
1 & Z & Zero Flag & Result is zero \\
0 & C & Carry Flag & Carry occurs in arithmetic \\
\end{longtable}
}

\begin{itemize}
\tightlist
\item
  \textbf{Arithmetic Feedback}: Indicates result status
\item
  \textbf{Conditional Branches}: Used by branch instructions
\item
  \textbf{Interrupt Control}: I-bit enables/disables all interrupts
\item
  \textbf{Access Methods}: Directly addressable via IN/OUT instructions
\end{itemize}

\end{solutionbox}
\begin{mnemonicbox}
``Interrupts Track Half Sign Overflow Negative Zero
Carry''

\end{mnemonicbox}
\subsection*{Question 3(a) [3 marks]}\label{q3a}

\textbf{Write a short note on Harvard Architecture of AVR
microcontroller.}

\begin{solutionbox}

Harvard Architecture is a fundamental design principle of AVR
microcontrollers, separating program and data memory.

\textbf{Diagram:}

\includegraphics[width=1\linewidth,height=\textheight,keepaspectratio]{mermaid-47fbc9d9.pdf}

\begin{itemize}
\tightlist
\item
  \textbf{Separate Buses}: Independent buses for program and data memory
\item
  \textbf{Parallel Access}: Can fetch instructions and access data
  simultaneously
\item
  \textbf{Performance}: Increases execution speed by eliminating memory
  bottlenecks
\item
  \textbf{Different Widths}: Program memory is organized in 16-bit
  words, data memory in 8-bit bytes
\end{itemize}

\end{solutionbox}
\begin{mnemonicbox}
``Program and Data Paths Are Separate''

\end{mnemonicbox}
\subsection*{Question 3(b) [4 marks]}\label{q3b}

\textbf{List Registers associated with Serial Communication (RS232) and
explain steps to interface it with ATmega32.}

\begin{solutionbox}

ATmega32 uses USART (Universal Synchronous Asynchronous Receiver
Transmitter) for serial communication.


{\def\LTcaptype{none} % do not increment counter
\vspace{-5pt}
\captionof{table}{USART Registers}
\vspace{-10pt}
\begin{longtable}[]{@{}ll@{}}
\toprule\noalign{}
Register & Function \\
\midrule\noalign{}
\endhead
\bottomrule\noalign{}
\endlastfoot
UDR & USART Data Register (transmit/receive) \\
UCSRA & USART Control and Status Register A \\
UCSRB & USART Control and Status Register B \\
UCSRC & USART Control and Status Register C \\
UBRRH/UBRRL & USART Baud Rate Registers \\
\end{longtable}
}

\textbf{Steps to Interface RS232:}

\begin{enumerate}
\tightlist
\item
  \textbf{Hardware Connection}:

  \begin{itemize}
  \tightlist
  \item
    Connect ATmega32's TXD (PD1) and RXD (PD0) to MAX232
  \item
    Connect MAX232 to RS232 port or connector
  \end{itemize}
\item
  \textbf{Initialize USART}:

  \begin{itemize}
  \tightlist
  \item
    Set baud rate (UBRR)
  \item
    Set frame format (data bits, parity, stop bits)
  \item
    Enable transmitter and/or receiver
  \end{itemize}
\item
  \textbf{Data Transmission/Reception}:

  \begin{itemize}
  \tightlist
  \item
    Check status flags before operation
  \item
    Write to UDR to transmit
  \item
    Read from UDR to receive
  \end{itemize}
\end{enumerate}

\end{solutionbox}
\begin{mnemonicbox}
``Connect, Configure Baud, Enable, Transmit/Receive''

\end{mnemonicbox}
\subsection*{Question 3(c) [7 marks]}\label{q3c}

\textbf{Explain Bit-wise logical operations in AVR C programming with
necessary examples.}

\begin{solutionbox}

Bit-wise operations manipulate individual bits in a byte or word,
essential for embedded programming.


{\def\LTcaptype{none} % do not increment counter
\vspace{-5pt}
\captionof{table}{Bit-wise Operators in AVR C}
\vspace{-10pt}
\begin{longtable}[]{@{}llll@{}}
\toprule\noalign{}
Operator & Operation & Example & Result \\
\midrule\noalign{}
\endhead
\bottomrule\noalign{}
\endlastfoot
\& & AND & 0xA5 \& 0x0F & 0x05 \\
\textbar{} & OR & 0x50 \textbar{} 0x0F & 0x5F \\
\^{} & XOR & 0x55 \^{} 0xFF & 0xAA \\
\textasciitilde{} & NOT & \textasciitilde0x55 & 0xAA \\
\textless\textless{} & Left Shift & 0x01 \textless\textless{} 3 &
0x08 \\
\textgreater\textgreater{} & Right Shift & 0x80
\textgreater\textgreater{} 3 & 0x10 \\
\end{longtable}
}

\textbf{Example: Setting and Clearing Bits}

\begin{lstlisting}[language=C]
// Set bit 3 of PORTB
PORTB |= (1 << 3);   // PORTB = PORTB | 0b00001000

// Clear bit 5 of PORTB
PORTB &= ~(1 << 5);  // PORTB = PORTB & 0b11011111

// Toggle bit 2 of PORTB
PORTB ^= (1 << 2);   // PORTB = PORTB ^ 0b00000100

// Check if bit 4 is set
if (PINB & (1 << 4)) {
    // Bit 4 is set
}
\end{lstlisting}

\end{solutionbox}
\begin{mnemonicbox}
``AND clears, OR sets, XOR toggles, Shift
multiplies/divides''

\end{mnemonicbox}
\subsection*{Question 3(a OR) [3
marks]}\label{question-3a-or-3-marks}

\textbf{Explain RESET circuit for the ATmega32 microcontroller.}

\begin{solutionbox}

The reset circuit ensures proper initialization of ATmega32 when power
is applied or during system reset.

\textbf{Diagram:}

\begin{lstlisting}
         VCC
          |
          |
         +++
         | | 10KΩ (Pull-up)
         +++
          |
          +------+
          |      |
      +---+      |
      |   |      |
      |   C      |
   +--+--+ 100nF |
   |RESET|       |
   |     |       |
   | MCU |      GND
   +-----+
\end{lstlisting}

\begin{itemize}
\tightlist
\item
  \textbf{Active-Low RESET}: Held low to reset the microcontroller
\item
  \textbf{External Reset}: Manual reset button connects RESET pin to
  ground
\item
  \textbf{Power-on Reset}: Auto-reset when power is first applied
\item
  \textbf{Brown-out Detection}: Reset when voltage drops below threshold
\item
  \textbf{Watchdog Timer}: Reset on software malfunction
\end{itemize}

\end{solutionbox}
\begin{mnemonicbox}
``Pull Up, Push Button, Power Starts, Voltage Drops''

\end{mnemonicbox}
\subsection*{Question 3(b OR) [4
marks]}\label{question-3b-or-4-marks}

\textbf{List Registers associated with EEPROM and write steps to
interface EEPROM of ATmega32.}

\begin{solutionbox}

ATmega32 has on-chip EEPROM with dedicated registers for access control.


{\def\LTcaptype{none} % do not increment counter
\vspace{-5pt}
\captionof{table}{EEPROM Registers}
\vspace{-10pt}
\begin{longtable}[]{@{}ll@{}}
\toprule\noalign{}
Register & Function \\
\midrule\noalign{}
\endhead
\bottomrule\noalign{}
\endlastfoot
EEARH/EEARL & EEPROM Address Registers \\
EEDR & EEPROM Data Register \\
EECR & EEPROM Control Register \\
\end{longtable}
}

\textbf{Steps to Interface EEPROM:}

\begin{enumerate}
\tightlist
\item
  \textbf{Wait for Completion}:

  \begin{itemize}
  \tightlist
  \item
    Check if previous write operation is complete (EEWE bit in EECR)
  \end{itemize}
\item
  \textbf{Set Address}:

  \begin{itemize}
  \tightlist
  \item
    Load address into EEARH:EEARL (16-bit address)
  \end{itemize}
\item
  \textbf{Read or Write Operation}:

  \begin{itemize}
  \tightlist
  \item
    For read: Set EERE bit in EECR, then read EEDR
  \item
    For write: Write data to EEDR, then set EEMWE and EEWE bits in EECR
  \end{itemize}
\item
  \textbf{Wait for Completion}:

  \begin{itemize}
  \tightlist
  \item
    Poll EEWE bit until it becomes zero
  \end{itemize}
\end{enumerate}

\end{solutionbox}
\begin{mnemonicbox}
``Wait, Address, Data, Control, Wait''

\end{mnemonicbox}
\subsection*{Question 3(c OR) [7
marks]}\label{question-3c-or-7-marks}

\textbf{Write a C program to generate square wave of 1KHz on the PORTC.2
pin continuously. Use Timer0, Normal mode, and 1:8 pre-scaler to create
the delay. Assume XTAL = 8 MHz.}

\begin{solutionbox}

\begin{lstlisting}[language=C]
#include <avr/io.h>

int main(void)
{
    // Configure PORTC.2 as output
    DDRC |= (1 << 2);  // Set PC2 as output
    
    // Timer0 configuration - Normal mode, 1:8 prescaler
    TCCR0 = (0 << WGM01) | (0 << WGM00) | (0 << CS02) | (1 << CS01) | (0 << CS00);
    
    // Calculate timer value for 1KHz (500μs period, 250μs half-period)
    // 8MHz/8 = 1MHz timer clock, 250 cycles for 250μs
    // 256-250 = 6 (starting value for 250μs)
    
    while (1)
    {
        // Toggle PORTC.2
        PORTC ^= (1 << 2);
        
        // Reset timer
        TCNT0 = 6;
        
        // Wait until timer overflows
        while (!(TIFR & (1 << TOV0)));
        
        // Clear overflow flag
        TIFR |= (1 << TOV0);
    }
    
    return 0;
}
\end{lstlisting}

\begin{itemize}
\tightlist
\item
  \textbf{Frequency Calculation}: 1KHz = 1000Hz = 1ms period = 500μs
  half-period
\item
  \textbf{Timer Clock}: 8MHz \div 8 = 1MHz = 1μs per tick
\item
  \textbf{Timer Ticks}: 250μs \div 1μs = 250 ticks
\item
  \textbf{Initial Value}: 256 - 250 = 6 (for overflow after 250 ticks)
\end{itemize}

\end{solutionbox}
\begin{mnemonicbox}
``Configure, Calculate, Toggle, Reset, Wait, Clear,
Repeat''

\end{mnemonicbox}
\subsection*{Question 4(a) [3 marks]}\label{q4a}

\textbf{Draw and Explain SPI based device interfacing diagram with
ATmega32.}

\begin{solutionbox}

SPI (Serial Peripheral Interface) is a synchronous serial communication
protocol used to interface ATmega32 with peripheral devices.

\textbf{Diagram:}

\begin{lstlisting}
            ATmega32                  SPI Device
          +----------+               +----------+
          |          |               |          |
  (SS)  PB4 ---------|-------------> CS         |
 (MOSI) PB5 ---------|-------------> SDI        |
(MISO) PB6 <---------|-------------- SDO        |
 (SCK)  PB7 ---------|-------------> SCK        |
          |          |               |          |
          +----------+               +----------+
\end{lstlisting}

\begin{itemize}
\tightlist
\item
  \textbf{MOSI (Master Out Slave In)}: Data from master to slave
\item
  \textbf{MISO (Master In Slave Out)}: Data from slave to master
\item
  \textbf{SCK (Serial Clock)}: Synchronization clock provided by master
\item
  \textbf{SS (Slave Select)}: Active-low signal to select specific slave
  device
\end{itemize}

\end{solutionbox}
\begin{mnemonicbox}
``Master Outputs, Slave Inputs, Clock Keeps
Synchronization''

\end{mnemonicbox}
\subsection*{Question 4(b) [4 marks]}\label{q4b}

\textbf{Draw and explain interfacing of Relay using ULN2803 with
ATmega32.}

\begin{solutionbox}

ULN2803 is an array of Darlington transistor pairs used to drive
high-current devices like relays from microcontroller pins.

\textbf{Diagram:}

\begin{lstlisting}
  ATmega32          ULN2803            Relay
+---------+      +-----------+        +---------+
|         |      |           |        |         |
|     PD0 |----->| IN1  OUT1 |------->|+      K |
|         |      |           |  |     |         |
|     PD1 |----->| IN2  OUT2 |--┘     |         |
|         |      |           |        |         |
+---------+      |           |        +---------+
                 |       COM |------->| GND     |
     VCC ------->| VCC       |        |         |
                 +-----------+        +---------+
                                        ^
                                        |
                                       VCC
\end{lstlisting}

\begin{itemize}
\tightlist
\item
  \textbf{Current Amplification}: ULN2803 can sink up to 500mA per
  channel
\item
  \textbf{Voltage Isolation}: Built-in diodes protect against inductive
  kickback
\item
  \textbf{Multiple Channels}: 8 Darlington pairs in one package
\item
  \textbf{High Voltage Rating}: Can handle up to 50V at outputs
\end{itemize}

\end{solutionbox}
\begin{mnemonicbox}
``Low Current Controls High Current Loads''

\end{mnemonicbox}
\subsection*{Question 4(c) [7 marks]}\label{q4c}

\textbf{Draw an interfacing diagram of LM35 connected on ADC0 (pin 40)
of ATmega32 and write AVR C program to display digital result on Port B.
(use ADC in 8-bit mode).}

\begin{solutionbox}

LM35 is a precision temperature sensor that outputs an analog voltage
proportional to temperature.

\textbf{Circuit Diagram:}

\begin{lstlisting}
    +5V
     |
     |
  +--+--+
  |     |
  | LM35|
  |     |
  +--+--+
     |
     +---------> To ADC0 (PA0/Pin 40)
     |
     |
    GND
\end{lstlisting}

\textbf{C Program:}

\begin{lstlisting}[language=C]
#include <avr/io.h>
#include <util/delay.h>

int main(void)
{
    // Configure PORTB as output for displaying result
    DDRB = 0xFF;
    
    // Configure ADC
    ADMUX = (0 << REFS1) | (1 << REFS0) | // AVCC as reference
            (1 << ADLAR) |               // Left adjust result for 8-bit
            (0 << MUX4) | (0 << MUX3) | (0 << MUX2) | (0 << MUX1) | (0 << MUX0); // ADC0
    
    ADCSRA = (1 << ADEN) |               // Enable ADC
             (1 << ADPS2) | (1 << ADPS1) | (1 << ADPS0); // Prescaler 128
    
    while (1)
    {
        // Start conversion
        ADCSRA |= (1 << ADSC);
        
        // Wait for conversion to complete
        while (ADCSRA & (1 << ADSC));
        
        // Display result on PORTB (8-bit from ADCH)
        PORTB = ADCH;
        
        // Wait before next reading
        _delay_ms(500);
    }
    
    return 0;
}
\end{lstlisting}

\begin{itemize}
\tightlist
\item
  \textbf{Temperature Calculation}: LM35 outputs 10mV/^\circC
\item
  \textbf{ADC Configuration}: Left-adjusted for easy 8-bit reading
\item
  \textbf{Resolution}: Using 8-bit mode gives approximately 1^\circC
  resolution with 5V reference
\item
  \textbf{Range}: Can measure 0-255^\circC range (limited by 8-bit register)
\end{itemize}

\end{solutionbox}
\begin{mnemonicbox}
``Connect, Configure, Convert, Capture, Display''

\end{mnemonicbox}
\subsection*{Question 4(a OR) [3
marks]}\label{question-4a-or-3-marks}

\textbf{Write an AVR C program to continuous monitor PA0 pin of port A.
If it is HIGH, send HIGH to PC0 pin of port C; otherwise, send LOW to
PC0 pin of port C.}

\begin{solutionbox}

\begin{lstlisting}[language=C]
#include <avr/io.h>

int main(void)
{
    // Configure PA0 as input
    DDRA &= ~(1 << PA0);
    
    // Enable pull-up resistor on PA0
    PORTA |= (1 << PA0);
    
    // Configure PC0 as output
    DDRC |= (1 << PC0);
    
    while (1)
    {
        // Check if PA0 is HIGH
        if (PINA & (1 << PA0))
        {
            // Set PC0 HIGH
            PORTC |= (1 << PC0);
        }
        else
        {
            // Set PC0 LOW
            PORTC &= ~(1 << PC0);
        }
    }
    
    return 0;
}
\end{lstlisting}

\begin{itemize}
\tightlist
\item
  \textbf{Input Configuration}: Set as input with pull-up resistor
\item
  \textbf{Continuous Monitoring}: Infinite loop checks pin state
\item
  \textbf{Output Action}: PC0 mirrors PA0 state
\item
  \textbf{Efficient Code}: Simple conditional statement for pin
  monitoring
\end{itemize}

\end{solutionbox}
\begin{mnemonicbox}
``Configure, Monitor, Mirror''

\end{mnemonicbox}
\subsection*{Question 4(b OR) [4
marks]}\label{question-4b-or-4-marks}

\textbf{Draw ATmega32 pin diagram and write function of Vcc, AVcc and
Aref pin.}

\begin{solutionbox}

ATmega32 has 40 pins organized in a DIP package, with power supply pins
having distinct functions.

\textbf{Simplified Pin Diagram:}

\begin{lstlisting}
                 +------+
      (XCK) PB0 -|1   40|- PA0 (ADC0)
           PB1  -|2   39|- PA1 (ADC1)
(INT2/AIN0) PB2 -|3   38|- PA2 (ADC2)
 (OC0/AIN1) PB3 -|4   37|- PA3 (ADC3)
         SS PB4 -|5   36|- PA4 (ADC4)
       MOSI PB5 -|6   35|- PA5 (ADC5)
       MISO PB6 -|7   34|- PA6 (ADC6)
        SCK PB7 -|8   33|- PA7 (ADC7)
         RESET  -|9   32|- AREF
           VCC  -|10  31|- GND
           GND  -|11  30|- AVCC
         XTAL2  -|12  29|- PC7
         XTAL1  -|13  28|- PC6
     (RXD) PD0  -|14  27|- PC5
     (TXD) PD1  -|15  26|- PC4
    (INT0) PD2  -|16  25|- PC3
    (INT1) PD3  -|17  24|- PC2
    (OC1B) PD4  -|18  23|- PC1
    (OC1A) PD5  -|19  22|- PC0
     (ICP) PD6  -|20  21|- PD7 (OC2)
                 +------+
\end{lstlisting}


{\def\LTcaptype{none} % do not increment counter
\vspace{-5pt}
\captionof{table}{Power Supply Pins}
\vspace{-10pt}
\begin{longtable}[]{@{}
  >{\raggedright\arraybackslash}p{(\linewidth - 4\tabcolsep) * \real{0.1786}}
  >{\raggedright\arraybackslash}p{(\linewidth - 4\tabcolsep) * \real{0.3571}}
  >{\raggedright\arraybackslash}p{(\linewidth - 4\tabcolsep) * \real{0.4643}}@{}}
\toprule\noalign{}
\begin{minipage}[b]{\linewidth}\raggedright
Pin
\end{minipage} & \begin{minipage}[b]{\linewidth}\raggedright
Function
\end{minipage} & \begin{minipage}[b]{\linewidth}\raggedright
Description
\end{minipage} \\
\midrule\noalign{}
\endhead
\bottomrule\noalign{}
\endlastfoot
VCC & Digital Power & Main supply voltage for digital circuits (5V
typical) \\
AVCC & Analog Power & Supply for analog circuitry, particularly ADC (5V
typical) \\
AREF & Analog Reference & External reference voltage for ADC \\
\end{longtable}
}

\begin{itemize}
\tightlist
\item
  \textbf{VCC}: Powers digital logic and I/O ports
\item
  \textbf{AVCC}: Must be within \pm0.3V of VCC, even if ADC unused
\item
  \textbf{AREF}: Optional external reference for ADC, otherwise connect
  to AVCC
\end{itemize}

\end{solutionbox}
\begin{mnemonicbox}
``VCC for Core Circuits, AVCC for Analog, AREF for
Reference''

\end{mnemonicbox}
\subsection*{Question 4(c OR) [7
marks]}\label{question-4c-or-7-marks}

\textbf{Draw and explain interfacing of MAX7221 with ATmega32.}

\begin{solutionbox}

MAX7221 is an LED display driver IC that interfaces with ATmega32 using
SPI communication.

\textbf{Circuit Diagram:}

\begin{lstlisting}
 ATmega32                MAX7221                 Display
+--------+              +--------+              +--------+
|        |              |        |              |        |
|    PB4 |------------->|CS/LOAD |              |        |
|    PB5 |------------->|DIN     |              |        |
|    PB6 |<-------------|DOUT    |              |  7-SEG |
|    PB7 |------------->|CLK     |------------->| DISPLAY|
|        |              |        |              |        |
+--------+              +--------+              +--------+
\end{lstlisting}


{\def\LTcaptype{none} % do not increment counter
\vspace{-5pt}
\captionof{table}{Connection Details}
\vspace{-10pt}
\begin{longtable}[]{@{}lll@{}}
\toprule\noalign{}
ATmega32 Pin & MAX7221 Pin & Function \\
\midrule\noalign{}
\endhead
\bottomrule\noalign{}
\endlastfoot
PB4 (SS) & CS/LOAD & Chip select/Load data \\
PB5 (MOSI) & DIN & Data input to MAX7221 \\
PB6 (MISO) & DOUT & Data output (often unused) \\
PB7 (SCK) & CLK & Clock signal \\
\end{longtable}
}

\textbf{Interfacing Steps:}

\begin{enumerate}
\tightlist
\item
  \textbf{Initialize SPI:}

  \begin{itemize}
  \tightlist
  \item
    Configure SPI in master mode
  \item
    Set appropriate clock polarity and phase
  \item
    Set SS (PB4) as output and initially high
  \end{itemize}
\item
  \textbf{Initialize MAX7221:}

  \begin{itemize}
  \tightlist
  \item
    Set decode mode (BCD decode or no-decode)
  \item
    Set scan limit (number of digits)
  \item
    Set intensity (brightness)
  \item
    Turn on display
  \end{itemize}
\item
  \textbf{Send Data:}

  \begin{itemize}
  \tightlist
  \item
    Pull SS low
  \item
    Send register address followed by data
  \item
    Pull SS high to latch the data
  \end{itemize}
\end{enumerate}

\begin{lstlisting}[language=C]
// Example initialization code
void MAX7221_init() {
    // Initialize SPI
    DDRB |= (1<<PB4)|(1<<PB5)|(1<<PB7);  // SS, MOSI, SCK as outputs
    SPCR = (1<<SPE)|(1<<MSTR)|(1<<SPR0); // Enable SPI, Master, clk/16
    
    // Initialize MAX7221
    MAX7221_send(0x09, 0xFF);  // Decode mode: BCD for all digits
    MAX7221_send(0x0A, 0x0F);  // Intensity: 15/32 duty (max)
    MAX7221_send(0x0B, 0x07);  // Scan limit: display all digits
    MAX7221_send(0x0C, 0x01);  // Shutdown mode: normal operation
    MAX7221_send(0x0F, 0x00);  // Display test: normal operation
}
\end{lstlisting}

\end{solutionbox}
\begin{mnemonicbox}
``Send, Select, Clock, Data, Display''

\end{mnemonicbox}
\subsection*{Question 5(a) [3 marks]}\label{q5a}

\textbf{Draw and explain pin diagram of L293D motor driver IC.}

\begin{solutionbox}

L293D is a quadruple half-H driver designed for bidirectional control of
DC motors.

\textbf{Diagram:}

\begin{lstlisting}
        +------+
        | 1  16| 
    EN1-|      |-VCC1
    IN1-|      |-IN4
   OUT1-|      |-OUT4
    GND-| L293D|-GND
    GND-|      |-GND
   OUT2-|      |-OUT3
    IN2-|      |-IN3
   VCC2-|      |-EN2
        +------+
\end{lstlisting}


{\def\LTcaptype{none} % do not increment counter
\vspace{-5pt}
\captionof{table}{L293D Pin Functions}
\vspace{-10pt}
\begin{longtable}[]{@{}lll@{}}
\toprule\noalign{}
Pin & Name & Function \\
\midrule\noalign{}
\endhead
\bottomrule\noalign{}
\endlastfoot
1, 9 & EN1, EN2 & Enable inputs (can be PWM signals) \\
2, 7, 10, 15 & IN1-IN4 & Logic inputs \\
3, 6, 11, 14 & OUT1-OUT4 & Output pins to motors \\
4, 5, 12, 13 & GND & Ground connections \\
8 & VCC2 & Motor supply voltage (4.5V-36V) \\
16 & VCC1 & Logic supply voltage (5V) \\
\end{longtable}
}

\begin{itemize}
\tightlist
\item
  \textbf{Dual H-Bridges}: Can control two DC motors independently
\item
  \textbf{Heat Sink}: Ground pins provide heat dissipation
\item
  \textbf{High Current}: Can drive up to 600mA per channel
\item
  \textbf{Protection Diodes}: Internal flyback diodes for inductive
  loads
\end{itemize}

\end{solutionbox}
\begin{mnemonicbox}
``Enable, Input, Output, Power''

\end{mnemonicbox}
\subsection*{Question 5(b) [4 marks]}\label{q5b}

\textbf{Draw and explain ADMUX register.}

\begin{solutionbox}

ADMUX (ADC Multiplexer Selection Register) controls analog channel
selection and result format in ATmega32.

\textbf{Diagram:}

\begin{lstlisting}
+------+------+------+------+------+------+------+------+
| REFS1| REFS0| ADLAR|  --  | MUX3 | MUX2 | MUX1 | MUX0 |
+------+------+------+------+------+------+------+------+
    7      6      5      4      3      2      1      0
\end{lstlisting}


{\def\LTcaptype{none} % do not increment counter
\vspace{-5pt}
\captionof{table}{ADMUX Bit Functions}
\vspace{-10pt}
\begin{longtable}[]{@{}lll@{}}
\toprule\noalign{}
Bits & Name & Function \\
\midrule\noalign{}
\endhead
\bottomrule\noalign{}
\endlastfoot
7:6 & REFS1:0 & Reference voltage selection \\
5 & ADLAR & ADC Left Adjust Result \\
3:0 & MUX3:0 & Analog channel selection \\
\end{longtable}
}

\textbf{REFS1:0 Settings:}

\begin{itemize}
\item
  00: AREF pin (external reference)
\item
  01: AVCC with external capacitor
\item
  11: Internal 2.56V reference
\item
  \textbf{Channel Selection}: MUX3:0 selects which ADC input to connect
\item
  \textbf{Result Alignment}: ADLAR=1 shifts result left (for 8-bit
  readings)
\item
  \textbf{Differential Inputs}: Some MUX combinations allow differential
  measurements
\end{itemize}

\end{solutionbox}
\begin{mnemonicbox}
``Reference, Alignment, Multiplexer''

\end{mnemonicbox}
\subsection*{Question 5(c) [7 marks]}\label{q5c}

\textbf{Explain Smart Irrigation System.}

\begin{solutionbox}

A Smart Irrigation System uses embedded technology to efficiently manage
water for plant cultivation based on environmental conditions.

\textbf{Diagram:}

\includegraphics[width=1\linewidth,height=\textheight,keepaspectratio]{mermaid-78617e28.pdf}


{\def\LTcaptype{none} % do not increment counter
\vspace{-5pt}
\captionof{table}{Smart Irrigation Components}
\vspace{-10pt}
\begin{longtable}[]{@{}ll@{}}
\toprule\noalign{}
Component & Function \\
\midrule\noalign{}
\endhead
\bottomrule\noalign{}
\endlastfoot
Soil Moisture Sensors & Measure water content in soil \\
Temperature/Humidity Sensors & Monitor environmental conditions \\
Valves & Control water flow to different zones \\
Pump Control & Activate water pumps when needed \\
Microcontroller & Process sensor data and control outputs \\
User Interface & Allow monitoring and manual control \\
\end{longtable}
}

\textbf{Key Features:}

\begin{enumerate}
\tightlist
\item
  \textbf{Automated Watering}: Waters plants only when soil moisture
  falls below threshold
\item
  \textbf{Weather Adaptation}: Adjusts watering schedule based on
  temperature, humidity, and rain forecast
\item
  \textbf{Zone Control}: Different areas can have individual watering
  schedules
\item
  \textbf{Water Conservation}: Uses minimum necessary water for optimal
  plant growth
\item
  \textbf{Remote Monitoring}: Mobile app or web interface for system
  status and control
\item
  \textbf{Scheduling}: Time-based and condition-based watering options
\end{enumerate}

\end{solutionbox}
\begin{mnemonicbox}
``Sense, Decide, Conserve, Grow''

\end{mnemonicbox}
\subsection*{Question 5(a OR) [3
marks]}\label{question-5a-or-3-marks}

\textbf{Draw circuit diagram to interface DC motor with ATmega32 using
L293D motor driver.}

\begin{solutionbox}

The circuit connects a DC motor to ATmega32 through L293D for
bidirectional control.

\textbf{Circuit Diagram:}

\begin{lstlisting}
         ATmega32               L293D                  DC Motor
        +--------+           +--------+              +----------+
        |        |           |        |              |          |
        |     PB0|---------->|IN1     |              |          |
        |     PB1|---------->|IN2     |              |          |
        |     PB2|---------->|EN1     |              |          |
        |        |           |OUT1 >--|------------->|+         |
        |        |           |OUT2 >--|------------->|-         |
        |        |           |        |              |          |
        +--------+           +--------+              +----------+
                                 |
                                 | VCC2 (Motor power)
                              +--+--+
                              |     |
                              | 12V |
                              |     |
                              +-----+
\end{lstlisting}

\textbf{Control Logic:}

{\def\LTcaptype{none} % do not increment counter
\begin{longtable}[]{@{}llll@{}}
\toprule\noalign{}
PB0 (IN1) & PB1 (IN2) & PB2 (EN1) & Motor Status \\
\midrule\noalign{}
\endhead
\bottomrule\noalign{}
\endlastfoot
0 & 0 & 1 & Stop (brake) \\
1 & 0 & 1 & Rotate clockwise \\
0 & 1 & 1 & Rotate counter-clockwise \\
1 & 1 & 1 & Stop (brake) \\
X & X & 0 & Motor disabled \\
\end{longtable}
}

\begin{itemize}
\tightlist
\item
  \textbf{Speed Control}: PWM signal on EN1 can control motor speed
\item
  \textbf{Direction Control}: IN1 and IN2 control rotation direction
\item
  \textbf{Power Separation}: Logic powered by microcontroller, motor by
  separate supply
\end{itemize}

\end{solutionbox}
\begin{mnemonicbox}
``Enable and Direction Control Motor''

\end{mnemonicbox}
\subsection*{Question 5(b OR) [4
marks]}\label{question-5b-or-4-marks}

\textbf{Draw and Explain I2C based device interfacing diagram with
ATmega32.}

\begin{solutionbox}

I2C (Inter-Integrated Circuit) is a two-wire serial bus for connecting
multiple devices to a microcontroller.

\textbf{Diagram:}

\begin{lstlisting}
           VCC
            |
            |
         +--+--+        +-----------+        +-----------+
         |     |        |           |        |           |
         | 4.7K|        | I2C       |        | I2C       |
         | ohm |        | Device 1  |        | Device 2  |
         +--+--+        | (EEPROM)  |        | (Sensor)  |
            |           |           |        |           |
            |           |           |        |           |
 ATmega32   |           |           |        |           |
+--------+  |           |           |        |           |
|        |  |           |           |        |           |
|     PC0|--+-----------|-SDA-------|--------|-SDA-------|
|        |              |           |        |           |
|     PC1|--+-----------|-SCL-------|--------|-SCL-------|
|        |  |           |           |        |           |
+--------+  |           +-----------+        +-----------+
            |
         +--+--+
         |     |
         | 4.7K|
         | ohm |
         +--+--+
            |
            |
           VCC
\end{lstlisting}

\textbf{Key Components:}

\begin{itemize}
\tightlist
\item
  \textbf{SDA (Serial Data Line)}: Bidirectional data transfer line
\item
  \textbf{SCL (Serial Clock Line)}: Clock signal generated by master
\item
  \textbf{Pull-up Resistors}: Required on both lines (typically 4.7kΩ)
\item
  \textbf{Multiple Devices}: Each I2C device has a unique address
\end{itemize}

\textbf{Communication Process:}

\begin{enumerate}
\tightlist
\item
  \textbf{Start Condition}: SDA transitions high-to-low while SCL is
  high
\item
  \textbf{Address Transmission}: 7-bit device address followed by R/W
  bit
\item
  \textbf{Acknowledgment}: Receiving device pulls SDA low
\item
  \textbf{Data Transfer}: 8-bit data bytes with acknowledgment
\item
  \textbf{Stop Condition}: SDA transitions low-to-high while SCL is high
\end{enumerate}

\end{solutionbox}
\begin{mnemonicbox}
``Start, Address, Acknowledge, Data, Stop''

\end{mnemonicbox}
\subsection*{Question 5(c OR) [7
marks]}\label{question-5c-or-7-marks}

\textbf{Explain IoT based Home Automation System.}

\begin{solutionbox}

An IoT-based Home Automation System connects household devices to the
internet for remote monitoring and control.

\textbf{Diagram:}

\includegraphics[width=1\linewidth,height=\textheight,keepaspectratio]{mermaid-959047cd.pdf}


{\def\LTcaptype{none} % do not increment counter
\vspace{-5pt}
\captionof{table}{Home Automation Components}
\vspace{-10pt}
\begin{longtable}[]{@{}ll@{}}
\toprule\noalign{}
Component & Function \\
\midrule\noalign{}
\endhead
\bottomrule\noalign{}
\endlastfoot
Controller & Central processing unit (microcontroller/SBC) \\
Sensors & Monitor temperature, motion, light, humidity \\
Actuators & Control lights, appliances, locks, HVAC \\
Gateway & Connects to internet and local devices \\
User Interface & Mobile app, voice control, web dashboard \\
Cloud Services & Data storage, processing, and remote access \\
\end{longtable}
}

\textbf{Key Features:}

\begin{enumerate}
\tightlist
\item
  \textbf{Remote Access}: Control home devices from anywhere
\item
  \textbf{Voice Control}: Integration with voice assistants (Alexa,
  Google Home)
\item
  \textbf{Energy Management}: Monitor and optimize power consumption
\item
  \textbf{Security}: Control and monitor doors, windows, and cameras
\item
  \textbf{Scheduling}: Automate device operation based on time or events
\item
  \textbf{Scene Setting}: Predefined configurations for multiple devices
\item
  \textbf{Adaptive Control}: Learning user preferences and patterns
\end{enumerate}

\end{solutionbox}
\begin{mnemonicbox}
``Connect, Control, Monitor, Automate, Learn''

\end{mnemonicbox}

\end{document}
