\documentclass[10pt,a4paper]{article}

% content/resources/templates/preamble.tex
\usepackage[margin=0.6in]{geometry}
\author{Milav Dabgar}
\usepackage{amsmath,amssymb,amsthm}
\usepackage{booktabs}
\usepackage{multirow}
\usepackage{xcolor}
\usepackage{tcolorbox}
\tcbuselibrary{breakable,skins}
\usepackage[colorlinks=true,linkcolor=blue]{hyperref}
\usepackage{titlesec}
\usepackage{enumitem}
\usepackage{tikz}
\usepackage{pgfplots}
\usepackage{circuitikz}
\usepackage[version=4]{mhchem}
\usepackage{longtable}
\usepackage{array}
\usepackage{float}
\usepackage{caption}
\usepackage{listings}

\lstset{
  basicstyle=\small\ttfamily,
  breaklines=true,
  breakatwhitespace=false,
  postbreak=\mbox{\textcolor{red}{$\hookrightarrow$}\space},
  float=false,
  numbers=left,
  numberstyle=\tiny\color{gray},
  numbersep=10pt,
  xleftmargin=2em,
  keywordstyle=\color{blue},
  commentstyle=\color{green!60!black},
  stringstyle=\color{purple},
  backgroundcolor=\color{gray!5},
  showstringspaces=false,
  tabsize=2,
  captionpos=b,
  keepspaces=true,
  columns=flexible
}

\pgfplotsset{compat=1.18}
\usetikzlibrary{shapes,arrows,positioning,calc,patterns,decorations.pathmorphing,decorations.markings,arrows.meta}

% Color scheme
\definecolor{headcolor}{RGB}{0,102,204}
\definecolor{keycolor}{RGB}{220,20,60}
\definecolor{solutioncolor}{RGB}{34,139,34}
\definecolor{mnemoniccolor}{RGB}{148,0,211}
\definecolor{codecolor}{RGB}{0,0,100}

% Spacing
\setlength{\parskip}{3pt}
\setlist[itemize]{nosep}
\setlist[enumerate]{nosep}

% Title formatting
\titleformat{\section}{\Large\bfseries\color{headcolor}}{\thesection}{1em}{}
\titleformat{\subsection}{\large\bfseries\color{headcolor}}{\thesubsection}{1em}{}

% Pandoc tightlist compatibility
\providecommand{\tightlist}{%
  \setlength{\itemsep}{0pt}\setlength{\parskip}{0pt}}

% Pandoc longtable compatibility
\newcounter{none}
\def\thenone{}


% content/resources/templates/gujarati-boxes.tex
\usepackage{fontspec}
\usepackage{polyglossia}

% Set Gujarati as main language (document is primarily in Gujarati)
% Note: gloss-gujarati.ldf doesn't exist in polyglossia, but it will use hyphenation patterns
\setdefaultlanguage{gujarati}
\setotherlanguage{english}

% Configure Gujarati font properly
% Use Language=Default to prevent polyglossia from trying to add language-specific features
% that don't exist for Gujarati, which causes "empty feature" warnings
\newfontfamily\gujaratifont[Script=Gujarati,AutoFakeBold=2.5,AutoFakeSlant=0.3]{Noto Sans Gujarati}
\setmainfont[Script=Gujarati,AutoFakeBold=2.5,AutoFakeSlant=0.3]{Noto Sans Gujarati}
% Use Noto Sans Gujarati for monospace to support Gujarati in text
\setmonofont[Scale=0.9]{Noto Sans Gujarati}

% Configure English to use the same font
\newfontfamily\englishfont[Script=Gujarati,AutoFakeBold=2.5,AutoFakeSlant=0.3]{Noto Sans Gujarati}

% Translations for polyglossia
\gappto\captionsgujarati{
  \renewcommand{\tablename}{કોષ્ટક}
  \renewcommand{\figurename}{આકૃતિ}
}

% Helper for TikZ nodes to ensure Gujarati font
\newcommand{\gu}[1]{{\gujaratifont #1}}

% Custom environments
\newtcolorbox{solutionbox}{
    breakable,
    enhanced,
    colback=solutioncolor!5!white,
    colframe=solutioncolor!75!black,
    fonttitle=\bfseries,
    title=જવાબ
}

\newtcolorbox{solutionboxnobreak}{
 colback=solutioncolor!5!white,
 colframe=solutioncolor!75!black,
 fonttitle=\bfseries,
 title=જવાબ
}

\newtcolorbox{keyformula}{
 breakable,
 enhanced,
 colback=keycolor!5!white,
 colframe=keycolor!75!black,
 fonttitle=\bfseries,
 title=રાસાયણિક સમીકરણ/સૂત્ર
}

\newtcolorbox{mnemonicbox}{
 breakable,
 enhanced,
 colback=mnemoniccolor!5!white,
 colframe=mnemoniccolor!75!black,
 fonttitle=\bfseries,
 title=મેમરી ટ્રીક
}


\begin{document}

\begin{center}
{\Huge\bfseries\color{headcolor} Subject Name (Gujarati)}\\[5pt]
{\LARGE 4343204 -- Summer 2024}\\[3pt]
{\large Semester 1 Study Material}\\[3pt]
{\normalsize\textit{Detailed Solutions and Explanations}}
\end{center}

\vspace{10pt}

\subsection*{પ્રશ્ન 1(અ) [3
ગુણ]}\label{uxaaauxab0uxab6uxaa8-1uxa85-3-uxa97uxaa3}

\textbf{AVR સ્ટેટસ રજિસ્ટર દોરો.}

\begin{solutionbox}

AVR સ્ટેટસ રજિસ્ટર (SREG) એરિથમેટિક ઓપરેશન્સના પરિણામની માહિતી ધરાવે છે અને
ઇન્ટરપ્ટ્સને નિયંત્રિત કરે છે.

\textbf{ડાયાગ્રામ:}

\begin{verbatim}
+{-{-}{-}+{-}{-}{-}+{-}{-}{-}+{-}{-}{-}+{-}{-}{-}+{-}{-}{-}+{-}{-}{-}+{-}{-}{-}+}
| I | T | H | S | V | N | Z | C |
+{-{-}{-}+{-}{-}{-}+{-}{-}{-}+{-}{-}{-}+{-}{-}{-}+{-}{-}{-}+{-}{-}{-}+{-}{-}{-}+}
  7   6   5   4   3   2   1   0
\end{verbatim}

\begin{itemize}
\tightlist
\item
  \textbf{I (બિટ 7)}: ગ્લોબલ ઇન્ટરપ્ટ એનેબલ
\item
  \textbf{T (બિટ 6)}: બિટ કોપી સ્ટોરેજ
\item
  \textbf{H (બિટ 5)}: હાફ કેરી ફ્લેગ
\item
  \textbf{S (બિટ 4)}: સાઇન ફ્લેગ (S = N\oplusV)
\item
  \textbf{V (બિટ 3)}: ટુ'સ કોમ્પલિમેન્ટ ઓવરફ્લો
\item
  \textbf{N (બિટ 2)}: નેગેટિવ ફ્લેગ
\item
  \textbf{Z (બિટ 1)}: ઝીરો ફ્લેગ
\item
  \textbf{C (બિટ 0)}: કેરી ફ્લેગ
\end{itemize}

\end{solutionbox}
\begin{mnemonicbox}
``ઈ ટેક હેલ્થ સીરિયસલી, વેરી નાઈસ ઝીરો કેરી''

\end{mnemonicbox}
\subsection*{પ્રશ્ન 1(બ) [4
ગુણ]}\label{uxaaauxab0uxab6uxaa8-1uxaac-4-uxa97uxaa3}

\textbf{AVR માં હાર્વર્ડ આર્કિટેક્ચર સમજાવો.}

\begin{solutionbox}

AVR માં હાર્વર્ડ આર્કિટેક્ચર પ્રોગ્રામ અને ડેટા મેમરી અલગ રાખે છે, જેનાથી બંને પર એક સાથે
એક્સેસ કરી શકાય છે.

\textbf{ડાયાગ્રામ:}

\begin{center}
\textbf{Mermaid Diagram (Code)}
\begin{verbatim}
{Shaded}
{Highlighting}[]
graph TD
    CPU[CPU]
    PM[Program Memory]
    DM[Data Memory]
    CPU {-{-}{}|Instruction Bus| PM}
    CPU {-{-}{}|Data Bus| DM}
{Highlighting}
{Shaded}
\end{verbatim}
\end{center}

\begin{itemize}
\tightlist
\item
  \textbf{Program Memory}: Flash મેમરીમાં ઇન્સ્ટ્રક્શન્સ સ્ટોર કરે છે
\item
  \textbf{Data Memory}: SRAM, રજિસ્ટર્સ અને I/O રજિસ્ટર્સ ધરાવે છે
\item
  \textbf{અલગ બસ}: પ્રોગ્રામ અને ડેટા માટે અલગ બસ
\item
  \textbf{પેરેલલ એક્સેસ}: એક સાથે ઇન્સ્ટ્રક્શન ફેચ અને ડેટા એક્સેસ કરી શકાય છે
\end{itemize}

\end{solutionbox}
\begin{mnemonicbox}
``ડેટા અને પ્રોગ્રામ માટે અલગ જગ્યા''

\end{mnemonicbox}
\subsection*{પ્રશ્ન 1(ક) [7
ગુણ]}\label{uxaaauxab0uxab6uxaa8-1uxa95-7-uxa97uxaa3}

\textbf{રીયલ ટાઇમ ઓપરેટિંગ સિસ્ટમ ચર્ચો.}

\begin{solutionbox}

રીયલ-ટાઇમ ઓપરેટિંગ સિસ્ટમ (RTOS) ચુસ્ત ટાઇમિંગ જરૂરિયાતો ધરાવતા ટાસ્ક્સનું મેનેજમેન્ટ
કરે છે, અને નિશ્ચિત રિસ્પોન્સ ટાઇમ સુનિશ્ચિત કરે છે.


{\def\LTcaptype{none} % do not increment counter
\vspace{-5pt}
\captionof{table}{RTOS ની મુખ્ય વિશેષતાઓ}
\vspace{-10pt}
\begin{longtable}[]{@{}ll@{}}
\toprule\noalign{}
વિશેષતા & વર્ણન \\
\midrule\noalign{}
\endhead
\bottomrule\noalign{}
\endlastfoot
ટાસ્ક શેડ્યુલિંગ & તાત્કાલિકતાના આધારે ટાસ્ક્સને પ્રાધાન્ય આપે છે \\
નિશ્ચિત & ઘટનાઓ માટે ગેરંટેડ રિસ્પોન્સ ટાઇમ \\
પ્રિએમ્પ્ટિવ & ક્રિટિકલ ટાસ્ક ઓછા પ્રાધાન્યવાળા ટાસ્કને ઇન્ટરપ્ટ કરી શકે છે \\
મેમરી મેનેજમેન્ટ & ફ્રેગમેન્ટેશન વગર કાર્યક્ષમ મેમરી ફાળવણી \\
ઓછો લેટન્સી & ઘટના અને પ્રતિક્રિયા વચ્ચે ન્યૂનતમ વિલંબ \\
મલ્ટીટાસ્કિંગ & એકસાથે અનેક ટાસ્ક હેન્ડલ કરે છે \\
\end{longtable}
}

\begin{itemize}
\tightlist
\item
  \textbf{ટાસ્ક-બેઝ્ડ}: પ્રોગ્રામને સ્વતંત્ર ટાસ્ક્સમાં વિભાજિત કરે છે
\item
  \textbf{ઇન્ટરપ્ટ હેન્ડલિંગ}: બાહ્ય ઘટનાઓ માટે ઝડપી પ્રતિક્રિયા
\item
  \textbf{સિંક્રોનાઇઝેશન}: ટાસ્ક કોઓર્ડિનેશન માટે સેમાફોર અને મ્યુટેક્સ પૂરા પાડે છે
\item
  \textbf{રિસોર્સ મેનેજમેન્ટ}: રિસોર્સ કોન્ફ્લિક્ટ્સ અટકાવે છે
\item
  \textbf{નાનો ફૂટપ્રિન્ટ}: મર્યાદિત હાર્ડવેર રિસોર્સ માટે ઓપ્ટિમાઇઝ કરેલ છે
\end{itemize}

\end{solutionbox}
\begin{mnemonicbox}
``ટાસ્ક ચલાવે ચુસ્ત સમય પર''

\end{mnemonicbox}
\subsection*{પ્રશ્ન 1(ક OR) [7
ગુણ]}\label{uxaaauxab0uxab6uxaa8-1uxa95-or-7-uxa97uxaa3}

\textbf{એમ્બેડેડ સિસ્ટમ માટે માઇક્રોકન્ટ્રોલર પસંદ કરવા માટેના ક્રાઈટેરીયા ચર્ચો.}

\begin{solutionbox}

યોગ્ય માઇક્રોકન્ટ્રોલર પસંદ કરવા માટે એપ્લિકેશન જરૂરિયાતોને મેચ કરવા અનેક મુખ્ય
પરિબળોનું મૂલ્યાંકન કરવું જરૂરી છે.


{\def\LTcaptype{none} % do not increment counter
\vspace{-5pt}
\captionof{table}{માઇક્રોકન્ટ્રોલર પસંદગી માપદંડ}
\vspace{-10pt}
\begin{longtable}[]{@{}ll@{}}
\toprule\noalign{}
માપદંડ & વિચારણાઓ \\
\midrule\noalign{}
\endhead
\bottomrule\noalign{}
\endlastfoot
પ્રોસેસિંગ પાવર & CPU સ્પીડ, બિટ વિડ્થ (8/16/32-બિટ) \\
મેમરી & Flash, RAM, EEPROM સાઇઝ \\
પાવર કન્ઝમ્પશન & સ્લીપ મોડ, ઓપરેટિંગ વોલ્ટેજ \\
I/O કેપેબિલિટીઝ & પોર્ટ્સની સંખ્યા, સ્પેશિયલ ફંક્શન્સ \\
પેરિફેરલ્સ & ટાઇમર, ADC, કમ્યુનિકેશન ઇન્ટરફેસ \\
કોસ્ટ & યુનિટ પ્રાઇસ, ડેવલપમેન્ટ ટૂલ્સ કોસ્ટ \\
ડેવલપમેન્ટ સપોર્ટ & ટૂલ્સ, ડોક્યુમેન્ટેશન, કમ્યુનિટી \\
\end{longtable}
}

\begin{itemize}
\tightlist
\item
  \textbf{એપ્લિકેશન નીડ્સ}: કન્ટ્રોલરને ટાસ્કની જટિલતા સાથે મેચ કરવો
\item
  \textbf{રીયલ-ટાઇમ રિક્વાયરમેન્ટ}: રિસ્પોન્સ ટાઇમની મર્યાદાઓ
\item
  \textbf{એન્વાયર્નમેન્ટલ ફેક્ટર્સ}: તાપમાન, નોઇઝ, વાઇબ્રેશન
\item
  \textbf{ફોર્મ ફેક્ટર}: ભૌતિક આકાર અને પેકેજિંગ
\item
  \textbf{ભવિષ્યની એક્સ્પાન્શન}: ફીચર ગ્રોથ માટે જગ્યા
\end{itemize}

\end{solutionbox}
\begin{mnemonicbox}
``પાવર, મેમરી, I/O, પેરિફેરલ્સ, કોસ્ટ''

\end{mnemonicbox}
\subsection*{પ્રશ્ન 2(અ) [3
ગુણ]}\label{uxaaauxab0uxab6uxaa8-2uxa85-3-uxa97uxaa3}

\textbf{એમ્બેડેડ સિસ્ટમ વ્યાખ્યાયીત કરો અને તેનો જનરલ બ્લોક ડાયાગ્રામ દોરો.}

\begin{solutionbox}

એમ્બેડેડ સિસ્ટમ એ એક ડેડિકેટેડ કમ્પ્યુટર સિસ્ટમ છે જે મોટી મિકેનિકલ કે ઇલેક્ટ્રિકલ સિસ્ટમમાં
ચોક્કસ કાર્યો માટે ડિઝાઇન કરેલ છે.

\textbf{ડાયાગ્રામ:}

\begin{verbatim}
+{-{-}{-}{-}{-}{-}{-}{-}{-}{-}{-}{-}{-}+      +{-}{-}{-}{-}{-}{-}{-}{-}{-}{-}{-}{-}{-}+      +{-}{-}{-}{-}{-}{-}{-}{-}{-}{-}{-}{-}{-}{-}+}
| Input       |{-{-}{-}{-}{-}| Processing  |{-}{-}{-}{-}{-}| Output       |}
| Devices     |      | Unit        |      | Devices      |
+{-{-}{-}{-}{-}{-}{-}{-}{-}{-}{-}{-}{-}+      +{-}{-}{-}{-}{-}{-}{-}{-}{-}{-}{-}{-}{-}+      +{-}{-}{-}{-}{-}{-}{-}{-}{-}{-}{-}{-}{-}{-}+}
      \^{                    \^{}                    \^{}}
      |                    |                    |
      v                    v                    v
+{-{-}{-}{-}{-}{-}{-}{-}{-}{-}{-}{-}{-}+      +{-}{-}{-}{-}{-}{-}{-}{-}{-}{-}{-}{-}{-}+      +{-}{-}{-}{-}{-}{-}{-}{-}{-}{-}{-}{-}{-}{-}+}
| Sensors     |      | Memory      |      | Actuators    |
+{-{-}{-}{-}{-}{-}{-}{-}{-}{-}{-}{-}{-}+      +{-}{-}{-}{-}{-}{-}{-}{-}{-}{-}{-}{-}{-}+      +{-}{-}{-}{-}{-}{-}{-}{-}{-}{-}{-}{-}{-}{-}+}
                           \^{}
                           |
                           v
                     +{-{-}{-}{-}{-}{-}{-}{-}{-}{-}{-}{-}{-}+}
                     | Power       |
                     | Supply      |
                     +{-{-}{-}{-}{-}{-}{-}{-}{-}{-}{-}{-}{-}+}
\end{verbatim}

\begin{itemize}
\tightlist
\item
  \textbf{પ્રોસેસિંગ યુનિટ}: માઇક્રોકન્ટ્રોલર/માઇક્રોપ્રોસેસર
\item
  \textbf{મેમરી}: પ્રોગ્રામ અને ડેટા સ્ટોર કરે છે
\item
  \textbf{ઇનપુટ/આઉટપુટ}: બાહ્ય દુનિયા સાથે ઇન્ટરફેસ
\end{itemize}

\end{solutionbox}
\begin{mnemonicbox}
``પ્રોસેસિંગ મેમરી I/O પાવર''

\end{mnemonicbox}
\subsection*{પ્રશ્ન 2(બ) [4
ગુણ]}\label{uxaaauxab0uxab6uxaa8-2uxaac-4-uxa97uxaa3}

\textbf{દરેક પોર્ટ સાથે સંકળાયેલ I/O રજીસ્ટરની યાદી બનાવો.}

\begin{solutionbox}

AVR માઇક્રોકન્ટ્રોલર દરેક I/O પોર્ટ કંટ્રોલ કરવા માટે ત્રણ મુખ્ય રજિસ્ટર ધરાવે છે.


{\def\LTcaptype{none} % do not increment counter
\vspace{-5pt}
\captionof{table}{I/O પોર્ટ રજિસ્ટર્સ}
\vspace{-10pt}
\begin{longtable}[]{@{}lll@{}}
\toprule\noalign{}
રજિસ્ટર & ફંક્શન & વર્ણન \\
\midrule\noalign{}
\endhead
\bottomrule\noalign{}
\endlastfoot
PORTx & ડેટા રજિસ્ટર & આઉટપુટ વેલ્યુ અથવા પુલ-અપ સેટ કરે છે \\
DDRx & ડેટા ડિરેક્શન રજિસ્ટર & પિન ડિરેક્શન સેટ કરે છે (1=આઉટપુટ, 0=ઇનપુટ) \\
PINx & પોર્ટ ઇનપુટ પિન્સ & વાસ્તવિક પિન સ્ટેટસ વાંચે છે \\
\end{longtable}
}

\begin{itemize}
\tightlist
\item
  \textbf{x દર્શાવે છે}: A, B, C, D (પોર્ટનો અક્ષર)
\item
  \textbf{વધારાનાં સ્પેશિયલ}: કેટલાક પોર્ટ્સ PCMSK (પિન ચેન્જ માસ્ક) રજિસ્ટર ધરાવે
  છે
\end{itemize}

\end{solutionbox}
\begin{mnemonicbox}
``ડિરેક્શન, ડેટા, પિન રીડિંગ''

\end{mnemonicbox}
\subsection*{પ્રશ્ન 2(ક) [7
ગુણ]}\label{uxaaauxab0uxab6uxaa8-2uxa95-7-uxa97uxaa3}

\textbf{AVR માટેની ક્લોક અને રીસેટ સકીટ સમજાવો.}

\begin{solutionbox}

ક્લોક અને રીસેટ સર્કિટ્સ AVR ઓપરેશન્સના યોગ્ય ઇનિશિયલાઇઝેશન અને ટાઇમિંગ સુનિશ્ચિત કરે
છે.

\textbf{ક્લોક સર્કિટ ડાયાગ્રામ:}

\begin{verbatim}
          +{-{-}{-}{-}{-}{-}{-}+}
          |       |
     +{-{-}{-}{-}|  AVR  |{-}{-}{-}{-}+}
     |    |       |    |
     |    +{-{-}{-}{-}{-}{-}{-}+    |}
     |                 |
+{-{-}{-}{-}+{-}{-}{-}{-}+       +{-}{-}{-}{-}+{-}{-}{-}{-}+}
|         |       |         |
|  XTAL1  |       |  XTAL2  |
|         |       |         |
+{-{-}{-}{-}|{-}{-}{-}{-}+       +{-}{-}{-}{-}|{-}{-}{-}{-}+}
     |                 |
     |                 |
     +{-{-}{-}{-}{-}{-}{-}{-}+{-}{-}{-}{-}{-}{-}{-}{-}+}
              |
         +{-{-}{-}{-}+{-}{-}{-}{-}+}
         |         |
         |  XTAL   |
         |         |
         +{-{-}{-}{-}|{-}{-}{-}{-}+}
              |
              |
             GND
\end{verbatim}

\textbf{રીસેટ સર્કિટ:}

\begin{verbatim}
        VCC
         |
         |
        +++
        | | 10KΩ
        +++
         |
         +{-{-}{-}{-}{-}{-}{-}{-}+}
         |        |
         |   C    |
      +{-{-}+{-}{-}+     |}
      |RESET|     |
      |     |    GND
      | AVR |
      +{-{-}{-}{-}{-}+}
\end{verbatim}

\begin{itemize}
\tightlist
\item
  \textbf{ક્લોક સોર્સ}: એક્સટર્નલ ક્રિસ્ટલ, RC ઓસિલેટર, અથવા ઇન્ટરનલ ઓસિલેટર
\item
  \textbf{ક્રિસ્ટલ}: ચોક્કસ ટાઇમિંગ પૂરું પાડે છે (1-16 MHz)
\item
  \textbf{રીસેટ પિન}: સિસ્ટમ રીસ્ટાર્ટ માટે એક્ટિવ-લો ઇનપુટ
\item
  \textbf{પાવર-ઓન રીસેટ}: પાવર આપતી વખતે ઓટોમેટિક રીસેટ
\item
  \textbf{બ્રાઉન-આઉટ ડિટેક્શન}: જો વોલ્ટેજ નિશ્ચિત થ્રેશોલ્ડથી નીચે જાય તો રીસેટ
\end{itemize}

\end{solutionbox}
\begin{mnemonicbox}
``ક્રિસ્ટલ ઓસિલેટ કરે, રીસેટ શરૂઆત કરાવે''

\end{mnemonicbox}
\subsection*{પ્રશ્ન 2(અ OR) [3
ગુણ]}\label{uxaaauxab0uxab6uxaa8-2uxa85-or-3-uxa97uxaa3}

\textbf{એમ્બેડેડ સિસ્ટમની લાક્ષણિકતાઓ લખો.}

\begin{solutionbox}

એમ્બેડેડ સિસ્ટમની અનન્ય લાક્ષણિકતાઓ તેને જનરલ-પરપઝ કમ્પ્યુટરથી અલગ પાડે છે.


{\def\LTcaptype{none} % do not increment counter
\vspace{-5pt}
\captionof{table}{એમ્બેડેડ સિસ્ટમની લાક્ષણિકતાઓ}
\vspace{-10pt}
\begin{longtable}[]{@{}ll@{}}
\toprule\noalign{}
લાક્ષણિકતા & વર્ણન \\
\midrule\noalign{}
\endhead
\bottomrule\noalign{}
\endlastfoot
સિંગલ-ફંક્શન & ચોક્કસ ટાસ્ક માટે સમર્પિત \\
રીયલ-ટાઇમ & અનુમાનિત પ્રતિક્રિયા સમય \\
રિસોર્સ-કન્સ્ટ્રેઇન્ડ & મર્યાદિત મેમરી, પાવર, પ્રોસેસિંગ \\
વિશ્વસનીયતા & નિષ્ફળતા વગર સતત ચાલવું જોઈએ \\
રીએક્ટિવ & પર્યાવરણીય ફેરફારોને પ્રતિસાદ આપે છે \\
\end{longtable}
}

\begin{itemize}
\tightlist
\item
  \textbf{લાંબું આયુષ્ય}: ઘણીવાર વર્ષો સુધી હસ્તક્ષેપ વિના કામ કરે છે
\item
  \textbf{ઘણીવાર છુપાયેલ}: મોટી સિસ્ટમમાં એકીકૃત
\end{itemize}

\end{solutionbox}
\begin{mnemonicbox}
``સિંગલ, રીયલ-ટાઇમ, રિસોર્સ-મર્યાદિત, વિશ્વસનીય''

\end{mnemonicbox}
\subsection*{પ્રશ્ન 2(બ OR) [4
ગુણ]}\label{uxaaauxab0uxab6uxaa8-2uxaac-or-4-uxa97uxaa3}

\textbf{ડેટા આઉટપુટ અને ઇનપુટ કરવામાં DDRx રજીસ્ટરની ભૂમિકાની ચર્ચા કરો.}

\begin{solutionbox}

DDRx (ડેટા ડાઇરેક્શન રજિસ્ટર) પોર્ટ x ના દરેક પિનને ઇનપુટ કે આઉટપુટ તરીકે કન્ફિગર કરે
છે.


{\def\LTcaptype{none} % do not increment counter
\vspace{-5pt}
\captionof{table}{I/O ઓપરેશન્સમાં DDRx ની ભૂમિકા}
\vspace{-10pt}
\begin{longtable}[]{@{}llll@{}}
\toprule\noalign{}
DDRx વેલ્યુ & PORTx વેલ્યુ & મોડ & ફંક્શન \\
\midrule\noalign{}
\endhead
\bottomrule\noalign{}
\endlastfoot
0 & 0 & ઇનપુટ & હાઇ-ઇમ્પીડન્સ મોડ \\
0 & 1 & ઇનપુટ & પુલ-અપ એનેબલ્ડ \\
1 & 0 & આઉટપુટ & આઉટપુટ લો (0V) \\
1 & 1 & આઉટપુટ & આઉટપુટ હાઇ (VCC) \\
\end{longtable}
}

\begin{itemize}
\tightlist
\item
  \textbf{ડિરેક્શન કંટ્રોલ}: 1 = આઉટપુટ, 0 = ઇનપુટ
\item
  \textbf{પિન-સ્પેસિફિક}: દરેક બિટ વ્યક્તિગત પિન નિયંત્રિત કરે છે
\item
  \textbf{ઇનિશિયલ સ્ટેટ}: ડિફોલ્ટ ઇનપુટ (બધા 0s) છે
\end{itemize}

\end{solutionbox}
\begin{mnemonicbox}
``ડિરેક્શન નક્કી કરે ડેટા ફ્લો''

\end{mnemonicbox}
\subsection*{પ્રશ્ન 2(ક OR) [7
ગુણ]}\label{uxaaauxab0uxab6uxaa8-2uxa95-or-7-uxa97uxaa3}

\textbf{ATmega32નો પીન ડાયાગ્રામ દોરી સમજાવો.}

\begin{solutionbox}

ATmega32 એ 40 પિન ધરાવતો લોકપ્રિય 8-બિટ AVR માઇક્રોકન્ટ્રોલર છે જે વિવિધ
કાર્યક્ષમતા પ્રદાન કરે છે.

\textbf{ડાયાગ્રામ:}

\begin{verbatim}
               +{-{-}{-}{-}{-}{-}+}
    (XCK) PB0 {-|1   40|{-} PA0 (ADC0)}
         PB1  {-|2   39|{-} PA1 (ADC1)}
(INT2/AIN0)PB2{-|3   38|{-} PA2 (ADC2)}
(OC0/AIN1)PB3 {-|4   37|{-} PA3 (ADC3)}
       SS PB4 {-|5   36|{-} PA4 (ADC4)}
     MOSI PB5 {-|6   35|{-} PA5 (ADC5)}
     MISO PB6 {-|7   34|{-} PA6 (ADC6)}
      SCK PB7 {-|8   33|{-} PA7 (ADC7)}
       RESET  {-|9   32|{-} AREF}
         VCC  {-|10  31|{-} GND}
         GND  {-|11  30|{-} AVCC}
       XTAL2  {-|12  29|{-} PC7 (TOSC2)}
       XTAL1  {-|13  28|{-} PC6 (TOSC1)}
   (RXD) PD0  {-|14  27|{-} PC5}
   (TXD) PD1  {-|15  26|{-} PC4}
  (INT0) PD2  {-|16  25|{-} PC3}
  (INT1) PD3  {-|17  24|{-} PC2}
  (OC1B) PD4  {-|18  23|{-} PC1}
  (OC1A) PD5  {-|19  22|{-} PC0}
   (ICP) PD6  {-|20  21|{-} PD7 (OC2)}
               +{-{-}{-}{-}{-}{-}+}
\end{verbatim}

\begin{itemize}
\tightlist
\item
  \textbf{પોર્ટ A (PA0-PA7)}: 8-બિટ બાયડાયરેક્શનલ પોર્ટ ADC ઇનપુટ સાથે
\item
  \textbf{પોર્ટ B (PB0-PB7)}: 8-બિટ પોર્ટ SPI, ટાઇમર્સ, અને એક્સટર્નલ ઇન્ટરપ્ટ
  સાથે
\item
  \textbf{પોર્ટ C (PC0-PC7)}: 8-બિટ બાયડાયરેક્શનલ પોર્ટ TWI સપોર્ટ સાથે
\item
  \textbf{પોર્ટ D (PD0-PD7)}: 8-બિટ પોર્ટ USART, એક્સટર્નલ ઇન્ટરપ્ટ, અને PWM
  સાથે
\item
  \textbf{પાવર/ગ્રાઉન્ડ}: VCC, GND, AVCC, AREF
\item
  \textbf{ક્લોક}: XTAL1/XTAL2 એક્સટર્નલ ઓસિલેટર માટે
\item
  \textbf{રીસેટ}: એક્ટિવ-લો રીસેટ ઇનપુટ
\end{itemize}

\end{solutionbox}
\begin{mnemonicbox}
``ABCD પોર્ટ્સ, પાવર, ક્લોક, રીસેટની ચારે બાજુ''

\end{mnemonicbox}
\subsection*{પ્રશ્ન 3(અ) [3
ગુણ]}\label{uxaaauxab0uxab6uxaa8-3uxa85-3-uxa97uxaa3}

\textbf{ATmega32 માટે પ્રોગ્રામ કાઉન્ટર (PC) રજિસ્ટર સમજાવો.}

\begin{solutionbox}

પ્રોગ્રામ કાઉન્ટર (PC) એ 16-બિટ રજિસ્ટર છે જે એક્ઝિક્યુટ કરવા માટેના આગામી
ઇન્સ્ટ્રક્શનના એડ્રેસને ટ્રેક કરે છે.

\textbf{ડાયાગ્રામ:}

\begin{verbatim}
+{-{-}{-}{-}{-}{-}{-}{-}{-}+{-}{-}{-}{-}{-}{-}{-}{-}+}
| PC High | PC Low |
+{-{-}{-}{-}{-}{-}{-}{-}{-}+{-}{-}{-}{-}{-}{-}{-}{-}+}
    15:8     7:0
\end{verbatim}

\begin{itemize}
\tightlist
\item
  \textbf{ફંક્શન}: પ્રોગ્રામ મેમરીમાં આગામી ઇન્સ્ટ્રક્શન તરફ પોઇન્ટ કરે છે
\item
  \textbf{સાઇઝ}: 16-બિટ (64K શબ્દો સુધી એડ્રેસ કરી શકાય)
\item
  \textbf{ઓટો-ઇન્ક્રિમેન્ટ}: ઇન્સ્ટ્રક્શન ફેચ પછી આપોઆપ વધે છે
\item
  \textbf{જમ્પ કંટ્રોલ}: બ્રાન્ચ અને જમ્પ ઇન્સ્ટ્રક્શન્સ દ્વારા મોડિફાય થાય છે
\end{itemize}

\end{solutionbox}
\begin{mnemonicbox}
``કોડ એક્ઝિક્યુશન તરફ પોઇન્ટ કરે''

\end{mnemonicbox}
\subsection*{પ્રશ્ન 3(બ) [4
ગુણ]}\label{uxaaauxab0uxab6uxaa8-3uxaac-4-uxa97uxaa3}

\textbf{EEPROM ના 0x005F લોકેશન પરથી ડેટા રીડ કરી PORTB પર મોકલવા માટે AVR
C પ્રોગ્રામ લખો.}

\begin{solutionbox}

\begin{verbatim}
\#include {avr/io.h}
\#include {avr/eeprom.h}

int main(void)
\{
    // PORTB ને આઉટપુટ તરીકે સેટ કરો
    DDRB = 0xFF;
    
    // EEPROM લોકેશન 0x005F પરથી વાંચો અને PORTB પર આઉટપુટ કરો
    PORTB = eeprom\_read\_byte((uint8\_t*)0x005F);
    
    while(1) \{
        // મુખ્ય લૂપ
    \}
    return 0;
\}
\end{verbatim}

\begin{itemize}
\tightlist
\item
  \textbf{DDRB = 0xFF}: બધા PORTB પિન્સને આઉટપુટ તરીકે કન્ફિગર કરે છે
\item
  \textbf{eeprom\_read\_byte()}: EEPROM વાંચવા માટે AVR લાઇબ્રેરી ફંક્શન
\item
  \textbf{while(1)}: આઉટપુટ જાળવવા માટે અનંત લૂપ
\end{itemize}

\end{solutionbox}
\begin{mnemonicbox}
``ડિરેક્શન, EEPROM વાંચો, પોર્ટ પર આઉટપુટ''

\end{mnemonicbox}
\subsection*{પ્રશ્ન 3(ક) [7
ગુણ]}\label{uxaaauxab0uxab6uxaa8-3uxa95-7-uxa97uxaa3}

\textbf{TCCR0 રજિસ્ટર દોરી વિગતવાર સમજાવો.}

\begin{solutionbox}

ટાઇમર/કાઉન્ટર કંટ્રોલ રજિસ્ટર 0 (TCCR0) ટાઇમર/કાઉન્ટર0ના ઓપરેશનને કંટ્રોલ કરે છે.

\textbf{ડાયાગ્રામ:}

\begin{verbatim}
+{-{-}{-}{-}{-}+{-}{-}{-}{-}{-}{-}+{-}{-}{-}{-}{-}{-}+{-}{-}{-}{-}{-}+{-}{-}{-}{-}{-}+{-}{-}{-}{-}{-}+{-}{-}{-}{-}{-}+{-}{-}{-}{-}{-}+}
| FOC0| WGM00| COM01|COM00|WGM01| CS02| CS01| CS00|
+{-{-}{-}{-}{-}+{-}{-}{-}{-}{-}{-}+{-}{-}{-}{-}{-}{-}+{-}{-}{-}{-}{-}+{-}{-}{-}{-}{-}+{-}{-}{-}{-}{-}+{-}{-}{-}{-}{-}+{-}{-}{-}{-}{-}+}
   7     6     5     4     3     2     1     0
\end{verbatim}


{\def\LTcaptype{none} % do not increment counter
\vspace{-5pt}
\captionof{table}{TCCR0 બિટ્સ ફંક્શન}
\vspace{-10pt}
\begin{longtable}[]{@{}lll@{}}
\toprule\noalign{}
બિટ(સ) & નામ & ફંક્શન \\
\midrule\noalign{}
\endhead
\bottomrule\noalign{}
\endlastfoot
7 & FOC0 & ફોર્સ આઉટપુટ કમ્પેર \\
6,3 & WGM01:0 & વેવફોર્મ જનરેશન મોડ \\
5,4 & COM01:0 & કમ્પેર મેચ આઉટપુટ મોડ \\
2,1,0 & CS02:0 & ક્લોક સિલેક્ટ \\
\end{longtable}
}

\begin{itemize}
\tightlist
\item
  \textbf{WGM01:0}: નોર્મલ, CTC, અથવા PWM મોડ પસંદ કરે છે
\item
  \textbf{COM01:0}: કમ્પેર મેચ પર OC0 પિન વર્તણૂક વ્યાખ્યાયિત કરે છે
\item
  \textbf{CS02:0}: ક્લોક સોર્સ અને પ્રીસ્કેલર સેટ કરે છે (1, 8, 64, 256, 1024)
\end{itemize}

\end{solutionbox}
\begin{mnemonicbox}
``ફોર્સિંગ વેવફોર્મ્સ, કમ્પેરિંગ, સિલેક્ટિંગ ક્લોક''

\end{mnemonicbox}
\subsection*{પ્રશ્ન 3(અ OR) [3
ગુણ]}\label{uxaaauxab0uxab6uxaa8-3uxa85-or-3-uxa97uxaa3}

\textbf{AVR ડેટા મેમરી સમજાવો.}

\begin{solutionbox}

AVR ડેટા મેમરીમાં વિવિધ પ્રકારના ડેટા સ્ટોરેજ માટે અનેક સેક્શન્સ હોય છે.

\textbf{ડાયાગ્રામ:}

\begin{center}
\textbf{Mermaid Diagram (Code)}
\begin{verbatim}
{Shaded}
{Highlighting}[]
graph TD
    A[AVR ડેટા મેમરી]
    A {-{-}{} B[રજિસ્ટર્સ]}
    A {-{-}{} C[I/O રજિસ્ટર્સ]}
    A {-{-}{} D[ઇન્ટરનલ SRAM]}
    A {-{-}{} E[EEPROM]}
{Highlighting}
{Shaded}
\end{verbatim}
\end{center}

\begin{itemize}
\tightlist
\item
  \textbf{રજિસ્ટર્સ}: 32 જનરલ-પરપઝ રજિસ્ટર્સ (R0-R31)
\item
  \textbf{I/O મેમરી}: પેરિફેરલ્સ માટે સ્પેશિયલ ફંક્શન રજિસ્ટર્સ
\item
  \textbf{SRAM}: વેરિએબલ્સ માટે ઇન્ટરનલ RAM (વોલેટાઇલ)
\item
  \textbf{EEPROM}: સાતત્યપૂર્ણ ડેટા માટે નોન-વોલેટાઇલ મેમરી
\end{itemize}

\end{solutionbox}
\begin{mnemonicbox}
``રજિસ્ટર્સ I/O SRAM EEPROM''

\end{mnemonicbox}
\subsection*{પ્રશ્ન 3(બ OR) [4
ગુણ]}\label{uxaaauxab0uxab6uxaa8-3uxaac-or-4-uxa97uxaa3}

\textbf{EEPROM ના 0x005F લોકેશન પર `G' સ્ટોર કરવા માટે AVR C પ્રોગ્રામ લખો.}

\begin{solutionbox}

\begin{verbatim}
\#include {avr/io.h}
\#include {avr/eeprom.h}

int main(void)
\{
    // {G કેરેક્ટરને EEPROM લોકેશન 0x005F પર સ્ટોર કરો}
    eeprom\_write\_byte((uint8\_t*)0x005F, {G});
    
    while(1) \{
        // મુખ્ય લૂપ
    \}
    return 0;
\}
\end{verbatim}

\begin{itemize}
\tightlist
\item
  \textbf{eeprom\_write\_byte()}: EEPROM માં લખવા માટે AVR લાઇબ્રેરી ફંક્શન
\item
  \textbf{`G'}: ASCII વેલ્યુ 71 (0x47) EEPROM માં સ્ટોર થાય છે
\item
  \textbf{0x005F}: ટાર્ગેટ EEPROM એડ્રેસ
\item
  \textbf{while(1)}: લખ્યા પછી અનંત લૂપ
\end{itemize}

\end{solutionbox}
\begin{mnemonicbox}
``એક વાર લખો, હંમેશા માટે યાદ રાખો''

\end{mnemonicbox}
\subsection*{પ્રશ્ન 3(ક OR) [7
ગુણ]}\label{uxaaauxab0uxab6uxaa8-3uxa95-or-7-uxa97uxaa3}

\textbf{TIFR રજિસ્ટર દોરી વિગતવાર સમજાવો.}

\begin{solutionbox}

ટાઇમર/કાઉન્ટર ઇન્ટરપ્ટ ફ્લેગ રજિસ્ટર (TIFR) ટાઇમર ઇવેન્ટ્સ સૂચવતા ફ્લેગ ધરાવે છે.

\textbf{ડાયાગ્રામ:}

\begin{verbatim}
+{-{-}{-}{-}{-}+{-}{-}{-}{-}{-}+{-}{-}{-}{-}{-}+{-}{-}{-}{-}{-}+{-}{-}{-}{-}{-}+{-}{-}{-}{-}{-}+{-}{-}{-}{-}{-}+{-}{-}{-}{-}{-}+}
|  {-  |  {-}  |  {-}  |  {-}  |  {-}  |OCF2 |TOV2 |TOV0 |}
+{-{-}{-}{-}{-}+{-}{-}{-}{-}{-}+{-}{-}{-}{-}{-}+{-}{-}{-}{-}{-}+{-}{-}{-}{-}{-}+{-}{-}{-}{-}{-}+{-}{-}{-}{-}{-}+{-}{-}{-}{-}{-}+}
   7     6     5     4     3     2     1     0
\end{verbatim}


{\def\LTcaptype{none} % do not increment counter
\vspace{-5pt}
\captionof{table}{TIFR બિટ્સ ફંક્શન}
\vspace{-10pt}
\begin{longtable}[]{@{}lll@{}}
\toprule\noalign{}
બિટ & નામ & ફંક્શન \\
\midrule\noalign{}
\endhead
\bottomrule\noalign{}
\endlastfoot
0 & TOV0 & ટાઇમર/કાઉન્ટર0 ઓવરફ્લો ફ્લેગ \\
1 & TOV2 & ટાઇમર/કાઉન્ટર2 ઓવરફ્લો ફ્લેગ \\
2 & OCF2 & આઉટપુટ કમ્પેર ફ્લેગ 2 \\
3-7 & - & રિઝર્વ્ડ બિટ્સ \\
\end{longtable}
}

\begin{itemize}
\tightlist
\item
  \textbf{TOV0}: ટાઇમર0 ઓવરફ્લો થતાં સેટ થાય છે, ISR એક્ઝિક્યુટ થતાં ક્લિયર થાય છે
\item
  \textbf{TOV2}: ટાઇમર2 ઓવરફ્લો થતાં સેટ થાય છે
\item
  \textbf{OCF2}: ટાઇમર2 કમ્પેર મેચ થતાં સેટ થાય છે
\item
  \textbf{ફ્લેગ ક્લિયરિંગ}: ફ્લેગ ક્લિયર કરવા બિટને `1' લખો
\end{itemize}

\end{solutionbox}
\begin{mnemonicbox}
``ટાઇમર્સ ઓવરફ્લો, કમ્પેરિઝન ફ્લેગ''

\end{mnemonicbox}
\subsection*{પ્રશ્ન 4(અ) [3
ગુણ]}\label{uxaaauxab0uxab6uxaa8-4uxa85-3-uxa97uxaa3}

\textbf{AVRમાં ટાઇમ ડીલે જનરેટ કરવાની વિવિધ રીતો લખો.}

\begin{solutionbox}

AVR માઇક્રોકન્ટ્રોલર્સ ટાઇમ ડિલે જનરેટ કરવા માટે અનેક પદ્ધતિઓ ઓફર કરે છે.


{\def\LTcaptype{none} % do not increment counter
\vspace{-5pt}
\captionof{table}{ડિલે જનરેશન પદ્ધતિઓ}
\vspace{-10pt}
\begin{longtable}[]{@{}lll@{}}
\toprule\noalign{}
પદ્ધતિ & વર્ણન & પ્રિસિઝન \\
\midrule\noalign{}
\endhead
\bottomrule\noalign{}
\endlastfoot
સોફ્ટવેર લૂપ્સ & CPU સાયકલ્સ કાઉન્ટિંગ & ઓછી \\
ટાઇમર ઇન્ટરપ્ટ્સ & ISR સાથે હાર્ડવેર ટાઇમર્સ & ઉચ્ચ \\
ટાઇમર પોલિંગ & ફ્લેગ ચેકિંગ સાથે હાર્ડવેર ટાઇમર્સ & મધ્યમ \\
ડિલે ફંક્શન્સ & લાઇબ્રેરી ફંક્શન્સ (\_delay\_ms/\_delay\_us) & મધ્યમ \\
\end{longtable}
}

\begin{itemize}
\tightlist
\item
  \textbf{સોફ્ટવેર}: સરળ પરંતુ ઓપ્ટિમાઇઝેશન્સથી અસર પામે
\item
  \textbf{હાર્ડવેર}: વધુ ચોક્કસ પરંતુ ટાઇમર સેટઅપની જરૂર
\item
  \textbf{લાઇબ્રેરી}: સુવિધાજનક પરંતુ કોન્સ્ટન્ટ વેલ્યુ સુધી મર્યાદિત
\end{itemize}

\end{solutionbox}
\begin{mnemonicbox}
``લૂપ્સ, ઇન્ટરપ્ટ્સ, પોલિંગ, ફંક્શન્સ''

\end{mnemonicbox}
\subsection*{પ્રશ્ન 4(બ) [4
ગુણ]}\label{uxaaauxab0uxab6uxaa8-4uxaac-4-uxa97uxaa3}

\textbf{LM35નુ ATmega32 સાથે ઇન્ટરફેસિંગ દોરો અને સમજાવો.}

\begin{solutionbox}

LM35 એ તાપમાનના પ્રમાણસર એનાલોગ વોલ્ટેજ આઉટપુટ આપતો તાપમાન સેન્સર છે.

\textbf{સર્કિટ ડાયાગ્રામ:}

\begin{verbatim}
    VCC (+5V)
      |
      |
  +{-{-}{-}+{-}{-}{-}+}
  |       |
  | LM35  |
  |       |
  +{-{-}{-}+{-}{-}{-}+}
      |
      +{-{-}{-}{-}{-}{-}{-} To ADC0 (PA0)}
      |
      |
     GND
\end{verbatim}

\begin{itemize}
\tightlist
\item
  \textbf{કનેક્શન}: LM35 આઉટપુટ ATmega32 ના ADC0 (PA0) પર
\item
  \textbf{સ્કેલિંગ}: 10mV/^\circC આઉટપુટ (0^\circC = 0V, 25^\circC = 250mV)
\item
  \textbf{ADC સેટઅપ}: ADC0 પસંદ કરવા ADMUX કન્ફિગર કરો
\item
  \textbf{ગણતરી}: તાપમાન = (ADC\_value * 5 * 100) / 1024
\end{itemize}

\end{solutionbox}
\begin{mnemonicbox}
``એનાલોગ વોલ્ટેજ તાપમાન બદલે''

\end{mnemonicbox}
\subsection*{પ્રશ્ન 4(ક) [7
ગુણ]}\label{uxaaauxab0uxab6uxaa8-4uxa95-7-uxa97uxaa3}

\textbf{MAX7221નુ ATmega32 સાથે ઇન્ટરફેસિંગ વિગતવાર સમજાવો.}

\begin{solutionbox}

MAX7221 એ SPI કમ્યુનિકેશન દ્વારા AVR સાથે જોડાતી LED ડિસ્પ્લે ડ્રાઇવર IC છે.

\textbf{સર્કિટ ડાયાગ્રામ:}

\begin{verbatim}
 ATmega32                MAX7221
+{-{-}{-}{-}{-}{-}{-}{-}+              +{-}{-}{-}{-}{-}{-}{-}{-}+}
|        |              |        |
|     PB7|{-{-}{-}{-}{-}{-}{-}{-}{-}{-}{-}{-}{-}|CLK     |}
|     PB5|{-{-}{-}{-}{-}{-}{-}{-}{-}{-}{-}{-}{-}|DIN     |}
|     PB4|{-{-}{-}{-}{-}{-}{-}{-}{-}{-}{-}{-}{-}|LOAD    |}
|        |              |        |
+{-{-}{-}{-}{-}{-}{-}{-}+              +{-}{-}{-}{-}{-}{-}{-}{-}+}
                             |
                        +{-{-}{-}{-}+{-}{-}{-}{-}+}
                        |         |
                        | 7{-SEG   |}
                        | DISPLAY |
                        |         |
                        +{-{-}{-}{-}{-}{-}{-}{-}{-}+}
\end{verbatim}


{\def\LTcaptype{none} % do not increment counter
\vspace{-5pt}
\captionof{table}{કનેક્શન્સ અને ફંક્શનાલિટી}
\vspace{-10pt}
\begin{longtable}[]{@{}lll@{}}
\toprule\noalign{}
ATmega32 પિન & MAX7221 પિન & ફંક્શન \\
\midrule\noalign{}
\endhead
\bottomrule\noalign{}
\endlastfoot
PB7 (SCK) & CLK & સીરિયલ ક્લોક \\
PB5 (MOSI) & DIN & ડેટા ઇનપુટ \\
PB4 (SS) & LOAD & ચિપ સિલેક્ટ \\
\end{longtable}
}

\begin{itemize}
\tightlist
\item
  \textbf{SPI મોડ}: માસ્ટર મોડ, MSB ફર્સ્ટ
\item
  \textbf{ઇનિશિયલાઇઝેશન}: ડિકોડ મોડ, ઇન્ટેન્સિટી, સ્કેન લિમિટ સેટ કરે
\item
  \textbf{ડેટા ટ્રાન્સફર}: એડ્રેસ બાય્ટ પછી ડેટા બાય્ટ મોકલે
\item
  \textbf{મલ્ટિપ્લેક્સિંગ}: 8 ડિજિટ્સ સુધી ડ્રાઇવ કરી શકે
\item
  \textbf{બ્રાઇટનેસ કંટ્રોલ}: ઇન્ટેન્સિટી રજિસ્ટર દ્વારા 16 લેવલ
\end{itemize}

\end{solutionbox}
\begin{mnemonicbox}
``ક્લોક ડેટા લોડ ડિસ્પ્લે મોકલો''

\end{mnemonicbox}
\subsection*{પ્રશ્ન 4(અ OR) [3
ગુણ]}\label{uxaaauxab0uxab6uxaa8-4uxa85-or-3-uxa97uxaa3}

\textbf{MAX232 લાઈન ડ્રાઈવર સમજાવો.}

\begin{solutionbox}

MAX232 એ TTL/CMOS લોજિક લેવલ્સને RS-232 વોલ્ટેજ લેવલ્સમાં સીરિયલ કમ્યુનિકેશન માટે
કન્વર્ટ કરતી IC છે.

\textbf{ડાયાગ્રામ:}

\begin{verbatim}
    +{-{-}{-}{-}{-}{-}{-}+          +{-}{-}{-}{-}{-}{-}{-}+}
    |       |C1+    C1{-|       |}
+{-{-}{-}|T1IN   |          |  T1OUT|{-}{-}{-}+}
|   |       |          |       |   |
|   |       |C2+    C2{-|       |   |}
|   |       |          |       |   |
|   |       |          |       |   |
+{-{-}{-}|R1OUT  |          |   R1IN|{-}{-}{-}+}
    |       |          |       |
    |MAX232 |          |  RS232|
    +{-{-}{-}{-}{-}{-}{-}+          +{-}{-}{-}{-}{-}{-}{-}+}
\end{verbatim}

\begin{itemize}
\tightlist
\item
  \textbf{વોલ્ટેજ કન્વર્ઝન}: TTL (0/5V) થી RS-232 (\pm12V)
\item
  \textbf{ચાર્જ પમ્પ્સ}: જરૂરી વોલ્ટેજ જનરેટ કરવા કેપેસિટર્સ વાપરે છે
\item
  \textbf{એપ્લિકેશન્સ}: PC, મોડેમ સાથે સીરિયલ કમ્યુનિકેશન
\item
  \textbf{બાયડાયરેક્શનલ}: ટ્રાન્સમિટ અને રિસીવ બંને સિગ્નલ હેન્ડલ કરે છે
\end{itemize}

\end{solutionbox}
\begin{mnemonicbox}
``TTL થી RS-232 કન્વર્ઝન''

\end{mnemonicbox}
\subsection*{પ્રશ્ન 4(બ OR) [4
ગુણ]}\label{uxaaauxab0uxab6uxaa8-4uxaac-or-4-uxa97uxaa3}

\textbf{ADMUX રજીસ્ટર સમજાવો.}

\begin{solutionbox}

ADC મલ્ટિપ્લેક્સર સિલેક્શન રજિસ્ટર (ADMUX) એનાલોગ ઇનપુટ ચેનલ સિલેક્શન અને રિઝલ્ટ
ફોર્મેટ કંટ્રોલ કરે છે.

\textbf{ડાયાગ્રામ:}

\begin{verbatim}
+{-{-}{-}{-}{-}+{-}{-}{-}{-}{-}+{-}{-}{-}{-}{-}+{-}{-}{-}{-}{-}+{-}{-}{-}{-}{-}+{-}{-}{-}{-}{-}+{-}{-}{-}{-}{-}+{-}{-}{-}{-}{-}+}
|REFS1|REFS0|ADLAR| {-   |MUX3 |MUX2 |MUX1 |MUX0 |}
+{-{-}{-}{-}{-}+{-}{-}{-}{-}{-}+{-}{-}{-}{-}{-}+{-}{-}{-}{-}{-}+{-}{-}{-}{-}{-}+{-}{-}{-}{-}{-}+{-}{-}{-}{-}{-}+{-}{-}{-}{-}{-}+}
   7     6     5     4     3     2     1     0
\end{verbatim}


{\def\LTcaptype{none} % do not increment counter
\vspace{-5pt}
\captionof{table}{ADMUX બિટ ફંક્શન્સ}
\vspace{-10pt}
\begin{longtable}[]{@{}lll@{}}
\toprule\noalign{}
બિટ્સ & નામ & ફંક્શન \\
\midrule\noalign{}
\endhead
\bottomrule\noalign{}
\endlastfoot
7:6 & REFS1:0 & રેફરન્સ સિલેક્શન \\
5 & ADLAR & ADC લેફ્ટ એડજસ્ટ રિઝલ્ટ \\
3:0 & MUX3:0 & એનાલોગ ચેનલ સિલેક્શન \\
\end{longtable}
}

\begin{itemize}
\tightlist
\item
  \textbf{REFS1:0}: વોલ્ટેજ રેફરન્સ (AREF, AVCC, ઇન્ટરનલ) પસંદ કરે
\item
  \textbf{ADLAR}: ADC રજિસ્ટર્સમાં રિઝલ્ટ એલાઇનમેન્ટ
\item
  \textbf{MUX3:0}: ઇનપુટ ચેનલ (ADC0-ADC7) પસંદ કરે
\end{itemize}

\end{solutionbox}
\begin{mnemonicbox}
``રેફરન્સ, એલાઇનમેન્ટ, મલ્ટિપ્લેક્સર''

\end{mnemonicbox}
\subsection*{પ્રશ્ન 4(ક OR) [7
ગુણ]}\label{uxaaauxab0uxab6uxaa8-4uxa95-or-7-uxa97uxaa3}

\textbf{AVRની Two Wire serial Interface (TWI)ની ચર્ચા કરો.}

\begin{solutionbox}

ટુ વાયર ઇન્ટરફેસ (TWI) એ પેરિફેરલ ડિવાઇસ સાથે કમ્યુનિકેશન માટે AVRનો I^{2}C
પ્રોટોકોલનો અમલ છે.

\textbf{ડાયાગ્રામ:}

\begin{center}
\textbf{Mermaid Diagram (Code)}
\begin{verbatim}
{Shaded}
{Highlighting}[]
graph LR
    A[માસ્ટર AVR] {-{-} SDA {-}{-}{-} B[સ્લેવ 1]}
    A {-{-} SCL {-}{-}{-} B}
    B {-{-} SDA {-}{-}{-} C[સ્લેવ 2]}
    B {-{-} SCL {-}{-}{-} C}
{Highlighting}
{Shaded}
\end{verbatim}
\end{center}


{\def\LTcaptype{none} % do not increment counter
\vspace{-5pt}
\captionof{table}{TWI લાક્ષણિકતાઓ}
\vspace{-10pt}
\begin{longtable}[]{@{}ll@{}}
\toprule\noalign{}
ફીચર & વર્ણન \\
\midrule\noalign{}
\endhead
\bottomrule\noalign{}
\endlastfoot
પિન્સ & SCL (સીરિયલ ક્લોક) અને SDA (સીરિયલ ડેટા) \\
સ્પીડ & સ્ટાન્ડર્ડ (100kHz), ફાસ્ટ (400kHz) \\
એડ્રેસિંગ & 7-બિટ અથવા 10-બિટ ડિવાઇસ એડ્રેસિંગ \\
ઓપરેશન & માસ્ટર અથવા સ્લેવ મોડ \\
બસ સ્ટ્રક્ચર & મલ્ટી-માસ્ટર, મલ્ટી-સ્લેવ \\
\end{longtable}
}

\begin{itemize}
\tightlist
\item
  \textbf{બાયડાયરેક્શનલ}: બંને ડિવાઇસ ટ્રાન્સમિટ અને રિસીવ કરી શકે
\item
  \textbf{રજિસ્ટર્સ}: TWBR, TWCR, TWSR, TWDR, TWAR
\item
  \textbf{ACK/NACK}: વિશ્વસનીય ટ્રાન્સફર માટે એક્નોલેજમેન્ટ
\item
  \textbf{સ્ટાર્ટ/સ્ટોપ}: ટ્રાન્સમિશન શરૂ/સમાપ્ત કરવા માટે ખાસ કન્ડિશન્સ
\item
  \textbf{સામાન્ય ઉપયોગ}: EEPROM, RTC, સેન્સર્સ, ડિસ્પ્લે
\end{itemize}

\end{solutionbox}
\begin{mnemonicbox}
``સીરિયલ ક્લોક અને ડેટા ટ્રાન્સફર''

\end{mnemonicbox}
\subsection*{પ્રશ્ન 5(અ) [3
ગુણ]}\label{uxaaauxab0uxab6uxaa8-5uxa85-3-uxa97uxaa3}

\textbf{L293D મોટર ડ્રાઇવરનો ઉપયોગ કરી DC મોટરને ATmega32 સાથે ઇન્ટરફેસ કરવા
માટે સર્કિટ ડાયાગ્રામ દોરો.}

\begin{solutionbox}

L293D માઇક્રોકન્ટ્રોલર્સ સાથે DC મોટર કંટ્રોલ કરવા માટે બાયડાયરેક્શનલ ડ્રાઇવ કરંટ
પ્રદાન કરે છે.

\textbf{સર્કિટ ડાયાગ્રામ:}

\begin{verbatim}
         ATmega32               L293D                DC Motor
        +{-{-}{-}{-}{-}{-}{-}{-}+           +{-}{-}{-}{-}{-}{-}{-}{-}+              +{-}{-}{-}{-}{-}+}
        |        |           |        |              |     |
        |     PD0|{-{-}{-}{-}{-}{-}{-}{-}{-}{-}|IN1     |              |     |}
        |     PD1|{-{-}{-}{-}{-}{-}{-}{-}{-}{-}|IN2     |              |     |}
        |        |           |        |              |     |
        |        |           |OUT1 {{-}{-}|{-}{-}{-}{-}{-}{-}{-}{-}{-}{-}{-}{-}{-}{-}|+    |}
        |        |           |OUT2 {{-}{-}|{-}{-}{-}{-}{-}{-}{-}{-}{-}{-}{-}{-}{-}{-}|−    |}
        |        |           |        |              |     |
        +{-{-}{-}{-}{-}{-}{-}{-}+           +{-}{-}{-}{-}{-}{-}{-}{-}+              +{-}{-}{-}{-}{-}+}
                                 |
                                 | +5V
                              +{-{-}+{-}{-}+}
                              |     |
                              |Power|
                              |     |
                              +{-{-}{-}{-}{-}+}
\end{verbatim}

\begin{itemize}
\tightlist
\item
  \textbf{કંટ્રોલ પિન્સ}: PD0, PD1 મોટર દિશા નિયંત્રિત કરે છે
\item
  \textbf{ડ્રાઇવર પાવર}: લોજિક અને મોટર માટે અલગ
\item
  \textbf{H-બ્રિજ}: ફોરવર્ડ/રિવર્સ ઓપરેશન સક્ષમ કરે છે
\item
  \textbf{એનેબલ પિન}: PWM સ્પીડ કંટ્રોલ માટે વાપરી શકાય
\end{itemize}

\end{solutionbox}
\begin{mnemonicbox}
``બ્રિજ દ્વારા દિશા નિયંત્રણ''

\end{mnemonicbox}
\subsection*{પ્રશ્ન 5(બ) [4
ગુણ]}\label{uxaaauxab0uxab6uxaa8-5uxaac-4-uxa97uxaa3}

\textbf{ATmega32 માં ઓન ચિપ ADCની લાક્ષણિકતા લખો.}

\begin{solutionbox}

ATmega32 એનાલોગ સિગ્નલ્સ માપવા માટે વર્સેટાઇલ એનાલોગ-ટુ-ડિજિટલ કન્વર્ટર ધરાવે છે.


{\def\LTcaptype{none} % do not increment counter
\vspace{-5pt}
\captionof{table}{ATmega32 ADC ફીચર્સ}
\vspace{-10pt}
\begin{longtable}[]{@{}ll@{}}
\toprule\noalign{}
ફીચર & સ્પેસિફિકેશન \\
\midrule\noalign{}
\endhead
\bottomrule\noalign{}
\endlastfoot
રેઝોલ્યુશન & 10-બિટ \\
ચેનલ્સ & 8 સિંગલ-એન્ડેડ ઇનપુટ્સ \\
કન્વર્ઝન ટાઇમ & 65-260 μs \\
રેફરન્સ વોલ્ટેજ & AREF, AVCC, અથવા 2.56V ઇન્ટરનલ \\
એક્યુરસી & \pm2 LSB \\
કન્વર્ઝન મોડ્સ & સિંગલ અને ફ્રી રનિંગ \\
ઇનપુટ રેન્જ & 0V થી VREF \\
\end{longtable}
}

\begin{itemize}
\tightlist
\item
  \textbf{સક્સેસિવ એપ્રોક્સિમેશન}: કન્વર્ઝન ટેકનિક
\item
  \textbf{મલ્ટિપ્લેક્સર}: 8 ઇનપુટ ચેનલ્સ વચ્ચે પસંદ કરે છે
\item
  \textbf{ઇન્ટરપ્ટ}: પૂર્ણ થયા પર વૈકલ્પિક ઇન્ટરપ્ટ
\item
  \textbf{સેમ્પલિંગ રેટ}: મહત્તમ રેઝોલ્યુશન પર 15 KSPS સુધી
\end{itemize}

\end{solutionbox}
\begin{mnemonicbox}
``મલ્ટિપલ ચેનલ્સ, ટેન-બિટ રેઝોલ્યુશન''

\end{mnemonicbox}
\subsection*{પ્રશ્ન 5(ક) [7
ગુણ]}\label{uxaaauxab0uxab6uxaa8-5uxa95-7-uxa97uxaa3}

\textbf{સ્માર્ટ ઇરીગેશન સિસ્ટમ સમજાવો.}

\begin{solutionbox}

સ્માર્ટ ઇરીગેશન સિસ્ટમ માઇક્રોકન્ટ્રોલર ટેકનોલોજીનો ઉપયોગ કરીને પર્યાવરણીય
પરિસ્થિતિઓના આધારે વોટરિંગને ઓટોમેટ કરે છે.

\textbf{ડાયાગ્રામ:}

\begin{center}
\textbf{Mermaid Diagram (Code)}
\begin{verbatim}
{Shaded}
{Highlighting}[]
graph TD
    A[ATmega32] {-{-}{} B[સોઇલ મોઇસ્ચર સેન્સર]}
    A {-{-}{} C[તાપમાન સેન્સર]}
    A {-{-}{} D[ભેજ સેન્સર]}
    A {-{-}{} E[વોટર પમ્પ કંટ્રોલ]}
    A {-{-}{} F[વાલ્વ કંટ્રોલ]}
    A {-{-}{} G[LCD ડિસ્પ્લે]}
    H[RTC મોડ્યુલ] {-{-}{} A}
{Highlighting}
{Shaded}
\end{verbatim}
\end{center}


{\def\LTcaptype{none} % do not increment counter
\vspace{-5pt}
\captionof{table}{સિસ્ટમ કોમ્પોનન્ટ્સ}
\vspace{-10pt}
\begin{longtable}[]{@{}ll@{}}
\toprule\noalign{}
કોમ્પોનન્ટ & ફંક્શન \\
\midrule\noalign{}
\endhead
\bottomrule\noalign{}
\endlastfoot
સોઇલ મોઇસ્ચર સેન્સર & માટીમાં પાણીની માત્રા માપે છે \\
તાપમાન/ભેજ & પર્યાવરણીય પરિસ્થિતિનું મોનિટરિંગ કરે છે \\
વોટર પમ્પ & જરૂર પડે ત્યારે પાણી આપે છે \\
વાલ્વ્સ & વિવિધ ઝોન્સમાં પાણી ફ્લોને નિયંત્રિત કરે છે \\
LCD ડિસ્પ્લે & સિસ્ટમ સ્ટેટસ બતાવે છે \\
RTC મોડ્યુલ & શેડ્યૂલ્ડ ઇરીગેશન માટે સમય ટ્રેક કરે છે \\
\end{longtable}
}

\begin{itemize}
\tightlist
\item
  \textbf{એડેપ્ટિવ કંટ્રોલ}: પરિસ્થિતિઓના આધારે વોટરિંગ એડજસ્ટ કરે છે
\item
  \textbf{વોટર કન્ઝર્વેશન}: માત્ર જરૂરી પ્રમાણમાં પાણીનો ઉપયોગ કરે છે
\item
  \textbf{રિમોટ મોનિટરિંગ}: વૈકલ્પિક WiFi/GSM કનેક્ટિવિટી
\item
  \textbf{ડેટા લોગિંગ}: ભેજના સ્તર અને વોટરિંગ ઇવેન્ટ્સની નોંધ રાખે છે
\item
  \textbf{બેટરી બેકઅપ}: પાવર આઉટેજ દરમિયાન ઓપરેશન સુનિશ્ચિત કરે છે
\end{itemize}

\end{solutionbox}
\begin{mnemonicbox}
``ભેજ સેન્સ કરો, પાણી ઓટોમેટિક કંટ્રોલ કરો''

\end{mnemonicbox}
\subsection*{પ્રશ્ન 5(અ OR) [3
ગુણ]}\label{uxaaauxab0uxab6uxaa8-5uxa85-or-3-uxa97uxaa3}

\textbf{L293D મોટર ડ્રાઇવર IC નો પિન ડાયાગ્રામ દોરો અને સમજાવો.}

\begin{solutionbox}

L293D એ મોટર્સ અને અન્ય ઇન્ડક્ટિવ લોડ્સ કંટ્રોલ કરવા માટે વપરાતી ક્વાડ્રુપલ હાફ-H
ડ્રાઇવર IC છે.

\textbf{ડાયાગ્રામ:}

\begin{verbatim}
        +{-{-}{-}{-}{-}{-}+}
        | 1 16 | 
    EN1{-|      |{-}VCC1}
    IN1{-|      |{-}IN4}
   OUT1{-|      |{-}OUT4}
    GND{-| L293D|{-}GND}
    GND{-|      |{-}GND}
   OUT2{-|      |{-}OUT3}
    IN2{-|      |{-}IN3}
   VCC2{-|      |{-}EN2}
        +{-{-}{-}{-}{-}{-}+}
\end{verbatim}

\begin{itemize}
\tightlist
\item
  \textbf{VCC1 (પિન 16)}: લોજિક સપ્લાય વોલ્ટેજ (5V)
\item
  \textbf{VCC2 (પિન 8)}: મોટર સપ્લાય વોલ્ટેજ (4.5V-36V)
\item
  \textbf{EN1/EN2}: એનેબલ ઇનપુટ્સ (સ્પીડ કંટ્રોલ માટે PWM થઈ શકે)
\item
  \textbf{IN1-IN4}: દિશા નિયંત્રિત કરવા માટે લોજિક ઇનપુટ્સ
\item
  \textbf{OUT1-OUT4}: મોટર્સ કનેક્ટ કરવા માટે આઉટપુટ્સ
\item
  \textbf{GND}: ગ્રાઉન્ડ કનેક્શન્સ
\end{itemize}

\end{solutionbox}
\begin{mnemonicbox}
``એનેબલ, ઇનપુટ, આઉટપુટ, પાવર''

\end{mnemonicbox}
\subsection*{પ્રશ્ન 5(બ OR) [4
ગુણ]}\label{uxaaauxab0uxab6uxaa8-5uxaac-or-4-uxa97uxaa3}

\textbf{AVR માં ADC સાથે સંકળાયેલ રજીસ્ટરોની યાદી બનાવો.}

\begin{solutionbox}

AVRની ADC સિસ્ટમ તેના ઓપરેશન કંટ્રોલ કરવા અને પરિણામો સ્ટોર કરવા માટે અનેક
રજિસ્ટર્સનો ઉપયોગ કરે છે.


{\def\LTcaptype{none} % do not increment counter
\vspace{-5pt}
\captionof{table}{ADC રજિસ્ટર્સ}
\vspace{-10pt}
\begin{longtable}[]{@{}lll@{}}
\toprule\noalign{}
રજિસ્ટર & ફંક્શન & વર્ણન \\
\midrule\noalign{}
\endhead
\bottomrule\noalign{}
\endlastfoot
ADMUX & મલ્ટિપ્લેક્સર & ચેનલ સિલેક્શન અને રેફરન્સ ઓપ્શન્સ \\
ADCSRA & કંટ્રોલ \& સ્ટેટસ & કંટ્રોલ બિટ્સ અને ફ્લેગ્સ \\
ADCH & ડેટા હાઇ & કન્વર્ઝન રિઝલ્ટનો હાઇ બાઇટ \\
ADCL & ડેટા લો & કન્વર્ઝન રિઝલ્ટનો લો બાઇટ \\
SFIOR & સ્પેશિયલ ફંક્શન & ADC ટ્રિગર સોર્સ સિલેક્શન \\
\end{longtable}
}

\begin{itemize}
\tightlist
\item
  \textbf{ADMUX}: ચેનલ અને રેફરન્સ સિલેક્શન
\item
  \textbf{ADCSRA}: ADC એનેબલ, કન્વર્ઝન સ્ટાર્ટ, પ્રીસ્કેલર
\item
  \textbf{ADCH/ADCL}: રિઝલ્ટ રજિસ્ટર્સ (10-બિટ વેલ્યુ)
\item
  \textbf{SFIOR}: ઓટો-ટ્રિગર સોર્સ (ટાઇમર, એક્સટર્નલ)
\end{itemize}

\end{solutionbox}
\begin{mnemonicbox}
``મલ્ટિપ્લેક્સર કંટ્રોલ કરે અને રિઝલ્ટ મેળવે''

\end{mnemonicbox}
\subsection*{પ્રશ્ન 5(ક OR) [7
ગુણ]}\label{uxaaauxab0uxab6uxaa8-5uxa95-or-7-uxa97uxaa3}

\textbf{IoT આધારિત હોમ ઓટોમેશન સિસ્ટમ સમજાવો.}

\begin{solutionbox}

IoT હોમ ઓટોમેશન ઘરના ઉપકરણોને રિમોટ મોનિટરિંગ અને કંટ્રોલ માટે ઇન્ટરનેટ સાથે જોડે
છે.

\textbf{ડાયાગ્રામ:}

\begin{center}
\textbf{Mermaid Diagram (Code)}
\begin{verbatim}
{Shaded}
{Highlighting}[]
graph TD
    A[ઇન્ટરનેટ] {-{-}{-} B[WiFi ગેટવે]}
    B {-{-}{-} C[AVR કંટ્રોલર]}
    C {-{-}{-} D[લાઇટ કંટ્રોલ]}
    C {-{-}{-} E[પંખા કંટ્રોલ]}
    C {-{-}{-} F[દરવાજા લોક]}
    C {-{-}{-} G[તાપમાન સેન્સર્સ]}
    C {-{-}{-} H[મોશન સેન્સર્સ]}
    B {-{-}{-} I[મોબાઇલ એપ]}
    B {-{-}{-} J[ક્લાઉડ સર્વિસીસ]}
{Highlighting}
{Shaded}
\end{verbatim}
\end{center}


{\def\LTcaptype{none} % do not increment counter
\vspace{-5pt}
\captionof{table}{સિસ્ટમ કોમ્પોનન્ટ્સ}
\vspace{-10pt}
\begin{longtable}[]{@{}ll@{}}
\toprule\noalign{}
કોમ્પોનન્ટ & ફંક્શન \\
\midrule\noalign{}
\endhead
\bottomrule\noalign{}
\endlastfoot
કંટ્રોલર & સેન્સર ડેટા અને કમાન્ડ્સ પ્રોસેસ કરે છે \\
સેન્સર્સ & પર્યાવરણીય પરિસ્થિતિઓનું મોનિટરિંગ કરે છે \\
એક્ચ્યુએટર્સ & ઉપકરણો અને સિસ્ટમ્સ કંટ્રોલ કરે છે \\
કમ્યુનિકેશન & WiFi/ઇથરનેટ/બ્લુટુથ કનેક્ટિવિટી \\
ગેટવે & લોકલ નેટવર્કને ઇન્ટરનેટ સાથે જોડે છે \\
મોબાઇલ એપ & રિમોટ કંટ્રોલ માટે યુઝર ઇન્ટરફેસ \\
\end{longtable}
}

\begin{itemize}
\tightlist
\item
  \textbf{રિમોટ એક્સેસ}: ગમે ત્યાંથી ઘર કંટ્રોલ કરો
\item
  \textbf{શેડ્યુલિંગ}: સમય આધારિત ડિવાઇસ ઓપરેશન ઓટોમેટ કરો
\item
  \textbf{વોઇસ કંટ્રોલ}: ડિજિટલ આસિસ્ટન્ટ સાથે એકીકરણ
\item
  \textbf{એનર્જી મોનિટરિંગ}: પાવર કન્ઝમ્પશન ટ્રેક કરો
\item
  \textbf{સિક્યુરિટી}: અસામાન્ય પ્રવૃત્તિઓ માટે એલર્ટ
\item
  \textbf{સીન સેટિંગ}: અનેક ડિવાઇસનું વન-ટચ કંટ્રોલ
\end{itemize}

\end{solutionbox}
\begin{mnemonicbox}
``કનેક્ટ, કંટ્રોલ, ઓટોમેટ, મોનિટર''

\end{mnemonicbox}

\end{document}
