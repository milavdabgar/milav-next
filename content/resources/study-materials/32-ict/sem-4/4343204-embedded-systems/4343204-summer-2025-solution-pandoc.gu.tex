\documentclass[10pt,a4paper]{article}

% content/resources/templates/preamble.tex
\usepackage[margin=0.6in]{geometry}
\author{Milav Dabgar}
\usepackage{amsmath,amssymb,amsthm}
\usepackage{booktabs}
\usepackage{multirow}
\usepackage{xcolor}
\usepackage{tcolorbox}
\tcbuselibrary{breakable,skins}
\usepackage[colorlinks=true,linkcolor=blue]{hyperref}
\usepackage{titlesec}
\usepackage{enumitem}
\usepackage{tikz}
\usepackage{pgfplots}
\usepackage{circuitikz}
\usepackage[version=4]{mhchem}
\usepackage{longtable}
\usepackage{array}
\usepackage{float}
\usepackage{caption}
\usepackage{listings}

\lstset{
  basicstyle=\small\ttfamily,
  breaklines=true,
  breakatwhitespace=false,
  postbreak=\mbox{\textcolor{red}{$\hookrightarrow$}\space},
  float=false,
  numbers=left,
  numberstyle=\tiny\color{gray},
  numbersep=10pt,
  xleftmargin=2em,
  keywordstyle=\color{blue},
  commentstyle=\color{green!60!black},
  stringstyle=\color{purple},
  backgroundcolor=\color{gray!5},
  showstringspaces=false,
  tabsize=2,
  captionpos=b,
  keepspaces=true,
  columns=flexible
}

\pgfplotsset{compat=1.18}
\usetikzlibrary{shapes,arrows,positioning,calc,patterns,decorations.pathmorphing,decorations.markings,arrows.meta}

% Color scheme
\definecolor{headcolor}{RGB}{0,102,204}
\definecolor{keycolor}{RGB}{220,20,60}
\definecolor{solutioncolor}{RGB}{34,139,34}
\definecolor{mnemoniccolor}{RGB}{148,0,211}
\definecolor{codecolor}{RGB}{0,0,100}

% Spacing
\setlength{\parskip}{3pt}
\setlist[itemize]{nosep}
\setlist[enumerate]{nosep}

% Title formatting
\titleformat{\section}{\Large\bfseries\color{headcolor}}{\thesection}{1em}{}
\titleformat{\subsection}{\large\bfseries\color{headcolor}}{\thesubsection}{1em}{}

% Pandoc tightlist compatibility
\providecommand{\tightlist}{%
  \setlength{\itemsep}{0pt}\setlength{\parskip}{0pt}}

% Pandoc longtable compatibility
\newcounter{none}
\def\thenone{}


% content/resources/templates/gujarati-boxes.tex
\usepackage{fontspec}
\usepackage{polyglossia}

% Set Gujarati as main language (document is primarily in Gujarati)
% Note: gloss-gujarati.ldf doesn't exist in polyglossia, but it will use hyphenation patterns
\setdefaultlanguage{gujarati}
\setotherlanguage{english}

% Configure Gujarati font properly
% Use Language=Default to prevent polyglossia from trying to add language-specific features
% that don't exist for Gujarati, which causes "empty feature" warnings
\newfontfamily\gujaratifont[Script=Gujarati,AutoFakeBold=2.5,AutoFakeSlant=0.3]{Noto Sans Gujarati}
\setmainfont[Script=Gujarati,AutoFakeBold=2.5,AutoFakeSlant=0.3]{Noto Sans Gujarati}
% Use Noto Sans Gujarati for monospace to support Gujarati in text
\setmonofont[Scale=0.9]{Noto Sans Gujarati}

% Configure English to use the same font
\newfontfamily\englishfont[Script=Gujarati,AutoFakeBold=2.5,AutoFakeSlant=0.3]{Noto Sans Gujarati}

% Translations for polyglossia
\gappto\captionsgujarati{
  \renewcommand{\tablename}{કોષ્ટક}
  \renewcommand{\figurename}{આકૃતિ}
}

% Helper for TikZ nodes to ensure Gujarati font
\newcommand{\gu}[1]{{\gujaratifont #1}}

% Custom environments
\newtcolorbox{solutionbox}{
    breakable,
    enhanced,
    colback=solutioncolor!5!white,
    colframe=solutioncolor!75!black,
    fonttitle=\bfseries,
    title=જવાબ
}

\newtcolorbox{solutionboxnobreak}{
 colback=solutioncolor!5!white,
 colframe=solutioncolor!75!black,
 fonttitle=\bfseries,
 title=જવાબ
}

\newtcolorbox{keyformula}{
 breakable,
 enhanced,
 colback=keycolor!5!white,
 colframe=keycolor!75!black,
 fonttitle=\bfseries,
 title=રાસાયણિક સમીકરણ/સૂત્ર
}

\newtcolorbox{mnemonicbox}{
 breakable,
 enhanced,
 colback=mnemoniccolor!5!white,
 colframe=mnemoniccolor!75!black,
 fonttitle=\bfseries,
 title=મેમરી ટ્રીક
}


\begin{document}

\begin{center}
{\Huge\bfseries\color{headcolor} Subject Name (Gujarati)}\\[5pt]
{\LARGE 4343204 -- Summer 2025}\\[3pt]
{\large Semester 1 Study Material}\\[3pt]
{\normalsize\textit{Detailed Solutions and Explanations}}
\end{center}

\vspace{10pt}

\subsection*{પ્રશ્ન 1(અ) [3
ગુણ]}\label{uxaaauxab0uxab6uxaa8-1uxa85-3-uxa97uxaa3}

\textbf{રીઅલ ટાઇમ ઓપરેટિંગ સિસ્ટમની લાક્ષણિકતાઓની ચર્ચા કરો.}

\begin{solutionbox}


{\def\LTcaptype{none} % do not increment counter
\vspace{-5pt}
\captionof{table}{RTOS લાક્ષણિકતાઓ}
\vspace{-10pt}
\begin{longtable}[]{@{}ll@{}}
\toprule\noalign{}
લાક્ષણિકતા & વર્ણન \\
\midrule\noalign{}
\endhead
\bottomrule\noalign{}
\endlastfoot
\textbf{નિર્ધારિત વર્તન} & અનુમાનિત પ્રતિસાદ સમય \\
\textbf{સમય મર્યાદા} & કઠિન અને નરમ ડેડલાઇન \\
\textbf{પ્રાથમિકતા શેડ્યુલિંગ} & પ્રાથમિકતા દ્વારા કાર્ય અમલ \\
\textbf{સંસાધન વ્યવસ્થાપન} & કાર્યક્ષમ મેમરી અને CPU ઉપયોગ \\
\end{longtable}
}

\begin{itemize}
\tightlist
\item
  \textbf{નિર્ધારિત વર્તન}: સિસ્ટમ ગેરંટીવાળા સમય મર્યાદામાં પ્રતિસાદ આપે છે
\item
  \textbf{મલ્ટિટાસ્કિંગ સપોર્ટ}: બહુવિધ કાર્યો પ્રાથમિકતા સાથે સમાંતર ચાલે છે
\item
  \textbf{ઇન્ટરપ્ટ હેન્ડલિંગ}: બાહ્ય ઘટનાઓને ઝડપી પ્રતિસાદ
\end{itemize}

\textbf{સ્મરણ સહાયક:} ``RTOS કાર્યો યોગ્ય રીતે વિતરિત કરે છે''

\end{solutionbox}
\begin{center}\rule{0.5\linewidth}{0.5pt}\end{center}

\subsection*{પ્રશ્ન 1(બ) [4
ગુણ]}\label{uxaaauxab0uxab6uxaa8-1uxaac-4-uxa97uxaa3}

\textbf{AVR I/O પોર્ટ રજિસ્ટરનું વર્ણન કરો.}

\begin{solutionbox}


{\def\LTcaptype{none} % do not increment counter
\vspace{-5pt}
\captionof{table}{AVR I/O પોર્ટ રજિસ્ટર}
\vspace{-10pt}
\begin{longtable}[]{@{}lll@{}}
\toprule\noalign{}
રજિસ્ટર & કાર્ય & પ્રવેશ \\
\midrule\noalign{}
\endhead
\bottomrule\noalign{}
\endlastfoot
\textbf{DDRx} & ડેટા દિશા રજિસ્ટર & વાંચો/લખો \\
\textbf{PORTx} & પોર્ટ આઉટપુટ રજિસ્ટર & વાંચો/લખો \\
\textbf{PINx} & પોર્ટ ઇનપુટ રજિસ્ટર & ફક્ત વાંચો \\
\end{longtable}
}

\begin{itemize}
\tightlist
\item
  \textbf{DDRx રજિસ્ટર}: પિન દિશા નિયંત્રિત કરે છે (0=ઇનપુટ, 1=આઉટપુટ)
\item
  \textbf{PORTx રજિસ્ટર}: આઉટપુટ મૂલ્યો સેટ કરે છે અથવા pull-up રેઝિસ્ટર સક્રિય કરે
  છે
\item
  \textbf{PINx રજિસ્ટર}: ઇનપુટ ઓપરેશન માટે વર્તમાન પિન સ્થિતિ વાંચે છે
\end{itemize}

\textbf{સ્મરણ સહાયક:} ``દિશા, પોર્ટ, પિન - DPP''

\end{solutionbox}
\begin{center}\rule{0.5\linewidth}{0.5pt}\end{center}

\subsection*{પ્રશ્ન 1(ક) [7
ગુણ]}\label{uxaaauxab0uxab6uxaa8-1uxa95-7-uxa97uxaa3}

\textbf{વિવિધ AVR માઇક્રોકન્ટ્રોલરની સરખામણી કરો અને એમ્બેડેડ સિસ્ટમ માટે
માઇક્રોકન્ટ્રોલર પસંદ કરવા માટે કયા પરિબળો ધ્યાનમાં લેવા જોઈએ?}

\begin{solutionbox}


{\def\LTcaptype{none} % do not increment counter
\vspace{-5pt}
\captionof{table}{AVR માઇક્રોકન્ટ્રોલર સરખામણી}
\vspace{-10pt}
\begin{longtable}[]{@{}llll@{}}
\toprule\noalign{}
લક્ષણ & ATmega8 & ATmega32 & ATmega128 \\
\midrule\noalign{}
\endhead
\bottomrule\noalign{}
\endlastfoot
\textbf{Flash મેમરી} & 8KB & 32KB & 128KB \\
\textbf{SRAM} & 1KB & 2KB & 4KB \\
\textbf{EEPROM} & 512B & 1KB & 4KB \\
\textbf{I/O પિન} & 23 & 32 & 53 \\
\textbf{ટાઇમર} & 3 & 3 & 4 \\
\end{longtable}
}

\textbf{પસંદગીના પરિબળો:}

\begin{itemize}
\tightlist
\item
  \textbf{પ્રોસેસિંગ સ્પીડ}: એપ્લિકેશન માટે ક્લોક ફ્રીક્વન્સી જરૂરિયાત
\item
  \textbf{મેમરી જરૂરિયાત}: પ્રોગ્રામ અને ડેટા સ્ટોરેજની જરૂર
\item
  \textbf{I/O જરૂરિયાત}: ઇન્ટરફેસિંગ માટે જરૂરી પિનોની સંખ્યા
\item
  \textbf{પાવર વપરાશ}: પોર્ટેબલ ઉપકરણો માટે બેટરી જીવનની વિચારણા
\item
  \textbf{કિંમત પરિબળ}: બજેટ મર્યાદા અને વોલ્યુમ જરૂરિયાત
\item
  \textbf{ડેવલપમેન્ટ ટૂલ્સ}: કમ્પાઇલર અને ડીબગરની ઉપલબ્ધતા
\end{itemize}

\textbf{સ્મરણ સહાયક:} ``સ્પીડ, મેમરી, I/O, પાવર, કિંમત, ટૂલ્સ - SMIPCT''

\end{solutionbox}
\begin{center}\rule{0.5\linewidth}{0.5pt}\end{center}

\subsection*{પ્રશ્ન 1(ક અથવા) [7
ગુણ]}\label{uxaaauxab0uxab6uxaa8-1uxa95-uxa85uxaa5uxab5-7-uxa97uxaa3}

\textbf{એમ્બેડેડ સિસ્ટમનો સામાન્ય બ્લોક ડાયાગ્રામ દોરો અને સમજાવો.}

\begin{solutionbox}

\textbf{આકૃતિ:}

\begin{verbatim}
    +{-{-}{-}{-}{-}{-}{-}{-}{-}{-}{-}{-}{-}{-}{-}{-}{-}{-}+    +{-}{-}{-}{-}{-}{-}{-}{-}{-}{-}{-}{-}{-}{-}{-}{-}{-}{-}+    +{-}{-}{-}{-}{-}{-}{-}{-}{-}{-}{-}{-}{-}{-}{-}{-}{-}{-}+}
    |   Input Devices  |    |  Microcontroller |    |  Output Devices  |
    |                  |{-{-}{-}|                  |{-}{-}{-}|                  |}
    | • Sensors        |    | • CPU            |    | • Actuators      |
    | • Switches       |    | • Memory         |    | • Display        |
    | • Keypad         |    | • I/O Ports      |    | • LEDs           |
    +{-{-}{-}{-}{-}{-}{-}{-}{-}{-}{-}{-}{-}{-}{-}{-}{-}{-}+    +{-}{-}{-}{-}{-}{-}{-}{-}{-}{-}{-}{-}{-}{-}{-}{-}{-}{-}+    +{-}{-}{-}{-}{-}{-}{-}{-}{-}{-}{-}{-}{-}{-}{-}{-}{-}{-}+}
                                      |
                                      v
                            +{-{-}{-}{-}{-}{-}{-}{-}{-}{-}{-}{-}{-}{-}{-}{-}{-}{-}+}
                            |  Power Supply    |
                            |                  |
                            | • Voltage Reg.   |
                            | • Battery        |
                            +{-{-}{-}{-}{-}{-}{-}{-}{-}{-}{-}{-}{-}{-}{-}{-}{-}{-}+}
\end{verbatim}

\textbf{ઘટકો:}

\begin{itemize}
\tightlist
\item
  \textbf{ઇનપુટ વિભાગ}: સેન્સર અને સ્વિચ સિસ્ટમને ડેટા પ્રદાન કરે છે
\item
  \textbf{પ્રોસેસિંગ યુનિટ}: માઇક્રોકન્ટ્રોલર પ્રોગ્રામ ચલાવે છે અને ઓપરેશન કંટ્રોલ કરે
  છે
\item
  \textbf{આઉટપુટ વિભાગ}: પરિણામો દર્શાવે છે અને બાહ્ય ઉપકરણો કંટ્રોલ કરે છે
\item
  \textbf{પાવર સપ્લાય}: બધા ઘટકોને નિયંત્રિત પાવર પ્રદાન કરે છે
\item
  \textbf{મેમરી}: પ્રોગ્રામ કોડ અને ડેટાને કાયમી ધોરણે સંગ્રહિત કરે છે
\item
  \textbf{કમ્યુનિકેશન}: સીરીયલ/વાયરલેસ દ્વારા બાહ્ય સિસ્ટમ સાથે ઇન્ટરફેસ
\end{itemize}

\textbf{સ્મરણ સહાયક:} ``ઇનપુટ, પ્રોસેસ, આઉટપુટ, પાવર, મેમરી, કમ્યુનિકેશન -
IPOPMC''

\end{solutionbox}
\begin{center}\rule{0.5\linewidth}{0.5pt}\end{center}

\subsection*{પ્રશ્ન 2(અ) [3
ગુણ]}\label{uxaaauxab0uxab6uxaa8-2uxa85-3-uxa97uxaa3}

\textbf{ATMega32 ના EEPROM સાથે SRAM ની સરખામણી કરો.}

\begin{solutionbox}


{\def\LTcaptype{none} % do not increment counter
\vspace{-5pt}
\captionof{table}{SRAM વિ EEPROM સરખામણી}
\vspace{-10pt}
\begin{longtable}[]{@{}lll@{}}
\toprule\noalign{}
પેરામીટર & SRAM & EEPROM \\
\midrule\noalign{}
\endhead
\bottomrule\noalign{}
\endlastfoot
\textbf{કદ} & 2KB & 1KB \\
\textbf{અસ્થિરતા} & અસ્થિર & બિન-અસ્થિર \\
\textbf{પ્રવેશ ઝડપ} & ઝડપી & ધીમી \\
\textbf{લેખન ચક્ર} & અમર્યાદિત & 100,000 ચક્ર \\
\end{longtable}
}

\begin{itemize}
\tightlist
\item
  \textbf{ડેટા રીટેન્શન}: SRAM પાવર-ઓફ પર ડેટા ખોવાય છે, EEPROM ડેટા જાળવે છે
\item
  \textbf{ઉપયોગ હેતુ}: SRAM વેરિએબલ માટે, EEPROM કૉન્ફિગરેશન ડેટા માટે
\end{itemize}

\textbf{સ્મરણ સહાયક:} ``SRAM ઝડપી પણ ભૂલી જાય, EEPROM ટકી રહે''

\end{solutionbox}
\begin{center}\rule{0.5\linewidth}{0.5pt}\end{center}

\subsection*{પ્રશ્ન 2(બ) [4
ગુણ]}\label{uxaaauxab0uxab6uxaa8-2uxaac-4-uxa97uxaa3}

\textbf{ટાઈમર/કાઉન્ટર 0 ઑપરેશન મોડની સૂચિ બનાવો અને કોઈપણને સમજાવો.}

\begin{solutionbox}


{\def\LTcaptype{none} % do not increment counter
\vspace{-5pt}
\captionof{table}{Timer0 ઑપરેશન મોડ}
\vspace{-10pt}
\begin{longtable}[]{@{}lll@{}}
\toprule\noalign{}
મોડ & નામ & વર્ણન \\
\midrule\noalign{}
\endhead
\bottomrule\noalign{}
\endlastfoot
\textbf{0} & સામાન્ય & 0xFF સુધી ગણતરી, ઓવરફ્લો \\
\textbf{1} & PWM ફેઝ કરેક્ટ & ફેઝ કરેક્શન સાથે PWM \\
\textbf{2} & CTC & કંપેર પર ટાઇમર ક્લિયર \\
\textbf{3} & ફાસ્ટ PWM & ઉચ્ચ ફ્રીક્વન્સી PWM \\
\end{longtable}
}

\textbf{સામાન્ય મોડ સમજૂતી:}

\begin{itemize}
\tightlist
\item
  \textbf{કાઉન્ટર ઑપરેશન}: સતત 0x00 થી 0xFF સુધી ગણતરી કરે છે
\item
  \textbf{ઓવરફ્લો ફ્લેગ}: કાઉન્ટર 0x00 પર ઓવરફ્લો થાય છે ત્યારે TOV0 ફ્લેગ સેટ થાય
  છે
\item
  \textbf{ઇન્ટરપ્ટ જનરેશન}: ઓવરફ્લો કન્ડિશન પર ઇન્ટરપ્ટ જનરેટ કરી શકે છે
\end{itemize}

\textbf{સ્મરણ સહાયક:} ``સામાન્ય ગણે, PWM પલ્સ કરે, CTC ક્લિયર કરે''

\end{solutionbox}
\begin{center}\rule{0.5\linewidth}{0.5pt}\end{center}

\subsection*{પ્રશ્ન 2(ક) [7
ગુણ]}\label{uxaaauxab0uxab6uxaa8-2uxa95-7-uxa97uxaa3}

\textbf{સ્કેચ સાથે, ATmega32 ની દરેક પિનનું કાર્ય ઓળખો અને લખો.}

\begin{solutionbox}

\textbf{આકૃતિ: ATmega32 પિન કૉન્ફિગરેશન}

\begin{verbatim}
                    ATmega32
                 +{-{-}{-}{-}{-}{-}{-}{-}{-}{-}{-}{-}{-}{-}{-}{-}{-}{-}{-}{-}+}
    (XCK/T0) PB0 |1                 40| PA0 (ADC0)
        (T1) PB1 |2                 39| PA1 (ADC1)
 (INT2/AIN0) PB2 |3                 38| PA2 (ADC2)
  (OC0/AIN1) PB3 |4                 37| PA3 (ADC3)
        (SS) PB4 |5                 36| PA4 (ADC4)
      (MOSI) PB5 |6                 35| PA5 (ADC5)
      (MISO) PB6 |7                 34| PA6 (ADC6)
       (SCK) PB7 |8                 33| PA7 (ADC7)
             RST |9                 32| AREF
             VCC |10                31| GND
             GND |11                30| AVCC
           XTAL2 |12                29| PC7 (TOSC2)
           XTAL1 |13                28| PC6 (TOSC1)
       (RXD) PD0 |14                27| PC5 (TDI)
       (TXD) PD1 |15                26| PC4 (TDO)
      (INT0) PD2 |16                25| PC3 (TMS)
      (INT1) PD3 |17                24| PC2 (TCK)
      (OC1B) PD4 |18                23| PC1 (SDA)
      (OC1A) PD5 |19                22| PC0 (SCL)
      (ICP1) PD6 |20                21| PD7 (OC2)
                 +{-{-}{-}{-}{-}{-}{-}{-}{-}{-}{-}{-}{-}{-}{-}{-}{-}{-}{-}{-}+}
\end{verbatim}

\textbf{પિન કાર્યો:}

\begin{itemize}
\tightlist
\item
  \textbf{પોર્ટ A}: 8-બિટ ADC ઇનપુટ પિન (PA0-PA7)
\item
  \textbf{પોર્ટ B}: SPI કમ્યુનિકેશન અને ટાઇમર કાર્યો
\item
  \textbf{પોર્ટ C}: JTAG ઇન્ટરફેસ અને I2C કમ્યુનિકેશન
\item
  \textbf{પોર્ટ D}: UART કમ્યુનિકેશન અને બાહ્ય ઇન્ટરપ્ટ
\item
  \textbf{પાવર પિન}: VCC, GND, AVCC એનાલોગ સપ્લાય માટે
\item
  \textbf{ક્રિસ્ટલ પિન}: XTAL1, XTAL2 બાહ્ય ઓસિલેટર માટે
\end{itemize}

\textbf{સ્મરણ સહાયક:} ``એનાલોગ-A, બસ-B, કમ્યુનિકેશન-C, ડેટા-D''

\end{solutionbox}
\begin{center}\rule{0.5\linewidth}{0.5pt}\end{center}

\subsection*{પ્રશ્ન 2(અ અથવા) [3
ગુણ]}\label{uxaaauxab0uxab6uxaa8-2uxa85-uxa85uxaa5uxab5-3-uxa97uxaa3}

\textbf{ATmega32 ની ડેટા મેમરીની રચના સમજાવો.}

\begin{solutionbox}


{\def\LTcaptype{none} % do not increment counter
\vspace{-5pt}
\captionof{table}{ATmega32 મેમરી ઓર્ગેનાઈઝેશન}
\vspace{-10pt}
\begin{longtable}[]{@{}lll@{}}
\toprule\noalign{}
મેમરી પ્રકાર & એડ્રેસ રેન્જ & કદ \\
\midrule\noalign{}
\endhead
\bottomrule\noalign{}
\endlastfoot
\textbf{રજિસ્ટર} & 0x00-0x1F & 32 બાઇટ \\
\textbf{I/O રજિસ્ટર} & 0x20-0x5F & 64 બાઇટ \\
\textbf{આંતરિક SRAM} & 0x60-0x25F & 2048 બાઇટ \\
\end{longtable}
}

\begin{itemize}
\tightlist
\item
  \textbf{સામાન્ય હેતુ રજિસ્ટર}: અંકગણિત ઓપરેશન માટે R0-R31
\item
  \textbf{I/O મેમરી જગ્યા}: પેરિફેરલ માટે કંટ્રોલ રજિસ્ટર
\item
  \textbf{આંતરિક SRAM}: પ્રોગ્રામ એક્ઝિક્યુશન દરમિયાન વેરિએબલ સ્ટોરેજ
\end{itemize}

\textbf{સ્મરણ સહાયક:} ``રજિસ્ટર, I/O, SRAM - RIS''

\end{solutionbox}
\begin{center}\rule{0.5\linewidth}{0.5pt}\end{center}

\subsection*{પ્રશ્ન 2(બ અથવા) [4
ગુણ]}\label{uxaaauxab0uxab6uxaa8-2uxaac-uxa85uxaa5uxab5-4-uxa97uxaa3}

\textbf{ટાઈમર/કાઉન્ટર 0 ના TIFR અને TCCR રજિસ્ટર દોરો.}

\begin{solutionbox}

\textbf{આકૃતિ: Timer0 રજિસ્ટર}

\begin{verbatim}
TIFR (ટાઇમર ઇન્ટરપ્ટ ફ્લેગ રજિસ્ટર)
+{-{-}{-}+{-}{-}{-}+{-}{-}{-}+{-}{-}{-}+{-}{-}{-}+{-}{-}{-}+{-}{-}{-}+{-}{-}{-}+}
| {- | {-} | {-} | {-} | {-} |OCF2|TOV2|TOV0|OCF0|TOV1|OCF1A|ICF1|OCF1B|}
+{-{-}{-}+{-}{-}{-}+{-}{-}{-}+{-}{-}{-}+{-}{-}{-}+{-}{-}{-}+{-}{-}{-}+{-}{-}{-}+}
  7   6   5   4   3   2   1   0

TCCR0 (ટાઇમર/કાઉન્ટર કંટ્રોલ રજિસ્ટર 0)
+{-{-}{-}+{-}{-}{-}+{-}{-}{-}+{-}{-}{-}+{-}{-}{-}+{-}{-}{-}+{-}{-}{-}+{-}{-}{-}+}
|FOC0|WGM00|COM01|COM00|WGM01| {- |CS02|CS01|CS00|}
+{-{-}{-}+{-}{-}{-}+{-}{-}{-}+{-}{-}{-}+{-}{-}{-}+{-}{-}{-}+{-}{-}{-}+{-}{-}{-}+}
  7   6    5    4    3   2   1   0
\end{verbatim}

\textbf{બિટ કાર્યો:}

\begin{itemize}
\tightlist
\item
  \textbf{TOV0}: Timer0 ઓવરફ્લો ફ્લેગ બિટ
\item
  \textbf{OCF0}: Timer0 આઉટપુટ કંપેર મેચ ફ્લેગ
\item
  \textbf{CS02:CS00}: પ્રીસ્કેલર માટે ક્લોક સિલેક્ટ બિટ
\item
  \textbf{WGM01:WGM00}: વેવફોર્મ જનરેશન મોડ બિટ
\end{itemize}

\textbf{સ્મરણ સહાયક:} ``TIFR ફ્લેગ બતાવે, TCCR ક્લોક કંટ્રોલ કરે''

\end{solutionbox}
\begin{center}\rule{0.5\linewidth}{0.5pt}\end{center}

\subsection*{પ્રશ્ન 2(ક અથવા) [7
ગુણ]}\label{uxaaauxab0uxab6uxaa8-2uxa95-uxa85uxaa5uxab5-7-uxa97uxaa3}

\textbf{AVR માઇક્રોકન્ટ્રોલરનો સામાન્ય બ્લોક ડાયાગ્રામ દોરો અને સમજાવો.}

\begin{solutionbox}

\textbf{આકૃતિ: AVR આર્કિટેક્ચર}

\begin{verbatim}
    +{-{-}{-}{-}{-}{-}{-}{-}{-}{-}{-}{-}{-}{-}{-}{-}{-}{-}+    +{-}{-}{-}{-}{-}{-}{-}{-}{-}{-}{-}{-}{-}{-}{-}{-}{-}{-}+}
    |   Program Memory |    |   Data Memory    |
    |     (Flash)      |    |     (SRAM)       |
    +{-{-}{-}{-}{-}{-}{-}{-}{-}{-}{-}{-}{-}{-}{-}{-}{-}{-}+    +{-}{-}{-}{-}{-}{-}{-}{-}{-}{-}{-}{-}{-}{-}{-}{-}{-}{-}+}
             |                       |
             v                       v
    +{-{-}{-}{-}{-}{-}{-}{-}{-}{-}{-}{-}{-}{-}{-}{-}{-}{-}{-}{-}{-}{-}{-}{-}{-}{-}{-}{-}{-}{-}{-}{-}{-}{-}{-}{-}{-}{-}{-}{-}+}
    |              CPU Core                  |
    |  +{-{-}{-}{-}{-}{-}{-}{-}{-}{-}+  +{-}{-}{-}{-}{-}{-}{-}{-}{-}{-}+            |}
    |  |   ALU    |  | Register |            |
    |  |          |  |   File   |            |
    |  +{-{-}{-}{-}{-}{-}{-}{-}{-}{-}+  +{-}{-}{-}{-}{-}{-}{-}{-}{-}{-}+            |}
    +{-{-}{-}{-}{-}{-}{-}{-}{-}{-}{-}{-}{-}{-}{-}{-}{-}{-}{-}{-}{-}{-}{-}{-}{-}{-}{-}{-}{-}{-}{-}{-}{-}{-}{-}{-}{-}{-}{-}{-}+}
             |
             v
    +{-{-}{-}{-}{-}{-}{-}{-}{-}{-}{-}{-}{-}{-}{-}{-}{-}{-}+    +{-}{-}{-}{-}{-}{-}{-}{-}{-}{-}{-}{-}{-}{-}{-}{-}{-}{-}+}
    |   I/O Registers  |    |   Peripherals    |
    |                  |    | • Timers         |
    |                  |    | • UART           |
    |                  |    | • ADC            |
    +{-{-}{-}{-}{-}{-}{-}{-}{-}{-}{-}{-}{-}{-}{-}{-}{-}{-}+    +{-}{-}{-}{-}{-}{-}{-}{-}{-}{-}{-}{-}{-}{-}{-}{-}{-}{-}+}
\end{verbatim}

\textbf{ઘટકો:}

\begin{itemize}
\tightlist
\item
  \textbf{CPU કોર}: ઇન્સ્ટ્રક્શન એક્ઝિક્યુટ કરે છે અને સિસ્ટમ ઓપરેશન કંટ્રોલ કરે છે
\item
  \textbf{પ્રોગ્રામ મેમરી}: બિન-અસ્થિર flash માં એપ્લિકેશન કોડ સ્ટોર કરે છે
\item
  \textbf{ડેટા મેમરી}: વેરિએબલ અને સ્ટેક માટે અસ્થાયી સ્ટોરેજ
\item
  \textbf{ALU}: અંકગણિત અને તાર્કિક ઓપરેશન કરે છે
\item
  \textbf{રજિસ્ટર ફાઇલ}: 32 સામાન્ય-હેતુ વર્કિંગ રજિસ્ટર
\item
  \textbf{I/O સિસ્ટમ}: બાહ્ય હાર્ડવેર ઘટકો સાથે ઇન્ટરફેસ
\item
  \textbf{પેરિફેરલ}: બિલ્ટ-ઇન મોડ્યુલ જેમ કે ટાઇમર, UART, ADC
\end{itemize}

\textbf{સ્મરણ સહાયક:} ``CPU પ્રોગ્રામ, ડેટા, I/O, પેરિફેરલ કંટ્રોલ કરે - CPDIP''

\end{solutionbox}
\begin{center}\rule{0.5\linewidth}{0.5pt}\end{center}

\subsection*{પ્રશ્ન 3(અ) [3
ગુણ]}\label{uxaaauxab0uxab6uxaa8-3uxa85-3-uxa97uxaa3}

\textbf{10 ms વિલંબ સાથે સતત પોર્ટ B ના તમામ બિટ્સને ટૉગલ કરવા માટે AVR C
પ્રોગ્રામ લખો.}

\begin{solutionbox}

\begin{verbatim}
\#include {avr/io.h}
\#include {util/delay.h}

int main()
\{
    DDRB = 0xFF;        // પોર્ટ B ને આઉટપુટ તરીકે સેટ કરો
    
    while(1)
    \{
        PORTB = 0xFF;    // બધા બિટ હાઇ સેટ કરો
        \_delay\_ms(10);   // 10ms વિલંબ
        PORTB = 0x00;    // બધા બિટ લો સેટ કરો
        \_delay\_ms(10);   // 10ms વિલંબ
    \}
\}
\end{verbatim}

\textbf{મુખ્ય મુદ્દાઓ:}

\begin{itemize}
\tightlist
\item
  \textbf{DDRB = 0xFF}: પોર્ટ B ના બધા પિનને આઉટપુટ તરીકે કૉન્ફિગર કરે છે
\item
  \textbf{PORTB ટૉગલ}: 0xFF અને 0x00 વચ્ચે બદલાય છે
\end{itemize}

\textbf{સ્મરણ સહાયક:} ``DDR દિશા, PORT આઉટપુટ''

\end{solutionbox}
\begin{center}\rule{0.5\linewidth}{0.5pt}\end{center}

\subsection*{પ્રશ્ન 3(બ) [4
ગુણ]}\label{uxaaauxab0uxab6uxaa8-3uxaac-4-uxa97uxaa3}

\textbf{MAX232 નું કાર્ય સમજાવો.}

\begin{solutionbox}


{\def\LTcaptype{none} % do not increment counter
\vspace{-5pt}
\captionof{table}{MAX232 કાર્યો}
\vspace{-10pt}
\begin{longtable}[]{@{}ll@{}}
\toprule\noalign{}
કાર્ય & વર્ણન \\
\midrule\noalign{}
\endhead
\bottomrule\noalign{}
\endlastfoot
\textbf{લેવલ કન્વર્ઝન} & TTL થી RS232 વોલ્ટેજ લેવલ \\
\textbf{ચાર્જ પંપ} & +5V સપ્લાયથી \pm10V જનરેટ કરે છે \\
\textbf{લાઇન ડ્રાઇવર} & બે ટ્રાન્સમિટ ડ્રાઇવર \\
\textbf{લાઇન રિસીવર} & બે રિસીવ રિસીવર \\
\end{longtable}
}

\begin{itemize}
\tightlist
\item
  \textbf{વોલ્ટેજ કન્વર્ઝન}: 0-5V TTL ને \pm12V RS232 લેવલમાં કન્વર્ટ કરે છે
\item
  \textbf{સીરીયલ કમ્યુનિકેશન}: માઇક્રોકન્ટ્રોલરને PC સાથે કમ્યુનિકેટ કરવા સક્ષમ બનાવે
  છે
\item
  \textbf{ડ્યુઅલ ચેનલ}: બે-દિશાવાળી કમ્યુનિકેશનને સમાંતર સપોર્ટ કરે છે
\end{itemize}

\textbf{સ્મરણ સહાયક:} ``MAX232 માઇક્રોકન્ટ્રોલરને PC સાથે મળાવે છે''

\end{solutionbox}
\begin{center}\rule{0.5\linewidth}{0.5pt}\end{center}

\subsection*{પ્રશ્ન 3(ક) [7
ગુણ]}\label{uxaaauxab0uxab6uxaa8-3uxa95-7-uxa97uxaa3}

\textbf{કેટલાક વિલંબ સાથે સતત PORTC ના તમામ બિટ્સને ટૉગલ કરવા માટે AVR C
પ્રોગ્રામ લખો. વિલંબ જનરેટ કરવા માટે પ્રીસ્કેલર વિકલ્પ વગર અને ટાઈમર 0, મોડ 0 નો
ઉપયોગ કરવો.}

\begin{solutionbox}

\begin{verbatim}
\#include {avr/io.h}

void timer0\_delay()
\{
    TCNT0 = 0;          // કાઉન્ટર ઇનિશિયલાઇઝ કરો
    TCCR0 = 0x01;       // કોઈ પ્રીસ્કેલર નહીં, સામાન્ય મોડ
    while(!(TIFR \& (1{}TOV0))); // ઓવરફ્લો માટે રાહ જુઓ
    TIFR |= (1{}TOV0);  // ઓવરફ્લો ફ્લેગ ક્લિયર કરો
    TCCR0 = 0;          // ટાઇમર સ્ટોપ કરો
\}

int main()
\{
    DDRC = 0xFF;        // પોર્ટ C આઉટપુટ તરીકે
    
    while(1)
    \{
        PORTC = 0xFF;    // બધા બિટ હાઇ
        for(int i=0; i{}100; i++)
            timer0\_delay(); // બહુવિધ વિલંબ
            
        PORTC = 0x00;    // બધા બિટ લો
        for(int i=0; i{}100; i++)
            timer0\_delay(); // બહુવિધ વિલંબ
    \}
\}
\end{verbatim}

\textbf{મુખ્ય લક્ષણો:}

\begin{itemize}
\tightlist
\item
  \textbf{Timer0 સામાન્ય મોડ}: 0 થી 255 સુધી ગણે છે પછી ઓવરફ્લો
\item
  \textbf{કોઈ પ્રીસ્કેલર નહીં}: ટાઇમર સિસ્ટમ ક્લોક સ્પીડે ચાલે છે
\item
  \textbf{ઓવરફ્લો ડિટેક્શન}: TOV0 ફ્લેગ ટાઇમર ઓવરફ્લો દર્શાવે છે
\item
  \textbf{વિલંબ જનરેશન}: બહુવિધ ટાઇમર ચક્ર દૃશ્યમાન વિલંબ બનાવે છે
\end{itemize}

\textbf{સ્મરણ સહાયક:} ``ટાઇમર ગણે, ઓવરફ્લો ફ્લેગ, વિલંબ જનરેટ કરે''

\end{solutionbox}
\begin{center}\rule{0.5\linewidth}{0.5pt}\end{center}

\subsection*{પ્રશ્ન 3(અ અથવા) [3
ગુણ]}\label{uxaaauxab0uxab6uxaa8-3uxa85-uxa85uxaa5uxab5-3-uxa97uxaa3}

\textbf{EEPROM ના સ્થાન 0X011F માં \#30h સ્ટોર કરવા માટે AVR C પ્રોગ્રામ લખો.}

\begin{solutionbox}

\begin{verbatim}
\#include {avr/io.h}
\#include {avr/eeprom.h}

int main()
\{
    eeprom\_write\_byte((uint8\_t*)0x011F, 0x30);
    return 0;
\}
\end{verbatim}

\textbf{વૈકલ્પિક પદ્ધતિ:}

\begin{verbatim}
\#include {avr/io.h}

int main()
\{
    while(EECR \& (1{}EEWE));    // અગાઉના લેખન માટે રાહ જુઓ
    EEAR = 0x011F;              // એડ્રેસ સેટ કરો
    EEDR = 0x30;                // ડેટા સેટ કરો
    EECR |= (1{}EEMWE);         // માસ્ટર લેખન સક્ષમ
    EECR |= (1{}EEWE);          // લેખન સક્ષમ
\}
\end{verbatim}

\textbf{સ્મરણ સહાયક:} ``એડ્રેસ, ડેટા, માસ્ટર, લેખન - ADMW''

\end{solutionbox}
\begin{center}\rule{0.5\linewidth}{0.5pt}\end{center}

\subsection*{પ્રશ્ન 3(બ અથવા) [4
ગુણ]}\label{uxaaauxab0uxab6uxaa8-3uxaac-uxa85uxaa5uxab5-4-uxa97uxaa3}

\textbf{C માં AVR પ્રોગ્રામિંગ માટે વિવિધ ડેટા પ્રકારોની ચર્ચા કરો.}

\begin{solutionbox}


{\def\LTcaptype{none} % do not increment counter
\vspace{-5pt}
\captionof{table}{AVR C ડેટા પ્રકાર}
\vspace{-10pt}
\begin{longtable}[]{@{}lll@{}}
\toprule\noalign{}
ડેટા પ્રકાર & કદ & રેન્જ \\
\midrule\noalign{}
\endhead
\bottomrule\noalign{}
\endlastfoot
\textbf{char} & 1 બાઇટ & -128 થી 127 \\
\textbf{unsigned char} & 1 બાઇટ & 0 થી 255 \\
\textbf{int} & 2 બાઇટ & -32768 થી 32767 \\
\textbf{unsigned int} & 2 બાઇટ & 0 થી 65535 \\
\textbf{long} & 4 બાઇટ & -2^{3}^{1} થી 2^{3}^{1}-1 \\
\textbf{float} & 4 બાઇટ & IEEE 754 ફોર્મેટ \\
\end{longtable}
}

\begin{itemize}
\tightlist
\item
  \textbf{મેમરી કાર્યક્ષમતા}: સૌથી નાના યોગ્ય ડેટા પ્રકારની પસંદગી કરો
\item
  \textbf{Unsigned પ્રકાર}: જ્યારે નેગેટિવ મૂલ્યોની જરૂર ન હોય ત્યારે ઉપયોગ કરો
\item
  \textbf{Integer અંકગણિત}: ફ્લોટિંગ-પોઇન્ટ ઓપરેશન કરતાં ઝડપી
\end{itemize}

\textbf{સ્મરણ સહાયક:} ``મેમરી કાર્યક્ષમતા માટે યોગ્ય કદ પસંદ કરો''

\end{solutionbox}
\begin{center}\rule{0.5\linewidth}{0.5pt}\end{center}

\subsection*{પ્રશ્ન 3(ક અથવા) [7
ગુણ]}\label{uxaaauxab0uxab6uxaa8-3uxa95-uxa85uxaa5uxab5-7-uxa97uxaa3}

\textbf{સીરીયલ ડેટા ટ્રાન્સમિશન માટે AVR C પ્રોગ્રામ્સ લખો.}

\begin{solutionbox}

\begin{verbatim}
\#include {avr/io.h}

void uart\_init(unsigned int baud)
\{
    UBRRH = (unsigned char)(baud{}8);
    UBRRL = (unsigned char)baud;
    UCSRB = (1{}TXEN);          // ટ્રાન્સમિટર સક્ષમ કરો
    UCSRC = (1{}URSEL)|(3{}UCSZ0); // 8{-બિટ ડેટા}
\}

void uart\_transmit(unsigned char data)
\{
    while(!(UCSRA \& (1{}UDRE))); // ખાલી બફર માટે રાહ જુઓ
    UDR = data;                  // ડેટા મોકલો
\}

void uart\_send\_string(char *str)
\{
    while(*str)
    \{
        uart\_transmit(*str++);
    \}
\}

int main()
\{
    uart\_init(51);              // 8MHz પર 9600 baud
    
    while(1)
    \{
        uart\_send\_string("Hello World{rn}");
        for(long i=0; i{}100000; i++); // વિલંબ
    \}
\}
\end{verbatim}

\textbf{મુખ્ય ઘટકો:}

\begin{itemize}
\tightlist
\item
  \textbf{બોડ રેટ સેટિંગ}: UBRR રજિસ્ટર કમ્યુનિકેશન સ્પીડ સેટ કરે છે
\item
  \textbf{ટ્રાન્સમિટ સક્ષમ}: TXEN બિટ UART ટ્રાન્સમિટર સક્ષમ કરે છે
\item
  \textbf{ડેટા ટ્રાન્સમિશન}: UDR રજિસ્ટર ટ્રાન્સમિટ કરવાનો ડેટા હોલ્ડ કરે છે
\item
  \textbf{બફર ચેક}: UDRE ફ્લેગ ટ્રાન્સમિટ બફર ખાલી દર્શાવે છે
\end{itemize}

\textbf{સ્મરણ સહાયક:} ``ઇનિટ, સક્ષમ, ચેક, ટ્રાન્સમિટ - IECT''

\end{solutionbox}
\begin{center}\rule{0.5\linewidth}{0.5pt}\end{center}

\subsection*{પ્રશ્ન 4(અ) [3
ગુણ]}\label{uxaaauxab0uxab6uxaa8-4uxa85-3-uxa97uxaa3}

\textbf{ADMUX રજિસ્ટર સમજાવો.}

\begin{solutionbox}


{\def\LTcaptype{none} % do not increment counter
\vspace{-5pt}
\captionof{table}{ADMUX રજિસ્ટર બિટ્સ}
\vspace{-10pt}
\begin{longtable}[]{@{}lll@{}}
\toprule\noalign{}
બિટ & નામ & કાર્ય \\
\midrule\noalign{}
\endhead
\bottomrule\noalign{}
\endlastfoot
\textbf{REFS1:0} & રેફરન્સ સિલેક્ટ & વોલ્ટેજ રેફરન્સ પસંદગી \\
\textbf{ADLAR} & લેફ્ટ એડજસ્ટ & પરિણામ ડાબે એડજસ્ટમેન્ટ \\
\textbf{MUX4:0} & ચેનલ સિલેક્ટ & ADC ઇનપુટ ચેનલ પસંદગી \\
\end{longtable}
}

\begin{itemize}
\tightlist
\item
  \textbf{રેફરન્સ વોલ્ટેજ}: આંતરિક/બાહ્ય વોલ્ટેજ રેફરન્સ પસંદ કરે છે
\item
  \textbf{પરિણામ ફોર્મેટ}: ADLAR બિટ 10-બિટ પરિણામ એલાઇનમેન્ટ એડજસ્ટ કરે છે
\item
  \textbf{ચેનલ પસંદગી}: MUX બિટ્સ કયા ADC પિનને વાંચવો તે પસંદ કરે છે
\end{itemize}

\textbf{સ્મરણ સહાયક:} ``રેફરન્સ, એડજસ્ટ, ચેનલ - RAC''

\end{solutionbox}
\begin{center}\rule{0.5\linewidth}{0.5pt}\end{center}

\subsection*{પ્રશ્ન 4(બ) [4
ગુણ]}\label{uxaaauxab0uxab6uxaa8-4uxaac-4-uxa97uxaa3}

\textbf{ATmega32 સાથે ઇન્ટરફેસિંગ રિલે દોરો અને સમજાવો.}

\begin{solutionbox}

\textbf{આકૃતિ: રિલે ઇન્ટરફેસિંગ}

\begin{verbatim}
ATmega32                    Relay Circuit
                         
  PA0 {-{-}{-}{-}+                +12V}
          |                 |
          R            [Relay Coil]
          |                 |
          |     +{-{-}{-}{-}{-}+     |}
          +{-{-}{-}{-}{-}|  T  |{-}{-}{-}{-}{-}+}
                | NPN |
                +{-{-}{-}{-}{-}+}
                  |
                 GND
                 
T = BC547 Transistor
R = 1K Resistor
\end{verbatim}

\textbf{ઘટકો:}

\begin{itemize}
\tightlist
\item
  \textbf{ટ્રાન્ઝિસ્ટર સ્વિચ}: BC547 NPN ટ્રાન્ઝિસ્ટર ઇલેક્ટ્રોનિક સ્વિચ તરીકે કામ
  કરે છે
\item
  \textbf{બેઝ રેઝિસ્ટર}: 1KΩ માઇક્રોકન્ટ્રોલરથી બેઝ કરન્ટ મર્યાદિત કરે છે
\item
  \textbf{રિલે કોઇલ}: 12V રિલે બાહ્ય હાઇ-પાવર ઉપકરણો ઓપરેટ કરે છે
\item
  \textbf{પ્રોટેક્શન ડાયોડ}: બેક EMF થી બચાવવા માટે ફ્રીવ્હીલિંગ ડાયોડ
\end{itemize}

\textbf{સ્મરણ સહાયક:} ``માઇક્રો ટ્રાન્ઝિસ્ટર કંટ્રોલ કરે, ટ્રાન્ઝિસ્ટર રિલે કંટ્રોલ
કરે''

\end{solutionbox}
\begin{center}\rule{0.5\linewidth}{0.5pt}\end{center}

\subsection*{પ્રશ્ન 4(ક) [7
ગુણ]}\label{uxaaauxab0uxab6uxaa8-4uxa95-7-uxa97uxaa3}

\textbf{AVR માં TWI રજિસ્ટર દોરો અને સમજાવો.}

\begin{solutionbox}

\textbf{આકૃતિ: TWI રજિસ્ટર સ્ટ્રક્ચર}

\begin{verbatim}
TWCR (TWI Control Register)
+{-{-}{-}{-}{-}+{-}{-}{-}{-}+{-}{-}{-}{-}{-}+{-}{-}{-}{-}{-}+{-}{-}{-}{-}+{-}{-}{-}{-}+{-}{-}{-}+{-}{-}{-}{-}+}
|TWINT|TWEA|TWSTA|TWSTO|TWWC|TWEN| {- |TWIE|}
+{-{-}{-}{-}{-}+{-}{-}{-}{-}+{-}{-}{-}{-}{-}+{-}{-}{-}{-}{-}+{-}{-}{-}{-}+{-}{-}{-}{-}+{-}{-}{-}+{-}{-}{-}{-}+}
 7      6     5     4    3     2   1    0

TWSR (TWI Status Register)  
+{-{-}{-}{-}+{-}{-}{-}{-}+{-}{-}{-}{-}+{-}{-}{-}{-}+{-}{-}{-}{-}+{-}{-}{-}+{-}{-}{-}{-}{-}+{-}{-}{-}{-}{-}+}
|TWS7|TWS6|TWS5|TWS4|TWS3| {- |TWPS1|TWPS0|}
+{-{-}{-}{-}+{-}{-}{-}{-}+{-}{-}{-}{-}+{-}{-}{-}{-}+{-}{-}{-}{-}+{-}{-}{-}+{-}{-}{-}{-}{-}+{-}{-}{-}{-}{-}+}
 7     6     5    4    3   2    1     0

TWDR (TWI Data Register)
+{-{-}{-}{-}+{-}{-}{-}{-}+{-}{-}{-}{-}+{-}{-}{-}{-}+{-}{-}{-}{-}+{-}{-}{-}{-}+{-}{-}{-}{-}+{-}{-}{-}{-}+}
|TWD7|TWD6|TWD5|TWD4|TWD3|TWD2|TWD1|TWD0|
+{-{-}{-}{-}+{-}{-}{-}{-}+{-}{-}{-}{-}+{-}{-}{-}{-}+{-}{-}{-}{-}+{-}{-}{-}{-}+{-}{-}{-}{-}+{-}{-}{-}{-}+}
 7     6     5    4    3    2    1    0
\end{verbatim}

\textbf{રજિસ્ટર કાર્યો:}

\begin{itemize}
\tightlist
\item
  \textbf{TWCR}: TWI ઓપરેશન અને ઇન્ટરપ્ટ હેન્ડલિંગ કંટ્રોલ કરે છે
\item
  \textbf{TWSR}: સ્ટેટસ માહિતી અને પ્રીસ્કેલર સેટિંગ પ્રદાન કરે છે
\item
  \textbf{TWDR}: ટ્રાન્સમિશન/રિસેપ્શન માટે ડેટા હોલ્ડ કરે છે
\item
  \textbf{TWAR}: સ્લેવ તરીકે ઓપરેટ કરતી વખતે સ્લેવ એડ્રેસ સેટ કરે છે
\item
  \textbf{TWBR}: TWI કમ્યુનિકેશન માટે બિટ રેટ સેટ કરે છે
\item
  \textbf{TWINT}: ઇન્ટરપ્ટ ફ્લેગ 1 લખીને ક્લિયર થાય છે
\item
  \textbf{Start/Stop}: TWSTA અને TWSTO I2C કન્ડિશન કંટ્રોલ કરે છે
\end{itemize}

\textbf{સ્મરણ સહાયક:} ``કંટ્રોલ, સ્ટેટસ, ડેટા, એડ્રેસ, બિટ રેટ - CSDAB''

\end{solutionbox}
\begin{center}\rule{0.5\linewidth}{0.5pt}\end{center}

\subsection*{પ્રશ્ન 4(અ અથવા) [3
ગુણ]}\label{uxaaauxab0uxab6uxaa8-4uxa85-uxa85uxaa5uxab5-3-uxa97uxaa3}

\textbf{ADCSRA રજિસ્ટર સમજાવો.}

\begin{solutionbox}


{\def\LTcaptype{none} % do not increment counter
\vspace{-5pt}
\captionof{table}{ADCSRA રજિસ્ટર બિટ્સ}
\vspace{-10pt}
\begin{longtable}[]{@{}lll@{}}
\toprule\noalign{}
બિટ & નામ & કાર્ય \\
\midrule\noalign{}
\endhead
\bottomrule\noalign{}
\endlastfoot
\textbf{ADEN} & ADC સક્ષમ & ADC મોડ્યુલ સક્ષમ કરે છે \\
\textbf{ADSC} & કન્વર્ઝન શરૂ કરો & ADC કન્વર્ઝન શરૂ કરે છે \\
\textbf{ADATE} & ઓટો ટ્રિગર & ઓટો ટ્રિગર મોડ સક્ષમ કરે છે \\
\textbf{ADIF} & ઇન્ટરપ્ટ ફ્લેગ & ADC કન્વર્ઝન પૂર્ણ ફ્લેગ \\
\textbf{ADIE} & ઇન્ટરપ્ટ સક્ષમ & ADC ઇન્ટરપ્ટ સક્ષમ કરે છે \\
\textbf{ADPS2:0} & પ્રીસ્કેલર & ADC ક્લોક પ્રીસ્કેલર સેટ કરે છે \\
\end{longtable}
}

\begin{itemize}
\tightlist
\item
  \textbf{ADC કંટ્રોલ}: ADEN ADC સક્ષમ કરે છે, ADSC કન્વર્ઝન શરૂ કરે છે
\item
  \textbf{ઇન્ટરપ્ટ સિસ્ટમ}: કન્વર્ઝન પૂર્ણ થાય ત્યારે ADIF ફ્લેગ સેટ થાય છે
\end{itemize}

\textbf{સ્મરણ સહાયક:} ``સક્ષમ, શરૂ, ટ્રિગર, ઇન્ટરપ્ટ, પ્રીસ્કેલ - ESTIP''

\end{solutionbox}
\begin{center}\rule{0.5\linewidth}{0.5pt}\end{center}

\subsection*{પ્રશ્ન 4(બ અથવા) [4
ગુણ]}\label{uxaaauxab0uxab6uxaa8-4uxaac-uxa85uxaa5uxab5-4-uxa97uxaa3}

\textbf{ATmega32 સાથે LM35 નું ઇન્ટરફેસિંગ દોરો અને સમજાવો.}

\begin{solutionbox}

\textbf{આકૃતિ: LM35 ઇન્ટરફેસિંગ}

\begin{verbatim}
    LM35                 ATmega32
                         
   +5V {-{-}{-}{-}+              }
           |              
         [LM35]          
           |              
   GND {-{-}{-}{-}+              }
           |              
   Vout {-{-}{-}+{-}{-}{-}{-}{-}{-}{-}{-}{-}{-}{-}{-}{-} PA0 (ADC0)}
                         
   Temperature Sensor
   Output: 10mV/^
\end{verbatim}

\textbf{કનેક્શન વિગતો:}

\begin{itemize}
\tightlist
\item
  \textbf{પાવર સપ્લાય}: LM35 ને +5V અને ગ્રાઉન્ડ કનેક્શનની જરૂર છે
\item
  \textbf{આઉટપુટ વોલ્ટેજ}: પ્રતિ ડિગ્રી સેલ્સિયસ 10mV ઉત્પન્ન કરે છે
\item
  \textbf{ADC ઇનપુટ}: LM35 આઉટપુટને ADC ચેનલ (PA0) સાથે કનેક્ટ કરો
\item
  \textbf{ટેમ્પરેચર ગણતરી}: ^\circC = (ADC\_Value \times 5000mV) / (1024 \times 10mV)
\end{itemize}

\textbf{કોડ ઉદાહરણ:}

\begin{verbatim}
float temp = (adc\_read() * 5.0 * 100.0) / 1024.0;
\end{verbatim}

\textbf{સ્મરણ સહાયક:} ``LM35 પ્રતિ ડિગ્રી 10mV આપે છે''

\end{solutionbox}
\begin{center}\rule{0.5\linewidth}{0.5pt}\end{center}

\subsection*{પ્રશ્ન 4(ક અથવા) [7
ગુણ]}\label{uxaaauxab0uxab6uxaa8-4uxa95-uxa85uxaa5uxab5-7-uxa97uxaa3}

\textbf{ATmega32 સાથે MAX7221 નો ઉપયોગ કરીને બહુવિધ 7-સેગમેન્ટ ડિસ્પ્લેના
ઇન્ટરફેસિંગ દોરો અને સમજાવો.}

\begin{solutionbox}

\textbf{આકૃતિ: MAX7221 ઇન્ટરફેસિંગ}

\begin{verbatim}
ATmega32                MAX7221              7{-Segment Displays}

PB5(MOSI) {-{-}{-}{-}{-}{-}{-}{-}{-}{-}{-}{-}{-} DIN                   DIG0 {-}{-}{-}{-} Display 1}
PB7(SCK)  {-{-}{-}{-}{-}{-}{-}{-}{-}{-}{-}{-}{-} CLK                   DIG1 {-}{-}{-}{-} Display 2  }
PB4(SS)   {-{-}{-}{-}{-}{-}{-}{-}{-}{-}{-}{-}{-} CS                    DIG2 {-}{-}{-}{-} Display 3}
                                              DIG3 {-{-}{-}{-} Display 4}
          +5V {-{-}{-}{-}{-}{-}{-}{-}{-} VCC                   DIG4 {-}{-}{-}{-} Display 5}
          GND {-{-}{-}{-}{-}{-}{-}{-}{-} GND                   DIG5 {-}{-}{-}{-} Display 6}
                                              DIG6 {-{-}{-}{-} Display 7}
                        SEGA {-{-}{-}{-} Common segments}
                        SEGB     to all displays
                        SEGC
                        SEGD
                        SEGE
                        SEGF
                        SEGG
                        SEGDP
\end{verbatim}

\textbf{લક્ષણો:}

\begin{itemize}
\tightlist
\item
  \textbf{SPI કમ્યુનિકેશન}: કંટ્રોલ માટે સીરીયલ પેરિફેરલ ઇન્ટરફેસ ઉપયોગ કરે છે
\item
  \textbf{બહુવિધ ડિસ્પ્લે}: 8 સુધી સેવન-સેગમેન્ટ ડિસ્પ્લે કંટ્રોલ કરે છે
\item
  \textbf{ઓટોમેટિક સ્કેનિંગ}: MAX7221 મલ્ટિપ્લેક્સિંગ ઓટોમેટિક હેન્ડલ કરે છે
\item
  \textbf{બ્રાઇટનેસ કંટ્રોલ}: સોફ્ટવેર-કંટ્રોલ્ડ બ્રાઇટનેસ લેવલ
\item
  \textbf{ડીકોડ મોડ}: બિલ્ટ-ઇન BCD થી 7-સેગમેન્ટ ડીકોડર
\item
  \textbf{ઓછા ઘટકો}: જરૂરી બાહ્ય ઘટકો ઘટાડે છે
\end{itemize}

\textbf{મુખ્ય રજિસ્ટર:}

\begin{itemize}
\tightlist
\item
  \textbf{ડીકોડ મોડ રજિસ્ટર}: BCD ડીકોડિંગ સક્ષમ/અક્ષમ કરે છે
\item
  \textbf{ઇન્ટેન્સિટી રજિસ્ટર}: ડિસ્પ્લે બ્રાઇટનેસ કંટ્રોલ કરે છે
\item
  \textbf{સ્કેન લિમિટ રજિસ્ટર}: સક્રિય ડિસ્પ્લેની સંખ્યા સેટ કરે છે
\item
  \textbf{શટડાઉન રજિસ્ટર}: સામાન્ય ઓપરેશન અથવા શટડાઉન મોડ
\end{itemize}

\textbf{સ્મરણ સહાયક:} ``SPI બહુવિધ ડિસ્પ્લે માટે સીરીયલ ડેટા મોકલે છે''

\end{solutionbox}
\begin{center}\rule{0.5\linewidth}{0.5pt}\end{center}

\subsection*{પ્રશ્ન 5(અ) [3
ગુણ]}\label{uxaaauxab0uxab6uxaa8-5uxa85-3-uxa97uxaa3}

\textbf{SPCR રજિસ્ટર સમજાવો.}

\begin{solutionbox}


{\def\LTcaptype{none} % do not increment counter
\vspace{-5pt}
\captionof{table}{SPCR રજિસ્ટર બિટ્સ}
\vspace{-10pt}
\begin{longtable}[]{@{}lll@{}}
\toprule\noalign{}
બિટ & નામ & કાર્ય \\
\midrule\noalign{}
\endhead
\bottomrule\noalign{}
\endlastfoot
\textbf{SPIE} & ઇન્ટરપ્ટ સક્ષમ & SPI ઇન્ટરપ્ટ સક્ષમ કરે છે \\
\textbf{SPE} & SPI સક્ષમ & SPI મોડ્યુલ સક્ષમ કરે છે \\
\textbf{DORD} & ડેટા ઓર્ડર & LSB/MSB પ્રથમ પસંદગી \\
\textbf{MSTR} & માસ્ટર/સ્લેવ & માસ્ટર અથવા સ્લેવ મોડ પસંદ કરે છે \\
\textbf{CPOL} & ક્લોક પોલેરિટી & ક્લોક આઈડલ સ્ટેટ પસંદગી \\
\textbf{CPHA} & ક્લોક ફેઝ & ડેટા સેમ્પલિંગ માટે ક્લોક એજ \\
\textbf{SPR1:0} & ક્લોક રેટ & SPI ક્લોક રેટ પસંદગી \\
\end{longtable}
}

\begin{itemize}
\tightlist
\item
  \textbf{SPI સક્ષમ}: SPI કાર્યક્ષમતા સક્ષમ કરવા માટે SPE બિટ સેટ કરવું જરૂરી છે
\item
  \textbf{માસ્ટર મોડ}: MSTR બિટ નક્કી કરે છે કે ઉપકરણ માસ્ટર છે કે સ્લેવ
\end{itemize}

\textbf{સ્મરણ સહાયક:} ``ઇન્ટરપ્ટ, સક્ષમ, ડેટા, માસ્ટર, ક્લોક સેટિંગ્સ - IEDMC''

\end{solutionbox}
\begin{center}\rule{0.5\linewidth}{0.5pt}\end{center}

\subsection*{પ્રશ્ન 5(બ) [4
ગુણ]}\label{uxaaauxab0uxab6uxaa8-5uxaac-4-uxa97uxaa3}

\textbf{L293D મોટર ડ્રાઇવરનો ઉપયોગ કરીને ATmega32 સાથે DC મોટરને ઇન્ટરફેસ કરવા
માટે સર્કિટ ડાયાગ્રામ દોરો.}

\begin{solutionbox}

\textbf{આકૃતિ: DC મોટર ઇન્ટરફેસિંગ}

\begin{verbatim}
ATmega32              L293D                DC Motor

PA0 {-{-}{-}{-}{-}{-}{-}{-}{-}{-}{-} IN1    OUT1 {-}{-}{-}{-}{-}{-}{-}{-}{-}{-}+}
PA1 {-{-}{-}{-}{-}{-}{-}{-}{-}{-}{-} IN2    OUT2 {-}{-}{-}{-}{-}{-}{-}{-}{-}{-}+    [Motor]}
                                      |      M
+5V {-{-}{-}{-}{-}{-}{-}{-}{-}{-}{-} VCC1   VCC2 {-}{-}{-}{-}{-}{-}{-}{-}{-} +12V  |}
GND {-{-}{-}{-}{-}{-}{-}{-}{-}{-}{-} GND    GND  {-}{-}{-}{-}{-}{-}{-}{-}{-} GND   |}
PA2 {-{-}{-}{-}{-}{-}{-}{-}{-}{-}{-} EN1                         |}
                                            |
               Input Logic Table:           |
               IN1  IN2  Motor              |
                0    0   Stop               |
                0    1   CCW                |
                1    0   CW                 |
                1    1   Brake              |
\end{verbatim}

\textbf{ઘટકો:}

\begin{itemize}
\tightlist
\item
  \textbf{L293D ડ્રાઇવર}: મોટર કંટ્રોલ માટે કરન્ટ એમ્પ્લિફિકેશન પ્રદાન કરે છે
\item
  \textbf{પાવર સપ્લાય}: લૉજિક માટે +5V, મોટર પાવર માટે +12V
\item
  \textbf{કંટ્રોલ સિગ્નલ}: IN1, IN2 મોટરની દિશા નક્કી કરે છે
\item
  \textbf{સક્ષમ પિન}: EN1 મોટર ઓન/ઓફ અને સ્પીડ (PWM) કંટ્રોલ કરે છે
\end{itemize}

\textbf{સ્મરણ સહાયક:} ``લૉજિક દિશા કંટ્રોલ કરે, સક્ષમ સ્પીડ કંટ્રોલ કરે''

\end{solutionbox}
\begin{center}\rule{0.5\linewidth}{0.5pt}\end{center}

\subsection*{પ્રશ્ન 5(ક) [7
ગુણ]}\label{uxaaauxab0uxab6uxaa8-5uxa95-7-uxa97uxaa3}

\textbf{IoT આધારિત હોમ ઓટોમેશન સિસ્ટમ સમજાવો.}

\begin{solutionbox}

\textbf{આકૃતિ: IoT હોમ ઓટોમેશન સિસ્ટમ}

\begin{verbatim}
    Internet Cloud
          |
    +{-{-}{-}{-}{-}{-}{-}{-}{-}{-}+}
    |  Router  |
    +{-{-}{-}{-}{-}{-}{-}{-}{-}{-}+}
          |
    +{-{-}{-}{-}{-}{-}{-}{-}{-}{-}+      +{-}{-}{-}{-}{-}{-}{-}{-}{-}{-}+      +{-}{-}{-}{-}{-}{-}{-}{-}{-}{-}+}
    |   ESP32  |{-{-}{-}{-}{-}{-}|ATmega32  |{-}{-}{-}{-}{-}{-}| Devices  |}
    | WiFi MCU |      |Main MCU  |      |• Lights  |
    +{-{-}{-}{-}{-}{-}{-}{-}{-}{-}+      +{-}{-}{-}{-}{-}{-}{-}{-}{-}{-}+      |• Fan     |}
          |                  |          |• AC      |
    +{-{-}{-}{-}{-}{-}{-}{-}{-}{-}+      +{-}{-}{-}{-}{-}{-}{-}{-}{-}{-}+      |• Security|}
    |   App    |      | Sensors  |      +{-{-}{-}{-}{-}{-}{-}{-}{-}{-}+}
    |Smartphone|      |• Temp    |
    +{-{-}{-}{-}{-}{-}{-}{-}{-}{-}+      |• Motion  |}
                      |• LDR     |
                      +{-{-}{-}{-}{-}{-}{-}{-}{-}{-}+}
\end{verbatim}

\textbf{સિસ્ટમ ઘટકો:}

\begin{itemize}
\tightlist
\item
  \textbf{ઇન્ટરનેટ કનેક્ટિવિટી}: WiFi મોડ્યુલ સિસ્ટમને ઇન્ટરનેટ સાથે કનેક્ટ કરે છે
\item
  \textbf{મોબાઇલ એપ્લિકેશન}: રિમોટ કંટ્રોલ અને મોનિટરિંગ માટે યુઝર ઇન્ટરફેસ
\item
  \textbf{સેન્સર નેટવર્ક}: ઓટોમેશન માટે ટેમ્પરેચર, મોશન, લાઇટ સેન્સર
\item
  \textbf{કંટ્રોલ ઉપકરણો}: રિલે ઘરના ઉપકરણો અને લાઇટ કંટ્રોલ કરે છે
\item
  \textbf{સેન્ટ્રલ કંટ્રોલર}: માઇક્રોકન્ટ્રોલર કમાન્ડ અને સેન્સર ડેટા પ્રોસેસ કરે છે
\item
  \textbf{ક્લાઉડ સેવાઓ}: ડેટા સ્ટોર કરે છે અને રિમોટ એક્સેસ સક્ષમ કરે છે
\end{itemize}

\textbf{લક્ષણો:}

\begin{itemize}
\tightlist
\item
  \textbf{રિમોટ કંટ્રોલ}: ઇન્ટરનેટ દ્વારા ગમે ત્યાંથી ઉપકરણો કંટ્રોલ કરો
\item
  \textbf{ઓટોમેશન}: સેન્સર રીડિંગ આધારે ઓટોમેટિક કંટ્રોલ
\item
  \textbf{એનર્જી સેવિંગ}: સ્માર્ટ શેડ્યુલિંગ પાવર વપરાશ ઘટાડે છે
\item
  \textbf{સુરક્ષા મોનિટરિંગ}: સુરક્ષા માટે મોશન સેન્સર અને કેમેરા
\item
  \textbf{ડેટા લૉગિંગ}: વિશ્લેષણ માટે ઐતિહાસિક ડેટા સ્ટોરેજ
\end{itemize}

\textbf{સ્મરણ સહાયક:} ``ઇન્ટરનેટ ફોનને ઘરના ઉપકરણો સાથે જોડે છે - IPHD''

\end{solutionbox}
\begin{center}\rule{0.5\linewidth}{0.5pt}\end{center}

\subsection*{પ્રશ્ન 5(અ અથવા) [3
ગુણ]}\label{uxaaauxab0uxab6uxaa8-5uxa85-uxa85uxaa5uxab5-3-uxa97uxaa3}

\textbf{SPSR રજિસ્ટર સમજાવો.}

\begin{solutionbox}


{\def\LTcaptype{none} % do not increment counter
\vspace{-5pt}
\captionof{table}{SPSR રજિસ્ટર બિટ્સ}
\vspace{-10pt}
\begin{longtable}[]{@{}lll@{}}
\toprule\noalign{}
બિટ & નામ & કાર્ય \\
\midrule\noalign{}
\endhead
\bottomrule\noalign{}
\endlastfoot
\textbf{SPIF} & ઇન્ટરપ્ટ ફ્લેગ & SPI ટ્રાન્સફર પૂર્ણ ફ્લેગ \\
\textbf{WCOL} & રાઇટ કોલિશન & ડેટા કોલિશન એરર ફ્લેગ \\
\textbf{SPI2X} & ડબલ સ્પીડ & SPI ક્લોક રેટ બમણી કરે છે \\
\end{longtable}
}

\begin{itemize}
\tightlist
\item
  \textbf{ટ્રાન્સફર પૂર્ણ}: SPIF ફ્લેગ SPI ટ્રાન્સમિશન સમાપ્ત થયું દર્શાવે છે
\item
  \textbf{કોલિશન ડિટેક્શન}: WCOL ફ્લેગ રાઇટ કોલિશન થયું બતાવે છે
\item
  \textbf{સ્પીડ કંટ્રોલ}: SPI2X સેટ કરવાથી કમ્યુનિકેશન સ્પીડ બમણી થાય છે
\end{itemize}

\textbf{સ્મરણ સહાયક:} ``ફ્લેગ, કોલિશન, સ્પીડ - FCS''

\end{solutionbox}
\begin{center}\rule{0.5\linewidth}{0.5pt}\end{center}

\subsection*{પ્રશ્ન 5(બ અથવા) [4
ગુણ]}\label{uxaaauxab0uxab6uxaa8-5uxaac-uxa85uxaa5uxab5-4-uxa97uxaa3}

\textbf{L293D મોટર ડ્રાઇવર IC નો પિન ડાયાગ્રામ દોરો અને સમજાવો.}

\begin{solutionbox}

\textbf{આકૃતિ: L293D પિન કૉન્ફિગરેશન}

\begin{verbatim}
      L293D (16{-pin DIP)}
    +{-{-}{-}{-}{-}{-}{-}{-}{-}{-}{-}{-}{-}{-}{-}{-}{-}{-}{-}{-}+}
EN1 |1                 16| VCC1
IN1 |2                 15| IN4  
OUT1|3                 14| OUT4
GND |4                 13| GND
GND |5                 12| GND
OUT2|6                 11| OUT3
IN2 |7                 10| IN3
VCC2|8                  9| EN2
    +{-{-}{-}{-}{-}{-}{-}{-}{-}{-}{-}{-}{-}{-}{-}{-}{-}{-}{-}{-}+}
\end{verbatim}

\textbf{પિન કાર્યો:}

\begin{itemize}
\tightlist
\item
  \textbf{સક્ષમ પિન (EN1, EN2)}: PWM દ્વારા મોટર ઓન/ઓફ અને સ્પીડ કંટ્રોલ કરે છે
\item
  \textbf{ઇનપુટ પિન (IN1-IN4)}: માઇક્રોકન્ટ્રોલરથી લૉજિક ઇનપુટ
\item
  \textbf{આઉટપુટ પિન (OUT1-OUT4)}: મોટર માટે હાઇ કરન્ટ આઉટપુટ
\item
  \textbf{પાવર સપ્લાય (VCC1)}: IC ઓપરેશન માટે +5V લૉજિક સપ્લાય
\item
  \textbf{મોટર સપ્લાય (VCC2)}: મોટર પાવર માટે +12V સપ્લાય
\item
  \textbf{ગ્રાઉન્ડ પિન}: હીટ ડિસિપેશન માટે બહુવિધ ગ્રાઉન્ડ કનેક્શન
\end{itemize}

\textbf{લક્ષણો:}

\begin{itemize}
\tightlist
\item
  \textbf{ડ્યુઅલ H-બ્રિજ}: બે DC મોટર સમાંતર કંટ્રોલ કરી શકે છે
\item
  \textbf{કરન્ટ કેપેસિટી}: પ્રતિ ચેનલ 600mA, 1.2A પીક
\item
  \textbf{પ્રોટેક્શન}: મોટર પ્રોટેક્શન માટે બિલ્ટ-ઇન ફ્લાયબેક ડાયોડ
\end{itemize}

\textbf{સ્મરણ સહાયક:} ``સક્ષમ, ઇનપુટ, આઉટપુટ, પાવર - EIOP''

\end{solutionbox}
\begin{center}\rule{0.5\linewidth}{0.5pt}\end{center}

\subsection*{પ્રશ્ન 5(ક અથવા) [7
ગુણ]}\label{uxaaauxab0uxab6uxaa8-5uxa95-uxa85uxaa5uxab5-7-uxa97uxaa3}

\textbf{મોટરાઇઝ્ડ કંટ્રોલ રોબોટિક્સ સિસ્ટમ સમજાવો.}

\begin{solutionbox}

\textbf{આકૃતિ: રોબોટિક્સ કંટ્રોલ સિસ્ટમ}

\begin{verbatim}
    +{-{-}{-}{-}{-}{-}{-}{-}{-}{-}{-}{-}{-}{-}{-}{-}+      +{-}{-}{-}{-}{-}{-}{-}{-}{-}{-}{-}{-}{-}{-}{-}{-}+      +{-}{-}{-}{-}{-}{-}{-}{-}{-}{-}{-}{-}{-}{-}{-}{-}+}
    |   Sensors      |      | Microcontroller|      |   Actuators    |
    |                |{-{-}{-}{-}{-}|                |{-}{-}{-}{-}{-}|                |}
    | • Ultrasonic   |      | • ATmega32     |      | • DC Motors    |
    | • IR Sensor    |      | • Processing   |      | • Servo Motors |
    | • Gyroscope    |      | • Decision     |      | • Stepper      |
    | • Camera       |      | • Control      |      | • Gripper      |
    +{-{-}{-}{-}{-}{-}{-}{-}{-}{-}{-}{-}{-}{-}{-}{-}+      +{-}{-}{-}{-}{-}{-}{-}{-}{-}{-}{-}{-}{-}{-}{-}{-}+      +{-}{-}{-}{-}{-}{-}{-}{-}{-}{-}{-}{-}{-}{-}{-}{-}+}
             |                        |                        |
             v                        v                        v
    +{-{-}{-}{-}{-}{-}{-}{-}{-}{-}{-}{-}{-}{-}{-}{-}+      +{-}{-}{-}{-}{-}{-}{-}{-}{-}{-}{-}{-}{-}{-}{-}{-}+      +{-}{-}{-}{-}{-}{-}{-}{-}{-}{-}{-}{-}{-}{-}{-}{-}+}
    |  Communication |      |   Power Supply |      |   Feedback     |
    |                |      |                |      |                |
    | • Bluetooth    |      | • Battery      |      | • Encoders     |
    | • WiFi         |      | • Regulators   |      | • Position     |
    | • RF Module    |      | • Protection   |      | • Speed        |
    +{-{-}{-}{-}{-}{-}{-}{-}{-}{-}{-}{-}{-}{-}{-}{-}+      +{-}{-}{-}{-}{-}{-}{-}{-}{-}{-}{-}{-}{-}{-}{-}{-}+      +{-}{-}{-}{-}{-}{-}{-}{-}{-}{-}{-}{-}{-}{-}{-}{-}+}
\end{verbatim}

\textbf{સિસ્ટમ ઘટકો:}


{\def\LTcaptype{none} % do not increment counter
\vspace{-5pt}
\captionof{table}{રોબોટિક્સ સિસ્ટમ એલિમેન્ટ્સ}
\vspace{-10pt}
\begin{longtable}[]{@{}lll@{}}
\toprule\noalign{}
ઘટક & કાર્ય & ઉદાહરણો \\
\midrule\noalign{}
\endhead
\bottomrule\noalign{}
\endlastfoot
\textbf{સેન્સર} & પર્યાવરણ સેન્સિંગ & અલ્ટ્રાસોનિક, IR, કેમેરા \\
\textbf{કંટ્રોલર} & નિર્ણય લેવો & ATmega32, Arduino \\
\textbf{એક્ચ્યુએટર} & ભૌતિક હલનચલન & મોટર, સર્વો \\
\textbf{કમ્યુનિકેશન} & રિમોટ કંટ્રોલ & બ્લૂટૂથ, WiFi \\
\textbf{પાવર} & એનર્જી સપ્લાય & બેટરી, રેગ્યુલેટર \\
\textbf{ફીડબેક} & પોઝિશન સેન્સિંગ & એન્કોડર, જાયરોસ્કોપ \\
\end{longtable}
}

\textbf{કંટ્રોલ અલ્ગોરિધમ:}

\begin{itemize}
\tightlist
\item
  \textbf{સેન્સ}: સેન્સર ઉપયોગ કરીને પર્યાવરણથી ડેટા એકત્રિત કરો
\item
  \textbf{પ્રોસેસ}: સેન્સર ડેટાનું વિશ્લેષણ કરો અને નિર્ણયો લો
\item
  \textbf{એક્ટ}: નિર્ણયો આધારે મોટર અને એક્ચ્યુએટર કંટ્રોલ કરો
\item
  \textbf{ફીડબેક}: વાસ્તવિક હલનચલન મોનિટર કરો અને કંટ્રોલ એડજસ્ટ કરો
\item
  \textbf{કમ્યુનિકેટ}: સ્ટેટસ મોકલો અને વાયરલેસ કમાન્ડ રિસીવ કરો
\end{itemize}

\textbf{એપ્લિકેશન:}

\begin{itemize}
\tightlist
\item
  \textbf{સ્વાયત્ત નેવિગેશન}: રોબોટ સેન્સર ઉપયોગ કરીને સ્વતંત્ર રીતે મૂવ કરે છે
\item
  \textbf{ઓબ્જેક્ટ મેનિપ્યુલેશન}: પિક અને પ્લેસ કાર્યો માટે ગ્રિપર કંટ્રોલ
\item
  \textbf{રિમોટ ઓપરેશન}: વાયરલેસ કમ્યુનિકેશન દ્વારા મેન્યુઅલ કંટ્રોલ
\item
  \textbf{પાથ ફોલોવિંગ}: લાઇન ફોલોવિંગ અથવા પૂર્વનિર્ધારિત રૂટ નેવિગેશન
\item
  \textbf{ઓબ્સ્ટેકલ એવોઇડન્સ}: અવરોધોની આસપાસ ડાયનેમિક પાથ પ્લાનિંગ
\end{itemize}

\textbf{પ્રોગ્રામિંગ સ્ટ્રક્ચર:}

\begin{verbatim}
while(1) \{
    read\_sensors();
    process\_data();
    make\_decision();
    control\_motors();
    check\_feedback();
    communicate\_status();
\}
\end{verbatim}

\textbf{સ્મરણ સહાયક:} ``સેન્સ, પ્રોસેસ, એક્ટ, ફીડબેક, કમ્યુનિકેટ - SPACF''

\end{solutionbox}

\end{document}
