\documentclass[10pt,a4paper]{article}

% content/resources/templates/preamble.tex
\usepackage[margin=0.6in]{geometry}
\author{Milav Dabgar}
\usepackage{amsmath,amssymb,amsthm}
\usepackage{booktabs}
\usepackage{multirow}
\usepackage{xcolor}
\usepackage{tcolorbox}
\tcbuselibrary{breakable,skins}
\usepackage[colorlinks=true,linkcolor=blue]{hyperref}
\usepackage{titlesec}
\usepackage{enumitem}
\usepackage{tikz}
\usepackage{pgfplots}
\usepackage{circuitikz}
\usepackage[version=4]{mhchem}
\usepackage{longtable}
\usepackage{array}
\usepackage{float}
\usepackage{caption}
\usepackage{listings}

\lstset{
  basicstyle=\small\ttfamily,
  breaklines=true,
  breakatwhitespace=false,
  postbreak=\mbox{\textcolor{red}{$\hookrightarrow$}\space},
  float=false,
  numbers=left,
  numberstyle=\tiny\color{gray},
  numbersep=10pt,
  xleftmargin=2em,
  keywordstyle=\color{blue},
  commentstyle=\color{green!60!black},
  stringstyle=\color{purple},
  backgroundcolor=\color{gray!5},
  showstringspaces=false,
  tabsize=2,
  captionpos=b,
  keepspaces=true,
  columns=flexible
}

\pgfplotsset{compat=1.18}
\usetikzlibrary{shapes,arrows,positioning,calc,patterns,decorations.pathmorphing,decorations.markings,arrows.meta}

% Color scheme
\definecolor{headcolor}{RGB}{0,102,204}
\definecolor{keycolor}{RGB}{220,20,60}
\definecolor{solutioncolor}{RGB}{34,139,34}
\definecolor{mnemoniccolor}{RGB}{148,0,211}
\definecolor{codecolor}{RGB}{0,0,100}

% Spacing
\setlength{\parskip}{3pt}
\setlist[itemize]{nosep}
\setlist[enumerate]{nosep}

% Title formatting
\titleformat{\section}{\Large\bfseries\color{headcolor}}{\thesection}{1em}{}
\titleformat{\subsection}{\large\bfseries\color{headcolor}}{\thesubsection}{1em}{}

% Pandoc tightlist compatibility
\providecommand{\tightlist}{%
  \setlength{\itemsep}{0pt}\setlength{\parskip}{0pt}}

% Pandoc longtable compatibility
\newcounter{none}
\def\thenone{}


% content/resources/templates/english-boxes.tex
% This file is currently empty - it exists to maintain consistency with the import structure.
% Add custom environments here if needed in the future.


\begin{document}

\begin{center}
{\Huge\bfseries\color{headcolor} Subject Name Solutions}\\[5pt]
{\LARGE 1313202 -- Summer 2024}\\[3pt]
{\large Semester 1 Study Material}\\[3pt]
{\normalsize\textit{Detailed Solutions and Explanations}}
\end{center}

\vspace{10pt}

\subsection*{Question 1(a) [3 marks]}\label{q1a}

\textbf{Define: 1. Node, 2. Loop, 3. Branch}

\begin{solutionbox}

{\def\LTcaptype{none} % do not increment counter
\begin{longtable}[]{@{}
  >{\raggedright\arraybackslash}p{(\linewidth - 2\tabcolsep) * \real{0.3333}}
  >{\raggedright\arraybackslash}p{(\linewidth - 2\tabcolsep) * \real{0.6667}}@{}}
\toprule\noalign{}
\begin{minipage}[b]{\linewidth}\raggedright
Term
\end{minipage} & \begin{minipage}[b]{\linewidth}\raggedright
Definition
\end{minipage} \\
\midrule\noalign{}
\endhead
\bottomrule\noalign{}
\endlastfoot
\textbf{Node} & A point in a circuit where two or more circuit elements
meet or connect \\
\textbf{Loop} & A closed path in a circuit that starts and ends at the
same point without passing through any node more than once \\
\textbf{Branch} & A path or element connecting two nodes in a circuit \\
\end{longtable}
}

\end{solutionbox}
\begin{mnemonicbox}
``Never Loop Between'' - Nodes Link, Loops Bound,
Branches Establish connections

\end{mnemonicbox}
\subsection*{Question 1(b) [4 marks]}\label{q1b}

\textbf{Write statement of Superposition theorem and Maximum power
transfer theorem.}

\begin{solutionbox}

{\def\LTcaptype{none} % do not increment counter
\begin{longtable}[]{@{}
  >{\raggedright\arraybackslash}p{(\linewidth - 2\tabcolsep) * \real{0.4500}}
  >{\raggedright\arraybackslash}p{(\linewidth - 2\tabcolsep) * \real{0.5500}}@{}}
\toprule\noalign{}
\begin{minipage}[b]{\linewidth}\raggedright
Theorem
\end{minipage} & \begin{minipage}[b]{\linewidth}\raggedright
Statement
\end{minipage} \\
\midrule\noalign{}
\endhead
\bottomrule\noalign{}
\endlastfoot
\textbf{Superposition Theorem} & In a linear circuit with multiple
sources, the response (voltage or current) in any element equals the
algebraic sum of responses caused by each source acting alone, with all
other sources replaced by their internal impedances \\
\textbf{Maximum Power Transfer Theorem} & Maximum power is transferred
from source to load when the load resistance equals the source's
internal resistance \\
\end{longtable}
}

\textbf{Diagram:}

\includegraphics[width=1\linewidth,height=\textheight,keepaspectratio]{mermaid-5950b027.pdf}

\end{solutionbox}
\begin{mnemonicbox}
``Sum Powers Matched'' - Sum individual powers; Match
resistance for maximum

\end{mnemonicbox}
\subsection*{Question 1(c) [7 marks]}\label{q1c}

\textbf{Explain Kirchhoff's Voltage Law and Kirchhoff's current Law.}

\begin{solutionbox}

{\def\LTcaptype{none} % do not increment counter
\begin{longtable}[]{@{}
  >{\raggedright\arraybackslash}p{(\linewidth - 4\tabcolsep) * \real{0.1351}}
  >{\raggedright\arraybackslash}p{(\linewidth - 4\tabcolsep) * \real{0.3514}}
  >{\raggedright\arraybackslash}p{(\linewidth - 4\tabcolsep) * \real{0.5135}}@{}}
\toprule\noalign{}
\begin{minipage}[b]{\linewidth}\raggedright
Law
\end{minipage} & \begin{minipage}[b]{\linewidth}\raggedright
Explanation
\end{minipage} & \begin{minipage}[b]{\linewidth}\raggedright
Mathematical Form
\end{minipage} \\
\midrule\noalign{}
\endhead
\bottomrule\noalign{}
\endlastfoot
\textbf{Kirchhoff's Voltage Law (KVL)} & The algebraic sum of all
voltages around any closed loop in a circuit equals zero & Σ V = 0 \\
\textbf{Kirchhoff's Current Law (KCL)} & The algebraic sum of all
currents entering and leaving a node equals zero & Σ I = 0 \\
\end{longtable}
}

\textbf{Diagram:}

\includegraphics[width=1\linewidth,height=\textheight,keepaspectratio]{mermaid-dc8a2499.pdf}

\begin{itemize}
\tightlist
\item
  \textbf{Physical interpretation of KVL}: Energy is conserved in a
  circuit loop
\item
  \textbf{Physical interpretation of KCL}: Charge is conserved at
  circuit nodes
\item
  \textbf{Application of KVL}: Finding unknown voltages in circuit loops
\item
  \textbf{Application of KCL}: Finding unknown currents at circuit
  junctions
\end{itemize}

\end{solutionbox}
\begin{mnemonicbox}
``Voltages Loop to Zero, Currents Node to Zero''

\end{mnemonicbox}
\subsection*{Question 1(c) OR [7
marks]}\label{q1c}

\textbf{Explain series and parallel connection of resistors with
necessary equations.}

\begin{solutionbox}

{\def\LTcaptype{none} % do not increment counter
\begin{longtable}[]{@{}
  >{\raggedright\arraybackslash}p{(\linewidth - 6\tabcolsep) * \real{0.1500}}
  >{\raggedright\arraybackslash}p{(\linewidth - 6\tabcolsep) * \real{0.2000}}
  >{\raggedright\arraybackslash}p{(\linewidth - 6\tabcolsep) * \real{0.2750}}
  >{\raggedright\arraybackslash}p{(\linewidth - 6\tabcolsep) * \real{0.3750}}@{}}
\toprule\noalign{}
\begin{minipage}[b]{\linewidth}\raggedright
Connection
\end{minipage} & \begin{minipage}[b]{\linewidth}\raggedright
Characteristics
\end{minipage} & \begin{minipage}[b]{\linewidth}\raggedright
Equivalent Resistance
\end{minipage} & \begin{minipage}[b]{\linewidth}\raggedright
Current-Voltage Relationship
\end{minipage} \\
\midrule\noalign{}
\endhead
\bottomrule\noalign{}
\endlastfoot
\textbf{Series Connection} & Same current flows through all resistors &
Req = R1 + R2 + R3 + \ldots{} + Rn &

I = V/Req \\

\textbf{Parallel Connection} & Same voltage appears across all resistors
& 1/Req = 1/R1 + 1/R2 + 1/R3 + \ldots{} + 1/Rn &

I = I1 + I2 + I3 +

\ldots{} + In \\
\end{longtable}
}

\textbf{Diagram:}

\includegraphics[width=1\linewidth,height=\textheight,keepaspectratio]{mermaid-185db78d.pdf}

\begin{itemize}
\tightlist
\item
  \textbf{Current in series}: I = I1 = I2 = I3 = \ldots{} = In
\item
  \textbf{Voltage in series}: V = V1 + V2 + V3 + \ldots{} + Vn
\item
  \textbf{Current in parallel}: I = I1 + I2 + I3 + \ldots{} + In\\
\item
  \textbf{Voltage in parallel}: V = V1 = V2 = V3 = \ldots{} = Vn
\end{itemize}

\end{solutionbox}
\begin{mnemonicbox}
``Same Current Series, Same Voltage Parallel''

\end{mnemonicbox}
\subsection*{Question 2(a) [3 marks]}\label{q2a}

\textbf{State limitations of Ohm's law.}

\begin{solutionbox}

{\def\LTcaptype{none} % do not increment counter
\begin{longtable}[]{@{}
  >{\raggedright\arraybackslash}p{(\linewidth - 0\tabcolsep) * \real{1.0000}}@{}}
\toprule\noalign{}
\begin{minipage}[b]{\linewidth}\raggedright
Limitations of Ohm's Law
\end{minipage} \\
\midrule\noalign{}
\endhead
\bottomrule\noalign{}
\endlastfoot
\textbf{Non-linear components}: Does not apply to components like
diodes, transistors \\
\textbf{Temperature changes}: Not valid when temperature varies
significantly \\
\textbf{High frequencies}: Breaks down at very high frequencies \\
\end{longtable}
}

\end{solutionbox}
\begin{mnemonicbox}
``Ohm's Not Linear Thermal High'' - Non-linear,
Temperature, High frequency

\end{mnemonicbox}
\subsection*{Question 2(b) [4 marks]}\label{q2b}

\textbf{Define: 1. Doping, 2. Intrinsic Semiconductor, 3. Extrinsic
Semiconductor, 4. Dopant}

\begin{solutionbox}

{\def\LTcaptype{none} % do not increment counter
\begin{longtable}[]{@{}
  >{\raggedright\arraybackslash}p{(\linewidth - 2\tabcolsep) * \real{0.3333}}
  >{\raggedright\arraybackslash}p{(\linewidth - 2\tabcolsep) * \real{0.6667}}@{}}
\toprule\noalign{}
\begin{minipage}[b]{\linewidth}\raggedright
Term
\end{minipage} & \begin{minipage}[b]{\linewidth}\raggedright
Definition
\end{minipage} \\
\midrule\noalign{}
\endhead
\bottomrule\noalign{}
\endlastfoot
\textbf{Doping} & Process of adding impurity atoms to pure semiconductor
to modify electrical properties \\
\textbf{Intrinsic Semiconductor} & Pure semiconductor with equal number
of electrons and holes \\
\textbf{Extrinsic Semiconductor} & Doped semiconductor with unequal
number of electrons and holes \\
\textbf{Dopant} & Impurity element added to semiconductor during doping
process \\
\end{longtable}
}

\end{solutionbox}
\begin{mnemonicbox}
``Do In-Ex-Do'' - Doping Introduces Extrinsic
properties through Dopants

\end{mnemonicbox}
\subsection*{Question 2(c) [7 marks]}\label{q2c}

\textbf{Define Trivalent material and give examples of it. Explain
Formation of P-type Semiconductor with the help of proper diagram.}

\begin{solutionbox}

\textbf{Trivalent material}: Elements with 3 valence electrons in their
outermost shell.

\textbf{Examples}: Boron (B), Aluminum (Al), Gallium (Ga), Indium (In)

\textbf{P-type Semiconductor Formation}:

\textbf{Diagram:}

\begin{lstlisting}
          Silicon atom (4 valence e-)    Trivalent atom (3 valence e-)
             ┌───┐                          ┌───┐
             │   │                          │   │
          ┌──┤ Si├──┐                    ┌──┤ B ├──┐
          │  │   │  │                    │  │   │  │
       ───┼──┴───┴──┼───              ───┼──┴───┴──┼───
          │         │                    │         │
          │         │                    │    ↑    │
       ───┼─────────┼───              ───┼────┼────┼───
          │         │                    │    │    │
          │         │                    │    │    │
       ───┴─────────┴───              ───┴────┘────┴───
                                          hole
\end{lstlisting}

{\def\LTcaptype{none} % do not increment counter
\begin{longtable}[]{@{}
  >{\raggedright\arraybackslash}p{(\linewidth - 2\tabcolsep) * \real{0.5294}}
  >{\raggedright\arraybackslash}p{(\linewidth - 2\tabcolsep) * \real{0.4706}}@{}}
\toprule\noalign{}
\begin{minipage}[b]{\linewidth}\raggedright
Process
\end{minipage} & \begin{minipage}[b]{\linewidth}\raggedright
Result
\end{minipage} \\
\midrule\noalign{}
\endhead
\bottomrule\noalign{}
\endlastfoot
\textbf{Doping} & Silicon doped with trivalent atoms like Boron \\
\textbf{Bond formation} & Trivalent atoms form 3 covalent bonds with 4
neighboring Silicon atoms \\
\textbf{Hole creation} & One bond remains incomplete, creating a hole
(positive charge carrier) \\
\textbf{Majority carriers} & Holes become majority carriers \\
\textbf{Minority carriers} & Electrons become minority carriers \\
\end{longtable}
}

\end{solutionbox}
\begin{mnemonicbox}
``Three Makes Positive'' - Three valence electrons
make a Positive hole

\end{mnemonicbox}
\subsection*{Question 2(a) OR [3
marks]}\label{q2a}

\textbf{Enlist factors affecting Resistance and explain any one of
them.}

\begin{solutionbox}

{\def\LTcaptype{none} % do not increment counter
\begin{longtable}[]{@{}l@{}}
\toprule\noalign{}
Factors Affecting Resistance \\
\midrule\noalign{}
\endhead
\bottomrule\noalign{}
\endlastfoot
\textbf{Length of conductor} \\
\textbf{Cross-sectional area} \\
\textbf{Material (resistivity)} \\
\textbf{Temperature} \\
\end{longtable}
}

\textbf{Explanation of Temperature effect}: The resistance of most
metallic conductors increases with temperature as: R = R_{0}[1 + α(T -
T_{0})] where:

\begin{itemize}
\tightlist
\item
  R = Resistance at temperature T
\item
  R_{0} = Resistance at reference temperature T_{0}
\item
  α = Temperature coefficient of resistance
\end{itemize}

\end{solutionbox}
\begin{mnemonicbox}
``LAMT'' - Length, Area, Material, Temperature affect
resistance

\end{mnemonicbox}
\subsection*{Question 2(b) OR [4
marks]}\label{q2b}

\textbf{Define: 1. Valance band, 2. Conduction band, 3. Forbidden energy
gap, 4. Free electron}

\begin{solutionbox}

{\def\LTcaptype{none} % do not increment counter
\begin{longtable}[]{@{}
  >{\raggedright\arraybackslash}p{(\linewidth - 2\tabcolsep) * \real{0.3333}}
  >{\raggedright\arraybackslash}p{(\linewidth - 2\tabcolsep) * \real{0.6667}}@{}}
\toprule\noalign{}
\begin{minipage}[b]{\linewidth}\raggedright
Term
\end{minipage} & \begin{minipage}[b]{\linewidth}\raggedright
Definition
\end{minipage} \\
\midrule\noalign{}
\endhead
\bottomrule\noalign{}
\endlastfoot
\textbf{Valence band} & Energy band filled with valence electrons that
are bound to atoms \\
\textbf{Conduction band} & Higher energy band where electrons can move
freely and conduct electricity \\
\textbf{Forbidden energy gap} & Energy range between valence and
conduction bands where no electron states exist \\
\textbf{Free electron} & Electron that has gained enough energy to
escape from valence band to conduction band \\
\end{longtable}
}

\textbf{Diagram:}

\includegraphics[width=1\linewidth,height=\textheight,keepaspectratio]{mermaid-908bc74a.pdf}

\end{solutionbox}
\begin{mnemonicbox}
``Very Clearly Freedom Follows'' - Valence,
Conduction, Forbidden gap, Free electrons

\end{mnemonicbox}
\subsection*{Question 2(c) OR [7
marks]}\label{q2c}

\textbf{Define Pentavalent material and give examples of it. Explain
Formation of N-type material with the help of proper diagram.}

\begin{solutionbox}

\textbf{Pentavalent material}: Elements with 5 valence electrons in
their outermost shell.

\textbf{Examples}: Phosphorus (P), Arsenic (As), Antimony (Sb)

\textbf{N-type Semiconductor Formation}:

\textbf{Diagram:}

\begin{lstlisting}
          Silicon atom (4 valence e-)    Pentavalent atom (5 valence e-)
             ┌───┐                          ┌───┐
             │   │                          │   │
          ┌──┤ Si├──┐                    ┌──┤ P ├──┐
          │  │   │  │                    │  │   │  │
       ───┼──┴───┴──┼───              ───┼──┴───┴──┼───
          │         │                    │         │
          │         │                    │         │
       ───┼─────────┼───              ───┼─────────┼───
          │         │                    │    ↓    │
          │         │                    │    │    │
       ───┴─────────┴───              ───┴────┼────┴───
                                          free electron
\end{lstlisting}

{\def\LTcaptype{none} % do not increment counter
\begin{longtable}[]{@{}
  >{\raggedright\arraybackslash}p{(\linewidth - 2\tabcolsep) * \real{0.5294}}
  >{\raggedright\arraybackslash}p{(\linewidth - 2\tabcolsep) * \real{0.4706}}@{}}
\toprule\noalign{}
\begin{minipage}[b]{\linewidth}\raggedright
Process
\end{minipage} & \begin{minipage}[b]{\linewidth}\raggedright
Result
\end{minipage} \\
\midrule\noalign{}
\endhead
\bottomrule\noalign{}
\endlastfoot
\textbf{Doping} & Silicon doped with pentavalent atoms like
Phosphorus \\
\textbf{Bond formation} & Pentavalent atoms form 4 covalent bonds with 4
neighboring Silicon atoms \\
\textbf{Free electron} & Fifth valence electron remains free (negative
charge carrier) \\
\textbf{Majority carriers} & Electrons become majority carriers \\
\textbf{Minority carriers} & Holes become minority carriers \\
\end{longtable}
}

\end{solutionbox}
\begin{mnemonicbox}
``Five Makes Negative'' - Five valence electrons make
a Negative carrier

\end{mnemonicbox}
\subsection*{Question 3(a) [3 marks]}\label{q3a}

\textbf{Define: 1. Depletion region, 2. Knee voltage, 3. Breakdown
voltage in accordance of diode.}

\begin{solutionbox}

{\def\LTcaptype{none} % do not increment counter
\begin{longtable}[]{@{}
  >{\raggedright\arraybackslash}p{(\linewidth - 2\tabcolsep) * \real{0.3333}}
  >{\raggedright\arraybackslash}p{(\linewidth - 2\tabcolsep) * \real{0.6667}}@{}}
\toprule\noalign{}
\begin{minipage}[b]{\linewidth}\raggedright
Term
\end{minipage} & \begin{minipage}[b]{\linewidth}\raggedright
Definition
\end{minipage} \\
\midrule\noalign{}
\endhead
\bottomrule\noalign{}
\endlastfoot
\textbf{Depletion region} & Region at P-N junction devoid of mobile
charge carriers due to diffusion and recombination \\
\textbf{Knee voltage} & Forward voltage at which current begins to
increase rapidly (typically 0.7V for silicon, 0.3V for germanium) \\
\textbf{Breakdown voltage} & Reverse voltage at which diode rapidly
conducts current in reverse direction \\
\end{longtable}
}

\end{solutionbox}
\begin{mnemonicbox}
``Depleted Knees Break'' - Depletion occurs, Knee
begins conduction, Breakdown ends blocking

\end{mnemonicbox}
\subsection*{Question 3(b) [4 marks]}\label{q3b}

\textbf{Explain V-I characteristics of P-N junction diode with necessary
graph.}

\begin{solutionbox}

\textbf{V-I Characteristics of P-N Junction Diode}:

\textbf{Diagram:}

\begin{lstlisting}
    I
    ↑                          
    │                        /
    │                      /
    │                    /
    │                  /
    │                /
    │              /
    │            /
    │          /
    │        /
    │      /
    │    /
    │ Knee voltage (\approx0.7V)
    │  /
    │/
────┼────────────────────────── V
    │
    │
    │
    │
    │
    │          Breakdown
    │          voltage
    │         /
    │       /
    │     /
    │   /
    v
\end{lstlisting}

{\def\LTcaptype{none} % do not increment counter
\begin{longtable}[]{@{}
  >{\raggedright\arraybackslash}p{(\linewidth - 2\tabcolsep) * \real{0.4444}}
  >{\raggedright\arraybackslash}p{(\linewidth - 2\tabcolsep) * \real{0.5556}}@{}}
\toprule\noalign{}
\begin{minipage}[b]{\linewidth}\raggedright
Region
\end{minipage} & \begin{minipage}[b]{\linewidth}\raggedright
Behavior
\end{minipage} \\
\midrule\noalign{}
\endhead
\bottomrule\noalign{}
\endlastfoot
\textbf{Forward Bias (V \textgreater{} 0)} & Current increases
exponentially after knee voltage \\
\textbf{Reverse Bias (V \textless{} 0)} & Very small leakage current
until breakdown voltage \\
\textbf{Breakdown Region} & Sharp increase in reverse current at
breakdown voltage \\
\end{longtable}
}

\begin{itemize}
\tightlist
\item
  \textbf{Forward equation}: I = Is(e\^{}(qV/nkT) - 1)
\item
  \textbf{Knee voltage}: \textasciitilde0.7V for silicon,
  \textasciitilde0.3V for germanium
\end{itemize}

\end{solutionbox}
\begin{mnemonicbox}
``Forward Flows, Reverse Restricts, Breakdown
Bursts''

\end{mnemonicbox}
\subsection*{Question 3(c) [7 marks]}\label{q3c}

\textbf{Draw characteristic of Varactor diode. Explain working of
Varactor diode with diagram and write its application.}

\begin{solutionbox}

\textbf{Varactor Diode Characteristics}:

\textbf{Diagram:}

\begin{lstlisting}
    C
    ↑                          
    │\
    │ \
    │  \
    │   \
    │    \
    │     \
    │      \
    │       \
    │        \
    │         \
    │          \
    │           \
    │            \
    │             \
    │              \
    │               \
────┼────────────────────────── VR
    │                            \rightarrow
\end{lstlisting}

\textbf{Working of Varactor Diode}:

\textbf{Circuit Symbol:}

\begin{lstlisting}
     │
    ┌┴┐
  ──┤ ├──
    └┬┘
     │
\end{lstlisting}

{\def\LTcaptype{none} % do not increment counter
\begin{longtable}[]{@{}
  >{\raggedright\arraybackslash}p{(\linewidth - 2\tabcolsep) * \real{0.4583}}
  >{\raggedright\arraybackslash}p{(\linewidth - 2\tabcolsep) * \real{0.5417}}@{}}
\toprule\noalign{}
\begin{minipage}[b]{\linewidth}\raggedright
Principle
\end{minipage} & \begin{minipage}[b]{\linewidth}\raggedright
Explanation
\end{minipage} \\
\midrule\noalign{}
\endhead
\bottomrule\noalign{}
\endlastfoot
\textbf{Basic structure} & Special P-N junction diode optimized for
variable capacitance \\
\textbf{Reverse bias operation} & Always operated in reverse bias
condition \\
\textbf{Depletion region} & Width varies with applied reverse voltage \\
\textbf{Capacitance variation} & Capacitance decreases as reverse
voltage increases \\
\textbf{Mathematical relation} & C ∝ 1/\sqrtVR where VR is reverse
voltage \\
\end{longtable}
}

\textbf{Applications of Varactor Diode}:

\begin{itemize}
\tightlist
\item
  Voltage-controlled oscillators (VCOs)
\item
  Frequency modulators
\item
  Electronic tuning circuits
\item
  Automatic frequency control circuits
\item
  Phase-locked loops (PLLs)
\end{itemize}

\end{solutionbox}
\begin{mnemonicbox}
``Capacitance Varies Reversely'' - Capacitance Varies
with Reverse voltage

\end{mnemonicbox}
\subsection*{Question 3(a) OR [3
marks]}\label{q3a}

\textbf{Write application of following diode: 1. Varactor diode, 2.
Photo diode, 3. Light Emitting Diode}

\begin{solutionbox}

{\def\LTcaptype{none} % do not increment counter
\begin{longtable}[]{@{}
  >{\raggedright\arraybackslash}p{(\linewidth - 2\tabcolsep) * \real{0.4615}}
  >{\raggedright\arraybackslash}p{(\linewidth - 2\tabcolsep) * \real{0.5385}}@{}}
\toprule\noalign{}
\begin{minipage}[b]{\linewidth}\raggedright
Diode Type
\end{minipage} & \begin{minipage}[b]{\linewidth}\raggedright
Applications
\end{minipage} \\
\midrule\noalign{}
\endhead
\bottomrule\noalign{}
\endlastfoot
\textbf{Varactor Diode} & Voltage-controlled oscillators, Frequency
modulators, Electronic tuning circuits \\
\textbf{Photo Diode} & Light sensors, Optical communication, Smoke
detectors, Camera light meters \\
\textbf{Light Emitting Diode (LED)} & Display devices, Indicators,
Lighting systems, Optical communication \\
\end{longtable}
}

\end{solutionbox}
\begin{mnemonicbox}
``Vary Photo Emit'' - Varactor varies frequency,
Photo detects light, LED emits light

\end{mnemonicbox}
\subsection*{Question 3(b) OR [4
marks]}\label{q3b}

\textbf{Explain working of P-N junction diode in forward bias and
reverse bias.}

\begin{solutionbox}

{\def\LTcaptype{none} % do not increment counter
\begin{longtable}[]{@{}
  >{\raggedright\arraybackslash}p{(\linewidth - 4\tabcolsep) * \real{0.3137}}
  >{\raggedright\arraybackslash}p{(\linewidth - 4\tabcolsep) * \real{0.3725}}
  >{\raggedright\arraybackslash}p{(\linewidth - 4\tabcolsep) * \real{0.3137}}@{}}
\toprule\noalign{}
\begin{minipage}[b]{\linewidth}\raggedright
Bias Condition
\end{minipage} & \begin{minipage}[b]{\linewidth}\raggedright
Working Principle
\end{minipage} & \begin{minipage}[b]{\linewidth}\raggedright
Characteristics
\end{minipage} \\
\midrule\noalign{}
\endhead
\bottomrule\noalign{}
\endlastfoot
\textbf{Forward Bias} & P-side connected to positive terminal, N-side to
negative terminal & Depletion region narrows, current flows easily after
knee voltage (\textasciitilde0.7V) \\
\textbf{Reverse Bias} & P-side connected to negative terminal, N-side to
positive terminal & Depletion region widens, only small leakage current
flows until breakdown \\
\end{longtable}
}

\textbf{Diagram:}

\includegraphics[width=1\linewidth,height=\textheight,keepaspectratio]{mermaid-ca6403b2.pdf}

\end{solutionbox}
\begin{mnemonicbox}
``Forward Flows, Reverse Resists''

\end{mnemonicbox}
\subsection*{Question 3(c) OR [7
marks]}\label{q3c}

\textbf{Draw characteristic of Photo diode. Explain working of Photo
diode with diagram and write its application.}

\begin{solutionbox}

\textbf{Photo Diode Characteristics}:

\textbf{Diagram:}

\begin{lstlisting}
     I
     ↑
     │                     Light intensity
     │                     increasing
     │                /
     │              /
     │            /
     │          /
     │        /
     │      /
     │    /
     │  /
     │/
─────┼─────────────────────────── V
     │\                           \rightarrow
     │ \
     │  \
     │   \
     │    \
     │     \
     │
     v
\end{lstlisting}

\textbf{Working of Photo Diode}:

\textbf{Circuit Symbol:}

\begin{lstlisting}
     │   ↙↙
    ┌┴┐ ↙
  ──┤ ├──
    └┬┘
     │
\end{lstlisting}

{\def\LTcaptype{none} % do not increment counter
\begin{longtable}[]{@{}
  >{\raggedright\arraybackslash}p{(\linewidth - 2\tabcolsep) * \real{0.4583}}
  >{\raggedright\arraybackslash}p{(\linewidth - 2\tabcolsep) * \real{0.5417}}@{}}
\toprule\noalign{}
\begin{minipage}[b]{\linewidth}\raggedright
Principle
\end{minipage} & \begin{minipage}[b]{\linewidth}\raggedright
Explanation
\end{minipage} \\
\midrule\noalign{}
\endhead
\bottomrule\noalign{}
\endlastfoot
\textbf{Basic structure} & P-N junction diode with transparent window or
lens \\
\textbf{Reverse bias operation} & Typically operated in reverse bias
condition \\
\textbf{Light absorption} & Photons create electron-hole pairs in
depletion region \\
\textbf{Carrier generation} & Light intensity proportional to generated
carriers \\
\textbf{Current generation} & Reverse current increases with light
intensity \\
\end{longtable}
}

\textbf{Applications of Photo Diode}:

\begin{itemize}
\tightlist
\item
  Light detectors in optical communication
\item
  Photometers and light meters
\item
  Smoke detectors
\item
  Barcode readers
\item
  Medical equipment (pulse oximeters)
\end{itemize}

\end{solutionbox}
\begin{mnemonicbox}
``Light In, Current Out'' - Light intensity controls
current output

\end{mnemonicbox}
\subsection*{Question 4(a) [3 marks]}\label{q4a}

\textbf{Explain working of Half wave rectifier with circuit diagram.}

\begin{solutionbox}

\textbf{Half Wave Rectifier}:

\textbf{Circuit Diagram:}

\begin{lstlisting}
           D
    AC    ┌─┬─┐     R    
    o─────┤>├─┼─────┳─────o
          └─┘ │     │     
              │     │     
    o─────────┘     ┗─────o
                      Output
\end{lstlisting}

{\def\LTcaptype{none} % do not increment counter
\begin{longtable}[]{@{}
  >{\raggedright\arraybackslash}p{(\linewidth - 2\tabcolsep) * \real{0.5667}}
  >{\raggedright\arraybackslash}p{(\linewidth - 2\tabcolsep) * \real{0.4333}}@{}}
\toprule\noalign{}
\begin{minipage}[b]{\linewidth}\raggedright
Operation Phase
\end{minipage} & \begin{minipage}[b]{\linewidth}\raggedright
Description
\end{minipage} \\
\midrule\noalign{}
\endhead
\bottomrule\noalign{}
\endlastfoot
\textbf{Positive Half Cycle} & Diode conducts, current flows through
load, output follows input \\
\textbf{Negative Half Cycle} & Diode blocks, no current flows, output is
zero \\
\end{longtable}
}

\begin{itemize}
\tightlist
\item
  \textbf{Output frequency}: Same as input frequency
\item
  \textbf{Form factor}: 1.57
\item
  \textbf{Ripple factor}: 1.21
\item
  \textbf{Efficiency}: 40.6\%
\item
  \textbf{PIV of diode}: Vmax
\end{itemize}

\end{solutionbox}
\begin{mnemonicbox}
``Half Passes Positive'' - Only positive half-cycle
passes through

\end{mnemonicbox}
\subsection*{Question 4(b) [4 marks]}\label{q4b}

\textbf{Explain Zener diode as a voltage regulator.}

\begin{solutionbox}

\textbf{Zener Diode Voltage Regulator}:

\textbf{Circuit Diagram:}

\begin{lstlisting}
           Rs             
    o─────┳─────┐           
    Vin   │     │     
          │     │   Zener   RL    Vout
          │     ├──┐┌┬┐┌───┳─────o
          │     │  ││>││   │     
          │     │  │└┬┘│   │     
    o─────┴─────┴──┴─┴─┴───┴─────o
                      
\end{lstlisting}

{\def\LTcaptype{none} % do not increment counter
\begin{longtable}[]{@{}ll@{}}
\toprule\noalign{}
Component & Function \\
\midrule\noalign{}
\endhead
\bottomrule\noalign{}
\endlastfoot
\textbf{Series resistor Rs} & Limits current and drops excess voltage \\
\textbf{Zener diode} & Maintains constant voltage across load \\
\textbf{Load resistor RL} & Represents the circuit being powered \\
\end{longtable}
}

\textbf{Working Principle}:

\begin{itemize}
\tightlist
\item
  Zener operates in reverse breakdown region
\item
  Maintains constant voltage regardless of input changes
\item
  Excess current flows through Zener diode
\item
  Voltage regulation equation: Vout = Vz (Zener voltage)
\end{itemize}

\end{solutionbox}
\begin{mnemonicbox}
``Zener Zeros Voltage Variations''

\end{mnemonicbox}
\subsection*{Question 4(c) [7 marks]}\label{q4c}

\textbf{Write need of Rectifier. Explain Bridge wave rectifier with
circuit diagram and draw its input and output waveform.}

\begin{solutionbox}

\textbf{Need of Rectifier}:

\begin{itemize}
\tightlist
\item
  To convert AC voltage to DC voltage
\item
  Most electronic devices require DC for operation
\item
  Power supply systems need DC output from AC mains
\end{itemize}

\textbf{Bridge Wave Rectifier}:

\textbf{Circuit Diagram:}

\begin{lstlisting}
                 D1      D3
              ┌─┬─┐    ┌─┬─┐
              │>├─┼────┤<├─┐
              └─┘ │    └─┘ │
    AC            │         RL   Output
    o─────────────┼────────┳────o
                  │        │
             ┌─┬─┐│    ┌─┬─┐
             │<├─┼────┤>├─┘
             └─┘      └─┘
                D2      D4
\end{lstlisting}

\textbf{Input and Output Waveform}:

\begin{lstlisting}
    Input
      ↑
      │    /\      /\      /\
      │   /  \    /  \    /  \
      │  /    \  /    \  /    \
    ──┼─┼──────┼┼──────┼┼──────┼──────► t
      │ │\    /││\    /││\    /│
      │ │ \  / ││ \  / ││ \  / │
      │ │  \/  ││  \/  ││  \/  │
      v
    
    Output
      ↑
      │    /\      /\      /\
      │   /  \    /  \    /  \
      │  /    \  /    \  /    \
    ──┼─┼──────┼┼──────┼┼──────┼──────► t
      │
      │
      v
\end{lstlisting}

{\def\LTcaptype{none} % do not increment counter
\begin{longtable}[]{@{}
  >{\raggedright\arraybackslash}p{(\linewidth - 2\tabcolsep) * \real{0.5000}}
  >{\raggedright\arraybackslash}p{(\linewidth - 2\tabcolsep) * \real{0.5000}}@{}}
\toprule\noalign{}
\begin{minipage}[b]{\linewidth}\raggedright
Working in Positive Half Cycle
\end{minipage} & \begin{minipage}[b]{\linewidth}\raggedright
Working in Negative Half Cycle
\end{minipage} \\
\midrule\noalign{}
\endhead
\bottomrule\noalign{}
\endlastfoot
D1 and D4 conduct & D2 and D3 conduct \\
Current flows through load in same direction & Current flows through
load in same direction \\
\end{longtable}
}

\begin{itemize}
\tightlist
\item
  \textbf{Output frequency}: Twice the input frequency
\item
  \textbf{Form factor}: 1.11
\item
  \textbf{Ripple factor}: 0.48
\item
  \textbf{Efficiency}: 81.2\%
\item
  \textbf{PIV of diode}: Vmax
\end{itemize}

\end{solutionbox}
\begin{mnemonicbox}
``Bridge Both Better'' - Bridge rectifier uses both
half cycles

\end{mnemonicbox}
\subsection*{Question 4(a) OR [3
marks]}\label{q4a}

\textbf{Explain working of Shunt capacitor filter.}

\begin{solutionbox}

\textbf{Shunt Capacitor Filter}:

\textbf{Circuit Diagram:}

\begin{lstlisting}
                  D
               ┌─┬─┐
               │>├─┐
               └─┘ │
    AC             │       C    RL    
    o──────────────┼───────┳────┳────o
                   │       │    │   Output
                   │       │    │
    o──────────────┴───────┴────┴────o
\end{lstlisting}

{\def\LTcaptype{none} % do not increment counter
\begin{longtable}[]{@{}
  >{\raggedright\arraybackslash}p{(\linewidth - 2\tabcolsep) * \real{0.4583}}
  >{\raggedright\arraybackslash}p{(\linewidth - 2\tabcolsep) * \real{0.5417}}@{}}
\toprule\noalign{}
\begin{minipage}[b]{\linewidth}\raggedright
Operation
\end{minipage} & \begin{minipage}[b]{\linewidth}\raggedright
Description
\end{minipage} \\
\midrule\noalign{}
\endhead
\bottomrule\noalign{}
\endlastfoot
\textbf{Charging} & Capacitor charges during peak of rectified output \\
\textbf{Discharging} & Capacitor discharges slowly through load when
voltage drops \\
\textbf{Smoothing effect} & Provides almost constant DC output by
filling gaps \\
\end{longtable}
}

\begin{itemize}
\tightlist
\item
  \textbf{Ripple reduction}: Significant reduction in ripple voltage
\item
  \textbf{Time constant}: RC must be much larger than period of input
\item
  \textbf{Discharge equation}: V = V_{0}e\^{}(-t/RC)
\end{itemize}

\end{solutionbox}
\begin{mnemonicbox}
``Capacitor Catches Peaks'' - Capacitor stores peak
voltage

\end{mnemonicbox}
\subsection*{Question 4(b) OR [4
marks]}\label{q4b}

\textbf{Compare Center tap full wave rectifier and Bridge wave
rectifier}

\begin{solutionbox}

{\def\LTcaptype{none} % do not increment counter
\begin{longtable}[]{@{}
  >{\raggedright\arraybackslash}p{(\linewidth - 4\tabcolsep) * \real{0.1642}}
  >{\raggedright\arraybackslash}p{(\linewidth - 4\tabcolsep) * \real{0.4776}}
  >{\raggedright\arraybackslash}p{(\linewidth - 4\tabcolsep) * \real{0.3582}}@{}}
\toprule\noalign{}
\begin{minipage}[b]{\linewidth}\raggedright
Parameter
\end{minipage} & \begin{minipage}[b]{\linewidth}\raggedright
Center Tap Full Wave Rectifier
\end{minipage} & \begin{minipage}[b]{\linewidth}\raggedright
Bridge Wave Rectifier
\end{minipage} \\
\midrule\noalign{}
\endhead
\bottomrule\noalign{}
\endlastfoot
\textbf{Number of diodes} & 2 & 4 \\
\textbf{Transformer} & Center-tapped transformer required & Simple
transformer sufficient \\
\textbf{PIV of diode} & 2Vmax & Vmax \\
\textbf{Efficiency} & 81.2\% & 81.2\% \\
\textbf{Output frequency} & Twice input frequency & Twice input
frequency \\
\textbf{Cost} & Higher due to center-tapped transformer & Lower, simpler
transformer but more diodes \\
\textbf{Size} & Larger & Smaller \\
\end{longtable}
}

\end{solutionbox}
\begin{mnemonicbox}
``Center Taps Transformer, Bridge Bypasses Tapping''

\end{mnemonicbox}
\subsection*{Question 4(c) OR [7
marks]}\label{q4c}

\textbf{Write need of Filter circuit in rectifier. Explain π filter with
circuit diagram and draw its input and output waveform.}

\begin{solutionbox}

\textbf{Need of Filter Circuit in Rectifier}:

\begin{itemize}
\tightlist
\item
  Reduces ripple in rectified output
\item
  Provides steady DC voltage required by electronic circuits
\item
  Improves efficiency of power supply
\item
  Prevents damage to sensitive electronic components
\end{itemize}

\textbf{π Filter}:

\textbf{Circuit Diagram:}

\begin{lstlisting}
                  D
               ┌─┬─┐      L
               │>├─┼──────┳──────┐
               └─┘ │      │      │
    AC             │      │      │
    o──────────────┼──────┘      │
                   │             │
                   │  C1    C2   │  RL    
    o──────────────┴───┳────┳────┴───┳────o
                       │    │        │   Output
                       │    │        │
                       ┴────┴────────┴────o
\end{lstlisting}

\textbf{Input and Output Waveform}:

\begin{lstlisting}
    Input (Rectified)
      ↑
      │    /\      /\      /\
      │   /  \    /  \    /  \
      │  /    \  /    \  /    \
    ──┼─┼──────┼┼──────┼┼──────┼──────► t
      │
      │
      v
    
    Output
      ↑
      │─────────────────────────────
      │
      │
    ──┼─────────────────────────────► t
      │
      │
      v
\end{lstlisting}

{\def\LTcaptype{none} % do not increment counter
\begin{longtable}[]{@{}ll@{}}
\toprule\noalign{}
Component & Function \\
\midrule\noalign{}
\endhead
\bottomrule\noalign{}
\endlastfoot
\textbf{Input capacitor (C1)} & Initial filtering of rectified output \\
\textbf{Choke (L)} & Blocks AC ripple and allows DC to pass \\
\textbf{Output capacitor (C2)} & Further filtering for smoother
output \\
\end{longtable}
}

\begin{itemize}
\tightlist
\item
  \textbf{Superior filtering}: Better ripple reduction than simple
  capacitor filter
\item
  \textbf{Ripple factor}: Much lower than capacitor filter alone
\item
  \textbf{Voltage regulation}: Better voltage regulation under load
  variations
\end{itemize}

\end{solutionbox}
\begin{mnemonicbox}
``Capacitor-Inductor-Capacitor Perfectly Irons'' (π
shape resembling CIC filter)

\end{mnemonicbox}
\subsection*{Question 5(a) [3 marks]}\label{q5a}

\textbf{Explain Working of PNP Transistor with the necessary diagram.}

\begin{solutionbox}

\textbf{PNP Transistor Working}:

\textbf{Diagram:}

\begin{lstlisting}
              Collector
                  ↑
                  │
                  P
              ┌───┴───┐
    Base \rightarrow    │       │
              N       │
              │       │
              P       │
                  │
                  ↓
              Emitter
\end{lstlisting}

{\def\LTcaptype{none} % do not increment counter
\begin{longtable}[]{@{}ll@{}}
\toprule\noalign{}
Biasing & Working \\
\midrule\noalign{}
\endhead
\bottomrule\noalign{}
\endlastfoot
\textbf{Base-Emitter junction} & Forward biased \\
\textbf{Base-Collector junction} & Reverse biased \\
\textbf{Majority carriers} & Holes \\
\textbf{Current flow} & Emitter to Collector \\
\end{longtable}
}

\begin{itemize}
\tightlist
\item
  \textbf{Emitter}: Heavily doped P-region that emits holes
\item
  \textbf{Base}: Thin, lightly doped N-region that controls current flow
\item
  \textbf{Collector}: Moderately doped P-region that collects holes
\end{itemize}

\end{solutionbox}
\begin{mnemonicbox}
``Positive-Negative-Positive'' - PNP structure

\end{mnemonicbox}
\subsection*{Question 5(b) [4 marks]}\label{q5b}

\textbf{Explain working of N-channel JFET with diagram.}

\begin{solutionbox}

\textbf{N-channel JFET Working}:

\textbf{Diagram:}

\begin{lstlisting}
                  Drain
                    ↑
                    │
           ┌────────┴────────┐
           │                 │
    Gate \rightarrow  P               P  \leftarrow Gate
           │                 │
           │        N        │
           │                 │
           └────────┬────────┘
                    │
                    ↓
                  Source
\end{lstlisting}

{\def\LTcaptype{none} % do not increment counter
\begin{longtable}[]{@{}ll@{}}
\toprule\noalign{}
Terminal & Function \\
\midrule\noalign{}
\endhead
\bottomrule\noalign{}
\endlastfoot
\textbf{Source} & Source of charge carriers (electrons) \\
\textbf{Drain} & Collects charge carriers \\
\textbf{Gate} & Controls width of the channel \\
\end{longtable}
}

\textbf{Working Principle}:

\begin{itemize}
\tightlist
\item
  Channel formed by N-type material between source and drain
\item
  P-type gate regions form PN junctions with channel
\item
  Gate-to-source junction always reverse biased
\item
  Increasing negative gate voltage widens depletion region
\item
  Narrower channel increases resistance between source and drain
\item
  FET operates as voltage-controlled resistor
\end{itemize}

\end{solutionbox}
\begin{mnemonicbox}
``Negative Channel Junction Effect'' - N-channel JFET

\end{mnemonicbox}
\subsection*{Question 5(c) [7 marks]}\label{q5c}

\textbf{Compare BJT and JFET}

\begin{solutionbox}

{\def\LTcaptype{none} % do not increment counter
\begin{longtable}[]{@{}
  >{\raggedright\arraybackslash}p{(\linewidth - 4\tabcolsep) * \real{0.1264}}
  >{\raggedright\arraybackslash}p{(\linewidth - 4\tabcolsep) * \real{0.4023}}
  >{\raggedright\arraybackslash}p{(\linewidth - 4\tabcolsep) * \real{0.4713}}@{}}
\toprule\noalign{}
\begin{minipage}[b]{\linewidth}\raggedright
Parameter
\end{minipage} & \begin{minipage}[b]{\linewidth}\raggedright
BJT (Bipolar Junction Transistor)
\end{minipage} & \begin{minipage}[b]{\linewidth}\raggedright
JFET (Junction Field Effect Transistor)
\end{minipage} \\
\midrule\noalign{}
\endhead
\bottomrule\noalign{}
\endlastfoot
\textbf{Structure} & Three-layer structure (NPN or PNP) & Single channel
with gate junctions \\
\textbf{Control mechanism} & Current-controlled device &
Voltage-controlled device \\
\textbf{Carriers} & Both majority and minority carriers (bipolar) & Only
majority carriers (unipolar) \\
\textbf{Input impedance} & Low to medium (1-10 kΩ) & Very high (10^{8}-10^{1}^{2}
Ω) \\
\textbf{Noise} & Higher noise & Lower noise \\
\textbf{Power consumption} & Higher & Lower \\
\textbf{Switching speed} & Slower due to charge storage & Faster due to
absence of charge storage \\
\textbf{Temperature stability} & Less stable & More stable \\
\end{longtable}
}

\textbf{Diagram:}

\includegraphics[width=1\linewidth,height=\textheight,keepaspectratio]{mermaid-14e55487.pdf}

\end{solutionbox}
\begin{mnemonicbox}
``Current Bipolar Low, Voltage Unipolar High'' - BJT
vs JFET key differences

\end{mnemonicbox}
\subsection*{Question 5(a) OR [3
marks]}\label{q5a}

\textbf{Enlist methods to dispose E-waste and explain any one method of
them.}

\begin{solutionbox}

{\def\LTcaptype{none} % do not increment counter
\begin{longtable}[]{@{}l@{}}
\toprule\noalign{}
E-waste Disposal Methods \\
\midrule\noalign{}
\endhead
\bottomrule\noalign{}
\endlastfoot
\textbf{Recycling} \\
\textbf{Reuse} \\
\textbf{Incineration} \\
\textbf{Landfilling} \\
\textbf{Take-back systems} \\
\end{longtable}
}

\textbf{Explanation of Recycling}: E-waste recycling involves
collecting, dismantling, and separating electronic waste into
recoverable materials. Components are shredded and sorted into raw
materials like plastic, glass, and metals (including precious metals
like gold, silver, copper). These materials are then processed and can
be used to manufacture new products. Recycling reduces environmental
impact, conserves resources, and recovers valuable materials.

\end{solutionbox}
\begin{mnemonicbox}
``RRIL-T'' - Recycling, Reuse, Incineration,
Landfill, Take-back

\end{mnemonicbox}
\subsection*{Question 5(b) OR [4
marks]}\label{q5b}

\textbf{Compare PNP and NPN Transistor.}

\begin{solutionbox}

{\def\LTcaptype{none} % do not increment counter
\begin{longtable}[]{@{}
  >{\raggedright\arraybackslash}p{(\linewidth - 4\tabcolsep) * \real{0.2619}}
  >{\raggedright\arraybackslash}p{(\linewidth - 4\tabcolsep) * \real{0.3571}}
  >{\raggedright\arraybackslash}p{(\linewidth - 4\tabcolsep) * \real{0.3810}}@{}}
\toprule\noalign{}
\begin{minipage}[b]{\linewidth}\raggedright
Parameter
\end{minipage} & \begin{minipage}[b]{\linewidth}\raggedright
PNP Transistor
\end{minipage} & \begin{minipage}[b]{\linewidth}\raggedright
NPN Transistor
\end{minipage} \\
\midrule\noalign{}
\endhead
\bottomrule\noalign{}
\endlastfoot
\textbf{Symbol} & Arrow points inward to base & Arrow points outward
from base \\
\textbf{Structure} & P-type, N-type, P-type layers & N-type, P-type,
N-type layers \\
\textbf{Majority carriers} & Holes & Electrons \\
\textbf{Biasing voltage} & Base negative with respect to emitter & Base
positive with respect to emitter \\
\textbf{Current direction} & Emitter to collector & Collector to
emitter \\
\textbf{Speed} & Slower (holes mobility is less) & Faster (electrons
mobility is more) \\
\end{longtable}
}

\textbf{Diagram:}

\includegraphics[width=1\linewidth,height=\textheight,keepaspectratio]{mermaid-ef7d5744.pdf}

\end{solutionbox}
\begin{mnemonicbox}
``Positive-Negative-Positive (Holes),
Negative-Positive-Negative (Electrons)''

\end{mnemonicbox}
\subsection*{Question 5(c) OR [7
marks]}\label{q5c}

\textbf{Draw and explain Input and Output Characteristics of CE
configuration.}

\begin{solutionbox}

\textbf{Input Characteristics of CE Configuration}:

\textbf{Diagram:}

\begin{lstlisting}
   Ib(μA)
    ↑                          
    │                       VCE=10V
    │                     /
    │                   /
    │                 / VCE=5V
    │               /
    │             /
    │           / VCE=0V
    │         /
    │       /
    │     /
    │   /
    │ /
    │/
────┼───────────────────────── VBE(V)
    │                           \rightarrow
\end{lstlisting}

\textbf{Output Characteristics of CE Configuration}:

\textbf{Diagram:}

\begin{lstlisting}
    Ic(mA)
    ↑                          
    │                  Ib=50μA
    │                /─────────────
    │               /
    │              / Ib=40μA
    │             /─────────────
    │            /
    │           / Ib=30μA
    │          /─────────────
    │         /
    │        / Ib=20μA
    │       /─────────────
    │      /
    │     / Ib=10μA
    │    /─────────────
    │   /
    │  / Ib=0
    │ /
    │/
────┼───────────────────────── VCE(V)
    │                           \rightarrow
    │
    │ Active    |  Saturation
    │ Region    |  Region
    v           v
\end{lstlisting}

{\def\LTcaptype{none} % do not increment counter
\begin{longtable}[]{@{}
  >{\raggedright\arraybackslash}p{(\linewidth - 2\tabcolsep) * \real{0.5517}}
  >{\raggedright\arraybackslash}p{(\linewidth - 2\tabcolsep) * \real{0.4483}}@{}}
\toprule\noalign{}
\begin{minipage}[b]{\linewidth}\raggedright
Characteristic
\end{minipage} & \begin{minipage}[b]{\linewidth}\raggedright
Description
\end{minipage} \\
\midrule\noalign{}
\endhead
\bottomrule\noalign{}
\endlastfoot
\textbf{Input Characteristics} & Relationship between base current (IB)
and base-emitter voltage (VBE) at constant collector-emitter voltage
(VCE) \\
\textbf{Output Characteristics} & Relationship between collector current
(IC) and collector-emitter voltage (VCE) at constant base current
(IB) \\
\end{longtable}
}

\textbf{Regions in Output Characteristics}:

{\def\LTcaptype{none} % do not increment counter
\begin{longtable}[]{@{}
  >{\raggedright\arraybackslash}p{(\linewidth - 2\tabcolsep) * \real{0.3810}}
  >{\raggedright\arraybackslash}p{(\linewidth - 2\tabcolsep) * \real{0.6190}}@{}}
\toprule\noalign{}
\begin{minipage}[b]{\linewidth}\raggedright
Region
\end{minipage} & \begin{minipage}[b]{\linewidth}\raggedright
Description
\end{minipage} \\
\midrule\noalign{}
\endhead
\bottomrule\noalign{}
\endlastfoot
\textbf{Saturation Region} & Both junctions forward biased, VCE is
small, IC is almost constant regardless of VCE \\
\textbf{Active Region} & Base-emitter junction forward biased,
base-collector junction reverse biased, IC proportional to IB \\
\textbf{Cutoff Region} & Both junctions reverse biased, negligible
current flows \\
\end{longtable}
}

\textbf{Important Parameters}:

\begin{itemize}
\tightlist
\item
  \textbf{Current gain (β)}: Ratio of collector current to base current
  (IC/IB)
\item
  \textbf{Input resistance}: Ratio of change in VBE to change in IB at
  constant VCE
\item
  \textbf{Output resistance}: Ratio of change in VCE to change in IC at
  constant IB
\end{itemize}

\end{solutionbox}
\begin{mnemonicbox}
``Input Shows Voltage Effects, Output Shows Current
Control''

\end{mnemonicbox}

\end{document}
