\documentclass[10pt,a4paper]{article}

% content/resources/templates/preamble.tex
\usepackage[margin=0.6in]{geometry}
\author{Milav Dabgar}
\usepackage{amsmath,amssymb,amsthm}
\usepackage{booktabs}
\usepackage{multirow}
\usepackage{xcolor}
\usepackage{tcolorbox}
\tcbuselibrary{breakable,skins}
\usepackage[colorlinks=true,linkcolor=blue]{hyperref}
\usepackage{titlesec}
\usepackage{enumitem}
\usepackage{tikz}
\usepackage{pgfplots}
\usepackage{circuitikz}
\usepackage[version=4]{mhchem}
\usepackage{longtable}
\usepackage{array}
\usepackage{float}
\usepackage{caption}
\usepackage{listings}

\lstset{
  basicstyle=\small\ttfamily,
  breaklines=true,
  breakatwhitespace=false,
  postbreak=\mbox{\textcolor{red}{$\hookrightarrow$}\space},
  float=false,
  numbers=left,
  numberstyle=\tiny\color{gray},
  numbersep=10pt,
  xleftmargin=2em,
  keywordstyle=\color{blue},
  commentstyle=\color{green!60!black},
  stringstyle=\color{purple},
  backgroundcolor=\color{gray!5},
  showstringspaces=false,
  tabsize=2,
  captionpos=b,
  keepspaces=true,
  columns=flexible
}

\pgfplotsset{compat=1.18}
\usetikzlibrary{shapes,arrows,positioning,calc,patterns,decorations.pathmorphing,decorations.markings,arrows.meta}

% Color scheme
\definecolor{headcolor}{RGB}{0,102,204}
\definecolor{keycolor}{RGB}{220,20,60}
\definecolor{solutioncolor}{RGB}{34,139,34}
\definecolor{mnemoniccolor}{RGB}{148,0,211}
\definecolor{codecolor}{RGB}{0,0,100}

% Spacing
\setlength{\parskip}{3pt}
\setlist[itemize]{nosep}
\setlist[enumerate]{nosep}

% Title formatting
\titleformat{\section}{\Large\bfseries\color{headcolor}}{\thesection}{1em}{}
\titleformat{\subsection}{\large\bfseries\color{headcolor}}{\thesubsection}{1em}{}

% Pandoc tightlist compatibility
\providecommand{\tightlist}{%
  \setlength{\itemsep}{0pt}\setlength{\parskip}{0pt}}

% Pandoc longtable compatibility
\newcounter{none}
\def\thenone{}


% content/resources/templates/english-boxes.tex
% This file is currently empty - it exists to maintain consistency with the import structure.
% Add custom environments here if needed in the future.


\begin{document}

\begin{center}
{\Huge\bfseries\color{headcolor} Subject Name Solutions}\\[5pt]
{\LARGE 1313202 -- Summer 2023}\\[3pt]
{\large Semester 1 Study Material}\\[3pt]
{\normalsize\textit{Detailed Solutions and Explanations}}
\end{center}

\vspace{10pt}

\subsection*{Question 1(a) [3 marks]}\label{q1a}

\textbf{Find mesh currents in following circuit.}

\begin{solutionbox}

\textbf{Diagram/Table:}

\begin{lstlisting}
    2kΩ      2kΩ
    ┌───┐    ┌───┐
    │   │    │   │
    │   │    │   │
┌───┴───┴────┴───┴───┐
│   │              │ │
│  ┌┴┐             ┌┴┐
│  │ │   1kΩ       │ │
5V ┤ ├─────────────┤ ├ 2V
│  │ │   │         │ │
│  └┬┘   │         └┬┘
│   │    │          │ │
└───┴────┴──────────┴─┘
\end{lstlisting}

Applying Mesh Analysis:

\begin{itemize}
\tightlist
\item
  Write KVL equations for two meshes
\item
  I_{1} flows clockwise in left loop
\item
  I_{2} flows clockwise in right loop
\end{itemize}

\textbf{Steps to solve:}

\begin{itemize}
\tightlist
\item
  \textbf{Mesh 1 equation}: 5V - 2kΩ\timesI_{1} - 1kΩ\times(I_{1}-I_{2}) = 0
\item
  \textbf{Mesh 2 equation}: -2V + 2kΩ\timesI_{2} + 1kΩ\times(I_{2}-I_{1}) = 0
\end{itemize}

Simplifying:

\begin{itemize}
\item
  5 - 2000I_{1} - 1000I_{1} + 1000I_{2} = 0
\item
  -2 + 2000I_{2} + 1000I_{2} - 1000I_{1} = 0
\item
  3000I_{1} - 1000I_{2} = 5
\item
  -1000I_{1} + 3000I_{2} = 2
\end{itemize}

Solving: I_{1} = 2 mA I_{2} = 1 mA

\end{solutionbox}
\begin{mnemonicbox}
``Mesh Matters: Write KVL, Solve Simultaneous''

\end{mnemonicbox}
\subsection*{Question 1(b) [4 marks]}\label{q1b}

\textbf{State and explain Kirchhoff's Voltage Law (KVL) with the help of
diagram.}

\begin{solutionbox}

Kirchhoff's Voltage Law (KVL) states that the algebraic sum of all
voltages around any closed loop in a circuit is zero.

\textbf{Diagram:}

\includegraphics[width=1\linewidth,height=\textheight,keepaspectratio]{mermaid-a9a0e9f2.pdf}

\textbf{Key points:}

\begin{itemize}
\tightlist
\item
  \textbf{Loop rule}: V_{1} + V_{2} + V_{3} + V_{4} = 0
\item
  \textbf{Sign convention}: Voltage rise (battery positive terminal) is
  positive, voltage drop (across resistor) is negative
\item
  \textbf{Conservation principle}: Total energy gained equals total
  energy lost in any closed loop
\item
  \textbf{Application}: Used to analyze and solve complex circuits with
  multiple voltage sources
\end{itemize}

\end{solutionbox}
\begin{mnemonicbox}
``Voltages Around a Loop Sum to Zero'' (VALSZ)

\end{mnemonicbox}
\subsection*{Question 1(c) [7 marks]}\label{q1c}

\textbf{State and explain Superposition theorem.}

\begin{solutionbox}

Superposition theorem states that in a linear circuit with multiple
sources, the response in any element is the sum of responses caused by
each source acting alone, with all other sources replaced by their
internal impedances.

\textbf{Diagram:}

\includegraphics[width=1\linewidth,height=\textheight,keepaspectratio]{mermaid-25fcc738.pdf}

\textbf{Steps to apply:}

\begin{itemize}
\tightlist
\item
  \textbf{Step 1}: Consider one source at a time
\item
  \textbf{Step 2}: Replace voltage sources with short circuits (0Ω)
\item
  \textbf{Step 3}: Replace current sources with open circuits (\inftyΩ)
\item
  \textbf{Step 4}: Calculate the response (voltage/current) due to each
  source
\item
  \textbf{Step 5}: Add all responses algebraically to get total response
\end{itemize}

\textbf{Applications:}

\begin{itemize}
\tightlist
\item
  \textbf{Circuit analysis}: Simplifies complex circuits with multiple
  sources
\item
  \textbf{Network theory}: Foundation for more advanced analysis methods
\item
  \textbf{Practical circuits}: Analyzing superimposed signals in
  communication systems
\end{itemize}

\end{solutionbox}
\begin{mnemonicbox}
``Sources Separately, Sum Successfully'' (SSSS)

\end{mnemonicbox}
\subsection*{Question 1(c) OR [7
marks]}\label{q1c}

\textbf{State and explain Thevenin's theorem.}

\begin{solutionbox}

Thevenin's theorem states that any linear circuit with voltage and
current sources can be replaced by an equivalent circuit consisting of a
voltage source (VTH) in series with a resistance (RTH).

\textbf{Diagram:}

\includegraphics[width=1\linewidth,height=\textheight,keepaspectratio]{mermaid-1692ceda.pdf}

\textbf{Steps to find Thevenin equivalent:}

\begin{itemize}
\tightlist
\item
  \textbf{Step 1}: Remove load resistor from original circuit
\item
  \textbf{Step 2}: Calculate open-circuit voltage (VOC) across load
  terminals (= VTH)
\item
  \textbf{Step 3}: Calculate equivalent resistance (RTH) by:

  \begin{itemize}
  \tightlist
  \item
    Deactivating all sources (replacing voltage sources with short
    circuits and current sources with open circuits)
  \item
    Finding resistance between load terminals
  \end{itemize}
\end{itemize}

\textbf{Applications:}

\begin{itemize}
\tightlist
\item
  \textbf{Circuit simplification}: Reduces complex networks to simple
  equivalent
\item
  \textbf{Load analysis}: Easily calculate effects of changing loads
\item
  \textbf{Maximum power transfer}: Determine conditions for maximum
  power
\end{itemize}

\end{solutionbox}
\begin{mnemonicbox}
``Two Handy Elements: Voltage and Resistance''
(THEVR)

\end{mnemonicbox}
\subsection*{Question 2(a) [3 marks]}\label{q2a}

\textbf{Give comparison of trivalent, tetravalent and pentavalent
materials.}

\begin{solutionbox}

{\def\LTcaptype{none} % do not increment counter
\begin{longtable}[]{@{}
  >{\raggedright\arraybackslash}p{(\linewidth - 6\tabcolsep) * \real{0.1333}}
  >{\raggedright\arraybackslash}p{(\linewidth - 6\tabcolsep) * \real{0.2800}}
  >{\raggedright\arraybackslash}p{(\linewidth - 6\tabcolsep) * \real{0.2933}}
  >{\raggedright\arraybackslash}p{(\linewidth - 6\tabcolsep) * \real{0.2933}}@{}}
\toprule\noalign{}
\begin{minipage}[b]{\linewidth}\raggedright
Property
\end{minipage} & \begin{minipage}[b]{\linewidth}\raggedright
Trivalent Materials
\end{minipage} & \begin{minipage}[b]{\linewidth}\raggedright
Tetravalent Materials
\end{minipage} & \begin{minipage}[b]{\linewidth}\raggedright
Pentavalent Materials
\end{minipage} \\
\midrule\noalign{}
\endhead
\bottomrule\noalign{}
\endlastfoot
\textbf{Valence electrons} & 3 & 4 & 5 \\
\textbf{Examples} & Boron, Aluminum, Gallium & Silicon, Germanium,
Carbon & Phosphorus, Arsenic, Antimony \\
\textbf{Doping type} & Used as P-type dopants & Base semiconductor
materials & Used as N-type dopants \\
\textbf{Bond formation} & Makes 3 covalent bonds & Makes 4 covalent
bonds & Makes 5 covalent bonds \\
\textbf{Charge carriers} & Creates holes (positive) & Creates balanced
structure & Creates free electrons (negative) \\
\end{longtable}
}

\end{solutionbox}
\begin{mnemonicbox}
``Three-Four-Five: Holes-Balance-Electrons''
(TFF:HBE)

\end{mnemonicbox}
\subsection*{Question 2(b) [4 marks]}\label{q2b}

\textbf{State and explain Kirchhoff's Current Law (KCL) with the help of
diagram.}

\begin{solutionbox}

Kirchhoff's Current Law (KCL) states that the algebraic sum of all
currents entering and leaving any node in an electrical circuit is zero.

\textbf{Diagram:}

\includegraphics[width=1\linewidth,height=\textheight,keepaspectratio]{mermaid-4bbfce1a.pdf}

\textbf{Key points:}

\begin{itemize}
\tightlist
\item
  \textbf{Node equation}: I_{1} + I_{2} - I_{3} - I_{4} - I_{5} = 0 (or I_{1} + I_{2} = I_{3} +
  I_{4} + I_{5})
\item
  \textbf{Sign convention}: Currents entering node are positive, leaving
  are negative
\item
  \textbf{Conservation principle}: Based on conservation of electric
  charge
\item
  \textbf{Application}: Essential for solving circuits with parallel
  components
\end{itemize}

\end{solutionbox}
\begin{mnemonicbox}
``Currents In Equals Currents Out'' (CIECO)

\end{mnemonicbox}
\subsection*{Question 2(c) [7 marks]}\label{q2c}

\textbf{Define: Extrinsic Semiconductor. Explain formation of N-type
Semiconductor with the help of diagram.}

\begin{solutionbox}

\textbf{Extrinsic Semiconductor}: A semiconductor whose electrical
properties are modified by adding impurity atoms (doping) to change its
conductivity.

\textbf{N-type Semiconductor Formation:}

\textbf{Diagram:}

\includegraphics[width=1\linewidth,height=\textheight,keepaspectratio]{mermaid-26f9c764.pdf}

\textbf{Process:}

\begin{itemize}
\tightlist
\item
  \textbf{Doping process}: Pentavalent impurity (P, As, Sb) added to
  tetravalent semiconductor (Si, Ge)
\item
  \textbf{Bond formation}: Impurity atom forms 4 covalent bonds with
  neighboring Si atoms
\item
  \textbf{Free electron}: 5th electron has no bond to form and becomes
  free to move
\item
  \textbf{Charge carriers}: Majority carriers are electrons, minority
  carriers are holes
\item
  \textbf{Conductivity}: Higher than intrinsic semiconductor due to more
  free electrons
\end{itemize}

\textbf{Properties of N-type semiconductor:}

\begin{itemize}
\tightlist
\item
  \textbf{Fermi level}: Closer to conduction band
\item
  \textbf{Donor level}: Energy level created near conduction band
\item
  \textbf{Room temperature}: Most donor atoms are ionized
\end{itemize}

\end{solutionbox}
\begin{mnemonicbox}
``Phosphorus Provides Plus-one electron'' (PPP)

\end{mnemonicbox}
\subsection*{Question 2(a) OR [3
marks]}\label{q2a}

\textbf{Draw energy band diagrams for Conductor, Semiconductor and
Insulator.}

\begin{solutionbox}

\textbf{Diagram:}

\includegraphics[width=1\linewidth,height=\textheight,keepaspectratio]{mermaid-fa19e8c5.pdf}

\textbf{Key characteristics:}

\begin{itemize}
\tightlist
\item
  \textbf{Conductor}: Overlapping bands or partially filled band
\item
  \textbf{Semiconductor}: Small energy gap (\textasciitilde1 eV)
\item
  \textbf{Insulator}: Large energy gap (\textgreater5 eV)
\end{itemize}

\end{solutionbox}
\begin{mnemonicbox}
``Gaps Determine Flow: None, Small, Huge'' (GDF:NSH)

\end{mnemonicbox}
\subsection*{Question 2(b) OR [4
marks]}\label{q2b}

\textbf{Give the difference between EMF and Potential difference.}

\begin{solutionbox}

{\def\LTcaptype{none} % do not increment counter
\begin{longtable}[]{@{}
  >{\raggedright\arraybackslash}p{(\linewidth - 4\tabcolsep) * \real{0.1833}}
  >{\raggedright\arraybackslash}p{(\linewidth - 4\tabcolsep) * \real{0.4500}}
  >{\raggedright\arraybackslash}p{(\linewidth - 4\tabcolsep) * \real{0.3667}}@{}}
\toprule\noalign{}
\begin{minipage}[b]{\linewidth}\raggedright
Parameter
\end{minipage} & \begin{minipage}[b]{\linewidth}\raggedright
EMF (Electromotive Force)
\end{minipage} & \begin{minipage}[b]{\linewidth}\raggedright
Potential Difference
\end{minipage} \\
\midrule\noalign{}
\endhead
\bottomrule\noalign{}
\endlastfoot
\textbf{Definition} & Energy supplied per unit charge by a source &
Energy consumed per unit charge in a component \\
\textbf{Symbol \& Unit} & ξ or E, measured in Volts & V, measured in
Volts \\
\textbf{Cause} & Chemical, mechanical, thermal or light energy
conversion & Result of current flowing through a resistance \\
\textbf{Measurement} & Measured across source terminals with no current
flowing & Measured across components when current flows \\
\textbf{Direction} & From negative to positive inside source & From
positive to negative outside source \\
\textbf{Device example} & Battery, generator, solar cell & Resistor,
lamp, motor \\
\textbf{Conservation} & Not conserved in a circuit & Conserved in a
closed circuit (KVL) \\
\end{longtable}
}

\end{solutionbox}
\begin{mnemonicbox}
``EMF Creates, PD Consumes'' (ECPC)

\end{mnemonicbox}
\subsection*{Question 2(c) OR [7
marks]}\label{q2c}

\textbf{Explain the formation of depletion region or space-charge region
in P-N junction.}

\begin{solutionbox}

\textbf{Diagram:}

\includegraphics[width=1\linewidth,height=\textheight,keepaspectratio]{mermaid-0fcb8259.pdf}

\textbf{Formation process:}

\begin{itemize}
\tightlist
\item
  \textbf{Junction creation}: When P-type and N-type semiconductors are
  joined
\item
  \textbf{Diffusion}: Free electrons from N-side diffuse to P-side;
  holes from P-side diffuse to N-side
\item
  \textbf{Recombination}: Electrons recombine with holes near junction
\item
  \textbf{Ion formation}: Immobile positive ions left in N-region;
  negative ions in P-region
\item
  \textbf{Electric field}: Created across the junction pointing from N
  to P
\item
  \textbf{Equilibrium}: Diffusion current balanced by drift current due
  to electric field
\item
  \textbf{Barrier potential}: Typically 0.7V for silicon, 0.3V for
  germanium
\end{itemize}

\textbf{Characteristics:}

\begin{itemize}
\tightlist
\item
  \textbf{Width}: Typically 0.5 μm, depends on doping concentration
\item
  \textbf{Capacitance}: Acts as variable capacitor
\item
  \textbf{Barrier}: Prevents further diffusion of majority carriers
\end{itemize}

\end{solutionbox}
\begin{mnemonicbox}
``Diffusion Creates, Field Balances'' (DCFB)

\end{mnemonicbox}
\subsection*{Question 3(a) [3 marks]}\label{q3a}

\textbf{Define forbidden energy gap. How does it occur? What is its
magnitude for Ge and Si?}

\begin{solutionbox}

\textbf{Forbidden energy gap} is the energy range between valence band
and conduction band where no electron energy states exist in a
semiconductor.

\textbf{Occurrence:}

\begin{itemize}
\tightlist
\item
  Results from quantum mechanical interaction of atoms in crystal
  lattice
\item
  Forms due to splitting of energy levels when atoms are brought close
  together
\item
  Creates band structure with allowed and forbidden regions
\end{itemize}

\textbf{Magnitude:}

\begin{itemize}
\tightlist
\item
  \textbf{Germanium (Ge)}: 0.67 eV at 300K
\item
  \textbf{Silicon (Si)}: 1.1 eV at 300K
\end{itemize}

\end{solutionbox}
\begin{mnemonicbox}
``Greater Silicon, Lower Germanium'' (GSLG)

\end{mnemonicbox}
\subsection*{Question 3(b) [4 marks]}\label{q3b}

\textbf{Define the following terms:} \textbf{(i) Knee voltage (ii)
Reverse saturation current (iii) Reverse breakdown voltage (iv) Peak
Inverse Voltage (PIV)}

\begin{solutionbox}

\textbf{Table of Definitions:}

{\def\LTcaptype{none} % do not increment counter
\begin{longtable}[]{@{}
  >{\raggedright\arraybackslash}p{(\linewidth - 2\tabcolsep) * \real{0.3333}}
  >{\raggedright\arraybackslash}p{(\linewidth - 2\tabcolsep) * \real{0.6667}}@{}}
\toprule\noalign{}
\begin{minipage}[b]{\linewidth}\raggedright
Term
\end{minipage} & \begin{minipage}[b]{\linewidth}\raggedright
Definition
\end{minipage} \\
\midrule\noalign{}
\endhead
\bottomrule\noalign{}
\endlastfoot
\textbf{Knee voltage} & The forward voltage at which current through
diode starts increasing rapidly (0.3V for Ge, 0.7V for Si) \\
\textbf{Reverse saturation current} & The small current that flows when
diode is reverse biased, due to minority carriers (typically nA or
μA) \\
\textbf{Reverse breakdown voltage} & The reverse voltage at which the
diode conducts heavily in reverse direction due to breakdown
mechanisms \\
\textbf{Peak Inverse Voltage (PIV)} & Maximum reverse voltage a diode
can withstand without breakdown in a rectifier circuit \\
\end{longtable}
}

\end{solutionbox}
\begin{mnemonicbox}
``Knee Rises, Saturation Trickles, Breakdown Bursts,
PIV Protects'' (KRSBBP)

\end{mnemonicbox}
\subsection*{Question 3(c) [7 marks]}\label{q3c}

\textbf{Explain construction, working and characteristics of LASER diode
and write its applications.}

\begin{solutionbox}

\textbf{Diagram:}

\begin{lstlisting}
                  +-------+
+--------+        |       |  
| p-type |~~~~~~~~|       |----> Laser Beam
+--------+        |       |
| active |~~~~~~~~|       |
| layer  |        |       |
+--------+        |       |
| n-type |~~~~~~~~|       |
+--------+        |       |
                  +-------+
                Reflective Surfaces
\end{lstlisting}

\textbf{Construction:}

\begin{itemize}
\tightlist
\item
  \textbf{P-N junction}: Made of direct bandgap semiconductor (GaAs,
  InGaAsP)
\item
  \textbf{Active region}: Thin layer between P and N regions where
  recombination occurs
\item
  \textbf{Cavity design}: Parallel reflective surfaces (cleaved facets)
  form optical resonator
\item
  \textbf{Packaging}: Includes heat sink, optical window, monitoring
  photodiode
\end{itemize}

\textbf{Working principle:}

\begin{itemize}
\tightlist
\item
  \textbf{Injection}: Forward biasing injects electrons and holes into
  active region
\item
  \textbf{Population inversion}: More electrons in excited state than
  ground state
\item
  \textbf{Stimulated emission}: Photon triggers release of identical
  photons (same wavelength, phase)
\item
  \textbf{Optical feedback}: Photons reflect between mirrors, amplifying
  light
\item
  \textbf{Threshold current}: Minimum current for lasing action
\end{itemize}

\textbf{Characteristics:}

\begin{itemize}
\tightlist
\item
  \textbf{Coherent light}: Single wavelength, in-phase light emission
\item
  \textbf{Directionality}: Highly directional, narrow beam
\item
  \textbf{High intensity}: Concentrated energy output
\item
  \textbf{Threshold behavior}: Laser action only above threshold current
\end{itemize}

\textbf{Applications:}

\begin{itemize}
\tightlist
\item
  Optical fiber communications
\item
  DVD/Blu-ray players
\item
  Laser printers
\item
  Barcode scanners
\item
  Medical surgery instruments
\end{itemize}

\end{solutionbox}
\begin{mnemonicbox}
``Population Inversion Creates Coherent Light''
(PICL)

\end{mnemonicbox}
\subsection*{Question 3(a) OR [3
marks]}\label{q3a}

\textbf{Draw V-I characteristics of P-N junction diode and Zener diode.}

\begin{solutionbox}

\textbf{Diagram:}

\begin{lstlisting}
   I↑
    |                 /
    |                /
    |               /
    |              /
    |             /
Forward |            /
    |           /
    |          /         P-N Junction Diode
    |         /
    |        /
----+-------------------- V \rightarrow
    |       /
    |      /
    |     /
Reverse|
    |
    |
    |                     Zener
    |                     Breakdown
    |                     Region
    |                       |
    |                     \ |
    |                      \|
    |                       |
    |                       |
    |                       v
    
   I↑
    |                 /
    |                /
    |               /
    |              /
    |             /
Forward |            /
    |           /
    |          /         Zener Diode
    |         /
    |        /
----+-------------------- V \rightarrow
    |       /
    |      /
    |     /
Reverse|
    |       ______
    |      /
    |     /
    |    /         Zener
    |   /          Region
    |  /
    | /
    |/
    
\end{lstlisting}

\textbf{Key differences:}

\begin{itemize}
\tightlist
\item
  \textbf{P-N Junction diode}: Conducts in forward bias, blocks in
  reverse until breakdown
\item
  \textbf{Zener diode}: Specially designed to operate in reverse
  breakdown region at precise voltage
\end{itemize}

\end{solutionbox}
\begin{mnemonicbox}
``Forward Same, Reverse Different'' (FSRD)

\end{mnemonicbox}
\subsection*{Question 3(b) OR [4
marks]}\label{q3b}

\textbf{Explain working of P-N junction diode in forward bias with
circuit diagram.}

\begin{solutionbox}

\textbf{Diagram:}

\begin{lstlisting}
        +
    V   |     R
    ___/\/\/\__
   |            |
   |            |
   |    |>|     |
   |    D1      |
   |            |
   |____________|
        -
\end{lstlisting}

\textbf{Working in forward bias:}

\begin{itemize}
\tightlist
\item
  \textbf{Connection}: P-side connected to positive terminal, N-side to
  negative terminal
\item
  \textbf{Depletion region}: Width decreases as applied voltage
  increases
\item
  \textbf{Barrier potential}: Must overcome threshold (0.7V for Si, 0.3V
  for Ge)
\item
  \textbf{Current flow}: Above threshold, current increases
  exponentially with voltage
\item
  \textbf{Majority carriers}: Electrons from N-side and holes from
  P-side are pushed toward junction
\item
  \textbf{Recombination}: Electrons and holes recombine, creating
  continuous current flow
\end{itemize}

\textbf{Current equation}: I = I_{0}(e\^{}(qV/kT) - 1), where I_{0} is reverse
saturation current

\end{solutionbox}
\begin{mnemonicbox}
``Positive to P, Reduces Barrier, Current Flows''
(PPRBCF)

\end{mnemonicbox}
\subsection*{Question 3(c) OR [7
marks]}\label{q3c}

\textbf{Explain working of Light Emitting diode (LED) and Photodiode
with diagram.}

\begin{solutionbox}

\textbf{LED Diagram:}

\begin{lstlisting}
     Current
       flow
        ↓
    +-------+
    |       |
+---+       +---+
|   | P-type|   |
|   +-------+   |
|   | N-type|   |
|   |       |   |
+---+       +---+
    |       |
    +-------+
       ↑
     Photon
    Emission
\end{lstlisting}

\textbf{LED Working:}

\begin{itemize}
\tightlist
\item
  \textbf{Direct bandgap}: Made of GaAs, GaP compounds with direct
  bandgap
\item
  \textbf{Forward bias}: Applied to inject carriers across junction
\item
  \textbf{Recombination}: Electrons from N-side recombine with holes
  from P-side
\item
  \textbf{Photon emission}: Energy released during recombination emitted
  as photons
\item
  \textbf{Wavelength control}: Different materials produce different
  colors
\item
  \textbf{Efficiency}: Modern LEDs achieve 80-90\% efficiency
\end{itemize}

\textbf{Photodiode Diagram:}

\begin{lstlisting}
    +-------+
    |       |
+---+       +---+
|   | P-type|   |
|   +-------+   |
|   | N-type|   |
|   |       |   |
+---+       +---+
    |       |
    +-------+
       ↑
     Photon
    Absorption
\end{lstlisting}

\textbf{Photodiode Working:}

\begin{itemize}
\tightlist
\item
  \textbf{Reverse bias}: Operated in reverse bias typically
\item
  \textbf{Light absorption}: Photons absorbed in depletion region
\item
  \textbf{Electron-hole pairs}: Created by photon energy
\item
  \textbf{Carrier separation}: Electric field separates electrons and
  holes
\item
  \textbf{Current generation}: Photocurrent proportional to light
  intensity
\item
  \textbf{Response time}: Faster in reverse bias due to wider depletion
  region
\end{itemize}

\textbf{Comparison table:}

{\def\LTcaptype{none} % do not increment counter
\begin{longtable}[]{@{}
  >{\raggedright\arraybackslash}p{(\linewidth - 4\tabcolsep) * \real{0.3929}}
  >{\raggedright\arraybackslash}p{(\linewidth - 4\tabcolsep) * \real{0.1786}}
  >{\raggedright\arraybackslash}p{(\linewidth - 4\tabcolsep) * \real{0.4286}}@{}}
\toprule\noalign{}
\begin{minipage}[b]{\linewidth}\raggedright
Parameter
\end{minipage} & \begin{minipage}[b]{\linewidth}\raggedright
LED
\end{minipage} & \begin{minipage}[b]{\linewidth}\raggedright
Photodiode
\end{minipage} \\
\midrule\noalign{}
\endhead
\bottomrule\noalign{}
\endlastfoot
\textbf{Function} & Converts electrical energy to light & Converts light
to electrical energy \\
\textbf{Bias mode} & Forward bias & Reverse bias (typically) \\
\textbf{Direction} & Energy output (emitter) & Energy input
(detector) \\
\textbf{Application} & Displays, indicators, lighting & Light sensors,
optical communications \\
\end{longtable}
}

\end{solutionbox}
\begin{mnemonicbox}
``LEDs Emit, Photodiodes Detect'' (LEPD)

\end{mnemonicbox}
\subsection*{Question 4(a) [3 marks]}\label{q4a}

\textbf{Define the following terms:} \textbf{(i) Rectifier efficiency
(η) (ii) Ripple factor (γ) (iii) Voltage regulation}

\begin{solutionbox}

\textbf{Table of Definitions:}

{\def\LTcaptype{none} % do not increment counter
\begin{longtable}[]{@{}
  >{\raggedright\arraybackslash}p{(\linewidth - 2\tabcolsep) * \real{0.3333}}
  >{\raggedright\arraybackslash}p{(\linewidth - 2\tabcolsep) * \real{0.6667}}@{}}
\toprule\noalign{}
\begin{minipage}[b]{\linewidth}\raggedright
Term
\end{minipage} & \begin{minipage}[b]{\linewidth}\raggedright
Definition
\end{minipage} \\
\midrule\noalign{}
\endhead
\bottomrule\noalign{}
\endlastfoot
\textbf{Rectifier efficiency (η)} & Ratio of DC power output to AC power
input in a rectifier circuit (η = P\_DC/P\_AC \times 100\%) \\
\textbf{Ripple factor (γ)} & Ratio of RMS value of AC component to DC
component in rectifier output (γ = V\_rms(ac)/V\_dc) \\
\textbf{Voltage regulation} & Measure of how well a power supply
maintains constant output voltage despite changes in load (VR =
[(V\_NL - V\_FL)/V\_FL] \times 100\%) \\
\end{longtable}
}

\end{solutionbox}
\begin{mnemonicbox}
``Efficiency Powers, Ripple Varies, Regulation
Stabilizes'' (EPRVS)

\end{mnemonicbox}
\subsection*{Question 4(b) [4 marks]}\label{q4b}

\textbf{Explain zener diode as a voltage regulator.}

\begin{solutionbox}

\textbf{Diagram:}

\begin{lstlisting}
    R
   /\/\/\
Vi +----+----+ Vout
    |    |    |
    |   [Z]   RL
    |    |    |
    +----+----+
         -
\end{lstlisting}

\textbf{Working principle:}

\begin{itemize}
\tightlist
\item
  \textbf{Zener breakdown}: Operates in reverse breakdown region at
  specific voltage
\item
  \textbf{Series resistor}: Limits current and drops excess voltage
\item
  \textbf{Parallel connection}: Zener connected in parallel with load
\item
  \textbf{Regulation mechanism}:

  \begin{itemize}
  \tightlist
  \item
    When input voltage increases: More current through Zener, voltage
    across load remains constant
  \item
    When load current increases: Less current through Zener, voltage
    remains constant
  \end{itemize}
\end{itemize}

\textbf{Characteristics:}

\begin{itemize}
\tightlist
\item
  \textbf{Load regulation}: Maintains constant voltage despite load
  changes
\item
  \textbf{Line regulation}: Maintains constant voltage despite input
  voltage changes
\item
  \textbf{Power rating}: Zener must handle maximum power dissipation (P
  = V\_Z \times I\_Z)
\item
  \textbf{Design equation}: R = (V\_in - V\_Z)/I\_L + I\_Z)
\end{itemize}

\end{solutionbox}
\begin{mnemonicbox}
``Zener Shunts Excess Current'' (ZSEC)

\end{mnemonicbox}
\subsection*{Question 4(c) [7 marks]}\label{q4c}

\textbf{Explain full wave bridge rectifier with circuit diagram and
input-output waveform.}

\begin{solutionbox}

\textbf{Circuit Diagram:}

\begin{lstlisting}
         D1        D3
         |>|       |>|
          |         |
Vin ------+----+----+----- Vout
          |    |    |
          |    RL   |
          |    |    |
          +----+----+
         |>|       |>|
         D2        D4
\end{lstlisting}

\textbf{Working principle:}

\begin{itemize}
\tightlist
\item
  \textbf{First half cycle (positive)}: D1 and D4 conduct, D2 and D3
  block
\item
  \textbf{Second half cycle (negative)}: D2 and D3 conduct, D1 and D4
  block
\item
  \textbf{Both half cycles}: Current flows through load in same
  direction
\end{itemize}

\textbf{Waveforms:}

\begin{lstlisting}
Input:         Output:
    ^              ^
    |              |
    |  /\    /\    |   /\    /\    /\
    | /  \  /  \   |  /  \  /  \  /  \
----+------+---+---+-+----+----+----+-->
    |      \  /    |
    |       \/     |
    |              |
    v              v
\end{lstlisting}

\textbf{Characteristics:}

\begin{itemize}
\tightlist
\item
  \textbf{Ripple frequency}: Twice the input frequency
\item
  \textbf{Output voltage}: V\_dc = 2V\_m/π \approx 0.636V\_m
\item
  \textbf{PIV}: Each diode must withstand V\_m
\item
  \textbf{Efficiency}: η = 81.2\%
\item
  \textbf{Ripple factor}: γ = 0.48
\item
  \textbf{Uses}: Higher current applications, no center-tapped
  transformer needed
\end{itemize}

\textbf{Advantages over center-tapped:}

\begin{itemize}
\tightlist
\item
  No center-tapped transformer required
\item
  Lower PIV requirement for diodes
\item
  Better transformer utilization
\end{itemize}

\end{solutionbox}
\begin{mnemonicbox}
``Bridge Brings Both Halves'' (BBBH)

\end{mnemonicbox}
\subsection*{Question 4(a) OR [3
marks]}\label{q4a}

\textbf{Give the applications of rectifier.}

\begin{solutionbox}

\textbf{Applications of Rectifiers:}

{\def\LTcaptype{none} % do not increment counter
\begin{longtable}[]{@{}
  >{\raggedright\arraybackslash}p{(\linewidth - 2\tabcolsep) * \real{0.5455}}
  >{\raggedright\arraybackslash}p{(\linewidth - 2\tabcolsep) * \real{0.4545}}@{}}
\toprule\noalign{}
\begin{minipage}[b]{\linewidth}\raggedright
Application Area
\end{minipage} & \begin{minipage}[b]{\linewidth}\raggedright
Specific Uses
\end{minipage} \\
\midrule\noalign{}
\endhead
\bottomrule\noalign{}
\endlastfoot
\textbf{Power supplies} & DC power supplies for electronic devices,
battery chargers, adaptors \\
\textbf{Industrial applications} & Electroplating, welding machines,
motor drives, induction heating \\
\textbf{Transport systems} & Electric locomotives, metro trains,
electric vehicles \\
\textbf{Renewable energy} & Solar inverters, wind power generation \\
\textbf{Consumer electronics} & Mobile phone chargers, laptop adapters,
TV power supplies \\
\textbf{Telecommunications} & Base stations, transmission equipment,
signal processing devices \\
\end{longtable}
}

\end{solutionbox}
\begin{mnemonicbox}
``Power Perfectly Transformed in Consumer Devices''
(PPTICD)

\end{mnemonicbox}
\subsection*{Question 4(b) OR [4
marks]}\label{q4b}

\textbf{Compare half wave, full wave center tapped and full wave bridge
rectifier with four parameters.}

\begin{solutionbox}

{\def\LTcaptype{none} % do not increment counter
\begin{longtable}[]{@{}
  >{\raggedright\arraybackslash}p{(\linewidth - 6\tabcolsep) * \real{0.1692}}
  >{\raggedright\arraybackslash}p{(\linewidth - 6\tabcolsep) * \real{0.1692}}
  >{\raggedright\arraybackslash}p{(\linewidth - 6\tabcolsep) * \real{0.3846}}
  >{\raggedright\arraybackslash}p{(\linewidth - 6\tabcolsep) * \real{0.2769}}@{}}
\toprule\noalign{}
\begin{minipage}[b]{\linewidth}\raggedright
Parameter
\end{minipage} & \begin{minipage}[b]{\linewidth}\raggedright
Half Wave
\end{minipage} & \begin{minipage}[b]{\linewidth}\raggedright
Full Wave Center Tapped
\end{minipage} & \begin{minipage}[b]{\linewidth}\raggedright
Full Wave Bridge
\end{minipage} \\
\midrule\noalign{}
\endhead
\bottomrule\noalign{}
\endlastfoot
\textbf{Number of diodes} & 1 & 2 & 4 \\
\textbf{DC output voltage} & V\_m/π (0.318V\_m) & 2V\_m/π (0.636V\_m) &
2V\_m/π (0.636V\_m) \\
\textbf{Ripple frequency} & Same as input & Twice the input & Twice the
input \\
\textbf{Efficiency} & 40.6\% & 81.2\% & 81.2\% \\
\textbf{Transformer utilization} & Poor & Medium (center tap needed) &
Good (no center tap) \\
\textbf{PIV of diodes} & V\_m & 2V\_m & V\_m \\
\textbf{Ripple factor} & 1.21 & 0.48 & 0.48 \\
\textbf{Form factor} & 1.57 & 1.11 & 1.11 \\
\end{longtable}
}

\end{solutionbox}
\begin{mnemonicbox}
``Half Wastes, Center Tapped Improves, Bridge
Optimizes'' (HWCTIBO)

\end{mnemonicbox}
\subsection*{Question 4(c) OR [7
marks]}\label{q4c}

\textbf{Explain Shunt capacitor filter and π-filter with circuit
diagram.}

\begin{solutionbox}

\textbf{Shunt Capacitor Filter:}

\textbf{Diagram:}

\begin{lstlisting}
      Rectifier   C
         |        |
Vin --->|M|-------+------ Vout
         |        |
         |        RL
         |        |
         +--------+------
\end{lstlisting}

\textbf{Working principle:}

\begin{itemize}
\tightlist
\item
  \textbf{Charging}: Capacitor charges rapidly during voltage rise in
  rectifier output
\item
  \textbf{Discharging}: Capacitor discharges slowly through load during
  voltage fall
\item
  \textbf{Smoothing effect}: Reduces ripple by storing energy when
  voltage is high
\item
  \textbf{Time constant}: RC should be much larger than ripple period
\item
  \textbf{Performance}: Ripple factor γ = 1/(4\sqrt3·f·R·C)
\end{itemize}

\textbf{π-Filter:}

\textbf{Diagram:}

\begin{lstlisting}
      Rectifier    L
         |        /\/\/\
Vin --->|M|-------+------ Vout
         |        |
         |        |
         |        |
         +---||---+---||--+
             C1       C2  |
             |        |   RL
             |        |   |
             +--------+---+
\end{lstlisting}

\textbf{Working principle:}

\begin{itemize}
\tightlist
\item
  \textbf{First capacitor (C1)}: Provides initial filtering like shunt
  capacitor
\item
  \textbf{Choke (L)}: Blocks AC components, allows DC to pass
\item
  \textbf{Second capacitor (C2)}: Further reduces remaining ripple
\item
  \textbf{Combined effect}: Acts as cascade of low-pass filters
\end{itemize}

\textbf{Comparison:}

{\def\LTcaptype{none} % do not increment counter
\begin{longtable}[]{@{}
  >{\raggedright\arraybackslash}p{(\linewidth - 4\tabcolsep) * \real{0.2444}}
  >{\raggedright\arraybackslash}p{(\linewidth - 4\tabcolsep) * \real{0.5333}}
  >{\raggedright\arraybackslash}p{(\linewidth - 4\tabcolsep) * \real{0.2222}}@{}}
\toprule\noalign{}
\begin{minipage}[b]{\linewidth}\raggedright
Parameter
\end{minipage} & \begin{minipage}[b]{\linewidth}\raggedright
Shunt Capacitor Filter
\end{minipage} & \begin{minipage}[b]{\linewidth}\raggedright
π-Filter
\end{minipage} \\
\midrule\noalign{}
\endhead
\bottomrule\noalign{}
\endlastfoot
\textbf{Components} & Single capacitor & Two capacitors and inductor \\
\textbf{Ripple reduction} & Moderate & Excellent \\
\textbf{Cost} & Low & High \\
\textbf{Size} & Small & Large \\
\textbf{Voltage regulation} & Poor & Good \\
\textbf{Suitable for} & Low current applications & High current
applications \\
\end{longtable}
}

\end{solutionbox}
\begin{mnemonicbox}
``Capacitor Smooths, Pi-Filter Perfects'' (CSPFP)

\end{mnemonicbox}
\subsection*{Question 5(a) [3 marks]}\label{q5a}

\textbf{Draw the symbols of following components:} \textbf{(i) PNP
transistor (ii) N channel JFET (iii) N channel enhancement mode MOSFET}

\begin{solutionbox}

\textbf{Diagram:}

\begin{lstlisting}
PNP Transistor:       N-channel JFET:      N-channel enhancement MOSFET:
     C                     D                        D
     |                     |                        |
     |                     |                        |
  >--+                     +---<                    |
 /    \                   /|                        |
|  E   |                 / |                     +--+
 \    /                 /  |                     |  |
  +--+--B              /   |                 G---+  |
  |                    |   |                     |  |
  |                 G--+   +--S                  +--+--S
  E                    |                            |
                       |                            |
                       S
\end{lstlisting}

\textbf{Characteristics:}

\begin{itemize}
\tightlist
\item
  \textbf{PNP Transistor}: Arrow points inward at emitter
\item
  \textbf{N-channel JFET}: Gate controls channel between source and
  drain
\item
  \textbf{N-channel enhancement MOSFET}: Gap in channel, requires
  positive gate voltage
\end{itemize}

\end{solutionbox}
\begin{mnemonicbox}
``PNP Points IN, JFET Joins Gates, MOSFET Makes
Gaps'' (PPIJJGMMG)

\end{mnemonicbox}
\subsection*{Question 5(b) [4 marks]}\label{q5b}

\textbf{Explain working of NPN transistor with diagram.}

\begin{solutionbox}

\textbf{Diagram:}

\begin{lstlisting}
        Collector (C)
            |
            |
    +-----------------+
    |      N-type     |
    +-----------------+
    |      P-type     |
    +-----------------+
    |      N-type     |
    +-----------------+
            |
            |
        Emitter (E)
            
  B---/\/\/\--+   +--/\/\/\--C
    (RB)      |   |  (RC)
              |   |
              V_BE|   V_CE
      +-------|---+-------+
      |       |           |
      |       +--[NPN]----+
      |          |        |
      |          |        |
      +----------+--------+
                 |
                 |
                 E
\end{lstlisting}

\textbf{Working principle:}

\begin{itemize}
\tightlist
\item
  \textbf{Structure}: Two N-type regions separated by thin P-type region
\item
  \textbf{Biasing}: E-B junction forward biased, C-B junction reverse
  biased
\item
  \textbf{Current flow}:

  \begin{itemize}
  \tightlist
  \item
    Electrons from emitter cross into base
  \item
    \textasciitilde98\% electrons continue to collector due to thin base
    region
  \item
    \textasciitilde2\% electrons recombine in base region
  \end{itemize}
\item
  \textbf{Amplification}: Small base current controls larger collector
  current
\item
  \textbf{Current relationship}: I\_C = β \times I\_B (where β is current
  gain)
\end{itemize}

\textbf{Junction behavior:}

\begin{itemize}
\tightlist
\item
  \textbf{Emitter-Base junction}: Forward biased, low resistance path
\item
  \textbf{Collector-Base junction}: Reverse biased, high resistance path
\end{itemize}

\end{solutionbox}
\begin{mnemonicbox}
``Electrons Enter, Barely Pause, Collect Above''
(EEBPCA)

\end{mnemonicbox}
\subsection*{Question 5(c) [7 marks]}\label{q5c}

\textbf{Draw and explain common emitter (CE) transistor with its input
output characteristic.}

\begin{solutionbox}

\textbf{Circuit Diagram:}

\begin{lstlisting}
     +VCC
      |
      R_C
      |
      +-----o V_out
      |
      |
  B---+--[NPN]
  |      |
  R_B    |
  |      |
  +      E
V_in     |
  -      |
 GND    GND
\end{lstlisting}

\textbf{Input Characteristics (I\_B vs V\_BE with V\_CE constant):}

\begin{lstlisting}
  I_B ↑
   |
   |            V_CE = 10V
   |           /
   |          /
   |         / V_CE = 5V
   |        /
   |       /
   |      /
   |     /
   |    /
   |   /
   |  /
   | /
   |/
   +--------------> V_BE
       0.7V
\end{lstlisting}

\textbf{Output Characteristics (I\_C vs V\_CE with I\_B constant):}

\begin{lstlisting}
  I_C ↑
   |                   I_B = 50μA
   |                 /-----------
   |                /
   |               /  I_B = 40μA
   |              /------------
   |             /
   |            /   I_B = 30μA
   |           /--------------
   |          /
   |         /    I_B = 20μA
   |        /---------------
   |       /
   |      /     I_B = 10μA
   |     /----------------
   |    /
   |   /      I_B = 0
   |  /------------------
   | /
   |/
   +--+-----+----------> V_CE
      |     |
    Saturation|Active
      Region |Region
\end{lstlisting}

\textbf{Operating regions:}

\begin{itemize}
\tightlist
\item
  \textbf{Cut-off}: I\_B \approx 0, I\_C \approx 0, transistor OFF
\item
  \textbf{Active}: E-B junction forward biased, C-B junction reverse
  biased, linear amplification
\item
  \textbf{Saturation}: Both junctions forward biased, transistor fully
  ON
\end{itemize}

\textbf{Parameters:}

\begin{itemize}
\tightlist
\item
  \textbf{Current gain (β)}: Ratio of collector current to base current
  (β = I\_C/I\_B)
\item
  \textbf{Input resistance}: Ratio of change in V\_BE to change in I\_B
\item
  \textbf{Output resistance}: Ratio of change in V\_CE to change in I\_C
\end{itemize}

\textbf{Applications:}

\begin{itemize}
\tightlist
\item
  \textbf{Amplification}: Voltage, current, and power amplification
\item
  \textbf{Switching}: Digital circuits, logic gates
\item
  \textbf{Signal processing}: Oscillators, filters, modulators
\end{itemize}

\end{solutionbox}
\begin{mnemonicbox}
``Cut-Active-Saturate: Off-Amplify-On'' (CASOAO)

\end{mnemonicbox}
\subsection*{Question 5(a) OR [3
marks]}\label{q5a}

\textbf{Derive relationship between current gain alpha (α) and beta
(β).}

\begin{solutionbox}

\textbf{Key definitions:}

\begin{itemize}
\tightlist
\item
  \textbf{Alpha (α)}: Common-base current gain = I\_C/I\_E
\item
  \textbf{Beta (β)}: Common-emitter current gain = I\_C/I\_B
\end{itemize}

\textbf{Diagram:}

\begin{lstlisting}
         I_C
         ^
         |
    +----+----+
    |         |
I_E >    T    > I_B
    |         |
    +---------+
\end{lstlisting}

\textbf{Current relationship in transistor:}

\begin{itemize}
\tightlist
\item
  I\_E = I\_B + I\_C (Kirchhoff's Current Law)
\end{itemize}

\textbf{Derivation steps:}

\begin{enumerate}
\tightlist
\item
  α = I\_C/I\_E
\item
  I\_E = I\_B + I\_C
\item
  α = I\_C/(I\_B + I\_C)
\item
  β = I\_C/I\_B
\item
  I\_C = β \times I\_B
\item
Substituting in equation 3:

α = (β \times I\_B)/(I\_B + β \times I\_B)

α = β/(1

  + β)
\item
Solving for β: α(1 + β) = β α + αβ = β

α = β - αβ

α = β(1 - α)

β =

  α/(1 - α)
\end{enumerate}

\textbf{Final relationships:}

\begin{itemize}
\tightlist
\item
  β = α/(1 - α)
\item
  α = β/(1 + β)
\end{itemize}

\textbf{Typical values:}

\begin{itemize}
\tightlist
\item
  α is always less than 1 (typically 0.95 to 0.99)
\item
  β typically ranges from 20 to 200
\end{itemize}

\end{solutionbox}
\begin{mnemonicbox}
``Alpha Approaches One, Beta Becomes Infinite''
(AAOBBI)

\end{mnemonicbox}
\subsection*{Question 5(b) OR [4
marks]}\label{q5b}

\textbf{Explain different operating regions for transistor.}

\begin{solutionbox}

\textbf{Diagram:}

\begin{lstlisting}
  I_C ↑
   |
   |      +-------------+
   |      |             |
   |      |             |
   |      |             |
   |      |   Active    |
   |      |   Region    |
   | Saturation         |
   | Region|            |
   |      |             |
   |      |             |
   |      |             |
   +------+-------------+-------> V_CE
   |                    |
   |                    |
   |   Cut-off Region   |
   |                    |
   |                    |
   +--------------------+
\end{lstlisting}

\textbf{Operating regions:}

{\def\LTcaptype{none} % do not increment counter
\begin{longtable}[]{@{}
  >{\raggedright\arraybackslash}p{(\linewidth - 6\tabcolsep) * \real{0.1538}}
  >{\raggedright\arraybackslash}p{(\linewidth - 6\tabcolsep) * \real{0.2885}}
  >{\raggedright\arraybackslash}p{(\linewidth - 6\tabcolsep) * \real{0.3077}}
  >{\raggedright\arraybackslash}p{(\linewidth - 6\tabcolsep) * \real{0.2500}}@{}}
\toprule\noalign{}
\begin{minipage}[b]{\linewidth}\raggedright
Region
\end{minipage} & \begin{minipage}[b]{\linewidth}\raggedright
Junction Bias
\end{minipage} & \begin{minipage}[b]{\linewidth}\raggedright
Characteristics
\end{minipage} & \begin{minipage}[b]{\linewidth}\raggedright
Applications
\end{minipage} \\
\midrule\noalign{}
\endhead
\bottomrule\noalign{}
\endlastfoot
\textbf{Cut-off} & E-B: OFFC-B: OFF & • I\_B \approx 0, I\_C \approx 0• Transistor
is OFF• V\_CE \approx V\_CC & Digital circuits (OFF state)Switching
applications \\
\textbf{Active} & E-B: ONC-B: OFF & • Linear relationship between I\_C
and I\_B• I\_C = β \times I\_B• Used for amplification & Analog
amplifiersSignal processing \\
\textbf{Saturation} & E-B: ONC-B: ON & • Both junctions forward biased•
Transistor fully ON• V\_CE \approx 0.2V & Digital circuits (ON state)Switching
applications \\
\textbf{Breakdown} & E-B: OFFC-B: Breakdown & • Exceeds breakdown
voltage• Can damage transistor• Should be avoided & Avoid this region in
normal operation \\
\end{longtable}
}

\end{solutionbox}
\begin{mnemonicbox}
``Cut Active Saturate: Off Amplify Switch'' (CASOAS)

\end{mnemonicbox}
\subsection*{Question 5(c) OR [7
marks]}\label{q5c}

\textbf{Write a short note on MOSFET.}

\begin{solutionbox}

\textbf{MOSFET (Metal Oxide Semiconductor Field Effect Transistor)}

\textbf{Structure Diagram:}

\begin{lstlisting}
    Gate (G)
       |
       v
    +-----+    Drain (D)
    |  M  |      |
    +-----+      v
    |  O  |    +---+
    +-----+    |   |
    |  S  |    | N |
    +-----+----+---+----+
    |                   |
    |        P          |
    |                   |
    +---+-------------+-+
        |             |
        v             v
    Source (S)    Substrate
\end{lstlisting}

\textbf{Types of MOSFETs:}

\begin{itemize}
\tightlist
\item
  \textbf{Enhancement mode}: Channel does not exist without gate voltage

  \begin{itemize}
  \tightlist
  \item
    N-channel: Positive gate voltage creates channel
  \item
    P-channel: Negative gate voltage creates channel
  \end{itemize}
\item
  \textbf{Depletion mode}: Channel exists without gate voltage

  \begin{itemize}
  \tightlist
  \item
    N-channel: Negative gate voltage depletes channel
  \item
    P-channel: Positive gate voltage depletes channel
  \end{itemize}
\end{itemize}

\textbf{Working principle:}

\begin{itemize}
\tightlist
\item
  \textbf{Insulated gate}: Gate isolated from channel by oxide layer
\item
  \textbf{Field effect}: Electric field controls channel conductivity
\item
  \textbf{Voltage controlled}: Gate voltage controls drain current
\item
  \textbf{No gate current}: Extremely high input impedance
\end{itemize}

\textbf{Characteristics:}

\begin{itemize}
\tightlist
\item
  \textbf{Transfer characteristic}: I\_D vs V\_GS
\item
  \textbf{Output characteristic}: I\_D vs V\_DS
\item
  \textbf{Threshold voltage}: Minimum V\_GS required to create channel
\item
  \textbf{Transconductance}: Change in I\_D per unit change in V\_GS
\end{itemize}

\textbf{Advantages over BJT:}

\begin{itemize}
\tightlist
\item
  \textbf{High input impedance}: Virtually no input current
\item
  \textbf{Faster switching}: Lower capacitance, no minority carrier
  storage
\item
  \textbf{Higher packing density}: Smaller size for same function
\item
  \textbf{Lower power consumption}: Less heat generation
\item
  \textbf{Simpler biasing}: Single polarity supply often sufficient
\end{itemize}

\textbf{Applications:}

\begin{itemize}
\tightlist
\item
  \textbf{Digital circuits}: CMOS logic, memory devices
\item
  \textbf{Analog circuits}: Amplifiers, current sources
\item
  \textbf{Power electronics}: High-power switching
\item
  \textbf{RF applications}: Low-noise amplifiers
\item
  \textbf{Integrated circuits}: Processors, ASICs
\end{itemize}

\end{solutionbox}
\begin{mnemonicbox}
``Metal Oxide Separate Gate Enables Field Control''
(MOSGFC)

\end{mnemonicbox}

\end{document}
