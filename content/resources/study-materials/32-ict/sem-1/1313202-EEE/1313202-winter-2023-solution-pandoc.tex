\documentclass[10pt,a4paper]{article}

% content/resources/templates/preamble.tex
\usepackage[margin=0.6in]{geometry}
\author{Milav Dabgar}
\usepackage{amsmath,amssymb,amsthm}
\usepackage{booktabs}
\usepackage{multirow}
\usepackage{xcolor}
\usepackage{tcolorbox}
\tcbuselibrary{breakable,skins}
\usepackage[colorlinks=true,linkcolor=blue]{hyperref}
\usepackage{titlesec}
\usepackage{enumitem}
\usepackage{tikz}
\usepackage{pgfplots}
\usepackage{circuitikz}
\usepackage[version=4]{mhchem}
\usepackage{longtable}
\usepackage{array}
\usepackage{float}
\usepackage{caption}
\usepackage{listings}

\lstset{
  basicstyle=\small\ttfamily,
  breaklines=true,
  breakatwhitespace=false,
  postbreak=\mbox{\textcolor{red}{$\hookrightarrow$}\space},
  float=false,
  numbers=left,
  numberstyle=\tiny\color{gray},
  numbersep=10pt,
  xleftmargin=2em,
  keywordstyle=\color{blue},
  commentstyle=\color{green!60!black},
  stringstyle=\color{purple},
  backgroundcolor=\color{gray!5},
  showstringspaces=false,
  tabsize=2,
  captionpos=b,
  keepspaces=true,
  columns=flexible
}

\pgfplotsset{compat=1.18}
\usetikzlibrary{shapes,arrows,positioning,calc,patterns,decorations.pathmorphing,decorations.markings,arrows.meta}

% Color scheme
\definecolor{headcolor}{RGB}{0,102,204}
\definecolor{keycolor}{RGB}{220,20,60}
\definecolor{solutioncolor}{RGB}{34,139,34}
\definecolor{mnemoniccolor}{RGB}{148,0,211}
\definecolor{codecolor}{RGB}{0,0,100}

% Spacing
\setlength{\parskip}{3pt}
\setlist[itemize]{nosep}
\setlist[enumerate]{nosep}

% Title formatting
\titleformat{\section}{\Large\bfseries\color{headcolor}}{\thesection}{1em}{}
\titleformat{\subsection}{\large\bfseries\color{headcolor}}{\thesubsection}{1em}{}

% Pandoc tightlist compatibility
\providecommand{\tightlist}{%
  \setlength{\itemsep}{0pt}\setlength{\parskip}{0pt}}

% Pandoc longtable compatibility
\newcounter{none}
\def\thenone{}


% content/resources/templates/english-boxes.tex
% This file is currently empty - it exists to maintain consistency with the import structure.
% Add custom environments here if needed in the future.


\begin{document}

\begin{center}
{\Huge\bfseries\color{headcolor} Subject Name Solutions}\\[5pt]
{\LARGE 1313202 -- Winter 2023}\\[3pt]
{\large Semester 1 Study Material}\\[3pt]
{\normalsize\textit{Detailed Solutions and Explanations}}
\end{center}

\vspace{10pt}

\subsection*{Question 1(a) [3 marks]}\label{q1a}

\textbf{Explain difference between Active and passive network.}

\begin{solutionbox}

{\def\LTcaptype{none} % do not increment counter
\begin{longtable}[]{@{}
  >{\raggedright\arraybackslash}p{(\linewidth - 2\tabcolsep) * \real{0.4750}}
  >{\raggedright\arraybackslash}p{(\linewidth - 2\tabcolsep) * \real{0.5250}}@{}}
\toprule\noalign{}
\begin{minipage}[b]{\linewidth}\raggedright
\textbf{Active Network}
\end{minipage} & \begin{minipage}[b]{\linewidth}\raggedright
\textbf{Passive Network}
\end{minipage} \\
\midrule\noalign{}
\endhead
\bottomrule\noalign{}
\endlastfoot
Contains at least one energy source & Contains no energy source \\
Can deliver power to other elements & Cannot deliver power to other
elements \\
Examples: Transistors, Op-amps, Batteries & Examples: Resistors,
Capacitors, Inductors \\
\end{longtable}
}

\end{solutionbox}
\begin{mnemonicbox}
``Active Adds Power, Passive Pulls Power''

\end{mnemonicbox}
\subsection*{Question 1(b) [4 marks]}\label{q1b}

\textbf{State and explain Kirchhoff's voltage law (KVL).}

\begin{solutionbox}

\textbf{Kirchhoff's Voltage Law (KVL)}: The algebraic sum of all
voltages around any closed path (loop) in a circuit is zero.

\textbf{Diagram:}

\includegraphics[width=1\linewidth,height=\textheight,keepaspectratio]{mermaid-28c30a58.pdf}

\textbf{Mathematical Form}: V_{1} + V_{2} + V_{3} + V_{4} = 0

\begin{itemize}
\tightlist
\item
  \textbf{Circuit Application}: When moving around a loop, voltage rises
  (batteries) are positive and voltage drops (components) are negative
\item
  \textbf{Physical Meaning}: Total energy in a closed loop is conserved
\end{itemize}

\end{solutionbox}
\begin{mnemonicbox}
``Voltage Loop Sum Zero''

\end{mnemonicbox}
\subsection*{Question 1(c) [7 marks]}\label{q1c}

\textbf{Define the following terms: (1) Charge (2) Current (3) Potential
(4) E.M.F. (5) Inductance (6) Capacitance (7) Frequency.}

\begin{solutionbox}

{\def\LTcaptype{none} % do not increment counter
\begin{longtable}[]{@{}
  >{\raggedright\arraybackslash}p{(\linewidth - 2\tabcolsep) * \real{0.3846}}
  >{\raggedright\arraybackslash}p{(\linewidth - 2\tabcolsep) * \real{0.6154}}@{}}
\toprule\noalign{}
\begin{minipage}[b]{\linewidth}\raggedright
\textbf{Term}
\end{minipage} & \begin{minipage}[b]{\linewidth}\raggedright
\textbf{Definition}
\end{minipage} \\
\midrule\noalign{}
\endhead
\bottomrule\noalign{}
\endlastfoot
\textbf{Charge} & The basic electrical quantity measured in coulombs
(C); flow of electrons creates electricity \\
\textbf{Current} & The rate of flow of electric charge, measured in
amperes (A); I = dQ/dt \\
\textbf{Potential} & Electric potential energy per unit charge, measured
in volts (V) \\
\textbf{E.M.F.} & Electromotive force, energy supplied by source per
unit charge, measured in volts (V) \\
\textbf{Inductance} & Property of a conductor to oppose change in
current, measured in henry (H) \\
\textbf{Capacitance} & Ability of a component to store electric charge,
measured in farad (F) \\
\textbf{Frequency} & Number of cycles per second of an alternating
quantity, measured in hertz (Hz) \\
\end{longtable}
}

\end{solutionbox}
\begin{mnemonicbox}
``Careful Currents Pass Easily Into Circuit
Frequently''

\end{mnemonicbox}
\subsection*{Question 1(c) OR [7
marks]}\label{q1c}

\textbf{State Ohm's law. Write its application and limitation.}

\begin{solutionbox}

\textbf{Ohm's Law}: The current flowing through a conductor is directly
proportional to the potential difference across it and inversely
proportional to its resistance.

\textbf{Mathematical Form}: I = V/R

\textbf{Diagram:}

\begin{lstlisting}
            +
            |
            V
            |
  A---------/\/\/\---------B
            R
            |
            |
            -
\end{lstlisting}

\textbf{Applications of Ohm's Law}:

\begin{itemize}
\tightlist
\item
  Computing current, voltage, resistance in circuits
\item
  Design of electrical networks
\item
  Power calculations (P = VI = I^{2}R = V^{2}/R)
\item
  Voltage division and current division
\end{itemize}

\textbf{Limitations of Ohm's Law}:

\begin{itemize}
\tightlist
\item
  Not valid for non-linear elements (diodes, transistors)
\item
  Not applicable at very high frequencies
\item
  Not valid for non-metallic conductors like semiconductors
\item
  Not applicable for vacuum tubes and gaseous devices
\end{itemize}

\end{solutionbox}
\begin{mnemonicbox}
``Voltage Drives, Resistance Restricts''

\end{mnemonicbox}
\subsection*{Question 2(a) [3 marks]}\label{q2a}

\textbf{Draw and explain energy band diagrams for insulator, conductor
and Semiconductor.}

\begin{solutionbox}

\textbf{Energy Band Diagrams}:

\includegraphics[width=1\linewidth,height=\textheight,keepaspectratio]{mermaid-ca9f3da2.pdf}

\begin{itemize}
\tightlist
\item
  \textbf{Conductor}: Valence and conduction bands overlap, allowing
  easy electron flow
\item
  \textbf{Semiconductor}: Small energy gap (\textasciitilde1eV) between
  bands; electrons can jump with thermal energy
\item
  \textbf{Insulator}: Large energy gap (\textgreater5eV) prevents
  electron movement between bands
\end{itemize}

\end{solutionbox}
\begin{mnemonicbox}
``Conductors Connect, Semiconductors Sometimes,
Insulators Impede''

\end{mnemonicbox}
\subsection*{Question 2(b) [4 marks]}\label{q2b}

\textbf{Write statement of Maximum power transfer theorem and
reciprocity theorem.}

\begin{solutionbox}

{\def\LTcaptype{none} % do not increment counter
\begin{longtable}[]{@{}
  >{\raggedright\arraybackslash}p{(\linewidth - 2\tabcolsep) * \real{0.4643}}
  >{\raggedright\arraybackslash}p{(\linewidth - 2\tabcolsep) * \real{0.5357}}@{}}
\toprule\noalign{}
\begin{minipage}[b]{\linewidth}\raggedright
\textbf{Theorem}
\end{minipage} & \begin{minipage}[b]{\linewidth}\raggedright
\textbf{Statement}
\end{minipage} \\
\midrule\noalign{}
\endhead
\bottomrule\noalign{}
\endlastfoot
\textbf{Maximum Power Transfer Theorem} & Maximum power is transferred
from source to load when load resistance equals the source internal
resistance (RL = RS) \\
\textbf{Reciprocity Theorem} & In a linear passive network with a single
source, if the source is moved from position A to B, the current at A
due to source at B will equal the current at B when source was at A \\
\end{longtable}
}

\textbf{Diagram:}

\begin{lstlisting}
Maximum Power Transfer:
        +---[Source]---+
        |              |
        R(source)      R(load)
        |              |
        +------+-------+
\end{lstlisting}

\end{solutionbox}
\begin{mnemonicbox}
``Match Resistance to Maximize Power; Switch Source
and Sink, Current Stays Same''

\end{mnemonicbox}
\subsection*{Question 2(c) [7 marks]}\label{q2c}

\textbf{Explain the formation and conduction of N-type materials.}

\begin{solutionbox}

\textbf{N-type Semiconductor Formation}:

\includegraphics[width=1\linewidth,height=\textheight,keepaspectratio]{mermaid-b09f9c8f.pdf}

\begin{itemize}
\tightlist
\item
  \textbf{Doping Process}: Silicon/Germanium (4 valence e^{-}) doped with
  pentavalent elements (P, As, Sb)
\item
  \textbf{Extra Electron}: Each dopant atom provides 1 extra electron
  after covalent bonding
\item
  \textbf{Conduction Mechanism}:

  \begin{itemize}
  \tightlist
  \item
    \textbf{Majority Carriers}: Free electrons (negative charge
    carriers)
  \item
    \textbf{Minority Carriers}: Holes (very few)
  \end{itemize}
\item
  \textbf{Electrical Properties}: Increased conductivity and negative
  charge carriers
\end{itemize}

\end{solutionbox}
\begin{mnemonicbox}
``Pentavalent Provides Plus one Electron,
Negative-type''

\end{mnemonicbox}
\subsection*{Question 2(a) OR [3
marks]}\label{q2a}

\textbf{Define valence band, conduction band and forbidden gap.}

\begin{solutionbox}

{\def\LTcaptype{none} % do not increment counter
\begin{longtable}[]{@{}
  >{\raggedright\arraybackslash}p{(\linewidth - 2\tabcolsep) * \real{0.3846}}
  >{\raggedright\arraybackslash}p{(\linewidth - 2\tabcolsep) * \real{0.6154}}@{}}
\toprule\noalign{}
\begin{minipage}[b]{\linewidth}\raggedright
\textbf{Term}
\end{minipage} & \begin{minipage}[b]{\linewidth}\raggedright
\textbf{Definition}
\end{minipage} \\
\midrule\noalign{}
\endhead
\bottomrule\noalign{}
\endlastfoot
\textbf{Valence Band} & The highest energy band filled with electrons,
where electrons are bound to atoms \\
\textbf{Conduction Band} & The band above valence band where electrons
move freely and contribute to electrical conduction \\
\textbf{Forbidden Gap} & The energy range between valence and conduction
bands where no electron states exist \\
\end{longtable}
}

\textbf{Diagram:}

\includegraphics[width=1\linewidth,height=\textheight,keepaspectratio]{mermaid-dba9a145.pdf}

\end{solutionbox}
\begin{mnemonicbox}
``Valence Holds, Forbidden Blocks, Conduction Flows''

\end{mnemonicbox}
\subsection*{Question 2(b) OR [4
marks]}\label{q2b}

\textbf{Define the terms active power, reactive power and power factor
with power triangle.}

\begin{solutionbox}

\textbf{Power Terms in AC Circuits}:

{\def\LTcaptype{none} % do not increment counter
\begin{longtable}[]{@{}
  >{\raggedright\arraybackslash}p{(\linewidth - 2\tabcolsep) * \real{0.3846}}
  >{\raggedright\arraybackslash}p{(\linewidth - 2\tabcolsep) * \real{0.6154}}@{}}
\toprule\noalign{}
\begin{minipage}[b]{\linewidth}\raggedright
\textbf{Term}
\end{minipage} & \begin{minipage}[b]{\linewidth}\raggedright
\textbf{Definition}
\end{minipage} \\
\midrule\noalign{}
\endhead
\bottomrule\noalign{}
\endlastfoot
\textbf{Active Power (P)} & Actual power consumed, measured in watts
(W); P = VI cosθ \\
\textbf{Reactive Power (Q)} & Power oscillating between source and load,
measured in VAR; Q = VI sinθ \\
\textbf{Power Factor (PF)} & Ratio of active power to apparent power; PF
= cosθ \\
\end{longtable}
}

\textbf{Power Triangle:}

\begin{lstlisting}
                S (VA)
               /|
              / |
             /  |
            /   |
           /θ   |
          /_____|
         P(W)   Q(VAR)
\end{lstlisting}

\begin{itemize}
\tightlist
\item
  \textbf{Apparent Power (S)}: Vector sum of active and reactive power
\item
  \textbf{Power Triangle}: Right triangle with P, Q, and S as sides
\item
  \textbf{Power Factor}: cos θ = P/S (0 to 1)
\end{itemize}

\end{solutionbox}
\begin{mnemonicbox}
``Active Power Works, Reactive Power Waits''

\end{mnemonicbox}
\subsection*{Question 2(c) OR [7
marks]}\label{q2c}

\textbf{Explain the structure of atom of trivalent, tetravalent and
pentavalent elements.}

\begin{solutionbox}

\textbf{Atomic Structures:}

{\def\LTcaptype{none} % do not increment counter
\begin{longtable}[]{@{}
  >{\raggedright\arraybackslash}p{(\linewidth - 6\tabcolsep) * \real{0.2118}}
  >{\raggedright\arraybackslash}p{(\linewidth - 6\tabcolsep) * \real{0.2588}}
  >{\raggedright\arraybackslash}p{(\linewidth - 6\tabcolsep) * \real{0.1647}}
  >{\raggedright\arraybackslash}p{(\linewidth - 6\tabcolsep) * \real{0.3647}}@{}}
\toprule\noalign{}
\begin{minipage}[b]{\linewidth}\raggedright
\textbf{Element Type}
\end{minipage} & \begin{minipage}[b]{\linewidth}\raggedright
\textbf{Valence Electrons}
\end{minipage} & \begin{minipage}[b]{\linewidth}\raggedright
\textbf{Examples}
\end{minipage} & \begin{minipage}[b]{\linewidth}\raggedright
\textbf{Electronic Configuration}
\end{minipage} \\
\midrule\noalign{}
\endhead
\bottomrule\noalign{}
\endlastfoot
\textbf{Trivalent} & 3 & Boron, Aluminum, Gallium & 3 electrons in
outermost shell \\
\textbf{Tetravalent} & 4 & Carbon, Silicon, Germanium & 4 electrons in
outermost shell \\
\textbf{Pentavalent} & 5 & Nitrogen, Phosphorus, Arsenic & 5 electrons
in outermost shell \\
\end{longtable}
}

\textbf{Diagram:}

\includegraphics[width=1\linewidth,height=\textheight,keepaspectratio]{mermaid-c93a1fb6.pdf}

\begin{itemize}
\tightlist
\item
  \textbf{Trivalent Elements}: Used as p-type dopants in semiconductors
\item
  \textbf{Tetravalent Elements}: Form semiconductor base materials
\item
  \textbf{Pentavalent Elements}: Used as n-type dopants in
  semiconductors
\end{itemize}

\end{solutionbox}
\begin{mnemonicbox}
``Three Tries to Bond, Four Forms Full bonds, Five
Frees an Electron''

\end{mnemonicbox}
\subsection*{Question 3(a) [3 marks]}\label{q3a}

\textbf{Draw the symbol of photodiode and state it's application.}

\begin{solutionbox}

\textbf{Photodiode Symbol:}

\begin{lstlisting}
    --------|>|--------
             |
            / \
           /   \
\end{lstlisting}

\textbf{Applications of Photodiode:}

\begin{itemize}
\tightlist
\item
  Light sensors and detectors
\item
  Optical communication systems
\item
  Camera exposure controls
\item
  Barcode scanners
\item
  Medical instruments
\item
  Solar cells
\end{itemize}

\end{solutionbox}
\begin{mnemonicbox}
``Photons Produce Current''

\end{mnemonicbox}
\subsection*{Question 3(b) [4 marks]}\label{q3b}

\textbf{Write a Short note on LED.}

\begin{solutionbox}

\textbf{LED (Light Emitting Diode)}:

{\def\LTcaptype{none} % do not increment counter
\begin{longtable}[]{@{}
  >{\raggedright\arraybackslash}p{(\linewidth - 2\tabcolsep) * \real{0.4688}}
  >{\raggedright\arraybackslash}p{(\linewidth - 2\tabcolsep) * \real{0.5312}}@{}}
\toprule\noalign{}
\begin{minipage}[b]{\linewidth}\raggedright
\textbf{Parameter}
\end{minipage} & \begin{minipage}[b]{\linewidth}\raggedright
\textbf{Description}
\end{minipage} \\
\midrule\noalign{}
\endhead
\bottomrule\noalign{}
\endlastfoot
\textbf{Structure} & p-n junction with special doping materials \\
\textbf{Working} & Electrons recombine with holes, releasing energy as
photons \\
\textbf{Materials} & GaAs (red), GaP (green), GaN (blue), etc. \\
\textbf{Voltage} & Forward voltage typically 1.8V to 3.3V depending on
color \\
\end{longtable}
}

\textbf{Advantages}:

\begin{itemize}
\tightlist
\item
  High efficiency (low power consumption)
\item
  Long life (50,000+ hours)
\item
  Small size and durability
\item
  Various colors available
\end{itemize}

\textbf{Applications}:

\begin{itemize}
\tightlist
\item
  Indicators and displays
\item
  Lighting systems
\item
  TV/monitor backlights
\item
  Traffic signals
\end{itemize}

\end{solutionbox}
\begin{mnemonicbox}
``Light Emits when Diode conducts''

\end{mnemonicbox}
\subsection*{Question 3(c) [7 marks]}\label{q3c}

\textbf{Draw and explain VI characteristic of PN junction diode.}

\begin{solutionbox}

\textbf{P-N Junction Diode V-I Characteristic:}

\begin{lstlisting}
                        |
                        |         /
                        |        /
                        |       /
                        |      /
                        |     /
                        |    /
-----------+------------+---+----------
           |            |  /|
           |            | / |
           |            |/  |
           |            |   |
           |            |   |
           +            +   +
       Reverse       Origin Forward
       Region               Region
       
\end{lstlisting}

\textbf{Forward Bias Region:}

\begin{itemize}
\tightlist
\item
  \textbf{Knee Voltage}: 0.3V (Ge), 0.7V (Si) where current starts
  flowing
\item
  \textbf{Current Equation}: I = Is(e\^{}(qV/kT) - 1)
\item
  \textbf{Conductivity}: High (low resistance)
\end{itemize}

\textbf{Reverse Bias Region:}

\begin{itemize}
\tightlist
\item
  \textbf{Leakage Current}: Very small reverse current (micro-amps)
\item
  \textbf{Breakdown Region}: Sharp increase in current at breakdown
  voltage
\item
  \textbf{Conductivity}: Very low (high resistance)
\end{itemize}

\textbf{Key Points}:

\begin{itemize}
\tightlist
\item
  \textbf{Barrier Potential}: Decreases in forward bias, increases in
  reverse bias
\item
  \textbf{Diode Resistance}: Dynamic resistance changes with applied
  voltage
\item
  \textbf{Temperature Effect}: Voltage drop decreases with temperature
  increase
\end{itemize}

\end{solutionbox}
\begin{mnemonicbox}
``Forward Flows Freely, Reverse Resists''

\end{mnemonicbox}
\subsection*{Question 3(a) OR [3
marks]}\label{q3a}

\textbf{List the applications of PN junction diode.}

\begin{solutionbox}

\textbf{Applications of PN Junction Diode:}

{\def\LTcaptype{none} % do not increment counter
\begin{longtable}[]{@{}
  >{\raggedright\arraybackslash}p{(\linewidth - 2\tabcolsep) * \real{0.6500}}
  >{\raggedright\arraybackslash}p{(\linewidth - 2\tabcolsep) * \real{0.3500}}@{}}
\toprule\noalign{}
\begin{minipage}[b]{\linewidth}\raggedright
\textbf{Application Category}
\end{minipage} & \begin{minipage}[b]{\linewidth}\raggedright
\textbf{Examples}
\end{minipage} \\
\midrule\noalign{}
\endhead
\bottomrule\noalign{}
\endlastfoot
\textbf{Rectification} & Half-wave rectifier, Full-wave rectifier,
Bridge rectifier \\
\textbf{Signal Processing} & Signal demodulation, Clipping circuits,
Clamping circuits \\
\textbf{Protection} & Voltage spike protection, Reverse polarity
protection \\
\textbf{Logic Gates} & Diode logic circuits, Switching applications \\
\textbf{Voltage Regulation} & Zener diodes for voltage references \\
\textbf{Light Applications} & LEDs, Photodiodes, Solar cells \\
\end{longtable}
}

\end{solutionbox}
\begin{mnemonicbox}
``Rectify, Process, Protect, Logic, Regulate, Light''

\end{mnemonicbox}
\subsection*{Question 3(b) OR [4
marks]}\label{q3b}

\textbf{Explain the formation of depletion region in unbiased P-N
junction.}

\begin{solutionbox}

\textbf{Depletion Region Formation:}

\includegraphics[width=1\linewidth,height=\textheight,keepaspectratio]{mermaid-ed06d906.pdf}

\textbf{Process:}

\begin{itemize}
\tightlist
\item
  \textbf{Diffusion}: Electrons from n-side diffuse to p-side; holes
  from p-side diffuse to n-side
\item
  \textbf{Recombination}: Electrons and holes recombine at the junction
\item
  \textbf{Immobile Ions}: Exposed positive ions in n-region, negative
  ions in p-region
\item
  \textbf{Electric Field}: Forms between positive and negative ions,
  opposing further diffusion
\item
  \textbf{Equilibrium}: Diffusion current equals drift current; no net
  current flows
\end{itemize}

\textbf{Properties of Depletion Region:}

\begin{itemize}
\tightlist
\item
  No free charge carriers
\item
  Acts as insulator
\item
  Width depends on doping levels
\item
  Contains built-in potential barrier
\end{itemize}

\end{solutionbox}
\begin{mnemonicbox}
``Diffusion Depletes Carriers, Creating Electric
barrier''

\end{mnemonicbox}
\subsection*{Question 3(c) OR [7
marks]}\label{q3c}

\textbf{Explain construction, working and applications of PN junction
diode.}

\begin{solutionbox}

\textbf{Construction of PN Junction Diode:}

\begin{lstlisting}
    +--------+--------+
    |        |        |
    |  P-Type|N-Type  |
    |        |        |
    +--------+--------+
       |     |     |
       |Depletion|
       |  Region |
\end{lstlisting}

\begin{itemize}
\tightlist
\item
  \textbf{P-Type Region}: Silicon/Germanium doped with trivalent
  impurities (boron, aluminum)
\item
  \textbf{N-Type Region}: Silicon/Germanium doped with pentavalent
  impurities (phosphorus, arsenic)
\item
  \textbf{Junction}: Interface between p and n regions with depletion
  layer
\item
  \textbf{Terminals}: Anode (p-side) and Cathode (n-side)
\end{itemize}

\textbf{Working Principle:}

{\def\LTcaptype{none} % do not increment counter
\begin{longtable}[]{@{}
  >{\raggedright\arraybackslash}p{(\linewidth - 2\tabcolsep) * \real{0.5882}}
  >{\raggedright\arraybackslash}p{(\linewidth - 2\tabcolsep) * \real{0.4118}}@{}}
\toprule\noalign{}
\begin{minipage}[b]{\linewidth}\raggedright
\textbf{Bias Condition}
\end{minipage} & \begin{minipage}[b]{\linewidth}\raggedright
\textbf{Behavior}
\end{minipage} \\
\midrule\noalign{}
\endhead
\bottomrule\noalign{}
\endlastfoot
\textbf{Forward Bias} & Depletion region narrows, current flows when V
\textgreater{} 0.7V (Si) \\
\textbf{Reverse Bias} & Depletion region widens, only small leakage
current flows \\
\end{longtable}
}

\textbf{Applications:}

\begin{itemize}
\tightlist
\item
  Rectification in power supplies
\item
  Signal demodulation in radios
\item
  Voltage regulation (Zener)
\item
  Signal clipping and clamping
\item
  Logic gates and switching
\item
  Light emission and detection
\end{itemize}

\end{solutionbox}
\begin{mnemonicbox}
``Forward Flow, Reverse Restrict, Convert AC to DC''

\end{mnemonicbox}
\subsection*{Question 4(a) [3 marks]}\label{q4a}

\textbf{Define: (1) Ripple frequency (2) Ripple factor (3) PIV of a
diode.}

\begin{solutionbox}

{\def\LTcaptype{none} % do not increment counter
\begin{longtable}[]{@{}
  >{\raggedright\arraybackslash}p{(\linewidth - 2\tabcolsep) * \real{0.3846}}
  >{\raggedright\arraybackslash}p{(\linewidth - 2\tabcolsep) * \real{0.6154}}@{}}
\toprule\noalign{}
\begin{minipage}[b]{\linewidth}\raggedright
\textbf{Term}
\end{minipage} & \begin{minipage}[b]{\linewidth}\raggedright
\textbf{Definition}
\end{minipage} \\
\midrule\noalign{}
\endhead
\bottomrule\noalign{}
\endlastfoot
\textbf{Ripple Frequency} & The frequency of AC component present in
rectified DC output; for half-wave f = supply frequency, for full-wave f
= 2 \times supply frequency \\
\textbf{Ripple Factor (γ)} & Ratio of RMS value of AC component to DC
component in rectifier output; γ = Vac(rms)/Vdc \\
\textbf{PIV of Diode} & Peak Inverse Voltage - maximum reverse voltage a
diode can withstand without breakdown \\
\end{longtable}
}

\end{solutionbox}
\begin{mnemonicbox}
``Ripples Per second, Ripple Proportion, Reverse Peak
Voltage''

\end{mnemonicbox}
\subsection*{Question 4(b) [4 marks]}\label{q4b}

\textbf{Give comparison between full wave rectifier with two diodes and
full wave bridge rectifier.}

\begin{solutionbox}

{\def\LTcaptype{none} % do not increment counter
\begin{longtable}[]{@{}lll@{}}
\toprule\noalign{}
\textbf{Parameter} & \textbf{Center-Tapped Full Wave} & \textbf{Bridge
Rectifier} \\
\midrule\noalign{}
\endhead
\bottomrule\noalign{}
\endlastfoot
\textbf{Diodes Used} & 2 diodes & 4 diodes \\
\textbf{Transformer} & Center-tapped required & No center tap needed \\
\textbf{PIV of Diode} & 2Vm & Vm \\
\textbf{Output Voltage} & Vdc = 0.637Vm & Vdc = 0.637Vm \\
\textbf{Ripple Factor} & 0.48 & 0.48 \\
\textbf{Efficiency} & 81.2\% & 81.2\% \\
\textbf{TUF} & 0.693 & 0.693 \\
\end{longtable}
}

\textbf{Diagram:}

\includegraphics[width=1\linewidth,height=\textheight,keepaspectratio]{mermaid-14d609de.pdf}

\end{solutionbox}
\begin{mnemonicbox}
``Bridge Beats Tap with Lower PIV but Needs More
Diodes''

\end{mnemonicbox}
\subsection*{Question 4(c) [7 marks]}\label{q4c}

\textbf{Explain zener diode as voltage regulator.}

\begin{solutionbox}

\textbf{Zener Diode Voltage Regulator:}

\begin{lstlisting}
    Vin     Rs          
    +------|\/\/\|------+--------+ Vout
    |                   |        |
    |                   Z        RL
    |                   Z Zener  |
    |                   Z        |
    +-------------------+--------+
                        |
                       GND
\end{lstlisting}

\textbf{Working Principle:}

\begin{itemize}
\tightlist
\item
  \textbf{Reverse Biased}: Zener operates in breakdown region
\item
  \textbf{Constant Voltage}: Maintains fixed voltage (Vz) across its
  terminals
\item
  \textbf{Current Regulation}: Series resistor (Rs) limits current
\item
  \textbf{Load Changes}: When load current changes, Zener current
  changes to maintain constant output voltage
\end{itemize}

\textbf{Design Equations:}

\begin{itemize}
\tightlist
\item
  Rs = (Vin - Vz) / (IL + Iz)
\item
  Power rating of Zener: Pz = Vz \times Iz(max)
\end{itemize}

\textbf{Advantages:}

\begin{itemize}
\tightlist
\item
  Simple circuit
\item
  Low cost
\item
  Good regulation for small loads
\item
  Fast response to load changes
\end{itemize}

\textbf{Limitations:}

\begin{itemize}
\tightlist
\item
  Power wastage in Rs and Zener
\item
  Limited output current capability
\item
  Temperature dependence of Vz
\end{itemize}

\end{solutionbox}
\begin{mnemonicbox}
``Zener Stays at breakdown Voltage despite Current
changes''

\end{mnemonicbox}
\subsection*{Question 4(a) OR [3
marks]}\label{q4a}

\textbf{What is rectifier? Explain full wave rectifier with waveforms.}

\begin{solutionbox}

\textbf{Rectifier}: A circuit that converts AC voltage to pulsating DC
voltage by allowing current flow in one direction only.

\textbf{Full Wave Rectifier:}

\begin{lstlisting}
                    D1
   AC     +--------->|-------+
   Input  |                  |
   o------+                  +-----o
          |                  |     DC
          |                  |     Output
   o------+                  +-----o
          |                  |
          +--------|<--------+
                   D2
\end{lstlisting}

\textbf{Waveforms:}

\begin{lstlisting}
Input:    ^     ^     ^
          |     |     |
   0 -----+-----+-----+----
          |     |     |
          v     v     v

Output:   ^     ^     ^
          |     |     |
   0 -----+-----+-----+----
          
\end{lstlisting}

\begin{itemize}
\tightlist
\item
  \textbf{Operation}: Both half cycles of AC input are converted to same
  polarity
\item
  \textbf{Frequency}: Output ripple frequency is twice the input
  frequency
\item
  \textbf{Voltage}: Vdc = 0.637Vm (where Vm is peak input voltage)
\end{itemize}

\end{solutionbox}
\begin{mnemonicbox}
``Full Wave Forms Full Output''

\end{mnemonicbox}
\subsection*{Question 4(b) OR [4
marks]}\label{q4b}

\textbf{Why filter is required in rectifier? State the different types
of filter and explain any one type of filter.}

\begin{solutionbox}

\textbf{Need for Filters}: Rectifiers produce pulsating DC with large
ripples; filters smooth this output to provide steady DC voltage.

\textbf{Types of Filters:}

\begin{itemize}
\tightlist
\item
  Capacitor (C) filter
\item
  Inductor (L) filter
\item
  LC filter
\item
  π (Pi) filter
\item
  RC filter
\end{itemize}

\textbf{Capacitor Filter:}

\begin{lstlisting}
    +-------+-----+
    |       |     |
    |       C     RL
    |       |     |
    +-------+-----+
\end{lstlisting}

\textbf{Working:}

\begin{itemize}
\tightlist
\item
  Capacitor charges during voltage rise
\item
  Discharges slowly through load during voltage fall
\item
  Acts as temporary storage element
\item
  Time constant RC determines discharge rate
\item
  Reduces ripple by providing discharge path
\end{itemize}

\textbf{Advantages:}

\begin{itemize}
\tightlist
\item
  Simple and economical
\item
  Good smoothing for light loads
\item
  Increases DC output voltage
\end{itemize}

\end{solutionbox}
\begin{mnemonicbox}
``Capacitor Catches Charge and Releases Slowly''

\end{mnemonicbox}
\subsection*{Question 4(c) OR [7
marks]}\label{q4c}

\textbf{Write the need of rectifier. Explain bridge rectifier with
circuit diagram and draw its input and output waveforms.}

\begin{solutionbox}

\textbf{Need for Rectifiers:}

\begin{itemize}
\tightlist
\item
  Convert AC to DC for electronic devices
\item
  Power supplies for DC-operated equipment
\item
  Battery charging circuits
\item
  DC power for industrial drives
\item
  Signal demodulation in communication
\end{itemize}

\textbf{Bridge Rectifier Circuit:}

\begin{lstlisting}
           D1       D3
     +----->|----+--|>----+
     |             |      |
AC   |             |      | DC
Input|             |      | Output
     |             |      |
     +------|<----+--|<---+
            D2       D4
\end{lstlisting}

\textbf{Working Principle:}

\begin{itemize}
\tightlist
\item
  \textbf{Positive Half Cycle}: D1 and D4 conduct, D2 and D3 block
\item
  \textbf{Negative Half Cycle}: D2 and D3 conduct, D1 and D4 block
\item
  \textbf{Both Half Cycles}: Current flows in same direction through
  load
\end{itemize}

\textbf{Input-Output Waveforms:}

\begin{lstlisting}
Input:     ^      ^      ^
           |      |      |
    0 -----+------+------+-----
           |      |      |
           v      v      v

Output:    ^      ^      ^      ^      ^
           |      |      |      |      |
    0 ------+----+------+------+------+----
\end{lstlisting}

\textbf{Characteristics:}

\begin{itemize}
\tightlist
\item
  Vdc = 0.637Vm (Vm: peak input voltage)
\item
  PIV of each diode = Vm
\item
  Ripple factor = 0.48
\item
  Efficiency = 81.2\%
\item
  TUF = 0.693
\end{itemize}

\end{solutionbox}
\begin{mnemonicbox}
``Bridge Brings Both halves to Direct Current''

\end{mnemonicbox}
\subsection*{Question 5(a) [3 marks]}\label{q5a}

\textbf{Explain causes of electronic waste.}

\begin{solutionbox}

\textbf{Causes of Electronic Waste:}

{\def\LTcaptype{none} % do not increment counter
\begin{longtable}[]{@{}
  >{\raggedright\arraybackslash}p{(\linewidth - 2\tabcolsep) * \real{0.3929}}
  >{\raggedright\arraybackslash}p{(\linewidth - 2\tabcolsep) * \real{0.6071}}@{}}
\toprule\noalign{}
\begin{minipage}[b]{\linewidth}\raggedright
\textbf{Cause}
\end{minipage} & \begin{minipage}[b]{\linewidth}\raggedright
\textbf{Description}
\end{minipage} \\
\midrule\noalign{}
\endhead
\bottomrule\noalign{}
\endlastfoot
\textbf{Rapid Technology Change} & Frequent upgrades and obsolescence of
electronics \\
\textbf{Short Lifecycle} & Devices designed with limited useful life \\
\textbf{Consumer Behavior} & Preference for new gadgets over repair \\
\textbf{Manufacturing Issues} & Poor quality leading to early
failures \\
\textbf{Economic Factors} & Sometimes cheaper to replace than repair \\
\textbf{Marketing Strategies} & Promoting new models through planned
obsolescence \\
\end{longtable}
}

\end{solutionbox}
\begin{mnemonicbox}
``Upgrade, Use, Throw, Repeat''

\end{mnemonicbox}
\subsection*{Question 5(b) [4 marks]}\label{q5b}

\textbf{Compare PNP and NPN transistors.}

\begin{solutionbox}

{\def\LTcaptype{none} % do not increment counter
\begin{longtable}[]{@{}
  >{\raggedright\arraybackslash}p{(\linewidth - 4\tabcolsep) * \real{0.2727}}
  >{\raggedright\arraybackslash}p{(\linewidth - 4\tabcolsep) * \real{0.3636}}
  >{\raggedright\arraybackslash}p{(\linewidth - 4\tabcolsep) * \real{0.3636}}@{}}
\toprule\noalign{}
\begin{minipage}[b]{\linewidth}\raggedright
\textbf{Parameter}
\end{minipage} & \begin{minipage}[b]{\linewidth}\raggedright
\textbf{PNP Transistor}
\end{minipage} & \begin{minipage}[b]{\linewidth}\raggedright
\textbf{NPN Transistor}
\end{minipage} \\
\midrule\noalign{}
\endhead
\bottomrule\noalign{}
\endlastfoot
\textbf{Symbol} & & \\
\textbf{Current Flow} & Emitter to Collector & Collector to Emitter \\
\textbf{Majority Carriers} & Holes & Electrons \\
\textbf{Biasing} & Emitter positive, Collector negative & Collector
positive, Emitter negative \\
\textbf{Switching Speed} & Slower & Faster \\
\textbf{Usage} & Less common & More common \\
\end{longtable}
}

\end{solutionbox}
\begin{mnemonicbox}
``PNP: Positive to Negative to Positive; NPN:
Negative to Positive to Negative''

\end{mnemonicbox}
\subsection*{Question 5(c) [7 marks]}\label{q5c}

\textbf{Draw the symbol, explain the construction and working of
MOSFET.}

\begin{solutionbox}

\textbf{MOSFET Symbol (N-Channel Enhancement):}

\begin{lstlisting}
        D
        |
        |
    G---|
        |
        |
        S
\end{lstlisting}

\textbf{Construction:}

\includegraphics[width=1\linewidth,height=\textheight,keepaspectratio]{mermaid-ebefc790.pdf}

\textbf{Components:}

\begin{itemize}
\tightlist
\item
  \textbf{Substrate}: P-type semiconductor body
\item
  \textbf{Source/Drain}: Heavily doped n+ regions
\item
  \textbf{Gate}: Metal electrode separated by insulator (SiO2)
\item
  \textbf{Channel}: Forms between source and drain when biased
\end{itemize}

\textbf{Working Principle:}

\begin{itemize}
\tightlist
\item
  \textbf{Enhancement Mode}: No channel exists initially; gate voltage
  creates channel
\item
  \textbf{Threshold Voltage (VT)}: Minimum gate voltage needed to form
  channel
\item
  \textbf{Conducting State}: When VGS \textgreater{} VT, electrons form
  channel, allowing current flow
\item
  \textbf{Saturation Region}: Current remains constant despite increase
  in VDS
\item
  \textbf{Linear Region}: Current proportional to VDS at low drain
  voltages
\end{itemize}

\textbf{Applications:}

\begin{itemize}
\tightlist
\item
  Digital circuits (logic gates)
\item
  Power amplifiers
\item
  Switching applications
\item
  Memory devices
\end{itemize}

\end{solutionbox}
\begin{mnemonicbox}
``Gate Voltage Controls Electron Channel''

\end{mnemonicbox}
\subsection*{Question 5(a) OR [3
marks]}\label{q5a}

\textbf{Explain methods to handle electronic waste.}

\begin{solutionbox}

\textbf{Methods to Handle Electronic Waste:}

{\def\LTcaptype{none} % do not increment counter
\begin{longtable}[]{@{}
  >{\raggedright\arraybackslash}p{(\linewidth - 2\tabcolsep) * \real{0.4138}}
  >{\raggedright\arraybackslash}p{(\linewidth - 2\tabcolsep) * \real{0.5862}}@{}}
\toprule\noalign{}
\begin{minipage}[b]{\linewidth}\raggedright
\textbf{Method}
\end{minipage} & \begin{minipage}[b]{\linewidth}\raggedright
\textbf{Description}
\end{minipage} \\
\midrule\noalign{}
\endhead
\bottomrule\noalign{}
\endlastfoot
\textbf{Reduce} & Designing products with longer lifecycle and
upgradability \\
\textbf{Reuse} & Refurbishing and donating electronics for secondary
use \\
\textbf{Recycle} & Systematic disassembly to recover valuable
materials \\
\textbf{Responsible Disposal} & Proper collection and processing by
certified facilities \\
\textbf{Extended Producer Responsibility} & Manufacturers take back used
products \\
\textbf{Urban Mining} & Recovering precious metals from discarded
electronics \\
\end{longtable}
}

\textbf{Diagram:}

\includegraphics[width=1\linewidth,height=\textheight,keepaspectratio]{mermaid-53166b1b.pdf}

\end{solutionbox}
\begin{mnemonicbox}
``Reduce, Reuse, Recycle, Recover Resources''

\end{mnemonicbox}
\subsection*{Question 5(b) OR [4
marks]}\label{q5b}

\textbf{Derive the relationship between αdc and βdc.}

\begin{solutionbox}

\textbf{Relationship between α and β:}

\textbf{Given:}

\begin{itemize}
\tightlist
\item
  αdc = IC/IE (Common base current gain)
\item
  βdc = IC/IB (Common emitter current gain)
\end{itemize}

\textbf{Derivation:} From Kirchhoff's current law: IE = IC + IB

Dividing both sides by IC: IE/IC = 1 + IB/IC

Since αdc = IC/IE: 1/αdc = 1 + IB/IC

Since βdc = IC/IB: 1/αdc = 1 + 1/βdc

\textbf{Final Relations:}

\begin{itemize}
\tightlist
\item
  αdc = βdc/(1 + βdc)
\item
  βdc = αdc/(1 - αdc)
\end{itemize}

\textbf{Table: } \textbar{} \textbf{α Value} \textbar{} \textbf{β Value}
\textbar{} \textbar-------------\textbar-------------\textbar{}
\textbar{} 0.9 \textbar{} 9 \textbar{} \textbar{} 0.95 \textbar{} 19
\textbar{} \textbar{} 0.99 \textbar{} 99 \textbar{}

\end{solutionbox}
\begin{mnemonicbox}
``Alpha approaches One as Beta approaches Infinity''

\end{mnemonicbox}
\subsection*{Question 5(c) OR [7
marks]}\label{q5c}

\textbf{Explain common collector configuration with its input and output
characteristics.}

\begin{solutionbox}

\textbf{Common Collector (Emitter Follower) Configuration:}

\begin{lstlisting}
                   +Vcc
                    |
                    |
                    R
                    |
                    |
    +------+--------+-------+
    |      |                |
    |    B |    C           E
    +-----|       |-------+-+
           |      |       |
         --+      +--     R
           |                |
           |                |
         +-+-+              |
         |   |              |
         GND GND            GND
\end{lstlisting}

\textbf{Input Characteristics:}

\begin{lstlisting}
   Ib
   ^
   |      -------
   |     /
   |    /
   |   /
   |  /
   | /
   |/
   +-----------------> Vbe
\end{lstlisting}

\textbf{Output Characteristics:}

\begin{lstlisting}
   Ie
   ^
   |       --------
   |      /
   |     /
   |    /
   |   /
   |  /
   | /
   |/
   +-----------------> Vce
\end{lstlisting}

\textbf{Key Features:}

\begin{itemize}
\tightlist
\item
  \textbf{Voltage Gain (Av)}: Approximately 1 (unity)
\item
  \textbf{Current Gain (Ai)}: High (β + 1)
\item
  \textbf{Input Impedance}: High (β \times RE)
\item
  \textbf{Output Impedance}: Low (1/gm) where gm is transconductance
\item
  \textbf{Phase Relationship}: No phase inversion between input and
  output
\item
  \textbf{Applications}: Impedance matching, buffers, voltage regulators
\end{itemize}

\textbf{Characteristics:}

\begin{itemize}
\tightlist
\item
  \textbf{Input Resistance}: Ri = β \times (re + RL)
\item
  \textbf{Output Resistance}: Ro = (rs + re)/(β + 1)
\item
  \textbf{Voltage Gain}: Av = RL/(RL + re) \approx 1
\item
  \textbf{Current Gain}: Ai = (β + 1)
\end{itemize}

\textbf{Advantages:}

\begin{itemize}
\tightlist
\item
  Very high input impedance
\item
  Low output impedance
\item
  Good impedance matching properties
\item
  No phase inversion
\end{itemize}

\textbf{Limitations:}

\begin{itemize}
\tightlist
\item
  No voltage gain (slightly less than 1)
\item
  Used only for impedance matching
\end{itemize}

\end{solutionbox}
\begin{mnemonicbox}
``Collector Common, Current amplifies, Voltage
follows''

This completes the full solutions for the Elements of Electrical \&
Electronics Engineering (1313202) Winter 2023 examination.

\end{mnemonicbox}

\end{document}
