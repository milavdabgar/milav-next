\documentclass[10pt,a4paper]{article}

% content/resources/templates/preamble.tex
\usepackage[margin=0.6in]{geometry}
\author{Milav Dabgar}
\usepackage{amsmath,amssymb,amsthm}
\usepackage{booktabs}
\usepackage{multirow}
\usepackage{xcolor}
\usepackage{tcolorbox}
\tcbuselibrary{breakable,skins}
\usepackage[colorlinks=true,linkcolor=blue]{hyperref}
\usepackage{titlesec}
\usepackage{enumitem}
\usepackage{tikz}
\usepackage{pgfplots}
\usepackage{circuitikz}
\usepackage[version=4]{mhchem}
\usepackage{longtable}
\usepackage{array}
\usepackage{float}
\usepackage{caption}
\usepackage{listings}

\lstset{
  basicstyle=\small\ttfamily,
  breaklines=true,
  breakatwhitespace=false,
  postbreak=\mbox{\textcolor{red}{$\hookrightarrow$}\space},
  float=false,
  numbers=left,
  numberstyle=\tiny\color{gray},
  numbersep=10pt,
  xleftmargin=2em,
  keywordstyle=\color{blue},
  commentstyle=\color{green!60!black},
  stringstyle=\color{purple},
  backgroundcolor=\color{gray!5},
  showstringspaces=false,
  tabsize=2,
  captionpos=b,
  keepspaces=true,
  columns=flexible
}

\pgfplotsset{compat=1.18}
\usetikzlibrary{shapes,arrows,positioning,calc,patterns,decorations.pathmorphing,decorations.markings,arrows.meta}

% Color scheme
\definecolor{headcolor}{RGB}{0,102,204}
\definecolor{keycolor}{RGB}{220,20,60}
\definecolor{solutioncolor}{RGB}{34,139,34}
\definecolor{mnemoniccolor}{RGB}{148,0,211}
\definecolor{codecolor}{RGB}{0,0,100}

% Spacing
\setlength{\parskip}{3pt}
\setlist[itemize]{nosep}
\setlist[enumerate]{nosep}

% Title formatting
\titleformat{\section}{\Large\bfseries\color{headcolor}}{\thesection}{1em}{}
\titleformat{\subsection}{\large\bfseries\color{headcolor}}{\thesubsection}{1em}{}

% Pandoc tightlist compatibility
\providecommand{\tightlist}{%
  \setlength{\itemsep}{0pt}\setlength{\parskip}{0pt}}

% Pandoc longtable compatibility
\newcounter{none}
\def\thenone{}


% content/resources/templates/gujarati-boxes.tex
\usepackage{fontspec}
\usepackage{polyglossia}

% Set Gujarati as main language (document is primarily in Gujarati)
% Note: gloss-gujarati.ldf doesn't exist in polyglossia, but it will use hyphenation patterns
\setdefaultlanguage{gujarati}
\setotherlanguage{english}

% Configure Gujarati font properly
% Use Language=Default to prevent polyglossia from trying to add language-specific features
% that don't exist for Gujarati, which causes "empty feature" warnings
\newfontfamily\gujaratifont[Script=Gujarati,AutoFakeBold=2.5,AutoFakeSlant=0.3]{Noto Sans Gujarati}
\setmainfont[Script=Gujarati,AutoFakeBold=2.5,AutoFakeSlant=0.3]{Noto Sans Gujarati}
% Use Noto Sans Gujarati for monospace to support Gujarati in text
\setmonofont[Scale=0.9]{Noto Sans Gujarati}

% Configure English to use the same font
\newfontfamily\englishfont[Script=Gujarati,AutoFakeBold=2.5,AutoFakeSlant=0.3]{Noto Sans Gujarati}

% Translations for polyglossia
\gappto\captionsgujarati{
  \renewcommand{\tablename}{કોષ્ટક}
  \renewcommand{\figurename}{આકૃતિ}
}

% Helper for TikZ nodes to ensure Gujarati font
\newcommand{\gu}[1]{{\gujaratifont #1}}

% Custom environments
\newtcolorbox{solutionbox}{
    breakable,
    enhanced,
    colback=solutioncolor!5!white,
    colframe=solutioncolor!75!black,
    fonttitle=\bfseries,
    title=જવાબ
}

\newtcolorbox{solutionboxnobreak}{
 colback=solutioncolor!5!white,
 colframe=solutioncolor!75!black,
 fonttitle=\bfseries,
 title=જવાબ
}

\newtcolorbox{keyformula}{
 breakable,
 enhanced,
 colback=keycolor!5!white,
 colframe=keycolor!75!black,
 fonttitle=\bfseries,
 title=રાસાયણિક સમીકરણ/સૂત્ર
}

\newtcolorbox{mnemonicbox}{
 breakable,
 enhanced,
 colback=mnemoniccolor!5!white,
 colframe=mnemoniccolor!75!black,
 fonttitle=\bfseries,
 title=મેમરી ટ્રીક
}


\begin{document}

\begin{center}
{\Huge\bfseries\color{headcolor} Subject Name (Gujarati)}\\[5pt]
{\LARGE 1313202 -- Winter 2024}\\[3pt]
{\large Semester 1 Study Material}\\[3pt]
{\normalsize\textit{Detailed Solutions and Explanations}}
\end{center}

\vspace{10pt}

\subsection*{પ્રશ્ન 1(અ) [3
માર્ક્સ]}\label{uxaaauxab0uxab6uxaa8-1uxa85-3-uxaaeuxab0uxa95uxab8}

\textbf{એક્ટિવ અને પેસિવ નેટવર્ક નો તફાવત સમજાવો.}

\begin{solutionbox}

{\def\LTcaptype{none} % do not increment counter
\begin{longtable}[]{@{}
  >{\raggedright\arraybackslash}p{(\linewidth - 2\tabcolsep) * \real{0.4750}}
  >{\raggedright\arraybackslash}p{(\linewidth - 2\tabcolsep) * \real{0.5250}}@{}}
\toprule\noalign{}
\begin{minipage}[b]{\linewidth}\raggedright
\textbf{એક્ટિવ નેટવર્ક}
\end{minipage} & \begin{minipage}[b]{\linewidth}\raggedright
\textbf{પેસિવ નેટવર્ક}
\end{minipage} \\
\midrule\noalign{}
\endhead
\bottomrule\noalign{}
\endlastfoot
ઓછામાં ઓછું એક એક્ટિવ ઘટક (વોલ્ટેજ/કરંટ સ્ત્રોત) ધરાવે છે & માત્ર પેસિવ ઘટકો (R, L,
C) ધરાવે છે \\
સર્કિટમાં ઊર્જા આપી શકે છે & સર્કિટમાં ઊર્જા આપી શકતું નથી \\
સિગ્નલ પાવરને વધારી શકે છે & સિગ્નલ પાવરને વધારી શકતું નથી \\
\end{longtable}
}

\end{solutionbox}
\begin{mnemonicbox}
``એક્ટિવ ઊર્જા આપે, પેસિવ ઊર્જા લે''

\end{mnemonicbox}
\subsection*{પ્રશ્ન 1(બ) [4
માર્ક્સ]}\label{uxaaauxab0uxab6uxaa8-1uxaac-4-uxaaeuxab0uxa95uxab8}

\textbf{કિર્ચોફનો વોલ્ટેજનો નિયમ જણાવો અને સમજાવો.}

\begin{solutionbox}

કિર્ચોફનો વોલ્ટેજનો નિયમ (KVL) કહે છે કે સર્કિટમાં કોઈપણ બંધ લૂપની અંદર બધા વોલ્ટેજનો
બીજગણિતીય સરવાળો શૂન્ય થાય છે.

\textbf{આકૃતિ:}

\includegraphics[width=1\linewidth,height=\textheight,keepaspectratio]{mermaid-28c30a58.pdf}

ગણિતશાસ્ત્ર મુજબ: V1 + V2 + V3 + V4 = 0

\begin{itemize}
\tightlist
\item
  \textbf{વોલ્ટેજ ડ્રોપ}: જ્યારે કરંટની દિશામાં રેઝિસ્ટર વાટે પસાર થતાં વોલ્ટેજ નેગેટિવ
  છે
\item
  \textbf{વોલ્ટેજ વધારો}: જ્યારે નેગેટિવથી પોઝિટિવ તરફ સ્ત્રોત વાટે પસાર થતાં
  વોલ્ટેજ પોઝિટિવ છે
\end{itemize}

\end{solutionbox}
\begin{mnemonicbox}
``વોલ્ટેજ લૂપનો સરવાળો શૂન્ય''

\end{mnemonicbox}
\subsection*{પ્રશ્ન 1(ક) [7
માર્ક્સ]}\label{uxaaauxab0uxab6uxaa8-1uxa95-7-uxaaeuxab0uxa95uxab8}

\textbf{વ્યાખ્યા આપો: (1) ચાર્જ (2) કરંટ (3) પોટેન્શિયલ (4) E.M.F. (5) ઇન્ડક્ટન્સ
(6) કેપેસિટન્સ (7) આવૃત્તિ.}

\begin{solutionbox}

{\def\LTcaptype{none} % do not increment counter
\begin{longtable}[]{@{}
  >{\raggedright\arraybackslash}p{(\linewidth - 2\tabcolsep) * \real{0.3846}}
  >{\raggedright\arraybackslash}p{(\linewidth - 2\tabcolsep) * \real{0.6154}}@{}}
\toprule\noalign{}
\begin{minipage}[b]{\linewidth}\raggedright
\textbf{શબ્દ}
\end{minipage} & \begin{minipage}[b]{\linewidth}\raggedright
\textbf{વ્યાખ્યા}
\end{minipage} \\
\midrule\noalign{}
\endhead
\bottomrule\noalign{}
\endlastfoot
\textbf{ચાર્જ} & કૂલમ્બ (C)માં માપવામાં આવતો વીજળીનો જથ્થો \\
\textbf{કરંટ} & એમ્પિયર (A)માં માપવામાં આવતો વીજળીના ચાર્જનો પ્રવાહ દર \\
\textbf{પોટેન્શિયલ} & વોલ્ટ (V)માં માપવામાં આવતું એકમ ચાર્જ દીઠ વીજળીય દબાણ અથવા
ઊર્જા \\
\textbf{E.M.F.} & ઇલેક્ટ્રોમોટિવ ફોર્સ એટલે એકમ ચાર્જ દીઠ સ્ત્રોત દ્વારા પ્રદાન
કરેલી ઊર્જા, જે વોલ્ટ (V)માં માપવામાં આવે છે \\
\textbf{ઇન્ડક્ટન્સ} & હેનરી (H)માં માપવામાં આવતો વીજળીય સર્કિટનો ગુણ જે કરંટમાં
ફેરફારનો વિરોધ કરે છે \\
\textbf{કેપેસિટન્સ} & ફેરડ (F)માં માપવામાં આવતી કોઈ વસ્તુની વીજળીય ચાર્જ સંગ્રહ
કરવાની ક્ષમતા \\
\textbf{આવૃત્તિ} & હર્ટ્ઝ (Hz)માં માપવામાં આવતી પ્રતિ સેકન્ડ પૂર્ણ થયેલા ચક્રોની
સંખ્યા \\
\end{longtable}
}

\end{solutionbox}
\begin{mnemonicbox}
``ચાર્જનો પ્રવાહ દબાણથી ઊર્જા ઇન્ડ્યુસ કરે કેપેસિટિવ
ફ્લક્ચ્યુએશન''

\end{mnemonicbox}
\subsection*{પ્રશ્ન 1(ક) OR [7
માર્ક્સ]}\label{uxaaauxab0uxab6uxaa8-1uxa95-or-7-uxaaeuxab0uxa95uxab8}

\textbf{ઓહમનો નિયમ જણાવો. તેના ઉપયોગો અને મર્યાદા લખો.}

\begin{solutionbox}

ઓહમનો નિયમ કહે છે કે વાહક દ્વારા વહેતો કરંટ પોટેન્શિયલ તફાવતના સમપ્રમાણમાં અને
અવરોધના વ્યસ્ત પ્રમાણમાં હોય છે.

\textbf{આકૃતિ:}

\begin{lstlisting}
V = I \times R
\end{lstlisting}

જ્યાં:

\begin{itemize}
\tightlist
\item
  V = વોલ્ટેજ (વોલ્ટ)
\item
  I = કરંટ (એમ્પિયર)
\item
  R = અવરોધ (ઓહમ)
\end{itemize}

\textbf{ઉપયોગો:}

\begin{itemize}
\tightlist
\item
  સર્કિટ ડિઝાઇન અને વિશ્લેષણ
\item
  પાવર વપરાશની ગણતરીઓ
\item
  ઘટક મૂલ્ય નક્કી કરવા
\item
  વોલ્ટેજ ડિવાઇડર નેટવર્ક
\item
  કરંટ ડિવાઇડર નેટવર્ક
\end{itemize}

\textbf{મર્યાદાઓ:}

\begin{itemize}
\tightlist
\item
  માત્ર લીનિયર ઘટકો માટે માન્ય
\item
  નોન-ઓહમિક ઉપકરણો (ડાયોડ, ટ્રાન્ઝિસ્ટર) માટે લાગુ પડતો નથી
\item
  ઉચ્ચ તાપમાને અમાન્ય
\item
  સેમિકન્ડક્ટર્સ માટે માન્ય નથી
\item
  નોન-લીનિયર રેઝિસ્ટિવ ઘટકો માટે લાગુ કરી શકાતું નથી
\end{itemize}

\end{solutionbox}
\begin{mnemonicbox}
``વોલ્ટેજ કરંટને અવરોધ દ્વારા નિયંત્રિત કરે''

\end{mnemonicbox}
\subsection*{પ્રશ્ન 2(અ) [3
માર્ક્સ]}\label{uxaaauxab0uxab6uxaa8-2uxa85-3-uxaaeuxab0uxa95uxab8}

\textbf{વાહક, અવાહક અને અર્ધવાહક નો એનર્જી બેન્ડ ની આકૃતિ દોરી સમજાવો.}

\begin{solutionbox}

\textbf{આકૃતિ:}

\includegraphics[width=1\linewidth,height=\textheight,keepaspectratio]{mermaid-4adf126a.pdf}

\begin{itemize}
\tightlist
\item
  \textbf{વાહક}: વેલેન્સ અને કન્ડક્શન બેન્ડ ઓવરલેપ થાય છે, જે ઇલેક્ટ્રોનને મુક્ત રીતે
  ફરવાની મંજૂરી આપે છે
\item
  \textbf{અર્ધવાહક}: બેન્ડ વચ્ચે નાની ઊર્જા ગેપ (0.7-3 eV) મર્યાદિત કન્ડક્શનને મંજૂરી
  આપે છે
\item
  \textbf{અવાહક}: મોટી ઊર્જા ગેપ (\textgreater3 eV) ઇલેક્ટ્રોનને કન્ડક્શન બેન્ડમાં
  જતાં અટકાવે છે
\end{itemize}

\end{solutionbox}
\begin{mnemonicbox}
``વાહક ઓવરલેપ, અર્ધવાહક નાનો ગેપ કૂદે, અવાહક બ્લોક કરે''

\end{mnemonicbox}
\subsection*{પ્રશ્ન 2(બ) [4
માર્ક્સ]}\label{uxaaauxab0uxab6uxaa8-2uxaac-4-uxaaeuxab0uxa95uxab8}

\textbf{Maximum power transfer theorem અને reciprocity theorem નું સ્ટેટમેન્ટ
લખો.}

\begin{solutionbox}

{\def\LTcaptype{none} % do not increment counter
\begin{longtable}[]{@{}
  >{\raggedright\arraybackslash}p{(\linewidth - 2\tabcolsep) * \real{0.4643}}
  >{\raggedright\arraybackslash}p{(\linewidth - 2\tabcolsep) * \real{0.5357}}@{}}
\toprule\noalign{}
\begin{minipage}[b]{\linewidth}\raggedright
\textbf{થિયરમ}
\end{minipage} & \begin{minipage}[b]{\linewidth}\raggedright
\textbf{સ્ટેટમેન્ટ}
\end{minipage} \\
\midrule\noalign{}
\endhead
\bottomrule\noalign{}
\endlastfoot
\textbf{મેક્સિમમ પાવર ટ્રાન્સફર થિયરમ} & સ્ત્રોતથી લોડમાં મહત્તમ પાવર ત્યારે
ટ્રાન્સફર થાય છે જ્યારે લોડ રેઝિસ્ટન્સ સ્ત્રોતના આંતરિક અવરોધ જેટલો હોય (RL = RS) \\
\textbf{રેસિપ્રોસિટી થિયરમ} & લીનિયર, બાઇલેટરલ નેટવર્કમાં, જો બ્રાન્ચ 1માં વોલ્ટેજ
સ્ત્રોત E બ્રાન્ચ 2માં કરંટ I ઉત્પન્ન કરે છે, તો એ જ વોલ્ટેજ સ્ત્રોત E બ્રાન્ચ 2માં
મૂકવાથી બ્રાન્ચ 1માં એ જ કરંટ I ઉત્પન્ન થશે \\
\end{longtable}
}

\end{solutionbox}
\begin{mnemonicbox}
``અવરોધ મેળવો મહત્તમ પાવર માટે; સ્ત્રોત બદલો, કરંટ એક સરખો
રહે''

\end{mnemonicbox}
\subsection*{પ્રશ્ન 2(ક) [7
માર્ક્સ]}\label{uxaaauxab0uxab6uxaa8-2uxa95-7-uxaaeuxab0uxa95uxab8}

\textbf{N-type મટીરીઅલ ની રચના અને તેનું કંડક્શન સમજાવો.}

\begin{solutionbox}

\textbf{આકૃતિ:}

\includegraphics[width=1\linewidth,height=\textheight,keepaspectratio]{mermaid-e38d7d1b.pdf}

\begin{itemize}
\tightlist
\item
  \textbf{રચના પ્રક્રિયા}:

  \begin{itemize}
  \tightlist
  \item
    શુદ્ધ સિલિકોન/જર્મેનિયમમાં પેન્ટાવેલેન્ટ અશુદ્ધિ અણુઓ (P, As, Sb) ઉમેરવામાં આવે છે
  \item
    અશુદ્ધિ અણુઓમાં 5 વેલેન્સ ઇલેક્ટ્રોન હોય છે (સિલિકોનમાં 4 હોય છે)
  \item
    ચાર ઇલેક્ટ્રોન કોવેલેન્ટ બોન્ડ બનાવે છે, પાંચમો ફ્રી ઇલેક્ટ્રોન બને છે
  \item
    વધારાના નેગેટિવ ચાર્જ કેરિયર્સ બનાવે છે
  \end{itemize}
\item
  \textbf{કંડક્શન મેકેનિઝમ}:

  \begin{itemize}
  \tightlist
  \item
    મેજોરિટી કેરિયર્સ: ઇલેક્ટ્રોન
  \item
    માઇનોરિટી કેરિયર્સ: હોલ્સ
  \item
    ઇલેક્ટ્રોનની ગતિ વીજળીય કંડક્શન પ્રદાન કરે છે
  \item
    રૂમ ટેમ્પરેચર પર પણ, ફ્રી ઇલેક્ટ્રોન કરંટ પ્રવાહને સક્ષમ બનાવે છે
  \end{itemize}
\end{itemize}

\end{solutionbox}
\begin{mnemonicbox}
``પેન્ટાવેલેન્ટ એક વધારાનો ઇલેક્ટ્રોન આપે''

\end{mnemonicbox}
\subsection*{પ્રશ્ન 2(અ) OR [3
માર્ક્સ]}\label{uxaaauxab0uxab6uxaa8-2uxa85-or-3-uxaaeuxab0uxa95uxab8}

\textbf{વેલેન્સ બેન્ડ, કંડક્શન બેન્ડ અને ફોર્બિડન ગેપ ની વ્યાખ્યા આપો.}

\begin{solutionbox}

{\def\LTcaptype{none} % do not increment counter
\begin{longtable}[]{@{}
  >{\raggedright\arraybackslash}p{(\linewidth - 2\tabcolsep) * \real{0.3846}}
  >{\raggedright\arraybackslash}p{(\linewidth - 2\tabcolsep) * \real{0.6154}}@{}}
\toprule\noalign{}
\begin{minipage}[b]{\linewidth}\raggedright
\textbf{શબ્દ}
\end{minipage} & \begin{minipage}[b]{\linewidth}\raggedright
\textbf{વ્યાખ્યા}
\end{minipage} \\
\midrule\noalign{}
\endhead
\bottomrule\noalign{}
\endlastfoot
\textbf{વેલેન્સ બેન્ડ} & ઊર્જા બેન્ડ જેમાં વેલેન્સ ઇલેક્ટ્રોન હોય છે જે ઘન પદાર્થમાં ચોક્કસ
અણુઓ સાથે બંધાયેલા હોય છે \\
\textbf{કંડક્શન બેન્ડ} & ઉચ્ચ ઊર્જા બેન્ડ જ્યાં ઇલેક્ટ્રોન સમગ્ર પદાર્થમાં મુક્તપણે હરીફરી
શકે છે, જે વીજળીય કંડક્શન સક્ષમ બનાવે છે \\
\textbf{ફોર્બિડન ગેપ} & વેલેન્સ અને કંડક્શન બેન્ડ વચ્ચેનો ઊર્જા પ્રદેશ જ્યાં કોઈ ઇલેક્ટ્રોન
સ્ટેટ હોતા નથી \\
\end{longtable}
}

\end{solutionbox}
\begin{mnemonicbox}
``વેલેન્સ બાંધે, કંડક્શન વહાવે, ફોર્બિડન રોકે''

\end{mnemonicbox}
\subsection*{પ્રશ્ન 2(બ) OR [4
માર્ક્સ]}\label{uxaaauxab0uxab6uxaa8-2uxaac-or-4-uxaaeuxab0uxa95uxab8}

\textbf{એક્ટિવ પાવર, રિએક્ટિવ પાવર અને પાવર ફેક્ટર ની વ્યાખ્યા આપો અને પાવર
ત્રિકોણ દોરો.}

\begin{solutionbox}

\textbf{આકૃતિ:}

\begin{lstlisting}
    |    
    |   S (Apparent Power)
    |  /|
    | / |
    |/__|
    P   Q

P = Active Power
Q = Reactive Power
S = Apparent Power
cosθ = Power Factor
\end{lstlisting}

\begin{itemize}
\tightlist
\item
  \textbf{એક્ટિવ પાવર (P)}: વાસ્તવિક વપરાયેલો પાવર, વોટ (W)માં માપવામાં આવે છે,
  P = VI cosθ
\item
  \textbf{રિએક્ટિવ પાવર (Q)}: સ્ત્રોત અને લોડ વચ્ચે આગળ-પાછળ થતો પાવર,
  વોલ્ટ-એમ્પિયર રિએક્ટિવ (VAR)માં માપવામાં આવે છે, Q = VI sinθ
\item
  \textbf{પાવર ફેક્ટર}: એક્ટિવ પાવરનો એપેરન્ટ પાવર સાથેનો ગુણોત્તર, PF = cosθ =
  P/S
\end{itemize}

\end{solutionbox}
\begin{mnemonicbox}
``વાસ્તવિક પાવર કામ કરે, રિએક્ટિવ પાવર રાહ જુએ''

\end{mnemonicbox}
\subsection*{પ્રશ્ન 2(ક) OR [7
માર્ક્સ]}\label{uxaaauxab0uxab6uxaa8-2uxa95-or-7-uxaaeuxab0uxa95uxab8}

\textbf{ટ્રાઇવેલેન્ટ, ટેટ્રાવેલેન્ટ અને પેન્ટાવેલેન્ટ તત્વોના અણુની રચના સમજાવો.}

\begin{solutionbox}

\textbf{આકૃતિ:}

\includegraphics[width=1\linewidth,height=\textheight,keepaspectratio]{mermaid-485bf1f9.pdf}

{\def\LTcaptype{none} % do not increment counter
\begin{longtable}[]{@{}
  >{\raggedright\arraybackslash}p{(\linewidth - 6\tabcolsep) * \real{0.2535}}
  >{\raggedright\arraybackslash}p{(\linewidth - 6\tabcolsep) * \real{0.2113}}
  >{\raggedright\arraybackslash}p{(\linewidth - 6\tabcolsep) * \real{0.1972}}
  >{\raggedright\arraybackslash}p{(\linewidth - 6\tabcolsep) * \real{0.3380}}@{}}
\toprule\noalign{}
\begin{minipage}[b]{\linewidth}\raggedright
\textbf{તત્વનો પ્રકાર}
\end{minipage} & \begin{minipage}[b]{\linewidth}\raggedright
\textbf{રચના}
\end{minipage} & \begin{minipage}[b]{\linewidth}\raggedright
\textbf{ઉદાહરણો}
\end{minipage} & \begin{minipage}[b]{\linewidth}\raggedright
\textbf{સેમિકન્ડક્ટર ઉપયોગ}
\end{minipage} \\
\midrule\noalign{}
\endhead
\bottomrule\noalign{}
\endlastfoot
\textbf{ટ્રાઇવેલેન્ટ} & સૌથી બહારના શેલમાં 3 ઇલેક્ટ્રોન & B, Al, Ga, In & P-ટાઇપ
ડોપન્ટ \\
\textbf{ટેટ્રાવેલેન્ટ} & સૌથી બહારના શેલમાં 4 ઇલેક્ટ્રોન & Si, Ge, C & સેમિકન્ડક્ટર
બેઝ \\
\textbf{પેન્ટાવેલેન્ટ} & સૌથી બહારના શેલમાં 5 ઇલેક્ટ્રોન & P, As, Sb & N-ટાઇપ
ડોપન્ટ \\
\end{longtable}
}

\end{solutionbox}
\begin{mnemonicbox}
``ત્રણ સ્વીકારે, ચાર બનાવે, પાંચ આપે''

\end{mnemonicbox}
\subsection*{પ્રશ્ન 3(અ) [3
માર્ક્સ]}\label{uxaaauxab0uxab6uxaa8-3uxa85-3-uxaaeuxab0uxa95uxab8}

\textbf{ફોટોડિઓડનું પ્રતીક દોરો અને તેનો ઉપયોગ જણાવો.}

\begin{solutionbox}

\textbf{આકૃતિ:}

\begin{lstlisting}
    |\ 
    | \  
    |  \   
-->|| \--->
    |  /     
    | /   
    |/  
\end{lstlisting}

\textbf{ફોટોડિઓડના ઉપયોગો:}

\begin{itemize}
\tightlist
\item
  લાઇટ સેન્સર અને ડિટેક્ટર્સ
\item
  ઓપ્ટિકલ કમ્યુનિકેશન સિસ્ટમ્સ
\item
  સોલર સેલ્સ અને ફોટોવોલ્ટેઇક એપ્લિકેશન્સ
\item
  કેમેરા એક્સપોઝર કંટ્રોલ્સ
\item
  મેડિકલ ઉપકરણો (પલ્સ ઓક્સિમીટર)
\end{itemize}

\end{solutionbox}
\begin{mnemonicbox}
``પ્રકાશ કરંટને ઉત્તેજિત કરે''

\end{mnemonicbox}
\subsection*{પ્રશ્ન 3(બ) [4
માર્ક્સ]}\label{uxaaauxab0uxab6uxaa8-3uxaac-4-uxaaeuxab0uxa95uxab8}

\textbf{LED પર ટૂંકી નોંધ લખો.}

\begin{solutionbox}

\textbf{આકૃતિ:}

\begin{lstlisting}
    |\ 
    | \  
    |  \   
<---|| \--->
    |  /     
    | /   
    |/  
    ▼ ▼
   Light
\end{lstlisting}

\begin{itemize}
\tightlist
\item
  \textbf{રચના}: ફોરવર્ડ બાયસ થયેલ હોય ત્યારે પ્રકાશ ઉત્સર્જિત કરતો P-N જંક્શન
  ડાયોડ
\item
  \textbf{કાર્ય સિદ્ધાંત}: ઇલેક્ટ્રોન-હોલ રીકોમ્બિનેશન ફોટોન્સના રૂપમાં ઊર્જા છોડે છે
\item
  \textbf{પ્રકારો}: સેમિકન્ડક્ટર મટીરિયલ (GaAs, GaP, GaN) પર આધારિત વિવિધ
  રંગો
\item
  \textbf{ફાયદાઓ}: ઓછો પાવર વપરાશ, લાંબી લાઇફ, નાનું કદ, ઝડપી સ્વિચિંગ
\item
  \textbf{ઉપયોગો}: ડિસ્પ્લે, ઇન્ડિકેટર્સ, લાઇટિંગ, રિમોટ કંટ્રોલ, ઓપ્ટિકલ કમ્યુનિકેશન
\end{itemize}

\end{solutionbox}
\begin{mnemonicbox}
``ઇલેક્ટ્રોન કૂદે, ફોટોન નીકળે''

\end{mnemonicbox}
\subsection*{પ્રશ્ન 3(ક) [7
માર્ક્સ]}\label{uxaaauxab0uxab6uxaa8-3uxa95-7-uxaaeuxab0uxa95uxab8}

\textbf{PN જંક્શન ડાયોડની VI લાક્ષણિકતા દોરીને સમજાવો.}

\begin{solutionbox}

\textbf{આકૃતિ:}

\begin{lstlisting}
    Current
    ^
    |           /
    |          /
    |         /
    |        /
    |       /
    |      /
    |_____/_________> Voltage
    |    /|
    |   / |
    |  /  |
    | /   |
    |/    |
    |     |
    
    Forward bias  | Reverse bias
\end{lstlisting}

\textbf{P-N જંક્શન ડાયોડની V-I લાક્ષણિકતાઓ:}

\begin{itemize}
\tightlist
\item
  \textbf{ફોરવર્ડ બાયસ રીજન}:

  \begin{itemize}
  \tightlist
  \item
    ડાયોડ ત્યારે કંડક્ટ કરે છે જ્યારે વોલ્ટેજ ની/કટ-ઇન વોલ્ટેજને (Ge માટે 0.3V, Si માટે
    0.7V) ઓળંગે
  \item
    વોલ્ટેજની સાથે કરંટ એક્સપોનેન્શિયલી વધે છે
  \item
    ઓછી રેઝિસ્ટન્સ સ્ટેટ
  \end{itemize}
\item
  \textbf{રિવર્સ બાયસ રીજન}:

  \begin{itemize}
  \tightlist
  \item
    ખૂબ જ નાનો લીકેજ કરંટ વહે છે
  \item
    રિવર્સ વોલ્ટેજ વધવા છતાં કરંટ લગભગ સ્થિર રહે છે
  \item
    ઉચ્ચ રેઝિસ્ટન્સ સ્ટેટ
  \item
    ઉચ્ચ રિવર્સ વોલ્ટેજ પર બ્રેકડાઉન થાય છે
  \end{itemize}
\item
  \textbf{મુખ્ય બિંદુઓ}:

  \begin{itemize}
  \tightlist
  \item
    નોન-લીનિયર ઉપકરણ
  \item
    એક દિશામાં કરંટ પ્રવાહ
  \item
    તાપમાન પર આધારિત
  \end{itemize}
\end{itemize}

\end{solutionbox}
\begin{mnemonicbox}
``ફોરવર્ડ સરળતાથી વહે, રિવર્સ દૃઢતાથી અટકાવે''

\end{mnemonicbox}
\subsection*{પ્રશ્ન 3(અ) OR [3
માર્ક્સ]}\label{uxaaauxab0uxab6uxaa8-3uxa85-or-3-uxaaeuxab0uxa95uxab8}

\textbf{PN જંક્શન ડાયોડના ઉપયોગોની યાદી બનાવો.}

\begin{solutionbox}

\textbf{PN જંક્શન ડાયોડના ઉપયોગો:}

\begin{itemize}
\tightlist
\item
  પાવર સપ્લાયમાં રેક્ટિફિકેશન
\item
  સિગ્નલ ડિમોડ્યુલેશન
\item
  ડિજિટલ સર્કિટમાં લોજિક ગેટ્સ
\item
  વોલ્ટેજ રેગ્યુલેશન (ઝેનર ડાયોડ સાથે)
\item
  સિગ્નલ ક્લિપિંગ અને ક્લેમ્પિંગ સર્કિટ્સ
\item
  રિવર્સ પોલારિટી સામે પ્રોટેક્શન સર્કિટ્સ
\end{itemize}

\end{solutionbox}
\begin{mnemonicbox}
``રેક્ટિફાય, ડિટેક્ટ, ક્લિપ, પ્રોટેક્ટ''

\end{mnemonicbox}
\subsection*{પ્રશ્ન 3(બ) OR [4
માર્ક્સ]}\label{uxaaauxab0uxab6uxaa8-3uxaac-or-4-uxaaeuxab0uxa95uxab8}

\textbf{અનબાયસ PN જંક્શન ડાયોડ ના ડીપલીશન રીજીયન ની રચના સમજાવો.}

\begin{solutionbox}

\textbf{આકૃતિ:}

\includegraphics[width=1\linewidth,height=\textheight,keepaspectratio]{mermaid-7581ab79.pdf}

\begin{itemize}
\tightlist
\item
  \textbf{રચના પ્રક્રિયા}:

  \begin{itemize}
  \tightlist
  \item
    N-સાઇડના ઇલેક્ટ્રોન P-સાઇડમાં ડિફ્યુઝ થાય છે
  \item
    P-સાઇડના હોલ્સ N-સાઇડમાં ડિફ્યુઝ થાય છે
  \item
    જંક્શન પર રીકોમ્બિનેશન થાય છે
  \item
    ઇમોબાઇલ આયન બાકી રહે છે (N-સાઇડમાં પોઝિટિવ, P-સાઇડમાં નેગેટિવ)
  \item
    ઇલેક્ટ્રિક ફીલ્ડ વિકસે છે, જે વધુ ડિફ્યુઝનનો વિરોધ કરે છે
  \item
    સમતુલન સ્થાપિત થાય છે, જે ડિપ્લેશન રીજિયન બનાવે છે
  \end{itemize}
\item
  \textbf{લાક્ષણિકતાઓ}:

  \begin{itemize}
  \tightlist
  \item
    ચાર્જ કેરિયર્સથી મુક્ત
  \item
    અવાહક/અવરોધક તરીકે કાર્ય કરે છે
  \item
    બિલ્ટ-ઇન પોટેન્શિયલ બનાવે છે
  \end{itemize}
\end{itemize}

\end{solutionbox}
\begin{mnemonicbox}
``ડિફ્યુઝન બેરિયર ફીલ્ડ બનાવે''

\end{mnemonicbox}
\subsection*{પ્રશ્ન 3(ક) OR [7
માર્ક્સ]}\label{uxaaauxab0uxab6uxaa8-3uxa95-or-7-uxaaeuxab0uxa95uxab8}

\textbf{PN જંક્શન ડાયોડનું બાંધકામ, કાર્ય અને એપ્લિકેશન સમજાવો.}

\begin{solutionbox}

\textbf{આકૃતિ:}

\includegraphics[width=1\linewidth,height=\textheight,keepaspectratio]{mermaid-b33ec5bf.pdf}

\textbf{બાંધકામ:}

\begin{itemize}
\tightlist
\item
  P-ટાઇપ સેમિકન્ડક્ટરને N-ટાઇપ સેમિકન્ડક્ટર સાથે જોડવામાં આવે છે
\item
  સિલિકોન અથવા જર્મેનિયમના સિંગલ ક્રિસ્ટલમાંથી બનાવવામાં આવે છે
\item
  P અને N રીજન સાથે મેટલ કોન્ટેક્ટ્સ જોડાયેલા હોય છે
\end{itemize}

\textbf{કાર્ય:}

\begin{itemize}
\tightlist
\item
  \textbf{ફોરવર્ડ બાયસ}:

  \begin{itemize}
  \tightlist
  \item
    P પર પોઝિટિવ, N પર નેગેટિવ
  \item
    ડિપ્લેશન રીજિયન સાંકડો થાય છે
  \item
    વોલ્ટેજ બેરિયર પોટેન્શિયલને ઓળંગે ત્યારે કરંટ વહે છે
  \end{itemize}
\item
  \textbf{રિવર્સ બાયસ}:

  \begin{itemize}
  \tightlist
  \item
    N પર પોઝિટિવ, P પર નેગેટિવ
  \item
    ડિપ્લેશન રીજિયન પહોળો થાય છે
  \item
    માત્ર નાનો લીકેજ કરંટ વહે છે
  \end{itemize}
\end{itemize}

\textbf{એપ્લિકેશન:}

\begin{itemize}
\tightlist
\item
  પાવર રેક્ટિફિકેશન
\item
  સિગ્નલ ડિટેક્શન
\item
  વોલ્ટેજ રેગ્યુલેશન
\item
  સ્વિચિંગ એપ્લિકેશન
\item
  પ્રોટેક્શન સર્કિટ્સ
\item
  લોજિક ગેટ્સ
\end{itemize}

\end{solutionbox}
\begin{mnemonicbox}
``P-N જોડો, કરંટ દિશા નિયંત્રિત કરો''

\end{mnemonicbox}
\subsection*{પ્રશ્ન 4(અ) [3
માર્ક્સ]}\label{uxaaauxab0uxab6uxaa8-4uxa85-3-uxaaeuxab0uxa95uxab8}

\textbf{વ્યાખ્યા આપો (1) રીપપલ આવૃત્તિ (2) રીપપલ ફેક્ટર (3) ડાયોડ નો PIV.}

\begin{solutionbox}

{\def\LTcaptype{none} % do not increment counter
\begin{longtable}[]{@{}
  >{\raggedright\arraybackslash}p{(\linewidth - 2\tabcolsep) * \real{0.3846}}
  >{\raggedright\arraybackslash}p{(\linewidth - 2\tabcolsep) * \real{0.6154}}@{}}
\toprule\noalign{}
\begin{minipage}[b]{\linewidth}\raggedright
\textbf{શબ્દ}
\end{minipage} & \begin{minipage}[b]{\linewidth}\raggedright
\textbf{વ્યાખ્યા}
\end{minipage} \\
\midrule\noalign{}
\endhead
\bottomrule\noalign{}
\endlastfoot
\textbf{રીપપલ આવૃત્તિ} & રેક્ટિફાયડ DC આઉટપુટમાં બાકી રહેલ AC ઘટકની આવૃત્તિ
(ફુલ-વેવ માટે 2\times ઇનપુટ આવૃત્તિ, હાફ-વેવ માટે 1\times) \\
\textbf{રીપપલ ફેક્ટર} & રેક્ટિફાયર આઉટપુટમાં DC ઘટક સાથે AC ઘટકના RMS મૂલ્યનો
ગુણોત્તર (γ = Vac(rms)/Vdc) \\
\textbf{PIV of a diode} & પીક ઇન્વર્સ વોલ્ટેજ એ મહત્તમ રિવર્સ વોલ્ટેજ છે જે ડાયોડ
બ્રેકડાઉન વિના સહન કરી શકે છે \\
\end{longtable}
}

\end{solutionbox}
\begin{mnemonicbox}
``આવૃત્તિ ફ્લક્ચ્યુએટ કરે, ફેક્ટર માપે, PIV સુરક્ષા આપે''

\end{mnemonicbox}
\subsection*{પ્રશ્ન 4(બ) [4
માર્ક્સ]}\label{uxaaauxab0uxab6uxaa8-4uxaac-4-uxaaeuxab0uxa95uxab8}

\textbf{બે ડાયોડ ફુલ વેવ રેક્ટિફાયર અને બ્રિજ રેક્ટિફાયર નો તફાવત આપો.}

\begin{solutionbox}

{\def\LTcaptype{none} % do not increment counter
\begin{longtable}[]{@{}lll@{}}
\toprule\noalign{}
\textbf{પેરામીટર} & \textbf{સેન્ટર-ટેપ્ડ ફુલ વેવ} & \textbf{બ્રિજ રેક્ટિફાયર} \\
\midrule\noalign{}
\endhead
\bottomrule\noalign{}
\endlastfoot
\textbf{ડાયોડની સંખ્યા} & 2 & 4 \\
\textbf{ટ્રાન્સફોર્મર} & સેન્ટર-ટેપ્ડ જરૂરી & સામાન્ય ટ્રાન્સફોર્મર \\
\textbf{PIV} & 2Vm & Vm \\
\textbf{કાર્યક્ષમતા} & 81.2\% & 81.2\% \\
\textbf{રીપપલ ફેક્ટર} & 0.48 & 0.48 \\
\textbf{આઉટપુટ} & Vm/π & 2Vm/π \\
\textbf{ખર્ચ} & ઊંચો ટ્રાન્સફોર્મર ખર્ચ & ઊંચો ડાયોડ ખર્ચ \\
\end{longtable}
}

\end{solutionbox}
\begin{mnemonicbox}
``બે ડાયોડ સેન્ટર ટેપ, ચાર બ્રિજ બનાવે''

\end{mnemonicbox}
\subsection*{પ્રશ્ન 4(ક) [7
માર્ક્સ]}\label{uxaaauxab0uxab6uxaa8-4uxa95-7-uxaaeuxab0uxa95uxab8}

\textbf{ઝેનર ડાયોડને વોલ્ટેજ રેગ્યુલેટર તરીકે સમજાવો.}

\begin{solutionbox}

\textbf{આકૃતિ:}

\begin{lstlisting}
        Rs            
    +---www-----+
    |           |
Vin |           | Zener    RL    Vout
    |           Z Diode     R     
    |           |           R     
    +-----------+-----------+
\end{lstlisting}

\textbf{કાર્ય સિદ્ધાંત:}

\begin{itemize}
\tightlist
\item
  ઝેનર ડાયોડ રિવર્સ બ્રેકડાઉન રીજીયનમાં કાર્ય કરે છે
\item
  તેના ટર્મિનલ્સ પર સ્થિર વોલ્ટેજ જાળવે છે
\item
  વોલ્ટેજ રેફરન્સ તરીકે કાર્ય કરે છે
\end{itemize}

\textbf{સર્કિટ ઓપરેશન:}

\begin{itemize}
\tightlist
\item
  સીરીઝ રેઝિસ્ટર Rs કરંટને મર્યાદિત કરે છે
\item
  જ્યારે ઇનપુટ બ્રેકડાઉન વોલ્ટેજથી વધે છે ત્યારે ઝેનર કંડક્ટ કરે છે
\item
  વધારાનો કરંટ ઝેનર ડાયોડ મારફતે વહે છે
\item
  આઉટપુટ વોલ્ટેજ ઝેનર વોલ્ટેજ પર સ્થિર રહે છે
\end{itemize}

\textbf{ફાયદાઓ:}

\begin{itemize}
\tightlist
\item
  સરળ સર્કિટ
\item
  ઓછી કિંમત
\item
  નાના લોડ ફેરફારો માટે સારું રેગ્યુલેશન
\end{itemize}

\textbf{મર્યાદાઓ:}

\begin{itemize}
\tightlist
\item
  ઝેનર અને સીરીઝ રેઝિસ્ટરમાં પાવર ડિસિપેશન
\item
  મર્યાદિત કરંટ ક્ષમતા
\item
  તાપમાન પર આધારિતતા
\end{itemize}

\end{solutionbox}
\begin{mnemonicbox}
``ઝેનર બ્રેકડાઉન થઈ વોલ્ટેજ સ્થિર રાખે''

\end{mnemonicbox}
\subsection*{પ્રશ્ન 4(અ) OR [3
માર્ક્સ]}\label{uxaaauxab0uxab6uxaa8-4uxa85-or-3-uxaaeuxab0uxa95uxab8}

\textbf{રેક્ટિફાયર શું છે? ફુલ વેવ રેક્ટિફાયરને વેવફોર્મ્સ સાથે સમજાવો.}

\begin{solutionbox}

\textbf{રેક્ટિફાયર:} એક સર્કિટ જે AC વોલ્ટેજને પલ્સેટિંગ DC વોલ્ટેજમાં રૂપાંતરિત કરે છે.

\textbf{આકૃતિ:}

\begin{lstlisting}
    +-------+
    |       |
A --+       +-- C
    | XFRMR |         D1
    |       |--+------|>|----+--+
    |       |  |             |  |
    |       |  |             |  |  RL   Output
    |       |  |             |  |  
B --+       +--+             |  |
    |       |  |             |  |
    |       |  |             |  |
    |       |--+------|<|----+--+
    |       |         D2
    +-------+
\end{lstlisting}

\textbf{વેવફોર્મ્સ:}

\begin{lstlisting}
Input:    ^
          |   /\    /\    /\
          |  /  \  /  \  /  \
          | /    \/    \/    \
          +--------------------
          |
          |\    /\    /\    /
          | \  /  \  /  \  / 
          |  \/    \/    \/

Output:   ^
          |   /\    /\    /\
          |  /  \  /  \  /  \
          | /    \/    \/    \
          +--------------------
\end{lstlisting}

\end{solutionbox}
\begin{mnemonicbox}
``બંને હાફ-સાયકલ પોઝિટિવ બને''

\end{mnemonicbox}
\subsection*{પ્રશ્ન 4(બ) OR [4
માર્ક્સ]}\label{uxaaauxab0uxab6uxaa8-4uxaac-or-4-uxaaeuxab0uxa95uxab8}

\textbf{રેક્ટિફાયરમાં ફિલ્ટર શા માટે જરૂરી છે? ફિલ્ટરના વિવિધ પ્રકારો જણાવો અને
કોઈપણ એક પ્રકારનું ફિલ્ટર સમજાવો.}

\begin{solutionbox}

\textbf{ફિલ્ટરની જરૂરિયાત:}

\begin{itemize}
\tightlist
\item
  રેક્ટિફાયર આઉટપુટમાં AC રિપપલ ઘટક હોય છે
\item
  ઇલેક્ટ્રોનિક સર્કિટ્સ માટે શુદ્ધ DC જરૂરી છે
\item
  ફિલ્ટર્સ AC ઘટકોને દૂર કરીને પલ્સેટિંગ DCને સ્મૂધ કરે છે
\end{itemize}

\textbf{ફિલ્ટરના પ્રકારો:}

\begin{itemize}
\tightlist
\item
  કેપેસિટર ફિલ્ટર (C-ફિલ્ટર)
\item
  ઇન્ડક્ટર ફિલ્ટર (L-ફિલ્ટર)
\item
  LC ફિલ્ટર
\item
  π (પાઇ) ફિલ્ટર
\item
  CLC ફિલ્ટર
\end{itemize}

\textbf{કેપેસિટર ફિલ્ટર:}

\includegraphics[width=1\linewidth,height=\textheight,keepaspectratio]{mermaid-ed62c8e1.pdf}

\textbf{કાર્ય:}

\begin{itemize}
\tightlist
\item
  કેપેસિટર વોલ્ટેજ વધારા દરમિયાન ચાર્જ થાય છે
\item
  વોલ્ટેજ ઘટાડા દરમિયાન ધીમે ધીમે ડિસ્ચાર્જ થાય છે
\item
  ઇનપુટ ઘટે ત્યારે કરંટ પ્રદાન કરે છે
\item
  રિપપલ વોલ્ટેજ ઘટાડે છે
\end{itemize}

\textbf{ફાયદાઓ:}

\begin{itemize}
\tightlist
\item
  સરળ અને સસ્તું
\item
  હળવા લોડ માટે અસરકારક
\item
  રિપપલ નોંધપાત્ર રીતે ઘટાડે છે
\end{itemize}

\end{solutionbox}
\begin{mnemonicbox}
``કેપેસિટર પીક્સ પકડે, ધીમેથી છોડે''

\end{mnemonicbox}
\subsection*{પ્રશ્ન 4(ક) OR [7
માર્ક્સ]}\label{uxaaauxab0uxab6uxaa8-4uxa95-or-7-uxaaeuxab0uxa95uxab8}

\textbf{રેક્ટિફાયરની જરૂરિયાત લખો. સર્કિટ ડાયાગ્રામ વડે બ્રિજ રેક્ટિફાયર સમજાવો
અને તેના ઇનપુટ અને આઉટપુટ વેવફોર્મ્સ દોરો.}

\begin{solutionbox}

\textbf{રેક્ટિફાયરની જરૂરિયાત:}

\begin{itemize}
\tightlist
\item
  ઇલેક્ટ્રોનિક ઉપકરણો માટે AC થી DC માં રૂપાંતર કરવા
\item
  મોટાભાગના ઇલેક્ટ્રોનિક સર્કિટ્સને DC પાવરની જરૂર પડે છે
\item
  બેટરી DC પ્રદાન કરે છે પરંતુ AC વિતરિત થાય છે
\item
  પાવર સપ્લાયનો બિલ્ડિંગ બ્લોક
\item
  ચાર્જિંગ સિસ્ટમ્સ માટે આવશ્યક
\end{itemize}

\textbf{બ્રિજ રેક્ટિફાયર સર્કિટ:}

\begin{lstlisting}
          D1        D3
    +-----|>|--------+
    |                |
A --+                +-- DC+
    |                |
    |                |    RL
    |                |
B --+                +-- DC-
    |                |
    +-----|<|--------+
          D2        D4
\end{lstlisting}

\textbf{ઇનપુટ વેવફોર્મ:}

\begin{lstlisting}
    ^
    |    /\      /\
    |   /  \    /  \
    |  /    \  /    \
    | /      \/      \
    +-------------------
    |       /\      /\
    |      /  \    /  \
    |     /    \  /    \
    |\   /      \/      
    | \ /
    |  V
\end{lstlisting}

\textbf{આઉટપુટ વેવફોર્મ:}

\begin{lstlisting}
    ^
    |    /\      /\
    |   /  \    /  \
    |  /    \  /    \
    | /      \/      \
    +-------------------
\end{lstlisting}

\textbf{કાર્ય:}

\begin{itemize}
\tightlist
\item
  પોઝિટિવ હાફ સાયકલ દરમિયાન: D1 અને D4 કંડક્ટ કરે છે
\item
  નેગેટિવ હાફ સાયકલ દરમિયાન: D2 અને D3 કંડક્ટ કરે છે
\item
  લોડને બંને સાયકલમાં એક જ દિશામાં કરંટ મળે છે
\item
  ઇનપુટ વેવફોર્મના બંને અર્ધ-ચક્રનો ઉપયોગ કરે છે
\end{itemize}

\end{solutionbox}
\begin{mnemonicbox}
``ચાર ડાયોડ બધા કરંટને એક દિશામાં વાળે''

\end{mnemonicbox}
\subsection*{પ્રશ્ન 5(અ) [3
માર્ક્સ]}\label{uxaaauxab0uxab6uxaa8-5uxa85-3-uxaaeuxab0uxa95uxab8}

\textbf{ઇલેક્ટ્રોનિક કચરાના કારણો સમજાવો.}

\begin{solutionbox}

\textbf{ઇલેક્ટ્રોનિક કચરાના કારણો:}

\begin{itemize}
\tightlist
\item
  ઝડપી ટેકનોલોજીકલ અદ્યતનીકરણ
\item
  ઉત્પાદનોની આયોજિત કાલગ્રસ્તતા
\item
  ઉત્પાદનોનું ઘટતું જીવનકાળ
\item
  નવા ઉપકરણોને પસંદ કરતી ગ્રાહક વર્તણૂક
\item
  ઇલેક્ટ્રોનિક્સ માટે મર્યાદિત રિપેર વિકલ્પો
\item
  રિપ્લેસમેન્ટની તુલનામાં ઊંચા રિપેર ખર્ચ
\end{itemize}

\end{solutionbox}
\begin{mnemonicbox}
``ટેક્નોલોજી આગળ વધે, ઉત્પાદન જલ્દી બગડે''

\end{mnemonicbox}
\subsection*{પ્રશ્ન 5(બ) [4
માર્ક્સ]}\label{uxaaauxab0uxab6uxaa8-5uxaac-4-uxaaeuxab0uxa95uxab8}

\textbf{PNP અને NPN ટ્રાન્ઝિસ્ટરની સરખામણી કરો.}

\begin{solutionbox}

{\def\LTcaptype{none} % do not increment counter
\begin{longtable}[]{@{}
  >{\raggedright\arraybackslash}p{(\linewidth - 4\tabcolsep) * \real{0.2727}}
  >{\raggedright\arraybackslash}p{(\linewidth - 4\tabcolsep) * \real{0.3636}}
  >{\raggedright\arraybackslash}p{(\linewidth - 4\tabcolsep) * \real{0.3636}}@{}}
\toprule\noalign{}
\begin{minipage}[b]{\linewidth}\raggedright
\textbf{પેરામીટર}
\end{minipage} & \begin{minipage}[b]{\linewidth}\raggedright
\textbf{PNP ટ્રાન્ઝિસ્ટર}
\end{minipage} & \begin{minipage}[b]{\linewidth}\raggedright
\textbf{NPN ટ્રાન્ઝિસ્ટર}
\end{minipage} \\
\midrule\noalign{}
\endhead
\bottomrule\noalign{}
\endlastfoot
\textbf{સિમ્બોલ} &
\pandocbounded{\includegraphics[keepaspectratio,alt={PNP}]{https://example.com/pnp.jpg}}
&
\pandocbounded{\includegraphics[keepaspectratio,alt={NPN}]{https://example.com/npn.jpg}} \\
\textbf{મેજોરિટી કેરિયર્સ} & હોલ્સ & ઇલેક્ટ્રોન્સ \\
\textbf{કરંટ પ્રવાહ} & એમિટરથી કલેક્ટર & કલેક્ટરથી એમિટર \\
\textbf{બાયસિંગ} & એમિટર બેઝ કરતાં વધુ પોઝિટિવ & બેઝ એમિટર કરતાં વધુ પોઝિટિવ \\
\textbf{સ્વિચિંગ સ્પીડ} & ધીમી & ઝડપી \\
\textbf{એપ્લિકેશન્સ} & લો ફ્રિક્વન્સી, હાઇ કરંટ & હાઇ ફ્રિક્વન્સી, સ્વિચિંગ \\
\end{longtable}
}

\textbf{આકૃતિ:}

\begin{lstlisting}
    NPN:         PNP:
    
    C            C
    |            |
    |            |
    B---->       <----B
    |            |
    |            |
    E            E
\end{lstlisting}

\end{solutionbox}
\begin{mnemonicbox}
``નેગેટિવ-પોઝિટિવ-નેગેટિવ વિરુદ્ધ પોઝિટિવ-નેગેટિવ-પોઝિટિવ''

\end{mnemonicbox}
\subsection*{પ્રશ્ન 5(ક) [7
માર્ક્સ]}\label{uxaaauxab0uxab6uxaa8-5uxa95-7-uxaaeuxab0uxa95uxab8}

\textbf{પ્રતીક દોરો, MOSFET નું બાંધકામ અને કાર્ય સમજાવો.}

\begin{solutionbox}

\textbf{સિમ્બોલ:}

\begin{lstlisting}
         D (Drain)
         |
         |
G (Gate) |
----||---+
         |
         |
         S (Source)
\end{lstlisting}

\textbf{બાંધકામ:}

\includegraphics[width=1\linewidth,height=\textheight,keepaspectratio]{mermaid-74d305d6.pdf}

\textbf{કાર્ય સિદ્ધાંત:}

\begin{itemize}
\tightlist
\item
  \textbf{એન્હાન્સમેન્ટ મોડ N-ચેનલ MOSFET:}

  \begin{itemize}
  \tightlist
  \item
    ગેટ વોલ્ટેજ વિના કોઈ ચેનલ અસ્તિત્વમાં નથી
  \item
    પોઝિટિવ ગેટ વોલ્ટેજ સબસ્ટ્રેટમાંથી ઇલેક્ટ્રોન્સને આકર્ષે છે
  \item
    ઉત્પન્ન થયેલી ચેનલ ડ્રેનથી સોર્સ સુધી કરંટ પ્રવાહને મંજૂરી આપે છે
  \item
    ગેટ વોલ્ટેજ વધારવાથી કન્ડક્ટિવિટી વધે છે
  \end{itemize}
\item
  \textbf{મુખ્ય વિશેષતાઓ:}

  \begin{itemize}
  \tightlist
  \item
    વોલ્ટેજ-નિયંત્રિત ઉપકરણ (ઉચ્ચ ઇનપુટ ઇમ્પેડન્સ)
  \item
    ગેટ કરંટની જરૂર નથી (BJT થી અલગ)
  \item
    BJT કરતાં ઝડપી સ્વિચિંગ
  \item
    ઓછુ પાવર ડિસિપેશન
  \end{itemize}
\end{itemize}

\textbf{એપ્લિકેશન્સ:}

\begin{itemize}
\tightlist
\item
  ડિજિટલ લોજિક સર્કિટ્સ
\item
  સ્વિચિંગ એપ્લિકેશન્સ
\item
  એમ્પ્લિફાયર્સ
\item
  પાવર કન્ટ્રોલ ડિવાઇસીસ
\end{itemize}

\end{solutionbox}
\begin{mnemonicbox}
``ગેટ વોલ્ટેજ ઇલેક્ટ્રોન ચેનલ બનાવે''

\end{mnemonicbox}
\subsection*{પ્રશ્ન 5(અ) OR [3
માર્ક્સ]}\label{uxaaauxab0uxab6uxaa8-5uxa85-or-3-uxaaeuxab0uxa95uxab8}

\textbf{ઈલેક્ટ્રોનિક કચરાને હેન્ડલ કરવાની પદ્ધતિઓ સમજાવો.}

\begin{solutionbox}

\textbf{ઈલેક્ટ્રોનિક કચરાને હેન્ડલ કરવાની પદ્ધતિઓ:}

{\def\LTcaptype{none} % do not increment counter
\begin{longtable}[]{@{}
  >{\raggedright\arraybackslash}p{(\linewidth - 2\tabcolsep) * \real{0.4138}}
  >{\raggedright\arraybackslash}p{(\linewidth - 2\tabcolsep) * \real{0.5862}}@{}}
\toprule\noalign{}
\begin{minipage}[b]{\linewidth}\raggedright
\textbf{પદ્ધતિ}
\end{minipage} & \begin{minipage}[b]{\linewidth}\raggedright
\textbf{વર્ણન}
\end{minipage} \\
\midrule\noalign{}
\endhead
\bottomrule\noalign{}
\endlastfoot
\textbf{ઘટાડો (Reduce)} & લાંબા સમય સુધી ચાલે તેવા ઇલેક્ટ્રોનિક્સનું ડિઝાઇન, અપગ્રેડ
માટે મોડ્યુલર ડિઝાઇન \\
\textbf{પુન:ઉપયોગ (Reuse)} & કાર્યરત ઉપકરણોનું દાન અથવા વેચાણ, ઘટકોનો
પુન:ઉપયોગ \\
\textbf{રિસાયકલ (Recycle)} & યોગ્ય વિઘટન અને સામગ્રી પુનઃપ્રાપ્તિ (કિંમતી ધાતુઓ,
પ્લાસ્ટિક) \\
\textbf{નિયમન (Regulation)} & ઇ-વેસ્ટ મેનેજમેન્ટ નીતિઓ, વિસ્તારિત ઉત્પાદક
જવાબદારી \\
\textbf{રિકવરી (Recovery)} & વિશિષ્ટ પ્રક્રિયાઓ દ્વારા મૂલ્યવાન સામગ્રીનું
નિષ્કર્ષણ \\
\end{longtable}
}

\end{solutionbox}
\begin{mnemonicbox}
``ઘટાડો, પુન:ઉપયોગ, રિસાયકલ, નિયમન, પુનઃપ્રાપ્તિ''

\end{mnemonicbox}
\subsection*{પ્રશ્ન 5(બ) OR [4
માર્ક્સ]}\label{uxaaauxab0uxab6uxaa8-5uxaac-or-4-uxaaeuxab0uxa95uxab8}

\textbf{αdc અને βdc વચ્ચેનો સંબંધ મેળવો.}

\begin{solutionbox}

\textbf{આકૃતિ:}

\begin{lstlisting}
               IC
              ↑
              |  
              |
    IB \rightarrow      |      \rightarrow  
        B ----+---- C
              |
              |
              |
              E
              ↓
              IE
\end{lstlisting}

\textbf{ટ્રાન્ઝિસ્ટર કરંટ સંબંધો:}

\begin{itemize}
\tightlist
\item
  IE = IC + IB (પ્રવેશ કરતો કરંટ નીકળતા કરંટ બરાબર)
\item
  αdc = IC/IE (કોમન બેઝ કરંટ ગેઇન)
\item
  βdc = IC/IB (કોમન એમિટર કરંટ ગેઇન)
\end{itemize}

\textbf{ડેરિવેશન:}

\begin{itemize}
\tightlist
\item
  IE = IC + IB માંથી
\item
  બંને બાજુઓને IC થી ભાગો: IE/IC = 1 + IB/IC
\item
  તેથી: 1/αdc = 1 + 1/βdc
\item
  βdc માટે હલ કરતાં: βdc = αdc/(1-αdc)
\item
  અને αdc માટે: αdc = βdc/(1+βdc)
\end{itemize}

\textbf{મૂલ્યોની ટેબલ:}

{\def\LTcaptype{none} % do not increment counter
\begin{longtable}[]{@{}ll@{}}
\toprule\noalign{}
αdc & βdc \\
\midrule\noalign{}
\endhead
\bottomrule\noalign{}
\endlastfoot
0.9 & 9 \\
0.95 & 19 \\
0.99 & 99 \\
\end{longtable}
}

\end{solutionbox}
\begin{mnemonicbox}
``આલ્ફા-બીટા સંબંધિત છે αdc = βdc/(1+βdc)''

\end{mnemonicbox}
\subsection*{પ્રશ્ન 5(ક) OR [7
માર્ક્સ]}\label{uxaaauxab0uxab6uxaa8-5uxa95-or-7-uxaaeuxab0uxa95uxab8}

\textbf{તેના ઇનપુટ અને આઉટપુટ લાક્ષણિકતાઓ સાથે CC ની રચના સમજાવો.}

\begin{solutionbox}

\textbf{કોમન કલેક્ટર સર્કિટ (એમિટર ફોલોઅર):}

\begin{lstlisting}
              +VCC
               |
               R
               |
               C
    +----+     |
    |    |     |
    Vin  |     +---+ Output
    |    |     |
    +----+-----+
         |     |
         B     E
         |     |
         +-----+
               |
               RE
               |
              GND
\end{lstlisting}

\textbf{ઇનપુટ લાક્ષણિકતાઓ:} (IB vs VBE)

\begin{lstlisting}
    IB ^
       |           /
       |          /
       |         /
       |        /
       |       /
       |      /
       |     /
       |    /
       |   /
       |  /
       | /
       |/
       +--------------> VBE
\end{lstlisting}

\textbf{આઉટપુટ લાક્ષણિકતાઓ:} (IE vs VCE)

\begin{lstlisting}
    IE ^
       |    ---------------
       |   /
       |  /
       | /
       |/
       +--------------> VCE
       
       IB3 > IB2 > IB1 > 0
\end{lstlisting}

\textbf{મુખ્ય વિશેષતાઓ:}

\begin{itemize}
\tightlist
\item
  વોલ્ટેજ ગેઇન \approx 1 (થોડો ઓછો)
\item
  ઉચ્ચ કરંટ ગેઇન (β+1)
\item
  ઉચ્ચ ઇનપુટ ઇમ્પેડન્સ
\item
  નીચું આઉટપુટ ઇમ્પેડન્સ
\item
  ઇનપુટ અને આઉટપુટ વચ્ચે કોઈ ફેઝ ઇન્વર્ઝન નહીં
\item
  બફર/ઇમ્પેડન્સ મેચિંગ સર્કિટ તરીકે ઉપયોગ
\end{itemize}

\end{solutionbox}
\begin{mnemonicbox}
``એમિટર બેઝ વોલ્ટેજને અનુસરે છે''

\end{mnemonicbox}

\end{document}
