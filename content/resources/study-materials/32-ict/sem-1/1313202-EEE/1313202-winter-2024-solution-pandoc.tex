\documentclass[10pt,a4paper]{article}

% content/resources/templates/preamble.tex
\usepackage[margin=0.6in]{geometry}
\author{Milav Dabgar}
\usepackage{amsmath,amssymb,amsthm}
\usepackage{booktabs}
\usepackage{multirow}
\usepackage{xcolor}
\usepackage{tcolorbox}
\tcbuselibrary{breakable,skins}
\usepackage[colorlinks=true,linkcolor=blue]{hyperref}
\usepackage{titlesec}
\usepackage{enumitem}
\usepackage{tikz}
\usepackage{pgfplots}
\usepackage{circuitikz}
\usepackage[version=4]{mhchem}
\usepackage{longtable}
\usepackage{array}
\usepackage{float}
\usepackage{caption}
\usepackage{listings}

\lstset{
  basicstyle=\small\ttfamily,
  breaklines=true,
  breakatwhitespace=false,
  postbreak=\mbox{\textcolor{red}{$\hookrightarrow$}\space},
  float=false,
  numbers=left,
  numberstyle=\tiny\color{gray},
  numbersep=10pt,
  xleftmargin=2em,
  keywordstyle=\color{blue},
  commentstyle=\color{green!60!black},
  stringstyle=\color{purple},
  backgroundcolor=\color{gray!5},
  showstringspaces=false,
  tabsize=2,
  captionpos=b,
  keepspaces=true,
  columns=flexible
}

\pgfplotsset{compat=1.18}
\usetikzlibrary{shapes,arrows,positioning,calc,patterns,decorations.pathmorphing,decorations.markings,arrows.meta}

% Color scheme
\definecolor{headcolor}{RGB}{0,102,204}
\definecolor{keycolor}{RGB}{220,20,60}
\definecolor{solutioncolor}{RGB}{34,139,34}
\definecolor{mnemoniccolor}{RGB}{148,0,211}
\definecolor{codecolor}{RGB}{0,0,100}

% Spacing
\setlength{\parskip}{3pt}
\setlist[itemize]{nosep}
\setlist[enumerate]{nosep}

% Title formatting
\titleformat{\section}{\Large\bfseries\color{headcolor}}{\thesection}{1em}{}
\titleformat{\subsection}{\large\bfseries\color{headcolor}}{\thesubsection}{1em}{}

% Pandoc tightlist compatibility
\providecommand{\tightlist}{%
  \setlength{\itemsep}{0pt}\setlength{\parskip}{0pt}}

% Pandoc longtable compatibility
\newcounter{none}
\def\thenone{}


% content/resources/templates/english-boxes.tex
% This file is currently empty - it exists to maintain consistency with the import structure.
% Add custom environments here if needed in the future.


\begin{document}

\begin{center}
{\Huge\bfseries\color{headcolor} Subject Name Solutions}\\[5pt]
{\LARGE 1313202 -- Winter 2024}\\[3pt]
{\large Semester 1 Study Material}\\[3pt]
{\normalsize\textit{Detailed Solutions and Explanations}}
\end{center}

\vspace{10pt}

\subsection*{Question 1(a) [3 marks]}\label{q1a}

\textbf{Explain difference between Active and passive network.}

\begin{solutionbox}

{\def\LTcaptype{none} % do not increment counter
\begin{longtable}[]{@{}
  >{\raggedright\arraybackslash}p{(\linewidth - 2\tabcolsep) * \real{0.4750}}
  >{\raggedright\arraybackslash}p{(\linewidth - 2\tabcolsep) * \real{0.5250}}@{}}
\toprule\noalign{}
\begin{minipage}[b]{\linewidth}\raggedright
\textbf{Active Network}
\end{minipage} & \begin{minipage}[b]{\linewidth}\raggedright
\textbf{Passive Network}
\end{minipage} \\
\midrule\noalign{}
\endhead
\bottomrule\noalign{}
\endlastfoot
Contains at least one active element (voltage/current source) & Contains
only passive elements (R, L, C) \\
Can deliver energy to the circuit & Cannot deliver energy to the
circuit \\
Can amplify signal power & Cannot amplify signal power \\
\end{longtable}
}

\end{solutionbox}
\begin{mnemonicbox}
``Active Adds Power, Passive Parts Take''

\end{mnemonicbox}
\subsection*{Question 1(b) [4 marks]}\label{q1b}

\textbf{State and explain Kirchhoff's voltage law (KVL).}

\begin{solutionbox}

Kirchhoff's Voltage Law (KVL) states that the algebraic sum of all
voltages around any closed loop in a circuit is zero.

\textbf{Diagram:}

\includegraphics[width=1\linewidth,height=\textheight,keepaspectratio]{mermaid-28c30a58.pdf}

Mathematically: V1 + V2 + V3 + V4 = 0

\begin{itemize}
\tightlist
\item
  \textbf{Voltage drops}: When passing through a resistor in direction
  of current, voltage is negative
\item
  \textbf{Voltage rises}: When passing through a source from negative to
  positive, voltage is positive
\end{itemize}

\end{solutionbox}
\begin{mnemonicbox}
``Voltage Loop Equals Zero''

\end{mnemonicbox}
\subsection*{Question 1(c) [7 marks]}\label{q1c}

\textbf{Define the following terms: (1) Charge (2) Current (3) Potential
(4) E.M.F. (5) Inductance (6) Capacitance (7) Frequency.}

\begin{solutionbox}

{\def\LTcaptype{none} % do not increment counter
\begin{longtable}[]{@{}
  >{\raggedright\arraybackslash}p{(\linewidth - 2\tabcolsep) * \real{0.3846}}
  >{\raggedright\arraybackslash}p{(\linewidth - 2\tabcolsep) * \real{0.6154}}@{}}
\toprule\noalign{}
\begin{minipage}[b]{\linewidth}\raggedright
\textbf{Term}
\end{minipage} & \begin{minipage}[b]{\linewidth}\raggedright
\textbf{Definition}
\end{minipage} \\
\midrule\noalign{}
\endhead
\bottomrule\noalign{}
\endlastfoot
\textbf{Charge} & The quantity of electricity measured in coulombs
(C) \\
\textbf{Current} & The rate of flow of electric charge measured in
amperes (A) \\
\textbf{Potential} & The electrical pressure or energy per unit charge
measured in volts (V) \\
\textbf{E.M.F.} & Electromotive Force is the energy supplied by a source
per unit charge measured in volts (V) \\
\textbf{Inductance} & The property of an electric circuit that opposes
change in current, measured in henries (H) \\
\textbf{Capacitance} & The ability of a body to store electrical charge,
measured in farads (F) \\
\textbf{Frequency} & Number of complete cycles per second, measured in
hertz (Hz) \\
\end{longtable}
}

\end{solutionbox}
\begin{mnemonicbox}
``Coulombs' Flow Pressurized by Energy Induces
Capacitive Fluctuations''

\end{mnemonicbox}
\subsection*{Question 1(c) OR [7
marks]}\label{q1c}

\textbf{State Ohm's law. Write its application and limitation.}

\begin{solutionbox}

Ohm's Law states that the current flowing through a conductor is
directly proportional to the potential difference and inversely
proportional to the resistance.

\textbf{Diagram:}

\begin{lstlisting}
V = I \times R
\end{lstlisting}

Where:

\begin{itemize}
\tightlist
\item
  V = Voltage (volts)
\item
  I = Current (amperes)
\item
  R = Resistance (ohms)
\end{itemize}

\textbf{Applications:}

\begin{itemize}
\tightlist
\item
  Circuit design and analysis
\item
  Power consumption calculations
\item
  Component value determination
\item
  Voltage divider networks
\item
  Current divider networks
\end{itemize}

\textbf{Limitations:}

\begin{itemize}
\tightlist
\item
  Valid only for linear components
\item
  Not applicable to non-ohmic devices (diodes, transistors)
\item
  Invalid at high temperatures
\item
  Not valid for semiconductors
\item
  Cannot be applied to non-linear resistive elements
\end{itemize}

\end{solutionbox}
\begin{mnemonicbox}
``Volts Reveal Amps' Motion''

\end{mnemonicbox}
\subsection*{Question 2(a) [3 marks]}\label{q2a}

\textbf{Draw and explain energy band diagrams for insulator, conductor
and Semiconductor.}

\begin{solutionbox}

\textbf{Diagram:}

\includegraphics[width=1\linewidth,height=\textheight,keepaspectratio]{mermaid-4adf126a.pdf}

\begin{itemize}
\tightlist
\item
  \textbf{Conductor}: Valence and conduction bands overlap, allowing
  free electron movement
\item
  \textbf{Semiconductor}: Small energy gap (0.7-3 eV) between bands
  allows limited conduction
\item
  \textbf{Insulator}: Large energy gap (\textgreater3 eV) prevents
  electrons from moving to conduction band
\end{itemize}

\end{solutionbox}
\begin{mnemonicbox}
``Conductors Overlap, Semiconductors Jump Small,
Insulators Block All''

\end{mnemonicbox}
\subsection*{Question 2(b) [4 marks]}\label{q2b}

\textbf{Write statement of Maximum power transfer theorem and
reciprocity theorem.}

\begin{solutionbox}

{\def\LTcaptype{none} % do not increment counter
\begin{longtable}[]{@{}
  >{\raggedright\arraybackslash}p{(\linewidth - 2\tabcolsep) * \real{0.4643}}
  >{\raggedright\arraybackslash}p{(\linewidth - 2\tabcolsep) * \real{0.5357}}@{}}
\toprule\noalign{}
\begin{minipage}[b]{\linewidth}\raggedright
\textbf{Theorem}
\end{minipage} & \begin{minipage}[b]{\linewidth}\raggedright
\textbf{Statement}
\end{minipage} \\
\midrule\noalign{}
\endhead
\bottomrule\noalign{}
\endlastfoot
\textbf{Maximum Power Transfer Theorem} & Maximum power is transferred
from source to load when the load resistance equals the source internal
resistance (RL = RS) \\
\textbf{Reciprocity Theorem} & In a linear, bilateral network, if
voltage source E in branch 1 produces current I in branch 2, then the
same voltage source E in branch 2 will produce the same current I in
branch 1 \\
\end{longtable}
}

\end{solutionbox}
\begin{mnemonicbox}
``Match Resistance for Maximum Power; Swap Sources,
Current Stays''

\end{mnemonicbox}
\subsection*{Question 2(c) [7 marks]}\label{q2c}

\textbf{Explain the formation and conduction of N-type materials.}

\begin{solutionbox}

\textbf{Diagram:}

\includegraphics[width=1\linewidth,height=\textheight,keepaspectratio]{mermaid-e38d7d1b.pdf}

\begin{itemize}
\tightlist
\item
  \textbf{Formation Process}:

  \begin{itemize}
  \tightlist
  \item
    Pure silicon/germanium doped with pentavalent impurity atoms (P, As,
    Sb)
  \item
    Impurity atoms have 5 valence electrons (silicon has 4)
  \item
    Four electrons form covalent bonds, fifth becomes free electron
  \item
    Creates excess negative charge carriers
  \end{itemize}
\item
  \textbf{Conduction Mechanism}:

  \begin{itemize}
  \tightlist
  \item
    Majority carriers: Electrons
  \item
    Minority carriers: Holes
  \item
    Electron movement provides electrical conduction
  \item
    Even at room temperature, free electrons enable current flow
  \end{itemize}
\end{itemize}

\end{solutionbox}
\begin{mnemonicbox}
``Pentavalent Provides Plus-One Electron''

\end{mnemonicbox}
\subsection*{Question 2(a) OR [3
marks]}\label{q2a}

\textbf{Define valence band, conduction band and forbidden gap.}

\begin{solutionbox}

{\def\LTcaptype{none} % do not increment counter
\begin{longtable}[]{@{}
  >{\raggedright\arraybackslash}p{(\linewidth - 2\tabcolsep) * \real{0.3846}}
  >{\raggedright\arraybackslash}p{(\linewidth - 2\tabcolsep) * \real{0.6154}}@{}}
\toprule\noalign{}
\begin{minipage}[b]{\linewidth}\raggedright
\textbf{Term}
\end{minipage} & \begin{minipage}[b]{\linewidth}\raggedright
\textbf{Definition}
\end{minipage} \\
\midrule\noalign{}
\endhead
\bottomrule\noalign{}
\endlastfoot
\textbf{Valence Band} & Energy band occupied by valence electrons that
are bound to specific atoms in the solid \\
\textbf{Conduction Band} & Higher energy band where electrons can move
freely throughout the material, enabling electrical conduction \\
\textbf{Forbidden Gap} & Energy region between valence and conduction
bands where no electron states exist \\
\end{longtable}
}

\end{solutionbox}
\begin{mnemonicbox}
``Valence Binds, Conduction Flows, Forbidden Gaps
Block''

\end{mnemonicbox}
\subsection*{Question 2(b) OR [4
marks]}\label{q2b}

\textbf{Define the terms active power, reactive power and power factor
with power triangle.}

\begin{solutionbox}

\textbf{Diagram:}

\begin{lstlisting}
    |    
    |   S (Apparent Power)
    |  /|
    | / |
    |/__|
    P   Q

P = Active Power
Q = Reactive Power
S = Apparent Power
cosθ = Power Factor
\end{lstlisting}

\begin{itemize}
\tightlist
\item
  \textbf{Active Power (P)}: Actual power consumed, measured in watts
  (W), P = VI cosθ
\item
  \textbf{Reactive Power (Q)}: Power oscillating between source and
  load, measured in volt-amperes reactive (VAR), Q = VI sinθ
\item
  \textbf{Power Factor}: Ratio of active power to apparent power, PF =
  cosθ = P/S
\end{itemize}

\end{solutionbox}
\begin{mnemonicbox}
``Real Power Works, Reactive Power Waits''

\end{mnemonicbox}
\subsection*{Question 2(c) OR [7
marks]}\label{q2c}

\textbf{Explain the structure of atom of trivalent, tetravalent and
pentavalent elements.}

\begin{solutionbox}

\textbf{Diagram:}

\includegraphics[width=1\linewidth,height=\textheight,keepaspectratio]{mermaid-485bf1f9.pdf}

{\def\LTcaptype{none} % do not increment counter
\begin{longtable}[]{@{}
  >{\raggedright\arraybackslash}p{(\linewidth - 6\tabcolsep) * \real{0.2535}}
  >{\raggedright\arraybackslash}p{(\linewidth - 6\tabcolsep) * \real{0.2113}}
  >{\raggedright\arraybackslash}p{(\linewidth - 6\tabcolsep) * \real{0.1972}}
  >{\raggedright\arraybackslash}p{(\linewidth - 6\tabcolsep) * \real{0.3380}}@{}}
\toprule\noalign{}
\begin{minipage}[b]{\linewidth}\raggedright
\textbf{Element Type}
\end{minipage} & \begin{minipage}[b]{\linewidth}\raggedright
\textbf{Structure}
\end{minipage} & \begin{minipage}[b]{\linewidth}\raggedright
\textbf{Examples}
\end{minipage} & \begin{minipage}[b]{\linewidth}\raggedright
\textbf{Semiconductor Use}
\end{minipage} \\
\midrule\noalign{}
\endhead
\bottomrule\noalign{}
\endlastfoot
\textbf{Trivalent} & 3 electrons in outermost shell & B, Al, Ga, In &
P-type dopant \\
\textbf{Tetravalent} & 4 electrons in outermost shell & Si, Ge, C &
Semiconductor base \\
\textbf{Pentavalent} & 5 electrons in outermost shell & P, As, Sb &
N-type dopant \\
\end{longtable}
}

\end{solutionbox}
\begin{mnemonicbox}
``Three Accepts, Four Forms, Five Donates''

\end{mnemonicbox}
\subsection*{Question 3(a) [3 marks]}\label{q3a}

\textbf{Draw the symbol of photodiode and state its application.}

\begin{solutionbox}

\textbf{Diagram:}

\begin{lstlisting}
    |\ 
    | \  
    |  \   
-->|| \--->
    |  /     
    | /   
    |/  
\end{lstlisting}

\textbf{Applications of Photodiode:}

\begin{itemize}
\tightlist
\item
  Light sensors and detectors
\item
  Optical communication systems
\item
  Solar cells and photovoltaic applications
\item
  Camera exposure controls
\item
  Medical equipment (pulse oximeters)
\end{itemize}

\end{solutionbox}
\begin{mnemonicbox}
``Light Triggers Electric Current''

\end{mnemonicbox}
\subsection*{Question 3(b) [4 marks]}\label{q3b}

\textbf{Write a Short note on LED.}

\begin{solutionbox}

\textbf{Diagram:}

\begin{lstlisting}
    |\ 
    | \  
    |  \   
<---|| \--->
    |  /     
    | /   
    |/  
    ▼ ▼
   Light
\end{lstlisting}

\begin{itemize}
\tightlist
\item
  \textbf{Structure}: P-N junction diode that emits light when forward
  biased
\item
  \textbf{Working Principle}: Electron-hole recombination releases
  energy as photons
\item
  \textbf{Types}: Various colors based on semiconductor material (GaAs,
  GaP, GaN)
\item
  \textbf{Advantages}: Low power consumption, long life, small size,
  fast switching
\item
  \textbf{Applications}: Displays, indicators, lighting, remote
  controls, optical communications
\end{itemize}

\end{solutionbox}
\begin{mnemonicbox}
``Electrons Jump, Photons Emit''

\end{mnemonicbox}
\subsection*{Question 3(c) [7 marks]}\label{q3c}

\textbf{Draw and explain VI characteristic of PN junction diode.}

\begin{solutionbox}

\textbf{Diagram:}

\begin{lstlisting}
    Current
    ^
    |           /
    |          /
    |         /
    |        /
    |       /
    |      /
    |_____/_________> Voltage
    |    /|
    |   / |
    |  /  |
    | /   |
    |/    |
    |     |
    
    Forward bias  | Reverse bias
\end{lstlisting}

\textbf{P-N Junction Diode V-I Characteristics:}

\begin{itemize}
\tightlist
\item
  \textbf{Forward Bias Region}:

  \begin{itemize}
  \tightlist
  \item
    Diode conducts when voltage exceeds knee/cut-in voltage (0.3V for
    Ge, 0.7V for Si)
  \item
    Current increases exponentially with voltage
  \item
    Low resistance state
  \end{itemize}
\item
  \textbf{Reverse Bias Region}:

  \begin{itemize}
  \tightlist
  \item
    Very small leakage current flows
  \item
    Current remains almost constant with increasing reverse voltage
  \item
    High resistance state
  \item
    Breakdown occurs at high reverse voltage
  \end{itemize}
\item
  \textbf{Key Points}:

  \begin{itemize}
  \tightlist
  \item
    Non-linear device
  \item
    Unidirectional current flow
  \item
    Temperature dependent
  \end{itemize}
\end{itemize}

\end{solutionbox}
\begin{mnemonicbox}
``Forward Flows Freely, Reverse Resists Rigidly''

\end{mnemonicbox}
\subsection*{Question 3(a) OR [3
marks]}\label{q3a}

\textbf{List the applications of PN junction diode.}

\begin{solutionbox}

\textbf{Applications of PN Junction Diode:}

\begin{itemize}
\tightlist
\item
  Rectification in power supplies
\item
  Signal demodulation
\item
  Logic gates in digital circuits
\item
  Voltage regulation (with zener diodes)
\item
  Signal clipping and clamping circuits
\item
  Protection circuits against reverse polarity
\end{itemize}

\end{solutionbox}
\begin{mnemonicbox}
``Rectify, Detect, Clip, Protect''

\end{mnemonicbox}
\subsection*{Question 3(b) OR [4
marks]}\label{q3b}

\textbf{Explain the formation of depletion region in unbiased P-N
junction.}

\begin{solutionbox}

\textbf{Diagram:}

\includegraphics[width=1\linewidth,height=\textheight,keepaspectratio]{mermaid-7581ab79.pdf}

\begin{itemize}
\tightlist
\item
  \textbf{Formation Process}:

  \begin{itemize}
  \tightlist
  \item
    Electrons from N-side diffuse into P-side
  \item
    Holes from P-side diffuse into N-side
  \item
    Recombination occurs at junction
  \item
    Immobile ions remain (positive in N-side, negative in P-side)
  \item
    Electric field develops, opposing further diffusion
  \item
    Equilibrium is established, creating depletion region
  \end{itemize}
\item
  \textbf{Characteristics}:

  \begin{itemize}
  \tightlist
  \item
    Free of charge carriers
  \item
    Acts as insulator/barrier
  \item
    Creates built-in potential
  \end{itemize}
\end{itemize}

\end{solutionbox}
\begin{mnemonicbox}
``Diffusion Creates Barrier Field''

\end{mnemonicbox}
\subsection*{Question 3(c) OR [7
marks]}\label{q3c}

\textbf{Explain construction, working and applications of PN junction
diode.}

\begin{solutionbox}

\textbf{Diagram:}

\includegraphics[width=1\linewidth,height=\textheight,keepaspectratio]{mermaid-b33ec5bf.pdf}

\textbf{Construction:}

\begin{itemize}
\tightlist
\item
  P-type semiconductor joined with N-type semiconductor
\item
  Made from single crystal of silicon or germanium
\item
  Metal contacts connected to P and N regions
\end{itemize}

\textbf{Working:}

\begin{itemize}
\tightlist
\item
  \textbf{Forward Bias}:

  \begin{itemize}
  \tightlist
  \item
    Positive to P, negative to N
  \item
    Depletion region narrows
  \item
    Current flows when voltage exceeds barrier potential
  \end{itemize}
\item
  \textbf{Reverse Bias}:

  \begin{itemize}
  \tightlist
  \item
    Positive to N, negative to P
  \item
    Depletion region widens
  \item
    Only small leakage current flows
  \end{itemize}
\end{itemize}

\textbf{Applications:}

\begin{itemize}
\tightlist
\item
  Power rectification
\item
  Signal detection
\item
  Voltage regulation
\item
  Switching applications
\item
  Protection circuits
\item
  Logic gates
\end{itemize}

\end{solutionbox}
\begin{mnemonicbox}
``Join P-N, Control Current Direction''

\end{mnemonicbox}
\subsection*{Question 4(a) [3 marks]}\label{q4a}

\textbf{Define: (1) Ripple frequency (2) Ripple factor (3) PIV of a
diode.}

\begin{solutionbox}

{\def\LTcaptype{none} % do not increment counter
\begin{longtable}[]{@{}
  >{\raggedright\arraybackslash}p{(\linewidth - 2\tabcolsep) * \real{0.3846}}
  >{\raggedright\arraybackslash}p{(\linewidth - 2\tabcolsep) * \real{0.6154}}@{}}
\toprule\noalign{}
\begin{minipage}[b]{\linewidth}\raggedright
\textbf{Term}
\end{minipage} & \begin{minipage}[b]{\linewidth}\raggedright
\textbf{Definition}
\end{minipage} \\
\midrule\noalign{}
\endhead
\bottomrule\noalign{}
\endlastfoot
\textbf{Ripple Frequency} & Frequency of the AC component remaining in
the rectified DC output (2\times input frequency for full-wave, 1\times for
half-wave) \\
\textbf{Ripple Factor} & Ratio of RMS value of AC component to the DC
component in rectifier output (γ = Vac(rms)/Vdc) \\
\textbf{PIV of a diode} & Peak Inverse Voltage is the maximum reverse
voltage a diode can withstand without breakdown \\
\end{longtable}
}

\end{solutionbox}
\begin{mnemonicbox}
``Frequency Fluctuates, Factor Measures, PIV
Protects''

\end{mnemonicbox}
\subsection*{Question 4(b) [4 marks]}\label{q4b}

\textbf{Give comparison between full wave rectifier with two diodes and
full wave bridge rectifier.}

\begin{solutionbox}

{\def\LTcaptype{none} % do not increment counter
\begin{longtable}[]{@{}lll@{}}
\toprule\noalign{}
\textbf{Parameter} & \textbf{Center-Tapped Full Wave} & \textbf{Bridge
Rectifier} \\
\midrule\noalign{}
\endhead
\bottomrule\noalign{}
\endlastfoot
\textbf{Number of Diodes} & 2 & 4 \\
\textbf{Transformer} & Center-tapped required & Simple transformer \\
\textbf{PIV} & 2Vm & Vm \\
\textbf{Efficiency} & 81.2\% & 81.2\% \\
\textbf{Ripple Factor} & 0.48 & 0.48 \\
\textbf{Output} & Vm/π & 2Vm/π \\
\textbf{Cost} & Higher transformer cost & Higher diode cost \\
\end{longtable}
}

\end{solutionbox}
\begin{mnemonicbox}
``Two Diodes Tap Center, Four Make Bridge''

\end{mnemonicbox}
\subsection*{Question 4(c) [7 marks]}\label{q4c}

\textbf{Explain zener diode as voltage regulator.}

\begin{solutionbox}

\textbf{Diagram:}

\begin{lstlisting}
        Rs            
    +---www-----+
    |           |
Vin |           | Zener    RL    Vout
    |           Z Diode     R     
    |           |           R     
    +-----------+-----------+
\end{lstlisting}

\textbf{Working Principle:}

\begin{itemize}
\tightlist
\item
  Zener diode operates in reverse breakdown region
\item
  Maintains constant voltage across its terminals
\item
  Acts as voltage reference
\end{itemize}

\textbf{Circuit Operation:}

\begin{itemize}
\tightlist
\item
  Series resistor Rs limits current
\item
  Zener conducts when input exceeds breakdown voltage
\item
  Excess current flows through zener diode
\item
  Output voltage remains constant at zener voltage
\end{itemize}

\textbf{Advantages:}

\begin{itemize}
\tightlist
\item
  Simple circuit
\item
  Low cost
\item
  Good regulation for small load changes
\end{itemize}

\textbf{Limitations:}

\begin{itemize}
\tightlist
\item
  Power dissipation in zener and series resistor
\item
  Limited current capability
\item
  Temperature dependency
\end{itemize}

\end{solutionbox}
\begin{mnemonicbox}
``Zener Breaks Down to Hold Voltage Steady''

\end{mnemonicbox}
\subsection*{Question 4(a) OR [3
marks]}\label{q4a}

\textbf{What is rectifier? Explain full wave rectifier with waveforms.}

\begin{solutionbox}

\textbf{Rectifier:} A circuit that converts AC voltage to pulsating DC
voltage.

\textbf{Diagram:}

\begin{lstlisting}
    +-------+
    |       |
A --+       +-- C
    | XFRMR |         D1
    |       |--+------|>|----+--+
    |       |  |             |  |
    |       |  |             |  |  RL   Output
    |       |  |             |  |  
B --+       +--+             |  |
    |       |  |             |  |
    |       |  |             |  |
    |       |--+------|<|----+--+
    |       |         D2
    +-------+
\end{lstlisting}

\textbf{Waveforms:}

\begin{lstlisting}
Input:    ^
          |   /\    /\    /\
          |  /  \  /  \  /  \
          | /    \/    \/    \
          +--------------------
          |
          |\    /\    /\    /
          | \  /  \  /  \  / 
          |  \/    \/    \/

Output:   ^
          |   /\    /\    /\
          |  /  \  /  \  /  \
          | /    \/    \/    \
          +--------------------
\end{lstlisting}

\end{solutionbox}
\begin{mnemonicbox}
``Both Half-Cycles Become Positive''

\end{mnemonicbox}
\subsection*{Question 4(b) OR [4
marks]}\label{q4b}

\textbf{Why filter is required in rectifier? State the different types
of filter and explain any one type of filter.}

\begin{solutionbox}

\textbf{Need for Filter:}

\begin{itemize}
\tightlist
\item
  Rectifier output contains AC ripple component
\item
  Pure DC required for electronic circuits
\item
  Filters smooth pulsating DC by removing AC components
\end{itemize}

\textbf{Types of Filters:}

\begin{itemize}
\tightlist
\item
  Capacitor filter (C-filter)
\item
  Inductor filter (L-filter)
\item
  LC filter
\item
  π (Pi) filter
\item
  CLC filter
\end{itemize}

\textbf{Capacitor Filter:}

\includegraphics[width=1\linewidth,height=\textheight,keepaspectratio]{mermaid-ed62c8e1.pdf}

\textbf{Working:}

\begin{itemize}
\tightlist
\item
  Capacitor charges during voltage rise
\item
  Discharges slowly during voltage fall
\item
  Provides current when input decreases
\item
  Reduces ripple voltage
\end{itemize}

\textbf{Advantages:}

\begin{itemize}
\tightlist
\item
  Simple and inexpensive
\item
  Effective for light loads
\item
  Reduces ripple significantly
\end{itemize}

\end{solutionbox}
\begin{mnemonicbox}
``Capacitor Catches Peaks, Releases Slowly''

\end{mnemonicbox}
\subsection*{Question 4(c) OR [7
marks]}\label{q4c}

\textbf{Write the need of rectifier. Explain bridge rectifier with
circuit diagram and draw its input and output waveforms.}

\begin{solutionbox}

\textbf{Need of Rectifier:}

\begin{itemize}
\tightlist
\item
  Convert AC to DC for electronic devices
\item
  Most electronic circuits require DC power
\item
  Batteries provide DC but AC is distributed
\item
  Building block of power supplies
\item
  Essential for charging systems
\end{itemize}

\textbf{Bridge Rectifier Circuit:}

\begin{lstlisting}
          D1        D3
    +-----|>|--------+
    |                |
A --+                +-- DC+
    |                |
    |                |    RL
    |                |
B --+                +-- DC-
    |                |
    +-----|<|--------+
          D2        D4
\end{lstlisting}

\textbf{Input Waveform:}

\begin{lstlisting}
    ^
    |    /\      /\
    |   /  \    /  \
    |  /    \  /    \
    | /      \/      \
    +-------------------
    |       /\      /\
    |      /  \    /  \
    |     /    \  /    \
    |\   /      \/      
    | \ /
    |  V
\end{lstlisting}

\textbf{Output Waveform:}

\begin{lstlisting}
    ^
    |    /\      /\
    |   /  \    /  \
    |  /    \  /    \
    | /      \/      \
    +-------------------
\end{lstlisting}

\textbf{Working:}

\begin{itemize}
\tightlist
\item
  During positive half cycle: D1 and D4 conduct
\item
  During negative half cycle: D2 and D3 conduct
\item
  Load receives unidirectional current in both cycles
\item
  Utilizes both halves of input waveform
\end{itemize}

\end{solutionbox}
\begin{mnemonicbox}
``Four Diodes Direct All Current One Way''

\end{mnemonicbox}
\subsection*{Question 5(a) [3 marks]}\label{q5a}

\textbf{Explain causes of electronic waste.}

\begin{solutionbox}

\textbf{Causes of Electronic Waste:}

\begin{itemize}
\tightlist
\item
  Rapid technological advancement
\item
  Planned obsolescence of products
\item
  Decreasing product lifespan
\item
  Consumer behavior preferring new devices
\item
  Limited repair options for electronics
\item
  High repair costs compared to replacement
\end{itemize}

\end{solutionbox}
\begin{mnemonicbox}
``Technology Advances, Products Expire Rapidly''

\end{mnemonicbox}
\subsection*{Question 5(b) [4 marks]}\label{q5b}

\textbf{Compare PNP and NPN transistors.}

\begin{solutionbox}

{\def\LTcaptype{none} % do not increment counter
\begin{longtable}[]{@{}
  >{\raggedright\arraybackslash}p{(\linewidth - 4\tabcolsep) * \real{0.2727}}
  >{\raggedright\arraybackslash}p{(\linewidth - 4\tabcolsep) * \real{0.3636}}
  >{\raggedright\arraybackslash}p{(\linewidth - 4\tabcolsep) * \real{0.3636}}@{}}
\toprule\noalign{}
\begin{minipage}[b]{\linewidth}\raggedright
\textbf{Parameter}
\end{minipage} & \begin{minipage}[b]{\linewidth}\raggedright
\textbf{PNP Transistor}
\end{minipage} & \begin{minipage}[b]{\linewidth}\raggedright
\textbf{NPN Transistor}
\end{minipage} \\
\midrule\noalign{}
\endhead
\bottomrule\noalign{}
\endlastfoot
\textbf{Symbol} &
\pandocbounded{\includegraphics[keepaspectratio,alt={PNP}]{https://example.com/pnp.jpg}}
&
\pandocbounded{\includegraphics[keepaspectratio,alt={NPN}]{https://example.com/npn.jpg}} \\
\textbf{Majority Carriers} & Holes & Electrons \\
\textbf{Current Flow} & Emitter to Collector & Collector to Emitter \\
\textbf{Biasing} & Emitter more positive than Base & Base more positive
than Emitter \\
\textbf{Switching Speed} & Slower & Faster \\
\textbf{Applications} & Low frequency, high current & High frequency,
switching \\
\end{longtable}
}

\textbf{Diagram:}

\begin{lstlisting}
    NPN:         PNP:
    
    C            C
    |            |
    |            |
    B---->       <----B
    |            |
    |            |
    E            E
\end{lstlisting}

\end{solutionbox}
\begin{mnemonicbox}
``Negative-Positive-Negative vs
Positive-Negative-Positive''

\end{mnemonicbox}
\subsection*{Question 5(c) [7 marks]}\label{q5c}

\textbf{Draw the symbol, explain the construction and working of
MOSFET.}

\begin{solutionbox}

\textbf{Symbol:}

\begin{lstlisting}
         D (Drain)
         |
         |
G (Gate) |
----||---+
         |
         |
         S (Source)
\end{lstlisting}

\textbf{Construction:}

\includegraphics[width=1\linewidth,height=\textheight,keepaspectratio]{mermaid-74d305d6.pdf}

\textbf{Working Principle:}

\begin{itemize}
\tightlist
\item
  \textbf{Enhancement Mode N-Channel MOSFET:}

  \begin{itemize}
  \tightlist
  \item
    No channel exists without gate voltage
  \item
    Positive gate voltage attracts electrons from substrate
  \item
    Induced channel allows current flow from drain to source
  \item
    Increasing gate voltage enhances conductivity
  \end{itemize}
\item
  \textbf{Key Features:}

  \begin{itemize}
  \tightlist
  \item
    Voltage-controlled device (high input impedance)
  \item
    No gate current required (unlike BJT)
  \item
    Faster switching than BJT
  \item
    Lower power dissipation
  \end{itemize}
\end{itemize}

\textbf{Applications:}

\begin{itemize}
\tightlist
\item
  Digital logic circuits
\item
  Switching applications
\item
  Amplifiers
\item
  Power control devices
\end{itemize}

\end{solutionbox}
\begin{mnemonicbox}
``Gate Voltage Creates Electron Channel''

\end{mnemonicbox}
\subsection*{Question 5(a) OR [3
marks]}\label{q5a}

\textbf{Explain methods to handle electronic waste.}

\begin{solutionbox}

\textbf{Methods to Handle Electronic Waste:}

{\def\LTcaptype{none} % do not increment counter
\begin{longtable}[]{@{}
  >{\raggedright\arraybackslash}p{(\linewidth - 2\tabcolsep) * \real{0.4138}}
  >{\raggedright\arraybackslash}p{(\linewidth - 2\tabcolsep) * \real{0.5862}}@{}}
\toprule\noalign{}
\begin{minipage}[b]{\linewidth}\raggedright
\textbf{Method}
\end{minipage} & \begin{minipage}[b]{\linewidth}\raggedright
\textbf{Description}
\end{minipage} \\
\midrule\noalign{}
\endhead
\bottomrule\noalign{}
\endlastfoot
\textbf{Reduce} & Designing longer-lasting electronics, modular design
for upgrading \\
\textbf{Reuse} & Donating or selling functional devices, repurposing
components \\
\textbf{Recycle} & Proper dismantling and material recovery (precious
metals, plastics) \\
\textbf{Regulation} & E-waste management policies, extended producer
responsibility \\
\textbf{Recovery} & Extracting valuable materials through specialized
processes \\
\end{longtable}
}

\end{solutionbox}
\begin{mnemonicbox}
``Reduce, Reuse, Recycle, Regulate, Recover''

\end{mnemonicbox}
\subsection*{Question 5(b) OR [4
marks]}\label{q5b}

\textbf{Derive the relationship between αdc and βdc.}

\begin{solutionbox}

\textbf{Diagram:}

\begin{lstlisting}
               IC
              ↑
              |  
              |
    IB \rightarrow      |      \rightarrow  
        B ----+---- C
              |
              |
              |
              E
              ↓
              IE
\end{lstlisting}

\textbf{Transistor Current Relationships:}

\begin{itemize}
\tightlist
\item
  IE = IC + IB (Current entering equals current leaving)
\item
  αdc = IC/IE (Common Base current gain)
\item
  βdc = IC/IB (Common Emitter current gain)
\end{itemize}

\textbf{Derivation:}

\begin{itemize}
\tightlist
\item
  From IE = IC + IB
\item
  Divide both sides by IC: IE/IC = 1 + IB/IC
\item
  Therefore: 1/αdc = 1 + 1/βdc
\item
  Solving for βdc: βdc = αdc/(1-αdc)
\item
  And for αdc: αdc = βdc/(1+βdc)
\end{itemize}

\textbf{Table of Values:} \textbar{} αdc \textbar{} βdc \textbar{}
\textbar-----\textbar-----\textbar{} \textbar{} 0.9 \textbar{} 9
\textbar{} \textbar{} 0.95 \textbar{} 19 \textbar{} \textbar{} 0.99
\textbar{} 99 \textbar{}

\end{solutionbox}
\begin{mnemonicbox}
``Alpha-Beta Relate as αdc = βdc/(1+βdc)''

\end{mnemonicbox}
\subsection*{Question 5(c) OR [7
marks]}\label{q5c}

\textbf{Explain common collector configuration with its input and output
characteristics.}

\begin{solutionbox}

\textbf{Common Collector Circuit (Emitter Follower):}

\begin{lstlisting}
              +VCC
               |
               R
               |
               C
    +----+     |
    |    |     |
    Vin  |     +---+ Output
    |    |     |
    +----+-----+
         |     |
         B     E
         |     |
         +-----+
               |
               RE
               |
              GND
\end{lstlisting}

\textbf{Input Characteristics:} (IB vs VBE)

\begin{lstlisting}
    IB ^
       |           /
       |          /
       |         /
       |        /
       |       /
       |      /
       |     /
       |    /
       |   /
       |  /
       | /
       |/
       +--------------> VBE
\end{lstlisting}

\textbf{Output Characteristics:} (IE vs VCE)

\begin{lstlisting}
    IE ^
       |    ---------------
       |   /
       |  /
       | /
       |/
       +--------------> VCE
       
       IB3 > IB2 > IB1 > 0
\end{lstlisting}

\textbf{Key Features:}

\begin{itemize}
\tightlist
\item
  Voltage gain \approx 1 (slightly less)
\item
  High current gain (β+1)
\item
  High input impedance
\item
  Low output impedance
\item
  No phase inversion between input and output
\item
  Used as buffer/impedance matching circuit
\end{itemize}

\end{solutionbox}
\begin{mnemonicbox}
``Emitter Follows Base Voltage''

\end{mnemonicbox}

\end{document}
