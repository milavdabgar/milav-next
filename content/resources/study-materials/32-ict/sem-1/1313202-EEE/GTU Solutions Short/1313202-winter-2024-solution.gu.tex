\documentclass{article}

% content/resources/templates/preamble.tex
\usepackage[margin=0.6in]{geometry}
\author{Milav Dabgar}
\usepackage{amsmath,amssymb,amsthm}
\usepackage{booktabs}
\usepackage{multirow}
\usepackage{xcolor}
\usepackage{tcolorbox}
\tcbuselibrary{breakable,skins}
\usepackage[colorlinks=true,linkcolor=blue]{hyperref}
\usepackage{titlesec}
\usepackage{enumitem}
\usepackage{tikz}
\usepackage{pgfplots}
\usepackage{circuitikz}
\usepackage[version=4]{mhchem}
\usepackage{longtable}
\usepackage{array}
\usepackage{float}
\usepackage{caption}
\usepackage{listings}

\lstset{
  basicstyle=\small\ttfamily,
  breaklines=true,
  breakatwhitespace=false,
  postbreak=\mbox{\textcolor{red}{$\hookrightarrow$}\space},
  float=false,
  numbers=left,
  numberstyle=\tiny\color{gray},
  numbersep=10pt,
  xleftmargin=2em,
  keywordstyle=\color{blue},
  commentstyle=\color{green!60!black},
  stringstyle=\color{purple},
  backgroundcolor=\color{gray!5},
  showstringspaces=false,
  tabsize=2,
  captionpos=b,
  keepspaces=true,
  columns=flexible
}

\pgfplotsset{compat=1.18}
\usetikzlibrary{shapes,arrows,positioning,calc,patterns,decorations.pathmorphing,decorations.markings,arrows.meta}

% Color scheme
\definecolor{headcolor}{RGB}{0,102,204}
\definecolor{keycolor}{RGB}{220,20,60}
\definecolor{solutioncolor}{RGB}{34,139,34}
\definecolor{mnemoniccolor}{RGB}{148,0,211}
\definecolor{codecolor}{RGB}{0,0,100}

% Spacing
\setlength{\parskip}{3pt}
\setlist[itemize]{nosep}
\setlist[enumerate]{nosep}

% Title formatting
\titleformat{\section}{\Large\bfseries\color{headcolor}}{\thesection}{1em}{}
\titleformat{\subsection}{\large\bfseries\color{headcolor}}{\thesubsection}{1em}{}

% Pandoc tightlist compatibility
\providecommand{\tightlist}{%
  \setlength{\itemsep}{0pt}\setlength{\parskip}{0pt}}

% Pandoc longtable compatibility
\newcounter{none}
\def\thenone{}


% content/resources/templates/gujarati-boxes.tex
\usepackage{fontspec}
\usepackage{polyglossia}

% Set Gujarati as main language (document is primarily in Gujarati)
% Note: gloss-gujarati.ldf doesn't exist in polyglossia, but it will use hyphenation patterns
\setdefaultlanguage{gujarati}
\setotherlanguage{english}

% Configure Gujarati font properly
% Use Language=Default to prevent polyglossia from trying to add language-specific features
% that don't exist for Gujarati, which causes "empty feature" warnings
\newfontfamily\gujaratifont[Script=Gujarati,AutoFakeBold=2.5,AutoFakeSlant=0.3]{Noto Sans Gujarati}
\setmainfont[Script=Gujarati,AutoFakeBold=2.5,AutoFakeSlant=0.3]{Noto Sans Gujarati}
% Use Noto Sans Gujarati for monospace to support Gujarati in text
\setmonofont[Scale=0.9]{Noto Sans Gujarati}

% Configure English to use the same font
\newfontfamily\englishfont[Script=Gujarati,AutoFakeBold=2.5,AutoFakeSlant=0.3]{Noto Sans Gujarati}

% Translations for polyglossia
\gappto\captionsgujarati{
  \renewcommand{\tablename}{કોષ્ટક}
  \renewcommand{\figurename}{આકૃતિ}
}

% Helper for TikZ nodes to ensure Gujarati font
\newcommand{\gu}[1]{{\gujaratifont #1}}

% Custom environments
\newtcolorbox{solutionbox}{
    breakable,
    enhanced,
    colback=solutioncolor!5!white,
    colframe=solutioncolor!75!black,
    fonttitle=\bfseries,
    title=જવાબ
}

\newtcolorbox{solutionboxnobreak}{
 colback=solutioncolor!5!white,
 colframe=solutioncolor!75!black,
 fonttitle=\bfseries,
 title=જવાબ
}

\newtcolorbox{keyformula}{
 breakable,
 enhanced,
 colback=keycolor!5!white,
 colframe=keycolor!75!black,
 fonttitle=\bfseries,
 title=રાસાયણિક સમીકરણ/સૂત્ર
}

\newtcolorbox{mnemonicbox}{
 breakable,
 enhanced,
 colback=mnemoniccolor!5!white,
 colframe=mnemoniccolor!75!black,
 fonttitle=\bfseries,
 title=મેમરી ટ્રીક
}


% Custom commands for GTU solutions
% This file defines semantic commands for consistent formatting

% Question command with automatic formatting
\newcommand{\question}[2]{%
  \section*{Question #1}%
  \textbf{#2}%
}

% OR question variant
\newcommand{\questionor}[2]{%
  \section*{Question #1 OR}%
  \textbf{#2}%
}

% Proper table environment with caption
\newenvironment{answertable}[1]{%
  \begin{table}[htbp]
  \centering
  \caption{#1}
}{%
  \end{table}
}

% Proper figure environment for diagrams
\newenvironment{answerdiagram}[1]{%
  \begin{figure}[htbp]
  \centering
  \caption{#1}
}{%
  \end{figure}
}

% Semantic markup for key terms
\newcommand{\keyword}[1]{\textbf{#1}}
\newcommand{\code}[1]{\texttt{#1}}
\newcommand{\classname}[1]{\texttt{#1}}
\newcommand{\methodname}[1]{\texttt{#1}}

% Proper quotation marks
\newcommand{\mnemonic}[1]{``#1''}


\title{Elements of Electrical \& Electronics Engineering (1313202) - Winter 2024 Solution}
\date{January 17, 2024}

\begin{document}
\maketitle

\questionmarks{1(a)}{3}{એક્ટિવ અને પેસિવ નેટવર્ક નો તફાવત સમજાવો.}

\begin{solutionbox}
\begin{tabulary}{\linewidth}{|L|L|}
\hline
\textbf{એક્ટિવ નેટવર્ક} & \textbf{પેસિવ નેટવર્ક} \\ \hline
ઓછામાં ઓછું એક એક્ટિવ ઘટક (વોલ્ટેજ/કરંટ સ્ત્રોત) ધરાવે છે. & માત્ર પેસિવ ઘટકો (R, L, C) ધરાવે છે. \\ \hline
સર્કિટમાં ઊર્જા આપી શકે છે. & સર્કિટમાં ઊર્જા આપી શકતું નથી. \\ \hline
સિગ્નલ પાવરને વધારી શકે છે. & સિગ્નલ પાવરને વધારી શકતું નથી. \\ \hline
\end{tabulary}
\end{solutionbox}

\begin{mnemonicbox}
\mnemonic{Active Adds Power, Passive Parts Take}
\end{mnemonicbox}

\questionmarks{1(b)}{4}{કિર્ચોફનો વોલ્ટેજનો નિયમ જણાવો અને સમજાવો.}

\begin{solutionbox}
\textbf{નિયમ:} KVL કહે છે કે સર્કિટમાં કોઈપણ બંધ લૂપની અંદર બધા વોલ્ટેજનો બીજગણિતીય સરવાળો શૂન્ય થાય છે.

\textbf{ગણિતશાસ્ત્ર મુજબ:} $V_1 + V_2 + V_3 + V_4 = 0$

\begin{center}
\begin{tikzpicture}[
    node distance=2.5cm,
    state/.style={circle, draw, minimum size=1cm}
]
    \node[state] (A) {A};
    \node[state] (B) [right of=A] {B};
    \node[state] (D) [below of=A] {D};
    \node[state] (C) [below of=B] {C};

    \draw[->] (A) -- node[above] {$V_1$} (B);
    \draw[->] (B) -- node[right] {$V_2$} (C);
    \draw[->] (C) -- node[below] {$V_3$} (D);
    \draw[->] (D) -- node[left] {$V_4$} (A);
\end{tikzpicture}
\end{center}

\textbf{સંજ્ઞા પ્રણાલી:}
\begin{itemize}
    \item \textbf{વોલ્ટેજ ડ્રોપ}: કરંટની દિશામાં રેઝિસ્ટર પસાર કરતી વખતે વોલ્ટેજ નેગેટિવ છે.
    \item \textbf{વોલ્ટેજ વધારો}: નેગેટિવથી પોઝિટિવ તરફ સ્ત્રોત પસાર કરતી વખતે વોલ્ટેજ પોઝિટિવ છે.
\end{itemize}
\end{solutionbox}

\begin{mnemonicbox}
\mnemonic{Voltage Loop Equals Zero}
\end{mnemonicbox}

\questionmarks{1(c)}{7}{વ્યાખ્યા આપો: (1) ચાર્જ (2) કરંટ (3) પોટેન્શિયલ (4) E.M.F. (5) ઇન્ડક્ટન્સ (6) કેપેસિટન્સ (7) આવૃત્તિ.}

\begin{solutionbox}
\begin{tabulary}{\linewidth}{|L|L|}
\hline
\textbf{શબ્દ} & \textbf{વ્યાખ્યા} \\ \hline
\textbf{ચાર્જ} & કૂલમ્બ (C)માં માપવામાં આવતો વીજળીનો જથ્થો. \\ \hline
\textbf{કરંટ} & એમ્પિયર (A)માં માપવામાં આવતો વીજળીના ચાર્જનો પ્રવાહ દર. \\ \hline
\textbf{પોટેન્શિયલ} & વોલ્ટ (V)માં માપવામાં આવતું એકમ ચાર્જ દીઠ વીજળીય દબાણ અથવા ઊર્જા. \\ \hline
\textbf{E.M.F.} & એકમ ચાર્જ દીઠ સ્ત્રોત દ્વારા પ્રદાન કરેલી ઊર્જા (V). \\ \hline
\textbf{ઇન્ડક્ટન્સ} & હેનરી (H)માં માપવામાં આવતો વીજળીય સર્કિટનો ગુણ જે કરંટમાં ફેરફારનો વિરોધ કરે છે. \\ \hline
\textbf{કેપેસિટન્સ} & ફેરડ (F)માં માપવામાં આવતી વીજળીય ચાર્જ સંગ્રહ કરવાની ક્ષમતા. \\ \hline
\textbf{આવૃત્તિ} & હર્ટ્ઝ (Hz)માં માપવામાં આવતી પ્રતિ સેકન્ડ પૂર્ણ થયેલા ચક્રોની સંખ્યા. \\ \hline
\end{tabulary}
\end{solutionbox}

\begin{mnemonicbox}
\mnemonic{Coulombs' Flow Pressurized by Energy Induces Capacitive Fluctuations}
\end{mnemonicbox}

\questionmarks{1(c) OR}{7}{ઓહમનો નિયમ જણાવો. તેના ઉપયોગો અને મર્યાદા લખો.}

\begin{solutionbox}
\textbf{નિયમ:} ઓહમનો નિયમ કહે છે કે વાહક દ્વારા વહેતો કરંટ પોટેન્શિયલ તફાવતના સમપ્રમાણમાં અને અવરોધના વ્યસ્ત પ્રમાણમાં હોય છે (તાપમાન અચળ રહેવું જોઈએ).

\textbf{સૂત્ર:} $V = I \times R$

\begin{center}
\begin{circuitikz}
    \draw (0,0) to[battery1, l=$V$] (0,2) -- (2,2) to[R, l=$R$, i=$I$] (2,0) -- (0,0);
\end{circuitikz}
\end{center}

\textbf{ઉપયોગો:}
\begin{itemize}
    \item સર્કિટ ડિઝાઇન અને વિશ્લેષણ.
    \item પાવર વપરાશની ગણતરીઓ.
    \item ઘટક મૂલ્ય નક્કી કરવા.
    \item વોલ્ટેજ/કરંટ ડિવાઇડર નેટવર્ક.
\end{itemize}

\textbf{મર્યાદાઓ:}
\begin{itemize}
    \item માત્ર લીનિયર ઘટકો માટે માન્ય.
    \item ડાયોડ, ટ્રાન્ઝિસ્ટર માટે લાગુ પડતો નથી.
    \item ઉચ્ચ તાપમાને અમાન્ય.
    \item સેમિકન્ડક્ટર્સ માટે માન્ય નથી.
\end{itemize}
\end{solutionbox}

\begin{mnemonicbox}
\mnemonic{Volts Reveal Amps' Motion}
\end{mnemonicbox}

\questionmarks{2(a)}{3}{વાહક, અવાહક અને અર્ધવાહક નો એનર્જી બેન્ડ ની આકૃતિ દોરી સમજાવો.}

\begin{solutionbox}
\begin{center}
\begin{tikzpicture}[scale=0.7]
    \node at (1.5,4) {\textbf{વાહક}};
    \draw (0,0) rectangle (3,1.5); \node at (1.5,0.75) {Valence};
    \draw (0,1.2) rectangle (3,2.7); \node at (1.5,2.2) {Conduction};
    \node at (1.5,1.35) {\small Overlap};

    \node at (5.5,4) {\textbf{અર્ધવાહક}};
    \draw (4,0) rectangle (7,1.2); \node at (5.5,0.6) {Valence};
    \draw (4,2.0) rectangle (7,3.2); \node at (5.5,2.6) {Conduction};
    \node at (5.5,1.6) {\small Small Gap};

    \node at (9.5,4) {\textbf{અવાહક}};
    \draw (8,0) rectangle (11,1.0); \node at (9.5,0.5) {Valence};
    \draw (8,2.5) rectangle (11,3.5); \node at (9.5,3.0) {Conduction};
    \node at (9.5,1.75) {\small Large Gap};
\end{tikzpicture}
\end{center}

\begin{itemize}
    \item \textbf{વાહક}: વેલેન્સ અને કન્ડક્શન બેન્ડ ઓવરલેપ થાય છે.
    \item \textbf{અર્ધવાહક}: નાની ઊર્જા ગેપ (0.7-3 eV) મર્યાદિત કન્ડક્શન આપે છે.
    \item \textbf{અવાહક}: મોટી ઊર્જા ગેપ ($>3$ eV) કન્ડક્શન અટકાવે છે.
\end{itemize}
\end{solutionbox}

\begin{mnemonicbox}
\mnemonic{Conductors Overlap, Semiconductors Jump Small, Insulators Block All}
\end{mnemonicbox}

\questionmarks{2(b)}{4}{Maximum power transfer theorem અને reciprocity theorem નું સ્ટેટમેન્ટ લખો.}

\begin{solutionbox}
\begin{tabulary}{\linewidth}{|L|L|}
\hline
\textbf{થિયરમ} & \textbf{સ્ટેટમેન્ટ} \\ \hline
\textbf{MPT થિયરમ} & સ્ત્રોતથી લોડમાં મહત્તમ પાવર ત્યારે ટ્રાન્સફર થાય છે જ્યારે લોડ રેઝિસ્ટન્સ સ્ત્રોતના આંતરિક અવરોધ જેટલો હોય ($R_L = R_S$). \\ \hline
\textbf{રેસિપ્રોસિટી થિયરમ} & લીનિયર નેટવર્કમાં, જો વોલ્ટેજ સ્ત્રોત $E$ બ્રાન્ચ 1માં કરંટ $I$ બ્રાન્ચ 2માં આપે છે, તો એ જ સ્ત્રોત બ્રાન્ચ 2માં મૂકવાથી બ્રાન્ચ 1માં એ જ કરંટ $I$ મળશે. \\ \hline
\end{tabulary}
\end{solutionbox}

\begin{mnemonicbox}
\mnemonic{Match Resistance for Maximum Power; Swap Sources, Current Stays}
\end{mnemonicbox}

\questionmarks{2(c)}{7}{N-type મટીરીઅલ ની રચના અને તેનું કંડક્શન સમજાવો.}

\begin{solutionbox}
\textbf{રચના પ્રક્રિયા:}
\begin{itemize}
    \item શુદ્ધ સિલિકોન/જર્મેનિયમમાં પેન્ટાવેલેન્ટ અશુદ્ધિ (P, As, Sb) ઉમેરવામાં આવે છે.
    \item અશુદ્ધિ અણુઓમાં 5 વેલેન્સ ઇલેક્ટ્રોન હોય છે.
    \item ચાર કોવેલેન્ટ બોન્ડ બનાવે છે, પાંચમો ફ્રી ઇલેક્ટ્રોન બને છે.
\end{itemize}

\begin{center}
\begin{tikzpicture}
    \foreach \x in {0,2,4} \foreach \y in {0,2} \node[circle,draw,inner sep=2pt] at (\x,\y) {Si};
    \node[circle,draw,fill=gray!20,inner sep=2pt] at (2,2) {P};
    \draw (0.3,0)--(1.7,0); \draw (2.3,0)--(3.7,0);
    \draw (0.3,2)--(1.7,2); \draw (2.3,2)--(3.7,2);
    \draw (0,0.3)--(0,1.7); \draw (2,0.3)--(2,1.7); \draw (4,0.3)--(4,1.7);
    \draw[fill=black] (2.5,2.5) circle (0.1) node[right] {Free $e^-$};
\end{tikzpicture}
\end{center}

\textbf{કંડક્શન મેકેનિઝમ:}
\begin{itemize}
    \item \textbf{મેજોરિટી કેરિયર્સ}: ઇલેક્ટ્રોન.
    \item \textbf{માઇનોરિટી કેરિયર્સ}: હોલ્સ.
    \item ઇલેક્ટ્રોનની ગતિ વીજળીય કંડક્શન આપે છે.
\end{itemize}
\end{solutionbox}

\begin{mnemonicbox}
\mnemonic{Pentavalent Provides Plus-One Electron}
\end{mnemonicbox}

\questionmarks{2(a) OR}{3}{વેલેન્સ બેન્ડ, કંડક્શન બેન્ડ અને ફોર્બિડન ગેપ ની વ્યાખ્યા આપો.}

\begin{solutionbox}
\begin{tabulary}{\linewidth}{|L|L|}
\hline
\textbf{શબ્દ} & \textbf{વ્યાખ્યા} \\ \hline
\textbf{વેલેન્સ બેન્ડ} & કણો સાથે બંધાયેલા વેલેન્સ ઇલેક્ટ્રોન દ્વારા રોકાયેલ ઊર્જા બેન્ડ. \\ \hline
\textbf{કંડક્શન બેન્ડ} & ઉચ્ચ ઊર્જા બેન્ડ જ્યાં ઇલેક્ટ્રોન મુક્તરીતે ફરી શકે છે. \\ \hline
\textbf{ફોર્બિડન ગેપ} & વેલેન્સ અને કંડક્શન બેન્ડ વચ્ચેનો ઊર્જા વિસ્તાર જ્યાં કોઈ ઇલેક્ટ્રોન હોતા નથી. \\ \hline
\end{tabulary}
\end{solutionbox}

\begin{mnemonicbox}
\mnemonic{Valence Binds, Conduction Flows, Forbidden Gaps Block}
\end{mnemonicbox}

\questionmarks{2(b) OR}{4}{એક્ટિવ પાવર, રિએક્ટિવ પાવર અને પાવર ફેક્ટર ની વ્યાખ્યા આપો અને પાવર ત્રિકોણ દોરો.}

\begin{solutionbox}
\textbf{પાવર ત્રિકોણ:}
\begin{center}
\begin{tikzpicture}
    \draw[thick] (0,0) -- (4,0) node[midway, below] {Active ($P$)};
    \draw[thick] (4,0) -- (4,3) node[midway, right] {Reactive ($Q$)};
    \draw[thick] (0,0) -- (4,3) node[midway, above left] {Apparent ($S$)};
    \draw (0.5,0) arc (0:36.87:0.5); \node at (0.8,0.3) {$\theta$};
\end{tikzpicture}
\end{center}

\begin{itemize}
    \item \textbf{એક્ટિવ પાવર (P)}: વાસ્તવિક પાવર (W), $P = VI \cos\theta$.
    \item \textbf{રિએક્ટિવ પાવર (Q)}: ઓસિલેટિંગ પાવર (VAR), $Q = VI \sin\theta$.
    \item \textbf{પાવર ફેક્ટર}: $PF = \cos\theta = P/S$.
\end{itemize}
\end{solutionbox}

\begin{mnemonicbox}
\mnemonic{Real Power Works, Reactive Power Waits}
\end{mnemonicbox}

\questionmarks{2(c) OR}{7}{ટ્રાઇવેલેન્ટ, ટેટ્રાવેલેન્ટ અને પેન્ટાવેલેન્ટ તત્વોના અણુની રચના સમજાવો.}

\begin{solutionbox}
\begin{center}
\begin{tikzpicture}[scale=0.8]
    \node at (2,4) {\textbf{Trivalent}}; \draw (2,2) circle (1.5); \node at (2,0) {3 $e^-$};
    \node at (6,4) {\textbf{Tetravalent}}; \draw (6,2) circle (1.5); \node at (6,0) {4 $e^-$};
    \node at (10,4) {\textbf{Pentavalent}}; \draw (10,2) circle (1.5); \node at (10,0) {5 $e^-$};
\end{tikzpicture}
\end{center}

\begin{tabulary}{\linewidth}{|L|L|L|}
\hline
\textbf{પ્રકાર} & \textbf{રચના (બાહ્ય કક્ષા)} & \textbf{ઉદાહરણો} \\ \hline
\textbf{ટ્રાઇવેલેન્ટ} & 3 ઇલેક્ટ્રોન & B, Al, Ga (P-ટાઇપ) \\ \hline
\textbf{ટેટ્રાવેલેન્ટ} & 4 ઇલેક્ટ્રોન & Si, Ge, C (સેમિકન્ડક્ટર) \\ \hline
\textbf{પેન્ટાવેલેન્ટ} & 5 ઇલેક્ટ્રોન & P, As, Sb (N-ટાઇપ) \\ \hline
\end{tabulary}
\end{solutionbox}

\begin{mnemonicbox}
\mnemonic{Three Accepts, Four Forms, Five Donates}
\end{mnemonicbox}

\questionmarks{3(a)}{3}{ફોટોડિઓડનું પ્રતીક દોરો અને તેનો ઉપયોગ જણાવો.}

\begin{solutionbox}
\textbf{પ્રતીક:}
\begin{center}
\begin{circuitikz} \draw (0,0) to[photodiode] (2,0); \end{circuitikz}
\captionof{figure}{ફોટોડાયોડ}
\end{center}

\textbf{ઉપયોગો:} લાઇટ સેન્સર, ઓપ્ટિકલ કમ્યુનિકેશન, સોલર સેલ્સ, કેમેરા, મેડિકલ સાધનો.
\end{solutionbox}

\begin{mnemonicbox}
\mnemonic{Light Triggers Electric Current}
\end{mnemonicbox}

\questionmarks{3(b)}{4}{LED પર ટૂંકી નોંધ લખો.}

\begin{solutionbox}
\textbf{પ્રતીક:}
\begin{center}
\begin{circuitikz} \draw (0,0) to[led] (2,0); \end{circuitikz}
\captionof{figure}{LED}
\end{center}

\textbf{માહિતી:}
\begin{itemize}
    \item P-N જંક્શન ડાયોડ જે ફોરવર્ડ બાયસમાં પ્રકાશ આપે છે.
    \item રીકોમ્બિનેશનથી ફોટોન્સ મુક્ત થાય છે.
    \item \textbf{ફાયદા}: ઓછો પાવર, લાંબી લાઇફ.
    \item \textbf{ઉપયોગો}: ડિસ્પ્લે, ઇન્ડિકેટર્સ, લાઇટિંગ.
\end{itemize}
\end{solutionbox}

\begin{mnemonicbox}
\mnemonic{Electrons Jump, Photons Emit}
\end{mnemonicbox}

\questionmarks{3(c)}{7}{PN જંક્શન ડાયોડની VI લાક્ષણિકતા દોરીને સમજાવો.}

\begin{solutionbox}
\textbf{V-I લાક્ષણિકતાઓ:}
\begin{center}
\begin{tikzpicture}
    \draw[->] (-3,0) -- (3,0) node[right] {$V$};
    \draw[->] (0,-3) -- (0,3) node[above] {$I$};
    \draw[blue, thick] plot[domain=0:2.5] (\x, {0.1*exp(\x)});
    \draw[red, thick] (-3,-0.5) -- (0,-0.1);
    \node at (2,2) {Forward}; \node at (-2,-1) {Reverse};
\end{tikzpicture}
\end{center}

\begin{itemize}
    \item \textbf{ફોરવર્ડ બાયસ}: કટ-ઇન વોલ્ટેજ (Si: 0.7V) પછી કરંટ ઝડપથી વધે છે.
    \item \textbf{રિવર્સ બાયસ}: ખૂબ જ નાનો લીકેજ કરંટ વહે છે. ઉચ્ચ વોલ્ટેજે બ્રેકડાઉન થાય છે.
\end{itemize}
\end{solutionbox}

\begin{mnemonicbox}
\mnemonic{Forward Flows Freely, Reverse Resists Rigidly}
\end{mnemonicbox}

\questionmarks{3(a) OR}{3}{PN જંક્શન ડાયોડના ઉપયોગોની યાદી બનાવો.}

\begin{solutionbox}
\textbf{ઉપયોગો:} રેક્ટિફિકેશન, ડિમોડ્યુલેશન, લોજિક ગેટ્સ, વોલ્ટેજ રેગ્યુલેશન, ક્લિપિંગ/ક્લેમ્પિંગ, પ્રોટેક્શન.
\end{solutionbox}

\begin{mnemonicbox}
\mnemonic{Rectify, Detect, Clip, Protect}
\end{mnemonicbox}

\questionmarks{3(b) OR}{4}{અનબાયસ PN જંક્શન ડાયોડ ના ડીપલીશન રીજીયન ની રચના સમજાવો.}

\begin{solutionbox}
\textbf{રચના:}
\begin{itemize}
    \item ડિફ્યુઝન: N થી ઇલેક્ટ્રોન P માં, P થી હોલ્સ N માં જાય છે.
    \item રીકોમ્બિનેશન થાય છે.
    \item જંક્શન પાસે આયનો (N માં +, P માં -) જમા થાય છે.
    \item ઇલેક્ટ્રિક ફીલ્ડ વધુ ડિફ્યુઝન અટકાવે છે $\rightarrow$ ડિપ્લેશન રીજિયન.
\end{itemize}

\begin{center}
\begin{tikzpicture}
    \draw (0,0) rectangle (4,2); \draw (2,0)--(2,2);
    \node at (1,1) {P (- Ions)}; \node at (3,1) {N (+ Ions)};
    \node at (2,2.3) {Depletion Region};
\end{tikzpicture}
\end{center}
\end{solutionbox}

\begin{mnemonicbox}
\mnemonic{Diffusion Creates Barrier Field}
\end{mnemonicbox}

\questionmarks{3(c) OR}{7}{PN જંક્શન ડાયોડનું બાંધકામ, કાર્ય અને એપ્લિકેશન સમજાવો.}

\begin{solutionbox}
\textbf{બાંધકામ:} P અને N સેમિકન્ડક્ટરનું જોડાણ. એનોડ અને કેથોડ સંપર્કો.

\begin{center}
\begin{tikzpicture}
    \draw (0,0) rectangle (4,1.5); \draw (2,0)--(2,1.5);
    \node at (1,0.75) {P (Anode)}; \node at (3,0.75) {N (Cathode)};
\end{tikzpicture}
\end{center}

\textbf{કાર્ય:}
\begin{itemize}
    \item \textbf{ફોરવર્ડ (P to +, N to -)}: ડિપ્લેશન રીજિયન ઘટે છે, કરંટ વહે છે.
    \item \textbf{રિવર્સ (N to +, P to -)}: ડિપ્લેશન રીજિયન વધે છે, કરંટ અટકે છે.
\end{itemize}

\textbf{એપ્લિકેશન:} રેક્ટિફાયર, સ્વિચ, વગેરે.
\end{solutionbox}

\begin{mnemonicbox}
\mnemonic{Join P-N, Control Current Direction}
\end{mnemonicbox}

\questionmarks{4(a)}{3}{વ્યાખ્યા આપો: (1) રીપપલ આવૃત્તિ (2) રીપપલ ફેક્ટર (3) ડાયોડ નો PIV.}

\begin{solutionbox}
\begin{tabulary}{\linewidth}{|L|L|}
\hline
\textbf{શબ્દ} & \textbf{વ્યાખ્યા} \\ \hline
\textbf{રીપપલ આવૃત્તિ} & રેક્ટિફાયડ DC આઉટપુટમાં બાકી રહેલ AC ઘટકની આવૃત્તિ (ફુલ-વેવ માટે 2$\times$ ઇનપુટ આવૃત્તિ, હાફ-વેવ માટે 1$\times$). \\ \hline
\textbf{રીપપલ ફેક્ટર} & રેક્ટિફાયર આઉટપુટમાં DC ઘટક સાથે AC ઘટકના RMS મૂલ્યનો ગુણોત્તર ($\gamma = V_{ac(rms)}/V_{dc}$). \\ \hline
\textbf{PIV} & પીક ઇન્વર્સ વોલ્ટેજ એ મહત્તમ રિવર્સ વોલ્ટેજ છે જે ડાયોડ બ્રેકડાઉન વિના સહન કરી શકે છે. \\ \hline
\end{tabulary}
\end{solutionbox}

\begin{mnemonicbox}
\mnemonic{Frequency Fluctuates, Factor Measures, PIV Protects}
\end{mnemonicbox}

\questionmarks{4(b)}{4}{બે ડાયોડ ફુલ વેવ રેક્ટિફાયર અને બ્રિજ રેક્ટિફાયર નો તફાવત આપો.}

\begin{solutionbox}
\begin{tabulary}{\linewidth}{|L|L|L|}
\hline
\textbf{પેરામીટર} & \textbf{સેન્ટર-ટેપ્ડ ફુલ વેવ} & \textbf{બ્રિજ રેક્ટિફાયર} \\ \hline
\textbf{ડાયોડની સંખ્યા} & 2 & 4 \\ \hline
\textbf{ટ્રાન્સફોર્મર} & સેન્ટર-ટેપ્ડ જરૂરી & સાદું ટ્રાન્સફોર્મર \\ \hline
\textbf{PIV} & $2V_m$ & $V_m$ \\ \hline
\textbf{કાર્યક્ષમતા} & 81.2\% & 81.2\% \\ \hline
\textbf{રીપપલ ફેક્ટર} & 0.48 & 0.48 \\ \hline
\textbf{આઉટપુટ} & $V_m/\pi$ & $2V_m/\pi$ \\ \hline
\textbf{ખર્ચ} & ઊંચો ટ્રાન્સફોર્મર ખર્ચ & ઊંચો ડાયોડ ખર્ચ \\ \hline
\end{tabulary}
\end{solutionbox}

\begin{mnemonicbox}
\mnemonic{Two Diodes Tap Center, Four Make Bridge}
\end{mnemonicbox}

\questionmarks{4(c)}{7}{ઝેનર ડાયોડને વોલ્ટેજ રેગ્યુલેટર તરીકે સમજાવો.}

\begin{solutionbox}
\textbf{સર્કિટ ડાયાગ્રામ:}
\begin{center}
\begin{circuitikz}[scale=0.9]
    \draw (0,0) to[sV, l=$V_{in}$] (0,3) to[R, l=$R_S$] (3,3) -- (5,3);
    \draw (3,3) to[zD*, l=$D_Z$] (3,0);
    \draw (5,3) to[R, l=$R_L$] (5,0);
    \draw (0,0) -- (5,0);
    \node at (5.5, 1.5) {$V_{out} = V_Z$};
\end{circuitikz}
\captionof{figure}{ઝેનર વોલ્ટેજ રેગ્યુલેટર}
\end{center}

\textbf{કાર્ય સિદ્ધાંત:}
\begin{itemize}
    \item ઝેનર ડાયોડ રિવર્સ બ્રેકડાઉન રીજીયનમાં કાર્ય કરે છે.
    \item તેના ટર્મિનલ્સ પર સ્થિર વોલ્ટેજ જાળવે છે.
\end{itemize}

\textbf{સર્કિટ ઓપરેશન:}
\begin{itemize}
    \item સીરીઝ રેઝિસ્ટર $R_S$ કરંટને મર્યાદિત કરે છે.
    \item જ્યારે ઇનપુટ $> V_Z$ હોય ત્યારે ઝેનર કંડક્ટ કરે છે.
    \item વધારાનો કરંટ ઝેનર ડાયોડ મારફતે વહે છે, લોડ વોલ્ટેજ $V_Z$ પર સ્થિર રહે છે.
\end{itemize}
\end{solutionbox}

\begin{mnemonicbox}
\mnemonic{Zener Breaks Down to Hold Voltage Steady}
\end{mnemonicbox}

\questionmarks{4(a) OR}{3}{રેક્ટિફાયર શું છે? ફુલ વેવ રેક્ટિફાયરને વેવફોર્મ્સ સાથે સમજાવો.}

\begin{solutionbox}
\textbf{રેક્ટિફાયર:} AC વોલ્ટેજને પલ્સેટિંગ DC વોલ્ટેજમાં રૂપાંતરિત કરતી સિસ્ટમ.

\textbf{ફુલ વેવ રેક્ટિફાયર:}
\begin{center}
\begin{circuitikz}[scale=0.8]
    \draw (0,0) node[transformer core] (T) {};
    \draw (T.A1) -- ++(-0.5,0) node[left] {AC};
    \draw (T.A2) -- ++(-0.5,0);
    \draw (T.B1) to[D*, l=$D_1$] (3,1);
    \draw (T.B2) to[D*, l=$D_2$] (3,-1);
    \draw (3,1) -- (3,-1);
    \draw (3,0) to[R, l=$R_L$] (5,0);
    \draw (T.base) -- (5,0) |- (5,0);
\end{circuitikz}
\end{center}

\textbf{વેવફોર્મ્સ:} ઇનપુટ સાઇન વેવ છે, આઉટપુટમાં બંને હાફ-સાયકલ પોઝિટિવ હોય છે.
\end{solutionbox}

\begin{mnemonicbox}
\mnemonic{Both Half-Cycles Become Positive}
\end{mnemonicbox}

\questionmarks{4(b) OR}{4}{રેક્ટિફાયરમાં ફિલ્ટર શા માટે જરૂરી છે? ફિલ્ટરના વિવિધ પ્રકારો જણાવો અને કોઈપણ એક પ્રકારનું ફિલ્ટર સમજાવો.}

\begin{solutionbox}
\textbf{જરૂરિયાત:} રેક્ટિફાયર આઉટપુટમાં AC રિપપલ હોય છે. ઇલેક્ટ્રોનિક્સ માટે શુદ્ધ DC જરૂરી છે. ફિલ્ટર્સ રિપપલ દૂર કરે છે.

\textbf{પ્રકારો:} C, L, LC, $\pi$, CLC.

\textbf{કેપેસિટર ફિલ્ટર:}
\begin{center}
\begin{circuitikz}
    \draw (0,0) to[short, o-o] (0,2);
    \draw (0,2) -- (2,2) to[C, l=C] (2,0) -- (0,0);
    \draw (2,2) -- (4,2) to[R, l=$R_L$] (4,0) -- (2,0);
\end{circuitikz}
\captionof{figure}{કેપેસિટર ફિલ્ટર}
\end{center}

\textbf{કાર્ય:} વોલ્ટેજ વધે ત્યારે કેપેસિટર ચાર્જ થાય છે, ઘટે ત્યારે ડિસ્ચાર્જ થાય છે $\rightarrow$ સ્મૂધ આઉટપુટ.
\end{solutionbox}

\begin{mnemonicbox}
\mnemonic{Capacitor Catches Peaks, Releases Slowly}
\end{mnemonicbox}

\questionmarks{4(c) OR}{7}{રેક્ટિફાયરની જરૂરિયાત લખો. સર્કિટ ડાયાગ્રામ વડે બ્રિજ રેક્ટિફાયર સમજાવો અને તેના ઇનપુટ અને આઉટપુટ વેવફોર્મ્સ દોરો.}

\begin{solutionbox}
\textbf{જરૂરિયાત:} AC ને DC માં ફેરવવા (ઇલેક્ટ્રોનિક્સ, ચાર્જિંગ વગેરે માટે).

\textbf{બ્રિજ રેક્ટિફાયર:}
\begin{center}
\begin{circuitikz}[scale=0.8]
    \draw (0,0) to[sV, l=$V_{in}$] (0,3);
    \draw (0,3) -- (2,3) -- (3,2);
    \draw (0,0) -- (2,0) -- (5,0) -- (5,2);
    \draw (3,2) to[D*, l=$D_1$] (4,3);
    \draw (4,3) to[D*, l=$D_2$] (5,2);
    \draw (5,2) to[D*, l=$D_3$] (4,1);
    \draw (4,1) to[D*, l=$D_4$] (3,2);
    \draw (4,3) -- (4,4) -- (7,4) to[R, l=$R_L$] (7,0) -- (5,0);
    \draw (4,1) -- (4,0);
\end{circuitikz}
\captionof{figure}{બ્રિજ રેક્ટિફાયર}
\end{center}

\textbf{વેવફોર્મ્સ:} $D_1, D_4$ (+ સાયકલ) અને $D_2, D_3$ (- સાયકલ) કંડક્ટ કરે છે. આઉટપુટ યુનિડાયરેક્શનલ છે.
\end{solutionbox}

\begin{mnemonicbox}
\mnemonic{Four Diodes Direct All Current One Way}
\end{mnemonicbox}

\questionmarks{5(a)}{3}{ઇલેક્ટ્રોનિક કચરાના કારણો સમજાવો.}

\begin{solutionbox}
\textbf{કારણો:}
\begin{itemize}
    \item ઝડપી ટેકનોલોજીકલ અદ્યતનીકરણ.
    \item ઉત્પાદનોની આયોજિત કાલગ્રસ્તતા (Planned Obsolescence).
    \item ઉત્પાદનોનું ઘટતું જીવનકાળ.
    \item નવા ઉપકરણોને પસંદ કરતી ગ્રાહક વર્તણૂક.
    \item રિપેરિંગને બદલે રિપ્લેસમેન્ટ.
\end{itemize}
\end{solutionbox}

\begin{mnemonicbox}
\mnemonic{Technology Advances, Products Expire Rapidly}
\end{mnemonicbox}

\questionmarks{5(b)}{4}{PNP અને NPN ટ્રાન્ઝિસ્ટરની સરખામણી કરો.}

\begin{solutionbox}
\begin{tabulary}{\linewidth}{|L|L|L|}
\hline
\textbf{પેરામીટર} & \textbf{PNP ટ્રાન્ઝિસ્ટર} & \textbf{NPN ટ્રાન્ઝિસ્ટર} \\ \hline
\textbf{સિમ્બોલ} & \begin{circuitikz}[scale=0.6]\draw(0,0)node[pnp]{};\end{circuitikz} & \begin{circuitikz}[scale=0.6]\draw(0,0)node[npn]{};\end{circuitikz} \\ \hline
\textbf{મેજોરિટી કેરિયર્સ} & હોલ્સ & ઇલેક્ટ્રોન્સ \\ \hline
\textbf{કરંટ પ્રવાહ} & એમિટરથી કલેક્ટર & કલેક્ટરથી એમિટર \\ \hline
\textbf{બાયસિંગ} & એમિટર +ve, બેઝ -ve & કલેક્ટર +ve, બેઝ +ve \\ \hline
\textbf{સ્વિચિંગ સ્પીડ} & ધીમી & ઝડપી \\ \hline
\end{tabulary}
\end{solutionbox}

\begin{mnemonicbox}
\mnemonic{Negative-Positive-Negative vs Positive-Negative-Positive}
\end{mnemonicbox}

\questionmarks{5(c)}{7}{પ્રતીક દોરો, MOSFET નું બાંધકામ અને કાર્ય સમજાવો.}

\begin{solutionbox}
\textbf{સિમ્બોલ:}
\begin{center}
\begin{circuitikz}
    \draw (0,0) node[nmos] (Q) {};
    \node[right] at (Q.D) {D}; \node[right] at (Q.S) {S}; \node[left] at (Q.G) {G};
\end{circuitikz}
\captionof{figure}{MOSFET સિમ્બોલ}
\end{center}

\textbf{બાંધકામ:}
\begin{center}
\begin{tikzpicture}[scale=0.8]
    \draw (0,0) rectangle (5,3);
    \node at (2.5,0.5) {P-Substrate};
    \draw[fill=white] (0.5,2) rectangle (1.5,3); \node at (1,2.5) {S};
    \draw[fill=white] (3.5,2) rectangle (4.5,3); \node at (4,2.5) {D};
    \draw[fill=gray!30] (1.5,3) rectangle (3.5,3.2); \node[right] at (3.5,3.1) {SiO2};
    \draw[fill=black] (1.5,3.2) rectangle (3.5,3.4); \node[right] at (3.5,3.3) {Gate};
\end{tikzpicture}
\captionof{figure}{MOSFET બાંધકામ}
\end{center}

\textbf{કાર્ય (એન્હાન્સમેન્ટ મોડ):}
\begin{itemize}
    \item ગેટ વોલ્ટેજ વિના કોઈ ચેનલ હોતી નથી.
    \item પોઝિટિવ ગેટ વોલ્ટેજ ઇલેક્ટ્રોન્સ આકર્ષે છે.
    \item ચેનલ બને છે અને કરંટ વહે છે.
\end{itemize}
\end{solutionbox}

\begin{mnemonicbox}
\mnemonic{Gate Voltage Creates Electron Channel}
\end{mnemonicbox}

\questionmarks{5(a) OR}{3}{ઈલેક્ટ્રોનિક કચરાને હેન્ડલ કરવાની પદ્ધતિઓ સમજાવો.}

\begin{solutionbox}
\textbf{પદ્ધતિઓ:}
\begin{tabulary}{\linewidth}{|L|L|}
\hline
\textbf{પદ્ધતિ} & \textbf{વર્ણન} \\ \hline
\textbf{ઘટાડો (Reduce)} & લાંબા આયુષ્યવાળા ઉપકરણો. \\ \hline
\textbf{પુન:ઉપયોગ (Reuse)} & દાન/વેચાણ. \\ \hline
\textbf{રિસાયકલ (Recycle)} & સામગ્રી પુનઃપ્રાપ્તિ. \\ \hline
\textbf{નિયમન (Regulation)} & નીતિઓ. \\ \hline
\textbf{રિકવરી (Recovery)} & ધાતુઓનું નિષ્કર્ષણ. \\ \hline
\end{tabulary}
\end{solutionbox}

\begin{mnemonicbox}
\mnemonic{Reduce, Reuse, Recycle, Regulate, Recover}
\end{mnemonicbox}

\questionmarks{5(b) OR}{4}{αdc અને βdc વચ્ચેનો સંબંધ મેળવો.}

\begin{solutionbox}
\textbf{સંબંધો:} $I_E = I_C + I_B$, $\alpha_{dc} = I_C/I_E$, $\beta_{dc} = I_C/I_B$.

\textbf{તારવણી:}
\begin{itemize}
    \item $I_E = I_C + I_B$ ને $I_C$ વડે ભાગો:
    $$ \frac{I_E}{I_C} = 1 + \frac{I_B}{I_C} $$
    $$ \frac{1}{\alpha_{dc}} = 1 + \frac{1}{\beta_{dc}} $$
    $$ \alpha_{dc} = \frac{\beta_{dc}}{1+\beta_{dc}} $$
    \item $\beta_{dc}$ માટે: $\beta_{dc} = \frac{\alpha_{dc}}{1-\alpha_{dc}}$
\end{itemize}
\end{solutionbox}

\begin{mnemonicbox}
\mnemonic{Alpha-Beta Relate as Alpha = Beta/(1+Beta)}
\end{mnemonicbox}

\questionmarks{5(c) OR}{7}{તેના ઇનપુટ અને આઉટપુટ લાક્ષણિકતાઓ સાથે CC ની રચના સમજાવો.}

\begin{solutionbox}
\textbf{કોમન કલેક્ટર (એમિટર ફોલોઅર):}
\begin{center}
\begin{circuitikz}
    \draw (0,0) node[npn] (Q) {};
    \draw (Q.C) -- ++(0,1) node[above] {$V_{CC}$};
    \draw (Q.B) to[R, l=$R_B$] (-2,0) to[sV, l=$V_{in}$] (-2,-2) -- (0,-2) -- (Q.E);
    \draw (Q.E) to[R, l=$R_E$] (0,-2) node[ground] {};
    \draw (Q.E) -- ++(2,0) to[short, -o] ++(0,0) node[right] {$V_{out}$};
\end{circuitikz}
\captionof{figure}{CC સર્કિટ}
\end{center}

\textbf{લાક્ષણિકતાઓ:}
\begin{itemize}
    \item \textbf{ઇનપુટ}: $I_B$ વિ $V_{BC}$. ઉચ્ચ ઇમ્પેડન્સ.
    \item \textbf{આઉટપુટ}: $I_E$ વિ $V_{CE}$. નીચું ઇમ્પેડન્સ.
\end{itemize}

\textbf{વિશેષતાઓ:} વોલ્ટેજ ગેઇન $\approx 1$. બફર તરીકે ઉપયોગી.
\end{solutionbox}

\begin{mnemonicbox}
\mnemonic{Emitter Follows Base Voltage}
\end{mnemonicbox}

\end{document}
