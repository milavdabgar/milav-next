\documentclass{article}

% content/resources/templates/preamble.tex
\usepackage[margin=0.6in]{geometry}
\author{Milav Dabgar}
\usepackage{amsmath,amssymb,amsthm}
\usepackage{booktabs}
\usepackage{multirow}
\usepackage{xcolor}
\usepackage{tcolorbox}
\tcbuselibrary{breakable,skins}
\usepackage[colorlinks=true,linkcolor=blue]{hyperref}
\usepackage{titlesec}
\usepackage{enumitem}
\usepackage{tikz}
\usepackage{pgfplots}
\usepackage{circuitikz}
\usepackage[version=4]{mhchem}
\usepackage{longtable}
\usepackage{array}
\usepackage{float}
\usepackage{caption}
\usepackage{listings}

\lstset{
  basicstyle=\small\ttfamily,
  breaklines=true,
  breakatwhitespace=false,
  postbreak=\mbox{\textcolor{red}{$\hookrightarrow$}\space},
  float=false,
  numbers=left,
  numberstyle=\tiny\color{gray},
  numbersep=10pt,
  xleftmargin=2em,
  keywordstyle=\color{blue},
  commentstyle=\color{green!60!black},
  stringstyle=\color{purple},
  backgroundcolor=\color{gray!5},
  showstringspaces=false,
  tabsize=2,
  captionpos=b,
  keepspaces=true,
  columns=flexible
}

\pgfplotsset{compat=1.18}
\usetikzlibrary{shapes,arrows,positioning,calc,patterns,decorations.pathmorphing,decorations.markings,arrows.meta}

% Color scheme
\definecolor{headcolor}{RGB}{0,102,204}
\definecolor{keycolor}{RGB}{220,20,60}
\definecolor{solutioncolor}{RGB}{34,139,34}
\definecolor{mnemoniccolor}{RGB}{148,0,211}
\definecolor{codecolor}{RGB}{0,0,100}

% Spacing
\setlength{\parskip}{3pt}
\setlist[itemize]{nosep}
\setlist[enumerate]{nosep}

% Title formatting
\titleformat{\section}{\Large\bfseries\color{headcolor}}{\thesection}{1em}{}
\titleformat{\subsection}{\large\bfseries\color{headcolor}}{\thesubsection}{1em}{}

% Pandoc tightlist compatibility
\providecommand{\tightlist}{%
  \setlength{\itemsep}{0pt}\setlength{\parskip}{0pt}}

% Pandoc longtable compatibility
\newcounter{none}
\def\thenone{}


% content/resources/templates/gujarati-boxes.tex
\usepackage{fontspec}
\usepackage{polyglossia}

% Set Gujarati as main language (document is primarily in Gujarati)
% Note: gloss-gujarati.ldf doesn't exist in polyglossia, but it will use hyphenation patterns
\setdefaultlanguage{gujarati}
\setotherlanguage{english}

% Configure Gujarati font properly
% Use Language=Default to prevent polyglossia from trying to add language-specific features
% that don't exist for Gujarati, which causes "empty feature" warnings
\newfontfamily\gujaratifont[Script=Gujarati,AutoFakeBold=2.5,AutoFakeSlant=0.3]{Noto Sans Gujarati}
\setmainfont[Script=Gujarati,AutoFakeBold=2.5,AutoFakeSlant=0.3]{Noto Sans Gujarati}
% Use Noto Sans Gujarati for monospace to support Gujarati in text
\setmonofont[Scale=0.9]{Noto Sans Gujarati}

% Configure English to use the same font
\newfontfamily\englishfont[Script=Gujarati,AutoFakeBold=2.5,AutoFakeSlant=0.3]{Noto Sans Gujarati}

% Translations for polyglossia
\gappto\captionsgujarati{
  \renewcommand{\tablename}{કોષ્ટક}
  \renewcommand{\figurename}{આકૃતિ}
}

% Helper for TikZ nodes to ensure Gujarati font
\newcommand{\gu}[1]{{\gujaratifont #1}}

% Custom environments
\newtcolorbox{solutionbox}{
    breakable,
    enhanced,
    colback=solutioncolor!5!white,
    colframe=solutioncolor!75!black,
    fonttitle=\bfseries,
    title=જવાબ
}

\newtcolorbox{solutionboxnobreak}{
 colback=solutioncolor!5!white,
 colframe=solutioncolor!75!black,
 fonttitle=\bfseries,
 title=જવાબ
}

\newtcolorbox{keyformula}{
 breakable,
 enhanced,
 colback=keycolor!5!white,
 colframe=keycolor!75!black,
 fonttitle=\bfseries,
 title=રાસાયણિક સમીકરણ/સૂત્ર
}

\newtcolorbox{mnemonicbox}{
 breakable,
 enhanced,
 colback=mnemoniccolor!5!white,
 colframe=mnemoniccolor!75!black,
 fonttitle=\bfseries,
 title=મેમરી ટ્રીક
}


% Custom commands for GTU solutions
% This file defines semantic commands for consistent formatting

% Question command with automatic formatting
\newcommand{\question}[2]{%
  \section*{Question #1}%
  \textbf{#2}%
}

% OR question variant
\newcommand{\questionor}[2]{%
  \section*{Question #1 OR}%
  \textbf{#2}%
}

% Proper table environment with caption
\newenvironment{answertable}[1]{%
  \begin{table}[htbp]
  \centering
  \caption{#1}
}{%
  \end{table}
}

% Proper figure environment for diagrams
\newenvironment{answerdiagram}[1]{%
  \begin{figure}[htbp]
  \centering
  \caption{#1}
}{%
  \end{figure}
}

% Semantic markup for key terms
\newcommand{\keyword}[1]{\textbf{#1}}
\newcommand{\code}[1]{\texttt{#1}}
\newcommand{\classname}[1]{\texttt{#1}}
\newcommand{\methodname}[1]{\texttt{#1}}

% Proper quotation marks
\newcommand{\mnemonic}[1]{``#1''}


\title{ઇલેક્ટ્રિકલ અને ઇલેક્ટ્રોનિક્સ એન્જિનિયરિંગના તત્વો (1313202) - ઉનાળુ 2024 સોલ્યુશન}
\date{June 13, 2024}

\begin{document}
\maketitle

\questionmarks{1(a)}{3}{વ્યાખ્યા આપો: 1. નોડ, 2. લૂપ, 3. બ્રાંચ}

\begin{solutionbox}
\begin{enumerate}
    \item \textbf{નોડ (Node):} સર્કિટમાં એવો બિંદુ જ્યાં બે અથવા વધુ સર્કિટ એલિમેન્ટ મળે છે અથવા જોડાય છે.
    \item \textbf{લૂપ (Loop):} સર્કિટમાં એક બંધ માર્ગ જે એક જ બિંદુથી શરૂ થઈને એ જ બિંદુ પર પરત આવે છે, કોઈપણ નોડને એક વખતથી વધુ ઓળંગીને નહીં.
    \item \textbf{બ્રાંચ (Branch):} સર્કિટમાં બે નોડને જોડતો માર્ગ અથવા એલિમેન્ટ.
\end{enumerate}
\end{solutionbox}

\begin{mnemonicbox}
\mnemonic{Never Loop Between: Nodes Link, Loops Bound, Branches Establish connections}
\end{mnemonicbox}

\questionmarks{1(b)}{4}{Superposition થીયરમ અને Maximum power transfer થીયરમ નું સ્ટેટમેંટ લખો.}

\begin{solutionbox}
\begin{itemize}
    \item \textbf{Superposition થીયરમ:} લીનિયર સર્કિટમાં મલ્ટીપલ સોર્સ હોય ત્યારે, કોઈપણ એલિમેન્ટમાં રિસ્પોન્સ (વોલ્ટેજ અથવા કરંટ) એ દરેક સોર્સના એકલા કાર્ય કરવાથી થતા રિસ્પોન્સના બીજગણિતીય સરવાળાની બરાબર હોય છે, જ્યારે બીજા બધા સોર્સને તેમના આંતરિક ઇમ્પિડન્સથી બદલી દેવામાં આવે.
    \item \textbf{Maximum Power Transfer થીયરમ:} સોર્સથી લોડમાં મહત્તમ પાવર ત્યારે ટ્રાન્સફર થાય છે જ્યારે લોડ રેઝિસ્ટન્સ સોર્સના આંતરિક રેઝિસ્ટન્સની બરાબર હોય ($R_L = R_S$).
\end{itemize}

\begin{center}
\begin{tikzpicture}[node distance=2cm, auto]
    % Superposition
    \node [gtu block] (src) {સોર્સ};
    \node [gtu block, right=of src] (ind) {વ્યક્તિગત\\પ્રતિક્રિયાઓ};
    \node [gtu block, right=of ind] (sum) {સરવાળો = કુલ\\પ્રતિક્રિયા};
    
    \draw [gtu arrow] (src) -- (ind);
    \draw [gtu arrow] (ind) -- (sum);
    
    % Max Power
    \node [gtu block, below=1.5cm of src] (source) {સોર્સ $R_S$};
    \node [gtu block, right=of source] (load) {લોડ $R_L$};
    \node [gtu block, right=of load] (cond) {મેક્સ પાવર\\જ્યારે $R_S = R_L$};
    
    \draw [gtu dashed arrow] (source) -- (load);
    \draw [gtu arrow] (load) -- (cond);
\end{tikzpicture}
\captionof{figure}{Superposition અને Max Power Transfer ના કોન્સેપ્ટ}
\end{center}
\end{solutionbox}

\begin{mnemonicbox}
\mnemonic{Sum Powers Matched: Sum individual powers; Match resistance for maximum}
\end{mnemonicbox}

\questionmarks{1(c)}{7}{કિરચોફનો વોલ્ટેજ નો નિયમ અને કિરચોફનો કરંટનો નિયમ સમજાવો.}

\begin{solutionbox}
\begin{tabulary}{\linewidth}{|L|L|L|}
\hline
\textbf{નિયમ} & \textbf{સમજૂતી} & \textbf{ગાણિતિક સ્વરૂપ} \\ \hline
\textbf{કિરચોફનો વોલ્ટેજ નો નિયમ (KVL)} & સર્કિટમાં કોઈપણ બંધ લૂપમાં બધા વોલ્ટેજનો બીજગણિતીય સરવાળો શૂન્ય થાય છે. & $\sum V = 0$ \\ \hline
\textbf{કિરચોફનો કરંટનો નિયમ (KCL)} & નોડમાં પ્રવેશતા અને નીકળતા બધા કરંટનો બીજગણિતીય સરવાળો શૂન્ય થાય છે. & $\sum I = 0$ \\ \hline
\end{tabulary}

\begin{center}
\begin{tabular}{cc}
\begin{circuitikz}[scale=0.8]
    \draw (0,0) to[battery1, l=$V_1$] (0,3) -- (3,3) to[R, l=$V_2$] (3,0) -- (0,0);
    \draw (1.5,1.5) node[circle, draw] {Loop};
    \node at (1.5,-0.5) {KVL: $V_1 - V_2 = 0$};
\end{circuitikz}
&
\begin{circuitikz}[scale=0.8]
    \draw (0,0) node[circle,fill,inner sep=2pt,label=above:Node] (N) {};
    \draw[<-] (N) -- (-1.5, 1) node[left] {$I_1$};
    \draw[<-] (N) -- (-1.5, -1) node[left] {$I_2$};
    \draw[->] (N) -- (1.5, 1) node[right] {$I_3$};
    \draw[->] (N) -- (1.5, -1) node[right] {$I_4$};
    \node at (0,-1.5) {KCL: $I_1 + I_2 = I_3 + I_4$};
\end{circuitikz}
\end{tabular}
\captionof{figure}{KVL અને KCL ડાયાગ્રામ}
\end{center}

\begin{itemize}
    \item \textbf{KVL નું ભૌતિક અર્થઘટન:} સર્કિટ લૂપમાં ઊર્જા સંરક્ષિત રહે છે.
    \item \textbf{KCL નું ભૌતિક અર્થઘટન:} સર્કિટ નોડમાં ચાર્જ સંરક્ષિત રહે છે.
    \item \textbf{KVL નો ઉપયોગ:} સર્કિટ લૂપમાં અજ્ઞાત વોલ્ટેજ શોધવા.
    \item \textbf{KCL નો ઉપયોગ:} સર્કિટ જંક્શનમાં અજ્ઞાત કરંટ શોધવા.
\end{itemize}
\end{solutionbox}

\begin{mnemonicbox}
\mnemonic{Voltages Loop to Zero, Currents Node to Zero}
\end{mnemonicbox}

\questionmarks{1(c) OR}{7}{રેસિસ્ટન્સ ના સીરીઝ અને પેરેલલ કનેક્શન જરુરી સમીકરણો સાથે સમજાવો.}

\begin{solutionbox}
\begin{tabulary}{\linewidth}{|L|L|L|L|}
\hline
\textbf{કનેક્શન} & \textbf{લાક્ષણિકતાઓ} & \textbf{સમતુલ્ય રેસિસ્ટન્સ} & \textbf{કરંટ-વોલ્ટેજ સંબંધ} \\ \hline
\textbf{સીરીઝ} & બધા રેસિસ્ટર્સમાંથી એક સરખો કરંટ વહે છે. & $R_{eq} = R_1 + R_2 + \dots + R_n$ & $I = V/R_{eq}$ \\ \hline
\textbf{પેરેલલ} & બધા રેસિસ્ટર્સ પર એક સરખો વોલ્ટેજ આવે છે. & $1/R_{eq} = 1/R_1 + 1/R_2 + \dots + 1/R_n$ & $I = I_1 + I_2 + \dots + I_n$ \\ \hline
\end{tabulary}

\begin{center}
\begin{tabular}{cc}
\begin{circuitikz}[scale=0.7]
    \draw (0,0) to[battery1, l=$V$] (0,2) -- (1,2)
    to[R, l=$R_1$] (2,2)
    to[R, l=$R_2$] (3,2)
    to[R, l=$R_3$] (4,2) -- (4,0) -- (0,0);
    \node at (2,-0.5) {સીરીઝ કનેક્શન};
\end{circuitikz}
&
\begin{circuitikz}[scale=0.7]
    \draw (0,0) to[battery1, l=$V$] (0,2) -- (3,2);
    \draw (1,2) to[R, l=$R_1$] (1,0);
    \draw (2,2) to[R, l=$R_2$] (2,0);
    \draw (3,2) to[R, l=$R_3$] (3,0);
    \draw (0,0) -- (3,0);
    \node at (1.5,-0.5) {પેરેલલ કનેક્શન};
\end{circuitikz}
\end{tabular}
\captionof{figure}{સીરીઝ અને પેરેલલ કનેક્શન}
\end{center}

\begin{itemize}
    \item \textbf{સીરીઝમાં કરંટ:} $I = I_1 = I_2 = \dots = I_n$
    \item \textbf{સીરીઝમાં વોલ્ટેજ:} $V = V_1 + V_2 + \dots + V_n$
    \item \textbf{પેરેલલમાં કરંટ:} $I = I_1 + I_2 + \dots + I_n$
    \item \textbf{પેરેલલમાં વોલ્ટેજ:} $V = V_1 = V_2 = \dots = V_n$
\end{itemize}
\end{solutionbox}

\begin{mnemonicbox}
\mnemonic{Same Current Series, Same Voltage Parallel}
\end{mnemonicbox}

\questionmarks{2(a)}{3}{Ohm's law ની મર્યાદાઓ જણાવો.}

\begin{solutionbox}
\begin{itemize}
    \item \textbf{નોન-લિનિયર કંપોનન્ટ્સ:} ડાયોડ, ટ્રાન્ઝિસ્ટર જેવા કંપોનન્ટ્સને લાગુ પડતો નથી, જે નોન-લિનિયર V-I લાક્ષણિકતાઓ ધરાવે છે.
    \item \textbf{તાપમાન ફેરફાર:} જ્યારે તાપમાન નોંધપાત્ર રીતે બદલાય છે ત્યારે માન્ય રહેતો નથી, કારણ કે તાપમાન સાથે રેઝિસ્ટન્સ બદલાય છે.
    \item \textbf{ઉચ્ચ ફ્રિક્વન્સી:} ખૂબ ઊંચી ફ્રિક્વન્સી પર સ્કિન ઇફેક્ટ અને અન્ય અસરોને કારણે નિષ્ફળ જાય છે.
\end{itemize}
\end{solutionbox}

\begin{mnemonicbox}
\mnemonic{Ohm's Not Linear Thermal High: Non-linear, Temperature, High frequency}
\end{mnemonicbox}

\questionmarks{2(b)}{4}{વ્યાખ્યા આપો: 1. ડોપીંગ, 2. ઈંટ્રાસીક સેમીકંડક્ટર, 3. એક્સ્ટ્રાસીક સેમીકંડક્ટર, 4. ડોપંટ}

\begin{solutionbox}
\begin{enumerate}
    \item \textbf{ડોપીંગ (Doping):} શુદ્ધ સેમીકંડક્ટરમાં અશુદ્ધિના પરમાણુઓ ઉમેરવાની પ્રક્રિયા જેનાથી ઇલેક્ટ્રિકલ ગુણધર્મો બદલાય છે.
    \item \textbf{ઈંટ્રાસીક સેમીકંડક્ટર (Intrinsic Semiconductor):} શુદ્ધ સેમીકંડક્ટર (જેમ કે શુદ્ધ Si અથવા Ge) જેમાં ઇલેક્ટ્રોન અને હોલની સંખ્યા સરખી હોય છે.
    \item \textbf{એક્સ્ટ્રાસીક સેમીકંડક્ટર (Extrinsic Semiconductor):} ડોપ કરેલા સેમીકંડક્ટર જેમાં ઇલેક્ટ્રોન અને હોલની સંખ્યા અસરખી હોય છે (N-type અથવા P-type).
    \item \textbf{ડોપંટ (Dopant):} ડોપિંગ પ્રક્રિયા દરમિયાન સેમીકંડક્ટરમાં ઉમેરાતા અશુદ્ધિના તત્વો (જેમ કે Boron, Phosphorus).
\end{enumerate}
\end{solutionbox}

\begin{mnemonicbox}
\mnemonic{Do In-Ex-Do: Doping Introduces Extrinsic properties through Dopants}
\end{mnemonicbox}

\questionmarks{2(c)}{7}{ટ્રાયવેલેંટ મટીરીયલ ની વ્યાખ્યા આપો અને તેના ઉદાહરણ આપો. P-type સેમીકંડક્ટરની રચના જરુરી આકૃતિ સાથે સમજાવો.}

\begin{solutionbox}
\begin{itemize}
    \item \textbf{ટ્રાયવેલેંટ મટીરીયલ:} એવા તત્વો જેમના બાહ્યતમ કોશમાં 3 વેલેન્સ ઇલેક્ટ્રોન હોય છે.
    \item \textbf{ઉદાહરણો:} Boron (B), Aluminum (Al), Gallium (Ga), Indium (In).
\end{itemize}

\textbf{P-type સેમીકંડક્ટરની રચના:}
જ્યારે શુદ્ધ સિલિકોન ક્રિસ્ટલને ટ્રાયવેલેંટ ઇમ્પ્યોરિટી (જેમ કે Boron) સાથે ડોપ કરવામાં આવે છે:
\begin{enumerate}
    \item \textbf{બોન્ડ ફોર્મેશન:} Boron ના 3 વેલેન્સ ઇલેક્ટ્રોન 3 પડોશી Silicon એટમ સાથે કોવેલેન્ટ બોન્ડ બનાવે છે.
    \item \textbf{હોલ ક્રિએશન:} ચોથો બોન્ડ અપૂર્ણ રહે છે કારણ કે Boron પાસે ચોથો ઇલેક્ટ્રોન નથી. આ ખૂટતો ઇલેક્ટ્રોન એક ખાલી જગ્યા બનાવે છે જેને \textbf{હોલ (Hole)} કહેવાય છે.
    \item \textbf{ચાર્જ કેરિયર્સ:} હોલ પોઝિટિવ ચાર્જ કેરિયર્સ છે. હોલ ડોપિંગ દ્વારા ઉત્પન્ન થતા હોવાથી તે \textbf{મેજોરિટી કેરિયર્સ} બને છે, જ્યારે ઇલેક્ટ્રોન \textbf{માઇનોરિટી કેરિયર્સ} હોય છે.
\end{enumerate}

\begin{center}
\begin{tikzpicture}[scale=0.8]
    \foreach \x in {0,2,4}
    \foreach \y in {0,2,4}
        \node [draw, circle, minimum size=0.8cm] at (\x,\y) {Si};
    
    \node [draw, circle, minimum size=0.8cm, fill=yellow!20] at (2,2) {B};
    
    % Bonds
    \draw (0.4, 2.1) -- (1.6, 2.1); \draw (0.4, 1.9) -- (1.6, 1.9); % Left
    \draw (2.4, 2.1) -- (3.6, 2.1); \draw (2.4, 1.9) -- (3.6, 1.9); % Right
    \draw (2.1, 2.4) -- (2.1, 3.6); \draw (1.9, 2.4) -- (1.9, 3.6); % Top
    \draw (2.1, 1.6) -- (2.1, 0.4); \draw (1.9, 1.6) -- (1.9, 0.4); % Bottom
    
    % Hole
    \draw [red, dashed] (2, 2.8) circle (0.15);
    \node [red, right] at (2.2, 2.8) {હોલ (Hole)};
\end{tikzpicture}
\captionof{figure}{સિલિકોન લેટિસમાં બોરોન એટમ હોલ બનાવે છે}
\end{center}
\end{solutionbox}

\begin{mnemonicbox}
\mnemonic{Three Makes Positive: Three valence electrons make a Positive hole}
\end{mnemonicbox}

\questionmarks{2(a) OR}{3}{રેસિસ્ટન્સને અસર કરતા પરિબળો જણાવો અને તેમાથી કોઈપણ એક સમજાવો.}

\begin{solutionbox}
\textbf{રેસિસ્ટન્સને અસર કરતા પરિબળો:}
\begin{enumerate}
    \item કન્ડક્ટરની લંબાઈ ($L$)
    \item ક્રોસ-સેક્શનલ એરિયા ($A$)
    \item મટીરિયલ (રેસિસ્ટિવિટી, $\rho$)
    \item તાપમાન ($T$)
\end{enumerate}

\textbf{તાપમાનની અસરની સમજૂતી:}
મોટાભાગના મેટાલિક કન્ડક્ટરનો રેસિસ્ટન્સ તાપમાન સાથે વધે છે.
\[ R = R_0[1 + \alpha(T - T_0)] \]
જ્યાં:
\begin{itemize}
    \item $R$ = તાપમાન $T$ પર રેસિસ્ટન્સ
    \item $R_0$ = રેફરન્સ તાપમાન $T_0$ પર રેસિસ્ટન્સ
    \item $\alpha$ = રેસિસ્ટન્સનો તાપમાન કોએફિશિયન્ટ
\end{itemize}
તાપમાન વધતા પરમાણુઓ વધુ કંપન કરે છે, જે ઇલેક્ટ્રોન પ્રવાહને અવરોધે છે, જેથી રેસિસ્ટન્સ વધે છે.
\end{solutionbox}

\begin{mnemonicbox}
\mnemonic{LAMT: Length, Area, Material, Temperature}
\end{mnemonicbox}

\questionmarks{2(b) OR}{4}{વ્યાખ્યા આપો: 1. વેલેન્સ બેન્ડ, 2. કંડકશન બેન્ડ, 3. ફોરબિડન એનર્જી ગેપ, 4. ફ્રી ઇલેક્ટ્રોન}

\begin{solutionbox}
\begin{itemize}
    \item \textbf{વેલેન્સ બેન્ડ:} એનર્જી બેન્ડ જેમાં એટમ સાથે બંધાયેલા વેલેન્સ ઇલેક્ટ્રોન ભરેલા હોય છે.
    \item \textbf{કંડકશન બેન્ડ:} ઉચ્ચ એનર્જી બેન્ડ જ્યાં ઇલેક્ટ્રોન મુક્તપણે ફરી શકે છે અને વીજળી વહન કરી શકે છે.
    \item \textbf{ફોરબિડન એનર્જી ગેપ:} વેલેન્સ અને કંડકશન બેન્ડ વચ્ચેની એનર્જી રેન્જ જ્યાં કોઈ ઇલેક્ટ્રોન સ્ટેટ્સ અસ્તિત્વમાં હોતા નથી.
    \item \textbf{ફ્રી ઇલેક્ટ્રોન:} ઇલેક્ટ્રોન જે એટલી ઊર્જા મેળવે છે કે તે વેલેન્સ બેન્ડમાંથી કંડકશન બેન્ડમાં કૂદી શકે છે, પાછળ હોલ છોડીને.
\end{itemize}

\begin{center}
\begin{tikzpicture}[scale=0.8]
    \draw[thick] (0, 3) rectangle (4, 4);
    \node at (2, 3.5) {કંડકશન બેન્ડ (ફ્રી $e^-$)};
    
    \draw[thick] (0, 0) rectangle (4, 1);
    \node at (2, 0.5) {વેલેન્સ બેન્ડ (બાઉન્ડ $e^-$)};
    
    \draw[<->] (4.2, 1) -- (4.2, 3);
    \node[right, align=left] at (4.2, 2) {ફોરબિડન\\ગેપ ($E_g$)};
\end{tikzpicture}
\captionof{figure}{એનર્જી બેન્ડ ડાયાગ્રામ}
\end{center}
\end{solutionbox}

\begin{mnemonicbox}
\mnemonic{Very Clearly Freedom Follows: Valence, Conduction, Forbidden gap, Free electrons}
\end{mnemonicbox}

\questionmarks{2(c) OR}{7}{પેન્ટાવેલેંટ મટીરીયલ ની વ્યાખ્યા આપો અને તેના ઉદાહરણ આપો. N-type સેમીકંડક્ટરની રચના જરુરી આકૃતિ સાથે સમજાવો.}

\begin{solutionbox}
\begin{itemize}
    \item \textbf{પેન્ટાવેલેંટ મટીરીયલ:} એવા તત્વો જેમના બાહ્યતમ કોશમાં 5 વેલેન્સ ઇલેક્ટ્રોન હોય છે.
    \item \textbf{ઉદાહરણો:} Phosphorus (P), Arsenic (As), Antimony (Sb).
\end{itemize}

\textbf{N-type સેમીકંડક્ટરની રચના:}
જ્યારે શુદ્ધ સિલિકોન ક્રિસ્ટલને પેન્ટાવેલેંટ ઇમ્પ્યોરિટી (જેમ કે Phosphorus) સાથે ડોપ કરવામાં આવે છે:
\begin{enumerate}
    \item \textbf{બોન્ડ ફોર્મેશન:} Phosphorus ના 4 વેલેન્સ ઇલેક્ટ્રોન 4 પડોશી Silicon એટમ સાથે કોવેલેન્ટ બોન્ડ બનાવે છે.
    \item \textbf{ફ્રી ઇલેક્ટ્રોન:} પાંચમો વેલેન્સ ઇલેક્ટ્રોન ઢીલી રીતે બંધાયેલ રહે છે અને સરળતાથી મુક્ત થઈ શકે છે.
    \item \textbf{ચાર્જ કેરિયર્સ:} ઇલેક્ટ્રોન નેગેટિવ ચાર્જ કેરિયર્સ છે. ઇલેક્ટ્રોન \textbf{મેજોરિટી કેરિયર્સ} બને છે, જ્યારે હોલ \textbf{માઇનોરિટી કેરિયર્સ} હોય છે.
\end{enumerate}

\begin{center}
\begin{tikzpicture}[scale=0.8]
    \foreach \x in {0,2,4}
    \foreach \y in {0,2,4}
        \node [draw, circle, minimum size=0.8cm] at (\x,\y) {Si};
    
    \node [draw, circle, minimum size=0.8cm, fill=yellow!20] at (2,2) {P};
    
    % Bonds
    \draw (0.4, 2.1) -- (1.6, 2.1); \draw (0.4, 1.9) -- (1.6, 1.9); % Left
    \draw (2.4, 2.1) -- (3.6, 2.1); \draw (2.4, 1.9) -- (3.6, 1.9); % Right
    \draw (2.1, 2.4) -- (2.1, 3.6); \draw (1.9, 2.4) -- (1.9, 3.6); % Top
    \draw (2.1, 1.6) -- (2.1, 0.4); \draw (1.9, 1.6) -- (1.9, 0.4); % Bottom
    
    % Free Electron
    \draw [fill=black] (2.8, 2.8) circle (0.05);
    \node [right] at (2.9, 2.8) {ફ્રી ઇલેક્ટ્રોન};
\end{tikzpicture}
\captionof{figure}{સિલિકોન લેટિસમાં ફોસ્ફરસ એટમ ફ્રી ઇલેક્ટ્રોન બનાવે છે}
\end{center}
\end{solutionbox}

\begin{mnemonicbox}
\mnemonic{Five Makes Negative: Five valence electrons make a Negative carrier}
\end{mnemonicbox}

\questionmarks{3(a)}{3}{ડાયોડની સાપેક્ષમાં 1. ડીપ્લીશન રીજીયન, 2. ની વોલ્ટેજ, અને 3. બ્રેકડાઉન વોલ્ટેજની વ્યાખ્યા આપો}

\begin{solutionbox}
\begin{enumerate}
    \item \textbf{ડીપ્લીશન રીજીયન (Depletion Region):} P-N જંક્શન પાસેનો વિસ્તાર જે ઇલેક્ટ્રોન અને હોલના ડિફ્યુઝન અને રિકોમ્બિનેશનને કારણે મોબાઇલ ચાર્જ કેરિયર્સથી ખાલી હોય છે. તેમાં માત્ર સ્થિર આયનો હોય છે.
    \item \textbf{ની વોલ્ટેજ (Knee Voltage, $V_k$):} લઘુત્તમ ફોરવર્ડ વોલ્ટેજ જેના પછી કરંટ ઝડપથી વધવાનું શરૂ થાય છે. (Si માટે ~0.7V, Ge માટે ~0.3V). તેને કટ-ઈન વોલ્ટેજ પણ કહે છે.
    \item \textbf{બ્રેકડાઉન વોલ્ટેજ (Breakdown Voltage, $V_{BR}$):} રિવર્સ વોલ્ટેજ જેના પર P-N જંક્શન બ્રેક થાય છે અને મોટો રિવર્સ કરંટ વહેવા દે છે, જે ડાયોડને નુકસાન પહોંચાડી શકે છે.
\end{enumerate}
\end{solutionbox}

\begin{mnemonicbox}
\mnemonic{Depleted Knees Break: Depletion occurs, Knee begins conduction, Breakdown ends blocking}
\end{mnemonicbox}

\questionmarks{3(b)}{4}{P-N જંક્શન ડાયોડ ની V-I લાક્ષણિકતા જરુરી ગ્રાફ સાથે સમજાવો.}

\begin{solutionbox}
V-I લાક્ષણિકતા ડાયોડ પરના વોલ્ટેજ અને તેમાંથી વહેતા કરંટ વચ્ચેનો સંબંધ દર્શાવે છે.

\begin{enumerate}
    \item \textbf{ફોરવર્ડ બાયસ ($V > 0$):} શરૂઆતમાં, કરંટ ઓછો હોય છે. એકવાર વોલ્ટેજ ની વોલ્ટેજ ($V_k \approx 0.7V$) કરતા વધી જાય, કરંટ એક્સપોનેન્શિયલી વધે છે.
    \item \textbf{રિવર્સ બાયસ ($V < 0$):} માત્ર ખૂબ જ નાનો લીકેજ કરંટ (સેચુરેશન કરંટ $I_s$) વહે છે.
    \item \textbf{બ્રેકડાઉન:} જો રિવર્સ વોલ્ટેજ બ્રેકડાઉન વોલ્ટેજ ($V_{BR}$) કરતા વધી જાય, તો કરંટમાં તીવ્ર વધારો થાય છે.
\end{enumerate}

\begin{center}
\begin{tikzpicture}[scale=0.8]
    \draw[->] (-3,0) -- (3,0) node[right] {$V$};
    \draw[->] (0,-3) -- (0,3) node[above] {$I$};
    
    % Forward
    \draw[blue, thick] (0,0) -- (0.5,0.1) plot[domain=0.5:1.2] (\x, {exp(2*(\x-0.7))});
    \node [right] at (1.5, 2) {ફોરવર્ડ બાયસ};
    \draw[dashed] (0.7,0) -- (0.7,1);
    \node[below] at (0.7,0) {$V_k$};
    
    % Reverse
    \draw[red, thick] (0,0) -- (-2, -0.1) -- (-2, -2.5);
    \node [left] at (-1.5, -1) {રિવર્સ બાયસ};
    \node[above] at (-2,0) {$V_{BR}$};
    \node at (-1, -3) {બ્રેકડાઉન};
\end{tikzpicture}
\captionof{figure}{PN જંક્શન ડાયોડની V-I લાક્ષણિકતાઓ}
\end{center}
\end{solutionbox}

\begin{mnemonicbox}
\mnemonic{Forward Flows, Reverse Restricts, Breakdown Bursts}
\end{mnemonicbox}

\questionmarks{3(c)}{7}{Varactor ડાયોડ ની લાક્ષણિકતા દોરો. Varactor ડાયોડની કાર્યપધ્ધતિ આકૃતિ સાથે સમજાવો અને તેની એપ્લીકેશન લખો.}

\begin{solutionbox}
\textbf{Varactor ડાયોડની કાર્યપધ્ધતિ:}
Varactor (Variable Capacitor) ડાયોડ એ સિદ્ધાંત પર કાર્ય કરે છે કે ડીપ્લીશન રીજીયન P અને N કન્ડક્ટિવ રીજીયન (પ્લેટ્સ તરીકે વર્તે છે) વચ્ચે ડાઇ-ઇલેક્ટ્રિક તરીકે કામ કરે છે.
\begin{itemize}
    \item તે વોલ્ટેજ-ડિપેન્ડન્ટ કેપેસિટર તરીકે વર્તે છે.
    \item \textbf{રિવર્સ બાયસ}માં ઓપરેટ થાય છે.
    \item જેમ રિવર્સ વોલ્ટેજ ($V_R$) વધે છે, ડીપ્લીશન વિડ્થ ($W$) વધે છે. $C \propto 1/W$ હોવાથી, કેપેસિટન્સ \textbf{ઘટે છે}.
    \item સંબંધ: $C = \frac{K}{\sqrt{V_R}}$
\end{itemize}

\begin{center}
\begin{tabular}{cc}
\begin{circuitikz}[scale=0.8]
    \draw (0,0) to[vC, l=Varactor] (2,0);
\end{circuitikz} &
\begin{tikzpicture}[scale=0.8]
    \draw[->] (0,0) -- (4,0) node[right] {$V_R$};
    \draw[->] (0,0) -- (0,3) node[above] {$C$};
    \draw[blue, thick] plot[domain=0.2:3.5] (\x, {2.5/sqrt(\x)});
\end{tikzpicture} \\
સિમ્બોલ & લાક્ષણિકતાઓ ($C$ વિરુદ્ધ $V_R$)
\end{tabular}
\end{center}

\textbf{એપ્લિકેશન્સ:}
\begin{itemize}
    \item વોલ્ટેજ કંટ્રોલ્ડ ઓસીલેટર્સ (VCOs)
    \item ફ્રિક્વન્સી મોડ્યુલેશન (FM) ટ્રાન્સમિટર્સ
    \item રેડિયોમાં ઓટોમેટિક ફ્રિક્વન્સી કંટ્રોલ (AFC)
    \item ટીવી ટ્યુનિંગ સર્કિટ્સ
\end{itemize}
\end{solutionbox}

\begin{mnemonicbox}
\mnemonic{Capacitance Varies Reversely: Capacitance Varies with Reverse voltage}
\end{mnemonicbox}

\questionmarks{3(a) OR}{3}{નીચે દર્શાવેલ ડાયોડની એપ્લીકેશન લખો. 1. Varactor ડાયોડ, 2. Photo ડાયોડ, 3. Light Emitting ડાયોડ}

\begin{solutionbox}
\begin{tabulary}{\linewidth}{|L|L|}
\hline
\textbf{ડાયોડનો પ્રકાર} & \textbf{એપ્લિકેશન્સ} \\ \hline
\textbf{Varactor ડાયોડ} & ટ્યુનિંગ સર્કિટ્સ, VCOs, ફ્રિક્વન્સી મોડ્યુલેટર્સ. \\ \hline
\textbf{Photo ડાયોડ} & લાઇટ ડિટેક્ટર્સ, સોલર સેલ, ઓપ્ટિકલ કોમ્યુનિકેશન રિસીવર્સ, સ્મોક ડિટેક્ટર્સ. \\ \hline
\textbf{LED} & ઇન્ડીકેટર્સ, ડિજિટલ ડિસ્પ્લે (7-સેગમેન્ટ), ટ્રાફિક લાઈટ્સ, લાઇટિંગ. \\ \hline
\end{tabulary}
\end{solutionbox}

\begin{mnemonicbox}
\mnemonic{Vary Photo Emit: Varactor varies frequency, Photo detects light, LED emits light}
\end{mnemonicbox}

\questionmarks{3(b) OR}{4}{P-N junction ડાયોડની કાર્યપધ્ધતિ ફોરવર્ડ બાયસ અને રીવર્સ બાયસ માં સમજાવો.}

\begin{solutionbox}
\begin{enumerate}
    \item \textbf{ફોરવર્ડ બાયસ:} P-ટર્મિનલ પોઝિટિવ સાથે અને N-ટર્મિનલ નેગેટિવ સાથે જોડાયેલ.
    \begin{itemize}
        \item P માંથી હોલ અને N માંથી ઇલેક્ટ્રોન જંક્શન તરફ ધકેલાય છે.
        \item ડીપ્લીશન વિડ્થ ઘટે છે.
        \item ઓછો રેઝિસ્ટન્સ પાથ; કરંટ સરળતાથી વહે છે.
    \end{itemize}
    \item \textbf{રિવર્સ બાયસ:} P-ટર્મિનલ નેગેટિવ સાથે અને N-ટર્મિનલ પોઝિટિવ સાથે જોડાયેલ.
    \begin{itemize}
        \item P માંથી હોલ અને N માંથી ઇલેક્ટ્રોન જંક્શનથી દૂર ખેંચાય છે.
        \item ડીપ્લીશન વિડ્થ વધે છે.
        \item ઉચ્ચ રેઝિસ્ટન્સ પાથ; લગભગ કોઈ કરંટ વહેતો નથી (લીકેજ સિવાય).
    \end{itemize}
\end{enumerate}

\begin{center}
\begin{tabular}{cc}
\begin{circuitikz}[scale=0.7]
    \draw (0,0) to[battery1, l=$V$] (0,2) -- (1,2)
    to[D*, l=Diode] (3,2) -- (3,0) -- (0,0);
    \node at (1.5,-0.5) {ફોરવર્ડ બાયસ};
\end{circuitikz} &
\begin{circuitikz}[scale=0.7]
    \draw (0,0) to[battery1, l=$V$, invert] (0,2) -- (1,2)
    to[D*, l=Diode] (3,2) -- (3,0) -- (0,0);
    \node at (1.5,-0.5) {રિવર્સ બાયસ};
\end{circuitikz}
\end{tabular}
\end{center}
\end{solutionbox}

\begin{mnemonicbox}
\mnemonic{Forward Flows, Reverse Resists}
\end{mnemonicbox}

\questionmarks{3(c) OR}{7}{Photo ડાયોડ ની લાક્ષણિકતા દોરો. Photo ડાયોડની કાર્યપધ્ધિત આકૃતિ સાથે સમજાવો અને તેની એપ્લીકેશન લખો.}

\begin{solutionbox}
\textbf{Photo ડાયોડની કાર્યપધ્ધિત:}
Photo ડાયોડ એ P-N જંક્શન ડાયોડ છે જે \textbf{રિવર્સ બાયસ}માં ઓપરેટ કરવા માટે ડિઝાઈન કરેલ છે. તેમાં પ્રકાશને જંક્શન પર પડવા દેવા માટે પારદર્શક બારી હોય છે.
\begin{itemize}
    \item જ્યારે પ્રકાશ (ફોટોન્સ) ડીપ્લીશન રીજીયન પર પડે છે, ત્યારે તે ઇલેક્ટ્રોન-હોલ પેર ઉત્પન્ન કરે છે.
    \item રિવર્સ બાયસ ઇલેક્ટ્રિક ફિલ્ડ આ કેરિયર્સને જંક્શન પાર કરાવે છે, જેનાથી કરંટ ઉત્પન્ન થાય છે.
    \item આ રિવર્સ કરંટ પ્રકાશની તીવ્રતાના પ્રમાણમાં હોય છે.
\end{itemize}

\begin{center}
\begin{tabular}{cc}
\begin{circuitikz}[scale=0.8]
    \draw (0,0) to[photodiode, l=PD] (2,0);
\end{circuitikz} &
\begin{tikzpicture}[scale=0.8]
    \draw[->] (-3,0) -- (0,0) node[right] {$V_R$};
    \draw[->] (0,-3) -- (0,0) node[above] {$I_R$};
    
    \draw[blue, thick] (0,0) -- (-2.5, -0.5) node[left] {Low Light};
    \draw[blue, thick] (0,0) -- (-2.5, -1.5) node[left] {Medium};
    \draw[blue, thick] (0,0) -- (-2.5, -2.5) node[left] {High Light};
\end{tikzpicture} \\
સિમ્બોલ & લાક્ષણિકતાઓ
\end{tabular}
\end{center}

\textbf{એપ્લિકેશન્સ:}
\begin{itemize}
    \item લાઈટ સેન્સર્સ (સ્ટ્રીટ લાઈટ્સ)
    \item ઓપ્ટિકલ ફાઈબર રિસીવર્સ
    \item રિમોટ કંટ્રોલ રિસીવર્સ
    \item સિક્યુરિટી એલાર્મ્સ
\end{itemize}
\end{solutionbox}

\begin{mnemonicbox}
\mnemonic{Light In, Current Out: Light intensity controls current output}
\end{mnemonicbox}
\questionmarks{4(a)}{3}{ટૂંકનોંધ લખો: હાફ વેવ રેક્ટિફાયર}

\begin{solutionbox}
\textbf{હાફ વેવ રેક્ટિફાયર:}
\begin{center}
\begin{tabular}{cc}
\begin{circuitikz}[scale=0.8]
    \draw (0,0) to[sV, l=$V_{in}$] (0,2) 
    to[D, l=D] (2,2) 
    to[R, l=$R_L$] (2,0) -- (0,0);
    \draw (2,2) -- (3,2) node[right] {$+$};
    \draw (2,0) -- (3,0) node[right] {$-$ $V_{out}$};
\end{circuitikz} &
\begin{tikzpicture}[scale=0.6]
    \draw[->] (0,0) -- (4,0) node[right] {$t$};
    \draw[->] (0,-1.5) -- (0,1.5) node[above] {$V$};
    \draw[blue, dashed] plot[domain=0:3.5] (\x, {sin(\x r * 3)});
    \draw[red, thick] plot[domain=0:3.5] (\x, {max(0, sin(\x r * 3))});
    \node at (2, -1) {ઈનપુટ/આઉટપુટ};
\end{tikzpicture}
\end{tabular}
\captionof{figure}{હાફ વેવ રેક્ટિફાયર સર્કિટ અને વેવફોર્મ}
\end{center}

\begin{tabulary}{\linewidth}{|L|L|}
\hline
\textbf{ઓપરેશન ફેઝ} & \textbf{વર્ણન} \\ \hline
\textbf{પોઝિટિવ હાફ સાયકલ} & ડાયોડ ફોરવર્ડ બાયસ થાય છે અને કંડક્ટ કરે છે. લોડમાંથી કરંટ વહે છે. આઉટપુટ ઈનપુટને અનુસરે છે. \\ \hline
\textbf{નેગેટિવ હાફ સાયકલ} & ડાયોડ રિવર્સ બાયસ થાય છે અને કરંટ બ્લોક કરે છે. આઉટપુટ વોલ્ટેજ શૂન્ય હોય છે. \\ \hline
\end{tabulary}

\begin{itemize}
    \item \textbf{આઉટપુટ ફ્રિક્વન્સી:} $f_{out} = f_{in}$
    \item \textbf{કાર્યક્ષમતા (Efficiency):} 40.6\%
    \item \textbf{રિપલ ફેક્ટર:} 1.21
\end{itemize}
\end{solutionbox}

\begin{mnemonicbox}
\mnemonic{Half Passes Positive: Only positive half-cycle passes through}
\end{mnemonicbox}

\questionmarks{4(b)}{4}{વોલ્ટેજ રેગ્યુલેટર તરીકે ઝેનર ડાયોડ સમજાવો.}

\begin{solutionbox}
\textbf{ઝેનર ડાયોડ વોલ્ટેજ રેગ્યુલેટર:}
\begin{center}
\begin{circuitikz}[scale=0.9]
    \draw (0,0) to[battery1, l=$V_{in}$] (0,3) to[R, l=$R_S$] (3,3) -- (5,3);
    \draw (3,3) to[zD*, l=$D_Z$] (3,0);
    \draw (5,3) to[R, l=$R_L$] (5,0);
    \draw (0,0) -- (5,0);
    \node at (5.5, 1.5) {$V_{out} = V_Z$};
\end{circuitikz}
\captionof{figure}{ઝેનર વોલ્ટેજ રેગ્યુલેટર}
\end{center}

\textbf{કાર્ય સિદ્ધાંત:}
\begin{enumerate}
    \item ઝેનર ડાયોડ લોડની સમાંતરમાં \textbf{રિવર્સ બાયસ}માં જોડાયેલ હોય છે.
    \item તે બ્રેકડાઉન રીજીયનમાં કાર્ય કરે છે જ્યાં તેના પરનો વોલ્ટેજ ($V_Z$) કરંટની વિશાળ શ્રેણી માટે અચળ રહે છે.
    \item \textbf{લાઇન રેગ્યુલેશન:} જો ઈનપુટ વોલ્ટેજ $V_{in}$ વધે છે, તો ઝેનર કરંટ વધે છે, પરંતુ $R_L$ પર વોલ્ટેજ $V_Z$ પર સ્થિર રહે છે. વધારાનો વોલ્ટેજ $R_S$ પર ડ્રોપ થાય છે.
    \item \textbf{લોડ રેગ્યુલેશન:} જો લોડ કરંટ બદલાય છે, તો ઝેનર કરંટ એડજસ્ટ થાય છે જેથી કુલ કરંટ અચળ રહે અને $V_{out} = V_Z$ જળવાઈ રહે.
\end{enumerate}
\end{solutionbox}

\begin{mnemonicbox}
\mnemonic{Zener Zeros Voltage Variations}
\end{mnemonicbox}

\questionmarks{4(c)}{7}{રેક્ટિફાયરની જરૂરિયાત લખો. બ્રિજ વેવ રેક્ટિફાયર સર્કિટ ડાયાગ્રામ સાથે સમજાવો અને તેના ઇનપુટ અને આઉટપુટ વેવફોર્મ દોરો.}

\begin{solutionbox}
\textbf{રેક્ટિફાયરની જરૂરિયાત:}
\begin{itemize}
    \item AC વોલ્ટેજ (મેઈન્સમાંથી) ને DC વોલ્ટેજમાં કન્વર્ટ કરવા માટે.
    \item મોટાભાગના ઇલેક્ટ્રોનિક ઉપકરણો (ટીવી, કોમ્પ્યુટર, મોબાઈલ) ને કાર્ય કરવા માટે DC પાવરની જરૂર હોય છે.
\end{itemize}

\textbf{બ્રિજ વેવ રેક્ટિફાયર:}
\begin{center}
\begin{circuitikz}[scale=0.8]
    \draw (0,0) to[sV, l=$V_{in}$] (0,3);
    \draw (0,3) -- (2,3) -- (3,2);
    \draw (0,0) -- (2,0) -- (5,0) -- (5,2);
    
    % Bridge
    \draw (3,2) to[D*, l=$D_1$] (4,3);
    \draw (4,3) to[D*, l=$D_2$] (5,2);
    \draw (5,2) to[D*, l=$D_3$] (4,1);
    \draw (4,1) to[D*, l=$D_4$] (3,2);
    
    % Load
    \draw (4,3) -- (4,4) -- (7,4) to[R, l=$R_L$] (7,0) -- (5,0);
    \draw (4,1) to[short,-*] (4,1) -- (4,0); % Ground connection visually
    \node at (7.5, 2) {$V_{out}$};
\end{circuitikz}
\captionof{figure}{બ્રિજ રેક્ટિફાયર સર્કિટ}
\end{center}

\textbf{વેવફોર્મ્સ:}
\begin{center}
\begin{tikzpicture}[scale=0.7]
    \draw[->] (0,0) -- (7,0) node[right] {$t$};
    \draw[->] (0,2) -- (0,5) node[above] {$V_{in}$};
    \draw[blue] (0,3.5) sin (1,4.5) cos (2,3.5) sin (3,2.5) cos (4,3.5) sin (5,4.5) cos (6,3.5);
    \node at (8, 3.5) {ઈનપુટ};
    
    \draw[->] (0,-2) -- (7,-2) node[right] {$t$};
    \draw[->] (0,-2) -- (0,1) node[above] {$V_{out}$};
    \draw[red, thick] (0,-2) sin (1,-1) cos (2,-2) sin (3,-1) cos (4,-2) sin (5,-1) cos (6,-2);
    \node at (8, -1) {આઉટપુટ};
\end{tikzpicture}
\captionof{figure}{ઈનપુટ અને આઉટપુટ વેવફોર્મ}
\end{center}

\begin{itemize}
    \item \textbf{પોઝિટિવ હાફ સાયકલ:} $D_1$ અને $D_3$ કંડક્ટ કરે છે.
    \item \textbf{નેગેટિવ હાફ સાયકલ:} $D_2$ અને $D_4$ કંડક્ટ કરે છે.
    \item બંને હાફ સાયકલમાં $R_L$ માંથી એક જ દિશામાં કરંટ વહે છે.
    \item કાર્યક્ષમતા: 81.2\%
\end{itemize}
\end{solutionbox}

\begin{mnemonicbox}
\mnemonic{Bridge Both Better: Bridge rectifier uses both half cycles}
\end{mnemonicbox}

\questionmarks{4(a) OR}{3}{શંટ કેપેસિટર ફિલ્ટરની કાર્યપધ્ધતિ સમજાવો.}

\begin{solutionbox}
\textbf{શંટ કેપેસિટર ફિલ્ટર:}
\begin{center}
\begin{circuitikz}[scale=0.8]
    \draw (0,0) to[sV, l=Rectifier] (0,2) -- (2,2);
    \draw (2,2) to[C, l=$C$] (2,0);
    \draw (2,2) -- (4,2) to[R, l=$R_L$] (4,0) -- (0,0);
    \draw (2,0) -- (0,0);
\end{circuitikz}
\end{center}
\begin{itemize}
    \item \textbf{ચાર્જિંગ:} જ્યારે રેક્ટિફાયર વોલ્ટેજ વધે છે, ત્યારે કેપેસિટર પીક વોલ્ટેજ $V_m$ સુધી ચાર્જ થાય છે.
    \item \textbf{ડિસ્ચાર્જિંગ:} જ્યારે રેક્ટિફાયર વોલ્ટેજ ઘટે છે, ત્યારે કેપેસિટર $R_L$ દ્વારા ડિસ્ચાર્જ થાય છે, વોલ્ટેજ જાળવી રાખે છે.
    \item \textbf{પરિણામ:} ઓછા રિપલ સાથે સ્મૂધ DC આઉટપુટ.
\end{itemize}
\end{solutionbox}

\begin{mnemonicbox}
\mnemonic{Capacitor Catches Peaks: Capacitor stores peak voltage}
\end{mnemonicbox}

\questionmarks{4(b) OR}{4}{સેન્ટર ટેપ ફૂલ વેવ રેક્ટિફાયર અને બ્રિજ વેવ રેક્ટિફાયરની સરખામણી કરો.}

\begin{solutionbox}
\begin{tabulary}{\linewidth}{|L|L|L|}
\hline
\textbf{પેરામીટર} & \textbf{સેન્ટર ટેપ FW રેક્ટિફાયર} & \textbf{બ્રિજ રેક્ટિફાયર} \\ \hline
\textbf{ડાયોડની સંખ્યા} & 2 & 4 \\ \hline
\textbf{ટ્રાન્સફોર્મર} & સેન્ટર-ટેપ્ડ જરૂરી (મોટું, ખર્ચાળ) & સાદું ટ્રાન્સફોર્મર (નાનું, સસ્તું) \\ \hline
\textbf{PIV રેટિંગ} & $2V_m$ & $V_m$ \\ \hline
\textbf{કાર્યક્ષમતા (Efficiency)} & 81.2\% & 81.2\% \\ \hline
\textbf{ખર્ચ} & વધારે (ટ્રાન્સફોર્મરને કારણે) & ઓછો \\ \hline
\end{tabulary}
\end{solutionbox}

\begin{mnemonicbox}
\mnemonic{Center Taps Transformer, Bridge Bypasses Tapping}
\end{mnemonicbox}

\questionmarks{4(c) OR}{7}{રેક્ટિફાયરમાં ફિલ્ટર સર્કિટની જરૂરિયાત લખો. $\pi$ ફિલ્ટર સર્કિટ ડાયાગ્રામ સાથે સમજાવો અને તેના ઇનપુટ અને આઉટપુટ વેવફોર્મ દોરો.}

\begin{solutionbox}
\textbf{ફિલ્ટરની જરૂરિયાત:}
રેક્ટિફાયર આઉટપુટમાંથી AC કમ્પોનન્ટ્સ (રિપલ) દૂર કરવા અને ઇલેક્ટ્રોનિક સર્કિટ માટે યોગ્ય સ્ટેડી/સ્મૂધ DC વોલ્ટેજ પ્રદાન કરવા.

\textbf{$\pi$ ફિલ્ટર (CLC ફિલ્ટર):}
\begin{center}
\begin{circuitikz}[scale=0.8]
    \draw (0,0) to[sV, l=In] (0,2) -- (1,2);
    \draw (1,2) to[C, l=$C_1$] (1,0);
    \draw (1,2) to[L, l=$L$] (3,2);
    \draw (3,2) to[C, l=$C_2$] (3,0);
    \draw (3,2) -- (4,2) to[R, l=$R_L$] (4,0) -- (0,0);
    \draw (1,0) -- (0,0); % ગ્રાઉન્ડ
    \draw (3,0) -- (1,0);
\end{circuitikz}
\captionof{figure}{$\pi$ (Pi) ફિલ્ટર સર્કિટ}
\end{center}

\begin{itemize}
    \item \textbf{$C_1$}: મોટાભાગના AC રિપલને ગ્રાઉન્ડમાં બાયપાસ કરે છે.
    \item \textbf{$L$}: બાકીના AC ને બ્લોક કરે છે (AC માટે ઉચ્ચ ઈમ્પિડન્સ) પરંતુ DC ને પસાર થવા દે છે.
    \item \textbf{$C_2$}: કોઈપણ બાકી રહેલા રિપલને ફિલ્ટર કરે છે.
\end{itemize}

\textbf{વેવફોર્મ્સ:}
\begin{center}
\begin{tikzpicture}[scale=0.7]
    \draw[->] (0,0) -- (5,0) node[right] {$t$};
    \draw[->] (0,0) -- (0,2) node[above] {$V$};
    \draw[red, thick] plot[domain=0:5] (\x, {1.5 + 0.1*sin(\x r * 5)});
    \node at (2.5, 2) {ફિલ્ટર કરેલ આઉટપુટ (લગભગ DC)};
\end{tikzpicture}
\end{center}
\end{solutionbox}

\begin{mnemonicbox}
\mnemonic{Capacitor-Inductor-Capacitor Perfectly Irons}
\end{mnemonicbox}

\questionmarks{5(a)}{3}{PNP ટ્રાન્ઝિસ્ટરની કાર્યપધ્ધતિ આકૃતિ સાથે સમજાવો.}

\begin{solutionbox}
\textbf{PNP ટ્રાન્ઝિસ્ટર:}
PNP ટ્રાન્ઝિસ્ટરમાં બે P-ટાઈપ વિસ્તારો (એમિટર અને કલેક્ટર) વચ્ચે પાતળું N-ટાઈપ બેઝ સેન્ડવીચ કરેલું હોય છે.

\begin{center}
\begin{tikzpicture}[scale=0.8]
    \draw (0,0) rectangle (3,2);
    \draw (1,0) -- (1,2);
    \draw (2,0) -- (2,2);
    \node at (0.5,1) {P (E)};
    \node at (1.5,1) {N (B)};
    \node at (2.5,1) {P (C)};
    
    % Biasing
    \draw (0.5,0) -- (0.5,-1) -- (1.5,-1);
    \draw (1.5,0) -- (1.5,-1);
    \node at (1,-1.3) {ફોરવર્ડ બાયસ્ડ};
    
    \draw (2.5,0) -- (2.5,-1) -- (1.5,-1);
    \node at (2,-1.3) {રિવર્સ બાયસ્ડ};
\end{tikzpicture}
\end{center}

\begin{itemize}
    \item \textbf{એમિટર-બેઝ જંક્શન:} ફોરવર્ડ બાયસ્ડ. હોલ એમિટરથી બેઝ તરફ જાય છે.
    \item \textbf{કલેક્ટર-બેઝ જંક્શન:} રિવર્સ બાયસ્ડ. બેઝને પાર કરતા હોલ કલેક્ટર દ્વારા એકત્રિત કરવામાં આવે છે.
    \item મેજોરિટી કેરિયર્સ: હોલ. કરંટ પ્રવાહ એમિટરથી કલેક્ટર તરફ હોય છે.
\end{itemize}
\end{solutionbox}

\begin{mnemonicbox}
\mnemonic{Positive-Negative-Positive}
\end{mnemonicbox}

\questionmarks{5(b)}{4}{N-channel JFET ની કાર્યપધ્ધતિ આકૃતિ સાથે સમજાવો.}

\begin{solutionbox}
\textbf{N-channel JFET:}
\begin{center}
\begin{circuitikz}[scale=0.8]
    \draw (0,0) node[njfet] (Q) {};
    \node[right] at (Q.D) {ડ્રેન};
    \node[right] at (Q.S) {સોર્સ};
    \node[left] at (Q.G) {ગેટ};
    
    % Channel illustration
    \draw (2, -1) rectangle (3, 1); \node at (2.5,0) {N-Ch};
    \draw[fill=gray] (1.8, 0) rectangle (2, 0.5); \node[left] at (1.8, 0.25) {P (Gate)};
    \draw[fill=gray] (3, 0) rectangle (3.2, 0.5); \node[right] at (3.2, 0.25) {P (Gate)};
\end{circuitikz}
\end{center}

\textbf{કાર્યપધ્ધતિ:}
\begin{itemize}
    \item કરંટ N-ચેનલ દ્વારા ડ્રેનથી સોર્સ તરફ વહે છે.
    \item ગેટ પર રિવર્સ વોલ્ટેજ લાગુ કરવામાં આવે છે ($V_{GS} < 0$).
    \item નેગેટિવ ગેટ વોલ્ટેજ વધારવાથી ડીપ્લીશન રીજન પહોળી થાય છે, ચેનલ સાંકડી થાય છે.
    \item આ રેસિસ્ટન્સ વધારે છે અને ડ્રેન કરંટ ($I_D$) ઘટાડે છે. તેથી, તે Voltage Controlled Device છે.
\end{itemize}
\end{solutionbox}

\begin{mnemonicbox}
\mnemonic{Negative Channel Junction Effect}
\end{mnemonicbox}

\questionmarks{5(c)}{7}{BJT અને JFET ની સરખામણી કરો.}

\begin{solutionbox}
\begin{tabulary}{\linewidth}{|L|L|L|}
\hline
\textbf{પેરામીટર} & \textbf{BJT} & \textbf{JFET} \\ \hline
\textbf{પ્રકાર} & બાયપોલર (ઇલેક્ટ્રોન અને હોલ) & યુનિપોલર (માત્ર મેજોરિટી કેરિયર્સ) \\ \hline
\textbf{કંટ્રોલ} & કરંટ-કંટ્રોલ્ડ ડિવાઇસ ($I_B$ $I_C$ ને કંટ્રોલ કરે છે) & વોલ્ટેજ-કંટ્રોલ્ડ ડિવાઇસ ($V_{GS}$ $I_D$ ને કંટ્રોલ કરે છે) \\ \hline
\textbf{ઇનપુટ ઇમ્પીડન્સ} & ઓછો ($k\Omega$ રેન્જ) & ખૂબ વધારે ($M\Omega$ રેન્જ) \\ \hline
\textbf{નોઈઝ} & વધારે નોઈઝ & ઓછો નોઈઝ \\ \hline
\textbf{તાપમાન સ્ટેબિલિટી} & ઓછી & વધારે \\ \hline
\textbf{કદ} & મોટું & નાનું (ICs માં બનાવવું સરળ) \\ \hline
\end{tabulary}
\end{solutionbox}

\begin{mnemonicbox}
\mnemonic{Current Bipolar Low, Voltage Unipolar High}
\end{mnemonicbox}

\questionmarks{5(a) OR}{3}{E-waste ને નાબૂદ કરવાની પદ્ધતિ જણાવો અને તેમાથી કોઈપણ એક સમજાવો.}

\begin{solutionbox}
\textbf{પદ્ધતિઓ:}
\begin{enumerate}
    \item રિસાયકલિંગ (Recycling)
    \item રીયુઝ (Reuse)
    \item ઇન્સિનરેશન (Incineration)
    \item લેન્ડફિલિંગ (Landfilling)
    \item ટેક-બેક સિસ્ટમ્સ (Take-back systems)
\end{enumerate}

\textbf{રિસાયકલિંગ:}
આમાં ઇલેક્ટ્રોનિક ઉપકરણોને ડિસમેન્ટલ કરવા અને પ્લાસ્ટિક, ગ્લાસ અને મેટલ્સ (સોનું, કોપર વગેરે) જેવા મટીરિયલ્સને અલગ પાડવાનો સમાવેશ થાય છે. આ રિકવર કરેલા મટીરિયલ્સને પ્રોસેસ કરીને નવા ઉત્પાદનો બનાવવા માટે ઉપયોગમાં લેવાય છે. તે પર્યાવરણીય પ્રદૂષણ ઘટાડે છે અને કુદરતી સંસાધનોનું સંરક્ષણ કરે છે.
\end{solutionbox}

\begin{mnemonicbox}
\mnemonic{RRIL-T: Recycling, Reuse, Incineration, Landfill, Take-back}
\end{mnemonicbox}

\questionmarks{5(b) OR}{4}{PNP અને NPN Transistor ની સરખામણી કરો.}

\begin{solutionbox}
\begin{tabulary}{\linewidth}{|L|L|L|}
\hline
\textbf{પેરામીટર} & \textbf{PNP} & \textbf{NPN} \\ \hline
\textbf{સ્ટ્રક્ચર} & P-N-P લેયર્સ & N-P-N લેયર્સ \\ \hline
\textbf{મેજોરિટી કેરિયર્સ} & હોલ & ઇલેક્ટ્રોન \\ \hline
\textbf{કલેક્ટર કરંટ} & હોલના પ્રવાહને કારણે & ઇલેક્ટ્રોનના પ્રવાહને કારણે \\ \hline
\textbf{બાયસિંગ} & બેઝ એમિટરની સાપેક્ષે નેગેટિવ & બેઝ એમિટરની સાપેક્ષે પોઝિટિવ \\ \hline
\textbf{સ્પીડ} & ધીમી (હોલ મોબિલિટી ઓછી છે) & ઝડપી (ઇલેક્ટ્રોન મોબિલિટી વધારે છે) \\ \hline
\textbf{ઉપયોગ} & ઓછો સામાન્ય & સૌથી સામાન્ય \\ \hline
\end{tabulary}
\end{solutionbox}

\begin{mnemonicbox}
\mnemonic{Positive-Negative-Positive (Holes), Negative-Positive-Negative (Electrons)}
\end{mnemonicbox}

\questionmarks{5(c) OR}{7}{CE કોંફીગરેશન ની ઈનપુટ અને આઉટપુટ લાક્ષણિકતા દોરો અને સમજાવો.}

\begin{solutionbox}
\textbf{1. ઈનપુટ લાક્ષણિકતા ($I_B$ વિરુદ્ધ $V_{BE}$):}
આ સ્થિર $V_{CE}$ પર બેઝ કરંટ અને બેઝ-એમિટર વોલ્ટેજ વચ્ચેનો ગ્રાફ છે. તે ફોરવર્ડ-બાયસ્ડ ડાયોડ કર્વ જેવો દેખાય છે.
\begin{center}
\begin{tikzpicture}[scale=0.7]
    \draw[->] (0,0) -- (4,0) node[right] {$V_{BE}$};
    \draw[->] (0,0) -- (0,3) node[above] {$I_B$};
    \draw[blue, thick] (0.7,0) .. controls (1,0.5) and (1.2,2) .. (1.5,3);
    \node at (2,1) {$V_{CE} > 0$};
\end{tikzpicture}
\end{center}

\textbf{2. આઉટપુટ લાક્ષણિકતા ($I_C$ વિરુદ્ધ $V_{CE}$):}
સ્થિર $I_B$ પર કલેક્ટર કરંટ અને કલેક્ટર-એમિટર વોલ્ટેજ વચ્ચેનો ગ્રાફ.
\begin{itemize}
    \item \textbf{કટ-ઓફ ક્ષેત્ર (Cut-off):} બંને જંક્શન રિવર્સ બાયસ્ડ. $I_C \approx 0$.
    \item \textbf{એક્ટિવ ક્ષેત્ર (Active):} JE ફોરવર્ડ, JC રિવર્સ બાયસ્ડ. $I_C = \beta I_B$. એમ્પ્લીફિકેશન માટે વપરાય છે.
    \item \textbf{સેચુરેશન ક્ષેત્ર (Saturation):} બંને જંક્શન ફોરવર્ડ બાયસ્ડ. $V_{CE}$ ખૂબ ઓછો હોય છે. સ્વિચિંગ (ON) માટે વપરાય છે.
\end{itemize}

\begin{center}
\begin{tikzpicture}[scale=0.7]
    \draw[->] (0,0) -- (4,0) node[right] {$V_{CE}$};
    \draw[->] (0,0) -- (0,4) node[above] {$I_C$};
    \draw (0,0) -- (0.5, 3) -- (4, 3.2) node[right] {$I_{B3}$};
    \draw (0,0) -- (0.5, 2) -- (4, 2.2) node[right] {$I_{B2}$};
    \draw (0,0) -- (0.5, 1) -- (4, 1.2) node[right] {$I_{B1}$};
    
    \node[draw, dashed, fit={(0,0) (0.5,3.5)}] (sat) {}; \node[above] at (sat.north) {Sat};
    \node[draw, dashed, fit={(0,0) (4,0.5)}] (cut) {}; \node[right] at (cut.east) {Cut-off};
    \node at (2,2.5) {Active};
\end{tikzpicture}
\end{center}
\end{solutionbox}

\begin{mnemonicbox}
\mnemonic{Input Shows Voltage Effects, Output Shows Current Control}
\end{mnemonicbox}

\end{document}
