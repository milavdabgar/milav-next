\documentclass{article}

% content/resources/templates/preamble.tex
\usepackage[margin=0.6in]{geometry}
\author{Milav Dabgar}
\usepackage{amsmath,amssymb,amsthm}
\usepackage{booktabs}
\usepackage{multirow}
\usepackage{xcolor}
\usepackage{tcolorbox}
\tcbuselibrary{breakable,skins}
\usepackage[colorlinks=true,linkcolor=blue]{hyperref}
\usepackage{titlesec}
\usepackage{enumitem}
\usepackage{tikz}
\usepackage{pgfplots}
\usepackage{circuitikz}
\usepackage[version=4]{mhchem}
\usepackage{longtable}
\usepackage{array}
\usepackage{float}
\usepackage{caption}
\usepackage{listings}

\lstset{
  basicstyle=\small\ttfamily,
  breaklines=true,
  breakatwhitespace=false,
  postbreak=\mbox{\textcolor{red}{$\hookrightarrow$}\space},
  float=false,
  numbers=left,
  numberstyle=\tiny\color{gray},
  numbersep=10pt,
  xleftmargin=2em,
  keywordstyle=\color{blue},
  commentstyle=\color{green!60!black},
  stringstyle=\color{purple},
  backgroundcolor=\color{gray!5},
  showstringspaces=false,
  tabsize=2,
  captionpos=b,
  keepspaces=true,
  columns=flexible
}

\pgfplotsset{compat=1.18}
\usetikzlibrary{shapes,arrows,positioning,calc,patterns,decorations.pathmorphing,decorations.markings,arrows.meta}

% Color scheme
\definecolor{headcolor}{RGB}{0,102,204}
\definecolor{keycolor}{RGB}{220,20,60}
\definecolor{solutioncolor}{RGB}{34,139,34}
\definecolor{mnemoniccolor}{RGB}{148,0,211}
\definecolor{codecolor}{RGB}{0,0,100}

% Spacing
\setlength{\parskip}{3pt}
\setlist[itemize]{nosep}
\setlist[enumerate]{nosep}

% Title formatting
\titleformat{\section}{\Large\bfseries\color{headcolor}}{\thesection}{1em}{}
\titleformat{\subsection}{\large\bfseries\color{headcolor}}{\thesubsection}{1em}{}

% Pandoc tightlist compatibility
\providecommand{\tightlist}{%
  \setlength{\itemsep}{0pt}\setlength{\parskip}{0pt}}

% Pandoc longtable compatibility
\newcounter{none}
\def\thenone{}


% content/resources/templates/gujarati-boxes.tex
\usepackage{fontspec}
\usepackage{polyglossia}

% Set Gujarati as main language (document is primarily in Gujarati)
% Note: gloss-gujarati.ldf doesn't exist in polyglossia, but it will use hyphenation patterns
\setdefaultlanguage{gujarati}
\setotherlanguage{english}

% Configure Gujarati font properly
% Use Language=Default to prevent polyglossia from trying to add language-specific features
% that don't exist for Gujarati, which causes "empty feature" warnings
\newfontfamily\gujaratifont[Script=Gujarati,AutoFakeBold=2.5,AutoFakeSlant=0.3]{Noto Sans Gujarati}
\setmainfont[Script=Gujarati,AutoFakeBold=2.5,AutoFakeSlant=0.3]{Noto Sans Gujarati}
% Use Noto Sans Gujarati for monospace to support Gujarati in text
\setmonofont[Scale=0.9]{Noto Sans Gujarati}

% Configure English to use the same font
\newfontfamily\englishfont[Script=Gujarati,AutoFakeBold=2.5,AutoFakeSlant=0.3]{Noto Sans Gujarati}

% Translations for polyglossia
\gappto\captionsgujarati{
  \renewcommand{\tablename}{કોષ્ટક}
  \renewcommand{\figurename}{આકૃતિ}
}

% Helper for TikZ nodes to ensure Gujarati font
\newcommand{\gu}[1]{{\gujaratifont #1}}

% Custom environments
\newtcolorbox{solutionbox}{
    breakable,
    enhanced,
    colback=solutioncolor!5!white,
    colframe=solutioncolor!75!black,
    fonttitle=\bfseries,
    title=જવાબ
}

\newtcolorbox{solutionboxnobreak}{
 colback=solutioncolor!5!white,
 colframe=solutioncolor!75!black,
 fonttitle=\bfseries,
 title=જવાબ
}

\newtcolorbox{keyformula}{
 breakable,
 enhanced,
 colback=keycolor!5!white,
 colframe=keycolor!75!black,
 fonttitle=\bfseries,
 title=રાસાયણિક સમીકરણ/સૂત્ર
}

\newtcolorbox{mnemonicbox}{
 breakable,
 enhanced,
 colback=mnemoniccolor!5!white,
 colframe=mnemoniccolor!75!black,
 fonttitle=\bfseries,
 title=મેમરી ટ્રીક
}


% Custom commands for GTU solutions
% This file defines semantic commands for consistent formatting

% Question command with automatic formatting
\newcommand{\question}[2]{%
  \section*{Question #1}%
  \textbf{#2}%
}

% OR question variant
\newcommand{\questionor}[2]{%
  \section*{Question #1 OR}%
  \textbf{#2}%
}

% Proper table environment with caption
\newenvironment{answertable}[1]{%
  \begin{table}[htbp]
  \centering
  \caption{#1}
}{%
  \end{table}
}

% Proper figure environment for diagrams
\newenvironment{answerdiagram}[1]{%
  \begin{figure}[htbp]
  \centering
  \caption{#1}
}{%
  \end{figure}
}

% Semantic markup for key terms
\newcommand{\keyword}[1]{\textbf{#1}}
\newcommand{\code}[1]{\texttt{#1}}
\newcommand{\classname}[1]{\texttt{#1}}
\newcommand{\methodname}[1]{\texttt{#1}}

% Proper quotation marks
\newcommand{\mnemonic}[1]{``#1''}


\title{Elements of Electrical \& Electronics Engineering (1313202) - Winter 2023 Solution}
\date{January 17, 2024}

\begin{document}
\maketitle

\questionmarks{1(a)}{3}{એક્ટિવ અને પેસિવ નેટવર્કનો તફાવત સમજાવો.}

\begin{solutionbox}
\begin{tabulary}{\linewidth}{|L|L|}
\hline
\textbf{એક્ટિવ નેટવર્ક} & \textbf{પેસિવ નેટવર્ક} \\ \hline
ઓછામાં ઓછા એક ઊર્જા સ્ત્રોત ધરાવે છે. & કોઈ ઊર્જા સ્ત્રોત ધરાવતું નથી. \\ \hline
અન્ય તત્વોને પાવર આપી શકે છે. & અન્ય તત્વોને પાવર આપી શકતું નથી. \\ \hline
ઉદાહરણ: ટ્રાન્ઝિસ્ટર, ઑપ-એમ્પ, બેટરી & ઉદાહરણ: રેઝિસ્ટર, કેપેસિટર, ઇન્ડક્ટર \\ \hline
\end{tabulary}
\end{solutionbox}

\begin{mnemonicbox}
\mnemonic{Active Adds Power, Passive Pulls Power}
\end{mnemonicbox}

\questionmarks{1(b)}{4}{કિર્ચોફનો વોલ્ટેજનો નિયમ જણાવો અને સમજાવો.}

\begin{solutionbox}
\textbf{કિર્ચોફનો વોલ્ટેજનો નિયમ (KVL)}: સર્કિટમાં કોઈપણ બંધ પથ (લૂપ) ની આસપાસના તમામ વોલ્ટેજનો બીજગણિતીય સરવાળો શૂન્ય હોય છે.

\textbf{ગણિતીય સ્વરૂપ}: $\sum V = 0$ અથવા $V_1 + V_2 + V_3 + V_4 = 0$

\begin{center}
\begin{circuitikz}
    \draw (0,0) to[sV, l=$V_s$] (0,2) -- (2,2) to[R, l=$R_1$, v=$V_1$] (2,0) -- (0,0);
    \draw (2,2) -- (4,2) to[R, l=$R_2$, v=$V_2$] (4,0) -- (2,0);
\end{circuitikz}
\captionof{figure}{KVL માટે બંધ લૂપ}
\end{center}

\begin{itemize}
    \item \textbf{સર્કિટ એપ્લિકેશન}: જ્યારે એક લૂપની આસપાસ ફરતી વખતે, વોલ્ટેજમાં વધારો (બેટરી) ધન અને વોલ્ટેજમાં ઘટાડો (ઘટકો) ઋણ હોય છે.
    \item \textbf{ભૌતિક અર્થ}: બંધ લૂપમાં કુલ ઊર્જા સંરક્ષિત રહે છે.
\end{itemize}
\end{solutionbox}

\begin{mnemonicbox}
\mnemonic{Voltage Loop Sum Zero}
\end{mnemonicbox}

\questionmarks{1(c)}{7}{નીચેના પદોની વ્યાખ્યા આપો: (1) ચાર્જ (2) કરંટ (3) પોટેન્શિયલ (4) E.M.F. (5) ઇન્ડક્ટન્સ (6) કેપેસિટન્સ (7) આવૃત્તિ.}

\begin{solutionbox}
\begin{tabulary}{\linewidth}{|L|L|}
\hline
\textbf{પદ} & \textbf{વ્યાખ્યા} \\ \hline
\textbf{ચાર્જ} & મૂળભૂત વિદ્યુત માત્રા જે કૂલોમ્બ (C)માં માપવામાં આવે છે; ઇલેક્ટ્રોનોનો પ્રવાહ વીજળી બનાવે છે. \\ \hline
\textbf{કરંટ} & વિદ્યુત ચાર્જનો પ્રવાહ દર, એમ્પિયર (A)માં માપવામાં આવે છે; $I = dQ/dt$. \\ \hline
\textbf{પોટેન્શિયલ} & એકમ ચાર્જ દીઠ વિદ્યુત પોટેન્શિયલ ઊર્જા, વોલ્ટ (V)માં માપવામાં આવે છે. \\ \hline
\textbf{E.M.F.} & ઇલેક્ટ્રો મોટિવ ફોર્સ, સ્ત્રોત દ્વારા એકમ ચાર્જ દીઠ પૂરી પાડવામાં આવતી ઊર્જા, વોલ્ટ (V)માં. \\ \hline
\textbf{ઇન્ડક્ટન્સ} & કરંટમાં ફેરફારનો વિરોધ કરવાની વાહકની ક્ષમતા, હેનરી (H)માં માપવામાં આવે છે. \\ \hline
\textbf{કેપેસિટન્સ} & વિદ્યુત ચાર્જ સંગ્રહ કરવાની ઘટકની ક્ષમતા, ફેરડ (F)માં માપવામાં આવે છે. \\ \hline
\textbf{આવૃત્તિ} & એક અલ્ટરનેટિંગ રાશિના એક સેકંડમાં થતા ચક્રોની સંખ્યા, હર્ટ્ઝ (Hz)માં. \\ \hline
\end{tabulary}
\end{solutionbox}

\begin{mnemonicbox}
\mnemonic{Careful Currents Pass Easily Into Circuit Frequently}
\end{mnemonicbox}

\questionmarks{1(c) OR}{7}{ઓહમનો નિયમ જણાવો. તેના ઉપયોગો અને મર્યાદા લખો.}

\begin{solutionbox}
\textbf{ઓહમનો નિયમ}: વાહક વડે પસાર થતો કરંટ, તેના છેડા વચ્ચેના પોટેન્શિયલ ડિફરન્સના સમપ્રમાણમાં અને તેના રેઝિસ્ટન્સના વ્યસ્ત પ્રમાણમાં હોય છે.

\textbf{ગણિતીય સ્વરૂપ}: $I = V/R$

\begin{center}
\begin{circuitikz}
    \draw (0,0) to[battery1, l=$V$] (0,2) -- (2,2) to[R, l=$R$, i=$I$] (2,0) -- (0,0);
\end{circuitikz}
\captionof{figure}{ઓહમનો નિયમ સર્કિટ}
\end{center}

\textbf{ઓહમના નિયમના ઉપયોગો}:
\begin{itemize}
    \item સર્કિટમાં કરંટ, વોલ્ટેજ, રેઝિસ્ટન્સની ગણતરી.
    \item વિદ્યુત નેટવર્કની ડિઝાઇન.
    \item પાવર ગણતરી ($P = VI = I^2R = V^2/R$).
    \item વોલ્ટેજ ડિવિઝન અને કરંટ ડિવિઝન.
\end{itemize}

\textbf{ઓહમના નિયમની મર્યાદાઓ}:
\begin{itemize}
    \item નોન-લિનિયર ઘટકો (ડાયોડ, ટ્રાન્ઝિસ્ટર) માટે માન્ય નથી.
    \item ખૂબ ઉચ્ચ આવૃત્તિઓ પર લાગુ પડતો નથી.
    \item અર્ધવાહકો જેવા બિન-ધાતુના વાહકો માટે લાગુ પડતો નથી.
    \item વેક્યુમ ટ્યુબ અને વાયુ ઉપકરણો માટે લાગુ પડતો નથી.
\end{itemize}
\end{solutionbox}

\begin{mnemonicbox}
\mnemonic{Voltage Drives, Resistance Restricts}
\end{mnemonicbox}

\questionmarks{2(a)}{3}{વાહક, અવાહક અને અર્ધવાહક નો એનર્જી બેન્ડ ની આકૃતિ દોરી સમજાવો.}

\begin{solutionbox}
\textbf{એનર્જી બેન્ડ આરેખ}:

\begin{center}
\begin{tikzpicture}[scale=0.7]
    % Conductor
    \draw (0,0) rectangle (2,1) node[midway] {Valence};
    \draw (0,0.8) rectangle (2,1.8) node[midway] {Conduction};
    \node[above] at (1,1.8) {વાહક};
    \node[below] at (1,0) {ઓવરલેપ};
    
    % Semiconductor
    \draw (3,0) rectangle (5,1) node[midway] {Valence};
    \draw (3,2) rectangle (5,3) node[midway] {Conduction};
    \node[above] at (4,3) {અર્ધવાહક};
    \node at (4,1.5) {નાનું અંતર ($\approx 1eV$)};
    
    % Insulator
    \draw (6,0) rectangle (8,1) node[midway] {Valence};
    \draw (6,3) rectangle (8,4) node[midway] {Conduction};
    \node[above] at (7,4) {અવાહક};
    \node at (7,2) {મોટું અંતર ($> 5eV$)};
\end{tikzpicture}
\captionof{figure}{એનર્જી બેન્ડ આરેખ}
\end{center}

\begin{itemize}
    \item \textbf{વાહક}: વેલેન્સ અને કન્ડકશન બેન્ડ ઓવરલેપ થાય છે, જે ઇલેક્ટ્રોનને સરળતાથી વહેવા દે છે.
    \item \textbf{અર્ધવાહક}: બેન્ડ વચ્ચે નાનું એનર્જી ગેપ ($\approx 1$ eV), ઇલેક્ટ્રોન થર્મલ એનર્જી સાથે જંપ કરી શકે છે.
    \item \textbf{અવાહક}: મોટું એનર્જી ગેપ ($> 5$ eV) બેન્ડ વચ્ચે ઇલેક્ટ્રોન મૂવમેન્ટને અટકાવે છે.
\end{itemize}
\end{solutionbox}

\begin{mnemonicbox}
\mnemonic{Conductors Connect, Semiconductors Sometimes, Insulators Impede}
\end{mnemonicbox}

\questionmarks{2(b)}{4}{Maximum power transfer theorem અને reciprocity theorem નું સ્ટેટમેન્ટ લખો.}

\begin{solutionbox}
\begin{tabulary}{\linewidth}{|L|L|}
\hline
\textbf{થિયરમ} & \textbf{સ્ટેટમેન્ટ} \\ \hline
\textbf{Maximum Power Transfer Theorem} & સ્ત્રોતમાંથી લોડમાં મહત્તમ પાવર ત્યારે ટ્રાન્સફર થાય જ્યારે લોડ રેઝિસ્ટન્સ સ્ત્રોતના આંતરિક રેઝિસ્ટન્સ જેટલો હોય ($R_L = R_S$). \\ \hline
\textbf{Reciprocity Theorem} & એક લિનિયર પેસિવ નેટવર્કમાં એક સિંગલ સ્ત્રોત સાથે, જો સ્ત્રોત પોઝિશન Aથી Bમાં ખસેડવામાં આવે, તો Bમાં સ્ત્રોત હોય ત્યારે Aમાં જે કરંટ મળે તે Aમાં સ્ત્રોત હોય ત્યારે Bમાં મળતા કરંટ જેટલો જ હશે. \\ \hline
\end{tabulary}

\begin{center}
\begin{circuitikz}[scale=0.8]
    \draw (0,0) to[V, l=$V_S$] (0,2) to[R, l=$R_S$] (2,2) -- (3,2) to[R, l=$R_L$] (3,0) -- (0,0);
    \node at (1.5, -0.5) {મહત્તમ પાવર જ્યારે $R_L = R_S$};
\end{circuitikz}
\end{center}
\end{solutionbox}

\begin{mnemonicbox}
\mnemonic{Match Resistance to Maximize Power; Switch Source and Sink, Current Stays Same}
\end{mnemonicbox}

\questionmarks{2(c)}{7}{N-type મટીરીઅલ ની રચના અને તેનું કંડક્શન સમજાવો.}

\begin{solutionbox}
\textbf{N-type અર્ધવાહક રચના}:

\begin{center}
\begin{tikzpicture}[scale=0.8]
    \foreach \x in {0,2,4}
        \foreach \y in {0,2,4}
            \node[circle, draw, minimum size=0.8cm] at (\x,\y) {Si};
    
    \node[circle, draw, fill=gray!20, minimum size=0.8cm] at (2,2) {P}; % Phosphorus dopant
    
    \foreach \x in {0,2,4} {
        \draw (\x,0.4) -- (\x,1.6);
        \draw (\x,2.4) -- (\x,3.6);
    }
    \foreach \y in {0,2,4} {
        \draw (0.4,\y) -- (1.6,\y);
        \draw (2.4,\y) -- (3.6,\y);
    }
    
    \node[circle, fill=black, inner sep=1.5pt] at (2.5,2.5) {};
    \node[right] at (2.6,2.6) {Free $e^-$};
    \node[below] at (2,-0.5) {N-type Lattice};
\end{tikzpicture}
\captionof{figure}{પેન્ટાવેલેન્ટ ડોપિંગ (N-type)}
\end{center}

\begin{itemize}
    \item \textbf{ડોપિંગ પ્રક્રિયા}: સિલિકોન/જર્મેનિયમ (4 વેલેન્સ $e^-$) પેન્ટાવેલેન્ટ તત્વો (P, As, Sb) સાથે ડોપ કરવામાં આવે છે.
    \item \textbf{વધારાનો ઇલેક્ટ્રોન}: કોવેલન્ટ બોન્ડિંગ પછી દરેક ડોપન્ટ અણુ 1 વધારાનો ઇલેક્ટ્રોન આપે છે.
    \item \textbf{કન્ડક્શન મેકેનિઝમ}: 
    \begin{itemize}
        \item \textbf{મેજોરિટી કેરિયર}: ફ્રી ઇલેક્ટ્રોન (નેગેટિવ ચાર્જ કેરિયર).
        \item \textbf{માઇનોરિટી કેરિયર}: હોલ (ખૂબ ઓછા).
    \end{itemize}
    \item \textbf{વિદ્યુત ગુણધર્મો}: વધેલી વાહકતા અને નેગેટિવ ચાર્જ કેરિયર.
\end{itemize}
\end{solutionbox}

\begin{mnemonicbox}
\mnemonic{Pentavalent Provides Plus one Electron, Negative-type}
\end{mnemonicbox}

\questionmarks{2(a) OR}{3}{વેલેન્સ બેન્ડ, કંડક્શન બેન્ડ અને ફોર્બિડન ગેપ ની વ્યાખ્યા આપો.}

\begin{solutionbox}
\begin{tabulary}{\linewidth}{|L|L|}
\hline
\textbf{પદ} & \textbf{વ્યાખ્યા} \\ \hline
\textbf{વેલેન્સ બેન્ડ} & ઇલેક્ટ્રોનથી ભરેલી સૌથી ઉચ્ચ ઊર્જા બેન્ડ, જ્યાં ઇલેક્ટ્રોન અણુઓ સાથે બંધાયેલા હોય છે. \\ \hline
\textbf{કંડક્શન બેન્ડ} & વેલેન્સ બેન્ડની ઉપરની બેન્ડ જ્યાં ઇલેક્ટ્રોન મુક્તપણે ફરે છે અને વિદ્યુત વાહકતામાં યોગદાન આપે છે. \\ \hline
\textbf{ફોર્બિડન ગેપ} & વેલેન્સ અને કંડક્શન બેન્ડ વચ્ચેની ઊર્જા શ્રેણી જ્યાં કોઈ ઇલેક્ટ્રોન સ્ટેટ્સ હોતા નથી. \\ \hline
\end{tabulary}

\begin{center}
\begin{tikzpicture}
    \draw (0,0) rectangle (3,1) node[midway] {Valence Band};
    \draw (0,2) rectangle (3,3) node[midway] {Conduction Band};
    \draw[<->] (3.2,1) -- (3.2,2) node[midway, right] {Forbidden Gap ($E_g$)};
\end{tikzpicture}
\end{center}
\end{solutionbox}

\begin{mnemonicbox}
\mnemonic{Valence Holds, Forbidden Blocks, Conduction Flows}
\end{mnemonicbox}

\questionmarks{2(b) OR}{4}{એક્ટીવ પાવર, રિએક્ટીવ પાવર અને પાવર ફેક્ટર ની વ્યાખ્યા આપો અને પાવર ત્રિકોણ દોરો.}

\begin{solutionbox}
\textbf{AC સર્કિટમાં પાવર સંબંધિત પદો}:

\begin{tabulary}{\linewidth}{|L|L|}
\hline
\textbf{પદ} & \textbf{વ્યાખ્યા} \\ \hline
\textbf{એક્ટિવ પાવર (P)} & વાસ્તવિક વપરાયેલી પાવર, વોટ (W)માં માપવામાં આવે છે; $P = VI \cos\theta$. \\ \hline
\textbf{રિએક્ટિવ પાવર (Q)} & સ્ત્રોત અને લોડ વચ્ચે આગળ-પાછળ થતી પાવર, VAR માં માપવામાં આવે છે; $Q = VI \sin\theta$. \\ \hline
\textbf{પાવર ફેક્ટર (PF)} & એક્ટિવ પાવરનો એપરન્ટ પાવર સાથેનો ગુણોત્તર; $PF = \cos\theta$. \\ \hline
\end{tabulary}

\textbf{પાવર ત્રિકોણ:}
\begin{center}
\begin{tikzpicture}
    \draw (0,0) -- (4,0) node[midway, below] {$P$ (Watts)};
    \draw (4,0) -- (4,3) node[midway, right] {$Q$ (VAR)};
    \draw (0,0) -- (4,3) node[midway, above left] {$S$ (VA)};
    \draw (0.5,0) arc (0:36.87:0.5);
    \node at (0.8, 0.3) {$\theta$};
\end{tikzpicture}
\captionof{figure}{પાવર ત્રિકોણ}
\end{center}

\begin{itemize}
    \item \textbf{એપરન્ટ પાવર (S)}: એક્ટિવ અને રિએક્ટિવ પાવરનો વેક્ટર સરવાળો.
    \item \textbf{પાવર ફેક્ટર}: $\cos \theta = P/S$ (0 થી 1).
\end{itemize}
\end{solutionbox}

\begin{mnemonicbox}
\mnemonic{Active Power Works, Reactive Power Waits}
\end{mnemonicbox}

\questionmarks{2(c) OR}{7}{ટ્રાઇવેલેન્ટ, ટેટ્રાવેલેન્ટ અને પેન્ટાવેલેન્ટ તત્વોના અણુની રચના સમજાવો.}

\begin{solutionbox}
\textbf{અણુ રચના:}

\begin{tabulary}{\linewidth}{|L|L|L|L|}
\hline
\textbf{તત્વનો પ્રકાર} & \textbf{વેલેન્સ ઇલેક્ટ્રોન} & \textbf{ઉદાહરણ} & \textbf{ઇલેક્ટ્રોનિક કોન્ફિગરેશન} \\ \hline
\textbf{ટ્રાઇવેલેન્ટ} & 3 & બોરોન, એલ્યુમિનિયમ, ગેલિયમ & સૌથી બહારના શેલમાં 3 ઇલેક્ટ્રોન \\ \hline
\textbf{ટેટ્રાવેલેન્ટ} & 4 & કાર્બન, સિલિકોન, જર્મેનિયમ & સૌથી બહારના શેલમાં 4 ઇલેક્ટ્રોન \\ \hline
\textbf{પેન્ટાવેલેન્ટ} & 5 & નાઇટ્રોજન, ફોસ્ફરસ, આર્સેનિક & સૌથી બહારના શેલમાં 5 ઇલેક્ટ્રોન \\ \hline
\end{tabulary}

\vspace{0.5cm}

\begin{center}
\begin{tikzpicture}[scale=0.7]
    % Trivalent
    \node[circle, draw] (n1) at (0,0) {+3};
    \draw (n1) circle (1.5cm);
    \foreach \a in {90, 210, 330}
        \filldraw[black] (\a:1.5) circle (2pt);
    \node[below] at (0,-2) {ટ્રાઇવેલેન્ટ (3 $e^-$)};

    % Tetravalent
    \node[circle, draw] (n2) at (5,0) {+4};
    \draw (n2) circle (1.5cm);
    \foreach \a in {45, 135, 225, 315}
        \filldraw[black] (5,0) +(\a:1.5) circle (2pt);
    \node[below] at (5,-2) {ટેટ્રાવેલેન્ટ (4 $e^-$)};

    % Pentavalent
    \node[circle, draw] (n3) at (10,0) {+5};
    \draw (n3) circle (1.5cm);
    \foreach \a in {18, 90, 162, 234, 306}
        \filldraw[black] (10,0) +(\a:1.5) circle (2pt);
    \node[below] at (10,-2) {પેન્ટાવેલેન્ટ (5 $e^-$)};
\end{tikzpicture}
\captionof{figure}{વેલેન્સ શેલ ઇલેક્ટ્રોન્સ}
\end{center}

\begin{itemize}
    \item \textbf{ટ્રાઇવેલેન્ટ તત્વો}: અર્ધવાહકોમાં p-ટાઇપ ડોપન્ટ્સ તરીકે વપરાય છે.
    \item \textbf{ટેટ્રાવેલેન્ટ તત્વો}: અર્ધવાહક બેઝ મટિરિયલ્સ બનાવે છે.
    \item \textbf{પેન્ટાવેલેન્ટ તત્વો}: અર્ધવાહકોમાં n-ટાઇપ ડોપન્ટ્સ તરીકે વપરાય છે.
\end{itemize}
\end{solutionbox}

\begin{mnemonicbox}
\mnemonic{Three Tries to Bond, Four Forms Full bonds, Five Frees an Electron}
\end{mnemonicbox}

\questionmarks{3(a)}{3}{ફોટોડીઓડનું પ્રતીક દોરો અને તેનો ઉપયોગ જણાવો.}

\begin{solutionbox}
\textbf{ફોટોડાયોડ પ્રતીક:}
\begin{center}
\begin{circuitikz}
    \draw (0,0) to[photodiode, l=Photodiode] (2,0);
\end{circuitikz}
\end{center}

\textbf{ફોટોડાયોડના ઉપયોગો:}
\begin{itemize}
    \item લાઇટ સેન્સર અને ડિટેક્ટર.
    \item ઓપ્ટિકલ કોમ્યુનિકેશન સિસ્ટમ્સ.
    \item કેમેરા એક્સપોઝર કંટ્રોલ.
    \item બારકોડ સ્કેનર.
    \item મેડિકલ ઇન્સ્ટ્રુમેન્ટ્સ.
    \item સોલાર સેલ.
\end{itemize}
\end{solutionbox}

\begin{mnemonicbox}
\mnemonic{Photons Produce Current}
\end{mnemonicbox}

\questionmarks{3(b)}{4}{LED પર ટૂંકી નોંધ લખો.}

\begin{solutionbox}
\textbf{LED (લાઇટ એમિટિંગ ડાયોડ)}:

\begin{tabulary}{\linewidth}{|L|L|}
\hline
\textbf{પેરામીટર} & \textbf{વર્ણન} \\ \hline
\textbf{બંધારણ} & વિશેષ ડોપિંગ મટિરિયલ્સ સાથે p-n જંક્શન. \\ \hline
\textbf{કાર્યપદ્ધતિ} & ઇલેક્ટ્રોન હોલ્સ સાથે રિકોમ્બાઇન થઈને ફોટોન્સ રૂપે ઊર્જા છોડે છે. \\ \hline
\textbf{મટિરિયલ્સ} & GaAs (લાલ), GaP (લીલો), GaN (વાદળી), વગેરે. \\ \hline
\textbf{વોલ્ટેજ} & ફોરવર્ડ વોલ્ટેજ સામાન્ય રીતે 1.8V થી 3.3V (રંગ પર આધારિત). \\ \hline
\end{tabulary}

\textbf{ફાયદાઓ}:
\begin{itemize}
    \item ઉચ્ચ કાર્યક્ષમતા (ઓછી પાવર વપરાશ).
    \item લાંબી લાઇફ (50,000+ કલાક).
    \item નાનું કદ અને મજબૂતાઈ.
    \item વિવિધ રંગો ઉપલબ્ધ.
\end{itemize}

\textbf{ઉપયોગો}:
\begin{itemize}
    \item ઇન્ડિકેટર અને ડિસ્પ્લે.
    \item લાઇટિંગ સિસ્ટમ્સ.
    \item TV/મોનિટર બેકલાઇટ્સ.
    \item ટ્રાફિક સિગ્નલ.
\end{itemize}
\end{solutionbox}

\begin{mnemonicbox}
\mnemonic{Light Emits when Diode conducts}
\end{mnemonicbox}

\questionmarks{3(c)}{7}{PN જંક્શન ડાયોડની લાક્ષણિકતા દોરીને સમજાવો.}

\begin{solutionbox}
\textbf{P-N જંક્શન ડાયોડની V-I લાક્ષણિકતા:}

\begin{center}
\begin{tikzpicture}[scale=0.8]
    \draw[->] (-3,0) -- (3,0) node[right] {$V$};
    \draw[->] (0,-2) -- (0,3) node[above] {$I$};
    \draw[blue, thick] (0,0) -- (0.5,0) .. controls (0.7,0.1) and (0.8,1) .. (1,3) node[right] {Forward};
    \draw[red, thick] (0,0) -- (-1,0) -- (-2.5,-0.1) -- (-2.5,-2) node[below] {Breakdown};
    
    \node at (1.5,1) {Knee Voltage};
    \node at (-1.5,-0.5) {Leakage Current};
    \node at (2.5,2.5) {mA};
    \node at (-2.5,-2.5) {$\mu$A};
\end{tikzpicture}
\captionof{figure}{V-I લાક્ષણિકતાઓ}
\end{center}

\textbf{ફોરવર્ડ બાયસ રીજન:}
\begin{itemize}
    \item \textbf{ની વોલ્ટેજ}: 0.3V (Ge), 0.7V (Si) જ્યાં કરંટ વહેવાનું શરૂ થાય છે.
    \item \textbf{કરંટ સમીકરણ}: $I = I_s(e^{qV/kT} - 1)$.
    \item \textbf{વાહકતા}: ઉચ્ચ (ઓછો અવરોધ).
\end{itemize}

\textbf{રિવર્સ બાયસ રીજન:}
\begin{itemize}
    \item \textbf{લીકેજ કરંટ}: ખૂબ જ નાનો રિવર્સ કરંટ (માઇક્રો-એમ્પિયર).
    \item \textbf{બ્રેકડાઉન રીજન}: બ્રેકડાઉન વોલ્ટેજ પર કરંટનો તીવ્ર વધારો.
    \item \textbf{વાહકતા}: ખૂબ ઓછી (ઉચ્ચ અવરોધ).
\end{itemize}

\textbf{મુખ્ય પોઇન્ટ્સ}:
\begin{itemize}
    \item \textbf{બેરિયર પોટેન્શિયલ}: ફોરવર્ડ બાયસમાં ઘટે છે, રિવર્સ બાયસમાં વધે છે.
    \item \textbf{ડાયોડ રેઝિસ્ટન્સ}: ડાયનેમિક રેઝિસ્ટન્સ એપ્લાઇડ વોલ્ટેજ સાથે બદલાય છે.
    \item \textbf{તાપમાન અસર}: તાપમાન વધવાથી વોલ્ટેજ ડ્રોપ ઘટે છે.
\end{itemize}
\end{solutionbox}

\begin{mnemonicbox}
\mnemonic{Forward Flows Freely, Reverse Resists}
\end{mnemonicbox}

\questionmarks{3(a) OR}{3}{PN જંક્શન ડાયોડના ઉપયોગોની યાદી બનાવો.}

\begin{solutionbox}
\textbf{PN જંક્શન ડાયોડના ઉપયોગો:}

\begin{tabulary}{\linewidth}{|L|L|}
\hline
\textbf{ઉપયોગ કેટેગરી} & \textbf{ઉદાહરણો} \\ \hline
\textbf{રેક્ટિફિકેશન} & હાફ-વેવ રેક્ટિફાયર, ફુલ-વેવ રેક્ટિફાયર, બ્રિજ રેક્ટિફાયર. \\ \hline
\textbf{સિગ્નલ પ્રોસેસિંગ} & સિગ્નલ ડિમોડ્યુલેશન, ક્લિપિંગ સર્કિટ્સ, ક્લેમ્પિંગ સર્કિટ્સ. \\ \hline
\textbf{પ્રોટેક્શન} & વોલ્ટેજ સ્પાઇક પ્રોટેક્શન, રિવર્સ પોલારિટી પ્રોટેક્શન. \\ \hline
\textbf{લોજિક ગેટ્સ} & ડાયોડ લોજિક સર્કિટ્સ, સ્વિચિંગ એપ્લિકેશન્સ. \\ \hline
\textbf{વોલ્ટેજ રેગ્યુલેશન} & ઝેનર ડાયોડ વોલ્ટેજ રેફરન્સિસ. \\ \hline
\textbf{લાઇટ એપ્લિકેશન્સ} & LEDs, ફોટોડાયોડ, સોલાર સેલ. \\ \hline
\end{tabulary}
\end{solutionbox}

\begin{mnemonicbox}
\mnemonic{Rectify, Process, Protect, Logic, Regulate, Light}
\end{mnemonicbox}

\questionmarks{3(b) OR}{4}{અનબાયસ PN જંક્શન ડાયોડ ના ડિપ્લીશન રીજીયન ની રચના સમજાવો.}

\begin{solutionbox}
\textbf{ડિપ્લીશન રીજન ફોર્મેશન:}

\begin{center}
\begin{tikzpicture}[scale=0.8]
    \draw (0,0) rectangle (6,3);
    \draw (3,0) -- (3,3); % Junction
    \node at (1.5,2.5) {P-Type};
    \node at (4.5,2.5) {N-Type};
    
    % Charge carriers
    \foreach \x in {0.5, 1, 1.5, 2}
        \node at (\x, 1.5) {h+};
    \foreach \x in {4, 4.5, 5, 5.5}
        \node at (\x, 1.5) {e-};
        
    % Depletion zone ions
    \draw[fill=gray!20] (2.5,0) rectangle (3.5,3);
    \node at (2.7,1.5) {-};
    \node at (3.3,1.5) {+};
    
    \node[below] at (3,0) {ડિપ્લીશન રીજન};
    \draw[->] (2.2,0.5) -- (3.8,0.5) node[right] {E-field};
\end{tikzpicture}
\captionof{figure}{ડિપ્લીશન રીજન}
\end{center}

\textbf{પ્રક્રિયા:}
\begin{itemize}
    \item \textbf{ડિફ્યુઝન}: n-સાઇડમાંથી ઇલેક્ટ્રોન p-સાઇડ તરફ ડિફ્યુઝ થાય છે; p-સાઇડમાંથી હોલ્સ n-સાઇડ તરફ ડિફ્યુઝ થાય છે.
    \item \textbf{રિકોમ્બિનેશન}: ઇલેક્ટ્રોન અને હોલ્સ જંક્શન પર રિકોમ્બાઇન થાય છે.
    \item \textbf{ઇમોબાઇલ આયન્સ}: n-રિજનમાં એક્સપોઝ્ડ પોઝિટિવ આયન્સ, p-રિજનમાં નેગેટિવ આયન્સ.
    \item \textbf{ઇલેક્ટ્રિક ફિલ્ડ}: પોઝિટિવ અને નેગેટિવ આયન્સ વચ્ચે બને છે, જે વધુ ડિફ્યુઝનનો વિરોધ કરે છે.
    \item \textbf{ઇક્વિલિબ્રિયમ}: ડિફ્યુઝન કરંટ ડ્રિફ્ટ કરંટ જેટલો થાય છે; કોઈ નેટ કરંટ વહેતો નથી.
\end{itemize}

\textbf{ડિપ્લીશન રીજનના ગુણધર્મો:}
\begin{itemize}
    \item ફ્રી ચાર્જ કેરિયર નથી.
    \item અવાહક તરીકે કામ કરે છે.
    \item પહોળાઈ ડોપિંગ લેવલ પર આધાર રાખે છે.
    \item બિલ્ટ-ઇન પોટેન્શિયલ બેરિયર ધરાવે છે.
\end{itemize}
\end{solutionbox}

\begin{mnemonicbox}
\mnemonic{Diffusion Depletes Carriers, Creating Electric barrier}
\end{mnemonicbox}

\questionmarks{3(c) OR}{7}{PN જંક્શન ડાયોડનું બાંધકામ, કાર્ય અને એપ્લિકેશન સમજાવો.}

\begin{solutionbox}
\textbf{PN જંક્શન ડાયોડનું બાંધકામ:}

\begin{center}
\begin{tikzpicture}
    \draw[fill=blue!10] (0,0) rectangle (2,2) node[midway] {P-Type};
    \draw[fill=red!10] (2,0) rectangle (4,2) node[midway] {N-Type};
    \draw (0,1) -- (-1,1) node[left] {Anode (+)};
    \draw (4,1) -- (5,1) node[right] {Cathode (-)};
    \draw[pattern=north east lines] (1.8,0) rectangle (2.2,2);
    \node[below] at (2,0) {Junction};
\end{tikzpicture}
\captionof{figure}{PN જંક્શન બાંધકામ}
\end{center}

\begin{itemize}
    \item \textbf{P-Type રીજન}: ટ્રાઇવેલેન્ટ અશુદ્ધિઓ (બોરોન, એલ્યુમિનિયમ) સાથે ડોપ કરેલ સિલિકોન/જર્મેનિયમ.
    \item \textbf{N-Type રીજન}: પેન્ટાવેલેન્ટ અશુદ્ધિઓ (ફોસ્ફરસ, આર્સેનિક) સાથે ડોપ કરેલ સિલિકોન/જર્મેનિયમ.
    \item \textbf{જંક્શન}: ડિપ્લીશન લેયર સાથે p અને n રીજન વચ્ચેનું ઇન્ટરફેસ.
    \item \textbf{ટર્મિનલ્સ}: એનોડ (p-સાઇડ) અને કેથોડ (n-સાઇડ).
\end{itemize}

\textbf{કાર્યપદ્ધતિ:}
\begin{tabulary}{\linewidth}{|L|L|}
\hline
\textbf{બાયસ કન્ડિશન} & \textbf{વર્તન} \\ \hline
\textbf{ફોરવર્ડ બાયસ} & ડિપ્લીશન રીજન સાંકડી થાય છે, $V > 0.7V$ (Si) થાય ત્યારે કરંટ વહે છે. \\ \hline
\textbf{રિવર્સ બાયસ} & ડિપ્લીશન રીજન પહોળી થાય છે, માત્ર નાનો લીકેજ કરંટ વહે છે. \\ \hline
\end{tabulary}

\textbf{ઉપયોગો:}
\begin{itemize}
    \item પાવર સપ્લાયમાં રેક્ટિફિકેશન.
    \item રેડિયોમાં સિગ્નલ ડિમોડ્યુલેશન.
    \item વોલ્ટેજ રેગ્યુલેશન (ઝેનર).
    \item સિગ્નલ ક્લિપિંગ અને ક્લેમ્પિંગ.
    \item લોજિક ગેટ્સ અને સ્વિચિંગ.
    \item લાઇટ એમિશન અને ડિટેક્શન.
\end{itemize}
\end{solutionbox}

\begin{mnemonicbox}
\mnemonic{Forward Flow, Reverse Restrict, Convert AC to DC}
\end{mnemonicbox}


\questionmarks{4(a)}{3}{વ્યાખ્યા આપો: (1) રિપલ ફ્રિકવન્સી (2) રિપલ ફેક્ટર (3) ડાયોડની PIV.}

\begin{solutionbox}
\begin{tabulary}{\linewidth}{|L|L|}
\hline
\textbf{પદ} & \textbf{વ્યાખ્યા} \\ \hline
\textbf{રિપલ ફ્રિકવન્સી} & રેક્ટિફાઇડ DC આઉટપુટમાં હાજર AC ઘટકની આવૃત્તિ; હાફ-વેવ માટે $f = f_{in}$, ફુલ-વેવ માટે $f = 2f_{in}$. \\ \hline
\textbf{રિપલ ફેક્ટર ($\gamma$)} & રેક્ટિફાયર આઉટપુટમાં AC ઘટકના RMS મૂલ્ય અને DC ઘટકના ગુણોત્તરને રિપલ ફેક્ટર કહે છે; $\gamma = V_{ac(rms)}/V_{dc}$. \\ \hline
\textbf{ડાયોડની PIV} & પીક ઇન્વર્સ વોલ્ટેજ - મહત્તમ રિવર્સ વોલ્ટેજ જે ડાયોડ બ્રેકડાઉન વગર સહન કરી શકે છે. \\ \hline
\end{tabulary}
\end{solutionbox}

\begin{mnemonicbox}
\mnemonic{Ripples Per second, Ripple Proportion, Reverse Peak Voltage}
\end{mnemonicbox}

\questionmarks{4(b)}{4}{બે ડાયોડ સાથેના ફુલ વેવ રેક્ટિફાયર અને ફુલ વેવ બ્રિજ રેક્ટિફાયર વચ્ચેનો તફાવત આપો.}

\begin{solutionbox}
\begin{tabulary}{\linewidth}{|L|L|L|}
\hline
\textbf{પેરામીટર} & \textbf{સેન્ટર-ટેપ્ડ ફુલ વેવ} & \textbf{બ્રિજ રેક્ટિફાયર} \\ \hline
\textbf{વપરાતા ડાયોડ} & 2 ડાયોડ & 4 ડાયોડ \\ \hline
\textbf{ટ્રાન્સફોર્મર} & સેન્ટર-ટેપ્ડ જરૂરી & સેન્ટર-ટેપની જરૂર નથી \\ \hline
\textbf{ડાયોડની PIV} & $2V_m$ & $V_m$ \\ \hline
\textbf{આઉટપુટ વોલ્ટેજ} & $V_{dc} = 0.637V_m$ & $V_{dc} = 0.637V_m$ \\ \hline
\textbf{રિપલ ફેક્ટર} & 0.48 & 0.48 \\ \hline
\textbf{કાર્યક્ષમતા} & 81.2\% & 81.2\% \\ \hline
\textbf{TUF} & 0.693 & 0.812 \\ \hline
\end{tabulary}
\end{solutionbox}

\begin{mnemonicbox}
\mnemonic{Bridge Beats Tap with Lower PIV but Needs More Diodes}
\end{mnemonicbox}

\questionmarks{4(c)}{7}{વોલ્ટેજ રેગ્યુલેટર તરીકે ઝેનર ડાયોડ સમજાવો.}

\begin{solutionbox}
\textbf{ઝેનર ડાયોડ વોલ્ટેજ રેગ્યુલેટર:}

\begin{center}
\begin{circuitikz}[scale=0.9]
    \draw (0,0) to[sV, l=$V_{in}$] (0,3) to[R, l=$R_S$] (3,3) -- (5,3);
    \draw (3,3) to[zD*, l=$D_Z$] (3,0);
    \draw (5,3) to[R, l=$R_L$] (5,0);
    \draw (0,0) -- (5,0);
    \node at (5.5, 1.5) {$V_{out} = V_Z$};
\end{circuitikz}
\captionof{figure}{ઝેનર વોલ્ટેજ રેગ્યુલેટર}
\end{center}

\textbf{કાર્યપદ્ધતિ:}
\begin{itemize}
    \item \textbf{રિવર્સ બાયસ્ડ}: ઝેનર બ્રેકડાઉન રીજનમાં કામ કરે છે.
    \item \textbf{અચળ વોલ્ટેજ}: તેના ટર્મિનલ્સ પર ચોક્કસ વોલ્ટેજ ($V_Z$) જાળવી રાખે છે.
    \item \textbf{કરંટ રેગ્યુલેશન}: સીરીઝ રેઝિસ્ટર ($R_S$) કરંટને મર્યાદિત કરે છે.
    \item \textbf{લોડ ફેરફાર}: જ્યારે લોડ કરંટ બદલાય છે, ત્યારે ઝેનર કરંટ બદલાય છે જેથી આઉટપુટ વોલ્ટેજ અચળ રહે.
\end{itemize}

\textbf{ડિઝાઇન સમીકરણો:}
\begin{itemize}
    \item $R_S = (V_{in} - V_Z) / (I_L + I_Z)$.
    \item ઝેનર પાવર રેટિંગ: $P_Z = V_Z \times I_{Z(max)}$.
\end{itemize}
\end{solutionbox}

\begin{mnemonicbox}
\mnemonic{Zener Stays at breakdown Voltage despite Current changes}
\end{mnemonicbox}

\questionmarks{4(a) OR}{3}{રેક્ટિફાયર એટલે શું? ફુલ વેવ રેક્ટિફાયર વેવફોર્મ્સ સાથે સમજાવો.}

\begin{solutionbox}
\textbf{રેક્ટિફાયર}: એક સર્કિટ જે AC વોલ્ટેજને પલ્સેટિંગ DC વોલ્ટેજમાં રૂપાંતરિત કરે છે, માત્ર એક જ દિશામાં કરંટ પ્રવાહને મંજૂરી આપીને.

\textbf{ફુલ વેવ રેક્ટિફાયર (સેન્ટર-ટેપ્ડ):}

\begin{center}
\begin{circuitikz}[scale=0.8]
    \draw (0,0) node[transformer core] (T) {};
    \draw (T.A1) -- ++(-1,0) node[left] {AC In};
    \draw (T.A2) -- ++(-1,0);
    \draw (T.B1) to[D*, l=$D_1$] (4,1);
    \draw (T.B2) to[D*, l=$D_2$] (4,-1);
    \draw (4,1) -- (4,-1);
    \draw (4,0) to[R, l=$R_L$] (6,0);
    \draw (T.base) -- (6,0) |- (6,0);
\end{circuitikz}
\end{center}

\textbf{વેવફોર્મ્સ:}
\begin{center}
\begin{tikzpicture}[scale=0.6]
    % Input
    \draw[->] (0,3) -- (6,3) node[right] {t};
    \draw[->] (0,3) -- (0,5) node[above] {$V_{in}$};
    \draw[blue] plot[domain=0:6] (\x, {3 + sin(\x r * 3)});
    
    % Output
    \draw[->] (0,0) -- (6,0) node[right] {t};
    \draw[->] (0,0) -- (0,2) node[above] {$V_{out}$};
    \draw[red, thick] plot[domain=0:6] (\x, {abs(sin(\x r * 3))});
\end{tikzpicture}
\captionof{figure}{ફુલ વેવ રેક્ટિફાયર વેવફોર્મ્સ}
\end{center}

\begin{itemize}
    \item \textbf{કાર્ય}: AC ઇનપુટના બંને હાફ સાયકલ સમાન પોલારિટીમાં રૂપાંતરિત થાય છે.
    \item \textbf{આવૃત્તિ}: આઉટપુટ રિપલ ફ્રિકવન્સી ઇનપુટ ફ્રિકવન્સી કરતાં બમણી હોય છે.
    \item \textbf{વોલ્ટેજ}: $V_{dc} = 0.637V_m$.
\end{itemize}
\end{solutionbox}

\begin{mnemonicbox}
\mnemonic{Full Wave Forms Full Output}
\end{mnemonicbox}

\questionmarks{4(b) OR}{4}{રેક્ટિફાયરમાં ફિલ્ટર શા માટે જરૂરી છે? વિવિધ પ્રકારના ફિલ્ટર જણાવો અને કોઈપણ એક સમજાવો.}

\begin{solutionbox}
\textbf{ફિલ્ટરની જરૂરિયાત}: રેક્ટિફાયર્સ મોટા રિપલ સાથે પલ્સેટિંગ DC પેદા કરે છે; ફિલ્ટર્સ આ આઉટપુટને સ્મૂધ બનાવીને સ્ટેડી DC વોલ્ટેજ આપે છે.

\textbf{ફિલ્ટરના પ્રકારો:}
\begin{itemize}
    \item કેપેસિટર (C) ફિલ્ટર.
    \item ઇન્ડક્ટર (L) ફિલ્ટર.
    \item LC ફિલ્ટર.
    \item $\pi$ (Pi) ફિલ્ટર.
    \item RC ફિલ્ટર.
\end{itemize}

\textbf{કેપેસિટર ફિલ્ટર:}
\begin{center}
\begin{circuitikz}
    \draw (0,0) to[short, o-o] (0,2); % Input
    \draw (0,2) -- (2,2) to[C, l=C] (2,0) -- (0,0);
    \draw (2,2) -- (4,2) to[R, l=$R_L$] (4,0) -- (2,0);
    \node at (-0.5,1) {Rectifier Out};
    \node at (5,1) {DC Out};
\end{circuitikz}
\captionof{figure}{કેપેસિટર ફિલ્ટર}
\end{center}

\textbf{કાર્યપદ્ધતિ:}
\begin{itemize}
    \item વોલ્ટેજ પીક સુધી વધે ત્યારે કેપેસિટર ચાર્જ થાય છે.
    \item વોલ્ટેજ ઘટે ત્યારે લોડ દ્વારા કેપેસિટર ધીમે ધીમે ડિસ્ચાર્જ થાય છે.
    \item $RC$ ટાઈમ કોન્સ્ટન્ટ સાથે ડિસ્ચાર્જ પાથ પૂરો પાડીને રિપલ ઘટાડે છે.
\end{itemize}
\end{solutionbox}

\begin{mnemonicbox}
\mnemonic{Capacitor Catches Charge and Releases Slowly}
\end{mnemonicbox}

\questionmarks{4(c) OR}{7}{રેક્ટિફાયરની જરૂરિયાત લખો. બ્રિજ રેક્ટિફાયર સર્કિટ અને વેવફોર્મ્સ સાથે સમજાવો.}

\begin{solutionbox}
\textbf{રેક્ટિફાયરની જરૂરિયાત:}
\begin{itemize}
    \item ઇલેક્ટ્રોનિક ઉપકરણો માટે AC ને DC માં ફેરવવા.
    \item પાવર સપ્લાય અને બેટરી ચાર્જિંગ.
    \item સિગ્નલ ડિમોડ્યુલેશન.
\end{itemize}

\textbf{બ્રિજ રેક્ટિફાયર સર્કિટ:}
\begin{center}
\begin{circuitikz}[scale=0.8]
    \draw (0,0) to[sV, l=$V_{in}$] (0,3);
    \draw (0,3) -- (2,3) -- (3,2);
    \draw (0,0) -- (2,0) -- (5,0) -- (5,2);
    
    % Bridge
    \draw (3,2) to[D*, l=$D_1$] (4,3);
    \draw (4,3) to[D*, l=$D_2$] (5,2);
    \draw (5,2) to[D*, l=$D_3$] (4,1);
    \draw (4,1) to[D*, l=$D_4$] (3,2);
    
    % Load
    \draw (4,3) -- (4,4) -- (7,4) to[R, l=$R_L$] (7,0) -- (5,0);
    \draw (4,1) -- (4,0);
    \node at (7.5, 2) {$V_{out}$};
\end{circuitikz}
\captionof{figure}{બ્રિજ રેક્ટિફાયર}
\end{center}

\textbf{કાર્યપદ્ધતિ:}
\begin{itemize}
    \item \textbf{પોઝિટિવ હાફ સાયકલ}: $D_1$ અને $D_3$ કન્ડક્ટ થાય છે.
    \item \textbf{નેગેટિવ હાફ સાયકલ}: $D_2$ અને $D_4$ કન્ડક્ટ થાય છે.
    \item \textbf{પરિણામ}: $R_L$ માંથી એક જ દિશામાં પ્રવાહ વહે છે.
\end{itemize}

\textbf{વેવફોર્મ્સ:}
ઇનપુટ સાઈન વેવ છે, આઉટપુટ પલ્સેટિંગ DC (ફુલ-વેવ રેક્ટિફાઇડ) છે.
\end{solutionbox}

\begin{mnemonicbox}
\mnemonic{Bridge Brings Both halves to Direct Current}
\end{mnemonicbox}

\questionmarks{5(a)}{3}{ઇલેક્ટ્રોનિક કચરાના કારણો સમજાવો.}

\begin{solutionbox}
\textbf{ઇલેક્ટ્રોનિક કચરાના કારણો:}
\begin{tabulary}{\linewidth}{|L|L|}
\hline
\textbf{કારણ} & \textbf{વર્ણન} \\ \hline
\textbf{ઝડપી ટેકનોલોજી ફેરફાર} & ઇલેક્ટ્રોનિક્સમાં વારંવાર અપગ્રેડ અને જૂના થવાનું પ્રમાણ. \\ \hline
\textbf{ટૂંકી લાઈફસાયકલ} & મર્યાદિત ઉપયોગી જીવન સાથે ડિઝાઇન કરેલ ઉપકરણો. \\ \hline
\textbf{ગ્રાહક વર્તણૂક} & રિપેરિંગને બદલે નવા ગેજેટ્સની પસંદગી. \\ \hline
\textbf{ઉત્પાદન સમસ્યાઓ} & નબળી ગુણવત્તા જે વહેલા બગાડ તરફ દોરી જાય છે. \\ \hline
\textbf{માર્કેટિંગ વ્યૂહરચના} & પ્લાન્ડ ઓબ્સોલેસન્સ દ્વારા નવા મોડલ્સનું પ્રમોશન. \\ \hline
\end{tabulary}
\end{solutionbox}

\begin{mnemonicbox}
\mnemonic{Upgrade, Use, Throw, Repeat}
\end{mnemonicbox}

\questionmarks{5(b)}{4}{PNP અને NPN ટ્રાન્ઝિસ્ટરની તુલના કરો.}

\begin{solutionbox}
\begin{tabulary}{\linewidth}{|L|L|L|}
\hline
\textbf{પેરામીટર} & \textbf{PNP ટ્રાન્ઝિસ્ટર} & \textbf{NPN ટ્રાન્ઝિસ્ટર} \\ \hline
\textbf{મેજોરિટી કેરિયર્સ} & હોલ્સ & ઇલેક્ટ્રોન્સ \\ \hline
\textbf{કરંટ પ્રવાહ} & એમિટરથી કલેક્ટર & કલેક્ટરથી એમિટર \\ \hline
\textbf{બાયસિંગ} & એમિટર +ve, કલેક્ટર -ve & કલેક્ટર +ve, એમિટર -ve \\ \hline
\textbf{સ્વિચિંગ સ્પીડ} & ધીમી & ઝડપી \\ \hline
\textbf{વપરાશ} & ઓછો સામાન્ય & વધુ સામાન્ય \\ \hline
\end{tabulary}
\end{solutionbox}

\begin{mnemonicbox}
\mnemonic{PNP: Positive-Negative-Positive; NPN: Negative-Positive-Negative}
\end{mnemonicbox}

\questionmarks{5(c)}{7}{MOSFET નું પ્રતીક દોરો, તેનું કામકાજ અને રચના સમજાવો.}

\begin{solutionbox}
\textbf{MOSFET (N- ચેનલ એન્હાન્સમેન્ટ):}

\begin{center}
\begin{circuitikz}
    \draw (0,0) node[nmos] (Q) {};
    \node[right] at (Q.D) {D};
    \node[right] at (Q.S) {S};
    \node[left] at (Q.G) {G};
\end{circuitikz}
\captionof{figure}{MOSFET પ્રતીક}
\end{center}

\textbf{રચના:}
\begin{center}
\begin{tikzpicture}[scale=0.8]
    \draw (0,0) rectangle (5,3);
    \node at (2.5,0.5) {P-Substrate};
    \draw[fill=white] (0.5,2) rectangle (1.5,3); \node at (1,2.5) {n+ Source};
    \draw[fill=white] (3.5,2) rectangle (4.5,3); \node at (4,2.5) {n+ Drain};
    \draw[fill=gray!30] (1.5,3) rectangle (3.5,3.2); \node[right] at (3.5,3.1) {SiO2};
    \draw[fill=black] (1.5,3.2) rectangle (3.5,3.4); \node[right] at (3.5,3.3) {Gate Metal};
\end{tikzpicture}
\captionof{figure}{MOSFET રચના}
\end{center}

\textbf{કાર્યપદ્ધતિ (એન્હાન્સમેન્ટ મોડ):}
\begin{itemize}
    \item શરૂઆતમાં કોઈ ચેનલ અસ્તિત્વમાં નથી.
    \item જ્યારે ગેટ પર પોઝિટિવ વોલ્ટેજ આપવામાં આવે ($V_{GS} > V_{Th}$), ત્યારે ઇલેક્ટ્રોન સપાટી પર આકર્ષાય છે.
    \item એક N-ચેનલ સોર્સ અને ડ્રેનને જોડે છે, જે કરંટ પ્રવાહને મંજૂરી આપે છે.
\end{itemize}
\end{solutionbox}

\begin{mnemonicbox}
\mnemonic{Gate Voltage Controls Electron Channel}
\end{mnemonicbox}

\questionmarks{5(a) OR}{3}{ઈલેક્ટ્રોનિક કચરાને હેન્ડલ કરવાની પદ્ધતિઓ સમજાવો.}

\begin{solutionbox}
\textbf{ઇલેક્ટ્રોનિક કચરા હેન્ડલિંગની પદ્ધતિઓ:}
\begin{itemize}
    \item \textbf{રિડ્યુસ}: લાંબી લાઇફસાયકલવાળા ઉત્પાદનોની ડિઝાઇન.
    \item \textbf{રિયુઝ}: વપરાયેલા ઇલેક્ટ્રોનિક્સનું રિફર્બિશિંગ.
    \item \textbf{રિસાયકલ}: સામગ્રી પુનઃપ્રાપ્ત કરવી.
    \item \textbf{રિકવર}: ઊર્જા/ધાતુઓ કાઢવી.
\end{itemize}

\begin{center}
\begin{tikzpicture}
    \node[draw, rounded corners] (A) at (0,0) {કલેક્શન};
    \node[draw, rounded corners] (B) at (3,0) {સોર્ટિંગ};
    \node[draw, rounded corners] (C) at (6,0) {રિસાયક્લિંગ};
    \draw[->] (A) -- (B);
    \draw[->] (B) -- (C);
\end{tikzpicture}
\end{center}
\end{solutionbox}

\begin{mnemonicbox}
\mnemonic{Reduce, Reuse, Recycle, Recover Resources}
\end{mnemonicbox}

\questionmarks{5(b) OR}{4}{αdc અને βdc વચ્ચેનો સંબંધ મેળવો.}

\begin{solutionbox}
\textbf{આપેલ:}
\begin{itemize}
    \item $\alpha_{dc} = I_C/I_E$
    \item $\beta_{dc} = I_C/I_B$
\end{itemize}

\textbf{ગણતરી:}
KCL મુજબ: $I_E = I_C + I_B$
$I_C$ વડે ભાગતા:
$$ \frac{I_E}{I_C} = 1 + \frac{I_B}{I_C} $$
$$ \frac{1}{\alpha} = 1 + \frac{1}{\beta} $$
$$ \frac{1}{\alpha} = \frac{\beta + 1}{\beta} $$
$$ \alpha = \frac{\beta}{1+\beta} $$

તેવી જ રીતે,
$$ \beta = \frac{\alpha}{1-\alpha} $$
\end{solutionbox}

\begin{mnemonicbox}
\mnemonic{Alpha approaches One as Beta approaches Infinity}
\end{mnemonicbox}

\questionmarks{5(c) OR}{7}{તેના ઇનપુટ અને આઉટપુટ લાક્ષણિકતાઓ સાથે CC ની રચના સમજાવો.}

\begin{solutionbox}
\textbf{કોમન કલેક્ટર (એમિટર ફોલોઅર):}
\begin{center}
\begin{circuitikz}
    \draw (0,0) node[npn] (Q) {};
    \draw (Q.C) -- ++(0,1) node[above] {$V_{CC}$};
    \draw (Q.B) to[R, l=$R_B$] (-2,0) to[sV, l=$V_{in}$] (-2,-2) -- (0,-2) -- (Q.E);
    \draw (Q.E) to[R, l=$R_E$] (0,-2) node[ground] {};
    \draw (Q.E) -- ++(2,0) to[short, -o] ++(0,0) node[right] {$V_{out}$};
\end{circuitikz}
\captionof{figure}{કોમન કલેક્ટર સર્કિટ}
\end{center}

\textbf{લાક્ષણિકતાઓ:}
\begin{itemize}
    \item \textbf{ઇનપુટ}: $I_B$ વિ $V_{BC}$ નો આલેખ. ઉચ્ચ ઇનપુટ ઇમ્પીડન્સ.
    \item \textbf{આઉટપુટ}: $I_E$ વિ $V_{CE}$ નો આલેખ. નીચું આઉટપુટ ઇમ્પીડન્સ.
    \item \textbf{વોલ્ટેજ ગેઇન}: $\approx 1$.
    \item \textbf{કરંટ ગેઇન}: ઉચ્ચ ($\beta + 1$).
\end{itemize}
\end{solutionbox}

\begin{mnemonicbox}
\mnemonic{Collector Common, Current amplifies, Voltage follows}
\end{mnemonicbox}

\end{document}
