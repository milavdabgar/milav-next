\documentclass[10pt,a4paper]{article}

% content/resources/templates/preamble.tex
\usepackage[margin=0.6in]{geometry}
\author{Milav Dabgar}
\usepackage{amsmath,amssymb,amsthm}
\usepackage{booktabs}
\usepackage{multirow}
\usepackage{xcolor}
\usepackage{tcolorbox}
\tcbuselibrary{breakable,skins}
\usepackage[colorlinks=true,linkcolor=blue]{hyperref}
\usepackage{titlesec}
\usepackage{enumitem}
\usepackage{tikz}
\usepackage{pgfplots}
\usepackage{circuitikz}
\usepackage[version=4]{mhchem}
\usepackage{longtable}
\usepackage{array}
\usepackage{float}
\usepackage{caption}
\usepackage{listings}

\lstset{
  basicstyle=\small\ttfamily,
  breaklines=true,
  breakatwhitespace=false,
  postbreak=\mbox{\textcolor{red}{$\hookrightarrow$}\space},
  float=false,
  numbers=left,
  numberstyle=\tiny\color{gray},
  numbersep=10pt,
  xleftmargin=2em,
  keywordstyle=\color{blue},
  commentstyle=\color{green!60!black},
  stringstyle=\color{purple},
  backgroundcolor=\color{gray!5},
  showstringspaces=false,
  tabsize=2,
  captionpos=b,
  keepspaces=true,
  columns=flexible
}

\pgfplotsset{compat=1.18}
\usetikzlibrary{shapes,arrows,positioning,calc,patterns,decorations.pathmorphing,decorations.markings,arrows.meta}

% Color scheme
\definecolor{headcolor}{RGB}{0,102,204}
\definecolor{keycolor}{RGB}{220,20,60}
\definecolor{solutioncolor}{RGB}{34,139,34}
\definecolor{mnemoniccolor}{RGB}{148,0,211}
\definecolor{codecolor}{RGB}{0,0,100}

% Spacing
\setlength{\parskip}{3pt}
\setlist[itemize]{nosep}
\setlist[enumerate]{nosep}

% Title formatting
\titleformat{\section}{\Large\bfseries\color{headcolor}}{\thesection}{1em}{}
\titleformat{\subsection}{\large\bfseries\color{headcolor}}{\thesubsection}{1em}{}

% Pandoc tightlist compatibility
\providecommand{\tightlist}{%
  \setlength{\itemsep}{0pt}\setlength{\parskip}{0pt}}

% Pandoc longtable compatibility
\newcounter{none}
\def\thenone{}


% content/resources/templates/english-boxes.tex
% This file is currently empty - it exists to maintain consistency with the import structure.
% Add custom environments here if needed in the future.


\begin{document}

\begin{center}
{\Huge\bfseries\color{headcolor} Subject Name Solutions}\\[5pt]
{\LARGE 4353204 -- Summer 2025}\\[3pt]
{\large Semester 1 Study Material}\\[3pt]
{\normalsize\textit{Detailed Solutions and Explanations}}
\end{center}

\vspace{10pt}

\subsection*{Question 1(a) [3 marks]}\label{q1a}

\textbf{Describe CIA triad with example.}

\begin{solutionbox}

\textbf{CIA Triad Components:}

\begin{figure}
\centering
\pandocbounded{\includesvg[keepaspectratio]{diagrams/cia-triad.svg}}
\caption{CIA Triad}
\end{figure}

{\def\LTcaptype{none} % do not increment counter
\begin{longtable}[]{@{}
  >{\raggedright\arraybackslash}p{(\linewidth - 4\tabcolsep) * \real{0.3438}}
  >{\raggedright\arraybackslash}p{(\linewidth - 4\tabcolsep) * \real{0.3750}}
  >{\raggedright\arraybackslash}p{(\linewidth - 4\tabcolsep) * \real{0.2812}}@{}}
\toprule\noalign{}
\begin{minipage}[b]{\linewidth}\raggedright
Component
\end{minipage} & \begin{minipage}[b]{\linewidth}\raggedright
Definition
\end{minipage} & \begin{minipage}[b]{\linewidth}\raggedright
Example
\end{minipage} \\
\midrule\noalign{}
\endhead
\bottomrule\noalign{}
\endlastfoot
\textbf{Confidentiality} & Protecting data from unauthorized access &
Password protection on bank accounts \\
\textbf{Integrity} & Ensuring data accuracy and completeness & Digital
signatures on documents \\
\textbf{Availability} & Ensuring systems are accessible when needed &
24/7 online banking services \\
\end{longtable}
}

\begin{itemize}
\tightlist
\item
  \textbf{Confidentiality}: Only authorized users can access sensitive
  information
\item
  \textbf{Integrity}: Data remains accurate and unaltered during
  transmission
\item
  \textbf{Availability}: Systems remain operational and accessible to
  legitimate users
\end{itemize}

\end{solutionbox}
\begin{mnemonicbox}
``CIA Keeps Information Safe''

\end{mnemonicbox}
\subsection*{Question 1(b) [4 marks]}\label{q1b}

\textbf{Explain Public key and Private Key cryptography.}

\begin{solutionbox}

\textbf{Public Key Cryptography (Asymmetric):}

\begin{figure}
\centering
\pandocbounded{\includesvg[keepaspectratio]{diagrams/public-key-cryptography.svg}}
\caption{Public Key Cryptography}
\end{figure}

\textbf{Key Characteristics:}

{\def\LTcaptype{none} % do not increment counter
\begin{longtable}[]{@{}lll@{}}
\toprule\noalign{}
Feature & Public Key & Private Key \\
\midrule\noalign{}
\endhead
\bottomrule\noalign{}
\endlastfoot
\textbf{Distribution} & Freely shared & Kept secret \\
\textbf{Usage} & Encryption/Verification & Decryption/Signing \\
\textbf{Security} & Can be public & Must be protected \\
\end{longtable}
}

\begin{itemize}
\tightlist
\item
  \textbf{Public Key}: Used for encryption and signature verification
\item
  \textbf{Private Key}: Used for decryption and digital signing
\item
  \textbf{Security}: Based on mathematical complexity (RSA, ECC
  algorithms)
\end{itemize}

\end{solutionbox}
\begin{mnemonicbox}
``Public Encrypts, Private Decrypts''

\end{mnemonicbox}
\subsection*{Question 1(c) [7 marks]}\label{q1c}

\textbf{Explain various security attacks, mechanisms, and services
associated with each layer of the OSI model.}

\begin{solutionbox}

\textbf{OSI Security Framework:}

\begin{figure}
\centering
\pandocbounded{\includesvg[keepaspectratio]{diagrams/osi-security-framework.svg}}
\caption{OSI Security Framework}
\end{figure}

{\def\LTcaptype{none} % do not increment counter
\begin{longtable}[]{@{}
  >{\raggedright\arraybackslash}p{(\linewidth - 6\tabcolsep) * \real{0.1842}}
  >{\raggedright\arraybackslash}p{(\linewidth - 6\tabcolsep) * \real{0.2368}}
  >{\raggedright\arraybackslash}p{(\linewidth - 6\tabcolsep) * \real{0.3158}}
  >{\raggedright\arraybackslash}p{(\linewidth - 6\tabcolsep) * \real{0.2632}}@{}}
\toprule\noalign{}
\begin{minipage}[b]{\linewidth}\raggedright
Layer
\end{minipage} & \begin{minipage}[b]{\linewidth}\raggedright
Attacks
\end{minipage} & \begin{minipage}[b]{\linewidth}\raggedright
Mechanisms
\end{minipage} & \begin{minipage}[b]{\linewidth}\raggedright
Services
\end{minipage} \\
\midrule\noalign{}
\endhead
\bottomrule\noalign{}
\endlastfoot
\textbf{Physical} & Wiretapping, Jamming & Physical security, Shielding
& Access control \\
\textbf{Data Link} & MAC flooding, ARP poisoning & Encryption,
Authentication & Frame integrity \\
\textbf{Network} & IP spoofing, Routing attacks & IPSec, Firewalls &
Packet filtering \\
\textbf{Transport} & Session hijacking, SYN flooding & SSL/TLS, Port
security & End-to-end security \\
\textbf{Session} & Session replay, Hijacking & Session tokens, Timeouts
& Session management \\
\textbf{Presentation} & Data corruption, Format attacks & Encryption,
Compression & Data transformation \\
\textbf{Application} & Malware, Social engineering & Antivirus, User
training & Application security \\
\end{longtable}
}

\textbf{Key Security Services:}

\begin{itemize}
\tightlist
\item
  \textbf{Authentication}: Verifying user identity
\item
  \textbf{Authorization}: Controlling access permissions
\item
  \textbf{Non-repudiation}: Preventing denial of actions
\item
  \textbf{Data integrity}: Ensuring data accuracy
\end{itemize}

\end{solutionbox}
\begin{mnemonicbox}
``All People Seem To Need Data Protection''

\end{mnemonicbox}
\subsection*{Question 1(c OR) [7
marks]}\label{question-1c-or-7-marks}

\textbf{Explain MD5 hashing and Secure Hash Function (SHA) algorithms.}

\begin{solutionbox}

\textbf{Hash Function Comparison:}

\begin{figure}
\centering
\pandocbounded{\includesvg[keepaspectratio]{diagrams/hash-function-process.svg}}
\caption{Hash Function Process}
\end{figure}

{\def\LTcaptype{none} % do not increment counter
\begin{longtable}[]{@{}llll@{}}
\toprule\noalign{}
Feature & MD5 & SHA-1 & SHA-256 \\
\midrule\noalign{}
\endhead
\bottomrule\noalign{}
\endlastfoot
\textbf{Output Size} & 128 bits & 160 bits & 256 bits \\
\textbf{Security Level} & Weak & Weak & Strong \\
\textbf{Speed} & Fast & Moderate & Slower \\
\textbf{Current Status} & Deprecated & Deprecated & Recommended \\
\end{longtable}
}

\includegraphics[width=1\linewidth,height=\textheight,keepaspectratio]{mermaid-a81a0eff.pdf}

\textbf{Hash Properties:}

\begin{itemize}
\tightlist
\item
  \textbf{Deterministic}: Same input produces same hash
\item
  \textbf{Avalanche Effect}: Small input change causes major hash change
\item
  \textbf{One-way Function}: Cannot reverse hash to original data
\item
  \textbf{Collision Resistant}: Difficult to find two inputs with same
  hash
\end{itemize}

\textbf{Applications:}

\begin{itemize}
\tightlist
\item
  Password storage and verification
\item
  Digital signatures and certificates
\item
  Data integrity verification
\end{itemize}

\end{solutionbox}
\begin{mnemonicbox}
``Hash Always Produces Same Output''

\end{mnemonicbox}
\subsection*{Question 2(a) [3 marks]}\label{q2a}

\textbf{What is firewall? List out types of firewall.}

\begin{solutionbox}

\textbf{Firewall Definition:} Network security device that monitors and
controls incoming/outgoing traffic based on security rules.

\begin{figure}
\centering
\pandocbounded{\includesvg[keepaspectratio]{diagrams/firewall-architecture.svg}}
\caption{Firewall Architecture}
\end{figure}

\textbf{Types of Firewalls:}

{\def\LTcaptype{none} % do not increment counter
\begin{longtable}[]{@{}
  >{\raggedright\arraybackslash}p{(\linewidth - 4\tabcolsep) * \real{0.2609}}
  >{\raggedright\arraybackslash}p{(\linewidth - 4\tabcolsep) * \real{0.4348}}
  >{\raggedright\arraybackslash}p{(\linewidth - 4\tabcolsep) * \real{0.3043}}@{}}
\toprule\noalign{}
\begin{minipage}[b]{\linewidth}\raggedright
Type
\end{minipage} & \begin{minipage}[b]{\linewidth}\raggedright
Function
\end{minipage} & \begin{minipage}[b]{\linewidth}\raggedright
Level
\end{minipage} \\
\midrule\noalign{}
\endhead
\bottomrule\noalign{}
\endlastfoot
\textbf{Packet Filter} & Examines packet headers & Network Layer \\
\textbf{Stateful} & Tracks connection state & Transport Layer \\
\textbf{Application Proxy} & Inspects application data & Application
Layer \\
\textbf{Personal Firewall} & Protects individual devices & Host-based \\
\end{longtable}
}

\begin{itemize}
\tightlist
\item
  \textbf{Hardware Firewall}: Dedicated network appliance
\item
  \textbf{Software Firewall}: Installed on individual computers
\item
  \textbf{Cloud Firewall}: Delivered as a service (FWaaS)
\end{itemize}

\end{solutionbox}
\begin{mnemonicbox}
``Firewalls Protect Networks Always''

\end{mnemonicbox}
\subsection*{Question 2(b) [4 marks]}\label{q2b}

\textbf{Define: HTTPS and describe working of HTTPS.}

\begin{solutionbox}

\textbf{HTTPS Definition:} Hypertext Transfer Protocol Secure - HTTP
over SSL/TLS encryption.

\textbf{HTTPS Working Process:}

\begin{figure}
\centering
\pandocbounded{\includesvg[keepaspectratio]{diagrams/https-process.svg}}
\caption{HTTPS Process}
\end{figure}

\textbf{HTTPS Components:}

\begin{itemize}
\tightlist
\item
  \textbf{Port 443}: Standard HTTPS port
\item
  \textbf{SSL/TLS}: Encryption protocols
\item
  \textbf{Digital Certificates}: Server authentication
\item
  \textbf{Symmetric Encryption}: Data transmission security
\end{itemize}

\textbf{Benefits:}

\begin{itemize}
\tightlist
\item
  Data encryption during transmission
\item
  Server authentication verification
\item
  Data integrity protection
\item
  SEO ranking improvement
\end{itemize}

\end{solutionbox}
\begin{mnemonicbox}
``HTTPS Secures Web Traffic''

\end{mnemonicbox}
\subsection*{Question 2(c) [7 marks]}\label{q2c}

\textbf{Explain different types of malicious software and their effect.}

\begin{solutionbox}

\textbf{Malware Classification:}

\begin{figure}
\centering
\pandocbounded{\includesvg[keepaspectratio]{diagrams/malware-classification.svg}}
\caption{Malware Classification}
\end{figure}

{\def\LTcaptype{none} % do not increment counter
\begin{longtable}[]{@{}
  >{\raggedright\arraybackslash}p{(\linewidth - 6\tabcolsep) * \real{0.1818}}
  >{\raggedright\arraybackslash}p{(\linewidth - 6\tabcolsep) * \real{0.3030}}
  >{\raggedright\arraybackslash}p{(\linewidth - 6\tabcolsep) * \real{0.2424}}
  >{\raggedright\arraybackslash}p{(\linewidth - 6\tabcolsep) * \real{0.2727}}@{}}
\toprule\noalign{}
\begin{minipage}[b]{\linewidth}\raggedright
Type
\end{minipage} & \begin{minipage}[b]{\linewidth}\raggedright
Behavior
\end{minipage} & \begin{minipage}[b]{\linewidth}\raggedright
Effect
\end{minipage} & \begin{minipage}[b]{\linewidth}\raggedright
Example
\end{minipage} \\
\midrule\noalign{}
\endhead
\bottomrule\noalign{}
\endlastfoot
\textbf{Virus} & Attaches to files & File corruption & Boot sector
virus \\
\textbf{Worm} & Self-replicating & Network congestion & Conficker
worm \\
\textbf{Trojan} & Disguised malware & Data theft & Banking Trojans \\
\textbf{Ransomware} & Encrypts files & Data hostage & WannaCry \\
\textbf{Spyware} & Monitors activity & Privacy breach & Keyloggers \\
\textbf{Adware} & Shows unwanted ads & Performance degradation & Pop-up
ads \\
\textbf{Rootkit} & Hides presence & System compromise & Kernel
rootkits \\
\end{longtable}
}

\textbf{Effects on Systems:}

\begin{itemize}
\tightlist
\item
  \textbf{Performance}: Slow system response
\item
  \textbf{Data}: Loss, corruption, or theft
\item
  \textbf{Privacy}: Unauthorized monitoring
\item
  \textbf{Financial}: Direct monetary loss
\end{itemize}

\textbf{Prevention Methods:}

\begin{itemize}
\tightlist
\item
  Regular antivirus updates
\item
  Safe browsing practices
\item
  Email attachment caution
\item
  System security patches
\end{itemize}

\end{solutionbox}
\begin{mnemonicbox}
``Viruses Worms Trojans Really Steal All Resources''

\end{mnemonicbox}
\subsection*{Question 2(a OR) [3
marks]}\label{question-2a-or-3-marks}

\textbf{What is authentication? Explain different methods of
authentication.}

\begin{solutionbox}

\textbf{Authentication Definition:} Process of verifying user identity
before granting system access.

\textbf{Authentication Methods:}

\begin{figure}
\centering
\pandocbounded{\includesvg[keepaspectratio]{diagrams/authentication-methods.svg}}
\caption{Authentication Methods}
\end{figure}

{\def\LTcaptype{none} % do not increment counter
\begin{longtable}[]{@{}lll@{}}
\toprule\noalign{}
Method & Description & Example \\
\midrule\noalign{}
\endhead
\bottomrule\noalign{}
\endlastfoot
\textbf{Password} & Something you know & PIN, passphrase \\
\textbf{Biometric} & Something you are & Fingerprint, iris \\
\textbf{Token} & Something you have & Smart card, USB key \\
\end{longtable}
}

\begin{itemize}
\tightlist
\item
  \textbf{Single-Factor}: Uses one authentication method
\item
  \textbf{Multi-Factor}: Combines multiple methods
\item
  \textbf{Two-Factor (2FA)}: Uses exactly two factors
\end{itemize}

\end{solutionbox}
\begin{mnemonicbox}
``Password Biometric Token Authentication''

\end{mnemonicbox}
\subsection*{Question 2(b OR) [4
marks]}\label{question-2b-or-4-marks}

\textbf{Define: Trojans, Rootkit, Backdoors, Keylogger}

\begin{solutionbox}

\textbf{Malware Definitions:}

{\def\LTcaptype{none} % do not increment counter
\begin{longtable}[]{@{}
  >{\raggedright\arraybackslash}p{(\linewidth - 4\tabcolsep) * \real{0.1714}}
  >{\raggedright\arraybackslash}p{(\linewidth - 4\tabcolsep) * \real{0.3429}}
  >{\raggedright\arraybackslash}p{(\linewidth - 4\tabcolsep) * \real{0.4857}}@{}}
\toprule\noalign{}
\begin{minipage}[b]{\linewidth}\raggedright
Term
\end{minipage} & \begin{minipage}[b]{\linewidth}\raggedright
Definition
\end{minipage} & \begin{minipage}[b]{\linewidth}\raggedright
Characteristics
\end{minipage} \\
\midrule\noalign{}
\endhead
\bottomrule\noalign{}
\endlastfoot
\textbf{Trojans} & Malware disguised as legitimate software & Appears
harmless, hidden payload \\
\textbf{Rootkit} & Software that hides malware presence & Deep system
access, stealth operation \\
\textbf{Backdoors} & Unauthorized access method & Bypasses normal
authentication \\
\textbf{Keylogger} & Records keyboard input & Captures passwords,
sensitive data \\
\end{longtable}
}

\begin{itemize}
\tightlist
\item
  \textbf{Trojans}: Named after Greek Trojan Horse
\item
  \textbf{Rootkit}: Operates at kernel level
\item
  \textbf{Backdoors}: Can be hardware or software based
\item
  \textbf{Keylogger}: Can be software or hardware device
\end{itemize}

\end{solutionbox}
\begin{mnemonicbox}
``Trojans Root Backdoors Keylog''

\end{mnemonicbox}
\subsection*{Question 2(c OR) [7
marks]}\label{question-2c-or-7-marks}

\textbf{Explain Secure Socket Layer (SSL) and Transport Layer Security
(TLS) protocols.}

\begin{solutionbox}

\textbf{SSL/TLS Protocol Evolution:}

\begin{figure}
\centering
\pandocbounded{\includesvg[keepaspectratio]{diagrams/ssl-tls-handshake.svg}}
\caption{SSL/TLS Handshake}
\end{figure}

{\def\LTcaptype{none} % do not increment counter
\begin{longtable}[]{@{}llll@{}}
\toprule\noalign{}
Version & Year & Status & Security Level \\
\midrule\noalign{}
\endhead
\bottomrule\noalign{}
\endlastfoot
\textbf{SSL 2.0} & 1995 & Deprecated & Weak \\
\textbf{SSL 3.0} & 1996 & Deprecated & Vulnerable \\
\textbf{TLS 1.0} & 1999 & Legacy & Limited \\
\textbf{TLS 1.2} & 2008 & Widely used & Good \\
\textbf{TLS 1.3} & 2018 & Current & Strong \\
\end{longtable}
}

\textbf{TLS Handshake Process:}

\includegraphics[width=1\linewidth,height=\textheight,keepaspectratio]{mermaid-d2ba59b6.pdf}

\textbf{Key Features:}

\begin{itemize}
\tightlist
\item
  \textbf{Encryption}: Symmetric and asymmetric algorithms
\item
  \textbf{Authentication}: Server and client verification
\item
  \textbf{Integrity}: Message authentication codes
\item
  \textbf{Forward Secrecy}: Session key protection
\end{itemize}

\textbf{Applications:}

\begin{itemize}
\tightlist
\item
  HTTPS web browsing
\item
  Email security (SMTPS)
\item
  VPN connections
\item
  Secure file transfers
\end{itemize}

\end{solutionbox}
\begin{mnemonicbox}
``TLS Encrypts All Network Traffic''

\end{mnemonicbox}
\subsection*{Question 3(a) [3 marks]}\label{q3a}

\textbf{Explain in detail cybercrime and cybercriminal.}

\begin{solutionbox}

\textbf{Cybercrime Definition:} Criminal activities conducted through
computers or internet networks.

\textbf{Diagram:}

\begin{figure}
\centering
\pandocbounded{\includesvg[keepaspectratio]{diagrams/cybercrime-overview.svg}}
\caption{Cybercrime Overview}
\end{figure}

\textbf{Cybercriminal Types:}

{\def\LTcaptype{none} % do not increment counter
\begin{longtable}[]{@{}llll@{}}
\toprule\noalign{}
Type & Motivation & Skills & Target \\
\midrule\noalign{}
\endhead
\bottomrule\noalign{}
\endlastfoot
\textbf{Script Kiddies} & Fun/Fame & Low & Random \\
\textbf{Hacktivists} & Political/Social & Moderate & Organizations \\
\textbf{Cybercriminals} & Financial Gain & High & Individuals/Banks \\
\end{longtable}
}

\begin{itemize}
\tightlist
\item
  \textbf{Cybercrime}: Illegal activities using digital technology
\item
  \textbf{Cybercriminal}: Person who commits cybercrimes
\item
  \textbf{Impact}: Financial loss, privacy breach, system damage
\end{itemize}

\end{solutionbox}
\begin{mnemonicbox}
``Cyber Criminals Create Chaos''

\end{mnemonicbox}
\subsection*{Question 3(b) [4 marks]}\label{q3b}

\textbf{Describe cyber stalking and cyber bullying in detail.}

\begin{solutionbox}

\textbf{Digital Harassment Comparison:}

\begin{figure}
\centering
\pandocbounded{\includesvg[keepaspectratio]{diagrams/cyber-stalking-vs-bullying.svg}}
\caption{Cyber Stalking vs Cyber Bullying}
\end{figure}

{\def\LTcaptype{none} % do not increment counter
\begin{longtable}[]{@{}lll@{}}
\toprule\noalign{}
Aspect & Cyber Stalking & Cyber Bullying \\
\midrule\noalign{}
\endhead
\bottomrule\noalign{}
\endlastfoot
\textbf{Target} & Specific individual & Often minors \\
\textbf{Duration} & Persistent, long-term & Can be episodic \\
\textbf{Intent} & Intimidation, control & Harassment, humiliation \\
\textbf{Platform} & Social media, email & Schools, gaming platforms \\
\end{longtable}
}

\textbf{Cyber Stalking Characteristics:}

\begin{itemize}
\tightlist
\item
  Persistent unwanted contact
\item
  Monitoring victim's online activity
\item
  Threatening messages or behavior
\item
  Identity theft or impersonation
\end{itemize}

\textbf{Cyber Bullying Forms:}

\begin{itemize}
\tightlist
\item
  Public humiliation online
\item
  Exclusion from digital groups
\item
  Spreading false information
\item
  Sharing private content without consent
\end{itemize}

\textbf{Prevention Measures:}

\begin{itemize}
\tightlist
\item
  Privacy settings on social media
\item
  Reporting harassment to platforms
\item
  Legal action when necessary
\item
  Digital literacy education
\end{itemize}

\end{solutionbox}
\begin{mnemonicbox}
``Stop Bullying, Report Stalking''

\end{mnemonicbox}
\subsection*{Question 3(c) [7 marks]}\label{q3c}

\textbf{Explain Property based classification in cybercrime.}

\begin{solutionbox}

\textbf{Property-Based Cybercrime Categories:}

{\def\LTcaptype{none} % do not increment counter
\begin{longtable}[]{@{}
  >{\raggedright\arraybackslash}p{(\linewidth - 6\tabcolsep) * \real{0.2273}}
  >{\raggedright\arraybackslash}p{(\linewidth - 6\tabcolsep) * \real{0.2727}}
  >{\raggedright\arraybackslash}p{(\linewidth - 6\tabcolsep) * \real{0.2955}}
  >{\raggedright\arraybackslash}p{(\linewidth - 6\tabcolsep) * \real{0.2045}}@{}}
\toprule\noalign{}
\begin{minipage}[b]{\linewidth}\raggedright
Category
\end{minipage} & \begin{minipage}[b]{\linewidth}\raggedright
Crime Type
\end{minipage} & \begin{minipage}[b]{\linewidth}\raggedright
Description
\end{minipage} & \begin{minipage}[b]{\linewidth}\raggedright
Example
\end{minipage} \\
\midrule\noalign{}
\endhead
\bottomrule\noalign{}
\endlastfoot
\textbf{Intellectual Property} & Copyright infringement & Unauthorized
use of copyrighted material & Software piracy \\
\textbf{Financial Property} & Credit card fraud & Unauthorized use of
financial information & Online shopping fraud \\
\textbf{Digital Property} & Data theft & Stealing digital information &
Database breaches \\
\textbf{Virtual Property} & Gaming asset theft & Stealing virtual goods
& Online game currency theft \\
\end{longtable}
}

\textbf{Diagram:}

\begin{figure}
\centering
\pandocbounded{\includesvg[keepaspectratio]{diagrams/property-based-cybercrime.svg}}
\caption{Property-Based Cybercrime Classification}
\end{figure}

\textbf{Legal Aspects:}

\begin{itemize}
\tightlist
\item
  \textbf{Copyright Laws}: Protect creative works
\item
  \textbf{Trademark Laws}: Protect brand identity
\item
  \textbf{Patent Laws}: Protect inventions
\item
  \textbf{Trade Secret Laws}: Protect confidential information
\end{itemize}

\textbf{Impact on Economy:}

\begin{itemize}
\tightlist
\item
  Revenue loss for legitimate businesses
\item
  Reduced innovation incentives
\item
  Consumer trust erosion
\item
  Legal enforcement costs
\end{itemize}

\textbf{Prevention Strategies:}

\begin{itemize}
\tightlist
\item
  Digital rights management (DRM)
\item
  Watermarking and tracking
\item
  Legal enforcement mechanisms
\item
  Public awareness campaigns
\end{itemize}

\end{solutionbox}
\begin{mnemonicbox}
``Property Protection Prevents Piracy''

\end{mnemonicbox}
\subsection*{Question 3(a OR) [3
marks]}\label{question-3a-or-3-marks}

\textbf{Explain Data diddling.}

\begin{solutionbox}

\textbf{Data Diddling Definition:} Unauthorized alteration of data
before or during input into computer systems.

\begin{figure}
\centering
\pandocbounded{\includesvg[keepaspectratio]{diagrams/data-diddling-process.svg}}
\caption{Data Diddling Process}
\end{figure}

\textbf{Characteristics:}

{\def\LTcaptype{none} % do not increment counter
\begin{longtable}[]{@{}ll@{}}
\toprule\noalign{}
Aspect & Description \\
\midrule\noalign{}
\endhead
\bottomrule\noalign{}
\endlastfoot
\textbf{Method} & Changing data values \\
\textbf{Timing} & Before system processing \\
\textbf{Detection} & Often difficult to identify \\
\end{longtable}
}

\begin{itemize}
\tightlist
\item
  \textbf{Examples}: Changing salary figures, altering exam scores
\item
  \textbf{Target}: Input data during entry process
\item
  \textbf{Impact}: Financial loss, incorrect records
\end{itemize}

\end{solutionbox}
\begin{mnemonicbox}
``Data Diddling Damages Databases''

\end{mnemonicbox}
\subsection*{Question 3(b OR) [4
marks]}\label{question-3b-or-4-marks}

\textbf{Explain cyber spying and cyber terrorism.}

\begin{solutionbox}

\textbf{Cyber Threats Comparison:}

\begin{figure}
\centering
\pandocbounded{\includesvg[keepaspectratio]{diagrams/cyber-spying-vs-terrorism.svg}}
\caption{Cyber Spying vs Cyber Terrorism}
\end{figure}

{\def\LTcaptype{none} % do not increment counter
\begin{longtable}[]{@{}lll@{}}
\toprule\noalign{}
Aspect & Cyber Spying & Cyber Terrorism \\
\midrule\noalign{}
\endhead
\bottomrule\noalign{}
\endlastfoot
\textbf{Purpose} & Information gathering & Causing fear/disruption \\
\textbf{Target} & Government, corporations & Critical infrastructure \\
\textbf{Methods} & Stealth infiltration & Destructive attacks \\
\textbf{Impact} & Intelligence loss & Public safety risk \\
\end{longtable}
}

\textbf{Cyber Spying Activities:}

\begin{itemize}
\tightlist
\item
  Corporate espionage
\item
  Government surveillance
\item
  Trade secret theft
\item
  Personal information gathering
\end{itemize}

\textbf{Cyber Terrorism Methods:}

\begin{itemize}
\tightlist
\item
  Infrastructure attacks
\item
  Mass disruption campaigns
\item
  Psychological warfare
\item
  Economic damage
\end{itemize}

\textbf{Prevention Measures:}

\begin{itemize}
\tightlist
\item
  Network security monitoring
\item
  Incident response planning
\item
  International cooperation
\item
  Public-private partnerships
\end{itemize}

\end{solutionbox}
\begin{mnemonicbox}
``Spies Steal, Terrorists Terror''

\end{mnemonicbox}
\subsection*{Question 3(c OR) [7
marks]}\label{question-3c-or-7-marks}

\textbf{Explain the role of digital signatures and digital certificates
in cybersecurity.}

\begin{solutionbox}

\textbf{Digital Security Components:}

\begin{figure}
\centering
\pandocbounded{\includesvg[keepaspectratio]{diagrams/digital-signatures-certificates.svg}}
\caption{Digital Signatures and Certificates}
\end{figure}

{\def\LTcaptype{none} % do not increment counter
\begin{longtable}[]{@{}
  >{\raggedright\arraybackslash}p{(\linewidth - 6\tabcolsep) * \real{0.2821}}
  >{\raggedright\arraybackslash}p{(\linewidth - 6\tabcolsep) * \real{0.2308}}
  >{\raggedright\arraybackslash}p{(\linewidth - 6\tabcolsep) * \real{0.2564}}
  >{\raggedright\arraybackslash}p{(\linewidth - 6\tabcolsep) * \real{0.2308}}@{}}
\toprule\noalign{}
\begin{minipage}[b]{\linewidth}\raggedright
Component
\end{minipage} & \begin{minipage}[b]{\linewidth}\raggedright
Purpose
\end{minipage} & \begin{minipage}[b]{\linewidth}\raggedright
Function
\end{minipage} & \begin{minipage}[b]{\linewidth}\raggedright
Benefit
\end{minipage} \\
\midrule\noalign{}
\endhead
\bottomrule\noalign{}
\endlastfoot
\textbf{Digital Signature} & Authentication & Proves sender identity &
Non-repudiation \\
\textbf{Digital Certificate} & Verification & Validates public keys &
Trust establishment \\
\end{longtable}
}

\textbf{Digital Signature Process:}

\includegraphics[width=1\linewidth,height=\textheight,keepaspectratio]{mermaid-e023f6d1.pdf}

\textbf{Digital Certificate Components:}

\begin{itemize}
\tightlist
\item
  \textbf{Subject Information}: Certificate owner details
\item
  \textbf{Public Key}: For encryption/verification
\item
  \textbf{Digital Signature}: CA's signature
\item
  \textbf{Validity Period}: Certificate expiration date
\end{itemize}

\textbf{Certificate Authority (CA) Role:}

\begin{itemize}
\tightlist
\item
  Issues digital certificates
\item
  Verifies identity before issuance
\item
  Maintains certificate revocation lists
\item
  Provides trust infrastructure
\end{itemize}

\textbf{Applications in Cybersecurity:}

\begin{itemize}
\tightlist
\item
  Email security (S/MIME)
\item
  Code signing for software
\item
  SSL/TLS certificates for websites
\item
  Document authentication
\end{itemize}

\textbf{Security Benefits:}

\begin{itemize}
\tightlist
\item
  \textbf{Authentication}: Verifies sender identity
\item
  \textbf{Integrity}: Ensures data hasn't been modified
\item
  \textbf{Non-repudiation}: Prevents denial of actions
\item
  \textbf{Confidentiality}: Enables secure communication
\end{itemize}

\end{solutionbox}
\begin{mnemonicbox}
``Digital Signatures Authenticate Documents
Securely''

\end{mnemonicbox}
\subsection*{Question 4(a) [3 marks]}\label{q4a}

\textbf{What is Hacking? List out types of Hackers.}

\begin{solutionbox}

\textbf{Hacking Definition:} Unauthorized access to computer systems or
networks to exploit vulnerabilities.

\textbf{Hacker Classifications:}

\begin{figure}
\centering
\pandocbounded{\includesvg[keepaspectratio]{diagrams/hacker-types.svg}}
\caption{Types of Hackers}
\end{figure}

{\def\LTcaptype{none} % do not increment counter
\begin{longtable}[]{@{}lll@{}}
\toprule\noalign{}
Type & Intent & Legal Status \\
\midrule\noalign{}
\endhead
\bottomrule\noalign{}
\endlastfoot
\textbf{White Hat} & Security improvement & Legal \\
\textbf{Black Hat} & Malicious activities & Illegal \\
\textbf{Gray Hat} & Mixed motivations & Questionable \\
\end{longtable}
}

\begin{itemize}
\tightlist
\item
  \textbf{White Hat}: Ethical hackers, security researchers
\item
  \textbf{Black Hat}: Cybercriminals, malicious intent
\item
  \textbf{Gray Hat}: Sometimes legal, sometimes not
\end{itemize}

\end{solutionbox}
\begin{mnemonicbox}
``White Good, Black Bad, Gray Questionable''

\end{mnemonicbox}
\subsection*{Question 4(b) [4 marks]}\label{q4b}

\textbf{Explain Vulnerability and 0-Day terminology of Hacking.}

\begin{solutionbox}

\textbf{Security Terminology:}

\begin{figure}
\centering
\pandocbounded{\includesvg[keepaspectratio]{diagrams/vulnerability-vs-0day.svg}}
\caption{Vulnerability vs 0-Day}
\end{figure}

{\def\LTcaptype{none} % do not increment counter
\begin{longtable}[]{@{}llll@{}}
\toprule\noalign{}
Term & Definition & Risk Level & Example \\
\midrule\noalign{}
\endhead
\bottomrule\noalign{}
\endlastfoot
\textbf{Vulnerability} & System weakness & Varies & Unpatched
software \\
\textbf{0-Day} & Unknown vulnerability & Critical & Undiscovered flaw \\
\end{longtable}
}

\textbf{Vulnerability Characteristics:}

\begin{itemize}
\tightlist
\item
  \textbf{Discovery}: Found through security testing
\item
  \textbf{Disclosure}: Responsible reporting to vendors
\item
  \textbf{Patching}: Vendor provides security updates
\item
  \textbf{Window}: Time between discovery and patch
\end{itemize}

\textbf{0-Day Attack Process:}

\begin{itemize}
\tightlist
\item
  Hacker discovers unknown vulnerability
\item
  Exploits flaw before vendor awareness
\item
  No available patches or defenses
\item
  High success rate due to surprise element
\end{itemize}

\textbf{Protection Strategies:}

\begin{itemize}
\tightlist
\item
  Regular security updates
\item
  Intrusion detection systems
\item
  Behavioral analysis tools
\item
  Zero-trust security models
\end{itemize}

\end{solutionbox}
\begin{mnemonicbox}
``Vulnerabilities Need Patches, Zero-Days Need
Vigilance''

\end{mnemonicbox}
\subsection*{Question 4(c) [7 marks]}\label{q4c}

\textbf{Explain Five Steps of Hacking.}

\begin{solutionbox}

\textbf{Hacking Methodology:}

\begin{figure}
\centering
\pandocbounded{\includesvg[keepaspectratio]{diagrams/hacking-steps.svg}}
\caption{Hacking Steps}
\end{figure}

\textbf{Detailed Steps:}

{\def\LTcaptype{none} % do not increment counter
\begin{longtable}[]{@{}
  >{\raggedright\arraybackslash}p{(\linewidth - 6\tabcolsep) * \real{0.1333}}
  >{\raggedright\arraybackslash}p{(\linewidth - 6\tabcolsep) * \real{0.2889}}
  >{\raggedright\arraybackslash}p{(\linewidth - 6\tabcolsep) * \real{0.3333}}
  >{\raggedright\arraybackslash}p{(\linewidth - 6\tabcolsep) * \real{0.2444}}@{}}
\toprule\noalign{}
\begin{minipage}[b]{\linewidth}\raggedright
Step
\end{minipage} & \begin{minipage}[b]{\linewidth}\raggedright
Description
\end{minipage} & \begin{minipage}[b]{\linewidth}\raggedright
Tools/Methods
\end{minipage} & \begin{minipage}[b]{\linewidth}\raggedright
Objective
\end{minipage} \\
\midrule\noalign{}
\endhead
\bottomrule\noalign{}
\endlastfoot
\textbf{Reconnaissance} & Information gathering & Google dorking, Social
media & Target profiling \\
\textbf{Scanning} & System enumeration & Nmap, Nessus & Vulnerability
identification \\
\textbf{Gaining Access} & Exploit vulnerabilities & Metasploit, Custom
exploits & System compromise \\
\textbf{Maintaining Access} & Persistent presence & Backdoors, Rootkits
& Long-term control \\
\textbf{Covering Tracks} & Evidence removal & Log cleaning, File
deletion & Avoid detection \\
\end{longtable}
}

\textbf{Information Gathering Types:}

\begin{itemize}
\tightlist
\item
  \textbf{Passive}: No direct target contact
\item
  \textbf{Active}: Direct interaction with target systems
\end{itemize}

\textbf{Scanning Techniques:}

\begin{itemize}
\tightlist
\item
  Port scanning for open services
\item
  Vulnerability scanning for weaknesses
\item
  Network mapping for topology
\end{itemize}

\textbf{Access Methods:}

\begin{itemize}
\tightlist
\item
  Password attacks (brute force, dictionary)
\item
  Exploit vulnerabilities
\item
  Social engineering
\item
  Physical access
\end{itemize}

\textbf{Persistence Mechanisms:}

\begin{itemize}
\tightlist
\item
  Installing backdoors
\item
  Creating user accounts
\item
  Scheduling tasks
\item
  Registry modifications
\end{itemize}

\textbf{Track Covering Methods:}

\begin{itemize}
\tightlist
\item
  Clearing system logs
\item
  Deleting temporary files
\item
  Modifying timestamps
\item
  Using encryption
\end{itemize}

\end{solutionbox}
\begin{mnemonicbox}
``Reconnaissance Scans Generate Access, Maintain
Coverage''

\end{mnemonicbox}
\subsection*{Question 4(a OR) [3
marks]}\label{question-4a-or-3-marks}

\textbf{Explain any three basic commands of Kali Linux with suitable
example.}

\begin{solutionbox}

\textbf{Essential Kali Linux Commands:}

\begin{figure}
\centering
\pandocbounded{\includesvg[keepaspectratio]{diagrams/kali-linux-commands-detailed.svg}}
\caption{Kali Linux Commands}
\end{figure}

{\def\LTcaptype{none} % do not increment counter
\begin{longtable}[]{@{}
  >{\raggedright\arraybackslash}p{(\linewidth - 4\tabcolsep) * \real{0.3214}}
  >{\raggedright\arraybackslash}p{(\linewidth - 4\tabcolsep) * \real{0.3571}}
  >{\raggedright\arraybackslash}p{(\linewidth - 4\tabcolsep) * \real{0.3214}}@{}}
\toprule\noalign{}
\begin{minipage}[b]{\linewidth}\raggedright
Command
\end{minipage} & \begin{minipage}[b]{\linewidth}\raggedright
Function
\end{minipage} & \begin{minipage}[b]{\linewidth}\raggedright
Example
\end{minipage} \\
\midrule\noalign{}
\endhead
\bottomrule\noalign{}
\endlastfoot
\textbf{nmap} & Network scanning &
\passthrough{\lstinline!nmap -sS 192.168.1.1!} \\
\textbf{netcat} & Network communication &
\passthrough{\lstinline!nc -l -p 1234!} \\
\textbf{hydra} & Password cracking &
\passthrough{\lstinline!hydra -l admin -P passwords.txt ssh://target!} \\
\end{longtable}
}

\begin{itemize}
\tightlist
\item
  \textbf{Nmap}: Discovers hosts and services on network
\item
  \textbf{Netcat}: Creates network connections for data transfer
\item
  \textbf{Hydra}: Performs brute-force password attacks
\end{itemize}

\end{solutionbox}
\begin{mnemonicbox}
``Network Map, Connect, Crack''

\end{mnemonicbox}
\subsection*{Question 4(b OR) [4
marks]}\label{question-4b-or-4-marks}

\textbf{Describe Session Hijacking in detail.}

\begin{solutionbox}

\textbf{Session Hijacking Overview:} Attack where attacker takes over
legitimate user's session.

\begin{figure}
\centering
\pandocbounded{\includesvg[keepaspectratio]{diagrams/session-hijacking-process.svg}}
\caption{Session Hijacking Process}
\end{figure}

\textbf{Types of Session Hijacking:}

{\def\LTcaptype{none} % do not increment counter
\begin{longtable}[]{@{}lll@{}}
\toprule\noalign{}
Type & Method & Prevention \\
\midrule\noalign{}
\endhead
\bottomrule\noalign{}
\endlastfoot
\textbf{Active} & Takes over session & Strong session management \\
\textbf{Passive} & Monitors session & Encryption (HTTPS) \\
\textbf{Network-level} & TCP hijacking & Secure protocols \\
\textbf{Application-level} & Cookie theft & Secure cookie attributes \\
\end{longtable}
}

\textbf{Attack Process:}

\begin{enumerate}
\tightlist
\item
  Monitor network traffic
\item
  Capture session identifiers
\item
  Replay session tokens
\item
  Access user account
\end{enumerate}

\textbf{Prevention Measures:}

\begin{itemize}
\tightlist
\item
  Use HTTPS for all communications
\item
  Implement secure session management
\item
  Set secure cookie attributes
\item
  Monitor for suspicious activity
\end{itemize}

\end{solutionbox}
\begin{mnemonicbox}
``Sessions Hijacked Need Secure Handling''

\end{mnemonicbox}
\subsection*{Question 4(c OR) [7
marks]}\label{question-4c-or-7-marks}

\textbf{Explain how Virtual Private Networks (VPNs) create secure,
encrypted connections over public networks.}

\begin{solutionbox}

\textbf{VPN Architecture:}

\begin{figure}
\centering
\pandocbounded{\includesvg[keepaspectratio]{diagrams/vpn-architecture.svg}}
\caption{VPN Architecture}
\end{figure}

\textbf{VPN Components:}

{\def\LTcaptype{none} % do not increment counter
\begin{longtable}[]{@{}lll@{}}
\toprule\noalign{}
Component & Function & Benefit \\
\midrule\noalign{}
\endhead
\bottomrule\noalign{}
\endlastfoot
\textbf{Tunneling} & Creates secure pathway & Data protection \\
\textbf{Encryption} & Scrambles data & Confidentiality \\
\textbf{Authentication} & Verifies identity & Access control \\
\textbf{IP Masking} & Hides real IP & Anonymity \\
\end{longtable}
}

\textbf{VPN Protocols:}

{\def\LTcaptype{none} % do not increment counter
\begin{longtable}[]{@{}llll@{}}
\toprule\noalign{}
Protocol & Security Level & Speed & Use Case \\
\midrule\noalign{}
\endhead
\bottomrule\noalign{}
\endlastfoot
\textbf{OpenVPN} & High & Good & General purpose \\
\textbf{IPSec} & Very High & Moderate & Enterprise \\
\textbf{WireGuard} & High & Excellent & Modern solution \\
\textbf{PPTP} & Low & Fast & Legacy (deprecated) \\
\end{longtable}
}

\textbf{VPN Working Process:}

\begin{enumerate}
\tightlist
\item
  \textbf{Connection}: Client connects to VPN server
\item
  \textbf{Authentication}: User credentials verified
\item
  \textbf{Tunnel Creation}: Encrypted pathway established
\item
  \textbf{Data Encryption}: All traffic encrypted
\item
  \textbf{Routing}: Traffic routed through VPN server
\item
  \textbf{Decryption}: Data decrypted at destination
\end{enumerate}

\textbf{Security Benefits:}

\begin{itemize}
\tightlist
\item
  \textbf{Data Protection}: Encryption prevents eavesdropping
\item
  \textbf{Privacy}: IP address masking
\item
  \textbf{Access Control}: Authenticate before connection
\item
  \textbf{Bypass Restrictions}: Access geo-blocked content
\end{itemize}

\textbf{Business Applications:}

\begin{itemize}
\tightlist
\item
  Remote worker access
\item
  Site-to-site connectivity
\item
  Secure cloud access
\item
  Compliance requirements
\end{itemize}

\textbf{Personal Use Cases:}

\begin{itemize}
\tightlist
\item
  Public Wi-Fi protection
\item
  Privacy enhancement
\item
  Content access
\item
  Location privacy
\end{itemize}

\end{solutionbox}
\begin{mnemonicbox}
``VPNs Provide Network Privacy''

\end{mnemonicbox}
\subsection*{Question 5(a) [3 marks]}\label{q5a}

\textbf{Explain Network forensics.}

\begin{solutionbox}

\textbf{Network Forensics Definition:} Investigation of network traffic
to detect and analyze security incidents.

\begin{figure}
\centering
\pandocbounded{\includesvg[keepaspectratio]{diagrams/network-forensics-process.svg}}
\caption{Network Forensics Process}
\end{figure}

\textbf{Key Components:}

{\def\LTcaptype{none} % do not increment counter
\begin{longtable}[]{@{}lll@{}}
\toprule\noalign{}
Component & Purpose & Tools \\
\midrule\noalign{}
\endhead
\bottomrule\noalign{}
\endlastfoot
\textbf{Traffic Capture} & Record network data & Wireshark, tcpdump \\
\textbf{Analysis} & Examine patterns & NetworkMiner, Snort \\
\textbf{Evidence} & Document findings & Forensic reports \\
\end{longtable}
}

\begin{itemize}
\tightlist
\item
  \textbf{Scope}: Analyzes packets, flows, and network behavior
\item
  \textbf{Objective}: Identify security breaches and attack patterns
\item
  \textbf{Challenge}: Large data volumes and real-time processing
\end{itemize}

\end{solutionbox}
\begin{mnemonicbox}
``Network Forensics Finds Facts''

\end{mnemonicbox}
\subsection*{Question 5(b) [4 marks]}\label{q5b}

\textbf{Explain why CCTV plays an important role as evidence in digital
forensics investigations.}

\begin{solutionbox}

\textbf{CCTV in Digital Forensics:}

\begin{figure}
\centering
\pandocbounded{\includesvg[keepaspectratio]{diagrams/cctv-digital-forensics.svg}}
\caption{CCTV in Digital Forensics}
\end{figure}

{\def\LTcaptype{none} % do not increment counter
\begin{longtable}[]{@{}lll@{}}
\toprule\noalign{}
Aspect & Importance & Value \\
\midrule\noalign{}
\endhead
\bottomrule\noalign{}
\endlastfoot
\textbf{Visual Evidence} & Direct observation & High credibility \\
\textbf{Timeline} & Time-stamped records & Event correlation \\
\textbf{Digital Format} & Easy to analyze & Metadata extraction \\
\textbf{Backup} & Multiple copies & Evidence preservation \\
\end{longtable}
}

\textbf{Evidence Value:}

\begin{itemize}
\tightlist
\item
  \textbf{Corroboration}: Supports other digital evidence
\item
  \textbf{Timeline}: Establishes sequence of events
\item
  \textbf{Identity}: May reveal perpetrator identity
\item
  \textbf{Context}: Shows physical environment during incident
\end{itemize}

\textbf{Forensic Considerations:}

\begin{itemize}
\tightlist
\item
  \textbf{Chain of Custody}: Proper evidence handling
\item
  \textbf{Authentication}: Verify video integrity
\item
  \textbf{Analysis}: Enhancement and interpretation
\item
  \textbf{Legal Admissibility}: Court-acceptable format
\end{itemize}

\end{solutionbox}
\begin{mnemonicbox}
``CCTV Captures Criminal Conduct Clearly''

\end{mnemonicbox}
\subsection*{Question 5(c) [7 marks]}\label{q5c}

\textbf{Explain phases of Digital forensic investigation.}

\begin{solutionbox}

\textbf{Digital Forensics Investigation Phases:}

\begin{figure}
\centering
\pandocbounded{\includesvg[keepaspectratio]{diagrams/digital-forensics-phases.svg}}
\caption{Digital Forensics Phases}
\end{figure}

\textbf{Detailed Phase Breakdown:}

{\def\LTcaptype{none} % do not increment counter
\begin{longtable}[]{@{}
  >{\raggedright\arraybackslash}p{(\linewidth - 6\tabcolsep) * \real{0.1892}}
  >{\raggedright\arraybackslash}p{(\linewidth - 6\tabcolsep) * \real{0.3243}}
  >{\raggedright\arraybackslash}p{(\linewidth - 6\tabcolsep) * \real{0.1892}}
  >{\raggedright\arraybackslash}p{(\linewidth - 6\tabcolsep) * \real{0.2973}}@{}}
\toprule\noalign{}
\begin{minipage}[b]{\linewidth}\raggedright
Phase
\end{minipage} & \begin{minipage}[b]{\linewidth}\raggedright
Activities
\end{minipage} & \begin{minipage}[b]{\linewidth}\raggedright
Tools
\end{minipage} & \begin{minipage}[b]{\linewidth}\raggedright
Objective
\end{minipage} \\
\midrule\noalign{}
\endhead
\bottomrule\noalign{}
\endlastfoot
\textbf{Identification} & Recognize potential evidence & Visual
inspection & Scope definition \\
\textbf{Preservation} & Prevent evidence contamination & Write blockers
& Evidence integrity \\
\textbf{Collection} & Acquire digital evidence & Forensic imaging &
Complete data capture \\
\textbf{Examination} & Extract relevant data & Autopsy, FTK & Data
recovery \\
\textbf{Analysis} & Interpret findings & Timeline tools & Pattern
identification \\
\textbf{Presentation} & Document results & Report generators & Legal
presentation \\
\end{longtable}
}

\textbf{Phase 1 - Identification:}

\begin{itemize}
\tightlist
\item
  Survey the scene
\item
  Identify potential evidence sources
\item
  Document initial observations
\item
  Establish investigation scope
\end{itemize}

\textbf{Phase 2 - Preservation:}

\begin{itemize}
\tightlist
\item
  Secure the crime scene
\item
  Prevent evidence contamination
\item
  Use write-protection mechanisms
\item
  Document evidence condition
\end{itemize}

\textbf{Phase 3 - Collection:}

\begin{itemize}
\tightlist
\item
  Create forensic images
\item
  Maintain chain of custody
\item
  Use proper collection techniques
\item
  Generate hash values for verification
\end{itemize}

\textbf{Phase 4 - Examination:}

\begin{itemize}
\tightlist
\item
  Extract file systems
\item
  Recover deleted data
\item
  Identify relevant files
\item
  Document findings
\end{itemize}

\textbf{Phase 5 - Analysis:}

\begin{itemize}
\tightlist
\item
  Correlate evidence
\item
  Reconstruct events
\item
  Identify patterns
\item
  Form conclusions
\end{itemize}

\textbf{Phase 6 - Presentation:}

\begin{itemize}
\tightlist
\item
  Prepare detailed reports
\item
  Create visual presentations
\item
  Explain technical findings
\item
  Support legal proceedings
\end{itemize}

\textbf{Quality Assurance:}

\begin{itemize}
\tightlist
\item
  \textbf{Documentation}: Detailed records at each phase
\item
  \textbf{Validation}: Verify procedures and results
\item
  \textbf{Reproducibility}: Ensure results can be replicated
\item
  \textbf{Legal Compliance}: Follow jurisdictional requirements
\end{itemize}

\end{solutionbox}
\begin{mnemonicbox}
``Investigators Preserve, Collect, Examine, Analyze,
Present''

\end{mnemonicbox}
\subsection*{Question 5(a OR) [3
marks]}\label{question-5a-or-3-marks}

\textbf{List applications of microcontrollers in various fields related
to cybersecurity.}

\begin{solutionbox}

\textbf{Microcontroller Security Applications:}

\begin{figure}
\centering
\pandocbounded{\includesvg[keepaspectratio]{diagrams/microcontroller-security-applications.svg}}
\caption{Microcontroller Security Applications}
\end{figure}

{\def\LTcaptype{none} % do not increment counter
\begin{longtable}[]{@{}lll@{}}
\toprule\noalign{}
Field & Application & Security Function \\
\midrule\noalign{}
\endhead
\bottomrule\noalign{}
\endlastfoot
\textbf{IoT Security} & Smart home devices & Authentication,
encryption \\
\textbf{Access Control} & Key cards, biometric & Identity
verification \\
\textbf{Network Security} & Hardware firewalls & Packet filtering \\
\end{longtable}
}

\begin{itemize}
\tightlist
\item
  \textbf{Smart Cards}: Secure authentication tokens
\item
  \textbf{HSM (Hardware Security Modules)}: Cryptographic processing
\item
  \textbf{Embedded Systems}: Secure boot, tamper detection
\end{itemize}

\end{solutionbox}
\begin{mnemonicbox}
``Microcontrollers Manage Multiple Security
Functions''

\end{mnemonicbox}
\subsection*{Question 5(b OR) [4
marks]}\label{question-5b-or-4-marks}

\textbf{Explain the importance of port scanning in ethical hacking.}

\begin{solutionbox}

\textbf{Port Scanning in Ethical Hacking:}

\begin{figure}
\centering
\pandocbounded{\includesvg[keepaspectratio]{diagrams/port-scanning-ethical-hacking.svg}}
\caption{Port Scanning in Ethical Hacking}
\end{figure}

{\def\LTcaptype{none} % do not increment counter
\begin{longtable}[]{@{}
  >{\raggedright\arraybackslash}p{(\linewidth - 4\tabcolsep) * \real{0.2759}}
  >{\raggedright\arraybackslash}p{(\linewidth - 4\tabcolsep) * \real{0.4138}}
  >{\raggedright\arraybackslash}p{(\linewidth - 4\tabcolsep) * \real{0.3103}}@{}}
\toprule\noalign{}
\begin{minipage}[b]{\linewidth}\raggedright
Aspect
\end{minipage} & \begin{minipage}[b]{\linewidth}\raggedright
Importance
\end{minipage} & \begin{minipage}[b]{\linewidth}\raggedright
Benefit
\end{minipage} \\
\midrule\noalign{}
\endhead
\bottomrule\noalign{}
\endlastfoot
\textbf{Service Discovery} & Identify running services & Attack surface
mapping \\
\textbf{Vulnerability Assessment} & Find open ports & Security gap
identification \\
\textbf{Network Mapping} & Understand topology & Infrastructure
analysis \\
\textbf{Security Testing} & Validate configurations & Compliance
verification \\
\end{longtable}
}

\textbf{Port Scanning Techniques:}

\begin{itemize}
\tightlist
\item
  \textbf{TCP Connect}: Full connection establishment
\item
  \textbf{SYN Scan}: Stealth scanning method
\item
  \textbf{UDP Scan}: User Datagram Protocol scanning
\item
  \textbf{Service Detection}: Identify service versions
\end{itemize}

\textbf{Ethical Considerations:}

\begin{itemize}
\tightlist
\item
  \textbf{Authorization}: Obtain proper permissions
\item
  \textbf{Scope}: Stay within defined boundaries
\item
  \textbf{Documentation}: Record all activities
\item
  \textbf{Reporting}: Provide detailed findings
\end{itemize}

\end{solutionbox}
\begin{mnemonicbox}
``Port Scanning Provides Security Insights''

\end{mnemonicbox}
\subsection*{Question 5(c OR) [7
marks]}\label{question-5c-or-7-marks}

\textbf{Describe the process of conducting a vulnerability assessment
using Kali Linux tools.}

\begin{solutionbox}

\textbf{Vulnerability Assessment Process:}

\begin{figure}
\centering
\pandocbounded{\includesvg[keepaspectratio]{diagrams/vulnerability-assessment-process.svg}}
\caption{Vulnerability Assessment Process}
\end{figure}

\textbf{Kali Linux Tools and Commands:}

\begin{figure}
\centering
\pandocbounded{\includesvg[keepaspectratio]{diagrams/kali-linux-commands.svg}}
\caption{Kali Linux Commands}
\end{figure}

\textbf{Step-by-Step Process:}

{\def\LTcaptype{none} % do not increment counter
\begin{longtable}[]{@{}
  >{\raggedright\arraybackslash}p{(\linewidth - 6\tabcolsep) * \real{0.1395}}
  >{\raggedright\arraybackslash}p{(\linewidth - 6\tabcolsep) * \real{0.2558}}
  >{\raggedright\arraybackslash}p{(\linewidth - 6\tabcolsep) * \real{0.3953}}
  >{\raggedright\arraybackslash}p{(\linewidth - 6\tabcolsep) * \real{0.2093}}@{}}
\toprule\noalign{}
\begin{minipage}[b]{\linewidth}\raggedright
Step
\end{minipage} & \begin{minipage}[b]{\linewidth}\raggedright
Kali Tool
\end{minipage} & \begin{minipage}[b]{\linewidth}\raggedright
Command Example
\end{minipage} & \begin{minipage}[b]{\linewidth}\raggedright
Purpose
\end{minipage} \\
\midrule\noalign{}
\endhead
\bottomrule\noalign{}
\endlastfoot
\textbf{Reconnaissance} & Nmap &
\passthrough{\lstinline!nmap -sn 192.168.1.0/24!} & Host discovery \\
\textbf{Port Scanning} & Nmap &
\passthrough{\lstinline!nmap -sS -O target!} & Open port
identification \\
\textbf{Service Enumeration} & Nmap, Banner grabbing &
\passthrough{\lstinline!nmap -sV target!} & Service version detection \\
\textbf{Vulnerability Scanning} & OpenVAS, Nessus &
\passthrough{\lstinline!openvas-start!} & Automated vulnerability
detection \\
\textbf{Web Application Testing} & Nikto, Dirb &
\passthrough{\lstinline!nikto -h target!} & Web vulnerability
scanning \\
\end{longtable}
}

\textbf{Detailed Process:}

\textbf{Phase 1 - Target Identification:}

\begin{itemize}
\tightlist
\item
  Use Nmap for network discovery
\item
  Identify live hosts and their IP addresses
\item
  Document network topology
\item
  Determine target scope
\end{itemize}

\textbf{Phase 2 - Port and Service Analysis:}

\begin{itemize}
\tightlist
\item
  Perform comprehensive port scans
\item
  Identify running services and versions
\item
  Check for default credentials
\item
  Analyze service configurations
\end{itemize}

\textbf{Phase 3 - Automated Vulnerability Scanning:}

\begin{itemize}
\tightlist
\item
  Configure vulnerability scanners (OpenVAS)
\item
  Run comprehensive scans
\item
  Analyze scan results
\item
  Prioritize findings by severity
\end{itemize}

\textbf{Phase 4 - Manual Testing:}

\begin{itemize}
\tightlist
\item
  Verify automated findings
\item
  Perform targeted testing
\item
  Test for specific vulnerabilities
\item
  Validate false positives
\end{itemize}

\textbf{Phase 5 - Web Application Assessment:}

\begin{itemize}
\tightlist
\item
  Use web vulnerability scanners
\item
  Test for OWASP Top 10 vulnerabilities
\item
  Analyze application logic
\item
  Check for misconfigurations
\end{itemize}

\textbf{Common Kali Tools:}

{\def\LTcaptype{none} % do not increment counter
\begin{longtable}[]{@{}lll@{}}
\toprule\noalign{}
Tool & Function & Use Case \\
\midrule\noalign{}
\endhead
\bottomrule\noalign{}
\endlastfoot
\textbf{Nmap} & Network scanning & Port and service discovery \\
\textbf{OpenVAS} & Vulnerability scanning & Automated assessment \\
\textbf{Nikto} & Web scanning & Web server vulnerabilities \\
\textbf{Dirb} & Directory brute forcing & Hidden file discovery \\
\textbf{SQLmap} & SQL injection testing & Database vulnerabilities \\
\textbf{Burp Suite} & Web proxy & Manual web testing \\
\textbf{Metasploit} & Exploitation framework & Vulnerability
validation \\
\end{longtable}
}

\textbf{Assessment Methodology:}

\begin{itemize}
\tightlist
\item
  \textbf{Scope Definition}: Clearly define assessment boundaries
\item
  \textbf{Information Gathering}: Collect target intelligence
\item
  \textbf{Vulnerability Detection}: Use multiple scanning methods
\item
  \textbf{Risk Assessment}: Evaluate impact and likelihood
\item
  \textbf{Remediation Planning}: Provide actionable recommendations
\end{itemize}

\textbf{Reporting Components:}

\begin{itemize}
\tightlist
\item
  \textbf{Executive Summary}: High-level findings for management
\item
  \textbf{Technical Details}: Detailed vulnerability descriptions
\item
  \textbf{Risk Ratings}: CVSS scores and business impact
\item
  \textbf{Remediation Steps}: Specific mitigation recommendations
\item
  \textbf{Supporting Evidence}: Screenshots and proof-of-concept
\end{itemize}

\textbf{Best Practices:}

\begin{itemize}
\tightlist
\item
  \textbf{Authorization}: Always obtain written permission
\item
  \textbf{Documentation}: Maintain detailed logs of all activities
\item
  \textbf{Minimal Impact}: Avoid disrupting production systems
\item
  \textbf{Confidentiality}: Protect sensitive information discovered
\end{itemize}

\end{solutionbox}
\begin{mnemonicbox}
``Vulnerability Assessment Validates Application
Security''

\end{mnemonicbox}

\end{document}
